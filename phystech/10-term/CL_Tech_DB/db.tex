%\documentclass[12pt]{article}
%\usepackage{jmlda}
\documentclass[12pt,a4paper]{article}
\usepackage[utf8]{inputenc}
\usepackage{graphicx}
\usepackage{color}
\usepackage{epstopdf} % http://tex.stackexchange.com/questions/29664/latex-error-unknown-graphics-extension-eps
\usepackage[colorlinks,unicode,pdfpagelabels=true]{hyperref}
\usepackage[english,russian]{babel}
\usepackage{amssymb,amsmath,enumerate}
\usepackage{enumitem}  \setlist{nolistsep}
\usepackage{tikz}
\usepackage{pgfplots} \pgfplotsset{compat=1.6}
\usepackage{textcase} % for UTF-8 in header

%%%%%%%%%%%%%%%%%%%%%%%%%%%%%%%%%%%%%%%%%%%%%%%%%%%%%%%%%%%%
%%%%%%%%%%%%%%%%%%%% 	\input macro.tex
%%%%%%%%%%%%%%%%%%%%%%%%%%%%%%%%%%%%%%%%%%%%%%%%%%%%%%%%%%%%
	\newcommand\ol[1]{\overline{#1}}

	\def\cond{\,|\,}
	\newcommand\normal[2]{\mathcal{N}\!\cbr{#1,#2}}

	\newcommand\al[1]{\begin{align*} #1 \end{align*}}
	\newcommand\begcas[1]{\begin{cases}#1\end{cases}}

	\def\le{\leqslant}
	\def\ge{\geqslant}
	\def\Ell{\mathcal{L}}
	\def\Rn{\ensuremath{\mathbb{R}^n}}

	\newcommand\mb[1]{\ensuremath{\boldsymbol{\mathbf{#1}}}}
	% \newcommand\argmax[1]{\arg\underset{#1}\max\,} % \operatornamewithlimits
	\def\argmax{\arg\,\max} % \operatornamewithlimits
	\newcommand{\prodl}{\prod\limits}
	\newcommand{\suml}{\sum\limits}

	\newcommand\cbr[1]{\left(#1\right)} %circled brackets
	\newcommand\fbr[1]{\left\{#1\right\}} %figure brackets
	\newcommand\sbr[1]{\left[#1\right]} %square brackets
	\newcommand\modul[1]{\left|#1\right|}
	\newcommand\norm[1]{\ensuremath{\left\|{#1}\right\|}}

	\newcommand{\T}{^{\text{\tiny\sffamily\upshape\mdseries T}}}
	\newcommand\dd[2]{\frac{\partial#1}{\partial#2}}

\begin{document}
%%%%%%%%%%%%%%%%%%%%%%%%%%%%%%%%%%%%%%%%%%%%%%%%%%%%%%%%%%%%
%%%%%%%%%%%%%%%%%%%% 			\begin{abstract}
Решается задача одноклассовой классификации электронных писем на предмет наличия в них спама. В работе вводится квазивероятностная модель для классической эмпирической постановки задачи одноклассовой классификации. Произведена модификация, позволяющая накладывать различные требования отбора признаков. Построенные методы классификации проверяются вычислительными экспериментами на модельных и реальных данных.
\end{abstract}
%%%%%%%%%%%%%%%%%%%%%%%%%%%%%%%%%%%%%%%%%%%%%%%%%%%%%%%%%%%%

	\title{Вероятностная модель одноклассовой классификации}
	\author{М.\,О.\,Бурмистров, Л.\,Н.\,Сандуляну}
	% \organization{Московский физико-технический институт, ФУПМ, каф. <<Интеллектуальные системы>>}
	% \thanks{Научный руководитель О.\,В.\,Красоткина}
	% \email{burmisha@gmail.com, liubov.sanduleanu@gmail.com}
	\date{декабрь 2012\,г.}

	\abstract{
	Решается задача одноклассовой классификации электронных писем на предмет наличия в них спама. 
	В работе вводится квазивероятностная модель для классической эмпирической постановки задачи одноклассовой классификации и 
	производится сведение классического подхода к новой модели.
	%Произведена модификация, позволяющая накладывать различные требования отбора признаков. 
	Построенные методы классификации проверяются вычислительными экспериментами на модельных и реальных данных.
	}

\maketitle

Пусть имеем 2 биржи: $A$ и $B$. Объемы, доступные для продажи: $Q_A, Q_B.$ Цены: 

$$s = \min\cbr{\frac{Y_A\cdot C}{p_A}, \frac{Y_B\cdot Q}{1-\delta}, R_A, \frac{R_B}{1-\delta}}$$
\vspace{-5pt}
$$
c = \frac{\cfrac{p_A}{p_B}(1-\delta)^2 - (1-\delta)^4}{1-(1-\delta)^4}
$$
\vspace{-15pt}
\begin{figure} [!ht] %lrp
		\centering
		\begin{tikzpicture}[x=1.2cm,y=1.2cm,thick,domain=-1.2:1.2]
			\draw[-latex] (-1.2,0) -- (1.5,0) node[below] {$\frac{p_A}{p_B}$};
			\draw[-latex] (0,-0.3) -- (0,1.3) node[right] {$|c|^5$};
			\draw[domain=-1.005:1.005] plot (\x,{(1+\x)*(1+\x)*(1+\x)*(1+\x)*(1+\x)/32});
			\node[left] at (0,1) {$1$};
			\node[below] at (1,0) {$1$};
			\node[below left] at (0,0) {$0$};
		\end{tikzpicture}
	\end{figure}

\end{document}
