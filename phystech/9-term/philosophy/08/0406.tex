\lectiondate{6 апреля}
Платон. Продолжение. 
Философская драматургия. 
Самые известные:

\begin{enumerate} 
	\item Апология Сократа. 
	\item Пир, симпозиум. Основная тема~--- любовь, варианты её проявление, обсуждается Эрот. Разделение людей. 
	\item Федр. 
	\item Федон. Рассматриваются аргументы о защите души. 
	\item Государства. Не только соцустройство, но и много чего еще. Дает представление о его взглядах. 
	\item Тимей. Это астроном. Диалог о философии астрологии. Есть единый бог-демиург, который создает мир его упорядочивает. Обладал тоталитарными взглядами. 
	\item Законы. Весьма суровая картина идеального государства~--- очень много ограничений и запретов.
\end{enumerate}

Он же рассказал об Атлантиде. 
Утверждал, что вначале люди были однополы. Но потом боги рассекли их на половинки, которые теперь и ходят по земле. 
Рассказ про Сороса. Он был сторонником открытого общества и учеником Карла Поппера. 
Поппер написал книгу «Открытое общество и его враги», которая была запрещена в СССР. 
Там он описывает историю и идеологию тоталитаризма, считает, что первый его идеолог~--- Платон. 

3 периода работы. 
1 период. Сократический. После смерти Сократа и до основания академии. Сократ присутствует во всех диалогах. Этический идеализм (эдиализм? \texttt{o\_O}). Это идеи, а не идеалы. 

2 этап. Путешествие в южную Италию и Сицилию. Формируется система. Философия Платона является систематической. 
Формируется система объективного идеализма: существование идей в неком идеальном месте, которые влияют на весь остальной существующий мир. Теория идей. Концепция души как материальной и неучтожимой сущности. 
Теория познания как припоминания. Критика чувственного познания и «мира вещей». Сократ есть, но идеи Платона. 

3 этап. Сиракузы и т.\,д. диалог Парменид, где критикует сам себя. Мысль о том, что диалектика существует не только в области аргументации, это не только логическая процедура. 
Но диалектика возможна и в сфере самих идей (сравнить с концепцией компа). Также смотри диалог Софист. Пытается построить иерархию мира идей, пытается найти место для чисел и включить их в мир идей (связь с пифагорйцами). 
Есть демиург~--- творит богов~--- творят десонов~--- творят числа. Диалог законы~--- относится к этому же периоду. 
В описанном обществе~--- все весьма жестоко (например убивать слабых детей) запрещены все искусства кроме ритуальных танцев и хорового пения. 
Он считал что искусство~--- подобие мира вещей так же как мир вещей~--- подобие мира идей. Считал что структура должна быть небольшой и стабильной. 

Писаная философия Платона~--- его тексты, разрозненные и порой противоречивые. И к тому же не все вопросы для полной системы освещены. Система разбивается на части. 
Неписанная философия Платона. Здесь все острые моменты игнорируются. Именно так преподавалось в платоновской академии. Это воспроизводимо и лучше для преподавания, но не выражает всю глубину. 

Аристотель 20 лет провел в академии. Но потом начал высказывать аргументы против. Простые аргументы, даже слишком. Но это потому, что изучили лишь простую версию платонизма. 

Зачем же Платон удваивал мир до мира идей и мира вещей?
Размышлял о природе этического. В Сократе ценил момент поворота мышления к думам о сущности знания. Никогда не поступать по привычке или сообразно обычаям, а всегда на основе сознательных соображений. 
Хотение по правилам, полученное в результате размышлений. Надо при добродетели ориентироваться на идею блага (недостижимую), и лишь если человек сознательно следует на эти правила, то он действительно добродетелен, не завязываясь на частности. 
Иначе~--- любую подлость можно объявить добродетелью, мол «жизнь у него такая». 
И всё это должно быть выражено в формате понятий~--- строгих дефениций. Кстати где живет общее правило? Оно не такое, как наши эмпирические представления, ведь мы можем уйти, а правило останется. 

Другой мотив~--- интерес к математике. Видимо, от знакомства с пифагорейцами. 
Платон был знаком с теорией чисел~--- арифметикой в пифагорейском варианте. Видимо, даже сам что-то придумывал. Еще и геометрия, учение о пропорциях. 
Решение делосской проблемы~--- удвоение куба, якобы. Интересовался структурой математического вывода~--- аксиомы и определения с доказательства. 

К тому Платон стремился к общему. Итого следствия. 
Идея абсолютного этического блага не может полностью и в точности воплотиться, хотя и существует. И число 4 не может материализоваться. Считал, что ни один из чувственных образов никогда не сумеет точно выразить идею, причем не важно: поступки или математику. Поэтому противопоставляются мышление и восприятие. 
Восприятие~--- лишь повод, вызывающий понятия, но не могут служить источником идей. Идея по сравнению с восприятием являются идеалом. Считал, что есть базовая общая идея. Где же она существует?

У Платона синаптическая трактовка общего~--- озарение, инсайт, «а-понял!». Именно так постигается общее, происходит с помощью интуиции. Считали, есть человек и мир, и человек его созерцает, не изменяя. 
Знания нужно откуда-то черпать. У человека должен быть доступ к высшему знанию. Откуда? Наша душа уже видела все эти идеи и способна их припомнить, т. е. в момент рождения душа уже всё знает. 
Но помнит она мало. Надо настроить душу, чтобы её ничто не отвлекало и она спокойно всё вспоминала. 
Есть выход в иммортиальный мир, душа может попадать туда, вечна, всё там видела. Процесс припоминания религиозный, этический, нравственный (?). Платоническая любовь. Связана чувствами лишь отчасти. Высшая норма любви~--- стремление к вечному, прекрасному, абсолютному. Это характерно не только для человека, но и для всех чувственных вещей. 
