\lectiondate{2 марта}
Кант.

Стоит идти от понятий к вещам, а не наоборот. Потому что понятия ближе к нам, мы именно ими и располагаем. И как вещи сообразуются с понятиями~--- Раньше было движение от объекта к субъекту. 
Но не Кант: человек --- главный участник познания, поэтому и надо его исследовать и понять, как он организует процесс познания. 
Возможно, поняв, как это происходит мы поймем, почему предметы представляются именно такими, а не другими. Главное: грамотно сформулировать вопрос.
Мир есть --- крайность догматизма. Другая крайность --- эмпиризм (близко к Юмовскому).

 Вопросы: 
 \begin{enumerate}
 	\item что я могу знать?
 	\item что я должен делать?
 	\item на что я могу надеяться? (работа «Религия в пределах одного только разума» --- Кант считает, что понятие бога --- философское понятие)
 	\item что такое человек? главный вопрос, поэтому философию Канта называют антропологической. Работа «Антропология», она не закончена, Кант, видимо, не смог до конца разобраться.
 \end{enumerate}
 Практическая философия --- этика. Ответ на второй вопрос не зависит от первого. Это совершенно разные области. 

 Юм же считал, что следует четко разделять сферу жизни и сферу философии (да, съехать с катушек можно, если его внимательно почитать и проникнуться). Не стоит следовать философским выводам в реальной жизни.

 Кант заложил многие основы для дальнейшего развития философии. После Канта начались поиски таких пластов в человеке, которые до этого вообще не интересовали классическую философию.

Антропологическому повороту способствовала то, что Кант показал, что человек занимает особое положение человека. Трагизм: человек одинок: лишь только один задумывается о себе, куда он движется. 

Посмотрим на критику чистого разума.

Что Кант понимает под критикой: дисциплину (а не опровержение ложных суждений). 
Это дисциплина о границах компетенции тех или иных разумных возможностей, определение того, до каких пределов распространяется теоретическая философия, и должна начаться практическая.

Ноумен --- вещь в себе. В отличие от феномена --- предмета чувственного созерцания. Ноумен --- продукт умственной деятельности, демаркационное понятие, указывающее на пределы нашего познания. 
% Skitch in Evernote
\parpic[r]{
	\begin{tikzpicture}[x=1cm,y=1cm,thick]
		\draw[very thick] (0,0) circle (1);
		\draw[dashed] (0,0) circle (1.2);
		\draw[dashed] (0,0) circle (1.5);
		\draw (0,0) node{феномены};
		\draw (0.5,1.3) node{ноумены (вещи в себе)};
		\draw (0,0) ++(-120:1) -- ++(-120:1) node[right]{трансцендентальное};
	\end{tikzpicture}
}

Чем крупнее ученый, тем большего он хочет постичь.


Природа, которую мы можем исследовать, и мир в целом --- не тождественные понятия. Опыт не полон.

Кант предупреждает, что теоретическая философия (и основанная на ней практическая) должна находиться в сфере опыта. Вне него начинаются спекуляции: противоречия и всё такое.
Допущение объективной реальности важно для Канта, как для человека, принимающего науку всерьёз. В этой части он материалист и признает существование реального мира, 
что это некая устойчивая данность, которую можно изучать (но было бы необоснованно считать, что настанет день и мы всю её изучим). Для этого ему и были нужны ноумены. Ноумен --- Вещь (с большой буквы, по Канту).

Еще раз повторим. В классической философии мир воспринимался как зеркало. Кант же говорил про синтез материи чувств, идущей извне, и стремления человека её освоить, вот и образуется картина реальности. 
Причем эта картина не нечто нейтральное, а результат взаимодействие, это изменяющийся процесс.

Что-то непонятное происходит на лекции.
Кант хочет понять, что это за структуры. 

Трансцендентальная --- не то, что выходит за пределы всякого опыта, а то что опыту хотя и предшествует (в том смысле, что без опыта никогда бы не проявилось, 
т.\,е. они не из опыта берутся, а непостижимом берутся уже при первом же сознательном акте), но является априорным.
Условия возможного опыта, без них опыт бы растекался. «Когда эти понятия выходят за пределы опыта, тогда они являются трансцендентного, что отличается от имманентного опыта (внутреннего)». Трансцендентальность --- пограничная область (см. рис.). 

Кант дистанцируется от ... Говорит, что есть логика формальная и содержательная (трансцендентальная). Формальная не привносит новых знаний.

Математика, метафизика, теоретическая физика --- главные науки.

Части «Критики чистого разума»:
\begin{enumerate}
	\item Трансцендентальная эстетика. Является ли математика наукой и за счет чего
	\item Трансцендентальная логика. физика и её научность
	\item Трансцендентальная диалектика. Является ли метафизика наукой.
\end{enumerate}

У человека можно выделить 3 познавательных способности. 
\begin{enumerate}
	\item Эстетика --- учение о чувственном восприятии.
	\item рассудок
	\item разум (здесь в узком смысле. А в названии имеется ввиду вся умственная деятельность)
\end{enumerate}

Математика (вроде) --- базовая наука. отвечает за способность человека ориентироваться в окружающем пространстве и времени. 
Пространство и время у Канта становятся априорными формами чувственности.

Смотрим с листок.

12 категорий.

Человека интересует, как именно связаны предметы. +Модальность: часть предметов важнее других. Мир мы можем измерять, обобщать, разделять на классы, упорядочивать. Всё это с помощью категорий. 
Вычленения причинно-следственных связей позволяет предсказывать, что будет дальше. Поэтому теорфиз очень полезен.

Для знания необходим синтез чувственного созерцания и наложения на них форм категориальных. В результате возникает целостная картина природы. 

Трансцендентальное единство аперцепции --- указание на единообразие категориальных синтезов, безо всяких флуктуаций.

Самая хитрая вещь происходит с естествознанием. Представление об объекте должно складываться в образ природы. А ученые понимают под природой объективный мир, расширяя природу до мира целиком. 
Однако, мир как таковой не может быть постигнут в рамках опыта. Образ природы построен категориальным синтезом. Со временем наука получает всё больший авторитет. Научные понятия и положения об устройстве мира действительно принимаются как истинные. 

Но тут есть шанс забыть, что люди, которые живут уже довольно долго жили с неправильными представлениями о мире и ничего страшного.
Кант: образ природы может меняться, человек может жить в разных мирах в зависимости от поставленных перед собой целей. Одно дело: просто прожить и прокормить себя и детей, а другое --- поехать неизвестно куда, поплыть, придумать, как это сделать.