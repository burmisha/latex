\lectiondate{26 октября}
Сегодня перенесемся на Британские острова и посмотрим на проект <<Эмпиризм>>.

Томас Гоббс. 1588--1679
Продолжатель Ф. Бэкона (даже был какое-то время его секретарем). Интересная и яркая личность. Но жизнь у него была посложнее, чем у Декарта и Спинозы потому, что в это время в Англии шла ужасная революция.
Родился в провинциальной семье малообразованного священника и крестьянки. В год его рождения обострилась борьба между Англией и Испанией (соперничество на море было очень угрожающим, граждане Англии были напуганы) и он (видимо, от испуга матери) родился семимесячным. Гоббс признавался, что он и страх~--- близнецы-братья, что они родились вместе. Искал решение социальных вопросов, в отличие от Декарта и Бэкона. Дядя помог ему получить образование и поступить в универ. Работал воспитателем в аристократических семьях. Это и обеспечивало, и давало возможность путешествовать вместе со своими воспитанниками. В 1649 встречался с Декартом. Первый набросок философской системы~--- 1640. В этот же год началась гражданская война (после революции). Столкновения шли под религиозной подоплекой. Гоббс не был глубоковерующим, мягко говоря, а считал религию плодом страха и невежества, но по долгу службы ему приходилось приспосабливаться. После казни короля Гоббс оказался во Франции в эмиграции и даже был наставником Карла Второго. В результате сделал свой выбор в пользу сторонников Кромбеля. 
В 1651 тайно возвращается в Лондон, выходит <<Левиафан>> (Это чудовище, упоминаемое в библии. (Антагонист~--- Бегемот тоже фигурирует в <<Бегемот или долгий парламент>>). ) Работа не очень большая, можно и прочитать. 
В более полном виде взгляды изложены в трилогии: <<О теле>>, <<О человеке>>, <<О гражданине>>. Идет по нарастающей. Бла-бла-бла\ldots Государство~--- искусственное тело\ldots
В период реакции Гоббсу пришлось несладко, как стороннику Кромеля. Подвергся гонениям, работы запретили, умер в нищите. К тому же его считали неверующим, гоббист~--- вообще было ругательное слово.

Теперь его общефилософские взгляды.
Вот, например, Бэкон~--- основоположник эмпиризма~--- написал не очень много несистематизированных неоконченых работ, не было системы у него. А у Гоббса была система, общее видение мира, он молодец. Система очень проста, тривиальна для нас, выстроена по образу и подобию механики. Речь идет о том, что философия должна служить практическим и прагматическим интересам. Механика была полезна людям: устройства всякие классные, флот. Так давайте же и дальше её использовать. Механика базируется на геометрической модели Евклида. Итак. Философия. Всё описывается законами. Гоббс считает, что хоть все тела и находятся в состоянии движения (то все вызвано механикой), всё сводится к перемещению, которые вызываются столкновениями и всякими толчками. Процесс познание начинается с деятельности чувств. Сюда распространяется механика: предметы давят на наши органы чувств. <<Фантазмы>>. У животных похожий процесс восприятия мира. Но у человека есть особенность: человек может оперировать не образами, а именами. Поименованный образ~--- знак. Наше знание~--- знаковая система. Гоббс~--- первый семиотик, считал что в практике именования возникает новая система. Практика транслируется в жизни поколений. С помощью знаков люди передают свои мысли, желания, надежды и т.\,п. Считал, что мы обладаем личным языком (что, кстати, является вообще нетривиальным даже ныне). Классифицирует знаки.
\begin{enumerate}
	\item Знаки-метки. Для мыслей и воспоминания мыслей. Для оперирования своим опытом.Каждый человек делает это на свой лад. Конкретные мыслительные образы. Лошадь Красавица~--- конкретная кобыла, на которой я катался.
	\item Знаки знаков. Общие понятия. Вот, например, лошадь~--- это знак знаков, не конкретная. Животные~--- знак знаков знаков.
\end{enumerate}
	
С помощью знаков вырабатывают суждение. Истинность и ложность~--- свойство суждений, а не предметов (т.\,е. соответствует ли суждение нашей реальности или нет). 
Касается анализа/синтеза и дедукции/индукции. Гоббс считает, что дедуктивный и индуктивный методы должны одинаково (т.\,е. оба, вместе) выстраивать наш мир. Сначала делаем индукцию, а потом дедукцию. Это повышает производительность. Анализ преобладает в физике. В геометрии~--- дедукция. Гоббс считает, что разумное мышление~--- аналогия калькуляции (не то что Декарт). Потому, что он либо ищет сумму, складывая воедино части, либо отсекает лишнее~--- вычитает.

Реально существуют лишь тела и их свойства~--- акциденции. Главная характеристика~--- протяженность. Жизнь и мышление~--- виды перемещения. Гоббс отрицает реальность цвета, вкуса, запаха\ldots реальны лишь тела и форма их движения.

Люди способны создавать новые тела. Природа человека~--- совокупность его способностей, страстей, желаний и стремлений~--- это различные движения. От рождения люди приблизительно равны. В общем и целом нет оснований считать, что кто-то от природы лучше/хуже. Эголитаризм. Главное: природа дала каждому равные права и возможности, права на всё. Права на всё~--- столкновение интересов, война всех против всех. Соперничество за землю, продукты, интернет. Если бы человек ничего не предпринял~--- наш род бы погиб. Гоббс считает, что в античной традиции (человек~--- существо общественное) была ошибка, а человек лишь тело. От природы человек эгоист. Но такая ситуация приводит к необходимости ограничить свои притязания. Нужно как-то решать конфликты. Гоббс говорит о наличии естественного разума. Естественный разум побуждает человека вступать в отношения с другими людьми (не только для безопасности, но и для решения задач а-ля <<мамонта ловить>>). Общественный договор переводит человека в состояние гражданина (никаких промежуточных или переходных состояний: либо естественное~--- всё моё, либо гражданское~--- не всё моё, а лишь то-то и то-то). Не путать естественное и первобытное состояния~--- нет привязки к времени. Гоббс специально подчеркивает хрупкость процесса гражданского становления (к тому же он видел гражданскую войну), его пугает перспектива возвращения в естественное состояние. Гос-во может возникнуть и естественным путём: в рез-те завоевания. Но всё-же общественный договор лучше. Носитель власти~--- либо король, либо коллектив. Самые главные права у человека отчуждаются, а передаются гражданам. Люди сами отдают свои права суверену. Это порок системы Гоббса: гос-ва не могут пропасть, люди раз и навсегда отдают свои права. Но лучше какой-то порядок, чем вообще никакого. Гоббс считал что не стоит делить власть на ветви (они начнут конфликтовать). И должен быть начальник, верховная власть должна принадлежать не народу. Но частную и хозяйственную жизнь власть никак не ограничивает, но вполне может искоренять даже ростки еретических идей. Это диктатура, детка.

У Гоббса был вообще первый опыт чисто светского социально-философского учения и явно виден перекос. Какую же роль Гоббс отводил религии? Очень простой ответ. Религия~--- плод страха и невежества. Возникла в силу компенсации людьми своих страхов, смягчения действий каких-то сил. С появлением религии, власть поняла, это всё очень удобно. Ну так и пускай остаётся (какая именно~--- пусть решают власти). 

Локк. 1632--1704.
Родился. Получил образование в универе. Работал в лаборатории Бойля. Даже использовал химическую терминологию в философии. Работал воспитателем, в частности в семействе Швебсдери.
Сочетал свой педагогический опыт с медицинским опытом~--- лечил. Находился в эмиграции во Франции (революция ж). Познакомился с работами всякими. Познакомился с будущим Вильгельмом Третьим. Причем сам он должностей не занимал. Жизнь закончил в доме леди Мэшам.

Создал классическую концепцию сенсуализма~--- учения о чувствах. Работа <<Опыт о человеческом разумении>>. Ему еще принадлежат работы социальной проблематики. Его считают отцом западного либерализма. Выступил с теоретикой конституционной монархии, говорил о разделении властей. Говорил, что люди не настолько уж злы и эгоистичны, люди от природы равны и свободны. Главное естественное право~--- на собственность~--- должно быть закреплено и обеспечено с помощью разумных законов. Гос-во создается на основе добровольного решения граждан. Причем лишь часть своих прав граждане передают власти, и гос-во подотчетно. Гос-во должно содержать в себе системы сдерживания и противовеса. Сосредоточить всё у одного человека~--- слишком рискованно. Система должна смягчать индивидуальные особенности правителей, корректировать. Для этого он предложил 3 независимых ветви власти
\begin{enumerate}
	\item Законодательная
	\item Исполнительная (в т.ч. судебная) 
	\item Федеративная (типа МИД) 
\end{enumerate}
	
Локк считал, что государство может обеспечить безопасность лишь внутри страны, а вопросы внешние необходимо решать отдельно. Был решительным противником божественного происхождения королевской власти, говорил, что это сугубо человеческое изобретение (очень полезное, но нуждающееся в контроле). Считал, что система ради граждан, а не наоборот.
