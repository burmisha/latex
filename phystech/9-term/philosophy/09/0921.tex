\lectiondate{21 сентября}
Остановились на Ансельме Кентерберийском. 1033--1109

Он выдвинул док-во. Поскольку бог всеобъемлющ, то он существует. Он обладает все полнотой предикатов. Существование бога выводится из существования понятия о боге. Так себе док-во. (И даже в те времена оно не было общепризнанным. Фома Аквинский считал, что у нас нет представления о боге.) 
Это было проявление средневекового реализма. Это не обычный реализм (нечто очень похожее на реальное, максимальное сходство с чувственным опытом), ведь средневекового человека чувства не очень волнуют. Признается существование общих родовых и видовых понятий. Стол, человек~--- наделяются отдельным существованием. т.\,е. отделяются от конкретного стола или человека. Слоган: универсалия сунт реалие (универсалии обладают реальным существованием, одновременно с реальными вещами, хоть и не воспринимаются чувственно.) В рамках средневекового мышления род ближе к богу, чем вид, т.\,е. он реальнее, он обладает большим числом сущностей. Такая онтологическая картина уже встречалась у Платона (платонизм) еще в античности. Средневековый реализм как онтологическая позиция: у универсалий больше бытия и реальности. Это было основой доказательство бытия бога.

Этому противостоял номинализм. (Номина~--- имя). Универсалия сунт номина. За универсалями нет реальности. Важны лишь реальные вещи, имеющие имена.

Крайние сторонники.
Реализм

\begin{itemize}
	\item Ансельм Кентерберийский
	\item Шартрская (под Парижем) школа: Бернар Шартрский, Гильберт Порретанский
\end{itemize}

Номиналисты
\begin{itemize}
	\item Росцелин из Компьена
	\item Уильям Оккам (14 век) (ударение на первый слог) 
\end{itemize}

Концептуализм (умеренный номинализм, нечто промежуточное) 
\begin{itemize}
	\item Пьер Абеляр
\end{itemize}
Фома Акримский~--- умеренный реализм.

Бернард Шартрский. 12 век. Умер где-то в 1124--1130. Учил, что реален лишь бог, идеи и материя. Это 3 вида бытия. А всё остальное~--- смесь 3 сущностей. Выше остальных стоит бог, затем идеи, ниже материя.
Его ученик~--- Гильберт Порретанский. Конец 11- начало 12. Причина бытия отдельного человека~--- родовая сущность, соответствующая человечности~--- родовой сущности. Вместе с сущностью приобретаем и свойства. Становимся людьми через св-во быть человеком. А форма быть человека наследуется от бытия человечности, от общего понятия. Реализм принижает конкретное в пользу всеобщего.

Номинализм.
Росцелин. 1056--1122. Реально существуют лишь единичные вещи. Понятия~--- лишь <<звучание голоса>>. Роды и виды не имеют реального существования. Существовать может лишь что-то одно, что-то единое и уникальное, но никак не общее. Выводы для церкви были не очень приятны. (То ли дело реализм, который оправдывал религию). Три лица троицы~--- 3 субстанции. Бог это только слово, звучание голоса. Был обвинен в трехбожье и его позиция была осуждена церковью, но не отлучили и не казнили.

Концептуализм. 1079--1142 Пьер Абеляр. Умеренный номиналист. Талантливый педагог, боролся с невежеством и религиозным догматизм. Логик. Религиозный рационалист. Его оскорбили: с девушкой он связался богатой, а сам был монахом. См. книгу <<История моих бедствий>>, есть и фильм, вроде. Реальное существование принадлежит только реальным вещам. Был сильным логиком. Главное в универсалиях~--- их смысловое значение. Выступают в качестве предикатов для субъекта. Ввел понятие <<концепт>>. Концепт~--- нечто общее, объективно присутствующее в вещах, на основе чего возникает слово. Такое определение~--- не полный произвол. Утверждается, что есть некоторая принудительность в именовании вещей.

Мистика. Пыталась объяснить через чувства, а не через рассуждения. Дозированно принималась церковью. Существовала ортодоксальная мистика, которая пыталась вытеснить схоластику как основной путь богопознания. Бернард Клервосский. Говорил, что душа во время экстаза уподобляется богу, поэтому надо почаще в нем бывать. Человек в такие моменты: душа исступает из тела и подключается к высшему миру, воспаряет в богу и пытается слиться с ним, но не удается. Но как достичь экстаза он не пояснял.

Пантеистическая народная мистика. Явилась духовной почвой для протестантизма и его принятия. В этой среде (живущей в духе народной мистики) ~--- отдаленные крестьянские среды (секты катар, альбигойцы). Человек обладает бессмертной душой и должен использовать этот дар для общения с богом и возвышения. Фишка: не нужны посредники, можно самому погрузиться в библию и самому проникнуться божественным светом. Отрицал посредничество церкви в общении человека с богом. Церковь боролась с народными мистиками. Но они всё равно партизанами по лесам растекались. Особенно по лесам нынешней Германии.

13 век. Духовная жизнь Европы существенно меняется. Важный фактор: создание университетов и новой социальной группы: преподаватели (статус, привилегии). Одновременно распространяется ученость, а то что учить только религию. Возникает интерес к античным авторам и Аристотелю (он не был запрещен церковью). Схоластика уже не хотела идти и давить что-либо самостоятельно в поисках новых форм улучшения земной жизни. Самая цивильная форма схоластики~--- в трудах Фомы Аквинского. 1225--1285. Фома Аквинский учился в Неаполе и Париже, был религиозен, в 17 лет вступил в доминиканский орден. Ангельский доктор~--- его прозвище за мягкость. Ни с кем не ссорился и писал правильно. <<Сумма теологии>>~--- многотомное сочинение. <<Сума против язычников>>~--- самые весомые аргументы в защиту христианства против неверующих и язычников. <<Учение о гармонии веры и разума>>. Фома говорил, что нужно разделить сферы, где можно исп-ть разум и нет. Есть истины разумные и сверхразумные (не антиразумные и не неразумные. Политкорретно, крутяк ваще. Троебожие, воскресение и т.\,п.). Разуму есть чем заняться. Натуральная теология. В своей философии опирался на Аристотеля. Мир иерархичен и систематизирован, сверху бог.

Сущность~--- эссенция~--- essentia
Существование~--- экзистенция~--- existencia. Их надо различать. \texttt{o\_O}
Во всех сотворенных вещах они расходятся. Во всех сотворенных вещах больше существования. Только в боге они совпадают. Позиция умеренного реализма. Универсалии существуют трояким образом.

\begin{enumerate}
	\item Сущностная форма. В вещах.
	\item Пострес. После вещей. Абстрагирование
	\item Антерес. До вещей. Что-то с богами.
\end{enumerate}Версия очень удобна для церкви. Универсалии сущ-ют в вещах, до и после них.
Фома е соглсен с Ансельмом. Мы е можем познать сущность бога. В силу своей малости и ограниченности. Непосредственно постичь не удастся. Но существование бога, воплощенное в природе, вещах, в нас самих~--- можно. Восходить от вещей к их вторцу. Так можно и должно доказывать сущ-ие бога. 5 космологических или физико-телеологических доказательств. Те же самые, что и у Аристотеля про перводвигатель.

\begin{enumerate}
	\item Идем от существования движимых вещей к их двигателю
	\item От иерархии причин к первопричине~--- причине причин.
	\item Существование вещей, способных приобретать и терять свойства
	\item От несовершенства
	\item От целевой причинности к сущ-ию ума, ответственных\ldots бла-бла-бла. Не могло же всё само так сложиться. (У неё муж~--- технарь.) 
\end{enumerate}
Невозможно продолжать ряд до бесконечности.
Вернул разуму приоритет.
Самое классное государство~--- монархия. А монарх прислушивается к церкви, под её духовным руководством.


См. Томизм

Разобрать текст самостоятельно. Пришлет на почту. См. Джордано Бруно и Кузанского. Увы, оригинальных идей не было.

И еще 2 текста. По реформации. О Лютере. Для общего развития.

Уильям Оккам
Крайний номиналист. Последний схоласт. Дальше рассуждать было не очень, самоуничтожение. 1300--1349.
Абсолютная свобода божественное воли в акте творения не связана даже идеями. Нет идей вовсе и бог ими не пользуется, а значит универсалий нет в боге и тем более в вещах. Бог сразу творит вещи. Раз нет идей, то нет и идей видов, род~--- это лишь человеческие предположения. Лишь идеи индивидов. В индивиде заключена реальность, заключенная вне бога. Правило: не следует умножать сущности сверх необходимого. Есть бог и реалии, никаких универсалий. Нет ничего общего в вещах, нет отношений и закономерностей. Ничто не объединяет вещи. Нет причинно-следственной связей. Но наше знание о мире формируются на основе общих понятий~--- чисто человеческих понятий. Реальность сюда не относится. С помощью наших инструментов нельзя познать истину. Общие идеи неприменимы к истинной реальности. Мы упускаем из индивидов самое главное, оставляя лишь общее. Отрицается рациональности мири и гармония слова и бытия~--- основ схоластики. Бытийная и смысловая составляющие противопоставляются. Мир состоит из неповторимых вещей, а мы их не схватываем, используя общие слова. Анализ становится безпредметным. Схоластика более не может никого удовлетворить: несостоятельно по своей сути.
Открытия не работают без человеческого фактора, без принятия сообществом.

Зарождение <<экспериментальной философии>>.
Роберт Гроссетест. 1175--1253
Роджер Бекон. 1214--1292
Это произошло на британских островах. А в Париже изучали гуманитарные науки. В Оксфорде~--- квадрилиум (\ldots). 
Гроссетест. Большая голова~--- умный. Доктор теологии, первый канцлер факультеты. О свете или о начале форм. О линиях и углах фигур. Физика света. Ценил геометрию и оптику. Все начинается со света\ldots сосредоточен в сферах космоса\ldots какой-то бред.
Роджер бекон. Тоже фронсисканец. Ученик Гроссетеста. Истина~--- дитя времени, дочь всего человечества. Новое поколение исправляет ошибки предшественников, пытается сделать что-то новой
4 препятствия познанию истины

\begin{enumerate}
	\item Доверие сомнительному авторитету
	\item Привычка
	\item Вульгарные глупости
	\item Невежество под маской всезнайства
\end{enumerate}
2 пути познания

\begin{enumerate}
	\item Аргументация~--- дают вывод, но лишь формальный
	\item Эксперимент~--- даст уверенность. Опыт внешний (через чувства) и внутренний (божественный). 
\end{enumerate}
Френсис бекон всё содрал у своего однофамильца.
Для внешнего опыта важна математика. Занимался оптикой, линзами, сделал очки. Думал о механике. Аэроплан, мост без опор, карета без лошадей, ~подлодка. <<Знание~--- сила>>.
