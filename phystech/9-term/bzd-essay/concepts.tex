Пусть  $D$~--- множество (коллекция) документов, $W$~--- множество встречающихся в них слов (словарь). Тогда каждый документ $d\in D$ можно рассматривать как упорядоченную последовательность слов из $W$ : $d = \cbr{w_1, \ldots, w_{n_d}},$ где $n_d$~--- количество слов в документе с учетом возможных повторов.

Каждый документ может относиться к некоторому набору тем из множества $T.$ Пусть каждому документу $d\in D$ соответствует его распределение по темам: $p\cbr{t\cond d}$. При этом сами темы различаются между собой частотами употребления слов. Пусть каждой теме соответствует её распределение по всем словам словаря: $p\cbr{w\cond t}.$ К введенным распределениям предъявляются обычные требования нормировки и неотрицательности:
$$\suml{w\in W}p\cbr{w\cond t} = 1; \quad p\cbr{w\cond t} \ge 0; 
\qquad
\suml{t\in T}p\cbr{t\cond d} = 1; \quad p\cbr{t\cond d} \ge 0.$$

Вероятностной тематической моделью будем называть совокупность множества тем $T$ и распределений $p\cbr{t\cond d}$ и $p\cbr{w\cond t}$ на всем множестве слов, документов и тем. 
