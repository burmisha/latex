\documentclass[unicode,lefteqn,c,hyperref={pdfpagelabels=false},12pt]{beamer}
\usepackage[utf8]{inputenc}
\usepackage{amssymb}
\usepackage{amsmath,mathrsfs}
\usepackage[russian]{babel}
\usepackage{ulem}\normalem
\usepackage{color}
\usepackage[noend]{algorithmic}

\input macro.tex

\usetheme{Warsaw}
\usefonttheme[onlylarge]{structurebold}
\setbeamerfont*{frametitle}{size=\normalsize,series=\bfseries}
\setbeamertemplate{navigation symbols}{}
\setbeameroption{show notes}
\definecolor{beamer@blendedblue}{RGB}{15,80,120}
\let\Tiny=\tiny
\def\shortspace{\hspace{1.5pt}}

%%%%%%%%%%%%%%%%%%%%%%%%%%%%%%%%%%%%%%%%%%%%%%%%%%%%%%%%%%%%%%%%%%%%%%%%%%%%%%%
\title[\hbox to 56mm{Проникающая радиация\hfill\insertframenumber\,/\,\inserttotalframenumber}]{Поражающие факторы \\ядерного взрыва: \\проникающая радиация}
\author[П.\shortspace C.\shortspaceАндрианов, М.\shortspaceО.\shortspaceБурмистров]{П.\shortspace C.\shortspaceАндрианов, М.\shortspaceО.\shortspaceБурмистров}
\institute{\vfill Московский физико-технический институт
		%\vfill Факультет управления и прикладной математики
		\vfill Военная кафедра}
\date{\today}
%%%%%%%%%%%%%%%%%%%%%%%%%%%%%%%%%%%%%%%%%%%%%%%%%%%%%%%%%%%%%%%%%%%%%%%%%%%%%%%
\begin{document}
\begin{frame}
    \titlepage
\end{frame}
%%%%%%%%%%%%%%%%%%%%%%%%%%%%%%%%%%%%%%%%%%%%%%%%%%%%%%%%%%%%%%%%%%%%%%%%%%%%%%%
%%%%%%%%%%%%%%%%%%%%%%%%%%%%%%%%%%%%%%%%%%%%%%%%%%%%%%%%%%%%%%%%%%%%%%%%%%%%%%%
%\section{Введение}
%%%%%%%%%%%%%%%%%%%%%%%%%%%%%%%%%%%%%%%%%%%%%%%%%%%%%%%%%%%%%%%%%%%%%%%%%%%%%%%
%\subsection{Данные и вероятностные гипотезы}
%%%%%%%%%%%%%%%%%%%%%%%%%%%%%%%%%%%%%%%%%%%%%%%%%%%%%%%%%%%%%%%%%%%%%%%%%%%%%%%
\begin{frame}{Поражающие факторы ядерного взрыва}
    \textbf{Наземный ядерный взрыв}
    \begin{itemize}
    		\item ударная волна (50\% энергии);
    		\item световое излучение (30--40\% энергии);
    		\item радиоактивное заражение местности ($<$15\% энергии);
            \item проникающая радиация ($<$5\% энергии).
    \end{itemize}
    \bigskip
    \textbf{Воздушный ядерный взрыв:}
    \begin{itemize}
            \item проникающая радиация ($<$85\% энергии);
            \item ударная волна ($<$10\% энергии);
            \item световое излучение (5--8\% энергии).
    \end{itemize}

  
\end{frame}

\begin{frame}{Определение и основные особенности}
    \textbf{Проникающая радиация (ионизирующее излучение)}  представляет собой гамма-излучение и поток нейтронов, испускаемых из зоны ядерного взрыва в течение единиц или десятков секунд

    \smallskip

    \textbf{Основные особенности}
    \begin{itemize}
            \item малый радиус поражения при взрыве в атмосфере (2--3~км);
            \item способность вызывать обратимые и необратимые изменения в материалах, электронных, оптических и других приборах, не разрушая их полностью;
            \item защитой от проникающей радиации служат различные материалы, ослабляющие гамма-излучение и поток нейтронов.
    \end{itemize}
\end{frame}

\begin{frame}{Характерные параметры и поражающие факторы}
    \begin{itemize}
        \item Доза
        \item Мощность дозы излучения
        \item Поток
        \item Плотность потока частиц
    \end{itemize}
    \smallskip

    \textbf{Воздействие радиации на организм:}
    \begin{itemize}
        \item В долгосрочном плане проявляется мутациями.
        \item В краткосрочном --- лучевой болезнью различной степени тяжести.
        \item В первую очередь, страдают клетки лимфатической системы и костного мозга.
        \item Затем идут волосяные фолликулы и клетки желудочно-кишечного тракта.
    \end{itemize}
\end{frame}


\begin{frame}{Защита от ионизирующего излучения}


\begin{columns}
    \column{0.5\textwidth}
        \begin{tabular}{l|c}
        \multicolumn{2}{c}{\textbf{Гамма-излучение}}\\
        материал & толщина\\ 
        \hline
        Свинец & 2\,см \\
        Сталь & 3\,см \\
        Бетон & 10\,см  \\ 
        Каменная кладка & 12\,см \\
        Грунт & 14\,см \\
        Вода & 22\,см \\
        Древесина & 31\,см \\
        \end{tabular}
    \column{0.5\textwidth}
        \begin{tabular}{l|c}
        \multicolumn{2}{c}{\textbf{Нейтронное излучение}} \\
        материал & толщина\\ 
        \hline
        Свинец & 9--20\,см \\
        Сталь & 5--12\,см \\
        Бетон & 9--12\,см  \\
        Каменная кладка & 12\,см \\
        Грунт & 14\,см \\
        Вода и пластмасса  & 3--6\,см \\
        Древесина & 10--15\,см \\
        \end{tabular}

\end{columns}
\end{frame}



\end{document}
