\simpletitle{15 сентября}

$$\int_a^b x(t)dt = \lim_{k\to\infty}\sigma_{T_k}$$

\begin{enumerate}
	\item $\norm{\integr ab{x(t)dt}} \le \integr ab{\norm{x(t)}}dt$
	\item $\integr ab{\cbr{\alpha x_1(t) + \beta x_2(t)}dt} = \alpha\integr ab{ x_1(t)dt} + \beta\integr ab{x_2(t)dt}$ --- линейность интеграла
	\item $\integr ab{} = \integr ac{}  + \integr cb{}$
	\item $\psi(t) = \integr {t_0}t{x(\tau)d\tau} \Rightarrow \psi'(t) = x(t)$
\end{enumerate}

\lemma{о среднем} Если $x(t)$ дифференцируема на $(a,b)$ и непрерывна на $[a,b]$, то $$\norm{x(b) - x(a)} \le \norm{x'(a+\theta(b-a))}(b-a), \foral {0<\theta<1}.$$
\proof{
	\al{
		\phi(t) &= \cbr{x^*, x(t)} \\
		\frac{phi(t+\Delta t) - \phi(t)}{\Delta t} &= \cbr{x^*, \frac{x(t+\Delta t) - x(t)}{\Delta t}} \\
		\phi'(t) &= \cbr{x^*, x'(t)}\\
		\phi(b) - \phi(a) &= \phi'(a +\theta(b-a))(b-a)\\
		\cbr{x^*, x(b)- x(a)} &= \cbr{x^*, x'(a + \theta(b-a))(b-a)}\\
		\mod{x^*, x(b)- x(a)} &\le \norm{x^*}\norm{x'(a + \theta(b-a))}(b-a) \\
		\text{По теореме }&\text{Хана-Банаха подберем такой функционал, что  ... и получим, что} \\
		\norm{x(b)- x(a)} &\le \norm{x'(a + \theta(b-a))}(b-a)
	}
}

Следствие: $x'(t) = 0 \Rightarrow x(t) = x_0$

Еще следствие: $\norm{x(t_2) - x(t_1) - x'(t_1)(t_2-t_1)} \le \norm{x'(t_1+\theta(t_2-t_1)) - x'(t_1)}(t_2-t_1)$

\proof{
	Д-м свойство 4:
	$\difflim \psi t = \frac1{\Delta t}\integr{}{}{}$ .... и по лемме
}

\theorem{Ньютона-Лейбница} $\integr ab{x'(t)dt} = x(a) - x(b).$
\proof{
	\al{\psi(t) &= \integr at{x'(\tau)d\tau} \Rightarrow \psi'(t) = x'(t)  \Rightarrow\\
	\Rightarrow \psi(t) - x(t) &= x_0 \Rightarrow \psi(a) - x(a) = x_0, \quad \psi(b) - x(b) = -x(a)\\
	\integr ab x'(t)dt &\le x(a) - x(b).
	}
}

Отсюда получается формула интегрирования по частям и формула замены переменной.

Справедлива формула Тейлора (если есть производные до $n$-го порядка) с остаточным членом в интегральной форме.
\al{
	x(t) &= \sum_{k=0}^n \frac{x^{\cbr{k}}(t_0)}{k!}\cbr(t-t_0)^k + r_n(x
	) \\
	r_n(x) &= \frac{(t-t_0)^{n+1}}{n!} \integr 01 (1-u)^nx^{n+1}(t_0+u(t-t_0))du 
}
\proof{
	Доказательство по индукции. База: $n=0$
	\al{
		x(t) &= x(t_0) + r_0(t) \\
		r_0(t) &= \frac1{1!}\integr 01{x'(t_0+u(t-t_0))(t -t_0)du} \text{ через замену переменной}
	}
	Пусть верна при $n$, докажем при $n+1$
	\al{
		r_n(t) &= \frac{(t-t_0)^{n+1}}{n!} 
			\sbr{
				- \frac{(1-u)^{n+1}}{n+1}x^{(n+1)}(t_0 + u(t-t_0)) |_0^1 
				+ \integr 01{\frac{(1-u)^{n+1}}{n+1}x^{(n+2)}(t_0 + u(t- t_0))(t-t_0)du}
			} \\
		&= \frac{(t-t_0)^{n+1}}{(n+1)!}x^{(n+1)}(t_0) + \frac{(t-t_0)^{n+2}}{(n+1)!}\integr 01{(1-u)^{n+1}x^{(n+2)}(t_0 + u(t-t_0))du} \\
		&= \frac{(t-t_0)^{n+1}}{(n+1)!}x^{(n+1)}(t_0) + r_{n+1} 
	}
	$$x(t)-T_n(x) = \frac{(t-t_0)^{n+1}}{(n+1)!}x^{(n+1)}(t_0) + r_{n+1} \Rightarrow x(t) = T_{n+1}(x) + r_{n+1.}$$
}

$\norm{r_n(x)} \le \frac{(t-t_0)^{n+1}}{n!}C\integr01{(1-u)^n}du$ --- если равномерно ограничено, то сходится.


Несобственный интеграл: $\integr 0{+\infty}{x(t)dt} =_{\text{def}} \lim_{k\to\infty}\integr0{t_k}x(t)dt$ и не зависит от последовательности $\fbr{t_k}_{k=1}^{\infty}$

$\norm{x(t)} \le \phi(t), \int \phi <+\infty \Rightarrow \int x \text{ --- сходится}$
\proof{
	$y_k = \integr 0{t_k}x(t)dt. $
	$\norm{y_{k+p} - y_k} $ --- оценить, нер-во для интегралов, показать фундаментальность.
}

\hrule

$$f\colon O(x_0)\to B_2, O(x_0) \subset B-1.$$
\defenition{}
	Ф $f(x)$ дифф в т $x_0$, если $\exist{\text{лин огран оператор\ } A} f(x) - f(x_0) = A(x-x_0) +\omega(x)(x-x_0),$ где $\lim_{x\to x_0}\omega(x) = 0.$

$A\in L[B_1\to B_2]$ --- сильная производная (или производная Фреше)

$A(x-x_0)$ --- дифференциал Фреше. $df(x, x_0) = df(x_0).$

\theorem{Производная сложной функции} Если $f(y)$ дифф в т. $y_0$, а функция $\psi(x)$ дифф в т $x_0$, $y_0 = \psi(x_0)$. Тогда сл. ф.  $f(y(x))$ дифф в т $x_0$ и $f(\psi(x))'_{x=x_0} = f'(\psi(x_0))\psi'(x_0)$
\proof {
	\al {
		f(y) - f(y_0) &= f'(y_0)(y-y_0) + \omega_1(y)\norm{y-y_0} \\
		\psi(x) - \psi(x_0) &= \psi'(x_0)(x-x_0) + \omega_1(x)\norm{x-x_0} \\
		f(\psi(x)) - f(\psi(x_0)) &= f'(\psi(x_0))(\psi(x) - \psi(x_0) + \omega_1(\psi(x))\norm{\psi(x) - \psi(x_0)} \\
		f(\psi(x)) - f(\psi(x_0)) &= f'(\psi(x_0))\psi'(x_0)(x -x_0) + \omega_3(\psi(x))\norm{x- x_0} \\
		\omega_3 &= f'(\psi(x_0))\omega_2(x) + \frac1{\norm{x-x_0}}\omega_1{\psi(x)}\cbr{\psi'(x_0)(x- x_0) + \omega_1(x)\norm{x-x_0}}\\
		\text{сказать, что бесконечно малая}
	}
}