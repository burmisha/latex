\documentclass[12pt,a4paper,oneside]{article}
\usepackage[T2A]{fontenc}
\usepackage[utf8]{inputenc}       
\usepackage[english,russian]{babel} % Download Russian fonts.

\begin{document}

	\title{Рецензия на работу А.\,П.\,Мотренко \\ <<Оценка плотности совместного распределения>>}
	\author{М.\,О.\,Бурмистров}
	\date{январь 2013 г.}
	\maketitle

В работе А.\,П.\,Мотренко исследовалась проблема оценки плотности совместного неоднородного распределения набора случайных величин, включающего в себя одновременно дискретные и непрерывные величины. 
Порождающие алгоритмы классификации зачастую приходится применять к данным, содержащим оба типа случайных величин, поэтому необходимо разработать общий подход к описанию плотности совместного распределения.

В работе рассмотрены два подхода к этой оценке: факторизация и непараметрическое оценивание. 
При исследовании факторизации получены явные выражения для совместной плотности смешанного распределения в предположении о том, что одна из случайных величин является смесью гауссовских, а другая бинарной. 
В случае, когда какие-либо предположения о природе распределений сделать быть не могут, применено непараметрическая оценка плотности совместного распределения. 
Изучен механизм подбора оптимального ядра сглаживания. 
Студенткой проведены эксперименты на синтетических и реальных данных, демонстрирующие эффективность подходов.

Достоинством работы является разносторонний и детальный анализ подходов к проблеме оценки плотности совместного распределения.

Работа является самостоятельным научным исследованием и удовлетворяет всем предъявляемым требованиям.
\end{document}