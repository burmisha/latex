\input header.tex
\input pagestyle.tex
\input macro.tex

\begin{document}
\simpletitle{Построение классификатора,\\ разделяющего облака точек. \\ Михаил Бурмистров. \today}


\maintext{Дано:} $X = \fbr{
                          \cbr{
                              \fbr{\cbr{x_i^j, y_i^j}}_{i=1}^{n_i}, y_j
                          }
                      }_{j=1}^N$ ~--- набор облаков точек на плоскости ($\cbr{x_i^j, y_i^j} \in \mathbb R^2$) и их меток: $y_i\in Y=\fbr{0,1}$.

\maintext{Построить:} Классификатор $a: \fbr{\cbr{x_i, y_i}}_{i=1}^n \to Y$.

\simpletitle{Идея решения}
В данном решении предлагается выделить признаки из облаков точек и по ним построить классификатор стандартным алгоритмом машинного обучения (был выбран метод опорных векторов).

Для каждого облака проделаем следующее:
\begin{enumerate}
  \item Чтобы учесть возможный различный масштаб облаков, предлагается линейным преобразованием перевести их в квадрат $[0, 100]\times [0, 100].$ (рис. \ref{eps:cloud})
  \item После этого  строится выпуклая оболочка полученного облака, и в качестве 3 признаков были выбраны её длина, площадь и число точек в выпуклой оболочке. (рис. \ref{eps:convex})
  \item Еще 3 признака получаем с помощью алгоритма, построенного в предыдущей работе, который был применен с параметром $r = 10$ (и здесь в качестве признаков были выбраны длина, площадь и число точек в построенной оболочке.) (рис. \ref{eps:contour})
  \item Также по построенной оболочке можно построить силуэт нашего облака, получив бинарное изображение (рис. \ref{png:binarised}). По этому силуэту строим скелет (рис. \ref{png:skel}) и обрезаем все ребра длиной менее 40 (рис. \ref{png:prunned}) , а затем считаем число узловых точек --- это число берем в качестве 7-го признака. 
\end{enumerate}
После этого мы имеем набор признаков объекта (облака точек) и метку его класса. Теперь можно строить классификатор. 

Для оценки качества использованного подхода к построению классификатора применим следующий подход: будем поочередно (не последовательно) удалять по одному объекту из нашей выборки, по оставшейся строить классификатор, а потом проверять, верно ли он классифицирует удаленный объект.
В результате получаем, что алгоритм ошибается на 11 объектах (6 рыб и 5 птиц) из 107, доступных для обучения (т.\,е. доля неверно классифицированных объектов составляет $0{,}1028$).

\addeps{cloud}{Облако точек, переведенное в квадрат}
\addeps{convex}{Выпуклая оболочка}
\addeps{contour}{Построенная оболочка при $R = 10$}
\addpng{binarised}{Бинаризованный контур}
\addpng{skel}{Скелет контура}
\addpng{prunned}{Скелет контура с удаленными короткими ветвями}

\end{document}
