\documentclass[12pt]{article}
\usepackage[utf8]{inputenc}
\usepackage[T2A]{fontenc}
\usepackage{graphics,graphicx,epsfig}
\usepackage{amssymb,amsfonts,amsthm,amsmath,mathtext,cite,enumerate,float}
\usepackage[english,russian]{babel}
\usepackage{color}
\usepackage{algorithm}
\usepackage[noend]{algorithmic}
\usepackage[vflt]{floatflt}
\definecolor{linkcolor}{RGB}{7,31,63}%{15,80,120}
\usepackage[colorlinks,unicode,pdfpagelabels=false, linkcolor = linkcolor]{hyperref}
\usepackage{tikz}
\usepackage{pgfplots} \pgfplotsset{compat=1.6}
% \textheight=24cm		\textwidth=18cm
% \oddsidemargin=-10mm 		\evensidemargin=-10mm
% \topmargin=-2,5cm
% \parindent=24pt 		\parskip=3pt
% \footnotesep=3ex
% \raggedbottom %\flushbottom
% \clubpenalty=10000		\widowpenalty=10000 	\tolerance=500
% \renewcommand{\baselinestretch}{1}%{1.4}

\textheight=24cm		\textwidth=16cm
\oddsidemargin=0mm 		\evensidemargin=0mm
\topmargin=-2,5cm
\parindent=24pt 		\parskip=3pt 
\footnotesep=3ex
\raggedbottom %\flushbottom
\clubpenalty=10000		\widowpenalty=10000 	\tolerance=500
\renewcommand{\baselinestretch}{1.4}

\input macro.tex
\input common.tex

\begin{document}
	{
	\renewcommand{\baselinestretch}{1}
	\thispagestyle{empty}
	\begin{center}
		\sc
			Министерство образования и науки Российской Федерации\\
			Московский физико-технический институт {\rm(государственный университет)}\\
			Факультет управления и прикладной математики\\
			Вычислительный центр им. А. А. Дородницына РАН\\
			Кафедра <<Интеллектуальные системы>>\\[40mm]
		\rm\large
			Бурмистров Михаил Олегович\\[5mm] 
		\bf\Large
			Дифференциальный скользящий контроль \\
			при выборе параметров регуляризации \\
			в задаче регрессионного анализа \\
			с помощью Elastic Net \\ [5mm]
			% Методы оптимизации параметров\\
			% вероятностных тематических моделей
		\rm\normalsize
			{511656 --- Математические и информационные технологии}\\[5mm]
		\sc
		Магистерская диссертация
		%Выпускная квалификационная работа бакалавра
		\vspace{40mm}
	\end{center}
	\hfill \parbox{80mm} { 
		\begin{flushleft}
			\bf{Научный руководитель:}\\
			\rm д.т.н., профессор \\
			в.н.с. ВЦ РАН \\
			Моттль Вадим Вячеславович
			% \rm д.ф.--м.н. Воронцов Константин Вячеславович\\
		\end{flushleft}
	}
	\\ \vspace{2cm}
	\begin{center}
		Москва\\
		2014
	\end{center}
}
	\newpage \tableofcontents
	\newpage \begin{abstract}
Решается задача одноклассовой классификации электронных писем на предмет наличия в них спама. В работе вводится квазивероятностная модель для классической эмпирической постановки задачи одноклассовой классификации. Произведена модификация, позволяющая накладывать различные требования отбора признаков. Построенные методы классификации проверяются вычислительными экспериментами на модельных и реальных данных.
%Решается задача автоматического разделения текстов по тематикам. Рассмотрены два подхода к решению задачи: вероятностный латентный семантический анализ и алгоритм латентного размещения Дирихле, основанные на различных вероятностных предположениях о текстах, однако обладающие схожей вычислительной техникой. Произведено сведение алгоритмов и их модификаций к новой обобщающей вычислительной схеме и построен новый алгоритм. Проанализировано качество и скорость сходимости построенного алгоритма в зависимости от внутренних параметров и числа тем, ассоциируемых с каждым словом в тексте. Результаты подтверждены численным экспериментом на реальных текстах. 
\end{abstract}
	\newpage 

	\section{Введение} 
		С широким развитием сети интернет и её проникновением в большую часть всех сфер жизни, у людей появилась возможность свободно обмениваться информацией и получать доступ к разнообразным ресурсом. 
Одним из наиболее распространенных способов общения людей через интернет является использование электронной почты \cite{}. 
В силу большой открытости этого канала связи с точки зрения возможности передачи любого сообщения произвольному пользователю он активно используется мошенниками, злоумышленниками и распространителями рекламных материалов. При этом создается не только повышенная нагрузка на техническую инфраструктуру, но и тратится время людей, которым приходится отделять полезную информацию, от всей остальной \cite{}. 
Поэтому задача автоматизации фильтрации электронной почты будет оставаться актуальной в течение всего времени её существования.

Задача фильтрации спама уже решалась многими методами \cite{}, однако они в большой степени являлись эвристическими и не имели под собой четкой вероятностной модели. 
Также проблемой является корректное составление обучающей выборки. 
Дело в том, что спам-письма зачастую шаблонны и имеют много общего в своей структуре, к тому же они широко доступны. 
Составить же обучающую выборку, содержащую письма, полезные для пользователей, гораздо сложнее по следующим причинам:
\begin{itemize}
	\item меньшая доступность,
	\item высокая разнородность,
	\item большое число шаблонных писем (разнообразные уведомления от сервисов).
\end{itemize}
По этим причинам предлагается использовать методы одноклассовой классификации \cite{}, чтобы отказаться от требования к обучающей выборки содержать достаточно широкой множество разнообразных представителей обоих классов.

В работе будет предложена квазивероятностная постановка задачи одноклассовой классификации. 
За счет такого подхода становятся яснее области применимости построенной модели и предъявляемые требования к данным.

Поскольку количество признаков, которые можно извлечь из текстов спам-писем, очень велико, то предлагается применить отбор признаков. 
На основе полученной вероятностной постановки задачи, строится новая вероятностная модель порождения объектов, в ходе оптимизации которой происходит требуемый отбор признаков.

Полученные методы построения одноклассовых классификаторов применяются к модельным и реальным данным.

%Современный интернет обладает широкой, распределенной и сложной архитектурой, в которой зачатую затруднительно найти новую требуемую информацию. Помочь пользователю решить задачу поиска призваны поисковые системы, которые автоматически сканируют интернет и выявляют наиболее подходящие пользователю по некоторым ключевым словам. 

%Алгоритм определения степени соответствия сайта запросу пользователя основан на множестве характеристик, а владельцы ресурсов заинтересованы, чтобы поисковые системы как можно выше оценивали их сайт. Зачастую

%Злоумышленники пытаются вывести подконтрольные им сайты в топ поисковой выдачи, искусственно изменяя характеристики сайта, видимые для поисковой машины. Такие действия ухудшают качество поиска и опасны для пользователя, поэтому необходимо либо полностью блокировать такие сайты, либо существенно опускать их в выдаче.

%Задача могла бы рассматривать как традиционная задача бинарной классификации, если бы знание поисковой машины характеристик сайта не влияло на эти характеристики. Например, если сайт признан по какой-либо причине опасным и удален из выдачи, у него резко (в разы) снижается посещаемость, при этом сайт мог исправить свою проблему, но не разбанится автоматически и посещаемость останется низкой. Одновременно с этим существуют признаки (количество переходов), которые, по-видимому, отражают степень полезность сайта, однако слабо зависят от действий поисковой системы. Задача выделения признаков, не зависящих от поисковой машины также представляет интерес в нашем исследовании.

%Идея: пусть каждый сайтовладелец в каждый момент времени $t\in T$ (см. презентацию) думает: сделать ему сайт хуже или лучше (спамерское поведение или, напротив, добропорядочное --- это и есть на самом деле класс $y\in Y$). В зависимости от этого он меняет характеристики своего сайта (тут чуть веселее, на самом деле: он одинаково меняет характеристики {\it всех} сайтов, которыми владеет, а это можно узнавать по whois-данным. {\it но я думаю, что в работе с этим заморачиваться не будем}). В таком случае наблюдаемые характеристики сайта (те, которые зависят) есть, на самом деле, функция от его наблюдаемого класса (то, что люди видят глазами). Надо додумать. Статей по теме не находил (да и не искал: придумал поздно вечером).
		\subsection{Постановка задачи}
Рассматривается задача восстановления числовой регрессии по обучающей выборке объектов, заданных своим признаковым описанием. 
Пусть $\Omega$~--- некоторое множество объектов реального мира, каждому элементу $\omega$ которого сопоставлено число $y(\omega)\in \mathbb Y \subset \mathbb R,$ 
а каждый объект $\omega \in \Omega$ представлен конечным множеством своих числовых признаков: $\mb x(\omega) = \cbr{x_1, \ldots, x_n}^T \in \mathbb X \subset \mathbb R^n.$
То есть предположим существование функций $y\colon \Omega \to \mathbb R$ и $\mb x \colon \Omega \to \mathbb R^n.$

Пусть наблюдателю известны значения этих функций в пределах некоторой конечной обучающей совокупности $\Omega^* = \fbr{\omega_j \cond j = 1,\ldots, N}$: $D = \fbr{\mb x(\omega), y(\omega) \cond \omega\in\Omega^*}.$
Ставится задача построить функцию $\hat y: \mathbb X \to \mathbb Y$ по известному $D$ такую, 
что функция $\mb x \circ \hat y \colon \Omega \to \mathbb Y$ будет как можно точнее описывать функцию $y.$

\subsection{Задача линейной регрессии}
Такая постановка задачи является весьма общей, и для получения конкретного решения требуется введение дополнительных ограничений. 
Будем искать решение в классе линейных функций, параметризованных следующим образом:
\begin{equation*}
	\hat y(\mb x) = \mb a^T \mb x + b, 
	\:\text{где $\mb a = (a_1, \ldots, a_n)^T \in \mathbb R^n$ и $b\in \mathbb R.$}
\end{equation*}

Введём обозначения:
\begin{align*}
	\mb x(\omega_j)	&= \mb x_j, j=1, \ldots, N  \text{ --- векторы признаков объектов,}\\
	\mb X 			&= \norm{x_{ij}}_{i=1, j=1}^{n,N} = \cbr{\mb x_1, \ldots, \mb x_N}^T \text{ --- матрица объекты-признаки,} \\
	y(\omega_j) 	&= y_j, j=1, \ldots, N, \\
	\hat y(x_j) 	&= \hat y_j, j=1, \ldots, N, \\
\end{align*}

В качестве меры близости функций $\mb x \circ \hat y$ и $y$ предлагается использовать средний квадрат их разницы на обучающей совокупности:
\begin{equation*}
	\frac1N\suml_{j=1}^N \cbr{y_j - \hat y_j}^2.
\end{equation*}

Тогда мы имеем следующую задачу оптимизации 
\begin{equation}
	\label{bestAB}
	\cbr{\hat{\mb a}, \hat b} = \argmin \suml_{j=1}^N\cbr{y_j - \mb a^T \mb x_j - b}^2, 
\end{equation}
однако такая постановка задачи может вызвать трудности при непосредственном поиске оптимальных параметров $\cbr{\hat{\mb a}, \hat b}$ в случае мультиколлинеарности используемых признаков, 
выражающееся в численной неустойчивости ответа и невозможности найти единственный оптимум, т.е. задача оказывается нерегулярной.

\subsection{Регуляризация Elastic Net}
С целью регуляризации задачи (\ref{bestAB}) в [] предлагается ввести штраф, называемый регуляризацией Elastic Net:
\begin{equation}
	\label{ENregularization}
	\beta \norm{\mb a}_{\mathbb R^2}^2 + \mu \norm{\mb a}_{\mathbb R} 
	= \beta \suml_{i=1}^n a_i^2 + \mu \suml_{i=1}^n \abs{a_i}.
\end{equation}

Что приводит к задаче оптимизации: 
\begin{equation}
	\label{basicEN}
	\sum_{i=1}^n\cbr{\beta a_i^2 + \mu \modul{a_i}} 
	+ \sum_{j=1}^N\cbr{y_j - \mb a^T \mb x_j - b}^2 
	\to \min_{\mb a, b}
\end{equation}
Эта задача не является квадратичной при $\mu > 0,$ и её решение мы обсудим позднее.

Отметим, что этот подход обобщает в себе регуляризации, используемые в методе лассо (LASSO, []) и методе опорных вектором (SVM) в задачах регрессии ([]).

Регуляризация (\ref{ENregularization}) в равной степени штрафует все компоненты вектора $\mb a\in \mathbb R^n,$ 
поэтому необходимо убедиться в том, что все признаки нормированы. 
Этого можно добиться линейным преобразованием признаков. 
Одновременно с этим предлагается центрировать признаки, чтобы избавиться от параметра $b$.

Тогда по обучающей совокупности $D$ линейным преобразованием построим совокупность
\begin{equation*}
	D^* = \fbr{\mb x^*_i, y^*_i \cond i = 1, \ldots, N}, 
\end{equation*}
удовлетворяющую следующим условиям:
\begin{align}
	\label{normalization-start}
	\mb 0 	&= \frac1N\suml_{i=1}^N\mb x^*_i, \\
	0 		&= \frac1N\suml_{i=1}^N y^*_i, \\
	\label{normalization-end}
	1 		&= \frac1N\suml_{i=1}^N\cbr{\mb x^*_{ij}}^2, j=1,\ldots, n.
\end{align}

Подходящее линейное преобразование задаётся соотношениями:
\begin{align*}
	x^*_{ij} 	&= \frac{x_{ij} - \bar x_j}{d_j}, \oi iN, \oi jn;\\
	\bar x_j 	&= \frac1N\suml_{i=1}^N x_{ij}, \oi jn;\\
	d_j 		&= \sqrt{\frac1N\suml_{i=1}^N \cbr{x_{ij}-\bar x_j}^2}, \oi jn;\\
	y^*_i 		&= y_i - \bar y, \oi iN;\\
	\bar y 		&= \frac1N\suml_{i=1}^N y_i. \\
\end{align*}

В дальнейшем для упрощения изложения мы будем везде считать, что это преобразование уже произведено, и условия (\ref{normalization-start}--\ref{normalization-end}) выполнены для совокупности $D.$

В таком случае задача (\ref{basicEN}) принимает вид
\begin{equation}
	\label{mainEN}
	\sum_{i=1}^n\cbr{\beta a_i^2 + \mu \modul{a_i}} 
	+ \sum_{j=1}^N\cbr{y_j - \mb a^T \mb x_j}^2 
	\to \min_{\mb a}.
\end{equation}

\subsection{Скользящий контроль для поиска оптимальных параметров регуляризации} % (fold)
\label{sub:intro:LOO}
Для подбора оптимальных значений параметров $\beta, \mu$ предлагается воспользоваться критерием скользящего контроля leave-one-out (LOO, []).
Для этого построим совокупность множеств обучения $D^{(k)}, \oi kN$ из $D$ путем поочерёдного исключения из него каждого из объектов: 
\begin{equation}
	D^{(k)}=\fbr{(\mb x_i, y_i) \cond \oi iN, i\ne k}.
\end{equation}
Для каждого полученного обучающего множества решим задачу (\ref{mainEN}) и получим оптимальный вектор 
\begin{equation*}
	\hat{\mb a}^{(k)} 
	= \argmin \suml_{i=1}^n\cbr{\beta a_i^2 + \mu \modul{a_i}} 
	+ \suml_{\oi jN, j\ne k}\cbr{y_j - \mb a^T \mb x_j}^2.
\end{equation*}

Критерием качества будет являться среднее ошибок алгоритмов на объектах исключённых из исходного множества обучения:
\begin{equation*}
	% \label{LOO}
	S_{LOO}(\beta, \mu) = \frac1N\suml_{k=1}^N \cbr{y_k - \mb x_k^T\hat{\mb a}^{(k)}}^2.
\end{equation*}

Таким образом критерий традиционной кросс-валидации принимает вид:
\begin{equation}
	\label{LOOcriteria}
	\cbr{\hat \beta, \hat \mu} = \argmin \frac1N\suml_{k=1}^N \cbr{y_k - \mb x_k^T\hat{\mb a}^{(k)}}^2.
\end{equation}

В такой форме явно видно, что получение оценки $S_{LOO}(\beta, \mu)$ требует построение всех векторов $\hat{\mb a}^{(k)}, \oi kN,$ то есть придётся обучаться $N$ раз. 
% subsection subsection_name (end)

% \subsection{Скользящий контроль для поиска оптимальных параметров регуляризации}

\subsection{Взвешивание объектов обучающей совокупности}
\label{sub:intro:weight}
В работе [] была предложена идея дифференциального скользящего контроля. 
Она заключается в присвоении каждому объекту $\omega_j, \oi jN$ из обучающей совокупности некоторого числа $p_j, \oi jN,$ которое означает вес объекта при обучении. 
Задача оптимизации (\ref{mainEN}) при этом принимает вид: 
\begin{equation}
	\label{weighedEN}
	\sum_{i=1}^n\cbr{\beta a_i^2 + \mu \modul{a_i}} 
	+ \sum_{j=1}^Np_j\cbr{y_j - \mb a^T \mb x_j}^2 
	\to \min_{\mb a, b}.
\end{equation}

Теперь оптимальный вектор $\hat{\mb a}$ зависит не только от параметров $\mu, \beta$, но и от вектора весов $\mb p = (p_1, \ldots, p_N).$
Тогда критерий (\ref{LOOcriteria}) можно переписать в виде:
\begin{align*}
	\cbr{\hat \beta, \hat \mu} &= \argmin \frac1N\suml_{k=1}^N \cbr{y_k - \mb x_k^T\hat{\mb a}(\beta, \mu, \mb e^{(i)})}^2,\\
	\text{ где } \mb e^{(i)} &= (e^{(i)}_1, \ldots, e^{(i)}_N)^T, e^{(i)}_k=\begcas{1, i\ne k, \\ 0, i = k.}
\end{align*}

Задача в форме (\ref{weighedEN}) обобщает ранее рассмотренные (\ref{basicEN}, \ref{mainEN}), поэтому далее мы будем решать именно её.

		% \subsection{Постановка задачи оптимизации} 	\begin{equation}
	\label{basicENdelta}
	\begin{cases}
		\suml_{i=1}^n\cbr{\beta a_i^2 + \mu \modul{a_i}} 
	+ \suml_{j=1}^N \delta_j^2 \to \min_{\mb a}, \\
	\delta_j = y_j - \suml_{i=1}^nx_{ij}a_i, j=1,\ldots,N. 
	\end{cases}
\end{equation}
	\section{Исследование задачи оптимизации}
		\subsection{Решение двойственной задачи}	Рассмотрим задачу (\ref{weighedEN}) и перепишем её в виде
\begin{equation}
	\label{weighedENdelta}
	\begin{cases}
		\suml_{i=1}^n\cbr{\beta a_i^2 + \mu \modul{a_i}} 
	+ \suml_{j=1}^N p_j\delta_j^2 \to \min_{\mb a}, \\
	\delta_j = y_j - \suml_{i=1}^nx_{ij}a_i, j=1,\ldots,N,
	\end{cases}
\end{equation}
где $\delta = \cbr{\delta_1, \ldots, \delta_N}^T$ --- вектор регрессионных остатков.

Функция Лагранжа:
\begin{equation}
	\label{LagrWEN}
	\begin{split}
		L(\mb a,\idots \delta1N, \idots \lambda1N) 
	&= \suml_{i=1}^n\cbr{\beta a_i^2 + \mu \modul{a_i}} + \suml_{j=1}^N p_j\delta_j^2 - \sum_{j=1}^N\lambda_j\cbr{\delta_j - y_j + \suml_{i=1}^nx_{ij}a_i} = \\
	&= \suml_{i=1}^n\cbr{\beta a_i^2 + \mu \modul{a_i} - \suml_{j=1}^N\lambda_jx_{ij}a_i} + \suml_{j=1}^N \cbr{p_j\delta_j^2  - \lambda_j\cbr{\delta_j - y_j}} \to \\
	&\to 
	\begin{cases}
		\min_{\mb a,\idots \delta1N}\\
		\max_{\idots \lambda1N}\\
	\end{cases}
	\end{split}
\end{equation}



\def\LiPart{L_i(a_i,\idots \lambda1N)}
\def\sumLambdaX{\suml_{j=1}^N\lambda_jx_{ij}}
\newcommand\diag[1]{\text{diag}\cbr{#1}}
Положим 
\begin{equation}
	\label{partialLagr}
	\LiPart = \beta a_i^2 + \mu \modul{a_i} - \cbr{\sumLambdaX}a_i \to \min_{a_i}.
\end{equation}

Каждая из функций $L_i$ является кусочно заданной комбинаций двух квадратных трехчленов:
\begin{equation}
	\label{partialLagrParted}
	\begin{split}
		\LiPart 
		&= \begin{cases}
			\beta a_i^2 + \mu a_i - \cbr{\sumLambdaX}a_i, &a_i > 0,\\
			0, &a_i = 0,\\
			\beta a_i^2 - \mu a_i - \cbr{\sumLambdaX}a_i, &a_i < 0,\\
		\end{cases} \quad= \\
		&= \begin{cases}
			\beta a_i^2 - \cbr{\sumLambdaX-\mu}a_i, &a_i > 0,\\
			0, &a_i = 0,\\
			\beta a_i^2 - \cbr{\sumLambdaX+\mu}a_i, &a_i < 0.\\
		\end{cases}	
	\end{split}
\end{equation}

Исходная постановка задачи Elastic Net \ref{basicEN} предполагает, что $\beta\ge0, \mu\ge0.$ Изучим случай $\beta > 0:$ 
\begin{equation}
	\label{hatAi}
	\begin{split}
		\hat a_i &= \argmin_{a_i} \LiPart = \\
				 &=	\argmin_{a_i} \fbr{
				 	\begin{aligned}
						\beta a_i^2 - \cbr{\sumLambdaX-\mu}a_i, &a_i > 0,\\
						0, \qquad\qquad &a_i = 0,\\
						\beta a_i^2 - \cbr{\sumLambdaX+\mu}a_i, &a_i < 0.\\
					\end{aligned}	
					}
	\end{split}
\end{equation}

Для поиска точки минимума рассмотрим 3 случая:
\begin{align}
	\label{hatAicases}
	         &\sumLambdaX \le -\mu 	&\Rightarrow \hat a_i &= \frac{\sumLambdaX+\mu}{2\beta}, \\
	-\mu <   &\sumLambdaX < \mu 	&\Rightarrow \hat a_i &= 0, \\
	\mu \le &\sumLambdaX 			&\Rightarrow \hat a_i &= \frac{\sumLambdaX-\mu}{2\beta},
\end{align}

% a x^2 + b x + c = a(x-b/{2a})^2 + c - b^2/{4a} = a(x-x_0)^2 + c - a x_0^2

Подставляя значения из \ref{hatAicases} в \ref{partialLagrParted}, получим решение задачи минимизации по $a_i$:
\begin{align}
	\label{hatLiSquared}
	\begin{split}
		\hat L_i(\idots \lambda1N) &= \min_{a_i}L_i(a_i, \idots \lambda1N) =
		\begin{cases}
			-\cfrac1{4\beta}\cbr{\sumLambdaX+\mu}^2, &\sumLambdaX \le -\mu,\\
			0, 										&-\mu < \sumLambdaX < \mu,\\
			-\cfrac1{4\beta}\cbr{\sumLambdaX-\mu}^2, &\mu\le\sumLambdaX,
		\end{cases} = \\
		&= -\frac1{4\beta}
		\begin{cases}
			\cbr{-\sumLambdaX-\mu}^2, &\sumLambdaX \le -\mu,\\
			0, 	 					&-\mu < \sumLambdaX < \mu,\\
			\cbr{\sumLambdaX-\mu}^2, &\mu\le\sumLambdaX.
		\end{cases} 	\\
		&= -\frac1{4\beta} \cbr{\min\fbr{\mu + \sumLambdaX, 0, \mu-\sumLambdaX}}^2
	\end{split}
\end{align}
(Здесь не сходится с результатами doc-файла)

Теперь перейдём к минимизации по $\cbr{\delta_1, \ldots, \delta_N}:$
\begin{equation}
	0 = \dd{}{\delta_j}L(\mb a,\idots \delta1N, \idots \lambda1N) = 2p_j\delta_j - \lambda_j 
\end{equation}
\begin{equation}
	\label{deltaJ}
	\delta_j = \frac{\lambda_j}{2p_j}
\end{equation}

Подставим результаты (\ref{hatLiSquared}) в (\ref{LagrWEN}):
\begin{align} 
	\hat L&\cbr{\mb a,\idots \delta1N, \idots \lambda1N} 
		= \suml_{i=1}^n \hat L_i(a_i,\idots \lambda1N)
		+ \suml_{j=1}^N \cbr{p_j\delta_j^2  - \lambda_j\cbr{\delta_j - y_j}} = \\
		&= -\frac1{4\beta} \suml_{i=1}^n \cbr{\min\fbr{\mu + \sumLambdaX, 0, \mu-\sumLambdaX}}^2
		+ \suml_{j=1}^N \cbr{p_j\cbr{\frac{\lambda_j}{2p_j}}^2  - \lambda_j\cbr{\frac{\lambda_j}{2p_j} - y_j}} = \\
		&= -\frac1{4\beta} \suml_{i=1}^n \cbr{\min\fbr{\mu + \sumLambdaX, 0, \mu - \sumLambdaX}}^2
		- \suml_{j=1}^N \cbr{ \frac{\lambda_j^2}{4p_j}  - \lambda_j y_j} 
\end{align}

Таким образом, задача (\ref{LagrWEN}) свелась к 
\begin{align}
-\frac1{4\beta} \suml_{i=1}^n \cbr{\min\fbr{\mu + \sumLambdaX, 0, \mu - \sumLambdaX}}^2
		- \suml_{j=1}^N \cbr{ \frac{\lambda_j^2}{4p_j}  - \lambda_j y_j} \to \max_{\idots \lambda1N}
\end{align}
Используя соотношение (\ref{deltaJ}), получим эквивалентную задачу:
\begin{align} 
	% \hat L&\cbr{\mb a,\idots \delta1N, \idots \lambda1N} 
	% 	= \suml_{i=1}^n \hat L_i(a_i,\idots \lambda1N)
	% 	+ \suml_{j=1}^N % \cbr{p_j\cbr{\frac{\lambda_j}{2p_j}}^2  - \lambda_j\cbr{\frac{\lambda_j}{2p_j} - y_j}} = \\
	% 		\cbr{p_j\delta_j^2  - \lambda_j\cbr{\delta_j - y_j}} = \\
	% 	&= -\frac1{4\beta} \suml_{i=1}^n \cbr{\min\fbr{\mu + \sumLambdaX, 0, \mu-\sumLambdaX}}^2
	% 	- \suml_{j=1}^N \cbr{p_j\delta_j^2  -  2 \delta_j p_j (\delta_j - y_j)} = \\
	% 	&= 
	-\frac1{4\beta} \suml_{i=1}^n \cbr{\min\fbr{\mu + \suml_{j=1}^N2 \delta_j p_jx_{ij}, 0, \mu-\suml_{j=1}^N2 \delta_j p_jx_{ij}}}^2
		- \suml_{j=1}^N \cbr{ \delta_j^2 p_j  - 2\delta_j p_j y_j} \to \max_{\idots \delta1N}
\end{align}
Отметим, что задача осталась задачей максимизации.

Положим 
\begin{align}
	W(\idots \delta1N) 
	= \frac1{4\beta} \suml_{i=1}^n \cbr{\min\fbr{\mu + \suml_{j=1}^N2 \delta_j p_jx_{ij}, 0, \mu-\suml_{j=1}^N2 \delta_j p_jx_{ij}}}^2
		+ \suml_{j=1}^N \cbr{ \delta_j^2 p_j  - 2\delta_j p_j y_j}
\end{align}

Итак, достаточно решить задачу 
\begin{align}
	W(\idots \delta1N) \to \min_{\idots \delta1N},
\end{align}
или же, используя матричные обозначения $\mb \delta = \cbr{\idots \delta1N}^T \in \mathbb R^N, \mb x_i = \cbr{x_{i1}, \ldots, , x_{iN}} \in \mathbb R^N, \mb y = \cbr{\idots y1N}^T \in \mathbb R^N, P = P^T= \diag{\sqrt{p_1}, \ldots, \sqrt{p_N}},$

\begin{align}
	W(\mb \delta) = \frac1{2\beta} \suml_{i=1}^n \cbr{\min\fbr{\frac\mu2 + \mb \delta^T P^2\mb x_i, 0, \frac\mu2- \mb \delta^T P^2\mb x_i}}^2
		+ (\mb \delta - \mb y)^T P (\mb \delta - \mb y) \to \min_{\mb \delta}.
\end{align}

Пусть $\hat{\mb \delta}$~--- решение этой задачи. Оно единственно в силу выпуклости функции $W(\mb \delta)$ по $\mb \delta$. Тогда оптимальное $\hat{\mb a}$ можно определить из соотношений  (\ref{deltaJ}) и (\ref{hatAicases}).
	\section{Скользящий контроль}
		\begin{align}
	\hat S(\lambda_1, \lambda_2) &= \frac1N\sum_{j=1}^N \hat\delta_{j, \lambda_1, \lambda_2}^2, \\
	\hat \delta_{j, \lambda_1, \lambda_2} &= y_j - \mb x_j^T \hat {\mb a}_{\lambda_1, \lambda_2}.
\end{align}
В используемых обозначениях кросс-валидацию можно рассматривать как обучение с поочередным присвоением объектам нулевых весов ($p_j=0$).
Усреденные квадратичные остатки 


\begin{align}
	\hat{S}_{\text{LOO}}&=\frac1N\sum_{j=1}^N\sqr{\hat{\delta}_{j, \lambda_1, \lambda_2}^{(j)}}, \\
	\hat \delta_{j, \lambda_1, \lambda_2}^{(j)} &= y_j - \mb x_j^T \hat {\mb a}_{\lambda_1, \lambda_2}^{(j)}.
\end{align}

\begin{align}
	% \dd{}{\mb p}\hat{S}_{\lambda_1, \lambda_2}&= \frac2N\cbr{y_j - x_j^T \dd{}{\mb p}\hat {\mb a}_{\lambda_1, \lambda_2}}.
	\dd{}{p_k}\hat{S}_{\lambda_1, \lambda_2}&= \frac2N\sum_{j=1}^N\cbr{y_j - x_j^T \dd{}{p_k}\hat {\mb a}_{\lambda_1, \lambda_2}}
\end{align}

	\section{Вычислительный эксперимент}
		Для экспериментального сравнения предложенных техник проведём вычислительный эксперимент на синтетических данных.

\subsection{Сравнение оценок скользящего контроля} % (fold)
В этом эксперименте данные сгенерированно $N=100$ объектов, обладающие $n=5$ признаками, по следующей вероятностной модели:
	\begin{align*}
		x_{ij} &\sim Exp(j), \: i=1,\ldots,N, \: j=1,\ldots,n,\\
		\mb a &= (0, 2, 0, -3, 0)^T, \\
		y_{i} &= \mb x_i^T \mb a + \eps_i, \eps_i \sim \Norm0{0{.}01}, i=1,\ldots,100. 
	\end{align*}

Вычислим оценки скользящего контроля по соотношениям (\ref{diffLOO}, \ref{LOOfixed}, \ref{LOO}) для определения оптимальных параметров регуляризации. Соответствующие графики приведены на рисунках (\ref{pic:diffLOO}, \ref{pic:LOOfixed}, \ref{pic:LOO}).

\begin{figure}[H]
	\centering
	\includegraphics[height=240px]{graph/LOO_I.png}
	\caption{$S_{LOO, \hat I_\lams}(\lams)$}
	\label{pic:LOOfixed}
\end{figure}

\begin{figure}[H]
	\centering
	\includegraphics[height=240px]{graph/diff_CV.png}
	\caption{$S_{Diff}(\lams)$}
	\label{pic:diffLOO}
\end{figure}

\begin{figure}[H]
	\centering
	\includegraphics[height=240px]{graph/honest_CV.png}
	\caption{$S_{LOO}(\lams)$}
	\label{pic:LOO}
\end{figure}

При этом оптимальные значения оказались равными:
\begin{align*}
	\hat S_{LOO, \hat I} 	&=0.0136, &\qquad	\hat \beta&=0.156, &\qquad	\hat \mu&=47.75 \\
	\hat S_{Diff} 			&=0.0119, &\qquad	\hat \beta&=0.160, &\qquad	\hat \mu&=48.55 \\
	\hat S_{LOO} 			&=0.0119, &\qquad	\hat \beta&=0.159, &\qquad	\hat \mu&=48.70.
\end{align*}
Видно, что результаты практически совпадают, что подтверждает теоретическое сравнение.

\subsection{Исследование зависимости от одного из параметров}

Одно из основных преимуществ Elastic Net перед гребневой регрессией заключается в способности учитывать кореллированные признаки совместно, а не выбирать из них всего один. 
Поэтому на практике часто оказывается разумно зафиксировать значение $\beta$ на некотором небольшом значении (в нашем эксперименте положим $\beta = 0{.}01$) и подбирать лишь параметр $\mu$. 
Обратим внимание, что таком использовании критерий дифференциалоной кросс-валидации всё так же вычислительно эффективен. 

На рисунке \ref{tikz:CV} изображена зависимость оценок скользящего контроля от параметра $\mu.$ Видно, что графики в целом совпадают, однако на графике для $S_{Diff}$ отчётливо выражены «ступеньки». 
Каждая такая ступень соответствует изменению множества информативных признаков. 
На $S_{LOO}$ таких ступенек нет, поскольку дифференциальный подход в отличие от классического может быть интерпретирован в терминах предположения о неизменности множества ненулевых компонент вектора оптимальных параметров регрессионной модели, 
однако такое предположение слишком сильно, чтобы быть верным, что и вызывает скачки функции при измении множества информативных признаков.

Также обратим внимание на то, что $S(\mb e, \lams)$ в сумме не оказывает существенного вклада в поиск оптимальных структурных параметров (см. рис. \ref{tikz:newDiffLOO}), что приводит к иному виду критерия дифференциальной кросс-валидации, который, однако, имеет другую интерпретацию и требует дальнейшего исследования: 
\begin{equation*}
	S^\prime_{Diff}(\lams) 
	= -\frac1N\suml_{k=1}^N \left. 
		\dd{\hat{\delta}_k^2(\mb p, \lams)}{p_k} 
	\right|_{\mb p=\mb e} \to \min_{\lams}.
\end{equation*}

\begin{figure}[H]
	\centering
	\begin{tikzpicture}%[x=3cm,y=3cm]
		\begin{axis}[xlabel=$\mu$, ylabel=$S$, enlargelimits=false,
		%title={Модельные данные},
		scaled x ticks=false,
		xticklabel style={/pgf/number format/fixed, /pgf/number format/precision=3}]
		\addplot[blue, mark=none, very thick] coordinates {(190.00,0.0115864)(190.01,0.0115856)(190.02,0.0115848)(190.03,0.0115841)(190.04,0.0115833)(190.05,0.0115826)(190.06,0.0115818)(190.07,0.0115811)(190.08,0.0115803)(190.09,0.0115795)(190.10,0.0115787)(190.11,0.0115780)(190.12,0.0115772)(190.13,0.0115764)(190.14,0.0115757)(190.15,0.0115750)(190.16,0.0115742)(190.17,0.0115734)(190.18,0.0115727)(190.19,0.0115719)(190.20,0.0115711)(190.21,0.0115703)(190.22,0.0115695)(190.23,0.0115687)(190.24,0.0115680)(190.25,0.0115672)(190.26,0.0115664)(190.27,0.0115656)(190.28,0.0115648)(190.29,0.0115640)(190.30,0.0115632)(190.31,0.0115621)(190.32,0.0115613)(190.33,0.0115605)(190.34,0.0115598)(190.35,0.0115590)(190.36,0.0115582)(190.37,0.0115574)(190.38,0.0115566)(190.39,0.0115559)(190.40,0.0115551)(190.41,0.0115543)(190.42,0.0115534)(190.43,0.0115526)(190.44,0.0115518)(190.45,0.0115510)(190.46,0.0115503)(190.47,0.0115495)(190.48,0.0115487)(190.49,0.0115479)(190.50,0.0115470)(190.51,0.0115463)(190.52,0.0115455)(190.53,0.0115447)(190.54,0.0115439)(190.55,0.0115431)(190.56,0.0115423)(190.57,0.0115415)(190.58,0.0115408)(190.59,0.0115400)(190.60,0.0115390)(190.61,0.0115382)(190.62,0.0115375)(190.63,0.0115367)(190.64,0.0115360)(190.65,0.0115352)(190.66,0.0115344)(190.67,0.0115336)(190.68,0.0115329)(190.69,0.0115321)(190.70,0.0115310)(190.71,0.0115302)(190.72,0.0115294)(190.73,0.0115286)(190.74,0.0115279)(190.75,0.0115271)(190.76,0.0115263)(190.77,0.0115255)(190.78,0.0115247)(190.79,0.0115239)(190.80,0.0115231)(190.81,0.0115221)(190.82,0.0115213)(190.83,0.0115203)(190.84,0.0115196)(190.85,0.0115189)(190.86,0.0115182)(190.87,0.0115175)(190.88,0.0115166)(190.89,0.0115159)(190.90,0.0115152)(190.91,0.0115145)(190.92,0.0115138)(190.93,0.0115131)(190.94,0.0115124)(190.95,0.0115117)(190.96,0.0115109)(190.97,0.0115102)(190.98,0.0115095)(190.99,0.0115088)(191.00,0.0115081)(191.01,0.0115074)(191.02,0.0115067)(191.03,0.0115059)(191.04,0.0115052)(191.05,0.0115045)(191.06,0.0115039)(191.07,0.0115032)(191.08,0.0115022)(191.09,0.0115018)(191.10,0.0115011)(191.11,0.0115003)(191.12,0.0114996)(191.13,0.0114989)(191.14,0.0114981)(191.15,0.0114974)(191.16,0.0114966)(191.17,0.0114959)(191.18,0.0114952)(191.19,0.0114944)(191.20,0.0114937)(191.21,0.0114929)(191.22,0.0114922)(191.23,0.0114914)(191.24,0.0114907)(191.25,0.0114899)(191.26,0.0114892)(191.27,0.0114884)(191.28,0.0114879)(191.29,0.0114872)(191.30,0.0114864)(191.31,0.0114856)(191.32,0.0114849)(191.33,0.0114841)(191.34,0.0114833)(191.35,0.0114826)(191.36,0.0114818)(191.37,0.0114810)(191.38,0.0114802)(191.39,0.0114793)(191.40,0.0114785)(191.41,0.0114777)(191.42,0.0114770)(191.43,0.0114761)(191.44,0.0114753)(191.45,0.0114745)(191.46,0.0114734)(191.47,0.0114726)(191.48,0.0114719)(191.49,0.0114712)(191.50,0.0114704)(191.51,0.0114697)(191.52,0.0114690)(191.53,0.0114682)(191.54,0.0114675)(191.55,0.0114668)(191.56,0.0114661)(191.57,0.0114653)(191.58,0.0114646)(191.59,0.0114639)(191.60,0.0114632)(191.61,0.0114624)(191.62,0.0114618)(191.63,0.0114611)(191.64,0.0114603)(191.65,0.0114596)(191.66,0.0114588)(191.67,0.0114581)(191.68,0.0114574)(191.69,0.0114566)(191.70,0.0114559)(191.71,0.0114552)(191.72,0.0114544)(191.73,0.0114537)(191.74,0.0114529)(191.75,0.0114516)(191.76,0.0114509)(191.77,0.0114502)(191.78,0.0114495)(191.79,0.0114488)(191.80,0.0114481)(191.81,0.0114475)(191.82,0.0114468)(191.83,0.0114461)(191.84,0.0114454)(191.85,0.0114447)(191.86,0.0114440)(191.87,0.0114433)(191.88,0.0114424)(191.89,0.0114418)(191.90,0.0114409)(191.91,0.0114403)(191.92,0.0114397)(191.93,0.0114390)(191.94,0.0114384)(191.95,0.0114376)(191.96,0.0114371)(191.97,0.0114365)(191.98,0.0114363)(191.99,0.0114357)(192.00,0.0114351)(192.01,0.0114345)(192.02,0.0114340)(192.03,0.0114334)(192.04,0.0114328)(192.05,0.0114317)(192.06,0.0114311)(192.07,0.0114305)(192.08,0.0114299)(192.09,0.0114293)(192.10,0.0114287)(192.11,0.0114281)(192.12,0.0114275)(192.13,0.0114269)(192.14,0.0114263)(192.15,0.0114257)(192.16,0.0114251)(192.17,0.0114245)(192.18,0.0114239)(192.19,0.0114232)(192.20,0.0114226)(192.21,0.0114220)(192.22,0.0114213)(192.23,0.0114207)(192.24,0.0114201)(192.25,0.0114195)(192.26,0.0114188)(192.27,0.0114182)(192.28,0.0114176)(192.29,0.0114170)(192.30,0.0114164)(192.31,0.0114158)(192.32,0.0114152)(192.33,0.0114145)(192.34,0.0114139)(192.35,0.0114133)(192.36,0.0114122)(192.37,0.0114117)(192.38,0.0114111)(192.39,0.0114105)(192.40,0.0114099)(192.41,0.0114093)(192.42,0.0114086)(192.43,0.0114080)(192.44,0.0114075)(192.45,0.0114070)(192.46,0.0114065)(192.47,0.0114062)(192.48,0.0114055)(192.49,0.0114049)(192.50,0.0114043)(192.51,0.0114037)(192.52,0.0114031)(192.53,0.0114024)(192.54,0.0114016)(192.55,0.0114010)(192.56,0.0114004)(192.57,0.0113997)(192.58,0.0113991)(192.59,0.0113983)(192.60,0.0113977)(192.61,0.0113971)(192.62,0.0113965)(192.63,0.0113959)(192.64,0.0113952)(192.65,0.0113946)(192.66,0.0113940)(192.67,0.0113934)(192.68,0.0113928)(192.69,0.0113921)(192.70,0.0113915)(192.71,0.0113909)(192.72,0.0113903)(192.73,0.0113898)(192.74,0.0113892)(192.75,0.0113887)(192.76,0.0113881)(192.77,0.0113874)(192.78,0.0113869)(192.79,0.0113862)(192.80,0.0113854)(192.81,0.0113848)(192.82,0.0113841)(192.83,0.0113835)(192.84,0.0113829)(192.85,0.0113823)(192.86,0.0113816)(192.87,0.0113810)(192.88,0.0113803)(192.89,0.0113796)(192.90,0.0113790)(192.91,0.0113784)(192.92,0.0113776)(192.93,0.0113768)(192.94,0.0113761)(192.95,0.0113754)(192.96,0.0113747)(192.97,0.0113739)(192.98,0.0113732)(192.99,0.0113729)(193.00,0.0113720)(193.01,0.0113712)(193.02,0.0113704)(193.03,0.0113696)(193.04,0.0113689)(193.05,0.0113681)(193.06,0.0113674)(193.07,0.0113666)(193.08,0.0113659)(193.09,0.0113649)(193.10,0.0113637)(193.11,0.0113630)(193.12,0.0113622)(193.13,0.0113614)(193.14,0.0113608)(193.15,0.0113601)(193.16,0.0113594)(193.17,0.0113587)(193.18,0.0113581)(193.19,0.0113574)(193.20,0.0113567)(193.21,0.0113561)(193.22,0.0113553)(193.23,0.0113546)(193.24,0.0113539)(193.25,0.0113532)(193.26,0.0113525)(193.27,0.0113518)(193.28,0.0113511)(193.29,0.0113504)(193.30,0.0113497)(193.31,0.0113488)(193.32,0.0113480)(193.33,0.0113473)(193.34,0.0113466)(193.35,0.0113457)(193.36,0.0113449)(193.37,0.0113442)(193.38,0.0113436)(193.39,0.0113429)(193.40,0.0113421)(193.41,0.0113414)(193.42,0.0113407)(193.43,0.0113400)(193.44,0.0113391)(193.45,0.0113384)(193.46,0.0113378)(193.47,0.0113367)(193.48,0.0113361)(193.49,0.0113355)(193.50,0.0113348)(193.51,0.0113342)(193.52,0.0113335)(193.53,0.0113329)(193.54,0.0113322)(193.55,0.0113316)(193.56,0.0113309)(193.57,0.0113302)(193.58,0.0113295)(193.59,0.0113289)(193.60,0.0113283)(193.61,0.0113277)(193.62,0.0113271)(193.63,0.0113265)(193.64,0.0113259)(193.65,0.0113253)(193.66,0.0113247)(193.67,0.0113241)(193.68,0.0113235)(193.69,0.0113229)(193.70,0.0113222)(193.71,0.0113215)(193.72,0.0113209)(193.73,0.0113202)(193.74,0.0113195)(193.75,0.0113188)(193.76,0.0113181)(193.77,0.0113173)(193.78,0.0113166)(193.79,0.0113159)(193.80,0.0113151)(193.81,0.0113145)(193.82,0.0113138)(193.83,0.0113131)(193.84,0.0113123)(193.85,0.0113116)(193.86,0.0113108)(193.87,0.0113101)(193.88,0.0113095)(193.89,0.0113087)(193.90,0.0113079)(193.91,0.0113072)(193.92,0.0113064)(193.93,0.0113056)(193.94,0.0113048)(193.95,0.0113040)(193.96,0.0113032)(193.97,0.0113024)(193.98,0.0113017)(193.99,0.0113009)(194.00,0.0113001)(194.01,0.0112994)(194.02,0.0112986)(194.03,0.0112978)(194.04,0.0112970)(194.05,0.0112963)(194.06,0.0112951)(194.07,0.0112944)(194.08,0.0112936)(194.09,0.0112924)(194.10,0.0112917)(194.11,0.0112909)(194.12,0.0112893)(194.13,0.0112887)(194.14,0.0112882)(194.15,0.0112876)(194.16,0.0112870)(194.17,0.0112864)(194.18,0.0112858)(194.19,0.0112854)(194.20,0.0112848)(194.21,0.0112838)(194.22,0.0112833)(194.23,0.0112827)(194.24,0.0112820)(194.25,0.0112814)(194.26,0.0112811)(194.27,0.0112804)(194.28,0.0112799)(194.29,0.0112792)(194.30,0.0112785)(194.31,0.0112779)(194.32,0.0112772)(194.33,0.0112765)(194.34,0.0112758)(194.35,0.0112751)(194.36,0.0112745)(194.37,0.0112738)(194.38,0.0112732)(194.39,0.0112725)(194.40,0.0112719)(194.41,0.0112712)(194.42,0.0112706)(194.43,0.0112700)(194.44,0.0112694)(194.45,0.0112688)(194.46,0.0112682)(194.47,0.0112676)(194.48,0.0112663)(194.49,0.0112658)(194.50,0.0112654)(194.51,0.0112649)(194.52,0.0112644)(194.53,0.0112639)(194.54,0.0112633)(194.55,0.0112628)(194.56,0.0112623)(194.57,0.0112618)(194.58,0.0112614)(194.59,0.0112609)(194.60,0.0112603)(194.61,0.0112587)(194.62,0.0112583)(194.63,0.0112578)(194.64,0.0112574)(194.65,0.0112563)(194.66,0.0112559)(194.67,0.0112556)(194.68,0.0112551)(194.69,0.0112547)(194.70,0.0112543)(194.71,0.0112540)(194.72,0.0112536)(194.73,0.0112532)(194.74,0.0112522)(194.75,0.0112519)(194.76,0.0112515)(194.77,0.0112512)(194.78,0.0112508)(194.79,0.0112504)(194.80,0.0112500)(194.81,0.0112497)(194.82,0.0112493)(194.83,0.0112489)(194.84,0.0112485)(194.85,0.0112481)(194.86,0.0112477)(194.87,0.0112473)(194.88,0.0112469)(194.89,0.0112464)(194.90,0.0112461)(194.91,0.0112457)(194.92,0.0112454)(194.93,0.0112451)(194.94,0.0112448)(194.95,0.0112445)(194.96,0.0112443)(194.97,0.0112441)(194.98,0.0112439)(194.99,0.0112437)(195.00,0.0112435)(195.01,0.0112435)(195.02,0.0112432)(195.03,0.0112429)(195.04,0.0112426)(195.05,0.0112422)(195.06,0.0112420)(195.07,0.0112417)(195.08,0.0112414)(195.09,0.0112411)(195.10,0.0112409)(195.11,0.0112408)(195.12,0.0112407)(195.13,0.0112407)(195.14,0.0112407)(195.15,0.0112407)(195.16,0.0112406)(195.17,0.0112406)(195.18,0.0112406)(195.19,0.0112405)(195.20,0.0112405)(195.21,0.0112404)(195.22,0.0112404)(195.23,0.0112397)(195.24,0.0112398)(195.25,0.0112400)(195.26,0.0112402)(195.27,0.0112403)(195.28,0.0112404)(195.29,0.0112405)(195.30,0.0112407)(195.31,0.0112410)(195.32,0.0112412)(195.33,0.0112414)(195.34,0.0112417)(195.35,0.0112420)(195.36,0.0112423)(195.37,0.0112426)(195.38,0.0112429)(195.39,0.0112432)(195.40,0.0112435)(195.41,0.0112439)(195.42,0.0112443)(195.43,0.0112447)(195.44,0.0112451)(195.45,0.0112455)(195.46,0.0112459)(195.47,0.0112463)(195.48,0.0112468)(195.49,0.0112472)(195.50,0.0112476)(195.51,0.0112481)(195.52,0.0112486)(195.53,0.0112490)(195.54,0.0112494)(195.55,0.0112499)(195.56,0.0112503)(195.57,0.0112507)(195.58,0.0112512)(195.59,0.0112516)(195.60,0.0112521)(195.61,0.0112525)(195.62,0.0112530)(195.63,0.0112531)(195.64,0.0112537)(195.65,0.0112543)(195.66,0.0112550)(195.67,0.0112558)(195.68,0.0112565)(195.69,0.0112573)(195.70,0.0112580)(195.71,0.0112589)(195.72,0.0112596)(195.73,0.0112603)(195.74,0.0112611)(195.75,0.0112619)(195.76,0.0112626)(195.77,0.0112634)(195.78,0.0112642)(195.79,0.0112649)(195.80,0.0112657)(195.81,0.0112665)(195.82,0.0112673)(195.83,0.0112681)(195.84,0.0112689)(195.85,0.0112697)(195.86,0.0112705)(195.87,0.0112714)(195.88,0.0112722)(195.89,0.0112730)(195.90,0.0112739)(195.91,0.0112747)(195.92,0.0112756)(195.93,0.0112764)(195.94,0.0112773)(195.95,0.0112781)(195.96,0.0112790)(195.97,0.0112799)(195.98,0.0112808)(195.99,0.0112817)(196.00,0.0112826)(196.01,0.0112835)(196.02,0.0112845)(196.03,0.0112854)(196.04,0.0112863)(196.05,0.0112873)(196.06,0.0112882)(196.07,0.0112892)(196.08,0.0112901)(196.09,0.0112911)(196.10,0.0112921)(196.11,0.0112931)(196.12,0.0112941)(196.13,0.0112951)(196.14,0.0112961)(196.15,0.0112971)(196.16,0.0112982)(196.17,0.0112992)(196.18,0.0113002)(196.19,0.0113013)(196.20,0.0113024)(196.21,0.0113034)(196.22,0.0113045)(196.23,0.0113056)(196.24,0.0113067)(196.25,0.0113080)(196.26,0.0113094)(196.27,0.0113108)(196.28,0.0113122)(196.29,0.0113137)(196.30,0.0113151)(196.31,0.0113166)(196.32,0.0113181)(196.33,0.0113196)(196.34,0.0113211)(196.35,0.0113226)(196.36,0.0113241)(196.37,0.0113256)(196.38,0.0113272)(196.39,0.0113287)(196.40,0.0113303)(196.41,0.0113319)(196.42,0.0113335)(196.43,0.0113351)(196.44,0.0113367)(196.45,0.0113384)(196.46,0.0113400)(196.47,0.0113417)(196.48,0.0113433)(196.49,0.0113450)(196.50,0.0113467)(196.51,0.0113485)(196.52,0.0113502)(196.53,0.0113519)(196.54,0.0113537)(196.55,0.0113555)(196.56,0.0113573)(196.57,0.0113591)(196.58,0.0113609)(196.59,0.0113627)(196.60,0.0113646)(196.61,0.0113664)(196.62,0.0113683)(196.63,0.0113702)(196.64,0.0113721)(196.65,0.0113741)(196.66,0.0113760)(196.67,0.0113780)(196.68,0.0113799)(196.69,0.0113819)(196.70,0.0113840)(196.71,0.0113860)(196.72,0.0113880)(196.73,0.0113901)(196.74,0.0113922)(196.75,0.0113943)(196.76,0.0113964)(196.77,0.0113986)(196.78,0.0114007)(196.79,0.0114029)(196.80,0.0114051)(196.81,0.0114073)(196.82,0.0114096)(196.83,0.0114118)(196.84,0.0114141)(196.85,0.0114164)(196.86,0.0114187)(196.87,0.0114211)(196.88,0.0114234)(196.89,0.0114258)(196.90,0.0114282)(196.91,0.0114307)(196.92,0.0114331)(196.93,0.0114356)(196.94,0.0114381)(196.95,0.0114406)(196.96,0.0114432)(196.97,0.0114457)(196.98,0.0114483)(196.99,0.0114510)(197.00,0.0114536)(197.01,0.0114563)(197.02,0.0114590)(197.03,0.0114617)(197.04,0.0114645)(197.05,0.0114672)(197.06,0.0114700)(197.07,0.0114729)(197.08,0.0114757)(197.09,0.0114786)(197.10,0.0114816)(197.11,0.0114845)(197.12,0.0114875)(197.13,0.0114905)(197.14,0.0114935)(197.15,0.0114966)(197.16,0.0114997)(197.17,0.0115028)(197.18,0.0115060)(197.19,0.0115092)(197.20,0.0115124)(197.21,0.0115157)(197.22,0.0115190)(197.23,0.0115223)(197.24,0.0115257)(197.25,0.0115291)(197.26,0.0115325)(197.27,0.0115360)(197.28,0.0115395)(197.29,0.0115430)(197.30,0.0115466)(197.31,0.0115502)(197.32,0.0115539)(197.33,0.0115576)(197.34,0.0115613)(197.35,0.0115651)(197.36,0.0115689)(197.37,0.0115728)(197.38,0.0115767)(197.39,0.0115807)(197.40,0.0115847)(197.41,0.0115887)(197.42,0.0115928)(197.43,0.0115969)(197.44,0.0116011)(197.45,0.0116053)(197.46,0.0116096)(197.47,0.0116139)(197.48,0.0116183)(197.49,0.0116227)(197.50,0.0116272)(197.51,0.0116318)(197.52,0.0116363)(197.53,0.0116410)(197.54,0.0116457)(197.55,0.0116504)(197.56,0.0116552)(197.57,0.0116601)(197.58,0.0116650)(197.59,0.0116700)(197.60,0.0116750)(197.61,0.0116801)(197.62,0.0116853)(197.63,0.0116905)(197.64,0.0116958)(197.65,0.0117011)(197.66,0.0117065)(197.67,0.0117120)(197.68,0.0117176)(197.69,0.0117232)(197.70,0.0117289)(197.71,0.0117347)(197.72,0.0117405)(197.73,0.0117464)(197.74,0.0117524)(197.75,0.0117585)(197.76,0.0117647)(197.77,0.0117709)(197.78,0.0117772)(197.79,0.0117836)(197.80,0.0117901)(197.81,0.0117966)(197.82,0.0118033)(197.83,0.0118100)(197.84,0.0118169)(197.85,0.0118238)(197.86,0.0118308)(197.87,0.0118379)(197.88,0.0118452)(197.89,0.0118525)(197.90,0.0118599)(197.91,0.0118674)(197.92,0.0118750)(197.93,0.0118828)(197.94,0.0118906)(197.95,0.0118986)(197.96,0.0119066)(197.97,0.0119148)(197.98,0.0119231)(197.99,0.0119315)(198.00,0.0119401)};
		\addplot[red,  mark=none, very thick] coordinates {(190.00,0.0117790)(190.01,0.0117786)(190.02,0.0117782)(190.03,0.0117777)(190.04,0.0117773)(190.05,0.0117769)(190.06,0.0117765)(190.07,0.0117760)(190.08,0.0117756)(190.09,0.0117752)(190.10,0.0117747)(190.11,0.0117743)(190.12,0.0117739)(190.13,0.0117734)(190.14,0.0117730)(190.15,0.0117726)(190.16,0.0117722)(190.17,0.0117717)(190.18,0.0117713)(190.19,0.0117709)(190.20,0.0117704)(190.21,0.0117700)(190.22,0.0117696)(190.23,0.0117692)(190.24,0.0117687)(190.25,0.0117683)(190.26,0.0117679)(190.27,0.0117674)(190.28,0.0117670)(190.29,0.0117666)(190.30,0.0117661)(190.31,0.0117657)(190.32,0.0117653)(190.33,0.0117649)(190.34,0.0117644)(190.35,0.0117640)(190.36,0.0115294)(190.37,0.0115292)(190.38,0.0115290)(190.39,0.0115288)(190.40,0.0115286)(190.41,0.0115284)(190.42,0.0115282)(190.43,0.0115280)(190.44,0.0115278)(190.45,0.0115276)(190.46,0.0115274)(190.47,0.0115272)(190.48,0.0115270)(190.49,0.0115268)(190.50,0.0115266)(190.51,0.0115264)(190.52,0.0115261)(190.53,0.0115259)(190.54,0.0115259)(190.55,0.0115257)(190.56,0.0115255)(190.57,0.0115253)(190.58,0.0115251)(190.59,0.0115249)(190.60,0.0115247)(190.61,0.0115245)(190.62,0.0115243)(190.63,0.0115241)(190.64,0.0115239)(190.65,0.0115237)(190.66,0.0115235)(190.67,0.0115233)(190.68,0.0115230)(190.69,0.0115228)(190.70,0.0115226)(190.71,0.0115224)(190.72,0.0115222)(190.73,0.0115220)(190.74,0.0115218)(190.75,0.0115216)(190.76,0.0115214)(190.77,0.0115212)(190.78,0.0115210)(190.79,0.0115208)(190.80,0.0115206)(190.81,0.0115204)(190.82,0.0115202)(190.83,0.0115200)(190.84,0.0115198)(190.85,0.0115196)(190.86,0.0115194)(190.87,0.0115192)(190.88,0.0115190)(190.89,0.0115188)(190.90,0.0115186)(190.91,0.0115184)(190.92,0.0115182)(190.93,0.0115180)(190.94,0.0115178)(190.95,0.0115176)(190.96,0.0115174)(190.97,0.0115172)(190.98,0.0115170)(190.99,0.0115168)(191.00,0.0115166)(191.01,0.0115164)(191.02,0.0115162)(191.03,0.0115160)(191.04,0.0115158)(191.05,0.0115156)(191.06,0.0115154)(191.07,0.0115152)(191.08,0.0115150)(191.09,0.0115148)(191.10,0.0115146)(191.11,0.0115144)(191.12,0.0115142)(191.13,0.0115140)(191.14,0.0115138)(191.15,0.0115136)(191.16,0.0115134)(191.17,0.0115132)(191.18,0.0115130)(191.19,0.0115128)(191.20,0.0115126)(191.21,0.0115124)(191.22,0.0115122)(191.23,0.0115120)(191.24,0.0115118)(191.25,0.0115116)(191.26,0.0115114)(191.27,0.0115112)(191.28,0.0115110)(191.29,0.0115109)(191.30,0.0115107)(191.31,0.0115105)(191.32,0.0115103)(191.33,0.0115101)(191.34,0.0115099)(191.35,0.0115097)(191.36,0.0115095)(191.37,0.0115093)(191.38,0.0115091)(191.39,0.0115089)(191.40,0.0115087)(191.41,0.0115085)(191.42,0.0115083)(191.43,0.0115081)(191.44,0.0115079)(191.45,0.0115077)(191.46,0.0115075)(191.47,0.0115073)(191.48,0.0115071)(191.49,0.0115070)(191.50,0.0115068)(191.51,0.0115066)(191.52,0.0115064)(191.53,0.0115062)(191.54,0.0115060)(191.55,0.0115058)(191.56,0.0115056)(191.57,0.0115054)(191.58,0.0115052)(191.59,0.0115050)(191.60,0.0115048)(191.61,0.0115047)(191.62,0.0115045)(191.63,0.0115043)(191.64,0.0115041)(191.65,0.0115039)(191.66,0.0115037)(191.67,0.0115035)(191.68,0.0115033)(191.69,0.0115031)(191.70,0.0115030)(191.71,0.0115028)(191.72,0.0115026)(191.73,0.0115024)(191.74,0.0115022)(191.75,0.0115020)(191.76,0.0115018)(191.77,0.0115017)(191.78,0.0115015)(191.79,0.0115013)(191.80,0.0115011)(191.81,0.0115009)(191.82,0.0115007)(191.83,0.0115005)(191.84,0.0115004)(191.85,0.0115002)(191.86,0.0115000)(191.87,0.0114998)(191.88,0.0114996)(191.89,0.0114995)(191.90,0.0114993)(191.91,0.0114991)(191.92,0.0114989)(191.93,0.0114987)(191.94,0.0114985)(191.95,0.0114984)(191.96,0.0114982)(191.97,0.0114980)(191.98,0.0114978)(191.99,0.0114977)(192.00,0.0114975)(192.01,0.0114973)(192.02,0.0114971)(192.03,0.0114970)(192.04,0.0114968)(192.05,0.0114966)(192.06,0.0114964)(192.07,0.0114963)(192.08,0.0114961)(192.09,0.0114959)(192.10,0.0114957)(192.11,0.0114956)(192.12,0.0114954)(192.13,0.0114952)(192.14,0.0114950)(192.15,0.0114949)(192.16,0.0114947)(192.17,0.0114945)(192.18,0.0114944)(192.19,0.0114942)(192.20,0.0114940)(192.21,0.0114939)(192.22,0.0114937)(192.23,0.0114935)(192.24,0.0114934)(192.25,0.0114932)(192.26,0.0114930)(192.27,0.0114929)(192.28,0.0114927)(192.29,0.0114925)(192.30,0.0114924)(192.31,0.0114922)(192.32,0.0114921)(192.33,0.0114919)(192.34,0.0114917)(192.35,0.0114916)(192.36,0.0114914)(192.37,0.0114913)(192.38,0.0114911)(192.39,0.0114909)(192.40,0.0114908)(192.41,0.0114906)(192.42,0.0114905)(192.43,0.0114903)(192.44,0.0114902)(192.45,0.0114900)(192.46,0.0114899)(192.47,0.0114897)(192.48,0.0114896)(192.49,0.0114894)(192.50,0.0114893)(192.51,0.0114891)(192.52,0.0114890)(192.53,0.0114888)(192.54,0.0114887)(192.55,0.0114885)(192.56,0.0114884)(192.57,0.0114882)(192.58,0.0114881)(192.59,0.0114879)(192.60,0.0114878)(192.61,0.0114877)(192.62,0.0114875)(192.63,0.0114874)(192.64,0.0114872)(192.65,0.0114871)(192.66,0.0114870)(192.67,0.0114868)(192.68,0.0114867)(192.69,0.0114866)(192.70,0.0114864)(192.71,0.0114863)(192.72,0.0114862)(192.73,0.0114860)(192.74,0.0114859)(192.75,0.0114858)(192.76,0.0114856)(192.77,0.0114855)(192.78,0.0114854)(192.79,0.0114853)(192.80,0.0114851)(192.81,0.0114850)(192.82,0.0114849)(192.83,0.0114848)(192.84,0.0114846)(192.85,0.0114845)(192.86,0.0114844)(192.87,0.0114843)(192.88,0.0114842)(192.89,0.0114841)(192.90,0.0114839)(192.91,0.0114838)(192.92,0.0114837)(192.93,0.0114836)(192.94,0.0114835)(192.95,0.0114834)(192.96,0.0114833)(192.97,0.0114832)(192.98,0.0114831)(192.99,0.0112692)(193.00,0.0112692)(193.01,0.0112693)(193.02,0.0112693)(193.03,0.0112693)(193.04,0.0112693)(193.05,0.0112694)(193.06,0.0112694)(193.07,0.0112694)(193.08,0.0112695)(193.09,0.0112695)(193.10,0.0112695)(193.11,0.0112695)(193.12,0.0112695)(193.13,0.0112696)(193.14,0.0112696)(193.15,0.0112697)(193.16,0.0112697)(193.17,0.0112697)(193.18,0.0112698)(193.19,0.0112698)(193.20,0.0112699)(193.21,0.0112699)(193.22,0.0112700)(193.23,0.0112700)(193.24,0.0112701)(193.25,0.0112701)(193.26,0.0112702)(193.27,0.0112702)(193.28,0.0112703)(193.29,0.0112703)(193.30,0.0112704)(193.31,0.0112704)(193.32,0.0112705)(193.33,0.0112706)(193.34,0.0112706)(193.35,0.0112707)(193.36,0.0112707)(193.37,0.0112708)(193.38,0.0112709)(193.39,0.0112709)(193.40,0.0112710)(193.41,0.0112711)(193.42,0.0112711)(193.43,0.0112712)(193.44,0.0112713)(193.45,0.0112713)(193.46,0.0112714)(193.47,0.0112715)(193.48,0.0112716)(193.49,0.0112716)(193.50,0.0112717)(193.51,0.0112718)(193.52,0.0112719)(193.53,0.0112719)(193.54,0.0112720)(193.55,0.0112721)(193.56,0.0112722)(193.57,0.0112723)(193.58,0.0112724)(193.59,0.0112725)(193.60,0.0112725)(193.61,0.0112726)(193.62,0.0112727)(193.63,0.0112728)(193.64,0.0112729)(193.65,0.0112730)(193.66,0.0112731)(193.67,0.0112732)(193.68,0.0112733)(193.69,0.0112734)(193.70,0.0112735)(193.71,0.0112736)(193.72,0.0112737)(193.73,0.0112738)(193.74,0.0112739)(193.75,0.0112740)(193.76,0.0112742)(193.77,0.0112743)(193.78,0.0112744)(193.79,0.0112745)(193.80,0.0112746)(193.81,0.0112747)(193.82,0.0112749)(193.83,0.0112750)(193.84,0.0112751)(193.85,0.0112752)(193.86,0.0112754)(193.87,0.0112755)(193.88,0.0112756)(193.89,0.0112757)(193.90,0.0112759)(193.91,0.0112760)(193.92,0.0112761)(193.93,0.0112763)(193.94,0.0112764)(193.95,0.0112766)(193.96,0.0112767)(193.97,0.0112769)(193.98,0.0112770)(193.99,0.0112772)(194.00,0.0112773)(194.01,0.0112775)(194.02,0.0112776)(194.03,0.0112778)(194.04,0.0112779)(194.05,0.0112781)(194.06,0.0112783)(194.07,0.0112784)(194.08,0.0112786)(194.09,0.0112787)(194.10,0.0112789)(194.11,0.0112791)(194.12,0.0112793)(194.13,0.0112794)(194.14,0.0112796)(194.15,0.0112798)(194.16,0.0112800)(194.17,0.0112802)(194.18,0.0112804)(194.19,0.0112805)(194.20,0.0112807)(194.21,0.0112809)(194.22,0.0112811)(194.23,0.0112813)(194.24,0.0112815)(194.25,0.0112817)(194.26,0.0112819)(194.27,0.0112821)(194.28,0.0112823)(194.29,0.0112825)(194.30,0.0112828)(194.31,0.0112830)(194.32,0.0112832)(194.33,0.0112834)(194.34,0.0112836)(194.35,0.0112839)(194.36,0.0112841)(194.37,0.0112843)(194.38,0.0112846)(194.39,0.0112848)(194.40,0.0112850)(194.41,0.0112853)(194.42,0.0112855)(194.43,0.0112858)(194.44,0.0112860)(194.45,0.0112863)(194.46,0.0112865)(194.47,0.0112868)(194.48,0.0112870)(194.49,0.0112873)(194.50,0.0112876)(194.51,0.0112878)(194.52,0.0112881)(194.53,0.0112884)(194.54,0.0112886)(194.55,0.0112889)(194.56,0.0112892)(194.57,0.0112895)(194.58,0.0112898)(194.59,0.0112901)(194.60,0.0112904)(194.61,0.0112907)(194.62,0.0112910)(194.63,0.0112913)(194.64,0.0112916)(194.65,0.0112919)(194.66,0.0112922)(194.67,0.0112925)(194.68,0.0112928)(194.69,0.0112932)(194.70,0.0112935)(194.71,0.0112938)(194.72,0.0112941)(194.73,0.0112945)(194.74,0.0112948)(194.75,0.0112952)(194.76,0.0112955)(194.77,0.0112959)(194.78,0.0112962)(194.79,0.0112966)(194.80,0.0112969)(194.81,0.0112973)(194.82,0.0112977)(194.83,0.0112981)(194.84,0.0112984)(194.85,0.0112988)(194.86,0.0112992)(194.87,0.0112996)(194.88,0.0113000)(194.89,0.0113004)(194.90,0.0113008)(194.91,0.0111828)(194.92,0.0111834)(194.93,0.0111840)(194.94,0.0111847)(194.95,0.0111853)(194.96,0.0111859)(194.97,0.0111866)(194.98,0.0111872)(194.99,0.0111878)(195.00,0.0111885)(195.01,0.0111891)(195.02,0.0111898)(195.03,0.0111905)(195.04,0.0111911)(195.05,0.0111918)(195.06,0.0111925)(195.07,0.0111931)(195.08,0.0111938)(195.09,0.0111945)(195.10,0.0111952)(195.11,0.0111959)(195.12,0.0111966)(195.13,0.0111973)(195.14,0.0111980)(195.15,0.0111987)(195.16,0.0111994)(195.17,0.0112001)(195.18,0.0112008)(195.19,0.0112015)(195.20,0.0112023)(195.21,0.0112030)(195.22,0.0112037)(195.23,0.0112045)(195.24,0.0112052)(195.25,0.0112060)(195.26,0.0112067)(195.27,0.0112075)(195.28,0.0112083)(195.29,0.0112090)(195.30,0.0112098)(195.31,0.0112106)(195.32,0.0112114)(195.33,0.0112121)(195.34,0.0112129)(195.35,0.0112137)(195.36,0.0112145)(195.37,0.0112153)(195.38,0.0112161)(195.39,0.0112170)(195.40,0.0112178)(195.41,0.0112186)(195.42,0.0112194)(195.43,0.0112203)(195.44,0.0112211)(195.45,0.0112220)(195.46,0.0112228)(195.47,0.0112237)(195.48,0.0112245)(195.49,0.0112254)(195.50,0.0112263)(195.51,0.0112271)(195.52,0.0112280)(195.53,0.0112289)(195.54,0.0112298)(195.55,0.0112307)(195.56,0.0112316)(195.57,0.0112325)(195.58,0.0112334)(195.59,0.0112344)(195.60,0.0112353)(195.61,0.0112362)(195.62,0.0112372)(195.63,0.0112381)(195.64,0.0112391)(195.65,0.0112400)(195.66,0.0112410)(195.67,0.0112420)(195.68,0.0112429)(195.69,0.0112439)(195.70,0.0112449)(195.71,0.0112459)(195.72,0.0112469)(195.73,0.0112479)(195.74,0.0112490)(195.75,0.0112500)(195.76,0.0112510)(195.77,0.0112521)(195.78,0.0112531)(195.79,0.0112542)(195.80,0.0112552)(195.81,0.0112563)(195.82,0.0112574)(195.83,0.0112584)(195.84,0.0112595)(195.85,0.0112606)(195.86,0.0112617)(195.87,0.0112628)(195.88,0.0112640)(195.89,0.0112651)(195.90,0.0112662)(195.91,0.0112674)(195.92,0.0112685)(195.93,0.0112697)(195.94,0.0112708)(195.95,0.0112720)(195.96,0.0112732)(195.97,0.0112744)(195.98,0.0112756)(195.99,0.0112768)(196.00,0.0112780)(196.01,0.0112792)(196.02,0.0112805)(196.03,0.0112817)(196.04,0.0112830)(196.05,0.0112842)(196.06,0.0112855)(196.07,0.0112868)(196.08,0.0112881)(196.09,0.0112894)(196.10,0.0112907)(196.11,0.0112920)(196.12,0.0112933)(196.13,0.0112946)(196.14,0.0112960)(196.15,0.0112973)(196.16,0.0112987)(196.17,0.0113001)(196.18,0.0113015)(196.19,0.0113029)(196.20,0.0113043)(196.21,0.0113057)(196.22,0.0113071)(196.23,0.0113086)(196.24,0.0113100)(196.25,0.0113115)(196.26,0.0113129)(196.27,0.0113144)(196.28,0.0113159)(196.29,0.0113174)(196.30,0.0113190)(196.31,0.0113205)(196.32,0.0113220)(196.33,0.0113236)(196.34,0.0113251)(196.35,0.0113267)(196.36,0.0113283)(196.37,0.0113299)(196.38,0.0113315)(196.39,0.0113332)(196.40,0.0113348)(196.41,0.0113365)(196.42,0.0113381)(196.43,0.0113398)(196.44,0.0113415)(196.45,0.0113432)(196.46,0.0113449)(196.47,0.0113467)(196.48,0.0113484)(196.49,0.0113502)(196.50,0.0113519)(196.51,0.0113537)(196.52,0.0113555)(196.53,0.0113574)(196.54,0.0113592)(196.55,0.0113611)(196.56,0.0113629)(196.57,0.0113648)(196.58,0.0113667)(196.59,0.0113686)(196.60,0.0113706)(196.61,0.0113725)(196.62,0.0113745)(196.63,0.0113765)(196.64,0.0113784)(196.65,0.0113805)(196.66,0.0113825)(196.67,0.0113845)(196.68,0.0113866)(196.69,0.0113887)(196.70,0.0113908)(196.71,0.0113929)(196.72,0.0113951)(196.73,0.0113972)(196.74,0.0113994)(196.75,0.0114016)(196.76,0.0114038)(196.77,0.0114060)(196.78,0.0114083)(196.79,0.0114106)(196.80,0.0114129)(196.81,0.0114152)(196.82,0.0114175)(196.83,0.0114199)(196.84,0.0114223)(196.85,0.0114247)(196.86,0.0114271)(196.87,0.0114295)(196.88,0.0114320)(196.89,0.0114345)(196.90,0.0114370)(196.91,0.0114396)(196.92,0.0114421)(196.93,0.0114447)(196.94,0.0114473)(196.95,0.0114500)(196.96,0.0114526)(196.97,0.0114553)(196.98,0.0114580)(196.99,0.0114607)(197.00,0.0114635)(197.01,0.0114663)(197.02,0.0114691)(197.03,0.0114720)(197.04,0.0114748)(197.05,0.0114777)(197.06,0.0114807)(197.07,0.0114836)(197.08,0.0114866)(197.09,0.0114896)(197.10,0.0114927)(197.11,0.0114957)(197.12,0.0114988)(197.13,0.0115020)(197.14,0.0115051)(197.15,0.0115083)(197.16,0.0115116)(197.17,0.0115148)(197.18,0.0115181)(197.19,0.0115215)(197.20,0.0115248)(197.21,0.0115283)(197.22,0.0115317)(197.23,0.0115352)(197.24,0.0115387)(197.25,0.0115422)(197.26,0.0115458)(197.27,0.0115494)(197.28,0.0115531)(197.29,0.0115568)(197.30,0.0115605)(197.31,0.0115643)(197.32,0.0115681)(197.33,0.0115720)(197.34,0.0115759)(197.35,0.0115798)(197.36,0.0115838)(197.37,0.0115879)(197.38,0.0115919)(197.39,0.0115961)(197.40,0.0116002)(197.41,0.0116044)(197.42,0.0116087)(197.43,0.0116130)(197.44,0.0116174)(197.45,0.0116218)(197.46,0.0116262)(197.47,0.0116308)(197.48,0.0116353)(197.49,0.0116399)(197.50,0.0116446)(197.51,0.0116493)(197.52,0.0116541)(197.53,0.0116589)(197.54,0.0116638)(197.55,0.0116688)(197.56,0.0116738)(197.57,0.0116788)(197.58,0.0116840)(197.59,0.0116892)(197.60,0.0116944)(197.61,0.0116997)(197.62,0.0117051)(197.63,0.0117106)(197.64,0.0117161)(197.65,0.0117216)(197.66,0.0117273)(197.67,0.0117330)(197.68,0.0117388)(197.69,0.0117447)(197.70,0.0117506)(197.71,0.0117566)(197.72,0.0117627)(197.73,0.0117689)(197.74,0.0117751)(197.75,0.0117815)(197.76,0.0117879)(197.77,0.0117944)(197.78,0.0118009)(197.79,0.0118076)(197.80,0.0118144)(197.81,0.0118212)(197.82,0.0118282)(197.83,0.0118352)(197.84,0.0118423)(197.85,0.0118495)(197.86,0.0118568)(197.87,0.0118643)(197.88,0.0118718)(197.89,0.0118794)(197.90,0.0118871)(197.91,0.0118950)(197.92,0.0119029)(197.93,0.0119110)(197.94,0.0119191)(197.95,0.0119274)(197.96,0.0119358)(197.97,0.0119444)(197.98,0.0119530)(197.99,0.0119618)(198.00,0.0119707)};
		\legend{$S_{Diff}$, $S_{LOO}$}
		\end{axis}
	\end{tikzpicture}
	\caption{Зависимость критериев кросс-валидации от параметра регуляризации $\mu$.}
	\label{tikz:CV}
\end{figure}

\begin{figure}[H]
	\centering
	\begin{tikzpicture}%[x=3cm,y=3cm]
		\begin{axis}[xlabel=$\mu$, ylabel=$S$, enlargelimits=false,
		%title={Модельные данные},
		scaled x ticks=false,
		xticklabel style={/pgf/number format/fixed, /pgf/number format/precision=3}]
		\addplot[blue, mark=none, very thick] coordinates {(190.00,0.0015864)(190.01,0.0015856)(190.02,0.0015848)(190.03,0.0015841)(190.04,0.0015833)(190.05,0.0015826)(190.06,0.0015818)(190.07,0.0015811)(190.08,0.0015803)(190.09,0.0015795)(190.10,0.0015787)(190.11,0.0015780)(190.12,0.0015772)(190.13,0.0015764)(190.14,0.0015757)(190.15,0.0015750)(190.16,0.0015742)(190.17,0.0015734)(190.18,0.0015727)(190.19,0.0015719)(190.20,0.0015711)(190.21,0.0015703)(190.22,0.0015695)(190.23,0.0015687)(190.24,0.0015680)(190.25,0.0015672)(190.26,0.0015664)(190.27,0.0015656)(190.28,0.0015648)(190.29,0.0015640)(190.30,0.0015632)(190.31,0.0015621)(190.32,0.0015613)(190.33,0.0015605)(190.34,0.0015598)(190.35,0.0015590)(190.36,0.0015582)(190.37,0.0015574)(190.38,0.0015566)(190.39,0.0015559)(190.40,0.0015551)(190.41,0.0015543)(190.42,0.0015534)(190.43,0.0015526)(190.44,0.0015518)(190.45,0.0015510)(190.46,0.0015503)(190.47,0.0015495)(190.48,0.0015487)(190.49,0.0015479)(190.50,0.0015470)(190.51,0.0015463)(190.52,0.0015455)(190.53,0.0015447)(190.54,0.0015439)(190.55,0.0015431)(190.56,0.0015423)(190.57,0.0015415)(190.58,0.0015408)(190.59,0.0015400)(190.60,0.0015390)(190.61,0.0015382)(190.62,0.0015375)(190.63,0.0015367)(190.64,0.0015360)(190.65,0.0015352)(190.66,0.0015344)(190.67,0.0015336)(190.68,0.0015329)(190.69,0.0015321)(190.70,0.0015310)(190.71,0.0015302)(190.72,0.0015294)(190.73,0.0015286)(190.74,0.0015279)(190.75,0.0015271)(190.76,0.0015263)(190.77,0.0015255)(190.78,0.0015247)(190.79,0.0015239)(190.80,0.0015231)(190.81,0.0015221)(190.82,0.0015213)(190.83,0.0015203)(190.84,0.0015196)(190.85,0.0015189)(190.86,0.0015182)(190.87,0.0015175)(190.88,0.0015166)(190.89,0.0015159)(190.90,0.0015152)(190.91,0.0015145)(190.92,0.0015138)(190.93,0.0015131)(190.94,0.0015124)(190.95,0.0015117)(190.96,0.0015109)(190.97,0.0015102)(190.98,0.0015095)(190.99,0.0015088)(191.00,0.0015081)(191.01,0.0015074)(191.02,0.0015067)(191.03,0.0015059)(191.04,0.0015052)(191.05,0.0015045)(191.06,0.0015039)(191.07,0.0015032)(191.08,0.0015022)(191.09,0.0015018)(191.10,0.0015011)(191.11,0.0015003)(191.12,0.0014996)(191.13,0.0014989)(191.14,0.0014981)(191.15,0.0014974)(191.16,0.0014966)(191.17,0.0014959)(191.18,0.0014952)(191.19,0.0014944)(191.20,0.0014937)(191.21,0.0014929)(191.22,0.0014922)(191.23,0.0014914)(191.24,0.0014907)(191.25,0.0014899)(191.26,0.0014892)(191.27,0.0014884)(191.28,0.0014879)(191.29,0.0014872)(191.30,0.0014864)(191.31,0.0014856)(191.32,0.0014849)(191.33,0.0014841)(191.34,0.0014833)(191.35,0.0014826)(191.36,0.0014818)(191.37,0.0014810)(191.38,0.0014802)(191.39,0.0014793)(191.40,0.0014785)(191.41,0.0014777)(191.42,0.0014770)(191.43,0.0014761)(191.44,0.0014753)(191.45,0.0014745)(191.46,0.0014734)(191.47,0.0014726)(191.48,0.0014719)(191.49,0.0014712)(191.50,0.0014704)(191.51,0.0014697)(191.52,0.0014690)(191.53,0.0014682)(191.54,0.0014675)(191.55,0.0014668)(191.56,0.0014661)(191.57,0.0014653)(191.58,0.0014646)(191.59,0.0014639)(191.60,0.0014632)(191.61,0.0014624)(191.62,0.0014618)(191.63,0.0014611)(191.64,0.0014603)(191.65,0.0014596)(191.66,0.0014588)(191.67,0.0014581)(191.68,0.0014574)(191.69,0.0014566)(191.70,0.0014559)(191.71,0.0014552)(191.72,0.0014544)(191.73,0.0014537)(191.74,0.0014529)(191.75,0.0014516)(191.76,0.0014509)(191.77,0.0014502)(191.78,0.0014495)(191.79,0.0014488)(191.80,0.0014481)(191.81,0.0014475)(191.82,0.0014468)(191.83,0.0014461)(191.84,0.0014454)(191.85,0.0014447)(191.86,0.0014440)(191.87,0.0014433)(191.88,0.0014424)(191.89,0.0014418)(191.90,0.0014409)(191.91,0.0014403)(191.92,0.0014397)(191.93,0.0014390)(191.94,0.0014384)(191.95,0.0014376)(191.96,0.0014371)(191.97,0.0014365)(191.98,0.0014363)(191.99,0.0014357)(192.00,0.0014351)(192.01,0.0014345)(192.02,0.0014340)(192.03,0.0014334)(192.04,0.0014328)(192.05,0.0014317)(192.06,0.0014311)(192.07,0.0014305)(192.08,0.0014299)(192.09,0.0014293)(192.10,0.0014287)(192.11,0.0014281)(192.12,0.0014275)(192.13,0.0014269)(192.14,0.0014263)(192.15,0.0014257)(192.16,0.0014251)(192.17,0.0014245)(192.18,0.0014239)(192.19,0.0014232)(192.20,0.0014226)(192.21,0.0014220)(192.22,0.0014213)(192.23,0.0014207)(192.24,0.0014201)(192.25,0.0014195)(192.26,0.0014188)(192.27,0.0014182)(192.28,0.0014176)(192.29,0.0014170)(192.30,0.0014164)(192.31,0.0014158)(192.32,0.0014152)(192.33,0.0014145)(192.34,0.0014139)(192.35,0.0014133)(192.36,0.0014122)(192.37,0.0014117)(192.38,0.0014111)(192.39,0.0014105)(192.40,0.0014099)(192.41,0.0014093)(192.42,0.0014086)(192.43,0.0014080)(192.44,0.0014075)(192.45,0.0014070)(192.46,0.0014065)(192.47,0.0014062)(192.48,0.0014055)(192.49,0.0014049)(192.50,0.0014043)(192.51,0.0014037)(192.52,0.0014031)(192.53,0.0014024)(192.54,0.0014016)(192.55,0.0014010)(192.56,0.0014004)(192.57,0.0013997)(192.58,0.0013991)(192.59,0.0013983)(192.60,0.0013977)(192.61,0.0013971)(192.62,0.0013965)(192.63,0.0013959)(192.64,0.0013952)(192.65,0.0013946)(192.66,0.0013940)(192.67,0.0013934)(192.68,0.0013928)(192.69,0.0013921)(192.70,0.0013915)(192.71,0.0013909)(192.72,0.0013903)(192.73,0.0013898)(192.74,0.0013892)(192.75,0.0013887)(192.76,0.0013881)(192.77,0.0013874)(192.78,0.0013869)(192.79,0.0013862)(192.80,0.0013854)(192.81,0.0013848)(192.82,0.0013841)(192.83,0.0013835)(192.84,0.0013829)(192.85,0.0013823)(192.86,0.0013816)(192.87,0.0013810)(192.88,0.0013803)(192.89,0.0013796)(192.90,0.0013790)(192.91,0.0013784)(192.92,0.0013776)(192.93,0.0013768)(192.94,0.0013761)(192.95,0.0013754)(192.96,0.0013747)(192.97,0.0013739)(192.98,0.0013732)(192.99,0.0013729)(193.00,0.0013720)(193.01,0.0013712)(193.02,0.0013704)(193.03,0.0013696)(193.04,0.0013689)(193.05,0.0013681)(193.06,0.0013674)(193.07,0.0013666)(193.08,0.0013659)(193.09,0.0013649)(193.10,0.0013637)(193.11,0.0013630)(193.12,0.0013622)(193.13,0.0013614)(193.14,0.0013608)(193.15,0.0013601)(193.16,0.0013594)(193.17,0.0013587)(193.18,0.0013581)(193.19,0.0013574)(193.20,0.0013567)(193.21,0.0013561)(193.22,0.0013553)(193.23,0.0013546)(193.24,0.0013539)(193.25,0.0013532)(193.26,0.0013525)(193.27,0.0013518)(193.28,0.0013511)(193.29,0.0013504)(193.30,0.0013497)(193.31,0.0013488)(193.32,0.0013480)(193.33,0.0013473)(193.34,0.0013466)(193.35,0.0013457)(193.36,0.0013449)(193.37,0.0013442)(193.38,0.0013436)(193.39,0.0013429)(193.40,0.0013421)(193.41,0.0013414)(193.42,0.0013407)(193.43,0.0013400)(193.44,0.0013391)(193.45,0.0013384)(193.46,0.0013378)(193.47,0.0013367)(193.48,0.0013361)(193.49,0.0013355)(193.50,0.0013348)(193.51,0.0013342)(193.52,0.0013335)(193.53,0.0013329)(193.54,0.0013322)(193.55,0.0013316)(193.56,0.0013309)(193.57,0.0013302)(193.58,0.0013295)(193.59,0.0013289)(193.60,0.0013283)(193.61,0.0013277)(193.62,0.0013271)(193.63,0.0013265)(193.64,0.0013259)(193.65,0.0013253)(193.66,0.0013247)(193.67,0.0013241)(193.68,0.0013235)(193.69,0.0013229)(193.70,0.0013222)(193.71,0.0013215)(193.72,0.0013209)(193.73,0.0013202)(193.74,0.0013195)(193.75,0.0013188)(193.76,0.0013181)(193.77,0.0013173)(193.78,0.0013166)(193.79,0.0013159)(193.80,0.0013151)(193.81,0.0013145)(193.82,0.0013138)(193.83,0.0013131)(193.84,0.0013123)(193.85,0.0013116)(193.86,0.0013108)(193.87,0.0013101)(193.88,0.0013095)(193.89,0.0013087)(193.90,0.0013079)(193.91,0.0013072)(193.92,0.0013064)(193.93,0.0013056)(193.94,0.0013048)(193.95,0.0013040)(193.96,0.0013032)(193.97,0.0013024)(193.98,0.0013017)(193.99,0.0013009)(194.00,0.0013001)(194.01,0.0012994)(194.02,0.0012986)(194.03,0.0012978)(194.04,0.0012970)(194.05,0.0012963)(194.06,0.0012951)(194.07,0.0012944)(194.08,0.0012936)(194.09,0.0012924)(194.10,0.0012917)(194.11,0.0012909)(194.12,0.0012893)(194.13,0.0012887)(194.14,0.0012882)(194.15,0.0012876)(194.16,0.0012870)(194.17,0.0012864)(194.18,0.0012858)(194.19,0.0012854)(194.20,0.0012848)(194.21,0.0012838)(194.22,0.0012833)(194.23,0.0012827)(194.24,0.0012820)(194.25,0.0012814)(194.26,0.0012811)(194.27,0.0012804)(194.28,0.0012799)(194.29,0.0012792)(194.30,0.0012785)(194.31,0.0012779)(194.32,0.0012772)(194.33,0.0012765)(194.34,0.0012758)(194.35,0.0012751)(194.36,0.0012745)(194.37,0.0012738)(194.38,0.0012732)(194.39,0.0012725)(194.40,0.0012719)(194.41,0.0012712)(194.42,0.0012706)(194.43,0.0012700)(194.44,0.0012694)(194.45,0.0012688)(194.46,0.0012682)(194.47,0.0012676)(194.48,0.0012663)(194.49,0.0012658)(194.50,0.0012654)(194.51,0.0012649)(194.52,0.0012644)(194.53,0.0012639)(194.54,0.0012633)(194.55,0.0012628)(194.56,0.0012623)(194.57,0.0012618)(194.58,0.0012614)(194.59,0.0012609)(194.60,0.0012603)(194.61,0.0012587)(194.62,0.0012583)(194.63,0.0012578)(194.64,0.0012574)(194.65,0.0012563)(194.66,0.0012559)(194.67,0.0012556)(194.68,0.0012551)(194.69,0.0012547)(194.70,0.0012543)(194.71,0.0012540)(194.72,0.0012536)(194.73,0.0012532)(194.74,0.0012522)(194.75,0.0012519)(194.76,0.0012515)(194.77,0.0012512)(194.78,0.0012508)(194.79,0.0012504)(194.80,0.0012500)(194.81,0.0012497)(194.82,0.0012493)(194.83,0.0012489)(194.84,0.0012485)(194.85,0.0012481)(194.86,0.0012477)(194.87,0.0012473)(194.88,0.0012469)(194.89,0.0012464)(194.90,0.0012461)(194.91,0.0012457)(194.92,0.0012454)(194.93,0.0012451)(194.94,0.0012448)(194.95,0.0012445)(194.96,0.0012443)(194.97,0.0012441)(194.98,0.0012439)(194.99,0.0012437)(195.00,0.0012435)(195.01,0.0012435)(195.02,0.0012432)(195.03,0.0012429)(195.04,0.0012426)(195.05,0.0012422)(195.06,0.0012420)(195.07,0.0012417)(195.08,0.0012414)(195.09,0.0012411)(195.10,0.0012409)(195.11,0.0012408)(195.12,0.0012407)(195.13,0.0012407)(195.14,0.0012407)(195.15,0.0012407)(195.16,0.0012406)(195.17,0.0012406)(195.18,0.0012406)(195.19,0.0012405)(195.20,0.0012405)(195.21,0.0012404)(195.22,0.0012404)(195.23,0.0012397)(195.24,0.0012398)(195.25,0.0012400)(195.26,0.0012402)(195.27,0.0012403)(195.28,0.0012404)(195.29,0.0012405)(195.30,0.0012407)(195.31,0.0012410)(195.32,0.0012412)(195.33,0.0012414)(195.34,0.0012417)(195.35,0.0012420)(195.36,0.0012423)(195.37,0.0012426)(195.38,0.0012429)(195.39,0.0012432)(195.40,0.0012435)(195.41,0.0012439)(195.42,0.0012443)(195.43,0.0012447)(195.44,0.0012451)(195.45,0.0012455)(195.46,0.0012459)(195.47,0.0012463)(195.48,0.0012468)(195.49,0.0012472)(195.50,0.0012476)(195.51,0.0012481)(195.52,0.0012486)(195.53,0.0012490)(195.54,0.0012494)(195.55,0.0012499)(195.56,0.0012503)(195.57,0.0012507)(195.58,0.0012512)(195.59,0.0012516)(195.60,0.0012521)(195.61,0.0012525)(195.62,0.0012530)(195.63,0.0012531)(195.64,0.0012537)(195.65,0.0012543)(195.66,0.0012550)(195.67,0.0012558)(195.68,0.0012565)(195.69,0.0012573)(195.70,0.0012580)(195.71,0.0012589)(195.72,0.0012596)(195.73,0.0012603)(195.74,0.0012611)(195.75,0.0012619)(195.76,0.0012626)(195.77,0.0012634)(195.78,0.0012642)(195.79,0.0012649)(195.80,0.0012657)(195.81,0.0012665)(195.82,0.0012673)(195.83,0.0012681)(195.84,0.0012689)(195.85,0.0012697)(195.86,0.0012705)(195.87,0.0012714)(195.88,0.0012722)(195.89,0.0012730)(195.90,0.0012739)(195.91,0.0012747)(195.92,0.0012756)(195.93,0.0012764)(195.94,0.0012773)(195.95,0.0012781)(195.96,0.0012790)(195.97,0.0012799)(195.98,0.0012808)(195.99,0.0012817)(196.00,0.0012826)(196.01,0.0012835)(196.02,0.0012845)(196.03,0.0012854)(196.04,0.0012863)(196.05,0.0012873)(196.06,0.0012882)(196.07,0.0012892)(196.08,0.0012901)(196.09,0.0012911)(196.10,0.0012921)(196.11,0.0012931)(196.12,0.0012941)(196.13,0.0012951)(196.14,0.0012961)(196.15,0.0012971)(196.16,0.0012982)(196.17,0.0012992)(196.18,0.0013002)(196.19,0.0013013)(196.20,0.0013024)(196.21,0.0013034)(196.22,0.0013045)(196.23,0.0013056)(196.24,0.0013067)(196.25,0.0013080)(196.26,0.0013094)(196.27,0.0013108)(196.28,0.0013122)(196.29,0.0013137)(196.30,0.0013151)(196.31,0.0013166)(196.32,0.0013181)(196.33,0.0013196)(196.34,0.0013211)(196.35,0.0013226)(196.36,0.0013241)(196.37,0.0013256)(196.38,0.0013272)(196.39,0.0013287)(196.40,0.0013303)(196.41,0.0013319)(196.42,0.0013335)(196.43,0.0013351)(196.44,0.0013367)(196.45,0.0013384)(196.46,0.0013400)(196.47,0.0013417)(196.48,0.0013433)(196.49,0.0013450)(196.50,0.0013467)(196.51,0.0013485)(196.52,0.0013502)(196.53,0.0013519)(196.54,0.0013537)(196.55,0.0013555)(196.56,0.0013573)(196.57,0.0013591)(196.58,0.0013609)(196.59,0.0013627)(196.60,0.0013646)(196.61,0.0013664)(196.62,0.0013683)(196.63,0.0013702)(196.64,0.0013721)(196.65,0.0013741)(196.66,0.0013760)(196.67,0.0013780)(196.68,0.0013799)(196.69,0.0013819)(196.70,0.0013840)(196.71,0.0013860)(196.72,0.0013880)(196.73,0.0013901)(196.74,0.0013922)(196.75,0.0013943)(196.76,0.0013964)(196.77,0.0013986)(196.78,0.0014007)(196.79,0.0014029)(196.80,0.0014051)(196.81,0.0014073)(196.82,0.0014096)(196.83,0.0014118)(196.84,0.0014141)(196.85,0.0014164)(196.86,0.0014187)(196.87,0.0014211)(196.88,0.0014234)(196.89,0.0014258)(196.90,0.0014282)(196.91,0.0014307)(196.92,0.0014331)(196.93,0.0014356)(196.94,0.0014381)(196.95,0.0014406)(196.96,0.0014432)(196.97,0.0014457)(196.98,0.0014483)(196.99,0.0014510)(197.00,0.0014536)(197.01,0.0014563)(197.02,0.0014590)(197.03,0.0014617)(197.04,0.0014645)(197.05,0.0014672)(197.06,0.0014700)(197.07,0.0014729)(197.08,0.0014757)(197.09,0.0014786)(197.10,0.0014816)(197.11,0.0014845)(197.12,0.0014875)(197.13,0.0014905)(197.14,0.0014935)(197.15,0.0014966)(197.16,0.0014997)(197.17,0.0015028)(197.18,0.0015060)(197.19,0.0015092)(197.20,0.0015124)(197.21,0.0015157)(197.22,0.0015190)(197.23,0.0015223)(197.24,0.0015257)(197.25,0.0015291)(197.26,0.0015325)(197.27,0.0015360)(197.28,0.0015395)(197.29,0.0015430)(197.30,0.0015466)(197.31,0.0015502)(197.32,0.0015539)(197.33,0.0015576)(197.34,0.0015613)(197.35,0.0015651)(197.36,0.0015689)(197.37,0.0015728)(197.38,0.0015767)(197.39,0.0015807)(197.40,0.0015847)(197.41,0.0015887)(197.42,0.0015928)(197.43,0.0015969)(197.44,0.0016011)(197.45,0.0016053)(197.46,0.0016096)(197.47,0.0016139)(197.48,0.0016183)(197.49,0.0016227)(197.50,0.0016272)(197.51,0.0016318)(197.52,0.0016363)(197.53,0.0016410)(197.54,0.0016457)(197.55,0.0016504)(197.56,0.0016552)(197.57,0.0016601)(197.58,0.0016650)(197.59,0.0016700)(197.60,0.0016750)(197.61,0.0016801)(197.62,0.0016853)(197.63,0.0016905)(197.64,0.0016958)(197.65,0.0017011)(197.66,0.0017065)(197.67,0.0017120)(197.68,0.0017176)(197.69,0.0017232)(197.70,0.0017289)(197.71,0.0017347)(197.72,0.0017405)(197.73,0.0017464)(197.74,0.0017524)(197.75,0.0017585)(197.76,0.0017647)(197.77,0.0017709)(197.78,0.0017772)(197.79,0.0017836)(197.80,0.0017901)(197.81,0.0017966)(197.82,0.0018033)(197.83,0.0018100)(197.84,0.0018169)(197.85,0.0018238)(197.86,0.0018308)(197.87,0.0018379)(197.88,0.0018452)(197.89,0.0018525)(197.90,0.0018599)(197.91,0.0018674)(197.92,0.0018750)(197.93,0.0018828)(197.94,0.0018906)(197.95,0.0018986)(197.96,0.0019066)(197.97,0.0019148)(197.98,0.0019231)(197.99,0.0019315)(198.00,0.0019401)};
		\addplot[red,  mark=none, very thick] coordinates {(190.00,0.0019632)(190.01,0.0019632)(190.02,0.0019633)(190.03,0.0019633)(190.04,0.0019633)(190.05,0.0019633)(190.06,0.0019633)(190.07,0.0019634)(190.08,0.0019634)(190.09,0.0019634)(190.10,0.0019634)(190.11,0.0019635)(190.12,0.0019635)(190.13,0.0019635)(190.14,0.0019635)(190.15,0.0019635)(190.16,0.0019636)(190.17,0.0019636)(190.18,0.0019636)(190.19,0.0019636)(190.20,0.0019637)(190.21,0.0019637)(190.22,0.0019637)(190.23,0.0019638)(190.24,0.0019638)(190.25,0.0019638)(190.26,0.0019638)(190.27,0.0019639)(190.28,0.0019639)(190.29,0.0019639)(190.30,0.0019639)(190.31,0.0019640)(190.32,0.0019640)(190.33,0.0019640)(190.34,0.0019641)(190.35,0.0019641)(190.36,0.0015045)(190.37,0.0015045)(190.38,0.0015045)(190.39,0.0015045)(190.40,0.0015045)(190.41,0.0015045)(190.42,0.0015046)(190.43,0.0015046)(190.44,0.0015046)(190.45,0.0015046)(190.46,0.0015046)(190.47,0.0015046)(190.48,0.0015047)(190.49,0.0015047)(190.50,0.0015047)(190.51,0.0015047)(190.52,0.0015047)(190.53,0.0015047)(190.54,0.0015048)(190.55,0.0015048)(190.56,0.0015048)(190.57,0.0015048)(190.58,0.0015048)(190.59,0.0015049)(190.60,0.0015049)(190.61,0.0015049)(190.62,0.0015049)(190.63,0.0015049)(190.64,0.0015050)(190.65,0.0015050)(190.66,0.0015050)(190.67,0.0015050)(190.68,0.0015050)(190.69,0.0015050)(190.70,0.0015051)(190.71,0.0015051)(190.72,0.0015051)(190.73,0.0015051)(190.74,0.0015051)(190.75,0.0015052)(190.76,0.0015052)(190.77,0.0015052)(190.78,0.0015052)(190.79,0.0015053)(190.80,0.0015053)(190.81,0.0015053)(190.82,0.0015053)(190.83,0.0015053)(190.84,0.0015054)(190.85,0.0015054)(190.86,0.0015054)(190.87,0.0015054)(190.88,0.0015055)(190.89,0.0015055)(190.90,0.0015055)(190.91,0.0015055)(190.92,0.0015055)(190.93,0.0015056)(190.94,0.0015056)(190.95,0.0015056)(190.96,0.0015056)(190.97,0.0015057)(190.98,0.0015057)(190.99,0.0015057)(191.00,0.0015057)(191.01,0.0015058)(191.02,0.0015058)(191.03,0.0015058)(191.04,0.0015059)(191.05,0.0015059)(191.06,0.0015059)(191.07,0.0015059)(191.08,0.0015060)(191.09,0.0015060)(191.10,0.0015060)(191.11,0.0015060)(191.12,0.0015061)(191.13,0.0015061)(191.14,0.0015061)(191.15,0.0015062)(191.16,0.0015062)(191.17,0.0015062)(191.18,0.0015063)(191.19,0.0015063)(191.20,0.0015063)(191.21,0.0015063)(191.22,0.0015064)(191.23,0.0015064)(191.24,0.0015064)(191.25,0.0015065)(191.26,0.0015065)(191.27,0.0015065)(191.28,0.0015066)(191.29,0.0015066)(191.30,0.0015066)(191.31,0.0015067)(191.32,0.0015067)(191.33,0.0015067)(191.34,0.0015068)(191.35,0.0015068)(191.36,0.0015068)(191.37,0.0015069)(191.38,0.0015069)(191.39,0.0015070)(191.40,0.0015070)(191.41,0.0015070)(191.42,0.0015071)(191.43,0.0015071)(191.44,0.0015071)(191.45,0.0015072)(191.46,0.0015072)(191.47,0.0015072)(191.48,0.0015073)(191.49,0.0015073)(191.50,0.0015074)(191.51,0.0015074)(191.52,0.0015074)(191.53,0.0015075)(191.54,0.0015075)(191.55,0.0015076)(191.56,0.0015076)(191.57,0.0015077)(191.58,0.0015077)(191.59,0.0015077)(191.60,0.0015078)(191.61,0.0015078)(191.62,0.0015079)(191.63,0.0015079)(191.64,0.0015079)(191.65,0.0015080)(191.66,0.0015080)(191.67,0.0015081)(191.68,0.0015081)(191.69,0.0015082)(191.70,0.0015082)(191.71,0.0015083)(191.72,0.0015083)(191.73,0.0015084)(191.74,0.0015084)(191.75,0.0015085)(191.76,0.0015085)(191.77,0.0015086)(191.78,0.0015086)(191.79,0.0015086)(191.80,0.0015087)(191.81,0.0015087)(191.82,0.0015088)(191.83,0.0015088)(191.84,0.0015089)(191.85,0.0015090)(191.86,0.0015090)(191.87,0.0015091)(191.88,0.0015091)(191.89,0.0015092)(191.90,0.0015092)(191.91,0.0015093)(191.92,0.0015093)(191.93,0.0015094)(191.94,0.0015094)(191.95,0.0015095)(191.96,0.0015095)(191.97,0.0015096)(191.98,0.0015097)(191.99,0.0015097)(192.00,0.0015098)(192.01,0.0015098)(192.02,0.0015099)(192.03,0.0015099)(192.04,0.0015100)(192.05,0.0015101)(192.06,0.0015101)(192.07,0.0015102)(192.08,0.0015102)(192.09,0.0015103)(192.10,0.0015104)(192.11,0.0015104)(192.12,0.0015105)(192.13,0.0015106)(192.14,0.0015106)(192.15,0.0015107)(192.16,0.0015107)(192.17,0.0015108)(192.18,0.0015109)(192.19,0.0015109)(192.20,0.0015110)(192.21,0.0015111)(192.22,0.0015111)(192.23,0.0015112)(192.24,0.0015113)(192.25,0.0015113)(192.26,0.0015114)(192.27,0.0015115)(192.28,0.0015116)(192.29,0.0015116)(192.30,0.0015117)(192.31,0.0015118)(192.32,0.0015118)(192.33,0.0015119)(192.34,0.0015120)(192.35,0.0015121)(192.36,0.0015121)(192.37,0.0015122)(192.38,0.0015123)(192.39,0.0015124)(192.40,0.0015124)(192.41,0.0015125)(192.42,0.0015126)(192.43,0.0015127)(192.44,0.0015127)(192.45,0.0015128)(192.46,0.0015129)(192.47,0.0015130)(192.48,0.0015131)(192.49,0.0015131)(192.50,0.0015132)(192.51,0.0015133)(192.52,0.0015134)(192.53,0.0015135)(192.54,0.0015136)(192.55,0.0015137)(192.56,0.0015137)(192.57,0.0015138)(192.58,0.0015139)(192.59,0.0015140)(192.60,0.0015141)(192.61,0.0015142)(192.62,0.0015143)(192.63,0.0015144)(192.64,0.0015144)(192.65,0.0015145)(192.66,0.0015146)(192.67,0.0015147)(192.68,0.0015148)(192.69,0.0015149)(192.70,0.0015150)(192.71,0.0015151)(192.72,0.0015152)(192.73,0.0015153)(192.74,0.0015154)(192.75,0.0015155)(192.76,0.0015156)(192.77,0.0015157)(192.78,0.0015158)(192.79,0.0015159)(192.80,0.0015160)(192.81,0.0015161)(192.82,0.0015162)(192.83,0.0015163)(192.84,0.0015164)(192.85,0.0015165)(192.86,0.0015166)(192.87,0.0015167)(192.88,0.0015168)(192.89,0.0015169)(192.90,0.0015170)(192.91,0.0015171)(192.92,0.0015172)(192.93,0.0015174)(192.94,0.0015175)(192.95,0.0015176)(192.96,0.0015177)(192.97,0.0015178)(192.98,0.0015179)(192.99,0.0010948)(193.00,0.0010948)(193.01,0.0010948)(193.02,0.0010948)(193.03,0.0010948)(193.04,0.0010948)(193.05,0.0010948)(193.06,0.0010948)(193.07,0.0010948)(193.08,0.0010948)(193.09,0.0010948)(193.10,0.0010948)(193.11,0.0010948)(193.12,0.0010948)(193.13,0.0010948)(193.14,0.0010948)(193.15,0.0010948)(193.16,0.0010948)(193.17,0.0010948)(193.18,0.0010948)(193.19,0.0010948)(193.20,0.0010948)(193.21,0.0010948)(193.22,0.0010948)(193.23,0.0010948)(193.24,0.0010949)(193.25,0.0010949)(193.26,0.0010949)(193.27,0.0010949)(193.28,0.0010949)(193.29,0.0010949)(193.30,0.0010949)(193.31,0.0010949)(193.32,0.0010949)(193.33,0.0010949)(193.34,0.0010949)(193.35,0.0010949)(193.36,0.0010950)(193.37,0.0010950)(193.38,0.0010950)(193.39,0.0010950)(193.40,0.0010950)(193.41,0.0010950)(193.42,0.0010950)(193.43,0.0010950)(193.44,0.0010950)(193.45,0.0010951)(193.46,0.0010951)(193.47,0.0010951)(193.48,0.0010951)(193.49,0.0010951)(193.50,0.0010951)(193.51,0.0010951)(193.52,0.0010952)(193.53,0.0010952)(193.54,0.0010952)(193.55,0.0010952)(193.56,0.0010952)(193.57,0.0010952)(193.58,0.0010952)(193.59,0.0010953)(193.60,0.0010953)(193.61,0.0010953)(193.62,0.0010953)(193.63,0.0010953)(193.64,0.0010954)(193.65,0.0010954)(193.66,0.0010954)(193.67,0.0010954)(193.68,0.0010954)(193.69,0.0010954)(193.70,0.0010955)(193.71,0.0010955)(193.72,0.0010955)(193.73,0.0010955)(193.74,0.0010956)(193.75,0.0010956)(193.76,0.0010956)(193.77,0.0010956)(193.78,0.0010956)(193.79,0.0010957)(193.80,0.0010957)(193.81,0.0010957)(193.82,0.0010957)(193.83,0.0010958)(193.84,0.0010958)(193.85,0.0010958)(193.86,0.0010958)(193.87,0.0010959)(193.88,0.0010959)(193.89,0.0010959)(193.90,0.0010960)(193.91,0.0010960)(193.92,0.0010960)(193.93,0.0010960)(193.94,0.0010961)(193.95,0.0010961)(193.96,0.0010961)(193.97,0.0010962)(193.98,0.0010962)(193.99,0.0010962)(194.00,0.0010963)(194.01,0.0010963)(194.02,0.0010963)(194.03,0.0010964)(194.04,0.0010964)(194.05,0.0010964)(194.06,0.0010965)(194.07,0.0010965)(194.08,0.0010965)(194.09,0.0010966)(194.10,0.0010966)(194.11,0.0010967)(194.12,0.0010967)(194.13,0.0010967)(194.14,0.0010968)(194.15,0.0010968)(194.16,0.0010969)(194.17,0.0010969)(194.18,0.0010969)(194.19,0.0010970)(194.20,0.0010970)(194.21,0.0010971)(194.22,0.0010971)(194.23,0.0010972)(194.24,0.0010972)(194.25,0.0010973)(194.26,0.0010973)(194.27,0.0010973)(194.28,0.0010974)(194.29,0.0010974)(194.30,0.0010975)(194.31,0.0010975)(194.32,0.0010976)(194.33,0.0010976)(194.34,0.0010977)(194.35,0.0010977)(194.36,0.0010978)(194.37,0.0010978)(194.38,0.0010979)(194.39,0.0010980)(194.40,0.0010980)(194.41,0.0010981)(194.42,0.0010981)(194.43,0.0010982)(194.44,0.0010982)(194.45,0.0010983)(194.46,0.0010984)(194.47,0.0010984)(194.48,0.0010985)(194.49,0.0010985)(194.50,0.0010986)(194.51,0.0010987)(194.52,0.0010987)(194.53,0.0010988)(194.54,0.0010989)(194.55,0.0010989)(194.56,0.0010990)(194.57,0.0010991)(194.58,0.0010991)(194.59,0.0010992)(194.60,0.0010993)(194.61,0.0010993)(194.62,0.0010994)(194.63,0.0010995)(194.64,0.0010996)(194.65,0.0010996)(194.66,0.0010997)(194.67,0.0010998)(194.68,0.0010999)(194.69,0.0010999)(194.70,0.0011000)(194.71,0.0011001)(194.72,0.0011002)(194.73,0.0011003)(194.74,0.0011003)(194.75,0.0011004)(194.76,0.0011005)(194.77,0.0011006)(194.78,0.0011007)(194.79,0.0011008)(194.80,0.0011008)(194.81,0.0011009)(194.82,0.0011010)(194.83,0.0011011)(194.84,0.0011012)(194.85,0.0011013)(194.86,0.0011014)(194.87,0.0011015)(194.88,0.0011016)(194.89,0.0011017)(194.90,0.0011018)(194.91,0.0008651)(194.92,0.0008652)(194.93,0.0008653)(194.94,0.0008654)(194.95,0.0008656)(194.96,0.0008657)(194.97,0.0008658)(194.98,0.0008659)(194.99,0.0008661)(195.00,0.0008662)(195.01,0.0008663)(195.02,0.0008665)(195.03,0.0008666)(195.04,0.0008667)(195.05,0.0008669)(195.06,0.0008670)(195.07,0.0008671)(195.08,0.0008673)(195.09,0.0008674)(195.10,0.0008675)(195.11,0.0008677)(195.12,0.0008678)(195.13,0.0008679)(195.14,0.0008681)(195.15,0.0008682)(195.16,0.0008684)(195.17,0.0008685)(195.18,0.0008687)(195.19,0.0008688)(195.20,0.0008689)(195.21,0.0008691)(195.22,0.0008692)(195.23,0.0008694)(195.24,0.0008695)(195.25,0.0008697)(195.26,0.0008698)(195.27,0.0008700)(195.28,0.0008701)(195.29,0.0008703)(195.30,0.0008705)(195.31,0.0008706)(195.32,0.0008708)(195.33,0.0008709)(195.34,0.0008711)(195.35,0.0008712)(195.36,0.0008714)(195.37,0.0008716)(195.38,0.0008717)(195.39,0.0008719)(195.40,0.0008721)(195.41,0.0008722)(195.42,0.0008724)(195.43,0.0008726)(195.44,0.0008727)(195.45,0.0008729)(195.46,0.0008731)(195.47,0.0008733)(195.48,0.0008734)(195.49,0.0008736)(195.50,0.0008738)(195.51,0.0008740)(195.52,0.0008741)(195.53,0.0008743)(195.54,0.0008745)(195.55,0.0008747)(195.56,0.0008749)(195.57,0.0008751)(195.58,0.0008752)(195.59,0.0008754)(195.60,0.0008756)(195.61,0.0008758)(195.62,0.0008760)(195.63,0.0008762)(195.64,0.0008764)(195.65,0.0008766)(195.66,0.0008768)(195.67,0.0008770)(195.68,0.0008772)(195.69,0.0008774)(195.70,0.0008776)(195.71,0.0008778)(195.72,0.0008780)(195.73,0.0008782)(195.74,0.0008784)(195.75,0.0008786)(195.76,0.0008788)(195.77,0.0008790)(195.78,0.0008792)(195.79,0.0008795)(195.80,0.0008797)(195.81,0.0008799)(195.82,0.0008801)(195.83,0.0008803)(195.84,0.0008806)(195.85,0.0008808)(195.86,0.0008810)(195.87,0.0008812)(195.88,0.0008815)(195.89,0.0008817)(195.90,0.0008819)(195.91,0.0008822)(195.92,0.0008824)(195.93,0.0008826)(195.94,0.0008829)(195.95,0.0008831)(195.96,0.0008834)(195.97,0.0008836)(195.98,0.0008839)(195.99,0.0008841)(196.00,0.0008844)(196.01,0.0008846)(196.02,0.0008849)(196.03,0.0008851)(196.04,0.0008854)(196.05,0.0008856)(196.06,0.0008859)(196.07,0.0008862)(196.08,0.0008864)(196.09,0.0008867)(196.10,0.0008870)(196.11,0.0008872)(196.12,0.0008875)(196.13,0.0008878)(196.14,0.0008881)(196.15,0.0008884)(196.16,0.0008886)(196.17,0.0008889)(196.18,0.0008892)(196.19,0.0008895)(196.20,0.0008898)(196.21,0.0008901)(196.22,0.0008904)(196.23,0.0008907)(196.24,0.0008910)(196.25,0.0008913)(196.26,0.0008916)(196.27,0.0008919)(196.28,0.0008922)(196.29,0.0008925)(196.30,0.0008928)(196.31,0.0008932)(196.32,0.0008935)(196.33,0.0008938)(196.34,0.0008941)(196.35,0.0008945)(196.36,0.0008948)(196.37,0.0008951)(196.38,0.0008955)(196.39,0.0008958)(196.40,0.0008962)(196.41,0.0008965)(196.42,0.0008969)(196.43,0.0008972)(196.44,0.0008976)(196.45,0.0008979)(196.46,0.0008983)(196.47,0.0008986)(196.48,0.0008990)(196.49,0.0008994)(196.50,0.0008998)(196.51,0.0009001)(196.52,0.0009005)(196.53,0.0009009)(196.54,0.0009013)(196.55,0.0009017)(196.56,0.0009021)(196.57,0.0009025)(196.58,0.0009029)(196.59,0.0009033)(196.60,0.0009037)(196.61,0.0009041)(196.62,0.0009045)(196.63,0.0009049)(196.64,0.0009053)(196.65,0.0009058)(196.66,0.0009062)(196.67,0.0009066)(196.68,0.0009071)(196.69,0.0009075)(196.70,0.0009079)(196.71,0.0009084)(196.72,0.0009088)(196.73,0.0009093)(196.74,0.0009098)(196.75,0.0009102)(196.76,0.0009107)(196.77,0.0009112)(196.78,0.0009116)(196.79,0.0009121)(196.80,0.0009126)(196.81,0.0009131)(196.82,0.0009136)(196.83,0.0009141)(196.84,0.0009146)(196.85,0.0009151)(196.86,0.0009156)(196.87,0.0009162)(196.88,0.0009167)(196.89,0.0009172)(196.90,0.0009177)(196.91,0.0009183)(196.92,0.0009188)(196.93,0.0009194)(196.94,0.0009199)(196.95,0.0009205)(196.96,0.0009211)(196.97,0.0009216)(196.98,0.0009222)(196.99,0.0009228)(197.00,0.0009234)(197.01,0.0009240)(197.02,0.0009246)(197.03,0.0009252)(197.04,0.0009258)(197.05,0.0009264)(197.06,0.0009271)(197.07,0.0009277)(197.08,0.0009283)(197.09,0.0009290)(197.10,0.0009296)(197.11,0.0009303)(197.12,0.0009310)(197.13,0.0009316)(197.14,0.0009323)(197.15,0.0009330)(197.16,0.0009337)(197.17,0.0009344)(197.18,0.0009351)(197.19,0.0009358)(197.20,0.0009366)(197.21,0.0009373)(197.22,0.0009380)(197.23,0.0009388)(197.24,0.0009396)(197.25,0.0009403)(197.26,0.0009411)(197.27,0.0009419)(197.28,0.0009427)(197.29,0.0009435)(197.30,0.0009443)(197.31,0.0009451)(197.32,0.0009459)(197.33,0.0009468)(197.34,0.0009476)(197.35,0.0009485)(197.36,0.0009493)(197.37,0.0009502)(197.38,0.0009511)(197.39,0.0009520)(197.40,0.0009529)(197.41,0.0009538)(197.42,0.0009547)(197.43,0.0009557)(197.44,0.0009566)(197.45,0.0009576)(197.46,0.0009585)(197.47,0.0009595)(197.48,0.0009605)(197.49,0.0009615)(197.50,0.0009625)(197.51,0.0009636)(197.52,0.0009646)(197.53,0.0009657)(197.54,0.0009667)(197.55,0.0009678)(197.56,0.0009689)(197.57,0.0009700)(197.58,0.0009711)(197.59,0.0009723)(197.60,0.0009734)(197.61,0.0009746)(197.62,0.0009757)(197.63,0.0009769)(197.64,0.0009781)(197.65,0.0009794)(197.66,0.0009806)(197.67,0.0009819)(197.68,0.0009831)(197.69,0.0009844)(197.70,0.0009857)(197.71,0.0009870)(197.72,0.0009884)(197.73,0.0009897)(197.74,0.0009911)(197.75,0.0009925)(197.76,0.0009939)(197.77,0.0009953)(197.78,0.0009968)(197.79,0.0009983)(197.80,0.0009998)(197.81,0.0010013)(197.82,0.0010028)(197.83,0.0010043)(197.84,0.0010059)(197.85,0.0010075)(197.86,0.0010091)(197.87,0.0010108)(197.88,0.0010124)(197.89,0.0010141)(197.90,0.0010158)(197.91,0.0010176)(197.92,0.0010193)(197.93,0.0010211)(197.94,0.0010229)(197.95,0.0010248)(197.96,0.0010266)(197.97,0.0010285)(197.98,0.0010304)(197.99,0.0010324)(198.00,0.0010344)};
		\legend{$2S^\prime_{Diff}$, $S_{LOO}-0{.}01$}
		\end{axis}
	\end{tikzpicture}
	\caption{Зависимость критериев кросс-валидации от параметра регуляризации $\mu$.}
	\label{tikz:newDiffLOO}
\end{figure}

		% \addcolumn{LOO_I}{}{
		% 	\hat S_{LOO, \hat I} &=0.0136, \\ \hat \beta &=0.156, \\ \hat \mu &=47.75
		% }
		% \addcolumn{diff_CV}{}{
		% 	\hat S_{Diff} &=0.0119, \\ \hat \beta &=0.160, \\ \hat \mu &=48.55
		% }
		% \addcolumn{honest_CV}{S_{LOO}(\lams)}{
		% 	\hat S_{LOO} &=0.0119, \\ \hat \beta &=0.159, \\ \hat \mu &=48.70
		% }

% subsection  (end)
	\section{Заключение}
		В работе изучены два алгоритма автоматического разделения текстов по тематикам, их модификаций, и выполнено их сведение к общей вычислительной схеме, позволяющей гибко настраивать параметры вероятностных тематических моделей.

\begin{itemize}
	\item Проведен анализ и произведено обобщение алгоритмов вероятностного латентного семантического анализа и латентого размещения Дирихле, сводящее их к единой вычислительной процедуре.
	\item Выполнен анализ возможностей варьирования параметров обобщенного алгоритма, влияющие на строгость и формальность производимых вычислений, а также изменяющие вероятностную модель порождения текстов.
	\item Проведен ряд численных экспериментов на реальных текстах, результаты которых позволяют говорить о возможности существенного ускорения сходимости алгоритмов путем грамотного подбора параметров.
\end{itemize}
	\newpage 
	\begin{thebibliography}{99}

\bibitem{hoff99} Hofmann T. \textit{Probabilistic latent semantic indexing}~// Proceedings of the 22nd annual international ACM SIGIR conference on Research and development in information retrieval, New York, NY, USA: ACM, 1999, Pp. 50–57.

\bibitem{hoff01} Hofmann T.	\textit{Unsupervised Learning by Probabilistic Latent Semantic Analysis}~// Machine Learning, 2001, Vol.~42, Pp.~177--196.

\bibitem{blei03}  D. M. Blei, A. Y. Ng, and M. I. Jordan. \textit{Latent dirichlet allocation} // Journal of Machine Learning Research, 2003, 3:993–1022

\bibitem{demp77} Dempster A. P., Laird N. M., Rubin D. B. \textit{Maximum likelihood from incomplete data via the EM algorithm}~// Journal of the Royal Statistical Society, Series B, 1977, no. 34, Pp. 1-38.

\bibitem{styv04} Steyvers M., Griffiths T. \textit{Finding scientific topics}~// Proceedings of the National
Academy of Sciences, 2004, Vol. 101, no. Suppl. 1, Pp. 5228–5235.

\bibitem{wang08} Wang Y.  \textit{Distributed Gibbs sampling of latent dirichlet allocation: The gritty
details}, 2008.

\bibitem{ronn89} Ronning G. \textit{Maximum likelihood estimation of dirichlet distributions}~// Journal of Statistical Computation and Simulation, 1989, 32:4, Pp.~215-221.

\bibitem{kull51} Kullback S., Leibler R. A.  \textit{On information and sufficiency}~// The Annals of Mathematical Statistics, 1951, Vol.~22, 1, Pp.~79-86.

\bibitem{asuc09} Asuncion A., Welling M., Smyth P., Teh Y. W. \textit{On smoothing and inference for topic models}~// Proceedings of the International Conference on Uncertainty in Artificial Intelligence, 2009.
\end{thebibliography}
		% 
У Жени всё в терминах $c$
$$
\hat y(x) = \sum_{j=1}^N c_j x_{ij}
$$

\begin{equation}
	\begin{cases}
		\suml_{i=1}^n c_i^2 + \suml_{j=1}^N \delta_j^2 \to \min_{\mb c, \mb \delta}, \\
	\delta_j = y_j - \suml_{i=1}^n c_ix_{ij}, j=1,\ldots,N. 
	\end{cases}
\end{equation}

\begin{equation}
	\begin{cases}
		\suml_{i=1}^n c_i^2 + \suml_{j=1}^N p_j\delta_j^2 \to \min_{\mb c, \mb \delta}, \\
	\delta_j = y_j - \suml_{i=1}^n c_ix_{ij}, j=1,\ldots,N. 
	\end{cases}
\end{equation}



\begin{equation}
L(\mb c,\idots \delta1N, \idots \lambda1N) 
	= \frac12 \mb c^T\mb c +  \frac12 \suml_{j=1}^N p_j\delta_j^2 + \sum_{j=1}^N\lambda_j\cbr{y_j  - \suml_{i=1}^n c_i x_{ij} - \delta_j }
\end{equation}

\begin{equation}
L(\beta) = - \left.\sum_{i=1}^N \dd{\delta_j^2(\mb p\cond \beta)}{p_j}\right|_{\mb p = (1, \ldots, 1)} \to \min_{\beta}
\end{equation}

\begin{equation}
\dd{\delta_j^2(\mb p)}{p_j} = 2 \delta_j(\mb p\cond \beta)\dd{\delta_j(\mb p)}{p_j} 
\end{equation}


		% 
Задача: найти выражание, связывающее $\hat{S}_{\text{LOO}}$ и $\hat S(\lambda_1, \lambda_2) - \mb e^T\dd{\hat{S}_{\lambda_1, \lambda_2}}{\mb p},$ где $\mb e = (1, \ldots, 1) \in \mathbb R^N$, при $\mb p = (1, \ldots, 1) \in \mathbb R^N$


Добавить:
- про природу, и почему мы используем эмпирическю функцию ошибки
- почему нет b: мы центрируем и нормируем
- написать то же самое, без b
- про индексы и их стабильность

\end{document}