\documentclass[12pt,a4paper]{amsart}
\usepackage{graphics,graphicx}
\usepackage{epsfig}
\usepackage{a4wide}
\usepackage[utf8]{inputenc}
\usepackage[T2A]{fontenc}
\usepackage[english,russian]{babel}
\usepackage{verbatim}
\usepackage{amssymb,amsfonts,amsthm,amsmath,mathtext,cite,enumerate,float}
\renewcommand{\baselinestretch}{1}

\begin{document}
\thispagestyle{empty}

\begin{center}
\sc
			Министерство образования и науки Российской Федерации\\
			Московский физико-технический институт {\rm(государственный университет)}\\
			Факультет управления и прикладной математики\\
			Вычислительный центр им. А. А. Дородницына РАН\\
			Кафедра <<Интеллектуальные системы>>\\[5mm]
\rm
\Large Отзыв на магистерскую диссертацию


\vspace{4pt}
\large
Дифференциальный скользящий контроль при выборе параметров регуляризации в задаче регрессионного анализа с помощью Elastic Net

\vspace{4pt}
\normalsize
студента 6 курса Бурмистрова Михаила Олеговича
\end{center}

Целью дипломной работы М.\,О. Бурмистрова является получение вычислительно эффективного подхода к кросс-валидации в задаче линейной регрессии с регуляризацией Elastic Net.
Рассматривается применение взвешивания объектов обучающей совокупности и дальнейшая вариация весов в окрестности единичных значений с целью оценить, как их изменение влияет на критерий качества в кросс-валидации.

\vspace{4pt}
В работе предложен принцип дифференциальной кросс-валидации, который определяется зависимостью регрессионных остатков на объектах от соответствующих им весов, а  также проведено теоретическое и экспериментально сравнение с существующим подходом к беспереборной кросс-валидации.

\vspace{4pt}
В ходе выполнения работы М.\,О. Бурмистров продемонстрировал способность самостоятельно проводить теоретические и экспериментальные исследования, адекватному сопоставлению полученных результатов с существующими. Предложенный в работе принцип дифференциального скользящего контроля является новым.

\vspace{4pt}
Работа является актуальным исследованием, удовлетворяет требованиям, предъявляемым к магистерским диссертациям на ФУПМ МФТИ, и заслуживает оценки <<отлично>>, а М. О. Бурмистров~--- присвоения квалификации магистра.

\vspace{5pt}
Научный руководитель: д.т.н., в.н.с. проф. ВЦ РАН В.\,В. Моттль.
\end{document}