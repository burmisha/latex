\documentclass[unicode,lefteqn,c,hyperref={pdfpagelabels=true}]{beamer}
\usepackage[utf8]{inputenc}
\usepackage{amssymb}
\usepackage{amsmath,mathrsfs}
\usepackage[russian]{babel}
\usepackage{ulem}\normalem
\usepackage{color}
\usepackage[noend]{algorithmic}

\usetheme{Warsaw}
\usefonttheme[onlylarge]{structurebold}
\setbeamerfont*{frametitle}{size=\normalsize,series=\bfseries}
\setbeamertemplate{navigation symbols}{}
\setbeameroption{show notes}
\definecolor{beamer@blendedblue}{RGB}{15,80,120}
\let\Tiny=\tiny
\def\shortspace{\hspace{1.5pt}}

\title[\hbox to 56mm{MDL principle \hfill\insertframenumber\,/\,\inserttotalframenumber}]{Принцип минимальной длины описания \\ Minimum description length principle}
\author[Михаил Бурмистров]{Михаил Бурмистров}
\institute{Основано на работе 
		Peter Gr\"unwald
		\vfill \textit{A Tutorial Introduction to 
		\vfill the Minimum Description Length Principle} 
		\vfill Centrum voor Wiskunde en Informatica, 
		\vfill The Netherlands, 2004.}
\date{\today}

\begin{document}

\begin{frame}
    \maketitle
\end{frame}

\begin{frame}
	\begin{block}{Основная задача}
	Построить критерий сравнения моделей обучения.
	\end{block}
	\begin{block}{Основная идея}
	Любая закономерность в данных может быть использована для того, чтобы их сжать. Поиск закономерностей и регулярностей в данных в подходе отождестваляется с обучением.
	\end{block}
	\textbf{Сжатие данных}~--- описание данных меньшим количеством символов, чем потребовалось бы для из записи обычным способом.
\end{frame}

\begin{frame}
	\textbf{Особенности подхода:}
	\begin{enumerate}
		\item выбор простейшей модели, по аналогии с бритвой Оккама,
		\item отсутствие переобучения, 
		\item наличие Байесовской интерпретации,
		\item отсутствие необходимости в априорных предположениях,
		\item построенная/выбранная модель обладает предсказательной силой.
	\end{enumerate}
\end{frame}

\begin{frame}{Примеры}
	\textbf{Последовательности из $0$ и $1$:}
	\begin{enumerate}
		\item \texttt{00010001000100010001\ldots 00010001000100010001}
		\item \texttt{01110100110100100110\ldots 10111011000101100010}
		\item \texttt{00011000001010100000\ldots 00010000001000110000}
	\end{enumerate}

	\textbf{Реальные неслучайные данные, допускающие сжатие:}
	\begin{enumerate}
		\item Последовательность цифр в записи $\pi$.
		\item Зависимость времени свободного падения шара от высоты, с которой его отпустили.
		\item Естественный язык, поскольку не любая последовательность слов 
	\end{enumerate}
\end{frame}


\end{document}
