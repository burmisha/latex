\documentclass[unicode,lefteqn,c,hyperref={pdfpagelabels=true}]{beamer}
\usepackage{burslides}
\newcommand\bivec[2]{\begin{pmatrix} #1 \\ #2 \end{pmatrix}}

\newcommand\ol[1]{\overline{#1}}

\newcommand\p[1]{\Prob\!\left(#1\right)}
\newcommand\e[1]{\mathsf{E}\!\left(#1\right)}
\newcommand\disp[1]{\mathsf{D}\!\left(#1\right)}
%\newcommand\norm[2]{\mathcal{N}\!\cbr{#1,#2}}
\newcommand\sign{\text{ sign }}

\newcommand\al[1]{\begin{align*} #1 \end{align*}}
\newcommand\begcas[1]{\begin{cases}#1\end{cases}}
\newcommand\tab[2]{	\vspace{-#1pt}
						\begin{tabbing} 
						#2
						\end{tabbing}
					\vspace{-#1pt}
					}

\newcommand\maintext[1]{{\bfseries\sffamily{#1}}}
\newcommand\skipped[1]{ {\ensuremath{\text{\small{\sffamily{Пропущено:} #1} } } } }
\newcommand\simpletitle[1]{\begin{center} \maintext{#1} \end{center}}

\def\le{\leqslant}
\def\ge{\geqslant}
\def\Ell{\mathcal{L}}
\def\eps{{\varepsilon}}
\def\Rn{\mathbb{R}^n}
\def\RSS{\mathsf{RSS}}

\newcommand\foral[1]{\forall\,#1\:}
\newcommand\exist[1]{\exists\,#1\:\colon}

\newcommand\cbr[1]{\left(#1\right)} %circled brackets
\newcommand\fbr[1]{\left\{#1\right\}} %figure brackets
\newcommand\sbr[1]{\left[#1\right]} %square brackets
\newcommand\modul[1]{\left|#1\right|}

\newcommand\sqr[1]{\cbr{#1}^2}
\newcommand\inv[1]{\cbr{#1}^{-1}}

\newcommand\cdf[2]{\cdot\frac{#1}{#2}}
\newcommand\dd[2]{\frac{\partial#1}{\partial#2}}

\newcommand\integr[2]{\int\limits_{#1}^{#2}}
\newcommand\suml[2]{\sum\limits_{#1}^{#2}}
\newcommand\isum[2]{\sum\limits_{#1=#2}^{+\infty}}
\newcommand\idots[3]{#1_{#2},\ldots,#1_{#3}}
\newcommand\fdots[5]{#4{#1_{#2}}#5\ldots#5#4{#1_{#3}}}

\newcommand\obol[1]{O\!\cbr{#1}}
\newcommand\omal[1]{o\!\cbr{#1}}

\newcommand\addeps[2]{
	\begin{figure} [!ht] %lrp
		\centering
		\includegraphics[height=320px]{#1.eps}
		\vspace{-10pt}
		\caption{#2}
		\label{eps:#1}
	\end{figure}
}

\newcommand\addepssize[3]{
	\begin{figure} [!ht] %lrp hp
		\centering
		\includegraphics[height=#3px]{#1.eps}
		\vspace{-10pt}
		\caption{#2}
		\label{eps:#1}
	\end{figure}
}


\newcommand\norm[1]{\ensuremath{\left\|{#1}\right\|}}
\newcommand\ort{\bot}
\newcommand\theorem[1]{{\sffamily Теорема #1\ }}
\newcommand\lemma[1]{{\sffamily Лемма #1\ }}
\newcommand\difflim[2]{\frac{#1\cbr{#2 + \Delta#2} - #1\cbr{#2}}{\Delta #2}}
\renewcommand\proof[1]{\par\noindent$\square$ #1 \hfill$\blacksquare$\par}
\newcommand\defenition[1]{{\sffamilyОпределение #1\ }}

\title[\hbox to 56mm{MDL principle \hfill\insertframenumber\,/\,\inserttotalframenumber}]{Принцип минимальной длины описания \\ Minimum description length principle}
\author[Михаил Бурмистров]{Михаил Бурмистров}
\institute{Основано на работе 
		Peter Gr\"unwald
		\vfill \textit{A Tutorial Introduction to 
		\vfill the Minimum Description Length Principle} 
		\vfill Centrum voor Wiskunde en Informatica, 
		\vfill The Netherlands, 2004.}
\date{\today}

\begin{document}

\begin{frame}
    \maketitle
\end{frame}

\begin{frame}
	\begin{block}{Основная задача}
	Построить критерий сравнения моделей обучения.
	\end{block}
	\begin{block}{Основная идея}
	Любая закономерность в данных может быть использована для того, чтобы их сжать. Поиск закономерностей и регулярностей в данных в подходе отождестваляется с обучением.
	\end{block}
	\textbf{Сжатие данных}~--- описание данных меньшим количеством символов, чем потребовалось бы для из записи обычным способом.
\end{frame}

\begin{frame}
	\textbf{Особенности подхода:}
	\begin{enumerate}
		\item выбор простейшей модели, по аналогии с бритвой Оккама,
		\item отсутствие переобучения, 
		\item наличие Байесовской интерпретации,
		\item отсутствие необходимости в априорных предположениях,
		\item построенная/выбранная модель обладает предсказательной силой.
	\end{enumerate}
\end{frame}

\begin{frame}{Примеры}
	\textbf{Реальные неслучайные данные, допускающие сжатие:}
	\begin{enumerate}
		\item Последовательность цифр в записи $\pi$.
		\item Зависимость времени свободного падения шара от высоты, с которой его отпустили.
		\item Естественный язык, поскольку не любая последовательность слов синтаксически корректна.
	\end{enumerate}

	\textbf{Последовательности из $0$ и $1$:}
	\begin{enumerate}
		\item \texttt{00010001000100010001\ldots 00010001000100010001}
		\item \texttt{01110100110100100110\ldots 10111011000101100010}
		\item \texttt{00011000001010100000\ldots 00010000001000110000}
	\end{enumerate}
\end{frame}

\begin{frame}{Идеализированный подход}
	\begin{block}{Идеализированный MDL}
	Использование в качестве критерия сложности модели данных Колмогоровскую сложность её описания.
	\end{block}

	\textbf{Проблемы:}
	\begin{enumerate}
		\item невычислимость,
		\item зависимость результата от выбранной универсальной машины Тьюринга.
	\end{enumerate}
\end{frame}

\begin{frame}{Практический подход}
	\begin{block}{Практический MDL}
	Ограничиться при изучении моделей лишь частью из всех возможных. 
	При этом язык описания моделей должен быть достаточно богат, чтобы сжимать <<интуитивно>> регулярные последовательности.
	\end{block}

	\textbf{Преимущества:}
	\begin{enumerate}
		\item вычислимость.
	\end{enumerate}
\end{frame}

\newcommand\mch[1]{\mathcal H^{\cbr{#1}}}
\begin{frame}{Черновой вариант MDL}
	\textbf{Терминология:}
	\begin{enumerate}
		\item Точечная гипотеза~--- конкретная функция, алгоритм, распределение. 
		Например, полином $4x^2+0{,}119x+93.$
		\item Модель~--- совокупность точечных гипотез. Например, все полиномы степени 3.
	\end{enumerate}
	\begin{block}{Практический MDL}
	Пусть $\mch 1, \mch 2, \ldots$~--- изучаемые модели данных. Лучшей точечной гипотезой $H \in \mch 1 \cup \mch 2 \cup \ldots$ для данных $D$ назовём 
	$H = \argmin \cbr{L(H) + L(D|H)},$
	где 
	\begin{itemize}
		\item $L(H)$~--- длина описания гипотезы,
		\item $L(D|H)$~--- длина описания данных при известной гипотезе.
	\end{itemize}
	Лучшая модель~--- $\hat {\mathcal H} = \mch i: H \in \mch i.$
	\end{block}
\end{frame}

\begin{frame}{Черновой вариант MDL}
	\textbf{Требуемые уточнения:}
	\begin{itemize}
		\item $L(D|H)$~--- отклонение данных от гипотезы. Например, если $Y=H(X) + Z, Z \sim \mathcal N(0,\sigma^2),$ то алгоритм кодирования Шеннона-Фано даёт $L(D|H)=-\log P(D|H).$ 
		\item $L(H)$~--- всё также зависит от способа кодирования,
		\begin{itemize}
			\item минимаксный подход (наикратчайший код для описания самых плохих входных данных) ведёт к необходимости сложной дискретизации пространства моделей $\mathcal{H}$
		\end{itemize}
	\end{itemize}

	\textbf{Refined MDL:} будем кодировать сразу все данные целиком, без разбиения на модель и поправку. Получим $\bar L(D|\mathcal H),$ т.н. стохастическую сложность данных по отношению к модели.
	
\end{frame}

\end{document}
