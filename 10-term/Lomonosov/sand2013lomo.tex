\documentclass[12pt,a4paper,oneside]{article}
\usepackage[T2A]{fontenc}
\usepackage[utf8]{inputenc}       
\usepackage[english,russian]{babel}

\begin{document}
    \title{Вероятностный подход к одноклассовой классификации}
    \author{Любовь Сандуляну}
	\date{\today}
	\maketitle

Проникновение компьюьтерных технологий в современную жизнь позволяет использовать методы машинного обучения в самых различных сферах жизни. При этом существуют задачи, в которых исследуемые классы объектов существенно различны. Так, например, объекты одного из классов могут
\begin{itemize}
    	\item быть менее доступны,
		\item обладать высокой разнородностью,
		\item содержать большое число крупных компактных кластеров,
\end{itemize}
а объекты другого класса обладать противоположными свойствами. 

В таких случаях может оказаться целесообразным полностью отказаться от построения традиционных классификаторов, а попытаться использовать одноклассовый классификатор, обученные лишь на объектах одного из классов. Ранее в работе \cite{Tax2001} предлагалось строить сферический пороговый классификатор на основе эвристических соображений. 

В данной работе \cite{JMLDA2012no4} была сделана вероятностная постановка задачи обучения классификатора. Для этого делается вероятностное предположение о распределении точек в признаковом пространсте и об априорных параметрах этого распределения. 
При этом полученный классификатор также является пороговым сферическим, а параметры сферы (центр и радиус) определяются из принципа максимума апостериорной вероятности. Вероятностная постановка задачи обучения позволяет определить область применимости построенной модели.

Полученная задача оптимизации оказывается задачей квадратичной выпуклой оптимизации, которую можно решать стандартными подходами. В работе поставлены эксперименты на модельных
% и реальных 
данных, показывающие эффективность подхода.

%В работе вводится квазивероятностная модель для классической эмпирической постановки задачи одноклассовой классификации и 
%производится сведение классического подхода к новой модели.
%
%Рассматривается одноклассовая классификация объектов генеральной совокупности. В работе \cite{Tax2001}  предлагается строить
% сферический пороговый классификатор без вероятностного обоснования такого подхода. В нашей работе будем придерживаться 
%вероятностной модели распределения объектов генеральной совокупности в признаковом пространстве. Делается предположение о виде этого
%распределения, которое зависит от двух параметров: центра и радиуса сферы. Из принципа максимума плотности апостериорного
%распределения в пространстве параметров модели генеральной совокупности было получено байесовское правило обучения. В 
%работе доказано, что задача в такой постановке является обобщением классической эмпирической постановки задачи 

%TODO: поменять списки литературы, чтобы не совпадали.

\begin{thebibliography}{99}
	\bibitem{JMLDA2012no4} Бурмистров М.\,О., Сандуляну Л.\,Н. \textit{Вероятностная модель одноклассовой классификации} // Машинное обучение и анализ данных, М., 2012. T. 1, №4, стр. 420--427.
	\bibitem{Tax2001} D.\,Tax \textit{One-class classification; Concept-learning in the absence of
	counterexamples}, Ph.D thesis, 2001.
\end{thebibliography}

\end{document}
