\documentclass[12pt,a4paper,oneside]{article}
\usepackage[T2A]{fontenc}
\usepackage[utf8]{inputenc}       
\usepackage[english,russian]{babel}

\textheight=28cm
\textwidth=17.5cm
\oddsidemargin=-1cm
\evensidemargin=-1cm
\topmargin=-4cm

\begin{document}
	\title{Одноклассовая классификация с использованием потенциальных функций.}
	\author{Михаил Бурмистров}
	\date{\today}
	\maketitle

В современном мире электронная почта широко распространена и используется во многих сферах человеческой жизни. 
В силу высокой открытости этот канала обмена сообщениями стал активно использоваться мошенниками и злоумышленниками, что поставило задачу фильтрации спама. 
Эта задача решается различными методами, использующие как эвристические \cite{Islam2007, Sun2008}, так и вероятностные постановки . 
При этом отдельно стоит задача составления «хорошей» обучающей выборки, поскольку при разбиении писем на 2 класса «спам»/«не спам», получившиеся классы существенно разнородны. Так, письма, представляющие интерес для пользователя, обладают
\begin{itemize}
		\item меньшей доступностью,
		\item высокой разнородностью,
		\item большим числом шаблонных писем (разнообразные уведомления от сервисов).
\end{itemize}
По этим причинам представляется разумным построить классификатор, обученный лишь на объектах одного класса. 

В предыдущей работе \cite{JMLDA2012no4} предложен вероятностный подход к решению задачи одноклассовой классификации. 
Решающее правило при этом оказывается основано на определении принадлежности точки в признаковом пространстве, соответствующей рассматриваемому объекту, некоторой гиперсфере, радиус и центр которой определяются на этапе обучения модели. 
Этот результат совпадает с эвристической постановкой задачи, предложенной в \cite{Tax2001}. 

Однако, использование гиперсферы при принятии решения оказывается слишком частным случаем, и качество классификации можно повысить, заменив в модели скалярные произведения на потенциальные функции. 
При этом классификатор способен строить более богатое семейство разделяющих поверхностей, что позволяет повысить обобщающую способность. 
В работе приведено соответствующее обобщение модели и проведены вычислительные эксперименты на модельных и реальных данных, показывающие эффективность использования такого подхода. 

\begin{thebibliography}{99}
	\bibitem{Islam2007} R.\,Islam, U.\,Chowdhury \textit{Spam filtering using ML algorithms} // Universitetets Okonomiske Institute, IADIS International Conference on WWW/Internet, 2007.
	\bibitem{Sun2008} J.\,Sun, Q.\,Zhang, Z.\,Yuan, W.\,Huang, X.\,Yan, J.\,Dong \textit{Research of Spam Filtering system based on LSA and SHA} // Advances in neural networks --- ISNN 2008, 2008.
	\bibitem{JMLDA2012no4} Бурмистров М.\,О., Сандуляну Л.\,Н. \textit{Вероятностная модель одноклассовой классификации} // Машинное обучение и анализ данных, М., 2012. T. 1, №4, стр. 420--427.
	\bibitem{Tax2001} D.\,Tax \textit{One-class classification; Concept-learning in the absence of
	counterexamples}, Ph.D thesis, 2001.
\end{thebibliography}

\end{document}