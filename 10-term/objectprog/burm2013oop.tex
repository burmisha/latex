\input header.tex
\input pagestyle.tex
\input macro.tex

\begin{document}
\simpletitle{Методы оптимизации параметров \\ вероятностных тематических моделей. \\ Михаил Бурмистров. \today}

\textbf{Исходные данные:}
\begin{itemize}
		\item $W$~--- конечное множество (словарь) слов;
		\item $D$~--- конечное множество (коллекция) документов;
		\item $n_{dw}$~--- число вхождений слова~$w\in W$ в~документ $d\in D$;
\end{itemize}

\textbf{Найти:}
\begin{itemize}
		\item $\theta_{td} = p(t|d)$ --- распределение тем в~каждом документе~$d$;
		\item $\phi_{wt} = p(w|t)$ --- распределение слов в~каждой теме~$t$;
\end{itemize}

\textbf{Основные гипотезы вероятностной тематической модели:}
\begin{itemize}
		\item $\exists{T}$ --- конечное множество латентных тем;
		\item $D\times W\times T$ --- вероятностное пространство с~$p(d,w,t)$;
		\item порядок слов в~документах не~важен;
		\item $p(w|d,t) = p(w|t)$ --- гипотеза условной независимости;
\end{itemize}

\simpletitle{Сравнение, модификации и обобщение методов}

Для идентификации параметров $\Theta$, $\Phi$ используется \rlap{EM-алгоритм}.

\textbf{Основные модификации:}
\begin{itemize}
	\item PLSA-GEM позволяет ускорить сходимость за счет более частого обновления параметров;
	\item LDA-GS: одно вхождение слова в документ соответствует одной теме
\end{itemize}

\textbf{Цели работы:}
\begin{itemize}
	\item обобщение известных алгоритмов тематического моделирования (PLSA-GEM и LDA-GS) в рамках одного численного метода;
	\item исследование зависимости качества построенной модели от параметров полученного метода.
\end{itemize}

\simpletitle{Соединение PLSA-GEM и LDA-GS в общий численный метод}
\begin{algorithmic}[1]
    \REQUIRE
		$D$, $|T|$, приближения $\Theta$~и~$\Phi$, гиперпараметры $\alpha$, $\beta$;
    \ENSURE
        распределения $\Theta$~и~$\Phi$;
	\STATE $(t; h)_{dwi} = 0 $ для всех $d\in D, w \in W, {i \in I_{dw}}$;
    \REPEAT
        \FORALL{$d\in D$,\; $w\in d$,\; $i\in I_{dw}$}
			\STATE подготовить тему $t_{dwi}$; \; $t$ = $t_{dwi}$;			
			\STATE вычислить значение $\delta$;
			\STATE увеличить $h_{dwi}$, $\varphi(w|t)$, $\theta(t|d)$ на $\delta$;		
			\IF{пора обновить параметры $\Phi$, $\Theta$}
				\STATE обновить $\Theta$~и~$\Phi$;
			\ENDIF
        \ENDFOR
    \UNTIL $\Theta$ и~$\Phi$ не~стабилизируются.
\end{algorithmic} 

\simpletitle{Объектно-ориентированный подход}
Для удобства использования и тестирования различных параметров программы предлагается применить подходы ООП.

Есть классы-прототипы, обеспечивающие общие интерфейсы доступа к пересчёту параметров, а также дающие возможность изменять критерии частоты обновления параметров и останова алгоритма.
%\newcommand\newcl[1]{\item $\mathfrak{#1}$~--- }
\newcommand\newcl[1]{\item \texttt{#1}~--- }

Классы:
\begin{itemize}
	\newcl{Dictionary} предоставляет общий интерфейс для управления используемым словарём.
	\newcl{Distance} позволяет использовать различные метрики качества при сравнении распределений.
	\newcl{GetDelta} вычисление приращения параметров модели.
	\newcl{Prepare} предобработка каждой новой пары документ-слово.
	\newcl{UpdateTime} критерий обновления параметров модели.
\end{itemize}

От каждого из классов-прототипов наследуются классы, реализующие конкретные способы обработки распределений $\Theta$ и~$\Phi$.

\begin{itemize}
	\newcl{Dictionary} 
	\begin{itemize}
		\newcl{Map} использование \texttt{std::map}
	\end{itemize}
	\newcl{Distance} 
	\begin{itemize}
		\newcl{KullbackLeibler} расстояние Кульбака-Лейблера
		\newcl{Chi\_Squared} расстояние $\chi^2$
		\newcl{Hellinger} метрика Хеллингера
	\end{itemize}
	\newcl{GetDelta} 
	\begin{itemize}
		\newcl{PLSA\_GetDelta} для PLSA
		\newcl{LDA\_GetDelta} для LDA
	\end{itemize}
	\newcl{Prepare} 
	\begin{itemize}
		\newcl{PLSA\_prepare} для PLSA
		\newcl{LDA\_prepare} для LDA
	\end{itemize}
	\newcl{UpdateTime} 
	\begin{itemize}
		\newcl{update\_every} обновлять каждые несколько раз, когда встаёт вопрос об этом,
		\newcl{update\_after\_document} обновлять после каждого обработанного документа
		\newcl{update\_after\_collection} обновлять после всей коллекции
	\end{itemize}
\end{itemize}

\end{document}
