\setdate{30~марта~2021}
\setclass{10«АБ»}

\addpersonalvariant{Михаил Бурмистров}

\tasknumber{1}%
\task{%
    Два одинаковых маленьких проводящих заряженных шарика находятся на расстоянии~$l$ друг от друга.
    Заряд первого равен~$+3Q$, второго~--- $+8Q$.
    Шарики приводят в соприкосновение, а после опять разводят на расстояние~$3l$.
    \begin{itemize}
        \item Каким стал заряд каждого из шариков?
        \item Определите характер (притяжение или отталкивание) и силу взаимодействия шариков до и после соприкосновения.
        \item Как изменилась сила взаимодействия шариков после соприкосновения?
    \end{itemize}
}
\answer{%
    \begin{align*}
    F &= k\frac{\abs{q_1}\abs {q_2}}{\sqr{3 l}}   = k\frac{\abs{+3Q} \cdot \abs{+8Q}}{3^2 \cdot l^2}, \text{отталкивание}; \\
        q'_1 &= q'_2, q_1 + q_2 = q'_1 + q'_2 \implies  q'_1 = q'_2 = \frac{q_1 + q_2}2 = \frac{+3Q + +8Q}2 = \frac{11}2Q \implies \\
        \implies F'  &= k\frac{\abs{q'_1}\abs{q'_2}}{\sqr{3 l}}
            = k\frac{\sqr{\frac{11}2Q}}{3^2 \cdot l^2},
        \text{отталкивание}, \\
    \frac{F'}F &= \frac{\sqr{\frac{11}2Q}}{3^2 \cdot \abs{+3Q} \cdot \abs{+8Q}} = \frac{121}{864}.
    \end{align*}
}
\solutionspace{120pt}

\tasknumber{2}%
\task{%
    На координатной плоскости в точках $(-d; 0)$ и $(d; 0)$
    находятся заряды, соответственно, $+q$ и $-q$.
    Сделайте рисунок, определите величину напряжённости электрического поля
    и укажите её направление в двух точках: $(0; d)$ и $(-2d; 0)$.
}
\solutionspace{120pt}

\tasknumber{3}%
\task{%
    Напряжение между двумя точками, лежащими на одной линии напряжённости
    однородного электрического поля, равно $U = 4\,\text{кВ}$.
    Расстояние между точками $r = 20\,\text{см}$.
    Какова напряжённость этого поля?
}
\answer{%
    $
        E_x = -\frac{\Delta \varphi}{\Delta x} \implies
        E = \frac{U}{r} = \frac{4\,\text{кВ}}{20\,\text{см}} = 20{,}0\,\frac{\text{кВ}}{\text{м}}.
    $
}
\solutionspace{40pt}

\tasknumber{4}%
\task{%
    В однородном электрическом поле напряжённостью $E = 20\,\frac{\text{кВ}}{\text{м}}$
    переместили заряд $Q = -40\,\text{нКл}$ в направлении силовой линии
    на $l = 10\,\text{см}$.
    Определите
    \begin{itemize}
        \item работу поля,
        \item изменение потенциальной энергии заряда.
        % \item напряжение между начальной и конечной точками перемещения.
    \end{itemize}
}
\answer{%
    \begin{align*}
    A &= F \cdot l \cdot \cos \alpha = EQ \cdot l \cdot 1 = EQl = 20\,\frac{\text{кВ}}{\text{м}} \cdot -40\,\text{нКл} \cdot 10\,\text{см} = -80{,}00\,\text{мкДж}, \\
    \Delta E_\text{пот.} &= -A = 80{,}0\,\text{мкДж}
    \end{align*}
}
\solutionspace{80pt}

\tasknumber{5}%
\task{%
    \begin{enumerate}
        \item Запишите (формулой) закон Кулона (в вакууме).
        \item Зарисуйте электрическое поле точечного положительного электрического заряда.
        \item Запишите формулу для вычисления напряжённости электрического поля точечного заряда.
        \item Запишите принцип суперпозиции (правило сложения) потенциалов.
    \end{enumerate}
}

\variantsplitter

\addpersonalvariant{Ирина Ан}

\tasknumber{1}%
\task{%
    Два одинаковых маленьких проводящих заряженных шарика находятся на расстоянии~$d$ друг от друга.
    Заряд первого равен~$+5q$, второго~--- $+4q$.
    Шарики приводят в соприкосновение, а после опять разводят на расстояние~$2d$.
    \begin{itemize}
        \item Каким стал заряд каждого из шариков?
        \item Определите характер (притяжение или отталкивание) и силу взаимодействия шариков до и после соприкосновения.
        \item Как изменилась сила взаимодействия шариков после соприкосновения?
    \end{itemize}
}
\answer{%
    \begin{align*}
    F &= k\frac{\abs{q_1}\abs {q_2}}{\sqr{2 d}}   = k\frac{\abs{+5q} \cdot \abs{+4q}}{2^2 \cdot d^2}, \text{отталкивание}; \\
        q'_1 &= q'_2, q_1 + q_2 = q'_1 + q'_2 \implies  q'_1 = q'_2 = \frac{q_1 + q_2}2 = \frac{+5q + +4q}2 = \frac92q \implies \\
        \implies F'  &= k\frac{\abs{q'_1}\abs{q'_2}}{\sqr{2 d}}
            = k\frac{\sqr{\frac92q}}{2^2 \cdot d^2},
        \text{отталкивание}, \\
    \frac{F'}F &= \frac{\sqr{\frac92q}}{2^2 \cdot \abs{+5q} \cdot \abs{+4q}} = \frac{81}{320}.
    \end{align*}
}
\solutionspace{120pt}

\tasknumber{2}%
\task{%
    На координатной плоскости в точках $(-d; 0)$ и $(d; 0)$
    находятся заряды, соответственно, $-q$ и $-q$.
    Сделайте рисунок, определите величину напряжённости электрического поля
    и укажите её направление в двух точках: $(0; -d)$ и $(2d; 0)$.
}
\solutionspace{120pt}

\tasknumber{3}%
\task{%
    Напряжение между двумя точками, лежащими на одной линии напряжённости
    однородного электрического поля, равно $U = 6\,\text{кВ}$.
    Расстояние между точками $r = 30\,\text{см}$.
    Какова напряжённость этого поля?
}
\answer{%
    $
        E_x = -\frac{\Delta \varphi}{\Delta x} \implies
        E = \frac{U}{r} = \frac{6\,\text{кВ}}{30\,\text{см}} = 20{,}0\,\frac{\text{кВ}}{\text{м}}.
    $
}
\solutionspace{40pt}

\tasknumber{4}%
\task{%
    В однородном электрическом поле напряжённостью $E = 4\,\frac{\text{кВ}}{\text{м}}$
    переместили заряд $Q = -25\,\text{нКл}$ в направлении силовой линии
    на $r = 2\,\text{см}$.
    Определите
    \begin{itemize}
        \item работу поля,
        \item изменение потенциальной энергии заряда.
        % \item напряжение между начальной и конечной точками перемещения.
    \end{itemize}
}
\answer{%
    \begin{align*}
    A &= F \cdot r \cdot \cos \alpha = EQ \cdot r \cdot 1 = EQr = 4\,\frac{\text{кВ}}{\text{м}} \cdot -25\,\text{нКл} \cdot 2\,\text{см} = -2{,}00\,\text{мкДж}, \\
    \Delta E_\text{пот.} &= -A = 2{,}0\,\text{мкДж}
    \end{align*}
}
\solutionspace{80pt}

\tasknumber{5}%
\task{%
    \begin{enumerate}
        \item Запишите (формулой) закон Кулона (в вакууме).
        \item Зарисуйте электрическое поле точечного отрицательного электрического заряда.
        \item Запишите формулу для вычисления потенциала электрического поля точечного заряда.
        \item Запишите принцип суперпозиции (правило сложения) потенциалов.
    \end{enumerate}
}

\variantsplitter

\addpersonalvariant{Софья Андрианова}

\tasknumber{1}%
\task{%
    Два одинаковых маленьких проводящих заряженных шарика находятся на расстоянии~$l$ друг от друга.
    Заряд первого равен~$+5q$, второго~--- $+8q$.
    Шарики приводят в соприкосновение, а после опять разводят на расстояние~$3l$.
    \begin{itemize}
        \item Каким стал заряд каждого из шариков?
        \item Определите характер (притяжение или отталкивание) и силу взаимодействия шариков до и после соприкосновения.
        \item Как изменилась сила взаимодействия шариков после соприкосновения?
    \end{itemize}
}
\answer{%
    \begin{align*}
    F &= k\frac{\abs{q_1}\abs {q_2}}{\sqr{3 l}}   = k\frac{\abs{+5q} \cdot \abs{+8q}}{3^2 \cdot l^2}, \text{отталкивание}; \\
        q'_1 &= q'_2, q_1 + q_2 = q'_1 + q'_2 \implies  q'_1 = q'_2 = \frac{q_1 + q_2}2 = \frac{+5q + +8q}2 = \frac{13}2q \implies \\
        \implies F'  &= k\frac{\abs{q'_1}\abs{q'_2}}{\sqr{3 l}}
            = k\frac{\sqr{\frac{13}2q}}{3^2 \cdot l^2},
        \text{отталкивание}, \\
    \frac{F'}F &= \frac{\sqr{\frac{13}2q}}{3^2 \cdot \abs{+5q} \cdot \abs{+8q}} = \frac{169}{1440}.
    \end{align*}
}
\solutionspace{120pt}

\tasknumber{2}%
\task{%
    На координатной плоскости в точках $(-d; 0)$ и $(d; 0)$
    находятся заряды, соответственно, $+q$ и $-q$.
    Сделайте рисунок, определите величину напряжённости электрического поля
    и укажите её направление в двух точках: $(0; -d)$ и $(-2d; 0)$.
}
\solutionspace{120pt}

\tasknumber{3}%
\task{%
    Напряжение между двумя точками, лежащими на одной линии напряжённости
    однородного электрического поля, равно $V = 5\,\text{кВ}$.
    Расстояние между точками $d = 10\,\text{см}$.
    Какова напряжённость этого поля?
}
\answer{%
    $
        E_x = -\frac{\Delta \varphi}{\Delta x} \implies
        E = \frac{V}{d} = \frac{5\,\text{кВ}}{10\,\text{см}} = 50{,}0\,\frac{\text{кВ}}{\text{м}}.
    $
}
\solutionspace{40pt}

\tasknumber{4}%
\task{%
    В однородном электрическом поле напряжённостью $E = 4\,\frac{\text{кВ}}{\text{м}}$
    переместили заряд $q = 10\,\text{нКл}$ в направлении силовой линии
    на $l = 4\,\text{см}$.
    Определите
    \begin{itemize}
        \item работу поля,
        \item изменение потенциальной энергии заряда.
        % \item напряжение между начальной и конечной точками перемещения.
    \end{itemize}
}
\answer{%
    \begin{align*}
    A &= F \cdot l \cdot \cos \alpha = Eq \cdot l \cdot 1 = Eql = 4\,\frac{\text{кВ}}{\text{м}} \cdot 10\,\text{нКл} \cdot 4\,\text{см} = 1{,}6\,\text{мкДж}, \\
    \Delta E_\text{пот.} &= -A = -1{,}600\,\text{мкДж}
    \end{align*}
}
\solutionspace{80pt}

\tasknumber{5}%
\task{%
    \begin{enumerate}
        \item Запишите (формулой) закон сохранения электрического заряда.
        \item Зарисуйте электрическое поле точечного положительного электрического заряда.
        \item Запишите формулу для вычисления напряжённости электрического поля точечного заряда.
        \item Запишите принцип суперпозиции (правило сложения) потенциалов.
    \end{enumerate}
}

\variantsplitter

\addpersonalvariant{Владимир Артемчук}

\tasknumber{1}%
\task{%
    Два одинаковых маленьких проводящих заряженных шарика находятся на расстоянии~$d$ друг от друга.
    Заряд первого равен~$-5q$, второго~--- $-8q$.
    Шарики приводят в соприкосновение, а после опять разводят на расстояние~$4d$.
    \begin{itemize}
        \item Каким стал заряд каждого из шариков?
        \item Определите характер (притяжение или отталкивание) и силу взаимодействия шариков до и после соприкосновения.
        \item Как изменилась сила взаимодействия шариков после соприкосновения?
    \end{itemize}
}
\answer{%
    \begin{align*}
    F &= k\frac{\abs{q_1}\abs {q_2}}{\sqr{4 d}}   = k\frac{\abs{-5q} \cdot \abs{-8q}}{4^2 \cdot d^2}, \text{отталкивание}; \\
        q'_1 &= q'_2, q_1 + q_2 = q'_1 + q'_2 \implies  q'_1 = q'_2 = \frac{q_1 + q_2}2 = \frac{-5q -8q}2 = -\frac{13}2q \implies \\
        \implies F'  &= k\frac{\abs{q'_1}\abs{q'_2}}{\sqr{4 d}}
            = k\frac{\sqr{-\frac{13}2q}}{4^2 \cdot d^2},
        \text{отталкивание}, \\
    \frac{F'}F &= \frac{\sqr{-\frac{13}2q}}{4^2 \cdot \abs{-5q} \cdot \abs{-8q}} = \frac{169}{2560}.
    \end{align*}
}
\solutionspace{120pt}

\tasknumber{2}%
\task{%
    На координатной плоскости в точках $(-l; 0)$ и $(l; 0)$
    находятся заряды, соответственно, $-q$ и $-q$.
    Сделайте рисунок, определите величину напряжённости электрического поля
    и укажите её направление в двух точках: $(0; -l)$ и $(-2l; 0)$.
}
\solutionspace{120pt}

\tasknumber{3}%
\task{%
    Напряжение между двумя точками, лежащими на одной линии напряжённости
    однородного электрического поля, равно $U = 2\,\text{кВ}$.
    Расстояние между точками $d = 10\,\text{см}$.
    Какова напряжённость этого поля?
}
\answer{%
    $
        E_x = -\frac{\Delta \varphi}{\Delta x} \implies
        E = \frac{U}{d} = \frac{2\,\text{кВ}}{10\,\text{см}} = 20{,}0\,\frac{\text{кВ}}{\text{м}}.
    $
}
\solutionspace{40pt}

\tasknumber{4}%
\task{%
    В однородном электрическом поле напряжённостью $E = 20\,\frac{\text{кВ}}{\text{м}}$
    переместили заряд $q = 40\,\text{нКл}$ в направлении силовой линии
    на $r = 4\,\text{см}$.
    Определите
    \begin{itemize}
        \item работу поля,
        \item изменение потенциальной энергии заряда.
        % \item напряжение между начальной и конечной точками перемещения.
    \end{itemize}
}
\answer{%
    \begin{align*}
    A &= F \cdot r \cdot \cos \alpha = Eq \cdot r \cdot 1 = Eqr = 20\,\frac{\text{кВ}}{\text{м}} \cdot 40\,\text{нКл} \cdot 4\,\text{см} = 32{,}0\,\text{мкДж}, \\
    \Delta E_\text{пот.} &= -A = -32{,}00\,\text{мкДж}
    \end{align*}
}
\solutionspace{80pt}

\tasknumber{5}%
\task{%
    \begin{enumerate}
        \item Запишите (формулой) закон Кулона (в вакууме).
        \item Зарисуйте электрическое поле точечного положительного электрического заряда.
        \item Запишите формулу для вычисления потенциала электрического поля точечного заряда.
        \item Запишите принцип суперпозиции (правило сложения) потенциалов.
    \end{enumerate}
}

\variantsplitter

\addpersonalvariant{Софья Белянкина}

\tasknumber{1}%
\task{%
    Два одинаковых маленьких проводящих заряженных шарика находятся на расстоянии~$r$ друг от друга.
    Заряд первого равен~$-3Q$, второго~--- $-4Q$.
    Шарики приводят в соприкосновение, а после опять разводят на расстояние~$4r$.
    \begin{itemize}
        \item Каким стал заряд каждого из шариков?
        \item Определите характер (притяжение или отталкивание) и силу взаимодействия шариков до и после соприкосновения.
        \item Как изменилась сила взаимодействия шариков после соприкосновения?
    \end{itemize}
}
\answer{%
    \begin{align*}
    F &= k\frac{\abs{q_1}\abs {q_2}}{\sqr{4 r}}   = k\frac{\abs{-3Q} \cdot \abs{-4Q}}{4^2 \cdot r^2}, \text{отталкивание}; \\
        q'_1 &= q'_2, q_1 + q_2 = q'_1 + q'_2 \implies  q'_1 = q'_2 = \frac{q_1 + q_2}2 = \frac{-3Q -4Q}2 = -\frac72Q \implies \\
        \implies F'  &= k\frac{\abs{q'_1}\abs{q'_2}}{\sqr{4 r}}
            = k\frac{\sqr{-\frac72Q}}{4^2 \cdot r^2},
        \text{отталкивание}, \\
    \frac{F'}F &= \frac{\sqr{-\frac72Q}}{4^2 \cdot \abs{-3Q} \cdot \abs{-4Q}} = \frac{49}{768}.
    \end{align*}
}
\solutionspace{120pt}

\tasknumber{2}%
\task{%
    На координатной плоскости в точках $(-r; 0)$ и $(r; 0)$
    находятся заряды, соответственно, $+Q$ и $+Q$.
    Сделайте рисунок, определите величину напряжённости электрического поля
    и укажите её направление в двух точках: $(0; -r)$ и $(2r; 0)$.
}
\solutionspace{120pt}

\tasknumber{3}%
\task{%
    Напряжение между двумя точками, лежащими на одной линии напряжённости
    однородного электрического поля, равно $U = 6\,\text{кВ}$.
    Расстояние между точками $d = 20\,\text{см}$.
    Какова напряжённость этого поля?
}
\answer{%
    $
        E_x = -\frac{\Delta \varphi}{\Delta x} \implies
        E = \frac{U}{d} = \frac{6\,\text{кВ}}{20\,\text{см}} = 30{,}0\,\frac{\text{кВ}}{\text{м}}.
    $
}
\solutionspace{40pt}

\tasknumber{4}%
\task{%
    В однородном электрическом поле напряжённостью $E = 2\,\frac{\text{кВ}}{\text{м}}$
    переместили заряд $Q = 40\,\text{нКл}$ в направлении силовой линии
    на $l = 4\,\text{см}$.
    Определите
    \begin{itemize}
        \item работу поля,
        \item изменение потенциальной энергии заряда.
        % \item напряжение между начальной и конечной точками перемещения.
    \end{itemize}
}
\answer{%
    \begin{align*}
    A &= F \cdot l \cdot \cos \alpha = EQ \cdot l \cdot 1 = EQl = 2\,\frac{\text{кВ}}{\text{м}} \cdot 40\,\text{нКл} \cdot 4\,\text{см} = 3{,}2\,\text{мкДж}, \\
    \Delta E_\text{пот.} &= -A = -3{,}20\,\text{мкДж}
    \end{align*}
}
\solutionspace{80pt}

\tasknumber{5}%
\task{%
    \begin{enumerate}
        \item Запишите (формулой) закон сохранения электрического заряда.
        \item Зарисуйте электрическое поле точечного отрицательного электрического заряда.
        \item Запишите формулу для вычисления напряжённости электрического поля точечного заряда.
        \item Запишите принцип суперпозиции (правило сложения) потенциалов.
    \end{enumerate}
}

\variantsplitter

\addpersonalvariant{Варвара Егиазарян}

\tasknumber{1}%
\task{%
    Два одинаковых маленьких проводящих заряженных шарика находятся на расстоянии~$d$ друг от друга.
    Заряд первого равен~$-3q$, второго~--- $+4q$.
    Шарики приводят в соприкосновение, а после опять разводят на расстояние~$3d$.
    \begin{itemize}
        \item Каким стал заряд каждого из шариков?
        \item Определите характер (притяжение или отталкивание) и силу взаимодействия шариков до и после соприкосновения.
        \item Как изменилась сила взаимодействия шариков после соприкосновения?
    \end{itemize}
}
\answer{%
    \begin{align*}
    F &= k\frac{\abs{q_1}\abs {q_2}}{\sqr{3 d}}   = k\frac{\abs{-3q} \cdot \abs{+4q}}{3^2 \cdot d^2}, \text{притяжение}; \\
        q'_1 &= q'_2, q_1 + q_2 = q'_1 + q'_2 \implies  q'_1 = q'_2 = \frac{q_1 + q_2}2 = \frac{-3q + +4q}2 = \frac12q \implies \\
        \implies F'  &= k\frac{\abs{q'_1}\abs{q'_2}}{\sqr{3 d}}
            = k\frac{\sqr{\frac12q}}{3^2 \cdot d^2},
        \text{отталкивание}, \\
    \frac{F'}F &= \frac{\sqr{\frac12q}}{3^2 \cdot \abs{-3q} \cdot \abs{+4q}} = \frac1{432}.
    \end{align*}
}
\solutionspace{120pt}

\tasknumber{2}%
\task{%
    На координатной плоскости в точках $(-r; 0)$ и $(r; 0)$
    находятся заряды, соответственно, $+q$ и $-q$.
    Сделайте рисунок, определите величину напряжённости электрического поля
    и укажите её направление в двух точках: $(0; r)$ и $(-2r; 0)$.
}
\solutionspace{120pt}

\tasknumber{3}%
\task{%
    Напряжение между двумя точками, лежащими на одной линии напряжённости
    однородного электрического поля, равно $V = 3\,\text{кВ}$.
    Расстояние между точками $l = 40\,\text{см}$.
    Какова напряжённость этого поля?
}
\answer{%
    $
        E_x = -\frac{\Delta \varphi}{\Delta x} \implies
        E = \frac{V}{l} = \frac{3\,\text{кВ}}{40\,\text{см}} = 7{,}5\,\frac{\text{кВ}}{\text{м}}.
    $
}
\solutionspace{40pt}

\tasknumber{4}%
\task{%
    В однородном электрическом поле напряжённостью $E = 2\,\frac{\text{кВ}}{\text{м}}$
    переместили заряд $q = 40\,\text{нКл}$ в направлении силовой линии
    на $r = 10\,\text{см}$.
    Определите
    \begin{itemize}
        \item работу поля,
        \item изменение потенциальной энергии заряда.
        % \item напряжение между начальной и конечной точками перемещения.
    \end{itemize}
}
\answer{%
    \begin{align*}
    A &= F \cdot r \cdot \cos \alpha = Eq \cdot r \cdot 1 = Eqr = 2\,\frac{\text{кВ}}{\text{м}} \cdot 40\,\text{нКл} \cdot 10\,\text{см} = 8{,}0\,\text{мкДж}, \\
    \Delta E_\text{пот.} &= -A = -8{,}00\,\text{мкДж}
    \end{align*}
}
\solutionspace{80pt}

\tasknumber{5}%
\task{%
    \begin{enumerate}
        \item Запишите (формулой) закон Кулона (в вакууме).
        \item Зарисуйте электрическое поле точечного отрицательного электрического заряда.
        \item Запишите формулу для вычисления потенциала электрического поля точечного заряда.
        \item Запишите принцип суперпозиции (правило сложения) напряжённостей.
    \end{enumerate}
}

\variantsplitter

\addpersonalvariant{Владислав Емелин}

\tasknumber{1}%
\task{%
    Два одинаковых маленьких проводящих заряженных шарика находятся на расстоянии~$r$ друг от друга.
    Заряд первого равен~$+3Q$, второго~--- $+6Q$.
    Шарики приводят в соприкосновение, а после опять разводят на расстояние~$3r$.
    \begin{itemize}
        \item Каким стал заряд каждого из шариков?
        \item Определите характер (притяжение или отталкивание) и силу взаимодействия шариков до и после соприкосновения.
        \item Как изменилась сила взаимодействия шариков после соприкосновения?
    \end{itemize}
}
\answer{%
    \begin{align*}
    F &= k\frac{\abs{q_1}\abs {q_2}}{\sqr{3 r}}   = k\frac{\abs{+3Q} \cdot \abs{+6Q}}{3^2 \cdot r^2}, \text{отталкивание}; \\
        q'_1 &= q'_2, q_1 + q_2 = q'_1 + q'_2 \implies  q'_1 = q'_2 = \frac{q_1 + q_2}2 = \frac{+3Q + +6Q}2 = \frac92Q \implies \\
        \implies F'  &= k\frac{\abs{q'_1}\abs{q'_2}}{\sqr{3 r}}
            = k\frac{\sqr{\frac92Q}}{3^2 \cdot r^2},
        \text{отталкивание}, \\
    \frac{F'}F &= \frac{\sqr{\frac92Q}}{3^2 \cdot \abs{+3Q} \cdot \abs{+6Q}} = \frac18.
    \end{align*}
}
\solutionspace{120pt}

\tasknumber{2}%
\task{%
    На координатной плоскости в точках $(-r; 0)$ и $(r; 0)$
    находятся заряды, соответственно, $-q$ и $-q$.
    Сделайте рисунок, определите величину напряжённости электрического поля
    и укажите её направление в двух точках: $(0; r)$ и $(-2r; 0)$.
}
\solutionspace{120pt}

\tasknumber{3}%
\task{%
    Напряжение между двумя точками, лежащими на одной линии напряжённости
    однородного электрического поля, равно $U = 6\,\text{кВ}$.
    Расстояние между точками $l = 30\,\text{см}$.
    Какова напряжённость этого поля?
}
\answer{%
    $
        E_x = -\frac{\Delta \varphi}{\Delta x} \implies
        E = \frac{U}{l} = \frac{6\,\text{кВ}}{30\,\text{см}} = 20{,}0\,\frac{\text{кВ}}{\text{м}}.
    $
}
\solutionspace{40pt}

\tasknumber{4}%
\task{%
    В однородном электрическом поле напряжённостью $E = 4\,\frac{\text{кВ}}{\text{м}}$
    переместили заряд $Q = 25\,\text{нКл}$ в направлении силовой линии
    на $l = 2\,\text{см}$.
    Определите
    \begin{itemize}
        \item работу поля,
        \item изменение потенциальной энергии заряда.
        % \item напряжение между начальной и конечной точками перемещения.
    \end{itemize}
}
\answer{%
    \begin{align*}
    A &= F \cdot l \cdot \cos \alpha = EQ \cdot l \cdot 1 = EQl = 4\,\frac{\text{кВ}}{\text{м}} \cdot 25\,\text{нКл} \cdot 2\,\text{см} = 2{,}0\,\text{мкДж}, \\
    \Delta E_\text{пот.} &= -A = -2{,}00\,\text{мкДж}
    \end{align*}
}
\solutionspace{80pt}

\tasknumber{5}%
\task{%
    \begin{enumerate}
        \item Запишите (формулой) закон Кулона (в вакууме).
        \item Зарисуйте электрическое поле точечного положительного электрического заряда.
        \item Запишите формулу для вычисления потенциала электрического поля точечного заряда.
        \item Запишите принцип суперпозиции (правило сложения) напряжённостей.
    \end{enumerate}
}

\variantsplitter

\addpersonalvariant{Артём Жичин}

\tasknumber{1}%
\task{%
    Два одинаковых маленьких проводящих заряженных шарика находятся на расстоянии~$l$ друг от друга.
    Заряд первого равен~$+5q$, второго~--- $+4q$.
    Шарики приводят в соприкосновение, а после опять разводят на расстояние~$3l$.
    \begin{itemize}
        \item Каким стал заряд каждого из шариков?
        \item Определите характер (притяжение или отталкивание) и силу взаимодействия шариков до и после соприкосновения.
        \item Как изменилась сила взаимодействия шариков после соприкосновения?
    \end{itemize}
}
\answer{%
    \begin{align*}
    F &= k\frac{\abs{q_1}\abs {q_2}}{\sqr{3 l}}   = k\frac{\abs{+5q} \cdot \abs{+4q}}{3^2 \cdot l^2}, \text{отталкивание}; \\
        q'_1 &= q'_2, q_1 + q_2 = q'_1 + q'_2 \implies  q'_1 = q'_2 = \frac{q_1 + q_2}2 = \frac{+5q + +4q}2 = \frac92q \implies \\
        \implies F'  &= k\frac{\abs{q'_1}\abs{q'_2}}{\sqr{3 l}}
            = k\frac{\sqr{\frac92q}}{3^2 \cdot l^2},
        \text{отталкивание}, \\
    \frac{F'}F &= \frac{\sqr{\frac92q}}{3^2 \cdot \abs{+5q} \cdot \abs{+4q}} = \frac9{80}.
    \end{align*}
}
\solutionspace{120pt}

\tasknumber{2}%
\task{%
    На координатной плоскости в точках $(-a; 0)$ и $(a; 0)$
    находятся заряды, соответственно, $+Q$ и $+Q$.
    Сделайте рисунок, определите величину напряжённости электрического поля
    и укажите её направление в двух точках: $(0; -a)$ и $(2a; 0)$.
}
\solutionspace{120pt}

\tasknumber{3}%
\task{%
    Напряжение между двумя точками, лежащими на одной линии напряжённости
    однородного электрического поля, равно $V = 2\,\text{кВ}$.
    Расстояние между точками $r = 40\,\text{см}$.
    Какова напряжённость этого поля?
}
\answer{%
    $
        E_x = -\frac{\Delta \varphi}{\Delta x} \implies
        E = \frac{V}{r} = \frac{2\,\text{кВ}}{40\,\text{см}} = 5{,}0\,\frac{\text{кВ}}{\text{м}}.
    $
}
\solutionspace{40pt}

\tasknumber{4}%
\task{%
    В однородном электрическом поле напряжённостью $E = 2\,\frac{\text{кВ}}{\text{м}}$
    переместили заряд $q = -10\,\text{нКл}$ в направлении силовой линии
    на $r = 5\,\text{см}$.
    Определите
    \begin{itemize}
        \item работу поля,
        \item изменение потенциальной энергии заряда.
        % \item напряжение между начальной и конечной точками перемещения.
    \end{itemize}
}
\answer{%
    \begin{align*}
    A &= F \cdot r \cdot \cos \alpha = Eq \cdot r \cdot 1 = Eqr = 2\,\frac{\text{кВ}}{\text{м}} \cdot -10\,\text{нКл} \cdot 5\,\text{см} = -1{,}000\,\text{мкДж}, \\
    \Delta E_\text{пот.} &= -A = 1{,}0\,\text{мкДж}
    \end{align*}
}
\solutionspace{80pt}

\tasknumber{5}%
\task{%
    \begin{enumerate}
        \item Запишите (формулой) закон сохранения электрического заряда.
        \item Зарисуйте электрическое поле точечного положительного электрического заряда.
        \item Запишите формулу для вычисления напряжённости электрического поля точечного заряда.
        \item Запишите принцип суперпозиции (правило сложения) напряжённостей.
    \end{enumerate}
}

\variantsplitter

\addpersonalvariant{Дарья Кошман}

\tasknumber{1}%
\task{%
    Два одинаковых маленьких проводящих заряженных шарика находятся на расстоянии~$r$ друг от друга.
    Заряд первого равен~$-5Q$, второго~--- $+4Q$.
    Шарики приводят в соприкосновение, а после опять разводят на расстояние~$3r$.
    \begin{itemize}
        \item Каким стал заряд каждого из шариков?
        \item Определите характер (притяжение или отталкивание) и силу взаимодействия шариков до и после соприкосновения.
        \item Как изменилась сила взаимодействия шариков после соприкосновения?
    \end{itemize}
}
\answer{%
    \begin{align*}
    F &= k\frac{\abs{q_1}\abs {q_2}}{\sqr{3 r}}   = k\frac{\abs{-5Q} \cdot \abs{+4Q}}{3^2 \cdot r^2}, \text{притяжение}; \\
        q'_1 &= q'_2, q_1 + q_2 = q'_1 + q'_2 \implies  q'_1 = q'_2 = \frac{q_1 + q_2}2 = \frac{-5Q + +4Q}2 = -\frac12Q \implies \\
        \implies F'  &= k\frac{\abs{q'_1}\abs{q'_2}}{\sqr{3 r}}
            = k\frac{\sqr{-\frac12Q}}{3^2 \cdot r^2},
        \text{отталкивание}, \\
    \frac{F'}F &= \frac{\sqr{-\frac12Q}}{3^2 \cdot \abs{-5Q} \cdot \abs{+4Q}} = \frac1{720}.
    \end{align*}
}
\solutionspace{120pt}

\tasknumber{2}%
\task{%
    На координатной плоскости в точках $(-l; 0)$ и $(l; 0)$
    находятся заряды, соответственно, $-q$ и $-q$.
    Сделайте рисунок, определите величину напряжённости электрического поля
    и укажите её направление в двух точках: $(0; -l)$ и $(-2l; 0)$.
}
\solutionspace{120pt}

\tasknumber{3}%
\task{%
    Напряжение между двумя точками, лежащими на одной линии напряжённости
    однородного электрического поля, равно $U = 4\,\text{кВ}$.
    Расстояние между точками $l = 40\,\text{см}$.
    Какова напряжённость этого поля?
}
\answer{%
    $
        E_x = -\frac{\Delta \varphi}{\Delta x} \implies
        E = \frac{U}{l} = \frac{4\,\text{кВ}}{40\,\text{см}} = 10{,}0\,\frac{\text{кВ}}{\text{м}}.
    $
}
\solutionspace{40pt}

\tasknumber{4}%
\task{%
    В однородном электрическом поле напряжённостью $E = 4\,\frac{\text{кВ}}{\text{м}}$
    переместили заряд $Q = -40\,\text{нКл}$ в направлении силовой линии
    на $d = 10\,\text{см}$.
    Определите
    \begin{itemize}
        \item работу поля,
        \item изменение потенциальной энергии заряда.
        % \item напряжение между начальной и конечной точками перемещения.
    \end{itemize}
}
\answer{%
    \begin{align*}
    A &= F \cdot d \cdot \cos \alpha = EQ \cdot d \cdot 1 = EQd = 4\,\frac{\text{кВ}}{\text{м}} \cdot -40\,\text{нКл} \cdot 10\,\text{см} = -16{,}000\,\text{мкДж}, \\
    \Delta E_\text{пот.} &= -A = 16{,}0\,\text{мкДж}
    \end{align*}
}
\solutionspace{80pt}

\tasknumber{5}%
\task{%
    \begin{enumerate}
        \item Запишите (формулой) закон сохранения электрического заряда.
        \item Зарисуйте электрическое поле точечного положительного электрического заряда.
        \item Запишите формулу для вычисления потенциала электрического поля точечного заряда.
        \item Запишите принцип суперпозиции (правило сложения) потенциалов.
    \end{enumerate}
}

\variantsplitter

\addpersonalvariant{Анна Кузьмичёва}

\tasknumber{1}%
\task{%
    Два одинаковых маленьких проводящих заряженных шарика находятся на расстоянии~$r$ друг от друга.
    Заряд первого равен~$+5q$, второго~--- $-6q$.
    Шарики приводят в соприкосновение, а после опять разводят на расстояние~$3r$.
    \begin{itemize}
        \item Каким стал заряд каждого из шариков?
        \item Определите характер (притяжение или отталкивание) и силу взаимодействия шариков до и после соприкосновения.
        \item Как изменилась сила взаимодействия шариков после соприкосновения?
    \end{itemize}
}
\answer{%
    \begin{align*}
    F &= k\frac{\abs{q_1}\abs {q_2}}{\sqr{3 r}}   = k\frac{\abs{+5q} \cdot \abs{-6q}}{3^2 \cdot r^2}, \text{притяжение}; \\
        q'_1 &= q'_2, q_1 + q_2 = q'_1 + q'_2 \implies  q'_1 = q'_2 = \frac{q_1 + q_2}2 = \frac{+5q -6q}2 = -\frac12q \implies \\
        \implies F'  &= k\frac{\abs{q'_1}\abs{q'_2}}{\sqr{3 r}}
            = k\frac{\sqr{-\frac12q}}{3^2 \cdot r^2},
        \text{отталкивание}, \\
    \frac{F'}F &= \frac{\sqr{-\frac12q}}{3^2 \cdot \abs{+5q} \cdot \abs{-6q}} = \frac1{1080}.
    \end{align*}
}
\solutionspace{120pt}

\tasknumber{2}%
\task{%
    На координатной плоскости в точках $(-l; 0)$ и $(l; 0)$
    находятся заряды, соответственно, $-Q$ и $+Q$.
    Сделайте рисунок, определите величину напряжённости электрического поля
    и укажите её направление в двух точках: $(0; -l)$ и $(2l; 0)$.
}
\solutionspace{120pt}

\tasknumber{3}%
\task{%
    Напряжение между двумя точками, лежащими на одной линии напряжённости
    однородного электрического поля, равно $V = 4\,\text{кВ}$.
    Расстояние между точками $d = 10\,\text{см}$.
    Какова напряжённость этого поля?
}
\answer{%
    $
        E_x = -\frac{\Delta \varphi}{\Delta x} \implies
        E = \frac{V}{d} = \frac{4\,\text{кВ}}{10\,\text{см}} = 40{,}0\,\frac{\text{кВ}}{\text{м}}.
    $
}
\solutionspace{40pt}

\tasknumber{4}%
\task{%
    В однородном электрическом поле напряжённостью $E = 4\,\frac{\text{кВ}}{\text{м}}$
    переместили заряд $Q = 10\,\text{нКл}$ в направлении силовой линии
    на $d = 4\,\text{см}$.
    Определите
    \begin{itemize}
        \item работу поля,
        \item изменение потенциальной энергии заряда.
        % \item напряжение между начальной и конечной точками перемещения.
    \end{itemize}
}
\answer{%
    \begin{align*}
    A &= F \cdot d \cdot \cos \alpha = EQ \cdot d \cdot 1 = EQd = 4\,\frac{\text{кВ}}{\text{м}} \cdot 10\,\text{нКл} \cdot 4\,\text{см} = 1{,}6\,\text{мкДж}, \\
    \Delta E_\text{пот.} &= -A = -1{,}600\,\text{мкДж}
    \end{align*}
}
\solutionspace{80pt}

\tasknumber{5}%
\task{%
    \begin{enumerate}
        \item Запишите (формулой) закон Кулона (в вакууме).
        \item Зарисуйте электрическое поле точечного положительного электрического заряда.
        \item Запишите формулу для вычисления потенциала электрического поля точечного заряда.
        \item Запишите принцип суперпозиции (правило сложения) напряжённостей.
    \end{enumerate}
}

\variantsplitter

\addpersonalvariant{Алёна Куприянова}

\tasknumber{1}%
\task{%
    Два одинаковых маленьких проводящих заряженных шарика находятся на расстоянии~$l$ друг от друга.
    Заряд первого равен~$+5q$, второго~--- $-6q$.
    Шарики приводят в соприкосновение, а после опять разводят на расстояние~$3l$.
    \begin{itemize}
        \item Каким стал заряд каждого из шариков?
        \item Определите характер (притяжение или отталкивание) и силу взаимодействия шариков до и после соприкосновения.
        \item Как изменилась сила взаимодействия шариков после соприкосновения?
    \end{itemize}
}
\answer{%
    \begin{align*}
    F &= k\frac{\abs{q_1}\abs {q_2}}{\sqr{3 l}}   = k\frac{\abs{+5q} \cdot \abs{-6q}}{3^2 \cdot l^2}, \text{притяжение}; \\
        q'_1 &= q'_2, q_1 + q_2 = q'_1 + q'_2 \implies  q'_1 = q'_2 = \frac{q_1 + q_2}2 = \frac{+5q -6q}2 = -\frac12q \implies \\
        \implies F'  &= k\frac{\abs{q'_1}\abs{q'_2}}{\sqr{3 l}}
            = k\frac{\sqr{-\frac12q}}{3^2 \cdot l^2},
        \text{отталкивание}, \\
    \frac{F'}F &= \frac{\sqr{-\frac12q}}{3^2 \cdot \abs{+5q} \cdot \abs{-6q}} = \frac1{1080}.
    \end{align*}
}
\solutionspace{120pt}

\tasknumber{2}%
\task{%
    На координатной плоскости в точках $(-a; 0)$ и $(a; 0)$
    находятся заряды, соответственно, $-Q$ и $+Q$.
    Сделайте рисунок, определите величину напряжённости электрического поля
    и укажите её направление в двух точках: $(0; -a)$ и $(-2a; 0)$.
}
\solutionspace{120pt}

\tasknumber{3}%
\task{%
    Напряжение между двумя точками, лежащими на одной линии напряжённости
    однородного электрического поля, равно $V = 2\,\text{кВ}$.
    Расстояние между точками $r = 40\,\text{см}$.
    Какова напряжённость этого поля?
}
\answer{%
    $
        E_x = -\frac{\Delta \varphi}{\Delta x} \implies
        E = \frac{V}{r} = \frac{2\,\text{кВ}}{40\,\text{см}} = 5{,}0\,\frac{\text{кВ}}{\text{м}}.
    $
}
\solutionspace{40pt}

\tasknumber{4}%
\task{%
    В однородном электрическом поле напряжённостью $E = 2\,\frac{\text{кВ}}{\text{м}}$
    переместили заряд $Q = -25\,\text{нКл}$ в направлении силовой линии
    на $d = 10\,\text{см}$.
    Определите
    \begin{itemize}
        \item работу поля,
        \item изменение потенциальной энергии заряда.
        % \item напряжение между начальной и конечной точками перемещения.
    \end{itemize}
}
\answer{%
    \begin{align*}
    A &= F \cdot d \cdot \cos \alpha = EQ \cdot d \cdot 1 = EQd = 2\,\frac{\text{кВ}}{\text{м}} \cdot -25\,\text{нКл} \cdot 10\,\text{см} = -5{,}00\,\text{мкДж}, \\
    \Delta E_\text{пот.} &= -A = 5{,}0\,\text{мкДж}
    \end{align*}
}
\solutionspace{80pt}

\tasknumber{5}%
\task{%
    \begin{enumerate}
        \item Запишите (формулой) закон сохранения электрического заряда.
        \item Зарисуйте электрическое поле точечного положительного электрического заряда.
        \item Запишите формулу для вычисления потенциала электрического поля точечного заряда.
        \item Запишите принцип суперпозиции (правило сложения) потенциалов.
    \end{enumerate}
}

\variantsplitter

\addpersonalvariant{Ярослав Лавровский}

\tasknumber{1}%
\task{%
    Два одинаковых маленьких проводящих заряженных шарика находятся на расстоянии~$l$ друг от друга.
    Заряд первого равен~$-5q$, второго~--- $-2q$.
    Шарики приводят в соприкосновение, а после опять разводят на расстояние~$4l$.
    \begin{itemize}
        \item Каким стал заряд каждого из шариков?
        \item Определите характер (притяжение или отталкивание) и силу взаимодействия шариков до и после соприкосновения.
        \item Как изменилась сила взаимодействия шариков после соприкосновения?
    \end{itemize}
}
\answer{%
    \begin{align*}
    F &= k\frac{\abs{q_1}\abs {q_2}}{\sqr{4 l}}   = k\frac{\abs{-5q} \cdot \abs{-2q}}{4^2 \cdot l^2}, \text{отталкивание}; \\
        q'_1 &= q'_2, q_1 + q_2 = q'_1 + q'_2 \implies  q'_1 = q'_2 = \frac{q_1 + q_2}2 = \frac{-5q -2q}2 = -\frac72q \implies \\
        \implies F'  &= k\frac{\abs{q'_1}\abs{q'_2}}{\sqr{4 l}}
            = k\frac{\sqr{-\frac72q}}{4^2 \cdot l^2},
        \text{отталкивание}, \\
    \frac{F'}F &= \frac{\sqr{-\frac72q}}{4^2 \cdot \abs{-5q} \cdot \abs{-2q}} = \frac{49}{640}.
    \end{align*}
}
\solutionspace{120pt}

\tasknumber{2}%
\task{%
    На координатной плоскости в точках $(-r; 0)$ и $(r; 0)$
    находятся заряды, соответственно, $-Q$ и $+Q$.
    Сделайте рисунок, определите величину напряжённости электрического поля
    и укажите её направление в двух точках: $(0; r)$ и $(-2r; 0)$.
}
\solutionspace{120pt}

\tasknumber{3}%
\task{%
    Напряжение между двумя точками, лежащими на одной линии напряжённости
    однородного электрического поля, равно $V = 4\,\text{кВ}$.
    Расстояние между точками $l = 20\,\text{см}$.
    Какова напряжённость этого поля?
}
\answer{%
    $
        E_x = -\frac{\Delta \varphi}{\Delta x} \implies
        E = \frac{V}{l} = \frac{4\,\text{кВ}}{20\,\text{см}} = 20{,}0\,\frac{\text{кВ}}{\text{м}}.
    $
}
\solutionspace{40pt}

\tasknumber{4}%
\task{%
    В однородном электрическом поле напряжённостью $E = 20\,\frac{\text{кВ}}{\text{м}}$
    переместили заряд $Q = 40\,\text{нКл}$ в направлении силовой линии
    на $l = 5\,\text{см}$.
    Определите
    \begin{itemize}
        \item работу поля,
        \item изменение потенциальной энергии заряда.
        % \item напряжение между начальной и конечной точками перемещения.
    \end{itemize}
}
\answer{%
    \begin{align*}
    A &= F \cdot l \cdot \cos \alpha = EQ \cdot l \cdot 1 = EQl = 20\,\frac{\text{кВ}}{\text{м}} \cdot 40\,\text{нКл} \cdot 5\,\text{см} = 40{,}0\,\text{мкДж}, \\
    \Delta E_\text{пот.} &= -A = -40{,}00\,\text{мкДж}
    \end{align*}
}
\solutionspace{80pt}

\tasknumber{5}%
\task{%
    \begin{enumerate}
        \item Запишите (формулой) закон Кулона (в вакууме).
        \item Зарисуйте электрическое поле точечного отрицательного электрического заряда.
        \item Запишите формулу для вычисления напряжённости электрического поля точечного заряда.
        \item Запишите принцип суперпозиции (правило сложения) напряжённостей.
    \end{enumerate}
}

\variantsplitter

\addpersonalvariant{Анастасия Ламанова}

\tasknumber{1}%
\task{%
    Два одинаковых маленьких проводящих заряженных шарика находятся на расстоянии~$l$ друг от друга.
    Заряд первого равен~$+7Q$, второго~--- $-6Q$.
    Шарики приводят в соприкосновение, а после опять разводят на расстояние~$4l$.
    \begin{itemize}
        \item Каким стал заряд каждого из шариков?
        \item Определите характер (притяжение или отталкивание) и силу взаимодействия шариков до и после соприкосновения.
        \item Как изменилась сила взаимодействия шариков после соприкосновения?
    \end{itemize}
}
\answer{%
    \begin{align*}
    F &= k\frac{\abs{q_1}\abs {q_2}}{\sqr{4 l}}   = k\frac{\abs{+7Q} \cdot \abs{-6Q}}{4^2 \cdot l^2}, \text{притяжение}; \\
        q'_1 &= q'_2, q_1 + q_2 = q'_1 + q'_2 \implies  q'_1 = q'_2 = \frac{q_1 + q_2}2 = \frac{+7Q -6Q}2 = \frac12Q \implies \\
        \implies F'  &= k\frac{\abs{q'_1}\abs{q'_2}}{\sqr{4 l}}
            = k\frac{\sqr{\frac12Q}}{4^2 \cdot l^2},
        \text{отталкивание}, \\
    \frac{F'}F &= \frac{\sqr{\frac12Q}}{4^2 \cdot \abs{+7Q} \cdot \abs{-6Q}} = \frac1{2688}.
    \end{align*}
}
\solutionspace{120pt}

\tasknumber{2}%
\task{%
    На координатной плоскости в точках $(-l; 0)$ и $(l; 0)$
    находятся заряды, соответственно, $-Q$ и $+Q$.
    Сделайте рисунок, определите величину напряжённости электрического поля
    и укажите её направление в двух точках: $(0; -l)$ и $(-2l; 0)$.
}
\solutionspace{120pt}

\tasknumber{3}%
\task{%
    Напряжение между двумя точками, лежащими на одной линии напряжённости
    однородного электрического поля, равно $U = 4\,\text{кВ}$.
    Расстояние между точками $l = 20\,\text{см}$.
    Какова напряжённость этого поля?
}
\answer{%
    $
        E_x = -\frac{\Delta \varphi}{\Delta x} \implies
        E = \frac{U}{l} = \frac{4\,\text{кВ}}{20\,\text{см}} = 20{,}0\,\frac{\text{кВ}}{\text{м}}.
    $
}
\solutionspace{40pt}

\tasknumber{4}%
\task{%
    В однородном электрическом поле напряжённостью $E = 4\,\frac{\text{кВ}}{\text{м}}$
    переместили заряд $Q = -40\,\text{нКл}$ в направлении силовой линии
    на $d = 5\,\text{см}$.
    Определите
    \begin{itemize}
        \item работу поля,
        \item изменение потенциальной энергии заряда.
        % \item напряжение между начальной и конечной точками перемещения.
    \end{itemize}
}
\answer{%
    \begin{align*}
    A &= F \cdot d \cdot \cos \alpha = EQ \cdot d \cdot 1 = EQd = 4\,\frac{\text{кВ}}{\text{м}} \cdot -40\,\text{нКл} \cdot 5\,\text{см} = -8{,}00\,\text{мкДж}, \\
    \Delta E_\text{пот.} &= -A = 8{,}0\,\text{мкДж}
    \end{align*}
}
\solutionspace{80pt}

\tasknumber{5}%
\task{%
    \begin{enumerate}
        \item Запишите (формулой) закон Кулона (в вакууме).
        \item Зарисуйте электрическое поле точечного положительного электрического заряда.
        \item Запишите формулу для вычисления напряжённости электрического поля точечного заряда.
        \item Запишите принцип суперпозиции (правило сложения) потенциалов.
    \end{enumerate}
}

\variantsplitter

\addpersonalvariant{Виктория Легонькова}

\tasknumber{1}%
\task{%
    Два одинаковых маленьких проводящих заряженных шарика находятся на расстоянии~$d$ друг от друга.
    Заряд первого равен~$+5q$, второго~--- $+4q$.
    Шарики приводят в соприкосновение, а после опять разводят на расстояние~$2d$.
    \begin{itemize}
        \item Каким стал заряд каждого из шариков?
        \item Определите характер (притяжение или отталкивание) и силу взаимодействия шариков до и после соприкосновения.
        \item Как изменилась сила взаимодействия шариков после соприкосновения?
    \end{itemize}
}
\answer{%
    \begin{align*}
    F &= k\frac{\abs{q_1}\abs {q_2}}{\sqr{2 d}}   = k\frac{\abs{+5q} \cdot \abs{+4q}}{2^2 \cdot d^2}, \text{отталкивание}; \\
        q'_1 &= q'_2, q_1 + q_2 = q'_1 + q'_2 \implies  q'_1 = q'_2 = \frac{q_1 + q_2}2 = \frac{+5q + +4q}2 = \frac92q \implies \\
        \implies F'  &= k\frac{\abs{q'_1}\abs{q'_2}}{\sqr{2 d}}
            = k\frac{\sqr{\frac92q}}{2^2 \cdot d^2},
        \text{отталкивание}, \\
    \frac{F'}F &= \frac{\sqr{\frac92q}}{2^2 \cdot \abs{+5q} \cdot \abs{+4q}} = \frac{81}{320}.
    \end{align*}
}
\solutionspace{120pt}

\tasknumber{2}%
\task{%
    На координатной плоскости в точках $(-a; 0)$ и $(a; 0)$
    находятся заряды, соответственно, $+q$ и $-q$.
    Сделайте рисунок, определите величину напряжённости электрического поля
    и укажите её направление в двух точках: $(0; -a)$ и $(2a; 0)$.
}
\solutionspace{120pt}

\tasknumber{3}%
\task{%
    Напряжение между двумя точками, лежащими на одной линии напряжённости
    однородного электрического поля, равно $V = 6\,\text{кВ}$.
    Расстояние между точками $d = 20\,\text{см}$.
    Какова напряжённость этого поля?
}
\answer{%
    $
        E_x = -\frac{\Delta \varphi}{\Delta x} \implies
        E = \frac{V}{d} = \frac{6\,\text{кВ}}{20\,\text{см}} = 30{,}0\,\frac{\text{кВ}}{\text{м}}.
    $
}
\solutionspace{40pt}

\tasknumber{4}%
\task{%
    В однородном электрическом поле напряжённостью $E = 20\,\frac{\text{кВ}}{\text{м}}$
    переместили заряд $Q = 10\,\text{нКл}$ в направлении силовой линии
    на $r = 4\,\text{см}$.
    Определите
    \begin{itemize}
        \item работу поля,
        \item изменение потенциальной энергии заряда.
        % \item напряжение между начальной и конечной точками перемещения.
    \end{itemize}
}
\answer{%
    \begin{align*}
    A &= F \cdot r \cdot \cos \alpha = EQ \cdot r \cdot 1 = EQr = 20\,\frac{\text{кВ}}{\text{м}} \cdot 10\,\text{нКл} \cdot 4\,\text{см} = 8{,}0\,\text{мкДж}, \\
    \Delta E_\text{пот.} &= -A = -8{,}00\,\text{мкДж}
    \end{align*}
}
\solutionspace{80pt}

\tasknumber{5}%
\task{%
    \begin{enumerate}
        \item Запишите (формулой) закон Кулона (в вакууме).
        \item Зарисуйте электрическое поле точечного положительного электрического заряда.
        \item Запишите формулу для вычисления напряжённости электрического поля точечного заряда.
        \item Запишите принцип суперпозиции (правило сложения) напряжённостей.
    \end{enumerate}
}

\variantsplitter

\addpersonalvariant{Семён Мартынов}

\tasknumber{1}%
\task{%
    Два одинаковых маленьких проводящих заряженных шарика находятся на расстоянии~$r$ друг от друга.
    Заряд первого равен~$+5Q$, второго~--- $+8Q$.
    Шарики приводят в соприкосновение, а после опять разводят на расстояние~$4r$.
    \begin{itemize}
        \item Каким стал заряд каждого из шариков?
        \item Определите характер (притяжение или отталкивание) и силу взаимодействия шариков до и после соприкосновения.
        \item Как изменилась сила взаимодействия шариков после соприкосновения?
    \end{itemize}
}
\answer{%
    \begin{align*}
    F &= k\frac{\abs{q_1}\abs {q_2}}{\sqr{4 r}}   = k\frac{\abs{+5Q} \cdot \abs{+8Q}}{4^2 \cdot r^2}, \text{отталкивание}; \\
        q'_1 &= q'_2, q_1 + q_2 = q'_1 + q'_2 \implies  q'_1 = q'_2 = \frac{q_1 + q_2}2 = \frac{+5Q + +8Q}2 = \frac{13}2Q \implies \\
        \implies F'  &= k\frac{\abs{q'_1}\abs{q'_2}}{\sqr{4 r}}
            = k\frac{\sqr{\frac{13}2Q}}{4^2 \cdot r^2},
        \text{отталкивание}, \\
    \frac{F'}F &= \frac{\sqr{\frac{13}2Q}}{4^2 \cdot \abs{+5Q} \cdot \abs{+8Q}} = \frac{169}{2560}.
    \end{align*}
}
\solutionspace{120pt}

\tasknumber{2}%
\task{%
    На координатной плоскости в точках $(-d; 0)$ и $(d; 0)$
    находятся заряды, соответственно, $-Q$ и $+Q$.
    Сделайте рисунок, определите величину напряжённости электрического поля
    и укажите её направление в двух точках: $(0; -d)$ и $(2d; 0)$.
}
\solutionspace{120pt}

\tasknumber{3}%
\task{%
    Напряжение между двумя точками, лежащими на одной линии напряжённости
    однородного электрического поля, равно $U = 5\,\text{кВ}$.
    Расстояние между точками $l = 10\,\text{см}$.
    Какова напряжённость этого поля?
}
\answer{%
    $
        E_x = -\frac{\Delta \varphi}{\Delta x} \implies
        E = \frac{U}{l} = \frac{5\,\text{кВ}}{10\,\text{см}} = 50{,}0\,\frac{\text{кВ}}{\text{м}}.
    $
}
\solutionspace{40pt}

\tasknumber{4}%
\task{%
    В однородном электрическом поле напряжённостью $E = 4\,\frac{\text{кВ}}{\text{м}}$
    переместили заряд $q = -40\,\text{нКл}$ в направлении силовой линии
    на $l = 10\,\text{см}$.
    Определите
    \begin{itemize}
        \item работу поля,
        \item изменение потенциальной энергии заряда.
        % \item напряжение между начальной и конечной точками перемещения.
    \end{itemize}
}
\answer{%
    \begin{align*}
    A &= F \cdot l \cdot \cos \alpha = Eq \cdot l \cdot 1 = Eql = 4\,\frac{\text{кВ}}{\text{м}} \cdot -40\,\text{нКл} \cdot 10\,\text{см} = -16{,}000\,\text{мкДж}, \\
    \Delta E_\text{пот.} &= -A = 16{,}0\,\text{мкДж}
    \end{align*}
}
\solutionspace{80pt}

\tasknumber{5}%
\task{%
    \begin{enumerate}
        \item Запишите (формулой) закон Кулона (в вакууме).
        \item Зарисуйте электрическое поле точечного положительного электрического заряда.
        \item Запишите формулу для вычисления напряжённости электрического поля точечного заряда.
        \item Запишите принцип суперпозиции (правило сложения) напряжённостей.
    \end{enumerate}
}

\variantsplitter

\addpersonalvariant{Варвара Минаева}

\tasknumber{1}%
\task{%
    Два одинаковых маленьких проводящих заряженных шарика находятся на расстоянии~$d$ друг от друга.
    Заряд первого равен~$+5Q$, второго~--- $+6Q$.
    Шарики приводят в соприкосновение, а после опять разводят на расстояние~$3d$.
    \begin{itemize}
        \item Каким стал заряд каждого из шариков?
        \item Определите характер (притяжение или отталкивание) и силу взаимодействия шариков до и после соприкосновения.
        \item Как изменилась сила взаимодействия шариков после соприкосновения?
    \end{itemize}
}
\answer{%
    \begin{align*}
    F &= k\frac{\abs{q_1}\abs {q_2}}{\sqr{3 d}}   = k\frac{\abs{+5Q} \cdot \abs{+6Q}}{3^2 \cdot d^2}, \text{отталкивание}; \\
        q'_1 &= q'_2, q_1 + q_2 = q'_1 + q'_2 \implies  q'_1 = q'_2 = \frac{q_1 + q_2}2 = \frac{+5Q + +6Q}2 = \frac{11}2Q \implies \\
        \implies F'  &= k\frac{\abs{q'_1}\abs{q'_2}}{\sqr{3 d}}
            = k\frac{\sqr{\frac{11}2Q}}{3^2 \cdot d^2},
        \text{отталкивание}, \\
    \frac{F'}F &= \frac{\sqr{\frac{11}2Q}}{3^2 \cdot \abs{+5Q} \cdot \abs{+6Q}} = \frac{121}{1080}.
    \end{align*}
}
\solutionspace{120pt}

\tasknumber{2}%
\task{%
    На координатной плоскости в точках $(-r; 0)$ и $(r; 0)$
    находятся заряды, соответственно, $+q$ и $-q$.
    Сделайте рисунок, определите величину напряжённости электрического поля
    и укажите её направление в двух точках: $(0; -r)$ и $(-2r; 0)$.
}
\solutionspace{120pt}

\tasknumber{3}%
\task{%
    Напряжение между двумя точками, лежащими на одной линии напряжённости
    однородного электрического поля, равно $U = 2\,\text{кВ}$.
    Расстояние между точками $d = 30\,\text{см}$.
    Какова напряжённость этого поля?
}
\answer{%
    $
        E_x = -\frac{\Delta \varphi}{\Delta x} \implies
        E = \frac{U}{d} = \frac{2\,\text{кВ}}{30\,\text{см}} = 6{,}7\,\frac{\text{кВ}}{\text{м}}.
    $
}
\solutionspace{40pt}

\tasknumber{4}%
\task{%
    В однородном электрическом поле напряжённостью $E = 2\,\frac{\text{кВ}}{\text{м}}$
    переместили заряд $q = 25\,\text{нКл}$ в направлении силовой линии
    на $d = 10\,\text{см}$.
    Определите
    \begin{itemize}
        \item работу поля,
        \item изменение потенциальной энергии заряда.
        % \item напряжение между начальной и конечной точками перемещения.
    \end{itemize}
}
\answer{%
    \begin{align*}
    A &= F \cdot d \cdot \cos \alpha = Eq \cdot d \cdot 1 = Eqd = 2\,\frac{\text{кВ}}{\text{м}} \cdot 25\,\text{нКл} \cdot 10\,\text{см} = 5{,}0\,\text{мкДж}, \\
    \Delta E_\text{пот.} &= -A = -5{,}00\,\text{мкДж}
    \end{align*}
}
\solutionspace{80pt}

\tasknumber{5}%
\task{%
    \begin{enumerate}
        \item Запишите (формулой) закон сохранения электрического заряда.
        \item Зарисуйте электрическое поле точечного отрицательного электрического заряда.
        \item Запишите формулу для вычисления потенциала электрического поля точечного заряда.
        \item Запишите принцип суперпозиции (правило сложения) потенциалов.
    \end{enumerate}
}

\variantsplitter

\addpersonalvariant{Леонид Никитин}

\tasknumber{1}%
\task{%
    Два одинаковых маленьких проводящих заряженных шарика находятся на расстоянии~$l$ друг от друга.
    Заряд первого равен~$-3q$, второго~--- $+6q$.
    Шарики приводят в соприкосновение, а после опять разводят на расстояние~$4l$.
    \begin{itemize}
        \item Каким стал заряд каждого из шариков?
        \item Определите характер (притяжение или отталкивание) и силу взаимодействия шариков до и после соприкосновения.
        \item Как изменилась сила взаимодействия шариков после соприкосновения?
    \end{itemize}
}
\answer{%
    \begin{align*}
    F &= k\frac{\abs{q_1}\abs {q_2}}{\sqr{4 l}}   = k\frac{\abs{-3q} \cdot \abs{+6q}}{4^2 \cdot l^2}, \text{притяжение}; \\
        q'_1 &= q'_2, q_1 + q_2 = q'_1 + q'_2 \implies  q'_1 = q'_2 = \frac{q_1 + q_2}2 = \frac{-3q + +6q}2 = \frac32q \implies \\
        \implies F'  &= k\frac{\abs{q'_1}\abs{q'_2}}{\sqr{4 l}}
            = k\frac{\sqr{\frac32q}}{4^2 \cdot l^2},
        \text{отталкивание}, \\
    \frac{F'}F &= \frac{\sqr{\frac32q}}{4^2 \cdot \abs{-3q} \cdot \abs{+6q}} = \frac1{128}.
    \end{align*}
}
\solutionspace{120pt}

\tasknumber{2}%
\task{%
    На координатной плоскости в точках $(-d; 0)$ и $(d; 0)$
    находятся заряды, соответственно, $-q$ и $-q$.
    Сделайте рисунок, определите величину напряжённости электрического поля
    и укажите её направление в двух точках: $(0; d)$ и $(-2d; 0)$.
}
\solutionspace{120pt}

\tasknumber{3}%
\task{%
    Напряжение между двумя точками, лежащими на одной линии напряжённости
    однородного электрического поля, равно $U = 2\,\text{кВ}$.
    Расстояние между точками $d = 40\,\text{см}$.
    Какова напряжённость этого поля?
}
\answer{%
    $
        E_x = -\frac{\Delta \varphi}{\Delta x} \implies
        E = \frac{U}{d} = \frac{2\,\text{кВ}}{40\,\text{см}} = 5{,}0\,\frac{\text{кВ}}{\text{м}}.
    $
}
\solutionspace{40pt}

\tasknumber{4}%
\task{%
    В однородном электрическом поле напряжённостью $E = 4\,\frac{\text{кВ}}{\text{м}}$
    переместили заряд $q = 40\,\text{нКл}$ в направлении силовой линии
    на $l = 5\,\text{см}$.
    Определите
    \begin{itemize}
        \item работу поля,
        \item изменение потенциальной энергии заряда.
        % \item напряжение между начальной и конечной точками перемещения.
    \end{itemize}
}
\answer{%
    \begin{align*}
    A &= F \cdot l \cdot \cos \alpha = Eq \cdot l \cdot 1 = Eql = 4\,\frac{\text{кВ}}{\text{м}} \cdot 40\,\text{нКл} \cdot 5\,\text{см} = 8{,}0\,\text{мкДж}, \\
    \Delta E_\text{пот.} &= -A = -8{,}00\,\text{мкДж}
    \end{align*}
}
\solutionspace{80pt}

\tasknumber{5}%
\task{%
    \begin{enumerate}
        \item Запишите (формулой) закон сохранения электрического заряда.
        \item Зарисуйте электрическое поле точечного отрицательного электрического заряда.
        \item Запишите формулу для вычисления потенциала электрического поля точечного заряда.
        \item Запишите принцип суперпозиции (правило сложения) потенциалов.
    \end{enumerate}
}

\variantsplitter

\addpersonalvariant{Тимофей Полетаев}

\tasknumber{1}%
\task{%
    Два одинаковых маленьких проводящих заряженных шарика находятся на расстоянии~$d$ друг от друга.
    Заряд первого равен~$-3Q$, второго~--- $+6Q$.
    Шарики приводят в соприкосновение, а после опять разводят на расстояние~$4d$.
    \begin{itemize}
        \item Каким стал заряд каждого из шариков?
        \item Определите характер (притяжение или отталкивание) и силу взаимодействия шариков до и после соприкосновения.
        \item Как изменилась сила взаимодействия шариков после соприкосновения?
    \end{itemize}
}
\answer{%
    \begin{align*}
    F &= k\frac{\abs{q_1}\abs {q_2}}{\sqr{4 d}}   = k\frac{\abs{-3Q} \cdot \abs{+6Q}}{4^2 \cdot d^2}, \text{притяжение}; \\
        q'_1 &= q'_2, q_1 + q_2 = q'_1 + q'_2 \implies  q'_1 = q'_2 = \frac{q_1 + q_2}2 = \frac{-3Q + +6Q}2 = \frac32Q \implies \\
        \implies F'  &= k\frac{\abs{q'_1}\abs{q'_2}}{\sqr{4 d}}
            = k\frac{\sqr{\frac32Q}}{4^2 \cdot d^2},
        \text{отталкивание}, \\
    \frac{F'}F &= \frac{\sqr{\frac32Q}}{4^2 \cdot \abs{-3Q} \cdot \abs{+6Q}} = \frac1{128}.
    \end{align*}
}
\solutionspace{120pt}

\tasknumber{2}%
\task{%
    На координатной плоскости в точках $(-a; 0)$ и $(a; 0)$
    находятся заряды, соответственно, $-q$ и $-q$.
    Сделайте рисунок, определите величину напряжённости электрического поля
    и укажите её направление в двух точках: $(0; -a)$ и $(-2a; 0)$.
}
\solutionspace{120pt}

\tasknumber{3}%
\task{%
    Напряжение между двумя точками, лежащими на одной линии напряжённости
    однородного электрического поля, равно $V = 6\,\text{кВ}$.
    Расстояние между точками $l = 20\,\text{см}$.
    Какова напряжённость этого поля?
}
\answer{%
    $
        E_x = -\frac{\Delta \varphi}{\Delta x} \implies
        E = \frac{V}{l} = \frac{6\,\text{кВ}}{20\,\text{см}} = 30{,}0\,\frac{\text{кВ}}{\text{м}}.
    $
}
\solutionspace{40pt}

\tasknumber{4}%
\task{%
    В однородном электрическом поле напряжённостью $E = 2\,\frac{\text{кВ}}{\text{м}}$
    переместили заряд $Q = 25\,\text{нКл}$ в направлении силовой линии
    на $l = 2\,\text{см}$.
    Определите
    \begin{itemize}
        \item работу поля,
        \item изменение потенциальной энергии заряда.
        % \item напряжение между начальной и конечной точками перемещения.
    \end{itemize}
}
\answer{%
    \begin{align*}
    A &= F \cdot l \cdot \cos \alpha = EQ \cdot l \cdot 1 = EQl = 2\,\frac{\text{кВ}}{\text{м}} \cdot 25\,\text{нКл} \cdot 2\,\text{см} = 1{,}0\,\text{мкДж}, \\
    \Delta E_\text{пот.} &= -A = -1{,}000\,\text{мкДж}
    \end{align*}
}
\solutionspace{80pt}

\tasknumber{5}%
\task{%
    \begin{enumerate}
        \item Запишите (формулой) закон сохранения электрического заряда.
        \item Зарисуйте электрическое поле точечного положительного электрического заряда.
        \item Запишите формулу для вычисления напряжённости электрического поля точечного заряда.
        \item Запишите принцип суперпозиции (правило сложения) потенциалов.
    \end{enumerate}
}

\variantsplitter

\addpersonalvariant{Андрей Рожков}

\tasknumber{1}%
\task{%
    Два одинаковых маленьких проводящих заряженных шарика находятся на расстоянии~$d$ друг от друга.
    Заряд первого равен~$-5q$, второго~--- $-8q$.
    Шарики приводят в соприкосновение, а после опять разводят на расстояние~$2d$.
    \begin{itemize}
        \item Каким стал заряд каждого из шариков?
        \item Определите характер (притяжение или отталкивание) и силу взаимодействия шариков до и после соприкосновения.
        \item Как изменилась сила взаимодействия шариков после соприкосновения?
    \end{itemize}
}
\answer{%
    \begin{align*}
    F &= k\frac{\abs{q_1}\abs {q_2}}{\sqr{2 d}}   = k\frac{\abs{-5q} \cdot \abs{-8q}}{2^2 \cdot d^2}, \text{отталкивание}; \\
        q'_1 &= q'_2, q_1 + q_2 = q'_1 + q'_2 \implies  q'_1 = q'_2 = \frac{q_1 + q_2}2 = \frac{-5q -8q}2 = -\frac{13}2q \implies \\
        \implies F'  &= k\frac{\abs{q'_1}\abs{q'_2}}{\sqr{2 d}}
            = k\frac{\sqr{-\frac{13}2q}}{2^2 \cdot d^2},
        \text{отталкивание}, \\
    \frac{F'}F &= \frac{\sqr{-\frac{13}2q}}{2^2 \cdot \abs{-5q} \cdot \abs{-8q}} = \frac{169}{640}.
    \end{align*}
}
\solutionspace{120pt}

\tasknumber{2}%
\task{%
    На координатной плоскости в точках $(-r; 0)$ и $(r; 0)$
    находятся заряды, соответственно, $+Q$ и $+Q$.
    Сделайте рисунок, определите величину напряжённости электрического поля
    и укажите её направление в двух точках: $(0; r)$ и $(-2r; 0)$.
}
\solutionspace{120pt}

\tasknumber{3}%
\task{%
    Напряжение между двумя точками, лежащими на одной линии напряжённости
    однородного электрического поля, равно $V = 3\,\text{кВ}$.
    Расстояние между точками $l = 20\,\text{см}$.
    Какова напряжённость этого поля?
}
\answer{%
    $
        E_x = -\frac{\Delta \varphi}{\Delta x} \implies
        E = \frac{V}{l} = \frac{3\,\text{кВ}}{20\,\text{см}} = 15{,}0\,\frac{\text{кВ}}{\text{м}}.
    $
}
\solutionspace{40pt}

\tasknumber{4}%
\task{%
    В однородном электрическом поле напряжённостью $E = 4\,\frac{\text{кВ}}{\text{м}}$
    переместили заряд $Q = -40\,\text{нКл}$ в направлении силовой линии
    на $l = 4\,\text{см}$.
    Определите
    \begin{itemize}
        \item работу поля,
        \item изменение потенциальной энергии заряда.
        % \item напряжение между начальной и конечной точками перемещения.
    \end{itemize}
}
\answer{%
    \begin{align*}
    A &= F \cdot l \cdot \cos \alpha = EQ \cdot l \cdot 1 = EQl = 4\,\frac{\text{кВ}}{\text{м}} \cdot -40\,\text{нКл} \cdot 4\,\text{см} = -6{,}40\,\text{мкДж}, \\
    \Delta E_\text{пот.} &= -A = 6{,}4\,\text{мкДж}
    \end{align*}
}
\solutionspace{80pt}

\tasknumber{5}%
\task{%
    \begin{enumerate}
        \item Запишите (формулой) закон Кулона (в вакууме).
        \item Зарисуйте электрическое поле точечного положительного электрического заряда.
        \item Запишите формулу для вычисления потенциала электрического поля точечного заряда.
        \item Запишите принцип суперпозиции (правило сложения) потенциалов.
    \end{enumerate}
}

\variantsplitter

\addpersonalvariant{Рената Таржиманова}

\tasknumber{1}%
\task{%
    Два одинаковых маленьких проводящих заряженных шарика находятся на расстоянии~$l$ друг от друга.
    Заряд первого равен~$-5Q$, второго~--- $+2Q$.
    Шарики приводят в соприкосновение, а после опять разводят на расстояние~$4l$.
    \begin{itemize}
        \item Каким стал заряд каждого из шариков?
        \item Определите характер (притяжение или отталкивание) и силу взаимодействия шариков до и после соприкосновения.
        \item Как изменилась сила взаимодействия шариков после соприкосновения?
    \end{itemize}
}
\answer{%
    \begin{align*}
    F &= k\frac{\abs{q_1}\abs {q_2}}{\sqr{4 l}}   = k\frac{\abs{-5Q} \cdot \abs{+2Q}}{4^2 \cdot l^2}, \text{притяжение}; \\
        q'_1 &= q'_2, q_1 + q_2 = q'_1 + q'_2 \implies  q'_1 = q'_2 = \frac{q_1 + q_2}2 = \frac{-5Q + +2Q}2 = -\frac32Q \implies \\
        \implies F'  &= k\frac{\abs{q'_1}\abs{q'_2}}{\sqr{4 l}}
            = k\frac{\sqr{-\frac32Q}}{4^2 \cdot l^2},
        \text{отталкивание}, \\
    \frac{F'}F &= \frac{\sqr{-\frac32Q}}{4^2 \cdot \abs{-5Q} \cdot \abs{+2Q}} = \frac9{640}.
    \end{align*}
}
\solutionspace{120pt}

\tasknumber{2}%
\task{%
    На координатной плоскости в точках $(-a; 0)$ и $(a; 0)$
    находятся заряды, соответственно, $-Q$ и $+Q$.
    Сделайте рисунок, определите величину напряжённости электрического поля
    и укажите её направление в двух точках: $(0; -a)$ и $(-2a; 0)$.
}
\solutionspace{120pt}

\tasknumber{3}%
\task{%
    Напряжение между двумя точками, лежащими на одной линии напряжённости
    однородного электрического поля, равно $U = 5\,\text{кВ}$.
    Расстояние между точками $r = 20\,\text{см}$.
    Какова напряжённость этого поля?
}
\answer{%
    $
        E_x = -\frac{\Delta \varphi}{\Delta x} \implies
        E = \frac{U}{r} = \frac{5\,\text{кВ}}{20\,\text{см}} = 25{,}0\,\frac{\text{кВ}}{\text{м}}.
    $
}
\solutionspace{40pt}

\tasknumber{4}%
\task{%
    В однородном электрическом поле напряжённостью $E = 4\,\frac{\text{кВ}}{\text{м}}$
    переместили заряд $Q = 10\,\text{нКл}$ в направлении силовой линии
    на $r = 4\,\text{см}$.
    Определите
    \begin{itemize}
        \item работу поля,
        \item изменение потенциальной энергии заряда.
        % \item напряжение между начальной и конечной точками перемещения.
    \end{itemize}
}
\answer{%
    \begin{align*}
    A &= F \cdot r \cdot \cos \alpha = EQ \cdot r \cdot 1 = EQr = 4\,\frac{\text{кВ}}{\text{м}} \cdot 10\,\text{нКл} \cdot 4\,\text{см} = 1{,}6\,\text{мкДж}, \\
    \Delta E_\text{пот.} &= -A = -1{,}600\,\text{мкДж}
    \end{align*}
}
\solutionspace{80pt}

\tasknumber{5}%
\task{%
    \begin{enumerate}
        \item Запишите (формулой) закон Кулона (в вакууме).
        \item Зарисуйте электрическое поле точечного отрицательного электрического заряда.
        \item Запишите формулу для вычисления потенциала электрического поля точечного заряда.
        \item Запишите принцип суперпозиции (правило сложения) потенциалов.
    \end{enumerate}
}

\variantsplitter

\addpersonalvariant{Андрей Щербаков}

\tasknumber{1}%
\task{%
    Два одинаковых маленьких проводящих заряженных шарика находятся на расстоянии~$d$ друг от друга.
    Заряд первого равен~$+7q$, второго~--- $-8q$.
    Шарики приводят в соприкосновение, а после опять разводят на расстояние~$3d$.
    \begin{itemize}
        \item Каким стал заряд каждого из шариков?
        \item Определите характер (притяжение или отталкивание) и силу взаимодействия шариков до и после соприкосновения.
        \item Как изменилась сила взаимодействия шариков после соприкосновения?
    \end{itemize}
}
\answer{%
    \begin{align*}
    F &= k\frac{\abs{q_1}\abs {q_2}}{\sqr{3 d}}   = k\frac{\abs{+7q} \cdot \abs{-8q}}{3^2 \cdot d^2}, \text{притяжение}; \\
        q'_1 &= q'_2, q_1 + q_2 = q'_1 + q'_2 \implies  q'_1 = q'_2 = \frac{q_1 + q_2}2 = \frac{+7q -8q}2 = -\frac12q \implies \\
        \implies F'  &= k\frac{\abs{q'_1}\abs{q'_2}}{\sqr{3 d}}
            = k\frac{\sqr{-\frac12q}}{3^2 \cdot d^2},
        \text{отталкивание}, \\
    \frac{F'}F &= \frac{\sqr{-\frac12q}}{3^2 \cdot \abs{+7q} \cdot \abs{-8q}} = \frac1{2016}.
    \end{align*}
}
\solutionspace{120pt}

\tasknumber{2}%
\task{%
    На координатной плоскости в точках $(-d; 0)$ и $(d; 0)$
    находятся заряды, соответственно, $-Q$ и $+Q$.
    Сделайте рисунок, определите величину напряжённости электрического поля
    и укажите её направление в двух точках: $(0; d)$ и $(-2d; 0)$.
}
\solutionspace{120pt}

\tasknumber{3}%
\task{%
    Напряжение между двумя точками, лежащими на одной линии напряжённости
    однородного электрического поля, равно $V = 3\,\text{кВ}$.
    Расстояние между точками $d = 30\,\text{см}$.
    Какова напряжённость этого поля?
}
\answer{%
    $
        E_x = -\frac{\Delta \varphi}{\Delta x} \implies
        E = \frac{V}{d} = \frac{3\,\text{кВ}}{30\,\text{см}} = 10{,}0\,\frac{\text{кВ}}{\text{м}}.
    $
}
\solutionspace{40pt}

\tasknumber{4}%
\task{%
    В однородном электрическом поле напряжённостью $E = 20\,\frac{\text{кВ}}{\text{м}}$
    переместили заряд $q = 25\,\text{нКл}$ в направлении силовой линии
    на $l = 2\,\text{см}$.
    Определите
    \begin{itemize}
        \item работу поля,
        \item изменение потенциальной энергии заряда.
        % \item напряжение между начальной и конечной точками перемещения.
    \end{itemize}
}
\answer{%
    \begin{align*}
    A &= F \cdot l \cdot \cos \alpha = Eq \cdot l \cdot 1 = Eql = 20\,\frac{\text{кВ}}{\text{м}} \cdot 25\,\text{нКл} \cdot 2\,\text{см} = 10{,}0\,\text{мкДж}, \\
    \Delta E_\text{пот.} &= -A = -10{,}000\,\text{мкДж}
    \end{align*}
}
\solutionspace{80pt}

\tasknumber{5}%
\task{%
    \begin{enumerate}
        \item Запишите (формулой) закон Кулона (в вакууме).
        \item Зарисуйте электрическое поле точечного отрицательного электрического заряда.
        \item Запишите формулу для вычисления потенциала электрического поля точечного заряда.
        \item Запишите принцип суперпозиции (правило сложения) потенциалов.
    \end{enumerate}
}

\variantsplitter

\addpersonalvariant{Михаил Ярошевский}

\tasknumber{1}%
\task{%
    Два одинаковых маленьких проводящих заряженных шарика находятся на расстоянии~$r$ друг от друга.
    Заряд первого равен~$+5Q$, второго~--- $+4Q$.
    Шарики приводят в соприкосновение, а после опять разводят на расстояние~$2r$.
    \begin{itemize}
        \item Каким стал заряд каждого из шариков?
        \item Определите характер (притяжение или отталкивание) и силу взаимодействия шариков до и после соприкосновения.
        \item Как изменилась сила взаимодействия шариков после соприкосновения?
    \end{itemize}
}
\answer{%
    \begin{align*}
    F &= k\frac{\abs{q_1}\abs {q_2}}{\sqr{2 r}}   = k\frac{\abs{+5Q} \cdot \abs{+4Q}}{2^2 \cdot r^2}, \text{отталкивание}; \\
        q'_1 &= q'_2, q_1 + q_2 = q'_1 + q'_2 \implies  q'_1 = q'_2 = \frac{q_1 + q_2}2 = \frac{+5Q + +4Q}2 = \frac92Q \implies \\
        \implies F'  &= k\frac{\abs{q'_1}\abs{q'_2}}{\sqr{2 r}}
            = k\frac{\sqr{\frac92Q}}{2^2 \cdot r^2},
        \text{отталкивание}, \\
    \frac{F'}F &= \frac{\sqr{\frac92Q}}{2^2 \cdot \abs{+5Q} \cdot \abs{+4Q}} = \frac{81}{320}.
    \end{align*}
}
\solutionspace{120pt}

\tasknumber{2}%
\task{%
    На координатной плоскости в точках $(-l; 0)$ и $(l; 0)$
    находятся заряды, соответственно, $-q$ и $-q$.
    Сделайте рисунок, определите величину напряжённости электрического поля
    и укажите её направление в двух точках: $(0; l)$ и $(-2l; 0)$.
}
\solutionspace{120pt}

\tasknumber{3}%
\task{%
    Напряжение между двумя точками, лежащими на одной линии напряжённости
    однородного электрического поля, равно $V = 3\,\text{кВ}$.
    Расстояние между точками $l = 10\,\text{см}$.
    Какова напряжённость этого поля?
}
\answer{%
    $
        E_x = -\frac{\Delta \varphi}{\Delta x} \implies
        E = \frac{V}{l} = \frac{3\,\text{кВ}}{10\,\text{см}} = 30{,}0\,\frac{\text{кВ}}{\text{м}}.
    $
}
\solutionspace{40pt}

\tasknumber{4}%
\task{%
    В однородном электрическом поле напряжённостью $E = 20\,\frac{\text{кВ}}{\text{м}}$
    переместили заряд $Q = 25\,\text{нКл}$ в направлении силовой линии
    на $l = 5\,\text{см}$.
    Определите
    \begin{itemize}
        \item работу поля,
        \item изменение потенциальной энергии заряда.
        % \item напряжение между начальной и конечной точками перемещения.
    \end{itemize}
}
\answer{%
    \begin{align*}
    A &= F \cdot l \cdot \cos \alpha = EQ \cdot l \cdot 1 = EQl = 20\,\frac{\text{кВ}}{\text{м}} \cdot 25\,\text{нКл} \cdot 5\,\text{см} = 25{,}0\,\text{мкДж}, \\
    \Delta E_\text{пот.} &= -A = -25{,}00\,\text{мкДж}
    \end{align*}
}
\solutionspace{80pt}

\tasknumber{5}%
\task{%
    \begin{enumerate}
        \item Запишите (формулой) закон Кулона (в вакууме).
        \item Зарисуйте электрическое поле точечного положительного электрического заряда.
        \item Запишите формулу для вычисления напряжённости электрического поля точечного заряда.
        \item Запишите принцип суперпозиции (правило сложения) напряжённостей.
    \end{enumerate}
}

\variantsplitter

\addpersonalvariant{Алексей Алимпиев}

\tasknumber{1}%
\task{%
    Два одинаковых маленьких проводящих заряженных шарика находятся на расстоянии~$d$ друг от друга.
    Заряд первого равен~$+5Q$, второго~--- $-2Q$.
    Шарики приводят в соприкосновение, а после опять разводят на расстояние~$3d$.
    \begin{itemize}
        \item Каким стал заряд каждого из шариков?
        \item Определите характер (притяжение или отталкивание) и силу взаимодействия шариков до и после соприкосновения.
        \item Как изменилась сила взаимодействия шариков после соприкосновения?
    \end{itemize}
}
\answer{%
    \begin{align*}
    F &= k\frac{\abs{q_1}\abs {q_2}}{\sqr{3 d}}   = k\frac{\abs{+5Q} \cdot \abs{-2Q}}{3^2 \cdot d^2}, \text{притяжение}; \\
        q'_1 &= q'_2, q_1 + q_2 = q'_1 + q'_2 \implies  q'_1 = q'_2 = \frac{q_1 + q_2}2 = \frac{+5Q -2Q}2 = \frac32Q \implies \\
        \implies F'  &= k\frac{\abs{q'_1}\abs{q'_2}}{\sqr{3 d}}
            = k\frac{\sqr{\frac32Q}}{3^2 \cdot d^2},
        \text{отталкивание}, \\
    \frac{F'}F &= \frac{\sqr{\frac32Q}}{3^2 \cdot \abs{+5Q} \cdot \abs{-2Q}} = \frac1{40}.
    \end{align*}
}
\solutionspace{120pt}

\tasknumber{2}%
\task{%
    На координатной плоскости в точках $(-d; 0)$ и $(d; 0)$
    находятся заряды, соответственно, $+Q$ и $+Q$.
    Сделайте рисунок, определите величину напряжённости электрического поля
    и укажите её направление в двух точках: $(0; -d)$ и $(2d; 0)$.
}
\solutionspace{120pt}

\tasknumber{3}%
\task{%
    Напряжение между двумя точками, лежащими на одной линии напряжённости
    однородного электрического поля, равно $U = 3\,\text{кВ}$.
    Расстояние между точками $r = 30\,\text{см}$.
    Какова напряжённость этого поля?
}
\answer{%
    $
        E_x = -\frac{\Delta \varphi}{\Delta x} \implies
        E = \frac{U}{r} = \frac{3\,\text{кВ}}{30\,\text{см}} = 10{,}0\,\frac{\text{кВ}}{\text{м}}.
    $
}
\solutionspace{40pt}

\tasknumber{4}%
\task{%
    В однородном электрическом поле напряжённостью $E = 20\,\frac{\text{кВ}}{\text{м}}$
    переместили заряд $q = 40\,\text{нКл}$ в направлении силовой линии
    на $d = 10\,\text{см}$.
    Определите
    \begin{itemize}
        \item работу поля,
        \item изменение потенциальной энергии заряда.
        % \item напряжение между начальной и конечной точками перемещения.
    \end{itemize}
}
\answer{%
    \begin{align*}
    A &= F \cdot d \cdot \cos \alpha = Eq \cdot d \cdot 1 = Eqd = 20\,\frac{\text{кВ}}{\text{м}} \cdot 40\,\text{нКл} \cdot 10\,\text{см} = 80{,}0\,\text{мкДж}, \\
    \Delta E_\text{пот.} &= -A = -80{,}00\,\text{мкДж}
    \end{align*}
}
\solutionspace{80pt}

\tasknumber{5}%
\task{%
    \begin{enumerate}
        \item Запишите (формулой) закон Кулона (в вакууме).
        \item Зарисуйте электрическое поле точечного отрицательного электрического заряда.
        \item Запишите формулу для вычисления напряжённости электрического поля точечного заряда.
        \item Запишите принцип суперпозиции (правило сложения) потенциалов.
    \end{enumerate}
}

\variantsplitter

\addpersonalvariant{Евгений Васин}

\tasknumber{1}%
\task{%
    Два одинаковых маленьких проводящих заряженных шарика находятся на расстоянии~$l$ друг от друга.
    Заряд первого равен~$+7Q$, второго~--- $+4Q$.
    Шарики приводят в соприкосновение, а после опять разводят на расстояние~$2l$.
    \begin{itemize}
        \item Каким стал заряд каждого из шариков?
        \item Определите характер (притяжение или отталкивание) и силу взаимодействия шариков до и после соприкосновения.
        \item Как изменилась сила взаимодействия шариков после соприкосновения?
    \end{itemize}
}
\answer{%
    \begin{align*}
    F &= k\frac{\abs{q_1}\abs {q_2}}{\sqr{2 l}}   = k\frac{\abs{+7Q} \cdot \abs{+4Q}}{2^2 \cdot l^2}, \text{отталкивание}; \\
        q'_1 &= q'_2, q_1 + q_2 = q'_1 + q'_2 \implies  q'_1 = q'_2 = \frac{q_1 + q_2}2 = \frac{+7Q + +4Q}2 = \frac{11}2Q \implies \\
        \implies F'  &= k\frac{\abs{q'_1}\abs{q'_2}}{\sqr{2 l}}
            = k\frac{\sqr{\frac{11}2Q}}{2^2 \cdot l^2},
        \text{отталкивание}, \\
    \frac{F'}F &= \frac{\sqr{\frac{11}2Q}}{2^2 \cdot \abs{+7Q} \cdot \abs{+4Q}} = \frac{121}{448}.
    \end{align*}
}
\solutionspace{120pt}

\tasknumber{2}%
\task{%
    На координатной плоскости в точках $(-l; 0)$ и $(l; 0)$
    находятся заряды, соответственно, $+q$ и $-q$.
    Сделайте рисунок, определите величину напряжённости электрического поля
    и укажите её направление в двух точках: $(0; l)$ и $(2l; 0)$.
}
\solutionspace{120pt}

\tasknumber{3}%
\task{%
    Напряжение между двумя точками, лежащими на одной линии напряжённости
    однородного электрического поля, равно $V = 5\,\text{кВ}$.
    Расстояние между точками $d = 10\,\text{см}$.
    Какова напряжённость этого поля?
}
\answer{%
    $
        E_x = -\frac{\Delta \varphi}{\Delta x} \implies
        E = \frac{V}{d} = \frac{5\,\text{кВ}}{10\,\text{см}} = 50{,}0\,\frac{\text{кВ}}{\text{м}}.
    $
}
\solutionspace{40pt}

\tasknumber{4}%
\task{%
    В однородном электрическом поле напряжённостью $E = 20\,\frac{\text{кВ}}{\text{м}}$
    переместили заряд $q = 25\,\text{нКл}$ в направлении силовой линии
    на $d = 2\,\text{см}$.
    Определите
    \begin{itemize}
        \item работу поля,
        \item изменение потенциальной энергии заряда.
        % \item напряжение между начальной и конечной точками перемещения.
    \end{itemize}
}
\answer{%
    \begin{align*}
    A &= F \cdot d \cdot \cos \alpha = Eq \cdot d \cdot 1 = Eqd = 20\,\frac{\text{кВ}}{\text{м}} \cdot 25\,\text{нКл} \cdot 2\,\text{см} = 10{,}0\,\text{мкДж}, \\
    \Delta E_\text{пот.} &= -A = -10{,}000\,\text{мкДж}
    \end{align*}
}
\solutionspace{80pt}

\tasknumber{5}%
\task{%
    \begin{enumerate}
        \item Запишите (формулой) закон сохранения электрического заряда.
        \item Зарисуйте электрическое поле точечного отрицательного электрического заряда.
        \item Запишите формулу для вычисления потенциала электрического поля точечного заряда.
        \item Запишите принцип суперпозиции (правило сложения) потенциалов.
    \end{enumerate}
}

\variantsplitter

\addpersonalvariant{Вячеслав Волохов}

\tasknumber{1}%
\task{%
    Два одинаковых маленьких проводящих заряженных шарика находятся на расстоянии~$l$ друг от друга.
    Заряд первого равен~$-3q$, второго~--- $-4q$.
    Шарики приводят в соприкосновение, а после опять разводят на расстояние~$3l$.
    \begin{itemize}
        \item Каким стал заряд каждого из шариков?
        \item Определите характер (притяжение или отталкивание) и силу взаимодействия шариков до и после соприкосновения.
        \item Как изменилась сила взаимодействия шариков после соприкосновения?
    \end{itemize}
}
\answer{%
    \begin{align*}
    F &= k\frac{\abs{q_1}\abs {q_2}}{\sqr{3 l}}   = k\frac{\abs{-3q} \cdot \abs{-4q}}{3^2 \cdot l^2}, \text{отталкивание}; \\
        q'_1 &= q'_2, q_1 + q_2 = q'_1 + q'_2 \implies  q'_1 = q'_2 = \frac{q_1 + q_2}2 = \frac{-3q -4q}2 = -\frac72q \implies \\
        \implies F'  &= k\frac{\abs{q'_1}\abs{q'_2}}{\sqr{3 l}}
            = k\frac{\sqr{-\frac72q}}{3^2 \cdot l^2},
        \text{отталкивание}, \\
    \frac{F'}F &= \frac{\sqr{-\frac72q}}{3^2 \cdot \abs{-3q} \cdot \abs{-4q}} = \frac{49}{432}.
    \end{align*}
}
\solutionspace{120pt}

\tasknumber{2}%
\task{%
    На координатной плоскости в точках $(-l; 0)$ и $(l; 0)$
    находятся заряды, соответственно, $+q$ и $-q$.
    Сделайте рисунок, определите величину напряжённости электрического поля
    и укажите её направление в двух точках: $(0; l)$ и $(-2l; 0)$.
}
\solutionspace{120pt}

\tasknumber{3}%
\task{%
    Напряжение между двумя точками, лежащими на одной линии напряжённости
    однородного электрического поля, равно $U = 6\,\text{кВ}$.
    Расстояние между точками $r = 30\,\text{см}$.
    Какова напряжённость этого поля?
}
\answer{%
    $
        E_x = -\frac{\Delta \varphi}{\Delta x} \implies
        E = \frac{U}{r} = \frac{6\,\text{кВ}}{30\,\text{см}} = 20{,}0\,\frac{\text{кВ}}{\text{м}}.
    $
}
\solutionspace{40pt}

\tasknumber{4}%
\task{%
    В однородном электрическом поле напряжённостью $E = 2\,\frac{\text{кВ}}{\text{м}}$
    переместили заряд $q = -25\,\text{нКл}$ в направлении силовой линии
    на $r = 10\,\text{см}$.
    Определите
    \begin{itemize}
        \item работу поля,
        \item изменение потенциальной энергии заряда.
        % \item напряжение между начальной и конечной точками перемещения.
    \end{itemize}
}
\answer{%
    \begin{align*}
    A &= F \cdot r \cdot \cos \alpha = Eq \cdot r \cdot 1 = Eqr = 2\,\frac{\text{кВ}}{\text{м}} \cdot -25\,\text{нКл} \cdot 10\,\text{см} = -5{,}00\,\text{мкДж}, \\
    \Delta E_\text{пот.} &= -A = 5{,}0\,\text{мкДж}
    \end{align*}
}
\solutionspace{80pt}

\tasknumber{5}%
\task{%
    \begin{enumerate}
        \item Запишите (формулой) закон Кулона (в вакууме).
        \item Зарисуйте электрическое поле точечного отрицательного электрического заряда.
        \item Запишите формулу для вычисления напряжённости электрического поля точечного заряда.
        \item Запишите принцип суперпозиции (правило сложения) потенциалов.
    \end{enumerate}
}

\variantsplitter

\addpersonalvariant{Герман Говоров}

\tasknumber{1}%
\task{%
    Два одинаковых маленьких проводящих заряженных шарика находятся на расстоянии~$r$ друг от друга.
    Заряд первого равен~$-7Q$, второго~--- $-4Q$.
    Шарики приводят в соприкосновение, а после опять разводят на расстояние~$2r$.
    \begin{itemize}
        \item Каким стал заряд каждого из шариков?
        \item Определите характер (притяжение или отталкивание) и силу взаимодействия шариков до и после соприкосновения.
        \item Как изменилась сила взаимодействия шариков после соприкосновения?
    \end{itemize}
}
\answer{%
    \begin{align*}
    F &= k\frac{\abs{q_1}\abs {q_2}}{\sqr{2 r}}   = k\frac{\abs{-7Q} \cdot \abs{-4Q}}{2^2 \cdot r^2}, \text{отталкивание}; \\
        q'_1 &= q'_2, q_1 + q_2 = q'_1 + q'_2 \implies  q'_1 = q'_2 = \frac{q_1 + q_2}2 = \frac{-7Q -4Q}2 = -\frac{11}2Q \implies \\
        \implies F'  &= k\frac{\abs{q'_1}\abs{q'_2}}{\sqr{2 r}}
            = k\frac{\sqr{-\frac{11}2Q}}{2^2 \cdot r^2},
        \text{отталкивание}, \\
    \frac{F'}F &= \frac{\sqr{-\frac{11}2Q}}{2^2 \cdot \abs{-7Q} \cdot \abs{-4Q}} = \frac{121}{448}.
    \end{align*}
}
\solutionspace{120pt}

\tasknumber{2}%
\task{%
    На координатной плоскости в точках $(-l; 0)$ и $(l; 0)$
    находятся заряды, соответственно, $+Q$ и $+Q$.
    Сделайте рисунок, определите величину напряжённости электрического поля
    и укажите её направление в двух точках: $(0; -l)$ и $(2l; 0)$.
}
\solutionspace{120pt}

\tasknumber{3}%
\task{%
    Напряжение между двумя точками, лежащими на одной линии напряжённости
    однородного электрического поля, равно $V = 4\,\text{кВ}$.
    Расстояние между точками $r = 40\,\text{см}$.
    Какова напряжённость этого поля?
}
\answer{%
    $
        E_x = -\frac{\Delta \varphi}{\Delta x} \implies
        E = \frac{V}{r} = \frac{4\,\text{кВ}}{40\,\text{см}} = 10{,}0\,\frac{\text{кВ}}{\text{м}}.
    $
}
\solutionspace{40pt}

\tasknumber{4}%
\task{%
    В однородном электрическом поле напряжённостью $E = 2\,\frac{\text{кВ}}{\text{м}}$
    переместили заряд $Q = -40\,\text{нКл}$ в направлении силовой линии
    на $r = 10\,\text{см}$.
    Определите
    \begin{itemize}
        \item работу поля,
        \item изменение потенциальной энергии заряда.
        % \item напряжение между начальной и конечной точками перемещения.
    \end{itemize}
}
\answer{%
    \begin{align*}
    A &= F \cdot r \cdot \cos \alpha = EQ \cdot r \cdot 1 = EQr = 2\,\frac{\text{кВ}}{\text{м}} \cdot -40\,\text{нКл} \cdot 10\,\text{см} = -8{,}00\,\text{мкДж}, \\
    \Delta E_\text{пот.} &= -A = 8{,}0\,\text{мкДж}
    \end{align*}
}
\solutionspace{80pt}

\tasknumber{5}%
\task{%
    \begin{enumerate}
        \item Запишите (формулой) закон сохранения электрического заряда.
        \item Зарисуйте электрическое поле точечного отрицательного электрического заряда.
        \item Запишите формулу для вычисления потенциала электрического поля точечного заряда.
        \item Запишите принцип суперпозиции (правило сложения) напряжённостей.
    \end{enumerate}
}

\variantsplitter

\addpersonalvariant{София Журавлёва}

\tasknumber{1}%
\task{%
    Два одинаковых маленьких проводящих заряженных шарика находятся на расстоянии~$l$ друг от друга.
    Заряд первого равен~$-3q$, второго~--- $-8q$.
    Шарики приводят в соприкосновение, а после опять разводят на расстояние~$4l$.
    \begin{itemize}
        \item Каким стал заряд каждого из шариков?
        \item Определите характер (притяжение или отталкивание) и силу взаимодействия шариков до и после соприкосновения.
        \item Как изменилась сила взаимодействия шариков после соприкосновения?
    \end{itemize}
}
\answer{%
    \begin{align*}
    F &= k\frac{\abs{q_1}\abs {q_2}}{\sqr{4 l}}   = k\frac{\abs{-3q} \cdot \abs{-8q}}{4^2 \cdot l^2}, \text{отталкивание}; \\
        q'_1 &= q'_2, q_1 + q_2 = q'_1 + q'_2 \implies  q'_1 = q'_2 = \frac{q_1 + q_2}2 = \frac{-3q -8q}2 = -\frac{11}2q \implies \\
        \implies F'  &= k\frac{\abs{q'_1}\abs{q'_2}}{\sqr{4 l}}
            = k\frac{\sqr{-\frac{11}2q}}{4^2 \cdot l^2},
        \text{отталкивание}, \\
    \frac{F'}F &= \frac{\sqr{-\frac{11}2q}}{4^2 \cdot \abs{-3q} \cdot \abs{-8q}} = \frac{121}{1536}.
    \end{align*}
}
\solutionspace{120pt}

\tasknumber{2}%
\task{%
    На координатной плоскости в точках $(-d; 0)$ и $(d; 0)$
    находятся заряды, соответственно, $+Q$ и $+Q$.
    Сделайте рисунок, определите величину напряжённости электрического поля
    и укажите её направление в двух точках: $(0; -d)$ и $(2d; 0)$.
}
\solutionspace{120pt}

\tasknumber{3}%
\task{%
    Напряжение между двумя точками, лежащими на одной линии напряжённости
    однородного электрического поля, равно $V = 6\,\text{кВ}$.
    Расстояние между точками $d = 40\,\text{см}$.
    Какова напряжённость этого поля?
}
\answer{%
    $
        E_x = -\frac{\Delta \varphi}{\Delta x} \implies
        E = \frac{V}{d} = \frac{6\,\text{кВ}}{40\,\text{см}} = 15{,}0\,\frac{\text{кВ}}{\text{м}}.
    $
}
\solutionspace{40pt}

\tasknumber{4}%
\task{%
    В однородном электрическом поле напряжённостью $E = 20\,\frac{\text{кВ}}{\text{м}}$
    переместили заряд $Q = 25\,\text{нКл}$ в направлении силовой линии
    на $d = 10\,\text{см}$.
    Определите
    \begin{itemize}
        \item работу поля,
        \item изменение потенциальной энергии заряда.
        % \item напряжение между начальной и конечной точками перемещения.
    \end{itemize}
}
\answer{%
    \begin{align*}
    A &= F \cdot d \cdot \cos \alpha = EQ \cdot d \cdot 1 = EQd = 20\,\frac{\text{кВ}}{\text{м}} \cdot 25\,\text{нКл} \cdot 10\,\text{см} = 50{,}0\,\text{мкДж}, \\
    \Delta E_\text{пот.} &= -A = -50{,}00\,\text{мкДж}
    \end{align*}
}
\solutionspace{80pt}

\tasknumber{5}%
\task{%
    \begin{enumerate}
        \item Запишите (формулой) закон сохранения электрического заряда.
        \item Зарисуйте электрическое поле точечного отрицательного электрического заряда.
        \item Запишите формулу для вычисления напряжённости электрического поля точечного заряда.
        \item Запишите принцип суперпозиции (правило сложения) потенциалов.
    \end{enumerate}
}

\variantsplitter

\addpersonalvariant{Константин Козлов}

\tasknumber{1}%
\task{%
    Два одинаковых маленьких проводящих заряженных шарика находятся на расстоянии~$l$ друг от друга.
    Заряд первого равен~$-7q$, второго~--- $+4q$.
    Шарики приводят в соприкосновение, а после опять разводят на расстояние~$3l$.
    \begin{itemize}
        \item Каким стал заряд каждого из шариков?
        \item Определите характер (притяжение или отталкивание) и силу взаимодействия шариков до и после соприкосновения.
        \item Как изменилась сила взаимодействия шариков после соприкосновения?
    \end{itemize}
}
\answer{%
    \begin{align*}
    F &= k\frac{\abs{q_1}\abs {q_2}}{\sqr{3 l}}   = k\frac{\abs{-7q} \cdot \abs{+4q}}{3^2 \cdot l^2}, \text{притяжение}; \\
        q'_1 &= q'_2, q_1 + q_2 = q'_1 + q'_2 \implies  q'_1 = q'_2 = \frac{q_1 + q_2}2 = \frac{-7q + +4q}2 = -\frac32q \implies \\
        \implies F'  &= k\frac{\abs{q'_1}\abs{q'_2}}{\sqr{3 l}}
            = k\frac{\sqr{-\frac32q}}{3^2 \cdot l^2},
        \text{отталкивание}, \\
    \frac{F'}F &= \frac{\sqr{-\frac32q}}{3^2 \cdot \abs{-7q} \cdot \abs{+4q}} = \frac1{112}.
    \end{align*}
}
\solutionspace{120pt}

\tasknumber{2}%
\task{%
    На координатной плоскости в точках $(-l; 0)$ и $(l; 0)$
    находятся заряды, соответственно, $-q$ и $-q$.
    Сделайте рисунок, определите величину напряжённости электрического поля
    и укажите её направление в двух точках: $(0; l)$ и $(-2l; 0)$.
}
\solutionspace{120pt}

\tasknumber{3}%
\task{%
    Напряжение между двумя точками, лежащими на одной линии напряжённости
    однородного электрического поля, равно $V = 5\,\text{кВ}$.
    Расстояние между точками $r = 30\,\text{см}$.
    Какова напряжённость этого поля?
}
\answer{%
    $
        E_x = -\frac{\Delta \varphi}{\Delta x} \implies
        E = \frac{V}{r} = \frac{5\,\text{кВ}}{30\,\text{см}} = 16{,}7\,\frac{\text{кВ}}{\text{м}}.
    $
}
\solutionspace{40pt}

\tasknumber{4}%
\task{%
    В однородном электрическом поле напряжённостью $E = 4\,\frac{\text{кВ}}{\text{м}}$
    переместили заряд $Q = -40\,\text{нКл}$ в направлении силовой линии
    на $r = 10\,\text{см}$.
    Определите
    \begin{itemize}
        \item работу поля,
        \item изменение потенциальной энергии заряда.
        % \item напряжение между начальной и конечной точками перемещения.
    \end{itemize}
}
\answer{%
    \begin{align*}
    A &= F \cdot r \cdot \cos \alpha = EQ \cdot r \cdot 1 = EQr = 4\,\frac{\text{кВ}}{\text{м}} \cdot -40\,\text{нКл} \cdot 10\,\text{см} = -16{,}000\,\text{мкДж}, \\
    \Delta E_\text{пот.} &= -A = 16{,}0\,\text{мкДж}
    \end{align*}
}
\solutionspace{80pt}

\tasknumber{5}%
\task{%
    \begin{enumerate}
        \item Запишите (формулой) закон Кулона (в вакууме).
        \item Зарисуйте электрическое поле точечного положительного электрического заряда.
        \item Запишите формулу для вычисления напряжённости электрического поля точечного заряда.
        \item Запишите принцип суперпозиции (правило сложения) потенциалов.
    \end{enumerate}
}

\variantsplitter

\addpersonalvariant{Наталья Кравченко}

\tasknumber{1}%
\task{%
    Два одинаковых маленьких проводящих заряженных шарика находятся на расстоянии~$r$ друг от друга.
    Заряд первого равен~$+7q$, второго~--- $-4q$.
    Шарики приводят в соприкосновение, а после опять разводят на расстояние~$3r$.
    \begin{itemize}
        \item Каким стал заряд каждого из шариков?
        \item Определите характер (притяжение или отталкивание) и силу взаимодействия шариков до и после соприкосновения.
        \item Как изменилась сила взаимодействия шариков после соприкосновения?
    \end{itemize}
}
\answer{%
    \begin{align*}
    F &= k\frac{\abs{q_1}\abs {q_2}}{\sqr{3 r}}   = k\frac{\abs{+7q} \cdot \abs{-4q}}{3^2 \cdot r^2}, \text{притяжение}; \\
        q'_1 &= q'_2, q_1 + q_2 = q'_1 + q'_2 \implies  q'_1 = q'_2 = \frac{q_1 + q_2}2 = \frac{+7q -4q}2 = \frac32q \implies \\
        \implies F'  &= k\frac{\abs{q'_1}\abs{q'_2}}{\sqr{3 r}}
            = k\frac{\sqr{\frac32q}}{3^2 \cdot r^2},
        \text{отталкивание}, \\
    \frac{F'}F &= \frac{\sqr{\frac32q}}{3^2 \cdot \abs{+7q} \cdot \abs{-4q}} = \frac1{112}.
    \end{align*}
}
\solutionspace{120pt}

\tasknumber{2}%
\task{%
    На координатной плоскости в точках $(-r; 0)$ и $(r; 0)$
    находятся заряды, соответственно, $-q$ и $-q$.
    Сделайте рисунок, определите величину напряжённости электрического поля
    и укажите её направление в двух точках: $(0; -r)$ и $(-2r; 0)$.
}
\solutionspace{120pt}

\tasknumber{3}%
\task{%
    Напряжение между двумя точками, лежащими на одной линии напряжённости
    однородного электрического поля, равно $U = 2\,\text{кВ}$.
    Расстояние между точками $r = 20\,\text{см}$.
    Какова напряжённость этого поля?
}
\answer{%
    $
        E_x = -\frac{\Delta \varphi}{\Delta x} \implies
        E = \frac{U}{r} = \frac{2\,\text{кВ}}{20\,\text{см}} = 10{,}0\,\frac{\text{кВ}}{\text{м}}.
    $
}
\solutionspace{40pt}

\tasknumber{4}%
\task{%
    В однородном электрическом поле напряжённостью $E = 20\,\frac{\text{кВ}}{\text{м}}$
    переместили заряд $q = -40\,\text{нКл}$ в направлении силовой линии
    на $d = 4\,\text{см}$.
    Определите
    \begin{itemize}
        \item работу поля,
        \item изменение потенциальной энергии заряда.
        % \item напряжение между начальной и конечной точками перемещения.
    \end{itemize}
}
\answer{%
    \begin{align*}
    A &= F \cdot d \cdot \cos \alpha = Eq \cdot d \cdot 1 = Eqd = 20\,\frac{\text{кВ}}{\text{м}} \cdot -40\,\text{нКл} \cdot 4\,\text{см} = -32{,}00\,\text{мкДж}, \\
    \Delta E_\text{пот.} &= -A = 32{,}0\,\text{мкДж}
    \end{align*}
}
\solutionspace{80pt}

\tasknumber{5}%
\task{%
    \begin{enumerate}
        \item Запишите (формулой) закон Кулона (в вакууме).
        \item Зарисуйте электрическое поле точечного положительного электрического заряда.
        \item Запишите формулу для вычисления напряжённости электрического поля точечного заряда.
        \item Запишите принцип суперпозиции (правило сложения) напряжённостей.
    \end{enumerate}
}

\variantsplitter

\addpersonalvariant{Матвей Кузьмин}

\tasknumber{1}%
\task{%
    Два одинаковых маленьких проводящих заряженных шарика находятся на расстоянии~$d$ друг от друга.
    Заряд первого равен~$-7q$, второго~--- $+8q$.
    Шарики приводят в соприкосновение, а после опять разводят на расстояние~$4d$.
    \begin{itemize}
        \item Каким стал заряд каждого из шариков?
        \item Определите характер (притяжение или отталкивание) и силу взаимодействия шариков до и после соприкосновения.
        \item Как изменилась сила взаимодействия шариков после соприкосновения?
    \end{itemize}
}
\answer{%
    \begin{align*}
    F &= k\frac{\abs{q_1}\abs {q_2}}{\sqr{4 d}}   = k\frac{\abs{-7q} \cdot \abs{+8q}}{4^2 \cdot d^2}, \text{притяжение}; \\
        q'_1 &= q'_2, q_1 + q_2 = q'_1 + q'_2 \implies  q'_1 = q'_2 = \frac{q_1 + q_2}2 = \frac{-7q + +8q}2 = \frac12q \implies \\
        \implies F'  &= k\frac{\abs{q'_1}\abs{q'_2}}{\sqr{4 d}}
            = k\frac{\sqr{\frac12q}}{4^2 \cdot d^2},
        \text{отталкивание}, \\
    \frac{F'}F &= \frac{\sqr{\frac12q}}{4^2 \cdot \abs{-7q} \cdot \abs{+8q}} = \frac1{3584}.
    \end{align*}
}
\solutionspace{120pt}

\tasknumber{2}%
\task{%
    На координатной плоскости в точках $(-a; 0)$ и $(a; 0)$
    находятся заряды, соответственно, $-q$ и $-q$.
    Сделайте рисунок, определите величину напряжённости электрического поля
    и укажите её направление в двух точках: $(0; -a)$ и $(-2a; 0)$.
}
\solutionspace{120pt}

\tasknumber{3}%
\task{%
    Напряжение между двумя точками, лежащими на одной линии напряжённости
    однородного электрического поля, равно $U = 3\,\text{кВ}$.
    Расстояние между точками $r = 20\,\text{см}$.
    Какова напряжённость этого поля?
}
\answer{%
    $
        E_x = -\frac{\Delta \varphi}{\Delta x} \implies
        E = \frac{U}{r} = \frac{3\,\text{кВ}}{20\,\text{см}} = 15{,}0\,\frac{\text{кВ}}{\text{м}}.
    $
}
\solutionspace{40pt}

\tasknumber{4}%
\task{%
    В однородном электрическом поле напряжённостью $E = 20\,\frac{\text{кВ}}{\text{м}}$
    переместили заряд $Q = 40\,\text{нКл}$ в направлении силовой линии
    на $l = 5\,\text{см}$.
    Определите
    \begin{itemize}
        \item работу поля,
        \item изменение потенциальной энергии заряда.
        % \item напряжение между начальной и конечной точками перемещения.
    \end{itemize}
}
\answer{%
    \begin{align*}
    A &= F \cdot l \cdot \cos \alpha = EQ \cdot l \cdot 1 = EQl = 20\,\frac{\text{кВ}}{\text{м}} \cdot 40\,\text{нКл} \cdot 5\,\text{см} = 40{,}0\,\text{мкДж}, \\
    \Delta E_\text{пот.} &= -A = -40{,}00\,\text{мкДж}
    \end{align*}
}
\solutionspace{80pt}

\tasknumber{5}%
\task{%
    \begin{enumerate}
        \item Запишите (формулой) закон сохранения электрического заряда.
        \item Зарисуйте электрическое поле точечного положительного электрического заряда.
        \item Запишите формулу для вычисления напряжённости электрического поля точечного заряда.
        \item Запишите принцип суперпозиции (правило сложения) напряжённостей.
    \end{enumerate}
}

\variantsplitter

\addpersonalvariant{Сергей Малышев}

\tasknumber{1}%
\task{%
    Два одинаковых маленьких проводящих заряженных шарика находятся на расстоянии~$r$ друг от друга.
    Заряд первого равен~$+3Q$, второго~--- $-4Q$.
    Шарики приводят в соприкосновение, а после опять разводят на расстояние~$4r$.
    \begin{itemize}
        \item Каким стал заряд каждого из шариков?
        \item Определите характер (притяжение или отталкивание) и силу взаимодействия шариков до и после соприкосновения.
        \item Как изменилась сила взаимодействия шариков после соприкосновения?
    \end{itemize}
}
\answer{%
    \begin{align*}
    F &= k\frac{\abs{q_1}\abs {q_2}}{\sqr{4 r}}   = k\frac{\abs{+3Q} \cdot \abs{-4Q}}{4^2 \cdot r^2}, \text{притяжение}; \\
        q'_1 &= q'_2, q_1 + q_2 = q'_1 + q'_2 \implies  q'_1 = q'_2 = \frac{q_1 + q_2}2 = \frac{+3Q -4Q}2 = -\frac12Q \implies \\
        \implies F'  &= k\frac{\abs{q'_1}\abs{q'_2}}{\sqr{4 r}}
            = k\frac{\sqr{-\frac12Q}}{4^2 \cdot r^2},
        \text{отталкивание}, \\
    \frac{F'}F &= \frac{\sqr{-\frac12Q}}{4^2 \cdot \abs{+3Q} \cdot \abs{-4Q}} = \frac1{768}.
    \end{align*}
}
\solutionspace{120pt}

\tasknumber{2}%
\task{%
    На координатной плоскости в точках $(-r; 0)$ и $(r; 0)$
    находятся заряды, соответственно, $+q$ и $-q$.
    Сделайте рисунок, определите величину напряжённости электрического поля
    и укажите её направление в двух точках: $(0; -r)$ и $(-2r; 0)$.
}
\solutionspace{120pt}

\tasknumber{3}%
\task{%
    Напряжение между двумя точками, лежащими на одной линии напряжённости
    однородного электрического поля, равно $V = 5\,\text{кВ}$.
    Расстояние между точками $r = 10\,\text{см}$.
    Какова напряжённость этого поля?
}
\answer{%
    $
        E_x = -\frac{\Delta \varphi}{\Delta x} \implies
        E = \frac{V}{r} = \frac{5\,\text{кВ}}{10\,\text{см}} = 50{,}0\,\frac{\text{кВ}}{\text{м}}.
    $
}
\solutionspace{40pt}

\tasknumber{4}%
\task{%
    В однородном электрическом поле напряжённостью $E = 2\,\frac{\text{кВ}}{\text{м}}$
    переместили заряд $q = -25\,\text{нКл}$ в направлении силовой линии
    на $l = 2\,\text{см}$.
    Определите
    \begin{itemize}
        \item работу поля,
        \item изменение потенциальной энергии заряда.
        % \item напряжение между начальной и конечной точками перемещения.
    \end{itemize}
}
\answer{%
    \begin{align*}
    A &= F \cdot l \cdot \cos \alpha = Eq \cdot l \cdot 1 = Eql = 2\,\frac{\text{кВ}}{\text{м}} \cdot -25\,\text{нКл} \cdot 2\,\text{см} = -1{,}000\,\text{мкДж}, \\
    \Delta E_\text{пот.} &= -A = 1{,}0\,\text{мкДж}
    \end{align*}
}
\solutionspace{80pt}

\tasknumber{5}%
\task{%
    \begin{enumerate}
        \item Запишите (формулой) закон сохранения электрического заряда.
        \item Зарисуйте электрическое поле точечного положительного электрического заряда.
        \item Запишите формулу для вычисления напряжённости электрического поля точечного заряда.
        \item Запишите принцип суперпозиции (правило сложения) потенциалов.
    \end{enumerate}
}

\variantsplitter

\addpersonalvariant{Алина Полканова}

\tasknumber{1}%
\task{%
    Два одинаковых маленьких проводящих заряженных шарика находятся на расстоянии~$r$ друг от друга.
    Заряд первого равен~$-5q$, второго~--- $-6q$.
    Шарики приводят в соприкосновение, а после опять разводят на расстояние~$3r$.
    \begin{itemize}
        \item Каким стал заряд каждого из шариков?
        \item Определите характер (притяжение или отталкивание) и силу взаимодействия шариков до и после соприкосновения.
        \item Как изменилась сила взаимодействия шариков после соприкосновения?
    \end{itemize}
}
\answer{%
    \begin{align*}
    F &= k\frac{\abs{q_1}\abs {q_2}}{\sqr{3 r}}   = k\frac{\abs{-5q} \cdot \abs{-6q}}{3^2 \cdot r^2}, \text{отталкивание}; \\
        q'_1 &= q'_2, q_1 + q_2 = q'_1 + q'_2 \implies  q'_1 = q'_2 = \frac{q_1 + q_2}2 = \frac{-5q -6q}2 = -\frac{11}2q \implies \\
        \implies F'  &= k\frac{\abs{q'_1}\abs{q'_2}}{\sqr{3 r}}
            = k\frac{\sqr{-\frac{11}2q}}{3^2 \cdot r^2},
        \text{отталкивание}, \\
    \frac{F'}F &= \frac{\sqr{-\frac{11}2q}}{3^2 \cdot \abs{-5q} \cdot \abs{-6q}} = \frac{121}{1080}.
    \end{align*}
}
\solutionspace{120pt}

\tasknumber{2}%
\task{%
    На координатной плоскости в точках $(-a; 0)$ и $(a; 0)$
    находятся заряды, соответственно, $+Q$ и $+Q$.
    Сделайте рисунок, определите величину напряжённости электрического поля
    и укажите её направление в двух точках: $(0; -a)$ и $(-2a; 0)$.
}
\solutionspace{120pt}

\tasknumber{3}%
\task{%
    Напряжение между двумя точками, лежащими на одной линии напряжённости
    однородного электрического поля, равно $V = 6\,\text{кВ}$.
    Расстояние между точками $d = 40\,\text{см}$.
    Какова напряжённость этого поля?
}
\answer{%
    $
        E_x = -\frac{\Delta \varphi}{\Delta x} \implies
        E = \frac{V}{d} = \frac{6\,\text{кВ}}{40\,\text{см}} = 15{,}0\,\frac{\text{кВ}}{\text{м}}.
    $
}
\solutionspace{40pt}

\tasknumber{4}%
\task{%
    В однородном электрическом поле напряжённостью $E = 20\,\frac{\text{кВ}}{\text{м}}$
    переместили заряд $Q = -25\,\text{нКл}$ в направлении силовой линии
    на $r = 5\,\text{см}$.
    Определите
    \begin{itemize}
        \item работу поля,
        \item изменение потенциальной энергии заряда.
        % \item напряжение между начальной и конечной точками перемещения.
    \end{itemize}
}
\answer{%
    \begin{align*}
    A &= F \cdot r \cdot \cos \alpha = EQ \cdot r \cdot 1 = EQr = 20\,\frac{\text{кВ}}{\text{м}} \cdot -25\,\text{нКл} \cdot 5\,\text{см} = -25{,}00\,\text{мкДж}, \\
    \Delta E_\text{пот.} &= -A = 25{,}0\,\text{мкДж}
    \end{align*}
}
\solutionspace{80pt}

\tasknumber{5}%
\task{%
    \begin{enumerate}
        \item Запишите (формулой) закон сохранения электрического заряда.
        \item Зарисуйте электрическое поле точечного положительного электрического заряда.
        \item Запишите формулу для вычисления напряжённости электрического поля точечного заряда.
        \item Запишите принцип суперпозиции (правило сложения) потенциалов.
    \end{enumerate}
}

\variantsplitter

\addpersonalvariant{Сергей Пономарёв}

\tasknumber{1}%
\task{%
    Два одинаковых маленьких проводящих заряженных шарика находятся на расстоянии~$l$ друг от друга.
    Заряд первого равен~$-3Q$, второго~--- $+2Q$.
    Шарики приводят в соприкосновение, а после опять разводят на расстояние~$2l$.
    \begin{itemize}
        \item Каким стал заряд каждого из шариков?
        \item Определите характер (притяжение или отталкивание) и силу взаимодействия шариков до и после соприкосновения.
        \item Как изменилась сила взаимодействия шариков после соприкосновения?
    \end{itemize}
}
\answer{%
    \begin{align*}
    F &= k\frac{\abs{q_1}\abs {q_2}}{\sqr{2 l}}   = k\frac{\abs{-3Q} \cdot \abs{+2Q}}{2^2 \cdot l^2}, \text{притяжение}; \\
        q'_1 &= q'_2, q_1 + q_2 = q'_1 + q'_2 \implies  q'_1 = q'_2 = \frac{q_1 + q_2}2 = \frac{-3Q + +2Q}2 = -\frac12Q \implies \\
        \implies F'  &= k\frac{\abs{q'_1}\abs{q'_2}}{\sqr{2 l}}
            = k\frac{\sqr{-\frac12Q}}{2^2 \cdot l^2},
        \text{отталкивание}, \\
    \frac{F'}F &= \frac{\sqr{-\frac12Q}}{2^2 \cdot \abs{-3Q} \cdot \abs{+2Q}} = \frac1{96}.
    \end{align*}
}
\solutionspace{120pt}

\tasknumber{2}%
\task{%
    На координатной плоскости в точках $(-d; 0)$ и $(d; 0)$
    находятся заряды, соответственно, $-Q$ и $+Q$.
    Сделайте рисунок, определите величину напряжённости электрического поля
    и укажите её направление в двух точках: $(0; d)$ и $(-2d; 0)$.
}
\solutionspace{120pt}

\tasknumber{3}%
\task{%
    Напряжение между двумя точками, лежащими на одной линии напряжённости
    однородного электрического поля, равно $U = 2\,\text{кВ}$.
    Расстояние между точками $d = 10\,\text{см}$.
    Какова напряжённость этого поля?
}
\answer{%
    $
        E_x = -\frac{\Delta \varphi}{\Delta x} \implies
        E = \frac{U}{d} = \frac{2\,\text{кВ}}{10\,\text{см}} = 20{,}0\,\frac{\text{кВ}}{\text{м}}.
    $
}
\solutionspace{40pt}

\tasknumber{4}%
\task{%
    В однородном электрическом поле напряжённостью $E = 4\,\frac{\text{кВ}}{\text{м}}$
    переместили заряд $Q = 25\,\text{нКл}$ в направлении силовой линии
    на $r = 4\,\text{см}$.
    Определите
    \begin{itemize}
        \item работу поля,
        \item изменение потенциальной энергии заряда.
        % \item напряжение между начальной и конечной точками перемещения.
    \end{itemize}
}
\answer{%
    \begin{align*}
    A &= F \cdot r \cdot \cos \alpha = EQ \cdot r \cdot 1 = EQr = 4\,\frac{\text{кВ}}{\text{м}} \cdot 25\,\text{нКл} \cdot 4\,\text{см} = 4{,}0\,\text{мкДж}, \\
    \Delta E_\text{пот.} &= -A = -4{,}00\,\text{мкДж}
    \end{align*}
}
\solutionspace{80pt}

\tasknumber{5}%
\task{%
    \begin{enumerate}
        \item Запишите (формулой) закон Кулона (в вакууме).
        \item Зарисуйте электрическое поле точечного положительного электрического заряда.
        \item Запишите формулу для вычисления потенциала электрического поля точечного заряда.
        \item Запишите принцип суперпозиции (правило сложения) потенциалов.
    \end{enumerate}
}

\variantsplitter

\addpersonalvariant{Егор Свистушкин}

\tasknumber{1}%
\task{%
    Два одинаковых маленьких проводящих заряженных шарика находятся на расстоянии~$r$ друг от друга.
    Заряд первого равен~$+3q$, второго~--- $+2q$.
    Шарики приводят в соприкосновение, а после опять разводят на расстояние~$4r$.
    \begin{itemize}
        \item Каким стал заряд каждого из шариков?
        \item Определите характер (притяжение или отталкивание) и силу взаимодействия шариков до и после соприкосновения.
        \item Как изменилась сила взаимодействия шариков после соприкосновения?
    \end{itemize}
}
\answer{%
    \begin{align*}
    F &= k\frac{\abs{q_1}\abs {q_2}}{\sqr{4 r}}   = k\frac{\abs{+3q} \cdot \abs{+2q}}{4^2 \cdot r^2}, \text{отталкивание}; \\
        q'_1 &= q'_2, q_1 + q_2 = q'_1 + q'_2 \implies  q'_1 = q'_2 = \frac{q_1 + q_2}2 = \frac{+3q + +2q}2 = \frac52q \implies \\
        \implies F'  &= k\frac{\abs{q'_1}\abs{q'_2}}{\sqr{4 r}}
            = k\frac{\sqr{\frac52q}}{4^2 \cdot r^2},
        \text{отталкивание}, \\
    \frac{F'}F &= \frac{\sqr{\frac52q}}{4^2 \cdot \abs{+3q} \cdot \abs{+2q}} = \frac{25}{384}.
    \end{align*}
}
\solutionspace{120pt}

\tasknumber{2}%
\task{%
    На координатной плоскости в точках $(-a; 0)$ и $(a; 0)$
    находятся заряды, соответственно, $-Q$ и $+Q$.
    Сделайте рисунок, определите величину напряжённости электрического поля
    и укажите её направление в двух точках: $(0; -a)$ и $(2a; 0)$.
}
\solutionspace{120pt}

\tasknumber{3}%
\task{%
    Напряжение между двумя точками, лежащими на одной линии напряжённости
    однородного электрического поля, равно $U = 6\,\text{кВ}$.
    Расстояние между точками $r = 30\,\text{см}$.
    Какова напряжённость этого поля?
}
\answer{%
    $
        E_x = -\frac{\Delta \varphi}{\Delta x} \implies
        E = \frac{U}{r} = \frac{6\,\text{кВ}}{30\,\text{см}} = 20{,}0\,\frac{\text{кВ}}{\text{м}}.
    $
}
\solutionspace{40pt}

\tasknumber{4}%
\task{%
    В однородном электрическом поле напряжённостью $E = 2\,\frac{\text{кВ}}{\text{м}}$
    переместили заряд $q = 10\,\text{нКл}$ в направлении силовой линии
    на $d = 2\,\text{см}$.
    Определите
    \begin{itemize}
        \item работу поля,
        \item изменение потенциальной энергии заряда.
        % \item напряжение между начальной и конечной точками перемещения.
    \end{itemize}
}
\answer{%
    \begin{align*}
    A &= F \cdot d \cdot \cos \alpha = Eq \cdot d \cdot 1 = Eqd = 2\,\frac{\text{кВ}}{\text{м}} \cdot 10\,\text{нКл} \cdot 2\,\text{см} = 0{,}4\,\text{мкДж}, \\
    \Delta E_\text{пот.} &= -A = -0{,}400\,\text{мкДж}
    \end{align*}
}
\solutionspace{80pt}

\tasknumber{5}%
\task{%
    \begin{enumerate}
        \item Запишите (формулой) закон Кулона (в вакууме).
        \item Зарисуйте электрическое поле точечного положительного электрического заряда.
        \item Запишите формулу для вычисления потенциала электрического поля точечного заряда.
        \item Запишите принцип суперпозиции (правило сложения) напряжённостей.
    \end{enumerate}
}

\variantsplitter

\addpersonalvariant{Дмитрий Соколов}

\tasknumber{1}%
\task{%
    Два одинаковых маленьких проводящих заряженных шарика находятся на расстоянии~$r$ друг от друга.
    Заряд первого равен~$+5Q$, второго~--- $+8Q$.
    Шарики приводят в соприкосновение, а после опять разводят на расстояние~$4r$.
    \begin{itemize}
        \item Каким стал заряд каждого из шариков?
        \item Определите характер (притяжение или отталкивание) и силу взаимодействия шариков до и после соприкосновения.
        \item Как изменилась сила взаимодействия шариков после соприкосновения?
    \end{itemize}
}
\answer{%
    \begin{align*}
    F &= k\frac{\abs{q_1}\abs {q_2}}{\sqr{4 r}}   = k\frac{\abs{+5Q} \cdot \abs{+8Q}}{4^2 \cdot r^2}, \text{отталкивание}; \\
        q'_1 &= q'_2, q_1 + q_2 = q'_1 + q'_2 \implies  q'_1 = q'_2 = \frac{q_1 + q_2}2 = \frac{+5Q + +8Q}2 = \frac{13}2Q \implies \\
        \implies F'  &= k\frac{\abs{q'_1}\abs{q'_2}}{\sqr{4 r}}
            = k\frac{\sqr{\frac{13}2Q}}{4^2 \cdot r^2},
        \text{отталкивание}, \\
    \frac{F'}F &= \frac{\sqr{\frac{13}2Q}}{4^2 \cdot \abs{+5Q} \cdot \abs{+8Q}} = \frac{169}{2560}.
    \end{align*}
}
\solutionspace{120pt}

\tasknumber{2}%
\task{%
    На координатной плоскости в точках $(-d; 0)$ и $(d; 0)$
    находятся заряды, соответственно, $+q$ и $-q$.
    Сделайте рисунок, определите величину напряжённости электрического поля
    и укажите её направление в двух точках: $(0; d)$ и $(2d; 0)$.
}
\solutionspace{120pt}

\tasknumber{3}%
\task{%
    Напряжение между двумя точками, лежащими на одной линии напряжённости
    однородного электрического поля, равно $U = 5\,\text{кВ}$.
    Расстояние между точками $d = 10\,\text{см}$.
    Какова напряжённость этого поля?
}
\answer{%
    $
        E_x = -\frac{\Delta \varphi}{\Delta x} \implies
        E = \frac{U}{d} = \frac{5\,\text{кВ}}{10\,\text{см}} = 50{,}0\,\frac{\text{кВ}}{\text{м}}.
    $
}
\solutionspace{40pt}

\tasknumber{4}%
\task{%
    В однородном электрическом поле напряжённостью $E = 2\,\frac{\text{кВ}}{\text{м}}$
    переместили заряд $Q = 40\,\text{нКл}$ в направлении силовой линии
    на $d = 10\,\text{см}$.
    Определите
    \begin{itemize}
        \item работу поля,
        \item изменение потенциальной энергии заряда.
        % \item напряжение между начальной и конечной точками перемещения.
    \end{itemize}
}
\answer{%
    \begin{align*}
    A &= F \cdot d \cdot \cos \alpha = EQ \cdot d \cdot 1 = EQd = 2\,\frac{\text{кВ}}{\text{м}} \cdot 40\,\text{нКл} \cdot 10\,\text{см} = 8{,}0\,\text{мкДж}, \\
    \Delta E_\text{пот.} &= -A = -8{,}00\,\text{мкДж}
    \end{align*}
}
\solutionspace{80pt}

\tasknumber{5}%
\task{%
    \begin{enumerate}
        \item Запишите (формулой) закон сохранения электрического заряда.
        \item Зарисуйте электрическое поле точечного положительного электрического заряда.
        \item Запишите формулу для вычисления потенциала электрического поля точечного заряда.
        \item Запишите принцип суперпозиции (правило сложения) потенциалов.
    \end{enumerate}
}

\variantsplitter

\addpersonalvariant{Арсений Трофимов}

\tasknumber{1}%
\task{%
    Два одинаковых маленьких проводящих заряженных шарика находятся на расстоянии~$l$ друг от друга.
    Заряд первого равен~$+7Q$, второго~--- $+8Q$.
    Шарики приводят в соприкосновение, а после опять разводят на расстояние~$2l$.
    \begin{itemize}
        \item Каким стал заряд каждого из шариков?
        \item Определите характер (притяжение или отталкивание) и силу взаимодействия шариков до и после соприкосновения.
        \item Как изменилась сила взаимодействия шариков после соприкосновения?
    \end{itemize}
}
\answer{%
    \begin{align*}
    F &= k\frac{\abs{q_1}\abs {q_2}}{\sqr{2 l}}   = k\frac{\abs{+7Q} \cdot \abs{+8Q}}{2^2 \cdot l^2}, \text{отталкивание}; \\
        q'_1 &= q'_2, q_1 + q_2 = q'_1 + q'_2 \implies  q'_1 = q'_2 = \frac{q_1 + q_2}2 = \frac{+7Q + +8Q}2 = \frac{15}2Q \implies \\
        \implies F'  &= k\frac{\abs{q'_1}\abs{q'_2}}{\sqr{2 l}}
            = k\frac{\sqr{\frac{15}2Q}}{2^2 \cdot l^2},
        \text{отталкивание}, \\
    \frac{F'}F &= \frac{\sqr{\frac{15}2Q}}{2^2 \cdot \abs{+7Q} \cdot \abs{+8Q}} = \frac{225}{896}.
    \end{align*}
}
\solutionspace{120pt}

\tasknumber{2}%
\task{%
    На координатной плоскости в точках $(-d; 0)$ и $(d; 0)$
    находятся заряды, соответственно, $-Q$ и $+Q$.
    Сделайте рисунок, определите величину напряжённости электрического поля
    и укажите её направление в двух точках: $(0; -d)$ и $(2d; 0)$.
}
\solutionspace{120pt}

\tasknumber{3}%
\task{%
    Напряжение между двумя точками, лежащими на одной линии напряжённости
    однородного электрического поля, равно $V = 2\,\text{кВ}$.
    Расстояние между точками $r = 20\,\text{см}$.
    Какова напряжённость этого поля?
}
\answer{%
    $
        E_x = -\frac{\Delta \varphi}{\Delta x} \implies
        E = \frac{V}{r} = \frac{2\,\text{кВ}}{20\,\text{см}} = 10{,}0\,\frac{\text{кВ}}{\text{м}}.
    $
}
\solutionspace{40pt}

\tasknumber{4}%
\task{%
    В однородном электрическом поле напряжённостью $E = 20\,\frac{\text{кВ}}{\text{м}}$
    переместили заряд $Q = -40\,\text{нКл}$ в направлении силовой линии
    на $d = 10\,\text{см}$.
    Определите
    \begin{itemize}
        \item работу поля,
        \item изменение потенциальной энергии заряда.
        % \item напряжение между начальной и конечной точками перемещения.
    \end{itemize}
}
\answer{%
    \begin{align*}
    A &= F \cdot d \cdot \cos \alpha = EQ \cdot d \cdot 1 = EQd = 20\,\frac{\text{кВ}}{\text{м}} \cdot -40\,\text{нКл} \cdot 10\,\text{см} = -80{,}00\,\text{мкДж}, \\
    \Delta E_\text{пот.} &= -A = 80{,}0\,\text{мкДж}
    \end{align*}
}
\solutionspace{80pt}

\tasknumber{5}%
\task{%
    \begin{enumerate}
        \item Запишите (формулой) закон Кулона (в вакууме).
        \item Зарисуйте электрическое поле точечного отрицательного электрического заряда.
        \item Запишите формулу для вычисления потенциала электрического поля точечного заряда.
        \item Запишите принцип суперпозиции (правило сложения) потенциалов.
    \end{enumerate}
}
% autogenerated
