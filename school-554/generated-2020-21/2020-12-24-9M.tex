\setdate{24~декабря~2020}
\setclass{9«М»}

\addpersonalvariant{Михаил Бурмистров}

\tasknumber{1}%
\task{%
    Дайте определения:
    \begin{itemize}
        \item гармонические колебания,
        \item вынужденные колебания,
        \item незатухающие колебания,
        \item амплитуда колебаний.
    \end{itemize}
}
\solutionspace{120pt}

\tasknumber{2}%
\task{%
    Определите частоту колебаний, если их период составляет $T = 20\,\text{мс}$.
}
\answer{%
    $\nu = \frac 1T = \frac 1{20\,\text{мс}} = 50\,\text{Гц}$
}
\solutionspace{40pt}

\tasknumber{3}%
\task{%
    Определите период колебаний, если их частота составляет $\nu = 10\,\text{кГц}$.
    Сколько колебаний произойдёт за $t = 5\,\text{мин}$?
}
\answer{%
    \begin{align*}
    T &= \frac 1\nu = \frac 1{10\,\text{кГц}} = 0{,}100\,\text{мc}, \\
    N &= \nu t = 10\,\text{кГц} \cdot5\,\text{мин} = 3000000\,\text{колебаний}.
    \end{align*}
}
\solutionspace{40pt}

\tasknumber{4}%
\task{%
    Амплитуда колебаний точки составляет $A = 15\,\text{см}$, а частота~--- $\nu = 20\,\text{Гц}$.
    Определите, какой путь преодолеет эта точка за $t = 40\,\text{с}$.
}
\answer{%
    $s = 4A \cdot N = 4A \cdot \frac tT = 4A \cdot t\nu = 4 \cdot 15\,\text{см} \cdot 40\,\text{с} \cdot 20\,\text{Гц} = 480{,}0\,\text{м}$
}
\solutionspace{120pt}

\tasknumber{5}%
\task{%
    Изобразите график гармонических колебаний,
    амплитуда которых составляла бы $A = 75\,\text{см}$, а период $T = 6\,\text{с}$.
}

\variantsplitter

\addpersonalvariant{Артём Глембо}

\tasknumber{1}%
\task{%
    Дайте определения:
    \begin{itemize}
        \item механические колебания,
        \item свободные колебания,
        \item незатухающие колебания,
        \item период колебаний.
    \end{itemize}
}
\solutionspace{120pt}

\tasknumber{2}%
\task{%
    Определите частоту колебаний, если их период составляет $T = 20\,\text{мс}$.
}
\answer{%
    $\nu = \frac 1T = \frac 1{20\,\text{мс}} = 50\,\text{Гц}$
}
\solutionspace{40pt}

\tasknumber{3}%
\task{%
    Определите период колебаний, если их частота составляет $\nu = 4\,\text{кГц}$.
    Сколько колебаний произойдёт за $t = 3\,\text{мин}$?
}
\answer{%
    \begin{align*}
    T &= \frac 1\nu = \frac 1{4\,\text{кГц}} = 0{,}250\,\text{мc}, \\
    N &= \nu t = 4\,\text{кГц} \cdot3\,\text{мин} = 720000\,\text{колебаний}.
    \end{align*}
}
\solutionspace{40pt}

\tasknumber{4}%
\task{%
    Амплитуда колебаний точки составляет $A = 5\,\text{см}$, а частота~--- $\nu = 6\,\text{Гц}$.
    Определите, какой путь преодолеет эта точка за $t = 80\,\text{с}$.
}
\answer{%
    $s = 4A \cdot N = 4A \cdot \frac tT = 4A \cdot t\nu = 4 \cdot 5\,\text{см} \cdot 80\,\text{с} \cdot 6\,\text{Гц} = 96{,}0\,\text{м}$
}
\solutionspace{120pt}

\tasknumber{5}%
\task{%
    Изобразите график гармонических колебаний,
    амплитуда которых составляла бы $A = 75\,\text{см}$, а период $T = 10\,\text{с}$.
}

\variantsplitter

\addpersonalvariant{Наталья Гончарова}

\tasknumber{1}%
\task{%
    Дайте определения:
    \begin{itemize}
        \item гармонические колебания,
        \item вынужденные колебания,
        \item незатухающие колебания,
        \item амплитуда колебаний.
    \end{itemize}
}
\solutionspace{120pt}

\tasknumber{2}%
\task{%
    Определите частоту колебаний, если их период составляет $T = 10\,\text{мс}$.
}
\answer{%
    $\nu = \frac 1T = \frac 1{10\,\text{мс}} = 100\,\text{Гц}$
}
\solutionspace{40pt}

\tasknumber{3}%
\task{%
    Определите период колебаний, если их частота составляет $\nu = 5\,\text{кГц}$.
    Сколько колебаний произойдёт за $t = 3\,\text{мин}$?
}
\answer{%
    \begin{align*}
    T &= \frac 1\nu = \frac 1{5\,\text{кГц}} = 0{,}200\,\text{мc}, \\
    N &= \nu t = 5\,\text{кГц} \cdot3\,\text{мин} = 900000\,\text{колебаний}.
    \end{align*}
}
\solutionspace{40pt}

\tasknumber{4}%
\task{%
    Амплитуда колебаний точки составляет $A = 10\,\text{см}$, а частота~--- $\nu = 2\,\text{Гц}$.
    Определите, какой путь преодолеет эта точка за $t = 80\,\text{с}$.
}
\answer{%
    $s = 4A \cdot N = 4A \cdot \frac tT = 4A \cdot t\nu = 4 \cdot 10\,\text{см} \cdot 80\,\text{с} \cdot 2\,\text{Гц} = 64{,}0\,\text{м}$
}
\solutionspace{120pt}

\tasknumber{5}%
\task{%
    Изобразите график гармонических колебаний,
    амплитуда которых составляла бы $A = 40\,\text{см}$, а период $T = 10\,\text{с}$.
}

\variantsplitter

\addpersonalvariant{Файёзбек Касымов}

\tasknumber{1}%
\task{%
    Дайте определения:
    \begin{itemize}
        \item механические колебания,
        \item вынужденные колебания,
        \item незатухающие колебания,
        \item амплитуда колебаний.
    \end{itemize}
}
\solutionspace{120pt}

\tasknumber{2}%
\task{%
    Определите частоту колебаний, если их период составляет $T = 10\,\text{мс}$.
}
\answer{%
    $\nu = \frac 1T = \frac 1{10\,\text{мс}} = 100\,\text{Гц}$
}
\solutionspace{40pt}

\tasknumber{3}%
\task{%
    Определите период колебаний, если их частота составляет $\nu = 2\,\text{кГц}$.
    Сколько колебаний произойдёт за $t = 1\,\text{мин}$?
}
\answer{%
    \begin{align*}
    T &= \frac 1\nu = \frac 1{2\,\text{кГц}} = 0{,}500\,\text{мc}, \\
    N &= \nu t = 2\,\text{кГц} \cdot1\,\text{мин} = 120000\,\text{колебаний}.
    \end{align*}
}
\solutionspace{40pt}

\tasknumber{4}%
\task{%
    Амплитуда колебаний точки составляет $A = 2\,\text{см}$, а частота~--- $\nu = 20\,\text{Гц}$.
    Определите, какой путь преодолеет эта точка за $t = 40\,\text{с}$.
}
\answer{%
    $s = 4A \cdot N = 4A \cdot \frac tT = 4A \cdot t\nu = 4 \cdot 2\,\text{см} \cdot 40\,\text{с} \cdot 20\,\text{Гц} = 64{,}0\,\text{м}$
}
\solutionspace{120pt}

\tasknumber{5}%
\task{%
    Изобразите график гармонических колебаний,
    амплитуда которых составляла бы $A = 75\,\text{см}$, а период $T = 10\,\text{с}$.
}

\variantsplitter

\addpersonalvariant{Александр Козинец}

\tasknumber{1}%
\task{%
    Дайте определения:
    \begin{itemize}
        \item механические колебания,
        \item свободные колебания,
        \item затухающие колебания,
        \item амплитуда колебаний.
    \end{itemize}
}
\solutionspace{120pt}

\tasknumber{2}%
\task{%
    Определите частоту колебаний, если их период составляет $T = 4\,\text{мс}$.
}
\answer{%
    $\nu = \frac 1T = \frac 1{4\,\text{мс}} = 250\,\text{Гц}$
}
\solutionspace{40pt}

\tasknumber{3}%
\task{%
    Определите период колебаний, если их частота составляет $\nu = 50\,\text{кГц}$.
    Сколько колебаний произойдёт за $t = 10\,\text{мин}$?
}
\answer{%
    \begin{align*}
    T &= \frac 1\nu = \frac 1{50\,\text{кГц}} = 0{,}020\,\text{мc}, \\
    N &= \nu t = 50\,\text{кГц} \cdot10\,\text{мин} = 30000000\,\text{колебаний}.
    \end{align*}
}
\solutionspace{40pt}

\tasknumber{4}%
\task{%
    Амплитуда колебаний точки составляет $A = 5\,\text{см}$, а частота~--- $\nu = 20\,\text{Гц}$.
    Определите, какой путь преодолеет эта точка за $t = 10\,\text{с}$.
}
\answer{%
    $s = 4A \cdot N = 4A \cdot \frac tT = 4A \cdot t\nu = 4 \cdot 5\,\text{см} \cdot 10\,\text{с} \cdot 20\,\text{Гц} = 40{,}0\,\text{м}$
}
\solutionspace{120pt}

\tasknumber{5}%
\task{%
    Изобразите график гармонических колебаний,
    амплитуда которых составляла бы $A = 30\,\text{см}$, а период $T = 6\,\text{с}$.
}

\variantsplitter

\addpersonalvariant{Андрей Куликовский}

\tasknumber{1}%
\task{%
    Дайте определения:
    \begin{itemize}
        \item механические колебания,
        \item свободные колебания,
        \item затухающие колебания,
        \item амплитуда колебаний.
    \end{itemize}
}
\solutionspace{120pt}

\tasknumber{2}%
\task{%
    Определите частоту колебаний, если их период составляет $T = 4\,\text{мс}$.
}
\answer{%
    $\nu = \frac 1T = \frac 1{4\,\text{мс}} = 250\,\text{Гц}$
}
\solutionspace{40pt}

\tasknumber{3}%
\task{%
    Определите период колебаний, если их частота составляет $\nu = 4\,\text{кГц}$.
    Сколько колебаний произойдёт за $t = 5\,\text{мин}$?
}
\answer{%
    \begin{align*}
    T &= \frac 1\nu = \frac 1{4\,\text{кГц}} = 0{,}250\,\text{мc}, \\
    N &= \nu t = 4\,\text{кГц} \cdot5\,\text{мин} = 1200000\,\text{колебаний}.
    \end{align*}
}
\solutionspace{40pt}

\tasknumber{4}%
\task{%
    Амплитуда колебаний точки составляет $A = 2\,\text{см}$, а частота~--- $\nu = 5\,\text{Гц}$.
    Определите, какой путь преодолеет эта точка за $t = 40\,\text{с}$.
}
\answer{%
    $s = 4A \cdot N = 4A \cdot \frac tT = 4A \cdot t\nu = 4 \cdot 2\,\text{см} \cdot 40\,\text{с} \cdot 5\,\text{Гц} = 16{,}0\,\text{м}$
}
\solutionspace{120pt}

\tasknumber{5}%
\task{%
    Изобразите график гармонических колебаний,
    амплитуда которых составляла бы $A = 40\,\text{см}$, а период $T = 2\,\text{с}$.
}

\variantsplitter

\addpersonalvariant{Полина Лоткова}

\tasknumber{1}%
\task{%
    Дайте определения:
    \begin{itemize}
        \item гармонические колебания,
        \item свободные колебания,
        \item незатухающие колебания,
        \item амплитуда колебаний.
    \end{itemize}
}
\solutionspace{120pt}

\tasknumber{2}%
\task{%
    Определите частоту колебаний, если их период составляет $T = 50\,\text{мс}$.
}
\answer{%
    $\nu = \frac 1T = \frac 1{50\,\text{мс}} = 20\,\text{Гц}$
}
\solutionspace{40pt}

\tasknumber{3}%
\task{%
    Определите период колебаний, если их частота составляет $\nu = 50\,\text{кГц}$.
    Сколько колебаний произойдёт за $t = 10\,\text{мин}$?
}
\answer{%
    \begin{align*}
    T &= \frac 1\nu = \frac 1{50\,\text{кГц}} = 0{,}020\,\text{мc}, \\
    N &= \nu t = 50\,\text{кГц} \cdot10\,\text{мин} = 30000000\,\text{колебаний}.
    \end{align*}
}
\solutionspace{40pt}

\tasknumber{4}%
\task{%
    Амплитуда колебаний точки составляет $A = 3\,\text{см}$, а частота~--- $\nu = 20\,\text{Гц}$.
    Определите, какой путь преодолеет эта точка за $t = 40\,\text{с}$.
}
\answer{%
    $s = 4A \cdot N = 4A \cdot \frac tT = 4A \cdot t\nu = 4 \cdot 3\,\text{см} \cdot 40\,\text{с} \cdot 20\,\text{Гц} = 96{,}0\,\text{м}$
}
\solutionspace{120pt}

\tasknumber{5}%
\task{%
    Изобразите график гармонических колебаний,
    амплитуда которых составляла бы $A = 40\,\text{см}$, а период $T = 2\,\text{с}$.
}

\variantsplitter

\addpersonalvariant{Екатерина Медведева}

\tasknumber{1}%
\task{%
    Дайте определения:
    \begin{itemize}
        \item механические колебания,
        \item свободные колебания,
        \item незатухающие колебания,
        \item период колебаний.
    \end{itemize}
}
\solutionspace{120pt}

\tasknumber{2}%
\task{%
    Определите частоту колебаний, если их период составляет $T = 10\,\text{мс}$.
}
\answer{%
    $\nu = \frac 1T = \frac 1{10\,\text{мс}} = 100\,\text{Гц}$
}
\solutionspace{40pt}

\tasknumber{3}%
\task{%
    Определите период колебаний, если их частота составляет $\nu = 5\,\text{кГц}$.
    Сколько колебаний произойдёт за $t = 2\,\text{мин}$?
}
\answer{%
    \begin{align*}
    T &= \frac 1\nu = \frac 1{5\,\text{кГц}} = 0{,}200\,\text{мc}, \\
    N &= \nu t = 5\,\text{кГц} \cdot2\,\text{мин} = 600000\,\text{колебаний}.
    \end{align*}
}
\solutionspace{40pt}

\tasknumber{4}%
\task{%
    Амплитуда колебаний точки составляет $A = 2\,\text{см}$, а частота~--- $\nu = 5\,\text{Гц}$.
    Определите, какой путь преодолеет эта точка за $t = 40\,\text{с}$.
}
\answer{%
    $s = 4A \cdot N = 4A \cdot \frac tT = 4A \cdot t\nu = 4 \cdot 2\,\text{см} \cdot 40\,\text{с} \cdot 5\,\text{Гц} = 16{,}0\,\text{м}$
}
\solutionspace{120pt}

\tasknumber{5}%
\task{%
    Изобразите график гармонических колебаний,
    амплитуда которых составляла бы $A = 15\,\text{см}$, а период $T = 4\,\text{с}$.
}

\variantsplitter

\addpersonalvariant{Константин Мельник}

\tasknumber{1}%
\task{%
    Дайте определения:
    \begin{itemize}
        \item механические колебания,
        \item вынужденные колебания,
        \item затухающие колебания,
        \item период колебаний.
    \end{itemize}
}
\solutionspace{120pt}

\tasknumber{2}%
\task{%
    Определите частоту колебаний, если их период составляет $T = 40\,\text{мс}$.
}
\answer{%
    $\nu = \frac 1T = \frac 1{40\,\text{мс}} = 25\,\text{Гц}$
}
\solutionspace{40pt}

\tasknumber{3}%
\task{%
    Определите период колебаний, если их частота составляет $\nu = 20\,\text{кГц}$.
    Сколько колебаний произойдёт за $t = 5\,\text{мин}$?
}
\answer{%
    \begin{align*}
    T &= \frac 1\nu = \frac 1{20\,\text{кГц}} = 0{,}050\,\text{мc}, \\
    N &= \nu t = 20\,\text{кГц} \cdot5\,\text{мин} = 6000000\,\text{колебаний}.
    \end{align*}
}
\solutionspace{40pt}

\tasknumber{4}%
\task{%
    Амплитуда колебаний точки составляет $A = 2\,\text{см}$, а частота~--- $\nu = 2\,\text{Гц}$.
    Определите, какой путь преодолеет эта точка за $t = 10\,\text{с}$.
}
\answer{%
    $s = 4A \cdot N = 4A \cdot \frac tT = 4A \cdot t\nu = 4 \cdot 2\,\text{см} \cdot 10\,\text{с} \cdot 2\,\text{Гц} = 1{,}6\,\text{м}$
}
\solutionspace{120pt}

\tasknumber{5}%
\task{%
    Изобразите график гармонических колебаний,
    амплитуда которых составляла бы $A = 30\,\text{см}$, а период $T = 10\,\text{с}$.
}

\variantsplitter

\addpersonalvariant{Степан Небоваренков}

\tasknumber{1}%
\task{%
    Дайте определения:
    \begin{itemize}
        \item гармонические колебания,
        \item свободные колебания,
        \item незатухающие колебания,
        \item период колебаний.
    \end{itemize}
}
\solutionspace{120pt}

\tasknumber{2}%
\task{%
    Определите частоту колебаний, если их период составляет $T = 40\,\text{мс}$.
}
\answer{%
    $\nu = \frac 1T = \frac 1{40\,\text{мс}} = 25\,\text{Гц}$
}
\solutionspace{40pt}

\tasknumber{3}%
\task{%
    Определите период колебаний, если их частота составляет $\nu = 40\,\text{кГц}$.
    Сколько колебаний произойдёт за $t = 1\,\text{мин}$?
}
\answer{%
    \begin{align*}
    T &= \frac 1\nu = \frac 1{40\,\text{кГц}} = 0{,}025\,\text{мc}, \\
    N &= \nu t = 40\,\text{кГц} \cdot1\,\text{мин} = 2400000\,\text{колебаний}.
    \end{align*}
}
\solutionspace{40pt}

\tasknumber{4}%
\task{%
    Амплитуда колебаний точки составляет $A = 10\,\text{см}$, а частота~--- $\nu = 6\,\text{Гц}$.
    Определите, какой путь преодолеет эта точка за $t = 80\,\text{с}$.
}
\answer{%
    $s = 4A \cdot N = 4A \cdot \frac tT = 4A \cdot t\nu = 4 \cdot 10\,\text{см} \cdot 80\,\text{с} \cdot 6\,\text{Гц} = 192{,}0\,\text{м}$
}
\solutionspace{120pt}

\tasknumber{5}%
\task{%
    Изобразите график гармонических колебаний,
    амплитуда которых составляла бы $A = 2\,\text{см}$, а период $T = 10\,\text{с}$.
}

\variantsplitter

\addpersonalvariant{Матвей Неретин}

\tasknumber{1}%
\task{%
    Дайте определения:
    \begin{itemize}
        \item механические колебания,
        \item вынужденные колебания,
        \item незатухающие колебания,
        \item период колебаний.
    \end{itemize}
}
\solutionspace{120pt}

\tasknumber{2}%
\task{%
    Определите частоту колебаний, если их период составляет $T = 10\,\text{мс}$.
}
\answer{%
    $\nu = \frac 1T = \frac 1{10\,\text{мс}} = 100\,\text{Гц}$
}
\solutionspace{40pt}

\tasknumber{3}%
\task{%
    Определите период колебаний, если их частота составляет $\nu = 5\,\text{кГц}$.
    Сколько колебаний произойдёт за $t = 10\,\text{мин}$?
}
\answer{%
    \begin{align*}
    T &= \frac 1\nu = \frac 1{5\,\text{кГц}} = 0{,}200\,\text{мc}, \\
    N &= \nu t = 5\,\text{кГц} \cdot10\,\text{мин} = 3000000\,\text{колебаний}.
    \end{align*}
}
\solutionspace{40pt}

\tasknumber{4}%
\task{%
    Амплитуда колебаний точки составляет $A = 3\,\text{см}$, а частота~--- $\nu = 10\,\text{Гц}$.
    Определите, какой путь преодолеет эта точка за $t = 40\,\text{с}$.
}
\answer{%
    $s = 4A \cdot N = 4A \cdot \frac tT = 4A \cdot t\nu = 4 \cdot 3\,\text{см} \cdot 40\,\text{с} \cdot 10\,\text{Гц} = 48{,}0\,\text{м}$
}
\solutionspace{120pt}

\tasknumber{5}%
\task{%
    Изобразите график гармонических колебаний,
    амплитуда которых составляла бы $A = 40\,\text{см}$, а период $T = 10\,\text{с}$.
}

\variantsplitter

\addpersonalvariant{Мария Никонова}

\tasknumber{1}%
\task{%
    Дайте определения:
    \begin{itemize}
        \item гармонические колебания,
        \item свободные колебания,
        \item затухающие колебания,
        \item амплитуда колебаний.
    \end{itemize}
}
\solutionspace{120pt}

\tasknumber{2}%
\task{%
    Определите частоту колебаний, если их период составляет $T = 20\,\text{мс}$.
}
\answer{%
    $\nu = \frac 1T = \frac 1{20\,\text{мс}} = 50\,\text{Гц}$
}
\solutionspace{40pt}

\tasknumber{3}%
\task{%
    Определите период колебаний, если их частота составляет $\nu = 40\,\text{кГц}$.
    Сколько колебаний произойдёт за $t = 2\,\text{мин}$?
}
\answer{%
    \begin{align*}
    T &= \frac 1\nu = \frac 1{40\,\text{кГц}} = 0{,}025\,\text{мc}, \\
    N &= \nu t = 40\,\text{кГц} \cdot2\,\text{мин} = 4800000\,\text{колебаний}.
    \end{align*}
}
\solutionspace{40pt}

\tasknumber{4}%
\task{%
    Амплитуда колебаний точки составляет $A = 15\,\text{см}$, а частота~--- $\nu = 2\,\text{Гц}$.
    Определите, какой путь преодолеет эта точка за $t = 40\,\text{с}$.
}
\answer{%
    $s = 4A \cdot N = 4A \cdot \frac tT = 4A \cdot t\nu = 4 \cdot 15\,\text{см} \cdot 40\,\text{с} \cdot 2\,\text{Гц} = 48{,}0\,\text{м}$
}
\solutionspace{120pt}

\tasknumber{5}%
\task{%
    Изобразите график гармонических колебаний,
    амплитуда которых составляла бы $A = 2\,\text{см}$, а период $T = 8\,\text{с}$.
}

\variantsplitter

\addpersonalvariant{Даниил Палаткин}

\tasknumber{1}%
\task{%
    Дайте определения:
    \begin{itemize}
        \item механические колебания,
        \item свободные колебания,
        \item незатухающие колебания,
        \item амплитуда колебаний.
    \end{itemize}
}
\solutionspace{120pt}

\tasknumber{2}%
\task{%
    Определите частоту колебаний, если их период составляет $T = 2\,\text{мс}$.
}
\answer{%
    $\nu = \frac 1T = \frac 1{2\,\text{мс}} = 500\,\text{Гц}$
}
\solutionspace{40pt}

\tasknumber{3}%
\task{%
    Определите период колебаний, если их частота составляет $\nu = 10\,\text{кГц}$.
    Сколько колебаний произойдёт за $t = 3\,\text{мин}$?
}
\answer{%
    \begin{align*}
    T &= \frac 1\nu = \frac 1{10\,\text{кГц}} = 0{,}100\,\text{мc}, \\
    N &= \nu t = 10\,\text{кГц} \cdot3\,\text{мин} = 1800000\,\text{колебаний}.
    \end{align*}
}
\solutionspace{40pt}

\tasknumber{4}%
\task{%
    Амплитуда колебаний точки составляет $A = 15\,\text{см}$, а частота~--- $\nu = 20\,\text{Гц}$.
    Определите, какой путь преодолеет эта точка за $t = 80\,\text{с}$.
}
\answer{%
    $s = 4A \cdot N = 4A \cdot \frac tT = 4A \cdot t\nu = 4 \cdot 15\,\text{см} \cdot 80\,\text{с} \cdot 20\,\text{Гц} = 960{,}0\,\text{м}$
}
\solutionspace{120pt}

\tasknumber{5}%
\task{%
    Изобразите график гармонических колебаний,
    амплитуда которых составляла бы $A = 1\,\text{см}$, а период $T = 10\,\text{с}$.
}

\variantsplitter

\addpersonalvariant{Станислав Пикун}

\tasknumber{1}%
\task{%
    Дайте определения:
    \begin{itemize}
        \item гармонические колебания,
        \item свободные колебания,
        \item незатухающие колебания,
        \item период колебаний.
    \end{itemize}
}
\solutionspace{120pt}

\tasknumber{2}%
\task{%
    Определите частоту колебаний, если их период составляет $T = 20\,\text{мс}$.
}
\answer{%
    $\nu = \frac 1T = \frac 1{20\,\text{мс}} = 50\,\text{Гц}$
}
\solutionspace{40pt}

\tasknumber{3}%
\task{%
    Определите период колебаний, если их частота составляет $\nu = 2\,\text{кГц}$.
    Сколько колебаний произойдёт за $t = 5\,\text{мин}$?
}
\answer{%
    \begin{align*}
    T &= \frac 1\nu = \frac 1{2\,\text{кГц}} = 0{,}500\,\text{мc}, \\
    N &= \nu t = 2\,\text{кГц} \cdot5\,\text{мин} = 600000\,\text{колебаний}.
    \end{align*}
}
\solutionspace{40pt}

\tasknumber{4}%
\task{%
    Амплитуда колебаний точки составляет $A = 5\,\text{см}$, а частота~--- $\nu = 10\,\text{Гц}$.
    Определите, какой путь преодолеет эта точка за $t = 80\,\text{с}$.
}
\answer{%
    $s = 4A \cdot N = 4A \cdot \frac tT = 4A \cdot t\nu = 4 \cdot 5\,\text{см} \cdot 80\,\text{с} \cdot 10\,\text{Гц} = 160{,}0\,\text{м}$
}
\solutionspace{120pt}

\tasknumber{5}%
\task{%
    Изобразите график гармонических колебаний,
    амплитуда которых составляла бы $A = 5\,\text{см}$, а период $T = 10\,\text{с}$.
}

\variantsplitter

\addpersonalvariant{Илья Пичугин}

\tasknumber{1}%
\task{%
    Дайте определения:
    \begin{itemize}
        \item механические колебания,
        \item свободные колебания,
        \item затухающие колебания,
        \item период колебаний.
    \end{itemize}
}
\solutionspace{120pt}

\tasknumber{2}%
\task{%
    Определите частоту колебаний, если их период составляет $T = 10\,\text{мс}$.
}
\answer{%
    $\nu = \frac 1T = \frac 1{10\,\text{мс}} = 100\,\text{Гц}$
}
\solutionspace{40pt}

\tasknumber{3}%
\task{%
    Определите период колебаний, если их частота составляет $\nu = 50\,\text{кГц}$.
    Сколько колебаний произойдёт за $t = 1\,\text{мин}$?
}
\answer{%
    \begin{align*}
    T &= \frac 1\nu = \frac 1{50\,\text{кГц}} = 0{,}020\,\text{мc}, \\
    N &= \nu t = 50\,\text{кГц} \cdot1\,\text{мин} = 3000000\,\text{колебаний}.
    \end{align*}
}
\solutionspace{40pt}

\tasknumber{4}%
\task{%
    Амплитуда колебаний точки составляет $A = 5\,\text{см}$, а частота~--- $\nu = 5\,\text{Гц}$.
    Определите, какой путь преодолеет эта точка за $t = 40\,\text{с}$.
}
\answer{%
    $s = 4A \cdot N = 4A \cdot \frac tT = 4A \cdot t\nu = 4 \cdot 5\,\text{см} \cdot 40\,\text{с} \cdot 5\,\text{Гц} = 40{,}0\,\text{м}$
}
\solutionspace{120pt}

\tasknumber{5}%
\task{%
    Изобразите график гармонических колебаний,
    амплитуда которых составляла бы $A = 40\,\text{см}$, а период $T = 8\,\text{с}$.
}

\variantsplitter

\addpersonalvariant{Кирилл Севрюгин}

\tasknumber{1}%
\task{%
    Дайте определения:
    \begin{itemize}
        \item механические колебания,
        \item вынужденные колебания,
        \item незатухающие колебания,
        \item период колебаний.
    \end{itemize}
}
\solutionspace{120pt}

\tasknumber{2}%
\task{%
    Определите частоту колебаний, если их период составляет $T = 20\,\text{мс}$.
}
\answer{%
    $\nu = \frac 1T = \frac 1{20\,\text{мс}} = 50\,\text{Гц}$
}
\solutionspace{40pt}

\tasknumber{3}%
\task{%
    Определите период колебаний, если их частота составляет $\nu = 4\,\text{кГц}$.
    Сколько колебаний произойдёт за $t = 5\,\text{мин}$?
}
\answer{%
    \begin{align*}
    T &= \frac 1\nu = \frac 1{4\,\text{кГц}} = 0{,}250\,\text{мc}, \\
    N &= \nu t = 4\,\text{кГц} \cdot5\,\text{мин} = 1200000\,\text{колебаний}.
    \end{align*}
}
\solutionspace{40pt}

\tasknumber{4}%
\task{%
    Амплитуда колебаний точки составляет $A = 10\,\text{см}$, а частота~--- $\nu = 2\,\text{Гц}$.
    Определите, какой путь преодолеет эта точка за $t = 40\,\text{с}$.
}
\answer{%
    $s = 4A \cdot N = 4A \cdot \frac tT = 4A \cdot t\nu = 4 \cdot 10\,\text{см} \cdot 40\,\text{с} \cdot 2\,\text{Гц} = 32{,}0\,\text{м}$
}
\solutionspace{120pt}

\tasknumber{5}%
\task{%
    Изобразите график гармонических колебаний,
    амплитуда которых составляла бы $A = 15\,\text{см}$, а период $T = 2\,\text{с}$.
}

\variantsplitter

\addpersonalvariant{Илья Стратонников}

\tasknumber{1}%
\task{%
    Дайте определения:
    \begin{itemize}
        \item механические колебания,
        \item вынужденные колебания,
        \item незатухающие колебания,
        \item амплитуда колебаний.
    \end{itemize}
}
\solutionspace{120pt}

\tasknumber{2}%
\task{%
    Определите частоту колебаний, если их период составляет $T = 2\,\text{мс}$.
}
\answer{%
    $\nu = \frac 1T = \frac 1{2\,\text{мс}} = 500\,\text{Гц}$
}
\solutionspace{40pt}

\tasknumber{3}%
\task{%
    Определите период колебаний, если их частота составляет $\nu = 50\,\text{кГц}$.
    Сколько колебаний произойдёт за $t = 10\,\text{мин}$?
}
\answer{%
    \begin{align*}
    T &= \frac 1\nu = \frac 1{50\,\text{кГц}} = 0{,}020\,\text{мc}, \\
    N &= \nu t = 50\,\text{кГц} \cdot10\,\text{мин} = 30000000\,\text{колебаний}.
    \end{align*}
}
\solutionspace{40pt}

\tasknumber{4}%
\task{%
    Амплитуда колебаний точки составляет $A = 5\,\text{см}$, а частота~--- $\nu = 2\,\text{Гц}$.
    Определите, какой путь преодолеет эта точка за $t = 10\,\text{с}$.
}
\answer{%
    $s = 4A \cdot N = 4A \cdot \frac tT = 4A \cdot t\nu = 4 \cdot 5\,\text{см} \cdot 10\,\text{с} \cdot 2\,\text{Гц} = 4{,}0\,\text{м}$
}
\solutionspace{120pt}

\tasknumber{5}%
\task{%
    Изобразите график гармонических колебаний,
    амплитуда которых составляла бы $A = 1\,\text{см}$, а период $T = 4\,\text{с}$.
}

\variantsplitter

\addpersonalvariant{Иван Шустов}

\tasknumber{1}%
\task{%
    Дайте определения:
    \begin{itemize}
        \item гармонические колебания,
        \item свободные колебания,
        \item затухающие колебания,
        \item амплитуда колебаний.
    \end{itemize}
}
\solutionspace{120pt}

\tasknumber{2}%
\task{%
    Определите частоту колебаний, если их период составляет $T = 5\,\text{мс}$.
}
\answer{%
    $\nu = \frac 1T = \frac 1{5\,\text{мс}} = 200\,\text{Гц}$
}
\solutionspace{40pt}

\tasknumber{3}%
\task{%
    Определите период колебаний, если их частота составляет $\nu = 5\,\text{кГц}$.
    Сколько колебаний произойдёт за $t = 5\,\text{мин}$?
}
\answer{%
    \begin{align*}
    T &= \frac 1\nu = \frac 1{5\,\text{кГц}} = 0{,}200\,\text{мc}, \\
    N &= \nu t = 5\,\text{кГц} \cdot5\,\text{мин} = 1500000\,\text{колебаний}.
    \end{align*}
}
\solutionspace{40pt}

\tasknumber{4}%
\task{%
    Амплитуда колебаний точки составляет $A = 2\,\text{см}$, а частота~--- $\nu = 2\,\text{Гц}$.
    Определите, какой путь преодолеет эта точка за $t = 40\,\text{с}$.
}
\answer{%
    $s = 4A \cdot N = 4A \cdot \frac tT = 4A \cdot t\nu = 4 \cdot 2\,\text{см} \cdot 40\,\text{с} \cdot 2\,\text{Гц} = 6{,}4\,\text{м}$
}
\solutionspace{120pt}

\tasknumber{5}%
\task{%
    Изобразите график гармонических колебаний,
    амплитуда которых составляла бы $A = 2\,\text{см}$, а период $T = 10\,\text{с}$.
}
% autogenerated
