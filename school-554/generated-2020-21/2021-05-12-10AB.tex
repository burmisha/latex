\setdate{12~мая~2021}
\setclass{10«АБ»}

\addpersonalvariant{Михаил Бурмистров}

\tasknumber{1}%
\task{%
    Определите эквивалентное сопротивление цепи на рисунке (между выделенными на рисунке контактами),
    если известны сопротивления всех резисторов: $R_1 = 1\,\text{Ом}$, $R_2 = 4\,\text{Ом}$, $R_3 = 2\,\text{Ом}$, $R_4 = 2\,\text{Ом}$.
    При каком напряжении поданном на эту цепь, в ней потечёт ток равный $\eli = 10\,\text{А}$?

    \begin{tikzpicture}[rotate=180, circuit ee IEC, thick]
        \node [contact]  (contact1) at (-1.5, 0) {};
        \draw  (0, 0) to [resistor={info=$R_1$}] ++(left:1.5);
        \draw  (0, 0) -- ++(up:1.5) to [resistor={near start, info=$R_2$}, resistor={near end, info=$R_3$}] ++(right:3);
        \draw  (0, 0) to [resistor={info=$R_4$}] ++(right:3) -- ++(up:1.5);
        \draw  (3, 1.5) -- ++(right:0.5); \node [contact] (contact2) at (3.5, 1.5) {};
    \end{tikzpicture}
}
\answer{%
    $R=\frac52\units{Ом} \approx 2{,}50\,\text{Ом} \implies U = \eli R \approx 25{,}0\,\text{В}.$
}
\solutionspace{120pt}

\tasknumber{2}%
\task{%
    Определите показания амперметра $1$ (см.
    рис.) и разность потенциалов на резисторе $3$,
    если сопротивления всех резисторов равны: $R_1 = R_2 = R_3 = R_4 = R_5 = R_6 = R = 5\,\text{Ом}$,
    а напряжение, поданное на цепь, равно $U = 120\,\text{В}$.
    Ответы получите в виде несократимых дробей, а также определите приближённые значения.
    Амперметры считать идеальными.

    \begin{tikzpicture}[circuit ee IEC, thick]
        \node [contact]  (left contact) at (3, 0) {};
        \node [contact]  (right contact) at (9, 0) {};
        \draw  (left contact) -- ++(up:2) to [resistor={very near start, info=$R_2$}, amperemeter={midway, info=$1$}, resistor={very near end, info=$R_3$} ] ++(right:6) -- (right contact);
        \draw  (left contact) -- ++(down:2) to [resistor={very near start, info=$R_5$}, resistor={midway, info=$R_6$}, amperemeter={very near end, info=$3$}] ++(right:6) -- (right contact);
        \draw  (left contact) ++(left:3) to [resistor={info=$R_1$}] (left contact) to [amperemeter={near start, info=$2$}, resistor={near end , info=$R_4$}] (right contact) -- ++(right:0.5);
    \end{tikzpicture}
}
\answer{%
    \begin{align*}
    R_0 &= R + \frac 1{\frac 1{R+R} + \frac 1R + \frac 1{R+R}} = R + \frac 1{\frac 2R} = \frac 32 R, \\
    \eli &= \frac U{R_0} = \frac {2U}{3R}, \\
    U_1 &= \eli R_1 = \frac {2U}{3R} \cdot R = \frac 23 U = 80{,}0\,\text{В}, \\
    U_{23} &= U_{56} = U_4 = U - \eli R_1 = U - \frac {2U}{3R} \cdot R = \frac U3 = 40{,}0\,\text{В}, \\
    \eli_2 &= \frac{U_4}{R_4} = \frac U{3R} \approx 8{,}0\,\text{А}, \\
    \eli_1 &= \frac{U_{23}}{R_{23}} = \frac{\frac U3}{R+R} = \frac U{6R} \approx 4{,}0\,\text{А}, \\
    \eli_3 &= \frac{U_{56}}{R_{56}} = \frac{\frac U3}{R+R} = \frac U{6R} \approx 4{,}0\,\text{А}, \\
    U_2 &= \eli_1 R_2 = \frac U{6R} \cdot R = \frac U6 = 20{,}0\,\text{В}, \\
    U_3 &= \eli_1 R_3 = \frac U{6R} \cdot R = \frac U3 = 20{,}0\,\text{В}, \\
    U_5 &= \eli_3 R_5 = \frac U{6R} \cdot R = \frac U5 = 20{,}0\,\text{В}, \\
    U_6 &= \eli_3 R_6 = \frac U{6R} \cdot R = \frac U6 = 20{,}0\,\text{В}.
    \end{align*}
}
\solutionspace{120pt}

\tasknumber{3}%
\task{%
    Определите ток, протекающий через резистор $R = 18\,\text{Ом}$ и разность потенциалов на нём (см.
    рис.),
    если $\ele_1 = 18\,\text{В}$, $\ele_2 = 25\,\text{В}$, $r_1 = 2\,\text{Ом}$, $r_2 = 6\,\text{Ом}$.

    \begin{tikzpicture}[circuit ee IEC, thick]
        \draw  (0, 0) to [battery={rotate=-180,info={$\ele_1, r_1$}}] (0, 3)
                -- (5, 3)
                to [battery={rotate=-180, info'={$\ele_2, r_2$}}] (5, 0)
                -- (0, 0)
                (2.5, 0) to [resistor={info=$R$}] (2.5, 3);
    \end{tikzpicture}
}
\answer{%
    Выберем 2 контура и один узел, запишем для них законы Кирхгофа:

    \begin{tikzpicture}[circuit ee IEC, thick]
        \draw  (0, 0) to [battery={rotate=-180,info={$\ele_1, r_1$}}, current direction={near end, info=$\eli_1$}] (0, 3)
                -- (5, 3)
                to [battery={rotate=-180, info'={$\ele_2, r_2$}}, current direction={near end, info=$\eli_2$}] (5, 0)
                -- (0, 0)
                (2.5, 0) to [resistor={info=$R$}, current direction'={near end, info=$\eli$}] (2.5, 3);
        \draw [-{Latex},color=red] (0.8, 1.9) arc [start angle = 135, end angle = -160, radius = 0.6];
        \draw [-{Latex},color=blue] (3.5, 1.9) arc [start angle = 135, end angle = -160, radius = 0.6];
        \node [contact,color=green!71!black] (topc) at (2.5, 3) {};
        \node [above] (top) at (2.5, 3) {$1$};
    \end{tikzpicture}

    \begin{align*}
        &\begin{cases}
            {\color{red} \ele_1 = \eli_1 r_1 + \eli R}, \\
            {\color{blue} \ele_2 = \eli_2 r_2 - \eli R}, \\
            {\color{green!71!black} \eli - \eli_1 - \eli_2 = 0};
        \end{cases}
        \qquad \implies \qquad
        \begin{cases}
            \eli_1 = \frac{\ele_1 - \eli R}{r_1}, \\
            \eli_2 = \frac{\ele_2 + \eli R}{r_2}, \\
            \eli - \eli_1 - \eli_2 = 0;
        \end{cases} \implies \\
        &\implies \eli - \frac{\ele_1 - \eli R}{r_1} + \frac{\ele_2 + \eli R}{r_2} = 0, \\
        &\eli\cbr{ 1 + \frac{R}{r_1} + \frac{R}{r_2}} - \frac{\ele_1}{r_1} + \frac{\ele_2}{r_2} = 0, \\
        &\eli
            = \frac{\frac{\ele_1}{r_1} - \frac{\ele_2}{r_2}}{ 1 + \frac{R}{r_1} + \frac{R}{r_2}}
            = \frac{\frac{18\,\text{В}}{2\,\text{Ом}} - \frac{25\,\text{В}}{6\,\text{Ом}}}{ 1 + \frac{18\,\text{Ом}}{2\,\text{Ом}} + \frac{18\,\text{Ом}}{6\,\text{Ом}}}
            = \frac{29}{78}\units{А}
            \approx 0{,}372\,\text{А}, \\
        &U  = \eli R = \frac{\frac{\ele_1}{r_1} - \frac{\ele_2}{r_2}}{ 1 + \frac{R}{r_1} + \frac{R}{r_2}} R
            \approx 6{,}692\,\text{В}.
    \end{align*}
}

\variantsplitter

\addpersonalvariant{Ирина Ан}

\tasknumber{1}%
\task{%
    Определите эквивалентное сопротивление цепи на рисунке (между выделенными на рисунке контактами),
    если известны сопротивления всех резисторов: $R_1 = 2\,\text{Ом}$, $R_2 = 4\,\text{Ом}$, $R_3 = 2\,\text{Ом}$, $R_4 = 2\,\text{Ом}$.
    При каком напряжении поданном на эту цепь, в ней потечёт ток равный $\eli = 2\,\text{А}$?

    \begin{tikzpicture}[rotate=90, circuit ee IEC, thick]
        \node [contact]  (contact1) at (-1.5, 0) {};
        \draw  (0, 0) to [resistor={info=$R_1$}] ++(left:1.5);
        \draw  (0, 0) -- ++(up:1.5) to [resistor={near start, info=$R_2$}, resistor={near end, info=$R_3$}] ++(right:3);
        \draw  (0, 0) to [resistor={info=$R_4$}] ++(right:3) -- ++(up:1.5);
        \draw  (1.5, 1.5) -- ++(up:1); \node [contact] (contact2) at (1.5, 2.5) {};
    \end{tikzpicture}
}
\answer{%
    $R=4\units{Ом} \approx 4{,}00\,\text{Ом} \implies U = \eli R \approx 8{,}0\,\text{В}.$
}
\solutionspace{120pt}

\tasknumber{2}%
\task{%
    Определите показания амперметра $1$ (см.
    рис.) и разность потенциалов на резисторе $5$,
    если сопротивления всех резисторов равны: $R_1 = R_2 = R_3 = R_4 = R_5 = R_6 = R = 10\,\text{Ом}$,
    а напряжение, поданное на цепь, равно $U = 90\,\text{В}$.
    Ответы получите в виде несократимых дробей, а также определите приближённые значения.
    Амперметры считать идеальными.

    \begin{tikzpicture}[circuit ee IEC, thick]
        \node [contact]  (left contact) at (3, 0) {};
        \node [contact]  (right contact) at (9, 0) {};
        \draw  (left contact) -- ++(up:2) to [resistor={very near start, info=$R_2$}, amperemeter={midway, info=$1$}, resistor={very near end, info=$R_3$} ] ++(right:6) -- (right contact);
        \draw  (left contact) -- ++(down:2) to [resistor={very near start, info=$R_5$}, resistor={midway, info=$R_6$}, amperemeter={very near end, info=$3$}] ++(right:6) -- (right contact);
        \draw  (left contact) ++(left:3) to [resistor={info=$R_1$}] (left contact) to [amperemeter={near start, info=$2$}, resistor={near end , info=$R_4$}] (right contact) -- ++(right:0.5);
    \end{tikzpicture}
}
\answer{%
    \begin{align*}
    R_0 &= R + \frac 1{\frac 1{R+R} + \frac 1R + \frac 1{R+R}} = R + \frac 1{\frac 2R} = \frac 32 R, \\
    \eli &= \frac U{R_0} = \frac {2U}{3R}, \\
    U_1 &= \eli R_1 = \frac {2U}{3R} \cdot R = \frac 23 U = 60{,}0\,\text{В}, \\
    U_{23} &= U_{56} = U_4 = U - \eli R_1 = U - \frac {2U}{3R} \cdot R = \frac U3 = 30{,}0\,\text{В}, \\
    \eli_2 &= \frac{U_4}{R_4} = \frac U{3R} \approx 3{,}0\,\text{А}, \\
    \eli_1 &= \frac{U_{23}}{R_{23}} = \frac{\frac U3}{R+R} = \frac U{6R} \approx 1{,}5\,\text{А}, \\
    \eli_3 &= \frac{U_{56}}{R_{56}} = \frac{\frac U3}{R+R} = \frac U{6R} \approx 1{,}5\,\text{А}, \\
    U_2 &= \eli_1 R_2 = \frac U{6R} \cdot R = \frac U6 = 15{,}0\,\text{В}, \\
    U_3 &= \eli_1 R_3 = \frac U{6R} \cdot R = \frac U3 = 15{,}0\,\text{В}, \\
    U_5 &= \eli_3 R_5 = \frac U{6R} \cdot R = \frac U5 = 15{,}0\,\text{В}, \\
    U_6 &= \eli_3 R_6 = \frac U{6R} \cdot R = \frac U6 = 15{,}0\,\text{В}.
    \end{align*}
}
\solutionspace{120pt}

\tasknumber{3}%
\task{%
    Определите ток, протекающий через резистор $R = 10\,\text{Ом}$ и разность потенциалов на нём (см.
    рис.),
    если $\ele_1 = 12\,\text{В}$, $\ele_2 = 25\,\text{В}$, $r_1 = 2\,\text{Ом}$, $r_2 = 6\,\text{Ом}$.

    \begin{tikzpicture}[circuit ee IEC, thick]
        \draw  (0, 0) to [battery={rotate=-180,info={$\ele_1, r_1$}}] (0, 3)
                -- (5, 3)
                to [battery={rotate=-180, info'={$\ele_2, r_2$}}] (5, 0)
                -- (0, 0)
                (2.5, 0) to [resistor={info=$R$}] (2.5, 3);
    \end{tikzpicture}
}
\answer{%
    Выберем 2 контура и один узел, запишем для них законы Кирхгофа:

    \begin{tikzpicture}[circuit ee IEC, thick]
        \draw  (0, 0) to [battery={rotate=-180,info={$\ele_1, r_1$}}, current direction={near end, info=$\eli_1$}] (0, 3)
                -- (5, 3)
                to [battery={rotate=-180, info'={$\ele_2, r_2$}}, current direction={near end, info=$\eli_2$}] (5, 0)
                -- (0, 0)
                (2.5, 0) to [resistor={info=$R$}, current direction'={near end, info=$\eli$}] (2.5, 3);
        \draw [-{Latex},color=red] (0.8, 1.9) arc [start angle = 135, end angle = -160, radius = 0.6];
        \draw [-{Latex},color=blue] (3.5, 1.9) arc [start angle = 135, end angle = -160, radius = 0.6];
        \node [contact,color=green!71!black] (topc) at (2.5, 3) {};
        \node [above] (top) at (2.5, 3) {$1$};
    \end{tikzpicture}

    \begin{align*}
        &\begin{cases}
            {\color{red} \ele_1 = \eli_1 r_1 + \eli R}, \\
            {\color{blue} \ele_2 = \eli_2 r_2 - \eli R}, \\
            {\color{green!71!black} \eli - \eli_1 - \eli_2 = 0};
        \end{cases}
        \qquad \implies \qquad
        \begin{cases}
            \eli_1 = \frac{\ele_1 - \eli R}{r_1}, \\
            \eli_2 = \frac{\ele_2 + \eli R}{r_2}, \\
            \eli - \eli_1 - \eli_2 = 0;
        \end{cases} \implies \\
        &\implies \eli - \frac{\ele_1 - \eli R}{r_1} + \frac{\ele_2 + \eli R}{r_2} = 0, \\
        &\eli\cbr{ 1 + \frac{R}{r_1} + \frac{R}{r_2}} - \frac{\ele_1}{r_1} + \frac{\ele_2}{r_2} = 0, \\
        &\eli
            = \frac{\frac{\ele_1}{r_1} - \frac{\ele_2}{r_2}}{ 1 + \frac{R}{r_1} + \frac{R}{r_2}}
            = \frac{\frac{12\,\text{В}}{2\,\text{Ом}} - \frac{25\,\text{В}}{6\,\text{Ом}}}{ 1 + \frac{10\,\text{Ом}}{2\,\text{Ом}} + \frac{10\,\text{Ом}}{6\,\text{Ом}}}
            = \frac{11}{46}\units{А}
            \approx 0{,}239\,\text{А}, \\
        &U  = \eli R = \frac{\frac{\ele_1}{r_1} - \frac{\ele_2}{r_2}}{ 1 + \frac{R}{r_1} + \frac{R}{r_2}} R
            \approx 2{,}391\,\text{В}.
    \end{align*}
}

\variantsplitter

\addpersonalvariant{Софья Андрианова}

\tasknumber{1}%
\task{%
    Определите эквивалентное сопротивление цепи на рисунке (между выделенными на рисунке контактами),
    если известны сопротивления всех резисторов: $R_1 = 2\,\text{Ом}$, $R_2 = 4\,\text{Ом}$, $R_3 = 2\,\text{Ом}$, $R_4 = 3\,\text{Ом}$.
    При каком напряжении поданном на эту цепь, в ней потечёт ток равный $\eli = 10\,\text{А}$?

    \begin{tikzpicture}[rotate=90, circuit ee IEC, thick]
        \node [contact]  (contact1) at (-1.5, 0) {};
        \draw  (0, 0) to [resistor={info=$R_1$}] ++(left:1.5);
        \draw  (0, 0) -- ++(up:1.5) to [resistor={near start, info=$R_2$}, resistor={near end, info=$R_3$}] ++(right:3);
        \draw  (0, 0) to [resistor={info=$R_4$}] ++(right:3) -- ++(up:1.5);
        \draw  (1.5, 1.5) -- ++(up:1); \node [contact] (contact2) at (1.5, 2.5) {};
    \end{tikzpicture}
}
\answer{%
    $R=\frac{38}9\units{Ом} \approx 4{,}22\,\text{Ом} \implies U = \eli R \approx 42{,}2\,\text{В}.$
}
\solutionspace{120pt}

\tasknumber{2}%
\task{%
    Определите показания амперметра $3$ (см.
    рис.) и разность потенциалов на резисторе $3$,
    если сопротивления всех резисторов равны: $R_1 = R_2 = R_3 = R_4 = R_5 = R_6 = R = 10\,\text{Ом}$,
    а напряжение, поданное на цепь, равно $U = 30\,\text{В}$.
    Ответы получите в виде несократимых дробей, а также определите приближённые значения.
    Амперметры считать идеальными.

    \begin{tikzpicture}[circuit ee IEC, thick]
        \node [contact]  (left contact) at (3, 0) {};
        \node [contact]  (right contact) at (9, 0) {};
        \draw  (left contact) -- ++(up:2) to [resistor={very near start, info=$R_2$}, amperemeter={midway, info=$1$}, resistor={very near end, info=$R_3$} ] ++(right:6) -- (right contact);
        \draw  (left contact) -- ++(down:2) to [resistor={very near start, info=$R_5$}, resistor={midway, info=$R_6$}, amperemeter={very near end, info=$3$}] ++(right:6) -- (right contact);
        \draw  (left contact) ++(left:3) to [resistor={info=$R_1$}] (left contact) to [amperemeter={near start, info=$2$}, resistor={near end , info=$R_4$}] (right contact) -- ++(right:0.5);
    \end{tikzpicture}
}
\answer{%
    \begin{align*}
    R_0 &= R + \frac 1{\frac 1{R+R} + \frac 1R + \frac 1{R+R}} = R + \frac 1{\frac 2R} = \frac 32 R, \\
    \eli &= \frac U{R_0} = \frac {2U}{3R}, \\
    U_1 &= \eli R_1 = \frac {2U}{3R} \cdot R = \frac 23 U = 20{,}0\,\text{В}, \\
    U_{23} &= U_{56} = U_4 = U - \eli R_1 = U - \frac {2U}{3R} \cdot R = \frac U3 = 10{,}0\,\text{В}, \\
    \eli_2 &= \frac{U_4}{R_4} = \frac U{3R} \approx 1{,}0\,\text{А}, \\
    \eli_1 &= \frac{U_{23}}{R_{23}} = \frac{\frac U3}{R+R} = \frac U{6R} \approx 0{,}5\,\text{А}, \\
    \eli_3 &= \frac{U_{56}}{R_{56}} = \frac{\frac U3}{R+R} = \frac U{6R} \approx 0{,}5\,\text{А}, \\
    U_2 &= \eli_1 R_2 = \frac U{6R} \cdot R = \frac U6 = 5{,}0\,\text{В}, \\
    U_3 &= \eli_1 R_3 = \frac U{6R} \cdot R = \frac U3 = 5{,}0\,\text{В}, \\
    U_5 &= \eli_3 R_5 = \frac U{6R} \cdot R = \frac U5 = 5{,}0\,\text{В}, \\
    U_6 &= \eli_3 R_6 = \frac U{6R} \cdot R = \frac U6 = 5{,}0\,\text{В}.
    \end{align*}
}
\solutionspace{120pt}

\tasknumber{3}%
\task{%
    Определите ток, протекающий через резистор $R = 20\,\text{Ом}$ и разность потенциалов на нём (см.
    рис.),
    если $\ele_1 = 18\,\text{В}$, $\ele_2 = 5\,\text{В}$, $r_1 = 3\,\text{Ом}$, $r_2 = 2\,\text{Ом}$.

    \begin{tikzpicture}[circuit ee IEC, thick]
        \draw  (0, 0) to [battery={rotate=-180,info={$\ele_1, r_1$}}] (0, 3)
                -- (5, 3)
                to [battery={rotate=-180, info'={$\ele_2, r_2$}}] (5, 0)
                -- (0, 0)
                (2.5, 0) to [resistor={info=$R$}] (2.5, 3);
    \end{tikzpicture}
}
\answer{%
    Выберем 2 контура и один узел, запишем для них законы Кирхгофа:

    \begin{tikzpicture}[circuit ee IEC, thick]
        \draw  (0, 0) to [battery={rotate=-180,info={$\ele_1, r_1$}}, current direction={near end, info=$\eli_1$}] (0, 3)
                -- (5, 3)
                to [battery={rotate=-180, info'={$\ele_2, r_2$}}, current direction={near end, info=$\eli_2$}] (5, 0)
                -- (0, 0)
                (2.5, 0) to [resistor={info=$R$}, current direction'={near end, info=$\eli$}] (2.5, 3);
        \draw [-{Latex},color=red] (0.8, 1.9) arc [start angle = 135, end angle = -160, radius = 0.6];
        \draw [-{Latex},color=blue] (3.5, 1.9) arc [start angle = 135, end angle = -160, radius = 0.6];
        \node [contact,color=green!71!black] (topc) at (2.5, 3) {};
        \node [above] (top) at (2.5, 3) {$1$};
    \end{tikzpicture}

    \begin{align*}
        &\begin{cases}
            {\color{red} \ele_1 = \eli_1 r_1 + \eli R}, \\
            {\color{blue} \ele_2 = \eli_2 r_2 - \eli R}, \\
            {\color{green!71!black} \eli - \eli_1 - \eli_2 = 0};
        \end{cases}
        \qquad \implies \qquad
        \begin{cases}
            \eli_1 = \frac{\ele_1 - \eli R}{r_1}, \\
            \eli_2 = \frac{\ele_2 + \eli R}{r_2}, \\
            \eli - \eli_1 - \eli_2 = 0;
        \end{cases} \implies \\
        &\implies \eli - \frac{\ele_1 - \eli R}{r_1} + \frac{\ele_2 + \eli R}{r_2} = 0, \\
        &\eli\cbr{ 1 + \frac{R}{r_1} + \frac{R}{r_2}} - \frac{\ele_1}{r_1} + \frac{\ele_2}{r_2} = 0, \\
        &\eli
            = \frac{\frac{\ele_1}{r_1} - \frac{\ele_2}{r_2}}{ 1 + \frac{R}{r_1} + \frac{R}{r_2}}
            = \frac{\frac{18\,\text{В}}{3\,\text{Ом}} - \frac{5\,\text{В}}{2\,\text{Ом}}}{ 1 + \frac{20\,\text{Ом}}{3\,\text{Ом}} + \frac{20\,\text{Ом}}{2\,\text{Ом}}}
            = \frac{21}{106}\units{А}
            \approx 0{,}198\,\text{А}, \\
        &U  = \eli R = \frac{\frac{\ele_1}{r_1} - \frac{\ele_2}{r_2}}{ 1 + \frac{R}{r_1} + \frac{R}{r_2}} R
            \approx 3{,}962\,\text{В}.
    \end{align*}
}

\variantsplitter

\addpersonalvariant{Владимир Артемчук}

\tasknumber{1}%
\task{%
    Определите эквивалентное сопротивление цепи на рисунке (между выделенными на рисунке контактами),
    если известны сопротивления всех резисторов: $R_1 = 1\,\text{Ом}$, $R_2 = 3\,\text{Ом}$, $R_3 = 1\,\text{Ом}$, $R_4 = 4\,\text{Ом}$.
    При каком напряжении поданном на эту цепь, в ней потечёт ток равный $\eli = 5\,\text{А}$?

    \begin{tikzpicture}[rotate=270, circuit ee IEC, thick]
        \node [contact]  (contact1) at (-1.5, 0) {};
        \draw  (0, 0) to [resistor={info=$R_1$}] ++(left:1.5);
        \draw  (0, 0) -- ++(up:1.5) to [resistor={near start, info=$R_2$}, resistor={near end, info=$R_3$}] ++(right:3);
        \draw  (0, 0) to [resistor={info=$R_4$}] ++(right:3) -- ++(up:1.5);
        \draw  (1.5, 1.5) -- ++(up:1); \node [contact] (contact2) at (1.5, 2.5) {};
    \end{tikzpicture}
}
\answer{%
    $R=\frac{23}8\units{Ом} \approx 2{,}88\,\text{Ом} \implies U = \eli R \approx 14{,}4\,\text{В}.$
}
\solutionspace{120pt}

\tasknumber{2}%
\task{%
    Определите показания амперметра $3$ (см.
    рис.) и разность потенциалов на резисторе $2$,
    если сопротивления всех резисторов равны: $R_1 = R_2 = R_3 = R_4 = R_5 = R_6 = R = 2\,\text{Ом}$,
    а напряжение, поданное на цепь, равно $U = 150\,\text{В}$.
    Ответы получите в виде несократимых дробей, а также определите приближённые значения.
    Амперметры считать идеальными.

    \begin{tikzpicture}[circuit ee IEC, thick]
        \node [contact]  (left contact) at (3, 0) {};
        \node [contact]  (right contact) at (9, 0) {};
        \draw  (left contact) -- ++(up:2) to [resistor={very near start, info=$R_2$}, amperemeter={midway, info=$1$}, resistor={very near end, info=$R_3$} ] ++(right:6) -- (right contact);
        \draw  (left contact) -- ++(down:2) to [resistor={very near start, info=$R_5$}, resistor={midway, info=$R_6$}, amperemeter={very near end, info=$3$}] ++(right:6) -- (right contact);
        \draw  (left contact) ++(left:3) to [resistor={info=$R_1$}] (left contact) to [amperemeter={near start, info=$2$}, resistor={near end , info=$R_4$}] (right contact) -- ++(right:0.5);
    \end{tikzpicture}
}
\answer{%
    \begin{align*}
    R_0 &= R + \frac 1{\frac 1{R+R} + \frac 1R + \frac 1{R+R}} = R + \frac 1{\frac 2R} = \frac 32 R, \\
    \eli &= \frac U{R_0} = \frac {2U}{3R}, \\
    U_1 &= \eli R_1 = \frac {2U}{3R} \cdot R = \frac 23 U = 100{,}0\,\text{В}, \\
    U_{23} &= U_{56} = U_4 = U - \eli R_1 = U - \frac {2U}{3R} \cdot R = \frac U3 = 50{,}0\,\text{В}, \\
    \eli_2 &= \frac{U_4}{R_4} = \frac U{3R} \approx 25{,}0\,\text{А}, \\
    \eli_1 &= \frac{U_{23}}{R_{23}} = \frac{\frac U3}{R+R} = \frac U{6R} \approx 12{,}5\,\text{А}, \\
    \eli_3 &= \frac{U_{56}}{R_{56}} = \frac{\frac U3}{R+R} = \frac U{6R} \approx 12{,}5\,\text{А}, \\
    U_2 &= \eli_1 R_2 = \frac U{6R} \cdot R = \frac U6 = 25{,}0\,\text{В}, \\
    U_3 &= \eli_1 R_3 = \frac U{6R} \cdot R = \frac U3 = 25{,}0\,\text{В}, \\
    U_5 &= \eli_3 R_5 = \frac U{6R} \cdot R = \frac U5 = 25{,}0\,\text{В}, \\
    U_6 &= \eli_3 R_6 = \frac U{6R} \cdot R = \frac U6 = 25{,}0\,\text{В}.
    \end{align*}
}
\solutionspace{120pt}

\tasknumber{3}%
\task{%
    Определите ток, протекающий через резистор $R = 15\,\text{Ом}$ и разность потенциалов на нём (см.
    рис.),
    если $\ele_1 = 12\,\text{В}$, $\ele_2 = 25\,\text{В}$, $r_1 = 1\,\text{Ом}$, $r_2 = 6\,\text{Ом}$.

    \begin{tikzpicture}[circuit ee IEC, thick]
        \draw  (0, 0) to [battery={rotate=-180,info={$\ele_1, r_1$}}] (0, 3)
                -- (5, 3)
                to [battery={rotate=-180, info'={$\ele_2, r_2$}}] (5, 0)
                -- (0, 0)
                (2.5, 0) to [resistor={info=$R$}] (2.5, 3);
    \end{tikzpicture}
}
\answer{%
    Выберем 2 контура и один узел, запишем для них законы Кирхгофа:

    \begin{tikzpicture}[circuit ee IEC, thick]
        \draw  (0, 0) to [battery={rotate=-180,info={$\ele_1, r_1$}}, current direction={near end, info=$\eli_1$}] (0, 3)
                -- (5, 3)
                to [battery={rotate=-180, info'={$\ele_2, r_2$}}, current direction={near end, info=$\eli_2$}] (5, 0)
                -- (0, 0)
                (2.5, 0) to [resistor={info=$R$}, current direction'={near end, info=$\eli$}] (2.5, 3);
        \draw [-{Latex},color=red] (0.8, 1.9) arc [start angle = 135, end angle = -160, radius = 0.6];
        \draw [-{Latex},color=blue] (3.5, 1.9) arc [start angle = 135, end angle = -160, radius = 0.6];
        \node [contact,color=green!71!black] (topc) at (2.5, 3) {};
        \node [above] (top) at (2.5, 3) {$1$};
    \end{tikzpicture}

    \begin{align*}
        &\begin{cases}
            {\color{red} \ele_1 = \eli_1 r_1 + \eli R}, \\
            {\color{blue} \ele_2 = \eli_2 r_2 - \eli R}, \\
            {\color{green!71!black} \eli - \eli_1 - \eli_2 = 0};
        \end{cases}
        \qquad \implies \qquad
        \begin{cases}
            \eli_1 = \frac{\ele_1 - \eli R}{r_1}, \\
            \eli_2 = \frac{\ele_2 + \eli R}{r_2}, \\
            \eli - \eli_1 - \eli_2 = 0;
        \end{cases} \implies \\
        &\implies \eli - \frac{\ele_1 - \eli R}{r_1} + \frac{\ele_2 + \eli R}{r_2} = 0, \\
        &\eli\cbr{ 1 + \frac{R}{r_1} + \frac{R}{r_2}} - \frac{\ele_1}{r_1} + \frac{\ele_2}{r_2} = 0, \\
        &\eli
            = \frac{\frac{\ele_1}{r_1} - \frac{\ele_2}{r_2}}{ 1 + \frac{R}{r_1} + \frac{R}{r_2}}
            = \frac{\frac{12\,\text{В}}{1\,\text{Ом}} - \frac{25\,\text{В}}{6\,\text{Ом}}}{ 1 + \frac{15\,\text{Ом}}{1\,\text{Ом}} + \frac{15\,\text{Ом}}{6\,\text{Ом}}}
            = \frac{47}{111}\units{А}
            \approx 0{,}423\,\text{А}, \\
        &U  = \eli R = \frac{\frac{\ele_1}{r_1} - \frac{\ele_2}{r_2}}{ 1 + \frac{R}{r_1} + \frac{R}{r_2}} R
            \approx 6{,}351\,\text{В}.
    \end{align*}
}

\variantsplitter

\addpersonalvariant{Софья Белянкина}

\tasknumber{1}%
\task{%
    Определите эквивалентное сопротивление цепи на рисунке (между выделенными на рисунке контактами),
    если известны сопротивления всех резисторов: $R_1 = 2\,\text{Ом}$, $R_2 = 5\,\text{Ом}$, $R_3 = 1\,\text{Ом}$, $R_4 = 3\,\text{Ом}$.
    При каком напряжении поданном на эту цепь, в ней потечёт ток равный $\eli = 5\,\text{А}$?

    \begin{tikzpicture}[rotate=270, circuit ee IEC, thick]
        \node [contact]  (contact1) at (-1.5, 0) {};
        \draw  (0, 0) to [resistor={info=$R_1$}] ++(left:1.5);
        \draw  (0, 0) -- ++(up:1.5) to [resistor={near start, info=$R_2$}, resistor={near end, info=$R_3$}] ++(right:3);
        \draw  (0, 0) to [resistor={info=$R_4$}] ++(right:3) -- ++(up:1.5);
        \draw  (1.5, 1.5) -- ++(up:1); \node [contact] (contact2) at (1.5, 2.5) {};
    \end{tikzpicture}
}
\answer{%
    $R=\frac{38}9\units{Ом} \approx 4{,}22\,\text{Ом} \implies U = \eli R \approx 21{,}1\,\text{В}.$
}
\solutionspace{120pt}

\tasknumber{2}%
\task{%
    Определите показания амперметра $3$ (см.
    рис.) и разность потенциалов на резисторе $1$,
    если сопротивления всех резисторов равны: $R_1 = R_2 = R_3 = R_4 = R_5 = R_6 = R = 4\,\text{Ом}$,
    а напряжение, поданное на цепь, равно $U = 60\,\text{В}$.
    Ответы получите в виде несократимых дробей, а также определите приближённые значения.
    Амперметры считать идеальными.

    \begin{tikzpicture}[circuit ee IEC, thick]
        \node [contact]  (left contact) at (3, 0) {};
        \node [contact]  (right contact) at (9, 0) {};
        \draw  (left contact) -- ++(up:2) to [resistor={very near start, info=$R_2$}, amperemeter={midway, info=$1$}, resistor={very near end, info=$R_3$} ] ++(right:6) -- (right contact);
        \draw  (left contact) -- ++(down:2) to [resistor={very near start, info=$R_5$}, resistor={midway, info=$R_6$}, amperemeter={very near end, info=$3$}] ++(right:6) -- (right contact);
        \draw  (left contact) ++(left:3) to [resistor={info=$R_1$}] (left contact) to [amperemeter={near start, info=$2$}, resistor={near end , info=$R_4$}] (right contact) -- ++(right:0.5);
    \end{tikzpicture}
}
\answer{%
    \begin{align*}
    R_0 &= R + \frac 1{\frac 1{R+R} + \frac 1R + \frac 1{R+R}} = R + \frac 1{\frac 2R} = \frac 32 R, \\
    \eli &= \frac U{R_0} = \frac {2U}{3R}, \\
    U_1 &= \eli R_1 = \frac {2U}{3R} \cdot R = \frac 23 U = 40{,}0\,\text{В}, \\
    U_{23} &= U_{56} = U_4 = U - \eli R_1 = U - \frac {2U}{3R} \cdot R = \frac U3 = 20{,}0\,\text{В}, \\
    \eli_2 &= \frac{U_4}{R_4} = \frac U{3R} \approx 5{,}0\,\text{А}, \\
    \eli_1 &= \frac{U_{23}}{R_{23}} = \frac{\frac U3}{R+R} = \frac U{6R} \approx 2{,}5\,\text{А}, \\
    \eli_3 &= \frac{U_{56}}{R_{56}} = \frac{\frac U3}{R+R} = \frac U{6R} \approx 2{,}5\,\text{А}, \\
    U_2 &= \eli_1 R_2 = \frac U{6R} \cdot R = \frac U6 = 10{,}0\,\text{В}, \\
    U_3 &= \eli_1 R_3 = \frac U{6R} \cdot R = \frac U3 = 10{,}0\,\text{В}, \\
    U_5 &= \eli_3 R_5 = \frac U{6R} \cdot R = \frac U5 = 10{,}0\,\text{В}, \\
    U_6 &= \eli_3 R_6 = \frac U{6R} \cdot R = \frac U6 = 10{,}0\,\text{В}.
    \end{align*}
}
\solutionspace{120pt}

\tasknumber{3}%
\task{%
    Определите ток, протекающий через резистор $R = 20\,\text{Ом}$ и разность потенциалов на нём (см.
    рис.),
    если $\ele_1 = 12\,\text{В}$, $\ele_2 = 25\,\text{В}$, $r_1 = 1\,\text{Ом}$, $r_2 = 4\,\text{Ом}$.

    \begin{tikzpicture}[circuit ee IEC, thick]
        \draw  (0, 0) to [battery={rotate=-180,info={$\ele_1, r_1$}}] (0, 3)
                -- (5, 3)
                to [battery={rotate=-180, info'={$\ele_2, r_2$}}] (5, 0)
                -- (0, 0)
                (2.5, 0) to [resistor={info=$R$}] (2.5, 3);
    \end{tikzpicture}
}
\answer{%
    Выберем 2 контура и один узел, запишем для них законы Кирхгофа:

    \begin{tikzpicture}[circuit ee IEC, thick]
        \draw  (0, 0) to [battery={rotate=-180,info={$\ele_1, r_1$}}, current direction={near end, info=$\eli_1$}] (0, 3)
                -- (5, 3)
                to [battery={rotate=-180, info'={$\ele_2, r_2$}}, current direction={near end, info=$\eli_2$}] (5, 0)
                -- (0, 0)
                (2.5, 0) to [resistor={info=$R$}, current direction'={near end, info=$\eli$}] (2.5, 3);
        \draw [-{Latex},color=red] (0.8, 1.9) arc [start angle = 135, end angle = -160, radius = 0.6];
        \draw [-{Latex},color=blue] (3.5, 1.9) arc [start angle = 135, end angle = -160, radius = 0.6];
        \node [contact,color=green!71!black] (topc) at (2.5, 3) {};
        \node [above] (top) at (2.5, 3) {$1$};
    \end{tikzpicture}

    \begin{align*}
        &\begin{cases}
            {\color{red} \ele_1 = \eli_1 r_1 + \eli R}, \\
            {\color{blue} \ele_2 = \eli_2 r_2 - \eli R}, \\
            {\color{green!71!black} \eli - \eli_1 - \eli_2 = 0};
        \end{cases}
        \qquad \implies \qquad
        \begin{cases}
            \eli_1 = \frac{\ele_1 - \eli R}{r_1}, \\
            \eli_2 = \frac{\ele_2 + \eli R}{r_2}, \\
            \eli - \eli_1 - \eli_2 = 0;
        \end{cases} \implies \\
        &\implies \eli - \frac{\ele_1 - \eli R}{r_1} + \frac{\ele_2 + \eli R}{r_2} = 0, \\
        &\eli\cbr{ 1 + \frac{R}{r_1} + \frac{R}{r_2}} - \frac{\ele_1}{r_1} + \frac{\ele_2}{r_2} = 0, \\
        &\eli
            = \frac{\frac{\ele_1}{r_1} - \frac{\ele_2}{r_2}}{ 1 + \frac{R}{r_1} + \frac{R}{r_2}}
            = \frac{\frac{12\,\text{В}}{1\,\text{Ом}} - \frac{25\,\text{В}}{4\,\text{Ом}}}{ 1 + \frac{20\,\text{Ом}}{1\,\text{Ом}} + \frac{20\,\text{Ом}}{4\,\text{Ом}}}
            = \frac{23}{104}\units{А}
            \approx 0{,}221\,\text{А}, \\
        &U  = \eli R = \frac{\frac{\ele_1}{r_1} - \frac{\ele_2}{r_2}}{ 1 + \frac{R}{r_1} + \frac{R}{r_2}} R
            \approx 4{,}423\,\text{В}.
    \end{align*}
}

\variantsplitter

\addpersonalvariant{Варвара Егиазарян}

\tasknumber{1}%
\task{%
    Определите эквивалентное сопротивление цепи на рисунке (между выделенными на рисунке контактами),
    если известны сопротивления всех резисторов: $R_1 = 2\,\text{Ом}$, $R_2 = 3\,\text{Ом}$, $R_3 = 3\,\text{Ом}$, $R_4 = 2\,\text{Ом}$.
    При каком напряжении поданном на эту цепь, в ней потечёт ток равный $\eli = 10\,\text{А}$?

    \begin{tikzpicture}[rotate=0, circuit ee IEC, thick]
        \node [contact]  (contact1) at (-1.5, 0) {};
        \draw  (0, 0) to [resistor={info=$R_1$}] ++(left:1.5);
        \draw  (0, 0) -- ++(up:1.5) to [resistor={near start, info=$R_2$}, resistor={near end, info=$R_3$}] ++(right:3);
        \draw  (0, 0) to [resistor={info=$R_4$}] ++(right:3) -- ++(up:1.5);
        \draw  (3, 1.5) -- ++(right:0.5); \node [contact] (contact2) at (3.5, 1.5) {};
    \end{tikzpicture}
}
\answer{%
    $R=\frac72\units{Ом} \approx 3{,}50\,\text{Ом} \implies U = \eli R \approx 35{,}0\,\text{В}.$
}
\solutionspace{120pt}

\tasknumber{2}%
\task{%
    Определите показания амперметра $2$ (см.
    рис.) и разность потенциалов на резисторе $2$,
    если сопротивления всех резисторов равны: $R_1 = R_2 = R_3 = R_4 = R_5 = R_6 = R = 5\,\text{Ом}$,
    а напряжение, поданное на цепь, равно $U = 90\,\text{В}$.
    Ответы получите в виде несократимых дробей, а также определите приближённые значения.
    Амперметры считать идеальными.

    \begin{tikzpicture}[circuit ee IEC, thick]
        \node [contact]  (left contact) at (3, 0) {};
        \node [contact]  (right contact) at (9, 0) {};
        \draw  (left contact) -- ++(up:2) to [resistor={very near start, info=$R_2$}, amperemeter={midway, info=$1$}, resistor={very near end, info=$R_3$} ] ++(right:6) -- (right contact);
        \draw  (left contact) -- ++(down:2) to [resistor={very near start, info=$R_5$}, resistor={midway, info=$R_6$}, amperemeter={very near end, info=$3$}] ++(right:6) -- (right contact);
        \draw  (left contact) ++(left:3) to [resistor={info=$R_1$}] (left contact) to [amperemeter={near start, info=$2$}, resistor={near end , info=$R_4$}] (right contact) -- ++(right:0.5);
    \end{tikzpicture}
}
\answer{%
    \begin{align*}
    R_0 &= R + \frac 1{\frac 1{R+R} + \frac 1R + \frac 1{R+R}} = R + \frac 1{\frac 2R} = \frac 32 R, \\
    \eli &= \frac U{R_0} = \frac {2U}{3R}, \\
    U_1 &= \eli R_1 = \frac {2U}{3R} \cdot R = \frac 23 U = 60{,}0\,\text{В}, \\
    U_{23} &= U_{56} = U_4 = U - \eli R_1 = U - \frac {2U}{3R} \cdot R = \frac U3 = 30{,}0\,\text{В}, \\
    \eli_2 &= \frac{U_4}{R_4} = \frac U{3R} \approx 6{,}0\,\text{А}, \\
    \eli_1 &= \frac{U_{23}}{R_{23}} = \frac{\frac U3}{R+R} = \frac U{6R} \approx 3{,}0\,\text{А}, \\
    \eli_3 &= \frac{U_{56}}{R_{56}} = \frac{\frac U3}{R+R} = \frac U{6R} \approx 3{,}0\,\text{А}, \\
    U_2 &= \eli_1 R_2 = \frac U{6R} \cdot R = \frac U6 = 15{,}0\,\text{В}, \\
    U_3 &= \eli_1 R_3 = \frac U{6R} \cdot R = \frac U3 = 15{,}0\,\text{В}, \\
    U_5 &= \eli_3 R_5 = \frac U{6R} \cdot R = \frac U5 = 15{,}0\,\text{В}, \\
    U_6 &= \eli_3 R_6 = \frac U{6R} \cdot R = \frac U6 = 15{,}0\,\text{В}.
    \end{align*}
}
\solutionspace{120pt}

\tasknumber{3}%
\task{%
    Определите ток, протекающий через резистор $R = 18\,\text{Ом}$ и разность потенциалов на нём (см.
    рис.),
    если $\ele_1 = 12\,\text{В}$, $\ele_2 = 15\,\text{В}$, $r_1 = 1\,\text{Ом}$, $r_2 = 2\,\text{Ом}$.

    \begin{tikzpicture}[circuit ee IEC, thick]
        \draw  (0, 0) to [battery={rotate=-180,info={$\ele_1, r_1$}}] (0, 3)
                -- (5, 3)
                to [battery={rotate=-180, info'={$\ele_2, r_2$}}] (5, 0)
                -- (0, 0)
                (2.5, 0) to [resistor={info=$R$}] (2.5, 3);
    \end{tikzpicture}
}
\answer{%
    Выберем 2 контура и один узел, запишем для них законы Кирхгофа:

    \begin{tikzpicture}[circuit ee IEC, thick]
        \draw  (0, 0) to [battery={rotate=-180,info={$\ele_1, r_1$}}, current direction={near end, info=$\eli_1$}] (0, 3)
                -- (5, 3)
                to [battery={rotate=-180, info'={$\ele_2, r_2$}}, current direction={near end, info=$\eli_2$}] (5, 0)
                -- (0, 0)
                (2.5, 0) to [resistor={info=$R$}, current direction'={near end, info=$\eli$}] (2.5, 3);
        \draw [-{Latex},color=red] (0.8, 1.9) arc [start angle = 135, end angle = -160, radius = 0.6];
        \draw [-{Latex},color=blue] (3.5, 1.9) arc [start angle = 135, end angle = -160, radius = 0.6];
        \node [contact,color=green!71!black] (topc) at (2.5, 3) {};
        \node [above] (top) at (2.5, 3) {$1$};
    \end{tikzpicture}

    \begin{align*}
        &\begin{cases}
            {\color{red} \ele_1 = \eli_1 r_1 + \eli R}, \\
            {\color{blue} \ele_2 = \eli_2 r_2 - \eli R}, \\
            {\color{green!71!black} \eli - \eli_1 - \eli_2 = 0};
        \end{cases}
        \qquad \implies \qquad
        \begin{cases}
            \eli_1 = \frac{\ele_1 - \eli R}{r_1}, \\
            \eli_2 = \frac{\ele_2 + \eli R}{r_2}, \\
            \eli - \eli_1 - \eli_2 = 0;
        \end{cases} \implies \\
        &\implies \eli - \frac{\ele_1 - \eli R}{r_1} + \frac{\ele_2 + \eli R}{r_2} = 0, \\
        &\eli\cbr{ 1 + \frac{R}{r_1} + \frac{R}{r_2}} - \frac{\ele_1}{r_1} + \frac{\ele_2}{r_2} = 0, \\
        &\eli
            = \frac{\frac{\ele_1}{r_1} - \frac{\ele_2}{r_2}}{ 1 + \frac{R}{r_1} + \frac{R}{r_2}}
            = \frac{\frac{12\,\text{В}}{1\,\text{Ом}} - \frac{15\,\text{В}}{2\,\text{Ом}}}{ 1 + \frac{18\,\text{Ом}}{1\,\text{Ом}} + \frac{18\,\text{Ом}}{2\,\text{Ом}}}
            = \frac9{56}\units{А}
            \approx 0{,}161\,\text{А}, \\
        &U  = \eli R = \frac{\frac{\ele_1}{r_1} - \frac{\ele_2}{r_2}}{ 1 + \frac{R}{r_1} + \frac{R}{r_2}} R
            \approx 2{,}893\,\text{В}.
    \end{align*}
}

\variantsplitter

\addpersonalvariant{Владислав Емелин}

\tasknumber{1}%
\task{%
    Определите эквивалентное сопротивление цепи на рисунке (между выделенными на рисунке контактами),
    если известны сопротивления всех резисторов: $R_1 = 2\,\text{Ом}$, $R_2 = 4\,\text{Ом}$, $R_3 = 3\,\text{Ом}$, $R_4 = 2\,\text{Ом}$.
    При каком напряжении поданном на эту цепь, в ней потечёт ток равный $\eli = 5\,\text{А}$?

    \begin{tikzpicture}[rotate=0, circuit ee IEC, thick]
        \node [contact]  (contact1) at (-1.5, 0) {};
        \draw  (0, 0) to [resistor={info=$R_1$}] ++(left:1.5);
        \draw  (0, 0) -- ++(up:1.5) to [resistor={near start, info=$R_2$}, resistor={near end, info=$R_3$}] ++(right:3);
        \draw  (0, 0) to [resistor={info=$R_4$}] ++(right:3) -- ++(up:1.5);
        \draw  (3, 1.5) -- ++(right:0.5); \node [contact] (contact2) at (3.5, 1.5) {};
    \end{tikzpicture}
}
\answer{%
    $R=\frac{32}9\units{Ом} \approx 3{,}56\,\text{Ом} \implies U = \eli R \approx 17{,}8\,\text{В}.$
}
\solutionspace{120pt}

\tasknumber{2}%
\task{%
    Определите показания амперметра $3$ (см.
    рис.) и разность потенциалов на резисторе $4$,
    если сопротивления всех резисторов равны: $R_1 = R_2 = R_3 = R_4 = R_5 = R_6 = R = 2\,\text{Ом}$,
    а напряжение, поданное на цепь, равно $U = 60\,\text{В}$.
    Ответы получите в виде несократимых дробей, а также определите приближённые значения.
    Амперметры считать идеальными.

    \begin{tikzpicture}[circuit ee IEC, thick]
        \node [contact]  (left contact) at (3, 0) {};
        \node [contact]  (right contact) at (9, 0) {};
        \draw  (left contact) -- ++(up:2) to [resistor={very near start, info=$R_2$}, amperemeter={midway, info=$1$}, resistor={very near end, info=$R_3$} ] ++(right:6) -- (right contact);
        \draw  (left contact) -- ++(down:2) to [resistor={very near start, info=$R_5$}, resistor={midway, info=$R_6$}, amperemeter={very near end, info=$3$}] ++(right:6) -- (right contact);
        \draw  (left contact) ++(left:3) to [resistor={info=$R_1$}] (left contact) to [amperemeter={near start, info=$2$}, resistor={near end , info=$R_4$}] (right contact) -- ++(right:0.5);
    \end{tikzpicture}
}
\answer{%
    \begin{align*}
    R_0 &= R + \frac 1{\frac 1{R+R} + \frac 1R + \frac 1{R+R}} = R + \frac 1{\frac 2R} = \frac 32 R, \\
    \eli &= \frac U{R_0} = \frac {2U}{3R}, \\
    U_1 &= \eli R_1 = \frac {2U}{3R} \cdot R = \frac 23 U = 40{,}0\,\text{В}, \\
    U_{23} &= U_{56} = U_4 = U - \eli R_1 = U - \frac {2U}{3R} \cdot R = \frac U3 = 20{,}0\,\text{В}, \\
    \eli_2 &= \frac{U_4}{R_4} = \frac U{3R} \approx 10{,}0\,\text{А}, \\
    \eli_1 &= \frac{U_{23}}{R_{23}} = \frac{\frac U3}{R+R} = \frac U{6R} \approx 5{,}0\,\text{А}, \\
    \eli_3 &= \frac{U_{56}}{R_{56}} = \frac{\frac U3}{R+R} = \frac U{6R} \approx 5{,}0\,\text{А}, \\
    U_2 &= \eli_1 R_2 = \frac U{6R} \cdot R = \frac U6 = 10{,}0\,\text{В}, \\
    U_3 &= \eli_1 R_3 = \frac U{6R} \cdot R = \frac U3 = 10{,}0\,\text{В}, \\
    U_5 &= \eli_3 R_5 = \frac U{6R} \cdot R = \frac U5 = 10{,}0\,\text{В}, \\
    U_6 &= \eli_3 R_6 = \frac U{6R} \cdot R = \frac U6 = 10{,}0\,\text{В}.
    \end{align*}
}
\solutionspace{120pt}

\tasknumber{3}%
\task{%
    Определите ток, протекающий через резистор $R = 12\,\text{Ом}$ и разность потенциалов на нём (см.
    рис.),
    если $\ele_1 = 12\,\text{В}$, $\ele_2 = 15\,\text{В}$, $r_1 = 2\,\text{Ом}$, $r_2 = 4\,\text{Ом}$.

    \begin{tikzpicture}[circuit ee IEC, thick]
        \draw  (0, 0) to [battery={rotate=-180,info={$\ele_1, r_1$}}] (0, 3)
                -- (5, 3)
                to [battery={rotate=-180, info'={$\ele_2, r_2$}}] (5, 0)
                -- (0, 0)
                (2.5, 0) to [resistor={info=$R$}] (2.5, 3);
    \end{tikzpicture}
}
\answer{%
    Выберем 2 контура и один узел, запишем для них законы Кирхгофа:

    \begin{tikzpicture}[circuit ee IEC, thick]
        \draw  (0, 0) to [battery={rotate=-180,info={$\ele_1, r_1$}}, current direction={near end, info=$\eli_1$}] (0, 3)
                -- (5, 3)
                to [battery={rotate=-180, info'={$\ele_2, r_2$}}, current direction={near end, info=$\eli_2$}] (5, 0)
                -- (0, 0)
                (2.5, 0) to [resistor={info=$R$}, current direction'={near end, info=$\eli$}] (2.5, 3);
        \draw [-{Latex},color=red] (0.8, 1.9) arc [start angle = 135, end angle = -160, radius = 0.6];
        \draw [-{Latex},color=blue] (3.5, 1.9) arc [start angle = 135, end angle = -160, radius = 0.6];
        \node [contact,color=green!71!black] (topc) at (2.5, 3) {};
        \node [above] (top) at (2.5, 3) {$1$};
    \end{tikzpicture}

    \begin{align*}
        &\begin{cases}
            {\color{red} \ele_1 = \eli_1 r_1 + \eli R}, \\
            {\color{blue} \ele_2 = \eli_2 r_2 - \eli R}, \\
            {\color{green!71!black} \eli - \eli_1 - \eli_2 = 0};
        \end{cases}
        \qquad \implies \qquad
        \begin{cases}
            \eli_1 = \frac{\ele_1 - \eli R}{r_1}, \\
            \eli_2 = \frac{\ele_2 + \eli R}{r_2}, \\
            \eli - \eli_1 - \eli_2 = 0;
        \end{cases} \implies \\
        &\implies \eli - \frac{\ele_1 - \eli R}{r_1} + \frac{\ele_2 + \eli R}{r_2} = 0, \\
        &\eli\cbr{ 1 + \frac{R}{r_1} + \frac{R}{r_2}} - \frac{\ele_1}{r_1} + \frac{\ele_2}{r_2} = 0, \\
        &\eli
            = \frac{\frac{\ele_1}{r_1} - \frac{\ele_2}{r_2}}{ 1 + \frac{R}{r_1} + \frac{R}{r_2}}
            = \frac{\frac{12\,\text{В}}{2\,\text{Ом}} - \frac{15\,\text{В}}{4\,\text{Ом}}}{ 1 + \frac{12\,\text{Ом}}{2\,\text{Ом}} + \frac{12\,\text{Ом}}{4\,\text{Ом}}}
            = \frac9{40}\units{А}
            \approx 0{,}225\,\text{А}, \\
        &U  = \eli R = \frac{\frac{\ele_1}{r_1} - \frac{\ele_2}{r_2}}{ 1 + \frac{R}{r_1} + \frac{R}{r_2}} R
            \approx 2{,}700\,\text{В}.
    \end{align*}
}

\variantsplitter

\addpersonalvariant{Артём Жичин}

\tasknumber{1}%
\task{%
    Определите эквивалентное сопротивление цепи на рисунке (между выделенными на рисунке контактами),
    если известны сопротивления всех резисторов: $R_1 = 1\,\text{Ом}$, $R_2 = 5\,\text{Ом}$, $R_3 = 2\,\text{Ом}$, $R_4 = 2\,\text{Ом}$.
    При каком напряжении поданном на эту цепь, в ней потечёт ток равный $\eli = 2\,\text{А}$?

    \begin{tikzpicture}[rotate=0, circuit ee IEC, thick]
        \node [contact]  (contact1) at (-1.5, 0) {};
        \draw  (0, 0) to [resistor={info=$R_1$}] ++(left:1.5);
        \draw  (0, 0) -- ++(up:1.5) to [resistor={near start, info=$R_2$}, resistor={near end, info=$R_3$}] ++(right:3);
        \draw  (0, 0) to [resistor={info=$R_4$}] ++(right:3) -- ++(up:1.5);
        \draw  (3, 1.5) -- ++(right:0.5); \node [contact] (contact2) at (3.5, 1.5) {};
    \end{tikzpicture}
}
\answer{%
    $R=\frac{23}9\units{Ом} \approx 2{,}56\,\text{Ом} \implies U = \eli R \approx 5{,}1\,\text{В}.$
}
\solutionspace{120pt}

\tasknumber{2}%
\task{%
    Определите показания амперметра $2$ (см.
    рис.) и разность потенциалов на резисторе $5$,
    если сопротивления всех резисторов равны: $R_1 = R_2 = R_3 = R_4 = R_5 = R_6 = R = 5\,\text{Ом}$,
    а напряжение, поданное на цепь, равно $U = 120\,\text{В}$.
    Ответы получите в виде несократимых дробей, а также определите приближённые значения.
    Амперметры считать идеальными.

    \begin{tikzpicture}[circuit ee IEC, thick]
        \node [contact]  (left contact) at (3, 0) {};
        \node [contact]  (right contact) at (9, 0) {};
        \draw  (left contact) -- ++(up:2) to [resistor={very near start, info=$R_2$}, amperemeter={midway, info=$1$}, resistor={very near end, info=$R_3$} ] ++(right:6) -- (right contact);
        \draw  (left contact) -- ++(down:2) to [resistor={very near start, info=$R_5$}, resistor={midway, info=$R_6$}, amperemeter={very near end, info=$3$}] ++(right:6) -- (right contact);
        \draw  (left contact) ++(left:3) to [resistor={info=$R_1$}] (left contact) to [amperemeter={near start, info=$2$}, resistor={near end , info=$R_4$}] (right contact) -- ++(right:0.5);
    \end{tikzpicture}
}
\answer{%
    \begin{align*}
    R_0 &= R + \frac 1{\frac 1{R+R} + \frac 1R + \frac 1{R+R}} = R + \frac 1{\frac 2R} = \frac 32 R, \\
    \eli &= \frac U{R_0} = \frac {2U}{3R}, \\
    U_1 &= \eli R_1 = \frac {2U}{3R} \cdot R = \frac 23 U = 80{,}0\,\text{В}, \\
    U_{23} &= U_{56} = U_4 = U - \eli R_1 = U - \frac {2U}{3R} \cdot R = \frac U3 = 40{,}0\,\text{В}, \\
    \eli_2 &= \frac{U_4}{R_4} = \frac U{3R} \approx 8{,}0\,\text{А}, \\
    \eli_1 &= \frac{U_{23}}{R_{23}} = \frac{\frac U3}{R+R} = \frac U{6R} \approx 4{,}0\,\text{А}, \\
    \eli_3 &= \frac{U_{56}}{R_{56}} = \frac{\frac U3}{R+R} = \frac U{6R} \approx 4{,}0\,\text{А}, \\
    U_2 &= \eli_1 R_2 = \frac U{6R} \cdot R = \frac U6 = 20{,}0\,\text{В}, \\
    U_3 &= \eli_1 R_3 = \frac U{6R} \cdot R = \frac U3 = 20{,}0\,\text{В}, \\
    U_5 &= \eli_3 R_5 = \frac U{6R} \cdot R = \frac U5 = 20{,}0\,\text{В}, \\
    U_6 &= \eli_3 R_6 = \frac U{6R} \cdot R = \frac U6 = 20{,}0\,\text{В}.
    \end{align*}
}
\solutionspace{120pt}

\tasknumber{3}%
\task{%
    Определите ток, протекающий через резистор $R = 12\,\text{Ом}$ и разность потенциалов на нём (см.
    рис.),
    если $\ele_1 = 18\,\text{В}$, $\ele_2 = 25\,\text{В}$, $r_1 = 3\,\text{Ом}$, $r_2 = 6\,\text{Ом}$.

    \begin{tikzpicture}[circuit ee IEC, thick]
        \draw  (0, 0) to [battery={rotate=-180,info={$\ele_1, r_1$}}] (0, 3)
                -- (5, 3)
                to [battery={rotate=-180, info'={$\ele_2, r_2$}}] (5, 0)
                -- (0, 0)
                (2.5, 0) to [resistor={info=$R$}] (2.5, 3);
    \end{tikzpicture}
}
\answer{%
    Выберем 2 контура и один узел, запишем для них законы Кирхгофа:

    \begin{tikzpicture}[circuit ee IEC, thick]
        \draw  (0, 0) to [battery={rotate=-180,info={$\ele_1, r_1$}}, current direction={near end, info=$\eli_1$}] (0, 3)
                -- (5, 3)
                to [battery={rotate=-180, info'={$\ele_2, r_2$}}, current direction={near end, info=$\eli_2$}] (5, 0)
                -- (0, 0)
                (2.5, 0) to [resistor={info=$R$}, current direction'={near end, info=$\eli$}] (2.5, 3);
        \draw [-{Latex},color=red] (0.8, 1.9) arc [start angle = 135, end angle = -160, radius = 0.6];
        \draw [-{Latex},color=blue] (3.5, 1.9) arc [start angle = 135, end angle = -160, radius = 0.6];
        \node [contact,color=green!71!black] (topc) at (2.5, 3) {};
        \node [above] (top) at (2.5, 3) {$1$};
    \end{tikzpicture}

    \begin{align*}
        &\begin{cases}
            {\color{red} \ele_1 = \eli_1 r_1 + \eli R}, \\
            {\color{blue} \ele_2 = \eli_2 r_2 - \eli R}, \\
            {\color{green!71!black} \eli - \eli_1 - \eli_2 = 0};
        \end{cases}
        \qquad \implies \qquad
        \begin{cases}
            \eli_1 = \frac{\ele_1 - \eli R}{r_1}, \\
            \eli_2 = \frac{\ele_2 + \eli R}{r_2}, \\
            \eli - \eli_1 - \eli_2 = 0;
        \end{cases} \implies \\
        &\implies \eli - \frac{\ele_1 - \eli R}{r_1} + \frac{\ele_2 + \eli R}{r_2} = 0, \\
        &\eli\cbr{ 1 + \frac{R}{r_1} + \frac{R}{r_2}} - \frac{\ele_1}{r_1} + \frac{\ele_2}{r_2} = 0, \\
        &\eli
            = \frac{\frac{\ele_1}{r_1} - \frac{\ele_2}{r_2}}{ 1 + \frac{R}{r_1} + \frac{R}{r_2}}
            = \frac{\frac{18\,\text{В}}{3\,\text{Ом}} - \frac{25\,\text{В}}{6\,\text{Ом}}}{ 1 + \frac{12\,\text{Ом}}{3\,\text{Ом}} + \frac{12\,\text{Ом}}{6\,\text{Ом}}}
            = \frac{11}{42}\units{А}
            \approx 0{,}262\,\text{А}, \\
        &U  = \eli R = \frac{\frac{\ele_1}{r_1} - \frac{\ele_2}{r_2}}{ 1 + \frac{R}{r_1} + \frac{R}{r_2}} R
            \approx 3{,}143\,\text{В}.
    \end{align*}
}

\variantsplitter

\addpersonalvariant{Дарья Кошман}

\tasknumber{1}%
\task{%
    Определите эквивалентное сопротивление цепи на рисунке (между выделенными на рисунке контактами),
    если известны сопротивления всех резисторов: $R_1 = 2\,\text{Ом}$, $R_2 = 5\,\text{Ом}$, $R_3 = 1\,\text{Ом}$, $R_4 = 3\,\text{Ом}$.
    При каком напряжении поданном на эту цепь, в ней потечёт ток равный $\eli = 10\,\text{А}$?

    \begin{tikzpicture}[rotate=90, circuit ee IEC, thick]
        \node [contact]  (contact1) at (-1.5, 0) {};
        \draw  (0, 0) to [resistor={info=$R_1$}] ++(left:1.5);
        \draw  (0, 0) -- ++(up:1.5) to [resistor={near start, info=$R_2$}, resistor={near end, info=$R_3$}] ++(right:3);
        \draw  (0, 0) to [resistor={info=$R_4$}] ++(right:3) -- ++(up:1.5);
        \draw  (3, 1.5) -- ++(right:0.5); \node [contact] (contact2) at (3.5, 1.5) {};
    \end{tikzpicture}
}
\answer{%
    $R=4\units{Ом} \approx 4{,}00\,\text{Ом} \implies U = \eli R \approx 40{,}0\,\text{В}.$
}
\solutionspace{120pt}

\tasknumber{2}%
\task{%
    Определите показания амперметра $3$ (см.
    рис.) и разность потенциалов на резисторе $2$,
    если сопротивления всех резисторов равны: $R_1 = R_2 = R_3 = R_4 = R_5 = R_6 = R = 5\,\text{Ом}$,
    а напряжение, поданное на цепь, равно $U = 30\,\text{В}$.
    Ответы получите в виде несократимых дробей, а также определите приближённые значения.
    Амперметры считать идеальными.

    \begin{tikzpicture}[circuit ee IEC, thick]
        \node [contact]  (left contact) at (3, 0) {};
        \node [contact]  (right contact) at (9, 0) {};
        \draw  (left contact) -- ++(up:2) to [resistor={very near start, info=$R_2$}, amperemeter={midway, info=$1$}, resistor={very near end, info=$R_3$} ] ++(right:6) -- (right contact);
        \draw  (left contact) -- ++(down:2) to [resistor={very near start, info=$R_5$}, resistor={midway, info=$R_6$}, amperemeter={very near end, info=$3$}] ++(right:6) -- (right contact);
        \draw  (left contact) ++(left:3) to [resistor={info=$R_1$}] (left contact) to [amperemeter={near start, info=$2$}, resistor={near end , info=$R_4$}] (right contact) -- ++(right:0.5);
    \end{tikzpicture}
}
\answer{%
    \begin{align*}
    R_0 &= R + \frac 1{\frac 1{R+R} + \frac 1R + \frac 1{R+R}} = R + \frac 1{\frac 2R} = \frac 32 R, \\
    \eli &= \frac U{R_0} = \frac {2U}{3R}, \\
    U_1 &= \eli R_1 = \frac {2U}{3R} \cdot R = \frac 23 U = 20{,}0\,\text{В}, \\
    U_{23} &= U_{56} = U_4 = U - \eli R_1 = U - \frac {2U}{3R} \cdot R = \frac U3 = 10{,}0\,\text{В}, \\
    \eli_2 &= \frac{U_4}{R_4} = \frac U{3R} \approx 2{,}0\,\text{А}, \\
    \eli_1 &= \frac{U_{23}}{R_{23}} = \frac{\frac U3}{R+R} = \frac U{6R} \approx 1{,}0\,\text{А}, \\
    \eli_3 &= \frac{U_{56}}{R_{56}} = \frac{\frac U3}{R+R} = \frac U{6R} \approx 1{,}0\,\text{А}, \\
    U_2 &= \eli_1 R_2 = \frac U{6R} \cdot R = \frac U6 = 5{,}0\,\text{В}, \\
    U_3 &= \eli_1 R_3 = \frac U{6R} \cdot R = \frac U3 = 5{,}0\,\text{В}, \\
    U_5 &= \eli_3 R_5 = \frac U{6R} \cdot R = \frac U5 = 5{,}0\,\text{В}, \\
    U_6 &= \eli_3 R_6 = \frac U{6R} \cdot R = \frac U6 = 5{,}0\,\text{В}.
    \end{align*}
}
\solutionspace{120pt}

\tasknumber{3}%
\task{%
    Определите ток, протекающий через резистор $R = 12\,\text{Ом}$ и разность потенциалов на нём (см.
    рис.),
    если $\ele_1 = 18\,\text{В}$, $\ele_2 = 25\,\text{В}$, $r_1 = 3\,\text{Ом}$, $r_2 = 4\,\text{Ом}$.

    \begin{tikzpicture}[circuit ee IEC, thick]
        \draw  (0, 0) to [battery={rotate=-180,info={$\ele_1, r_1$}}] (0, 3)
                -- (5, 3)
                to [battery={rotate=-180, info'={$\ele_2, r_2$}}] (5, 0)
                -- (0, 0)
                (2.5, 0) to [resistor={info=$R$}] (2.5, 3);
    \end{tikzpicture}
}
\answer{%
    Выберем 2 контура и один узел, запишем для них законы Кирхгофа:

    \begin{tikzpicture}[circuit ee IEC, thick]
        \draw  (0, 0) to [battery={rotate=-180,info={$\ele_1, r_1$}}, current direction={near end, info=$\eli_1$}] (0, 3)
                -- (5, 3)
                to [battery={rotate=-180, info'={$\ele_2, r_2$}}, current direction={near end, info=$\eli_2$}] (5, 0)
                -- (0, 0)
                (2.5, 0) to [resistor={info=$R$}, current direction'={near end, info=$\eli$}] (2.5, 3);
        \draw [-{Latex},color=red] (0.8, 1.9) arc [start angle = 135, end angle = -160, radius = 0.6];
        \draw [-{Latex},color=blue] (3.5, 1.9) arc [start angle = 135, end angle = -160, radius = 0.6];
        \node [contact,color=green!71!black] (topc) at (2.5, 3) {};
        \node [above] (top) at (2.5, 3) {$1$};
    \end{tikzpicture}

    \begin{align*}
        &\begin{cases}
            {\color{red} \ele_1 = \eli_1 r_1 + \eli R}, \\
            {\color{blue} \ele_2 = \eli_2 r_2 - \eli R}, \\
            {\color{green!71!black} \eli - \eli_1 - \eli_2 = 0};
        \end{cases}
        \qquad \implies \qquad
        \begin{cases}
            \eli_1 = \frac{\ele_1 - \eli R}{r_1}, \\
            \eli_2 = \frac{\ele_2 + \eli R}{r_2}, \\
            \eli - \eli_1 - \eli_2 = 0;
        \end{cases} \implies \\
        &\implies \eli - \frac{\ele_1 - \eli R}{r_1} + \frac{\ele_2 + \eli R}{r_2} = 0, \\
        &\eli\cbr{ 1 + \frac{R}{r_1} + \frac{R}{r_2}} - \frac{\ele_1}{r_1} + \frac{\ele_2}{r_2} = 0, \\
        &\eli
            = \frac{\frac{\ele_1}{r_1} - \frac{\ele_2}{r_2}}{ 1 + \frac{R}{r_1} + \frac{R}{r_2}}
            = \frac{\frac{18\,\text{В}}{3\,\text{Ом}} - \frac{25\,\text{В}}{4\,\text{Ом}}}{ 1 + \frac{12\,\text{Ом}}{3\,\text{Ом}} + \frac{12\,\text{Ом}}{4\,\text{Ом}}}
            = -\frac1{32}\units{А}
            \approx -0{,}031000\,\text{А}, \\
        &U  = \eli R = \frac{\frac{\ele_1}{r_1} - \frac{\ele_2}{r_2}}{ 1 + \frac{R}{r_1} + \frac{R}{r_2}} R
            \approx -0{,}37500\,\text{В}.
    \end{align*}
}

\variantsplitter

\addpersonalvariant{Анна Кузьмичёва}

\tasknumber{1}%
\task{%
    Определите эквивалентное сопротивление цепи на рисунке (между выделенными на рисунке контактами),
    если известны сопротивления всех резисторов: $R_1 = 1\,\text{Ом}$, $R_2 = 3\,\text{Ом}$, $R_3 = 3\,\text{Ом}$, $R_4 = 3\,\text{Ом}$.
    При каком напряжении поданном на эту цепь, в ней потечёт ток равный $\eli = 5\,\text{А}$?

    \begin{tikzpicture}[rotate=270, circuit ee IEC, thick]
        \node [contact]  (contact1) at (-1.5, 0) {};
        \draw  (0, 0) to [resistor={info=$R_1$}] ++(left:1.5);
        \draw  (0, 0) -- ++(up:1.5) to [resistor={near start, info=$R_2$}, resistor={near end, info=$R_3$}] ++(right:3);
        \draw  (0, 0) to [resistor={info=$R_4$}] ++(right:3) -- ++(up:1.5);
        \draw  (1.5, 1.5) -- ++(up:1); \node [contact] (contact2) at (1.5, 2.5) {};
    \end{tikzpicture}
}
\answer{%
    $R=3\units{Ом} \approx 3{,}00\,\text{Ом} \implies U = \eli R \approx 15{,}0\,\text{В}.$
}
\solutionspace{120pt}

\tasknumber{2}%
\task{%
    Определите показания амперметра $2$ (см.
    рис.) и разность потенциалов на резисторе $5$,
    если сопротивления всех резисторов равны: $R_1 = R_2 = R_3 = R_4 = R_5 = R_6 = R = 5\,\text{Ом}$,
    а напряжение, поданное на цепь, равно $U = 120\,\text{В}$.
    Ответы получите в виде несократимых дробей, а также определите приближённые значения.
    Амперметры считать идеальными.

    \begin{tikzpicture}[circuit ee IEC, thick]
        \node [contact]  (left contact) at (3, 0) {};
        \node [contact]  (right contact) at (9, 0) {};
        \draw  (left contact) -- ++(up:2) to [resistor={very near start, info=$R_2$}, amperemeter={midway, info=$1$}, resistor={very near end, info=$R_3$} ] ++(right:6) -- (right contact);
        \draw  (left contact) -- ++(down:2) to [resistor={very near start, info=$R_5$}, resistor={midway, info=$R_6$}, amperemeter={very near end, info=$3$}] ++(right:6) -- (right contact);
        \draw  (left contact) ++(left:3) to [resistor={info=$R_1$}] (left contact) to [amperemeter={near start, info=$2$}, resistor={near end , info=$R_4$}] (right contact) -- ++(right:0.5);
    \end{tikzpicture}
}
\answer{%
    \begin{align*}
    R_0 &= R + \frac 1{\frac 1{R+R} + \frac 1R + \frac 1{R+R}} = R + \frac 1{\frac 2R} = \frac 32 R, \\
    \eli &= \frac U{R_0} = \frac {2U}{3R}, \\
    U_1 &= \eli R_1 = \frac {2U}{3R} \cdot R = \frac 23 U = 80{,}0\,\text{В}, \\
    U_{23} &= U_{56} = U_4 = U - \eli R_1 = U - \frac {2U}{3R} \cdot R = \frac U3 = 40{,}0\,\text{В}, \\
    \eli_2 &= \frac{U_4}{R_4} = \frac U{3R} \approx 8{,}0\,\text{А}, \\
    \eli_1 &= \frac{U_{23}}{R_{23}} = \frac{\frac U3}{R+R} = \frac U{6R} \approx 4{,}0\,\text{А}, \\
    \eli_3 &= \frac{U_{56}}{R_{56}} = \frac{\frac U3}{R+R} = \frac U{6R} \approx 4{,}0\,\text{А}, \\
    U_2 &= \eli_1 R_2 = \frac U{6R} \cdot R = \frac U6 = 20{,}0\,\text{В}, \\
    U_3 &= \eli_1 R_3 = \frac U{6R} \cdot R = \frac U3 = 20{,}0\,\text{В}, \\
    U_5 &= \eli_3 R_5 = \frac U{6R} \cdot R = \frac U5 = 20{,}0\,\text{В}, \\
    U_6 &= \eli_3 R_6 = \frac U{6R} \cdot R = \frac U6 = 20{,}0\,\text{В}.
    \end{align*}
}
\solutionspace{120pt}

\tasknumber{3}%
\task{%
    Определите ток, протекающий через резистор $R = 10\,\text{Ом}$ и разность потенциалов на нём (см.
    рис.),
    если $\ele_1 = 18\,\text{В}$, $\ele_2 = 25\,\text{В}$, $r_1 = 2\,\text{Ом}$, $r_2 = 2\,\text{Ом}$.

    \begin{tikzpicture}[circuit ee IEC, thick]
        \draw  (0, 0) to [battery={rotate=-180,info={$\ele_1, r_1$}}] (0, 3)
                -- (5, 3)
                to [battery={rotate=-180, info'={$\ele_2, r_2$}}] (5, 0)
                -- (0, 0)
                (2.5, 0) to [resistor={info=$R$}] (2.5, 3);
    \end{tikzpicture}
}
\answer{%
    Выберем 2 контура и один узел, запишем для них законы Кирхгофа:

    \begin{tikzpicture}[circuit ee IEC, thick]
        \draw  (0, 0) to [battery={rotate=-180,info={$\ele_1, r_1$}}, current direction={near end, info=$\eli_1$}] (0, 3)
                -- (5, 3)
                to [battery={rotate=-180, info'={$\ele_2, r_2$}}, current direction={near end, info=$\eli_2$}] (5, 0)
                -- (0, 0)
                (2.5, 0) to [resistor={info=$R$}, current direction'={near end, info=$\eli$}] (2.5, 3);
        \draw [-{Latex},color=red] (0.8, 1.9) arc [start angle = 135, end angle = -160, radius = 0.6];
        \draw [-{Latex},color=blue] (3.5, 1.9) arc [start angle = 135, end angle = -160, radius = 0.6];
        \node [contact,color=green!71!black] (topc) at (2.5, 3) {};
        \node [above] (top) at (2.5, 3) {$1$};
    \end{tikzpicture}

    \begin{align*}
        &\begin{cases}
            {\color{red} \ele_1 = \eli_1 r_1 + \eli R}, \\
            {\color{blue} \ele_2 = \eli_2 r_2 - \eli R}, \\
            {\color{green!71!black} \eli - \eli_1 - \eli_2 = 0};
        \end{cases}
        \qquad \implies \qquad
        \begin{cases}
            \eli_1 = \frac{\ele_1 - \eli R}{r_1}, \\
            \eli_2 = \frac{\ele_2 + \eli R}{r_2}, \\
            \eli - \eli_1 - \eli_2 = 0;
        \end{cases} \implies \\
        &\implies \eli - \frac{\ele_1 - \eli R}{r_1} + \frac{\ele_2 + \eli R}{r_2} = 0, \\
        &\eli\cbr{ 1 + \frac{R}{r_1} + \frac{R}{r_2}} - \frac{\ele_1}{r_1} + \frac{\ele_2}{r_2} = 0, \\
        &\eli
            = \frac{\frac{\ele_1}{r_1} - \frac{\ele_2}{r_2}}{ 1 + \frac{R}{r_1} + \frac{R}{r_2}}
            = \frac{\frac{18\,\text{В}}{2\,\text{Ом}} - \frac{25\,\text{В}}{2\,\text{Ом}}}{ 1 + \frac{10\,\text{Ом}}{2\,\text{Ом}} + \frac{10\,\text{Ом}}{2\,\text{Ом}}}
            = -\frac7{22}\units{А}
            \approx -0{,}31800\,\text{А}, \\
        &U  = \eli R = \frac{\frac{\ele_1}{r_1} - \frac{\ele_2}{r_2}}{ 1 + \frac{R}{r_1} + \frac{R}{r_2}} R
            \approx -3{,}1820\,\text{В}.
    \end{align*}
}

\variantsplitter

\addpersonalvariant{Алёна Куприянова}

\tasknumber{1}%
\task{%
    Определите эквивалентное сопротивление цепи на рисунке (между выделенными на рисунке контактами),
    если известны сопротивления всех резисторов: $R_1 = 1\,\text{Ом}$, $R_2 = 3\,\text{Ом}$, $R_3 = 3\,\text{Ом}$, $R_4 = 3\,\text{Ом}$.
    При каком напряжении поданном на эту цепь, в ней потечёт ток равный $\eli = 5\,\text{А}$?

    \begin{tikzpicture}[rotate=270, circuit ee IEC, thick]
        \node [contact]  (contact1) at (-1.5, 0) {};
        \draw  (0, 0) to [resistor={info=$R_1$}] ++(left:1.5);
        \draw  (0, 0) -- ++(up:1.5) to [resistor={near start, info=$R_2$}, resistor={near end, info=$R_3$}] ++(right:3);
        \draw  (0, 0) to [resistor={info=$R_4$}] ++(right:3) -- ++(up:1.5);
        \draw  (1.5, 1.5) -- ++(up:1); \node [contact] (contact2) at (1.5, 2.5) {};
    \end{tikzpicture}
}
\answer{%
    $R=3\units{Ом} \approx 3{,}00\,\text{Ом} \implies U = \eli R \approx 15{,}0\,\text{В}.$
}
\solutionspace{120pt}

\tasknumber{2}%
\task{%
    Определите показания амперметра $3$ (см.
    рис.) и разность потенциалов на резисторе $5$,
    если сопротивления всех резисторов равны: $R_1 = R_2 = R_3 = R_4 = R_5 = R_6 = R = 10\,\text{Ом}$,
    а напряжение, поданное на цепь, равно $U = 30\,\text{В}$.
    Ответы получите в виде несократимых дробей, а также определите приближённые значения.
    Амперметры считать идеальными.

    \begin{tikzpicture}[circuit ee IEC, thick]
        \node [contact]  (left contact) at (3, 0) {};
        \node [contact]  (right contact) at (9, 0) {};
        \draw  (left contact) -- ++(up:2) to [resistor={very near start, info=$R_2$}, amperemeter={midway, info=$1$}, resistor={very near end, info=$R_3$} ] ++(right:6) -- (right contact);
        \draw  (left contact) -- ++(down:2) to [resistor={very near start, info=$R_5$}, resistor={midway, info=$R_6$}, amperemeter={very near end, info=$3$}] ++(right:6) -- (right contact);
        \draw  (left contact) ++(left:3) to [resistor={info=$R_1$}] (left contact) to [amperemeter={near start, info=$2$}, resistor={near end , info=$R_4$}] (right contact) -- ++(right:0.5);
    \end{tikzpicture}
}
\answer{%
    \begin{align*}
    R_0 &= R + \frac 1{\frac 1{R+R} + \frac 1R + \frac 1{R+R}} = R + \frac 1{\frac 2R} = \frac 32 R, \\
    \eli &= \frac U{R_0} = \frac {2U}{3R}, \\
    U_1 &= \eli R_1 = \frac {2U}{3R} \cdot R = \frac 23 U = 20{,}0\,\text{В}, \\
    U_{23} &= U_{56} = U_4 = U - \eli R_1 = U - \frac {2U}{3R} \cdot R = \frac U3 = 10{,}0\,\text{В}, \\
    \eli_2 &= \frac{U_4}{R_4} = \frac U{3R} \approx 1{,}0\,\text{А}, \\
    \eli_1 &= \frac{U_{23}}{R_{23}} = \frac{\frac U3}{R+R} = \frac U{6R} \approx 0{,}5\,\text{А}, \\
    \eli_3 &= \frac{U_{56}}{R_{56}} = \frac{\frac U3}{R+R} = \frac U{6R} \approx 0{,}5\,\text{А}, \\
    U_2 &= \eli_1 R_2 = \frac U{6R} \cdot R = \frac U6 = 5{,}0\,\text{В}, \\
    U_3 &= \eli_1 R_3 = \frac U{6R} \cdot R = \frac U3 = 5{,}0\,\text{В}, \\
    U_5 &= \eli_3 R_5 = \frac U{6R} \cdot R = \frac U5 = 5{,}0\,\text{В}, \\
    U_6 &= \eli_3 R_6 = \frac U{6R} \cdot R = \frac U6 = 5{,}0\,\text{В}.
    \end{align*}
}
\solutionspace{120pt}

\tasknumber{3}%
\task{%
    Определите ток, протекающий через резистор $R = 20\,\text{Ом}$ и разность потенциалов на нём (см.
    рис.),
    если $\ele_1 = 6\,\text{В}$, $\ele_2 = 5\,\text{В}$, $r_1 = 1\,\text{Ом}$, $r_2 = 4\,\text{Ом}$.

    \begin{tikzpicture}[circuit ee IEC, thick]
        \draw  (0, 0) to [battery={rotate=-180,info={$\ele_1, r_1$}}] (0, 3)
                -- (5, 3)
                to [battery={rotate=-180, info'={$\ele_2, r_2$}}] (5, 0)
                -- (0, 0)
                (2.5, 0) to [resistor={info=$R$}] (2.5, 3);
    \end{tikzpicture}
}
\answer{%
    Выберем 2 контура и один узел, запишем для них законы Кирхгофа:

    \begin{tikzpicture}[circuit ee IEC, thick]
        \draw  (0, 0) to [battery={rotate=-180,info={$\ele_1, r_1$}}, current direction={near end, info=$\eli_1$}] (0, 3)
                -- (5, 3)
                to [battery={rotate=-180, info'={$\ele_2, r_2$}}, current direction={near end, info=$\eli_2$}] (5, 0)
                -- (0, 0)
                (2.5, 0) to [resistor={info=$R$}, current direction'={near end, info=$\eli$}] (2.5, 3);
        \draw [-{Latex},color=red] (0.8, 1.9) arc [start angle = 135, end angle = -160, radius = 0.6];
        \draw [-{Latex},color=blue] (3.5, 1.9) arc [start angle = 135, end angle = -160, radius = 0.6];
        \node [contact,color=green!71!black] (topc) at (2.5, 3) {};
        \node [above] (top) at (2.5, 3) {$1$};
    \end{tikzpicture}

    \begin{align*}
        &\begin{cases}
            {\color{red} \ele_1 = \eli_1 r_1 + \eli R}, \\
            {\color{blue} \ele_2 = \eli_2 r_2 - \eli R}, \\
            {\color{green!71!black} \eli - \eli_1 - \eli_2 = 0};
        \end{cases}
        \qquad \implies \qquad
        \begin{cases}
            \eli_1 = \frac{\ele_1 - \eli R}{r_1}, \\
            \eli_2 = \frac{\ele_2 + \eli R}{r_2}, \\
            \eli - \eli_1 - \eli_2 = 0;
        \end{cases} \implies \\
        &\implies \eli - \frac{\ele_1 - \eli R}{r_1} + \frac{\ele_2 + \eli R}{r_2} = 0, \\
        &\eli\cbr{ 1 + \frac{R}{r_1} + \frac{R}{r_2}} - \frac{\ele_1}{r_1} + \frac{\ele_2}{r_2} = 0, \\
        &\eli
            = \frac{\frac{\ele_1}{r_1} - \frac{\ele_2}{r_2}}{ 1 + \frac{R}{r_1} + \frac{R}{r_2}}
            = \frac{\frac{6\,\text{В}}{1\,\text{Ом}} - \frac{5\,\text{В}}{4\,\text{Ом}}}{ 1 + \frac{20\,\text{Ом}}{1\,\text{Ом}} + \frac{20\,\text{Ом}}{4\,\text{Ом}}}
            = \frac{19}{104}\units{А}
            \approx 0{,}183\,\text{А}, \\
        &U  = \eli R = \frac{\frac{\ele_1}{r_1} - \frac{\ele_2}{r_2}}{ 1 + \frac{R}{r_1} + \frac{R}{r_2}} R
            \approx 3{,}654\,\text{В}.
    \end{align*}
}

\variantsplitter

\addpersonalvariant{Ярослав Лавровский}

\tasknumber{1}%
\task{%
    Определите эквивалентное сопротивление цепи на рисунке (между выделенными на рисунке контактами),
    если известны сопротивления всех резисторов: $R_1 = 2\,\text{Ом}$, $R_2 = 5\,\text{Ом}$, $R_3 = 1\,\text{Ом}$, $R_4 = 4\,\text{Ом}$.
    При каком напряжении поданном на эту цепь, в ней потечёт ток равный $\eli = 2\,\text{А}$?

    \begin{tikzpicture}[rotate=270, circuit ee IEC, thick]
        \node [contact]  (contact1) at (-1.5, 0) {};
        \draw  (0, 0) to [resistor={info=$R_1$}] ++(left:1.5);
        \draw  (0, 0) -- ++(up:1.5) to [resistor={near start, info=$R_2$}, resistor={near end, info=$R_3$}] ++(right:3);
        \draw  (0, 0) to [resistor={info=$R_4$}] ++(right:3) -- ++(up:1.5);
        \draw  (1.5, 1.5) -- ++(up:1); \node [contact] (contact2) at (1.5, 2.5) {};
    \end{tikzpicture}
}
\answer{%
    $R=\frac92\units{Ом} \approx 4{,}50\,\text{Ом} \implies U = \eli R \approx 9{,}0\,\text{В}.$
}
\solutionspace{120pt}

\tasknumber{2}%
\task{%
    Определите показания амперметра $1$ (см.
    рис.) и разность потенциалов на резисторе $1$,
    если сопротивления всех резисторов равны: $R_1 = R_2 = R_3 = R_4 = R_5 = R_6 = R = 5\,\text{Ом}$,
    а напряжение, поданное на цепь, равно $U = 90\,\text{В}$.
    Ответы получите в виде несократимых дробей, а также определите приближённые значения.
    Амперметры считать идеальными.

    \begin{tikzpicture}[circuit ee IEC, thick]
        \node [contact]  (left contact) at (3, 0) {};
        \node [contact]  (right contact) at (9, 0) {};
        \draw  (left contact) -- ++(up:2) to [resistor={very near start, info=$R_2$}, amperemeter={midway, info=$1$}, resistor={very near end, info=$R_3$} ] ++(right:6) -- (right contact);
        \draw  (left contact) -- ++(down:2) to [resistor={very near start, info=$R_5$}, resistor={midway, info=$R_6$}, amperemeter={very near end, info=$3$}] ++(right:6) -- (right contact);
        \draw  (left contact) ++(left:3) to [resistor={info=$R_1$}] (left contact) to [amperemeter={near start, info=$2$}, resistor={near end , info=$R_4$}] (right contact) -- ++(right:0.5);
    \end{tikzpicture}
}
\answer{%
    \begin{align*}
    R_0 &= R + \frac 1{\frac 1{R+R} + \frac 1R + \frac 1{R+R}} = R + \frac 1{\frac 2R} = \frac 32 R, \\
    \eli &= \frac U{R_0} = \frac {2U}{3R}, \\
    U_1 &= \eli R_1 = \frac {2U}{3R} \cdot R = \frac 23 U = 60{,}0\,\text{В}, \\
    U_{23} &= U_{56} = U_4 = U - \eli R_1 = U - \frac {2U}{3R} \cdot R = \frac U3 = 30{,}0\,\text{В}, \\
    \eli_2 &= \frac{U_4}{R_4} = \frac U{3R} \approx 6{,}0\,\text{А}, \\
    \eli_1 &= \frac{U_{23}}{R_{23}} = \frac{\frac U3}{R+R} = \frac U{6R} \approx 3{,}0\,\text{А}, \\
    \eli_3 &= \frac{U_{56}}{R_{56}} = \frac{\frac U3}{R+R} = \frac U{6R} \approx 3{,}0\,\text{А}, \\
    U_2 &= \eli_1 R_2 = \frac U{6R} \cdot R = \frac U6 = 15{,}0\,\text{В}, \\
    U_3 &= \eli_1 R_3 = \frac U{6R} \cdot R = \frac U3 = 15{,}0\,\text{В}, \\
    U_5 &= \eli_3 R_5 = \frac U{6R} \cdot R = \frac U5 = 15{,}0\,\text{В}, \\
    U_6 &= \eli_3 R_6 = \frac U{6R} \cdot R = \frac U6 = 15{,}0\,\text{В}.
    \end{align*}
}
\solutionspace{120pt}

\tasknumber{3}%
\task{%
    Определите ток, протекающий через резистор $R = 10\,\text{Ом}$ и разность потенциалов на нём (см.
    рис.),
    если $\ele_1 = 18\,\text{В}$, $\ele_2 = 5\,\text{В}$, $r_1 = 3\,\text{Ом}$, $r_2 = 2\,\text{Ом}$.

    \begin{tikzpicture}[circuit ee IEC, thick]
        \draw  (0, 0) to [battery={rotate=-180,info={$\ele_1, r_1$}}] (0, 3)
                -- (5, 3)
                to [battery={rotate=-180, info'={$\ele_2, r_2$}}] (5, 0)
                -- (0, 0)
                (2.5, 0) to [resistor={info=$R$}] (2.5, 3);
    \end{tikzpicture}
}
\answer{%
    Выберем 2 контура и один узел, запишем для них законы Кирхгофа:

    \begin{tikzpicture}[circuit ee IEC, thick]
        \draw  (0, 0) to [battery={rotate=-180,info={$\ele_1, r_1$}}, current direction={near end, info=$\eli_1$}] (0, 3)
                -- (5, 3)
                to [battery={rotate=-180, info'={$\ele_2, r_2$}}, current direction={near end, info=$\eli_2$}] (5, 0)
                -- (0, 0)
                (2.5, 0) to [resistor={info=$R$}, current direction'={near end, info=$\eli$}] (2.5, 3);
        \draw [-{Latex},color=red] (0.8, 1.9) arc [start angle = 135, end angle = -160, radius = 0.6];
        \draw [-{Latex},color=blue] (3.5, 1.9) arc [start angle = 135, end angle = -160, radius = 0.6];
        \node [contact,color=green!71!black] (topc) at (2.5, 3) {};
        \node [above] (top) at (2.5, 3) {$1$};
    \end{tikzpicture}

    \begin{align*}
        &\begin{cases}
            {\color{red} \ele_1 = \eli_1 r_1 + \eli R}, \\
            {\color{blue} \ele_2 = \eli_2 r_2 - \eli R}, \\
            {\color{green!71!black} \eli - \eli_1 - \eli_2 = 0};
        \end{cases}
        \qquad \implies \qquad
        \begin{cases}
            \eli_1 = \frac{\ele_1 - \eli R}{r_1}, \\
            \eli_2 = \frac{\ele_2 + \eli R}{r_2}, \\
            \eli - \eli_1 - \eli_2 = 0;
        \end{cases} \implies \\
        &\implies \eli - \frac{\ele_1 - \eli R}{r_1} + \frac{\ele_2 + \eli R}{r_2} = 0, \\
        &\eli\cbr{ 1 + \frac{R}{r_1} + \frac{R}{r_2}} - \frac{\ele_1}{r_1} + \frac{\ele_2}{r_2} = 0, \\
        &\eli
            = \frac{\frac{\ele_1}{r_1} - \frac{\ele_2}{r_2}}{ 1 + \frac{R}{r_1} + \frac{R}{r_2}}
            = \frac{\frac{18\,\text{В}}{3\,\text{Ом}} - \frac{5\,\text{В}}{2\,\text{Ом}}}{ 1 + \frac{10\,\text{Ом}}{3\,\text{Ом}} + \frac{10\,\text{Ом}}{2\,\text{Ом}}}
            = \frac38\units{А}
            \approx 0{,}375\,\text{А}, \\
        &U  = \eli R = \frac{\frac{\ele_1}{r_1} - \frac{\ele_2}{r_2}}{ 1 + \frac{R}{r_1} + \frac{R}{r_2}} R
            \approx 3{,}750\,\text{В}.
    \end{align*}
}

\variantsplitter

\addpersonalvariant{Анастасия Ламанова}

\tasknumber{1}%
\task{%
    Определите эквивалентное сопротивление цепи на рисунке (между выделенными на рисунке контактами),
    если известны сопротивления всех резисторов: $R_1 = 2\,\text{Ом}$, $R_2 = 4\,\text{Ом}$, $R_3 = 1\,\text{Ом}$, $R_4 = 4\,\text{Ом}$.
    При каком напряжении поданном на эту цепь, в ней потечёт ток равный $\eli = 5\,\text{А}$?

    \begin{tikzpicture}[rotate=270, circuit ee IEC, thick]
        \node [contact]  (contact1) at (-1.5, 0) {};
        \draw  (0, 0) to [resistor={info=$R_1$}] ++(left:1.5);
        \draw  (0, 0) -- ++(up:1.5) to [resistor={near start, info=$R_2$}, resistor={near end, info=$R_3$}] ++(right:3);
        \draw  (0, 0) to [resistor={info=$R_4$}] ++(right:3) -- ++(up:1.5);
        \draw  (1.5, 1.5) -- ++(up:1); \node [contact] (contact2) at (1.5, 2.5) {};
    \end{tikzpicture}
}
\answer{%
    $R=\frac{38}9\units{Ом} \approx 4{,}22\,\text{Ом} \implies U = \eli R \approx 21{,}1\,\text{В}.$
}
\solutionspace{120pt}

\tasknumber{2}%
\task{%
    Определите показания амперметра $1$ (см.
    рис.) и разность потенциалов на резисторе $1$,
    если сопротивления всех резисторов равны: $R_1 = R_2 = R_3 = R_4 = R_5 = R_6 = R = 5\,\text{Ом}$,
    а напряжение, поданное на цепь, равно $U = 150\,\text{В}$.
    Ответы получите в виде несократимых дробей, а также определите приближённые значения.
    Амперметры считать идеальными.

    \begin{tikzpicture}[circuit ee IEC, thick]
        \node [contact]  (left contact) at (3, 0) {};
        \node [contact]  (right contact) at (9, 0) {};
        \draw  (left contact) -- ++(up:2) to [resistor={very near start, info=$R_2$}, amperemeter={midway, info=$1$}, resistor={very near end, info=$R_3$} ] ++(right:6) -- (right contact);
        \draw  (left contact) -- ++(down:2) to [resistor={very near start, info=$R_5$}, resistor={midway, info=$R_6$}, amperemeter={very near end, info=$3$}] ++(right:6) -- (right contact);
        \draw  (left contact) ++(left:3) to [resistor={info=$R_1$}] (left contact) to [amperemeter={near start, info=$2$}, resistor={near end , info=$R_4$}] (right contact) -- ++(right:0.5);
    \end{tikzpicture}
}
\answer{%
    \begin{align*}
    R_0 &= R + \frac 1{\frac 1{R+R} + \frac 1R + \frac 1{R+R}} = R + \frac 1{\frac 2R} = \frac 32 R, \\
    \eli &= \frac U{R_0} = \frac {2U}{3R}, \\
    U_1 &= \eli R_1 = \frac {2U}{3R} \cdot R = \frac 23 U = 100{,}0\,\text{В}, \\
    U_{23} &= U_{56} = U_4 = U - \eli R_1 = U - \frac {2U}{3R} \cdot R = \frac U3 = 50{,}0\,\text{В}, \\
    \eli_2 &= \frac{U_4}{R_4} = \frac U{3R} \approx 10{,}0\,\text{А}, \\
    \eli_1 &= \frac{U_{23}}{R_{23}} = \frac{\frac U3}{R+R} = \frac U{6R} \approx 5{,}0\,\text{А}, \\
    \eli_3 &= \frac{U_{56}}{R_{56}} = \frac{\frac U3}{R+R} = \frac U{6R} \approx 5{,}0\,\text{А}, \\
    U_2 &= \eli_1 R_2 = \frac U{6R} \cdot R = \frac U6 = 25{,}0\,\text{В}, \\
    U_3 &= \eli_1 R_3 = \frac U{6R} \cdot R = \frac U3 = 25{,}0\,\text{В}, \\
    U_5 &= \eli_3 R_5 = \frac U{6R} \cdot R = \frac U5 = 25{,}0\,\text{В}, \\
    U_6 &= \eli_3 R_6 = \frac U{6R} \cdot R = \frac U6 = 25{,}0\,\text{В}.
    \end{align*}
}
\solutionspace{120pt}

\tasknumber{3}%
\task{%
    Определите ток, протекающий через резистор $R = 20\,\text{Ом}$ и разность потенциалов на нём (см.
    рис.),
    если $\ele_1 = 18\,\text{В}$, $\ele_2 = 5\,\text{В}$, $r_1 = 2\,\text{Ом}$, $r_2 = 4\,\text{Ом}$.

    \begin{tikzpicture}[circuit ee IEC, thick]
        \draw  (0, 0) to [battery={rotate=-180,info={$\ele_1, r_1$}}] (0, 3)
                -- (5, 3)
                to [battery={rotate=-180, info'={$\ele_2, r_2$}}] (5, 0)
                -- (0, 0)
                (2.5, 0) to [resistor={info=$R$}] (2.5, 3);
    \end{tikzpicture}
}
\answer{%
    Выберем 2 контура и один узел, запишем для них законы Кирхгофа:

    \begin{tikzpicture}[circuit ee IEC, thick]
        \draw  (0, 0) to [battery={rotate=-180,info={$\ele_1, r_1$}}, current direction={near end, info=$\eli_1$}] (0, 3)
                -- (5, 3)
                to [battery={rotate=-180, info'={$\ele_2, r_2$}}, current direction={near end, info=$\eli_2$}] (5, 0)
                -- (0, 0)
                (2.5, 0) to [resistor={info=$R$}, current direction'={near end, info=$\eli$}] (2.5, 3);
        \draw [-{Latex},color=red] (0.8, 1.9) arc [start angle = 135, end angle = -160, radius = 0.6];
        \draw [-{Latex},color=blue] (3.5, 1.9) arc [start angle = 135, end angle = -160, radius = 0.6];
        \node [contact,color=green!71!black] (topc) at (2.5, 3) {};
        \node [above] (top) at (2.5, 3) {$1$};
    \end{tikzpicture}

    \begin{align*}
        &\begin{cases}
            {\color{red} \ele_1 = \eli_1 r_1 + \eli R}, \\
            {\color{blue} \ele_2 = \eli_2 r_2 - \eli R}, \\
            {\color{green!71!black} \eli - \eli_1 - \eli_2 = 0};
        \end{cases}
        \qquad \implies \qquad
        \begin{cases}
            \eli_1 = \frac{\ele_1 - \eli R}{r_1}, \\
            \eli_2 = \frac{\ele_2 + \eli R}{r_2}, \\
            \eli - \eli_1 - \eli_2 = 0;
        \end{cases} \implies \\
        &\implies \eli - \frac{\ele_1 - \eli R}{r_1} + \frac{\ele_2 + \eli R}{r_2} = 0, \\
        &\eli\cbr{ 1 + \frac{R}{r_1} + \frac{R}{r_2}} - \frac{\ele_1}{r_1} + \frac{\ele_2}{r_2} = 0, \\
        &\eli
            = \frac{\frac{\ele_1}{r_1} - \frac{\ele_2}{r_2}}{ 1 + \frac{R}{r_1} + \frac{R}{r_2}}
            = \frac{\frac{18\,\text{В}}{2\,\text{Ом}} - \frac{5\,\text{В}}{4\,\text{Ом}}}{ 1 + \frac{20\,\text{Ом}}{2\,\text{Ом}} + \frac{20\,\text{Ом}}{4\,\text{Ом}}}
            = \frac{31}{64}\units{А}
            \approx 0{,}484\,\text{А}, \\
        &U  = \eli R = \frac{\frac{\ele_1}{r_1} - \frac{\ele_2}{r_2}}{ 1 + \frac{R}{r_1} + \frac{R}{r_2}} R
            \approx 9{,}688\,\text{В}.
    \end{align*}
}

\variantsplitter

\addpersonalvariant{Виктория Легонькова}

\tasknumber{1}%
\task{%
    Определите эквивалентное сопротивление цепи на рисунке (между выделенными на рисунке контактами),
    если известны сопротивления всех резисторов: $R_1 = 2\,\text{Ом}$, $R_2 = 3\,\text{Ом}$, $R_3 = 1\,\text{Ом}$, $R_4 = 4\,\text{Ом}$.
    При каком напряжении поданном на эту цепь, в ней потечёт ток равный $\eli = 5\,\text{А}$?

    \begin{tikzpicture}[rotate=180, circuit ee IEC, thick]
        \node [contact]  (contact1) at (-1.5, 0) {};
        \draw  (0, 0) to [resistor={info=$R_1$}] ++(left:1.5);
        \draw  (0, 0) -- ++(up:1.5) to [resistor={near start, info=$R_2$}, resistor={near end, info=$R_3$}] ++(right:3);
        \draw  (0, 0) to [resistor={info=$R_4$}] ++(right:3) -- ++(up:1.5);
        \draw  (1.5, 1.5) -- ++(up:1); \node [contact] (contact2) at (1.5, 2.5) {};
    \end{tikzpicture}
}
\answer{%
    $R=\frac{31}8\units{Ом} \approx 3{,}88\,\text{Ом} \implies U = \eli R \approx 19{,}4\,\text{В}.$
}
\solutionspace{120pt}

\tasknumber{2}%
\task{%
    Определите показания амперметра $3$ (см.
    рис.) и разность потенциалов на резисторе $3$,
    если сопротивления всех резисторов равны: $R_1 = R_2 = R_3 = R_4 = R_5 = R_6 = R = 2\,\text{Ом}$,
    а напряжение, поданное на цепь, равно $U = 90\,\text{В}$.
    Ответы получите в виде несократимых дробей, а также определите приближённые значения.
    Амперметры считать идеальными.

    \begin{tikzpicture}[circuit ee IEC, thick]
        \node [contact]  (left contact) at (3, 0) {};
        \node [contact]  (right contact) at (9, 0) {};
        \draw  (left contact) -- ++(up:2) to [resistor={very near start, info=$R_2$}, amperemeter={midway, info=$1$}, resistor={very near end, info=$R_3$} ] ++(right:6) -- (right contact);
        \draw  (left contact) -- ++(down:2) to [resistor={very near start, info=$R_5$}, resistor={midway, info=$R_6$}, amperemeter={very near end, info=$3$}] ++(right:6) -- (right contact);
        \draw  (left contact) ++(left:3) to [resistor={info=$R_1$}] (left contact) to [amperemeter={near start, info=$2$}, resistor={near end , info=$R_4$}] (right contact) -- ++(right:0.5);
    \end{tikzpicture}
}
\answer{%
    \begin{align*}
    R_0 &= R + \frac 1{\frac 1{R+R} + \frac 1R + \frac 1{R+R}} = R + \frac 1{\frac 2R} = \frac 32 R, \\
    \eli &= \frac U{R_0} = \frac {2U}{3R}, \\
    U_1 &= \eli R_1 = \frac {2U}{3R} \cdot R = \frac 23 U = 60{,}0\,\text{В}, \\
    U_{23} &= U_{56} = U_4 = U - \eli R_1 = U - \frac {2U}{3R} \cdot R = \frac U3 = 30{,}0\,\text{В}, \\
    \eli_2 &= \frac{U_4}{R_4} = \frac U{3R} \approx 15{,}0\,\text{А}, \\
    \eli_1 &= \frac{U_{23}}{R_{23}} = \frac{\frac U3}{R+R} = \frac U{6R} \approx 7{,}5\,\text{А}, \\
    \eli_3 &= \frac{U_{56}}{R_{56}} = \frac{\frac U3}{R+R} = \frac U{6R} \approx 7{,}5\,\text{А}, \\
    U_2 &= \eli_1 R_2 = \frac U{6R} \cdot R = \frac U6 = 15{,}0\,\text{В}, \\
    U_3 &= \eli_1 R_3 = \frac U{6R} \cdot R = \frac U3 = 15{,}0\,\text{В}, \\
    U_5 &= \eli_3 R_5 = \frac U{6R} \cdot R = \frac U5 = 15{,}0\,\text{В}, \\
    U_6 &= \eli_3 R_6 = \frac U{6R} \cdot R = \frac U6 = 15{,}0\,\text{В}.
    \end{align*}
}
\solutionspace{120pt}

\tasknumber{3}%
\task{%
    Определите ток, протекающий через резистор $R = 18\,\text{Ом}$ и разность потенциалов на нём (см.
    рис.),
    если $\ele_1 = 6\,\text{В}$, $\ele_2 = 5\,\text{В}$, $r_1 = 2\,\text{Ом}$, $r_2 = 6\,\text{Ом}$.

    \begin{tikzpicture}[circuit ee IEC, thick]
        \draw  (0, 0) to [battery={rotate=-180,info={$\ele_1, r_1$}}] (0, 3)
                -- (5, 3)
                to [battery={rotate=-180, info'={$\ele_2, r_2$}}] (5, 0)
                -- (0, 0)
                (2.5, 0) to [resistor={info=$R$}] (2.5, 3);
    \end{tikzpicture}
}
\answer{%
    Выберем 2 контура и один узел, запишем для них законы Кирхгофа:

    \begin{tikzpicture}[circuit ee IEC, thick]
        \draw  (0, 0) to [battery={rotate=-180,info={$\ele_1, r_1$}}, current direction={near end, info=$\eli_1$}] (0, 3)
                -- (5, 3)
                to [battery={rotate=-180, info'={$\ele_2, r_2$}}, current direction={near end, info=$\eli_2$}] (5, 0)
                -- (0, 0)
                (2.5, 0) to [resistor={info=$R$}, current direction'={near end, info=$\eli$}] (2.5, 3);
        \draw [-{Latex},color=red] (0.8, 1.9) arc [start angle = 135, end angle = -160, radius = 0.6];
        \draw [-{Latex},color=blue] (3.5, 1.9) arc [start angle = 135, end angle = -160, radius = 0.6];
        \node [contact,color=green!71!black] (topc) at (2.5, 3) {};
        \node [above] (top) at (2.5, 3) {$1$};
    \end{tikzpicture}

    \begin{align*}
        &\begin{cases}
            {\color{red} \ele_1 = \eli_1 r_1 + \eli R}, \\
            {\color{blue} \ele_2 = \eli_2 r_2 - \eli R}, \\
            {\color{green!71!black} \eli - \eli_1 - \eli_2 = 0};
        \end{cases}
        \qquad \implies \qquad
        \begin{cases}
            \eli_1 = \frac{\ele_1 - \eli R}{r_1}, \\
            \eli_2 = \frac{\ele_2 + \eli R}{r_2}, \\
            \eli - \eli_1 - \eli_2 = 0;
        \end{cases} \implies \\
        &\implies \eli - \frac{\ele_1 - \eli R}{r_1} + \frac{\ele_2 + \eli R}{r_2} = 0, \\
        &\eli\cbr{ 1 + \frac{R}{r_1} + \frac{R}{r_2}} - \frac{\ele_1}{r_1} + \frac{\ele_2}{r_2} = 0, \\
        &\eli
            = \frac{\frac{\ele_1}{r_1} - \frac{\ele_2}{r_2}}{ 1 + \frac{R}{r_1} + \frac{R}{r_2}}
            = \frac{\frac{6\,\text{В}}{2\,\text{Ом}} - \frac{5\,\text{В}}{6\,\text{Ом}}}{ 1 + \frac{18\,\text{Ом}}{2\,\text{Ом}} + \frac{18\,\text{Ом}}{6\,\text{Ом}}}
            = \frac16\units{А}
            \approx 0{,}167\,\text{А}, \\
        &U  = \eli R = \frac{\frac{\ele_1}{r_1} - \frac{\ele_2}{r_2}}{ 1 + \frac{R}{r_1} + \frac{R}{r_2}} R
            \approx 3{,}000\,\text{В}.
    \end{align*}
}

\variantsplitter

\addpersonalvariant{Семён Мартынов}

\tasknumber{1}%
\task{%
    Определите эквивалентное сопротивление цепи на рисунке (между выделенными на рисунке контактами),
    если известны сопротивления всех резисторов: $R_1 = 2\,\text{Ом}$, $R_2 = 4\,\text{Ом}$, $R_3 = 2\,\text{Ом}$, $R_4 = 4\,\text{Ом}$.
    При каком напряжении поданном на эту цепь, в ней потечёт ток равный $\eli = 2\,\text{А}$?

    \begin{tikzpicture}[rotate=0, circuit ee IEC, thick]
        \node [contact]  (contact1) at (-1.5, 0) {};
        \draw  (0, 0) to [resistor={info=$R_1$}] ++(left:1.5);
        \draw  (0, 0) -- ++(up:1.5) to [resistor={near start, info=$R_2$}, resistor={near end, info=$R_3$}] ++(right:3);
        \draw  (0, 0) to [resistor={info=$R_4$}] ++(right:3) -- ++(up:1.5);
        \draw  (3, 1.5) -- ++(right:0.5); \node [contact] (contact2) at (3.5, 1.5) {};
    \end{tikzpicture}
}
\answer{%
    $R=\frac{22}5\units{Ом} \approx 4{,}40\,\text{Ом} \implies U = \eli R \approx 8{,}8\,\text{В}.$
}
\solutionspace{120pt}

\tasknumber{2}%
\task{%
    Определите показания амперметра $1$ (см.
    рис.) и разность потенциалов на резисторе $1$,
    если сопротивления всех резисторов равны: $R_1 = R_2 = R_3 = R_4 = R_5 = R_6 = R = 10\,\text{Ом}$,
    а напряжение, поданное на цепь, равно $U = 150\,\text{В}$.
    Ответы получите в виде несократимых дробей, а также определите приближённые значения.
    Амперметры считать идеальными.

    \begin{tikzpicture}[circuit ee IEC, thick]
        \node [contact]  (left contact) at (3, 0) {};
        \node [contact]  (right contact) at (9, 0) {};
        \draw  (left contact) -- ++(up:2) to [resistor={very near start, info=$R_2$}, amperemeter={midway, info=$1$}, resistor={very near end, info=$R_3$} ] ++(right:6) -- (right contact);
        \draw  (left contact) -- ++(down:2) to [resistor={very near start, info=$R_5$}, resistor={midway, info=$R_6$}, amperemeter={very near end, info=$3$}] ++(right:6) -- (right contact);
        \draw  (left contact) ++(left:3) to [resistor={info=$R_1$}] (left contact) to [amperemeter={near start, info=$2$}, resistor={near end , info=$R_4$}] (right contact) -- ++(right:0.5);
    \end{tikzpicture}
}
\answer{%
    \begin{align*}
    R_0 &= R + \frac 1{\frac 1{R+R} + \frac 1R + \frac 1{R+R}} = R + \frac 1{\frac 2R} = \frac 32 R, \\
    \eli &= \frac U{R_0} = \frac {2U}{3R}, \\
    U_1 &= \eli R_1 = \frac {2U}{3R} \cdot R = \frac 23 U = 100{,}0\,\text{В}, \\
    U_{23} &= U_{56} = U_4 = U - \eli R_1 = U - \frac {2U}{3R} \cdot R = \frac U3 = 50{,}0\,\text{В}, \\
    \eli_2 &= \frac{U_4}{R_4} = \frac U{3R} \approx 5{,}0\,\text{А}, \\
    \eli_1 &= \frac{U_{23}}{R_{23}} = \frac{\frac U3}{R+R} = \frac U{6R} \approx 2{,}5\,\text{А}, \\
    \eli_3 &= \frac{U_{56}}{R_{56}} = \frac{\frac U3}{R+R} = \frac U{6R} \approx 2{,}5\,\text{А}, \\
    U_2 &= \eli_1 R_2 = \frac U{6R} \cdot R = \frac U6 = 25{,}0\,\text{В}, \\
    U_3 &= \eli_1 R_3 = \frac U{6R} \cdot R = \frac U3 = 25{,}0\,\text{В}, \\
    U_5 &= \eli_3 R_5 = \frac U{6R} \cdot R = \frac U5 = 25{,}0\,\text{В}, \\
    U_6 &= \eli_3 R_6 = \frac U{6R} \cdot R = \frac U6 = 25{,}0\,\text{В}.
    \end{align*}
}
\solutionspace{120pt}

\tasknumber{3}%
\task{%
    Определите ток, протекающий через резистор $R = 20\,\text{Ом}$ и разность потенциалов на нём (см.
    рис.),
    если $\ele_1 = 18\,\text{В}$, $\ele_2 = 15\,\text{В}$, $r_1 = 3\,\text{Ом}$, $r_2 = 4\,\text{Ом}$.

    \begin{tikzpicture}[circuit ee IEC, thick]
        \draw  (0, 0) to [battery={rotate=-180,info={$\ele_1, r_1$}}] (0, 3)
                -- (5, 3)
                to [battery={rotate=-180, info'={$\ele_2, r_2$}}] (5, 0)
                -- (0, 0)
                (2.5, 0) to [resistor={info=$R$}] (2.5, 3);
    \end{tikzpicture}
}
\answer{%
    Выберем 2 контура и один узел, запишем для них законы Кирхгофа:

    \begin{tikzpicture}[circuit ee IEC, thick]
        \draw  (0, 0) to [battery={rotate=-180,info={$\ele_1, r_1$}}, current direction={near end, info=$\eli_1$}] (0, 3)
                -- (5, 3)
                to [battery={rotate=-180, info'={$\ele_2, r_2$}}, current direction={near end, info=$\eli_2$}] (5, 0)
                -- (0, 0)
                (2.5, 0) to [resistor={info=$R$}, current direction'={near end, info=$\eli$}] (2.5, 3);
        \draw [-{Latex},color=red] (0.8, 1.9) arc [start angle = 135, end angle = -160, radius = 0.6];
        \draw [-{Latex},color=blue] (3.5, 1.9) arc [start angle = 135, end angle = -160, radius = 0.6];
        \node [contact,color=green!71!black] (topc) at (2.5, 3) {};
        \node [above] (top) at (2.5, 3) {$1$};
    \end{tikzpicture}

    \begin{align*}
        &\begin{cases}
            {\color{red} \ele_1 = \eli_1 r_1 + \eli R}, \\
            {\color{blue} \ele_2 = \eli_2 r_2 - \eli R}, \\
            {\color{green!71!black} \eli - \eli_1 - \eli_2 = 0};
        \end{cases}
        \qquad \implies \qquad
        \begin{cases}
            \eli_1 = \frac{\ele_1 - \eli R}{r_1}, \\
            \eli_2 = \frac{\ele_2 + \eli R}{r_2}, \\
            \eli - \eli_1 - \eli_2 = 0;
        \end{cases} \implies \\
        &\implies \eli - \frac{\ele_1 - \eli R}{r_1} + \frac{\ele_2 + \eli R}{r_2} = 0, \\
        &\eli\cbr{ 1 + \frac{R}{r_1} + \frac{R}{r_2}} - \frac{\ele_1}{r_1} + \frac{\ele_2}{r_2} = 0, \\
        &\eli
            = \frac{\frac{\ele_1}{r_1} - \frac{\ele_2}{r_2}}{ 1 + \frac{R}{r_1} + \frac{R}{r_2}}
            = \frac{\frac{18\,\text{В}}{3\,\text{Ом}} - \frac{15\,\text{В}}{4\,\text{Ом}}}{ 1 + \frac{20\,\text{Ом}}{3\,\text{Ом}} + \frac{20\,\text{Ом}}{4\,\text{Ом}}}
            = \frac{27}{152}\units{А}
            \approx 0{,}178\,\text{А}, \\
        &U  = \eli R = \frac{\frac{\ele_1}{r_1} - \frac{\ele_2}{r_2}}{ 1 + \frac{R}{r_1} + \frac{R}{r_2}} R
            \approx 3{,}553\,\text{В}.
    \end{align*}
}

\variantsplitter

\addpersonalvariant{Варвара Минаева}

\tasknumber{1}%
\task{%
    Определите эквивалентное сопротивление цепи на рисунке (между выделенными на рисунке контактами),
    если известны сопротивления всех резисторов: $R_1 = 1\,\text{Ом}$, $R_2 = 4\,\text{Ом}$, $R_3 = 3\,\text{Ом}$, $R_4 = 3\,\text{Ом}$.
    При каком напряжении поданном на эту цепь, в ней потечёт ток равный $\eli = 10\,\text{А}$?

    \begin{tikzpicture}[rotate=90, circuit ee IEC, thick]
        \node [contact]  (contact1) at (-1.5, 0) {};
        \draw  (0, 0) to [resistor={info=$R_1$}] ++(left:1.5);
        \draw  (0, 0) -- ++(up:1.5) to [resistor={near start, info=$R_2$}, resistor={near end, info=$R_3$}] ++(right:3);
        \draw  (0, 0) to [resistor={info=$R_4$}] ++(right:3) -- ++(up:1.5);
        \draw  (3, 1.5) -- ++(right:0.5); \node [contact] (contact2) at (3.5, 1.5) {};
    \end{tikzpicture}
}
\answer{%
    $R=\frac{31}{10}\units{Ом} \approx 3{,}10\,\text{Ом} \implies U = \eli R \approx 31{,}0\,\text{В}.$
}
\solutionspace{120pt}

\tasknumber{2}%
\task{%
    Определите показания амперметра $1$ (см.
    рис.) и разность потенциалов на резисторе $5$,
    если сопротивления всех резисторов равны: $R_1 = R_2 = R_3 = R_4 = R_5 = R_6 = R = 4\,\text{Ом}$,
    а напряжение, поданное на цепь, равно $U = 30\,\text{В}$.
    Ответы получите в виде несократимых дробей, а также определите приближённые значения.
    Амперметры считать идеальными.

    \begin{tikzpicture}[circuit ee IEC, thick]
        \node [contact]  (left contact) at (3, 0) {};
        \node [contact]  (right contact) at (9, 0) {};
        \draw  (left contact) -- ++(up:2) to [resistor={very near start, info=$R_2$}, amperemeter={midway, info=$1$}, resistor={very near end, info=$R_3$} ] ++(right:6) -- (right contact);
        \draw  (left contact) -- ++(down:2) to [resistor={very near start, info=$R_5$}, resistor={midway, info=$R_6$}, amperemeter={very near end, info=$3$}] ++(right:6) -- (right contact);
        \draw  (left contact) ++(left:3) to [resistor={info=$R_1$}] (left contact) to [amperemeter={near start, info=$2$}, resistor={near end , info=$R_4$}] (right contact) -- ++(right:0.5);
    \end{tikzpicture}
}
\answer{%
    \begin{align*}
    R_0 &= R + \frac 1{\frac 1{R+R} + \frac 1R + \frac 1{R+R}} = R + \frac 1{\frac 2R} = \frac 32 R, \\
    \eli &= \frac U{R_0} = \frac {2U}{3R}, \\
    U_1 &= \eli R_1 = \frac {2U}{3R} \cdot R = \frac 23 U = 20{,}0\,\text{В}, \\
    U_{23} &= U_{56} = U_4 = U - \eli R_1 = U - \frac {2U}{3R} \cdot R = \frac U3 = 10{,}0\,\text{В}, \\
    \eli_2 &= \frac{U_4}{R_4} = \frac U{3R} \approx 2{,}5\,\text{А}, \\
    \eli_1 &= \frac{U_{23}}{R_{23}} = \frac{\frac U3}{R+R} = \frac U{6R} \approx 1{,}2\,\text{А}, \\
    \eli_3 &= \frac{U_{56}}{R_{56}} = \frac{\frac U3}{R+R} = \frac U{6R} \approx 1{,}2\,\text{А}, \\
    U_2 &= \eli_1 R_2 = \frac U{6R} \cdot R = \frac U6 = 5{,}0\,\text{В}, \\
    U_3 &= \eli_1 R_3 = \frac U{6R} \cdot R = \frac U3 = 5{,}0\,\text{В}, \\
    U_5 &= \eli_3 R_5 = \frac U{6R} \cdot R = \frac U5 = 5{,}0\,\text{В}, \\
    U_6 &= \eli_3 R_6 = \frac U{6R} \cdot R = \frac U6 = 5{,}0\,\text{В}.
    \end{align*}
}
\solutionspace{120pt}

\tasknumber{3}%
\task{%
    Определите ток, протекающий через резистор $R = 12\,\text{Ом}$ и разность потенциалов на нём (см.
    рис.),
    если $\ele_1 = 18\,\text{В}$, $\ele_2 = 25\,\text{В}$, $r_1 = 2\,\text{Ом}$, $r_2 = 2\,\text{Ом}$.

    \begin{tikzpicture}[circuit ee IEC, thick]
        \draw  (0, 0) to [battery={rotate=-180,info={$\ele_1, r_1$}}] (0, 3)
                -- (5, 3)
                to [battery={rotate=-180, info'={$\ele_2, r_2$}}] (5, 0)
                -- (0, 0)
                (2.5, 0) to [resistor={info=$R$}] (2.5, 3);
    \end{tikzpicture}
}
\answer{%
    Выберем 2 контура и один узел, запишем для них законы Кирхгофа:

    \begin{tikzpicture}[circuit ee IEC, thick]
        \draw  (0, 0) to [battery={rotate=-180,info={$\ele_1, r_1$}}, current direction={near end, info=$\eli_1$}] (0, 3)
                -- (5, 3)
                to [battery={rotate=-180, info'={$\ele_2, r_2$}}, current direction={near end, info=$\eli_2$}] (5, 0)
                -- (0, 0)
                (2.5, 0) to [resistor={info=$R$}, current direction'={near end, info=$\eli$}] (2.5, 3);
        \draw [-{Latex},color=red] (0.8, 1.9) arc [start angle = 135, end angle = -160, radius = 0.6];
        \draw [-{Latex},color=blue] (3.5, 1.9) arc [start angle = 135, end angle = -160, radius = 0.6];
        \node [contact,color=green!71!black] (topc) at (2.5, 3) {};
        \node [above] (top) at (2.5, 3) {$1$};
    \end{tikzpicture}

    \begin{align*}
        &\begin{cases}
            {\color{red} \ele_1 = \eli_1 r_1 + \eli R}, \\
            {\color{blue} \ele_2 = \eli_2 r_2 - \eli R}, \\
            {\color{green!71!black} \eli - \eli_1 - \eli_2 = 0};
        \end{cases}
        \qquad \implies \qquad
        \begin{cases}
            \eli_1 = \frac{\ele_1 - \eli R}{r_1}, \\
            \eli_2 = \frac{\ele_2 + \eli R}{r_2}, \\
            \eli - \eli_1 - \eli_2 = 0;
        \end{cases} \implies \\
        &\implies \eli - \frac{\ele_1 - \eli R}{r_1} + \frac{\ele_2 + \eli R}{r_2} = 0, \\
        &\eli\cbr{ 1 + \frac{R}{r_1} + \frac{R}{r_2}} - \frac{\ele_1}{r_1} + \frac{\ele_2}{r_2} = 0, \\
        &\eli
            = \frac{\frac{\ele_1}{r_1} - \frac{\ele_2}{r_2}}{ 1 + \frac{R}{r_1} + \frac{R}{r_2}}
            = \frac{\frac{18\,\text{В}}{2\,\text{Ом}} - \frac{25\,\text{В}}{2\,\text{Ом}}}{ 1 + \frac{12\,\text{Ом}}{2\,\text{Ом}} + \frac{12\,\text{Ом}}{2\,\text{Ом}}}
            = -\frac7{26}\units{А}
            \approx -0{,}26900\,\text{А}, \\
        &U  = \eli R = \frac{\frac{\ele_1}{r_1} - \frac{\ele_2}{r_2}}{ 1 + \frac{R}{r_1} + \frac{R}{r_2}} R
            \approx -3{,}2310\,\text{В}.
    \end{align*}
}

\variantsplitter

\addpersonalvariant{Леонид Никитин}

\tasknumber{1}%
\task{%
    Определите эквивалентное сопротивление цепи на рисунке (между выделенными на рисунке контактами),
    если известны сопротивления всех резисторов: $R_1 = 2\,\text{Ом}$, $R_2 = 3\,\text{Ом}$, $R_3 = 3\,\text{Ом}$, $R_4 = 3\,\text{Ом}$.
    При каком напряжении поданном на эту цепь, в ней потечёт ток равный $\eli = 2\,\text{А}$?

    \begin{tikzpicture}[rotate=180, circuit ee IEC, thick]
        \node [contact]  (contact1) at (-1.5, 0) {};
        \draw  (0, 0) to [resistor={info=$R_1$}] ++(left:1.5);
        \draw  (0, 0) -- ++(up:1.5) to [resistor={near start, info=$R_2$}, resistor={near end, info=$R_3$}] ++(right:3);
        \draw  (0, 0) to [resistor={info=$R_4$}] ++(right:3) -- ++(up:1.5);
        \draw  (1.5, 1.5) -- ++(up:1); \node [contact] (contact2) at (1.5, 2.5) {};
    \end{tikzpicture}
}
\answer{%
    $R=4\units{Ом} \approx 4{,}00\,\text{Ом} \implies U = \eli R \approx 8{,}0\,\text{В}.$
}
\solutionspace{120pt}

\tasknumber{2}%
\task{%
    Определите показания амперметра $1$ (см.
    рис.) и разность потенциалов на резисторе $5$,
    если сопротивления всех резисторов равны: $R_1 = R_2 = R_3 = R_4 = R_5 = R_6 = R = 10\,\text{Ом}$,
    а напряжение, поданное на цепь, равно $U = 30\,\text{В}$.
    Ответы получите в виде несократимых дробей, а также определите приближённые значения.
    Амперметры считать идеальными.

    \begin{tikzpicture}[circuit ee IEC, thick]
        \node [contact]  (left contact) at (3, 0) {};
        \node [contact]  (right contact) at (9, 0) {};
        \draw  (left contact) -- ++(up:2) to [resistor={very near start, info=$R_2$}, amperemeter={midway, info=$1$}, resistor={very near end, info=$R_3$} ] ++(right:6) -- (right contact);
        \draw  (left contact) -- ++(down:2) to [resistor={very near start, info=$R_5$}, resistor={midway, info=$R_6$}, amperemeter={very near end, info=$3$}] ++(right:6) -- (right contact);
        \draw  (left contact) ++(left:3) to [resistor={info=$R_1$}] (left contact) to [amperemeter={near start, info=$2$}, resistor={near end , info=$R_4$}] (right contact) -- ++(right:0.5);
    \end{tikzpicture}
}
\answer{%
    \begin{align*}
    R_0 &= R + \frac 1{\frac 1{R+R} + \frac 1R + \frac 1{R+R}} = R + \frac 1{\frac 2R} = \frac 32 R, \\
    \eli &= \frac U{R_0} = \frac {2U}{3R}, \\
    U_1 &= \eli R_1 = \frac {2U}{3R} \cdot R = \frac 23 U = 20{,}0\,\text{В}, \\
    U_{23} &= U_{56} = U_4 = U - \eli R_1 = U - \frac {2U}{3R} \cdot R = \frac U3 = 10{,}0\,\text{В}, \\
    \eli_2 &= \frac{U_4}{R_4} = \frac U{3R} \approx 1{,}0\,\text{А}, \\
    \eli_1 &= \frac{U_{23}}{R_{23}} = \frac{\frac U3}{R+R} = \frac U{6R} \approx 0{,}5\,\text{А}, \\
    \eli_3 &= \frac{U_{56}}{R_{56}} = \frac{\frac U3}{R+R} = \frac U{6R} \approx 0{,}5\,\text{А}, \\
    U_2 &= \eli_1 R_2 = \frac U{6R} \cdot R = \frac U6 = 5{,}0\,\text{В}, \\
    U_3 &= \eli_1 R_3 = \frac U{6R} \cdot R = \frac U3 = 5{,}0\,\text{В}, \\
    U_5 &= \eli_3 R_5 = \frac U{6R} \cdot R = \frac U5 = 5{,}0\,\text{В}, \\
    U_6 &= \eli_3 R_6 = \frac U{6R} \cdot R = \frac U6 = 5{,}0\,\text{В}.
    \end{align*}
}
\solutionspace{120pt}

\tasknumber{3}%
\task{%
    Определите ток, протекающий через резистор $R = 10\,\text{Ом}$ и разность потенциалов на нём (см.
    рис.),
    если $\ele_1 = 18\,\text{В}$, $\ele_2 = 5\,\text{В}$, $r_1 = 1\,\text{Ом}$, $r_2 = 2\,\text{Ом}$.

    \begin{tikzpicture}[circuit ee IEC, thick]
        \draw  (0, 0) to [battery={rotate=-180,info={$\ele_1, r_1$}}] (0, 3)
                -- (5, 3)
                to [battery={rotate=-180, info'={$\ele_2, r_2$}}] (5, 0)
                -- (0, 0)
                (2.5, 0) to [resistor={info=$R$}] (2.5, 3);
    \end{tikzpicture}
}
\answer{%
    Выберем 2 контура и один узел, запишем для них законы Кирхгофа:

    \begin{tikzpicture}[circuit ee IEC, thick]
        \draw  (0, 0) to [battery={rotate=-180,info={$\ele_1, r_1$}}, current direction={near end, info=$\eli_1$}] (0, 3)
                -- (5, 3)
                to [battery={rotate=-180, info'={$\ele_2, r_2$}}, current direction={near end, info=$\eli_2$}] (5, 0)
                -- (0, 0)
                (2.5, 0) to [resistor={info=$R$}, current direction'={near end, info=$\eli$}] (2.5, 3);
        \draw [-{Latex},color=red] (0.8, 1.9) arc [start angle = 135, end angle = -160, radius = 0.6];
        \draw [-{Latex},color=blue] (3.5, 1.9) arc [start angle = 135, end angle = -160, radius = 0.6];
        \node [contact,color=green!71!black] (topc) at (2.5, 3) {};
        \node [above] (top) at (2.5, 3) {$1$};
    \end{tikzpicture}

    \begin{align*}
        &\begin{cases}
            {\color{red} \ele_1 = \eli_1 r_1 + \eli R}, \\
            {\color{blue} \ele_2 = \eli_2 r_2 - \eli R}, \\
            {\color{green!71!black} \eli - \eli_1 - \eli_2 = 0};
        \end{cases}
        \qquad \implies \qquad
        \begin{cases}
            \eli_1 = \frac{\ele_1 - \eli R}{r_1}, \\
            \eli_2 = \frac{\ele_2 + \eli R}{r_2}, \\
            \eli - \eli_1 - \eli_2 = 0;
        \end{cases} \implies \\
        &\implies \eli - \frac{\ele_1 - \eli R}{r_1} + \frac{\ele_2 + \eli R}{r_2} = 0, \\
        &\eli\cbr{ 1 + \frac{R}{r_1} + \frac{R}{r_2}} - \frac{\ele_1}{r_1} + \frac{\ele_2}{r_2} = 0, \\
        &\eli
            = \frac{\frac{\ele_1}{r_1} - \frac{\ele_2}{r_2}}{ 1 + \frac{R}{r_1} + \frac{R}{r_2}}
            = \frac{\frac{18\,\text{В}}{1\,\text{Ом}} - \frac{5\,\text{В}}{2\,\text{Ом}}}{ 1 + \frac{10\,\text{Ом}}{1\,\text{Ом}} + \frac{10\,\text{Ом}}{2\,\text{Ом}}}
            = \frac{31}{32}\units{А}
            \approx 0{,}969\,\text{А}, \\
        &U  = \eli R = \frac{\frac{\ele_1}{r_1} - \frac{\ele_2}{r_2}}{ 1 + \frac{R}{r_1} + \frac{R}{r_2}} R
            \approx 9{,}688\,\text{В}.
    \end{align*}
}

\variantsplitter

\addpersonalvariant{Тимофей Полетаев}

\tasknumber{1}%
\task{%
    Определите эквивалентное сопротивление цепи на рисунке (между выделенными на рисунке контактами),
    если известны сопротивления всех резисторов: $R_1 = 1\,\text{Ом}$, $R_2 = 5\,\text{Ом}$, $R_3 = 1\,\text{Ом}$, $R_4 = 3\,\text{Ом}$.
    При каком напряжении поданном на эту цепь, в ней потечёт ток равный $\eli = 10\,\text{А}$?

    \begin{tikzpicture}[rotate=0, circuit ee IEC, thick]
        \node [contact]  (contact1) at (-1.5, 0) {};
        \draw  (0, 0) to [resistor={info=$R_1$}] ++(left:1.5);
        \draw  (0, 0) -- ++(up:1.5) to [resistor={near start, info=$R_2$}, resistor={near end, info=$R_3$}] ++(right:3);
        \draw  (0, 0) to [resistor={info=$R_4$}] ++(right:3) -- ++(up:1.5);
        \draw  (3, 1.5) -- ++(right:0.5); \node [contact] (contact2) at (3.5, 1.5) {};
    \end{tikzpicture}
}
\answer{%
    $R=3\units{Ом} \approx 3{,}00\,\text{Ом} \implies U = \eli R \approx 30{,}0\,\text{В}.$
}
\solutionspace{120pt}

\tasknumber{2}%
\task{%
    Определите показания амперметра $3$ (см.
    рис.) и разность потенциалов на резисторе $6$,
    если сопротивления всех резисторов равны: $R_1 = R_2 = R_3 = R_4 = R_5 = R_6 = R = 4\,\text{Ом}$,
    а напряжение, поданное на цепь, равно $U = 150\,\text{В}$.
    Ответы получите в виде несократимых дробей, а также определите приближённые значения.
    Амперметры считать идеальными.

    \begin{tikzpicture}[circuit ee IEC, thick]
        \node [contact]  (left contact) at (3, 0) {};
        \node [contact]  (right contact) at (9, 0) {};
        \draw  (left contact) -- ++(up:2) to [resistor={very near start, info=$R_2$}, amperemeter={midway, info=$1$}, resistor={very near end, info=$R_3$} ] ++(right:6) -- (right contact);
        \draw  (left contact) -- ++(down:2) to [resistor={very near start, info=$R_5$}, resistor={midway, info=$R_6$}, amperemeter={very near end, info=$3$}] ++(right:6) -- (right contact);
        \draw  (left contact) ++(left:3) to [resistor={info=$R_1$}] (left contact) to [amperemeter={near start, info=$2$}, resistor={near end , info=$R_4$}] (right contact) -- ++(right:0.5);
    \end{tikzpicture}
}
\answer{%
    \begin{align*}
    R_0 &= R + \frac 1{\frac 1{R+R} + \frac 1R + \frac 1{R+R}} = R + \frac 1{\frac 2R} = \frac 32 R, \\
    \eli &= \frac U{R_0} = \frac {2U}{3R}, \\
    U_1 &= \eli R_1 = \frac {2U}{3R} \cdot R = \frac 23 U = 100{,}0\,\text{В}, \\
    U_{23} &= U_{56} = U_4 = U - \eli R_1 = U - \frac {2U}{3R} \cdot R = \frac U3 = 50{,}0\,\text{В}, \\
    \eli_2 &= \frac{U_4}{R_4} = \frac U{3R} \approx 12{,}5\,\text{А}, \\
    \eli_1 &= \frac{U_{23}}{R_{23}} = \frac{\frac U3}{R+R} = \frac U{6R} \approx 6{,}2\,\text{А}, \\
    \eli_3 &= \frac{U_{56}}{R_{56}} = \frac{\frac U3}{R+R} = \frac U{6R} \approx 6{,}2\,\text{А}, \\
    U_2 &= \eli_1 R_2 = \frac U{6R} \cdot R = \frac U6 = 25{,}0\,\text{В}, \\
    U_3 &= \eli_1 R_3 = \frac U{6R} \cdot R = \frac U3 = 25{,}0\,\text{В}, \\
    U_5 &= \eli_3 R_5 = \frac U{6R} \cdot R = \frac U5 = 25{,}0\,\text{В}, \\
    U_6 &= \eli_3 R_6 = \frac U{6R} \cdot R = \frac U6 = 25{,}0\,\text{В}.
    \end{align*}
}
\solutionspace{120pt}

\tasknumber{3}%
\task{%
    Определите ток, протекающий через резистор $R = 15\,\text{Ом}$ и разность потенциалов на нём (см.
    рис.),
    если $\ele_1 = 12\,\text{В}$, $\ele_2 = 5\,\text{В}$, $r_1 = 1\,\text{Ом}$, $r_2 = 6\,\text{Ом}$.

    \begin{tikzpicture}[circuit ee IEC, thick]
        \draw  (0, 0) to [battery={rotate=-180,info={$\ele_1, r_1$}}] (0, 3)
                -- (5, 3)
                to [battery={rotate=-180, info'={$\ele_2, r_2$}}] (5, 0)
                -- (0, 0)
                (2.5, 0) to [resistor={info=$R$}] (2.5, 3);
    \end{tikzpicture}
}
\answer{%
    Выберем 2 контура и один узел, запишем для них законы Кирхгофа:

    \begin{tikzpicture}[circuit ee IEC, thick]
        \draw  (0, 0) to [battery={rotate=-180,info={$\ele_1, r_1$}}, current direction={near end, info=$\eli_1$}] (0, 3)
                -- (5, 3)
                to [battery={rotate=-180, info'={$\ele_2, r_2$}}, current direction={near end, info=$\eli_2$}] (5, 0)
                -- (0, 0)
                (2.5, 0) to [resistor={info=$R$}, current direction'={near end, info=$\eli$}] (2.5, 3);
        \draw [-{Latex},color=red] (0.8, 1.9) arc [start angle = 135, end angle = -160, radius = 0.6];
        \draw [-{Latex},color=blue] (3.5, 1.9) arc [start angle = 135, end angle = -160, radius = 0.6];
        \node [contact,color=green!71!black] (topc) at (2.5, 3) {};
        \node [above] (top) at (2.5, 3) {$1$};
    \end{tikzpicture}

    \begin{align*}
        &\begin{cases}
            {\color{red} \ele_1 = \eli_1 r_1 + \eli R}, \\
            {\color{blue} \ele_2 = \eli_2 r_2 - \eli R}, \\
            {\color{green!71!black} \eli - \eli_1 - \eli_2 = 0};
        \end{cases}
        \qquad \implies \qquad
        \begin{cases}
            \eli_1 = \frac{\ele_1 - \eli R}{r_1}, \\
            \eli_2 = \frac{\ele_2 + \eli R}{r_2}, \\
            \eli - \eli_1 - \eli_2 = 0;
        \end{cases} \implies \\
        &\implies \eli - \frac{\ele_1 - \eli R}{r_1} + \frac{\ele_2 + \eli R}{r_2} = 0, \\
        &\eli\cbr{ 1 + \frac{R}{r_1} + \frac{R}{r_2}} - \frac{\ele_1}{r_1} + \frac{\ele_2}{r_2} = 0, \\
        &\eli
            = \frac{\frac{\ele_1}{r_1} - \frac{\ele_2}{r_2}}{ 1 + \frac{R}{r_1} + \frac{R}{r_2}}
            = \frac{\frac{12\,\text{В}}{1\,\text{Ом}} - \frac{5\,\text{В}}{6\,\text{Ом}}}{ 1 + \frac{15\,\text{Ом}}{1\,\text{Ом}} + \frac{15\,\text{Ом}}{6\,\text{Ом}}}
            = \frac{67}{111}\units{А}
            \approx 0{,}604\,\text{А}, \\
        &U  = \eli R = \frac{\frac{\ele_1}{r_1} - \frac{\ele_2}{r_2}}{ 1 + \frac{R}{r_1} + \frac{R}{r_2}} R
            \approx 9{,}054\,\text{В}.
    \end{align*}
}

\variantsplitter

\addpersonalvariant{Андрей Рожков}

\tasknumber{1}%
\task{%
    Определите эквивалентное сопротивление цепи на рисунке (между выделенными на рисунке контактами),
    если известны сопротивления всех резисторов: $R_1 = 1\,\text{Ом}$, $R_2 = 3\,\text{Ом}$, $R_3 = 3\,\text{Ом}$, $R_4 = 3\,\text{Ом}$.
    При каком напряжении поданном на эту цепь, в ней потечёт ток равный $\eli = 5\,\text{А}$?

    \begin{tikzpicture}[rotate=180, circuit ee IEC, thick]
        \node [contact]  (contact1) at (-1.5, 0) {};
        \draw  (0, 0) to [resistor={info=$R_1$}] ++(left:1.5);
        \draw  (0, 0) -- ++(up:1.5) to [resistor={near start, info=$R_2$}, resistor={near end, info=$R_3$}] ++(right:3);
        \draw  (0, 0) to [resistor={info=$R_4$}] ++(right:3) -- ++(up:1.5);
        \draw  (1.5, 1.5) -- ++(up:1); \node [contact] (contact2) at (1.5, 2.5) {};
    \end{tikzpicture}
}
\answer{%
    $R=3\units{Ом} \approx 3{,}00\,\text{Ом} \implies U = \eli R \approx 15{,}0\,\text{В}.$
}
\solutionspace{120pt}

\tasknumber{2}%
\task{%
    Определите показания амперметра $1$ (см.
    рис.) и разность потенциалов на резисторе $2$,
    если сопротивления всех резисторов равны: $R_1 = R_2 = R_3 = R_4 = R_5 = R_6 = R = 5\,\text{Ом}$,
    а напряжение, поданное на цепь, равно $U = 90\,\text{В}$.
    Ответы получите в виде несократимых дробей, а также определите приближённые значения.
    Амперметры считать идеальными.

    \begin{tikzpicture}[circuit ee IEC, thick]
        \node [contact]  (left contact) at (3, 0) {};
        \node [contact]  (right contact) at (9, 0) {};
        \draw  (left contact) -- ++(up:2) to [resistor={very near start, info=$R_2$}, amperemeter={midway, info=$1$}, resistor={very near end, info=$R_3$} ] ++(right:6) -- (right contact);
        \draw  (left contact) -- ++(down:2) to [resistor={very near start, info=$R_5$}, resistor={midway, info=$R_6$}, amperemeter={very near end, info=$3$}] ++(right:6) -- (right contact);
        \draw  (left contact) ++(left:3) to [resistor={info=$R_1$}] (left contact) to [amperemeter={near start, info=$2$}, resistor={near end , info=$R_4$}] (right contact) -- ++(right:0.5);
    \end{tikzpicture}
}
\answer{%
    \begin{align*}
    R_0 &= R + \frac 1{\frac 1{R+R} + \frac 1R + \frac 1{R+R}} = R + \frac 1{\frac 2R} = \frac 32 R, \\
    \eli &= \frac U{R_0} = \frac {2U}{3R}, \\
    U_1 &= \eli R_1 = \frac {2U}{3R} \cdot R = \frac 23 U = 60{,}0\,\text{В}, \\
    U_{23} &= U_{56} = U_4 = U - \eli R_1 = U - \frac {2U}{3R} \cdot R = \frac U3 = 30{,}0\,\text{В}, \\
    \eli_2 &= \frac{U_4}{R_4} = \frac U{3R} \approx 6{,}0\,\text{А}, \\
    \eli_1 &= \frac{U_{23}}{R_{23}} = \frac{\frac U3}{R+R} = \frac U{6R} \approx 3{,}0\,\text{А}, \\
    \eli_3 &= \frac{U_{56}}{R_{56}} = \frac{\frac U3}{R+R} = \frac U{6R} \approx 3{,}0\,\text{А}, \\
    U_2 &= \eli_1 R_2 = \frac U{6R} \cdot R = \frac U6 = 15{,}0\,\text{В}, \\
    U_3 &= \eli_1 R_3 = \frac U{6R} \cdot R = \frac U3 = 15{,}0\,\text{В}, \\
    U_5 &= \eli_3 R_5 = \frac U{6R} \cdot R = \frac U5 = 15{,}0\,\text{В}, \\
    U_6 &= \eli_3 R_6 = \frac U{6R} \cdot R = \frac U6 = 15{,}0\,\text{В}.
    \end{align*}
}
\solutionspace{120pt}

\tasknumber{3}%
\task{%
    Определите ток, протекающий через резистор $R = 12\,\text{Ом}$ и разность потенциалов на нём (см.
    рис.),
    если $\ele_1 = 6\,\text{В}$, $\ele_2 = 5\,\text{В}$, $r_1 = 1\,\text{Ом}$, $r_2 = 4\,\text{Ом}$.

    \begin{tikzpicture}[circuit ee IEC, thick]
        \draw  (0, 0) to [battery={rotate=-180,info={$\ele_1, r_1$}}] (0, 3)
                -- (5, 3)
                to [battery={rotate=-180, info'={$\ele_2, r_2$}}] (5, 0)
                -- (0, 0)
                (2.5, 0) to [resistor={info=$R$}] (2.5, 3);
    \end{tikzpicture}
}
\answer{%
    Выберем 2 контура и один узел, запишем для них законы Кирхгофа:

    \begin{tikzpicture}[circuit ee IEC, thick]
        \draw  (0, 0) to [battery={rotate=-180,info={$\ele_1, r_1$}}, current direction={near end, info=$\eli_1$}] (0, 3)
                -- (5, 3)
                to [battery={rotate=-180, info'={$\ele_2, r_2$}}, current direction={near end, info=$\eli_2$}] (5, 0)
                -- (0, 0)
                (2.5, 0) to [resistor={info=$R$}, current direction'={near end, info=$\eli$}] (2.5, 3);
        \draw [-{Latex},color=red] (0.8, 1.9) arc [start angle = 135, end angle = -160, radius = 0.6];
        \draw [-{Latex},color=blue] (3.5, 1.9) arc [start angle = 135, end angle = -160, radius = 0.6];
        \node [contact,color=green!71!black] (topc) at (2.5, 3) {};
        \node [above] (top) at (2.5, 3) {$1$};
    \end{tikzpicture}

    \begin{align*}
        &\begin{cases}
            {\color{red} \ele_1 = \eli_1 r_1 + \eli R}, \\
            {\color{blue} \ele_2 = \eli_2 r_2 - \eli R}, \\
            {\color{green!71!black} \eli - \eli_1 - \eli_2 = 0};
        \end{cases}
        \qquad \implies \qquad
        \begin{cases}
            \eli_1 = \frac{\ele_1 - \eli R}{r_1}, \\
            \eli_2 = \frac{\ele_2 + \eli R}{r_2}, \\
            \eli - \eli_1 - \eli_2 = 0;
        \end{cases} \implies \\
        &\implies \eli - \frac{\ele_1 - \eli R}{r_1} + \frac{\ele_2 + \eli R}{r_2} = 0, \\
        &\eli\cbr{ 1 + \frac{R}{r_1} + \frac{R}{r_2}} - \frac{\ele_1}{r_1} + \frac{\ele_2}{r_2} = 0, \\
        &\eli
            = \frac{\frac{\ele_1}{r_1} - \frac{\ele_2}{r_2}}{ 1 + \frac{R}{r_1} + \frac{R}{r_2}}
            = \frac{\frac{6\,\text{В}}{1\,\text{Ом}} - \frac{5\,\text{В}}{4\,\text{Ом}}}{ 1 + \frac{12\,\text{Ом}}{1\,\text{Ом}} + \frac{12\,\text{Ом}}{4\,\text{Ом}}}
            = \frac{19}{64}\units{А}
            \approx 0{,}297\,\text{А}, \\
        &U  = \eli R = \frac{\frac{\ele_1}{r_1} - \frac{\ele_2}{r_2}}{ 1 + \frac{R}{r_1} + \frac{R}{r_2}} R
            \approx 3{,}562\,\text{В}.
    \end{align*}
}

\variantsplitter

\addpersonalvariant{Рената Таржиманова}

\tasknumber{1}%
\task{%
    Определите эквивалентное сопротивление цепи на рисунке (между выделенными на рисунке контактами),
    если известны сопротивления всех резисторов: $R_1 = 2\,\text{Ом}$, $R_2 = 5\,\text{Ом}$, $R_3 = 3\,\text{Ом}$, $R_4 = 2\,\text{Ом}$.
    При каком напряжении поданном на эту цепь, в ней потечёт ток равный $\eli = 10\,\text{А}$?

    \begin{tikzpicture}[rotate=0, circuit ee IEC, thick]
        \node [contact]  (contact1) at (-1.5, 0) {};
        \draw  (0, 0) to [resistor={info=$R_1$}] ++(left:1.5);
        \draw  (0, 0) -- ++(up:1.5) to [resistor={near start, info=$R_2$}, resistor={near end, info=$R_3$}] ++(right:3);
        \draw  (0, 0) to [resistor={info=$R_4$}] ++(right:3) -- ++(up:1.5);
        \draw  (1.5, 1.5) -- ++(up:1); \node [contact] (contact2) at (1.5, 2.5) {};
    \end{tikzpicture}
}
\answer{%
    $R=\frac92\units{Ом} \approx 4{,}50\,\text{Ом} \implies U = \eli R \approx 45{,}0\,\text{В}.$
}
\solutionspace{120pt}

\tasknumber{2}%
\task{%
    Определите показания амперметра $2$ (см.
    рис.) и разность потенциалов на резисторе $6$,
    если сопротивления всех резисторов равны: $R_1 = R_2 = R_3 = R_4 = R_5 = R_6 = R = 2\,\text{Ом}$,
    а напряжение, поданное на цепь, равно $U = 30\,\text{В}$.
    Ответы получите в виде несократимых дробей, а также определите приближённые значения.
    Амперметры считать идеальными.

    \begin{tikzpicture}[circuit ee IEC, thick]
        \node [contact]  (left contact) at (3, 0) {};
        \node [contact]  (right contact) at (9, 0) {};
        \draw  (left contact) -- ++(up:2) to [resistor={very near start, info=$R_2$}, amperemeter={midway, info=$1$}, resistor={very near end, info=$R_3$} ] ++(right:6) -- (right contact);
        \draw  (left contact) -- ++(down:2) to [resistor={very near start, info=$R_5$}, resistor={midway, info=$R_6$}, amperemeter={very near end, info=$3$}] ++(right:6) -- (right contact);
        \draw  (left contact) ++(left:3) to [resistor={info=$R_1$}] (left contact) to [amperemeter={near start, info=$2$}, resistor={near end , info=$R_4$}] (right contact) -- ++(right:0.5);
    \end{tikzpicture}
}
\answer{%
    \begin{align*}
    R_0 &= R + \frac 1{\frac 1{R+R} + \frac 1R + \frac 1{R+R}} = R + \frac 1{\frac 2R} = \frac 32 R, \\
    \eli &= \frac U{R_0} = \frac {2U}{3R}, \\
    U_1 &= \eli R_1 = \frac {2U}{3R} \cdot R = \frac 23 U = 20{,}0\,\text{В}, \\
    U_{23} &= U_{56} = U_4 = U - \eli R_1 = U - \frac {2U}{3R} \cdot R = \frac U3 = 10{,}0\,\text{В}, \\
    \eli_2 &= \frac{U_4}{R_4} = \frac U{3R} \approx 5{,}0\,\text{А}, \\
    \eli_1 &= \frac{U_{23}}{R_{23}} = \frac{\frac U3}{R+R} = \frac U{6R} \approx 2{,}5\,\text{А}, \\
    \eli_3 &= \frac{U_{56}}{R_{56}} = \frac{\frac U3}{R+R} = \frac U{6R} \approx 2{,}5\,\text{А}, \\
    U_2 &= \eli_1 R_2 = \frac U{6R} \cdot R = \frac U6 = 5{,}0\,\text{В}, \\
    U_3 &= \eli_1 R_3 = \frac U{6R} \cdot R = \frac U3 = 5{,}0\,\text{В}, \\
    U_5 &= \eli_3 R_5 = \frac U{6R} \cdot R = \frac U5 = 5{,}0\,\text{В}, \\
    U_6 &= \eli_3 R_6 = \frac U{6R} \cdot R = \frac U6 = 5{,}0\,\text{В}.
    \end{align*}
}
\solutionspace{120pt}

\tasknumber{3}%
\task{%
    Определите ток, протекающий через резистор $R = 10\,\text{Ом}$ и разность потенциалов на нём (см.
    рис.),
    если $\ele_1 = 6\,\text{В}$, $\ele_2 = 25\,\text{В}$, $r_1 = 2\,\text{Ом}$, $r_2 = 4\,\text{Ом}$.

    \begin{tikzpicture}[circuit ee IEC, thick]
        \draw  (0, 0) to [battery={rotate=-180,info={$\ele_1, r_1$}}] (0, 3)
                -- (5, 3)
                to [battery={rotate=-180, info'={$\ele_2, r_2$}}] (5, 0)
                -- (0, 0)
                (2.5, 0) to [resistor={info=$R$}] (2.5, 3);
    \end{tikzpicture}
}
\answer{%
    Выберем 2 контура и один узел, запишем для них законы Кирхгофа:

    \begin{tikzpicture}[circuit ee IEC, thick]
        \draw  (0, 0) to [battery={rotate=-180,info={$\ele_1, r_1$}}, current direction={near end, info=$\eli_1$}] (0, 3)
                -- (5, 3)
                to [battery={rotate=-180, info'={$\ele_2, r_2$}}, current direction={near end, info=$\eli_2$}] (5, 0)
                -- (0, 0)
                (2.5, 0) to [resistor={info=$R$}, current direction'={near end, info=$\eli$}] (2.5, 3);
        \draw [-{Latex},color=red] (0.8, 1.9) arc [start angle = 135, end angle = -160, radius = 0.6];
        \draw [-{Latex},color=blue] (3.5, 1.9) arc [start angle = 135, end angle = -160, radius = 0.6];
        \node [contact,color=green!71!black] (topc) at (2.5, 3) {};
        \node [above] (top) at (2.5, 3) {$1$};
    \end{tikzpicture}

    \begin{align*}
        &\begin{cases}
            {\color{red} \ele_1 = \eli_1 r_1 + \eli R}, \\
            {\color{blue} \ele_2 = \eli_2 r_2 - \eli R}, \\
            {\color{green!71!black} \eli - \eli_1 - \eli_2 = 0};
        \end{cases}
        \qquad \implies \qquad
        \begin{cases}
            \eli_1 = \frac{\ele_1 - \eli R}{r_1}, \\
            \eli_2 = \frac{\ele_2 + \eli R}{r_2}, \\
            \eli - \eli_1 - \eli_2 = 0;
        \end{cases} \implies \\
        &\implies \eli - \frac{\ele_1 - \eli R}{r_1} + \frac{\ele_2 + \eli R}{r_2} = 0, \\
        &\eli\cbr{ 1 + \frac{R}{r_1} + \frac{R}{r_2}} - \frac{\ele_1}{r_1} + \frac{\ele_2}{r_2} = 0, \\
        &\eli
            = \frac{\frac{\ele_1}{r_1} - \frac{\ele_2}{r_2}}{ 1 + \frac{R}{r_1} + \frac{R}{r_2}}
            = \frac{\frac{6\,\text{В}}{2\,\text{Ом}} - \frac{25\,\text{В}}{4\,\text{Ом}}}{ 1 + \frac{10\,\text{Ом}}{2\,\text{Ом}} + \frac{10\,\text{Ом}}{4\,\text{Ом}}}
            = -\frac{13}{34}\units{А}
            \approx -0{,}38200\,\text{А}, \\
        &U  = \eli R = \frac{\frac{\ele_1}{r_1} - \frac{\ele_2}{r_2}}{ 1 + \frac{R}{r_1} + \frac{R}{r_2}} R
            \approx -3{,}8240\,\text{В}.
    \end{align*}
}

\variantsplitter

\addpersonalvariant{Андрей Щербаков}

\tasknumber{1}%
\task{%
    Определите эквивалентное сопротивление цепи на рисунке (между выделенными на рисунке контактами),
    если известны сопротивления всех резисторов: $R_1 = 2\,\text{Ом}$, $R_2 = 4\,\text{Ом}$, $R_3 = 2\,\text{Ом}$, $R_4 = 3\,\text{Ом}$.
    При каком напряжении поданном на эту цепь, в ней потечёт ток равный $\eli = 10\,\text{А}$?

    \begin{tikzpicture}[rotate=90, circuit ee IEC, thick]
        \node [contact]  (contact1) at (-1.5, 0) {};
        \draw  (0, 0) to [resistor={info=$R_1$}] ++(left:1.5);
        \draw  (0, 0) -- ++(up:1.5) to [resistor={near start, info=$R_2$}, resistor={near end, info=$R_3$}] ++(right:3);
        \draw  (0, 0) to [resistor={info=$R_4$}] ++(right:3) -- ++(up:1.5);
        \draw  (1.5, 1.5) -- ++(up:1); \node [contact] (contact2) at (1.5, 2.5) {};
    \end{tikzpicture}
}
\answer{%
    $R=\frac{38}9\units{Ом} \approx 4{,}22\,\text{Ом} \implies U = \eli R \approx 42{,}2\,\text{В}.$
}
\solutionspace{120pt}

\tasknumber{2}%
\task{%
    Определите показания амперметра $3$ (см.
    рис.) и разность потенциалов на резисторе $5$,
    если сопротивления всех резисторов равны: $R_1 = R_2 = R_3 = R_4 = R_5 = R_6 = R = 5\,\text{Ом}$,
    а напряжение, поданное на цепь, равно $U = 90\,\text{В}$.
    Ответы получите в виде несократимых дробей, а также определите приближённые значения.
    Амперметры считать идеальными.

    \begin{tikzpicture}[circuit ee IEC, thick]
        \node [contact]  (left contact) at (3, 0) {};
        \node [contact]  (right contact) at (9, 0) {};
        \draw  (left contact) -- ++(up:2) to [resistor={very near start, info=$R_2$}, amperemeter={midway, info=$1$}, resistor={very near end, info=$R_3$} ] ++(right:6) -- (right contact);
        \draw  (left contact) -- ++(down:2) to [resistor={very near start, info=$R_5$}, resistor={midway, info=$R_6$}, amperemeter={very near end, info=$3$}] ++(right:6) -- (right contact);
        \draw  (left contact) ++(left:3) to [resistor={info=$R_1$}] (left contact) to [amperemeter={near start, info=$2$}, resistor={near end , info=$R_4$}] (right contact) -- ++(right:0.5);
    \end{tikzpicture}
}
\answer{%
    \begin{align*}
    R_0 &= R + \frac 1{\frac 1{R+R} + \frac 1R + \frac 1{R+R}} = R + \frac 1{\frac 2R} = \frac 32 R, \\
    \eli &= \frac U{R_0} = \frac {2U}{3R}, \\
    U_1 &= \eli R_1 = \frac {2U}{3R} \cdot R = \frac 23 U = 60{,}0\,\text{В}, \\
    U_{23} &= U_{56} = U_4 = U - \eli R_1 = U - \frac {2U}{3R} \cdot R = \frac U3 = 30{,}0\,\text{В}, \\
    \eli_2 &= \frac{U_4}{R_4} = \frac U{3R} \approx 6{,}0\,\text{А}, \\
    \eli_1 &= \frac{U_{23}}{R_{23}} = \frac{\frac U3}{R+R} = \frac U{6R} \approx 3{,}0\,\text{А}, \\
    \eli_3 &= \frac{U_{56}}{R_{56}} = \frac{\frac U3}{R+R} = \frac U{6R} \approx 3{,}0\,\text{А}, \\
    U_2 &= \eli_1 R_2 = \frac U{6R} \cdot R = \frac U6 = 15{,}0\,\text{В}, \\
    U_3 &= \eli_1 R_3 = \frac U{6R} \cdot R = \frac U3 = 15{,}0\,\text{В}, \\
    U_5 &= \eli_3 R_5 = \frac U{6R} \cdot R = \frac U5 = 15{,}0\,\text{В}, \\
    U_6 &= \eli_3 R_6 = \frac U{6R} \cdot R = \frac U6 = 15{,}0\,\text{В}.
    \end{align*}
}
\solutionspace{120pt}

\tasknumber{3}%
\task{%
    Определите ток, протекающий через резистор $R = 20\,\text{Ом}$ и разность потенциалов на нём (см.
    рис.),
    если $\ele_1 = 6\,\text{В}$, $\ele_2 = 15\,\text{В}$, $r_1 = 1\,\text{Ом}$, $r_2 = 2\,\text{Ом}$.

    \begin{tikzpicture}[circuit ee IEC, thick]
        \draw  (0, 0) to [battery={rotate=-180,info={$\ele_1, r_1$}}] (0, 3)
                -- (5, 3)
                to [battery={rotate=-180, info'={$\ele_2, r_2$}}] (5, 0)
                -- (0, 0)
                (2.5, 0) to [resistor={info=$R$}] (2.5, 3);
    \end{tikzpicture}
}
\answer{%
    Выберем 2 контура и один узел, запишем для них законы Кирхгофа:

    \begin{tikzpicture}[circuit ee IEC, thick]
        \draw  (0, 0) to [battery={rotate=-180,info={$\ele_1, r_1$}}, current direction={near end, info=$\eli_1$}] (0, 3)
                -- (5, 3)
                to [battery={rotate=-180, info'={$\ele_2, r_2$}}, current direction={near end, info=$\eli_2$}] (5, 0)
                -- (0, 0)
                (2.5, 0) to [resistor={info=$R$}, current direction'={near end, info=$\eli$}] (2.5, 3);
        \draw [-{Latex},color=red] (0.8, 1.9) arc [start angle = 135, end angle = -160, radius = 0.6];
        \draw [-{Latex},color=blue] (3.5, 1.9) arc [start angle = 135, end angle = -160, radius = 0.6];
        \node [contact,color=green!71!black] (topc) at (2.5, 3) {};
        \node [above] (top) at (2.5, 3) {$1$};
    \end{tikzpicture}

    \begin{align*}
        &\begin{cases}
            {\color{red} \ele_1 = \eli_1 r_1 + \eli R}, \\
            {\color{blue} \ele_2 = \eli_2 r_2 - \eli R}, \\
            {\color{green!71!black} \eli - \eli_1 - \eli_2 = 0};
        \end{cases}
        \qquad \implies \qquad
        \begin{cases}
            \eli_1 = \frac{\ele_1 - \eli R}{r_1}, \\
            \eli_2 = \frac{\ele_2 + \eli R}{r_2}, \\
            \eli - \eli_1 - \eli_2 = 0;
        \end{cases} \implies \\
        &\implies \eli - \frac{\ele_1 - \eli R}{r_1} + \frac{\ele_2 + \eli R}{r_2} = 0, \\
        &\eli\cbr{ 1 + \frac{R}{r_1} + \frac{R}{r_2}} - \frac{\ele_1}{r_1} + \frac{\ele_2}{r_2} = 0, \\
        &\eli
            = \frac{\frac{\ele_1}{r_1} - \frac{\ele_2}{r_2}}{ 1 + \frac{R}{r_1} + \frac{R}{r_2}}
            = \frac{\frac{6\,\text{В}}{1\,\text{Ом}} - \frac{15\,\text{В}}{2\,\text{Ом}}}{ 1 + \frac{20\,\text{Ом}}{1\,\text{Ом}} + \frac{20\,\text{Ом}}{2\,\text{Ом}}}
            = -\frac3{62}\units{А}
            \approx -0{,}048000\,\text{А}, \\
        &U  = \eli R = \frac{\frac{\ele_1}{r_1} - \frac{\ele_2}{r_2}}{ 1 + \frac{R}{r_1} + \frac{R}{r_2}} R
            \approx -0{,}96800\,\text{В}.
    \end{align*}
}

\variantsplitter

\addpersonalvariant{Михаил Ярошевский}

\tasknumber{1}%
\task{%
    Определите эквивалентное сопротивление цепи на рисунке (между выделенными на рисунке контактами),
    если известны сопротивления всех резисторов: $R_1 = 1\,\text{Ом}$, $R_2 = 4\,\text{Ом}$, $R_3 = 1\,\text{Ом}$, $R_4 = 2\,\text{Ом}$.
    При каком напряжении поданном на эту цепь, в ней потечёт ток равный $\eli = 10\,\text{А}$?

    \begin{tikzpicture}[rotate=90, circuit ee IEC, thick]
        \node [contact]  (contact1) at (-1.5, 0) {};
        \draw  (0, 0) to [resistor={info=$R_1$}] ++(left:1.5);
        \draw  (0, 0) -- ++(up:1.5) to [resistor={near start, info=$R_2$}, resistor={near end, info=$R_3$}] ++(right:3);
        \draw  (0, 0) to [resistor={info=$R_4$}] ++(right:3) -- ++(up:1.5);
        \draw  (1.5, 1.5) -- ++(up:1); \node [contact] (contact2) at (1.5, 2.5) {};
    \end{tikzpicture}
}
\answer{%
    $R=\frac{19}7\units{Ом} \approx 2{,}71\,\text{Ом} \implies U = \eli R \approx 27{,}1\,\text{В}.$
}
\solutionspace{120pt}

\tasknumber{2}%
\task{%
    Определите показания амперметра $2$ (см.
    рис.) и разность потенциалов на резисторе $4$,
    если сопротивления всех резисторов равны: $R_1 = R_2 = R_3 = R_4 = R_5 = R_6 = R = 4\,\text{Ом}$,
    а напряжение, поданное на цепь, равно $U = 120\,\text{В}$.
    Ответы получите в виде несократимых дробей, а также определите приближённые значения.
    Амперметры считать идеальными.

    \begin{tikzpicture}[circuit ee IEC, thick]
        \node [contact]  (left contact) at (3, 0) {};
        \node [contact]  (right contact) at (9, 0) {};
        \draw  (left contact) -- ++(up:2) to [resistor={very near start, info=$R_2$}, amperemeter={midway, info=$1$}, resistor={very near end, info=$R_3$} ] ++(right:6) -- (right contact);
        \draw  (left contact) -- ++(down:2) to [resistor={very near start, info=$R_5$}, resistor={midway, info=$R_6$}, amperemeter={very near end, info=$3$}] ++(right:6) -- (right contact);
        \draw  (left contact) ++(left:3) to [resistor={info=$R_1$}] (left contact) to [amperemeter={near start, info=$2$}, resistor={near end , info=$R_4$}] (right contact) -- ++(right:0.5);
    \end{tikzpicture}
}
\answer{%
    \begin{align*}
    R_0 &= R + \frac 1{\frac 1{R+R} + \frac 1R + \frac 1{R+R}} = R + \frac 1{\frac 2R} = \frac 32 R, \\
    \eli &= \frac U{R_0} = \frac {2U}{3R}, \\
    U_1 &= \eli R_1 = \frac {2U}{3R} \cdot R = \frac 23 U = 80{,}0\,\text{В}, \\
    U_{23} &= U_{56} = U_4 = U - \eli R_1 = U - \frac {2U}{3R} \cdot R = \frac U3 = 40{,}0\,\text{В}, \\
    \eli_2 &= \frac{U_4}{R_4} = \frac U{3R} \approx 10{,}0\,\text{А}, \\
    \eli_1 &= \frac{U_{23}}{R_{23}} = \frac{\frac U3}{R+R} = \frac U{6R} \approx 5{,}0\,\text{А}, \\
    \eli_3 &= \frac{U_{56}}{R_{56}} = \frac{\frac U3}{R+R} = \frac U{6R} \approx 5{,}0\,\text{А}, \\
    U_2 &= \eli_1 R_2 = \frac U{6R} \cdot R = \frac U6 = 20{,}0\,\text{В}, \\
    U_3 &= \eli_1 R_3 = \frac U{6R} \cdot R = \frac U3 = 20{,}0\,\text{В}, \\
    U_5 &= \eli_3 R_5 = \frac U{6R} \cdot R = \frac U5 = 20{,}0\,\text{В}, \\
    U_6 &= \eli_3 R_6 = \frac U{6R} \cdot R = \frac U6 = 20{,}0\,\text{В}.
    \end{align*}
}
\solutionspace{120pt}

\tasknumber{3}%
\task{%
    Определите ток, протекающий через резистор $R = 20\,\text{Ом}$ и разность потенциалов на нём (см.
    рис.),
    если $\ele_1 = 12\,\text{В}$, $\ele_2 = 25\,\text{В}$, $r_1 = 3\,\text{Ом}$, $r_2 = 6\,\text{Ом}$.

    \begin{tikzpicture}[circuit ee IEC, thick]
        \draw  (0, 0) to [battery={rotate=-180,info={$\ele_1, r_1$}}] (0, 3)
                -- (5, 3)
                to [battery={rotate=-180, info'={$\ele_2, r_2$}}] (5, 0)
                -- (0, 0)
                (2.5, 0) to [resistor={info=$R$}] (2.5, 3);
    \end{tikzpicture}
}
\answer{%
    Выберем 2 контура и один узел, запишем для них законы Кирхгофа:

    \begin{tikzpicture}[circuit ee IEC, thick]
        \draw  (0, 0) to [battery={rotate=-180,info={$\ele_1, r_1$}}, current direction={near end, info=$\eli_1$}] (0, 3)
                -- (5, 3)
                to [battery={rotate=-180, info'={$\ele_2, r_2$}}, current direction={near end, info=$\eli_2$}] (5, 0)
                -- (0, 0)
                (2.5, 0) to [resistor={info=$R$}, current direction'={near end, info=$\eli$}] (2.5, 3);
        \draw [-{Latex},color=red] (0.8, 1.9) arc [start angle = 135, end angle = -160, radius = 0.6];
        \draw [-{Latex},color=blue] (3.5, 1.9) arc [start angle = 135, end angle = -160, radius = 0.6];
        \node [contact,color=green!71!black] (topc) at (2.5, 3) {};
        \node [above] (top) at (2.5, 3) {$1$};
    \end{tikzpicture}

    \begin{align*}
        &\begin{cases}
            {\color{red} \ele_1 = \eli_1 r_1 + \eli R}, \\
            {\color{blue} \ele_2 = \eli_2 r_2 - \eli R}, \\
            {\color{green!71!black} \eli - \eli_1 - \eli_2 = 0};
        \end{cases}
        \qquad \implies \qquad
        \begin{cases}
            \eli_1 = \frac{\ele_1 - \eli R}{r_1}, \\
            \eli_2 = \frac{\ele_2 + \eli R}{r_2}, \\
            \eli - \eli_1 - \eli_2 = 0;
        \end{cases} \implies \\
        &\implies \eli - \frac{\ele_1 - \eli R}{r_1} + \frac{\ele_2 + \eli R}{r_2} = 0, \\
        &\eli\cbr{ 1 + \frac{R}{r_1} + \frac{R}{r_2}} - \frac{\ele_1}{r_1} + \frac{\ele_2}{r_2} = 0, \\
        &\eli
            = \frac{\frac{\ele_1}{r_1} - \frac{\ele_2}{r_2}}{ 1 + \frac{R}{r_1} + \frac{R}{r_2}}
            = \frac{\frac{12\,\text{В}}{3\,\text{Ом}} - \frac{25\,\text{В}}{6\,\text{Ом}}}{ 1 + \frac{20\,\text{Ом}}{3\,\text{Ом}} + \frac{20\,\text{Ом}}{6\,\text{Ом}}}
            = -\frac1{66}\units{А}
            \approx -0{,}0150000\,\text{А}, \\
        &U  = \eli R = \frac{\frac{\ele_1}{r_1} - \frac{\ele_2}{r_2}}{ 1 + \frac{R}{r_1} + \frac{R}{r_2}} R
            \approx -0{,}30300\,\text{В}.
    \end{align*}
}

\variantsplitter

\addpersonalvariant{Алексей Алимпиев}

\tasknumber{1}%
\task{%
    Определите эквивалентное сопротивление цепи на рисунке (между выделенными на рисунке контактами),
    если известны сопротивления всех резисторов: $R_1 = 1\,\text{Ом}$, $R_2 = 5\,\text{Ом}$, $R_3 = 1\,\text{Ом}$, $R_4 = 3\,\text{Ом}$.
    При каком напряжении поданном на эту цепь, в ней потечёт ток равный $\eli = 2\,\text{А}$?

    \begin{tikzpicture}[rotate=270, circuit ee IEC, thick]
        \node [contact]  (contact1) at (-1.5, 0) {};
        \draw  (0, 0) to [resistor={info=$R_1$}] ++(left:1.5);
        \draw  (0, 0) -- ++(up:1.5) to [resistor={near start, info=$R_2$}, resistor={near end, info=$R_3$}] ++(right:3);
        \draw  (0, 0) to [resistor={info=$R_4$}] ++(right:3) -- ++(up:1.5);
        \draw  (1.5, 1.5) -- ++(up:1); \node [contact] (contact2) at (1.5, 2.5) {};
    \end{tikzpicture}
}
\answer{%
    $R=\frac{29}9\units{Ом} \approx 3{,}22\,\text{Ом} \implies U = \eli R \approx 6{,}4\,\text{В}.$
}
\solutionspace{120pt}

\tasknumber{2}%
\task{%
    Определите показания амперметра $3$ (см.
    рис.) и разность потенциалов на резисторе $1$,
    если сопротивления всех резисторов равны: $R_1 = R_2 = R_3 = R_4 = R_5 = R_6 = R = 10\,\text{Ом}$,
    а напряжение, поданное на цепь, равно $U = 150\,\text{В}$.
    Ответы получите в виде несократимых дробей, а также определите приближённые значения.
    Амперметры считать идеальными.

    \begin{tikzpicture}[circuit ee IEC, thick]
        \node [contact]  (left contact) at (3, 0) {};
        \node [contact]  (right contact) at (9, 0) {};
        \draw  (left contact) -- ++(up:2) to [resistor={very near start, info=$R_2$}, amperemeter={midway, info=$1$}, resistor={very near end, info=$R_3$} ] ++(right:6) -- (right contact);
        \draw  (left contact) -- ++(down:2) to [resistor={very near start, info=$R_5$}, resistor={midway, info=$R_6$}, amperemeter={very near end, info=$3$}] ++(right:6) -- (right contact);
        \draw  (left contact) ++(left:3) to [resistor={info=$R_1$}] (left contact) to [amperemeter={near start, info=$2$}, resistor={near end , info=$R_4$}] (right contact) -- ++(right:0.5);
    \end{tikzpicture}
}
\answer{%
    \begin{align*}
    R_0 &= R + \frac 1{\frac 1{R+R} + \frac 1R + \frac 1{R+R}} = R + \frac 1{\frac 2R} = \frac 32 R, \\
    \eli &= \frac U{R_0} = \frac {2U}{3R}, \\
    U_1 &= \eli R_1 = \frac {2U}{3R} \cdot R = \frac 23 U = 100{,}0\,\text{В}, \\
    U_{23} &= U_{56} = U_4 = U - \eli R_1 = U - \frac {2U}{3R} \cdot R = \frac U3 = 50{,}0\,\text{В}, \\
    \eli_2 &= \frac{U_4}{R_4} = \frac U{3R} \approx 5{,}0\,\text{А}, \\
    \eli_1 &= \frac{U_{23}}{R_{23}} = \frac{\frac U3}{R+R} = \frac U{6R} \approx 2{,}5\,\text{А}, \\
    \eli_3 &= \frac{U_{56}}{R_{56}} = \frac{\frac U3}{R+R} = \frac U{6R} \approx 2{,}5\,\text{А}, \\
    U_2 &= \eli_1 R_2 = \frac U{6R} \cdot R = \frac U6 = 25{,}0\,\text{В}, \\
    U_3 &= \eli_1 R_3 = \frac U{6R} \cdot R = \frac U3 = 25{,}0\,\text{В}, \\
    U_5 &= \eli_3 R_5 = \frac U{6R} \cdot R = \frac U5 = 25{,}0\,\text{В}, \\
    U_6 &= \eli_3 R_6 = \frac U{6R} \cdot R = \frac U6 = 25{,}0\,\text{В}.
    \end{align*}
}
\solutionspace{120pt}

\tasknumber{3}%
\task{%
    Определите ток, протекающий через резистор $R = 20\,\text{Ом}$ и разность потенциалов на нём (см.
    рис.),
    если $\ele_1 = 6\,\text{В}$, $\ele_2 = 5\,\text{В}$, $r_1 = 2\,\text{Ом}$, $r_2 = 2\,\text{Ом}$.

    \begin{tikzpicture}[circuit ee IEC, thick]
        \draw  (0, 0) to [battery={rotate=-180,info={$\ele_1, r_1$}}] (0, 3)
                -- (5, 3)
                to [battery={rotate=-180, info'={$\ele_2, r_2$}}] (5, 0)
                -- (0, 0)
                (2.5, 0) to [resistor={info=$R$}] (2.5, 3);
    \end{tikzpicture}
}
\answer{%
    Выберем 2 контура и один узел, запишем для них законы Кирхгофа:

    \begin{tikzpicture}[circuit ee IEC, thick]
        \draw  (0, 0) to [battery={rotate=-180,info={$\ele_1, r_1$}}, current direction={near end, info=$\eli_1$}] (0, 3)
                -- (5, 3)
                to [battery={rotate=-180, info'={$\ele_2, r_2$}}, current direction={near end, info=$\eli_2$}] (5, 0)
                -- (0, 0)
                (2.5, 0) to [resistor={info=$R$}, current direction'={near end, info=$\eli$}] (2.5, 3);
        \draw [-{Latex},color=red] (0.8, 1.9) arc [start angle = 135, end angle = -160, radius = 0.6];
        \draw [-{Latex},color=blue] (3.5, 1.9) arc [start angle = 135, end angle = -160, radius = 0.6];
        \node [contact,color=green!71!black] (topc) at (2.5, 3) {};
        \node [above] (top) at (2.5, 3) {$1$};
    \end{tikzpicture}

    \begin{align*}
        &\begin{cases}
            {\color{red} \ele_1 = \eli_1 r_1 + \eli R}, \\
            {\color{blue} \ele_2 = \eli_2 r_2 - \eli R}, \\
            {\color{green!71!black} \eli - \eli_1 - \eli_2 = 0};
        \end{cases}
        \qquad \implies \qquad
        \begin{cases}
            \eli_1 = \frac{\ele_1 - \eli R}{r_1}, \\
            \eli_2 = \frac{\ele_2 + \eli R}{r_2}, \\
            \eli - \eli_1 - \eli_2 = 0;
        \end{cases} \implies \\
        &\implies \eli - \frac{\ele_1 - \eli R}{r_1} + \frac{\ele_2 + \eli R}{r_2} = 0, \\
        &\eli\cbr{ 1 + \frac{R}{r_1} + \frac{R}{r_2}} - \frac{\ele_1}{r_1} + \frac{\ele_2}{r_2} = 0, \\
        &\eli
            = \frac{\frac{\ele_1}{r_1} - \frac{\ele_2}{r_2}}{ 1 + \frac{R}{r_1} + \frac{R}{r_2}}
            = \frac{\frac{6\,\text{В}}{2\,\text{Ом}} - \frac{5\,\text{В}}{2\,\text{Ом}}}{ 1 + \frac{20\,\text{Ом}}{2\,\text{Ом}} + \frac{20\,\text{Ом}}{2\,\text{Ом}}}
            = \frac1{42}\units{А}
            \approx 0{,}024\,\text{А}, \\
        &U  = \eli R = \frac{\frac{\ele_1}{r_1} - \frac{\ele_2}{r_2}}{ 1 + \frac{R}{r_1} + \frac{R}{r_2}} R
            \approx 0{,}476\,\text{В}.
    \end{align*}
}

\variantsplitter

\addpersonalvariant{Евгений Васин}

\tasknumber{1}%
\task{%
    Определите эквивалентное сопротивление цепи на рисунке (между выделенными на рисунке контактами),
    если известны сопротивления всех резисторов: $R_1 = 2\,\text{Ом}$, $R_2 = 4\,\text{Ом}$, $R_3 = 3\,\text{Ом}$, $R_4 = 2\,\text{Ом}$.
    При каком напряжении поданном на эту цепь, в ней потечёт ток равный $\eli = 5\,\text{А}$?

    \begin{tikzpicture}[rotate=90, circuit ee IEC, thick]
        \node [contact]  (contact1) at (-1.5, 0) {};
        \draw  (0, 0) to [resistor={info=$R_1$}] ++(left:1.5);
        \draw  (0, 0) -- ++(up:1.5) to [resistor={near start, info=$R_2$}, resistor={near end, info=$R_3$}] ++(right:3);
        \draw  (0, 0) to [resistor={info=$R_4$}] ++(right:3) -- ++(up:1.5);
        \draw  (3, 1.5) -- ++(right:0.5); \node [contact] (contact2) at (3.5, 1.5) {};
    \end{tikzpicture}
}
\answer{%
    $R=\frac{32}9\units{Ом} \approx 3{,}56\,\text{Ом} \implies U = \eli R \approx 17{,}8\,\text{В}.$
}
\solutionspace{120pt}

\tasknumber{2}%
\task{%
    Определите показания амперметра $2$ (см.
    рис.) и разность потенциалов на резисторе $5$,
    если сопротивления всех резисторов равны: $R_1 = R_2 = R_3 = R_4 = R_5 = R_6 = R = 2\,\text{Ом}$,
    а напряжение, поданное на цепь, равно $U = 150\,\text{В}$.
    Ответы получите в виде несократимых дробей, а также определите приближённые значения.
    Амперметры считать идеальными.

    \begin{tikzpicture}[circuit ee IEC, thick]
        \node [contact]  (left contact) at (3, 0) {};
        \node [contact]  (right contact) at (9, 0) {};
        \draw  (left contact) -- ++(up:2) to [resistor={very near start, info=$R_2$}, amperemeter={midway, info=$1$}, resistor={very near end, info=$R_3$} ] ++(right:6) -- (right contact);
        \draw  (left contact) -- ++(down:2) to [resistor={very near start, info=$R_5$}, resistor={midway, info=$R_6$}, amperemeter={very near end, info=$3$}] ++(right:6) -- (right contact);
        \draw  (left contact) ++(left:3) to [resistor={info=$R_1$}] (left contact) to [amperemeter={near start, info=$2$}, resistor={near end , info=$R_4$}] (right contact) -- ++(right:0.5);
    \end{tikzpicture}
}
\answer{%
    \begin{align*}
    R_0 &= R + \frac 1{\frac 1{R+R} + \frac 1R + \frac 1{R+R}} = R + \frac 1{\frac 2R} = \frac 32 R, \\
    \eli &= \frac U{R_0} = \frac {2U}{3R}, \\
    U_1 &= \eli R_1 = \frac {2U}{3R} \cdot R = \frac 23 U = 100{,}0\,\text{В}, \\
    U_{23} &= U_{56} = U_4 = U - \eli R_1 = U - \frac {2U}{3R} \cdot R = \frac U3 = 50{,}0\,\text{В}, \\
    \eli_2 &= \frac{U_4}{R_4} = \frac U{3R} \approx 25{,}0\,\text{А}, \\
    \eli_1 &= \frac{U_{23}}{R_{23}} = \frac{\frac U3}{R+R} = \frac U{6R} \approx 12{,}5\,\text{А}, \\
    \eli_3 &= \frac{U_{56}}{R_{56}} = \frac{\frac U3}{R+R} = \frac U{6R} \approx 12{,}5\,\text{А}, \\
    U_2 &= \eli_1 R_2 = \frac U{6R} \cdot R = \frac U6 = 25{,}0\,\text{В}, \\
    U_3 &= \eli_1 R_3 = \frac U{6R} \cdot R = \frac U3 = 25{,}0\,\text{В}, \\
    U_5 &= \eli_3 R_5 = \frac U{6R} \cdot R = \frac U5 = 25{,}0\,\text{В}, \\
    U_6 &= \eli_3 R_6 = \frac U{6R} \cdot R = \frac U6 = 25{,}0\,\text{В}.
    \end{align*}
}
\solutionspace{120pt}

\tasknumber{3}%
\task{%
    Определите ток, протекающий через резистор $R = 20\,\text{Ом}$ и разность потенциалов на нём (см.
    рис.),
    если $\ele_1 = 12\,\text{В}$, $\ele_2 = 5\,\text{В}$, $r_1 = 3\,\text{Ом}$, $r_2 = 2\,\text{Ом}$.

    \begin{tikzpicture}[circuit ee IEC, thick]
        \draw  (0, 0) to [battery={rotate=-180,info={$\ele_1, r_1$}}] (0, 3)
                -- (5, 3)
                to [battery={rotate=-180, info'={$\ele_2, r_2$}}] (5, 0)
                -- (0, 0)
                (2.5, 0) to [resistor={info=$R$}] (2.5, 3);
    \end{tikzpicture}
}
\answer{%
    Выберем 2 контура и один узел, запишем для них законы Кирхгофа:

    \begin{tikzpicture}[circuit ee IEC, thick]
        \draw  (0, 0) to [battery={rotate=-180,info={$\ele_1, r_1$}}, current direction={near end, info=$\eli_1$}] (0, 3)
                -- (5, 3)
                to [battery={rotate=-180, info'={$\ele_2, r_2$}}, current direction={near end, info=$\eli_2$}] (5, 0)
                -- (0, 0)
                (2.5, 0) to [resistor={info=$R$}, current direction'={near end, info=$\eli$}] (2.5, 3);
        \draw [-{Latex},color=red] (0.8, 1.9) arc [start angle = 135, end angle = -160, radius = 0.6];
        \draw [-{Latex},color=blue] (3.5, 1.9) arc [start angle = 135, end angle = -160, radius = 0.6];
        \node [contact,color=green!71!black] (topc) at (2.5, 3) {};
        \node [above] (top) at (2.5, 3) {$1$};
    \end{tikzpicture}

    \begin{align*}
        &\begin{cases}
            {\color{red} \ele_1 = \eli_1 r_1 + \eli R}, \\
            {\color{blue} \ele_2 = \eli_2 r_2 - \eli R}, \\
            {\color{green!71!black} \eli - \eli_1 - \eli_2 = 0};
        \end{cases}
        \qquad \implies \qquad
        \begin{cases}
            \eli_1 = \frac{\ele_1 - \eli R}{r_1}, \\
            \eli_2 = \frac{\ele_2 + \eli R}{r_2}, \\
            \eli - \eli_1 - \eli_2 = 0;
        \end{cases} \implies \\
        &\implies \eli - \frac{\ele_1 - \eli R}{r_1} + \frac{\ele_2 + \eli R}{r_2} = 0, \\
        &\eli\cbr{ 1 + \frac{R}{r_1} + \frac{R}{r_2}} - \frac{\ele_1}{r_1} + \frac{\ele_2}{r_2} = 0, \\
        &\eli
            = \frac{\frac{\ele_1}{r_1} - \frac{\ele_2}{r_2}}{ 1 + \frac{R}{r_1} + \frac{R}{r_2}}
            = \frac{\frac{12\,\text{В}}{3\,\text{Ом}} - \frac{5\,\text{В}}{2\,\text{Ом}}}{ 1 + \frac{20\,\text{Ом}}{3\,\text{Ом}} + \frac{20\,\text{Ом}}{2\,\text{Ом}}}
            = \frac9{106}\units{А}
            \approx 0{,}085\,\text{А}, \\
        &U  = \eli R = \frac{\frac{\ele_1}{r_1} - \frac{\ele_2}{r_2}}{ 1 + \frac{R}{r_1} + \frac{R}{r_2}} R
            \approx 1{,}698\,\text{В}.
    \end{align*}
}

\variantsplitter

\addpersonalvariant{Вячеслав Волохов}

\tasknumber{1}%
\task{%
    Определите эквивалентное сопротивление цепи на рисунке (между выделенными на рисунке контактами),
    если известны сопротивления всех резисторов: $R_1 = 2\,\text{Ом}$, $R_2 = 4\,\text{Ом}$, $R_3 = 2\,\text{Ом}$, $R_4 = 4\,\text{Ом}$.
    При каком напряжении поданном на эту цепь, в ней потечёт ток равный $\eli = 2\,\text{А}$?

    \begin{tikzpicture}[rotate=0, circuit ee IEC, thick]
        \node [contact]  (contact1) at (-1.5, 0) {};
        \draw  (0, 0) to [resistor={info=$R_1$}] ++(left:1.5);
        \draw  (0, 0) -- ++(up:1.5) to [resistor={near start, info=$R_2$}, resistor={near end, info=$R_3$}] ++(right:3);
        \draw  (0, 0) to [resistor={info=$R_4$}] ++(right:3) -- ++(up:1.5);
        \draw  (3, 1.5) -- ++(right:0.5); \node [contact] (contact2) at (3.5, 1.5) {};
    \end{tikzpicture}
}
\answer{%
    $R=\frac{22}5\units{Ом} \approx 4{,}40\,\text{Ом} \implies U = \eli R \approx 8{,}8\,\text{В}.$
}
\solutionspace{120pt}

\tasknumber{2}%
\task{%
    Определите показания амперметра $1$ (см.
    рис.) и разность потенциалов на резисторе $3$,
    если сопротивления всех резисторов равны: $R_1 = R_2 = R_3 = R_4 = R_5 = R_6 = R = 10\,\text{Ом}$,
    а напряжение, поданное на цепь, равно $U = 90\,\text{В}$.
    Ответы получите в виде несократимых дробей, а также определите приближённые значения.
    Амперметры считать идеальными.

    \begin{tikzpicture}[circuit ee IEC, thick]
        \node [contact]  (left contact) at (3, 0) {};
        \node [contact]  (right contact) at (9, 0) {};
        \draw  (left contact) -- ++(up:2) to [resistor={very near start, info=$R_2$}, amperemeter={midway, info=$1$}, resistor={very near end, info=$R_3$} ] ++(right:6) -- (right contact);
        \draw  (left contact) -- ++(down:2) to [resistor={very near start, info=$R_5$}, resistor={midway, info=$R_6$}, amperemeter={very near end, info=$3$}] ++(right:6) -- (right contact);
        \draw  (left contact) ++(left:3) to [resistor={info=$R_1$}] (left contact) to [amperemeter={near start, info=$2$}, resistor={near end , info=$R_4$}] (right contact) -- ++(right:0.5);
    \end{tikzpicture}
}
\answer{%
    \begin{align*}
    R_0 &= R + \frac 1{\frac 1{R+R} + \frac 1R + \frac 1{R+R}} = R + \frac 1{\frac 2R} = \frac 32 R, \\
    \eli &= \frac U{R_0} = \frac {2U}{3R}, \\
    U_1 &= \eli R_1 = \frac {2U}{3R} \cdot R = \frac 23 U = 60{,}0\,\text{В}, \\
    U_{23} &= U_{56} = U_4 = U - \eli R_1 = U - \frac {2U}{3R} \cdot R = \frac U3 = 30{,}0\,\text{В}, \\
    \eli_2 &= \frac{U_4}{R_4} = \frac U{3R} \approx 3{,}0\,\text{А}, \\
    \eli_1 &= \frac{U_{23}}{R_{23}} = \frac{\frac U3}{R+R} = \frac U{6R} \approx 1{,}5\,\text{А}, \\
    \eli_3 &= \frac{U_{56}}{R_{56}} = \frac{\frac U3}{R+R} = \frac U{6R} \approx 1{,}5\,\text{А}, \\
    U_2 &= \eli_1 R_2 = \frac U{6R} \cdot R = \frac U6 = 15{,}0\,\text{В}, \\
    U_3 &= \eli_1 R_3 = \frac U{6R} \cdot R = \frac U3 = 15{,}0\,\text{В}, \\
    U_5 &= \eli_3 R_5 = \frac U{6R} \cdot R = \frac U5 = 15{,}0\,\text{В}, \\
    U_6 &= \eli_3 R_6 = \frac U{6R} \cdot R = \frac U6 = 15{,}0\,\text{В}.
    \end{align*}
}
\solutionspace{120pt}

\tasknumber{3}%
\task{%
    Определите ток, протекающий через резистор $R = 10\,\text{Ом}$ и разность потенциалов на нём (см.
    рис.),
    если $\ele_1 = 12\,\text{В}$, $\ele_2 = 25\,\text{В}$, $r_1 = 2\,\text{Ом}$, $r_2 = 2\,\text{Ом}$.

    \begin{tikzpicture}[circuit ee IEC, thick]
        \draw  (0, 0) to [battery={rotate=-180,info={$\ele_1, r_1$}}] (0, 3)
                -- (5, 3)
                to [battery={rotate=-180, info'={$\ele_2, r_2$}}] (5, 0)
                -- (0, 0)
                (2.5, 0) to [resistor={info=$R$}] (2.5, 3);
    \end{tikzpicture}
}
\answer{%
    Выберем 2 контура и один узел, запишем для них законы Кирхгофа:

    \begin{tikzpicture}[circuit ee IEC, thick]
        \draw  (0, 0) to [battery={rotate=-180,info={$\ele_1, r_1$}}, current direction={near end, info=$\eli_1$}] (0, 3)
                -- (5, 3)
                to [battery={rotate=-180, info'={$\ele_2, r_2$}}, current direction={near end, info=$\eli_2$}] (5, 0)
                -- (0, 0)
                (2.5, 0) to [resistor={info=$R$}, current direction'={near end, info=$\eli$}] (2.5, 3);
        \draw [-{Latex},color=red] (0.8, 1.9) arc [start angle = 135, end angle = -160, radius = 0.6];
        \draw [-{Latex},color=blue] (3.5, 1.9) arc [start angle = 135, end angle = -160, radius = 0.6];
        \node [contact,color=green!71!black] (topc) at (2.5, 3) {};
        \node [above] (top) at (2.5, 3) {$1$};
    \end{tikzpicture}

    \begin{align*}
        &\begin{cases}
            {\color{red} \ele_1 = \eli_1 r_1 + \eli R}, \\
            {\color{blue} \ele_2 = \eli_2 r_2 - \eli R}, \\
            {\color{green!71!black} \eli - \eli_1 - \eli_2 = 0};
        \end{cases}
        \qquad \implies \qquad
        \begin{cases}
            \eli_1 = \frac{\ele_1 - \eli R}{r_1}, \\
            \eli_2 = \frac{\ele_2 + \eli R}{r_2}, \\
            \eli - \eli_1 - \eli_2 = 0;
        \end{cases} \implies \\
        &\implies \eli - \frac{\ele_1 - \eli R}{r_1} + \frac{\ele_2 + \eli R}{r_2} = 0, \\
        &\eli\cbr{ 1 + \frac{R}{r_1} + \frac{R}{r_2}} - \frac{\ele_1}{r_1} + \frac{\ele_2}{r_2} = 0, \\
        &\eli
            = \frac{\frac{\ele_1}{r_1} - \frac{\ele_2}{r_2}}{ 1 + \frac{R}{r_1} + \frac{R}{r_2}}
            = \frac{\frac{12\,\text{В}}{2\,\text{Ом}} - \frac{25\,\text{В}}{2\,\text{Ом}}}{ 1 + \frac{10\,\text{Ом}}{2\,\text{Ом}} + \frac{10\,\text{Ом}}{2\,\text{Ом}}}
            = -\frac{13}{22}\units{А}
            \approx -0{,}59100\,\text{А}, \\
        &U  = \eli R = \frac{\frac{\ele_1}{r_1} - \frac{\ele_2}{r_2}}{ 1 + \frac{R}{r_1} + \frac{R}{r_2}} R
            \approx -5{,}9090\,\text{В}.
    \end{align*}
}

\variantsplitter

\addpersonalvariant{Герман Говоров}

\tasknumber{1}%
\task{%
    Определите эквивалентное сопротивление цепи на рисунке (между выделенными на рисунке контактами),
    если известны сопротивления всех резисторов: $R_1 = 1\,\text{Ом}$, $R_2 = 4\,\text{Ом}$, $R_3 = 3\,\text{Ом}$, $R_4 = 3\,\text{Ом}$.
    При каком напряжении поданном на эту цепь, в ней потечёт ток равный $\eli = 2\,\text{А}$?

    \begin{tikzpicture}[rotate=0, circuit ee IEC, thick]
        \node [contact]  (contact1) at (-1.5, 0) {};
        \draw  (0, 0) to [resistor={info=$R_1$}] ++(left:1.5);
        \draw  (0, 0) -- ++(up:1.5) to [resistor={near start, info=$R_2$}, resistor={near end, info=$R_3$}] ++(right:3);
        \draw  (0, 0) to [resistor={info=$R_4$}] ++(right:3) -- ++(up:1.5);
        \draw  (1.5, 1.5) -- ++(up:1); \node [contact] (contact2) at (1.5, 2.5) {};
    \end{tikzpicture}
}
\answer{%
    $R=\frac{17}5\units{Ом} \approx 3{,}40\,\text{Ом} \implies U = \eli R \approx 6{,}8\,\text{В}.$
}
\solutionspace{120pt}

\tasknumber{2}%
\task{%
    Определите показания амперметра $3$ (см.
    рис.) и разность потенциалов на резисторе $1$,
    если сопротивления всех резисторов равны: $R_1 = R_2 = R_3 = R_4 = R_5 = R_6 = R = 10\,\text{Ом}$,
    а напряжение, поданное на цепь, равно $U = 120\,\text{В}$.
    Ответы получите в виде несократимых дробей, а также определите приближённые значения.
    Амперметры считать идеальными.

    \begin{tikzpicture}[circuit ee IEC, thick]
        \node [contact]  (left contact) at (3, 0) {};
        \node [contact]  (right contact) at (9, 0) {};
        \draw  (left contact) -- ++(up:2) to [resistor={very near start, info=$R_2$}, amperemeter={midway, info=$1$}, resistor={very near end, info=$R_3$} ] ++(right:6) -- (right contact);
        \draw  (left contact) -- ++(down:2) to [resistor={very near start, info=$R_5$}, resistor={midway, info=$R_6$}, amperemeter={very near end, info=$3$}] ++(right:6) -- (right contact);
        \draw  (left contact) ++(left:3) to [resistor={info=$R_1$}] (left contact) to [amperemeter={near start, info=$2$}, resistor={near end , info=$R_4$}] (right contact) -- ++(right:0.5);
    \end{tikzpicture}
}
\answer{%
    \begin{align*}
    R_0 &= R + \frac 1{\frac 1{R+R} + \frac 1R + \frac 1{R+R}} = R + \frac 1{\frac 2R} = \frac 32 R, \\
    \eli &= \frac U{R_0} = \frac {2U}{3R}, \\
    U_1 &= \eli R_1 = \frac {2U}{3R} \cdot R = \frac 23 U = 80{,}0\,\text{В}, \\
    U_{23} &= U_{56} = U_4 = U - \eli R_1 = U - \frac {2U}{3R} \cdot R = \frac U3 = 40{,}0\,\text{В}, \\
    \eli_2 &= \frac{U_4}{R_4} = \frac U{3R} \approx 4{,}0\,\text{А}, \\
    \eli_1 &= \frac{U_{23}}{R_{23}} = \frac{\frac U3}{R+R} = \frac U{6R} \approx 2{,}0\,\text{А}, \\
    \eli_3 &= \frac{U_{56}}{R_{56}} = \frac{\frac U3}{R+R} = \frac U{6R} \approx 2{,}0\,\text{А}, \\
    U_2 &= \eli_1 R_2 = \frac U{6R} \cdot R = \frac U6 = 20{,}0\,\text{В}, \\
    U_3 &= \eli_1 R_3 = \frac U{6R} \cdot R = \frac U3 = 20{,}0\,\text{В}, \\
    U_5 &= \eli_3 R_5 = \frac U{6R} \cdot R = \frac U5 = 20{,}0\,\text{В}, \\
    U_6 &= \eli_3 R_6 = \frac U{6R} \cdot R = \frac U6 = 20{,}0\,\text{В}.
    \end{align*}
}
\solutionspace{120pt}

\tasknumber{3}%
\task{%
    Определите ток, протекающий через резистор $R = 12\,\text{Ом}$ и разность потенциалов на нём (см.
    рис.),
    если $\ele_1 = 18\,\text{В}$, $\ele_2 = 25\,\text{В}$, $r_1 = 3\,\text{Ом}$, $r_2 = 4\,\text{Ом}$.

    \begin{tikzpicture}[circuit ee IEC, thick]
        \draw  (0, 0) to [battery={rotate=-180,info={$\ele_1, r_1$}}] (0, 3)
                -- (5, 3)
                to [battery={rotate=-180, info'={$\ele_2, r_2$}}] (5, 0)
                -- (0, 0)
                (2.5, 0) to [resistor={info=$R$}] (2.5, 3);
    \end{tikzpicture}
}
\answer{%
    Выберем 2 контура и один узел, запишем для них законы Кирхгофа:

    \begin{tikzpicture}[circuit ee IEC, thick]
        \draw  (0, 0) to [battery={rotate=-180,info={$\ele_1, r_1$}}, current direction={near end, info=$\eli_1$}] (0, 3)
                -- (5, 3)
                to [battery={rotate=-180, info'={$\ele_2, r_2$}}, current direction={near end, info=$\eli_2$}] (5, 0)
                -- (0, 0)
                (2.5, 0) to [resistor={info=$R$}, current direction'={near end, info=$\eli$}] (2.5, 3);
        \draw [-{Latex},color=red] (0.8, 1.9) arc [start angle = 135, end angle = -160, radius = 0.6];
        \draw [-{Latex},color=blue] (3.5, 1.9) arc [start angle = 135, end angle = -160, radius = 0.6];
        \node [contact,color=green!71!black] (topc) at (2.5, 3) {};
        \node [above] (top) at (2.5, 3) {$1$};
    \end{tikzpicture}

    \begin{align*}
        &\begin{cases}
            {\color{red} \ele_1 = \eli_1 r_1 + \eli R}, \\
            {\color{blue} \ele_2 = \eli_2 r_2 - \eli R}, \\
            {\color{green!71!black} \eli - \eli_1 - \eli_2 = 0};
        \end{cases}
        \qquad \implies \qquad
        \begin{cases}
            \eli_1 = \frac{\ele_1 - \eli R}{r_1}, \\
            \eli_2 = \frac{\ele_2 + \eli R}{r_2}, \\
            \eli - \eli_1 - \eli_2 = 0;
        \end{cases} \implies \\
        &\implies \eli - \frac{\ele_1 - \eli R}{r_1} + \frac{\ele_2 + \eli R}{r_2} = 0, \\
        &\eli\cbr{ 1 + \frac{R}{r_1} + \frac{R}{r_2}} - \frac{\ele_1}{r_1} + \frac{\ele_2}{r_2} = 0, \\
        &\eli
            = \frac{\frac{\ele_1}{r_1} - \frac{\ele_2}{r_2}}{ 1 + \frac{R}{r_1} + \frac{R}{r_2}}
            = \frac{\frac{18\,\text{В}}{3\,\text{Ом}} - \frac{25\,\text{В}}{4\,\text{Ом}}}{ 1 + \frac{12\,\text{Ом}}{3\,\text{Ом}} + \frac{12\,\text{Ом}}{4\,\text{Ом}}}
            = -\frac1{32}\units{А}
            \approx -0{,}031000\,\text{А}, \\
        &U  = \eli R = \frac{\frac{\ele_1}{r_1} - \frac{\ele_2}{r_2}}{ 1 + \frac{R}{r_1} + \frac{R}{r_2}} R
            \approx -0{,}37500\,\text{В}.
    \end{align*}
}

\variantsplitter

\addpersonalvariant{София Журавлёва}

\tasknumber{1}%
\task{%
    Определите эквивалентное сопротивление цепи на рисунке (между выделенными на рисунке контактами),
    если известны сопротивления всех резисторов: $R_1 = 2\,\text{Ом}$, $R_2 = 5\,\text{Ом}$, $R_3 = 3\,\text{Ом}$, $R_4 = 4\,\text{Ом}$.
    При каком напряжении поданном на эту цепь, в ней потечёт ток равный $\eli = 5\,\text{А}$?

    \begin{tikzpicture}[rotate=180, circuit ee IEC, thick]
        \node [contact]  (contact1) at (-1.5, 0) {};
        \draw  (0, 0) to [resistor={info=$R_1$}] ++(left:1.5);
        \draw  (0, 0) -- ++(up:1.5) to [resistor={near start, info=$R_2$}, resistor={near end, info=$R_3$}] ++(right:3);
        \draw  (0, 0) to [resistor={info=$R_4$}] ++(right:3) -- ++(up:1.5);
        \draw  (1.5, 1.5) -- ++(up:1); \node [contact] (contact2) at (1.5, 2.5) {};
    \end{tikzpicture}
}
\answer{%
    $R=\frac{59}{12}\units{Ом} \approx 4{,}92\,\text{Ом} \implies U = \eli R \approx 24{,}6\,\text{В}.$
}
\solutionspace{120pt}

\tasknumber{2}%
\task{%
    Определите показания амперметра $3$ (см.
    рис.) и разность потенциалов на резисторе $6$,
    если сопротивления всех резисторов равны: $R_1 = R_2 = R_3 = R_4 = R_5 = R_6 = R = 5\,\text{Ом}$,
    а напряжение, поданное на цепь, равно $U = 150\,\text{В}$.
    Ответы получите в виде несократимых дробей, а также определите приближённые значения.
    Амперметры считать идеальными.

    \begin{tikzpicture}[circuit ee IEC, thick]
        \node [contact]  (left contact) at (3, 0) {};
        \node [contact]  (right contact) at (9, 0) {};
        \draw  (left contact) -- ++(up:2) to [resistor={very near start, info=$R_2$}, amperemeter={midway, info=$1$}, resistor={very near end, info=$R_3$} ] ++(right:6) -- (right contact);
        \draw  (left contact) -- ++(down:2) to [resistor={very near start, info=$R_5$}, resistor={midway, info=$R_6$}, amperemeter={very near end, info=$3$}] ++(right:6) -- (right contact);
        \draw  (left contact) ++(left:3) to [resistor={info=$R_1$}] (left contact) to [amperemeter={near start, info=$2$}, resistor={near end , info=$R_4$}] (right contact) -- ++(right:0.5);
    \end{tikzpicture}
}
\answer{%
    \begin{align*}
    R_0 &= R + \frac 1{\frac 1{R+R} + \frac 1R + \frac 1{R+R}} = R + \frac 1{\frac 2R} = \frac 32 R, \\
    \eli &= \frac U{R_0} = \frac {2U}{3R}, \\
    U_1 &= \eli R_1 = \frac {2U}{3R} \cdot R = \frac 23 U = 100{,}0\,\text{В}, \\
    U_{23} &= U_{56} = U_4 = U - \eli R_1 = U - \frac {2U}{3R} \cdot R = \frac U3 = 50{,}0\,\text{В}, \\
    \eli_2 &= \frac{U_4}{R_4} = \frac U{3R} \approx 10{,}0\,\text{А}, \\
    \eli_1 &= \frac{U_{23}}{R_{23}} = \frac{\frac U3}{R+R} = \frac U{6R} \approx 5{,}0\,\text{А}, \\
    \eli_3 &= \frac{U_{56}}{R_{56}} = \frac{\frac U3}{R+R} = \frac U{6R} \approx 5{,}0\,\text{А}, \\
    U_2 &= \eli_1 R_2 = \frac U{6R} \cdot R = \frac U6 = 25{,}0\,\text{В}, \\
    U_3 &= \eli_1 R_3 = \frac U{6R} \cdot R = \frac U3 = 25{,}0\,\text{В}, \\
    U_5 &= \eli_3 R_5 = \frac U{6R} \cdot R = \frac U5 = 25{,}0\,\text{В}, \\
    U_6 &= \eli_3 R_6 = \frac U{6R} \cdot R = \frac U6 = 25{,}0\,\text{В}.
    \end{align*}
}
\solutionspace{120pt}

\tasknumber{3}%
\task{%
    Определите ток, протекающий через резистор $R = 10\,\text{Ом}$ и разность потенциалов на нём (см.
    рис.),
    если $\ele_1 = 12\,\text{В}$, $\ele_2 = 25\,\text{В}$, $r_1 = 1\,\text{Ом}$, $r_2 = 6\,\text{Ом}$.

    \begin{tikzpicture}[circuit ee IEC, thick]
        \draw  (0, 0) to [battery={rotate=-180,info={$\ele_1, r_1$}}] (0, 3)
                -- (5, 3)
                to [battery={rotate=-180, info'={$\ele_2, r_2$}}] (5, 0)
                -- (0, 0)
                (2.5, 0) to [resistor={info=$R$}] (2.5, 3);
    \end{tikzpicture}
}
\answer{%
    Выберем 2 контура и один узел, запишем для них законы Кирхгофа:

    \begin{tikzpicture}[circuit ee IEC, thick]
        \draw  (0, 0) to [battery={rotate=-180,info={$\ele_1, r_1$}}, current direction={near end, info=$\eli_1$}] (0, 3)
                -- (5, 3)
                to [battery={rotate=-180, info'={$\ele_2, r_2$}}, current direction={near end, info=$\eli_2$}] (5, 0)
                -- (0, 0)
                (2.5, 0) to [resistor={info=$R$}, current direction'={near end, info=$\eli$}] (2.5, 3);
        \draw [-{Latex},color=red] (0.8, 1.9) arc [start angle = 135, end angle = -160, radius = 0.6];
        \draw [-{Latex},color=blue] (3.5, 1.9) arc [start angle = 135, end angle = -160, radius = 0.6];
        \node [contact,color=green!71!black] (topc) at (2.5, 3) {};
        \node [above] (top) at (2.5, 3) {$1$};
    \end{tikzpicture}

    \begin{align*}
        &\begin{cases}
            {\color{red} \ele_1 = \eli_1 r_1 + \eli R}, \\
            {\color{blue} \ele_2 = \eli_2 r_2 - \eli R}, \\
            {\color{green!71!black} \eli - \eli_1 - \eli_2 = 0};
        \end{cases}
        \qquad \implies \qquad
        \begin{cases}
            \eli_1 = \frac{\ele_1 - \eli R}{r_1}, \\
            \eli_2 = \frac{\ele_2 + \eli R}{r_2}, \\
            \eli - \eli_1 - \eli_2 = 0;
        \end{cases} \implies \\
        &\implies \eli - \frac{\ele_1 - \eli R}{r_1} + \frac{\ele_2 + \eli R}{r_2} = 0, \\
        &\eli\cbr{ 1 + \frac{R}{r_1} + \frac{R}{r_2}} - \frac{\ele_1}{r_1} + \frac{\ele_2}{r_2} = 0, \\
        &\eli
            = \frac{\frac{\ele_1}{r_1} - \frac{\ele_2}{r_2}}{ 1 + \frac{R}{r_1} + \frac{R}{r_2}}
            = \frac{\frac{12\,\text{В}}{1\,\text{Ом}} - \frac{25\,\text{В}}{6\,\text{Ом}}}{ 1 + \frac{10\,\text{Ом}}{1\,\text{Ом}} + \frac{10\,\text{Ом}}{6\,\text{Ом}}}
            = \frac{47}{76}\units{А}
            \approx 0{,}618\,\text{А}, \\
        &U  = \eli R = \frac{\frac{\ele_1}{r_1} - \frac{\ele_2}{r_2}}{ 1 + \frac{R}{r_1} + \frac{R}{r_2}} R
            \approx 6{,}184\,\text{В}.
    \end{align*}
}

\variantsplitter

\addpersonalvariant{Константин Козлов}

\tasknumber{1}%
\task{%
    Определите эквивалентное сопротивление цепи на рисунке (между выделенными на рисунке контактами),
    если известны сопротивления всех резисторов: $R_1 = 1\,\text{Ом}$, $R_2 = 5\,\text{Ом}$, $R_3 = 3\,\text{Ом}$, $R_4 = 2\,\text{Ом}$.
    При каком напряжении поданном на эту цепь, в ней потечёт ток равный $\eli = 5\,\text{А}$?

    \begin{tikzpicture}[rotate=90, circuit ee IEC, thick]
        \node [contact]  (contact1) at (-1.5, 0) {};
        \draw  (0, 0) to [resistor={info=$R_1$}] ++(left:1.5);
        \draw  (0, 0) -- ++(up:1.5) to [resistor={near start, info=$R_2$}, resistor={near end, info=$R_3$}] ++(right:3);
        \draw  (0, 0) to [resistor={info=$R_4$}] ++(right:3) -- ++(up:1.5);
        \draw  (1.5, 1.5) -- ++(up:1); \node [contact] (contact2) at (1.5, 2.5) {};
    \end{tikzpicture}
}
\answer{%
    $R=\frac72\units{Ом} \approx 3{,}50\,\text{Ом} \implies U = \eli R \approx 17{,}5\,\text{В}.$
}
\solutionspace{120pt}

\tasknumber{2}%
\task{%
    Определите показания амперметра $1$ (см.
    рис.) и разность потенциалов на резисторе $6$,
    если сопротивления всех резисторов равны: $R_1 = R_2 = R_3 = R_4 = R_5 = R_6 = R = 10\,\text{Ом}$,
    а напряжение, поданное на цепь, равно $U = 150\,\text{В}$.
    Ответы получите в виде несократимых дробей, а также определите приближённые значения.
    Амперметры считать идеальными.

    \begin{tikzpicture}[circuit ee IEC, thick]
        \node [contact]  (left contact) at (3, 0) {};
        \node [contact]  (right contact) at (9, 0) {};
        \draw  (left contact) -- ++(up:2) to [resistor={very near start, info=$R_2$}, amperemeter={midway, info=$1$}, resistor={very near end, info=$R_3$} ] ++(right:6) -- (right contact);
        \draw  (left contact) -- ++(down:2) to [resistor={very near start, info=$R_5$}, resistor={midway, info=$R_6$}, amperemeter={very near end, info=$3$}] ++(right:6) -- (right contact);
        \draw  (left contact) ++(left:3) to [resistor={info=$R_1$}] (left contact) to [amperemeter={near start, info=$2$}, resistor={near end , info=$R_4$}] (right contact) -- ++(right:0.5);
    \end{tikzpicture}
}
\answer{%
    \begin{align*}
    R_0 &= R + \frac 1{\frac 1{R+R} + \frac 1R + \frac 1{R+R}} = R + \frac 1{\frac 2R} = \frac 32 R, \\
    \eli &= \frac U{R_0} = \frac {2U}{3R}, \\
    U_1 &= \eli R_1 = \frac {2U}{3R} \cdot R = \frac 23 U = 100{,}0\,\text{В}, \\
    U_{23} &= U_{56} = U_4 = U - \eli R_1 = U - \frac {2U}{3R} \cdot R = \frac U3 = 50{,}0\,\text{В}, \\
    \eli_2 &= \frac{U_4}{R_4} = \frac U{3R} \approx 5{,}0\,\text{А}, \\
    \eli_1 &= \frac{U_{23}}{R_{23}} = \frac{\frac U3}{R+R} = \frac U{6R} \approx 2{,}5\,\text{А}, \\
    \eli_3 &= \frac{U_{56}}{R_{56}} = \frac{\frac U3}{R+R} = \frac U{6R} \approx 2{,}5\,\text{А}, \\
    U_2 &= \eli_1 R_2 = \frac U{6R} \cdot R = \frac U6 = 25{,}0\,\text{В}, \\
    U_3 &= \eli_1 R_3 = \frac U{6R} \cdot R = \frac U3 = 25{,}0\,\text{В}, \\
    U_5 &= \eli_3 R_5 = \frac U{6R} \cdot R = \frac U5 = 25{,}0\,\text{В}, \\
    U_6 &= \eli_3 R_6 = \frac U{6R} \cdot R = \frac U6 = 25{,}0\,\text{В}.
    \end{align*}
}
\solutionspace{120pt}

\tasknumber{3}%
\task{%
    Определите ток, протекающий через резистор $R = 10\,\text{Ом}$ и разность потенциалов на нём (см.
    рис.),
    если $\ele_1 = 12\,\text{В}$, $\ele_2 = 5\,\text{В}$, $r_1 = 1\,\text{Ом}$, $r_2 = 2\,\text{Ом}$.

    \begin{tikzpicture}[circuit ee IEC, thick]
        \draw  (0, 0) to [battery={rotate=-180,info={$\ele_1, r_1$}}] (0, 3)
                -- (5, 3)
                to [battery={rotate=-180, info'={$\ele_2, r_2$}}] (5, 0)
                -- (0, 0)
                (2.5, 0) to [resistor={info=$R$}] (2.5, 3);
    \end{tikzpicture}
}
\answer{%
    Выберем 2 контура и один узел, запишем для них законы Кирхгофа:

    \begin{tikzpicture}[circuit ee IEC, thick]
        \draw  (0, 0) to [battery={rotate=-180,info={$\ele_1, r_1$}}, current direction={near end, info=$\eli_1$}] (0, 3)
                -- (5, 3)
                to [battery={rotate=-180, info'={$\ele_2, r_2$}}, current direction={near end, info=$\eli_2$}] (5, 0)
                -- (0, 0)
                (2.5, 0) to [resistor={info=$R$}, current direction'={near end, info=$\eli$}] (2.5, 3);
        \draw [-{Latex},color=red] (0.8, 1.9) arc [start angle = 135, end angle = -160, radius = 0.6];
        \draw [-{Latex},color=blue] (3.5, 1.9) arc [start angle = 135, end angle = -160, radius = 0.6];
        \node [contact,color=green!71!black] (topc) at (2.5, 3) {};
        \node [above] (top) at (2.5, 3) {$1$};
    \end{tikzpicture}

    \begin{align*}
        &\begin{cases}
            {\color{red} \ele_1 = \eli_1 r_1 + \eli R}, \\
            {\color{blue} \ele_2 = \eli_2 r_2 - \eli R}, \\
            {\color{green!71!black} \eli - \eli_1 - \eli_2 = 0};
        \end{cases}
        \qquad \implies \qquad
        \begin{cases}
            \eli_1 = \frac{\ele_1 - \eli R}{r_1}, \\
            \eli_2 = \frac{\ele_2 + \eli R}{r_2}, \\
            \eli - \eli_1 - \eli_2 = 0;
        \end{cases} \implies \\
        &\implies \eli - \frac{\ele_1 - \eli R}{r_1} + \frac{\ele_2 + \eli R}{r_2} = 0, \\
        &\eli\cbr{ 1 + \frac{R}{r_1} + \frac{R}{r_2}} - \frac{\ele_1}{r_1} + \frac{\ele_2}{r_2} = 0, \\
        &\eli
            = \frac{\frac{\ele_1}{r_1} - \frac{\ele_2}{r_2}}{ 1 + \frac{R}{r_1} + \frac{R}{r_2}}
            = \frac{\frac{12\,\text{В}}{1\,\text{Ом}} - \frac{5\,\text{В}}{2\,\text{Ом}}}{ 1 + \frac{10\,\text{Ом}}{1\,\text{Ом}} + \frac{10\,\text{Ом}}{2\,\text{Ом}}}
            = \frac{19}{32}\units{А}
            \approx 0{,}594\,\text{А}, \\
        &U  = \eli R = \frac{\frac{\ele_1}{r_1} - \frac{\ele_2}{r_2}}{ 1 + \frac{R}{r_1} + \frac{R}{r_2}} R
            \approx 5{,}938\,\text{В}.
    \end{align*}
}

\variantsplitter

\addpersonalvariant{Наталья Кравченко}

\tasknumber{1}%
\task{%
    Определите эквивалентное сопротивление цепи на рисунке (между выделенными на рисунке контактами),
    если известны сопротивления всех резисторов: $R_1 = 1\,\text{Ом}$, $R_2 = 3\,\text{Ом}$, $R_3 = 3\,\text{Ом}$, $R_4 = 2\,\text{Ом}$.
    При каком напряжении поданном на эту цепь, в ней потечёт ток равный $\eli = 2\,\text{А}$?

    \begin{tikzpicture}[rotate=180, circuit ee IEC, thick]
        \node [contact]  (contact1) at (-1.5, 0) {};
        \draw  (0, 0) to [resistor={info=$R_1$}] ++(left:1.5);
        \draw  (0, 0) -- ++(up:1.5) to [resistor={near start, info=$R_2$}, resistor={near end, info=$R_3$}] ++(right:3);
        \draw  (0, 0) to [resistor={info=$R_4$}] ++(right:3) -- ++(up:1.5);
        \draw  (1.5, 1.5) -- ++(up:1); \node [contact] (contact2) at (1.5, 2.5) {};
    \end{tikzpicture}
}
\answer{%
    $R=\frac{23}8\units{Ом} \approx 2{,}88\,\text{Ом} \implies U = \eli R \approx 5{,}8\,\text{В}.$
}
\solutionspace{120pt}

\tasknumber{2}%
\task{%
    Определите показания амперметра $2$ (см.
    рис.) и разность потенциалов на резисторе $2$,
    если сопротивления всех резисторов равны: $R_1 = R_2 = R_3 = R_4 = R_5 = R_6 = R = 5\,\text{Ом}$,
    а напряжение, поданное на цепь, равно $U = 90\,\text{В}$.
    Ответы получите в виде несократимых дробей, а также определите приближённые значения.
    Амперметры считать идеальными.

    \begin{tikzpicture}[circuit ee IEC, thick]
        \node [contact]  (left contact) at (3, 0) {};
        \node [contact]  (right contact) at (9, 0) {};
        \draw  (left contact) -- ++(up:2) to [resistor={very near start, info=$R_2$}, amperemeter={midway, info=$1$}, resistor={very near end, info=$R_3$} ] ++(right:6) -- (right contact);
        \draw  (left contact) -- ++(down:2) to [resistor={very near start, info=$R_5$}, resistor={midway, info=$R_6$}, amperemeter={very near end, info=$3$}] ++(right:6) -- (right contact);
        \draw  (left contact) ++(left:3) to [resistor={info=$R_1$}] (left contact) to [amperemeter={near start, info=$2$}, resistor={near end , info=$R_4$}] (right contact) -- ++(right:0.5);
    \end{tikzpicture}
}
\answer{%
    \begin{align*}
    R_0 &= R + \frac 1{\frac 1{R+R} + \frac 1R + \frac 1{R+R}} = R + \frac 1{\frac 2R} = \frac 32 R, \\
    \eli &= \frac U{R_0} = \frac {2U}{3R}, \\
    U_1 &= \eli R_1 = \frac {2U}{3R} \cdot R = \frac 23 U = 60{,}0\,\text{В}, \\
    U_{23} &= U_{56} = U_4 = U - \eli R_1 = U - \frac {2U}{3R} \cdot R = \frac U3 = 30{,}0\,\text{В}, \\
    \eli_2 &= \frac{U_4}{R_4} = \frac U{3R} \approx 6{,}0\,\text{А}, \\
    \eli_1 &= \frac{U_{23}}{R_{23}} = \frac{\frac U3}{R+R} = \frac U{6R} \approx 3{,}0\,\text{А}, \\
    \eli_3 &= \frac{U_{56}}{R_{56}} = \frac{\frac U3}{R+R} = \frac U{6R} \approx 3{,}0\,\text{А}, \\
    U_2 &= \eli_1 R_2 = \frac U{6R} \cdot R = \frac U6 = 15{,}0\,\text{В}, \\
    U_3 &= \eli_1 R_3 = \frac U{6R} \cdot R = \frac U3 = 15{,}0\,\text{В}, \\
    U_5 &= \eli_3 R_5 = \frac U{6R} \cdot R = \frac U5 = 15{,}0\,\text{В}, \\
    U_6 &= \eli_3 R_6 = \frac U{6R} \cdot R = \frac U6 = 15{,}0\,\text{В}.
    \end{align*}
}
\solutionspace{120pt}

\tasknumber{3}%
\task{%
    Определите ток, протекающий через резистор $R = 15\,\text{Ом}$ и разность потенциалов на нём (см.
    рис.),
    если $\ele_1 = 18\,\text{В}$, $\ele_2 = 25\,\text{В}$, $r_1 = 2\,\text{Ом}$, $r_2 = 4\,\text{Ом}$.

    \begin{tikzpicture}[circuit ee IEC, thick]
        \draw  (0, 0) to [battery={rotate=-180,info={$\ele_1, r_1$}}] (0, 3)
                -- (5, 3)
                to [battery={rotate=-180, info'={$\ele_2, r_2$}}] (5, 0)
                -- (0, 0)
                (2.5, 0) to [resistor={info=$R$}] (2.5, 3);
    \end{tikzpicture}
}
\answer{%
    Выберем 2 контура и один узел, запишем для них законы Кирхгофа:

    \begin{tikzpicture}[circuit ee IEC, thick]
        \draw  (0, 0) to [battery={rotate=-180,info={$\ele_1, r_1$}}, current direction={near end, info=$\eli_1$}] (0, 3)
                -- (5, 3)
                to [battery={rotate=-180, info'={$\ele_2, r_2$}}, current direction={near end, info=$\eli_2$}] (5, 0)
                -- (0, 0)
                (2.5, 0) to [resistor={info=$R$}, current direction'={near end, info=$\eli$}] (2.5, 3);
        \draw [-{Latex},color=red] (0.8, 1.9) arc [start angle = 135, end angle = -160, radius = 0.6];
        \draw [-{Latex},color=blue] (3.5, 1.9) arc [start angle = 135, end angle = -160, radius = 0.6];
        \node [contact,color=green!71!black] (topc) at (2.5, 3) {};
        \node [above] (top) at (2.5, 3) {$1$};
    \end{tikzpicture}

    \begin{align*}
        &\begin{cases}
            {\color{red} \ele_1 = \eli_1 r_1 + \eli R}, \\
            {\color{blue} \ele_2 = \eli_2 r_2 - \eli R}, \\
            {\color{green!71!black} \eli - \eli_1 - \eli_2 = 0};
        \end{cases}
        \qquad \implies \qquad
        \begin{cases}
            \eli_1 = \frac{\ele_1 - \eli R}{r_1}, \\
            \eli_2 = \frac{\ele_2 + \eli R}{r_2}, \\
            \eli - \eli_1 - \eli_2 = 0;
        \end{cases} \implies \\
        &\implies \eli - \frac{\ele_1 - \eli R}{r_1} + \frac{\ele_2 + \eli R}{r_2} = 0, \\
        &\eli\cbr{ 1 + \frac{R}{r_1} + \frac{R}{r_2}} - \frac{\ele_1}{r_1} + \frac{\ele_2}{r_2} = 0, \\
        &\eli
            = \frac{\frac{\ele_1}{r_1} - \frac{\ele_2}{r_2}}{ 1 + \frac{R}{r_1} + \frac{R}{r_2}}
            = \frac{\frac{18\,\text{В}}{2\,\text{Ом}} - \frac{25\,\text{В}}{4\,\text{Ом}}}{ 1 + \frac{15\,\text{Ом}}{2\,\text{Ом}} + \frac{15\,\text{Ом}}{4\,\text{Ом}}}
            = \frac{11}{49}\units{А}
            \approx 0{,}224\,\text{А}, \\
        &U  = \eli R = \frac{\frac{\ele_1}{r_1} - \frac{\ele_2}{r_2}}{ 1 + \frac{R}{r_1} + \frac{R}{r_2}} R
            \approx 3{,}367\,\text{В}.
    \end{align*}
}

\variantsplitter

\addpersonalvariant{Матвей Кузьмин}

\tasknumber{1}%
\task{%
    Определите эквивалентное сопротивление цепи на рисунке (между выделенными на рисунке контактами),
    если известны сопротивления всех резисторов: $R_1 = 1\,\text{Ом}$, $R_2 = 3\,\text{Ом}$, $R_3 = 3\,\text{Ом}$, $R_4 = 4\,\text{Ом}$.
    При каком напряжении поданном на эту цепь, в ней потечёт ток равный $\eli = 5\,\text{А}$?

    \begin{tikzpicture}[rotate=0, circuit ee IEC, thick]
        \node [contact]  (contact1) at (-1.5, 0) {};
        \draw  (0, 0) to [resistor={info=$R_1$}] ++(left:1.5);
        \draw  (0, 0) -- ++(up:1.5) to [resistor={near start, info=$R_2$}, resistor={near end, info=$R_3$}] ++(right:3);
        \draw  (0, 0) to [resistor={info=$R_4$}] ++(right:3) -- ++(up:1.5);
        \draw  (1.5, 1.5) -- ++(up:1); \node [contact] (contact2) at (1.5, 2.5) {};
    \end{tikzpicture}
}
\answer{%
    $R=\frac{31}{10}\units{Ом} \approx 3{,}10\,\text{Ом} \implies U = \eli R \approx 15{,}5\,\text{В}.$
}
\solutionspace{120pt}

\tasknumber{2}%
\task{%
    Определите показания амперметра $2$ (см.
    рис.) и разность потенциалов на резисторе $5$,
    если сопротивления всех резисторов равны: $R_1 = R_2 = R_3 = R_4 = R_5 = R_6 = R = 4\,\text{Ом}$,
    а напряжение, поданное на цепь, равно $U = 30\,\text{В}$.
    Ответы получите в виде несократимых дробей, а также определите приближённые значения.
    Амперметры считать идеальными.

    \begin{tikzpicture}[circuit ee IEC, thick]
        \node [contact]  (left contact) at (3, 0) {};
        \node [contact]  (right contact) at (9, 0) {};
        \draw  (left contact) -- ++(up:2) to [resistor={very near start, info=$R_2$}, amperemeter={midway, info=$1$}, resistor={very near end, info=$R_3$} ] ++(right:6) -- (right contact);
        \draw  (left contact) -- ++(down:2) to [resistor={very near start, info=$R_5$}, resistor={midway, info=$R_6$}, amperemeter={very near end, info=$3$}] ++(right:6) -- (right contact);
        \draw  (left contact) ++(left:3) to [resistor={info=$R_1$}] (left contact) to [amperemeter={near start, info=$2$}, resistor={near end , info=$R_4$}] (right contact) -- ++(right:0.5);
    \end{tikzpicture}
}
\answer{%
    \begin{align*}
    R_0 &= R + \frac 1{\frac 1{R+R} + \frac 1R + \frac 1{R+R}} = R + \frac 1{\frac 2R} = \frac 32 R, \\
    \eli &= \frac U{R_0} = \frac {2U}{3R}, \\
    U_1 &= \eli R_1 = \frac {2U}{3R} \cdot R = \frac 23 U = 20{,}0\,\text{В}, \\
    U_{23} &= U_{56} = U_4 = U - \eli R_1 = U - \frac {2U}{3R} \cdot R = \frac U3 = 10{,}0\,\text{В}, \\
    \eli_2 &= \frac{U_4}{R_4} = \frac U{3R} \approx 2{,}5\,\text{А}, \\
    \eli_1 &= \frac{U_{23}}{R_{23}} = \frac{\frac U3}{R+R} = \frac U{6R} \approx 1{,}2\,\text{А}, \\
    \eli_3 &= \frac{U_{56}}{R_{56}} = \frac{\frac U3}{R+R} = \frac U{6R} \approx 1{,}2\,\text{А}, \\
    U_2 &= \eli_1 R_2 = \frac U{6R} \cdot R = \frac U6 = 5{,}0\,\text{В}, \\
    U_3 &= \eli_1 R_3 = \frac U{6R} \cdot R = \frac U3 = 5{,}0\,\text{В}, \\
    U_5 &= \eli_3 R_5 = \frac U{6R} \cdot R = \frac U5 = 5{,}0\,\text{В}, \\
    U_6 &= \eli_3 R_6 = \frac U{6R} \cdot R = \frac U6 = 5{,}0\,\text{В}.
    \end{align*}
}
\solutionspace{120pt}

\tasknumber{3}%
\task{%
    Определите ток, протекающий через резистор $R = 15\,\text{Ом}$ и разность потенциалов на нём (см.
    рис.),
    если $\ele_1 = 12\,\text{В}$, $\ele_2 = 5\,\text{В}$, $r_1 = 2\,\text{Ом}$, $r_2 = 6\,\text{Ом}$.

    \begin{tikzpicture}[circuit ee IEC, thick]
        \draw  (0, 0) to [battery={rotate=-180,info={$\ele_1, r_1$}}] (0, 3)
                -- (5, 3)
                to [battery={rotate=-180, info'={$\ele_2, r_2$}}] (5, 0)
                -- (0, 0)
                (2.5, 0) to [resistor={info=$R$}] (2.5, 3);
    \end{tikzpicture}
}
\answer{%
    Выберем 2 контура и один узел, запишем для них законы Кирхгофа:

    \begin{tikzpicture}[circuit ee IEC, thick]
        \draw  (0, 0) to [battery={rotate=-180,info={$\ele_1, r_1$}}, current direction={near end, info=$\eli_1$}] (0, 3)
                -- (5, 3)
                to [battery={rotate=-180, info'={$\ele_2, r_2$}}, current direction={near end, info=$\eli_2$}] (5, 0)
                -- (0, 0)
                (2.5, 0) to [resistor={info=$R$}, current direction'={near end, info=$\eli$}] (2.5, 3);
        \draw [-{Latex},color=red] (0.8, 1.9) arc [start angle = 135, end angle = -160, radius = 0.6];
        \draw [-{Latex},color=blue] (3.5, 1.9) arc [start angle = 135, end angle = -160, radius = 0.6];
        \node [contact,color=green!71!black] (topc) at (2.5, 3) {};
        \node [above] (top) at (2.5, 3) {$1$};
    \end{tikzpicture}

    \begin{align*}
        &\begin{cases}
            {\color{red} \ele_1 = \eli_1 r_1 + \eli R}, \\
            {\color{blue} \ele_2 = \eli_2 r_2 - \eli R}, \\
            {\color{green!71!black} \eli - \eli_1 - \eli_2 = 0};
        \end{cases}
        \qquad \implies \qquad
        \begin{cases}
            \eli_1 = \frac{\ele_1 - \eli R}{r_1}, \\
            \eli_2 = \frac{\ele_2 + \eli R}{r_2}, \\
            \eli - \eli_1 - \eli_2 = 0;
        \end{cases} \implies \\
        &\implies \eli - \frac{\ele_1 - \eli R}{r_1} + \frac{\ele_2 + \eli R}{r_2} = 0, \\
        &\eli\cbr{ 1 + \frac{R}{r_1} + \frac{R}{r_2}} - \frac{\ele_1}{r_1} + \frac{\ele_2}{r_2} = 0, \\
        &\eli
            = \frac{\frac{\ele_1}{r_1} - \frac{\ele_2}{r_2}}{ 1 + \frac{R}{r_1} + \frac{R}{r_2}}
            = \frac{\frac{12\,\text{В}}{2\,\text{Ом}} - \frac{5\,\text{В}}{6\,\text{Ом}}}{ 1 + \frac{15\,\text{Ом}}{2\,\text{Ом}} + \frac{15\,\text{Ом}}{6\,\text{Ом}}}
            = \frac{31}{66}\units{А}
            \approx 0{,}470\,\text{А}, \\
        &U  = \eli R = \frac{\frac{\ele_1}{r_1} - \frac{\ele_2}{r_2}}{ 1 + \frac{R}{r_1} + \frac{R}{r_2}} R
            \approx 7{,}045\,\text{В}.
    \end{align*}
}

\variantsplitter

\addpersonalvariant{Сергей Малышев}

\tasknumber{1}%
\task{%
    Определите эквивалентное сопротивление цепи на рисунке (между выделенными на рисунке контактами),
    если известны сопротивления всех резисторов: $R_1 = 2\,\text{Ом}$, $R_2 = 3\,\text{Ом}$, $R_3 = 2\,\text{Ом}$, $R_4 = 2\,\text{Ом}$.
    При каком напряжении поданном на эту цепь, в ней потечёт ток равный $\eli = 10\,\text{А}$?

    \begin{tikzpicture}[rotate=180, circuit ee IEC, thick]
        \node [contact]  (contact1) at (-1.5, 0) {};
        \draw  (0, 0) to [resistor={info=$R_1$}] ++(left:1.5);
        \draw  (0, 0) -- ++(up:1.5) to [resistor={near start, info=$R_2$}, resistor={near end, info=$R_3$}] ++(right:3);
        \draw  (0, 0) to [resistor={info=$R_4$}] ++(right:3) -- ++(up:1.5);
        \draw  (1.5, 1.5) -- ++(up:1); \node [contact] (contact2) at (1.5, 2.5) {};
    \end{tikzpicture}
}
\answer{%
    $R=\frac{26}7\units{Ом} \approx 3{,}71\,\text{Ом} \implies U = \eli R \approx 37{,}1\,\text{В}.$
}
\solutionspace{120pt}

\tasknumber{2}%
\task{%
    Определите показания амперметра $1$ (см.
    рис.) и разность потенциалов на резисторе $1$,
    если сопротивления всех резисторов равны: $R_1 = R_2 = R_3 = R_4 = R_5 = R_6 = R = 5\,\text{Ом}$,
    а напряжение, поданное на цепь, равно $U = 150\,\text{В}$.
    Ответы получите в виде несократимых дробей, а также определите приближённые значения.
    Амперметры считать идеальными.

    \begin{tikzpicture}[circuit ee IEC, thick]
        \node [contact]  (left contact) at (3, 0) {};
        \node [contact]  (right contact) at (9, 0) {};
        \draw  (left contact) -- ++(up:2) to [resistor={very near start, info=$R_2$}, amperemeter={midway, info=$1$}, resistor={very near end, info=$R_3$} ] ++(right:6) -- (right contact);
        \draw  (left contact) -- ++(down:2) to [resistor={very near start, info=$R_5$}, resistor={midway, info=$R_6$}, amperemeter={very near end, info=$3$}] ++(right:6) -- (right contact);
        \draw  (left contact) ++(left:3) to [resistor={info=$R_1$}] (left contact) to [amperemeter={near start, info=$2$}, resistor={near end , info=$R_4$}] (right contact) -- ++(right:0.5);
    \end{tikzpicture}
}
\answer{%
    \begin{align*}
    R_0 &= R + \frac 1{\frac 1{R+R} + \frac 1R + \frac 1{R+R}} = R + \frac 1{\frac 2R} = \frac 32 R, \\
    \eli &= \frac U{R_0} = \frac {2U}{3R}, \\
    U_1 &= \eli R_1 = \frac {2U}{3R} \cdot R = \frac 23 U = 100{,}0\,\text{В}, \\
    U_{23} &= U_{56} = U_4 = U - \eli R_1 = U - \frac {2U}{3R} \cdot R = \frac U3 = 50{,}0\,\text{В}, \\
    \eli_2 &= \frac{U_4}{R_4} = \frac U{3R} \approx 10{,}0\,\text{А}, \\
    \eli_1 &= \frac{U_{23}}{R_{23}} = \frac{\frac U3}{R+R} = \frac U{6R} \approx 5{,}0\,\text{А}, \\
    \eli_3 &= \frac{U_{56}}{R_{56}} = \frac{\frac U3}{R+R} = \frac U{6R} \approx 5{,}0\,\text{А}, \\
    U_2 &= \eli_1 R_2 = \frac U{6R} \cdot R = \frac U6 = 25{,}0\,\text{В}, \\
    U_3 &= \eli_1 R_3 = \frac U{6R} \cdot R = \frac U3 = 25{,}0\,\text{В}, \\
    U_5 &= \eli_3 R_5 = \frac U{6R} \cdot R = \frac U5 = 25{,}0\,\text{В}, \\
    U_6 &= \eli_3 R_6 = \frac U{6R} \cdot R = \frac U6 = 25{,}0\,\text{В}.
    \end{align*}
}
\solutionspace{120pt}

\tasknumber{3}%
\task{%
    Определите ток, протекающий через резистор $R = 15\,\text{Ом}$ и разность потенциалов на нём (см.
    рис.),
    если $\ele_1 = 6\,\text{В}$, $\ele_2 = 25\,\text{В}$, $r_1 = 3\,\text{Ом}$, $r_2 = 6\,\text{Ом}$.

    \begin{tikzpicture}[circuit ee IEC, thick]
        \draw  (0, 0) to [battery={rotate=-180,info={$\ele_1, r_1$}}] (0, 3)
                -- (5, 3)
                to [battery={rotate=-180, info'={$\ele_2, r_2$}}] (5, 0)
                -- (0, 0)
                (2.5, 0) to [resistor={info=$R$}] (2.5, 3);
    \end{tikzpicture}
}
\answer{%
    Выберем 2 контура и один узел, запишем для них законы Кирхгофа:

    \begin{tikzpicture}[circuit ee IEC, thick]
        \draw  (0, 0) to [battery={rotate=-180,info={$\ele_1, r_1$}}, current direction={near end, info=$\eli_1$}] (0, 3)
                -- (5, 3)
                to [battery={rotate=-180, info'={$\ele_2, r_2$}}, current direction={near end, info=$\eli_2$}] (5, 0)
                -- (0, 0)
                (2.5, 0) to [resistor={info=$R$}, current direction'={near end, info=$\eli$}] (2.5, 3);
        \draw [-{Latex},color=red] (0.8, 1.9) arc [start angle = 135, end angle = -160, radius = 0.6];
        \draw [-{Latex},color=blue] (3.5, 1.9) arc [start angle = 135, end angle = -160, radius = 0.6];
        \node [contact,color=green!71!black] (topc) at (2.5, 3) {};
        \node [above] (top) at (2.5, 3) {$1$};
    \end{tikzpicture}

    \begin{align*}
        &\begin{cases}
            {\color{red} \ele_1 = \eli_1 r_1 + \eli R}, \\
            {\color{blue} \ele_2 = \eli_2 r_2 - \eli R}, \\
            {\color{green!71!black} \eli - \eli_1 - \eli_2 = 0};
        \end{cases}
        \qquad \implies \qquad
        \begin{cases}
            \eli_1 = \frac{\ele_1 - \eli R}{r_1}, \\
            \eli_2 = \frac{\ele_2 + \eli R}{r_2}, \\
            \eli - \eli_1 - \eli_2 = 0;
        \end{cases} \implies \\
        &\implies \eli - \frac{\ele_1 - \eli R}{r_1} + \frac{\ele_2 + \eli R}{r_2} = 0, \\
        &\eli\cbr{ 1 + \frac{R}{r_1} + \frac{R}{r_2}} - \frac{\ele_1}{r_1} + \frac{\ele_2}{r_2} = 0, \\
        &\eli
            = \frac{\frac{\ele_1}{r_1} - \frac{\ele_2}{r_2}}{ 1 + \frac{R}{r_1} + \frac{R}{r_2}}
            = \frac{\frac{6\,\text{В}}{3\,\text{Ом}} - \frac{25\,\text{В}}{6\,\text{Ом}}}{ 1 + \frac{15\,\text{Ом}}{3\,\text{Ом}} + \frac{15\,\text{Ом}}{6\,\text{Ом}}}
            = -\frac{13}{51}\units{А}
            \approx -0{,}25500\,\text{А}, \\
        &U  = \eli R = \frac{\frac{\ele_1}{r_1} - \frac{\ele_2}{r_2}}{ 1 + \frac{R}{r_1} + \frac{R}{r_2}} R
            \approx -3{,}8240\,\text{В}.
    \end{align*}
}

\variantsplitter

\addpersonalvariant{Алина Полканова}

\tasknumber{1}%
\task{%
    Определите эквивалентное сопротивление цепи на рисунке (между выделенными на рисунке контактами),
    если известны сопротивления всех резисторов: $R_1 = 2\,\text{Ом}$, $R_2 = 4\,\text{Ом}$, $R_3 = 2\,\text{Ом}$, $R_4 = 2\,\text{Ом}$.
    При каком напряжении поданном на эту цепь, в ней потечёт ток равный $\eli = 10\,\text{А}$?

    \begin{tikzpicture}[rotate=90, circuit ee IEC, thick]
        \node [contact]  (contact1) at (-1.5, 0) {};
        \draw  (0, 0) to [resistor={info=$R_1$}] ++(left:1.5);
        \draw  (0, 0) -- ++(up:1.5) to [resistor={near start, info=$R_2$}, resistor={near end, info=$R_3$}] ++(right:3);
        \draw  (0, 0) to [resistor={info=$R_4$}] ++(right:3) -- ++(up:1.5);
        \draw  (1.5, 1.5) -- ++(up:1); \node [contact] (contact2) at (1.5, 2.5) {};
    \end{tikzpicture}
}
\answer{%
    $R=4\units{Ом} \approx 4{,}00\,\text{Ом} \implies U = \eli R \approx 40{,}0\,\text{В}.$
}
\solutionspace{120pt}

\tasknumber{2}%
\task{%
    Определите показания амперметра $3$ (см.
    рис.) и разность потенциалов на резисторе $3$,
    если сопротивления всех резисторов равны: $R_1 = R_2 = R_3 = R_4 = R_5 = R_6 = R = 5\,\text{Ом}$,
    а напряжение, поданное на цепь, равно $U = 30\,\text{В}$.
    Ответы получите в виде несократимых дробей, а также определите приближённые значения.
    Амперметры считать идеальными.

    \begin{tikzpicture}[circuit ee IEC, thick]
        \node [contact]  (left contact) at (3, 0) {};
        \node [contact]  (right contact) at (9, 0) {};
        \draw  (left contact) -- ++(up:2) to [resistor={very near start, info=$R_2$}, amperemeter={midway, info=$1$}, resistor={very near end, info=$R_3$} ] ++(right:6) -- (right contact);
        \draw  (left contact) -- ++(down:2) to [resistor={very near start, info=$R_5$}, resistor={midway, info=$R_6$}, amperemeter={very near end, info=$3$}] ++(right:6) -- (right contact);
        \draw  (left contact) ++(left:3) to [resistor={info=$R_1$}] (left contact) to [amperemeter={near start, info=$2$}, resistor={near end , info=$R_4$}] (right contact) -- ++(right:0.5);
    \end{tikzpicture}
}
\answer{%
    \begin{align*}
    R_0 &= R + \frac 1{\frac 1{R+R} + \frac 1R + \frac 1{R+R}} = R + \frac 1{\frac 2R} = \frac 32 R, \\
    \eli &= \frac U{R_0} = \frac {2U}{3R}, \\
    U_1 &= \eli R_1 = \frac {2U}{3R} \cdot R = \frac 23 U = 20{,}0\,\text{В}, \\
    U_{23} &= U_{56} = U_4 = U - \eli R_1 = U - \frac {2U}{3R} \cdot R = \frac U3 = 10{,}0\,\text{В}, \\
    \eli_2 &= \frac{U_4}{R_4} = \frac U{3R} \approx 2{,}0\,\text{А}, \\
    \eli_1 &= \frac{U_{23}}{R_{23}} = \frac{\frac U3}{R+R} = \frac U{6R} \approx 1{,}0\,\text{А}, \\
    \eli_3 &= \frac{U_{56}}{R_{56}} = \frac{\frac U3}{R+R} = \frac U{6R} \approx 1{,}0\,\text{А}, \\
    U_2 &= \eli_1 R_2 = \frac U{6R} \cdot R = \frac U6 = 5{,}0\,\text{В}, \\
    U_3 &= \eli_1 R_3 = \frac U{6R} \cdot R = \frac U3 = 5{,}0\,\text{В}, \\
    U_5 &= \eli_3 R_5 = \frac U{6R} \cdot R = \frac U5 = 5{,}0\,\text{В}, \\
    U_6 &= \eli_3 R_6 = \frac U{6R} \cdot R = \frac U6 = 5{,}0\,\text{В}.
    \end{align*}
}
\solutionspace{120pt}

\tasknumber{3}%
\task{%
    Определите ток, протекающий через резистор $R = 18\,\text{Ом}$ и разность потенциалов на нём (см.
    рис.),
    если $\ele_1 = 12\,\text{В}$, $\ele_2 = 15\,\text{В}$, $r_1 = 1\,\text{Ом}$, $r_2 = 2\,\text{Ом}$.

    \begin{tikzpicture}[circuit ee IEC, thick]
        \draw  (0, 0) to [battery={rotate=-180,info={$\ele_1, r_1$}}] (0, 3)
                -- (5, 3)
                to [battery={rotate=-180, info'={$\ele_2, r_2$}}] (5, 0)
                -- (0, 0)
                (2.5, 0) to [resistor={info=$R$}] (2.5, 3);
    \end{tikzpicture}
}
\answer{%
    Выберем 2 контура и один узел, запишем для них законы Кирхгофа:

    \begin{tikzpicture}[circuit ee IEC, thick]
        \draw  (0, 0) to [battery={rotate=-180,info={$\ele_1, r_1$}}, current direction={near end, info=$\eli_1$}] (0, 3)
                -- (5, 3)
                to [battery={rotate=-180, info'={$\ele_2, r_2$}}, current direction={near end, info=$\eli_2$}] (5, 0)
                -- (0, 0)
                (2.5, 0) to [resistor={info=$R$}, current direction'={near end, info=$\eli$}] (2.5, 3);
        \draw [-{Latex},color=red] (0.8, 1.9) arc [start angle = 135, end angle = -160, radius = 0.6];
        \draw [-{Latex},color=blue] (3.5, 1.9) arc [start angle = 135, end angle = -160, radius = 0.6];
        \node [contact,color=green!71!black] (topc) at (2.5, 3) {};
        \node [above] (top) at (2.5, 3) {$1$};
    \end{tikzpicture}

    \begin{align*}
        &\begin{cases}
            {\color{red} \ele_1 = \eli_1 r_1 + \eli R}, \\
            {\color{blue} \ele_2 = \eli_2 r_2 - \eli R}, \\
            {\color{green!71!black} \eli - \eli_1 - \eli_2 = 0};
        \end{cases}
        \qquad \implies \qquad
        \begin{cases}
            \eli_1 = \frac{\ele_1 - \eli R}{r_1}, \\
            \eli_2 = \frac{\ele_2 + \eli R}{r_2}, \\
            \eli - \eli_1 - \eli_2 = 0;
        \end{cases} \implies \\
        &\implies \eli - \frac{\ele_1 - \eli R}{r_1} + \frac{\ele_2 + \eli R}{r_2} = 0, \\
        &\eli\cbr{ 1 + \frac{R}{r_1} + \frac{R}{r_2}} - \frac{\ele_1}{r_1} + \frac{\ele_2}{r_2} = 0, \\
        &\eli
            = \frac{\frac{\ele_1}{r_1} - \frac{\ele_2}{r_2}}{ 1 + \frac{R}{r_1} + \frac{R}{r_2}}
            = \frac{\frac{12\,\text{В}}{1\,\text{Ом}} - \frac{15\,\text{В}}{2\,\text{Ом}}}{ 1 + \frac{18\,\text{Ом}}{1\,\text{Ом}} + \frac{18\,\text{Ом}}{2\,\text{Ом}}}
            = \frac9{56}\units{А}
            \approx 0{,}161\,\text{А}, \\
        &U  = \eli R = \frac{\frac{\ele_1}{r_1} - \frac{\ele_2}{r_2}}{ 1 + \frac{R}{r_1} + \frac{R}{r_2}} R
            \approx 2{,}893\,\text{В}.
    \end{align*}
}

\variantsplitter

\addpersonalvariant{Сергей Пономарёв}

\tasknumber{1}%
\task{%
    Определите эквивалентное сопротивление цепи на рисунке (между выделенными на рисунке контактами),
    если известны сопротивления всех резисторов: $R_1 = 2\,\text{Ом}$, $R_2 = 4\,\text{Ом}$, $R_3 = 1\,\text{Ом}$, $R_4 = 4\,\text{Ом}$.
    При каком напряжении поданном на эту цепь, в ней потечёт ток равный $\eli = 10\,\text{А}$?

    \begin{tikzpicture}[rotate=270, circuit ee IEC, thick]
        \node [contact]  (contact1) at (-1.5, 0) {};
        \draw  (0, 0) to [resistor={info=$R_1$}] ++(left:1.5);
        \draw  (0, 0) -- ++(up:1.5) to [resistor={near start, info=$R_2$}, resistor={near end, info=$R_3$}] ++(right:3);
        \draw  (0, 0) to [resistor={info=$R_4$}] ++(right:3) -- ++(up:1.5);
        \draw  (1.5, 1.5) -- ++(up:1); \node [contact] (contact2) at (1.5, 2.5) {};
    \end{tikzpicture}
}
\answer{%
    $R=\frac{38}9\units{Ом} \approx 4{,}22\,\text{Ом} \implies U = \eli R \approx 42{,}2\,\text{В}.$
}
\solutionspace{120pt}

\tasknumber{2}%
\task{%
    Определите показания амперметра $3$ (см.
    рис.) и разность потенциалов на резисторе $6$,
    если сопротивления всех резисторов равны: $R_1 = R_2 = R_3 = R_4 = R_5 = R_6 = R = 4\,\text{Ом}$,
    а напряжение, поданное на цепь, равно $U = 60\,\text{В}$.
    Ответы получите в виде несократимых дробей, а также определите приближённые значения.
    Амперметры считать идеальными.

    \begin{tikzpicture}[circuit ee IEC, thick]
        \node [contact]  (left contact) at (3, 0) {};
        \node [contact]  (right contact) at (9, 0) {};
        \draw  (left contact) -- ++(up:2) to [resistor={very near start, info=$R_2$}, amperemeter={midway, info=$1$}, resistor={very near end, info=$R_3$} ] ++(right:6) -- (right contact);
        \draw  (left contact) -- ++(down:2) to [resistor={very near start, info=$R_5$}, resistor={midway, info=$R_6$}, amperemeter={very near end, info=$3$}] ++(right:6) -- (right contact);
        \draw  (left contact) ++(left:3) to [resistor={info=$R_1$}] (left contact) to [amperemeter={near start, info=$2$}, resistor={near end , info=$R_4$}] (right contact) -- ++(right:0.5);
    \end{tikzpicture}
}
\answer{%
    \begin{align*}
    R_0 &= R + \frac 1{\frac 1{R+R} + \frac 1R + \frac 1{R+R}} = R + \frac 1{\frac 2R} = \frac 32 R, \\
    \eli &= \frac U{R_0} = \frac {2U}{3R}, \\
    U_1 &= \eli R_1 = \frac {2U}{3R} \cdot R = \frac 23 U = 40{,}0\,\text{В}, \\
    U_{23} &= U_{56} = U_4 = U - \eli R_1 = U - \frac {2U}{3R} \cdot R = \frac U3 = 20{,}0\,\text{В}, \\
    \eli_2 &= \frac{U_4}{R_4} = \frac U{3R} \approx 5{,}0\,\text{А}, \\
    \eli_1 &= \frac{U_{23}}{R_{23}} = \frac{\frac U3}{R+R} = \frac U{6R} \approx 2{,}5\,\text{А}, \\
    \eli_3 &= \frac{U_{56}}{R_{56}} = \frac{\frac U3}{R+R} = \frac U{6R} \approx 2{,}5\,\text{А}, \\
    U_2 &= \eli_1 R_2 = \frac U{6R} \cdot R = \frac U6 = 10{,}0\,\text{В}, \\
    U_3 &= \eli_1 R_3 = \frac U{6R} \cdot R = \frac U3 = 10{,}0\,\text{В}, \\
    U_5 &= \eli_3 R_5 = \frac U{6R} \cdot R = \frac U5 = 10{,}0\,\text{В}, \\
    U_6 &= \eli_3 R_6 = \frac U{6R} \cdot R = \frac U6 = 10{,}0\,\text{В}.
    \end{align*}
}
\solutionspace{120pt}

\tasknumber{3}%
\task{%
    Определите ток, протекающий через резистор $R = 10\,\text{Ом}$ и разность потенциалов на нём (см.
    рис.),
    если $\ele_1 = 18\,\text{В}$, $\ele_2 = 5\,\text{В}$, $r_1 = 3\,\text{Ом}$, $r_2 = 4\,\text{Ом}$.

    \begin{tikzpicture}[circuit ee IEC, thick]
        \draw  (0, 0) to [battery={rotate=-180,info={$\ele_1, r_1$}}] (0, 3)
                -- (5, 3)
                to [battery={rotate=-180, info'={$\ele_2, r_2$}}] (5, 0)
                -- (0, 0)
                (2.5, 0) to [resistor={info=$R$}] (2.5, 3);
    \end{tikzpicture}
}
\answer{%
    Выберем 2 контура и один узел, запишем для них законы Кирхгофа:

    \begin{tikzpicture}[circuit ee IEC, thick]
        \draw  (0, 0) to [battery={rotate=-180,info={$\ele_1, r_1$}}, current direction={near end, info=$\eli_1$}] (0, 3)
                -- (5, 3)
                to [battery={rotate=-180, info'={$\ele_2, r_2$}}, current direction={near end, info=$\eli_2$}] (5, 0)
                -- (0, 0)
                (2.5, 0) to [resistor={info=$R$}, current direction'={near end, info=$\eli$}] (2.5, 3);
        \draw [-{Latex},color=red] (0.8, 1.9) arc [start angle = 135, end angle = -160, radius = 0.6];
        \draw [-{Latex},color=blue] (3.5, 1.9) arc [start angle = 135, end angle = -160, radius = 0.6];
        \node [contact,color=green!71!black] (topc) at (2.5, 3) {};
        \node [above] (top) at (2.5, 3) {$1$};
    \end{tikzpicture}

    \begin{align*}
        &\begin{cases}
            {\color{red} \ele_1 = \eli_1 r_1 + \eli R}, \\
            {\color{blue} \ele_2 = \eli_2 r_2 - \eli R}, \\
            {\color{green!71!black} \eli - \eli_1 - \eli_2 = 0};
        \end{cases}
        \qquad \implies \qquad
        \begin{cases}
            \eli_1 = \frac{\ele_1 - \eli R}{r_1}, \\
            \eli_2 = \frac{\ele_2 + \eli R}{r_2}, \\
            \eli - \eli_1 - \eli_2 = 0;
        \end{cases} \implies \\
        &\implies \eli - \frac{\ele_1 - \eli R}{r_1} + \frac{\ele_2 + \eli R}{r_2} = 0, \\
        &\eli\cbr{ 1 + \frac{R}{r_1} + \frac{R}{r_2}} - \frac{\ele_1}{r_1} + \frac{\ele_2}{r_2} = 0, \\
        &\eli
            = \frac{\frac{\ele_1}{r_1} - \frac{\ele_2}{r_2}}{ 1 + \frac{R}{r_1} + \frac{R}{r_2}}
            = \frac{\frac{18\,\text{В}}{3\,\text{Ом}} - \frac{5\,\text{В}}{4\,\text{Ом}}}{ 1 + \frac{10\,\text{Ом}}{3\,\text{Ом}} + \frac{10\,\text{Ом}}{4\,\text{Ом}}}
            = \frac{57}{82}\units{А}
            \approx 0{,}695\,\text{А}, \\
        &U  = \eli R = \frac{\frac{\ele_1}{r_1} - \frac{\ele_2}{r_2}}{ 1 + \frac{R}{r_1} + \frac{R}{r_2}} R
            \approx 6{,}951\,\text{В}.
    \end{align*}
}

\variantsplitter

\addpersonalvariant{Егор Свистушкин}

\tasknumber{1}%
\task{%
    Определите эквивалентное сопротивление цепи на рисунке (между выделенными на рисунке контактами),
    если известны сопротивления всех резисторов: $R_1 = 2\,\text{Ом}$, $R_2 = 4\,\text{Ом}$, $R_3 = 1\,\text{Ом}$, $R_4 = 4\,\text{Ом}$.
    При каком напряжении поданном на эту цепь, в ней потечёт ток равный $\eli = 10\,\text{А}$?

    \begin{tikzpicture}[rotate=180, circuit ee IEC, thick]
        \node [contact]  (contact1) at (-1.5, 0) {};
        \draw  (0, 0) to [resistor={info=$R_1$}] ++(left:1.5);
        \draw  (0, 0) -- ++(up:1.5) to [resistor={near start, info=$R_2$}, resistor={near end, info=$R_3$}] ++(right:3);
        \draw  (0, 0) to [resistor={info=$R_4$}] ++(right:3) -- ++(up:1.5);
        \draw  (1.5, 1.5) -- ++(up:1); \node [contact] (contact2) at (1.5, 2.5) {};
    \end{tikzpicture}
}
\answer{%
    $R=\frac{38}9\units{Ом} \approx 4{,}22\,\text{Ом} \implies U = \eli R \approx 42{,}2\,\text{В}.$
}
\solutionspace{120pt}

\tasknumber{2}%
\task{%
    Определите показания амперметра $3$ (см.
    рис.) и разность потенциалов на резисторе $6$,
    если сопротивления всех резисторов равны: $R_1 = R_2 = R_3 = R_4 = R_5 = R_6 = R = 5\,\text{Ом}$,
    а напряжение, поданное на цепь, равно $U = 90\,\text{В}$.
    Ответы получите в виде несократимых дробей, а также определите приближённые значения.
    Амперметры считать идеальными.

    \begin{tikzpicture}[circuit ee IEC, thick]
        \node [contact]  (left contact) at (3, 0) {};
        \node [contact]  (right contact) at (9, 0) {};
        \draw  (left contact) -- ++(up:2) to [resistor={very near start, info=$R_2$}, amperemeter={midway, info=$1$}, resistor={very near end, info=$R_3$} ] ++(right:6) -- (right contact);
        \draw  (left contact) -- ++(down:2) to [resistor={very near start, info=$R_5$}, resistor={midway, info=$R_6$}, amperemeter={very near end, info=$3$}] ++(right:6) -- (right contact);
        \draw  (left contact) ++(left:3) to [resistor={info=$R_1$}] (left contact) to [amperemeter={near start, info=$2$}, resistor={near end , info=$R_4$}] (right contact) -- ++(right:0.5);
    \end{tikzpicture}
}
\answer{%
    \begin{align*}
    R_0 &= R + \frac 1{\frac 1{R+R} + \frac 1R + \frac 1{R+R}} = R + \frac 1{\frac 2R} = \frac 32 R, \\
    \eli &= \frac U{R_0} = \frac {2U}{3R}, \\
    U_1 &= \eli R_1 = \frac {2U}{3R} \cdot R = \frac 23 U = 60{,}0\,\text{В}, \\
    U_{23} &= U_{56} = U_4 = U - \eli R_1 = U - \frac {2U}{3R} \cdot R = \frac U3 = 30{,}0\,\text{В}, \\
    \eli_2 &= \frac{U_4}{R_4} = \frac U{3R} \approx 6{,}0\,\text{А}, \\
    \eli_1 &= \frac{U_{23}}{R_{23}} = \frac{\frac U3}{R+R} = \frac U{6R} \approx 3{,}0\,\text{А}, \\
    \eli_3 &= \frac{U_{56}}{R_{56}} = \frac{\frac U3}{R+R} = \frac U{6R} \approx 3{,}0\,\text{А}, \\
    U_2 &= \eli_1 R_2 = \frac U{6R} \cdot R = \frac U6 = 15{,}0\,\text{В}, \\
    U_3 &= \eli_1 R_3 = \frac U{6R} \cdot R = \frac U3 = 15{,}0\,\text{В}, \\
    U_5 &= \eli_3 R_5 = \frac U{6R} \cdot R = \frac U5 = 15{,}0\,\text{В}, \\
    U_6 &= \eli_3 R_6 = \frac U{6R} \cdot R = \frac U6 = 15{,}0\,\text{В}.
    \end{align*}
}
\solutionspace{120pt}

\tasknumber{3}%
\task{%
    Определите ток, протекающий через резистор $R = 15\,\text{Ом}$ и разность потенциалов на нём (см.
    рис.),
    если $\ele_1 = 12\,\text{В}$, $\ele_2 = 25\,\text{В}$, $r_1 = 3\,\text{Ом}$, $r_2 = 6\,\text{Ом}$.

    \begin{tikzpicture}[circuit ee IEC, thick]
        \draw  (0, 0) to [battery={rotate=-180,info={$\ele_1, r_1$}}] (0, 3)
                -- (5, 3)
                to [battery={rotate=-180, info'={$\ele_2, r_2$}}] (5, 0)
                -- (0, 0)
                (2.5, 0) to [resistor={info=$R$}] (2.5, 3);
    \end{tikzpicture}
}
\answer{%
    Выберем 2 контура и один узел, запишем для них законы Кирхгофа:

    \begin{tikzpicture}[circuit ee IEC, thick]
        \draw  (0, 0) to [battery={rotate=-180,info={$\ele_1, r_1$}}, current direction={near end, info=$\eli_1$}] (0, 3)
                -- (5, 3)
                to [battery={rotate=-180, info'={$\ele_2, r_2$}}, current direction={near end, info=$\eli_2$}] (5, 0)
                -- (0, 0)
                (2.5, 0) to [resistor={info=$R$}, current direction'={near end, info=$\eli$}] (2.5, 3);
        \draw [-{Latex},color=red] (0.8, 1.9) arc [start angle = 135, end angle = -160, radius = 0.6];
        \draw [-{Latex},color=blue] (3.5, 1.9) arc [start angle = 135, end angle = -160, radius = 0.6];
        \node [contact,color=green!71!black] (topc) at (2.5, 3) {};
        \node [above] (top) at (2.5, 3) {$1$};
    \end{tikzpicture}

    \begin{align*}
        &\begin{cases}
            {\color{red} \ele_1 = \eli_1 r_1 + \eli R}, \\
            {\color{blue} \ele_2 = \eli_2 r_2 - \eli R}, \\
            {\color{green!71!black} \eli - \eli_1 - \eli_2 = 0};
        \end{cases}
        \qquad \implies \qquad
        \begin{cases}
            \eli_1 = \frac{\ele_1 - \eli R}{r_1}, \\
            \eli_2 = \frac{\ele_2 + \eli R}{r_2}, \\
            \eli - \eli_1 - \eli_2 = 0;
        \end{cases} \implies \\
        &\implies \eli - \frac{\ele_1 - \eli R}{r_1} + \frac{\ele_2 + \eli R}{r_2} = 0, \\
        &\eli\cbr{ 1 + \frac{R}{r_1} + \frac{R}{r_2}} - \frac{\ele_1}{r_1} + \frac{\ele_2}{r_2} = 0, \\
        &\eli
            = \frac{\frac{\ele_1}{r_1} - \frac{\ele_2}{r_2}}{ 1 + \frac{R}{r_1} + \frac{R}{r_2}}
            = \frac{\frac{12\,\text{В}}{3\,\text{Ом}} - \frac{25\,\text{В}}{6\,\text{Ом}}}{ 1 + \frac{15\,\text{Ом}}{3\,\text{Ом}} + \frac{15\,\text{Ом}}{6\,\text{Ом}}}
            = -\frac1{51}\units{А}
            \approx -0{,}020000\,\text{А}, \\
        &U  = \eli R = \frac{\frac{\ele_1}{r_1} - \frac{\ele_2}{r_2}}{ 1 + \frac{R}{r_1} + \frac{R}{r_2}} R
            \approx -0{,}29400\,\text{В}.
    \end{align*}
}

\variantsplitter

\addpersonalvariant{Дмитрий Соколов}

\tasknumber{1}%
\task{%
    Определите эквивалентное сопротивление цепи на рисунке (между выделенными на рисунке контактами),
    если известны сопротивления всех резисторов: $R_1 = 2\,\text{Ом}$, $R_2 = 5\,\text{Ом}$, $R_3 = 1\,\text{Ом}$, $R_4 = 3\,\text{Ом}$.
    При каком напряжении поданном на эту цепь, в ней потечёт ток равный $\eli = 5\,\text{А}$?

    \begin{tikzpicture}[rotate=90, circuit ee IEC, thick]
        \node [contact]  (contact1) at (-1.5, 0) {};
        \draw  (0, 0) to [resistor={info=$R_1$}] ++(left:1.5);
        \draw  (0, 0) -- ++(up:1.5) to [resistor={near start, info=$R_2$}, resistor={near end, info=$R_3$}] ++(right:3);
        \draw  (0, 0) to [resistor={info=$R_4$}] ++(right:3) -- ++(up:1.5);
        \draw  (1.5, 1.5) -- ++(up:1); \node [contact] (contact2) at (1.5, 2.5) {};
    \end{tikzpicture}
}
\answer{%
    $R=\frac{38}9\units{Ом} \approx 4{,}22\,\text{Ом} \implies U = \eli R \approx 21{,}1\,\text{В}.$
}
\solutionspace{120pt}

\tasknumber{2}%
\task{%
    Определите показания амперметра $3$ (см.
    рис.) и разность потенциалов на резисторе $2$,
    если сопротивления всех резисторов равны: $R_1 = R_2 = R_3 = R_4 = R_5 = R_6 = R = 5\,\text{Ом}$,
    а напряжение, поданное на цепь, равно $U = 120\,\text{В}$.
    Ответы получите в виде несократимых дробей, а также определите приближённые значения.
    Амперметры считать идеальными.

    \begin{tikzpicture}[circuit ee IEC, thick]
        \node [contact]  (left contact) at (3, 0) {};
        \node [contact]  (right contact) at (9, 0) {};
        \draw  (left contact) -- ++(up:2) to [resistor={very near start, info=$R_2$}, amperemeter={midway, info=$1$}, resistor={very near end, info=$R_3$} ] ++(right:6) -- (right contact);
        \draw  (left contact) -- ++(down:2) to [resistor={very near start, info=$R_5$}, resistor={midway, info=$R_6$}, amperemeter={very near end, info=$3$}] ++(right:6) -- (right contact);
        \draw  (left contact) ++(left:3) to [resistor={info=$R_1$}] (left contact) to [amperemeter={near start, info=$2$}, resistor={near end , info=$R_4$}] (right contact) -- ++(right:0.5);
    \end{tikzpicture}
}
\answer{%
    \begin{align*}
    R_0 &= R + \frac 1{\frac 1{R+R} + \frac 1R + \frac 1{R+R}} = R + \frac 1{\frac 2R} = \frac 32 R, \\
    \eli &= \frac U{R_0} = \frac {2U}{3R}, \\
    U_1 &= \eli R_1 = \frac {2U}{3R} \cdot R = \frac 23 U = 80{,}0\,\text{В}, \\
    U_{23} &= U_{56} = U_4 = U - \eli R_1 = U - \frac {2U}{3R} \cdot R = \frac U3 = 40{,}0\,\text{В}, \\
    \eli_2 &= \frac{U_4}{R_4} = \frac U{3R} \approx 8{,}0\,\text{А}, \\
    \eli_1 &= \frac{U_{23}}{R_{23}} = \frac{\frac U3}{R+R} = \frac U{6R} \approx 4{,}0\,\text{А}, \\
    \eli_3 &= \frac{U_{56}}{R_{56}} = \frac{\frac U3}{R+R} = \frac U{6R} \approx 4{,}0\,\text{А}, \\
    U_2 &= \eli_1 R_2 = \frac U{6R} \cdot R = \frac U6 = 20{,}0\,\text{В}, \\
    U_3 &= \eli_1 R_3 = \frac U{6R} \cdot R = \frac U3 = 20{,}0\,\text{В}, \\
    U_5 &= \eli_3 R_5 = \frac U{6R} \cdot R = \frac U5 = 20{,}0\,\text{В}, \\
    U_6 &= \eli_3 R_6 = \frac U{6R} \cdot R = \frac U6 = 20{,}0\,\text{В}.
    \end{align*}
}
\solutionspace{120pt}

\tasknumber{3}%
\task{%
    Определите ток, протекающий через резистор $R = 18\,\text{Ом}$ и разность потенциалов на нём (см.
    рис.),
    если $\ele_1 = 6\,\text{В}$, $\ele_2 = 25\,\text{В}$, $r_1 = 1\,\text{Ом}$, $r_2 = 2\,\text{Ом}$.

    \begin{tikzpicture}[circuit ee IEC, thick]
        \draw  (0, 0) to [battery={rotate=-180,info={$\ele_1, r_1$}}] (0, 3)
                -- (5, 3)
                to [battery={rotate=-180, info'={$\ele_2, r_2$}}] (5, 0)
                -- (0, 0)
                (2.5, 0) to [resistor={info=$R$}] (2.5, 3);
    \end{tikzpicture}
}
\answer{%
    Выберем 2 контура и один узел, запишем для них законы Кирхгофа:

    \begin{tikzpicture}[circuit ee IEC, thick]
        \draw  (0, 0) to [battery={rotate=-180,info={$\ele_1, r_1$}}, current direction={near end, info=$\eli_1$}] (0, 3)
                -- (5, 3)
                to [battery={rotate=-180, info'={$\ele_2, r_2$}}, current direction={near end, info=$\eli_2$}] (5, 0)
                -- (0, 0)
                (2.5, 0) to [resistor={info=$R$}, current direction'={near end, info=$\eli$}] (2.5, 3);
        \draw [-{Latex},color=red] (0.8, 1.9) arc [start angle = 135, end angle = -160, radius = 0.6];
        \draw [-{Latex},color=blue] (3.5, 1.9) arc [start angle = 135, end angle = -160, radius = 0.6];
        \node [contact,color=green!71!black] (topc) at (2.5, 3) {};
        \node [above] (top) at (2.5, 3) {$1$};
    \end{tikzpicture}

    \begin{align*}
        &\begin{cases}
            {\color{red} \ele_1 = \eli_1 r_1 + \eli R}, \\
            {\color{blue} \ele_2 = \eli_2 r_2 - \eli R}, \\
            {\color{green!71!black} \eli - \eli_1 - \eli_2 = 0};
        \end{cases}
        \qquad \implies \qquad
        \begin{cases}
            \eli_1 = \frac{\ele_1 - \eli R}{r_1}, \\
            \eli_2 = \frac{\ele_2 + \eli R}{r_2}, \\
            \eli - \eli_1 - \eli_2 = 0;
        \end{cases} \implies \\
        &\implies \eli - \frac{\ele_1 - \eli R}{r_1} + \frac{\ele_2 + \eli R}{r_2} = 0, \\
        &\eli\cbr{ 1 + \frac{R}{r_1} + \frac{R}{r_2}} - \frac{\ele_1}{r_1} + \frac{\ele_2}{r_2} = 0, \\
        &\eli
            = \frac{\frac{\ele_1}{r_1} - \frac{\ele_2}{r_2}}{ 1 + \frac{R}{r_1} + \frac{R}{r_2}}
            = \frac{\frac{6\,\text{В}}{1\,\text{Ом}} - \frac{25\,\text{В}}{2\,\text{Ом}}}{ 1 + \frac{18\,\text{Ом}}{1\,\text{Ом}} + \frac{18\,\text{Ом}}{2\,\text{Ом}}}
            = -\frac{13}{56}\units{А}
            \approx -0{,}23200\,\text{А}, \\
        &U  = \eli R = \frac{\frac{\ele_1}{r_1} - \frac{\ele_2}{r_2}}{ 1 + \frac{R}{r_1} + \frac{R}{r_2}} R
            \approx -4{,}1790\,\text{В}.
    \end{align*}
}
% autogenerated
