\setdate{11~декабря~2020}
\setclass{10«АБ»}

\addpersonalvariant{Михаил Бурмистров}

\tasknumber{1}%
\task{%
    Гидростатическое давление столба нефти равно $40\,\text{кПа}$.
    Определите высоту столба жидкости.
    Принять $\rho_{\text{н}} = 800\,\frac{\text{кг}}{\text{м}^{3}}$, $g = 10\,\frac{\text{м}}{\text{с}^{2}}$.
}
\answer{%
    $p = \rho_{\text{н}}gh \implies h = \frac{p}{g\rho_{\text{н}}} = \frac{{40\,\text{кПа}}}{10\,\frac{\text{м}}{\text{с}^{2}} \cdot 800\,\frac{\text{кг}}{\text{м}^{3}}} = 5\,\text{м}.$
}

\tasknumber{2}%
\task{%
    На какой глубине полное давление пресной воды превышает атмосферное в 7 раз?
    Принять $p_{\text{aтм}} = 100\,\text{кПа}$, $g = 10\,\frac{\text{м}}{\text{с}^{2}}$, $\rho_{\text{в}} = 1000\,\frac{\text{кг}}{\text{м}^{3}}$.
}
\answer{%
    $p = \rho_{\text{в}} g h + p_{\text{aтм}} = 7 p_{\text{aтм}} \implies h = \frac{(7-1) p_{\text{aтм}}}{g \rho_{\text{в}}} = \frac{(7-1) \cdot 100\,\text{кПа}}{10\,\frac{\text{м}}{\text{с}^{2}} \cdot 1000\,\frac{\text{кг}}{\text{м}^{3}}} = 60\,\text{м}.$
}

\tasknumber{3}%
\task{%
    В сосуд с вертикальными стенками и площадью горизонтального поперечного сечения $S = 0{,}05\,\text{м}^{2}$
    налили воду.
    На сколько увеличится давление на дно сосуда, если
    на поверхности воды ещё будет плавать тело массой $900\,\text{г}$.
    Принять $p_{\text{aтм}} = 100\,\text{кПа}$, $g = 10\,\frac{\text{м}}{\text{с}^{2}}$.
}
\answer{%
    $\Delta F = mg, \Delta p = \frac{\Delta F}{S} = \frac{mg}{S}, \Delta p = 180\,\text{Па}.$
}

\tasknumber{4}%
\task{%
    В два сообщающихся сосуда налита вода.
    В один из соcудов наливают масло так,
    что столб этой жидкости имеет высоту $50\,\text{см}$.
    На сколько теперь уровень воды в этом сосуде ниже, чем в другом?
    Ответ выразите в сантиметрах.
    $\rho_{\text{в}} = 1000\,\frac{\text{кг}}{\text{м}^{3}}$, $\rho_{\text{м}} = 900\,\frac{\text{кг}}{\text{м}^{3}}$.
}
\answer{%
    $\rho_{\text{м}}gh_1 = \rho_{\text{в}}gh_2 \implies h_2 = h_1 \frac{\rho_{\text{м}}}{\rho_{\text{в}}} = 50\,\text{см} \cdot \frac{900\,\frac{\text{кг}}{\text{м}^{3}}}{1000\,\frac{\text{кг}}{\text{м}^{3}}} = 45\,\text{см}$
}

\tasknumber{5}%
\task{%
    В два сообщающихся сосуда сечений $60\,\text{см}^{2}$ и $40\,\text{см}^{2}$ налита вода.
    Оба сосуда закрыты лёгкими поршнями и находятся в равновесии.
    На больший из поршней кладут груз массой $300\,\text{г}$.
    Определите, на сколько поднимется меньший поршень.
    Ответ выразите в сантиметрах.
    $\rho_{\text{в}} = 1000\,\frac{\text{кг}}{\text{м}^{3}}$.
}
\answer{%
    \begin{align*}
    S_1h_1 &= S_2h_2 \implies h_1 = h_2 \frac{S_2}{S_1}  \\
    \frac{mg}{S_1} &= \rho_{\text{в}}g(h_1 + h_2) = \rho_{\text{в}}g \cdot h_2 \cbr{1 + \frac{S_2}{S_1}} \\
    h_2 &= \frac{mg}{S_1\rho_{\text{в}}g} \cdot \frac{S_1}{S_1 + S_2} = \frac{m}{\rho_{\text{в}}(S_1 + S_2)} = 3\,\text{см}
    \end{align*}
}

\variantsplitter

\addpersonalvariant{Ирина Ан}

\tasknumber{1}%
\task{%
    Гидростатическое давление столба масла равно $27\,\text{кПа}$.
    Определите высоту столба жидкости.
    Принять $\rho_{\text{м}} = 900\,\frac{\text{кг}}{\text{м}^{3}}$, $g = 10\,\frac{\text{м}}{\text{с}^{2}}$.
}
\answer{%
    $p = \rho_{\text{м}}gh \implies h = \frac{p}{g\rho_{\text{м}}} = \frac{{27\,\text{кПа}}}{10\,\frac{\text{м}}{\text{с}^{2}} \cdot 900\,\frac{\text{кг}}{\text{м}^{3}}} = 3\,\text{м}.$
}

\tasknumber{2}%
\task{%
    На какой глубине полное давление пресной воды превышает атмосферное в 7 раз?
    Принять $p_{\text{aтм}} = 100\,\text{кПа}$, $g = 10\,\frac{\text{м}}{\text{с}^{2}}$, $\rho_{\text{в}} = 1000\,\frac{\text{кг}}{\text{м}^{3}}$.
}
\answer{%
    $p = \rho_{\text{в}} g h + p_{\text{aтм}} = 7 p_{\text{aтм}} \implies h = \frac{(7-1) p_{\text{aтм}}}{g \rho_{\text{в}}} = \frac{(7-1) \cdot 100\,\text{кПа}}{10\,\frac{\text{м}}{\text{с}^{2}} \cdot 1000\,\frac{\text{кг}}{\text{м}^{3}}} = 60\,\text{м}.$
}

\tasknumber{3}%
\task{%
    В сосуд с вертикальными стенками и площадью горизонтального поперечного сечения $S = 0{,}010\,\text{м}^{2}$
    налили воду.
    На сколько увеличится сила давления на дно сосуда, если
    на поверхности воды ещё будет плавать тело массой $300\,\text{г}$.
    Принять $p_{\text{aтм}} = 100\,\text{кПа}$, $g = 10\,\frac{\text{м}}{\text{с}^{2}}$.
}
\answer{%
    $\Delta F = mg, \Delta p = \frac{\Delta F}{S} = \frac{mg}{S}, \Delta F = 3\,\text{Н}.$
}

\tasknumber{4}%
\task{%
    В два сообщающихся сосуда налита вода.
    В один из соcудов наливают нефть так,
    что столб этой жидкости имеет высоту $10\,\text{см}$.
    На сколько теперь уровень воды в этом сосуде ниже, чем в другом?
    Ответ выразите в сантиметрах.
    $\rho_{\text{в}} = 1000\,\frac{\text{кг}}{\text{м}^{3}}$, $\rho_{\text{н}} = 800\,\frac{\text{кг}}{\text{м}^{3}}$.
}
\answer{%
    $\rho_{\text{н}}gh_1 = \rho_{\text{в}}gh_2 \implies h_2 = h_1 \frac{\rho_{\text{н}}}{\rho_{\text{в}}} = 10\,\text{см} \cdot \frac{800\,\frac{\text{кг}}{\text{м}^{3}}}{1000\,\frac{\text{кг}}{\text{м}^{3}}} = 8\,\text{см}$
}

\tasknumber{5}%
\task{%
    В два сообщающихся сосуда сечений $90\,\text{см}^{2}$ и $10\,\text{см}^{2}$ налита вода.
    Оба сосуда закрыты лёгкими поршнями и находятся в равновесии.
    На больший из поршней кладут груз массой $200\,\text{г}$.
    Определите, на сколько поднимется меньший поршень.
    Ответ выразите в сантиметрах.
    $\rho_{\text{в}} = 1000\,\frac{\text{кг}}{\text{м}^{3}}$.
}
\answer{%
    \begin{align*}
    S_1h_1 &= S_2h_2 \implies h_1 = h_2 \frac{S_2}{S_1}  \\
    \frac{mg}{S_1} &= \rho_{\text{в}}g(h_1 + h_2) = \rho_{\text{в}}g \cdot h_2 \cbr{1 + \frac{S_2}{S_1}} \\
    h_2 &= \frac{mg}{S_1\rho_{\text{в}}g} \cdot \frac{S_1}{S_1 + S_2} = \frac{m}{\rho_{\text{в}}(S_1 + S_2)} = 2\,\text{см}
    \end{align*}
}

\variantsplitter

\addpersonalvariant{Софья Андрианова}

\tasknumber{1}%
\task{%
    Гидростатическое давление столба нефти равно $20\,\text{кПа}$.
    Определите высоту столба жидкости.
    Принять $\rho_{\text{н}} = 800\,\frac{\text{кг}}{\text{м}^{3}}$, $g = 10\,\frac{\text{м}}{\text{с}^{2}}$.
}
\answer{%
    $p = \rho_{\text{н}}gh \implies h = \frac{p}{g\rho_{\text{н}}} = \frac{{20\,\text{кПа}}}{10\,\frac{\text{м}}{\text{с}^{2}} \cdot 800\,\frac{\text{кг}}{\text{м}^{3}}} = 2{,}5\,\text{м}.$
}

\tasknumber{2}%
\task{%
    На какой глубине полное давление пресной воды превышает атмосферное в 7 раз?
    Принять $p_{\text{aтм}} = 100\,\text{кПа}$, $g = 10\,\frac{\text{м}}{\text{с}^{2}}$, $\rho_{\text{в}} = 1000\,\frac{\text{кг}}{\text{м}^{3}}$.
}
\answer{%
    $p = \rho_{\text{в}} g h + p_{\text{aтм}} = 7 p_{\text{aтм}} \implies h = \frac{(7-1) p_{\text{aтм}}}{g \rho_{\text{в}}} = \frac{(7-1) \cdot 100\,\text{кПа}}{10\,\frac{\text{м}}{\text{с}^{2}} \cdot 1000\,\frac{\text{кг}}{\text{м}^{3}}} = 60\,\text{м}.$
}

\tasknumber{3}%
\task{%
    В сосуд с вертикальными стенками и площадью горизонтального поперечного сечения $S = 0{,}05\,\text{м}^{2}$
    налили воду.
    На сколько увеличится сила давления на дно сосуда, если
    на поверхности воды ещё будет плавать тело массой $900\,\text{г}$.
    Принять $p_{\text{aтм}} = 100\,\text{кПа}$, $g = 10\,\frac{\text{м}}{\text{с}^{2}}$.
}
\answer{%
    $\Delta F = mg, \Delta p = \frac{\Delta F}{S} = \frac{mg}{S}, \Delta F = 9\,\text{Н}.$
}

\tasknumber{4}%
\task{%
    В два сообщающихся сосуда налита вода.
    В один из соcудов наливают масло так,
    что столб этой жидкости имеет высоту $70\,\text{см}$.
    На сколько теперь уровень воды в этом сосуде ниже, чем в другом?
    Ответ выразите в сантиметрах.
    $\rho_{\text{в}} = 1000\,\frac{\text{кг}}{\text{м}^{3}}$, $\rho_{\text{м}} = 900\,\frac{\text{кг}}{\text{м}^{3}}$.
}
\answer{%
    $\rho_{\text{м}}gh_1 = \rho_{\text{в}}gh_2 \implies h_2 = h_1 \frac{\rho_{\text{м}}}{\rho_{\text{в}}} = 70\,\text{см} \cdot \frac{900\,\frac{\text{кг}}{\text{м}^{3}}}{1000\,\frac{\text{кг}}{\text{м}^{3}}} = 63\,\text{см}$
}

\tasknumber{5}%
\task{%
    В два сообщающихся сосуда сечений $80\,\text{см}^{2}$ и $20\,\text{см}^{2}$ налита вода.
    Оба сосуда закрыты лёгкими поршнями и находятся в равновесии.
    На больший из поршней кладут груз массой $400\,\text{г}$.
    Определите, на сколько поднимется меньший поршень.
    Ответ выразите в сантиметрах.
    $\rho_{\text{в}} = 1000\,\frac{\text{кг}}{\text{м}^{3}}$.
}
\answer{%
    \begin{align*}
    S_1h_1 &= S_2h_2 \implies h_1 = h_2 \frac{S_2}{S_1}  \\
    \frac{mg}{S_1} &= \rho_{\text{в}}g(h_1 + h_2) = \rho_{\text{в}}g \cdot h_2 \cbr{1 + \frac{S_2}{S_1}} \\
    h_2 &= \frac{mg}{S_1\rho_{\text{в}}g} \cdot \frac{S_1}{S_1 + S_2} = \frac{m}{\rho_{\text{в}}(S_1 + S_2)} = 4\,\text{см}
    \end{align*}
}

\variantsplitter

\addpersonalvariant{Владимир Артемчук}

\tasknumber{1}%
\task{%
    Гидростатическое давление столба воды равно $50\,\text{кПа}$.
    Определите высоту столба жидкости.
    Принять $\rho_{\text{в}} = 1000\,\frac{\text{кг}}{\text{м}^{3}}$, $g = 10\,\frac{\text{м}}{\text{с}^{2}}$.
}
\answer{%
    $p = \rho_{\text{в}}gh \implies h = \frac{p}{g\rho_{\text{в}}} = \frac{{50\,\text{кПа}}}{10\,\frac{\text{м}}{\text{с}^{2}} \cdot 1000\,\frac{\text{кг}}{\text{м}^{3}}} = 5\,\text{м}.$
}

\tasknumber{2}%
\task{%
    На какой глубине полное давление пресной воды превышает атмосферное в 9 раз?
    Принять $p_{\text{aтм}} = 100\,\text{кПа}$, $g = 10\,\frac{\text{м}}{\text{с}^{2}}$, $\rho_{\text{в}} = 1000\,\frac{\text{кг}}{\text{м}^{3}}$.
}
\answer{%
    $p = \rho_{\text{в}} g h + p_{\text{aтм}} = 9 p_{\text{aтм}} \implies h = \frac{(9-1) p_{\text{aтм}}}{g \rho_{\text{в}}} = \frac{(9-1) \cdot 100\,\text{кПа}}{10\,\frac{\text{м}}{\text{с}^{2}} \cdot 1000\,\frac{\text{кг}}{\text{м}^{3}}} = 80\,\text{м}.$
}

\tasknumber{3}%
\task{%
    В сосуд с вертикальными стенками и площадью горизонтального поперечного сечения $S = 0{,}05\,\text{м}^{2}$
    налили воду.
    На сколько увеличится давление на дно сосуда, если
    на поверхности воды ещё будет плавать тело массой $900\,\text{г}$.
    Принять $p_{\text{aтм}} = 100\,\text{кПа}$, $g = 10\,\frac{\text{м}}{\text{с}^{2}}$.
}
\answer{%
    $\Delta F = mg, \Delta p = \frac{\Delta F}{S} = \frac{mg}{S}, \Delta p = 180\,\text{Па}.$
}

\tasknumber{4}%
\task{%
    В два сообщающихся сосуда налита вода.
    В один из соcудов наливают масло так,
    что столб этой жидкости имеет высоту $20\,\text{см}$.
    На сколько теперь уровень воды в этом сосуде ниже, чем в другом?
    Ответ выразите в сантиметрах.
    $\rho_{\text{в}} = 1000\,\frac{\text{кг}}{\text{м}^{3}}$, $\rho_{\text{м}} = 900\,\frac{\text{кг}}{\text{м}^{3}}$.
}
\answer{%
    $\rho_{\text{м}}gh_1 = \rho_{\text{в}}gh_2 \implies h_2 = h_1 \frac{\rho_{\text{м}}}{\rho_{\text{в}}} = 20\,\text{см} \cdot \frac{900\,\frac{\text{кг}}{\text{м}^{3}}}{1000\,\frac{\text{кг}}{\text{м}^{3}}} = 18\,\text{см}$
}

\tasknumber{5}%
\task{%
    В два сообщающихся сосуда сечений $45\,\text{см}^{2}$ и $5\,\text{см}^{2}$ налита вода.
    Оба сосуда закрыты лёгкими поршнями и находятся в равновесии.
    На больший из поршней кладут груз массой $400\,\text{г}$.
    Определите, на сколько поднимется меньший поршень.
    Ответ выразите в сантиметрах.
    $\rho_{\text{в}} = 1000\,\frac{\text{кг}}{\text{м}^{3}}$.
}
\answer{%
    \begin{align*}
    S_1h_1 &= S_2h_2 \implies h_1 = h_2 \frac{S_2}{S_1}  \\
    \frac{mg}{S_1} &= \rho_{\text{в}}g(h_1 + h_2) = \rho_{\text{в}}g \cdot h_2 \cbr{1 + \frac{S_2}{S_1}} \\
    h_2 &= \frac{mg}{S_1\rho_{\text{в}}g} \cdot \frac{S_1}{S_1 + S_2} = \frac{m}{\rho_{\text{в}}(S_1 + S_2)} = 8\,\text{см}
    \end{align*}
}

\variantsplitter

\addpersonalvariant{Софья Белянкина}

\tasknumber{1}%
\task{%
    Гидростатическое давление столба нефти равно $400\,\text{кПа}$.
    Определите высоту столба жидкости.
    Принять $\rho_{\text{н}} = 800\,\frac{\text{кг}}{\text{м}^{3}}$, $g = 10\,\frac{\text{м}}{\text{с}^{2}}$.
}
\answer{%
    $p = \rho_{\text{н}}gh \implies h = \frac{p}{g\rho_{\text{н}}} = \frac{{400\,\text{кПа}}}{10\,\frac{\text{м}}{\text{с}^{2}} \cdot 800\,\frac{\text{кг}}{\text{м}^{3}}} = 50\,\text{м}.$
}

\tasknumber{2}%
\task{%
    На какой глубине полное давление пресной воды превышает атмосферное в 6 раз?
    Принять $p_{\text{aтм}} = 100\,\text{кПа}$, $g = 10\,\frac{\text{м}}{\text{с}^{2}}$, $\rho_{\text{в}} = 1000\,\frac{\text{кг}}{\text{м}^{3}}$.
}
\answer{%
    $p = \rho_{\text{в}} g h + p_{\text{aтм}} = 6 p_{\text{aтм}} \implies h = \frac{(6-1) p_{\text{aтм}}}{g \rho_{\text{в}}} = \frac{(6-1) \cdot 100\,\text{кПа}}{10\,\frac{\text{м}}{\text{с}^{2}} \cdot 1000\,\frac{\text{кг}}{\text{м}^{3}}} = 50\,\text{м}.$
}

\tasknumber{3}%
\task{%
    В сосуд с вертикальными стенками и площадью горизонтального поперечного сечения $S = 0{,}02\,\text{м}^{2}$
    налили воду.
    На сколько увеличится сила давления на дно сосуда, если
    на поверхности воды ещё будет плавать тело массой $600\,\text{г}$.
    Принять $p_{\text{aтм}} = 100\,\text{кПа}$, $g = 10\,\frac{\text{м}}{\text{с}^{2}}$.
}
\answer{%
    $\Delta F = mg, \Delta p = \frac{\Delta F}{S} = \frac{mg}{S}, \Delta F = 6\,\text{Н}.$
}

\tasknumber{4}%
\task{%
    В два сообщающихся сосуда налита вода.
    В один из соcудов наливают нефть так,
    что столб этой жидкости имеет высоту $20\,\text{см}$.
    На сколько теперь уровень воды в этом сосуде ниже, чем в другом?
    Ответ выразите в сантиметрах.
    $\rho_{\text{в}} = 1000\,\frac{\text{кг}}{\text{м}^{3}}$, $\rho_{\text{н}} = 800\,\frac{\text{кг}}{\text{м}^{3}}$.
}
\answer{%
    $\rho_{\text{н}}gh_1 = \rho_{\text{в}}gh_2 \implies h_2 = h_1 \frac{\rho_{\text{н}}}{\rho_{\text{в}}} = 20\,\text{см} \cdot \frac{800\,\frac{\text{кг}}{\text{м}^{3}}}{1000\,\frac{\text{кг}}{\text{м}^{3}}} = 16\,\text{см}$
}

\tasknumber{5}%
\task{%
    В два сообщающихся сосуда сечений $90\,\text{см}^{2}$ и $10\,\text{см}^{2}$ налита вода.
    Оба сосуда закрыты лёгкими поршнями и находятся в равновесии.
    На больший из поршней кладут груз массой $300\,\text{г}$.
    Определите, на сколько поднимется меньший поршень.
    Ответ выразите в сантиметрах.
    $\rho_{\text{в}} = 1000\,\frac{\text{кг}}{\text{м}^{3}}$.
}
\answer{%
    \begin{align*}
    S_1h_1 &= S_2h_2 \implies h_1 = h_2 \frac{S_2}{S_1}  \\
    \frac{mg}{S_1} &= \rho_{\text{в}}g(h_1 + h_2) = \rho_{\text{в}}g \cdot h_2 \cbr{1 + \frac{S_2}{S_1}} \\
    h_2 &= \frac{mg}{S_1\rho_{\text{в}}g} \cdot \frac{S_1}{S_1 + S_2} = \frac{m}{\rho_{\text{в}}(S_1 + S_2)} = 3\,\text{см}
    \end{align*}
}

\variantsplitter

\addpersonalvariant{Варвара Егиазарян}

\tasknumber{1}%
\task{%
    Гидростатическое давление столба нефти равно $20\,\text{кПа}$.
    Определите высоту столба жидкости.
    Принять $\rho_{\text{н}} = 800\,\frac{\text{кг}}{\text{м}^{3}}$, $g = 10\,\frac{\text{м}}{\text{с}^{2}}$.
}
\answer{%
    $p = \rho_{\text{н}}gh \implies h = \frac{p}{g\rho_{\text{н}}} = \frac{{20\,\text{кПа}}}{10\,\frac{\text{м}}{\text{с}^{2}} \cdot 800\,\frac{\text{кг}}{\text{м}^{3}}} = 2{,}5\,\text{м}.$
}

\tasknumber{2}%
\task{%
    На какой глубине полное давление пресной воды превышает атмосферное в 9 раз?
    Принять $p_{\text{aтм}} = 100\,\text{кПа}$, $g = 10\,\frac{\text{м}}{\text{с}^{2}}$, $\rho_{\text{в}} = 1000\,\frac{\text{кг}}{\text{м}^{3}}$.
}
\answer{%
    $p = \rho_{\text{в}} g h + p_{\text{aтм}} = 9 p_{\text{aтм}} \implies h = \frac{(9-1) p_{\text{aтм}}}{g \rho_{\text{в}}} = \frac{(9-1) \cdot 100\,\text{кПа}}{10\,\frac{\text{м}}{\text{с}^{2}} \cdot 1000\,\frac{\text{кг}}{\text{м}^{3}}} = 80\,\text{м}.$
}

\tasknumber{3}%
\task{%
    В сосуд с вертикальными стенками и площадью горизонтального поперечного сечения $S = 0{,}010\,\text{м}^{2}$
    налили воду.
    На сколько увеличится сила давления на дно сосуда, если
    на поверхности воды ещё будет плавать тело массой $900\,\text{г}$.
    Принять $p_{\text{aтм}} = 100\,\text{кПа}$, $g = 10\,\frac{\text{м}}{\text{с}^{2}}$.
}
\answer{%
    $\Delta F = mg, \Delta p = \frac{\Delta F}{S} = \frac{mg}{S}, \Delta F = 9\,\text{Н}.$
}

\tasknumber{4}%
\task{%
    В два сообщающихся сосуда налита вода.
    В один из соcудов наливают масло так,
    что столб этой жидкости имеет высоту $40\,\text{см}$.
    На сколько теперь уровень воды в этом сосуде ниже, чем в другом?
    Ответ выразите в сантиметрах.
    $\rho_{\text{в}} = 1000\,\frac{\text{кг}}{\text{м}^{3}}$, $\rho_{\text{м}} = 900\,\frac{\text{кг}}{\text{м}^{3}}$.
}
\answer{%
    $\rho_{\text{м}}gh_1 = \rho_{\text{в}}gh_2 \implies h_2 = h_1 \frac{\rho_{\text{м}}}{\rho_{\text{в}}} = 40\,\text{см} \cdot \frac{900\,\frac{\text{кг}}{\text{м}^{3}}}{1000\,\frac{\text{кг}}{\text{м}^{3}}} = 36\,\text{см}$
}

\tasknumber{5}%
\task{%
    В два сообщающихся сосуда сечений $64\,\text{см}^{2}$ и $36\,\text{см}^{2}$ налита вода.
    Оба сосуда закрыты лёгкими поршнями и находятся в равновесии.
    На больший из поршней кладут груз массой $400\,\text{г}$.
    Определите, на сколько поднимется меньший поршень.
    Ответ выразите в сантиметрах.
    $\rho_{\text{в}} = 1000\,\frac{\text{кг}}{\text{м}^{3}}$.
}
\answer{%
    \begin{align*}
    S_1h_1 &= S_2h_2 \implies h_1 = h_2 \frac{S_2}{S_1}  \\
    \frac{mg}{S_1} &= \rho_{\text{в}}g(h_1 + h_2) = \rho_{\text{в}}g \cdot h_2 \cbr{1 + \frac{S_2}{S_1}} \\
    h_2 &= \frac{mg}{S_1\rho_{\text{в}}g} \cdot \frac{S_1}{S_1 + S_2} = \frac{m}{\rho_{\text{в}}(S_1 + S_2)} = 4\,\text{см}
    \end{align*}
}

\variantsplitter

\addpersonalvariant{Владислав Емелин}

\tasknumber{1}%
\task{%
    Гидростатическое давление столба нефти равно $40\,\text{кПа}$.
    Определите высоту столба жидкости.
    Принять $\rho_{\text{н}} = 800\,\frac{\text{кг}}{\text{м}^{3}}$, $g = 10\,\frac{\text{м}}{\text{с}^{2}}$.
}
\answer{%
    $p = \rho_{\text{н}}gh \implies h = \frac{p}{g\rho_{\text{н}}} = \frac{{40\,\text{кПа}}}{10\,\frac{\text{м}}{\text{с}^{2}} \cdot 800\,\frac{\text{кг}}{\text{м}^{3}}} = 5\,\text{м}.$
}

\tasknumber{2}%
\task{%
    На какой глубине полное давление пресной воды превышает атмосферное в 9 раз?
    Принять $p_{\text{aтм}} = 100\,\text{кПа}$, $g = 10\,\frac{\text{м}}{\text{с}^{2}}$, $\rho_{\text{в}} = 1000\,\frac{\text{кг}}{\text{м}^{3}}$.
}
\answer{%
    $p = \rho_{\text{в}} g h + p_{\text{aтм}} = 9 p_{\text{aтм}} \implies h = \frac{(9-1) p_{\text{aтм}}}{g \rho_{\text{в}}} = \frac{(9-1) \cdot 100\,\text{кПа}}{10\,\frac{\text{м}}{\text{с}^{2}} \cdot 1000\,\frac{\text{кг}}{\text{м}^{3}}} = 80\,\text{м}.$
}

\tasknumber{3}%
\task{%
    В сосуд с вертикальными стенками и площадью горизонтального поперечного сечения $S = 0{,}05\,\text{м}^{2}$
    налили воду.
    На сколько увеличится давление на дно сосуда, если
    на поверхности воды ещё будет плавать тело массой $300\,\text{г}$.
    Принять $p_{\text{aтм}} = 100\,\text{кПа}$, $g = 10\,\frac{\text{м}}{\text{с}^{2}}$.
}
\answer{%
    $\Delta F = mg, \Delta p = \frac{\Delta F}{S} = \frac{mg}{S}, \Delta p = 60\,\text{Па}.$
}

\tasknumber{4}%
\task{%
    В два сообщающихся сосуда налита вода.
    В один из соcудов наливают масло так,
    что столб этой жидкости имеет высоту $70\,\text{см}$.
    На сколько теперь уровень воды в этом сосуде ниже, чем в другом?
    Ответ выразите в сантиметрах.
    $\rho_{\text{в}} = 1000\,\frac{\text{кг}}{\text{м}^{3}}$, $\rho_{\text{м}} = 900\,\frac{\text{кг}}{\text{м}^{3}}$.
}
\answer{%
    $\rho_{\text{м}}gh_1 = \rho_{\text{в}}gh_2 \implies h_2 = h_1 \frac{\rho_{\text{м}}}{\rho_{\text{в}}} = 70\,\text{см} \cdot \frac{900\,\frac{\text{кг}}{\text{м}^{3}}}{1000\,\frac{\text{кг}}{\text{м}^{3}}} = 63\,\text{см}$
}

\tasknumber{5}%
\task{%
    В два сообщающихся сосуда сечений $32\,\text{см}^{2}$ и $18\,\text{см}^{2}$ налита вода.
    Оба сосуда закрыты лёгкими поршнями и находятся в равновесии.
    На больший из поршней кладут груз массой $200\,\text{г}$.
    Определите, на сколько поднимется меньший поршень.
    Ответ выразите в сантиметрах.
    $\rho_{\text{в}} = 1000\,\frac{\text{кг}}{\text{м}^{3}}$.
}
\answer{%
    \begin{align*}
    S_1h_1 &= S_2h_2 \implies h_1 = h_2 \frac{S_2}{S_1}  \\
    \frac{mg}{S_1} &= \rho_{\text{в}}g(h_1 + h_2) = \rho_{\text{в}}g \cdot h_2 \cbr{1 + \frac{S_2}{S_1}} \\
    h_2 &= \frac{mg}{S_1\rho_{\text{в}}g} \cdot \frac{S_1}{S_1 + S_2} = \frac{m}{\rho_{\text{в}}(S_1 + S_2)} = 4\,\text{см}
    \end{align*}
}

\variantsplitter

\addpersonalvariant{Артём Жичин}

\tasknumber{1}%
\task{%
    Гидростатическое давление столба нефти равно $40\,\text{кПа}$.
    Определите высоту столба жидкости.
    Принять $\rho_{\text{н}} = 800\,\frac{\text{кг}}{\text{м}^{3}}$, $g = 10\,\frac{\text{м}}{\text{с}^{2}}$.
}
\answer{%
    $p = \rho_{\text{н}}gh \implies h = \frac{p}{g\rho_{\text{н}}} = \frac{{40\,\text{кПа}}}{10\,\frac{\text{м}}{\text{с}^{2}} \cdot 800\,\frac{\text{кг}}{\text{м}^{3}}} = 5\,\text{м}.$
}

\tasknumber{2}%
\task{%
    На какой глубине полное давление пресной воды превышает атмосферное в 4 раз?
    Принять $p_{\text{aтм}} = 100\,\text{кПа}$, $g = 10\,\frac{\text{м}}{\text{с}^{2}}$, $\rho_{\text{в}} = 1000\,\frac{\text{кг}}{\text{м}^{3}}$.
}
\answer{%
    $p = \rho_{\text{в}} g h + p_{\text{aтм}} = 4 p_{\text{aтм}} \implies h = \frac{(4-1) p_{\text{aтм}}}{g \rho_{\text{в}}} = \frac{(4-1) \cdot 100\,\text{кПа}}{10\,\frac{\text{м}}{\text{с}^{2}} \cdot 1000\,\frac{\text{кг}}{\text{м}^{3}}} = 30\,\text{м}.$
}

\tasknumber{3}%
\task{%
    В сосуд с вертикальными стенками и площадью горизонтального поперечного сечения $S = 0{,}010\,\text{м}^{2}$
    налили воду.
    На сколько увеличится давление на дно сосуда, если
    на поверхности воды ещё будет плавать тело массой $900\,\text{г}$.
    Принять $p_{\text{aтм}} = 100\,\text{кПа}$, $g = 10\,\frac{\text{м}}{\text{с}^{2}}$.
}
\answer{%
    $\Delta F = mg, \Delta p = \frac{\Delta F}{S} = \frac{mg}{S}, \Delta p = 900\,\text{Па}.$
}

\tasknumber{4}%
\task{%
    В два сообщающихся сосуда налита вода.
    В один из соcудов наливают нефть так,
    что столб этой жидкости имеет высоту $20\,\text{см}$.
    На сколько теперь уровень воды в этом сосуде ниже, чем в другом?
    Ответ выразите в сантиметрах.
    $\rho_{\text{в}} = 1000\,\frac{\text{кг}}{\text{м}^{3}}$, $\rho_{\text{н}} = 800\,\frac{\text{кг}}{\text{м}^{3}}$.
}
\answer{%
    $\rho_{\text{н}}gh_1 = \rho_{\text{в}}gh_2 \implies h_2 = h_1 \frac{\rho_{\text{н}}}{\rho_{\text{в}}} = 20\,\text{см} \cdot \frac{800\,\frac{\text{кг}}{\text{м}^{3}}}{1000\,\frac{\text{кг}}{\text{м}^{3}}} = 16\,\text{см}$
}

\tasknumber{5}%
\task{%
    В два сообщающихся сосуда сечений $64\,\text{см}^{2}$ и $36\,\text{см}^{2}$ налита вода.
    Оба сосуда закрыты лёгкими поршнями и находятся в равновесии.
    На больший из поршней кладут груз массой $300\,\text{г}$.
    Определите, на сколько поднимется меньший поршень.
    Ответ выразите в сантиметрах.
    $\rho_{\text{в}} = 1000\,\frac{\text{кг}}{\text{м}^{3}}$.
}
\answer{%
    \begin{align*}
    S_1h_1 &= S_2h_2 \implies h_1 = h_2 \frac{S_2}{S_1}  \\
    \frac{mg}{S_1} &= \rho_{\text{в}}g(h_1 + h_2) = \rho_{\text{в}}g \cdot h_2 \cbr{1 + \frac{S_2}{S_1}} \\
    h_2 &= \frac{mg}{S_1\rho_{\text{в}}g} \cdot \frac{S_1}{S_1 + S_2} = \frac{m}{\rho_{\text{в}}(S_1 + S_2)} = 3\,\text{см}
    \end{align*}
}

\variantsplitter

\addpersonalvariant{Дарья Кошман}

\tasknumber{1}%
\task{%
    Гидростатическое давление столба нефти равно $40\,\text{кПа}$.
    Определите высоту столба жидкости.
    Принять $\rho_{\text{н}} = 800\,\frac{\text{кг}}{\text{м}^{3}}$, $g = 10\,\frac{\text{м}}{\text{с}^{2}}$.
}
\answer{%
    $p = \rho_{\text{н}}gh \implies h = \frac{p}{g\rho_{\text{н}}} = \frac{{40\,\text{кПа}}}{10\,\frac{\text{м}}{\text{с}^{2}} \cdot 800\,\frac{\text{кг}}{\text{м}^{3}}} = 5\,\text{м}.$
}

\tasknumber{2}%
\task{%
    На какой глубине полное давление пресной воды превышает атмосферное в 5 раз?
    Принять $p_{\text{aтм}} = 100\,\text{кПа}$, $g = 10\,\frac{\text{м}}{\text{с}^{2}}$, $\rho_{\text{в}} = 1000\,\frac{\text{кг}}{\text{м}^{3}}$.
}
\answer{%
    $p = \rho_{\text{в}} g h + p_{\text{aтм}} = 5 p_{\text{aтм}} \implies h = \frac{(5-1) p_{\text{aтм}}}{g \rho_{\text{в}}} = \frac{(5-1) \cdot 100\,\text{кПа}}{10\,\frac{\text{м}}{\text{с}^{2}} \cdot 1000\,\frac{\text{кг}}{\text{м}^{3}}} = 40\,\text{м}.$
}

\tasknumber{3}%
\task{%
    В сосуд с вертикальными стенками и площадью горизонтального поперечного сечения $S = 0{,}010\,\text{м}^{2}$
    налили воду.
    На сколько увеличится сила давления на дно сосуда, если
    на поверхности воды ещё будет плавать тело массой $900\,\text{г}$.
    Принять $p_{\text{aтм}} = 100\,\text{кПа}$, $g = 10\,\frac{\text{м}}{\text{с}^{2}}$.
}
\answer{%
    $\Delta F = mg, \Delta p = \frac{\Delta F}{S} = \frac{mg}{S}, \Delta F = 9\,\text{Н}.$
}

\tasknumber{4}%
\task{%
    В два сообщающихся сосуда налита вода.
    В один из соcудов наливают нефть так,
    что столб этой жидкости имеет высоту $20\,\text{см}$.
    На сколько теперь уровень воды в этом сосуде ниже, чем в другом?
    Ответ выразите в сантиметрах.
    $\rho_{\text{в}} = 1000\,\frac{\text{кг}}{\text{м}^{3}}$, $\rho_{\text{н}} = 800\,\frac{\text{кг}}{\text{м}^{3}}$.
}
\answer{%
    $\rho_{\text{н}}gh_1 = \rho_{\text{в}}gh_2 \implies h_2 = h_1 \frac{\rho_{\text{н}}}{\rho_{\text{в}}} = 20\,\text{см} \cdot \frac{800\,\frac{\text{кг}}{\text{м}^{3}}}{1000\,\frac{\text{кг}}{\text{м}^{3}}} = 16\,\text{см}$
}

\tasknumber{5}%
\task{%
    В два сообщающихся сосуда сечений $90\,\text{см}^{2}$ и $10\,\text{см}^{2}$ налита вода.
    Оба сосуда закрыты лёгкими поршнями и находятся в равновесии.
    На больший из поршней кладут груз массой $300\,\text{г}$.
    Определите, на сколько поднимется меньший поршень.
    Ответ выразите в сантиметрах.
    $\rho_{\text{в}} = 1000\,\frac{\text{кг}}{\text{м}^{3}}$.
}
\answer{%
    \begin{align*}
    S_1h_1 &= S_2h_2 \implies h_1 = h_2 \frac{S_2}{S_1}  \\
    \frac{mg}{S_1} &= \rho_{\text{в}}g(h_1 + h_2) = \rho_{\text{в}}g \cdot h_2 \cbr{1 + \frac{S_2}{S_1}} \\
    h_2 &= \frac{mg}{S_1\rho_{\text{в}}g} \cdot \frac{S_1}{S_1 + S_2} = \frac{m}{\rho_{\text{в}}(S_1 + S_2)} = 3\,\text{см}
    \end{align*}
}

\variantsplitter

\addpersonalvariant{Анна Кузьмичёва}

\tasknumber{1}%
\task{%
    Гидростатическое давление столба нефти равно $40\,\text{кПа}$.
    Определите высоту столба жидкости.
    Принять $\rho_{\text{н}} = 800\,\frac{\text{кг}}{\text{м}^{3}}$, $g = 10\,\frac{\text{м}}{\text{с}^{2}}$.
}
\answer{%
    $p = \rho_{\text{н}}gh \implies h = \frac{p}{g\rho_{\text{н}}} = \frac{{40\,\text{кПа}}}{10\,\frac{\text{м}}{\text{с}^{2}} \cdot 800\,\frac{\text{кг}}{\text{м}^{3}}} = 5\,\text{м}.$
}

\tasknumber{2}%
\task{%
    На какой глубине полное давление пресной воды превышает атмосферное в 3 раз?
    Принять $p_{\text{aтм}} = 100\,\text{кПа}$, $g = 10\,\frac{\text{м}}{\text{с}^{2}}$, $\rho_{\text{в}} = 1000\,\frac{\text{кг}}{\text{м}^{3}}$.
}
\answer{%
    $p = \rho_{\text{в}} g h + p_{\text{aтм}} = 3 p_{\text{aтм}} \implies h = \frac{(3-1) p_{\text{aтм}}}{g \rho_{\text{в}}} = \frac{(3-1) \cdot 100\,\text{кПа}}{10\,\frac{\text{м}}{\text{с}^{2}} \cdot 1000\,\frac{\text{кг}}{\text{м}^{3}}} = 20\,\text{м}.$
}

\tasknumber{3}%
\task{%
    В сосуд с вертикальными стенками и площадью горизонтального поперечного сечения $S = 0{,}02\,\text{м}^{2}$
    налили воду.
    На сколько увеличится давление на дно сосуда, если
    на поверхности воды ещё будет плавать тело массой $600\,\text{г}$.
    Принять $p_{\text{aтм}} = 100\,\text{кПа}$, $g = 10\,\frac{\text{м}}{\text{с}^{2}}$.
}
\answer{%
    $\Delta F = mg, \Delta p = \frac{\Delta F}{S} = \frac{mg}{S}, \Delta p = 300\,\text{Па}.$
}

\tasknumber{4}%
\task{%
    В два сообщающихся сосуда налита вода.
    В один из соcудов наливают нефть так,
    что столб этой жидкости имеет высоту $40\,\text{см}$.
    На сколько теперь уровень воды в этом сосуде ниже, чем в другом?
    Ответ выразите в сантиметрах.
    $\rho_{\text{в}} = 1000\,\frac{\text{кг}}{\text{м}^{3}}$, $\rho_{\text{н}} = 800\,\frac{\text{кг}}{\text{м}^{3}}$.
}
\answer{%
    $\rho_{\text{н}}gh_1 = \rho_{\text{в}}gh_2 \implies h_2 = h_1 \frac{\rho_{\text{н}}}{\rho_{\text{в}}} = 40\,\text{см} \cdot \frac{800\,\frac{\text{кг}}{\text{м}^{3}}}{1000\,\frac{\text{кг}}{\text{м}^{3}}} = 32\,\text{см}$
}

\tasknumber{5}%
\task{%
    В два сообщающихся сосуда сечений $40\,\text{см}^{2}$ и $10\,\text{см}^{2}$ налита вода.
    Оба сосуда закрыты лёгкими поршнями и находятся в равновесии.
    На больший из поршней кладут груз массой $400\,\text{г}$.
    Определите, на сколько поднимется меньший поршень.
    Ответ выразите в сантиметрах.
    $\rho_{\text{в}} = 1000\,\frac{\text{кг}}{\text{м}^{3}}$.
}
\answer{%
    \begin{align*}
    S_1h_1 &= S_2h_2 \implies h_1 = h_2 \frac{S_2}{S_1}  \\
    \frac{mg}{S_1} &= \rho_{\text{в}}g(h_1 + h_2) = \rho_{\text{в}}g \cdot h_2 \cbr{1 + \frac{S_2}{S_1}} \\
    h_2 &= \frac{mg}{S_1\rho_{\text{в}}g} \cdot \frac{S_1}{S_1 + S_2} = \frac{m}{\rho_{\text{в}}(S_1 + S_2)} = 8\,\text{см}
    \end{align*}
}

\variantsplitter

\addpersonalvariant{Алёна Куприянова}

\tasknumber{1}%
\task{%
    Гидростатическое давление столба масла равно $18\,\text{кПа}$.
    Определите высоту столба жидкости.
    Принять $\rho_{\text{м}} = 900\,\frac{\text{кг}}{\text{м}^{3}}$, $g = 10\,\frac{\text{м}}{\text{с}^{2}}$.
}
\answer{%
    $p = \rho_{\text{м}}gh \implies h = \frac{p}{g\rho_{\text{м}}} = \frac{{18\,\text{кПа}}}{10\,\frac{\text{м}}{\text{с}^{2}} \cdot 900\,\frac{\text{кг}}{\text{м}^{3}}} = 2\,\text{м}.$
}

\tasknumber{2}%
\task{%
    На какой глубине полное давление пресной воды превышает атмосферное в 9 раз?
    Принять $p_{\text{aтм}} = 100\,\text{кПа}$, $g = 10\,\frac{\text{м}}{\text{с}^{2}}$, $\rho_{\text{в}} = 1000\,\frac{\text{кг}}{\text{м}^{3}}$.
}
\answer{%
    $p = \rho_{\text{в}} g h + p_{\text{aтм}} = 9 p_{\text{aтм}} \implies h = \frac{(9-1) p_{\text{aтм}}}{g \rho_{\text{в}}} = \frac{(9-1) \cdot 100\,\text{кПа}}{10\,\frac{\text{м}}{\text{с}^{2}} \cdot 1000\,\frac{\text{кг}}{\text{м}^{3}}} = 80\,\text{м}.$
}

\tasknumber{3}%
\task{%
    В сосуд с вертикальными стенками и площадью горизонтального поперечного сечения $S = 0{,}05\,\text{м}^{2}$
    налили воду.
    На сколько увеличится сила давления на дно сосуда, если
    на поверхности воды ещё будет плавать тело массой $150\,\text{г}$.
    Принять $p_{\text{aтм}} = 100\,\text{кПа}$, $g = 10\,\frac{\text{м}}{\text{с}^{2}}$.
}
\answer{%
    $\Delta F = mg, \Delta p = \frac{\Delta F}{S} = \frac{mg}{S}, \Delta F = 1\,\text{Н}.$
}

\tasknumber{4}%
\task{%
    В два сообщающихся сосуда налита вода.
    В один из соcудов наливают нефть так,
    что столб этой жидкости имеет высоту $25\,\text{см}$.
    На сколько теперь уровень воды в этом сосуде ниже, чем в другом?
    Ответ выразите в сантиметрах.
    $\rho_{\text{в}} = 1000\,\frac{\text{кг}}{\text{м}^{3}}$, $\rho_{\text{н}} = 800\,\frac{\text{кг}}{\text{м}^{3}}$.
}
\answer{%
    $\rho_{\text{н}}gh_1 = \rho_{\text{в}}gh_2 \implies h_2 = h_1 \frac{\rho_{\text{н}}}{\rho_{\text{в}}} = 25\,\text{см} \cdot \frac{800\,\frac{\text{кг}}{\text{м}^{3}}}{1000\,\frac{\text{кг}}{\text{м}^{3}}} = 20\,\text{см}$
}

\tasknumber{5}%
\task{%
    В два сообщающихся сосуда сечений $80\,\text{см}^{2}$ и $20\,\text{см}^{2}$ налита вода.
    Оба сосуда закрыты лёгкими поршнями и находятся в равновесии.
    На больший из поршней кладут груз массой $500\,\text{г}$.
    Определите, на сколько поднимется меньший поршень.
    Ответ выразите в сантиметрах.
    $\rho_{\text{в}} = 1000\,\frac{\text{кг}}{\text{м}^{3}}$.
}
\answer{%
    \begin{align*}
    S_1h_1 &= S_2h_2 \implies h_1 = h_2 \frac{S_2}{S_1}  \\
    \frac{mg}{S_1} &= \rho_{\text{в}}g(h_1 + h_2) = \rho_{\text{в}}g \cdot h_2 \cbr{1 + \frac{S_2}{S_1}} \\
    h_2 &= \frac{mg}{S_1\rho_{\text{в}}g} \cdot \frac{S_1}{S_1 + S_2} = \frac{m}{\rho_{\text{в}}(S_1 + S_2)} = 5\,\text{см}
    \end{align*}
}

\variantsplitter

\addpersonalvariant{Ярослав Лавровский}

\tasknumber{1}%
\task{%
    Гидростатическое давление столба воды равно $150\,\text{кПа}$.
    Определите высоту столба жидкости.
    Принять $\rho_{\text{в}} = 1000\,\frac{\text{кг}}{\text{м}^{3}}$, $g = 10\,\frac{\text{м}}{\text{с}^{2}}$.
}
\answer{%
    $p = \rho_{\text{в}}gh \implies h = \frac{p}{g\rho_{\text{в}}} = \frac{{150\,\text{кПа}}}{10\,\frac{\text{м}}{\text{с}^{2}} \cdot 1000\,\frac{\text{кг}}{\text{м}^{3}}} = 15\,\text{м}.$
}

\tasknumber{2}%
\task{%
    На какой глубине полное давление пресной воды превышает атмосферное в 9 раз?
    Принять $p_{\text{aтм}} = 100\,\text{кПа}$, $g = 10\,\frac{\text{м}}{\text{с}^{2}}$, $\rho_{\text{в}} = 1000\,\frac{\text{кг}}{\text{м}^{3}}$.
}
\answer{%
    $p = \rho_{\text{в}} g h + p_{\text{aтм}} = 9 p_{\text{aтм}} \implies h = \frac{(9-1) p_{\text{aтм}}}{g \rho_{\text{в}}} = \frac{(9-1) \cdot 100\,\text{кПа}}{10\,\frac{\text{м}}{\text{с}^{2}} \cdot 1000\,\frac{\text{кг}}{\text{м}^{3}}} = 80\,\text{м}.$
}

\tasknumber{3}%
\task{%
    В сосуд с вертикальными стенками и площадью горизонтального поперечного сечения $S = 0{,}010\,\text{м}^{2}$
    налили воду.
    На сколько увеличится сила давления на дно сосуда, если
    на поверхности воды ещё будет плавать тело массой $300\,\text{г}$.
    Принять $p_{\text{aтм}} = 100\,\text{кПа}$, $g = 10\,\frac{\text{м}}{\text{с}^{2}}$.
}
\answer{%
    $\Delta F = mg, \Delta p = \frac{\Delta F}{S} = \frac{mg}{S}, \Delta F = 3\,\text{Н}.$
}

\tasknumber{4}%
\task{%
    В два сообщающихся сосуда налита вода.
    В один из соcудов наливают нефть так,
    что столб этой жидкости имеет высоту $5\,\text{см}$.
    На сколько теперь уровень воды в этом сосуде ниже, чем в другом?
    Ответ выразите в сантиметрах.
    $\rho_{\text{в}} = 1000\,\frac{\text{кг}}{\text{м}^{3}}$, $\rho_{\text{н}} = 800\,\frac{\text{кг}}{\text{м}^{3}}$.
}
\answer{%
    $\rho_{\text{н}}gh_1 = \rho_{\text{в}}gh_2 \implies h_2 = h_1 \frac{\rho_{\text{н}}}{\rho_{\text{в}}} = 5\,\text{см} \cdot \frac{800\,\frac{\text{кг}}{\text{м}^{3}}}{1000\,\frac{\text{кг}}{\text{м}^{3}}} = 4\,\text{см}$
}

\tasknumber{5}%
\task{%
    В два сообщающихся сосуда сечений $30\,\text{см}^{2}$ и $20\,\text{см}^{2}$ налита вода.
    Оба сосуда закрыты лёгкими поршнями и находятся в равновесии.
    На больший из поршней кладут груз массой $400\,\text{г}$.
    Определите, на сколько поднимется меньший поршень.
    Ответ выразите в сантиметрах.
    $\rho_{\text{в}} = 1000\,\frac{\text{кг}}{\text{м}^{3}}$.
}
\answer{%
    \begin{align*}
    S_1h_1 &= S_2h_2 \implies h_1 = h_2 \frac{S_2}{S_1}  \\
    \frac{mg}{S_1} &= \rho_{\text{в}}g(h_1 + h_2) = \rho_{\text{в}}g \cdot h_2 \cbr{1 + \frac{S_2}{S_1}} \\
    h_2 &= \frac{mg}{S_1\rho_{\text{в}}g} \cdot \frac{S_1}{S_1 + S_2} = \frac{m}{\rho_{\text{в}}(S_1 + S_2)} = 8\,\text{см}
    \end{align*}
}

\variantsplitter

\addpersonalvariant{Анастасия Ламанова}

\tasknumber{1}%
\task{%
    Гидростатическое давление столба воды равно $100\,\text{кПа}$.
    Определите высоту столба жидкости.
    Принять $\rho_{\text{в}} = 1000\,\frac{\text{кг}}{\text{м}^{3}}$, $g = 10\,\frac{\text{м}}{\text{с}^{2}}$.
}
\answer{%
    $p = \rho_{\text{в}}gh \implies h = \frac{p}{g\rho_{\text{в}}} = \frac{{100\,\text{кПа}}}{10\,\frac{\text{м}}{\text{с}^{2}} \cdot 1000\,\frac{\text{кг}}{\text{м}^{3}}} = 10\,\text{м}.$
}

\tasknumber{2}%
\task{%
    На какой глубине полное давление пресной воды превышает атмосферное в 6 раз?
    Принять $p_{\text{aтм}} = 100\,\text{кПа}$, $g = 10\,\frac{\text{м}}{\text{с}^{2}}$, $\rho_{\text{в}} = 1000\,\frac{\text{кг}}{\text{м}^{3}}$.
}
\answer{%
    $p = \rho_{\text{в}} g h + p_{\text{aтм}} = 6 p_{\text{aтм}} \implies h = \frac{(6-1) p_{\text{aтм}}}{g \rho_{\text{в}}} = \frac{(6-1) \cdot 100\,\text{кПа}}{10\,\frac{\text{м}}{\text{с}^{2}} \cdot 1000\,\frac{\text{кг}}{\text{м}^{3}}} = 50\,\text{м}.$
}

\tasknumber{3}%
\task{%
    В сосуд с вертикальными стенками и площадью горизонтального поперечного сечения $S = 0{,}010\,\text{м}^{2}$
    налили воду.
    На сколько увеличится сила давления на дно сосуда, если
    на поверхности воды ещё будет плавать тело массой $600\,\text{г}$.
    Принять $p_{\text{aтм}} = 100\,\text{кПа}$, $g = 10\,\frac{\text{м}}{\text{с}^{2}}$.
}
\answer{%
    $\Delta F = mg, \Delta p = \frac{\Delta F}{S} = \frac{mg}{S}, \Delta F = 6\,\text{Н}.$
}

\tasknumber{4}%
\task{%
    В два сообщающихся сосуда налита вода.
    В один из соcудов наливают масло так,
    что столб этой жидкости имеет высоту $30\,\text{см}$.
    На сколько теперь уровень воды в этом сосуде ниже, чем в другом?
    Ответ выразите в сантиметрах.
    $\rho_{\text{в}} = 1000\,\frac{\text{кг}}{\text{м}^{3}}$, $\rho_{\text{м}} = 900\,\frac{\text{кг}}{\text{м}^{3}}$.
}
\answer{%
    $\rho_{\text{м}}gh_1 = \rho_{\text{в}}gh_2 \implies h_2 = h_1 \frac{\rho_{\text{м}}}{\rho_{\text{в}}} = 30\,\text{см} \cdot \frac{900\,\frac{\text{кг}}{\text{м}^{3}}}{1000\,\frac{\text{кг}}{\text{м}^{3}}} = 27\,\text{см}$
}

\tasknumber{5}%
\task{%
    В два сообщающихся сосуда сечений $40\,\text{см}^{2}$ и $10\,\text{см}^{2}$ налита вода.
    Оба сосуда закрыты лёгкими поршнями и находятся в равновесии.
    На больший из поршней кладут груз массой $300\,\text{г}$.
    Определите, на сколько поднимется меньший поршень.
    Ответ выразите в сантиметрах.
    $\rho_{\text{в}} = 1000\,\frac{\text{кг}}{\text{м}^{3}}$.
}
\answer{%
    \begin{align*}
    S_1h_1 &= S_2h_2 \implies h_1 = h_2 \frac{S_2}{S_1}  \\
    \frac{mg}{S_1} &= \rho_{\text{в}}g(h_1 + h_2) = \rho_{\text{в}}g \cdot h_2 \cbr{1 + \frac{S_2}{S_1}} \\
    h_2 &= \frac{mg}{S_1\rho_{\text{в}}g} \cdot \frac{S_1}{S_1 + S_2} = \frac{m}{\rho_{\text{в}}(S_1 + S_2)} = 6\,\text{см}
    \end{align*}
}

\variantsplitter

\addpersonalvariant{Виктория Легонькова}

\tasknumber{1}%
\task{%
    Гидростатическое давление столба воды равно $150\,\text{кПа}$.
    Определите высоту столба жидкости.
    Принять $\rho_{\text{в}} = 1000\,\frac{\text{кг}}{\text{м}^{3}}$, $g = 10\,\frac{\text{м}}{\text{с}^{2}}$.
}
\answer{%
    $p = \rho_{\text{в}}gh \implies h = \frac{p}{g\rho_{\text{в}}} = \frac{{150\,\text{кПа}}}{10\,\frac{\text{м}}{\text{с}^{2}} \cdot 1000\,\frac{\text{кг}}{\text{м}^{3}}} = 15\,\text{м}.$
}

\tasknumber{2}%
\task{%
    На какой глубине полное давление пресной воды превышает атмосферное в 6 раз?
    Принять $p_{\text{aтм}} = 100\,\text{кПа}$, $g = 10\,\frac{\text{м}}{\text{с}^{2}}$, $\rho_{\text{в}} = 1000\,\frac{\text{кг}}{\text{м}^{3}}$.
}
\answer{%
    $p = \rho_{\text{в}} g h + p_{\text{aтм}} = 6 p_{\text{aтм}} \implies h = \frac{(6-1) p_{\text{aтм}}}{g \rho_{\text{в}}} = \frac{(6-1) \cdot 100\,\text{кПа}}{10\,\frac{\text{м}}{\text{с}^{2}} \cdot 1000\,\frac{\text{кг}}{\text{м}^{3}}} = 50\,\text{м}.$
}

\tasknumber{3}%
\task{%
    В сосуд с вертикальными стенками и площадью горизонтального поперечного сечения $S = 0{,}03\,\text{м}^{2}$
    налили воду.
    На сколько увеличится сила давления на дно сосуда, если
    на поверхности воды ещё будет плавать тело массой $600\,\text{г}$.
    Принять $p_{\text{aтм}} = 100\,\text{кПа}$, $g = 10\,\frac{\text{м}}{\text{с}^{2}}$.
}
\answer{%
    $\Delta F = mg, \Delta p = \frac{\Delta F}{S} = \frac{mg}{S}, \Delta F = 6\,\text{Н}.$
}

\tasknumber{4}%
\task{%
    В два сообщающихся сосуда налита вода.
    В один из соcудов наливают нефть так,
    что столб этой жидкости имеет высоту $25\,\text{см}$.
    На сколько теперь уровень воды в этом сосуде ниже, чем в другом?
    Ответ выразите в сантиметрах.
    $\rho_{\text{в}} = 1000\,\frac{\text{кг}}{\text{м}^{3}}$, $\rho_{\text{н}} = 800\,\frac{\text{кг}}{\text{м}^{3}}$.
}
\answer{%
    $\rho_{\text{н}}gh_1 = \rho_{\text{в}}gh_2 \implies h_2 = h_1 \frac{\rho_{\text{н}}}{\rho_{\text{в}}} = 25\,\text{см} \cdot \frac{800\,\frac{\text{кг}}{\text{м}^{3}}}{1000\,\frac{\text{кг}}{\text{м}^{3}}} = 20\,\text{см}$
}

\tasknumber{5}%
\task{%
    В два сообщающихся сосуда сечений $32\,\text{см}^{2}$ и $18\,\text{см}^{2}$ налита вода.
    Оба сосуда закрыты лёгкими поршнями и находятся в равновесии.
    На больший из поршней кладут груз массой $400\,\text{г}$.
    Определите, на сколько поднимется меньший поршень.
    Ответ выразите в сантиметрах.
    $\rho_{\text{в}} = 1000\,\frac{\text{кг}}{\text{м}^{3}}$.
}
\answer{%
    \begin{align*}
    S_1h_1 &= S_2h_2 \implies h_1 = h_2 \frac{S_2}{S_1}  \\
    \frac{mg}{S_1} &= \rho_{\text{в}}g(h_1 + h_2) = \rho_{\text{в}}g \cdot h_2 \cbr{1 + \frac{S_2}{S_1}} \\
    h_2 &= \frac{mg}{S_1\rho_{\text{в}}g} \cdot \frac{S_1}{S_1 + S_2} = \frac{m}{\rho_{\text{в}}(S_1 + S_2)} = 8\,\text{см}
    \end{align*}
}

\variantsplitter

\addpersonalvariant{Семён Мартынов}

\tasknumber{1}%
\task{%
    Гидростатическое давление столба нефти равно $20\,\text{кПа}$.
    Определите высоту столба жидкости.
    Принять $\rho_{\text{н}} = 800\,\frac{\text{кг}}{\text{м}^{3}}$, $g = 10\,\frac{\text{м}}{\text{с}^{2}}$.
}
\answer{%
    $p = \rho_{\text{н}}gh \implies h = \frac{p}{g\rho_{\text{н}}} = \frac{{20\,\text{кПа}}}{10\,\frac{\text{м}}{\text{с}^{2}} \cdot 800\,\frac{\text{кг}}{\text{м}^{3}}} = 2{,}5\,\text{м}.$
}

\tasknumber{2}%
\task{%
    На какой глубине полное давление пресной воды превышает атмосферное в 6 раз?
    Принять $p_{\text{aтм}} = 100\,\text{кПа}$, $g = 10\,\frac{\text{м}}{\text{с}^{2}}$, $\rho_{\text{в}} = 1000\,\frac{\text{кг}}{\text{м}^{3}}$.
}
\answer{%
    $p = \rho_{\text{в}} g h + p_{\text{aтм}} = 6 p_{\text{aтм}} \implies h = \frac{(6-1) p_{\text{aтм}}}{g \rho_{\text{в}}} = \frac{(6-1) \cdot 100\,\text{кПа}}{10\,\frac{\text{м}}{\text{с}^{2}} \cdot 1000\,\frac{\text{кг}}{\text{м}^{3}}} = 50\,\text{м}.$
}

\tasknumber{3}%
\task{%
    В сосуд с вертикальными стенками и площадью горизонтального поперечного сечения $S = 0{,}02\,\text{м}^{2}$
    налили воду.
    На сколько увеличится сила давления на дно сосуда, если
    на поверхности воды ещё будет плавать тело массой $150\,\text{г}$.
    Принять $p_{\text{aтм}} = 100\,\text{кПа}$, $g = 10\,\frac{\text{м}}{\text{с}^{2}}$.
}
\answer{%
    $\Delta F = mg, \Delta p = \frac{\Delta F}{S} = \frac{mg}{S}, \Delta F = 1\,\text{Н}.$
}

\tasknumber{4}%
\task{%
    В два сообщающихся сосуда налита вода.
    В один из соcудов наливают нефть так,
    что столб этой жидкости имеет высоту $30\,\text{см}$.
    На сколько теперь уровень воды в этом сосуде ниже, чем в другом?
    Ответ выразите в сантиметрах.
    $\rho_{\text{в}} = 1000\,\frac{\text{кг}}{\text{м}^{3}}$, $\rho_{\text{н}} = 800\,\frac{\text{кг}}{\text{м}^{3}}$.
}
\answer{%
    $\rho_{\text{н}}gh_1 = \rho_{\text{в}}gh_2 \implies h_2 = h_1 \frac{\rho_{\text{н}}}{\rho_{\text{в}}} = 30\,\text{см} \cdot \frac{800\,\frac{\text{кг}}{\text{м}^{3}}}{1000\,\frac{\text{кг}}{\text{м}^{3}}} = 24\,\text{см}$
}

\tasknumber{5}%
\task{%
    В два сообщающихся сосуда сечений $60\,\text{см}^{2}$ и $40\,\text{см}^{2}$ налита вода.
    Оба сосуда закрыты лёгкими поршнями и находятся в равновесии.
    На больший из поршней кладут груз массой $400\,\text{г}$.
    Определите, на сколько поднимется меньший поршень.
    Ответ выразите в сантиметрах.
    $\rho_{\text{в}} = 1000\,\frac{\text{кг}}{\text{м}^{3}}$.
}
\answer{%
    \begin{align*}
    S_1h_1 &= S_2h_2 \implies h_1 = h_2 \frac{S_2}{S_1}  \\
    \frac{mg}{S_1} &= \rho_{\text{в}}g(h_1 + h_2) = \rho_{\text{в}}g \cdot h_2 \cbr{1 + \frac{S_2}{S_1}} \\
    h_2 &= \frac{mg}{S_1\rho_{\text{в}}g} \cdot \frac{S_1}{S_1 + S_2} = \frac{m}{\rho_{\text{в}}(S_1 + S_2)} = 4\,\text{см}
    \end{align*}
}

\variantsplitter

\addpersonalvariant{Варвара Минаева}

\tasknumber{1}%
\task{%
    Гидростатическое давление столба масла равно $27\,\text{кПа}$.
    Определите высоту столба жидкости.
    Принять $\rho_{\text{м}} = 900\,\frac{\text{кг}}{\text{м}^{3}}$, $g = 10\,\frac{\text{м}}{\text{с}^{2}}$.
}
\answer{%
    $p = \rho_{\text{м}}gh \implies h = \frac{p}{g\rho_{\text{м}}} = \frac{{27\,\text{кПа}}}{10\,\frac{\text{м}}{\text{с}^{2}} \cdot 900\,\frac{\text{кг}}{\text{м}^{3}}} = 3\,\text{м}.$
}

\tasknumber{2}%
\task{%
    На какой глубине полное давление пресной воды превышает атмосферное в 10 раз?
    Принять $p_{\text{aтм}} = 100\,\text{кПа}$, $g = 10\,\frac{\text{м}}{\text{с}^{2}}$, $\rho_{\text{в}} = 1000\,\frac{\text{кг}}{\text{м}^{3}}$.
}
\answer{%
    $p = \rho_{\text{в}} g h + p_{\text{aтм}} = 10 p_{\text{aтм}} \implies h = \frac{(10-1) p_{\text{aтм}}}{g \rho_{\text{в}}} = \frac{(10-1) \cdot 100\,\text{кПа}}{10\,\frac{\text{м}}{\text{с}^{2}} \cdot 1000\,\frac{\text{кг}}{\text{м}^{3}}} = 90\,\text{м}.$
}

\tasknumber{3}%
\task{%
    В сосуд с вертикальными стенками и площадью горизонтального поперечного сечения $S = 0{,}010\,\text{м}^{2}$
    налили воду.
    На сколько увеличится сила давления на дно сосуда, если
    на поверхности воды ещё будет плавать тело массой $150\,\text{г}$.
    Принять $p_{\text{aтм}} = 100\,\text{кПа}$, $g = 10\,\frac{\text{м}}{\text{с}^{2}}$.
}
\answer{%
    $\Delta F = mg, \Delta p = \frac{\Delta F}{S} = \frac{mg}{S}, \Delta F = 1\,\text{Н}.$
}

\tasknumber{4}%
\task{%
    В два сообщающихся сосуда налита вода.
    В один из соcудов наливают масло так,
    что столб этой жидкости имеет высоту $80\,\text{см}$.
    На сколько теперь уровень воды в этом сосуде ниже, чем в другом?
    Ответ выразите в сантиметрах.
    $\rho_{\text{в}} = 1000\,\frac{\text{кг}}{\text{м}^{3}}$, $\rho_{\text{м}} = 900\,\frac{\text{кг}}{\text{м}^{3}}$.
}
\answer{%
    $\rho_{\text{м}}gh_1 = \rho_{\text{в}}gh_2 \implies h_2 = h_1 \frac{\rho_{\text{м}}}{\rho_{\text{в}}} = 80\,\text{см} \cdot \frac{900\,\frac{\text{кг}}{\text{м}^{3}}}{1000\,\frac{\text{кг}}{\text{м}^{3}}} = 72\,\text{см}$
}

\tasknumber{5}%
\task{%
    В два сообщающихся сосуда сечений $40\,\text{см}^{2}$ и $10\,\text{см}^{2}$ налита вода.
    Оба сосуда закрыты лёгкими поршнями и находятся в равновесии.
    На больший из поршней кладут груз массой $400\,\text{г}$.
    Определите, на сколько поднимется меньший поршень.
    Ответ выразите в сантиметрах.
    $\rho_{\text{в}} = 1000\,\frac{\text{кг}}{\text{м}^{3}}$.
}
\answer{%
    \begin{align*}
    S_1h_1 &= S_2h_2 \implies h_1 = h_2 \frac{S_2}{S_1}  \\
    \frac{mg}{S_1} &= \rho_{\text{в}}g(h_1 + h_2) = \rho_{\text{в}}g \cdot h_2 \cbr{1 + \frac{S_2}{S_1}} \\
    h_2 &= \frac{mg}{S_1\rho_{\text{в}}g} \cdot \frac{S_1}{S_1 + S_2} = \frac{m}{\rho_{\text{в}}(S_1 + S_2)} = 8\,\text{см}
    \end{align*}
}

\variantsplitter

\addpersonalvariant{Леонид Никитин}

\tasknumber{1}%
\task{%
    Гидростатическое давление столба масла равно $18\,\text{кПа}$.
    Определите высоту столба жидкости.
    Принять $\rho_{\text{м}} = 900\,\frac{\text{кг}}{\text{м}^{3}}$, $g = 10\,\frac{\text{м}}{\text{с}^{2}}$.
}
\answer{%
    $p = \rho_{\text{м}}gh \implies h = \frac{p}{g\rho_{\text{м}}} = \frac{{18\,\text{кПа}}}{10\,\frac{\text{м}}{\text{с}^{2}} \cdot 900\,\frac{\text{кг}}{\text{м}^{3}}} = 2\,\text{м}.$
}

\tasknumber{2}%
\task{%
    На какой глубине полное давление пресной воды превышает атмосферное в 6 раз?
    Принять $p_{\text{aтм}} = 100\,\text{кПа}$, $g = 10\,\frac{\text{м}}{\text{с}^{2}}$, $\rho_{\text{в}} = 1000\,\frac{\text{кг}}{\text{м}^{3}}$.
}
\answer{%
    $p = \rho_{\text{в}} g h + p_{\text{aтм}} = 6 p_{\text{aтм}} \implies h = \frac{(6-1) p_{\text{aтм}}}{g \rho_{\text{в}}} = \frac{(6-1) \cdot 100\,\text{кПа}}{10\,\frac{\text{м}}{\text{с}^{2}} \cdot 1000\,\frac{\text{кг}}{\text{м}^{3}}} = 50\,\text{м}.$
}

\tasknumber{3}%
\task{%
    В сосуд с вертикальными стенками и площадью горизонтального поперечного сечения $S = 0{,}05\,\text{м}^{2}$
    налили воду.
    На сколько увеличится сила давления на дно сосуда, если
    на поверхности воды ещё будет плавать тело массой $600\,\text{г}$.
    Принять $p_{\text{aтм}} = 100\,\text{кПа}$, $g = 10\,\frac{\text{м}}{\text{с}^{2}}$.
}
\answer{%
    $\Delta F = mg, \Delta p = \frac{\Delta F}{S} = \frac{mg}{S}, \Delta F = 6\,\text{Н}.$
}

\tasknumber{4}%
\task{%
    В два сообщающихся сосуда налита вода.
    В один из соcудов наливают нефть так,
    что столб этой жидкости имеет высоту $5\,\text{см}$.
    На сколько теперь уровень воды в этом сосуде ниже, чем в другом?
    Ответ выразите в сантиметрах.
    $\rho_{\text{в}} = 1000\,\frac{\text{кг}}{\text{м}^{3}}$, $\rho_{\text{н}} = 800\,\frac{\text{кг}}{\text{м}^{3}}$.
}
\answer{%
    $\rho_{\text{н}}gh_1 = \rho_{\text{в}}gh_2 \implies h_2 = h_1 \frac{\rho_{\text{н}}}{\rho_{\text{в}}} = 5\,\text{см} \cdot \frac{800\,\frac{\text{кг}}{\text{м}^{3}}}{1000\,\frac{\text{кг}}{\text{м}^{3}}} = 4\,\text{см}$
}

\tasknumber{5}%
\task{%
    В два сообщающихся сосуда сечений $90\,\text{см}^{2}$ и $10\,\text{см}^{2}$ налита вода.
    Оба сосуда закрыты лёгкими поршнями и находятся в равновесии.
    На больший из поршней кладут груз массой $400\,\text{г}$.
    Определите, на сколько поднимется меньший поршень.
    Ответ выразите в сантиметрах.
    $\rho_{\text{в}} = 1000\,\frac{\text{кг}}{\text{м}^{3}}$.
}
\answer{%
    \begin{align*}
    S_1h_1 &= S_2h_2 \implies h_1 = h_2 \frac{S_2}{S_1}  \\
    \frac{mg}{S_1} &= \rho_{\text{в}}g(h_1 + h_2) = \rho_{\text{в}}g \cdot h_2 \cbr{1 + \frac{S_2}{S_1}} \\
    h_2 &= \frac{mg}{S_1\rho_{\text{в}}g} \cdot \frac{S_1}{S_1 + S_2} = \frac{m}{\rho_{\text{в}}(S_1 + S_2)} = 4\,\text{см}
    \end{align*}
}

\variantsplitter

\addpersonalvariant{Тимофей Полетаев}

\tasknumber{1}%
\task{%
    Гидростатическое давление столба масла равно $9\,\text{кПа}$.
    Определите высоту столба жидкости.
    Принять $\rho_{\text{м}} = 900\,\frac{\text{кг}}{\text{м}^{3}}$, $g = 10\,\frac{\text{м}}{\text{с}^{2}}$.
}
\answer{%
    $p = \rho_{\text{м}}gh \implies h = \frac{p}{g\rho_{\text{м}}} = \frac{{9\,\text{кПа}}}{10\,\frac{\text{м}}{\text{с}^{2}} \cdot 900\,\frac{\text{кг}}{\text{м}^{3}}} = 1\,\text{м}.$
}

\tasknumber{2}%
\task{%
    На какой глубине полное давление пресной воды превышает атмосферное в 10 раз?
    Принять $p_{\text{aтм}} = 100\,\text{кПа}$, $g = 10\,\frac{\text{м}}{\text{с}^{2}}$, $\rho_{\text{в}} = 1000\,\frac{\text{кг}}{\text{м}^{3}}$.
}
\answer{%
    $p = \rho_{\text{в}} g h + p_{\text{aтм}} = 10 p_{\text{aтм}} \implies h = \frac{(10-1) p_{\text{aтм}}}{g \rho_{\text{в}}} = \frac{(10-1) \cdot 100\,\text{кПа}}{10\,\frac{\text{м}}{\text{с}^{2}} \cdot 1000\,\frac{\text{кг}}{\text{м}^{3}}} = 90\,\text{м}.$
}

\tasknumber{3}%
\task{%
    В сосуд с вертикальными стенками и площадью горизонтального поперечного сечения $S = 0{,}010\,\text{м}^{2}$
    налили воду.
    На сколько увеличится сила давления на дно сосуда, если
    на поверхности воды ещё будет плавать тело массой $300\,\text{г}$.
    Принять $p_{\text{aтм}} = 100\,\text{кПа}$, $g = 10\,\frac{\text{м}}{\text{с}^{2}}$.
}
\answer{%
    $\Delta F = mg, \Delta p = \frac{\Delta F}{S} = \frac{mg}{S}, \Delta F = 3\,\text{Н}.$
}

\tasknumber{4}%
\task{%
    В два сообщающихся сосуда налита вода.
    В один из соcудов наливают масло так,
    что столб этой жидкости имеет высоту $80\,\text{см}$.
    На сколько теперь уровень воды в этом сосуде ниже, чем в другом?
    Ответ выразите в сантиметрах.
    $\rho_{\text{в}} = 1000\,\frac{\text{кг}}{\text{м}^{3}}$, $\rho_{\text{м}} = 900\,\frac{\text{кг}}{\text{м}^{3}}$.
}
\answer{%
    $\rho_{\text{м}}gh_1 = \rho_{\text{в}}gh_2 \implies h_2 = h_1 \frac{\rho_{\text{м}}}{\rho_{\text{в}}} = 80\,\text{см} \cdot \frac{900\,\frac{\text{кг}}{\text{м}^{3}}}{1000\,\frac{\text{кг}}{\text{м}^{3}}} = 72\,\text{см}$
}

\tasknumber{5}%
\task{%
    В два сообщающихся сосуда сечений $80\,\text{см}^{2}$ и $20\,\text{см}^{2}$ налита вода.
    Оба сосуда закрыты лёгкими поршнями и находятся в равновесии.
    На больший из поршней кладут груз массой $400\,\text{г}$.
    Определите, на сколько поднимется меньший поршень.
    Ответ выразите в сантиметрах.
    $\rho_{\text{в}} = 1000\,\frac{\text{кг}}{\text{м}^{3}}$.
}
\answer{%
    \begin{align*}
    S_1h_1 &= S_2h_2 \implies h_1 = h_2 \frac{S_2}{S_1}  \\
    \frac{mg}{S_1} &= \rho_{\text{в}}g(h_1 + h_2) = \rho_{\text{в}}g \cdot h_2 \cbr{1 + \frac{S_2}{S_1}} \\
    h_2 &= \frac{mg}{S_1\rho_{\text{в}}g} \cdot \frac{S_1}{S_1 + S_2} = \frac{m}{\rho_{\text{в}}(S_1 + S_2)} = 4\,\text{см}
    \end{align*}
}

\variantsplitter

\addpersonalvariant{Андрей Рожков}

\tasknumber{1}%
\task{%
    Гидростатическое давление столба воды равно $100\,\text{кПа}$.
    Определите высоту столба жидкости.
    Принять $\rho_{\text{в}} = 1000\,\frac{\text{кг}}{\text{м}^{3}}$, $g = 10\,\frac{\text{м}}{\text{с}^{2}}$.
}
\answer{%
    $p = \rho_{\text{в}}gh \implies h = \frac{p}{g\rho_{\text{в}}} = \frac{{100\,\text{кПа}}}{10\,\frac{\text{м}}{\text{с}^{2}} \cdot 1000\,\frac{\text{кг}}{\text{м}^{3}}} = 10\,\text{м}.$
}

\tasknumber{2}%
\task{%
    На какой глубине полное давление пресной воды превышает атмосферное в 5 раз?
    Принять $p_{\text{aтм}} = 100\,\text{кПа}$, $g = 10\,\frac{\text{м}}{\text{с}^{2}}$, $\rho_{\text{в}} = 1000\,\frac{\text{кг}}{\text{м}^{3}}$.
}
\answer{%
    $p = \rho_{\text{в}} g h + p_{\text{aтм}} = 5 p_{\text{aтм}} \implies h = \frac{(5-1) p_{\text{aтм}}}{g \rho_{\text{в}}} = \frac{(5-1) \cdot 100\,\text{кПа}}{10\,\frac{\text{м}}{\text{с}^{2}} \cdot 1000\,\frac{\text{кг}}{\text{м}^{3}}} = 40\,\text{м}.$
}

\tasknumber{3}%
\task{%
    В сосуд с вертикальными стенками и площадью горизонтального поперечного сечения $S = 0{,}03\,\text{м}^{2}$
    налили воду.
    На сколько увеличится давление на дно сосуда, если
    на поверхности воды ещё будет плавать тело массой $300\,\text{г}$.
    Принять $p_{\text{aтм}} = 100\,\text{кПа}$, $g = 10\,\frac{\text{м}}{\text{с}^{2}}$.
}
\answer{%
    $\Delta F = mg, \Delta p = \frac{\Delta F}{S} = \frac{mg}{S}, \Delta p = 100\,\text{Па}.$
}

\tasknumber{4}%
\task{%
    В два сообщающихся сосуда налита вода.
    В один из соcудов наливают нефть так,
    что столб этой жидкости имеет высоту $20\,\text{см}$.
    На сколько теперь уровень воды в этом сосуде ниже, чем в другом?
    Ответ выразите в сантиметрах.
    $\rho_{\text{в}} = 1000\,\frac{\text{кг}}{\text{м}^{3}}$, $\rho_{\text{н}} = 800\,\frac{\text{кг}}{\text{м}^{3}}$.
}
\answer{%
    $\rho_{\text{н}}gh_1 = \rho_{\text{в}}gh_2 \implies h_2 = h_1 \frac{\rho_{\text{н}}}{\rho_{\text{в}}} = 20\,\text{см} \cdot \frac{800\,\frac{\text{кг}}{\text{м}^{3}}}{1000\,\frac{\text{кг}}{\text{м}^{3}}} = 16\,\text{см}$
}

\tasknumber{5}%
\task{%
    В два сообщающихся сосуда сечений $30\,\text{см}^{2}$ и $20\,\text{см}^{2}$ налита вода.
    Оба сосуда закрыты лёгкими поршнями и находятся в равновесии.
    На больший из поршней кладут груз массой $200\,\text{г}$.
    Определите, на сколько поднимется меньший поршень.
    Ответ выразите в сантиметрах.
    $\rho_{\text{в}} = 1000\,\frac{\text{кг}}{\text{м}^{3}}$.
}
\answer{%
    \begin{align*}
    S_1h_1 &= S_2h_2 \implies h_1 = h_2 \frac{S_2}{S_1}  \\
    \frac{mg}{S_1} &= \rho_{\text{в}}g(h_1 + h_2) = \rho_{\text{в}}g \cdot h_2 \cbr{1 + \frac{S_2}{S_1}} \\
    h_2 &= \frac{mg}{S_1\rho_{\text{в}}g} \cdot \frac{S_1}{S_1 + S_2} = \frac{m}{\rho_{\text{в}}(S_1 + S_2)} = 4\,\text{см}
    \end{align*}
}

\variantsplitter

\addpersonalvariant{Рената Таржиманова}

\tasknumber{1}%
\task{%
    Гидростатическое давление столба нефти равно $20\,\text{кПа}$.
    Определите высоту столба жидкости.
    Принять $\rho_{\text{н}} = 800\,\frac{\text{кг}}{\text{м}^{3}}$, $g = 10\,\frac{\text{м}}{\text{с}^{2}}$.
}
\answer{%
    $p = \rho_{\text{н}}gh \implies h = \frac{p}{g\rho_{\text{н}}} = \frac{{20\,\text{кПа}}}{10\,\frac{\text{м}}{\text{с}^{2}} \cdot 800\,\frac{\text{кг}}{\text{м}^{3}}} = 2{,}5\,\text{м}.$
}

\tasknumber{2}%
\task{%
    На какой глубине полное давление пресной воды превышает атмосферное в 6 раз?
    Принять $p_{\text{aтм}} = 100\,\text{кПа}$, $g = 10\,\frac{\text{м}}{\text{с}^{2}}$, $\rho_{\text{в}} = 1000\,\frac{\text{кг}}{\text{м}^{3}}$.
}
\answer{%
    $p = \rho_{\text{в}} g h + p_{\text{aтм}} = 6 p_{\text{aтм}} \implies h = \frac{(6-1) p_{\text{aтм}}}{g \rho_{\text{в}}} = \frac{(6-1) \cdot 100\,\text{кПа}}{10\,\frac{\text{м}}{\text{с}^{2}} \cdot 1000\,\frac{\text{кг}}{\text{м}^{3}}} = 50\,\text{м}.$
}

\tasknumber{3}%
\task{%
    В сосуд с вертикальными стенками и площадью горизонтального поперечного сечения $S = 0{,}03\,\text{м}^{2}$
    налили воду.
    На сколько увеличится сила давления на дно сосуда, если
    на поверхности воды ещё будет плавать тело массой $600\,\text{г}$.
    Принять $p_{\text{aтм}} = 100\,\text{кПа}$, $g = 10\,\frac{\text{м}}{\text{с}^{2}}$.
}
\answer{%
    $\Delta F = mg, \Delta p = \frac{\Delta F}{S} = \frac{mg}{S}, \Delta F = 6\,\text{Н}.$
}

\tasknumber{4}%
\task{%
    В два сообщающихся сосуда налита вода.
    В один из соcудов наливают нефть так,
    что столб этой жидкости имеет высоту $25\,\text{см}$.
    На сколько теперь уровень воды в этом сосуде ниже, чем в другом?
    Ответ выразите в сантиметрах.
    $\rho_{\text{в}} = 1000\,\frac{\text{кг}}{\text{м}^{3}}$, $\rho_{\text{н}} = 800\,\frac{\text{кг}}{\text{м}^{3}}$.
}
\answer{%
    $\rho_{\text{н}}gh_1 = \rho_{\text{в}}gh_2 \implies h_2 = h_1 \frac{\rho_{\text{н}}}{\rho_{\text{в}}} = 25\,\text{см} \cdot \frac{800\,\frac{\text{кг}}{\text{м}^{3}}}{1000\,\frac{\text{кг}}{\text{м}^{3}}} = 20\,\text{см}$
}

\tasknumber{5}%
\task{%
    В два сообщающихся сосуда сечений $40\,\text{см}^{2}$ и $10\,\text{см}^{2}$ налита вода.
    Оба сосуда закрыты лёгкими поршнями и находятся в равновесии.
    На больший из поршней кладут груз массой $200\,\text{г}$.
    Определите, на сколько поднимется меньший поршень.
    Ответ выразите в сантиметрах.
    $\rho_{\text{в}} = 1000\,\frac{\text{кг}}{\text{м}^{3}}$.
}
\answer{%
    \begin{align*}
    S_1h_1 &= S_2h_2 \implies h_1 = h_2 \frac{S_2}{S_1}  \\
    \frac{mg}{S_1} &= \rho_{\text{в}}g(h_1 + h_2) = \rho_{\text{в}}g \cdot h_2 \cbr{1 + \frac{S_2}{S_1}} \\
    h_2 &= \frac{mg}{S_1\rho_{\text{в}}g} \cdot \frac{S_1}{S_1 + S_2} = \frac{m}{\rho_{\text{в}}(S_1 + S_2)} = 4\,\text{см}
    \end{align*}
}

\variantsplitter

\addpersonalvariant{Андрей Щербаков}

\tasknumber{1}%
\task{%
    Гидростатическое давление столба воды равно $200\,\text{кПа}$.
    Определите высоту столба жидкости.
    Принять $\rho_{\text{в}} = 1000\,\frac{\text{кг}}{\text{м}^{3}}$, $g = 10\,\frac{\text{м}}{\text{с}^{2}}$.
}
\answer{%
    $p = \rho_{\text{в}}gh \implies h = \frac{p}{g\rho_{\text{в}}} = \frac{{200\,\text{кПа}}}{10\,\frac{\text{м}}{\text{с}^{2}} \cdot 1000\,\frac{\text{кг}}{\text{м}^{3}}} = 20\,\text{м}.$
}

\tasknumber{2}%
\task{%
    На какой глубине полное давление пресной воды превышает атмосферное в 8 раз?
    Принять $p_{\text{aтм}} = 100\,\text{кПа}$, $g = 10\,\frac{\text{м}}{\text{с}^{2}}$, $\rho_{\text{в}} = 1000\,\frac{\text{кг}}{\text{м}^{3}}$.
}
\answer{%
    $p = \rho_{\text{в}} g h + p_{\text{aтм}} = 8 p_{\text{aтм}} \implies h = \frac{(8-1) p_{\text{aтм}}}{g \rho_{\text{в}}} = \frac{(8-1) \cdot 100\,\text{кПа}}{10\,\frac{\text{м}}{\text{с}^{2}} \cdot 1000\,\frac{\text{кг}}{\text{м}^{3}}} = 70\,\text{м}.$
}

\tasknumber{3}%
\task{%
    В сосуд с вертикальными стенками и площадью горизонтального поперечного сечения $S = 0{,}05\,\text{м}^{2}$
    налили воду.
    На сколько увеличится сила давления на дно сосуда, если
    на поверхности воды ещё будет плавать тело массой $900\,\text{г}$.
    Принять $p_{\text{aтм}} = 100\,\text{кПа}$, $g = 10\,\frac{\text{м}}{\text{с}^{2}}$.
}
\answer{%
    $\Delta F = mg, \Delta p = \frac{\Delta F}{S} = \frac{mg}{S}, \Delta F = 9\,\text{Н}.$
}

\tasknumber{4}%
\task{%
    В два сообщающихся сосуда налита вода.
    В один из соcудов наливают масло так,
    что столб этой жидкости имеет высоту $20\,\text{см}$.
    На сколько теперь уровень воды в этом сосуде ниже, чем в другом?
    Ответ выразите в сантиметрах.
    $\rho_{\text{в}} = 1000\,\frac{\text{кг}}{\text{м}^{3}}$, $\rho_{\text{м}} = 900\,\frac{\text{кг}}{\text{м}^{3}}$.
}
\answer{%
    $\rho_{\text{м}}gh_1 = \rho_{\text{в}}gh_2 \implies h_2 = h_1 \frac{\rho_{\text{м}}}{\rho_{\text{в}}} = 20\,\text{см} \cdot \frac{900\,\frac{\text{кг}}{\text{м}^{3}}}{1000\,\frac{\text{кг}}{\text{м}^{3}}} = 18\,\text{см}$
}

\tasknumber{5}%
\task{%
    В два сообщающихся сосуда сечений $30\,\text{см}^{2}$ и $20\,\text{см}^{2}$ налита вода.
    Оба сосуда закрыты лёгкими поршнями и находятся в равновесии.
    На больший из поршней кладут груз массой $400\,\text{г}$.
    Определите, на сколько поднимется меньший поршень.
    Ответ выразите в сантиметрах.
    $\rho_{\text{в}} = 1000\,\frac{\text{кг}}{\text{м}^{3}}$.
}
\answer{%
    \begin{align*}
    S_1h_1 &= S_2h_2 \implies h_1 = h_2 \frac{S_2}{S_1}  \\
    \frac{mg}{S_1} &= \rho_{\text{в}}g(h_1 + h_2) = \rho_{\text{в}}g \cdot h_2 \cbr{1 + \frac{S_2}{S_1}} \\
    h_2 &= \frac{mg}{S_1\rho_{\text{в}}g} \cdot \frac{S_1}{S_1 + S_2} = \frac{m}{\rho_{\text{в}}(S_1 + S_2)} = 8\,\text{см}
    \end{align*}
}

\variantsplitter

\addpersonalvariant{Михаил Ярошевский}

\tasknumber{1}%
\task{%
    Гидростатическое давление столба воды равно $50\,\text{кПа}$.
    Определите высоту столба жидкости.
    Принять $\rho_{\text{в}} = 1000\,\frac{\text{кг}}{\text{м}^{3}}$, $g = 10\,\frac{\text{м}}{\text{с}^{2}}$.
}
\answer{%
    $p = \rho_{\text{в}}gh \implies h = \frac{p}{g\rho_{\text{в}}} = \frac{{50\,\text{кПа}}}{10\,\frac{\text{м}}{\text{с}^{2}} \cdot 1000\,\frac{\text{кг}}{\text{м}^{3}}} = 5\,\text{м}.$
}

\tasknumber{2}%
\task{%
    На какой глубине полное давление пресной воды превышает атмосферное в 3 раз?
    Принять $p_{\text{aтм}} = 100\,\text{кПа}$, $g = 10\,\frac{\text{м}}{\text{с}^{2}}$, $\rho_{\text{в}} = 1000\,\frac{\text{кг}}{\text{м}^{3}}$.
}
\answer{%
    $p = \rho_{\text{в}} g h + p_{\text{aтм}} = 3 p_{\text{aтм}} \implies h = \frac{(3-1) p_{\text{aтм}}}{g \rho_{\text{в}}} = \frac{(3-1) \cdot 100\,\text{кПа}}{10\,\frac{\text{м}}{\text{с}^{2}} \cdot 1000\,\frac{\text{кг}}{\text{м}^{3}}} = 20\,\text{м}.$
}

\tasknumber{3}%
\task{%
    В сосуд с вертикальными стенками и площадью горизонтального поперечного сечения $S = 0{,}010\,\text{м}^{2}$
    налили воду.
    На сколько увеличится давление на дно сосуда, если
    на поверхности воды ещё будет плавать тело массой $600\,\text{г}$.
    Принять $p_{\text{aтм}} = 100\,\text{кПа}$, $g = 10\,\frac{\text{м}}{\text{с}^{2}}$.
}
\answer{%
    $\Delta F = mg, \Delta p = \frac{\Delta F}{S} = \frac{mg}{S}, \Delta p = 600\,\text{Па}.$
}

\tasknumber{4}%
\task{%
    В два сообщающихся сосуда налита вода.
    В один из соcудов наливают масло так,
    что столб этой жидкости имеет высоту $20\,\text{см}$.
    На сколько теперь уровень воды в этом сосуде ниже, чем в другом?
    Ответ выразите в сантиметрах.
    $\rho_{\text{в}} = 1000\,\frac{\text{кг}}{\text{м}^{3}}$, $\rho_{\text{м}} = 900\,\frac{\text{кг}}{\text{м}^{3}}$.
}
\answer{%
    $\rho_{\text{м}}gh_1 = \rho_{\text{в}}gh_2 \implies h_2 = h_1 \frac{\rho_{\text{м}}}{\rho_{\text{в}}} = 20\,\text{см} \cdot \frac{900\,\frac{\text{кг}}{\text{м}^{3}}}{1000\,\frac{\text{кг}}{\text{м}^{3}}} = 18\,\text{см}$
}

\tasknumber{5}%
\task{%
    В два сообщающихся сосуда сечений $30\,\text{см}^{2}$ и $20\,\text{см}^{2}$ налита вода.
    Оба сосуда закрыты лёгкими поршнями и находятся в равновесии.
    На больший из поршней кладут груз массой $300\,\text{г}$.
    Определите, на сколько поднимется меньший поршень.
    Ответ выразите в сантиметрах.
    $\rho_{\text{в}} = 1000\,\frac{\text{кг}}{\text{м}^{3}}$.
}
\answer{%
    \begin{align*}
    S_1h_1 &= S_2h_2 \implies h_1 = h_2 \frac{S_2}{S_1}  \\
    \frac{mg}{S_1} &= \rho_{\text{в}}g(h_1 + h_2) = \rho_{\text{в}}g \cdot h_2 \cbr{1 + \frac{S_2}{S_1}} \\
    h_2 &= \frac{mg}{S_1\rho_{\text{в}}g} \cdot \frac{S_1}{S_1 + S_2} = \frac{m}{\rho_{\text{в}}(S_1 + S_2)} = 6\,\text{см}
    \end{align*}
}

\variantsplitter

\addpersonalvariant{Алексей Алимпиев}

\tasknumber{1}%
\task{%
    Гидростатическое давление столба воды равно $200\,\text{кПа}$.
    Определите высоту столба жидкости.
    Принять $\rho_{\text{в}} = 1000\,\frac{\text{кг}}{\text{м}^{3}}$, $g = 10\,\frac{\text{м}}{\text{с}^{2}}$.
}
\answer{%
    $p = \rho_{\text{в}}gh \implies h = \frac{p}{g\rho_{\text{в}}} = \frac{{200\,\text{кПа}}}{10\,\frac{\text{м}}{\text{с}^{2}} \cdot 1000\,\frac{\text{кг}}{\text{м}^{3}}} = 20\,\text{м}.$
}

\tasknumber{2}%
\task{%
    На какой глубине полное давление пресной воды превышает атмосферное в 10 раз?
    Принять $p_{\text{aтм}} = 100\,\text{кПа}$, $g = 10\,\frac{\text{м}}{\text{с}^{2}}$, $\rho_{\text{в}} = 1000\,\frac{\text{кг}}{\text{м}^{3}}$.
}
\answer{%
    $p = \rho_{\text{в}} g h + p_{\text{aтм}} = 10 p_{\text{aтм}} \implies h = \frac{(10-1) p_{\text{aтм}}}{g \rho_{\text{в}}} = \frac{(10-1) \cdot 100\,\text{кПа}}{10\,\frac{\text{м}}{\text{с}^{2}} \cdot 1000\,\frac{\text{кг}}{\text{м}^{3}}} = 90\,\text{м}.$
}

\tasknumber{3}%
\task{%
    В сосуд с вертикальными стенками и площадью горизонтального поперечного сечения $S = 0{,}05\,\text{м}^{2}$
    налили воду.
    На сколько увеличится сила давления на дно сосуда, если
    на поверхности воды ещё будет плавать тело массой $900\,\text{г}$.
    Принять $p_{\text{aтм}} = 100\,\text{кПа}$, $g = 10\,\frac{\text{м}}{\text{с}^{2}}$.
}
\answer{%
    $\Delta F = mg, \Delta p = \frac{\Delta F}{S} = \frac{mg}{S}, \Delta F = 9\,\text{Н}.$
}

\tasknumber{4}%
\task{%
    В два сообщающихся сосуда налита вода.
    В один из соcудов наливают масло так,
    что столб этой жидкости имеет высоту $30\,\text{см}$.
    На сколько теперь уровень воды в этом сосуде ниже, чем в другом?
    Ответ выразите в сантиметрах.
    $\rho_{\text{в}} = 1000\,\frac{\text{кг}}{\text{м}^{3}}$, $\rho_{\text{м}} = 900\,\frac{\text{кг}}{\text{м}^{3}}$.
}
\answer{%
    $\rho_{\text{м}}gh_1 = \rho_{\text{в}}gh_2 \implies h_2 = h_1 \frac{\rho_{\text{м}}}{\rho_{\text{в}}} = 30\,\text{см} \cdot \frac{900\,\frac{\text{кг}}{\text{м}^{3}}}{1000\,\frac{\text{кг}}{\text{м}^{3}}} = 27\,\text{см}$
}

\tasknumber{5}%
\task{%
    В два сообщающихся сосуда сечений $32\,\text{см}^{2}$ и $18\,\text{см}^{2}$ налита вода.
    Оба сосуда закрыты лёгкими поршнями и находятся в равновесии.
    На больший из поршней кладут груз массой $500\,\text{г}$.
    Определите, на сколько поднимется меньший поршень.
    Ответ выразите в сантиметрах.
    $\rho_{\text{в}} = 1000\,\frac{\text{кг}}{\text{м}^{3}}$.
}
\answer{%
    \begin{align*}
    S_1h_1 &= S_2h_2 \implies h_1 = h_2 \frac{S_2}{S_1}  \\
    \frac{mg}{S_1} &= \rho_{\text{в}}g(h_1 + h_2) = \rho_{\text{в}}g \cdot h_2 \cbr{1 + \frac{S_2}{S_1}} \\
    h_2 &= \frac{mg}{S_1\rho_{\text{в}}g} \cdot \frac{S_1}{S_1 + S_2} = \frac{m}{\rho_{\text{в}}(S_1 + S_2)} = 10\,\text{см}
    \end{align*}
}

\variantsplitter

\addpersonalvariant{Евгений Васин}

\tasknumber{1}%
\task{%
    Гидростатическое давление столба нефти равно $40\,\text{кПа}$.
    Определите высоту столба жидкости.
    Принять $\rho_{\text{н}} = 800\,\frac{\text{кг}}{\text{м}^{3}}$, $g = 10\,\frac{\text{м}}{\text{с}^{2}}$.
}
\answer{%
    $p = \rho_{\text{н}}gh \implies h = \frac{p}{g\rho_{\text{н}}} = \frac{{40\,\text{кПа}}}{10\,\frac{\text{м}}{\text{с}^{2}} \cdot 800\,\frac{\text{кг}}{\text{м}^{3}}} = 5\,\text{м}.$
}

\tasknumber{2}%
\task{%
    На какой глубине полное давление пресной воды превышает атмосферное в 3 раз?
    Принять $p_{\text{aтм}} = 100\,\text{кПа}$, $g = 10\,\frac{\text{м}}{\text{с}^{2}}$, $\rho_{\text{в}} = 1000\,\frac{\text{кг}}{\text{м}^{3}}$.
}
\answer{%
    $p = \rho_{\text{в}} g h + p_{\text{aтм}} = 3 p_{\text{aтм}} \implies h = \frac{(3-1) p_{\text{aтм}}}{g \rho_{\text{в}}} = \frac{(3-1) \cdot 100\,\text{кПа}}{10\,\frac{\text{м}}{\text{с}^{2}} \cdot 1000\,\frac{\text{кг}}{\text{м}^{3}}} = 20\,\text{м}.$
}

\tasknumber{3}%
\task{%
    В сосуд с вертикальными стенками и площадью горизонтального поперечного сечения $S = 0{,}05\,\text{м}^{2}$
    налили воду.
    На сколько увеличится сила давления на дно сосуда, если
    на поверхности воды ещё будет плавать тело массой $300\,\text{г}$.
    Принять $p_{\text{aтм}} = 100\,\text{кПа}$, $g = 10\,\frac{\text{м}}{\text{с}^{2}}$.
}
\answer{%
    $\Delta F = mg, \Delta p = \frac{\Delta F}{S} = \frac{mg}{S}, \Delta F = 3\,\text{Н}.$
}

\tasknumber{4}%
\task{%
    В два сообщающихся сосуда налита вода.
    В один из соcудов наливают масло так,
    что столб этой жидкости имеет высоту $50\,\text{см}$.
    На сколько теперь уровень воды в этом сосуде ниже, чем в другом?
    Ответ выразите в сантиметрах.
    $\rho_{\text{в}} = 1000\,\frac{\text{кг}}{\text{м}^{3}}$, $\rho_{\text{м}} = 900\,\frac{\text{кг}}{\text{м}^{3}}$.
}
\answer{%
    $\rho_{\text{м}}gh_1 = \rho_{\text{в}}gh_2 \implies h_2 = h_1 \frac{\rho_{\text{м}}}{\rho_{\text{в}}} = 50\,\text{см} \cdot \frac{900\,\frac{\text{кг}}{\text{м}^{3}}}{1000\,\frac{\text{кг}}{\text{м}^{3}}} = 45\,\text{см}$
}

\tasknumber{5}%
\task{%
    В два сообщающихся сосуда сечений $32\,\text{см}^{2}$ и $18\,\text{см}^{2}$ налита вода.
    Оба сосуда закрыты лёгкими поршнями и находятся в равновесии.
    На больший из поршней кладут груз массой $300\,\text{г}$.
    Определите, на сколько поднимется меньший поршень.
    Ответ выразите в сантиметрах.
    $\rho_{\text{в}} = 1000\,\frac{\text{кг}}{\text{м}^{3}}$.
}
\answer{%
    \begin{align*}
    S_1h_1 &= S_2h_2 \implies h_1 = h_2 \frac{S_2}{S_1}  \\
    \frac{mg}{S_1} &= \rho_{\text{в}}g(h_1 + h_2) = \rho_{\text{в}}g \cdot h_2 \cbr{1 + \frac{S_2}{S_1}} \\
    h_2 &= \frac{mg}{S_1\rho_{\text{в}}g} \cdot \frac{S_1}{S_1 + S_2} = \frac{m}{\rho_{\text{в}}(S_1 + S_2)} = 6\,\text{см}
    \end{align*}
}

\variantsplitter

\addpersonalvariant{Вячеслав Волохов}

\tasknumber{1}%
\task{%
    Гидростатическое давление столба нефти равно $40\,\text{кПа}$.
    Определите высоту столба жидкости.
    Принять $\rho_{\text{н}} = 800\,\frac{\text{кг}}{\text{м}^{3}}$, $g = 10\,\frac{\text{м}}{\text{с}^{2}}$.
}
\answer{%
    $p = \rho_{\text{н}}gh \implies h = \frac{p}{g\rho_{\text{н}}} = \frac{{40\,\text{кПа}}}{10\,\frac{\text{м}}{\text{с}^{2}} \cdot 800\,\frac{\text{кг}}{\text{м}^{3}}} = 5\,\text{м}.$
}

\tasknumber{2}%
\task{%
    На какой глубине полное давление пресной воды превышает атмосферное в 8 раз?
    Принять $p_{\text{aтм}} = 100\,\text{кПа}$, $g = 10\,\frac{\text{м}}{\text{с}^{2}}$, $\rho_{\text{в}} = 1000\,\frac{\text{кг}}{\text{м}^{3}}$.
}
\answer{%
    $p = \rho_{\text{в}} g h + p_{\text{aтм}} = 8 p_{\text{aтм}} \implies h = \frac{(8-1) p_{\text{aтм}}}{g \rho_{\text{в}}} = \frac{(8-1) \cdot 100\,\text{кПа}}{10\,\frac{\text{м}}{\text{с}^{2}} \cdot 1000\,\frac{\text{кг}}{\text{м}^{3}}} = 70\,\text{м}.$
}

\tasknumber{3}%
\task{%
    В сосуд с вертикальными стенками и площадью горизонтального поперечного сечения $S = 0{,}010\,\text{м}^{2}$
    налили воду.
    На сколько увеличится сила давления на дно сосуда, если
    на поверхности воды ещё будет плавать тело массой $150\,\text{г}$.
    Принять $p_{\text{aтм}} = 100\,\text{кПа}$, $g = 10\,\frac{\text{м}}{\text{с}^{2}}$.
}
\answer{%
    $\Delta F = mg, \Delta p = \frac{\Delta F}{S} = \frac{mg}{S}, \Delta F = 1\,\text{Н}.$
}

\tasknumber{4}%
\task{%
    В два сообщающихся сосуда налита вода.
    В один из соcудов наливают масло так,
    что столб этой жидкости имеет высоту $50\,\text{см}$.
    На сколько теперь уровень воды в этом сосуде ниже, чем в другом?
    Ответ выразите в сантиметрах.
    $\rho_{\text{в}} = 1000\,\frac{\text{кг}}{\text{м}^{3}}$, $\rho_{\text{м}} = 900\,\frac{\text{кг}}{\text{м}^{3}}$.
}
\answer{%
    $\rho_{\text{м}}gh_1 = \rho_{\text{в}}gh_2 \implies h_2 = h_1 \frac{\rho_{\text{м}}}{\rho_{\text{в}}} = 50\,\text{см} \cdot \frac{900\,\frac{\text{кг}}{\text{м}^{3}}}{1000\,\frac{\text{кг}}{\text{м}^{3}}} = 45\,\text{см}$
}

\tasknumber{5}%
\task{%
    В два сообщающихся сосуда сечений $32\,\text{см}^{2}$ и $18\,\text{см}^{2}$ налита вода.
    Оба сосуда закрыты лёгкими поршнями и находятся в равновесии.
    На больший из поршней кладут груз массой $500\,\text{г}$.
    Определите, на сколько поднимется меньший поршень.
    Ответ выразите в сантиметрах.
    $\rho_{\text{в}} = 1000\,\frac{\text{кг}}{\text{м}^{3}}$.
}
\answer{%
    \begin{align*}
    S_1h_1 &= S_2h_2 \implies h_1 = h_2 \frac{S_2}{S_1}  \\
    \frac{mg}{S_1} &= \rho_{\text{в}}g(h_1 + h_2) = \rho_{\text{в}}g \cdot h_2 \cbr{1 + \frac{S_2}{S_1}} \\
    h_2 &= \frac{mg}{S_1\rho_{\text{в}}g} \cdot \frac{S_1}{S_1 + S_2} = \frac{m}{\rho_{\text{в}}(S_1 + S_2)} = 10\,\text{см}
    \end{align*}
}

\variantsplitter

\addpersonalvariant{Герман Говоров}

\tasknumber{1}%
\task{%
    Гидростатическое давление столба нефти равно $800\,\text{кПа}$.
    Определите высоту столба жидкости.
    Принять $\rho_{\text{н}} = 800\,\frac{\text{кг}}{\text{м}^{3}}$, $g = 10\,\frac{\text{м}}{\text{с}^{2}}$.
}
\answer{%
    $p = \rho_{\text{н}}gh \implies h = \frac{p}{g\rho_{\text{н}}} = \frac{{800\,\text{кПа}}}{10\,\frac{\text{м}}{\text{с}^{2}} \cdot 800\,\frac{\text{кг}}{\text{м}^{3}}} = 100\,\text{м}.$
}

\tasknumber{2}%
\task{%
    На какой глубине полное давление пресной воды превышает атмосферное в 9 раз?
    Принять $p_{\text{aтм}} = 100\,\text{кПа}$, $g = 10\,\frac{\text{м}}{\text{с}^{2}}$, $\rho_{\text{в}} = 1000\,\frac{\text{кг}}{\text{м}^{3}}$.
}
\answer{%
    $p = \rho_{\text{в}} g h + p_{\text{aтм}} = 9 p_{\text{aтм}} \implies h = \frac{(9-1) p_{\text{aтм}}}{g \rho_{\text{в}}} = \frac{(9-1) \cdot 100\,\text{кПа}}{10\,\frac{\text{м}}{\text{с}^{2}} \cdot 1000\,\frac{\text{кг}}{\text{м}^{3}}} = 80\,\text{м}.$
}

\tasknumber{3}%
\task{%
    В сосуд с вертикальными стенками и площадью горизонтального поперечного сечения $S = 0{,}02\,\text{м}^{2}$
    налили воду.
    На сколько увеличится давление на дно сосуда, если
    на поверхности воды ещё будет плавать тело массой $150\,\text{г}$.
    Принять $p_{\text{aтм}} = 100\,\text{кПа}$, $g = 10\,\frac{\text{м}}{\text{с}^{2}}$.
}
\answer{%
    $\Delta F = mg, \Delta p = \frac{\Delta F}{S} = \frac{mg}{S}, \Delta p = 75\,\text{Па}.$
}

\tasknumber{4}%
\task{%
    В два сообщающихся сосуда налита вода.
    В один из соcудов наливают масло так,
    что столб этой жидкости имеет высоту $40\,\text{см}$.
    На сколько теперь уровень воды в этом сосуде ниже, чем в другом?
    Ответ выразите в сантиметрах.
    $\rho_{\text{в}} = 1000\,\frac{\text{кг}}{\text{м}^{3}}$, $\rho_{\text{м}} = 900\,\frac{\text{кг}}{\text{м}^{3}}$.
}
\answer{%
    $\rho_{\text{м}}gh_1 = \rho_{\text{в}}gh_2 \implies h_2 = h_1 \frac{\rho_{\text{м}}}{\rho_{\text{в}}} = 40\,\text{см} \cdot \frac{900\,\frac{\text{кг}}{\text{м}^{3}}}{1000\,\frac{\text{кг}}{\text{м}^{3}}} = 36\,\text{см}$
}

\tasknumber{5}%
\task{%
    В два сообщающихся сосуда сечений $45\,\text{см}^{2}$ и $5\,\text{см}^{2}$ налита вода.
    Оба сосуда закрыты лёгкими поршнями и находятся в равновесии.
    На больший из поршней кладут груз массой $400\,\text{г}$.
    Определите, на сколько поднимется меньший поршень.
    Ответ выразите в сантиметрах.
    $\rho_{\text{в}} = 1000\,\frac{\text{кг}}{\text{м}^{3}}$.
}
\answer{%
    \begin{align*}
    S_1h_1 &= S_2h_2 \implies h_1 = h_2 \frac{S_2}{S_1}  \\
    \frac{mg}{S_1} &= \rho_{\text{в}}g(h_1 + h_2) = \rho_{\text{в}}g \cdot h_2 \cbr{1 + \frac{S_2}{S_1}} \\
    h_2 &= \frac{mg}{S_1\rho_{\text{в}}g} \cdot \frac{S_1}{S_1 + S_2} = \frac{m}{\rho_{\text{в}}(S_1 + S_2)} = 8\,\text{см}
    \end{align*}
}

\variantsplitter

\addpersonalvariant{София Журавлёва}

\tasknumber{1}%
\task{%
    Гидростатическое давление столба воды равно $200\,\text{кПа}$.
    Определите высоту столба жидкости.
    Принять $\rho_{\text{в}} = 1000\,\frac{\text{кг}}{\text{м}^{3}}$, $g = 10\,\frac{\text{м}}{\text{с}^{2}}$.
}
\answer{%
    $p = \rho_{\text{в}}gh \implies h = \frac{p}{g\rho_{\text{в}}} = \frac{{200\,\text{кПа}}}{10\,\frac{\text{м}}{\text{с}^{2}} \cdot 1000\,\frac{\text{кг}}{\text{м}^{3}}} = 20\,\text{м}.$
}

\tasknumber{2}%
\task{%
    На какой глубине полное давление пресной воды превышает атмосферное в 2 раз?
    Принять $p_{\text{aтм}} = 100\,\text{кПа}$, $g = 10\,\frac{\text{м}}{\text{с}^{2}}$, $\rho_{\text{в}} = 1000\,\frac{\text{кг}}{\text{м}^{3}}$.
}
\answer{%
    $p = \rho_{\text{в}} g h + p_{\text{aтм}} = 2 p_{\text{aтм}} \implies h = \frac{(2-1) p_{\text{aтм}}}{g \rho_{\text{в}}} = \frac{(2-1) \cdot 100\,\text{кПа}}{10\,\frac{\text{м}}{\text{с}^{2}} \cdot 1000\,\frac{\text{кг}}{\text{м}^{3}}} = 10\,\text{м}.$
}

\tasknumber{3}%
\task{%
    В сосуд с вертикальными стенками и площадью горизонтального поперечного сечения $S = 0{,}010\,\text{м}^{2}$
    налили воду.
    На сколько увеличится сила давления на дно сосуда, если
    на поверхности воды ещё будет плавать тело массой $600\,\text{г}$.
    Принять $p_{\text{aтм}} = 100\,\text{кПа}$, $g = 10\,\frac{\text{м}}{\text{с}^{2}}$.
}
\answer{%
    $\Delta F = mg, \Delta p = \frac{\Delta F}{S} = \frac{mg}{S}, \Delta F = 6\,\text{Н}.$
}

\tasknumber{4}%
\task{%
    В два сообщающихся сосуда налита вода.
    В один из соcудов наливают нефть так,
    что столб этой жидкости имеет высоту $10\,\text{см}$.
    На сколько теперь уровень воды в этом сосуде ниже, чем в другом?
    Ответ выразите в сантиметрах.
    $\rho_{\text{в}} = 1000\,\frac{\text{кг}}{\text{м}^{3}}$, $\rho_{\text{н}} = 800\,\frac{\text{кг}}{\text{м}^{3}}$.
}
\answer{%
    $\rho_{\text{н}}gh_1 = \rho_{\text{в}}gh_2 \implies h_2 = h_1 \frac{\rho_{\text{н}}}{\rho_{\text{в}}} = 10\,\text{см} \cdot \frac{800\,\frac{\text{кг}}{\text{м}^{3}}}{1000\,\frac{\text{кг}}{\text{м}^{3}}} = 8\,\text{см}$
}

\tasknumber{5}%
\task{%
    В два сообщающихся сосуда сечений $30\,\text{см}^{2}$ и $20\,\text{см}^{2}$ налита вода.
    Оба сосуда закрыты лёгкими поршнями и находятся в равновесии.
    На больший из поршней кладут груз массой $200\,\text{г}$.
    Определите, на сколько поднимется меньший поршень.
    Ответ выразите в сантиметрах.
    $\rho_{\text{в}} = 1000\,\frac{\text{кг}}{\text{м}^{3}}$.
}
\answer{%
    \begin{align*}
    S_1h_1 &= S_2h_2 \implies h_1 = h_2 \frac{S_2}{S_1}  \\
    \frac{mg}{S_1} &= \rho_{\text{в}}g(h_1 + h_2) = \rho_{\text{в}}g \cdot h_2 \cbr{1 + \frac{S_2}{S_1}} \\
    h_2 &= \frac{mg}{S_1\rho_{\text{в}}g} \cdot \frac{S_1}{S_1 + S_2} = \frac{m}{\rho_{\text{в}}(S_1 + S_2)} = 4\,\text{см}
    \end{align*}
}

\variantsplitter

\addpersonalvariant{Константин Козлов}

\tasknumber{1}%
\task{%
    Гидростатическое давление столба масла равно $9\,\text{кПа}$.
    Определите высоту столба жидкости.
    Принять $\rho_{\text{м}} = 900\,\frac{\text{кг}}{\text{м}^{3}}$, $g = 10\,\frac{\text{м}}{\text{с}^{2}}$.
}
\answer{%
    $p = \rho_{\text{м}}gh \implies h = \frac{p}{g\rho_{\text{м}}} = \frac{{9\,\text{кПа}}}{10\,\frac{\text{м}}{\text{с}^{2}} \cdot 900\,\frac{\text{кг}}{\text{м}^{3}}} = 1\,\text{м}.$
}

\tasknumber{2}%
\task{%
    На какой глубине полное давление пресной воды превышает атмосферное в 6 раз?
    Принять $p_{\text{aтм}} = 100\,\text{кПа}$, $g = 10\,\frac{\text{м}}{\text{с}^{2}}$, $\rho_{\text{в}} = 1000\,\frac{\text{кг}}{\text{м}^{3}}$.
}
\answer{%
    $p = \rho_{\text{в}} g h + p_{\text{aтм}} = 6 p_{\text{aтм}} \implies h = \frac{(6-1) p_{\text{aтм}}}{g \rho_{\text{в}}} = \frac{(6-1) \cdot 100\,\text{кПа}}{10\,\frac{\text{м}}{\text{с}^{2}} \cdot 1000\,\frac{\text{кг}}{\text{м}^{3}}} = 50\,\text{м}.$
}

\tasknumber{3}%
\task{%
    В сосуд с вертикальными стенками и площадью горизонтального поперечного сечения $S = 0{,}05\,\text{м}^{2}$
    налили воду.
    На сколько увеличится давление на дно сосуда, если
    на поверхности воды ещё будет плавать тело массой $300\,\text{г}$.
    Принять $p_{\text{aтм}} = 100\,\text{кПа}$, $g = 10\,\frac{\text{м}}{\text{с}^{2}}$.
}
\answer{%
    $\Delta F = mg, \Delta p = \frac{\Delta F}{S} = \frac{mg}{S}, \Delta p = 60\,\text{Па}.$
}

\tasknumber{4}%
\task{%
    В два сообщающихся сосуда налита вода.
    В один из соcудов наливают масло так,
    что столб этой жидкости имеет высоту $40\,\text{см}$.
    На сколько теперь уровень воды в этом сосуде ниже, чем в другом?
    Ответ выразите в сантиметрах.
    $\rho_{\text{в}} = 1000\,\frac{\text{кг}}{\text{м}^{3}}$, $\rho_{\text{м}} = 900\,\frac{\text{кг}}{\text{м}^{3}}$.
}
\answer{%
    $\rho_{\text{м}}gh_1 = \rho_{\text{в}}gh_2 \implies h_2 = h_1 \frac{\rho_{\text{м}}}{\rho_{\text{в}}} = 40\,\text{см} \cdot \frac{900\,\frac{\text{кг}}{\text{м}^{3}}}{1000\,\frac{\text{кг}}{\text{м}^{3}}} = 36\,\text{см}$
}

\tasknumber{5}%
\task{%
    В два сообщающихся сосуда сечений $45\,\text{см}^{2}$ и $5\,\text{см}^{2}$ налита вода.
    Оба сосуда закрыты лёгкими поршнями и находятся в равновесии.
    На больший из поршней кладут груз массой $200\,\text{г}$.
    Определите, на сколько поднимется меньший поршень.
    Ответ выразите в сантиметрах.
    $\rho_{\text{в}} = 1000\,\frac{\text{кг}}{\text{м}^{3}}$.
}
\answer{%
    \begin{align*}
    S_1h_1 &= S_2h_2 \implies h_1 = h_2 \frac{S_2}{S_1}  \\
    \frac{mg}{S_1} &= \rho_{\text{в}}g(h_1 + h_2) = \rho_{\text{в}}g \cdot h_2 \cbr{1 + \frac{S_2}{S_1}} \\
    h_2 &= \frac{mg}{S_1\rho_{\text{в}}g} \cdot \frac{S_1}{S_1 + S_2} = \frac{m}{\rho_{\text{в}}(S_1 + S_2)} = 4\,\text{см}
    \end{align*}
}

\variantsplitter

\addpersonalvariant{Наталья Кравченко}

\tasknumber{1}%
\task{%
    Гидростатическое давление столба воды равно $150\,\text{кПа}$.
    Определите высоту столба жидкости.
    Принять $\rho_{\text{в}} = 1000\,\frac{\text{кг}}{\text{м}^{3}}$, $g = 10\,\frac{\text{м}}{\text{с}^{2}}$.
}
\answer{%
    $p = \rho_{\text{в}}gh \implies h = \frac{p}{g\rho_{\text{в}}} = \frac{{150\,\text{кПа}}}{10\,\frac{\text{м}}{\text{с}^{2}} \cdot 1000\,\frac{\text{кг}}{\text{м}^{3}}} = 15\,\text{м}.$
}

\tasknumber{2}%
\task{%
    На какой глубине полное давление пресной воды превышает атмосферное в 3 раз?
    Принять $p_{\text{aтм}} = 100\,\text{кПа}$, $g = 10\,\frac{\text{м}}{\text{с}^{2}}$, $\rho_{\text{в}} = 1000\,\frac{\text{кг}}{\text{м}^{3}}$.
}
\answer{%
    $p = \rho_{\text{в}} g h + p_{\text{aтм}} = 3 p_{\text{aтм}} \implies h = \frac{(3-1) p_{\text{aтм}}}{g \rho_{\text{в}}} = \frac{(3-1) \cdot 100\,\text{кПа}}{10\,\frac{\text{м}}{\text{с}^{2}} \cdot 1000\,\frac{\text{кг}}{\text{м}^{3}}} = 20\,\text{м}.$
}

\tasknumber{3}%
\task{%
    В сосуд с вертикальными стенками и площадью горизонтального поперечного сечения $S = 0{,}010\,\text{м}^{2}$
    налили воду.
    На сколько увеличится давление на дно сосуда, если
    на поверхности воды ещё будет плавать тело массой $600\,\text{г}$.
    Принять $p_{\text{aтм}} = 100\,\text{кПа}$, $g = 10\,\frac{\text{м}}{\text{с}^{2}}$.
}
\answer{%
    $\Delta F = mg, \Delta p = \frac{\Delta F}{S} = \frac{mg}{S}, \Delta p = 600\,\text{Па}.$
}

\tasknumber{4}%
\task{%
    В два сообщающихся сосуда налита вода.
    В один из соcудов наливают масло так,
    что столб этой жидкости имеет высоту $90\,\text{см}$.
    На сколько теперь уровень воды в этом сосуде ниже, чем в другом?
    Ответ выразите в сантиметрах.
    $\rho_{\text{в}} = 1000\,\frac{\text{кг}}{\text{м}^{3}}$, $\rho_{\text{м}} = 900\,\frac{\text{кг}}{\text{м}^{3}}$.
}
\answer{%
    $\rho_{\text{м}}gh_1 = \rho_{\text{в}}gh_2 \implies h_2 = h_1 \frac{\rho_{\text{м}}}{\rho_{\text{в}}} = 90\,\text{см} \cdot \frac{900\,\frac{\text{кг}}{\text{м}^{3}}}{1000\,\frac{\text{кг}}{\text{м}^{3}}} = 81\,\text{см}$
}

\tasknumber{5}%
\task{%
    В два сообщающихся сосуда сечений $45\,\text{см}^{2}$ и $5\,\text{см}^{2}$ налита вода.
    Оба сосуда закрыты лёгкими поршнями и находятся в равновесии.
    На больший из поршней кладут груз массой $500\,\text{г}$.
    Определите, на сколько поднимется меньший поршень.
    Ответ выразите в сантиметрах.
    $\rho_{\text{в}} = 1000\,\frac{\text{кг}}{\text{м}^{3}}$.
}
\answer{%
    \begin{align*}
    S_1h_1 &= S_2h_2 \implies h_1 = h_2 \frac{S_2}{S_1}  \\
    \frac{mg}{S_1} &= \rho_{\text{в}}g(h_1 + h_2) = \rho_{\text{в}}g \cdot h_2 \cbr{1 + \frac{S_2}{S_1}} \\
    h_2 &= \frac{mg}{S_1\rho_{\text{в}}g} \cdot \frac{S_1}{S_1 + S_2} = \frac{m}{\rho_{\text{в}}(S_1 + S_2)} = 10\,\text{см}
    \end{align*}
}

\variantsplitter

\addpersonalvariant{Матвей Кузьмин}

\tasknumber{1}%
\task{%
    Гидростатическое давление столба воды равно $50\,\text{кПа}$.
    Определите высоту столба жидкости.
    Принять $\rho_{\text{в}} = 1000\,\frac{\text{кг}}{\text{м}^{3}}$, $g = 10\,\frac{\text{м}}{\text{с}^{2}}$.
}
\answer{%
    $p = \rho_{\text{в}}gh \implies h = \frac{p}{g\rho_{\text{в}}} = \frac{{50\,\text{кПа}}}{10\,\frac{\text{м}}{\text{с}^{2}} \cdot 1000\,\frac{\text{кг}}{\text{м}^{3}}} = 5\,\text{м}.$
}

\tasknumber{2}%
\task{%
    На какой глубине полное давление пресной воды превышает атмосферное в 6 раз?
    Принять $p_{\text{aтм}} = 100\,\text{кПа}$, $g = 10\,\frac{\text{м}}{\text{с}^{2}}$, $\rho_{\text{в}} = 1000\,\frac{\text{кг}}{\text{м}^{3}}$.
}
\answer{%
    $p = \rho_{\text{в}} g h + p_{\text{aтм}} = 6 p_{\text{aтм}} \implies h = \frac{(6-1) p_{\text{aтм}}}{g \rho_{\text{в}}} = \frac{(6-1) \cdot 100\,\text{кПа}}{10\,\frac{\text{м}}{\text{с}^{2}} \cdot 1000\,\frac{\text{кг}}{\text{м}^{3}}} = 50\,\text{м}.$
}

\tasknumber{3}%
\task{%
    В сосуд с вертикальными стенками и площадью горизонтального поперечного сечения $S = 0{,}05\,\text{м}^{2}$
    налили воду.
    На сколько увеличится сила давления на дно сосуда, если
    на поверхности воды ещё будет плавать тело массой $150\,\text{г}$.
    Принять $p_{\text{aтм}} = 100\,\text{кПа}$, $g = 10\,\frac{\text{м}}{\text{с}^{2}}$.
}
\answer{%
    $\Delta F = mg, \Delta p = \frac{\Delta F}{S} = \frac{mg}{S}, \Delta F = 1\,\text{Н}.$
}

\tasknumber{4}%
\task{%
    В два сообщающихся сосуда налита вода.
    В один из соcудов наливают нефть так,
    что столб этой жидкости имеет высоту $20\,\text{см}$.
    На сколько теперь уровень воды в этом сосуде ниже, чем в другом?
    Ответ выразите в сантиметрах.
    $\rho_{\text{в}} = 1000\,\frac{\text{кг}}{\text{м}^{3}}$, $\rho_{\text{н}} = 800\,\frac{\text{кг}}{\text{м}^{3}}$.
}
\answer{%
    $\rho_{\text{н}}gh_1 = \rho_{\text{в}}gh_2 \implies h_2 = h_1 \frac{\rho_{\text{н}}}{\rho_{\text{в}}} = 20\,\text{см} \cdot \frac{800\,\frac{\text{кг}}{\text{м}^{3}}}{1000\,\frac{\text{кг}}{\text{м}^{3}}} = 16\,\text{см}$
}

\tasknumber{5}%
\task{%
    В два сообщающихся сосуда сечений $64\,\text{см}^{2}$ и $36\,\text{см}^{2}$ налита вода.
    Оба сосуда закрыты лёгкими поршнями и находятся в равновесии.
    На больший из поршней кладут груз массой $200\,\text{г}$.
    Определите, на сколько поднимется меньший поршень.
    Ответ выразите в сантиметрах.
    $\rho_{\text{в}} = 1000\,\frac{\text{кг}}{\text{м}^{3}}$.
}
\answer{%
    \begin{align*}
    S_1h_1 &= S_2h_2 \implies h_1 = h_2 \frac{S_2}{S_1}  \\
    \frac{mg}{S_1} &= \rho_{\text{в}}g(h_1 + h_2) = \rho_{\text{в}}g \cdot h_2 \cbr{1 + \frac{S_2}{S_1}} \\
    h_2 &= \frac{mg}{S_1\rho_{\text{в}}g} \cdot \frac{S_1}{S_1 + S_2} = \frac{m}{\rho_{\text{в}}(S_1 + S_2)} = 2\,\text{см}
    \end{align*}
}

\variantsplitter

\addpersonalvariant{Сергей Малышев}

\tasknumber{1}%
\task{%
    Гидростатическое давление столба воды равно $200\,\text{кПа}$.
    Определите высоту столба жидкости.
    Принять $\rho_{\text{в}} = 1000\,\frac{\text{кг}}{\text{м}^{3}}$, $g = 10\,\frac{\text{м}}{\text{с}^{2}}$.
}
\answer{%
    $p = \rho_{\text{в}}gh \implies h = \frac{p}{g\rho_{\text{в}}} = \frac{{200\,\text{кПа}}}{10\,\frac{\text{м}}{\text{с}^{2}} \cdot 1000\,\frac{\text{кг}}{\text{м}^{3}}} = 20\,\text{м}.$
}

\tasknumber{2}%
\task{%
    На какой глубине полное давление пресной воды превышает атмосферное в 9 раз?
    Принять $p_{\text{aтм}} = 100\,\text{кПа}$, $g = 10\,\frac{\text{м}}{\text{с}^{2}}$, $\rho_{\text{в}} = 1000\,\frac{\text{кг}}{\text{м}^{3}}$.
}
\answer{%
    $p = \rho_{\text{в}} g h + p_{\text{aтм}} = 9 p_{\text{aтм}} \implies h = \frac{(9-1) p_{\text{aтм}}}{g \rho_{\text{в}}} = \frac{(9-1) \cdot 100\,\text{кПа}}{10\,\frac{\text{м}}{\text{с}^{2}} \cdot 1000\,\frac{\text{кг}}{\text{м}^{3}}} = 80\,\text{м}.$
}

\tasknumber{3}%
\task{%
    В сосуд с вертикальными стенками и площадью горизонтального поперечного сечения $S = 0{,}010\,\text{м}^{2}$
    налили воду.
    На сколько увеличится сила давления на дно сосуда, если
    на поверхности воды ещё будет плавать тело массой $600\,\text{г}$.
    Принять $p_{\text{aтм}} = 100\,\text{кПа}$, $g = 10\,\frac{\text{м}}{\text{с}^{2}}$.
}
\answer{%
    $\Delta F = mg, \Delta p = \frac{\Delta F}{S} = \frac{mg}{S}, \Delta F = 6\,\text{Н}.$
}

\tasknumber{4}%
\task{%
    В два сообщающихся сосуда налита вода.
    В один из соcудов наливают нефть так,
    что столб этой жидкости имеет высоту $10\,\text{см}$.
    На сколько теперь уровень воды в этом сосуде ниже, чем в другом?
    Ответ выразите в сантиметрах.
    $\rho_{\text{в}} = 1000\,\frac{\text{кг}}{\text{м}^{3}}$, $\rho_{\text{н}} = 800\,\frac{\text{кг}}{\text{м}^{3}}$.
}
\answer{%
    $\rho_{\text{н}}gh_1 = \rho_{\text{в}}gh_2 \implies h_2 = h_1 \frac{\rho_{\text{н}}}{\rho_{\text{в}}} = 10\,\text{см} \cdot \frac{800\,\frac{\text{кг}}{\text{м}^{3}}}{1000\,\frac{\text{кг}}{\text{м}^{3}}} = 8\,\text{см}$
}

\tasknumber{5}%
\task{%
    В два сообщающихся сосуда сечений $80\,\text{см}^{2}$ и $20\,\text{см}^{2}$ налита вода.
    Оба сосуда закрыты лёгкими поршнями и находятся в равновесии.
    На больший из поршней кладут груз массой $500\,\text{г}$.
    Определите, на сколько поднимется меньший поршень.
    Ответ выразите в сантиметрах.
    $\rho_{\text{в}} = 1000\,\frac{\text{кг}}{\text{м}^{3}}$.
}
\answer{%
    \begin{align*}
    S_1h_1 &= S_2h_2 \implies h_1 = h_2 \frac{S_2}{S_1}  \\
    \frac{mg}{S_1} &= \rho_{\text{в}}g(h_1 + h_2) = \rho_{\text{в}}g \cdot h_2 \cbr{1 + \frac{S_2}{S_1}} \\
    h_2 &= \frac{mg}{S_1\rho_{\text{в}}g} \cdot \frac{S_1}{S_1 + S_2} = \frac{m}{\rho_{\text{в}}(S_1 + S_2)} = 5\,\text{см}
    \end{align*}
}

\variantsplitter

\addpersonalvariant{Алина Полканова}

\tasknumber{1}%
\task{%
    Гидростатическое давление столба воды равно $50\,\text{кПа}$.
    Определите высоту столба жидкости.
    Принять $\rho_{\text{в}} = 1000\,\frac{\text{кг}}{\text{м}^{3}}$, $g = 10\,\frac{\text{м}}{\text{с}^{2}}$.
}
\answer{%
    $p = \rho_{\text{в}}gh \implies h = \frac{p}{g\rho_{\text{в}}} = \frac{{50\,\text{кПа}}}{10\,\frac{\text{м}}{\text{с}^{2}} \cdot 1000\,\frac{\text{кг}}{\text{м}^{3}}} = 5\,\text{м}.$
}

\tasknumber{2}%
\task{%
    На какой глубине полное давление пресной воды превышает атмосферное в 8 раз?
    Принять $p_{\text{aтм}} = 100\,\text{кПа}$, $g = 10\,\frac{\text{м}}{\text{с}^{2}}$, $\rho_{\text{в}} = 1000\,\frac{\text{кг}}{\text{м}^{3}}$.
}
\answer{%
    $p = \rho_{\text{в}} g h + p_{\text{aтм}} = 8 p_{\text{aтм}} \implies h = \frac{(8-1) p_{\text{aтм}}}{g \rho_{\text{в}}} = \frac{(8-1) \cdot 100\,\text{кПа}}{10\,\frac{\text{м}}{\text{с}^{2}} \cdot 1000\,\frac{\text{кг}}{\text{м}^{3}}} = 70\,\text{м}.$
}

\tasknumber{3}%
\task{%
    В сосуд с вертикальными стенками и площадью горизонтального поперечного сечения $S = 0{,}02\,\text{м}^{2}$
    налили воду.
    На сколько увеличится сила давления на дно сосуда, если
    на поверхности воды ещё будет плавать тело массой $300\,\text{г}$.
    Принять $p_{\text{aтм}} = 100\,\text{кПа}$, $g = 10\,\frac{\text{м}}{\text{с}^{2}}$.
}
\answer{%
    $\Delta F = mg, \Delta p = \frac{\Delta F}{S} = \frac{mg}{S}, \Delta F = 3\,\text{Н}.$
}

\tasknumber{4}%
\task{%
    В два сообщающихся сосуда налита вода.
    В один из соcудов наливают нефть так,
    что столб этой жидкости имеет высоту $40\,\text{см}$.
    На сколько теперь уровень воды в этом сосуде ниже, чем в другом?
    Ответ выразите в сантиметрах.
    $\rho_{\text{в}} = 1000\,\frac{\text{кг}}{\text{м}^{3}}$, $\rho_{\text{н}} = 800\,\frac{\text{кг}}{\text{м}^{3}}$.
}
\answer{%
    $\rho_{\text{н}}gh_1 = \rho_{\text{в}}gh_2 \implies h_2 = h_1 \frac{\rho_{\text{н}}}{\rho_{\text{в}}} = 40\,\text{см} \cdot \frac{800\,\frac{\text{кг}}{\text{м}^{3}}}{1000\,\frac{\text{кг}}{\text{м}^{3}}} = 32\,\text{см}$
}

\tasknumber{5}%
\task{%
    В два сообщающихся сосуда сечений $80\,\text{см}^{2}$ и $20\,\text{см}^{2}$ налита вода.
    Оба сосуда закрыты лёгкими поршнями и находятся в равновесии.
    На больший из поршней кладут груз массой $300\,\text{г}$.
    Определите, на сколько поднимется меньший поршень.
    Ответ выразите в сантиметрах.
    $\rho_{\text{в}} = 1000\,\frac{\text{кг}}{\text{м}^{3}}$.
}
\answer{%
    \begin{align*}
    S_1h_1 &= S_2h_2 \implies h_1 = h_2 \frac{S_2}{S_1}  \\
    \frac{mg}{S_1} &= \rho_{\text{в}}g(h_1 + h_2) = \rho_{\text{в}}g \cdot h_2 \cbr{1 + \frac{S_2}{S_1}} \\
    h_2 &= \frac{mg}{S_1\rho_{\text{в}}g} \cdot \frac{S_1}{S_1 + S_2} = \frac{m}{\rho_{\text{в}}(S_1 + S_2)} = 3\,\text{см}
    \end{align*}
}

\variantsplitter

\addpersonalvariant{Сергей Пономарёв}

\tasknumber{1}%
\task{%
    Гидростатическое давление столба масла равно $18\,\text{кПа}$.
    Определите высоту столба жидкости.
    Принять $\rho_{\text{м}} = 900\,\frac{\text{кг}}{\text{м}^{3}}$, $g = 10\,\frac{\text{м}}{\text{с}^{2}}$.
}
\answer{%
    $p = \rho_{\text{м}}gh \implies h = \frac{p}{g\rho_{\text{м}}} = \frac{{18\,\text{кПа}}}{10\,\frac{\text{м}}{\text{с}^{2}} \cdot 900\,\frac{\text{кг}}{\text{м}^{3}}} = 2\,\text{м}.$
}

\tasknumber{2}%
\task{%
    На какой глубине полное давление пресной воды превышает атмосферное в 10 раз?
    Принять $p_{\text{aтм}} = 100\,\text{кПа}$, $g = 10\,\frac{\text{м}}{\text{с}^{2}}$, $\rho_{\text{в}} = 1000\,\frac{\text{кг}}{\text{м}^{3}}$.
}
\answer{%
    $p = \rho_{\text{в}} g h + p_{\text{aтм}} = 10 p_{\text{aтм}} \implies h = \frac{(10-1) p_{\text{aтм}}}{g \rho_{\text{в}}} = \frac{(10-1) \cdot 100\,\text{кПа}}{10\,\frac{\text{м}}{\text{с}^{2}} \cdot 1000\,\frac{\text{кг}}{\text{м}^{3}}} = 90\,\text{м}.$
}

\tasknumber{3}%
\task{%
    В сосуд с вертикальными стенками и площадью горизонтального поперечного сечения $S = 0{,}03\,\text{м}^{2}$
    налили воду.
    На сколько увеличится давление на дно сосуда, если
    на поверхности воды ещё будет плавать тело массой $150\,\text{г}$.
    Принять $p_{\text{aтм}} = 100\,\text{кПа}$, $g = 10\,\frac{\text{м}}{\text{с}^{2}}$.
}
\answer{%
    $\Delta F = mg, \Delta p = \frac{\Delta F}{S} = \frac{mg}{S}, \Delta p = 50\,\text{Па}.$
}

\tasknumber{4}%
\task{%
    В два сообщающихся сосуда налита вода.
    В один из соcудов наливают нефть так,
    что столб этой жидкости имеет высоту $30\,\text{см}$.
    На сколько теперь уровень воды в этом сосуде ниже, чем в другом?
    Ответ выразите в сантиметрах.
    $\rho_{\text{в}} = 1000\,\frac{\text{кг}}{\text{м}^{3}}$, $\rho_{\text{н}} = 800\,\frac{\text{кг}}{\text{м}^{3}}$.
}
\answer{%
    $\rho_{\text{н}}gh_1 = \rho_{\text{в}}gh_2 \implies h_2 = h_1 \frac{\rho_{\text{н}}}{\rho_{\text{в}}} = 30\,\text{см} \cdot \frac{800\,\frac{\text{кг}}{\text{м}^{3}}}{1000\,\frac{\text{кг}}{\text{м}^{3}}} = 24\,\text{см}$
}

\tasknumber{5}%
\task{%
    В два сообщающихся сосуда сечений $40\,\text{см}^{2}$ и $10\,\text{см}^{2}$ налита вода.
    Оба сосуда закрыты лёгкими поршнями и находятся в равновесии.
    На больший из поршней кладут груз массой $200\,\text{г}$.
    Определите, на сколько поднимется меньший поршень.
    Ответ выразите в сантиметрах.
    $\rho_{\text{в}} = 1000\,\frac{\text{кг}}{\text{м}^{3}}$.
}
\answer{%
    \begin{align*}
    S_1h_1 &= S_2h_2 \implies h_1 = h_2 \frac{S_2}{S_1}  \\
    \frac{mg}{S_1} &= \rho_{\text{в}}g(h_1 + h_2) = \rho_{\text{в}}g \cdot h_2 \cbr{1 + \frac{S_2}{S_1}} \\
    h_2 &= \frac{mg}{S_1\rho_{\text{в}}g} \cdot \frac{S_1}{S_1 + S_2} = \frac{m}{\rho_{\text{в}}(S_1 + S_2)} = 4\,\text{см}
    \end{align*}
}

\variantsplitter

\addpersonalvariant{Егор Свистушкин}

\tasknumber{1}%
\task{%
    Гидростатическое давление столба масла равно $27\,\text{кПа}$.
    Определите высоту столба жидкости.
    Принять $\rho_{\text{м}} = 900\,\frac{\text{кг}}{\text{м}^{3}}$, $g = 10\,\frac{\text{м}}{\text{с}^{2}}$.
}
\answer{%
    $p = \rho_{\text{м}}gh \implies h = \frac{p}{g\rho_{\text{м}}} = \frac{{27\,\text{кПа}}}{10\,\frac{\text{м}}{\text{с}^{2}} \cdot 900\,\frac{\text{кг}}{\text{м}^{3}}} = 3\,\text{м}.$
}

\tasknumber{2}%
\task{%
    На какой глубине полное давление пресной воды превышает атмосферное в 2 раз?
    Принять $p_{\text{aтм}} = 100\,\text{кПа}$, $g = 10\,\frac{\text{м}}{\text{с}^{2}}$, $\rho_{\text{в}} = 1000\,\frac{\text{кг}}{\text{м}^{3}}$.
}
\answer{%
    $p = \rho_{\text{в}} g h + p_{\text{aтм}} = 2 p_{\text{aтм}} \implies h = \frac{(2-1) p_{\text{aтм}}}{g \rho_{\text{в}}} = \frac{(2-1) \cdot 100\,\text{кПа}}{10\,\frac{\text{м}}{\text{с}^{2}} \cdot 1000\,\frac{\text{кг}}{\text{м}^{3}}} = 10\,\text{м}.$
}

\tasknumber{3}%
\task{%
    В сосуд с вертикальными стенками и площадью горизонтального поперечного сечения $S = 0{,}05\,\text{м}^{2}$
    налили воду.
    На сколько увеличится давление на дно сосуда, если
    на поверхности воды ещё будет плавать тело массой $600\,\text{г}$.
    Принять $p_{\text{aтм}} = 100\,\text{кПа}$, $g = 10\,\frac{\text{м}}{\text{с}^{2}}$.
}
\answer{%
    $\Delta F = mg, \Delta p = \frac{\Delta F}{S} = \frac{mg}{S}, \Delta p = 120\,\text{Па}.$
}

\tasknumber{4}%
\task{%
    В два сообщающихся сосуда налита вода.
    В один из соcудов наливают нефть так,
    что столб этой жидкости имеет высоту $5\,\text{см}$.
    На сколько теперь уровень воды в этом сосуде ниже, чем в другом?
    Ответ выразите в сантиметрах.
    $\rho_{\text{в}} = 1000\,\frac{\text{кг}}{\text{м}^{3}}$, $\rho_{\text{н}} = 800\,\frac{\text{кг}}{\text{м}^{3}}$.
}
\answer{%
    $\rho_{\text{н}}gh_1 = \rho_{\text{в}}gh_2 \implies h_2 = h_1 \frac{\rho_{\text{н}}}{\rho_{\text{в}}} = 5\,\text{см} \cdot \frac{800\,\frac{\text{кг}}{\text{м}^{3}}}{1000\,\frac{\text{кг}}{\text{м}^{3}}} = 4\,\text{см}$
}

\tasknumber{5}%
\task{%
    В два сообщающихся сосуда сечений $80\,\text{см}^{2}$ и $20\,\text{см}^{2}$ налита вода.
    Оба сосуда закрыты лёгкими поршнями и находятся в равновесии.
    На больший из поршней кладут груз массой $300\,\text{г}$.
    Определите, на сколько поднимется меньший поршень.
    Ответ выразите в сантиметрах.
    $\rho_{\text{в}} = 1000\,\frac{\text{кг}}{\text{м}^{3}}$.
}
\answer{%
    \begin{align*}
    S_1h_1 &= S_2h_2 \implies h_1 = h_2 \frac{S_2}{S_1}  \\
    \frac{mg}{S_1} &= \rho_{\text{в}}g(h_1 + h_2) = \rho_{\text{в}}g \cdot h_2 \cbr{1 + \frac{S_2}{S_1}} \\
    h_2 &= \frac{mg}{S_1\rho_{\text{в}}g} \cdot \frac{S_1}{S_1 + S_2} = \frac{m}{\rho_{\text{в}}(S_1 + S_2)} = 3\,\text{см}
    \end{align*}
}

\variantsplitter

\addpersonalvariant{Дмитрий Соколов}

\tasknumber{1}%
\task{%
    Гидростатическое давление столба нефти равно $600\,\text{кПа}$.
    Определите высоту столба жидкости.
    Принять $\rho_{\text{н}} = 800\,\frac{\text{кг}}{\text{м}^{3}}$, $g = 10\,\frac{\text{м}}{\text{с}^{2}}$.
}
\answer{%
    $p = \rho_{\text{н}}gh \implies h = \frac{p}{g\rho_{\text{н}}} = \frac{{600\,\text{кПа}}}{10\,\frac{\text{м}}{\text{с}^{2}} \cdot 800\,\frac{\text{кг}}{\text{м}^{3}}} = 75\,\text{м}.$
}

\tasknumber{2}%
\task{%
    На какой глубине полное давление пресной воды превышает атмосферное в 2 раз?
    Принять $p_{\text{aтм}} = 100\,\text{кПа}$, $g = 10\,\frac{\text{м}}{\text{с}^{2}}$, $\rho_{\text{в}} = 1000\,\frac{\text{кг}}{\text{м}^{3}}$.
}
\answer{%
    $p = \rho_{\text{в}} g h + p_{\text{aтм}} = 2 p_{\text{aтм}} \implies h = \frac{(2-1) p_{\text{aтм}}}{g \rho_{\text{в}}} = \frac{(2-1) \cdot 100\,\text{кПа}}{10\,\frac{\text{м}}{\text{с}^{2}} \cdot 1000\,\frac{\text{кг}}{\text{м}^{3}}} = 10\,\text{м}.$
}

\tasknumber{3}%
\task{%
    В сосуд с вертикальными стенками и площадью горизонтального поперечного сечения $S = 0{,}03\,\text{м}^{2}$
    налили воду.
    На сколько увеличится давление на дно сосуда, если
    на поверхности воды ещё будет плавать тело массой $150\,\text{г}$.
    Принять $p_{\text{aтм}} = 100\,\text{кПа}$, $g = 10\,\frac{\text{м}}{\text{с}^{2}}$.
}
\answer{%
    $\Delta F = mg, \Delta p = \frac{\Delta F}{S} = \frac{mg}{S}, \Delta p = 50\,\text{Па}.$
}

\tasknumber{4}%
\task{%
    В два сообщающихся сосуда налита вода.
    В один из соcудов наливают масло так,
    что столб этой жидкости имеет высоту $90\,\text{см}$.
    На сколько теперь уровень воды в этом сосуде ниже, чем в другом?
    Ответ выразите в сантиметрах.
    $\rho_{\text{в}} = 1000\,\frac{\text{кг}}{\text{м}^{3}}$, $\rho_{\text{м}} = 900\,\frac{\text{кг}}{\text{м}^{3}}$.
}
\answer{%
    $\rho_{\text{м}}gh_1 = \rho_{\text{в}}gh_2 \implies h_2 = h_1 \frac{\rho_{\text{м}}}{\rho_{\text{в}}} = 90\,\text{см} \cdot \frac{900\,\frac{\text{кг}}{\text{м}^{3}}}{1000\,\frac{\text{кг}}{\text{м}^{3}}} = 81\,\text{см}$
}

\tasknumber{5}%
\task{%
    В два сообщающихся сосуда сечений $30\,\text{см}^{2}$ и $20\,\text{см}^{2}$ налита вода.
    Оба сосуда закрыты лёгкими поршнями и находятся в равновесии.
    На больший из поршней кладут груз массой $400\,\text{г}$.
    Определите, на сколько поднимется меньший поршень.
    Ответ выразите в сантиметрах.
    $\rho_{\text{в}} = 1000\,\frac{\text{кг}}{\text{м}^{3}}$.
}
\answer{%
    \begin{align*}
    S_1h_1 &= S_2h_2 \implies h_1 = h_2 \frac{S_2}{S_1}  \\
    \frac{mg}{S_1} &= \rho_{\text{в}}g(h_1 + h_2) = \rho_{\text{в}}g \cdot h_2 \cbr{1 + \frac{S_2}{S_1}} \\
    h_2 &= \frac{mg}{S_1\rho_{\text{в}}g} \cdot \frac{S_1}{S_1 + S_2} = \frac{m}{\rho_{\text{в}}(S_1 + S_2)} = 8\,\text{см}
    \end{align*}
}

\variantsplitter

\addpersonalvariant{Арсений Трофимов}

\tasknumber{1}%
\task{%
    Гидростатическое давление столба масла равно $9\,\text{кПа}$.
    Определите высоту столба жидкости.
    Принять $\rho_{\text{м}} = 900\,\frac{\text{кг}}{\text{м}^{3}}$, $g = 10\,\frac{\text{м}}{\text{с}^{2}}$.
}
\answer{%
    $p = \rho_{\text{м}}gh \implies h = \frac{p}{g\rho_{\text{м}}} = \frac{{9\,\text{кПа}}}{10\,\frac{\text{м}}{\text{с}^{2}} \cdot 900\,\frac{\text{кг}}{\text{м}^{3}}} = 1\,\text{м}.$
}

\tasknumber{2}%
\task{%
    На какой глубине полное давление пресной воды превышает атмосферное в 6 раз?
    Принять $p_{\text{aтм}} = 100\,\text{кПа}$, $g = 10\,\frac{\text{м}}{\text{с}^{2}}$, $\rho_{\text{в}} = 1000\,\frac{\text{кг}}{\text{м}^{3}}$.
}
\answer{%
    $p = \rho_{\text{в}} g h + p_{\text{aтм}} = 6 p_{\text{aтм}} \implies h = \frac{(6-1) p_{\text{aтм}}}{g \rho_{\text{в}}} = \frac{(6-1) \cdot 100\,\text{кПа}}{10\,\frac{\text{м}}{\text{с}^{2}} \cdot 1000\,\frac{\text{кг}}{\text{м}^{3}}} = 50\,\text{м}.$
}

\tasknumber{3}%
\task{%
    В сосуд с вертикальными стенками и площадью горизонтального поперечного сечения $S = 0{,}05\,\text{м}^{2}$
    налили воду.
    На сколько увеличится сила давления на дно сосуда, если
    на поверхности воды ещё будет плавать тело массой $300\,\text{г}$.
    Принять $p_{\text{aтм}} = 100\,\text{кПа}$, $g = 10\,\frac{\text{м}}{\text{с}^{2}}$.
}
\answer{%
    $\Delta F = mg, \Delta p = \frac{\Delta F}{S} = \frac{mg}{S}, \Delta F = 3\,\text{Н}.$
}

\tasknumber{4}%
\task{%
    В два сообщающихся сосуда налита вода.
    В один из соcудов наливают масло так,
    что столб этой жидкости имеет высоту $80\,\text{см}$.
    На сколько теперь уровень воды в этом сосуде ниже, чем в другом?
    Ответ выразите в сантиметрах.
    $\rho_{\text{в}} = 1000\,\frac{\text{кг}}{\text{м}^{3}}$, $\rho_{\text{м}} = 900\,\frac{\text{кг}}{\text{м}^{3}}$.
}
\answer{%
    $\rho_{\text{м}}gh_1 = \rho_{\text{в}}gh_2 \implies h_2 = h_1 \frac{\rho_{\text{м}}}{\rho_{\text{в}}} = 80\,\text{см} \cdot \frac{900\,\frac{\text{кг}}{\text{м}^{3}}}{1000\,\frac{\text{кг}}{\text{м}^{3}}} = 72\,\text{см}$
}

\tasknumber{5}%
\task{%
    В два сообщающихся сосуда сечений $60\,\text{см}^{2}$ и $40\,\text{см}^{2}$ налита вода.
    Оба сосуда закрыты лёгкими поршнями и находятся в равновесии.
    На больший из поршней кладут груз массой $300\,\text{г}$.
    Определите, на сколько поднимется меньший поршень.
    Ответ выразите в сантиметрах.
    $\rho_{\text{в}} = 1000\,\frac{\text{кг}}{\text{м}^{3}}$.
}
\answer{%
    \begin{align*}
    S_1h_1 &= S_2h_2 \implies h_1 = h_2 \frac{S_2}{S_1}  \\
    \frac{mg}{S_1} &= \rho_{\text{в}}g(h_1 + h_2) = \rho_{\text{в}}g \cdot h_2 \cbr{1 + \frac{S_2}{S_1}} \\
    h_2 &= \frac{mg}{S_1\rho_{\text{в}}g} \cdot \frac{S_1}{S_1 + S_2} = \frac{m}{\rho_{\text{в}}(S_1 + S_2)} = 3\,\text{см}
    \end{align*}
}
% autogenerated
