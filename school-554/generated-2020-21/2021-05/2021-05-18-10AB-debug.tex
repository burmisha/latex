\newcommand\rootpath{../../..}
\documentclass[12pt,a4paper]{amsart}%DVI-mode.
\usepackage{graphics,graphicx,epsfig}%DVI-mode.
%\documentclass[pdftex,12pt]{amsart} %PDF-mode.
%\usepackage[pdftex]{graphicx}       %PDF-mode.

%\usepackage{a4wide}                 % Fit the text to A4 page tightly.
\usepackage[utf8]{inputenc}
\usepackage[T2A]{fontenc}
\usepackage[english,russian]{babel} % Download Russian fonts.
\usepackage{amsmath,amsfonts,amssymb,amsthm,amscd,mathrsfs} % Use AMS symbols.
\usepackage{tikz}
\usetikzlibrary{circuits.ee.IEC}
\usetikzlibrary{shapes.geometric}
\usetikzlibrary{decorations.markings}
%\usetikzlibrary{dashs}
%\usetikzlibrary{info}


\textheight=29cm % высота текста
\textwidth=18cm % ширина текста
\topmargin=-2.5cm % отступ от верхнего края
\parskip=6pt % интервал между абзацами
\oddsidemargin=-1.5cm
\evensidemargin=-1.5cm 

% wide docs
% \oddsidemargin=0cm
% \evensidemargin=0cm 
% \textheight=29cm % высота текста
% \textwidth=15cm % ширина текста
% \topmargin=-1.5cm % отступ от верхнего края
% \parskip=18pt % интервал между абзацами


\parindent=0pt % абзацный отступ
\tolerance=500 % терпимость к "жидким" строкам
\binoppenalty=10000 % штраф за перенос формул - 10000 - абсолютный запрет
\relpenalty=10000
\flushbottom % выравнивание высоты страниц
\def\baselinestretch{1.00}
\pagenumbering{gobble}

\begin{document}
\newcommand\bivec[2]{\begin{pmatrix} #1 \\ #2 \end{pmatrix}}

\newcommand\ol[1]{\overline{#1}}

\newcommand\p[1]{\ensuremath{\Prob\!\left(#1\right)}}
\def\cond{\,|\,}
\newcommand\e[1]{\mathsf{E}\!\left(#1\right)}
\newcommand\disp[1]{\mathsf{D}\!\left(#1\right)}
%\newcommand\norm[2]{\mathcal{N}\!\cbr{#1,#2}}
\newcommand\sign{\text{ sign }}

\newcommand\al[1]{\begin{align*} #1 \end{align*}}
\newcommand\begcas[1]{\begin{cases}#1\end{cases}}
\newcommand\tab[2]{	\vspace{-#1pt}
						\begin{tabbing} 
						#2
						\end{tabbing}
					\vspace{-#1pt}
					}


\newcommand\maintext[1]{{\bfseries\sffamily{#1}}}
\newcommand\simpletitle[1]{\begin{center} \maintext{#1} \end{center}}

\def\le{\leqslant}
\def\ge{\geqslant}
\def\Ell{\mathcal{L}}
\def\eps{\varepsilon}
\def\x{\ensuremath{\textbf{x}}}
\def\y{\ensuremath{\textbf{y}}}
\def\Rn{\ensuremath{\mathbb{R}^n}}
\def\RSS{\mathsf{RSS}}

\newcommand\mb[1]{\ensuremath{\boldsymbol{\mathbf{#1}}}}
\newcommand\argmax[1]{\arg\underset{#1}\max\,} % \operatornamewithlimits
%\newcommand{\prodl}{\mathop{\textstyle\prod}\limits}
\newcommand{\prodl}{\prod\limits}
\newcommand{\suml}{\sum\limits}
\newcommand\foral[1]{\forall\,#1\:}
\newcommand\exist[1]{\exists\,#1\:\colon}

\newcommand\cbr[1]{\left(#1\right)} %circled brackets
\newcommand\fbr[1]{\left\{#1\right\}} %figure brackets
\newcommand\sbr[1]{\left[#1\right]} %square brackets
\newcommand\modul[1]{\left|#1\right|}
\newcommand\cdf[2]{\cdot\frac{#1}{#2}}
\newcommand\integr[3]{\int\limits_{#1}^{#2}{#3}}
\newcommand\obol[1]{O\!\cbr{#1}}
\newcommand\norm[1]{\ensuremath{\left\|{#1}\right\|}}

\newcommand\dd[2]{\frac{\partial#1}{\partial#2}}

\newcommand\addeps[2]{
	\begin{figure} [!ht] %lrp
		\centering
		\includegraphics[height=240px]{#1.eps}
		\vspace{-10pt}
		\caption{#2}
		\label{eps:#1}
	\end{figure}
}

\newcommand\addtikz[4]{
	\begin{figure} [!ht] %lrp
		\centering
		\begin{tikzpicture}[x=#2cm,y=#2cm,#3]
			\input{#1.tikz}
		\end{tikzpicture}
		\vspace{-10pt}
		\caption{#4}
		\label{tikz:#1}	
	\end{figure}
}



\newcommand\addepssize[3]{
	\begin{figure} [!ht] %lrp hp
		\centering
		\includegraphics[height=#3px]{#1.eps}
		\vspace{-10pt}
		\caption{#2}
		\label{eps:#1}
	\end{figure}
}

\def\algorithmicrequire{\textbf{Вход:}}
\def\algorithmicensure{\textbf{Выход:}}
\def\algorithmicif{\textbf{если}}
\def\algorithmicthen{\textbf{то}}
\def\algorithmicelse{\textbf{иначе}}
\def\algorithmicelsif{\textbf{иначе если}}
\def\algorithmicfor{\textbf{для}}
\def\algorithmicforall{\textbf{для всех}}
\def\algorithmicdo{}
\def\algorithmicwhile{\textbf{пока}}
\def\algorithmicrepeat{\textbf{повторять}}
\def\algorithmicuntil{\textbf{пока}}
\def\algorithmicloop{\textbf{цикл}}
% переопределение стиля комментариев
\def\algorithmiccomment#1{\quad// {\sl #1}}
%\raggedright
\classdate{7}{20 апреля 2018}

\task 1
Площадь большого поршня гидравлического домкрата $S_1 = 20\units{см}^2$, а малого $S_2 = 0{,}5\units{см}^2.$ Груз какой максимальной массы можно поднять этим домкратом, если на малый поршень давить с силой не более $F=200\units{Н}?$ Силой трения от стенки цилиндров пренебречь.

\task 2
В сосуд налита вода. Расстояние от поверхности воды до дна $H = 0{,}5\units{м},$ площадь дна $S = 0{,}1\units{м}^2.$ Найти гидростатическое давление $P_1$ и полное давление $P_2$ вблизи дна. Найти силу давления воды на дно. Плотность воды \rhowater

\task 3
На лёгкий поршень площадью $S=900\units{см}^2,$ касающийся поверхности воды, поставили гирю массы $m=3\units{кг}$. Высота слоя воды в сосуде с вертикальными стенками $H = 20\units{см}$. Определить давление жидкости вблизи дна, если плотность воды \rhowater

\task 4
Давление газов в конце сгорания в цилиндре дизельного двигателя трактора $P = 9\units{МПа}.$ Диаметр цилиндра $d = 130\units{мм}.$ С какой силой газы давят на поршень в цилиндре? Площадь круга диаметром $D$ равна $S = \cfrac{\pi D^2}4.$

\task 5
Площадь малого поршня гидравлического подъёмника $S_1 = 0{,}8\units{см}^2$, а большого $S_2 = 40\units{см}^2.$ Какую силу $F$ надо приложить к малому поршню, чтобы поднять груз весом $P = 8\units{кН}?$

\task 6
Герметичный сосуд полностью заполнен водой и стоит на столе. На небольшой поршень площадью $S$ давят рукой с силой $F$. Поршень находится ниже крышки сосуда на $H_1$, выше дна на $H_2$ и может свободно перемещаться. Плотность воды $\rho$, атмосферное давление $P_A$. Найти давления $P_1$ и $P_2$ в воде вблизи крышки и дна сосуда.
\\ \\
\classdate{7}{20 апреля 2018}

\task 1
Площадь большого поршня гидравлического домкрата $S_1 = 20\units{см}^2$, а малого $S_2 = 0{,}5\units{см}^2.$ Груз какой максимальной массы можно поднять этим домкратом, если на малый поршень давить с силой не более $F=200\units{Н}?$ Силой трения от стенки цилиндров пренебречь.

\task 2
В сосуд налита вода. Расстояние от поверхности воды до дна $H = 0{,}5\units{м},$ площадь дна $S = 0{,}1\units{м}^2.$ Найти гидростатическое давление $P_1$ и полное давление $P_2$ вблизи дна. Найти силу давления воды на дно. Плотность воды \rhowater

\task 3
На лёгкий поршень площадью $S=900\units{см}^2,$ касающийся поверхности воды, поставили гирю массы $m=3\units{кг}$. Высота слоя воды в сосуде с вертикальными стенками $H = 20\units{см}$. Определить давление жидкости вблизи дна, если плотность воды \rhowater

\task 4
Давление газов в конце сгорания в цилиндре дизельного двигателя трактора $P = 9\units{МПа}.$ Диаметр цилиндра $d = 130\units{мм}.$ С какой силой газы давят на поршень в цилиндре? Площадь круга диаметром $D$ равна $S = \cfrac{\pi D^2}4.$

\task 5
Площадь малого поршня гидравлического подъёмника $S_1 = 0{,}8\units{см}^2$, а большого $S_2 = 40\units{см}^2.$ Какую силу $F$ надо приложить к малому поршню, чтобы поднять груз весом $P = 8\units{кН}?$

\task 6
Герметичный сосуд полностью заполнен водой и стоит на столе. На небольшой поршень площадью $S$ давят рукой с силой $F$. Поршень находится ниже крышки сосуда на $H_1$, выше дна на $H_2$ и может свободно перемещаться. Плотность воды $\rho$, атмосферное давление $P_A$. Найти давления $P_1$ и $P_2$ в воде вблизи крышки и дна сосуда.

\newpage

\adddate{8 класс. 20 апреля 2018}

\task 1
Между точками $A$ и $B$ электрической цепи подключены последовательно резисторы $R_1 = 10\units{Ом}$ и $R_2 = 20\units{Ом}$ и параллельно им $R_3 = 30\units{Ом}.$ Найдите эквивалентное сопротивление $R_{AB}$ этого участка цепи.

\task 2
Электрическая цепь состоит из последовательности $N$ одинаковых звеньев, в которых каждый резистор имеет сопротивление $r$. Последнее звено замкнуто резистором сопротивлением $R$. При каком соотношении $\cfrac{R}{r}$ сопротивление цепи не зависит от числа звеньев?

\task 3
Для измерения сопротивления $R$ проводника собрана электрическая цепь. Вольтметр $V$ показывает напряжение $U_V = 5\units{В},$ показание амперметра $A$ равно $I_A = 25\units{мА}.$ Найдите величину $R$ сопротивления проводника. Внутреннее сопротивление вольтметра $R_V = 1{,}0\units{кОм},$ внутреннее сопротивление амперметра $R_A = 2{,}0\units{Ом}.$

\task 4
Шкала гальванометра имеет $N=100$ делений, цена деления $\delta = 1\units{мкА}$. Внутреннее сопротивление гальванометра $R_G = 1{,}0\units{кОм}.$ Как из этого прибора сделать вольтметр для измерения напряжений до $U = 100\units{В}$ или амперметр для измерения токов силой до $I = 1\units{А}?$

\\ \\ \\ \\ \\ \\ \\ \\
\adddate{8 класс. 20 апреля 2018}

\task 1
Между точками $A$ и $B$ электрической цепи подключены последовательно резисторы $R_1 = 10\units{Ом}$ и $R_2 = 20\units{Ом}$ и параллельно им $R_3 = 30\units{Ом}.$ Найдите эквивалентное сопротивление $R_{AB}$ этого участка цепи.

\task 2
Электрическая цепь состоит из последовательности $N$ одинаковых звеньев, в которых каждый резистор имеет сопротивление $r$. Последнее звено замкнуто резистором сопротивлением $R$. При каком соотношении $\cfrac{R}{r}$ сопротивление цепи не зависит от числа звеньев?

\task 3
Для измерения сопротивления $R$ проводника собрана электрическая цепь. Вольтметр $V$ показывает напряжение $U_V = 5\units{В},$ показание амперметра $A$ равно $I_A = 25\units{мА}.$ Найдите величину $R$ сопротивления проводника. Внутреннее сопротивление вольтметра $R_V = 1{,}0\units{кОм},$ внутреннее сопротивление амперметра $R_A = 2{,}0\units{Ом}.$

\task 4
Шкала гальванометра имеет $N=100$ делений, цена деления $\delta = 1\units{мкА}$. Внутреннее сопротивление гальванометра $R_G = 1{,}0\units{кОм}.$ Как из этого прибора сделать вольтметр для измерения напряжений до $U = 100\units{В}$ или амперметр для измерения токов силой до $I = 1\units{А}?$


% \begin{flushright}
\textsc{ГБОУ школа №554, 20 ноября 2018\,г.}
\end{flushright}

\begin{center}
\LARGE \textsc{Математический бой, 8 класс}
\end{center}

\problem{1} Есть тридцать карточек, на каждой написано по одному числу: на десяти карточках~–~$a$,  на десяти других~–~$b$ и на десяти оставшихся~–~$c$ (числа  различны). Известно, что к любым пяти карточкам можно подобрать ещё пять так, что сумма чисел на этих десяти карточках будет равна нулю. Докажите, что~одно из~чисел~$a, b, c$ равно нулю.

\problem{2} Вокруг стола стола пустили пакет с орешками. Первый взял один орешек, второй — 2, третий — 3 и так далее: каждый следующий брал на 1 орешек больше. Известно, что на втором круге было взято в сумме на 100 орешков больше, чем на первом. Сколько человек сидело за столом?

% \problem{2} Натуральное число разрешено увеличить на любое целое число процентов от 1 до 100, если при этом получаем натуральное число. Найдите наименьшее натуральное число, которое нельзя при помощи таких операций получить из~числа 1.

% \problem{3} Найти сумму $1^2 - 2^2 + 3^2 - 4^2 + 5^2 + \ldots - 2018^2$.

\problem{3} В кружке рукоделия, где занимается Валя, более 93\% участников~—~девочки. Какое наименьшее число участников может быть в таком кружке?

\problem{4} Произведение 2018 целых чисел равно 1. Может ли их сумма оказаться равной~0?

% \problem{4} Можно ли все натуральные числа от~1 до~9 записать в~клетки таблицы~$3\times3$ так, чтобы сумма в~любых двух соседних (по~вертикали или горизонтали) клетках равнялось простому числу?

\problem{5} На доске написано 2018 нулей и 2019 единиц. Женя стирает 2 числа и, если они были одинаковы, дописывает к оставшимся один ноль, а~если разные — единицу. Потом Женя повторяет эту операцию снова, потом ещё и~так далее. В~результате на~доске останется только одно число. Что это за~число?

\problem{6} Докажите, что в~любой компании людей найдутся 2~человека, имеющие равное число знакомых в этой компании (если $A$~знаком с~$B$, то~и $B$~знаком с~$A$).

\problem{7} Три колокола начинают бить одновременно. Интервалы между ударами колоколов соответственно составляют $\cfrac43$~секунды, $\cfrac53$~секунды и $2$~секунды. Совпавшие по времени удары воспринимаются за~один. Сколько ударов будет услышано за 1~минуту, включая первый и последний удары?

\problem{8} Восемь одинаковых момент расположены по кругу. Известно, что три из~них~— фальшивые, и они расположены рядом друг с~другом. Вес фальшивой монеты отличается от~веса настоящей. Все фальшивые монеты весят одинаково, но неизвестно, тяжелее или легче фальшивая монета настоящей. Покажите, что за~3~взвешивания на~чашечных весах без~гирь можно определить все фальшивые монеты.

\end{document}

\begin{document}

\setdate{18~мая~2021}
\setclass{10«АБ»}

\addpersonalvariant{Михаил Бурмистров}

\tasknumber{1}%
\task{%
    Саша стартует на велосипеде и в течение $t = 4\,\text{c}$ двигается с постоянным ускорением $0{,}5\,\frac{\text{м}}{\text{с}^{2}}$.
    Определите
    \begin{itemize}
        \item какую скорость при этом удастся достичь,
        \item какой путь за это время будет пройден,
        \item среднюю скорость за всё время движения, если после начального ускорения продолжить движение равномерно ещё в течение времени $2t$
    \end{itemize}
}
\answer{%
    \begin{align*}
    v &= v_0 + a t = at = 0{,}5\,\frac{\text{м}}{\text{с}^{2}} \cdot 4\,\text{c} = 2\,\frac{\text{м}}{\text{с}}, \\
    s_x &= v_0t + \frac{a t^2}2 = \frac{a t^2}2 = \frac{0{,}5\,\frac{\text{м}}{\text{с}^{2}} \cdot \sqr{4\,\text{c}}}2 = 4\,\text{м}, \\
    v_\text{сред.} &= \frac{s_\text{общ}}{t_\text{общ.}} = \frac{s_x + v \cdot 2t}{t + 2t} = \frac{\frac{a t^2}2 + at \cdot 2t}{t (1 + 2)} = \\
    &= at \cdot \frac{\frac 12 + 2}{1 + 2} = 0{,}5\,\frac{\text{м}}{\text{с}^{2}} \cdot 4\,\text{c} \cdot \frac{\frac 12 + 2}{1 + 2} \approx 1{,}67\,\frac{\text{м}}{\text{c}}.
    \end{align*}
}
\solutionspace{120pt}

\tasknumber{2}%
\task{%
    Какой путь тело пройдёт за вторую секунду после начала свободного падения?
    Какую скорость в начале этой секунды оно имеет?
}
\answer{%
    \begin{align*}
    s &= -s_y = -(y_2-y_1) = y_1 - y_2 = \cbr{y_0 + v_{0y}t_1 - \frac{gt_1^2}2} - \cbr{y_0 + v_{0y}t_2 - \frac{gt_2^2}2} = \\
    &= \frac{gt_2^2}2 - \frac{gt_1^2}2 = \frac g2\cbr{t_2^2 - t_1^2} = 15\,\text{м}, \\
    v_y &= v_{0y} - gt = -gt = 10\,\frac{\text{м}}{\text{с}^{2}} \cdot 1\,\text{с} = -10\,\frac{\text{м}}{\text{с}}.
    \end{align*}
}
\solutionspace{120pt}

\tasknumber{3}%
\task{%
    Карусель диаметром $3\,\text{м}$ равномерно совершает 6 оборотов в минуту.
    Определите
    \begin{itemize}
        \item период и частоту её обращения,
        \item скорость и ускорение крайних её точек.
    \end{itemize}
}
\answer{%
    \begin{align*}
    t &= 60\,\text{с}, r = 1{,}5\,\text{м}, n = 6\units{оборотов}, \\
    T &= \frac tN = \frac{60\,\text{с}}{6} \approx 10\,\text{c}, \\
    \nu &= \frac 1T = \frac{6}{60\,\text{с}} \approx 0{,}10\,\text{Гц}, \\
    v &= \frac{2 \pi r}T = \frac{2 \pi r}T =  \frac{2 \pi r n}t \approx 0{,}94\,\frac{\text{м}}{\text{c}}, \\
    a &= \frac{v^2}r =  \frac{4 \pi^2 r n^2}{t^2} \approx 0{,}59\,\frac{\text{м}}{\text{с}^{2}}.
    \end{align*}
}
\solutionspace{80pt}

\tasknumber{4}%
\task{%
    Паша стоит на обрыве над рекой и методично и строго горизонтально кидает в неё камушки.
    За этим всем наблюдает экспериментатор Глюк, который уже выяснил, что камушки падают в реку спустя $1{,}6\,\text{с}$ после броска,
    а вот дальность полёта оценить сложнее: придётся лезть в воду.
    Выручите Глюка и определите:
    \begin{itemize}
        \item высоту обрыва (вместе с ростом Паши).
        \item дальность полёта камушков (по горизонтали) и их скорость при падении, приняв начальную скорость броска равной $v_0 = 18\,\frac{\text{м}}{\text{с}}$.
    \end{itemize}
    Сопротивлением воздуха пренебречь.
}
\answer{%
    \begin{align*}
    y &= y_0 + v_{0y}t - \frac{gt^2}2 = h - \frac{gt^2}2, \qquad y(\tau) = 0 \implies h - \frac{g\tau^2}2 = 0 \implies h = \frac{g\tau^2}2 \approx 12{,}8\,\text{м}.
    \\
    x &= x_0 + v_{0x}t = v_0t \implies L = v_0\tau \approx 28{,}8\,\text{м}.
    \\
    &v = \sqrt{v_x^2 + v_y^2} = \sqrt{v_{0x}^2 + \sqr{v_{0y} - g\tau}} = \sqrt{v_0^2 + \sqr{g\tau}} \approx 24{,}1\,\frac{\text{м}}{\text{c}}.
    \end{align*}
}
\solutionspace{120pt}

\tasknumber{5}%
\task{%
    Четыре одинаковых брусков массой $3\,\text{кг}$ каждый лежат на гладком горизонтальном столе.
    Бруски пронумерованы от 1 до 4 и последовательно связаны между собой
    невесомыми нерастяжимыми нитями: 1 со 2, 2 с 3 (ну и с 1) и т.д.
    Экспериментатор Глюк прикладывает постоянную горизонтальную силу $90\,\text{Н}$ к бруску с наибольшим номером.
    С каким ускорением двигается система? Чему равна сила натяжения нити, связывающей бруски 1 и 2?
}
\answer{%
    \begin{align*}
    a &= \frac{F}{4 m} = \frac{90\,\text{Н}}{4 \cdot 3\,\text{кг}} \approx 7{,}5\,\frac{\text{м}}{\text{c}^{2}}, \\
    T &= m'a = 1m \cdot \frac{F}{4 m} = \frac{1}{4} F \approx 22{,}5\,\text{Н}.
    \end{align*}
}
\solutionspace{120pt}

\tasknumber{6}%
\task{%
    Два бруска связаны лёгкой нерастяжимой нитью и перекинуты через неподвижный блок (см.
    рис.).
    Определите силу натяжения нити и ускорения брусков.
    Силами трения пренебречь, массы брусков
    равны $m_1 = 5\,\text{кг}$ и $m_2 = 14\,\text{кг}$.
    % $g = 10\,\frac{\text{м}}{\text{с}^{2}}$.

    \begin{tikzpicture}[x=1.5cm,y=1.5cm,thick]
        \draw
            (-0.4, 0) rectangle (-0.2, 1.2)
            (0.15, 0.5) rectangle (0.45, 1)
            (0, 2) circle [radius=0.3] -- ++(up:0.5)
            (-0.3, 1.2) -- ++(up:0.8)
            (0.3, 1) -- ++(up:1)
            (-0.7, 2.5) -- (0.7, 2.5)
            ;
        \draw[pattern={Lines[angle=51,distance=3pt]},pattern color=black,draw=none] (-0.7, 2.5) rectangle (0.7, 2.75);
        \node [left] (left) at (-0.4, 0.6) {$m_1$};
        \node [right] (right) at (0.4, 0.75) {$m_2$};
    \end{tikzpicture}
}
\answer{%
    Предположим, что левый брусок ускоряется вверх, тогда правый ускоряется вниз (с тем же ускорением).
    Запишем 2-й закон Ньютона 2 раза (для обоих тел) в проекции на вертикальную оси, направив её вверх.
    \begin{align*}
        &\begin{cases}
            T - m_1g = m_1a, \\
            T - m_2g = -m_2a,
        \end{cases} \\
        &\begin{cases}
            m_2g - m_1g = m_1a + m_2a, \\
            T = m_1a + m_1g, \\
        \end{cases} \\
        a &= \frac{m_2 - m_1}{m_1 + m_2} \cdot g = \frac{14\,\text{кг} - 5\,\text{кг}}{5\,\text{кг} + 14\,\text{кг}} \cdot 10\,\frac{\text{м}}{\text{с}^{2}} \approx 4{,}74\,\frac{\text{м}}{\text{c}^{2}}, \\
        T &= m_1(a + g) = m_1 \cdot g \cdot \cbr{\frac{m_2 - m_1}{m_1 + m_2} + 1} = m_1 \cdot g \cdot \frac{2m_2}{m_1 + m_2} = \\
            &= \frac{2 m_2 m_1 g}{m_1 + m_2} = \frac{2 \cdot 14\,\text{кг} \cdot 5\,\text{кг} \cdot 10\,\frac{\text{м}}{\text{с}^{2}}}{5\,\text{кг} + 14\,\text{кг}} \approx 73{,}7\,\text{Н}.
    \end{align*}
    Отрицательный ответ говорит, что мы лишь не угадали с направлением ускорений.
    Сила же всегда положительна.
}
\solutionspace{80pt}

\tasknumber{7}%
\task{%
    Тело массой $1{,}4\,\text{кг}$ лежит на горизонтальной поверхности.
    Коэффициент трения между поверхностью и телом $0{,}25$.
    К телу приложена горизонтальная сила $5{,}5\,\text{Н}$.
    Определите силу трения, действующую на тело, и ускорение тела.
    % $g = 10\,\frac{\text{м}}{\text{с}^{2}}$.
}
\answer{%
    \begin{align*}
    &F_\text{трения покоя $\max$} = \mu N = \mu m g = 0{,}25 \cdot 1{,}4\,\text{кг} \cdot 10\,\frac{\text{м}}{\text{с}^{2}} = 3{,}50\,\text{Н}, \\
    &F_\text{трения покоя $\max$} \le F \implies F_\text{трения} = 3{,}50\,\text{Н}, a = \frac{F - F_\text{трения}}{m} = 1{,}43\,\frac{\text{м}}{\text{c}^{2}}, \\
    &\text{ при равенстве возможны оба варианта: и едет, и не едет, но на ответы это не влияет.}
    \end{align*}
}
\solutionspace{120pt}

\tasknumber{8}%
\task{%
    Определите плотность неизвестного вещества, если известно, что опускании тела из него
    в подсолнечное масло оно будет плавать и на треть выступать над поверхностью жидкости.
}
\answer{%
    $F_\text{Арх.} = F_\text{тяж.} \implies \rho_\text{ж.} g V_\text{погр.} = m g \implies\rho_\text{ж.} g \cbr{V -\frac V3} = \rho V g \implies \rho = \rho_\text{ж.}\cbr{1 -\frac 13} \approx 600\,\frac{\text{кг}}{\text{м}^{3}}$
}
\solutionspace{120pt}

\tasknumber{9}%
\task{%
    Определите силу, действующую на левую опору однородного горизонтального стержня длиной $l = 7\,\text{м}$
    и массой $M = 5\,\text{кг}$, к которому подвешен груз массой $m = 2\,\text{кг}$ на расстоянии $4\,\text{м}$ от правого конца (см.
    рис.).

    \begin{tikzpicture}[thick]
        \draw
            (-2, -0.1) rectangle (2, 0.1)
            (-0.5, -0.1) -- (-0.5, -1)
            (-0.7, -1) rectangle (-0.3, -1.3)
            (-2, -0.1) -- +(0.15,-0.9) -- +(-0.15,-0.9) -- cycle
            (2, -0.1) -- +(0.15,-0.9) -- +(-0.15,-0.9) -- cycle
        ;
        \draw[pattern={Lines[angle=51,distance=2pt]},pattern color=black,draw=none]
            (-2.15, -1.15) rectangle +(0.3, 0.15)
            (2.15, -1.15) rectangle +(-0.3, 0.15)
        ;
        \node [right] (m_small) at (-0.3, -1.15) {$m$};
        \node [above] (M_big) at (0, 0.1) {$M$};
    \end{tikzpicture}
}
\answer{%
    \begin{align*}
        &\begin{cases}
            F_1 + F_2 - mg - Mg= 0, \\
            F_1 \cdot 0 - mg \cdot a - Mg \cdot \frac l2 + F_2 \cdot l = 0,
        \end{cases} \\
        F_2 &= \frac{mga + Mg\frac l2}l = \frac al \cdot mg + \frac{Mg}2 \approx 33{,}6\,\text{Н}, \\
        F_1 &= mg + Mg - F_2 = mg + Mg - \frac al \cdot mg - \frac{Mg}2 = \frac bl \cdot mg + \frac{Mg}2 \approx 36{,}4\,\text{Н}.
    \end{align*}
}
\solutionspace{80pt}

\tasknumber{10}%
\task{%
    Тонкий однородный лом длиной $1\,\text{м}$ и массой $10\,\text{кг}$ лежит на горизонтальной поверхности.
    \begin{itemize}
        \item Какую минимальную силу надо приложить к одному из его концов, чтобы оторвать его от этой поверхности?
        \item Какую минимальную работу надо совершить, чтобы поставить его на землю в вертикальное положение?
    \end{itemize}
    % Примите $g = 10\,\frac{\text{м}}{\text{с}^{2}}$.
}
\answer{%
    $F = \frac{mg}2 \approx 100\,\text{Н}, A = mg\frac l2 = 50\,\text{Дж}$
}
\solutionspace{120pt}

\tasknumber{11}%
\task{%
    Определите работу силы, которая обеспечит спуск тела массой $2\,\text{кг}$ на высоту $5\,\text{м}$ с постоянным ускорением $3\,\frac{\text{м}}{\text{с}^{2}}$.
    % Примите $g = 10\,\frac{\text{м}}{\text{с}^{2}}$.
}
\answer{%
    \begin{align*}
    &\text{Для подъёма: } A = Fh = (mg + ma) h = m(g+a)h, \\
    &\text{Для спуска: } A = -Fh = -(mg - ma) h = -m(g-a)h, \\
    &\text{В результате получаем: } A = -70\,\text{Дж}.
    \end{align*}
}
\solutionspace{60pt}

\tasknumber{12}%
\task{%
    Тело бросили вертикально вверх со скоростью $14\,\frac{\text{м}}{\text{с}}$.
    На какой высоте кинетическая энергия тела составит половину от потенциальной?
}
\answer{%
    \begin{align*}
    &0 + \frac{mv_0^2}2 = E_p + E_k, E_k = \frac 12 E_p \implies \\
    &\implies \frac{mv_0^2}2 = E_p + \frac 12 E_p = E_p\cbr{1 + \frac 12} = mgh\cbr{1 + \frac 12} \implies \\
    &\implies h = \frac{\frac{mv_0^2}2}{mg\cbr{1 + \frac 12}} = \frac{v_0^2}{2g} \cdot \frac 1{1 + \frac 12} \approx 6{,}5\,\text{м}.
    \end{align*}
}
\solutionspace{100pt}

\tasknumber{13}%
\task{%
    Плотность воздуха при нормальных условиях равна $1{,}3\,\frac{\text{кг}}{\text{м}^{3}}$.
    Чему равна плотность воздуха
    при температуре $200\celsius$ и давлении $50\,\text{кПа}$?
}
\answer{%
    \begin{align*}
    &\text{В общем случае:} PV = \frac m{\mu} RT \implies \rho = \frac mV = \frac m{\frac{\frac m{\mu} RT}P} = \frac{P\mu}{RT}, \\
    &\text{У нас 2 состояния:} \rho_1 = \frac{P_1\mu}{RT_1}, \rho_2 = \frac{P_2\mu}{RT_2} \implies \frac{\rho_2}{\rho_1} = \frac{\frac{P_2\mu}{RT_2}}{\frac{P_1\mu}{RT_1}} = \frac{P_2T_1}{P_1T_2} \implies \\
    &\implies \rho_2 = \rho_1 \cdot \frac{P_2T_1}{P_1T_2} = 1{,}3\,\frac{\text{кг}}{\text{м}^{3}} \cdot \frac{50\,\text{кПа} \cdot 273\units{К}}{100\,\text{кПа} \cdot 473\units{К}} \approx 0{,}38\,\frac{\text{кг}}{\text{м}^{3}}.
    \end{align*}
}
\solutionspace{120pt}

\tasknumber{14}%
\task{%
    Небольшую цилиндрическую пробирку с воздухом погружают на некоторую глубину в глубокое пресное озеро,
    после чего воздух занимает в ней лишь пятую часть от общего объема.
    Определите глубину, на которую погрузили пробирку.
    Температуру считать постоянной $T = 280\,\text{К}$, давлением паров воды пренебречь,
    атмосферное давление принять равным $p_{\text{aтм}} = 100\,\text{кПа}$.
}
\answer{%
    \begin{align*}
    T\text{— const} &\implies P_1V_1 = \nu RT = P_2V_2.
    \\
    V_2 = \frac 15 V_1 &\implies P_1V_1 = P_2 \cdot \frac 15V_1 \implies P_2 = 5P_1 = 5p_{\text{aтм}}.
    \\
    P_2 = p_{\text{aтм}} + \rho_{\text{в}} g h \implies h = \frac{P_2 - p_{\text{aтм}}}{\rho_{\text{в}} g} &= \frac{5p_{\text{aтм}} - p_{\text{aтм}}}{\rho_{\text{в}} g} = \frac{4 \cdot p_{\text{aтм}}}{\rho_{\text{в}} g} =  \\
     &= \frac{4 \cdot 100\,\text{кПа}}{1000\,\frac{\text{кг}}{\text{м}^{3}} \cdot  10\,\frac{\text{м}}{\text{с}^{2}}} \approx 40\,\text{м}.
    \end{align*}
}
\solutionspace{120pt}

\tasknumber{15}%
\task{%
    Газу сообщили некоторое количество теплоты,
    при этом треть его он потратил на совершение работы,
    одновременно увеличив свою внутреннюю энергию на $1200\,\text{Дж}$.
    Определите работу, совершённую газом.
}
\answer{%
    \begin{align*}
    Q &= A' + \Delta U, A' = \frac 13 Q \implies Q \cdot \cbr{1 - \frac 13} = \Delta U \implies Q = \frac{\Delta U}{1 - \frac 13} = \frac{1200\,\text{Дж}}{1 - \frac 13} \approx 1800\,\text{Дж}.
    \\
    A' &= \frac 13 Q
        = \frac 13 \cdot \frac{\Delta U}{1 - \frac 13}
        = \frac{\Delta U}{3 - 1}
        = \frac{1200\,\text{Дж}}{3 - 1} \approx 600\,\text{Дж}.
    \end{align*}
}
\solutionspace{60pt}

\tasknumber{16}%
\task{%
    Два конденсатора ёмкостей $C_1 = 40\,\text{нФ}$ и $C_2 = 60\,\text{нФ}$ последовательно подключают
    к источнику напряжения $V = 400\,\text{В}$ (см.
    рис.).
    % Определите заряды каждого из конденсаторов.
    Определите заряд второго конденсатора.

    \begin{tikzpicture}[circuit ee IEC, semithick]
        \draw  (0, 0) to [capacitor={info={$C_1$}}] (1, 0)
                       to [capacitor={info={$C_2$}}] (2, 0)
        ;
        % \draw [-o] (0, 0) -- ++(-0.5, 0) node[left] {$-$};
        % \draw [-o] (2, 0) -- ++(0.5, 0) node[right] {$+$};
        \draw [-o] (0, 0) -- ++(-0.5, 0) node[left] {};
        \draw [-o] (2, 0) -- ++(0.5, 0) node[right] {};
    \end{tikzpicture}
}
\answer{%
    $
        Q_1
            = Q_2
            = CV
            = \frac{V}{\frac1{C_1} + \frac1{C_2}}
            = \frac{C_1C_2V}{C_1 + C_2}
            = \frac{
                40\,\text{нФ} \cdot 60\,\text{нФ} \cdot 400\,\text{В}
             }{
                40\,\text{нФ} + 60\,\text{нФ}
             }
            = 9{,}60\,\text{мкКл}
    $
}
\solutionspace{120pt}

\tasknumber{17}%
\task{%
    В вакууме вдоль одной прямой расположены четыре отрицательных заряда так,
    что расстояние между соседними зарядами равно $r$.
    Сделайте рисунок,
    и определите силу, действующую на крайний заряд.
    Модули всех зарядов равны $q$ ($q > 0$).
}
\answer{%
    $F = \sum_i F_i = \ldots = \frac{49}{36} \frac{kq^2}{r^2}.$
}
\solutionspace{80pt}

\tasknumber{18}%
\task{%
    Юлия проводит эксперименты c 2 кусками одинаковой алюминиевой проволки, причём второй кусок в четыре раза длиннее первого.
    В одном из экспериментов Юлия подаёт на первый кусок проволки напряжение в шесть раз раз больше, чем на второй.
    Определите отношения в двух проволках в этом эксперименте (второй к первой):
    \begin{itemize}
        \item отношение сил тока,
        \item отношение выделяющихся мощностей.
    \end{itemize}
}
\answer{%
    $R_2 = 4R_1, U_1 = 6U_2 \implies  \eli_2 / \eli_1 = \frac{U_2 / R_2}{U_1 / R_1} = \frac{U_2}{U_1} \cdot \frac{R_1}{R_2} = \frac1{24}, P_2 / P_1 = \frac{U_2^2 / R_2}{U_1^2 / R_1} = \sqr{\frac{U_2}{U_1}} \cdot \frac{R_1}{R_2} = \frac1{144}.$
}

\end{document}
% autogenerated
