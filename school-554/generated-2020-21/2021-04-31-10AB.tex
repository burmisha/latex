\setdate{31~апреля~2021}
\setclass{10«АБ»}

\addpersonalvariant{Михаил Бурмистров}

\tasknumber{1}%
\task{%
    Получите выражение:
    \begin{enumerate}
        \item длины проводника через его сопротивление,
        \item сопротивление из закона Ома,
        \item внешнее сопротивление цепи из закона Ома для полной цепи,
        \item эквивалентное сопротивление $n$ резисторов, соединённых параллельно, каждый сопротивлением $R$.
    \end{enumerate}
}
\solutionspace{40pt}

\tasknumber{2}%
\task{%
    Получите выражение:
    \begin{enumerate}
        \item силы тока через выделяемую мощность и сопротивление резистора,
        \item силы тока через выделенную теплоту и напряжение на резисторе,
        \item напряжение на резисторе через выделяемую мощность и силу тока через него,
        \item напряжение на резисторе через выделенную в нём теплоту и сопротивление резистора.
    \end{enumerate}
}
\solutionspace{80pt}

\tasknumber{3}%
\task{%
    Определите ток, протекающий через резистор $R = 18\,\text{Ом}$ и разность потенциалов на нём (см.
    рис.
    на доске),
    если $r_1 = 1\,\text{Ом}$, $r_2 = 2\,\text{Ом}$, $\mathcal{E}_1 = 60\,\text{В}$, $\mathcal{E}_2 = 20\,\text{В}$.
}

\variantsplitter

\addpersonalvariant{Ирина Ан}

\tasknumber{1}%
\task{%
    Получите выражение:
    \begin{enumerate}
        \item площади поперечного сечения проводника через его сопротивление,
        \item удельное сопротивление из закона Ома,
        \item внешнее сопротивление цепи из закона Ома для полной цепи,
        \item эквивалентное сопротивление $n$ резисторов, соединённых параллельно, каждый сопротивлением $R$.
    \end{enumerate}
}
\solutionspace{40pt}

\tasknumber{2}%
\task{%
    Получите выражение:
    \begin{enumerate}
        \item силы тока через выделяемую мощность и сопротивление резистора,
        \item силы тока через выделенную теплоту и разность потенциалов на резисторе,
        \item напряжение на резисторе через выделяемую мощность и силу тока через него,
        \item напряжение на резисторе через выделенную в нём теплоту и сопротивление резистора.
    \end{enumerate}
}
\solutionspace{80pt}

\tasknumber{3}%
\task{%
    Определите ток, протекающий через резистор $R = 10\,\text{Ом}$ и разность потенциалов на нём (см.
    рис.
    на доске),
    если $r_1 = 1\,\text{Ом}$, $r_2 = 1\,\text{Ом}$, $\mathcal{E}_1 = 30\,\text{В}$, $\mathcal{E}_2 = 40\,\text{В}$.
}

\variantsplitter

\addpersonalvariant{Софья Андрианова}

\tasknumber{1}%
\task{%
    Получите выражение:
    \begin{enumerate}
        \item площади поперечного сечения проводника через его сопротивление,
        \item сопротивление из закона Ома,
        \item внешнее сопротивление цепи из закона Ома для полной цепи,
        \item эквивалентное сопротивление $n$ резисторов, соединённых параллельно, каждый сопротивлением $R$.
    \end{enumerate}
}
\solutionspace{40pt}

\tasknumber{2}%
\task{%
    Получите выражение:
    \begin{enumerate}
        \item силы тока через выделяемую мощность и сопротивление резистора,
        \item силы тока через выделенную теплоту и сопротивление резистора,
        \item напряжение на резисторе через выделяемую мощность и силу тока через него,
        \item напряжение на резисторе через выделенную в нём теплоту и сопротивление резистора.
    \end{enumerate}
}
\solutionspace{80pt}

\tasknumber{3}%
\task{%
    Определите ток, протекающий через резистор $R = 20\,\text{Ом}$ и разность потенциалов на нём (см.
    рис.
    на доске),
    если $r_1 = 2\,\text{Ом}$, $r_2 = 2\,\text{Ом}$, $\mathcal{E}_1 = 20\,\text{В}$, $\mathcal{E}_2 = 60\,\text{В}$.
}

\variantsplitter

\addpersonalvariant{Владимир Артемчук}

\tasknumber{1}%
\task{%
    Получите выражение:
    \begin{enumerate}
        \item длины проводника через его сопротивление,
        \item удельное сопротивление из закона Ома,
        \item внешнее сопротивление цепи из закона Ома для полной цепи,
        \item эквивалентное сопротивление $n$ резисторов, соединённых последовательно, каждый сопротивлением $R$.
    \end{enumerate}
}
\solutionspace{40pt}

\tasknumber{2}%
\task{%
    Получите выражение:
    \begin{enumerate}
        \item силы тока через выделяемую мощность и сопротивление резистора,
        \item силы тока через выделенную теплоту и напряжение на резисторе,
        \item напряжение на резисторе через выделяемую мощность и силу тока через него,
        \item напряжение на резисторе через выделенную в нём теплоту и силу тока через него.
    \end{enumerate}
}
\solutionspace{80pt}

\tasknumber{3}%
\task{%
    Определите ток, протекающий через резистор $R = 20\,\text{Ом}$ и разность потенциалов на нём (см.
    рис.
    на доске),
    если $r_1 = 1\,\text{Ом}$, $r_2 = 1\,\text{Ом}$, $\mathcal{E}_1 = 60\,\text{В}$, $\mathcal{E}_2 = 40\,\text{В}$.
}

\variantsplitter

\addpersonalvariant{Софья Белянкина}

\tasknumber{1}%
\task{%
    Получите выражение:
    \begin{enumerate}
        \item длины проводника через его сопротивление,
        \item удельное сопротивление из закона Ома,
        \item внешнее сопротивление цепи из закона Ома для полной цепи,
        \item эквивалентное сопротивление $n$ резисторов, соединённых параллельно, каждый сопротивлением $R$.
    \end{enumerate}
}
\solutionspace{40pt}

\tasknumber{2}%
\task{%
    Получите выражение:
    \begin{enumerate}
        \item силы тока через выделяемую мощность и сопротивление резистора,
        \item силы тока через выделенную теплоту и сопротивление резистора,
        \item напряжение на резисторе через выделяемую мощность и сопротивление резистора,
        \item напряжение на резисторе через выделенную в нём теплоту и силу тока через него.
    \end{enumerate}
}
\solutionspace{80pt}

\tasknumber{3}%
\task{%
    Определите ток, протекающий через резистор $R = 20\,\text{Ом}$ и разность потенциалов на нём (см.
    рис.
    на доске),
    если $r_1 = 1\,\text{Ом}$, $r_2 = 1\,\text{Ом}$, $\mathcal{E}_1 = 60\,\text{В}$, $\mathcal{E}_2 = 60\,\text{В}$.
}

\variantsplitter

\addpersonalvariant{Варвара Егиазарян}

\tasknumber{1}%
\task{%
    Получите выражение:
    \begin{enumerate}
        \item площади поперечного сечения проводника через его сопротивление,
        \item сопротивление из закона Ома,
        \item внешнее сопротивление цепи из закона Ома для полной цепи,
        \item эквивалентное сопротивление $n$ резисторов, соединённых последовательно, каждый сопротивлением $R$.
    \end{enumerate}
}
\solutionspace{40pt}

\tasknumber{2}%
\task{%
    Получите выражение:
    \begin{enumerate}
        \item силы тока через выделяемую мощность и сопротивление резистора,
        \item силы тока через выделенную теплоту и напряжение на резисторе,
        \item напряжение на резисторе через выделяемую мощность и сопротивление резистора,
        \item напряжение на резисторе через выделенную в нём теплоту и силу тока через него.
    \end{enumerate}
}
\solutionspace{80pt}

\tasknumber{3}%
\task{%
    Определите ток, протекающий через резистор $R = 20\,\text{Ом}$ и разность потенциалов на нём (см.
    рис.
    на доске),
    если $r_1 = 1\,\text{Ом}$, $r_2 = 1\,\text{Ом}$, $\mathcal{E}_1 = 60\,\text{В}$, $\mathcal{E}_2 = 60\,\text{В}$.
}

\variantsplitter

\addpersonalvariant{Владислав Емелин}

\tasknumber{1}%
\task{%
    Получите выражение:
    \begin{enumerate}
        \item площади поперечного сечения проводника через его сопротивление,
        \item сопротивление из закона Ома,
        \item внутреннее сопротивление цепи из закона Ома для полной цепи,
        \item эквивалентное сопротивление $n$ резисторов, соединённых последовательно, каждый сопротивлением $R$.
    \end{enumerate}
}
\solutionspace{40pt}

\tasknumber{2}%
\task{%
    Получите выражение:
    \begin{enumerate}
        \item силы тока через выделяемую мощность и напряжение на резисторе,
        \item силы тока через выделенную теплоту и сопротивление резистора,
        \item напряжение на резисторе через выделяемую мощность и сопротивление резистора,
        \item напряжение на резисторе через выделенную в нём теплоту и сопротивление резистора.
    \end{enumerate}
}
\solutionspace{80pt}

\tasknumber{3}%
\task{%
    Определите ток, протекающий через резистор $R = 15\,\text{Ом}$ и разность потенциалов на нём (см.
    рис.
    на доске),
    если $r_1 = 1\,\text{Ом}$, $r_2 = 2\,\text{Ом}$, $\mathcal{E}_1 = 40\,\text{В}$, $\mathcal{E}_2 = 40\,\text{В}$.
}

\variantsplitter

\addpersonalvariant{Артём Жичин}

\tasknumber{1}%
\task{%
    Получите выражение:
    \begin{enumerate}
        \item площади поперечного сечения проводника через его сопротивление,
        \item удельное сопротивление из закона Ома,
        \item внешнее сопротивление цепи из закона Ома для полной цепи,
        \item эквивалентное сопротивление $n$ резисторов, соединённых последовательно, каждый сопротивлением $R$.
    \end{enumerate}
}
\solutionspace{40pt}

\tasknumber{2}%
\task{%
    Получите выражение:
    \begin{enumerate}
        \item силы тока через выделяемую мощность и разность потенциалов на резисторе,
        \item силы тока через выделенную теплоту и сопротивление резистора,
        \item напряжение на резисторе через выделяемую мощность и силу тока через него,
        \item напряжение на резисторе через выделенную в нём теплоту и сопротивление резистора.
    \end{enumerate}
}
\solutionspace{80pt}

\tasknumber{3}%
\task{%
    Определите ток, протекающий через резистор $R = 20\,\text{Ом}$ и разность потенциалов на нём (см.
    рис.
    на доске),
    если $r_1 = 3\,\text{Ом}$, $r_2 = 3\,\text{Ом}$, $\mathcal{E}_1 = 40\,\text{В}$, $\mathcal{E}_2 = 20\,\text{В}$.
}

\variantsplitter

\addpersonalvariant{Дарья Кошман}

\tasknumber{1}%
\task{%
    Получите выражение:
    \begin{enumerate}
        \item площади поперечного сечения проводника через его сопротивление,
        \item сопротивление из закона Ома,
        \item внешнее сопротивление цепи из закона Ома для полной цепи,
        \item эквивалентное сопротивление $n$ резисторов, соединённых последовательно, каждый сопротивлением $R$.
    \end{enumerate}
}
\solutionspace{40pt}

\tasknumber{2}%
\task{%
    Получите выражение:
    \begin{enumerate}
        \item силы тока через выделяемую мощность и разность потенциалов на резисторе,
        \item силы тока через выделенную теплоту и сопротивление резистора,
        \item напряжение на резисторе через выделяемую мощность и силу тока через него,
        \item напряжение на резисторе через выделенную в нём теплоту и силу тока через него.
    \end{enumerate}
}
\solutionspace{80pt}

\tasknumber{3}%
\task{%
    Определите ток, протекающий через резистор $R = 18\,\text{Ом}$ и разность потенциалов на нём (см.
    рис.
    на доске),
    если $r_1 = 1\,\text{Ом}$, $r_2 = 3\,\text{Ом}$, $\mathcal{E}_1 = 20\,\text{В}$, $\mathcal{E}_2 = 40\,\text{В}$.
}

\variantsplitter

\addpersonalvariant{Анна Кузьмичёва}

\tasknumber{1}%
\task{%
    Получите выражение:
    \begin{enumerate}
        \item площади поперечного сечения проводника через его сопротивление,
        \item сопротивление из закона Ома,
        \item внутреннее сопротивление цепи из закона Ома для полной цепи,
        \item эквивалентное сопротивление $n$ резисторов, соединённых параллельно, каждый сопротивлением $R$.
    \end{enumerate}
}
\solutionspace{40pt}

\tasknumber{2}%
\task{%
    Получите выражение:
    \begin{enumerate}
        \item силы тока через выделяемую мощность и разность потенциалов на резисторе,
        \item силы тока через выделенную теплоту и сопротивление резистора,
        \item напряжение на резисторе через выделяемую мощность и силу тока через него,
        \item напряжение на резисторе через выделенную в нём теплоту и силу тока через него.
    \end{enumerate}
}
\solutionspace{80pt}

\tasknumber{3}%
\task{%
    Определите ток, протекающий через резистор $R = 18\,\text{Ом}$ и разность потенциалов на нём (см.
    рис.
    на доске),
    если $r_1 = 3\,\text{Ом}$, $r_2 = 2\,\text{Ом}$, $\mathcal{E}_1 = 30\,\text{В}$, $\mathcal{E}_2 = 40\,\text{В}$.
}

\variantsplitter

\addpersonalvariant{Алёна Куприянова}

\tasknumber{1}%
\task{%
    Получите выражение:
    \begin{enumerate}
        \item длины проводника через его сопротивление,
        \item сопротивление из закона Ома,
        \item внешнее сопротивление цепи из закона Ома для полной цепи,
        \item эквивалентное сопротивление $n$ резисторов, соединённых параллельно, каждый сопротивлением $R$.
    \end{enumerate}
}
\solutionspace{40pt}

\tasknumber{2}%
\task{%
    Получите выражение:
    \begin{enumerate}
        \item силы тока через выделяемую мощность и напряжение на резисторе,
        \item силы тока через выделенную теплоту и разность потенциалов на резисторе,
        \item напряжение на резисторе через выделяемую мощность и силу тока через него,
        \item напряжение на резисторе через выделенную в нём теплоту и сопротивление резистора.
    \end{enumerate}
}
\solutionspace{80pt}

\tasknumber{3}%
\task{%
    Определите ток, протекающий через резистор $R = 12\,\text{Ом}$ и разность потенциалов на нём (см.
    рис.
    на доске),
    если $r_1 = 2\,\text{Ом}$, $r_2 = 1\,\text{Ом}$, $\mathcal{E}_1 = 40\,\text{В}$, $\mathcal{E}_2 = 40\,\text{В}$.
}

\variantsplitter

\addpersonalvariant{Ярослав Лавровский}

\tasknumber{1}%
\task{%
    Получите выражение:
    \begin{enumerate}
        \item длины проводника через его сопротивление,
        \item удельное сопротивление из закона Ома,
        \item внутреннее сопротивление цепи из закона Ома для полной цепи,
        \item эквивалентное сопротивление $n$ резисторов, соединённых параллельно, каждый сопротивлением $R$.
    \end{enumerate}
}
\solutionspace{40pt}

\tasknumber{2}%
\task{%
    Получите выражение:
    \begin{enumerate}
        \item силы тока через выделяемую мощность и сопротивление резистора,
        \item силы тока через выделенную теплоту и сопротивление резистора,
        \item напряжение на резисторе через выделяемую мощность и сопротивление резистора,
        \item напряжение на резисторе через выделенную в нём теплоту и силу тока через него.
    \end{enumerate}
}
\solutionspace{80pt}

\tasknumber{3}%
\task{%
    Определите ток, протекающий через резистор $R = 20\,\text{Ом}$ и разность потенциалов на нём (см.
    рис.
    на доске),
    если $r_1 = 3\,\text{Ом}$, $r_2 = 3\,\text{Ом}$, $\mathcal{E}_1 = 30\,\text{В}$, $\mathcal{E}_2 = 30\,\text{В}$.
}

\variantsplitter

\addpersonalvariant{Анастасия Ламанова}

\tasknumber{1}%
\task{%
    Получите выражение:
    \begin{enumerate}
        \item длины проводника через его сопротивление,
        \item удельное сопротивление из закона Ома,
        \item внутреннее сопротивление цепи из закона Ома для полной цепи,
        \item эквивалентное сопротивление $n$ резисторов, соединённых последовательно, каждый сопротивлением $R$.
    \end{enumerate}
}
\solutionspace{40pt}

\tasknumber{2}%
\task{%
    Получите выражение:
    \begin{enumerate}
        \item силы тока через выделяемую мощность и разность потенциалов на резисторе,
        \item силы тока через выделенную теплоту и разность потенциалов на резисторе,
        \item напряжение на резисторе через выделяемую мощность и сопротивление резистора,
        \item напряжение на резисторе через выделенную в нём теплоту и силу тока через него.
    \end{enumerate}
}
\solutionspace{80pt}

\tasknumber{3}%
\task{%
    Определите ток, протекающий через резистор $R = 15\,\text{Ом}$ и разность потенциалов на нём (см.
    рис.
    на доске),
    если $r_1 = 2\,\text{Ом}$, $r_2 = 3\,\text{Ом}$, $\mathcal{E}_1 = 60\,\text{В}$, $\mathcal{E}_2 = 40\,\text{В}$.
}

\variantsplitter

\addpersonalvariant{Виктория Легонькова}

\tasknumber{1}%
\task{%
    Получите выражение:
    \begin{enumerate}
        \item площади поперечного сечения проводника через его сопротивление,
        \item сопротивление из закона Ома,
        \item внутреннее сопротивление цепи из закона Ома для полной цепи,
        \item эквивалентное сопротивление $n$ резисторов, соединённых параллельно, каждый сопротивлением $R$.
    \end{enumerate}
}
\solutionspace{40pt}

\tasknumber{2}%
\task{%
    Получите выражение:
    \begin{enumerate}
        \item силы тока через выделяемую мощность и напряжение на резисторе,
        \item силы тока через выделенную теплоту и сопротивление резистора,
        \item напряжение на резисторе через выделяемую мощность и сопротивление резистора,
        \item напряжение на резисторе через выделенную в нём теплоту и сопротивление резистора.
    \end{enumerate}
}
\solutionspace{80pt}

\tasknumber{3}%
\task{%
    Определите ток, протекающий через резистор $R = 12\,\text{Ом}$ и разность потенциалов на нём (см.
    рис.
    на доске),
    если $r_1 = 3\,\text{Ом}$, $r_2 = 2\,\text{Ом}$, $\mathcal{E}_1 = 60\,\text{В}$, $\mathcal{E}_2 = 60\,\text{В}$.
}

\variantsplitter

\addpersonalvariant{Семён Мартынов}

\tasknumber{1}%
\task{%
    Получите выражение:
    \begin{enumerate}
        \item длины проводника через его сопротивление,
        \item удельное сопротивление из закона Ома,
        \item внутреннее сопротивление цепи из закона Ома для полной цепи,
        \item эквивалентное сопротивление $n$ резисторов, соединённых параллельно, каждый сопротивлением $R$.
    \end{enumerate}
}
\solutionspace{40pt}

\tasknumber{2}%
\task{%
    Получите выражение:
    \begin{enumerate}
        \item силы тока через выделяемую мощность и напряжение на резисторе,
        \item силы тока через выделенную теплоту и разность потенциалов на резисторе,
        \item напряжение на резисторе через выделяемую мощность и силу тока через него,
        \item напряжение на резисторе через выделенную в нём теплоту и силу тока через него.
    \end{enumerate}
}
\solutionspace{80pt}

\tasknumber{3}%
\task{%
    Определите ток, протекающий через резистор $R = 20\,\text{Ом}$ и разность потенциалов на нём (см.
    рис.
    на доске),
    если $r_1 = 1\,\text{Ом}$, $r_2 = 1\,\text{Ом}$, $\mathcal{E}_1 = 30\,\text{В}$, $\mathcal{E}_2 = 40\,\text{В}$.
}

\variantsplitter

\addpersonalvariant{Варвара Минаева}

\tasknumber{1}%
\task{%
    Получите выражение:
    \begin{enumerate}
        \item длины проводника через его сопротивление,
        \item сопротивление из закона Ома,
        \item внешнее сопротивление цепи из закона Ома для полной цепи,
        \item эквивалентное сопротивление $n$ резисторов, соединённых последовательно, каждый сопротивлением $R$.
    \end{enumerate}
}
\solutionspace{40pt}

\tasknumber{2}%
\task{%
    Получите выражение:
    \begin{enumerate}
        \item силы тока через выделяемую мощность и сопротивление резистора,
        \item силы тока через выделенную теплоту и напряжение на резисторе,
        \item напряжение на резисторе через выделяемую мощность и силу тока через него,
        \item напряжение на резисторе через выделенную в нём теплоту и сопротивление резистора.
    \end{enumerate}
}
\solutionspace{80pt}

\tasknumber{3}%
\task{%
    Определите ток, протекающий через резистор $R = 12\,\text{Ом}$ и разность потенциалов на нём (см.
    рис.
    на доске),
    если $r_1 = 2\,\text{Ом}$, $r_2 = 3\,\text{Ом}$, $\mathcal{E}_1 = 40\,\text{В}$, $\mathcal{E}_2 = 40\,\text{В}$.
}

\variantsplitter

\addpersonalvariant{Леонид Никитин}

\tasknumber{1}%
\task{%
    Получите выражение:
    \begin{enumerate}
        \item площади поперечного сечения проводника через его сопротивление,
        \item удельное сопротивление из закона Ома,
        \item внутреннее сопротивление цепи из закона Ома для полной цепи,
        \item эквивалентное сопротивление $n$ резисторов, соединённых параллельно, каждый сопротивлением $R$.
    \end{enumerate}
}
\solutionspace{40pt}

\tasknumber{2}%
\task{%
    Получите выражение:
    \begin{enumerate}
        \item силы тока через выделяемую мощность и разность потенциалов на резисторе,
        \item силы тока через выделенную теплоту и сопротивление резистора,
        \item напряжение на резисторе через выделяемую мощность и силу тока через него,
        \item напряжение на резисторе через выделенную в нём теплоту и сопротивление резистора.
    \end{enumerate}
}
\solutionspace{80pt}

\tasknumber{3}%
\task{%
    Определите ток, протекающий через резистор $R = 18\,\text{Ом}$ и разность потенциалов на нём (см.
    рис.
    на доске),
    если $r_1 = 1\,\text{Ом}$, $r_2 = 3\,\text{Ом}$, $\mathcal{E}_1 = 20\,\text{В}$, $\mathcal{E}_2 = 40\,\text{В}$.
}

\variantsplitter

\addpersonalvariant{Тимофей Полетаев}

\tasknumber{1}%
\task{%
    Получите выражение:
    \begin{enumerate}
        \item длины проводника через его сопротивление,
        \item сопротивление из закона Ома,
        \item внешнее сопротивление цепи из закона Ома для полной цепи,
        \item эквивалентное сопротивление $n$ резисторов, соединённых последовательно, каждый сопротивлением $R$.
    \end{enumerate}
}
\solutionspace{40pt}

\tasknumber{2}%
\task{%
    Получите выражение:
    \begin{enumerate}
        \item силы тока через выделяемую мощность и разность потенциалов на резисторе,
        \item силы тока через выделенную теплоту и напряжение на резисторе,
        \item напряжение на резисторе через выделяемую мощность и силу тока через него,
        \item напряжение на резисторе через выделенную в нём теплоту и силу тока через него.
    \end{enumerate}
}
\solutionspace{80pt}

\tasknumber{3}%
\task{%
    Определите ток, протекающий через резистор $R = 20\,\text{Ом}$ и разность потенциалов на нём (см.
    рис.
    на доске),
    если $r_1 = 2\,\text{Ом}$, $r_2 = 2\,\text{Ом}$, $\mathcal{E}_1 = 60\,\text{В}$, $\mathcal{E}_2 = 40\,\text{В}$.
}

\variantsplitter

\addpersonalvariant{Андрей Рожков}

\tasknumber{1}%
\task{%
    Получите выражение:
    \begin{enumerate}
        \item длины проводника через его сопротивление,
        \item удельное сопротивление из закона Ома,
        \item внешнее сопротивление цепи из закона Ома для полной цепи,
        \item эквивалентное сопротивление $n$ резисторов, соединённых параллельно, каждый сопротивлением $R$.
    \end{enumerate}
}
\solutionspace{40pt}

\tasknumber{2}%
\task{%
    Получите выражение:
    \begin{enumerate}
        \item силы тока через выделяемую мощность и сопротивление резистора,
        \item силы тока через выделенную теплоту и напряжение на резисторе,
        \item напряжение на резисторе через выделяемую мощность и силу тока через него,
        \item напряжение на резисторе через выделенную в нём теплоту и силу тока через него.
    \end{enumerate}
}
\solutionspace{80pt}

\tasknumber{3}%
\task{%
    Определите ток, протекающий через резистор $R = 10\,\text{Ом}$ и разность потенциалов на нём (см.
    рис.
    на доске),
    если $r_1 = 1\,\text{Ом}$, $r_2 = 1\,\text{Ом}$, $\mathcal{E}_1 = 30\,\text{В}$, $\mathcal{E}_2 = 30\,\text{В}$.
}

\variantsplitter

\addpersonalvariant{Рената Таржиманова}

\tasknumber{1}%
\task{%
    Получите выражение:
    \begin{enumerate}
        \item площади поперечного сечения проводника через его сопротивление,
        \item удельное сопротивление из закона Ома,
        \item внутреннее сопротивление цепи из закона Ома для полной цепи,
        \item эквивалентное сопротивление $n$ резисторов, соединённых последовательно, каждый сопротивлением $R$.
    \end{enumerate}
}
\solutionspace{40pt}

\tasknumber{2}%
\task{%
    Получите выражение:
    \begin{enumerate}
        \item силы тока через выделяемую мощность и сопротивление резистора,
        \item силы тока через выделенную теплоту и разность потенциалов на резисторе,
        \item напряжение на резисторе через выделяемую мощность и силу тока через него,
        \item напряжение на резисторе через выделенную в нём теплоту и сопротивление резистора.
    \end{enumerate}
}
\solutionspace{80pt}

\tasknumber{3}%
\task{%
    Определите ток, протекающий через резистор $R = 10\,\text{Ом}$ и разность потенциалов на нём (см.
    рис.
    на доске),
    если $r_1 = 3\,\text{Ом}$, $r_2 = 3\,\text{Ом}$, $\mathcal{E}_1 = 30\,\text{В}$, $\mathcal{E}_2 = 60\,\text{В}$.
}

\variantsplitter

\addpersonalvariant{Андрей Щербаков}

\tasknumber{1}%
\task{%
    Получите выражение:
    \begin{enumerate}
        \item площади поперечного сечения проводника через его сопротивление,
        \item удельное сопротивление из закона Ома,
        \item внешнее сопротивление цепи из закона Ома для полной цепи,
        \item эквивалентное сопротивление $n$ резисторов, соединённых параллельно, каждый сопротивлением $R$.
    \end{enumerate}
}
\solutionspace{40pt}

\tasknumber{2}%
\task{%
    Получите выражение:
    \begin{enumerate}
        \item силы тока через выделяемую мощность и напряжение на резисторе,
        \item силы тока через выделенную теплоту и сопротивление резистора,
        \item напряжение на резисторе через выделяемую мощность и сопротивление резистора,
        \item напряжение на резисторе через выделенную в нём теплоту и силу тока через него.
    \end{enumerate}
}
\solutionspace{80pt}

\tasknumber{3}%
\task{%
    Определите ток, протекающий через резистор $R = 15\,\text{Ом}$ и разность потенциалов на нём (см.
    рис.
    на доске),
    если $r_1 = 2\,\text{Ом}$, $r_2 = 2\,\text{Ом}$, $\mathcal{E}_1 = 30\,\text{В}$, $\mathcal{E}_2 = 20\,\text{В}$.
}

\variantsplitter

\addpersonalvariant{Михаил Ярошевский}

\tasknumber{1}%
\task{%
    Получите выражение:
    \begin{enumerate}
        \item площади поперечного сечения проводника через его сопротивление,
        \item удельное сопротивление из закона Ома,
        \item внешнее сопротивление цепи из закона Ома для полной цепи,
        \item эквивалентное сопротивление $n$ резисторов, соединённых параллельно, каждый сопротивлением $R$.
    \end{enumerate}
}
\solutionspace{40pt}

\tasknumber{2}%
\task{%
    Получите выражение:
    \begin{enumerate}
        \item силы тока через выделяемую мощность и напряжение на резисторе,
        \item силы тока через выделенную теплоту и сопротивление резистора,
        \item напряжение на резисторе через выделяемую мощность и сопротивление резистора,
        \item напряжение на резисторе через выделенную в нём теплоту и силу тока через него.
    \end{enumerate}
}
\solutionspace{80pt}

\tasknumber{3}%
\task{%
    Определите ток, протекающий через резистор $R = 10\,\text{Ом}$ и разность потенциалов на нём (см.
    рис.
    на доске),
    если $r_1 = 2\,\text{Ом}$, $r_2 = 1\,\text{Ом}$, $\mathcal{E}_1 = 60\,\text{В}$, $\mathcal{E}_2 = 40\,\text{В}$.
}

\variantsplitter

\addpersonalvariant{Алексей Алимпиев}

\tasknumber{1}%
\task{%
    Получите выражение:
    \begin{enumerate}
        \item площади поперечного сечения проводника через его сопротивление,
        \item сопротивление из закона Ома,
        \item внешнее сопротивление цепи из закона Ома для полной цепи,
        \item эквивалентное сопротивление $n$ резисторов, соединённых параллельно, каждый сопротивлением $R$.
    \end{enumerate}
}
\solutionspace{40pt}

\tasknumber{2}%
\task{%
    Получите выражение:
    \begin{enumerate}
        \item силы тока через выделяемую мощность и напряжение на резисторе,
        \item силы тока через выделенную теплоту и разность потенциалов на резисторе,
        \item напряжение на резисторе через выделяемую мощность и сопротивление резистора,
        \item напряжение на резисторе через выделенную в нём теплоту и силу тока через него.
    \end{enumerate}
}
\solutionspace{80pt}

\tasknumber{3}%
\task{%
    Определите ток, протекающий через резистор $R = 12\,\text{Ом}$ и разность потенциалов на нём (см.
    рис.
    на доске),
    если $r_1 = 3\,\text{Ом}$, $r_2 = 2\,\text{Ом}$, $\mathcal{E}_1 = 20\,\text{В}$, $\mathcal{E}_2 = 40\,\text{В}$.
}

\variantsplitter

\addpersonalvariant{Евгений Васин}

\tasknumber{1}%
\task{%
    Получите выражение:
    \begin{enumerate}
        \item площади поперечного сечения проводника через его сопротивление,
        \item удельное сопротивление из закона Ома,
        \item внутреннее сопротивление цепи из закона Ома для полной цепи,
        \item эквивалентное сопротивление $n$ резисторов, соединённых последовательно, каждый сопротивлением $R$.
    \end{enumerate}
}
\solutionspace{40pt}

\tasknumber{2}%
\task{%
    Получите выражение:
    \begin{enumerate}
        \item силы тока через выделяемую мощность и сопротивление резистора,
        \item силы тока через выделенную теплоту и разность потенциалов на резисторе,
        \item напряжение на резисторе через выделяемую мощность и силу тока через него,
        \item напряжение на резисторе через выделенную в нём теплоту и сопротивление резистора.
    \end{enumerate}
}
\solutionspace{80pt}

\tasknumber{3}%
\task{%
    Определите ток, протекающий через резистор $R = 12\,\text{Ом}$ и разность потенциалов на нём (см.
    рис.
    на доске),
    если $r_1 = 1\,\text{Ом}$, $r_2 = 1\,\text{Ом}$, $\mathcal{E}_1 = 20\,\text{В}$, $\mathcal{E}_2 = 40\,\text{В}$.
}

\variantsplitter

\addpersonalvariant{Вячеслав Волохов}

\tasknumber{1}%
\task{%
    Получите выражение:
    \begin{enumerate}
        \item площади поперечного сечения проводника через его сопротивление,
        \item сопротивление из закона Ома,
        \item внешнее сопротивление цепи из закона Ома для полной цепи,
        \item эквивалентное сопротивление $n$ резисторов, соединённых последовательно, каждый сопротивлением $R$.
    \end{enumerate}
}
\solutionspace{40pt}

\tasknumber{2}%
\task{%
    Получите выражение:
    \begin{enumerate}
        \item силы тока через выделяемую мощность и напряжение на резисторе,
        \item силы тока через выделенную теплоту и сопротивление резистора,
        \item напряжение на резисторе через выделяемую мощность и сопротивление резистора,
        \item напряжение на резисторе через выделенную в нём теплоту и сопротивление резистора.
    \end{enumerate}
}
\solutionspace{80pt}

\tasknumber{3}%
\task{%
    Определите ток, протекающий через резистор $R = 18\,\text{Ом}$ и разность потенциалов на нём (см.
    рис.
    на доске),
    если $r_1 = 1\,\text{Ом}$, $r_2 = 3\,\text{Ом}$, $\mathcal{E}_1 = 30\,\text{В}$, $\mathcal{E}_2 = 40\,\text{В}$.
}

\variantsplitter

\addpersonalvariant{Герман Говоров}

\tasknumber{1}%
\task{%
    Получите выражение:
    \begin{enumerate}
        \item длины проводника через его сопротивление,
        \item сопротивление из закона Ома,
        \item внутреннее сопротивление цепи из закона Ома для полной цепи,
        \item эквивалентное сопротивление $n$ резисторов, соединённых параллельно, каждый сопротивлением $R$.
    \end{enumerate}
}
\solutionspace{40pt}

\tasknumber{2}%
\task{%
    Получите выражение:
    \begin{enumerate}
        \item силы тока через выделяемую мощность и сопротивление резистора,
        \item силы тока через выделенную теплоту и сопротивление резистора,
        \item напряжение на резисторе через выделяемую мощность и сопротивление резистора,
        \item напряжение на резисторе через выделенную в нём теплоту и сопротивление резистора.
    \end{enumerate}
}
\solutionspace{80pt}

\tasknumber{3}%
\task{%
    Определите ток, протекающий через резистор $R = 12\,\text{Ом}$ и разность потенциалов на нём (см.
    рис.
    на доске),
    если $r_1 = 1\,\text{Ом}$, $r_2 = 2\,\text{Ом}$, $\mathcal{E}_1 = 60\,\text{В}$, $\mathcal{E}_2 = 60\,\text{В}$.
}

\variantsplitter

\addpersonalvariant{София Журавлёва}

\tasknumber{1}%
\task{%
    Получите выражение:
    \begin{enumerate}
        \item площади поперечного сечения проводника через его сопротивление,
        \item удельное сопротивление из закона Ома,
        \item внутреннее сопротивление цепи из закона Ома для полной цепи,
        \item эквивалентное сопротивление $n$ резисторов, соединённых параллельно, каждый сопротивлением $R$.
    \end{enumerate}
}
\solutionspace{40pt}

\tasknumber{2}%
\task{%
    Получите выражение:
    \begin{enumerate}
        \item силы тока через выделяемую мощность и напряжение на резисторе,
        \item силы тока через выделенную теплоту и разность потенциалов на резисторе,
        \item напряжение на резисторе через выделяемую мощность и сопротивление резистора,
        \item напряжение на резисторе через выделенную в нём теплоту и сопротивление резистора.
    \end{enumerate}
}
\solutionspace{80pt}

\tasknumber{3}%
\task{%
    Определите ток, протекающий через резистор $R = 18\,\text{Ом}$ и разность потенциалов на нём (см.
    рис.
    на доске),
    если $r_1 = 2\,\text{Ом}$, $r_2 = 2\,\text{Ом}$, $\mathcal{E}_1 = 30\,\text{В}$, $\mathcal{E}_2 = 60\,\text{В}$.
}

\variantsplitter

\addpersonalvariant{Константин Козлов}

\tasknumber{1}%
\task{%
    Получите выражение:
    \begin{enumerate}
        \item длины проводника через его сопротивление,
        \item удельное сопротивление из закона Ома,
        \item внешнее сопротивление цепи из закона Ома для полной цепи,
        \item эквивалентное сопротивление $n$ резисторов, соединённых последовательно, каждый сопротивлением $R$.
    \end{enumerate}
}
\solutionspace{40pt}

\tasknumber{2}%
\task{%
    Получите выражение:
    \begin{enumerate}
        \item силы тока через выделяемую мощность и сопротивление резистора,
        \item силы тока через выделенную теплоту и напряжение на резисторе,
        \item напряжение на резисторе через выделяемую мощность и силу тока через него,
        \item напряжение на резисторе через выделенную в нём теплоту и силу тока через него.
    \end{enumerate}
}
\solutionspace{80pt}

\tasknumber{3}%
\task{%
    Определите ток, протекающий через резистор $R = 10\,\text{Ом}$ и разность потенциалов на нём (см.
    рис.
    на доске),
    если $r_1 = 3\,\text{Ом}$, $r_2 = 2\,\text{Ом}$, $\mathcal{E}_1 = 30\,\text{В}$, $\mathcal{E}_2 = 30\,\text{В}$.
}

\variantsplitter

\addpersonalvariant{Наталья Кравченко}

\tasknumber{1}%
\task{%
    Получите выражение:
    \begin{enumerate}
        \item длины проводника через его сопротивление,
        \item удельное сопротивление из закона Ома,
        \item внутреннее сопротивление цепи из закона Ома для полной цепи,
        \item эквивалентное сопротивление $n$ резисторов, соединённых последовательно, каждый сопротивлением $R$.
    \end{enumerate}
}
\solutionspace{40pt}

\tasknumber{2}%
\task{%
    Получите выражение:
    \begin{enumerate}
        \item силы тока через выделяемую мощность и напряжение на резисторе,
        \item силы тока через выделенную теплоту и разность потенциалов на резисторе,
        \item напряжение на резисторе через выделяемую мощность и силу тока через него,
        \item напряжение на резисторе через выделенную в нём теплоту и силу тока через него.
    \end{enumerate}
}
\solutionspace{80pt}

\tasknumber{3}%
\task{%
    Определите ток, протекающий через резистор $R = 10\,\text{Ом}$ и разность потенциалов на нём (см.
    рис.
    на доске),
    если $r_1 = 1\,\text{Ом}$, $r_2 = 2\,\text{Ом}$, $\mathcal{E}_1 = 30\,\text{В}$, $\mathcal{E}_2 = 30\,\text{В}$.
}

\variantsplitter

\addpersonalvariant{Матвей Кузьмин}

\tasknumber{1}%
\task{%
    Получите выражение:
    \begin{enumerate}
        \item длины проводника через его сопротивление,
        \item удельное сопротивление из закона Ома,
        \item внутреннее сопротивление цепи из закона Ома для полной цепи,
        \item эквивалентное сопротивление $n$ резисторов, соединённых параллельно, каждый сопротивлением $R$.
    \end{enumerate}
}
\solutionspace{40pt}

\tasknumber{2}%
\task{%
    Получите выражение:
    \begin{enumerate}
        \item силы тока через выделяемую мощность и разность потенциалов на резисторе,
        \item силы тока через выделенную теплоту и напряжение на резисторе,
        \item напряжение на резисторе через выделяемую мощность и силу тока через него,
        \item напряжение на резисторе через выделенную в нём теплоту и сопротивление резистора.
    \end{enumerate}
}
\solutionspace{80pt}

\tasknumber{3}%
\task{%
    Определите ток, протекающий через резистор $R = 20\,\text{Ом}$ и разность потенциалов на нём (см.
    рис.
    на доске),
    если $r_1 = 3\,\text{Ом}$, $r_2 = 1\,\text{Ом}$, $\mathcal{E}_1 = 30\,\text{В}$, $\mathcal{E}_2 = 40\,\text{В}$.
}

\variantsplitter

\addpersonalvariant{Сергей Малышев}

\tasknumber{1}%
\task{%
    Получите выражение:
    \begin{enumerate}
        \item длины проводника через его сопротивление,
        \item сопротивление из закона Ома,
        \item внешнее сопротивление цепи из закона Ома для полной цепи,
        \item эквивалентное сопротивление $n$ резисторов, соединённых последовательно, каждый сопротивлением $R$.
    \end{enumerate}
}
\solutionspace{40pt}

\tasknumber{2}%
\task{%
    Получите выражение:
    \begin{enumerate}
        \item силы тока через выделяемую мощность и сопротивление резистора,
        \item силы тока через выделенную теплоту и напряжение на резисторе,
        \item напряжение на резисторе через выделяемую мощность и силу тока через него,
        \item напряжение на резисторе через выделенную в нём теплоту и сопротивление резистора.
    \end{enumerate}
}
\solutionspace{80pt}

\tasknumber{3}%
\task{%
    Определите ток, протекающий через резистор $R = 12\,\text{Ом}$ и разность потенциалов на нём (см.
    рис.
    на доске),
    если $r_1 = 2\,\text{Ом}$, $r_2 = 1\,\text{Ом}$, $\mathcal{E}_1 = 60\,\text{В}$, $\mathcal{E}_2 = 60\,\text{В}$.
}

\variantsplitter

\addpersonalvariant{Алина Полканова}

\tasknumber{1}%
\task{%
    Получите выражение:
    \begin{enumerate}
        \item площади поперечного сечения проводника через его сопротивление,
        \item сопротивление из закона Ома,
        \item внутреннее сопротивление цепи из закона Ома для полной цепи,
        \item эквивалентное сопротивление $n$ резисторов, соединённых параллельно, каждый сопротивлением $R$.
    \end{enumerate}
}
\solutionspace{40pt}

\tasknumber{2}%
\task{%
    Получите выражение:
    \begin{enumerate}
        \item силы тока через выделяемую мощность и напряжение на резисторе,
        \item силы тока через выделенную теплоту и напряжение на резисторе,
        \item напряжение на резисторе через выделяемую мощность и силу тока через него,
        \item напряжение на резисторе через выделенную в нём теплоту и сопротивление резистора.
    \end{enumerate}
}
\solutionspace{80pt}

\tasknumber{3}%
\task{%
    Определите ток, протекающий через резистор $R = 10\,\text{Ом}$ и разность потенциалов на нём (см.
    рис.
    на доске),
    если $r_1 = 3\,\text{Ом}$, $r_2 = 1\,\text{Ом}$, $\mathcal{E}_1 = 20\,\text{В}$, $\mathcal{E}_2 = 40\,\text{В}$.
}

\variantsplitter

\addpersonalvariant{Сергей Пономарёв}

\tasknumber{1}%
\task{%
    Получите выражение:
    \begin{enumerate}
        \item площади поперечного сечения проводника через его сопротивление,
        \item сопротивление из закона Ома,
        \item внешнее сопротивление цепи из закона Ома для полной цепи,
        \item эквивалентное сопротивление $n$ резисторов, соединённых параллельно, каждый сопротивлением $R$.
    \end{enumerate}
}
\solutionspace{40pt}

\tasknumber{2}%
\task{%
    Получите выражение:
    \begin{enumerate}
        \item силы тока через выделяемую мощность и разность потенциалов на резисторе,
        \item силы тока через выделенную теплоту и разность потенциалов на резисторе,
        \item напряжение на резисторе через выделяемую мощность и силу тока через него,
        \item напряжение на резисторе через выделенную в нём теплоту и сопротивление резистора.
    \end{enumerate}
}
\solutionspace{80pt}

\tasknumber{3}%
\task{%
    Определите ток, протекающий через резистор $R = 20\,\text{Ом}$ и разность потенциалов на нём (см.
    рис.
    на доске),
    если $r_1 = 1\,\text{Ом}$, $r_2 = 2\,\text{Ом}$, $\mathcal{E}_1 = 30\,\text{В}$, $\mathcal{E}_2 = 40\,\text{В}$.
}

\variantsplitter

\addpersonalvariant{Егор Свистушкин}

\tasknumber{1}%
\task{%
    Получите выражение:
    \begin{enumerate}
        \item длины проводника через его сопротивление,
        \item удельное сопротивление из закона Ома,
        \item внешнее сопротивление цепи из закона Ома для полной цепи,
        \item эквивалентное сопротивление $n$ резисторов, соединённых параллельно, каждый сопротивлением $R$.
    \end{enumerate}
}
\solutionspace{40pt}

\tasknumber{2}%
\task{%
    Получите выражение:
    \begin{enumerate}
        \item силы тока через выделяемую мощность и сопротивление резистора,
        \item силы тока через выделенную теплоту и разность потенциалов на резисторе,
        \item напряжение на резисторе через выделяемую мощность и сопротивление резистора,
        \item напряжение на резисторе через выделенную в нём теплоту и сопротивление резистора.
    \end{enumerate}
}
\solutionspace{80pt}

\tasknumber{3}%
\task{%
    Определите ток, протекающий через резистор $R = 18\,\text{Ом}$ и разность потенциалов на нём (см.
    рис.
    на доске),
    если $r_1 = 1\,\text{Ом}$, $r_2 = 2\,\text{Ом}$, $\mathcal{E}_1 = 60\,\text{В}$, $\mathcal{E}_2 = 60\,\text{В}$.
}

\variantsplitter

\addpersonalvariant{Дмитрий Соколов}

\tasknumber{1}%
\task{%
    Получите выражение:
    \begin{enumerate}
        \item площади поперечного сечения проводника через его сопротивление,
        \item сопротивление из закона Ома,
        \item внешнее сопротивление цепи из закона Ома для полной цепи,
        \item эквивалентное сопротивление $n$ резисторов, соединённых последовательно, каждый сопротивлением $R$.
    \end{enumerate}
}
\solutionspace{40pt}

\tasknumber{2}%
\task{%
    Получите выражение:
    \begin{enumerate}
        \item силы тока через выделяемую мощность и сопротивление резистора,
        \item силы тока через выделенную теплоту и напряжение на резисторе,
        \item напряжение на резисторе через выделяемую мощность и сопротивление резистора,
        \item напряжение на резисторе через выделенную в нём теплоту и силу тока через него.
    \end{enumerate}
}
\solutionspace{80pt}

\tasknumber{3}%
\task{%
    Определите ток, протекающий через резистор $R = 10\,\text{Ом}$ и разность потенциалов на нём (см.
    рис.
    на доске),
    если $r_1 = 1\,\text{Ом}$, $r_2 = 3\,\text{Ом}$, $\mathcal{E}_1 = 60\,\text{В}$, $\mathcal{E}_2 = 30\,\text{В}$.
}

\variantsplitter

\addpersonalvariant{Арсений Трофимов}

\tasknumber{1}%
\task{%
    Получите выражение:
    \begin{enumerate}
        \item площади поперечного сечения проводника через его сопротивление,
        \item удельное сопротивление из закона Ома,
        \item внешнее сопротивление цепи из закона Ома для полной цепи,
        \item эквивалентное сопротивление $n$ резисторов, соединённых параллельно, каждый сопротивлением $R$.
    \end{enumerate}
}
\solutionspace{40pt}

\tasknumber{2}%
\task{%
    Получите выражение:
    \begin{enumerate}
        \item силы тока через выделяемую мощность и разность потенциалов на резисторе,
        \item силы тока через выделенную теплоту и разность потенциалов на резисторе,
        \item напряжение на резисторе через выделяемую мощность и силу тока через него,
        \item напряжение на резисторе через выделенную в нём теплоту и силу тока через него.
    \end{enumerate}
}
\solutionspace{80pt}

\tasknumber{3}%
\task{%
    Определите ток, протекающий через резистор $R = 10\,\text{Ом}$ и разность потенциалов на нём (см.
    рис.
    на доске),
    если $r_1 = 1\,\text{Ом}$, $r_2 = 2\,\text{Ом}$, $\mathcal{E}_1 = 40\,\text{В}$, $\mathcal{E}_2 = 20\,\text{В}$.
}
% autogenerated
