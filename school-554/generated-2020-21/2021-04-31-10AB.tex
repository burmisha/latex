\setdate{31~апреля~2021}
\setclass{10«АБ»}

\addpersonalvariant{Михаил Бурмистров}

\tasknumber{1}%
\task{%
    Получите выражение:
    \begin{enumerate}
        \item длины проводника через его сопротивление,
        \item сопротивление из закона Ома,
        \item внешнее сопротивление цепи из закона Ома для полной цепи,
        \item эквивалентное сопротивление $n$ резисторов, соединённых параллельно, каждый сопротивлением $R$.
    \end{enumerate}
}
\solutionspace{40pt}

\tasknumber{2}%
\task{%
    Получите выражение:
    \begin{enumerate}
        \item силы тока через выделяемую мощность и сопротивление резистора,
        \item силы тока через выделенную теплоту и напряжение на резисторе,
        \item напряжение на резисторе через выделяемую мощность и силу тока через него,
        \item напряжение на резисторе через выделенную в нём теплоту и сопротивление резистора.
    \end{enumerate}
}
\solutionspace{80pt}

\tasknumber{3}%
\task{%
    Определите ток, протекающий через резистор $R = 18\,\text{Ом}$ и разность потенциалов на нём (см.
    рис.
    на доске),
    если $r_1 = 1\,\text{Ом}$, $r_2 = 2\,\text{Ом}$, $\ele_1 = 60\,\text{В}$, $\ele_2 = 20\,\text{В}$.
}
\answer{%
    Обозначим на рисунке все токи: направление произвольно, но его надо зафиксировать.
    Всего на рисунке 3 контура и 2 узла.
    Поэтому можно записать $3 - 1 = 2$ уравнения законов Кирхгофа для замкнутого контура и $2 - 1 = 1$ — для узлов
    (остальные уравнения тоже можно записать, но они не дадут полезной информации, а будут лишь следствиями уже записанных).

    Отметим на рисунке 2 контура (и не забуем указать направление) и 1 узел (точка «1»ы, выделена жирным).
    Выбор контуров и узлов не критичен: получившаяся система может быть чуть проще или сложнее, но не слишком.

    \begin{tikzpicture}[circuit ee IEC, thick]
        \draw  (0, 0) to [current direction={near end, info=$\eli_1$}] (0, 3)
                to [battery={rotate=-180,info={$\ele_1, r_1$}}]
                (3, 3)
                to [battery={info'={$\ele_2, r_2$}}]
                (6, 3) to [current direction'={near start, info=$\eli_2$}] (6, 0) -- (0, 0)
                (3, 0) to [current direction={near start, info=$\eli$}, resistor={near end, info=$R$}] (3, 3);
        \draw [-{Latex},color=red] (1.2, 1.7) arc [start angle = 135, end angle = -160, radius = 0.6];
        \draw [-{Latex},color=blue] (4.2, 1.7) arc [start angle = 135, end angle = -160, radius = 0.6];
        \node [contact,color=green!71!black] (bottomc) at (3, 0) {};
        \node [below] (bottom) at (3, 0) {$2$};
        \node [above] (top) at (3, 3) {$1$};
    \end{tikzpicture}

    \begin{align*}
        &\begin{cases}
            {\color{red} \ele_1 = \eli_1 r_1 - \eli R}, \\
            {\color{blue} -\ele_2 = -\eli_2 r_2 + \eli R}, \\
            {\color{green!71!black} - \eli - \eli_1 - \eli_2 = 0 };
        \end{cases}
        \qquad \implies \qquad
        \begin{cases}
            \eli_1 = \frac{\ele_1 + \eli R}{r_1}, \\
            \eli_2 = \frac{\ele_2 + \eli R}{r_2}, \\
            \eli + \eli_1 + \eli_2 = 0;
        \end{cases} \implies \\
        &\implies
         \eli + \frac{\ele_1 + \eli R}{r_1:L} + \frac{\ele_2 + \eli R}{r_2:L} = 0, \\
        &\eli\cbr{ 1 + \frac R{r_1:L} + \frac R{r_2:L}} + \frac{\ele_1 }{r_1:L} + \frac{\ele_2 }{r_2:L} = 0, \\
        &\eli
            = - \frac{\frac{\ele_1 }{r_1:L} + \frac{\ele_2 }{r_2:L}}{ 1 + \frac R{r_1:L} + \frac R{r_2:L}}
            = - \frac{\frac{60\,\text{В}}{1\,\text{Ом}} + \frac{20\,\text{В}}{2\,\text{Ом}}}{ 1 + \frac{18\,\text{Ом}}{1\,\text{Ом}} + \frac{18\,\text{Ом}}{2\,\text{Ом}}}
            = - \frac52\units{А}
            \approx -2{,}50\,\text{А}, \\
        &U  = \varphi_2 - \varphi_1 = \eli R
            = - \frac{\frac{\ele_1 }{r_1:L} + \frac{\ele_2 }{r_2:L}}{ 1 + \frac R{r_1:L} + \frac R{r_2:L}} R
            \approx -45{,}00\,\text{В}.
    \end{align*}
    Оба ответа отрицательны, потому что мы изначально «не угадали» с направлением тока.
    Расчёт же показал,
    что ток через резистор $R$ течёт в противоположную сторону: вниз на рисунке, а потенциал точки 1 больше потенциала точки 2,
    а электрический ток ожидаемо течёт из точки с большим потенциалов в точку с меньшим.

    Кстати, если продолжить расчёт и вычислить значения ещё двух токов (формулы для $\eli_1$ и $\eli_2$, куда подставлять, выписаны выше),
    то по их знакам можно будет понять: угадали ли мы с их направлением или нет.
}

\variantsplitter

\addpersonalvariant{Ирина Ан}

\tasknumber{1}%
\task{%
    Получите выражение:
    \begin{enumerate}
        \item площади поперечного сечения проводника через его сопротивление,
        \item удельное сопротивление из закона Ома,
        \item внешнее сопротивление цепи из закона Ома для полной цепи,
        \item эквивалентное сопротивление $n$ резисторов, соединённых параллельно, каждый сопротивлением $R$.
    \end{enumerate}
}
\solutionspace{40pt}

\tasknumber{2}%
\task{%
    Получите выражение:
    \begin{enumerate}
        \item силы тока через выделяемую мощность и сопротивление резистора,
        \item силы тока через выделенную теплоту и разность потенциалов на резисторе,
        \item напряжение на резисторе через выделяемую мощность и силу тока через него,
        \item напряжение на резисторе через выделенную в нём теплоту и сопротивление резистора.
    \end{enumerate}
}
\solutionspace{80pt}

\tasknumber{3}%
\task{%
    Определите ток, протекающий через резистор $R = 10\,\text{Ом}$ и разность потенциалов на нём (см.
    рис.
    на доске),
    если $r_1 = 1\,\text{Ом}$, $r_2 = 1\,\text{Ом}$, $\ele_1 = 30\,\text{В}$, $\ele_2 = 40\,\text{В}$.
}
\answer{%
    Обозначим на рисунке все токи: направление произвольно, но его надо зафиксировать.
    Всего на рисунке 3 контура и 2 узла.
    Поэтому можно записать $3 - 1 = 2$ уравнения законов Кирхгофа для замкнутого контура и $2 - 1 = 1$ — для узлов
    (остальные уравнения тоже можно записать, но они не дадут полезной информации, а будут лишь следствиями уже записанных).

    Отметим на рисунке 2 контура (и не забуем указать направление) и 1 узел (точка «1»ы, выделена жирным).
    Выбор контуров и узлов не критичен: получившаяся система может быть чуть проще или сложнее, но не слишком.

    \begin{tikzpicture}[circuit ee IEC, thick]
        \draw  (0, 0) to [current direction={near end, info=$\eli_1$}] (0, 3)
                to [battery={rotate=-180,info={$\ele_1, r_1$}}]
                (3, 3)
                to [battery={info'={$\ele_2, r_2$}}]
                (6, 3) to [current direction'={near start, info=$\eli_2$}] (6, 0) -- (0, 0)
                (3, 0) to [current direction={near start, info=$\eli$}, resistor={near end, info=$R$}] (3, 3);
        \draw [-{Latex},color=red] (1.2, 1.7) arc [start angle = 135, end angle = -160, radius = 0.6];
        \draw [-{Latex},color=blue] (4.2, 1.7) arc [start angle = 135, end angle = -160, radius = 0.6];
        \node [contact,color=green!71!black] (bottomc) at (3, 0) {};
        \node [below] (bottom) at (3, 0) {$2$};
        \node [above] (top) at (3, 3) {$1$};
    \end{tikzpicture}

    \begin{align*}
        &\begin{cases}
            {\color{red} \ele_1 = \eli_1 r_1 - \eli R}, \\
            {\color{blue} -\ele_2 = -\eli_2 r_2 + \eli R}, \\
            {\color{green!71!black} - \eli - \eli_1 - \eli_2 = 0 };
        \end{cases}
        \qquad \implies \qquad
        \begin{cases}
            \eli_1 = \frac{\ele_1 + \eli R}{r_1}, \\
            \eli_2 = \frac{\ele_2 + \eli R}{r_2}, \\
            \eli + \eli_1 + \eli_2 = 0;
        \end{cases} \implies \\
        &\implies
         \eli + \frac{\ele_1 + \eli R}{r_1:L} + \frac{\ele_2 + \eli R}{r_2:L} = 0, \\
        &\eli\cbr{ 1 + \frac R{r_1:L} + \frac R{r_2:L}} + \frac{\ele_1 }{r_1:L} + \frac{\ele_2 }{r_2:L} = 0, \\
        &\eli
            = - \frac{\frac{\ele_1 }{r_1:L} + \frac{\ele_2 }{r_2:L}}{ 1 + \frac R{r_1:L} + \frac R{r_2:L}}
            = - \frac{\frac{30\,\text{В}}{1\,\text{Ом}} + \frac{40\,\text{В}}{1\,\text{Ом}}}{ 1 + \frac{10\,\text{Ом}}{1\,\text{Ом}} + \frac{10\,\text{Ом}}{1\,\text{Ом}}}
            = - \frac{10}3\units{А}
            \approx -3{,}30\,\text{А}, \\
        &U  = \varphi_2 - \varphi_1 = \eli R
            = - \frac{\frac{\ele_1 }{r_1:L} + \frac{\ele_2 }{r_2:L}}{ 1 + \frac R{r_1:L} + \frac R{r_2:L}} R
            \approx -33{,}30\,\text{В}.
    \end{align*}
    Оба ответа отрицательны, потому что мы изначально «не угадали» с направлением тока.
    Расчёт же показал,
    что ток через резистор $R$ течёт в противоположную сторону: вниз на рисунке, а потенциал точки 1 больше потенциала точки 2,
    а электрический ток ожидаемо течёт из точки с большим потенциалов в точку с меньшим.

    Кстати, если продолжить расчёт и вычислить значения ещё двух токов (формулы для $\eli_1$ и $\eli_2$, куда подставлять, выписаны выше),
    то по их знакам можно будет понять: угадали ли мы с их направлением или нет.
}

\variantsplitter

\addpersonalvariant{Софья Андрианова}

\tasknumber{1}%
\task{%
    Получите выражение:
    \begin{enumerate}
        \item площади поперечного сечения проводника через его сопротивление,
        \item сопротивление из закона Ома,
        \item внешнее сопротивление цепи из закона Ома для полной цепи,
        \item эквивалентное сопротивление $n$ резисторов, соединённых параллельно, каждый сопротивлением $R$.
    \end{enumerate}
}
\solutionspace{40pt}

\tasknumber{2}%
\task{%
    Получите выражение:
    \begin{enumerate}
        \item силы тока через выделяемую мощность и сопротивление резистора,
        \item силы тока через выделенную теплоту и сопротивление резистора,
        \item напряжение на резисторе через выделяемую мощность и силу тока через него,
        \item напряжение на резисторе через выделенную в нём теплоту и сопротивление резистора.
    \end{enumerate}
}
\solutionspace{80pt}

\tasknumber{3}%
\task{%
    Определите ток, протекающий через резистор $R = 20\,\text{Ом}$ и разность потенциалов на нём (см.
    рис.
    на доске),
    если $r_1 = 2\,\text{Ом}$, $r_2 = 2\,\text{Ом}$, $\ele_1 = 20\,\text{В}$, $\ele_2 = 60\,\text{В}$.
}
\answer{%
    Обозначим на рисунке все токи: направление произвольно, но его надо зафиксировать.
    Всего на рисунке 3 контура и 2 узла.
    Поэтому можно записать $3 - 1 = 2$ уравнения законов Кирхгофа для замкнутого контура и $2 - 1 = 1$ — для узлов
    (остальные уравнения тоже можно записать, но они не дадут полезной информации, а будут лишь следствиями уже записанных).

    Отметим на рисунке 2 контура (и не забуем указать направление) и 1 узел (точка «1»ы, выделена жирным).
    Выбор контуров и узлов не критичен: получившаяся система может быть чуть проще или сложнее, но не слишком.

    \begin{tikzpicture}[circuit ee IEC, thick]
        \draw  (0, 0) to [current direction={near end, info=$\eli_1$}] (0, 3)
                to [battery={rotate=-180,info={$\ele_1, r_1$}}]
                (3, 3)
                to [battery={info'={$\ele_2, r_2$}}]
                (6, 3) to [current direction'={near start, info=$\eli_2$}] (6, 0) -- (0, 0)
                (3, 0) to [current direction={near start, info=$\eli$}, resistor={near end, info=$R$}] (3, 3);
        \draw [-{Latex},color=red] (1.2, 1.7) arc [start angle = 135, end angle = -160, radius = 0.6];
        \draw [-{Latex},color=blue] (4.2, 1.7) arc [start angle = 135, end angle = -160, radius = 0.6];
        \node [contact,color=green!71!black] (bottomc) at (3, 0) {};
        \node [below] (bottom) at (3, 0) {$2$};
        \node [above] (top) at (3, 3) {$1$};
    \end{tikzpicture}

    \begin{align*}
        &\begin{cases}
            {\color{red} \ele_1 = \eli_1 r_1 - \eli R}, \\
            {\color{blue} -\ele_2 = -\eli_2 r_2 + \eli R}, \\
            {\color{green!71!black} - \eli - \eli_1 - \eli_2 = 0 };
        \end{cases}
        \qquad \implies \qquad
        \begin{cases}
            \eli_1 = \frac{\ele_1 + \eli R}{r_1}, \\
            \eli_2 = \frac{\ele_2 + \eli R}{r_2}, \\
            \eli + \eli_1 + \eli_2 = 0;
        \end{cases} \implies \\
        &\implies
         \eli + \frac{\ele_1 + \eli R}{r_1:L} + \frac{\ele_2 + \eli R}{r_2:L} = 0, \\
        &\eli\cbr{ 1 + \frac R{r_1:L} + \frac R{r_2:L}} + \frac{\ele_1 }{r_1:L} + \frac{\ele_2 }{r_2:L} = 0, \\
        &\eli
            = - \frac{\frac{\ele_1 }{r_1:L} + \frac{\ele_2 }{r_2:L}}{ 1 + \frac R{r_1:L} + \frac R{r_2:L}}
            = - \frac{\frac{20\,\text{В}}{2\,\text{Ом}} + \frac{60\,\text{В}}{2\,\text{Ом}}}{ 1 + \frac{20\,\text{Ом}}{2\,\text{Ом}} + \frac{20\,\text{Ом}}{2\,\text{Ом}}}
            = - \frac{40}{21}\units{А}
            \approx -1{,}900\,\text{А}, \\
        &U  = \varphi_2 - \varphi_1 = \eli R
            = - \frac{\frac{\ele_1 }{r_1:L} + \frac{\ele_2 }{r_2:L}}{ 1 + \frac R{r_1:L} + \frac R{r_2:L}} R
            \approx -38{,}10\,\text{В}.
    \end{align*}
    Оба ответа отрицательны, потому что мы изначально «не угадали» с направлением тока.
    Расчёт же показал,
    что ток через резистор $R$ течёт в противоположную сторону: вниз на рисунке, а потенциал точки 1 больше потенциала точки 2,
    а электрический ток ожидаемо течёт из точки с большим потенциалов в точку с меньшим.

    Кстати, если продолжить расчёт и вычислить значения ещё двух токов (формулы для $\eli_1$ и $\eli_2$, куда подставлять, выписаны выше),
    то по их знакам можно будет понять: угадали ли мы с их направлением или нет.
}

\variantsplitter

\addpersonalvariant{Владимир Артемчук}

\tasknumber{1}%
\task{%
    Получите выражение:
    \begin{enumerate}
        \item длины проводника через его сопротивление,
        \item удельное сопротивление из закона Ома,
        \item внешнее сопротивление цепи из закона Ома для полной цепи,
        \item эквивалентное сопротивление $n$ резисторов, соединённых последовательно, каждый сопротивлением $R$.
    \end{enumerate}
}
\solutionspace{40pt}

\tasknumber{2}%
\task{%
    Получите выражение:
    \begin{enumerate}
        \item силы тока через выделяемую мощность и сопротивление резистора,
        \item силы тока через выделенную теплоту и напряжение на резисторе,
        \item напряжение на резисторе через выделяемую мощность и силу тока через него,
        \item напряжение на резисторе через выделенную в нём теплоту и силу тока через него.
    \end{enumerate}
}
\solutionspace{80pt}

\tasknumber{3}%
\task{%
    Определите ток, протекающий через резистор $R = 20\,\text{Ом}$ и разность потенциалов на нём (см.
    рис.
    на доске),
    если $r_1 = 1\,\text{Ом}$, $r_2 = 1\,\text{Ом}$, $\ele_1 = 60\,\text{В}$, $\ele_2 = 40\,\text{В}$.
}
\answer{%
    Обозначим на рисунке все токи: направление произвольно, но его надо зафиксировать.
    Всего на рисунке 3 контура и 2 узла.
    Поэтому можно записать $3 - 1 = 2$ уравнения законов Кирхгофа для замкнутого контура и $2 - 1 = 1$ — для узлов
    (остальные уравнения тоже можно записать, но они не дадут полезной информации, а будут лишь следствиями уже записанных).

    Отметим на рисунке 2 контура (и не забуем указать направление) и 1 узел (точка «1»ы, выделена жирным).
    Выбор контуров и узлов не критичен: получившаяся система может быть чуть проще или сложнее, но не слишком.

    \begin{tikzpicture}[circuit ee IEC, thick]
        \draw  (0, 0) to [current direction={near end, info=$\eli_1$}] (0, 3)
                to [battery={rotate=-180,info={$\ele_1, r_1$}}]
                (3, 3)
                to [battery={info'={$\ele_2, r_2$}}]
                (6, 3) to [current direction'={near start, info=$\eli_2$}] (6, 0) -- (0, 0)
                (3, 0) to [current direction={near start, info=$\eli$}, resistor={near end, info=$R$}] (3, 3);
        \draw [-{Latex},color=red] (1.2, 1.7) arc [start angle = 135, end angle = -160, radius = 0.6];
        \draw [-{Latex},color=blue] (4.2, 1.7) arc [start angle = 135, end angle = -160, radius = 0.6];
        \node [contact,color=green!71!black] (bottomc) at (3, 0) {};
        \node [below] (bottom) at (3, 0) {$2$};
        \node [above] (top) at (3, 3) {$1$};
    \end{tikzpicture}

    \begin{align*}
        &\begin{cases}
            {\color{red} \ele_1 = \eli_1 r_1 - \eli R}, \\
            {\color{blue} -\ele_2 = -\eli_2 r_2 + \eli R}, \\
            {\color{green!71!black} - \eli - \eli_1 - \eli_2 = 0 };
        \end{cases}
        \qquad \implies \qquad
        \begin{cases}
            \eli_1 = \frac{\ele_1 + \eli R}{r_1}, \\
            \eli_2 = \frac{\ele_2 + \eli R}{r_2}, \\
            \eli + \eli_1 + \eli_2 = 0;
        \end{cases} \implies \\
        &\implies
         \eli + \frac{\ele_1 + \eli R}{r_1:L} + \frac{\ele_2 + \eli R}{r_2:L} = 0, \\
        &\eli\cbr{ 1 + \frac R{r_1:L} + \frac R{r_2:L}} + \frac{\ele_1 }{r_1:L} + \frac{\ele_2 }{r_2:L} = 0, \\
        &\eli
            = - \frac{\frac{\ele_1 }{r_1:L} + \frac{\ele_2 }{r_2:L}}{ 1 + \frac R{r_1:L} + \frac R{r_2:L}}
            = - \frac{\frac{60\,\text{В}}{1\,\text{Ом}} + \frac{40\,\text{В}}{1\,\text{Ом}}}{ 1 + \frac{20\,\text{Ом}}{1\,\text{Ом}} + \frac{20\,\text{Ом}}{1\,\text{Ом}}}
            = - \frac{100}{41}\units{А}
            \approx -2{,}40\,\text{А}, \\
        &U  = \varphi_2 - \varphi_1 = \eli R
            = - \frac{\frac{\ele_1 }{r_1:L} + \frac{\ele_2 }{r_2:L}}{ 1 + \frac R{r_1:L} + \frac R{r_2:L}} R
            \approx -48{,}80\,\text{В}.
    \end{align*}
    Оба ответа отрицательны, потому что мы изначально «не угадали» с направлением тока.
    Расчёт же показал,
    что ток через резистор $R$ течёт в противоположную сторону: вниз на рисунке, а потенциал точки 1 больше потенциала точки 2,
    а электрический ток ожидаемо течёт из точки с большим потенциалов в точку с меньшим.

    Кстати, если продолжить расчёт и вычислить значения ещё двух токов (формулы для $\eli_1$ и $\eli_2$, куда подставлять, выписаны выше),
    то по их знакам можно будет понять: угадали ли мы с их направлением или нет.
}

\variantsplitter

\addpersonalvariant{Софья Белянкина}

\tasknumber{1}%
\task{%
    Получите выражение:
    \begin{enumerate}
        \item длины проводника через его сопротивление,
        \item удельное сопротивление из закона Ома,
        \item внешнее сопротивление цепи из закона Ома для полной цепи,
        \item эквивалентное сопротивление $n$ резисторов, соединённых параллельно, каждый сопротивлением $R$.
    \end{enumerate}
}
\solutionspace{40pt}

\tasknumber{2}%
\task{%
    Получите выражение:
    \begin{enumerate}
        \item силы тока через выделяемую мощность и сопротивление резистора,
        \item силы тока через выделенную теплоту и сопротивление резистора,
        \item напряжение на резисторе через выделяемую мощность и сопротивление резистора,
        \item напряжение на резисторе через выделенную в нём теплоту и силу тока через него.
    \end{enumerate}
}
\solutionspace{80pt}

\tasknumber{3}%
\task{%
    Определите ток, протекающий через резистор $R = 20\,\text{Ом}$ и разность потенциалов на нём (см.
    рис.
    на доске),
    если $r_1 = 1\,\text{Ом}$, $r_2 = 1\,\text{Ом}$, $\ele_1 = 60\,\text{В}$, $\ele_2 = 60\,\text{В}$.
}
\answer{%
    Обозначим на рисунке все токи: направление произвольно, но его надо зафиксировать.
    Всего на рисунке 3 контура и 2 узла.
    Поэтому можно записать $3 - 1 = 2$ уравнения законов Кирхгофа для замкнутого контура и $2 - 1 = 1$ — для узлов
    (остальные уравнения тоже можно записать, но они не дадут полезной информации, а будут лишь следствиями уже записанных).

    Отметим на рисунке 2 контура (и не забуем указать направление) и 1 узел (точка «1»ы, выделена жирным).
    Выбор контуров и узлов не критичен: получившаяся система может быть чуть проще или сложнее, но не слишком.

    \begin{tikzpicture}[circuit ee IEC, thick]
        \draw  (0, 0) to [current direction={near end, info=$\eli_1$}] (0, 3)
                to [battery={rotate=-180,info={$\ele_1, r_1$}}]
                (3, 3)
                to [battery={info'={$\ele_2, r_2$}}]
                (6, 3) to [current direction'={near start, info=$\eli_2$}] (6, 0) -- (0, 0)
                (3, 0) to [current direction={near start, info=$\eli$}, resistor={near end, info=$R$}] (3, 3);
        \draw [-{Latex},color=red] (1.2, 1.7) arc [start angle = 135, end angle = -160, radius = 0.6];
        \draw [-{Latex},color=blue] (4.2, 1.7) arc [start angle = 135, end angle = -160, radius = 0.6];
        \node [contact,color=green!71!black] (bottomc) at (3, 0) {};
        \node [below] (bottom) at (3, 0) {$2$};
        \node [above] (top) at (3, 3) {$1$};
    \end{tikzpicture}

    \begin{align*}
        &\begin{cases}
            {\color{red} \ele_1 = \eli_1 r_1 - \eli R}, \\
            {\color{blue} -\ele_2 = -\eli_2 r_2 + \eli R}, \\
            {\color{green!71!black} - \eli - \eli_1 - \eli_2 = 0 };
        \end{cases}
        \qquad \implies \qquad
        \begin{cases}
            \eli_1 = \frac{\ele_1 + \eli R}{r_1}, \\
            \eli_2 = \frac{\ele_2 + \eli R}{r_2}, \\
            \eli + \eli_1 + \eli_2 = 0;
        \end{cases} \implies \\
        &\implies
         \eli + \frac{\ele_1 + \eli R}{r_1:L} + \frac{\ele_2 + \eli R}{r_2:L} = 0, \\
        &\eli\cbr{ 1 + \frac R{r_1:L} + \frac R{r_2:L}} + \frac{\ele_1 }{r_1:L} + \frac{\ele_2 }{r_2:L} = 0, \\
        &\eli
            = - \frac{\frac{\ele_1 }{r_1:L} + \frac{\ele_2 }{r_2:L}}{ 1 + \frac R{r_1:L} + \frac R{r_2:L}}
            = - \frac{\frac{60\,\text{В}}{1\,\text{Ом}} + \frac{60\,\text{В}}{1\,\text{Ом}}}{ 1 + \frac{20\,\text{Ом}}{1\,\text{Ом}} + \frac{20\,\text{Ом}}{1\,\text{Ом}}}
            = - \frac{120}{41}\units{А}
            \approx -2{,}90\,\text{А}, \\
        &U  = \varphi_2 - \varphi_1 = \eli R
            = - \frac{\frac{\ele_1 }{r_1:L} + \frac{\ele_2 }{r_2:L}}{ 1 + \frac R{r_1:L} + \frac R{r_2:L}} R
            \approx -58{,}50\,\text{В}.
    \end{align*}
    Оба ответа отрицательны, потому что мы изначально «не угадали» с направлением тока.
    Расчёт же показал,
    что ток через резистор $R$ течёт в противоположную сторону: вниз на рисунке, а потенциал точки 1 больше потенциала точки 2,
    а электрический ток ожидаемо течёт из точки с большим потенциалов в точку с меньшим.

    Кстати, если продолжить расчёт и вычислить значения ещё двух токов (формулы для $\eli_1$ и $\eli_2$, куда подставлять, выписаны выше),
    то по их знакам можно будет понять: угадали ли мы с их направлением или нет.
}

\variantsplitter

\addpersonalvariant{Варвара Егиазарян}

\tasknumber{1}%
\task{%
    Получите выражение:
    \begin{enumerate}
        \item площади поперечного сечения проводника через его сопротивление,
        \item сопротивление из закона Ома,
        \item внешнее сопротивление цепи из закона Ома для полной цепи,
        \item эквивалентное сопротивление $n$ резисторов, соединённых последовательно, каждый сопротивлением $R$.
    \end{enumerate}
}
\solutionspace{40pt}

\tasknumber{2}%
\task{%
    Получите выражение:
    \begin{enumerate}
        \item силы тока через выделяемую мощность и сопротивление резистора,
        \item силы тока через выделенную теплоту и напряжение на резисторе,
        \item напряжение на резисторе через выделяемую мощность и сопротивление резистора,
        \item напряжение на резисторе через выделенную в нём теплоту и силу тока через него.
    \end{enumerate}
}
\solutionspace{80pt}

\tasknumber{3}%
\task{%
    Определите ток, протекающий через резистор $R = 20\,\text{Ом}$ и разность потенциалов на нём (см.
    рис.
    на доске),
    если $r_1 = 1\,\text{Ом}$, $r_2 = 1\,\text{Ом}$, $\ele_1 = 60\,\text{В}$, $\ele_2 = 60\,\text{В}$.
}
\answer{%
    Обозначим на рисунке все токи: направление произвольно, но его надо зафиксировать.
    Всего на рисунке 3 контура и 2 узла.
    Поэтому можно записать $3 - 1 = 2$ уравнения законов Кирхгофа для замкнутого контура и $2 - 1 = 1$ — для узлов
    (остальные уравнения тоже можно записать, но они не дадут полезной информации, а будут лишь следствиями уже записанных).

    Отметим на рисунке 2 контура (и не забуем указать направление) и 1 узел (точка «1»ы, выделена жирным).
    Выбор контуров и узлов не критичен: получившаяся система может быть чуть проще или сложнее, но не слишком.

    \begin{tikzpicture}[circuit ee IEC, thick]
        \draw  (0, 0) to [current direction={near end, info=$\eli_1$}] (0, 3)
                to [battery={rotate=-180,info={$\ele_1, r_1$}}]
                (3, 3)
                to [battery={info'={$\ele_2, r_2$}}]
                (6, 3) to [current direction'={near start, info=$\eli_2$}] (6, 0) -- (0, 0)
                (3, 0) to [current direction={near start, info=$\eli$}, resistor={near end, info=$R$}] (3, 3);
        \draw [-{Latex},color=red] (1.2, 1.7) arc [start angle = 135, end angle = -160, radius = 0.6];
        \draw [-{Latex},color=blue] (4.2, 1.7) arc [start angle = 135, end angle = -160, radius = 0.6];
        \node [contact,color=green!71!black] (bottomc) at (3, 0) {};
        \node [below] (bottom) at (3, 0) {$2$};
        \node [above] (top) at (3, 3) {$1$};
    \end{tikzpicture}

    \begin{align*}
        &\begin{cases}
            {\color{red} \ele_1 = \eli_1 r_1 - \eli R}, \\
            {\color{blue} -\ele_2 = -\eli_2 r_2 + \eli R}, \\
            {\color{green!71!black} - \eli - \eli_1 - \eli_2 = 0 };
        \end{cases}
        \qquad \implies \qquad
        \begin{cases}
            \eli_1 = \frac{\ele_1 + \eli R}{r_1}, \\
            \eli_2 = \frac{\ele_2 + \eli R}{r_2}, \\
            \eli + \eli_1 + \eli_2 = 0;
        \end{cases} \implies \\
        &\implies
         \eli + \frac{\ele_1 + \eli R}{r_1:L} + \frac{\ele_2 + \eli R}{r_2:L} = 0, \\
        &\eli\cbr{ 1 + \frac R{r_1:L} + \frac R{r_2:L}} + \frac{\ele_1 }{r_1:L} + \frac{\ele_2 }{r_2:L} = 0, \\
        &\eli
            = - \frac{\frac{\ele_1 }{r_1:L} + \frac{\ele_2 }{r_2:L}}{ 1 + \frac R{r_1:L} + \frac R{r_2:L}}
            = - \frac{\frac{60\,\text{В}}{1\,\text{Ом}} + \frac{60\,\text{В}}{1\,\text{Ом}}}{ 1 + \frac{20\,\text{Ом}}{1\,\text{Ом}} + \frac{20\,\text{Ом}}{1\,\text{Ом}}}
            = - \frac{120}{41}\units{А}
            \approx -2{,}90\,\text{А}, \\
        &U  = \varphi_2 - \varphi_1 = \eli R
            = - \frac{\frac{\ele_1 }{r_1:L} + \frac{\ele_2 }{r_2:L}}{ 1 + \frac R{r_1:L} + \frac R{r_2:L}} R
            \approx -58{,}50\,\text{В}.
    \end{align*}
    Оба ответа отрицательны, потому что мы изначально «не угадали» с направлением тока.
    Расчёт же показал,
    что ток через резистор $R$ течёт в противоположную сторону: вниз на рисунке, а потенциал точки 1 больше потенциала точки 2,
    а электрический ток ожидаемо течёт из точки с большим потенциалов в точку с меньшим.

    Кстати, если продолжить расчёт и вычислить значения ещё двух токов (формулы для $\eli_1$ и $\eli_2$, куда подставлять, выписаны выше),
    то по их знакам можно будет понять: угадали ли мы с их направлением или нет.
}

\variantsplitter

\addpersonalvariant{Владислав Емелин}

\tasknumber{1}%
\task{%
    Получите выражение:
    \begin{enumerate}
        \item площади поперечного сечения проводника через его сопротивление,
        \item сопротивление из закона Ома,
        \item внутреннее сопротивление цепи из закона Ома для полной цепи,
        \item эквивалентное сопротивление $n$ резисторов, соединённых последовательно, каждый сопротивлением $R$.
    \end{enumerate}
}
\solutionspace{40pt}

\tasknumber{2}%
\task{%
    Получите выражение:
    \begin{enumerate}
        \item силы тока через выделяемую мощность и напряжение на резисторе,
        \item силы тока через выделенную теплоту и сопротивление резистора,
        \item напряжение на резисторе через выделяемую мощность и сопротивление резистора,
        \item напряжение на резисторе через выделенную в нём теплоту и сопротивление резистора.
    \end{enumerate}
}
\solutionspace{80pt}

\tasknumber{3}%
\task{%
    Определите ток, протекающий через резистор $R = 15\,\text{Ом}$ и разность потенциалов на нём (см.
    рис.
    на доске),
    если $r_1 = 1\,\text{Ом}$, $r_2 = 2\,\text{Ом}$, $\ele_1 = 40\,\text{В}$, $\ele_2 = 40\,\text{В}$.
}
\answer{%
    Обозначим на рисунке все токи: направление произвольно, но его надо зафиксировать.
    Всего на рисунке 3 контура и 2 узла.
    Поэтому можно записать $3 - 1 = 2$ уравнения законов Кирхгофа для замкнутого контура и $2 - 1 = 1$ — для узлов
    (остальные уравнения тоже можно записать, но они не дадут полезной информации, а будут лишь следствиями уже записанных).

    Отметим на рисунке 2 контура (и не забуем указать направление) и 1 узел (точка «1»ы, выделена жирным).
    Выбор контуров и узлов не критичен: получившаяся система может быть чуть проще или сложнее, но не слишком.

    \begin{tikzpicture}[circuit ee IEC, thick]
        \draw  (0, 0) to [current direction={near end, info=$\eli_1$}] (0, 3)
                to [battery={rotate=-180,info={$\ele_1, r_1$}}]
                (3, 3)
                to [battery={info'={$\ele_2, r_2$}}]
                (6, 3) to [current direction'={near start, info=$\eli_2$}] (6, 0) -- (0, 0)
                (3, 0) to [current direction={near start, info=$\eli$}, resistor={near end, info=$R$}] (3, 3);
        \draw [-{Latex},color=red] (1.2, 1.7) arc [start angle = 135, end angle = -160, radius = 0.6];
        \draw [-{Latex},color=blue] (4.2, 1.7) arc [start angle = 135, end angle = -160, radius = 0.6];
        \node [contact,color=green!71!black] (bottomc) at (3, 0) {};
        \node [below] (bottom) at (3, 0) {$2$};
        \node [above] (top) at (3, 3) {$1$};
    \end{tikzpicture}

    \begin{align*}
        &\begin{cases}
            {\color{red} \ele_1 = \eli_1 r_1 - \eli R}, \\
            {\color{blue} -\ele_2 = -\eli_2 r_2 + \eli R}, \\
            {\color{green!71!black} - \eli - \eli_1 - \eli_2 = 0 };
        \end{cases}
        \qquad \implies \qquad
        \begin{cases}
            \eli_1 = \frac{\ele_1 + \eli R}{r_1}, \\
            \eli_2 = \frac{\ele_2 + \eli R}{r_2}, \\
            \eli + \eli_1 + \eli_2 = 0;
        \end{cases} \implies \\
        &\implies
         \eli + \frac{\ele_1 + \eli R}{r_1:L} + \frac{\ele_2 + \eli R}{r_2:L} = 0, \\
        &\eli\cbr{ 1 + \frac R{r_1:L} + \frac R{r_2:L}} + \frac{\ele_1 }{r_1:L} + \frac{\ele_2 }{r_2:L} = 0, \\
        &\eli
            = - \frac{\frac{\ele_1 }{r_1:L} + \frac{\ele_2 }{r_2:L}}{ 1 + \frac R{r_1:L} + \frac R{r_2:L}}
            = - \frac{\frac{40\,\text{В}}{1\,\text{Ом}} + \frac{40\,\text{В}}{2\,\text{Ом}}}{ 1 + \frac{15\,\text{Ом}}{1\,\text{Ом}} + \frac{15\,\text{Ом}}{2\,\text{Ом}}}
            = - \frac{120}{47}\units{А}
            \approx -2{,}60\,\text{А}, \\
        &U  = \varphi_2 - \varphi_1 = \eli R
            = - \frac{\frac{\ele_1 }{r_1:L} + \frac{\ele_2 }{r_2:L}}{ 1 + \frac R{r_1:L} + \frac R{r_2:L}} R
            \approx -38{,}30\,\text{В}.
    \end{align*}
    Оба ответа отрицательны, потому что мы изначально «не угадали» с направлением тока.
    Расчёт же показал,
    что ток через резистор $R$ течёт в противоположную сторону: вниз на рисунке, а потенциал точки 1 больше потенциала точки 2,
    а электрический ток ожидаемо течёт из точки с большим потенциалов в точку с меньшим.

    Кстати, если продолжить расчёт и вычислить значения ещё двух токов (формулы для $\eli_1$ и $\eli_2$, куда подставлять, выписаны выше),
    то по их знакам можно будет понять: угадали ли мы с их направлением или нет.
}

\variantsplitter

\addpersonalvariant{Артём Жичин}

\tasknumber{1}%
\task{%
    Получите выражение:
    \begin{enumerate}
        \item площади поперечного сечения проводника через его сопротивление,
        \item удельное сопротивление из закона Ома,
        \item внешнее сопротивление цепи из закона Ома для полной цепи,
        \item эквивалентное сопротивление $n$ резисторов, соединённых последовательно, каждый сопротивлением $R$.
    \end{enumerate}
}
\solutionspace{40pt}

\tasknumber{2}%
\task{%
    Получите выражение:
    \begin{enumerate}
        \item силы тока через выделяемую мощность и разность потенциалов на резисторе,
        \item силы тока через выделенную теплоту и сопротивление резистора,
        \item напряжение на резисторе через выделяемую мощность и силу тока через него,
        \item напряжение на резисторе через выделенную в нём теплоту и сопротивление резистора.
    \end{enumerate}
}
\solutionspace{80pt}

\tasknumber{3}%
\task{%
    Определите ток, протекающий через резистор $R = 20\,\text{Ом}$ и разность потенциалов на нём (см.
    рис.
    на доске),
    если $r_1 = 3\,\text{Ом}$, $r_2 = 3\,\text{Ом}$, $\ele_1 = 40\,\text{В}$, $\ele_2 = 20\,\text{В}$.
}
\answer{%
    Обозначим на рисунке все токи: направление произвольно, но его надо зафиксировать.
    Всего на рисунке 3 контура и 2 узла.
    Поэтому можно записать $3 - 1 = 2$ уравнения законов Кирхгофа для замкнутого контура и $2 - 1 = 1$ — для узлов
    (остальные уравнения тоже можно записать, но они не дадут полезной информации, а будут лишь следствиями уже записанных).

    Отметим на рисунке 2 контура (и не забуем указать направление) и 1 узел (точка «1»ы, выделена жирным).
    Выбор контуров и узлов не критичен: получившаяся система может быть чуть проще или сложнее, но не слишком.

    \begin{tikzpicture}[circuit ee IEC, thick]
        \draw  (0, 0) to [current direction={near end, info=$\eli_1$}] (0, 3)
                to [battery={rotate=-180,info={$\ele_1, r_1$}}]
                (3, 3)
                to [battery={info'={$\ele_2, r_2$}}]
                (6, 3) to [current direction'={near start, info=$\eli_2$}] (6, 0) -- (0, 0)
                (3, 0) to [current direction={near start, info=$\eli$}, resistor={near end, info=$R$}] (3, 3);
        \draw [-{Latex},color=red] (1.2, 1.7) arc [start angle = 135, end angle = -160, radius = 0.6];
        \draw [-{Latex},color=blue] (4.2, 1.7) arc [start angle = 135, end angle = -160, radius = 0.6];
        \node [contact,color=green!71!black] (bottomc) at (3, 0) {};
        \node [below] (bottom) at (3, 0) {$2$};
        \node [above] (top) at (3, 3) {$1$};
    \end{tikzpicture}

    \begin{align*}
        &\begin{cases}
            {\color{red} \ele_1 = \eli_1 r_1 - \eli R}, \\
            {\color{blue} -\ele_2 = -\eli_2 r_2 + \eli R}, \\
            {\color{green!71!black} - \eli - \eli_1 - \eli_2 = 0 };
        \end{cases}
        \qquad \implies \qquad
        \begin{cases}
            \eli_1 = \frac{\ele_1 + \eli R}{r_1}, \\
            \eli_2 = \frac{\ele_2 + \eli R}{r_2}, \\
            \eli + \eli_1 + \eli_2 = 0;
        \end{cases} \implies \\
        &\implies
         \eli + \frac{\ele_1 + \eli R}{r_1:L} + \frac{\ele_2 + \eli R}{r_2:L} = 0, \\
        &\eli\cbr{ 1 + \frac R{r_1:L} + \frac R{r_2:L}} + \frac{\ele_1 }{r_1:L} + \frac{\ele_2 }{r_2:L} = 0, \\
        &\eli
            = - \frac{\frac{\ele_1 }{r_1:L} + \frac{\ele_2 }{r_2:L}}{ 1 + \frac R{r_1:L} + \frac R{r_2:L}}
            = - \frac{\frac{40\,\text{В}}{3\,\text{Ом}} + \frac{20\,\text{В}}{3\,\text{Ом}}}{ 1 + \frac{20\,\text{Ом}}{3\,\text{Ом}} + \frac{20\,\text{Ом}}{3\,\text{Ом}}}
            = - \frac{60}{43}\units{А}
            \approx -1{,}400\,\text{А}, \\
        &U  = \varphi_2 - \varphi_1 = \eli R
            = - \frac{\frac{\ele_1 }{r_1:L} + \frac{\ele_2 }{r_2:L}}{ 1 + \frac R{r_1:L} + \frac R{r_2:L}} R
            \approx -27{,}90\,\text{В}.
    \end{align*}
    Оба ответа отрицательны, потому что мы изначально «не угадали» с направлением тока.
    Расчёт же показал,
    что ток через резистор $R$ течёт в противоположную сторону: вниз на рисунке, а потенциал точки 1 больше потенциала точки 2,
    а электрический ток ожидаемо течёт из точки с большим потенциалов в точку с меньшим.

    Кстати, если продолжить расчёт и вычислить значения ещё двух токов (формулы для $\eli_1$ и $\eli_2$, куда подставлять, выписаны выше),
    то по их знакам можно будет понять: угадали ли мы с их направлением или нет.
}

\variantsplitter

\addpersonalvariant{Дарья Кошман}

\tasknumber{1}%
\task{%
    Получите выражение:
    \begin{enumerate}
        \item площади поперечного сечения проводника через его сопротивление,
        \item сопротивление из закона Ома,
        \item внешнее сопротивление цепи из закона Ома для полной цепи,
        \item эквивалентное сопротивление $n$ резисторов, соединённых последовательно, каждый сопротивлением $R$.
    \end{enumerate}
}
\solutionspace{40pt}

\tasknumber{2}%
\task{%
    Получите выражение:
    \begin{enumerate}
        \item силы тока через выделяемую мощность и разность потенциалов на резисторе,
        \item силы тока через выделенную теплоту и сопротивление резистора,
        \item напряжение на резисторе через выделяемую мощность и силу тока через него,
        \item напряжение на резисторе через выделенную в нём теплоту и силу тока через него.
    \end{enumerate}
}
\solutionspace{80pt}

\tasknumber{3}%
\task{%
    Определите ток, протекающий через резистор $R = 18\,\text{Ом}$ и разность потенциалов на нём (см.
    рис.
    на доске),
    если $r_1 = 1\,\text{Ом}$, $r_2 = 3\,\text{Ом}$, $\ele_1 = 20\,\text{В}$, $\ele_2 = 40\,\text{В}$.
}
\answer{%
    Обозначим на рисунке все токи: направление произвольно, но его надо зафиксировать.
    Всего на рисунке 3 контура и 2 узла.
    Поэтому можно записать $3 - 1 = 2$ уравнения законов Кирхгофа для замкнутого контура и $2 - 1 = 1$ — для узлов
    (остальные уравнения тоже можно записать, но они не дадут полезной информации, а будут лишь следствиями уже записанных).

    Отметим на рисунке 2 контура (и не забуем указать направление) и 1 узел (точка «1»ы, выделена жирным).
    Выбор контуров и узлов не критичен: получившаяся система может быть чуть проще или сложнее, но не слишком.

    \begin{tikzpicture}[circuit ee IEC, thick]
        \draw  (0, 0) to [current direction={near end, info=$\eli_1$}] (0, 3)
                to [battery={rotate=-180,info={$\ele_1, r_1$}}]
                (3, 3)
                to [battery={info'={$\ele_2, r_2$}}]
                (6, 3) to [current direction'={near start, info=$\eli_2$}] (6, 0) -- (0, 0)
                (3, 0) to [current direction={near start, info=$\eli$}, resistor={near end, info=$R$}] (3, 3);
        \draw [-{Latex},color=red] (1.2, 1.7) arc [start angle = 135, end angle = -160, radius = 0.6];
        \draw [-{Latex},color=blue] (4.2, 1.7) arc [start angle = 135, end angle = -160, radius = 0.6];
        \node [contact,color=green!71!black] (bottomc) at (3, 0) {};
        \node [below] (bottom) at (3, 0) {$2$};
        \node [above] (top) at (3, 3) {$1$};
    \end{tikzpicture}

    \begin{align*}
        &\begin{cases}
            {\color{red} \ele_1 = \eli_1 r_1 - \eli R}, \\
            {\color{blue} -\ele_2 = -\eli_2 r_2 + \eli R}, \\
            {\color{green!71!black} - \eli - \eli_1 - \eli_2 = 0 };
        \end{cases}
        \qquad \implies \qquad
        \begin{cases}
            \eli_1 = \frac{\ele_1 + \eli R}{r_1}, \\
            \eli_2 = \frac{\ele_2 + \eli R}{r_2}, \\
            \eli + \eli_1 + \eli_2 = 0;
        \end{cases} \implies \\
        &\implies
         \eli + \frac{\ele_1 + \eli R}{r_1:L} + \frac{\ele_2 + \eli R}{r_2:L} = 0, \\
        &\eli\cbr{ 1 + \frac R{r_1:L} + \frac R{r_2:L}} + \frac{\ele_1 }{r_1:L} + \frac{\ele_2 }{r_2:L} = 0, \\
        &\eli
            = - \frac{\frac{\ele_1 }{r_1:L} + \frac{\ele_2 }{r_2:L}}{ 1 + \frac R{r_1:L} + \frac R{r_2:L}}
            = - \frac{\frac{20\,\text{В}}{1\,\text{Ом}} + \frac{40\,\text{В}}{3\,\text{Ом}}}{ 1 + \frac{18\,\text{Ом}}{1\,\text{Ом}} + \frac{18\,\text{Ом}}{3\,\text{Ом}}}
            = - \frac43\units{А}
            \approx -1{,}300\,\text{А}, \\
        &U  = \varphi_2 - \varphi_1 = \eli R
            = - \frac{\frac{\ele_1 }{r_1:L} + \frac{\ele_2 }{r_2:L}}{ 1 + \frac R{r_1:L} + \frac R{r_2:L}} R
            \approx -24{,}00\,\text{В}.
    \end{align*}
    Оба ответа отрицательны, потому что мы изначально «не угадали» с направлением тока.
    Расчёт же показал,
    что ток через резистор $R$ течёт в противоположную сторону: вниз на рисунке, а потенциал точки 1 больше потенциала точки 2,
    а электрический ток ожидаемо течёт из точки с большим потенциалов в точку с меньшим.

    Кстати, если продолжить расчёт и вычислить значения ещё двух токов (формулы для $\eli_1$ и $\eli_2$, куда подставлять, выписаны выше),
    то по их знакам можно будет понять: угадали ли мы с их направлением или нет.
}

\variantsplitter

\addpersonalvariant{Анна Кузьмичёва}

\tasknumber{1}%
\task{%
    Получите выражение:
    \begin{enumerate}
        \item площади поперечного сечения проводника через его сопротивление,
        \item сопротивление из закона Ома,
        \item внутреннее сопротивление цепи из закона Ома для полной цепи,
        \item эквивалентное сопротивление $n$ резисторов, соединённых параллельно, каждый сопротивлением $R$.
    \end{enumerate}
}
\solutionspace{40pt}

\tasknumber{2}%
\task{%
    Получите выражение:
    \begin{enumerate}
        \item силы тока через выделяемую мощность и разность потенциалов на резисторе,
        \item силы тока через выделенную теплоту и сопротивление резистора,
        \item напряжение на резисторе через выделяемую мощность и силу тока через него,
        \item напряжение на резисторе через выделенную в нём теплоту и силу тока через него.
    \end{enumerate}
}
\solutionspace{80pt}

\tasknumber{3}%
\task{%
    Определите ток, протекающий через резистор $R = 18\,\text{Ом}$ и разность потенциалов на нём (см.
    рис.
    на доске),
    если $r_1 = 3\,\text{Ом}$, $r_2 = 2\,\text{Ом}$, $\ele_1 = 30\,\text{В}$, $\ele_2 = 40\,\text{В}$.
}
\answer{%
    Обозначим на рисунке все токи: направление произвольно, но его надо зафиксировать.
    Всего на рисунке 3 контура и 2 узла.
    Поэтому можно записать $3 - 1 = 2$ уравнения законов Кирхгофа для замкнутого контура и $2 - 1 = 1$ — для узлов
    (остальные уравнения тоже можно записать, но они не дадут полезной информации, а будут лишь следствиями уже записанных).

    Отметим на рисунке 2 контура (и не забуем указать направление) и 1 узел (точка «1»ы, выделена жирным).
    Выбор контуров и узлов не критичен: получившаяся система может быть чуть проще или сложнее, но не слишком.

    \begin{tikzpicture}[circuit ee IEC, thick]
        \draw  (0, 0) to [current direction={near end, info=$\eli_1$}] (0, 3)
                to [battery={rotate=-180,info={$\ele_1, r_1$}}]
                (3, 3)
                to [battery={info'={$\ele_2, r_2$}}]
                (6, 3) to [current direction'={near start, info=$\eli_2$}] (6, 0) -- (0, 0)
                (3, 0) to [current direction={near start, info=$\eli$}, resistor={near end, info=$R$}] (3, 3);
        \draw [-{Latex},color=red] (1.2, 1.7) arc [start angle = 135, end angle = -160, radius = 0.6];
        \draw [-{Latex},color=blue] (4.2, 1.7) arc [start angle = 135, end angle = -160, radius = 0.6];
        \node [contact,color=green!71!black] (bottomc) at (3, 0) {};
        \node [below] (bottom) at (3, 0) {$2$};
        \node [above] (top) at (3, 3) {$1$};
    \end{tikzpicture}

    \begin{align*}
        &\begin{cases}
            {\color{red} \ele_1 = \eli_1 r_1 - \eli R}, \\
            {\color{blue} -\ele_2 = -\eli_2 r_2 + \eli R}, \\
            {\color{green!71!black} - \eli - \eli_1 - \eli_2 = 0 };
        \end{cases}
        \qquad \implies \qquad
        \begin{cases}
            \eli_1 = \frac{\ele_1 + \eli R}{r_1}, \\
            \eli_2 = \frac{\ele_2 + \eli R}{r_2}, \\
            \eli + \eli_1 + \eli_2 = 0;
        \end{cases} \implies \\
        &\implies
         \eli + \frac{\ele_1 + \eli R}{r_1:L} + \frac{\ele_2 + \eli R}{r_2:L} = 0, \\
        &\eli\cbr{ 1 + \frac R{r_1:L} + \frac R{r_2:L}} + \frac{\ele_1 }{r_1:L} + \frac{\ele_2 }{r_2:L} = 0, \\
        &\eli
            = - \frac{\frac{\ele_1 }{r_1:L} + \frac{\ele_2 }{r_2:L}}{ 1 + \frac R{r_1:L} + \frac R{r_2:L}}
            = - \frac{\frac{30\,\text{В}}{3\,\text{Ом}} + \frac{40\,\text{В}}{2\,\text{Ом}}}{ 1 + \frac{18\,\text{Ом}}{3\,\text{Ом}} + \frac{18\,\text{Ом}}{2\,\text{Ом}}}
            = - \frac{15}8\units{А}
            \approx -1{,}900\,\text{А}, \\
        &U  = \varphi_2 - \varphi_1 = \eli R
            = - \frac{\frac{\ele_1 }{r_1:L} + \frac{\ele_2 }{r_2:L}}{ 1 + \frac R{r_1:L} + \frac R{r_2:L}} R
            \approx -33{,}80\,\text{В}.
    \end{align*}
    Оба ответа отрицательны, потому что мы изначально «не угадали» с направлением тока.
    Расчёт же показал,
    что ток через резистор $R$ течёт в противоположную сторону: вниз на рисунке, а потенциал точки 1 больше потенциала точки 2,
    а электрический ток ожидаемо течёт из точки с большим потенциалов в точку с меньшим.

    Кстати, если продолжить расчёт и вычислить значения ещё двух токов (формулы для $\eli_1$ и $\eli_2$, куда подставлять, выписаны выше),
    то по их знакам можно будет понять: угадали ли мы с их направлением или нет.
}

\variantsplitter

\addpersonalvariant{Алёна Куприянова}

\tasknumber{1}%
\task{%
    Получите выражение:
    \begin{enumerate}
        \item длины проводника через его сопротивление,
        \item сопротивление из закона Ома,
        \item внешнее сопротивление цепи из закона Ома для полной цепи,
        \item эквивалентное сопротивление $n$ резисторов, соединённых параллельно, каждый сопротивлением $R$.
    \end{enumerate}
}
\solutionspace{40pt}

\tasknumber{2}%
\task{%
    Получите выражение:
    \begin{enumerate}
        \item силы тока через выделяемую мощность и напряжение на резисторе,
        \item силы тока через выделенную теплоту и разность потенциалов на резисторе,
        \item напряжение на резисторе через выделяемую мощность и силу тока через него,
        \item напряжение на резисторе через выделенную в нём теплоту и сопротивление резистора.
    \end{enumerate}
}
\solutionspace{80pt}

\tasknumber{3}%
\task{%
    Определите ток, протекающий через резистор $R = 12\,\text{Ом}$ и разность потенциалов на нём (см.
    рис.
    на доске),
    если $r_1 = 2\,\text{Ом}$, $r_2 = 1\,\text{Ом}$, $\ele_1 = 40\,\text{В}$, $\ele_2 = 40\,\text{В}$.
}
\answer{%
    Обозначим на рисунке все токи: направление произвольно, но его надо зафиксировать.
    Всего на рисунке 3 контура и 2 узла.
    Поэтому можно записать $3 - 1 = 2$ уравнения законов Кирхгофа для замкнутого контура и $2 - 1 = 1$ — для узлов
    (остальные уравнения тоже можно записать, но они не дадут полезной информации, а будут лишь следствиями уже записанных).

    Отметим на рисунке 2 контура (и не забуем указать направление) и 1 узел (точка «1»ы, выделена жирным).
    Выбор контуров и узлов не критичен: получившаяся система может быть чуть проще или сложнее, но не слишком.

    \begin{tikzpicture}[circuit ee IEC, thick]
        \draw  (0, 0) to [current direction={near end, info=$\eli_1$}] (0, 3)
                to [battery={rotate=-180,info={$\ele_1, r_1$}}]
                (3, 3)
                to [battery={info'={$\ele_2, r_2$}}]
                (6, 3) to [current direction'={near start, info=$\eli_2$}] (6, 0) -- (0, 0)
                (3, 0) to [current direction={near start, info=$\eli$}, resistor={near end, info=$R$}] (3, 3);
        \draw [-{Latex},color=red] (1.2, 1.7) arc [start angle = 135, end angle = -160, radius = 0.6];
        \draw [-{Latex},color=blue] (4.2, 1.7) arc [start angle = 135, end angle = -160, radius = 0.6];
        \node [contact,color=green!71!black] (bottomc) at (3, 0) {};
        \node [below] (bottom) at (3, 0) {$2$};
        \node [above] (top) at (3, 3) {$1$};
    \end{tikzpicture}

    \begin{align*}
        &\begin{cases}
            {\color{red} \ele_1 = \eli_1 r_1 - \eli R}, \\
            {\color{blue} -\ele_2 = -\eli_2 r_2 + \eli R}, \\
            {\color{green!71!black} - \eli - \eli_1 - \eli_2 = 0 };
        \end{cases}
        \qquad \implies \qquad
        \begin{cases}
            \eli_1 = \frac{\ele_1 + \eli R}{r_1}, \\
            \eli_2 = \frac{\ele_2 + \eli R}{r_2}, \\
            \eli + \eli_1 + \eli_2 = 0;
        \end{cases} \implies \\
        &\implies
         \eli + \frac{\ele_1 + \eli R}{r_1:L} + \frac{\ele_2 + \eli R}{r_2:L} = 0, \\
        &\eli\cbr{ 1 + \frac R{r_1:L} + \frac R{r_2:L}} + \frac{\ele_1 }{r_1:L} + \frac{\ele_2 }{r_2:L} = 0, \\
        &\eli
            = - \frac{\frac{\ele_1 }{r_1:L} + \frac{\ele_2 }{r_2:L}}{ 1 + \frac R{r_1:L} + \frac R{r_2:L}}
            = - \frac{\frac{40\,\text{В}}{2\,\text{Ом}} + \frac{40\,\text{В}}{1\,\text{Ом}}}{ 1 + \frac{12\,\text{Ом}}{2\,\text{Ом}} + \frac{12\,\text{Ом}}{1\,\text{Ом}}}
            = - \frac{60}{19}\units{А}
            \approx -3{,}20\,\text{А}, \\
        &U  = \varphi_2 - \varphi_1 = \eli R
            = - \frac{\frac{\ele_1 }{r_1:L} + \frac{\ele_2 }{r_2:L}}{ 1 + \frac R{r_1:L} + \frac R{r_2:L}} R
            \approx -37{,}90\,\text{В}.
    \end{align*}
    Оба ответа отрицательны, потому что мы изначально «не угадали» с направлением тока.
    Расчёт же показал,
    что ток через резистор $R$ течёт в противоположную сторону: вниз на рисунке, а потенциал точки 1 больше потенциала точки 2,
    а электрический ток ожидаемо течёт из точки с большим потенциалов в точку с меньшим.

    Кстати, если продолжить расчёт и вычислить значения ещё двух токов (формулы для $\eli_1$ и $\eli_2$, куда подставлять, выписаны выше),
    то по их знакам можно будет понять: угадали ли мы с их направлением или нет.
}

\variantsplitter

\addpersonalvariant{Ярослав Лавровский}

\tasknumber{1}%
\task{%
    Получите выражение:
    \begin{enumerate}
        \item длины проводника через его сопротивление,
        \item удельное сопротивление из закона Ома,
        \item внутреннее сопротивление цепи из закона Ома для полной цепи,
        \item эквивалентное сопротивление $n$ резисторов, соединённых параллельно, каждый сопротивлением $R$.
    \end{enumerate}
}
\solutionspace{40pt}

\tasknumber{2}%
\task{%
    Получите выражение:
    \begin{enumerate}
        \item силы тока через выделяемую мощность и сопротивление резистора,
        \item силы тока через выделенную теплоту и сопротивление резистора,
        \item напряжение на резисторе через выделяемую мощность и сопротивление резистора,
        \item напряжение на резисторе через выделенную в нём теплоту и силу тока через него.
    \end{enumerate}
}
\solutionspace{80pt}

\tasknumber{3}%
\task{%
    Определите ток, протекающий через резистор $R = 20\,\text{Ом}$ и разность потенциалов на нём (см.
    рис.
    на доске),
    если $r_1 = 3\,\text{Ом}$, $r_2 = 3\,\text{Ом}$, $\ele_1 = 30\,\text{В}$, $\ele_2 = 30\,\text{В}$.
}
\answer{%
    Обозначим на рисунке все токи: направление произвольно, но его надо зафиксировать.
    Всего на рисунке 3 контура и 2 узла.
    Поэтому можно записать $3 - 1 = 2$ уравнения законов Кирхгофа для замкнутого контура и $2 - 1 = 1$ — для узлов
    (остальные уравнения тоже можно записать, но они не дадут полезной информации, а будут лишь следствиями уже записанных).

    Отметим на рисунке 2 контура (и не забуем указать направление) и 1 узел (точка «1»ы, выделена жирным).
    Выбор контуров и узлов не критичен: получившаяся система может быть чуть проще или сложнее, но не слишком.

    \begin{tikzpicture}[circuit ee IEC, thick]
        \draw  (0, 0) to [current direction={near end, info=$\eli_1$}] (0, 3)
                to [battery={rotate=-180,info={$\ele_1, r_1$}}]
                (3, 3)
                to [battery={info'={$\ele_2, r_2$}}]
                (6, 3) to [current direction'={near start, info=$\eli_2$}] (6, 0) -- (0, 0)
                (3, 0) to [current direction={near start, info=$\eli$}, resistor={near end, info=$R$}] (3, 3);
        \draw [-{Latex},color=red] (1.2, 1.7) arc [start angle = 135, end angle = -160, radius = 0.6];
        \draw [-{Latex},color=blue] (4.2, 1.7) arc [start angle = 135, end angle = -160, radius = 0.6];
        \node [contact,color=green!71!black] (bottomc) at (3, 0) {};
        \node [below] (bottom) at (3, 0) {$2$};
        \node [above] (top) at (3, 3) {$1$};
    \end{tikzpicture}

    \begin{align*}
        &\begin{cases}
            {\color{red} \ele_1 = \eli_1 r_1 - \eli R}, \\
            {\color{blue} -\ele_2 = -\eli_2 r_2 + \eli R}, \\
            {\color{green!71!black} - \eli - \eli_1 - \eli_2 = 0 };
        \end{cases}
        \qquad \implies \qquad
        \begin{cases}
            \eli_1 = \frac{\ele_1 + \eli R}{r_1}, \\
            \eli_2 = \frac{\ele_2 + \eli R}{r_2}, \\
            \eli + \eli_1 + \eli_2 = 0;
        \end{cases} \implies \\
        &\implies
         \eli + \frac{\ele_1 + \eli R}{r_1:L} + \frac{\ele_2 + \eli R}{r_2:L} = 0, \\
        &\eli\cbr{ 1 + \frac R{r_1:L} + \frac R{r_2:L}} + \frac{\ele_1 }{r_1:L} + \frac{\ele_2 }{r_2:L} = 0, \\
        &\eli
            = - \frac{\frac{\ele_1 }{r_1:L} + \frac{\ele_2 }{r_2:L}}{ 1 + \frac R{r_1:L} + \frac R{r_2:L}}
            = - \frac{\frac{30\,\text{В}}{3\,\text{Ом}} + \frac{30\,\text{В}}{3\,\text{Ом}}}{ 1 + \frac{20\,\text{Ом}}{3\,\text{Ом}} + \frac{20\,\text{Ом}}{3\,\text{Ом}}}
            = - \frac{60}{43}\units{А}
            \approx -1{,}400\,\text{А}, \\
        &U  = \varphi_2 - \varphi_1 = \eli R
            = - \frac{\frac{\ele_1 }{r_1:L} + \frac{\ele_2 }{r_2:L}}{ 1 + \frac R{r_1:L} + \frac R{r_2:L}} R
            \approx -27{,}90\,\text{В}.
    \end{align*}
    Оба ответа отрицательны, потому что мы изначально «не угадали» с направлением тока.
    Расчёт же показал,
    что ток через резистор $R$ течёт в противоположную сторону: вниз на рисунке, а потенциал точки 1 больше потенциала точки 2,
    а электрический ток ожидаемо течёт из точки с большим потенциалов в точку с меньшим.

    Кстати, если продолжить расчёт и вычислить значения ещё двух токов (формулы для $\eli_1$ и $\eli_2$, куда подставлять, выписаны выше),
    то по их знакам можно будет понять: угадали ли мы с их направлением или нет.
}

\variantsplitter

\addpersonalvariant{Анастасия Ламанова}

\tasknumber{1}%
\task{%
    Получите выражение:
    \begin{enumerate}
        \item длины проводника через его сопротивление,
        \item удельное сопротивление из закона Ома,
        \item внутреннее сопротивление цепи из закона Ома для полной цепи,
        \item эквивалентное сопротивление $n$ резисторов, соединённых последовательно, каждый сопротивлением $R$.
    \end{enumerate}
}
\solutionspace{40pt}

\tasknumber{2}%
\task{%
    Получите выражение:
    \begin{enumerate}
        \item силы тока через выделяемую мощность и разность потенциалов на резисторе,
        \item силы тока через выделенную теплоту и разность потенциалов на резисторе,
        \item напряжение на резисторе через выделяемую мощность и сопротивление резистора,
        \item напряжение на резисторе через выделенную в нём теплоту и силу тока через него.
    \end{enumerate}
}
\solutionspace{80pt}

\tasknumber{3}%
\task{%
    Определите ток, протекающий через резистор $R = 15\,\text{Ом}$ и разность потенциалов на нём (см.
    рис.
    на доске),
    если $r_1 = 2\,\text{Ом}$, $r_2 = 3\,\text{Ом}$, $\ele_1 = 60\,\text{В}$, $\ele_2 = 40\,\text{В}$.
}
\answer{%
    Обозначим на рисунке все токи: направление произвольно, но его надо зафиксировать.
    Всего на рисунке 3 контура и 2 узла.
    Поэтому можно записать $3 - 1 = 2$ уравнения законов Кирхгофа для замкнутого контура и $2 - 1 = 1$ — для узлов
    (остальные уравнения тоже можно записать, но они не дадут полезной информации, а будут лишь следствиями уже записанных).

    Отметим на рисунке 2 контура (и не забуем указать направление) и 1 узел (точка «1»ы, выделена жирным).
    Выбор контуров и узлов не критичен: получившаяся система может быть чуть проще или сложнее, но не слишком.

    \begin{tikzpicture}[circuit ee IEC, thick]
        \draw  (0, 0) to [current direction={near end, info=$\eli_1$}] (0, 3)
                to [battery={rotate=-180,info={$\ele_1, r_1$}}]
                (3, 3)
                to [battery={info'={$\ele_2, r_2$}}]
                (6, 3) to [current direction'={near start, info=$\eli_2$}] (6, 0) -- (0, 0)
                (3, 0) to [current direction={near start, info=$\eli$}, resistor={near end, info=$R$}] (3, 3);
        \draw [-{Latex},color=red] (1.2, 1.7) arc [start angle = 135, end angle = -160, radius = 0.6];
        \draw [-{Latex},color=blue] (4.2, 1.7) arc [start angle = 135, end angle = -160, radius = 0.6];
        \node [contact,color=green!71!black] (bottomc) at (3, 0) {};
        \node [below] (bottom) at (3, 0) {$2$};
        \node [above] (top) at (3, 3) {$1$};
    \end{tikzpicture}

    \begin{align*}
        &\begin{cases}
            {\color{red} \ele_1 = \eli_1 r_1 - \eli R}, \\
            {\color{blue} -\ele_2 = -\eli_2 r_2 + \eli R}, \\
            {\color{green!71!black} - \eli - \eli_1 - \eli_2 = 0 };
        \end{cases}
        \qquad \implies \qquad
        \begin{cases}
            \eli_1 = \frac{\ele_1 + \eli R}{r_1}, \\
            \eli_2 = \frac{\ele_2 + \eli R}{r_2}, \\
            \eli + \eli_1 + \eli_2 = 0;
        \end{cases} \implies \\
        &\implies
         \eli + \frac{\ele_1 + \eli R}{r_1:L} + \frac{\ele_2 + \eli R}{r_2:L} = 0, \\
        &\eli\cbr{ 1 + \frac R{r_1:L} + \frac R{r_2:L}} + \frac{\ele_1 }{r_1:L} + \frac{\ele_2 }{r_2:L} = 0, \\
        &\eli
            = - \frac{\frac{\ele_1 }{r_1:L} + \frac{\ele_2 }{r_2:L}}{ 1 + \frac R{r_1:L} + \frac R{r_2:L}}
            = - \frac{\frac{60\,\text{В}}{2\,\text{Ом}} + \frac{40\,\text{В}}{3\,\text{Ом}}}{ 1 + \frac{15\,\text{Ом}}{2\,\text{Ом}} + \frac{15\,\text{Ом}}{3\,\text{Ом}}}
            = - \frac{260}{81}\units{А}
            \approx -3{,}20\,\text{А}, \\
        &U  = \varphi_2 - \varphi_1 = \eli R
            = - \frac{\frac{\ele_1 }{r_1:L} + \frac{\ele_2 }{r_2:L}}{ 1 + \frac R{r_1:L} + \frac R{r_2:L}} R
            \approx -48{,}10\,\text{В}.
    \end{align*}
    Оба ответа отрицательны, потому что мы изначально «не угадали» с направлением тока.
    Расчёт же показал,
    что ток через резистор $R$ течёт в противоположную сторону: вниз на рисунке, а потенциал точки 1 больше потенциала точки 2,
    а электрический ток ожидаемо течёт из точки с большим потенциалов в точку с меньшим.

    Кстати, если продолжить расчёт и вычислить значения ещё двух токов (формулы для $\eli_1$ и $\eli_2$, куда подставлять, выписаны выше),
    то по их знакам можно будет понять: угадали ли мы с их направлением или нет.
}

\variantsplitter

\addpersonalvariant{Виктория Легонькова}

\tasknumber{1}%
\task{%
    Получите выражение:
    \begin{enumerate}
        \item площади поперечного сечения проводника через его сопротивление,
        \item сопротивление из закона Ома,
        \item внутреннее сопротивление цепи из закона Ома для полной цепи,
        \item эквивалентное сопротивление $n$ резисторов, соединённых параллельно, каждый сопротивлением $R$.
    \end{enumerate}
}
\solutionspace{40pt}

\tasknumber{2}%
\task{%
    Получите выражение:
    \begin{enumerate}
        \item силы тока через выделяемую мощность и напряжение на резисторе,
        \item силы тока через выделенную теплоту и сопротивление резистора,
        \item напряжение на резисторе через выделяемую мощность и сопротивление резистора,
        \item напряжение на резисторе через выделенную в нём теплоту и сопротивление резистора.
    \end{enumerate}
}
\solutionspace{80pt}

\tasknumber{3}%
\task{%
    Определите ток, протекающий через резистор $R = 12\,\text{Ом}$ и разность потенциалов на нём (см.
    рис.
    на доске),
    если $r_1 = 3\,\text{Ом}$, $r_2 = 2\,\text{Ом}$, $\ele_1 = 60\,\text{В}$, $\ele_2 = 60\,\text{В}$.
}
\answer{%
    Обозначим на рисунке все токи: направление произвольно, но его надо зафиксировать.
    Всего на рисунке 3 контура и 2 узла.
    Поэтому можно записать $3 - 1 = 2$ уравнения законов Кирхгофа для замкнутого контура и $2 - 1 = 1$ — для узлов
    (остальные уравнения тоже можно записать, но они не дадут полезной информации, а будут лишь следствиями уже записанных).

    Отметим на рисунке 2 контура (и не забуем указать направление) и 1 узел (точка «1»ы, выделена жирным).
    Выбор контуров и узлов не критичен: получившаяся система может быть чуть проще или сложнее, но не слишком.

    \begin{tikzpicture}[circuit ee IEC, thick]
        \draw  (0, 0) to [current direction={near end, info=$\eli_1$}] (0, 3)
                to [battery={rotate=-180,info={$\ele_1, r_1$}}]
                (3, 3)
                to [battery={info'={$\ele_2, r_2$}}]
                (6, 3) to [current direction'={near start, info=$\eli_2$}] (6, 0) -- (0, 0)
                (3, 0) to [current direction={near start, info=$\eli$}, resistor={near end, info=$R$}] (3, 3);
        \draw [-{Latex},color=red] (1.2, 1.7) arc [start angle = 135, end angle = -160, radius = 0.6];
        \draw [-{Latex},color=blue] (4.2, 1.7) arc [start angle = 135, end angle = -160, radius = 0.6];
        \node [contact,color=green!71!black] (bottomc) at (3, 0) {};
        \node [below] (bottom) at (3, 0) {$2$};
        \node [above] (top) at (3, 3) {$1$};
    \end{tikzpicture}

    \begin{align*}
        &\begin{cases}
            {\color{red} \ele_1 = \eli_1 r_1 - \eli R}, \\
            {\color{blue} -\ele_2 = -\eli_2 r_2 + \eli R}, \\
            {\color{green!71!black} - \eli - \eli_1 - \eli_2 = 0 };
        \end{cases}
        \qquad \implies \qquad
        \begin{cases}
            \eli_1 = \frac{\ele_1 + \eli R}{r_1}, \\
            \eli_2 = \frac{\ele_2 + \eli R}{r_2}, \\
            \eli + \eli_1 + \eli_2 = 0;
        \end{cases} \implies \\
        &\implies
         \eli + \frac{\ele_1 + \eli R}{r_1:L} + \frac{\ele_2 + \eli R}{r_2:L} = 0, \\
        &\eli\cbr{ 1 + \frac R{r_1:L} + \frac R{r_2:L}} + \frac{\ele_1 }{r_1:L} + \frac{\ele_2 }{r_2:L} = 0, \\
        &\eli
            = - \frac{\frac{\ele_1 }{r_1:L} + \frac{\ele_2 }{r_2:L}}{ 1 + \frac R{r_1:L} + \frac R{r_2:L}}
            = - \frac{\frac{60\,\text{В}}{3\,\text{Ом}} + \frac{60\,\text{В}}{2\,\text{Ом}}}{ 1 + \frac{12\,\text{Ом}}{3\,\text{Ом}} + \frac{12\,\text{Ом}}{2\,\text{Ом}}}
            = - \frac{50}{11}\units{А}
            \approx -4{,}50\,\text{А}, \\
        &U  = \varphi_2 - \varphi_1 = \eli R
            = - \frac{\frac{\ele_1 }{r_1:L} + \frac{\ele_2 }{r_2:L}}{ 1 + \frac R{r_1:L} + \frac R{r_2:L}} R
            \approx -54{,}50\,\text{В}.
    \end{align*}
    Оба ответа отрицательны, потому что мы изначально «не угадали» с направлением тока.
    Расчёт же показал,
    что ток через резистор $R$ течёт в противоположную сторону: вниз на рисунке, а потенциал точки 1 больше потенциала точки 2,
    а электрический ток ожидаемо течёт из точки с большим потенциалов в точку с меньшим.

    Кстати, если продолжить расчёт и вычислить значения ещё двух токов (формулы для $\eli_1$ и $\eli_2$, куда подставлять, выписаны выше),
    то по их знакам можно будет понять: угадали ли мы с их направлением или нет.
}

\variantsplitter

\addpersonalvariant{Семён Мартынов}

\tasknumber{1}%
\task{%
    Получите выражение:
    \begin{enumerate}
        \item длины проводника через его сопротивление,
        \item удельное сопротивление из закона Ома,
        \item внутреннее сопротивление цепи из закона Ома для полной цепи,
        \item эквивалентное сопротивление $n$ резисторов, соединённых параллельно, каждый сопротивлением $R$.
    \end{enumerate}
}
\solutionspace{40pt}

\tasknumber{2}%
\task{%
    Получите выражение:
    \begin{enumerate}
        \item силы тока через выделяемую мощность и напряжение на резисторе,
        \item силы тока через выделенную теплоту и разность потенциалов на резисторе,
        \item напряжение на резисторе через выделяемую мощность и силу тока через него,
        \item напряжение на резисторе через выделенную в нём теплоту и силу тока через него.
    \end{enumerate}
}
\solutionspace{80pt}

\tasknumber{3}%
\task{%
    Определите ток, протекающий через резистор $R = 20\,\text{Ом}$ и разность потенциалов на нём (см.
    рис.
    на доске),
    если $r_1 = 1\,\text{Ом}$, $r_2 = 1\,\text{Ом}$, $\ele_1 = 30\,\text{В}$, $\ele_2 = 40\,\text{В}$.
}
\answer{%
    Обозначим на рисунке все токи: направление произвольно, но его надо зафиксировать.
    Всего на рисунке 3 контура и 2 узла.
    Поэтому можно записать $3 - 1 = 2$ уравнения законов Кирхгофа для замкнутого контура и $2 - 1 = 1$ — для узлов
    (остальные уравнения тоже можно записать, но они не дадут полезной информации, а будут лишь следствиями уже записанных).

    Отметим на рисунке 2 контура (и не забуем указать направление) и 1 узел (точка «1»ы, выделена жирным).
    Выбор контуров и узлов не критичен: получившаяся система может быть чуть проще или сложнее, но не слишком.

    \begin{tikzpicture}[circuit ee IEC, thick]
        \draw  (0, 0) to [current direction={near end, info=$\eli_1$}] (0, 3)
                to [battery={rotate=-180,info={$\ele_1, r_1$}}]
                (3, 3)
                to [battery={info'={$\ele_2, r_2$}}]
                (6, 3) to [current direction'={near start, info=$\eli_2$}] (6, 0) -- (0, 0)
                (3, 0) to [current direction={near start, info=$\eli$}, resistor={near end, info=$R$}] (3, 3);
        \draw [-{Latex},color=red] (1.2, 1.7) arc [start angle = 135, end angle = -160, radius = 0.6];
        \draw [-{Latex},color=blue] (4.2, 1.7) arc [start angle = 135, end angle = -160, radius = 0.6];
        \node [contact,color=green!71!black] (bottomc) at (3, 0) {};
        \node [below] (bottom) at (3, 0) {$2$};
        \node [above] (top) at (3, 3) {$1$};
    \end{tikzpicture}

    \begin{align*}
        &\begin{cases}
            {\color{red} \ele_1 = \eli_1 r_1 - \eli R}, \\
            {\color{blue} -\ele_2 = -\eli_2 r_2 + \eli R}, \\
            {\color{green!71!black} - \eli - \eli_1 - \eli_2 = 0 };
        \end{cases}
        \qquad \implies \qquad
        \begin{cases}
            \eli_1 = \frac{\ele_1 + \eli R}{r_1}, \\
            \eli_2 = \frac{\ele_2 + \eli R}{r_2}, \\
            \eli + \eli_1 + \eli_2 = 0;
        \end{cases} \implies \\
        &\implies
         \eli + \frac{\ele_1 + \eli R}{r_1:L} + \frac{\ele_2 + \eli R}{r_2:L} = 0, \\
        &\eli\cbr{ 1 + \frac R{r_1:L} + \frac R{r_2:L}} + \frac{\ele_1 }{r_1:L} + \frac{\ele_2 }{r_2:L} = 0, \\
        &\eli
            = - \frac{\frac{\ele_1 }{r_1:L} + \frac{\ele_2 }{r_2:L}}{ 1 + \frac R{r_1:L} + \frac R{r_2:L}}
            = - \frac{\frac{30\,\text{В}}{1\,\text{Ом}} + \frac{40\,\text{В}}{1\,\text{Ом}}}{ 1 + \frac{20\,\text{Ом}}{1\,\text{Ом}} + \frac{20\,\text{Ом}}{1\,\text{Ом}}}
            = - \frac{70}{41}\units{А}
            \approx -1{,}700\,\text{А}, \\
        &U  = \varphi_2 - \varphi_1 = \eli R
            = - \frac{\frac{\ele_1 }{r_1:L} + \frac{\ele_2 }{r_2:L}}{ 1 + \frac R{r_1:L} + \frac R{r_2:L}} R
            \approx -34{,}10\,\text{В}.
    \end{align*}
    Оба ответа отрицательны, потому что мы изначально «не угадали» с направлением тока.
    Расчёт же показал,
    что ток через резистор $R$ течёт в противоположную сторону: вниз на рисунке, а потенциал точки 1 больше потенциала точки 2,
    а электрический ток ожидаемо течёт из точки с большим потенциалов в точку с меньшим.

    Кстати, если продолжить расчёт и вычислить значения ещё двух токов (формулы для $\eli_1$ и $\eli_2$, куда подставлять, выписаны выше),
    то по их знакам можно будет понять: угадали ли мы с их направлением или нет.
}

\variantsplitter

\addpersonalvariant{Варвара Минаева}

\tasknumber{1}%
\task{%
    Получите выражение:
    \begin{enumerate}
        \item длины проводника через его сопротивление,
        \item сопротивление из закона Ома,
        \item внешнее сопротивление цепи из закона Ома для полной цепи,
        \item эквивалентное сопротивление $n$ резисторов, соединённых последовательно, каждый сопротивлением $R$.
    \end{enumerate}
}
\solutionspace{40pt}

\tasknumber{2}%
\task{%
    Получите выражение:
    \begin{enumerate}
        \item силы тока через выделяемую мощность и сопротивление резистора,
        \item силы тока через выделенную теплоту и напряжение на резисторе,
        \item напряжение на резисторе через выделяемую мощность и силу тока через него,
        \item напряжение на резисторе через выделенную в нём теплоту и сопротивление резистора.
    \end{enumerate}
}
\solutionspace{80pt}

\tasknumber{3}%
\task{%
    Определите ток, протекающий через резистор $R = 12\,\text{Ом}$ и разность потенциалов на нём (см.
    рис.
    на доске),
    если $r_1 = 2\,\text{Ом}$, $r_2 = 3\,\text{Ом}$, $\ele_1 = 40\,\text{В}$, $\ele_2 = 40\,\text{В}$.
}
\answer{%
    Обозначим на рисунке все токи: направление произвольно, но его надо зафиксировать.
    Всего на рисунке 3 контура и 2 узла.
    Поэтому можно записать $3 - 1 = 2$ уравнения законов Кирхгофа для замкнутого контура и $2 - 1 = 1$ — для узлов
    (остальные уравнения тоже можно записать, но они не дадут полезной информации, а будут лишь следствиями уже записанных).

    Отметим на рисунке 2 контура (и не забуем указать направление) и 1 узел (точка «1»ы, выделена жирным).
    Выбор контуров и узлов не критичен: получившаяся система может быть чуть проще или сложнее, но не слишком.

    \begin{tikzpicture}[circuit ee IEC, thick]
        \draw  (0, 0) to [current direction={near end, info=$\eli_1$}] (0, 3)
                to [battery={rotate=-180,info={$\ele_1, r_1$}}]
                (3, 3)
                to [battery={info'={$\ele_2, r_2$}}]
                (6, 3) to [current direction'={near start, info=$\eli_2$}] (6, 0) -- (0, 0)
                (3, 0) to [current direction={near start, info=$\eli$}, resistor={near end, info=$R$}] (3, 3);
        \draw [-{Latex},color=red] (1.2, 1.7) arc [start angle = 135, end angle = -160, radius = 0.6];
        \draw [-{Latex},color=blue] (4.2, 1.7) arc [start angle = 135, end angle = -160, radius = 0.6];
        \node [contact,color=green!71!black] (bottomc) at (3, 0) {};
        \node [below] (bottom) at (3, 0) {$2$};
        \node [above] (top) at (3, 3) {$1$};
    \end{tikzpicture}

    \begin{align*}
        &\begin{cases}
            {\color{red} \ele_1 = \eli_1 r_1 - \eli R}, \\
            {\color{blue} -\ele_2 = -\eli_2 r_2 + \eli R}, \\
            {\color{green!71!black} - \eli - \eli_1 - \eli_2 = 0 };
        \end{cases}
        \qquad \implies \qquad
        \begin{cases}
            \eli_1 = \frac{\ele_1 + \eli R}{r_1}, \\
            \eli_2 = \frac{\ele_2 + \eli R}{r_2}, \\
            \eli + \eli_1 + \eli_2 = 0;
        \end{cases} \implies \\
        &\implies
         \eli + \frac{\ele_1 + \eli R}{r_1:L} + \frac{\ele_2 + \eli R}{r_2:L} = 0, \\
        &\eli\cbr{ 1 + \frac R{r_1:L} + \frac R{r_2:L}} + \frac{\ele_1 }{r_1:L} + \frac{\ele_2 }{r_2:L} = 0, \\
        &\eli
            = - \frac{\frac{\ele_1 }{r_1:L} + \frac{\ele_2 }{r_2:L}}{ 1 + \frac R{r_1:L} + \frac R{r_2:L}}
            = - \frac{\frac{40\,\text{В}}{2\,\text{Ом}} + \frac{40\,\text{В}}{3\,\text{Ом}}}{ 1 + \frac{12\,\text{Ом}}{2\,\text{Ом}} + \frac{12\,\text{Ом}}{3\,\text{Ом}}}
            = - \frac{100}{33}\units{А}
            \approx -3{,}00\,\text{А}, \\
        &U  = \varphi_2 - \varphi_1 = \eli R
            = - \frac{\frac{\ele_1 }{r_1:L} + \frac{\ele_2 }{r_2:L}}{ 1 + \frac R{r_1:L} + \frac R{r_2:L}} R
            \approx -36{,}40\,\text{В}.
    \end{align*}
    Оба ответа отрицательны, потому что мы изначально «не угадали» с направлением тока.
    Расчёт же показал,
    что ток через резистор $R$ течёт в противоположную сторону: вниз на рисунке, а потенциал точки 1 больше потенциала точки 2,
    а электрический ток ожидаемо течёт из точки с большим потенциалов в точку с меньшим.

    Кстати, если продолжить расчёт и вычислить значения ещё двух токов (формулы для $\eli_1$ и $\eli_2$, куда подставлять, выписаны выше),
    то по их знакам можно будет понять: угадали ли мы с их направлением или нет.
}

\variantsplitter

\addpersonalvariant{Леонид Никитин}

\tasknumber{1}%
\task{%
    Получите выражение:
    \begin{enumerate}
        \item площади поперечного сечения проводника через его сопротивление,
        \item удельное сопротивление из закона Ома,
        \item внутреннее сопротивление цепи из закона Ома для полной цепи,
        \item эквивалентное сопротивление $n$ резисторов, соединённых параллельно, каждый сопротивлением $R$.
    \end{enumerate}
}
\solutionspace{40pt}

\tasknumber{2}%
\task{%
    Получите выражение:
    \begin{enumerate}
        \item силы тока через выделяемую мощность и разность потенциалов на резисторе,
        \item силы тока через выделенную теплоту и сопротивление резистора,
        \item напряжение на резисторе через выделяемую мощность и силу тока через него,
        \item напряжение на резисторе через выделенную в нём теплоту и сопротивление резистора.
    \end{enumerate}
}
\solutionspace{80pt}

\tasknumber{3}%
\task{%
    Определите ток, протекающий через резистор $R = 18\,\text{Ом}$ и разность потенциалов на нём (см.
    рис.
    на доске),
    если $r_1 = 1\,\text{Ом}$, $r_2 = 3\,\text{Ом}$, $\ele_1 = 20\,\text{В}$, $\ele_2 = 40\,\text{В}$.
}
\answer{%
    Обозначим на рисунке все токи: направление произвольно, но его надо зафиксировать.
    Всего на рисунке 3 контура и 2 узла.
    Поэтому можно записать $3 - 1 = 2$ уравнения законов Кирхгофа для замкнутого контура и $2 - 1 = 1$ — для узлов
    (остальные уравнения тоже можно записать, но они не дадут полезной информации, а будут лишь следствиями уже записанных).

    Отметим на рисунке 2 контура (и не забуем указать направление) и 1 узел (точка «1»ы, выделена жирным).
    Выбор контуров и узлов не критичен: получившаяся система может быть чуть проще или сложнее, но не слишком.

    \begin{tikzpicture}[circuit ee IEC, thick]
        \draw  (0, 0) to [current direction={near end, info=$\eli_1$}] (0, 3)
                to [battery={rotate=-180,info={$\ele_1, r_1$}}]
                (3, 3)
                to [battery={info'={$\ele_2, r_2$}}]
                (6, 3) to [current direction'={near start, info=$\eli_2$}] (6, 0) -- (0, 0)
                (3, 0) to [current direction={near start, info=$\eli$}, resistor={near end, info=$R$}] (3, 3);
        \draw [-{Latex},color=red] (1.2, 1.7) arc [start angle = 135, end angle = -160, radius = 0.6];
        \draw [-{Latex},color=blue] (4.2, 1.7) arc [start angle = 135, end angle = -160, radius = 0.6];
        \node [contact,color=green!71!black] (bottomc) at (3, 0) {};
        \node [below] (bottom) at (3, 0) {$2$};
        \node [above] (top) at (3, 3) {$1$};
    \end{tikzpicture}

    \begin{align*}
        &\begin{cases}
            {\color{red} \ele_1 = \eli_1 r_1 - \eli R}, \\
            {\color{blue} -\ele_2 = -\eli_2 r_2 + \eli R}, \\
            {\color{green!71!black} - \eli - \eli_1 - \eli_2 = 0 };
        \end{cases}
        \qquad \implies \qquad
        \begin{cases}
            \eli_1 = \frac{\ele_1 + \eli R}{r_1}, \\
            \eli_2 = \frac{\ele_2 + \eli R}{r_2}, \\
            \eli + \eli_1 + \eli_2 = 0;
        \end{cases} \implies \\
        &\implies
         \eli + \frac{\ele_1 + \eli R}{r_1:L} + \frac{\ele_2 + \eli R}{r_2:L} = 0, \\
        &\eli\cbr{ 1 + \frac R{r_1:L} + \frac R{r_2:L}} + \frac{\ele_1 }{r_1:L} + \frac{\ele_2 }{r_2:L} = 0, \\
        &\eli
            = - \frac{\frac{\ele_1 }{r_1:L} + \frac{\ele_2 }{r_2:L}}{ 1 + \frac R{r_1:L} + \frac R{r_2:L}}
            = - \frac{\frac{20\,\text{В}}{1\,\text{Ом}} + \frac{40\,\text{В}}{3\,\text{Ом}}}{ 1 + \frac{18\,\text{Ом}}{1\,\text{Ом}} + \frac{18\,\text{Ом}}{3\,\text{Ом}}}
            = - \frac43\units{А}
            \approx -1{,}300\,\text{А}, \\
        &U  = \varphi_2 - \varphi_1 = \eli R
            = - \frac{\frac{\ele_1 }{r_1:L} + \frac{\ele_2 }{r_2:L}}{ 1 + \frac R{r_1:L} + \frac R{r_2:L}} R
            \approx -24{,}00\,\text{В}.
    \end{align*}
    Оба ответа отрицательны, потому что мы изначально «не угадали» с направлением тока.
    Расчёт же показал,
    что ток через резистор $R$ течёт в противоположную сторону: вниз на рисунке, а потенциал точки 1 больше потенциала точки 2,
    а электрический ток ожидаемо течёт из точки с большим потенциалов в точку с меньшим.

    Кстати, если продолжить расчёт и вычислить значения ещё двух токов (формулы для $\eli_1$ и $\eli_2$, куда подставлять, выписаны выше),
    то по их знакам можно будет понять: угадали ли мы с их направлением или нет.
}

\variantsplitter

\addpersonalvariant{Тимофей Полетаев}

\tasknumber{1}%
\task{%
    Получите выражение:
    \begin{enumerate}
        \item длины проводника через его сопротивление,
        \item сопротивление из закона Ома,
        \item внешнее сопротивление цепи из закона Ома для полной цепи,
        \item эквивалентное сопротивление $n$ резисторов, соединённых последовательно, каждый сопротивлением $R$.
    \end{enumerate}
}
\solutionspace{40pt}

\tasknumber{2}%
\task{%
    Получите выражение:
    \begin{enumerate}
        \item силы тока через выделяемую мощность и разность потенциалов на резисторе,
        \item силы тока через выделенную теплоту и напряжение на резисторе,
        \item напряжение на резисторе через выделяемую мощность и силу тока через него,
        \item напряжение на резисторе через выделенную в нём теплоту и силу тока через него.
    \end{enumerate}
}
\solutionspace{80pt}

\tasknumber{3}%
\task{%
    Определите ток, протекающий через резистор $R = 20\,\text{Ом}$ и разность потенциалов на нём (см.
    рис.
    на доске),
    если $r_1 = 2\,\text{Ом}$, $r_2 = 2\,\text{Ом}$, $\ele_1 = 60\,\text{В}$, $\ele_2 = 40\,\text{В}$.
}
\answer{%
    Обозначим на рисунке все токи: направление произвольно, но его надо зафиксировать.
    Всего на рисунке 3 контура и 2 узла.
    Поэтому можно записать $3 - 1 = 2$ уравнения законов Кирхгофа для замкнутого контура и $2 - 1 = 1$ — для узлов
    (остальные уравнения тоже можно записать, но они не дадут полезной информации, а будут лишь следствиями уже записанных).

    Отметим на рисунке 2 контура (и не забуем указать направление) и 1 узел (точка «1»ы, выделена жирным).
    Выбор контуров и узлов не критичен: получившаяся система может быть чуть проще или сложнее, но не слишком.

    \begin{tikzpicture}[circuit ee IEC, thick]
        \draw  (0, 0) to [current direction={near end, info=$\eli_1$}] (0, 3)
                to [battery={rotate=-180,info={$\ele_1, r_1$}}]
                (3, 3)
                to [battery={info'={$\ele_2, r_2$}}]
                (6, 3) to [current direction'={near start, info=$\eli_2$}] (6, 0) -- (0, 0)
                (3, 0) to [current direction={near start, info=$\eli$}, resistor={near end, info=$R$}] (3, 3);
        \draw [-{Latex},color=red] (1.2, 1.7) arc [start angle = 135, end angle = -160, radius = 0.6];
        \draw [-{Latex},color=blue] (4.2, 1.7) arc [start angle = 135, end angle = -160, radius = 0.6];
        \node [contact,color=green!71!black] (bottomc) at (3, 0) {};
        \node [below] (bottom) at (3, 0) {$2$};
        \node [above] (top) at (3, 3) {$1$};
    \end{tikzpicture}

    \begin{align*}
        &\begin{cases}
            {\color{red} \ele_1 = \eli_1 r_1 - \eli R}, \\
            {\color{blue} -\ele_2 = -\eli_2 r_2 + \eli R}, \\
            {\color{green!71!black} - \eli - \eli_1 - \eli_2 = 0 };
        \end{cases}
        \qquad \implies \qquad
        \begin{cases}
            \eli_1 = \frac{\ele_1 + \eli R}{r_1}, \\
            \eli_2 = \frac{\ele_2 + \eli R}{r_2}, \\
            \eli + \eli_1 + \eli_2 = 0;
        \end{cases} \implies \\
        &\implies
         \eli + \frac{\ele_1 + \eli R}{r_1:L} + \frac{\ele_2 + \eli R}{r_2:L} = 0, \\
        &\eli\cbr{ 1 + \frac R{r_1:L} + \frac R{r_2:L}} + \frac{\ele_1 }{r_1:L} + \frac{\ele_2 }{r_2:L} = 0, \\
        &\eli
            = - \frac{\frac{\ele_1 }{r_1:L} + \frac{\ele_2 }{r_2:L}}{ 1 + \frac R{r_1:L} + \frac R{r_2:L}}
            = - \frac{\frac{60\,\text{В}}{2\,\text{Ом}} + \frac{40\,\text{В}}{2\,\text{Ом}}}{ 1 + \frac{20\,\text{Ом}}{2\,\text{Ом}} + \frac{20\,\text{Ом}}{2\,\text{Ом}}}
            = - \frac{50}{21}\units{А}
            \approx -2{,}40\,\text{А}, \\
        &U  = \varphi_2 - \varphi_1 = \eli R
            = - \frac{\frac{\ele_1 }{r_1:L} + \frac{\ele_2 }{r_2:L}}{ 1 + \frac R{r_1:L} + \frac R{r_2:L}} R
            \approx -47{,}60\,\text{В}.
    \end{align*}
    Оба ответа отрицательны, потому что мы изначально «не угадали» с направлением тока.
    Расчёт же показал,
    что ток через резистор $R$ течёт в противоположную сторону: вниз на рисунке, а потенциал точки 1 больше потенциала точки 2,
    а электрический ток ожидаемо течёт из точки с большим потенциалов в точку с меньшим.

    Кстати, если продолжить расчёт и вычислить значения ещё двух токов (формулы для $\eli_1$ и $\eli_2$, куда подставлять, выписаны выше),
    то по их знакам можно будет понять: угадали ли мы с их направлением или нет.
}

\variantsplitter

\addpersonalvariant{Андрей Рожков}

\tasknumber{1}%
\task{%
    Получите выражение:
    \begin{enumerate}
        \item длины проводника через его сопротивление,
        \item удельное сопротивление из закона Ома,
        \item внешнее сопротивление цепи из закона Ома для полной цепи,
        \item эквивалентное сопротивление $n$ резисторов, соединённых параллельно, каждый сопротивлением $R$.
    \end{enumerate}
}
\solutionspace{40pt}

\tasknumber{2}%
\task{%
    Получите выражение:
    \begin{enumerate}
        \item силы тока через выделяемую мощность и сопротивление резистора,
        \item силы тока через выделенную теплоту и напряжение на резисторе,
        \item напряжение на резисторе через выделяемую мощность и силу тока через него,
        \item напряжение на резисторе через выделенную в нём теплоту и силу тока через него.
    \end{enumerate}
}
\solutionspace{80pt}

\tasknumber{3}%
\task{%
    Определите ток, протекающий через резистор $R = 10\,\text{Ом}$ и разность потенциалов на нём (см.
    рис.
    на доске),
    если $r_1 = 1\,\text{Ом}$, $r_2 = 1\,\text{Ом}$, $\ele_1 = 30\,\text{В}$, $\ele_2 = 30\,\text{В}$.
}
\answer{%
    Обозначим на рисунке все токи: направление произвольно, но его надо зафиксировать.
    Всего на рисунке 3 контура и 2 узла.
    Поэтому можно записать $3 - 1 = 2$ уравнения законов Кирхгофа для замкнутого контура и $2 - 1 = 1$ — для узлов
    (остальные уравнения тоже можно записать, но они не дадут полезной информации, а будут лишь следствиями уже записанных).

    Отметим на рисунке 2 контура (и не забуем указать направление) и 1 узел (точка «1»ы, выделена жирным).
    Выбор контуров и узлов не критичен: получившаяся система может быть чуть проще или сложнее, но не слишком.

    \begin{tikzpicture}[circuit ee IEC, thick]
        \draw  (0, 0) to [current direction={near end, info=$\eli_1$}] (0, 3)
                to [battery={rotate=-180,info={$\ele_1, r_1$}}]
                (3, 3)
                to [battery={info'={$\ele_2, r_2$}}]
                (6, 3) to [current direction'={near start, info=$\eli_2$}] (6, 0) -- (0, 0)
                (3, 0) to [current direction={near start, info=$\eli$}, resistor={near end, info=$R$}] (3, 3);
        \draw [-{Latex},color=red] (1.2, 1.7) arc [start angle = 135, end angle = -160, radius = 0.6];
        \draw [-{Latex},color=blue] (4.2, 1.7) arc [start angle = 135, end angle = -160, radius = 0.6];
        \node [contact,color=green!71!black] (bottomc) at (3, 0) {};
        \node [below] (bottom) at (3, 0) {$2$};
        \node [above] (top) at (3, 3) {$1$};
    \end{tikzpicture}

    \begin{align*}
        &\begin{cases}
            {\color{red} \ele_1 = \eli_1 r_1 - \eli R}, \\
            {\color{blue} -\ele_2 = -\eli_2 r_2 + \eli R}, \\
            {\color{green!71!black} - \eli - \eli_1 - \eli_2 = 0 };
        \end{cases}
        \qquad \implies \qquad
        \begin{cases}
            \eli_1 = \frac{\ele_1 + \eli R}{r_1}, \\
            \eli_2 = \frac{\ele_2 + \eli R}{r_2}, \\
            \eli + \eli_1 + \eli_2 = 0;
        \end{cases} \implies \\
        &\implies
         \eli + \frac{\ele_1 + \eli R}{r_1:L} + \frac{\ele_2 + \eli R}{r_2:L} = 0, \\
        &\eli\cbr{ 1 + \frac R{r_1:L} + \frac R{r_2:L}} + \frac{\ele_1 }{r_1:L} + \frac{\ele_2 }{r_2:L} = 0, \\
        &\eli
            = - \frac{\frac{\ele_1 }{r_1:L} + \frac{\ele_2 }{r_2:L}}{ 1 + \frac R{r_1:L} + \frac R{r_2:L}}
            = - \frac{\frac{30\,\text{В}}{1\,\text{Ом}} + \frac{30\,\text{В}}{1\,\text{Ом}}}{ 1 + \frac{10\,\text{Ом}}{1\,\text{Ом}} + \frac{10\,\text{Ом}}{1\,\text{Ом}}}
            = - \frac{20}7\units{А}
            \approx -2{,}90\,\text{А}, \\
        &U  = \varphi_2 - \varphi_1 = \eli R
            = - \frac{\frac{\ele_1 }{r_1:L} + \frac{\ele_2 }{r_2:L}}{ 1 + \frac R{r_1:L} + \frac R{r_2:L}} R
            \approx -28{,}60\,\text{В}.
    \end{align*}
    Оба ответа отрицательны, потому что мы изначально «не угадали» с направлением тока.
    Расчёт же показал,
    что ток через резистор $R$ течёт в противоположную сторону: вниз на рисунке, а потенциал точки 1 больше потенциала точки 2,
    а электрический ток ожидаемо течёт из точки с большим потенциалов в точку с меньшим.

    Кстати, если продолжить расчёт и вычислить значения ещё двух токов (формулы для $\eli_1$ и $\eli_2$, куда подставлять, выписаны выше),
    то по их знакам можно будет понять: угадали ли мы с их направлением или нет.
}

\variantsplitter

\addpersonalvariant{Рената Таржиманова}

\tasknumber{1}%
\task{%
    Получите выражение:
    \begin{enumerate}
        \item площади поперечного сечения проводника через его сопротивление,
        \item удельное сопротивление из закона Ома,
        \item внутреннее сопротивление цепи из закона Ома для полной цепи,
        \item эквивалентное сопротивление $n$ резисторов, соединённых последовательно, каждый сопротивлением $R$.
    \end{enumerate}
}
\solutionspace{40pt}

\tasknumber{2}%
\task{%
    Получите выражение:
    \begin{enumerate}
        \item силы тока через выделяемую мощность и сопротивление резистора,
        \item силы тока через выделенную теплоту и разность потенциалов на резисторе,
        \item напряжение на резисторе через выделяемую мощность и силу тока через него,
        \item напряжение на резисторе через выделенную в нём теплоту и сопротивление резистора.
    \end{enumerate}
}
\solutionspace{80pt}

\tasknumber{3}%
\task{%
    Определите ток, протекающий через резистор $R = 10\,\text{Ом}$ и разность потенциалов на нём (см.
    рис.
    на доске),
    если $r_1 = 3\,\text{Ом}$, $r_2 = 3\,\text{Ом}$, $\ele_1 = 30\,\text{В}$, $\ele_2 = 60\,\text{В}$.
}
\answer{%
    Обозначим на рисунке все токи: направление произвольно, но его надо зафиксировать.
    Всего на рисунке 3 контура и 2 узла.
    Поэтому можно записать $3 - 1 = 2$ уравнения законов Кирхгофа для замкнутого контура и $2 - 1 = 1$ — для узлов
    (остальные уравнения тоже можно записать, но они не дадут полезной информации, а будут лишь следствиями уже записанных).

    Отметим на рисунке 2 контура (и не забуем указать направление) и 1 узел (точка «1»ы, выделена жирным).
    Выбор контуров и узлов не критичен: получившаяся система может быть чуть проще или сложнее, но не слишком.

    \begin{tikzpicture}[circuit ee IEC, thick]
        \draw  (0, 0) to [current direction={near end, info=$\eli_1$}] (0, 3)
                to [battery={rotate=-180,info={$\ele_1, r_1$}}]
                (3, 3)
                to [battery={info'={$\ele_2, r_2$}}]
                (6, 3) to [current direction'={near start, info=$\eli_2$}] (6, 0) -- (0, 0)
                (3, 0) to [current direction={near start, info=$\eli$}, resistor={near end, info=$R$}] (3, 3);
        \draw [-{Latex},color=red] (1.2, 1.7) arc [start angle = 135, end angle = -160, radius = 0.6];
        \draw [-{Latex},color=blue] (4.2, 1.7) arc [start angle = 135, end angle = -160, radius = 0.6];
        \node [contact,color=green!71!black] (bottomc) at (3, 0) {};
        \node [below] (bottom) at (3, 0) {$2$};
        \node [above] (top) at (3, 3) {$1$};
    \end{tikzpicture}

    \begin{align*}
        &\begin{cases}
            {\color{red} \ele_1 = \eli_1 r_1 - \eli R}, \\
            {\color{blue} -\ele_2 = -\eli_2 r_2 + \eli R}, \\
            {\color{green!71!black} - \eli - \eli_1 - \eli_2 = 0 };
        \end{cases}
        \qquad \implies \qquad
        \begin{cases}
            \eli_1 = \frac{\ele_1 + \eli R}{r_1}, \\
            \eli_2 = \frac{\ele_2 + \eli R}{r_2}, \\
            \eli + \eli_1 + \eli_2 = 0;
        \end{cases} \implies \\
        &\implies
         \eli + \frac{\ele_1 + \eli R}{r_1:L} + \frac{\ele_2 + \eli R}{r_2:L} = 0, \\
        &\eli\cbr{ 1 + \frac R{r_1:L} + \frac R{r_2:L}} + \frac{\ele_1 }{r_1:L} + \frac{\ele_2 }{r_2:L} = 0, \\
        &\eli
            = - \frac{\frac{\ele_1 }{r_1:L} + \frac{\ele_2 }{r_2:L}}{ 1 + \frac R{r_1:L} + \frac R{r_2:L}}
            = - \frac{\frac{30\,\text{В}}{3\,\text{Ом}} + \frac{60\,\text{В}}{3\,\text{Ом}}}{ 1 + \frac{10\,\text{Ом}}{3\,\text{Ом}} + \frac{10\,\text{Ом}}{3\,\text{Ом}}}
            = - \frac{90}{23}\units{А}
            \approx -3{,}90\,\text{А}, \\
        &U  = \varphi_2 - \varphi_1 = \eli R
            = - \frac{\frac{\ele_1 }{r_1:L} + \frac{\ele_2 }{r_2:L}}{ 1 + \frac R{r_1:L} + \frac R{r_2:L}} R
            \approx -39{,}10\,\text{В}.
    \end{align*}
    Оба ответа отрицательны, потому что мы изначально «не угадали» с направлением тока.
    Расчёт же показал,
    что ток через резистор $R$ течёт в противоположную сторону: вниз на рисунке, а потенциал точки 1 больше потенциала точки 2,
    а электрический ток ожидаемо течёт из точки с большим потенциалов в точку с меньшим.

    Кстати, если продолжить расчёт и вычислить значения ещё двух токов (формулы для $\eli_1$ и $\eli_2$, куда подставлять, выписаны выше),
    то по их знакам можно будет понять: угадали ли мы с их направлением или нет.
}

\variantsplitter

\addpersonalvariant{Андрей Щербаков}

\tasknumber{1}%
\task{%
    Получите выражение:
    \begin{enumerate}
        \item площади поперечного сечения проводника через его сопротивление,
        \item удельное сопротивление из закона Ома,
        \item внешнее сопротивление цепи из закона Ома для полной цепи,
        \item эквивалентное сопротивление $n$ резисторов, соединённых параллельно, каждый сопротивлением $R$.
    \end{enumerate}
}
\solutionspace{40pt}

\tasknumber{2}%
\task{%
    Получите выражение:
    \begin{enumerate}
        \item силы тока через выделяемую мощность и напряжение на резисторе,
        \item силы тока через выделенную теплоту и сопротивление резистора,
        \item напряжение на резисторе через выделяемую мощность и сопротивление резистора,
        \item напряжение на резисторе через выделенную в нём теплоту и силу тока через него.
    \end{enumerate}
}
\solutionspace{80pt}

\tasknumber{3}%
\task{%
    Определите ток, протекающий через резистор $R = 15\,\text{Ом}$ и разность потенциалов на нём (см.
    рис.
    на доске),
    если $r_1 = 2\,\text{Ом}$, $r_2 = 2\,\text{Ом}$, $\ele_1 = 30\,\text{В}$, $\ele_2 = 20\,\text{В}$.
}
\answer{%
    Обозначим на рисунке все токи: направление произвольно, но его надо зафиксировать.
    Всего на рисунке 3 контура и 2 узла.
    Поэтому можно записать $3 - 1 = 2$ уравнения законов Кирхгофа для замкнутого контура и $2 - 1 = 1$ — для узлов
    (остальные уравнения тоже можно записать, но они не дадут полезной информации, а будут лишь следствиями уже записанных).

    Отметим на рисунке 2 контура (и не забуем указать направление) и 1 узел (точка «1»ы, выделена жирным).
    Выбор контуров и узлов не критичен: получившаяся система может быть чуть проще или сложнее, но не слишком.

    \begin{tikzpicture}[circuit ee IEC, thick]
        \draw  (0, 0) to [current direction={near end, info=$\eli_1$}] (0, 3)
                to [battery={rotate=-180,info={$\ele_1, r_1$}}]
                (3, 3)
                to [battery={info'={$\ele_2, r_2$}}]
                (6, 3) to [current direction'={near start, info=$\eli_2$}] (6, 0) -- (0, 0)
                (3, 0) to [current direction={near start, info=$\eli$}, resistor={near end, info=$R$}] (3, 3);
        \draw [-{Latex},color=red] (1.2, 1.7) arc [start angle = 135, end angle = -160, radius = 0.6];
        \draw [-{Latex},color=blue] (4.2, 1.7) arc [start angle = 135, end angle = -160, radius = 0.6];
        \node [contact,color=green!71!black] (bottomc) at (3, 0) {};
        \node [below] (bottom) at (3, 0) {$2$};
        \node [above] (top) at (3, 3) {$1$};
    \end{tikzpicture}

    \begin{align*}
        &\begin{cases}
            {\color{red} \ele_1 = \eli_1 r_1 - \eli R}, \\
            {\color{blue} -\ele_2 = -\eli_2 r_2 + \eli R}, \\
            {\color{green!71!black} - \eli - \eli_1 - \eli_2 = 0 };
        \end{cases}
        \qquad \implies \qquad
        \begin{cases}
            \eli_1 = \frac{\ele_1 + \eli R}{r_1}, \\
            \eli_2 = \frac{\ele_2 + \eli R}{r_2}, \\
            \eli + \eli_1 + \eli_2 = 0;
        \end{cases} \implies \\
        &\implies
         \eli + \frac{\ele_1 + \eli R}{r_1:L} + \frac{\ele_2 + \eli R}{r_2:L} = 0, \\
        &\eli\cbr{ 1 + \frac R{r_1:L} + \frac R{r_2:L}} + \frac{\ele_1 }{r_1:L} + \frac{\ele_2 }{r_2:L} = 0, \\
        &\eli
            = - \frac{\frac{\ele_1 }{r_1:L} + \frac{\ele_2 }{r_2:L}}{ 1 + \frac R{r_1:L} + \frac R{r_2:L}}
            = - \frac{\frac{30\,\text{В}}{2\,\text{Ом}} + \frac{20\,\text{В}}{2\,\text{Ом}}}{ 1 + \frac{15\,\text{Ом}}{2\,\text{Ом}} + \frac{15\,\text{Ом}}{2\,\text{Ом}}}
            = - \frac{25}{16}\units{А}
            \approx -1{,}600\,\text{А}, \\
        &U  = \varphi_2 - \varphi_1 = \eli R
            = - \frac{\frac{\ele_1 }{r_1:L} + \frac{\ele_2 }{r_2:L}}{ 1 + \frac R{r_1:L} + \frac R{r_2:L}} R
            \approx -23{,}40\,\text{В}.
    \end{align*}
    Оба ответа отрицательны, потому что мы изначально «не угадали» с направлением тока.
    Расчёт же показал,
    что ток через резистор $R$ течёт в противоположную сторону: вниз на рисунке, а потенциал точки 1 больше потенциала точки 2,
    а электрический ток ожидаемо течёт из точки с большим потенциалов в точку с меньшим.

    Кстати, если продолжить расчёт и вычислить значения ещё двух токов (формулы для $\eli_1$ и $\eli_2$, куда подставлять, выписаны выше),
    то по их знакам можно будет понять: угадали ли мы с их направлением или нет.
}

\variantsplitter

\addpersonalvariant{Михаил Ярошевский}

\tasknumber{1}%
\task{%
    Получите выражение:
    \begin{enumerate}
        \item площади поперечного сечения проводника через его сопротивление,
        \item удельное сопротивление из закона Ома,
        \item внешнее сопротивление цепи из закона Ома для полной цепи,
        \item эквивалентное сопротивление $n$ резисторов, соединённых параллельно, каждый сопротивлением $R$.
    \end{enumerate}
}
\solutionspace{40pt}

\tasknumber{2}%
\task{%
    Получите выражение:
    \begin{enumerate}
        \item силы тока через выделяемую мощность и напряжение на резисторе,
        \item силы тока через выделенную теплоту и сопротивление резистора,
        \item напряжение на резисторе через выделяемую мощность и сопротивление резистора,
        \item напряжение на резисторе через выделенную в нём теплоту и силу тока через него.
    \end{enumerate}
}
\solutionspace{80pt}

\tasknumber{3}%
\task{%
    Определите ток, протекающий через резистор $R = 10\,\text{Ом}$ и разность потенциалов на нём (см.
    рис.
    на доске),
    если $r_1 = 2\,\text{Ом}$, $r_2 = 1\,\text{Ом}$, $\ele_1 = 60\,\text{В}$, $\ele_2 = 40\,\text{В}$.
}
\answer{%
    Обозначим на рисунке все токи: направление произвольно, но его надо зафиксировать.
    Всего на рисунке 3 контура и 2 узла.
    Поэтому можно записать $3 - 1 = 2$ уравнения законов Кирхгофа для замкнутого контура и $2 - 1 = 1$ — для узлов
    (остальные уравнения тоже можно записать, но они не дадут полезной информации, а будут лишь следствиями уже записанных).

    Отметим на рисунке 2 контура (и не забуем указать направление) и 1 узел (точка «1»ы, выделена жирным).
    Выбор контуров и узлов не критичен: получившаяся система может быть чуть проще или сложнее, но не слишком.

    \begin{tikzpicture}[circuit ee IEC, thick]
        \draw  (0, 0) to [current direction={near end, info=$\eli_1$}] (0, 3)
                to [battery={rotate=-180,info={$\ele_1, r_1$}}]
                (3, 3)
                to [battery={info'={$\ele_2, r_2$}}]
                (6, 3) to [current direction'={near start, info=$\eli_2$}] (6, 0) -- (0, 0)
                (3, 0) to [current direction={near start, info=$\eli$}, resistor={near end, info=$R$}] (3, 3);
        \draw [-{Latex},color=red] (1.2, 1.7) arc [start angle = 135, end angle = -160, radius = 0.6];
        \draw [-{Latex},color=blue] (4.2, 1.7) arc [start angle = 135, end angle = -160, radius = 0.6];
        \node [contact,color=green!71!black] (bottomc) at (3, 0) {};
        \node [below] (bottom) at (3, 0) {$2$};
        \node [above] (top) at (3, 3) {$1$};
    \end{tikzpicture}

    \begin{align*}
        &\begin{cases}
            {\color{red} \ele_1 = \eli_1 r_1 - \eli R}, \\
            {\color{blue} -\ele_2 = -\eli_2 r_2 + \eli R}, \\
            {\color{green!71!black} - \eli - \eli_1 - \eli_2 = 0 };
        \end{cases}
        \qquad \implies \qquad
        \begin{cases}
            \eli_1 = \frac{\ele_1 + \eli R}{r_1}, \\
            \eli_2 = \frac{\ele_2 + \eli R}{r_2}, \\
            \eli + \eli_1 + \eli_2 = 0;
        \end{cases} \implies \\
        &\implies
         \eli + \frac{\ele_1 + \eli R}{r_1:L} + \frac{\ele_2 + \eli R}{r_2:L} = 0, \\
        &\eli\cbr{ 1 + \frac R{r_1:L} + \frac R{r_2:L}} + \frac{\ele_1 }{r_1:L} + \frac{\ele_2 }{r_2:L} = 0, \\
        &\eli
            = - \frac{\frac{\ele_1 }{r_1:L} + \frac{\ele_2 }{r_2:L}}{ 1 + \frac R{r_1:L} + \frac R{r_2:L}}
            = - \frac{\frac{60\,\text{В}}{2\,\text{Ом}} + \frac{40\,\text{В}}{1\,\text{Ом}}}{ 1 + \frac{10\,\text{Ом}}{2\,\text{Ом}} + \frac{10\,\text{Ом}}{1\,\text{Ом}}}
            = - \frac{35}8\units{А}
            \approx -4{,}40\,\text{А}, \\
        &U  = \varphi_2 - \varphi_1 = \eli R
            = - \frac{\frac{\ele_1 }{r_1:L} + \frac{\ele_2 }{r_2:L}}{ 1 + \frac R{r_1:L} + \frac R{r_2:L}} R
            \approx -43{,}80\,\text{В}.
    \end{align*}
    Оба ответа отрицательны, потому что мы изначально «не угадали» с направлением тока.
    Расчёт же показал,
    что ток через резистор $R$ течёт в противоположную сторону: вниз на рисунке, а потенциал точки 1 больше потенциала точки 2,
    а электрический ток ожидаемо течёт из точки с большим потенциалов в точку с меньшим.

    Кстати, если продолжить расчёт и вычислить значения ещё двух токов (формулы для $\eli_1$ и $\eli_2$, куда подставлять, выписаны выше),
    то по их знакам можно будет понять: угадали ли мы с их направлением или нет.
}

\variantsplitter

\addpersonalvariant{Алексей Алимпиев}

\tasknumber{1}%
\task{%
    Получите выражение:
    \begin{enumerate}
        \item площади поперечного сечения проводника через его сопротивление,
        \item сопротивление из закона Ома,
        \item внешнее сопротивление цепи из закона Ома для полной цепи,
        \item эквивалентное сопротивление $n$ резисторов, соединённых параллельно, каждый сопротивлением $R$.
    \end{enumerate}
}
\solutionspace{40pt}

\tasknumber{2}%
\task{%
    Получите выражение:
    \begin{enumerate}
        \item силы тока через выделяемую мощность и напряжение на резисторе,
        \item силы тока через выделенную теплоту и разность потенциалов на резисторе,
        \item напряжение на резисторе через выделяемую мощность и сопротивление резистора,
        \item напряжение на резисторе через выделенную в нём теплоту и силу тока через него.
    \end{enumerate}
}
\solutionspace{80pt}

\tasknumber{3}%
\task{%
    Определите ток, протекающий через резистор $R = 12\,\text{Ом}$ и разность потенциалов на нём (см.
    рис.
    на доске),
    если $r_1 = 3\,\text{Ом}$, $r_2 = 2\,\text{Ом}$, $\ele_1 = 20\,\text{В}$, $\ele_2 = 40\,\text{В}$.
}
\answer{%
    Обозначим на рисунке все токи: направление произвольно, но его надо зафиксировать.
    Всего на рисунке 3 контура и 2 узла.
    Поэтому можно записать $3 - 1 = 2$ уравнения законов Кирхгофа для замкнутого контура и $2 - 1 = 1$ — для узлов
    (остальные уравнения тоже можно записать, но они не дадут полезной информации, а будут лишь следствиями уже записанных).

    Отметим на рисунке 2 контура (и не забуем указать направление) и 1 узел (точка «1»ы, выделена жирным).
    Выбор контуров и узлов не критичен: получившаяся система может быть чуть проще или сложнее, но не слишком.

    \begin{tikzpicture}[circuit ee IEC, thick]
        \draw  (0, 0) to [current direction={near end, info=$\eli_1$}] (0, 3)
                to [battery={rotate=-180,info={$\ele_1, r_1$}}]
                (3, 3)
                to [battery={info'={$\ele_2, r_2$}}]
                (6, 3) to [current direction'={near start, info=$\eli_2$}] (6, 0) -- (0, 0)
                (3, 0) to [current direction={near start, info=$\eli$}, resistor={near end, info=$R$}] (3, 3);
        \draw [-{Latex},color=red] (1.2, 1.7) arc [start angle = 135, end angle = -160, radius = 0.6];
        \draw [-{Latex},color=blue] (4.2, 1.7) arc [start angle = 135, end angle = -160, radius = 0.6];
        \node [contact,color=green!71!black] (bottomc) at (3, 0) {};
        \node [below] (bottom) at (3, 0) {$2$};
        \node [above] (top) at (3, 3) {$1$};
    \end{tikzpicture}

    \begin{align*}
        &\begin{cases}
            {\color{red} \ele_1 = \eli_1 r_1 - \eli R}, \\
            {\color{blue} -\ele_2 = -\eli_2 r_2 + \eli R}, \\
            {\color{green!71!black} - \eli - \eli_1 - \eli_2 = 0 };
        \end{cases}
        \qquad \implies \qquad
        \begin{cases}
            \eli_1 = \frac{\ele_1 + \eli R}{r_1}, \\
            \eli_2 = \frac{\ele_2 + \eli R}{r_2}, \\
            \eli + \eli_1 + \eli_2 = 0;
        \end{cases} \implies \\
        &\implies
         \eli + \frac{\ele_1 + \eli R}{r_1:L} + \frac{\ele_2 + \eli R}{r_2:L} = 0, \\
        &\eli\cbr{ 1 + \frac R{r_1:L} + \frac R{r_2:L}} + \frac{\ele_1 }{r_1:L} + \frac{\ele_2 }{r_2:L} = 0, \\
        &\eli
            = - \frac{\frac{\ele_1 }{r_1:L} + \frac{\ele_2 }{r_2:L}}{ 1 + \frac R{r_1:L} + \frac R{r_2:L}}
            = - \frac{\frac{20\,\text{В}}{3\,\text{Ом}} + \frac{40\,\text{В}}{2\,\text{Ом}}}{ 1 + \frac{12\,\text{Ом}}{3\,\text{Ом}} + \frac{12\,\text{Ом}}{2\,\text{Ом}}}
            = - \frac{80}{33}\units{А}
            \approx -2{,}40\,\text{А}, \\
        &U  = \varphi_2 - \varphi_1 = \eli R
            = - \frac{\frac{\ele_1 }{r_1:L} + \frac{\ele_2 }{r_2:L}}{ 1 + \frac R{r_1:L} + \frac R{r_2:L}} R
            \approx -29{,}10\,\text{В}.
    \end{align*}
    Оба ответа отрицательны, потому что мы изначально «не угадали» с направлением тока.
    Расчёт же показал,
    что ток через резистор $R$ течёт в противоположную сторону: вниз на рисунке, а потенциал точки 1 больше потенциала точки 2,
    а электрический ток ожидаемо течёт из точки с большим потенциалов в точку с меньшим.

    Кстати, если продолжить расчёт и вычислить значения ещё двух токов (формулы для $\eli_1$ и $\eli_2$, куда подставлять, выписаны выше),
    то по их знакам можно будет понять: угадали ли мы с их направлением или нет.
}

\variantsplitter

\addpersonalvariant{Евгений Васин}

\tasknumber{1}%
\task{%
    Получите выражение:
    \begin{enumerate}
        \item площади поперечного сечения проводника через его сопротивление,
        \item удельное сопротивление из закона Ома,
        \item внутреннее сопротивление цепи из закона Ома для полной цепи,
        \item эквивалентное сопротивление $n$ резисторов, соединённых последовательно, каждый сопротивлением $R$.
    \end{enumerate}
}
\solutionspace{40pt}

\tasknumber{2}%
\task{%
    Получите выражение:
    \begin{enumerate}
        \item силы тока через выделяемую мощность и сопротивление резистора,
        \item силы тока через выделенную теплоту и разность потенциалов на резисторе,
        \item напряжение на резисторе через выделяемую мощность и силу тока через него,
        \item напряжение на резисторе через выделенную в нём теплоту и сопротивление резистора.
    \end{enumerate}
}
\solutionspace{80pt}

\tasknumber{3}%
\task{%
    Определите ток, протекающий через резистор $R = 12\,\text{Ом}$ и разность потенциалов на нём (см.
    рис.
    на доске),
    если $r_1 = 1\,\text{Ом}$, $r_2 = 1\,\text{Ом}$, $\ele_1 = 20\,\text{В}$, $\ele_2 = 40\,\text{В}$.
}
\answer{%
    Обозначим на рисунке все токи: направление произвольно, но его надо зафиксировать.
    Всего на рисунке 3 контура и 2 узла.
    Поэтому можно записать $3 - 1 = 2$ уравнения законов Кирхгофа для замкнутого контура и $2 - 1 = 1$ — для узлов
    (остальные уравнения тоже можно записать, но они не дадут полезной информации, а будут лишь следствиями уже записанных).

    Отметим на рисунке 2 контура (и не забуем указать направление) и 1 узел (точка «1»ы, выделена жирным).
    Выбор контуров и узлов не критичен: получившаяся система может быть чуть проще или сложнее, но не слишком.

    \begin{tikzpicture}[circuit ee IEC, thick]
        \draw  (0, 0) to [current direction={near end, info=$\eli_1$}] (0, 3)
                to [battery={rotate=-180,info={$\ele_1, r_1$}}]
                (3, 3)
                to [battery={info'={$\ele_2, r_2$}}]
                (6, 3) to [current direction'={near start, info=$\eli_2$}] (6, 0) -- (0, 0)
                (3, 0) to [current direction={near start, info=$\eli$}, resistor={near end, info=$R$}] (3, 3);
        \draw [-{Latex},color=red] (1.2, 1.7) arc [start angle = 135, end angle = -160, radius = 0.6];
        \draw [-{Latex},color=blue] (4.2, 1.7) arc [start angle = 135, end angle = -160, radius = 0.6];
        \node [contact,color=green!71!black] (bottomc) at (3, 0) {};
        \node [below] (bottom) at (3, 0) {$2$};
        \node [above] (top) at (3, 3) {$1$};
    \end{tikzpicture}

    \begin{align*}
        &\begin{cases}
            {\color{red} \ele_1 = \eli_1 r_1 - \eli R}, \\
            {\color{blue} -\ele_2 = -\eli_2 r_2 + \eli R}, \\
            {\color{green!71!black} - \eli - \eli_1 - \eli_2 = 0 };
        \end{cases}
        \qquad \implies \qquad
        \begin{cases}
            \eli_1 = \frac{\ele_1 + \eli R}{r_1}, \\
            \eli_2 = \frac{\ele_2 + \eli R}{r_2}, \\
            \eli + \eli_1 + \eli_2 = 0;
        \end{cases} \implies \\
        &\implies
         \eli + \frac{\ele_1 + \eli R}{r_1:L} + \frac{\ele_2 + \eli R}{r_2:L} = 0, \\
        &\eli\cbr{ 1 + \frac R{r_1:L} + \frac R{r_2:L}} + \frac{\ele_1 }{r_1:L} + \frac{\ele_2 }{r_2:L} = 0, \\
        &\eli
            = - \frac{\frac{\ele_1 }{r_1:L} + \frac{\ele_2 }{r_2:L}}{ 1 + \frac R{r_1:L} + \frac R{r_2:L}}
            = - \frac{\frac{20\,\text{В}}{1\,\text{Ом}} + \frac{40\,\text{В}}{1\,\text{Ом}}}{ 1 + \frac{12\,\text{Ом}}{1\,\text{Ом}} + \frac{12\,\text{Ом}}{1\,\text{Ом}}}
            = - \frac{12}5\units{А}
            \approx -2{,}40\,\text{А}, \\
        &U  = \varphi_2 - \varphi_1 = \eli R
            = - \frac{\frac{\ele_1 }{r_1:L} + \frac{\ele_2 }{r_2:L}}{ 1 + \frac R{r_1:L} + \frac R{r_2:L}} R
            \approx -28{,}80\,\text{В}.
    \end{align*}
    Оба ответа отрицательны, потому что мы изначально «не угадали» с направлением тока.
    Расчёт же показал,
    что ток через резистор $R$ течёт в противоположную сторону: вниз на рисунке, а потенциал точки 1 больше потенциала точки 2,
    а электрический ток ожидаемо течёт из точки с большим потенциалов в точку с меньшим.

    Кстати, если продолжить расчёт и вычислить значения ещё двух токов (формулы для $\eli_1$ и $\eli_2$, куда подставлять, выписаны выше),
    то по их знакам можно будет понять: угадали ли мы с их направлением или нет.
}

\variantsplitter

\addpersonalvariant{Вячеслав Волохов}

\tasknumber{1}%
\task{%
    Получите выражение:
    \begin{enumerate}
        \item площади поперечного сечения проводника через его сопротивление,
        \item сопротивление из закона Ома,
        \item внешнее сопротивление цепи из закона Ома для полной цепи,
        \item эквивалентное сопротивление $n$ резисторов, соединённых последовательно, каждый сопротивлением $R$.
    \end{enumerate}
}
\solutionspace{40pt}

\tasknumber{2}%
\task{%
    Получите выражение:
    \begin{enumerate}
        \item силы тока через выделяемую мощность и напряжение на резисторе,
        \item силы тока через выделенную теплоту и сопротивление резистора,
        \item напряжение на резисторе через выделяемую мощность и сопротивление резистора,
        \item напряжение на резисторе через выделенную в нём теплоту и сопротивление резистора.
    \end{enumerate}
}
\solutionspace{80pt}

\tasknumber{3}%
\task{%
    Определите ток, протекающий через резистор $R = 18\,\text{Ом}$ и разность потенциалов на нём (см.
    рис.
    на доске),
    если $r_1 = 1\,\text{Ом}$, $r_2 = 3\,\text{Ом}$, $\ele_1 = 30\,\text{В}$, $\ele_2 = 40\,\text{В}$.
}
\answer{%
    Обозначим на рисунке все токи: направление произвольно, но его надо зафиксировать.
    Всего на рисунке 3 контура и 2 узла.
    Поэтому можно записать $3 - 1 = 2$ уравнения законов Кирхгофа для замкнутого контура и $2 - 1 = 1$ — для узлов
    (остальные уравнения тоже можно записать, но они не дадут полезной информации, а будут лишь следствиями уже записанных).

    Отметим на рисунке 2 контура (и не забуем указать направление) и 1 узел (точка «1»ы, выделена жирным).
    Выбор контуров и узлов не критичен: получившаяся система может быть чуть проще или сложнее, но не слишком.

    \begin{tikzpicture}[circuit ee IEC, thick]
        \draw  (0, 0) to [current direction={near end, info=$\eli_1$}] (0, 3)
                to [battery={rotate=-180,info={$\ele_1, r_1$}}]
                (3, 3)
                to [battery={info'={$\ele_2, r_2$}}]
                (6, 3) to [current direction'={near start, info=$\eli_2$}] (6, 0) -- (0, 0)
                (3, 0) to [current direction={near start, info=$\eli$}, resistor={near end, info=$R$}] (3, 3);
        \draw [-{Latex},color=red] (1.2, 1.7) arc [start angle = 135, end angle = -160, radius = 0.6];
        \draw [-{Latex},color=blue] (4.2, 1.7) arc [start angle = 135, end angle = -160, radius = 0.6];
        \node [contact,color=green!71!black] (bottomc) at (3, 0) {};
        \node [below] (bottom) at (3, 0) {$2$};
        \node [above] (top) at (3, 3) {$1$};
    \end{tikzpicture}

    \begin{align*}
        &\begin{cases}
            {\color{red} \ele_1 = \eli_1 r_1 - \eli R}, \\
            {\color{blue} -\ele_2 = -\eli_2 r_2 + \eli R}, \\
            {\color{green!71!black} - \eli - \eli_1 - \eli_2 = 0 };
        \end{cases}
        \qquad \implies \qquad
        \begin{cases}
            \eli_1 = \frac{\ele_1 + \eli R}{r_1}, \\
            \eli_2 = \frac{\ele_2 + \eli R}{r_2}, \\
            \eli + \eli_1 + \eli_2 = 0;
        \end{cases} \implies \\
        &\implies
         \eli + \frac{\ele_1 + \eli R}{r_1:L} + \frac{\ele_2 + \eli R}{r_2:L} = 0, \\
        &\eli\cbr{ 1 + \frac R{r_1:L} + \frac R{r_2:L}} + \frac{\ele_1 }{r_1:L} + \frac{\ele_2 }{r_2:L} = 0, \\
        &\eli
            = - \frac{\frac{\ele_1 }{r_1:L} + \frac{\ele_2 }{r_2:L}}{ 1 + \frac R{r_1:L} + \frac R{r_2:L}}
            = - \frac{\frac{30\,\text{В}}{1\,\text{Ом}} + \frac{40\,\text{В}}{3\,\text{Ом}}}{ 1 + \frac{18\,\text{Ом}}{1\,\text{Ом}} + \frac{18\,\text{Ом}}{3\,\text{Ом}}}
            = - \frac{26}{15}\units{А}
            \approx -1{,}700\,\text{А}, \\
        &U  = \varphi_2 - \varphi_1 = \eli R
            = - \frac{\frac{\ele_1 }{r_1:L} + \frac{\ele_2 }{r_2:L}}{ 1 + \frac R{r_1:L} + \frac R{r_2:L}} R
            \approx -31{,}20\,\text{В}.
    \end{align*}
    Оба ответа отрицательны, потому что мы изначально «не угадали» с направлением тока.
    Расчёт же показал,
    что ток через резистор $R$ течёт в противоположную сторону: вниз на рисунке, а потенциал точки 1 больше потенциала точки 2,
    а электрический ток ожидаемо течёт из точки с большим потенциалов в точку с меньшим.

    Кстати, если продолжить расчёт и вычислить значения ещё двух токов (формулы для $\eli_1$ и $\eli_2$, куда подставлять, выписаны выше),
    то по их знакам можно будет понять: угадали ли мы с их направлением или нет.
}

\variantsplitter

\addpersonalvariant{Герман Говоров}

\tasknumber{1}%
\task{%
    Получите выражение:
    \begin{enumerate}
        \item длины проводника через его сопротивление,
        \item сопротивление из закона Ома,
        \item внутреннее сопротивление цепи из закона Ома для полной цепи,
        \item эквивалентное сопротивление $n$ резисторов, соединённых параллельно, каждый сопротивлением $R$.
    \end{enumerate}
}
\solutionspace{40pt}

\tasknumber{2}%
\task{%
    Получите выражение:
    \begin{enumerate}
        \item силы тока через выделяемую мощность и сопротивление резистора,
        \item силы тока через выделенную теплоту и сопротивление резистора,
        \item напряжение на резисторе через выделяемую мощность и сопротивление резистора,
        \item напряжение на резисторе через выделенную в нём теплоту и сопротивление резистора.
    \end{enumerate}
}
\solutionspace{80pt}

\tasknumber{3}%
\task{%
    Определите ток, протекающий через резистор $R = 12\,\text{Ом}$ и разность потенциалов на нём (см.
    рис.
    на доске),
    если $r_1 = 1\,\text{Ом}$, $r_2 = 2\,\text{Ом}$, $\ele_1 = 60\,\text{В}$, $\ele_2 = 60\,\text{В}$.
}
\answer{%
    Обозначим на рисунке все токи: направление произвольно, но его надо зафиксировать.
    Всего на рисунке 3 контура и 2 узла.
    Поэтому можно записать $3 - 1 = 2$ уравнения законов Кирхгофа для замкнутого контура и $2 - 1 = 1$ — для узлов
    (остальные уравнения тоже можно записать, но они не дадут полезной информации, а будут лишь следствиями уже записанных).

    Отметим на рисунке 2 контура (и не забуем указать направление) и 1 узел (точка «1»ы, выделена жирным).
    Выбор контуров и узлов не критичен: получившаяся система может быть чуть проще или сложнее, но не слишком.

    \begin{tikzpicture}[circuit ee IEC, thick]
        \draw  (0, 0) to [current direction={near end, info=$\eli_1$}] (0, 3)
                to [battery={rotate=-180,info={$\ele_1, r_1$}}]
                (3, 3)
                to [battery={info'={$\ele_2, r_2$}}]
                (6, 3) to [current direction'={near start, info=$\eli_2$}] (6, 0) -- (0, 0)
                (3, 0) to [current direction={near start, info=$\eli$}, resistor={near end, info=$R$}] (3, 3);
        \draw [-{Latex},color=red] (1.2, 1.7) arc [start angle = 135, end angle = -160, radius = 0.6];
        \draw [-{Latex},color=blue] (4.2, 1.7) arc [start angle = 135, end angle = -160, radius = 0.6];
        \node [contact,color=green!71!black] (bottomc) at (3, 0) {};
        \node [below] (bottom) at (3, 0) {$2$};
        \node [above] (top) at (3, 3) {$1$};
    \end{tikzpicture}

    \begin{align*}
        &\begin{cases}
            {\color{red} \ele_1 = \eli_1 r_1 - \eli R}, \\
            {\color{blue} -\ele_2 = -\eli_2 r_2 + \eli R}, \\
            {\color{green!71!black} - \eli - \eli_1 - \eli_2 = 0 };
        \end{cases}
        \qquad \implies \qquad
        \begin{cases}
            \eli_1 = \frac{\ele_1 + \eli R}{r_1}, \\
            \eli_2 = \frac{\ele_2 + \eli R}{r_2}, \\
            \eli + \eli_1 + \eli_2 = 0;
        \end{cases} \implies \\
        &\implies
         \eli + \frac{\ele_1 + \eli R}{r_1:L} + \frac{\ele_2 + \eli R}{r_2:L} = 0, \\
        &\eli\cbr{ 1 + \frac R{r_1:L} + \frac R{r_2:L}} + \frac{\ele_1 }{r_1:L} + \frac{\ele_2 }{r_2:L} = 0, \\
        &\eli
            = - \frac{\frac{\ele_1 }{r_1:L} + \frac{\ele_2 }{r_2:L}}{ 1 + \frac R{r_1:L} + \frac R{r_2:L}}
            = - \frac{\frac{60\,\text{В}}{1\,\text{Ом}} + \frac{60\,\text{В}}{2\,\text{Ом}}}{ 1 + \frac{12\,\text{Ом}}{1\,\text{Ом}} + \frac{12\,\text{Ом}}{2\,\text{Ом}}}
            = - \frac{90}{19}\units{А}
            \approx -4{,}70\,\text{А}, \\
        &U  = \varphi_2 - \varphi_1 = \eli R
            = - \frac{\frac{\ele_1 }{r_1:L} + \frac{\ele_2 }{r_2:L}}{ 1 + \frac R{r_1:L} + \frac R{r_2:L}} R
            \approx -56{,}80\,\text{В}.
    \end{align*}
    Оба ответа отрицательны, потому что мы изначально «не угадали» с направлением тока.
    Расчёт же показал,
    что ток через резистор $R$ течёт в противоположную сторону: вниз на рисунке, а потенциал точки 1 больше потенциала точки 2,
    а электрический ток ожидаемо течёт из точки с большим потенциалов в точку с меньшим.

    Кстати, если продолжить расчёт и вычислить значения ещё двух токов (формулы для $\eli_1$ и $\eli_2$, куда подставлять, выписаны выше),
    то по их знакам можно будет понять: угадали ли мы с их направлением или нет.
}

\variantsplitter

\addpersonalvariant{София Журавлёва}

\tasknumber{1}%
\task{%
    Получите выражение:
    \begin{enumerate}
        \item площади поперечного сечения проводника через его сопротивление,
        \item удельное сопротивление из закона Ома,
        \item внутреннее сопротивление цепи из закона Ома для полной цепи,
        \item эквивалентное сопротивление $n$ резисторов, соединённых параллельно, каждый сопротивлением $R$.
    \end{enumerate}
}
\solutionspace{40pt}

\tasknumber{2}%
\task{%
    Получите выражение:
    \begin{enumerate}
        \item силы тока через выделяемую мощность и напряжение на резисторе,
        \item силы тока через выделенную теплоту и разность потенциалов на резисторе,
        \item напряжение на резисторе через выделяемую мощность и сопротивление резистора,
        \item напряжение на резисторе через выделенную в нём теплоту и сопротивление резистора.
    \end{enumerate}
}
\solutionspace{80pt}

\tasknumber{3}%
\task{%
    Определите ток, протекающий через резистор $R = 18\,\text{Ом}$ и разность потенциалов на нём (см.
    рис.
    на доске),
    если $r_1 = 2\,\text{Ом}$, $r_2 = 2\,\text{Ом}$, $\ele_1 = 30\,\text{В}$, $\ele_2 = 60\,\text{В}$.
}
\answer{%
    Обозначим на рисунке все токи: направление произвольно, но его надо зафиксировать.
    Всего на рисунке 3 контура и 2 узла.
    Поэтому можно записать $3 - 1 = 2$ уравнения законов Кирхгофа для замкнутого контура и $2 - 1 = 1$ — для узлов
    (остальные уравнения тоже можно записать, но они не дадут полезной информации, а будут лишь следствиями уже записанных).

    Отметим на рисунке 2 контура (и не забуем указать направление) и 1 узел (точка «1»ы, выделена жирным).
    Выбор контуров и узлов не критичен: получившаяся система может быть чуть проще или сложнее, но не слишком.

    \begin{tikzpicture}[circuit ee IEC, thick]
        \draw  (0, 0) to [current direction={near end, info=$\eli_1$}] (0, 3)
                to [battery={rotate=-180,info={$\ele_1, r_1$}}]
                (3, 3)
                to [battery={info'={$\ele_2, r_2$}}]
                (6, 3) to [current direction'={near start, info=$\eli_2$}] (6, 0) -- (0, 0)
                (3, 0) to [current direction={near start, info=$\eli$}, resistor={near end, info=$R$}] (3, 3);
        \draw [-{Latex},color=red] (1.2, 1.7) arc [start angle = 135, end angle = -160, radius = 0.6];
        \draw [-{Latex},color=blue] (4.2, 1.7) arc [start angle = 135, end angle = -160, radius = 0.6];
        \node [contact,color=green!71!black] (bottomc) at (3, 0) {};
        \node [below] (bottom) at (3, 0) {$2$};
        \node [above] (top) at (3, 3) {$1$};
    \end{tikzpicture}

    \begin{align*}
        &\begin{cases}
            {\color{red} \ele_1 = \eli_1 r_1 - \eli R}, \\
            {\color{blue} -\ele_2 = -\eli_2 r_2 + \eli R}, \\
            {\color{green!71!black} - \eli - \eli_1 - \eli_2 = 0 };
        \end{cases}
        \qquad \implies \qquad
        \begin{cases}
            \eli_1 = \frac{\ele_1 + \eli R}{r_1}, \\
            \eli_2 = \frac{\ele_2 + \eli R}{r_2}, \\
            \eli + \eli_1 + \eli_2 = 0;
        \end{cases} \implies \\
        &\implies
         \eli + \frac{\ele_1 + \eli R}{r_1:L} + \frac{\ele_2 + \eli R}{r_2:L} = 0, \\
        &\eli\cbr{ 1 + \frac R{r_1:L} + \frac R{r_2:L}} + \frac{\ele_1 }{r_1:L} + \frac{\ele_2 }{r_2:L} = 0, \\
        &\eli
            = - \frac{\frac{\ele_1 }{r_1:L} + \frac{\ele_2 }{r_2:L}}{ 1 + \frac R{r_1:L} + \frac R{r_2:L}}
            = - \frac{\frac{30\,\text{В}}{2\,\text{Ом}} + \frac{60\,\text{В}}{2\,\text{Ом}}}{ 1 + \frac{18\,\text{Ом}}{2\,\text{Ом}} + \frac{18\,\text{Ом}}{2\,\text{Ом}}}
            = - \frac{45}{19}\units{А}
            \approx -2{,}40\,\text{А}, \\
        &U  = \varphi_2 - \varphi_1 = \eli R
            = - \frac{\frac{\ele_1 }{r_1:L} + \frac{\ele_2 }{r_2:L}}{ 1 + \frac R{r_1:L} + \frac R{r_2:L}} R
            \approx -42{,}60\,\text{В}.
    \end{align*}
    Оба ответа отрицательны, потому что мы изначально «не угадали» с направлением тока.
    Расчёт же показал,
    что ток через резистор $R$ течёт в противоположную сторону: вниз на рисунке, а потенциал точки 1 больше потенциала точки 2,
    а электрический ток ожидаемо течёт из точки с большим потенциалов в точку с меньшим.

    Кстати, если продолжить расчёт и вычислить значения ещё двух токов (формулы для $\eli_1$ и $\eli_2$, куда подставлять, выписаны выше),
    то по их знакам можно будет понять: угадали ли мы с их направлением или нет.
}

\variantsplitter

\addpersonalvariant{Константин Козлов}

\tasknumber{1}%
\task{%
    Получите выражение:
    \begin{enumerate}
        \item длины проводника через его сопротивление,
        \item удельное сопротивление из закона Ома,
        \item внешнее сопротивление цепи из закона Ома для полной цепи,
        \item эквивалентное сопротивление $n$ резисторов, соединённых последовательно, каждый сопротивлением $R$.
    \end{enumerate}
}
\solutionspace{40pt}

\tasknumber{2}%
\task{%
    Получите выражение:
    \begin{enumerate}
        \item силы тока через выделяемую мощность и сопротивление резистора,
        \item силы тока через выделенную теплоту и напряжение на резисторе,
        \item напряжение на резисторе через выделяемую мощность и силу тока через него,
        \item напряжение на резисторе через выделенную в нём теплоту и силу тока через него.
    \end{enumerate}
}
\solutionspace{80pt}

\tasknumber{3}%
\task{%
    Определите ток, протекающий через резистор $R = 10\,\text{Ом}$ и разность потенциалов на нём (см.
    рис.
    на доске),
    если $r_1 = 3\,\text{Ом}$, $r_2 = 2\,\text{Ом}$, $\ele_1 = 30\,\text{В}$, $\ele_2 = 30\,\text{В}$.
}
\answer{%
    Обозначим на рисунке все токи: направление произвольно, но его надо зафиксировать.
    Всего на рисунке 3 контура и 2 узла.
    Поэтому можно записать $3 - 1 = 2$ уравнения законов Кирхгофа для замкнутого контура и $2 - 1 = 1$ — для узлов
    (остальные уравнения тоже можно записать, но они не дадут полезной информации, а будут лишь следствиями уже записанных).

    Отметим на рисунке 2 контура (и не забуем указать направление) и 1 узел (точка «1»ы, выделена жирным).
    Выбор контуров и узлов не критичен: получившаяся система может быть чуть проще или сложнее, но не слишком.

    \begin{tikzpicture}[circuit ee IEC, thick]
        \draw  (0, 0) to [current direction={near end, info=$\eli_1$}] (0, 3)
                to [battery={rotate=-180,info={$\ele_1, r_1$}}]
                (3, 3)
                to [battery={info'={$\ele_2, r_2$}}]
                (6, 3) to [current direction'={near start, info=$\eli_2$}] (6, 0) -- (0, 0)
                (3, 0) to [current direction={near start, info=$\eli$}, resistor={near end, info=$R$}] (3, 3);
        \draw [-{Latex},color=red] (1.2, 1.7) arc [start angle = 135, end angle = -160, radius = 0.6];
        \draw [-{Latex},color=blue] (4.2, 1.7) arc [start angle = 135, end angle = -160, radius = 0.6];
        \node [contact,color=green!71!black] (bottomc) at (3, 0) {};
        \node [below] (bottom) at (3, 0) {$2$};
        \node [above] (top) at (3, 3) {$1$};
    \end{tikzpicture}

    \begin{align*}
        &\begin{cases}
            {\color{red} \ele_1 = \eli_1 r_1 - \eli R}, \\
            {\color{blue} -\ele_2 = -\eli_2 r_2 + \eli R}, \\
            {\color{green!71!black} - \eli - \eli_1 - \eli_2 = 0 };
        \end{cases}
        \qquad \implies \qquad
        \begin{cases}
            \eli_1 = \frac{\ele_1 + \eli R}{r_1}, \\
            \eli_2 = \frac{\ele_2 + \eli R}{r_2}, \\
            \eli + \eli_1 + \eli_2 = 0;
        \end{cases} \implies \\
        &\implies
         \eli + \frac{\ele_1 + \eli R}{r_1:L} + \frac{\ele_2 + \eli R}{r_2:L} = 0, \\
        &\eli\cbr{ 1 + \frac R{r_1:L} + \frac R{r_2:L}} + \frac{\ele_1 }{r_1:L} + \frac{\ele_2 }{r_2:L} = 0, \\
        &\eli
            = - \frac{\frac{\ele_1 }{r_1:L} + \frac{\ele_2 }{r_2:L}}{ 1 + \frac R{r_1:L} + \frac R{r_2:L}}
            = - \frac{\frac{30\,\text{В}}{3\,\text{Ом}} + \frac{30\,\text{В}}{2\,\text{Ом}}}{ 1 + \frac{10\,\text{Ом}}{3\,\text{Ом}} + \frac{10\,\text{Ом}}{2\,\text{Ом}}}
            = - \frac{75}{28}\units{А}
            \approx -2{,}70\,\text{А}, \\
        &U  = \varphi_2 - \varphi_1 = \eli R
            = - \frac{\frac{\ele_1 }{r_1:L} + \frac{\ele_2 }{r_2:L}}{ 1 + \frac R{r_1:L} + \frac R{r_2:L}} R
            \approx -26{,}80\,\text{В}.
    \end{align*}
    Оба ответа отрицательны, потому что мы изначально «не угадали» с направлением тока.
    Расчёт же показал,
    что ток через резистор $R$ течёт в противоположную сторону: вниз на рисунке, а потенциал точки 1 больше потенциала точки 2,
    а электрический ток ожидаемо течёт из точки с большим потенциалов в точку с меньшим.

    Кстати, если продолжить расчёт и вычислить значения ещё двух токов (формулы для $\eli_1$ и $\eli_2$, куда подставлять, выписаны выше),
    то по их знакам можно будет понять: угадали ли мы с их направлением или нет.
}

\variantsplitter

\addpersonalvariant{Наталья Кравченко}

\tasknumber{1}%
\task{%
    Получите выражение:
    \begin{enumerate}
        \item длины проводника через его сопротивление,
        \item удельное сопротивление из закона Ома,
        \item внутреннее сопротивление цепи из закона Ома для полной цепи,
        \item эквивалентное сопротивление $n$ резисторов, соединённых последовательно, каждый сопротивлением $R$.
    \end{enumerate}
}
\solutionspace{40pt}

\tasknumber{2}%
\task{%
    Получите выражение:
    \begin{enumerate}
        \item силы тока через выделяемую мощность и напряжение на резисторе,
        \item силы тока через выделенную теплоту и разность потенциалов на резисторе,
        \item напряжение на резисторе через выделяемую мощность и силу тока через него,
        \item напряжение на резисторе через выделенную в нём теплоту и силу тока через него.
    \end{enumerate}
}
\solutionspace{80pt}

\tasknumber{3}%
\task{%
    Определите ток, протекающий через резистор $R = 10\,\text{Ом}$ и разность потенциалов на нём (см.
    рис.
    на доске),
    если $r_1 = 1\,\text{Ом}$, $r_2 = 2\,\text{Ом}$, $\ele_1 = 30\,\text{В}$, $\ele_2 = 30\,\text{В}$.
}
\answer{%
    Обозначим на рисунке все токи: направление произвольно, но его надо зафиксировать.
    Всего на рисунке 3 контура и 2 узла.
    Поэтому можно записать $3 - 1 = 2$ уравнения законов Кирхгофа для замкнутого контура и $2 - 1 = 1$ — для узлов
    (остальные уравнения тоже можно записать, но они не дадут полезной информации, а будут лишь следствиями уже записанных).

    Отметим на рисунке 2 контура (и не забуем указать направление) и 1 узел (точка «1»ы, выделена жирным).
    Выбор контуров и узлов не критичен: получившаяся система может быть чуть проще или сложнее, но не слишком.

    \begin{tikzpicture}[circuit ee IEC, thick]
        \draw  (0, 0) to [current direction={near end, info=$\eli_1$}] (0, 3)
                to [battery={rotate=-180,info={$\ele_1, r_1$}}]
                (3, 3)
                to [battery={info'={$\ele_2, r_2$}}]
                (6, 3) to [current direction'={near start, info=$\eli_2$}] (6, 0) -- (0, 0)
                (3, 0) to [current direction={near start, info=$\eli$}, resistor={near end, info=$R$}] (3, 3);
        \draw [-{Latex},color=red] (1.2, 1.7) arc [start angle = 135, end angle = -160, radius = 0.6];
        \draw [-{Latex},color=blue] (4.2, 1.7) arc [start angle = 135, end angle = -160, radius = 0.6];
        \node [contact,color=green!71!black] (bottomc) at (3, 0) {};
        \node [below] (bottom) at (3, 0) {$2$};
        \node [above] (top) at (3, 3) {$1$};
    \end{tikzpicture}

    \begin{align*}
        &\begin{cases}
            {\color{red} \ele_1 = \eli_1 r_1 - \eli R}, \\
            {\color{blue} -\ele_2 = -\eli_2 r_2 + \eli R}, \\
            {\color{green!71!black} - \eli - \eli_1 - \eli_2 = 0 };
        \end{cases}
        \qquad \implies \qquad
        \begin{cases}
            \eli_1 = \frac{\ele_1 + \eli R}{r_1}, \\
            \eli_2 = \frac{\ele_2 + \eli R}{r_2}, \\
            \eli + \eli_1 + \eli_2 = 0;
        \end{cases} \implies \\
        &\implies
         \eli + \frac{\ele_1 + \eli R}{r_1:L} + \frac{\ele_2 + \eli R}{r_2:L} = 0, \\
        &\eli\cbr{ 1 + \frac R{r_1:L} + \frac R{r_2:L}} + \frac{\ele_1 }{r_1:L} + \frac{\ele_2 }{r_2:L} = 0, \\
        &\eli
            = - \frac{\frac{\ele_1 }{r_1:L} + \frac{\ele_2 }{r_2:L}}{ 1 + \frac R{r_1:L} + \frac R{r_2:L}}
            = - \frac{\frac{30\,\text{В}}{1\,\text{Ом}} + \frac{30\,\text{В}}{2\,\text{Ом}}}{ 1 + \frac{10\,\text{Ом}}{1\,\text{Ом}} + \frac{10\,\text{Ом}}{2\,\text{Ом}}}
            = - \frac{45}{16}\units{А}
            \approx -2{,}80\,\text{А}, \\
        &U  = \varphi_2 - \varphi_1 = \eli R
            = - \frac{\frac{\ele_1 }{r_1:L} + \frac{\ele_2 }{r_2:L}}{ 1 + \frac R{r_1:L} + \frac R{r_2:L}} R
            \approx -28{,}10\,\text{В}.
    \end{align*}
    Оба ответа отрицательны, потому что мы изначально «не угадали» с направлением тока.
    Расчёт же показал,
    что ток через резистор $R$ течёт в противоположную сторону: вниз на рисунке, а потенциал точки 1 больше потенциала точки 2,
    а электрический ток ожидаемо течёт из точки с большим потенциалов в точку с меньшим.

    Кстати, если продолжить расчёт и вычислить значения ещё двух токов (формулы для $\eli_1$ и $\eli_2$, куда подставлять, выписаны выше),
    то по их знакам можно будет понять: угадали ли мы с их направлением или нет.
}

\variantsplitter

\addpersonalvariant{Матвей Кузьмин}

\tasknumber{1}%
\task{%
    Получите выражение:
    \begin{enumerate}
        \item длины проводника через его сопротивление,
        \item удельное сопротивление из закона Ома,
        \item внутреннее сопротивление цепи из закона Ома для полной цепи,
        \item эквивалентное сопротивление $n$ резисторов, соединённых параллельно, каждый сопротивлением $R$.
    \end{enumerate}
}
\solutionspace{40pt}

\tasknumber{2}%
\task{%
    Получите выражение:
    \begin{enumerate}
        \item силы тока через выделяемую мощность и разность потенциалов на резисторе,
        \item силы тока через выделенную теплоту и напряжение на резисторе,
        \item напряжение на резисторе через выделяемую мощность и силу тока через него,
        \item напряжение на резисторе через выделенную в нём теплоту и сопротивление резистора.
    \end{enumerate}
}
\solutionspace{80pt}

\tasknumber{3}%
\task{%
    Определите ток, протекающий через резистор $R = 20\,\text{Ом}$ и разность потенциалов на нём (см.
    рис.
    на доске),
    если $r_1 = 3\,\text{Ом}$, $r_2 = 1\,\text{Ом}$, $\ele_1 = 30\,\text{В}$, $\ele_2 = 40\,\text{В}$.
}
\answer{%
    Обозначим на рисунке все токи: направление произвольно, но его надо зафиксировать.
    Всего на рисунке 3 контура и 2 узла.
    Поэтому можно записать $3 - 1 = 2$ уравнения законов Кирхгофа для замкнутого контура и $2 - 1 = 1$ — для узлов
    (остальные уравнения тоже можно записать, но они не дадут полезной информации, а будут лишь следствиями уже записанных).

    Отметим на рисунке 2 контура (и не забуем указать направление) и 1 узел (точка «1»ы, выделена жирным).
    Выбор контуров и узлов не критичен: получившаяся система может быть чуть проще или сложнее, но не слишком.

    \begin{tikzpicture}[circuit ee IEC, thick]
        \draw  (0, 0) to [current direction={near end, info=$\eli_1$}] (0, 3)
                to [battery={rotate=-180,info={$\ele_1, r_1$}}]
                (3, 3)
                to [battery={info'={$\ele_2, r_2$}}]
                (6, 3) to [current direction'={near start, info=$\eli_2$}] (6, 0) -- (0, 0)
                (3, 0) to [current direction={near start, info=$\eli$}, resistor={near end, info=$R$}] (3, 3);
        \draw [-{Latex},color=red] (1.2, 1.7) arc [start angle = 135, end angle = -160, radius = 0.6];
        \draw [-{Latex},color=blue] (4.2, 1.7) arc [start angle = 135, end angle = -160, radius = 0.6];
        \node [contact,color=green!71!black] (bottomc) at (3, 0) {};
        \node [below] (bottom) at (3, 0) {$2$};
        \node [above] (top) at (3, 3) {$1$};
    \end{tikzpicture}

    \begin{align*}
        &\begin{cases}
            {\color{red} \ele_1 = \eli_1 r_1 - \eli R}, \\
            {\color{blue} -\ele_2 = -\eli_2 r_2 + \eli R}, \\
            {\color{green!71!black} - \eli - \eli_1 - \eli_2 = 0 };
        \end{cases}
        \qquad \implies \qquad
        \begin{cases}
            \eli_1 = \frac{\ele_1 + \eli R}{r_1}, \\
            \eli_2 = \frac{\ele_2 + \eli R}{r_2}, \\
            \eli + \eli_1 + \eli_2 = 0;
        \end{cases} \implies \\
        &\implies
         \eli + \frac{\ele_1 + \eli R}{r_1:L} + \frac{\ele_2 + \eli R}{r_2:L} = 0, \\
        &\eli\cbr{ 1 + \frac R{r_1:L} + \frac R{r_2:L}} + \frac{\ele_1 }{r_1:L} + \frac{\ele_2 }{r_2:L} = 0, \\
        &\eli
            = - \frac{\frac{\ele_1 }{r_1:L} + \frac{\ele_2 }{r_2:L}}{ 1 + \frac R{r_1:L} + \frac R{r_2:L}}
            = - \frac{\frac{30\,\text{В}}{3\,\text{Ом}} + \frac{40\,\text{В}}{1\,\text{Ом}}}{ 1 + \frac{20\,\text{Ом}}{3\,\text{Ом}} + \frac{20\,\text{Ом}}{1\,\text{Ом}}}
            = - \frac{150}{83}\units{А}
            \approx -1{,}800\,\text{А}, \\
        &U  = \varphi_2 - \varphi_1 = \eli R
            = - \frac{\frac{\ele_1 }{r_1:L} + \frac{\ele_2 }{r_2:L}}{ 1 + \frac R{r_1:L} + \frac R{r_2:L}} R
            \approx -36{,}10\,\text{В}.
    \end{align*}
    Оба ответа отрицательны, потому что мы изначально «не угадали» с направлением тока.
    Расчёт же показал,
    что ток через резистор $R$ течёт в противоположную сторону: вниз на рисунке, а потенциал точки 1 больше потенциала точки 2,
    а электрический ток ожидаемо течёт из точки с большим потенциалов в точку с меньшим.

    Кстати, если продолжить расчёт и вычислить значения ещё двух токов (формулы для $\eli_1$ и $\eli_2$, куда подставлять, выписаны выше),
    то по их знакам можно будет понять: угадали ли мы с их направлением или нет.
}

\variantsplitter

\addpersonalvariant{Сергей Малышев}

\tasknumber{1}%
\task{%
    Получите выражение:
    \begin{enumerate}
        \item длины проводника через его сопротивление,
        \item сопротивление из закона Ома,
        \item внешнее сопротивление цепи из закона Ома для полной цепи,
        \item эквивалентное сопротивление $n$ резисторов, соединённых последовательно, каждый сопротивлением $R$.
    \end{enumerate}
}
\solutionspace{40pt}

\tasknumber{2}%
\task{%
    Получите выражение:
    \begin{enumerate}
        \item силы тока через выделяемую мощность и сопротивление резистора,
        \item силы тока через выделенную теплоту и напряжение на резисторе,
        \item напряжение на резисторе через выделяемую мощность и силу тока через него,
        \item напряжение на резисторе через выделенную в нём теплоту и сопротивление резистора.
    \end{enumerate}
}
\solutionspace{80pt}

\tasknumber{3}%
\task{%
    Определите ток, протекающий через резистор $R = 12\,\text{Ом}$ и разность потенциалов на нём (см.
    рис.
    на доске),
    если $r_1 = 2\,\text{Ом}$, $r_2 = 1\,\text{Ом}$, $\ele_1 = 60\,\text{В}$, $\ele_2 = 60\,\text{В}$.
}
\answer{%
    Обозначим на рисунке все токи: направление произвольно, но его надо зафиксировать.
    Всего на рисунке 3 контура и 2 узла.
    Поэтому можно записать $3 - 1 = 2$ уравнения законов Кирхгофа для замкнутого контура и $2 - 1 = 1$ — для узлов
    (остальные уравнения тоже можно записать, но они не дадут полезной информации, а будут лишь следствиями уже записанных).

    Отметим на рисунке 2 контура (и не забуем указать направление) и 1 узел (точка «1»ы, выделена жирным).
    Выбор контуров и узлов не критичен: получившаяся система может быть чуть проще или сложнее, но не слишком.

    \begin{tikzpicture}[circuit ee IEC, thick]
        \draw  (0, 0) to [current direction={near end, info=$\eli_1$}] (0, 3)
                to [battery={rotate=-180,info={$\ele_1, r_1$}}]
                (3, 3)
                to [battery={info'={$\ele_2, r_2$}}]
                (6, 3) to [current direction'={near start, info=$\eli_2$}] (6, 0) -- (0, 0)
                (3, 0) to [current direction={near start, info=$\eli$}, resistor={near end, info=$R$}] (3, 3);
        \draw [-{Latex},color=red] (1.2, 1.7) arc [start angle = 135, end angle = -160, radius = 0.6];
        \draw [-{Latex},color=blue] (4.2, 1.7) arc [start angle = 135, end angle = -160, radius = 0.6];
        \node [contact,color=green!71!black] (bottomc) at (3, 0) {};
        \node [below] (bottom) at (3, 0) {$2$};
        \node [above] (top) at (3, 3) {$1$};
    \end{tikzpicture}

    \begin{align*}
        &\begin{cases}
            {\color{red} \ele_1 = \eli_1 r_1 - \eli R}, \\
            {\color{blue} -\ele_2 = -\eli_2 r_2 + \eli R}, \\
            {\color{green!71!black} - \eli - \eli_1 - \eli_2 = 0 };
        \end{cases}
        \qquad \implies \qquad
        \begin{cases}
            \eli_1 = \frac{\ele_1 + \eli R}{r_1}, \\
            \eli_2 = \frac{\ele_2 + \eli R}{r_2}, \\
            \eli + \eli_1 + \eli_2 = 0;
        \end{cases} \implies \\
        &\implies
         \eli + \frac{\ele_1 + \eli R}{r_1:L} + \frac{\ele_2 + \eli R}{r_2:L} = 0, \\
        &\eli\cbr{ 1 + \frac R{r_1:L} + \frac R{r_2:L}} + \frac{\ele_1 }{r_1:L} + \frac{\ele_2 }{r_2:L} = 0, \\
        &\eli
            = - \frac{\frac{\ele_1 }{r_1:L} + \frac{\ele_2 }{r_2:L}}{ 1 + \frac R{r_1:L} + \frac R{r_2:L}}
            = - \frac{\frac{60\,\text{В}}{2\,\text{Ом}} + \frac{60\,\text{В}}{1\,\text{Ом}}}{ 1 + \frac{12\,\text{Ом}}{2\,\text{Ом}} + \frac{12\,\text{Ом}}{1\,\text{Ом}}}
            = - \frac{90}{19}\units{А}
            \approx -4{,}70\,\text{А}, \\
        &U  = \varphi_2 - \varphi_1 = \eli R
            = - \frac{\frac{\ele_1 }{r_1:L} + \frac{\ele_2 }{r_2:L}}{ 1 + \frac R{r_1:L} + \frac R{r_2:L}} R
            \approx -56{,}80\,\text{В}.
    \end{align*}
    Оба ответа отрицательны, потому что мы изначально «не угадали» с направлением тока.
    Расчёт же показал,
    что ток через резистор $R$ течёт в противоположную сторону: вниз на рисунке, а потенциал точки 1 больше потенциала точки 2,
    а электрический ток ожидаемо течёт из точки с большим потенциалов в точку с меньшим.

    Кстати, если продолжить расчёт и вычислить значения ещё двух токов (формулы для $\eli_1$ и $\eli_2$, куда подставлять, выписаны выше),
    то по их знакам можно будет понять: угадали ли мы с их направлением или нет.
}

\variantsplitter

\addpersonalvariant{Алина Полканова}

\tasknumber{1}%
\task{%
    Получите выражение:
    \begin{enumerate}
        \item площади поперечного сечения проводника через его сопротивление,
        \item сопротивление из закона Ома,
        \item внутреннее сопротивление цепи из закона Ома для полной цепи,
        \item эквивалентное сопротивление $n$ резисторов, соединённых параллельно, каждый сопротивлением $R$.
    \end{enumerate}
}
\solutionspace{40pt}

\tasknumber{2}%
\task{%
    Получите выражение:
    \begin{enumerate}
        \item силы тока через выделяемую мощность и напряжение на резисторе,
        \item силы тока через выделенную теплоту и напряжение на резисторе,
        \item напряжение на резисторе через выделяемую мощность и силу тока через него,
        \item напряжение на резисторе через выделенную в нём теплоту и сопротивление резистора.
    \end{enumerate}
}
\solutionspace{80pt}

\tasknumber{3}%
\task{%
    Определите ток, протекающий через резистор $R = 10\,\text{Ом}$ и разность потенциалов на нём (см.
    рис.
    на доске),
    если $r_1 = 3\,\text{Ом}$, $r_2 = 1\,\text{Ом}$, $\ele_1 = 20\,\text{В}$, $\ele_2 = 40\,\text{В}$.
}
\answer{%
    Обозначим на рисунке все токи: направление произвольно, но его надо зафиксировать.
    Всего на рисунке 3 контура и 2 узла.
    Поэтому можно записать $3 - 1 = 2$ уравнения законов Кирхгофа для замкнутого контура и $2 - 1 = 1$ — для узлов
    (остальные уравнения тоже можно записать, но они не дадут полезной информации, а будут лишь следствиями уже записанных).

    Отметим на рисунке 2 контура (и не забуем указать направление) и 1 узел (точка «1»ы, выделена жирным).
    Выбор контуров и узлов не критичен: получившаяся система может быть чуть проще или сложнее, но не слишком.

    \begin{tikzpicture}[circuit ee IEC, thick]
        \draw  (0, 0) to [current direction={near end, info=$\eli_1$}] (0, 3)
                to [battery={rotate=-180,info={$\ele_1, r_1$}}]
                (3, 3)
                to [battery={info'={$\ele_2, r_2$}}]
                (6, 3) to [current direction'={near start, info=$\eli_2$}] (6, 0) -- (0, 0)
                (3, 0) to [current direction={near start, info=$\eli$}, resistor={near end, info=$R$}] (3, 3);
        \draw [-{Latex},color=red] (1.2, 1.7) arc [start angle = 135, end angle = -160, radius = 0.6];
        \draw [-{Latex},color=blue] (4.2, 1.7) arc [start angle = 135, end angle = -160, radius = 0.6];
        \node [contact,color=green!71!black] (bottomc) at (3, 0) {};
        \node [below] (bottom) at (3, 0) {$2$};
        \node [above] (top) at (3, 3) {$1$};
    \end{tikzpicture}

    \begin{align*}
        &\begin{cases}
            {\color{red} \ele_1 = \eli_1 r_1 - \eli R}, \\
            {\color{blue} -\ele_2 = -\eli_2 r_2 + \eli R}, \\
            {\color{green!71!black} - \eli - \eli_1 - \eli_2 = 0 };
        \end{cases}
        \qquad \implies \qquad
        \begin{cases}
            \eli_1 = \frac{\ele_1 + \eli R}{r_1}, \\
            \eli_2 = \frac{\ele_2 + \eli R}{r_2}, \\
            \eli + \eli_1 + \eli_2 = 0;
        \end{cases} \implies \\
        &\implies
         \eli + \frac{\ele_1 + \eli R}{r_1:L} + \frac{\ele_2 + \eli R}{r_2:L} = 0, \\
        &\eli\cbr{ 1 + \frac R{r_1:L} + \frac R{r_2:L}} + \frac{\ele_1 }{r_1:L} + \frac{\ele_2 }{r_2:L} = 0, \\
        &\eli
            = - \frac{\frac{\ele_1 }{r_1:L} + \frac{\ele_2 }{r_2:L}}{ 1 + \frac R{r_1:L} + \frac R{r_2:L}}
            = - \frac{\frac{20\,\text{В}}{3\,\text{Ом}} + \frac{40\,\text{В}}{1\,\text{Ом}}}{ 1 + \frac{10\,\text{Ом}}{3\,\text{Ом}} + \frac{10\,\text{Ом}}{1\,\text{Ом}}}
            = - \frac{140}{43}\units{А}
            \approx -3{,}30\,\text{А}, \\
        &U  = \varphi_2 - \varphi_1 = \eli R
            = - \frac{\frac{\ele_1 }{r_1:L} + \frac{\ele_2 }{r_2:L}}{ 1 + \frac R{r_1:L} + \frac R{r_2:L}} R
            \approx -32{,}60\,\text{В}.
    \end{align*}
    Оба ответа отрицательны, потому что мы изначально «не угадали» с направлением тока.
    Расчёт же показал,
    что ток через резистор $R$ течёт в противоположную сторону: вниз на рисунке, а потенциал точки 1 больше потенциала точки 2,
    а электрический ток ожидаемо течёт из точки с большим потенциалов в точку с меньшим.

    Кстати, если продолжить расчёт и вычислить значения ещё двух токов (формулы для $\eli_1$ и $\eli_2$, куда подставлять, выписаны выше),
    то по их знакам можно будет понять: угадали ли мы с их направлением или нет.
}

\variantsplitter

\addpersonalvariant{Сергей Пономарёв}

\tasknumber{1}%
\task{%
    Получите выражение:
    \begin{enumerate}
        \item площади поперечного сечения проводника через его сопротивление,
        \item сопротивление из закона Ома,
        \item внешнее сопротивление цепи из закона Ома для полной цепи,
        \item эквивалентное сопротивление $n$ резисторов, соединённых параллельно, каждый сопротивлением $R$.
    \end{enumerate}
}
\solutionspace{40pt}

\tasknumber{2}%
\task{%
    Получите выражение:
    \begin{enumerate}
        \item силы тока через выделяемую мощность и разность потенциалов на резисторе,
        \item силы тока через выделенную теплоту и разность потенциалов на резисторе,
        \item напряжение на резисторе через выделяемую мощность и силу тока через него,
        \item напряжение на резисторе через выделенную в нём теплоту и сопротивление резистора.
    \end{enumerate}
}
\solutionspace{80pt}

\tasknumber{3}%
\task{%
    Определите ток, протекающий через резистор $R = 20\,\text{Ом}$ и разность потенциалов на нём (см.
    рис.
    на доске),
    если $r_1 = 1\,\text{Ом}$, $r_2 = 2\,\text{Ом}$, $\ele_1 = 30\,\text{В}$, $\ele_2 = 40\,\text{В}$.
}
\answer{%
    Обозначим на рисунке все токи: направление произвольно, но его надо зафиксировать.
    Всего на рисунке 3 контура и 2 узла.
    Поэтому можно записать $3 - 1 = 2$ уравнения законов Кирхгофа для замкнутого контура и $2 - 1 = 1$ — для узлов
    (остальные уравнения тоже можно записать, но они не дадут полезной информации, а будут лишь следствиями уже записанных).

    Отметим на рисунке 2 контура (и не забуем указать направление) и 1 узел (точка «1»ы, выделена жирным).
    Выбор контуров и узлов не критичен: получившаяся система может быть чуть проще или сложнее, но не слишком.

    \begin{tikzpicture}[circuit ee IEC, thick]
        \draw  (0, 0) to [current direction={near end, info=$\eli_1$}] (0, 3)
                to [battery={rotate=-180,info={$\ele_1, r_1$}}]
                (3, 3)
                to [battery={info'={$\ele_2, r_2$}}]
                (6, 3) to [current direction'={near start, info=$\eli_2$}] (6, 0) -- (0, 0)
                (3, 0) to [current direction={near start, info=$\eli$}, resistor={near end, info=$R$}] (3, 3);
        \draw [-{Latex},color=red] (1.2, 1.7) arc [start angle = 135, end angle = -160, radius = 0.6];
        \draw [-{Latex},color=blue] (4.2, 1.7) arc [start angle = 135, end angle = -160, radius = 0.6];
        \node [contact,color=green!71!black] (bottomc) at (3, 0) {};
        \node [below] (bottom) at (3, 0) {$2$};
        \node [above] (top) at (3, 3) {$1$};
    \end{tikzpicture}

    \begin{align*}
        &\begin{cases}
            {\color{red} \ele_1 = \eli_1 r_1 - \eli R}, \\
            {\color{blue} -\ele_2 = -\eli_2 r_2 + \eli R}, \\
            {\color{green!71!black} - \eli - \eli_1 - \eli_2 = 0 };
        \end{cases}
        \qquad \implies \qquad
        \begin{cases}
            \eli_1 = \frac{\ele_1 + \eli R}{r_1}, \\
            \eli_2 = \frac{\ele_2 + \eli R}{r_2}, \\
            \eli + \eli_1 + \eli_2 = 0;
        \end{cases} \implies \\
        &\implies
         \eli + \frac{\ele_1 + \eli R}{r_1:L} + \frac{\ele_2 + \eli R}{r_2:L} = 0, \\
        &\eli\cbr{ 1 + \frac R{r_1:L} + \frac R{r_2:L}} + \frac{\ele_1 }{r_1:L} + \frac{\ele_2 }{r_2:L} = 0, \\
        &\eli
            = - \frac{\frac{\ele_1 }{r_1:L} + \frac{\ele_2 }{r_2:L}}{ 1 + \frac R{r_1:L} + \frac R{r_2:L}}
            = - \frac{\frac{30\,\text{В}}{1\,\text{Ом}} + \frac{40\,\text{В}}{2\,\text{Ом}}}{ 1 + \frac{20\,\text{Ом}}{1\,\text{Ом}} + \frac{20\,\text{Ом}}{2\,\text{Ом}}}
            = - \frac{50}{31}\units{А}
            \approx -1{,}600\,\text{А}, \\
        &U  = \varphi_2 - \varphi_1 = \eli R
            = - \frac{\frac{\ele_1 }{r_1:L} + \frac{\ele_2 }{r_2:L}}{ 1 + \frac R{r_1:L} + \frac R{r_2:L}} R
            \approx -32{,}30\,\text{В}.
    \end{align*}
    Оба ответа отрицательны, потому что мы изначально «не угадали» с направлением тока.
    Расчёт же показал,
    что ток через резистор $R$ течёт в противоположную сторону: вниз на рисунке, а потенциал точки 1 больше потенциала точки 2,
    а электрический ток ожидаемо течёт из точки с большим потенциалов в точку с меньшим.

    Кстати, если продолжить расчёт и вычислить значения ещё двух токов (формулы для $\eli_1$ и $\eli_2$, куда подставлять, выписаны выше),
    то по их знакам можно будет понять: угадали ли мы с их направлением или нет.
}

\variantsplitter

\addpersonalvariant{Егор Свистушкин}

\tasknumber{1}%
\task{%
    Получите выражение:
    \begin{enumerate}
        \item длины проводника через его сопротивление,
        \item удельное сопротивление из закона Ома,
        \item внешнее сопротивление цепи из закона Ома для полной цепи,
        \item эквивалентное сопротивление $n$ резисторов, соединённых параллельно, каждый сопротивлением $R$.
    \end{enumerate}
}
\solutionspace{40pt}

\tasknumber{2}%
\task{%
    Получите выражение:
    \begin{enumerate}
        \item силы тока через выделяемую мощность и сопротивление резистора,
        \item силы тока через выделенную теплоту и разность потенциалов на резисторе,
        \item напряжение на резисторе через выделяемую мощность и сопротивление резистора,
        \item напряжение на резисторе через выделенную в нём теплоту и сопротивление резистора.
    \end{enumerate}
}
\solutionspace{80pt}

\tasknumber{3}%
\task{%
    Определите ток, протекающий через резистор $R = 18\,\text{Ом}$ и разность потенциалов на нём (см.
    рис.
    на доске),
    если $r_1 = 1\,\text{Ом}$, $r_2 = 2\,\text{Ом}$, $\ele_1 = 60\,\text{В}$, $\ele_2 = 60\,\text{В}$.
}
\answer{%
    Обозначим на рисунке все токи: направление произвольно, но его надо зафиксировать.
    Всего на рисунке 3 контура и 2 узла.
    Поэтому можно записать $3 - 1 = 2$ уравнения законов Кирхгофа для замкнутого контура и $2 - 1 = 1$ — для узлов
    (остальные уравнения тоже можно записать, но они не дадут полезной информации, а будут лишь следствиями уже записанных).

    Отметим на рисунке 2 контура (и не забуем указать направление) и 1 узел (точка «1»ы, выделена жирным).
    Выбор контуров и узлов не критичен: получившаяся система может быть чуть проще или сложнее, но не слишком.

    \begin{tikzpicture}[circuit ee IEC, thick]
        \draw  (0, 0) to [current direction={near end, info=$\eli_1$}] (0, 3)
                to [battery={rotate=-180,info={$\ele_1, r_1$}}]
                (3, 3)
                to [battery={info'={$\ele_2, r_2$}}]
                (6, 3) to [current direction'={near start, info=$\eli_2$}] (6, 0) -- (0, 0)
                (3, 0) to [current direction={near start, info=$\eli$}, resistor={near end, info=$R$}] (3, 3);
        \draw [-{Latex},color=red] (1.2, 1.7) arc [start angle = 135, end angle = -160, radius = 0.6];
        \draw [-{Latex},color=blue] (4.2, 1.7) arc [start angle = 135, end angle = -160, radius = 0.6];
        \node [contact,color=green!71!black] (bottomc) at (3, 0) {};
        \node [below] (bottom) at (3, 0) {$2$};
        \node [above] (top) at (3, 3) {$1$};
    \end{tikzpicture}

    \begin{align*}
        &\begin{cases}
            {\color{red} \ele_1 = \eli_1 r_1 - \eli R}, \\
            {\color{blue} -\ele_2 = -\eli_2 r_2 + \eli R}, \\
            {\color{green!71!black} - \eli - \eli_1 - \eli_2 = 0 };
        \end{cases}
        \qquad \implies \qquad
        \begin{cases}
            \eli_1 = \frac{\ele_1 + \eli R}{r_1}, \\
            \eli_2 = \frac{\ele_2 + \eli R}{r_2}, \\
            \eli + \eli_1 + \eli_2 = 0;
        \end{cases} \implies \\
        &\implies
         \eli + \frac{\ele_1 + \eli R}{r_1:L} + \frac{\ele_2 + \eli R}{r_2:L} = 0, \\
        &\eli\cbr{ 1 + \frac R{r_1:L} + \frac R{r_2:L}} + \frac{\ele_1 }{r_1:L} + \frac{\ele_2 }{r_2:L} = 0, \\
        &\eli
            = - \frac{\frac{\ele_1 }{r_1:L} + \frac{\ele_2 }{r_2:L}}{ 1 + \frac R{r_1:L} + \frac R{r_2:L}}
            = - \frac{\frac{60\,\text{В}}{1\,\text{Ом}} + \frac{60\,\text{В}}{2\,\text{Ом}}}{ 1 + \frac{18\,\text{Ом}}{1\,\text{Ом}} + \frac{18\,\text{Ом}}{2\,\text{Ом}}}
            = - \frac{45}{14}\units{А}
            \approx -3{,}20\,\text{А}, \\
        &U  = \varphi_2 - \varphi_1 = \eli R
            = - \frac{\frac{\ele_1 }{r_1:L} + \frac{\ele_2 }{r_2:L}}{ 1 + \frac R{r_1:L} + \frac R{r_2:L}} R
            \approx -57{,}90\,\text{В}.
    \end{align*}
    Оба ответа отрицательны, потому что мы изначально «не угадали» с направлением тока.
    Расчёт же показал,
    что ток через резистор $R$ течёт в противоположную сторону: вниз на рисунке, а потенциал точки 1 больше потенциала точки 2,
    а электрический ток ожидаемо течёт из точки с большим потенциалов в точку с меньшим.

    Кстати, если продолжить расчёт и вычислить значения ещё двух токов (формулы для $\eli_1$ и $\eli_2$, куда подставлять, выписаны выше),
    то по их знакам можно будет понять: угадали ли мы с их направлением или нет.
}

\variantsplitter

\addpersonalvariant{Дмитрий Соколов}

\tasknumber{1}%
\task{%
    Получите выражение:
    \begin{enumerate}
        \item площади поперечного сечения проводника через его сопротивление,
        \item сопротивление из закона Ома,
        \item внешнее сопротивление цепи из закона Ома для полной цепи,
        \item эквивалентное сопротивление $n$ резисторов, соединённых последовательно, каждый сопротивлением $R$.
    \end{enumerate}
}
\solutionspace{40pt}

\tasknumber{2}%
\task{%
    Получите выражение:
    \begin{enumerate}
        \item силы тока через выделяемую мощность и сопротивление резистора,
        \item силы тока через выделенную теплоту и напряжение на резисторе,
        \item напряжение на резисторе через выделяемую мощность и сопротивление резистора,
        \item напряжение на резисторе через выделенную в нём теплоту и силу тока через него.
    \end{enumerate}
}
\solutionspace{80pt}

\tasknumber{3}%
\task{%
    Определите ток, протекающий через резистор $R = 10\,\text{Ом}$ и разность потенциалов на нём (см.
    рис.
    на доске),
    если $r_1 = 1\,\text{Ом}$, $r_2 = 3\,\text{Ом}$, $\ele_1 = 60\,\text{В}$, $\ele_2 = 30\,\text{В}$.
}
\answer{%
    Обозначим на рисунке все токи: направление произвольно, но его надо зафиксировать.
    Всего на рисунке 3 контура и 2 узла.
    Поэтому можно записать $3 - 1 = 2$ уравнения законов Кирхгофа для замкнутого контура и $2 - 1 = 1$ — для узлов
    (остальные уравнения тоже можно записать, но они не дадут полезной информации, а будут лишь следствиями уже записанных).

    Отметим на рисунке 2 контура (и не забуем указать направление) и 1 узел (точка «1»ы, выделена жирным).
    Выбор контуров и узлов не критичен: получившаяся система может быть чуть проще или сложнее, но не слишком.

    \begin{tikzpicture}[circuit ee IEC, thick]
        \draw  (0, 0) to [current direction={near end, info=$\eli_1$}] (0, 3)
                to [battery={rotate=-180,info={$\ele_1, r_1$}}]
                (3, 3)
                to [battery={info'={$\ele_2, r_2$}}]
                (6, 3) to [current direction'={near start, info=$\eli_2$}] (6, 0) -- (0, 0)
                (3, 0) to [current direction={near start, info=$\eli$}, resistor={near end, info=$R$}] (3, 3);
        \draw [-{Latex},color=red] (1.2, 1.7) arc [start angle = 135, end angle = -160, radius = 0.6];
        \draw [-{Latex},color=blue] (4.2, 1.7) arc [start angle = 135, end angle = -160, radius = 0.6];
        \node [contact,color=green!71!black] (bottomc) at (3, 0) {};
        \node [below] (bottom) at (3, 0) {$2$};
        \node [above] (top) at (3, 3) {$1$};
    \end{tikzpicture}

    \begin{align*}
        &\begin{cases}
            {\color{red} \ele_1 = \eli_1 r_1 - \eli R}, \\
            {\color{blue} -\ele_2 = -\eli_2 r_2 + \eli R}, \\
            {\color{green!71!black} - \eli - \eli_1 - \eli_2 = 0 };
        \end{cases}
        \qquad \implies \qquad
        \begin{cases}
            \eli_1 = \frac{\ele_1 + \eli R}{r_1}, \\
            \eli_2 = \frac{\ele_2 + \eli R}{r_2}, \\
            \eli + \eli_1 + \eli_2 = 0;
        \end{cases} \implies \\
        &\implies
         \eli + \frac{\ele_1 + \eli R}{r_1:L} + \frac{\ele_2 + \eli R}{r_2:L} = 0, \\
        &\eli\cbr{ 1 + \frac R{r_1:L} + \frac R{r_2:L}} + \frac{\ele_1 }{r_1:L} + \frac{\ele_2 }{r_2:L} = 0, \\
        &\eli
            = - \frac{\frac{\ele_1 }{r_1:L} + \frac{\ele_2 }{r_2:L}}{ 1 + \frac R{r_1:L} + \frac R{r_2:L}}
            = - \frac{\frac{60\,\text{В}}{1\,\text{Ом}} + \frac{30\,\text{В}}{3\,\text{Ом}}}{ 1 + \frac{10\,\text{Ом}}{1\,\text{Ом}} + \frac{10\,\text{Ом}}{3\,\text{Ом}}}
            = - \frac{210}{43}\units{А}
            \approx -4{,}90\,\text{А}, \\
        &U  = \varphi_2 - \varphi_1 = \eli R
            = - \frac{\frac{\ele_1 }{r_1:L} + \frac{\ele_2 }{r_2:L}}{ 1 + \frac R{r_1:L} + \frac R{r_2:L}} R
            \approx -48{,}80\,\text{В}.
    \end{align*}
    Оба ответа отрицательны, потому что мы изначально «не угадали» с направлением тока.
    Расчёт же показал,
    что ток через резистор $R$ течёт в противоположную сторону: вниз на рисунке, а потенциал точки 1 больше потенциала точки 2,
    а электрический ток ожидаемо течёт из точки с большим потенциалов в точку с меньшим.

    Кстати, если продолжить расчёт и вычислить значения ещё двух токов (формулы для $\eli_1$ и $\eli_2$, куда подставлять, выписаны выше),
    то по их знакам можно будет понять: угадали ли мы с их направлением или нет.
}

\variantsplitter

\addpersonalvariant{Арсений Трофимов}

\tasknumber{1}%
\task{%
    Получите выражение:
    \begin{enumerate}
        \item площади поперечного сечения проводника через его сопротивление,
        \item удельное сопротивление из закона Ома,
        \item внешнее сопротивление цепи из закона Ома для полной цепи,
        \item эквивалентное сопротивление $n$ резисторов, соединённых параллельно, каждый сопротивлением $R$.
    \end{enumerate}
}
\solutionspace{40pt}

\tasknumber{2}%
\task{%
    Получите выражение:
    \begin{enumerate}
        \item силы тока через выделяемую мощность и разность потенциалов на резисторе,
        \item силы тока через выделенную теплоту и разность потенциалов на резисторе,
        \item напряжение на резисторе через выделяемую мощность и силу тока через него,
        \item напряжение на резисторе через выделенную в нём теплоту и силу тока через него.
    \end{enumerate}
}
\solutionspace{80pt}

\tasknumber{3}%
\task{%
    Определите ток, протекающий через резистор $R = 10\,\text{Ом}$ и разность потенциалов на нём (см.
    рис.
    на доске),
    если $r_1 = 1\,\text{Ом}$, $r_2 = 2\,\text{Ом}$, $\ele_1 = 40\,\text{В}$, $\ele_2 = 20\,\text{В}$.
}
\answer{%
    Обозначим на рисунке все токи: направление произвольно, но его надо зафиксировать.
    Всего на рисунке 3 контура и 2 узла.
    Поэтому можно записать $3 - 1 = 2$ уравнения законов Кирхгофа для замкнутого контура и $2 - 1 = 1$ — для узлов
    (остальные уравнения тоже можно записать, но они не дадут полезной информации, а будут лишь следствиями уже записанных).

    Отметим на рисунке 2 контура (и не забуем указать направление) и 1 узел (точка «1»ы, выделена жирным).
    Выбор контуров и узлов не критичен: получившаяся система может быть чуть проще или сложнее, но не слишком.

    \begin{tikzpicture}[circuit ee IEC, thick]
        \draw  (0, 0) to [current direction={near end, info=$\eli_1$}] (0, 3)
                to [battery={rotate=-180,info={$\ele_1, r_1$}}]
                (3, 3)
                to [battery={info'={$\ele_2, r_2$}}]
                (6, 3) to [current direction'={near start, info=$\eli_2$}] (6, 0) -- (0, 0)
                (3, 0) to [current direction={near start, info=$\eli$}, resistor={near end, info=$R$}] (3, 3);
        \draw [-{Latex},color=red] (1.2, 1.7) arc [start angle = 135, end angle = -160, radius = 0.6];
        \draw [-{Latex},color=blue] (4.2, 1.7) arc [start angle = 135, end angle = -160, radius = 0.6];
        \node [contact,color=green!71!black] (bottomc) at (3, 0) {};
        \node [below] (bottom) at (3, 0) {$2$};
        \node [above] (top) at (3, 3) {$1$};
    \end{tikzpicture}

    \begin{align*}
        &\begin{cases}
            {\color{red} \ele_1 = \eli_1 r_1 - \eli R}, \\
            {\color{blue} -\ele_2 = -\eli_2 r_2 + \eli R}, \\
            {\color{green!71!black} - \eli - \eli_1 - \eli_2 = 0 };
        \end{cases}
        \qquad \implies \qquad
        \begin{cases}
            \eli_1 = \frac{\ele_1 + \eli R}{r_1}, \\
            \eli_2 = \frac{\ele_2 + \eli R}{r_2}, \\
            \eli + \eli_1 + \eli_2 = 0;
        \end{cases} \implies \\
        &\implies
         \eli + \frac{\ele_1 + \eli R}{r_1:L} + \frac{\ele_2 + \eli R}{r_2:L} = 0, \\
        &\eli\cbr{ 1 + \frac R{r_1:L} + \frac R{r_2:L}} + \frac{\ele_1 }{r_1:L} + \frac{\ele_2 }{r_2:L} = 0, \\
        &\eli
            = - \frac{\frac{\ele_1 }{r_1:L} + \frac{\ele_2 }{r_2:L}}{ 1 + \frac R{r_1:L} + \frac R{r_2:L}}
            = - \frac{\frac{40\,\text{В}}{1\,\text{Ом}} + \frac{20\,\text{В}}{2\,\text{Ом}}}{ 1 + \frac{10\,\text{Ом}}{1\,\text{Ом}} + \frac{10\,\text{Ом}}{2\,\text{Ом}}}
            = - \frac{25}8\units{А}
            \approx -3{,}10\,\text{А}, \\
        &U  = \varphi_2 - \varphi_1 = \eli R
            = - \frac{\frac{\ele_1 }{r_1:L} + \frac{\ele_2 }{r_2:L}}{ 1 + \frac R{r_1:L} + \frac R{r_2:L}} R
            \approx -31{,}20\,\text{В}.
    \end{align*}
    Оба ответа отрицательны, потому что мы изначально «не угадали» с направлением тока.
    Расчёт же показал,
    что ток через резистор $R$ течёт в противоположную сторону: вниз на рисунке, а потенциал точки 1 больше потенциала точки 2,
    а электрический ток ожидаемо течёт из точки с большим потенциалов в точку с меньшим.

    Кстати, если продолжить расчёт и вычислить значения ещё двух токов (формулы для $\eli_1$ и $\eli_2$, куда подставлять, выписаны выше),
    то по их знакам можно будет понять: угадали ли мы с их направлением или нет.
}
% autogenerated
