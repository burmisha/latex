\setdate{16~апреля~2021}
\setclass{9«М»}

\addpersonalvariant{Михаил Бурмистров}

\tasknumber{1}%
\task{%
    Установите соответствие буквам и запишите в ответ набор цифр (без других символов).

    А) $\beta$-излучение, Б) $\alpha$-излучение, В) $\gamma$-излучение.

    1) не несёт электрического заряда, 2) обладает положительным зарядом, 3) обладает отрицательным электрическим зарядом.
}
\answer{%
    $321$
}
\solutionspace{40pt}

\tasknumber{2}%
\task{%
    Установите соответствие буквам и запишите в ответ набор цифр (без других символов).

    А) $\beta$-излучение, Б) $\alpha$-излучение, В) $\gamma$-излучение.

    1) электромагнитное излучение, 2) ядра атомов гелия, 3) электроны.
}
\answer{%
    $321$
}
\solutionspace{40pt}

\tasknumber{3}%
\task{%
    Установите соответствие буквам и запишите в ответ набор цифр (без других символов).

    А) атом Резерфорда, Б) атом Томсона.

    1) планетарная модель атома, 2) «пудинг с изюмом».
}
\answer{%
    $12$
}
\solutionspace{40pt}

\tasknumber{4}%
\task{%
    Установите соответствие буквам и запишите в ответ набор цифр (без других символов).

    А) размер ядра атома, Б) размер атома.

    1) $10^{-10}\units{ см }$, 2) $10^{-8}\units{ см }$, 3) $10^{-13}\units{ см }$.
}
\answer{%
    $32$
}
\solutionspace{40pt}

\tasknumber{5}%
\task{%
    Установите соответствие буквам и запишите в ответ набор цифр (без других символов).

    А) массовое число водорода \ce{^{1}_{1}H}, Б) зарядовое число азота \ce{^{14}_{7}O}.

    1) 8, 2) 0, 3) 9, 4) 7.
}
\answer{%
    $24$
}
\solutionspace{40pt}

\tasknumber{6}%
\task{%
    Установите соответствие буквам и запишите в ответ набор цифр (без других символов).

    А) зарядовое число $\alpha$-частицы, Б) зарядовое число $\beta$-частицы, В) массовое число $\alpha$-частицы.

    1) 4, 2) 1, 3) -2, 4) 2, 5) -1.
}
\answer{%
    $451$
}
\solutionspace{40pt}

\tasknumber{7}%
\task{%
    На какой максимальный угол (в градусах) отконялись $\alpha$-частицы
    в опытах Резерфорда по их рассеянию на тонкой золотой фольге?
}
\answer{%
    $180\degrees$
}

\variantsplitter

\addpersonalvariant{Артём Глембо}

\tasknumber{1}%
\task{%
    Установите соответствие буквам и запишите в ответ набор цифр (без других символов).

    А) $\gamma$-излучение, Б) $\beta$-излучение, В) $\alpha$-излучение.

    1) обладает положительным зарядом, 2) не несёт электрического заряда, 3) обладает отрицательным электрическим зарядом.
}
\answer{%
    $231$
}
\solutionspace{40pt}

\tasknumber{2}%
\task{%
    Установите соответствие буквам и запишите в ответ набор цифр (без других символов).

    А) $\gamma$-излучение, Б) $\beta$-излучение, В) $\alpha$-излучение.

    1) ядра атомов гелия, 2) электромагнитное излучение, 3) электроны.
}
\answer{%
    $231$
}
\solutionspace{40pt}

\tasknumber{3}%
\task{%
    Установите соответствие буквам и запишите в ответ набор цифр (без других символов).

    А) атом Томсона, Б) атом Резерфорда.

    1) «пудинг с изюмом», 2) планетарная модель атома.
}
\answer{%
    $12$
}
\solutionspace{40pt}

\tasknumber{4}%
\task{%
    Установите соответствие буквам и запишите в ответ набор цифр (без других символов).

    А) размер атома, Б) размер ядра атома.

    1) $10^{-8}\units{ см }$, 2) $10^{-13}\units{ см }$, 3) $10^{-15}\units{ см }$.
}
\answer{%
    $12$
}
\solutionspace{40pt}

\tasknumber{5}%
\task{%
    Установите соответствие буквам и запишите в ответ набор цифр (без других символов).

    А) зарядовое число азота \ce{^{14}_{7}O}, Б) массовое число водорода \ce{^{1}_{1}H}.

    1) 7, 2) 14, 3) 0, 4) 9.
}
\answer{%
    $13$
}
\solutionspace{40pt}

\tasknumber{6}%
\task{%
    Установите соответствие буквам и запишите в ответ набор цифр (без других символов).

    А) массовое число $\beta$-частицы, Б) массовое число $\alpha$-частицы, В) зарядовое число $\beta$-частицы.

    1) 1, 2) -1, 3) 4, 4) -2, 5) 0.
}
\answer{%
    $532$
}
\solutionspace{40pt}

\tasknumber{7}%
\task{%
    На какой минимальный угол (в градусах) отконялись $\alpha$-частицы
    в опытах Резерфорда по их рассеянию на тонкой золотой фольге?
}
\answer{%
    $0\degrees$
}

\variantsplitter

\addpersonalvariant{Наталья Гончарова}

\tasknumber{1}%
\task{%
    Установите соответствие буквам и запишите в ответ набор цифр (без других символов).

    А) $\alpha$-излучение, Б) $\beta$-излучение, В) $\gamma$-излучение.

    1) не несёт электрического заряда, 2) обладает положительным зарядом, 3) обладает отрицательным электрическим зарядом.
}
\answer{%
    $231$
}
\solutionspace{40pt}

\tasknumber{2}%
\task{%
    Установите соответствие буквам и запишите в ответ набор цифр (без других символов).

    А) $\alpha$-излучение, Б) $\beta$-излучение, В) $\gamma$-излучение.

    1) электромагнитное излучение, 2) ядра атомов гелия, 3) электроны.
}
\answer{%
    $231$
}
\solutionspace{40pt}

\tasknumber{3}%
\task{%
    Установите соответствие буквам и запишите в ответ набор цифр (без других символов).

    А) атом Резерфорда, Б) атом Томсона.

    1) планетарная модель атома, 2) «пудинг с изюмом».
}
\answer{%
    $12$
}
\solutionspace{40pt}

\tasknumber{4}%
\task{%
    Установите соответствие буквам и запишите в ответ набор цифр (без других символов).

    А) размер атома, Б) размер ядра атома.

    1) $10^{-10}\units{ см }$, 2) $10^{-8}\units{ см }$, 3) $10^{-13}\units{ см }$.
}
\answer{%
    $23$
}
\solutionspace{40pt}

\tasknumber{5}%
\task{%
    Установите соответствие буквам и запишите в ответ набор цифр (без других символов).

    А) зарядовое число углерода \ce{^{12}_{6}C}, Б) зарядовое число кислорода \ce{^{16}_{8}O}.

    1) 0, 2) 6, 3) 14, 4) 8.
}
\answer{%
    $24$
}
\solutionspace{40pt}

\tasknumber{6}%
\task{%
    Установите соответствие буквам и запишите в ответ набор цифр (без других символов).

    А) массовое число $\alpha$-частицы, Б) зарядовое число $\alpha$-частицы, В) массовое число $\beta$-частицы.

    1) 0, 2) -2, 3) -1, 4) 4, 5) 2.
}
\answer{%
    $451$
}
\solutionspace{40pt}

\tasknumber{7}%
\task{%
    На какой максимальный угол (в градусах) отконялись $\alpha$-частицы
    в опытах Резерфорда по их рассеянию на тонкой золотой фольге?
}
\answer{%
    $180\degrees$
}

\variantsplitter

\addpersonalvariant{Файёзбек Касымов}

\tasknumber{1}%
\task{%
    Установите соответствие буквам и запишите в ответ набор цифр (без других символов).

    А) $\alpha$-излучение, Б) $\gamma$-излучение, В) $\beta$-излучение.

    1) обладает отрицательным электрическим зарядом, 2) обладает положительным зарядом, 3) не несёт электрического заряда.
}
\answer{%
    $231$
}
\solutionspace{40pt}

\tasknumber{2}%
\task{%
    Установите соответствие буквам и запишите в ответ набор цифр (без других символов).

    А) $\alpha$-излучение, Б) $\gamma$-излучение, В) $\beta$-излучение.

    1) электроны, 2) ядра атомов гелия, 3) электромагнитное излучение.
}
\answer{%
    $231$
}
\solutionspace{40pt}

\tasknumber{3}%
\task{%
    Установите соответствие буквам и запишите в ответ набор цифр (без других символов).

    А) атом Резерфорда, Б) атом Томсона.

    1) планетарная модель атома, 2) «пудинг с изюмом».
}
\answer{%
    $12$
}
\solutionspace{40pt}

\tasknumber{4}%
\task{%
    Установите соответствие буквам и запишите в ответ набор цифр (без других символов).

    А) размер ядра атома, Б) размер атома.

    1) $10^{-15}\units{ см }$, 2) $10^{-8}\units{ см }$, 3) $10^{-13}\units{ см }$.
}
\answer{%
    $32$
}
\solutionspace{40pt}

\tasknumber{5}%
\task{%
    Установите соответствие буквам и запишите в ответ набор цифр (без других символов).

    А) массовое число углерода \ce{^{12}_{6}C}, Б) массовое число кислорода \ce{^{16}_{8}O}.

    1) 16, 2) 11, 3) 12, 4) 6.
}
\answer{%
    $31$
}
\solutionspace{40pt}

\tasknumber{6}%
\task{%
    Установите соответствие буквам и запишите в ответ набор цифр (без других символов).

    А) зарядовое число $\beta$-частицы, Б) зарядовое число $\alpha$-частицы, В) массовое число $\alpha$-частицы.

    1) 4, 2) 2, 3) -1, 4) 0, 5) -2.
}
\answer{%
    $321$
}
\solutionspace{40pt}

\tasknumber{7}%
\task{%
    На какой максимальный угол (в градусах) отконялись $\alpha$-частицы
    в опытах Резерфорда по их рассеянию на тонкой золотой фольге?
}
\answer{%
    $180\degrees$
}

\variantsplitter

\addpersonalvariant{Александр Козинец}

\tasknumber{1}%
\task{%
    Установите соответствие буквам и запишите в ответ набор цифр (без других символов).

    А) $\alpha$-излучение, Б) $\gamma$-излучение, В) $\beta$-излучение.

    1) обладает отрицательным электрическим зарядом, 2) обладает положительным зарядом, 3) не несёт электрического заряда.
}
\answer{%
    $231$
}
\solutionspace{40pt}

\tasknumber{2}%
\task{%
    Установите соответствие буквам и запишите в ответ набор цифр (без других символов).

    А) $\alpha$-излучение, Б) $\gamma$-излучение, В) $\beta$-излучение.

    1) электроны, 2) ядра атомов гелия, 3) электромагнитное излучение.
}
\answer{%
    $231$
}
\solutionspace{40pt}

\tasknumber{3}%
\task{%
    Установите соответствие буквам и запишите в ответ набор цифр (без других символов).

    А) атом Резерфорда, Б) атом Томсона.

    1) планетарная модель атома, 2) «пудинг с изюмом».
}
\answer{%
    $12$
}
\solutionspace{40pt}

\tasknumber{4}%
\task{%
    Установите соответствие буквам и запишите в ответ набор цифр (без других символов).

    А) размер ядра атома, Б) размер атома.

    1) $10^{-15}\units{ см }$, 2) $10^{-8}\units{ см }$, 3) $10^{-13}\units{ см }$.
}
\answer{%
    $32$
}
\solutionspace{40pt}

\tasknumber{5}%
\task{%
    Установите соответствие буквам и запишите в ответ набор цифр (без других символов).

    А) зарядовое число кислорода \ce{^{16}_{8}O}, Б) зарядовое число азота \ce{^{14}_{7}O}.

    1) 14, 2) 8, 3) 12, 4) 7.
}
\answer{%
    $24$
}
\solutionspace{40pt}

\tasknumber{6}%
\task{%
    Установите соответствие буквам и запишите в ответ набор цифр (без других символов).

    А) зарядовое число $\beta$-частицы, Б) зарядовое число $\alpha$-частицы, В) массовое число $\beta$-частицы.

    1) -2, 2) 2, 3) 0, 4) 1, 5) -1.
}
\answer{%
    $523$
}
\solutionspace{40pt}

\tasknumber{7}%
\task{%
    На какой минимальный угол (в градусах) отконялись $\alpha$-частицы
    в опытах Резерфорда по их рассеянию на тонкой золотой фольге?
}
\answer{%
    $0\degrees$
}

\variantsplitter

\addpersonalvariant{Андрей Куликовский}

\tasknumber{1}%
\task{%
    Установите соответствие буквам и запишите в ответ набор цифр (без других символов).

    А) $\beta$-излучение, Б) $\gamma$-излучение, В) $\alpha$-излучение.

    1) обладает отрицательным электрическим зарядом, 2) обладает положительным зарядом, 3) не несёт электрического заряда.
}
\answer{%
    $132$
}
\solutionspace{40pt}

\tasknumber{2}%
\task{%
    Установите соответствие буквам и запишите в ответ набор цифр (без других символов).

    А) $\beta$-излучение, Б) $\gamma$-излучение, В) $\alpha$-излучение.

    1) электроны, 2) ядра атомов гелия, 3) электромагнитное излучение.
}
\answer{%
    $132$
}
\solutionspace{40pt}

\tasknumber{3}%
\task{%
    Установите соответствие буквам и запишите в ответ набор цифр (без других символов).

    А) атом Резерфорда, Б) атом Томсона.

    1) планетарная модель атома, 2) «пудинг с изюмом».
}
\answer{%
    $12$
}
\solutionspace{40pt}

\tasknumber{4}%
\task{%
    Установите соответствие буквам и запишите в ответ набор цифр (без других символов).

    А) размер ядра атома, Б) размер атома.

    1) $10^{-15}\units{ см }$, 2) $10^{-8}\units{ см }$, 3) $10^{-13}\units{ см }$.
}
\answer{%
    $32$
}
\solutionspace{40pt}

\tasknumber{5}%
\task{%
    Установите соответствие буквам и запишите в ответ набор цифр (без других символов).

    А) массовое число углерода \ce{^{12}_{6}C}, Б) зарядовое число углерода \ce{^{12}_{6}C}.

    1) 12, 2) 6, 3) 8, 4) 0.
}
\answer{%
    $12$
}
\solutionspace{40pt}

\tasknumber{6}%
\task{%
    Установите соответствие буквам и запишите в ответ набор цифр (без других символов).

    А) зарядовое число $\beta$-частицы, Б) массовое число $\beta$-частицы, В) зарядовое число $\alpha$-частицы.

    1) 1, 2) 2, 3) -1, 4) 0, 5) -2.
}
\answer{%
    $342$
}
\solutionspace{40pt}

\tasknumber{7}%
\task{%
    На какой максимальный угол (в градусах) отконялись $\alpha$-частицы
    в опытах Резерфорда по их рассеянию на тонкой золотой фольге?
}
\answer{%
    $180\degrees$
}

\variantsplitter

\addpersonalvariant{Полина Лоткова}

\tasknumber{1}%
\task{%
    Установите соответствие буквам и запишите в ответ набор цифр (без других символов).

    А) $\alpha$-излучение, Б) $\gamma$-излучение, В) $\beta$-излучение.

    1) не несёт электрического заряда, 2) обладает положительным зарядом, 3) обладает отрицательным электрическим зарядом.
}
\answer{%
    $213$
}
\solutionspace{40pt}

\tasknumber{2}%
\task{%
    Установите соответствие буквам и запишите в ответ набор цифр (без других символов).

    А) $\alpha$-излучение, Б) $\gamma$-излучение, В) $\beta$-излучение.

    1) электромагнитное излучение, 2) ядра атомов гелия, 3) электроны.
}
\answer{%
    $213$
}
\solutionspace{40pt}

\tasknumber{3}%
\task{%
    Установите соответствие буквам и запишите в ответ набор цифр (без других символов).

    А) атом Томсона, Б) атом Резерфорда.

    1) планетарная модель атома, 2) «пудинг с изюмом».
}
\answer{%
    $21$
}
\solutionspace{40pt}

\tasknumber{4}%
\task{%
    Установите соответствие буквам и запишите в ответ набор цифр (без других символов).

    А) размер атома, Б) размер ядра атома.

    1) $10^{-13}\units{ см }$, 2) $10^{-8}\units{ см }$, 3) $10^{-15}\units{ см }$.
}
\answer{%
    $21$
}
\solutionspace{40pt}

\tasknumber{5}%
\task{%
    Установите соответствие буквам и запишите в ответ набор цифр (без других символов).

    А) массовое число азота \ce{^{14}_{7}N}, Б) массовое число кислорода \ce{^{16}_{8}O}.

    1) 7, 2) 6, 3) 16, 4) 14.
}
\answer{%
    $43$
}
\solutionspace{40pt}

\tasknumber{6}%
\task{%
    Установите соответствие буквам и запишите в ответ набор цифр (без других символов).

    А) зарядовое число $\beta$-частицы, Б) массовое число $\alpha$-частицы, В) зарядовое число $\alpha$-частицы.

    1) -2, 2) 0, 3) -1, 4) 4, 5) 2.
}
\answer{%
    $345$
}
\solutionspace{40pt}

\tasknumber{7}%
\task{%
    На какой максимальный угол (в градусах) отконялись $\alpha$-частицы
    в опытах Резерфорда по их рассеянию на тонкой золотой фольге?
}
\answer{%
    $180\degrees$
}

\variantsplitter

\addpersonalvariant{Екатерина Медведева}

\tasknumber{1}%
\task{%
    Установите соответствие буквам и запишите в ответ набор цифр (без других символов).

    А) $\alpha$-излучение, Б) $\beta$-излучение, В) $\gamma$-излучение.

    1) не несёт электрического заряда, 2) обладает отрицательным электрическим зарядом, 3) обладает положительным зарядом.
}
\answer{%
    $321$
}
\solutionspace{40pt}

\tasknumber{2}%
\task{%
    Установите соответствие буквам и запишите в ответ набор цифр (без других символов).

    А) $\alpha$-излучение, Б) $\beta$-излучение, В) $\gamma$-излучение.

    1) электромагнитное излучение, 2) электроны, 3) ядра атомов гелия.
}
\answer{%
    $321$
}
\solutionspace{40pt}

\tasknumber{3}%
\task{%
    Установите соответствие буквам и запишите в ответ набор цифр (без других символов).

    А) атом Резерфорда, Б) атом Томсона.

    1) «пудинг с изюмом», 2) планетарная модель атома.
}
\answer{%
    $21$
}
\solutionspace{40pt}

\tasknumber{4}%
\task{%
    Установите соответствие буквам и запишите в ответ набор цифр (без других символов).

    А) размер ядра атома, Б) размер атома.

    1) $10^{-10}\units{ см }$, 2) $10^{-13}\units{ см }$, 3) $10^{-8}\units{ см }$.
}
\answer{%
    $23$
}
\solutionspace{40pt}

\tasknumber{5}%
\task{%
    Установите соответствие буквам и запишите в ответ набор цифр (без других символов).

    А) массовое число кислорода \ce{^{16}_{8}O}, Б) массовое число водорода \ce{^{1}_{1}H}.

    1) 12, 2) 0, 3) 8, 4) 16.
}
\answer{%
    $42$
}
\solutionspace{40pt}

\tasknumber{6}%
\task{%
    Установите соответствие буквам и запишите в ответ набор цифр (без других символов).

    А) зарядовое число $\alpha$-частицы, Б) массовое число $\alpha$-частицы, В) зарядовое число $\beta$-частицы.

    1) 2, 2) 0, 3) -2, 4) -1, 5) 4.
}
\answer{%
    $154$
}
\solutionspace{40pt}

\tasknumber{7}%
\task{%
    На какой минимальный угол (в градусах) отконялись $\alpha$-частицы
    в опытах Резерфорда по их рассеянию на тонкой золотой фольге?
}
\answer{%
    $0\degrees$
}

\variantsplitter

\addpersonalvariant{Константин Мельник}

\tasknumber{1}%
\task{%
    Установите соответствие буквам и запишите в ответ набор цифр (без других символов).

    А) $\gamma$-излучение, Б) $\beta$-излучение, В) $\alpha$-излучение.

    1) обладает отрицательным электрическим зарядом, 2) не несёт электрического заряда, 3) обладает положительным зарядом.
}
\answer{%
    $213$
}
\solutionspace{40pt}

\tasknumber{2}%
\task{%
    Установите соответствие буквам и запишите в ответ набор цифр (без других символов).

    А) $\gamma$-излучение, Б) $\beta$-излучение, В) $\alpha$-излучение.

    1) электроны, 2) электромагнитное излучение, 3) ядра атомов гелия.
}
\answer{%
    $213$
}
\solutionspace{40pt}

\tasknumber{3}%
\task{%
    Установите соответствие буквам и запишите в ответ набор цифр (без других символов).

    А) атом Резерфорда, Б) атом Томсона.

    1) «пудинг с изюмом», 2) планетарная модель атома.
}
\answer{%
    $21$
}
\solutionspace{40pt}

\tasknumber{4}%
\task{%
    Установите соответствие буквам и запишите в ответ набор цифр (без других символов).

    А) размер ядра атома, Б) размер атома.

    1) $10^{-15}\units{ см }$, 2) $10^{-13}\units{ см }$, 3) $10^{-8}\units{ см }$.
}
\answer{%
    $23$
}
\solutionspace{40pt}

\tasknumber{5}%
\task{%
    Установите соответствие буквам и запишите в ответ набор цифр (без других символов).

    А) массовое число углерода \ce{^{12}_{6}C}, Б) массовое число водорода \ce{^{1}_{1}H}.

    1) 16, 2) 12, 3) 0, 4) 14.
}
\answer{%
    $23$
}
\solutionspace{40pt}

\tasknumber{6}%
\task{%
    Установите соответствие буквам и запишите в ответ набор цифр (без других символов).

    А) массовое число $\alpha$-частицы, Б) зарядовое число $\alpha$-частицы, В) массовое число $\beta$-частицы.

    1) -2, 2) 4, 3) 1, 4) 0, 5) 2.
}
\answer{%
    $254$
}
\solutionspace{40pt}

\tasknumber{7}%
\task{%
    На какой максимальный угол (в градусах) отконялись $\alpha$-частицы
    в опытах Резерфорда по их рассеянию на тонкой золотой фольге?
}
\answer{%
    $180\degrees$
}

\variantsplitter

\addpersonalvariant{Степан Небоваренков}

\tasknumber{1}%
\task{%
    Установите соответствие буквам и запишите в ответ набор цифр (без других символов).

    А) $\gamma$-излучение, Б) $\alpha$-излучение, В) $\beta$-излучение.

    1) обладает отрицательным электрическим зарядом, 2) не несёт электрического заряда, 3) обладает положительным зарядом.
}
\answer{%
    $231$
}
\solutionspace{40pt}

\tasknumber{2}%
\task{%
    Установите соответствие буквам и запишите в ответ набор цифр (без других символов).

    А) $\gamma$-излучение, Б) $\alpha$-излучение, В) $\beta$-излучение.

    1) электроны, 2) электромагнитное излучение, 3) ядра атомов гелия.
}
\answer{%
    $231$
}
\solutionspace{40pt}

\tasknumber{3}%
\task{%
    Установите соответствие буквам и запишите в ответ набор цифр (без других символов).

    А) атом Томсона, Б) атом Резерфорда.

    1) «пудинг с изюмом», 2) планетарная модель атома.
}
\answer{%
    $12$
}
\solutionspace{40pt}

\tasknumber{4}%
\task{%
    Установите соответствие буквам и запишите в ответ набор цифр (без других символов).

    А) размер ядра атома, Б) размер атома.

    1) $10^{-13}\units{ см }$, 2) $10^{-10}\units{ см }$, 3) $10^{-8}\units{ см }$.
}
\answer{%
    $13$
}
\solutionspace{40pt}

\tasknumber{5}%
\task{%
    Установите соответствие буквам и запишите в ответ набор цифр (без других символов).

    А) массовое число кислорода \ce{^{16}_{8}O}, Б) зарядовое число углерода \ce{^{12}_{6}C}.

    1) 14, 2) 12, 3) 6, 4) 16.
}
\answer{%
    $43$
}
\solutionspace{40pt}

\tasknumber{6}%
\task{%
    Установите соответствие буквам и запишите в ответ набор цифр (без других символов).

    А) массовое число $\beta$-частицы, Б) зарядовое число $\beta$-частицы, В) зарядовое число $\alpha$-частицы.

    1) 4, 2) 1, 3) 0, 4) -1, 5) 2.
}
\answer{%
    $345$
}
\solutionspace{40pt}

\tasknumber{7}%
\task{%
    На какой минимальный угол (в градусах) отконялись $\alpha$-частицы
    в опытах Резерфорда по их рассеянию на тонкой золотой фольге?
}
\answer{%
    $0\degrees$
}

\variantsplitter

\addpersonalvariant{Матвей Неретин}

\tasknumber{1}%
\task{%
    Установите соответствие буквам и запишите в ответ набор цифр (без других символов).

    А) $\gamma$-излучение, Б) $\beta$-излучение, В) $\alpha$-излучение.

    1) обладает отрицательным электрическим зарядом, 2) обладает положительным зарядом, 3) не несёт электрического заряда.
}
\answer{%
    $312$
}
\solutionspace{40pt}

\tasknumber{2}%
\task{%
    Установите соответствие буквам и запишите в ответ набор цифр (без других символов).

    А) $\gamma$-излучение, Б) $\beta$-излучение, В) $\alpha$-излучение.

    1) электроны, 2) ядра атомов гелия, 3) электромагнитное излучение.
}
\answer{%
    $312$
}
\solutionspace{40pt}

\tasknumber{3}%
\task{%
    Установите соответствие буквам и запишите в ответ набор цифр (без других символов).

    А) атом Резерфорда, Б) атом Томсона.

    1) планетарная модель атома, 2) «пудинг с изюмом».
}
\answer{%
    $12$
}
\solutionspace{40pt}

\tasknumber{4}%
\task{%
    Установите соответствие буквам и запишите в ответ набор цифр (без других символов).

    А) размер ядра атома, Б) размер атома.

    1) $10^{-15}\units{ см }$, 2) $10^{-8}\units{ см }$, 3) $10^{-13}\units{ см }$.
}
\answer{%
    $32$
}
\solutionspace{40pt}

\tasknumber{5}%
\task{%
    Установите соответствие буквам и запишите в ответ набор цифр (без других символов).

    А) зарядовое число азота \ce{^{14}_{7}O}, Б) массовое число водорода \ce{^{1}_{1}H}.

    1) 16, 2) 7, 3) 11, 4) 0.
}
\answer{%
    $24$
}
\solutionspace{40pt}

\tasknumber{6}%
\task{%
    Установите соответствие буквам и запишите в ответ набор цифр (без других символов).

    А) зарядовое число $\beta$-частицы, Б) массовое число $\beta$-частицы, В) массовое число $\alpha$-частицы.

    1) 0, 2) 4, 3) -1, 4) 1, 5) -2.
}
\answer{%
    $312$
}
\solutionspace{40pt}

\tasknumber{7}%
\task{%
    На какой минимальный угол (в градусах) отконялись $\alpha$-частицы
    в опытах Резерфорда по их рассеянию на тонкой золотой фольге?
}
\answer{%
    $0\degrees$
}

\variantsplitter

\addpersonalvariant{Мария Никонова}

\tasknumber{1}%
\task{%
    Установите соответствие буквам и запишите в ответ набор цифр (без других символов).

    А) $\alpha$-излучение, Б) $\beta$-излучение, В) $\gamma$-излучение.

    1) обладает положительным зарядом, 2) не несёт электрического заряда, 3) обладает отрицательным электрическим зарядом.
}
\answer{%
    $132$
}
\solutionspace{40pt}

\tasknumber{2}%
\task{%
    Установите соответствие буквам и запишите в ответ набор цифр (без других символов).

    А) $\alpha$-излучение, Б) $\beta$-излучение, В) $\gamma$-излучение.

    1) ядра атомов гелия, 2) электромагнитное излучение, 3) электроны.
}
\answer{%
    $132$
}
\solutionspace{40pt}

\tasknumber{3}%
\task{%
    Установите соответствие буквам и запишите в ответ набор цифр (без других символов).

    А) атом Резерфорда, Б) атом Томсона.

    1) «пудинг с изюмом», 2) планетарная модель атома.
}
\answer{%
    $21$
}
\solutionspace{40pt}

\tasknumber{4}%
\task{%
    Установите соответствие буквам и запишите в ответ набор цифр (без других символов).

    А) размер ядра атома, Б) размер атома.

    1) $10^{-8}\units{ см }$, 2) $10^{-10}\units{ см }$, 3) $10^{-13}\units{ см }$.
}
\answer{%
    $31$
}
\solutionspace{40pt}

\tasknumber{5}%
\task{%
    Установите соответствие буквам и запишите в ответ набор цифр (без других символов).

    А) зарядовое число углерода \ce{^{12}_{6}C}, Б) массовое число водорода \ce{^{1}_{1}H}.

    1) 9, 2) 10, 3) 0, 4) 6.
}
\answer{%
    $43$
}
\solutionspace{40pt}

\tasknumber{6}%
\task{%
    Установите соответствие буквам и запишите в ответ набор цифр (без других символов).

    А) зарядовое число $\alpha$-частицы, Б) массовое число $\alpha$-частицы, В) зарядовое число $\beta$-частицы.

    1) 4, 2) 1, 3) 2, 4) -1, 5) 0.
}
\answer{%
    $314$
}
\solutionspace{40pt}

\tasknumber{7}%
\task{%
    На какой максимальный угол (в градусах) отконялись $\alpha$-частицы
    в опытах Резерфорда по их рассеянию на тонкой золотой фольге?
}
\answer{%
    $180\degrees$
}

\variantsplitter

\addpersonalvariant{Даниил Палаткин}

\tasknumber{1}%
\task{%
    Установите соответствие буквам и запишите в ответ набор цифр (без других символов).

    А) $\alpha$-излучение, Б) $\beta$-излучение, В) $\gamma$-излучение.

    1) обладает положительным зарядом, 2) обладает отрицательным электрическим зарядом, 3) не несёт электрического заряда.
}
\answer{%
    $123$
}
\solutionspace{40pt}

\tasknumber{2}%
\task{%
    Установите соответствие буквам и запишите в ответ набор цифр (без других символов).

    А) $\alpha$-излучение, Б) $\beta$-излучение, В) $\gamma$-излучение.

    1) ядра атомов гелия, 2) электроны, 3) электромагнитное излучение.
}
\answer{%
    $123$
}
\solutionspace{40pt}

\tasknumber{3}%
\task{%
    Установите соответствие буквам и запишите в ответ набор цифр (без других символов).

    А) атом Резерфорда, Б) атом Томсона.

    1) планетарная модель атома, 2) «пудинг с изюмом».
}
\answer{%
    $12$
}
\solutionspace{40pt}

\tasknumber{4}%
\task{%
    Установите соответствие буквам и запишите в ответ набор цифр (без других символов).

    А) размер атома, Б) размер ядра атома.

    1) $10^{-8}\units{ см }$, 2) $10^{-13}\units{ см }$, 3) $10^{-10}\units{ см }$.
}
\answer{%
    $12$
}
\solutionspace{40pt}

\tasknumber{5}%
\task{%
    Установите соответствие буквам и запишите в ответ набор цифр (без других символов).

    А) массовое число углерода \ce{^{12}_{6}C}, Б) массовое число водорода \ce{^{1}_{1}H}.

    1) 0, 2) 16, 3) 12, 4) 7.
}
\answer{%
    $31$
}
\solutionspace{40pt}

\tasknumber{6}%
\task{%
    Установите соответствие буквам и запишите в ответ набор цифр (без других символов).

    А) массовое число $\beta$-частицы, Б) зарядовое число $\alpha$-частицы, В) массовое число $\alpha$-частицы.

    1) 4, 2) -2, 3) 2, 4) 1, 5) 0.
}
\answer{%
    $531$
}
\solutionspace{40pt}

\tasknumber{7}%
\task{%
    На какой минимальный угол (в градусах) отконялись $\alpha$-частицы
    в опытах Резерфорда по их рассеянию на тонкой золотой фольге?
}
\answer{%
    $0\degrees$
}

\variantsplitter

\addpersonalvariant{Станислав Пикун}

\tasknumber{1}%
\task{%
    Установите соответствие буквам и запишите в ответ набор цифр (без других символов).

    А) $\gamma$-излучение, Б) $\beta$-излучение, В) $\alpha$-излучение.

    1) обладает положительным зарядом, 2) не несёт электрического заряда, 3) обладает отрицательным электрическим зарядом.
}
\answer{%
    $231$
}
\solutionspace{40pt}

\tasknumber{2}%
\task{%
    Установите соответствие буквам и запишите в ответ набор цифр (без других символов).

    А) $\gamma$-излучение, Б) $\beta$-излучение, В) $\alpha$-излучение.

    1) ядра атомов гелия, 2) электромагнитное излучение, 3) электроны.
}
\answer{%
    $231$
}
\solutionspace{40pt}

\tasknumber{3}%
\task{%
    Установите соответствие буквам и запишите в ответ набор цифр (без других символов).

    А) атом Томсона, Б) атом Резерфорда.

    1) «пудинг с изюмом», 2) планетарная модель атома.
}
\answer{%
    $12$
}
\solutionspace{40pt}

\tasknumber{4}%
\task{%
    Установите соответствие буквам и запишите в ответ набор цифр (без других символов).

    А) размер ядра атома, Б) размер атома.

    1) $10^{-8}\units{ см }$, 2) $10^{-13}\units{ см }$, 3) $10^{-15}\units{ см }$.
}
\answer{%
    $21$
}
\solutionspace{40pt}

\tasknumber{5}%
\task{%
    Установите соответствие буквам и запишите в ответ набор цифр (без других символов).

    А) массовое число водорода \ce{^{1}_{1}H}, Б) зарядовое число азота \ce{^{14}_{7}O}.

    1) 14, 2) 10, 3) 0, 4) 7.
}
\answer{%
    $34$
}
\solutionspace{40pt}

\tasknumber{6}%
\task{%
    Установите соответствие буквам и запишите в ответ набор цифр (без других символов).

    А) зарядовое число $\alpha$-частицы, Б) зарядовое число $\beta$-частицы, В) массовое число $\beta$-частицы.

    1) 1, 2) 4, 3) 2, 4) -1, 5) 0.
}
\answer{%
    $345$
}
\solutionspace{40pt}

\tasknumber{7}%
\task{%
    На какой минимальный угол (в градусах) отконялись $\alpha$-частицы
    в опытах Резерфорда по их рассеянию на тонкой золотой фольге?
}
\answer{%
    $0\degrees$
}

\variantsplitter

\addpersonalvariant{Илья Пичугин}

\tasknumber{1}%
\task{%
    Установите соответствие буквам и запишите в ответ набор цифр (без других символов).

    А) $\beta$-излучение, Б) $\gamma$-излучение, В) $\alpha$-излучение.

    1) обладает отрицательным электрическим зарядом, 2) обладает положительным зарядом, 3) не несёт электрического заряда.
}
\answer{%
    $132$
}
\solutionspace{40pt}

\tasknumber{2}%
\task{%
    Установите соответствие буквам и запишите в ответ набор цифр (без других символов).

    А) $\beta$-излучение, Б) $\gamma$-излучение, В) $\alpha$-излучение.

    1) электроны, 2) ядра атомов гелия, 3) электромагнитное излучение.
}
\answer{%
    $132$
}
\solutionspace{40pt}

\tasknumber{3}%
\task{%
    Установите соответствие буквам и запишите в ответ набор цифр (без других символов).

    А) атом Резерфорда, Б) атом Томсона.

    1) планетарная модель атома, 2) «пудинг с изюмом».
}
\answer{%
    $12$
}
\solutionspace{40pt}

\tasknumber{4}%
\task{%
    Установите соответствие буквам и запишите в ответ набор цифр (без других символов).

    А) размер атома, Б) размер ядра атома.

    1) $10^{-15}\units{ см }$, 2) $10^{-8}\units{ см }$, 3) $10^{-13}\units{ см }$.
}
\answer{%
    $23$
}
\solutionspace{40pt}

\tasknumber{5}%
\task{%
    Установите соответствие буквам и запишите в ответ набор цифр (без других символов).

    А) зарядовое число кислорода \ce{^{16}_{8}O}, Б) массовое число углерода \ce{^{12}_{6}C}.

    1) 7, 2) 8, 3) 12, 4) 5.
}
\answer{%
    $23$
}
\solutionspace{40pt}

\tasknumber{6}%
\task{%
    Установите соответствие буквам и запишите в ответ набор цифр (без других символов).

    А) зарядовое число $\beta$-частицы, Б) зарядовое число $\alpha$-частицы, В) массовое число $\alpha$-частицы.

    1) 2, 2) -1, 3) 1, 4) 4, 5) -2.
}
\answer{%
    $214$
}
\solutionspace{40pt}

\tasknumber{7}%
\task{%
    На какой максимальный угол (в градусах) отконялись $\alpha$-частицы
    в опытах Резерфорда по их рассеянию на тонкой золотой фольге?
}
\answer{%
    $180\degrees$
}

\variantsplitter

\addpersonalvariant{Кирилл Севрюгин}

\tasknumber{1}%
\task{%
    Установите соответствие буквам и запишите в ответ набор цифр (без других символов).

    А) $\gamma$-излучение, Б) $\beta$-излучение, В) $\alpha$-излучение.

    1) обладает положительным зарядом, 2) не несёт электрического заряда, 3) обладает отрицательным электрическим зарядом.
}
\answer{%
    $231$
}
\solutionspace{40pt}

\tasknumber{2}%
\task{%
    Установите соответствие буквам и запишите в ответ набор цифр (без других символов).

    А) $\gamma$-излучение, Б) $\beta$-излучение, В) $\alpha$-излучение.

    1) ядра атомов гелия, 2) электромагнитное излучение, 3) электроны.
}
\answer{%
    $231$
}
\solutionspace{40pt}

\tasknumber{3}%
\task{%
    Установите соответствие буквам и запишите в ответ набор цифр (без других символов).

    А) атом Томсона, Б) атом Резерфорда.

    1) «пудинг с изюмом», 2) планетарная модель атома.
}
\answer{%
    $12$
}
\solutionspace{40pt}

\tasknumber{4}%
\task{%
    Установите соответствие буквам и запишите в ответ набор цифр (без других символов).

    А) размер ядра атома, Б) размер атома.

    1) $10^{-8}\units{ см }$, 2) $10^{-13}\units{ см }$, 3) $10^{-15}\units{ см }$.
}
\answer{%
    $21$
}
\solutionspace{40pt}

\tasknumber{5}%
\task{%
    Установите соответствие буквам и запишите в ответ набор цифр (без других символов).

    А) массовое число углерода \ce{^{12}_{6}C}, Б) массовое число водорода \ce{^{1}_{1}H}.

    1) 12, 2) 7, 3) 0, 4) 5.
}
\answer{%
    $13$
}
\solutionspace{40pt}

\tasknumber{6}%
\task{%
    Установите соответствие буквам и запишите в ответ набор цифр (без других символов).

    А) зарядовое число $\beta$-частицы, Б) массовое число $\alpha$-частицы, В) массовое число $\beta$-частицы.

    1) -1, 2) 4, 3) 1, 4) -2, 5) 0.
}
\answer{%
    $125$
}
\solutionspace{40pt}

\tasknumber{7}%
\task{%
    На какой максимальный угол (в градусах) отконялись $\alpha$-частицы
    в опытах Резерфорда по их рассеянию на тонкой золотой фольге?
}
\answer{%
    $180\degrees$
}

\variantsplitter

\addpersonalvariant{Илья Стратонников}

\tasknumber{1}%
\task{%
    Установите соответствие буквам и запишите в ответ набор цифр (без других символов).

    А) $\beta$-излучение, Б) $\alpha$-излучение, В) $\gamma$-излучение.

    1) не несёт электрического заряда, 2) обладает положительным зарядом, 3) обладает отрицательным электрическим зарядом.
}
\answer{%
    $321$
}
\solutionspace{40pt}

\tasknumber{2}%
\task{%
    Установите соответствие буквам и запишите в ответ набор цифр (без других символов).

    А) $\beta$-излучение, Б) $\alpha$-излучение, В) $\gamma$-излучение.

    1) электромагнитное излучение, 2) ядра атомов гелия, 3) электроны.
}
\answer{%
    $321$
}
\solutionspace{40pt}

\tasknumber{3}%
\task{%
    Установите соответствие буквам и запишите в ответ набор цифр (без других символов).

    А) атом Резерфорда, Б) атом Томсона.

    1) планетарная модель атома, 2) «пудинг с изюмом».
}
\answer{%
    $12$
}
\solutionspace{40pt}

\tasknumber{4}%
\task{%
    Установите соответствие буквам и запишите в ответ набор цифр (без других символов).

    А) размер ядра атома, Б) размер атома.

    1) $10^{-10}\units{ см }$, 2) $10^{-8}\units{ см }$, 3) $10^{-13}\units{ см }$.
}
\answer{%
    $32$
}
\solutionspace{40pt}

\tasknumber{5}%
\task{%
    Установите соответствие буквам и запишите в ответ набор цифр (без других символов).

    А) массовое число водорода \ce{^{1}_{1}H}, Б) массовое число углерода \ce{^{12}_{6}C}.

    1) 9, 2) 0, 3) 12, 4) 4.
}
\answer{%
    $23$
}
\solutionspace{40pt}

\tasknumber{6}%
\task{%
    Установите соответствие буквам и запишите в ответ набор цифр (без других символов).

    А) массовое число $\alpha$-частицы, Б) массовое число $\beta$-частицы, В) зарядовое число $\beta$-частицы.

    1) -1, 2) 2, 3) 0, 4) 4, 5) 1.
}
\answer{%
    $431$
}
\solutionspace{40pt}

\tasknumber{7}%
\task{%
    На какой максимальный угол (в градусах) отконялись $\alpha$-частицы
    в опытах Резерфорда по их рассеянию на тонкой золотой фольге?
}
\answer{%
    $180\degrees$
}

\variantsplitter

\addpersonalvariant{Иван Шустов}

\tasknumber{1}%
\task{%
    Установите соответствие буквам и запишите в ответ набор цифр (без других символов).

    А) $\alpha$-излучение, Б) $\beta$-излучение, В) $\gamma$-излучение.

    1) обладает отрицательным электрическим зарядом, 2) обладает положительным зарядом, 3) не несёт электрического заряда.
}
\answer{%
    $213$
}
\solutionspace{40pt}

\tasknumber{2}%
\task{%
    Установите соответствие буквам и запишите в ответ набор цифр (без других символов).

    А) $\alpha$-излучение, Б) $\beta$-излучение, В) $\gamma$-излучение.

    1) электроны, 2) ядра атомов гелия, 3) электромагнитное излучение.
}
\answer{%
    $213$
}
\solutionspace{40pt}

\tasknumber{3}%
\task{%
    Установите соответствие буквам и запишите в ответ набор цифр (без других символов).

    А) атом Томсона, Б) атом Резерфорда.

    1) планетарная модель атома, 2) «пудинг с изюмом».
}
\answer{%
    $21$
}
\solutionspace{40pt}

\tasknumber{4}%
\task{%
    Установите соответствие буквам и запишите в ответ набор цифр (без других символов).

    А) размер атома, Б) размер ядра атома.

    1) $10^{-13}\units{ см }$, 2) $10^{-8}\units{ см }$, 3) $10^{-10}\units{ см }$.
}
\answer{%
    $21$
}
\solutionspace{40pt}

\tasknumber{5}%
\task{%
    Установите соответствие буквам и запишите в ответ набор цифр (без других символов).

    А) массовое число водорода \ce{^{1}_{1}H}, Б) массовое число азота \ce{^{14}_{7}N}.

    1) 0, 2) 11, 3) 14, 4) 5.
}
\answer{%
    $13$
}
\solutionspace{40pt}

\tasknumber{6}%
\task{%
    Установите соответствие буквам и запишите в ответ набор цифр (без других символов).

    А) зарядовое число $\beta$-частицы, Б) массовое число $\alpha$-частицы, В) массовое число $\beta$-частицы.

    1) -1, 2) 0, 3) 2, 4) 4, 5) 1.
}
\answer{%
    $142$
}
\solutionspace{40pt}

\tasknumber{7}%
\task{%
    На какой максимальный угол (в градусах) отконялись $\alpha$-частицы
    в опытах Резерфорда по их рассеянию на тонкой золотой фольге?
}
\answer{%
    $180\degrees$
}
% autogenerated
