\setdate{2~марта~2021}
\setclass{10«АБ»}

\addpersonalvariant{Михаил Бурмистров}

\tasknumber{1}%
\task{%
    Из уравнения состояния идеального газа выведите или выразите...
    \begin{enumerate}
        \item давление,
        \item температуру,
        \item плотность газа.
    \end{enumerate}
}

\tasknumber{2}%
\task{%
    Почему плазму (одно из агрегатных состояний вещества, при котором часть молекул распадаются на ионы и электроны) нельзя считать идеальным газом?
}
\solutionspace{40pt}

\tasknumber{3}%
\task{%
    Изобразите в координатах $PV$/$VT$/$PT$ графики изобарического сжатия в 2 раза (все 3 графика).
    Не забудьте указать оси и масштаб, начальную и конечную точки, направление движения на графике.
}
\solutionspace{100pt}

\tasknumber{4}%
\task{%
    Изобразите в координатах $PV$, соблюдая масштаб, процесс 1234,
    в котором 12 — изохорическое нагревание в 2 раза,
    23 — изотермическое расширение в 3 раза,
    34 — изобарическое нагревание в 3 раза.
}
\solutionspace{160pt}

\tasknumber{5}%
\task{%
    Небольшую цилиндрическую пробирку с воздухом погружают на некоторую глубину в глубокое пресное озеро,
    после чего воздух занимает в ней лишь третью часть от общего объема.
    Определите глубину, на которую погрузили пробирку.
    Температуру считать постоянной $T = 291\,\text{К}$, давлением паров воды пренебречь,
    атмосферное давление принять равным $p_{\text{aтм}} = 100\,\text{кПа}$.
}
\answer{%
    \begin{align*}
    T\text{— const} &\implies P_1V_1 = \nu RT = P_2V_2.
    \\
    V_2 = \frac 13 V_1 &\implies P_1V_1 = P_2 \cdot \frac 13V_1 \implies P_2 = 3P_1 = 3p_{\text{aтм}}.
    \\
    P_2 = p_{\text{aтм}} + \rho_{\text{в}} g h \implies h = \frac{P_2 - p_{\text{aтм}}}{\rho_{\text{в}} g} &= \frac{3p_{\text{aтм}} - p_{\text{aтм}}}{\rho_{\text{в}} g} = \frac{2 \cdot p_{\text{aтм}}}{\rho_{\text{в}} g} =  \\
     &= \frac{2 \cdot 100\,\text{кПа}}{1000\,\frac{\text{кг}}{\text{м}^{3}} \cdot  10\,\frac{\text{м}}{\text{с}^{2}}} \approx 20\,\text{м}.
    \end{align*}
}
\solutionspace{120pt}

\tasknumber{6}%
\task{%
    В замкнутом сосуде объёмом $2\,\text{л}$ находится {gas.Name} ($\mu = 20\,\frac{\text{г}}{\text{моль}}$) под давлением $5\units{атм}$.
    Определите массу газа в сосуде и выразите её в граммах, приняв температуру газа равной $47\celsius$.
}
\answer{%
    $
        PV = \frac m\mu RT \implies m = \frac{PV \mu}{RT} =
        \frac{5{,}0\,\text{атм} \cdot 2\,\text{л} \cdot 20\,\frac{\text{г}}{\text{моль}}}{8{,}31\,\frac{\text{Дж}}{\text{моль}\cdot\text{К}} \cdot \cbr{47 + 273}\units{К}}
        \approx 7{,}52\,\text{г}.
    $
}
\solutionspace{120pt}

\tasknumber{7}%
\task{%
    Идеальный газ в экспериментальной установке подвергут политропному процессу $PV^n\text{— const}$
    с показателем политропы $n=1{,}5$.
    В одном из экспериментов объём газа уменьшился в $3$ раза.
    Как при этом изменилась температура газа (выросла или уменьшилась, на сколько или во сколько раз)?
}
\answer{%
    \begin{align*}
    P_1V_1^n &= P_2V_2^n, P_1V_1 = \nu R T_1, P_2V_2 = \nu R T_2 \implies\frac{\nu R T_1}{V_1} V_1^n = \frac{\nu R T_2}{V_2} V_2^n \implies \\
    \implies T_1V_1^{n-1} &= T_2V_2^{n-1} \implies\frac{T_2}{T_1} = \cbr{\frac{V_1}{V_2}}^{n-1} \approx 1{,}732
    \end{align*}
}

\variantsplitter

\addpersonalvariant{Ирина Ан}

\tasknumber{1}%
\task{%
    Из уравнения состояния идеального газа выведите или выразите...
    \begin{enumerate}
        \item давление,
        \item температуру,
        \item плотность газа.
    \end{enumerate}
}

\tasknumber{2}%
\task{%
    В межзвездном пространстве встречаются молекулы (до нескольких десятков на $\text{{м}}^3$).
    Почему такой газ нельзя считать идеальным?
}
\solutionspace{40pt}

\tasknumber{3}%
\task{%
    Изобразите в координатах $PV$/$VT$/$PT$ графики изохорического нагрева в 2 раза (все 3 графика).
    Не забудьте указать оси и масштаб, начальную и конечную точки, направление движения на графике.
}
\solutionspace{100pt}

\tasknumber{4}%
\task{%
    Изобразите в координатах $PV$, соблюдая масштаб, процесс 1234,
    в котором 12 — изохорическое охлаждение в 2 раза,
    23 — изотермическое расширение в 3 раза,
    34 — изобарическое нагревание в 3 раза.
}
\solutionspace{160pt}

\tasknumber{5}%
\task{%
    Небольшую цилиндрическую пробирку с воздухом погружают на некоторую глубину в глубокое пресное озеро,
    после чего воздух занимает в ней лишь шестую часть от общего объема.
    Определите глубину, на которую погрузили пробирку.
    Температуру считать постоянной $T = 292\,\text{К}$, давлением паров воды пренебречь,
    атмосферное давление принять равным $p_{\text{aтм}} = 100\,\text{кПа}$.
}
\answer{%
    \begin{align*}
    T\text{— const} &\implies P_1V_1 = \nu RT = P_2V_2.
    \\
    V_2 = \frac 16 V_1 &\implies P_1V_1 = P_2 \cdot \frac 16V_1 \implies P_2 = 6P_1 = 6p_{\text{aтм}}.
    \\
    P_2 = p_{\text{aтм}} + \rho_{\text{в}} g h \implies h = \frac{P_2 - p_{\text{aтм}}}{\rho_{\text{в}} g} &= \frac{6p_{\text{aтм}} - p_{\text{aтм}}}{\rho_{\text{в}} g} = \frac{5 \cdot p_{\text{aтм}}}{\rho_{\text{в}} g} =  \\
     &= \frac{5 \cdot 100\,\text{кПа}}{1000\,\frac{\text{кг}}{\text{м}^{3}} \cdot  10\,\frac{\text{м}}{\text{с}^{2}}} \approx 50\,\text{м}.
    \end{align*}
}
\solutionspace{120pt}

\tasknumber{6}%
\task{%
    В замкнутом сосуде объёмом $5\,\text{л}$ находится {gas.Name} ($\mu = 29\,\frac{\text{г}}{\text{моль}}$) под давлением $2{,}5\units{атм}$.
    Определите массу газа в сосуде и выразите её в граммах, приняв температуру газа равной $17\celsius$.
}
\answer{%
    $
        PV = \frac m\mu RT \implies m = \frac{PV \mu}{RT} =
        \frac{2{,}5\,\text{атм} \cdot 5\,\text{л} \cdot 29\,\frac{\text{г}}{\text{моль}}}{8{,}31\,\frac{\text{Дж}}{\text{моль}\cdot\text{К}} \cdot \cbr{17 + 273}\units{К}}
        \approx 15{,}04\,\text{г}.
    $
}
\solutionspace{120pt}

\tasknumber{7}%
\task{%
    Идеальный газ в экспериментальной установке подвергут политропному процессу $PV^n\text{— const}$
    с показателем политропы $n=0{,}7$.
    В одном из экспериментов объём газа уменьшился в $3$ раза.
    Как при этом изменилась температура газа (выросла или уменьшилась, на сколько или во сколько раз)?
}
\answer{%
    \begin{align*}
    P_1V_1^n &= P_2V_2^n, P_1V_1 = \nu R T_1, P_2V_2 = \nu R T_2 \implies\frac{\nu R T_1}{V_1} V_1^n = \frac{\nu R T_2}{V_2} V_2^n \implies \\
    \implies T_1V_1^{n-1} &= T_2V_2^{n-1} \implies\frac{T_2}{T_1} = \cbr{\frac{V_1}{V_2}}^{n-1} \approx 0{,}719
    \end{align*}
}

\variantsplitter

\addpersonalvariant{Софья Андрианова}

\tasknumber{1}%
\task{%
    Из уравнения состояния идеального газа выведите или выразите...
    \begin{enumerate}
        \item давление,
        \item температуру,
        \item плотность газа.
    \end{enumerate}
}

\tasknumber{2}%
\task{%
    Почему плазму (одно из агрегатных состояний вещества, при котором часть молекул распадаются на ионы и электроны) нельзя считать идеальным газом?
}
\solutionspace{40pt}

\tasknumber{3}%
\task{%
    Изобразите в координатах $PV$/$VT$/$PT$ графики изобарического сжатия в 2 раза (все 3 графика).
    Не забудьте указать оси и масштаб, начальную и конечную точки, направление движения на графике.
}
\solutionspace{100pt}

\tasknumber{4}%
\task{%
    Изобразите в координатах $PV$, соблюдая масштаб, процесс 1234,
    в котором 12 — изобарическое охлаждение в 3 раза,
    23 — изотермическое расширение в 2 раза,
    34 — изобарическое нагревание в 3 раза.
}
\solutionspace{160pt}

\tasknumber{5}%
\task{%
    Небольшую цилиндрическую пробирку с воздухом погружают на некоторую глубину в глубокое пресное озеро,
    после чего воздух занимает в ней лишь шестую часть от общего объема.
    Определите глубину, на которую погрузили пробирку.
    Температуру считать постоянной $T = 281\,\text{К}$, давлением паров воды пренебречь,
    атмосферное давление принять равным $p_{\text{aтм}} = 100\,\text{кПа}$.
}
\answer{%
    \begin{align*}
    T\text{— const} &\implies P_1V_1 = \nu RT = P_2V_2.
    \\
    V_2 = \frac 16 V_1 &\implies P_1V_1 = P_2 \cdot \frac 16V_1 \implies P_2 = 6P_1 = 6p_{\text{aтм}}.
    \\
    P_2 = p_{\text{aтм}} + \rho_{\text{в}} g h \implies h = \frac{P_2 - p_{\text{aтм}}}{\rho_{\text{в}} g} &= \frac{6p_{\text{aтм}} - p_{\text{aтм}}}{\rho_{\text{в}} g} = \frac{5 \cdot p_{\text{aтм}}}{\rho_{\text{в}} g} =  \\
     &= \frac{5 \cdot 100\,\text{кПа}}{1000\,\frac{\text{кг}}{\text{м}^{3}} \cdot  10\,\frac{\text{м}}{\text{с}^{2}}} \approx 50\,\text{м}.
    \end{align*}
}
\solutionspace{120pt}

\tasknumber{6}%
\task{%
    В замкнутом сосуде объёмом $2\,\text{л}$ находится {gas.Name} ($\mu = 32\,\frac{\text{г}}{\text{моль}}$) под давлением $4\units{атм}$.
    Определите массу газа в сосуде и выразите её в граммах, приняв температуру газа равной $17\celsius$.
}
\answer{%
    $
        PV = \frac m\mu RT \implies m = \frac{PV \mu}{RT} =
        \frac{4{,}0\,\text{атм} \cdot 2\,\text{л} \cdot 32\,\frac{\text{г}}{\text{моль}}}{8{,}31\,\frac{\text{Дж}}{\text{моль}\cdot\text{К}} \cdot \cbr{17 + 273}\units{К}}
        \approx 10{,}62\,\text{г}.
    $
}
\solutionspace{120pt}

\tasknumber{7}%
\task{%
    Идеальный газ в экспериментальной установке подвергут политропному процессу $PV^n\text{— const}$
    с показателем политропы $n=0{,}4$.
    В одном из экспериментов объём газа увеличился в $4$ раза.
    Как при этом изменилась температура газа (выросла или уменьшилась, на сколько или во сколько раз)?
}
\answer{%
    \begin{align*}
    P_1V_1^n &= P_2V_2^n, P_1V_1 = \nu R T_1, P_2V_2 = \nu R T_2 \implies\frac{\nu R T_1}{V_1} V_1^n = \frac{\nu R T_2}{V_2} V_2^n \implies \\
    \implies T_1V_1^{n-1} &= T_2V_2^{n-1} \implies\frac{T_2}{T_1} = \cbr{\frac{V_1}{V_2}}^{n-1} \approx 2{,}297
    \end{align*}
}

\variantsplitter

\addpersonalvariant{Владимир Артемчук}

\tasknumber{1}%
\task{%
    Из уравнения состояния идеального газа выведите или выразите...
    \begin{enumerate}
        \item объём,
        \item температуру,
        \item плотность газа.
    \end{enumerate}
}

\tasknumber{2}%
\task{%
    В межзвездном пространстве встречаются молекулы (до нескольких десятков на $\text{{м}}^3$).
    Почему такой газ нельзя считать идеальным?
}
\solutionspace{40pt}

\tasknumber{3}%
\task{%
    Изобразите в координатах $PV$/$VT$/$PT$ графики изотермического повышения давления в 4 раза (все 3 графика).
    Не забудьте указать оси и масштаб, начальную и конечную точки, направление движения на графике.
}
\solutionspace{100pt}

\tasknumber{4}%
\task{%
    Изобразите в координатах $PV$, соблюдая масштаб, процесс 1234,
    в котором 12 — изобарическое нагревание в 2 раза,
    23 — изотермическое сжатие в 3 раза,
    34 — изохорическое нагревание в 2 раза.
}
\solutionspace{160pt}

\tasknumber{5}%
\task{%
    Небольшую цилиндрическую пробирку с воздухом погружают на некоторую глубину в глубокое пресное озеро,
    после чего воздух занимает в ней лишь четвертую часть от общего объема.
    Определите глубину, на которую погрузили пробирку.
    Температуру считать постоянной $T = 288\,\text{К}$, давлением паров воды пренебречь,
    атмосферное давление принять равным $p_{\text{aтм}} = 100\,\text{кПа}$.
}
\answer{%
    \begin{align*}
    T\text{— const} &\implies P_1V_1 = \nu RT = P_2V_2.
    \\
    V_2 = \frac 14 V_1 &\implies P_1V_1 = P_2 \cdot \frac 14V_1 \implies P_2 = 4P_1 = 4p_{\text{aтм}}.
    \\
    P_2 = p_{\text{aтм}} + \rho_{\text{в}} g h \implies h = \frac{P_2 - p_{\text{aтм}}}{\rho_{\text{в}} g} &= \frac{4p_{\text{aтм}} - p_{\text{aтм}}}{\rho_{\text{в}} g} = \frac{3 \cdot p_{\text{aтм}}}{\rho_{\text{в}} g} =  \\
     &= \frac{3 \cdot 100\,\text{кПа}}{1000\,\frac{\text{кг}}{\text{м}^{3}} \cdot  10\,\frac{\text{м}}{\text{с}^{2}}} \approx 30\,\text{м}.
    \end{align*}
}
\solutionspace{120pt}

\tasknumber{6}%
\task{%
    В замкнутом сосуде объёмом $2\,\text{л}$ находится {gas.Name} ($\mu = 40\,\frac{\text{г}}{\text{моль}}$) под давлением $4{,}5\units{атм}$.
    Определите массу газа в сосуде и выразите её в граммах, приняв температуру газа равной $27\celsius$.
}
\answer{%
    $
        PV = \frac m\mu RT \implies m = \frac{PV \mu}{RT} =
        \frac{4{,}5\,\text{атм} \cdot 2\,\text{л} \cdot 40\,\frac{\text{г}}{\text{моль}}}{8{,}31\,\frac{\text{Дж}}{\text{моль}\cdot\text{К}} \cdot \cbr{27 + 273}\units{К}}
        \approx 14{,}44\,\text{г}.
    $
}
\solutionspace{120pt}

\tasknumber{7}%
\task{%
    Идеальный газ в экспериментальной установке подвергут политропному процессу $PV^n\text{— const}$
    с показателем политропы $n=0{,}7$.
    В одном из экспериментов объём газа увеличился в $3$ раза.
    Как при этом изменилась температура газа (выросла или уменьшилась, на сколько или во сколько раз)?
}
\answer{%
    \begin{align*}
    P_1V_1^n &= P_2V_2^n, P_1V_1 = \nu R T_1, P_2V_2 = \nu R T_2 \implies\frac{\nu R T_1}{V_1} V_1^n = \frac{\nu R T_2}{V_2} V_2^n \implies \\
    \implies T_1V_1^{n-1} &= T_2V_2^{n-1} \implies\frac{T_2}{T_1} = \cbr{\frac{V_1}{V_2}}^{n-1} \approx 1{,}390
    \end{align*}
}

\variantsplitter

\addpersonalvariant{Софья Белянкина}

\tasknumber{1}%
\task{%
    Из уравнения состояния идеального газа выведите или выразите...
    \begin{enumerate}
        \item давление,
        \item температуру,
        \item плотность газа.
    \end{enumerate}
}

\tasknumber{2}%
\task{%
    Почему плазму (одно из агрегатных состояний вещества, при котором часть молекул распадаются на ионы и электроны) нельзя считать идеальным газом?
}
\solutionspace{40pt}

\tasknumber{3}%
\task{%
    Изобразите в координатах $PV$/$VT$/$PT$ графики изотермического понижения давления в 3 раза (все 3 графика).
    Не забудьте указать оси и масштаб, начальную и конечную точки, направление движения на графике.
}
\solutionspace{100pt}

\tasknumber{4}%
\task{%
    Изобразите в координатах $PV$, соблюдая масштаб, процесс 1234,
    в котором 12 — изобарическое нагревание в 3 раза,
    23 — изотермическое расширение в 2 раза,
    34 — изобарическое нагревание в 3 раза.
}
\solutionspace{160pt}

\tasknumber{5}%
\task{%
    Небольшую цилиндрическую пробирку с воздухом погружают на некоторую глубину в глубокое пресное озеро,
    после чего воздух занимает в ней лишь пятую часть от общего объема.
    Определите глубину, на которую погрузили пробирку.
    Температуру считать постоянной $T = 283\,\text{К}$, давлением паров воды пренебречь,
    атмосферное давление принять равным $p_{\text{aтм}} = 100\,\text{кПа}$.
}
\answer{%
    \begin{align*}
    T\text{— const} &\implies P_1V_1 = \nu RT = P_2V_2.
    \\
    V_2 = \frac 15 V_1 &\implies P_1V_1 = P_2 \cdot \frac 15V_1 \implies P_2 = 5P_1 = 5p_{\text{aтм}}.
    \\
    P_2 = p_{\text{aтм}} + \rho_{\text{в}} g h \implies h = \frac{P_2 - p_{\text{aтм}}}{\rho_{\text{в}} g} &= \frac{5p_{\text{aтм}} - p_{\text{aтм}}}{\rho_{\text{в}} g} = \frac{4 \cdot p_{\text{aтм}}}{\rho_{\text{в}} g} =  \\
     &= \frac{4 \cdot 100\,\text{кПа}}{1000\,\frac{\text{кг}}{\text{м}^{3}} \cdot  10\,\frac{\text{м}}{\text{с}^{2}}} \approx 40\,\text{м}.
    \end{align*}
}
\solutionspace{120pt}

\tasknumber{6}%
\task{%
    В замкнутом сосуде объёмом $5\,\text{л}$ находится {gas.Name} ($\mu = 20\,\frac{\text{г}}{\text{моль}}$) под давлением $4\units{атм}$.
    Определите массу газа в сосуде и выразите её в граммах, приняв температуру газа равной $27\celsius$.
}
\answer{%
    $
        PV = \frac m\mu RT \implies m = \frac{PV \mu}{RT} =
        \frac{4{,}0\,\text{атм} \cdot 5\,\text{л} \cdot 20\,\frac{\text{г}}{\text{моль}}}{8{,}31\,\frac{\text{Дж}}{\text{моль}\cdot\text{К}} \cdot \cbr{27 + 273}\units{К}}
        \approx 16{,}04\,\text{г}.
    $
}
\solutionspace{120pt}

\tasknumber{7}%
\task{%
    Идеальный газ в экспериментальной установке подвергут политропному процессу $PV^n\text{— const}$
    с показателем политропы $n=0{,}4$.
    В одном из экспериментов объём газа уменьшился в $2$ раза.
    Как при этом изменилась температура газа (выросла или уменьшилась, на сколько или во сколько раз)?
}
\answer{%
    \begin{align*}
    P_1V_1^n &= P_2V_2^n, P_1V_1 = \nu R T_1, P_2V_2 = \nu R T_2 \implies\frac{\nu R T_1}{V_1} V_1^n = \frac{\nu R T_2}{V_2} V_2^n \implies \\
    \implies T_1V_1^{n-1} &= T_2V_2^{n-1} \implies\frac{T_2}{T_1} = \cbr{\frac{V_1}{V_2}}^{n-1} \approx 0{,}660
    \end{align*}
}

\variantsplitter

\addpersonalvariant{Варвара Егиазарян}

\tasknumber{1}%
\task{%
    Из уравнения состояния идеального газа выведите или выразите...
    \begin{enumerate}
        \item объём,
        \item температуру,
        \item плотность газа.
    \end{enumerate}
}

\tasknumber{2}%
\task{%
    В межзвездном пространстве встречаются молекулы (до нескольких десятков на $\text{{м}}^3$).
    Почему такой газ нельзя считать идеальным?
}
\solutionspace{40pt}

\tasknumber{3}%
\task{%
    Изобразите в координатах $PV$/$VT$/$PT$ графики изотермического понижения давления в 4 раза (все 3 графика).
    Не забудьте указать оси и масштаб, начальную и конечную точки, направление движения на графике.
}
\solutionspace{100pt}

\tasknumber{4}%
\task{%
    Изобразите в координатах $PV$, соблюдая масштаб, процесс 1234,
    в котором 12 — изохорическое нагревание в 3 раза,
    23 — изотермическое сжатие в 2 раза,
    34 — изобарическое охлаждение в 3 раза.
}
\solutionspace{160pt}

\tasknumber{5}%
\task{%
    Небольшую цилиндрическую пробирку с воздухом погружают на некоторую глубину в глубокое пресное озеро,
    после чего воздух занимает в ней лишь пятую часть от общего объема.
    Определите глубину, на которую погрузили пробирку.
    Температуру считать постоянной $T = 278\,\text{К}$, давлением паров воды пренебречь,
    атмосферное давление принять равным $p_{\text{aтм}} = 100\,\text{кПа}$.
}
\answer{%
    \begin{align*}
    T\text{— const} &\implies P_1V_1 = \nu RT = P_2V_2.
    \\
    V_2 = \frac 15 V_1 &\implies P_1V_1 = P_2 \cdot \frac 15V_1 \implies P_2 = 5P_1 = 5p_{\text{aтм}}.
    \\
    P_2 = p_{\text{aтм}} + \rho_{\text{в}} g h \implies h = \frac{P_2 - p_{\text{aтм}}}{\rho_{\text{в}} g} &= \frac{5p_{\text{aтм}} - p_{\text{aтм}}}{\rho_{\text{в}} g} = \frac{4 \cdot p_{\text{aтм}}}{\rho_{\text{в}} g} =  \\
     &= \frac{4 \cdot 100\,\text{кПа}}{1000\,\frac{\text{кг}}{\text{м}^{3}} \cdot  10\,\frac{\text{м}}{\text{с}^{2}}} \approx 40\,\text{м}.
    \end{align*}
}
\solutionspace{120pt}

\tasknumber{6}%
\task{%
    В замкнутом сосуде объёмом $2\,\text{л}$ находится {gas.Name} ($\mu = 20\,\frac{\text{г}}{\text{моль}}$) под давлением $4\units{атм}$.
    Определите массу газа в сосуде и выразите её в граммах, приняв температуру газа равной $17\celsius$.
}
\answer{%
    $
        PV = \frac m\mu RT \implies m = \frac{PV \mu}{RT} =
        \frac{4{,}0\,\text{атм} \cdot 2\,\text{л} \cdot 20\,\frac{\text{г}}{\text{моль}}}{8{,}31\,\frac{\text{Дж}}{\text{моль}\cdot\text{К}} \cdot \cbr{17 + 273}\units{К}}
        \approx 6{,}64\,\text{г}.
    $
}
\solutionspace{120pt}

\tasknumber{7}%
\task{%
    Идеальный газ в экспериментальной установке подвергут политропному процессу $PV^n\text{— const}$
    с показателем политропы $n=1{,}8$.
    В одном из экспериментов объём газа уменьшился в $3$ раза.
    Как при этом изменилась температура газа (выросла или уменьшилась, на сколько или во сколько раз)?
}
\answer{%
    \begin{align*}
    P_1V_1^n &= P_2V_2^n, P_1V_1 = \nu R T_1, P_2V_2 = \nu R T_2 \implies\frac{\nu R T_1}{V_1} V_1^n = \frac{\nu R T_2}{V_2} V_2^n \implies \\
    \implies T_1V_1^{n-1} &= T_2V_2^{n-1} \implies\frac{T_2}{T_1} = \cbr{\frac{V_1}{V_2}}^{n-1} \approx 2{,}408
    \end{align*}
}

\variantsplitter

\addpersonalvariant{Владислав Емелин}

\tasknumber{1}%
\task{%
    Из уравнения состояния идеального газа выведите или выразите...
    \begin{enumerate}
        \item давление,
        \item температуру,
        \item концентрацию молекул газа.
    \end{enumerate}
}

\tasknumber{2}%
\task{%
    В межзвездном пространстве встречаются молекулы (до нескольких десятков на $\text{{м}}^3$).
    Почему такой газ нельзя считать идеальным?
}
\solutionspace{40pt}

\tasknumber{3}%
\task{%
    Изобразите в координатах $PV$/$VT$/$PT$ графики изотермического понижения давления в 4 раза (все 3 графика).
    Не забудьте указать оси и масштаб, начальную и конечную точки, направление движения на графике.
}
\solutionspace{100pt}

\tasknumber{4}%
\task{%
    Изобразите в координатах $PV$, соблюдая масштаб, процесс 1234,
    в котором 12 — изобарическое нагревание в 3 раза,
    23 — изотермическое расширение в 2 раза,
    34 — изобарическое нагревание в 2 раза.
}
\solutionspace{160pt}

\tasknumber{5}%
\task{%
    Небольшую цилиндрическую пробирку с воздухом погружают на некоторую глубину в глубокое пресное озеро,
    после чего воздух занимает в ней лишь шестую часть от общего объема.
    Определите глубину, на которую погрузили пробирку.
    Температуру считать постоянной $T = 285\,\text{К}$, давлением паров воды пренебречь,
    атмосферное давление принять равным $p_{\text{aтм}} = 100\,\text{кПа}$.
}
\answer{%
    \begin{align*}
    T\text{— const} &\implies P_1V_1 = \nu RT = P_2V_2.
    \\
    V_2 = \frac 16 V_1 &\implies P_1V_1 = P_2 \cdot \frac 16V_1 \implies P_2 = 6P_1 = 6p_{\text{aтм}}.
    \\
    P_2 = p_{\text{aтм}} + \rho_{\text{в}} g h \implies h = \frac{P_2 - p_{\text{aтм}}}{\rho_{\text{в}} g} &= \frac{6p_{\text{aтм}} - p_{\text{aтм}}}{\rho_{\text{в}} g} = \frac{5 \cdot p_{\text{aтм}}}{\rho_{\text{в}} g} =  \\
     &= \frac{5 \cdot 100\,\text{кПа}}{1000\,\frac{\text{кг}}{\text{м}^{3}} \cdot  10\,\frac{\text{м}}{\text{с}^{2}}} \approx 50\,\text{м}.
    \end{align*}
}
\solutionspace{120pt}

\tasknumber{6}%
\task{%
    В замкнутом сосуде объёмом $4\,\text{л}$ находится {gas.Name} ($\mu = 29\,\frac{\text{г}}{\text{моль}}$) под давлением $5\units{атм}$.
    Определите массу газа в сосуде и выразите её в граммах, приняв температуру газа равной $27\celsius$.
}
\answer{%
    $
        PV = \frac m\mu RT \implies m = \frac{PV \mu}{RT} =
        \frac{5{,}0\,\text{атм} \cdot 4\,\text{л} \cdot 29\,\frac{\text{г}}{\text{моль}}}{8{,}31\,\frac{\text{Дж}}{\text{моль}\cdot\text{К}} \cdot \cbr{27 + 273}\units{К}}
        \approx 23{,}27\,\text{г}.
    $
}
\solutionspace{120pt}

\tasknumber{7}%
\task{%
    Идеальный газ в экспериментальной установке подвергут политропному процессу $PV^n\text{— const}$
    с показателем политропы $n=1{,}2$.
    В одном из экспериментов объём газа уменьшился в $3$ раза.
    Как при этом изменилась температура газа (выросла или уменьшилась, на сколько или во сколько раз)?
}
\answer{%
    \begin{align*}
    P_1V_1^n &= P_2V_2^n, P_1V_1 = \nu R T_1, P_2V_2 = \nu R T_2 \implies\frac{\nu R T_1}{V_1} V_1^n = \frac{\nu R T_2}{V_2} V_2^n \implies \\
    \implies T_1V_1^{n-1} &= T_2V_2^{n-1} \implies\frac{T_2}{T_1} = \cbr{\frac{V_1}{V_2}}^{n-1} \approx 1{,}246
    \end{align*}
}

\variantsplitter

\addpersonalvariant{Артём Жичин}

\tasknumber{1}%
\task{%
    Из уравнения состояния идеального газа выведите или выразите...
    \begin{enumerate}
        \item объём,
        \item температуру,
        \item плотность газа.
    \end{enumerate}
}

\tasknumber{2}%
\task{%
    Почему плазму (одно из агрегатных состояний вещества, при котором часть молекул распадаются на ионы и электроны) нельзя считать идеальным газом?
}
\solutionspace{40pt}

\tasknumber{3}%
\task{%
    Изобразите в координатах $PV$/$VT$/$PT$ графики изохорического нагрева в 3 раза (все 3 графика).
    Не забудьте указать оси и масштаб, начальную и конечную точки, направление движения на графике.
}
\solutionspace{100pt}

\tasknumber{4}%
\task{%
    Изобразите в координатах $PV$, соблюдая масштаб, процесс 1234,
    в котором 12 — изохорическое нагревание в 3 раза,
    23 — изотермическое расширение в 3 раза,
    34 — изохорическое нагревание в 3 раза.
}
\solutionspace{160pt}

\tasknumber{5}%
\task{%
    Небольшую цилиндрическую пробирку с воздухом погружают на некоторую глубину в глубокое пресное озеро,
    после чего воздух занимает в ней лишь четвертую часть от общего объема.
    Определите глубину, на которую погрузили пробирку.
    Температуру считать постоянной $T = 289\,\text{К}$, давлением паров воды пренебречь,
    атмосферное давление принять равным $p_{\text{aтм}} = 100\,\text{кПа}$.
}
\answer{%
    \begin{align*}
    T\text{— const} &\implies P_1V_1 = \nu RT = P_2V_2.
    \\
    V_2 = \frac 14 V_1 &\implies P_1V_1 = P_2 \cdot \frac 14V_1 \implies P_2 = 4P_1 = 4p_{\text{aтм}}.
    \\
    P_2 = p_{\text{aтм}} + \rho_{\text{в}} g h \implies h = \frac{P_2 - p_{\text{aтм}}}{\rho_{\text{в}} g} &= \frac{4p_{\text{aтм}} - p_{\text{aтм}}}{\rho_{\text{в}} g} = \frac{3 \cdot p_{\text{aтм}}}{\rho_{\text{в}} g} =  \\
     &= \frac{3 \cdot 100\,\text{кПа}}{1000\,\frac{\text{кг}}{\text{м}^{3}} \cdot  10\,\frac{\text{м}}{\text{с}^{2}}} \approx 30\,\text{м}.
    \end{align*}
}
\solutionspace{120pt}

\tasknumber{6}%
\task{%
    В замкнутом сосуде объёмом $2\,\text{л}$ находится {gas.Name} ($\mu = 20\,\frac{\text{г}}{\text{моль}}$) под давлением $4{,}5\units{атм}$.
    Определите массу газа в сосуде и выразите её в граммах, приняв температуру газа равной $17\celsius$.
}
\answer{%
    $
        PV = \frac m\mu RT \implies m = \frac{PV \mu}{RT} =
        \frac{4{,}5\,\text{атм} \cdot 2\,\text{л} \cdot 20\,\frac{\text{г}}{\text{моль}}}{8{,}31\,\frac{\text{Дж}}{\text{моль}\cdot\text{К}} \cdot \cbr{17 + 273}\units{К}}
        \approx 7{,}47\,\text{г}.
    $
}
\solutionspace{120pt}

\tasknumber{7}%
\task{%
    Идеальный газ в экспериментальной установке подвергут политропному процессу $PV^n\text{— const}$
    с показателем политропы $n=1{,}5$.
    В одном из экспериментов объём газа увеличился в $3$ раза.
    Как при этом изменилась температура газа (выросла или уменьшилась, на сколько или во сколько раз)?
}
\answer{%
    \begin{align*}
    P_1V_1^n &= P_2V_2^n, P_1V_1 = \nu R T_1, P_2V_2 = \nu R T_2 \implies\frac{\nu R T_1}{V_1} V_1^n = \frac{\nu R T_2}{V_2} V_2^n \implies \\
    \implies T_1V_1^{n-1} &= T_2V_2^{n-1} \implies\frac{T_2}{T_1} = \cbr{\frac{V_1}{V_2}}^{n-1} \approx 0{,}577
    \end{align*}
}

\variantsplitter

\addpersonalvariant{Дарья Кошман}

\tasknumber{1}%
\task{%
    Из уравнения состояния идеального газа выведите или выразите...
    \begin{enumerate}
        \item давление,
        \item молярную массу,
        \item концентрацию молекул газа.
    \end{enumerate}
}

\tasknumber{2}%
\task{%
    В межзвездном пространстве встречаются молекулы (до нескольких десятков на $\text{{м}}^3$).
    Почему такой газ нельзя считать идеальным?
}
\solutionspace{40pt}

\tasknumber{3}%
\task{%
    Изобразите в координатах $PV$/$VT$/$PT$ графики изобарического сжатия в 3 раза (все 3 графика).
    Не забудьте указать оси и масштаб, начальную и конечную точки, направление движения на графике.
}
\solutionspace{100pt}

\tasknumber{4}%
\task{%
    Изобразите в координатах $PV$, соблюдая масштаб, процесс 1234,
    в котором 12 — изобарическое нагревание в 3 раза,
    23 — изотермическое расширение в 3 раза,
    34 — изобарическое нагревание в 3 раза.
}
\solutionspace{160pt}

\tasknumber{5}%
\task{%
    Небольшую цилиндрическую пробирку с воздухом погружают на некоторую глубину в глубокое пресное озеро,
    после чего воздух занимает в ней лишь третью часть от общего объема.
    Определите глубину, на которую погрузили пробирку.
    Температуру считать постоянной $T = 278\,\text{К}$, давлением паров воды пренебречь,
    атмосферное давление принять равным $p_{\text{aтм}} = 100\,\text{кПа}$.
}
\answer{%
    \begin{align*}
    T\text{— const} &\implies P_1V_1 = \nu RT = P_2V_2.
    \\
    V_2 = \frac 13 V_1 &\implies P_1V_1 = P_2 \cdot \frac 13V_1 \implies P_2 = 3P_1 = 3p_{\text{aтм}}.
    \\
    P_2 = p_{\text{aтм}} + \rho_{\text{в}} g h \implies h = \frac{P_2 - p_{\text{aтм}}}{\rho_{\text{в}} g} &= \frac{3p_{\text{aтм}} - p_{\text{aтм}}}{\rho_{\text{в}} g} = \frac{2 \cdot p_{\text{aтм}}}{\rho_{\text{в}} g} =  \\
     &= \frac{2 \cdot 100\,\text{кПа}}{1000\,\frac{\text{кг}}{\text{м}^{3}} \cdot  10\,\frac{\text{м}}{\text{с}^{2}}} \approx 20\,\text{м}.
    \end{align*}
}
\solutionspace{120pt}

\tasknumber{6}%
\task{%
    В замкнутом сосуде объёмом $5\,\text{л}$ находится {gas.Name} ($\mu = 32\,\frac{\text{г}}{\text{моль}}$) под давлением $3\units{атм}$.
    Определите массу газа в сосуде и выразите её в граммах, приняв температуру газа равной $47\celsius$.
}
\answer{%
    $
        PV = \frac m\mu RT \implies m = \frac{PV \mu}{RT} =
        \frac{3{,}0\,\text{атм} \cdot 5\,\text{л} \cdot 32\,\frac{\text{г}}{\text{моль}}}{8{,}31\,\frac{\text{Дж}}{\text{моль}\cdot\text{К}} \cdot \cbr{47 + 273}\units{К}}
        \approx 18{,}05\,\text{г}.
    $
}
\solutionspace{120pt}

\tasknumber{7}%
\task{%
    Идеальный газ в экспериментальной установке подвергут политропному процессу $PV^n\text{— const}$
    с показателем политропы $n=1{,}2$.
    В одном из экспериментов объём газа увеличился в $2$ раза.
    Как при этом изменилась температура газа (выросла или уменьшилась, на сколько или во сколько раз)?
}
\answer{%
    \begin{align*}
    P_1V_1^n &= P_2V_2^n, P_1V_1 = \nu R T_1, P_2V_2 = \nu R T_2 \implies\frac{\nu R T_1}{V_1} V_1^n = \frac{\nu R T_2}{V_2} V_2^n \implies \\
    \implies T_1V_1^{n-1} &= T_2V_2^{n-1} \implies\frac{T_2}{T_1} = \cbr{\frac{V_1}{V_2}}^{n-1} \approx 0{,}871
    \end{align*}
}

\variantsplitter

\addpersonalvariant{Анна Кузьмичёва}

\tasknumber{1}%
\task{%
    Из уравнения состояния идеального газа выведите или выразите...
    \begin{enumerate}
        \item объём,
        \item молярную массу,
        \item концентрацию молекул газа.
    \end{enumerate}
}

\tasknumber{2}%
\task{%
    Почему плазму (одно из агрегатных состояний вещества, при котором часть молекул распадаются на ионы и электроны) нельзя считать идеальным газом?
}
\solutionspace{40pt}

\tasknumber{3}%
\task{%
    Изобразите в координатах $PV$/$VT$/$PT$ графики изотермического повышения давления в 4 раза (все 3 графика).
    Не забудьте указать оси и масштаб, начальную и конечную точки, направление движения на графике.
}
\solutionspace{100pt}

\tasknumber{4}%
\task{%
    Изобразите в координатах $PV$, соблюдая масштаб, процесс 1234,
    в котором 12 — изохорическое охлаждение в 3 раза,
    23 — изотермическое расширение в 3 раза,
    34 — изобарическое нагревание в 2 раза.
}
\solutionspace{160pt}

\tasknumber{5}%
\task{%
    Небольшую цилиндрическую пробирку с воздухом погружают на некоторую глубину в глубокое пресное озеро,
    после чего воздух занимает в ней лишь шестую часть от общего объема.
    Определите глубину, на которую погрузили пробирку.
    Температуру считать постоянной $T = 289\,\text{К}$, давлением паров воды пренебречь,
    атмосферное давление принять равным $p_{\text{aтм}} = 100\,\text{кПа}$.
}
\answer{%
    \begin{align*}
    T\text{— const} &\implies P_1V_1 = \nu RT = P_2V_2.
    \\
    V_2 = \frac 16 V_1 &\implies P_1V_1 = P_2 \cdot \frac 16V_1 \implies P_2 = 6P_1 = 6p_{\text{aтм}}.
    \\
    P_2 = p_{\text{aтм}} + \rho_{\text{в}} g h \implies h = \frac{P_2 - p_{\text{aтм}}}{\rho_{\text{в}} g} &= \frac{6p_{\text{aтм}} - p_{\text{aтм}}}{\rho_{\text{в}} g} = \frac{5 \cdot p_{\text{aтм}}}{\rho_{\text{в}} g} =  \\
     &= \frac{5 \cdot 100\,\text{кПа}}{1000\,\frac{\text{кг}}{\text{м}^{3}} \cdot  10\,\frac{\text{м}}{\text{с}^{2}}} \approx 50\,\text{м}.
    \end{align*}
}
\solutionspace{120pt}

\tasknumber{6}%
\task{%
    В замкнутом сосуде объёмом $5\,\text{л}$ находится {gas.Name} ($\mu = 40\,\frac{\text{г}}{\text{моль}}$) под давлением $3\units{атм}$.
    Определите массу газа в сосуде и выразите её в граммах, приняв температуру газа равной $47\celsius$.
}
\answer{%
    $
        PV = \frac m\mu RT \implies m = \frac{PV \mu}{RT} =
        \frac{3{,}0\,\text{атм} \cdot 5\,\text{л} \cdot 40\,\frac{\text{г}}{\text{моль}}}{8{,}31\,\frac{\text{Дж}}{\text{моль}\cdot\text{К}} \cdot \cbr{47 + 273}\units{К}}
        \approx 22{,}56\,\text{г}.
    $
}
\solutionspace{120pt}

\tasknumber{7}%
\task{%
    Идеальный газ в экспериментальной установке подвергут политропному процессу $PV^n\text{— const}$
    с показателем политропы $n=1{,}2$.
    В одном из экспериментов объём газа увеличился в $2$ раза.
    Как при этом изменилась температура газа (выросла или уменьшилась, на сколько или во сколько раз)?
}
\answer{%
    \begin{align*}
    P_1V_1^n &= P_2V_2^n, P_1V_1 = \nu R T_1, P_2V_2 = \nu R T_2 \implies\frac{\nu R T_1}{V_1} V_1^n = \frac{\nu R T_2}{V_2} V_2^n \implies \\
    \implies T_1V_1^{n-1} &= T_2V_2^{n-1} \implies\frac{T_2}{T_1} = \cbr{\frac{V_1}{V_2}}^{n-1} \approx 0{,}871
    \end{align*}
}

\variantsplitter

\addpersonalvariant{Алёна Куприянова}

\tasknumber{1}%
\task{%
    Из уравнения состояния идеального газа выведите или выразите...
    \begin{enumerate}
        \item объём,
        \item молярную массу,
        \item плотность газа.
    \end{enumerate}
}

\tasknumber{2}%
\task{%
    Почему плазму (одно из агрегатных состояний вещества, при котором часть молекул распадаются на ионы и электроны) нельзя считать идеальным газом?
}
\solutionspace{40pt}

\tasknumber{3}%
\task{%
    Изобразите в координатах $PV$/$VT$/$PT$ графики изохорического нагрева в 4 раза (все 3 графика).
    Не забудьте указать оси и масштаб, начальную и конечную точки, направление движения на графике.
}
\solutionspace{100pt}

\tasknumber{4}%
\task{%
    Изобразите в координатах $PV$, соблюдая масштаб, процесс 1234,
    в котором 12 — изобарическое охлаждение в 3 раза,
    23 — изотермическое расширение в 2 раза,
    34 — изобарическое нагревание в 2 раза.
}
\solutionspace{160pt}

\tasknumber{5}%
\task{%
    Небольшую цилиндрическую пробирку с воздухом погружают на некоторую глубину в глубокое пресное озеро,
    после чего воздух занимает в ней лишь шестую часть от общего объема.
    Определите глубину, на которую погрузили пробирку.
    Температуру считать постоянной $T = 291\,\text{К}$, давлением паров воды пренебречь,
    атмосферное давление принять равным $p_{\text{aтм}} = 100\,\text{кПа}$.
}
\answer{%
    \begin{align*}
    T\text{— const} &\implies P_1V_1 = \nu RT = P_2V_2.
    \\
    V_2 = \frac 16 V_1 &\implies P_1V_1 = P_2 \cdot \frac 16V_1 \implies P_2 = 6P_1 = 6p_{\text{aтм}}.
    \\
    P_2 = p_{\text{aтм}} + \rho_{\text{в}} g h \implies h = \frac{P_2 - p_{\text{aтм}}}{\rho_{\text{в}} g} &= \frac{6p_{\text{aтм}} - p_{\text{aтм}}}{\rho_{\text{в}} g} = \frac{5 \cdot p_{\text{aтм}}}{\rho_{\text{в}} g} =  \\
     &= \frac{5 \cdot 100\,\text{кПа}}{1000\,\frac{\text{кг}}{\text{м}^{3}} \cdot  10\,\frac{\text{м}}{\text{с}^{2}}} \approx 50\,\text{м}.
    \end{align*}
}
\solutionspace{120pt}

\tasknumber{6}%
\task{%
    В замкнутом сосуде объёмом $3\,\text{л}$ находится {gas.Name} ($\mu = 32\,\frac{\text{г}}{\text{моль}}$) под давлением $3\units{атм}$.
    Определите массу газа в сосуде и выразите её в граммах, приняв температуру газа равной $7\celsius$.
}
\answer{%
    $
        PV = \frac m\mu RT \implies m = \frac{PV \mu}{RT} =
        \frac{3{,}0\,\text{атм} \cdot 3\,\text{л} \cdot 32\,\frac{\text{г}}{\text{моль}}}{8{,}31\,\frac{\text{Дж}}{\text{моль}\cdot\text{К}} \cdot \cbr{7 + 273}\units{К}}
        \approx 12{,}38\,\text{г}.
    $
}
\solutionspace{120pt}

\tasknumber{7}%
\task{%
    Идеальный газ в экспериментальной установке подвергут политропному процессу $PV^n\text{— const}$
    с показателем политропы $n=1{,}5$.
    В одном из экспериментов объём газа увеличился в $3$ раза.
    Как при этом изменилась температура газа (выросла или уменьшилась, на сколько или во сколько раз)?
}
\answer{%
    \begin{align*}
    P_1V_1^n &= P_2V_2^n, P_1V_1 = \nu R T_1, P_2V_2 = \nu R T_2 \implies\frac{\nu R T_1}{V_1} V_1^n = \frac{\nu R T_2}{V_2} V_2^n \implies \\
    \implies T_1V_1^{n-1} &= T_2V_2^{n-1} \implies\frac{T_2}{T_1} = \cbr{\frac{V_1}{V_2}}^{n-1} \approx 0{,}577
    \end{align*}
}

\variantsplitter

\addpersonalvariant{Ярослав Лавровский}

\tasknumber{1}%
\task{%
    Из уравнения состояния идеального газа выведите или выразите...
    \begin{enumerate}
        \item давление,
        \item молярную массу,
        \item плотность газа.
    \end{enumerate}
}

\tasknumber{2}%
\task{%
    Почему плазму (одно из агрегатных состояний вещества, при котором часть молекул распадаются на ионы и электроны) нельзя считать идеальным газом?
}
\solutionspace{40pt}

\tasknumber{3}%
\task{%
    Изобразите в координатах $PV$/$VT$/$PT$ графики изохорического охлаждения в 4 раза (все 3 графика).
    Не забудьте указать оси и масштаб, начальную и конечную точки, направление движения на графике.
}
\solutionspace{100pt}

\tasknumber{4}%
\task{%
    Изобразите в координатах $PV$, соблюдая масштаб, процесс 1234,
    в котором 12 — изобарическое охлаждение в 3 раза,
    23 — изотермическое расширение в 3 раза,
    34 — изобарическое нагревание в 2 раза.
}
\solutionspace{160pt}

\tasknumber{5}%
\task{%
    Небольшую цилиндрическую пробирку с воздухом погружают на некоторую глубину в глубокое пресное озеро,
    после чего воздух занимает в ней лишь четвертую часть от общего объема.
    Определите глубину, на которую погрузили пробирку.
    Температуру считать постоянной $T = 282\,\text{К}$, давлением паров воды пренебречь,
    атмосферное давление принять равным $p_{\text{aтм}} = 100\,\text{кПа}$.
}
\answer{%
    \begin{align*}
    T\text{— const} &\implies P_1V_1 = \nu RT = P_2V_2.
    \\
    V_2 = \frac 14 V_1 &\implies P_1V_1 = P_2 \cdot \frac 14V_1 \implies P_2 = 4P_1 = 4p_{\text{aтм}}.
    \\
    P_2 = p_{\text{aтм}} + \rho_{\text{в}} g h \implies h = \frac{P_2 - p_{\text{aтм}}}{\rho_{\text{в}} g} &= \frac{4p_{\text{aтм}} - p_{\text{aтм}}}{\rho_{\text{в}} g} = \frac{3 \cdot p_{\text{aтм}}}{\rho_{\text{в}} g} =  \\
     &= \frac{3 \cdot 100\,\text{кПа}}{1000\,\frac{\text{кг}}{\text{м}^{3}} \cdot  10\,\frac{\text{м}}{\text{с}^{2}}} \approx 30\,\text{м}.
    \end{align*}
}
\solutionspace{120pt}

\tasknumber{6}%
\task{%
    В замкнутом сосуде объёмом $3\,\text{л}$ находится {gas.Name} ($\mu = 32\,\frac{\text{г}}{\text{моль}}$) под давлением $4\units{атм}$.
    Определите массу газа в сосуде и выразите её в граммах, приняв температуру газа равной $17\celsius$.
}
\answer{%
    $
        PV = \frac m\mu RT \implies m = \frac{PV \mu}{RT} =
        \frac{4{,}0\,\text{атм} \cdot 3\,\text{л} \cdot 32\,\frac{\text{г}}{\text{моль}}}{8{,}31\,\frac{\text{Дж}}{\text{моль}\cdot\text{К}} \cdot \cbr{17 + 273}\units{К}}
        \approx 15{,}93\,\text{г}.
    $
}
\solutionspace{120pt}

\tasknumber{7}%
\task{%
    Идеальный газ в экспериментальной установке подвергут политропному процессу $PV^n\text{— const}$
    с показателем политропы $n=1{,}5$.
    В одном из экспериментов объём газа уменьшился в $2$ раза.
    Как при этом изменилась температура газа (выросла или уменьшилась, на сколько или во сколько раз)?
}
\answer{%
    \begin{align*}
    P_1V_1^n &= P_2V_2^n, P_1V_1 = \nu R T_1, P_2V_2 = \nu R T_2 \implies\frac{\nu R T_1}{V_1} V_1^n = \frac{\nu R T_2}{V_2} V_2^n \implies \\
    \implies T_1V_1^{n-1} &= T_2V_2^{n-1} \implies\frac{T_2}{T_1} = \cbr{\frac{V_1}{V_2}}^{n-1} \approx 1{,}414
    \end{align*}
}

\variantsplitter

\addpersonalvariant{Анастасия Ламанова}

\tasknumber{1}%
\task{%
    Из уравнения состояния идеального газа выведите или выразите...
    \begin{enumerate}
        \item давление,
        \item температуру,
        \item плотность газа.
    \end{enumerate}
}

\tasknumber{2}%
\task{%
    В межзвездном пространстве встречаются молекулы (до нескольких десятков на $\text{{м}}^3$).
    Почему такой газ нельзя считать идеальным?
}
\solutionspace{40pt}

\tasknumber{3}%
\task{%
    Изобразите в координатах $PV$/$VT$/$PT$ графики изобарического сжатия в 3 раза (все 3 графика).
    Не забудьте указать оси и масштаб, начальную и конечную точки, направление движения на графике.
}
\solutionspace{100pt}

\tasknumber{4}%
\task{%
    Изобразите в координатах $PV$, соблюдая масштаб, процесс 1234,
    в котором 12 — изохорическое охлаждение в 3 раза,
    23 — изотермическое сжатие в 3 раза,
    34 — изобарическое нагревание в 3 раза.
}
\solutionspace{160pt}

\tasknumber{5}%
\task{%
    Небольшую цилиндрическую пробирку с воздухом погружают на некоторую глубину в глубокое пресное озеро,
    после чего воздух занимает в ней лишь шестую часть от общего объема.
    Определите глубину, на которую погрузили пробирку.
    Температуру считать постоянной $T = 291\,\text{К}$, давлением паров воды пренебречь,
    атмосферное давление принять равным $p_{\text{aтм}} = 100\,\text{кПа}$.
}
\answer{%
    \begin{align*}
    T\text{— const} &\implies P_1V_1 = \nu RT = P_2V_2.
    \\
    V_2 = \frac 16 V_1 &\implies P_1V_1 = P_2 \cdot \frac 16V_1 \implies P_2 = 6P_1 = 6p_{\text{aтм}}.
    \\
    P_2 = p_{\text{aтм}} + \rho_{\text{в}} g h \implies h = \frac{P_2 - p_{\text{aтм}}}{\rho_{\text{в}} g} &= \frac{6p_{\text{aтм}} - p_{\text{aтм}}}{\rho_{\text{в}} g} = \frac{5 \cdot p_{\text{aтм}}}{\rho_{\text{в}} g} =  \\
     &= \frac{5 \cdot 100\,\text{кПа}}{1000\,\frac{\text{кг}}{\text{м}^{3}} \cdot  10\,\frac{\text{м}}{\text{с}^{2}}} \approx 50\,\text{м}.
    \end{align*}
}
\solutionspace{120pt}

\tasknumber{6}%
\task{%
    В замкнутом сосуде объёмом $5\,\text{л}$ находится {gas.Name} ($\mu = 40\,\frac{\text{г}}{\text{моль}}$) под давлением $4{,}5\units{атм}$.
    Определите массу газа в сосуде и выразите её в граммах, приняв температуру газа равной $47\celsius$.
}
\answer{%
    $
        PV = \frac m\mu RT \implies m = \frac{PV \mu}{RT} =
        \frac{4{,}5\,\text{атм} \cdot 5\,\text{л} \cdot 40\,\frac{\text{г}}{\text{моль}}}{8{,}31\,\frac{\text{Дж}}{\text{моль}\cdot\text{К}} \cdot \cbr{47 + 273}\units{К}}
        \approx 33{,}84\,\text{г}.
    $
}
\solutionspace{120pt}

\tasknumber{7}%
\task{%
    Идеальный газ в экспериментальной установке подвергут политропному процессу $PV^n\text{— const}$
    с показателем политропы $n=0{,}7$.
    В одном из экспериментов объём газа увеличился в $3$ раза.
    Как при этом изменилась температура газа (выросла или уменьшилась, на сколько или во сколько раз)?
}
\answer{%
    \begin{align*}
    P_1V_1^n &= P_2V_2^n, P_1V_1 = \nu R T_1, P_2V_2 = \nu R T_2 \implies\frac{\nu R T_1}{V_1} V_1^n = \frac{\nu R T_2}{V_2} V_2^n \implies \\
    \implies T_1V_1^{n-1} &= T_2V_2^{n-1} \implies\frac{T_2}{T_1} = \cbr{\frac{V_1}{V_2}}^{n-1} \approx 1{,}390
    \end{align*}
}

\variantsplitter

\addpersonalvariant{Виктория Легонькова}

\tasknumber{1}%
\task{%
    Из уравнения состояния идеального газа выведите или выразите...
    \begin{enumerate}
        \item объём,
        \item молярную массу,
        \item плотность газа.
    \end{enumerate}
}

\tasknumber{2}%
\task{%
    В межзвездном пространстве встречаются молекулы (до нескольких десятков на $\text{{м}}^3$).
    Почему такой газ нельзя считать идеальным?
}
\solutionspace{40pt}

\tasknumber{3}%
\task{%
    Изобразите в координатах $PV$/$VT$/$PT$ графики изобарического сжатия в 2 раза (все 3 графика).
    Не забудьте указать оси и масштаб, начальную и конечную точки, направление движения на графике.
}
\solutionspace{100pt}

\tasknumber{4}%
\task{%
    Изобразите в координатах $PV$, соблюдая масштаб, процесс 1234,
    в котором 12 — изохорическое охлаждение в 2 раза,
    23 — изотермическое сжатие в 2 раза,
    34 — изохорическое охлаждение в 2 раза.
}
\solutionspace{160pt}

\tasknumber{5}%
\task{%
    Небольшую цилиндрическую пробирку с воздухом погружают на некоторую глубину в глубокое пресное озеро,
    после чего воздух занимает в ней лишь четвертую часть от общего объема.
    Определите глубину, на которую погрузили пробирку.
    Температуру считать постоянной $T = 286\,\text{К}$, давлением паров воды пренебречь,
    атмосферное давление принять равным $p_{\text{aтм}} = 100\,\text{кПа}$.
}
\answer{%
    \begin{align*}
    T\text{— const} &\implies P_1V_1 = \nu RT = P_2V_2.
    \\
    V_2 = \frac 14 V_1 &\implies P_1V_1 = P_2 \cdot \frac 14V_1 \implies P_2 = 4P_1 = 4p_{\text{aтм}}.
    \\
    P_2 = p_{\text{aтм}} + \rho_{\text{в}} g h \implies h = \frac{P_2 - p_{\text{aтм}}}{\rho_{\text{в}} g} &= \frac{4p_{\text{aтм}} - p_{\text{aтм}}}{\rho_{\text{в}} g} = \frac{3 \cdot p_{\text{aтм}}}{\rho_{\text{в}} g} =  \\
     &= \frac{3 \cdot 100\,\text{кПа}}{1000\,\frac{\text{кг}}{\text{м}^{3}} \cdot  10\,\frac{\text{м}}{\text{с}^{2}}} \approx 30\,\text{м}.
    \end{align*}
}
\solutionspace{120pt}

\tasknumber{6}%
\task{%
    В замкнутом сосуде объёмом $2\,\text{л}$ находится {gas.Name} ($\mu = 44\,\frac{\text{г}}{\text{моль}}$) под давлением $2{,}5\units{атм}$.
    Определите массу газа в сосуде и выразите её в граммах, приняв температуру газа равной $17\celsius$.
}
\answer{%
    $
        PV = \frac m\mu RT \implies m = \frac{PV \mu}{RT} =
        \frac{2{,}5\,\text{атм} \cdot 2\,\text{л} \cdot 44\,\frac{\text{г}}{\text{моль}}}{8{,}31\,\frac{\text{Дж}}{\text{моль}\cdot\text{К}} \cdot \cbr{17 + 273}\units{К}}
        \approx 9{,}13\,\text{г}.
    $
}
\solutionspace{120pt}

\tasknumber{7}%
\task{%
    Идеальный газ в экспериментальной установке подвергут политропному процессу $PV^n\text{— const}$
    с показателем политропы $n=1{,}8$.
    В одном из экспериментов объём газа уменьшился в $3$ раза.
    Как при этом изменилась температура газа (выросла или уменьшилась, на сколько или во сколько раз)?
}
\answer{%
    \begin{align*}
    P_1V_1^n &= P_2V_2^n, P_1V_1 = \nu R T_1, P_2V_2 = \nu R T_2 \implies\frac{\nu R T_1}{V_1} V_1^n = \frac{\nu R T_2}{V_2} V_2^n \implies \\
    \implies T_1V_1^{n-1} &= T_2V_2^{n-1} \implies\frac{T_2}{T_1} = \cbr{\frac{V_1}{V_2}}^{n-1} \approx 2{,}408
    \end{align*}
}

\variantsplitter

\addpersonalvariant{Семён Мартынов}

\tasknumber{1}%
\task{%
    Из уравнения состояния идеального газа выведите или выразите...
    \begin{enumerate}
        \item давление,
        \item температуру,
        \item плотность газа.
    \end{enumerate}
}

\tasknumber{2}%
\task{%
    В межзвездном пространстве встречаются молекулы (до нескольких десятков на $\text{{м}}^3$).
    Почему такой газ нельзя считать идеальным?
}
\solutionspace{40pt}

\tasknumber{3}%
\task{%
    Изобразите в координатах $PV$/$VT$/$PT$ графики изотермического понижения давления в 4 раза (все 3 графика).
    Не забудьте указать оси и масштаб, начальную и конечную точки, направление движения на графике.
}
\solutionspace{100pt}

\tasknumber{4}%
\task{%
    Изобразите в координатах $PV$, соблюдая масштаб, процесс 1234,
    в котором 12 — изохорическое нагревание в 3 раза,
    23 — изотермическое сжатие в 3 раза,
    34 — изобарическое нагревание в 2 раза.
}
\solutionspace{160pt}

\tasknumber{5}%
\task{%
    Небольшую цилиндрическую пробирку с воздухом погружают на некоторую глубину в глубокое пресное озеро,
    после чего воздух занимает в ней лишь шестую часть от общего объема.
    Определите глубину, на которую погрузили пробирку.
    Температуру считать постоянной $T = 290\,\text{К}$, давлением паров воды пренебречь,
    атмосферное давление принять равным $p_{\text{aтм}} = 100\,\text{кПа}$.
}
\answer{%
    \begin{align*}
    T\text{— const} &\implies P_1V_1 = \nu RT = P_2V_2.
    \\
    V_2 = \frac 16 V_1 &\implies P_1V_1 = P_2 \cdot \frac 16V_1 \implies P_2 = 6P_1 = 6p_{\text{aтм}}.
    \\
    P_2 = p_{\text{aтм}} + \rho_{\text{в}} g h \implies h = \frac{P_2 - p_{\text{aтм}}}{\rho_{\text{в}} g} &= \frac{6p_{\text{aтм}} - p_{\text{aтм}}}{\rho_{\text{в}} g} = \frac{5 \cdot p_{\text{aтм}}}{\rho_{\text{в}} g} =  \\
     &= \frac{5 \cdot 100\,\text{кПа}}{1000\,\frac{\text{кг}}{\text{м}^{3}} \cdot  10\,\frac{\text{м}}{\text{с}^{2}}} \approx 50\,\text{м}.
    \end{align*}
}
\solutionspace{120pt}

\tasknumber{6}%
\task{%
    В замкнутом сосуде объёмом $2\,\text{л}$ находится {gas.Name} ($\mu = 28\,\frac{\text{г}}{\text{моль}}$) под давлением $3\units{атм}$.
    Определите массу газа в сосуде и выразите её в граммах, приняв температуру газа равной $37\celsius$.
}
\answer{%
    $
        PV = \frac m\mu RT \implies m = \frac{PV \mu}{RT} =
        \frac{3{,}0\,\text{атм} \cdot 2\,\text{л} \cdot 28\,\frac{\text{г}}{\text{моль}}}{8{,}31\,\frac{\text{Дж}}{\text{моль}\cdot\text{К}} \cdot \cbr{37 + 273}\units{К}}
        \approx 6{,}52\,\text{г}.
    $
}
\solutionspace{120pt}

\tasknumber{7}%
\task{%
    Идеальный газ в экспериментальной установке подвергут политропному процессу $PV^n\text{— const}$
    с показателем политропы $n=0{,}7$.
    В одном из экспериментов объём газа уменьшился в $2$ раза.
    Как при этом изменилась температура газа (выросла или уменьшилась, на сколько или во сколько раз)?
}
\answer{%
    \begin{align*}
    P_1V_1^n &= P_2V_2^n, P_1V_1 = \nu R T_1, P_2V_2 = \nu R T_2 \implies\frac{\nu R T_1}{V_1} V_1^n = \frac{\nu R T_2}{V_2} V_2^n \implies \\
    \implies T_1V_1^{n-1} &= T_2V_2^{n-1} \implies\frac{T_2}{T_1} = \cbr{\frac{V_1}{V_2}}^{n-1} \approx 0{,}812
    \end{align*}
}

\variantsplitter

\addpersonalvariant{Варвара Минаева}

\tasknumber{1}%
\task{%
    Из уравнения состояния идеального газа выведите или выразите...
    \begin{enumerate}
        \item давление,
        \item температуру,
        \item плотность газа.
    \end{enumerate}
}

\tasknumber{2}%
\task{%
    Почему плазму (одно из агрегатных состояний вещества, при котором часть молекул распадаются на ионы и электроны) нельзя считать идеальным газом?
}
\solutionspace{40pt}

\tasknumber{3}%
\task{%
    Изобразите в координатах $PV$/$VT$/$PT$ графики изотермического понижения давления в 3 раза (все 3 графика).
    Не забудьте указать оси и масштаб, начальную и конечную точки, направление движения на графике.
}
\solutionspace{100pt}

\tasknumber{4}%
\task{%
    Изобразите в координатах $PV$, соблюдая масштаб, процесс 1234,
    в котором 12 — изобарическое охлаждение в 3 раза,
    23 — изотермическое расширение в 2 раза,
    34 — изохорическое нагревание в 2 раза.
}
\solutionspace{160pt}

\tasknumber{5}%
\task{%
    Небольшую цилиндрическую пробирку с воздухом погружают на некоторую глубину в глубокое пресное озеро,
    после чего воздух занимает в ней лишь шестую часть от общего объема.
    Определите глубину, на которую погрузили пробирку.
    Температуру считать постоянной $T = 287\,\text{К}$, давлением паров воды пренебречь,
    атмосферное давление принять равным $p_{\text{aтм}} = 100\,\text{кПа}$.
}
\answer{%
    \begin{align*}
    T\text{— const} &\implies P_1V_1 = \nu RT = P_2V_2.
    \\
    V_2 = \frac 16 V_1 &\implies P_1V_1 = P_2 \cdot \frac 16V_1 \implies P_2 = 6P_1 = 6p_{\text{aтм}}.
    \\
    P_2 = p_{\text{aтм}} + \rho_{\text{в}} g h \implies h = \frac{P_2 - p_{\text{aтм}}}{\rho_{\text{в}} g} &= \frac{6p_{\text{aтм}} - p_{\text{aтм}}}{\rho_{\text{в}} g} = \frac{5 \cdot p_{\text{aтм}}}{\rho_{\text{в}} g} =  \\
     &= \frac{5 \cdot 100\,\text{кПа}}{1000\,\frac{\text{кг}}{\text{м}^{3}} \cdot  10\,\frac{\text{м}}{\text{с}^{2}}} \approx 50\,\text{м}.
    \end{align*}
}
\solutionspace{120pt}

\tasknumber{6}%
\task{%
    В замкнутом сосуде объёмом $3\,\text{л}$ находится {gas.Name} ($\mu = 20\,\frac{\text{г}}{\text{моль}}$) под давлением $5\units{атм}$.
    Определите массу газа в сосуде и выразите её в граммах, приняв температуру газа равной $7\celsius$.
}
\answer{%
    $
        PV = \frac m\mu RT \implies m = \frac{PV \mu}{RT} =
        \frac{5{,}0\,\text{атм} \cdot 3\,\text{л} \cdot 20\,\frac{\text{г}}{\text{моль}}}{8{,}31\,\frac{\text{Дж}}{\text{моль}\cdot\text{К}} \cdot \cbr{7 + 273}\units{К}}
        \approx 12{,}89\,\text{г}.
    $
}
\solutionspace{120pt}

\tasknumber{7}%
\task{%
    Идеальный газ в экспериментальной установке подвергут политропному процессу $PV^n\text{— const}$
    с показателем политропы $n=0{,}4$.
    В одном из экспериментов объём газа уменьшился в $4$ раза.
    Как при этом изменилась температура газа (выросла или уменьшилась, на сколько или во сколько раз)?
}
\answer{%
    \begin{align*}
    P_1V_1^n &= P_2V_2^n, P_1V_1 = \nu R T_1, P_2V_2 = \nu R T_2 \implies\frac{\nu R T_1}{V_1} V_1^n = \frac{\nu R T_2}{V_2} V_2^n \implies \\
    \implies T_1V_1^{n-1} &= T_2V_2^{n-1} \implies\frac{T_2}{T_1} = \cbr{\frac{V_1}{V_2}}^{n-1} \approx 0{,}435
    \end{align*}
}

\variantsplitter

\addpersonalvariant{Леонид Никитин}

\tasknumber{1}%
\task{%
    Из уравнения состояния идеального газа выведите или выразите...
    \begin{enumerate}
        \item объём,
        \item молярную массу,
        \item концентрацию молекул газа.
    \end{enumerate}
}

\tasknumber{2}%
\task{%
    Почему плазму (одно из агрегатных состояний вещества, при котором часть молекул распадаются на ионы и электроны) нельзя считать идеальным газом?
}
\solutionspace{40pt}

\tasknumber{3}%
\task{%
    Изобразите в координатах $PV$/$VT$/$PT$ графики изобарического сжатия в 3 раза (все 3 графика).
    Не забудьте указать оси и масштаб, начальную и конечную точки, направление движения на графике.
}
\solutionspace{100pt}

\tasknumber{4}%
\task{%
    Изобразите в координатах $PV$, соблюдая масштаб, процесс 1234,
    в котором 12 — изохорическое нагревание в 2 раза,
    23 — изотермическое сжатие в 3 раза,
    34 — изохорическое охлаждение в 3 раза.
}
\solutionspace{160pt}

\tasknumber{5}%
\task{%
    Небольшую цилиндрическую пробирку с воздухом погружают на некоторую глубину в глубокое пресное озеро,
    после чего воздух занимает в ней лишь шестую часть от общего объема.
    Определите глубину, на которую погрузили пробирку.
    Температуру считать постоянной $T = 279\,\text{К}$, давлением паров воды пренебречь,
    атмосферное давление принять равным $p_{\text{aтм}} = 100\,\text{кПа}$.
}
\answer{%
    \begin{align*}
    T\text{— const} &\implies P_1V_1 = \nu RT = P_2V_2.
    \\
    V_2 = \frac 16 V_1 &\implies P_1V_1 = P_2 \cdot \frac 16V_1 \implies P_2 = 6P_1 = 6p_{\text{aтм}}.
    \\
    P_2 = p_{\text{aтм}} + \rho_{\text{в}} g h \implies h = \frac{P_2 - p_{\text{aтм}}}{\rho_{\text{в}} g} &= \frac{6p_{\text{aтм}} - p_{\text{aтм}}}{\rho_{\text{в}} g} = \frac{5 \cdot p_{\text{aтм}}}{\rho_{\text{в}} g} =  \\
     &= \frac{5 \cdot 100\,\text{кПа}}{1000\,\frac{\text{кг}}{\text{м}^{3}} \cdot  10\,\frac{\text{м}}{\text{с}^{2}}} \approx 50\,\text{м}.
    \end{align*}
}
\solutionspace{120pt}

\tasknumber{6}%
\task{%
    В замкнутом сосуде объёмом $5\,\text{л}$ находится {gas.Name} ($\mu = 32\,\frac{\text{г}}{\text{моль}}$) под давлением $2{,}5\units{атм}$.
    Определите массу газа в сосуде и выразите её в граммах, приняв температуру газа равной $7\celsius$.
}
\answer{%
    $
        PV = \frac m\mu RT \implies m = \frac{PV \mu}{RT} =
        \frac{2{,}5\,\text{атм} \cdot 5\,\text{л} \cdot 32\,\frac{\text{г}}{\text{моль}}}{8{,}31\,\frac{\text{Дж}}{\text{моль}\cdot\text{К}} \cdot \cbr{7 + 273}\units{К}}
        \approx 17{,}19\,\text{г}.
    $
}
\solutionspace{120pt}

\tasknumber{7}%
\task{%
    Идеальный газ в экспериментальной установке подвергут политропному процессу $PV^n\text{— const}$
    с показателем политропы $n=0{,}7$.
    В одном из экспериментов объём газа уменьшился в $2$ раза.
    Как при этом изменилась температура газа (выросла или уменьшилась, на сколько или во сколько раз)?
}
\answer{%
    \begin{align*}
    P_1V_1^n &= P_2V_2^n, P_1V_1 = \nu R T_1, P_2V_2 = \nu R T_2 \implies\frac{\nu R T_1}{V_1} V_1^n = \frac{\nu R T_2}{V_2} V_2^n \implies \\
    \implies T_1V_1^{n-1} &= T_2V_2^{n-1} \implies\frac{T_2}{T_1} = \cbr{\frac{V_1}{V_2}}^{n-1} \approx 0{,}812
    \end{align*}
}

\variantsplitter

\addpersonalvariant{Тимофей Полетаев}

\tasknumber{1}%
\task{%
    Из уравнения состояния идеального газа выведите или выразите...
    \begin{enumerate}
        \item давление,
        \item молярную массу,
        \item концентрацию молекул газа.
    \end{enumerate}
}

\tasknumber{2}%
\task{%
    В межзвездном пространстве встречаются молекулы (до нескольких десятков на $\text{{м}}^3$).
    Почему такой газ нельзя считать идеальным?
}
\solutionspace{40pt}

\tasknumber{3}%
\task{%
    Изобразите в координатах $PV$/$VT$/$PT$ графики изобарического сжатия в 4 раза (все 3 графика).
    Не забудьте указать оси и масштаб, начальную и конечную точки, направление движения на графике.
}
\solutionspace{100pt}

\tasknumber{4}%
\task{%
    Изобразите в координатах $PV$, соблюдая масштаб, процесс 1234,
    в котором 12 — изобарическое нагревание в 3 раза,
    23 — изотермическое расширение в 3 раза,
    34 — изохорическое нагревание в 3 раза.
}
\solutionspace{160pt}

\tasknumber{5}%
\task{%
    Небольшую цилиндрическую пробирку с воздухом погружают на некоторую глубину в глубокое пресное озеро,
    после чего воздух занимает в ней лишь четвертую часть от общего объема.
    Определите глубину, на которую погрузили пробирку.
    Температуру считать постоянной $T = 284\,\text{К}$, давлением паров воды пренебречь,
    атмосферное давление принять равным $p_{\text{aтм}} = 100\,\text{кПа}$.
}
\answer{%
    \begin{align*}
    T\text{— const} &\implies P_1V_1 = \nu RT = P_2V_2.
    \\
    V_2 = \frac 14 V_1 &\implies P_1V_1 = P_2 \cdot \frac 14V_1 \implies P_2 = 4P_1 = 4p_{\text{aтм}}.
    \\
    P_2 = p_{\text{aтм}} + \rho_{\text{в}} g h \implies h = \frac{P_2 - p_{\text{aтм}}}{\rho_{\text{в}} g} &= \frac{4p_{\text{aтм}} - p_{\text{aтм}}}{\rho_{\text{в}} g} = \frac{3 \cdot p_{\text{aтм}}}{\rho_{\text{в}} g} =  \\
     &= \frac{3 \cdot 100\,\text{кПа}}{1000\,\frac{\text{кг}}{\text{м}^{3}} \cdot  10\,\frac{\text{м}}{\text{с}^{2}}} \approx 30\,\text{м}.
    \end{align*}
}
\solutionspace{120pt}

\tasknumber{6}%
\task{%
    В замкнутом сосуде объёмом $5\,\text{л}$ находится {gas.Name} ($\mu = 32\,\frac{\text{г}}{\text{моль}}$) под давлением $5\units{атм}$.
    Определите массу газа в сосуде и выразите её в граммах, приняв температуру газа равной $17\celsius$.
}
\answer{%
    $
        PV = \frac m\mu RT \implies m = \frac{PV \mu}{RT} =
        \frac{5{,}0\,\text{атм} \cdot 5\,\text{л} \cdot 32\,\frac{\text{г}}{\text{моль}}}{8{,}31\,\frac{\text{Дж}}{\text{моль}\cdot\text{К}} \cdot \cbr{17 + 273}\units{К}}
        \approx 33{,}20\,\text{г}.
    $
}
\solutionspace{120pt}

\tasknumber{7}%
\task{%
    Идеальный газ в экспериментальной установке подвергут политропному процессу $PV^n\text{— const}$
    с показателем политропы $n=1{,}5$.
    В одном из экспериментов объём газа увеличился в $4$ раза.
    Как при этом изменилась температура газа (выросла или уменьшилась, на сколько или во сколько раз)?
}
\answer{%
    \begin{align*}
    P_1V_1^n &= P_2V_2^n, P_1V_1 = \nu R T_1, P_2V_2 = \nu R T_2 \implies\frac{\nu R T_1}{V_1} V_1^n = \frac{\nu R T_2}{V_2} V_2^n \implies \\
    \implies T_1V_1^{n-1} &= T_2V_2^{n-1} \implies\frac{T_2}{T_1} = \cbr{\frac{V_1}{V_2}}^{n-1} \approx 0{,}500
    \end{align*}
}

\variantsplitter

\addpersonalvariant{Андрей Рожков}

\tasknumber{1}%
\task{%
    Из уравнения состояния идеального газа выведите или выразите...
    \begin{enumerate}
        \item давление,
        \item молярную массу,
        \item плотность газа.
    \end{enumerate}
}

\tasknumber{2}%
\task{%
    В межзвездном пространстве встречаются молекулы (до нескольких десятков на $\text{{м}}^3$).
    Почему такой газ нельзя считать идеальным?
}
\solutionspace{40pt}

\tasknumber{3}%
\task{%
    Изобразите в координатах $PV$/$VT$/$PT$ графики изотермического повышения давления в 4 раза (все 3 графика).
    Не забудьте указать оси и масштаб, начальную и конечную точки, направление движения на графике.
}
\solutionspace{100pt}

\tasknumber{4}%
\task{%
    Изобразите в координатах $PV$, соблюдая масштаб, процесс 1234,
    в котором 12 — изохорическое нагревание в 2 раза,
    23 — изотермическое сжатие в 3 раза,
    34 — изобарическое охлаждение в 3 раза.
}
\solutionspace{160pt}

\tasknumber{5}%
\task{%
    Небольшую цилиндрическую пробирку с воздухом погружают на некоторую глубину в глубокое пресное озеро,
    после чего воздух занимает в ней лишь шестую часть от общего объема.
    Определите глубину, на которую погрузили пробирку.
    Температуру считать постоянной $T = 288\,\text{К}$, давлением паров воды пренебречь,
    атмосферное давление принять равным $p_{\text{aтм}} = 100\,\text{кПа}$.
}
\answer{%
    \begin{align*}
    T\text{— const} &\implies P_1V_1 = \nu RT = P_2V_2.
    \\
    V_2 = \frac 16 V_1 &\implies P_1V_1 = P_2 \cdot \frac 16V_1 \implies P_2 = 6P_1 = 6p_{\text{aтм}}.
    \\
    P_2 = p_{\text{aтм}} + \rho_{\text{в}} g h \implies h = \frac{P_2 - p_{\text{aтм}}}{\rho_{\text{в}} g} &= \frac{6p_{\text{aтм}} - p_{\text{aтм}}}{\rho_{\text{в}} g} = \frac{5 \cdot p_{\text{aтм}}}{\rho_{\text{в}} g} =  \\
     &= \frac{5 \cdot 100\,\text{кПа}}{1000\,\frac{\text{кг}}{\text{м}^{3}} \cdot  10\,\frac{\text{м}}{\text{с}^{2}}} \approx 50\,\text{м}.
    \end{align*}
}
\solutionspace{120pt}

\tasknumber{6}%
\task{%
    В замкнутом сосуде объёмом $5\,\text{л}$ находится {gas.Name} ($\mu = 20\,\frac{\text{г}}{\text{моль}}$) под давлением $5\units{атм}$.
    Определите массу газа в сосуде и выразите её в граммах, приняв температуру газа равной $37\celsius$.
}
\answer{%
    $
        PV = \frac m\mu RT \implies m = \frac{PV \mu}{RT} =
        \frac{5{,}0\,\text{атм} \cdot 5\,\text{л} \cdot 20\,\frac{\text{г}}{\text{моль}}}{8{,}31\,\frac{\text{Дж}}{\text{моль}\cdot\text{К}} \cdot \cbr{37 + 273}\units{К}}
        \approx 19{,}41\,\text{г}.
    $
}
\solutionspace{120pt}

\tasknumber{7}%
\task{%
    Идеальный газ в экспериментальной установке подвергут политропному процессу $PV^n\text{— const}$
    с показателем политропы $n=0{,}4$.
    В одном из экспериментов объём газа увеличился в $2$ раза.
    Как при этом изменилась температура газа (выросла или уменьшилась, на сколько или во сколько раз)?
}
\answer{%
    \begin{align*}
    P_1V_1^n &= P_2V_2^n, P_1V_1 = \nu R T_1, P_2V_2 = \nu R T_2 \implies\frac{\nu R T_1}{V_1} V_1^n = \frac{\nu R T_2}{V_2} V_2^n \implies \\
    \implies T_1V_1^{n-1} &= T_2V_2^{n-1} \implies\frac{T_2}{T_1} = \cbr{\frac{V_1}{V_2}}^{n-1} \approx 1{,}516
    \end{align*}
}

\variantsplitter

\addpersonalvariant{Рената Таржиманова}

\tasknumber{1}%
\task{%
    Из уравнения состояния идеального газа выведите или выразите...
    \begin{enumerate}
        \item давление,
        \item молярную массу,
        \item концентрацию молекул газа.
    \end{enumerate}
}

\tasknumber{2}%
\task{%
    Почему плазму (одно из агрегатных состояний вещества, при котором часть молекул распадаются на ионы и электроны) нельзя считать идеальным газом?
}
\solutionspace{40pt}

\tasknumber{3}%
\task{%
    Изобразите в координатах $PV$/$VT$/$PT$ графики изотермического повышения давления в 2 раза (все 3 графика).
    Не забудьте указать оси и масштаб, начальную и конечную точки, направление движения на графике.
}
\solutionspace{100pt}

\tasknumber{4}%
\task{%
    Изобразите в координатах $PV$, соблюдая масштаб, процесс 1234,
    в котором 12 — изохорическое нагревание в 2 раза,
    23 — изотермическое сжатие в 3 раза,
    34 — изобарическое охлаждение в 2 раза.
}
\solutionspace{160pt}

\tasknumber{5}%
\task{%
    Небольшую цилиндрическую пробирку с воздухом погружают на некоторую глубину в глубокое пресное озеро,
    после чего воздух занимает в ней лишь третью часть от общего объема.
    Определите глубину, на которую погрузили пробирку.
    Температуру считать постоянной $T = 282\,\text{К}$, давлением паров воды пренебречь,
    атмосферное давление принять равным $p_{\text{aтм}} = 100\,\text{кПа}$.
}
\answer{%
    \begin{align*}
    T\text{— const} &\implies P_1V_1 = \nu RT = P_2V_2.
    \\
    V_2 = \frac 13 V_1 &\implies P_1V_1 = P_2 \cdot \frac 13V_1 \implies P_2 = 3P_1 = 3p_{\text{aтм}}.
    \\
    P_2 = p_{\text{aтм}} + \rho_{\text{в}} g h \implies h = \frac{P_2 - p_{\text{aтм}}}{\rho_{\text{в}} g} &= \frac{3p_{\text{aтм}} - p_{\text{aтм}}}{\rho_{\text{в}} g} = \frac{2 \cdot p_{\text{aтм}}}{\rho_{\text{в}} g} =  \\
     &= \frac{2 \cdot 100\,\text{кПа}}{1000\,\frac{\text{кг}}{\text{м}^{3}} \cdot  10\,\frac{\text{м}}{\text{с}^{2}}} \approx 20\,\text{м}.
    \end{align*}
}
\solutionspace{120pt}

\tasknumber{6}%
\task{%
    В замкнутом сосуде объёмом $2\,\text{л}$ находится {gas.Name} ($\mu = 29\,\frac{\text{г}}{\text{моль}}$) под давлением $3\units{атм}$.
    Определите массу газа в сосуде и выразите её в граммах, приняв температуру газа равной $17\celsius$.
}
\answer{%
    $
        PV = \frac m\mu RT \implies m = \frac{PV \mu}{RT} =
        \frac{3{,}0\,\text{атм} \cdot 2\,\text{л} \cdot 29\,\frac{\text{г}}{\text{моль}}}{8{,}31\,\frac{\text{Дж}}{\text{моль}\cdot\text{К}} \cdot \cbr{17 + 273}\units{К}}
        \approx 7{,}22\,\text{г}.
    $
}
\solutionspace{120pt}

\tasknumber{7}%
\task{%
    Идеальный газ в экспериментальной установке подвергут политропному процессу $PV^n\text{— const}$
    с показателем политропы $n=0{,}7$.
    В одном из экспериментов объём газа увеличился в $3$ раза.
    Как при этом изменилась температура газа (выросла или уменьшилась, на сколько или во сколько раз)?
}
\answer{%
    \begin{align*}
    P_1V_1^n &= P_2V_2^n, P_1V_1 = \nu R T_1, P_2V_2 = \nu R T_2 \implies\frac{\nu R T_1}{V_1} V_1^n = \frac{\nu R T_2}{V_2} V_2^n \implies \\
    \implies T_1V_1^{n-1} &= T_2V_2^{n-1} \implies\frac{T_2}{T_1} = \cbr{\frac{V_1}{V_2}}^{n-1} \approx 1{,}390
    \end{align*}
}

\variantsplitter

\addpersonalvariant{Андрей Щербаков}

\tasknumber{1}%
\task{%
    Из уравнения состояния идеального газа выведите или выразите...
    \begin{enumerate}
        \item объём,
        \item молярную массу,
        \item концентрацию молекул газа.
    \end{enumerate}
}

\tasknumber{2}%
\task{%
    Почему плазму (одно из агрегатных состояний вещества, при котором часть молекул распадаются на ионы и электроны) нельзя считать идеальным газом?
}
\solutionspace{40pt}

\tasknumber{3}%
\task{%
    Изобразите в координатах $PV$/$VT$/$PT$ графики изохорического охлаждения в 4 раза (все 3 графика).
    Не забудьте указать оси и масштаб, начальную и конечную точки, направление движения на графике.
}
\solutionspace{100pt}

\tasknumber{4}%
\task{%
    Изобразите в координатах $PV$, соблюдая масштаб, процесс 1234,
    в котором 12 — изохорическое нагревание в 3 раза,
    23 — изотермическое расширение в 3 раза,
    34 — изобарическое нагревание в 3 раза.
}
\solutionspace{160pt}

\tasknumber{5}%
\task{%
    Небольшую цилиндрическую пробирку с воздухом погружают на некоторую глубину в глубокое пресное озеро,
    после чего воздух занимает в ней лишь пятую часть от общего объема.
    Определите глубину, на которую погрузили пробирку.
    Температуру считать постоянной $T = 284\,\text{К}$, давлением паров воды пренебречь,
    атмосферное давление принять равным $p_{\text{aтм}} = 100\,\text{кПа}$.
}
\answer{%
    \begin{align*}
    T\text{— const} &\implies P_1V_1 = \nu RT = P_2V_2.
    \\
    V_2 = \frac 15 V_1 &\implies P_1V_1 = P_2 \cdot \frac 15V_1 \implies P_2 = 5P_1 = 5p_{\text{aтм}}.
    \\
    P_2 = p_{\text{aтм}} + \rho_{\text{в}} g h \implies h = \frac{P_2 - p_{\text{aтм}}}{\rho_{\text{в}} g} &= \frac{5p_{\text{aтм}} - p_{\text{aтм}}}{\rho_{\text{в}} g} = \frac{4 \cdot p_{\text{aтм}}}{\rho_{\text{в}} g} =  \\
     &= \frac{4 \cdot 100\,\text{кПа}}{1000\,\frac{\text{кг}}{\text{м}^{3}} \cdot  10\,\frac{\text{м}}{\text{с}^{2}}} \approx 40\,\text{м}.
    \end{align*}
}
\solutionspace{120pt}

\tasknumber{6}%
\task{%
    В замкнутом сосуде объёмом $4\,\text{л}$ находится {gas.Name} ($\mu = 32\,\frac{\text{г}}{\text{моль}}$) под давлением $4\units{атм}$.
    Определите массу газа в сосуде и выразите её в граммах, приняв температуру газа равной $17\celsius$.
}
\answer{%
    $
        PV = \frac m\mu RT \implies m = \frac{PV \mu}{RT} =
        \frac{4{,}0\,\text{атм} \cdot 4\,\text{л} \cdot 32\,\frac{\text{г}}{\text{моль}}}{8{,}31\,\frac{\text{Дж}}{\text{моль}\cdot\text{К}} \cdot \cbr{17 + 273}\units{К}}
        \approx 21{,}25\,\text{г}.
    $
}
\solutionspace{120pt}

\tasknumber{7}%
\task{%
    Идеальный газ в экспериментальной установке подвергут политропному процессу $PV^n\text{— const}$
    с показателем политропы $n=0{,}4$.
    В одном из экспериментов объём газа уменьшился в $4$ раза.
    Как при этом изменилась температура газа (выросла или уменьшилась, на сколько или во сколько раз)?
}
\answer{%
    \begin{align*}
    P_1V_1^n &= P_2V_2^n, P_1V_1 = \nu R T_1, P_2V_2 = \nu R T_2 \implies\frac{\nu R T_1}{V_1} V_1^n = \frac{\nu R T_2}{V_2} V_2^n \implies \\
    \implies T_1V_1^{n-1} &= T_2V_2^{n-1} \implies\frac{T_2}{T_1} = \cbr{\frac{V_1}{V_2}}^{n-1} \approx 0{,}435
    \end{align*}
}

\variantsplitter

\addpersonalvariant{Михаил Ярошевский}

\tasknumber{1}%
\task{%
    Из уравнения состояния идеального газа выведите или выразите...
    \begin{enumerate}
        \item объём,
        \item температуру,
        \item концентрацию молекул газа.
    \end{enumerate}
}

\tasknumber{2}%
\task{%
    В межзвездном пространстве встречаются молекулы (до нескольких десятков на $\text{{м}}^3$).
    Почему такой газ нельзя считать идеальным?
}
\solutionspace{40pt}

\tasknumber{3}%
\task{%
    Изобразите в координатах $PV$/$VT$/$PT$ графики изохорического охлаждения в 2 раза (все 3 графика).
    Не забудьте указать оси и масштаб, начальную и конечную точки, направление движения на графике.
}
\solutionspace{100pt}

\tasknumber{4}%
\task{%
    Изобразите в координатах $PV$, соблюдая масштаб, процесс 1234,
    в котором 12 — изобарическое нагревание в 3 раза,
    23 — изотермическое сжатие в 3 раза,
    34 — изохорическое нагревание в 2 раза.
}
\solutionspace{160pt}

\tasknumber{5}%
\task{%
    Небольшую цилиндрическую пробирку с воздухом погружают на некоторую глубину в глубокое пресное озеро,
    после чего воздух занимает в ней лишь четвертую часть от общего объема.
    Определите глубину, на которую погрузили пробирку.
    Температуру считать постоянной $T = 291\,\text{К}$, давлением паров воды пренебречь,
    атмосферное давление принять равным $p_{\text{aтм}} = 100\,\text{кПа}$.
}
\answer{%
    \begin{align*}
    T\text{— const} &\implies P_1V_1 = \nu RT = P_2V_2.
    \\
    V_2 = \frac 14 V_1 &\implies P_1V_1 = P_2 \cdot \frac 14V_1 \implies P_2 = 4P_1 = 4p_{\text{aтм}}.
    \\
    P_2 = p_{\text{aтм}} + \rho_{\text{в}} g h \implies h = \frac{P_2 - p_{\text{aтм}}}{\rho_{\text{в}} g} &= \frac{4p_{\text{aтм}} - p_{\text{aтм}}}{\rho_{\text{в}} g} = \frac{3 \cdot p_{\text{aтм}}}{\rho_{\text{в}} g} =  \\
     &= \frac{3 \cdot 100\,\text{кПа}}{1000\,\frac{\text{кг}}{\text{м}^{3}} \cdot  10\,\frac{\text{м}}{\text{с}^{2}}} \approx 30\,\text{м}.
    \end{align*}
}
\solutionspace{120pt}

\tasknumber{6}%
\task{%
    В замкнутом сосуде объёмом $3\,\text{л}$ находится {gas.Name} ($\mu = 28\,\frac{\text{г}}{\text{моль}}$) под давлением $3{,}5\units{атм}$.
    Определите массу газа в сосуде и выразите её в граммах, приняв температуру газа равной $27\celsius$.
}
\answer{%
    $
        PV = \frac m\mu RT \implies m = \frac{PV \mu}{RT} =
        \frac{3{,}5\,\text{атм} \cdot 3\,\text{л} \cdot 28\,\frac{\text{г}}{\text{моль}}}{8{,}31\,\frac{\text{Дж}}{\text{моль}\cdot\text{К}} \cdot \cbr{27 + 273}\units{К}}
        \approx 11{,}79\,\text{г}.
    $
}
\solutionspace{120pt}

\tasknumber{7}%
\task{%
    Идеальный газ в экспериментальной установке подвергут политропному процессу $PV^n\text{— const}$
    с показателем политропы $n=1{,}5$.
    В одном из экспериментов объём газа увеличился в $3$ раза.
    Как при этом изменилась температура газа (выросла или уменьшилась, на сколько или во сколько раз)?
}
\answer{%
    \begin{align*}
    P_1V_1^n &= P_2V_2^n, P_1V_1 = \nu R T_1, P_2V_2 = \nu R T_2 \implies\frac{\nu R T_1}{V_1} V_1^n = \frac{\nu R T_2}{V_2} V_2^n \implies \\
    \implies T_1V_1^{n-1} &= T_2V_2^{n-1} \implies\frac{T_2}{T_1} = \cbr{\frac{V_1}{V_2}}^{n-1} \approx 0{,}577
    \end{align*}
}

\variantsplitter

\addpersonalvariant{Алексей Алимпиев}

\tasknumber{1}%
\task{%
    Из уравнения состояния идеального газа выведите или выразите...
    \begin{enumerate}
        \item давление,
        \item молярную массу,
        \item концентрацию молекул газа.
    \end{enumerate}
}

\tasknumber{2}%
\task{%
    Почему плазму (одно из агрегатных состояний вещества, при котором часть молекул распадаются на ионы и электроны) нельзя считать идеальным газом?
}
\solutionspace{40pt}

\tasknumber{3}%
\task{%
    Изобразите в координатах $PV$/$VT$/$PT$ графики изобарического сжатия в 2 раза (все 3 графика).
    Не забудьте указать оси и масштаб, начальную и конечную точки, направление движения на графике.
}
\solutionspace{100pt}

\tasknumber{4}%
\task{%
    Изобразите в координатах $PV$, соблюдая масштаб, процесс 1234,
    в котором 12 — изохорическое охлаждение в 2 раза,
    23 — изотермическое сжатие в 3 раза,
    34 — изохорическое нагревание в 3 раза.
}
\solutionspace{160pt}

\tasknumber{5}%
\task{%
    Небольшую цилиндрическую пробирку с воздухом погружают на некоторую глубину в глубокое пресное озеро,
    после чего воздух занимает в ней лишь третью часть от общего объема.
    Определите глубину, на которую погрузили пробирку.
    Температуру считать постоянной $T = 284\,\text{К}$, давлением паров воды пренебречь,
    атмосферное давление принять равным $p_{\text{aтм}} = 100\,\text{кПа}$.
}
\answer{%
    \begin{align*}
    T\text{— const} &\implies P_1V_1 = \nu RT = P_2V_2.
    \\
    V_2 = \frac 13 V_1 &\implies P_1V_1 = P_2 \cdot \frac 13V_1 \implies P_2 = 3P_1 = 3p_{\text{aтм}}.
    \\
    P_2 = p_{\text{aтм}} + \rho_{\text{в}} g h \implies h = \frac{P_2 - p_{\text{aтм}}}{\rho_{\text{в}} g} &= \frac{3p_{\text{aтм}} - p_{\text{aтм}}}{\rho_{\text{в}} g} = \frac{2 \cdot p_{\text{aтм}}}{\rho_{\text{в}} g} =  \\
     &= \frac{2 \cdot 100\,\text{кПа}}{1000\,\frac{\text{кг}}{\text{м}^{3}} \cdot  10\,\frac{\text{м}}{\text{с}^{2}}} \approx 20\,\text{м}.
    \end{align*}
}
\solutionspace{120pt}

\tasknumber{6}%
\task{%
    В замкнутом сосуде объёмом $2\,\text{л}$ находится {gas.Name} ($\mu = 32\,\frac{\text{г}}{\text{моль}}$) под давлением $4{,}5\units{атм}$.
    Определите массу газа в сосуде и выразите её в граммах, приняв температуру газа равной $7\celsius$.
}
\answer{%
    $
        PV = \frac m\mu RT \implies m = \frac{PV \mu}{RT} =
        \frac{4{,}5\,\text{атм} \cdot 2\,\text{л} \cdot 32\,\frac{\text{г}}{\text{моль}}}{8{,}31\,\frac{\text{Дж}}{\text{моль}\cdot\text{К}} \cdot \cbr{7 + 273}\units{К}}
        \approx 12{,}38\,\text{г}.
    $
}
\solutionspace{120pt}

\tasknumber{7}%
\task{%
    Идеальный газ в экспериментальной установке подвергут политропному процессу $PV^n\text{— const}$
    с показателем политропы $n=0{,}7$.
    В одном из экспериментов объём газа увеличился в $4$ раза.
    Как при этом изменилась температура газа (выросла или уменьшилась, на сколько или во сколько раз)?
}
\answer{%
    \begin{align*}
    P_1V_1^n &= P_2V_2^n, P_1V_1 = \nu R T_1, P_2V_2 = \nu R T_2 \implies\frac{\nu R T_1}{V_1} V_1^n = \frac{\nu R T_2}{V_2} V_2^n \implies \\
    \implies T_1V_1^{n-1} &= T_2V_2^{n-1} \implies\frac{T_2}{T_1} = \cbr{\frac{V_1}{V_2}}^{n-1} \approx 1{,}516
    \end{align*}
}

\variantsplitter

\addpersonalvariant{Евгений Васин}

\tasknumber{1}%
\task{%
    Из уравнения состояния идеального газа выведите или выразите...
    \begin{enumerate}
        \item давление,
        \item молярную массу,
        \item плотность газа.
    \end{enumerate}
}

\tasknumber{2}%
\task{%
    Почему плазму (одно из агрегатных состояний вещества, при котором часть молекул распадаются на ионы и электроны) нельзя считать идеальным газом?
}
\solutionspace{40pt}

\tasknumber{3}%
\task{%
    Изобразите в координатах $PV$/$VT$/$PT$ графики изотермического понижения давления в 2 раза (все 3 графика).
    Не забудьте указать оси и масштаб, начальную и конечную точки, направление движения на графике.
}
\solutionspace{100pt}

\tasknumber{4}%
\task{%
    Изобразите в координатах $PV$, соблюдая масштаб, процесс 1234,
    в котором 12 — изохорическое нагревание в 2 раза,
    23 — изотермическое расширение в 3 раза,
    34 — изохорическое охлаждение в 2 раза.
}
\solutionspace{160pt}

\tasknumber{5}%
\task{%
    Небольшую цилиндрическую пробирку с воздухом погружают на некоторую глубину в глубокое пресное озеро,
    после чего воздух занимает в ней лишь пятую часть от общего объема.
    Определите глубину, на которую погрузили пробирку.
    Температуру считать постоянной $T = 288\,\text{К}$, давлением паров воды пренебречь,
    атмосферное давление принять равным $p_{\text{aтм}} = 100\,\text{кПа}$.
}
\answer{%
    \begin{align*}
    T\text{— const} &\implies P_1V_1 = \nu RT = P_2V_2.
    \\
    V_2 = \frac 15 V_1 &\implies P_1V_1 = P_2 \cdot \frac 15V_1 \implies P_2 = 5P_1 = 5p_{\text{aтм}}.
    \\
    P_2 = p_{\text{aтм}} + \rho_{\text{в}} g h \implies h = \frac{P_2 - p_{\text{aтм}}}{\rho_{\text{в}} g} &= \frac{5p_{\text{aтм}} - p_{\text{aтм}}}{\rho_{\text{в}} g} = \frac{4 \cdot p_{\text{aтм}}}{\rho_{\text{в}} g} =  \\
     &= \frac{4 \cdot 100\,\text{кПа}}{1000\,\frac{\text{кг}}{\text{м}^{3}} \cdot  10\,\frac{\text{м}}{\text{с}^{2}}} \approx 40\,\text{м}.
    \end{align*}
}
\solutionspace{120pt}

\tasknumber{6}%
\task{%
    В замкнутом сосуде объёмом $5\,\text{л}$ находится {gas.Name} ($\mu = 44\,\frac{\text{г}}{\text{моль}}$) под давлением $4\units{атм}$.
    Определите массу газа в сосуде и выразите её в граммах, приняв температуру газа равной $47\celsius$.
}
\answer{%
    $
        PV = \frac m\mu RT \implies m = \frac{PV \mu}{RT} =
        \frac{4{,}0\,\text{атм} \cdot 5\,\text{л} \cdot 44\,\frac{\text{г}}{\text{моль}}}{8{,}31\,\frac{\text{Дж}}{\text{моль}\cdot\text{К}} \cdot \cbr{47 + 273}\units{К}}
        \approx 33{,}09\,\text{г}.
    $
}
\solutionspace{120pt}

\tasknumber{7}%
\task{%
    Идеальный газ в экспериментальной установке подвергут политропному процессу $PV^n\text{— const}$
    с показателем политропы $n=1{,}8$.
    В одном из экспериментов объём газа уменьшился в $2$ раза.
    Как при этом изменилась температура газа (выросла или уменьшилась, на сколько или во сколько раз)?
}
\answer{%
    \begin{align*}
    P_1V_1^n &= P_2V_2^n, P_1V_1 = \nu R T_1, P_2V_2 = \nu R T_2 \implies\frac{\nu R T_1}{V_1} V_1^n = \frac{\nu R T_2}{V_2} V_2^n \implies \\
    \implies T_1V_1^{n-1} &= T_2V_2^{n-1} \implies\frac{T_2}{T_1} = \cbr{\frac{V_1}{V_2}}^{n-1} \approx 1{,}741
    \end{align*}
}

\variantsplitter

\addpersonalvariant{Вячеслав Волохов}

\tasknumber{1}%
\task{%
    Из уравнения состояния идеального газа выведите или выразите...
    \begin{enumerate}
        \item объём,
        \item молярную массу,
        \item плотность газа.
    \end{enumerate}
}

\tasknumber{2}%
\task{%
    Почему плазму (одно из агрегатных состояний вещества, при котором часть молекул распадаются на ионы и электроны) нельзя считать идеальным газом?
}
\solutionspace{40pt}

\tasknumber{3}%
\task{%
    Изобразите в координатах $PV$/$VT$/$PT$ графики изотермического понижения давления в 2 раза (все 3 графика).
    Не забудьте указать оси и масштаб, начальную и конечную точки, направление движения на графике.
}
\solutionspace{100pt}

\tasknumber{4}%
\task{%
    Изобразите в координатах $PV$, соблюдая масштаб, процесс 1234,
    в котором 12 — изобарическое нагревание в 2 раза,
    23 — изотермическое расширение в 2 раза,
    34 — изохорическое нагревание в 2 раза.
}
\solutionspace{160pt}

\tasknumber{5}%
\task{%
    Небольшую цилиндрическую пробирку с воздухом погружают на некоторую глубину в глубокое пресное озеро,
    после чего воздух занимает в ней лишь третью часть от общего объема.
    Определите глубину, на которую погрузили пробирку.
    Температуру считать постоянной $T = 288\,\text{К}$, давлением паров воды пренебречь,
    атмосферное давление принять равным $p_{\text{aтм}} = 100\,\text{кПа}$.
}
\answer{%
    \begin{align*}
    T\text{— const} &\implies P_1V_1 = \nu RT = P_2V_2.
    \\
    V_2 = \frac 13 V_1 &\implies P_1V_1 = P_2 \cdot \frac 13V_1 \implies P_2 = 3P_1 = 3p_{\text{aтм}}.
    \\
    P_2 = p_{\text{aтм}} + \rho_{\text{в}} g h \implies h = \frac{P_2 - p_{\text{aтм}}}{\rho_{\text{в}} g} &= \frac{3p_{\text{aтм}} - p_{\text{aтм}}}{\rho_{\text{в}} g} = \frac{2 \cdot p_{\text{aтм}}}{\rho_{\text{в}} g} =  \\
     &= \frac{2 \cdot 100\,\text{кПа}}{1000\,\frac{\text{кг}}{\text{м}^{3}} \cdot  10\,\frac{\text{м}}{\text{с}^{2}}} \approx 20\,\text{м}.
    \end{align*}
}
\solutionspace{120pt}

\tasknumber{6}%
\task{%
    В замкнутом сосуде объёмом $2\,\text{л}$ находится {gas.Name} ($\mu = 20\,\frac{\text{г}}{\text{моль}}$) под давлением $4{,}5\units{атм}$.
    Определите массу газа в сосуде и выразите её в граммах, приняв температуру газа равной $27\celsius$.
}
\answer{%
    $
        PV = \frac m\mu RT \implies m = \frac{PV \mu}{RT} =
        \frac{4{,}5\,\text{атм} \cdot 2\,\text{л} \cdot 20\,\frac{\text{г}}{\text{моль}}}{8{,}31\,\frac{\text{Дж}}{\text{моль}\cdot\text{К}} \cdot \cbr{27 + 273}\units{К}}
        \approx 7{,}22\,\text{г}.
    $
}
\solutionspace{120pt}

\tasknumber{7}%
\task{%
    Идеальный газ в экспериментальной установке подвергут политропному процессу $PV^n\text{— const}$
    с показателем политропы $n=1{,}8$.
    В одном из экспериментов объём газа уменьшился в $3$ раза.
    Как при этом изменилась температура газа (выросла или уменьшилась, на сколько или во сколько раз)?
}
\answer{%
    \begin{align*}
    P_1V_1^n &= P_2V_2^n, P_1V_1 = \nu R T_1, P_2V_2 = \nu R T_2 \implies\frac{\nu R T_1}{V_1} V_1^n = \frac{\nu R T_2}{V_2} V_2^n \implies \\
    \implies T_1V_1^{n-1} &= T_2V_2^{n-1} \implies\frac{T_2}{T_1} = \cbr{\frac{V_1}{V_2}}^{n-1} \approx 2{,}408
    \end{align*}
}

\variantsplitter

\addpersonalvariant{Герман Говоров}

\tasknumber{1}%
\task{%
    Из уравнения состояния идеального газа выведите или выразите...
    \begin{enumerate}
        \item давление,
        \item температуру,
        \item концентрацию молекул газа.
    \end{enumerate}
}

\tasknumber{2}%
\task{%
    Почему плазму (одно из агрегатных состояний вещества, при котором часть молекул распадаются на ионы и электроны) нельзя считать идеальным газом?
}
\solutionspace{40pt}

\tasknumber{3}%
\task{%
    Изобразите в координатах $PV$/$VT$/$PT$ графики изотермического понижения давления в 3 раза (все 3 графика).
    Не забудьте указать оси и масштаб, начальную и конечную точки, направление движения на графике.
}
\solutionspace{100pt}

\tasknumber{4}%
\task{%
    Изобразите в координатах $PV$, соблюдая масштаб, процесс 1234,
    в котором 12 — изохорическое охлаждение в 2 раза,
    23 — изотермическое сжатие в 2 раза,
    34 — изохорическое нагревание в 3 раза.
}
\solutionspace{160pt}

\tasknumber{5}%
\task{%
    Небольшую цилиндрическую пробирку с воздухом погружают на некоторую глубину в глубокое пресное озеро,
    после чего воздух занимает в ней лишь шестую часть от общего объема.
    Определите глубину, на которую погрузили пробирку.
    Температуру считать постоянной $T = 280\,\text{К}$, давлением паров воды пренебречь,
    атмосферное давление принять равным $p_{\text{aтм}} = 100\,\text{кПа}$.
}
\answer{%
    \begin{align*}
    T\text{— const} &\implies P_1V_1 = \nu RT = P_2V_2.
    \\
    V_2 = \frac 16 V_1 &\implies P_1V_1 = P_2 \cdot \frac 16V_1 \implies P_2 = 6P_1 = 6p_{\text{aтм}}.
    \\
    P_2 = p_{\text{aтм}} + \rho_{\text{в}} g h \implies h = \frac{P_2 - p_{\text{aтм}}}{\rho_{\text{в}} g} &= \frac{6p_{\text{aтм}} - p_{\text{aтм}}}{\rho_{\text{в}} g} = \frac{5 \cdot p_{\text{aтм}}}{\rho_{\text{в}} g} =  \\
     &= \frac{5 \cdot 100\,\text{кПа}}{1000\,\frac{\text{кг}}{\text{м}^{3}} \cdot  10\,\frac{\text{м}}{\text{с}^{2}}} \approx 50\,\text{м}.
    \end{align*}
}
\solutionspace{120pt}

\tasknumber{6}%
\task{%
    В замкнутом сосуде объёмом $5\,\text{л}$ находится {gas.Name} ($\mu = 40\,\frac{\text{г}}{\text{моль}}$) под давлением $3\units{атм}$.
    Определите массу газа в сосуде и выразите её в граммах, приняв температуру газа равной $27\celsius$.
}
\answer{%
    $
        PV = \frac m\mu RT \implies m = \frac{PV \mu}{RT} =
        \frac{3{,}0\,\text{атм} \cdot 5\,\text{л} \cdot 40\,\frac{\text{г}}{\text{моль}}}{8{,}31\,\frac{\text{Дж}}{\text{моль}\cdot\text{К}} \cdot \cbr{27 + 273}\units{К}}
        \approx 24{,}07\,\text{г}.
    $
}
\solutionspace{120pt}

\tasknumber{7}%
\task{%
    Идеальный газ в экспериментальной установке подвергут политропному процессу $PV^n\text{— const}$
    с показателем политропы $n=1{,}5$.
    В одном из экспериментов объём газа уменьшился в $3$ раза.
    Как при этом изменилась температура газа (выросла или уменьшилась, на сколько или во сколько раз)?
}
\answer{%
    \begin{align*}
    P_1V_1^n &= P_2V_2^n, P_1V_1 = \nu R T_1, P_2V_2 = \nu R T_2 \implies\frac{\nu R T_1}{V_1} V_1^n = \frac{\nu R T_2}{V_2} V_2^n \implies \\
    \implies T_1V_1^{n-1} &= T_2V_2^{n-1} \implies\frac{T_2}{T_1} = \cbr{\frac{V_1}{V_2}}^{n-1} \approx 1{,}732
    \end{align*}
}

\variantsplitter

\addpersonalvariant{София Журавлёва}

\tasknumber{1}%
\task{%
    Из уравнения состояния идеального газа выведите или выразите...
    \begin{enumerate}
        \item давление,
        \item температуру,
        \item плотность газа.
    \end{enumerate}
}

\tasknumber{2}%
\task{%
    В межзвездном пространстве встречаются молекулы (до нескольких десятков на $\text{{м}}^3$).
    Почему такой газ нельзя считать идеальным?
}
\solutionspace{40pt}

\tasknumber{3}%
\task{%
    Изобразите в координатах $PV$/$VT$/$PT$ графики изохорического охлаждения в 2 раза (все 3 графика).
    Не забудьте указать оси и масштаб, начальную и конечную точки, направление движения на графике.
}
\solutionspace{100pt}

\tasknumber{4}%
\task{%
    Изобразите в координатах $PV$, соблюдая масштаб, процесс 1234,
    в котором 12 — изобарическое охлаждение в 2 раза,
    23 — изотермическое расширение в 3 раза,
    34 — изохорическое нагревание в 3 раза.
}
\solutionspace{160pt}

\tasknumber{5}%
\task{%
    Небольшую цилиндрическую пробирку с воздухом погружают на некоторую глубину в глубокое пресное озеро,
    после чего воздух занимает в ней лишь третью часть от общего объема.
    Определите глубину, на которую погрузили пробирку.
    Температуру считать постоянной $T = 283\,\text{К}$, давлением паров воды пренебречь,
    атмосферное давление принять равным $p_{\text{aтм}} = 100\,\text{кПа}$.
}
\answer{%
    \begin{align*}
    T\text{— const} &\implies P_1V_1 = \nu RT = P_2V_2.
    \\
    V_2 = \frac 13 V_1 &\implies P_1V_1 = P_2 \cdot \frac 13V_1 \implies P_2 = 3P_1 = 3p_{\text{aтм}}.
    \\
    P_2 = p_{\text{aтм}} + \rho_{\text{в}} g h \implies h = \frac{P_2 - p_{\text{aтм}}}{\rho_{\text{в}} g} &= \frac{3p_{\text{aтм}} - p_{\text{aтм}}}{\rho_{\text{в}} g} = \frac{2 \cdot p_{\text{aтм}}}{\rho_{\text{в}} g} =  \\
     &= \frac{2 \cdot 100\,\text{кПа}}{1000\,\frac{\text{кг}}{\text{м}^{3}} \cdot  10\,\frac{\text{м}}{\text{с}^{2}}} \approx 20\,\text{м}.
    \end{align*}
}
\solutionspace{120pt}

\tasknumber{6}%
\task{%
    В замкнутом сосуде объёмом $4\,\text{л}$ находится {gas.Name} ($\mu = 28\,\frac{\text{г}}{\text{моль}}$) под давлением $3{,}5\units{атм}$.
    Определите массу газа в сосуде и выразите её в граммах, приняв температуру газа равной $37\celsius$.
}
\answer{%
    $
        PV = \frac m\mu RT \implies m = \frac{PV \mu}{RT} =
        \frac{3{,}5\,\text{атм} \cdot 4\,\text{л} \cdot 28\,\frac{\text{г}}{\text{моль}}}{8{,}31\,\frac{\text{Дж}}{\text{моль}\cdot\text{К}} \cdot \cbr{37 + 273}\units{К}}
        \approx 15{,}22\,\text{г}.
    $
}
\solutionspace{120pt}

\tasknumber{7}%
\task{%
    Идеальный газ в экспериментальной установке подвергут политропному процессу $PV^n\text{— const}$
    с показателем политропы $n=0{,}4$.
    В одном из экспериментов объём газа уменьшился в $4$ раза.
    Как при этом изменилась температура газа (выросла или уменьшилась, на сколько или во сколько раз)?
}
\answer{%
    \begin{align*}
    P_1V_1^n &= P_2V_2^n, P_1V_1 = \nu R T_1, P_2V_2 = \nu R T_2 \implies\frac{\nu R T_1}{V_1} V_1^n = \frac{\nu R T_2}{V_2} V_2^n \implies \\
    \implies T_1V_1^{n-1} &= T_2V_2^{n-1} \implies\frac{T_2}{T_1} = \cbr{\frac{V_1}{V_2}}^{n-1} \approx 0{,}435
    \end{align*}
}

\variantsplitter

\addpersonalvariant{Константин Козлов}

\tasknumber{1}%
\task{%
    Из уравнения состояния идеального газа выведите или выразите...
    \begin{enumerate}
        \item объём,
        \item молярную массу,
        \item концентрацию молекул газа.
    \end{enumerate}
}

\tasknumber{2}%
\task{%
    В межзвездном пространстве встречаются молекулы (до нескольких десятков на $\text{{м}}^3$).
    Почему такой газ нельзя считать идеальным?
}
\solutionspace{40pt}

\tasknumber{3}%
\task{%
    Изобразите в координатах $PV$/$VT$/$PT$ графики изотермического повышения давления в 2 раза (все 3 графика).
    Не забудьте указать оси и масштаб, начальную и конечную точки, направление движения на графике.
}
\solutionspace{100pt}

\tasknumber{4}%
\task{%
    Изобразите в координатах $PV$, соблюдая масштаб, процесс 1234,
    в котором 12 — изохорическое охлаждение в 3 раза,
    23 — изотермическое сжатие в 2 раза,
    34 — изохорическое охлаждение в 3 раза.
}
\solutionspace{160pt}

\tasknumber{5}%
\task{%
    Небольшую цилиндрическую пробирку с воздухом погружают на некоторую глубину в глубокое пресное озеро,
    после чего воздух занимает в ней лишь четвертую часть от общего объема.
    Определите глубину, на которую погрузили пробирку.
    Температуру считать постоянной $T = 292\,\text{К}$, давлением паров воды пренебречь,
    атмосферное давление принять равным $p_{\text{aтм}} = 100\,\text{кПа}$.
}
\answer{%
    \begin{align*}
    T\text{— const} &\implies P_1V_1 = \nu RT = P_2V_2.
    \\
    V_2 = \frac 14 V_1 &\implies P_1V_1 = P_2 \cdot \frac 14V_1 \implies P_2 = 4P_1 = 4p_{\text{aтм}}.
    \\
    P_2 = p_{\text{aтм}} + \rho_{\text{в}} g h \implies h = \frac{P_2 - p_{\text{aтм}}}{\rho_{\text{в}} g} &= \frac{4p_{\text{aтм}} - p_{\text{aтм}}}{\rho_{\text{в}} g} = \frac{3 \cdot p_{\text{aтм}}}{\rho_{\text{в}} g} =  \\
     &= \frac{3 \cdot 100\,\text{кПа}}{1000\,\frac{\text{кг}}{\text{м}^{3}} \cdot  10\,\frac{\text{м}}{\text{с}^{2}}} \approx 30\,\text{м}.
    \end{align*}
}
\solutionspace{120pt}

\tasknumber{6}%
\task{%
    В замкнутом сосуде объёмом $3\,\text{л}$ находится {gas.Name} ($\mu = 29\,\frac{\text{г}}{\text{моль}}$) под давлением $3\units{атм}$.
    Определите массу газа в сосуде и выразите её в граммах, приняв температуру газа равной $7\celsius$.
}
\answer{%
    $
        PV = \frac m\mu RT \implies m = \frac{PV \mu}{RT} =
        \frac{3{,}0\,\text{атм} \cdot 3\,\text{л} \cdot 29\,\frac{\text{г}}{\text{моль}}}{8{,}31\,\frac{\text{Дж}}{\text{моль}\cdot\text{К}} \cdot \cbr{7 + 273}\units{К}}
        \approx 11{,}22\,\text{г}.
    $
}
\solutionspace{120pt}

\tasknumber{7}%
\task{%
    Идеальный газ в экспериментальной установке подвергут политропному процессу $PV^n\text{— const}$
    с показателем политропы $n=1{,}2$.
    В одном из экспериментов объём газа увеличился в $2$ раза.
    Как при этом изменилась температура газа (выросла или уменьшилась, на сколько или во сколько раз)?
}
\answer{%
    \begin{align*}
    P_1V_1^n &= P_2V_2^n, P_1V_1 = \nu R T_1, P_2V_2 = \nu R T_2 \implies\frac{\nu R T_1}{V_1} V_1^n = \frac{\nu R T_2}{V_2} V_2^n \implies \\
    \implies T_1V_1^{n-1} &= T_2V_2^{n-1} \implies\frac{T_2}{T_1} = \cbr{\frac{V_1}{V_2}}^{n-1} \approx 0{,}871
    \end{align*}
}

\variantsplitter

\addpersonalvariant{Наталья Кравченко}

\tasknumber{1}%
\task{%
    Из уравнения состояния идеального газа выведите или выразите...
    \begin{enumerate}
        \item объём,
        \item температуру,
        \item концентрацию молекул газа.
    \end{enumerate}
}

\tasknumber{2}%
\task{%
    В межзвездном пространстве встречаются молекулы (до нескольких десятков на $\text{{м}}^3$).
    Почему такой газ нельзя считать идеальным?
}
\solutionspace{40pt}

\tasknumber{3}%
\task{%
    Изобразите в координатах $PV$/$VT$/$PT$ графики изохорического нагрева в 2 раза (все 3 графика).
    Не забудьте указать оси и масштаб, начальную и конечную точки, направление движения на графике.
}
\solutionspace{100pt}

\tasknumber{4}%
\task{%
    Изобразите в координатах $PV$, соблюдая масштаб, процесс 1234,
    в котором 12 — изобарическое охлаждение в 3 раза,
    23 — изотермическое сжатие в 2 раза,
    34 — изохорическое охлаждение в 2 раза.
}
\solutionspace{160pt}

\tasknumber{5}%
\task{%
    Небольшую цилиндрическую пробирку с воздухом погружают на некоторую глубину в глубокое пресное озеро,
    после чего воздух занимает в ней лишь третью часть от общего объема.
    Определите глубину, на которую погрузили пробирку.
    Температуру считать постоянной $T = 283\,\text{К}$, давлением паров воды пренебречь,
    атмосферное давление принять равным $p_{\text{aтм}} = 100\,\text{кПа}$.
}
\answer{%
    \begin{align*}
    T\text{— const} &\implies P_1V_1 = \nu RT = P_2V_2.
    \\
    V_2 = \frac 13 V_1 &\implies P_1V_1 = P_2 \cdot \frac 13V_1 \implies P_2 = 3P_1 = 3p_{\text{aтм}}.
    \\
    P_2 = p_{\text{aтм}} + \rho_{\text{в}} g h \implies h = \frac{P_2 - p_{\text{aтм}}}{\rho_{\text{в}} g} &= \frac{3p_{\text{aтм}} - p_{\text{aтм}}}{\rho_{\text{в}} g} = \frac{2 \cdot p_{\text{aтм}}}{\rho_{\text{в}} g} =  \\
     &= \frac{2 \cdot 100\,\text{кПа}}{1000\,\frac{\text{кг}}{\text{м}^{3}} \cdot  10\,\frac{\text{м}}{\text{с}^{2}}} \approx 20\,\text{м}.
    \end{align*}
}
\solutionspace{120pt}

\tasknumber{6}%
\task{%
    В замкнутом сосуде объёмом $4\,\text{л}$ находится {gas.Name} ($\mu = 20\,\frac{\text{г}}{\text{моль}}$) под давлением $2{,}5\units{атм}$.
    Определите массу газа в сосуде и выразите её в граммах, приняв температуру газа равной $37\celsius$.
}
\answer{%
    $
        PV = \frac m\mu RT \implies m = \frac{PV \mu}{RT} =
        \frac{2{,}5\,\text{атм} \cdot 4\,\text{л} \cdot 20\,\frac{\text{г}}{\text{моль}}}{8{,}31\,\frac{\text{Дж}}{\text{моль}\cdot\text{К}} \cdot \cbr{37 + 273}\units{К}}
        \approx 7{,}76\,\text{г}.
    $
}
\solutionspace{120pt}

\tasknumber{7}%
\task{%
    Идеальный газ в экспериментальной установке подвергут политропному процессу $PV^n\text{— const}$
    с показателем политропы $n=1{,}8$.
    В одном из экспериментов объём газа увеличился в $3$ раза.
    Как при этом изменилась температура газа (выросла или уменьшилась, на сколько или во сколько раз)?
}
\answer{%
    \begin{align*}
    P_1V_1^n &= P_2V_2^n, P_1V_1 = \nu R T_1, P_2V_2 = \nu R T_2 \implies\frac{\nu R T_1}{V_1} V_1^n = \frac{\nu R T_2}{V_2} V_2^n \implies \\
    \implies T_1V_1^{n-1} &= T_2V_2^{n-1} \implies\frac{T_2}{T_1} = \cbr{\frac{V_1}{V_2}}^{n-1} \approx 0{,}415
    \end{align*}
}

\variantsplitter

\addpersonalvariant{Матвей Кузьмин}

\tasknumber{1}%
\task{%
    Из уравнения состояния идеального газа выведите или выразите...
    \begin{enumerate}
        \item давление,
        \item молярную массу,
        \item концентрацию молекул газа.
    \end{enumerate}
}

\tasknumber{2}%
\task{%
    Почему плазму (одно из агрегатных состояний вещества, при котором часть молекул распадаются на ионы и электроны) нельзя считать идеальным газом?
}
\solutionspace{40pt}

\tasknumber{3}%
\task{%
    Изобразите в координатах $PV$/$VT$/$PT$ графики изобарического расширения в 3 раза (все 3 графика).
    Не забудьте указать оси и масштаб, начальную и конечную точки, направление движения на графике.
}
\solutionspace{100pt}

\tasknumber{4}%
\task{%
    Изобразите в координатах $PV$, соблюдая масштаб, процесс 1234,
    в котором 12 — изохорическое нагревание в 2 раза,
    23 — изотермическое расширение в 3 раза,
    34 — изохорическое охлаждение в 2 раза.
}
\solutionspace{160pt}

\tasknumber{5}%
\task{%
    Небольшую цилиндрическую пробирку с воздухом погружают на некоторую глубину в глубокое пресное озеро,
    после чего воздух занимает в ней лишь третью часть от общего объема.
    Определите глубину, на которую погрузили пробирку.
    Температуру считать постоянной $T = 283\,\text{К}$, давлением паров воды пренебречь,
    атмосферное давление принять равным $p_{\text{aтм}} = 100\,\text{кПа}$.
}
\answer{%
    \begin{align*}
    T\text{— const} &\implies P_1V_1 = \nu RT = P_2V_2.
    \\
    V_2 = \frac 13 V_1 &\implies P_1V_1 = P_2 \cdot \frac 13V_1 \implies P_2 = 3P_1 = 3p_{\text{aтм}}.
    \\
    P_2 = p_{\text{aтм}} + \rho_{\text{в}} g h \implies h = \frac{P_2 - p_{\text{aтм}}}{\rho_{\text{в}} g} &= \frac{3p_{\text{aтм}} - p_{\text{aтм}}}{\rho_{\text{в}} g} = \frac{2 \cdot p_{\text{aтм}}}{\rho_{\text{в}} g} =  \\
     &= \frac{2 \cdot 100\,\text{кПа}}{1000\,\frac{\text{кг}}{\text{м}^{3}} \cdot  10\,\frac{\text{м}}{\text{с}^{2}}} \approx 20\,\text{м}.
    \end{align*}
}
\solutionspace{120pt}

\tasknumber{6}%
\task{%
    В замкнутом сосуде объёмом $5\,\text{л}$ находится {gas.Name} ($\mu = 44\,\frac{\text{г}}{\text{моль}}$) под давлением $3\units{атм}$.
    Определите массу газа в сосуде и выразите её в граммах, приняв температуру газа равной $17\celsius$.
}
\answer{%
    $
        PV = \frac m\mu RT \implies m = \frac{PV \mu}{RT} =
        \frac{3{,}0\,\text{атм} \cdot 5\,\text{л} \cdot 44\,\frac{\text{г}}{\text{моль}}}{8{,}31\,\frac{\text{Дж}}{\text{моль}\cdot\text{К}} \cdot \cbr{17 + 273}\units{К}}
        \approx 27{,}39\,\text{г}.
    $
}
\solutionspace{120pt}

\tasknumber{7}%
\task{%
    Идеальный газ в экспериментальной установке подвергут политропному процессу $PV^n\text{— const}$
    с показателем политропы $n=1{,}8$.
    В одном из экспериментов объём газа увеличился в $4$ раза.
    Как при этом изменилась температура газа (выросла или уменьшилась, на сколько или во сколько раз)?
}
\answer{%
    \begin{align*}
    P_1V_1^n &= P_2V_2^n, P_1V_1 = \nu R T_1, P_2V_2 = \nu R T_2 \implies\frac{\nu R T_1}{V_1} V_1^n = \frac{\nu R T_2}{V_2} V_2^n \implies \\
    \implies T_1V_1^{n-1} &= T_2V_2^{n-1} \implies\frac{T_2}{T_1} = \cbr{\frac{V_1}{V_2}}^{n-1} \approx 0{,}330
    \end{align*}
}

\variantsplitter

\addpersonalvariant{Сергей Малышев}

\tasknumber{1}%
\task{%
    Из уравнения состояния идеального газа выведите или выразите...
    \begin{enumerate}
        \item давление,
        \item молярную массу,
        \item концентрацию молекул газа.
    \end{enumerate}
}

\tasknumber{2}%
\task{%
    Почему плазму (одно из агрегатных состояний вещества, при котором часть молекул распадаются на ионы и электроны) нельзя считать идеальным газом?
}
\solutionspace{40pt}

\tasknumber{3}%
\task{%
    Изобразите в координатах $PV$/$VT$/$PT$ графики изохорического охлаждения в 2 раза (все 3 графика).
    Не забудьте указать оси и масштаб, начальную и конечную точки, направление движения на графике.
}
\solutionspace{100pt}

\tasknumber{4}%
\task{%
    Изобразите в координатах $PV$, соблюдая масштаб, процесс 1234,
    в котором 12 — изохорическое охлаждение в 2 раза,
    23 — изотермическое сжатие в 2 раза,
    34 — изохорическое нагревание в 3 раза.
}
\solutionspace{160pt}

\tasknumber{5}%
\task{%
    Небольшую цилиндрическую пробирку с воздухом погружают на некоторую глубину в глубокое пресное озеро,
    после чего воздух занимает в ней лишь четвертую часть от общего объема.
    Определите глубину, на которую погрузили пробирку.
    Температуру считать постоянной $T = 282\,\text{К}$, давлением паров воды пренебречь,
    атмосферное давление принять равным $p_{\text{aтм}} = 100\,\text{кПа}$.
}
\answer{%
    \begin{align*}
    T\text{— const} &\implies P_1V_1 = \nu RT = P_2V_2.
    \\
    V_2 = \frac 14 V_1 &\implies P_1V_1 = P_2 \cdot \frac 14V_1 \implies P_2 = 4P_1 = 4p_{\text{aтм}}.
    \\
    P_2 = p_{\text{aтм}} + \rho_{\text{в}} g h \implies h = \frac{P_2 - p_{\text{aтм}}}{\rho_{\text{в}} g} &= \frac{4p_{\text{aтм}} - p_{\text{aтм}}}{\rho_{\text{в}} g} = \frac{3 \cdot p_{\text{aтм}}}{\rho_{\text{в}} g} =  \\
     &= \frac{3 \cdot 100\,\text{кПа}}{1000\,\frac{\text{кг}}{\text{м}^{3}} \cdot  10\,\frac{\text{м}}{\text{с}^{2}}} \approx 30\,\text{м}.
    \end{align*}
}
\solutionspace{120pt}

\tasknumber{6}%
\task{%
    В замкнутом сосуде объёмом $2\,\text{л}$ находится {gas.Name} ($\mu = 29\,\frac{\text{г}}{\text{моль}}$) под давлением $3{,}5\units{атм}$.
    Определите массу газа в сосуде и выразите её в граммах, приняв температуру газа равной $17\celsius$.
}
\answer{%
    $
        PV = \frac m\mu RT \implies m = \frac{PV \mu}{RT} =
        \frac{3{,}5\,\text{атм} \cdot 2\,\text{л} \cdot 29\,\frac{\text{г}}{\text{моль}}}{8{,}31\,\frac{\text{Дж}}{\text{моль}\cdot\text{К}} \cdot \cbr{17 + 273}\units{К}}
        \approx 8{,}42\,\text{г}.
    $
}
\solutionspace{120pt}

\tasknumber{7}%
\task{%
    Идеальный газ в экспериментальной установке подвергут политропному процессу $PV^n\text{— const}$
    с показателем политропы $n=0{,}7$.
    В одном из экспериментов объём газа увеличился в $2$ раза.
    Как при этом изменилась температура газа (выросла или уменьшилась, на сколько или во сколько раз)?
}
\answer{%
    \begin{align*}
    P_1V_1^n &= P_2V_2^n, P_1V_1 = \nu R T_1, P_2V_2 = \nu R T_2 \implies\frac{\nu R T_1}{V_1} V_1^n = \frac{\nu R T_2}{V_2} V_2^n \implies \\
    \implies T_1V_1^{n-1} &= T_2V_2^{n-1} \implies\frac{T_2}{T_1} = \cbr{\frac{V_1}{V_2}}^{n-1} \approx 1{,}231
    \end{align*}
}

\variantsplitter

\addpersonalvariant{Алина Полканова}

\tasknumber{1}%
\task{%
    Из уравнения состояния идеального газа выведите или выразите...
    \begin{enumerate}
        \item давление,
        \item молярную массу,
        \item плотность газа.
    \end{enumerate}
}

\tasknumber{2}%
\task{%
    Почему плазму (одно из агрегатных состояний вещества, при котором часть молекул распадаются на ионы и электроны) нельзя считать идеальным газом?
}
\solutionspace{40pt}

\tasknumber{3}%
\task{%
    Изобразите в координатах $PV$/$VT$/$PT$ графики изохорического нагрева в 4 раза (все 3 графика).
    Не забудьте указать оси и масштаб, начальную и конечную точки, направление движения на графике.
}
\solutionspace{100pt}

\tasknumber{4}%
\task{%
    Изобразите в координатах $PV$, соблюдая масштаб, процесс 1234,
    в котором 12 — изобарическое нагревание в 3 раза,
    23 — изотермическое сжатие в 2 раза,
    34 — изохорическое охлаждение в 2 раза.
}
\solutionspace{160pt}

\tasknumber{5}%
\task{%
    Небольшую цилиндрическую пробирку с воздухом погружают на некоторую глубину в глубокое пресное озеро,
    после чего воздух занимает в ней лишь четвертую часть от общего объема.
    Определите глубину, на которую погрузили пробирку.
    Температуру считать постоянной $T = 284\,\text{К}$, давлением паров воды пренебречь,
    атмосферное давление принять равным $p_{\text{aтм}} = 100\,\text{кПа}$.
}
\answer{%
    \begin{align*}
    T\text{— const} &\implies P_1V_1 = \nu RT = P_2V_2.
    \\
    V_2 = \frac 14 V_1 &\implies P_1V_1 = P_2 \cdot \frac 14V_1 \implies P_2 = 4P_1 = 4p_{\text{aтм}}.
    \\
    P_2 = p_{\text{aтм}} + \rho_{\text{в}} g h \implies h = \frac{P_2 - p_{\text{aтм}}}{\rho_{\text{в}} g} &= \frac{4p_{\text{aтм}} - p_{\text{aтм}}}{\rho_{\text{в}} g} = \frac{3 \cdot p_{\text{aтм}}}{\rho_{\text{в}} g} =  \\
     &= \frac{3 \cdot 100\,\text{кПа}}{1000\,\frac{\text{кг}}{\text{м}^{3}} \cdot  10\,\frac{\text{м}}{\text{с}^{2}}} \approx 30\,\text{м}.
    \end{align*}
}
\solutionspace{120pt}

\tasknumber{6}%
\task{%
    В замкнутом сосуде объёмом $4\,\text{л}$ находится {gas.Name} ($\mu = 28\,\frac{\text{г}}{\text{моль}}$) под давлением $5\units{атм}$.
    Определите массу газа в сосуде и выразите её в граммах, приняв температуру газа равной $37\celsius$.
}
\answer{%
    $
        PV = \frac m\mu RT \implies m = \frac{PV \mu}{RT} =
        \frac{5{,}0\,\text{атм} \cdot 4\,\text{л} \cdot 28\,\frac{\text{г}}{\text{моль}}}{8{,}31\,\frac{\text{Дж}}{\text{моль}\cdot\text{К}} \cdot \cbr{37 + 273}\units{К}}
        \approx 21{,}74\,\text{г}.
    $
}
\solutionspace{120pt}

\tasknumber{7}%
\task{%
    Идеальный газ в экспериментальной установке подвергут политропному процессу $PV^n\text{— const}$
    с показателем политропы $n=0{,}7$.
    В одном из экспериментов объём газа уменьшился в $4$ раза.
    Как при этом изменилась температура газа (выросла или уменьшилась, на сколько или во сколько раз)?
}
\answer{%
    \begin{align*}
    P_1V_1^n &= P_2V_2^n, P_1V_1 = \nu R T_1, P_2V_2 = \nu R T_2 \implies\frac{\nu R T_1}{V_1} V_1^n = \frac{\nu R T_2}{V_2} V_2^n \implies \\
    \implies T_1V_1^{n-1} &= T_2V_2^{n-1} \implies\frac{T_2}{T_1} = \cbr{\frac{V_1}{V_2}}^{n-1} \approx 0{,}660
    \end{align*}
}

\variantsplitter

\addpersonalvariant{Сергей Пономарёв}

\tasknumber{1}%
\task{%
    Из уравнения состояния идеального газа выведите или выразите...
    \begin{enumerate}
        \item объём,
        \item температуру,
        \item концентрацию молекул газа.
    \end{enumerate}
}

\tasknumber{2}%
\task{%
    Почему плазму (одно из агрегатных состояний вещества, при котором часть молекул распадаются на ионы и электроны) нельзя считать идеальным газом?
}
\solutionspace{40pt}

\tasknumber{3}%
\task{%
    Изобразите в координатах $PV$/$VT$/$PT$ графики изотермического повышения давления в 3 раза (все 3 графика).
    Не забудьте указать оси и масштаб, начальную и конечную точки, направление движения на графике.
}
\solutionspace{100pt}

\tasknumber{4}%
\task{%
    Изобразите в координатах $PV$, соблюдая масштаб, процесс 1234,
    в котором 12 — изохорическое нагревание в 2 раза,
    23 — изотермическое расширение в 3 раза,
    34 — изобарическое нагревание в 2 раза.
}
\solutionspace{160pt}

\tasknumber{5}%
\task{%
    Небольшую цилиндрическую пробирку с воздухом погружают на некоторую глубину в глубокое пресное озеро,
    после чего воздух занимает в ней лишь пятую часть от общего объема.
    Определите глубину, на которую погрузили пробирку.
    Температуру считать постоянной $T = 286\,\text{К}$, давлением паров воды пренебречь,
    атмосферное давление принять равным $p_{\text{aтм}} = 100\,\text{кПа}$.
}
\answer{%
    \begin{align*}
    T\text{— const} &\implies P_1V_1 = \nu RT = P_2V_2.
    \\
    V_2 = \frac 15 V_1 &\implies P_1V_1 = P_2 \cdot \frac 15V_1 \implies P_2 = 5P_1 = 5p_{\text{aтм}}.
    \\
    P_2 = p_{\text{aтм}} + \rho_{\text{в}} g h \implies h = \frac{P_2 - p_{\text{aтм}}}{\rho_{\text{в}} g} &= \frac{5p_{\text{aтм}} - p_{\text{aтм}}}{\rho_{\text{в}} g} = \frac{4 \cdot p_{\text{aтм}}}{\rho_{\text{в}} g} =  \\
     &= \frac{4 \cdot 100\,\text{кПа}}{1000\,\frac{\text{кг}}{\text{м}^{3}} \cdot  10\,\frac{\text{м}}{\text{с}^{2}}} \approx 40\,\text{м}.
    \end{align*}
}
\solutionspace{120pt}

\tasknumber{6}%
\task{%
    В замкнутом сосуде объёмом $4\,\text{л}$ находится {gas.Name} ($\mu = 28\,\frac{\text{г}}{\text{моль}}$) под давлением $3\units{атм}$.
    Определите массу газа в сосуде и выразите её в граммах, приняв температуру газа равной $27\celsius$.
}
\answer{%
    $
        PV = \frac m\mu RT \implies m = \frac{PV \mu}{RT} =
        \frac{3{,}0\,\text{атм} \cdot 4\,\text{л} \cdot 28\,\frac{\text{г}}{\text{моль}}}{8{,}31\,\frac{\text{Дж}}{\text{моль}\cdot\text{К}} \cdot \cbr{27 + 273}\units{К}}
        \approx 13{,}48\,\text{г}.
    $
}
\solutionspace{120pt}

\tasknumber{7}%
\task{%
    Идеальный газ в экспериментальной установке подвергут политропному процессу $PV^n\text{— const}$
    с показателем политропы $n=1{,}8$.
    В одном из экспериментов объём газа уменьшился в $2$ раза.
    Как при этом изменилась температура газа (выросла или уменьшилась, на сколько или во сколько раз)?
}
\answer{%
    \begin{align*}
    P_1V_1^n &= P_2V_2^n, P_1V_1 = \nu R T_1, P_2V_2 = \nu R T_2 \implies\frac{\nu R T_1}{V_1} V_1^n = \frac{\nu R T_2}{V_2} V_2^n \implies \\
    \implies T_1V_1^{n-1} &= T_2V_2^{n-1} \implies\frac{T_2}{T_1} = \cbr{\frac{V_1}{V_2}}^{n-1} \approx 1{,}741
    \end{align*}
}

\variantsplitter

\addpersonalvariant{Егор Свистушкин}

\tasknumber{1}%
\task{%
    Из уравнения состояния идеального газа выведите или выразите...
    \begin{enumerate}
        \item давление,
        \item температуру,
        \item концентрацию молекул газа.
    \end{enumerate}
}

\tasknumber{2}%
\task{%
    Почему плазму (одно из агрегатных состояний вещества, при котором часть молекул распадаются на ионы и электроны) нельзя считать идеальным газом?
}
\solutionspace{40pt}

\tasknumber{3}%
\task{%
    Изобразите в координатах $PV$/$VT$/$PT$ графики изотермического понижения давления в 4 раза (все 3 графика).
    Не забудьте указать оси и масштаб, начальную и конечную точки, направление движения на графике.
}
\solutionspace{100pt}

\tasknumber{4}%
\task{%
    Изобразите в координатах $PV$, соблюдая масштаб, процесс 1234,
    в котором 12 — изохорическое нагревание в 3 раза,
    23 — изотермическое расширение в 2 раза,
    34 — изобарическое нагревание в 2 раза.
}
\solutionspace{160pt}

\tasknumber{5}%
\task{%
    Небольшую цилиндрическую пробирку с воздухом погружают на некоторую глубину в глубокое пресное озеро,
    после чего воздух занимает в ней лишь пятую часть от общего объема.
    Определите глубину, на которую погрузили пробирку.
    Температуру считать постоянной $T = 283\,\text{К}$, давлением паров воды пренебречь,
    атмосферное давление принять равным $p_{\text{aтм}} = 100\,\text{кПа}$.
}
\answer{%
    \begin{align*}
    T\text{— const} &\implies P_1V_1 = \nu RT = P_2V_2.
    \\
    V_2 = \frac 15 V_1 &\implies P_1V_1 = P_2 \cdot \frac 15V_1 \implies P_2 = 5P_1 = 5p_{\text{aтм}}.
    \\
    P_2 = p_{\text{aтм}} + \rho_{\text{в}} g h \implies h = \frac{P_2 - p_{\text{aтм}}}{\rho_{\text{в}} g} &= \frac{5p_{\text{aтм}} - p_{\text{aтм}}}{\rho_{\text{в}} g} = \frac{4 \cdot p_{\text{aтм}}}{\rho_{\text{в}} g} =  \\
     &= \frac{4 \cdot 100\,\text{кПа}}{1000\,\frac{\text{кг}}{\text{м}^{3}} \cdot  10\,\frac{\text{м}}{\text{с}^{2}}} \approx 40\,\text{м}.
    \end{align*}
}
\solutionspace{120pt}

\tasknumber{6}%
\task{%
    В замкнутом сосуде объёмом $5\,\text{л}$ находится {gas.Name} ($\mu = 32\,\frac{\text{г}}{\text{моль}}$) под давлением $4\units{атм}$.
    Определите массу газа в сосуде и выразите её в граммах, приняв температуру газа равной $17\celsius$.
}
\answer{%
    $
        PV = \frac m\mu RT \implies m = \frac{PV \mu}{RT} =
        \frac{4{,}0\,\text{атм} \cdot 5\,\text{л} \cdot 32\,\frac{\text{г}}{\text{моль}}}{8{,}31\,\frac{\text{Дж}}{\text{моль}\cdot\text{К}} \cdot \cbr{17 + 273}\units{К}}
        \approx 26{,}56\,\text{г}.
    $
}
\solutionspace{120pt}

\tasknumber{7}%
\task{%
    Идеальный газ в экспериментальной установке подвергут политропному процессу $PV^n\text{— const}$
    с показателем политропы $n=0{,}4$.
    В одном из экспериментов объём газа увеличился в $2$ раза.
    Как при этом изменилась температура газа (выросла или уменьшилась, на сколько или во сколько раз)?
}
\answer{%
    \begin{align*}
    P_1V_1^n &= P_2V_2^n, P_1V_1 = \nu R T_1, P_2V_2 = \nu R T_2 \implies\frac{\nu R T_1}{V_1} V_1^n = \frac{\nu R T_2}{V_2} V_2^n \implies \\
    \implies T_1V_1^{n-1} &= T_2V_2^{n-1} \implies\frac{T_2}{T_1} = \cbr{\frac{V_1}{V_2}}^{n-1} \approx 1{,}516
    \end{align*}
}

\variantsplitter

\addpersonalvariant{Дмитрий Соколов}

\tasknumber{1}%
\task{%
    Из уравнения состояния идеального газа выведите или выразите...
    \begin{enumerate}
        \item давление,
        \item молярную массу,
        \item концентрацию молекул газа.
    \end{enumerate}
}

\tasknumber{2}%
\task{%
    Почему плазму (одно из агрегатных состояний вещества, при котором часть молекул распадаются на ионы и электроны) нельзя считать идеальным газом?
}
\solutionspace{40pt}

\tasknumber{3}%
\task{%
    Изобразите в координатах $PV$/$VT$/$PT$ графики изохорического охлаждения в 3 раза (все 3 графика).
    Не забудьте указать оси и масштаб, начальную и конечную точки, направление движения на графике.
}
\solutionspace{100pt}

\tasknumber{4}%
\task{%
    Изобразите в координатах $PV$, соблюдая масштаб, процесс 1234,
    в котором 12 — изобарическое охлаждение в 2 раза,
    23 — изотермическое сжатие в 2 раза,
    34 — изохорическое нагревание в 3 раза.
}
\solutionspace{160pt}

\tasknumber{5}%
\task{%
    Небольшую цилиндрическую пробирку с воздухом погружают на некоторую глубину в глубокое пресное озеро,
    после чего воздух занимает в ней лишь шестую часть от общего объема.
    Определите глубину, на которую погрузили пробирку.
    Температуру считать постоянной $T = 286\,\text{К}$, давлением паров воды пренебречь,
    атмосферное давление принять равным $p_{\text{aтм}} = 100\,\text{кПа}$.
}
\answer{%
    \begin{align*}
    T\text{— const} &\implies P_1V_1 = \nu RT = P_2V_2.
    \\
    V_2 = \frac 16 V_1 &\implies P_1V_1 = P_2 \cdot \frac 16V_1 \implies P_2 = 6P_1 = 6p_{\text{aтм}}.
    \\
    P_2 = p_{\text{aтм}} + \rho_{\text{в}} g h \implies h = \frac{P_2 - p_{\text{aтм}}}{\rho_{\text{в}} g} &= \frac{6p_{\text{aтм}} - p_{\text{aтм}}}{\rho_{\text{в}} g} = \frac{5 \cdot p_{\text{aтм}}}{\rho_{\text{в}} g} =  \\
     &= \frac{5 \cdot 100\,\text{кПа}}{1000\,\frac{\text{кг}}{\text{м}^{3}} \cdot  10\,\frac{\text{м}}{\text{с}^{2}}} \approx 50\,\text{м}.
    \end{align*}
}
\solutionspace{120pt}

\tasknumber{6}%
\task{%
    В замкнутом сосуде объёмом $2\,\text{л}$ находится {gas.Name} ($\mu = 44\,\frac{\text{г}}{\text{моль}}$) под давлением $4\units{атм}$.
    Определите массу газа в сосуде и выразите её в граммах, приняв температуру газа равной $7\celsius$.
}
\answer{%
    $
        PV = \frac m\mu RT \implies m = \frac{PV \mu}{RT} =
        \frac{4{,}0\,\text{атм} \cdot 2\,\text{л} \cdot 44\,\frac{\text{г}}{\text{моль}}}{8{,}31\,\frac{\text{Дж}}{\text{моль}\cdot\text{К}} \cdot \cbr{7 + 273}\units{К}}
        \approx 15{,}13\,\text{г}.
    $
}
\solutionspace{120pt}

\tasknumber{7}%
\task{%
    Идеальный газ в экспериментальной установке подвергут политропному процессу $PV^n\text{— const}$
    с показателем политропы $n=1{,}2$.
    В одном из экспериментов объём газа уменьшился в $4$ раза.
    Как при этом изменилась температура газа (выросла или уменьшилась, на сколько или во сколько раз)?
}
\answer{%
    \begin{align*}
    P_1V_1^n &= P_2V_2^n, P_1V_1 = \nu R T_1, P_2V_2 = \nu R T_2 \implies\frac{\nu R T_1}{V_1} V_1^n = \frac{\nu R T_2}{V_2} V_2^n \implies \\
    \implies T_1V_1^{n-1} &= T_2V_2^{n-1} \implies\frac{T_2}{T_1} = \cbr{\frac{V_1}{V_2}}^{n-1} \approx 1{,}320
    \end{align*}
}
% autogenerated
