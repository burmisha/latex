\setdate{14~мая~2021}
\setclass{10«АБ»}

\addpersonalvariant{Михаил Бурмистров}

\tasknumber{1}%
\task{%
    Два резистора сопротивлениями $R_1=5R$ и $R_2=8R$ подключены последовательно к источнику напряжения.
    Определите, в каком резисторе выделяется большая тепловая мощность и во сколько раз?
}
\solutionspace{120pt}

\tasknumber{2}%
\task{%
    Если батарею замкнуть на резистор сопротивлением $R_1$, то в цепи потечёт ток $\eli_1$,
    а если на другой $R_2$ — то $\eli_2$.
    Определите:
    \begin{itemize}
        \item ЭДС батареи,
        \item внутреннее сопротивление батареи,
        \item ток короткого замыкания.
    \end{itemize}
}
\solutionspace{120pt}

\tasknumber{3}%
\task{%
    Определите ток, протекающий через резистор $R_1$, разность потенциалов на нём (см.
    рис.)
    и выделяющуюся на нём мощность, если известны $r_1, r_2, \ele_1, \ele_2, R_1, R_2$.

    \begin{tikzpicture}[circuit ee IEC, thick]
        \draw  (0, 0) -- ++(up:2)
                to [
                    battery={very near start, rotate=-180, info={$\ele_1, r_1 $}},
                    resistor={midway, info=$R_1$},
                    battery={very near end, rotate=0, info={$\ele_2, r_2 $}}
                ] ++(right:5)
                -- ++(down:2)
                to [resistor={info=$R_2$}] ++(left:5);
    \end{tikzpicture}
}

\variantsplitter

\addpersonalvariant{Ирина Ан}

\tasknumber{1}%
\task{%
    Два резистора сопротивлениями $R_1=5R$ и $R_2=2R$ подключены последовательно к источнику напряжения.
    Определите, в каком резисторе выделяется большая тепловая мощность и во сколько раз?
}
\solutionspace{120pt}

\tasknumber{2}%
\task{%
    Если батарею замкнуть на резистор сопротивлением $R_1$, то в цепи потечёт ток $\eli_1$,
    а если на другой $R_2$ — то $\eli_2$.
    Определите:
    \begin{itemize}
        \item ЭДС батареи,
        \item внутреннее сопротивление батареи,
        \item ток короткого замыкания.
    \end{itemize}
}
\solutionspace{120pt}

\tasknumber{3}%
\task{%
    Определите ток, протекающий через резистор $R_2$, разность потенциалов на нём (см.
    рис.)
    и выделяющуюся на нём мощность, если известны $r_1, r_2, \ele_1, \ele_2, R_1, R_2$.

    \begin{tikzpicture}[circuit ee IEC, thick]
        \draw  (0, 0) -- ++(up:2)
                to [
                    battery={very near start, rotate=0, info={$\ele_1, r_1 $}},
                    resistor={midway, info=$R_1$},
                    battery={very near end, rotate=0, info={$\ele_2, r_2 $}}
                ] ++(right:5)
                -- ++(down:2)
                to [resistor={info=$R_2$}] ++(left:5);
    \end{tikzpicture}
}

\variantsplitter

\addpersonalvariant{Софья Андрианова}

\tasknumber{1}%
\task{%
    Два резистора сопротивлениями $R_1=5R$ и $R_2=2R$ подключены параллельно к источнику напряжения.
    Определите, в каком резисторе выделяется большая тепловая мощность и во сколько раз?
}
\solutionspace{120pt}

\tasknumber{2}%
\task{%
    Если батарею замкнуть на резистор сопротивлением $R_1$, то в цепи потечёт ток $\eli_1$,
    а если на другой $R_2$ — то $\eli_2$.
    Определите:
    \begin{itemize}
        \item ЭДС батареи,
        \item внутреннее сопротивление батареи,
        \item ток короткого замыкания.
    \end{itemize}
}
\solutionspace{120pt}

\tasknumber{3}%
\task{%
    Определите ток, протекающий через резистор $R_1$, разность потенциалов на нём (см.
    рис.)
    и выделяющуюся на нём мощность, если известны $r_1, r_2, \ele_1, \ele_2, R_1, R_2$.

    \begin{tikzpicture}[circuit ee IEC, thick]
        \draw  (0, 0) -- ++(up:2)
                to [
                    battery={very near start, rotate=0, info={$\ele_1, r_1 $}},
                    resistor={midway, info=$R_1$},
                    battery={very near end, rotate=0, info={$\ele_2, r_2 $}}
                ] ++(right:5)
                -- ++(down:2)
                to [resistor={info=$R_2$}] ++(left:5);
    \end{tikzpicture}
}

\variantsplitter

\addpersonalvariant{Владимир Артемчук}

\tasknumber{1}%
\task{%
    Два резистора сопротивлениями $R_1=3R$ и $R_2=2R$ подключены параллельно к источнику напряжения.
    Определите, в каком резисторе выделяется большая тепловая мощность и во сколько раз?
}
\solutionspace{120pt}

\tasknumber{2}%
\task{%
    Если батарею замкнуть на резистор сопротивлением $R_1$, то в цепи потечёт ток $\eli_1$,
    а если на другой $R_2$ — то $\eli_2$.
    Определите:
    \begin{itemize}
        \item ЭДС батареи,
        \item внутреннее сопротивление батареи,
        \item ток короткого замыкания.
    \end{itemize}
}
\solutionspace{120pt}

\tasknumber{3}%
\task{%
    Определите ток, протекающий через резистор $R_1$, разность потенциалов на нём (см.
    рис.)
    и выделяющуюся на нём мощность, если известны $r_1, r_2, \ele_1, \ele_2, R_1, R_2$.

    \begin{tikzpicture}[circuit ee IEC, thick]
        \draw  (0, 0) -- ++(up:2)
                to [
                    battery={very near start, rotate=0, info={$\ele_1, r_1 $}},
                    resistor={midway, info=$R_1$},
                    battery={very near end, rotate=-180, info={$\ele_2, r_2 $}}
                ] ++(right:5)
                -- ++(down:2)
                to [resistor={info=$R_2$}] ++(left:5);
    \end{tikzpicture}
}

\variantsplitter

\addpersonalvariant{Софья Белянкина}

\tasknumber{1}%
\task{%
    Два резистора сопротивлениями $R_1=5R$ и $R_2=2R$ подключены последовательно к источнику напряжения.
    Определите, в каком резисторе выделяется большая тепловая мощность и во сколько раз?
}
\solutionspace{120pt}

\tasknumber{2}%
\task{%
    Если батарею замкнуть на резистор сопротивлением $R_1$, то в цепи потечёт ток $\eli_1$,
    а если на другой $R_2$ — то $\eli_2$.
    Определите:
    \begin{itemize}
        \item ЭДС батареи,
        \item внутреннее сопротивление батареи,
        \item ток короткого замыкания.
    \end{itemize}
}
\solutionspace{120pt}

\tasknumber{3}%
\task{%
    Определите ток, протекающий через резистор $R_2$, разность потенциалов на нём (см.
    рис.)
    и выделяющуюся на нём мощность, если известны $r_1, r_2, \ele_1, \ele_2, R_1, R_2$.

    \begin{tikzpicture}[circuit ee IEC, thick]
        \draw  (0, 0) -- ++(up:2)
                to [
                    battery={very near start, rotate=-180, info={$\ele_1, r_1 $}},
                    resistor={midway, info=$R_1$},
                    battery={very near end, rotate=-180, info={$\ele_2, r_2 $}}
                ] ++(right:5)
                -- ++(down:2)
                to [resistor={info=$R_2$}] ++(left:5);
    \end{tikzpicture}
}

\variantsplitter

\addpersonalvariant{Варвара Егиазарян}

\tasknumber{1}%
\task{%
    Два резистора сопротивлениями $R_1=5R$ и $R_2=2R$ подключены последовательно к источнику напряжения.
    Определите, в каком резисторе выделяется большая тепловая мощность и во сколько раз?
}
\solutionspace{120pt}

\tasknumber{2}%
\task{%
    Если батарею замкнуть на резистор сопротивлением $R_1$, то в цепи потечёт ток $\eli_1$,
    а если на другой $R_2$ — то $\eli_2$.
    Определите:
    \begin{itemize}
        \item ЭДС батареи,
        \item внутреннее сопротивление батареи,
        \item ток короткого замыкания.
    \end{itemize}
}
\solutionspace{120pt}

\tasknumber{3}%
\task{%
    Определите ток, протекающий через резистор $R_2$, разность потенциалов на нём (см.
    рис.)
    и выделяющуюся на нём мощность, если известны $r_1, r_2, \ele_1, \ele_2, R_1, R_2$.

    \begin{tikzpicture}[circuit ee IEC, thick]
        \draw  (0, 0) -- ++(up:2)
                to [
                    battery={very near start, rotate=0, info={$\ele_1, r_1 $}},
                    resistor={midway, info=$R_1$},
                    battery={very near end, rotate=0, info={$\ele_2, r_2 $}}
                ] ++(right:5)
                -- ++(down:2)
                to [resistor={info=$R_2$}] ++(left:5);
    \end{tikzpicture}
}

\variantsplitter

\addpersonalvariant{Владислав Емелин}

\tasknumber{1}%
\task{%
    Два резистора сопротивлениями $R_1=3R$ и $R_2=2R$ подключены последовательно к источнику напряжения.
    Определите, в каком резисторе выделяется большая тепловая мощность и во сколько раз?
}
\solutionspace{120pt}

\tasknumber{2}%
\task{%
    Если батарею замкнуть на резистор сопротивлением $R_1$, то в цепи потечёт ток $\eli_1$,
    а если на другой $R_2$ — то $\eli_2$.
    Определите:
    \begin{itemize}
        \item ЭДС батареи,
        \item внутреннее сопротивление батареи,
        \item ток короткого замыкания.
    \end{itemize}
}
\solutionspace{120pt}

\tasknumber{3}%
\task{%
    Определите ток, протекающий через резистор $R_1$, разность потенциалов на нём (см.
    рис.)
    и выделяющуюся на нём мощность, если известны $r_1, r_2, \ele_1, \ele_2, R_1, R_2$.

    \begin{tikzpicture}[circuit ee IEC, thick]
        \draw  (0, 0) -- ++(up:2)
                to [
                    battery={very near start, rotate=0, info={$\ele_1, r_1 $}},
                    resistor={midway, info=$R_1$},
                    battery={very near end, rotate=-180, info={$\ele_2, r_2 $}}
                ] ++(right:5)
                -- ++(down:2)
                to [resistor={info=$R_2$}] ++(left:5);
    \end{tikzpicture}
}

\variantsplitter

\addpersonalvariant{Артём Жичин}

\tasknumber{1}%
\task{%
    Два резистора сопротивлениями $R_1=3R$ и $R_2=4R$ подключены параллельно к источнику напряжения.
    Определите, в каком резисторе выделяется большая тепловая мощность и во сколько раз?
}
\solutionspace{120pt}

\tasknumber{2}%
\task{%
    Если батарею замкнуть на резистор сопротивлением $R_1$, то в цепи потечёт ток $\eli_1$,
    а если на другой $R_2$ — то $\eli_2$.
    Определите:
    \begin{itemize}
        \item ЭДС батареи,
        \item внутреннее сопротивление батареи,
        \item ток короткого замыкания.
    \end{itemize}
}
\solutionspace{120pt}

\tasknumber{3}%
\task{%
    Определите ток, протекающий через резистор $R_1$, разность потенциалов на нём (см.
    рис.)
    и выделяющуюся на нём мощность, если известны $r_1, r_2, \ele_1, \ele_2, R_1, R_2$.

    \begin{tikzpicture}[circuit ee IEC, thick]
        \draw  (0, 0) -- ++(up:2)
                to [
                    battery={very near start, rotate=-180, info={$\ele_1, r_1 $}},
                    resistor={midway, info=$R_1$},
                    battery={very near end, rotate=0, info={$\ele_2, r_2 $}}
                ] ++(right:5)
                -- ++(down:2)
                to [resistor={info=$R_2$}] ++(left:5);
    \end{tikzpicture}
}

\variantsplitter

\addpersonalvariant{Дарья Кошман}

\tasknumber{1}%
\task{%
    Два резистора сопротивлениями $R_1=3R$ и $R_2=2R$ подключены последовательно к источнику напряжения.
    Определите, в каком резисторе выделяется большая тепловая мощность и во сколько раз?
}
\solutionspace{120pt}

\tasknumber{2}%
\task{%
    Если батарею замкнуть на резистор сопротивлением $R_1$, то в цепи потечёт ток $\eli_1$,
    а если на другой $R_2$ — то $\eli_2$.
    Определите:
    \begin{itemize}
        \item ЭДС батареи,
        \item внутреннее сопротивление батареи,
        \item ток короткого замыкания.
    \end{itemize}
}
\solutionspace{120pt}

\tasknumber{3}%
\task{%
    Определите ток, протекающий через резистор $R_1$, разность потенциалов на нём (см.
    рис.)
    и выделяющуюся на нём мощность, если известны $r_1, r_2, \ele_1, \ele_2, R_1, R_2$.

    \begin{tikzpicture}[circuit ee IEC, thick]
        \draw  (0, 0) -- ++(up:2)
                to [
                    battery={very near start, rotate=-180, info={$\ele_1, r_1 $}},
                    resistor={midway, info=$R_1$},
                    battery={very near end, rotate=-180, info={$\ele_2, r_2 $}}
                ] ++(right:5)
                -- ++(down:2)
                to [resistor={info=$R_2$}] ++(left:5);
    \end{tikzpicture}
}

\variantsplitter

\addpersonalvariant{Анна Кузьмичёва}

\tasknumber{1}%
\task{%
    Два резистора сопротивлениями $R_1=3R$ и $R_2=2R$ подключены параллельно к источнику напряжения.
    Определите, в каком резисторе выделяется большая тепловая мощность и во сколько раз?
}
\solutionspace{120pt}

\tasknumber{2}%
\task{%
    Если батарею замкнуть на резистор сопротивлением $R_1$, то в цепи потечёт ток $\eli_1$,
    а если на другой $R_2$ — то $\eli_2$.
    Определите:
    \begin{itemize}
        \item ЭДС батареи,
        \item внутреннее сопротивление батареи,
        \item ток короткого замыкания.
    \end{itemize}
}
\solutionspace{120pt}

\tasknumber{3}%
\task{%
    Определите ток, протекающий через резистор $R_1$, разность потенциалов на нём (см.
    рис.)
    и выделяющуюся на нём мощность, если известны $r_1, r_2, \ele_1, \ele_2, R_1, R_2$.

    \begin{tikzpicture}[circuit ee IEC, thick]
        \draw  (0, 0) -- ++(up:2)
                to [
                    battery={very near start, rotate=-180, info={$\ele_1, r_1 $}},
                    resistor={midway, info=$R_1$},
                    battery={very near end, rotate=-180, info={$\ele_2, r_2 $}}
                ] ++(right:5)
                -- ++(down:2)
                to [resistor={info=$R_2$}] ++(left:5);
    \end{tikzpicture}
}

\variantsplitter

\addpersonalvariant{Алёна Куприянова}

\tasknumber{1}%
\task{%
    Два резистора сопротивлениями $R_1=3R$ и $R_2=8R$ подключены последовательно к источнику напряжения.
    Определите, в каком резисторе выделяется большая тепловая мощность и во сколько раз?
}
\solutionspace{120pt}

\tasknumber{2}%
\task{%
    Если батарею замкнуть на резистор сопротивлением $R_1$, то в цепи потечёт ток $\eli_1$,
    а если на другой $R_2$ — то $\eli_2$.
    Определите:
    \begin{itemize}
        \item ЭДС батареи,
        \item внутреннее сопротивление батареи,
        \item ток короткого замыкания.
    \end{itemize}
}
\solutionspace{120pt}

\tasknumber{3}%
\task{%
    Определите ток, протекающий через резистор $R_2$, разность потенциалов на нём (см.
    рис.)
    и выделяющуюся на нём мощность, если известны $r_1, r_2, \ele_1, \ele_2, R_1, R_2$.

    \begin{tikzpicture}[circuit ee IEC, thick]
        \draw  (0, 0) -- ++(up:2)
                to [
                    battery={very near start, rotate=-180, info={$\ele_1, r_1 $}},
                    resistor={midway, info=$R_1$},
                    battery={very near end, rotate=0, info={$\ele_2, r_2 $}}
                ] ++(right:5)
                -- ++(down:2)
                to [resistor={info=$R_2$}] ++(left:5);
    \end{tikzpicture}
}

\variantsplitter

\addpersonalvariant{Ярослав Лавровский}

\tasknumber{1}%
\task{%
    Два резистора сопротивлениями $R_1=5R$ и $R_2=4R$ подключены параллельно к источнику напряжения.
    Определите, в каком резисторе выделяется большая тепловая мощность и во сколько раз?
}
\solutionspace{120pt}

\tasknumber{2}%
\task{%
    Если батарею замкнуть на резистор сопротивлением $R_1$, то в цепи потечёт ток $\eli_1$,
    а если на другой $R_2$ — то $\eli_2$.
    Определите:
    \begin{itemize}
        \item ЭДС батареи,
        \item внутреннее сопротивление батареи,
        \item ток короткого замыкания.
    \end{itemize}
}
\solutionspace{120pt}

\tasknumber{3}%
\task{%
    Определите ток, протекающий через резистор $R_1$, разность потенциалов на нём (см.
    рис.)
    и выделяющуюся на нём мощность, если известны $r_1, r_2, \ele_1, \ele_2, R_1, R_2$.

    \begin{tikzpicture}[circuit ee IEC, thick]
        \draw  (0, 0) -- ++(up:2)
                to [
                    battery={very near start, rotate=-180, info={$\ele_1, r_1 $}},
                    resistor={midway, info=$R_1$},
                    battery={very near end, rotate=0, info={$\ele_2, r_2 $}}
                ] ++(right:5)
                -- ++(down:2)
                to [resistor={info=$R_2$}] ++(left:5);
    \end{tikzpicture}
}

\variantsplitter

\addpersonalvariant{Анастасия Ламанова}

\tasknumber{1}%
\task{%
    Два резистора сопротивлениями $R_1=3R$ и $R_2=8R$ подключены последовательно к источнику напряжения.
    Определите, в каком резисторе выделяется большая тепловая мощность и во сколько раз?
}
\solutionspace{120pt}

\tasknumber{2}%
\task{%
    Если батарею замкнуть на резистор сопротивлением $R_1$, то в цепи потечёт ток $\eli_1$,
    а если на другой $R_2$ — то $\eli_2$.
    Определите:
    \begin{itemize}
        \item ЭДС батареи,
        \item внутреннее сопротивление батареи,
        \item ток короткого замыкания.
    \end{itemize}
}
\solutionspace{120pt}

\tasknumber{3}%
\task{%
    Определите ток, протекающий через резистор $R_1$, разность потенциалов на нём (см.
    рис.)
    и выделяющуюся на нём мощность, если известны $r_1, r_2, \ele_1, \ele_2, R_1, R_2$.

    \begin{tikzpicture}[circuit ee IEC, thick]
        \draw  (0, 0) -- ++(up:2)
                to [
                    battery={very near start, rotate=0, info={$\ele_1, r_1 $}},
                    resistor={midway, info=$R_1$},
                    battery={very near end, rotate=-180, info={$\ele_2, r_2 $}}
                ] ++(right:5)
                -- ++(down:2)
                to [resistor={info=$R_2$}] ++(left:5);
    \end{tikzpicture}
}

\variantsplitter

\addpersonalvariant{Виктория Легонькова}

\tasknumber{1}%
\task{%
    Два резистора сопротивлениями $R_1=5R$ и $R_2=4R$ подключены последовательно к источнику напряжения.
    Определите, в каком резисторе выделяется большая тепловая мощность и во сколько раз?
}
\solutionspace{120pt}

\tasknumber{2}%
\task{%
    Если батарею замкнуть на резистор сопротивлением $R_1$, то в цепи потечёт ток $\eli_1$,
    а если на другой $R_2$ — то $\eli_2$.
    Определите:
    \begin{itemize}
        \item ЭДС батареи,
        \item внутреннее сопротивление батареи,
        \item ток короткого замыкания.
    \end{itemize}
}
\solutionspace{120pt}

\tasknumber{3}%
\task{%
    Определите ток, протекающий через резистор $R_1$, разность потенциалов на нём (см.
    рис.)
    и выделяющуюся на нём мощность, если известны $r_1, r_2, \ele_1, \ele_2, R_1, R_2$.

    \begin{tikzpicture}[circuit ee IEC, thick]
        \draw  (0, 0) -- ++(up:2)
                to [
                    battery={very near start, rotate=0, info={$\ele_1, r_1 $}},
                    resistor={midway, info=$R_1$},
                    battery={very near end, rotate=0, info={$\ele_2, r_2 $}}
                ] ++(right:5)
                -- ++(down:2)
                to [resistor={info=$R_2$}] ++(left:5);
    \end{tikzpicture}
}

\variantsplitter

\addpersonalvariant{Семён Мартынов}

\tasknumber{1}%
\task{%
    Два резистора сопротивлениями $R_1=7R$ и $R_2=4R$ подключены последовательно к источнику напряжения.
    Определите, в каком резисторе выделяется большая тепловая мощность и во сколько раз?
}
\solutionspace{120pt}

\tasknumber{2}%
\task{%
    Если батарею замкнуть на резистор сопротивлением $R_1$, то в цепи потечёт ток $\eli_1$,
    а если на другой $R_2$ — то $\eli_2$.
    Определите:
    \begin{itemize}
        \item ЭДС батареи,
        \item внутреннее сопротивление батареи,
        \item ток короткого замыкания.
    \end{itemize}
}
\solutionspace{120pt}

\tasknumber{3}%
\task{%
    Определите ток, протекающий через резистор $R_2$, разность потенциалов на нём (см.
    рис.)
    и выделяющуюся на нём мощность, если известны $r_1, r_2, \ele_1, \ele_2, R_1, R_2$.

    \begin{tikzpicture}[circuit ee IEC, thick]
        \draw  (0, 0) -- ++(up:2)
                to [
                    battery={very near start, rotate=-180, info={$\ele_1, r_1 $}},
                    resistor={midway, info=$R_1$},
                    battery={very near end, rotate=-180, info={$\ele_2, r_2 $}}
                ] ++(right:5)
                -- ++(down:2)
                to [resistor={info=$R_2$}] ++(left:5);
    \end{tikzpicture}
}

\variantsplitter

\addpersonalvariant{Варвара Минаева}

\tasknumber{1}%
\task{%
    Два резистора сопротивлениями $R_1=5R$ и $R_2=6R$ подключены последовательно к источнику напряжения.
    Определите, в каком резисторе выделяется большая тепловая мощность и во сколько раз?
}
\solutionspace{120pt}

\tasknumber{2}%
\task{%
    Если батарею замкнуть на резистор сопротивлением $R_1$, то в цепи потечёт ток $\eli_1$,
    а если на другой $R_2$ — то $\eli_2$.
    Определите:
    \begin{itemize}
        \item ЭДС батареи,
        \item внутреннее сопротивление батареи,
        \item ток короткого замыкания.
    \end{itemize}
}
\solutionspace{120pt}

\tasknumber{3}%
\task{%
    Определите ток, протекающий через резистор $R_1$, разность потенциалов на нём (см.
    рис.)
    и выделяющуюся на нём мощность, если известны $r_1, r_2, \ele_1, \ele_2, R_1, R_2$.

    \begin{tikzpicture}[circuit ee IEC, thick]
        \draw  (0, 0) -- ++(up:2)
                to [
                    battery={very near start, rotate=-180, info={$\ele_1, r_1 $}},
                    resistor={midway, info=$R_1$},
                    battery={very near end, rotate=-180, info={$\ele_2, r_2 $}}
                ] ++(right:5)
                -- ++(down:2)
                to [resistor={info=$R_2$}] ++(left:5);
    \end{tikzpicture}
}

\variantsplitter

\addpersonalvariant{Леонид Никитин}

\tasknumber{1}%
\task{%
    Два резистора сопротивлениями $R_1=3R$ и $R_2=4R$ подключены параллельно к источнику напряжения.
    Определите, в каком резисторе выделяется большая тепловая мощность и во сколько раз?
}
\solutionspace{120pt}

\tasknumber{2}%
\task{%
    Если батарею замкнуть на резистор сопротивлением $R_1$, то в цепи потечёт ток $\eli_1$,
    а если на другой $R_2$ — то $\eli_2$.
    Определите:
    \begin{itemize}
        \item ЭДС батареи,
        \item внутреннее сопротивление батареи,
        \item ток короткого замыкания.
    \end{itemize}
}
\solutionspace{120pt}

\tasknumber{3}%
\task{%
    Определите ток, протекающий через резистор $R_2$, разность потенциалов на нём (см.
    рис.)
    и выделяющуюся на нём мощность, если известны $r_1, r_2, \ele_1, \ele_2, R_1, R_2$.

    \begin{tikzpicture}[circuit ee IEC, thick]
        \draw  (0, 0) -- ++(up:2)
                to [
                    battery={very near start, rotate=0, info={$\ele_1, r_1 $}},
                    resistor={midway, info=$R_1$},
                    battery={very near end, rotate=-180, info={$\ele_2, r_2 $}}
                ] ++(right:5)
                -- ++(down:2)
                to [resistor={info=$R_2$}] ++(left:5);
    \end{tikzpicture}
}

\variantsplitter

\addpersonalvariant{Тимофей Полетаев}

\tasknumber{1}%
\task{%
    Два резистора сопротивлениями $R_1=3R$ и $R_2=8R$ подключены последовательно к источнику напряжения.
    Определите, в каком резисторе выделяется большая тепловая мощность и во сколько раз?
}
\solutionspace{120pt}

\tasknumber{2}%
\task{%
    Если батарею замкнуть на резистор сопротивлением $R_1$, то в цепи потечёт ток $\eli_1$,
    а если на другой $R_2$ — то $\eli_2$.
    Определите:
    \begin{itemize}
        \item ЭДС батареи,
        \item внутреннее сопротивление батареи,
        \item ток короткого замыкания.
    \end{itemize}
}
\solutionspace{120pt}

\tasknumber{3}%
\task{%
    Определите ток, протекающий через резистор $R_2$, разность потенциалов на нём (см.
    рис.)
    и выделяющуюся на нём мощность, если известны $r_1, r_2, \ele_1, \ele_2, R_1, R_2$.

    \begin{tikzpicture}[circuit ee IEC, thick]
        \draw  (0, 0) -- ++(up:2)
                to [
                    battery={very near start, rotate=-180, info={$\ele_1, r_1 $}},
                    resistor={midway, info=$R_1$},
                    battery={very near end, rotate=0, info={$\ele_2, r_2 $}}
                ] ++(right:5)
                -- ++(down:2)
                to [resistor={info=$R_2$}] ++(left:5);
    \end{tikzpicture}
}

\variantsplitter

\addpersonalvariant{Андрей Рожков}

\tasknumber{1}%
\task{%
    Два резистора сопротивлениями $R_1=5R$ и $R_2=6R$ подключены параллельно к источнику напряжения.
    Определите, в каком резисторе выделяется большая тепловая мощность и во сколько раз?
}
\solutionspace{120pt}

\tasknumber{2}%
\task{%
    Если батарею замкнуть на резистор сопротивлением $R_1$, то в цепи потечёт ток $\eli_1$,
    а если на другой $R_2$ — то $\eli_2$.
    Определите:
    \begin{itemize}
        \item ЭДС батареи,
        \item внутреннее сопротивление батареи,
        \item ток короткого замыкания.
    \end{itemize}
}
\solutionspace{120pt}

\tasknumber{3}%
\task{%
    Определите ток, протекающий через резистор $R_2$, разность потенциалов на нём (см.
    рис.)
    и выделяющуюся на нём мощность, если известны $r_1, r_2, \ele_1, \ele_2, R_1, R_2$.

    \begin{tikzpicture}[circuit ee IEC, thick]
        \draw  (0, 0) -- ++(up:2)
                to [
                    battery={very near start, rotate=0, info={$\ele_1, r_1 $}},
                    resistor={midway, info=$R_1$},
                    battery={very near end, rotate=-180, info={$\ele_2, r_2 $}}
                ] ++(right:5)
                -- ++(down:2)
                to [resistor={info=$R_2$}] ++(left:5);
    \end{tikzpicture}
}

\variantsplitter

\addpersonalvariant{Рената Таржиманова}

\tasknumber{1}%
\task{%
    Два резистора сопротивлениями $R_1=5R$ и $R_2=6R$ подключены параллельно к источнику напряжения.
    Определите, в каком резисторе выделяется большая тепловая мощность и во сколько раз?
}
\solutionspace{120pt}

\tasknumber{2}%
\task{%
    Если батарею замкнуть на резистор сопротивлением $R_1$, то в цепи потечёт ток $\eli_1$,
    а если на другой $R_2$ — то $\eli_2$.
    Определите:
    \begin{itemize}
        \item ЭДС батареи,
        \item внутреннее сопротивление батареи,
        \item ток короткого замыкания.
    \end{itemize}
}
\solutionspace{120pt}

\tasknumber{3}%
\task{%
    Определите ток, протекающий через резистор $R_2$, разность потенциалов на нём (см.
    рис.)
    и выделяющуюся на нём мощность, если известны $r_1, r_2, \ele_1, \ele_2, R_1, R_2$.

    \begin{tikzpicture}[circuit ee IEC, thick]
        \draw  (0, 0) -- ++(up:2)
                to [
                    battery={very near start, rotate=0, info={$\ele_1, r_1 $}},
                    resistor={midway, info=$R_1$},
                    battery={very near end, rotate=0, info={$\ele_2, r_2 $}}
                ] ++(right:5)
                -- ++(down:2)
                to [resistor={info=$R_2$}] ++(left:5);
    \end{tikzpicture}
}

\variantsplitter

\addpersonalvariant{Андрей Щербаков}

\tasknumber{1}%
\task{%
    Два резистора сопротивлениями $R_1=7R$ и $R_2=4R$ подключены последовательно к источнику напряжения.
    Определите, в каком резисторе выделяется большая тепловая мощность и во сколько раз?
}
\solutionspace{120pt}

\tasknumber{2}%
\task{%
    Если батарею замкнуть на резистор сопротивлением $R_1$, то в цепи потечёт ток $\eli_1$,
    а если на другой $R_2$ — то $\eli_2$.
    Определите:
    \begin{itemize}
        \item ЭДС батареи,
        \item внутреннее сопротивление батареи,
        \item ток короткого замыкания.
    \end{itemize}
}
\solutionspace{120pt}

\tasknumber{3}%
\task{%
    Определите ток, протекающий через резистор $R_1$, разность потенциалов на нём (см.
    рис.)
    и выделяющуюся на нём мощность, если известны $r_1, r_2, \ele_1, \ele_2, R_1, R_2$.

    \begin{tikzpicture}[circuit ee IEC, thick]
        \draw  (0, 0) -- ++(up:2)
                to [
                    battery={very near start, rotate=-180, info={$\ele_1, r_1 $}},
                    resistor={midway, info=$R_1$},
                    battery={very near end, rotate=-180, info={$\ele_2, r_2 $}}
                ] ++(right:5)
                -- ++(down:2)
                to [resistor={info=$R_2$}] ++(left:5);
    \end{tikzpicture}
}

\variantsplitter

\addpersonalvariant{Михаил Ярошевский}

\tasknumber{1}%
\task{%
    Два резистора сопротивлениями $R_1=7R$ и $R_2=8R$ подключены параллельно к источнику напряжения.
    Определите, в каком резисторе выделяется большая тепловая мощность и во сколько раз?
}
\solutionspace{120pt}

\tasknumber{2}%
\task{%
    Если батарею замкнуть на резистор сопротивлением $R_1$, то в цепи потечёт ток $\eli_1$,
    а если на другой $R_2$ — то $\eli_2$.
    Определите:
    \begin{itemize}
        \item ЭДС батареи,
        \item внутреннее сопротивление батареи,
        \item ток короткого замыкания.
    \end{itemize}
}
\solutionspace{120pt}

\tasknumber{3}%
\task{%
    Определите ток, протекающий через резистор $R_2$, разность потенциалов на нём (см.
    рис.)
    и выделяющуюся на нём мощность, если известны $r_1, r_2, \ele_1, \ele_2, R_1, R_2$.

    \begin{tikzpicture}[circuit ee IEC, thick]
        \draw  (0, 0) -- ++(up:2)
                to [
                    battery={very near start, rotate=-180, info={$\ele_1, r_1 $}},
                    resistor={midway, info=$R_1$},
                    battery={very near end, rotate=-180, info={$\ele_2, r_2 $}}
                ] ++(right:5)
                -- ++(down:2)
                to [resistor={info=$R_2$}] ++(left:5);
    \end{tikzpicture}
}

\variantsplitter

\addpersonalvariant{Алексей Алимпиев}

\tasknumber{1}%
\task{%
    Два резистора сопротивлениями $R_1=3R$ и $R_2=6R$ подключены последовательно к источнику напряжения.
    Определите, в каком резисторе выделяется большая тепловая мощность и во сколько раз?
}
\solutionspace{120pt}

\tasknumber{2}%
\task{%
    Если батарею замкнуть на резистор сопротивлением $R_1$, то в цепи потечёт ток $\eli_1$,
    а если на другой $R_2$ — то $\eli_2$.
    Определите:
    \begin{itemize}
        \item ЭДС батареи,
        \item внутреннее сопротивление батареи,
        \item ток короткого замыкания.
    \end{itemize}
}
\solutionspace{120pt}

\tasknumber{3}%
\task{%
    Определите ток, протекающий через резистор $R_1$, разность потенциалов на нём (см.
    рис.)
    и выделяющуюся на нём мощность, если известны $r_1, r_2, \ele_1, \ele_2, R_1, R_2$.

    \begin{tikzpicture}[circuit ee IEC, thick]
        \draw  (0, 0) -- ++(up:2)
                to [
                    battery={very near start, rotate=-180, info={$\ele_1, r_1 $}},
                    resistor={midway, info=$R_1$},
                    battery={very near end, rotate=-180, info={$\ele_2, r_2 $}}
                ] ++(right:5)
                -- ++(down:2)
                to [resistor={info=$R_2$}] ++(left:5);
    \end{tikzpicture}
}

\variantsplitter

\addpersonalvariant{Евгений Васин}

\tasknumber{1}%
\task{%
    Два резистора сопротивлениями $R_1=7R$ и $R_2=4R$ подключены последовательно к источнику напряжения.
    Определите, в каком резисторе выделяется большая тепловая мощность и во сколько раз?
}
\solutionspace{120pt}

\tasknumber{2}%
\task{%
    Если батарею замкнуть на резистор сопротивлением $R_1$, то в цепи потечёт ток $\eli_1$,
    а если на другой $R_2$ — то $\eli_2$.
    Определите:
    \begin{itemize}
        \item ЭДС батареи,
        \item внутреннее сопротивление батареи,
        \item ток короткого замыкания.
    \end{itemize}
}
\solutionspace{120pt}

\tasknumber{3}%
\task{%
    Определите ток, протекающий через резистор $R_1$, разность потенциалов на нём (см.
    рис.)
    и выделяющуюся на нём мощность, если известны $r_1, r_2, \ele_1, \ele_2, R_1, R_2$.

    \begin{tikzpicture}[circuit ee IEC, thick]
        \draw  (0, 0) -- ++(up:2)
                to [
                    battery={very near start, rotate=0, info={$\ele_1, r_1 $}},
                    resistor={midway, info=$R_1$},
                    battery={very near end, rotate=-180, info={$\ele_2, r_2 $}}
                ] ++(right:5)
                -- ++(down:2)
                to [resistor={info=$R_2$}] ++(left:5);
    \end{tikzpicture}
}

\variantsplitter

\addpersonalvariant{Вячеслав Волохов}

\tasknumber{1}%
\task{%
    Два резистора сопротивлениями $R_1=5R$ и $R_2=8R$ подключены параллельно к источнику напряжения.
    Определите, в каком резисторе выделяется большая тепловая мощность и во сколько раз?
}
\solutionspace{120pt}

\tasknumber{2}%
\task{%
    Если батарею замкнуть на резистор сопротивлением $R_1$, то в цепи потечёт ток $\eli_1$,
    а если на другой $R_2$ — то $\eli_2$.
    Определите:
    \begin{itemize}
        \item ЭДС батареи,
        \item внутреннее сопротивление батареи,
        \item ток короткого замыкания.
    \end{itemize}
}
\solutionspace{120pt}

\tasknumber{3}%
\task{%
    Определите ток, протекающий через резистор $R_2$, разность потенциалов на нём (см.
    рис.)
    и выделяющуюся на нём мощность, если известны $r_1, r_2, \ele_1, \ele_2, R_1, R_2$.

    \begin{tikzpicture}[circuit ee IEC, thick]
        \draw  (0, 0) -- ++(up:2)
                to [
                    battery={very near start, rotate=-180, info={$\ele_1, r_1 $}},
                    resistor={midway, info=$R_1$},
                    battery={very near end, rotate=0, info={$\ele_2, r_2 $}}
                ] ++(right:5)
                -- ++(down:2)
                to [resistor={info=$R_2$}] ++(left:5);
    \end{tikzpicture}
}

\variantsplitter

\addpersonalvariant{Герман Говоров}

\tasknumber{1}%
\task{%
    Два резистора сопротивлениями $R_1=5R$ и $R_2=8R$ подключены последовательно к источнику напряжения.
    Определите, в каком резисторе выделяется большая тепловая мощность и во сколько раз?
}
\solutionspace{120pt}

\tasknumber{2}%
\task{%
    Если батарею замкнуть на резистор сопротивлением $R_1$, то в цепи потечёт ток $\eli_1$,
    а если на другой $R_2$ — то $\eli_2$.
    Определите:
    \begin{itemize}
        \item ЭДС батареи,
        \item внутреннее сопротивление батареи,
        \item ток короткого замыкания.
    \end{itemize}
}
\solutionspace{120pt}

\tasknumber{3}%
\task{%
    Определите ток, протекающий через резистор $R_1$, разность потенциалов на нём (см.
    рис.)
    и выделяющуюся на нём мощность, если известны $r_1, r_2, \ele_1, \ele_2, R_1, R_2$.

    \begin{tikzpicture}[circuit ee IEC, thick]
        \draw  (0, 0) -- ++(up:2)
                to [
                    battery={very near start, rotate=0, info={$\ele_1, r_1 $}},
                    resistor={midway, info=$R_1$},
                    battery={very near end, rotate=-180, info={$\ele_2, r_2 $}}
                ] ++(right:5)
                -- ++(down:2)
                to [resistor={info=$R_2$}] ++(left:5);
    \end{tikzpicture}
}

\variantsplitter

\addpersonalvariant{София Журавлёва}

\tasknumber{1}%
\task{%
    Два резистора сопротивлениями $R_1=5R$ и $R_2=6R$ подключены параллельно к источнику напряжения.
    Определите, в каком резисторе выделяется большая тепловая мощность и во сколько раз?
}
\solutionspace{120pt}

\tasknumber{2}%
\task{%
    Если батарею замкнуть на резистор сопротивлением $R_1$, то в цепи потечёт ток $\eli_1$,
    а если на другой $R_2$ — то $\eli_2$.
    Определите:
    \begin{itemize}
        \item ЭДС батареи,
        \item внутреннее сопротивление батареи,
        \item ток короткого замыкания.
    \end{itemize}
}
\solutionspace{120pt}

\tasknumber{3}%
\task{%
    Определите ток, протекающий через резистор $R_1$, разность потенциалов на нём (см.
    рис.)
    и выделяющуюся на нём мощность, если известны $r_1, r_2, \ele_1, \ele_2, R_1, R_2$.

    \begin{tikzpicture}[circuit ee IEC, thick]
        \draw  (0, 0) -- ++(up:2)
                to [
                    battery={very near start, rotate=-180, info={$\ele_1, r_1 $}},
                    resistor={midway, info=$R_1$},
                    battery={very near end, rotate=0, info={$\ele_2, r_2 $}}
                ] ++(right:5)
                -- ++(down:2)
                to [resistor={info=$R_2$}] ++(left:5);
    \end{tikzpicture}
}

\variantsplitter

\addpersonalvariant{Константин Козлов}

\tasknumber{1}%
\task{%
    Два резистора сопротивлениями $R_1=3R$ и $R_2=8R$ подключены параллельно к источнику напряжения.
    Определите, в каком резисторе выделяется большая тепловая мощность и во сколько раз?
}
\solutionspace{120pt}

\tasknumber{2}%
\task{%
    Если батарею замкнуть на резистор сопротивлением $R_1$, то в цепи потечёт ток $\eli_1$,
    а если на другой $R_2$ — то $\eli_2$.
    Определите:
    \begin{itemize}
        \item ЭДС батареи,
        \item внутреннее сопротивление батареи,
        \item ток короткого замыкания.
    \end{itemize}
}
\solutionspace{120pt}

\tasknumber{3}%
\task{%
    Определите ток, протекающий через резистор $R_2$, разность потенциалов на нём (см.
    рис.)
    и выделяющуюся на нём мощность, если известны $r_1, r_2, \ele_1, \ele_2, R_1, R_2$.

    \begin{tikzpicture}[circuit ee IEC, thick]
        \draw  (0, 0) -- ++(up:2)
                to [
                    battery={very near start, rotate=-180, info={$\ele_1, r_1 $}},
                    resistor={midway, info=$R_1$},
                    battery={very near end, rotate=0, info={$\ele_2, r_2 $}}
                ] ++(right:5)
                -- ++(down:2)
                to [resistor={info=$R_2$}] ++(left:5);
    \end{tikzpicture}
}

\variantsplitter

\addpersonalvariant{Наталья Кравченко}

\tasknumber{1}%
\task{%
    Два резистора сопротивлениями $R_1=3R$ и $R_2=8R$ подключены последовательно к источнику напряжения.
    Определите, в каком резисторе выделяется большая тепловая мощность и во сколько раз?
}
\solutionspace{120pt}

\tasknumber{2}%
\task{%
    Если батарею замкнуть на резистор сопротивлением $R_1$, то в цепи потечёт ток $\eli_1$,
    а если на другой $R_2$ — то $\eli_2$.
    Определите:
    \begin{itemize}
        \item ЭДС батареи,
        \item внутреннее сопротивление батареи,
        \item ток короткого замыкания.
    \end{itemize}
}
\solutionspace{120pt}

\tasknumber{3}%
\task{%
    Определите ток, протекающий через резистор $R_1$, разность потенциалов на нём (см.
    рис.)
    и выделяющуюся на нём мощность, если известны $r_1, r_2, \ele_1, \ele_2, R_1, R_2$.

    \begin{tikzpicture}[circuit ee IEC, thick]
        \draw  (0, 0) -- ++(up:2)
                to [
                    battery={very near start, rotate=0, info={$\ele_1, r_1 $}},
                    resistor={midway, info=$R_1$},
                    battery={very near end, rotate=0, info={$\ele_2, r_2 $}}
                ] ++(right:5)
                -- ++(down:2)
                to [resistor={info=$R_2$}] ++(left:5);
    \end{tikzpicture}
}

\variantsplitter

\addpersonalvariant{Матвей Кузьмин}

\tasknumber{1}%
\task{%
    Два резистора сопротивлениями $R_1=5R$ и $R_2=2R$ подключены последовательно к источнику напряжения.
    Определите, в каком резисторе выделяется большая тепловая мощность и во сколько раз?
}
\solutionspace{120pt}

\tasknumber{2}%
\task{%
    Если батарею замкнуть на резистор сопротивлением $R_1$, то в цепи потечёт ток $\eli_1$,
    а если на другой $R_2$ — то $\eli_2$.
    Определите:
    \begin{itemize}
        \item ЭДС батареи,
        \item внутреннее сопротивление батареи,
        \item ток короткого замыкания.
    \end{itemize}
}
\solutionspace{120pt}

\tasknumber{3}%
\task{%
    Определите ток, протекающий через резистор $R_2$, разность потенциалов на нём (см.
    рис.)
    и выделяющуюся на нём мощность, если известны $r_1, r_2, \ele_1, \ele_2, R_1, R_2$.

    \begin{tikzpicture}[circuit ee IEC, thick]
        \draw  (0, 0) -- ++(up:2)
                to [
                    battery={very near start, rotate=-180, info={$\ele_1, r_1 $}},
                    resistor={midway, info=$R_1$},
                    battery={very near end, rotate=-180, info={$\ele_2, r_2 $}}
                ] ++(right:5)
                -- ++(down:2)
                to [resistor={info=$R_2$}] ++(left:5);
    \end{tikzpicture}
}

\variantsplitter

\addpersonalvariant{Сергей Малышев}

\tasknumber{1}%
\task{%
    Два резистора сопротивлениями $R_1=7R$ и $R_2=4R$ подключены последовательно к источнику напряжения.
    Определите, в каком резисторе выделяется большая тепловая мощность и во сколько раз?
}
\solutionspace{120pt}

\tasknumber{2}%
\task{%
    Если батарею замкнуть на резистор сопротивлением $R_1$, то в цепи потечёт ток $\eli_1$,
    а если на другой $R_2$ — то $\eli_2$.
    Определите:
    \begin{itemize}
        \item ЭДС батареи,
        \item внутреннее сопротивление батареи,
        \item ток короткого замыкания.
    \end{itemize}
}
\solutionspace{120pt}

\tasknumber{3}%
\task{%
    Определите ток, протекающий через резистор $R_1$, разность потенциалов на нём (см.
    рис.)
    и выделяющуюся на нём мощность, если известны $r_1, r_2, \ele_1, \ele_2, R_1, R_2$.

    \begin{tikzpicture}[circuit ee IEC, thick]
        \draw  (0, 0) -- ++(up:2)
                to [
                    battery={very near start, rotate=0, info={$\ele_1, r_1 $}},
                    resistor={midway, info=$R_1$},
                    battery={very near end, rotate=0, info={$\ele_2, r_2 $}}
                ] ++(right:5)
                -- ++(down:2)
                to [resistor={info=$R_2$}] ++(left:5);
    \end{tikzpicture}
}

\variantsplitter

\addpersonalvariant{Алина Полканова}

\tasknumber{1}%
\task{%
    Два резистора сопротивлениями $R_1=3R$ и $R_2=4R$ подключены последовательно к источнику напряжения.
    Определите, в каком резисторе выделяется большая тепловая мощность и во сколько раз?
}
\solutionspace{120pt}

\tasknumber{2}%
\task{%
    Если батарею замкнуть на резистор сопротивлением $R_1$, то в цепи потечёт ток $\eli_1$,
    а если на другой $R_2$ — то $\eli_2$.
    Определите:
    \begin{itemize}
        \item ЭДС батареи,
        \item внутреннее сопротивление батареи,
        \item ток короткого замыкания.
    \end{itemize}
}
\solutionspace{120pt}

\tasknumber{3}%
\task{%
    Определите ток, протекающий через резистор $R_1$, разность потенциалов на нём (см.
    рис.)
    и выделяющуюся на нём мощность, если известны $r_1, r_2, \ele_1, \ele_2, R_1, R_2$.

    \begin{tikzpicture}[circuit ee IEC, thick]
        \draw  (0, 0) -- ++(up:2)
                to [
                    battery={very near start, rotate=0, info={$\ele_1, r_1 $}},
                    resistor={midway, info=$R_1$},
                    battery={very near end, rotate=-180, info={$\ele_2, r_2 $}}
                ] ++(right:5)
                -- ++(down:2)
                to [resistor={info=$R_2$}] ++(left:5);
    \end{tikzpicture}
}

\variantsplitter

\addpersonalvariant{Сергей Пономарёв}

\tasknumber{1}%
\task{%
    Два резистора сопротивлениями $R_1=5R$ и $R_2=6R$ подключены последовательно к источнику напряжения.
    Определите, в каком резисторе выделяется большая тепловая мощность и во сколько раз?
}
\solutionspace{120pt}

\tasknumber{2}%
\task{%
    Если батарею замкнуть на резистор сопротивлением $R_1$, то в цепи потечёт ток $\eli_1$,
    а если на другой $R_2$ — то $\eli_2$.
    Определите:
    \begin{itemize}
        \item ЭДС батареи,
        \item внутреннее сопротивление батареи,
        \item ток короткого замыкания.
    \end{itemize}
}
\solutionspace{120pt}

\tasknumber{3}%
\task{%
    Определите ток, протекающий через резистор $R_2$, разность потенциалов на нём (см.
    рис.)
    и выделяющуюся на нём мощность, если известны $r_1, r_2, \ele_1, \ele_2, R_1, R_2$.

    \begin{tikzpicture}[circuit ee IEC, thick]
        \draw  (0, 0) -- ++(up:2)
                to [
                    battery={very near start, rotate=-180, info={$\ele_1, r_1 $}},
                    resistor={midway, info=$R_1$},
                    battery={very near end, rotate=-180, info={$\ele_2, r_2 $}}
                ] ++(right:5)
                -- ++(down:2)
                to [resistor={info=$R_2$}] ++(left:5);
    \end{tikzpicture}
}

\variantsplitter

\addpersonalvariant{Егор Свистушкин}

\tasknumber{1}%
\task{%
    Два резистора сопротивлениями $R_1=3R$ и $R_2=4R$ подключены последовательно к источнику напряжения.
    Определите, в каком резисторе выделяется большая тепловая мощность и во сколько раз?
}
\solutionspace{120pt}

\tasknumber{2}%
\task{%
    Если батарею замкнуть на резистор сопротивлением $R_1$, то в цепи потечёт ток $\eli_1$,
    а если на другой $R_2$ — то $\eli_2$.
    Определите:
    \begin{itemize}
        \item ЭДС батареи,
        \item внутреннее сопротивление батареи,
        \item ток короткого замыкания.
    \end{itemize}
}
\solutionspace{120pt}

\tasknumber{3}%
\task{%
    Определите ток, протекающий через резистор $R_2$, разность потенциалов на нём (см.
    рис.)
    и выделяющуюся на нём мощность, если известны $r_1, r_2, \ele_1, \ele_2, R_1, R_2$.

    \begin{tikzpicture}[circuit ee IEC, thick]
        \draw  (0, 0) -- ++(up:2)
                to [
                    battery={very near start, rotate=-180, info={$\ele_1, r_1 $}},
                    resistor={midway, info=$R_1$},
                    battery={very near end, rotate=-180, info={$\ele_2, r_2 $}}
                ] ++(right:5)
                -- ++(down:2)
                to [resistor={info=$R_2$}] ++(left:5);
    \end{tikzpicture}
}

\variantsplitter

\addpersonalvariant{Дмитрий Соколов}

\tasknumber{1}%
\task{%
    Два резистора сопротивлениями $R_1=5R$ и $R_2=8R$ подключены последовательно к источнику напряжения.
    Определите, в каком резисторе выделяется большая тепловая мощность и во сколько раз?
}
\solutionspace{120pt}

\tasknumber{2}%
\task{%
    Если батарею замкнуть на резистор сопротивлением $R_1$, то в цепи потечёт ток $\eli_1$,
    а если на другой $R_2$ — то $\eli_2$.
    Определите:
    \begin{itemize}
        \item ЭДС батареи,
        \item внутреннее сопротивление батареи,
        \item ток короткого замыкания.
    \end{itemize}
}
\solutionspace{120pt}

\tasknumber{3}%
\task{%
    Определите ток, протекающий через резистор $R_2$, разность потенциалов на нём (см.
    рис.)
    и выделяющуюся на нём мощность, если известны $r_1, r_2, \ele_1, \ele_2, R_1, R_2$.

    \begin{tikzpicture}[circuit ee IEC, thick]
        \draw  (0, 0) -- ++(up:2)
                to [
                    battery={very near start, rotate=0, info={$\ele_1, r_1 $}},
                    resistor={midway, info=$R_1$},
                    battery={very near end, rotate=0, info={$\ele_2, r_2 $}}
                ] ++(right:5)
                -- ++(down:2)
                to [resistor={info=$R_2$}] ++(left:5);
    \end{tikzpicture}
}

\variantsplitter

\addpersonalvariant{Арсений Трофимов}

\tasknumber{1}%
\task{%
    Два резистора сопротивлениями $R_1=7R$ и $R_2=2R$ подключены последовательно к источнику напряжения.
    Определите, в каком резисторе выделяется большая тепловая мощность и во сколько раз?
}
\solutionspace{120pt}

\tasknumber{2}%
\task{%
    Если батарею замкнуть на резистор сопротивлением $R_1$, то в цепи потечёт ток $\eli_1$,
    а если на другой $R_2$ — то $\eli_2$.
    Определите:
    \begin{itemize}
        \item ЭДС батареи,
        \item внутреннее сопротивление батареи,
        \item ток короткого замыкания.
    \end{itemize}
}
\solutionspace{120pt}

\tasknumber{3}%
\task{%
    Определите ток, протекающий через резистор $R_1$, разность потенциалов на нём (см.
    рис.)
    и выделяющуюся на нём мощность, если известны $r_1, r_2, \ele_1, \ele_2, R_1, R_2$.

    \begin{tikzpicture}[circuit ee IEC, thick]
        \draw  (0, 0) -- ++(up:2)
                to [
                    battery={very near start, rotate=0, info={$\ele_1, r_1 $}},
                    resistor={midway, info=$R_1$},
                    battery={very near end, rotate=0, info={$\ele_2, r_2 $}}
                ] ++(right:5)
                -- ++(down:2)
                to [resistor={info=$R_2$}] ++(left:5);
    \end{tikzpicture}
}
% autogenerated
