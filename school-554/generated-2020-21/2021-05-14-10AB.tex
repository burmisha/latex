\setdate{14~мая~2021}
\setclass{10«АБ»}

\addpersonalvariant{Михаил Бурмистров}

\tasknumber{1}%
\task{%
    Два резистора сопротивлениями $R_1=5R$ и $R_2=8R$ подключены последовательно к источнику напряжения.
    Определите, в каком резисторе выделяется большая тепловая мощность и во сколько раз?
}
\answer{%
    Подключены последовательно, поэтому  $\eli_1 = \eli_2 = \eli \implies \frac{P_2}{P_1} = \frac{\eli_2^2 R_2}{\eli_1^2 R_1} = \frac{\eli_2^2R_2}{\eli_1R_1} = \frac{R_2}{R_1} = \frac85$.
}
\solutionspace{120pt}

\tasknumber{2}%
\task{%
    Если батарею замкнуть на резистор сопротивлением $R_1$, то в цепи потечёт ток $\eli_1$,
    а если на другой $R_2$ — то $\eli_2$.
    Определите:
    \begin{itemize}
        \item ЭДС батареи,
        \item внутреннее сопротивление батареи,
        \item ток короткого замыкания.
    \end{itemize}
}
\answer{%
    Запишем закон Ома для полной цепи 2 раза для обоих способов подключения (с $R_1$ и с $R_2$),
    а короткое замыкание рассмотрим позже.
    Отметим, что для такой простой схемы он совпадает
    с законом Кирхгофа.
    Получим систему из 2 уравнений и 2 неизвестных, решим в удобном порядке,
    ибо нам всё равно понадобятся обе.

    \begin{align*}
        &\begin{cases}
            \ele = \eli_1(R_1 + r), \\
            \ele = \eli_2(R_2 + r); \\
        \end{cases} \\
        &\eli_1(R_1 + r) = \eli_2(R_2 + r), \\
        &\eli_1 R_1 + \eli_1r = \eli_2 R_2 + \eli_2r, \\
        &\eli_1 R_1 - \eli_2 R_2 = - \eli_1r  + \eli_2r = (\eli_2 - \eli_1)r, \\
        r &= \frac{\eli_1 R_1 - \eli_2 R_2}{\eli_2 - \eli_1}
            \equiv \frac{\eli_2 R_2 - \eli_1 R_1}{\eli_1 - \eli_2}, \\
        \ele &= \eli_1(R_1 + r)
            = \eli_1\cbr{R_1 + \frac{\eli_1 R_1 - \eli_2 R_2}{\eli_2 - \eli_1}}
            = \eli_1 \cdot \frac{R_1\eli_2 - R_1\eli_1 + \eli_1 R_1 - \eli_2 R_2}{\eli_2 - \eli_1} \\
            &= \eli_1 \cdot \frac{R_1\eli_2 - \eli_2 R_2}{\eli_2 - \eli_1}
            = \frac{\eli_1 \eli_2 (R_1 - R_2)}{\eli_2 - \eli_1}
            \equiv \frac{\eli_1 \eli_2 (R_2 - R_1)}{\eli_1 - \eli_2}.
    \end{align*}

    Короткое замыкание происходит в ситуации, когда внешнее сопротивление равно 0
    (при этом цепь замкнута, хотя нагрузки и нет вовсе):
    $$
        \eli_\text{к.
        з.} = \frac \ele {0 + r} = \frac \ele r
            = \frac{\cfrac{\eli_1 \eli_2 (R_1 - R_2)}{\eli_2 - \eli_1}}{\cfrac{\eli_1 R_1 - \eli_2 R_2}{\eli_2 - \eli_1}}
            = \frac{\eli_1 \eli_2 (R_1 - R_2)}{\eli_1 R_1 - \eli_2 R_2}
            \equiv \frac{\eli_1 \eli_2 (R_2 - R_1)}{\eli_2 R_2 - \eli_1 R_1}.
    $$

    Важные пункты:
    \begin{itemize}
        \item В ответах есть только те величины, которые есть в условии
            (и ещё физические постоянные могут встретиться, но нам не понадобилось).
        \item Мы упростили выражения, который пошли в ответы (благо у нас даже получилось:
            приведение к общему знаменателю укоротило ответ).
            Надо доделывать.
        \item Всё ответы симметричны относительно замены резисторов 1 и 2 (ведь при этом изменятся и токи).
    \end{itemize}
}
\solutionspace{120pt}

\tasknumber{3}%
\task{%
    Определите ток, протекающий через резистор $R_1$, разность потенциалов на нём (см.
    рис.)
    и выделяющуюся на нём мощность, если известны $r_1, r_2, \ele_1, \ele_2, R_1, R_2$.

    \begin{tikzpicture}[circuit ee IEC, thick]
        \draw  (0, 0) -- ++(up:2)
                to [
                    battery={ very near start, rotate=-180, info={$\ele_1, r_1 $}},
                    resistor={ midway, info=$R_1$},
                    battery={ very near end, rotate=0, info={$\ele_2, r_2 $}}
                ] ++(right:5)
                -- ++(down:2)
                to [resistor={info=$R_2$}] ++(left:5);
    \end{tikzpicture}
}
\answer{%
    Нетривиальных узлов нет, поэтому все законы Кирхгофа для узлов будут иметь вид
    $\eli-\eli=0$ и ничем нам не помогут.
    Впрочем, если бы мы обозначили токи на разных участках контура $\eli_1, \eli_2, \eli_3, \ldots$,
    то именно эти законы бы помогли понять, что все эти токи равны: $\eli_1 - \eli_2 = 0$ и т.д.
    Так что запишем закон Кирхгофа для единственного замкнутого контура:

    \begin{tikzpicture}[circuit ee IEC, thick]
        \draw  (0, 0) -- ++(up:2)
                to [
                    battery={ very near start, rotate=-180, info={$\ele_1, r_1 $}},
                    resistor={ midway, info=$R_1$},
                    battery={ very near end, rotate=0, info={$\ele_2, r_2 $}}
                ] ++(right:5)
                -- ++(down:2)
                to [resistor={info=$R_2$}, current direction={near end, info=$\eli$}] ++(left:5);
        \draw [-{Latex}] (2, 1.4) arc [start angle = 135, end angle = -160, radius = 0.6];
    \end{tikzpicture}

    \begin{align*}
        & \ele_1 - \ele_2 = \eli R_1 + \eli r_2 + \eli R_2 + \eli r_1, \\
        & \ele_1 - \ele_2 = \eli (R_1 + r_2 + R_2 + r_1), \\
        &\eli = \frac{ \ele_1 - \ele_2 }{ R_1 + r_2 + R_2 + r_1 }, \\
        &U_1 = \eli R_1 = \frac{ \ele_1 - \ele_2 }{ R_1 + r_2 + R_2 + r_1 } \cdot R_1, \\
        &P_1 = \eli^2 R_1 = \frac{\sqr{ \ele_1 - \ele_2 } R_1}{ \sqr{ R_1 + r_2 + R_2 + r_1 }}.
    \end{align*}

    Отметим, что ответ для тока $\eli$ меняет знак, если отметить его на рисунке в другую сторону.
    Поэтому критично важно указывать на рисунке направление тока, иначе невозможно утверждать, что ответ верный.
    А вот выбор направления контура — не повлияет на ответ, но для проверки корректности записи законо Кирхгофа,
    там тоже необходимо направление.
}

\variantsplitter

\addpersonalvariant{Ирина Ан}

\tasknumber{1}%
\task{%
    Два резистора сопротивлениями $R_1=5R$ и $R_2=2R$ подключены последовательно к источнику напряжения.
    Определите, в каком резисторе выделяется большая тепловая мощность и во сколько раз?
}
\answer{%
    Подключены последовательно, поэтому  $\eli_1 = \eli_2 = \eli \implies \frac{P_2}{P_1} = \frac{\eli_2^2 R_2}{\eli_1^2 R_1} = \frac{\eli_2^2R_2}{\eli_1R_1} = \frac{R_2}{R_1} = \frac25$.
}
\solutionspace{120pt}

\tasknumber{2}%
\task{%
    Если батарею замкнуть на резистор сопротивлением $R_1$, то в цепи потечёт ток $\eli_1$,
    а если на другой $R_2$ — то $\eli_2$.
    Определите:
    \begin{itemize}
        \item ЭДС батареи,
        \item внутреннее сопротивление батареи,
        \item ток короткого замыкания.
    \end{itemize}
}
\answer{%
    Запишем закон Ома для полной цепи 2 раза для обоих способов подключения (с $R_1$ и с $R_2$),
    а короткое замыкание рассмотрим позже.
    Отметим, что для такой простой схемы он совпадает
    с законом Кирхгофа.
    Получим систему из 2 уравнений и 2 неизвестных, решим в удобном порядке,
    ибо нам всё равно понадобятся обе.

    \begin{align*}
        &\begin{cases}
            \ele = \eli_1(R_1 + r), \\
            \ele = \eli_2(R_2 + r); \\
        \end{cases} \\
        &\eli_1(R_1 + r) = \eli_2(R_2 + r), \\
        &\eli_1 R_1 + \eli_1r = \eli_2 R_2 + \eli_2r, \\
        &\eli_1 R_1 - \eli_2 R_2 = - \eli_1r  + \eli_2r = (\eli_2 - \eli_1)r, \\
        r &= \frac{\eli_1 R_1 - \eli_2 R_2}{\eli_2 - \eli_1}
            \equiv \frac{\eli_2 R_2 - \eli_1 R_1}{\eli_1 - \eli_2}, \\
        \ele &= \eli_1(R_1 + r)
            = \eli_1\cbr{R_1 + \frac{\eli_1 R_1 - \eli_2 R_2}{\eli_2 - \eli_1}}
            = \eli_1 \cdot \frac{R_1\eli_2 - R_1\eli_1 + \eli_1 R_1 - \eli_2 R_2}{\eli_2 - \eli_1} \\
            &= \eli_1 \cdot \frac{R_1\eli_2 - \eli_2 R_2}{\eli_2 - \eli_1}
            = \frac{\eli_1 \eli_2 (R_1 - R_2)}{\eli_2 - \eli_1}
            \equiv \frac{\eli_1 \eli_2 (R_2 - R_1)}{\eli_1 - \eli_2}.
    \end{align*}

    Короткое замыкание происходит в ситуации, когда внешнее сопротивление равно 0
    (при этом цепь замкнута, хотя нагрузки и нет вовсе):
    $$
        \eli_\text{к.
        з.} = \frac \ele {0 + r} = \frac \ele r
            = \frac{\cfrac{\eli_1 \eli_2 (R_1 - R_2)}{\eli_2 - \eli_1}}{\cfrac{\eli_1 R_1 - \eli_2 R_2}{\eli_2 - \eli_1}}
            = \frac{\eli_1 \eli_2 (R_1 - R_2)}{\eli_1 R_1 - \eli_2 R_2}
            \equiv \frac{\eli_1 \eli_2 (R_2 - R_1)}{\eli_2 R_2 - \eli_1 R_1}.
    $$

    Важные пункты:
    \begin{itemize}
        \item В ответах есть только те величины, которые есть в условии
            (и ещё физические постоянные могут встретиться, но нам не понадобилось).
        \item Мы упростили выражения, который пошли в ответы (благо у нас даже получилось:
            приведение к общему знаменателю укоротило ответ).
            Надо доделывать.
        \item Всё ответы симметричны относительно замены резисторов 1 и 2 (ведь при этом изменятся и токи).
    \end{itemize}
}
\solutionspace{120pt}

\tasknumber{3}%
\task{%
    Определите ток, протекающий через резистор $R_2$, разность потенциалов на нём (см.
    рис.)
    и выделяющуюся на нём мощность, если известны $r_1, r_2, \ele_1, \ele_2, R_1, R_2$.

    \begin{tikzpicture}[circuit ee IEC, thick]
        \draw  (0, 0) -- ++(up:2)
                to [
                    battery={ very near start, rotate=0, info={$\ele_1, r_1 $}},
                    resistor={ midway, info=$R_1$},
                    battery={ very near end, rotate=0, info={$\ele_2, r_2 $}}
                ] ++(right:5)
                -- ++(down:2)
                to [resistor={info=$R_2$}] ++(left:5);
    \end{tikzpicture}
}
\answer{%
    Нетривиальных узлов нет, поэтому все законы Кирхгофа для узлов будут иметь вид
    $\eli-\eli=0$ и ничем нам не помогут.
    Впрочем, если бы мы обозначили токи на разных участках контура $\eli_1, \eli_2, \eli_3, \ldots$,
    то именно эти законы бы помогли понять, что все эти токи равны: $\eli_1 - \eli_2 = 0$ и т.д.
    Так что запишем закон Кирхгофа для единственного замкнутого контура:

    \begin{tikzpicture}[circuit ee IEC, thick]
        \draw  (0, 0) -- ++(up:2)
                to [
                    battery={ very near start, rotate=0, info={$\ele_1, r_1 $}},
                    resistor={ midway, info=$R_1$},
                    battery={ very near end, rotate=0, info={$\ele_2, r_2 $}}
                ] ++(right:5)
                -- ++(down:2)
                to [resistor={info=$R_2$}, current direction={near end, info=$\eli$}] ++(left:5);
        \draw [-{Latex}] (2, 1.4) arc [start angle = 135, end angle = -160, radius = 0.6];
    \end{tikzpicture}

    \begin{align*}
        &- \ele_1 - \ele_2 = \eli R_1 + \eli r_2 + \eli R_2 + \eli r_1, \\
        &- \ele_1 - \ele_2 = \eli (R_1 + r_2 + R_2 + r_1), \\
        &\eli = \frac{- \ele_1 - \ele_2 }{ R_1 + r_2 + R_2 + r_1 }, \\
        &U_2 = \eli R_2 = \frac{- \ele_1 - \ele_2 }{ R_1 + r_2 + R_2 + r_1 } \cdot R_2, \\
        &P_2 = \eli^2 R_2 = \frac{\sqr{- \ele_1 - \ele_2 } R_2}{ \sqr{ R_1 + r_2 + R_2 + r_1 }}.
    \end{align*}

    Отметим, что ответ для тока $\eli$ меняет знак, если отметить его на рисунке в другую сторону.
    Поэтому критично важно указывать на рисунке направление тока, иначе невозможно утверждать, что ответ верный.
    А вот выбор направления контура — не повлияет на ответ, но для проверки корректности записи законо Кирхгофа,
    там тоже необходимо направление.
}

\variantsplitter

\addpersonalvariant{Софья Андрианова}

\tasknumber{1}%
\task{%
    Два резистора сопротивлениями $R_1=5R$ и $R_2=2R$ подключены параллельно к источнику напряжения.
    Определите, в каком резисторе выделяется большая тепловая мощность и во сколько раз?
}
\answer{%
    Подключены параллельно, поэтому  $U_1 = U_2 = U \implies \frac{P_2}{P_1} = \frac{\frac{U_2^2}{R_2}}{\frac{U_1^2}{R_1}} = \frac{U^2R_1}{U^2R_2} = \frac{R_1}{R_2} = \frac52$.
}
\solutionspace{120pt}

\tasknumber{2}%
\task{%
    Если батарею замкнуть на резистор сопротивлением $R_1$, то в цепи потечёт ток $\eli_1$,
    а если на другой $R_2$ — то $\eli_2$.
    Определите:
    \begin{itemize}
        \item ЭДС батареи,
        \item внутреннее сопротивление батареи,
        \item ток короткого замыкания.
    \end{itemize}
}
\answer{%
    Запишем закон Ома для полной цепи 2 раза для обоих способов подключения (с $R_1$ и с $R_2$),
    а короткое замыкание рассмотрим позже.
    Отметим, что для такой простой схемы он совпадает
    с законом Кирхгофа.
    Получим систему из 2 уравнений и 2 неизвестных, решим в удобном порядке,
    ибо нам всё равно понадобятся обе.

    \begin{align*}
        &\begin{cases}
            \ele = \eli_1(R_1 + r), \\
            \ele = \eli_2(R_2 + r); \\
        \end{cases} \\
        &\eli_1(R_1 + r) = \eli_2(R_2 + r), \\
        &\eli_1 R_1 + \eli_1r = \eli_2 R_2 + \eli_2r, \\
        &\eli_1 R_1 - \eli_2 R_2 = - \eli_1r  + \eli_2r = (\eli_2 - \eli_1)r, \\
        r &= \frac{\eli_1 R_1 - \eli_2 R_2}{\eli_2 - \eli_1}
            \equiv \frac{\eli_2 R_2 - \eli_1 R_1}{\eli_1 - \eli_2}, \\
        \ele &= \eli_1(R_1 + r)
            = \eli_1\cbr{R_1 + \frac{\eli_1 R_1 - \eli_2 R_2}{\eli_2 - \eli_1}}
            = \eli_1 \cdot \frac{R_1\eli_2 - R_1\eli_1 + \eli_1 R_1 - \eli_2 R_2}{\eli_2 - \eli_1} \\
            &= \eli_1 \cdot \frac{R_1\eli_2 - \eli_2 R_2}{\eli_2 - \eli_1}
            = \frac{\eli_1 \eli_2 (R_1 - R_2)}{\eli_2 - \eli_1}
            \equiv \frac{\eli_1 \eli_2 (R_2 - R_1)}{\eli_1 - \eli_2}.
    \end{align*}

    Короткое замыкание происходит в ситуации, когда внешнее сопротивление равно 0
    (при этом цепь замкнута, хотя нагрузки и нет вовсе):
    $$
        \eli_\text{к.
        з.} = \frac \ele {0 + r} = \frac \ele r
            = \frac{\cfrac{\eli_1 \eli_2 (R_1 - R_2)}{\eli_2 - \eli_1}}{\cfrac{\eli_1 R_1 - \eli_2 R_2}{\eli_2 - \eli_1}}
            = \frac{\eli_1 \eli_2 (R_1 - R_2)}{\eli_1 R_1 - \eli_2 R_2}
            \equiv \frac{\eli_1 \eli_2 (R_2 - R_1)}{\eli_2 R_2 - \eli_1 R_1}.
    $$

    Важные пункты:
    \begin{itemize}
        \item В ответах есть только те величины, которые есть в условии
            (и ещё физические постоянные могут встретиться, но нам не понадобилось).
        \item Мы упростили выражения, который пошли в ответы (благо у нас даже получилось:
            приведение к общему знаменателю укоротило ответ).
            Надо доделывать.
        \item Всё ответы симметричны относительно замены резисторов 1 и 2 (ведь при этом изменятся и токи).
    \end{itemize}
}
\solutionspace{120pt}

\tasknumber{3}%
\task{%
    Определите ток, протекающий через резистор $R_1$, разность потенциалов на нём (см.
    рис.)
    и выделяющуюся на нём мощность, если известны $r_1, r_2, \ele_1, \ele_2, R_1, R_2$.

    \begin{tikzpicture}[circuit ee IEC, thick]
        \draw  (0, 0) -- ++(up:2)
                to [
                    battery={ very near start, rotate=0, info={$\ele_1, r_1 $}},
                    resistor={ midway, info=$R_1$},
                    battery={ very near end, rotate=0, info={$\ele_2, r_2 $}}
                ] ++(right:5)
                -- ++(down:2)
                to [resistor={info=$R_2$}] ++(left:5);
    \end{tikzpicture}
}
\answer{%
    Нетривиальных узлов нет, поэтому все законы Кирхгофа для узлов будут иметь вид
    $\eli-\eli=0$ и ничем нам не помогут.
    Впрочем, если бы мы обозначили токи на разных участках контура $\eli_1, \eli_2, \eli_3, \ldots$,
    то именно эти законы бы помогли понять, что все эти токи равны: $\eli_1 - \eli_2 = 0$ и т.д.
    Так что запишем закон Кирхгофа для единственного замкнутого контура:

    \begin{tikzpicture}[circuit ee IEC, thick]
        \draw  (0, 0) -- ++(up:2)
                to [
                    battery={ very near start, rotate=0, info={$\ele_1, r_1 $}},
                    resistor={ midway, info=$R_1$},
                    battery={ very near end, rotate=0, info={$\ele_2, r_2 $}}
                ] ++(right:5)
                -- ++(down:2)
                to [resistor={info=$R_2$}, current direction={near end, info=$\eli$}] ++(left:5);
        \draw [-{Latex}] (2, 1.4) arc [start angle = 135, end angle = -160, radius = 0.6];
    \end{tikzpicture}

    \begin{align*}
        &- \ele_1 - \ele_2 = \eli R_1 + \eli r_2 + \eli R_2 + \eli r_1, \\
        &- \ele_1 - \ele_2 = \eli (R_1 + r_2 + R_2 + r_1), \\
        &\eli = \frac{- \ele_1 - \ele_2 }{ R_1 + r_2 + R_2 + r_1 }, \\
        &U_1 = \eli R_1 = \frac{- \ele_1 - \ele_2 }{ R_1 + r_2 + R_2 + r_1 } \cdot R_1, \\
        &P_1 = \eli^2 R_1 = \frac{\sqr{- \ele_1 - \ele_2 } R_1}{ \sqr{ R_1 + r_2 + R_2 + r_1 }}.
    \end{align*}

    Отметим, что ответ для тока $\eli$ меняет знак, если отметить его на рисунке в другую сторону.
    Поэтому критично важно указывать на рисунке направление тока, иначе невозможно утверждать, что ответ верный.
    А вот выбор направления контура — не повлияет на ответ, но для проверки корректности записи законо Кирхгофа,
    там тоже необходимо направление.
}

\variantsplitter

\addpersonalvariant{Владимир Артемчук}

\tasknumber{1}%
\task{%
    Два резистора сопротивлениями $R_1=3R$ и $R_2=2R$ подключены параллельно к источнику напряжения.
    Определите, в каком резисторе выделяется большая тепловая мощность и во сколько раз?
}
\answer{%
    Подключены параллельно, поэтому  $U_1 = U_2 = U \implies \frac{P_2}{P_1} = \frac{\frac{U_2^2}{R_2}}{\frac{U_1^2}{R_1}} = \frac{U^2R_1}{U^2R_2} = \frac{R_1}{R_2} = \frac32$.
}
\solutionspace{120pt}

\tasknumber{2}%
\task{%
    Если батарею замкнуть на резистор сопротивлением $R_1$, то в цепи потечёт ток $\eli_1$,
    а если на другой $R_2$ — то $\eli_2$.
    Определите:
    \begin{itemize}
        \item ЭДС батареи,
        \item внутреннее сопротивление батареи,
        \item ток короткого замыкания.
    \end{itemize}
}
\answer{%
    Запишем закон Ома для полной цепи 2 раза для обоих способов подключения (с $R_1$ и с $R_2$),
    а короткое замыкание рассмотрим позже.
    Отметим, что для такой простой схемы он совпадает
    с законом Кирхгофа.
    Получим систему из 2 уравнений и 2 неизвестных, решим в удобном порядке,
    ибо нам всё равно понадобятся обе.

    \begin{align*}
        &\begin{cases}
            \ele = \eli_1(R_1 + r), \\
            \ele = \eli_2(R_2 + r); \\
        \end{cases} \\
        &\eli_1(R_1 + r) = \eli_2(R_2 + r), \\
        &\eli_1 R_1 + \eli_1r = \eli_2 R_2 + \eli_2r, \\
        &\eli_1 R_1 - \eli_2 R_2 = - \eli_1r  + \eli_2r = (\eli_2 - \eli_1)r, \\
        r &= \frac{\eli_1 R_1 - \eli_2 R_2}{\eli_2 - \eli_1}
            \equiv \frac{\eli_2 R_2 - \eli_1 R_1}{\eli_1 - \eli_2}, \\
        \ele &= \eli_1(R_1 + r)
            = \eli_1\cbr{R_1 + \frac{\eli_1 R_1 - \eli_2 R_2}{\eli_2 - \eli_1}}
            = \eli_1 \cdot \frac{R_1\eli_2 - R_1\eli_1 + \eli_1 R_1 - \eli_2 R_2}{\eli_2 - \eli_1} \\
            &= \eli_1 \cdot \frac{R_1\eli_2 - \eli_2 R_2}{\eli_2 - \eli_1}
            = \frac{\eli_1 \eli_2 (R_1 - R_2)}{\eli_2 - \eli_1}
            \equiv \frac{\eli_1 \eli_2 (R_2 - R_1)}{\eli_1 - \eli_2}.
    \end{align*}

    Короткое замыкание происходит в ситуации, когда внешнее сопротивление равно 0
    (при этом цепь замкнута, хотя нагрузки и нет вовсе):
    $$
        \eli_\text{к.
        з.} = \frac \ele {0 + r} = \frac \ele r
            = \frac{\cfrac{\eli_1 \eli_2 (R_1 - R_2)}{\eli_2 - \eli_1}}{\cfrac{\eli_1 R_1 - \eli_2 R_2}{\eli_2 - \eli_1}}
            = \frac{\eli_1 \eli_2 (R_1 - R_2)}{\eli_1 R_1 - \eli_2 R_2}
            \equiv \frac{\eli_1 \eli_2 (R_2 - R_1)}{\eli_2 R_2 - \eli_1 R_1}.
    $$

    Важные пункты:
    \begin{itemize}
        \item В ответах есть только те величины, которые есть в условии
            (и ещё физические постоянные могут встретиться, но нам не понадобилось).
        \item Мы упростили выражения, который пошли в ответы (благо у нас даже получилось:
            приведение к общему знаменателю укоротило ответ).
            Надо доделывать.
        \item Всё ответы симметричны относительно замены резисторов 1 и 2 (ведь при этом изменятся и токи).
    \end{itemize}
}
\solutionspace{120pt}

\tasknumber{3}%
\task{%
    Определите ток, протекающий через резистор $R_1$, разность потенциалов на нём (см.
    рис.)
    и выделяющуюся на нём мощность, если известны $r_1, r_2, \ele_1, \ele_2, R_1, R_2$.

    \begin{tikzpicture}[circuit ee IEC, thick]
        \draw  (0, 0) -- ++(up:2)
                to [
                    battery={ very near start, rotate=0, info={$\ele_1, r_1 $}},
                    resistor={ midway, info=$R_1$},
                    battery={ very near end, rotate=-180, info={$\ele_2, r_2 $}}
                ] ++(right:5)
                -- ++(down:2)
                to [resistor={info=$R_2$}] ++(left:5);
    \end{tikzpicture}
}
\answer{%
    Нетривиальных узлов нет, поэтому все законы Кирхгофа для узлов будут иметь вид
    $\eli-\eli=0$ и ничем нам не помогут.
    Впрочем, если бы мы обозначили токи на разных участках контура $\eli_1, \eli_2, \eli_3, \ldots$,
    то именно эти законы бы помогли понять, что все эти токи равны: $\eli_1 - \eli_2 = 0$ и т.д.
    Так что запишем закон Кирхгофа для единственного замкнутого контура:

    \begin{tikzpicture}[circuit ee IEC, thick]
        \draw  (0, 0) -- ++(up:2)
                to [
                    battery={ very near start, rotate=0, info={$\ele_1, r_1 $}},
                    resistor={ midway, info=$R_1$},
                    battery={ very near end, rotate=-180, info={$\ele_2, r_2 $}}
                ] ++(right:5)
                -- ++(down:2)
                to [resistor={info=$R_2$}, current direction={near end, info=$\eli$}] ++(left:5);
        \draw [-{Latex}] (2, 1.4) arc [start angle = 135, end angle = -160, radius = 0.6];
    \end{tikzpicture}

    \begin{align*}
        &- \ele_1 +  \ele_2 = \eli R_1 + \eli r_2 + \eli R_2 + \eli r_1, \\
        &- \ele_1 +  \ele_2 = \eli (R_1 + r_2 + R_2 + r_1), \\
        &\eli = \frac{- \ele_1 +  \ele_2 }{ R_1 + r_2 + R_2 + r_1 }, \\
        &U_1 = \eli R_1 = \frac{- \ele_1 +  \ele_2 }{ R_1 + r_2 + R_2 + r_1 } \cdot R_1, \\
        &P_1 = \eli^2 R_1 = \frac{\sqr{- \ele_1 +  \ele_2 } R_1}{ \sqr{ R_1 + r_2 + R_2 + r_1 }}.
    \end{align*}

    Отметим, что ответ для тока $\eli$ меняет знак, если отметить его на рисунке в другую сторону.
    Поэтому критично важно указывать на рисунке направление тока, иначе невозможно утверждать, что ответ верный.
    А вот выбор направления контура — не повлияет на ответ, но для проверки корректности записи законо Кирхгофа,
    там тоже необходимо направление.
}

\variantsplitter

\addpersonalvariant{Софья Белянкина}

\tasknumber{1}%
\task{%
    Два резистора сопротивлениями $R_1=5R$ и $R_2=2R$ подключены последовательно к источнику напряжения.
    Определите, в каком резисторе выделяется большая тепловая мощность и во сколько раз?
}
\answer{%
    Подключены последовательно, поэтому  $\eli_1 = \eli_2 = \eli \implies \frac{P_2}{P_1} = \frac{\eli_2^2 R_2}{\eli_1^2 R_1} = \frac{\eli_2^2R_2}{\eli_1R_1} = \frac{R_2}{R_1} = \frac25$.
}
\solutionspace{120pt}

\tasknumber{2}%
\task{%
    Если батарею замкнуть на резистор сопротивлением $R_1$, то в цепи потечёт ток $\eli_1$,
    а если на другой $R_2$ — то $\eli_2$.
    Определите:
    \begin{itemize}
        \item ЭДС батареи,
        \item внутреннее сопротивление батареи,
        \item ток короткого замыкания.
    \end{itemize}
}
\answer{%
    Запишем закон Ома для полной цепи 2 раза для обоих способов подключения (с $R_1$ и с $R_2$),
    а короткое замыкание рассмотрим позже.
    Отметим, что для такой простой схемы он совпадает
    с законом Кирхгофа.
    Получим систему из 2 уравнений и 2 неизвестных, решим в удобном порядке,
    ибо нам всё равно понадобятся обе.

    \begin{align*}
        &\begin{cases}
            \ele = \eli_1(R_1 + r), \\
            \ele = \eli_2(R_2 + r); \\
        \end{cases} \\
        &\eli_1(R_1 + r) = \eli_2(R_2 + r), \\
        &\eli_1 R_1 + \eli_1r = \eli_2 R_2 + \eli_2r, \\
        &\eli_1 R_1 - \eli_2 R_2 = - \eli_1r  + \eli_2r = (\eli_2 - \eli_1)r, \\
        r &= \frac{\eli_1 R_1 - \eli_2 R_2}{\eli_2 - \eli_1}
            \equiv \frac{\eli_2 R_2 - \eli_1 R_1}{\eli_1 - \eli_2}, \\
        \ele &= \eli_1(R_1 + r)
            = \eli_1\cbr{R_1 + \frac{\eli_1 R_1 - \eli_2 R_2}{\eli_2 - \eli_1}}
            = \eli_1 \cdot \frac{R_1\eli_2 - R_1\eli_1 + \eli_1 R_1 - \eli_2 R_2}{\eli_2 - \eli_1} \\
            &= \eli_1 \cdot \frac{R_1\eli_2 - \eli_2 R_2}{\eli_2 - \eli_1}
            = \frac{\eli_1 \eli_2 (R_1 - R_2)}{\eli_2 - \eli_1}
            \equiv \frac{\eli_1 \eli_2 (R_2 - R_1)}{\eli_1 - \eli_2}.
    \end{align*}

    Короткое замыкание происходит в ситуации, когда внешнее сопротивление равно 0
    (при этом цепь замкнута, хотя нагрузки и нет вовсе):
    $$
        \eli_\text{к.
        з.} = \frac \ele {0 + r} = \frac \ele r
            = \frac{\cfrac{\eli_1 \eli_2 (R_1 - R_2)}{\eli_2 - \eli_1}}{\cfrac{\eli_1 R_1 - \eli_2 R_2}{\eli_2 - \eli_1}}
            = \frac{\eli_1 \eli_2 (R_1 - R_2)}{\eli_1 R_1 - \eli_2 R_2}
            \equiv \frac{\eli_1 \eli_2 (R_2 - R_1)}{\eli_2 R_2 - \eli_1 R_1}.
    $$

    Важные пункты:
    \begin{itemize}
        \item В ответах есть только те величины, которые есть в условии
            (и ещё физические постоянные могут встретиться, но нам не понадобилось).
        \item Мы упростили выражения, который пошли в ответы (благо у нас даже получилось:
            приведение к общему знаменателю укоротило ответ).
            Надо доделывать.
        \item Всё ответы симметричны относительно замены резисторов 1 и 2 (ведь при этом изменятся и токи).
    \end{itemize}
}
\solutionspace{120pt}

\tasknumber{3}%
\task{%
    Определите ток, протекающий через резистор $R_2$, разность потенциалов на нём (см.
    рис.)
    и выделяющуюся на нём мощность, если известны $r_1, r_2, \ele_1, \ele_2, R_1, R_2$.

    \begin{tikzpicture}[circuit ee IEC, thick]
        \draw  (0, 0) -- ++(up:2)
                to [
                    battery={ very near start, rotate=-180, info={$\ele_1, r_1 $}},
                    resistor={ midway, info=$R_1$},
                    battery={ very near end, rotate=-180, info={$\ele_2, r_2 $}}
                ] ++(right:5)
                -- ++(down:2)
                to [resistor={info=$R_2$}] ++(left:5);
    \end{tikzpicture}
}
\answer{%
    Нетривиальных узлов нет, поэтому все законы Кирхгофа для узлов будут иметь вид
    $\eli-\eli=0$ и ничем нам не помогут.
    Впрочем, если бы мы обозначили токи на разных участках контура $\eli_1, \eli_2, \eli_3, \ldots$,
    то именно эти законы бы помогли понять, что все эти токи равны: $\eli_1 - \eli_2 = 0$ и т.д.
    Так что запишем закон Кирхгофа для единственного замкнутого контура:

    \begin{tikzpicture}[circuit ee IEC, thick]
        \draw  (0, 0) -- ++(up:2)
                to [
                    battery={ very near start, rotate=-180, info={$\ele_1, r_1 $}},
                    resistor={ midway, info=$R_1$},
                    battery={ very near end, rotate=-180, info={$\ele_2, r_2 $}}
                ] ++(right:5)
                -- ++(down:2)
                to [resistor={info=$R_2$}, current direction={near end, info=$\eli$}] ++(left:5);
        \draw [-{Latex}] (2, 1.4) arc [start angle = 135, end angle = -160, radius = 0.6];
    \end{tikzpicture}

    \begin{align*}
        & \ele_1 +  \ele_2 = \eli R_1 + \eli r_2 + \eli R_2 + \eli r_1, \\
        & \ele_1 +  \ele_2 = \eli (R_1 + r_2 + R_2 + r_1), \\
        &\eli = \frac{ \ele_1 +  \ele_2 }{ R_1 + r_2 + R_2 + r_1 }, \\
        &U_2 = \eli R_2 = \frac{ \ele_1 +  \ele_2 }{ R_1 + r_2 + R_2 + r_1 } \cdot R_2, \\
        &P_2 = \eli^2 R_2 = \frac{\sqr{ \ele_1 +  \ele_2 } R_2}{ \sqr{ R_1 + r_2 + R_2 + r_1 }}.
    \end{align*}

    Отметим, что ответ для тока $\eli$ меняет знак, если отметить его на рисунке в другую сторону.
    Поэтому критично важно указывать на рисунке направление тока, иначе невозможно утверждать, что ответ верный.
    А вот выбор направления контура — не повлияет на ответ, но для проверки корректности записи законо Кирхгофа,
    там тоже необходимо направление.
}

\variantsplitter

\addpersonalvariant{Варвара Егиазарян}

\tasknumber{1}%
\task{%
    Два резистора сопротивлениями $R_1=5R$ и $R_2=2R$ подключены последовательно к источнику напряжения.
    Определите, в каком резисторе выделяется большая тепловая мощность и во сколько раз?
}
\answer{%
    Подключены последовательно, поэтому  $\eli_1 = \eli_2 = \eli \implies \frac{P_2}{P_1} = \frac{\eli_2^2 R_2}{\eli_1^2 R_1} = \frac{\eli_2^2R_2}{\eli_1R_1} = \frac{R_2}{R_1} = \frac25$.
}
\solutionspace{120pt}

\tasknumber{2}%
\task{%
    Если батарею замкнуть на резистор сопротивлением $R_1$, то в цепи потечёт ток $\eli_1$,
    а если на другой $R_2$ — то $\eli_2$.
    Определите:
    \begin{itemize}
        \item ЭДС батареи,
        \item внутреннее сопротивление батареи,
        \item ток короткого замыкания.
    \end{itemize}
}
\answer{%
    Запишем закон Ома для полной цепи 2 раза для обоих способов подключения (с $R_1$ и с $R_2$),
    а короткое замыкание рассмотрим позже.
    Отметим, что для такой простой схемы он совпадает
    с законом Кирхгофа.
    Получим систему из 2 уравнений и 2 неизвестных, решим в удобном порядке,
    ибо нам всё равно понадобятся обе.

    \begin{align*}
        &\begin{cases}
            \ele = \eli_1(R_1 + r), \\
            \ele = \eli_2(R_2 + r); \\
        \end{cases} \\
        &\eli_1(R_1 + r) = \eli_2(R_2 + r), \\
        &\eli_1 R_1 + \eli_1r = \eli_2 R_2 + \eli_2r, \\
        &\eli_1 R_1 - \eli_2 R_2 = - \eli_1r  + \eli_2r = (\eli_2 - \eli_1)r, \\
        r &= \frac{\eli_1 R_1 - \eli_2 R_2}{\eli_2 - \eli_1}
            \equiv \frac{\eli_2 R_2 - \eli_1 R_1}{\eli_1 - \eli_2}, \\
        \ele &= \eli_1(R_1 + r)
            = \eli_1\cbr{R_1 + \frac{\eli_1 R_1 - \eli_2 R_2}{\eli_2 - \eli_1}}
            = \eli_1 \cdot \frac{R_1\eli_2 - R_1\eli_1 + \eli_1 R_1 - \eli_2 R_2}{\eli_2 - \eli_1} \\
            &= \eli_1 \cdot \frac{R_1\eli_2 - \eli_2 R_2}{\eli_2 - \eli_1}
            = \frac{\eli_1 \eli_2 (R_1 - R_2)}{\eli_2 - \eli_1}
            \equiv \frac{\eli_1 \eli_2 (R_2 - R_1)}{\eli_1 - \eli_2}.
    \end{align*}

    Короткое замыкание происходит в ситуации, когда внешнее сопротивление равно 0
    (при этом цепь замкнута, хотя нагрузки и нет вовсе):
    $$
        \eli_\text{к.
        з.} = \frac \ele {0 + r} = \frac \ele r
            = \frac{\cfrac{\eli_1 \eli_2 (R_1 - R_2)}{\eli_2 - \eli_1}}{\cfrac{\eli_1 R_1 - \eli_2 R_2}{\eli_2 - \eli_1}}
            = \frac{\eli_1 \eli_2 (R_1 - R_2)}{\eli_1 R_1 - \eli_2 R_2}
            \equiv \frac{\eli_1 \eli_2 (R_2 - R_1)}{\eli_2 R_2 - \eli_1 R_1}.
    $$

    Важные пункты:
    \begin{itemize}
        \item В ответах есть только те величины, которые есть в условии
            (и ещё физические постоянные могут встретиться, но нам не понадобилось).
        \item Мы упростили выражения, который пошли в ответы (благо у нас даже получилось:
            приведение к общему знаменателю укоротило ответ).
            Надо доделывать.
        \item Всё ответы симметричны относительно замены резисторов 1 и 2 (ведь при этом изменятся и токи).
    \end{itemize}
}
\solutionspace{120pt}

\tasknumber{3}%
\task{%
    Определите ток, протекающий через резистор $R_2$, разность потенциалов на нём (см.
    рис.)
    и выделяющуюся на нём мощность, если известны $r_1, r_2, \ele_1, \ele_2, R_1, R_2$.

    \begin{tikzpicture}[circuit ee IEC, thick]
        \draw  (0, 0) -- ++(up:2)
                to [
                    battery={ very near start, rotate=0, info={$\ele_1, r_1 $}},
                    resistor={ midway, info=$R_1$},
                    battery={ very near end, rotate=0, info={$\ele_2, r_2 $}}
                ] ++(right:5)
                -- ++(down:2)
                to [resistor={info=$R_2$}] ++(left:5);
    \end{tikzpicture}
}
\answer{%
    Нетривиальных узлов нет, поэтому все законы Кирхгофа для узлов будут иметь вид
    $\eli-\eli=0$ и ничем нам не помогут.
    Впрочем, если бы мы обозначили токи на разных участках контура $\eli_1, \eli_2, \eli_3, \ldots$,
    то именно эти законы бы помогли понять, что все эти токи равны: $\eli_1 - \eli_2 = 0$ и т.д.
    Так что запишем закон Кирхгофа для единственного замкнутого контура:

    \begin{tikzpicture}[circuit ee IEC, thick]
        \draw  (0, 0) -- ++(up:2)
                to [
                    battery={ very near start, rotate=0, info={$\ele_1, r_1 $}},
                    resistor={ midway, info=$R_1$},
                    battery={ very near end, rotate=0, info={$\ele_2, r_2 $}}
                ] ++(right:5)
                -- ++(down:2)
                to [resistor={info=$R_2$}, current direction={near end, info=$\eli$}] ++(left:5);
        \draw [-{Latex}] (2, 1.4) arc [start angle = 135, end angle = -160, radius = 0.6];
    \end{tikzpicture}

    \begin{align*}
        &- \ele_1 - \ele_2 = \eli R_1 + \eli r_2 + \eli R_2 + \eli r_1, \\
        &- \ele_1 - \ele_2 = \eli (R_1 + r_2 + R_2 + r_1), \\
        &\eli = \frac{- \ele_1 - \ele_2 }{ R_1 + r_2 + R_2 + r_1 }, \\
        &U_2 = \eli R_2 = \frac{- \ele_1 - \ele_2 }{ R_1 + r_2 + R_2 + r_1 } \cdot R_2, \\
        &P_2 = \eli^2 R_2 = \frac{\sqr{- \ele_1 - \ele_2 } R_2}{ \sqr{ R_1 + r_2 + R_2 + r_1 }}.
    \end{align*}

    Отметим, что ответ для тока $\eli$ меняет знак, если отметить его на рисунке в другую сторону.
    Поэтому критично важно указывать на рисунке направление тока, иначе невозможно утверждать, что ответ верный.
    А вот выбор направления контура — не повлияет на ответ, но для проверки корректности записи законо Кирхгофа,
    там тоже необходимо направление.
}

\variantsplitter

\addpersonalvariant{Владислав Емелин}

\tasknumber{1}%
\task{%
    Два резистора сопротивлениями $R_1=3R$ и $R_2=2R$ подключены последовательно к источнику напряжения.
    Определите, в каком резисторе выделяется большая тепловая мощность и во сколько раз?
}
\answer{%
    Подключены последовательно, поэтому  $\eli_1 = \eli_2 = \eli \implies \frac{P_2}{P_1} = \frac{\eli_2^2 R_2}{\eli_1^2 R_1} = \frac{\eli_2^2R_2}{\eli_1R_1} = \frac{R_2}{R_1} = \frac23$.
}
\solutionspace{120pt}

\tasknumber{2}%
\task{%
    Если батарею замкнуть на резистор сопротивлением $R_1$, то в цепи потечёт ток $\eli_1$,
    а если на другой $R_2$ — то $\eli_2$.
    Определите:
    \begin{itemize}
        \item ЭДС батареи,
        \item внутреннее сопротивление батареи,
        \item ток короткого замыкания.
    \end{itemize}
}
\answer{%
    Запишем закон Ома для полной цепи 2 раза для обоих способов подключения (с $R_1$ и с $R_2$),
    а короткое замыкание рассмотрим позже.
    Отметим, что для такой простой схемы он совпадает
    с законом Кирхгофа.
    Получим систему из 2 уравнений и 2 неизвестных, решим в удобном порядке,
    ибо нам всё равно понадобятся обе.

    \begin{align*}
        &\begin{cases}
            \ele = \eli_1(R_1 + r), \\
            \ele = \eli_2(R_2 + r); \\
        \end{cases} \\
        &\eli_1(R_1 + r) = \eli_2(R_2 + r), \\
        &\eli_1 R_1 + \eli_1r = \eli_2 R_2 + \eli_2r, \\
        &\eli_1 R_1 - \eli_2 R_2 = - \eli_1r  + \eli_2r = (\eli_2 - \eli_1)r, \\
        r &= \frac{\eli_1 R_1 - \eli_2 R_2}{\eli_2 - \eli_1}
            \equiv \frac{\eli_2 R_2 - \eli_1 R_1}{\eli_1 - \eli_2}, \\
        \ele &= \eli_1(R_1 + r)
            = \eli_1\cbr{R_1 + \frac{\eli_1 R_1 - \eli_2 R_2}{\eli_2 - \eli_1}}
            = \eli_1 \cdot \frac{R_1\eli_2 - R_1\eli_1 + \eli_1 R_1 - \eli_2 R_2}{\eli_2 - \eli_1} \\
            &= \eli_1 \cdot \frac{R_1\eli_2 - \eli_2 R_2}{\eli_2 - \eli_1}
            = \frac{\eli_1 \eli_2 (R_1 - R_2)}{\eli_2 - \eli_1}
            \equiv \frac{\eli_1 \eli_2 (R_2 - R_1)}{\eli_1 - \eli_2}.
    \end{align*}

    Короткое замыкание происходит в ситуации, когда внешнее сопротивление равно 0
    (при этом цепь замкнута, хотя нагрузки и нет вовсе):
    $$
        \eli_\text{к.
        з.} = \frac \ele {0 + r} = \frac \ele r
            = \frac{\cfrac{\eli_1 \eli_2 (R_1 - R_2)}{\eli_2 - \eli_1}}{\cfrac{\eli_1 R_1 - \eli_2 R_2}{\eli_2 - \eli_1}}
            = \frac{\eli_1 \eli_2 (R_1 - R_2)}{\eli_1 R_1 - \eli_2 R_2}
            \equiv \frac{\eli_1 \eli_2 (R_2 - R_1)}{\eli_2 R_2 - \eli_1 R_1}.
    $$

    Важные пункты:
    \begin{itemize}
        \item В ответах есть только те величины, которые есть в условии
            (и ещё физические постоянные могут встретиться, но нам не понадобилось).
        \item Мы упростили выражения, который пошли в ответы (благо у нас даже получилось:
            приведение к общему знаменателю укоротило ответ).
            Надо доделывать.
        \item Всё ответы симметричны относительно замены резисторов 1 и 2 (ведь при этом изменятся и токи).
    \end{itemize}
}
\solutionspace{120pt}

\tasknumber{3}%
\task{%
    Определите ток, протекающий через резистор $R_1$, разность потенциалов на нём (см.
    рис.)
    и выделяющуюся на нём мощность, если известны $r_1, r_2, \ele_1, \ele_2, R_1, R_2$.

    \begin{tikzpicture}[circuit ee IEC, thick]
        \draw  (0, 0) -- ++(up:2)
                to [
                    battery={ very near start, rotate=0, info={$\ele_1, r_1 $}},
                    resistor={ midway, info=$R_1$},
                    battery={ very near end, rotate=-180, info={$\ele_2, r_2 $}}
                ] ++(right:5)
                -- ++(down:2)
                to [resistor={info=$R_2$}] ++(left:5);
    \end{tikzpicture}
}
\answer{%
    Нетривиальных узлов нет, поэтому все законы Кирхгофа для узлов будут иметь вид
    $\eli-\eli=0$ и ничем нам не помогут.
    Впрочем, если бы мы обозначили токи на разных участках контура $\eli_1, \eli_2, \eli_3, \ldots$,
    то именно эти законы бы помогли понять, что все эти токи равны: $\eli_1 - \eli_2 = 0$ и т.д.
    Так что запишем закон Кирхгофа для единственного замкнутого контура:

    \begin{tikzpicture}[circuit ee IEC, thick]
        \draw  (0, 0) -- ++(up:2)
                to [
                    battery={ very near start, rotate=0, info={$\ele_1, r_1 $}},
                    resistor={ midway, info=$R_1$},
                    battery={ very near end, rotate=-180, info={$\ele_2, r_2 $}}
                ] ++(right:5)
                -- ++(down:2)
                to [resistor={info=$R_2$}, current direction={near end, info=$\eli$}] ++(left:5);
        \draw [-{Latex}] (2, 1.4) arc [start angle = 135, end angle = -160, radius = 0.6];
    \end{tikzpicture}

    \begin{align*}
        &- \ele_1 +  \ele_2 = \eli R_1 + \eli r_2 + \eli R_2 + \eli r_1, \\
        &- \ele_1 +  \ele_2 = \eli (R_1 + r_2 + R_2 + r_1), \\
        &\eli = \frac{- \ele_1 +  \ele_2 }{ R_1 + r_2 + R_2 + r_1 }, \\
        &U_1 = \eli R_1 = \frac{- \ele_1 +  \ele_2 }{ R_1 + r_2 + R_2 + r_1 } \cdot R_1, \\
        &P_1 = \eli^2 R_1 = \frac{\sqr{- \ele_1 +  \ele_2 } R_1}{ \sqr{ R_1 + r_2 + R_2 + r_1 }}.
    \end{align*}

    Отметим, что ответ для тока $\eli$ меняет знак, если отметить его на рисунке в другую сторону.
    Поэтому критично важно указывать на рисунке направление тока, иначе невозможно утверждать, что ответ верный.
    А вот выбор направления контура — не повлияет на ответ, но для проверки корректности записи законо Кирхгофа,
    там тоже необходимо направление.
}

\variantsplitter

\addpersonalvariant{Артём Жичин}

\tasknumber{1}%
\task{%
    Два резистора сопротивлениями $R_1=3R$ и $R_2=4R$ подключены параллельно к источнику напряжения.
    Определите, в каком резисторе выделяется большая тепловая мощность и во сколько раз?
}
\answer{%
    Подключены параллельно, поэтому  $U_1 = U_2 = U \implies \frac{P_2}{P_1} = \frac{\frac{U_2^2}{R_2}}{\frac{U_1^2}{R_1}} = \frac{U^2R_1}{U^2R_2} = \frac{R_1}{R_2} = \frac34$.
}
\solutionspace{120pt}

\tasknumber{2}%
\task{%
    Если батарею замкнуть на резистор сопротивлением $R_1$, то в цепи потечёт ток $\eli_1$,
    а если на другой $R_2$ — то $\eli_2$.
    Определите:
    \begin{itemize}
        \item ЭДС батареи,
        \item внутреннее сопротивление батареи,
        \item ток короткого замыкания.
    \end{itemize}
}
\answer{%
    Запишем закон Ома для полной цепи 2 раза для обоих способов подключения (с $R_1$ и с $R_2$),
    а короткое замыкание рассмотрим позже.
    Отметим, что для такой простой схемы он совпадает
    с законом Кирхгофа.
    Получим систему из 2 уравнений и 2 неизвестных, решим в удобном порядке,
    ибо нам всё равно понадобятся обе.

    \begin{align*}
        &\begin{cases}
            \ele = \eli_1(R_1 + r), \\
            \ele = \eli_2(R_2 + r); \\
        \end{cases} \\
        &\eli_1(R_1 + r) = \eli_2(R_2 + r), \\
        &\eli_1 R_1 + \eli_1r = \eli_2 R_2 + \eli_2r, \\
        &\eli_1 R_1 - \eli_2 R_2 = - \eli_1r  + \eli_2r = (\eli_2 - \eli_1)r, \\
        r &= \frac{\eli_1 R_1 - \eli_2 R_2}{\eli_2 - \eli_1}
            \equiv \frac{\eli_2 R_2 - \eli_1 R_1}{\eli_1 - \eli_2}, \\
        \ele &= \eli_1(R_1 + r)
            = \eli_1\cbr{R_1 + \frac{\eli_1 R_1 - \eli_2 R_2}{\eli_2 - \eli_1}}
            = \eli_1 \cdot \frac{R_1\eli_2 - R_1\eli_1 + \eli_1 R_1 - \eli_2 R_2}{\eli_2 - \eli_1} \\
            &= \eli_1 \cdot \frac{R_1\eli_2 - \eli_2 R_2}{\eli_2 - \eli_1}
            = \frac{\eli_1 \eli_2 (R_1 - R_2)}{\eli_2 - \eli_1}
            \equiv \frac{\eli_1 \eli_2 (R_2 - R_1)}{\eli_1 - \eli_2}.
    \end{align*}

    Короткое замыкание происходит в ситуации, когда внешнее сопротивление равно 0
    (при этом цепь замкнута, хотя нагрузки и нет вовсе):
    $$
        \eli_\text{к.
        з.} = \frac \ele {0 + r} = \frac \ele r
            = \frac{\cfrac{\eli_1 \eli_2 (R_1 - R_2)}{\eli_2 - \eli_1}}{\cfrac{\eli_1 R_1 - \eli_2 R_2}{\eli_2 - \eli_1}}
            = \frac{\eli_1 \eli_2 (R_1 - R_2)}{\eli_1 R_1 - \eli_2 R_2}
            \equiv \frac{\eli_1 \eli_2 (R_2 - R_1)}{\eli_2 R_2 - \eli_1 R_1}.
    $$

    Важные пункты:
    \begin{itemize}
        \item В ответах есть только те величины, которые есть в условии
            (и ещё физические постоянные могут встретиться, но нам не понадобилось).
        \item Мы упростили выражения, который пошли в ответы (благо у нас даже получилось:
            приведение к общему знаменателю укоротило ответ).
            Надо доделывать.
        \item Всё ответы симметричны относительно замены резисторов 1 и 2 (ведь при этом изменятся и токи).
    \end{itemize}
}
\solutionspace{120pt}

\tasknumber{3}%
\task{%
    Определите ток, протекающий через резистор $R_1$, разность потенциалов на нём (см.
    рис.)
    и выделяющуюся на нём мощность, если известны $r_1, r_2, \ele_1, \ele_2, R_1, R_2$.

    \begin{tikzpicture}[circuit ee IEC, thick]
        \draw  (0, 0) -- ++(up:2)
                to [
                    battery={ very near start, rotate=-180, info={$\ele_1, r_1 $}},
                    resistor={ midway, info=$R_1$},
                    battery={ very near end, rotate=0, info={$\ele_2, r_2 $}}
                ] ++(right:5)
                -- ++(down:2)
                to [resistor={info=$R_2$}] ++(left:5);
    \end{tikzpicture}
}
\answer{%
    Нетривиальных узлов нет, поэтому все законы Кирхгофа для узлов будут иметь вид
    $\eli-\eli=0$ и ничем нам не помогут.
    Впрочем, если бы мы обозначили токи на разных участках контура $\eli_1, \eli_2, \eli_3, \ldots$,
    то именно эти законы бы помогли понять, что все эти токи равны: $\eli_1 - \eli_2 = 0$ и т.д.
    Так что запишем закон Кирхгофа для единственного замкнутого контура:

    \begin{tikzpicture}[circuit ee IEC, thick]
        \draw  (0, 0) -- ++(up:2)
                to [
                    battery={ very near start, rotate=-180, info={$\ele_1, r_1 $}},
                    resistor={ midway, info=$R_1$},
                    battery={ very near end, rotate=0, info={$\ele_2, r_2 $}}
                ] ++(right:5)
                -- ++(down:2)
                to [resistor={info=$R_2$}, current direction={near end, info=$\eli$}] ++(left:5);
        \draw [-{Latex}] (2, 1.4) arc [start angle = 135, end angle = -160, radius = 0.6];
    \end{tikzpicture}

    \begin{align*}
        & \ele_1 - \ele_2 = \eli R_1 + \eli r_2 + \eli R_2 + \eli r_1, \\
        & \ele_1 - \ele_2 = \eli (R_1 + r_2 + R_2 + r_1), \\
        &\eli = \frac{ \ele_1 - \ele_2 }{ R_1 + r_2 + R_2 + r_1 }, \\
        &U_1 = \eli R_1 = \frac{ \ele_1 - \ele_2 }{ R_1 + r_2 + R_2 + r_1 } \cdot R_1, \\
        &P_1 = \eli^2 R_1 = \frac{\sqr{ \ele_1 - \ele_2 } R_1}{ \sqr{ R_1 + r_2 + R_2 + r_1 }}.
    \end{align*}

    Отметим, что ответ для тока $\eli$ меняет знак, если отметить его на рисунке в другую сторону.
    Поэтому критично важно указывать на рисунке направление тока, иначе невозможно утверждать, что ответ верный.
    А вот выбор направления контура — не повлияет на ответ, но для проверки корректности записи законо Кирхгофа,
    там тоже необходимо направление.
}

\variantsplitter

\addpersonalvariant{Дарья Кошман}

\tasknumber{1}%
\task{%
    Два резистора сопротивлениями $R_1=3R$ и $R_2=2R$ подключены последовательно к источнику напряжения.
    Определите, в каком резисторе выделяется большая тепловая мощность и во сколько раз?
}
\answer{%
    Подключены последовательно, поэтому  $\eli_1 = \eli_2 = \eli \implies \frac{P_2}{P_1} = \frac{\eli_2^2 R_2}{\eli_1^2 R_1} = \frac{\eli_2^2R_2}{\eli_1R_1} = \frac{R_2}{R_1} = \frac23$.
}
\solutionspace{120pt}

\tasknumber{2}%
\task{%
    Если батарею замкнуть на резистор сопротивлением $R_1$, то в цепи потечёт ток $\eli_1$,
    а если на другой $R_2$ — то $\eli_2$.
    Определите:
    \begin{itemize}
        \item ЭДС батареи,
        \item внутреннее сопротивление батареи,
        \item ток короткого замыкания.
    \end{itemize}
}
\answer{%
    Запишем закон Ома для полной цепи 2 раза для обоих способов подключения (с $R_1$ и с $R_2$),
    а короткое замыкание рассмотрим позже.
    Отметим, что для такой простой схемы он совпадает
    с законом Кирхгофа.
    Получим систему из 2 уравнений и 2 неизвестных, решим в удобном порядке,
    ибо нам всё равно понадобятся обе.

    \begin{align*}
        &\begin{cases}
            \ele = \eli_1(R_1 + r), \\
            \ele = \eli_2(R_2 + r); \\
        \end{cases} \\
        &\eli_1(R_1 + r) = \eli_2(R_2 + r), \\
        &\eli_1 R_1 + \eli_1r = \eli_2 R_2 + \eli_2r, \\
        &\eli_1 R_1 - \eli_2 R_2 = - \eli_1r  + \eli_2r = (\eli_2 - \eli_1)r, \\
        r &= \frac{\eli_1 R_1 - \eli_2 R_2}{\eli_2 - \eli_1}
            \equiv \frac{\eli_2 R_2 - \eli_1 R_1}{\eli_1 - \eli_2}, \\
        \ele &= \eli_1(R_1 + r)
            = \eli_1\cbr{R_1 + \frac{\eli_1 R_1 - \eli_2 R_2}{\eli_2 - \eli_1}}
            = \eli_1 \cdot \frac{R_1\eli_2 - R_1\eli_1 + \eli_1 R_1 - \eli_2 R_2}{\eli_2 - \eli_1} \\
            &= \eli_1 \cdot \frac{R_1\eli_2 - \eli_2 R_2}{\eli_2 - \eli_1}
            = \frac{\eli_1 \eli_2 (R_1 - R_2)}{\eli_2 - \eli_1}
            \equiv \frac{\eli_1 \eli_2 (R_2 - R_1)}{\eli_1 - \eli_2}.
    \end{align*}

    Короткое замыкание происходит в ситуации, когда внешнее сопротивление равно 0
    (при этом цепь замкнута, хотя нагрузки и нет вовсе):
    $$
        \eli_\text{к.
        з.} = \frac \ele {0 + r} = \frac \ele r
            = \frac{\cfrac{\eli_1 \eli_2 (R_1 - R_2)}{\eli_2 - \eli_1}}{\cfrac{\eli_1 R_1 - \eli_2 R_2}{\eli_2 - \eli_1}}
            = \frac{\eli_1 \eli_2 (R_1 - R_2)}{\eli_1 R_1 - \eli_2 R_2}
            \equiv \frac{\eli_1 \eli_2 (R_2 - R_1)}{\eli_2 R_2 - \eli_1 R_1}.
    $$

    Важные пункты:
    \begin{itemize}
        \item В ответах есть только те величины, которые есть в условии
            (и ещё физические постоянные могут встретиться, но нам не понадобилось).
        \item Мы упростили выражения, который пошли в ответы (благо у нас даже получилось:
            приведение к общему знаменателю укоротило ответ).
            Надо доделывать.
        \item Всё ответы симметричны относительно замены резисторов 1 и 2 (ведь при этом изменятся и токи).
    \end{itemize}
}
\solutionspace{120pt}

\tasknumber{3}%
\task{%
    Определите ток, протекающий через резистор $R_1$, разность потенциалов на нём (см.
    рис.)
    и выделяющуюся на нём мощность, если известны $r_1, r_2, \ele_1, \ele_2, R_1, R_2$.

    \begin{tikzpicture}[circuit ee IEC, thick]
        \draw  (0, 0) -- ++(up:2)
                to [
                    battery={ very near start, rotate=-180, info={$\ele_1, r_1 $}},
                    resistor={ midway, info=$R_1$},
                    battery={ very near end, rotate=-180, info={$\ele_2, r_2 $}}
                ] ++(right:5)
                -- ++(down:2)
                to [resistor={info=$R_2$}] ++(left:5);
    \end{tikzpicture}
}
\answer{%
    Нетривиальных узлов нет, поэтому все законы Кирхгофа для узлов будут иметь вид
    $\eli-\eli=0$ и ничем нам не помогут.
    Впрочем, если бы мы обозначили токи на разных участках контура $\eli_1, \eli_2, \eli_3, \ldots$,
    то именно эти законы бы помогли понять, что все эти токи равны: $\eli_1 - \eli_2 = 0$ и т.д.
    Так что запишем закон Кирхгофа для единственного замкнутого контура:

    \begin{tikzpicture}[circuit ee IEC, thick]
        \draw  (0, 0) -- ++(up:2)
                to [
                    battery={ very near start, rotate=-180, info={$\ele_1, r_1 $}},
                    resistor={ midway, info=$R_1$},
                    battery={ very near end, rotate=-180, info={$\ele_2, r_2 $}}
                ] ++(right:5)
                -- ++(down:2)
                to [resistor={info=$R_2$}, current direction={near end, info=$\eli$}] ++(left:5);
        \draw [-{Latex}] (2, 1.4) arc [start angle = 135, end angle = -160, radius = 0.6];
    \end{tikzpicture}

    \begin{align*}
        & \ele_1 +  \ele_2 = \eli R_1 + \eli r_2 + \eli R_2 + \eli r_1, \\
        & \ele_1 +  \ele_2 = \eli (R_1 + r_2 + R_2 + r_1), \\
        &\eli = \frac{ \ele_1 +  \ele_2 }{ R_1 + r_2 + R_2 + r_1 }, \\
        &U_1 = \eli R_1 = \frac{ \ele_1 +  \ele_2 }{ R_1 + r_2 + R_2 + r_1 } \cdot R_1, \\
        &P_1 = \eli^2 R_1 = \frac{\sqr{ \ele_1 +  \ele_2 } R_1}{ \sqr{ R_1 + r_2 + R_2 + r_1 }}.
    \end{align*}

    Отметим, что ответ для тока $\eli$ меняет знак, если отметить его на рисунке в другую сторону.
    Поэтому критично важно указывать на рисунке направление тока, иначе невозможно утверждать, что ответ верный.
    А вот выбор направления контура — не повлияет на ответ, но для проверки корректности записи законо Кирхгофа,
    там тоже необходимо направление.
}

\variantsplitter

\addpersonalvariant{Анна Кузьмичёва}

\tasknumber{1}%
\task{%
    Два резистора сопротивлениями $R_1=3R$ и $R_2=2R$ подключены параллельно к источнику напряжения.
    Определите, в каком резисторе выделяется большая тепловая мощность и во сколько раз?
}
\answer{%
    Подключены параллельно, поэтому  $U_1 = U_2 = U \implies \frac{P_2}{P_1} = \frac{\frac{U_2^2}{R_2}}{\frac{U_1^2}{R_1}} = \frac{U^2R_1}{U^2R_2} = \frac{R_1}{R_2} = \frac32$.
}
\solutionspace{120pt}

\tasknumber{2}%
\task{%
    Если батарею замкнуть на резистор сопротивлением $R_1$, то в цепи потечёт ток $\eli_1$,
    а если на другой $R_2$ — то $\eli_2$.
    Определите:
    \begin{itemize}
        \item ЭДС батареи,
        \item внутреннее сопротивление батареи,
        \item ток короткого замыкания.
    \end{itemize}
}
\answer{%
    Запишем закон Ома для полной цепи 2 раза для обоих способов подключения (с $R_1$ и с $R_2$),
    а короткое замыкание рассмотрим позже.
    Отметим, что для такой простой схемы он совпадает
    с законом Кирхгофа.
    Получим систему из 2 уравнений и 2 неизвестных, решим в удобном порядке,
    ибо нам всё равно понадобятся обе.

    \begin{align*}
        &\begin{cases}
            \ele = \eli_1(R_1 + r), \\
            \ele = \eli_2(R_2 + r); \\
        \end{cases} \\
        &\eli_1(R_1 + r) = \eli_2(R_2 + r), \\
        &\eli_1 R_1 + \eli_1r = \eli_2 R_2 + \eli_2r, \\
        &\eli_1 R_1 - \eli_2 R_2 = - \eli_1r  + \eli_2r = (\eli_2 - \eli_1)r, \\
        r &= \frac{\eli_1 R_1 - \eli_2 R_2}{\eli_2 - \eli_1}
            \equiv \frac{\eli_2 R_2 - \eli_1 R_1}{\eli_1 - \eli_2}, \\
        \ele &= \eli_1(R_1 + r)
            = \eli_1\cbr{R_1 + \frac{\eli_1 R_1 - \eli_2 R_2}{\eli_2 - \eli_1}}
            = \eli_1 \cdot \frac{R_1\eli_2 - R_1\eli_1 + \eli_1 R_1 - \eli_2 R_2}{\eli_2 - \eli_1} \\
            &= \eli_1 \cdot \frac{R_1\eli_2 - \eli_2 R_2}{\eli_2 - \eli_1}
            = \frac{\eli_1 \eli_2 (R_1 - R_2)}{\eli_2 - \eli_1}
            \equiv \frac{\eli_1 \eli_2 (R_2 - R_1)}{\eli_1 - \eli_2}.
    \end{align*}

    Короткое замыкание происходит в ситуации, когда внешнее сопротивление равно 0
    (при этом цепь замкнута, хотя нагрузки и нет вовсе):
    $$
        \eli_\text{к.
        з.} = \frac \ele {0 + r} = \frac \ele r
            = \frac{\cfrac{\eli_1 \eli_2 (R_1 - R_2)}{\eli_2 - \eli_1}}{\cfrac{\eli_1 R_1 - \eli_2 R_2}{\eli_2 - \eli_1}}
            = \frac{\eli_1 \eli_2 (R_1 - R_2)}{\eli_1 R_1 - \eli_2 R_2}
            \equiv \frac{\eli_1 \eli_2 (R_2 - R_1)}{\eli_2 R_2 - \eli_1 R_1}.
    $$

    Важные пункты:
    \begin{itemize}
        \item В ответах есть только те величины, которые есть в условии
            (и ещё физические постоянные могут встретиться, но нам не понадобилось).
        \item Мы упростили выражения, который пошли в ответы (благо у нас даже получилось:
            приведение к общему знаменателю укоротило ответ).
            Надо доделывать.
        \item Всё ответы симметричны относительно замены резисторов 1 и 2 (ведь при этом изменятся и токи).
    \end{itemize}
}
\solutionspace{120pt}

\tasknumber{3}%
\task{%
    Определите ток, протекающий через резистор $R_1$, разность потенциалов на нём (см.
    рис.)
    и выделяющуюся на нём мощность, если известны $r_1, r_2, \ele_1, \ele_2, R_1, R_2$.

    \begin{tikzpicture}[circuit ee IEC, thick]
        \draw  (0, 0) -- ++(up:2)
                to [
                    battery={ very near start, rotate=-180, info={$\ele_1, r_1 $}},
                    resistor={ midway, info=$R_1$},
                    battery={ very near end, rotate=-180, info={$\ele_2, r_2 $}}
                ] ++(right:5)
                -- ++(down:2)
                to [resistor={info=$R_2$}] ++(left:5);
    \end{tikzpicture}
}
\answer{%
    Нетривиальных узлов нет, поэтому все законы Кирхгофа для узлов будут иметь вид
    $\eli-\eli=0$ и ничем нам не помогут.
    Впрочем, если бы мы обозначили токи на разных участках контура $\eli_1, \eli_2, \eli_3, \ldots$,
    то именно эти законы бы помогли понять, что все эти токи равны: $\eli_1 - \eli_2 = 0$ и т.д.
    Так что запишем закон Кирхгофа для единственного замкнутого контура:

    \begin{tikzpicture}[circuit ee IEC, thick]
        \draw  (0, 0) -- ++(up:2)
                to [
                    battery={ very near start, rotate=-180, info={$\ele_1, r_1 $}},
                    resistor={ midway, info=$R_1$},
                    battery={ very near end, rotate=-180, info={$\ele_2, r_2 $}}
                ] ++(right:5)
                -- ++(down:2)
                to [resistor={info=$R_2$}, current direction={near end, info=$\eli$}] ++(left:5);
        \draw [-{Latex}] (2, 1.4) arc [start angle = 135, end angle = -160, radius = 0.6];
    \end{tikzpicture}

    \begin{align*}
        & \ele_1 +  \ele_2 = \eli R_1 + \eli r_2 + \eli R_2 + \eli r_1, \\
        & \ele_1 +  \ele_2 = \eli (R_1 + r_2 + R_2 + r_1), \\
        &\eli = \frac{ \ele_1 +  \ele_2 }{ R_1 + r_2 + R_2 + r_1 }, \\
        &U_1 = \eli R_1 = \frac{ \ele_1 +  \ele_2 }{ R_1 + r_2 + R_2 + r_1 } \cdot R_1, \\
        &P_1 = \eli^2 R_1 = \frac{\sqr{ \ele_1 +  \ele_2 } R_1}{ \sqr{ R_1 + r_2 + R_2 + r_1 }}.
    \end{align*}

    Отметим, что ответ для тока $\eli$ меняет знак, если отметить его на рисунке в другую сторону.
    Поэтому критично важно указывать на рисунке направление тока, иначе невозможно утверждать, что ответ верный.
    А вот выбор направления контура — не повлияет на ответ, но для проверки корректности записи законо Кирхгофа,
    там тоже необходимо направление.
}

\variantsplitter

\addpersonalvariant{Алёна Куприянова}

\tasknumber{1}%
\task{%
    Два резистора сопротивлениями $R_1=3R$ и $R_2=8R$ подключены последовательно к источнику напряжения.
    Определите, в каком резисторе выделяется большая тепловая мощность и во сколько раз?
}
\answer{%
    Подключены последовательно, поэтому  $\eli_1 = \eli_2 = \eli \implies \frac{P_2}{P_1} = \frac{\eli_2^2 R_2}{\eli_1^2 R_1} = \frac{\eli_2^2R_2}{\eli_1R_1} = \frac{R_2}{R_1} = \frac83$.
}
\solutionspace{120pt}

\tasknumber{2}%
\task{%
    Если батарею замкнуть на резистор сопротивлением $R_1$, то в цепи потечёт ток $\eli_1$,
    а если на другой $R_2$ — то $\eli_2$.
    Определите:
    \begin{itemize}
        \item ЭДС батареи,
        \item внутреннее сопротивление батареи,
        \item ток короткого замыкания.
    \end{itemize}
}
\answer{%
    Запишем закон Ома для полной цепи 2 раза для обоих способов подключения (с $R_1$ и с $R_2$),
    а короткое замыкание рассмотрим позже.
    Отметим, что для такой простой схемы он совпадает
    с законом Кирхгофа.
    Получим систему из 2 уравнений и 2 неизвестных, решим в удобном порядке,
    ибо нам всё равно понадобятся обе.

    \begin{align*}
        &\begin{cases}
            \ele = \eli_1(R_1 + r), \\
            \ele = \eli_2(R_2 + r); \\
        \end{cases} \\
        &\eli_1(R_1 + r) = \eli_2(R_2 + r), \\
        &\eli_1 R_1 + \eli_1r = \eli_2 R_2 + \eli_2r, \\
        &\eli_1 R_1 - \eli_2 R_2 = - \eli_1r  + \eli_2r = (\eli_2 - \eli_1)r, \\
        r &= \frac{\eli_1 R_1 - \eli_2 R_2}{\eli_2 - \eli_1}
            \equiv \frac{\eli_2 R_2 - \eli_1 R_1}{\eli_1 - \eli_2}, \\
        \ele &= \eli_1(R_1 + r)
            = \eli_1\cbr{R_1 + \frac{\eli_1 R_1 - \eli_2 R_2}{\eli_2 - \eli_1}}
            = \eli_1 \cdot \frac{R_1\eli_2 - R_1\eli_1 + \eli_1 R_1 - \eli_2 R_2}{\eli_2 - \eli_1} \\
            &= \eli_1 \cdot \frac{R_1\eli_2 - \eli_2 R_2}{\eli_2 - \eli_1}
            = \frac{\eli_1 \eli_2 (R_1 - R_2)}{\eli_2 - \eli_1}
            \equiv \frac{\eli_1 \eli_2 (R_2 - R_1)}{\eli_1 - \eli_2}.
    \end{align*}

    Короткое замыкание происходит в ситуации, когда внешнее сопротивление равно 0
    (при этом цепь замкнута, хотя нагрузки и нет вовсе):
    $$
        \eli_\text{к.
        з.} = \frac \ele {0 + r} = \frac \ele r
            = \frac{\cfrac{\eli_1 \eli_2 (R_1 - R_2)}{\eli_2 - \eli_1}}{\cfrac{\eli_1 R_1 - \eli_2 R_2}{\eli_2 - \eli_1}}
            = \frac{\eli_1 \eli_2 (R_1 - R_2)}{\eli_1 R_1 - \eli_2 R_2}
            \equiv \frac{\eli_1 \eli_2 (R_2 - R_1)}{\eli_2 R_2 - \eli_1 R_1}.
    $$

    Важные пункты:
    \begin{itemize}
        \item В ответах есть только те величины, которые есть в условии
            (и ещё физические постоянные могут встретиться, но нам не понадобилось).
        \item Мы упростили выражения, который пошли в ответы (благо у нас даже получилось:
            приведение к общему знаменателю укоротило ответ).
            Надо доделывать.
        \item Всё ответы симметричны относительно замены резисторов 1 и 2 (ведь при этом изменятся и токи).
    \end{itemize}
}
\solutionspace{120pt}

\tasknumber{3}%
\task{%
    Определите ток, протекающий через резистор $R_2$, разность потенциалов на нём (см.
    рис.)
    и выделяющуюся на нём мощность, если известны $r_1, r_2, \ele_1, \ele_2, R_1, R_2$.

    \begin{tikzpicture}[circuit ee IEC, thick]
        \draw  (0, 0) -- ++(up:2)
                to [
                    battery={ very near start, rotate=-180, info={$\ele_1, r_1 $}},
                    resistor={ midway, info=$R_1$},
                    battery={ very near end, rotate=0, info={$\ele_2, r_2 $}}
                ] ++(right:5)
                -- ++(down:2)
                to [resistor={info=$R_2$}] ++(left:5);
    \end{tikzpicture}
}
\answer{%
    Нетривиальных узлов нет, поэтому все законы Кирхгофа для узлов будут иметь вид
    $\eli-\eli=0$ и ничем нам не помогут.
    Впрочем, если бы мы обозначили токи на разных участках контура $\eli_1, \eli_2, \eli_3, \ldots$,
    то именно эти законы бы помогли понять, что все эти токи равны: $\eli_1 - \eli_2 = 0$ и т.д.
    Так что запишем закон Кирхгофа для единственного замкнутого контура:

    \begin{tikzpicture}[circuit ee IEC, thick]
        \draw  (0, 0) -- ++(up:2)
                to [
                    battery={ very near start, rotate=-180, info={$\ele_1, r_1 $}},
                    resistor={ midway, info=$R_1$},
                    battery={ very near end, rotate=0, info={$\ele_2, r_2 $}}
                ] ++(right:5)
                -- ++(down:2)
                to [resistor={info=$R_2$}, current direction={near end, info=$\eli$}] ++(left:5);
        \draw [-{Latex}] (2, 1.4) arc [start angle = 135, end angle = -160, radius = 0.6];
    \end{tikzpicture}

    \begin{align*}
        & \ele_1 - \ele_2 = \eli R_1 + \eli r_2 + \eli R_2 + \eli r_1, \\
        & \ele_1 - \ele_2 = \eli (R_1 + r_2 + R_2 + r_1), \\
        &\eli = \frac{ \ele_1 - \ele_2 }{ R_1 + r_2 + R_2 + r_1 }, \\
        &U_2 = \eli R_2 = \frac{ \ele_1 - \ele_2 }{ R_1 + r_2 + R_2 + r_1 } \cdot R_2, \\
        &P_2 = \eli^2 R_2 = \frac{\sqr{ \ele_1 - \ele_2 } R_2}{ \sqr{ R_1 + r_2 + R_2 + r_1 }}.
    \end{align*}

    Отметим, что ответ для тока $\eli$ меняет знак, если отметить его на рисунке в другую сторону.
    Поэтому критично важно указывать на рисунке направление тока, иначе невозможно утверждать, что ответ верный.
    А вот выбор направления контура — не повлияет на ответ, но для проверки корректности записи законо Кирхгофа,
    там тоже необходимо направление.
}

\variantsplitter

\addpersonalvariant{Ярослав Лавровский}

\tasknumber{1}%
\task{%
    Два резистора сопротивлениями $R_1=5R$ и $R_2=4R$ подключены параллельно к источнику напряжения.
    Определите, в каком резисторе выделяется большая тепловая мощность и во сколько раз?
}
\answer{%
    Подключены параллельно, поэтому  $U_1 = U_2 = U \implies \frac{P_2}{P_1} = \frac{\frac{U_2^2}{R_2}}{\frac{U_1^2}{R_1}} = \frac{U^2R_1}{U^2R_2} = \frac{R_1}{R_2} = \frac54$.
}
\solutionspace{120pt}

\tasknumber{2}%
\task{%
    Если батарею замкнуть на резистор сопротивлением $R_1$, то в цепи потечёт ток $\eli_1$,
    а если на другой $R_2$ — то $\eli_2$.
    Определите:
    \begin{itemize}
        \item ЭДС батареи,
        \item внутреннее сопротивление батареи,
        \item ток короткого замыкания.
    \end{itemize}
}
\answer{%
    Запишем закон Ома для полной цепи 2 раза для обоих способов подключения (с $R_1$ и с $R_2$),
    а короткое замыкание рассмотрим позже.
    Отметим, что для такой простой схемы он совпадает
    с законом Кирхгофа.
    Получим систему из 2 уравнений и 2 неизвестных, решим в удобном порядке,
    ибо нам всё равно понадобятся обе.

    \begin{align*}
        &\begin{cases}
            \ele = \eli_1(R_1 + r), \\
            \ele = \eli_2(R_2 + r); \\
        \end{cases} \\
        &\eli_1(R_1 + r) = \eli_2(R_2 + r), \\
        &\eli_1 R_1 + \eli_1r = \eli_2 R_2 + \eli_2r, \\
        &\eli_1 R_1 - \eli_2 R_2 = - \eli_1r  + \eli_2r = (\eli_2 - \eli_1)r, \\
        r &= \frac{\eli_1 R_1 - \eli_2 R_2}{\eli_2 - \eli_1}
            \equiv \frac{\eli_2 R_2 - \eli_1 R_1}{\eli_1 - \eli_2}, \\
        \ele &= \eli_1(R_1 + r)
            = \eli_1\cbr{R_1 + \frac{\eli_1 R_1 - \eli_2 R_2}{\eli_2 - \eli_1}}
            = \eli_1 \cdot \frac{R_1\eli_2 - R_1\eli_1 + \eli_1 R_1 - \eli_2 R_2}{\eli_2 - \eli_1} \\
            &= \eli_1 \cdot \frac{R_1\eli_2 - \eli_2 R_2}{\eli_2 - \eli_1}
            = \frac{\eli_1 \eli_2 (R_1 - R_2)}{\eli_2 - \eli_1}
            \equiv \frac{\eli_1 \eli_2 (R_2 - R_1)}{\eli_1 - \eli_2}.
    \end{align*}

    Короткое замыкание происходит в ситуации, когда внешнее сопротивление равно 0
    (при этом цепь замкнута, хотя нагрузки и нет вовсе):
    $$
        \eli_\text{к.
        з.} = \frac \ele {0 + r} = \frac \ele r
            = \frac{\cfrac{\eli_1 \eli_2 (R_1 - R_2)}{\eli_2 - \eli_1}}{\cfrac{\eli_1 R_1 - \eli_2 R_2}{\eli_2 - \eli_1}}
            = \frac{\eli_1 \eli_2 (R_1 - R_2)}{\eli_1 R_1 - \eli_2 R_2}
            \equiv \frac{\eli_1 \eli_2 (R_2 - R_1)}{\eli_2 R_2 - \eli_1 R_1}.
    $$

    Важные пункты:
    \begin{itemize}
        \item В ответах есть только те величины, которые есть в условии
            (и ещё физические постоянные могут встретиться, но нам не понадобилось).
        \item Мы упростили выражения, который пошли в ответы (благо у нас даже получилось:
            приведение к общему знаменателю укоротило ответ).
            Надо доделывать.
        \item Всё ответы симметричны относительно замены резисторов 1 и 2 (ведь при этом изменятся и токи).
    \end{itemize}
}
\solutionspace{120pt}

\tasknumber{3}%
\task{%
    Определите ток, протекающий через резистор $R_1$, разность потенциалов на нём (см.
    рис.)
    и выделяющуюся на нём мощность, если известны $r_1, r_2, \ele_1, \ele_2, R_1, R_2$.

    \begin{tikzpicture}[circuit ee IEC, thick]
        \draw  (0, 0) -- ++(up:2)
                to [
                    battery={ very near start, rotate=-180, info={$\ele_1, r_1 $}},
                    resistor={ midway, info=$R_1$},
                    battery={ very near end, rotate=0, info={$\ele_2, r_2 $}}
                ] ++(right:5)
                -- ++(down:2)
                to [resistor={info=$R_2$}] ++(left:5);
    \end{tikzpicture}
}
\answer{%
    Нетривиальных узлов нет, поэтому все законы Кирхгофа для узлов будут иметь вид
    $\eli-\eli=0$ и ничем нам не помогут.
    Впрочем, если бы мы обозначили токи на разных участках контура $\eli_1, \eli_2, \eli_3, \ldots$,
    то именно эти законы бы помогли понять, что все эти токи равны: $\eli_1 - \eli_2 = 0$ и т.д.
    Так что запишем закон Кирхгофа для единственного замкнутого контура:

    \begin{tikzpicture}[circuit ee IEC, thick]
        \draw  (0, 0) -- ++(up:2)
                to [
                    battery={ very near start, rotate=-180, info={$\ele_1, r_1 $}},
                    resistor={ midway, info=$R_1$},
                    battery={ very near end, rotate=0, info={$\ele_2, r_2 $}}
                ] ++(right:5)
                -- ++(down:2)
                to [resistor={info=$R_2$}, current direction={near end, info=$\eli$}] ++(left:5);
        \draw [-{Latex}] (2, 1.4) arc [start angle = 135, end angle = -160, radius = 0.6];
    \end{tikzpicture}

    \begin{align*}
        & \ele_1 - \ele_2 = \eli R_1 + \eli r_2 + \eli R_2 + \eli r_1, \\
        & \ele_1 - \ele_2 = \eli (R_1 + r_2 + R_2 + r_1), \\
        &\eli = \frac{ \ele_1 - \ele_2 }{ R_1 + r_2 + R_2 + r_1 }, \\
        &U_1 = \eli R_1 = \frac{ \ele_1 - \ele_2 }{ R_1 + r_2 + R_2 + r_1 } \cdot R_1, \\
        &P_1 = \eli^2 R_1 = \frac{\sqr{ \ele_1 - \ele_2 } R_1}{ \sqr{ R_1 + r_2 + R_2 + r_1 }}.
    \end{align*}

    Отметим, что ответ для тока $\eli$ меняет знак, если отметить его на рисунке в другую сторону.
    Поэтому критично важно указывать на рисунке направление тока, иначе невозможно утверждать, что ответ верный.
    А вот выбор направления контура — не повлияет на ответ, но для проверки корректности записи законо Кирхгофа,
    там тоже необходимо направление.
}

\variantsplitter

\addpersonalvariant{Анастасия Ламанова}

\tasknumber{1}%
\task{%
    Два резистора сопротивлениями $R_1=3R$ и $R_2=8R$ подключены последовательно к источнику напряжения.
    Определите, в каком резисторе выделяется большая тепловая мощность и во сколько раз?
}
\answer{%
    Подключены последовательно, поэтому  $\eli_1 = \eli_2 = \eli \implies \frac{P_2}{P_1} = \frac{\eli_2^2 R_2}{\eli_1^2 R_1} = \frac{\eli_2^2R_2}{\eli_1R_1} = \frac{R_2}{R_1} = \frac83$.
}
\solutionspace{120pt}

\tasknumber{2}%
\task{%
    Если батарею замкнуть на резистор сопротивлением $R_1$, то в цепи потечёт ток $\eli_1$,
    а если на другой $R_2$ — то $\eli_2$.
    Определите:
    \begin{itemize}
        \item ЭДС батареи,
        \item внутреннее сопротивление батареи,
        \item ток короткого замыкания.
    \end{itemize}
}
\answer{%
    Запишем закон Ома для полной цепи 2 раза для обоих способов подключения (с $R_1$ и с $R_2$),
    а короткое замыкание рассмотрим позже.
    Отметим, что для такой простой схемы он совпадает
    с законом Кирхгофа.
    Получим систему из 2 уравнений и 2 неизвестных, решим в удобном порядке,
    ибо нам всё равно понадобятся обе.

    \begin{align*}
        &\begin{cases}
            \ele = \eli_1(R_1 + r), \\
            \ele = \eli_2(R_2 + r); \\
        \end{cases} \\
        &\eli_1(R_1 + r) = \eli_2(R_2 + r), \\
        &\eli_1 R_1 + \eli_1r = \eli_2 R_2 + \eli_2r, \\
        &\eli_1 R_1 - \eli_2 R_2 = - \eli_1r  + \eli_2r = (\eli_2 - \eli_1)r, \\
        r &= \frac{\eli_1 R_1 - \eli_2 R_2}{\eli_2 - \eli_1}
            \equiv \frac{\eli_2 R_2 - \eli_1 R_1}{\eli_1 - \eli_2}, \\
        \ele &= \eli_1(R_1 + r)
            = \eli_1\cbr{R_1 + \frac{\eli_1 R_1 - \eli_2 R_2}{\eli_2 - \eli_1}}
            = \eli_1 \cdot \frac{R_1\eli_2 - R_1\eli_1 + \eli_1 R_1 - \eli_2 R_2}{\eli_2 - \eli_1} \\
            &= \eli_1 \cdot \frac{R_1\eli_2 - \eli_2 R_2}{\eli_2 - \eli_1}
            = \frac{\eli_1 \eli_2 (R_1 - R_2)}{\eli_2 - \eli_1}
            \equiv \frac{\eli_1 \eli_2 (R_2 - R_1)}{\eli_1 - \eli_2}.
    \end{align*}

    Короткое замыкание происходит в ситуации, когда внешнее сопротивление равно 0
    (при этом цепь замкнута, хотя нагрузки и нет вовсе):
    $$
        \eli_\text{к.
        з.} = \frac \ele {0 + r} = \frac \ele r
            = \frac{\cfrac{\eli_1 \eli_2 (R_1 - R_2)}{\eli_2 - \eli_1}}{\cfrac{\eli_1 R_1 - \eli_2 R_2}{\eli_2 - \eli_1}}
            = \frac{\eli_1 \eli_2 (R_1 - R_2)}{\eli_1 R_1 - \eli_2 R_2}
            \equiv \frac{\eli_1 \eli_2 (R_2 - R_1)}{\eli_2 R_2 - \eli_1 R_1}.
    $$

    Важные пункты:
    \begin{itemize}
        \item В ответах есть только те величины, которые есть в условии
            (и ещё физические постоянные могут встретиться, но нам не понадобилось).
        \item Мы упростили выражения, который пошли в ответы (благо у нас даже получилось:
            приведение к общему знаменателю укоротило ответ).
            Надо доделывать.
        \item Всё ответы симметричны относительно замены резисторов 1 и 2 (ведь при этом изменятся и токи).
    \end{itemize}
}
\solutionspace{120pt}

\tasknumber{3}%
\task{%
    Определите ток, протекающий через резистор $R_1$, разность потенциалов на нём (см.
    рис.)
    и выделяющуюся на нём мощность, если известны $r_1, r_2, \ele_1, \ele_2, R_1, R_2$.

    \begin{tikzpicture}[circuit ee IEC, thick]
        \draw  (0, 0) -- ++(up:2)
                to [
                    battery={ very near start, rotate=0, info={$\ele_1, r_1 $}},
                    resistor={ midway, info=$R_1$},
                    battery={ very near end, rotate=-180, info={$\ele_2, r_2 $}}
                ] ++(right:5)
                -- ++(down:2)
                to [resistor={info=$R_2$}] ++(left:5);
    \end{tikzpicture}
}
\answer{%
    Нетривиальных узлов нет, поэтому все законы Кирхгофа для узлов будут иметь вид
    $\eli-\eli=0$ и ничем нам не помогут.
    Впрочем, если бы мы обозначили токи на разных участках контура $\eli_1, \eli_2, \eli_3, \ldots$,
    то именно эти законы бы помогли понять, что все эти токи равны: $\eli_1 - \eli_2 = 0$ и т.д.
    Так что запишем закон Кирхгофа для единственного замкнутого контура:

    \begin{tikzpicture}[circuit ee IEC, thick]
        \draw  (0, 0) -- ++(up:2)
                to [
                    battery={ very near start, rotate=0, info={$\ele_1, r_1 $}},
                    resistor={ midway, info=$R_1$},
                    battery={ very near end, rotate=-180, info={$\ele_2, r_2 $}}
                ] ++(right:5)
                -- ++(down:2)
                to [resistor={info=$R_2$}, current direction={near end, info=$\eli$}] ++(left:5);
        \draw [-{Latex}] (2, 1.4) arc [start angle = 135, end angle = -160, radius = 0.6];
    \end{tikzpicture}

    \begin{align*}
        &- \ele_1 +  \ele_2 = \eli R_1 + \eli r_2 + \eli R_2 + \eli r_1, \\
        &- \ele_1 +  \ele_2 = \eli (R_1 + r_2 + R_2 + r_1), \\
        &\eli = \frac{- \ele_1 +  \ele_2 }{ R_1 + r_2 + R_2 + r_1 }, \\
        &U_1 = \eli R_1 = \frac{- \ele_1 +  \ele_2 }{ R_1 + r_2 + R_2 + r_1 } \cdot R_1, \\
        &P_1 = \eli^2 R_1 = \frac{\sqr{- \ele_1 +  \ele_2 } R_1}{ \sqr{ R_1 + r_2 + R_2 + r_1 }}.
    \end{align*}

    Отметим, что ответ для тока $\eli$ меняет знак, если отметить его на рисунке в другую сторону.
    Поэтому критично важно указывать на рисунке направление тока, иначе невозможно утверждать, что ответ верный.
    А вот выбор направления контура — не повлияет на ответ, но для проверки корректности записи законо Кирхгофа,
    там тоже необходимо направление.
}

\variantsplitter

\addpersonalvariant{Виктория Легонькова}

\tasknumber{1}%
\task{%
    Два резистора сопротивлениями $R_1=5R$ и $R_2=4R$ подключены последовательно к источнику напряжения.
    Определите, в каком резисторе выделяется большая тепловая мощность и во сколько раз?
}
\answer{%
    Подключены последовательно, поэтому  $\eli_1 = \eli_2 = \eli \implies \frac{P_2}{P_1} = \frac{\eli_2^2 R_2}{\eli_1^2 R_1} = \frac{\eli_2^2R_2}{\eli_1R_1} = \frac{R_2}{R_1} = \frac45$.
}
\solutionspace{120pt}

\tasknumber{2}%
\task{%
    Если батарею замкнуть на резистор сопротивлением $R_1$, то в цепи потечёт ток $\eli_1$,
    а если на другой $R_2$ — то $\eli_2$.
    Определите:
    \begin{itemize}
        \item ЭДС батареи,
        \item внутреннее сопротивление батареи,
        \item ток короткого замыкания.
    \end{itemize}
}
\answer{%
    Запишем закон Ома для полной цепи 2 раза для обоих способов подключения (с $R_1$ и с $R_2$),
    а короткое замыкание рассмотрим позже.
    Отметим, что для такой простой схемы он совпадает
    с законом Кирхгофа.
    Получим систему из 2 уравнений и 2 неизвестных, решим в удобном порядке,
    ибо нам всё равно понадобятся обе.

    \begin{align*}
        &\begin{cases}
            \ele = \eli_1(R_1 + r), \\
            \ele = \eli_2(R_2 + r); \\
        \end{cases} \\
        &\eli_1(R_1 + r) = \eli_2(R_2 + r), \\
        &\eli_1 R_1 + \eli_1r = \eli_2 R_2 + \eli_2r, \\
        &\eli_1 R_1 - \eli_2 R_2 = - \eli_1r  + \eli_2r = (\eli_2 - \eli_1)r, \\
        r &= \frac{\eli_1 R_1 - \eli_2 R_2}{\eli_2 - \eli_1}
            \equiv \frac{\eli_2 R_2 - \eli_1 R_1}{\eli_1 - \eli_2}, \\
        \ele &= \eli_1(R_1 + r)
            = \eli_1\cbr{R_1 + \frac{\eli_1 R_1 - \eli_2 R_2}{\eli_2 - \eli_1}}
            = \eli_1 \cdot \frac{R_1\eli_2 - R_1\eli_1 + \eli_1 R_1 - \eli_2 R_2}{\eli_2 - \eli_1} \\
            &= \eli_1 \cdot \frac{R_1\eli_2 - \eli_2 R_2}{\eli_2 - \eli_1}
            = \frac{\eli_1 \eli_2 (R_1 - R_2)}{\eli_2 - \eli_1}
            \equiv \frac{\eli_1 \eli_2 (R_2 - R_1)}{\eli_1 - \eli_2}.
    \end{align*}

    Короткое замыкание происходит в ситуации, когда внешнее сопротивление равно 0
    (при этом цепь замкнута, хотя нагрузки и нет вовсе):
    $$
        \eli_\text{к.
        з.} = \frac \ele {0 + r} = \frac \ele r
            = \frac{\cfrac{\eli_1 \eli_2 (R_1 - R_2)}{\eli_2 - \eli_1}}{\cfrac{\eli_1 R_1 - \eli_2 R_2}{\eli_2 - \eli_1}}
            = \frac{\eli_1 \eli_2 (R_1 - R_2)}{\eli_1 R_1 - \eli_2 R_2}
            \equiv \frac{\eli_1 \eli_2 (R_2 - R_1)}{\eli_2 R_2 - \eli_1 R_1}.
    $$

    Важные пункты:
    \begin{itemize}
        \item В ответах есть только те величины, которые есть в условии
            (и ещё физические постоянные могут встретиться, но нам не понадобилось).
        \item Мы упростили выражения, который пошли в ответы (благо у нас даже получилось:
            приведение к общему знаменателю укоротило ответ).
            Надо доделывать.
        \item Всё ответы симметричны относительно замены резисторов 1 и 2 (ведь при этом изменятся и токи).
    \end{itemize}
}
\solutionspace{120pt}

\tasknumber{3}%
\task{%
    Определите ток, протекающий через резистор $R_1$, разность потенциалов на нём (см.
    рис.)
    и выделяющуюся на нём мощность, если известны $r_1, r_2, \ele_1, \ele_2, R_1, R_2$.

    \begin{tikzpicture}[circuit ee IEC, thick]
        \draw  (0, 0) -- ++(up:2)
                to [
                    battery={ very near start, rotate=0, info={$\ele_1, r_1 $}},
                    resistor={ midway, info=$R_1$},
                    battery={ very near end, rotate=0, info={$\ele_2, r_2 $}}
                ] ++(right:5)
                -- ++(down:2)
                to [resistor={info=$R_2$}] ++(left:5);
    \end{tikzpicture}
}
\answer{%
    Нетривиальных узлов нет, поэтому все законы Кирхгофа для узлов будут иметь вид
    $\eli-\eli=0$ и ничем нам не помогут.
    Впрочем, если бы мы обозначили токи на разных участках контура $\eli_1, \eli_2, \eli_3, \ldots$,
    то именно эти законы бы помогли понять, что все эти токи равны: $\eli_1 - \eli_2 = 0$ и т.д.
    Так что запишем закон Кирхгофа для единственного замкнутого контура:

    \begin{tikzpicture}[circuit ee IEC, thick]
        \draw  (0, 0) -- ++(up:2)
                to [
                    battery={ very near start, rotate=0, info={$\ele_1, r_1 $}},
                    resistor={ midway, info=$R_1$},
                    battery={ very near end, rotate=0, info={$\ele_2, r_2 $}}
                ] ++(right:5)
                -- ++(down:2)
                to [resistor={info=$R_2$}, current direction={near end, info=$\eli$}] ++(left:5);
        \draw [-{Latex}] (2, 1.4) arc [start angle = 135, end angle = -160, radius = 0.6];
    \end{tikzpicture}

    \begin{align*}
        &- \ele_1 - \ele_2 = \eli R_1 + \eli r_2 + \eli R_2 + \eli r_1, \\
        &- \ele_1 - \ele_2 = \eli (R_1 + r_2 + R_2 + r_1), \\
        &\eli = \frac{- \ele_1 - \ele_2 }{ R_1 + r_2 + R_2 + r_1 }, \\
        &U_1 = \eli R_1 = \frac{- \ele_1 - \ele_2 }{ R_1 + r_2 + R_2 + r_1 } \cdot R_1, \\
        &P_1 = \eli^2 R_1 = \frac{\sqr{- \ele_1 - \ele_2 } R_1}{ \sqr{ R_1 + r_2 + R_2 + r_1 }}.
    \end{align*}

    Отметим, что ответ для тока $\eli$ меняет знак, если отметить его на рисунке в другую сторону.
    Поэтому критично важно указывать на рисунке направление тока, иначе невозможно утверждать, что ответ верный.
    А вот выбор направления контура — не повлияет на ответ, но для проверки корректности записи законо Кирхгофа,
    там тоже необходимо направление.
}

\variantsplitter

\addpersonalvariant{Семён Мартынов}

\tasknumber{1}%
\task{%
    Два резистора сопротивлениями $R_1=7R$ и $R_2=4R$ подключены последовательно к источнику напряжения.
    Определите, в каком резисторе выделяется большая тепловая мощность и во сколько раз?
}
\answer{%
    Подключены последовательно, поэтому  $\eli_1 = \eli_2 = \eli \implies \frac{P_2}{P_1} = \frac{\eli_2^2 R_2}{\eli_1^2 R_1} = \frac{\eli_2^2R_2}{\eli_1R_1} = \frac{R_2}{R_1} = \frac47$.
}
\solutionspace{120pt}

\tasknumber{2}%
\task{%
    Если батарею замкнуть на резистор сопротивлением $R_1$, то в цепи потечёт ток $\eli_1$,
    а если на другой $R_2$ — то $\eli_2$.
    Определите:
    \begin{itemize}
        \item ЭДС батареи,
        \item внутреннее сопротивление батареи,
        \item ток короткого замыкания.
    \end{itemize}
}
\answer{%
    Запишем закон Ома для полной цепи 2 раза для обоих способов подключения (с $R_1$ и с $R_2$),
    а короткое замыкание рассмотрим позже.
    Отметим, что для такой простой схемы он совпадает
    с законом Кирхгофа.
    Получим систему из 2 уравнений и 2 неизвестных, решим в удобном порядке,
    ибо нам всё равно понадобятся обе.

    \begin{align*}
        &\begin{cases}
            \ele = \eli_1(R_1 + r), \\
            \ele = \eli_2(R_2 + r); \\
        \end{cases} \\
        &\eli_1(R_1 + r) = \eli_2(R_2 + r), \\
        &\eli_1 R_1 + \eli_1r = \eli_2 R_2 + \eli_2r, \\
        &\eli_1 R_1 - \eli_2 R_2 = - \eli_1r  + \eli_2r = (\eli_2 - \eli_1)r, \\
        r &= \frac{\eli_1 R_1 - \eli_2 R_2}{\eli_2 - \eli_1}
            \equiv \frac{\eli_2 R_2 - \eli_1 R_1}{\eli_1 - \eli_2}, \\
        \ele &= \eli_1(R_1 + r)
            = \eli_1\cbr{R_1 + \frac{\eli_1 R_1 - \eli_2 R_2}{\eli_2 - \eli_1}}
            = \eli_1 \cdot \frac{R_1\eli_2 - R_1\eli_1 + \eli_1 R_1 - \eli_2 R_2}{\eli_2 - \eli_1} \\
            &= \eli_1 \cdot \frac{R_1\eli_2 - \eli_2 R_2}{\eli_2 - \eli_1}
            = \frac{\eli_1 \eli_2 (R_1 - R_2)}{\eli_2 - \eli_1}
            \equiv \frac{\eli_1 \eli_2 (R_2 - R_1)}{\eli_1 - \eli_2}.
    \end{align*}

    Короткое замыкание происходит в ситуации, когда внешнее сопротивление равно 0
    (при этом цепь замкнута, хотя нагрузки и нет вовсе):
    $$
        \eli_\text{к.
        з.} = \frac \ele {0 + r} = \frac \ele r
            = \frac{\cfrac{\eli_1 \eli_2 (R_1 - R_2)}{\eli_2 - \eli_1}}{\cfrac{\eli_1 R_1 - \eli_2 R_2}{\eli_2 - \eli_1}}
            = \frac{\eli_1 \eli_2 (R_1 - R_2)}{\eli_1 R_1 - \eli_2 R_2}
            \equiv \frac{\eli_1 \eli_2 (R_2 - R_1)}{\eli_2 R_2 - \eli_1 R_1}.
    $$

    Важные пункты:
    \begin{itemize}
        \item В ответах есть только те величины, которые есть в условии
            (и ещё физические постоянные могут встретиться, но нам не понадобилось).
        \item Мы упростили выражения, который пошли в ответы (благо у нас даже получилось:
            приведение к общему знаменателю укоротило ответ).
            Надо доделывать.
        \item Всё ответы симметричны относительно замены резисторов 1 и 2 (ведь при этом изменятся и токи).
    \end{itemize}
}
\solutionspace{120pt}

\tasknumber{3}%
\task{%
    Определите ток, протекающий через резистор $R_2$, разность потенциалов на нём (см.
    рис.)
    и выделяющуюся на нём мощность, если известны $r_1, r_2, \ele_1, \ele_2, R_1, R_2$.

    \begin{tikzpicture}[circuit ee IEC, thick]
        \draw  (0, 0) -- ++(up:2)
                to [
                    battery={ very near start, rotate=-180, info={$\ele_1, r_1 $}},
                    resistor={ midway, info=$R_1$},
                    battery={ very near end, rotate=-180, info={$\ele_2, r_2 $}}
                ] ++(right:5)
                -- ++(down:2)
                to [resistor={info=$R_2$}] ++(left:5);
    \end{tikzpicture}
}
\answer{%
    Нетривиальных узлов нет, поэтому все законы Кирхгофа для узлов будут иметь вид
    $\eli-\eli=0$ и ничем нам не помогут.
    Впрочем, если бы мы обозначили токи на разных участках контура $\eli_1, \eli_2, \eli_3, \ldots$,
    то именно эти законы бы помогли понять, что все эти токи равны: $\eli_1 - \eli_2 = 0$ и т.д.
    Так что запишем закон Кирхгофа для единственного замкнутого контура:

    \begin{tikzpicture}[circuit ee IEC, thick]
        \draw  (0, 0) -- ++(up:2)
                to [
                    battery={ very near start, rotate=-180, info={$\ele_1, r_1 $}},
                    resistor={ midway, info=$R_1$},
                    battery={ very near end, rotate=-180, info={$\ele_2, r_2 $}}
                ] ++(right:5)
                -- ++(down:2)
                to [resistor={info=$R_2$}, current direction={near end, info=$\eli$}] ++(left:5);
        \draw [-{Latex}] (2, 1.4) arc [start angle = 135, end angle = -160, radius = 0.6];
    \end{tikzpicture}

    \begin{align*}
        & \ele_1 +  \ele_2 = \eli R_1 + \eli r_2 + \eli R_2 + \eli r_1, \\
        & \ele_1 +  \ele_2 = \eli (R_1 + r_2 + R_2 + r_1), \\
        &\eli = \frac{ \ele_1 +  \ele_2 }{ R_1 + r_2 + R_2 + r_1 }, \\
        &U_2 = \eli R_2 = \frac{ \ele_1 +  \ele_2 }{ R_1 + r_2 + R_2 + r_1 } \cdot R_2, \\
        &P_2 = \eli^2 R_2 = \frac{\sqr{ \ele_1 +  \ele_2 } R_2}{ \sqr{ R_1 + r_2 + R_2 + r_1 }}.
    \end{align*}

    Отметим, что ответ для тока $\eli$ меняет знак, если отметить его на рисунке в другую сторону.
    Поэтому критично важно указывать на рисунке направление тока, иначе невозможно утверждать, что ответ верный.
    А вот выбор направления контура — не повлияет на ответ, но для проверки корректности записи законо Кирхгофа,
    там тоже необходимо направление.
}

\variantsplitter

\addpersonalvariant{Варвара Минаева}

\tasknumber{1}%
\task{%
    Два резистора сопротивлениями $R_1=5R$ и $R_2=6R$ подключены последовательно к источнику напряжения.
    Определите, в каком резисторе выделяется большая тепловая мощность и во сколько раз?
}
\answer{%
    Подключены последовательно, поэтому  $\eli_1 = \eli_2 = \eli \implies \frac{P_2}{P_1} = \frac{\eli_2^2 R_2}{\eli_1^2 R_1} = \frac{\eli_2^2R_2}{\eli_1R_1} = \frac{R_2}{R_1} = \frac65$.
}
\solutionspace{120pt}

\tasknumber{2}%
\task{%
    Если батарею замкнуть на резистор сопротивлением $R_1$, то в цепи потечёт ток $\eli_1$,
    а если на другой $R_2$ — то $\eli_2$.
    Определите:
    \begin{itemize}
        \item ЭДС батареи,
        \item внутреннее сопротивление батареи,
        \item ток короткого замыкания.
    \end{itemize}
}
\answer{%
    Запишем закон Ома для полной цепи 2 раза для обоих способов подключения (с $R_1$ и с $R_2$),
    а короткое замыкание рассмотрим позже.
    Отметим, что для такой простой схемы он совпадает
    с законом Кирхгофа.
    Получим систему из 2 уравнений и 2 неизвестных, решим в удобном порядке,
    ибо нам всё равно понадобятся обе.

    \begin{align*}
        &\begin{cases}
            \ele = \eli_1(R_1 + r), \\
            \ele = \eli_2(R_2 + r); \\
        \end{cases} \\
        &\eli_1(R_1 + r) = \eli_2(R_2 + r), \\
        &\eli_1 R_1 + \eli_1r = \eli_2 R_2 + \eli_2r, \\
        &\eli_1 R_1 - \eli_2 R_2 = - \eli_1r  + \eli_2r = (\eli_2 - \eli_1)r, \\
        r &= \frac{\eli_1 R_1 - \eli_2 R_2}{\eli_2 - \eli_1}
            \equiv \frac{\eli_2 R_2 - \eli_1 R_1}{\eli_1 - \eli_2}, \\
        \ele &= \eli_1(R_1 + r)
            = \eli_1\cbr{R_1 + \frac{\eli_1 R_1 - \eli_2 R_2}{\eli_2 - \eli_1}}
            = \eli_1 \cdot \frac{R_1\eli_2 - R_1\eli_1 + \eli_1 R_1 - \eli_2 R_2}{\eli_2 - \eli_1} \\
            &= \eli_1 \cdot \frac{R_1\eli_2 - \eli_2 R_2}{\eli_2 - \eli_1}
            = \frac{\eli_1 \eli_2 (R_1 - R_2)}{\eli_2 - \eli_1}
            \equiv \frac{\eli_1 \eli_2 (R_2 - R_1)}{\eli_1 - \eli_2}.
    \end{align*}

    Короткое замыкание происходит в ситуации, когда внешнее сопротивление равно 0
    (при этом цепь замкнута, хотя нагрузки и нет вовсе):
    $$
        \eli_\text{к.
        з.} = \frac \ele {0 + r} = \frac \ele r
            = \frac{\cfrac{\eli_1 \eli_2 (R_1 - R_2)}{\eli_2 - \eli_1}}{\cfrac{\eli_1 R_1 - \eli_2 R_2}{\eli_2 - \eli_1}}
            = \frac{\eli_1 \eli_2 (R_1 - R_2)}{\eli_1 R_1 - \eli_2 R_2}
            \equiv \frac{\eli_1 \eli_2 (R_2 - R_1)}{\eli_2 R_2 - \eli_1 R_1}.
    $$

    Важные пункты:
    \begin{itemize}
        \item В ответах есть только те величины, которые есть в условии
            (и ещё физические постоянные могут встретиться, но нам не понадобилось).
        \item Мы упростили выражения, который пошли в ответы (благо у нас даже получилось:
            приведение к общему знаменателю укоротило ответ).
            Надо доделывать.
        \item Всё ответы симметричны относительно замены резисторов 1 и 2 (ведь при этом изменятся и токи).
    \end{itemize}
}
\solutionspace{120pt}

\tasknumber{3}%
\task{%
    Определите ток, протекающий через резистор $R_1$, разность потенциалов на нём (см.
    рис.)
    и выделяющуюся на нём мощность, если известны $r_1, r_2, \ele_1, \ele_2, R_1, R_2$.

    \begin{tikzpicture}[circuit ee IEC, thick]
        \draw  (0, 0) -- ++(up:2)
                to [
                    battery={ very near start, rotate=-180, info={$\ele_1, r_1 $}},
                    resistor={ midway, info=$R_1$},
                    battery={ very near end, rotate=-180, info={$\ele_2, r_2 $}}
                ] ++(right:5)
                -- ++(down:2)
                to [resistor={info=$R_2$}] ++(left:5);
    \end{tikzpicture}
}
\answer{%
    Нетривиальных узлов нет, поэтому все законы Кирхгофа для узлов будут иметь вид
    $\eli-\eli=0$ и ничем нам не помогут.
    Впрочем, если бы мы обозначили токи на разных участках контура $\eli_1, \eli_2, \eli_3, \ldots$,
    то именно эти законы бы помогли понять, что все эти токи равны: $\eli_1 - \eli_2 = 0$ и т.д.
    Так что запишем закон Кирхгофа для единственного замкнутого контура:

    \begin{tikzpicture}[circuit ee IEC, thick]
        \draw  (0, 0) -- ++(up:2)
                to [
                    battery={ very near start, rotate=-180, info={$\ele_1, r_1 $}},
                    resistor={ midway, info=$R_1$},
                    battery={ very near end, rotate=-180, info={$\ele_2, r_2 $}}
                ] ++(right:5)
                -- ++(down:2)
                to [resistor={info=$R_2$}, current direction={near end, info=$\eli$}] ++(left:5);
        \draw [-{Latex}] (2, 1.4) arc [start angle = 135, end angle = -160, radius = 0.6];
    \end{tikzpicture}

    \begin{align*}
        & \ele_1 +  \ele_2 = \eli R_1 + \eli r_2 + \eli R_2 + \eli r_1, \\
        & \ele_1 +  \ele_2 = \eli (R_1 + r_2 + R_2 + r_1), \\
        &\eli = \frac{ \ele_1 +  \ele_2 }{ R_1 + r_2 + R_2 + r_1 }, \\
        &U_1 = \eli R_1 = \frac{ \ele_1 +  \ele_2 }{ R_1 + r_2 + R_2 + r_1 } \cdot R_1, \\
        &P_1 = \eli^2 R_1 = \frac{\sqr{ \ele_1 +  \ele_2 } R_1}{ \sqr{ R_1 + r_2 + R_2 + r_1 }}.
    \end{align*}

    Отметим, что ответ для тока $\eli$ меняет знак, если отметить его на рисунке в другую сторону.
    Поэтому критично важно указывать на рисунке направление тока, иначе невозможно утверждать, что ответ верный.
    А вот выбор направления контура — не повлияет на ответ, но для проверки корректности записи законо Кирхгофа,
    там тоже необходимо направление.
}

\variantsplitter

\addpersonalvariant{Леонид Никитин}

\tasknumber{1}%
\task{%
    Два резистора сопротивлениями $R_1=3R$ и $R_2=4R$ подключены параллельно к источнику напряжения.
    Определите, в каком резисторе выделяется большая тепловая мощность и во сколько раз?
}
\answer{%
    Подключены параллельно, поэтому  $U_1 = U_2 = U \implies \frac{P_2}{P_1} = \frac{\frac{U_2^2}{R_2}}{\frac{U_1^2}{R_1}} = \frac{U^2R_1}{U^2R_2} = \frac{R_1}{R_2} = \frac34$.
}
\solutionspace{120pt}

\tasknumber{2}%
\task{%
    Если батарею замкнуть на резистор сопротивлением $R_1$, то в цепи потечёт ток $\eli_1$,
    а если на другой $R_2$ — то $\eli_2$.
    Определите:
    \begin{itemize}
        \item ЭДС батареи,
        \item внутреннее сопротивление батареи,
        \item ток короткого замыкания.
    \end{itemize}
}
\answer{%
    Запишем закон Ома для полной цепи 2 раза для обоих способов подключения (с $R_1$ и с $R_2$),
    а короткое замыкание рассмотрим позже.
    Отметим, что для такой простой схемы он совпадает
    с законом Кирхгофа.
    Получим систему из 2 уравнений и 2 неизвестных, решим в удобном порядке,
    ибо нам всё равно понадобятся обе.

    \begin{align*}
        &\begin{cases}
            \ele = \eli_1(R_1 + r), \\
            \ele = \eli_2(R_2 + r); \\
        \end{cases} \\
        &\eli_1(R_1 + r) = \eli_2(R_2 + r), \\
        &\eli_1 R_1 + \eli_1r = \eli_2 R_2 + \eli_2r, \\
        &\eli_1 R_1 - \eli_2 R_2 = - \eli_1r  + \eli_2r = (\eli_2 - \eli_1)r, \\
        r &= \frac{\eli_1 R_1 - \eli_2 R_2}{\eli_2 - \eli_1}
            \equiv \frac{\eli_2 R_2 - \eli_1 R_1}{\eli_1 - \eli_2}, \\
        \ele &= \eli_1(R_1 + r)
            = \eli_1\cbr{R_1 + \frac{\eli_1 R_1 - \eli_2 R_2}{\eli_2 - \eli_1}}
            = \eli_1 \cdot \frac{R_1\eli_2 - R_1\eli_1 + \eli_1 R_1 - \eli_2 R_2}{\eli_2 - \eli_1} \\
            &= \eli_1 \cdot \frac{R_1\eli_2 - \eli_2 R_2}{\eli_2 - \eli_1}
            = \frac{\eli_1 \eli_2 (R_1 - R_2)}{\eli_2 - \eli_1}
            \equiv \frac{\eli_1 \eli_2 (R_2 - R_1)}{\eli_1 - \eli_2}.
    \end{align*}

    Короткое замыкание происходит в ситуации, когда внешнее сопротивление равно 0
    (при этом цепь замкнута, хотя нагрузки и нет вовсе):
    $$
        \eli_\text{к.
        з.} = \frac \ele {0 + r} = \frac \ele r
            = \frac{\cfrac{\eli_1 \eli_2 (R_1 - R_2)}{\eli_2 - \eli_1}}{\cfrac{\eli_1 R_1 - \eli_2 R_2}{\eli_2 - \eli_1}}
            = \frac{\eli_1 \eli_2 (R_1 - R_2)}{\eli_1 R_1 - \eli_2 R_2}
            \equiv \frac{\eli_1 \eli_2 (R_2 - R_1)}{\eli_2 R_2 - \eli_1 R_1}.
    $$

    Важные пункты:
    \begin{itemize}
        \item В ответах есть только те величины, которые есть в условии
            (и ещё физические постоянные могут встретиться, но нам не понадобилось).
        \item Мы упростили выражения, который пошли в ответы (благо у нас даже получилось:
            приведение к общему знаменателю укоротило ответ).
            Надо доделывать.
        \item Всё ответы симметричны относительно замены резисторов 1 и 2 (ведь при этом изменятся и токи).
    \end{itemize}
}
\solutionspace{120pt}

\tasknumber{3}%
\task{%
    Определите ток, протекающий через резистор $R_2$, разность потенциалов на нём (см.
    рис.)
    и выделяющуюся на нём мощность, если известны $r_1, r_2, \ele_1, \ele_2, R_1, R_2$.

    \begin{tikzpicture}[circuit ee IEC, thick]
        \draw  (0, 0) -- ++(up:2)
                to [
                    battery={ very near start, rotate=0, info={$\ele_1, r_1 $}},
                    resistor={ midway, info=$R_1$},
                    battery={ very near end, rotate=-180, info={$\ele_2, r_2 $}}
                ] ++(right:5)
                -- ++(down:2)
                to [resistor={info=$R_2$}] ++(left:5);
    \end{tikzpicture}
}
\answer{%
    Нетривиальных узлов нет, поэтому все законы Кирхгофа для узлов будут иметь вид
    $\eli-\eli=0$ и ничем нам не помогут.
    Впрочем, если бы мы обозначили токи на разных участках контура $\eli_1, \eli_2, \eli_3, \ldots$,
    то именно эти законы бы помогли понять, что все эти токи равны: $\eli_1 - \eli_2 = 0$ и т.д.
    Так что запишем закон Кирхгофа для единственного замкнутого контура:

    \begin{tikzpicture}[circuit ee IEC, thick]
        \draw  (0, 0) -- ++(up:2)
                to [
                    battery={ very near start, rotate=0, info={$\ele_1, r_1 $}},
                    resistor={ midway, info=$R_1$},
                    battery={ very near end, rotate=-180, info={$\ele_2, r_2 $}}
                ] ++(right:5)
                -- ++(down:2)
                to [resistor={info=$R_2$}, current direction={near end, info=$\eli$}] ++(left:5);
        \draw [-{Latex}] (2, 1.4) arc [start angle = 135, end angle = -160, radius = 0.6];
    \end{tikzpicture}

    \begin{align*}
        &- \ele_1 +  \ele_2 = \eli R_1 + \eli r_2 + \eli R_2 + \eli r_1, \\
        &- \ele_1 +  \ele_2 = \eli (R_1 + r_2 + R_2 + r_1), \\
        &\eli = \frac{- \ele_1 +  \ele_2 }{ R_1 + r_2 + R_2 + r_1 }, \\
        &U_2 = \eli R_2 = \frac{- \ele_1 +  \ele_2 }{ R_1 + r_2 + R_2 + r_1 } \cdot R_2, \\
        &P_2 = \eli^2 R_2 = \frac{\sqr{- \ele_1 +  \ele_2 } R_2}{ \sqr{ R_1 + r_2 + R_2 + r_1 }}.
    \end{align*}

    Отметим, что ответ для тока $\eli$ меняет знак, если отметить его на рисунке в другую сторону.
    Поэтому критично важно указывать на рисунке направление тока, иначе невозможно утверждать, что ответ верный.
    А вот выбор направления контура — не повлияет на ответ, но для проверки корректности записи законо Кирхгофа,
    там тоже необходимо направление.
}

\variantsplitter

\addpersonalvariant{Тимофей Полетаев}

\tasknumber{1}%
\task{%
    Два резистора сопротивлениями $R_1=3R$ и $R_2=8R$ подключены последовательно к источнику напряжения.
    Определите, в каком резисторе выделяется большая тепловая мощность и во сколько раз?
}
\answer{%
    Подключены последовательно, поэтому  $\eli_1 = \eli_2 = \eli \implies \frac{P_2}{P_1} = \frac{\eli_2^2 R_2}{\eli_1^2 R_1} = \frac{\eli_2^2R_2}{\eli_1R_1} = \frac{R_2}{R_1} = \frac83$.
}
\solutionspace{120pt}

\tasknumber{2}%
\task{%
    Если батарею замкнуть на резистор сопротивлением $R_1$, то в цепи потечёт ток $\eli_1$,
    а если на другой $R_2$ — то $\eli_2$.
    Определите:
    \begin{itemize}
        \item ЭДС батареи,
        \item внутреннее сопротивление батареи,
        \item ток короткого замыкания.
    \end{itemize}
}
\answer{%
    Запишем закон Ома для полной цепи 2 раза для обоих способов подключения (с $R_1$ и с $R_2$),
    а короткое замыкание рассмотрим позже.
    Отметим, что для такой простой схемы он совпадает
    с законом Кирхгофа.
    Получим систему из 2 уравнений и 2 неизвестных, решим в удобном порядке,
    ибо нам всё равно понадобятся обе.

    \begin{align*}
        &\begin{cases}
            \ele = \eli_1(R_1 + r), \\
            \ele = \eli_2(R_2 + r); \\
        \end{cases} \\
        &\eli_1(R_1 + r) = \eli_2(R_2 + r), \\
        &\eli_1 R_1 + \eli_1r = \eli_2 R_2 + \eli_2r, \\
        &\eli_1 R_1 - \eli_2 R_2 = - \eli_1r  + \eli_2r = (\eli_2 - \eli_1)r, \\
        r &= \frac{\eli_1 R_1 - \eli_2 R_2}{\eli_2 - \eli_1}
            \equiv \frac{\eli_2 R_2 - \eli_1 R_1}{\eli_1 - \eli_2}, \\
        \ele &= \eli_1(R_1 + r)
            = \eli_1\cbr{R_1 + \frac{\eli_1 R_1 - \eli_2 R_2}{\eli_2 - \eli_1}}
            = \eli_1 \cdot \frac{R_1\eli_2 - R_1\eli_1 + \eli_1 R_1 - \eli_2 R_2}{\eli_2 - \eli_1} \\
            &= \eli_1 \cdot \frac{R_1\eli_2 - \eli_2 R_2}{\eli_2 - \eli_1}
            = \frac{\eli_1 \eli_2 (R_1 - R_2)}{\eli_2 - \eli_1}
            \equiv \frac{\eli_1 \eli_2 (R_2 - R_1)}{\eli_1 - \eli_2}.
    \end{align*}

    Короткое замыкание происходит в ситуации, когда внешнее сопротивление равно 0
    (при этом цепь замкнута, хотя нагрузки и нет вовсе):
    $$
        \eli_\text{к.
        з.} = \frac \ele {0 + r} = \frac \ele r
            = \frac{\cfrac{\eli_1 \eli_2 (R_1 - R_2)}{\eli_2 - \eli_1}}{\cfrac{\eli_1 R_1 - \eli_2 R_2}{\eli_2 - \eli_1}}
            = \frac{\eli_1 \eli_2 (R_1 - R_2)}{\eli_1 R_1 - \eli_2 R_2}
            \equiv \frac{\eli_1 \eli_2 (R_2 - R_1)}{\eli_2 R_2 - \eli_1 R_1}.
    $$

    Важные пункты:
    \begin{itemize}
        \item В ответах есть только те величины, которые есть в условии
            (и ещё физические постоянные могут встретиться, но нам не понадобилось).
        \item Мы упростили выражения, который пошли в ответы (благо у нас даже получилось:
            приведение к общему знаменателю укоротило ответ).
            Надо доделывать.
        \item Всё ответы симметричны относительно замены резисторов 1 и 2 (ведь при этом изменятся и токи).
    \end{itemize}
}
\solutionspace{120pt}

\tasknumber{3}%
\task{%
    Определите ток, протекающий через резистор $R_2$, разность потенциалов на нём (см.
    рис.)
    и выделяющуюся на нём мощность, если известны $r_1, r_2, \ele_1, \ele_2, R_1, R_2$.

    \begin{tikzpicture}[circuit ee IEC, thick]
        \draw  (0, 0) -- ++(up:2)
                to [
                    battery={ very near start, rotate=-180, info={$\ele_1, r_1 $}},
                    resistor={ midway, info=$R_1$},
                    battery={ very near end, rotate=0, info={$\ele_2, r_2 $}}
                ] ++(right:5)
                -- ++(down:2)
                to [resistor={info=$R_2$}] ++(left:5);
    \end{tikzpicture}
}
\answer{%
    Нетривиальных узлов нет, поэтому все законы Кирхгофа для узлов будут иметь вид
    $\eli-\eli=0$ и ничем нам не помогут.
    Впрочем, если бы мы обозначили токи на разных участках контура $\eli_1, \eli_2, \eli_3, \ldots$,
    то именно эти законы бы помогли понять, что все эти токи равны: $\eli_1 - \eli_2 = 0$ и т.д.
    Так что запишем закон Кирхгофа для единственного замкнутого контура:

    \begin{tikzpicture}[circuit ee IEC, thick]
        \draw  (0, 0) -- ++(up:2)
                to [
                    battery={ very near start, rotate=-180, info={$\ele_1, r_1 $}},
                    resistor={ midway, info=$R_1$},
                    battery={ very near end, rotate=0, info={$\ele_2, r_2 $}}
                ] ++(right:5)
                -- ++(down:2)
                to [resistor={info=$R_2$}, current direction={near end, info=$\eli$}] ++(left:5);
        \draw [-{Latex}] (2, 1.4) arc [start angle = 135, end angle = -160, radius = 0.6];
    \end{tikzpicture}

    \begin{align*}
        & \ele_1 - \ele_2 = \eli R_1 + \eli r_2 + \eli R_2 + \eli r_1, \\
        & \ele_1 - \ele_2 = \eli (R_1 + r_2 + R_2 + r_1), \\
        &\eli = \frac{ \ele_1 - \ele_2 }{ R_1 + r_2 + R_2 + r_1 }, \\
        &U_2 = \eli R_2 = \frac{ \ele_1 - \ele_2 }{ R_1 + r_2 + R_2 + r_1 } \cdot R_2, \\
        &P_2 = \eli^2 R_2 = \frac{\sqr{ \ele_1 - \ele_2 } R_2}{ \sqr{ R_1 + r_2 + R_2 + r_1 }}.
    \end{align*}

    Отметим, что ответ для тока $\eli$ меняет знак, если отметить его на рисунке в другую сторону.
    Поэтому критично важно указывать на рисунке направление тока, иначе невозможно утверждать, что ответ верный.
    А вот выбор направления контура — не повлияет на ответ, но для проверки корректности записи законо Кирхгофа,
    там тоже необходимо направление.
}

\variantsplitter

\addpersonalvariant{Андрей Рожков}

\tasknumber{1}%
\task{%
    Два резистора сопротивлениями $R_1=5R$ и $R_2=6R$ подключены параллельно к источнику напряжения.
    Определите, в каком резисторе выделяется большая тепловая мощность и во сколько раз?
}
\answer{%
    Подключены параллельно, поэтому  $U_1 = U_2 = U \implies \frac{P_2}{P_1} = \frac{\frac{U_2^2}{R_2}}{\frac{U_1^2}{R_1}} = \frac{U^2R_1}{U^2R_2} = \frac{R_1}{R_2} = \frac56$.
}
\solutionspace{120pt}

\tasknumber{2}%
\task{%
    Если батарею замкнуть на резистор сопротивлением $R_1$, то в цепи потечёт ток $\eli_1$,
    а если на другой $R_2$ — то $\eli_2$.
    Определите:
    \begin{itemize}
        \item ЭДС батареи,
        \item внутреннее сопротивление батареи,
        \item ток короткого замыкания.
    \end{itemize}
}
\answer{%
    Запишем закон Ома для полной цепи 2 раза для обоих способов подключения (с $R_1$ и с $R_2$),
    а короткое замыкание рассмотрим позже.
    Отметим, что для такой простой схемы он совпадает
    с законом Кирхгофа.
    Получим систему из 2 уравнений и 2 неизвестных, решим в удобном порядке,
    ибо нам всё равно понадобятся обе.

    \begin{align*}
        &\begin{cases}
            \ele = \eli_1(R_1 + r), \\
            \ele = \eli_2(R_2 + r); \\
        \end{cases} \\
        &\eli_1(R_1 + r) = \eli_2(R_2 + r), \\
        &\eli_1 R_1 + \eli_1r = \eli_2 R_2 + \eli_2r, \\
        &\eli_1 R_1 - \eli_2 R_2 = - \eli_1r  + \eli_2r = (\eli_2 - \eli_1)r, \\
        r &= \frac{\eli_1 R_1 - \eli_2 R_2}{\eli_2 - \eli_1}
            \equiv \frac{\eli_2 R_2 - \eli_1 R_1}{\eli_1 - \eli_2}, \\
        \ele &= \eli_1(R_1 + r)
            = \eli_1\cbr{R_1 + \frac{\eli_1 R_1 - \eli_2 R_2}{\eli_2 - \eli_1}}
            = \eli_1 \cdot \frac{R_1\eli_2 - R_1\eli_1 + \eli_1 R_1 - \eli_2 R_2}{\eli_2 - \eli_1} \\
            &= \eli_1 \cdot \frac{R_1\eli_2 - \eli_2 R_2}{\eli_2 - \eli_1}
            = \frac{\eli_1 \eli_2 (R_1 - R_2)}{\eli_2 - \eli_1}
            \equiv \frac{\eli_1 \eli_2 (R_2 - R_1)}{\eli_1 - \eli_2}.
    \end{align*}

    Короткое замыкание происходит в ситуации, когда внешнее сопротивление равно 0
    (при этом цепь замкнута, хотя нагрузки и нет вовсе):
    $$
        \eli_\text{к.
        з.} = \frac \ele {0 + r} = \frac \ele r
            = \frac{\cfrac{\eli_1 \eli_2 (R_1 - R_2)}{\eli_2 - \eli_1}}{\cfrac{\eli_1 R_1 - \eli_2 R_2}{\eli_2 - \eli_1}}
            = \frac{\eli_1 \eli_2 (R_1 - R_2)}{\eli_1 R_1 - \eli_2 R_2}
            \equiv \frac{\eli_1 \eli_2 (R_2 - R_1)}{\eli_2 R_2 - \eli_1 R_1}.
    $$

    Важные пункты:
    \begin{itemize}
        \item В ответах есть только те величины, которые есть в условии
            (и ещё физические постоянные могут встретиться, но нам не понадобилось).
        \item Мы упростили выражения, который пошли в ответы (благо у нас даже получилось:
            приведение к общему знаменателю укоротило ответ).
            Надо доделывать.
        \item Всё ответы симметричны относительно замены резисторов 1 и 2 (ведь при этом изменятся и токи).
    \end{itemize}
}
\solutionspace{120pt}

\tasknumber{3}%
\task{%
    Определите ток, протекающий через резистор $R_2$, разность потенциалов на нём (см.
    рис.)
    и выделяющуюся на нём мощность, если известны $r_1, r_2, \ele_1, \ele_2, R_1, R_2$.

    \begin{tikzpicture}[circuit ee IEC, thick]
        \draw  (0, 0) -- ++(up:2)
                to [
                    battery={ very near start, rotate=0, info={$\ele_1, r_1 $}},
                    resistor={ midway, info=$R_1$},
                    battery={ very near end, rotate=-180, info={$\ele_2, r_2 $}}
                ] ++(right:5)
                -- ++(down:2)
                to [resistor={info=$R_2$}] ++(left:5);
    \end{tikzpicture}
}
\answer{%
    Нетривиальных узлов нет, поэтому все законы Кирхгофа для узлов будут иметь вид
    $\eli-\eli=0$ и ничем нам не помогут.
    Впрочем, если бы мы обозначили токи на разных участках контура $\eli_1, \eli_2, \eli_3, \ldots$,
    то именно эти законы бы помогли понять, что все эти токи равны: $\eli_1 - \eli_2 = 0$ и т.д.
    Так что запишем закон Кирхгофа для единственного замкнутого контура:

    \begin{tikzpicture}[circuit ee IEC, thick]
        \draw  (0, 0) -- ++(up:2)
                to [
                    battery={ very near start, rotate=0, info={$\ele_1, r_1 $}},
                    resistor={ midway, info=$R_1$},
                    battery={ very near end, rotate=-180, info={$\ele_2, r_2 $}}
                ] ++(right:5)
                -- ++(down:2)
                to [resistor={info=$R_2$}, current direction={near end, info=$\eli$}] ++(left:5);
        \draw [-{Latex}] (2, 1.4) arc [start angle = 135, end angle = -160, radius = 0.6];
    \end{tikzpicture}

    \begin{align*}
        &- \ele_1 +  \ele_2 = \eli R_1 + \eli r_2 + \eli R_2 + \eli r_1, \\
        &- \ele_1 +  \ele_2 = \eli (R_1 + r_2 + R_2 + r_1), \\
        &\eli = \frac{- \ele_1 +  \ele_2 }{ R_1 + r_2 + R_2 + r_1 }, \\
        &U_2 = \eli R_2 = \frac{- \ele_1 +  \ele_2 }{ R_1 + r_2 + R_2 + r_1 } \cdot R_2, \\
        &P_2 = \eli^2 R_2 = \frac{\sqr{- \ele_1 +  \ele_2 } R_2}{ \sqr{ R_1 + r_2 + R_2 + r_1 }}.
    \end{align*}

    Отметим, что ответ для тока $\eli$ меняет знак, если отметить его на рисунке в другую сторону.
    Поэтому критично важно указывать на рисунке направление тока, иначе невозможно утверждать, что ответ верный.
    А вот выбор направления контура — не повлияет на ответ, но для проверки корректности записи законо Кирхгофа,
    там тоже необходимо направление.
}

\variantsplitter

\addpersonalvariant{Рената Таржиманова}

\tasknumber{1}%
\task{%
    Два резистора сопротивлениями $R_1=5R$ и $R_2=6R$ подключены параллельно к источнику напряжения.
    Определите, в каком резисторе выделяется большая тепловая мощность и во сколько раз?
}
\answer{%
    Подключены параллельно, поэтому  $U_1 = U_2 = U \implies \frac{P_2}{P_1} = \frac{\frac{U_2^2}{R_2}}{\frac{U_1^2}{R_1}} = \frac{U^2R_1}{U^2R_2} = \frac{R_1}{R_2} = \frac56$.
}
\solutionspace{120pt}

\tasknumber{2}%
\task{%
    Если батарею замкнуть на резистор сопротивлением $R_1$, то в цепи потечёт ток $\eli_1$,
    а если на другой $R_2$ — то $\eli_2$.
    Определите:
    \begin{itemize}
        \item ЭДС батареи,
        \item внутреннее сопротивление батареи,
        \item ток короткого замыкания.
    \end{itemize}
}
\answer{%
    Запишем закон Ома для полной цепи 2 раза для обоих способов подключения (с $R_1$ и с $R_2$),
    а короткое замыкание рассмотрим позже.
    Отметим, что для такой простой схемы он совпадает
    с законом Кирхгофа.
    Получим систему из 2 уравнений и 2 неизвестных, решим в удобном порядке,
    ибо нам всё равно понадобятся обе.

    \begin{align*}
        &\begin{cases}
            \ele = \eli_1(R_1 + r), \\
            \ele = \eli_2(R_2 + r); \\
        \end{cases} \\
        &\eli_1(R_1 + r) = \eli_2(R_2 + r), \\
        &\eli_1 R_1 + \eli_1r = \eli_2 R_2 + \eli_2r, \\
        &\eli_1 R_1 - \eli_2 R_2 = - \eli_1r  + \eli_2r = (\eli_2 - \eli_1)r, \\
        r &= \frac{\eli_1 R_1 - \eli_2 R_2}{\eli_2 - \eli_1}
            \equiv \frac{\eli_2 R_2 - \eli_1 R_1}{\eli_1 - \eli_2}, \\
        \ele &= \eli_1(R_1 + r)
            = \eli_1\cbr{R_1 + \frac{\eli_1 R_1 - \eli_2 R_2}{\eli_2 - \eli_1}}
            = \eli_1 \cdot \frac{R_1\eli_2 - R_1\eli_1 + \eli_1 R_1 - \eli_2 R_2}{\eli_2 - \eli_1} \\
            &= \eli_1 \cdot \frac{R_1\eli_2 - \eli_2 R_2}{\eli_2 - \eli_1}
            = \frac{\eli_1 \eli_2 (R_1 - R_2)}{\eli_2 - \eli_1}
            \equiv \frac{\eli_1 \eli_2 (R_2 - R_1)}{\eli_1 - \eli_2}.
    \end{align*}

    Короткое замыкание происходит в ситуации, когда внешнее сопротивление равно 0
    (при этом цепь замкнута, хотя нагрузки и нет вовсе):
    $$
        \eli_\text{к.
        з.} = \frac \ele {0 + r} = \frac \ele r
            = \frac{\cfrac{\eli_1 \eli_2 (R_1 - R_2)}{\eli_2 - \eli_1}}{\cfrac{\eli_1 R_1 - \eli_2 R_2}{\eli_2 - \eli_1}}
            = \frac{\eli_1 \eli_2 (R_1 - R_2)}{\eli_1 R_1 - \eli_2 R_2}
            \equiv \frac{\eli_1 \eli_2 (R_2 - R_1)}{\eli_2 R_2 - \eli_1 R_1}.
    $$

    Важные пункты:
    \begin{itemize}
        \item В ответах есть только те величины, которые есть в условии
            (и ещё физические постоянные могут встретиться, но нам не понадобилось).
        \item Мы упростили выражения, который пошли в ответы (благо у нас даже получилось:
            приведение к общему знаменателю укоротило ответ).
            Надо доделывать.
        \item Всё ответы симметричны относительно замены резисторов 1 и 2 (ведь при этом изменятся и токи).
    \end{itemize}
}
\solutionspace{120pt}

\tasknumber{3}%
\task{%
    Определите ток, протекающий через резистор $R_2$, разность потенциалов на нём (см.
    рис.)
    и выделяющуюся на нём мощность, если известны $r_1, r_2, \ele_1, \ele_2, R_1, R_2$.

    \begin{tikzpicture}[circuit ee IEC, thick]
        \draw  (0, 0) -- ++(up:2)
                to [
                    battery={ very near start, rotate=0, info={$\ele_1, r_1 $}},
                    resistor={ midway, info=$R_1$},
                    battery={ very near end, rotate=0, info={$\ele_2, r_2 $}}
                ] ++(right:5)
                -- ++(down:2)
                to [resistor={info=$R_2$}] ++(left:5);
    \end{tikzpicture}
}
\answer{%
    Нетривиальных узлов нет, поэтому все законы Кирхгофа для узлов будут иметь вид
    $\eli-\eli=0$ и ничем нам не помогут.
    Впрочем, если бы мы обозначили токи на разных участках контура $\eli_1, \eli_2, \eli_3, \ldots$,
    то именно эти законы бы помогли понять, что все эти токи равны: $\eli_1 - \eli_2 = 0$ и т.д.
    Так что запишем закон Кирхгофа для единственного замкнутого контура:

    \begin{tikzpicture}[circuit ee IEC, thick]
        \draw  (0, 0) -- ++(up:2)
                to [
                    battery={ very near start, rotate=0, info={$\ele_1, r_1 $}},
                    resistor={ midway, info=$R_1$},
                    battery={ very near end, rotate=0, info={$\ele_2, r_2 $}}
                ] ++(right:5)
                -- ++(down:2)
                to [resistor={info=$R_2$}, current direction={near end, info=$\eli$}] ++(left:5);
        \draw [-{Latex}] (2, 1.4) arc [start angle = 135, end angle = -160, radius = 0.6];
    \end{tikzpicture}

    \begin{align*}
        &- \ele_1 - \ele_2 = \eli R_1 + \eli r_2 + \eli R_2 + \eli r_1, \\
        &- \ele_1 - \ele_2 = \eli (R_1 + r_2 + R_2 + r_1), \\
        &\eli = \frac{- \ele_1 - \ele_2 }{ R_1 + r_2 + R_2 + r_1 }, \\
        &U_2 = \eli R_2 = \frac{- \ele_1 - \ele_2 }{ R_1 + r_2 + R_2 + r_1 } \cdot R_2, \\
        &P_2 = \eli^2 R_2 = \frac{\sqr{- \ele_1 - \ele_2 } R_2}{ \sqr{ R_1 + r_2 + R_2 + r_1 }}.
    \end{align*}

    Отметим, что ответ для тока $\eli$ меняет знак, если отметить его на рисунке в другую сторону.
    Поэтому критично важно указывать на рисунке направление тока, иначе невозможно утверждать, что ответ верный.
    А вот выбор направления контура — не повлияет на ответ, но для проверки корректности записи законо Кирхгофа,
    там тоже необходимо направление.
}

\variantsplitter

\addpersonalvariant{Андрей Щербаков}

\tasknumber{1}%
\task{%
    Два резистора сопротивлениями $R_1=7R$ и $R_2=4R$ подключены последовательно к источнику напряжения.
    Определите, в каком резисторе выделяется большая тепловая мощность и во сколько раз?
}
\answer{%
    Подключены последовательно, поэтому  $\eli_1 = \eli_2 = \eli \implies \frac{P_2}{P_1} = \frac{\eli_2^2 R_2}{\eli_1^2 R_1} = \frac{\eli_2^2R_2}{\eli_1R_1} = \frac{R_2}{R_1} = \frac47$.
}
\solutionspace{120pt}

\tasknumber{2}%
\task{%
    Если батарею замкнуть на резистор сопротивлением $R_1$, то в цепи потечёт ток $\eli_1$,
    а если на другой $R_2$ — то $\eli_2$.
    Определите:
    \begin{itemize}
        \item ЭДС батареи,
        \item внутреннее сопротивление батареи,
        \item ток короткого замыкания.
    \end{itemize}
}
\answer{%
    Запишем закон Ома для полной цепи 2 раза для обоих способов подключения (с $R_1$ и с $R_2$),
    а короткое замыкание рассмотрим позже.
    Отметим, что для такой простой схемы он совпадает
    с законом Кирхгофа.
    Получим систему из 2 уравнений и 2 неизвестных, решим в удобном порядке,
    ибо нам всё равно понадобятся обе.

    \begin{align*}
        &\begin{cases}
            \ele = \eli_1(R_1 + r), \\
            \ele = \eli_2(R_2 + r); \\
        \end{cases} \\
        &\eli_1(R_1 + r) = \eli_2(R_2 + r), \\
        &\eli_1 R_1 + \eli_1r = \eli_2 R_2 + \eli_2r, \\
        &\eli_1 R_1 - \eli_2 R_2 = - \eli_1r  + \eli_2r = (\eli_2 - \eli_1)r, \\
        r &= \frac{\eli_1 R_1 - \eli_2 R_2}{\eli_2 - \eli_1}
            \equiv \frac{\eli_2 R_2 - \eli_1 R_1}{\eli_1 - \eli_2}, \\
        \ele &= \eli_1(R_1 + r)
            = \eli_1\cbr{R_1 + \frac{\eli_1 R_1 - \eli_2 R_2}{\eli_2 - \eli_1}}
            = \eli_1 \cdot \frac{R_1\eli_2 - R_1\eli_1 + \eli_1 R_1 - \eli_2 R_2}{\eli_2 - \eli_1} \\
            &= \eli_1 \cdot \frac{R_1\eli_2 - \eli_2 R_2}{\eli_2 - \eli_1}
            = \frac{\eli_1 \eli_2 (R_1 - R_2)}{\eli_2 - \eli_1}
            \equiv \frac{\eli_1 \eli_2 (R_2 - R_1)}{\eli_1 - \eli_2}.
    \end{align*}

    Короткое замыкание происходит в ситуации, когда внешнее сопротивление равно 0
    (при этом цепь замкнута, хотя нагрузки и нет вовсе):
    $$
        \eli_\text{к.
        з.} = \frac \ele {0 + r} = \frac \ele r
            = \frac{\cfrac{\eli_1 \eli_2 (R_1 - R_2)}{\eli_2 - \eli_1}}{\cfrac{\eli_1 R_1 - \eli_2 R_2}{\eli_2 - \eli_1}}
            = \frac{\eli_1 \eli_2 (R_1 - R_2)}{\eli_1 R_1 - \eli_2 R_2}
            \equiv \frac{\eli_1 \eli_2 (R_2 - R_1)}{\eli_2 R_2 - \eli_1 R_1}.
    $$

    Важные пункты:
    \begin{itemize}
        \item В ответах есть только те величины, которые есть в условии
            (и ещё физические постоянные могут встретиться, но нам не понадобилось).
        \item Мы упростили выражения, который пошли в ответы (благо у нас даже получилось:
            приведение к общему знаменателю укоротило ответ).
            Надо доделывать.
        \item Всё ответы симметричны относительно замены резисторов 1 и 2 (ведь при этом изменятся и токи).
    \end{itemize}
}
\solutionspace{120pt}

\tasknumber{3}%
\task{%
    Определите ток, протекающий через резистор $R_1$, разность потенциалов на нём (см.
    рис.)
    и выделяющуюся на нём мощность, если известны $r_1, r_2, \ele_1, \ele_2, R_1, R_2$.

    \begin{tikzpicture}[circuit ee IEC, thick]
        \draw  (0, 0) -- ++(up:2)
                to [
                    battery={ very near start, rotate=-180, info={$\ele_1, r_1 $}},
                    resistor={ midway, info=$R_1$},
                    battery={ very near end, rotate=-180, info={$\ele_2, r_2 $}}
                ] ++(right:5)
                -- ++(down:2)
                to [resistor={info=$R_2$}] ++(left:5);
    \end{tikzpicture}
}
\answer{%
    Нетривиальных узлов нет, поэтому все законы Кирхгофа для узлов будут иметь вид
    $\eli-\eli=0$ и ничем нам не помогут.
    Впрочем, если бы мы обозначили токи на разных участках контура $\eli_1, \eli_2, \eli_3, \ldots$,
    то именно эти законы бы помогли понять, что все эти токи равны: $\eli_1 - \eli_2 = 0$ и т.д.
    Так что запишем закон Кирхгофа для единственного замкнутого контура:

    \begin{tikzpicture}[circuit ee IEC, thick]
        \draw  (0, 0) -- ++(up:2)
                to [
                    battery={ very near start, rotate=-180, info={$\ele_1, r_1 $}},
                    resistor={ midway, info=$R_1$},
                    battery={ very near end, rotate=-180, info={$\ele_2, r_2 $}}
                ] ++(right:5)
                -- ++(down:2)
                to [resistor={info=$R_2$}, current direction={near end, info=$\eli$}] ++(left:5);
        \draw [-{Latex}] (2, 1.4) arc [start angle = 135, end angle = -160, radius = 0.6];
    \end{tikzpicture}

    \begin{align*}
        & \ele_1 +  \ele_2 = \eli R_1 + \eli r_2 + \eli R_2 + \eli r_1, \\
        & \ele_1 +  \ele_2 = \eli (R_1 + r_2 + R_2 + r_1), \\
        &\eli = \frac{ \ele_1 +  \ele_2 }{ R_1 + r_2 + R_2 + r_1 }, \\
        &U_1 = \eli R_1 = \frac{ \ele_1 +  \ele_2 }{ R_1 + r_2 + R_2 + r_1 } \cdot R_1, \\
        &P_1 = \eli^2 R_1 = \frac{\sqr{ \ele_1 +  \ele_2 } R_1}{ \sqr{ R_1 + r_2 + R_2 + r_1 }}.
    \end{align*}

    Отметим, что ответ для тока $\eli$ меняет знак, если отметить его на рисунке в другую сторону.
    Поэтому критично важно указывать на рисунке направление тока, иначе невозможно утверждать, что ответ верный.
    А вот выбор направления контура — не повлияет на ответ, но для проверки корректности записи законо Кирхгофа,
    там тоже необходимо направление.
}

\variantsplitter

\addpersonalvariant{Михаил Ярошевский}

\tasknumber{1}%
\task{%
    Два резистора сопротивлениями $R_1=7R$ и $R_2=8R$ подключены параллельно к источнику напряжения.
    Определите, в каком резисторе выделяется большая тепловая мощность и во сколько раз?
}
\answer{%
    Подключены параллельно, поэтому  $U_1 = U_2 = U \implies \frac{P_2}{P_1} = \frac{\frac{U_2^2}{R_2}}{\frac{U_1^2}{R_1}} = \frac{U^2R_1}{U^2R_2} = \frac{R_1}{R_2} = \frac78$.
}
\solutionspace{120pt}

\tasknumber{2}%
\task{%
    Если батарею замкнуть на резистор сопротивлением $R_1$, то в цепи потечёт ток $\eli_1$,
    а если на другой $R_2$ — то $\eli_2$.
    Определите:
    \begin{itemize}
        \item ЭДС батареи,
        \item внутреннее сопротивление батареи,
        \item ток короткого замыкания.
    \end{itemize}
}
\answer{%
    Запишем закон Ома для полной цепи 2 раза для обоих способов подключения (с $R_1$ и с $R_2$),
    а короткое замыкание рассмотрим позже.
    Отметим, что для такой простой схемы он совпадает
    с законом Кирхгофа.
    Получим систему из 2 уравнений и 2 неизвестных, решим в удобном порядке,
    ибо нам всё равно понадобятся обе.

    \begin{align*}
        &\begin{cases}
            \ele = \eli_1(R_1 + r), \\
            \ele = \eli_2(R_2 + r); \\
        \end{cases} \\
        &\eli_1(R_1 + r) = \eli_2(R_2 + r), \\
        &\eli_1 R_1 + \eli_1r = \eli_2 R_2 + \eli_2r, \\
        &\eli_1 R_1 - \eli_2 R_2 = - \eli_1r  + \eli_2r = (\eli_2 - \eli_1)r, \\
        r &= \frac{\eli_1 R_1 - \eli_2 R_2}{\eli_2 - \eli_1}
            \equiv \frac{\eli_2 R_2 - \eli_1 R_1}{\eli_1 - \eli_2}, \\
        \ele &= \eli_1(R_1 + r)
            = \eli_1\cbr{R_1 + \frac{\eli_1 R_1 - \eli_2 R_2}{\eli_2 - \eli_1}}
            = \eli_1 \cdot \frac{R_1\eli_2 - R_1\eli_1 + \eli_1 R_1 - \eli_2 R_2}{\eli_2 - \eli_1} \\
            &= \eli_1 \cdot \frac{R_1\eli_2 - \eli_2 R_2}{\eli_2 - \eli_1}
            = \frac{\eli_1 \eli_2 (R_1 - R_2)}{\eli_2 - \eli_1}
            \equiv \frac{\eli_1 \eli_2 (R_2 - R_1)}{\eli_1 - \eli_2}.
    \end{align*}

    Короткое замыкание происходит в ситуации, когда внешнее сопротивление равно 0
    (при этом цепь замкнута, хотя нагрузки и нет вовсе):
    $$
        \eli_\text{к.
        з.} = \frac \ele {0 + r} = \frac \ele r
            = \frac{\cfrac{\eli_1 \eli_2 (R_1 - R_2)}{\eli_2 - \eli_1}}{\cfrac{\eli_1 R_1 - \eli_2 R_2}{\eli_2 - \eli_1}}
            = \frac{\eli_1 \eli_2 (R_1 - R_2)}{\eli_1 R_1 - \eli_2 R_2}
            \equiv \frac{\eli_1 \eli_2 (R_2 - R_1)}{\eli_2 R_2 - \eli_1 R_1}.
    $$

    Важные пункты:
    \begin{itemize}
        \item В ответах есть только те величины, которые есть в условии
            (и ещё физические постоянные могут встретиться, но нам не понадобилось).
        \item Мы упростили выражения, который пошли в ответы (благо у нас даже получилось:
            приведение к общему знаменателю укоротило ответ).
            Надо доделывать.
        \item Всё ответы симметричны относительно замены резисторов 1 и 2 (ведь при этом изменятся и токи).
    \end{itemize}
}
\solutionspace{120pt}

\tasknumber{3}%
\task{%
    Определите ток, протекающий через резистор $R_2$, разность потенциалов на нём (см.
    рис.)
    и выделяющуюся на нём мощность, если известны $r_1, r_2, \ele_1, \ele_2, R_1, R_2$.

    \begin{tikzpicture}[circuit ee IEC, thick]
        \draw  (0, 0) -- ++(up:2)
                to [
                    battery={ very near start, rotate=-180, info={$\ele_1, r_1 $}},
                    resistor={ midway, info=$R_1$},
                    battery={ very near end, rotate=-180, info={$\ele_2, r_2 $}}
                ] ++(right:5)
                -- ++(down:2)
                to [resistor={info=$R_2$}] ++(left:5);
    \end{tikzpicture}
}
\answer{%
    Нетривиальных узлов нет, поэтому все законы Кирхгофа для узлов будут иметь вид
    $\eli-\eli=0$ и ничем нам не помогут.
    Впрочем, если бы мы обозначили токи на разных участках контура $\eli_1, \eli_2, \eli_3, \ldots$,
    то именно эти законы бы помогли понять, что все эти токи равны: $\eli_1 - \eli_2 = 0$ и т.д.
    Так что запишем закон Кирхгофа для единственного замкнутого контура:

    \begin{tikzpicture}[circuit ee IEC, thick]
        \draw  (0, 0) -- ++(up:2)
                to [
                    battery={ very near start, rotate=-180, info={$\ele_1, r_1 $}},
                    resistor={ midway, info=$R_1$},
                    battery={ very near end, rotate=-180, info={$\ele_2, r_2 $}}
                ] ++(right:5)
                -- ++(down:2)
                to [resistor={info=$R_2$}, current direction={near end, info=$\eli$}] ++(left:5);
        \draw [-{Latex}] (2, 1.4) arc [start angle = 135, end angle = -160, radius = 0.6];
    \end{tikzpicture}

    \begin{align*}
        & \ele_1 +  \ele_2 = \eli R_1 + \eli r_2 + \eli R_2 + \eli r_1, \\
        & \ele_1 +  \ele_2 = \eli (R_1 + r_2 + R_2 + r_1), \\
        &\eli = \frac{ \ele_1 +  \ele_2 }{ R_1 + r_2 + R_2 + r_1 }, \\
        &U_2 = \eli R_2 = \frac{ \ele_1 +  \ele_2 }{ R_1 + r_2 + R_2 + r_1 } \cdot R_2, \\
        &P_2 = \eli^2 R_2 = \frac{\sqr{ \ele_1 +  \ele_2 } R_2}{ \sqr{ R_1 + r_2 + R_2 + r_1 }}.
    \end{align*}

    Отметим, что ответ для тока $\eli$ меняет знак, если отметить его на рисунке в другую сторону.
    Поэтому критично важно указывать на рисунке направление тока, иначе невозможно утверждать, что ответ верный.
    А вот выбор направления контура — не повлияет на ответ, но для проверки корректности записи законо Кирхгофа,
    там тоже необходимо направление.
}

\variantsplitter

\addpersonalvariant{Алексей Алимпиев}

\tasknumber{1}%
\task{%
    Два резистора сопротивлениями $R_1=3R$ и $R_2=6R$ подключены последовательно к источнику напряжения.
    Определите, в каком резисторе выделяется большая тепловая мощность и во сколько раз?
}
\answer{%
    Подключены последовательно, поэтому  $\eli_1 = \eli_2 = \eli \implies \frac{P_2}{P_1} = \frac{\eli_2^2 R_2}{\eli_1^2 R_1} = \frac{\eli_2^2R_2}{\eli_1R_1} = \frac{R_2}{R_1} = 2$.
}
\solutionspace{120pt}

\tasknumber{2}%
\task{%
    Если батарею замкнуть на резистор сопротивлением $R_1$, то в цепи потечёт ток $\eli_1$,
    а если на другой $R_2$ — то $\eli_2$.
    Определите:
    \begin{itemize}
        \item ЭДС батареи,
        \item внутреннее сопротивление батареи,
        \item ток короткого замыкания.
    \end{itemize}
}
\answer{%
    Запишем закон Ома для полной цепи 2 раза для обоих способов подключения (с $R_1$ и с $R_2$),
    а короткое замыкание рассмотрим позже.
    Отметим, что для такой простой схемы он совпадает
    с законом Кирхгофа.
    Получим систему из 2 уравнений и 2 неизвестных, решим в удобном порядке,
    ибо нам всё равно понадобятся обе.

    \begin{align*}
        &\begin{cases}
            \ele = \eli_1(R_1 + r), \\
            \ele = \eli_2(R_2 + r); \\
        \end{cases} \\
        &\eli_1(R_1 + r) = \eli_2(R_2 + r), \\
        &\eli_1 R_1 + \eli_1r = \eli_2 R_2 + \eli_2r, \\
        &\eli_1 R_1 - \eli_2 R_2 = - \eli_1r  + \eli_2r = (\eli_2 - \eli_1)r, \\
        r &= \frac{\eli_1 R_1 - \eli_2 R_2}{\eli_2 - \eli_1}
            \equiv \frac{\eli_2 R_2 - \eli_1 R_1}{\eli_1 - \eli_2}, \\
        \ele &= \eli_1(R_1 + r)
            = \eli_1\cbr{R_1 + \frac{\eli_1 R_1 - \eli_2 R_2}{\eli_2 - \eli_1}}
            = \eli_1 \cdot \frac{R_1\eli_2 - R_1\eli_1 + \eli_1 R_1 - \eli_2 R_2}{\eli_2 - \eli_1} \\
            &= \eli_1 \cdot \frac{R_1\eli_2 - \eli_2 R_2}{\eli_2 - \eli_1}
            = \frac{\eli_1 \eli_2 (R_1 - R_2)}{\eli_2 - \eli_1}
            \equiv \frac{\eli_1 \eli_2 (R_2 - R_1)}{\eli_1 - \eli_2}.
    \end{align*}

    Короткое замыкание происходит в ситуации, когда внешнее сопротивление равно 0
    (при этом цепь замкнута, хотя нагрузки и нет вовсе):
    $$
        \eli_\text{к.
        з.} = \frac \ele {0 + r} = \frac \ele r
            = \frac{\cfrac{\eli_1 \eli_2 (R_1 - R_2)}{\eli_2 - \eli_1}}{\cfrac{\eli_1 R_1 - \eli_2 R_2}{\eli_2 - \eli_1}}
            = \frac{\eli_1 \eli_2 (R_1 - R_2)}{\eli_1 R_1 - \eli_2 R_2}
            \equiv \frac{\eli_1 \eli_2 (R_2 - R_1)}{\eli_2 R_2 - \eli_1 R_1}.
    $$

    Важные пункты:
    \begin{itemize}
        \item В ответах есть только те величины, которые есть в условии
            (и ещё физические постоянные могут встретиться, но нам не понадобилось).
        \item Мы упростили выражения, который пошли в ответы (благо у нас даже получилось:
            приведение к общему знаменателю укоротило ответ).
            Надо доделывать.
        \item Всё ответы симметричны относительно замены резисторов 1 и 2 (ведь при этом изменятся и токи).
    \end{itemize}
}
\solutionspace{120pt}

\tasknumber{3}%
\task{%
    Определите ток, протекающий через резистор $R_1$, разность потенциалов на нём (см.
    рис.)
    и выделяющуюся на нём мощность, если известны $r_1, r_2, \ele_1, \ele_2, R_1, R_2$.

    \begin{tikzpicture}[circuit ee IEC, thick]
        \draw  (0, 0) -- ++(up:2)
                to [
                    battery={ very near start, rotate=-180, info={$\ele_1, r_1 $}},
                    resistor={ midway, info=$R_1$},
                    battery={ very near end, rotate=-180, info={$\ele_2, r_2 $}}
                ] ++(right:5)
                -- ++(down:2)
                to [resistor={info=$R_2$}] ++(left:5);
    \end{tikzpicture}
}
\answer{%
    Нетривиальных узлов нет, поэтому все законы Кирхгофа для узлов будут иметь вид
    $\eli-\eli=0$ и ничем нам не помогут.
    Впрочем, если бы мы обозначили токи на разных участках контура $\eli_1, \eli_2, \eli_3, \ldots$,
    то именно эти законы бы помогли понять, что все эти токи равны: $\eli_1 - \eli_2 = 0$ и т.д.
    Так что запишем закон Кирхгофа для единственного замкнутого контура:

    \begin{tikzpicture}[circuit ee IEC, thick]
        \draw  (0, 0) -- ++(up:2)
                to [
                    battery={ very near start, rotate=-180, info={$\ele_1, r_1 $}},
                    resistor={ midway, info=$R_1$},
                    battery={ very near end, rotate=-180, info={$\ele_2, r_2 $}}
                ] ++(right:5)
                -- ++(down:2)
                to [resistor={info=$R_2$}, current direction={near end, info=$\eli$}] ++(left:5);
        \draw [-{Latex}] (2, 1.4) arc [start angle = 135, end angle = -160, radius = 0.6];
    \end{tikzpicture}

    \begin{align*}
        & \ele_1 +  \ele_2 = \eli R_1 + \eli r_2 + \eli R_2 + \eli r_1, \\
        & \ele_1 +  \ele_2 = \eli (R_1 + r_2 + R_2 + r_1), \\
        &\eli = \frac{ \ele_1 +  \ele_2 }{ R_1 + r_2 + R_2 + r_1 }, \\
        &U_1 = \eli R_1 = \frac{ \ele_1 +  \ele_2 }{ R_1 + r_2 + R_2 + r_1 } \cdot R_1, \\
        &P_1 = \eli^2 R_1 = \frac{\sqr{ \ele_1 +  \ele_2 } R_1}{ \sqr{ R_1 + r_2 + R_2 + r_1 }}.
    \end{align*}

    Отметим, что ответ для тока $\eli$ меняет знак, если отметить его на рисунке в другую сторону.
    Поэтому критично важно указывать на рисунке направление тока, иначе невозможно утверждать, что ответ верный.
    А вот выбор направления контура — не повлияет на ответ, но для проверки корректности записи законо Кирхгофа,
    там тоже необходимо направление.
}

\variantsplitter

\addpersonalvariant{Евгений Васин}

\tasknumber{1}%
\task{%
    Два резистора сопротивлениями $R_1=7R$ и $R_2=4R$ подключены последовательно к источнику напряжения.
    Определите, в каком резисторе выделяется большая тепловая мощность и во сколько раз?
}
\answer{%
    Подключены последовательно, поэтому  $\eli_1 = \eli_2 = \eli \implies \frac{P_2}{P_1} = \frac{\eli_2^2 R_2}{\eli_1^2 R_1} = \frac{\eli_2^2R_2}{\eli_1R_1} = \frac{R_2}{R_1} = \frac47$.
}
\solutionspace{120pt}

\tasknumber{2}%
\task{%
    Если батарею замкнуть на резистор сопротивлением $R_1$, то в цепи потечёт ток $\eli_1$,
    а если на другой $R_2$ — то $\eli_2$.
    Определите:
    \begin{itemize}
        \item ЭДС батареи,
        \item внутреннее сопротивление батареи,
        \item ток короткого замыкания.
    \end{itemize}
}
\answer{%
    Запишем закон Ома для полной цепи 2 раза для обоих способов подключения (с $R_1$ и с $R_2$),
    а короткое замыкание рассмотрим позже.
    Отметим, что для такой простой схемы он совпадает
    с законом Кирхгофа.
    Получим систему из 2 уравнений и 2 неизвестных, решим в удобном порядке,
    ибо нам всё равно понадобятся обе.

    \begin{align*}
        &\begin{cases}
            \ele = \eli_1(R_1 + r), \\
            \ele = \eli_2(R_2 + r); \\
        \end{cases} \\
        &\eli_1(R_1 + r) = \eli_2(R_2 + r), \\
        &\eli_1 R_1 + \eli_1r = \eli_2 R_2 + \eli_2r, \\
        &\eli_1 R_1 - \eli_2 R_2 = - \eli_1r  + \eli_2r = (\eli_2 - \eli_1)r, \\
        r &= \frac{\eli_1 R_1 - \eli_2 R_2}{\eli_2 - \eli_1}
            \equiv \frac{\eli_2 R_2 - \eli_1 R_1}{\eli_1 - \eli_2}, \\
        \ele &= \eli_1(R_1 + r)
            = \eli_1\cbr{R_1 + \frac{\eli_1 R_1 - \eli_2 R_2}{\eli_2 - \eli_1}}
            = \eli_1 \cdot \frac{R_1\eli_2 - R_1\eli_1 + \eli_1 R_1 - \eli_2 R_2}{\eli_2 - \eli_1} \\
            &= \eli_1 \cdot \frac{R_1\eli_2 - \eli_2 R_2}{\eli_2 - \eli_1}
            = \frac{\eli_1 \eli_2 (R_1 - R_2)}{\eli_2 - \eli_1}
            \equiv \frac{\eli_1 \eli_2 (R_2 - R_1)}{\eli_1 - \eli_2}.
    \end{align*}

    Короткое замыкание происходит в ситуации, когда внешнее сопротивление равно 0
    (при этом цепь замкнута, хотя нагрузки и нет вовсе):
    $$
        \eli_\text{к.
        з.} = \frac \ele {0 + r} = \frac \ele r
            = \frac{\cfrac{\eli_1 \eli_2 (R_1 - R_2)}{\eli_2 - \eli_1}}{\cfrac{\eli_1 R_1 - \eli_2 R_2}{\eli_2 - \eli_1}}
            = \frac{\eli_1 \eli_2 (R_1 - R_2)}{\eli_1 R_1 - \eli_2 R_2}
            \equiv \frac{\eli_1 \eli_2 (R_2 - R_1)}{\eli_2 R_2 - \eli_1 R_1}.
    $$

    Важные пункты:
    \begin{itemize}
        \item В ответах есть только те величины, которые есть в условии
            (и ещё физические постоянные могут встретиться, но нам не понадобилось).
        \item Мы упростили выражения, который пошли в ответы (благо у нас даже получилось:
            приведение к общему знаменателю укоротило ответ).
            Надо доделывать.
        \item Всё ответы симметричны относительно замены резисторов 1 и 2 (ведь при этом изменятся и токи).
    \end{itemize}
}
\solutionspace{120pt}

\tasknumber{3}%
\task{%
    Определите ток, протекающий через резистор $R_1$, разность потенциалов на нём (см.
    рис.)
    и выделяющуюся на нём мощность, если известны $r_1, r_2, \ele_1, \ele_2, R_1, R_2$.

    \begin{tikzpicture}[circuit ee IEC, thick]
        \draw  (0, 0) -- ++(up:2)
                to [
                    battery={ very near start, rotate=0, info={$\ele_1, r_1 $}},
                    resistor={ midway, info=$R_1$},
                    battery={ very near end, rotate=-180, info={$\ele_2, r_2 $}}
                ] ++(right:5)
                -- ++(down:2)
                to [resistor={info=$R_2$}] ++(left:5);
    \end{tikzpicture}
}
\answer{%
    Нетривиальных узлов нет, поэтому все законы Кирхгофа для узлов будут иметь вид
    $\eli-\eli=0$ и ничем нам не помогут.
    Впрочем, если бы мы обозначили токи на разных участках контура $\eli_1, \eli_2, \eli_3, \ldots$,
    то именно эти законы бы помогли понять, что все эти токи равны: $\eli_1 - \eli_2 = 0$ и т.д.
    Так что запишем закон Кирхгофа для единственного замкнутого контура:

    \begin{tikzpicture}[circuit ee IEC, thick]
        \draw  (0, 0) -- ++(up:2)
                to [
                    battery={ very near start, rotate=0, info={$\ele_1, r_1 $}},
                    resistor={ midway, info=$R_1$},
                    battery={ very near end, rotate=-180, info={$\ele_2, r_2 $}}
                ] ++(right:5)
                -- ++(down:2)
                to [resistor={info=$R_2$}, current direction={near end, info=$\eli$}] ++(left:5);
        \draw [-{Latex}] (2, 1.4) arc [start angle = 135, end angle = -160, radius = 0.6];
    \end{tikzpicture}

    \begin{align*}
        &- \ele_1 +  \ele_2 = \eli R_1 + \eli r_2 + \eli R_2 + \eli r_1, \\
        &- \ele_1 +  \ele_2 = \eli (R_1 + r_2 + R_2 + r_1), \\
        &\eli = \frac{- \ele_1 +  \ele_2 }{ R_1 + r_2 + R_2 + r_1 }, \\
        &U_1 = \eli R_1 = \frac{- \ele_1 +  \ele_2 }{ R_1 + r_2 + R_2 + r_1 } \cdot R_1, \\
        &P_1 = \eli^2 R_1 = \frac{\sqr{- \ele_1 +  \ele_2 } R_1}{ \sqr{ R_1 + r_2 + R_2 + r_1 }}.
    \end{align*}

    Отметим, что ответ для тока $\eli$ меняет знак, если отметить его на рисунке в другую сторону.
    Поэтому критично важно указывать на рисунке направление тока, иначе невозможно утверждать, что ответ верный.
    А вот выбор направления контура — не повлияет на ответ, но для проверки корректности записи законо Кирхгофа,
    там тоже необходимо направление.
}

\variantsplitter

\addpersonalvariant{Вячеслав Волохов}

\tasknumber{1}%
\task{%
    Два резистора сопротивлениями $R_1=5R$ и $R_2=8R$ подключены параллельно к источнику напряжения.
    Определите, в каком резисторе выделяется большая тепловая мощность и во сколько раз?
}
\answer{%
    Подключены параллельно, поэтому  $U_1 = U_2 = U \implies \frac{P_2}{P_1} = \frac{\frac{U_2^2}{R_2}}{\frac{U_1^2}{R_1}} = \frac{U^2R_1}{U^2R_2} = \frac{R_1}{R_2} = \frac58$.
}
\solutionspace{120pt}

\tasknumber{2}%
\task{%
    Если батарею замкнуть на резистор сопротивлением $R_1$, то в цепи потечёт ток $\eli_1$,
    а если на другой $R_2$ — то $\eli_2$.
    Определите:
    \begin{itemize}
        \item ЭДС батареи,
        \item внутреннее сопротивление батареи,
        \item ток короткого замыкания.
    \end{itemize}
}
\answer{%
    Запишем закон Ома для полной цепи 2 раза для обоих способов подключения (с $R_1$ и с $R_2$),
    а короткое замыкание рассмотрим позже.
    Отметим, что для такой простой схемы он совпадает
    с законом Кирхгофа.
    Получим систему из 2 уравнений и 2 неизвестных, решим в удобном порядке,
    ибо нам всё равно понадобятся обе.

    \begin{align*}
        &\begin{cases}
            \ele = \eli_1(R_1 + r), \\
            \ele = \eli_2(R_2 + r); \\
        \end{cases} \\
        &\eli_1(R_1 + r) = \eli_2(R_2 + r), \\
        &\eli_1 R_1 + \eli_1r = \eli_2 R_2 + \eli_2r, \\
        &\eli_1 R_1 - \eli_2 R_2 = - \eli_1r  + \eli_2r = (\eli_2 - \eli_1)r, \\
        r &= \frac{\eli_1 R_1 - \eli_2 R_2}{\eli_2 - \eli_1}
            \equiv \frac{\eli_2 R_2 - \eli_1 R_1}{\eli_1 - \eli_2}, \\
        \ele &= \eli_1(R_1 + r)
            = \eli_1\cbr{R_1 + \frac{\eli_1 R_1 - \eli_2 R_2}{\eli_2 - \eli_1}}
            = \eli_1 \cdot \frac{R_1\eli_2 - R_1\eli_1 + \eli_1 R_1 - \eli_2 R_2}{\eli_2 - \eli_1} \\
            &= \eli_1 \cdot \frac{R_1\eli_2 - \eli_2 R_2}{\eli_2 - \eli_1}
            = \frac{\eli_1 \eli_2 (R_1 - R_2)}{\eli_2 - \eli_1}
            \equiv \frac{\eli_1 \eli_2 (R_2 - R_1)}{\eli_1 - \eli_2}.
    \end{align*}

    Короткое замыкание происходит в ситуации, когда внешнее сопротивление равно 0
    (при этом цепь замкнута, хотя нагрузки и нет вовсе):
    $$
        \eli_\text{к.
        з.} = \frac \ele {0 + r} = \frac \ele r
            = \frac{\cfrac{\eli_1 \eli_2 (R_1 - R_2)}{\eli_2 - \eli_1}}{\cfrac{\eli_1 R_1 - \eli_2 R_2}{\eli_2 - \eli_1}}
            = \frac{\eli_1 \eli_2 (R_1 - R_2)}{\eli_1 R_1 - \eli_2 R_2}
            \equiv \frac{\eli_1 \eli_2 (R_2 - R_1)}{\eli_2 R_2 - \eli_1 R_1}.
    $$

    Важные пункты:
    \begin{itemize}
        \item В ответах есть только те величины, которые есть в условии
            (и ещё физические постоянные могут встретиться, но нам не понадобилось).
        \item Мы упростили выражения, который пошли в ответы (благо у нас даже получилось:
            приведение к общему знаменателю укоротило ответ).
            Надо доделывать.
        \item Всё ответы симметричны относительно замены резисторов 1 и 2 (ведь при этом изменятся и токи).
    \end{itemize}
}
\solutionspace{120pt}

\tasknumber{3}%
\task{%
    Определите ток, протекающий через резистор $R_2$, разность потенциалов на нём (см.
    рис.)
    и выделяющуюся на нём мощность, если известны $r_1, r_2, \ele_1, \ele_2, R_1, R_2$.

    \begin{tikzpicture}[circuit ee IEC, thick]
        \draw  (0, 0) -- ++(up:2)
                to [
                    battery={ very near start, rotate=-180, info={$\ele_1, r_1 $}},
                    resistor={ midway, info=$R_1$},
                    battery={ very near end, rotate=0, info={$\ele_2, r_2 $}}
                ] ++(right:5)
                -- ++(down:2)
                to [resistor={info=$R_2$}] ++(left:5);
    \end{tikzpicture}
}
\answer{%
    Нетривиальных узлов нет, поэтому все законы Кирхгофа для узлов будут иметь вид
    $\eli-\eli=0$ и ничем нам не помогут.
    Впрочем, если бы мы обозначили токи на разных участках контура $\eli_1, \eli_2, \eli_3, \ldots$,
    то именно эти законы бы помогли понять, что все эти токи равны: $\eli_1 - \eli_2 = 0$ и т.д.
    Так что запишем закон Кирхгофа для единственного замкнутого контура:

    \begin{tikzpicture}[circuit ee IEC, thick]
        \draw  (0, 0) -- ++(up:2)
                to [
                    battery={ very near start, rotate=-180, info={$\ele_1, r_1 $}},
                    resistor={ midway, info=$R_1$},
                    battery={ very near end, rotate=0, info={$\ele_2, r_2 $}}
                ] ++(right:5)
                -- ++(down:2)
                to [resistor={info=$R_2$}, current direction={near end, info=$\eli$}] ++(left:5);
        \draw [-{Latex}] (2, 1.4) arc [start angle = 135, end angle = -160, radius = 0.6];
    \end{tikzpicture}

    \begin{align*}
        & \ele_1 - \ele_2 = \eli R_1 + \eli r_2 + \eli R_2 + \eli r_1, \\
        & \ele_1 - \ele_2 = \eli (R_1 + r_2 + R_2 + r_1), \\
        &\eli = \frac{ \ele_1 - \ele_2 }{ R_1 + r_2 + R_2 + r_1 }, \\
        &U_2 = \eli R_2 = \frac{ \ele_1 - \ele_2 }{ R_1 + r_2 + R_2 + r_1 } \cdot R_2, \\
        &P_2 = \eli^2 R_2 = \frac{\sqr{ \ele_1 - \ele_2 } R_2}{ \sqr{ R_1 + r_2 + R_2 + r_1 }}.
    \end{align*}

    Отметим, что ответ для тока $\eli$ меняет знак, если отметить его на рисунке в другую сторону.
    Поэтому критично важно указывать на рисунке направление тока, иначе невозможно утверждать, что ответ верный.
    А вот выбор направления контура — не повлияет на ответ, но для проверки корректности записи законо Кирхгофа,
    там тоже необходимо направление.
}

\variantsplitter

\addpersonalvariant{Герман Говоров}

\tasknumber{1}%
\task{%
    Два резистора сопротивлениями $R_1=5R$ и $R_2=8R$ подключены последовательно к источнику напряжения.
    Определите, в каком резисторе выделяется большая тепловая мощность и во сколько раз?
}
\answer{%
    Подключены последовательно, поэтому  $\eli_1 = \eli_2 = \eli \implies \frac{P_2}{P_1} = \frac{\eli_2^2 R_2}{\eli_1^2 R_1} = \frac{\eli_2^2R_2}{\eli_1R_1} = \frac{R_2}{R_1} = \frac85$.
}
\solutionspace{120pt}

\tasknumber{2}%
\task{%
    Если батарею замкнуть на резистор сопротивлением $R_1$, то в цепи потечёт ток $\eli_1$,
    а если на другой $R_2$ — то $\eli_2$.
    Определите:
    \begin{itemize}
        \item ЭДС батареи,
        \item внутреннее сопротивление батареи,
        \item ток короткого замыкания.
    \end{itemize}
}
\answer{%
    Запишем закон Ома для полной цепи 2 раза для обоих способов подключения (с $R_1$ и с $R_2$),
    а короткое замыкание рассмотрим позже.
    Отметим, что для такой простой схемы он совпадает
    с законом Кирхгофа.
    Получим систему из 2 уравнений и 2 неизвестных, решим в удобном порядке,
    ибо нам всё равно понадобятся обе.

    \begin{align*}
        &\begin{cases}
            \ele = \eli_1(R_1 + r), \\
            \ele = \eli_2(R_2 + r); \\
        \end{cases} \\
        &\eli_1(R_1 + r) = \eli_2(R_2 + r), \\
        &\eli_1 R_1 + \eli_1r = \eli_2 R_2 + \eli_2r, \\
        &\eli_1 R_1 - \eli_2 R_2 = - \eli_1r  + \eli_2r = (\eli_2 - \eli_1)r, \\
        r &= \frac{\eli_1 R_1 - \eli_2 R_2}{\eli_2 - \eli_1}
            \equiv \frac{\eli_2 R_2 - \eli_1 R_1}{\eli_1 - \eli_2}, \\
        \ele &= \eli_1(R_1 + r)
            = \eli_1\cbr{R_1 + \frac{\eli_1 R_1 - \eli_2 R_2}{\eli_2 - \eli_1}}
            = \eli_1 \cdot \frac{R_1\eli_2 - R_1\eli_1 + \eli_1 R_1 - \eli_2 R_2}{\eli_2 - \eli_1} \\
            &= \eli_1 \cdot \frac{R_1\eli_2 - \eli_2 R_2}{\eli_2 - \eli_1}
            = \frac{\eli_1 \eli_2 (R_1 - R_2)}{\eli_2 - \eli_1}
            \equiv \frac{\eli_1 \eli_2 (R_2 - R_1)}{\eli_1 - \eli_2}.
    \end{align*}

    Короткое замыкание происходит в ситуации, когда внешнее сопротивление равно 0
    (при этом цепь замкнута, хотя нагрузки и нет вовсе):
    $$
        \eli_\text{к.
        з.} = \frac \ele {0 + r} = \frac \ele r
            = \frac{\cfrac{\eli_1 \eli_2 (R_1 - R_2)}{\eli_2 - \eli_1}}{\cfrac{\eli_1 R_1 - \eli_2 R_2}{\eli_2 - \eli_1}}
            = \frac{\eli_1 \eli_2 (R_1 - R_2)}{\eli_1 R_1 - \eli_2 R_2}
            \equiv \frac{\eli_1 \eli_2 (R_2 - R_1)}{\eli_2 R_2 - \eli_1 R_1}.
    $$

    Важные пункты:
    \begin{itemize}
        \item В ответах есть только те величины, которые есть в условии
            (и ещё физические постоянные могут встретиться, но нам не понадобилось).
        \item Мы упростили выражения, который пошли в ответы (благо у нас даже получилось:
            приведение к общему знаменателю укоротило ответ).
            Надо доделывать.
        \item Всё ответы симметричны относительно замены резисторов 1 и 2 (ведь при этом изменятся и токи).
    \end{itemize}
}
\solutionspace{120pt}

\tasknumber{3}%
\task{%
    Определите ток, протекающий через резистор $R_1$, разность потенциалов на нём (см.
    рис.)
    и выделяющуюся на нём мощность, если известны $r_1, r_2, \ele_1, \ele_2, R_1, R_2$.

    \begin{tikzpicture}[circuit ee IEC, thick]
        \draw  (0, 0) -- ++(up:2)
                to [
                    battery={ very near start, rotate=0, info={$\ele_1, r_1 $}},
                    resistor={ midway, info=$R_1$},
                    battery={ very near end, rotate=-180, info={$\ele_2, r_2 $}}
                ] ++(right:5)
                -- ++(down:2)
                to [resistor={info=$R_2$}] ++(left:5);
    \end{tikzpicture}
}
\answer{%
    Нетривиальных узлов нет, поэтому все законы Кирхгофа для узлов будут иметь вид
    $\eli-\eli=0$ и ничем нам не помогут.
    Впрочем, если бы мы обозначили токи на разных участках контура $\eli_1, \eli_2, \eli_3, \ldots$,
    то именно эти законы бы помогли понять, что все эти токи равны: $\eli_1 - \eli_2 = 0$ и т.д.
    Так что запишем закон Кирхгофа для единственного замкнутого контура:

    \begin{tikzpicture}[circuit ee IEC, thick]
        \draw  (0, 0) -- ++(up:2)
                to [
                    battery={ very near start, rotate=0, info={$\ele_1, r_1 $}},
                    resistor={ midway, info=$R_1$},
                    battery={ very near end, rotate=-180, info={$\ele_2, r_2 $}}
                ] ++(right:5)
                -- ++(down:2)
                to [resistor={info=$R_2$}, current direction={near end, info=$\eli$}] ++(left:5);
        \draw [-{Latex}] (2, 1.4) arc [start angle = 135, end angle = -160, radius = 0.6];
    \end{tikzpicture}

    \begin{align*}
        &- \ele_1 +  \ele_2 = \eli R_1 + \eli r_2 + \eli R_2 + \eli r_1, \\
        &- \ele_1 +  \ele_2 = \eli (R_1 + r_2 + R_2 + r_1), \\
        &\eli = \frac{- \ele_1 +  \ele_2 }{ R_1 + r_2 + R_2 + r_1 }, \\
        &U_1 = \eli R_1 = \frac{- \ele_1 +  \ele_2 }{ R_1 + r_2 + R_2 + r_1 } \cdot R_1, \\
        &P_1 = \eli^2 R_1 = \frac{\sqr{- \ele_1 +  \ele_2 } R_1}{ \sqr{ R_1 + r_2 + R_2 + r_1 }}.
    \end{align*}

    Отметим, что ответ для тока $\eli$ меняет знак, если отметить его на рисунке в другую сторону.
    Поэтому критично важно указывать на рисунке направление тока, иначе невозможно утверждать, что ответ верный.
    А вот выбор направления контура — не повлияет на ответ, но для проверки корректности записи законо Кирхгофа,
    там тоже необходимо направление.
}

\variantsplitter

\addpersonalvariant{София Журавлёва}

\tasknumber{1}%
\task{%
    Два резистора сопротивлениями $R_1=5R$ и $R_2=6R$ подключены параллельно к источнику напряжения.
    Определите, в каком резисторе выделяется большая тепловая мощность и во сколько раз?
}
\answer{%
    Подключены параллельно, поэтому  $U_1 = U_2 = U \implies \frac{P_2}{P_1} = \frac{\frac{U_2^2}{R_2}}{\frac{U_1^2}{R_1}} = \frac{U^2R_1}{U^2R_2} = \frac{R_1}{R_2} = \frac56$.
}
\solutionspace{120pt}

\tasknumber{2}%
\task{%
    Если батарею замкнуть на резистор сопротивлением $R_1$, то в цепи потечёт ток $\eli_1$,
    а если на другой $R_2$ — то $\eli_2$.
    Определите:
    \begin{itemize}
        \item ЭДС батареи,
        \item внутреннее сопротивление батареи,
        \item ток короткого замыкания.
    \end{itemize}
}
\answer{%
    Запишем закон Ома для полной цепи 2 раза для обоих способов подключения (с $R_1$ и с $R_2$),
    а короткое замыкание рассмотрим позже.
    Отметим, что для такой простой схемы он совпадает
    с законом Кирхгофа.
    Получим систему из 2 уравнений и 2 неизвестных, решим в удобном порядке,
    ибо нам всё равно понадобятся обе.

    \begin{align*}
        &\begin{cases}
            \ele = \eli_1(R_1 + r), \\
            \ele = \eli_2(R_2 + r); \\
        \end{cases} \\
        &\eli_1(R_1 + r) = \eli_2(R_2 + r), \\
        &\eli_1 R_1 + \eli_1r = \eli_2 R_2 + \eli_2r, \\
        &\eli_1 R_1 - \eli_2 R_2 = - \eli_1r  + \eli_2r = (\eli_2 - \eli_1)r, \\
        r &= \frac{\eli_1 R_1 - \eli_2 R_2}{\eli_2 - \eli_1}
            \equiv \frac{\eli_2 R_2 - \eli_1 R_1}{\eli_1 - \eli_2}, \\
        \ele &= \eli_1(R_1 + r)
            = \eli_1\cbr{R_1 + \frac{\eli_1 R_1 - \eli_2 R_2}{\eli_2 - \eli_1}}
            = \eli_1 \cdot \frac{R_1\eli_2 - R_1\eli_1 + \eli_1 R_1 - \eli_2 R_2}{\eli_2 - \eli_1} \\
            &= \eli_1 \cdot \frac{R_1\eli_2 - \eli_2 R_2}{\eli_2 - \eli_1}
            = \frac{\eli_1 \eli_2 (R_1 - R_2)}{\eli_2 - \eli_1}
            \equiv \frac{\eli_1 \eli_2 (R_2 - R_1)}{\eli_1 - \eli_2}.
    \end{align*}

    Короткое замыкание происходит в ситуации, когда внешнее сопротивление равно 0
    (при этом цепь замкнута, хотя нагрузки и нет вовсе):
    $$
        \eli_\text{к.
        з.} = \frac \ele {0 + r} = \frac \ele r
            = \frac{\cfrac{\eli_1 \eli_2 (R_1 - R_2)}{\eli_2 - \eli_1}}{\cfrac{\eli_1 R_1 - \eli_2 R_2}{\eli_2 - \eli_1}}
            = \frac{\eli_1 \eli_2 (R_1 - R_2)}{\eli_1 R_1 - \eli_2 R_2}
            \equiv \frac{\eli_1 \eli_2 (R_2 - R_1)}{\eli_2 R_2 - \eli_1 R_1}.
    $$

    Важные пункты:
    \begin{itemize}
        \item В ответах есть только те величины, которые есть в условии
            (и ещё физические постоянные могут встретиться, но нам не понадобилось).
        \item Мы упростили выражения, который пошли в ответы (благо у нас даже получилось:
            приведение к общему знаменателю укоротило ответ).
            Надо доделывать.
        \item Всё ответы симметричны относительно замены резисторов 1 и 2 (ведь при этом изменятся и токи).
    \end{itemize}
}
\solutionspace{120pt}

\tasknumber{3}%
\task{%
    Определите ток, протекающий через резистор $R_1$, разность потенциалов на нём (см.
    рис.)
    и выделяющуюся на нём мощность, если известны $r_1, r_2, \ele_1, \ele_2, R_1, R_2$.

    \begin{tikzpicture}[circuit ee IEC, thick]
        \draw  (0, 0) -- ++(up:2)
                to [
                    battery={ very near start, rotate=-180, info={$\ele_1, r_1 $}},
                    resistor={ midway, info=$R_1$},
                    battery={ very near end, rotate=0, info={$\ele_2, r_2 $}}
                ] ++(right:5)
                -- ++(down:2)
                to [resistor={info=$R_2$}] ++(left:5);
    \end{tikzpicture}
}
\answer{%
    Нетривиальных узлов нет, поэтому все законы Кирхгофа для узлов будут иметь вид
    $\eli-\eli=0$ и ничем нам не помогут.
    Впрочем, если бы мы обозначили токи на разных участках контура $\eli_1, \eli_2, \eli_3, \ldots$,
    то именно эти законы бы помогли понять, что все эти токи равны: $\eli_1 - \eli_2 = 0$ и т.д.
    Так что запишем закон Кирхгофа для единственного замкнутого контура:

    \begin{tikzpicture}[circuit ee IEC, thick]
        \draw  (0, 0) -- ++(up:2)
                to [
                    battery={ very near start, rotate=-180, info={$\ele_1, r_1 $}},
                    resistor={ midway, info=$R_1$},
                    battery={ very near end, rotate=0, info={$\ele_2, r_2 $}}
                ] ++(right:5)
                -- ++(down:2)
                to [resistor={info=$R_2$}, current direction={near end, info=$\eli$}] ++(left:5);
        \draw [-{Latex}] (2, 1.4) arc [start angle = 135, end angle = -160, radius = 0.6];
    \end{tikzpicture}

    \begin{align*}
        & \ele_1 - \ele_2 = \eli R_1 + \eli r_2 + \eli R_2 + \eli r_1, \\
        & \ele_1 - \ele_2 = \eli (R_1 + r_2 + R_2 + r_1), \\
        &\eli = \frac{ \ele_1 - \ele_2 }{ R_1 + r_2 + R_2 + r_1 }, \\
        &U_1 = \eli R_1 = \frac{ \ele_1 - \ele_2 }{ R_1 + r_2 + R_2 + r_1 } \cdot R_1, \\
        &P_1 = \eli^2 R_1 = \frac{\sqr{ \ele_1 - \ele_2 } R_1}{ \sqr{ R_1 + r_2 + R_2 + r_1 }}.
    \end{align*}

    Отметим, что ответ для тока $\eli$ меняет знак, если отметить его на рисунке в другую сторону.
    Поэтому критично важно указывать на рисунке направление тока, иначе невозможно утверждать, что ответ верный.
    А вот выбор направления контура — не повлияет на ответ, но для проверки корректности записи законо Кирхгофа,
    там тоже необходимо направление.
}

\variantsplitter

\addpersonalvariant{Константин Козлов}

\tasknumber{1}%
\task{%
    Два резистора сопротивлениями $R_1=3R$ и $R_2=8R$ подключены параллельно к источнику напряжения.
    Определите, в каком резисторе выделяется большая тепловая мощность и во сколько раз?
}
\answer{%
    Подключены параллельно, поэтому  $U_1 = U_2 = U \implies \frac{P_2}{P_1} = \frac{\frac{U_2^2}{R_2}}{\frac{U_1^2}{R_1}} = \frac{U^2R_1}{U^2R_2} = \frac{R_1}{R_2} = \frac38$.
}
\solutionspace{120pt}

\tasknumber{2}%
\task{%
    Если батарею замкнуть на резистор сопротивлением $R_1$, то в цепи потечёт ток $\eli_1$,
    а если на другой $R_2$ — то $\eli_2$.
    Определите:
    \begin{itemize}
        \item ЭДС батареи,
        \item внутреннее сопротивление батареи,
        \item ток короткого замыкания.
    \end{itemize}
}
\answer{%
    Запишем закон Ома для полной цепи 2 раза для обоих способов подключения (с $R_1$ и с $R_2$),
    а короткое замыкание рассмотрим позже.
    Отметим, что для такой простой схемы он совпадает
    с законом Кирхгофа.
    Получим систему из 2 уравнений и 2 неизвестных, решим в удобном порядке,
    ибо нам всё равно понадобятся обе.

    \begin{align*}
        &\begin{cases}
            \ele = \eli_1(R_1 + r), \\
            \ele = \eli_2(R_2 + r); \\
        \end{cases} \\
        &\eli_1(R_1 + r) = \eli_2(R_2 + r), \\
        &\eli_1 R_1 + \eli_1r = \eli_2 R_2 + \eli_2r, \\
        &\eli_1 R_1 - \eli_2 R_2 = - \eli_1r  + \eli_2r = (\eli_2 - \eli_1)r, \\
        r &= \frac{\eli_1 R_1 - \eli_2 R_2}{\eli_2 - \eli_1}
            \equiv \frac{\eli_2 R_2 - \eli_1 R_1}{\eli_1 - \eli_2}, \\
        \ele &= \eli_1(R_1 + r)
            = \eli_1\cbr{R_1 + \frac{\eli_1 R_1 - \eli_2 R_2}{\eli_2 - \eli_1}}
            = \eli_1 \cdot \frac{R_1\eli_2 - R_1\eli_1 + \eli_1 R_1 - \eli_2 R_2}{\eli_2 - \eli_1} \\
            &= \eli_1 \cdot \frac{R_1\eli_2 - \eli_2 R_2}{\eli_2 - \eli_1}
            = \frac{\eli_1 \eli_2 (R_1 - R_2)}{\eli_2 - \eli_1}
            \equiv \frac{\eli_1 \eli_2 (R_2 - R_1)}{\eli_1 - \eli_2}.
    \end{align*}

    Короткое замыкание происходит в ситуации, когда внешнее сопротивление равно 0
    (при этом цепь замкнута, хотя нагрузки и нет вовсе):
    $$
        \eli_\text{к.
        з.} = \frac \ele {0 + r} = \frac \ele r
            = \frac{\cfrac{\eli_1 \eli_2 (R_1 - R_2)}{\eli_2 - \eli_1}}{\cfrac{\eli_1 R_1 - \eli_2 R_2}{\eli_2 - \eli_1}}
            = \frac{\eli_1 \eli_2 (R_1 - R_2)}{\eli_1 R_1 - \eli_2 R_2}
            \equiv \frac{\eli_1 \eli_2 (R_2 - R_1)}{\eli_2 R_2 - \eli_1 R_1}.
    $$

    Важные пункты:
    \begin{itemize}
        \item В ответах есть только те величины, которые есть в условии
            (и ещё физические постоянные могут встретиться, но нам не понадобилось).
        \item Мы упростили выражения, который пошли в ответы (благо у нас даже получилось:
            приведение к общему знаменателю укоротило ответ).
            Надо доделывать.
        \item Всё ответы симметричны относительно замены резисторов 1 и 2 (ведь при этом изменятся и токи).
    \end{itemize}
}
\solutionspace{120pt}

\tasknumber{3}%
\task{%
    Определите ток, протекающий через резистор $R_2$, разность потенциалов на нём (см.
    рис.)
    и выделяющуюся на нём мощность, если известны $r_1, r_2, \ele_1, \ele_2, R_1, R_2$.

    \begin{tikzpicture}[circuit ee IEC, thick]
        \draw  (0, 0) -- ++(up:2)
                to [
                    battery={ very near start, rotate=-180, info={$\ele_1, r_1 $}},
                    resistor={ midway, info=$R_1$},
                    battery={ very near end, rotate=0, info={$\ele_2, r_2 $}}
                ] ++(right:5)
                -- ++(down:2)
                to [resistor={info=$R_2$}] ++(left:5);
    \end{tikzpicture}
}
\answer{%
    Нетривиальных узлов нет, поэтому все законы Кирхгофа для узлов будут иметь вид
    $\eli-\eli=0$ и ничем нам не помогут.
    Впрочем, если бы мы обозначили токи на разных участках контура $\eli_1, \eli_2, \eli_3, \ldots$,
    то именно эти законы бы помогли понять, что все эти токи равны: $\eli_1 - \eli_2 = 0$ и т.д.
    Так что запишем закон Кирхгофа для единственного замкнутого контура:

    \begin{tikzpicture}[circuit ee IEC, thick]
        \draw  (0, 0) -- ++(up:2)
                to [
                    battery={ very near start, rotate=-180, info={$\ele_1, r_1 $}},
                    resistor={ midway, info=$R_1$},
                    battery={ very near end, rotate=0, info={$\ele_2, r_2 $}}
                ] ++(right:5)
                -- ++(down:2)
                to [resistor={info=$R_2$}, current direction={near end, info=$\eli$}] ++(left:5);
        \draw [-{Latex}] (2, 1.4) arc [start angle = 135, end angle = -160, radius = 0.6];
    \end{tikzpicture}

    \begin{align*}
        & \ele_1 - \ele_2 = \eli R_1 + \eli r_2 + \eli R_2 + \eli r_1, \\
        & \ele_1 - \ele_2 = \eli (R_1 + r_2 + R_2 + r_1), \\
        &\eli = \frac{ \ele_1 - \ele_2 }{ R_1 + r_2 + R_2 + r_1 }, \\
        &U_2 = \eli R_2 = \frac{ \ele_1 - \ele_2 }{ R_1 + r_2 + R_2 + r_1 } \cdot R_2, \\
        &P_2 = \eli^2 R_2 = \frac{\sqr{ \ele_1 - \ele_2 } R_2}{ \sqr{ R_1 + r_2 + R_2 + r_1 }}.
    \end{align*}

    Отметим, что ответ для тока $\eli$ меняет знак, если отметить его на рисунке в другую сторону.
    Поэтому критично важно указывать на рисунке направление тока, иначе невозможно утверждать, что ответ верный.
    А вот выбор направления контура — не повлияет на ответ, но для проверки корректности записи законо Кирхгофа,
    там тоже необходимо направление.
}

\variantsplitter

\addpersonalvariant{Наталья Кравченко}

\tasknumber{1}%
\task{%
    Два резистора сопротивлениями $R_1=3R$ и $R_2=8R$ подключены последовательно к источнику напряжения.
    Определите, в каком резисторе выделяется большая тепловая мощность и во сколько раз?
}
\answer{%
    Подключены последовательно, поэтому  $\eli_1 = \eli_2 = \eli \implies \frac{P_2}{P_1} = \frac{\eli_2^2 R_2}{\eli_1^2 R_1} = \frac{\eli_2^2R_2}{\eli_1R_1} = \frac{R_2}{R_1} = \frac83$.
}
\solutionspace{120pt}

\tasknumber{2}%
\task{%
    Если батарею замкнуть на резистор сопротивлением $R_1$, то в цепи потечёт ток $\eli_1$,
    а если на другой $R_2$ — то $\eli_2$.
    Определите:
    \begin{itemize}
        \item ЭДС батареи,
        \item внутреннее сопротивление батареи,
        \item ток короткого замыкания.
    \end{itemize}
}
\answer{%
    Запишем закон Ома для полной цепи 2 раза для обоих способов подключения (с $R_1$ и с $R_2$),
    а короткое замыкание рассмотрим позже.
    Отметим, что для такой простой схемы он совпадает
    с законом Кирхгофа.
    Получим систему из 2 уравнений и 2 неизвестных, решим в удобном порядке,
    ибо нам всё равно понадобятся обе.

    \begin{align*}
        &\begin{cases}
            \ele = \eli_1(R_1 + r), \\
            \ele = \eli_2(R_2 + r); \\
        \end{cases} \\
        &\eli_1(R_1 + r) = \eli_2(R_2 + r), \\
        &\eli_1 R_1 + \eli_1r = \eli_2 R_2 + \eli_2r, \\
        &\eli_1 R_1 - \eli_2 R_2 = - \eli_1r  + \eli_2r = (\eli_2 - \eli_1)r, \\
        r &= \frac{\eli_1 R_1 - \eli_2 R_2}{\eli_2 - \eli_1}
            \equiv \frac{\eli_2 R_2 - \eli_1 R_1}{\eli_1 - \eli_2}, \\
        \ele &= \eli_1(R_1 + r)
            = \eli_1\cbr{R_1 + \frac{\eli_1 R_1 - \eli_2 R_2}{\eli_2 - \eli_1}}
            = \eli_1 \cdot \frac{R_1\eli_2 - R_1\eli_1 + \eli_1 R_1 - \eli_2 R_2}{\eli_2 - \eli_1} \\
            &= \eli_1 \cdot \frac{R_1\eli_2 - \eli_2 R_2}{\eli_2 - \eli_1}
            = \frac{\eli_1 \eli_2 (R_1 - R_2)}{\eli_2 - \eli_1}
            \equiv \frac{\eli_1 \eli_2 (R_2 - R_1)}{\eli_1 - \eli_2}.
    \end{align*}

    Короткое замыкание происходит в ситуации, когда внешнее сопротивление равно 0
    (при этом цепь замкнута, хотя нагрузки и нет вовсе):
    $$
        \eli_\text{к.
        з.} = \frac \ele {0 + r} = \frac \ele r
            = \frac{\cfrac{\eli_1 \eli_2 (R_1 - R_2)}{\eli_2 - \eli_1}}{\cfrac{\eli_1 R_1 - \eli_2 R_2}{\eli_2 - \eli_1}}
            = \frac{\eli_1 \eli_2 (R_1 - R_2)}{\eli_1 R_1 - \eli_2 R_2}
            \equiv \frac{\eli_1 \eli_2 (R_2 - R_1)}{\eli_2 R_2 - \eli_1 R_1}.
    $$

    Важные пункты:
    \begin{itemize}
        \item В ответах есть только те величины, которые есть в условии
            (и ещё физические постоянные могут встретиться, но нам не понадобилось).
        \item Мы упростили выражения, который пошли в ответы (благо у нас даже получилось:
            приведение к общему знаменателю укоротило ответ).
            Надо доделывать.
        \item Всё ответы симметричны относительно замены резисторов 1 и 2 (ведь при этом изменятся и токи).
    \end{itemize}
}
\solutionspace{120pt}

\tasknumber{3}%
\task{%
    Определите ток, протекающий через резистор $R_1$, разность потенциалов на нём (см.
    рис.)
    и выделяющуюся на нём мощность, если известны $r_1, r_2, \ele_1, \ele_2, R_1, R_2$.

    \begin{tikzpicture}[circuit ee IEC, thick]
        \draw  (0, 0) -- ++(up:2)
                to [
                    battery={ very near start, rotate=0, info={$\ele_1, r_1 $}},
                    resistor={ midway, info=$R_1$},
                    battery={ very near end, rotate=0, info={$\ele_2, r_2 $}}
                ] ++(right:5)
                -- ++(down:2)
                to [resistor={info=$R_2$}] ++(left:5);
    \end{tikzpicture}
}
\answer{%
    Нетривиальных узлов нет, поэтому все законы Кирхгофа для узлов будут иметь вид
    $\eli-\eli=0$ и ничем нам не помогут.
    Впрочем, если бы мы обозначили токи на разных участках контура $\eli_1, \eli_2, \eli_3, \ldots$,
    то именно эти законы бы помогли понять, что все эти токи равны: $\eli_1 - \eli_2 = 0$ и т.д.
    Так что запишем закон Кирхгофа для единственного замкнутого контура:

    \begin{tikzpicture}[circuit ee IEC, thick]
        \draw  (0, 0) -- ++(up:2)
                to [
                    battery={ very near start, rotate=0, info={$\ele_1, r_1 $}},
                    resistor={ midway, info=$R_1$},
                    battery={ very near end, rotate=0, info={$\ele_2, r_2 $}}
                ] ++(right:5)
                -- ++(down:2)
                to [resistor={info=$R_2$}, current direction={near end, info=$\eli$}] ++(left:5);
        \draw [-{Latex}] (2, 1.4) arc [start angle = 135, end angle = -160, radius = 0.6];
    \end{tikzpicture}

    \begin{align*}
        &- \ele_1 - \ele_2 = \eli R_1 + \eli r_2 + \eli R_2 + \eli r_1, \\
        &- \ele_1 - \ele_2 = \eli (R_1 + r_2 + R_2 + r_1), \\
        &\eli = \frac{- \ele_1 - \ele_2 }{ R_1 + r_2 + R_2 + r_1 }, \\
        &U_1 = \eli R_1 = \frac{- \ele_1 - \ele_2 }{ R_1 + r_2 + R_2 + r_1 } \cdot R_1, \\
        &P_1 = \eli^2 R_1 = \frac{\sqr{- \ele_1 - \ele_2 } R_1}{ \sqr{ R_1 + r_2 + R_2 + r_1 }}.
    \end{align*}

    Отметим, что ответ для тока $\eli$ меняет знак, если отметить его на рисунке в другую сторону.
    Поэтому критично важно указывать на рисунке направление тока, иначе невозможно утверждать, что ответ верный.
    А вот выбор направления контура — не повлияет на ответ, но для проверки корректности записи законо Кирхгофа,
    там тоже необходимо направление.
}

\variantsplitter

\addpersonalvariant{Матвей Кузьмин}

\tasknumber{1}%
\task{%
    Два резистора сопротивлениями $R_1=5R$ и $R_2=2R$ подключены последовательно к источнику напряжения.
    Определите, в каком резисторе выделяется большая тепловая мощность и во сколько раз?
}
\answer{%
    Подключены последовательно, поэтому  $\eli_1 = \eli_2 = \eli \implies \frac{P_2}{P_1} = \frac{\eli_2^2 R_2}{\eli_1^2 R_1} = \frac{\eli_2^2R_2}{\eli_1R_1} = \frac{R_2}{R_1} = \frac25$.
}
\solutionspace{120pt}

\tasknumber{2}%
\task{%
    Если батарею замкнуть на резистор сопротивлением $R_1$, то в цепи потечёт ток $\eli_1$,
    а если на другой $R_2$ — то $\eli_2$.
    Определите:
    \begin{itemize}
        \item ЭДС батареи,
        \item внутреннее сопротивление батареи,
        \item ток короткого замыкания.
    \end{itemize}
}
\answer{%
    Запишем закон Ома для полной цепи 2 раза для обоих способов подключения (с $R_1$ и с $R_2$),
    а короткое замыкание рассмотрим позже.
    Отметим, что для такой простой схемы он совпадает
    с законом Кирхгофа.
    Получим систему из 2 уравнений и 2 неизвестных, решим в удобном порядке,
    ибо нам всё равно понадобятся обе.

    \begin{align*}
        &\begin{cases}
            \ele = \eli_1(R_1 + r), \\
            \ele = \eli_2(R_2 + r); \\
        \end{cases} \\
        &\eli_1(R_1 + r) = \eli_2(R_2 + r), \\
        &\eli_1 R_1 + \eli_1r = \eli_2 R_2 + \eli_2r, \\
        &\eli_1 R_1 - \eli_2 R_2 = - \eli_1r  + \eli_2r = (\eli_2 - \eli_1)r, \\
        r &= \frac{\eli_1 R_1 - \eli_2 R_2}{\eli_2 - \eli_1}
            \equiv \frac{\eli_2 R_2 - \eli_1 R_1}{\eli_1 - \eli_2}, \\
        \ele &= \eli_1(R_1 + r)
            = \eli_1\cbr{R_1 + \frac{\eli_1 R_1 - \eli_2 R_2}{\eli_2 - \eli_1}}
            = \eli_1 \cdot \frac{R_1\eli_2 - R_1\eli_1 + \eli_1 R_1 - \eli_2 R_2}{\eli_2 - \eli_1} \\
            &= \eli_1 \cdot \frac{R_1\eli_2 - \eli_2 R_2}{\eli_2 - \eli_1}
            = \frac{\eli_1 \eli_2 (R_1 - R_2)}{\eli_2 - \eli_1}
            \equiv \frac{\eli_1 \eli_2 (R_2 - R_1)}{\eli_1 - \eli_2}.
    \end{align*}

    Короткое замыкание происходит в ситуации, когда внешнее сопротивление равно 0
    (при этом цепь замкнута, хотя нагрузки и нет вовсе):
    $$
        \eli_\text{к.
        з.} = \frac \ele {0 + r} = \frac \ele r
            = \frac{\cfrac{\eli_1 \eli_2 (R_1 - R_2)}{\eli_2 - \eli_1}}{\cfrac{\eli_1 R_1 - \eli_2 R_2}{\eli_2 - \eli_1}}
            = \frac{\eli_1 \eli_2 (R_1 - R_2)}{\eli_1 R_1 - \eli_2 R_2}
            \equiv \frac{\eli_1 \eli_2 (R_2 - R_1)}{\eli_2 R_2 - \eli_1 R_1}.
    $$

    Важные пункты:
    \begin{itemize}
        \item В ответах есть только те величины, которые есть в условии
            (и ещё физические постоянные могут встретиться, но нам не понадобилось).
        \item Мы упростили выражения, который пошли в ответы (благо у нас даже получилось:
            приведение к общему знаменателю укоротило ответ).
            Надо доделывать.
        \item Всё ответы симметричны относительно замены резисторов 1 и 2 (ведь при этом изменятся и токи).
    \end{itemize}
}
\solutionspace{120pt}

\tasknumber{3}%
\task{%
    Определите ток, протекающий через резистор $R_2$, разность потенциалов на нём (см.
    рис.)
    и выделяющуюся на нём мощность, если известны $r_1, r_2, \ele_1, \ele_2, R_1, R_2$.

    \begin{tikzpicture}[circuit ee IEC, thick]
        \draw  (0, 0) -- ++(up:2)
                to [
                    battery={ very near start, rotate=-180, info={$\ele_1, r_1 $}},
                    resistor={ midway, info=$R_1$},
                    battery={ very near end, rotate=-180, info={$\ele_2, r_2 $}}
                ] ++(right:5)
                -- ++(down:2)
                to [resistor={info=$R_2$}] ++(left:5);
    \end{tikzpicture}
}
\answer{%
    Нетривиальных узлов нет, поэтому все законы Кирхгофа для узлов будут иметь вид
    $\eli-\eli=0$ и ничем нам не помогут.
    Впрочем, если бы мы обозначили токи на разных участках контура $\eli_1, \eli_2, \eli_3, \ldots$,
    то именно эти законы бы помогли понять, что все эти токи равны: $\eli_1 - \eli_2 = 0$ и т.д.
    Так что запишем закон Кирхгофа для единственного замкнутого контура:

    \begin{tikzpicture}[circuit ee IEC, thick]
        \draw  (0, 0) -- ++(up:2)
                to [
                    battery={ very near start, rotate=-180, info={$\ele_1, r_1 $}},
                    resistor={ midway, info=$R_1$},
                    battery={ very near end, rotate=-180, info={$\ele_2, r_2 $}}
                ] ++(right:5)
                -- ++(down:2)
                to [resistor={info=$R_2$}, current direction={near end, info=$\eli$}] ++(left:5);
        \draw [-{Latex}] (2, 1.4) arc [start angle = 135, end angle = -160, radius = 0.6];
    \end{tikzpicture}

    \begin{align*}
        & \ele_1 +  \ele_2 = \eli R_1 + \eli r_2 + \eli R_2 + \eli r_1, \\
        & \ele_1 +  \ele_2 = \eli (R_1 + r_2 + R_2 + r_1), \\
        &\eli = \frac{ \ele_1 +  \ele_2 }{ R_1 + r_2 + R_2 + r_1 }, \\
        &U_2 = \eli R_2 = \frac{ \ele_1 +  \ele_2 }{ R_1 + r_2 + R_2 + r_1 } \cdot R_2, \\
        &P_2 = \eli^2 R_2 = \frac{\sqr{ \ele_1 +  \ele_2 } R_2}{ \sqr{ R_1 + r_2 + R_2 + r_1 }}.
    \end{align*}

    Отметим, что ответ для тока $\eli$ меняет знак, если отметить его на рисунке в другую сторону.
    Поэтому критично важно указывать на рисунке направление тока, иначе невозможно утверждать, что ответ верный.
    А вот выбор направления контура — не повлияет на ответ, но для проверки корректности записи законо Кирхгофа,
    там тоже необходимо направление.
}

\variantsplitter

\addpersonalvariant{Сергей Малышев}

\tasknumber{1}%
\task{%
    Два резистора сопротивлениями $R_1=7R$ и $R_2=4R$ подключены последовательно к источнику напряжения.
    Определите, в каком резисторе выделяется большая тепловая мощность и во сколько раз?
}
\answer{%
    Подключены последовательно, поэтому  $\eli_1 = \eli_2 = \eli \implies \frac{P_2}{P_1} = \frac{\eli_2^2 R_2}{\eli_1^2 R_1} = \frac{\eli_2^2R_2}{\eli_1R_1} = \frac{R_2}{R_1} = \frac47$.
}
\solutionspace{120pt}

\tasknumber{2}%
\task{%
    Если батарею замкнуть на резистор сопротивлением $R_1$, то в цепи потечёт ток $\eli_1$,
    а если на другой $R_2$ — то $\eli_2$.
    Определите:
    \begin{itemize}
        \item ЭДС батареи,
        \item внутреннее сопротивление батареи,
        \item ток короткого замыкания.
    \end{itemize}
}
\answer{%
    Запишем закон Ома для полной цепи 2 раза для обоих способов подключения (с $R_1$ и с $R_2$),
    а короткое замыкание рассмотрим позже.
    Отметим, что для такой простой схемы он совпадает
    с законом Кирхгофа.
    Получим систему из 2 уравнений и 2 неизвестных, решим в удобном порядке,
    ибо нам всё равно понадобятся обе.

    \begin{align*}
        &\begin{cases}
            \ele = \eli_1(R_1 + r), \\
            \ele = \eli_2(R_2 + r); \\
        \end{cases} \\
        &\eli_1(R_1 + r) = \eli_2(R_2 + r), \\
        &\eli_1 R_1 + \eli_1r = \eli_2 R_2 + \eli_2r, \\
        &\eli_1 R_1 - \eli_2 R_2 = - \eli_1r  + \eli_2r = (\eli_2 - \eli_1)r, \\
        r &= \frac{\eli_1 R_1 - \eli_2 R_2}{\eli_2 - \eli_1}
            \equiv \frac{\eli_2 R_2 - \eli_1 R_1}{\eli_1 - \eli_2}, \\
        \ele &= \eli_1(R_1 + r)
            = \eli_1\cbr{R_1 + \frac{\eli_1 R_1 - \eli_2 R_2}{\eli_2 - \eli_1}}
            = \eli_1 \cdot \frac{R_1\eli_2 - R_1\eli_1 + \eli_1 R_1 - \eli_2 R_2}{\eli_2 - \eli_1} \\
            &= \eli_1 \cdot \frac{R_1\eli_2 - \eli_2 R_2}{\eli_2 - \eli_1}
            = \frac{\eli_1 \eli_2 (R_1 - R_2)}{\eli_2 - \eli_1}
            \equiv \frac{\eli_1 \eli_2 (R_2 - R_1)}{\eli_1 - \eli_2}.
    \end{align*}

    Короткое замыкание происходит в ситуации, когда внешнее сопротивление равно 0
    (при этом цепь замкнута, хотя нагрузки и нет вовсе):
    $$
        \eli_\text{к.
        з.} = \frac \ele {0 + r} = \frac \ele r
            = \frac{\cfrac{\eli_1 \eli_2 (R_1 - R_2)}{\eli_2 - \eli_1}}{\cfrac{\eli_1 R_1 - \eli_2 R_2}{\eli_2 - \eli_1}}
            = \frac{\eli_1 \eli_2 (R_1 - R_2)}{\eli_1 R_1 - \eli_2 R_2}
            \equiv \frac{\eli_1 \eli_2 (R_2 - R_1)}{\eli_2 R_2 - \eli_1 R_1}.
    $$

    Важные пункты:
    \begin{itemize}
        \item В ответах есть только те величины, которые есть в условии
            (и ещё физические постоянные могут встретиться, но нам не понадобилось).
        \item Мы упростили выражения, который пошли в ответы (благо у нас даже получилось:
            приведение к общему знаменателю укоротило ответ).
            Надо доделывать.
        \item Всё ответы симметричны относительно замены резисторов 1 и 2 (ведь при этом изменятся и токи).
    \end{itemize}
}
\solutionspace{120pt}

\tasknumber{3}%
\task{%
    Определите ток, протекающий через резистор $R_1$, разность потенциалов на нём (см.
    рис.)
    и выделяющуюся на нём мощность, если известны $r_1, r_2, \ele_1, \ele_2, R_1, R_2$.

    \begin{tikzpicture}[circuit ee IEC, thick]
        \draw  (0, 0) -- ++(up:2)
                to [
                    battery={ very near start, rotate=0, info={$\ele_1, r_1 $}},
                    resistor={ midway, info=$R_1$},
                    battery={ very near end, rotate=0, info={$\ele_2, r_2 $}}
                ] ++(right:5)
                -- ++(down:2)
                to [resistor={info=$R_2$}] ++(left:5);
    \end{tikzpicture}
}
\answer{%
    Нетривиальных узлов нет, поэтому все законы Кирхгофа для узлов будут иметь вид
    $\eli-\eli=0$ и ничем нам не помогут.
    Впрочем, если бы мы обозначили токи на разных участках контура $\eli_1, \eli_2, \eli_3, \ldots$,
    то именно эти законы бы помогли понять, что все эти токи равны: $\eli_1 - \eli_2 = 0$ и т.д.
    Так что запишем закон Кирхгофа для единственного замкнутого контура:

    \begin{tikzpicture}[circuit ee IEC, thick]
        \draw  (0, 0) -- ++(up:2)
                to [
                    battery={ very near start, rotate=0, info={$\ele_1, r_1 $}},
                    resistor={ midway, info=$R_1$},
                    battery={ very near end, rotate=0, info={$\ele_2, r_2 $}}
                ] ++(right:5)
                -- ++(down:2)
                to [resistor={info=$R_2$}, current direction={near end, info=$\eli$}] ++(left:5);
        \draw [-{Latex}] (2, 1.4) arc [start angle = 135, end angle = -160, radius = 0.6];
    \end{tikzpicture}

    \begin{align*}
        &- \ele_1 - \ele_2 = \eli R_1 + \eli r_2 + \eli R_2 + \eli r_1, \\
        &- \ele_1 - \ele_2 = \eli (R_1 + r_2 + R_2 + r_1), \\
        &\eli = \frac{- \ele_1 - \ele_2 }{ R_1 + r_2 + R_2 + r_1 }, \\
        &U_1 = \eli R_1 = \frac{- \ele_1 - \ele_2 }{ R_1 + r_2 + R_2 + r_1 } \cdot R_1, \\
        &P_1 = \eli^2 R_1 = \frac{\sqr{- \ele_1 - \ele_2 } R_1}{ \sqr{ R_1 + r_2 + R_2 + r_1 }}.
    \end{align*}

    Отметим, что ответ для тока $\eli$ меняет знак, если отметить его на рисунке в другую сторону.
    Поэтому критично важно указывать на рисунке направление тока, иначе невозможно утверждать, что ответ верный.
    А вот выбор направления контура — не повлияет на ответ, но для проверки корректности записи законо Кирхгофа,
    там тоже необходимо направление.
}

\variantsplitter

\addpersonalvariant{Алина Полканова}

\tasknumber{1}%
\task{%
    Два резистора сопротивлениями $R_1=3R$ и $R_2=4R$ подключены последовательно к источнику напряжения.
    Определите, в каком резисторе выделяется большая тепловая мощность и во сколько раз?
}
\answer{%
    Подключены последовательно, поэтому  $\eli_1 = \eli_2 = \eli \implies \frac{P_2}{P_1} = \frac{\eli_2^2 R_2}{\eli_1^2 R_1} = \frac{\eli_2^2R_2}{\eli_1R_1} = \frac{R_2}{R_1} = \frac43$.
}
\solutionspace{120pt}

\tasknumber{2}%
\task{%
    Если батарею замкнуть на резистор сопротивлением $R_1$, то в цепи потечёт ток $\eli_1$,
    а если на другой $R_2$ — то $\eli_2$.
    Определите:
    \begin{itemize}
        \item ЭДС батареи,
        \item внутреннее сопротивление батареи,
        \item ток короткого замыкания.
    \end{itemize}
}
\answer{%
    Запишем закон Ома для полной цепи 2 раза для обоих способов подключения (с $R_1$ и с $R_2$),
    а короткое замыкание рассмотрим позже.
    Отметим, что для такой простой схемы он совпадает
    с законом Кирхгофа.
    Получим систему из 2 уравнений и 2 неизвестных, решим в удобном порядке,
    ибо нам всё равно понадобятся обе.

    \begin{align*}
        &\begin{cases}
            \ele = \eli_1(R_1 + r), \\
            \ele = \eli_2(R_2 + r); \\
        \end{cases} \\
        &\eli_1(R_1 + r) = \eli_2(R_2 + r), \\
        &\eli_1 R_1 + \eli_1r = \eli_2 R_2 + \eli_2r, \\
        &\eli_1 R_1 - \eli_2 R_2 = - \eli_1r  + \eli_2r = (\eli_2 - \eli_1)r, \\
        r &= \frac{\eli_1 R_1 - \eli_2 R_2}{\eli_2 - \eli_1}
            \equiv \frac{\eli_2 R_2 - \eli_1 R_1}{\eli_1 - \eli_2}, \\
        \ele &= \eli_1(R_1 + r)
            = \eli_1\cbr{R_1 + \frac{\eli_1 R_1 - \eli_2 R_2}{\eli_2 - \eli_1}}
            = \eli_1 \cdot \frac{R_1\eli_2 - R_1\eli_1 + \eli_1 R_1 - \eli_2 R_2}{\eli_2 - \eli_1} \\
            &= \eli_1 \cdot \frac{R_1\eli_2 - \eli_2 R_2}{\eli_2 - \eli_1}
            = \frac{\eli_1 \eli_2 (R_1 - R_2)}{\eli_2 - \eli_1}
            \equiv \frac{\eli_1 \eli_2 (R_2 - R_1)}{\eli_1 - \eli_2}.
    \end{align*}

    Короткое замыкание происходит в ситуации, когда внешнее сопротивление равно 0
    (при этом цепь замкнута, хотя нагрузки и нет вовсе):
    $$
        \eli_\text{к.
        з.} = \frac \ele {0 + r} = \frac \ele r
            = \frac{\cfrac{\eli_1 \eli_2 (R_1 - R_2)}{\eli_2 - \eli_1}}{\cfrac{\eli_1 R_1 - \eli_2 R_2}{\eli_2 - \eli_1}}
            = \frac{\eli_1 \eli_2 (R_1 - R_2)}{\eli_1 R_1 - \eli_2 R_2}
            \equiv \frac{\eli_1 \eli_2 (R_2 - R_1)}{\eli_2 R_2 - \eli_1 R_1}.
    $$

    Важные пункты:
    \begin{itemize}
        \item В ответах есть только те величины, которые есть в условии
            (и ещё физические постоянные могут встретиться, но нам не понадобилось).
        \item Мы упростили выражения, который пошли в ответы (благо у нас даже получилось:
            приведение к общему знаменателю укоротило ответ).
            Надо доделывать.
        \item Всё ответы симметричны относительно замены резисторов 1 и 2 (ведь при этом изменятся и токи).
    \end{itemize}
}
\solutionspace{120pt}

\tasknumber{3}%
\task{%
    Определите ток, протекающий через резистор $R_1$, разность потенциалов на нём (см.
    рис.)
    и выделяющуюся на нём мощность, если известны $r_1, r_2, \ele_1, \ele_2, R_1, R_2$.

    \begin{tikzpicture}[circuit ee IEC, thick]
        \draw  (0, 0) -- ++(up:2)
                to [
                    battery={ very near start, rotate=0, info={$\ele_1, r_1 $}},
                    resistor={ midway, info=$R_1$},
                    battery={ very near end, rotate=-180, info={$\ele_2, r_2 $}}
                ] ++(right:5)
                -- ++(down:2)
                to [resistor={info=$R_2$}] ++(left:5);
    \end{tikzpicture}
}
\answer{%
    Нетривиальных узлов нет, поэтому все законы Кирхгофа для узлов будут иметь вид
    $\eli-\eli=0$ и ничем нам не помогут.
    Впрочем, если бы мы обозначили токи на разных участках контура $\eli_1, \eli_2, \eli_3, \ldots$,
    то именно эти законы бы помогли понять, что все эти токи равны: $\eli_1 - \eli_2 = 0$ и т.д.
    Так что запишем закон Кирхгофа для единственного замкнутого контура:

    \begin{tikzpicture}[circuit ee IEC, thick]
        \draw  (0, 0) -- ++(up:2)
                to [
                    battery={ very near start, rotate=0, info={$\ele_1, r_1 $}},
                    resistor={ midway, info=$R_1$},
                    battery={ very near end, rotate=-180, info={$\ele_2, r_2 $}}
                ] ++(right:5)
                -- ++(down:2)
                to [resistor={info=$R_2$}, current direction={near end, info=$\eli$}] ++(left:5);
        \draw [-{Latex}] (2, 1.4) arc [start angle = 135, end angle = -160, radius = 0.6];
    \end{tikzpicture}

    \begin{align*}
        &- \ele_1 +  \ele_2 = \eli R_1 + \eli r_2 + \eli R_2 + \eli r_1, \\
        &- \ele_1 +  \ele_2 = \eli (R_1 + r_2 + R_2 + r_1), \\
        &\eli = \frac{- \ele_1 +  \ele_2 }{ R_1 + r_2 + R_2 + r_1 }, \\
        &U_1 = \eli R_1 = \frac{- \ele_1 +  \ele_2 }{ R_1 + r_2 + R_2 + r_1 } \cdot R_1, \\
        &P_1 = \eli^2 R_1 = \frac{\sqr{- \ele_1 +  \ele_2 } R_1}{ \sqr{ R_1 + r_2 + R_2 + r_1 }}.
    \end{align*}

    Отметим, что ответ для тока $\eli$ меняет знак, если отметить его на рисунке в другую сторону.
    Поэтому критично важно указывать на рисунке направление тока, иначе невозможно утверждать, что ответ верный.
    А вот выбор направления контура — не повлияет на ответ, но для проверки корректности записи законо Кирхгофа,
    там тоже необходимо направление.
}

\variantsplitter

\addpersonalvariant{Сергей Пономарёв}

\tasknumber{1}%
\task{%
    Два резистора сопротивлениями $R_1=5R$ и $R_2=6R$ подключены последовательно к источнику напряжения.
    Определите, в каком резисторе выделяется большая тепловая мощность и во сколько раз?
}
\answer{%
    Подключены последовательно, поэтому  $\eli_1 = \eli_2 = \eli \implies \frac{P_2}{P_1} = \frac{\eli_2^2 R_2}{\eli_1^2 R_1} = \frac{\eli_2^2R_2}{\eli_1R_1} = \frac{R_2}{R_1} = \frac65$.
}
\solutionspace{120pt}

\tasknumber{2}%
\task{%
    Если батарею замкнуть на резистор сопротивлением $R_1$, то в цепи потечёт ток $\eli_1$,
    а если на другой $R_2$ — то $\eli_2$.
    Определите:
    \begin{itemize}
        \item ЭДС батареи,
        \item внутреннее сопротивление батареи,
        \item ток короткого замыкания.
    \end{itemize}
}
\answer{%
    Запишем закон Ома для полной цепи 2 раза для обоих способов подключения (с $R_1$ и с $R_2$),
    а короткое замыкание рассмотрим позже.
    Отметим, что для такой простой схемы он совпадает
    с законом Кирхгофа.
    Получим систему из 2 уравнений и 2 неизвестных, решим в удобном порядке,
    ибо нам всё равно понадобятся обе.

    \begin{align*}
        &\begin{cases}
            \ele = \eli_1(R_1 + r), \\
            \ele = \eli_2(R_2 + r); \\
        \end{cases} \\
        &\eli_1(R_1 + r) = \eli_2(R_2 + r), \\
        &\eli_1 R_1 + \eli_1r = \eli_2 R_2 + \eli_2r, \\
        &\eli_1 R_1 - \eli_2 R_2 = - \eli_1r  + \eli_2r = (\eli_2 - \eli_1)r, \\
        r &= \frac{\eli_1 R_1 - \eli_2 R_2}{\eli_2 - \eli_1}
            \equiv \frac{\eli_2 R_2 - \eli_1 R_1}{\eli_1 - \eli_2}, \\
        \ele &= \eli_1(R_1 + r)
            = \eli_1\cbr{R_1 + \frac{\eli_1 R_1 - \eli_2 R_2}{\eli_2 - \eli_1}}
            = \eli_1 \cdot \frac{R_1\eli_2 - R_1\eli_1 + \eli_1 R_1 - \eli_2 R_2}{\eli_2 - \eli_1} \\
            &= \eli_1 \cdot \frac{R_1\eli_2 - \eli_2 R_2}{\eli_2 - \eli_1}
            = \frac{\eli_1 \eli_2 (R_1 - R_2)}{\eli_2 - \eli_1}
            \equiv \frac{\eli_1 \eli_2 (R_2 - R_1)}{\eli_1 - \eli_2}.
    \end{align*}

    Короткое замыкание происходит в ситуации, когда внешнее сопротивление равно 0
    (при этом цепь замкнута, хотя нагрузки и нет вовсе):
    $$
        \eli_\text{к.
        з.} = \frac \ele {0 + r} = \frac \ele r
            = \frac{\cfrac{\eli_1 \eli_2 (R_1 - R_2)}{\eli_2 - \eli_1}}{\cfrac{\eli_1 R_1 - \eli_2 R_2}{\eli_2 - \eli_1}}
            = \frac{\eli_1 \eli_2 (R_1 - R_2)}{\eli_1 R_1 - \eli_2 R_2}
            \equiv \frac{\eli_1 \eli_2 (R_2 - R_1)}{\eli_2 R_2 - \eli_1 R_1}.
    $$

    Важные пункты:
    \begin{itemize}
        \item В ответах есть только те величины, которые есть в условии
            (и ещё физические постоянные могут встретиться, но нам не понадобилось).
        \item Мы упростили выражения, который пошли в ответы (благо у нас даже получилось:
            приведение к общему знаменателю укоротило ответ).
            Надо доделывать.
        \item Всё ответы симметричны относительно замены резисторов 1 и 2 (ведь при этом изменятся и токи).
    \end{itemize}
}
\solutionspace{120pt}

\tasknumber{3}%
\task{%
    Определите ток, протекающий через резистор $R_2$, разность потенциалов на нём (см.
    рис.)
    и выделяющуюся на нём мощность, если известны $r_1, r_2, \ele_1, \ele_2, R_1, R_2$.

    \begin{tikzpicture}[circuit ee IEC, thick]
        \draw  (0, 0) -- ++(up:2)
                to [
                    battery={ very near start, rotate=-180, info={$\ele_1, r_1 $}},
                    resistor={ midway, info=$R_1$},
                    battery={ very near end, rotate=-180, info={$\ele_2, r_2 $}}
                ] ++(right:5)
                -- ++(down:2)
                to [resistor={info=$R_2$}] ++(left:5);
    \end{tikzpicture}
}
\answer{%
    Нетривиальных узлов нет, поэтому все законы Кирхгофа для узлов будут иметь вид
    $\eli-\eli=0$ и ничем нам не помогут.
    Впрочем, если бы мы обозначили токи на разных участках контура $\eli_1, \eli_2, \eli_3, \ldots$,
    то именно эти законы бы помогли понять, что все эти токи равны: $\eli_1 - \eli_2 = 0$ и т.д.
    Так что запишем закон Кирхгофа для единственного замкнутого контура:

    \begin{tikzpicture}[circuit ee IEC, thick]
        \draw  (0, 0) -- ++(up:2)
                to [
                    battery={ very near start, rotate=-180, info={$\ele_1, r_1 $}},
                    resistor={ midway, info=$R_1$},
                    battery={ very near end, rotate=-180, info={$\ele_2, r_2 $}}
                ] ++(right:5)
                -- ++(down:2)
                to [resistor={info=$R_2$}, current direction={near end, info=$\eli$}] ++(left:5);
        \draw [-{Latex}] (2, 1.4) arc [start angle = 135, end angle = -160, radius = 0.6];
    \end{tikzpicture}

    \begin{align*}
        & \ele_1 +  \ele_2 = \eli R_1 + \eli r_2 + \eli R_2 + \eli r_1, \\
        & \ele_1 +  \ele_2 = \eli (R_1 + r_2 + R_2 + r_1), \\
        &\eli = \frac{ \ele_1 +  \ele_2 }{ R_1 + r_2 + R_2 + r_1 }, \\
        &U_2 = \eli R_2 = \frac{ \ele_1 +  \ele_2 }{ R_1 + r_2 + R_2 + r_1 } \cdot R_2, \\
        &P_2 = \eli^2 R_2 = \frac{\sqr{ \ele_1 +  \ele_2 } R_2}{ \sqr{ R_1 + r_2 + R_2 + r_1 }}.
    \end{align*}

    Отметим, что ответ для тока $\eli$ меняет знак, если отметить его на рисунке в другую сторону.
    Поэтому критично важно указывать на рисунке направление тока, иначе невозможно утверждать, что ответ верный.
    А вот выбор направления контура — не повлияет на ответ, но для проверки корректности записи законо Кирхгофа,
    там тоже необходимо направление.
}

\variantsplitter

\addpersonalvariant{Егор Свистушкин}

\tasknumber{1}%
\task{%
    Два резистора сопротивлениями $R_1=3R$ и $R_2=4R$ подключены последовательно к источнику напряжения.
    Определите, в каком резисторе выделяется большая тепловая мощность и во сколько раз?
}
\answer{%
    Подключены последовательно, поэтому  $\eli_1 = \eli_2 = \eli \implies \frac{P_2}{P_1} = \frac{\eli_2^2 R_2}{\eli_1^2 R_1} = \frac{\eli_2^2R_2}{\eli_1R_1} = \frac{R_2}{R_1} = \frac43$.
}
\solutionspace{120pt}

\tasknumber{2}%
\task{%
    Если батарею замкнуть на резистор сопротивлением $R_1$, то в цепи потечёт ток $\eli_1$,
    а если на другой $R_2$ — то $\eli_2$.
    Определите:
    \begin{itemize}
        \item ЭДС батареи,
        \item внутреннее сопротивление батареи,
        \item ток короткого замыкания.
    \end{itemize}
}
\answer{%
    Запишем закон Ома для полной цепи 2 раза для обоих способов подключения (с $R_1$ и с $R_2$),
    а короткое замыкание рассмотрим позже.
    Отметим, что для такой простой схемы он совпадает
    с законом Кирхгофа.
    Получим систему из 2 уравнений и 2 неизвестных, решим в удобном порядке,
    ибо нам всё равно понадобятся обе.

    \begin{align*}
        &\begin{cases}
            \ele = \eli_1(R_1 + r), \\
            \ele = \eli_2(R_2 + r); \\
        \end{cases} \\
        &\eli_1(R_1 + r) = \eli_2(R_2 + r), \\
        &\eli_1 R_1 + \eli_1r = \eli_2 R_2 + \eli_2r, \\
        &\eli_1 R_1 - \eli_2 R_2 = - \eli_1r  + \eli_2r = (\eli_2 - \eli_1)r, \\
        r &= \frac{\eli_1 R_1 - \eli_2 R_2}{\eli_2 - \eli_1}
            \equiv \frac{\eli_2 R_2 - \eli_1 R_1}{\eli_1 - \eli_2}, \\
        \ele &= \eli_1(R_1 + r)
            = \eli_1\cbr{R_1 + \frac{\eli_1 R_1 - \eli_2 R_2}{\eli_2 - \eli_1}}
            = \eli_1 \cdot \frac{R_1\eli_2 - R_1\eli_1 + \eli_1 R_1 - \eli_2 R_2}{\eli_2 - \eli_1} \\
            &= \eli_1 \cdot \frac{R_1\eli_2 - \eli_2 R_2}{\eli_2 - \eli_1}
            = \frac{\eli_1 \eli_2 (R_1 - R_2)}{\eli_2 - \eli_1}
            \equiv \frac{\eli_1 \eli_2 (R_2 - R_1)}{\eli_1 - \eli_2}.
    \end{align*}

    Короткое замыкание происходит в ситуации, когда внешнее сопротивление равно 0
    (при этом цепь замкнута, хотя нагрузки и нет вовсе):
    $$
        \eli_\text{к.
        з.} = \frac \ele {0 + r} = \frac \ele r
            = \frac{\cfrac{\eli_1 \eli_2 (R_1 - R_2)}{\eli_2 - \eli_1}}{\cfrac{\eli_1 R_1 - \eli_2 R_2}{\eli_2 - \eli_1}}
            = \frac{\eli_1 \eli_2 (R_1 - R_2)}{\eli_1 R_1 - \eli_2 R_2}
            \equiv \frac{\eli_1 \eli_2 (R_2 - R_1)}{\eli_2 R_2 - \eli_1 R_1}.
    $$

    Важные пункты:
    \begin{itemize}
        \item В ответах есть только те величины, которые есть в условии
            (и ещё физические постоянные могут встретиться, но нам не понадобилось).
        \item Мы упростили выражения, который пошли в ответы (благо у нас даже получилось:
            приведение к общему знаменателю укоротило ответ).
            Надо доделывать.
        \item Всё ответы симметричны относительно замены резисторов 1 и 2 (ведь при этом изменятся и токи).
    \end{itemize}
}
\solutionspace{120pt}

\tasknumber{3}%
\task{%
    Определите ток, протекающий через резистор $R_2$, разность потенциалов на нём (см.
    рис.)
    и выделяющуюся на нём мощность, если известны $r_1, r_2, \ele_1, \ele_2, R_1, R_2$.

    \begin{tikzpicture}[circuit ee IEC, thick]
        \draw  (0, 0) -- ++(up:2)
                to [
                    battery={ very near start, rotate=-180, info={$\ele_1, r_1 $}},
                    resistor={ midway, info=$R_1$},
                    battery={ very near end, rotate=-180, info={$\ele_2, r_2 $}}
                ] ++(right:5)
                -- ++(down:2)
                to [resistor={info=$R_2$}] ++(left:5);
    \end{tikzpicture}
}
\answer{%
    Нетривиальных узлов нет, поэтому все законы Кирхгофа для узлов будут иметь вид
    $\eli-\eli=0$ и ничем нам не помогут.
    Впрочем, если бы мы обозначили токи на разных участках контура $\eli_1, \eli_2, \eli_3, \ldots$,
    то именно эти законы бы помогли понять, что все эти токи равны: $\eli_1 - \eli_2 = 0$ и т.д.
    Так что запишем закон Кирхгофа для единственного замкнутого контура:

    \begin{tikzpicture}[circuit ee IEC, thick]
        \draw  (0, 0) -- ++(up:2)
                to [
                    battery={ very near start, rotate=-180, info={$\ele_1, r_1 $}},
                    resistor={ midway, info=$R_1$},
                    battery={ very near end, rotate=-180, info={$\ele_2, r_2 $}}
                ] ++(right:5)
                -- ++(down:2)
                to [resistor={info=$R_2$}, current direction={near end, info=$\eli$}] ++(left:5);
        \draw [-{Latex}] (2, 1.4) arc [start angle = 135, end angle = -160, radius = 0.6];
    \end{tikzpicture}

    \begin{align*}
        & \ele_1 +  \ele_2 = \eli R_1 + \eli r_2 + \eli R_2 + \eli r_1, \\
        & \ele_1 +  \ele_2 = \eli (R_1 + r_2 + R_2 + r_1), \\
        &\eli = \frac{ \ele_1 +  \ele_2 }{ R_1 + r_2 + R_2 + r_1 }, \\
        &U_2 = \eli R_2 = \frac{ \ele_1 +  \ele_2 }{ R_1 + r_2 + R_2 + r_1 } \cdot R_2, \\
        &P_2 = \eli^2 R_2 = \frac{\sqr{ \ele_1 +  \ele_2 } R_2}{ \sqr{ R_1 + r_2 + R_2 + r_1 }}.
    \end{align*}

    Отметим, что ответ для тока $\eli$ меняет знак, если отметить его на рисунке в другую сторону.
    Поэтому критично важно указывать на рисунке направление тока, иначе невозможно утверждать, что ответ верный.
    А вот выбор направления контура — не повлияет на ответ, но для проверки корректности записи законо Кирхгофа,
    там тоже необходимо направление.
}

\variantsplitter

\addpersonalvariant{Дмитрий Соколов}

\tasknumber{1}%
\task{%
    Два резистора сопротивлениями $R_1=5R$ и $R_2=8R$ подключены последовательно к источнику напряжения.
    Определите, в каком резисторе выделяется большая тепловая мощность и во сколько раз?
}
\answer{%
    Подключены последовательно, поэтому  $\eli_1 = \eli_2 = \eli \implies \frac{P_2}{P_1} = \frac{\eli_2^2 R_2}{\eli_1^2 R_1} = \frac{\eli_2^2R_2}{\eli_1R_1} = \frac{R_2}{R_1} = \frac85$.
}
\solutionspace{120pt}

\tasknumber{2}%
\task{%
    Если батарею замкнуть на резистор сопротивлением $R_1$, то в цепи потечёт ток $\eli_1$,
    а если на другой $R_2$ — то $\eli_2$.
    Определите:
    \begin{itemize}
        \item ЭДС батареи,
        \item внутреннее сопротивление батареи,
        \item ток короткого замыкания.
    \end{itemize}
}
\answer{%
    Запишем закон Ома для полной цепи 2 раза для обоих способов подключения (с $R_1$ и с $R_2$),
    а короткое замыкание рассмотрим позже.
    Отметим, что для такой простой схемы он совпадает
    с законом Кирхгофа.
    Получим систему из 2 уравнений и 2 неизвестных, решим в удобном порядке,
    ибо нам всё равно понадобятся обе.

    \begin{align*}
        &\begin{cases}
            \ele = \eli_1(R_1 + r), \\
            \ele = \eli_2(R_2 + r); \\
        \end{cases} \\
        &\eli_1(R_1 + r) = \eli_2(R_2 + r), \\
        &\eli_1 R_1 + \eli_1r = \eli_2 R_2 + \eli_2r, \\
        &\eli_1 R_1 - \eli_2 R_2 = - \eli_1r  + \eli_2r = (\eli_2 - \eli_1)r, \\
        r &= \frac{\eli_1 R_1 - \eli_2 R_2}{\eli_2 - \eli_1}
            \equiv \frac{\eli_2 R_2 - \eli_1 R_1}{\eli_1 - \eli_2}, \\
        \ele &= \eli_1(R_1 + r)
            = \eli_1\cbr{R_1 + \frac{\eli_1 R_1 - \eli_2 R_2}{\eli_2 - \eli_1}}
            = \eli_1 \cdot \frac{R_1\eli_2 - R_1\eli_1 + \eli_1 R_1 - \eli_2 R_2}{\eli_2 - \eli_1} \\
            &= \eli_1 \cdot \frac{R_1\eli_2 - \eli_2 R_2}{\eli_2 - \eli_1}
            = \frac{\eli_1 \eli_2 (R_1 - R_2)}{\eli_2 - \eli_1}
            \equiv \frac{\eli_1 \eli_2 (R_2 - R_1)}{\eli_1 - \eli_2}.
    \end{align*}

    Короткое замыкание происходит в ситуации, когда внешнее сопротивление равно 0
    (при этом цепь замкнута, хотя нагрузки и нет вовсе):
    $$
        \eli_\text{к.
        з.} = \frac \ele {0 + r} = \frac \ele r
            = \frac{\cfrac{\eli_1 \eli_2 (R_1 - R_2)}{\eli_2 - \eli_1}}{\cfrac{\eli_1 R_1 - \eli_2 R_2}{\eli_2 - \eli_1}}
            = \frac{\eli_1 \eli_2 (R_1 - R_2)}{\eli_1 R_1 - \eli_2 R_2}
            \equiv \frac{\eli_1 \eli_2 (R_2 - R_1)}{\eli_2 R_2 - \eli_1 R_1}.
    $$

    Важные пункты:
    \begin{itemize}
        \item В ответах есть только те величины, которые есть в условии
            (и ещё физические постоянные могут встретиться, но нам не понадобилось).
        \item Мы упростили выражения, который пошли в ответы (благо у нас даже получилось:
            приведение к общему знаменателю укоротило ответ).
            Надо доделывать.
        \item Всё ответы симметричны относительно замены резисторов 1 и 2 (ведь при этом изменятся и токи).
    \end{itemize}
}
\solutionspace{120pt}

\tasknumber{3}%
\task{%
    Определите ток, протекающий через резистор $R_2$, разность потенциалов на нём (см.
    рис.)
    и выделяющуюся на нём мощность, если известны $r_1, r_2, \ele_1, \ele_2, R_1, R_2$.

    \begin{tikzpicture}[circuit ee IEC, thick]
        \draw  (0, 0) -- ++(up:2)
                to [
                    battery={ very near start, rotate=0, info={$\ele_1, r_1 $}},
                    resistor={ midway, info=$R_1$},
                    battery={ very near end, rotate=0, info={$\ele_2, r_2 $}}
                ] ++(right:5)
                -- ++(down:2)
                to [resistor={info=$R_2$}] ++(left:5);
    \end{tikzpicture}
}
\answer{%
    Нетривиальных узлов нет, поэтому все законы Кирхгофа для узлов будут иметь вид
    $\eli-\eli=0$ и ничем нам не помогут.
    Впрочем, если бы мы обозначили токи на разных участках контура $\eli_1, \eli_2, \eli_3, \ldots$,
    то именно эти законы бы помогли понять, что все эти токи равны: $\eli_1 - \eli_2 = 0$ и т.д.
    Так что запишем закон Кирхгофа для единственного замкнутого контура:

    \begin{tikzpicture}[circuit ee IEC, thick]
        \draw  (0, 0) -- ++(up:2)
                to [
                    battery={ very near start, rotate=0, info={$\ele_1, r_1 $}},
                    resistor={ midway, info=$R_1$},
                    battery={ very near end, rotate=0, info={$\ele_2, r_2 $}}
                ] ++(right:5)
                -- ++(down:2)
                to [resistor={info=$R_2$}, current direction={near end, info=$\eli$}] ++(left:5);
        \draw [-{Latex}] (2, 1.4) arc [start angle = 135, end angle = -160, radius = 0.6];
    \end{tikzpicture}

    \begin{align*}
        &- \ele_1 - \ele_2 = \eli R_1 + \eli r_2 + \eli R_2 + \eli r_1, \\
        &- \ele_1 - \ele_2 = \eli (R_1 + r_2 + R_2 + r_1), \\
        &\eli = \frac{- \ele_1 - \ele_2 }{ R_1 + r_2 + R_2 + r_1 }, \\
        &U_2 = \eli R_2 = \frac{- \ele_1 - \ele_2 }{ R_1 + r_2 + R_2 + r_1 } \cdot R_2, \\
        &P_2 = \eli^2 R_2 = \frac{\sqr{- \ele_1 - \ele_2 } R_2}{ \sqr{ R_1 + r_2 + R_2 + r_1 }}.
    \end{align*}

    Отметим, что ответ для тока $\eli$ меняет знак, если отметить его на рисунке в другую сторону.
    Поэтому критично важно указывать на рисунке направление тока, иначе невозможно утверждать, что ответ верный.
    А вот выбор направления контура — не повлияет на ответ, но для проверки корректности записи законо Кирхгофа,
    там тоже необходимо направление.
}

\variantsplitter

\addpersonalvariant{Арсений Трофимов}

\tasknumber{1}%
\task{%
    Два резистора сопротивлениями $R_1=7R$ и $R_2=2R$ подключены последовательно к источнику напряжения.
    Определите, в каком резисторе выделяется большая тепловая мощность и во сколько раз?
}
\answer{%
    Подключены последовательно, поэтому  $\eli_1 = \eli_2 = \eli \implies \frac{P_2}{P_1} = \frac{\eli_2^2 R_2}{\eli_1^2 R_1} = \frac{\eli_2^2R_2}{\eli_1R_1} = \frac{R_2}{R_1} = \frac27$.
}
\solutionspace{120pt}

\tasknumber{2}%
\task{%
    Если батарею замкнуть на резистор сопротивлением $R_1$, то в цепи потечёт ток $\eli_1$,
    а если на другой $R_2$ — то $\eli_2$.
    Определите:
    \begin{itemize}
        \item ЭДС батареи,
        \item внутреннее сопротивление батареи,
        \item ток короткого замыкания.
    \end{itemize}
}
\answer{%
    Запишем закон Ома для полной цепи 2 раза для обоих способов подключения (с $R_1$ и с $R_2$),
    а короткое замыкание рассмотрим позже.
    Отметим, что для такой простой схемы он совпадает
    с законом Кирхгофа.
    Получим систему из 2 уравнений и 2 неизвестных, решим в удобном порядке,
    ибо нам всё равно понадобятся обе.

    \begin{align*}
        &\begin{cases}
            \ele = \eli_1(R_1 + r), \\
            \ele = \eli_2(R_2 + r); \\
        \end{cases} \\
        &\eli_1(R_1 + r) = \eli_2(R_2 + r), \\
        &\eli_1 R_1 + \eli_1r = \eli_2 R_2 + \eli_2r, \\
        &\eli_1 R_1 - \eli_2 R_2 = - \eli_1r  + \eli_2r = (\eli_2 - \eli_1)r, \\
        r &= \frac{\eli_1 R_1 - \eli_2 R_2}{\eli_2 - \eli_1}
            \equiv \frac{\eli_2 R_2 - \eli_1 R_1}{\eli_1 - \eli_2}, \\
        \ele &= \eli_1(R_1 + r)
            = \eli_1\cbr{R_1 + \frac{\eli_1 R_1 - \eli_2 R_2}{\eli_2 - \eli_1}}
            = \eli_1 \cdot \frac{R_1\eli_2 - R_1\eli_1 + \eli_1 R_1 - \eli_2 R_2}{\eli_2 - \eli_1} \\
            &= \eli_1 \cdot \frac{R_1\eli_2 - \eli_2 R_2}{\eli_2 - \eli_1}
            = \frac{\eli_1 \eli_2 (R_1 - R_2)}{\eli_2 - \eli_1}
            \equiv \frac{\eli_1 \eli_2 (R_2 - R_1)}{\eli_1 - \eli_2}.
    \end{align*}

    Короткое замыкание происходит в ситуации, когда внешнее сопротивление равно 0
    (при этом цепь замкнута, хотя нагрузки и нет вовсе):
    $$
        \eli_\text{к.
        з.} = \frac \ele {0 + r} = \frac \ele r
            = \frac{\cfrac{\eli_1 \eli_2 (R_1 - R_2)}{\eli_2 - \eli_1}}{\cfrac{\eli_1 R_1 - \eli_2 R_2}{\eli_2 - \eli_1}}
            = \frac{\eli_1 \eli_2 (R_1 - R_2)}{\eli_1 R_1 - \eli_2 R_2}
            \equiv \frac{\eli_1 \eli_2 (R_2 - R_1)}{\eli_2 R_2 - \eli_1 R_1}.
    $$

    Важные пункты:
    \begin{itemize}
        \item В ответах есть только те величины, которые есть в условии
            (и ещё физические постоянные могут встретиться, но нам не понадобилось).
        \item Мы упростили выражения, который пошли в ответы (благо у нас даже получилось:
            приведение к общему знаменателю укоротило ответ).
            Надо доделывать.
        \item Всё ответы симметричны относительно замены резисторов 1 и 2 (ведь при этом изменятся и токи).
    \end{itemize}
}
\solutionspace{120pt}

\tasknumber{3}%
\task{%
    Определите ток, протекающий через резистор $R_1$, разность потенциалов на нём (см.
    рис.)
    и выделяющуюся на нём мощность, если известны $r_1, r_2, \ele_1, \ele_2, R_1, R_2$.

    \begin{tikzpicture}[circuit ee IEC, thick]
        \draw  (0, 0) -- ++(up:2)
                to [
                    battery={ very near start, rotate=0, info={$\ele_1, r_1 $}},
                    resistor={ midway, info=$R_1$},
                    battery={ very near end, rotate=0, info={$\ele_2, r_2 $}}
                ] ++(right:5)
                -- ++(down:2)
                to [resistor={info=$R_2$}] ++(left:5);
    \end{tikzpicture}
}
\answer{%
    Нетривиальных узлов нет, поэтому все законы Кирхгофа для узлов будут иметь вид
    $\eli-\eli=0$ и ничем нам не помогут.
    Впрочем, если бы мы обозначили токи на разных участках контура $\eli_1, \eli_2, \eli_3, \ldots$,
    то именно эти законы бы помогли понять, что все эти токи равны: $\eli_1 - \eli_2 = 0$ и т.д.
    Так что запишем закон Кирхгофа для единственного замкнутого контура:

    \begin{tikzpicture}[circuit ee IEC, thick]
        \draw  (0, 0) -- ++(up:2)
                to [
                    battery={ very near start, rotate=0, info={$\ele_1, r_1 $}},
                    resistor={ midway, info=$R_1$},
                    battery={ very near end, rotate=0, info={$\ele_2, r_2 $}}
                ] ++(right:5)
                -- ++(down:2)
                to [resistor={info=$R_2$}, current direction={near end, info=$\eli$}] ++(left:5);
        \draw [-{Latex}] (2, 1.4) arc [start angle = 135, end angle = -160, radius = 0.6];
    \end{tikzpicture}

    \begin{align*}
        &- \ele_1 - \ele_2 = \eli R_1 + \eli r_2 + \eli R_2 + \eli r_1, \\
        &- \ele_1 - \ele_2 = \eli (R_1 + r_2 + R_2 + r_1), \\
        &\eli = \frac{- \ele_1 - \ele_2 }{ R_1 + r_2 + R_2 + r_1 }, \\
        &U_1 = \eli R_1 = \frac{- \ele_1 - \ele_2 }{ R_1 + r_2 + R_2 + r_1 } \cdot R_1, \\
        &P_1 = \eli^2 R_1 = \frac{\sqr{- \ele_1 - \ele_2 } R_1}{ \sqr{ R_1 + r_2 + R_2 + r_1 }}.
    \end{align*}

    Отметим, что ответ для тока $\eli$ меняет знак, если отметить его на рисунке в другую сторону.
    Поэтому критично важно указывать на рисунке направление тока, иначе невозможно утверждать, что ответ верный.
    А вот выбор направления контура — не повлияет на ответ, но для проверки корректности записи законо Кирхгофа,
    там тоже необходимо направление.
}
% autogenerated
