\setdate{12~марта~2021}
\setclass{10«АБ»}

\addpersonalvariant{Михаил Бурмистров}

\tasknumber{1}%
\task{%
    Определите КПД цикла 12341, рабочим телом которого является идеальный одноатомный газ, если
    12 — изобарическое расширение газа в шесть раз,
    23 — изохорическое охлаждение газа, при котором температура уменьшается в три раза,
    34 — изобара, 41 — изохора.
    % Для этого:
    % \begin{enumerate}
    %     \item сделайте рисунок в $PV$-координатах,
    %     \item выберите удобные обозначения, чтобы не запутаться в множестве температур, давлений и объёмов,
    %     \item вычислите необходимые соотнощения между температурами, давлениями и объёмами
    %     (некоторые сразу видны по рисунку, некоторые — надо считать),
    %     \item определите для каждого участка поглощается или отдаётся тепло (и сколько именно:
    %     потребуется первое начало термодинамики, отдельный расчёт работ на участках через площади фигур и изменений внутренней энергии),
    %     \item вычислите полную работу газа в цикле,
    %     \item подставьте всё в формулу для КПД, упростите и доведите до ответа.
    % \end{enumerate}
    Определите КПД цикла Карно, температура нагревателя которого равна максимальной температуре в цикле 12341, а холодильника — минимальной.
    Ответы в обоих случаях оставьте точными в виде нескоратимой дроби, никаких округлений.
}
\answer{%
    \begin{align*}
    A_{12} &> 0, \Delta U_{12} > 0, \implies Q_{12} = A_{12} + \Delta U_{12} > 0, \\
    A_{23} &= 0, \Delta U_{23} < 0, \implies Q_{23} = A_{23} + \Delta U_{23} < 0, \\
    A_{34} &< 0, \Delta U_{34} < 0, \implies Q_{34} = A_{34} + \Delta U_{34} < 0, \\
    A_{41} &= 0, \Delta U_{41} > 0, \implies Q_{41} = A_{41} + \Delta U_{41} > 0.
    \\
    P_1V_1 &= \nu R T_1, P_2V_2 = \nu R T_2, P_3V_3 = \nu R T_3, P_4V_4 = \nu R T_4 \text{ — уравнения состояния идеального газа}, \\
    &\text{Пусть $P_0$, $V_0$, $T_0$ — давление, объём и температура в точке 4 (минимальные во всём цикле):} \\
    P_1 &= P_2, P_3 = P_4 = P_0, V_1 = V_4 = V_0, V_2 = V_3 = 6 V_1 = 6 V_0,, \text{остальные соотношения между объёмами и давлениями не даны, нужно считать} \\
    T_3 &= \frac{T_2}3 \text{(по условию)} \implies \frac{P_2}{P_3} = \frac{P_2 V_2}{P_3 V_3}= \frac{\nu R T_2}{\nu R T_3} = \frac{T_2}{T_3} = 3 \implies P_1 = P_2 = 3 P_0 \\
    A_\text{цикл} &= (3P_0 - P_0)(6V_0 - V_0) = 10P_0V_0, \\
    A_{12} &= 3P_0 \cdot (6V_0 - V_0) = 15P_0V_0, \\
    \Delta U_{12} &= \frac 32 \nu R T_2 - \frac 32 \nu R T_1 = \frac 32 P_2 V_2 - \frac 32 P_1 V_1 = \frac 32 \cdot 3 P_0 \cdot 6 V_0 -  \frac 32 \cdot 3 P_0 \cdot V_0 = \frac 32 \cdot 15 \cdot P_0V_0, \\
    \Delta U_{41} &= \frac 32 \nu R T_1 - \frac 32 \nu R T_4 = \frac 32 P_1 V_1 - \frac 32 P_4 V_4 = \frac 32 \cdot 3 P_0 V_0 - \frac 32 P_0 V_0 = \frac 32 \cdot 2 \cdot P_0V_0.
    \\
    \eta &= \frac{A_\text{цикл}}{Q_+} = \frac{A_\text{цикл}}{Q_{12} + Q_{41}}  = \frac{A_\text{цикл}}{A_{12} + \Delta U_{12} + A_{41} + \Delta U_{41}} =  \\
     &= \frac{10P_0V_0}{15P_0V_0 + \frac 32 \cdot 15 \cdot P_0V_0 + 0 + \frac 32 \cdot 2 \cdot P_0V_0} = \frac{10}{15 + \frac 32 \cdot 15 + \frac 32 \cdot 2} = \frac{20}{81} \approx 0{,}247.
     \\
    \eta_\text{Карно} &= 1 - \frac{T_\text{х}}{T_\text{н}} = 1 - \frac{T_\text{4}}{T_\text{2}} = 1 - \frac{\frac{P_4V_4}{\nu R}}{\frac{P_2V_2}{\nu R}} = 1 - \frac{P_4V_4}{P_2V_2} = 1 - \frac{P_0V_0}{3P_0 \cdot 6V_0} = 1 - \frac 1{3 \cdot 6}  = \frac{17}{18} \approx 0{,}944.
    \end{align*}
}
\solutionspace{360pt}

\tasknumber{2}%
\task{%
    Порция идеального одноатомного газа перешла из состояния 1 в состояние 2: $P_1 = 4\,\text{МПа}$, $V_1 = 3\,\text{л}$, $P_2 = 2{,}5\,\text{МПа}$, $V_2 = 6\,\text{л}$.
    Известно, что в $PV$-координатах график процесса 12 представляет собой отрезок прямой.
    Определите,
    \begin{itemize}
        \item какую работу при этом совершил газ,
        \item чему равно изменение внутренней энергии газа,
        \item сколько теплоты подвели к нему в этом процессе?
    \end{itemize}
    При решении обратите внимание на знаки искомых величин.
}
\answer{%
    \begin{align*}
    P_1V_1 &= \nu R T_1, P_2V_2 = \nu R T_2, \\
    \Delta U &= U_2-U_1 = \frac 32 \nu R T_2- \frac 32 \nu R T_1 = \frac 32 P_2 V_2 - \frac 32 P_1 V_1= \frac 32 \cdot \cbr{2{,}5\,\text{МПа} \cdot 6\,\text{л} - 4\,\text{МПа} \cdot 3\,\text{л}} = 4{,}50\,\text{кДж}.
    \\
    A_\text{газа} &= \frac{P_2 + P_1} 2 \cdot (V_2 - V_1) = \frac{2{,}5\,\text{МПа} + 4\,\text{МПа}} 2 \cdot (6\,\text{л} - 3\,\text{л}) = 9{,}75\,\text{кДж}, \\
    Q &= A_\text{газа} + \Delta U = \frac 32 (P_2 V_2 - P_1 V_1) + \frac{P_2 + P_1} 2 \cdot (V_2 - V_1) = 4{,}50\,\text{кДж} + 9{,}75\,\text{кДж} = 14{,}25\,\text{кДж}.
    \end{align*}
}
\solutionspace{150pt}

\tasknumber{3}%
\task{%
    Запишите формулы и рядом с каждой физичической величиной укажите её название и единицы измерения в СИ:
    \begin{enumerate}
        \item первое начало термодинамики,
        \item внутренняя энергия идеального одноатомного газа.
    \end{enumerate}
}

\variantsplitter

\addpersonalvariant{Ирина Ан}

\tasknumber{1}%
\task{%
    Определите КПД цикла 12341, рабочим телом которого является идеальный одноатомный газ, если
    12 — изобарическое расширение газа в пять раз,
    23 — изохорическое охлаждение газа, при котором температура уменьшается в два раза,
    34 — изобара, 41 — изохора.
    % Для этого:
    % \begin{enumerate}
    %     \item сделайте рисунок в $PV$-координатах,
    %     \item выберите удобные обозначения, чтобы не запутаться в множестве температур, давлений и объёмов,
    %     \item вычислите необходимые соотнощения между температурами, давлениями и объёмами
    %     (некоторые сразу видны по рисунку, некоторые — надо считать),
    %     \item определите для каждого участка поглощается или отдаётся тепло (и сколько именно:
    %     потребуется первое начало термодинамики, отдельный расчёт работ на участках через площади фигур и изменений внутренней энергии),
    %     \item вычислите полную работу газа в цикле,
    %     \item подставьте всё в формулу для КПД, упростите и доведите до ответа.
    % \end{enumerate}
    Определите КПД цикла Карно, температура нагревателя которого равна максимальной температуре в цикле 12341, а холодильника — минимальной.
    Ответы в обоих случаях оставьте точными в виде нескоратимой дроби, никаких округлений.
}
\answer{%
    \begin{align*}
    A_{12} &> 0, \Delta U_{12} > 0, \implies Q_{12} = A_{12} + \Delta U_{12} > 0, \\
    A_{23} &= 0, \Delta U_{23} < 0, \implies Q_{23} = A_{23} + \Delta U_{23} < 0, \\
    A_{34} &< 0, \Delta U_{34} < 0, \implies Q_{34} = A_{34} + \Delta U_{34} < 0, \\
    A_{41} &= 0, \Delta U_{41} > 0, \implies Q_{41} = A_{41} + \Delta U_{41} > 0.
    \\
    P_1V_1 &= \nu R T_1, P_2V_2 = \nu R T_2, P_3V_3 = \nu R T_3, P_4V_4 = \nu R T_4 \text{ — уравнения состояния идеального газа}, \\
    &\text{Пусть $P_0$, $V_0$, $T_0$ — давление, объём и температура в точке 4 (минимальные во всём цикле):} \\
    P_1 &= P_2, P_3 = P_4 = P_0, V_1 = V_4 = V_0, V_2 = V_3 = 5 V_1 = 5 V_0,, \text{остальные соотношения между объёмами и давлениями не даны, нужно считать} \\
    T_3 &= \frac{T_2}2 \text{(по условию)} \implies \frac{P_2}{P_3} = \frac{P_2 V_2}{P_3 V_3}= \frac{\nu R T_2}{\nu R T_3} = \frac{T_2}{T_3} = 2 \implies P_1 = P_2 = 2 P_0 \\
    A_\text{цикл} &= (2P_0 - P_0)(5V_0 - V_0) = 4P_0V_0, \\
    A_{12} &= 2P_0 \cdot (5V_0 - V_0) = 8P_0V_0, \\
    \Delta U_{12} &= \frac 32 \nu R T_2 - \frac 32 \nu R T_1 = \frac 32 P_2 V_2 - \frac 32 P_1 V_1 = \frac 32 \cdot 2 P_0 \cdot 5 V_0 -  \frac 32 \cdot 2 P_0 \cdot V_0 = \frac 32 \cdot 8 \cdot P_0V_0, \\
    \Delta U_{41} &= \frac 32 \nu R T_1 - \frac 32 \nu R T_4 = \frac 32 P_1 V_1 - \frac 32 P_4 V_4 = \frac 32 \cdot 2 P_0 V_0 - \frac 32 P_0 V_0 = \frac 32 \cdot 1 \cdot P_0V_0.
    \\
    \eta &= \frac{A_\text{цикл}}{Q_+} = \frac{A_\text{цикл}}{Q_{12} + Q_{41}}  = \frac{A_\text{цикл}}{A_{12} + \Delta U_{12} + A_{41} + \Delta U_{41}} =  \\
     &= \frac{4P_0V_0}{8P_0V_0 + \frac 32 \cdot 8 \cdot P_0V_0 + 0 + \frac 32 \cdot 1 \cdot P_0V_0} = \frac{4}{8 + \frac 32 \cdot 8 + \frac 32 \cdot 1} = \frac8{43} \approx 0{,}186.
     \\
    \eta_\text{Карно} &= 1 - \frac{T_\text{х}}{T_\text{н}} = 1 - \frac{T_\text{4}}{T_\text{2}} = 1 - \frac{\frac{P_4V_4}{\nu R}}{\frac{P_2V_2}{\nu R}} = 1 - \frac{P_4V_4}{P_2V_2} = 1 - \frac{P_0V_0}{2P_0 \cdot 5V_0} = 1 - \frac 1{2 \cdot 5}  = \frac9{10} \approx 0{,}900.
    \end{align*}
}
\solutionspace{360pt}

\tasknumber{2}%
\task{%
    Порция идеального одноатомного газа перешла из состояния 1 в состояние 2: $P_1 = 3\,\text{МПа}$, $V_1 = 5\,\text{л}$, $P_2 = 4{,}5\,\text{МПа}$, $V_2 = 2\,\text{л}$.
    Известно, что в $PV$-координатах график процесса 12 представляет собой отрезок прямой.
    Определите,
    \begin{itemize}
        \item какую работу при этом совершил газ,
        \item чему равно изменение внутренней энергии газа,
        \item сколько теплоты подвели к нему в этом процессе?
    \end{itemize}
    При решении обратите внимание на знаки искомых величин.
}
\answer{%
    \begin{align*}
    P_1V_1 &= \nu R T_1, P_2V_2 = \nu R T_2, \\
    \Delta U &= U_2-U_1 = \frac 32 \nu R T_2- \frac 32 \nu R T_1 = \frac 32 P_2 V_2 - \frac 32 P_1 V_1= \frac 32 \cdot \cbr{4{,}5\,\text{МПа} \cdot 2\,\text{л} - 3\,\text{МПа} \cdot 5\,\text{л}} = -9{,}000\,\text{кДж}.
    \\
    A_\text{газа} &= \frac{P_2 + P_1} 2 \cdot (V_2 - V_1) = \frac{4{,}5\,\text{МПа} + 3\,\text{МПа}} 2 \cdot (2\,\text{л} - 5\,\text{л}) = -11{,}2500\,\text{кДж}, \\
    Q &= A_\text{газа} + \Delta U = \frac 32 (P_2 V_2 - P_1 V_1) + \frac{P_2 + P_1} 2 \cdot (V_2 - V_1) = -9{,}000\,\text{кДж} -11{,}2500\,\text{кДж} = -20{,}250\,\text{кДж}.
    \end{align*}
}
\solutionspace{150pt}

\tasknumber{3}%
\task{%
    Запишите формулы и рядом с каждой физичической величиной укажите её название и единицы измерения в СИ:
    \begin{enumerate}
        \item первое начало термодинамики,
        \item внутренняя энергия идеального одноатомного газа.
    \end{enumerate}
}

\variantsplitter

\addpersonalvariant{Софья Андрианова}

\tasknumber{1}%
\task{%
    Определите КПД цикла 12341, рабочим телом которого является идеальный одноатомный газ, если
    12 — изобарическое расширение газа в шесть раз,
    23 — изохорическое охлаждение газа, при котором температура уменьшается в два раза,
    34 — изобара, 41 — изохора.
    % Для этого:
    % \begin{enumerate}
    %     \item сделайте рисунок в $PV$-координатах,
    %     \item выберите удобные обозначения, чтобы не запутаться в множестве температур, давлений и объёмов,
    %     \item вычислите необходимые соотнощения между температурами, давлениями и объёмами
    %     (некоторые сразу видны по рисунку, некоторые — надо считать),
    %     \item определите для каждого участка поглощается или отдаётся тепло (и сколько именно:
    %     потребуется первое начало термодинамики, отдельный расчёт работ на участках через площади фигур и изменений внутренней энергии),
    %     \item вычислите полную работу газа в цикле,
    %     \item подставьте всё в формулу для КПД, упростите и доведите до ответа.
    % \end{enumerate}
    Определите КПД цикла Карно, температура нагревателя которого равна максимальной температуре в цикле 12341, а холодильника — минимальной.
    Ответы в обоих случаях оставьте точными в виде нескоратимой дроби, никаких округлений.
}
\answer{%
    \begin{align*}
    A_{12} &> 0, \Delta U_{12} > 0, \implies Q_{12} = A_{12} + \Delta U_{12} > 0, \\
    A_{23} &= 0, \Delta U_{23} < 0, \implies Q_{23} = A_{23} + \Delta U_{23} < 0, \\
    A_{34} &< 0, \Delta U_{34} < 0, \implies Q_{34} = A_{34} + \Delta U_{34} < 0, \\
    A_{41} &= 0, \Delta U_{41} > 0, \implies Q_{41} = A_{41} + \Delta U_{41} > 0.
    \\
    P_1V_1 &= \nu R T_1, P_2V_2 = \nu R T_2, P_3V_3 = \nu R T_3, P_4V_4 = \nu R T_4 \text{ — уравнения состояния идеального газа}, \\
    &\text{Пусть $P_0$, $V_0$, $T_0$ — давление, объём и температура в точке 4 (минимальные во всём цикле):} \\
    P_1 &= P_2, P_3 = P_4 = P_0, V_1 = V_4 = V_0, V_2 = V_3 = 6 V_1 = 6 V_0,, \text{остальные соотношения между объёмами и давлениями не даны, нужно считать} \\
    T_3 &= \frac{T_2}2 \text{(по условию)} \implies \frac{P_2}{P_3} = \frac{P_2 V_2}{P_3 V_3}= \frac{\nu R T_2}{\nu R T_3} = \frac{T_2}{T_3} = 2 \implies P_1 = P_2 = 2 P_0 \\
    A_\text{цикл} &= (2P_0 - P_0)(6V_0 - V_0) = 5P_0V_0, \\
    A_{12} &= 2P_0 \cdot (6V_0 - V_0) = 10P_0V_0, \\
    \Delta U_{12} &= \frac 32 \nu R T_2 - \frac 32 \nu R T_1 = \frac 32 P_2 V_2 - \frac 32 P_1 V_1 = \frac 32 \cdot 2 P_0 \cdot 6 V_0 -  \frac 32 \cdot 2 P_0 \cdot V_0 = \frac 32 \cdot 10 \cdot P_0V_0, \\
    \Delta U_{41} &= \frac 32 \nu R T_1 - \frac 32 \nu R T_4 = \frac 32 P_1 V_1 - \frac 32 P_4 V_4 = \frac 32 \cdot 2 P_0 V_0 - \frac 32 P_0 V_0 = \frac 32 \cdot 1 \cdot P_0V_0.
    \\
    \eta &= \frac{A_\text{цикл}}{Q_+} = \frac{A_\text{цикл}}{Q_{12} + Q_{41}}  = \frac{A_\text{цикл}}{A_{12} + \Delta U_{12} + A_{41} + \Delta U_{41}} =  \\
     &= \frac{5P_0V_0}{10P_0V_0 + \frac 32 \cdot 10 \cdot P_0V_0 + 0 + \frac 32 \cdot 1 \cdot P_0V_0} = \frac{5}{10 + \frac 32 \cdot 10 + \frac 32 \cdot 1} = \frac{10}{53} \approx 0{,}189.
     \\
    \eta_\text{Карно} &= 1 - \frac{T_\text{х}}{T_\text{н}} = 1 - \frac{T_\text{4}}{T_\text{2}} = 1 - \frac{\frac{P_4V_4}{\nu R}}{\frac{P_2V_2}{\nu R}} = 1 - \frac{P_4V_4}{P_2V_2} = 1 - \frac{P_0V_0}{2P_0 \cdot 6V_0} = 1 - \frac 1{2 \cdot 6}  = \frac{11}{12} \approx 0{,}917.
    \end{align*}
}
\solutionspace{360pt}

\tasknumber{2}%
\task{%
    Порция идеального одноатомного газа перешла из состояния 1 в состояние 2: $P_1 = 4\,\text{МПа}$, $V_1 = 3\,\text{л}$, $P_2 = 1{,}5\,\text{МПа}$, $V_2 = 4\,\text{л}$.
    Известно, что в $PV$-координатах график процесса 12 представляет собой отрезок прямой.
    Определите,
    \begin{itemize}
        \item какую работу при этом совершил газ,
        \item чему равно изменение внутренней энергии газа,
        \item сколько теплоты подвели к нему в этом процессе?
    \end{itemize}
    При решении обратите внимание на знаки искомых величин.
}
\answer{%
    \begin{align*}
    P_1V_1 &= \nu R T_1, P_2V_2 = \nu R T_2, \\
    \Delta U &= U_2-U_1 = \frac 32 \nu R T_2- \frac 32 \nu R T_1 = \frac 32 P_2 V_2 - \frac 32 P_1 V_1= \frac 32 \cdot \cbr{1{,}5\,\text{МПа} \cdot 4\,\text{л} - 4\,\text{МПа} \cdot 3\,\text{л}} = -9{,}000\,\text{кДж}.
    \\
    A_\text{газа} &= \frac{P_2 + P_1} 2 \cdot (V_2 - V_1) = \frac{1{,}5\,\text{МПа} + 4\,\text{МПа}} 2 \cdot (4\,\text{л} - 3\,\text{л}) = 2{,}75\,\text{кДж}, \\
    Q &= A_\text{газа} + \Delta U = \frac 32 (P_2 V_2 - P_1 V_1) + \frac{P_2 + P_1} 2 \cdot (V_2 - V_1) = -9{,}000\,\text{кДж} + 2{,}75\,\text{кДж} = -6{,}250\,\text{кДж}.
    \end{align*}
}
\solutionspace{150pt}

\tasknumber{3}%
\task{%
    Запишите формулы и рядом с каждой физичической величиной укажите её название и единицы измерения в СИ:
    \begin{enumerate}
        \item первое начало термодинамики,
        \item внутренняя энергия идеального одноатомного газа.
    \end{enumerate}
}

\variantsplitter

\addpersonalvariant{Владимир Артемчук}

\tasknumber{1}%
\task{%
    Определите КПД цикла 12341, рабочим телом которого является идеальный одноатомный газ, если
    12 — изобарическое расширение газа в шесть раз,
    23 — изохорическое охлаждение газа, при котором температура уменьшается в пять раз,
    34 — изобара, 41 — изохора.
    % Для этого:
    % \begin{enumerate}
    %     \item сделайте рисунок в $PV$-координатах,
    %     \item выберите удобные обозначения, чтобы не запутаться в множестве температур, давлений и объёмов,
    %     \item вычислите необходимые соотнощения между температурами, давлениями и объёмами
    %     (некоторые сразу видны по рисунку, некоторые — надо считать),
    %     \item определите для каждого участка поглощается или отдаётся тепло (и сколько именно:
    %     потребуется первое начало термодинамики, отдельный расчёт работ на участках через площади фигур и изменений внутренней энергии),
    %     \item вычислите полную работу газа в цикле,
    %     \item подставьте всё в формулу для КПД, упростите и доведите до ответа.
    % \end{enumerate}
    Определите КПД цикла Карно, температура нагревателя которого равна максимальной температуре в цикле 12341, а холодильника — минимальной.
    Ответы в обоих случаях оставьте точными в виде нескоратимой дроби, никаких округлений.
}
\answer{%
    \begin{align*}
    A_{12} &> 0, \Delta U_{12} > 0, \implies Q_{12} = A_{12} + \Delta U_{12} > 0, \\
    A_{23} &= 0, \Delta U_{23} < 0, \implies Q_{23} = A_{23} + \Delta U_{23} < 0, \\
    A_{34} &< 0, \Delta U_{34} < 0, \implies Q_{34} = A_{34} + \Delta U_{34} < 0, \\
    A_{41} &= 0, \Delta U_{41} > 0, \implies Q_{41} = A_{41} + \Delta U_{41} > 0.
    \\
    P_1V_1 &= \nu R T_1, P_2V_2 = \nu R T_2, P_3V_3 = \nu R T_3, P_4V_4 = \nu R T_4 \text{ — уравнения состояния идеального газа}, \\
    &\text{Пусть $P_0$, $V_0$, $T_0$ — давление, объём и температура в точке 4 (минимальные во всём цикле):} \\
    P_1 &= P_2, P_3 = P_4 = P_0, V_1 = V_4 = V_0, V_2 = V_3 = 6 V_1 = 6 V_0,, \text{остальные соотношения между объёмами и давлениями не даны, нужно считать} \\
    T_3 &= \frac{T_2}5 \text{(по условию)} \implies \frac{P_2}{P_3} = \frac{P_2 V_2}{P_3 V_3}= \frac{\nu R T_2}{\nu R T_3} = \frac{T_2}{T_3} = 5 \implies P_1 = P_2 = 5 P_0 \\
    A_\text{цикл} &= (5P_0 - P_0)(6V_0 - V_0) = 20P_0V_0, \\
    A_{12} &= 5P_0 \cdot (6V_0 - V_0) = 25P_0V_0, \\
    \Delta U_{12} &= \frac 32 \nu R T_2 - \frac 32 \nu R T_1 = \frac 32 P_2 V_2 - \frac 32 P_1 V_1 = \frac 32 \cdot 5 P_0 \cdot 6 V_0 -  \frac 32 \cdot 5 P_0 \cdot V_0 = \frac 32 \cdot 25 \cdot P_0V_0, \\
    \Delta U_{41} &= \frac 32 \nu R T_1 - \frac 32 \nu R T_4 = \frac 32 P_1 V_1 - \frac 32 P_4 V_4 = \frac 32 \cdot 5 P_0 V_0 - \frac 32 P_0 V_0 = \frac 32 \cdot 4 \cdot P_0V_0.
    \\
    \eta &= \frac{A_\text{цикл}}{Q_+} = \frac{A_\text{цикл}}{Q_{12} + Q_{41}}  = \frac{A_\text{цикл}}{A_{12} + \Delta U_{12} + A_{41} + \Delta U_{41}} =  \\
     &= \frac{20P_0V_0}{25P_0V_0 + \frac 32 \cdot 25 \cdot P_0V_0 + 0 + \frac 32 \cdot 4 \cdot P_0V_0} = \frac{20}{25 + \frac 32 \cdot 25 + \frac 32 \cdot 4} = \frac{40}{137} \approx 0{,}292.
     \\
    \eta_\text{Карно} &= 1 - \frac{T_\text{х}}{T_\text{н}} = 1 - \frac{T_\text{4}}{T_\text{2}} = 1 - \frac{\frac{P_4V_4}{\nu R}}{\frac{P_2V_2}{\nu R}} = 1 - \frac{P_4V_4}{P_2V_2} = 1 - \frac{P_0V_0}{5P_0 \cdot 6V_0} = 1 - \frac 1{5 \cdot 6}  = \frac{29}{30} \approx 0{,}967.
    \end{align*}
}
\solutionspace{360pt}

\tasknumber{2}%
\task{%
    Порция идеального одноатомного газа перешла из состояния 1 в состояние 2: $P_1 = 2\,\text{МПа}$, $V_1 = 5\,\text{л}$, $P_2 = 2{,}5\,\text{МПа}$, $V_2 = 8\,\text{л}$.
    Известно, что в $PV$-координатах график процесса 12 представляет собой отрезок прямой.
    Определите,
    \begin{itemize}
        \item какую работу при этом совершил газ,
        \item чему равно изменение внутренней энергии газа,
        \item сколько теплоты подвели к нему в этом процессе?
    \end{itemize}
    При решении обратите внимание на знаки искомых величин.
}
\answer{%
    \begin{align*}
    P_1V_1 &= \nu R T_1, P_2V_2 = \nu R T_2, \\
    \Delta U &= U_2-U_1 = \frac 32 \nu R T_2- \frac 32 \nu R T_1 = \frac 32 P_2 V_2 - \frac 32 P_1 V_1= \frac 32 \cdot \cbr{2{,}5\,\text{МПа} \cdot 8\,\text{л} - 2\,\text{МПа} \cdot 5\,\text{л}} = 15{,}00\,\text{кДж}.
    \\
    A_\text{газа} &= \frac{P_2 + P_1} 2 \cdot (V_2 - V_1) = \frac{2{,}5\,\text{МПа} + 2\,\text{МПа}} 2 \cdot (8\,\text{л} - 5\,\text{л}) = 6{,}75\,\text{кДж}, \\
    Q &= A_\text{газа} + \Delta U = \frac 32 (P_2 V_2 - P_1 V_1) + \frac{P_2 + P_1} 2 \cdot (V_2 - V_1) = 15{,}00\,\text{кДж} + 6{,}75\,\text{кДж} = 21{,}75\,\text{кДж}.
    \end{align*}
}
\solutionspace{150pt}

\tasknumber{3}%
\task{%
    Запишите формулы и рядом с каждой физичической величиной укажите её название и единицы измерения в СИ:
    \begin{enumerate}
        \item первое начало термодинамики,
        \item внутренняя энергия идеального одноатомного газа.
    \end{enumerate}
}

\variantsplitter

\addpersonalvariant{Софья Белянкина}

\tasknumber{1}%
\task{%
    Определите КПД цикла 12341, рабочим телом которого является идеальный одноатомный газ, если
    12 — изобарическое расширение газа в четыре раза,
    23 — изохорическое охлаждение газа, при котором температура уменьшается в три раза,
    34 — изобара, 41 — изохора.
    % Для этого:
    % \begin{enumerate}
    %     \item сделайте рисунок в $PV$-координатах,
    %     \item выберите удобные обозначения, чтобы не запутаться в множестве температур, давлений и объёмов,
    %     \item вычислите необходимые соотнощения между температурами, давлениями и объёмами
    %     (некоторые сразу видны по рисунку, некоторые — надо считать),
    %     \item определите для каждого участка поглощается или отдаётся тепло (и сколько именно:
    %     потребуется первое начало термодинамики, отдельный расчёт работ на участках через площади фигур и изменений внутренней энергии),
    %     \item вычислите полную работу газа в цикле,
    %     \item подставьте всё в формулу для КПД, упростите и доведите до ответа.
    % \end{enumerate}
    Определите КПД цикла Карно, температура нагревателя которого равна максимальной температуре в цикле 12341, а холодильника — минимальной.
    Ответы в обоих случаях оставьте точными в виде нескоратимой дроби, никаких округлений.
}
\answer{%
    \begin{align*}
    A_{12} &> 0, \Delta U_{12} > 0, \implies Q_{12} = A_{12} + \Delta U_{12} > 0, \\
    A_{23} &= 0, \Delta U_{23} < 0, \implies Q_{23} = A_{23} + \Delta U_{23} < 0, \\
    A_{34} &< 0, \Delta U_{34} < 0, \implies Q_{34} = A_{34} + \Delta U_{34} < 0, \\
    A_{41} &= 0, \Delta U_{41} > 0, \implies Q_{41} = A_{41} + \Delta U_{41} > 0.
    \\
    P_1V_1 &= \nu R T_1, P_2V_2 = \nu R T_2, P_3V_3 = \nu R T_3, P_4V_4 = \nu R T_4 \text{ — уравнения состояния идеального газа}, \\
    &\text{Пусть $P_0$, $V_0$, $T_0$ — давление, объём и температура в точке 4 (минимальные во всём цикле):} \\
    P_1 &= P_2, P_3 = P_4 = P_0, V_1 = V_4 = V_0, V_2 = V_3 = 4 V_1 = 4 V_0,, \text{остальные соотношения между объёмами и давлениями не даны, нужно считать} \\
    T_3 &= \frac{T_2}3 \text{(по условию)} \implies \frac{P_2}{P_3} = \frac{P_2 V_2}{P_3 V_3}= \frac{\nu R T_2}{\nu R T_3} = \frac{T_2}{T_3} = 3 \implies P_1 = P_2 = 3 P_0 \\
    A_\text{цикл} &= (3P_0 - P_0)(4V_0 - V_0) = 6P_0V_0, \\
    A_{12} &= 3P_0 \cdot (4V_0 - V_0) = 9P_0V_0, \\
    \Delta U_{12} &= \frac 32 \nu R T_2 - \frac 32 \nu R T_1 = \frac 32 P_2 V_2 - \frac 32 P_1 V_1 = \frac 32 \cdot 3 P_0 \cdot 4 V_0 -  \frac 32 \cdot 3 P_0 \cdot V_0 = \frac 32 \cdot 9 \cdot P_0V_0, \\
    \Delta U_{41} &= \frac 32 \nu R T_1 - \frac 32 \nu R T_4 = \frac 32 P_1 V_1 - \frac 32 P_4 V_4 = \frac 32 \cdot 3 P_0 V_0 - \frac 32 P_0 V_0 = \frac 32 \cdot 2 \cdot P_0V_0.
    \\
    \eta &= \frac{A_\text{цикл}}{Q_+} = \frac{A_\text{цикл}}{Q_{12} + Q_{41}}  = \frac{A_\text{цикл}}{A_{12} + \Delta U_{12} + A_{41} + \Delta U_{41}} =  \\
     &= \frac{6P_0V_0}{9P_0V_0 + \frac 32 \cdot 9 \cdot P_0V_0 + 0 + \frac 32 \cdot 2 \cdot P_0V_0} = \frac{6}{9 + \frac 32 \cdot 9 + \frac 32 \cdot 2} = \frac4{17} \approx 0{,}235.
     \\
    \eta_\text{Карно} &= 1 - \frac{T_\text{х}}{T_\text{н}} = 1 - \frac{T_\text{4}}{T_\text{2}} = 1 - \frac{\frac{P_4V_4}{\nu R}}{\frac{P_2V_2}{\nu R}} = 1 - \frac{P_4V_4}{P_2V_2} = 1 - \frac{P_0V_0}{3P_0 \cdot 4V_0} = 1 - \frac 1{3 \cdot 4}  = \frac{11}{12} \approx 0{,}917.
    \end{align*}
}
\solutionspace{360pt}

\tasknumber{2}%
\task{%
    Порция идеального одноатомного газа перешла из состояния 1 в состояние 2: $P_1 = 4\,\text{МПа}$, $V_1 = 7\,\text{л}$, $P_2 = 1{,}5\,\text{МПа}$, $V_2 = 8\,\text{л}$.
    Известно, что в $PV$-координатах график процесса 12 представляет собой отрезок прямой.
    Определите,
    \begin{itemize}
        \item какую работу при этом совершил газ,
        \item чему равно изменение внутренней энергии газа,
        \item сколько теплоты подвели к нему в этом процессе?
    \end{itemize}
    При решении обратите внимание на знаки искомых величин.
}
\answer{%
    \begin{align*}
    P_1V_1 &= \nu R T_1, P_2V_2 = \nu R T_2, \\
    \Delta U &= U_2-U_1 = \frac 32 \nu R T_2- \frac 32 \nu R T_1 = \frac 32 P_2 V_2 - \frac 32 P_1 V_1= \frac 32 \cdot \cbr{1{,}5\,\text{МПа} \cdot 8\,\text{л} - 4\,\text{МПа} \cdot 7\,\text{л}} = -24{,}000\,\text{кДж}.
    \\
    A_\text{газа} &= \frac{P_2 + P_1} 2 \cdot (V_2 - V_1) = \frac{1{,}5\,\text{МПа} + 4\,\text{МПа}} 2 \cdot (8\,\text{л} - 7\,\text{л}) = 2{,}75\,\text{кДж}, \\
    Q &= A_\text{газа} + \Delta U = \frac 32 (P_2 V_2 - P_1 V_1) + \frac{P_2 + P_1} 2 \cdot (V_2 - V_1) = -24{,}000\,\text{кДж} + 2{,}75\,\text{кДж} = -21{,}250\,\text{кДж}.
    \end{align*}
}
\solutionspace{150pt}

\tasknumber{3}%
\task{%
    Запишите формулы и рядом с каждой физичической величиной укажите её название и единицы измерения в СИ:
    \begin{enumerate}
        \item первое начало термодинамики,
        \item внутренняя энергия идеального одноатомного газа.
    \end{enumerate}
}

\variantsplitter

\addpersonalvariant{Варвара Егиазарян}

\tasknumber{1}%
\task{%
    Определите КПД цикла 12341, рабочим телом которого является идеальный одноатомный газ, если
    12 — изобарическое расширение газа в пять раз,
    23 — изохорическое охлаждение газа, при котором температура уменьшается в два раза,
    34 — изобара, 41 — изохора.
    % Для этого:
    % \begin{enumerate}
    %     \item сделайте рисунок в $PV$-координатах,
    %     \item выберите удобные обозначения, чтобы не запутаться в множестве температур, давлений и объёмов,
    %     \item вычислите необходимые соотнощения между температурами, давлениями и объёмами
    %     (некоторые сразу видны по рисунку, некоторые — надо считать),
    %     \item определите для каждого участка поглощается или отдаётся тепло (и сколько именно:
    %     потребуется первое начало термодинамики, отдельный расчёт работ на участках через площади фигур и изменений внутренней энергии),
    %     \item вычислите полную работу газа в цикле,
    %     \item подставьте всё в формулу для КПД, упростите и доведите до ответа.
    % \end{enumerate}
    Определите КПД цикла Карно, температура нагревателя которого равна максимальной температуре в цикле 12341, а холодильника — минимальной.
    Ответы в обоих случаях оставьте точными в виде нескоратимой дроби, никаких округлений.
}
\answer{%
    \begin{align*}
    A_{12} &> 0, \Delta U_{12} > 0, \implies Q_{12} = A_{12} + \Delta U_{12} > 0, \\
    A_{23} &= 0, \Delta U_{23} < 0, \implies Q_{23} = A_{23} + \Delta U_{23} < 0, \\
    A_{34} &< 0, \Delta U_{34} < 0, \implies Q_{34} = A_{34} + \Delta U_{34} < 0, \\
    A_{41} &= 0, \Delta U_{41} > 0, \implies Q_{41} = A_{41} + \Delta U_{41} > 0.
    \\
    P_1V_1 &= \nu R T_1, P_2V_2 = \nu R T_2, P_3V_3 = \nu R T_3, P_4V_4 = \nu R T_4 \text{ — уравнения состояния идеального газа}, \\
    &\text{Пусть $P_0$, $V_0$, $T_0$ — давление, объём и температура в точке 4 (минимальные во всём цикле):} \\
    P_1 &= P_2, P_3 = P_4 = P_0, V_1 = V_4 = V_0, V_2 = V_3 = 5 V_1 = 5 V_0,, \text{остальные соотношения между объёмами и давлениями не даны, нужно считать} \\
    T_3 &= \frac{T_2}2 \text{(по условию)} \implies \frac{P_2}{P_3} = \frac{P_2 V_2}{P_3 V_3}= \frac{\nu R T_2}{\nu R T_3} = \frac{T_2}{T_3} = 2 \implies P_1 = P_2 = 2 P_0 \\
    A_\text{цикл} &= (2P_0 - P_0)(5V_0 - V_0) = 4P_0V_0, \\
    A_{12} &= 2P_0 \cdot (5V_0 - V_0) = 8P_0V_0, \\
    \Delta U_{12} &= \frac 32 \nu R T_2 - \frac 32 \nu R T_1 = \frac 32 P_2 V_2 - \frac 32 P_1 V_1 = \frac 32 \cdot 2 P_0 \cdot 5 V_0 -  \frac 32 \cdot 2 P_0 \cdot V_0 = \frac 32 \cdot 8 \cdot P_0V_0, \\
    \Delta U_{41} &= \frac 32 \nu R T_1 - \frac 32 \nu R T_4 = \frac 32 P_1 V_1 - \frac 32 P_4 V_4 = \frac 32 \cdot 2 P_0 V_0 - \frac 32 P_0 V_0 = \frac 32 \cdot 1 \cdot P_0V_0.
    \\
    \eta &= \frac{A_\text{цикл}}{Q_+} = \frac{A_\text{цикл}}{Q_{12} + Q_{41}}  = \frac{A_\text{цикл}}{A_{12} + \Delta U_{12} + A_{41} + \Delta U_{41}} =  \\
     &= \frac{4P_0V_0}{8P_0V_0 + \frac 32 \cdot 8 \cdot P_0V_0 + 0 + \frac 32 \cdot 1 \cdot P_0V_0} = \frac{4}{8 + \frac 32 \cdot 8 + \frac 32 \cdot 1} = \frac8{43} \approx 0{,}186.
     \\
    \eta_\text{Карно} &= 1 - \frac{T_\text{х}}{T_\text{н}} = 1 - \frac{T_\text{4}}{T_\text{2}} = 1 - \frac{\frac{P_4V_4}{\nu R}}{\frac{P_2V_2}{\nu R}} = 1 - \frac{P_4V_4}{P_2V_2} = 1 - \frac{P_0V_0}{2P_0 \cdot 5V_0} = 1 - \frac 1{2 \cdot 5}  = \frac9{10} \approx 0{,}900.
    \end{align*}
}
\solutionspace{360pt}

\tasknumber{2}%
\task{%
    Порция идеального одноатомного газа перешла из состояния 1 в состояние 2: $P_1 = 4\,\text{МПа}$, $V_1 = 5\,\text{л}$, $P_2 = 4{,}5\,\text{МПа}$, $V_2 = 6\,\text{л}$.
    Известно, что в $PV$-координатах график процесса 12 представляет собой отрезок прямой.
    Определите,
    \begin{itemize}
        \item какую работу при этом совершил газ,
        \item чему равно изменение внутренней энергии газа,
        \item сколько теплоты подвели к нему в этом процессе?
    \end{itemize}
    При решении обратите внимание на знаки искомых величин.
}
\answer{%
    \begin{align*}
    P_1V_1 &= \nu R T_1, P_2V_2 = \nu R T_2, \\
    \Delta U &= U_2-U_1 = \frac 32 \nu R T_2- \frac 32 \nu R T_1 = \frac 32 P_2 V_2 - \frac 32 P_1 V_1= \frac 32 \cdot \cbr{4{,}5\,\text{МПа} \cdot 6\,\text{л} - 4\,\text{МПа} \cdot 5\,\text{л}} = 10{,}50\,\text{кДж}.
    \\
    A_\text{газа} &= \frac{P_2 + P_1} 2 \cdot (V_2 - V_1) = \frac{4{,}5\,\text{МПа} + 4\,\text{МПа}} 2 \cdot (6\,\text{л} - 5\,\text{л}) = 4{,}25\,\text{кДж}, \\
    Q &= A_\text{газа} + \Delta U = \frac 32 (P_2 V_2 - P_1 V_1) + \frac{P_2 + P_1} 2 \cdot (V_2 - V_1) = 10{,}50\,\text{кДж} + 4{,}25\,\text{кДж} = 14{,}75\,\text{кДж}.
    \end{align*}
}
\solutionspace{150pt}

\tasknumber{3}%
\task{%
    Запишите формулы и рядом с каждой физичической величиной укажите её название и единицы измерения в СИ:
    \begin{enumerate}
        \item первое начало термодинамики,
        \item внутренняя энергия идеального одноатомного газа.
    \end{enumerate}
}

\variantsplitter

\addpersonalvariant{Владислав Емелин}

\tasknumber{1}%
\task{%
    Определите КПД цикла 12341, рабочим телом которого является идеальный одноатомный газ, если
    12 — изобарическое расширение газа в четыре раза,
    23 — изохорическое охлаждение газа, при котором температура уменьшается в два раза,
    34 — изобара, 41 — изохора.
    % Для этого:
    % \begin{enumerate}
    %     \item сделайте рисунок в $PV$-координатах,
    %     \item выберите удобные обозначения, чтобы не запутаться в множестве температур, давлений и объёмов,
    %     \item вычислите необходимые соотнощения между температурами, давлениями и объёмами
    %     (некоторые сразу видны по рисунку, некоторые — надо считать),
    %     \item определите для каждого участка поглощается или отдаётся тепло (и сколько именно:
    %     потребуется первое начало термодинамики, отдельный расчёт работ на участках через площади фигур и изменений внутренней энергии),
    %     \item вычислите полную работу газа в цикле,
    %     \item подставьте всё в формулу для КПД, упростите и доведите до ответа.
    % \end{enumerate}
    Определите КПД цикла Карно, температура нагревателя которого равна максимальной температуре в цикле 12341, а холодильника — минимальной.
    Ответы в обоих случаях оставьте точными в виде нескоратимой дроби, никаких округлений.
}
\answer{%
    \begin{align*}
    A_{12} &> 0, \Delta U_{12} > 0, \implies Q_{12} = A_{12} + \Delta U_{12} > 0, \\
    A_{23} &= 0, \Delta U_{23} < 0, \implies Q_{23} = A_{23} + \Delta U_{23} < 0, \\
    A_{34} &< 0, \Delta U_{34} < 0, \implies Q_{34} = A_{34} + \Delta U_{34} < 0, \\
    A_{41} &= 0, \Delta U_{41} > 0, \implies Q_{41} = A_{41} + \Delta U_{41} > 0.
    \\
    P_1V_1 &= \nu R T_1, P_2V_2 = \nu R T_2, P_3V_3 = \nu R T_3, P_4V_4 = \nu R T_4 \text{ — уравнения состояния идеального газа}, \\
    &\text{Пусть $P_0$, $V_0$, $T_0$ — давление, объём и температура в точке 4 (минимальные во всём цикле):} \\
    P_1 &= P_2, P_3 = P_4 = P_0, V_1 = V_4 = V_0, V_2 = V_3 = 4 V_1 = 4 V_0,, \text{остальные соотношения между объёмами и давлениями не даны, нужно считать} \\
    T_3 &= \frac{T_2}2 \text{(по условию)} \implies \frac{P_2}{P_3} = \frac{P_2 V_2}{P_3 V_3}= \frac{\nu R T_2}{\nu R T_3} = \frac{T_2}{T_3} = 2 \implies P_1 = P_2 = 2 P_0 \\
    A_\text{цикл} &= (2P_0 - P_0)(4V_0 - V_0) = 3P_0V_0, \\
    A_{12} &= 2P_0 \cdot (4V_0 - V_0) = 6P_0V_0, \\
    \Delta U_{12} &= \frac 32 \nu R T_2 - \frac 32 \nu R T_1 = \frac 32 P_2 V_2 - \frac 32 P_1 V_1 = \frac 32 \cdot 2 P_0 \cdot 4 V_0 -  \frac 32 \cdot 2 P_0 \cdot V_0 = \frac 32 \cdot 6 \cdot P_0V_0, \\
    \Delta U_{41} &= \frac 32 \nu R T_1 - \frac 32 \nu R T_4 = \frac 32 P_1 V_1 - \frac 32 P_4 V_4 = \frac 32 \cdot 2 P_0 V_0 - \frac 32 P_0 V_0 = \frac 32 \cdot 1 \cdot P_0V_0.
    \\
    \eta &= \frac{A_\text{цикл}}{Q_+} = \frac{A_\text{цикл}}{Q_{12} + Q_{41}}  = \frac{A_\text{цикл}}{A_{12} + \Delta U_{12} + A_{41} + \Delta U_{41}} =  \\
     &= \frac{3P_0V_0}{6P_0V_0 + \frac 32 \cdot 6 \cdot P_0V_0 + 0 + \frac 32 \cdot 1 \cdot P_0V_0} = \frac{3}{6 + \frac 32 \cdot 6 + \frac 32 \cdot 1} = \frac2{11} \approx 0{,}182.
     \\
    \eta_\text{Карно} &= 1 - \frac{T_\text{х}}{T_\text{н}} = 1 - \frac{T_\text{4}}{T_\text{2}} = 1 - \frac{\frac{P_4V_4}{\nu R}}{\frac{P_2V_2}{\nu R}} = 1 - \frac{P_4V_4}{P_2V_2} = 1 - \frac{P_0V_0}{2P_0 \cdot 4V_0} = 1 - \frac 1{2 \cdot 4}  = \frac78 \approx 0{,}875.
    \end{align*}
}
\solutionspace{360pt}

\tasknumber{2}%
\task{%
    Порция идеального одноатомного газа перешла из состояния 1 в состояние 2: $P_1 = 2\,\text{МПа}$, $V_1 = 5\,\text{л}$, $P_2 = 4{,}5\,\text{МПа}$, $V_2 = 2\,\text{л}$.
    Известно, что в $PV$-координатах график процесса 12 представляет собой отрезок прямой.
    Определите,
    \begin{itemize}
        \item какую работу при этом совершил газ,
        \item чему равно изменение внутренней энергии газа,
        \item сколько теплоты подвели к нему в этом процессе?
    \end{itemize}
    При решении обратите внимание на знаки искомых величин.
}
\answer{%
    \begin{align*}
    P_1V_1 &= \nu R T_1, P_2V_2 = \nu R T_2, \\
    \Delta U &= U_2-U_1 = \frac 32 \nu R T_2- \frac 32 \nu R T_1 = \frac 32 P_2 V_2 - \frac 32 P_1 V_1= \frac 32 \cdot \cbr{4{,}5\,\text{МПа} \cdot 2\,\text{л} - 2\,\text{МПа} \cdot 5\,\text{л}} = -1{,}5000\,\text{кДж}.
    \\
    A_\text{газа} &= \frac{P_2 + P_1} 2 \cdot (V_2 - V_1) = \frac{4{,}5\,\text{МПа} + 2\,\text{МПа}} 2 \cdot (2\,\text{л} - 5\,\text{л}) = -9{,}750\,\text{кДж}, \\
    Q &= A_\text{газа} + \Delta U = \frac 32 (P_2 V_2 - P_1 V_1) + \frac{P_2 + P_1} 2 \cdot (V_2 - V_1) = -1{,}5000\,\text{кДж} -9{,}750\,\text{кДж} = -11{,}2500\,\text{кДж}.
    \end{align*}
}
\solutionspace{150pt}

\tasknumber{3}%
\task{%
    Запишите формулы и рядом с каждой физичической величиной укажите её название и единицы измерения в СИ:
    \begin{enumerate}
        \item первое начало термодинамики,
        \item внутренняя энергия идеального одноатомного газа.
    \end{enumerate}
}

\variantsplitter

\addpersonalvariant{Артём Жичин}

\tasknumber{1}%
\task{%
    Определите КПД цикла 12341, рабочим телом которого является идеальный одноатомный газ, если
    12 — изобарическое расширение газа в шесть раз,
    23 — изохорическое охлаждение газа, при котором температура уменьшается в шесть раз,
    34 — изобара, 41 — изохора.
    % Для этого:
    % \begin{enumerate}
    %     \item сделайте рисунок в $PV$-координатах,
    %     \item выберите удобные обозначения, чтобы не запутаться в множестве температур, давлений и объёмов,
    %     \item вычислите необходимые соотнощения между температурами, давлениями и объёмами
    %     (некоторые сразу видны по рисунку, некоторые — надо считать),
    %     \item определите для каждого участка поглощается или отдаётся тепло (и сколько именно:
    %     потребуется первое начало термодинамики, отдельный расчёт работ на участках через площади фигур и изменений внутренней энергии),
    %     \item вычислите полную работу газа в цикле,
    %     \item подставьте всё в формулу для КПД, упростите и доведите до ответа.
    % \end{enumerate}
    Определите КПД цикла Карно, температура нагревателя которого равна максимальной температуре в цикле 12341, а холодильника — минимальной.
    Ответы в обоих случаях оставьте точными в виде нескоратимой дроби, никаких округлений.
}
\answer{%
    \begin{align*}
    A_{12} &> 0, \Delta U_{12} > 0, \implies Q_{12} = A_{12} + \Delta U_{12} > 0, \\
    A_{23} &= 0, \Delta U_{23} < 0, \implies Q_{23} = A_{23} + \Delta U_{23} < 0, \\
    A_{34} &< 0, \Delta U_{34} < 0, \implies Q_{34} = A_{34} + \Delta U_{34} < 0, \\
    A_{41} &= 0, \Delta U_{41} > 0, \implies Q_{41} = A_{41} + \Delta U_{41} > 0.
    \\
    P_1V_1 &= \nu R T_1, P_2V_2 = \nu R T_2, P_3V_3 = \nu R T_3, P_4V_4 = \nu R T_4 \text{ — уравнения состояния идеального газа}, \\
    &\text{Пусть $P_0$, $V_0$, $T_0$ — давление, объём и температура в точке 4 (минимальные во всём цикле):} \\
    P_1 &= P_2, P_3 = P_4 = P_0, V_1 = V_4 = V_0, V_2 = V_3 = 6 V_1 = 6 V_0,, \text{остальные соотношения между объёмами и давлениями не даны, нужно считать} \\
    T_3 &= \frac{T_2}6 \text{(по условию)} \implies \frac{P_2}{P_3} = \frac{P_2 V_2}{P_3 V_3}= \frac{\nu R T_2}{\nu R T_3} = \frac{T_2}{T_3} = 6 \implies P_1 = P_2 = 6 P_0 \\
    A_\text{цикл} &= (6P_0 - P_0)(6V_0 - V_0) = 25P_0V_0, \\
    A_{12} &= 6P_0 \cdot (6V_0 - V_0) = 30P_0V_0, \\
    \Delta U_{12} &= \frac 32 \nu R T_2 - \frac 32 \nu R T_1 = \frac 32 P_2 V_2 - \frac 32 P_1 V_1 = \frac 32 \cdot 6 P_0 \cdot 6 V_0 -  \frac 32 \cdot 6 P_0 \cdot V_0 = \frac 32 \cdot 30 \cdot P_0V_0, \\
    \Delta U_{41} &= \frac 32 \nu R T_1 - \frac 32 \nu R T_4 = \frac 32 P_1 V_1 - \frac 32 P_4 V_4 = \frac 32 \cdot 6 P_0 V_0 - \frac 32 P_0 V_0 = \frac 32 \cdot 5 \cdot P_0V_0.
    \\
    \eta &= \frac{A_\text{цикл}}{Q_+} = \frac{A_\text{цикл}}{Q_{12} + Q_{41}}  = \frac{A_\text{цикл}}{A_{12} + \Delta U_{12} + A_{41} + \Delta U_{41}} =  \\
     &= \frac{25P_0V_0}{30P_0V_0 + \frac 32 \cdot 30 \cdot P_0V_0 + 0 + \frac 32 \cdot 5 \cdot P_0V_0} = \frac{25}{30 + \frac 32 \cdot 30 + \frac 32 \cdot 5} = \frac{10}{33} \approx 0{,}303.
     \\
    \eta_\text{Карно} &= 1 - \frac{T_\text{х}}{T_\text{н}} = 1 - \frac{T_\text{4}}{T_\text{2}} = 1 - \frac{\frac{P_4V_4}{\nu R}}{\frac{P_2V_2}{\nu R}} = 1 - \frac{P_4V_4}{P_2V_2} = 1 - \frac{P_0V_0}{6P_0 \cdot 6V_0} = 1 - \frac 1{6 \cdot 6}  = \frac{35}{36} \approx 0{,}972.
    \end{align*}
}
\solutionspace{360pt}

\tasknumber{2}%
\task{%
    Порция идеального одноатомного газа перешла из состояния 1 в состояние 2: $P_1 = 3\,\text{МПа}$, $V_1 = 5\,\text{л}$, $P_2 = 2{,}5\,\text{МПа}$, $V_2 = 2\,\text{л}$.
    Известно, что в $PV$-координатах график процесса 12 представляет собой отрезок прямой.
    Определите,
    \begin{itemize}
        \item какую работу при этом совершил газ,
        \item чему равно изменение внутренней энергии газа,
        \item сколько теплоты подвели к нему в этом процессе?
    \end{itemize}
    При решении обратите внимание на знаки искомых величин.
}
\answer{%
    \begin{align*}
    P_1V_1 &= \nu R T_1, P_2V_2 = \nu R T_2, \\
    \Delta U &= U_2-U_1 = \frac 32 \nu R T_2- \frac 32 \nu R T_1 = \frac 32 P_2 V_2 - \frac 32 P_1 V_1= \frac 32 \cdot \cbr{2{,}5\,\text{МПа} \cdot 2\,\text{л} - 3\,\text{МПа} \cdot 5\,\text{л}} = -15{,}0000\,\text{кДж}.
    \\
    A_\text{газа} &= \frac{P_2 + P_1} 2 \cdot (V_2 - V_1) = \frac{2{,}5\,\text{МПа} + 3\,\text{МПа}} 2 \cdot (2\,\text{л} - 5\,\text{л}) = -8{,}250\,\text{кДж}, \\
    Q &= A_\text{газа} + \Delta U = \frac 32 (P_2 V_2 - P_1 V_1) + \frac{P_2 + P_1} 2 \cdot (V_2 - V_1) = -15{,}0000\,\text{кДж} -8{,}250\,\text{кДж} = -23{,}250\,\text{кДж}.
    \end{align*}
}
\solutionspace{150pt}

\tasknumber{3}%
\task{%
    Запишите формулы и рядом с каждой физичической величиной укажите её название и единицы измерения в СИ:
    \begin{enumerate}
        \item первое начало термодинамики,
        \item внутренняя энергия идеального одноатомного газа.
    \end{enumerate}
}

\variantsplitter

\addpersonalvariant{Дарья Кошман}

\tasknumber{1}%
\task{%
    Определите КПД цикла 12341, рабочим телом которого является идеальный одноатомный газ, если
    12 — изобарическое расширение газа в два раза,
    23 — изохорическое охлаждение газа, при котором температура уменьшается в два раза,
    34 — изобара, 41 — изохора.
    % Для этого:
    % \begin{enumerate}
    %     \item сделайте рисунок в $PV$-координатах,
    %     \item выберите удобные обозначения, чтобы не запутаться в множестве температур, давлений и объёмов,
    %     \item вычислите необходимые соотнощения между температурами, давлениями и объёмами
    %     (некоторые сразу видны по рисунку, некоторые — надо считать),
    %     \item определите для каждого участка поглощается или отдаётся тепло (и сколько именно:
    %     потребуется первое начало термодинамики, отдельный расчёт работ на участках через площади фигур и изменений внутренней энергии),
    %     \item вычислите полную работу газа в цикле,
    %     \item подставьте всё в формулу для КПД, упростите и доведите до ответа.
    % \end{enumerate}
    Определите КПД цикла Карно, температура нагревателя которого равна максимальной температуре в цикле 12341, а холодильника — минимальной.
    Ответы в обоих случаях оставьте точными в виде нескоратимой дроби, никаких округлений.
}
\answer{%
    \begin{align*}
    A_{12} &> 0, \Delta U_{12} > 0, \implies Q_{12} = A_{12} + \Delta U_{12} > 0, \\
    A_{23} &= 0, \Delta U_{23} < 0, \implies Q_{23} = A_{23} + \Delta U_{23} < 0, \\
    A_{34} &< 0, \Delta U_{34} < 0, \implies Q_{34} = A_{34} + \Delta U_{34} < 0, \\
    A_{41} &= 0, \Delta U_{41} > 0, \implies Q_{41} = A_{41} + \Delta U_{41} > 0.
    \\
    P_1V_1 &= \nu R T_1, P_2V_2 = \nu R T_2, P_3V_3 = \nu R T_3, P_4V_4 = \nu R T_4 \text{ — уравнения состояния идеального газа}, \\
    &\text{Пусть $P_0$, $V_0$, $T_0$ — давление, объём и температура в точке 4 (минимальные во всём цикле):} \\
    P_1 &= P_2, P_3 = P_4 = P_0, V_1 = V_4 = V_0, V_2 = V_3 = 2 V_1 = 2 V_0,, \text{остальные соотношения между объёмами и давлениями не даны, нужно считать} \\
    T_3 &= \frac{T_2}2 \text{(по условию)} \implies \frac{P_2}{P_3} = \frac{P_2 V_2}{P_3 V_3}= \frac{\nu R T_2}{\nu R T_3} = \frac{T_2}{T_3} = 2 \implies P_1 = P_2 = 2 P_0 \\
    A_\text{цикл} &= (2P_0 - P_0)(2V_0 - V_0) = 1P_0V_0, \\
    A_{12} &= 2P_0 \cdot (2V_0 - V_0) = 2P_0V_0, \\
    \Delta U_{12} &= \frac 32 \nu R T_2 - \frac 32 \nu R T_1 = \frac 32 P_2 V_2 - \frac 32 P_1 V_1 = \frac 32 \cdot 2 P_0 \cdot 2 V_0 -  \frac 32 \cdot 2 P_0 \cdot V_0 = \frac 32 \cdot 2 \cdot P_0V_0, \\
    \Delta U_{41} &= \frac 32 \nu R T_1 - \frac 32 \nu R T_4 = \frac 32 P_1 V_1 - \frac 32 P_4 V_4 = \frac 32 \cdot 2 P_0 V_0 - \frac 32 P_0 V_0 = \frac 32 \cdot 1 \cdot P_0V_0.
    \\
    \eta &= \frac{A_\text{цикл}}{Q_+} = \frac{A_\text{цикл}}{Q_{12} + Q_{41}}  = \frac{A_\text{цикл}}{A_{12} + \Delta U_{12} + A_{41} + \Delta U_{41}} =  \\
     &= \frac{1P_0V_0}{2P_0V_0 + \frac 32 \cdot 2 \cdot P_0V_0 + 0 + \frac 32 \cdot 1 \cdot P_0V_0} = \frac{1}{2 + \frac 32 \cdot 2 + \frac 32 \cdot 1} = \frac2{13} \approx 0{,}154.
     \\
    \eta_\text{Карно} &= 1 - \frac{T_\text{х}}{T_\text{н}} = 1 - \frac{T_\text{4}}{T_\text{2}} = 1 - \frac{\frac{P_4V_4}{\nu R}}{\frac{P_2V_2}{\nu R}} = 1 - \frac{P_4V_4}{P_2V_2} = 1 - \frac{P_0V_0}{2P_0 \cdot 2V_0} = 1 - \frac 1{2 \cdot 2}  = \frac34 \approx 0{,}750.
    \end{align*}
}
\solutionspace{360pt}

\tasknumber{2}%
\task{%
    Порция идеального одноатомного газа перешла из состояния 1 в состояние 2: $P_1 = 4\,\text{МПа}$, $V_1 = 5\,\text{л}$, $P_2 = 4{,}5\,\text{МПа}$, $V_2 = 2\,\text{л}$.
    Известно, что в $PV$-координатах график процесса 12 представляет собой отрезок прямой.
    Определите,
    \begin{itemize}
        \item какую работу при этом совершил газ,
        \item чему равно изменение внутренней энергии газа,
        \item сколько теплоты подвели к нему в этом процессе?
    \end{itemize}
    При решении обратите внимание на знаки искомых величин.
}
\answer{%
    \begin{align*}
    P_1V_1 &= \nu R T_1, P_2V_2 = \nu R T_2, \\
    \Delta U &= U_2-U_1 = \frac 32 \nu R T_2- \frac 32 \nu R T_1 = \frac 32 P_2 V_2 - \frac 32 P_1 V_1= \frac 32 \cdot \cbr{4{,}5\,\text{МПа} \cdot 2\,\text{л} - 4\,\text{МПа} \cdot 5\,\text{л}} = -16{,}5000\,\text{кДж}.
    \\
    A_\text{газа} &= \frac{P_2 + P_1} 2 \cdot (V_2 - V_1) = \frac{4{,}5\,\text{МПа} + 4\,\text{МПа}} 2 \cdot (2\,\text{л} - 5\,\text{л}) = -12{,}7500\,\text{кДж}, \\
    Q &= A_\text{газа} + \Delta U = \frac 32 (P_2 V_2 - P_1 V_1) + \frac{P_2 + P_1} 2 \cdot (V_2 - V_1) = -16{,}5000\,\text{кДж} -12{,}7500\,\text{кДж} = -29{,}250\,\text{кДж}.
    \end{align*}
}
\solutionspace{150pt}

\tasknumber{3}%
\task{%
    Запишите формулы и рядом с каждой физичической величиной укажите её название и единицы измерения в СИ:
    \begin{enumerate}
        \item первое начало термодинамики,
        \item внутренняя энергия идеального одноатомного газа.
    \end{enumerate}
}

\variantsplitter

\addpersonalvariant{Анна Кузьмичёва}

\tasknumber{1}%
\task{%
    Определите КПД цикла 12341, рабочим телом которого является идеальный одноатомный газ, если
    12 — изобарическое расширение газа в четыре раза,
    23 — изохорическое охлаждение газа, при котором температура уменьшается в пять раз,
    34 — изобара, 41 — изохора.
    % Для этого:
    % \begin{enumerate}
    %     \item сделайте рисунок в $PV$-координатах,
    %     \item выберите удобные обозначения, чтобы не запутаться в множестве температур, давлений и объёмов,
    %     \item вычислите необходимые соотнощения между температурами, давлениями и объёмами
    %     (некоторые сразу видны по рисунку, некоторые — надо считать),
    %     \item определите для каждого участка поглощается или отдаётся тепло (и сколько именно:
    %     потребуется первое начало термодинамики, отдельный расчёт работ на участках через площади фигур и изменений внутренней энергии),
    %     \item вычислите полную работу газа в цикле,
    %     \item подставьте всё в формулу для КПД, упростите и доведите до ответа.
    % \end{enumerate}
    Определите КПД цикла Карно, температура нагревателя которого равна максимальной температуре в цикле 12341, а холодильника — минимальной.
    Ответы в обоих случаях оставьте точными в виде нескоратимой дроби, никаких округлений.
}
\answer{%
    \begin{align*}
    A_{12} &> 0, \Delta U_{12} > 0, \implies Q_{12} = A_{12} + \Delta U_{12} > 0, \\
    A_{23} &= 0, \Delta U_{23} < 0, \implies Q_{23} = A_{23} + \Delta U_{23} < 0, \\
    A_{34} &< 0, \Delta U_{34} < 0, \implies Q_{34} = A_{34} + \Delta U_{34} < 0, \\
    A_{41} &= 0, \Delta U_{41} > 0, \implies Q_{41} = A_{41} + \Delta U_{41} > 0.
    \\
    P_1V_1 &= \nu R T_1, P_2V_2 = \nu R T_2, P_3V_3 = \nu R T_3, P_4V_4 = \nu R T_4 \text{ — уравнения состояния идеального газа}, \\
    &\text{Пусть $P_0$, $V_0$, $T_0$ — давление, объём и температура в точке 4 (минимальные во всём цикле):} \\
    P_1 &= P_2, P_3 = P_4 = P_0, V_1 = V_4 = V_0, V_2 = V_3 = 4 V_1 = 4 V_0,, \text{остальные соотношения между объёмами и давлениями не даны, нужно считать} \\
    T_3 &= \frac{T_2}5 \text{(по условию)} \implies \frac{P_2}{P_3} = \frac{P_2 V_2}{P_3 V_3}= \frac{\nu R T_2}{\nu R T_3} = \frac{T_2}{T_3} = 5 \implies P_1 = P_2 = 5 P_0 \\
    A_\text{цикл} &= (5P_0 - P_0)(4V_0 - V_0) = 12P_0V_0, \\
    A_{12} &= 5P_0 \cdot (4V_0 - V_0) = 15P_0V_0, \\
    \Delta U_{12} &= \frac 32 \nu R T_2 - \frac 32 \nu R T_1 = \frac 32 P_2 V_2 - \frac 32 P_1 V_1 = \frac 32 \cdot 5 P_0 \cdot 4 V_0 -  \frac 32 \cdot 5 P_0 \cdot V_0 = \frac 32 \cdot 15 \cdot P_0V_0, \\
    \Delta U_{41} &= \frac 32 \nu R T_1 - \frac 32 \nu R T_4 = \frac 32 P_1 V_1 - \frac 32 P_4 V_4 = \frac 32 \cdot 5 P_0 V_0 - \frac 32 P_0 V_0 = \frac 32 \cdot 4 \cdot P_0V_0.
    \\
    \eta &= \frac{A_\text{цикл}}{Q_+} = \frac{A_\text{цикл}}{Q_{12} + Q_{41}}  = \frac{A_\text{цикл}}{A_{12} + \Delta U_{12} + A_{41} + \Delta U_{41}} =  \\
     &= \frac{12P_0V_0}{15P_0V_0 + \frac 32 \cdot 15 \cdot P_0V_0 + 0 + \frac 32 \cdot 4 \cdot P_0V_0} = \frac{12}{15 + \frac 32 \cdot 15 + \frac 32 \cdot 4} = \frac8{29} \approx 0{,}276.
     \\
    \eta_\text{Карно} &= 1 - \frac{T_\text{х}}{T_\text{н}} = 1 - \frac{T_\text{4}}{T_\text{2}} = 1 - \frac{\frac{P_4V_4}{\nu R}}{\frac{P_2V_2}{\nu R}} = 1 - \frac{P_4V_4}{P_2V_2} = 1 - \frac{P_0V_0}{5P_0 \cdot 4V_0} = 1 - \frac 1{5 \cdot 4}  = \frac{19}{20} \approx 0{,}950.
    \end{align*}
}
\solutionspace{360pt}

\tasknumber{2}%
\task{%
    Порция идеального одноатомного газа перешла из состояния 1 в состояние 2: $P_1 = 2\,\text{МПа}$, $V_1 = 5\,\text{л}$, $P_2 = 2{,}5\,\text{МПа}$, $V_2 = 8\,\text{л}$.
    Известно, что в $PV$-координатах график процесса 12 представляет собой отрезок прямой.
    Определите,
    \begin{itemize}
        \item какую работу при этом совершил газ,
        \item чему равно изменение внутренней энергии газа,
        \item сколько теплоты подвели к нему в этом процессе?
    \end{itemize}
    При решении обратите внимание на знаки искомых величин.
}
\answer{%
    \begin{align*}
    P_1V_1 &= \nu R T_1, P_2V_2 = \nu R T_2, \\
    \Delta U &= U_2-U_1 = \frac 32 \nu R T_2- \frac 32 \nu R T_1 = \frac 32 P_2 V_2 - \frac 32 P_1 V_1= \frac 32 \cdot \cbr{2{,}5\,\text{МПа} \cdot 8\,\text{л} - 2\,\text{МПа} \cdot 5\,\text{л}} = 15{,}00\,\text{кДж}.
    \\
    A_\text{газа} &= \frac{P_2 + P_1} 2 \cdot (V_2 - V_1) = \frac{2{,}5\,\text{МПа} + 2\,\text{МПа}} 2 \cdot (8\,\text{л} - 5\,\text{л}) = 6{,}75\,\text{кДж}, \\
    Q &= A_\text{газа} + \Delta U = \frac 32 (P_2 V_2 - P_1 V_1) + \frac{P_2 + P_1} 2 \cdot (V_2 - V_1) = 15{,}00\,\text{кДж} + 6{,}75\,\text{кДж} = 21{,}75\,\text{кДж}.
    \end{align*}
}
\solutionspace{150pt}

\tasknumber{3}%
\task{%
    Запишите формулы и рядом с каждой физичической величиной укажите её название и единицы измерения в СИ:
    \begin{enumerate}
        \item первое начало термодинамики,
        \item внутренняя энергия идеального одноатомного газа.
    \end{enumerate}
}

\variantsplitter

\addpersonalvariant{Алёна Куприянова}

\tasknumber{1}%
\task{%
    Определите КПД цикла 12341, рабочим телом которого является идеальный одноатомный газ, если
    12 — изобарическое расширение газа в четыре раза,
    23 — изохорическое охлаждение газа, при котором температура уменьшается в пять раз,
    34 — изобара, 41 — изохора.
    % Для этого:
    % \begin{enumerate}
    %     \item сделайте рисунок в $PV$-координатах,
    %     \item выберите удобные обозначения, чтобы не запутаться в множестве температур, давлений и объёмов,
    %     \item вычислите необходимые соотнощения между температурами, давлениями и объёмами
    %     (некоторые сразу видны по рисунку, некоторые — надо считать),
    %     \item определите для каждого участка поглощается или отдаётся тепло (и сколько именно:
    %     потребуется первое начало термодинамики, отдельный расчёт работ на участках через площади фигур и изменений внутренней энергии),
    %     \item вычислите полную работу газа в цикле,
    %     \item подставьте всё в формулу для КПД, упростите и доведите до ответа.
    % \end{enumerate}
    Определите КПД цикла Карно, температура нагревателя которого равна максимальной температуре в цикле 12341, а холодильника — минимальной.
    Ответы в обоих случаях оставьте точными в виде нескоратимой дроби, никаких округлений.
}
\answer{%
    \begin{align*}
    A_{12} &> 0, \Delta U_{12} > 0, \implies Q_{12} = A_{12} + \Delta U_{12} > 0, \\
    A_{23} &= 0, \Delta U_{23} < 0, \implies Q_{23} = A_{23} + \Delta U_{23} < 0, \\
    A_{34} &< 0, \Delta U_{34} < 0, \implies Q_{34} = A_{34} + \Delta U_{34} < 0, \\
    A_{41} &= 0, \Delta U_{41} > 0, \implies Q_{41} = A_{41} + \Delta U_{41} > 0.
    \\
    P_1V_1 &= \nu R T_1, P_2V_2 = \nu R T_2, P_3V_3 = \nu R T_3, P_4V_4 = \nu R T_4 \text{ — уравнения состояния идеального газа}, \\
    &\text{Пусть $P_0$, $V_0$, $T_0$ — давление, объём и температура в точке 4 (минимальные во всём цикле):} \\
    P_1 &= P_2, P_3 = P_4 = P_0, V_1 = V_4 = V_0, V_2 = V_3 = 4 V_1 = 4 V_0,, \text{остальные соотношения между объёмами и давлениями не даны, нужно считать} \\
    T_3 &= \frac{T_2}5 \text{(по условию)} \implies \frac{P_2}{P_3} = \frac{P_2 V_2}{P_3 V_3}= \frac{\nu R T_2}{\nu R T_3} = \frac{T_2}{T_3} = 5 \implies P_1 = P_2 = 5 P_0 \\
    A_\text{цикл} &= (5P_0 - P_0)(4V_0 - V_0) = 12P_0V_0, \\
    A_{12} &= 5P_0 \cdot (4V_0 - V_0) = 15P_0V_0, \\
    \Delta U_{12} &= \frac 32 \nu R T_2 - \frac 32 \nu R T_1 = \frac 32 P_2 V_2 - \frac 32 P_1 V_1 = \frac 32 \cdot 5 P_0 \cdot 4 V_0 -  \frac 32 \cdot 5 P_0 \cdot V_0 = \frac 32 \cdot 15 \cdot P_0V_0, \\
    \Delta U_{41} &= \frac 32 \nu R T_1 - \frac 32 \nu R T_4 = \frac 32 P_1 V_1 - \frac 32 P_4 V_4 = \frac 32 \cdot 5 P_0 V_0 - \frac 32 P_0 V_0 = \frac 32 \cdot 4 \cdot P_0V_0.
    \\
    \eta &= \frac{A_\text{цикл}}{Q_+} = \frac{A_\text{цикл}}{Q_{12} + Q_{41}}  = \frac{A_\text{цикл}}{A_{12} + \Delta U_{12} + A_{41} + \Delta U_{41}} =  \\
     &= \frac{12P_0V_0}{15P_0V_0 + \frac 32 \cdot 15 \cdot P_0V_0 + 0 + \frac 32 \cdot 4 \cdot P_0V_0} = \frac{12}{15 + \frac 32 \cdot 15 + \frac 32 \cdot 4} = \frac8{29} \approx 0{,}276.
     \\
    \eta_\text{Карно} &= 1 - \frac{T_\text{х}}{T_\text{н}} = 1 - \frac{T_\text{4}}{T_\text{2}} = 1 - \frac{\frac{P_4V_4}{\nu R}}{\frac{P_2V_2}{\nu R}} = 1 - \frac{P_4V_4}{P_2V_2} = 1 - \frac{P_0V_0}{5P_0 \cdot 4V_0} = 1 - \frac 1{5 \cdot 4}  = \frac{19}{20} \approx 0{,}950.
    \end{align*}
}
\solutionspace{360pt}

\tasknumber{2}%
\task{%
    Порция идеального одноатомного газа перешла из состояния 1 в состояние 2: $P_1 = 2\,\text{МПа}$, $V_1 = 7\,\text{л}$, $P_2 = 3{,}5\,\text{МПа}$, $V_2 = 4\,\text{л}$.
    Известно, что в $PV$-координатах график процесса 12 представляет собой отрезок прямой.
    Определите,
    \begin{itemize}
        \item какую работу при этом совершил газ,
        \item чему равно изменение внутренней энергии газа,
        \item сколько теплоты подвели к нему в этом процессе?
    \end{itemize}
    При решении обратите внимание на знаки искомых величин.
}
\answer{%
    \begin{align*}
    P_1V_1 &= \nu R T_1, P_2V_2 = \nu R T_2, \\
    \Delta U &= U_2-U_1 = \frac 32 \nu R T_2- \frac 32 \nu R T_1 = \frac 32 P_2 V_2 - \frac 32 P_1 V_1= \frac 32 \cdot \cbr{3{,}5\,\text{МПа} \cdot 4\,\text{л} - 2\,\text{МПа} \cdot 7\,\text{л}} = 0\,\text{кДж}.
    \\
    A_\text{газа} &= \frac{P_2 + P_1} 2 \cdot (V_2 - V_1) = \frac{3{,}5\,\text{МПа} + 2\,\text{МПа}} 2 \cdot (4\,\text{л} - 7\,\text{л}) = -8{,}250\,\text{кДж}, \\
    Q &= A_\text{газа} + \Delta U = \frac 32 (P_2 V_2 - P_1 V_1) + \frac{P_2 + P_1} 2 \cdot (V_2 - V_1) = 0\,\text{кДж} -8{,}250\,\text{кДж} = -8{,}250\,\text{кДж}.
    \end{align*}
}
\solutionspace{150pt}

\tasknumber{3}%
\task{%
    Запишите формулы и рядом с каждой физичической величиной укажите её название и единицы измерения в СИ:
    \begin{enumerate}
        \item первое начало термодинамики,
        \item внутренняя энергия идеального одноатомного газа.
    \end{enumerate}
}

\variantsplitter

\addpersonalvariant{Ярослав Лавровский}

\tasknumber{1}%
\task{%
    Определите КПД цикла 12341, рабочим телом которого является идеальный одноатомный газ, если
    12 — изобарическое расширение газа в пять раз,
    23 — изохорическое охлаждение газа, при котором температура уменьшается в четыре раза,
    34 — изобара, 41 — изохора.
    % Для этого:
    % \begin{enumerate}
    %     \item сделайте рисунок в $PV$-координатах,
    %     \item выберите удобные обозначения, чтобы не запутаться в множестве температур, давлений и объёмов,
    %     \item вычислите необходимые соотнощения между температурами, давлениями и объёмами
    %     (некоторые сразу видны по рисунку, некоторые — надо считать),
    %     \item определите для каждого участка поглощается или отдаётся тепло (и сколько именно:
    %     потребуется первое начало термодинамики, отдельный расчёт работ на участках через площади фигур и изменений внутренней энергии),
    %     \item вычислите полную работу газа в цикле,
    %     \item подставьте всё в формулу для КПД, упростите и доведите до ответа.
    % \end{enumerate}
    Определите КПД цикла Карно, температура нагревателя которого равна максимальной температуре в цикле 12341, а холодильника — минимальной.
    Ответы в обоих случаях оставьте точными в виде нескоратимой дроби, никаких округлений.
}
\answer{%
    \begin{align*}
    A_{12} &> 0, \Delta U_{12} > 0, \implies Q_{12} = A_{12} + \Delta U_{12} > 0, \\
    A_{23} &= 0, \Delta U_{23} < 0, \implies Q_{23} = A_{23} + \Delta U_{23} < 0, \\
    A_{34} &< 0, \Delta U_{34} < 0, \implies Q_{34} = A_{34} + \Delta U_{34} < 0, \\
    A_{41} &= 0, \Delta U_{41} > 0, \implies Q_{41} = A_{41} + \Delta U_{41} > 0.
    \\
    P_1V_1 &= \nu R T_1, P_2V_2 = \nu R T_2, P_3V_3 = \nu R T_3, P_4V_4 = \nu R T_4 \text{ — уравнения состояния идеального газа}, \\
    &\text{Пусть $P_0$, $V_0$, $T_0$ — давление, объём и температура в точке 4 (минимальные во всём цикле):} \\
    P_1 &= P_2, P_3 = P_4 = P_0, V_1 = V_4 = V_0, V_2 = V_3 = 5 V_1 = 5 V_0,, \text{остальные соотношения между объёмами и давлениями не даны, нужно считать} \\
    T_3 &= \frac{T_2}4 \text{(по условию)} \implies \frac{P_2}{P_3} = \frac{P_2 V_2}{P_3 V_3}= \frac{\nu R T_2}{\nu R T_3} = \frac{T_2}{T_3} = 4 \implies P_1 = P_2 = 4 P_0 \\
    A_\text{цикл} &= (4P_0 - P_0)(5V_0 - V_0) = 12P_0V_0, \\
    A_{12} &= 4P_0 \cdot (5V_0 - V_0) = 16P_0V_0, \\
    \Delta U_{12} &= \frac 32 \nu R T_2 - \frac 32 \nu R T_1 = \frac 32 P_2 V_2 - \frac 32 P_1 V_1 = \frac 32 \cdot 4 P_0 \cdot 5 V_0 -  \frac 32 \cdot 4 P_0 \cdot V_0 = \frac 32 \cdot 16 \cdot P_0V_0, \\
    \Delta U_{41} &= \frac 32 \nu R T_1 - \frac 32 \nu R T_4 = \frac 32 P_1 V_1 - \frac 32 P_4 V_4 = \frac 32 \cdot 4 P_0 V_0 - \frac 32 P_0 V_0 = \frac 32 \cdot 3 \cdot P_0V_0.
    \\
    \eta &= \frac{A_\text{цикл}}{Q_+} = \frac{A_\text{цикл}}{Q_{12} + Q_{41}}  = \frac{A_\text{цикл}}{A_{12} + \Delta U_{12} + A_{41} + \Delta U_{41}} =  \\
     &= \frac{12P_0V_0}{16P_0V_0 + \frac 32 \cdot 16 \cdot P_0V_0 + 0 + \frac 32 \cdot 3 \cdot P_0V_0} = \frac{12}{16 + \frac 32 \cdot 16 + \frac 32 \cdot 3} = \frac{24}{89} \approx 0{,}270.
     \\
    \eta_\text{Карно} &= 1 - \frac{T_\text{х}}{T_\text{н}} = 1 - \frac{T_\text{4}}{T_\text{2}} = 1 - \frac{\frac{P_4V_4}{\nu R}}{\frac{P_2V_2}{\nu R}} = 1 - \frac{P_4V_4}{P_2V_2} = 1 - \frac{P_0V_0}{4P_0 \cdot 5V_0} = 1 - \frac 1{4 \cdot 5}  = \frac{19}{20} \approx 0{,}950.
    \end{align*}
}
\solutionspace{360pt}

\tasknumber{2}%
\task{%
    Порция идеального одноатомного газа перешла из состояния 1 в состояние 2: $P_1 = 2\,\text{МПа}$, $V_1 = 5\,\text{л}$, $P_2 = 4{,}5\,\text{МПа}$, $V_2 = 2\,\text{л}$.
    Известно, что в $PV$-координатах график процесса 12 представляет собой отрезок прямой.
    Определите,
    \begin{itemize}
        \item какую работу при этом совершил газ,
        \item чему равно изменение внутренней энергии газа,
        \item сколько теплоты подвели к нему в этом процессе?
    \end{itemize}
    При решении обратите внимание на знаки искомых величин.
}
\answer{%
    \begin{align*}
    P_1V_1 &= \nu R T_1, P_2V_2 = \nu R T_2, \\
    \Delta U &= U_2-U_1 = \frac 32 \nu R T_2- \frac 32 \nu R T_1 = \frac 32 P_2 V_2 - \frac 32 P_1 V_1= \frac 32 \cdot \cbr{4{,}5\,\text{МПа} \cdot 2\,\text{л} - 2\,\text{МПа} \cdot 5\,\text{л}} = -1{,}5000\,\text{кДж}.
    \\
    A_\text{газа} &= \frac{P_2 + P_1} 2 \cdot (V_2 - V_1) = \frac{4{,}5\,\text{МПа} + 2\,\text{МПа}} 2 \cdot (2\,\text{л} - 5\,\text{л}) = -9{,}750\,\text{кДж}, \\
    Q &= A_\text{газа} + \Delta U = \frac 32 (P_2 V_2 - P_1 V_1) + \frac{P_2 + P_1} 2 \cdot (V_2 - V_1) = -1{,}5000\,\text{кДж} -9{,}750\,\text{кДж} = -11{,}2500\,\text{кДж}.
    \end{align*}
}
\solutionspace{150pt}

\tasknumber{3}%
\task{%
    Запишите формулы и рядом с каждой физичической величиной укажите её название и единицы измерения в СИ:
    \begin{enumerate}
        \item первое начало термодинамики,
        \item внутренняя энергия идеального одноатомного газа.
    \end{enumerate}
}

\variantsplitter

\addpersonalvariant{Анастасия Ламанова}

\tasknumber{1}%
\task{%
    Определите КПД цикла 12341, рабочим телом которого является идеальный одноатомный газ, если
    12 — изобарическое расширение газа в шесть раз,
    23 — изохорическое охлаждение газа, при котором температура уменьшается в два раза,
    34 — изобара, 41 — изохора.
    % Для этого:
    % \begin{enumerate}
    %     \item сделайте рисунок в $PV$-координатах,
    %     \item выберите удобные обозначения, чтобы не запутаться в множестве температур, давлений и объёмов,
    %     \item вычислите необходимые соотнощения между температурами, давлениями и объёмами
    %     (некоторые сразу видны по рисунку, некоторые — надо считать),
    %     \item определите для каждого участка поглощается или отдаётся тепло (и сколько именно:
    %     потребуется первое начало термодинамики, отдельный расчёт работ на участках через площади фигур и изменений внутренней энергии),
    %     \item вычислите полную работу газа в цикле,
    %     \item подставьте всё в формулу для КПД, упростите и доведите до ответа.
    % \end{enumerate}
    Определите КПД цикла Карно, температура нагревателя которого равна максимальной температуре в цикле 12341, а холодильника — минимальной.
    Ответы в обоих случаях оставьте точными в виде нескоратимой дроби, никаких округлений.
}
\answer{%
    \begin{align*}
    A_{12} &> 0, \Delta U_{12} > 0, \implies Q_{12} = A_{12} + \Delta U_{12} > 0, \\
    A_{23} &= 0, \Delta U_{23} < 0, \implies Q_{23} = A_{23} + \Delta U_{23} < 0, \\
    A_{34} &< 0, \Delta U_{34} < 0, \implies Q_{34} = A_{34} + \Delta U_{34} < 0, \\
    A_{41} &= 0, \Delta U_{41} > 0, \implies Q_{41} = A_{41} + \Delta U_{41} > 0.
    \\
    P_1V_1 &= \nu R T_1, P_2V_2 = \nu R T_2, P_3V_3 = \nu R T_3, P_4V_4 = \nu R T_4 \text{ — уравнения состояния идеального газа}, \\
    &\text{Пусть $P_0$, $V_0$, $T_0$ — давление, объём и температура в точке 4 (минимальные во всём цикле):} \\
    P_1 &= P_2, P_3 = P_4 = P_0, V_1 = V_4 = V_0, V_2 = V_3 = 6 V_1 = 6 V_0,, \text{остальные соотношения между объёмами и давлениями не даны, нужно считать} \\
    T_3 &= \frac{T_2}2 \text{(по условию)} \implies \frac{P_2}{P_3} = \frac{P_2 V_2}{P_3 V_3}= \frac{\nu R T_2}{\nu R T_3} = \frac{T_2}{T_3} = 2 \implies P_1 = P_2 = 2 P_0 \\
    A_\text{цикл} &= (2P_0 - P_0)(6V_0 - V_0) = 5P_0V_0, \\
    A_{12} &= 2P_0 \cdot (6V_0 - V_0) = 10P_0V_0, \\
    \Delta U_{12} &= \frac 32 \nu R T_2 - \frac 32 \nu R T_1 = \frac 32 P_2 V_2 - \frac 32 P_1 V_1 = \frac 32 \cdot 2 P_0 \cdot 6 V_0 -  \frac 32 \cdot 2 P_0 \cdot V_0 = \frac 32 \cdot 10 \cdot P_0V_0, \\
    \Delta U_{41} &= \frac 32 \nu R T_1 - \frac 32 \nu R T_4 = \frac 32 P_1 V_1 - \frac 32 P_4 V_4 = \frac 32 \cdot 2 P_0 V_0 - \frac 32 P_0 V_0 = \frac 32 \cdot 1 \cdot P_0V_0.
    \\
    \eta &= \frac{A_\text{цикл}}{Q_+} = \frac{A_\text{цикл}}{Q_{12} + Q_{41}}  = \frac{A_\text{цикл}}{A_{12} + \Delta U_{12} + A_{41} + \Delta U_{41}} =  \\
     &= \frac{5P_0V_0}{10P_0V_0 + \frac 32 \cdot 10 \cdot P_0V_0 + 0 + \frac 32 \cdot 1 \cdot P_0V_0} = \frac{5}{10 + \frac 32 \cdot 10 + \frac 32 \cdot 1} = \frac{10}{53} \approx 0{,}189.
     \\
    \eta_\text{Карно} &= 1 - \frac{T_\text{х}}{T_\text{н}} = 1 - \frac{T_\text{4}}{T_\text{2}} = 1 - \frac{\frac{P_4V_4}{\nu R}}{\frac{P_2V_2}{\nu R}} = 1 - \frac{P_4V_4}{P_2V_2} = 1 - \frac{P_0V_0}{2P_0 \cdot 6V_0} = 1 - \frac 1{2 \cdot 6}  = \frac{11}{12} \approx 0{,}917.
    \end{align*}
}
\solutionspace{360pt}

\tasknumber{2}%
\task{%
    Порция идеального одноатомного газа перешла из состояния 1 в состояние 2: $P_1 = 3\,\text{МПа}$, $V_1 = 5\,\text{л}$, $P_2 = 3{,}5\,\text{МПа}$, $V_2 = 6\,\text{л}$.
    Известно, что в $PV$-координатах график процесса 12 представляет собой отрезок прямой.
    Определите,
    \begin{itemize}
        \item какую работу при этом совершил газ,
        \item чему равно изменение внутренней энергии газа,
        \item сколько теплоты подвели к нему в этом процессе?
    \end{itemize}
    При решении обратите внимание на знаки искомых величин.
}
\answer{%
    \begin{align*}
    P_1V_1 &= \nu R T_1, P_2V_2 = \nu R T_2, \\
    \Delta U &= U_2-U_1 = \frac 32 \nu R T_2- \frac 32 \nu R T_1 = \frac 32 P_2 V_2 - \frac 32 P_1 V_1= \frac 32 \cdot \cbr{3{,}5\,\text{МПа} \cdot 6\,\text{л} - 3\,\text{МПа} \cdot 5\,\text{л}} = 9{,}00\,\text{кДж}.
    \\
    A_\text{газа} &= \frac{P_2 + P_1} 2 \cdot (V_2 - V_1) = \frac{3{,}5\,\text{МПа} + 3\,\text{МПа}} 2 \cdot (6\,\text{л} - 5\,\text{л}) = 3{,}25\,\text{кДж}, \\
    Q &= A_\text{газа} + \Delta U = \frac 32 (P_2 V_2 - P_1 V_1) + \frac{P_2 + P_1} 2 \cdot (V_2 - V_1) = 9{,}00\,\text{кДж} + 3{,}25\,\text{кДж} = 12{,}25\,\text{кДж}.
    \end{align*}
}
\solutionspace{150pt}

\tasknumber{3}%
\task{%
    Запишите формулы и рядом с каждой физичической величиной укажите её название и единицы измерения в СИ:
    \begin{enumerate}
        \item первое начало термодинамики,
        \item внутренняя энергия идеального одноатомного газа.
    \end{enumerate}
}

\variantsplitter

\addpersonalvariant{Виктория Легонькова}

\tasknumber{1}%
\task{%
    Определите КПД цикла 12341, рабочим телом которого является идеальный одноатомный газ, если
    12 — изобарическое расширение газа в три раза,
    23 — изохорическое охлаждение газа, при котором температура уменьшается в четыре раза,
    34 — изобара, 41 — изохора.
    % Для этого:
    % \begin{enumerate}
    %     \item сделайте рисунок в $PV$-координатах,
    %     \item выберите удобные обозначения, чтобы не запутаться в множестве температур, давлений и объёмов,
    %     \item вычислите необходимые соотнощения между температурами, давлениями и объёмами
    %     (некоторые сразу видны по рисунку, некоторые — надо считать),
    %     \item определите для каждого участка поглощается или отдаётся тепло (и сколько именно:
    %     потребуется первое начало термодинамики, отдельный расчёт работ на участках через площади фигур и изменений внутренней энергии),
    %     \item вычислите полную работу газа в цикле,
    %     \item подставьте всё в формулу для КПД, упростите и доведите до ответа.
    % \end{enumerate}
    Определите КПД цикла Карно, температура нагревателя которого равна максимальной температуре в цикле 12341, а холодильника — минимальной.
    Ответы в обоих случаях оставьте точными в виде нескоратимой дроби, никаких округлений.
}
\answer{%
    \begin{align*}
    A_{12} &> 0, \Delta U_{12} > 0, \implies Q_{12} = A_{12} + \Delta U_{12} > 0, \\
    A_{23} &= 0, \Delta U_{23} < 0, \implies Q_{23} = A_{23} + \Delta U_{23} < 0, \\
    A_{34} &< 0, \Delta U_{34} < 0, \implies Q_{34} = A_{34} + \Delta U_{34} < 0, \\
    A_{41} &= 0, \Delta U_{41} > 0, \implies Q_{41} = A_{41} + \Delta U_{41} > 0.
    \\
    P_1V_1 &= \nu R T_1, P_2V_2 = \nu R T_2, P_3V_3 = \nu R T_3, P_4V_4 = \nu R T_4 \text{ — уравнения состояния идеального газа}, \\
    &\text{Пусть $P_0$, $V_0$, $T_0$ — давление, объём и температура в точке 4 (минимальные во всём цикле):} \\
    P_1 &= P_2, P_3 = P_4 = P_0, V_1 = V_4 = V_0, V_2 = V_3 = 3 V_1 = 3 V_0,, \text{остальные соотношения между объёмами и давлениями не даны, нужно считать} \\
    T_3 &= \frac{T_2}4 \text{(по условию)} \implies \frac{P_2}{P_3} = \frac{P_2 V_2}{P_3 V_3}= \frac{\nu R T_2}{\nu R T_3} = \frac{T_2}{T_3} = 4 \implies P_1 = P_2 = 4 P_0 \\
    A_\text{цикл} &= (4P_0 - P_0)(3V_0 - V_0) = 6P_0V_0, \\
    A_{12} &= 4P_0 \cdot (3V_0 - V_0) = 8P_0V_0, \\
    \Delta U_{12} &= \frac 32 \nu R T_2 - \frac 32 \nu R T_1 = \frac 32 P_2 V_2 - \frac 32 P_1 V_1 = \frac 32 \cdot 4 P_0 \cdot 3 V_0 -  \frac 32 \cdot 4 P_0 \cdot V_0 = \frac 32 \cdot 8 \cdot P_0V_0, \\
    \Delta U_{41} &= \frac 32 \nu R T_1 - \frac 32 \nu R T_4 = \frac 32 P_1 V_1 - \frac 32 P_4 V_4 = \frac 32 \cdot 4 P_0 V_0 - \frac 32 P_0 V_0 = \frac 32 \cdot 3 \cdot P_0V_0.
    \\
    \eta &= \frac{A_\text{цикл}}{Q_+} = \frac{A_\text{цикл}}{Q_{12} + Q_{41}}  = \frac{A_\text{цикл}}{A_{12} + \Delta U_{12} + A_{41} + \Delta U_{41}} =  \\
     &= \frac{6P_0V_0}{8P_0V_0 + \frac 32 \cdot 8 \cdot P_0V_0 + 0 + \frac 32 \cdot 3 \cdot P_0V_0} = \frac{6}{8 + \frac 32 \cdot 8 + \frac 32 \cdot 3} = \frac{12}{49} \approx 0{,}245.
     \\
    \eta_\text{Карно} &= 1 - \frac{T_\text{х}}{T_\text{н}} = 1 - \frac{T_\text{4}}{T_\text{2}} = 1 - \frac{\frac{P_4V_4}{\nu R}}{\frac{P_2V_2}{\nu R}} = 1 - \frac{P_4V_4}{P_2V_2} = 1 - \frac{P_0V_0}{4P_0 \cdot 3V_0} = 1 - \frac 1{4 \cdot 3}  = \frac{11}{12} \approx 0{,}917.
    \end{align*}
}
\solutionspace{360pt}

\tasknumber{2}%
\task{%
    Порция идеального одноатомного газа перешла из состояния 1 в состояние 2: $P_1 = 2\,\text{МПа}$, $V_1 = 3\,\text{л}$, $P_2 = 3{,}5\,\text{МПа}$, $V_2 = 2\,\text{л}$.
    Известно, что в $PV$-координатах график процесса 12 представляет собой отрезок прямой.
    Определите,
    \begin{itemize}
        \item какую работу при этом совершил газ,
        \item чему равно изменение внутренней энергии газа,
        \item сколько теплоты подвели к нему в этом процессе?
    \end{itemize}
    При решении обратите внимание на знаки искомых величин.
}
\answer{%
    \begin{align*}
    P_1V_1 &= \nu R T_1, P_2V_2 = \nu R T_2, \\
    \Delta U &= U_2-U_1 = \frac 32 \nu R T_2- \frac 32 \nu R T_1 = \frac 32 P_2 V_2 - \frac 32 P_1 V_1= \frac 32 \cdot \cbr{3{,}5\,\text{МПа} \cdot 2\,\text{л} - 2\,\text{МПа} \cdot 3\,\text{л}} = 1{,}50\,\text{кДж}.
    \\
    A_\text{газа} &= \frac{P_2 + P_1} 2 \cdot (V_2 - V_1) = \frac{3{,}5\,\text{МПа} + 2\,\text{МПа}} 2 \cdot (2\,\text{л} - 3\,\text{л}) = -2{,}750\,\text{кДж}, \\
    Q &= A_\text{газа} + \Delta U = \frac 32 (P_2 V_2 - P_1 V_1) + \frac{P_2 + P_1} 2 \cdot (V_2 - V_1) = 1{,}50\,\text{кДж} -2{,}750\,\text{кДж} = -1{,}2500\,\text{кДж}.
    \end{align*}
}
\solutionspace{150pt}

\tasknumber{3}%
\task{%
    Запишите формулы и рядом с каждой физичической величиной укажите её название и единицы измерения в СИ:
    \begin{enumerate}
        \item первое начало термодинамики,
        \item внутренняя энергия идеального одноатомного газа.
    \end{enumerate}
}

\variantsplitter

\addpersonalvariant{Семён Мартынов}

\tasknumber{1}%
\task{%
    Определите КПД цикла 12341, рабочим телом которого является идеальный одноатомный газ, если
    12 — изобарическое расширение газа в два раза,
    23 — изохорическое охлаждение газа, при котором температура уменьшается в два раза,
    34 — изобара, 41 — изохора.
    % Для этого:
    % \begin{enumerate}
    %     \item сделайте рисунок в $PV$-координатах,
    %     \item выберите удобные обозначения, чтобы не запутаться в множестве температур, давлений и объёмов,
    %     \item вычислите необходимые соотнощения между температурами, давлениями и объёмами
    %     (некоторые сразу видны по рисунку, некоторые — надо считать),
    %     \item определите для каждого участка поглощается или отдаётся тепло (и сколько именно:
    %     потребуется первое начало термодинамики, отдельный расчёт работ на участках через площади фигур и изменений внутренней энергии),
    %     \item вычислите полную работу газа в цикле,
    %     \item подставьте всё в формулу для КПД, упростите и доведите до ответа.
    % \end{enumerate}
    Определите КПД цикла Карно, температура нагревателя которого равна максимальной температуре в цикле 12341, а холодильника — минимальной.
    Ответы в обоих случаях оставьте точными в виде нескоратимой дроби, никаких округлений.
}
\answer{%
    \begin{align*}
    A_{12} &> 0, \Delta U_{12} > 0, \implies Q_{12} = A_{12} + \Delta U_{12} > 0, \\
    A_{23} &= 0, \Delta U_{23} < 0, \implies Q_{23} = A_{23} + \Delta U_{23} < 0, \\
    A_{34} &< 0, \Delta U_{34} < 0, \implies Q_{34} = A_{34} + \Delta U_{34} < 0, \\
    A_{41} &= 0, \Delta U_{41} > 0, \implies Q_{41} = A_{41} + \Delta U_{41} > 0.
    \\
    P_1V_1 &= \nu R T_1, P_2V_2 = \nu R T_2, P_3V_3 = \nu R T_3, P_4V_4 = \nu R T_4 \text{ — уравнения состояния идеального газа}, \\
    &\text{Пусть $P_0$, $V_0$, $T_0$ — давление, объём и температура в точке 4 (минимальные во всём цикле):} \\
    P_1 &= P_2, P_3 = P_4 = P_0, V_1 = V_4 = V_0, V_2 = V_3 = 2 V_1 = 2 V_0,, \text{остальные соотношения между объёмами и давлениями не даны, нужно считать} \\
    T_3 &= \frac{T_2}2 \text{(по условию)} \implies \frac{P_2}{P_3} = \frac{P_2 V_2}{P_3 V_3}= \frac{\nu R T_2}{\nu R T_3} = \frac{T_2}{T_3} = 2 \implies P_1 = P_2 = 2 P_0 \\
    A_\text{цикл} &= (2P_0 - P_0)(2V_0 - V_0) = 1P_0V_0, \\
    A_{12} &= 2P_0 \cdot (2V_0 - V_0) = 2P_0V_0, \\
    \Delta U_{12} &= \frac 32 \nu R T_2 - \frac 32 \nu R T_1 = \frac 32 P_2 V_2 - \frac 32 P_1 V_1 = \frac 32 \cdot 2 P_0 \cdot 2 V_0 -  \frac 32 \cdot 2 P_0 \cdot V_0 = \frac 32 \cdot 2 \cdot P_0V_0, \\
    \Delta U_{41} &= \frac 32 \nu R T_1 - \frac 32 \nu R T_4 = \frac 32 P_1 V_1 - \frac 32 P_4 V_4 = \frac 32 \cdot 2 P_0 V_0 - \frac 32 P_0 V_0 = \frac 32 \cdot 1 \cdot P_0V_0.
    \\
    \eta &= \frac{A_\text{цикл}}{Q_+} = \frac{A_\text{цикл}}{Q_{12} + Q_{41}}  = \frac{A_\text{цикл}}{A_{12} + \Delta U_{12} + A_{41} + \Delta U_{41}} =  \\
     &= \frac{1P_0V_0}{2P_0V_0 + \frac 32 \cdot 2 \cdot P_0V_0 + 0 + \frac 32 \cdot 1 \cdot P_0V_0} = \frac{1}{2 + \frac 32 \cdot 2 + \frac 32 \cdot 1} = \frac2{13} \approx 0{,}154.
     \\
    \eta_\text{Карно} &= 1 - \frac{T_\text{х}}{T_\text{н}} = 1 - \frac{T_\text{4}}{T_\text{2}} = 1 - \frac{\frac{P_4V_4}{\nu R}}{\frac{P_2V_2}{\nu R}} = 1 - \frac{P_4V_4}{P_2V_2} = 1 - \frac{P_0V_0}{2P_0 \cdot 2V_0} = 1 - \frac 1{2 \cdot 2}  = \frac34 \approx 0{,}750.
    \end{align*}
}
\solutionspace{360pt}

\tasknumber{2}%
\task{%
    Порция идеального одноатомного газа перешла из состояния 1 в состояние 2: $P_1 = 2\,\text{МПа}$, $V_1 = 3\,\text{л}$, $P_2 = 4{,}5\,\text{МПа}$, $V_2 = 8\,\text{л}$.
    Известно, что в $PV$-координатах график процесса 12 представляет собой отрезок прямой.
    Определите,
    \begin{itemize}
        \item какую работу при этом совершил газ,
        \item чему равно изменение внутренней энергии газа,
        \item сколько теплоты подвели к нему в этом процессе?
    \end{itemize}
    При решении обратите внимание на знаки искомых величин.
}
\answer{%
    \begin{align*}
    P_1V_1 &= \nu R T_1, P_2V_2 = \nu R T_2, \\
    \Delta U &= U_2-U_1 = \frac 32 \nu R T_2- \frac 32 \nu R T_1 = \frac 32 P_2 V_2 - \frac 32 P_1 V_1= \frac 32 \cdot \cbr{4{,}5\,\text{МПа} \cdot 8\,\text{л} - 2\,\text{МПа} \cdot 3\,\text{л}} = 45{,}00\,\text{кДж}.
    \\
    A_\text{газа} &= \frac{P_2 + P_1} 2 \cdot (V_2 - V_1) = \frac{4{,}5\,\text{МПа} + 2\,\text{МПа}} 2 \cdot (8\,\text{л} - 3\,\text{л}) = 16{,}25\,\text{кДж}, \\
    Q &= A_\text{газа} + \Delta U = \frac 32 (P_2 V_2 - P_1 V_1) + \frac{P_2 + P_1} 2 \cdot (V_2 - V_1) = 45{,}00\,\text{кДж} + 16{,}25\,\text{кДж} = 61{,}25\,\text{кДж}.
    \end{align*}
}
\solutionspace{150pt}

\tasknumber{3}%
\task{%
    Запишите формулы и рядом с каждой физичической величиной укажите её название и единицы измерения в СИ:
    \begin{enumerate}
        \item первое начало термодинамики,
        \item внутренняя энергия идеального одноатомного газа.
    \end{enumerate}
}

\variantsplitter

\addpersonalvariant{Варвара Минаева}

\tasknumber{1}%
\task{%
    Определите КПД цикла 12341, рабочим телом которого является идеальный одноатомный газ, если
    12 — изобарическое расширение газа в три раза,
    23 — изохорическое охлаждение газа, при котором температура уменьшается в пять раз,
    34 — изобара, 41 — изохора.
    % Для этого:
    % \begin{enumerate}
    %     \item сделайте рисунок в $PV$-координатах,
    %     \item выберите удобные обозначения, чтобы не запутаться в множестве температур, давлений и объёмов,
    %     \item вычислите необходимые соотнощения между температурами, давлениями и объёмами
    %     (некоторые сразу видны по рисунку, некоторые — надо считать),
    %     \item определите для каждого участка поглощается или отдаётся тепло (и сколько именно:
    %     потребуется первое начало термодинамики, отдельный расчёт работ на участках через площади фигур и изменений внутренней энергии),
    %     \item вычислите полную работу газа в цикле,
    %     \item подставьте всё в формулу для КПД, упростите и доведите до ответа.
    % \end{enumerate}
    Определите КПД цикла Карно, температура нагревателя которого равна максимальной температуре в цикле 12341, а холодильника — минимальной.
    Ответы в обоих случаях оставьте точными в виде нескоратимой дроби, никаких округлений.
}
\answer{%
    \begin{align*}
    A_{12} &> 0, \Delta U_{12} > 0, \implies Q_{12} = A_{12} + \Delta U_{12} > 0, \\
    A_{23} &= 0, \Delta U_{23} < 0, \implies Q_{23} = A_{23} + \Delta U_{23} < 0, \\
    A_{34} &< 0, \Delta U_{34} < 0, \implies Q_{34} = A_{34} + \Delta U_{34} < 0, \\
    A_{41} &= 0, \Delta U_{41} > 0, \implies Q_{41} = A_{41} + \Delta U_{41} > 0.
    \\
    P_1V_1 &= \nu R T_1, P_2V_2 = \nu R T_2, P_3V_3 = \nu R T_3, P_4V_4 = \nu R T_4 \text{ — уравнения состояния идеального газа}, \\
    &\text{Пусть $P_0$, $V_0$, $T_0$ — давление, объём и температура в точке 4 (минимальные во всём цикле):} \\
    P_1 &= P_2, P_3 = P_4 = P_0, V_1 = V_4 = V_0, V_2 = V_3 = 3 V_1 = 3 V_0,, \text{остальные соотношения между объёмами и давлениями не даны, нужно считать} \\
    T_3 &= \frac{T_2}5 \text{(по условию)} \implies \frac{P_2}{P_3} = \frac{P_2 V_2}{P_3 V_3}= \frac{\nu R T_2}{\nu R T_3} = \frac{T_2}{T_3} = 5 \implies P_1 = P_2 = 5 P_0 \\
    A_\text{цикл} &= (5P_0 - P_0)(3V_0 - V_0) = 8P_0V_0, \\
    A_{12} &= 5P_0 \cdot (3V_0 - V_0) = 10P_0V_0, \\
    \Delta U_{12} &= \frac 32 \nu R T_2 - \frac 32 \nu R T_1 = \frac 32 P_2 V_2 - \frac 32 P_1 V_1 = \frac 32 \cdot 5 P_0 \cdot 3 V_0 -  \frac 32 \cdot 5 P_0 \cdot V_0 = \frac 32 \cdot 10 \cdot P_0V_0, \\
    \Delta U_{41} &= \frac 32 \nu R T_1 - \frac 32 \nu R T_4 = \frac 32 P_1 V_1 - \frac 32 P_4 V_4 = \frac 32 \cdot 5 P_0 V_0 - \frac 32 P_0 V_0 = \frac 32 \cdot 4 \cdot P_0V_0.
    \\
    \eta &= \frac{A_\text{цикл}}{Q_+} = \frac{A_\text{цикл}}{Q_{12} + Q_{41}}  = \frac{A_\text{цикл}}{A_{12} + \Delta U_{12} + A_{41} + \Delta U_{41}} =  \\
     &= \frac{8P_0V_0}{10P_0V_0 + \frac 32 \cdot 10 \cdot P_0V_0 + 0 + \frac 32 \cdot 4 \cdot P_0V_0} = \frac{8}{10 + \frac 32 \cdot 10 + \frac 32 \cdot 4} = \frac8{31} \approx 0{,}258.
     \\
    \eta_\text{Карно} &= 1 - \frac{T_\text{х}}{T_\text{н}} = 1 - \frac{T_\text{4}}{T_\text{2}} = 1 - \frac{\frac{P_4V_4}{\nu R}}{\frac{P_2V_2}{\nu R}} = 1 - \frac{P_4V_4}{P_2V_2} = 1 - \frac{P_0V_0}{5P_0 \cdot 3V_0} = 1 - \frac 1{5 \cdot 3}  = \frac{14}{15} \approx 0{,}933.
    \end{align*}
}
\solutionspace{360pt}

\tasknumber{2}%
\task{%
    Порция идеального одноатомного газа перешла из состояния 1 в состояние 2: $P_1 = 3\,\text{МПа}$, $V_1 = 7\,\text{л}$, $P_2 = 2{,}5\,\text{МПа}$, $V_2 = 8\,\text{л}$.
    Известно, что в $PV$-координатах график процесса 12 представляет собой отрезок прямой.
    Определите,
    \begin{itemize}
        \item какую работу при этом совершил газ,
        \item чему равно изменение внутренней энергии газа,
        \item сколько теплоты подвели к нему в этом процессе?
    \end{itemize}
    При решении обратите внимание на знаки искомых величин.
}
\answer{%
    \begin{align*}
    P_1V_1 &= \nu R T_1, P_2V_2 = \nu R T_2, \\
    \Delta U &= U_2-U_1 = \frac 32 \nu R T_2- \frac 32 \nu R T_1 = \frac 32 P_2 V_2 - \frac 32 P_1 V_1= \frac 32 \cdot \cbr{2{,}5\,\text{МПа} \cdot 8\,\text{л} - 3\,\text{МПа} \cdot 7\,\text{л}} = -1{,}5000\,\text{кДж}.
    \\
    A_\text{газа} &= \frac{P_2 + P_1} 2 \cdot (V_2 - V_1) = \frac{2{,}5\,\text{МПа} + 3\,\text{МПа}} 2 \cdot (8\,\text{л} - 7\,\text{л}) = 2{,}75\,\text{кДж}, \\
    Q &= A_\text{газа} + \Delta U = \frac 32 (P_2 V_2 - P_1 V_1) + \frac{P_2 + P_1} 2 \cdot (V_2 - V_1) = -1{,}5000\,\text{кДж} + 2{,}75\,\text{кДж} = 1{,}25\,\text{кДж}.
    \end{align*}
}
\solutionspace{150pt}

\tasknumber{3}%
\task{%
    Запишите формулы и рядом с каждой физичической величиной укажите её название и единицы измерения в СИ:
    \begin{enumerate}
        \item первое начало термодинамики,
        \item внутренняя энергия идеального одноатомного газа.
    \end{enumerate}
}

\variantsplitter

\addpersonalvariant{Леонид Никитин}

\tasknumber{1}%
\task{%
    Определите КПД цикла 12341, рабочим телом которого является идеальный одноатомный газ, если
    12 — изобарическое расширение газа в пять раз,
    23 — изохорическое охлаждение газа, при котором температура уменьшается в шесть раз,
    34 — изобара, 41 — изохора.
    % Для этого:
    % \begin{enumerate}
    %     \item сделайте рисунок в $PV$-координатах,
    %     \item выберите удобные обозначения, чтобы не запутаться в множестве температур, давлений и объёмов,
    %     \item вычислите необходимые соотнощения между температурами, давлениями и объёмами
    %     (некоторые сразу видны по рисунку, некоторые — надо считать),
    %     \item определите для каждого участка поглощается или отдаётся тепло (и сколько именно:
    %     потребуется первое начало термодинамики, отдельный расчёт работ на участках через площади фигур и изменений внутренней энергии),
    %     \item вычислите полную работу газа в цикле,
    %     \item подставьте всё в формулу для КПД, упростите и доведите до ответа.
    % \end{enumerate}
    Определите КПД цикла Карно, температура нагревателя которого равна максимальной температуре в цикле 12341, а холодильника — минимальной.
    Ответы в обоих случаях оставьте точными в виде нескоратимой дроби, никаких округлений.
}
\answer{%
    \begin{align*}
    A_{12} &> 0, \Delta U_{12} > 0, \implies Q_{12} = A_{12} + \Delta U_{12} > 0, \\
    A_{23} &= 0, \Delta U_{23} < 0, \implies Q_{23} = A_{23} + \Delta U_{23} < 0, \\
    A_{34} &< 0, \Delta U_{34} < 0, \implies Q_{34} = A_{34} + \Delta U_{34} < 0, \\
    A_{41} &= 0, \Delta U_{41} > 0, \implies Q_{41} = A_{41} + \Delta U_{41} > 0.
    \\
    P_1V_1 &= \nu R T_1, P_2V_2 = \nu R T_2, P_3V_3 = \nu R T_3, P_4V_4 = \nu R T_4 \text{ — уравнения состояния идеального газа}, \\
    &\text{Пусть $P_0$, $V_0$, $T_0$ — давление, объём и температура в точке 4 (минимальные во всём цикле):} \\
    P_1 &= P_2, P_3 = P_4 = P_0, V_1 = V_4 = V_0, V_2 = V_3 = 5 V_1 = 5 V_0,, \text{остальные соотношения между объёмами и давлениями не даны, нужно считать} \\
    T_3 &= \frac{T_2}6 \text{(по условию)} \implies \frac{P_2}{P_3} = \frac{P_2 V_2}{P_3 V_3}= \frac{\nu R T_2}{\nu R T_3} = \frac{T_2}{T_3} = 6 \implies P_1 = P_2 = 6 P_0 \\
    A_\text{цикл} &= (6P_0 - P_0)(5V_0 - V_0) = 20P_0V_0, \\
    A_{12} &= 6P_0 \cdot (5V_0 - V_0) = 24P_0V_0, \\
    \Delta U_{12} &= \frac 32 \nu R T_2 - \frac 32 \nu R T_1 = \frac 32 P_2 V_2 - \frac 32 P_1 V_1 = \frac 32 \cdot 6 P_0 \cdot 5 V_0 -  \frac 32 \cdot 6 P_0 \cdot V_0 = \frac 32 \cdot 24 \cdot P_0V_0, \\
    \Delta U_{41} &= \frac 32 \nu R T_1 - \frac 32 \nu R T_4 = \frac 32 P_1 V_1 - \frac 32 P_4 V_4 = \frac 32 \cdot 6 P_0 V_0 - \frac 32 P_0 V_0 = \frac 32 \cdot 5 \cdot P_0V_0.
    \\
    \eta &= \frac{A_\text{цикл}}{Q_+} = \frac{A_\text{цикл}}{Q_{12} + Q_{41}}  = \frac{A_\text{цикл}}{A_{12} + \Delta U_{12} + A_{41} + \Delta U_{41}} =  \\
     &= \frac{20P_0V_0}{24P_0V_0 + \frac 32 \cdot 24 \cdot P_0V_0 + 0 + \frac 32 \cdot 5 \cdot P_0V_0} = \frac{20}{24 + \frac 32 \cdot 24 + \frac 32 \cdot 5} = \frac8{27} \approx 0{,}296.
     \\
    \eta_\text{Карно} &= 1 - \frac{T_\text{х}}{T_\text{н}} = 1 - \frac{T_\text{4}}{T_\text{2}} = 1 - \frac{\frac{P_4V_4}{\nu R}}{\frac{P_2V_2}{\nu R}} = 1 - \frac{P_4V_4}{P_2V_2} = 1 - \frac{P_0V_0}{6P_0 \cdot 5V_0} = 1 - \frac 1{6 \cdot 5}  = \frac{29}{30} \approx 0{,}967.
    \end{align*}
}
\solutionspace{360pt}

\tasknumber{2}%
\task{%
    Порция идеального одноатомного газа перешла из состояния 1 в состояние 2: $P_1 = 3\,\text{МПа}$, $V_1 = 5\,\text{л}$, $P_2 = 2{,}5\,\text{МПа}$, $V_2 = 6\,\text{л}$.
    Известно, что в $PV$-координатах график процесса 12 представляет собой отрезок прямой.
    Определите,
    \begin{itemize}
        \item какую работу при этом совершил газ,
        \item чему равно изменение внутренней энергии газа,
        \item сколько теплоты подвели к нему в этом процессе?
    \end{itemize}
    При решении обратите внимание на знаки искомых величин.
}
\answer{%
    \begin{align*}
    P_1V_1 &= \nu R T_1, P_2V_2 = \nu R T_2, \\
    \Delta U &= U_2-U_1 = \frac 32 \nu R T_2- \frac 32 \nu R T_1 = \frac 32 P_2 V_2 - \frac 32 P_1 V_1= \frac 32 \cdot \cbr{2{,}5\,\text{МПа} \cdot 6\,\text{л} - 3\,\text{МПа} \cdot 5\,\text{л}} = 0\,\text{кДж}.
    \\
    A_\text{газа} &= \frac{P_2 + P_1} 2 \cdot (V_2 - V_1) = \frac{2{,}5\,\text{МПа} + 3\,\text{МПа}} 2 \cdot (6\,\text{л} - 5\,\text{л}) = 2{,}75\,\text{кДж}, \\
    Q &= A_\text{газа} + \Delta U = \frac 32 (P_2 V_2 - P_1 V_1) + \frac{P_2 + P_1} 2 \cdot (V_2 - V_1) = 0\,\text{кДж} + 2{,}75\,\text{кДж} = 2{,}75\,\text{кДж}.
    \end{align*}
}
\solutionspace{150pt}

\tasknumber{3}%
\task{%
    Запишите формулы и рядом с каждой физичической величиной укажите её название и единицы измерения в СИ:
    \begin{enumerate}
        \item первое начало термодинамики,
        \item внутренняя энергия идеального одноатомного газа.
    \end{enumerate}
}

\variantsplitter

\addpersonalvariant{Тимофей Полетаев}

\tasknumber{1}%
\task{%
    Определите КПД цикла 12341, рабочим телом которого является идеальный одноатомный газ, если
    12 — изобарическое расширение газа в пять раз,
    23 — изохорическое охлаждение газа, при котором температура уменьшается в четыре раза,
    34 — изобара, 41 — изохора.
    % Для этого:
    % \begin{enumerate}
    %     \item сделайте рисунок в $PV$-координатах,
    %     \item выберите удобные обозначения, чтобы не запутаться в множестве температур, давлений и объёмов,
    %     \item вычислите необходимые соотнощения между температурами, давлениями и объёмами
    %     (некоторые сразу видны по рисунку, некоторые — надо считать),
    %     \item определите для каждого участка поглощается или отдаётся тепло (и сколько именно:
    %     потребуется первое начало термодинамики, отдельный расчёт работ на участках через площади фигур и изменений внутренней энергии),
    %     \item вычислите полную работу газа в цикле,
    %     \item подставьте всё в формулу для КПД, упростите и доведите до ответа.
    % \end{enumerate}
    Определите КПД цикла Карно, температура нагревателя которого равна максимальной температуре в цикле 12341, а холодильника — минимальной.
    Ответы в обоих случаях оставьте точными в виде нескоратимой дроби, никаких округлений.
}
\answer{%
    \begin{align*}
    A_{12} &> 0, \Delta U_{12} > 0, \implies Q_{12} = A_{12} + \Delta U_{12} > 0, \\
    A_{23} &= 0, \Delta U_{23} < 0, \implies Q_{23} = A_{23} + \Delta U_{23} < 0, \\
    A_{34} &< 0, \Delta U_{34} < 0, \implies Q_{34} = A_{34} + \Delta U_{34} < 0, \\
    A_{41} &= 0, \Delta U_{41} > 0, \implies Q_{41} = A_{41} + \Delta U_{41} > 0.
    \\
    P_1V_1 &= \nu R T_1, P_2V_2 = \nu R T_2, P_3V_3 = \nu R T_3, P_4V_4 = \nu R T_4 \text{ — уравнения состояния идеального газа}, \\
    &\text{Пусть $P_0$, $V_0$, $T_0$ — давление, объём и температура в точке 4 (минимальные во всём цикле):} \\
    P_1 &= P_2, P_3 = P_4 = P_0, V_1 = V_4 = V_0, V_2 = V_3 = 5 V_1 = 5 V_0,, \text{остальные соотношения между объёмами и давлениями не даны, нужно считать} \\
    T_3 &= \frac{T_2}4 \text{(по условию)} \implies \frac{P_2}{P_3} = \frac{P_2 V_2}{P_3 V_3}= \frac{\nu R T_2}{\nu R T_3} = \frac{T_2}{T_3} = 4 \implies P_1 = P_2 = 4 P_0 \\
    A_\text{цикл} &= (4P_0 - P_0)(5V_0 - V_0) = 12P_0V_0, \\
    A_{12} &= 4P_0 \cdot (5V_0 - V_0) = 16P_0V_0, \\
    \Delta U_{12} &= \frac 32 \nu R T_2 - \frac 32 \nu R T_1 = \frac 32 P_2 V_2 - \frac 32 P_1 V_1 = \frac 32 \cdot 4 P_0 \cdot 5 V_0 -  \frac 32 \cdot 4 P_0 \cdot V_0 = \frac 32 \cdot 16 \cdot P_0V_0, \\
    \Delta U_{41} &= \frac 32 \nu R T_1 - \frac 32 \nu R T_4 = \frac 32 P_1 V_1 - \frac 32 P_4 V_4 = \frac 32 \cdot 4 P_0 V_0 - \frac 32 P_0 V_0 = \frac 32 \cdot 3 \cdot P_0V_0.
    \\
    \eta &= \frac{A_\text{цикл}}{Q_+} = \frac{A_\text{цикл}}{Q_{12} + Q_{41}}  = \frac{A_\text{цикл}}{A_{12} + \Delta U_{12} + A_{41} + \Delta U_{41}} =  \\
     &= \frac{12P_0V_0}{16P_0V_0 + \frac 32 \cdot 16 \cdot P_0V_0 + 0 + \frac 32 \cdot 3 \cdot P_0V_0} = \frac{12}{16 + \frac 32 \cdot 16 + \frac 32 \cdot 3} = \frac{24}{89} \approx 0{,}270.
     \\
    \eta_\text{Карно} &= 1 - \frac{T_\text{х}}{T_\text{н}} = 1 - \frac{T_\text{4}}{T_\text{2}} = 1 - \frac{\frac{P_4V_4}{\nu R}}{\frac{P_2V_2}{\nu R}} = 1 - \frac{P_4V_4}{P_2V_2} = 1 - \frac{P_0V_0}{4P_0 \cdot 5V_0} = 1 - \frac 1{4 \cdot 5}  = \frac{19}{20} \approx 0{,}950.
    \end{align*}
}
\solutionspace{360pt}

\tasknumber{2}%
\task{%
    Порция идеального одноатомного газа перешла из состояния 1 в состояние 2: $P_1 = 2\,\text{МПа}$, $V_1 = 5\,\text{л}$, $P_2 = 4{,}5\,\text{МПа}$, $V_2 = 8\,\text{л}$.
    Известно, что в $PV$-координатах график процесса 12 представляет собой отрезок прямой.
    Определите,
    \begin{itemize}
        \item какую работу при этом совершил газ,
        \item чему равно изменение внутренней энергии газа,
        \item сколько теплоты подвели к нему в этом процессе?
    \end{itemize}
    При решении обратите внимание на знаки искомых величин.
}
\answer{%
    \begin{align*}
    P_1V_1 &= \nu R T_1, P_2V_2 = \nu R T_2, \\
    \Delta U &= U_2-U_1 = \frac 32 \nu R T_2- \frac 32 \nu R T_1 = \frac 32 P_2 V_2 - \frac 32 P_1 V_1= \frac 32 \cdot \cbr{4{,}5\,\text{МПа} \cdot 8\,\text{л} - 2\,\text{МПа} \cdot 5\,\text{л}} = 39{,}00\,\text{кДж}.
    \\
    A_\text{газа} &= \frac{P_2 + P_1} 2 \cdot (V_2 - V_1) = \frac{4{,}5\,\text{МПа} + 2\,\text{МПа}} 2 \cdot (8\,\text{л} - 5\,\text{л}) = 9{,}75\,\text{кДж}, \\
    Q &= A_\text{газа} + \Delta U = \frac 32 (P_2 V_2 - P_1 V_1) + \frac{P_2 + P_1} 2 \cdot (V_2 - V_1) = 39{,}00\,\text{кДж} + 9{,}75\,\text{кДж} = 48{,}75\,\text{кДж}.
    \end{align*}
}
\solutionspace{150pt}

\tasknumber{3}%
\task{%
    Запишите формулы и рядом с каждой физичической величиной укажите её название и единицы измерения в СИ:
    \begin{enumerate}
        \item первое начало термодинамики,
        \item внутренняя энергия идеального одноатомного газа.
    \end{enumerate}
}

\variantsplitter

\addpersonalvariant{Андрей Рожков}

\tasknumber{1}%
\task{%
    Определите КПД цикла 12341, рабочим телом которого является идеальный одноатомный газ, если
    12 — изобарическое расширение газа в три раза,
    23 — изохорическое охлаждение газа, при котором температура уменьшается в пять раз,
    34 — изобара, 41 — изохора.
    % Для этого:
    % \begin{enumerate}
    %     \item сделайте рисунок в $PV$-координатах,
    %     \item выберите удобные обозначения, чтобы не запутаться в множестве температур, давлений и объёмов,
    %     \item вычислите необходимые соотнощения между температурами, давлениями и объёмами
    %     (некоторые сразу видны по рисунку, некоторые — надо считать),
    %     \item определите для каждого участка поглощается или отдаётся тепло (и сколько именно:
    %     потребуется первое начало термодинамики, отдельный расчёт работ на участках через площади фигур и изменений внутренней энергии),
    %     \item вычислите полную работу газа в цикле,
    %     \item подставьте всё в формулу для КПД, упростите и доведите до ответа.
    % \end{enumerate}
    Определите КПД цикла Карно, температура нагревателя которого равна максимальной температуре в цикле 12341, а холодильника — минимальной.
    Ответы в обоих случаях оставьте точными в виде нескоратимой дроби, никаких округлений.
}
\answer{%
    \begin{align*}
    A_{12} &> 0, \Delta U_{12} > 0, \implies Q_{12} = A_{12} + \Delta U_{12} > 0, \\
    A_{23} &= 0, \Delta U_{23} < 0, \implies Q_{23} = A_{23} + \Delta U_{23} < 0, \\
    A_{34} &< 0, \Delta U_{34} < 0, \implies Q_{34} = A_{34} + \Delta U_{34} < 0, \\
    A_{41} &= 0, \Delta U_{41} > 0, \implies Q_{41} = A_{41} + \Delta U_{41} > 0.
    \\
    P_1V_1 &= \nu R T_1, P_2V_2 = \nu R T_2, P_3V_3 = \nu R T_3, P_4V_4 = \nu R T_4 \text{ — уравнения состояния идеального газа}, \\
    &\text{Пусть $P_0$, $V_0$, $T_0$ — давление, объём и температура в точке 4 (минимальные во всём цикле):} \\
    P_1 &= P_2, P_3 = P_4 = P_0, V_1 = V_4 = V_0, V_2 = V_3 = 3 V_1 = 3 V_0,, \text{остальные соотношения между объёмами и давлениями не даны, нужно считать} \\
    T_3 &= \frac{T_2}5 \text{(по условию)} \implies \frac{P_2}{P_3} = \frac{P_2 V_2}{P_3 V_3}= \frac{\nu R T_2}{\nu R T_3} = \frac{T_2}{T_3} = 5 \implies P_1 = P_2 = 5 P_0 \\
    A_\text{цикл} &= (5P_0 - P_0)(3V_0 - V_0) = 8P_0V_0, \\
    A_{12} &= 5P_0 \cdot (3V_0 - V_0) = 10P_0V_0, \\
    \Delta U_{12} &= \frac 32 \nu R T_2 - \frac 32 \nu R T_1 = \frac 32 P_2 V_2 - \frac 32 P_1 V_1 = \frac 32 \cdot 5 P_0 \cdot 3 V_0 -  \frac 32 \cdot 5 P_0 \cdot V_0 = \frac 32 \cdot 10 \cdot P_0V_0, \\
    \Delta U_{41} &= \frac 32 \nu R T_1 - \frac 32 \nu R T_4 = \frac 32 P_1 V_1 - \frac 32 P_4 V_4 = \frac 32 \cdot 5 P_0 V_0 - \frac 32 P_0 V_0 = \frac 32 \cdot 4 \cdot P_0V_0.
    \\
    \eta &= \frac{A_\text{цикл}}{Q_+} = \frac{A_\text{цикл}}{Q_{12} + Q_{41}}  = \frac{A_\text{цикл}}{A_{12} + \Delta U_{12} + A_{41} + \Delta U_{41}} =  \\
     &= \frac{8P_0V_0}{10P_0V_0 + \frac 32 \cdot 10 \cdot P_0V_0 + 0 + \frac 32 \cdot 4 \cdot P_0V_0} = \frac{8}{10 + \frac 32 \cdot 10 + \frac 32 \cdot 4} = \frac8{31} \approx 0{,}258.
     \\
    \eta_\text{Карно} &= 1 - \frac{T_\text{х}}{T_\text{н}} = 1 - \frac{T_\text{4}}{T_\text{2}} = 1 - \frac{\frac{P_4V_4}{\nu R}}{\frac{P_2V_2}{\nu R}} = 1 - \frac{P_4V_4}{P_2V_2} = 1 - \frac{P_0V_0}{5P_0 \cdot 3V_0} = 1 - \frac 1{5 \cdot 3}  = \frac{14}{15} \approx 0{,}933.
    \end{align*}
}
\solutionspace{360pt}

\tasknumber{2}%
\task{%
    Порция идеального одноатомного газа перешла из состояния 1 в состояние 2: $P_1 = 3\,\text{МПа}$, $V_1 = 5\,\text{л}$, $P_2 = 4{,}5\,\text{МПа}$, $V_2 = 4\,\text{л}$.
    Известно, что в $PV$-координатах график процесса 12 представляет собой отрезок прямой.
    Определите,
    \begin{itemize}
        \item какую работу при этом совершил газ,
        \item чему равно изменение внутренней энергии газа,
        \item сколько теплоты подвели к нему в этом процессе?
    \end{itemize}
    При решении обратите внимание на знаки искомых величин.
}
\answer{%
    \begin{align*}
    P_1V_1 &= \nu R T_1, P_2V_2 = \nu R T_2, \\
    \Delta U &= U_2-U_1 = \frac 32 \nu R T_2- \frac 32 \nu R T_1 = \frac 32 P_2 V_2 - \frac 32 P_1 V_1= \frac 32 \cdot \cbr{4{,}5\,\text{МПа} \cdot 4\,\text{л} - 3\,\text{МПа} \cdot 5\,\text{л}} = 4{,}50\,\text{кДж}.
    \\
    A_\text{газа} &= \frac{P_2 + P_1} 2 \cdot (V_2 - V_1) = \frac{4{,}5\,\text{МПа} + 3\,\text{МПа}} 2 \cdot (4\,\text{л} - 5\,\text{л}) = -3{,}750\,\text{кДж}, \\
    Q &= A_\text{газа} + \Delta U = \frac 32 (P_2 V_2 - P_1 V_1) + \frac{P_2 + P_1} 2 \cdot (V_2 - V_1) = 4{,}50\,\text{кДж} -3{,}750\,\text{кДж} = 0{,}75\,\text{кДж}.
    \end{align*}
}
\solutionspace{150pt}

\tasknumber{3}%
\task{%
    Запишите формулы и рядом с каждой физичической величиной укажите её название и единицы измерения в СИ:
    \begin{enumerate}
        \item первое начало термодинамики,
        \item внутренняя энергия идеального одноатомного газа.
    \end{enumerate}
}

\variantsplitter

\addpersonalvariant{Рената Таржиманова}

\tasknumber{1}%
\task{%
    Определите КПД цикла 12341, рабочим телом которого является идеальный одноатомный газ, если
    12 — изобарическое расширение газа в два раза,
    23 — изохорическое охлаждение газа, при котором температура уменьшается в пять раз,
    34 — изобара, 41 — изохора.
    % Для этого:
    % \begin{enumerate}
    %     \item сделайте рисунок в $PV$-координатах,
    %     \item выберите удобные обозначения, чтобы не запутаться в множестве температур, давлений и объёмов,
    %     \item вычислите необходимые соотнощения между температурами, давлениями и объёмами
    %     (некоторые сразу видны по рисунку, некоторые — надо считать),
    %     \item определите для каждого участка поглощается или отдаётся тепло (и сколько именно:
    %     потребуется первое начало термодинамики, отдельный расчёт работ на участках через площади фигур и изменений внутренней энергии),
    %     \item вычислите полную работу газа в цикле,
    %     \item подставьте всё в формулу для КПД, упростите и доведите до ответа.
    % \end{enumerate}
    Определите КПД цикла Карно, температура нагревателя которого равна максимальной температуре в цикле 12341, а холодильника — минимальной.
    Ответы в обоих случаях оставьте точными в виде нескоратимой дроби, никаких округлений.
}
\answer{%
    \begin{align*}
    A_{12} &> 0, \Delta U_{12} > 0, \implies Q_{12} = A_{12} + \Delta U_{12} > 0, \\
    A_{23} &= 0, \Delta U_{23} < 0, \implies Q_{23} = A_{23} + \Delta U_{23} < 0, \\
    A_{34} &< 0, \Delta U_{34} < 0, \implies Q_{34} = A_{34} + \Delta U_{34} < 0, \\
    A_{41} &= 0, \Delta U_{41} > 0, \implies Q_{41} = A_{41} + \Delta U_{41} > 0.
    \\
    P_1V_1 &= \nu R T_1, P_2V_2 = \nu R T_2, P_3V_3 = \nu R T_3, P_4V_4 = \nu R T_4 \text{ — уравнения состояния идеального газа}, \\
    &\text{Пусть $P_0$, $V_0$, $T_0$ — давление, объём и температура в точке 4 (минимальные во всём цикле):} \\
    P_1 &= P_2, P_3 = P_4 = P_0, V_1 = V_4 = V_0, V_2 = V_3 = 2 V_1 = 2 V_0,, \text{остальные соотношения между объёмами и давлениями не даны, нужно считать} \\
    T_3 &= \frac{T_2}5 \text{(по условию)} \implies \frac{P_2}{P_3} = \frac{P_2 V_2}{P_3 V_3}= \frac{\nu R T_2}{\nu R T_3} = \frac{T_2}{T_3} = 5 \implies P_1 = P_2 = 5 P_0 \\
    A_\text{цикл} &= (5P_0 - P_0)(2V_0 - V_0) = 4P_0V_0, \\
    A_{12} &= 5P_0 \cdot (2V_0 - V_0) = 5P_0V_0, \\
    \Delta U_{12} &= \frac 32 \nu R T_2 - \frac 32 \nu R T_1 = \frac 32 P_2 V_2 - \frac 32 P_1 V_1 = \frac 32 \cdot 5 P_0 \cdot 2 V_0 -  \frac 32 \cdot 5 P_0 \cdot V_0 = \frac 32 \cdot 5 \cdot P_0V_0, \\
    \Delta U_{41} &= \frac 32 \nu R T_1 - \frac 32 \nu R T_4 = \frac 32 P_1 V_1 - \frac 32 P_4 V_4 = \frac 32 \cdot 5 P_0 V_0 - \frac 32 P_0 V_0 = \frac 32 \cdot 4 \cdot P_0V_0.
    \\
    \eta &= \frac{A_\text{цикл}}{Q_+} = \frac{A_\text{цикл}}{Q_{12} + Q_{41}}  = \frac{A_\text{цикл}}{A_{12} + \Delta U_{12} + A_{41} + \Delta U_{41}} =  \\
     &= \frac{4P_0V_0}{5P_0V_0 + \frac 32 \cdot 5 \cdot P_0V_0 + 0 + \frac 32 \cdot 4 \cdot P_0V_0} = \frac{4}{5 + \frac 32 \cdot 5 + \frac 32 \cdot 4} = \frac8{37} \approx 0{,}216.
     \\
    \eta_\text{Карно} &= 1 - \frac{T_\text{х}}{T_\text{н}} = 1 - \frac{T_\text{4}}{T_\text{2}} = 1 - \frac{\frac{P_4V_4}{\nu R}}{\frac{P_2V_2}{\nu R}} = 1 - \frac{P_4V_4}{P_2V_2} = 1 - \frac{P_0V_0}{5P_0 \cdot 2V_0} = 1 - \frac 1{5 \cdot 2}  = \frac9{10} \approx 0{,}900.
    \end{align*}
}
\solutionspace{360pt}

\tasknumber{2}%
\task{%
    Порция идеального одноатомного газа перешла из состояния 1 в состояние 2: $P_1 = 3\,\text{МПа}$, $V_1 = 3\,\text{л}$, $P_2 = 2{,}5\,\text{МПа}$, $V_2 = 4\,\text{л}$.
    Известно, что в $PV$-координатах график процесса 12 представляет собой отрезок прямой.
    Определите,
    \begin{itemize}
        \item какую работу при этом совершил газ,
        \item чему равно изменение внутренней энергии газа,
        \item сколько теплоты подвели к нему в этом процессе?
    \end{itemize}
    При решении обратите внимание на знаки искомых величин.
}
\answer{%
    \begin{align*}
    P_1V_1 &= \nu R T_1, P_2V_2 = \nu R T_2, \\
    \Delta U &= U_2-U_1 = \frac 32 \nu R T_2- \frac 32 \nu R T_1 = \frac 32 P_2 V_2 - \frac 32 P_1 V_1= \frac 32 \cdot \cbr{2{,}5\,\text{МПа} \cdot 4\,\text{л} - 3\,\text{МПа} \cdot 3\,\text{л}} = 1{,}50\,\text{кДж}.
    \\
    A_\text{газа} &= \frac{P_2 + P_1} 2 \cdot (V_2 - V_1) = \frac{2{,}5\,\text{МПа} + 3\,\text{МПа}} 2 \cdot (4\,\text{л} - 3\,\text{л}) = 2{,}75\,\text{кДж}, \\
    Q &= A_\text{газа} + \Delta U = \frac 32 (P_2 V_2 - P_1 V_1) + \frac{P_2 + P_1} 2 \cdot (V_2 - V_1) = 1{,}50\,\text{кДж} + 2{,}75\,\text{кДж} = 4{,}25\,\text{кДж}.
    \end{align*}
}
\solutionspace{150pt}

\tasknumber{3}%
\task{%
    Запишите формулы и рядом с каждой физичической величиной укажите её название и единицы измерения в СИ:
    \begin{enumerate}
        \item первое начало термодинамики,
        \item внутренняя энергия идеального одноатомного газа.
    \end{enumerate}
}

\variantsplitter

\addpersonalvariant{Андрей Щербаков}

\tasknumber{1}%
\task{%
    Определите КПД цикла 12341, рабочим телом которого является идеальный одноатомный газ, если
    12 — изобарическое расширение газа в шесть раз,
    23 — изохорическое охлаждение газа, при котором температура уменьшается в два раза,
    34 — изобара, 41 — изохора.
    % Для этого:
    % \begin{enumerate}
    %     \item сделайте рисунок в $PV$-координатах,
    %     \item выберите удобные обозначения, чтобы не запутаться в множестве температур, давлений и объёмов,
    %     \item вычислите необходимые соотнощения между температурами, давлениями и объёмами
    %     (некоторые сразу видны по рисунку, некоторые — надо считать),
    %     \item определите для каждого участка поглощается или отдаётся тепло (и сколько именно:
    %     потребуется первое начало термодинамики, отдельный расчёт работ на участках через площади фигур и изменений внутренней энергии),
    %     \item вычислите полную работу газа в цикле,
    %     \item подставьте всё в формулу для КПД, упростите и доведите до ответа.
    % \end{enumerate}
    Определите КПД цикла Карно, температура нагревателя которого равна максимальной температуре в цикле 12341, а холодильника — минимальной.
    Ответы в обоих случаях оставьте точными в виде нескоратимой дроби, никаких округлений.
}
\answer{%
    \begin{align*}
    A_{12} &> 0, \Delta U_{12} > 0, \implies Q_{12} = A_{12} + \Delta U_{12} > 0, \\
    A_{23} &= 0, \Delta U_{23} < 0, \implies Q_{23} = A_{23} + \Delta U_{23} < 0, \\
    A_{34} &< 0, \Delta U_{34} < 0, \implies Q_{34} = A_{34} + \Delta U_{34} < 0, \\
    A_{41} &= 0, \Delta U_{41} > 0, \implies Q_{41} = A_{41} + \Delta U_{41} > 0.
    \\
    P_1V_1 &= \nu R T_1, P_2V_2 = \nu R T_2, P_3V_3 = \nu R T_3, P_4V_4 = \nu R T_4 \text{ — уравнения состояния идеального газа}, \\
    &\text{Пусть $P_0$, $V_0$, $T_0$ — давление, объём и температура в точке 4 (минимальные во всём цикле):} \\
    P_1 &= P_2, P_3 = P_4 = P_0, V_1 = V_4 = V_0, V_2 = V_3 = 6 V_1 = 6 V_0,, \text{остальные соотношения между объёмами и давлениями не даны, нужно считать} \\
    T_3 &= \frac{T_2}2 \text{(по условию)} \implies \frac{P_2}{P_3} = \frac{P_2 V_2}{P_3 V_3}= \frac{\nu R T_2}{\nu R T_3} = \frac{T_2}{T_3} = 2 \implies P_1 = P_2 = 2 P_0 \\
    A_\text{цикл} &= (2P_0 - P_0)(6V_0 - V_0) = 5P_0V_0, \\
    A_{12} &= 2P_0 \cdot (6V_0 - V_0) = 10P_0V_0, \\
    \Delta U_{12} &= \frac 32 \nu R T_2 - \frac 32 \nu R T_1 = \frac 32 P_2 V_2 - \frac 32 P_1 V_1 = \frac 32 \cdot 2 P_0 \cdot 6 V_0 -  \frac 32 \cdot 2 P_0 \cdot V_0 = \frac 32 \cdot 10 \cdot P_0V_0, \\
    \Delta U_{41} &= \frac 32 \nu R T_1 - \frac 32 \nu R T_4 = \frac 32 P_1 V_1 - \frac 32 P_4 V_4 = \frac 32 \cdot 2 P_0 V_0 - \frac 32 P_0 V_0 = \frac 32 \cdot 1 \cdot P_0V_0.
    \\
    \eta &= \frac{A_\text{цикл}}{Q_+} = \frac{A_\text{цикл}}{Q_{12} + Q_{41}}  = \frac{A_\text{цикл}}{A_{12} + \Delta U_{12} + A_{41} + \Delta U_{41}} =  \\
     &= \frac{5P_0V_0}{10P_0V_0 + \frac 32 \cdot 10 \cdot P_0V_0 + 0 + \frac 32 \cdot 1 \cdot P_0V_0} = \frac{5}{10 + \frac 32 \cdot 10 + \frac 32 \cdot 1} = \frac{10}{53} \approx 0{,}189.
     \\
    \eta_\text{Карно} &= 1 - \frac{T_\text{х}}{T_\text{н}} = 1 - \frac{T_\text{4}}{T_\text{2}} = 1 - \frac{\frac{P_4V_4}{\nu R}}{\frac{P_2V_2}{\nu R}} = 1 - \frac{P_4V_4}{P_2V_2} = 1 - \frac{P_0V_0}{2P_0 \cdot 6V_0} = 1 - \frac 1{2 \cdot 6}  = \frac{11}{12} \approx 0{,}917.
    \end{align*}
}
\solutionspace{360pt}

\tasknumber{2}%
\task{%
    Порция идеального одноатомного газа перешла из состояния 1 в состояние 2: $P_1 = 4\,\text{МПа}$, $V_1 = 7\,\text{л}$, $P_2 = 4{,}5\,\text{МПа}$, $V_2 = 6\,\text{л}$.
    Известно, что в $PV$-координатах график процесса 12 представляет собой отрезок прямой.
    Определите,
    \begin{itemize}
        \item какую работу при этом совершил газ,
        \item чему равно изменение внутренней энергии газа,
        \item сколько теплоты подвели к нему в этом процессе?
    \end{itemize}
    При решении обратите внимание на знаки искомых величин.
}
\answer{%
    \begin{align*}
    P_1V_1 &= \nu R T_1, P_2V_2 = \nu R T_2, \\
    \Delta U &= U_2-U_1 = \frac 32 \nu R T_2- \frac 32 \nu R T_1 = \frac 32 P_2 V_2 - \frac 32 P_1 V_1= \frac 32 \cdot \cbr{4{,}5\,\text{МПа} \cdot 6\,\text{л} - 4\,\text{МПа} \cdot 7\,\text{л}} = -1{,}5000\,\text{кДж}.
    \\
    A_\text{газа} &= \frac{P_2 + P_1} 2 \cdot (V_2 - V_1) = \frac{4{,}5\,\text{МПа} + 4\,\text{МПа}} 2 \cdot (6\,\text{л} - 7\,\text{л}) = -4{,}250\,\text{кДж}, \\
    Q &= A_\text{газа} + \Delta U = \frac 32 (P_2 V_2 - P_1 V_1) + \frac{P_2 + P_1} 2 \cdot (V_2 - V_1) = -1{,}5000\,\text{кДж} -4{,}250\,\text{кДж} = -5{,}750\,\text{кДж}.
    \end{align*}
}
\solutionspace{150pt}

\tasknumber{3}%
\task{%
    Запишите формулы и рядом с каждой физичической величиной укажите её название и единицы измерения в СИ:
    \begin{enumerate}
        \item первое начало термодинамики,
        \item внутренняя энергия идеального одноатомного газа.
    \end{enumerate}
}

\variantsplitter

\addpersonalvariant{Михаил Ярошевский}

\tasknumber{1}%
\task{%
    Определите КПД цикла 12341, рабочим телом которого является идеальный одноатомный газ, если
    12 — изобарическое расширение газа в шесть раз,
    23 — изохорическое охлаждение газа, при котором температура уменьшается в три раза,
    34 — изобара, 41 — изохора.
    % Для этого:
    % \begin{enumerate}
    %     \item сделайте рисунок в $PV$-координатах,
    %     \item выберите удобные обозначения, чтобы не запутаться в множестве температур, давлений и объёмов,
    %     \item вычислите необходимые соотнощения между температурами, давлениями и объёмами
    %     (некоторые сразу видны по рисунку, некоторые — надо считать),
    %     \item определите для каждого участка поглощается или отдаётся тепло (и сколько именно:
    %     потребуется первое начало термодинамики, отдельный расчёт работ на участках через площади фигур и изменений внутренней энергии),
    %     \item вычислите полную работу газа в цикле,
    %     \item подставьте всё в формулу для КПД, упростите и доведите до ответа.
    % \end{enumerate}
    Определите КПД цикла Карно, температура нагревателя которого равна максимальной температуре в цикле 12341, а холодильника — минимальной.
    Ответы в обоих случаях оставьте точными в виде нескоратимой дроби, никаких округлений.
}
\answer{%
    \begin{align*}
    A_{12} &> 0, \Delta U_{12} > 0, \implies Q_{12} = A_{12} + \Delta U_{12} > 0, \\
    A_{23} &= 0, \Delta U_{23} < 0, \implies Q_{23} = A_{23} + \Delta U_{23} < 0, \\
    A_{34} &< 0, \Delta U_{34} < 0, \implies Q_{34} = A_{34} + \Delta U_{34} < 0, \\
    A_{41} &= 0, \Delta U_{41} > 0, \implies Q_{41} = A_{41} + \Delta U_{41} > 0.
    \\
    P_1V_1 &= \nu R T_1, P_2V_2 = \nu R T_2, P_3V_3 = \nu R T_3, P_4V_4 = \nu R T_4 \text{ — уравнения состояния идеального газа}, \\
    &\text{Пусть $P_0$, $V_0$, $T_0$ — давление, объём и температура в точке 4 (минимальные во всём цикле):} \\
    P_1 &= P_2, P_3 = P_4 = P_0, V_1 = V_4 = V_0, V_2 = V_3 = 6 V_1 = 6 V_0,, \text{остальные соотношения между объёмами и давлениями не даны, нужно считать} \\
    T_3 &= \frac{T_2}3 \text{(по условию)} \implies \frac{P_2}{P_3} = \frac{P_2 V_2}{P_3 V_3}= \frac{\nu R T_2}{\nu R T_3} = \frac{T_2}{T_3} = 3 \implies P_1 = P_2 = 3 P_0 \\
    A_\text{цикл} &= (3P_0 - P_0)(6V_0 - V_0) = 10P_0V_0, \\
    A_{12} &= 3P_0 \cdot (6V_0 - V_0) = 15P_0V_0, \\
    \Delta U_{12} &= \frac 32 \nu R T_2 - \frac 32 \nu R T_1 = \frac 32 P_2 V_2 - \frac 32 P_1 V_1 = \frac 32 \cdot 3 P_0 \cdot 6 V_0 -  \frac 32 \cdot 3 P_0 \cdot V_0 = \frac 32 \cdot 15 \cdot P_0V_0, \\
    \Delta U_{41} &= \frac 32 \nu R T_1 - \frac 32 \nu R T_4 = \frac 32 P_1 V_1 - \frac 32 P_4 V_4 = \frac 32 \cdot 3 P_0 V_0 - \frac 32 P_0 V_0 = \frac 32 \cdot 2 \cdot P_0V_0.
    \\
    \eta &= \frac{A_\text{цикл}}{Q_+} = \frac{A_\text{цикл}}{Q_{12} + Q_{41}}  = \frac{A_\text{цикл}}{A_{12} + \Delta U_{12} + A_{41} + \Delta U_{41}} =  \\
     &= \frac{10P_0V_0}{15P_0V_0 + \frac 32 \cdot 15 \cdot P_0V_0 + 0 + \frac 32 \cdot 2 \cdot P_0V_0} = \frac{10}{15 + \frac 32 \cdot 15 + \frac 32 \cdot 2} = \frac{20}{81} \approx 0{,}247.
     \\
    \eta_\text{Карно} &= 1 - \frac{T_\text{х}}{T_\text{н}} = 1 - \frac{T_\text{4}}{T_\text{2}} = 1 - \frac{\frac{P_4V_4}{\nu R}}{\frac{P_2V_2}{\nu R}} = 1 - \frac{P_4V_4}{P_2V_2} = 1 - \frac{P_0V_0}{3P_0 \cdot 6V_0} = 1 - \frac 1{3 \cdot 6}  = \frac{17}{18} \approx 0{,}944.
    \end{align*}
}
\solutionspace{360pt}

\tasknumber{2}%
\task{%
    Порция идеального одноатомного газа перешла из состояния 1 в состояние 2: $P_1 = 2\,\text{МПа}$, $V_1 = 3\,\text{л}$, $P_2 = 3{,}5\,\text{МПа}$, $V_2 = 4\,\text{л}$.
    Известно, что в $PV$-координатах график процесса 12 представляет собой отрезок прямой.
    Определите,
    \begin{itemize}
        \item какую работу при этом совершил газ,
        \item чему равно изменение внутренней энергии газа,
        \item сколько теплоты подвели к нему в этом процессе?
    \end{itemize}
    При решении обратите внимание на знаки искомых величин.
}
\answer{%
    \begin{align*}
    P_1V_1 &= \nu R T_1, P_2V_2 = \nu R T_2, \\
    \Delta U &= U_2-U_1 = \frac 32 \nu R T_2- \frac 32 \nu R T_1 = \frac 32 P_2 V_2 - \frac 32 P_1 V_1= \frac 32 \cdot \cbr{3{,}5\,\text{МПа} \cdot 4\,\text{л} - 2\,\text{МПа} \cdot 3\,\text{л}} = 12{,}00\,\text{кДж}.
    \\
    A_\text{газа} &= \frac{P_2 + P_1} 2 \cdot (V_2 - V_1) = \frac{3{,}5\,\text{МПа} + 2\,\text{МПа}} 2 \cdot (4\,\text{л} - 3\,\text{л}) = 2{,}75\,\text{кДж}, \\
    Q &= A_\text{газа} + \Delta U = \frac 32 (P_2 V_2 - P_1 V_1) + \frac{P_2 + P_1} 2 \cdot (V_2 - V_1) = 12{,}00\,\text{кДж} + 2{,}75\,\text{кДж} = 14{,}75\,\text{кДж}.
    \end{align*}
}
\solutionspace{150pt}

\tasknumber{3}%
\task{%
    Запишите формулы и рядом с каждой физичической величиной укажите её название и единицы измерения в СИ:
    \begin{enumerate}
        \item первое начало термодинамики,
        \item внутренняя энергия идеального одноатомного газа.
    \end{enumerate}
}

\variantsplitter

\addpersonalvariant{Алексей Алимпиев}

\tasknumber{1}%
\task{%
    Определите КПД цикла 12341, рабочим телом которого является идеальный одноатомный газ, если
    12 — изобарическое расширение газа в три раза,
    23 — изохорическое охлаждение газа, при котором температура уменьшается в шесть раз,
    34 — изобара, 41 — изохора.
    % Для этого:
    % \begin{enumerate}
    %     \item сделайте рисунок в $PV$-координатах,
    %     \item выберите удобные обозначения, чтобы не запутаться в множестве температур, давлений и объёмов,
    %     \item вычислите необходимые соотнощения между температурами, давлениями и объёмами
    %     (некоторые сразу видны по рисунку, некоторые — надо считать),
    %     \item определите для каждого участка поглощается или отдаётся тепло (и сколько именно:
    %     потребуется первое начало термодинамики, отдельный расчёт работ на участках через площади фигур и изменений внутренней энергии),
    %     \item вычислите полную работу газа в цикле,
    %     \item подставьте всё в формулу для КПД, упростите и доведите до ответа.
    % \end{enumerate}
    Определите КПД цикла Карно, температура нагревателя которого равна максимальной температуре в цикле 12341, а холодильника — минимальной.
    Ответы в обоих случаях оставьте точными в виде нескоратимой дроби, никаких округлений.
}
\answer{%
    \begin{align*}
    A_{12} &> 0, \Delta U_{12} > 0, \implies Q_{12} = A_{12} + \Delta U_{12} > 0, \\
    A_{23} &= 0, \Delta U_{23} < 0, \implies Q_{23} = A_{23} + \Delta U_{23} < 0, \\
    A_{34} &< 0, \Delta U_{34} < 0, \implies Q_{34} = A_{34} + \Delta U_{34} < 0, \\
    A_{41} &= 0, \Delta U_{41} > 0, \implies Q_{41} = A_{41} + \Delta U_{41} > 0.
    \\
    P_1V_1 &= \nu R T_1, P_2V_2 = \nu R T_2, P_3V_3 = \nu R T_3, P_4V_4 = \nu R T_4 \text{ — уравнения состояния идеального газа}, \\
    &\text{Пусть $P_0$, $V_0$, $T_0$ — давление, объём и температура в точке 4 (минимальные во всём цикле):} \\
    P_1 &= P_2, P_3 = P_4 = P_0, V_1 = V_4 = V_0, V_2 = V_3 = 3 V_1 = 3 V_0,, \text{остальные соотношения между объёмами и давлениями не даны, нужно считать} \\
    T_3 &= \frac{T_2}6 \text{(по условию)} \implies \frac{P_2}{P_3} = \frac{P_2 V_2}{P_3 V_3}= \frac{\nu R T_2}{\nu R T_3} = \frac{T_2}{T_3} = 6 \implies P_1 = P_2 = 6 P_0 \\
    A_\text{цикл} &= (6P_0 - P_0)(3V_0 - V_0) = 10P_0V_0, \\
    A_{12} &= 6P_0 \cdot (3V_0 - V_0) = 12P_0V_0, \\
    \Delta U_{12} &= \frac 32 \nu R T_2 - \frac 32 \nu R T_1 = \frac 32 P_2 V_2 - \frac 32 P_1 V_1 = \frac 32 \cdot 6 P_0 \cdot 3 V_0 -  \frac 32 \cdot 6 P_0 \cdot V_0 = \frac 32 \cdot 12 \cdot P_0V_0, \\
    \Delta U_{41} &= \frac 32 \nu R T_1 - \frac 32 \nu R T_4 = \frac 32 P_1 V_1 - \frac 32 P_4 V_4 = \frac 32 \cdot 6 P_0 V_0 - \frac 32 P_0 V_0 = \frac 32 \cdot 5 \cdot P_0V_0.
    \\
    \eta &= \frac{A_\text{цикл}}{Q_+} = \frac{A_\text{цикл}}{Q_{12} + Q_{41}}  = \frac{A_\text{цикл}}{A_{12} + \Delta U_{12} + A_{41} + \Delta U_{41}} =  \\
     &= \frac{10P_0V_0}{12P_0V_0 + \frac 32 \cdot 12 \cdot P_0V_0 + 0 + \frac 32 \cdot 5 \cdot P_0V_0} = \frac{10}{12 + \frac 32 \cdot 12 + \frac 32 \cdot 5} = \frac4{15} \approx 0{,}267.
     \\
    \eta_\text{Карно} &= 1 - \frac{T_\text{х}}{T_\text{н}} = 1 - \frac{T_\text{4}}{T_\text{2}} = 1 - \frac{\frac{P_4V_4}{\nu R}}{\frac{P_2V_2}{\nu R}} = 1 - \frac{P_4V_4}{P_2V_2} = 1 - \frac{P_0V_0}{6P_0 \cdot 3V_0} = 1 - \frac 1{6 \cdot 3}  = \frac{17}{18} \approx 0{,}944.
    \end{align*}
}
\solutionspace{360pt}

\tasknumber{2}%
\task{%
    Порция идеального одноатомного газа перешла из состояния 1 в состояние 2: $P_1 = 2\,\text{МПа}$, $V_1 = 3\,\text{л}$, $P_2 = 1{,}5\,\text{МПа}$, $V_2 = 2\,\text{л}$.
    Известно, что в $PV$-координатах график процесса 12 представляет собой отрезок прямой.
    Определите,
    \begin{itemize}
        \item какую работу при этом совершил газ,
        \item чему равно изменение внутренней энергии газа,
        \item сколько теплоты подвели к нему в этом процессе?
    \end{itemize}
    При решении обратите внимание на знаки искомых величин.
}
\answer{%
    \begin{align*}
    P_1V_1 &= \nu R T_1, P_2V_2 = \nu R T_2, \\
    \Delta U &= U_2-U_1 = \frac 32 \nu R T_2- \frac 32 \nu R T_1 = \frac 32 P_2 V_2 - \frac 32 P_1 V_1= \frac 32 \cdot \cbr{1{,}5\,\text{МПа} \cdot 2\,\text{л} - 2\,\text{МПа} \cdot 3\,\text{л}} = -4{,}500\,\text{кДж}.
    \\
    A_\text{газа} &= \frac{P_2 + P_1} 2 \cdot (V_2 - V_1) = \frac{1{,}5\,\text{МПа} + 2\,\text{МПа}} 2 \cdot (2\,\text{л} - 3\,\text{л}) = -1{,}7500\,\text{кДж}, \\
    Q &= A_\text{газа} + \Delta U = \frac 32 (P_2 V_2 - P_1 V_1) + \frac{P_2 + P_1} 2 \cdot (V_2 - V_1) = -4{,}500\,\text{кДж} -1{,}7500\,\text{кДж} = -6{,}250\,\text{кДж}.
    \end{align*}
}
\solutionspace{150pt}

\tasknumber{3}%
\task{%
    Запишите формулы и рядом с каждой физичической величиной укажите её название и единицы измерения в СИ:
    \begin{enumerate}
        \item первое начало термодинамики,
        \item внутренняя энергия идеального одноатомного газа.
    \end{enumerate}
}

\variantsplitter

\addpersonalvariant{Евгений Васин}

\tasknumber{1}%
\task{%
    Определите КПД цикла 12341, рабочим телом которого является идеальный одноатомный газ, если
    12 — изобарическое расширение газа в четыре раза,
    23 — изохорическое охлаждение газа, при котором температура уменьшается в пять раз,
    34 — изобара, 41 — изохора.
    % Для этого:
    % \begin{enumerate}
    %     \item сделайте рисунок в $PV$-координатах,
    %     \item выберите удобные обозначения, чтобы не запутаться в множестве температур, давлений и объёмов,
    %     \item вычислите необходимые соотнощения между температурами, давлениями и объёмами
    %     (некоторые сразу видны по рисунку, некоторые — надо считать),
    %     \item определите для каждого участка поглощается или отдаётся тепло (и сколько именно:
    %     потребуется первое начало термодинамики, отдельный расчёт работ на участках через площади фигур и изменений внутренней энергии),
    %     \item вычислите полную работу газа в цикле,
    %     \item подставьте всё в формулу для КПД, упростите и доведите до ответа.
    % \end{enumerate}
    Определите КПД цикла Карно, температура нагревателя которого равна максимальной температуре в цикле 12341, а холодильника — минимальной.
    Ответы в обоих случаях оставьте точными в виде нескоратимой дроби, никаких округлений.
}
\answer{%
    \begin{align*}
    A_{12} &> 0, \Delta U_{12} > 0, \implies Q_{12} = A_{12} + \Delta U_{12} > 0, \\
    A_{23} &= 0, \Delta U_{23} < 0, \implies Q_{23} = A_{23} + \Delta U_{23} < 0, \\
    A_{34} &< 0, \Delta U_{34} < 0, \implies Q_{34} = A_{34} + \Delta U_{34} < 0, \\
    A_{41} &= 0, \Delta U_{41} > 0, \implies Q_{41} = A_{41} + \Delta U_{41} > 0.
    \\
    P_1V_1 &= \nu R T_1, P_2V_2 = \nu R T_2, P_3V_3 = \nu R T_3, P_4V_4 = \nu R T_4 \text{ — уравнения состояния идеального газа}, \\
    &\text{Пусть $P_0$, $V_0$, $T_0$ — давление, объём и температура в точке 4 (минимальные во всём цикле):} \\
    P_1 &= P_2, P_3 = P_4 = P_0, V_1 = V_4 = V_0, V_2 = V_3 = 4 V_1 = 4 V_0,, \text{остальные соотношения между объёмами и давлениями не даны, нужно считать} \\
    T_3 &= \frac{T_2}5 \text{(по условию)} \implies \frac{P_2}{P_3} = \frac{P_2 V_2}{P_3 V_3}= \frac{\nu R T_2}{\nu R T_3} = \frac{T_2}{T_3} = 5 \implies P_1 = P_2 = 5 P_0 \\
    A_\text{цикл} &= (5P_0 - P_0)(4V_0 - V_0) = 12P_0V_0, \\
    A_{12} &= 5P_0 \cdot (4V_0 - V_0) = 15P_0V_0, \\
    \Delta U_{12} &= \frac 32 \nu R T_2 - \frac 32 \nu R T_1 = \frac 32 P_2 V_2 - \frac 32 P_1 V_1 = \frac 32 \cdot 5 P_0 \cdot 4 V_0 -  \frac 32 \cdot 5 P_0 \cdot V_0 = \frac 32 \cdot 15 \cdot P_0V_0, \\
    \Delta U_{41} &= \frac 32 \nu R T_1 - \frac 32 \nu R T_4 = \frac 32 P_1 V_1 - \frac 32 P_4 V_4 = \frac 32 \cdot 5 P_0 V_0 - \frac 32 P_0 V_0 = \frac 32 \cdot 4 \cdot P_0V_0.
    \\
    \eta &= \frac{A_\text{цикл}}{Q_+} = \frac{A_\text{цикл}}{Q_{12} + Q_{41}}  = \frac{A_\text{цикл}}{A_{12} + \Delta U_{12} + A_{41} + \Delta U_{41}} =  \\
     &= \frac{12P_0V_0}{15P_0V_0 + \frac 32 \cdot 15 \cdot P_0V_0 + 0 + \frac 32 \cdot 4 \cdot P_0V_0} = \frac{12}{15 + \frac 32 \cdot 15 + \frac 32 \cdot 4} = \frac8{29} \approx 0{,}276.
     \\
    \eta_\text{Карно} &= 1 - \frac{T_\text{х}}{T_\text{н}} = 1 - \frac{T_\text{4}}{T_\text{2}} = 1 - \frac{\frac{P_4V_4}{\nu R}}{\frac{P_2V_2}{\nu R}} = 1 - \frac{P_4V_4}{P_2V_2} = 1 - \frac{P_0V_0}{5P_0 \cdot 4V_0} = 1 - \frac 1{5 \cdot 4}  = \frac{19}{20} \approx 0{,}950.
    \end{align*}
}
\solutionspace{360pt}

\tasknumber{2}%
\task{%
    Порция идеального одноатомного газа перешла из состояния 1 в состояние 2: $P_1 = 2\,\text{МПа}$, $V_1 = 7\,\text{л}$, $P_2 = 3{,}5\,\text{МПа}$, $V_2 = 4\,\text{л}$.
    Известно, что в $PV$-координатах график процесса 12 представляет собой отрезок прямой.
    Определите,
    \begin{itemize}
        \item какую работу при этом совершил газ,
        \item чему равно изменение внутренней энергии газа,
        \item сколько теплоты подвели к нему в этом процессе?
    \end{itemize}
    При решении обратите внимание на знаки искомых величин.
}
\answer{%
    \begin{align*}
    P_1V_1 &= \nu R T_1, P_2V_2 = \nu R T_2, \\
    \Delta U &= U_2-U_1 = \frac 32 \nu R T_2- \frac 32 \nu R T_1 = \frac 32 P_2 V_2 - \frac 32 P_1 V_1= \frac 32 \cdot \cbr{3{,}5\,\text{МПа} \cdot 4\,\text{л} - 2\,\text{МПа} \cdot 7\,\text{л}} = 0\,\text{кДж}.
    \\
    A_\text{газа} &= \frac{P_2 + P_1} 2 \cdot (V_2 - V_1) = \frac{3{,}5\,\text{МПа} + 2\,\text{МПа}} 2 \cdot (4\,\text{л} - 7\,\text{л}) = -8{,}250\,\text{кДж}, \\
    Q &= A_\text{газа} + \Delta U = \frac 32 (P_2 V_2 - P_1 V_1) + \frac{P_2 + P_1} 2 \cdot (V_2 - V_1) = 0\,\text{кДж} -8{,}250\,\text{кДж} = -8{,}250\,\text{кДж}.
    \end{align*}
}
\solutionspace{150pt}

\tasknumber{3}%
\task{%
    Запишите формулы и рядом с каждой физичической величиной укажите её название и единицы измерения в СИ:
    \begin{enumerate}
        \item первое начало термодинамики,
        \item внутренняя энергия идеального одноатомного газа.
    \end{enumerate}
}

\variantsplitter

\addpersonalvariant{Вячеслав Волохов}

\tasknumber{1}%
\task{%
    Определите КПД цикла 12341, рабочим телом которого является идеальный одноатомный газ, если
    12 — изобарическое расширение газа в пять раз,
    23 — изохорическое охлаждение газа, при котором температура уменьшается в три раза,
    34 — изобара, 41 — изохора.
    % Для этого:
    % \begin{enumerate}
    %     \item сделайте рисунок в $PV$-координатах,
    %     \item выберите удобные обозначения, чтобы не запутаться в множестве температур, давлений и объёмов,
    %     \item вычислите необходимые соотнощения между температурами, давлениями и объёмами
    %     (некоторые сразу видны по рисунку, некоторые — надо считать),
    %     \item определите для каждого участка поглощается или отдаётся тепло (и сколько именно:
    %     потребуется первое начало термодинамики, отдельный расчёт работ на участках через площади фигур и изменений внутренней энергии),
    %     \item вычислите полную работу газа в цикле,
    %     \item подставьте всё в формулу для КПД, упростите и доведите до ответа.
    % \end{enumerate}
    Определите КПД цикла Карно, температура нагревателя которого равна максимальной температуре в цикле 12341, а холодильника — минимальной.
    Ответы в обоих случаях оставьте точными в виде нескоратимой дроби, никаких округлений.
}
\answer{%
    \begin{align*}
    A_{12} &> 0, \Delta U_{12} > 0, \implies Q_{12} = A_{12} + \Delta U_{12} > 0, \\
    A_{23} &= 0, \Delta U_{23} < 0, \implies Q_{23} = A_{23} + \Delta U_{23} < 0, \\
    A_{34} &< 0, \Delta U_{34} < 0, \implies Q_{34} = A_{34} + \Delta U_{34} < 0, \\
    A_{41} &= 0, \Delta U_{41} > 0, \implies Q_{41} = A_{41} + \Delta U_{41} > 0.
    \\
    P_1V_1 &= \nu R T_1, P_2V_2 = \nu R T_2, P_3V_3 = \nu R T_3, P_4V_4 = \nu R T_4 \text{ — уравнения состояния идеального газа}, \\
    &\text{Пусть $P_0$, $V_0$, $T_0$ — давление, объём и температура в точке 4 (минимальные во всём цикле):} \\
    P_1 &= P_2, P_3 = P_4 = P_0, V_1 = V_4 = V_0, V_2 = V_3 = 5 V_1 = 5 V_0,, \text{остальные соотношения между объёмами и давлениями не даны, нужно считать} \\
    T_3 &= \frac{T_2}3 \text{(по условию)} \implies \frac{P_2}{P_3} = \frac{P_2 V_2}{P_3 V_3}= \frac{\nu R T_2}{\nu R T_3} = \frac{T_2}{T_3} = 3 \implies P_1 = P_2 = 3 P_0 \\
    A_\text{цикл} &= (3P_0 - P_0)(5V_0 - V_0) = 8P_0V_0, \\
    A_{12} &= 3P_0 \cdot (5V_0 - V_0) = 12P_0V_0, \\
    \Delta U_{12} &= \frac 32 \nu R T_2 - \frac 32 \nu R T_1 = \frac 32 P_2 V_2 - \frac 32 P_1 V_1 = \frac 32 \cdot 3 P_0 \cdot 5 V_0 -  \frac 32 \cdot 3 P_0 \cdot V_0 = \frac 32 \cdot 12 \cdot P_0V_0, \\
    \Delta U_{41} &= \frac 32 \nu R T_1 - \frac 32 \nu R T_4 = \frac 32 P_1 V_1 - \frac 32 P_4 V_4 = \frac 32 \cdot 3 P_0 V_0 - \frac 32 P_0 V_0 = \frac 32 \cdot 2 \cdot P_0V_0.
    \\
    \eta &= \frac{A_\text{цикл}}{Q_+} = \frac{A_\text{цикл}}{Q_{12} + Q_{41}}  = \frac{A_\text{цикл}}{A_{12} + \Delta U_{12} + A_{41} + \Delta U_{41}} =  \\
     &= \frac{8P_0V_0}{12P_0V_0 + \frac 32 \cdot 12 \cdot P_0V_0 + 0 + \frac 32 \cdot 2 \cdot P_0V_0} = \frac{8}{12 + \frac 32 \cdot 12 + \frac 32 \cdot 2} = \frac8{33} \approx 0{,}242.
     \\
    \eta_\text{Карно} &= 1 - \frac{T_\text{х}}{T_\text{н}} = 1 - \frac{T_\text{4}}{T_\text{2}} = 1 - \frac{\frac{P_4V_4}{\nu R}}{\frac{P_2V_2}{\nu R}} = 1 - \frac{P_4V_4}{P_2V_2} = 1 - \frac{P_0V_0}{3P_0 \cdot 5V_0} = 1 - \frac 1{3 \cdot 5}  = \frac{14}{15} \approx 0{,}933.
    \end{align*}
}
\solutionspace{360pt}

\tasknumber{2}%
\task{%
    Порция идеального одноатомного газа перешла из состояния 1 в состояние 2: $P_1 = 4\,\text{МПа}$, $V_1 = 5\,\text{л}$, $P_2 = 3{,}5\,\text{МПа}$, $V_2 = 6\,\text{л}$.
    Известно, что в $PV$-координатах график процесса 12 представляет собой отрезок прямой.
    Определите,
    \begin{itemize}
        \item какую работу при этом совершил газ,
        \item чему равно изменение внутренней энергии газа,
        \item сколько теплоты подвели к нему в этом процессе?
    \end{itemize}
    При решении обратите внимание на знаки искомых величин.
}
\answer{%
    \begin{align*}
    P_1V_1 &= \nu R T_1, P_2V_2 = \nu R T_2, \\
    \Delta U &= U_2-U_1 = \frac 32 \nu R T_2- \frac 32 \nu R T_1 = \frac 32 P_2 V_2 - \frac 32 P_1 V_1= \frac 32 \cdot \cbr{3{,}5\,\text{МПа} \cdot 6\,\text{л} - 4\,\text{МПа} \cdot 5\,\text{л}} = 1{,}50\,\text{кДж}.
    \\
    A_\text{газа} &= \frac{P_2 + P_1} 2 \cdot (V_2 - V_1) = \frac{3{,}5\,\text{МПа} + 4\,\text{МПа}} 2 \cdot (6\,\text{л} - 5\,\text{л}) = 3{,}75\,\text{кДж}, \\
    Q &= A_\text{газа} + \Delta U = \frac 32 (P_2 V_2 - P_1 V_1) + \frac{P_2 + P_1} 2 \cdot (V_2 - V_1) = 1{,}50\,\text{кДж} + 3{,}75\,\text{кДж} = 5{,}25\,\text{кДж}.
    \end{align*}
}
\solutionspace{150pt}

\tasknumber{3}%
\task{%
    Запишите формулы и рядом с каждой физичической величиной укажите её название и единицы измерения в СИ:
    \begin{enumerate}
        \item первое начало термодинамики,
        \item внутренняя энергия идеального одноатомного газа.
    \end{enumerate}
}

\variantsplitter

\addpersonalvariant{Герман Говоров}

\tasknumber{1}%
\task{%
    Определите КПД цикла 12341, рабочим телом которого является идеальный одноатомный газ, если
    12 — изобарическое расширение газа в два раза,
    23 — изохорическое охлаждение газа, при котором температура уменьшается в пять раз,
    34 — изобара, 41 — изохора.
    % Для этого:
    % \begin{enumerate}
    %     \item сделайте рисунок в $PV$-координатах,
    %     \item выберите удобные обозначения, чтобы не запутаться в множестве температур, давлений и объёмов,
    %     \item вычислите необходимые соотнощения между температурами, давлениями и объёмами
    %     (некоторые сразу видны по рисунку, некоторые — надо считать),
    %     \item определите для каждого участка поглощается или отдаётся тепло (и сколько именно:
    %     потребуется первое начало термодинамики, отдельный расчёт работ на участках через площади фигур и изменений внутренней энергии),
    %     \item вычислите полную работу газа в цикле,
    %     \item подставьте всё в формулу для КПД, упростите и доведите до ответа.
    % \end{enumerate}
    Определите КПД цикла Карно, температура нагревателя которого равна максимальной температуре в цикле 12341, а холодильника — минимальной.
    Ответы в обоих случаях оставьте точными в виде нескоратимой дроби, никаких округлений.
}
\answer{%
    \begin{align*}
    A_{12} &> 0, \Delta U_{12} > 0, \implies Q_{12} = A_{12} + \Delta U_{12} > 0, \\
    A_{23} &= 0, \Delta U_{23} < 0, \implies Q_{23} = A_{23} + \Delta U_{23} < 0, \\
    A_{34} &< 0, \Delta U_{34} < 0, \implies Q_{34} = A_{34} + \Delta U_{34} < 0, \\
    A_{41} &= 0, \Delta U_{41} > 0, \implies Q_{41} = A_{41} + \Delta U_{41} > 0.
    \\
    P_1V_1 &= \nu R T_1, P_2V_2 = \nu R T_2, P_3V_3 = \nu R T_3, P_4V_4 = \nu R T_4 \text{ — уравнения состояния идеального газа}, \\
    &\text{Пусть $P_0$, $V_0$, $T_0$ — давление, объём и температура в точке 4 (минимальные во всём цикле):} \\
    P_1 &= P_2, P_3 = P_4 = P_0, V_1 = V_4 = V_0, V_2 = V_3 = 2 V_1 = 2 V_0,, \text{остальные соотношения между объёмами и давлениями не даны, нужно считать} \\
    T_3 &= \frac{T_2}5 \text{(по условию)} \implies \frac{P_2}{P_3} = \frac{P_2 V_2}{P_3 V_3}= \frac{\nu R T_2}{\nu R T_3} = \frac{T_2}{T_3} = 5 \implies P_1 = P_2 = 5 P_0 \\
    A_\text{цикл} &= (5P_0 - P_0)(2V_0 - V_0) = 4P_0V_0, \\
    A_{12} &= 5P_0 \cdot (2V_0 - V_0) = 5P_0V_0, \\
    \Delta U_{12} &= \frac 32 \nu R T_2 - \frac 32 \nu R T_1 = \frac 32 P_2 V_2 - \frac 32 P_1 V_1 = \frac 32 \cdot 5 P_0 \cdot 2 V_0 -  \frac 32 \cdot 5 P_0 \cdot V_0 = \frac 32 \cdot 5 \cdot P_0V_0, \\
    \Delta U_{41} &= \frac 32 \nu R T_1 - \frac 32 \nu R T_4 = \frac 32 P_1 V_1 - \frac 32 P_4 V_4 = \frac 32 \cdot 5 P_0 V_0 - \frac 32 P_0 V_0 = \frac 32 \cdot 4 \cdot P_0V_0.
    \\
    \eta &= \frac{A_\text{цикл}}{Q_+} = \frac{A_\text{цикл}}{Q_{12} + Q_{41}}  = \frac{A_\text{цикл}}{A_{12} + \Delta U_{12} + A_{41} + \Delta U_{41}} =  \\
     &= \frac{4P_0V_0}{5P_0V_0 + \frac 32 \cdot 5 \cdot P_0V_0 + 0 + \frac 32 \cdot 4 \cdot P_0V_0} = \frac{4}{5 + \frac 32 \cdot 5 + \frac 32 \cdot 4} = \frac8{37} \approx 0{,}216.
     \\
    \eta_\text{Карно} &= 1 - \frac{T_\text{х}}{T_\text{н}} = 1 - \frac{T_\text{4}}{T_\text{2}} = 1 - \frac{\frac{P_4V_4}{\nu R}}{\frac{P_2V_2}{\nu R}} = 1 - \frac{P_4V_4}{P_2V_2} = 1 - \frac{P_0V_0}{5P_0 \cdot 2V_0} = 1 - \frac 1{5 \cdot 2}  = \frac9{10} \approx 0{,}900.
    \end{align*}
}
\solutionspace{360pt}

\tasknumber{2}%
\task{%
    Порция идеального одноатомного газа перешла из состояния 1 в состояние 2: $P_1 = 2\,\text{МПа}$, $V_1 = 7\,\text{л}$, $P_2 = 2{,}5\,\text{МПа}$, $V_2 = 6\,\text{л}$.
    Известно, что в $PV$-координатах график процесса 12 представляет собой отрезок прямой.
    Определите,
    \begin{itemize}
        \item какую работу при этом совершил газ,
        \item чему равно изменение внутренней энергии газа,
        \item сколько теплоты подвели к нему в этом процессе?
    \end{itemize}
    При решении обратите внимание на знаки искомых величин.
}
\answer{%
    \begin{align*}
    P_1V_1 &= \nu R T_1, P_2V_2 = \nu R T_2, \\
    \Delta U &= U_2-U_1 = \frac 32 \nu R T_2- \frac 32 \nu R T_1 = \frac 32 P_2 V_2 - \frac 32 P_1 V_1= \frac 32 \cdot \cbr{2{,}5\,\text{МПа} \cdot 6\,\text{л} - 2\,\text{МПа} \cdot 7\,\text{л}} = 1{,}50\,\text{кДж}.
    \\
    A_\text{газа} &= \frac{P_2 + P_1} 2 \cdot (V_2 - V_1) = \frac{2{,}5\,\text{МПа} + 2\,\text{МПа}} 2 \cdot (6\,\text{л} - 7\,\text{л}) = -2{,}250\,\text{кДж}, \\
    Q &= A_\text{газа} + \Delta U = \frac 32 (P_2 V_2 - P_1 V_1) + \frac{P_2 + P_1} 2 \cdot (V_2 - V_1) = 1{,}50\,\text{кДж} -2{,}250\,\text{кДж} = -0{,}7500\,\text{кДж}.
    \end{align*}
}
\solutionspace{150pt}

\tasknumber{3}%
\task{%
    Запишите формулы и рядом с каждой физичической величиной укажите её название и единицы измерения в СИ:
    \begin{enumerate}
        \item первое начало термодинамики,
        \item внутренняя энергия идеального одноатомного газа.
    \end{enumerate}
}

\variantsplitter

\addpersonalvariant{София Журавлёва}

\tasknumber{1}%
\task{%
    Определите КПД цикла 12341, рабочим телом которого является идеальный одноатомный газ, если
    12 — изобарическое расширение газа в пять раз,
    23 — изохорическое охлаждение газа, при котором температура уменьшается в пять раз,
    34 — изобара, 41 — изохора.
    % Для этого:
    % \begin{enumerate}
    %     \item сделайте рисунок в $PV$-координатах,
    %     \item выберите удобные обозначения, чтобы не запутаться в множестве температур, давлений и объёмов,
    %     \item вычислите необходимые соотнощения между температурами, давлениями и объёмами
    %     (некоторые сразу видны по рисунку, некоторые — надо считать),
    %     \item определите для каждого участка поглощается или отдаётся тепло (и сколько именно:
    %     потребуется первое начало термодинамики, отдельный расчёт работ на участках через площади фигур и изменений внутренней энергии),
    %     \item вычислите полную работу газа в цикле,
    %     \item подставьте всё в формулу для КПД, упростите и доведите до ответа.
    % \end{enumerate}
    Определите КПД цикла Карно, температура нагревателя которого равна максимальной температуре в цикле 12341, а холодильника — минимальной.
    Ответы в обоих случаях оставьте точными в виде нескоратимой дроби, никаких округлений.
}
\answer{%
    \begin{align*}
    A_{12} &> 0, \Delta U_{12} > 0, \implies Q_{12} = A_{12} + \Delta U_{12} > 0, \\
    A_{23} &= 0, \Delta U_{23} < 0, \implies Q_{23} = A_{23} + \Delta U_{23} < 0, \\
    A_{34} &< 0, \Delta U_{34} < 0, \implies Q_{34} = A_{34} + \Delta U_{34} < 0, \\
    A_{41} &= 0, \Delta U_{41} > 0, \implies Q_{41} = A_{41} + \Delta U_{41} > 0.
    \\
    P_1V_1 &= \nu R T_1, P_2V_2 = \nu R T_2, P_3V_3 = \nu R T_3, P_4V_4 = \nu R T_4 \text{ — уравнения состояния идеального газа}, \\
    &\text{Пусть $P_0$, $V_0$, $T_0$ — давление, объём и температура в точке 4 (минимальные во всём цикле):} \\
    P_1 &= P_2, P_3 = P_4 = P_0, V_1 = V_4 = V_0, V_2 = V_3 = 5 V_1 = 5 V_0,, \text{остальные соотношения между объёмами и давлениями не даны, нужно считать} \\
    T_3 &= \frac{T_2}5 \text{(по условию)} \implies \frac{P_2}{P_3} = \frac{P_2 V_2}{P_3 V_3}= \frac{\nu R T_2}{\nu R T_3} = \frac{T_2}{T_3} = 5 \implies P_1 = P_2 = 5 P_0 \\
    A_\text{цикл} &= (5P_0 - P_0)(5V_0 - V_0) = 16P_0V_0, \\
    A_{12} &= 5P_0 \cdot (5V_0 - V_0) = 20P_0V_0, \\
    \Delta U_{12} &= \frac 32 \nu R T_2 - \frac 32 \nu R T_1 = \frac 32 P_2 V_2 - \frac 32 P_1 V_1 = \frac 32 \cdot 5 P_0 \cdot 5 V_0 -  \frac 32 \cdot 5 P_0 \cdot V_0 = \frac 32 \cdot 20 \cdot P_0V_0, \\
    \Delta U_{41} &= \frac 32 \nu R T_1 - \frac 32 \nu R T_4 = \frac 32 P_1 V_1 - \frac 32 P_4 V_4 = \frac 32 \cdot 5 P_0 V_0 - \frac 32 P_0 V_0 = \frac 32 \cdot 4 \cdot P_0V_0.
    \\
    \eta &= \frac{A_\text{цикл}}{Q_+} = \frac{A_\text{цикл}}{Q_{12} + Q_{41}}  = \frac{A_\text{цикл}}{A_{12} + \Delta U_{12} + A_{41} + \Delta U_{41}} =  \\
     &= \frac{16P_0V_0}{20P_0V_0 + \frac 32 \cdot 20 \cdot P_0V_0 + 0 + \frac 32 \cdot 4 \cdot P_0V_0} = \frac{16}{20 + \frac 32 \cdot 20 + \frac 32 \cdot 4} = \frac27 \approx 0{,}286.
     \\
    \eta_\text{Карно} &= 1 - \frac{T_\text{х}}{T_\text{н}} = 1 - \frac{T_\text{4}}{T_\text{2}} = 1 - \frac{\frac{P_4V_4}{\nu R}}{\frac{P_2V_2}{\nu R}} = 1 - \frac{P_4V_4}{P_2V_2} = 1 - \frac{P_0V_0}{5P_0 \cdot 5V_0} = 1 - \frac 1{5 \cdot 5}  = \frac{24}{25} \approx 0{,}960.
    \end{align*}
}
\solutionspace{360pt}

\tasknumber{2}%
\task{%
    Порция идеального одноатомного газа перешла из состояния 1 в состояние 2: $P_1 = 4\,\text{МПа}$, $V_1 = 5\,\text{л}$, $P_2 = 2{,}5\,\text{МПа}$, $V_2 = 8\,\text{л}$.
    Известно, что в $PV$-координатах график процесса 12 представляет собой отрезок прямой.
    Определите,
    \begin{itemize}
        \item какую работу при этом совершил газ,
        \item чему равно изменение внутренней энергии газа,
        \item сколько теплоты подвели к нему в этом процессе?
    \end{itemize}
    При решении обратите внимание на знаки искомых величин.
}
\answer{%
    \begin{align*}
    P_1V_1 &= \nu R T_1, P_2V_2 = \nu R T_2, \\
    \Delta U &= U_2-U_1 = \frac 32 \nu R T_2- \frac 32 \nu R T_1 = \frac 32 P_2 V_2 - \frac 32 P_1 V_1= \frac 32 \cdot \cbr{2{,}5\,\text{МПа} \cdot 8\,\text{л} - 4\,\text{МПа} \cdot 5\,\text{л}} = 0\,\text{кДж}.
    \\
    A_\text{газа} &= \frac{P_2 + P_1} 2 \cdot (V_2 - V_1) = \frac{2{,}5\,\text{МПа} + 4\,\text{МПа}} 2 \cdot (8\,\text{л} - 5\,\text{л}) = 9{,}75\,\text{кДж}, \\
    Q &= A_\text{газа} + \Delta U = \frac 32 (P_2 V_2 - P_1 V_1) + \frac{P_2 + P_1} 2 \cdot (V_2 - V_1) = 0\,\text{кДж} + 9{,}75\,\text{кДж} = 9{,}75\,\text{кДж}.
    \end{align*}
}
\solutionspace{150pt}

\tasknumber{3}%
\task{%
    Запишите формулы и рядом с каждой физичической величиной укажите её название и единицы измерения в СИ:
    \begin{enumerate}
        \item первое начало термодинамики,
        \item внутренняя энергия идеального одноатомного газа.
    \end{enumerate}
}

\variantsplitter

\addpersonalvariant{Константин Козлов}

\tasknumber{1}%
\task{%
    Определите КПД цикла 12341, рабочим телом которого является идеальный одноатомный газ, если
    12 — изобарическое расширение газа в шесть раз,
    23 — изохорическое охлаждение газа, при котором температура уменьшается в шесть раз,
    34 — изобара, 41 — изохора.
    % Для этого:
    % \begin{enumerate}
    %     \item сделайте рисунок в $PV$-координатах,
    %     \item выберите удобные обозначения, чтобы не запутаться в множестве температур, давлений и объёмов,
    %     \item вычислите необходимые соотнощения между температурами, давлениями и объёмами
    %     (некоторые сразу видны по рисунку, некоторые — надо считать),
    %     \item определите для каждого участка поглощается или отдаётся тепло (и сколько именно:
    %     потребуется первое начало термодинамики, отдельный расчёт работ на участках через площади фигур и изменений внутренней энергии),
    %     \item вычислите полную работу газа в цикле,
    %     \item подставьте всё в формулу для КПД, упростите и доведите до ответа.
    % \end{enumerate}
    Определите КПД цикла Карно, температура нагревателя которого равна максимальной температуре в цикле 12341, а холодильника — минимальной.
    Ответы в обоих случаях оставьте точными в виде нескоратимой дроби, никаких округлений.
}
\answer{%
    \begin{align*}
    A_{12} &> 0, \Delta U_{12} > 0, \implies Q_{12} = A_{12} + \Delta U_{12} > 0, \\
    A_{23} &= 0, \Delta U_{23} < 0, \implies Q_{23} = A_{23} + \Delta U_{23} < 0, \\
    A_{34} &< 0, \Delta U_{34} < 0, \implies Q_{34} = A_{34} + \Delta U_{34} < 0, \\
    A_{41} &= 0, \Delta U_{41} > 0, \implies Q_{41} = A_{41} + \Delta U_{41} > 0.
    \\
    P_1V_1 &= \nu R T_1, P_2V_2 = \nu R T_2, P_3V_3 = \nu R T_3, P_4V_4 = \nu R T_4 \text{ — уравнения состояния идеального газа}, \\
    &\text{Пусть $P_0$, $V_0$, $T_0$ — давление, объём и температура в точке 4 (минимальные во всём цикле):} \\
    P_1 &= P_2, P_3 = P_4 = P_0, V_1 = V_4 = V_0, V_2 = V_3 = 6 V_1 = 6 V_0,, \text{остальные соотношения между объёмами и давлениями не даны, нужно считать} \\
    T_3 &= \frac{T_2}6 \text{(по условию)} \implies \frac{P_2}{P_3} = \frac{P_2 V_2}{P_3 V_3}= \frac{\nu R T_2}{\nu R T_3} = \frac{T_2}{T_3} = 6 \implies P_1 = P_2 = 6 P_0 \\
    A_\text{цикл} &= (6P_0 - P_0)(6V_0 - V_0) = 25P_0V_0, \\
    A_{12} &= 6P_0 \cdot (6V_0 - V_0) = 30P_0V_0, \\
    \Delta U_{12} &= \frac 32 \nu R T_2 - \frac 32 \nu R T_1 = \frac 32 P_2 V_2 - \frac 32 P_1 V_1 = \frac 32 \cdot 6 P_0 \cdot 6 V_0 -  \frac 32 \cdot 6 P_0 \cdot V_0 = \frac 32 \cdot 30 \cdot P_0V_0, \\
    \Delta U_{41} &= \frac 32 \nu R T_1 - \frac 32 \nu R T_4 = \frac 32 P_1 V_1 - \frac 32 P_4 V_4 = \frac 32 \cdot 6 P_0 V_0 - \frac 32 P_0 V_0 = \frac 32 \cdot 5 \cdot P_0V_0.
    \\
    \eta &= \frac{A_\text{цикл}}{Q_+} = \frac{A_\text{цикл}}{Q_{12} + Q_{41}}  = \frac{A_\text{цикл}}{A_{12} + \Delta U_{12} + A_{41} + \Delta U_{41}} =  \\
     &= \frac{25P_0V_0}{30P_0V_0 + \frac 32 \cdot 30 \cdot P_0V_0 + 0 + \frac 32 \cdot 5 \cdot P_0V_0} = \frac{25}{30 + \frac 32 \cdot 30 + \frac 32 \cdot 5} = \frac{10}{33} \approx 0{,}303.
     \\
    \eta_\text{Карно} &= 1 - \frac{T_\text{х}}{T_\text{н}} = 1 - \frac{T_\text{4}}{T_\text{2}} = 1 - \frac{\frac{P_4V_4}{\nu R}}{\frac{P_2V_2}{\nu R}} = 1 - \frac{P_4V_4}{P_2V_2} = 1 - \frac{P_0V_0}{6P_0 \cdot 6V_0} = 1 - \frac 1{6 \cdot 6}  = \frac{35}{36} \approx 0{,}972.
    \end{align*}
}
\solutionspace{360pt}

\tasknumber{2}%
\task{%
    Порция идеального одноатомного газа перешла из состояния 1 в состояние 2: $P_1 = 2\,\text{МПа}$, $V_1 = 5\,\text{л}$, $P_2 = 1{,}5\,\text{МПа}$, $V_2 = 8\,\text{л}$.
    Известно, что в $PV$-координатах график процесса 12 представляет собой отрезок прямой.
    Определите,
    \begin{itemize}
        \item какую работу при этом совершил газ,
        \item чему равно изменение внутренней энергии газа,
        \item сколько теплоты подвели к нему в этом процессе?
    \end{itemize}
    При решении обратите внимание на знаки искомых величин.
}
\answer{%
    \begin{align*}
    P_1V_1 &= \nu R T_1, P_2V_2 = \nu R T_2, \\
    \Delta U &= U_2-U_1 = \frac 32 \nu R T_2- \frac 32 \nu R T_1 = \frac 32 P_2 V_2 - \frac 32 P_1 V_1= \frac 32 \cdot \cbr{1{,}5\,\text{МПа} \cdot 8\,\text{л} - 2\,\text{МПа} \cdot 5\,\text{л}} = 3{,}00\,\text{кДж}.
    \\
    A_\text{газа} &= \frac{P_2 + P_1} 2 \cdot (V_2 - V_1) = \frac{1{,}5\,\text{МПа} + 2\,\text{МПа}} 2 \cdot (8\,\text{л} - 5\,\text{л}) = 5{,}25\,\text{кДж}, \\
    Q &= A_\text{газа} + \Delta U = \frac 32 (P_2 V_2 - P_1 V_1) + \frac{P_2 + P_1} 2 \cdot (V_2 - V_1) = 3{,}00\,\text{кДж} + 5{,}25\,\text{кДж} = 8{,}25\,\text{кДж}.
    \end{align*}
}
\solutionspace{150pt}

\tasknumber{3}%
\task{%
    Запишите формулы и рядом с каждой физичической величиной укажите её название и единицы измерения в СИ:
    \begin{enumerate}
        \item первое начало термодинамики,
        \item внутренняя энергия идеального одноатомного газа.
    \end{enumerate}
}

\variantsplitter

\addpersonalvariant{Наталья Кравченко}

\tasknumber{1}%
\task{%
    Определите КПД цикла 12341, рабочим телом которого является идеальный одноатомный газ, если
    12 — изобарическое расширение газа в два раза,
    23 — изохорическое охлаждение газа, при котором температура уменьшается в шесть раз,
    34 — изобара, 41 — изохора.
    % Для этого:
    % \begin{enumerate}
    %     \item сделайте рисунок в $PV$-координатах,
    %     \item выберите удобные обозначения, чтобы не запутаться в множестве температур, давлений и объёмов,
    %     \item вычислите необходимые соотнощения между температурами, давлениями и объёмами
    %     (некоторые сразу видны по рисунку, некоторые — надо считать),
    %     \item определите для каждого участка поглощается или отдаётся тепло (и сколько именно:
    %     потребуется первое начало термодинамики, отдельный расчёт работ на участках через площади фигур и изменений внутренней энергии),
    %     \item вычислите полную работу газа в цикле,
    %     \item подставьте всё в формулу для КПД, упростите и доведите до ответа.
    % \end{enumerate}
    Определите КПД цикла Карно, температура нагревателя которого равна максимальной температуре в цикле 12341, а холодильника — минимальной.
    Ответы в обоих случаях оставьте точными в виде нескоратимой дроби, никаких округлений.
}
\answer{%
    \begin{align*}
    A_{12} &> 0, \Delta U_{12} > 0, \implies Q_{12} = A_{12} + \Delta U_{12} > 0, \\
    A_{23} &= 0, \Delta U_{23} < 0, \implies Q_{23} = A_{23} + \Delta U_{23} < 0, \\
    A_{34} &< 0, \Delta U_{34} < 0, \implies Q_{34} = A_{34} + \Delta U_{34} < 0, \\
    A_{41} &= 0, \Delta U_{41} > 0, \implies Q_{41} = A_{41} + \Delta U_{41} > 0.
    \\
    P_1V_1 &= \nu R T_1, P_2V_2 = \nu R T_2, P_3V_3 = \nu R T_3, P_4V_4 = \nu R T_4 \text{ — уравнения состояния идеального газа}, \\
    &\text{Пусть $P_0$, $V_0$, $T_0$ — давление, объём и температура в точке 4 (минимальные во всём цикле):} \\
    P_1 &= P_2, P_3 = P_4 = P_0, V_1 = V_4 = V_0, V_2 = V_3 = 2 V_1 = 2 V_0,, \text{остальные соотношения между объёмами и давлениями не даны, нужно считать} \\
    T_3 &= \frac{T_2}6 \text{(по условию)} \implies \frac{P_2}{P_3} = \frac{P_2 V_2}{P_3 V_3}= \frac{\nu R T_2}{\nu R T_3} = \frac{T_2}{T_3} = 6 \implies P_1 = P_2 = 6 P_0 \\
    A_\text{цикл} &= (6P_0 - P_0)(2V_0 - V_0) = 5P_0V_0, \\
    A_{12} &= 6P_0 \cdot (2V_0 - V_0) = 6P_0V_0, \\
    \Delta U_{12} &= \frac 32 \nu R T_2 - \frac 32 \nu R T_1 = \frac 32 P_2 V_2 - \frac 32 P_1 V_1 = \frac 32 \cdot 6 P_0 \cdot 2 V_0 -  \frac 32 \cdot 6 P_0 \cdot V_0 = \frac 32 \cdot 6 \cdot P_0V_0, \\
    \Delta U_{41} &= \frac 32 \nu R T_1 - \frac 32 \nu R T_4 = \frac 32 P_1 V_1 - \frac 32 P_4 V_4 = \frac 32 \cdot 6 P_0 V_0 - \frac 32 P_0 V_0 = \frac 32 \cdot 5 \cdot P_0V_0.
    \\
    \eta &= \frac{A_\text{цикл}}{Q_+} = \frac{A_\text{цикл}}{Q_{12} + Q_{41}}  = \frac{A_\text{цикл}}{A_{12} + \Delta U_{12} + A_{41} + \Delta U_{41}} =  \\
     &= \frac{5P_0V_0}{6P_0V_0 + \frac 32 \cdot 6 \cdot P_0V_0 + 0 + \frac 32 \cdot 5 \cdot P_0V_0} = \frac{5}{6 + \frac 32 \cdot 6 + \frac 32 \cdot 5} = \frac29 \approx 0{,}222.
     \\
    \eta_\text{Карно} &= 1 - \frac{T_\text{х}}{T_\text{н}} = 1 - \frac{T_\text{4}}{T_\text{2}} = 1 - \frac{\frac{P_4V_4}{\nu R}}{\frac{P_2V_2}{\nu R}} = 1 - \frac{P_4V_4}{P_2V_2} = 1 - \frac{P_0V_0}{6P_0 \cdot 2V_0} = 1 - \frac 1{6 \cdot 2}  = \frac{11}{12} \approx 0{,}917.
    \end{align*}
}
\solutionspace{360pt}

\tasknumber{2}%
\task{%
    Порция идеального одноатомного газа перешла из состояния 1 в состояние 2: $P_1 = 2\,\text{МПа}$, $V_1 = 7\,\text{л}$, $P_2 = 2{,}5\,\text{МПа}$, $V_2 = 6\,\text{л}$.
    Известно, что в $PV$-координатах график процесса 12 представляет собой отрезок прямой.
    Определите,
    \begin{itemize}
        \item какую работу при этом совершил газ,
        \item чему равно изменение внутренней энергии газа,
        \item сколько теплоты подвели к нему в этом процессе?
    \end{itemize}
    При решении обратите внимание на знаки искомых величин.
}
\answer{%
    \begin{align*}
    P_1V_1 &= \nu R T_1, P_2V_2 = \nu R T_2, \\
    \Delta U &= U_2-U_1 = \frac 32 \nu R T_2- \frac 32 \nu R T_1 = \frac 32 P_2 V_2 - \frac 32 P_1 V_1= \frac 32 \cdot \cbr{2{,}5\,\text{МПа} \cdot 6\,\text{л} - 2\,\text{МПа} \cdot 7\,\text{л}} = 1{,}50\,\text{кДж}.
    \\
    A_\text{газа} &= \frac{P_2 + P_1} 2 \cdot (V_2 - V_1) = \frac{2{,}5\,\text{МПа} + 2\,\text{МПа}} 2 \cdot (6\,\text{л} - 7\,\text{л}) = -2{,}250\,\text{кДж}, \\
    Q &= A_\text{газа} + \Delta U = \frac 32 (P_2 V_2 - P_1 V_1) + \frac{P_2 + P_1} 2 \cdot (V_2 - V_1) = 1{,}50\,\text{кДж} -2{,}250\,\text{кДж} = -0{,}7500\,\text{кДж}.
    \end{align*}
}
\solutionspace{150pt}

\tasknumber{3}%
\task{%
    Запишите формулы и рядом с каждой физичической величиной укажите её название и единицы измерения в СИ:
    \begin{enumerate}
        \item первое начало термодинамики,
        \item внутренняя энергия идеального одноатомного газа.
    \end{enumerate}
}

\variantsplitter

\addpersonalvariant{Матвей Кузьмин}

\tasknumber{1}%
\task{%
    Определите КПД цикла 12341, рабочим телом которого является идеальный одноатомный газ, если
    12 — изобарическое расширение газа в два раза,
    23 — изохорическое охлаждение газа, при котором температура уменьшается в три раза,
    34 — изобара, 41 — изохора.
    % Для этого:
    % \begin{enumerate}
    %     \item сделайте рисунок в $PV$-координатах,
    %     \item выберите удобные обозначения, чтобы не запутаться в множестве температур, давлений и объёмов,
    %     \item вычислите необходимые соотнощения между температурами, давлениями и объёмами
    %     (некоторые сразу видны по рисунку, некоторые — надо считать),
    %     \item определите для каждого участка поглощается или отдаётся тепло (и сколько именно:
    %     потребуется первое начало термодинамики, отдельный расчёт работ на участках через площади фигур и изменений внутренней энергии),
    %     \item вычислите полную работу газа в цикле,
    %     \item подставьте всё в формулу для КПД, упростите и доведите до ответа.
    % \end{enumerate}
    Определите КПД цикла Карно, температура нагревателя которого равна максимальной температуре в цикле 12341, а холодильника — минимальной.
    Ответы в обоих случаях оставьте точными в виде нескоратимой дроби, никаких округлений.
}
\answer{%
    \begin{align*}
    A_{12} &> 0, \Delta U_{12} > 0, \implies Q_{12} = A_{12} + \Delta U_{12} > 0, \\
    A_{23} &= 0, \Delta U_{23} < 0, \implies Q_{23} = A_{23} + \Delta U_{23} < 0, \\
    A_{34} &< 0, \Delta U_{34} < 0, \implies Q_{34} = A_{34} + \Delta U_{34} < 0, \\
    A_{41} &= 0, \Delta U_{41} > 0, \implies Q_{41} = A_{41} + \Delta U_{41} > 0.
    \\
    P_1V_1 &= \nu R T_1, P_2V_2 = \nu R T_2, P_3V_3 = \nu R T_3, P_4V_4 = \nu R T_4 \text{ — уравнения состояния идеального газа}, \\
    &\text{Пусть $P_0$, $V_0$, $T_0$ — давление, объём и температура в точке 4 (минимальные во всём цикле):} \\
    P_1 &= P_2, P_3 = P_4 = P_0, V_1 = V_4 = V_0, V_2 = V_3 = 2 V_1 = 2 V_0,, \text{остальные соотношения между объёмами и давлениями не даны, нужно считать} \\
    T_3 &= \frac{T_2}3 \text{(по условию)} \implies \frac{P_2}{P_3} = \frac{P_2 V_2}{P_3 V_3}= \frac{\nu R T_2}{\nu R T_3} = \frac{T_2}{T_3} = 3 \implies P_1 = P_2 = 3 P_0 \\
    A_\text{цикл} &= (3P_0 - P_0)(2V_0 - V_0) = 2P_0V_0, \\
    A_{12} &= 3P_0 \cdot (2V_0 - V_0) = 3P_0V_0, \\
    \Delta U_{12} &= \frac 32 \nu R T_2 - \frac 32 \nu R T_1 = \frac 32 P_2 V_2 - \frac 32 P_1 V_1 = \frac 32 \cdot 3 P_0 \cdot 2 V_0 -  \frac 32 \cdot 3 P_0 \cdot V_0 = \frac 32 \cdot 3 \cdot P_0V_0, \\
    \Delta U_{41} &= \frac 32 \nu R T_1 - \frac 32 \nu R T_4 = \frac 32 P_1 V_1 - \frac 32 P_4 V_4 = \frac 32 \cdot 3 P_0 V_0 - \frac 32 P_0 V_0 = \frac 32 \cdot 2 \cdot P_0V_0.
    \\
    \eta &= \frac{A_\text{цикл}}{Q_+} = \frac{A_\text{цикл}}{Q_{12} + Q_{41}}  = \frac{A_\text{цикл}}{A_{12} + \Delta U_{12} + A_{41} + \Delta U_{41}} =  \\
     &= \frac{2P_0V_0}{3P_0V_0 + \frac 32 \cdot 3 \cdot P_0V_0 + 0 + \frac 32 \cdot 2 \cdot P_0V_0} = \frac{2}{3 + \frac 32 \cdot 3 + \frac 32 \cdot 2} = \frac4{21} \approx 0{,}190.
     \\
    \eta_\text{Карно} &= 1 - \frac{T_\text{х}}{T_\text{н}} = 1 - \frac{T_\text{4}}{T_\text{2}} = 1 - \frac{\frac{P_4V_4}{\nu R}}{\frac{P_2V_2}{\nu R}} = 1 - \frac{P_4V_4}{P_2V_2} = 1 - \frac{P_0V_0}{3P_0 \cdot 2V_0} = 1 - \frac 1{3 \cdot 2}  = \frac56 \approx 0{,}833.
    \end{align*}
}
\solutionspace{360pt}

\tasknumber{2}%
\task{%
    Порция идеального одноатомного газа перешла из состояния 1 в состояние 2: $P_1 = 2\,\text{МПа}$, $V_1 = 3\,\text{л}$, $P_2 = 2{,}5\,\text{МПа}$, $V_2 = 8\,\text{л}$.
    Известно, что в $PV$-координатах график процесса 12 представляет собой отрезок прямой.
    Определите,
    \begin{itemize}
        \item какую работу при этом совершил газ,
        \item чему равно изменение внутренней энергии газа,
        \item сколько теплоты подвели к нему в этом процессе?
    \end{itemize}
    При решении обратите внимание на знаки искомых величин.
}
\answer{%
    \begin{align*}
    P_1V_1 &= \nu R T_1, P_2V_2 = \nu R T_2, \\
    \Delta U &= U_2-U_1 = \frac 32 \nu R T_2- \frac 32 \nu R T_1 = \frac 32 P_2 V_2 - \frac 32 P_1 V_1= \frac 32 \cdot \cbr{2{,}5\,\text{МПа} \cdot 8\,\text{л} - 2\,\text{МПа} \cdot 3\,\text{л}} = 21{,}00\,\text{кДж}.
    \\
    A_\text{газа} &= \frac{P_2 + P_1} 2 \cdot (V_2 - V_1) = \frac{2{,}5\,\text{МПа} + 2\,\text{МПа}} 2 \cdot (8\,\text{л} - 3\,\text{л}) = 11{,}25\,\text{кДж}, \\
    Q &= A_\text{газа} + \Delta U = \frac 32 (P_2 V_2 - P_1 V_1) + \frac{P_2 + P_1} 2 \cdot (V_2 - V_1) = 21{,}00\,\text{кДж} + 11{,}25\,\text{кДж} = 32{,}25\,\text{кДж}.
    \end{align*}
}
\solutionspace{150pt}

\tasknumber{3}%
\task{%
    Запишите формулы и рядом с каждой физичической величиной укажите её название и единицы измерения в СИ:
    \begin{enumerate}
        \item первое начало термодинамики,
        \item внутренняя энергия идеального одноатомного газа.
    \end{enumerate}
}

\variantsplitter

\addpersonalvariant{Сергей Малышев}

\tasknumber{1}%
\task{%
    Определите КПД цикла 12341, рабочим телом которого является идеальный одноатомный газ, если
    12 — изобарическое расширение газа в три раза,
    23 — изохорическое охлаждение газа, при котором температура уменьшается в два раза,
    34 — изобара, 41 — изохора.
    % Для этого:
    % \begin{enumerate}
    %     \item сделайте рисунок в $PV$-координатах,
    %     \item выберите удобные обозначения, чтобы не запутаться в множестве температур, давлений и объёмов,
    %     \item вычислите необходимые соотнощения между температурами, давлениями и объёмами
    %     (некоторые сразу видны по рисунку, некоторые — надо считать),
    %     \item определите для каждого участка поглощается или отдаётся тепло (и сколько именно:
    %     потребуется первое начало термодинамики, отдельный расчёт работ на участках через площади фигур и изменений внутренней энергии),
    %     \item вычислите полную работу газа в цикле,
    %     \item подставьте всё в формулу для КПД, упростите и доведите до ответа.
    % \end{enumerate}
    Определите КПД цикла Карно, температура нагревателя которого равна максимальной температуре в цикле 12341, а холодильника — минимальной.
    Ответы в обоих случаях оставьте точными в виде нескоратимой дроби, никаких округлений.
}
\answer{%
    \begin{align*}
    A_{12} &> 0, \Delta U_{12} > 0, \implies Q_{12} = A_{12} + \Delta U_{12} > 0, \\
    A_{23} &= 0, \Delta U_{23} < 0, \implies Q_{23} = A_{23} + \Delta U_{23} < 0, \\
    A_{34} &< 0, \Delta U_{34} < 0, \implies Q_{34} = A_{34} + \Delta U_{34} < 0, \\
    A_{41} &= 0, \Delta U_{41} > 0, \implies Q_{41} = A_{41} + \Delta U_{41} > 0.
    \\
    P_1V_1 &= \nu R T_1, P_2V_2 = \nu R T_2, P_3V_3 = \nu R T_3, P_4V_4 = \nu R T_4 \text{ — уравнения состояния идеального газа}, \\
    &\text{Пусть $P_0$, $V_0$, $T_0$ — давление, объём и температура в точке 4 (минимальные во всём цикле):} \\
    P_1 &= P_2, P_3 = P_4 = P_0, V_1 = V_4 = V_0, V_2 = V_3 = 3 V_1 = 3 V_0,, \text{остальные соотношения между объёмами и давлениями не даны, нужно считать} \\
    T_3 &= \frac{T_2}2 \text{(по условию)} \implies \frac{P_2}{P_3} = \frac{P_2 V_2}{P_3 V_3}= \frac{\nu R T_2}{\nu R T_3} = \frac{T_2}{T_3} = 2 \implies P_1 = P_2 = 2 P_0 \\
    A_\text{цикл} &= (2P_0 - P_0)(3V_0 - V_0) = 2P_0V_0, \\
    A_{12} &= 2P_0 \cdot (3V_0 - V_0) = 4P_0V_0, \\
    \Delta U_{12} &= \frac 32 \nu R T_2 - \frac 32 \nu R T_1 = \frac 32 P_2 V_2 - \frac 32 P_1 V_1 = \frac 32 \cdot 2 P_0 \cdot 3 V_0 -  \frac 32 \cdot 2 P_0 \cdot V_0 = \frac 32 \cdot 4 \cdot P_0V_0, \\
    \Delta U_{41} &= \frac 32 \nu R T_1 - \frac 32 \nu R T_4 = \frac 32 P_1 V_1 - \frac 32 P_4 V_4 = \frac 32 \cdot 2 P_0 V_0 - \frac 32 P_0 V_0 = \frac 32 \cdot 1 \cdot P_0V_0.
    \\
    \eta &= \frac{A_\text{цикл}}{Q_+} = \frac{A_\text{цикл}}{Q_{12} + Q_{41}}  = \frac{A_\text{цикл}}{A_{12} + \Delta U_{12} + A_{41} + \Delta U_{41}} =  \\
     &= \frac{2P_0V_0}{4P_0V_0 + \frac 32 \cdot 4 \cdot P_0V_0 + 0 + \frac 32 \cdot 1 \cdot P_0V_0} = \frac{2}{4 + \frac 32 \cdot 4 + \frac 32 \cdot 1} = \frac4{23} \approx 0{,}174.
     \\
    \eta_\text{Карно} &= 1 - \frac{T_\text{х}}{T_\text{н}} = 1 - \frac{T_\text{4}}{T_\text{2}} = 1 - \frac{\frac{P_4V_4}{\nu R}}{\frac{P_2V_2}{\nu R}} = 1 - \frac{P_4V_4}{P_2V_2} = 1 - \frac{P_0V_0}{2P_0 \cdot 3V_0} = 1 - \frac 1{2 \cdot 3}  = \frac56 \approx 0{,}833.
    \end{align*}
}
\solutionspace{360pt}

\tasknumber{2}%
\task{%
    Порция идеального одноатомного газа перешла из состояния 1 в состояние 2: $P_1 = 2\,\text{МПа}$, $V_1 = 5\,\text{л}$, $P_2 = 4{,}5\,\text{МПа}$, $V_2 = 4\,\text{л}$.
    Известно, что в $PV$-координатах график процесса 12 представляет собой отрезок прямой.
    Определите,
    \begin{itemize}
        \item какую работу при этом совершил газ,
        \item чему равно изменение внутренней энергии газа,
        \item сколько теплоты подвели к нему в этом процессе?
    \end{itemize}
    При решении обратите внимание на знаки искомых величин.
}
\answer{%
    \begin{align*}
    P_1V_1 &= \nu R T_1, P_2V_2 = \nu R T_2, \\
    \Delta U &= U_2-U_1 = \frac 32 \nu R T_2- \frac 32 \nu R T_1 = \frac 32 P_2 V_2 - \frac 32 P_1 V_1= \frac 32 \cdot \cbr{4{,}5\,\text{МПа} \cdot 4\,\text{л} - 2\,\text{МПа} \cdot 5\,\text{л}} = 12{,}00\,\text{кДж}.
    \\
    A_\text{газа} &= \frac{P_2 + P_1} 2 \cdot (V_2 - V_1) = \frac{4{,}5\,\text{МПа} + 2\,\text{МПа}} 2 \cdot (4\,\text{л} - 5\,\text{л}) = -3{,}250\,\text{кДж}, \\
    Q &= A_\text{газа} + \Delta U = \frac 32 (P_2 V_2 - P_1 V_1) + \frac{P_2 + P_1} 2 \cdot (V_2 - V_1) = 12{,}00\,\text{кДж} -3{,}250\,\text{кДж} = 8{,}75\,\text{кДж}.
    \end{align*}
}
\solutionspace{150pt}

\tasknumber{3}%
\task{%
    Запишите формулы и рядом с каждой физичической величиной укажите её название и единицы измерения в СИ:
    \begin{enumerate}
        \item первое начало термодинамики,
        \item внутренняя энергия идеального одноатомного газа.
    \end{enumerate}
}

\variantsplitter

\addpersonalvariant{Алина Полканова}

\tasknumber{1}%
\task{%
    Определите КПД цикла 12341, рабочим телом которого является идеальный одноатомный газ, если
    12 — изобарическое расширение газа в шесть раз,
    23 — изохорическое охлаждение газа, при котором температура уменьшается в четыре раза,
    34 — изобара, 41 — изохора.
    % Для этого:
    % \begin{enumerate}
    %     \item сделайте рисунок в $PV$-координатах,
    %     \item выберите удобные обозначения, чтобы не запутаться в множестве температур, давлений и объёмов,
    %     \item вычислите необходимые соотнощения между температурами, давлениями и объёмами
    %     (некоторые сразу видны по рисунку, некоторые — надо считать),
    %     \item определите для каждого участка поглощается или отдаётся тепло (и сколько именно:
    %     потребуется первое начало термодинамики, отдельный расчёт работ на участках через площади фигур и изменений внутренней энергии),
    %     \item вычислите полную работу газа в цикле,
    %     \item подставьте всё в формулу для КПД, упростите и доведите до ответа.
    % \end{enumerate}
    Определите КПД цикла Карно, температура нагревателя которого равна максимальной температуре в цикле 12341, а холодильника — минимальной.
    Ответы в обоих случаях оставьте точными в виде нескоратимой дроби, никаких округлений.
}
\answer{%
    \begin{align*}
    A_{12} &> 0, \Delta U_{12} > 0, \implies Q_{12} = A_{12} + \Delta U_{12} > 0, \\
    A_{23} &= 0, \Delta U_{23} < 0, \implies Q_{23} = A_{23} + \Delta U_{23} < 0, \\
    A_{34} &< 0, \Delta U_{34} < 0, \implies Q_{34} = A_{34} + \Delta U_{34} < 0, \\
    A_{41} &= 0, \Delta U_{41} > 0, \implies Q_{41} = A_{41} + \Delta U_{41} > 0.
    \\
    P_1V_1 &= \nu R T_1, P_2V_2 = \nu R T_2, P_3V_3 = \nu R T_3, P_4V_4 = \nu R T_4 \text{ — уравнения состояния идеального газа}, \\
    &\text{Пусть $P_0$, $V_0$, $T_0$ — давление, объём и температура в точке 4 (минимальные во всём цикле):} \\
    P_1 &= P_2, P_3 = P_4 = P_0, V_1 = V_4 = V_0, V_2 = V_3 = 6 V_1 = 6 V_0,, \text{остальные соотношения между объёмами и давлениями не даны, нужно считать} \\
    T_3 &= \frac{T_2}4 \text{(по условию)} \implies \frac{P_2}{P_3} = \frac{P_2 V_2}{P_3 V_3}= \frac{\nu R T_2}{\nu R T_3} = \frac{T_2}{T_3} = 4 \implies P_1 = P_2 = 4 P_0 \\
    A_\text{цикл} &= (4P_0 - P_0)(6V_0 - V_0) = 15P_0V_0, \\
    A_{12} &= 4P_0 \cdot (6V_0 - V_0) = 20P_0V_0, \\
    \Delta U_{12} &= \frac 32 \nu R T_2 - \frac 32 \nu R T_1 = \frac 32 P_2 V_2 - \frac 32 P_1 V_1 = \frac 32 \cdot 4 P_0 \cdot 6 V_0 -  \frac 32 \cdot 4 P_0 \cdot V_0 = \frac 32 \cdot 20 \cdot P_0V_0, \\
    \Delta U_{41} &= \frac 32 \nu R T_1 - \frac 32 \nu R T_4 = \frac 32 P_1 V_1 - \frac 32 P_4 V_4 = \frac 32 \cdot 4 P_0 V_0 - \frac 32 P_0 V_0 = \frac 32 \cdot 3 \cdot P_0V_0.
    \\
    \eta &= \frac{A_\text{цикл}}{Q_+} = \frac{A_\text{цикл}}{Q_{12} + Q_{41}}  = \frac{A_\text{цикл}}{A_{12} + \Delta U_{12} + A_{41} + \Delta U_{41}} =  \\
     &= \frac{15P_0V_0}{20P_0V_0 + \frac 32 \cdot 20 \cdot P_0V_0 + 0 + \frac 32 \cdot 3 \cdot P_0V_0} = \frac{15}{20 + \frac 32 \cdot 20 + \frac 32 \cdot 3} = \frac{30}{109} \approx 0{,}275.
     \\
    \eta_\text{Карно} &= 1 - \frac{T_\text{х}}{T_\text{н}} = 1 - \frac{T_\text{4}}{T_\text{2}} = 1 - \frac{\frac{P_4V_4}{\nu R}}{\frac{P_2V_2}{\nu R}} = 1 - \frac{P_4V_4}{P_2V_2} = 1 - \frac{P_0V_0}{4P_0 \cdot 6V_0} = 1 - \frac 1{4 \cdot 6}  = \frac{23}{24} \approx 0{,}958.
    \end{align*}
}
\solutionspace{360pt}

\tasknumber{2}%
\task{%
    Порция идеального одноатомного газа перешла из состояния 1 в состояние 2: $P_1 = 3\,\text{МПа}$, $V_1 = 7\,\text{л}$, $P_2 = 2{,}5\,\text{МПа}$, $V_2 = 6\,\text{л}$.
    Известно, что в $PV$-координатах график процесса 12 представляет собой отрезок прямой.
    Определите,
    \begin{itemize}
        \item какую работу при этом совершил газ,
        \item чему равно изменение внутренней энергии газа,
        \item сколько теплоты подвели к нему в этом процессе?
    \end{itemize}
    При решении обратите внимание на знаки искомых величин.
}
\answer{%
    \begin{align*}
    P_1V_1 &= \nu R T_1, P_2V_2 = \nu R T_2, \\
    \Delta U &= U_2-U_1 = \frac 32 \nu R T_2- \frac 32 \nu R T_1 = \frac 32 P_2 V_2 - \frac 32 P_1 V_1= \frac 32 \cdot \cbr{2{,}5\,\text{МПа} \cdot 6\,\text{л} - 3\,\text{МПа} \cdot 7\,\text{л}} = -9{,}000\,\text{кДж}.
    \\
    A_\text{газа} &= \frac{P_2 + P_1} 2 \cdot (V_2 - V_1) = \frac{2{,}5\,\text{МПа} + 3\,\text{МПа}} 2 \cdot (6\,\text{л} - 7\,\text{л}) = -2{,}750\,\text{кДж}, \\
    Q &= A_\text{газа} + \Delta U = \frac 32 (P_2 V_2 - P_1 V_1) + \frac{P_2 + P_1} 2 \cdot (V_2 - V_1) = -9{,}000\,\text{кДж} -2{,}750\,\text{кДж} = -11{,}7500\,\text{кДж}.
    \end{align*}
}
\solutionspace{150pt}

\tasknumber{3}%
\task{%
    Запишите формулы и рядом с каждой физичической величиной укажите её название и единицы измерения в СИ:
    \begin{enumerate}
        \item первое начало термодинамики,
        \item внутренняя энергия идеального одноатомного газа.
    \end{enumerate}
}

\variantsplitter

\addpersonalvariant{Сергей Пономарёв}

\tasknumber{1}%
\task{%
    Определите КПД цикла 12341, рабочим телом которого является идеальный одноатомный газ, если
    12 — изобарическое расширение газа в четыре раза,
    23 — изохорическое охлаждение газа, при котором температура уменьшается в три раза,
    34 — изобара, 41 — изохора.
    % Для этого:
    % \begin{enumerate}
    %     \item сделайте рисунок в $PV$-координатах,
    %     \item выберите удобные обозначения, чтобы не запутаться в множестве температур, давлений и объёмов,
    %     \item вычислите необходимые соотнощения между температурами, давлениями и объёмами
    %     (некоторые сразу видны по рисунку, некоторые — надо считать),
    %     \item определите для каждого участка поглощается или отдаётся тепло (и сколько именно:
    %     потребуется первое начало термодинамики, отдельный расчёт работ на участках через площади фигур и изменений внутренней энергии),
    %     \item вычислите полную работу газа в цикле,
    %     \item подставьте всё в формулу для КПД, упростите и доведите до ответа.
    % \end{enumerate}
    Определите КПД цикла Карно, температура нагревателя которого равна максимальной температуре в цикле 12341, а холодильника — минимальной.
    Ответы в обоих случаях оставьте точными в виде нескоратимой дроби, никаких округлений.
}
\answer{%
    \begin{align*}
    A_{12} &> 0, \Delta U_{12} > 0, \implies Q_{12} = A_{12} + \Delta U_{12} > 0, \\
    A_{23} &= 0, \Delta U_{23} < 0, \implies Q_{23} = A_{23} + \Delta U_{23} < 0, \\
    A_{34} &< 0, \Delta U_{34} < 0, \implies Q_{34} = A_{34} + \Delta U_{34} < 0, \\
    A_{41} &= 0, \Delta U_{41} > 0, \implies Q_{41} = A_{41} + \Delta U_{41} > 0.
    \\
    P_1V_1 &= \nu R T_1, P_2V_2 = \nu R T_2, P_3V_3 = \nu R T_3, P_4V_4 = \nu R T_4 \text{ — уравнения состояния идеального газа}, \\
    &\text{Пусть $P_0$, $V_0$, $T_0$ — давление, объём и температура в точке 4 (минимальные во всём цикле):} \\
    P_1 &= P_2, P_3 = P_4 = P_0, V_1 = V_4 = V_0, V_2 = V_3 = 4 V_1 = 4 V_0,, \text{остальные соотношения между объёмами и давлениями не даны, нужно считать} \\
    T_3 &= \frac{T_2}3 \text{(по условию)} \implies \frac{P_2}{P_3} = \frac{P_2 V_2}{P_3 V_3}= \frac{\nu R T_2}{\nu R T_3} = \frac{T_2}{T_3} = 3 \implies P_1 = P_2 = 3 P_0 \\
    A_\text{цикл} &= (3P_0 - P_0)(4V_0 - V_0) = 6P_0V_0, \\
    A_{12} &= 3P_0 \cdot (4V_0 - V_0) = 9P_0V_0, \\
    \Delta U_{12} &= \frac 32 \nu R T_2 - \frac 32 \nu R T_1 = \frac 32 P_2 V_2 - \frac 32 P_1 V_1 = \frac 32 \cdot 3 P_0 \cdot 4 V_0 -  \frac 32 \cdot 3 P_0 \cdot V_0 = \frac 32 \cdot 9 \cdot P_0V_0, \\
    \Delta U_{41} &= \frac 32 \nu R T_1 - \frac 32 \nu R T_4 = \frac 32 P_1 V_1 - \frac 32 P_4 V_4 = \frac 32 \cdot 3 P_0 V_0 - \frac 32 P_0 V_0 = \frac 32 \cdot 2 \cdot P_0V_0.
    \\
    \eta &= \frac{A_\text{цикл}}{Q_+} = \frac{A_\text{цикл}}{Q_{12} + Q_{41}}  = \frac{A_\text{цикл}}{A_{12} + \Delta U_{12} + A_{41} + \Delta U_{41}} =  \\
     &= \frac{6P_0V_0}{9P_0V_0 + \frac 32 \cdot 9 \cdot P_0V_0 + 0 + \frac 32 \cdot 2 \cdot P_0V_0} = \frac{6}{9 + \frac 32 \cdot 9 + \frac 32 \cdot 2} = \frac4{17} \approx 0{,}235.
     \\
    \eta_\text{Карно} &= 1 - \frac{T_\text{х}}{T_\text{н}} = 1 - \frac{T_\text{4}}{T_\text{2}} = 1 - \frac{\frac{P_4V_4}{\nu R}}{\frac{P_2V_2}{\nu R}} = 1 - \frac{P_4V_4}{P_2V_2} = 1 - \frac{P_0V_0}{3P_0 \cdot 4V_0} = 1 - \frac 1{3 \cdot 4}  = \frac{11}{12} \approx 0{,}917.
    \end{align*}
}
\solutionspace{360pt}

\tasknumber{2}%
\task{%
    Порция идеального одноатомного газа перешла из состояния 1 в состояние 2: $P_1 = 3\,\text{МПа}$, $V_1 = 7\,\text{л}$, $P_2 = 2{,}5\,\text{МПа}$, $V_2 = 8\,\text{л}$.
    Известно, что в $PV$-координатах график процесса 12 представляет собой отрезок прямой.
    Определите,
    \begin{itemize}
        \item какую работу при этом совершил газ,
        \item чему равно изменение внутренней энергии газа,
        \item сколько теплоты подвели к нему в этом процессе?
    \end{itemize}
    При решении обратите внимание на знаки искомых величин.
}
\answer{%
    \begin{align*}
    P_1V_1 &= \nu R T_1, P_2V_2 = \nu R T_2, \\
    \Delta U &= U_2-U_1 = \frac 32 \nu R T_2- \frac 32 \nu R T_1 = \frac 32 P_2 V_2 - \frac 32 P_1 V_1= \frac 32 \cdot \cbr{2{,}5\,\text{МПа} \cdot 8\,\text{л} - 3\,\text{МПа} \cdot 7\,\text{л}} = -1{,}5000\,\text{кДж}.
    \\
    A_\text{газа} &= \frac{P_2 + P_1} 2 \cdot (V_2 - V_1) = \frac{2{,}5\,\text{МПа} + 3\,\text{МПа}} 2 \cdot (8\,\text{л} - 7\,\text{л}) = 2{,}75\,\text{кДж}, \\
    Q &= A_\text{газа} + \Delta U = \frac 32 (P_2 V_2 - P_1 V_1) + \frac{P_2 + P_1} 2 \cdot (V_2 - V_1) = -1{,}5000\,\text{кДж} + 2{,}75\,\text{кДж} = 1{,}25\,\text{кДж}.
    \end{align*}
}
\solutionspace{150pt}

\tasknumber{3}%
\task{%
    Запишите формулы и рядом с каждой физичической величиной укажите её название и единицы измерения в СИ:
    \begin{enumerate}
        \item первое начало термодинамики,
        \item внутренняя энергия идеального одноатомного газа.
    \end{enumerate}
}

\variantsplitter

\addpersonalvariant{Егор Свистушкин}

\tasknumber{1}%
\task{%
    Определите КПД цикла 12341, рабочим телом которого является идеальный одноатомный газ, если
    12 — изобарическое расширение газа в два раза,
    23 — изохорическое охлаждение газа, при котором температура уменьшается в пять раз,
    34 — изобара, 41 — изохора.
    % Для этого:
    % \begin{enumerate}
    %     \item сделайте рисунок в $PV$-координатах,
    %     \item выберите удобные обозначения, чтобы не запутаться в множестве температур, давлений и объёмов,
    %     \item вычислите необходимые соотнощения между температурами, давлениями и объёмами
    %     (некоторые сразу видны по рисунку, некоторые — надо считать),
    %     \item определите для каждого участка поглощается или отдаётся тепло (и сколько именно:
    %     потребуется первое начало термодинамики, отдельный расчёт работ на участках через площади фигур и изменений внутренней энергии),
    %     \item вычислите полную работу газа в цикле,
    %     \item подставьте всё в формулу для КПД, упростите и доведите до ответа.
    % \end{enumerate}
    Определите КПД цикла Карно, температура нагревателя которого равна максимальной температуре в цикле 12341, а холодильника — минимальной.
    Ответы в обоих случаях оставьте точными в виде нескоратимой дроби, никаких округлений.
}
\answer{%
    \begin{align*}
    A_{12} &> 0, \Delta U_{12} > 0, \implies Q_{12} = A_{12} + \Delta U_{12} > 0, \\
    A_{23} &= 0, \Delta U_{23} < 0, \implies Q_{23} = A_{23} + \Delta U_{23} < 0, \\
    A_{34} &< 0, \Delta U_{34} < 0, \implies Q_{34} = A_{34} + \Delta U_{34} < 0, \\
    A_{41} &= 0, \Delta U_{41} > 0, \implies Q_{41} = A_{41} + \Delta U_{41} > 0.
    \\
    P_1V_1 &= \nu R T_1, P_2V_2 = \nu R T_2, P_3V_3 = \nu R T_3, P_4V_4 = \nu R T_4 \text{ — уравнения состояния идеального газа}, \\
    &\text{Пусть $P_0$, $V_0$, $T_0$ — давление, объём и температура в точке 4 (минимальные во всём цикле):} \\
    P_1 &= P_2, P_3 = P_4 = P_0, V_1 = V_4 = V_0, V_2 = V_3 = 2 V_1 = 2 V_0,, \text{остальные соотношения между объёмами и давлениями не даны, нужно считать} \\
    T_3 &= \frac{T_2}5 \text{(по условию)} \implies \frac{P_2}{P_3} = \frac{P_2 V_2}{P_3 V_3}= \frac{\nu R T_2}{\nu R T_3} = \frac{T_2}{T_3} = 5 \implies P_1 = P_2 = 5 P_0 \\
    A_\text{цикл} &= (5P_0 - P_0)(2V_0 - V_0) = 4P_0V_0, \\
    A_{12} &= 5P_0 \cdot (2V_0 - V_0) = 5P_0V_0, \\
    \Delta U_{12} &= \frac 32 \nu R T_2 - \frac 32 \nu R T_1 = \frac 32 P_2 V_2 - \frac 32 P_1 V_1 = \frac 32 \cdot 5 P_0 \cdot 2 V_0 -  \frac 32 \cdot 5 P_0 \cdot V_0 = \frac 32 \cdot 5 \cdot P_0V_0, \\
    \Delta U_{41} &= \frac 32 \nu R T_1 - \frac 32 \nu R T_4 = \frac 32 P_1 V_1 - \frac 32 P_4 V_4 = \frac 32 \cdot 5 P_0 V_0 - \frac 32 P_0 V_0 = \frac 32 \cdot 4 \cdot P_0V_0.
    \\
    \eta &= \frac{A_\text{цикл}}{Q_+} = \frac{A_\text{цикл}}{Q_{12} + Q_{41}}  = \frac{A_\text{цикл}}{A_{12} + \Delta U_{12} + A_{41} + \Delta U_{41}} =  \\
     &= \frac{4P_0V_0}{5P_0V_0 + \frac 32 \cdot 5 \cdot P_0V_0 + 0 + \frac 32 \cdot 4 \cdot P_0V_0} = \frac{4}{5 + \frac 32 \cdot 5 + \frac 32 \cdot 4} = \frac8{37} \approx 0{,}216.
     \\
    \eta_\text{Карно} &= 1 - \frac{T_\text{х}}{T_\text{н}} = 1 - \frac{T_\text{4}}{T_\text{2}} = 1 - \frac{\frac{P_4V_4}{\nu R}}{\frac{P_2V_2}{\nu R}} = 1 - \frac{P_4V_4}{P_2V_2} = 1 - \frac{P_0V_0}{5P_0 \cdot 2V_0} = 1 - \frac 1{5 \cdot 2}  = \frac9{10} \approx 0{,}900.
    \end{align*}
}
\solutionspace{360pt}

\tasknumber{2}%
\task{%
    Порция идеального одноатомного газа перешла из состояния 1 в состояние 2: $P_1 = 3\,\text{МПа}$, $V_1 = 5\,\text{л}$, $P_2 = 2{,}5\,\text{МПа}$, $V_2 = 8\,\text{л}$.
    Известно, что в $PV$-координатах график процесса 12 представляет собой отрезок прямой.
    Определите,
    \begin{itemize}
        \item какую работу при этом совершил газ,
        \item чему равно изменение внутренней энергии газа,
        \item сколько теплоты подвели к нему в этом процессе?
    \end{itemize}
    При решении обратите внимание на знаки искомых величин.
}
\answer{%
    \begin{align*}
    P_1V_1 &= \nu R T_1, P_2V_2 = \nu R T_2, \\
    \Delta U &= U_2-U_1 = \frac 32 \nu R T_2- \frac 32 \nu R T_1 = \frac 32 P_2 V_2 - \frac 32 P_1 V_1= \frac 32 \cdot \cbr{2{,}5\,\text{МПа} \cdot 8\,\text{л} - 3\,\text{МПа} \cdot 5\,\text{л}} = 7{,}50\,\text{кДж}.
    \\
    A_\text{газа} &= \frac{P_2 + P_1} 2 \cdot (V_2 - V_1) = \frac{2{,}5\,\text{МПа} + 3\,\text{МПа}} 2 \cdot (8\,\text{л} - 5\,\text{л}) = 8{,}25\,\text{кДж}, \\
    Q &= A_\text{газа} + \Delta U = \frac 32 (P_2 V_2 - P_1 V_1) + \frac{P_2 + P_1} 2 \cdot (V_2 - V_1) = 7{,}50\,\text{кДж} + 8{,}25\,\text{кДж} = 15{,}75\,\text{кДж}.
    \end{align*}
}
\solutionspace{150pt}

\tasknumber{3}%
\task{%
    Запишите формулы и рядом с каждой физичической величиной укажите её название и единицы измерения в СИ:
    \begin{enumerate}
        \item первое начало термодинамики,
        \item внутренняя энергия идеального одноатомного газа.
    \end{enumerate}
}

\variantsplitter

\addpersonalvariant{Дмитрий Соколов}

\tasknumber{1}%
\task{%
    Определите КПД цикла 12341, рабочим телом которого является идеальный одноатомный газ, если
    12 — изобарическое расширение газа в шесть раз,
    23 — изохорическое охлаждение газа, при котором температура уменьшается в два раза,
    34 — изобара, 41 — изохора.
    % Для этого:
    % \begin{enumerate}
    %     \item сделайте рисунок в $PV$-координатах,
    %     \item выберите удобные обозначения, чтобы не запутаться в множестве температур, давлений и объёмов,
    %     \item вычислите необходимые соотнощения между температурами, давлениями и объёмами
    %     (некоторые сразу видны по рисунку, некоторые — надо считать),
    %     \item определите для каждого участка поглощается или отдаётся тепло (и сколько именно:
    %     потребуется первое начало термодинамики, отдельный расчёт работ на участках через площади фигур и изменений внутренней энергии),
    %     \item вычислите полную работу газа в цикле,
    %     \item подставьте всё в формулу для КПД, упростите и доведите до ответа.
    % \end{enumerate}
    Определите КПД цикла Карно, температура нагревателя которого равна максимальной температуре в цикле 12341, а холодильника — минимальной.
    Ответы в обоих случаях оставьте точными в виде нескоратимой дроби, никаких округлений.
}
\answer{%
    \begin{align*}
    A_{12} &> 0, \Delta U_{12} > 0, \implies Q_{12} = A_{12} + \Delta U_{12} > 0, \\
    A_{23} &= 0, \Delta U_{23} < 0, \implies Q_{23} = A_{23} + \Delta U_{23} < 0, \\
    A_{34} &< 0, \Delta U_{34} < 0, \implies Q_{34} = A_{34} + \Delta U_{34} < 0, \\
    A_{41} &= 0, \Delta U_{41} > 0, \implies Q_{41} = A_{41} + \Delta U_{41} > 0.
    \\
    P_1V_1 &= \nu R T_1, P_2V_2 = \nu R T_2, P_3V_3 = \nu R T_3, P_4V_4 = \nu R T_4 \text{ — уравнения состояния идеального газа}, \\
    &\text{Пусть $P_0$, $V_0$, $T_0$ — давление, объём и температура в точке 4 (минимальные во всём цикле):} \\
    P_1 &= P_2, P_3 = P_4 = P_0, V_1 = V_4 = V_0, V_2 = V_3 = 6 V_1 = 6 V_0,, \text{остальные соотношения между объёмами и давлениями не даны, нужно считать} \\
    T_3 &= \frac{T_2}2 \text{(по условию)} \implies \frac{P_2}{P_3} = \frac{P_2 V_2}{P_3 V_3}= \frac{\nu R T_2}{\nu R T_3} = \frac{T_2}{T_3} = 2 \implies P_1 = P_2 = 2 P_0 \\
    A_\text{цикл} &= (2P_0 - P_0)(6V_0 - V_0) = 5P_0V_0, \\
    A_{12} &= 2P_0 \cdot (6V_0 - V_0) = 10P_0V_0, \\
    \Delta U_{12} &= \frac 32 \nu R T_2 - \frac 32 \nu R T_1 = \frac 32 P_2 V_2 - \frac 32 P_1 V_1 = \frac 32 \cdot 2 P_0 \cdot 6 V_0 -  \frac 32 \cdot 2 P_0 \cdot V_0 = \frac 32 \cdot 10 \cdot P_0V_0, \\
    \Delta U_{41} &= \frac 32 \nu R T_1 - \frac 32 \nu R T_4 = \frac 32 P_1 V_1 - \frac 32 P_4 V_4 = \frac 32 \cdot 2 P_0 V_0 - \frac 32 P_0 V_0 = \frac 32 \cdot 1 \cdot P_0V_0.
    \\
    \eta &= \frac{A_\text{цикл}}{Q_+} = \frac{A_\text{цикл}}{Q_{12} + Q_{41}}  = \frac{A_\text{цикл}}{A_{12} + \Delta U_{12} + A_{41} + \Delta U_{41}} =  \\
     &= \frac{5P_0V_0}{10P_0V_0 + \frac 32 \cdot 10 \cdot P_0V_0 + 0 + \frac 32 \cdot 1 \cdot P_0V_0} = \frac{5}{10 + \frac 32 \cdot 10 + \frac 32 \cdot 1} = \frac{10}{53} \approx 0{,}189.
     \\
    \eta_\text{Карно} &= 1 - \frac{T_\text{х}}{T_\text{н}} = 1 - \frac{T_\text{4}}{T_\text{2}} = 1 - \frac{\frac{P_4V_4}{\nu R}}{\frac{P_2V_2}{\nu R}} = 1 - \frac{P_4V_4}{P_2V_2} = 1 - \frac{P_0V_0}{2P_0 \cdot 6V_0} = 1 - \frac 1{2 \cdot 6}  = \frac{11}{12} \approx 0{,}917.
    \end{align*}
}
\solutionspace{360pt}

\tasknumber{2}%
\task{%
    Порция идеального одноатомного газа перешла из состояния 1 в состояние 2: $P_1 = 3\,\text{МПа}$, $V_1 = 3\,\text{л}$, $P_2 = 3{,}5\,\text{МПа}$, $V_2 = 4\,\text{л}$.
    Известно, что в $PV$-координатах график процесса 12 представляет собой отрезок прямой.
    Определите,
    \begin{itemize}
        \item какую работу при этом совершил газ,
        \item чему равно изменение внутренней энергии газа,
        \item сколько теплоты подвели к нему в этом процессе?
    \end{itemize}
    При решении обратите внимание на знаки искомых величин.
}
\answer{%
    \begin{align*}
    P_1V_1 &= \nu R T_1, P_2V_2 = \nu R T_2, \\
    \Delta U &= U_2-U_1 = \frac 32 \nu R T_2- \frac 32 \nu R T_1 = \frac 32 P_2 V_2 - \frac 32 P_1 V_1= \frac 32 \cdot \cbr{3{,}5\,\text{МПа} \cdot 4\,\text{л} - 3\,\text{МПа} \cdot 3\,\text{л}} = 7{,}50\,\text{кДж}.
    \\
    A_\text{газа} &= \frac{P_2 + P_1} 2 \cdot (V_2 - V_1) = \frac{3{,}5\,\text{МПа} + 3\,\text{МПа}} 2 \cdot (4\,\text{л} - 3\,\text{л}) = 3{,}25\,\text{кДж}, \\
    Q &= A_\text{газа} + \Delta U = \frac 32 (P_2 V_2 - P_1 V_1) + \frac{P_2 + P_1} 2 \cdot (V_2 - V_1) = 7{,}50\,\text{кДж} + 3{,}25\,\text{кДж} = 10{,}75\,\text{кДж}.
    \end{align*}
}
\solutionspace{150pt}

\tasknumber{3}%
\task{%
    Запишите формулы и рядом с каждой физичической величиной укажите её название и единицы измерения в СИ:
    \begin{enumerate}
        \item первое начало термодинамики,
        \item внутренняя энергия идеального одноатомного газа.
    \end{enumerate}
}

\variantsplitter

\addpersonalvariant{Арсений Трофимов}

\tasknumber{1}%
\task{%
    Определите КПД цикла 12341, рабочим телом которого является идеальный одноатомный газ, если
    12 — изобарическое расширение газа в два раза,
    23 — изохорическое охлаждение газа, при котором температура уменьшается в четыре раза,
    34 — изобара, 41 — изохора.
    % Для этого:
    % \begin{enumerate}
    %     \item сделайте рисунок в $PV$-координатах,
    %     \item выберите удобные обозначения, чтобы не запутаться в множестве температур, давлений и объёмов,
    %     \item вычислите необходимые соотнощения между температурами, давлениями и объёмами
    %     (некоторые сразу видны по рисунку, некоторые — надо считать),
    %     \item определите для каждого участка поглощается или отдаётся тепло (и сколько именно:
    %     потребуется первое начало термодинамики, отдельный расчёт работ на участках через площади фигур и изменений внутренней энергии),
    %     \item вычислите полную работу газа в цикле,
    %     \item подставьте всё в формулу для КПД, упростите и доведите до ответа.
    % \end{enumerate}
    Определите КПД цикла Карно, температура нагревателя которого равна максимальной температуре в цикле 12341, а холодильника — минимальной.
    Ответы в обоих случаях оставьте точными в виде нескоратимой дроби, никаких округлений.
}
\answer{%
    \begin{align*}
    A_{12} &> 0, \Delta U_{12} > 0, \implies Q_{12} = A_{12} + \Delta U_{12} > 0, \\
    A_{23} &= 0, \Delta U_{23} < 0, \implies Q_{23} = A_{23} + \Delta U_{23} < 0, \\
    A_{34} &< 0, \Delta U_{34} < 0, \implies Q_{34} = A_{34} + \Delta U_{34} < 0, \\
    A_{41} &= 0, \Delta U_{41} > 0, \implies Q_{41} = A_{41} + \Delta U_{41} > 0.
    \\
    P_1V_1 &= \nu R T_1, P_2V_2 = \nu R T_2, P_3V_3 = \nu R T_3, P_4V_4 = \nu R T_4 \text{ — уравнения состояния идеального газа}, \\
    &\text{Пусть $P_0$, $V_0$, $T_0$ — давление, объём и температура в точке 4 (минимальные во всём цикле):} \\
    P_1 &= P_2, P_3 = P_4 = P_0, V_1 = V_4 = V_0, V_2 = V_3 = 2 V_1 = 2 V_0,, \text{остальные соотношения между объёмами и давлениями не даны, нужно считать} \\
    T_3 &= \frac{T_2}4 \text{(по условию)} \implies \frac{P_2}{P_3} = \frac{P_2 V_2}{P_3 V_3}= \frac{\nu R T_2}{\nu R T_3} = \frac{T_2}{T_3} = 4 \implies P_1 = P_2 = 4 P_0 \\
    A_\text{цикл} &= (4P_0 - P_0)(2V_0 - V_0) = 3P_0V_0, \\
    A_{12} &= 4P_0 \cdot (2V_0 - V_0) = 4P_0V_0, \\
    \Delta U_{12} &= \frac 32 \nu R T_2 - \frac 32 \nu R T_1 = \frac 32 P_2 V_2 - \frac 32 P_1 V_1 = \frac 32 \cdot 4 P_0 \cdot 2 V_0 -  \frac 32 \cdot 4 P_0 \cdot V_0 = \frac 32 \cdot 4 \cdot P_0V_0, \\
    \Delta U_{41} &= \frac 32 \nu R T_1 - \frac 32 \nu R T_4 = \frac 32 P_1 V_1 - \frac 32 P_4 V_4 = \frac 32 \cdot 4 P_0 V_0 - \frac 32 P_0 V_0 = \frac 32 \cdot 3 \cdot P_0V_0.
    \\
    \eta &= \frac{A_\text{цикл}}{Q_+} = \frac{A_\text{цикл}}{Q_{12} + Q_{41}}  = \frac{A_\text{цикл}}{A_{12} + \Delta U_{12} + A_{41} + \Delta U_{41}} =  \\
     &= \frac{3P_0V_0}{4P_0V_0 + \frac 32 \cdot 4 \cdot P_0V_0 + 0 + \frac 32 \cdot 3 \cdot P_0V_0} = \frac{3}{4 + \frac 32 \cdot 4 + \frac 32 \cdot 3} = \frac6{29} \approx 0{,}207.
     \\
    \eta_\text{Карно} &= 1 - \frac{T_\text{х}}{T_\text{н}} = 1 - \frac{T_\text{4}}{T_\text{2}} = 1 - \frac{\frac{P_4V_4}{\nu R}}{\frac{P_2V_2}{\nu R}} = 1 - \frac{P_4V_4}{P_2V_2} = 1 - \frac{P_0V_0}{4P_0 \cdot 2V_0} = 1 - \frac 1{4 \cdot 2}  = \frac78 \approx 0{,}875.
    \end{align*}
}
\solutionspace{360pt}

\tasknumber{2}%
\task{%
    Порция идеального одноатомного газа перешла из состояния 1 в состояние 2: $P_1 = 3\,\text{МПа}$, $V_1 = 7\,\text{л}$, $P_2 = 2{,}5\,\text{МПа}$, $V_2 = 6\,\text{л}$.
    Известно, что в $PV$-координатах график процесса 12 представляет собой отрезок прямой.
    Определите,
    \begin{itemize}
        \item какую работу при этом совершил газ,
        \item чему равно изменение внутренней энергии газа,
        \item сколько теплоты подвели к нему в этом процессе?
    \end{itemize}
    При решении обратите внимание на знаки искомых величин.
}
\answer{%
    \begin{align*}
    P_1V_1 &= \nu R T_1, P_2V_2 = \nu R T_2, \\
    \Delta U &= U_2-U_1 = \frac 32 \nu R T_2- \frac 32 \nu R T_1 = \frac 32 P_2 V_2 - \frac 32 P_1 V_1= \frac 32 \cdot \cbr{2{,}5\,\text{МПа} \cdot 6\,\text{л} - 3\,\text{МПа} \cdot 7\,\text{л}} = -9{,}000\,\text{кДж}.
    \\
    A_\text{газа} &= \frac{P_2 + P_1} 2 \cdot (V_2 - V_1) = \frac{2{,}5\,\text{МПа} + 3\,\text{МПа}} 2 \cdot (6\,\text{л} - 7\,\text{л}) = -2{,}750\,\text{кДж}, \\
    Q &= A_\text{газа} + \Delta U = \frac 32 (P_2 V_2 - P_1 V_1) + \frac{P_2 + P_1} 2 \cdot (V_2 - V_1) = -9{,}000\,\text{кДж} -2{,}750\,\text{кДж} = -11{,}7500\,\text{кДж}.
    \end{align*}
}
\solutionspace{150pt}

\tasknumber{3}%
\task{%
    Запишите формулы и рядом с каждой физичической величиной укажите её название и единицы измерения в СИ:
    \begin{enumerate}
        \item первое начало термодинамики,
        \item внутренняя энергия идеального одноатомного газа.
    \end{enumerate}
}
% autogenerated
