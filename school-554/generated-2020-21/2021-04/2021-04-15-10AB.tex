\setdate{15~апреля~2021}
\setclass{10«АБ»}

\addpersonalvariant{Михаил Бурмистров}

\tasknumber{1}%
\task{%
    Установите соответствие и запишите в ответ набор цифр (без других символов).

    А) электроёмкость, Б) электрический заряд, В) напряжённость электрического поля.

    1) джоуль, 2) Н / Кл, 3) фарад, 4) ватт, 5) кулон.
}
\answer{%
    $352$
}
\solutionspace{40pt}

\tasknumber{2}%
\task{%
    Установите соответствие и запишите в ответ набор цифр (без других символов).

    А) электроёмкость, Б) энергия заряженного конденсатора, В) разность потенциалов.

    1) $r$, 2) $U$, 3) $C$, 4) $Q$, 5) $W$.
}
\answer{%
    $352$
}
\solutionspace{40pt}

\tasknumber{3}%
\task{%
    На конденсаторе указано: $C = 80\,\text{пФ}$, $V = 300\,\text{В}$.
    Удастся ли его использовать для накопления заряда $q = 60\,\text{нКл}$?
    (в ответе укажите «да» или «нет»)
}
\answer{%
    $
        q_{\text{max}} = CV = 80\,\text{пФ} \cdot 300\,\text{В} = 24\,\text{нКл}
        \implies q_{\text{max}}  <  q \implies \text{не удастся}
    $
}
\solutionspace{80pt}

\tasknumber{4}%
\task{%
    Конденсатор ёмкостью $40\,\text{нФ}$ был заряжен до напряжения $55\,\text{кВ}$.
    Затем напряжение уменьшают на $20\,\text{кВ}$.
    Определите на сколько уменьшится заряд конденсатора, ответ выразите в микрокулонах.
}
\answer{%
    $q = C \cdot \Delta U = 800\,\text{мкКл}$
}
\solutionspace{80pt}

\tasknumber{5}%
\task{%
    Определите ёмкость конденсатора, если при его зарядке до напряжения
    $U = 2\,\text{кВ}$ он приобретает заряд $q = 18\,\text{мКл}$.
    % Чему при этом равны заряды обкладок конденсатора (сделайте рисунок и укажите их)?
    Ответ выразите в нанофарадах.
}
\answer{%
    $
        q = CU \implies
        C = \frac{q}{U} = \frac{18\,\text{мКл}}{2\,\text{кВ}} = 9000\,\text{нФ}.
        \text{ Заряды обкладок: $q$ и $-q$}
    $
}
\solutionspace{120pt}

\tasknumber{6}%
\task{%
    Как и во сколько раз изменится ёмкость плоского конденсатора
    при уменьшении площади пластин в 5 раз
    и уменьшении расстояния между ними в 6 раз?
    В ответе укажите простую дробь или число — отношение новой ёмкости к старой.
}
\answer{%
    $
        \frac{C'}C
            = \frac{\eps_0\eps \frac S5}{\frac d6} \Big/ \frac{\eps_0\eps S}d
            = \frac{6}{5} = > 1 \implies \text{увеличится в $\frac65$ раз}
    $
}
\solutionspace{80pt}

\tasknumber{7}%
\task{%
    Электрическая ёмкость конденсатора равна $C = 400\,\text{пФ}$,
    при этом ему сообщён заряд $q = 500\,\text{нКл}$.
    Какова энергия заряженного конденсатора?
    Ответ выразите в микроджоулях и округлите до целого.
}
\answer{%
    $
        W
        = \frac{q^2}{2C}
        = \frac{\sqr{500\,\text{нКл}}}{2 \cdot 400\,\text{пФ}}
        = 312{,}50\,\text{мкДж}
    $
}

\variantsplitter

\addpersonalvariant{Ирина Ан}

\tasknumber{1}%
\task{%
    Установите соответствие и запишите в ответ набор цифр (без других символов).

    А) энергия заряженного конденсатора, Б) напряжённость электрического поля, В) электрический заряд.

    1) кулон, 2) джоуль, 3) Н / Кл, 4) фарад, 5) ампер.
}
\answer{%
    $231$
}
\solutionspace{40pt}

\tasknumber{2}%
\task{%
    Установите соответствие и запишите в ответ набор цифр (без других символов).

    А) разность потенциалов, Б) энергия заряженного конденсатора, В) электроёмкость.

    1) $C$, 2) $U$, 3) $W$, 4) $r$, 5) $E$.
}
\answer{%
    $231$
}
\solutionspace{40pt}

\tasknumber{3}%
\task{%
    На конденсаторе указано: $C = 80\,\text{пФ}$, $V = 450\,\text{В}$.
    Удастся ли его использовать для накопления заряда $q = 50\,\text{нКл}$?
    (в ответе укажите «да» или «нет»)
}
\answer{%
    $
        q_{\text{max}} = CV = 80\,\text{пФ} \cdot 450\,\text{В} = 36\,\text{нКл}
        \implies q_{\text{max}}  <  q \implies \text{не удастся}
    $
}
\solutionspace{80pt}

\tasknumber{4}%
\task{%
    Конденсатор ёмкостью $20\,\text{нФ}$ был заряжен до напряжения $45\,\text{кВ}$.
    Затем напряжение уменьшают до $10\,\text{кВ}$.
    Определите на сколько уменьшится заряд конденсатора, ответ выразите в микрокулонах.
}
\answer{%
    $q = C \cdot \Delta U = 700\,\text{мкКл}$
}
\solutionspace{80pt}

\tasknumber{5}%
\task{%
    Определите ёмкость конденсатора, если при его зарядке до напряжения
    $V = 40\,\text{кВ}$ он приобретает заряд $Q = 25\,\text{мКл}$.
    % Чему при этом равны заряды обкладок конденсатора (сделайте рисунок и укажите их)?
    Ответ выразите в нанофарадах.
}
\answer{%
    $
        Q = CV \implies
        C = \frac{Q}{V} = \frac{25\,\text{мКл}}{40\,\text{кВ}} = 625\,\text{нФ}.
        \text{ Заряды обкладок: $Q$ и $-Q$}
    $
}
\solutionspace{120pt}

\tasknumber{6}%
\task{%
    Как и во сколько раз изменится ёмкость плоского конденсатора
    при уменьшении площади пластин в 7 раз
    и уменьшении расстояния между ними в 8 раз?
    В ответе укажите простую дробь или число — отношение новой ёмкости к старой.
}
\answer{%
    $
        \frac{C'}C
            = \frac{\eps_0\eps \frac S7}{\frac d8} \Big/ \frac{\eps_0\eps S}d
            = \frac{8}{7} = > 1 \implies \text{увеличится в $\frac87$ раз}
    $
}
\solutionspace{80pt}

\tasknumber{7}%
\task{%
    Электрическая ёмкость конденсатора равна $C = 750\,\text{пФ}$,
    при этом ему сообщён заряд $Q = 900\,\text{нКл}$.
    Какова энергия заряженного конденсатора?
    Ответ выразите в микроджоулях и округлите до целого.
}
\answer{%
    $
        W
        = \frac{Q^2}{2C}
        = \frac{\sqr{900\,\text{нКл}}}{2 \cdot 750\,\text{пФ}}
        = 540{,}00\,\text{мкДж}
    $
}

\variantsplitter

\addpersonalvariant{Софья Андрианова}

\tasknumber{1}%
\task{%
    Установите соответствие и запишите в ответ набор цифр (без других символов).

    А) энергия заряженного конденсатора, Б) разность потенциалов, В) электроёмкость.

    1) джоуль, 2) ампер, 3) ватт, 4) фарад, 5) вольт.
}
\answer{%
    $154$
}
\solutionspace{40pt}

\tasknumber{2}%
\task{%
    Установите соответствие и запишите в ответ набор цифр (без других символов).

    А) напряжённость электрического поля, Б) разность потенциалов, В) энергия заряженного конденсатора.

    1) $E$, 2) $Q$, 3) $\eps$, 4) $W$, 5) $U$.
}
\answer{%
    $154$
}
\solutionspace{40pt}

\tasknumber{3}%
\task{%
    На конденсаторе указано: $C = 150\,\text{пФ}$, $V = 450\,\text{В}$.
    Удастся ли его использовать для накопления заряда $q = 50\,\text{нКл}$?
    (в ответе укажите «да» или «нет»)
}
\answer{%
    $
        q_{\text{max}} = CV = 150\,\text{пФ} \cdot 450\,\text{В} = 67\,\text{нКл}
        \implies q_{\text{max}} \ge q \implies \text{удастся}
    $
}
\solutionspace{80pt}

\tasknumber{4}%
\task{%
    Конденсатор ёмкостью $20\,\text{нФ}$ был заряжен до напряжения $55\,\text{кВ}$.
    Затем напряжение уменьшают до $10\,\text{кВ}$.
    Определите на сколько уменьшится заряд конденсатора, ответ выразите в микрокулонах.
}
\answer{%
    $q = C \cdot \Delta U = 900\,\text{мкКл}$
}
\solutionspace{80pt}

\tasknumber{5}%
\task{%
    Определите ёмкость конденсатора, если при его зарядке до напряжения
    $V = 40\,\text{кВ}$ он приобретает заряд $q = 4\,\text{мКл}$.
    % Чему при этом равны заряды обкладок конденсатора (сделайте рисунок и укажите их)?
    Ответ выразите в нанофарадах.
}
\answer{%
    $
        q = CV \implies
        C = \frac{q}{V} = \frac{4\,\text{мКл}}{40\,\text{кВ}} = 100\,\text{нФ}.
        \text{ Заряды обкладок: $q$ и $-q$}
    $
}
\solutionspace{120pt}

\tasknumber{6}%
\task{%
    Как и во сколько раз изменится ёмкость плоского конденсатора
    при уменьшении площади пластин в 8 раз
    и уменьшении расстояния между ними в 3 раз?
    В ответе укажите простую дробь или число — отношение новой ёмкости к старой.
}
\answer{%
    $
        \frac{C'}C
            = \frac{\eps_0\eps \frac S8}{\frac d3} \Big/ \frac{\eps_0\eps S}d
            = \frac{3}{8} = < 1 \implies \text{уменьшится в $\frac83$ раз}
    $
}
\solutionspace{80pt}

\tasknumber{7}%
\task{%
    Электрическая ёмкость конденсатора равна $C = 400\,\text{пФ}$,
    при этом ему сообщён заряд $Q = 300\,\text{нКл}$.
    Какова энергия заряженного конденсатора?
    Ответ выразите в микроджоулях и округлите до целого.
}
\answer{%
    $
        W
        = \frac{Q^2}{2C}
        = \frac{\sqr{300\,\text{нКл}}}{2 \cdot 400\,\text{пФ}}
        = 112{,}50\,\text{мкДж}
    $
}

\variantsplitter

\addpersonalvariant{Владимир Артемчук}

\tasknumber{1}%
\task{%
    Установите соответствие и запишите в ответ набор цифр (без других символов).

    А) напряжённость электрического поля, Б) электрический заряд, В) электроёмкость.

    1) Н / Кл, 2) ватт, 3) кулон, 4) фарад, 5) генри.
}
\answer{%
    $134$
}
\solutionspace{40pt}

\tasknumber{2}%
\task{%
    Установите соответствие и запишите в ответ набор цифр (без других символов).

    А) электрический заряд, Б) электрическая постоянная, В) энергия заряженного конденсатора.

    1) $Q$, 2) $E$, 3) $\eps_0$, 4) $W$, 5) $\eps$.
}
\answer{%
    $134$
}
\solutionspace{40pt}

\tasknumber{3}%
\task{%
    На конденсаторе указано: $C = 100\,\text{пФ}$, $V = 200\,\text{В}$.
    Удастся ли его использовать для накопления заряда $Q = 50\,\text{нКл}$?
    (в ответе укажите «да» или «нет»)
}
\answer{%
    $
        Q_{\text{max}} = CV = 100\,\text{пФ} \cdot 200\,\text{В} = 20\,\text{нКл}
        \implies Q_{\text{max}}  <  Q \implies \text{не удастся}
    $
}
\solutionspace{80pt}

\tasknumber{4}%
\task{%
    Конденсатор ёмкостью $20\,\text{нФ}$ был заряжен до напряжения $75\,\text{кВ}$.
    Затем напряжение уменьшают на $30\,\text{кВ}$.
    Определите на сколько уменьшится заряд конденсатора, ответ выразите в микрокулонах.
}
\answer{%
    $q = C \cdot \Delta U = 600\,\text{мкКл}$
}
\solutionspace{80pt}

\tasknumber{5}%
\task{%
    Определите ёмкость конденсатора, если при его зарядке до напряжения
    $V = 40\,\text{кВ}$ он приобретает заряд $Q = 4\,\text{мКл}$.
    % Чему при этом равны заряды обкладок конденсатора (сделайте рисунок и укажите их)?
    Ответ выразите в нанофарадах.
}
\answer{%
    $
        Q = CV \implies
        C = \frac{Q}{V} = \frac{4\,\text{мКл}}{40\,\text{кВ}} = 100\,\text{нФ}.
        \text{ Заряды обкладок: $Q$ и $-Q$}
    $
}
\solutionspace{120pt}

\tasknumber{6}%
\task{%
    Как и во сколько раз изменится ёмкость плоского конденсатора
    при уменьшении площади пластин в 8 раз
    и уменьшении расстояния между ними в 4 раз?
    В ответе укажите простую дробь или число — отношение новой ёмкости к старой.
}
\answer{%
    $
        \frac{C'}C
            = \frac{\eps_0\eps \frac S8}{\frac d4} \Big/ \frac{\eps_0\eps S}d
            = \frac{4}{8} = < 1 \implies \text{уменьшится в $2$ раз}
    $
}
\solutionspace{80pt}

\tasknumber{7}%
\task{%
    Электрическая ёмкость конденсатора равна $C = 400\,\text{пФ}$,
    при этом ему сообщён заряд $q = 900\,\text{нКл}$.
    Какова энергия заряженного конденсатора?
    Ответ выразите в микроджоулях и округлите до целого.
}
\answer{%
    $
        W
        = \frac{q^2}{2C}
        = \frac{\sqr{900\,\text{нКл}}}{2 \cdot 400\,\text{пФ}}
        = 1012{,}50\,\text{мкДж}
    $
}

\variantsplitter

\addpersonalvariant{Софья Белянкина}

\tasknumber{1}%
\task{%
    Установите соответствие и запишите в ответ набор цифр (без других символов).

    А) энергия заряженного конденсатора, Б) электроёмкость, В) напряжённость электрического поля.

    1) генри, 2) вольт, 3) джоуль, 4) Н / Кл, 5) фарад.
}
\answer{%
    $354$
}
\solutionspace{40pt}

\tasknumber{2}%
\task{%
    Установите соответствие и запишите в ответ набор цифр (без других символов).

    А) напряжённость электрического поля, Б) энергия заряженного конденсатора, В) электроёмкость.

    1) $\eps_0$, 2) $\eps$, 3) $E$, 4) $C$, 5) $W$.
}
\answer{%
    $354$
}
\solutionspace{40pt}

\tasknumber{3}%
\task{%
    На конденсаторе указано: $C = 150\,\text{пФ}$, $U = 400\,\text{В}$.
    Удастся ли его использовать для накопления заряда $Q = 50\,\text{нКл}$?
    (в ответе укажите «да» или «нет»)
}
\answer{%
    $
        Q_{\text{max}} = CU = 150\,\text{пФ} \cdot 400\,\text{В} = 60\,\text{нКл}
        \implies Q_{\text{max}} \ge Q \implies \text{удастся}
    $
}
\solutionspace{80pt}

\tasknumber{4}%
\task{%
    Конденсатор ёмкостью $30\,\text{нФ}$ был заряжен до напряжения $45\,\text{кВ}$.
    Затем напряжение уменьшают до $20\,\text{кВ}$.
    Определите на сколько уменьшится заряд конденсатора, ответ выразите в микрокулонах.
}
\answer{%
    $q = C \cdot \Delta U = 750\,\text{мкКл}$
}
\solutionspace{80pt}

\tasknumber{5}%
\task{%
    Определите ёмкость конденсатора, если при его зарядке до напряжения
    $V = 2\,\text{кВ}$ он приобретает заряд $Q = 6\,\text{мКл}$.
    % Чему при этом равны заряды обкладок конденсатора (сделайте рисунок и укажите их)?
    Ответ выразите в нанофарадах.
}
\answer{%
    $
        Q = CV \implies
        C = \frac{Q}{V} = \frac{6\,\text{мКл}}{2\,\text{кВ}} = 3000\,\text{нФ}.
        \text{ Заряды обкладок: $Q$ и $-Q$}
    $
}
\solutionspace{120pt}

\tasknumber{6}%
\task{%
    Как и во сколько раз изменится ёмкость плоского конденсатора
    при уменьшении площади пластин в 4 раз
    и уменьшении расстояния между ними в 4 раз?
    В ответе укажите простую дробь или число — отношение новой ёмкости к старой.
}
\answer{%
    $
        \frac{C'}C
            = \frac{\eps_0\eps \frac S4}{\frac d4} \Big/ \frac{\eps_0\eps S}d
            = \frac{4}{4} = = 1 \implies \text{не изменится}
    $
}
\solutionspace{80pt}

\tasknumber{7}%
\task{%
    Электрическая ёмкость конденсатора равна $C = 600\,\text{пФ}$,
    при этом ему сообщён заряд $Q = 500\,\text{нКл}$.
    Какова энергия заряженного конденсатора?
    Ответ выразите в микроджоулях и округлите до целого.
}
\answer{%
    $
        W
        = \frac{Q^2}{2C}
        = \frac{\sqr{500\,\text{нКл}}}{2 \cdot 600\,\text{пФ}}
        = 208{,}33\,\text{мкДж}
    $
}

\variantsplitter

\addpersonalvariant{Варвара Егиазарян}

\tasknumber{1}%
\task{%
    Установите соответствие и запишите в ответ набор цифр (без других символов).

    А) разность потенциалов, Б) напряжённость электрического поля, В) электроёмкость.

    1) ампер, 2) ватт, 3) Н / Кл, 4) фарад, 5) вольт.
}
\answer{%
    $534$
}
\solutionspace{40pt}

\tasknumber{2}%
\task{%
    Установите соответствие и запишите в ответ набор цифр (без других символов).

    А) энергия заряженного конденсатора, Б) электрический заряд, В) электрическая постоянная.

    1) $C$, 2) $E$, 3) $Q$, 4) $\eps_0$, 5) $W$.
}
\answer{%
    $534$
}
\solutionspace{40pt}

\tasknumber{3}%
\task{%
    На конденсаторе указано: $C = 150\,\text{пФ}$, $U = 450\,\text{В}$.
    Удастся ли его использовать для накопления заряда $Q = 60\,\text{нКл}$?
    (в ответе укажите «да» или «нет»)
}
\answer{%
    $
        Q_{\text{max}} = CU = 150\,\text{пФ} \cdot 450\,\text{В} = 67\,\text{нКл}
        \implies Q_{\text{max}} \ge Q \implies \text{удастся}
    $
}
\solutionspace{80pt}

\tasknumber{4}%
\task{%
    Конденсатор ёмкостью $20\,\text{нФ}$ был заряжен до напряжения $55\,\text{кВ}$.
    Затем напряжение уменьшают до $10\,\text{кВ}$.
    Определите на сколько уменьшится заряд конденсатора, ответ выразите в микрокулонах.
}
\answer{%
    $q = C \cdot \Delta U = 900\,\text{мкКл}$
}
\solutionspace{80pt}

\tasknumber{5}%
\task{%
    Определите ёмкость конденсатора, если при его зарядке до напряжения
    $V = 20\,\text{кВ}$ он приобретает заряд $q = 4\,\text{мКл}$.
    % Чему при этом равны заряды обкладок конденсатора (сделайте рисунок и укажите их)?
    Ответ выразите в нанофарадах.
}
\answer{%
    $
        q = CV \implies
        C = \frac{q}{V} = \frac{4\,\text{мКл}}{20\,\text{кВ}} = 200\,\text{нФ}.
        \text{ Заряды обкладок: $q$ и $-q$}
    $
}
\solutionspace{120pt}

\tasknumber{6}%
\task{%
    Как и во сколько раз изменится ёмкость плоского конденсатора
    при уменьшении площади пластин в 5 раз
    и уменьшении расстояния между ними в 4 раз?
    В ответе укажите простую дробь или число — отношение новой ёмкости к старой.
}
\answer{%
    $
        \frac{C'}C
            = \frac{\eps_0\eps \frac S5}{\frac d4} \Big/ \frac{\eps_0\eps S}d
            = \frac{4}{5} = < 1 \implies \text{уменьшится в $\frac54$ раз}
    $
}
\solutionspace{80pt}

\tasknumber{7}%
\task{%
    Электрическая ёмкость конденсатора равна $C = 600\,\text{пФ}$,
    при этом ему сообщён заряд $Q = 900\,\text{нКл}$.
    Какова энергия заряженного конденсатора?
    Ответ выразите в микроджоулях и округлите до целого.
}
\answer{%
    $
        W
        = \frac{Q^2}{2C}
        = \frac{\sqr{900\,\text{нКл}}}{2 \cdot 600\,\text{пФ}}
        = 675{,}00\,\text{мкДж}
    $
}

\variantsplitter

\addpersonalvariant{Владислав Емелин}

\tasknumber{1}%
\task{%
    Установите соответствие и запишите в ответ набор цифр (без других символов).

    А) разность потенциалов, Б) электрический заряд, В) электроёмкость.

    1) фарад, 2) вольт, 3) кулон, 4) генри, 5) ватт.
}
\answer{%
    $231$
}
\solutionspace{40pt}

\tasknumber{2}%
\task{%
    Установите соответствие и запишите в ответ набор цифр (без других символов).

    А) электрическая постоянная, Б) разность потенциалов, В) напряжённость электрического поля.

    1) $E$, 2) $\eps_0$, 3) $U$, 4) $\eps$, 5) $W$.
}
\answer{%
    $231$
}
\solutionspace{40pt}

\tasknumber{3}%
\task{%
    На конденсаторе указано: $C = 100\,\text{пФ}$, $U = 450\,\text{В}$.
    Удастся ли его использовать для накопления заряда $q = 30\,\text{нКл}$?
    (в ответе укажите «да» или «нет»)
}
\answer{%
    $
        q_{\text{max}} = CU = 100\,\text{пФ} \cdot 450\,\text{В} = 45\,\text{нКл}
        \implies q_{\text{max}} \ge q \implies \text{удастся}
    $
}
\solutionspace{80pt}

\tasknumber{4}%
\task{%
    Конденсатор ёмкостью $30\,\text{нФ}$ был заряжен до напряжения $55\,\text{кВ}$.
    Затем напряжение уменьшают до $30\,\text{кВ}$.
    Определите на сколько уменьшится заряд конденсатора, ответ выразите в микрокулонах.
}
\answer{%
    $q = C \cdot \Delta U = 750\,\text{мкКл}$
}
\solutionspace{80pt}

\tasknumber{5}%
\task{%
    Определите ёмкость конденсатора, если при его зарядке до напряжения
    $U = 40\,\text{кВ}$ он приобретает заряд $Q = 15\,\text{мКл}$.
    % Чему при этом равны заряды обкладок конденсатора (сделайте рисунок и укажите их)?
    Ответ выразите в нанофарадах.
}
\answer{%
    $
        Q = CU \implies
        C = \frac{Q}{U} = \frac{15\,\text{мКл}}{40\,\text{кВ}} = 375\,\text{нФ}.
        \text{ Заряды обкладок: $Q$ и $-Q$}
    $
}
\solutionspace{120pt}

\tasknumber{6}%
\task{%
    Как и во сколько раз изменится ёмкость плоского конденсатора
    при уменьшении площади пластин в 8 раз
    и уменьшении расстояния между ними в 3 раз?
    В ответе укажите простую дробь или число — отношение новой ёмкости к старой.
}
\answer{%
    $
        \frac{C'}C
            = \frac{\eps_0\eps \frac S8}{\frac d3} \Big/ \frac{\eps_0\eps S}d
            = \frac{3}{8} = < 1 \implies \text{уменьшится в $\frac83$ раз}
    $
}
\solutionspace{80pt}

\tasknumber{7}%
\task{%
    Электрическая ёмкость конденсатора равна $C = 200\,\text{пФ}$,
    при этом ему сообщён заряд $Q = 800\,\text{нКл}$.
    Какова энергия заряженного конденсатора?
    Ответ выразите в микроджоулях и округлите до целого.
}
\answer{%
    $
        W
        = \frac{Q^2}{2C}
        = \frac{\sqr{800\,\text{нКл}}}{2 \cdot 200\,\text{пФ}}
        = 1600{,}00\,\text{мкДж}
    $
}

\variantsplitter

\addpersonalvariant{Артём Жичин}

\tasknumber{1}%
\task{%
    Установите соответствие и запишите в ответ набор цифр (без других символов).

    А) разность потенциалов, Б) напряжённость электрического поля, В) электрический заряд.

    1) генри, 2) Н / Кл, 3) кулон, 4) ватт, 5) вольт.
}
\answer{%
    $523$
}
\solutionspace{40pt}

\tasknumber{2}%
\task{%
    Установите соответствие и запишите в ответ набор цифр (без других символов).

    А) электрическая постоянная, Б) электрический заряд, В) напряжённость электрического поля.

    1) $W$, 2) $Q$, 3) $E$, 4) $U$, 5) $\eps_0$.
}
\answer{%
    $523$
}
\solutionspace{40pt}

\tasknumber{3}%
\task{%
    На конденсаторе указано: $C = 150\,\text{пФ}$, $U = 300\,\text{В}$.
    Удастся ли его использовать для накопления заряда $q = 60\,\text{нКл}$?
    (в ответе укажите «да» или «нет»)
}
\answer{%
    $
        q_{\text{max}} = CU = 150\,\text{пФ} \cdot 300\,\text{В} = 45\,\text{нКл}
        \implies q_{\text{max}}  <  q \implies \text{не удастся}
    $
}
\solutionspace{80pt}

\tasknumber{4}%
\task{%
    Конденсатор ёмкостью $30\,\text{нФ}$ был заряжен до напряжения $55\,\text{кВ}$.
    Затем напряжение уменьшают до $10\,\text{кВ}$.
    Определите на сколько уменьшится заряд конденсатора, ответ выразите в микрокулонах.
}
\answer{%
    $q = C \cdot \Delta U = 1350\,\text{мкКл}$
}
\solutionspace{80pt}

\tasknumber{5}%
\task{%
    Определите ёмкость конденсатора, если при его зарядке до напряжения
    $V = 20\,\text{кВ}$ он приобретает заряд $Q = 15\,\text{мКл}$.
    % Чему при этом равны заряды обкладок конденсатора (сделайте рисунок и укажите их)?
    Ответ выразите в нанофарадах.
}
\answer{%
    $
        Q = CV \implies
        C = \frac{Q}{V} = \frac{15\,\text{мКл}}{20\,\text{кВ}} = 750\,\text{нФ}.
        \text{ Заряды обкладок: $Q$ и $-Q$}
    $
}
\solutionspace{120pt}

\tasknumber{6}%
\task{%
    Как и во сколько раз изменится ёмкость плоского конденсатора
    при уменьшении площади пластин в 6 раз
    и уменьшении расстояния между ними в 7 раз?
    В ответе укажите простую дробь или число — отношение новой ёмкости к старой.
}
\answer{%
    $
        \frac{C'}C
            = \frac{\eps_0\eps \frac S6}{\frac d7} \Big/ \frac{\eps_0\eps S}d
            = \frac{7}{6} = > 1 \implies \text{увеличится в $\frac76$ раз}
    $
}
\solutionspace{80pt}

\tasknumber{7}%
\task{%
    Электрическая ёмкость конденсатора равна $C = 750\,\text{пФ}$,
    при этом ему сообщён заряд $Q = 900\,\text{нКл}$.
    Какова энергия заряженного конденсатора?
    Ответ выразите в микроджоулях и округлите до целого.
}
\answer{%
    $
        W
        = \frac{Q^2}{2C}
        = \frac{\sqr{900\,\text{нКл}}}{2 \cdot 750\,\text{пФ}}
        = 540{,}00\,\text{мкДж}
    $
}

\variantsplitter

\addpersonalvariant{Дарья Кошман}

\tasknumber{1}%
\task{%
    Установите соответствие и запишите в ответ набор цифр (без других символов).

    А) электроёмкость, Б) электрический заряд, В) напряжённость электрического поля.

    1) кулон, 2) Н / Кл, 3) генри, 4) фарад, 5) ватт.
}
\answer{%
    $412$
}
\solutionspace{40pt}

\tasknumber{2}%
\task{%
    Установите соответствие и запишите в ответ набор цифр (без других символов).

    А) электроёмкость, Б) энергия заряженного конденсатора, В) разность потенциалов.

    1) $W$, 2) $U$, 3) $\eps_0$, 4) $C$, 5) $Q$.
}
\answer{%
    $412$
}
\solutionspace{40pt}

\tasknumber{3}%
\task{%
    На конденсаторе указано: $C = 120\,\text{пФ}$, $U = 300\,\text{В}$.
    Удастся ли его использовать для накопления заряда $Q = 50\,\text{нКл}$?
    (в ответе укажите «да» или «нет»)
}
\answer{%
    $
        Q_{\text{max}} = CU = 120\,\text{пФ} \cdot 300\,\text{В} = 36\,\text{нКл}
        \implies Q_{\text{max}}  <  Q \implies \text{не удастся}
    $
}
\solutionspace{80pt}

\tasknumber{4}%
\task{%
    Конденсатор ёмкостью $30\,\text{нФ}$ был заряжен до напряжения $75\,\text{кВ}$.
    Затем напряжение уменьшают на $10\,\text{кВ}$.
    Определите на сколько уменьшится заряд конденсатора, ответ выразите в микрокулонах.
}
\answer{%
    $q = C \cdot \Delta U = 300\,\text{мкКл}$
}
\solutionspace{80pt}

\tasknumber{5}%
\task{%
    Определите ёмкость конденсатора, если при его зарядке до напряжения
    $U = 20\,\text{кВ}$ он приобретает заряд $q = 24\,\text{мКл}$.
    % Чему при этом равны заряды обкладок конденсатора (сделайте рисунок и укажите их)?
    Ответ выразите в нанофарадах.
}
\answer{%
    $
        q = CU \implies
        C = \frac{q}{U} = \frac{24\,\text{мКл}}{20\,\text{кВ}} = 1200\,\text{нФ}.
        \text{ Заряды обкладок: $q$ и $-q$}
    $
}
\solutionspace{120pt}

\tasknumber{6}%
\task{%
    Как и во сколько раз изменится ёмкость плоского конденсатора
    при уменьшении площади пластин в 4 раз
    и уменьшении расстояния между ними в 8 раз?
    В ответе укажите простую дробь или число — отношение новой ёмкости к старой.
}
\answer{%
    $
        \frac{C'}C
            = \frac{\eps_0\eps \frac S4}{\frac d8} \Big/ \frac{\eps_0\eps S}d
            = \frac{8}{4} = > 1 \implies \text{увеличится в $2$ раз}
    $
}
\solutionspace{80pt}

\tasknumber{7}%
\task{%
    Электрическая ёмкость конденсатора равна $C = 200\,\text{пФ}$,
    при этом ему сообщён заряд $q = 300\,\text{нКл}$.
    Какова энергия заряженного конденсатора?
    Ответ выразите в микроджоулях и округлите до целого.
}
\answer{%
    $
        W
        = \frac{q^2}{2C}
        = \frac{\sqr{300\,\text{нКл}}}{2 \cdot 200\,\text{пФ}}
        = 225{,}00\,\text{мкДж}
    $
}

\variantsplitter

\addpersonalvariant{Анна Кузьмичёва}

\tasknumber{1}%
\task{%
    Установите соответствие и запишите в ответ набор цифр (без других символов).

    А) электрический заряд, Б) энергия заряженного конденсатора, В) электроёмкость.

    1) фарад, 2) кулон, 3) ватт, 4) джоуль, 5) генри.
}
\answer{%
    $241$
}
\solutionspace{40pt}

\tasknumber{2}%
\task{%
    Установите соответствие и запишите в ответ набор цифр (без других символов).

    А) энергия заряженного конденсатора, Б) электроёмкость, В) разность потенциалов.

    1) $U$, 2) $W$, 3) $\eps$, 4) $C$, 5) $r$.
}
\answer{%
    $241$
}
\solutionspace{40pt}

\tasknumber{3}%
\task{%
    На конденсаторе указано: $C = 100\,\text{пФ}$, $U = 400\,\text{В}$.
    Удастся ли его использовать для накопления заряда $Q = 50\,\text{нКл}$?
    (в ответе укажите «да» или «нет»)
}
\answer{%
    $
        Q_{\text{max}} = CU = 100\,\text{пФ} \cdot 400\,\text{В} = 40\,\text{нКл}
        \implies Q_{\text{max}}  <  Q \implies \text{не удастся}
    $
}
\solutionspace{80pt}

\tasknumber{4}%
\task{%
    Конденсатор ёмкостью $40\,\text{нФ}$ был заряжен до напряжения $55\,\text{кВ}$.
    Затем напряжение уменьшают до $30\,\text{кВ}$.
    Определите на сколько уменьшится заряд конденсатора, ответ выразите в микрокулонах.
}
\answer{%
    $q = C \cdot \Delta U = 1000\,\text{мкКл}$
}
\solutionspace{80pt}

\tasknumber{5}%
\task{%
    Определите ёмкость конденсатора, если при его зарядке до напряжения
    $U = 50\,\text{кВ}$ он приобретает заряд $q = 18\,\text{мКл}$.
    % Чему при этом равны заряды обкладок конденсатора (сделайте рисунок и укажите их)?
    Ответ выразите в нанофарадах.
}
\answer{%
    $
        q = CU \implies
        C = \frac{q}{U} = \frac{18\,\text{мКл}}{50\,\text{кВ}} = 360\,\text{нФ}.
        \text{ Заряды обкладок: $q$ и $-q$}
    $
}
\solutionspace{120pt}

\tasknumber{6}%
\task{%
    Как и во сколько раз изменится ёмкость плоского конденсатора
    при уменьшении площади пластин в 4 раз
    и уменьшении расстояния между ними в 2 раз?
    В ответе укажите простую дробь или число — отношение новой ёмкости к старой.
}
\answer{%
    $
        \frac{C'}C
            = \frac{\eps_0\eps \frac S4}{\frac d2} \Big/ \frac{\eps_0\eps S}d
            = \frac{2}{4} = < 1 \implies \text{уменьшится в $2$ раз}
    $
}
\solutionspace{80pt}

\tasknumber{7}%
\task{%
    Электрическая ёмкость конденсатора равна $C = 600\,\text{пФ}$,
    при этом ему сообщён заряд $Q = 800\,\text{нКл}$.
    Какова энергия заряженного конденсатора?
    Ответ выразите в микроджоулях и округлите до целого.
}
\answer{%
    $
        W
        = \frac{Q^2}{2C}
        = \frac{\sqr{800\,\text{нКл}}}{2 \cdot 600\,\text{пФ}}
        = 533{,}33\,\text{мкДж}
    $
}

\variantsplitter

\addpersonalvariant{Алёна Куприянова}

\tasknumber{1}%
\task{%
    Установите соответствие и запишите в ответ набор цифр (без других символов).

    А) разность потенциалов, Б) электроёмкость, В) напряжённость электрического поля.

    1) ампер, 2) ватт, 3) вольт, 4) Н / Кл, 5) фарад.
}
\answer{%
    $354$
}
\solutionspace{40pt}

\tasknumber{2}%
\task{%
    Установите соответствие и запишите в ответ набор цифр (без других символов).

    А) энергия заряженного конденсатора, Б) электрический заряд, В) электроёмкость.

    1) $E$, 2) $U$, 3) $W$, 4) $C$, 5) $Q$.
}
\answer{%
    $354$
}
\solutionspace{40pt}

\tasknumber{3}%
\task{%
    На конденсаторе указано: $C = 120\,\text{пФ}$, $U = 200\,\text{В}$.
    Удастся ли его использовать для накопления заряда $Q = 60\,\text{нКл}$?
    (в ответе укажите «да» или «нет»)
}
\answer{%
    $
        Q_{\text{max}} = CU = 120\,\text{пФ} \cdot 200\,\text{В} = 24\,\text{нКл}
        \implies Q_{\text{max}}  <  Q \implies \text{не удастся}
    $
}
\solutionspace{80pt}

\tasknumber{4}%
\task{%
    Конденсатор ёмкостью $20\,\text{нФ}$ был заряжен до напряжения $65\,\text{кВ}$.
    Затем напряжение уменьшают на $30\,\text{кВ}$.
    Определите на сколько уменьшится заряд конденсатора, ответ выразите в микрокулонах.
}
\answer{%
    $q = C \cdot \Delta U = 600\,\text{мкКл}$
}
\solutionspace{80pt}

\tasknumber{5}%
\task{%
    Определите ёмкость конденсатора, если при его зарядке до напряжения
    $U = 40\,\text{кВ}$ он приобретает заряд $Q = 15\,\text{мКл}$.
    % Чему при этом равны заряды обкладок конденсатора (сделайте рисунок и укажите их)?
    Ответ выразите в нанофарадах.
}
\answer{%
    $
        Q = CU \implies
        C = \frac{Q}{U} = \frac{15\,\text{мКл}}{40\,\text{кВ}} = 375\,\text{нФ}.
        \text{ Заряды обкладок: $Q$ и $-Q$}
    $
}
\solutionspace{120pt}

\tasknumber{6}%
\task{%
    Как и во сколько раз изменится ёмкость плоского конденсатора
    при уменьшении площади пластин в 7 раз
    и уменьшении расстояния между ними в 4 раз?
    В ответе укажите простую дробь или число — отношение новой ёмкости к старой.
}
\answer{%
    $
        \frac{C'}C
            = \frac{\eps_0\eps \frac S7}{\frac d4} \Big/ \frac{\eps_0\eps S}d
            = \frac{4}{7} = < 1 \implies \text{уменьшится в $\frac74$ раз}
    $
}
\solutionspace{80pt}

\tasknumber{7}%
\task{%
    Электрическая ёмкость конденсатора равна $C = 400\,\text{пФ}$,
    при этом ему сообщён заряд $q = 500\,\text{нКл}$.
    Какова энергия заряженного конденсатора?
    Ответ выразите в микроджоулях и округлите до целого.
}
\answer{%
    $
        W
        = \frac{q^2}{2C}
        = \frac{\sqr{500\,\text{нКл}}}{2 \cdot 400\,\text{пФ}}
        = 312{,}50\,\text{мкДж}
    $
}

\variantsplitter

\addpersonalvariant{Ярослав Лавровский}

\tasknumber{1}%
\task{%
    Установите соответствие и запишите в ответ набор цифр (без других символов).

    А) электрический заряд, Б) электроёмкость, В) разность потенциалов.

    1) джоуль, 2) вольт, 3) кулон, 4) Н / Кл, 5) фарад.
}
\answer{%
    $352$
}
\solutionspace{40pt}

\tasknumber{2}%
\task{%
    Установите соответствие и запишите в ответ набор цифр (без других символов).

    А) разность потенциалов, Б) электрическая постоянная, В) энергия заряженного конденсатора.

    1) $r$, 2) $W$, 3) $U$, 4) $C$, 5) $\eps_0$.
}
\answer{%
    $352$
}
\solutionspace{40pt}

\tasknumber{3}%
\task{%
    На конденсаторе указано: $C = 120\,\text{пФ}$, $V = 400\,\text{В}$.
    Удастся ли его использовать для накопления заряда $q = 50\,\text{нКл}$?
    (в ответе укажите «да» или «нет»)
}
\answer{%
    $
        q_{\text{max}} = CV = 120\,\text{пФ} \cdot 400\,\text{В} = 48\,\text{нКл}
        \implies q_{\text{max}}  <  q \implies \text{не удастся}
    $
}
\solutionspace{80pt}

\tasknumber{4}%
\task{%
    Конденсатор ёмкостью $40\,\text{нФ}$ был заряжен до напряжения $55\,\text{кВ}$.
    Затем напряжение уменьшают на $20\,\text{кВ}$.
    Определите на сколько уменьшится заряд конденсатора, ответ выразите в микрокулонах.
}
\answer{%
    $q = C \cdot \Delta U = 800\,\text{мкКл}$
}
\solutionspace{80pt}

\tasknumber{5}%
\task{%
    Определите ёмкость конденсатора, если при его зарядке до напряжения
    $U = 4\,\text{кВ}$ он приобретает заряд $q = 4\,\text{мКл}$.
    % Чему при этом равны заряды обкладок конденсатора (сделайте рисунок и укажите их)?
    Ответ выразите в нанофарадах.
}
\answer{%
    $
        q = CU \implies
        C = \frac{q}{U} = \frac{4\,\text{мКл}}{4\,\text{кВ}} = 1000\,\text{нФ}.
        \text{ Заряды обкладок: $q$ и $-q$}
    $
}
\solutionspace{120pt}

\tasknumber{6}%
\task{%
    Как и во сколько раз изменится ёмкость плоского конденсатора
    при уменьшении площади пластин в 3 раз
    и уменьшении расстояния между ними в 5 раз?
    В ответе укажите простую дробь или число — отношение новой ёмкости к старой.
}
\answer{%
    $
        \frac{C'}C
            = \frac{\eps_0\eps \frac S3}{\frac d5} \Big/ \frac{\eps_0\eps S}d
            = \frac{5}{3} = > 1 \implies \text{увеличится в $\frac53$ раз}
    $
}
\solutionspace{80pt}

\tasknumber{7}%
\task{%
    Электрическая ёмкость конденсатора равна $C = 600\,\text{пФ}$,
    при этом ему сообщён заряд $q = 800\,\text{нКл}$.
    Какова энергия заряженного конденсатора?
    Ответ выразите в микроджоулях и округлите до целого.
}
\answer{%
    $
        W
        = \frac{q^2}{2C}
        = \frac{\sqr{800\,\text{нКл}}}{2 \cdot 600\,\text{пФ}}
        = 533{,}33\,\text{мкДж}
    $
}

\variantsplitter

\addpersonalvariant{Анастасия Ламанова}

\tasknumber{1}%
\task{%
    Установите соответствие и запишите в ответ набор цифр (без других символов).

    А) электрический заряд, Б) электроёмкость, В) разность потенциалов.

    1) вольт, 2) джоуль, 3) фарад, 4) ватт, 5) кулон.
}
\answer{%
    $531$
}
\solutionspace{40pt}

\tasknumber{2}%
\task{%
    Установите соответствие и запишите в ответ набор цифр (без других символов).

    А) напряжённость электрического поля, Б) электрическая постоянная, В) электрический заряд.

    1) $Q$, 2) $r$, 3) $\eps_0$, 4) $W$, 5) $E$.
}
\answer{%
    $531$
}
\solutionspace{40pt}

\tasknumber{3}%
\task{%
    На конденсаторе указано: $C = 50\,\text{пФ}$, $V = 200\,\text{В}$.
    Удастся ли его использовать для накопления заряда $q = 60\,\text{нКл}$?
    (в ответе укажите «да» или «нет»)
}
\answer{%
    $
        q_{\text{max}} = CV = 50\,\text{пФ} \cdot 200\,\text{В} = 10\,\text{нКл}
        \implies q_{\text{max}}  <  q \implies \text{не удастся}
    $
}
\solutionspace{80pt}

\tasknumber{4}%
\task{%
    Конденсатор ёмкостью $30\,\text{нФ}$ был заряжен до напряжения $45\,\text{кВ}$.
    Затем напряжение уменьшают на $30\,\text{кВ}$.
    Определите на сколько уменьшится заряд конденсатора, ответ выразите в микрокулонах.
}
\answer{%
    $q = C \cdot \Delta U = 900\,\text{мкКл}$
}
\solutionspace{80pt}

\tasknumber{5}%
\task{%
    Определите ёмкость конденсатора, если при его зарядке до напряжения
    $U = 40\,\text{кВ}$ он приобретает заряд $Q = 6\,\text{мКл}$.
    % Чему при этом равны заряды обкладок конденсатора (сделайте рисунок и укажите их)?
    Ответ выразите в нанофарадах.
}
\answer{%
    $
        Q = CU \implies
        C = \frac{Q}{U} = \frac{6\,\text{мКл}}{40\,\text{кВ}} = 150\,\text{нФ}.
        \text{ Заряды обкладок: $Q$ и $-Q$}
    $
}
\solutionspace{120pt}

\tasknumber{6}%
\task{%
    Как и во сколько раз изменится ёмкость плоского конденсатора
    при уменьшении площади пластин в 5 раз
    и уменьшении расстояния между ними в 4 раз?
    В ответе укажите простую дробь или число — отношение новой ёмкости к старой.
}
\answer{%
    $
        \frac{C'}C
            = \frac{\eps_0\eps \frac S5}{\frac d4} \Big/ \frac{\eps_0\eps S}d
            = \frac{4}{5} = < 1 \implies \text{уменьшится в $\frac54$ раз}
    $
}
\solutionspace{80pt}

\tasknumber{7}%
\task{%
    Электрическая ёмкость конденсатора равна $C = 600\,\text{пФ}$,
    при этом ему сообщён заряд $q = 900\,\text{нКл}$.
    Какова энергия заряженного конденсатора?
    Ответ выразите в микроджоулях и округлите до целого.
}
\answer{%
    $
        W
        = \frac{q^2}{2C}
        = \frac{\sqr{900\,\text{нКл}}}{2 \cdot 600\,\text{пФ}}
        = 675{,}00\,\text{мкДж}
    $
}

\variantsplitter

\addpersonalvariant{Виктория Легонькова}

\tasknumber{1}%
\task{%
    Установите соответствие и запишите в ответ набор цифр (без других символов).

    А) электрический заряд, Б) разность потенциалов, В) электроёмкость.

    1) фарад, 2) джоуль, 3) ватт, 4) вольт, 5) кулон.
}
\answer{%
    $541$
}
\solutionspace{40pt}

\tasknumber{2}%
\task{%
    Установите соответствие и запишите в ответ набор цифр (без других символов).

    А) энергия заряженного конденсатора, Б) напряжённость электрического поля, В) электрический заряд.

    1) $Q$, 2) $r$, 3) $\eps$, 4) $E$, 5) $W$.
}
\answer{%
    $541$
}
\solutionspace{40pt}

\tasknumber{3}%
\task{%
    На конденсаторе указано: $C = 120\,\text{пФ}$, $V = 400\,\text{В}$.
    Удастся ли его использовать для накопления заряда $q = 60\,\text{нКл}$?
    (в ответе укажите «да» или «нет»)
}
\answer{%
    $
        q_{\text{max}} = CV = 120\,\text{пФ} \cdot 400\,\text{В} = 48\,\text{нКл}
        \implies q_{\text{max}}  <  q \implies \text{не удастся}
    $
}
\solutionspace{80pt}

\tasknumber{4}%
\task{%
    Конденсатор ёмкостью $40\,\text{нФ}$ был заряжен до напряжения $65\,\text{кВ}$.
    Затем напряжение уменьшают до $30\,\text{кВ}$.
    Определите на сколько уменьшится заряд конденсатора, ответ выразите в микрокулонах.
}
\answer{%
    $q = C \cdot \Delta U = 1400\,\text{мкКл}$
}
\solutionspace{80pt}

\tasknumber{5}%
\task{%
    Определите ёмкость конденсатора, если при его зарядке до напряжения
    $V = 4\,\text{кВ}$ он приобретает заряд $q = 25\,\text{мКл}$.
    % Чему при этом равны заряды обкладок конденсатора (сделайте рисунок и укажите их)?
    Ответ выразите в нанофарадах.
}
\answer{%
    $
        q = CV \implies
        C = \frac{q}{V} = \frac{25\,\text{мКл}}{4\,\text{кВ}} = 6250\,\text{нФ}.
        \text{ Заряды обкладок: $q$ и $-q$}
    $
}
\solutionspace{120pt}

\tasknumber{6}%
\task{%
    Как и во сколько раз изменится ёмкость плоского конденсатора
    при уменьшении площади пластин в 8 раз
    и уменьшении расстояния между ними в 3 раз?
    В ответе укажите простую дробь или число — отношение новой ёмкости к старой.
}
\answer{%
    $
        \frac{C'}C
            = \frac{\eps_0\eps \frac S8}{\frac d3} \Big/ \frac{\eps_0\eps S}d
            = \frac{3}{8} = < 1 \implies \text{уменьшится в $\frac83$ раз}
    $
}
\solutionspace{80pt}

\tasknumber{7}%
\task{%
    Электрическая ёмкость конденсатора равна $C = 600\,\text{пФ}$,
    при этом ему сообщён заряд $q = 800\,\text{нКл}$.
    Какова энергия заряженного конденсатора?
    Ответ выразите в микроджоулях и округлите до целого.
}
\answer{%
    $
        W
        = \frac{q^2}{2C}
        = \frac{\sqr{800\,\text{нКл}}}{2 \cdot 600\,\text{пФ}}
        = 533{,}33\,\text{мкДж}
    $
}

\variantsplitter

\addpersonalvariant{Семён Мартынов}

\tasknumber{1}%
\task{%
    Установите соответствие и запишите в ответ набор цифр (без других символов).

    А) разность потенциалов, Б) напряжённость электрического поля, В) электрический заряд.

    1) кулон, 2) Н / Кл, 3) вольт, 4) генри, 5) ампер.
}
\answer{%
    $321$
}
\solutionspace{40pt}

\tasknumber{2}%
\task{%
    Установите соответствие и запишите в ответ набор цифр (без других символов).

    А) электрическая постоянная, Б) электрический заряд, В) напряжённость электрического поля.

    1) $E$, 2) $Q$, 3) $\eps_0$, 4) $W$, 5) $C$.
}
\answer{%
    $321$
}
\solutionspace{40pt}

\tasknumber{3}%
\task{%
    На конденсаторе указано: $C = 120\,\text{пФ}$, $V = 200\,\text{В}$.
    Удастся ли его использовать для накопления заряда $Q = 60\,\text{нКл}$?
    (в ответе укажите «да» или «нет»)
}
\answer{%
    $
        Q_{\text{max}} = CV = 120\,\text{пФ} \cdot 200\,\text{В} = 24\,\text{нКл}
        \implies Q_{\text{max}}  <  Q \implies \text{не удастся}
    $
}
\solutionspace{80pt}

\tasknumber{4}%
\task{%
    Конденсатор ёмкостью $30\,\text{нФ}$ был заряжен до напряжения $75\,\text{кВ}$.
    Затем напряжение уменьшают на $10\,\text{кВ}$.
    Определите на сколько уменьшится заряд конденсатора, ответ выразите в микрокулонах.
}
\answer{%
    $q = C \cdot \Delta U = 300\,\text{мкКл}$
}
\solutionspace{80pt}

\tasknumber{5}%
\task{%
    Определите ёмкость конденсатора, если при его зарядке до напряжения
    $U = 4\,\text{кВ}$ он приобретает заряд $Q = 18\,\text{мКл}$.
    % Чему при этом равны заряды обкладок конденсатора (сделайте рисунок и укажите их)?
    Ответ выразите в нанофарадах.
}
\answer{%
    $
        Q = CU \implies
        C = \frac{Q}{U} = \frac{18\,\text{мКл}}{4\,\text{кВ}} = 4500\,\text{нФ}.
        \text{ Заряды обкладок: $Q$ и $-Q$}
    $
}
\solutionspace{120pt}

\tasknumber{6}%
\task{%
    Как и во сколько раз изменится ёмкость плоского конденсатора
    при уменьшении площади пластин в 2 раз
    и уменьшении расстояния между ними в 5 раз?
    В ответе укажите простую дробь или число — отношение новой ёмкости к старой.
}
\answer{%
    $
        \frac{C'}C
            = \frac{\eps_0\eps \frac S2}{\frac d5} \Big/ \frac{\eps_0\eps S}d
            = \frac{5}{2} = > 1 \implies \text{увеличится в $\frac52$ раз}
    $
}
\solutionspace{80pt}

\tasknumber{7}%
\task{%
    Электрическая ёмкость конденсатора равна $C = 750\,\text{пФ}$,
    при этом ему сообщён заряд $q = 300\,\text{нКл}$.
    Какова энергия заряженного конденсатора?
    Ответ выразите в микроджоулях и округлите до целого.
}
\answer{%
    $
        W
        = \frac{q^2}{2C}
        = \frac{\sqr{300\,\text{нКл}}}{2 \cdot 750\,\text{пФ}}
        = 60{,}00\,\text{мкДж}
    $
}

\variantsplitter

\addpersonalvariant{Варвара Минаева}

\tasknumber{1}%
\task{%
    Установите соответствие и запишите в ответ набор цифр (без других символов).

    А) напряжённость электрического поля, Б) энергия заряженного конденсатора, В) разность потенциалов.

    1) кулон, 2) Н / Кл, 3) ватт, 4) вольт, 5) джоуль.
}
\answer{%
    $254$
}
\solutionspace{40pt}

\tasknumber{2}%
\task{%
    Установите соответствие и запишите в ответ набор цифр (без других символов).

    А) электроёмкость, Б) электрическая постоянная, В) энергия заряженного конденсатора.

    1) $\eps$, 2) $C$, 3) $Q$, 4) $W$, 5) $\eps_0$.
}
\answer{%
    $254$
}
\solutionspace{40pt}

\tasknumber{3}%
\task{%
    На конденсаторе указано: $C = 80\,\text{пФ}$, $U = 400\,\text{В}$.
    Удастся ли его использовать для накопления заряда $Q = 60\,\text{нКл}$?
    (в ответе укажите «да» или «нет»)
}
\answer{%
    $
        Q_{\text{max}} = CU = 80\,\text{пФ} \cdot 400\,\text{В} = 32\,\text{нКл}
        \implies Q_{\text{max}}  <  Q \implies \text{не удастся}
    $
}
\solutionspace{80pt}

\tasknumber{4}%
\task{%
    Конденсатор ёмкостью $40\,\text{нФ}$ был заряжен до напряжения $45\,\text{кВ}$.
    Затем напряжение уменьшают до $20\,\text{кВ}$.
    Определите на сколько уменьшится заряд конденсатора, ответ выразите в микрокулонах.
}
\answer{%
    $q = C \cdot \Delta U = 1000\,\text{мкКл}$
}
\solutionspace{80pt}

\tasknumber{5}%
\task{%
    Определите ёмкость конденсатора, если при его зарядке до напряжения
    $V = 50\,\text{кВ}$ он приобретает заряд $Q = 25\,\text{мКл}$.
    % Чему при этом равны заряды обкладок конденсатора (сделайте рисунок и укажите их)?
    Ответ выразите в нанофарадах.
}
\answer{%
    $
        Q = CV \implies
        C = \frac{Q}{V} = \frac{25\,\text{мКл}}{50\,\text{кВ}} = 500\,\text{нФ}.
        \text{ Заряды обкладок: $Q$ и $-Q$}
    $
}
\solutionspace{120pt}

\tasknumber{6}%
\task{%
    Как и во сколько раз изменится ёмкость плоского конденсатора
    при уменьшении площади пластин в 2 раз
    и уменьшении расстояния между ними в 5 раз?
    В ответе укажите простую дробь или число — отношение новой ёмкости к старой.
}
\answer{%
    $
        \frac{C'}C
            = \frac{\eps_0\eps \frac S2}{\frac d5} \Big/ \frac{\eps_0\eps S}d
            = \frac{5}{2} = > 1 \implies \text{увеличится в $\frac52$ раз}
    $
}
\solutionspace{80pt}

\tasknumber{7}%
\task{%
    Электрическая ёмкость конденсатора равна $C = 200\,\text{пФ}$,
    при этом ему сообщён заряд $Q = 300\,\text{нКл}$.
    Какова энергия заряженного конденсатора?
    Ответ выразите в микроджоулях и округлите до целого.
}
\answer{%
    $
        W
        = \frac{Q^2}{2C}
        = \frac{\sqr{300\,\text{нКл}}}{2 \cdot 200\,\text{пФ}}
        = 225{,}00\,\text{мкДж}
    $
}

\variantsplitter

\addpersonalvariant{Леонид Никитин}

\tasknumber{1}%
\task{%
    Установите соответствие и запишите в ответ набор цифр (без других символов).

    А) разность потенциалов, Б) напряжённость электрического поля, В) электрический заряд.

    1) кулон, 2) вольт, 3) ватт, 4) генри, 5) Н / Кл.
}
\answer{%
    $251$
}
\solutionspace{40pt}

\tasknumber{2}%
\task{%
    Установите соответствие и запишите в ответ набор цифр (без других символов).

    А) энергия заряженного конденсатора, Б) разность потенциалов, В) напряжённость электрического поля.

    1) $E$, 2) $W$, 3) $Q$, 4) $\eps$, 5) $U$.
}
\answer{%
    $251$
}
\solutionspace{40pt}

\tasknumber{3}%
\task{%
    На конденсаторе указано: $C = 100\,\text{пФ}$, $V = 400\,\text{В}$.
    Удастся ли его использовать для накопления заряда $q = 60\,\text{нКл}$?
    (в ответе укажите «да» или «нет»)
}
\answer{%
    $
        q_{\text{max}} = CV = 100\,\text{пФ} \cdot 400\,\text{В} = 40\,\text{нКл}
        \implies q_{\text{max}}  <  q \implies \text{не удастся}
    $
}
\solutionspace{80pt}

\tasknumber{4}%
\task{%
    Конденсатор ёмкостью $20\,\text{нФ}$ был заряжен до напряжения $75\,\text{кВ}$.
    Затем напряжение уменьшают на $20\,\text{кВ}$.
    Определите на сколько уменьшится заряд конденсатора, ответ выразите в микрокулонах.
}
\answer{%
    $q = C \cdot \Delta U = 400\,\text{мкКл}$
}
\solutionspace{80pt}

\tasknumber{5}%
\task{%
    Определите ёмкость конденсатора, если при его зарядке до напряжения
    $U = 20\,\text{кВ}$ он приобретает заряд $q = 18\,\text{мКл}$.
    % Чему при этом равны заряды обкладок конденсатора (сделайте рисунок и укажите их)?
    Ответ выразите в нанофарадах.
}
\answer{%
    $
        q = CU \implies
        C = \frac{q}{U} = \frac{18\,\text{мКл}}{20\,\text{кВ}} = 900\,\text{нФ}.
        \text{ Заряды обкладок: $q$ и $-q$}
    $
}
\solutionspace{120pt}

\tasknumber{6}%
\task{%
    Как и во сколько раз изменится ёмкость плоского конденсатора
    при уменьшении площади пластин в 5 раз
    и уменьшении расстояния между ними в 4 раз?
    В ответе укажите простую дробь или число — отношение новой ёмкости к старой.
}
\answer{%
    $
        \frac{C'}C
            = \frac{\eps_0\eps \frac S5}{\frac d4} \Big/ \frac{\eps_0\eps S}d
            = \frac{4}{5} = < 1 \implies \text{уменьшится в $\frac54$ раз}
    $
}
\solutionspace{80pt}

\tasknumber{7}%
\task{%
    Электрическая ёмкость конденсатора равна $C = 200\,\text{пФ}$,
    при этом ему сообщён заряд $Q = 800\,\text{нКл}$.
    Какова энергия заряженного конденсатора?
    Ответ выразите в микроджоулях и округлите до целого.
}
\answer{%
    $
        W
        = \frac{Q^2}{2C}
        = \frac{\sqr{800\,\text{нКл}}}{2 \cdot 200\,\text{пФ}}
        = 1600{,}00\,\text{мкДж}
    $
}

\variantsplitter

\addpersonalvariant{Тимофей Полетаев}

\tasknumber{1}%
\task{%
    Установите соответствие и запишите в ответ набор цифр (без других символов).

    А) электроёмкость, Б) энергия заряженного конденсатора, В) разность потенциалов.

    1) ватт, 2) джоуль, 3) вольт, 4) кулон, 5) фарад.
}
\answer{%
    $523$
}
\solutionspace{40pt}

\tasknumber{2}%
\task{%
    Установите соответствие и запишите в ответ набор цифр (без других символов).

    А) электрический заряд, Б) напряжённость электрического поля, В) энергия заряженного конденсатора.

    1) $\eps$, 2) $E$, 3) $W$, 4) $r$, 5) $Q$.
}
\answer{%
    $523$
}
\solutionspace{40pt}

\tasknumber{3}%
\task{%
    На конденсаторе указано: $C = 150\,\text{пФ}$, $U = 400\,\text{В}$.
    Удастся ли его использовать для накопления заряда $q = 60\,\text{нКл}$?
    (в ответе укажите «да» или «нет»)
}
\answer{%
    $
        q_{\text{max}} = CU = 150\,\text{пФ} \cdot 400\,\text{В} = 60\,\text{нКл}
        \implies q_{\text{max}} \ge q \implies \text{удастся}
    $
}
\solutionspace{80pt}

\tasknumber{4}%
\task{%
    Конденсатор ёмкостью $20\,\text{нФ}$ был заряжен до напряжения $45\,\text{кВ}$.
    Затем напряжение уменьшают на $20\,\text{кВ}$.
    Определите на сколько уменьшится заряд конденсатора, ответ выразите в микрокулонах.
}
\answer{%
    $q = C \cdot \Delta U = 400\,\text{мкКл}$
}
\solutionspace{80pt}

\tasknumber{5}%
\task{%
    Определите ёмкость конденсатора, если при его зарядке до напряжения
    $V = 40\,\text{кВ}$ он приобретает заряд $Q = 15\,\text{мКл}$.
    % Чему при этом равны заряды обкладок конденсатора (сделайте рисунок и укажите их)?
    Ответ выразите в нанофарадах.
}
\answer{%
    $
        Q = CV \implies
        C = \frac{Q}{V} = \frac{15\,\text{мКл}}{40\,\text{кВ}} = 375\,\text{нФ}.
        \text{ Заряды обкладок: $Q$ и $-Q$}
    $
}
\solutionspace{120pt}

\tasknumber{6}%
\task{%
    Как и во сколько раз изменится ёмкость плоского конденсатора
    при уменьшении площади пластин в 8 раз
    и уменьшении расстояния между ними в 7 раз?
    В ответе укажите простую дробь или число — отношение новой ёмкости к старой.
}
\answer{%
    $
        \frac{C'}C
            = \frac{\eps_0\eps \frac S8}{\frac d7} \Big/ \frac{\eps_0\eps S}d
            = \frac{7}{8} = < 1 \implies \text{уменьшится в $\frac87$ раз}
    $
}
\solutionspace{80pt}

\tasknumber{7}%
\task{%
    Электрическая ёмкость конденсатора равна $C = 750\,\text{пФ}$,
    при этом ему сообщён заряд $q = 800\,\text{нКл}$.
    Какова энергия заряженного конденсатора?
    Ответ выразите в микроджоулях и округлите до целого.
}
\answer{%
    $
        W
        = \frac{q^2}{2C}
        = \frac{\sqr{800\,\text{нКл}}}{2 \cdot 750\,\text{пФ}}
        = 426{,}67\,\text{мкДж}
    $
}

\variantsplitter

\addpersonalvariant{Андрей Рожков}

\tasknumber{1}%
\task{%
    Установите соответствие и запишите в ответ набор цифр (без других символов).

    А) напряжённость электрического поля, Б) разность потенциалов, В) электрический заряд.

    1) кулон, 2) вольт, 3) Н / Кл, 4) джоуль, 5) фарад.
}
\answer{%
    $321$
}
\solutionspace{40pt}

\tasknumber{2}%
\task{%
    Установите соответствие и запишите в ответ набор цифр (без других символов).

    А) напряжённость электрического поля, Б) электроёмкость, В) электрическая постоянная.

    1) $\eps_0$, 2) $C$, 3) $E$, 4) $\eps$, 5) $r$.
}
\answer{%
    $321$
}
\solutionspace{40pt}

\tasknumber{3}%
\task{%
    На конденсаторе указано: $C = 100\,\text{пФ}$, $V = 300\,\text{В}$.
    Удастся ли его использовать для накопления заряда $Q = 60\,\text{нКл}$?
    (в ответе укажите «да» или «нет»)
}
\answer{%
    $
        Q_{\text{max}} = CV = 100\,\text{пФ} \cdot 300\,\text{В} = 30\,\text{нКл}
        \implies Q_{\text{max}}  <  Q \implies \text{не удастся}
    $
}
\solutionspace{80pt}

\tasknumber{4}%
\task{%
    Конденсатор ёмкостью $20\,\text{нФ}$ был заряжен до напряжения $75\,\text{кВ}$.
    Затем напряжение уменьшают до $10\,\text{кВ}$.
    Определите на сколько уменьшится заряд конденсатора, ответ выразите в микрокулонах.
}
\answer{%
    $q = C \cdot \Delta U = 1300\,\text{мкКл}$
}
\solutionspace{80pt}

\tasknumber{5}%
\task{%
    Определите ёмкость конденсатора, если при его зарядке до напряжения
    $V = 50\,\text{кВ}$ он приобретает заряд $Q = 18\,\text{мКл}$.
    % Чему при этом равны заряды обкладок конденсатора (сделайте рисунок и укажите их)?
    Ответ выразите в нанофарадах.
}
\answer{%
    $
        Q = CV \implies
        C = \frac{Q}{V} = \frac{18\,\text{мКл}}{50\,\text{кВ}} = 360\,\text{нФ}.
        \text{ Заряды обкладок: $Q$ и $-Q$}
    $
}
\solutionspace{120pt}

\tasknumber{6}%
\task{%
    Как и во сколько раз изменится ёмкость плоского конденсатора
    при уменьшении площади пластин в 5 раз
    и уменьшении расстояния между ними в 5 раз?
    В ответе укажите простую дробь или число — отношение новой ёмкости к старой.
}
\answer{%
    $
        \frac{C'}C
            = \frac{\eps_0\eps \frac S5}{\frac d5} \Big/ \frac{\eps_0\eps S}d
            = \frac{5}{5} = = 1 \implies \text{не изменится}
    $
}
\solutionspace{80pt}

\tasknumber{7}%
\task{%
    Электрическая ёмкость конденсатора равна $C = 750\,\text{пФ}$,
    при этом ему сообщён заряд $Q = 500\,\text{нКл}$.
    Какова энергия заряженного конденсатора?
    Ответ выразите в микроджоулях и округлите до целого.
}
\answer{%
    $
        W
        = \frac{Q^2}{2C}
        = \frac{\sqr{500\,\text{нКл}}}{2 \cdot 750\,\text{пФ}}
        = 166{,}67\,\text{мкДж}
    $
}

\variantsplitter

\addpersonalvariant{Рената Таржиманова}

\tasknumber{1}%
\task{%
    Установите соответствие и запишите в ответ набор цифр (без других символов).

    А) электрический заряд, Б) разность потенциалов, В) электроёмкость.

    1) кулон, 2) фарад, 3) вольт, 4) генри, 5) ампер.
}
\answer{%
    $132$
}
\solutionspace{40pt}

\tasknumber{2}%
\task{%
    Установите соответствие и запишите в ответ набор цифр (без других символов).

    А) энергия заряженного конденсатора, Б) напряжённость электрического поля, В) электрический заряд.

    1) $W$, 2) $Q$, 3) $E$, 4) $\eps_0$, 5) $U$.
}
\answer{%
    $132$
}
\solutionspace{40pt}

\tasknumber{3}%
\task{%
    На конденсаторе указано: $C = 100\,\text{пФ}$, $V = 450\,\text{В}$.
    Удастся ли его использовать для накопления заряда $Q = 50\,\text{нКл}$?
    (в ответе укажите «да» или «нет»)
}
\answer{%
    $
        Q_{\text{max}} = CV = 100\,\text{пФ} \cdot 450\,\text{В} = 45\,\text{нКл}
        \implies Q_{\text{max}}  <  Q \implies \text{не удастся}
    $
}
\solutionspace{80pt}

\tasknumber{4}%
\task{%
    Конденсатор ёмкостью $20\,\text{нФ}$ был заряжен до напряжения $65\,\text{кВ}$.
    Затем напряжение уменьшают до $10\,\text{кВ}$.
    Определите на сколько уменьшится заряд конденсатора, ответ выразите в микрокулонах.
}
\answer{%
    $q = C \cdot \Delta U = 1100\,\text{мкКл}$
}
\solutionspace{80pt}

\tasknumber{5}%
\task{%
    Определите ёмкость конденсатора, если при его зарядке до напряжения
    $V = 4\,\text{кВ}$ он приобретает заряд $q = 18\,\text{мКл}$.
    % Чему при этом равны заряды обкладок конденсатора (сделайте рисунок и укажите их)?
    Ответ выразите в нанофарадах.
}
\answer{%
    $
        q = CV \implies
        C = \frac{q}{V} = \frac{18\,\text{мКл}}{4\,\text{кВ}} = 4500\,\text{нФ}.
        \text{ Заряды обкладок: $q$ и $-q$}
    $
}
\solutionspace{120pt}

\tasknumber{6}%
\task{%
    Как и во сколько раз изменится ёмкость плоского конденсатора
    при уменьшении площади пластин в 8 раз
    и уменьшении расстояния между ними в 7 раз?
    В ответе укажите простую дробь или число — отношение новой ёмкости к старой.
}
\answer{%
    $
        \frac{C'}C
            = \frac{\eps_0\eps \frac S8}{\frac d7} \Big/ \frac{\eps_0\eps S}d
            = \frac{7}{8} = < 1 \implies \text{уменьшится в $\frac87$ раз}
    $
}
\solutionspace{80pt}

\tasknumber{7}%
\task{%
    Электрическая ёмкость конденсатора равна $C = 400\,\text{пФ}$,
    при этом ему сообщён заряд $Q = 500\,\text{нКл}$.
    Какова энергия заряженного конденсатора?
    Ответ выразите в микроджоулях и округлите до целого.
}
\answer{%
    $
        W
        = \frac{Q^2}{2C}
        = \frac{\sqr{500\,\text{нКл}}}{2 \cdot 400\,\text{пФ}}
        = 312{,}50\,\text{мкДж}
    $
}

\variantsplitter

\addpersonalvariant{Андрей Щербаков}

\tasknumber{1}%
\task{%
    Установите соответствие и запишите в ответ набор цифр (без других символов).

    А) электроёмкость, Б) напряжённость электрического поля, В) электрический заряд.

    1) ампер, 2) кулон, 3) ватт, 4) Н / Кл, 5) фарад.
}
\answer{%
    $542$
}
\solutionspace{40pt}

\tasknumber{2}%
\task{%
    Установите соответствие и запишите в ответ набор цифр (без других символов).

    А) электроёмкость, Б) напряжённость электрического поля, В) электрический заряд.

    1) $U$, 2) $Q$, 3) $W$, 4) $E$, 5) $C$.
}
\answer{%
    $542$
}
\solutionspace{40pt}

\tasknumber{3}%
\task{%
    На конденсаторе указано: $C = 150\,\text{пФ}$, $U = 200\,\text{В}$.
    Удастся ли его использовать для накопления заряда $Q = 50\,\text{нКл}$?
    (в ответе укажите «да» или «нет»)
}
\answer{%
    $
        Q_{\text{max}} = CU = 150\,\text{пФ} \cdot 200\,\text{В} = 30\,\text{нКл}
        \implies Q_{\text{max}}  <  Q \implies \text{не удастся}
    $
}
\solutionspace{80pt}

\tasknumber{4}%
\task{%
    Конденсатор ёмкостью $20\,\text{нФ}$ был заряжен до напряжения $75\,\text{кВ}$.
    Затем напряжение уменьшают до $30\,\text{кВ}$.
    Определите на сколько уменьшится заряд конденсатора, ответ выразите в микрокулонах.
}
\answer{%
    $q = C \cdot \Delta U = 900\,\text{мкКл}$
}
\solutionspace{80pt}

\tasknumber{5}%
\task{%
    Определите ёмкость конденсатора, если при его зарядке до напряжения
    $V = 20\,\text{кВ}$ он приобретает заряд $q = 4\,\text{мКл}$.
    % Чему при этом равны заряды обкладок конденсатора (сделайте рисунок и укажите их)?
    Ответ выразите в нанофарадах.
}
\answer{%
    $
        q = CV \implies
        C = \frac{q}{V} = \frac{4\,\text{мКл}}{20\,\text{кВ}} = 200\,\text{нФ}.
        \text{ Заряды обкладок: $q$ и $-q$}
    $
}
\solutionspace{120pt}

\tasknumber{6}%
\task{%
    Как и во сколько раз изменится ёмкость плоского конденсатора
    при уменьшении площади пластин в 2 раз
    и уменьшении расстояния между ними в 7 раз?
    В ответе укажите простую дробь или число — отношение новой ёмкости к старой.
}
\answer{%
    $
        \frac{C'}C
            = \frac{\eps_0\eps \frac S2}{\frac d7} \Big/ \frac{\eps_0\eps S}d
            = \frac{7}{2} = > 1 \implies \text{увеличится в $\frac72$ раз}
    $
}
\solutionspace{80pt}

\tasknumber{7}%
\task{%
    Электрическая ёмкость конденсатора равна $C = 200\,\text{пФ}$,
    при этом ему сообщён заряд $q = 900\,\text{нКл}$.
    Какова энергия заряженного конденсатора?
    Ответ выразите в микроджоулях и округлите до целого.
}
\answer{%
    $
        W
        = \frac{q^2}{2C}
        = \frac{\sqr{900\,\text{нКл}}}{2 \cdot 200\,\text{пФ}}
        = 2025{,}00\,\text{мкДж}
    $
}

\variantsplitter

\addpersonalvariant{Михаил Ярошевский}

\tasknumber{1}%
\task{%
    Установите соответствие и запишите в ответ набор цифр (без других символов).

    А) электрический заряд, Б) энергия заряженного конденсатора, В) напряжённость электрического поля.

    1) ватт, 2) джоуль, 3) кулон, 4) Н / Кл, 5) вольт.
}
\answer{%
    $324$
}
\solutionspace{40pt}

\tasknumber{2}%
\task{%
    Установите соответствие и запишите в ответ набор цифр (без других символов).

    А) энергия заряженного конденсатора, Б) электроёмкость, В) электрическая постоянная.

    1) $Q$, 2) $C$, 3) $W$, 4) $\eps_0$, 5) $U$.
}
\answer{%
    $324$
}
\solutionspace{40pt}

\tasknumber{3}%
\task{%
    На конденсаторе указано: $C = 100\,\text{пФ}$, $V = 400\,\text{В}$.
    Удастся ли его использовать для накопления заряда $Q = 50\,\text{нКл}$?
    (в ответе укажите «да» или «нет»)
}
\answer{%
    $
        Q_{\text{max}} = CV = 100\,\text{пФ} \cdot 400\,\text{В} = 40\,\text{нКл}
        \implies Q_{\text{max}}  <  Q \implies \text{не удастся}
    $
}
\solutionspace{80pt}

\tasknumber{4}%
\task{%
    Конденсатор ёмкостью $30\,\text{нФ}$ был заряжен до напряжения $45\,\text{кВ}$.
    Затем напряжение уменьшают до $10\,\text{кВ}$.
    Определите на сколько уменьшится заряд конденсатора, ответ выразите в микрокулонах.
}
\answer{%
    $q = C \cdot \Delta U = 1050\,\text{мкКл}$
}
\solutionspace{80pt}

\tasknumber{5}%
\task{%
    Определите ёмкость конденсатора, если при его зарядке до напряжения
    $U = 4\,\text{кВ}$ он приобретает заряд $Q = 18\,\text{мКл}$.
    % Чему при этом равны заряды обкладок конденсатора (сделайте рисунок и укажите их)?
    Ответ выразите в нанофарадах.
}
\answer{%
    $
        Q = CU \implies
        C = \frac{Q}{U} = \frac{18\,\text{мКл}}{4\,\text{кВ}} = 4500\,\text{нФ}.
        \text{ Заряды обкладок: $Q$ и $-Q$}
    $
}
\solutionspace{120pt}

\tasknumber{6}%
\task{%
    Как и во сколько раз изменится ёмкость плоского конденсатора
    при уменьшении площади пластин в 2 раз
    и уменьшении расстояния между ними в 2 раз?
    В ответе укажите простую дробь или число — отношение новой ёмкости к старой.
}
\answer{%
    $
        \frac{C'}C
            = \frac{\eps_0\eps \frac S2}{\frac d2} \Big/ \frac{\eps_0\eps S}d
            = \frac{2}{2} = = 1 \implies \text{не изменится}
    $
}
\solutionspace{80pt}

\tasknumber{7}%
\task{%
    Электрическая ёмкость конденсатора равна $C = 600\,\text{пФ}$,
    при этом ему сообщён заряд $q = 900\,\text{нКл}$.
    Какова энергия заряженного конденсатора?
    Ответ выразите в микроджоулях и округлите до целого.
}
\answer{%
    $
        W
        = \frac{q^2}{2C}
        = \frac{\sqr{900\,\text{нКл}}}{2 \cdot 600\,\text{пФ}}
        = 675{,}00\,\text{мкДж}
    $
}

\variantsplitter

\addpersonalvariant{Алексей Алимпиев}

\tasknumber{1}%
\task{%
    Установите соответствие и запишите в ответ набор цифр (без других символов).

    А) электрический заряд, Б) напряжённость электрического поля, В) разность потенциалов.

    1) кулон, 2) вольт, 3) ампер, 4) фарад, 5) Н / Кл.
}
\answer{%
    $152$
}
\solutionspace{40pt}

\tasknumber{2}%
\task{%
    Установите соответствие и запишите в ответ набор цифр (без других символов).

    А) энергия заряженного конденсатора, Б) электроёмкость, В) электрический заряд.

    1) $W$, 2) $Q$, 3) $E$, 4) $r$, 5) $C$.
}
\answer{%
    $152$
}
\solutionspace{40pt}

\tasknumber{3}%
\task{%
    На конденсаторе указано: $C = 50\,\text{пФ}$, $U = 300\,\text{В}$.
    Удастся ли его использовать для накопления заряда $Q = 60\,\text{нКл}$?
    (в ответе укажите «да» или «нет»)
}
\answer{%
    $
        Q_{\text{max}} = CU = 50\,\text{пФ} \cdot 300\,\text{В} = 15\,\text{нКл}
        \implies Q_{\text{max}}  <  Q \implies \text{не удастся}
    $
}
\solutionspace{80pt}

\tasknumber{4}%
\task{%
    Конденсатор ёмкостью $20\,\text{нФ}$ был заряжен до напряжения $45\,\text{кВ}$.
    Затем напряжение уменьшают на $10\,\text{кВ}$.
    Определите на сколько уменьшится заряд конденсатора, ответ выразите в микрокулонах.
}
\answer{%
    $q = C \cdot \Delta U = 200\,\text{мкКл}$
}
\solutionspace{80pt}

\tasknumber{5}%
\task{%
    Определите ёмкость конденсатора, если при его зарядке до напряжения
    $U = 2\,\text{кВ}$ он приобретает заряд $Q = 15\,\text{мКл}$.
    % Чему при этом равны заряды обкладок конденсатора (сделайте рисунок и укажите их)?
    Ответ выразите в нанофарадах.
}
\answer{%
    $
        Q = CU \implies
        C = \frac{Q}{U} = \frac{15\,\text{мКл}}{2\,\text{кВ}} = 7500\,\text{нФ}.
        \text{ Заряды обкладок: $Q$ и $-Q$}
    $
}
\solutionspace{120pt}

\tasknumber{6}%
\task{%
    Как и во сколько раз изменится ёмкость плоского конденсатора
    при уменьшении площади пластин в 4 раз
    и уменьшении расстояния между ними в 5 раз?
    В ответе укажите простую дробь или число — отношение новой ёмкости к старой.
}
\answer{%
    $
        \frac{C'}C
            = \frac{\eps_0\eps \frac S4}{\frac d5} \Big/ \frac{\eps_0\eps S}d
            = \frac{5}{4} = > 1 \implies \text{увеличится в $\frac54$ раз}
    $
}
\solutionspace{80pt}

\tasknumber{7}%
\task{%
    Электрическая ёмкость конденсатора равна $C = 200\,\text{пФ}$,
    при этом ему сообщён заряд $Q = 300\,\text{нКл}$.
    Какова энергия заряженного конденсатора?
    Ответ выразите в микроджоулях и округлите до целого.
}
\answer{%
    $
        W
        = \frac{Q^2}{2C}
        = \frac{\sqr{300\,\text{нКл}}}{2 \cdot 200\,\text{пФ}}
        = 225{,}00\,\text{мкДж}
    $
}

\variantsplitter

\addpersonalvariant{Евгений Васин}

\tasknumber{1}%
\task{%
    Установите соответствие и запишите в ответ набор цифр (без других символов).

    А) энергия заряженного конденсатора, Б) электрический заряд, В) напряжённость электрического поля.

    1) Н / Кл, 2) кулон, 3) генри, 4) ампер, 5) джоуль.
}
\answer{%
    $521$
}
\solutionspace{40pt}

\tasknumber{2}%
\task{%
    Установите соответствие и запишите в ответ набор цифр (без других символов).

    А) напряжённость электрического поля, Б) электрический заряд, В) электрическая постоянная.

    1) $\eps_0$, 2) $Q$, 3) $\eps$, 4) $C$, 5) $E$.
}
\answer{%
    $521$
}
\solutionspace{40pt}

\tasknumber{3}%
\task{%
    На конденсаторе указано: $C = 150\,\text{пФ}$, $V = 400\,\text{В}$.
    Удастся ли его использовать для накопления заряда $q = 50\,\text{нКл}$?
    (в ответе укажите «да» или «нет»)
}
\answer{%
    $
        q_{\text{max}} = CV = 150\,\text{пФ} \cdot 400\,\text{В} = 60\,\text{нКл}
        \implies q_{\text{max}} \ge q \implies \text{удастся}
    $
}
\solutionspace{80pt}

\tasknumber{4}%
\task{%
    Конденсатор ёмкостью $30\,\text{нФ}$ был заряжен до напряжения $75\,\text{кВ}$.
    Затем напряжение уменьшают на $10\,\text{кВ}$.
    Определите на сколько уменьшится заряд конденсатора, ответ выразите в микрокулонах.
}
\answer{%
    $q = C \cdot \Delta U = 300\,\text{мкКл}$
}
\solutionspace{80pt}

\tasknumber{5}%
\task{%
    Определите ёмкость конденсатора, если при его зарядке до напряжения
    $U = 40\,\text{кВ}$ он приобретает заряд $q = 6\,\text{мКл}$.
    % Чему при этом равны заряды обкладок конденсатора (сделайте рисунок и укажите их)?
    Ответ выразите в нанофарадах.
}
\answer{%
    $
        q = CU \implies
        C = \frac{q}{U} = \frac{6\,\text{мКл}}{40\,\text{кВ}} = 150\,\text{нФ}.
        \text{ Заряды обкладок: $q$ и $-q$}
    $
}
\solutionspace{120pt}

\tasknumber{6}%
\task{%
    Как и во сколько раз изменится ёмкость плоского конденсатора
    при уменьшении площади пластин в 8 раз
    и уменьшении расстояния между ними в 3 раз?
    В ответе укажите простую дробь или число — отношение новой ёмкости к старой.
}
\answer{%
    $
        \frac{C'}C
            = \frac{\eps_0\eps \frac S8}{\frac d3} \Big/ \frac{\eps_0\eps S}d
            = \frac{3}{8} = < 1 \implies \text{уменьшится в $\frac83$ раз}
    $
}
\solutionspace{80pt}

\tasknumber{7}%
\task{%
    Электрическая ёмкость конденсатора равна $C = 200\,\text{пФ}$,
    при этом ему сообщён заряд $q = 900\,\text{нКл}$.
    Какова энергия заряженного конденсатора?
    Ответ выразите в микроджоулях и округлите до целого.
}
\answer{%
    $
        W
        = \frac{q^2}{2C}
        = \frac{\sqr{900\,\text{нКл}}}{2 \cdot 200\,\text{пФ}}
        = 2025{,}00\,\text{мкДж}
    $
}

\variantsplitter

\addpersonalvariant{Вячеслав Волохов}

\tasknumber{1}%
\task{%
    Установите соответствие и запишите в ответ набор цифр (без других символов).

    А) электрический заряд, Б) напряжённость электрического поля, В) разность потенциалов.

    1) кулон, 2) вольт, 3) ватт, 4) джоуль, 5) Н / Кл.
}
\answer{%
    $152$
}
\solutionspace{40pt}

\tasknumber{2}%
\task{%
    Установите соответствие и запишите в ответ набор цифр (без других символов).

    А) энергия заряженного конденсатора, Б) электроёмкость, В) электрический заряд.

    1) $W$, 2) $Q$, 3) $U$, 4) $\eps_0$, 5) $C$.
}
\answer{%
    $152$
}
\solutionspace{40pt}

\tasknumber{3}%
\task{%
    На конденсаторе указано: $C = 100\,\text{пФ}$, $V = 450\,\text{В}$.
    Удастся ли его использовать для накопления заряда $q = 30\,\text{нКл}$?
    (в ответе укажите «да» или «нет»)
}
\answer{%
    $
        q_{\text{max}} = CV = 100\,\text{пФ} \cdot 450\,\text{В} = 45\,\text{нКл}
        \implies q_{\text{max}} \ge q \implies \text{удастся}
    $
}
\solutionspace{80pt}

\tasknumber{4}%
\task{%
    Конденсатор ёмкостью $40\,\text{нФ}$ был заряжен до напряжения $55\,\text{кВ}$.
    Затем напряжение уменьшают до $30\,\text{кВ}$.
    Определите на сколько уменьшится заряд конденсатора, ответ выразите в микрокулонах.
}
\answer{%
    $q = C \cdot \Delta U = 1000\,\text{мкКл}$
}
\solutionspace{80pt}

\tasknumber{5}%
\task{%
    Определите ёмкость конденсатора, если при его зарядке до напряжения
    $V = 4\,\text{кВ}$ он приобретает заряд $Q = 24\,\text{мКл}$.
    % Чему при этом равны заряды обкладок конденсатора (сделайте рисунок и укажите их)?
    Ответ выразите в нанофарадах.
}
\answer{%
    $
        Q = CV \implies
        C = \frac{Q}{V} = \frac{24\,\text{мКл}}{4\,\text{кВ}} = 6000\,\text{нФ}.
        \text{ Заряды обкладок: $Q$ и $-Q$}
    $
}
\solutionspace{120pt}

\tasknumber{6}%
\task{%
    Как и во сколько раз изменится ёмкость плоского конденсатора
    при уменьшении площади пластин в 7 раз
    и уменьшении расстояния между ними в 3 раз?
    В ответе укажите простую дробь или число — отношение новой ёмкости к старой.
}
\answer{%
    $
        \frac{C'}C
            = \frac{\eps_0\eps \frac S7}{\frac d3} \Big/ \frac{\eps_0\eps S}d
            = \frac{3}{7} = < 1 \implies \text{уменьшится в $\frac73$ раз}
    $
}
\solutionspace{80pt}

\tasknumber{7}%
\task{%
    Электрическая ёмкость конденсатора равна $C = 750\,\text{пФ}$,
    при этом ему сообщён заряд $Q = 500\,\text{нКл}$.
    Какова энергия заряженного конденсатора?
    Ответ выразите в микроджоулях и округлите до целого.
}
\answer{%
    $
        W
        = \frac{Q^2}{2C}
        = \frac{\sqr{500\,\text{нКл}}}{2 \cdot 750\,\text{пФ}}
        = 166{,}67\,\text{мкДж}
    $
}

\variantsplitter

\addpersonalvariant{Герман Говоров}

\tasknumber{1}%
\task{%
    Установите соответствие и запишите в ответ набор цифр (без других символов).

    А) электрический заряд, Б) энергия заряженного конденсатора, В) разность потенциалов.

    1) кулон, 2) вольт, 3) ватт, 4) генри, 5) джоуль.
}
\answer{%
    $152$
}
\solutionspace{40pt}

\tasknumber{2}%
\task{%
    Установите соответствие и запишите в ответ набор цифр (без других символов).

    А) электрическая постоянная, Б) напряжённость электрического поля, В) электроёмкость.

    1) $\eps_0$, 2) $C$, 3) $W$, 4) $\eps$, 5) $E$.
}
\answer{%
    $152$
}
\solutionspace{40pt}

\tasknumber{3}%
\task{%
    На конденсаторе указано: $C = 100\,\text{пФ}$, $U = 400\,\text{В}$.
    Удастся ли его использовать для накопления заряда $q = 50\,\text{нКл}$?
    (в ответе укажите «да» или «нет»)
}
\answer{%
    $
        q_{\text{max}} = CU = 100\,\text{пФ} \cdot 400\,\text{В} = 40\,\text{нКл}
        \implies q_{\text{max}}  <  q \implies \text{не удастся}
    $
}
\solutionspace{80pt}

\tasknumber{4}%
\task{%
    Конденсатор ёмкостью $40\,\text{нФ}$ был заряжен до напряжения $75\,\text{кВ}$.
    Затем напряжение уменьшают на $20\,\text{кВ}$.
    Определите на сколько уменьшится заряд конденсатора, ответ выразите в микрокулонах.
}
\answer{%
    $q = C \cdot \Delta U = 800\,\text{мкКл}$
}
\solutionspace{80pt}

\tasknumber{5}%
\task{%
    Определите ёмкость конденсатора, если при его зарядке до напряжения
    $U = 2\,\text{кВ}$ он приобретает заряд $Q = 18\,\text{мКл}$.
    % Чему при этом равны заряды обкладок конденсатора (сделайте рисунок и укажите их)?
    Ответ выразите в нанофарадах.
}
\answer{%
    $
        Q = CU \implies
        C = \frac{Q}{U} = \frac{18\,\text{мКл}}{2\,\text{кВ}} = 9000\,\text{нФ}.
        \text{ Заряды обкладок: $Q$ и $-Q$}
    $
}
\solutionspace{120pt}

\tasknumber{6}%
\task{%
    Как и во сколько раз изменится ёмкость плоского конденсатора
    при уменьшении площади пластин в 6 раз
    и уменьшении расстояния между ними в 4 раз?
    В ответе укажите простую дробь или число — отношение новой ёмкости к старой.
}
\answer{%
    $
        \frac{C'}C
            = \frac{\eps_0\eps \frac S6}{\frac d4} \Big/ \frac{\eps_0\eps S}d
            = \frac{4}{6} = < 1 \implies \text{уменьшится в $\frac32$ раз}
    $
}
\solutionspace{80pt}

\tasknumber{7}%
\task{%
    Электрическая ёмкость конденсатора равна $C = 750\,\text{пФ}$,
    при этом ему сообщён заряд $q = 800\,\text{нКл}$.
    Какова энергия заряженного конденсатора?
    Ответ выразите в микроджоулях и округлите до целого.
}
\answer{%
    $
        W
        = \frac{q^2}{2C}
        = \frac{\sqr{800\,\text{нКл}}}{2 \cdot 750\,\text{пФ}}
        = 426{,}67\,\text{мкДж}
    $
}

\variantsplitter

\addpersonalvariant{София Журавлёва}

\tasknumber{1}%
\task{%
    Установите соответствие и запишите в ответ набор цифр (без других символов).

    А) энергия заряженного конденсатора, Б) электроёмкость, В) электрический заряд.

    1) кулон, 2) генри, 3) фарад, 4) Н / Кл, 5) джоуль.
}
\answer{%
    $531$
}
\solutionspace{40pt}

\tasknumber{2}%
\task{%
    Установите соответствие и запишите в ответ набор цифр (без других символов).

    А) разность потенциалов, Б) напряжённость электрического поля, В) энергия заряженного конденсатора.

    1) $W$, 2) $r$, 3) $E$, 4) $C$, 5) $U$.
}
\answer{%
    $531$
}
\solutionspace{40pt}

\tasknumber{3}%
\task{%
    На конденсаторе указано: $C = 80\,\text{пФ}$, $U = 200\,\text{В}$.
    Удастся ли его использовать для накопления заряда $Q = 60\,\text{нКл}$?
    (в ответе укажите «да» или «нет»)
}
\answer{%
    $
        Q_{\text{max}} = CU = 80\,\text{пФ} \cdot 200\,\text{В} = 16\,\text{нКл}
        \implies Q_{\text{max}}  <  Q \implies \text{не удастся}
    $
}
\solutionspace{80pt}

\tasknumber{4}%
\task{%
    Конденсатор ёмкостью $40\,\text{нФ}$ был заряжен до напряжения $75\,\text{кВ}$.
    Затем напряжение уменьшают на $10\,\text{кВ}$.
    Определите на сколько уменьшится заряд конденсатора, ответ выразите в микрокулонах.
}
\answer{%
    $q = C \cdot \Delta U = 400\,\text{мкКл}$
}
\solutionspace{80pt}

\tasknumber{5}%
\task{%
    Определите ёмкость конденсатора, если при его зарядке до напряжения
    $V = 40\,\text{кВ}$ он приобретает заряд $q = 18\,\text{мКл}$.
    % Чему при этом равны заряды обкладок конденсатора (сделайте рисунок и укажите их)?
    Ответ выразите в нанофарадах.
}
\answer{%
    $
        q = CV \implies
        C = \frac{q}{V} = \frac{18\,\text{мКл}}{40\,\text{кВ}} = 450\,\text{нФ}.
        \text{ Заряды обкладок: $q$ и $-q$}
    $
}
\solutionspace{120pt}

\tasknumber{6}%
\task{%
    Как и во сколько раз изменится ёмкость плоского конденсатора
    при уменьшении площади пластин в 3 раз
    и уменьшении расстояния между ними в 4 раз?
    В ответе укажите простую дробь или число — отношение новой ёмкости к старой.
}
\answer{%
    $
        \frac{C'}C
            = \frac{\eps_0\eps \frac S3}{\frac d4} \Big/ \frac{\eps_0\eps S}d
            = \frac{4}{3} = > 1 \implies \text{увеличится в $\frac43$ раз}
    $
}
\solutionspace{80pt}

\tasknumber{7}%
\task{%
    Электрическая ёмкость конденсатора равна $C = 600\,\text{пФ}$,
    при этом ему сообщён заряд $Q = 900\,\text{нКл}$.
    Какова энергия заряженного конденсатора?
    Ответ выразите в микроджоулях и округлите до целого.
}
\answer{%
    $
        W
        = \frac{Q^2}{2C}
        = \frac{\sqr{900\,\text{нКл}}}{2 \cdot 600\,\text{пФ}}
        = 675{,}00\,\text{мкДж}
    $
}

\variantsplitter

\addpersonalvariant{Константин Козлов}

\tasknumber{1}%
\task{%
    Установите соответствие и запишите в ответ набор цифр (без других символов).

    А) энергия заряженного конденсатора, Б) электроёмкость, В) электрический заряд.

    1) фарад, 2) кулон, 3) генри, 4) вольт, 5) джоуль.
}
\answer{%
    $512$
}
\solutionspace{40pt}

\tasknumber{2}%
\task{%
    Установите соответствие и запишите в ответ набор цифр (без других символов).

    А) напряжённость электрического поля, Б) энергия заряженного конденсатора, В) электрический заряд.

    1) $W$, 2) $Q$, 3) $r$, 4) $\eps_0$, 5) $E$.
}
\answer{%
    $512$
}
\solutionspace{40pt}

\tasknumber{3}%
\task{%
    На конденсаторе указано: $C = 120\,\text{пФ}$, $U = 300\,\text{В}$.
    Удастся ли его использовать для накопления заряда $Q = 50\,\text{нКл}$?
    (в ответе укажите «да» или «нет»)
}
\answer{%
    $
        Q_{\text{max}} = CU = 120\,\text{пФ} \cdot 300\,\text{В} = 36\,\text{нКл}
        \implies Q_{\text{max}}  <  Q \implies \text{не удастся}
    $
}
\solutionspace{80pt}

\tasknumber{4}%
\task{%
    Конденсатор ёмкостью $40\,\text{нФ}$ был заряжен до напряжения $45\,\text{кВ}$.
    Затем напряжение уменьшают на $10\,\text{кВ}$.
    Определите на сколько уменьшится заряд конденсатора, ответ выразите в микрокулонах.
}
\answer{%
    $q = C \cdot \Delta U = 400\,\text{мкКл}$
}
\solutionspace{80pt}

\tasknumber{5}%
\task{%
    Определите ёмкость конденсатора, если при его зарядке до напряжения
    $V = 50\,\text{кВ}$ он приобретает заряд $Q = 24\,\text{мКл}$.
    % Чему при этом равны заряды обкладок конденсатора (сделайте рисунок и укажите их)?
    Ответ выразите в нанофарадах.
}
\answer{%
    $
        Q = CV \implies
        C = \frac{Q}{V} = \frac{24\,\text{мКл}}{50\,\text{кВ}} = 480\,\text{нФ}.
        \text{ Заряды обкладок: $Q$ и $-Q$}
    $
}
\solutionspace{120pt}

\tasknumber{6}%
\task{%
    Как и во сколько раз изменится ёмкость плоского конденсатора
    при уменьшении площади пластин в 6 раз
    и уменьшении расстояния между ними в 4 раз?
    В ответе укажите простую дробь или число — отношение новой ёмкости к старой.
}
\answer{%
    $
        \frac{C'}C
            = \frac{\eps_0\eps \frac S6}{\frac d4} \Big/ \frac{\eps_0\eps S}d
            = \frac{4}{6} = < 1 \implies \text{уменьшится в $\frac32$ раз}
    $
}
\solutionspace{80pt}

\tasknumber{7}%
\task{%
    Электрическая ёмкость конденсатора равна $C = 200\,\text{пФ}$,
    при этом ему сообщён заряд $q = 500\,\text{нКл}$.
    Какова энергия заряженного конденсатора?
    Ответ выразите в микроджоулях и округлите до целого.
}
\answer{%
    $
        W
        = \frac{q^2}{2C}
        = \frac{\sqr{500\,\text{нКл}}}{2 \cdot 200\,\text{пФ}}
        = 625{,}00\,\text{мкДж}
    $
}

\variantsplitter

\addpersonalvariant{Наталья Кравченко}

\tasknumber{1}%
\task{%
    Установите соответствие и запишите в ответ набор цифр (без других символов).

    А) энергия заряженного конденсатора, Б) электрический заряд, В) напряжённость электрического поля.

    1) джоуль, 2) ватт, 3) кулон, 4) Н / Кл, 5) фарад.
}
\answer{%
    $134$
}
\solutionspace{40pt}

\tasknumber{2}%
\task{%
    Установите соответствие и запишите в ответ набор цифр (без других символов).

    А) разность потенциалов, Б) энергия заряженного конденсатора, В) электрическая постоянная.

    1) $U$, 2) $E$, 3) $W$, 4) $\eps_0$, 5) $r$.
}
\answer{%
    $134$
}
\solutionspace{40pt}

\tasknumber{3}%
\task{%
    На конденсаторе указано: $C = 80\,\text{пФ}$, $U = 450\,\text{В}$.
    Удастся ли его использовать для накопления заряда $q = 50\,\text{нКл}$?
    (в ответе укажите «да» или «нет»)
}
\answer{%
    $
        q_{\text{max}} = CU = 80\,\text{пФ} \cdot 450\,\text{В} = 36\,\text{нКл}
        \implies q_{\text{max}}  <  q \implies \text{не удастся}
    $
}
\solutionspace{80pt}

\tasknumber{4}%
\task{%
    Конденсатор ёмкостью $30\,\text{нФ}$ был заряжен до напряжения $65\,\text{кВ}$.
    Затем напряжение уменьшают на $30\,\text{кВ}$.
    Определите на сколько уменьшится заряд конденсатора, ответ выразите в микрокулонах.
}
\answer{%
    $q = C \cdot \Delta U = 900\,\text{мкКл}$
}
\solutionspace{80pt}

\tasknumber{5}%
\task{%
    Определите ёмкость конденсатора, если при его зарядке до напряжения
    $V = 40\,\text{кВ}$ он приобретает заряд $q = 15\,\text{мКл}$.
    % Чему при этом равны заряды обкладок конденсатора (сделайте рисунок и укажите их)?
    Ответ выразите в нанофарадах.
}
\answer{%
    $
        q = CV \implies
        C = \frac{q}{V} = \frac{15\,\text{мКл}}{40\,\text{кВ}} = 375\,\text{нФ}.
        \text{ Заряды обкладок: $q$ и $-q$}
    $
}
\solutionspace{120pt}

\tasknumber{6}%
\task{%
    Как и во сколько раз изменится ёмкость плоского конденсатора
    при уменьшении площади пластин в 3 раз
    и уменьшении расстояния между ними в 7 раз?
    В ответе укажите простую дробь или число — отношение новой ёмкости к старой.
}
\answer{%
    $
        \frac{C'}C
            = \frac{\eps_0\eps \frac S3}{\frac d7} \Big/ \frac{\eps_0\eps S}d
            = \frac{7}{3} = > 1 \implies \text{увеличится в $\frac73$ раз}
    $
}
\solutionspace{80pt}

\tasknumber{7}%
\task{%
    Электрическая ёмкость конденсатора равна $C = 750\,\text{пФ}$,
    при этом ему сообщён заряд $Q = 900\,\text{нКл}$.
    Какова энергия заряженного конденсатора?
    Ответ выразите в микроджоулях и округлите до целого.
}
\answer{%
    $
        W
        = \frac{Q^2}{2C}
        = \frac{\sqr{900\,\text{нКл}}}{2 \cdot 750\,\text{пФ}}
        = 540{,}00\,\text{мкДж}
    $
}

\variantsplitter

\addpersonalvariant{Матвей Кузьмин}

\tasknumber{1}%
\task{%
    Установите соответствие и запишите в ответ набор цифр (без других символов).

    А) напряжённость электрического поля, Б) разность потенциалов, В) электроёмкость.

    1) вольт, 2) ватт, 3) джоуль, 4) фарад, 5) Н / Кл.
}
\answer{%
    $514$
}
\solutionspace{40pt}

\tasknumber{2}%
\task{%
    Установите соответствие и запишите в ответ набор цифр (без других символов).

    А) разность потенциалов, Б) электроёмкость, В) энергия заряженного конденсатора.

    1) $C$, 2) $Q$, 3) $\eps_0$, 4) $W$, 5) $U$.
}
\answer{%
    $514$
}
\solutionspace{40pt}

\tasknumber{3}%
\task{%
    На конденсаторе указано: $C = 150\,\text{пФ}$, $V = 450\,\text{В}$.
    Удастся ли его использовать для накопления заряда $Q = 50\,\text{нКл}$?
    (в ответе укажите «да» или «нет»)
}
\answer{%
    $
        Q_{\text{max}} = CV = 150\,\text{пФ} \cdot 450\,\text{В} = 67\,\text{нКл}
        \implies Q_{\text{max}} \ge Q \implies \text{удастся}
    $
}
\solutionspace{80pt}

\tasknumber{4}%
\task{%
    Конденсатор ёмкостью $20\,\text{нФ}$ был заряжен до напряжения $45\,\text{кВ}$.
    Затем напряжение уменьшают на $30\,\text{кВ}$.
    Определите на сколько уменьшится заряд конденсатора, ответ выразите в микрокулонах.
}
\answer{%
    $q = C \cdot \Delta U = 600\,\text{мкКл}$
}
\solutionspace{80pt}

\tasknumber{5}%
\task{%
    Определите ёмкость конденсатора, если при его зарядке до напряжения
    $U = 40\,\text{кВ}$ он приобретает заряд $q = 15\,\text{мКл}$.
    % Чему при этом равны заряды обкладок конденсатора (сделайте рисунок и укажите их)?
    Ответ выразите в нанофарадах.
}
\answer{%
    $
        q = CU \implies
        C = \frac{q}{U} = \frac{15\,\text{мКл}}{40\,\text{кВ}} = 375\,\text{нФ}.
        \text{ Заряды обкладок: $q$ и $-q$}
    $
}
\solutionspace{120pt}

\tasknumber{6}%
\task{%
    Как и во сколько раз изменится ёмкость плоского конденсатора
    при уменьшении площади пластин в 7 раз
    и уменьшении расстояния между ними в 4 раз?
    В ответе укажите простую дробь или число — отношение новой ёмкости к старой.
}
\answer{%
    $
        \frac{C'}C
            = \frac{\eps_0\eps \frac S7}{\frac d4} \Big/ \frac{\eps_0\eps S}d
            = \frac{4}{7} = < 1 \implies \text{уменьшится в $\frac74$ раз}
    $
}
\solutionspace{80pt}

\tasknumber{7}%
\task{%
    Электрическая ёмкость конденсатора равна $C = 400\,\text{пФ}$,
    при этом ему сообщён заряд $q = 900\,\text{нКл}$.
    Какова энергия заряженного конденсатора?
    Ответ выразите в микроджоулях и округлите до целого.
}
\answer{%
    $
        W
        = \frac{q^2}{2C}
        = \frac{\sqr{900\,\text{нКл}}}{2 \cdot 400\,\text{пФ}}
        = 1012{,}50\,\text{мкДж}
    $
}

\variantsplitter

\addpersonalvariant{Сергей Малышев}

\tasknumber{1}%
\task{%
    Установите соответствие и запишите в ответ набор цифр (без других символов).

    А) разность потенциалов, Б) напряжённость электрического поля, В) энергия заряженного конденсатора.

    1) джоуль, 2) вольт, 3) Н / Кл, 4) фарад, 5) ампер.
}
\answer{%
    $231$
}
\solutionspace{40pt}

\tasknumber{2}%
\task{%
    Установите соответствие и запишите в ответ набор цифр (без других символов).

    А) энергия заряженного конденсатора, Б) разность потенциалов, В) электроёмкость.

    1) $C$, 2) $W$, 3) $U$, 4) $r$, 5) $E$.
}
\answer{%
    $231$
}
\solutionspace{40pt}

\tasknumber{3}%
\task{%
    На конденсаторе указано: $C = 80\,\text{пФ}$, $V = 450\,\text{В}$.
    Удастся ли его использовать для накопления заряда $Q = 30\,\text{нКл}$?
    (в ответе укажите «да» или «нет»)
}
\answer{%
    $
        Q_{\text{max}} = CV = 80\,\text{пФ} \cdot 450\,\text{В} = 36\,\text{нКл}
        \implies Q_{\text{max}} \ge Q \implies \text{удастся}
    $
}
\solutionspace{80pt}

\tasknumber{4}%
\task{%
    Конденсатор ёмкостью $20\,\text{нФ}$ был заряжен до напряжения $65\,\text{кВ}$.
    Затем напряжение уменьшают на $10\,\text{кВ}$.
    Определите на сколько уменьшится заряд конденсатора, ответ выразите в микрокулонах.
}
\answer{%
    $q = C \cdot \Delta U = 200\,\text{мкКл}$
}
\solutionspace{80pt}

\tasknumber{5}%
\task{%
    Определите ёмкость конденсатора, если при его зарядке до напряжения
    $U = 50\,\text{кВ}$ он приобретает заряд $q = 6\,\text{мКл}$.
    % Чему при этом равны заряды обкладок конденсатора (сделайте рисунок и укажите их)?
    Ответ выразите в нанофарадах.
}
\answer{%
    $
        q = CU \implies
        C = \frac{q}{U} = \frac{6\,\text{мКл}}{50\,\text{кВ}} = 120\,\text{нФ}.
        \text{ Заряды обкладок: $q$ и $-q$}
    $
}
\solutionspace{120pt}

\tasknumber{6}%
\task{%
    Как и во сколько раз изменится ёмкость плоского конденсатора
    при уменьшении площади пластин в 7 раз
    и уменьшении расстояния между ними в 6 раз?
    В ответе укажите простую дробь или число — отношение новой ёмкости к старой.
}
\answer{%
    $
        \frac{C'}C
            = \frac{\eps_0\eps \frac S7}{\frac d6} \Big/ \frac{\eps_0\eps S}d
            = \frac{6}{7} = < 1 \implies \text{уменьшится в $\frac76$ раз}
    $
}
\solutionspace{80pt}

\tasknumber{7}%
\task{%
    Электрическая ёмкость конденсатора равна $C = 400\,\text{пФ}$,
    при этом ему сообщён заряд $q = 800\,\text{нКл}$.
    Какова энергия заряженного конденсатора?
    Ответ выразите в микроджоулях и округлите до целого.
}
\answer{%
    $
        W
        = \frac{q^2}{2C}
        = \frac{\sqr{800\,\text{нКл}}}{2 \cdot 400\,\text{пФ}}
        = 800{,}00\,\text{мкДж}
    $
}

\variantsplitter

\addpersonalvariant{Алина Полканова}

\tasknumber{1}%
\task{%
    Установите соответствие и запишите в ответ набор цифр (без других символов).

    А) энергия заряженного конденсатора, Б) разность потенциалов, В) электрический заряд.

    1) кулон, 2) вольт, 3) Н / Кл, 4) джоуль, 5) ватт.
}
\answer{%
    $421$
}
\solutionspace{40pt}

\tasknumber{2}%
\task{%
    Установите соответствие и запишите в ответ набор цифр (без других символов).

    А) напряжённость электрического поля, Б) разность потенциалов, В) электрическая постоянная.

    1) $\eps_0$, 2) $U$, 3) $C$, 4) $E$, 5) $W$.
}
\answer{%
    $421$
}
\solutionspace{40pt}

\tasknumber{3}%
\task{%
    На конденсаторе указано: $C = 150\,\text{пФ}$, $V = 450\,\text{В}$.
    Удастся ли его использовать для накопления заряда $Q = 50\,\text{нКл}$?
    (в ответе укажите «да» или «нет»)
}
\answer{%
    $
        Q_{\text{max}} = CV = 150\,\text{пФ} \cdot 450\,\text{В} = 67\,\text{нКл}
        \implies Q_{\text{max}} \ge Q \implies \text{удастся}
    $
}
\solutionspace{80pt}

\tasknumber{4}%
\task{%
    Конденсатор ёмкостью $30\,\text{нФ}$ был заряжен до напряжения $75\,\text{кВ}$.
    Затем напряжение уменьшают на $30\,\text{кВ}$.
    Определите на сколько уменьшится заряд конденсатора, ответ выразите в микрокулонах.
}
\answer{%
    $q = C \cdot \Delta U = 900\,\text{мкКл}$
}
\solutionspace{80pt}

\tasknumber{5}%
\task{%
    Определите ёмкость конденсатора, если при его зарядке до напряжения
    $V = 2\,\text{кВ}$ он приобретает заряд $q = 24\,\text{мКл}$.
    % Чему при этом равны заряды обкладок конденсатора (сделайте рисунок и укажите их)?
    Ответ выразите в нанофарадах.
}
\answer{%
    $
        q = CV \implies
        C = \frac{q}{V} = \frac{24\,\text{мКл}}{2\,\text{кВ}} = 12000\,\text{нФ}.
        \text{ Заряды обкладок: $q$ и $-q$}
    $
}
\solutionspace{120pt}

\tasknumber{6}%
\task{%
    Как и во сколько раз изменится ёмкость плоского конденсатора
    при уменьшении площади пластин в 4 раз
    и уменьшении расстояния между ними в 7 раз?
    В ответе укажите простую дробь или число — отношение новой ёмкости к старой.
}
\answer{%
    $
        \frac{C'}C
            = \frac{\eps_0\eps \frac S4}{\frac d7} \Big/ \frac{\eps_0\eps S}d
            = \frac{7}{4} = > 1 \implies \text{увеличится в $\frac74$ раз}
    $
}
\solutionspace{80pt}

\tasknumber{7}%
\task{%
    Электрическая ёмкость конденсатора равна $C = 600\,\text{пФ}$,
    при этом ему сообщён заряд $Q = 900\,\text{нКл}$.
    Какова энергия заряженного конденсатора?
    Ответ выразите в микроджоулях и округлите до целого.
}
\answer{%
    $
        W
        = \frac{Q^2}{2C}
        = \frac{\sqr{900\,\text{нКл}}}{2 \cdot 600\,\text{пФ}}
        = 675{,}00\,\text{мкДж}
    $
}

\variantsplitter

\addpersonalvariant{Сергей Пономарёв}

\tasknumber{1}%
\task{%
    Установите соответствие и запишите в ответ набор цифр (без других символов).

    А) разность потенциалов, Б) энергия заряженного конденсатора, В) напряжённость электрического поля.

    1) вольт, 2) джоуль, 3) Н / Кл, 4) ампер, 5) генри.
}
\answer{%
    $123$
}
\solutionspace{40pt}

\tasknumber{2}%
\task{%
    Установите соответствие и запишите в ответ набор цифр (без других символов).

    А) энергия заряженного конденсатора, Б) разность потенциалов, В) электрическая постоянная.

    1) $W$, 2) $U$, 3) $\eps_0$, 4) $C$, 5) $Q$.
}
\answer{%
    $123$
}
\solutionspace{40pt}

\tasknumber{3}%
\task{%
    На конденсаторе указано: $C = 80\,\text{пФ}$, $U = 300\,\text{В}$.
    Удастся ли его использовать для накопления заряда $q = 50\,\text{нКл}$?
    (в ответе укажите «да» или «нет»)
}
\answer{%
    $
        q_{\text{max}} = CU = 80\,\text{пФ} \cdot 300\,\text{В} = 24\,\text{нКл}
        \implies q_{\text{max}}  <  q \implies \text{не удастся}
    $
}
\solutionspace{80pt}

\tasknumber{4}%
\task{%
    Конденсатор ёмкостью $20\,\text{нФ}$ был заряжен до напряжения $65\,\text{кВ}$.
    Затем напряжение уменьшают на $30\,\text{кВ}$.
    Определите на сколько уменьшится заряд конденсатора, ответ выразите в микрокулонах.
}
\answer{%
    $q = C \cdot \Delta U = 600\,\text{мкКл}$
}
\solutionspace{80pt}

\tasknumber{5}%
\task{%
    Определите ёмкость конденсатора, если при его зарядке до напряжения
    $V = 20\,\text{кВ}$ он приобретает заряд $q = 18\,\text{мКл}$.
    % Чему при этом равны заряды обкладок конденсатора (сделайте рисунок и укажите их)?
    Ответ выразите в нанофарадах.
}
\answer{%
    $
        q = CV \implies
        C = \frac{q}{V} = \frac{18\,\text{мКл}}{20\,\text{кВ}} = 900\,\text{нФ}.
        \text{ Заряды обкладок: $q$ и $-q$}
    $
}
\solutionspace{120pt}

\tasknumber{6}%
\task{%
    Как и во сколько раз изменится ёмкость плоского конденсатора
    при уменьшении площади пластин в 5 раз
    и уменьшении расстояния между ними в 6 раз?
    В ответе укажите простую дробь или число — отношение новой ёмкости к старой.
}
\answer{%
    $
        \frac{C'}C
            = \frac{\eps_0\eps \frac S5}{\frac d6} \Big/ \frac{\eps_0\eps S}d
            = \frac{6}{5} = > 1 \implies \text{увеличится в $\frac65$ раз}
    $
}
\solutionspace{80pt}

\tasknumber{7}%
\task{%
    Электрическая ёмкость конденсатора равна $C = 750\,\text{пФ}$,
    при этом ему сообщён заряд $q = 500\,\text{нКл}$.
    Какова энергия заряженного конденсатора?
    Ответ выразите в микроджоулях и округлите до целого.
}
\answer{%
    $
        W
        = \frac{q^2}{2C}
        = \frac{\sqr{500\,\text{нКл}}}{2 \cdot 750\,\text{пФ}}
        = 166{,}67\,\text{мкДж}
    $
}

\variantsplitter

\addpersonalvariant{Егор Свистушкин}

\tasknumber{1}%
\task{%
    Установите соответствие и запишите в ответ набор цифр (без других символов).

    А) электрический заряд, Б) энергия заряженного конденсатора, В) разность потенциалов.

    1) кулон, 2) джоуль, 3) ватт, 4) ампер, 5) вольт.
}
\answer{%
    $125$
}
\solutionspace{40pt}

\tasknumber{2}%
\task{%
    Установите соответствие и запишите в ответ набор цифр (без других символов).

    А) энергия заряженного конденсатора, Б) напряжённость электрического поля, В) электроёмкость.

    1) $W$, 2) $E$, 3) $\eps$, 4) $U$, 5) $C$.
}
\answer{%
    $125$
}
\solutionspace{40pt}

\tasknumber{3}%
\task{%
    На конденсаторе указано: $C = 150\,\text{пФ}$, $U = 300\,\text{В}$.
    Удастся ли его использовать для накопления заряда $Q = 60\,\text{нКл}$?
    (в ответе укажите «да» или «нет»)
}
\answer{%
    $
        Q_{\text{max}} = CU = 150\,\text{пФ} \cdot 300\,\text{В} = 45\,\text{нКл}
        \implies Q_{\text{max}}  <  Q \implies \text{не удастся}
    $
}
\solutionspace{80pt}

\tasknumber{4}%
\task{%
    Конденсатор ёмкостью $40\,\text{нФ}$ был заряжен до напряжения $65\,\text{кВ}$.
    Затем напряжение уменьшают до $20\,\text{кВ}$.
    Определите на сколько уменьшится заряд конденсатора, ответ выразите в микрокулонах.
}
\answer{%
    $q = C \cdot \Delta U = 1800\,\text{мкКл}$
}
\solutionspace{80pt}

\tasknumber{5}%
\task{%
    Определите ёмкость конденсатора, если при его зарядке до напряжения
    $V = 20\,\text{кВ}$ он приобретает заряд $Q = 15\,\text{мКл}$.
    % Чему при этом равны заряды обкладок конденсатора (сделайте рисунок и укажите их)?
    Ответ выразите в нанофарадах.
}
\answer{%
    $
        Q = CV \implies
        C = \frac{Q}{V} = \frac{15\,\text{мКл}}{20\,\text{кВ}} = 750\,\text{нФ}.
        \text{ Заряды обкладок: $Q$ и $-Q$}
    $
}
\solutionspace{120pt}

\tasknumber{6}%
\task{%
    Как и во сколько раз изменится ёмкость плоского конденсатора
    при уменьшении площади пластин в 6 раз
    и уменьшении расстояния между ними в 7 раз?
    В ответе укажите простую дробь или число — отношение новой ёмкости к старой.
}
\answer{%
    $
        \frac{C'}C
            = \frac{\eps_0\eps \frac S6}{\frac d7} \Big/ \frac{\eps_0\eps S}d
            = \frac{7}{6} = > 1 \implies \text{увеличится в $\frac76$ раз}
    $
}
\solutionspace{80pt}

\tasknumber{7}%
\task{%
    Электрическая ёмкость конденсатора равна $C = 750\,\text{пФ}$,
    при этом ему сообщён заряд $q = 800\,\text{нКл}$.
    Какова энергия заряженного конденсатора?
    Ответ выразите в микроджоулях и округлите до целого.
}
\answer{%
    $
        W
        = \frac{q^2}{2C}
        = \frac{\sqr{800\,\text{нКл}}}{2 \cdot 750\,\text{пФ}}
        = 426{,}67\,\text{мкДж}
    $
}

\variantsplitter

\addpersonalvariant{Дмитрий Соколов}

\tasknumber{1}%
\task{%
    Установите соответствие и запишите в ответ набор цифр (без других символов).

    А) электроёмкость, Б) разность потенциалов, В) напряжённость электрического поля.

    1) Н / Кл, 2) вольт, 3) ампер, 4) фарад, 5) джоуль.
}
\answer{%
    $421$
}
\solutionspace{40pt}

\tasknumber{2}%
\task{%
    Установите соответствие и запишите в ответ набор цифр (без других символов).

    А) электрический заряд, Б) энергия заряженного конденсатора, В) напряжённость электрического поля.

    1) $E$, 2) $W$, 3) $C$, 4) $Q$, 5) $\eps_0$.
}
\answer{%
    $421$
}
\solutionspace{40pt}

\tasknumber{3}%
\task{%
    На конденсаторе указано: $C = 120\,\text{пФ}$, $U = 200\,\text{В}$.
    Удастся ли его использовать для накопления заряда $q = 60\,\text{нКл}$?
    (в ответе укажите «да» или «нет»)
}
\answer{%
    $
        q_{\text{max}} = CU = 120\,\text{пФ} \cdot 200\,\text{В} = 24\,\text{нКл}
        \implies q_{\text{max}}  <  q \implies \text{не удастся}
    $
}
\solutionspace{80pt}

\tasknumber{4}%
\task{%
    Конденсатор ёмкостью $20\,\text{нФ}$ был заряжен до напряжения $65\,\text{кВ}$.
    Затем напряжение уменьшают до $10\,\text{кВ}$.
    Определите на сколько уменьшится заряд конденсатора, ответ выразите в микрокулонах.
}
\answer{%
    $q = C \cdot \Delta U = 1100\,\text{мкКл}$
}
\solutionspace{80pt}

\tasknumber{5}%
\task{%
    Определите ёмкость конденсатора, если при его зарядке до напряжения
    $U = 50\,\text{кВ}$ он приобретает заряд $q = 18\,\text{мКл}$.
    % Чему при этом равны заряды обкладок конденсатора (сделайте рисунок и укажите их)?
    Ответ выразите в нанофарадах.
}
\answer{%
    $
        q = CU \implies
        C = \frac{q}{U} = \frac{18\,\text{мКл}}{50\,\text{кВ}} = 360\,\text{нФ}.
        \text{ Заряды обкладок: $q$ и $-q$}
    $
}
\solutionspace{120pt}

\tasknumber{6}%
\task{%
    Как и во сколько раз изменится ёмкость плоского конденсатора
    при уменьшении площади пластин в 2 раз
    и уменьшении расстояния между ними в 7 раз?
    В ответе укажите простую дробь или число — отношение новой ёмкости к старой.
}
\answer{%
    $
        \frac{C'}C
            = \frac{\eps_0\eps \frac S2}{\frac d7} \Big/ \frac{\eps_0\eps S}d
            = \frac{7}{2} = > 1 \implies \text{увеличится в $\frac72$ раз}
    $
}
\solutionspace{80pt}

\tasknumber{7}%
\task{%
    Электрическая ёмкость конденсатора равна $C = 600\,\text{пФ}$,
    при этом ему сообщён заряд $q = 500\,\text{нКл}$.
    Какова энергия заряженного конденсатора?
    Ответ выразите в микроджоулях и округлите до целого.
}
\answer{%
    $
        W
        = \frac{q^2}{2C}
        = \frac{\sqr{500\,\text{нКл}}}{2 \cdot 600\,\text{пФ}}
        = 208{,}33\,\text{мкДж}
    $
}

\variantsplitter

\addpersonalvariant{Арсений Трофимов}

\tasknumber{1}%
\task{%
    Установите соответствие и запишите в ответ набор цифр (без других символов).

    А) электроёмкость, Б) электрический заряд, В) разность потенциалов.

    1) фарад, 2) джоуль, 3) кулон, 4) генри, 5) вольт.
}
\answer{%
    $135$
}
\solutionspace{40pt}

\tasknumber{2}%
\task{%
    Установите соответствие и запишите в ответ набор цифр (без других символов).

    А) электроёмкость, Б) электрический заряд, В) напряжённость электрического поля.

    1) $C$, 2) $r$, 3) $Q$, 4) $\eps_0$, 5) $E$.
}
\answer{%
    $135$
}
\solutionspace{40pt}

\tasknumber{3}%
\task{%
    На конденсаторе указано: $C = 150\,\text{пФ}$, $U = 300\,\text{В}$.
    Удастся ли его использовать для накопления заряда $q = 30\,\text{нКл}$?
    (в ответе укажите «да» или «нет»)
}
\answer{%
    $
        q_{\text{max}} = CU = 150\,\text{пФ} \cdot 300\,\text{В} = 45\,\text{нКл}
        \implies q_{\text{max}} \ge q \implies \text{удастся}
    $
}
\solutionspace{80pt}

\tasknumber{4}%
\task{%
    Конденсатор ёмкостью $20\,\text{нФ}$ был заряжен до напряжения $75\,\text{кВ}$.
    Затем напряжение уменьшают до $10\,\text{кВ}$.
    Определите на сколько уменьшится заряд конденсатора, ответ выразите в микрокулонах.
}
\answer{%
    $q = C \cdot \Delta U = 1300\,\text{мкКл}$
}
\solutionspace{80pt}

\tasknumber{5}%
\task{%
    Определите ёмкость конденсатора, если при его зарядке до напряжения
    $U = 4\,\text{кВ}$ он приобретает заряд $Q = 18\,\text{мКл}$.
    % Чему при этом равны заряды обкладок конденсатора (сделайте рисунок и укажите их)?
    Ответ выразите в нанофарадах.
}
\answer{%
    $
        Q = CU \implies
        C = \frac{Q}{U} = \frac{18\,\text{мКл}}{4\,\text{кВ}} = 4500\,\text{нФ}.
        \text{ Заряды обкладок: $Q$ и $-Q$}
    $
}
\solutionspace{120pt}

\tasknumber{6}%
\task{%
    Как и во сколько раз изменится ёмкость плоского конденсатора
    при уменьшении площади пластин в 2 раз
    и уменьшении расстояния между ними в 7 раз?
    В ответе укажите простую дробь или число — отношение новой ёмкости к старой.
}
\answer{%
    $
        \frac{C'}C
            = \frac{\eps_0\eps \frac S2}{\frac d7} \Big/ \frac{\eps_0\eps S}d
            = \frac{7}{2} = > 1 \implies \text{увеличится в $\frac72$ раз}
    $
}
\solutionspace{80pt}

\tasknumber{7}%
\task{%
    Электрическая ёмкость конденсатора равна $C = 400\,\text{пФ}$,
    при этом ему сообщён заряд $q = 300\,\text{нКл}$.
    Какова энергия заряженного конденсатора?
    Ответ выразите в микроджоулях и округлите до целого.
}
\answer{%
    $
        W
        = \frac{q^2}{2C}
        = \frac{\sqr{300\,\text{нКл}}}{2 \cdot 400\,\text{пФ}}
        = 112{,}50\,\text{мкДж}
    $
}
% autogenerated
