\setdate{23~апреля~2021}
\setclass{10«АБ»}

\addpersonalvariant{Михаил Бурмистров}

\tasknumber{1}%
\task{%
    Установите каждой букве в соответствие ровно одну цифру и запишите ответ (только цифры, без других символов).

    А) сила тока, Б) разность потенциалов, В) удельное сопротивление проводника.

    1) $S$, 2) $\rho$, 3) $\lambda$, 4) $\mathcal{I}$, 5) $U$.
}
\answer{%
    $452$
}
\solutionspace{20pt}

\tasknumber{2}%
\task{%
    Установите каждой букве в соответствие ровно одну цифру и запишите ответ (только цифры, без других символов).

    А) электрический заряд, Б) сила тока, В) длина проводника.

    1) ватт, 2) метр, 3) ом, 4) кулон, 5) ампер.
}
\answer{%
    $452$
}
\solutionspace{20pt}

\tasknumber{3}%
\task{%
    Установите каждой букве в соответствие ровно одну цифру и запишите ответ (только цифры, без других символов).

    А) закон Ома, Б) эквивалентное сопротивление 2 резисторов (последовательно).

    1) $\frac{R_1R_2}{R_1 + R_2}$, 2) $R = \rho \frac Sl$, 3) $R = \rho \frac lS$, 4) $\mathcal{I} R = U$, 5) $R_1 + R_2$.
}
\answer{%
    $45$
}
\solutionspace{20pt}

\tasknumber{4}%
\task{%
    Установите каждой букве в соответствие ровно одну цифру и запишите ответ (только цифры, без других символов).

    А) эквивалентное сопротивление 3 резисторов (параллельно), Б) эквивалентное сопротивление 3 резисторов (последовательно).

    1) $\frac 3{\frac 1{R_1} + \frac 1{R_2} + \frac 1{R_3}}$, 2) $\sqrt{\frac{R_1^2 + R_2^2 + R_3^2}3}$, 3) $\frac{R_1 + R_2 + R_3}3$, 4) $\frac{R_1R_2R_3}{R_1R_2 + R_2R_3 + R_3R_1}$, 5) $R_1 + R_2 + R_3$.
}
\answer{%
    $45$
}
\solutionspace{20pt}

\tasknumber{5}%
\task{%
    На резистор сопротивлением $10\,\text{Ом}$ подали напряжение $240\,\text{В}$.
    Определите ток, который потечёт через резистор, ответ выразите в амперах.
}
\answer{%
    $\eli = \frac{U}{R} = \frac{240\,\text{В}}{10\,\text{Ом}} = 24\,\text{А}$
}
\solutionspace{20pt}

\tasknumber{6}%
\task{%
    Женя собирает электрическую цепь из $40$ одинаковых резисторов, каждый сопротивлением $240\,\text{Ом}$.
    Какое эквивалентное сопротивление этой цепи получится, если все резисторы подключены параллельно, ответ выразите в омах.
}
\answer{%
    $r = 6\,\text{Ом}$
}
\solutionspace{20pt}

\tasknumber{7}%
\task{%
    Два резистора сопротивлениями $R_1 = 4\,\text{кОм}$ и $R_2 = 6\,\text{кОм}$ подключены параллельно и на них подано напряжение.
    Определите, какое напряжение них подали, если в цепи идёт $\eli = 2\,\text{мА}$.
    Ответ выразите в вольтах и округлите до целого.
}
\answer{%
    $U = 5\,\text{В}$
}
\solutionspace{20pt}

\tasknumber{8}%
\task{%
    Валя проводит эксперименты c 2 кусками одинаковой медной проволки, причём второй кусок в 5 длиннее первого.
    В одном из экспериментов Валя подаёт на первый кусок проволки напряжение в 3 раз больше, чем на второй.
    Определите отношение сил тока в двух проволках в этом эксперименте: второй к первой.
    В ответе укажите простую дробь или целое число.
}
\answer{%
    $\frac1{15}$
}
\solutionspace{40pt}

\tasknumber{9}%
\task{%
    В распоряжении Маши имеется 10 одинаковых резисторов, каждый сопротивлением $3\,\text{кОм}$.
    Какое наибольшее эквивалентное сопротивление она может из них получить? Использовать все резисторы при этом не обязательно, ответ укажите в омах.
}
\answer{%
    $r = 30000\,\text{Ом}$
}

\variantsplitter

\addpersonalvariant{Ирина Ан}

\tasknumber{1}%
\task{%
    Установите каждой букве в соответствие ровно одну цифру и запишите ответ (только цифры, без других символов).

    А) удельное сопротивление проводника, Б) длина проводника, В) электрическое сопротивление резистора.

    1) $U$, 2) $R$, 3) $l$, 4) $\rho$, 5) $k$.
}
\answer{%
    $432$
}
\solutionspace{20pt}

\tasknumber{2}%
\task{%
    Установите каждой букве в соответствие ровно одну цифру и запишите ответ (только цифры, без других символов).

    А) электрическое сопротивление резистора, Б) сила тока, В) длина проводника.

    1) вольт, 2) метр, 3) ампер, 4) ом, 5) ватт.
}
\answer{%
    $432$
}
\solutionspace{20pt}

\tasknumber{3}%
\task{%
    Установите каждой букве в соответствие ровно одну цифру и запишите ответ (только цифры, без других символов).

    А) закон Ома, Б) эквивалентное сопротивление 2 резисторов (параллельно).

    1) $\rho = R l S$, 2) $R_1 + R_2$, 3) $\frac{R_1R_2}{R_1 + R_2}$, 4) $\mathcal{I} R = U$, 5) $\frac{2R_1R_2}{R_1 + R_2}$.
}
\answer{%
    $43$
}
\solutionspace{20pt}

\tasknumber{4}%
\task{%
    Установите каждой букве в соответствие ровно одну цифру и запишите ответ (только цифры, без других символов).

    А) эквивалентное сопротивление 3 резисторов (параллельно), Б) эквивалентное сопротивление 3 резисторов (последовательно).

    1) $\frac{R_1 + R_2 + R_3}3$, 2) $\sqrt{\frac{R_1^2 + R_2^2 + R_3^2}3}$, 3) $R_1 + R_2 + R_3$, 4) $\frac{R_1R_2R_3}{R_1R_2 + R_2R_3 + R_3R_1}$, 5) $\frac 3{\frac 1{R_1} + \frac 1{R_2} + \frac 1{R_3}}$.
}
\answer{%
    $43$
}
\solutionspace{20pt}

\tasknumber{5}%
\task{%
    На резистор сопротивлением $3\,\text{Ом}$ подали напряжение $120\,\text{В}$.
    Определите ток, который потечёт через резистор, ответ выразите в амперах.
}
\answer{%
    $\eli = \frac{U}{R} = \frac{120\,\text{В}}{3\,\text{Ом}} = 40\,\text{А}$
}
\solutionspace{20pt}

\tasknumber{6}%
\task{%
    Женя собирает электрическую цепь из $10$ одинаковых резисторов, каждый сопротивлением $240\,\text{Ом}$.
    Какое эквивалентное сопротивление этой цепи получится, если все резисторы подключены параллельно, ответ выразите в омах.
}
\answer{%
    $r = 24\,\text{Ом}$
}
\solutionspace{20pt}

\tasknumber{7}%
\task{%
    Два резистора сопротивлениями $R_1 = 10\,\text{кОм}$ и $R_2 = 6\,\text{кОм}$ подключены параллельно и на них подано напряжение.
    Определите, какое напряжение них подали, если в цепи идёт $\eli = 2\,\text{мА}$.
    Ответ выразите в вольтах и округлите до целого.
}
\answer{%
    $U = 8\,\text{В}$
}
\solutionspace{20pt}

\tasknumber{8}%
\task{%
    Валя проводит эксперименты c 2 кусками одинаковой алюминиевой проволки, причём второй кусок в 10 длиннее первого.
    В одном из экспериментов Валя подаёт на первый кусок проволки напряжение в 9 раз больше, чем на второй.
    Определите отношение сил тока в двух проволках в этом эксперименте: второй к первой.
    В ответе укажите простую дробь или целое число.
}
\answer{%
    $\frac1{90}$
}
\solutionspace{40pt}

\tasknumber{9}%
\task{%
    В распоряжении Маши имеется 20 одинаковых резисторов, каждый сопротивлением $3\,\text{кОм}$.
    Какое наименьшее эквивалентное сопротивление она может из них получить? Использовать все резисторы при этом не обязательно, ответ укажите в омах.
}
\answer{%
    $r = 150\,\text{Ом}$
}

\variantsplitter

\addpersonalvariant{Софья Андрианова}

\tasknumber{1}%
\task{%
    Установите каждой букве в соответствие ровно одну цифру и запишите ответ (только цифры, без других символов).

    А) разность потенциалов, Б) электрическое сопротивление резистора, В) удельное сопротивление проводника.

    1) $D$, 2) $U$, 3) $\rho$, 4) $\mathcal{I}$, 5) $R$.
}
\answer{%
    $253$
}
\solutionspace{20pt}

\tasknumber{2}%
\task{%
    Установите каждой букве в соответствие ровно одну цифру и запишите ответ (только цифры, без других символов).

    А) разность потенциалов, Б) электрическое сопротивление резистора, В) электрический заряд.

    1) ампер, 2) вольт, 3) кулон, 4) метр, 5) ом.
}
\answer{%
    $253$
}
\solutionspace{20pt}

\tasknumber{3}%
\task{%
    Установите каждой букве в соответствие ровно одну цифру и запишите ответ (только цифры, без других символов).

    А) электрическое сопротивление резистора, Б) закон Ома.

    1) $\sqrt{R_1R_2}$, 2) $R = \rho \frac lS$, 3) $\rho = R l S$, 4) $\frac{2R_1R_2}{R_1 + R_2}$, 5) $\mathcal{I} R = U$.
}
\answer{%
    $25$
}
\solutionspace{20pt}

\tasknumber{4}%
\task{%
    Установите каждой букве в соответствие ровно одну цифру и запишите ответ (только цифры, без других символов).

    А) эквивалентное сопротивление 3 резисторов (параллельно), Б) эквивалентное сопротивление 3 резисторов (последовательно).

    1) $\frac{R_1R_2R_3}{R_1 + R_2 + R_3}$, 2) $\frac{R_1R_2R_3}{R_1R_2 + R_2R_3 + R_3R_1}$, 3) $\frac 3{\frac 1{R_1} + \frac 1{R_2} + \frac 1{R_3}}$, 4) $\frac{R_1 + R_2 + R_3}3$, 5) $R_1 + R_2 + R_3$.
}
\answer{%
    $25$
}
\solutionspace{20pt}

\tasknumber{5}%
\task{%
    На резистор сопротивлением $3\,\text{Ом}$ подали напряжение $240\,\text{В}$.
    Определите ток, который потечёт через резистор, ответ выразите в амперах.
}
\answer{%
    $\eli = \frac{U}{R} = \frac{240\,\text{В}}{3\,\text{Ом}} = 80\,\text{А}$
}
\solutionspace{20pt}

\tasknumber{6}%
\task{%
    Женя собирает электрическую цепь из $40$ одинаковых резисторов, каждый сопротивлением $160\,\text{Ом}$.
    Какое эквивалентное сопротивление этой цепи получится, если все резисторы подключены параллельно, ответ выразите в омах.
}
\answer{%
    $r = 4\,\text{Ом}$
}
\solutionspace{20pt}

\tasknumber{7}%
\task{%
    Два резистора сопротивлениями $R_1 = 4\,\text{кОм}$ и $R_2 = 2\,\text{кОм}$ подключены параллельно и на них подано напряжение.
    Определите, какое напряжение них подали, если в цепи идёт $\eli = 5\,\text{мА}$.
    Ответ выразите в вольтах и округлите до целого.
}
\answer{%
    $U = 7\,\text{В}$
}
\solutionspace{20pt}

\tasknumber{8}%
\task{%
    Валя проводит эксперименты c 2 кусками одинаковой алюминиевой проволки, причём второй кусок в 10 длиннее первого.
    В одном из экспериментов Валя подаёт на первый кусок проволки напряжение в 5 раз больше, чем на второй.
    Определите отношение сил тока в двух проволках в этом эксперименте: второй к первой.
    В ответе укажите простую дробь или целое число.
}
\answer{%
    $\frac1{50}$
}
\solutionspace{40pt}

\tasknumber{9}%
\task{%
    В распоряжении Маши имеется 20 одинаковых резисторов, каждый сопротивлением $3\,\text{кОм}$.
    Какое наименьшее эквивалентное сопротивление она может из них получить? Использовать все резисторы при этом не обязательно, ответ укажите в омах.
}
\answer{%
    $r = 150\,\text{Ом}$
}

\variantsplitter

\addpersonalvariant{Владимир Артемчук}

\tasknumber{1}%
\task{%
    Установите каждой букве в соответствие ровно одну цифру и запишите ответ (только цифры, без других символов).

    А) сила тока, Б) удельное сопротивление проводника, В) электрическое сопротивление резистора.

    1) $\mathcal{I}$, 2) $\rho$, 3) $l$, 4) $S$, 5) $R$.
}
\answer{%
    $125$
}
\solutionspace{20pt}

\tasknumber{2}%
\task{%
    Установите каждой букве в соответствие ровно одну цифру и запишите ответ (только цифры, без других символов).

    А) сила тока, Б) электрический заряд, В) электрическое сопротивление резистора.

    1) ампер, 2) кулон, 3) метр, 4) сименс, 5) ом.
}
\answer{%
    $125$
}
\solutionspace{20pt}

\tasknumber{3}%
\task{%
    Установите каждой букве в соответствие ровно одну цифру и запишите ответ (только цифры, без других символов).

    А) эквивалентное сопротивление 2 резисторов (параллельно), Б) эквивалентное сопротивление 2 резисторов (последовательно).

    1) $\frac{R_1R_2}{R_1 + R_2}$, 2) $R_1 + R_2$, 3) $\frac{\mathcal{I}} R = U$, 4) $\mathcal{I} R = U$, 5) $\frac{2R_1R_2}{R_1 + R_2}$.
}
\answer{%
    $12$
}
\solutionspace{20pt}

\tasknumber{4}%
\task{%
    Установите каждой букве в соответствие ровно одну цифру и запишите ответ (только цифры, без других символов).

    А) эквивалентное сопротивление 3 резисторов (параллельно), Б) эквивалентное сопротивление 3 резисторов (последовательно).

    1) $\frac{R_1R_2R_3}{R_1R_2 + R_2R_3 + R_3R_1}$, 2) $R_1 + R_2 + R_3$, 3) $\sqrt{\frac{R_1^2 + R_2^2 + R_3^2}3}$, 4) $\frac{R_1 + R_2 + R_3}3$, 5) $\frac{R_1R_2R_3}{R_1 + R_2 + R_3}$.
}
\answer{%
    $12$
}
\solutionspace{20pt}

\tasknumber{5}%
\task{%
    На резистор сопротивлением $3\,\text{Ом}$ подали напряжение $240\,\text{В}$.
    Определите ток, который потечёт через резистор, ответ выразите в амперах.
}
\answer{%
    $\eli = \frac{U}{R} = \frac{240\,\text{В}}{3\,\text{Ом}} = 80\,\text{А}$
}
\solutionspace{20pt}

\tasknumber{6}%
\task{%
    Женя собирает электрическую цепь из $20$ одинаковых резисторов, каждый сопротивлением $320\,\text{Ом}$.
    Какое эквивалентное сопротивление этой цепи получится, если все резисторы подключены параллельно, ответ выразите в омах.
}
\answer{%
    $r = 16\,\text{Ом}$
}
\solutionspace{20pt}

\tasknumber{7}%
\task{%
    Два резистора сопротивлениями $R_1 = 15\,\text{кОм}$ и $R_2 = 6\,\text{кОм}$ подключены параллельно и на них подано напряжение.
    Определите, какое напряжение них подали, если в цепи идёт $\eli = 2\,\text{мА}$.
    Ответ выразите в вольтах и округлите до целого.
}
\answer{%
    $U = 9\,\text{В}$
}
\solutionspace{20pt}

\tasknumber{8}%
\task{%
    Валя проводит эксперименты c 2 кусками одинаковой медной проволки, причём второй кусок в 5 длиннее первого.
    В одном из экспериментов Валя подаёт на первый кусок проволки напряжение в 2 раз больше, чем на второй.
    Определите отношение сил тока в двух проволках в этом эксперименте: второй к первой.
    В ответе укажите простую дробь или целое число.
}
\answer{%
    $\frac1{10}$
}
\solutionspace{40pt}

\tasknumber{9}%
\task{%
    В распоряжении Маши имеется 20 одинаковых резисторов, каждый сопротивлением $2\,\text{кОм}$.
    Какое наибольшее эквивалентное сопротивление она может из них получить? Использовать все резисторы при этом не обязательно, ответ укажите в омах.
}
\answer{%
    $r = 40000\,\text{Ом}$
}

\variantsplitter

\addpersonalvariant{Софья Белянкина}

\tasknumber{1}%
\task{%
    Установите каждой букве в соответствие ровно одну цифру и запишите ответ (только цифры, без других символов).

    А) электрическое сопротивление резистора, Б) сила тока, В) площадь поперечного сечения проводника.

    1) $U$, 2) $l$, 3) $\mathcal{I}$, 4) $S$, 5) $R$.
}
\answer{%
    $534$
}
\solutionspace{20pt}

\tasknumber{2}%
\task{%
    Установите каждой букве в соответствие ровно одну цифру и запишите ответ (только цифры, без других символов).

    А) сила тока, Б) длина проводника, В) электрическое сопротивление резистора.

    1) сименс, 2) кулон, 3) метр, 4) ом, 5) ампер.
}
\answer{%
    $534$
}
\solutionspace{20pt}

\tasknumber{3}%
\task{%
    Установите каждой букве в соответствие ровно одну цифру и запишите ответ (только цифры, без других символов).

    А) электрическое сопротивление резистора, Б) эквивалентное сопротивление 2 резисторов (последовательно).

    1) $\frac{R_1 + R_2} 2$, 2) $\frac{\mathcal{I}} R = U$, 3) $R_1 + R_2$, 4) $\rho = R l S$, 5) $R = \rho \frac lS$.
}
\answer{%
    $53$
}
\solutionspace{20pt}

\tasknumber{4}%
\task{%
    Установите каждой букве в соответствие ровно одну цифру и запишите ответ (только цифры, без других символов).

    А) эквивалентное сопротивление 3 резисторов (последовательно), Б) эквивалентное сопротивление 3 резисторов (параллельно).

    1) $\frac{R_1R_2R_3}{R_1 + R_2 + R_3}$, 2) $\sqrt{\frac{R_1^2 + R_2^2 + R_3^2}3}$, 3) $\frac{R_1R_2R_3}{R_1R_2 + R_2R_3 + R_3R_1}$, 4) $\frac 3{\frac 1{R_1} + \frac 1{R_2} + \frac 1{R_3}}$, 5) $R_1 + R_2 + R_3$.
}
\answer{%
    $53$
}
\solutionspace{20pt}

\tasknumber{5}%
\task{%
    На резистор сопротивлением $15\,\text{Ом}$ подали напряжение $120\,\text{В}$.
    Определите ток, который потечёт через резистор, ответ выразите в амперах.
}
\answer{%
    $\eli = \frac{U}{R} = \frac{120\,\text{В}}{15\,\text{Ом}} = 8\,\text{А}$
}
\solutionspace{20pt}

\tasknumber{6}%
\task{%
    Женя собирает электрическую цепь из $40$ одинаковых резисторов, каждый сопротивлением $240\,\text{Ом}$.
    Какое эквивалентное сопротивление этой цепи получится, если все резисторы подключены параллельно, ответ выразите в омах.
}
\answer{%
    $r = 6\,\text{Ом}$
}
\solutionspace{20pt}

\tasknumber{7}%
\task{%
    Два резистора сопротивлениями $R_1 = 10\,\text{кОм}$ и $R_2 = 20\,\text{кОм}$ подключены последовательно и на них подано напряжение.
    Определите, какое напряжение них подали, если в цепи идёт $\eli = 2\,\text{мА}$.
    Ответ выразите в вольтах и округлите до целого.
}
\answer{%
    $U = 60\,\text{В}$
}
\solutionspace{20pt}

\tasknumber{8}%
\task{%
    Валя проводит эксперименты c 2 кусками одинаковой стальной проволки, причём второй кусок в 2 длиннее первого.
    В одном из экспериментов Валя подаёт на первый кусок проволки напряжение в 2 раз больше, чем на второй.
    Определите отношение сил тока в двух проволках в этом эксперименте: второй к первой.
    В ответе укажите простую дробь или целое число.
}
\answer{%
    $\frac14$
}
\solutionspace{40pt}

\tasknumber{9}%
\task{%
    В распоряжении Маши имеется 10 одинаковых резисторов, каждый сопротивлением $2\,\text{кОм}$.
    Какое наименьшее эквивалентное сопротивление она может из них получить? Использовать все резисторы при этом не обязательно, ответ укажите в омах.
}
\answer{%
    $r = 200\,\text{Ом}$
}

\variantsplitter

\addpersonalvariant{Варвара Егиазарян}

\tasknumber{1}%
\task{%
    Установите каждой букве в соответствие ровно одну цифру и запишите ответ (только цифры, без других символов).

    А) разность потенциалов, Б) электрическое сопротивление резистора, В) площадь поперечного сечения проводника.

    1) $S$, 2) $U$, 3) $R$, 4) $\mathcal{I}$, 5) $\lambda$.
}
\answer{%
    $231$
}
\solutionspace{20pt}

\tasknumber{2}%
\task{%
    Установите каждой букве в соответствие ровно одну цифру и запишите ответ (только цифры, без других символов).

    А) разность потенциалов, Б) электрический заряд, В) сила тока.

    1) ампер, 2) вольт, 3) кулон, 4) ватт, 5) сименс.
}
\answer{%
    $231$
}
\solutionspace{20pt}

\tasknumber{3}%
\task{%
    Установите каждой букве в соответствие ровно одну цифру и запишите ответ (только цифры, без других символов).

    А) эквивалентное сопротивление 2 резисторов (параллельно), Б) электрическое сопротивление резистора.

    1) $R_1 + R_2$, 2) $\frac{R_1R_2}{R_1 + R_2}$, 3) $R = \rho \frac lS$, 4) $R = \rho \frac Sl$, 5) $\frac{2R_1R_2}{R_1 + R_2}$.
}
\answer{%
    $23$
}
\solutionspace{20pt}

\tasknumber{4}%
\task{%
    Установите каждой букве в соответствие ровно одну цифру и запишите ответ (только цифры, без других символов).

    А) эквивалентное сопротивление 3 резисторов (параллельно), Б) эквивалентное сопротивление 3 резисторов (последовательно).

    1) $\frac 3{\frac 1{R_1} + \frac 1{R_2} + \frac 1{R_3}}$, 2) $\frac{R_1R_2R_3}{R_1R_2 + R_2R_3 + R_3R_1}$, 3) $R_1 + R_2 + R_3$, 4) $\frac{R_1R_2R_3}{R_1 + R_2 + R_3}$, 5) $\frac{R_1 + R_2 + R_3}3$.
}
\answer{%
    $23$
}
\solutionspace{20pt}

\tasknumber{5}%
\task{%
    На резистор сопротивлением $3\,\text{Ом}$ подали напряжение $180\,\text{В}$.
    Определите ток, который потечёт через резистор, ответ выразите в амперах.
}
\answer{%
    $\eli = \frac{U}{R} = \frac{180\,\text{В}}{3\,\text{Ом}} = 60\,\text{А}$
}
\solutionspace{20pt}

\tasknumber{6}%
\task{%
    Женя собирает электрическую цепь из $10$ одинаковых резисторов, каждый сопротивлением $320\,\text{Ом}$.
    Какое эквивалентное сопротивление этой цепи получится, если все резисторы подключены параллельно, ответ выразите в омах.
}
\answer{%
    $r = 32\,\text{Ом}$
}
\solutionspace{20pt}

\tasknumber{7}%
\task{%
    Два резистора сопротивлениями $R_1 = 10\,\text{кОм}$ и $R_2 = 2\,\text{кОм}$ подключены параллельно и на них подано напряжение.
    Определите, какое напряжение них подали, если в цепи идёт $\eli = 5\,\text{мА}$.
    Ответ выразите в вольтах и округлите до целого.
}
\answer{%
    $U = 8\,\text{В}$
}
\solutionspace{20pt}

\tasknumber{8}%
\task{%
    Валя проводит эксперименты c 2 кусками одинаковой медной проволки, причём второй кусок в 5 длиннее первого.
    В одном из экспериментов Валя подаёт на первый кусок проволки напряжение в 10 раз больше, чем на второй.
    Определите отношение сил тока в двух проволках в этом эксперименте: второй к первой.
    В ответе укажите простую дробь или целое число.
}
\answer{%
    $\frac1{50}$
}
\solutionspace{40pt}

\tasknumber{9}%
\task{%
    В распоряжении Маши имеется 20 одинаковых резисторов, каждый сопротивлением $3\,\text{кОм}$.
    Какое наименьшее эквивалентное сопротивление она может из них получить? Использовать все резисторы при этом не обязательно, ответ укажите в омах.
}
\answer{%
    $r = 150\,\text{Ом}$
}

\variantsplitter

\addpersonalvariant{Владислав Емелин}

\tasknumber{1}%
\task{%
    Установите каждой букве в соответствие ровно одну цифру и запишите ответ (только цифры, без других символов).

    А) электрическое сопротивление резистора, Б) разность потенциалов, В) сила тока.

    1) $\mathcal{I}$, 2) $l$, 3) $R$, 4) $\rho$, 5) $U$.
}
\answer{%
    $351$
}
\solutionspace{20pt}

\tasknumber{2}%
\task{%
    Установите каждой букве в соответствие ровно одну цифру и запишите ответ (только цифры, без других символов).

    А) электрический заряд, Б) сила тока, В) длина проводника.

    1) метр, 2) сименс, 3) кулон, 4) генри, 5) ампер.
}
\answer{%
    $351$
}
\solutionspace{20pt}

\tasknumber{3}%
\task{%
    Установите каждой букве в соответствие ровно одну цифру и запишите ответ (только цифры, без других символов).

    А) закон Ома, Б) электрическое сопротивление резистора.

    1) $\frac{\mathcal{I}} R = U$, 2) $\frac{2R_1R_2}{R_1 + R_2}$, 3) $\mathcal{I} R = U$, 4) $\rho = R l S$, 5) $R = \rho \frac lS$.
}
\answer{%
    $35$
}
\solutionspace{20pt}

\tasknumber{4}%
\task{%
    Установите каждой букве в соответствие ровно одну цифру и запишите ответ (только цифры, без других символов).

    А) эквивалентное сопротивление 3 резисторов (последовательно), Б) эквивалентное сопротивление 3 резисторов (параллельно).

    1) $\frac{R_1 + R_2 + R_3}3$, 2) $\sqrt{\frac{R_1^2 + R_2^2 + R_3^2}3}$, 3) $R_1 + R_2 + R_3$, 4) $\frac 3{\frac 1{R_1} + \frac 1{R_2} + \frac 1{R_3}}$, 5) $\frac{R_1R_2R_3}{R_1R_2 + R_2R_3 + R_3R_1}$.
}
\answer{%
    $35$
}
\solutionspace{20pt}

\tasknumber{5}%
\task{%
    На резистор сопротивлением $15\,\text{Ом}$ подали напряжение $180\,\text{В}$.
    Определите ток, который потечёт через резистор, ответ выразите в амперах.
}
\answer{%
    $\eli = \frac{U}{R} = \frac{180\,\text{В}}{15\,\text{Ом}} = 12\,\text{А}$
}
\solutionspace{20pt}

\tasknumber{6}%
\task{%
    Женя собирает электрическую цепь из $10$ одинаковых резисторов, каждый сопротивлением $160\,\text{Ом}$.
    Какое эквивалентное сопротивление этой цепи получится, если все резисторы подключены параллельно, ответ выразите в омах.
}
\answer{%
    $r = 16\,\text{Ом}$
}
\solutionspace{20pt}

\tasknumber{7}%
\task{%
    Два резистора сопротивлениями $R_1 = 10\,\text{кОм}$ и $R_2 = 20\,\text{кОм}$ подключены параллельно и на них подано напряжение.
    Определите, какое напряжение них подали, если в цепи идёт $\eli = 5\,\text{мА}$.
    Ответ выразите в вольтах и округлите до целого.
}
\answer{%
    $U = 33\,\text{В}$
}
\solutionspace{20pt}

\tasknumber{8}%
\task{%
    Валя проводит эксперименты c 2 кусками одинаковой стальной проволки, причём второй кусок в 6 длиннее первого.
    В одном из экспериментов Валя подаёт на первый кусок проволки напряжение в 5 раз больше, чем на второй.
    Определите отношение сил тока в двух проволках в этом эксперименте: второй к первой.
    В ответе укажите простую дробь или целое число.
}
\answer{%
    $\frac1{30}$
}
\solutionspace{40pt}

\tasknumber{9}%
\task{%
    В распоряжении Маши имеется 40 одинаковых резисторов, каждый сопротивлением $4\,\text{кОм}$.
    Какое наименьшее эквивалентное сопротивление она может из них получить? Использовать все резисторы при этом не обязательно, ответ укажите в омах.
}
\answer{%
    $r = 100\,\text{Ом}$
}

\variantsplitter

\addpersonalvariant{Артём Жичин}

\tasknumber{1}%
\task{%
    Установите каждой букве в соответствие ровно одну цифру и запишите ответ (только цифры, без других символов).

    А) электрическое сопротивление резистора, Б) разность потенциалов, В) сила тока.

    1) $\mathcal{I}$, 2) $k$, 3) $R$, 4) $U$, 5) $\rho$.
}
\answer{%
    $341$
}
\solutionspace{20pt}

\tasknumber{2}%
\task{%
    Установите каждой букве в соответствие ровно одну цифру и запишите ответ (только цифры, без других символов).

    А) электрический заряд, Б) сила тока, В) длина проводника.

    1) метр, 2) ом, 3) кулон, 4) ампер, 5) вольт.
}
\answer{%
    $341$
}
\solutionspace{20pt}

\tasknumber{3}%
\task{%
    Установите каждой букве в соответствие ровно одну цифру и запишите ответ (только цифры, без других символов).

    А) эквивалентное сопротивление 2 резисторов (параллельно), Б) закон Ома.

    1) $\frac{\mathcal{I}} R = U$, 2) $R = \rho \frac lS$, 3) $\frac{R_1R_2}{R_1 + R_2}$, 4) $\mathcal{I} R = U$, 5) $\sqrt{R_1R_2}$.
}
\answer{%
    $34$
}
\solutionspace{20pt}

\tasknumber{4}%
\task{%
    Установите каждой букве в соответствие ровно одну цифру и запишите ответ (только цифры, без других символов).

    А) эквивалентное сопротивление 3 резисторов (последовательно), Б) эквивалентное сопротивление 3 резисторов (параллельно).

    1) $\frac 3{\frac 1{R_1} + \frac 1{R_2} + \frac 1{R_3}}$, 2) $\sqrt{\frac{R_1^2 + R_2^2 + R_3^2}3}$, 3) $R_1 + R_2 + R_3$, 4) $\frac{R_1R_2R_3}{R_1R_2 + R_2R_3 + R_3R_1}$, 5) $\frac{R_1R_2R_3}{R_1 + R_2 + R_3}$.
}
\answer{%
    $34$
}
\solutionspace{20pt}

\tasknumber{5}%
\task{%
    На резистор сопротивлением $5\,\text{Ом}$ подали напряжение $180\,\text{В}$.
    Определите ток, который потечёт через резистор, ответ выразите в амперах.
}
\answer{%
    $\eli = \frac{U}{R} = \frac{180\,\text{В}}{5\,\text{Ом}} = 36\,\text{А}$
}
\solutionspace{20pt}

\tasknumber{6}%
\task{%
    Женя собирает электрическую цепь из $20$ одинаковых резисторов, каждый сопротивлением $320\,\text{Ом}$.
    Какое эквивалентное сопротивление этой цепи получится, если все резисторы подключены последовательно, ответ выразите в омах.
}
\answer{%
    $r = 6400\,\text{Ом}$
}
\solutionspace{20pt}

\tasknumber{7}%
\task{%
    Два резистора сопротивлениями $R_1 = 4\,\text{кОм}$ и $R_2 = 12\,\text{кОм}$ подключены параллельно и на них подано напряжение.
    Определите, какое напряжение них подали, если в цепи идёт $\eli = 5\,\text{мА}$.
    Ответ выразите в вольтах и округлите до целого.
}
\answer{%
    $U = 15\,\text{В}$
}
\solutionspace{20pt}

\tasknumber{8}%
\task{%
    Валя проводит эксперименты c 2 кусками одинаковой медной проволки, причём второй кусок в 5 длиннее первого.
    В одном из экспериментов Валя подаёт на первый кусок проволки напряжение в 5 раз больше, чем на второй.
    Определите отношение сил тока в двух проволках в этом эксперименте: второй к первой.
    В ответе укажите простую дробь или целое число.
}
\answer{%
    $\frac1{25}$
}
\solutionspace{40pt}

\tasknumber{9}%
\task{%
    В распоряжении Маши имеется 40 одинаковых резисторов, каждый сопротивлением $2\,\text{кОм}$.
    Какое наибольшее эквивалентное сопротивление она может из них получить? Использовать все резисторы при этом не обязательно, ответ укажите в омах.
}
\answer{%
    $r = 80000\,\text{Ом}$
}

\variantsplitter

\addpersonalvariant{Дарья Кошман}

\tasknumber{1}%
\task{%
    Установите каждой букве в соответствие ровно одну цифру и запишите ответ (только цифры, без других символов).

    А) площадь поперечного сечения проводника, Б) электрическое сопротивление резистора, В) удельное сопротивление проводника.

    1) $\rho$, 2) $k$, 3) $R$, 4) $l$, 5) $S$.
}
\answer{%
    $531$
}
\solutionspace{20pt}

\tasknumber{2}%
\task{%
    Установите каждой букве в соответствие ровно одну цифру и запишите ответ (только цифры, без других символов).

    А) сила тока, Б) разность потенциалов, В) длина проводника.

    1) метр, 2) ом, 3) вольт, 4) сименс, 5) ампер.
}
\answer{%
    $531$
}
\solutionspace{20pt}

\tasknumber{3}%
\task{%
    Установите каждой букве в соответствие ровно одну цифру и запишите ответ (только цифры, без других символов).

    А) закон Ома, Б) электрическое сопротивление резистора.

    1) $R = \rho \frac Sl$, 2) $\sqrt{R_1R_2}$, 3) $R = \rho \frac lS$, 4) $\frac{2R_1R_2}{R_1 + R_2}$, 5) $\mathcal{I} R = U$.
}
\answer{%
    $53$
}
\solutionspace{20pt}

\tasknumber{4}%
\task{%
    Установите каждой букве в соответствие ровно одну цифру и запишите ответ (только цифры, без других символов).

    А) эквивалентное сопротивление 3 резисторов (параллельно), Б) эквивалентное сопротивление 3 резисторов (последовательно).

    1) $\frac{R_1 + R_2 + R_3}3$, 2) $\frac{R_1R_2R_3}{R_1 + R_2 + R_3}$, 3) $R_1 + R_2 + R_3$, 4) $\sqrt{\frac{R_1^2 + R_2^2 + R_3^2}3}$, 5) $\frac{R_1R_2R_3}{R_1R_2 + R_2R_3 + R_3R_1}$.
}
\answer{%
    $53$
}
\solutionspace{20pt}

\tasknumber{5}%
\task{%
    На резистор сопротивлением $15\,\text{Ом}$ подали напряжение $240\,\text{В}$.
    Определите ток, который потечёт через резистор, ответ выразите в амперах.
}
\answer{%
    $\eli = \frac{U}{R} = \frac{240\,\text{В}}{15\,\text{Ом}} = 16\,\text{А}$
}
\solutionspace{20pt}

\tasknumber{6}%
\task{%
    Женя собирает электрическую цепь из $10$ одинаковых резисторов, каждый сопротивлением $160\,\text{Ом}$.
    Какое эквивалентное сопротивление этой цепи получится, если все резисторы подключены последовательно, ответ выразите в омах.
}
\answer{%
    $r = 1600\,\text{Ом}$
}
\solutionspace{20pt}

\tasknumber{7}%
\task{%
    Два резистора сопротивлениями $R_1 = 4\,\text{кОм}$ и $R_2 = 20\,\text{кОм}$ подключены последовательно и на них подано напряжение.
    Определите, какое напряжение них подали, если в цепи идёт $\eli = 5\,\text{мА}$.
    Ответ выразите в вольтах и округлите до целого.
}
\answer{%
    $U = 120\,\text{В}$
}
\solutionspace{20pt}

\tasknumber{8}%
\task{%
    Валя проводит эксперименты c 2 кусками одинаковой алюминиевой проволки, причём второй кусок в 9 длиннее первого.
    В одном из экспериментов Валя подаёт на первый кусок проволки напряжение в 8 раз больше, чем на второй.
    Определите отношение сил тока в двух проволках в этом эксперименте: второй к первой.
    В ответе укажите простую дробь или целое число.
}
\answer{%
    $\frac1{72}$
}
\solutionspace{40pt}

\tasknumber{9}%
\task{%
    В распоряжении Маши имеется 20 одинаковых резисторов, каждый сопротивлением $2\,\text{кОм}$.
    Какое наибольшее эквивалентное сопротивление она может из них получить? Использовать все резисторы при этом не обязательно, ответ укажите в омах.
}
\answer{%
    $r = 40000\,\text{Ом}$
}

\variantsplitter

\addpersonalvariant{Анна Кузьмичёва}

\tasknumber{1}%
\task{%
    Установите каждой букве в соответствие ровно одну цифру и запишите ответ (только цифры, без других символов).

    А) удельное сопротивление проводника, Б) разность потенциалов, В) электрическое сопротивление резистора.

    1) $\lambda$, 2) $U$, 3) $\mathcal{I}$, 4) $\rho$, 5) $R$.
}
\answer{%
    $425$
}
\solutionspace{20pt}

\tasknumber{2}%
\task{%
    Установите каждой букве в соответствие ровно одну цифру и запишите ответ (только цифры, без других символов).

    А) разность потенциалов, Б) электрическое сопротивление резистора, В) электрический заряд.

    1) метр, 2) ом, 3) ампер, 4) вольт, 5) кулон.
}
\answer{%
    $425$
}
\solutionspace{20pt}

\tasknumber{3}%
\task{%
    Установите каждой букве в соответствие ровно одну цифру и запишите ответ (только цифры, без других символов).

    А) закон Ома, Б) эквивалентное сопротивление 2 резисторов (последовательно).

    1) $\frac{\mathcal{I}} R = U$, 2) $R_1 + R_2$, 3) $R = \rho \frac Sl$, 4) $\mathcal{I} R = U$, 5) $\frac{2R_1R_2}{R_1 + R_2}$.
}
\answer{%
    $42$
}
\solutionspace{20pt}

\tasknumber{4}%
\task{%
    Установите каждой букве в соответствие ровно одну цифру и запишите ответ (только цифры, без других символов).

    А) эквивалентное сопротивление 3 резисторов (параллельно), Б) эквивалентное сопротивление 3 резисторов (последовательно).

    1) $\frac 3{\frac 1{R_1} + \frac 1{R_2} + \frac 1{R_3}}$, 2) $R_1 + R_2 + R_3$, 3) $\frac{R_1 + R_2 + R_3}3$, 4) $\frac{R_1R_2R_3}{R_1R_2 + R_2R_3 + R_3R_1}$, 5) $\frac{R_1R_2R_3}{R_1 + R_2 + R_3}$.
}
\answer{%
    $42$
}
\solutionspace{20pt}

\tasknumber{5}%
\task{%
    На резистор сопротивлением $20\,\text{Ом}$ подали напряжение $180\,\text{В}$.
    Определите ток, который потечёт через резистор, ответ выразите в амперах.
}
\answer{%
    $\eli = \frac{U}{R} = \frac{180\,\text{В}}{20\,\text{Ом}} = 9\,\text{А}$
}
\solutionspace{20pt}

\tasknumber{6}%
\task{%
    Женя собирает электрическую цепь из $20$ одинаковых резисторов, каждый сопротивлением $240\,\text{Ом}$.
    Какое эквивалентное сопротивление этой цепи получится, если все резисторы подключены последовательно, ответ выразите в омах.
}
\answer{%
    $r = 4800\,\text{Ом}$
}
\solutionspace{20pt}

\tasknumber{7}%
\task{%
    Два резистора сопротивлениями $R_1 = 4\,\text{кОм}$ и $R_2 = 6\,\text{кОм}$ подключены параллельно и на них подано напряжение.
    Определите, какое напряжение них подали, если в цепи идёт $\eli = 3\,\text{мА}$.
    Ответ выразите в вольтах и округлите до целого.
}
\answer{%
    $U = 7\,\text{В}$
}
\solutionspace{20pt}

\tasknumber{8}%
\task{%
    Валя проводит эксперименты c 2 кусками одинаковой алюминиевой проволки, причём второй кусок в 8 длиннее первого.
    В одном из экспериментов Валя подаёт на первый кусок проволки напряжение в 2 раз больше, чем на второй.
    Определите отношение сил тока в двух проволках в этом эксперименте: второй к первой.
    В ответе укажите простую дробь или целое число.
}
\answer{%
    $\frac1{16}$
}
\solutionspace{40pt}

\tasknumber{9}%
\task{%
    В распоряжении Маши имеется 40 одинаковых резисторов, каждый сопротивлением $4\,\text{кОм}$.
    Какое наибольшее эквивалентное сопротивление она может из них получить? Использовать все резисторы при этом не обязательно, ответ укажите в омах.
}
\answer{%
    $r = 160000\,\text{Ом}$
}

\variantsplitter

\addpersonalvariant{Алёна Куприянова}

\tasknumber{1}%
\task{%
    Установите каждой букве в соответствие ровно одну цифру и запишите ответ (только цифры, без других символов).

    А) длина проводника, Б) сила тока, В) удельное сопротивление проводника.

    1) $\mathcal{I}$, 2) $R$, 3) $l$, 4) $\rho$, 5) $k$.
}
\answer{%
    $314$
}
\solutionspace{20pt}

\tasknumber{2}%
\task{%
    Установите каждой букве в соответствие ровно одну цифру и запишите ответ (только цифры, без других символов).

    А) электрическое сопротивление резистора, Б) длина проводника, В) электрический заряд.

    1) метр, 2) генри, 3) ом, 4) кулон, 5) ватт.
}
\answer{%
    $314$
}
\solutionspace{20pt}

\tasknumber{3}%
\task{%
    Установите каждой букве в соответствие ровно одну цифру и запишите ответ (только цифры, без других символов).

    А) эквивалентное сопротивление 2 резисторов (последовательно), Б) закон Ома.

    1) $\mathcal{I} R = U$, 2) $\frac{R_1 + R_2} 2$, 3) $R_1 + R_2$, 4) $\rho = R l S$, 5) $R = \rho \frac Sl$.
}
\answer{%
    $31$
}
\solutionspace{20pt}

\tasknumber{4}%
\task{%
    Установите каждой букве в соответствие ровно одну цифру и запишите ответ (только цифры, без других символов).

    А) эквивалентное сопротивление 3 резисторов (параллельно), Б) эквивалентное сопротивление 3 резисторов (последовательно).

    1) $R_1 + R_2 + R_3$, 2) $\frac 3{\frac 1{R_1} + \frac 1{R_2} + \frac 1{R_3}}$, 3) $\frac{R_1R_2R_3}{R_1R_2 + R_2R_3 + R_3R_1}$, 4) $\frac{R_1R_2R_3}{R_1 + R_2 + R_3}$, 5) $\frac{R_1 + R_2 + R_3}3$.
}
\answer{%
    $31$
}
\solutionspace{20pt}

\tasknumber{5}%
\task{%
    На резистор сопротивлением $15\,\text{Ом}$ подали напряжение $240\,\text{В}$.
    Определите ток, который потечёт через резистор, ответ выразите в амперах.
}
\answer{%
    $\eli = \frac{U}{R} = \frac{240\,\text{В}}{15\,\text{Ом}} = 16\,\text{А}$
}
\solutionspace{20pt}

\tasknumber{6}%
\task{%
    Женя собирает электрическую цепь из $10$ одинаковых резисторов, каждый сопротивлением $160\,\text{Ом}$.
    Какое эквивалентное сопротивление этой цепи получится, если все резисторы подключены последовательно, ответ выразите в омах.
}
\answer{%
    $r = 1600\,\text{Ом}$
}
\solutionspace{20pt}

\tasknumber{7}%
\task{%
    Два резистора сопротивлениями $R_1 = 15\,\text{кОм}$ и $R_2 = 6\,\text{кОм}$ подключены параллельно и на них подано напряжение.
    Определите, какое напряжение них подали, если в цепи идёт $\eli = 2\,\text{мА}$.
    Ответ выразите в вольтах и округлите до целого.
}
\answer{%
    $U = 9\,\text{В}$
}
\solutionspace{20pt}

\tasknumber{8}%
\task{%
    Валя проводит эксперименты c 2 кусками одинаковой медной проволки, причём второй кусок в 3 длиннее первого.
    В одном из экспериментов Валя подаёт на первый кусок проволки напряжение в 5 раз больше, чем на второй.
    Определите отношение сил тока в двух проволках в этом эксперименте: второй к первой.
    В ответе укажите простую дробь или целое число.
}
\answer{%
    $\frac1{15}$
}
\solutionspace{40pt}

\tasknumber{9}%
\task{%
    В распоряжении Маши имеется 10 одинаковых резисторов, каждый сопротивлением $4\,\text{кОм}$.
    Какое наименьшее эквивалентное сопротивление она может из них получить? Использовать все резисторы при этом не обязательно, ответ укажите в омах.
}
\answer{%
    $r = 400\,\text{Ом}$
}

\variantsplitter

\addpersonalvariant{Ярослав Лавровский}

\tasknumber{1}%
\task{%
    Установите каждой букве в соответствие ровно одну цифру и запишите ответ (только цифры, без других символов).

    А) разность потенциалов, Б) длина проводника, В) электрическое сопротивление резистора.

    1) $S$, 2) $D$, 3) $l$, 4) $R$, 5) $U$.
}
\answer{%
    $534$
}
\solutionspace{20pt}

\tasknumber{2}%
\task{%
    Установите каждой букве в соответствие ровно одну цифру и запишите ответ (только цифры, без других символов).

    А) электрическое сопротивление резистора, Б) сила тока, В) разность потенциалов.

    1) генри, 2) кулон, 3) ампер, 4) вольт, 5) ом.
}
\answer{%
    $534$
}
\solutionspace{20pt}

\tasknumber{3}%
\task{%
    Установите каждой букве в соответствие ровно одну цифру и запишите ответ (только цифры, без других символов).

    А) закон Ома, Б) электрическое сопротивление резистора.

    1) $\rho = R l S$, 2) $\sqrt{R_1R_2}$, 3) $R = \rho \frac lS$, 4) $R_1 + R_2$, 5) $\mathcal{I} R = U$.
}
\answer{%
    $53$
}
\solutionspace{20pt}

\tasknumber{4}%
\task{%
    Установите каждой букве в соответствие ровно одну цифру и запишите ответ (только цифры, без других символов).

    А) эквивалентное сопротивление 3 резисторов (последовательно), Б) эквивалентное сопротивление 3 резисторов (параллельно).

    1) $\sqrt{\frac{R_1^2 + R_2^2 + R_3^2}3}$, 2) $\frac{R_1R_2R_3}{R_1 + R_2 + R_3}$, 3) $\frac{R_1R_2R_3}{R_1R_2 + R_2R_3 + R_3R_1}$, 4) $\frac 3{\frac 1{R_1} + \frac 1{R_2} + \frac 1{R_3}}$, 5) $R_1 + R_2 + R_3$.
}
\answer{%
    $53$
}
\solutionspace{20pt}

\tasknumber{5}%
\task{%
    На резистор сопротивлением $15\,\text{Ом}$ подали напряжение $120\,\text{В}$.
    Определите ток, который потечёт через резистор, ответ выразите в амперах.
}
\answer{%
    $\eli = \frac{U}{R} = \frac{120\,\text{В}}{15\,\text{Ом}} = 8\,\text{А}$
}
\solutionspace{20pt}

\tasknumber{6}%
\task{%
    Женя собирает электрическую цепь из $10$ одинаковых резисторов, каждый сопротивлением $160\,\text{Ом}$.
    Какое эквивалентное сопротивление этой цепи получится, если все резисторы подключены параллельно, ответ выразите в омах.
}
\answer{%
    $r = 16\,\text{Ом}$
}
\solutionspace{20pt}

\tasknumber{7}%
\task{%
    Два резистора сопротивлениями $R_1 = 4\,\text{кОм}$ и $R_2 = 2\,\text{кОм}$ подключены последовательно и на них подано напряжение.
    Определите, какое напряжение них подали, если в цепи идёт $\eli = 3\,\text{мА}$.
    Ответ выразите в вольтах и округлите до целого.
}
\answer{%
    $U = 18\,\text{В}$
}
\solutionspace{20pt}

\tasknumber{8}%
\task{%
    Валя проводит эксперименты c 2 кусками одинаковой стальной проволки, причём второй кусок в 4 длиннее первого.
    В одном из экспериментов Валя подаёт на первый кусок проволки напряжение в 2 раз больше, чем на второй.
    Определите отношение сил тока в двух проволках в этом эксперименте: второй к первой.
    В ответе укажите простую дробь или целое число.
}
\answer{%
    $\frac18$
}
\solutionspace{40pt}

\tasknumber{9}%
\task{%
    В распоряжении Маши имеется 40 одинаковых резисторов, каждый сопротивлением $3\,\text{кОм}$.
    Какое наибольшее эквивалентное сопротивление она может из них получить? Использовать все резисторы при этом не обязательно, ответ укажите в омах.
}
\answer{%
    $r = 120000\,\text{Ом}$
}

\variantsplitter

\addpersonalvariant{Анастасия Ламанова}

\tasknumber{1}%
\task{%
    Установите каждой букве в соответствие ровно одну цифру и запишите ответ (только цифры, без других символов).

    А) электрическое сопротивление резистора, Б) длина проводника, В) разность потенциалов.

    1) $S$, 2) $U$, 3) $l$, 4) $k$, 5) $R$.
}
\answer{%
    $532$
}
\solutionspace{20pt}

\tasknumber{2}%
\task{%
    Установите каждой букве в соответствие ровно одну цифру и запишите ответ (только цифры, без других символов).

    А) длина проводника, Б) разность потенциалов, В) электрический заряд.

    1) ом, 2) кулон, 3) вольт, 4) ватт, 5) метр.
}
\answer{%
    $532$
}
\solutionspace{20pt}

\tasknumber{3}%
\task{%
    Установите каждой букве в соответствие ровно одну цифру и запишите ответ (только цифры, без других символов).

    А) закон Ома, Б) эквивалентное сопротивление 2 резисторов (параллельно).

    1) $\frac{2R_1R_2}{R_1 + R_2}$, 2) $\frac{\mathcal{I}} R = U$, 3) $\frac{R_1R_2}{R_1 + R_2}$, 4) $R = \rho \frac lS$, 5) $\mathcal{I} R = U$.
}
\answer{%
    $53$
}
\solutionspace{20pt}

\tasknumber{4}%
\task{%
    Установите каждой букве в соответствие ровно одну цифру и запишите ответ (только цифры, без других символов).

    А) эквивалентное сопротивление 3 резисторов (последовательно), Б) эквивалентное сопротивление 3 резисторов (параллельно).

    1) $\frac{R_1 + R_2 + R_3}3$, 2) $\frac 3{\frac 1{R_1} + \frac 1{R_2} + \frac 1{R_3}}$, 3) $\frac{R_1R_2R_3}{R_1R_2 + R_2R_3 + R_3R_1}$, 4) $\frac{R_1R_2R_3}{R_1 + R_2 + R_3}$, 5) $R_1 + R_2 + R_3$.
}
\answer{%
    $53$
}
\solutionspace{20pt}

\tasknumber{5}%
\task{%
    На резистор сопротивлением $10\,\text{Ом}$ подали напряжение $120\,\text{В}$.
    Определите ток, который потечёт через резистор, ответ выразите в амперах.
}
\answer{%
    $\eli = \frac{U}{R} = \frac{120\,\text{В}}{10\,\text{Ом}} = 12\,\text{А}$
}
\solutionspace{20pt}

\tasknumber{6}%
\task{%
    Женя собирает электрическую цепь из $40$ одинаковых резисторов, каждый сопротивлением $240\,\text{Ом}$.
    Какое эквивалентное сопротивление этой цепи получится, если все резисторы подключены параллельно, ответ выразите в омах.
}
\answer{%
    $r = 6\,\text{Ом}$
}
\solutionspace{20pt}

\tasknumber{7}%
\task{%
    Два резистора сопротивлениями $R_1 = 4\,\text{кОм}$ и $R_2 = 6\,\text{кОм}$ подключены параллельно и на них подано напряжение.
    Определите, какое напряжение них подали, если в цепи идёт $\eli = 5\,\text{мА}$.
    Ответ выразите в вольтах и округлите до целого.
}
\answer{%
    $U = 12\,\text{В}$
}
\solutionspace{20pt}

\tasknumber{8}%
\task{%
    Валя проводит эксперименты c 2 кусками одинаковой алюминиевой проволки, причём второй кусок в 3 длиннее первого.
    В одном из экспериментов Валя подаёт на первый кусок проволки напряжение в 4 раз больше, чем на второй.
    Определите отношение сил тока в двух проволках в этом эксперименте: второй к первой.
    В ответе укажите простую дробь или целое число.
}
\answer{%
    $\frac1{12}$
}
\solutionspace{40pt}

\tasknumber{9}%
\task{%
    В распоряжении Маши имеется 20 одинаковых резисторов, каждый сопротивлением $2\,\text{кОм}$.
    Какое наименьшее эквивалентное сопротивление она может из них получить? Использовать все резисторы при этом не обязательно, ответ укажите в омах.
}
\answer{%
    $r = 100\,\text{Ом}$
}

\variantsplitter

\addpersonalvariant{Виктория Легонькова}

\tasknumber{1}%
\task{%
    Установите каждой букве в соответствие ровно одну цифру и запишите ответ (только цифры, без других символов).

    А) длина проводника, Б) сила тока, В) разность потенциалов.

    1) $R$, 2) $l$, 3) $\rho$, 4) $U$, 5) $\mathcal{I}$.
}
\answer{%
    $254$
}
\solutionspace{20pt}

\tasknumber{2}%
\task{%
    Установите каждой букве в соответствие ровно одну цифру и запишите ответ (только цифры, без других символов).

    А) электрическое сопротивление резистора, Б) сила тока, В) разность потенциалов.

    1) кулон, 2) ом, 3) ватт, 4) вольт, 5) ампер.
}
\answer{%
    $254$
}
\solutionspace{20pt}

\tasknumber{3}%
\task{%
    Установите каждой букве в соответствие ровно одну цифру и запишите ответ (только цифры, без других символов).

    А) электрическое сопротивление резистора, Б) закон Ома.

    1) $\frac{R_1R_2}{R_1 + R_2}$, 2) $R = \rho \frac lS$, 3) $\frac{2R_1R_2}{R_1 + R_2}$, 4) $\sqrt{R_1R_2}$, 5) $\mathcal{I} R = U$.
}
\answer{%
    $25$
}
\solutionspace{20pt}

\tasknumber{4}%
\task{%
    Установите каждой букве в соответствие ровно одну цифру и запишите ответ (только цифры, без других символов).

    А) эквивалентное сопротивление 3 резисторов (последовательно), Б) эквивалентное сопротивление 3 резисторов (параллельно).

    1) $\frac{R_1R_2R_3}{R_1 + R_2 + R_3}$, 2) $R_1 + R_2 + R_3$, 3) $\frac 3{\frac 1{R_1} + \frac 1{R_2} + \frac 1{R_3}}$, 4) $\frac{R_1 + R_2 + R_3}3$, 5) $\frac{R_1R_2R_3}{R_1R_2 + R_2R_3 + R_3R_1}$.
}
\answer{%
    $25$
}
\solutionspace{20pt}

\tasknumber{5}%
\task{%
    На резистор сопротивлением $20\,\text{Ом}$ подали напряжение $240\,\text{В}$.
    Определите ток, который потечёт через резистор, ответ выразите в амперах.
}
\answer{%
    $\eli = \frac{U}{R} = \frac{240\,\text{В}}{20\,\text{Ом}} = 12\,\text{А}$
}
\solutionspace{20pt}

\tasknumber{6}%
\task{%
    Женя собирает электрическую цепь из $20$ одинаковых резисторов, каждый сопротивлением $240\,\text{Ом}$.
    Какое эквивалентное сопротивление этой цепи получится, если все резисторы подключены последовательно, ответ выразите в омах.
}
\answer{%
    $r = 4800\,\text{Ом}$
}
\solutionspace{20pt}

\tasknumber{7}%
\task{%
    Два резистора сопротивлениями $R_1 = 15\,\text{кОм}$ и $R_2 = 12\,\text{кОм}$ подключены последовательно и на них подано напряжение.
    Определите, какое напряжение них подали, если в цепи идёт $\eli = 3\,\text{мА}$.
    Ответ выразите в вольтах и округлите до целого.
}
\answer{%
    $U = 81\,\text{В}$
}
\solutionspace{20pt}

\tasknumber{8}%
\task{%
    Валя проводит эксперименты c 2 кусками одинаковой алюминиевой проволки, причём второй кусок в 2 длиннее первого.
    В одном из экспериментов Валя подаёт на первый кусок проволки напряжение в 5 раз больше, чем на второй.
    Определите отношение сил тока в двух проволках в этом эксперименте: второй к первой.
    В ответе укажите простую дробь или целое число.
}
\answer{%
    $\frac1{10}$
}
\solutionspace{40pt}

\tasknumber{9}%
\task{%
    В распоряжении Маши имеется 10 одинаковых резисторов, каждый сопротивлением $4\,\text{кОм}$.
    Какое наименьшее эквивалентное сопротивление она может из них получить? Использовать все резисторы при этом не обязательно, ответ укажите в омах.
}
\answer{%
    $r = 400\,\text{Ом}$
}

\variantsplitter

\addpersonalvariant{Семён Мартынов}

\tasknumber{1}%
\task{%
    Установите каждой букве в соответствие ровно одну цифру и запишите ответ (только цифры, без других символов).

    А) сила тока, Б) электрическое сопротивление резистора, В) разность потенциалов.

    1) $\mathcal{I}$, 2) $U$, 3) $R$, 4) $\lambda$, 5) $\rho$.
}
\answer{%
    $132$
}
\solutionspace{20pt}

\tasknumber{2}%
\task{%
    Установите каждой букве в соответствие ровно одну цифру и запишите ответ (только цифры, без других символов).

    А) сила тока, Б) разность потенциалов, В) электрическое сопротивление резистора.

    1) ампер, 2) ом, 3) вольт, 4) генри, 5) сименс.
}
\answer{%
    $132$
}
\solutionspace{20pt}

\tasknumber{3}%
\task{%
    Установите каждой букве в соответствие ровно одну цифру и запишите ответ (только цифры, без других символов).

    А) эквивалентное сопротивление 2 резисторов (последовательно), Б) закон Ома.

    1) $R_1 + R_2$, 2) $R = \rho \frac lS$, 3) $\mathcal{I} R = U$, 4) $\sqrt{R_1R_2}$, 5) $\frac{R_1R_2}{R_1 + R_2}$.
}
\answer{%
    $13$
}
\solutionspace{20pt}

\tasknumber{4}%
\task{%
    Установите каждой букве в соответствие ровно одну цифру и запишите ответ (только цифры, без других символов).

    А) эквивалентное сопротивление 3 резисторов (параллельно), Б) эквивалентное сопротивление 3 резисторов (последовательно).

    1) $\frac{R_1R_2R_3}{R_1R_2 + R_2R_3 + R_3R_1}$, 2) $\frac{R_1R_2R_3}{R_1 + R_2 + R_3}$, 3) $R_1 + R_2 + R_3$, 4) $\frac{R_1 + R_2 + R_3}3$, 5) $\sqrt{\frac{R_1^2 + R_2^2 + R_3^2}3}$.
}
\answer{%
    $13$
}
\solutionspace{20pt}

\tasknumber{5}%
\task{%
    На резистор сопротивлением $3\,\text{Ом}$ подали напряжение $240\,\text{В}$.
    Определите ток, который потечёт через резистор, ответ выразите в амперах.
}
\answer{%
    $\eli = \frac{U}{R} = \frac{240\,\text{В}}{3\,\text{Ом}} = 80\,\text{А}$
}
\solutionspace{20pt}

\tasknumber{6}%
\task{%
    Женя собирает электрическую цепь из $10$ одинаковых резисторов, каждый сопротивлением $240\,\text{Ом}$.
    Какое эквивалентное сопротивление этой цепи получится, если все резисторы подключены последовательно, ответ выразите в омах.
}
\answer{%
    $r = 2400\,\text{Ом}$
}
\solutionspace{20pt}

\tasknumber{7}%
\task{%
    Два резистора сопротивлениями $R_1 = 15\,\text{кОм}$ и $R_2 = 2\,\text{кОм}$ подключены последовательно и на них подано напряжение.
    Определите, какое напряжение них подали, если в цепи идёт $\eli = 2\,\text{мА}$.
    Ответ выразите в вольтах и округлите до целого.
}
\answer{%
    $U = 34\,\text{В}$
}
\solutionspace{20pt}

\tasknumber{8}%
\task{%
    Валя проводит эксперименты c 2 кусками одинаковой стальной проволки, причём второй кусок в 3 длиннее первого.
    В одном из экспериментов Валя подаёт на первый кусок проволки напряжение в 8 раз больше, чем на второй.
    Определите отношение сил тока в двух проволках в этом эксперименте: второй к первой.
    В ответе укажите простую дробь или целое число.
}
\answer{%
    $\frac1{24}$
}
\solutionspace{40pt}

\tasknumber{9}%
\task{%
    В распоряжении Маши имеется 40 одинаковых резисторов, каждый сопротивлением $4\,\text{кОм}$.
    Какое наибольшее эквивалентное сопротивление она может из них получить? Использовать все резисторы при этом не обязательно, ответ укажите в омах.
}
\answer{%
    $r = 160000\,\text{Ом}$
}

\variantsplitter

\addpersonalvariant{Варвара Минаева}

\tasknumber{1}%
\task{%
    Установите каждой букве в соответствие ровно одну цифру и запишите ответ (только цифры, без других символов).

    А) длина проводника, Б) площадь поперечного сечения проводника, В) сила тока.

    1) $\mathcal{I}$, 2) $l$, 3) $S$, 4) $\lambda$, 5) $R$.
}
\answer{%
    $231$
}
\solutionspace{20pt}

\tasknumber{2}%
\task{%
    Установите каждой букве в соответствие ровно одну цифру и запишите ответ (только цифры, без других символов).

    А) электрический заряд, Б) электрическое сопротивление резистора, В) длина проводника.

    1) метр, 2) кулон, 3) ом, 4) ватт, 5) генри.
}
\answer{%
    $231$
}
\solutionspace{20pt}

\tasknumber{3}%
\task{%
    Установите каждой букве в соответствие ровно одну цифру и запишите ответ (только цифры, без других символов).

    А) эквивалентное сопротивление 2 резисторов (параллельно), Б) электрическое сопротивление резистора.

    1) $R_1 + R_2$, 2) $\frac{R_1R_2}{R_1 + R_2}$, 3) $R = \rho \frac lS$, 4) $\frac{R_1 + R_2} 2$, 5) $\rho = R l S$.
}
\answer{%
    $23$
}
\solutionspace{20pt}

\tasknumber{4}%
\task{%
    Установите каждой букве в соответствие ровно одну цифру и запишите ответ (только цифры, без других символов).

    А) эквивалентное сопротивление 3 резисторов (параллельно), Б) эквивалентное сопротивление 3 резисторов (последовательно).

    1) $\frac{R_1 + R_2 + R_3}3$, 2) $\frac{R_1R_2R_3}{R_1R_2 + R_2R_3 + R_3R_1}$, 3) $R_1 + R_2 + R_3$, 4) $\sqrt{\frac{R_1^2 + R_2^2 + R_3^2}3}$, 5) $\frac 3{\frac 1{R_1} + \frac 1{R_2} + \frac 1{R_3}}$.
}
\answer{%
    $23$
}
\solutionspace{20pt}

\tasknumber{5}%
\task{%
    На резистор сопротивлением $5\,\text{Ом}$ подали напряжение $180\,\text{В}$.
    Определите ток, который потечёт через резистор, ответ выразите в амперах.
}
\answer{%
    $\eli = \frac{U}{R} = \frac{180\,\text{В}}{5\,\text{Ом}} = 36\,\text{А}$
}
\solutionspace{20pt}

\tasknumber{6}%
\task{%
    Женя собирает электрическую цепь из $20$ одинаковых резисторов, каждый сопротивлением $240\,\text{Ом}$.
    Какое эквивалентное сопротивление этой цепи получится, если все резисторы подключены параллельно, ответ выразите в омах.
}
\answer{%
    $r = 12\,\text{Ом}$
}
\solutionspace{20pt}

\tasknumber{7}%
\task{%
    Два резистора сопротивлениями $R_1 = 4\,\text{кОм}$ и $R_2 = 20\,\text{кОм}$ подключены последовательно и на них подано напряжение.
    Определите, какое напряжение них подали, если в цепи идёт $\eli = 3\,\text{мА}$.
    Ответ выразите в вольтах и округлите до целого.
}
\answer{%
    $U = 72\,\text{В}$
}
\solutionspace{20pt}

\tasknumber{8}%
\task{%
    Валя проводит эксперименты c 2 кусками одинаковой стальной проволки, причём второй кусок в 5 длиннее первого.
    В одном из экспериментов Валя подаёт на первый кусок проволки напряжение в 5 раз больше, чем на второй.
    Определите отношение сил тока в двух проволках в этом эксперименте: второй к первой.
    В ответе укажите простую дробь или целое число.
}
\answer{%
    $\frac1{25}$
}
\solutionspace{40pt}

\tasknumber{9}%
\task{%
    В распоряжении Маши имеется 40 одинаковых резисторов, каждый сопротивлением $2\,\text{кОм}$.
    Какое наибольшее эквивалентное сопротивление она может из них получить? Использовать все резисторы при этом не обязательно, ответ укажите в омах.
}
\answer{%
    $r = 80000\,\text{Ом}$
}

\variantsplitter

\addpersonalvariant{Леонид Никитин}

\tasknumber{1}%
\task{%
    Установите каждой букве в соответствие ровно одну цифру и запишите ответ (только цифры, без других символов).

    А) удельное сопротивление проводника, Б) электрическое сопротивление резистора, В) длина проводника.

    1) $U$, 2) $\rho$, 3) $\lambda$, 4) $R$, 5) $l$.
}
\answer{%
    $245$
}
\solutionspace{20pt}

\tasknumber{2}%
\task{%
    Установите каждой букве в соответствие ровно одну цифру и запишите ответ (только цифры, без других символов).

    А) разность потенциалов, Б) длина проводника, В) электрическое сопротивление резистора.

    1) ватт, 2) вольт, 3) сименс, 4) метр, 5) ом.
}
\answer{%
    $245$
}
\solutionspace{20pt}

\tasknumber{3}%
\task{%
    Установите каждой букве в соответствие ровно одну цифру и запишите ответ (только цифры, без других символов).

    А) закон Ома, Б) электрическое сопротивление резистора.

    1) $\sqrt{R_1R_2}$, 2) $\mathcal{I} R = U$, 3) $\rho = R l S$, 4) $R = \rho \frac lS$, 5) $\frac{\mathcal{I}} R = U$.
}
\answer{%
    $24$
}
\solutionspace{20pt}

\tasknumber{4}%
\task{%
    Установите каждой букве в соответствие ровно одну цифру и запишите ответ (только цифры, без других символов).

    А) эквивалентное сопротивление 3 резисторов (параллельно), Б) эквивалентное сопротивление 3 резисторов (последовательно).

    1) $\frac{R_1R_2R_3}{R_1 + R_2 + R_3}$, 2) $\frac{R_1R_2R_3}{R_1R_2 + R_2R_3 + R_3R_1}$, 3) $\sqrt{\frac{R_1^2 + R_2^2 + R_3^2}3}$, 4) $R_1 + R_2 + R_3$, 5) $\frac 3{\frac 1{R_1} + \frac 1{R_2} + \frac 1{R_3}}$.
}
\answer{%
    $24$
}
\solutionspace{20pt}

\tasknumber{5}%
\task{%
    На резистор сопротивлением $20\,\text{Ом}$ подали напряжение $120\,\text{В}$.
    Определите ток, который потечёт через резистор, ответ выразите в амперах.
}
\answer{%
    $\eli = \frac{U}{R} = \frac{120\,\text{В}}{20\,\text{Ом}} = 6\,\text{А}$
}
\solutionspace{20pt}

\tasknumber{6}%
\task{%
    Женя собирает электрическую цепь из $10$ одинаковых резисторов, каждый сопротивлением $160\,\text{Ом}$.
    Какое эквивалентное сопротивление этой цепи получится, если все резисторы подключены параллельно, ответ выразите в омах.
}
\answer{%
    $r = 16\,\text{Ом}$
}
\solutionspace{20pt}

\tasknumber{7}%
\task{%
    Два резистора сопротивлениями $R_1 = 15\,\text{кОм}$ и $R_2 = 6\,\text{кОм}$ подключены параллельно и на них подано напряжение.
    Определите, какое напряжение них подали, если в цепи идёт $\eli = 5\,\text{мА}$.
    Ответ выразите в вольтах и округлите до целого.
}
\answer{%
    $U = 21\,\text{В}$
}
\solutionspace{20pt}

\tasknumber{8}%
\task{%
    Валя проводит эксперименты c 2 кусками одинаковой медной проволки, причём второй кусок в 4 длиннее первого.
    В одном из экспериментов Валя подаёт на первый кусок проволки напряжение в 3 раз больше, чем на второй.
    Определите отношение сил тока в двух проволках в этом эксперименте: второй к первой.
    В ответе укажите простую дробь или целое число.
}
\answer{%
    $\frac1{12}$
}
\solutionspace{40pt}

\tasknumber{9}%
\task{%
    В распоряжении Маши имеется 10 одинаковых резисторов, каждый сопротивлением $4\,\text{кОм}$.
    Какое наибольшее эквивалентное сопротивление она может из них получить? Использовать все резисторы при этом не обязательно, ответ укажите в омах.
}
\answer{%
    $r = 40000\,\text{Ом}$
}

\variantsplitter

\addpersonalvariant{Тимофей Полетаев}

\tasknumber{1}%
\task{%
    Установите каждой букве в соответствие ровно одну цифру и запишите ответ (только цифры, без других символов).

    А) площадь поперечного сечения проводника, Б) удельное сопротивление проводника, В) электрическое сопротивление резистора.

    1) $S$, 2) $\rho$, 3) $k$, 4) $R$, 5) $l$.
}
\answer{%
    $124$
}
\solutionspace{20pt}

\tasknumber{2}%
\task{%
    Установите каждой букве в соответствие ровно одну цифру и запишите ответ (только цифры, без других символов).

    А) сила тока, Б) электрический заряд, В) длина проводника.

    1) ампер, 2) кулон, 3) ом, 4) метр, 5) генри.
}
\answer{%
    $124$
}
\solutionspace{20pt}

\tasknumber{3}%
\task{%
    Установите каждой букве в соответствие ровно одну цифру и запишите ответ (только цифры, без других символов).

    А) эквивалентное сопротивление 2 резисторов (последовательно), Б) закон Ома.

    1) $R_1 + R_2$, 2) $\mathcal{I} R = U$, 3) $\frac{R_1 + R_2} 2$, 4) $\rho = R l S$, 5) $R = \rho \frac Sl$.
}
\answer{%
    $12$
}
\solutionspace{20pt}

\tasknumber{4}%
\task{%
    Установите каждой букве в соответствие ровно одну цифру и запишите ответ (только цифры, без других символов).

    А) эквивалентное сопротивление 3 резисторов (последовательно), Б) эквивалентное сопротивление 3 резисторов (параллельно).

    1) $R_1 + R_2 + R_3$, 2) $\frac{R_1R_2R_3}{R_1R_2 + R_2R_3 + R_3R_1}$, 3) $\frac 3{\frac 1{R_1} + \frac 1{R_2} + \frac 1{R_3}}$, 4) $\frac{R_1 + R_2 + R_3}3$, 5) $\frac{R_1R_2R_3}{R_1 + R_2 + R_3}$.
}
\answer{%
    $12$
}
\solutionspace{20pt}

\tasknumber{5}%
\task{%
    На резистор сопротивлением $15\,\text{Ом}$ подали напряжение $240\,\text{В}$.
    Определите ток, который потечёт через резистор, ответ выразите в амперах.
}
\answer{%
    $\eli = \frac{U}{R} = \frac{240\,\text{В}}{15\,\text{Ом}} = 16\,\text{А}$
}
\solutionspace{20pt}

\tasknumber{6}%
\task{%
    Женя собирает электрическую цепь из $20$ одинаковых резисторов, каждый сопротивлением $320\,\text{Ом}$.
    Какое эквивалентное сопротивление этой цепи получится, если все резисторы подключены параллельно, ответ выразите в омах.
}
\answer{%
    $r = 16\,\text{Ом}$
}
\solutionspace{20pt}

\tasknumber{7}%
\task{%
    Два резистора сопротивлениями $R_1 = 15\,\text{кОм}$ и $R_2 = 2\,\text{кОм}$ подключены параллельно и на них подано напряжение.
    Определите, какое напряжение них подали, если в цепи идёт $\eli = 5\,\text{мА}$.
    Ответ выразите в вольтах и округлите до целого.
}
\answer{%
    $U = 9\,\text{В}$
}
\solutionspace{20pt}

\tasknumber{8}%
\task{%
    Валя проводит эксперименты c 2 кусками одинаковой стальной проволки, причём второй кусок в 2 длиннее первого.
    В одном из экспериментов Валя подаёт на первый кусок проволки напряжение в 10 раз больше, чем на второй.
    Определите отношение сил тока в двух проволках в этом эксперименте: второй к первой.
    В ответе укажите простую дробь или целое число.
}
\answer{%
    $\frac1{20}$
}
\solutionspace{40pt}

\tasknumber{9}%
\task{%
    В распоряжении Маши имеется 20 одинаковых резисторов, каждый сопротивлением $2\,\text{кОм}$.
    Какое наибольшее эквивалентное сопротивление она может из них получить? Использовать все резисторы при этом не обязательно, ответ укажите в омах.
}
\answer{%
    $r = 40000\,\text{Ом}$
}

\variantsplitter

\addpersonalvariant{Андрей Рожков}

\tasknumber{1}%
\task{%
    Установите каждой букве в соответствие ровно одну цифру и запишите ответ (только цифры, без других символов).

    А) сила тока, Б) электрическое сопротивление резистора, В) разность потенциалов.

    1) $k$, 2) $\mathcal{I}$, 3) $U$, 4) $\rho$, 5) $R$.
}
\answer{%
    $253$
}
\solutionspace{20pt}

\tasknumber{2}%
\task{%
    Установите каждой букве в соответствие ровно одну цифру и запишите ответ (только цифры, без других символов).

    А) сила тока, Б) разность потенциалов, В) электрическое сопротивление резистора.

    1) кулон, 2) ампер, 3) ом, 4) сименс, 5) вольт.
}
\answer{%
    $253$
}
\solutionspace{20pt}

\tasknumber{3}%
\task{%
    Установите каждой букве в соответствие ровно одну цифру и запишите ответ (только цифры, без других символов).

    А) закон Ома, Б) эквивалентное сопротивление 2 резисторов (последовательно).

    1) $R = \rho \frac Sl$, 2) $\mathcal{I} R = U$, 3) $\rho = R l S$, 4) $R = \rho \frac lS$, 5) $R_1 + R_2$.
}
\answer{%
    $25$
}
\solutionspace{20pt}

\tasknumber{4}%
\task{%
    Установите каждой букве в соответствие ровно одну цифру и запишите ответ (только цифры, без других символов).

    А) эквивалентное сопротивление 3 резисторов (параллельно), Б) эквивалентное сопротивление 3 резисторов (последовательно).

    1) $\frac{R_1 + R_2 + R_3}3$, 2) $\frac{R_1R_2R_3}{R_1R_2 + R_2R_3 + R_3R_1}$, 3) $\sqrt{\frac{R_1^2 + R_2^2 + R_3^2}3}$, 4) $\frac 3{\frac 1{R_1} + \frac 1{R_2} + \frac 1{R_3}}$, 5) $R_1 + R_2 + R_3$.
}
\answer{%
    $25$
}
\solutionspace{20pt}

\tasknumber{5}%
\task{%
    На резистор сопротивлением $30\,\text{Ом}$ подали напряжение $120\,\text{В}$.
    Определите ток, который потечёт через резистор, ответ выразите в амперах.
}
\answer{%
    $\eli = \frac{U}{R} = \frac{120\,\text{В}}{30\,\text{Ом}} = 4\,\text{А}$
}
\solutionspace{20pt}

\tasknumber{6}%
\task{%
    Женя собирает электрическую цепь из $40$ одинаковых резисторов, каждый сопротивлением $240\,\text{Ом}$.
    Какое эквивалентное сопротивление этой цепи получится, если все резисторы подключены параллельно, ответ выразите в омах.
}
\answer{%
    $r = 6\,\text{Ом}$
}
\solutionspace{20pt}

\tasknumber{7}%
\task{%
    Два резистора сопротивлениями $R_1 = 10\,\text{кОм}$ и $R_2 = 2\,\text{кОм}$ подключены последовательно и на них подано напряжение.
    Определите, какое напряжение них подали, если в цепи идёт $\eli = 3\,\text{мА}$.
    Ответ выразите в вольтах и округлите до целого.
}
\answer{%
    $U = 36\,\text{В}$
}
\solutionspace{20pt}

\tasknumber{8}%
\task{%
    Валя проводит эксперименты c 2 кусками одинаковой стальной проволки, причём второй кусок в 2 длиннее первого.
    В одном из экспериментов Валя подаёт на первый кусок проволки напряжение в 10 раз больше, чем на второй.
    Определите отношение сил тока в двух проволках в этом эксперименте: второй к первой.
    В ответе укажите простую дробь или целое число.
}
\answer{%
    $\frac1{20}$
}
\solutionspace{40pt}

\tasknumber{9}%
\task{%
    В распоряжении Маши имеется 20 одинаковых резисторов, каждый сопротивлением $3\,\text{кОм}$.
    Какое наименьшее эквивалентное сопротивление она может из них получить? Использовать все резисторы при этом не обязательно, ответ укажите в омах.
}
\answer{%
    $r = 150\,\text{Ом}$
}

\variantsplitter

\addpersonalvariant{Рената Таржиманова}

\tasknumber{1}%
\task{%
    Установите каждой букве в соответствие ровно одну цифру и запишите ответ (только цифры, без других символов).

    А) сила тока, Б) удельное сопротивление проводника, В) разность потенциалов.

    1) $\rho$, 2) $\lambda$, 3) $\mathcal{I}$, 4) $S$, 5) $U$.
}
\answer{%
    $315$
}
\solutionspace{20pt}

\tasknumber{2}%
\task{%
    Установите каждой букве в соответствие ровно одну цифру и запишите ответ (только цифры, без других символов).

    А) сила тока, Б) электрический заряд, В) разность потенциалов.

    1) кулон, 2) ватт, 3) ампер, 4) ом, 5) вольт.
}
\answer{%
    $315$
}
\solutionspace{20pt}

\tasknumber{3}%
\task{%
    Установите каждой букве в соответствие ровно одну цифру и запишите ответ (только цифры, без других символов).

    А) закон Ома, Б) электрическое сопротивление резистора.

    1) $R = \rho \frac lS$, 2) $R = \rho \frac Sl$, 3) $\mathcal{I} R = U$, 4) $\frac{\mathcal{I}} R = U$, 5) $\sqrt{R_1R_2}$.
}
\answer{%
    $31$
}
\solutionspace{20pt}

\tasknumber{4}%
\task{%
    Установите каждой букве в соответствие ровно одну цифру и запишите ответ (только цифры, без других символов).

    А) эквивалентное сопротивление 3 резисторов (последовательно), Б) эквивалентное сопротивление 3 резисторов (параллельно).

    1) $\frac{R_1R_2R_3}{R_1R_2 + R_2R_3 + R_3R_1}$, 2) $\frac{R_1 + R_2 + R_3}3$, 3) $R_1 + R_2 + R_3$, 4) $\frac 3{\frac 1{R_1} + \frac 1{R_2} + \frac 1{R_3}}$, 5) $\frac{R_1R_2R_3}{R_1 + R_2 + R_3}$.
}
\answer{%
    $31$
}
\solutionspace{20pt}

\tasknumber{5}%
\task{%
    На резистор сопротивлением $30\,\text{Ом}$ подали напряжение $120\,\text{В}$.
    Определите ток, который потечёт через резистор, ответ выразите в амперах.
}
\answer{%
    $\eli = \frac{U}{R} = \frac{120\,\text{В}}{30\,\text{Ом}} = 4\,\text{А}$
}
\solutionspace{20pt}

\tasknumber{6}%
\task{%
    Женя собирает электрическую цепь из $40$ одинаковых резисторов, каждый сопротивлением $160\,\text{Ом}$.
    Какое эквивалентное сопротивление этой цепи получится, если все резисторы подключены последовательно, ответ выразите в омах.
}
\answer{%
    $r = 6400\,\text{Ом}$
}
\solutionspace{20pt}

\tasknumber{7}%
\task{%
    Два резистора сопротивлениями $R_1 = 10\,\text{кОм}$ и $R_2 = 20\,\text{кОм}$ подключены последовательно и на них подано напряжение.
    Определите, какое напряжение них подали, если в цепи идёт $\eli = 3\,\text{мА}$.
    Ответ выразите в вольтах и округлите до целого.
}
\answer{%
    $U = 90\,\text{В}$
}
\solutionspace{20pt}

\tasknumber{8}%
\task{%
    Валя проводит эксперименты c 2 кусками одинаковой алюминиевой проволки, причём второй кусок в 2 длиннее первого.
    В одном из экспериментов Валя подаёт на первый кусок проволки напряжение в 9 раз больше, чем на второй.
    Определите отношение сил тока в двух проволках в этом эксперименте: второй к первой.
    В ответе укажите простую дробь или целое число.
}
\answer{%
    $\frac1{18}$
}
\solutionspace{40pt}

\tasknumber{9}%
\task{%
    В распоряжении Маши имеется 40 одинаковых резисторов, каждый сопротивлением $4\,\text{кОм}$.
    Какое наименьшее эквивалентное сопротивление она может из них получить? Использовать все резисторы при этом не обязательно, ответ укажите в омах.
}
\answer{%
    $r = 100\,\text{Ом}$
}

\variantsplitter

\addpersonalvariant{Андрей Щербаков}

\tasknumber{1}%
\task{%
    Установите каждой букве в соответствие ровно одну цифру и запишите ответ (только цифры, без других символов).

    А) удельное сопротивление проводника, Б) длина проводника, В) электрическое сопротивление резистора.

    1) $R$, 2) $U$, 3) $\rho$, 4) $\mathcal{I}$, 5) $l$.
}
\answer{%
    $351$
}
\solutionspace{20pt}

\tasknumber{2}%
\task{%
    Установите каждой букве в соответствие ровно одну цифру и запишите ответ (только цифры, без других символов).

    А) электрическое сопротивление резистора, Б) сила тока, В) длина проводника.

    1) метр, 2) ватт, 3) ом, 4) сименс, 5) ампер.
}
\answer{%
    $351$
}
\solutionspace{20pt}

\tasknumber{3}%
\task{%
    Установите каждой букве в соответствие ровно одну цифру и запишите ответ (только цифры, без других символов).

    А) эквивалентное сопротивление 2 резисторов (последовательно), Б) эквивалентное сопротивление 2 резисторов (параллельно).

    1) $R = \rho \frac lS$, 2) $\mathcal{I} R = U$, 3) $R_1 + R_2$, 4) $\frac{2R_1R_2}{R_1 + R_2}$, 5) $\frac{R_1R_2}{R_1 + R_2}$.
}
\answer{%
    $35$
}
\solutionspace{20pt}

\tasknumber{4}%
\task{%
    Установите каждой букве в соответствие ровно одну цифру и запишите ответ (только цифры, без других символов).

    А) эквивалентное сопротивление 3 резисторов (параллельно), Б) эквивалентное сопротивление 3 резисторов (последовательно).

    1) $\sqrt{\frac{R_1^2 + R_2^2 + R_3^2}3}$, 2) $\frac 3{\frac 1{R_1} + \frac 1{R_2} + \frac 1{R_3}}$, 3) $\frac{R_1R_2R_3}{R_1R_2 + R_2R_3 + R_3R_1}$, 4) $\frac{R_1R_2R_3}{R_1 + R_2 + R_3}$, 5) $R_1 + R_2 + R_3$.
}
\answer{%
    $35$
}
\solutionspace{20pt}

\tasknumber{5}%
\task{%
    На резистор сопротивлением $10\,\text{Ом}$ подали напряжение $120\,\text{В}$.
    Определите ток, который потечёт через резистор, ответ выразите в амперах.
}
\answer{%
    $\eli = \frac{U}{R} = \frac{120\,\text{В}}{10\,\text{Ом}} = 12\,\text{А}$
}
\solutionspace{20pt}

\tasknumber{6}%
\task{%
    Женя собирает электрическую цепь из $20$ одинаковых резисторов, каждый сопротивлением $160\,\text{Ом}$.
    Какое эквивалентное сопротивление этой цепи получится, если все резисторы подключены параллельно, ответ выразите в омах.
}
\answer{%
    $r = 8\,\text{Ом}$
}
\solutionspace{20pt}

\tasknumber{7}%
\task{%
    Два резистора сопротивлениями $R_1 = 15\,\text{кОм}$ и $R_2 = 2\,\text{кОм}$ подключены последовательно и на них подано напряжение.
    Определите, какое напряжение них подали, если в цепи идёт $\eli = 3\,\text{мА}$.
    Ответ выразите в вольтах и округлите до целого.
}
\answer{%
    $U = 51\,\text{В}$
}
\solutionspace{20pt}

\tasknumber{8}%
\task{%
    Валя проводит эксперименты c 2 кусками одинаковой алюминиевой проволки, причём второй кусок в 9 длиннее первого.
    В одном из экспериментов Валя подаёт на первый кусок проволки напряжение в 10 раз больше, чем на второй.
    Определите отношение сил тока в двух проволках в этом эксперименте: второй к первой.
    В ответе укажите простую дробь или целое число.
}
\answer{%
    $\frac1{90}$
}
\solutionspace{40pt}

\tasknumber{9}%
\task{%
    В распоряжении Маши имеется 40 одинаковых резисторов, каждый сопротивлением $3\,\text{кОм}$.
    Какое наименьшее эквивалентное сопротивление она может из них получить? Использовать все резисторы при этом не обязательно, ответ укажите в омах.
}
\answer{%
    $r = 75\,\text{Ом}$
}

\variantsplitter

\addpersonalvariant{Михаил Ярошевский}

\tasknumber{1}%
\task{%
    Установите каждой букве в соответствие ровно одну цифру и запишите ответ (только цифры, без других символов).

    А) сила тока, Б) удельное сопротивление проводника, В) длина проводника.

    1) $l$, 2) $\mathcal{I}$, 3) $\rho$, 4) $k$, 5) $D$.
}
\answer{%
    $231$
}
\solutionspace{20pt}

\tasknumber{2}%
\task{%
    Установите каждой букве в соответствие ровно одну цифру и запишите ответ (только цифры, без других символов).

    А) сила тока, Б) длина проводника, В) электрическое сопротивление резистора.

    1) ом, 2) ампер, 3) метр, 4) вольт, 5) генри.
}
\answer{%
    $231$
}
\solutionspace{20pt}

\tasknumber{3}%
\task{%
    Установите каждой букве в соответствие ровно одну цифру и запишите ответ (только цифры, без других символов).

    А) электрическое сопротивление резистора, Б) эквивалентное сопротивление 2 резисторов (последовательно).

    1) $R = \rho \frac Sl$, 2) $R = \rho \frac lS$, 3) $R_1 + R_2$, 4) $\frac{R_1R_2}{R_1 + R_2}$, 5) $\frac{R_1 + R_2} 2$.
}
\answer{%
    $23$
}
\solutionspace{20pt}

\tasknumber{4}%
\task{%
    Установите каждой букве в соответствие ровно одну цифру и запишите ответ (только цифры, без других символов).

    А) эквивалентное сопротивление 3 резисторов (параллельно), Б) эквивалентное сопротивление 3 резисторов (последовательно).

    1) $\frac{R_1R_2R_3}{R_1 + R_2 + R_3}$, 2) $\frac{R_1R_2R_3}{R_1R_2 + R_2R_3 + R_3R_1}$, 3) $R_1 + R_2 + R_3$, 4) $\frac 3{\frac 1{R_1} + \frac 1{R_2} + \frac 1{R_3}}$, 5) $\sqrt{\frac{R_1^2 + R_2^2 + R_3^2}3}$.
}
\answer{%
    $23$
}
\solutionspace{20pt}

\tasknumber{5}%
\task{%
    На резистор сопротивлением $15\,\text{Ом}$ подали напряжение $120\,\text{В}$.
    Определите ток, который потечёт через резистор, ответ выразите в амперах.
}
\answer{%
    $\eli = \frac{U}{R} = \frac{120\,\text{В}}{15\,\text{Ом}} = 8\,\text{А}$
}
\solutionspace{20pt}

\tasknumber{6}%
\task{%
    Женя собирает электрическую цепь из $20$ одинаковых резисторов, каждый сопротивлением $240\,\text{Ом}$.
    Какое эквивалентное сопротивление этой цепи получится, если все резисторы подключены последовательно, ответ выразите в омах.
}
\answer{%
    $r = 4800\,\text{Ом}$
}
\solutionspace{20pt}

\tasknumber{7}%
\task{%
    Два резистора сопротивлениями $R_1 = 4\,\text{кОм}$ и $R_2 = 12\,\text{кОм}$ подключены параллельно и на них подано напряжение.
    Определите, какое напряжение них подали, если в цепи идёт $\eli = 3\,\text{мА}$.
    Ответ выразите в вольтах и округлите до целого.
}
\answer{%
    $U = 9\,\text{В}$
}
\solutionspace{20pt}

\tasknumber{8}%
\task{%
    Валя проводит эксперименты c 2 кусками одинаковой стальной проволки, причём второй кусок в 10 длиннее первого.
    В одном из экспериментов Валя подаёт на первый кусок проволки напряжение в 7 раз больше, чем на второй.
    Определите отношение сил тока в двух проволках в этом эксперименте: второй к первой.
    В ответе укажите простую дробь или целое число.
}
\answer{%
    $\frac1{70}$
}
\solutionspace{40pt}

\tasknumber{9}%
\task{%
    В распоряжении Маши имеется 40 одинаковых резисторов, каждый сопротивлением $3\,\text{кОм}$.
    Какое наименьшее эквивалентное сопротивление она может из них получить? Использовать все резисторы при этом не обязательно, ответ укажите в омах.
}
\answer{%
    $r = 75\,\text{Ом}$
}

\variantsplitter

\addpersonalvariant{Алексей Алимпиев}

\tasknumber{1}%
\task{%
    Установите каждой букве в соответствие ровно одну цифру и запишите ответ (только цифры, без других символов).

    А) длина проводника, Б) разность потенциалов, В) удельное сопротивление проводника.

    1) $D$, 2) $l$, 3) $k$, 4) $U$, 5) $\rho$.
}
\answer{%
    $245$
}
\solutionspace{20pt}

\tasknumber{2}%
\task{%
    Установите каждой букве в соответствие ровно одну цифру и запишите ответ (только цифры, без других символов).

    А) разность потенциалов, Б) электрическое сопротивление резистора, В) длина проводника.

    1) сименс, 2) вольт, 3) кулон, 4) ом, 5) метр.
}
\answer{%
    $245$
}
\solutionspace{20pt}

\tasknumber{3}%
\task{%
    Установите каждой букве в соответствие ровно одну цифру и запишите ответ (только цифры, без других символов).

    А) эквивалентное сопротивление 2 резисторов (последовательно), Б) закон Ома.

    1) $R = \rho \frac lS$, 2) $R_1 + R_2$, 3) $\frac{\mathcal{I}} R = U$, 4) $\mathcal{I} R = U$, 5) $\frac{R_1 + R_2} 2$.
}
\answer{%
    $24$
}
\solutionspace{20pt}

\tasknumber{4}%
\task{%
    Установите каждой букве в соответствие ровно одну цифру и запишите ответ (только цифры, без других символов).

    А) эквивалентное сопротивление 3 резисторов (параллельно), Б) эквивалентное сопротивление 3 резисторов (последовательно).

    1) $\sqrt{\frac{R_1^2 + R_2^2 + R_3^2}3}$, 2) $\frac{R_1R_2R_3}{R_1R_2 + R_2R_3 + R_3R_1}$, 3) $\frac 3{\frac 1{R_1} + \frac 1{R_2} + \frac 1{R_3}}$, 4) $R_1 + R_2 + R_3$, 5) $\frac{R_1 + R_2 + R_3}3$.
}
\answer{%
    $24$
}
\solutionspace{20pt}

\tasknumber{5}%
\task{%
    На резистор сопротивлением $20\,\text{Ом}$ подали напряжение $240\,\text{В}$.
    Определите ток, который потечёт через резистор, ответ выразите в амперах.
}
\answer{%
    $\eli = \frac{U}{R} = \frac{240\,\text{В}}{20\,\text{Ом}} = 12\,\text{А}$
}
\solutionspace{20pt}

\tasknumber{6}%
\task{%
    Женя собирает электрическую цепь из $20$ одинаковых резисторов, каждый сопротивлением $240\,\text{Ом}$.
    Какое эквивалентное сопротивление этой цепи получится, если все резисторы подключены параллельно, ответ выразите в омах.
}
\answer{%
    $r = 12\,\text{Ом}$
}
\solutionspace{20pt}

\tasknumber{7}%
\task{%
    Два резистора сопротивлениями $R_1 = 15\,\text{кОм}$ и $R_2 = 20\,\text{кОм}$ подключены параллельно и на них подано напряжение.
    Определите, какое напряжение них подали, если в цепи идёт $\eli = 2\,\text{мА}$.
    Ответ выразите в вольтах и округлите до целого.
}
\answer{%
    $U = 17\,\text{В}$
}
\solutionspace{20pt}

\tasknumber{8}%
\task{%
    Валя проводит эксперименты c 2 кусками одинаковой медной проволки, причём второй кусок в 9 длиннее первого.
    В одном из экспериментов Валя подаёт на первый кусок проволки напряжение в 6 раз больше, чем на второй.
    Определите отношение сил тока в двух проволках в этом эксперименте: второй к первой.
    В ответе укажите простую дробь или целое число.
}
\answer{%
    $\frac1{54}$
}
\solutionspace{40pt}

\tasknumber{9}%
\task{%
    В распоряжении Маши имеется 10 одинаковых резисторов, каждый сопротивлением $4\,\text{кОм}$.
    Какое наименьшее эквивалентное сопротивление она может из них получить? Использовать все резисторы при этом не обязательно, ответ укажите в омах.
}
\answer{%
    $r = 400\,\text{Ом}$
}

\variantsplitter

\addpersonalvariant{Евгений Васин}

\tasknumber{1}%
\task{%
    Установите каждой букве в соответствие ровно одну цифру и запишите ответ (только цифры, без других символов).

    А) площадь поперечного сечения проводника, Б) разность потенциалов, В) электрическое сопротивление резистора.

    1) $\lambda$, 2) $R$, 3) $U$, 4) $l$, 5) $S$.
}
\answer{%
    $532$
}
\solutionspace{20pt}

\tasknumber{2}%
\task{%
    Установите каждой букве в соответствие ровно одну цифру и запишите ответ (только цифры, без других символов).

    А) электрический заряд, Б) сила тока, В) электрическое сопротивление резистора.

    1) метр, 2) ом, 3) ампер, 4) вольт, 5) кулон.
}
\answer{%
    $532$
}
\solutionspace{20pt}

\tasknumber{3}%
\task{%
    Установите каждой букве в соответствие ровно одну цифру и запишите ответ (только цифры, без других символов).

    А) электрическое сопротивление резистора, Б) эквивалентное сопротивление 2 резисторов (последовательно).

    1) $\frac{R_1 + R_2} 2$, 2) $\frac{\mathcal{I}} R = U$, 3) $R_1 + R_2$, 4) $\mathcal{I} R = U$, 5) $R = \rho \frac lS$.
}
\answer{%
    $53$
}
\solutionspace{20pt}

\tasknumber{4}%
\task{%
    Установите каждой букве в соответствие ровно одну цифру и запишите ответ (только цифры, без других символов).

    А) эквивалентное сопротивление 3 резисторов (параллельно), Б) эквивалентное сопротивление 3 резисторов (последовательно).

    1) $\frac{R_1 + R_2 + R_3}3$, 2) $\frac 3{\frac 1{R_1} + \frac 1{R_2} + \frac 1{R_3}}$, 3) $R_1 + R_2 + R_3$, 4) $\frac{R_1R_2R_3}{R_1 + R_2 + R_3}$, 5) $\frac{R_1R_2R_3}{R_1R_2 + R_2R_3 + R_3R_1}$.
}
\answer{%
    $53$
}
\solutionspace{20pt}

\tasknumber{5}%
\task{%
    На резистор сопротивлением $5\,\text{Ом}$ подали напряжение $180\,\text{В}$.
    Определите ток, который потечёт через резистор, ответ выразите в амперах.
}
\answer{%
    $\eli = \frac{U}{R} = \frac{180\,\text{В}}{5\,\text{Ом}} = 36\,\text{А}$
}
\solutionspace{20pt}

\tasknumber{6}%
\task{%
    Женя собирает электрическую цепь из $20$ одинаковых резисторов, каждый сопротивлением $160\,\text{Ом}$.
    Какое эквивалентное сопротивление этой цепи получится, если все резисторы подключены параллельно, ответ выразите в омах.
}
\answer{%
    $r = 8\,\text{Ом}$
}
\solutionspace{20pt}

\tasknumber{7}%
\task{%
    Два резистора сопротивлениями $R_1 = 10\,\text{кОм}$ и $R_2 = 2\,\text{кОм}$ подключены параллельно и на них подано напряжение.
    Определите, какое напряжение них подали, если в цепи идёт $\eli = 2\,\text{мА}$.
    Ответ выразите в вольтах и округлите до целого.
}
\answer{%
    $U = 3\,\text{В}$
}
\solutionspace{20pt}

\tasknumber{8}%
\task{%
    Валя проводит эксперименты c 2 кусками одинаковой стальной проволки, причём второй кусок в 7 длиннее первого.
    В одном из экспериментов Валя подаёт на первый кусок проволки напряжение в 9 раз больше, чем на второй.
    Определите отношение сил тока в двух проволках в этом эксперименте: второй к первой.
    В ответе укажите простую дробь или целое число.
}
\answer{%
    $\frac1{63}$
}
\solutionspace{40pt}

\tasknumber{9}%
\task{%
    В распоряжении Маши имеется 40 одинаковых резисторов, каждый сопротивлением $4\,\text{кОм}$.
    Какое наименьшее эквивалентное сопротивление она может из них получить? Использовать все резисторы при этом не обязательно, ответ укажите в омах.
}
\answer{%
    $r = 100\,\text{Ом}$
}

\variantsplitter

\addpersonalvariant{Вячеслав Волохов}

\tasknumber{1}%
\task{%
    Установите каждой букве в соответствие ровно одну цифру и запишите ответ (только цифры, без других символов).

    А) длина проводника, Б) удельное сопротивление проводника, В) сила тока.

    1) $\mathcal{I}$, 2) $S$, 3) $\rho$, 4) $l$, 5) $D$.
}
\answer{%
    $431$
}
\solutionspace{20pt}

\tasknumber{2}%
\task{%
    Установите каждой букве в соответствие ровно одну цифру и запишите ответ (только цифры, без других символов).

    А) электрическое сопротивление резистора, Б) электрический заряд, В) разность потенциалов.

    1) вольт, 2) генри, 3) кулон, 4) ом, 5) ампер.
}
\answer{%
    $431$
}
\solutionspace{20pt}

\tasknumber{3}%
\task{%
    Установите каждой букве в соответствие ровно одну цифру и запишите ответ (только цифры, без других символов).

    А) электрическое сопротивление резистора, Б) закон Ома.

    1) $\rho = R l S$, 2) $\frac{2R_1R_2}{R_1 + R_2}$, 3) $\mathcal{I} R = U$, 4) $R = \rho \frac lS$, 5) $R = \rho \frac Sl$.
}
\answer{%
    $43$
}
\solutionspace{20pt}

\tasknumber{4}%
\task{%
    Установите каждой букве в соответствие ровно одну цифру и запишите ответ (только цифры, без других символов).

    А) эквивалентное сопротивление 3 резисторов (параллельно), Б) эквивалентное сопротивление 3 резисторов (последовательно).

    1) $\frac{R_1R_2R_3}{R_1 + R_2 + R_3}$, 2) $\sqrt{\frac{R_1^2 + R_2^2 + R_3^2}3}$, 3) $R_1 + R_2 + R_3$, 4) $\frac{R_1R_2R_3}{R_1R_2 + R_2R_3 + R_3R_1}$, 5) $\frac 3{\frac 1{R_1} + \frac 1{R_2} + \frac 1{R_3}}$.
}
\answer{%
    $43$
}
\solutionspace{20pt}

\tasknumber{5}%
\task{%
    На резистор сопротивлением $20\,\text{Ом}$ подали напряжение $180\,\text{В}$.
    Определите ток, который потечёт через резистор, ответ выразите в амперах.
}
\answer{%
    $\eli = \frac{U}{R} = \frac{180\,\text{В}}{20\,\text{Ом}} = 9\,\text{А}$
}
\solutionspace{20pt}

\tasknumber{6}%
\task{%
    Женя собирает электрическую цепь из $40$ одинаковых резисторов, каждый сопротивлением $160\,\text{Ом}$.
    Какое эквивалентное сопротивление этой цепи получится, если все резисторы подключены последовательно, ответ выразите в омах.
}
\answer{%
    $r = 6400\,\text{Ом}$
}
\solutionspace{20pt}

\tasknumber{7}%
\task{%
    Два резистора сопротивлениями $R_1 = 4\,\text{кОм}$ и $R_2 = 20\,\text{кОм}$ подключены параллельно и на них подано напряжение.
    Определите, какое напряжение них подали, если в цепи идёт $\eli = 5\,\text{мА}$.
    Ответ выразите в вольтах и округлите до целого.
}
\answer{%
    $U = 17\,\text{В}$
}
\solutionspace{20pt}

\tasknumber{8}%
\task{%
    Валя проводит эксперименты c 2 кусками одинаковой стальной проволки, причём второй кусок в 3 длиннее первого.
    В одном из экспериментов Валя подаёт на первый кусок проволки напряжение в 5 раз больше, чем на второй.
    Определите отношение сил тока в двух проволках в этом эксперименте: второй к первой.
    В ответе укажите простую дробь или целое число.
}
\answer{%
    $\frac1{15}$
}
\solutionspace{40pt}

\tasknumber{9}%
\task{%
    В распоряжении Маши имеется 40 одинаковых резисторов, каждый сопротивлением $3\,\text{кОм}$.
    Какое наибольшее эквивалентное сопротивление она может из них получить? Использовать все резисторы при этом не обязательно, ответ укажите в омах.
}
\answer{%
    $r = 120000\,\text{Ом}$
}

\variantsplitter

\addpersonalvariant{Герман Говоров}

\tasknumber{1}%
\task{%
    Установите каждой букве в соответствие ровно одну цифру и запишите ответ (только цифры, без других символов).

    А) сила тока, Б) удельное сопротивление проводника, В) разность потенциалов.

    1) $\mathcal{I}$, 2) $S$, 3) $R$, 4) $U$, 5) $\rho$.
}
\answer{%
    $154$
}
\solutionspace{20pt}

\tasknumber{2}%
\task{%
    Установите каждой букве в соответствие ровно одну цифру и запишите ответ (только цифры, без других символов).

    А) сила тока, Б) электрический заряд, В) разность потенциалов.

    1) ампер, 2) метр, 3) сименс, 4) вольт, 5) кулон.
}
\answer{%
    $154$
}
\solutionspace{20pt}

\tasknumber{3}%
\task{%
    Установите каждой букве в соответствие ровно одну цифру и запишите ответ (только цифры, без других символов).

    А) электрическое сопротивление резистора, Б) закон Ома.

    1) $R = \rho \frac lS$, 2) $\rho = R l S$, 3) $\sqrt{R_1R_2}$, 4) $R = \rho \frac Sl$, 5) $\mathcal{I} R = U$.
}
\answer{%
    $15$
}
\solutionspace{20pt}

\tasknumber{4}%
\task{%
    Установите каждой букве в соответствие ровно одну цифру и запишите ответ (только цифры, без других символов).

    А) эквивалентное сопротивление 3 резисторов (последовательно), Б) эквивалентное сопротивление 3 резисторов (параллельно).

    1) $R_1 + R_2 + R_3$, 2) $\sqrt{\frac{R_1^2 + R_2^2 + R_3^2}3}$, 3) $\frac{R_1R_2R_3}{R_1 + R_2 + R_3}$, 4) $\frac{R_1 + R_2 + R_3}3$, 5) $\frac{R_1R_2R_3}{R_1R_2 + R_2R_3 + R_3R_1}$.
}
\answer{%
    $15$
}
\solutionspace{20pt}

\tasknumber{5}%
\task{%
    На резистор сопротивлением $5\,\text{Ом}$ подали напряжение $180\,\text{В}$.
    Определите ток, который потечёт через резистор, ответ выразите в амперах.
}
\answer{%
    $\eli = \frac{U}{R} = \frac{180\,\text{В}}{5\,\text{Ом}} = 36\,\text{А}$
}
\solutionspace{20pt}

\tasknumber{6}%
\task{%
    Женя собирает электрическую цепь из $40$ одинаковых резисторов, каждый сопротивлением $320\,\text{Ом}$.
    Какое эквивалентное сопротивление этой цепи получится, если все резисторы подключены последовательно, ответ выразите в омах.
}
\answer{%
    $r = 12800\,\text{Ом}$
}
\solutionspace{20pt}

\tasknumber{7}%
\task{%
    Два резистора сопротивлениями $R_1 = 15\,\text{кОм}$ и $R_2 = 2\,\text{кОм}$ подключены параллельно и на них подано напряжение.
    Определите, какое напряжение них подали, если в цепи идёт $\eli = 2\,\text{мА}$.
    Ответ выразите в вольтах и округлите до целого.
}
\answer{%
    $U = 4\,\text{В}$
}
\solutionspace{20pt}

\tasknumber{8}%
\task{%
    Валя проводит эксперименты c 2 кусками одинаковой стальной проволки, причём второй кусок в 9 длиннее первого.
    В одном из экспериментов Валя подаёт на первый кусок проволки напряжение в 10 раз больше, чем на второй.
    Определите отношение сил тока в двух проволках в этом эксперименте: второй к первой.
    В ответе укажите простую дробь или целое число.
}
\answer{%
    $\frac1{90}$
}
\solutionspace{40pt}

\tasknumber{9}%
\task{%
    В распоряжении Маши имеется 20 одинаковых резисторов, каждый сопротивлением $4\,\text{кОм}$.
    Какое наибольшее эквивалентное сопротивление она может из них получить? Использовать все резисторы при этом не обязательно, ответ укажите в омах.
}
\answer{%
    $r = 80000\,\text{Ом}$
}

\variantsplitter

\addpersonalvariant{София Журавлёва}

\tasknumber{1}%
\task{%
    Установите каждой букве в соответствие ровно одну цифру и запишите ответ (только цифры, без других символов).

    А) длина проводника, Б) разность потенциалов, В) удельное сопротивление проводника.

    1) $R$, 2) $k$, 3) $\rho$, 4) $l$, 5) $U$.
}
\answer{%
    $453$
}
\solutionspace{20pt}

\tasknumber{2}%
\task{%
    Установите каждой букве в соответствие ровно одну цифру и запишите ответ (только цифры, без других символов).

    А) разность потенциалов, Б) электрическое сопротивление резистора, В) сила тока.

    1) сименс, 2) метр, 3) ампер, 4) вольт, 5) ом.
}
\answer{%
    $453$
}
\solutionspace{20pt}

\tasknumber{3}%
\task{%
    Установите каждой букве в соответствие ровно одну цифру и запишите ответ (только цифры, без других символов).

    А) эквивалентное сопротивление 2 резисторов (последовательно), Б) электрическое сопротивление резистора.

    1) $R = \rho \frac Sl$, 2) $\mathcal{I} R = U$, 3) $\frac{R_1 + R_2} 2$, 4) $R_1 + R_2$, 5) $R = \rho \frac lS$.
}
\answer{%
    $45$
}
\solutionspace{20pt}

\tasknumber{4}%
\task{%
    Установите каждой букве в соответствие ровно одну цифру и запишите ответ (только цифры, без других символов).

    А) эквивалентное сопротивление 3 резисторов (последовательно), Б) эквивалентное сопротивление 3 резисторов (параллельно).

    1) $\frac{R_1 + R_2 + R_3}3$, 2) $\sqrt{\frac{R_1^2 + R_2^2 + R_3^2}3}$, 3) $\frac{R_1R_2R_3}{R_1 + R_2 + R_3}$, 4) $R_1 + R_2 + R_3$, 5) $\frac{R_1R_2R_3}{R_1R_2 + R_2R_3 + R_3R_1}$.
}
\answer{%
    $45$
}
\solutionspace{20pt}

\tasknumber{5}%
\task{%
    На резистор сопротивлением $20\,\text{Ом}$ подали напряжение $180\,\text{В}$.
    Определите ток, который потечёт через резистор, ответ выразите в амперах.
}
\answer{%
    $\eli = \frac{U}{R} = \frac{180\,\text{В}}{20\,\text{Ом}} = 9\,\text{А}$
}
\solutionspace{20pt}

\tasknumber{6}%
\task{%
    Женя собирает электрическую цепь из $20$ одинаковых резисторов, каждый сопротивлением $320\,\text{Ом}$.
    Какое эквивалентное сопротивление этой цепи получится, если все резисторы подключены последовательно, ответ выразите в омах.
}
\answer{%
    $r = 6400\,\text{Ом}$
}
\solutionspace{20pt}

\tasknumber{7}%
\task{%
    Два резистора сопротивлениями $R_1 = 4\,\text{кОм}$ и $R_2 = 20\,\text{кОм}$ подключены параллельно и на них подано напряжение.
    Определите, какое напряжение них подали, если в цепи идёт $\eli = 3\,\text{мА}$.
    Ответ выразите в вольтах и округлите до целого.
}
\answer{%
    $U = 10\,\text{В}$
}
\solutionspace{20pt}

\tasknumber{8}%
\task{%
    Валя проводит эксперименты c 2 кусками одинаковой стальной проволки, причём второй кусок в 10 длиннее первого.
    В одном из экспериментов Валя подаёт на первый кусок проволки напряжение в 6 раз больше, чем на второй.
    Определите отношение сил тока в двух проволках в этом эксперименте: второй к первой.
    В ответе укажите простую дробь или целое число.
}
\answer{%
    $\frac1{60}$
}
\solutionspace{40pt}

\tasknumber{9}%
\task{%
    В распоряжении Маши имеется 20 одинаковых резисторов, каждый сопротивлением $2\,\text{кОм}$.
    Какое наибольшее эквивалентное сопротивление она может из них получить? Использовать все резисторы при этом не обязательно, ответ укажите в омах.
}
\answer{%
    $r = 40000\,\text{Ом}$
}

\variantsplitter

\addpersonalvariant{Константин Козлов}

\tasknumber{1}%
\task{%
    Установите каждой букве в соответствие ровно одну цифру и запишите ответ (только цифры, без других символов).

    А) длина проводника, Б) сила тока, В) электрическое сопротивление резистора.

    1) $R$, 2) $l$, 3) $\mathcal{I}$, 4) $k$, 5) $\rho$.
}
\answer{%
    $231$
}
\solutionspace{20pt}

\tasknumber{2}%
\task{%
    Установите каждой букве в соответствие ровно одну цифру и запишите ответ (только цифры, без других символов).

    А) электрическое сопротивление резистора, Б) сила тока, В) длина проводника.

    1) метр, 2) ом, 3) ампер, 4) генри, 5) кулон.
}
\answer{%
    $231$
}
\solutionspace{20pt}

\tasknumber{3}%
\task{%
    Установите каждой букве в соответствие ровно одну цифру и запишите ответ (только цифры, без других символов).

    А) эквивалентное сопротивление 2 резисторов (параллельно), Б) электрическое сопротивление резистора.

    1) $\sqrt{R_1R_2}$, 2) $\frac{R_1R_2}{R_1 + R_2}$, 3) $R = \rho \frac lS$, 4) $R = \rho \frac Sl$, 5) $\frac{\mathcal{I}} R = U$.
}
\answer{%
    $23$
}
\solutionspace{20pt}

\tasknumber{4}%
\task{%
    Установите каждой букве в соответствие ровно одну цифру и запишите ответ (только цифры, без других символов).

    А) эквивалентное сопротивление 3 резисторов (последовательно), Б) эквивалентное сопротивление 3 резисторов (параллельно).

    1) $\frac{R_1 + R_2 + R_3}3$, 2) $R_1 + R_2 + R_3$, 3) $\frac{R_1R_2R_3}{R_1R_2 + R_2R_3 + R_3R_1}$, 4) $\frac{R_1R_2R_3}{R_1 + R_2 + R_3}$, 5) $\frac 3{\frac 1{R_1} + \frac 1{R_2} + \frac 1{R_3}}$.
}
\answer{%
    $23$
}
\solutionspace{20pt}

\tasknumber{5}%
\task{%
    На резистор сопротивлением $5\,\text{Ом}$ подали напряжение $180\,\text{В}$.
    Определите ток, который потечёт через резистор, ответ выразите в амперах.
}
\answer{%
    $\eli = \frac{U}{R} = \frac{180\,\text{В}}{5\,\text{Ом}} = 36\,\text{А}$
}
\solutionspace{20pt}

\tasknumber{6}%
\task{%
    Женя собирает электрическую цепь из $20$ одинаковых резисторов, каждый сопротивлением $320\,\text{Ом}$.
    Какое эквивалентное сопротивление этой цепи получится, если все резисторы подключены параллельно, ответ выразите в омах.
}
\answer{%
    $r = 16\,\text{Ом}$
}
\solutionspace{20pt}

\tasknumber{7}%
\task{%
    Два резистора сопротивлениями $R_1 = 15\,\text{кОм}$ и $R_2 = 6\,\text{кОм}$ подключены параллельно и на них подано напряжение.
    Определите, какое напряжение них подали, если в цепи идёт $\eli = 3\,\text{мА}$.
    Ответ выразите в вольтах и округлите до целого.
}
\answer{%
    $U = 13\,\text{В}$
}
\solutionspace{20pt}

\tasknumber{8}%
\task{%
    Валя проводит эксперименты c 2 кусками одинаковой алюминиевой проволки, причём второй кусок в 6 длиннее первого.
    В одном из экспериментов Валя подаёт на первый кусок проволки напряжение в 9 раз больше, чем на второй.
    Определите отношение сил тока в двух проволках в этом эксперименте: второй к первой.
    В ответе укажите простую дробь или целое число.
}
\answer{%
    $\frac1{54}$
}
\solutionspace{40pt}

\tasknumber{9}%
\task{%
    В распоряжении Маши имеется 20 одинаковых резисторов, каждый сопротивлением $2\,\text{кОм}$.
    Какое наибольшее эквивалентное сопротивление она может из них получить? Использовать все резисторы при этом не обязательно, ответ укажите в омах.
}
\answer{%
    $r = 40000\,\text{Ом}$
}

\variantsplitter

\addpersonalvariant{Наталья Кравченко}

\tasknumber{1}%
\task{%
    Установите каждой букве в соответствие ровно одну цифру и запишите ответ (только цифры, без других символов).

    А) сила тока, Б) площадь поперечного сечения проводника, В) разность потенциалов.

    1) $l$, 2) $U$, 3) $S$, 4) $\rho$, 5) $\mathcal{I}$.
}
\answer{%
    $532$
}
\solutionspace{20pt}

\tasknumber{2}%
\task{%
    Установите каждой букве в соответствие ровно одну цифру и запишите ответ (только цифры, без других символов).

    А) длина проводника, Б) электрический заряд, В) электрическое сопротивление резистора.

    1) вольт, 2) ом, 3) кулон, 4) ампер, 5) метр.
}
\answer{%
    $532$
}
\solutionspace{20pt}

\tasknumber{3}%
\task{%
    Установите каждой букве в соответствие ровно одну цифру и запишите ответ (только цифры, без других символов).

    А) электрическое сопротивление резистора, Б) эквивалентное сопротивление 2 резисторов (последовательно).

    1) $\mathcal{I} R = U$, 2) $\frac{\mathcal{I}} R = U$, 3) $R_1 + R_2$, 4) $\rho = R l S$, 5) $R = \rho \frac lS$.
}
\answer{%
    $53$
}
\solutionspace{20pt}

\tasknumber{4}%
\task{%
    Установите каждой букве в соответствие ровно одну цифру и запишите ответ (только цифры, без других символов).

    А) эквивалентное сопротивление 3 резисторов (параллельно), Б) эквивалентное сопротивление 3 резисторов (последовательно).

    1) $\sqrt{\frac{R_1^2 + R_2^2 + R_3^2}3}$, 2) $\frac 3{\frac 1{R_1} + \frac 1{R_2} + \frac 1{R_3}}$, 3) $R_1 + R_2 + R_3$, 4) $\frac{R_1 + R_2 + R_3}3$, 5) $\frac{R_1R_2R_3}{R_1R_2 + R_2R_3 + R_3R_1}$.
}
\answer{%
    $53$
}
\solutionspace{20pt}

\tasknumber{5}%
\task{%
    На резистор сопротивлением $20\,\text{Ом}$ подали напряжение $120\,\text{В}$.
    Определите ток, который потечёт через резистор, ответ выразите в амперах.
}
\answer{%
    $\eli = \frac{U}{R} = \frac{120\,\text{В}}{20\,\text{Ом}} = 6\,\text{А}$
}
\solutionspace{20pt}

\tasknumber{6}%
\task{%
    Женя собирает электрическую цепь из $40$ одинаковых резисторов, каждый сопротивлением $160\,\text{Ом}$.
    Какое эквивалентное сопротивление этой цепи получится, если все резисторы подключены последовательно, ответ выразите в омах.
}
\answer{%
    $r = 6400\,\text{Ом}$
}
\solutionspace{20pt}

\tasknumber{7}%
\task{%
    Два резистора сопротивлениями $R_1 = 15\,\text{кОм}$ и $R_2 = 20\,\text{кОм}$ подключены последовательно и на них подано напряжение.
    Определите, какое напряжение них подали, если в цепи идёт $\eli = 2\,\text{мА}$.
    Ответ выразите в вольтах и округлите до целого.
}
\answer{%
    $U = 70\,\text{В}$
}
\solutionspace{20pt}

\tasknumber{8}%
\task{%
    Валя проводит эксперименты c 2 кусками одинаковой медной проволки, причём второй кусок в 7 длиннее первого.
    В одном из экспериментов Валя подаёт на первый кусок проволки напряжение в 9 раз больше, чем на второй.
    Определите отношение сил тока в двух проволках в этом эксперименте: второй к первой.
    В ответе укажите простую дробь или целое число.
}
\answer{%
    $\frac1{63}$
}
\solutionspace{40pt}

\tasknumber{9}%
\task{%
    В распоряжении Маши имеется 10 одинаковых резисторов, каждый сопротивлением $2\,\text{кОм}$.
    Какое наибольшее эквивалентное сопротивление она может из них получить? Использовать все резисторы при этом не обязательно, ответ укажите в омах.
}
\answer{%
    $r = 20000\,\text{Ом}$
}

\variantsplitter

\addpersonalvariant{Матвей Кузьмин}

\tasknumber{1}%
\task{%
    Установите каждой букве в соответствие ровно одну цифру и запишите ответ (только цифры, без других символов).

    А) площадь поперечного сечения проводника, Б) длина проводника, В) сила тока.

    1) $S$, 2) $l$, 3) $R$, 4) $\mathcal{I}$, 5) $U$.
}
\answer{%
    $124$
}
\solutionspace{20pt}

\tasknumber{2}%
\task{%
    Установите каждой букве в соответствие ровно одну цифру и запишите ответ (только цифры, без других символов).

    А) длина проводника, Б) сила тока, В) электрическое сопротивление резистора.

    1) метр, 2) ампер, 3) сименс, 4) ом, 5) ватт.
}
\answer{%
    $124$
}
\solutionspace{20pt}

\tasknumber{3}%
\task{%
    Установите каждой букве в соответствие ровно одну цифру и запишите ответ (только цифры, без других символов).

    А) электрическое сопротивление резистора, Б) эквивалентное сопротивление 2 резисторов (последовательно).

    1) $R = \rho \frac lS$, 2) $R_1 + R_2$, 3) $\frac{R_1R_2}{R_1 + R_2}$, 4) $\sqrt{R_1R_2}$, 5) $R = \rho \frac Sl$.
}
\answer{%
    $12$
}
\solutionspace{20pt}

\tasknumber{4}%
\task{%
    Установите каждой букве в соответствие ровно одну цифру и запишите ответ (только цифры, без других символов).

    А) эквивалентное сопротивление 3 резисторов (последовательно), Б) эквивалентное сопротивление 3 резисторов (параллельно).

    1) $R_1 + R_2 + R_3$, 2) $\frac{R_1R_2R_3}{R_1R_2 + R_2R_3 + R_3R_1}$, 3) $\frac{R_1 + R_2 + R_3}3$, 4) $\frac 3{\frac 1{R_1} + \frac 1{R_2} + \frac 1{R_3}}$, 5) $\frac{R_1R_2R_3}{R_1 + R_2 + R_3}$.
}
\answer{%
    $12$
}
\solutionspace{20pt}

\tasknumber{5}%
\task{%
    На резистор сопротивлением $30\,\text{Ом}$ подали напряжение $120\,\text{В}$.
    Определите ток, который потечёт через резистор, ответ выразите в амперах.
}
\answer{%
    $\eli = \frac{U}{R} = \frac{120\,\text{В}}{30\,\text{Ом}} = 4\,\text{А}$
}
\solutionspace{20pt}

\tasknumber{6}%
\task{%
    Женя собирает электрическую цепь из $40$ одинаковых резисторов, каждый сопротивлением $160\,\text{Ом}$.
    Какое эквивалентное сопротивление этой цепи получится, если все резисторы подключены последовательно, ответ выразите в омах.
}
\answer{%
    $r = 6400\,\text{Ом}$
}
\solutionspace{20pt}

\tasknumber{7}%
\task{%
    Два резистора сопротивлениями $R_1 = 15\,\text{кОм}$ и $R_2 = 20\,\text{кОм}$ подключены параллельно и на них подано напряжение.
    Определите, какое напряжение них подали, если в цепи идёт $\eli = 3\,\text{мА}$.
    Ответ выразите в вольтах и округлите до целого.
}
\answer{%
    $U = 26\,\text{В}$
}
\solutionspace{20pt}

\tasknumber{8}%
\task{%
    Валя проводит эксперименты c 2 кусками одинаковой медной проволки, причём второй кусок в 10 длиннее первого.
    В одном из экспериментов Валя подаёт на первый кусок проволки напряжение в 4 раз больше, чем на второй.
    Определите отношение сил тока в двух проволках в этом эксперименте: второй к первой.
    В ответе укажите простую дробь или целое число.
}
\answer{%
    $\frac1{40}$
}
\solutionspace{40pt}

\tasknumber{9}%
\task{%
    В распоряжении Маши имеется 20 одинаковых резисторов, каждый сопротивлением $2\,\text{кОм}$.
    Какое наибольшее эквивалентное сопротивление она может из них получить? Использовать все резисторы при этом не обязательно, ответ укажите в омах.
}
\answer{%
    $r = 40000\,\text{Ом}$
}

\variantsplitter

\addpersonalvariant{Сергей Малышев}

\tasknumber{1}%
\task{%
    Установите каждой букве в соответствие ровно одну цифру и запишите ответ (только цифры, без других символов).

    А) разность потенциалов, Б) электрическое сопротивление резистора, В) длина проводника.

    1) $R$, 2) $U$, 3) $\mathcal{I}$, 4) $\lambda$, 5) $l$.
}
\answer{%
    $215$
}
\solutionspace{20pt}

\tasknumber{2}%
\task{%
    Установите каждой букве в соответствие ровно одну цифру и запишите ответ (только цифры, без других символов).

    А) разность потенциалов, Б) электрический заряд, В) электрическое сопротивление резистора.

    1) кулон, 2) вольт, 3) сименс, 4) ватт, 5) ом.
}
\answer{%
    $215$
}
\solutionspace{20pt}

\tasknumber{3}%
\task{%
    Установите каждой букве в соответствие ровно одну цифру и запишите ответ (только цифры, без других символов).

    А) закон Ома, Б) эквивалентное сопротивление 2 резисторов (последовательно).

    1) $R_1 + R_2$, 2) $\mathcal{I} R = U$, 3) $\frac{2R_1R_2}{R_1 + R_2}$, 4) $\sqrt{R_1R_2}$, 5) $\rho = R l S$.
}
\answer{%
    $21$
}
\solutionspace{20pt}

\tasknumber{4}%
\task{%
    Установите каждой букве в соответствие ровно одну цифру и запишите ответ (только цифры, без других символов).

    А) эквивалентное сопротивление 3 резисторов (последовательно), Б) эквивалентное сопротивление 3 резисторов (параллельно).

    1) $\frac{R_1R_2R_3}{R_1R_2 + R_2R_3 + R_3R_1}$, 2) $R_1 + R_2 + R_3$, 3) $\sqrt{\frac{R_1^2 + R_2^2 + R_3^2}3}$, 4) $\frac{R_1R_2R_3}{R_1 + R_2 + R_3}$, 5) $\frac 3{\frac 1{R_1} + \frac 1{R_2} + \frac 1{R_3}}$.
}
\answer{%
    $21$
}
\solutionspace{20pt}

\tasknumber{5}%
\task{%
    На резистор сопротивлением $10\,\text{Ом}$ подали напряжение $240\,\text{В}$.
    Определите ток, который потечёт через резистор, ответ выразите в амперах.
}
\answer{%
    $\eli = \frac{U}{R} = \frac{240\,\text{В}}{10\,\text{Ом}} = 24\,\text{А}$
}
\solutionspace{20pt}

\tasknumber{6}%
\task{%
    Женя собирает электрическую цепь из $20$ одинаковых резисторов, каждый сопротивлением $240\,\text{Ом}$.
    Какое эквивалентное сопротивление этой цепи получится, если все резисторы подключены параллельно, ответ выразите в омах.
}
\answer{%
    $r = 12\,\text{Ом}$
}
\solutionspace{20pt}

\tasknumber{7}%
\task{%
    Два резистора сопротивлениями $R_1 = 10\,\text{кОм}$ и $R_2 = 20\,\text{кОм}$ подключены последовательно и на них подано напряжение.
    Определите, какое напряжение них подали, если в цепи идёт $\eli = 5\,\text{мА}$.
    Ответ выразите в вольтах и округлите до целого.
}
\answer{%
    $U = 150\,\text{В}$
}
\solutionspace{20pt}

\tasknumber{8}%
\task{%
    Валя проводит эксперименты c 2 кусками одинаковой алюминиевой проволки, причём второй кусок в 5 длиннее первого.
    В одном из экспериментов Валя подаёт на первый кусок проволки напряжение в 5 раз больше, чем на второй.
    Определите отношение сил тока в двух проволках в этом эксперименте: второй к первой.
    В ответе укажите простую дробь или целое число.
}
\answer{%
    $\frac1{25}$
}
\solutionspace{40pt}

\tasknumber{9}%
\task{%
    В распоряжении Маши имеется 10 одинаковых резисторов, каждый сопротивлением $4\,\text{кОм}$.
    Какое наименьшее эквивалентное сопротивление она может из них получить? Использовать все резисторы при этом не обязательно, ответ укажите в омах.
}
\answer{%
    $r = 400\,\text{Ом}$
}

\variantsplitter

\addpersonalvariant{Алина Полканова}

\tasknumber{1}%
\task{%
    Установите каждой букве в соответствие ровно одну цифру и запишите ответ (только цифры, без других символов).

    А) электрическое сопротивление резистора, Б) сила тока, В) площадь поперечного сечения проводника.

    1) $U$, 2) $\mathcal{I}$, 3) $S$, 4) $l$, 5) $R$.
}
\answer{%
    $523$
}
\solutionspace{20pt}

\tasknumber{2}%
\task{%
    Установите каждой букве в соответствие ровно одну цифру и запишите ответ (только цифры, без других символов).

    А) сила тока, Б) длина проводника, В) электрическое сопротивление резистора.

    1) сименс, 2) метр, 3) ом, 4) кулон, 5) ампер.
}
\answer{%
    $523$
}
\solutionspace{20pt}

\tasknumber{3}%
\task{%
    Установите каждой букве в соответствие ровно одну цифру и запишите ответ (только цифры, без других символов).

    А) эквивалентное сопротивление 2 резисторов (последовательно), Б) электрическое сопротивление резистора.

    1) $R = \rho \frac Sl$, 2) $R = \rho \frac lS$, 3) $\frac{\mathcal{I}} R = U$, 4) $\mathcal{I} R = U$, 5) $R_1 + R_2$.
}
\answer{%
    $52$
}
\solutionspace{20pt}

\tasknumber{4}%
\task{%
    Установите каждой букве в соответствие ровно одну цифру и запишите ответ (только цифры, без других символов).

    А) эквивалентное сопротивление 3 резисторов (последовательно), Б) эквивалентное сопротивление 3 резисторов (параллельно).

    1) $\frac{R_1R_2R_3}{R_1 + R_2 + R_3}$, 2) $\frac{R_1R_2R_3}{R_1R_2 + R_2R_3 + R_3R_1}$, 3) $\frac 3{\frac 1{R_1} + \frac 1{R_2} + \frac 1{R_3}}$, 4) $\sqrt{\frac{R_1^2 + R_2^2 + R_3^2}3}$, 5) $R_1 + R_2 + R_3$.
}
\answer{%
    $52$
}
\solutionspace{20pt}

\tasknumber{5}%
\task{%
    На резистор сопротивлением $3\,\text{Ом}$ подали напряжение $120\,\text{В}$.
    Определите ток, который потечёт через резистор, ответ выразите в амперах.
}
\answer{%
    $\eli = \frac{U}{R} = \frac{120\,\text{В}}{3\,\text{Ом}} = 40\,\text{А}$
}
\solutionspace{20pt}

\tasknumber{6}%
\task{%
    Женя собирает электрическую цепь из $40$ одинаковых резисторов, каждый сопротивлением $240\,\text{Ом}$.
    Какое эквивалентное сопротивление этой цепи получится, если все резисторы подключены параллельно, ответ выразите в омах.
}
\answer{%
    $r = 6\,\text{Ом}$
}
\solutionspace{20pt}

\tasknumber{7}%
\task{%
    Два резистора сопротивлениями $R_1 = 4\,\text{кОм}$ и $R_2 = 12\,\text{кОм}$ подключены параллельно и на них подано напряжение.
    Определите, какое напряжение них подали, если в цепи идёт $\eli = 2\,\text{мА}$.
    Ответ выразите в вольтах и округлите до целого.
}
\answer{%
    $U = 6\,\text{В}$
}
\solutionspace{20pt}

\tasknumber{8}%
\task{%
    Валя проводит эксперименты c 2 кусками одинаковой алюминиевой проволки, причём второй кусок в 6 длиннее первого.
    В одном из экспериментов Валя подаёт на первый кусок проволки напряжение в 7 раз больше, чем на второй.
    Определите отношение сил тока в двух проволках в этом эксперименте: второй к первой.
    В ответе укажите простую дробь или целое число.
}
\answer{%
    $\frac1{42}$
}
\solutionspace{40pt}

\tasknumber{9}%
\task{%
    В распоряжении Маши имеется 20 одинаковых резисторов, каждый сопротивлением $3\,\text{кОм}$.
    Какое наименьшее эквивалентное сопротивление она может из них получить? Использовать все резисторы при этом не обязательно, ответ укажите в омах.
}
\answer{%
    $r = 150\,\text{Ом}$
}

\variantsplitter

\addpersonalvariant{Сергей Пономарёв}

\tasknumber{1}%
\task{%
    Установите каждой букве в соответствие ровно одну цифру и запишите ответ (только цифры, без других символов).

    А) сила тока, Б) длина проводника, В) разность потенциалов.

    1) $l$, 2) $U$, 3) $R$, 4) $\mathcal{I}$, 5) $\lambda$.
}
\answer{%
    $412$
}
\solutionspace{20pt}

\tasknumber{2}%
\task{%
    Установите каждой букве в соответствие ровно одну цифру и запишите ответ (только цифры, без других символов).

    А) длина проводника, Б) разность потенциалов, В) электрическое сопротивление резистора.

    1) вольт, 2) ом, 3) ампер, 4) метр, 5) сименс.
}
\answer{%
    $412$
}
\solutionspace{20pt}

\tasknumber{3}%
\task{%
    Установите каждой букве в соответствие ровно одну цифру и запишите ответ (только цифры, без других символов).

    А) закон Ома, Б) эквивалентное сопротивление 2 резисторов (последовательно).

    1) $R_1 + R_2$, 2) $\frac{R_1R_2}{R_1 + R_2}$, 3) $\frac{\mathcal{I}} R = U$, 4) $\mathcal{I} R = U$, 5) $\sqrt{R_1R_2}$.
}
\answer{%
    $41$
}
\solutionspace{20pt}

\tasknumber{4}%
\task{%
    Установите каждой букве в соответствие ровно одну цифру и запишите ответ (только цифры, без других символов).

    А) эквивалентное сопротивление 3 резисторов (параллельно), Б) эквивалентное сопротивление 3 резисторов (последовательно).

    1) $R_1 + R_2 + R_3$, 2) $\frac{R_1R_2R_3}{R_1 + R_2 + R_3}$, 3) $\frac 3{\frac 1{R_1} + \frac 1{R_2} + \frac 1{R_3}}$, 4) $\frac{R_1R_2R_3}{R_1R_2 + R_2R_3 + R_3R_1}$, 5) $\sqrt{\frac{R_1^2 + R_2^2 + R_3^2}3}$.
}
\answer{%
    $41$
}
\solutionspace{20pt}

\tasknumber{5}%
\task{%
    На резистор сопротивлением $3\,\text{Ом}$ подали напряжение $240\,\text{В}$.
    Определите ток, который потечёт через резистор, ответ выразите в амперах.
}
\answer{%
    $\eli = \frac{U}{R} = \frac{240\,\text{В}}{3\,\text{Ом}} = 80\,\text{А}$
}
\solutionspace{20pt}

\tasknumber{6}%
\task{%
    Женя собирает электрическую цепь из $10$ одинаковых резисторов, каждый сопротивлением $160\,\text{Ом}$.
    Какое эквивалентное сопротивление этой цепи получится, если все резисторы подключены параллельно, ответ выразите в омах.
}
\answer{%
    $r = 16\,\text{Ом}$
}
\solutionspace{20pt}

\tasknumber{7}%
\task{%
    Два резистора сопротивлениями $R_1 = 4\,\text{кОм}$ и $R_2 = 2\,\text{кОм}$ подключены параллельно и на них подано напряжение.
    Определите, какое напряжение них подали, если в цепи идёт $\eli = 3\,\text{мА}$.
    Ответ выразите в вольтах и округлите до целого.
}
\answer{%
    $U = 4\,\text{В}$
}
\solutionspace{20pt}

\tasknumber{8}%
\task{%
    Валя проводит эксперименты c 2 кусками одинаковой стальной проволки, причём второй кусок в 4 длиннее первого.
    В одном из экспериментов Валя подаёт на первый кусок проволки напряжение в 5 раз больше, чем на второй.
    Определите отношение сил тока в двух проволках в этом эксперименте: второй к первой.
    В ответе укажите простую дробь или целое число.
}
\answer{%
    $\frac1{20}$
}
\solutionspace{40pt}

\tasknumber{9}%
\task{%
    В распоряжении Маши имеется 40 одинаковых резисторов, каждый сопротивлением $2\,\text{кОм}$.
    Какое наибольшее эквивалентное сопротивление она может из них получить? Использовать все резисторы при этом не обязательно, ответ укажите в омах.
}
\answer{%
    $r = 80000\,\text{Ом}$
}

\variantsplitter

\addpersonalvariant{Егор Свистушкин}

\tasknumber{1}%
\task{%
    Установите каждой букве в соответствие ровно одну цифру и запишите ответ (только цифры, без других символов).

    А) длина проводника, Б) площадь поперечного сечения проводника, В) электрическое сопротивление резистора.

    1) $S$, 2) $\mathcal{I}$, 3) $R$, 4) $l$, 5) $\lambda$.
}
\answer{%
    $413$
}
\solutionspace{20pt}

\tasknumber{2}%
\task{%
    Установите каждой букве в соответствие ровно одну цифру и запишите ответ (только цифры, без других символов).

    А) электрический заряд, Б) разность потенциалов, В) длина проводника.

    1) вольт, 2) ом, 3) метр, 4) кулон, 5) ампер.
}
\answer{%
    $413$
}
\solutionspace{20pt}

\tasknumber{3}%
\task{%
    Установите каждой букве в соответствие ровно одну цифру и запишите ответ (только цифры, без других символов).

    А) электрическое сопротивление резистора, Б) закон Ома.

    1) $\mathcal{I} R = U$, 2) $R = \rho \frac Sl$, 3) $\frac{R_1R_2}{R_1 + R_2}$, 4) $R = \rho \frac lS$, 5) $\frac{R_1 + R_2} 2$.
}
\answer{%
    $41$
}
\solutionspace{20pt}

\tasknumber{4}%
\task{%
    Установите каждой букве в соответствие ровно одну цифру и запишите ответ (только цифры, без других символов).

    А) эквивалентное сопротивление 3 резисторов (последовательно), Б) эквивалентное сопротивление 3 резисторов (параллельно).

    1) $\frac{R_1R_2R_3}{R_1R_2 + R_2R_3 + R_3R_1}$, 2) $\frac{R_1 + R_2 + R_3}3$, 3) $\sqrt{\frac{R_1^2 + R_2^2 + R_3^2}3}$, 4) $R_1 + R_2 + R_3$, 5) $\frac 3{\frac 1{R_1} + \frac 1{R_2} + \frac 1{R_3}}$.
}
\answer{%
    $41$
}
\solutionspace{20pt}

\tasknumber{5}%
\task{%
    На резистор сопротивлением $20\,\text{Ом}$ подали напряжение $240\,\text{В}$.
    Определите ток, который потечёт через резистор, ответ выразите в амперах.
}
\answer{%
    $\eli = \frac{U}{R} = \frac{240\,\text{В}}{20\,\text{Ом}} = 12\,\text{А}$
}
\solutionspace{20pt}

\tasknumber{6}%
\task{%
    Женя собирает электрическую цепь из $10$ одинаковых резисторов, каждый сопротивлением $320\,\text{Ом}$.
    Какое эквивалентное сопротивление этой цепи получится, если все резисторы подключены последовательно, ответ выразите в омах.
}
\answer{%
    $r = 3200\,\text{Ом}$
}
\solutionspace{20pt}

\tasknumber{7}%
\task{%
    Два резистора сопротивлениями $R_1 = 10\,\text{кОм}$ и $R_2 = 6\,\text{кОм}$ подключены параллельно и на них подано напряжение.
    Определите, какое напряжение них подали, если в цепи идёт $\eli = 2\,\text{мА}$.
    Ответ выразите в вольтах и округлите до целого.
}
\answer{%
    $U = 8\,\text{В}$
}
\solutionspace{20pt}

\tasknumber{8}%
\task{%
    Валя проводит эксперименты c 2 кусками одинаковой медной проволки, причём второй кусок в 8 длиннее первого.
    В одном из экспериментов Валя подаёт на первый кусок проволки напряжение в 10 раз больше, чем на второй.
    Определите отношение сил тока в двух проволках в этом эксперименте: второй к первой.
    В ответе укажите простую дробь или целое число.
}
\answer{%
    $\frac1{80}$
}
\solutionspace{40pt}

\tasknumber{9}%
\task{%
    В распоряжении Маши имеется 40 одинаковых резисторов, каждый сопротивлением $3\,\text{кОм}$.
    Какое наибольшее эквивалентное сопротивление она может из них получить? Использовать все резисторы при этом не обязательно, ответ укажите в омах.
}
\answer{%
    $r = 120000\,\text{Ом}$
}

\variantsplitter

\addpersonalvariant{Дмитрий Соколов}

\tasknumber{1}%
\task{%
    Установите каждой букве в соответствие ровно одну цифру и запишите ответ (только цифры, без других символов).

    А) площадь поперечного сечения проводника, Б) электрическое сопротивление резистора, В) удельное сопротивление проводника.

    1) $\mathcal{I}$, 2) $R$, 3) $\rho$, 4) $D$, 5) $S$.
}
\answer{%
    $523$
}
\solutionspace{20pt}

\tasknumber{2}%
\task{%
    Установите каждой букве в соответствие ровно одну цифру и запишите ответ (только цифры, без других символов).

    А) сила тока, Б) разность потенциалов, В) длина проводника.

    1) генри, 2) вольт, 3) метр, 4) кулон, 5) ампер.
}
\answer{%
    $523$
}
\solutionspace{20pt}

\tasknumber{3}%
\task{%
    Установите каждой букве в соответствие ровно одну цифру и запишите ответ (только цифры, без других символов).

    А) электрическое сопротивление резистора, Б) эквивалентное сопротивление 2 резисторов (последовательно).

    1) $\rho = R l S$, 2) $R_1 + R_2$, 3) $\sqrt{R_1R_2}$, 4) $\frac{R_1R_2}{R_1 + R_2}$, 5) $R = \rho \frac lS$.
}
\answer{%
    $52$
}
\solutionspace{20pt}

\tasknumber{4}%
\task{%
    Установите каждой букве в соответствие ровно одну цифру и запишите ответ (только цифры, без других символов).

    А) эквивалентное сопротивление 3 резисторов (последовательно), Б) эквивалентное сопротивление 3 резисторов (параллельно).

    1) $\frac{R_1 + R_2 + R_3}3$, 2) $\frac{R_1R_2R_3}{R_1R_2 + R_2R_3 + R_3R_1}$, 3) $\sqrt{\frac{R_1^2 + R_2^2 + R_3^2}3}$, 4) $\frac{R_1R_2R_3}{R_1 + R_2 + R_3}$, 5) $R_1 + R_2 + R_3$.
}
\answer{%
    $52$
}
\solutionspace{20pt}

\tasknumber{5}%
\task{%
    На резистор сопротивлением $15\,\text{Ом}$ подали напряжение $180\,\text{В}$.
    Определите ток, который потечёт через резистор, ответ выразите в амперах.
}
\answer{%
    $\eli = \frac{U}{R} = \frac{180\,\text{В}}{15\,\text{Ом}} = 12\,\text{А}$
}
\solutionspace{20pt}

\tasknumber{6}%
\task{%
    Женя собирает электрическую цепь из $10$ одинаковых резисторов, каждый сопротивлением $160\,\text{Ом}$.
    Какое эквивалентное сопротивление этой цепи получится, если все резисторы подключены последовательно, ответ выразите в омах.
}
\answer{%
    $r = 1600\,\text{Ом}$
}
\solutionspace{20pt}

\tasknumber{7}%
\task{%
    Два резистора сопротивлениями $R_1 = 10\,\text{кОм}$ и $R_2 = 12\,\text{кОм}$ подключены параллельно и на них подано напряжение.
    Определите, какое напряжение них подали, если в цепи идёт $\eli = 2\,\text{мА}$.
    Ответ выразите в вольтах и округлите до целого.
}
\answer{%
    $U = 11\,\text{В}$
}
\solutionspace{20pt}

\tasknumber{8}%
\task{%
    Валя проводит эксперименты c 2 кусками одинаковой алюминиевой проволки, причём второй кусок в 3 длиннее первого.
    В одном из экспериментов Валя подаёт на первый кусок проволки напряжение в 6 раз больше, чем на второй.
    Определите отношение сил тока в двух проволках в этом эксперименте: второй к первой.
    В ответе укажите простую дробь или целое число.
}
\answer{%
    $\frac1{18}$
}
\solutionspace{40pt}

\tasknumber{9}%
\task{%
    В распоряжении Маши имеется 20 одинаковых резисторов, каждый сопротивлением $3\,\text{кОм}$.
    Какое наибольшее эквивалентное сопротивление она может из них получить? Использовать все резисторы при этом не обязательно, ответ укажите в омах.
}
\answer{%
    $r = 60000\,\text{Ом}$
}

\variantsplitter

\addpersonalvariant{Арсений Трофимов}

\tasknumber{1}%
\task{%
    Установите каждой букве в соответствие ровно одну цифру и запишите ответ (только цифры, без других символов).

    А) площадь поперечного сечения проводника, Б) электрическое сопротивление резистора, В) сила тока.

    1) $\rho$, 2) $\mathcal{I}$, 3) $D$, 4) $S$, 5) $R$.
}
\answer{%
    $452$
}
\solutionspace{20pt}

\tasknumber{2}%
\task{%
    Установите каждой букве в соответствие ровно одну цифру и запишите ответ (только цифры, без других символов).

    А) сила тока, Б) электрическое сопротивление резистора, В) разность потенциалов.

    1) метр, 2) вольт, 3) сименс, 4) ампер, 5) ом.
}
\answer{%
    $452$
}
\solutionspace{20pt}

\tasknumber{3}%
\task{%
    Установите каждой букве в соответствие ровно одну цифру и запишите ответ (только цифры, без других символов).

    А) эквивалентное сопротивление 2 резисторов (последовательно), Б) закон Ома.

    1) $R = \rho \frac Sl$, 2) $\frac{R_1R_2}{R_1 + R_2}$, 3) $\frac{\mathcal{I}} R = U$, 4) $R_1 + R_2$, 5) $\mathcal{I} R = U$.
}
\answer{%
    $45$
}
\solutionspace{20pt}

\tasknumber{4}%
\task{%
    Установите каждой букве в соответствие ровно одну цифру и запишите ответ (только цифры, без других символов).

    А) эквивалентное сопротивление 3 резисторов (последовательно), Б) эквивалентное сопротивление 3 резисторов (параллельно).

    1) $\frac{R_1R_2R_3}{R_1 + R_2 + R_3}$, 2) $\frac 3{\frac 1{R_1} + \frac 1{R_2} + \frac 1{R_3}}$, 3) $\sqrt{\frac{R_1^2 + R_2^2 + R_3^2}3}$, 4) $R_1 + R_2 + R_3$, 5) $\frac{R_1R_2R_3}{R_1R_2 + R_2R_3 + R_3R_1}$.
}
\answer{%
    $45$
}
\solutionspace{20pt}

\tasknumber{5}%
\task{%
    На резистор сопротивлением $20\,\text{Ом}$ подали напряжение $180\,\text{В}$.
    Определите ток, который потечёт через резистор, ответ выразите в амперах.
}
\answer{%
    $\eli = \frac{U}{R} = \frac{180\,\text{В}}{20\,\text{Ом}} = 9\,\text{А}$
}
\solutionspace{20pt}

\tasknumber{6}%
\task{%
    Женя собирает электрическую цепь из $40$ одинаковых резисторов, каждый сопротивлением $160\,\text{Ом}$.
    Какое эквивалентное сопротивление этой цепи получится, если все резисторы подключены параллельно, ответ выразите в омах.
}
\answer{%
    $r = 4\,\text{Ом}$
}
\solutionspace{20pt}

\tasknumber{7}%
\task{%
    Два резистора сопротивлениями $R_1 = 4\,\text{кОм}$ и $R_2 = 20\,\text{кОм}$ подключены параллельно и на них подано напряжение.
    Определите, какое напряжение них подали, если в цепи идёт $\eli = 3\,\text{мА}$.
    Ответ выразите в вольтах и округлите до целого.
}
\answer{%
    $U = 10\,\text{В}$
}
\solutionspace{20pt}

\tasknumber{8}%
\task{%
    Валя проводит эксперименты c 2 кусками одинаковой медной проволки, причём второй кусок в 6 длиннее первого.
    В одном из экспериментов Валя подаёт на первый кусок проволки напряжение в 10 раз больше, чем на второй.
    Определите отношение сил тока в двух проволках в этом эксперименте: второй к первой.
    В ответе укажите простую дробь или целое число.
}
\answer{%
    $\frac1{60}$
}
\solutionspace{40pt}

\tasknumber{9}%
\task{%
    В распоряжении Маши имеется 20 одинаковых резисторов, каждый сопротивлением $3\,\text{кОм}$.
    Какое наименьшее эквивалентное сопротивление она может из них получить? Использовать все резисторы при этом не обязательно, ответ укажите в омах.
}
\answer{%
    $r = 150\,\text{Ом}$
}
% autogenerated
