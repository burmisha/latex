\setdate{23~апреля~2021}
\setclass{9«М»}

\addpersonalvariant{Михаил Бурмистров}

\tasknumber{1}%
\task{%
    Определите число нейтронов в атоме $\ce{^{1}_{1}{H}}$.
}
\answer{%
    $Z = 1$ протонов и столько же электронов $A = 1$ нуклонов, $A - Z = 0$ нейтронов.
    Ответ: 0
}

\tasknumber{2}%
\task{%
    Определите число протонов в атоме $\ce{^{3}_{2}{He}}$.
}
\answer{%
    $Z = 2$ протонов и столько же электронов $A = 3$ нуклонов, $A - Z = 1$ нейтронов.
    Ответ: 2
}

\tasknumber{3}%
\task{%
    Определите число электронов в атоме $\ce{^{7}_{3}{Li}}$.
}
\answer{%
    $Z = 3$ протонов и столько же электронов $A = 7$ нуклонов, $A - Z = 4$ нейтронов.
    Ответ: 3
}

\tasknumber{4}%
\task{%
    Определите число нейтронов в атоме $\ce{^{10}_{5}{B}}$.
}
\answer{%
    $Z = 5$ протонов и столько же электронов $A = 10$ нуклонов, $A - Z = 5$ нейтронов.
    Ответ: 5
}

\tasknumber{5}%
\task{%
    Определите число нейтронов в атоме $\ce{^{21}_{10}{Ne}}$.
}
\answer{%
    $Z = 10$ протонов и столько же электронов $A = 21$ нуклонов, $A - Z = 11$ нейтронов.
    Ответ: 11
}

\tasknumber{6}%
\task{%
    Определите число нейтронов в атоме $\ce{^{25}_{12}{Mg}}$.
}
\answer{%
    $Z = 12$ протонов и столько же электронов $A = 25$ нуклонов, $A - Z = 13$ нейтронов.
    Ответ: 13
}

\tasknumber{7}%
\task{%
    Определите число нейтронов в атоме $\ce{^{41}_{19}{K}}$.
}
\answer{%
    $Z = 19$ протонов и столько же электронов $A = 41$ нуклонов, $A - Z = 22$ нейтронов.
    Ответ: 22
}

\tasknumber{8}%
\task{%
    Определите число нейтронов в атоме $\ce{^{46}_{20}{Ca}}$.
}
\answer{%
    $Z = 20$ протонов и столько же электронов $A = 46$ нуклонов, $A - Z = 26$ нейтронов.
    Ответ: 26
}

\tasknumber{9}%
\task{%
    Определите число протонов в атоме $\ce{^{72}_{33}{As}}$.
}
\answer{%
    $Z = 33$ протонов и столько же электронов $A = 72$ нуклонов, $A - Z = 39$ нейтронов.
    Ответ: 33
}

\tasknumber{10}%
\task{%
    Определите число нуклонов в атоме $\ce{^{109}_{47}{Ag}}$.
}
\answer{%
    $Z = 47$ протонов и столько же электронов $A = 109$ нуклонов, $A - Z = 62$ нейтронов.
    Ответ: 109
}

\tasknumber{11}%
\task{%
    Определите число электронов в атоме $\ce{^{119}_{51}{Sb}}$.
}
\answer{%
    $Z = 51$ протонов и столько же электронов $A = 119$ нуклонов, $A - Z = 68$ нейтронов.
    Ответ: 51
}

\tasknumber{12}%
\task{%
    Определите число электронов в атоме $\ce{^{129}_{53}{I}}$.
}
\answer{%
    $Z = 53$ протонов и столько же электронов $A = 129$ нуклонов, $A - Z = 76$ нейтронов.
    Ответ: 53
}

\tasknumber{13}%
\task{%
    Определите число электронов в атоме $\text{тритий-3}$.
}
\answer{%
    $Z = 1$ протонов и столько же электронов, $A = 3$ нуклонов, $A - Z = 2$ нейтронов.
    Ответ: 1
}

\tasknumber{14}%
\task{%
    Определите число протонов в атоме $\text{гелий-4}$.
}
\answer{%
    $Z = 2$ протонов и столько же электронов, $A = 4$ нуклонов, $A - Z = 2$ нейтронов.
    Ответ: 2
}

\tasknumber{15}%
\task{%
    Определите число протонов в атоме $\text{бор-10}$.
}
\answer{%
    $Z = 5$ протонов и столько же электронов, $A = 10$ нуклонов, $A - Z = 5$ нейтронов.
    Ответ: 5
}

\tasknumber{16}%
\task{%
    Определите число нейтронов в атоме $\text{бор-10}$.
}
\answer{%
    $Z = 5$ протонов и столько же электронов, $A = 10$ нуклонов, $A - Z = 5$ нейтронов.
    Ответ: 5
}

\tasknumber{17}%
\task{%
    Определите число нуклонов в атоме $\text{неон-20}$.
}
\answer{%
    $Z = 10$ протонов и столько же электронов, $A = 20$ нуклонов, $A - Z = 10$ нейтронов.
    Ответ: 20
}

\tasknumber{18}%
\task{%
    Определите число электронов в атоме $\text{натрий-22}$.
}
\answer{%
    $Z = 11$ протонов и столько же электронов, $A = 22$ нуклонов, $A - Z = 11$ нейтронов.
    Ответ: 11
}

\tasknumber{19}%
\task{%
    Определите число нейтронов в атоме $\text{аргон-38}$.
}
\answer{%
    $Z = 18$ протонов и столько же электронов, $A = 38$ нуклонов, $A - Z = 20$ нейтронов.
    Ответ: 20
}

\tasknumber{20}%
\task{%
    Определите число электронов в атоме $\text{кальций-47}$.
}
\answer{%
    $Z = 20$ протонов и столько же электронов, $A = 47$ нуклонов, $A - Z = 27$ нейтронов.
    Ответ: 20
}

\tasknumber{21}%
\task{%
    Определите число нуклонов в атоме $\text{мышьяк-71}$.
}
\answer{%
    $Z = 33$ протонов и столько же электронов, $A = 71$ нуклонов, $A - Z = 38$ нейтронов.
    Ответ: 71
}

\tasknumber{22}%
\task{%
    Определите число нейтронов в атоме $\text{серебро-109}$.
}
\answer{%
    $Z = 47$ протонов и столько же электронов, $A = 109$ нуклонов, $A - Z = 62$ нейтронов.
    Ответ: 62
}

\tasknumber{23}%
\task{%
    Определите число нейтронов в атоме $\text{сурьма-123}$.
}
\answer{%
    $Z = 51$ протонов и столько же электронов, $A = 123$ нуклонов, $A - Z = 72$ нейтронов.
    Ответ: 72
}

\tasknumber{24}%
\task{%
    Определите число нейтронов в атоме $\text{самарий-148}$.
}
\answer{%
    $Z = 62$ протонов и столько же электронов, $A = 148$ нуклонов, $A - Z = 86$ нейтронов.
    Ответ: 86
}

\variantsplitter

\addpersonalvariant{Артём Глембо}

\tasknumber{1}%
\task{%
    Определите число электронов в атоме $\ce{^{1}_{1}{H}}$.
}
\answer{%
    $Z = 1$ протонов и столько же электронов $A = 1$ нуклонов, $A - Z = 0$ нейтронов.
    Ответ: 1
}

\tasknumber{2}%
\task{%
    Определите число нуклонов в атоме $\ce{^{8}_{2}{He}}$.
}
\answer{%
    $Z = 2$ протонов и столько же электронов $A = 8$ нуклонов, $A - Z = 6$ нейтронов.
    Ответ: 8
}

\tasknumber{3}%
\task{%
    Определите число нейтронов в атоме $\ce{^{6}_{3}{Li}}$.
}
\answer{%
    $Z = 3$ протонов и столько же электронов $A = 6$ нуклонов, $A - Z = 3$ нейтронов.
    Ответ: 3
}

\tasknumber{4}%
\task{%
    Определите число нейтронов в атоме $\ce{^{14}_{6}{C}}$.
}
\answer{%
    $Z = 6$ протонов и столько же электронов $A = 14$ нуклонов, $A - Z = 8$ нейтронов.
    Ответ: 8
}

\tasknumber{5}%
\task{%
    Определите число нейтронов в атоме $\ce{^{20}_{10}{Ne}}$.
}
\answer{%
    $Z = 10$ протонов и столько же электронов $A = 20$ нуклонов, $A - Z = 10$ нейтронов.
    Ответ: 10
}

\tasknumber{6}%
\task{%
    Определите число нуклонов в атоме $\ce{^{22}_{11}{Na}}$.
}
\answer{%
    $Z = 11$ протонов и столько же электронов $A = 22$ нуклонов, $A - Z = 11$ нейтронов.
    Ответ: 22
}

\tasknumber{7}%
\task{%
    Определите число электронов в атоме $\ce{^{38}_{18}{Ar}}$.
}
\answer{%
    $Z = 18$ протонов и столько же электронов $A = 38$ нуклонов, $A - Z = 20$ нейтронов.
    Ответ: 18
}

\tasknumber{8}%
\task{%
    Определите число нейтронов в атоме $\ce{^{48}_{20}{Ca}}$.
}
\answer{%
    $Z = 20$ протонов и столько же электронов $A = 48$ нуклонов, $A - Z = 28$ нейтронов.
    Ответ: 28
}

\tasknumber{9}%
\task{%
    Определите число нейтронов в атоме $\ce{^{73}_{33}{As}}$.
}
\answer{%
    $Z = 33$ протонов и столько же электронов $A = 73$ нуклонов, $A - Z = 40$ нейтронов.
    Ответ: 40
}

\tasknumber{10}%
\task{%
    Определите число нейтронов в атоме $\ce{^{76}_{34}{Se}}$.
}
\answer{%
    $Z = 34$ протонов и столько же электронов $A = 76$ нуклонов, $A - Z = 42$ нейтронов.
    Ответ: 42
}

\tasknumber{11}%
\task{%
    Определите число протонов в атоме $\ce{^{124}_{51}{Sb}}$.
}
\answer{%
    $Z = 51$ протонов и столько же электронов $A = 124$ нуклонов, $A - Z = 73$ нейтронов.
    Ответ: 51
}

\tasknumber{12}%
\task{%
    Определите число нейтронов в атоме $\ce{^{150}_{60}{Nd}}$.
}
\answer{%
    $Z = 60$ протонов и столько же электронов $A = 150$ нуклонов, $A - Z = 90$ нейтронов.
    Ответ: 90
}

\tasknumber{13}%
\task{%
    Определите число нейтронов в атоме $\text{дейтерий-2}$.
}
\answer{%
    $Z = 1$ протонов и столько же электронов, $A = 2$ нуклонов, $A - Z = 1$ нейтронов.
    Ответ: 1
}

\tasknumber{14}%
\task{%
    Определите число нейтронов в атоме $\text{гелий-6}$.
}
\answer{%
    $Z = 2$ протонов и столько же электронов, $A = 6$ нуклонов, $A - Z = 4$ нейтронов.
    Ответ: 4
}

\tasknumber{15}%
\task{%
    Определите число нейтронов в атоме $\text{литий-7}$.
}
\answer{%
    $Z = 3$ протонов и столько же электронов, $A = 7$ нуклонов, $A - Z = 4$ нейтронов.
    Ответ: 4
}

\tasknumber{16}%
\task{%
    Определите число электронов в атоме $\text{азот-14}$.
}
\answer{%
    $Z = 7$ протонов и столько же электронов, $A = 14$ нуклонов, $A - Z = 7$ нейтронов.
    Ответ: 7
}

\tasknumber{17}%
\task{%
    Определите число нейтронов в атоме $\text{кислород-17}$.
}
\answer{%
    $Z = 8$ протонов и столько же электронов, $A = 17$ нуклонов, $A - Z = 9$ нейтронов.
    Ответ: 9
}

\tasknumber{18}%
\task{%
    Определите число протонов в атоме $\text{натрий-23}$.
}
\answer{%
    $Z = 11$ протонов и столько же электронов, $A = 23$ нуклонов, $A - Z = 12$ нейтронов.
    Ответ: 11
}

\tasknumber{19}%
\task{%
    Определите число электронов в атоме $\text{аргон-37}$.
}
\answer{%
    $Z = 18$ протонов и столько же электронов, $A = 37$ нуклонов, $A - Z = 19$ нейтронов.
    Ответ: 18
}

\tasknumber{20}%
\task{%
    Определите число нейтронов в атоме $\text{кальций-44}$.
}
\answer{%
    $Z = 20$ протонов и столько же электронов, $A = 44$ нуклонов, $A - Z = 24$ нейтронов.
    Ответ: 24
}

\tasknumber{21}%
\task{%
    Определите число протонов в атоме $\text{мышьяк-72}$.
}
\answer{%
    $Z = 33$ протонов и столько же электронов, $A = 72$ нуклонов, $A - Z = 39$ нейтронов.
    Ответ: 33
}

\tasknumber{22}%
\task{%
    Определите число протонов в атоме $\text{индий-111}$.
}
\answer{%
    $Z = 49$ протонов и столько же электронов, $A = 111$ нуклонов, $A - Z = 62$ нейтронов.
    Ответ: 49
}

\tasknumber{23}%
\task{%
    Определите число нуклонов в атоме $\text{сурьма-119}$.
}
\answer{%
    $Z = 51$ протонов и столько же электронов, $A = 119$ нуклонов, $A - Z = 68$ нейтронов.
    Ответ: 119
}

\tasknumber{24}%
\task{%
    Определите число протонов в атоме $\text{цезий-137}$.
}
\answer{%
    $Z = 55$ протонов и столько же электронов, $A = 137$ нуклонов, $A - Z = 82$ нейтронов.
    Ответ: 55
}

\variantsplitter

\addpersonalvariant{Наталья Гончарова}

\tasknumber{1}%
\task{%
    Определите число нейтронов в атоме $\ce{^{3}_{1}{T}}$.
}
\answer{%
    $Z = 1$ протонов и столько же электронов $A = 3$ нуклонов, $A - Z = 2$ нейтронов.
    Ответ: 2
}

\tasknumber{2}%
\task{%
    Определите число нейтронов в атоме $\ce{^{6}_{2}{He}}$.
}
\answer{%
    $Z = 2$ протонов и столько же электронов $A = 6$ нуклонов, $A - Z = 4$ нейтронов.
    Ответ: 4
}

\tasknumber{3}%
\task{%
    Определите число нуклонов в атоме $\ce{^{10}_{4}{Be}}$.
}
\answer{%
    $Z = 4$ протонов и столько же электронов $A = 10$ нуклонов, $A - Z = 6$ нейтронов.
    Ответ: 10
}

\tasknumber{4}%
\task{%
    Определите число электронов в атоме $\ce{^{14}_{6}{C}}$.
}
\answer{%
    $Z = 6$ протонов и столько же электронов $A = 14$ нуклонов, $A - Z = 8$ нейтронов.
    Ответ: 6
}

\tasknumber{5}%
\task{%
    Определите число нуклонов в атоме $\ce{^{18}_{8}{O}}$.
}
\answer{%
    $Z = 8$ протонов и столько же электронов $A = 18$ нуклонов, $A - Z = 10$ нейтронов.
    Ответ: 18
}

\tasknumber{6}%
\task{%
    Определите число электронов в атоме $\ce{^{23}_{11}{Na}}$.
}
\answer{%
    $Z = 11$ протонов и столько же электронов $A = 23$ нуклонов, $A - Z = 12$ нейтронов.
    Ответ: 11
}

\tasknumber{7}%
\task{%
    Определите число протонов в атоме $\ce{^{40}_{20}{Ca}}$.
}
\answer{%
    $Z = 20$ протонов и столько же электронов $A = 40$ нуклонов, $A - Z = 20$ нейтронов.
    Ответ: 20
}

\tasknumber{8}%
\task{%
    Определите число протонов в атоме $\ce{^{46}_{20}{Ca}}$.
}
\answer{%
    $Z = 20$ протонов и столько же электронов $A = 46$ нуклонов, $A - Z = 26$ нейтронов.
    Ответ: 20
}

\tasknumber{9}%
\task{%
    Определите число нейтронов в атоме $\ce{^{76}_{33}{As}}$.
}
\answer{%
    $Z = 33$ протонов и столько же электронов $A = 76$ нуклонов, $A - Z = 43$ нейтронов.
    Ответ: 43
}

\tasknumber{10}%
\task{%
    Определите число электронов в атоме $\ce{^{79}_{34}{Se}}$.
}
\answer{%
    $Z = 34$ протонов и столько же электронов $A = 79$ нуклонов, $A - Z = 45$ нейтронов.
    Ответ: 34
}

\tasknumber{11}%
\task{%
    Определите число электронов в атоме $\ce{^{125}_{53}{I}}$.
}
\answer{%
    $Z = 53$ протонов и столько же электронов $A = 125$ нуклонов, $A - Z = 72$ нейтронов.
    Ответ: 53
}

\tasknumber{12}%
\task{%
    Определите число нуклонов в атоме $\ce{^{148}_{60}{Nd}}$.
}
\answer{%
    $Z = 60$ протонов и столько же электронов $A = 148$ нуклонов, $A - Z = 88$ нейтронов.
    Ответ: 148
}

\tasknumber{13}%
\task{%
    Определите число электронов в атоме $\text{дейтерий-2}$.
}
\answer{%
    $Z = 1$ протонов и столько же электронов, $A = 2$ нуклонов, $A - Z = 1$ нейтронов.
    Ответ: 1
}

\tasknumber{14}%
\task{%
    Определите число протонов в атоме $\text{гелий-4}$.
}
\answer{%
    $Z = 2$ протонов и столько же электронов, $A = 4$ нуклонов, $A - Z = 2$ нейтронов.
    Ответ: 2
}

\tasknumber{15}%
\task{%
    Определите число нейтронов в атоме $\text{бериллий-9}$.
}
\answer{%
    $Z = 4$ протонов и столько же электронов, $A = 9$ нуклонов, $A - Z = 5$ нейтронов.
    Ответ: 5
}

\tasknumber{16}%
\task{%
    Определите число протонов в атоме $\text{углерод-13}$.
}
\answer{%
    $Z = 6$ протонов и столько же электронов, $A = 13$ нуклонов, $A - Z = 7$ нейтронов.
    Ответ: 6
}

\tasknumber{17}%
\task{%
    Определите число электронов в атоме $\text{неон-21}$.
}
\answer{%
    $Z = 10$ протонов и столько же электронов, $A = 21$ нуклонов, $A - Z = 11$ нейтронов.
    Ответ: 10
}

\tasknumber{18}%
\task{%
    Определите число протонов в атоме $\text{аргон-36}$.
}
\answer{%
    $Z = 18$ протонов и столько же электронов, $A = 36$ нуклонов, $A - Z = 18$ нейтронов.
    Ответ: 18
}

\tasknumber{19}%
\task{%
    Определите число нейтронов в атоме $\text{аргон-38}$.
}
\answer{%
    $Z = 18$ протонов и столько же электронов, $A = 38$ нуклонов, $A - Z = 20$ нейтронов.
    Ответ: 20
}

\tasknumber{20}%
\task{%
    Определите число протонов в атоме $\text{медь-67}$.
}
\answer{%
    $Z = 29$ протонов и столько же электронов, $A = 67$ нуклонов, $A - Z = 38$ нейтронов.
    Ответ: 29
}

\tasknumber{21}%
\task{%
    Определите число протонов в атоме $\text{мышьяк-73}$.
}
\answer{%
    $Z = 33$ протонов и столько же электронов, $A = 73$ нуклонов, $A - Z = 40$ нейтронов.
    Ответ: 33
}

\tasknumber{22}%
\task{%
    Определите число протонов в атоме $\text{индий-115}$.
}
\answer{%
    $Z = 49$ протонов и столько же электронов, $A = 115$ нуклонов, $A - Z = 66$ нейтронов.
    Ответ: 49
}

\tasknumber{23}%
\task{%
    Определите число нуклонов в атоме $\text{йод-124}$.
}
\answer{%
    $Z = 53$ протонов и столько же электронов, $A = 124$ нуклонов, $A - Z = 71$ нейтронов.
    Ответ: 124
}

\tasknumber{24}%
\task{%
    Определите число нейтронов в атоме $\text{йод-131}$.
}
\answer{%
    $Z = 53$ протонов и столько же электронов, $A = 131$ нуклонов, $A - Z = 78$ нейтронов.
    Ответ: 78
}

\variantsplitter

\addpersonalvariant{Файёзбек Касымов}

\tasknumber{1}%
\task{%
    Определите число нуклонов в атоме $\ce{^{2}_{1}{D}}$.
}
\answer{%
    $Z = 1$ протонов и столько же электронов $A = 2$ нуклонов, $A - Z = 1$ нейтронов.
    Ответ: 2
}

\tasknumber{2}%
\task{%
    Определите число протонов в атоме $\ce{^{3}_{2}{He}}$.
}
\answer{%
    $Z = 2$ протонов и столько же электронов $A = 3$ нуклонов, $A - Z = 1$ нейтронов.
    Ответ: 2
}

\tasknumber{3}%
\task{%
    Определите число электронов в атоме $\ce{^{7}_{4}{Be}}$.
}
\answer{%
    $Z = 4$ протонов и столько же электронов $A = 7$ нуклонов, $A - Z = 3$ нейтронов.
    Ответ: 4
}

\tasknumber{4}%
\task{%
    Определите число протонов в атоме $\ce{^{10}_{5}{B}}$.
}
\answer{%
    $Z = 5$ протонов и столько же электронов $A = 10$ нуклонов, $A - Z = 5$ нейтронов.
    Ответ: 5
}

\tasknumber{5}%
\task{%
    Определите число протонов в атоме $\ce{^{16}_{8}{O}}$.
}
\answer{%
    $Z = 8$ протонов и столько же электронов $A = 16$ нуклонов, $A - Z = 8$ нейтронов.
    Ответ: 8
}

\tasknumber{6}%
\task{%
    Определите число нейтронов в атоме $\ce{^{33}_{15}{P}}$.
}
\answer{%
    $Z = 15$ протонов и столько же электронов $A = 33$ нуклонов, $A - Z = 18$ нейтронов.
    Ответ: 18
}

\tasknumber{7}%
\task{%
    Определите число протонов в атоме $\ce{^{37}_{18}{Ar}}$.
}
\answer{%
    $Z = 18$ протонов и столько же электронов $A = 37$ нуклонов, $A - Z = 19$ нейтронов.
    Ответ: 18
}

\tasknumber{8}%
\task{%
    Определите число нуклонов в атоме $\ce{^{42}_{20}{Ca}}$.
}
\answer{%
    $Z = 20$ протонов и столько же электронов $A = 42$ нуклонов, $A - Z = 22$ нейтронов.
    Ответ: 42
}

\tasknumber{9}%
\task{%
    Определите число протонов в атоме $\ce{^{67}_{29}{Cu}}$.
}
\answer{%
    $Z = 29$ протонов и столько же электронов $A = 67$ нуклонов, $A - Z = 38$ нейтронов.
    Ответ: 29
}

\tasknumber{10}%
\task{%
    Определите число электронов в атоме $\ce{^{80}_{34}{Se}}$.
}
\answer{%
    $Z = 34$ протонов и столько же электронов $A = 80$ нуклонов, $A - Z = 46$ нейтронов.
    Ответ: 34
}

\tasknumber{11}%
\task{%
    Определите число электронов в атоме $\ce{^{121}_{51}{Sb}}$.
}
\answer{%
    $Z = 51$ протонов и столько же электронов $A = 121$ нуклонов, $A - Z = 70$ нейтронов.
    Ответ: 51
}

\tasknumber{12}%
\task{%
    Определите число электронов в атоме $\ce{^{145}_{60}{Nd}}$.
}
\answer{%
    $Z = 60$ протонов и столько же электронов $A = 145$ нуклонов, $A - Z = 85$ нейтронов.
    Ответ: 60
}

\tasknumber{13}%
\task{%
    Определите число электронов в атоме $\text{тритий-3}$.
}
\answer{%
    $Z = 1$ протонов и столько же электронов, $A = 3$ нуклонов, $A - Z = 2$ нейтронов.
    Ответ: 1
}

\tasknumber{14}%
\task{%
    Определите число электронов в атоме $\text{гелий-8}$.
}
\answer{%
    $Z = 2$ протонов и столько же электронов, $A = 8$ нуклонов, $A - Z = 6$ нейтронов.
    Ответ: 2
}

\tasknumber{15}%
\task{%
    Определите число нейтронов в атоме $\text{бериллий-9}$.
}
\answer{%
    $Z = 4$ протонов и столько же электронов, $A = 9$ нуклонов, $A - Z = 5$ нейтронов.
    Ответ: 5
}

\tasknumber{16}%
\task{%
    Определите число электронов в атоме $\text{азот-14}$.
}
\answer{%
    $Z = 7$ протонов и столько же электронов, $A = 14$ нуклонов, $A - Z = 7$ нейтронов.
    Ответ: 7
}

\tasknumber{17}%
\task{%
    Определите число электронов в атоме $\text{фтор-19}$.
}
\answer{%
    $Z = 9$ протонов и столько же электронов, $A = 19$ нуклонов, $A - Z = 10$ нейтронов.
    Ответ: 9
}

\tasknumber{18}%
\task{%
    Определите число нейтронов в атоме $\text{натрий-22}$.
}
\answer{%
    $Z = 11$ протонов и столько же электронов, $A = 22$ нуклонов, $A - Z = 11$ нейтронов.
    Ответ: 11
}

\tasknumber{19}%
\task{%
    Определите число протонов в атоме $\text{калий-40}$.
}
\answer{%
    $Z = 19$ протонов и столько же электронов, $A = 40$ нуклонов, $A - Z = 21$ нейтронов.
    Ответ: 19
}

\tasknumber{20}%
\task{%
    Определите число электронов в атоме $\text{кальций-42}$.
}
\answer{%
    $Z = 20$ протонов и столько же электронов, $A = 42$ нуклонов, $A - Z = 22$ нейтронов.
    Ответ: 20
}

\tasknumber{21}%
\task{%
    Определите число нейтронов в атоме $\text{селен-76}$.
}
\answer{%
    $Z = 34$ протонов и столько же электронов, $A = 76$ нуклонов, $A - Z = 42$ нейтронов.
    Ответ: 42
}

\tasknumber{22}%
\task{%
    Определите число электронов в атоме $\text{серебро-111}$.
}
\answer{%
    $Z = 47$ протонов и столько же электронов, $A = 111$ нуклонов, $A - Z = 64$ нейтронов.
    Ответ: 47
}

\tasknumber{23}%
\task{%
    Определите число нейтронов в атоме $\text{сурьма-124}$.
}
\answer{%
    $Z = 51$ протонов и столько же электронов, $A = 124$ нуклонов, $A - Z = 73$ нейтронов.
    Ответ: 73
}

\tasknumber{24}%
\task{%
    Определите число протонов в атоме $\text{самарий-147}$.
}
\answer{%
    $Z = 62$ протонов и столько же электронов, $A = 147$ нуклонов, $A - Z = 85$ нейтронов.
    Ответ: 62
}

\variantsplitter

\addpersonalvariant{Александр Козинец}

\tasknumber{1}%
\task{%
    Определите число электронов в атоме $\ce{^{1}_{1}{H}}$.
}
\answer{%
    $Z = 1$ протонов и столько же электронов $A = 1$ нуклонов, $A - Z = 0$ нейтронов.
    Ответ: 1
}

\tasknumber{2}%
\task{%
    Определите число нейтронов в атоме $\ce{^{8}_{2}{He}}$.
}
\answer{%
    $Z = 2$ протонов и столько же электронов $A = 8$ нуклонов, $A - Z = 6$ нейтронов.
    Ответ: 6
}

\tasknumber{3}%
\task{%
    Определите число нуклонов в атоме $\ce{^{10}_{4}{Be}}$.
}
\answer{%
    $Z = 4$ протонов и столько же электронов $A = 10$ нуклонов, $A - Z = 6$ нейтронов.
    Ответ: 10
}

\tasknumber{4}%
\task{%
    Определите число протонов в атоме $\ce{^{12}_{6}{C}}$.
}
\answer{%
    $Z = 6$ протонов и столько же электронов $A = 12$ нуклонов, $A - Z = 6$ нейтронов.
    Ответ: 6
}

\tasknumber{5}%
\task{%
    Определите число нуклонов в атоме $\ce{^{18}_{8}{O}}$.
}
\answer{%
    $Z = 8$ протонов и столько же электронов $A = 18$ нуклонов, $A - Z = 10$ нейтронов.
    Ответ: 18
}

\tasknumber{6}%
\task{%
    Определите число протонов в атоме $\ce{^{22}_{11}{Na}}$.
}
\answer{%
    $Z = 11$ протонов и столько же электронов $A = 22$ нуклонов, $A - Z = 11$ нейтронов.
    Ответ: 11
}

\tasknumber{7}%
\task{%
    Определите число нейтронов в атоме $\ce{^{36}_{18}{Ar}}$.
}
\answer{%
    $Z = 18$ протонов и столько же электронов $A = 36$ нуклонов, $A - Z = 18$ нейтронов.
    Ответ: 18
}

\tasknumber{8}%
\task{%
    Определите число протонов в атоме $\ce{^{46}_{20}{Ca}}$.
}
\answer{%
    $Z = 20$ протонов и столько же электронов $A = 46$ нуклонов, $A - Z = 26$ нейтронов.
    Ответ: 20
}

\tasknumber{9}%
\task{%
    Определите число нуклонов в атоме $\ce{^{71}_{33}{As}}$.
}
\answer{%
    $Z = 33$ протонов и столько же электронов $A = 71$ нуклонов, $A - Z = 38$ нейтронов.
    Ответ: 71
}

\tasknumber{10}%
\task{%
    Определите число нуклонов в атоме $\ce{^{107}_{47}{Ag}}$.
}
\answer{%
    $Z = 47$ протонов и столько же электронов $A = 107$ нуклонов, $A - Z = 60$ нейтронов.
    Ответ: 107
}

\tasknumber{11}%
\task{%
    Определите число нуклонов в атоме $\ce{^{126}_{53}{I}}$.
}
\answer{%
    $Z = 53$ протонов и столько же электронов $A = 126$ нуклонов, $A - Z = 73$ нейтронов.
    Ответ: 126
}

\tasknumber{12}%
\task{%
    Определите число нуклонов в атоме $\ce{^{148}_{62}{Sm}}$.
}
\answer{%
    $Z = 62$ протонов и столько же электронов $A = 148$ нуклонов, $A - Z = 86$ нейтронов.
    Ответ: 148
}

\tasknumber{13}%
\task{%
    Определите число нуклонов в атоме $\text{дейтерий-2}$.
}
\answer{%
    $Z = 1$ протонов и столько же электронов, $A = 2$ нуклонов, $A - Z = 1$ нейтронов.
    Ответ: 2
}

\tasknumber{14}%
\task{%
    Определите число электронов в атоме $\text{литий-6}$.
}
\answer{%
    $Z = 3$ протонов и столько же электронов, $A = 6$ нуклонов, $A - Z = 3$ нейтронов.
    Ответ: 3
}

\tasknumber{15}%
\task{%
    Определите число нуклонов в атоме $\text{бор-10}$.
}
\answer{%
    $Z = 5$ протонов и столько же электронов, $A = 10$ нуклонов, $A - Z = 5$ нейтронов.
    Ответ: 10
}

\tasknumber{16}%
\task{%
    Определите число протонов в атоме $\text{азот-15}$.
}
\answer{%
    $Z = 7$ протонов и столько же электронов, $A = 15$ нуклонов, $A - Z = 8$ нейтронов.
    Ответ: 7
}

\tasknumber{17}%
\task{%
    Определите число электронов в атоме $\text{неон-20}$.
}
\answer{%
    $Z = 10$ протонов и столько же электронов, $A = 20$ нуклонов, $A - Z = 10$ нейтронов.
    Ответ: 10
}

\tasknumber{18}%
\task{%
    Определите число нуклонов в атоме $\text{натрий-23}$.
}
\answer{%
    $Z = 11$ протонов и столько же электронов, $A = 23$ нуклонов, $A - Z = 12$ нейтронов.
    Ответ: 23
}

\tasknumber{19}%
\task{%
    Определите число нейтронов в атоме $\text{аргон-37}$.
}
\answer{%
    $Z = 18$ протонов и столько же электронов, $A = 37$ нуклонов, $A - Z = 19$ нейтронов.
    Ответ: 19
}

\tasknumber{20}%
\task{%
    Определите число протонов в атоме $\text{кальций-45}$.
}
\answer{%
    $Z = 20$ протонов и столько же электронов, $A = 45$ нуклонов, $A - Z = 25$ нейтронов.
    Ответ: 20
}

\tasknumber{21}%
\task{%
    Определите число протонов в атоме $\text{мышьяк-76}$.
}
\answer{%
    $Z = 33$ протонов и столько же электронов, $A = 76$ нуклонов, $A - Z = 43$ нейтронов.
    Ответ: 33
}

\tasknumber{22}%
\task{%
    Определите число электронов в атоме $\text{индий-115}$.
}
\answer{%
    $Z = 49$ протонов и столько же электронов, $A = 115$ нуклонов, $A - Z = 66$ нейтронов.
    Ответ: 49
}

\tasknumber{23}%
\task{%
    Определите число протонов в атоме $\text{сурьма-126}$.
}
\answer{%
    $Z = 51$ протонов и столько же электронов, $A = 126$ нуклонов, $A - Z = 75$ нейтронов.
    Ответ: 51
}

\tasknumber{24}%
\task{%
    Определите число нуклонов в атоме $\text{йод-129}$.
}
\answer{%
    $Z = 53$ протонов и столько же электронов, $A = 129$ нуклонов, $A - Z = 76$ нейтронов.
    Ответ: 129
}

\variantsplitter

\addpersonalvariant{Андрей Куликовский}

\tasknumber{1}%
\task{%
    Определите число нейтронов в атоме $\ce{^{3}_{1}{T}}$.
}
\answer{%
    $Z = 1$ протонов и столько же электронов $A = 3$ нуклонов, $A - Z = 2$ нейтронов.
    Ответ: 2
}

\tasknumber{2}%
\task{%
    Определите число нуклонов в атоме $\ce{^{8}_{2}{He}}$.
}
\answer{%
    $Z = 2$ протонов и столько же электронов $A = 8$ нуклонов, $A - Z = 6$ нейтронов.
    Ответ: 8
}

\tasknumber{3}%
\task{%
    Определите число нейтронов в атоме $\ce{^{6}_{3}{Li}}$.
}
\answer{%
    $Z = 3$ протонов и столько же электронов $A = 6$ нуклонов, $A - Z = 3$ нейтронов.
    Ответ: 3
}

\tasknumber{4}%
\task{%
    Определите число нуклонов в атоме $\ce{^{10}_{5}{B}}$.
}
\answer{%
    $Z = 5$ протонов и столько же электронов $A = 10$ нуклонов, $A - Z = 5$ нейтронов.
    Ответ: 10
}

\tasknumber{5}%
\task{%
    Определите число протонов в атоме $\ce{^{17}_{8}{O}}$.
}
\answer{%
    $Z = 8$ протонов и столько же электронов $A = 17$ нуклонов, $A - Z = 9$ нейтронов.
    Ответ: 8
}

\tasknumber{6}%
\task{%
    Определите число нуклонов в атоме $\ce{^{23}_{11}{Na}}$.
}
\answer{%
    $Z = 11$ протонов и столько же электронов $A = 23$ нуклонов, $A - Z = 12$ нейтронов.
    Ответ: 23
}

\tasknumber{7}%
\task{%
    Определите число нейтронов в атоме $\ce{^{40}_{20}{Ca}}$.
}
\answer{%
    $Z = 20$ протонов и столько же электронов $A = 40$ нуклонов, $A - Z = 20$ нейтронов.
    Ответ: 20
}

\tasknumber{8}%
\task{%
    Определите число электронов в атоме $\ce{^{47}_{20}{Ca}}$.
}
\answer{%
    $Z = 20$ протонов и столько же электронов $A = 47$ нуклонов, $A - Z = 27$ нейтронов.
    Ответ: 20
}

\tasknumber{9}%
\task{%
    Определите число электронов в атоме $\ce{^{76}_{33}{As}}$.
}
\answer{%
    $Z = 33$ протонов и столько же электронов $A = 76$ нуклонов, $A - Z = 43$ нейтронов.
    Ответ: 33
}

\tasknumber{10}%
\task{%
    Определите число нейтронов в атоме $\ce{^{109}_{47}{Ag}}$.
}
\answer{%
    $Z = 47$ протонов и столько же электронов $A = 109$ нуклонов, $A - Z = 62$ нейтронов.
    Ответ: 62
}

\tasknumber{11}%
\task{%
    Определите число электронов в атоме $\ce{^{121}_{51}{Sb}}$.
}
\answer{%
    $Z = 51$ протонов и столько же электронов $A = 121$ нуклонов, $A - Z = 70$ нейтронов.
    Ответ: 51
}

\tasknumber{12}%
\task{%
    Определите число нейтронов в атоме $\ce{^{129}_{53}{I}}$.
}
\answer{%
    $Z = 53$ протонов и столько же электронов $A = 129$ нуклонов, $A - Z = 76$ нейтронов.
    Ответ: 76
}

\tasknumber{13}%
\task{%
    Определите число нейтронов в атоме $\text{тритий-3}$.
}
\answer{%
    $Z = 1$ протонов и столько же электронов, $A = 3$ нуклонов, $A - Z = 2$ нейтронов.
    Ответ: 2
}

\tasknumber{14}%
\task{%
    Определите число нуклонов в атоме $\text{литий-6}$.
}
\answer{%
    $Z = 3$ протонов и столько же электронов, $A = 6$ нуклонов, $A - Z = 3$ нейтронов.
    Ответ: 6
}

\tasknumber{15}%
\task{%
    Определите число электронов в атоме $\text{бериллий-9}$.
}
\answer{%
    $Z = 4$ протонов и столько же электронов, $A = 9$ нуклонов, $A - Z = 5$ нейтронов.
    Ответ: 4
}

\tasknumber{16}%
\task{%
    Определите число протонов в атоме $\text{углерод-13}$.
}
\answer{%
    $Z = 6$ протонов и столько же электронов, $A = 13$ нуклонов, $A - Z = 7$ нейтронов.
    Ответ: 6
}

\tasknumber{17}%
\task{%
    Определите число нейтронов в атоме $\text{кислород-18}$.
}
\answer{%
    $Z = 8$ протонов и столько же электронов, $A = 18$ нуклонов, $A - Z = 10$ нейтронов.
    Ответ: 10
}

\tasknumber{18}%
\task{%
    Определите число электронов в атоме $\text{магний-25}$.
}
\answer{%
    $Z = 12$ протонов и столько же электронов, $A = 25$ нуклонов, $A - Z = 13$ нейтронов.
    Ответ: 12
}

\tasknumber{19}%
\task{%
    Определите число нуклонов в атоме $\text{кальций-41}$.
}
\answer{%
    $Z = 20$ протонов и столько же электронов, $A = 41$ нуклонов, $A - Z = 21$ нейтронов.
    Ответ: 41
}

\tasknumber{20}%
\task{%
    Определите число нуклонов в атоме $\text{кальций-46}$.
}
\answer{%
    $Z = 20$ протонов и столько же электронов, $A = 46$ нуклонов, $A - Z = 26$ нейтронов.
    Ответ: 46
}

\tasknumber{21}%
\task{%
    Определите число нуклонов в атоме $\text{мышьяк-73}$.
}
\answer{%
    $Z = 33$ протонов и столько же электронов, $A = 73$ нуклонов, $A - Z = 40$ нейтронов.
    Ответ: 73
}

\tasknumber{22}%
\task{%
    Определите число нуклонов в атоме $\text{индий-115}$.
}
\answer{%
    $Z = 49$ протонов и столько же электронов, $A = 115$ нуклонов, $A - Z = 66$ нейтронов.
    Ответ: 115
}

\tasknumber{23}%
\task{%
    Определите число нейтронов в атоме $\text{сурьма-130}$.
}
\answer{%
    $Z = 51$ протонов и столько же электронов, $A = 130$ нуклонов, $A - Z = 79$ нейтронов.
    Ответ: 79
}

\tasknumber{24}%
\task{%
    Определите число протонов в атоме $\text{неодим-143}$.
}
\answer{%
    $Z = 60$ протонов и столько же электронов, $A = 143$ нуклонов, $A - Z = 83$ нейтронов.
    Ответ: 60
}

\variantsplitter

\addpersonalvariant{Полина Лоткова}

\tasknumber{1}%
\task{%
    Определите число нейтронов в атоме $\ce{^{2}_{1}{D}}$.
}
\answer{%
    $Z = 1$ протонов и столько же электронов $A = 2$ нуклонов, $A - Z = 1$ нейтронов.
    Ответ: 1
}

\tasknumber{2}%
\task{%
    Определите число нуклонов в атоме $\ce{^{4}_{2}{He}}$.
}
\answer{%
    $Z = 2$ протонов и столько же электронов $A = 4$ нуклонов, $A - Z = 2$ нейтронов.
    Ответ: 4
}

\tasknumber{3}%
\task{%
    Определите число нуклонов в атоме $\ce{^{10}_{4}{Be}}$.
}
\answer{%
    $Z = 4$ протонов и столько же электронов $A = 10$ нуклонов, $A - Z = 6$ нейтронов.
    Ответ: 10
}

\tasknumber{4}%
\task{%
    Определите число нуклонов в атоме $\ce{^{12}_{6}{C}}$.
}
\answer{%
    $Z = 6$ протонов и столько же электронов $A = 12$ нуклонов, $A - Z = 6$ нейтронов.
    Ответ: 12
}

\tasknumber{5}%
\task{%
    Определите число электронов в атоме $\ce{^{19}_{9}{F}}$.
}
\answer{%
    $Z = 9$ протонов и столько же электронов $A = 19$ нуклонов, $A - Z = 10$ нейтронов.
    Ответ: 9
}

\tasknumber{6}%
\task{%
    Определите число протонов в атоме $\ce{^{23}_{11}{Na}}$.
}
\answer{%
    $Z = 11$ протонов и столько же электронов $A = 23$ нуклонов, $A - Z = 12$ нейтронов.
    Ответ: 11
}

\tasknumber{7}%
\task{%
    Определите число нуклонов в атоме $\ce{^{40}_{20}{Ca}}$.
}
\answer{%
    $Z = 20$ протонов и столько же электронов $A = 40$ нуклонов, $A - Z = 20$ нейтронов.
    Ответ: 40
}

\tasknumber{8}%
\task{%
    Определите число нуклонов в атоме $\ce{^{44}_{20}{Ca}}$.
}
\answer{%
    $Z = 20$ протонов и столько же электронов $A = 44$ нуклонов, $A - Z = 24$ нейтронов.
    Ответ: 44
}

\tasknumber{9}%
\task{%
    Определите число протонов в атоме $\ce{^{67}_{29}{Cu}}$.
}
\answer{%
    $Z = 29$ протонов и столько же электронов $A = 67$ нуклонов, $A - Z = 38$ нейтронов.
    Ответ: 29
}

\tasknumber{10}%
\task{%
    Определите число протонов в атоме $\ce{^{80}_{34}{Se}}$.
}
\answer{%
    $Z = 34$ протонов и столько же электронов $A = 80$ нуклонов, $A - Z = 46$ нейтронов.
    Ответ: 34
}

\tasknumber{11}%
\task{%
    Определите число нейтронов в атоме $\ce{^{124}_{53}{I}}$.
}
\answer{%
    $Z = 53$ протонов и столько же электронов $A = 124$ нуклонов, $A - Z = 71$ нейтронов.
    Ответ: 71
}

\tasknumber{12}%
\task{%
    Определите число электронов в атоме $\ce{^{146}_{60}{Nd}}$.
}
\answer{%
    $Z = 60$ протонов и столько же электронов $A = 146$ нуклонов, $A - Z = 86$ нейтронов.
    Ответ: 60
}

\tasknumber{13}%
\task{%
    Определите число нейтронов в атоме $\text{дейтерий-2}$.
}
\answer{%
    $Z = 1$ протонов и столько же электронов, $A = 2$ нуклонов, $A - Z = 1$ нейтронов.
    Ответ: 1
}

\tasknumber{14}%
\task{%
    Определите число нуклонов в атоме $\text{гелий-4}$.
}
\answer{%
    $Z = 2$ протонов и столько же электронов, $A = 4$ нуклонов, $A - Z = 2$ нейтронов.
    Ответ: 4
}

\tasknumber{15}%
\task{%
    Определите число нуклонов в атоме $\text{литий-7}$.
}
\answer{%
    $Z = 3$ протонов и столько же электронов, $A = 7$ нуклонов, $A - Z = 4$ нейтронов.
    Ответ: 7
}

\tasknumber{16}%
\task{%
    Определите число электронов в атоме $\text{азот-15}$.
}
\answer{%
    $Z = 7$ протонов и столько же электронов, $A = 15$ нуклонов, $A - Z = 8$ нейтронов.
    Ответ: 7
}

\tasknumber{17}%
\task{%
    Определите число электронов в атоме $\text{фтор-19}$.
}
\answer{%
    $Z = 9$ протонов и столько же электронов, $A = 19$ нуклонов, $A - Z = 10$ нейтронов.
    Ответ: 9
}

\tasknumber{18}%
\task{%
    Определите число нуклонов в атоме $\text{фосфор-31}$.
}
\answer{%
    $Z = 15$ протонов и столько же электронов, $A = 31$ нуклонов, $A - Z = 16$ нейтронов.
    Ответ: 31
}

\tasknumber{19}%
\task{%
    Определите число нейтронов в атоме $\text{аргон-38}$.
}
\answer{%
    $Z = 18$ протонов и столько же электронов, $A = 38$ нуклонов, $A - Z = 20$ нейтронов.
    Ответ: 20
}

\tasknumber{20}%
\task{%
    Определите число нуклонов в атоме $\text{кальций-42}$.
}
\answer{%
    $Z = 20$ протонов и столько же электронов, $A = 42$ нуклонов, $A - Z = 22$ нейтронов.
    Ответ: 42
}

\tasknumber{21}%
\task{%
    Определите число нейтронов в атоме $\text{мышьяк-71}$.
}
\answer{%
    $Z = 33$ протонов и столько же электронов, $A = 71$ нуклонов, $A - Z = 38$ нейтронов.
    Ответ: 38
}

\tasknumber{22}%
\task{%
    Определите число нуклонов в атоме $\text{индий-111}$.
}
\answer{%
    $Z = 49$ протонов и столько же электронов, $A = 111$ нуклонов, $A - Z = 62$ нейтронов.
    Ответ: 111
}

\tasknumber{23}%
\task{%
    Определите число нуклонов в атоме $\text{сурьма-128}$.
}
\answer{%
    $Z = 51$ протонов и столько же электронов, $A = 128$ нуклонов, $A - Z = 77$ нейтронов.
    Ответ: 128
}

\tasknumber{24}%
\task{%
    Определите число нейтронов в атоме $\text{йод-131}$.
}
\answer{%
    $Z = 53$ протонов и столько же электронов, $A = 131$ нуклонов, $A - Z = 78$ нейтронов.
    Ответ: 78
}

\variantsplitter

\addpersonalvariant{Екатерина Медведева}

\tasknumber{1}%
\task{%
    Определите число нейтронов в атоме $\ce{^{2}_{1}{D}}$.
}
\answer{%
    $Z = 1$ протонов и столько же электронов $A = 2$ нуклонов, $A - Z = 1$ нейтронов.
    Ответ: 1
}

\tasknumber{2}%
\task{%
    Определите число электронов в атоме $\ce{^{6}_{2}{He}}$.
}
\answer{%
    $Z = 2$ протонов и столько же электронов $A = 6$ нуклонов, $A - Z = 4$ нейтронов.
    Ответ: 2
}

\tasknumber{3}%
\task{%
    Определите число протонов в атоме $\ce{^{7}_{4}{Be}}$.
}
\answer{%
    $Z = 4$ протонов и столько же электронов $A = 7$ нуклонов, $A - Z = 3$ нейтронов.
    Ответ: 4
}

\tasknumber{4}%
\task{%
    Определите число нейтронов в атоме $\ce{^{14}_{6}{C}}$.
}
\answer{%
    $Z = 6$ протонов и столько же электронов $A = 14$ нуклонов, $A - Z = 8$ нейтронов.
    Ответ: 8
}

\tasknumber{5}%
\task{%
    Определите число нуклонов в атоме $\ce{^{20}_{10}{Ne}}$.
}
\answer{%
    $Z = 10$ протонов и столько же электронов $A = 20$ нуклонов, $A - Z = 10$ нейтронов.
    Ответ: 20
}

\tasknumber{6}%
\task{%
    Определите число электронов в атоме $\ce{^{31}_{15}{P}}$.
}
\answer{%
    $Z = 15$ протонов и столько же электронов $A = 31$ нуклонов, $A - Z = 16$ нейтронов.
    Ответ: 15
}

\tasknumber{7}%
\task{%
    Определите число нуклонов в атоме $\ce{^{36}_{18}{Ar}}$.
}
\answer{%
    $Z = 18$ протонов и столько же электронов $A = 36$ нуклонов, $A - Z = 18$ нейтронов.
    Ответ: 36
}

\tasknumber{8}%
\task{%
    Определите число протонов в атоме $\ce{^{63}_{29}{Cu}}$.
}
\answer{%
    $Z = 29$ протонов и столько же электронов $A = 63$ нуклонов, $A - Z = 34$ нейтронов.
    Ответ: 29
}

\tasknumber{9}%
\task{%
    Определите число протонов в атоме $\ce{^{72}_{33}{As}}$.
}
\answer{%
    $Z = 33$ протонов и столько же электронов $A = 72$ нуклонов, $A - Z = 39$ нейтронов.
    Ответ: 33
}

\tasknumber{10}%
\task{%
    Определите число протонов в атоме $\ce{^{76}_{34}{Se}}$.
}
\answer{%
    $Z = 34$ протонов и столько же электронов $A = 76$ нуклонов, $A - Z = 42$ нейтронов.
    Ответ: 34
}

\tasknumber{11}%
\task{%
    Определите число протонов в атоме $\ce{^{124}_{51}{Sb}}$.
}
\answer{%
    $Z = 51$ протонов и столько же электронов $A = 124$ нуклонов, $A - Z = 73$ нейтронов.
    Ответ: 51
}

\tasknumber{12}%
\task{%
    Определите число нуклонов в атоме $\ce{^{148}_{62}{Sm}}$.
}
\answer{%
    $Z = 62$ протонов и столько же электронов $A = 148$ нуклонов, $A - Z = 86$ нейтронов.
    Ответ: 148
}

\tasknumber{13}%
\task{%
    Определите число нейтронов в атоме $\text{дейтерий-2}$.
}
\answer{%
    $Z = 1$ протонов и столько же электронов, $A = 2$ нуклонов, $A - Z = 1$ нейтронов.
    Ответ: 1
}

\tasknumber{14}%
\task{%
    Определите число электронов в атоме $\text{гелий-4}$.
}
\answer{%
    $Z = 2$ протонов и столько же электронов, $A = 4$ нуклонов, $A - Z = 2$ нейтронов.
    Ответ: 2
}

\tasknumber{15}%
\task{%
    Определите число протонов в атоме $\text{бериллий-10}$.
}
\answer{%
    $Z = 4$ протонов и столько же электронов, $A = 10$ нуклонов, $A - Z = 6$ нейтронов.
    Ответ: 4
}

\tasknumber{16}%
\task{%
    Определите число нейтронов в атоме $\text{азот-15}$.
}
\answer{%
    $Z = 7$ протонов и столько же электронов, $A = 15$ нуклонов, $A - Z = 8$ нейтронов.
    Ответ: 8
}

\tasknumber{17}%
\task{%
    Определите число нейтронов в атоме $\text{неон-21}$.
}
\answer{%
    $Z = 10$ протонов и столько же электронов, $A = 21$ нуклонов, $A - Z = 11$ нейтронов.
    Ответ: 11
}

\tasknumber{18}%
\task{%
    Определите число электронов в атоме $\text{магний-25}$.
}
\answer{%
    $Z = 12$ протонов и столько же электронов, $A = 25$ нуклонов, $A - Z = 13$ нейтронов.
    Ответ: 12
}

\tasknumber{19}%
\task{%
    Определите число нуклонов в атоме $\text{калий-41}$.
}
\answer{%
    $Z = 19$ протонов и столько же электронов, $A = 41$ нуклонов, $A - Z = 22$ нейтронов.
    Ответ: 41
}

\tasknumber{20}%
\task{%
    Определите число электронов в атоме $\text{кальций-42}$.
}
\answer{%
    $Z = 20$ протонов и столько же электронов, $A = 42$ нуклонов, $A - Z = 22$ нейтронов.
    Ответ: 20
}

\tasknumber{21}%
\task{%
    Определите число протонов в атоме $\text{селен-76}$.
}
\answer{%
    $Z = 34$ протонов и столько же электронов, $A = 76$ нуклонов, $A - Z = 42$ нейтронов.
    Ответ: 34
}

\tasknumber{22}%
\task{%
    Определите число нуклонов в атоме $\text{индий-111}$.
}
\answer{%
    $Z = 49$ протонов и столько же электронов, $A = 111$ нуклонов, $A - Z = 62$ нейтронов.
    Ответ: 111
}

\tasknumber{23}%
\task{%
    Определите число нуклонов в атоме $\text{йод-126}$.
}
\answer{%
    $Z = 53$ протонов и столько же электронов, $A = 126$ нуклонов, $A - Z = 73$ нейтронов.
    Ответ: 126
}

\tasknumber{24}%
\task{%
    Определите число нуклонов в атоме $\text{неодим-143}$.
}
\answer{%
    $Z = 60$ протонов и столько же электронов, $A = 143$ нуклонов, $A - Z = 83$ нейтронов.
    Ответ: 143
}

\variantsplitter

\addpersonalvariant{Константин Мельник}

\tasknumber{1}%
\task{%
    Определите число электронов в атоме $\ce{^{3}_{1}{T}}$.
}
\answer{%
    $Z = 1$ протонов и столько же электронов $A = 3$ нуклонов, $A - Z = 2$ нейтронов.
    Ответ: 1
}

\tasknumber{2}%
\task{%
    Определите число нуклонов в атоме $\ce{^{6}_{2}{He}}$.
}
\answer{%
    $Z = 2$ протонов и столько же электронов $A = 6$ нуклонов, $A - Z = 4$ нейтронов.
    Ответ: 6
}

\tasknumber{3}%
\task{%
    Определите число нуклонов в атоме $\ce{^{7}_{4}{Be}}$.
}
\answer{%
    $Z = 4$ протонов и столько же электронов $A = 7$ нуклонов, $A - Z = 3$ нейтронов.
    Ответ: 7
}

\tasknumber{4}%
\task{%
    Определите число протонов в атоме $\ce{^{12}_{6}{C}}$.
}
\answer{%
    $Z = 6$ протонов и столько же электронов $A = 12$ нуклонов, $A - Z = 6$ нейтронов.
    Ответ: 6
}

\tasknumber{5}%
\task{%
    Определите число нуклонов в атоме $\ce{^{19}_{9}{F}}$.
}
\answer{%
    $Z = 9$ протонов и столько же электронов $A = 19$ нуклонов, $A - Z = 10$ нейтронов.
    Ответ: 19
}

\tasknumber{6}%
\task{%
    Определите число нейтронов в атоме $\ce{^{24}_{12}{Mg}}$.
}
\answer{%
    $Z = 12$ протонов и столько же электронов $A = 24$ нуклонов, $A - Z = 12$ нейтронов.
    Ответ: 12
}

\tasknumber{7}%
\task{%
    Определите число нейтронов в атоме $\ce{^{41}_{19}{K}}$.
}
\answer{%
    $Z = 19$ протонов и столько же электронов $A = 41$ нуклонов, $A - Z = 22$ нейтронов.
    Ответ: 22
}

\tasknumber{8}%
\task{%
    Определите число нейтронов в атоме $\ce{^{44}_{20}{Ca}}$.
}
\answer{%
    $Z = 20$ протонов и столько же электронов $A = 44$ нуклонов, $A - Z = 24$ нейтронов.
    Ответ: 24
}

\tasknumber{9}%
\task{%
    Определите число нейтронов в атоме $\ce{^{75}_{34}{Se}}$.
}
\answer{%
    $Z = 34$ протонов и столько же электронов $A = 75$ нуклонов, $A - Z = 41$ нейтронов.
    Ответ: 41
}

\tasknumber{10}%
\task{%
    Определите число электронов в атоме $\ce{^{111}_{49}{In}}$.
}
\answer{%
    $Z = 49$ протонов и столько же электронов $A = 111$ нуклонов, $A - Z = 62$ нейтронов.
    Ответ: 49
}

\tasknumber{11}%
\task{%
    Определите число нейтронов в атоме $\ce{^{124}_{53}{I}}$.
}
\answer{%
    $Z = 53$ протонов и столько же электронов $A = 124$ нуклонов, $A - Z = 71$ нейтронов.
    Ответ: 71
}

\tasknumber{12}%
\task{%
    Определите число нейтронов в атоме $\ce{^{129}_{53}{I}}$.
}
\answer{%
    $Z = 53$ протонов и столько же электронов $A = 129$ нуклонов, $A - Z = 76$ нейтронов.
    Ответ: 76
}

\tasknumber{13}%
\task{%
    Определите число нейтронов в атоме $\text{гелий-3}$.
}
\answer{%
    $Z = 2$ протонов и столько же электронов, $A = 3$ нуклонов, $A - Z = 1$ нейтронов.
    Ответ: 1
}

\tasknumber{14}%
\task{%
    Определите число нейтронов в атоме $\text{гелий-4}$.
}
\answer{%
    $Z = 2$ протонов и столько же электронов, $A = 4$ нуклонов, $A - Z = 2$ нейтронов.
    Ответ: 2
}

\tasknumber{15}%
\task{%
    Определите число протонов в атоме $\text{бор-10}$.
}
\answer{%
    $Z = 5$ протонов и столько же электронов, $A = 10$ нуклонов, $A - Z = 5$ нейтронов.
    Ответ: 5
}

\tasknumber{16}%
\task{%
    Определите число нуклонов в атоме $\text{азот-14}$.
}
\answer{%
    $Z = 7$ протонов и столько же электронов, $A = 14$ нуклонов, $A - Z = 7$ нейтронов.
    Ответ: 14
}

\tasknumber{17}%
\task{%
    Определите число нуклонов в атоме $\text{неон-20}$.
}
\answer{%
    $Z = 10$ протонов и столько же электронов, $A = 20$ нуклонов, $A - Z = 10$ нейтронов.
    Ответ: 20
}

\tasknumber{18}%
\task{%
    Определите число протонов в атоме $\text{фосфор-33}$.
}
\answer{%
    $Z = 15$ протонов и столько же электронов, $A = 33$ нуклонов, $A - Z = 18$ нейтронов.
    Ответ: 15
}

\tasknumber{19}%
\task{%
    Определите число нейтронов в атоме $\text{калий-40}$.
}
\answer{%
    $Z = 19$ протонов и столько же электронов, $A = 40$ нуклонов, $A - Z = 21$ нейтронов.
    Ответ: 21
}

\tasknumber{20}%
\task{%
    Определите число нуклонов в атоме $\text{кальций-45}$.
}
\answer{%
    $Z = 20$ протонов и столько же электронов, $A = 45$ нуклонов, $A - Z = 25$ нейтронов.
    Ответ: 45
}

\tasknumber{21}%
\task{%
    Определите число протонов в атоме $\text{мышьяк-73}$.
}
\answer{%
    $Z = 33$ протонов и столько же электронов, $A = 73$ нуклонов, $A - Z = 40$ нейтронов.
    Ответ: 33
}

\tasknumber{22}%
\task{%
    Определите число протонов в атоме $\text{серебро-109}$.
}
\answer{%
    $Z = 47$ протонов и столько же электронов, $A = 109$ нуклонов, $A - Z = 62$ нейтронов.
    Ответ: 47
}

\tasknumber{23}%
\task{%
    Определите число нуклонов в атоме $\text{йод-127}$.
}
\answer{%
    $Z = 53$ протонов и столько же электронов, $A = 127$ нуклонов, $A - Z = 74$ нейтронов.
    Ответ: 127
}

\tasknumber{24}%
\task{%
    Определите число нейтронов в атоме $\text{самарий-147}$.
}
\answer{%
    $Z = 62$ протонов и столько же электронов, $A = 147$ нуклонов, $A - Z = 85$ нейтронов.
    Ответ: 85
}

\variantsplitter

\addpersonalvariant{Степан Небоваренков}

\tasknumber{1}%
\task{%
    Определите число электронов в атоме $\ce{^{2}_{1}{D}}$.
}
\answer{%
    $Z = 1$ протонов и столько же электронов $A = 2$ нуклонов, $A - Z = 1$ нейтронов.
    Ответ: 1
}

\tasknumber{2}%
\task{%
    Определите число нуклонов в атоме $\ce{^{4}_{2}{He}}$.
}
\answer{%
    $Z = 2$ протонов и столько же электронов $A = 4$ нуклонов, $A - Z = 2$ нейтронов.
    Ответ: 4
}

\tasknumber{3}%
\task{%
    Определите число электронов в атоме $\ce{^{10}_{4}{Be}}$.
}
\answer{%
    $Z = 4$ протонов и столько же электронов $A = 10$ нуклонов, $A - Z = 6$ нейтронов.
    Ответ: 4
}

\tasknumber{4}%
\task{%
    Определите число нуклонов в атоме $\ce{^{10}_{5}{B}}$.
}
\answer{%
    $Z = 5$ протонов и столько же электронов $A = 10$ нуклонов, $A - Z = 5$ нейтронов.
    Ответ: 10
}

\tasknumber{5}%
\task{%
    Определите число электронов в атоме $\ce{^{15}_{7}{N}}$.
}
\answer{%
    $Z = 7$ протонов и столько же электронов $A = 15$ нуклонов, $A - Z = 8$ нейтронов.
    Ответ: 7
}

\tasknumber{6}%
\task{%
    Определите число протонов в атоме $\ce{^{23}_{11}{Na}}$.
}
\answer{%
    $Z = 11$ протонов и столько же электронов $A = 23$ нуклонов, $A - Z = 12$ нейтронов.
    Ответ: 11
}

\tasknumber{7}%
\task{%
    Определите число электронов в атоме $\ce{^{40}_{20}{Ca}}$.
}
\answer{%
    $Z = 20$ протонов и столько же электронов $A = 40$ нуклонов, $A - Z = 20$ нейтронов.
    Ответ: 20
}

\tasknumber{8}%
\task{%
    Определите число нейтронов в атоме $\ce{^{42}_{20}{Ca}}$.
}
\answer{%
    $Z = 20$ протонов и столько же электронов $A = 42$ нуклонов, $A - Z = 22$ нейтронов.
    Ответ: 22
}

\tasknumber{9}%
\task{%
    Определите число протонов в атоме $\ce{^{74}_{33}{As}}$.
}
\answer{%
    $Z = 33$ протонов и столько же электронов $A = 74$ нуклонов, $A - Z = 41$ нейтронов.
    Ответ: 33
}

\tasknumber{10}%
\task{%
    Определите число протонов в атоме $\ce{^{79}_{34}{Se}}$.
}
\answer{%
    $Z = 34$ протонов и столько же электронов $A = 79$ нуклонов, $A - Z = 45$ нейтронов.
    Ответ: 34
}

\tasknumber{11}%
\task{%
    Определите число нуклонов в атоме $\ce{^{126}_{51}{Sb}}$.
}
\answer{%
    $Z = 51$ протонов и столько же электронов $A = 126$ нуклонов, $A - Z = 75$ нейтронов.
    Ответ: 126
}

\tasknumber{12}%
\task{%
    Определите число протонов в атоме $\ce{^{129}_{53}{I}}$.
}
\answer{%
    $Z = 53$ протонов и столько же электронов $A = 129$ нуклонов, $A - Z = 76$ нейтронов.
    Ответ: 53
}

\tasknumber{13}%
\task{%
    Определите число нейтронов в атоме $\text{тритий-3}$.
}
\answer{%
    $Z = 1$ протонов и столько же электронов, $A = 3$ нуклонов, $A - Z = 2$ нейтронов.
    Ответ: 2
}

\tasknumber{14}%
\task{%
    Определите число протонов в атоме $\text{гелий-6}$.
}
\answer{%
    $Z = 2$ протонов и столько же электронов, $A = 6$ нуклонов, $A - Z = 4$ нейтронов.
    Ответ: 2
}

\tasknumber{15}%
\task{%
    Определите число протонов в атоме $\text{бериллий-10}$.
}
\answer{%
    $Z = 4$ протонов и столько же электронов, $A = 10$ нуклонов, $A - Z = 6$ нейтронов.
    Ответ: 4
}

\tasknumber{16}%
\task{%
    Определите число протонов в атоме $\text{углерод-12}$.
}
\answer{%
    $Z = 6$ протонов и столько же электронов, $A = 12$ нуклонов, $A - Z = 6$ нейтронов.
    Ответ: 6
}

\tasknumber{17}%
\task{%
    Определите число протонов в атоме $\text{кислород-17}$.
}
\answer{%
    $Z = 8$ протонов и столько же электронов, $A = 17$ нуклонов, $A - Z = 9$ нейтронов.
    Ответ: 8
}

\tasknumber{18}%
\task{%
    Определите число протонов в атоме $\text{фосфор-33}$.
}
\answer{%
    $Z = 15$ протонов и столько же электронов, $A = 33$ нуклонов, $A - Z = 18$ нейтронов.
    Ответ: 15
}

\tasknumber{19}%
\task{%
    Определите число электронов в атоме $\text{аргон-42}$.
}
\answer{%
    $Z = 18$ протонов и столько же электронов, $A = 42$ нуклонов, $A - Z = 24$ нейтронов.
    Ответ: 18
}

\tasknumber{20}%
\task{%
    Определите число нуклонов в атоме $\text{медь-65}$.
}
\answer{%
    $Z = 29$ протонов и столько же электронов, $A = 65$ нуклонов, $A - Z = 36$ нейтронов.
    Ответ: 65
}

\tasknumber{21}%
\task{%
    Определите число протонов в атоме $\text{селен-76}$.
}
\answer{%
    $Z = 34$ протонов и столько же электронов, $A = 76$ нуклонов, $A - Z = 42$ нейтронов.
    Ответ: 34
}

\tasknumber{22}%
\task{%
    Определите число протонов в атоме $\text{серебро-107}$.
}
\answer{%
    $Z = 47$ протонов и столько же электронов, $A = 107$ нуклонов, $A - Z = 60$ нейтронов.
    Ответ: 47
}

\tasknumber{23}%
\task{%
    Определите число нейтронов в атоме $\text{йод-126}$.
}
\answer{%
    $Z = 53$ протонов и столько же электронов, $A = 126$ нуклонов, $A - Z = 73$ нейтронов.
    Ответ: 73
}

\tasknumber{24}%
\task{%
    Определите число нейтронов в атоме $\text{неодим-144}$.
}
\answer{%
    $Z = 60$ протонов и столько же электронов, $A = 144$ нуклонов, $A - Z = 84$ нейтронов.
    Ответ: 84
}

\variantsplitter

\addpersonalvariant{Матвей Неретин}

\tasknumber{1}%
\task{%
    Определите число нейтронов в атоме $\ce{^{3}_{1}{T}}$.
}
\answer{%
    $Z = 1$ протонов и столько же электронов $A = 3$ нуклонов, $A - Z = 2$ нейтронов.
    Ответ: 2
}

\tasknumber{2}%
\task{%
    Определите число электронов в атоме $\ce{^{4}_{2}{He}}$.
}
\answer{%
    $Z = 2$ протонов и столько же электронов $A = 4$ нуклонов, $A - Z = 2$ нейтронов.
    Ответ: 2
}

\tasknumber{3}%
\task{%
    Определите число электронов в атоме $\ce{^{7}_{4}{Be}}$.
}
\answer{%
    $Z = 4$ протонов и столько же электронов $A = 7$ нуклонов, $A - Z = 3$ нейтронов.
    Ответ: 4
}

\tasknumber{4}%
\task{%
    Определите число нуклонов в атоме $\ce{^{13}_{6}{C}}$.
}
\answer{%
    $Z = 6$ протонов и столько же электронов $A = 13$ нуклонов, $A - Z = 7$ нейтронов.
    Ответ: 13
}

\tasknumber{5}%
\task{%
    Определите число электронов в атоме $\ce{^{19}_{9}{F}}$.
}
\answer{%
    $Z = 9$ протонов и столько же электронов $A = 19$ нуклонов, $A - Z = 10$ нейтронов.
    Ответ: 9
}

\tasknumber{6}%
\task{%
    Определите число нуклонов в атоме $\ce{^{23}_{11}{Na}}$.
}
\answer{%
    $Z = 11$ протонов и столько же электронов $A = 23$ нуклонов, $A - Z = 12$ нейтронов.
    Ответ: 23
}

\tasknumber{7}%
\task{%
    Определите число протонов в атоме $\ce{^{41}_{19}{K}}$.
}
\answer{%
    $Z = 19$ протонов и столько же электронов $A = 41$ нуклонов, $A - Z = 22$ нейтронов.
    Ответ: 19
}

\tasknumber{8}%
\task{%
    Определите число нуклонов в атоме $\ce{^{47}_{20}{Ca}}$.
}
\answer{%
    $Z = 20$ протонов и столько же электронов $A = 47$ нуклонов, $A - Z = 27$ нейтронов.
    Ответ: 47
}

\tasknumber{9}%
\task{%
    Определите число нейтронов в атоме $\ce{^{72}_{33}{As}}$.
}
\answer{%
    $Z = 33$ протонов и столько же электронов $A = 72$ нуклонов, $A - Z = 39$ нейтронов.
    Ответ: 39
}

\tasknumber{10}%
\task{%
    Определите число нейтронов в атоме $\ce{^{76}_{34}{Se}}$.
}
\answer{%
    $Z = 34$ протонов и столько же электронов $A = 76$ нуклонов, $A - Z = 42$ нейтронов.
    Ответ: 42
}

\tasknumber{11}%
\task{%
    Определите число протонов в атоме $\ce{^{121}_{51}{Sb}}$.
}
\answer{%
    $Z = 51$ протонов и столько же электронов $A = 121$ нуклонов, $A - Z = 70$ нейтронов.
    Ответ: 51
}

\tasknumber{12}%
\task{%
    Определите число нуклонов в атоме $\ce{^{129}_{53}{I}}$.
}
\answer{%
    $Z = 53$ протонов и столько же электронов $A = 129$ нуклонов, $A - Z = 76$ нейтронов.
    Ответ: 129
}

\tasknumber{13}%
\task{%
    Определите число протонов в атоме $\text{тритий-3}$.
}
\answer{%
    $Z = 1$ протонов и столько же электронов, $A = 3$ нуклонов, $A - Z = 2$ нейтронов.
    Ответ: 1
}

\tasknumber{14}%
\task{%
    Определите число нуклонов в атоме $\text{гелий-4}$.
}
\answer{%
    $Z = 2$ протонов и столько же электронов, $A = 4$ нуклонов, $A - Z = 2$ нейтронов.
    Ответ: 4
}

\tasknumber{15}%
\task{%
    Определите число протонов в атоме $\text{бериллий-7}$.
}
\answer{%
    $Z = 4$ протонов и столько же электронов, $A = 7$ нуклонов, $A - Z = 3$ нейтронов.
    Ответ: 4
}

\tasknumber{16}%
\task{%
    Определите число электронов в атоме $\text{бор-10}$.
}
\answer{%
    $Z = 5$ протонов и столько же электронов, $A = 10$ нуклонов, $A - Z = 5$ нейтронов.
    Ответ: 5
}

\tasknumber{17}%
\task{%
    Определите число нейтронов в атоме $\text{кислород-16}$.
}
\answer{%
    $Z = 8$ протонов и столько же электронов, $A = 16$ нуклонов, $A - Z = 8$ нейтронов.
    Ответ: 8
}

\tasknumber{18}%
\task{%
    Определите число протонов в атоме $\text{фосфор-32}$.
}
\answer{%
    $Z = 15$ протонов и столько же электронов, $A = 32$ нуклонов, $A - Z = 17$ нейтронов.
    Ответ: 15
}

\tasknumber{19}%
\task{%
    Определите число электронов в атоме $\text{аргон-38}$.
}
\answer{%
    $Z = 18$ протонов и столько же электронов, $A = 38$ нуклонов, $A - Z = 20$ нейтронов.
    Ответ: 18
}

\tasknumber{20}%
\task{%
    Определите число протонов в атоме $\text{медь-63}$.
}
\answer{%
    $Z = 29$ протонов и столько же электронов, $A = 63$ нуклонов, $A - Z = 34$ нейтронов.
    Ответ: 29
}

\tasknumber{21}%
\task{%
    Определите число протонов в атоме $\text{мышьяк-74}$.
}
\answer{%
    $Z = 33$ протонов и столько же электронов, $A = 74$ нуклонов, $A - Z = 41$ нейтронов.
    Ответ: 33
}

\tasknumber{22}%
\task{%
    Определите число нейтронов в атоме $\text{серебро-105}$.
}
\answer{%
    $Z = 47$ протонов и столько же электронов, $A = 105$ нуклонов, $A - Z = 58$ нейтронов.
    Ответ: 58
}

\tasknumber{23}%
\task{%
    Определите число нейтронов в атоме $\text{йод-127}$.
}
\answer{%
    $Z = 53$ протонов и столько же электронов, $A = 127$ нуклонов, $A - Z = 74$ нейтронов.
    Ответ: 74
}

\tasknumber{24}%
\task{%
    Определите число нейтронов в атоме $\text{неодим-142}$.
}
\answer{%
    $Z = 60$ протонов и столько же электронов, $A = 142$ нуклонов, $A - Z = 82$ нейтронов.
    Ответ: 82
}

\variantsplitter

\addpersonalvariant{Мария Никонова}

\tasknumber{1}%
\task{%
    Определите число нуклонов в атоме $\ce{^{1}_{1}{H}}$.
}
\answer{%
    $Z = 1$ протонов и столько же электронов $A = 1$ нуклонов, $A - Z = 0$ нейтронов.
    Ответ: 1
}

\tasknumber{2}%
\task{%
    Определите число нуклонов в атоме $\ce{^{8}_{2}{He}}$.
}
\answer{%
    $Z = 2$ протонов и столько же электронов $A = 8$ нуклонов, $A - Z = 6$ нейтронов.
    Ответ: 8
}

\tasknumber{3}%
\task{%
    Определите число протонов в атоме $\ce{^{7}_{3}{Li}}$.
}
\answer{%
    $Z = 3$ протонов и столько же электронов $A = 7$ нуклонов, $A - Z = 4$ нейтронов.
    Ответ: 3
}

\tasknumber{4}%
\task{%
    Определите число протонов в атоме $\ce{^{14}_{6}{C}}$.
}
\answer{%
    $Z = 6$ протонов и столько же электронов $A = 14$ нуклонов, $A - Z = 8$ нейтронов.
    Ответ: 6
}

\tasknumber{5}%
\task{%
    Определите число протонов в атоме $\ce{^{18}_{8}{O}}$.
}
\answer{%
    $Z = 8$ протонов и столько же электронов $A = 18$ нуклонов, $A - Z = 10$ нейтронов.
    Ответ: 8
}

\tasknumber{6}%
\task{%
    Определите число нейтронов в атоме $\ce{^{22}_{11}{Na}}$.
}
\answer{%
    $Z = 11$ протонов и столько же электронов $A = 22$ нуклонов, $A - Z = 11$ нейтронов.
    Ответ: 11
}

\tasknumber{7}%
\task{%
    Определите число нуклонов в атоме $\ce{^{38}_{18}{Ar}}$.
}
\answer{%
    $Z = 18$ протонов и столько же электронов $A = 38$ нуклонов, $A - Z = 20$ нейтронов.
    Ответ: 38
}

\tasknumber{8}%
\task{%
    Определите число протонов в атоме $\ce{^{43}_{20}{Ca}}$.
}
\answer{%
    $Z = 20$ протонов и столько же электронов $A = 43$ нуклонов, $A - Z = 23$ нейтронов.
    Ответ: 20
}

\tasknumber{9}%
\task{%
    Определите число нейтронов в атоме $\ce{^{72}_{33}{As}}$.
}
\answer{%
    $Z = 33$ протонов и столько же электронов $A = 72$ нуклонов, $A - Z = 39$ нейтронов.
    Ответ: 39
}

\tasknumber{10}%
\task{%
    Определите число нуклонов в атоме $\ce{^{107}_{47}{Ag}}$.
}
\answer{%
    $Z = 47$ протонов и столько же электронов $A = 107$ нуклонов, $A - Z = 60$ нейтронов.
    Ответ: 107
}

\tasknumber{11}%
\task{%
    Определите число нуклонов в атоме $\ce{^{115}_{49}{In}}$.
}
\answer{%
    $Z = 49$ протонов и столько же электронов $A = 115$ нуклонов, $A - Z = 66$ нейтронов.
    Ответ: 115
}

\tasknumber{12}%
\task{%
    Определите число протонов в атоме $\ce{^{151}_{63}{Eu}}$.
}
\answer{%
    $Z = 63$ протонов и столько же электронов $A = 151$ нуклонов, $A - Z = 88$ нейтронов.
    Ответ: 63
}

\tasknumber{13}%
\task{%
    Определите число протонов в атоме $\text{тритий-3}$.
}
\answer{%
    $Z = 1$ протонов и столько же электронов, $A = 3$ нуклонов, $A - Z = 2$ нейтронов.
    Ответ: 1
}

\tasknumber{14}%
\task{%
    Определите число нуклонов в атоме $\text{литий-6}$.
}
\answer{%
    $Z = 3$ протонов и столько же электронов, $A = 6$ нуклонов, $A - Z = 3$ нейтронов.
    Ответ: 6
}

\tasknumber{15}%
\task{%
    Определите число электронов в атоме $\text{бериллий-7}$.
}
\answer{%
    $Z = 4$ протонов и столько же электронов, $A = 7$ нуклонов, $A - Z = 3$ нейтронов.
    Ответ: 4
}

\tasknumber{16}%
\task{%
    Определите число нейтронов в атоме $\text{азот-15}$.
}
\answer{%
    $Z = 7$ протонов и столько же электронов, $A = 15$ нуклонов, $A - Z = 8$ нейтронов.
    Ответ: 8
}

\tasknumber{17}%
\task{%
    Определите число нуклонов в атоме $\text{неон-21}$.
}
\answer{%
    $Z = 10$ протонов и столько же электронов, $A = 21$ нуклонов, $A - Z = 11$ нейтронов.
    Ответ: 21
}

\tasknumber{18}%
\task{%
    Определите число нейтронов в атоме $\text{фосфор-31}$.
}
\answer{%
    $Z = 15$ протонов и столько же электронов, $A = 31$ нуклонов, $A - Z = 16$ нейтронов.
    Ответ: 16
}

\tasknumber{19}%
\task{%
    Определите число протонов в атоме $\text{кальций-40}$.
}
\answer{%
    $Z = 20$ протонов и столько же электронов, $A = 40$ нуклонов, $A - Z = 20$ нейтронов.
    Ответ: 20
}

\tasknumber{20}%
\task{%
    Определите число нейтронов в атоме $\text{кальций-46}$.
}
\answer{%
    $Z = 20$ протонов и столько же электронов, $A = 46$ нуклонов, $A - Z = 26$ нейтронов.
    Ответ: 26
}

\tasknumber{21}%
\task{%
    Определите число протонов в атоме $\text{мышьяк-76}$.
}
\answer{%
    $Z = 33$ протонов и столько же электронов, $A = 76$ нуклонов, $A - Z = 43$ нейтронов.
    Ответ: 33
}

\tasknumber{22}%
\task{%
    Определите число протонов в атоме $\text{селен-82}$.
}
\answer{%
    $Z = 34$ протонов и столько же электронов, $A = 82$ нуклонов, $A - Z = 48$ нейтронов.
    Ответ: 34
}

\tasknumber{23}%
\task{%
    Определите число электронов в атоме $\text{сурьма-124}$.
}
\answer{%
    $Z = 51$ протонов и столько же электронов, $A = 124$ нуклонов, $A - Z = 73$ нейтронов.
    Ответ: 51
}

\tasknumber{24}%
\task{%
    Определите число электронов в атоме $\text{йод-131}$.
}
\answer{%
    $Z = 53$ протонов и столько же электронов, $A = 131$ нуклонов, $A - Z = 78$ нейтронов.
    Ответ: 53
}

\variantsplitter

\addpersonalvariant{Даниил Палаткин}

\tasknumber{1}%
\task{%
    Определите число нейтронов в атоме $\ce{^{1}_{1}{H}}$.
}
\answer{%
    $Z = 1$ протонов и столько же электронов $A = 1$ нуклонов, $A - Z = 0$ нейтронов.
    Ответ: 0
}

\tasknumber{2}%
\task{%
    Определите число электронов в атоме $\ce{^{3}_{2}{He}}$.
}
\answer{%
    $Z = 2$ протонов и столько же электронов $A = 3$ нуклонов, $A - Z = 1$ нейтронов.
    Ответ: 2
}

\tasknumber{3}%
\task{%
    Определите число нейтронов в атоме $\ce{^{7}_{3}{Li}}$.
}
\answer{%
    $Z = 3$ протонов и столько же электронов $A = 7$ нуклонов, $A - Z = 4$ нейтронов.
    Ответ: 4
}

\tasknumber{4}%
\task{%
    Определите число нуклонов в атоме $\ce{^{14}_{7}{N}}$.
}
\answer{%
    $Z = 7$ протонов и столько же электронов $A = 14$ нуклонов, $A - Z = 7$ нейтронов.
    Ответ: 14
}

\tasknumber{5}%
\task{%
    Определите число электронов в атоме $\ce{^{20}_{10}{Ne}}$.
}
\answer{%
    $Z = 10$ протонов и столько же электронов $A = 20$ нуклонов, $A - Z = 10$ нейтронов.
    Ответ: 10
}

\tasknumber{6}%
\task{%
    Определите число протонов в атоме $\ce{^{22}_{11}{Na}}$.
}
\answer{%
    $Z = 11$ протонов и столько же электронов $A = 22$ нуклонов, $A - Z = 11$ нейтронов.
    Ответ: 11
}

\tasknumber{7}%
\task{%
    Определите число протонов в атоме $\ce{^{42}_{18}{Ar}}$.
}
\answer{%
    $Z = 18$ протонов и столько же электронов $A = 42$ нуклонов, $A - Z = 24$ нейтронов.
    Ответ: 18
}

\tasknumber{8}%
\task{%
    Определите число нуклонов в атоме $\ce{^{43}_{20}{Ca}}$.
}
\answer{%
    $Z = 20$ протонов и столько же электронов $A = 43$ нуклонов, $A - Z = 23$ нейтронов.
    Ответ: 43
}

\tasknumber{9}%
\task{%
    Определите число нуклонов в атоме $\ce{^{74}_{33}{As}}$.
}
\answer{%
    $Z = 33$ протонов и столько же электронов $A = 74$ нуклонов, $A - Z = 41$ нейтронов.
    Ответ: 74
}

\tasknumber{10}%
\task{%
    Определите число нуклонов в атоме $\ce{^{82}_{34}{Se}}$.
}
\answer{%
    $Z = 34$ протонов и столько же электронов $A = 82$ нуклонов, $A - Z = 48$ нейтронов.
    Ответ: 82
}

\tasknumber{11}%
\task{%
    Определите число протонов в атоме $\ce{^{125}_{51}{Sb}}$.
}
\answer{%
    $Z = 51$ протонов и столько же электронов $A = 125$ нуклонов, $A - Z = 74$ нейтронов.
    Ответ: 51
}

\tasknumber{12}%
\task{%
    Определите число протонов в атоме $\ce{^{142}_{60}{Nd}}$.
}
\answer{%
    $Z = 60$ протонов и столько же электронов $A = 142$ нуклонов, $A - Z = 82$ нейтронов.
    Ответ: 60
}

\tasknumber{13}%
\task{%
    Определите число нейтронов в атоме $\text{тритий-3}$.
}
\answer{%
    $Z = 1$ протонов и столько же электронов, $A = 3$ нуклонов, $A - Z = 2$ нейтронов.
    Ответ: 2
}

\tasknumber{14}%
\task{%
    Определите число электронов в атоме $\text{литий-6}$.
}
\answer{%
    $Z = 3$ протонов и столько же электронов, $A = 6$ нуклонов, $A - Z = 3$ нейтронов.
    Ответ: 3
}

\tasknumber{15}%
\task{%
    Определите число электронов в атоме $\text{бериллий-10}$.
}
\answer{%
    $Z = 4$ протонов и столько же электронов, $A = 10$ нуклонов, $A - Z = 6$ нейтронов.
    Ответ: 4
}

\tasknumber{16}%
\task{%
    Определите число нуклонов в атоме $\text{бор-10}$.
}
\answer{%
    $Z = 5$ протонов и столько же электронов, $A = 10$ нуклонов, $A - Z = 5$ нейтронов.
    Ответ: 10
}

\tasknumber{17}%
\task{%
    Определите число нуклонов в атоме $\text{кислород-17}$.
}
\answer{%
    $Z = 8$ протонов и столько же электронов, $A = 17$ нуклонов, $A - Z = 9$ нейтронов.
    Ответ: 17
}

\tasknumber{18}%
\task{%
    Определите число нуклонов в атоме $\text{фосфор-33}$.
}
\answer{%
    $Z = 15$ протонов и столько же электронов, $A = 33$ нуклонов, $A - Z = 18$ нейтронов.
    Ответ: 33
}

\tasknumber{19}%
\task{%
    Определите число нуклонов в атоме $\text{кальций-40}$.
}
\answer{%
    $Z = 20$ протонов и столько же электронов, $A = 40$ нуклонов, $A - Z = 20$ нейтронов.
    Ответ: 40
}

\tasknumber{20}%
\task{%
    Определите число нейтронов в атоме $\text{кальций-44}$.
}
\answer{%
    $Z = 20$ протонов и столько же электронов, $A = 44$ нуклонов, $A - Z = 24$ нейтронов.
    Ответ: 24
}

\tasknumber{21}%
\task{%
    Определите число нуклонов в атоме $\text{селен-75}$.
}
\answer{%
    $Z = 34$ протонов и столько же электронов, $A = 75$ нуклонов, $A - Z = 41$ нейтронов.
    Ответ: 75
}

\tasknumber{22}%
\task{%
    Определите число нуклонов в атоме $\text{селен-77}$.
}
\answer{%
    $Z = 34$ протонов и столько же электронов, $A = 77$ нуклонов, $A - Z = 43$ нейтронов.
    Ответ: 77
}

\tasknumber{23}%
\task{%
    Определите число нуклонов в атоме $\text{сурьма-124}$.
}
\answer{%
    $Z = 51$ протонов и столько же электронов, $A = 124$ нуклонов, $A - Z = 73$ нейтронов.
    Ответ: 124
}

\tasknumber{24}%
\task{%
    Определите число протонов в атоме $\text{европий-151}$.
}
\answer{%
    $Z = 63$ протонов и столько же электронов, $A = 151$ нуклонов, $A - Z = 88$ нейтронов.
    Ответ: 63
}

\variantsplitter

\addpersonalvariant{Станислав Пикун}

\tasknumber{1}%
\task{%
    Определите число нуклонов в атоме $\ce{^{1}_{1}{H}}$.
}
\answer{%
    $Z = 1$ протонов и столько же электронов $A = 1$ нуклонов, $A - Z = 0$ нейтронов.
    Ответ: 1
}

\tasknumber{2}%
\task{%
    Определите число нейтронов в атоме $\ce{^{6}_{2}{He}}$.
}
\answer{%
    $Z = 2$ протонов и столько же электронов $A = 6$ нуклонов, $A - Z = 4$ нейтронов.
    Ответ: 4
}

\tasknumber{3}%
\task{%
    Определите число электронов в атоме $\ce{^{7}_{4}{Be}}$.
}
\answer{%
    $Z = 4$ протонов и столько же электронов $A = 7$ нуклонов, $A - Z = 3$ нейтронов.
    Ответ: 4
}

\tasknumber{4}%
\task{%
    Определите число нуклонов в атоме $\ce{^{14}_{7}{N}}$.
}
\answer{%
    $Z = 7$ протонов и столько же электронов $A = 14$ нуклонов, $A - Z = 7$ нейтронов.
    Ответ: 14
}

\tasknumber{5}%
\task{%
    Определите число нуклонов в атоме $\ce{^{21}_{10}{Ne}}$.
}
\answer{%
    $Z = 10$ протонов и столько же электронов $A = 21$ нуклонов, $A - Z = 11$ нейтронов.
    Ответ: 21
}

\tasknumber{6}%
\task{%
    Определите число электронов в атоме $\ce{^{22}_{10}{Ne}}$.
}
\answer{%
    $Z = 10$ протонов и столько же электронов $A = 22$ нуклонов, $A - Z = 12$ нейтронов.
    Ответ: 10
}

\tasknumber{7}%
\task{%
    Определите число электронов в атоме $\ce{^{40}_{19}{K}}$.
}
\answer{%
    $Z = 19$ протонов и столько же электронов $A = 40$ нуклонов, $A - Z = 21$ нейтронов.
    Ответ: 19
}

\tasknumber{8}%
\task{%
    Определите число протонов в атоме $\ce{^{65}_{29}{Cu}}$.
}
\answer{%
    $Z = 29$ протонов и столько же электронов $A = 65$ нуклонов, $A - Z = 36$ нейтронов.
    Ответ: 29
}

\tasknumber{9}%
\task{%
    Определите число нуклонов в атоме $\ce{^{75}_{33}{As}}$.
}
\answer{%
    $Z = 33$ протонов и столько же электронов $A = 75$ нуклонов, $A - Z = 42$ нейтронов.
    Ответ: 75
}

\tasknumber{10}%
\task{%
    Определите число электронов в атоме $\ce{^{77}_{34}{Se}}$.
}
\answer{%
    $Z = 34$ протонов и столько же электронов $A = 77$ нуклонов, $A - Z = 43$ нейтронов.
    Ответ: 34
}

\tasknumber{11}%
\task{%
    Определите число протонов в атоме $\ce{^{121}_{51}{Sb}}$.
}
\answer{%
    $Z = 51$ протонов и столько же электронов $A = 121$ нуклонов, $A - Z = 70$ нейтронов.
    Ответ: 51
}

\tasknumber{12}%
\task{%
    Определите число нуклонов в атоме $\ce{^{145}_{60}{Nd}}$.
}
\answer{%
    $Z = 60$ протонов и столько же электронов $A = 145$ нуклонов, $A - Z = 85$ нейтронов.
    Ответ: 145
}

\tasknumber{13}%
\task{%
    Определите число электронов в атоме $\text{дейтерий-2}$.
}
\answer{%
    $Z = 1$ протонов и столько же электронов, $A = 2$ нуклонов, $A - Z = 1$ нейтронов.
    Ответ: 1
}

\tasknumber{14}%
\task{%
    Определите число электронов в атоме $\text{гелий-6}$.
}
\answer{%
    $Z = 2$ протонов и столько же электронов, $A = 6$ нуклонов, $A - Z = 4$ нейтронов.
    Ответ: 2
}

\tasknumber{15}%
\task{%
    Определите число нейтронов в атоме $\text{бериллий-7}$.
}
\answer{%
    $Z = 4$ протонов и столько же электронов, $A = 7$ нуклонов, $A - Z = 3$ нейтронов.
    Ответ: 3
}

\tasknumber{16}%
\task{%
    Определите число нейтронов в атоме $\text{азот-15}$.
}
\answer{%
    $Z = 7$ протонов и столько же электронов, $A = 15$ нуклонов, $A - Z = 8$ нейтронов.
    Ответ: 8
}

\tasknumber{17}%
\task{%
    Определите число электронов в атоме $\text{неон-22}$.
}
\answer{%
    $Z = 10$ протонов и столько же электронов, $A = 22$ нуклонов, $A - Z = 12$ нейтронов.
    Ответ: 10
}

\tasknumber{18}%
\task{%
    Определите число протонов в атоме $\text{натрий-23}$.
}
\answer{%
    $Z = 11$ протонов и столько же электронов, $A = 23$ нуклонов, $A - Z = 12$ нейтронов.
    Ответ: 11
}

\tasknumber{19}%
\task{%
    Определите число протонов в атоме $\text{аргон-38}$.
}
\answer{%
    $Z = 18$ протонов и столько же электронов, $A = 38$ нуклонов, $A - Z = 20$ нейтронов.
    Ответ: 18
}

\tasknumber{20}%
\task{%
    Определите число нейтронов в атоме $\text{кальций-45}$.
}
\answer{%
    $Z = 20$ протонов и столько же электронов, $A = 45$ нуклонов, $A - Z = 25$ нейтронов.
    Ответ: 25
}

\tasknumber{21}%
\task{%
    Определите число протонов в атоме $\text{мышьяк-76}$.
}
\answer{%
    $Z = 33$ протонов и столько же электронов, $A = 76$ нуклонов, $A - Z = 43$ нейтронов.
    Ответ: 33
}

\tasknumber{22}%
\task{%
    Определите число протонов в атоме $\text{индий-115}$.
}
\answer{%
    $Z = 49$ протонов и столько же электронов, $A = 115$ нуклонов, $A - Z = 66$ нейтронов.
    Ответ: 49
}

\tasknumber{23}%
\task{%
    Определите число нуклонов в атоме $\text{сурьма-128}$.
}
\answer{%
    $Z = 51$ протонов и столько же электронов, $A = 128$ нуклонов, $A - Z = 77$ нейтронов.
    Ответ: 128
}

\tasknumber{24}%
\task{%
    Определите число нейтронов в атоме $\text{европий-153}$.
}
\answer{%
    $Z = 63$ протонов и столько же электронов, $A = 153$ нуклонов, $A - Z = 90$ нейтронов.
    Ответ: 90
}

\variantsplitter

\addpersonalvariant{Илья Пичугин}

\tasknumber{1}%
\task{%
    Определите число нуклонов в атоме $\ce{^{1}_{1}{H}}$.
}
\answer{%
    $Z = 1$ протонов и столько же электронов $A = 1$ нуклонов, $A - Z = 0$ нейтронов.
    Ответ: 1
}

\tasknumber{2}%
\task{%
    Определите число нуклонов в атоме $\ce{^{8}_{2}{He}}$.
}
\answer{%
    $Z = 2$ протонов и столько же электронов $A = 8$ нуклонов, $A - Z = 6$ нейтронов.
    Ответ: 8
}

\tasknumber{3}%
\task{%
    Определите число нейтронов в атоме $\ce{^{7}_{4}{Be}}$.
}
\answer{%
    $Z = 4$ протонов и столько же электронов $A = 7$ нуклонов, $A - Z = 3$ нейтронов.
    Ответ: 3
}

\tasknumber{4}%
\task{%
    Определите число нейтронов в атоме $\ce{^{14}_{6}{C}}$.
}
\answer{%
    $Z = 6$ протонов и столько же электронов $A = 14$ нуклонов, $A - Z = 8$ нейтронов.
    Ответ: 8
}

\tasknumber{5}%
\task{%
    Определите число нуклонов в атоме $\ce{^{21}_{10}{Ne}}$.
}
\answer{%
    $Z = 10$ протонов и столько же электронов $A = 21$ нуклонов, $A - Z = 11$ нейтронов.
    Ответ: 21
}

\tasknumber{6}%
\task{%
    Определите число протонов в атоме $\ce{^{31}_{15}{P}}$.
}
\answer{%
    $Z = 15$ протонов и столько же электронов $A = 31$ нуклонов, $A - Z = 16$ нейтронов.
    Ответ: 15
}

\tasknumber{7}%
\task{%
    Определите число электронов в атоме $\ce{^{38}_{18}{Ar}}$.
}
\answer{%
    $Z = 18$ протонов и столько же электронов $A = 38$ нуклонов, $A - Z = 20$ нейтронов.
    Ответ: 18
}

\tasknumber{8}%
\task{%
    Определите число электронов в атоме $\ce{^{43}_{20}{Ca}}$.
}
\answer{%
    $Z = 20$ протонов и столько же электронов $A = 43$ нуклонов, $A - Z = 23$ нейтронов.
    Ответ: 20
}

\tasknumber{9}%
\task{%
    Определите число нейтронов в атоме $\ce{^{67}_{29}{Cu}}$.
}
\answer{%
    $Z = 29$ протонов и столько же электронов $A = 67$ нуклонов, $A - Z = 38$ нейтронов.
    Ответ: 38
}

\tasknumber{10}%
\task{%
    Определите число протонов в атоме $\ce{^{79}_{34}{Se}}$.
}
\answer{%
    $Z = 34$ протонов и столько же электронов $A = 79$ нуклонов, $A - Z = 45$ нейтронов.
    Ответ: 34
}

\tasknumber{11}%
\task{%
    Определите число электронов в атоме $\ce{^{128}_{51}{Sb}}$.
}
\answer{%
    $Z = 51$ протонов и столько же электронов $A = 128$ нуклонов, $A - Z = 77$ нейтронов.
    Ответ: 51
}

\tasknumber{12}%
\task{%
    Определите число нуклонов в атоме $\ce{^{147}_{62}{Sm}}$.
}
\answer{%
    $Z = 62$ протонов и столько же электронов $A = 147$ нуклонов, $A - Z = 85$ нейтронов.
    Ответ: 147
}

\tasknumber{13}%
\task{%
    Определите число протонов в атоме $\text{тритий-3}$.
}
\answer{%
    $Z = 1$ протонов и столько же электронов, $A = 3$ нуклонов, $A - Z = 2$ нейтронов.
    Ответ: 1
}

\tasknumber{14}%
\task{%
    Определите число нейтронов в атоме $\text{гелий-6}$.
}
\answer{%
    $Z = 2$ протонов и столько же электронов, $A = 6$ нуклонов, $A - Z = 4$ нейтронов.
    Ответ: 4
}

\tasknumber{15}%
\task{%
    Определите число протонов в атоме $\text{бериллий-10}$.
}
\answer{%
    $Z = 4$ протонов и столько же электронов, $A = 10$ нуклонов, $A - Z = 6$ нейтронов.
    Ответ: 4
}

\tasknumber{16}%
\task{%
    Определите число нуклонов в атоме $\text{азот-14}$.
}
\answer{%
    $Z = 7$ протонов и столько же электронов, $A = 14$ нуклонов, $A - Z = 7$ нейтронов.
    Ответ: 14
}

\tasknumber{17}%
\task{%
    Определите число нуклонов в атоме $\text{кислород-16}$.
}
\answer{%
    $Z = 8$ протонов и столько же электронов, $A = 16$ нуклонов, $A - Z = 8$ нейтронов.
    Ответ: 16
}

\tasknumber{18}%
\task{%
    Определите число электронов в атоме $\text{аргон-36}$.
}
\answer{%
    $Z = 18$ протонов и столько же электронов, $A = 36$ нуклонов, $A - Z = 18$ нейтронов.
    Ответ: 18
}

\tasknumber{19}%
\task{%
    Определите число электронов в атоме $\text{калий-41}$.
}
\answer{%
    $Z = 19$ протонов и столько же электронов, $A = 41$ нуклонов, $A - Z = 22$ нейтронов.
    Ответ: 19
}

\tasknumber{20}%
\task{%
    Определите число нуклонов в атоме $\text{кальций-48}$.
}
\answer{%
    $Z = 20$ протонов и столько же электронов, $A = 48$ нуклонов, $A - Z = 28$ нейтронов.
    Ответ: 48
}

\tasknumber{21}%
\task{%
    Определите число протонов в атоме $\text{селен-74}$.
}
\answer{%
    $Z = 34$ протонов и столько же электронов, $A = 74$ нуклонов, $A - Z = 40$ нейтронов.
    Ответ: 34
}

\tasknumber{22}%
\task{%
    Определите число протонов в атоме $\text{селен-80}$.
}
\answer{%
    $Z = 34$ протонов и столько же электронов, $A = 80$ нуклонов, $A - Z = 46$ нейтронов.
    Ответ: 34
}

\tasknumber{23}%
\task{%
    Определите число электронов в атоме $\text{сурьма-123}$.
}
\answer{%
    $Z = 51$ протонов и столько же электронов, $A = 123$ нуклонов, $A - Z = 72$ нейтронов.
    Ответ: 51
}

\tasknumber{24}%
\task{%
    Определите число электронов в атоме $\text{европий-153}$.
}
\answer{%
    $Z = 63$ протонов и столько же электронов, $A = 153$ нуклонов, $A - Z = 90$ нейтронов.
    Ответ: 63
}

\variantsplitter

\addpersonalvariant{Кирилл Севрюгин}

\tasknumber{1}%
\task{%
    Определите число электронов в атоме $\ce{^{3}_{1}{T}}$.
}
\answer{%
    $Z = 1$ протонов и столько же электронов $A = 3$ нуклонов, $A - Z = 2$ нейтронов.
    Ответ: 1
}

\tasknumber{2}%
\task{%
    Определите число электронов в атоме $\ce{^{4}_{2}{He}}$.
}
\answer{%
    $Z = 2$ протонов и столько же электронов $A = 4$ нуклонов, $A - Z = 2$ нейтронов.
    Ответ: 2
}

\tasknumber{3}%
\task{%
    Определите число электронов в атоме $\ce{^{10}_{4}{Be}}$.
}
\answer{%
    $Z = 4$ протонов и столько же электронов $A = 10$ нуклонов, $A - Z = 6$ нейтронов.
    Ответ: 4
}

\tasknumber{4}%
\task{%
    Определите число нуклонов в атоме $\ce{^{10}_{5}{B}}$.
}
\answer{%
    $Z = 5$ протонов и столько же электронов $A = 10$ нуклонов, $A - Z = 5$ нейтронов.
    Ответ: 10
}

\tasknumber{5}%
\task{%
    Определите число нейтронов в атоме $\ce{^{15}_{7}{N}}$.
}
\answer{%
    $Z = 7$ протонов и столько же электронов $A = 15$ нуклонов, $A - Z = 8$ нейтронов.
    Ответ: 8
}

\tasknumber{6}%
\task{%
    Определите число электронов в атоме $\ce{^{23}_{11}{Na}}$.
}
\answer{%
    $Z = 11$ протонов и столько же электронов $A = 23$ нуклонов, $A - Z = 12$ нейтронов.
    Ответ: 11
}

\tasknumber{7}%
\task{%
    Определите число нейтронов в атоме $\ce{^{39}_{19}{K}}$.
}
\answer{%
    $Z = 19$ протонов и столько же электронов $A = 39$ нуклонов, $A - Z = 20$ нейтронов.
    Ответ: 20
}

\tasknumber{8}%
\task{%
    Определите число нуклонов в атоме $\ce{^{41}_{20}{Ca}}$.
}
\answer{%
    $Z = 20$ протонов и столько же электронов $A = 41$ нуклонов, $A - Z = 21$ нейтронов.
    Ответ: 41
}

\tasknumber{9}%
\task{%
    Определите число протонов в атоме $\ce{^{77}_{33}{As}}$.
}
\answer{%
    $Z = 33$ протонов и столько же электронов $A = 77$ нуклонов, $A - Z = 44$ нейтронов.
    Ответ: 33
}

\tasknumber{10}%
\task{%
    Определите число нуклонов в атоме $\ce{^{80}_{34}{Se}}$.
}
\answer{%
    $Z = 34$ протонов и столько же электронов $A = 80$ нуклонов, $A - Z = 46$ нейтронов.
    Ответ: 80
}

\tasknumber{11}%
\task{%
    Определите число протонов в атоме $\ce{^{128}_{51}{Sb}}$.
}
\answer{%
    $Z = 51$ протонов и столько же электронов $A = 128$ нуклонов, $A - Z = 77$ нейтронов.
    Ответ: 51
}

\tasknumber{12}%
\task{%
    Определите число нуклонов в атоме $\ce{^{150}_{60}{Nd}}$.
}
\answer{%
    $Z = 60$ протонов и столько же электронов $A = 150$ нуклонов, $A - Z = 90$ нейтронов.
    Ответ: 150
}

\tasknumber{13}%
\task{%
    Определите число электронов в атоме $\text{дейтерий-2}$.
}
\answer{%
    $Z = 1$ протонов и столько же электронов, $A = 2$ нуклонов, $A - Z = 1$ нейтронов.
    Ответ: 1
}

\tasknumber{14}%
\task{%
    Определите число протонов в атоме $\text{гелий-4}$.
}
\answer{%
    $Z = 2$ протонов и столько же электронов, $A = 4$ нуклонов, $A - Z = 2$ нейтронов.
    Ответ: 2
}

\tasknumber{15}%
\task{%
    Определите число протонов в атоме $\text{бериллий-7}$.
}
\answer{%
    $Z = 4$ протонов и столько же электронов, $A = 7$ нуклонов, $A - Z = 3$ нейтронов.
    Ответ: 4
}

\tasknumber{16}%
\task{%
    Определите число электронов в атоме $\text{азот-14}$.
}
\answer{%
    $Z = 7$ протонов и столько же электронов, $A = 14$ нуклонов, $A - Z = 7$ нейтронов.
    Ответ: 7
}

\tasknumber{17}%
\task{%
    Определите число нуклонов в атоме $\text{неон-20}$.
}
\answer{%
    $Z = 10$ протонов и столько же электронов, $A = 20$ нуклонов, $A - Z = 10$ нейтронов.
    Ответ: 20
}

\tasknumber{18}%
\task{%
    Определите число нейтронов в атоме $\text{магний-24}$.
}
\answer{%
    $Z = 12$ протонов и столько же электронов, $A = 24$ нуклонов, $A - Z = 12$ нейтронов.
    Ответ: 12
}

\tasknumber{19}%
\task{%
    Определите число протонов в атоме $\text{аргон-39}$.
}
\answer{%
    $Z = 18$ протонов и столько же электронов, $A = 39$ нуклонов, $A - Z = 21$ нейтронов.
    Ответ: 18
}

\tasknumber{20}%
\task{%
    Определите число нуклонов в атоме $\text{кальций-48}$.
}
\answer{%
    $Z = 20$ протонов и столько же электронов, $A = 48$ нуклонов, $A - Z = 28$ нейтронов.
    Ответ: 48
}

\tasknumber{21}%
\task{%
    Определите число нуклонов в атоме $\text{селен-76}$.
}
\answer{%
    $Z = 34$ протонов и столько же электронов, $A = 76$ нуклонов, $A - Z = 42$ нейтронов.
    Ответ: 76
}

\tasknumber{22}%
\task{%
    Определите число нуклонов в атоме $\text{серебро-107}$.
}
\answer{%
    $Z = 47$ протонов и столько же электронов, $A = 107$ нуклонов, $A - Z = 60$ нейтронов.
    Ответ: 107
}

\tasknumber{23}%
\task{%
    Определите число нуклонов в атоме $\text{сурьма-126}$.
}
\answer{%
    $Z = 51$ протонов и столько же электронов, $A = 126$ нуклонов, $A - Z = 75$ нейтронов.
    Ответ: 126
}

\tasknumber{24}%
\task{%
    Определите число нуклонов в атоме $\text{самарий-147}$.
}
\answer{%
    $Z = 62$ протонов и столько же электронов, $A = 147$ нуклонов, $A - Z = 85$ нейтронов.
    Ответ: 147
}

\variantsplitter

\addpersonalvariant{Илья Стратонников}

\tasknumber{1}%
\task{%
    Определите число электронов в атоме $\ce{^{3}_{1}{T}}$.
}
\answer{%
    $Z = 1$ протонов и столько же электронов $A = 3$ нуклонов, $A - Z = 2$ нейтронов.
    Ответ: 1
}

\tasknumber{2}%
\task{%
    Определите число электронов в атоме $\ce{^{6}_{2}{He}}$.
}
\answer{%
    $Z = 2$ протонов и столько же электронов $A = 6$ нуклонов, $A - Z = 4$ нейтронов.
    Ответ: 2
}

\tasknumber{3}%
\task{%
    Определите число нуклонов в атоме $\ce{^{9}_{4}{Be}}$.
}
\answer{%
    $Z = 4$ протонов и столько же электронов $A = 9$ нуклонов, $A - Z = 5$ нейтронов.
    Ответ: 9
}

\tasknumber{4}%
\task{%
    Определите число протонов в атоме $\ce{^{12}_{6}{C}}$.
}
\answer{%
    $Z = 6$ протонов и столько же электронов $A = 12$ нуклонов, $A - Z = 6$ нейтронов.
    Ответ: 6
}

\tasknumber{5}%
\task{%
    Определите число нуклонов в атоме $\ce{^{18}_{8}{O}}$.
}
\answer{%
    $Z = 8$ протонов и столько же электронов $A = 18$ нуклонов, $A - Z = 10$ нейтронов.
    Ответ: 18
}

\tasknumber{6}%
\task{%
    Определите число нуклонов в атоме $\ce{^{22}_{10}{Ne}}$.
}
\answer{%
    $Z = 10$ протонов и столько же электронов $A = 22$ нуклонов, $A - Z = 12$ нейтронов.
    Ответ: 22
}

\tasknumber{7}%
\task{%
    Определите число нейтронов в атоме $\ce{^{39}_{19}{K}}$.
}
\answer{%
    $Z = 19$ протонов и столько же электронов $A = 39$ нуклонов, $A - Z = 20$ нейтронов.
    Ответ: 20
}

\tasknumber{8}%
\task{%
    Определите число протонов в атоме $\ce{^{43}_{20}{Ca}}$.
}
\answer{%
    $Z = 20$ протонов и столько же электронов $A = 43$ нуклонов, $A - Z = 23$ нейтронов.
    Ответ: 20
}

\tasknumber{9}%
\task{%
    Определите число нуклонов в атоме $\ce{^{71}_{33}{As}}$.
}
\answer{%
    $Z = 33$ протонов и столько же электронов $A = 71$ нуклонов, $A - Z = 38$ нейтронов.
    Ответ: 71
}

\tasknumber{10}%
\task{%
    Определите число нейтронов в атоме $\ce{^{111}_{47}{Ag}}$.
}
\answer{%
    $Z = 47$ протонов и столько же электронов $A = 111$ нуклонов, $A - Z = 64$ нейтронов.
    Ответ: 64
}

\tasknumber{11}%
\task{%
    Определите число нуклонов в атоме $\ce{^{126}_{53}{I}}$.
}
\answer{%
    $Z = 53$ протонов и столько же электронов $A = 126$ нуклонов, $A - Z = 73$ нейтронов.
    Ответ: 126
}

\tasknumber{12}%
\task{%
    Определите число нейтронов в атоме $\ce{^{146}_{60}{Nd}}$.
}
\answer{%
    $Z = 60$ протонов и столько же электронов $A = 146$ нуклонов, $A - Z = 86$ нейтронов.
    Ответ: 86
}

\tasknumber{13}%
\task{%
    Определите число электронов в атоме $\text{гелий-3}$.
}
\answer{%
    $Z = 2$ протонов и столько же электронов, $A = 3$ нуклонов, $A - Z = 1$ нейтронов.
    Ответ: 2
}

\tasknumber{14}%
\task{%
    Определите число протонов в атоме $\text{гелий-6}$.
}
\answer{%
    $Z = 2$ протонов и столько же электронов, $A = 6$ нуклонов, $A - Z = 4$ нейтронов.
    Ответ: 2
}

\tasknumber{15}%
\task{%
    Определите число нейтронов в атоме $\text{бериллий-7}$.
}
\answer{%
    $Z = 4$ протонов и столько же электронов, $A = 7$ нуклонов, $A - Z = 3$ нейтронов.
    Ответ: 3
}

\tasknumber{16}%
\task{%
    Определите число протонов в атоме $\text{бор-10}$.
}
\answer{%
    $Z = 5$ протонов и столько же электронов, $A = 10$ нуклонов, $A - Z = 5$ нейтронов.
    Ответ: 5
}

\tasknumber{17}%
\task{%
    Определите число нуклонов в атоме $\text{кислород-18}$.
}
\answer{%
    $Z = 8$ протонов и столько же электронов, $A = 18$ нуклонов, $A - Z = 10$ нейтронов.
    Ответ: 18
}

\tasknumber{18}%
\task{%
    Определите число протонов в атоме $\text{фосфор-33}$.
}
\answer{%
    $Z = 15$ протонов и столько же электронов, $A = 33$ нуклонов, $A - Z = 18$ нейтронов.
    Ответ: 15
}

\tasknumber{19}%
\task{%
    Определите число протонов в атоме $\text{кальций-41}$.
}
\answer{%
    $Z = 20$ протонов и столько же электронов, $A = 41$ нуклонов, $A - Z = 21$ нейтронов.
    Ответ: 20
}

\tasknumber{20}%
\task{%
    Определите число нейтронов в атоме $\text{кальций-43}$.
}
\answer{%
    $Z = 20$ протонов и столько же электронов, $A = 43$ нуклонов, $A - Z = 23$ нейтронов.
    Ответ: 23
}

\tasknumber{21}%
\task{%
    Определите число протонов в атоме $\text{мышьяк-74}$.
}
\answer{%
    $Z = 33$ протонов и столько же электронов, $A = 74$ нуклонов, $A - Z = 41$ нейтронов.
    Ответ: 33
}

\tasknumber{22}%
\task{%
    Определите число протонов в атоме $\text{серебро-105}$.
}
\answer{%
    $Z = 47$ протонов и столько же электронов, $A = 105$ нуклонов, $A - Z = 58$ нейтронов.
    Ответ: 47
}

\tasknumber{23}%
\task{%
    Определите число электронов в атоме $\text{сурьма-128}$.
}
\answer{%
    $Z = 51$ протонов и столько же электронов, $A = 128$ нуклонов, $A - Z = 77$ нейтронов.
    Ответ: 51
}

\tasknumber{24}%
\task{%
    Определите число электронов в атоме $\text{неодим-145}$.
}
\answer{%
    $Z = 60$ протонов и столько же электронов, $A = 145$ нуклонов, $A - Z = 85$ нейтронов.
    Ответ: 60
}

\variantsplitter

\addpersonalvariant{Иван Шустов}

\tasknumber{1}%
\task{%
    Определите число электронов в атоме $\ce{^{3}_{1}{T}}$.
}
\answer{%
    $Z = 1$ протонов и столько же электронов $A = 3$ нуклонов, $A - Z = 2$ нейтронов.
    Ответ: 1
}

\tasknumber{2}%
\task{%
    Определите число электронов в атоме $\ce{^{8}_{2}{He}}$.
}
\answer{%
    $Z = 2$ протонов и столько же электронов $A = 8$ нуклонов, $A - Z = 6$ нейтронов.
    Ответ: 2
}

\tasknumber{3}%
\task{%
    Определите число электронов в атоме $\ce{^{10}_{4}{Be}}$.
}
\answer{%
    $Z = 4$ протонов и столько же электронов $A = 10$ нуклонов, $A - Z = 6$ нейтронов.
    Ответ: 4
}

\tasknumber{4}%
\task{%
    Определите число электронов в атоме $\ce{^{10}_{5}{B}}$.
}
\answer{%
    $Z = 5$ протонов и столько же электронов $A = 10$ нуклонов, $A - Z = 5$ нейтронов.
    Ответ: 5
}

\tasknumber{5}%
\task{%
    Определите число электронов в атоме $\ce{^{18}_{8}{O}}$.
}
\answer{%
    $Z = 8$ протонов и столько же электронов $A = 18$ нуклонов, $A - Z = 10$ нейтронов.
    Ответ: 8
}

\tasknumber{6}%
\task{%
    Определите число протонов в атоме $\ce{^{31}_{15}{P}}$.
}
\answer{%
    $Z = 15$ протонов и столько же электронов $A = 31$ нуклонов, $A - Z = 16$ нейтронов.
    Ответ: 15
}

\tasknumber{7}%
\task{%
    Определите число нуклонов в атоме $\ce{^{41}_{19}{K}}$.
}
\answer{%
    $Z = 19$ протонов и столько же электронов $A = 41$ нуклонов, $A - Z = 22$ нейтронов.
    Ответ: 41
}

\tasknumber{8}%
\task{%
    Определите число нейтронов в атоме $\ce{^{45}_{20}{Ca}}$.
}
\answer{%
    $Z = 20$ протонов и столько же электронов $A = 45$ нуклонов, $A - Z = 25$ нейтронов.
    Ответ: 25
}

\tasknumber{9}%
\task{%
    Определите число нейтронов в атоме $\ce{^{74}_{34}{Se}}$.
}
\answer{%
    $Z = 34$ протонов и столько же электронов $A = 74$ нуклонов, $A - Z = 40$ нейтронов.
    Ответ: 40
}

\tasknumber{10}%
\task{%
    Определите число электронов в атоме $\ce{^{79}_{34}{Se}}$.
}
\answer{%
    $Z = 34$ протонов и столько же электронов $A = 79$ нуклонов, $A - Z = 45$ нейтронов.
    Ответ: 34
}

\tasknumber{11}%
\task{%
    Определите число протонов в атоме $\ce{^{119}_{51}{Sb}}$.
}
\answer{%
    $Z = 51$ протонов и столько же электронов $A = 119$ нуклонов, $A - Z = 68$ нейтронов.
    Ответ: 51
}

\tasknumber{12}%
\task{%
    Определите число электронов в атоме $\ce{^{151}_{63}{Eu}}$.
}
\answer{%
    $Z = 63$ протонов и столько же электронов $A = 151$ нуклонов, $A - Z = 88$ нейтронов.
    Ответ: 63
}

\tasknumber{13}%
\task{%
    Определите число протонов в атоме $\text{гелий-3}$.
}
\answer{%
    $Z = 2$ протонов и столько же электронов, $A = 3$ нуклонов, $A - Z = 1$ нейтронов.
    Ответ: 2
}

\tasknumber{14}%
\task{%
    Определите число нейтронов в атоме $\text{гелий-4}$.
}
\answer{%
    $Z = 2$ протонов и столько же электронов, $A = 4$ нуклонов, $A - Z = 2$ нейтронов.
    Ответ: 2
}

\tasknumber{15}%
\task{%
    Определите число нуклонов в атоме $\text{бериллий-10}$.
}
\answer{%
    $Z = 4$ протонов и столько же электронов, $A = 10$ нуклонов, $A - Z = 6$ нейтронов.
    Ответ: 10
}

\tasknumber{16}%
\task{%
    Определите число нейтронов в атоме $\text{бор-10}$.
}
\answer{%
    $Z = 5$ протонов и столько же электронов, $A = 10$ нуклонов, $A - Z = 5$ нейтронов.
    Ответ: 5
}

\tasknumber{17}%
\task{%
    Определите число протонов в атоме $\text{кислород-16}$.
}
\answer{%
    $Z = 8$ протонов и столько же электронов, $A = 16$ нуклонов, $A - Z = 8$ нейтронов.
    Ответ: 8
}

\tasknumber{18}%
\task{%
    Определите число протонов в атоме $\text{аргон-36}$.
}
\answer{%
    $Z = 18$ протонов и столько же электронов, $A = 36$ нуклонов, $A - Z = 18$ нейтронов.
    Ответ: 18
}

\tasknumber{19}%
\task{%
    Определите число электронов в атоме $\text{кальций-41}$.
}
\answer{%
    $Z = 20$ протонов и столько же электронов, $A = 41$ нуклонов, $A - Z = 21$ нейтронов.
    Ответ: 20
}

\tasknumber{20}%
\task{%
    Определите число протонов в атоме $\text{медь-63}$.
}
\answer{%
    $Z = 29$ протонов и столько же электронов, $A = 63$ нуклонов, $A - Z = 34$ нейтронов.
    Ответ: 29
}

\tasknumber{21}%
\task{%
    Определите число протонов в атоме $\text{мышьяк-75}$.
}
\answer{%
    $Z = 33$ протонов и столько же электронов, $A = 75$ нуклонов, $A - Z = 42$ нейтронов.
    Ответ: 33
}

\tasknumber{22}%
\task{%
    Определите число протонов в атоме $\text{серебро-107}$.
}
\answer{%
    $Z = 47$ протонов и столько же электронов, $A = 107$ нуклонов, $A - Z = 60$ нейтронов.
    Ответ: 47
}

\tasknumber{23}%
\task{%
    Определите число нейтронов в атоме $\text{йод-125}$.
}
\answer{%
    $Z = 53$ протонов и столько же электронов, $A = 125$ нуклонов, $A - Z = 72$ нейтронов.
    Ответ: 72
}

\tasknumber{24}%
\task{%
    Определите число протонов в атоме $\text{самарий-148}$.
}
\answer{%
    $Z = 62$ протонов и столько же электронов, $A = 148$ нуклонов, $A - Z = 86$ нейтронов.
    Ответ: 62
}
% autogenerated
