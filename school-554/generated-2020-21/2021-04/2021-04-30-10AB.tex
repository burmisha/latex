\setdate{30~апреля~2021}
\setclass{10«АБ»}

\addpersonalvariant{Михаил Бурмистров}

\tasknumber{1}%
\task{%
    Напротив физических величин укажите их обозначения и единицы измерения в СИ:
    \begin{enumerate}
        \item сила тока,
        \item работа тока,
        \item ЭДС,
        \item внутреннее сопротивление полной цепи.
    \end{enumerate}
}
\solutionspace{20pt}

\tasknumber{2}%
\task{%
    Запишите физический закон или формулу:
    \begin{enumerate}
        \item правило Кирхгофа для узла цепи,
        \item закон Ома для однородного участка цепи,
        \item ЭДС (определение).
    \end{enumerate}
}
\solutionspace{40pt}

\tasknumber{3}%
\task{%
    На резистор сопротивлением $R = 5\,\text{Ом}$ подали напряжение $V = 120\,\text{В}$.
    Определите ток, который потечёт через резистор, и мощность, выделяющуюся на нём.
}
\answer{%
    \begin{align*}
    \eli &= \frac{V}{R} = \frac{120\,\text{В}}{5\,\text{Ом}} = 24\,\text{А},  \\
    P &= \frac{V^2}{R} = \frac{\sqr{120\,\text{В}}}{5\,\text{Ом}} = 2880\,\text{Вт}
    \end{align*}
}
\solutionspace{60pt}

\tasknumber{4}%
\task{%
    Через резистор сопротивлением $R = 12\,\text{Ом}$ протекает электрический ток $\eli = 3\,\text{А}$.
    Определите, чему равны напряжение на резисторе и мощность, выделяющаяся на нём.
}
\answer{%
    \begin{align*}
    U &= \eli R = 3\,\text{А} \cdot 12\,\text{Ом} = 36\,\text{В},  \\
    P &= \eli^2R = \sqr{3\,\text{А}} \cdot 12\,\text{Ом} = 108\,\text{Вт}
    \end{align*}
}
\solutionspace{60pt}

\tasknumber{5}%
\task{%
    Замкнутая электрическая цепь состоит из ЭДС $\ele = 2\,\text{В}$ и сопротивлением $r$
    и резистора $R = 24\,\text{Ом}$.
    Определите ток, протекающий в цепи.
    Какая тепловая энергия выделится на резисторе за время
    $\tau = 5\,\text{с}$? Какая работа будет совершена ЭДС за это время? Каков знак этой работы? Чему равен КПД цепи?
    Вычислите значения для 2 случаев: $r=0$ и $r = 10\,\text{Ом}$.
}
\answer{%
    \begin{align*}
    \eli_1 &= \frac{\ele}{R} = \frac{2\,\text{В}}{24\,\text{Ом}} = \frac1{12}\units{А} \approx 0{,}08\,\text{А},  \\
    \eli_2 &= \frac{\ele}{R + r} = \frac{2\,\text{В}}{24\,\text{Ом} + 10\,\text{Ом}} = \frac1{17}\units{А} \approx 0{,}06\,\text{А},  \\
    Q_1 &= \eli_1^2R\tau = \sqr{\frac{\ele}{R}} R \tau
            = \sqr{\frac{2\,\text{В}}{24\,\text{Ом}}} \cdot 24\,\text{Ом} \cdot 5\,\text{с} = \frac56\units{Дж} \approx 0{,}833\,\text{Дж},  \\
    Q_2 &= \eli_2^2R\tau = \sqr{\frac{\ele}{R + r}} R \tau
            = \sqr{\frac{2\,\text{В}}{24\,\text{Ом} + 10\,\text{Ом}}} \cdot 24\,\text{Ом} \cdot 5\,\text{с} = \frac{120}{289}\units{Дж} \approx 0{,}415\,\text{Дж},  \\
    A_1 &= q_1\ele = \eli_1\tau\ele = \frac{\ele}{R} \tau \ele
            = \frac{\ele^2 \tau}{R} = \frac{\sqr{2\,\text{В}} \cdot 5\,\text{с}}{24\,\text{Ом}}
            = \frac56\units{Дж} \approx 0{,}833\,\text{Дж}, \text{положительна},  \\
    A_2 &= q_2\ele = \eli_2\tau\ele = \frac{\ele}{R + r} \tau \ele
            = \frac{\ele^2 \tau}{R + r} = \frac{\sqr{2\,\text{В}} \cdot 5\,\text{с}}{24\,\text{Ом} + 10\,\text{Ом}}
            = \frac{10}{17}\units{Дж} \approx 0{,}588\,\text{Дж}, \text{положительна},  \\
    \eta_1 &= \frac{Q_1}{A_1} = \ldots = \frac{R}{R} = 1,  \\
    \eta_2 &= \frac{Q_2}{A_2} = \ldots = \frac{R}{R + r} = \frac{12}{17} \approx 0{,}71.
    \end{align*}
}
\solutionspace{180pt}

\tasknumber{6}%
\task{%
    Лампочки, сопротивления которых $R_1 = 3\,\text{Ом}$ и $R_2 = 12\,\text{Ом}$, поочерёдно подключённные к некоторому источнику тока,
    потребляют одинаковую мощность.
    Найти внутреннее сопротивление источника и КПД цепи в каждом случае.
}
\answer{%
    \begin{align*}
        P_1 &= \sqr{\frac{\ele}{R_1 + r}}R_1,
        P_2  = \sqr{\frac{\ele}{R_2 + r}}R_2,
        P_1 = P_2 \implies  \\
        &\implies R_1 \sqr{R_2 + r} = R_2 \sqr{R_1 + r} \implies  \\
        &\implies R_1 R_2^2 + 2 R_1 R_2 r + R_1 r^2 =
                    R_2 R_1^2 + 2 R_2 R_1 r + R_2 r^2  \implies  \\
    &\implies r^2 (R_2 - R_1) = R_2^2 R_2 - R_1^2 R_2 \implies  \\
    &\implies r
            = \sqrt{R_1 R_2 \frac{R_2 - R_1}{R_2 - R_1}}
            = \sqrt{R_1 R_2}
            = \sqrt{3\,\text{Ом} \cdot 12\,\text{Ом}}
            = 6\,\text{Ом}.
            \\
    \eta_1
            &= \frac{R_1}{R_1 + r}
            = \frac{\sqrt{R_1}}{\sqrt{R_1} + \sqrt{R_2}}
            = 0{,}333,  \\
    \eta_2
            &= \frac{R_2}{R_2 + r}
            = \frac{\sqrt{R_2}}{\sqrt{R_2} + \sqrt{R_1}}
            = 0{,}667
    \end{align*}
}

\variantsplitter

\addpersonalvariant{Ирина Ан}

\tasknumber{1}%
\task{%
    Напротив физических величин укажите их обозначения и единицы измерения в СИ:
    \begin{enumerate}
        \item разность потенциалов,
        \item работа тока,
        \item ЭДС,
        \item внутреннее сопротивление полной цепи.
    \end{enumerate}
}
\solutionspace{20pt}

\tasknumber{2}%
\task{%
    Запишите физический закон или формулу:
    \begin{enumerate}
        \item правило Кирхгофа для замкнутого контура,
        \item сопротивление резистора через удельное сопротивление,
        \item ЭДС (определение).
    \end{enumerate}
}
\solutionspace{40pt}

\tasknumber{3}%
\task{%
    На резистор сопротивлением $r = 18\,\text{Ом}$ подали напряжение $V = 120\,\text{В}$.
    Определите ток, который потечёт через резистор, и мощность, выделяющуюся на нём.
}
\answer{%
    \begin{align*}
    \eli &= \frac{V}{r} = \frac{120\,\text{В}}{18\,\text{Ом}} = 6{,}67\,\text{А},  \\
    P &= \frac{V^2}{r} = \frac{\sqr{120\,\text{В}}}{18\,\text{Ом}} = 800\,\text{Вт}
    \end{align*}
}
\solutionspace{60pt}

\tasknumber{4}%
\task{%
    Через резистор сопротивлением $R = 5\,\text{Ом}$ протекает электрический ток $\eli = 3\,\text{А}$.
    Определите, чему равны напряжение на резисторе и мощность, выделяющаяся на нём.
}
\answer{%
    \begin{align*}
    U &= \eli R = 3\,\text{А} \cdot 5\,\text{Ом} = 15\,\text{В},  \\
    P &= \eli^2R = \sqr{3\,\text{А}} \cdot 5\,\text{Ом} = 45\,\text{Вт}
    \end{align*}
}
\solutionspace{60pt}

\tasknumber{5}%
\task{%
    Замкнутая электрическая цепь состоит из ЭДС $\ele = 2\,\text{В}$ и сопротивлением $r$
    и резистора $R = 30\,\text{Ом}$.
    Определите ток, протекающий в цепи.
    Какая тепловая энергия выделится на резисторе за время
    $\tau = 5\,\text{с}$? Какая работа будет совершена ЭДС за это время? Каков знак этой работы? Чему равен КПД цепи?
    Вычислите значения для 2 случаев: $r=0$ и $r = 30\,\text{Ом}$.
}
\answer{%
    \begin{align*}
    \eli_1 &= \frac{\ele}{R} = \frac{2\,\text{В}}{30\,\text{Ом}} = \frac1{15}\units{А} \approx 0{,}07\,\text{А},  \\
    \eli_2 &= \frac{\ele}{R + r} = \frac{2\,\text{В}}{30\,\text{Ом} + 30\,\text{Ом}} = \frac1{30}\units{А} \approx 0{,}03\,\text{А},  \\
    Q_1 &= \eli_1^2R\tau = \sqr{\frac{\ele}{R}} R \tau
            = \sqr{\frac{2\,\text{В}}{30\,\text{Ом}}} \cdot 30\,\text{Ом} \cdot 5\,\text{с} = \frac23\units{Дж} \approx 0{,}667\,\text{Дж},  \\
    Q_2 &= \eli_2^2R\tau = \sqr{\frac{\ele}{R + r}} R \tau
            = \sqr{\frac{2\,\text{В}}{30\,\text{Ом} + 30\,\text{Ом}}} \cdot 30\,\text{Ом} \cdot 5\,\text{с} = \frac16\units{Дж} \approx 0{,}167\,\text{Дж},  \\
    A_1 &= q_1\ele = \eli_1\tau\ele = \frac{\ele}{R} \tau \ele
            = \frac{\ele^2 \tau}{R} = \frac{\sqr{2\,\text{В}} \cdot 5\,\text{с}}{30\,\text{Ом}}
            = \frac23\units{Дж} \approx 0{,}667\,\text{Дж}, \text{положительна},  \\
    A_2 &= q_2\ele = \eli_2\tau\ele = \frac{\ele}{R + r} \tau \ele
            = \frac{\ele^2 \tau}{R + r} = \frac{\sqr{2\,\text{В}} \cdot 5\,\text{с}}{30\,\text{Ом} + 30\,\text{Ом}}
            = \frac13\units{Дж} \approx 0{,}333\,\text{Дж}, \text{положительна},  \\
    \eta_1 &= \frac{Q_1}{A_1} = \ldots = \frac{R}{R} = 1,  \\
    \eta_2 &= \frac{Q_2}{A_2} = \ldots = \frac{R}{R + r} = \frac12 \approx 0{,}50.
    \end{align*}
}
\solutionspace{180pt}

\tasknumber{6}%
\task{%
    Лампочки, сопротивления которых $R_1 = 0{,}25\,\text{Ом}$ и $R_2 = 4\,\text{Ом}$, поочерёдно подключённные к некоторому источнику тока,
    потребляют одинаковую мощность.
    Найти внутреннее сопротивление источника и КПД цепи в каждом случае.
}
\answer{%
    \begin{align*}
        P_1 &= \sqr{\frac{\ele}{R_1 + r}}R_1,
        P_2  = \sqr{\frac{\ele}{R_2 + r}}R_2,
        P_1 = P_2 \implies  \\
        &\implies R_1 \sqr{R_2 + r} = R_2 \sqr{R_1 + r} \implies  \\
        &\implies R_1 R_2^2 + 2 R_1 R_2 r + R_1 r^2 =
                    R_2 R_1^2 + 2 R_2 R_1 r + R_2 r^2  \implies  \\
    &\implies r^2 (R_2 - R_1) = R_2^2 R_2 - R_1^2 R_2 \implies  \\
    &\implies r
            = \sqrt{R_1 R_2 \frac{R_2 - R_1}{R_2 - R_1}}
            = \sqrt{R_1 R_2}
            = \sqrt{0{,}25\,\text{Ом} \cdot 4\,\text{Ом}}
            = 1\,\text{Ом}.
            \\
    \eta_1
            &= \frac{R_1}{R_1 + r}
            = \frac{\sqrt{R_1}}{\sqrt{R_1} + \sqrt{R_2}}
            = 0{,}200,  \\
    \eta_2
            &= \frac{R_2}{R_2 + r}
            = \frac{\sqrt{R_2}}{\sqrt{R_2} + \sqrt{R_1}}
            = 0{,}800
    \end{align*}
}

\variantsplitter

\addpersonalvariant{Софья Андрианова}

\tasknumber{1}%
\task{%
    Напротив физических величин укажите их обозначения и единицы измерения в СИ:
    \begin{enumerate}
        \item сила тока,
        \item мощность тока,
        \item ЭДС,
        \item внешнее сопротивление полной цепи.
    \end{enumerate}
}
\solutionspace{20pt}

\tasknumber{2}%
\task{%
    Запишите физический закон или формулу:
    \begin{enumerate}
        \item правило Кирхгофа для узла цепи,
        \item сопротивление резистора через удельное сопротивление,
        \item ЭДС (определение).
    \end{enumerate}
}
\solutionspace{40pt}

\tasknumber{3}%
\task{%
    На резистор сопротивлением $r = 12\,\text{Ом}$ подали напряжение $V = 150\,\text{В}$.
    Определите ток, который потечёт через резистор, и мощность, выделяющуюся на нём.
}
\answer{%
    \begin{align*}
    \eli &= \frac{V}{r} = \frac{150\,\text{В}}{12\,\text{Ом}} = 12{,}50\,\text{А},  \\
    P &= \frac{V^2}{r} = \frac{\sqr{150\,\text{В}}}{12\,\text{Ом}} = 1875\,\text{Вт}
    \end{align*}
}
\solutionspace{60pt}

\tasknumber{4}%
\task{%
    Через резистор сопротивлением $R = 30\,\text{Ом}$ протекает электрический ток $\eli = 4\,\text{А}$.
    Определите, чему равны напряжение на резисторе и мощность, выделяющаяся на нём.
}
\answer{%
    \begin{align*}
    U &= \eli R = 4\,\text{А} \cdot 30\,\text{Ом} = 120\,\text{В},  \\
    P &= \eli^2R = \sqr{4\,\text{А}} \cdot 30\,\text{Ом} = 480\,\text{Вт}
    \end{align*}
}
\solutionspace{60pt}

\tasknumber{5}%
\task{%
    Замкнутая электрическая цепь состоит из ЭДС $\ele = 4\,\text{В}$ и сопротивлением $r$
    и резистора $R = 24\,\text{Ом}$.
    Определите ток, протекающий в цепи.
    Какая тепловая энергия выделится на резисторе за время
    $\tau = 10\,\text{с}$? Какая работа будет совершена ЭДС за это время? Каков знак этой работы? Чему равен КПД цепи?
    Вычислите значения для 2 случаев: $r=0$ и $r = 30\,\text{Ом}$.
}
\answer{%
    \begin{align*}
    \eli_1 &= \frac{\ele}{R} = \frac{4\,\text{В}}{24\,\text{Ом}} = \frac16\units{А} \approx 0{,}17\,\text{А},  \\
    \eli_2 &= \frac{\ele}{R + r} = \frac{4\,\text{В}}{24\,\text{Ом} + 30\,\text{Ом}} = \frac2{27}\units{А} \approx 0{,}07\,\text{А},  \\
    Q_1 &= \eli_1^2R\tau = \sqr{\frac{\ele}{R}} R \tau
            = \sqr{\frac{4\,\text{В}}{24\,\text{Ом}}} \cdot 24\,\text{Ом} \cdot 10\,\text{с} = \frac{20}3\units{Дж} \approx 6{,}667\,\text{Дж},  \\
    Q_2 &= \eli_2^2R\tau = \sqr{\frac{\ele}{R + r}} R \tau
            = \sqr{\frac{4\,\text{В}}{24\,\text{Ом} + 30\,\text{Ом}}} \cdot 24\,\text{Ом} \cdot 10\,\text{с} = \frac{320}{243}\units{Дж} \approx 1{,}317\,\text{Дж},  \\
    A_1 &= q_1\ele = \eli_1\tau\ele = \frac{\ele}{R} \tau \ele
            = \frac{\ele^2 \tau}{R} = \frac{\sqr{4\,\text{В}} \cdot 10\,\text{с}}{24\,\text{Ом}}
            = \frac{20}3\units{Дж} \approx 6{,}667\,\text{Дж}, \text{положительна},  \\
    A_2 &= q_2\ele = \eli_2\tau\ele = \frac{\ele}{R + r} \tau \ele
            = \frac{\ele^2 \tau}{R + r} = \frac{\sqr{4\,\text{В}} \cdot 10\,\text{с}}{24\,\text{Ом} + 30\,\text{Ом}}
            = \frac{80}{27}\units{Дж} \approx 2{,}963\,\text{Дж}, \text{положительна},  \\
    \eta_1 &= \frac{Q_1}{A_1} = \ldots = \frac{R}{R} = 1,  \\
    \eta_2 &= \frac{Q_2}{A_2} = \ldots = \frac{R}{R + r} = \frac49 \approx 0{,}44.
    \end{align*}
}
\solutionspace{180pt}

\tasknumber{6}%
\task{%
    Лампочки, сопротивления которых $R_1 = 4\,\text{Ом}$ и $R_2 = 100\,\text{Ом}$, поочерёдно подключённные к некоторому источнику тока,
    потребляют одинаковую мощность.
    Найти внутреннее сопротивление источника и КПД цепи в каждом случае.
}
\answer{%
    \begin{align*}
        P_1 &= \sqr{\frac{\ele}{R_1 + r}}R_1,
        P_2  = \sqr{\frac{\ele}{R_2 + r}}R_2,
        P_1 = P_2 \implies  \\
        &\implies R_1 \sqr{R_2 + r} = R_2 \sqr{R_1 + r} \implies  \\
        &\implies R_1 R_2^2 + 2 R_1 R_2 r + R_1 r^2 =
                    R_2 R_1^2 + 2 R_2 R_1 r + R_2 r^2  \implies  \\
    &\implies r^2 (R_2 - R_1) = R_2^2 R_2 - R_1^2 R_2 \implies  \\
    &\implies r
            = \sqrt{R_1 R_2 \frac{R_2 - R_1}{R_2 - R_1}}
            = \sqrt{R_1 R_2}
            = \sqrt{4\,\text{Ом} \cdot 100\,\text{Ом}}
            = 20\,\text{Ом}.
            \\
    \eta_1
            &= \frac{R_1}{R_1 + r}
            = \frac{\sqrt{R_1}}{\sqrt{R_1} + \sqrt{R_2}}
            = 0{,}167,  \\
    \eta_2
            &= \frac{R_2}{R_2 + r}
            = \frac{\sqrt{R_2}}{\sqrt{R_2} + \sqrt{R_1}}
            = 0{,}833
    \end{align*}
}

\variantsplitter

\addpersonalvariant{Владимир Артемчук}

\tasknumber{1}%
\task{%
    Напротив физических величин укажите их обозначения и единицы измерения в СИ:
    \begin{enumerate}
        \item напряжение,
        \item мощность тока,
        \item удельное сопротивление,
        \item внутреннее сопротивление полной цепи.
    \end{enumerate}
}
\solutionspace{20pt}

\tasknumber{2}%
\task{%
    Запишите физический закон или формулу:
    \begin{enumerate}
        \item правило Кирхгофа для замкнутого контура,
        \item закон Ома для однородного участка цепи,
        \item ЭДС (определение).
    \end{enumerate}
}
\solutionspace{40pt}

\tasknumber{3}%
\task{%
    На резистор сопротивлением $r = 18\,\text{Ом}$ подали напряжение $U = 180\,\text{В}$.
    Определите ток, который потечёт через резистор, и мощность, выделяющуюся на нём.
}
\answer{%
    \begin{align*}
    \eli &= \frac{U}{r} = \frac{180\,\text{В}}{18\,\text{Ом}} = 10\,\text{А},  \\
    P &= \frac{U^2}{r} = \frac{\sqr{180\,\text{В}}}{18\,\text{Ом}} = 1800\,\text{Вт}
    \end{align*}
}
\solutionspace{60pt}

\tasknumber{4}%
\task{%
    Через резистор сопротивлением $r = 12\,\text{Ом}$ протекает электрический ток $\eli = 2\,\text{А}$.
    Определите, чему равны напряжение на резисторе и мощность, выделяющаяся на нём.
}
\answer{%
    \begin{align*}
    U &= \eli r = 2\,\text{А} \cdot 12\,\text{Ом} = 24\,\text{В},  \\
    P &= \eli^2r = \sqr{2\,\text{А}} \cdot 12\,\text{Ом} = 48\,\text{Вт}
    \end{align*}
}
\solutionspace{60pt}

\tasknumber{5}%
\task{%
    Замкнутая электрическая цепь состоит из ЭДС $\ele = 1\,\text{В}$ и сопротивлением $r$
    и резистора $R = 30\,\text{Ом}$.
    Определите ток, протекающий в цепи.
    Какая тепловая энергия выделится на резисторе за время
    $\tau = 10\,\text{с}$? Какая работа будет совершена ЭДС за это время? Каков знак этой работы? Чему равен КПД цепи?
    Вычислите значения для 2 случаев: $r=0$ и $r = 30\,\text{Ом}$.
}
\answer{%
    \begin{align*}
    \eli_1 &= \frac{\ele}{R} = \frac{1\,\text{В}}{30\,\text{Ом}} = \frac1{30}\units{А} \approx 0{,}03\,\text{А},  \\
    \eli_2 &= \frac{\ele}{R + r} = \frac{1\,\text{В}}{30\,\text{Ом} + 30\,\text{Ом}} = \frac1{60}\units{А} \approx 0{,}02\,\text{А},  \\
    Q_1 &= \eli_1^2R\tau = \sqr{\frac{\ele}{R}} R \tau
            = \sqr{\frac{1\,\text{В}}{30\,\text{Ом}}} \cdot 30\,\text{Ом} \cdot 10\,\text{с} = \frac13\units{Дж} \approx 0{,}333\,\text{Дж},  \\
    Q_2 &= \eli_2^2R\tau = \sqr{\frac{\ele}{R + r}} R \tau
            = \sqr{\frac{1\,\text{В}}{30\,\text{Ом} + 30\,\text{Ом}}} \cdot 30\,\text{Ом} \cdot 10\,\text{с} = \frac1{12}\units{Дж} \approx 0{,}083\,\text{Дж},  \\
    A_1 &= q_1\ele = \eli_1\tau\ele = \frac{\ele}{R} \tau \ele
            = \frac{\ele^2 \tau}{R} = \frac{\sqr{1\,\text{В}} \cdot 10\,\text{с}}{30\,\text{Ом}}
            = \frac13\units{Дж} \approx 0{,}333\,\text{Дж}, \text{положительна},  \\
    A_2 &= q_2\ele = \eli_2\tau\ele = \frac{\ele}{R + r} \tau \ele
            = \frac{\ele^2 \tau}{R + r} = \frac{\sqr{1\,\text{В}} \cdot 10\,\text{с}}{30\,\text{Ом} + 30\,\text{Ом}}
            = \frac16\units{Дж} \approx 0{,}167\,\text{Дж}, \text{положительна},  \\
    \eta_1 &= \frac{Q_1}{A_1} = \ldots = \frac{R}{R} = 1,  \\
    \eta_2 &= \frac{Q_2}{A_2} = \ldots = \frac{R}{R + r} = \frac12 \approx 0{,}50.
    \end{align*}
}
\solutionspace{180pt}

\tasknumber{6}%
\task{%
    Лампочки, сопротивления которых $R_1 = 1\,\text{Ом}$ и $R_2 = 49\,\text{Ом}$, поочерёдно подключённные к некоторому источнику тока,
    потребляют одинаковую мощность.
    Найти внутреннее сопротивление источника и КПД цепи в каждом случае.
}
\answer{%
    \begin{align*}
        P_1 &= \sqr{\frac{\ele}{R_1 + r}}R_1,
        P_2  = \sqr{\frac{\ele}{R_2 + r}}R_2,
        P_1 = P_2 \implies  \\
        &\implies R_1 \sqr{R_2 + r} = R_2 \sqr{R_1 + r} \implies  \\
        &\implies R_1 R_2^2 + 2 R_1 R_2 r + R_1 r^2 =
                    R_2 R_1^2 + 2 R_2 R_1 r + R_2 r^2  \implies  \\
    &\implies r^2 (R_2 - R_1) = R_2^2 R_2 - R_1^2 R_2 \implies  \\
    &\implies r
            = \sqrt{R_1 R_2 \frac{R_2 - R_1}{R_2 - R_1}}
            = \sqrt{R_1 R_2}
            = \sqrt{1\,\text{Ом} \cdot 49\,\text{Ом}}
            = 7\,\text{Ом}.
            \\
    \eta_1
            &= \frac{R_1}{R_1 + r}
            = \frac{\sqrt{R_1}}{\sqrt{R_1} + \sqrt{R_2}}
            = 0{,}125,  \\
    \eta_2
            &= \frac{R_2}{R_2 + r}
            = \frac{\sqrt{R_2}}{\sqrt{R_2} + \sqrt{R_1}}
            = 0{,}875
    \end{align*}
}

\variantsplitter

\addpersonalvariant{Софья Белянкина}

\tasknumber{1}%
\task{%
    Напротив физических величин укажите их обозначения и единицы измерения в СИ:
    \begin{enumerate}
        \item разность потенциалов,
        \item мощность тока,
        \item ЭДС,
        \item внутреннее сопротивление полной цепи.
    \end{enumerate}
}
\solutionspace{20pt}

\tasknumber{2}%
\task{%
    Запишите физический закон или формулу:
    \begin{enumerate}
        \item правило Кирхгофа для замкнутого контура,
        \item закон Ома для однородного участка цепи,
        \item закон Ома для неоднородного участка цепи.
    \end{enumerate}
}
\solutionspace{40pt}

\tasknumber{3}%
\task{%
    На резистор сопротивлением $R = 30\,\text{Ом}$ подали напряжение $V = 120\,\text{В}$.
    Определите ток, который потечёт через резистор, и мощность, выделяющуюся на нём.
}
\answer{%
    \begin{align*}
    \eli &= \frac{V}{R} = \frac{120\,\text{В}}{30\,\text{Ом}} = 4\,\text{А},  \\
    P &= \frac{V^2}{R} = \frac{\sqr{120\,\text{В}}}{30\,\text{Ом}} = 480\,\text{Вт}
    \end{align*}
}
\solutionspace{60pt}

\tasknumber{4}%
\task{%
    Через резистор сопротивлением $R = 5\,\text{Ом}$ протекает электрический ток $\eli = 4\,\text{А}$.
    Определите, чему равны напряжение на резисторе и мощность, выделяющаяся на нём.
}
\answer{%
    \begin{align*}
    U &= \eli R = 4\,\text{А} \cdot 5\,\text{Ом} = 20\,\text{В},  \\
    P &= \eli^2R = \sqr{4\,\text{А}} \cdot 5\,\text{Ом} = 80\,\text{Вт}
    \end{align*}
}
\solutionspace{60pt}

\tasknumber{5}%
\task{%
    Замкнутая электрическая цепь состоит из ЭДС $\ele = 3\,\text{В}$ и сопротивлением $r$
    и резистора $R = 15\,\text{Ом}$.
    Определите ток, протекающий в цепи.
    Какая тепловая энергия выделится на резисторе за время
    $\tau = 10\,\text{с}$? Какая работа будет совершена ЭДС за это время? Каков знак этой работы? Чему равен КПД цепи?
    Вычислите значения для 2 случаев: $r=0$ и $r = 10\,\text{Ом}$.
}
\answer{%
    \begin{align*}
    \eli_1 &= \frac{\ele}{R} = \frac{3\,\text{В}}{15\,\text{Ом}} = \frac15\units{А} \approx 0{,}20\,\text{А},  \\
    \eli_2 &= \frac{\ele}{R + r} = \frac{3\,\text{В}}{15\,\text{Ом} + 10\,\text{Ом}} = \frac3{25}\units{А} \approx 0{,}12\,\text{А},  \\
    Q_1 &= \eli_1^2R\tau = \sqr{\frac{\ele}{R}} R \tau
            = \sqr{\frac{3\,\text{В}}{15\,\text{Ом}}} \cdot 15\,\text{Ом} \cdot 10\,\text{с} = 6\units{Дж} \approx 6\,\text{Дж},  \\
    Q_2 &= \eli_2^2R\tau = \sqr{\frac{\ele}{R + r}} R \tau
            = \sqr{\frac{3\,\text{В}}{15\,\text{Ом} + 10\,\text{Ом}}} \cdot 15\,\text{Ом} \cdot 10\,\text{с} = \frac{54}{25}\units{Дж} \approx 2{,}160\,\text{Дж},  \\
    A_1 &= q_1\ele = \eli_1\tau\ele = \frac{\ele}{R} \tau \ele
            = \frac{\ele^2 \tau}{R} = \frac{\sqr{3\,\text{В}} \cdot 10\,\text{с}}{15\,\text{Ом}}
            = 6\units{Дж} \approx 6\,\text{Дж}, \text{положительна},  \\
    A_2 &= q_2\ele = \eli_2\tau\ele = \frac{\ele}{R + r} \tau \ele
            = \frac{\ele^2 \tau}{R + r} = \frac{\sqr{3\,\text{В}} \cdot 10\,\text{с}}{15\,\text{Ом} + 10\,\text{Ом}}
            = \frac{18}5\units{Дж} \approx 3{,}600\,\text{Дж}, \text{положительна},  \\
    \eta_1 &= \frac{Q_1}{A_1} = \ldots = \frac{R}{R} = 1,  \\
    \eta_2 &= \frac{Q_2}{A_2} = \ldots = \frac{R}{R + r} = \frac35 \approx 0{,}60.
    \end{align*}
}
\solutionspace{180pt}

\tasknumber{6}%
\task{%
    Лампочки, сопротивления которых $R_1 = 3\,\text{Ом}$ и $R_2 = 48\,\text{Ом}$, поочерёдно подключённные к некоторому источнику тока,
    потребляют одинаковую мощность.
    Найти внутреннее сопротивление источника и КПД цепи в каждом случае.
}
\answer{%
    \begin{align*}
        P_1 &= \sqr{\frac{\ele}{R_1 + r}}R_1,
        P_2  = \sqr{\frac{\ele}{R_2 + r}}R_2,
        P_1 = P_2 \implies  \\
        &\implies R_1 \sqr{R_2 + r} = R_2 \sqr{R_1 + r} \implies  \\
        &\implies R_1 R_2^2 + 2 R_1 R_2 r + R_1 r^2 =
                    R_2 R_1^2 + 2 R_2 R_1 r + R_2 r^2  \implies  \\
    &\implies r^2 (R_2 - R_1) = R_2^2 R_2 - R_1^2 R_2 \implies  \\
    &\implies r
            = \sqrt{R_1 R_2 \frac{R_2 - R_1}{R_2 - R_1}}
            = \sqrt{R_1 R_2}
            = \sqrt{3\,\text{Ом} \cdot 48\,\text{Ом}}
            = 12\,\text{Ом}.
            \\
    \eta_1
            &= \frac{R_1}{R_1 + r}
            = \frac{\sqrt{R_1}}{\sqrt{R_1} + \sqrt{R_2}}
            = 0{,}200,  \\
    \eta_2
            &= \frac{R_2}{R_2 + r}
            = \frac{\sqrt{R_2}}{\sqrt{R_2} + \sqrt{R_1}}
            = 0{,}800
    \end{align*}
}

\variantsplitter

\addpersonalvariant{Варвара Егиазарян}

\tasknumber{1}%
\task{%
    Напротив физических величин укажите их обозначения и единицы измерения в СИ:
    \begin{enumerate}
        \item сила тока,
        \item работа тока,
        \item удельное сопротивление,
        \item внутреннее сопротивление полной цепи.
    \end{enumerate}
}
\solutionspace{20pt}

\tasknumber{2}%
\task{%
    Запишите физический закон или формулу:
    \begin{enumerate}
        \item правило Кирхгофа для замкнутого контура,
        \item закон Ома для однородного участка цепи,
        \item закон Ома для неоднородного участка цепи.
    \end{enumerate}
}
\solutionspace{40pt}

\tasknumber{3}%
\task{%
    На резистор сопротивлением $r = 12\,\text{Ом}$ подали напряжение $U = 180\,\text{В}$.
    Определите ток, который потечёт через резистор, и мощность, выделяющуюся на нём.
}
\answer{%
    \begin{align*}
    \eli &= \frac{U}{r} = \frac{180\,\text{В}}{12\,\text{Ом}} = 15\,\text{А},  \\
    P &= \frac{U^2}{r} = \frac{\sqr{180\,\text{В}}}{12\,\text{Ом}} = 2700\,\text{Вт}
    \end{align*}
}
\solutionspace{60pt}

\tasknumber{4}%
\task{%
    Через резистор сопротивлением $R = 18\,\text{Ом}$ протекает электрический ток $\eli = 8\,\text{А}$.
    Определите, чему равны напряжение на резисторе и мощность, выделяющаяся на нём.
}
\answer{%
    \begin{align*}
    U &= \eli R = 8\,\text{А} \cdot 18\,\text{Ом} = 144\,\text{В},  \\
    P &= \eli^2R = \sqr{8\,\text{А}} \cdot 18\,\text{Ом} = 1152\,\text{Вт}
    \end{align*}
}
\solutionspace{60pt}

\tasknumber{5}%
\task{%
    Замкнутая электрическая цепь состоит из ЭДС $\ele = 1\,\text{В}$ и сопротивлением $r$
    и резистора $R = 24\,\text{Ом}$.
    Определите ток, протекающий в цепи.
    Какая тепловая энергия выделится на резисторе за время
    $\tau = 2\,\text{с}$? Какая работа будет совершена ЭДС за это время? Каков знак этой работы? Чему равен КПД цепи?
    Вычислите значения для 2 случаев: $r=0$ и $r = 20\,\text{Ом}$.
}
\answer{%
    \begin{align*}
    \eli_1 &= \frac{\ele}{R} = \frac{1\,\text{В}}{24\,\text{Ом}} = \frac1{24}\units{А} \approx 0{,}04\,\text{А},  \\
    \eli_2 &= \frac{\ele}{R + r} = \frac{1\,\text{В}}{24\,\text{Ом} + 20\,\text{Ом}} = \frac1{44}\units{А} \approx 0{,}02\,\text{А},  \\
    Q_1 &= \eli_1^2R\tau = \sqr{\frac{\ele}{R}} R \tau
            = \sqr{\frac{1\,\text{В}}{24\,\text{Ом}}} \cdot 24\,\text{Ом} \cdot 2\,\text{с} = \frac1{12}\units{Дж} \approx 0{,}083\,\text{Дж},  \\
    Q_2 &= \eli_2^2R\tau = \sqr{\frac{\ele}{R + r}} R \tau
            = \sqr{\frac{1\,\text{В}}{24\,\text{Ом} + 20\,\text{Ом}}} \cdot 24\,\text{Ом} \cdot 2\,\text{с} = \frac3{121}\units{Дж} \approx 0{,}025\,\text{Дж},  \\
    A_1 &= q_1\ele = \eli_1\tau\ele = \frac{\ele}{R} \tau \ele
            = \frac{\ele^2 \tau}{R} = \frac{\sqr{1\,\text{В}} \cdot 2\,\text{с}}{24\,\text{Ом}}
            = \frac1{12}\units{Дж} \approx 0{,}083\,\text{Дж}, \text{положительна},  \\
    A_2 &= q_2\ele = \eli_2\tau\ele = \frac{\ele}{R + r} \tau \ele
            = \frac{\ele^2 \tau}{R + r} = \frac{\sqr{1\,\text{В}} \cdot 2\,\text{с}}{24\,\text{Ом} + 20\,\text{Ом}}
            = \frac1{22}\units{Дж} \approx 0{,}045\,\text{Дж}, \text{положительна},  \\
    \eta_1 &= \frac{Q_1}{A_1} = \ldots = \frac{R}{R} = 1,  \\
    \eta_2 &= \frac{Q_2}{A_2} = \ldots = \frac{R}{R + r} = \frac6{11} \approx 0{,}55.
    \end{align*}
}
\solutionspace{180pt}

\tasknumber{6}%
\task{%
    Лампочки, сопротивления которых $R_1 = 5\,\text{Ом}$ и $R_2 = 80\,\text{Ом}$, поочерёдно подключённные к некоторому источнику тока,
    потребляют одинаковую мощность.
    Найти внутреннее сопротивление источника и КПД цепи в каждом случае.
}
\answer{%
    \begin{align*}
        P_1 &= \sqr{\frac{\ele}{R_1 + r}}R_1,
        P_2  = \sqr{\frac{\ele}{R_2 + r}}R_2,
        P_1 = P_2 \implies  \\
        &\implies R_1 \sqr{R_2 + r} = R_2 \sqr{R_1 + r} \implies  \\
        &\implies R_1 R_2^2 + 2 R_1 R_2 r + R_1 r^2 =
                    R_2 R_1^2 + 2 R_2 R_1 r + R_2 r^2  \implies  \\
    &\implies r^2 (R_2 - R_1) = R_2^2 R_2 - R_1^2 R_2 \implies  \\
    &\implies r
            = \sqrt{R_1 R_2 \frac{R_2 - R_1}{R_2 - R_1}}
            = \sqrt{R_1 R_2}
            = \sqrt{5\,\text{Ом} \cdot 80\,\text{Ом}}
            = 20\,\text{Ом}.
            \\
    \eta_1
            &= \frac{R_1}{R_1 + r}
            = \frac{\sqrt{R_1}}{\sqrt{R_1} + \sqrt{R_2}}
            = 0{,}200,  \\
    \eta_2
            &= \frac{R_2}{R_2 + r}
            = \frac{\sqrt{R_2}}{\sqrt{R_2} + \sqrt{R_1}}
            = 0{,}800
    \end{align*}
}

\variantsplitter

\addpersonalvariant{Владислав Емелин}

\tasknumber{1}%
\task{%
    Напротив физических величин укажите их обозначения и единицы измерения в СИ:
    \begin{enumerate}
        \item напряжение,
        \item работа тока,
        \item ЭДС,
        \item внутреннее сопротивление полной цепи.
    \end{enumerate}
}
\solutionspace{20pt}

\tasknumber{2}%
\task{%
    Запишите физический закон или формулу:
    \begin{enumerate}
        \item правило Кирхгофа для узла цепи,
        \item сопротивление резистора через удельное сопротивление,
        \item закон Ома для неоднородного участка цепи.
    \end{enumerate}
}
\solutionspace{40pt}

\tasknumber{3}%
\task{%
    На резистор сопротивлением $r = 12\,\text{Ом}$ подали напряжение $V = 240\,\text{В}$.
    Определите ток, который потечёт через резистор, и мощность, выделяющуюся на нём.
}
\answer{%
    \begin{align*}
    \eli &= \frac{V}{r} = \frac{240\,\text{В}}{12\,\text{Ом}} = 20\,\text{А},  \\
    P &= \frac{V^2}{r} = \frac{\sqr{240\,\text{В}}}{12\,\text{Ом}} = 4800\,\text{Вт}
    \end{align*}
}
\solutionspace{60pt}

\tasknumber{4}%
\task{%
    Через резистор сопротивлением $R = 18\,\text{Ом}$ протекает электрический ток $\eli = 8\,\text{А}$.
    Определите, чему равны напряжение на резисторе и мощность, выделяющаяся на нём.
}
\answer{%
    \begin{align*}
    U &= \eli R = 8\,\text{А} \cdot 18\,\text{Ом} = 144\,\text{В},  \\
    P &= \eli^2R = \sqr{8\,\text{А}} \cdot 18\,\text{Ом} = 1152\,\text{Вт}
    \end{align*}
}
\solutionspace{60pt}

\tasknumber{5}%
\task{%
    Замкнутая электрическая цепь состоит из ЭДС $\ele = 4\,\text{В}$ и сопротивлением $r$
    и резистора $R = 24\,\text{Ом}$.
    Определите ток, протекающий в цепи.
    Какая тепловая энергия выделится на резисторе за время
    $\tau = 2\,\text{с}$? Какая работа будет совершена ЭДС за это время? Каков знак этой работы? Чему равен КПД цепи?
    Вычислите значения для 2 случаев: $r=0$ и $r = 60\,\text{Ом}$.
}
\answer{%
    \begin{align*}
    \eli_1 &= \frac{\ele}{R} = \frac{4\,\text{В}}{24\,\text{Ом}} = \frac16\units{А} \approx 0{,}17\,\text{А},  \\
    \eli_2 &= \frac{\ele}{R + r} = \frac{4\,\text{В}}{24\,\text{Ом} + 60\,\text{Ом}} = \frac1{21}\units{А} \approx 0{,}05\,\text{А},  \\
    Q_1 &= \eli_1^2R\tau = \sqr{\frac{\ele}{R}} R \tau
            = \sqr{\frac{4\,\text{В}}{24\,\text{Ом}}} \cdot 24\,\text{Ом} \cdot 2\,\text{с} = \frac43\units{Дж} \approx 1{,}333\,\text{Дж},  \\
    Q_2 &= \eli_2^2R\tau = \sqr{\frac{\ele}{R + r}} R \tau
            = \sqr{\frac{4\,\text{В}}{24\,\text{Ом} + 60\,\text{Ом}}} \cdot 24\,\text{Ом} \cdot 2\,\text{с} = \frac{16}{147}\units{Дж} \approx 0{,}109\,\text{Дж},  \\
    A_1 &= q_1\ele = \eli_1\tau\ele = \frac{\ele}{R} \tau \ele
            = \frac{\ele^2 \tau}{R} = \frac{\sqr{4\,\text{В}} \cdot 2\,\text{с}}{24\,\text{Ом}}
            = \frac43\units{Дж} \approx 1{,}333\,\text{Дж}, \text{положительна},  \\
    A_2 &= q_2\ele = \eli_2\tau\ele = \frac{\ele}{R + r} \tau \ele
            = \frac{\ele^2 \tau}{R + r} = \frac{\sqr{4\,\text{В}} \cdot 2\,\text{с}}{24\,\text{Ом} + 60\,\text{Ом}}
            = \frac8{21}\units{Дж} \approx 0{,}381\,\text{Дж}, \text{положительна},  \\
    \eta_1 &= \frac{Q_1}{A_1} = \ldots = \frac{R}{R} = 1,  \\
    \eta_2 &= \frac{Q_2}{A_2} = \ldots = \frac{R}{R + r} = \frac27 \approx 0{,}29.
    \end{align*}
}
\solutionspace{180pt}

\tasknumber{6}%
\task{%
    Лампочки, сопротивления которых $R_1 = 0{,}50\,\text{Ом}$ и $R_2 = 2\,\text{Ом}$, поочерёдно подключённные к некоторому источнику тока,
    потребляют одинаковую мощность.
    Найти внутреннее сопротивление источника и КПД цепи в каждом случае.
}
\answer{%
    \begin{align*}
        P_1 &= \sqr{\frac{\ele}{R_1 + r}}R_1,
        P_2  = \sqr{\frac{\ele}{R_2 + r}}R_2,
        P_1 = P_2 \implies  \\
        &\implies R_1 \sqr{R_2 + r} = R_2 \sqr{R_1 + r} \implies  \\
        &\implies R_1 R_2^2 + 2 R_1 R_2 r + R_1 r^2 =
                    R_2 R_1^2 + 2 R_2 R_1 r + R_2 r^2  \implies  \\
    &\implies r^2 (R_2 - R_1) = R_2^2 R_2 - R_1^2 R_2 \implies  \\
    &\implies r
            = \sqrt{R_1 R_2 \frac{R_2 - R_1}{R_2 - R_1}}
            = \sqrt{R_1 R_2}
            = \sqrt{0{,}50\,\text{Ом} \cdot 2\,\text{Ом}}
            = 1\,\text{Ом}.
            \\
    \eta_1
            &= \frac{R_1}{R_1 + r}
            = \frac{\sqrt{R_1}}{\sqrt{R_1} + \sqrt{R_2}}
            = 0{,}333,  \\
    \eta_2
            &= \frac{R_2}{R_2 + r}
            = \frac{\sqrt{R_2}}{\sqrt{R_2} + \sqrt{R_1}}
            = 0{,}667
    \end{align*}
}

\variantsplitter

\addpersonalvariant{Артём Жичин}

\tasknumber{1}%
\task{%
    Напротив физических величин укажите их обозначения и единицы измерения в СИ:
    \begin{enumerate}
        \item сила тока,
        \item работа тока,
        \item ЭДС,
        \item внутреннее сопротивление полной цепи.
    \end{enumerate}
}
\solutionspace{20pt}

\tasknumber{2}%
\task{%
    Запишите физический закон или формулу:
    \begin{enumerate}
        \item правило Кирхгофа для узла цепи,
        \item сопротивление резистора через удельное сопротивление,
        \item закон Ома для неоднородного участка цепи.
    \end{enumerate}
}
\solutionspace{40pt}

\tasknumber{3}%
\task{%
    На резистор сопротивлением $r = 30\,\text{Ом}$ подали напряжение $U = 120\,\text{В}$.
    Определите ток, который потечёт через резистор, и мощность, выделяющуюся на нём.
}
\answer{%
    \begin{align*}
    \eli &= \frac{U}{r} = \frac{120\,\text{В}}{30\,\text{Ом}} = 4\,\text{А},  \\
    P &= \frac{U^2}{r} = \frac{\sqr{120\,\text{В}}}{30\,\text{Ом}} = 480\,\text{Вт}
    \end{align*}
}
\solutionspace{60pt}

\tasknumber{4}%
\task{%
    Через резистор сопротивлением $r = 18\,\text{Ом}$ протекает электрический ток $\eli = 15\,\text{А}$.
    Определите, чему равны напряжение на резисторе и мощность, выделяющаяся на нём.
}
\answer{%
    \begin{align*}
    U &= \eli r = 15\,\text{А} \cdot 18\,\text{Ом} = 270\,\text{В},  \\
    P &= \eli^2r = \sqr{15\,\text{А}} \cdot 18\,\text{Ом} = 4050\,\text{Вт}
    \end{align*}
}
\solutionspace{60pt}

\tasknumber{5}%
\task{%
    Замкнутая электрическая цепь состоит из ЭДС $\ele = 2\,\text{В}$ и сопротивлением $r$
    и резистора $R = 10\,\text{Ом}$.
    Определите ток, протекающий в цепи.
    Какая тепловая энергия выделится на резисторе за время
    $\tau = 5\,\text{с}$? Какая работа будет совершена ЭДС за это время? Каков знак этой работы? Чему равен КПД цепи?
    Вычислите значения для 2 случаев: $r=0$ и $r = 60\,\text{Ом}$.
}
\answer{%
    \begin{align*}
    \eli_1 &= \frac{\ele}{R} = \frac{2\,\text{В}}{10\,\text{Ом}} = \frac15\units{А} \approx 0{,}20\,\text{А},  \\
    \eli_2 &= \frac{\ele}{R + r} = \frac{2\,\text{В}}{10\,\text{Ом} + 60\,\text{Ом}} = \frac1{35}\units{А} \approx 0{,}03\,\text{А},  \\
    Q_1 &= \eli_1^2R\tau = \sqr{\frac{\ele}{R}} R \tau
            = \sqr{\frac{2\,\text{В}}{10\,\text{Ом}}} \cdot 10\,\text{Ом} \cdot 5\,\text{с} = 2\units{Дж} \approx 2\,\text{Дж},  \\
    Q_2 &= \eli_2^2R\tau = \sqr{\frac{\ele}{R + r}} R \tau
            = \sqr{\frac{2\,\text{В}}{10\,\text{Ом} + 60\,\text{Ом}}} \cdot 10\,\text{Ом} \cdot 5\,\text{с} = \frac2{49}\units{Дж} \approx 0{,}041\,\text{Дж},  \\
    A_1 &= q_1\ele = \eli_1\tau\ele = \frac{\ele}{R} \tau \ele
            = \frac{\ele^2 \tau}{R} = \frac{\sqr{2\,\text{В}} \cdot 5\,\text{с}}{10\,\text{Ом}}
            = 2\units{Дж} \approx 2\,\text{Дж}, \text{положительна},  \\
    A_2 &= q_2\ele = \eli_2\tau\ele = \frac{\ele}{R + r} \tau \ele
            = \frac{\ele^2 \tau}{R + r} = \frac{\sqr{2\,\text{В}} \cdot 5\,\text{с}}{10\,\text{Ом} + 60\,\text{Ом}}
            = \frac27\units{Дж} \approx 0{,}286\,\text{Дж}, \text{положительна},  \\
    \eta_1 &= \frac{Q_1}{A_1} = \ldots = \frac{R}{R} = 1,  \\
    \eta_2 &= \frac{Q_2}{A_2} = \ldots = \frac{R}{R + r} = \frac17 \approx 0{,}14.
    \end{align*}
}
\solutionspace{180pt}

\tasknumber{6}%
\task{%
    Лампочки, сопротивления которых $R_1 = 1\,\text{Ом}$ и $R_2 = 49\,\text{Ом}$, поочерёдно подключённные к некоторому источнику тока,
    потребляют одинаковую мощность.
    Найти внутреннее сопротивление источника и КПД цепи в каждом случае.
}
\answer{%
    \begin{align*}
        P_1 &= \sqr{\frac{\ele}{R_1 + r}}R_1,
        P_2  = \sqr{\frac{\ele}{R_2 + r}}R_2,
        P_1 = P_2 \implies  \\
        &\implies R_1 \sqr{R_2 + r} = R_2 \sqr{R_1 + r} \implies  \\
        &\implies R_1 R_2^2 + 2 R_1 R_2 r + R_1 r^2 =
                    R_2 R_1^2 + 2 R_2 R_1 r + R_2 r^2  \implies  \\
    &\implies r^2 (R_2 - R_1) = R_2^2 R_2 - R_1^2 R_2 \implies  \\
    &\implies r
            = \sqrt{R_1 R_2 \frac{R_2 - R_1}{R_2 - R_1}}
            = \sqrt{R_1 R_2}
            = \sqrt{1\,\text{Ом} \cdot 49\,\text{Ом}}
            = 7\,\text{Ом}.
            \\
    \eta_1
            &= \frac{R_1}{R_1 + r}
            = \frac{\sqrt{R_1}}{\sqrt{R_1} + \sqrt{R_2}}
            = 0{,}125,  \\
    \eta_2
            &= \frac{R_2}{R_2 + r}
            = \frac{\sqrt{R_2}}{\sqrt{R_2} + \sqrt{R_1}}
            = 0{,}875
    \end{align*}
}

\variantsplitter

\addpersonalvariant{Дарья Кошман}

\tasknumber{1}%
\task{%
    Напротив физических величин укажите их обозначения и единицы измерения в СИ:
    \begin{enumerate}
        \item напряжение,
        \item работа тока,
        \item ЭДС,
        \item внутреннее сопротивление полной цепи.
    \end{enumerate}
}
\solutionspace{20pt}

\tasknumber{2}%
\task{%
    Запишите физический закон или формулу:
    \begin{enumerate}
        \item правило Кирхгофа для замкнутого контура,
        \item закон Ома для однородного участка цепи,
        \item закон Ома для неоднородного участка цепи.
    \end{enumerate}
}
\solutionspace{40pt}

\tasknumber{3}%
\task{%
    На резистор сопротивлением $R = 18\,\text{Ом}$ подали напряжение $V = 240\,\text{В}$.
    Определите ток, который потечёт через резистор, и мощность, выделяющуюся на нём.
}
\answer{%
    \begin{align*}
    \eli &= \frac{V}{R} = \frac{240\,\text{В}}{18\,\text{Ом}} = 13{,}33\,\text{А},  \\
    P &= \frac{V^2}{R} = \frac{\sqr{240\,\text{В}}}{18\,\text{Ом}} = 3200\,\text{Вт}
    \end{align*}
}
\solutionspace{60pt}

\tasknumber{4}%
\task{%
    Через резистор сопротивлением $r = 18\,\text{Ом}$ протекает электрический ток $\eli = 6\,\text{А}$.
    Определите, чему равны напряжение на резисторе и мощность, выделяющаяся на нём.
}
\answer{%
    \begin{align*}
    U &= \eli r = 6\,\text{А} \cdot 18\,\text{Ом} = 108\,\text{В},  \\
    P &= \eli^2r = \sqr{6\,\text{А}} \cdot 18\,\text{Ом} = 648\,\text{Вт}
    \end{align*}
}
\solutionspace{60pt}

\tasknumber{5}%
\task{%
    Замкнутая электрическая цепь состоит из ЭДС $\ele = 3\,\text{В}$ и сопротивлением $r$
    и резистора $R = 24\,\text{Ом}$.
    Определите ток, протекающий в цепи.
    Какая тепловая энергия выделится на резисторе за время
    $\tau = 10\,\text{с}$? Какая работа будет совершена ЭДС за это время? Каков знак этой работы? Чему равен КПД цепи?
    Вычислите значения для 2 случаев: $r=0$ и $r = 10\,\text{Ом}$.
}
\answer{%
    \begin{align*}
    \eli_1 &= \frac{\ele}{R} = \frac{3\,\text{В}}{24\,\text{Ом}} = \frac18\units{А} \approx 0{,}12\,\text{А},  \\
    \eli_2 &= \frac{\ele}{R + r} = \frac{3\,\text{В}}{24\,\text{Ом} + 10\,\text{Ом}} = \frac3{34}\units{А} \approx 0{,}09\,\text{А},  \\
    Q_1 &= \eli_1^2R\tau = \sqr{\frac{\ele}{R}} R \tau
            = \sqr{\frac{3\,\text{В}}{24\,\text{Ом}}} \cdot 24\,\text{Ом} \cdot 10\,\text{с} = \frac{15}4\units{Дж} \approx 3{,}750\,\text{Дж},  \\
    Q_2 &= \eli_2^2R\tau = \sqr{\frac{\ele}{R + r}} R \tau
            = \sqr{\frac{3\,\text{В}}{24\,\text{Ом} + 10\,\text{Ом}}} \cdot 24\,\text{Ом} \cdot 10\,\text{с} = \frac{540}{289}\units{Дж} \approx 1{,}869\,\text{Дж},  \\
    A_1 &= q_1\ele = \eli_1\tau\ele = \frac{\ele}{R} \tau \ele
            = \frac{\ele^2 \tau}{R} = \frac{\sqr{3\,\text{В}} \cdot 10\,\text{с}}{24\,\text{Ом}}
            = \frac{15}4\units{Дж} \approx 3{,}750\,\text{Дж}, \text{положительна},  \\
    A_2 &= q_2\ele = \eli_2\tau\ele = \frac{\ele}{R + r} \tau \ele
            = \frac{\ele^2 \tau}{R + r} = \frac{\sqr{3\,\text{В}} \cdot 10\,\text{с}}{24\,\text{Ом} + 10\,\text{Ом}}
            = \frac{45}{17}\units{Дж} \approx 2{,}647\,\text{Дж}, \text{положительна},  \\
    \eta_1 &= \frac{Q_1}{A_1} = \ldots = \frac{R}{R} = 1,  \\
    \eta_2 &= \frac{Q_2}{A_2} = \ldots = \frac{R}{R + r} = \frac{12}{17} \approx 0{,}71.
    \end{align*}
}
\solutionspace{180pt}

\tasknumber{6}%
\task{%
    Лампочки, сопротивления которых $R_1 = 0{,}25\,\text{Ом}$ и $R_2 = 16\,\text{Ом}$, поочерёдно подключённные к некоторому источнику тока,
    потребляют одинаковую мощность.
    Найти внутреннее сопротивление источника и КПД цепи в каждом случае.
}
\answer{%
    \begin{align*}
        P_1 &= \sqr{\frac{\ele}{R_1 + r}}R_1,
        P_2  = \sqr{\frac{\ele}{R_2 + r}}R_2,
        P_1 = P_2 \implies  \\
        &\implies R_1 \sqr{R_2 + r} = R_2 \sqr{R_1 + r} \implies  \\
        &\implies R_1 R_2^2 + 2 R_1 R_2 r + R_1 r^2 =
                    R_2 R_1^2 + 2 R_2 R_1 r + R_2 r^2  \implies  \\
    &\implies r^2 (R_2 - R_1) = R_2^2 R_2 - R_1^2 R_2 \implies  \\
    &\implies r
            = \sqrt{R_1 R_2 \frac{R_2 - R_1}{R_2 - R_1}}
            = \sqrt{R_1 R_2}
            = \sqrt{0{,}25\,\text{Ом} \cdot 16\,\text{Ом}}
            = 2\,\text{Ом}.
            \\
    \eta_1
            &= \frac{R_1}{R_1 + r}
            = \frac{\sqrt{R_1}}{\sqrt{R_1} + \sqrt{R_2}}
            = 0{,}111,  \\
    \eta_2
            &= \frac{R_2}{R_2 + r}
            = \frac{\sqrt{R_2}}{\sqrt{R_2} + \sqrt{R_1}}
            = 0{,}889
    \end{align*}
}

\variantsplitter

\addpersonalvariant{Анна Кузьмичёва}

\tasknumber{1}%
\task{%
    Напротив физических величин укажите их обозначения и единицы измерения в СИ:
    \begin{enumerate}
        \item разность потенциалов,
        \item работа тока,
        \item удельное сопротивление,
        \item внутреннее сопротивление полной цепи.
    \end{enumerate}
}
\solutionspace{20pt}

\tasknumber{2}%
\task{%
    Запишите физический закон или формулу:
    \begin{enumerate}
        \item правило Кирхгофа для замкнутого контура,
        \item сопротивление резистора через удельное сопротивление,
        \item ЭДС (определение).
    \end{enumerate}
}
\solutionspace{40pt}

\tasknumber{3}%
\task{%
    На резистор сопротивлением $R = 12\,\text{Ом}$ подали напряжение $U = 150\,\text{В}$.
    Определите ток, который потечёт через резистор, и мощность, выделяющуюся на нём.
}
\answer{%
    \begin{align*}
    \eli &= \frac{U}{R} = \frac{150\,\text{В}}{12\,\text{Ом}} = 12{,}50\,\text{А},  \\
    P &= \frac{U^2}{R} = \frac{\sqr{150\,\text{В}}}{12\,\text{Ом}} = 1875\,\text{Вт}
    \end{align*}
}
\solutionspace{60pt}

\tasknumber{4}%
\task{%
    Через резистор сопротивлением $R = 18\,\text{Ом}$ протекает электрический ток $\eli = 5\,\text{А}$.
    Определите, чему равны напряжение на резисторе и мощность, выделяющаяся на нём.
}
\answer{%
    \begin{align*}
    U &= \eli R = 5\,\text{А} \cdot 18\,\text{Ом} = 90\,\text{В},  \\
    P &= \eli^2R = \sqr{5\,\text{А}} \cdot 18\,\text{Ом} = 450\,\text{Вт}
    \end{align*}
}
\solutionspace{60pt}

\tasknumber{5}%
\task{%
    Замкнутая электрическая цепь состоит из ЭДС $\ele = 1\,\text{В}$ и сопротивлением $r$
    и резистора $R = 10\,\text{Ом}$.
    Определите ток, протекающий в цепи.
    Какая тепловая энергия выделится на резисторе за время
    $\tau = 10\,\text{с}$? Какая работа будет совершена ЭДС за это время? Каков знак этой работы? Чему равен КПД цепи?
    Вычислите значения для 2 случаев: $r=0$ и $r = 60\,\text{Ом}$.
}
\answer{%
    \begin{align*}
    \eli_1 &= \frac{\ele}{R} = \frac{1\,\text{В}}{10\,\text{Ом}} = \frac1{10}\units{А} \approx 0{,}10\,\text{А},  \\
    \eli_2 &= \frac{\ele}{R + r} = \frac{1\,\text{В}}{10\,\text{Ом} + 60\,\text{Ом}} = \frac1{70}\units{А} \approx 0{,}010\,\text{А},  \\
    Q_1 &= \eli_1^2R\tau = \sqr{\frac{\ele}{R}} R \tau
            = \sqr{\frac{1\,\text{В}}{10\,\text{Ом}}} \cdot 10\,\text{Ом} \cdot 10\,\text{с} = 1\units{Дж} \approx 1\,\text{Дж},  \\
    Q_2 &= \eli_2^2R\tau = \sqr{\frac{\ele}{R + r}} R \tau
            = \sqr{\frac{1\,\text{В}}{10\,\text{Ом} + 60\,\text{Ом}}} \cdot 10\,\text{Ом} \cdot 10\,\text{с} = \frac1{49}\units{Дж} \approx 0{,}020\,\text{Дж},  \\
    A_1 &= q_1\ele = \eli_1\tau\ele = \frac{\ele}{R} \tau \ele
            = \frac{\ele^2 \tau}{R} = \frac{\sqr{1\,\text{В}} \cdot 10\,\text{с}}{10\,\text{Ом}}
            = 1\units{Дж} \approx 1\,\text{Дж}, \text{положительна},  \\
    A_2 &= q_2\ele = \eli_2\tau\ele = \frac{\ele}{R + r} \tau \ele
            = \frac{\ele^2 \tau}{R + r} = \frac{\sqr{1\,\text{В}} \cdot 10\,\text{с}}{10\,\text{Ом} + 60\,\text{Ом}}
            = \frac17\units{Дж} \approx 0{,}143\,\text{Дж}, \text{положительна},  \\
    \eta_1 &= \frac{Q_1}{A_1} = \ldots = \frac{R}{R} = 1,  \\
    \eta_2 &= \frac{Q_2}{A_2} = \ldots = \frac{R}{R + r} = \frac17 \approx 0{,}14.
    \end{align*}
}
\solutionspace{180pt}

\tasknumber{6}%
\task{%
    Лампочки, сопротивления которых $R_1 = 6\,\text{Ом}$ и $R_2 = 54\,\text{Ом}$, поочерёдно подключённные к некоторому источнику тока,
    потребляют одинаковую мощность.
    Найти внутреннее сопротивление источника и КПД цепи в каждом случае.
}
\answer{%
    \begin{align*}
        P_1 &= \sqr{\frac{\ele}{R_1 + r}}R_1,
        P_2  = \sqr{\frac{\ele}{R_2 + r}}R_2,
        P_1 = P_2 \implies  \\
        &\implies R_1 \sqr{R_2 + r} = R_2 \sqr{R_1 + r} \implies  \\
        &\implies R_1 R_2^2 + 2 R_1 R_2 r + R_1 r^2 =
                    R_2 R_1^2 + 2 R_2 R_1 r + R_2 r^2  \implies  \\
    &\implies r^2 (R_2 - R_1) = R_2^2 R_2 - R_1^2 R_2 \implies  \\
    &\implies r
            = \sqrt{R_1 R_2 \frac{R_2 - R_1}{R_2 - R_1}}
            = \sqrt{R_1 R_2}
            = \sqrt{6\,\text{Ом} \cdot 54\,\text{Ом}}
            = 18\,\text{Ом}.
            \\
    \eta_1
            &= \frac{R_1}{R_1 + r}
            = \frac{\sqrt{R_1}}{\sqrt{R_1} + \sqrt{R_2}}
            = 0{,}250,  \\
    \eta_2
            &= \frac{R_2}{R_2 + r}
            = \frac{\sqrt{R_2}}{\sqrt{R_2} + \sqrt{R_1}}
            = 0{,}750
    \end{align*}
}

\variantsplitter

\addpersonalvariant{Алёна Куприянова}

\tasknumber{1}%
\task{%
    Напротив физических величин укажите их обозначения и единицы измерения в СИ:
    \begin{enumerate}
        \item разность потенциалов,
        \item мощность тока,
        \item удельное сопротивление,
        \item внешнее сопротивление полной цепи.
    \end{enumerate}
}
\solutionspace{20pt}

\tasknumber{2}%
\task{%
    Запишите физический закон или формулу:
    \begin{enumerate}
        \item правило Кирхгофа для узла цепи,
        \item закон Ома для однородного участка цепи,
        \item закон Ома для неоднородного участка цепи.
    \end{enumerate}
}
\solutionspace{40pt}

\tasknumber{3}%
\task{%
    На резистор сопротивлением $R = 18\,\text{Ом}$ подали напряжение $V = 240\,\text{В}$.
    Определите ток, который потечёт через резистор, и мощность, выделяющуюся на нём.
}
\answer{%
    \begin{align*}
    \eli &= \frac{V}{R} = \frac{240\,\text{В}}{18\,\text{Ом}} = 13{,}33\,\text{А},  \\
    P &= \frac{V^2}{R} = \frac{\sqr{240\,\text{В}}}{18\,\text{Ом}} = 3200\,\text{Вт}
    \end{align*}
}
\solutionspace{60pt}

\tasknumber{4}%
\task{%
    Через резистор сопротивлением $R = 18\,\text{Ом}$ протекает электрический ток $\eli = 2\,\text{А}$.
    Определите, чему равны напряжение на резисторе и мощность, выделяющаяся на нём.
}
\answer{%
    \begin{align*}
    U &= \eli R = 2\,\text{А} \cdot 18\,\text{Ом} = 36\,\text{В},  \\
    P &= \eli^2R = \sqr{2\,\text{А}} \cdot 18\,\text{Ом} = 72\,\text{Вт}
    \end{align*}
}
\solutionspace{60pt}

\tasknumber{5}%
\task{%
    Замкнутая электрическая цепь состоит из ЭДС $\ele = 4\,\text{В}$ и сопротивлением $r$
    и резистора $R = 15\,\text{Ом}$.
    Определите ток, протекающий в цепи.
    Какая тепловая энергия выделится на резисторе за время
    $\tau = 2\,\text{с}$? Какая работа будет совершена ЭДС за это время? Каков знак этой работы? Чему равен КПД цепи?
    Вычислите значения для 2 случаев: $r=0$ и $r = 60\,\text{Ом}$.
}
\answer{%
    \begin{align*}
    \eli_1 &= \frac{\ele}{R} = \frac{4\,\text{В}}{15\,\text{Ом}} = \frac4{15}\units{А} \approx 0{,}27\,\text{А},  \\
    \eli_2 &= \frac{\ele}{R + r} = \frac{4\,\text{В}}{15\,\text{Ом} + 60\,\text{Ом}} = \frac4{75}\units{А} \approx 0{,}05\,\text{А},  \\
    Q_1 &= \eli_1^2R\tau = \sqr{\frac{\ele}{R}} R \tau
            = \sqr{\frac{4\,\text{В}}{15\,\text{Ом}}} \cdot 15\,\text{Ом} \cdot 2\,\text{с} = \frac{32}{15}\units{Дж} \approx 2{,}133\,\text{Дж},  \\
    Q_2 &= \eli_2^2R\tau = \sqr{\frac{\ele}{R + r}} R \tau
            = \sqr{\frac{4\,\text{В}}{15\,\text{Ом} + 60\,\text{Ом}}} \cdot 15\,\text{Ом} \cdot 2\,\text{с} = \frac{32}{375}\units{Дж} \approx 0{,}085\,\text{Дж},  \\
    A_1 &= q_1\ele = \eli_1\tau\ele = \frac{\ele}{R} \tau \ele
            = \frac{\ele^2 \tau}{R} = \frac{\sqr{4\,\text{В}} \cdot 2\,\text{с}}{15\,\text{Ом}}
            = \frac{32}{15}\units{Дж} \approx 2{,}133\,\text{Дж}, \text{положительна},  \\
    A_2 &= q_2\ele = \eli_2\tau\ele = \frac{\ele}{R + r} \tau \ele
            = \frac{\ele^2 \tau}{R + r} = \frac{\sqr{4\,\text{В}} \cdot 2\,\text{с}}{15\,\text{Ом} + 60\,\text{Ом}}
            = \frac{32}{75}\units{Дж} \approx 0{,}427\,\text{Дж}, \text{положительна},  \\
    \eta_1 &= \frac{Q_1}{A_1} = \ldots = \frac{R}{R} = 1,  \\
    \eta_2 &= \frac{Q_2}{A_2} = \ldots = \frac{R}{R + r} = \frac15 \approx 0{,}20.
    \end{align*}
}
\solutionspace{180pt}

\tasknumber{6}%
\task{%
    Лампочки, сопротивления которых $R_1 = 0{,}25\,\text{Ом}$ и $R_2 = 64\,\text{Ом}$, поочерёдно подключённные к некоторому источнику тока,
    потребляют одинаковую мощность.
    Найти внутреннее сопротивление источника и КПД цепи в каждом случае.
}
\answer{%
    \begin{align*}
        P_1 &= \sqr{\frac{\ele}{R_1 + r}}R_1,
        P_2  = \sqr{\frac{\ele}{R_2 + r}}R_2,
        P_1 = P_2 \implies  \\
        &\implies R_1 \sqr{R_2 + r} = R_2 \sqr{R_1 + r} \implies  \\
        &\implies R_1 R_2^2 + 2 R_1 R_2 r + R_1 r^2 =
                    R_2 R_1^2 + 2 R_2 R_1 r + R_2 r^2  \implies  \\
    &\implies r^2 (R_2 - R_1) = R_2^2 R_2 - R_1^2 R_2 \implies  \\
    &\implies r
            = \sqrt{R_1 R_2 \frac{R_2 - R_1}{R_2 - R_1}}
            = \sqrt{R_1 R_2}
            = \sqrt{0{,}25\,\text{Ом} \cdot 64\,\text{Ом}}
            = 4\,\text{Ом}.
            \\
    \eta_1
            &= \frac{R_1}{R_1 + r}
            = \frac{\sqrt{R_1}}{\sqrt{R_1} + \sqrt{R_2}}
            = 0{,}059,  \\
    \eta_2
            &= \frac{R_2}{R_2 + r}
            = \frac{\sqrt{R_2}}{\sqrt{R_2} + \sqrt{R_1}}
            = 0{,}941
    \end{align*}
}

\variantsplitter

\addpersonalvariant{Ярослав Лавровский}

\tasknumber{1}%
\task{%
    Напротив физических величин укажите их обозначения и единицы измерения в СИ:
    \begin{enumerate}
        \item напряжение,
        \item мощность тока,
        \item удельное сопротивление,
        \item внешнее сопротивление полной цепи.
    \end{enumerate}
}
\solutionspace{20pt}

\tasknumber{2}%
\task{%
    Запишите физический закон или формулу:
    \begin{enumerate}
        \item правило Кирхгофа для замкнутого контура,
        \item сопротивление резистора через удельное сопротивление,
        \item ЭДС (определение).
    \end{enumerate}
}
\solutionspace{40pt}

\tasknumber{3}%
\task{%
    На резистор сопротивлением $r = 18\,\text{Ом}$ подали напряжение $U = 240\,\text{В}$.
    Определите ток, который потечёт через резистор, и мощность, выделяющуюся на нём.
}
\answer{%
    \begin{align*}
    \eli &= \frac{U}{r} = \frac{240\,\text{В}}{18\,\text{Ом}} = 13{,}33\,\text{А},  \\
    P &= \frac{U^2}{r} = \frac{\sqr{240\,\text{В}}}{18\,\text{Ом}} = 3200\,\text{Вт}
    \end{align*}
}
\solutionspace{60pt}

\tasknumber{4}%
\task{%
    Через резистор сопротивлением $r = 18\,\text{Ом}$ протекает электрический ток $\eli = 2\,\text{А}$.
    Определите, чему равны напряжение на резисторе и мощность, выделяющаяся на нём.
}
\answer{%
    \begin{align*}
    U &= \eli r = 2\,\text{А} \cdot 18\,\text{Ом} = 36\,\text{В},  \\
    P &= \eli^2r = \sqr{2\,\text{А}} \cdot 18\,\text{Ом} = 72\,\text{Вт}
    \end{align*}
}
\solutionspace{60pt}

\tasknumber{5}%
\task{%
    Замкнутая электрическая цепь состоит из ЭДС $\ele = 1\,\text{В}$ и сопротивлением $r$
    и резистора $R = 24\,\text{Ом}$.
    Определите ток, протекающий в цепи.
    Какая тепловая энергия выделится на резисторе за время
    $\tau = 2\,\text{с}$? Какая работа будет совершена ЭДС за это время? Каков знак этой работы? Чему равен КПД цепи?
    Вычислите значения для 2 случаев: $r=0$ и $r = 30\,\text{Ом}$.
}
\answer{%
    \begin{align*}
    \eli_1 &= \frac{\ele}{R} = \frac{1\,\text{В}}{24\,\text{Ом}} = \frac1{24}\units{А} \approx 0{,}04\,\text{А},  \\
    \eli_2 &= \frac{\ele}{R + r} = \frac{1\,\text{В}}{24\,\text{Ом} + 30\,\text{Ом}} = \frac1{54}\units{А} \approx 0{,}02\,\text{А},  \\
    Q_1 &= \eli_1^2R\tau = \sqr{\frac{\ele}{R}} R \tau
            = \sqr{\frac{1\,\text{В}}{24\,\text{Ом}}} \cdot 24\,\text{Ом} \cdot 2\,\text{с} = \frac1{12}\units{Дж} \approx 0{,}083\,\text{Дж},  \\
    Q_2 &= \eli_2^2R\tau = \sqr{\frac{\ele}{R + r}} R \tau
            = \sqr{\frac{1\,\text{В}}{24\,\text{Ом} + 30\,\text{Ом}}} \cdot 24\,\text{Ом} \cdot 2\,\text{с} = \frac4{243}\units{Дж} \approx 0{,}016\,\text{Дж},  \\
    A_1 &= q_1\ele = \eli_1\tau\ele = \frac{\ele}{R} \tau \ele
            = \frac{\ele^2 \tau}{R} = \frac{\sqr{1\,\text{В}} \cdot 2\,\text{с}}{24\,\text{Ом}}
            = \frac1{12}\units{Дж} \approx 0{,}083\,\text{Дж}, \text{положительна},  \\
    A_2 &= q_2\ele = \eli_2\tau\ele = \frac{\ele}{R + r} \tau \ele
            = \frac{\ele^2 \tau}{R + r} = \frac{\sqr{1\,\text{В}} \cdot 2\,\text{с}}{24\,\text{Ом} + 30\,\text{Ом}}
            = \frac1{27}\units{Дж} \approx 0{,}037\,\text{Дж}, \text{положительна},  \\
    \eta_1 &= \frac{Q_1}{A_1} = \ldots = \frac{R}{R} = 1,  \\
    \eta_2 &= \frac{Q_2}{A_2} = \ldots = \frac{R}{R + r} = \frac49 \approx 0{,}44.
    \end{align*}
}
\solutionspace{180pt}

\tasknumber{6}%
\task{%
    Лампочки, сопротивления которых $R_1 = 5\,\text{Ом}$ и $R_2 = 80\,\text{Ом}$, поочерёдно подключённные к некоторому источнику тока,
    потребляют одинаковую мощность.
    Найти внутреннее сопротивление источника и КПД цепи в каждом случае.
}
\answer{%
    \begin{align*}
        P_1 &= \sqr{\frac{\ele}{R_1 + r}}R_1,
        P_2  = \sqr{\frac{\ele}{R_2 + r}}R_2,
        P_1 = P_2 \implies  \\
        &\implies R_1 \sqr{R_2 + r} = R_2 \sqr{R_1 + r} \implies  \\
        &\implies R_1 R_2^2 + 2 R_1 R_2 r + R_1 r^2 =
                    R_2 R_1^2 + 2 R_2 R_1 r + R_2 r^2  \implies  \\
    &\implies r^2 (R_2 - R_1) = R_2^2 R_2 - R_1^2 R_2 \implies  \\
    &\implies r
            = \sqrt{R_1 R_2 \frac{R_2 - R_1}{R_2 - R_1}}
            = \sqrt{R_1 R_2}
            = \sqrt{5\,\text{Ом} \cdot 80\,\text{Ом}}
            = 20\,\text{Ом}.
            \\
    \eta_1
            &= \frac{R_1}{R_1 + r}
            = \frac{\sqrt{R_1}}{\sqrt{R_1} + \sqrt{R_2}}
            = 0{,}200,  \\
    \eta_2
            &= \frac{R_2}{R_2 + r}
            = \frac{\sqrt{R_2}}{\sqrt{R_2} + \sqrt{R_1}}
            = 0{,}800
    \end{align*}
}

\variantsplitter

\addpersonalvariant{Анастасия Ламанова}

\tasknumber{1}%
\task{%
    Напротив физических величин укажите их обозначения и единицы измерения в СИ:
    \begin{enumerate}
        \item разность потенциалов,
        \item мощность тока,
        \item ЭДС,
        \item внутреннее сопротивление полной цепи.
    \end{enumerate}
}
\solutionspace{20pt}

\tasknumber{2}%
\task{%
    Запишите физический закон или формулу:
    \begin{enumerate}
        \item правило Кирхгофа для замкнутого контура,
        \item сопротивление резистора через удельное сопротивление,
        \item ЭДС (определение).
    \end{enumerate}
}
\solutionspace{40pt}

\tasknumber{3}%
\task{%
    На резистор сопротивлением $r = 18\,\text{Ом}$ подали напряжение $V = 150\,\text{В}$.
    Определите ток, который потечёт через резистор, и мощность, выделяющуюся на нём.
}
\answer{%
    \begin{align*}
    \eli &= \frac{V}{r} = \frac{150\,\text{В}}{18\,\text{Ом}} = 8{,}33\,\text{А},  \\
    P &= \frac{V^2}{r} = \frac{\sqr{150\,\text{В}}}{18\,\text{Ом}} = 1250\,\text{Вт}
    \end{align*}
}
\solutionspace{60pt}

\tasknumber{4}%
\task{%
    Через резистор сопротивлением $r = 12\,\text{Ом}$ протекает электрический ток $\eli = 3\,\text{А}$.
    Определите, чему равны напряжение на резисторе и мощность, выделяющаяся на нём.
}
\answer{%
    \begin{align*}
    U &= \eli r = 3\,\text{А} \cdot 12\,\text{Ом} = 36\,\text{В},  \\
    P &= \eli^2r = \sqr{3\,\text{А}} \cdot 12\,\text{Ом} = 108\,\text{Вт}
    \end{align*}
}
\solutionspace{60pt}

\tasknumber{5}%
\task{%
    Замкнутая электрическая цепь состоит из ЭДС $\ele = 3\,\text{В}$ и сопротивлением $r$
    и резистора $R = 15\,\text{Ом}$.
    Определите ток, протекающий в цепи.
    Какая тепловая энергия выделится на резисторе за время
    $\tau = 10\,\text{с}$? Какая работа будет совершена ЭДС за это время? Каков знак этой работы? Чему равен КПД цепи?
    Вычислите значения для 2 случаев: $r=0$ и $r = 60\,\text{Ом}$.
}
\answer{%
    \begin{align*}
    \eli_1 &= \frac{\ele}{R} = \frac{3\,\text{В}}{15\,\text{Ом}} = \frac15\units{А} \approx 0{,}20\,\text{А},  \\
    \eli_2 &= \frac{\ele}{R + r} = \frac{3\,\text{В}}{15\,\text{Ом} + 60\,\text{Ом}} = \frac1{25}\units{А} \approx 0{,}04\,\text{А},  \\
    Q_1 &= \eli_1^2R\tau = \sqr{\frac{\ele}{R}} R \tau
            = \sqr{\frac{3\,\text{В}}{15\,\text{Ом}}} \cdot 15\,\text{Ом} \cdot 10\,\text{с} = 6\units{Дж} \approx 6\,\text{Дж},  \\
    Q_2 &= \eli_2^2R\tau = \sqr{\frac{\ele}{R + r}} R \tau
            = \sqr{\frac{3\,\text{В}}{15\,\text{Ом} + 60\,\text{Ом}}} \cdot 15\,\text{Ом} \cdot 10\,\text{с} = \frac6{25}\units{Дж} \approx 0{,}240\,\text{Дж},  \\
    A_1 &= q_1\ele = \eli_1\tau\ele = \frac{\ele}{R} \tau \ele
            = \frac{\ele^2 \tau}{R} = \frac{\sqr{3\,\text{В}} \cdot 10\,\text{с}}{15\,\text{Ом}}
            = 6\units{Дж} \approx 6\,\text{Дж}, \text{положительна},  \\
    A_2 &= q_2\ele = \eli_2\tau\ele = \frac{\ele}{R + r} \tau \ele
            = \frac{\ele^2 \tau}{R + r} = \frac{\sqr{3\,\text{В}} \cdot 10\,\text{с}}{15\,\text{Ом} + 60\,\text{Ом}}
            = \frac65\units{Дж} \approx 1{,}200\,\text{Дж}, \text{положительна},  \\
    \eta_1 &= \frac{Q_1}{A_1} = \ldots = \frac{R}{R} = 1,  \\
    \eta_2 &= \frac{Q_2}{A_2} = \ldots = \frac{R}{R + r} = \frac15 \approx 0{,}20.
    \end{align*}
}
\solutionspace{180pt}

\tasknumber{6}%
\task{%
    Лампочки, сопротивления которых $R_1 = 5\,\text{Ом}$ и $R_2 = 80\,\text{Ом}$, поочерёдно подключённные к некоторому источнику тока,
    потребляют одинаковую мощность.
    Найти внутреннее сопротивление источника и КПД цепи в каждом случае.
}
\answer{%
    \begin{align*}
        P_1 &= \sqr{\frac{\ele}{R_1 + r}}R_1,
        P_2  = \sqr{\frac{\ele}{R_2 + r}}R_2,
        P_1 = P_2 \implies  \\
        &\implies R_1 \sqr{R_2 + r} = R_2 \sqr{R_1 + r} \implies  \\
        &\implies R_1 R_2^2 + 2 R_1 R_2 r + R_1 r^2 =
                    R_2 R_1^2 + 2 R_2 R_1 r + R_2 r^2  \implies  \\
    &\implies r^2 (R_2 - R_1) = R_2^2 R_2 - R_1^2 R_2 \implies  \\
    &\implies r
            = \sqrt{R_1 R_2 \frac{R_2 - R_1}{R_2 - R_1}}
            = \sqrt{R_1 R_2}
            = \sqrt{5\,\text{Ом} \cdot 80\,\text{Ом}}
            = 20\,\text{Ом}.
            \\
    \eta_1
            &= \frac{R_1}{R_1 + r}
            = \frac{\sqrt{R_1}}{\sqrt{R_1} + \sqrt{R_2}}
            = 0{,}200,  \\
    \eta_2
            &= \frac{R_2}{R_2 + r}
            = \frac{\sqrt{R_2}}{\sqrt{R_2} + \sqrt{R_1}}
            = 0{,}800
    \end{align*}
}

\variantsplitter

\addpersonalvariant{Виктория Легонькова}

\tasknumber{1}%
\task{%
    Напротив физических величин укажите их обозначения и единицы измерения в СИ:
    \begin{enumerate}
        \item разность потенциалов,
        \item работа тока,
        \item ЭДС,
        \item внешнее сопротивление полной цепи.
    \end{enumerate}
}
\solutionspace{20pt}

\tasknumber{2}%
\task{%
    Запишите физический закон или формулу:
    \begin{enumerate}
        \item правило Кирхгофа для узла цепи,
        \item закон Ома для однородного участка цепи,
        \item закон Ома для неоднородного участка цепи.
    \end{enumerate}
}
\solutionspace{40pt}

\tasknumber{3}%
\task{%
    На резистор сопротивлением $R = 5\,\text{Ом}$ подали напряжение $U = 240\,\text{В}$.
    Определите ток, который потечёт через резистор, и мощность, выделяющуюся на нём.
}
\answer{%
    \begin{align*}
    \eli &= \frac{U}{R} = \frac{240\,\text{В}}{5\,\text{Ом}} = 48\,\text{А},  \\
    P &= \frac{U^2}{R} = \frac{\sqr{240\,\text{В}}}{5\,\text{Ом}} = 11520\,\text{Вт}
    \end{align*}
}
\solutionspace{60pt}

\tasknumber{4}%
\task{%
    Через резистор сопротивлением $R = 5\,\text{Ом}$ протекает электрический ток $\eli = 10\,\text{А}$.
    Определите, чему равны напряжение на резисторе и мощность, выделяющаяся на нём.
}
\answer{%
    \begin{align*}
    U &= \eli R = 10\,\text{А} \cdot 5\,\text{Ом} = 50\,\text{В},  \\
    P &= \eli^2R = \sqr{10\,\text{А}} \cdot 5\,\text{Ом} = 500\,\text{Вт}
    \end{align*}
}
\solutionspace{60pt}

\tasknumber{5}%
\task{%
    Замкнутая электрическая цепь состоит из ЭДС $\ele = 2\,\text{В}$ и сопротивлением $r$
    и резистора $R = 24\,\text{Ом}$.
    Определите ток, протекающий в цепи.
    Какая тепловая энергия выделится на резисторе за время
    $\tau = 10\,\text{с}$? Какая работа будет совершена ЭДС за это время? Каков знак этой работы? Чему равен КПД цепи?
    Вычислите значения для 2 случаев: $r=0$ и $r = 60\,\text{Ом}$.
}
\answer{%
    \begin{align*}
    \eli_1 &= \frac{\ele}{R} = \frac{2\,\text{В}}{24\,\text{Ом}} = \frac1{12}\units{А} \approx 0{,}08\,\text{А},  \\
    \eli_2 &= \frac{\ele}{R + r} = \frac{2\,\text{В}}{24\,\text{Ом} + 60\,\text{Ом}} = \frac1{42}\units{А} \approx 0{,}02\,\text{А},  \\
    Q_1 &= \eli_1^2R\tau = \sqr{\frac{\ele}{R}} R \tau
            = \sqr{\frac{2\,\text{В}}{24\,\text{Ом}}} \cdot 24\,\text{Ом} \cdot 10\,\text{с} = \frac53\units{Дж} \approx 1{,}667\,\text{Дж},  \\
    Q_2 &= \eli_2^2R\tau = \sqr{\frac{\ele}{R + r}} R \tau
            = \sqr{\frac{2\,\text{В}}{24\,\text{Ом} + 60\,\text{Ом}}} \cdot 24\,\text{Ом} \cdot 10\,\text{с} = \frac{20}{147}\units{Дж} \approx 0{,}136\,\text{Дж},  \\
    A_1 &= q_1\ele = \eli_1\tau\ele = \frac{\ele}{R} \tau \ele
            = \frac{\ele^2 \tau}{R} = \frac{\sqr{2\,\text{В}} \cdot 10\,\text{с}}{24\,\text{Ом}}
            = \frac53\units{Дж} \approx 1{,}667\,\text{Дж}, \text{положительна},  \\
    A_2 &= q_2\ele = \eli_2\tau\ele = \frac{\ele}{R + r} \tau \ele
            = \frac{\ele^2 \tau}{R + r} = \frac{\sqr{2\,\text{В}} \cdot 10\,\text{с}}{24\,\text{Ом} + 60\,\text{Ом}}
            = \frac{10}{21}\units{Дж} \approx 0{,}476\,\text{Дж}, \text{положительна},  \\
    \eta_1 &= \frac{Q_1}{A_1} = \ldots = \frac{R}{R} = 1,  \\
    \eta_2 &= \frac{Q_2}{A_2} = \ldots = \frac{R}{R + r} = \frac27 \approx 0{,}29.
    \end{align*}
}
\solutionspace{180pt}

\tasknumber{6}%
\task{%
    Лампочки, сопротивления которых $R_1 = 1\,\text{Ом}$ и $R_2 = 9\,\text{Ом}$, поочерёдно подключённные к некоторому источнику тока,
    потребляют одинаковую мощность.
    Найти внутреннее сопротивление источника и КПД цепи в каждом случае.
}
\answer{%
    \begin{align*}
        P_1 &= \sqr{\frac{\ele}{R_1 + r}}R_1,
        P_2  = \sqr{\frac{\ele}{R_2 + r}}R_2,
        P_1 = P_2 \implies  \\
        &\implies R_1 \sqr{R_2 + r} = R_2 \sqr{R_1 + r} \implies  \\
        &\implies R_1 R_2^2 + 2 R_1 R_2 r + R_1 r^2 =
                    R_2 R_1^2 + 2 R_2 R_1 r + R_2 r^2  \implies  \\
    &\implies r^2 (R_2 - R_1) = R_2^2 R_2 - R_1^2 R_2 \implies  \\
    &\implies r
            = \sqrt{R_1 R_2 \frac{R_2 - R_1}{R_2 - R_1}}
            = \sqrt{R_1 R_2}
            = \sqrt{1\,\text{Ом} \cdot 9\,\text{Ом}}
            = 3\,\text{Ом}.
            \\
    \eta_1
            &= \frac{R_1}{R_1 + r}
            = \frac{\sqrt{R_1}}{\sqrt{R_1} + \sqrt{R_2}}
            = 0{,}250,  \\
    \eta_2
            &= \frac{R_2}{R_2 + r}
            = \frac{\sqrt{R_2}}{\sqrt{R_2} + \sqrt{R_1}}
            = 0{,}750
    \end{align*}
}

\variantsplitter

\addpersonalvariant{Семён Мартынов}

\tasknumber{1}%
\task{%
    Напротив физических величин укажите их обозначения и единицы измерения в СИ:
    \begin{enumerate}
        \item сила тока,
        \item мощность тока,
        \item удельное сопротивление,
        \item внешнее сопротивление полной цепи.
    \end{enumerate}
}
\solutionspace{20pt}

\tasknumber{2}%
\task{%
    Запишите физический закон или формулу:
    \begin{enumerate}
        \item правило Кирхгофа для узла цепи,
        \item сопротивление резистора через удельное сопротивление,
        \item закон Ома для неоднородного участка цепи.
    \end{enumerate}
}
\solutionspace{40pt}

\tasknumber{3}%
\task{%
    На резистор сопротивлением $R = 30\,\text{Ом}$ подали напряжение $U = 180\,\text{В}$.
    Определите ток, который потечёт через резистор, и мощность, выделяющуюся на нём.
}
\answer{%
    \begin{align*}
    \eli &= \frac{U}{R} = \frac{180\,\text{В}}{30\,\text{Ом}} = 6\,\text{А},  \\
    P &= \frac{U^2}{R} = \frac{\sqr{180\,\text{В}}}{30\,\text{Ом}} = 1080\,\text{Вт}
    \end{align*}
}
\solutionspace{60pt}

\tasknumber{4}%
\task{%
    Через резистор сопротивлением $R = 18\,\text{Ом}$ протекает электрический ток $\eli = 15\,\text{А}$.
    Определите, чему равны напряжение на резисторе и мощность, выделяющаяся на нём.
}
\answer{%
    \begin{align*}
    U &= \eli R = 15\,\text{А} \cdot 18\,\text{Ом} = 270\,\text{В},  \\
    P &= \eli^2R = \sqr{15\,\text{А}} \cdot 18\,\text{Ом} = 4050\,\text{Вт}
    \end{align*}
}
\solutionspace{60pt}

\tasknumber{5}%
\task{%
    Замкнутая электрическая цепь состоит из ЭДС $\ele = 2\,\text{В}$ и сопротивлением $r$
    и резистора $R = 10\,\text{Ом}$.
    Определите ток, протекающий в цепи.
    Какая тепловая энергия выделится на резисторе за время
    $\tau = 2\,\text{с}$? Какая работа будет совершена ЭДС за это время? Каков знак этой работы? Чему равен КПД цепи?
    Вычислите значения для 2 случаев: $r=0$ и $r = 60\,\text{Ом}$.
}
\answer{%
    \begin{align*}
    \eli_1 &= \frac{\ele}{R} = \frac{2\,\text{В}}{10\,\text{Ом}} = \frac15\units{А} \approx 0{,}20\,\text{А},  \\
    \eli_2 &= \frac{\ele}{R + r} = \frac{2\,\text{В}}{10\,\text{Ом} + 60\,\text{Ом}} = \frac1{35}\units{А} \approx 0{,}03\,\text{А},  \\
    Q_1 &= \eli_1^2R\tau = \sqr{\frac{\ele}{R}} R \tau
            = \sqr{\frac{2\,\text{В}}{10\,\text{Ом}}} \cdot 10\,\text{Ом} \cdot 2\,\text{с} = \frac45\units{Дж} \approx 0{,}800\,\text{Дж},  \\
    Q_2 &= \eli_2^2R\tau = \sqr{\frac{\ele}{R + r}} R \tau
            = \sqr{\frac{2\,\text{В}}{10\,\text{Ом} + 60\,\text{Ом}}} \cdot 10\,\text{Ом} \cdot 2\,\text{с} = \frac4{245}\units{Дж} \approx 0{,}016\,\text{Дж},  \\
    A_1 &= q_1\ele = \eli_1\tau\ele = \frac{\ele}{R} \tau \ele
            = \frac{\ele^2 \tau}{R} = \frac{\sqr{2\,\text{В}} \cdot 2\,\text{с}}{10\,\text{Ом}}
            = \frac45\units{Дж} \approx 0{,}800\,\text{Дж}, \text{положительна},  \\
    A_2 &= q_2\ele = \eli_2\tau\ele = \frac{\ele}{R + r} \tau \ele
            = \frac{\ele^2 \tau}{R + r} = \frac{\sqr{2\,\text{В}} \cdot 2\,\text{с}}{10\,\text{Ом} + 60\,\text{Ом}}
            = \frac4{35}\units{Дж} \approx 0{,}114\,\text{Дж}, \text{положительна},  \\
    \eta_1 &= \frac{Q_1}{A_1} = \ldots = \frac{R}{R} = 1,  \\
    \eta_2 &= \frac{Q_2}{A_2} = \ldots = \frac{R}{R + r} = \frac17 \approx 0{,}14.
    \end{align*}
}
\solutionspace{180pt}

\tasknumber{6}%
\task{%
    Лампочки, сопротивления которых $R_1 = 0{,}50\,\text{Ом}$ и $R_2 = 18\,\text{Ом}$, поочерёдно подключённные к некоторому источнику тока,
    потребляют одинаковую мощность.
    Найти внутреннее сопротивление источника и КПД цепи в каждом случае.
}
\answer{%
    \begin{align*}
        P_1 &= \sqr{\frac{\ele}{R_1 + r}}R_1,
        P_2  = \sqr{\frac{\ele}{R_2 + r}}R_2,
        P_1 = P_2 \implies  \\
        &\implies R_1 \sqr{R_2 + r} = R_2 \sqr{R_1 + r} \implies  \\
        &\implies R_1 R_2^2 + 2 R_1 R_2 r + R_1 r^2 =
                    R_2 R_1^2 + 2 R_2 R_1 r + R_2 r^2  \implies  \\
    &\implies r^2 (R_2 - R_1) = R_2^2 R_2 - R_1^2 R_2 \implies  \\
    &\implies r
            = \sqrt{R_1 R_2 \frac{R_2 - R_1}{R_2 - R_1}}
            = \sqrt{R_1 R_2}
            = \sqrt{0{,}50\,\text{Ом} \cdot 18\,\text{Ом}}
            = 3\,\text{Ом}.
            \\
    \eta_1
            &= \frac{R_1}{R_1 + r}
            = \frac{\sqrt{R_1}}{\sqrt{R_1} + \sqrt{R_2}}
            = 0{,}143,  \\
    \eta_2
            &= \frac{R_2}{R_2 + r}
            = \frac{\sqrt{R_2}}{\sqrt{R_2} + \sqrt{R_1}}
            = 0{,}857
    \end{align*}
}

\variantsplitter

\addpersonalvariant{Варвара Минаева}

\tasknumber{1}%
\task{%
    Напротив физических величин укажите их обозначения и единицы измерения в СИ:
    \begin{enumerate}
        \item напряжение,
        \item работа тока,
        \item удельное сопротивление,
        \item внутреннее сопротивление полной цепи.
    \end{enumerate}
}
\solutionspace{20pt}

\tasknumber{2}%
\task{%
    Запишите физический закон или формулу:
    \begin{enumerate}
        \item правило Кирхгофа для замкнутого контура,
        \item закон Ома для однородного участка цепи,
        \item закон Ома для неоднородного участка цепи.
    \end{enumerate}
}
\solutionspace{40pt}

\tasknumber{3}%
\task{%
    На резистор сопротивлением $r = 5\,\text{Ом}$ подали напряжение $V = 180\,\text{В}$.
    Определите ток, который потечёт через резистор, и мощность, выделяющуюся на нём.
}
\answer{%
    \begin{align*}
    \eli &= \frac{V}{r} = \frac{180\,\text{В}}{5\,\text{Ом}} = 36\,\text{А},  \\
    P &= \frac{V^2}{r} = \frac{\sqr{180\,\text{В}}}{5\,\text{Ом}} = 6480\,\text{Вт}
    \end{align*}
}
\solutionspace{60pt}

\tasknumber{4}%
\task{%
    Через резистор сопротивлением $r = 30\,\text{Ом}$ протекает электрический ток $\eli = 6\,\text{А}$.
    Определите, чему равны напряжение на резисторе и мощность, выделяющаяся на нём.
}
\answer{%
    \begin{align*}
    U &= \eli r = 6\,\text{А} \cdot 30\,\text{Ом} = 180\,\text{В},  \\
    P &= \eli^2r = \sqr{6\,\text{А}} \cdot 30\,\text{Ом} = 1080\,\text{Вт}
    \end{align*}
}
\solutionspace{60pt}

\tasknumber{5}%
\task{%
    Замкнутая электрическая цепь состоит из ЭДС $\ele = 1\,\text{В}$ и сопротивлением $r$
    и резистора $R = 10\,\text{Ом}$.
    Определите ток, протекающий в цепи.
    Какая тепловая энергия выделится на резисторе за время
    $\tau = 5\,\text{с}$? Какая работа будет совершена ЭДС за это время? Каков знак этой работы? Чему равен КПД цепи?
    Вычислите значения для 2 случаев: $r=0$ и $r = 20\,\text{Ом}$.
}
\answer{%
    \begin{align*}
    \eli_1 &= \frac{\ele}{R} = \frac{1\,\text{В}}{10\,\text{Ом}} = \frac1{10}\units{А} \approx 0{,}10\,\text{А},  \\
    \eli_2 &= \frac{\ele}{R + r} = \frac{1\,\text{В}}{10\,\text{Ом} + 20\,\text{Ом}} = \frac1{30}\units{А} \approx 0{,}03\,\text{А},  \\
    Q_1 &= \eli_1^2R\tau = \sqr{\frac{\ele}{R}} R \tau
            = \sqr{\frac{1\,\text{В}}{10\,\text{Ом}}} \cdot 10\,\text{Ом} \cdot 5\,\text{с} = \frac12\units{Дж} \approx 0{,}500\,\text{Дж},  \\
    Q_2 &= \eli_2^2R\tau = \sqr{\frac{\ele}{R + r}} R \tau
            = \sqr{\frac{1\,\text{В}}{10\,\text{Ом} + 20\,\text{Ом}}} \cdot 10\,\text{Ом} \cdot 5\,\text{с} = \frac1{18}\units{Дж} \approx 0{,}056\,\text{Дж},  \\
    A_1 &= q_1\ele = \eli_1\tau\ele = \frac{\ele}{R} \tau \ele
            = \frac{\ele^2 \tau}{R} = \frac{\sqr{1\,\text{В}} \cdot 5\,\text{с}}{10\,\text{Ом}}
            = \frac12\units{Дж} \approx 0{,}500\,\text{Дж}, \text{положительна},  \\
    A_2 &= q_2\ele = \eli_2\tau\ele = \frac{\ele}{R + r} \tau \ele
            = \frac{\ele^2 \tau}{R + r} = \frac{\sqr{1\,\text{В}} \cdot 5\,\text{с}}{10\,\text{Ом} + 20\,\text{Ом}}
            = \frac16\units{Дж} \approx 0{,}167\,\text{Дж}, \text{положительна},  \\
    \eta_1 &= \frac{Q_1}{A_1} = \ldots = \frac{R}{R} = 1,  \\
    \eta_2 &= \frac{Q_2}{A_2} = \ldots = \frac{R}{R + r} = \frac13 \approx 0{,}33.
    \end{align*}
}
\solutionspace{180pt}

\tasknumber{6}%
\task{%
    Лампочки, сопротивления которых $R_1 = 0{,}25\,\text{Ом}$ и $R_2 = 4\,\text{Ом}$, поочерёдно подключённные к некоторому источнику тока,
    потребляют одинаковую мощность.
    Найти внутреннее сопротивление источника и КПД цепи в каждом случае.
}
\answer{%
    \begin{align*}
        P_1 &= \sqr{\frac{\ele}{R_1 + r}}R_1,
        P_2  = \sqr{\frac{\ele}{R_2 + r}}R_2,
        P_1 = P_2 \implies  \\
        &\implies R_1 \sqr{R_2 + r} = R_2 \sqr{R_1 + r} \implies  \\
        &\implies R_1 R_2^2 + 2 R_1 R_2 r + R_1 r^2 =
                    R_2 R_1^2 + 2 R_2 R_1 r + R_2 r^2  \implies  \\
    &\implies r^2 (R_2 - R_1) = R_2^2 R_2 - R_1^2 R_2 \implies  \\
    &\implies r
            = \sqrt{R_1 R_2 \frac{R_2 - R_1}{R_2 - R_1}}
            = \sqrt{R_1 R_2}
            = \sqrt{0{,}25\,\text{Ом} \cdot 4\,\text{Ом}}
            = 1\,\text{Ом}.
            \\
    \eta_1
            &= \frac{R_1}{R_1 + r}
            = \frac{\sqrt{R_1}}{\sqrt{R_1} + \sqrt{R_2}}
            = 0{,}200,  \\
    \eta_2
            &= \frac{R_2}{R_2 + r}
            = \frac{\sqrt{R_2}}{\sqrt{R_2} + \sqrt{R_1}}
            = 0{,}800
    \end{align*}
}

\variantsplitter

\addpersonalvariant{Леонид Никитин}

\tasknumber{1}%
\task{%
    Напротив физических величин укажите их обозначения и единицы измерения в СИ:
    \begin{enumerate}
        \item разность потенциалов,
        \item мощность тока,
        \item удельное сопротивление,
        \item внешнее сопротивление полной цепи.
    \end{enumerate}
}
\solutionspace{20pt}

\tasknumber{2}%
\task{%
    Запишите физический закон или формулу:
    \begin{enumerate}
        \item правило Кирхгофа для узла цепи,
        \item закон Ома для однородного участка цепи,
        \item ЭДС (определение).
    \end{enumerate}
}
\solutionspace{40pt}

\tasknumber{3}%
\task{%
    На резистор сопротивлением $r = 12\,\text{Ом}$ подали напряжение $U = 240\,\text{В}$.
    Определите ток, который потечёт через резистор, и мощность, выделяющуюся на нём.
}
\answer{%
    \begin{align*}
    \eli &= \frac{U}{r} = \frac{240\,\text{В}}{12\,\text{Ом}} = 20\,\text{А},  \\
    P &= \frac{U^2}{r} = \frac{\sqr{240\,\text{В}}}{12\,\text{Ом}} = 4800\,\text{Вт}
    \end{align*}
}
\solutionspace{60pt}

\tasknumber{4}%
\task{%
    Через резистор сопротивлением $r = 30\,\text{Ом}$ протекает электрический ток $\eli = 6\,\text{А}$.
    Определите, чему равны напряжение на резисторе и мощность, выделяющаяся на нём.
}
\answer{%
    \begin{align*}
    U &= \eli r = 6\,\text{А} \cdot 30\,\text{Ом} = 180\,\text{В},  \\
    P &= \eli^2r = \sqr{6\,\text{А}} \cdot 30\,\text{Ом} = 1080\,\text{Вт}
    \end{align*}
}
\solutionspace{60pt}

\tasknumber{5}%
\task{%
    Замкнутая электрическая цепь состоит из ЭДС $\ele = 4\,\text{В}$ и сопротивлением $r$
    и резистора $R = 15\,\text{Ом}$.
    Определите ток, протекающий в цепи.
    Какая тепловая энергия выделится на резисторе за время
    $\tau = 5\,\text{с}$? Какая работа будет совершена ЭДС за это время? Каков знак этой работы? Чему равен КПД цепи?
    Вычислите значения для 2 случаев: $r=0$ и $r = 10\,\text{Ом}$.
}
\answer{%
    \begin{align*}
    \eli_1 &= \frac{\ele}{R} = \frac{4\,\text{В}}{15\,\text{Ом}} = \frac4{15}\units{А} \approx 0{,}27\,\text{А},  \\
    \eli_2 &= \frac{\ele}{R + r} = \frac{4\,\text{В}}{15\,\text{Ом} + 10\,\text{Ом}} = \frac4{25}\units{А} \approx 0{,}16\,\text{А},  \\
    Q_1 &= \eli_1^2R\tau = \sqr{\frac{\ele}{R}} R \tau
            = \sqr{\frac{4\,\text{В}}{15\,\text{Ом}}} \cdot 15\,\text{Ом} \cdot 5\,\text{с} = \frac{16}3\units{Дж} \approx 5{,}333\,\text{Дж},  \\
    Q_2 &= \eli_2^2R\tau = \sqr{\frac{\ele}{R + r}} R \tau
            = \sqr{\frac{4\,\text{В}}{15\,\text{Ом} + 10\,\text{Ом}}} \cdot 15\,\text{Ом} \cdot 5\,\text{с} = \frac{48}{25}\units{Дж} \approx 1{,}920\,\text{Дж},  \\
    A_1 &= q_1\ele = \eli_1\tau\ele = \frac{\ele}{R} \tau \ele
            = \frac{\ele^2 \tau}{R} = \frac{\sqr{4\,\text{В}} \cdot 5\,\text{с}}{15\,\text{Ом}}
            = \frac{16}3\units{Дж} \approx 5{,}333\,\text{Дж}, \text{положительна},  \\
    A_2 &= q_2\ele = \eli_2\tau\ele = \frac{\ele}{R + r} \tau \ele
            = \frac{\ele^2 \tau}{R + r} = \frac{\sqr{4\,\text{В}} \cdot 5\,\text{с}}{15\,\text{Ом} + 10\,\text{Ом}}
            = \frac{16}5\units{Дж} \approx 3{,}200\,\text{Дж}, \text{положительна},  \\
    \eta_1 &= \frac{Q_1}{A_1} = \ldots = \frac{R}{R} = 1,  \\
    \eta_2 &= \frac{Q_2}{A_2} = \ldots = \frac{R}{R + r} = \frac35 \approx 0{,}60.
    \end{align*}
}
\solutionspace{180pt}

\tasknumber{6}%
\task{%
    Лампочки, сопротивления которых $R_1 = 0{,}50\,\text{Ом}$ и $R_2 = 18\,\text{Ом}$, поочерёдно подключённные к некоторому источнику тока,
    потребляют одинаковую мощность.
    Найти внутреннее сопротивление источника и КПД цепи в каждом случае.
}
\answer{%
    \begin{align*}
        P_1 &= \sqr{\frac{\ele}{R_1 + r}}R_1,
        P_2  = \sqr{\frac{\ele}{R_2 + r}}R_2,
        P_1 = P_2 \implies  \\
        &\implies R_1 \sqr{R_2 + r} = R_2 \sqr{R_1 + r} \implies  \\
        &\implies R_1 R_2^2 + 2 R_1 R_2 r + R_1 r^2 =
                    R_2 R_1^2 + 2 R_2 R_1 r + R_2 r^2  \implies  \\
    &\implies r^2 (R_2 - R_1) = R_2^2 R_2 - R_1^2 R_2 \implies  \\
    &\implies r
            = \sqrt{R_1 R_2 \frac{R_2 - R_1}{R_2 - R_1}}
            = \sqrt{R_1 R_2}
            = \sqrt{0{,}50\,\text{Ом} \cdot 18\,\text{Ом}}
            = 3\,\text{Ом}.
            \\
    \eta_1
            &= \frac{R_1}{R_1 + r}
            = \frac{\sqrt{R_1}}{\sqrt{R_1} + \sqrt{R_2}}
            = 0{,}143,  \\
    \eta_2
            &= \frac{R_2}{R_2 + r}
            = \frac{\sqrt{R_2}}{\sqrt{R_2} + \sqrt{R_1}}
            = 0{,}857
    \end{align*}
}

\variantsplitter

\addpersonalvariant{Тимофей Полетаев}

\tasknumber{1}%
\task{%
    Напротив физических величин укажите их обозначения и единицы измерения в СИ:
    \begin{enumerate}
        \item сила тока,
        \item мощность тока,
        \item ЭДС,
        \item внутреннее сопротивление полной цепи.
    \end{enumerate}
}
\solutionspace{20pt}

\tasknumber{2}%
\task{%
    Запишите физический закон или формулу:
    \begin{enumerate}
        \item правило Кирхгофа для замкнутого контура,
        \item закон Ома для однородного участка цепи,
        \item ЭДС (определение).
    \end{enumerate}
}
\solutionspace{40pt}

\tasknumber{3}%
\task{%
    На резистор сопротивлением $r = 12\,\text{Ом}$ подали напряжение $U = 180\,\text{В}$.
    Определите ток, который потечёт через резистор, и мощность, выделяющуюся на нём.
}
\answer{%
    \begin{align*}
    \eli &= \frac{U}{r} = \frac{180\,\text{В}}{12\,\text{Ом}} = 15\,\text{А},  \\
    P &= \frac{U^2}{r} = \frac{\sqr{180\,\text{В}}}{12\,\text{Ом}} = 2700\,\text{Вт}
    \end{align*}
}
\solutionspace{60pt}

\tasknumber{4}%
\task{%
    Через резистор сопротивлением $R = 30\,\text{Ом}$ протекает электрический ток $\eli = 5\,\text{А}$.
    Определите, чему равны напряжение на резисторе и мощность, выделяющаяся на нём.
}
\answer{%
    \begin{align*}
    U &= \eli R = 5\,\text{А} \cdot 30\,\text{Ом} = 150\,\text{В},  \\
    P &= \eli^2R = \sqr{5\,\text{А}} \cdot 30\,\text{Ом} = 750\,\text{Вт}
    \end{align*}
}
\solutionspace{60pt}

\tasknumber{5}%
\task{%
    Замкнутая электрическая цепь состоит из ЭДС $\ele = 3\,\text{В}$ и сопротивлением $r$
    и резистора $R = 24\,\text{Ом}$.
    Определите ток, протекающий в цепи.
    Какая тепловая энергия выделится на резисторе за время
    $\tau = 2\,\text{с}$? Какая работа будет совершена ЭДС за это время? Каков знак этой работы? Чему равен КПД цепи?
    Вычислите значения для 2 случаев: $r=0$ и $r = 10\,\text{Ом}$.
}
\answer{%
    \begin{align*}
    \eli_1 &= \frac{\ele}{R} = \frac{3\,\text{В}}{24\,\text{Ом}} = \frac18\units{А} \approx 0{,}12\,\text{А},  \\
    \eli_2 &= \frac{\ele}{R + r} = \frac{3\,\text{В}}{24\,\text{Ом} + 10\,\text{Ом}} = \frac3{34}\units{А} \approx 0{,}09\,\text{А},  \\
    Q_1 &= \eli_1^2R\tau = \sqr{\frac{\ele}{R}} R \tau
            = \sqr{\frac{3\,\text{В}}{24\,\text{Ом}}} \cdot 24\,\text{Ом} \cdot 2\,\text{с} = \frac34\units{Дж} \approx 0{,}750\,\text{Дж},  \\
    Q_2 &= \eli_2^2R\tau = \sqr{\frac{\ele}{R + r}} R \tau
            = \sqr{\frac{3\,\text{В}}{24\,\text{Ом} + 10\,\text{Ом}}} \cdot 24\,\text{Ом} \cdot 2\,\text{с} = \frac{108}{289}\units{Дж} \approx 0{,}374\,\text{Дж},  \\
    A_1 &= q_1\ele = \eli_1\tau\ele = \frac{\ele}{R} \tau \ele
            = \frac{\ele^2 \tau}{R} = \frac{\sqr{3\,\text{В}} \cdot 2\,\text{с}}{24\,\text{Ом}}
            = \frac34\units{Дж} \approx 0{,}750\,\text{Дж}, \text{положительна},  \\
    A_2 &= q_2\ele = \eli_2\tau\ele = \frac{\ele}{R + r} \tau \ele
            = \frac{\ele^2 \tau}{R + r} = \frac{\sqr{3\,\text{В}} \cdot 2\,\text{с}}{24\,\text{Ом} + 10\,\text{Ом}}
            = \frac9{17}\units{Дж} \approx 0{,}529\,\text{Дж}, \text{положительна},  \\
    \eta_1 &= \frac{Q_1}{A_1} = \ldots = \frac{R}{R} = 1,  \\
    \eta_2 &= \frac{Q_2}{A_2} = \ldots = \frac{R}{R + r} = \frac{12}{17} \approx 0{,}71.
    \end{align*}
}
\solutionspace{180pt}

\tasknumber{6}%
\task{%
    Лампочки, сопротивления которых $R_1 = 4\,\text{Ом}$ и $R_2 = 36\,\text{Ом}$, поочерёдно подключённные к некоторому источнику тока,
    потребляют одинаковую мощность.
    Найти внутреннее сопротивление источника и КПД цепи в каждом случае.
}
\answer{%
    \begin{align*}
        P_1 &= \sqr{\frac{\ele}{R_1 + r}}R_1,
        P_2  = \sqr{\frac{\ele}{R_2 + r}}R_2,
        P_1 = P_2 \implies  \\
        &\implies R_1 \sqr{R_2 + r} = R_2 \sqr{R_1 + r} \implies  \\
        &\implies R_1 R_2^2 + 2 R_1 R_2 r + R_1 r^2 =
                    R_2 R_1^2 + 2 R_2 R_1 r + R_2 r^2  \implies  \\
    &\implies r^2 (R_2 - R_1) = R_2^2 R_2 - R_1^2 R_2 \implies  \\
    &\implies r
            = \sqrt{R_1 R_2 \frac{R_2 - R_1}{R_2 - R_1}}
            = \sqrt{R_1 R_2}
            = \sqrt{4\,\text{Ом} \cdot 36\,\text{Ом}}
            = 12\,\text{Ом}.
            \\
    \eta_1
            &= \frac{R_1}{R_1 + r}
            = \frac{\sqrt{R_1}}{\sqrt{R_1} + \sqrt{R_2}}
            = 0{,}250,  \\
    \eta_2
            &= \frac{R_2}{R_2 + r}
            = \frac{\sqrt{R_2}}{\sqrt{R_2} + \sqrt{R_1}}
            = 0{,}750
    \end{align*}
}

\variantsplitter

\addpersonalvariant{Андрей Рожков}

\tasknumber{1}%
\task{%
    Напротив физических величин укажите их обозначения и единицы измерения в СИ:
    \begin{enumerate}
        \item разность потенциалов,
        \item мощность тока,
        \item ЭДС,
        \item внешнее сопротивление полной цепи.
    \end{enumerate}
}
\solutionspace{20pt}

\tasknumber{2}%
\task{%
    Запишите физический закон или формулу:
    \begin{enumerate}
        \item правило Кирхгофа для узла цепи,
        \item закон Ома для однородного участка цепи,
        \item закон Ома для неоднородного участка цепи.
    \end{enumerate}
}
\solutionspace{40pt}

\tasknumber{3}%
\task{%
    На резистор сопротивлением $r = 12\,\text{Ом}$ подали напряжение $U = 150\,\text{В}$.
    Определите ток, который потечёт через резистор, и мощность, выделяющуюся на нём.
}
\answer{%
    \begin{align*}
    \eli &= \frac{U}{r} = \frac{150\,\text{В}}{12\,\text{Ом}} = 12{,}50\,\text{А},  \\
    P &= \frac{U^2}{r} = \frac{\sqr{150\,\text{В}}}{12\,\text{Ом}} = 1875\,\text{Вт}
    \end{align*}
}
\solutionspace{60pt}

\tasknumber{4}%
\task{%
    Через резистор сопротивлением $r = 5\,\text{Ом}$ протекает электрический ток $\eli = 10\,\text{А}$.
    Определите, чему равны напряжение на резисторе и мощность, выделяющаяся на нём.
}
\answer{%
    \begin{align*}
    U &= \eli r = 10\,\text{А} \cdot 5\,\text{Ом} = 50\,\text{В},  \\
    P &= \eli^2r = \sqr{10\,\text{А}} \cdot 5\,\text{Ом} = 500\,\text{Вт}
    \end{align*}
}
\solutionspace{60pt}

\tasknumber{5}%
\task{%
    Замкнутая электрическая цепь состоит из ЭДС $\ele = 3\,\text{В}$ и сопротивлением $r$
    и резистора $R = 24\,\text{Ом}$.
    Определите ток, протекающий в цепи.
    Какая тепловая энергия выделится на резисторе за время
    $\tau = 5\,\text{с}$? Какая работа будет совершена ЭДС за это время? Каков знак этой работы? Чему равен КПД цепи?
    Вычислите значения для 2 случаев: $r=0$ и $r = 30\,\text{Ом}$.
}
\answer{%
    \begin{align*}
    \eli_1 &= \frac{\ele}{R} = \frac{3\,\text{В}}{24\,\text{Ом}} = \frac18\units{А} \approx 0{,}12\,\text{А},  \\
    \eli_2 &= \frac{\ele}{R + r} = \frac{3\,\text{В}}{24\,\text{Ом} + 30\,\text{Ом}} = \frac1{18}\units{А} \approx 0{,}06\,\text{А},  \\
    Q_1 &= \eli_1^2R\tau = \sqr{\frac{\ele}{R}} R \tau
            = \sqr{\frac{3\,\text{В}}{24\,\text{Ом}}} \cdot 24\,\text{Ом} \cdot 5\,\text{с} = \frac{15}8\units{Дж} \approx 1{,}875\,\text{Дж},  \\
    Q_2 &= \eli_2^2R\tau = \sqr{\frac{\ele}{R + r}} R \tau
            = \sqr{\frac{3\,\text{В}}{24\,\text{Ом} + 30\,\text{Ом}}} \cdot 24\,\text{Ом} \cdot 5\,\text{с} = \frac{10}{27}\units{Дж} \approx 0{,}370\,\text{Дж},  \\
    A_1 &= q_1\ele = \eli_1\tau\ele = \frac{\ele}{R} \tau \ele
            = \frac{\ele^2 \tau}{R} = \frac{\sqr{3\,\text{В}} \cdot 5\,\text{с}}{24\,\text{Ом}}
            = \frac{15}8\units{Дж} \approx 1{,}875\,\text{Дж}, \text{положительна},  \\
    A_2 &= q_2\ele = \eli_2\tau\ele = \frac{\ele}{R + r} \tau \ele
            = \frac{\ele^2 \tau}{R + r} = \frac{\sqr{3\,\text{В}} \cdot 5\,\text{с}}{24\,\text{Ом} + 30\,\text{Ом}}
            = \frac56\units{Дж} \approx 0{,}833\,\text{Дж}, \text{положительна},  \\
    \eta_1 &= \frac{Q_1}{A_1} = \ldots = \frac{R}{R} = 1,  \\
    \eta_2 &= \frac{Q_2}{A_2} = \ldots = \frac{R}{R + r} = \frac49 \approx 0{,}44.
    \end{align*}
}
\solutionspace{180pt}

\tasknumber{6}%
\task{%
    Лампочки, сопротивления которых $R_1 = 5\,\text{Ом}$ и $R_2 = 45\,\text{Ом}$, поочерёдно подключённные к некоторому источнику тока,
    потребляют одинаковую мощность.
    Найти внутреннее сопротивление источника и КПД цепи в каждом случае.
}
\answer{%
    \begin{align*}
        P_1 &= \sqr{\frac{\ele}{R_1 + r}}R_1,
        P_2  = \sqr{\frac{\ele}{R_2 + r}}R_2,
        P_1 = P_2 \implies  \\
        &\implies R_1 \sqr{R_2 + r} = R_2 \sqr{R_1 + r} \implies  \\
        &\implies R_1 R_2^2 + 2 R_1 R_2 r + R_1 r^2 =
                    R_2 R_1^2 + 2 R_2 R_1 r + R_2 r^2  \implies  \\
    &\implies r^2 (R_2 - R_1) = R_2^2 R_2 - R_1^2 R_2 \implies  \\
    &\implies r
            = \sqrt{R_1 R_2 \frac{R_2 - R_1}{R_2 - R_1}}
            = \sqrt{R_1 R_2}
            = \sqrt{5\,\text{Ом} \cdot 45\,\text{Ом}}
            = 15\,\text{Ом}.
            \\
    \eta_1
            &= \frac{R_1}{R_1 + r}
            = \frac{\sqrt{R_1}}{\sqrt{R_1} + \sqrt{R_2}}
            = 0{,}250,  \\
    \eta_2
            &= \frac{R_2}{R_2 + r}
            = \frac{\sqrt{R_2}}{\sqrt{R_2} + \sqrt{R_1}}
            = 0{,}750
    \end{align*}
}

\variantsplitter

\addpersonalvariant{Рената Таржиманова}

\tasknumber{1}%
\task{%
    Напротив физических величин укажите их обозначения и единицы измерения в СИ:
    \begin{enumerate}
        \item напряжение,
        \item работа тока,
        \item ЭДС,
        \item внутреннее сопротивление полной цепи.
    \end{enumerate}
}
\solutionspace{20pt}

\tasknumber{2}%
\task{%
    Запишите физический закон или формулу:
    \begin{enumerate}
        \item правило Кирхгофа для замкнутого контура,
        \item закон Ома для однородного участка цепи,
        \item ЭДС (определение).
    \end{enumerate}
}
\solutionspace{40pt}

\tasknumber{3}%
\task{%
    На резистор сопротивлением $R = 12\,\text{Ом}$ подали напряжение $U = 150\,\text{В}$.
    Определите ток, который потечёт через резистор, и мощность, выделяющуюся на нём.
}
\answer{%
    \begin{align*}
    \eli &= \frac{U}{R} = \frac{150\,\text{В}}{12\,\text{Ом}} = 12{,}50\,\text{А},  \\
    P &= \frac{U^2}{R} = \frac{\sqr{150\,\text{В}}}{12\,\text{Ом}} = 1875\,\text{Вт}
    \end{align*}
}
\solutionspace{60pt}

\tasknumber{4}%
\task{%
    Через резистор сопротивлением $R = 18\,\text{Ом}$ протекает электрический ток $\eli = 5\,\text{А}$.
    Определите, чему равны напряжение на резисторе и мощность, выделяющаяся на нём.
}
\answer{%
    \begin{align*}
    U &= \eli R = 5\,\text{А} \cdot 18\,\text{Ом} = 90\,\text{В},  \\
    P &= \eli^2R = \sqr{5\,\text{А}} \cdot 18\,\text{Ом} = 450\,\text{Вт}
    \end{align*}
}
\solutionspace{60pt}

\tasknumber{5}%
\task{%
    Замкнутая электрическая цепь состоит из ЭДС $\ele = 4\,\text{В}$ и сопротивлением $r$
    и резистора $R = 24\,\text{Ом}$.
    Определите ток, протекающий в цепи.
    Какая тепловая энергия выделится на резисторе за время
    $\tau = 2\,\text{с}$? Какая работа будет совершена ЭДС за это время? Каков знак этой работы? Чему равен КПД цепи?
    Вычислите значения для 2 случаев: $r=0$ и $r = 30\,\text{Ом}$.
}
\answer{%
    \begin{align*}
    \eli_1 &= \frac{\ele}{R} = \frac{4\,\text{В}}{24\,\text{Ом}} = \frac16\units{А} \approx 0{,}17\,\text{А},  \\
    \eli_2 &= \frac{\ele}{R + r} = \frac{4\,\text{В}}{24\,\text{Ом} + 30\,\text{Ом}} = \frac2{27}\units{А} \approx 0{,}07\,\text{А},  \\
    Q_1 &= \eli_1^2R\tau = \sqr{\frac{\ele}{R}} R \tau
            = \sqr{\frac{4\,\text{В}}{24\,\text{Ом}}} \cdot 24\,\text{Ом} \cdot 2\,\text{с} = \frac43\units{Дж} \approx 1{,}333\,\text{Дж},  \\
    Q_2 &= \eli_2^2R\tau = \sqr{\frac{\ele}{R + r}} R \tau
            = \sqr{\frac{4\,\text{В}}{24\,\text{Ом} + 30\,\text{Ом}}} \cdot 24\,\text{Ом} \cdot 2\,\text{с} = \frac{64}{243}\units{Дж} \approx 0{,}263\,\text{Дж},  \\
    A_1 &= q_1\ele = \eli_1\tau\ele = \frac{\ele}{R} \tau \ele
            = \frac{\ele^2 \tau}{R} = \frac{\sqr{4\,\text{В}} \cdot 2\,\text{с}}{24\,\text{Ом}}
            = \frac43\units{Дж} \approx 1{,}333\,\text{Дж}, \text{положительна},  \\
    A_2 &= q_2\ele = \eli_2\tau\ele = \frac{\ele}{R + r} \tau \ele
            = \frac{\ele^2 \tau}{R + r} = \frac{\sqr{4\,\text{В}} \cdot 2\,\text{с}}{24\,\text{Ом} + 30\,\text{Ом}}
            = \frac{16}{27}\units{Дж} \approx 0{,}593\,\text{Дж}, \text{положительна},  \\
    \eta_1 &= \frac{Q_1}{A_1} = \ldots = \frac{R}{R} = 1,  \\
    \eta_2 &= \frac{Q_2}{A_2} = \ldots = \frac{R}{R + r} = \frac49 \approx 0{,}44.
    \end{align*}
}
\solutionspace{180pt}

\tasknumber{6}%
\task{%
    Лампочки, сопротивления которых $R_1 = 4\,\text{Ом}$ и $R_2 = 36\,\text{Ом}$, поочерёдно подключённные к некоторому источнику тока,
    потребляют одинаковую мощность.
    Найти внутреннее сопротивление источника и КПД цепи в каждом случае.
}
\answer{%
    \begin{align*}
        P_1 &= \sqr{\frac{\ele}{R_1 + r}}R_1,
        P_2  = \sqr{\frac{\ele}{R_2 + r}}R_2,
        P_1 = P_2 \implies  \\
        &\implies R_1 \sqr{R_2 + r} = R_2 \sqr{R_1 + r} \implies  \\
        &\implies R_1 R_2^2 + 2 R_1 R_2 r + R_1 r^2 =
                    R_2 R_1^2 + 2 R_2 R_1 r + R_2 r^2  \implies  \\
    &\implies r^2 (R_2 - R_1) = R_2^2 R_2 - R_1^2 R_2 \implies  \\
    &\implies r
            = \sqrt{R_1 R_2 \frac{R_2 - R_1}{R_2 - R_1}}
            = \sqrt{R_1 R_2}
            = \sqrt{4\,\text{Ом} \cdot 36\,\text{Ом}}
            = 12\,\text{Ом}.
            \\
    \eta_1
            &= \frac{R_1}{R_1 + r}
            = \frac{\sqrt{R_1}}{\sqrt{R_1} + \sqrt{R_2}}
            = 0{,}250,  \\
    \eta_2
            &= \frac{R_2}{R_2 + r}
            = \frac{\sqrt{R_2}}{\sqrt{R_2} + \sqrt{R_1}}
            = 0{,}750
    \end{align*}
}

\variantsplitter

\addpersonalvariant{Андрей Щербаков}

\tasknumber{1}%
\task{%
    Напротив физических величин укажите их обозначения и единицы измерения в СИ:
    \begin{enumerate}
        \item напряжение,
        \item мощность тока,
        \item ЭДС,
        \item внешнее сопротивление полной цепи.
    \end{enumerate}
}
\solutionspace{20pt}

\tasknumber{2}%
\task{%
    Запишите физический закон или формулу:
    \begin{enumerate}
        \item правило Кирхгофа для замкнутого контура,
        \item сопротивление резистора через удельное сопротивление,
        \item ЭДС (определение).
    \end{enumerate}
}
\solutionspace{40pt}

\tasknumber{3}%
\task{%
    На резистор сопротивлением $R = 5\,\text{Ом}$ подали напряжение $U = 180\,\text{В}$.
    Определите ток, который потечёт через резистор, и мощность, выделяющуюся на нём.
}
\answer{%
    \begin{align*}
    \eli &= \frac{U}{R} = \frac{180\,\text{В}}{5\,\text{Ом}} = 36\,\text{А},  \\
    P &= \frac{U^2}{R} = \frac{\sqr{180\,\text{В}}}{5\,\text{Ом}} = 6480\,\text{Вт}
    \end{align*}
}
\solutionspace{60pt}

\tasknumber{4}%
\task{%
    Через резистор сопротивлением $r = 12\,\text{Ом}$ протекает электрический ток $\eli = 5\,\text{А}$.
    Определите, чему равны напряжение на резисторе и мощность, выделяющаяся на нём.
}
\answer{%
    \begin{align*}
    U &= \eli r = 5\,\text{А} \cdot 12\,\text{Ом} = 60\,\text{В},  \\
    P &= \eli^2r = \sqr{5\,\text{А}} \cdot 12\,\text{Ом} = 300\,\text{Вт}
    \end{align*}
}
\solutionspace{60pt}

\tasknumber{5}%
\task{%
    Замкнутая электрическая цепь состоит из ЭДС $\ele = 4\,\text{В}$ и сопротивлением $r$
    и резистора $R = 30\,\text{Ом}$.
    Определите ток, протекающий в цепи.
    Какая тепловая энергия выделится на резисторе за время
    $\tau = 2\,\text{с}$? Какая работа будет совершена ЭДС за это время? Каков знак этой работы? Чему равен КПД цепи?
    Вычислите значения для 2 случаев: $r=0$ и $r = 20\,\text{Ом}$.
}
\answer{%
    \begin{align*}
    \eli_1 &= \frac{\ele}{R} = \frac{4\,\text{В}}{30\,\text{Ом}} = \frac2{15}\units{А} \approx 0{,}13\,\text{А},  \\
    \eli_2 &= \frac{\ele}{R + r} = \frac{4\,\text{В}}{30\,\text{Ом} + 20\,\text{Ом}} = \frac2{25}\units{А} \approx 0{,}08\,\text{А},  \\
    Q_1 &= \eli_1^2R\tau = \sqr{\frac{\ele}{R}} R \tau
            = \sqr{\frac{4\,\text{В}}{30\,\text{Ом}}} \cdot 30\,\text{Ом} \cdot 2\,\text{с} = \frac{16}{15}\units{Дж} \approx 1{,}067\,\text{Дж},  \\
    Q_2 &= \eli_2^2R\tau = \sqr{\frac{\ele}{R + r}} R \tau
            = \sqr{\frac{4\,\text{В}}{30\,\text{Ом} + 20\,\text{Ом}}} \cdot 30\,\text{Ом} \cdot 2\,\text{с} = \frac{48}{125}\units{Дж} \approx 0{,}384\,\text{Дж},  \\
    A_1 &= q_1\ele = \eli_1\tau\ele = \frac{\ele}{R} \tau \ele
            = \frac{\ele^2 \tau}{R} = \frac{\sqr{4\,\text{В}} \cdot 2\,\text{с}}{30\,\text{Ом}}
            = \frac{16}{15}\units{Дж} \approx 1{,}067\,\text{Дж}, \text{положительна},  \\
    A_2 &= q_2\ele = \eli_2\tau\ele = \frac{\ele}{R + r} \tau \ele
            = \frac{\ele^2 \tau}{R + r} = \frac{\sqr{4\,\text{В}} \cdot 2\,\text{с}}{30\,\text{Ом} + 20\,\text{Ом}}
            = \frac{16}{25}\units{Дж} \approx 0{,}640\,\text{Дж}, \text{положительна},  \\
    \eta_1 &= \frac{Q_1}{A_1} = \ldots = \frac{R}{R} = 1,  \\
    \eta_2 &= \frac{Q_2}{A_2} = \ldots = \frac{R}{R + r} = \frac35 \approx 0{,}60.
    \end{align*}
}
\solutionspace{180pt}

\tasknumber{6}%
\task{%
    Лампочки, сопротивления которых $R_1 = 0{,}50\,\text{Ом}$ и $R_2 = 4{,}50\,\text{Ом}$, поочерёдно подключённные к некоторому источнику тока,
    потребляют одинаковую мощность.
    Найти внутреннее сопротивление источника и КПД цепи в каждом случае.
}
\answer{%
    \begin{align*}
        P_1 &= \sqr{\frac{\ele}{R_1 + r}}R_1,
        P_2  = \sqr{\frac{\ele}{R_2 + r}}R_2,
        P_1 = P_2 \implies  \\
        &\implies R_1 \sqr{R_2 + r} = R_2 \sqr{R_1 + r} \implies  \\
        &\implies R_1 R_2^2 + 2 R_1 R_2 r + R_1 r^2 =
                    R_2 R_1^2 + 2 R_2 R_1 r + R_2 r^2  \implies  \\
    &\implies r^2 (R_2 - R_1) = R_2^2 R_2 - R_1^2 R_2 \implies  \\
    &\implies r
            = \sqrt{R_1 R_2 \frac{R_2 - R_1}{R_2 - R_1}}
            = \sqrt{R_1 R_2}
            = \sqrt{0{,}50\,\text{Ом} \cdot 4{,}50\,\text{Ом}}
            = 1{,}5\,\text{Ом}.
            \\
    \eta_1
            &= \frac{R_1}{R_1 + r}
            = \frac{\sqrt{R_1}}{\sqrt{R_1} + \sqrt{R_2}}
            = 0{,}250,  \\
    \eta_2
            &= \frac{R_2}{R_2 + r}
            = \frac{\sqrt{R_2}}{\sqrt{R_2} + \sqrt{R_1}}
            = 0{,}750
    \end{align*}
}

\variantsplitter

\addpersonalvariant{Михаил Ярошевский}

\tasknumber{1}%
\task{%
    Напротив физических величин укажите их обозначения и единицы измерения в СИ:
    \begin{enumerate}
        \item сила тока,
        \item мощность тока,
        \item удельное сопротивление,
        \item внутреннее сопротивление полной цепи.
    \end{enumerate}
}
\solutionspace{20pt}

\tasknumber{2}%
\task{%
    Запишите физический закон или формулу:
    \begin{enumerate}
        \item правило Кирхгофа для замкнутого контура,
        \item сопротивление резистора через удельное сопротивление,
        \item закон Ома для неоднородного участка цепи.
    \end{enumerate}
}
\solutionspace{40pt}

\tasknumber{3}%
\task{%
    На резистор сопротивлением $r = 30\,\text{Ом}$ подали напряжение $V = 180\,\text{В}$.
    Определите ток, который потечёт через резистор, и мощность, выделяющуюся на нём.
}
\answer{%
    \begin{align*}
    \eli &= \frac{V}{r} = \frac{180\,\text{В}}{30\,\text{Ом}} = 6\,\text{А},  \\
    P &= \frac{V^2}{r} = \frac{\sqr{180\,\text{В}}}{30\,\text{Ом}} = 1080\,\text{Вт}
    \end{align*}
}
\solutionspace{60pt}

\tasknumber{4}%
\task{%
    Через резистор сопротивлением $r = 12\,\text{Ом}$ протекает электрический ток $\eli = 3\,\text{А}$.
    Определите, чему равны напряжение на резисторе и мощность, выделяющаяся на нём.
}
\answer{%
    \begin{align*}
    U &= \eli r = 3\,\text{А} \cdot 12\,\text{Ом} = 36\,\text{В},  \\
    P &= \eli^2r = \sqr{3\,\text{А}} \cdot 12\,\text{Ом} = 108\,\text{Вт}
    \end{align*}
}
\solutionspace{60pt}

\tasknumber{5}%
\task{%
    Замкнутая электрическая цепь состоит из ЭДС $\ele = 4\,\text{В}$ и сопротивлением $r$
    и резистора $R = 15\,\text{Ом}$.
    Определите ток, протекающий в цепи.
    Какая тепловая энергия выделится на резисторе за время
    $\tau = 10\,\text{с}$? Какая работа будет совершена ЭДС за это время? Каков знак этой работы? Чему равен КПД цепи?
    Вычислите значения для 2 случаев: $r=0$ и $r = 30\,\text{Ом}$.
}
\answer{%
    \begin{align*}
    \eli_1 &= \frac{\ele}{R} = \frac{4\,\text{В}}{15\,\text{Ом}} = \frac4{15}\units{А} \approx 0{,}27\,\text{А},  \\
    \eli_2 &= \frac{\ele}{R + r} = \frac{4\,\text{В}}{15\,\text{Ом} + 30\,\text{Ом}} = \frac4{45}\units{А} \approx 0{,}09\,\text{А},  \\
    Q_1 &= \eli_1^2R\tau = \sqr{\frac{\ele}{R}} R \tau
            = \sqr{\frac{4\,\text{В}}{15\,\text{Ом}}} \cdot 15\,\text{Ом} \cdot 10\,\text{с} = \frac{32}3\units{Дж} \approx 10{,}667\,\text{Дж},  \\
    Q_2 &= \eli_2^2R\tau = \sqr{\frac{\ele}{R + r}} R \tau
            = \sqr{\frac{4\,\text{В}}{15\,\text{Ом} + 30\,\text{Ом}}} \cdot 15\,\text{Ом} \cdot 10\,\text{с} = \frac{32}{27}\units{Дж} \approx 1{,}185\,\text{Дж},  \\
    A_1 &= q_1\ele = \eli_1\tau\ele = \frac{\ele}{R} \tau \ele
            = \frac{\ele^2 \tau}{R} = \frac{\sqr{4\,\text{В}} \cdot 10\,\text{с}}{15\,\text{Ом}}
            = \frac{32}3\units{Дж} \approx 10{,}667\,\text{Дж}, \text{положительна},  \\
    A_2 &= q_2\ele = \eli_2\tau\ele = \frac{\ele}{R + r} \tau \ele
            = \frac{\ele^2 \tau}{R + r} = \frac{\sqr{4\,\text{В}} \cdot 10\,\text{с}}{15\,\text{Ом} + 30\,\text{Ом}}
            = \frac{32}9\units{Дж} \approx 3{,}556\,\text{Дж}, \text{положительна},  \\
    \eta_1 &= \frac{Q_1}{A_1} = \ldots = \frac{R}{R} = 1,  \\
    \eta_2 &= \frac{Q_2}{A_2} = \ldots = \frac{R}{R + r} = \frac13 \approx 0{,}33.
    \end{align*}
}
\solutionspace{180pt}

\tasknumber{6}%
\task{%
    Лампочки, сопротивления которых $R_1 = 0{,}50\,\text{Ом}$ и $R_2 = 18\,\text{Ом}$, поочерёдно подключённные к некоторому источнику тока,
    потребляют одинаковую мощность.
    Найти внутреннее сопротивление источника и КПД цепи в каждом случае.
}
\answer{%
    \begin{align*}
        P_1 &= \sqr{\frac{\ele}{R_1 + r}}R_1,
        P_2  = \sqr{\frac{\ele}{R_2 + r}}R_2,
        P_1 = P_2 \implies  \\
        &\implies R_1 \sqr{R_2 + r} = R_2 \sqr{R_1 + r} \implies  \\
        &\implies R_1 R_2^2 + 2 R_1 R_2 r + R_1 r^2 =
                    R_2 R_1^2 + 2 R_2 R_1 r + R_2 r^2  \implies  \\
    &\implies r^2 (R_2 - R_1) = R_2^2 R_2 - R_1^2 R_2 \implies  \\
    &\implies r
            = \sqrt{R_1 R_2 \frac{R_2 - R_1}{R_2 - R_1}}
            = \sqrt{R_1 R_2}
            = \sqrt{0{,}50\,\text{Ом} \cdot 18\,\text{Ом}}
            = 3\,\text{Ом}.
            \\
    \eta_1
            &= \frac{R_1}{R_1 + r}
            = \frac{\sqrt{R_1}}{\sqrt{R_1} + \sqrt{R_2}}
            = 0{,}143,  \\
    \eta_2
            &= \frac{R_2}{R_2 + r}
            = \frac{\sqrt{R_2}}{\sqrt{R_2} + \sqrt{R_1}}
            = 0{,}857
    \end{align*}
}

\variantsplitter

\addpersonalvariant{Алексей Алимпиев}

\tasknumber{1}%
\task{%
    Напротив физических величин укажите их обозначения и единицы измерения в СИ:
    \begin{enumerate}
        \item сила тока,
        \item работа тока,
        \item удельное сопротивление,
        \item внутреннее сопротивление полной цепи.
    \end{enumerate}
}
\solutionspace{20pt}

\tasknumber{2}%
\task{%
    Запишите физический закон или формулу:
    \begin{enumerate}
        \item правило Кирхгофа для узла цепи,
        \item закон Ома для однородного участка цепи,
        \item ЭДС (определение).
    \end{enumerate}
}
\solutionspace{40pt}

\tasknumber{3}%
\task{%
    На резистор сопротивлением $R = 18\,\text{Ом}$ подали напряжение $V = 150\,\text{В}$.
    Определите ток, который потечёт через резистор, и мощность, выделяющуюся на нём.
}
\answer{%
    \begin{align*}
    \eli &= \frac{V}{R} = \frac{150\,\text{В}}{18\,\text{Ом}} = 8{,}33\,\text{А},  \\
    P &= \frac{V^2}{R} = \frac{\sqr{150\,\text{В}}}{18\,\text{Ом}} = 1250\,\text{Вт}
    \end{align*}
}
\solutionspace{60pt}

\tasknumber{4}%
\task{%
    Через резистор сопротивлением $r = 30\,\text{Ом}$ протекает электрический ток $\eli = 5\,\text{А}$.
    Определите, чему равны напряжение на резисторе и мощность, выделяющаяся на нём.
}
\answer{%
    \begin{align*}
    U &= \eli r = 5\,\text{А} \cdot 30\,\text{Ом} = 150\,\text{В},  \\
    P &= \eli^2r = \sqr{5\,\text{А}} \cdot 30\,\text{Ом} = 750\,\text{Вт}
    \end{align*}
}
\solutionspace{60pt}

\tasknumber{5}%
\task{%
    Замкнутая электрическая цепь состоит из ЭДС $\ele = 2\,\text{В}$ и сопротивлением $r$
    и резистора $R = 10\,\text{Ом}$.
    Определите ток, протекающий в цепи.
    Какая тепловая энергия выделится на резисторе за время
    $\tau = 10\,\text{с}$? Какая работа будет совершена ЭДС за это время? Каков знак этой работы? Чему равен КПД цепи?
    Вычислите значения для 2 случаев: $r=0$ и $r = 20\,\text{Ом}$.
}
\answer{%
    \begin{align*}
    \eli_1 &= \frac{\ele}{R} = \frac{2\,\text{В}}{10\,\text{Ом}} = \frac15\units{А} \approx 0{,}20\,\text{А},  \\
    \eli_2 &= \frac{\ele}{R + r} = \frac{2\,\text{В}}{10\,\text{Ом} + 20\,\text{Ом}} = \frac1{15}\units{А} \approx 0{,}07\,\text{А},  \\
    Q_1 &= \eli_1^2R\tau = \sqr{\frac{\ele}{R}} R \tau
            = \sqr{\frac{2\,\text{В}}{10\,\text{Ом}}} \cdot 10\,\text{Ом} \cdot 10\,\text{с} = 4\units{Дж} \approx 4\,\text{Дж},  \\
    Q_2 &= \eli_2^2R\tau = \sqr{\frac{\ele}{R + r}} R \tau
            = \sqr{\frac{2\,\text{В}}{10\,\text{Ом} + 20\,\text{Ом}}} \cdot 10\,\text{Ом} \cdot 10\,\text{с} = \frac49\units{Дж} \approx 0{,}444\,\text{Дж},  \\
    A_1 &= q_1\ele = \eli_1\tau\ele = \frac{\ele}{R} \tau \ele
            = \frac{\ele^2 \tau}{R} = \frac{\sqr{2\,\text{В}} \cdot 10\,\text{с}}{10\,\text{Ом}}
            = 4\units{Дж} \approx 4\,\text{Дж}, \text{положительна},  \\
    A_2 &= q_2\ele = \eli_2\tau\ele = \frac{\ele}{R + r} \tau \ele
            = \frac{\ele^2 \tau}{R + r} = \frac{\sqr{2\,\text{В}} \cdot 10\,\text{с}}{10\,\text{Ом} + 20\,\text{Ом}}
            = \frac43\units{Дж} \approx 1{,}333\,\text{Дж}, \text{положительна},  \\
    \eta_1 &= \frac{Q_1}{A_1} = \ldots = \frac{R}{R} = 1,  \\
    \eta_2 &= \frac{Q_2}{A_2} = \ldots = \frac{R}{R + r} = \frac13 \approx 0{,}33.
    \end{align*}
}
\solutionspace{180pt}

\tasknumber{6}%
\task{%
    Лампочки, сопротивления которых $R_1 = 0{,}50\,\text{Ом}$ и $R_2 = 2\,\text{Ом}$, поочерёдно подключённные к некоторому источнику тока,
    потребляют одинаковую мощность.
    Найти внутреннее сопротивление источника и КПД цепи в каждом случае.
}
\answer{%
    \begin{align*}
        P_1 &= \sqr{\frac{\ele}{R_1 + r}}R_1,
        P_2  = \sqr{\frac{\ele}{R_2 + r}}R_2,
        P_1 = P_2 \implies  \\
        &\implies R_1 \sqr{R_2 + r} = R_2 \sqr{R_1 + r} \implies  \\
        &\implies R_1 R_2^2 + 2 R_1 R_2 r + R_1 r^2 =
                    R_2 R_1^2 + 2 R_2 R_1 r + R_2 r^2  \implies  \\
    &\implies r^2 (R_2 - R_1) = R_2^2 R_2 - R_1^2 R_2 \implies  \\
    &\implies r
            = \sqrt{R_1 R_2 \frac{R_2 - R_1}{R_2 - R_1}}
            = \sqrt{R_1 R_2}
            = \sqrt{0{,}50\,\text{Ом} \cdot 2\,\text{Ом}}
            = 1\,\text{Ом}.
            \\
    \eta_1
            &= \frac{R_1}{R_1 + r}
            = \frac{\sqrt{R_1}}{\sqrt{R_1} + \sqrt{R_2}}
            = 0{,}333,  \\
    \eta_2
            &= \frac{R_2}{R_2 + r}
            = \frac{\sqrt{R_2}}{\sqrt{R_2} + \sqrt{R_1}}
            = 0{,}667
    \end{align*}
}

\variantsplitter

\addpersonalvariant{Евгений Васин}

\tasknumber{1}%
\task{%
    Напротив физических величин укажите их обозначения и единицы измерения в СИ:
    \begin{enumerate}
        \item сила тока,
        \item работа тока,
        \item ЭДС,
        \item внутреннее сопротивление полной цепи.
    \end{enumerate}
}
\solutionspace{20pt}

\tasknumber{2}%
\task{%
    Запишите физический закон или формулу:
    \begin{enumerate}
        \item правило Кирхгофа для узла цепи,
        \item закон Ома для однородного участка цепи,
        \item закон Ома для неоднородного участка цепи.
    \end{enumerate}
}
\solutionspace{40pt}

\tasknumber{3}%
\task{%
    На резистор сопротивлением $R = 5\,\text{Ом}$ подали напряжение $U = 240\,\text{В}$.
    Определите ток, который потечёт через резистор, и мощность, выделяющуюся на нём.
}
\answer{%
    \begin{align*}
    \eli &= \frac{U}{R} = \frac{240\,\text{В}}{5\,\text{Ом}} = 48\,\text{А},  \\
    P &= \frac{U^2}{R} = \frac{\sqr{240\,\text{В}}}{5\,\text{Ом}} = 11520\,\text{Вт}
    \end{align*}
}
\solutionspace{60pt}

\tasknumber{4}%
\task{%
    Через резистор сопротивлением $R = 30\,\text{Ом}$ протекает электрический ток $\eli = 8\,\text{А}$.
    Определите, чему равны напряжение на резисторе и мощность, выделяющаяся на нём.
}
\answer{%
    \begin{align*}
    U &= \eli R = 8\,\text{А} \cdot 30\,\text{Ом} = 240\,\text{В},  \\
    P &= \eli^2R = \sqr{8\,\text{А}} \cdot 30\,\text{Ом} = 1920\,\text{Вт}
    \end{align*}
}
\solutionspace{60pt}

\tasknumber{5}%
\task{%
    Замкнутая электрическая цепь состоит из ЭДС $\ele = 4\,\text{В}$ и сопротивлением $r$
    и резистора $R = 30\,\text{Ом}$.
    Определите ток, протекающий в цепи.
    Какая тепловая энергия выделится на резисторе за время
    $\tau = 5\,\text{с}$? Какая работа будет совершена ЭДС за это время? Каков знак этой работы? Чему равен КПД цепи?
    Вычислите значения для 2 случаев: $r=0$ и $r = 60\,\text{Ом}$.
}
\answer{%
    \begin{align*}
    \eli_1 &= \frac{\ele}{R} = \frac{4\,\text{В}}{30\,\text{Ом}} = \frac2{15}\units{А} \approx 0{,}13\,\text{А},  \\
    \eli_2 &= \frac{\ele}{R + r} = \frac{4\,\text{В}}{30\,\text{Ом} + 60\,\text{Ом}} = \frac2{45}\units{А} \approx 0{,}04\,\text{А},  \\
    Q_1 &= \eli_1^2R\tau = \sqr{\frac{\ele}{R}} R \tau
            = \sqr{\frac{4\,\text{В}}{30\,\text{Ом}}} \cdot 30\,\text{Ом} \cdot 5\,\text{с} = \frac83\units{Дж} \approx 2{,}667\,\text{Дж},  \\
    Q_2 &= \eli_2^2R\tau = \sqr{\frac{\ele}{R + r}} R \tau
            = \sqr{\frac{4\,\text{В}}{30\,\text{Ом} + 60\,\text{Ом}}} \cdot 30\,\text{Ом} \cdot 5\,\text{с} = \frac8{27}\units{Дж} \approx 0{,}296\,\text{Дж},  \\
    A_1 &= q_1\ele = \eli_1\tau\ele = \frac{\ele}{R} \tau \ele
            = \frac{\ele^2 \tau}{R} = \frac{\sqr{4\,\text{В}} \cdot 5\,\text{с}}{30\,\text{Ом}}
            = \frac83\units{Дж} \approx 2{,}667\,\text{Дж}, \text{положительна},  \\
    A_2 &= q_2\ele = \eli_2\tau\ele = \frac{\ele}{R + r} \tau \ele
            = \frac{\ele^2 \tau}{R + r} = \frac{\sqr{4\,\text{В}} \cdot 5\,\text{с}}{30\,\text{Ом} + 60\,\text{Ом}}
            = \frac89\units{Дж} \approx 0{,}889\,\text{Дж}, \text{положительна},  \\
    \eta_1 &= \frac{Q_1}{A_1} = \ldots = \frac{R}{R} = 1,  \\
    \eta_2 &= \frac{Q_2}{A_2} = \ldots = \frac{R}{R + r} = \frac13 \approx 0{,}33.
    \end{align*}
}
\solutionspace{180pt}

\tasknumber{6}%
\task{%
    Лампочки, сопротивления которых $R_1 = 0{,}50\,\text{Ом}$ и $R_2 = 4{,}50\,\text{Ом}$, поочерёдно подключённные к некоторому источнику тока,
    потребляют одинаковую мощность.
    Найти внутреннее сопротивление источника и КПД цепи в каждом случае.
}
\answer{%
    \begin{align*}
        P_1 &= \sqr{\frac{\ele}{R_1 + r}}R_1,
        P_2  = \sqr{\frac{\ele}{R_2 + r}}R_2,
        P_1 = P_2 \implies  \\
        &\implies R_1 \sqr{R_2 + r} = R_2 \sqr{R_1 + r} \implies  \\
        &\implies R_1 R_2^2 + 2 R_1 R_2 r + R_1 r^2 =
                    R_2 R_1^2 + 2 R_2 R_1 r + R_2 r^2  \implies  \\
    &\implies r^2 (R_2 - R_1) = R_2^2 R_2 - R_1^2 R_2 \implies  \\
    &\implies r
            = \sqrt{R_1 R_2 \frac{R_2 - R_1}{R_2 - R_1}}
            = \sqrt{R_1 R_2}
            = \sqrt{0{,}50\,\text{Ом} \cdot 4{,}50\,\text{Ом}}
            = 1{,}5\,\text{Ом}.
            \\
    \eta_1
            &= \frac{R_1}{R_1 + r}
            = \frac{\sqrt{R_1}}{\sqrt{R_1} + \sqrt{R_2}}
            = 0{,}250,  \\
    \eta_2
            &= \frac{R_2}{R_2 + r}
            = \frac{\sqrt{R_2}}{\sqrt{R_2} + \sqrt{R_1}}
            = 0{,}750
    \end{align*}
}

\variantsplitter

\addpersonalvariant{Вячеслав Волохов}

\tasknumber{1}%
\task{%
    Напротив физических величин укажите их обозначения и единицы измерения в СИ:
    \begin{enumerate}
        \item разность потенциалов,
        \item мощность тока,
        \item удельное сопротивление,
        \item внутреннее сопротивление полной цепи.
    \end{enumerate}
}
\solutionspace{20pt}

\tasknumber{2}%
\task{%
    Запишите физический закон или формулу:
    \begin{enumerate}
        \item правило Кирхгофа для узла цепи,
        \item сопротивление резистора через удельное сопротивление,
        \item закон Ома для неоднородного участка цепи.
    \end{enumerate}
}
\solutionspace{40pt}

\tasknumber{3}%
\task{%
    На резистор сопротивлением $r = 30\,\text{Ом}$ подали напряжение $V = 120\,\text{В}$.
    Определите ток, который потечёт через резистор, и мощность, выделяющуюся на нём.
}
\answer{%
    \begin{align*}
    \eli &= \frac{V}{r} = \frac{120\,\text{В}}{30\,\text{Ом}} = 4\,\text{А},  \\
    P &= \frac{V^2}{r} = \frac{\sqr{120\,\text{В}}}{30\,\text{Ом}} = 480\,\text{Вт}
    \end{align*}
}
\solutionspace{60pt}

\tasknumber{4}%
\task{%
    Через резистор сопротивлением $r = 5\,\text{Ом}$ протекает электрический ток $\eli = 15\,\text{А}$.
    Определите, чему равны напряжение на резисторе и мощность, выделяющаяся на нём.
}
\answer{%
    \begin{align*}
    U &= \eli r = 15\,\text{А} \cdot 5\,\text{Ом} = 75\,\text{В},  \\
    P &= \eli^2r = \sqr{15\,\text{А}} \cdot 5\,\text{Ом} = 1125\,\text{Вт}
    \end{align*}
}
\solutionspace{60pt}

\tasknumber{5}%
\task{%
    Замкнутая электрическая цепь состоит из ЭДС $\ele = 2\,\text{В}$ и сопротивлением $r$
    и резистора $R = 24\,\text{Ом}$.
    Определите ток, протекающий в цепи.
    Какая тепловая энергия выделится на резисторе за время
    $\tau = 5\,\text{с}$? Какая работа будет совершена ЭДС за это время? Каков знак этой работы? Чему равен КПД цепи?
    Вычислите значения для 2 случаев: $r=0$ и $r = 10\,\text{Ом}$.
}
\answer{%
    \begin{align*}
    \eli_1 &= \frac{\ele}{R} = \frac{2\,\text{В}}{24\,\text{Ом}} = \frac1{12}\units{А} \approx 0{,}08\,\text{А},  \\
    \eli_2 &= \frac{\ele}{R + r} = \frac{2\,\text{В}}{24\,\text{Ом} + 10\,\text{Ом}} = \frac1{17}\units{А} \approx 0{,}06\,\text{А},  \\
    Q_1 &= \eli_1^2R\tau = \sqr{\frac{\ele}{R}} R \tau
            = \sqr{\frac{2\,\text{В}}{24\,\text{Ом}}} \cdot 24\,\text{Ом} \cdot 5\,\text{с} = \frac56\units{Дж} \approx 0{,}833\,\text{Дж},  \\
    Q_2 &= \eli_2^2R\tau = \sqr{\frac{\ele}{R + r}} R \tau
            = \sqr{\frac{2\,\text{В}}{24\,\text{Ом} + 10\,\text{Ом}}} \cdot 24\,\text{Ом} \cdot 5\,\text{с} = \frac{120}{289}\units{Дж} \approx 0{,}415\,\text{Дж},  \\
    A_1 &= q_1\ele = \eli_1\tau\ele = \frac{\ele}{R} \tau \ele
            = \frac{\ele^2 \tau}{R} = \frac{\sqr{2\,\text{В}} \cdot 5\,\text{с}}{24\,\text{Ом}}
            = \frac56\units{Дж} \approx 0{,}833\,\text{Дж}, \text{положительна},  \\
    A_2 &= q_2\ele = \eli_2\tau\ele = \frac{\ele}{R + r} \tau \ele
            = \frac{\ele^2 \tau}{R + r} = \frac{\sqr{2\,\text{В}} \cdot 5\,\text{с}}{24\,\text{Ом} + 10\,\text{Ом}}
            = \frac{10}{17}\units{Дж} \approx 0{,}588\,\text{Дж}, \text{положительна},  \\
    \eta_1 &= \frac{Q_1}{A_1} = \ldots = \frac{R}{R} = 1,  \\
    \eta_2 &= \frac{Q_2}{A_2} = \ldots = \frac{R}{R + r} = \frac{12}{17} \approx 0{,}71.
    \end{align*}
}
\solutionspace{180pt}

\tasknumber{6}%
\task{%
    Лампочки, сопротивления которых $R_1 = 0{,}50\,\text{Ом}$ и $R_2 = 2\,\text{Ом}$, поочерёдно подключённные к некоторому источнику тока,
    потребляют одинаковую мощность.
    Найти внутреннее сопротивление источника и КПД цепи в каждом случае.
}
\answer{%
    \begin{align*}
        P_1 &= \sqr{\frac{\ele}{R_1 + r}}R_1,
        P_2  = \sqr{\frac{\ele}{R_2 + r}}R_2,
        P_1 = P_2 \implies  \\
        &\implies R_1 \sqr{R_2 + r} = R_2 \sqr{R_1 + r} \implies  \\
        &\implies R_1 R_2^2 + 2 R_1 R_2 r + R_1 r^2 =
                    R_2 R_1^2 + 2 R_2 R_1 r + R_2 r^2  \implies  \\
    &\implies r^2 (R_2 - R_1) = R_2^2 R_2 - R_1^2 R_2 \implies  \\
    &\implies r
            = \sqrt{R_1 R_2 \frac{R_2 - R_1}{R_2 - R_1}}
            = \sqrt{R_1 R_2}
            = \sqrt{0{,}50\,\text{Ом} \cdot 2\,\text{Ом}}
            = 1\,\text{Ом}.
            \\
    \eta_1
            &= \frac{R_1}{R_1 + r}
            = \frac{\sqrt{R_1}}{\sqrt{R_1} + \sqrt{R_2}}
            = 0{,}333,  \\
    \eta_2
            &= \frac{R_2}{R_2 + r}
            = \frac{\sqrt{R_2}}{\sqrt{R_2} + \sqrt{R_1}}
            = 0{,}667
    \end{align*}
}

\variantsplitter

\addpersonalvariant{Герман Говоров}

\tasknumber{1}%
\task{%
    Напротив физических величин укажите их обозначения и единицы измерения в СИ:
    \begin{enumerate}
        \item напряжение,
        \item работа тока,
        \item удельное сопротивление,
        \item внешнее сопротивление полной цепи.
    \end{enumerate}
}
\solutionspace{20pt}

\tasknumber{2}%
\task{%
    Запишите физический закон или формулу:
    \begin{enumerate}
        \item правило Кирхгофа для замкнутого контура,
        \item закон Ома для однородного участка цепи,
        \item ЭДС (определение).
    \end{enumerate}
}
\solutionspace{40pt}

\tasknumber{3}%
\task{%
    На резистор сопротивлением $r = 5\,\text{Ом}$ подали напряжение $V = 240\,\text{В}$.
    Определите ток, который потечёт через резистор, и мощность, выделяющуюся на нём.
}
\answer{%
    \begin{align*}
    \eli &= \frac{V}{r} = \frac{240\,\text{В}}{5\,\text{Ом}} = 48\,\text{А},  \\
    P &= \frac{V^2}{r} = \frac{\sqr{240\,\text{В}}}{5\,\text{Ом}} = 11520\,\text{Вт}
    \end{align*}
}
\solutionspace{60pt}

\tasknumber{4}%
\task{%
    Через резистор сопротивлением $R = 30\,\text{Ом}$ протекает электрический ток $\eli = 10\,\text{А}$.
    Определите, чему равны напряжение на резисторе и мощность, выделяющаяся на нём.
}
\answer{%
    \begin{align*}
    U &= \eli R = 10\,\text{А} \cdot 30\,\text{Ом} = 300\,\text{В},  \\
    P &= \eli^2R = \sqr{10\,\text{А}} \cdot 30\,\text{Ом} = 3000\,\text{Вт}
    \end{align*}
}
\solutionspace{60pt}

\tasknumber{5}%
\task{%
    Замкнутая электрическая цепь состоит из ЭДС $\ele = 2\,\text{В}$ и сопротивлением $r$
    и резистора $R = 30\,\text{Ом}$.
    Определите ток, протекающий в цепи.
    Какая тепловая энергия выделится на резисторе за время
    $\tau = 2\,\text{с}$? Какая работа будет совершена ЭДС за это время? Каков знак этой работы? Чему равен КПД цепи?
    Вычислите значения для 2 случаев: $r=0$ и $r = 60\,\text{Ом}$.
}
\answer{%
    \begin{align*}
    \eli_1 &= \frac{\ele}{R} = \frac{2\,\text{В}}{30\,\text{Ом}} = \frac1{15}\units{А} \approx 0{,}07\,\text{А},  \\
    \eli_2 &= \frac{\ele}{R + r} = \frac{2\,\text{В}}{30\,\text{Ом} + 60\,\text{Ом}} = \frac1{45}\units{А} \approx 0{,}02\,\text{А},  \\
    Q_1 &= \eli_1^2R\tau = \sqr{\frac{\ele}{R}} R \tau
            = \sqr{\frac{2\,\text{В}}{30\,\text{Ом}}} \cdot 30\,\text{Ом} \cdot 2\,\text{с} = \frac4{15}\units{Дж} \approx 0{,}267\,\text{Дж},  \\
    Q_2 &= \eli_2^2R\tau = \sqr{\frac{\ele}{R + r}} R \tau
            = \sqr{\frac{2\,\text{В}}{30\,\text{Ом} + 60\,\text{Ом}}} \cdot 30\,\text{Ом} \cdot 2\,\text{с} = \frac4{135}\units{Дж} \approx 0{,}030\,\text{Дж},  \\
    A_1 &= q_1\ele = \eli_1\tau\ele = \frac{\ele}{R} \tau \ele
            = \frac{\ele^2 \tau}{R} = \frac{\sqr{2\,\text{В}} \cdot 2\,\text{с}}{30\,\text{Ом}}
            = \frac4{15}\units{Дж} \approx 0{,}267\,\text{Дж}, \text{положительна},  \\
    A_2 &= q_2\ele = \eli_2\tau\ele = \frac{\ele}{R + r} \tau \ele
            = \frac{\ele^2 \tau}{R + r} = \frac{\sqr{2\,\text{В}} \cdot 2\,\text{с}}{30\,\text{Ом} + 60\,\text{Ом}}
            = \frac4{45}\units{Дж} \approx 0{,}089\,\text{Дж}, \text{положительна},  \\
    \eta_1 &= \frac{Q_1}{A_1} = \ldots = \frac{R}{R} = 1,  \\
    \eta_2 &= \frac{Q_2}{A_2} = \ldots = \frac{R}{R + r} = \frac13 \approx 0{,}33.
    \end{align*}
}
\solutionspace{180pt}

\tasknumber{6}%
\task{%
    Лампочки, сопротивления которых $R_1 = 0{,}50\,\text{Ом}$ и $R_2 = 18\,\text{Ом}$, поочерёдно подключённные к некоторому источнику тока,
    потребляют одинаковую мощность.
    Найти внутреннее сопротивление источника и КПД цепи в каждом случае.
}
\answer{%
    \begin{align*}
        P_1 &= \sqr{\frac{\ele}{R_1 + r}}R_1,
        P_2  = \sqr{\frac{\ele}{R_2 + r}}R_2,
        P_1 = P_2 \implies  \\
        &\implies R_1 \sqr{R_2 + r} = R_2 \sqr{R_1 + r} \implies  \\
        &\implies R_1 R_2^2 + 2 R_1 R_2 r + R_1 r^2 =
                    R_2 R_1^2 + 2 R_2 R_1 r + R_2 r^2  \implies  \\
    &\implies r^2 (R_2 - R_1) = R_2^2 R_2 - R_1^2 R_2 \implies  \\
    &\implies r
            = \sqrt{R_1 R_2 \frac{R_2 - R_1}{R_2 - R_1}}
            = \sqrt{R_1 R_2}
            = \sqrt{0{,}50\,\text{Ом} \cdot 18\,\text{Ом}}
            = 3\,\text{Ом}.
            \\
    \eta_1
            &= \frac{R_1}{R_1 + r}
            = \frac{\sqrt{R_1}}{\sqrt{R_1} + \sqrt{R_2}}
            = 0{,}143,  \\
    \eta_2
            &= \frac{R_2}{R_2 + r}
            = \frac{\sqrt{R_2}}{\sqrt{R_2} + \sqrt{R_1}}
            = 0{,}857
    \end{align*}
}

\variantsplitter

\addpersonalvariant{София Журавлёва}

\tasknumber{1}%
\task{%
    Напротив физических величин укажите их обозначения и единицы измерения в СИ:
    \begin{enumerate}
        \item сила тока,
        \item мощность тока,
        \item ЭДС,
        \item внутреннее сопротивление полной цепи.
    \end{enumerate}
}
\solutionspace{20pt}

\tasknumber{2}%
\task{%
    Запишите физический закон или формулу:
    \begin{enumerate}
        \item правило Кирхгофа для узла цепи,
        \item сопротивление резистора через удельное сопротивление,
        \item ЭДС (определение).
    \end{enumerate}
}
\solutionspace{40pt}

\tasknumber{3}%
\task{%
    На резистор сопротивлением $r = 18\,\text{Ом}$ подали напряжение $U = 240\,\text{В}$.
    Определите ток, который потечёт через резистор, и мощность, выделяющуюся на нём.
}
\answer{%
    \begin{align*}
    \eli &= \frac{U}{r} = \frac{240\,\text{В}}{18\,\text{Ом}} = 13{,}33\,\text{А},  \\
    P &= \frac{U^2}{r} = \frac{\sqr{240\,\text{В}}}{18\,\text{Ом}} = 3200\,\text{Вт}
    \end{align*}
}
\solutionspace{60pt}

\tasknumber{4}%
\task{%
    Через резистор сопротивлением $r = 30\,\text{Ом}$ протекает электрический ток $\eli = 5\,\text{А}$.
    Определите, чему равны напряжение на резисторе и мощность, выделяющаяся на нём.
}
\answer{%
    \begin{align*}
    U &= \eli r = 5\,\text{А} \cdot 30\,\text{Ом} = 150\,\text{В},  \\
    P &= \eli^2r = \sqr{5\,\text{А}} \cdot 30\,\text{Ом} = 750\,\text{Вт}
    \end{align*}
}
\solutionspace{60pt}

\tasknumber{5}%
\task{%
    Замкнутая электрическая цепь состоит из ЭДС $\ele = 4\,\text{В}$ и сопротивлением $r$
    и резистора $R = 24\,\text{Ом}$.
    Определите ток, протекающий в цепи.
    Какая тепловая энергия выделится на резисторе за время
    $\tau = 5\,\text{с}$? Какая работа будет совершена ЭДС за это время? Каков знак этой работы? Чему равен КПД цепи?
    Вычислите значения для 2 случаев: $r=0$ и $r = 30\,\text{Ом}$.
}
\answer{%
    \begin{align*}
    \eli_1 &= \frac{\ele}{R} = \frac{4\,\text{В}}{24\,\text{Ом}} = \frac16\units{А} \approx 0{,}17\,\text{А},  \\
    \eli_2 &= \frac{\ele}{R + r} = \frac{4\,\text{В}}{24\,\text{Ом} + 30\,\text{Ом}} = \frac2{27}\units{А} \approx 0{,}07\,\text{А},  \\
    Q_1 &= \eli_1^2R\tau = \sqr{\frac{\ele}{R}} R \tau
            = \sqr{\frac{4\,\text{В}}{24\,\text{Ом}}} \cdot 24\,\text{Ом} \cdot 5\,\text{с} = \frac{10}3\units{Дж} \approx 3{,}333\,\text{Дж},  \\
    Q_2 &= \eli_2^2R\tau = \sqr{\frac{\ele}{R + r}} R \tau
            = \sqr{\frac{4\,\text{В}}{24\,\text{Ом} + 30\,\text{Ом}}} \cdot 24\,\text{Ом} \cdot 5\,\text{с} = \frac{160}{243}\units{Дж} \approx 0{,}658\,\text{Дж},  \\
    A_1 &= q_1\ele = \eli_1\tau\ele = \frac{\ele}{R} \tau \ele
            = \frac{\ele^2 \tau}{R} = \frac{\sqr{4\,\text{В}} \cdot 5\,\text{с}}{24\,\text{Ом}}
            = \frac{10}3\units{Дж} \approx 3{,}333\,\text{Дж}, \text{положительна},  \\
    A_2 &= q_2\ele = \eli_2\tau\ele = \frac{\ele}{R + r} \tau \ele
            = \frac{\ele^2 \tau}{R + r} = \frac{\sqr{4\,\text{В}} \cdot 5\,\text{с}}{24\,\text{Ом} + 30\,\text{Ом}}
            = \frac{40}{27}\units{Дж} \approx 1{,}481\,\text{Дж}, \text{положительна},  \\
    \eta_1 &= \frac{Q_1}{A_1} = \ldots = \frac{R}{R} = 1,  \\
    \eta_2 &= \frac{Q_2}{A_2} = \ldots = \frac{R}{R + r} = \frac49 \approx 0{,}44.
    \end{align*}
}
\solutionspace{180pt}

\tasknumber{6}%
\task{%
    Лампочки, сопротивления которых $R_1 = 0{,}50\,\text{Ом}$ и $R_2 = 2\,\text{Ом}$, поочерёдно подключённные к некоторому источнику тока,
    потребляют одинаковую мощность.
    Найти внутреннее сопротивление источника и КПД цепи в каждом случае.
}
\answer{%
    \begin{align*}
        P_1 &= \sqr{\frac{\ele}{R_1 + r}}R_1,
        P_2  = \sqr{\frac{\ele}{R_2 + r}}R_2,
        P_1 = P_2 \implies  \\
        &\implies R_1 \sqr{R_2 + r} = R_2 \sqr{R_1 + r} \implies  \\
        &\implies R_1 R_2^2 + 2 R_1 R_2 r + R_1 r^2 =
                    R_2 R_1^2 + 2 R_2 R_1 r + R_2 r^2  \implies  \\
    &\implies r^2 (R_2 - R_1) = R_2^2 R_2 - R_1^2 R_2 \implies  \\
    &\implies r
            = \sqrt{R_1 R_2 \frac{R_2 - R_1}{R_2 - R_1}}
            = \sqrt{R_1 R_2}
            = \sqrt{0{,}50\,\text{Ом} \cdot 2\,\text{Ом}}
            = 1\,\text{Ом}.
            \\
    \eta_1
            &= \frac{R_1}{R_1 + r}
            = \frac{\sqrt{R_1}}{\sqrt{R_1} + \sqrt{R_2}}
            = 0{,}333,  \\
    \eta_2
            &= \frac{R_2}{R_2 + r}
            = \frac{\sqrt{R_2}}{\sqrt{R_2} + \sqrt{R_1}}
            = 0{,}667
    \end{align*}
}

\variantsplitter

\addpersonalvariant{Константин Козлов}

\tasknumber{1}%
\task{%
    Напротив физических величин укажите их обозначения и единицы измерения в СИ:
    \begin{enumerate}
        \item разность потенциалов,
        \item мощность тока,
        \item удельное сопротивление,
        \item внешнее сопротивление полной цепи.
    \end{enumerate}
}
\solutionspace{20pt}

\tasknumber{2}%
\task{%
    Запишите физический закон или формулу:
    \begin{enumerate}
        \item правило Кирхгофа для узла цепи,
        \item сопротивление резистора через удельное сопротивление,
        \item закон Ома для неоднородного участка цепи.
    \end{enumerate}
}
\solutionspace{40pt}

\tasknumber{3}%
\task{%
    На резистор сопротивлением $r = 12\,\text{Ом}$ подали напряжение $U = 150\,\text{В}$.
    Определите ток, который потечёт через резистор, и мощность, выделяющуюся на нём.
}
\answer{%
    \begin{align*}
    \eli &= \frac{U}{r} = \frac{150\,\text{В}}{12\,\text{Ом}} = 12{,}50\,\text{А},  \\
    P &= \frac{U^2}{r} = \frac{\sqr{150\,\text{В}}}{12\,\text{Ом}} = 1875\,\text{Вт}
    \end{align*}
}
\solutionspace{60pt}

\tasknumber{4}%
\task{%
    Через резистор сопротивлением $R = 30\,\text{Ом}$ протекает электрический ток $\eli = 10\,\text{А}$.
    Определите, чему равны напряжение на резисторе и мощность, выделяющаяся на нём.
}
\answer{%
    \begin{align*}
    U &= \eli R = 10\,\text{А} \cdot 30\,\text{Ом} = 300\,\text{В},  \\
    P &= \eli^2R = \sqr{10\,\text{А}} \cdot 30\,\text{Ом} = 3000\,\text{Вт}
    \end{align*}
}
\solutionspace{60pt}

\tasknumber{5}%
\task{%
    Замкнутая электрическая цепь состоит из ЭДС $\ele = 3\,\text{В}$ и сопротивлением $r$
    и резистора $R = 24\,\text{Ом}$.
    Определите ток, протекающий в цепи.
    Какая тепловая энергия выделится на резисторе за время
    $\tau = 2\,\text{с}$? Какая работа будет совершена ЭДС за это время? Каков знак этой работы? Чему равен КПД цепи?
    Вычислите значения для 2 случаев: $r=0$ и $r = 30\,\text{Ом}$.
}
\answer{%
    \begin{align*}
    \eli_1 &= \frac{\ele}{R} = \frac{3\,\text{В}}{24\,\text{Ом}} = \frac18\units{А} \approx 0{,}12\,\text{А},  \\
    \eli_2 &= \frac{\ele}{R + r} = \frac{3\,\text{В}}{24\,\text{Ом} + 30\,\text{Ом}} = \frac1{18}\units{А} \approx 0{,}06\,\text{А},  \\
    Q_1 &= \eli_1^2R\tau = \sqr{\frac{\ele}{R}} R \tau
            = \sqr{\frac{3\,\text{В}}{24\,\text{Ом}}} \cdot 24\,\text{Ом} \cdot 2\,\text{с} = \frac34\units{Дж} \approx 0{,}750\,\text{Дж},  \\
    Q_2 &= \eli_2^2R\tau = \sqr{\frac{\ele}{R + r}} R \tau
            = \sqr{\frac{3\,\text{В}}{24\,\text{Ом} + 30\,\text{Ом}}} \cdot 24\,\text{Ом} \cdot 2\,\text{с} = \frac4{27}\units{Дж} \approx 0{,}148\,\text{Дж},  \\
    A_1 &= q_1\ele = \eli_1\tau\ele = \frac{\ele}{R} \tau \ele
            = \frac{\ele^2 \tau}{R} = \frac{\sqr{3\,\text{В}} \cdot 2\,\text{с}}{24\,\text{Ом}}
            = \frac34\units{Дж} \approx 0{,}750\,\text{Дж}, \text{положительна},  \\
    A_2 &= q_2\ele = \eli_2\tau\ele = \frac{\ele}{R + r} \tau \ele
            = \frac{\ele^2 \tau}{R + r} = \frac{\sqr{3\,\text{В}} \cdot 2\,\text{с}}{24\,\text{Ом} + 30\,\text{Ом}}
            = \frac13\units{Дж} \approx 0{,}333\,\text{Дж}, \text{положительна},  \\
    \eta_1 &= \frac{Q_1}{A_1} = \ldots = \frac{R}{R} = 1,  \\
    \eta_2 &= \frac{Q_2}{A_2} = \ldots = \frac{R}{R + r} = \frac49 \approx 0{,}44.
    \end{align*}
}
\solutionspace{180pt}

\tasknumber{6}%
\task{%
    Лампочки, сопротивления которых $R_1 = 4\,\text{Ом}$ и $R_2 = 36\,\text{Ом}$, поочерёдно подключённные к некоторому источнику тока,
    потребляют одинаковую мощность.
    Найти внутреннее сопротивление источника и КПД цепи в каждом случае.
}
\answer{%
    \begin{align*}
        P_1 &= \sqr{\frac{\ele}{R_1 + r}}R_1,
        P_2  = \sqr{\frac{\ele}{R_2 + r}}R_2,
        P_1 = P_2 \implies  \\
        &\implies R_1 \sqr{R_2 + r} = R_2 \sqr{R_1 + r} \implies  \\
        &\implies R_1 R_2^2 + 2 R_1 R_2 r + R_1 r^2 =
                    R_2 R_1^2 + 2 R_2 R_1 r + R_2 r^2  \implies  \\
    &\implies r^2 (R_2 - R_1) = R_2^2 R_2 - R_1^2 R_2 \implies  \\
    &\implies r
            = \sqrt{R_1 R_2 \frac{R_2 - R_1}{R_2 - R_1}}
            = \sqrt{R_1 R_2}
            = \sqrt{4\,\text{Ом} \cdot 36\,\text{Ом}}
            = 12\,\text{Ом}.
            \\
    \eta_1
            &= \frac{R_1}{R_1 + r}
            = \frac{\sqrt{R_1}}{\sqrt{R_1} + \sqrt{R_2}}
            = 0{,}250,  \\
    \eta_2
            &= \frac{R_2}{R_2 + r}
            = \frac{\sqrt{R_2}}{\sqrt{R_2} + \sqrt{R_1}}
            = 0{,}750
    \end{align*}
}

\variantsplitter

\addpersonalvariant{Наталья Кравченко}

\tasknumber{1}%
\task{%
    Напротив физических величин укажите их обозначения и единицы измерения в СИ:
    \begin{enumerate}
        \item напряжение,
        \item работа тока,
        \item ЭДС,
        \item внутреннее сопротивление полной цепи.
    \end{enumerate}
}
\solutionspace{20pt}

\tasknumber{2}%
\task{%
    Запишите физический закон или формулу:
    \begin{enumerate}
        \item правило Кирхгофа для замкнутого контура,
        \item сопротивление резистора через удельное сопротивление,
        \item ЭДС (определение).
    \end{enumerate}
}
\solutionspace{40pt}

\tasknumber{3}%
\task{%
    На резистор сопротивлением $R = 30\,\text{Ом}$ подали напряжение $U = 180\,\text{В}$.
    Определите ток, который потечёт через резистор, и мощность, выделяющуюся на нём.
}
\answer{%
    \begin{align*}
    \eli &= \frac{U}{R} = \frac{180\,\text{В}}{30\,\text{Ом}} = 6\,\text{А},  \\
    P &= \frac{U^2}{R} = \frac{\sqr{180\,\text{В}}}{30\,\text{Ом}} = 1080\,\text{Вт}
    \end{align*}
}
\solutionspace{60pt}

\tasknumber{4}%
\task{%
    Через резистор сопротивлением $R = 5\,\text{Ом}$ протекает электрический ток $\eli = 5\,\text{А}$.
    Определите, чему равны напряжение на резисторе и мощность, выделяющаяся на нём.
}
\answer{%
    \begin{align*}
    U &= \eli R = 5\,\text{А} \cdot 5\,\text{Ом} = 25\,\text{В},  \\
    P &= \eli^2R = \sqr{5\,\text{А}} \cdot 5\,\text{Ом} = 125\,\text{Вт}
    \end{align*}
}
\solutionspace{60pt}

\tasknumber{5}%
\task{%
    Замкнутая электрическая цепь состоит из ЭДС $\ele = 2\,\text{В}$ и сопротивлением $r$
    и резистора $R = 15\,\text{Ом}$.
    Определите ток, протекающий в цепи.
    Какая тепловая энергия выделится на резисторе за время
    $\tau = 10\,\text{с}$? Какая работа будет совершена ЭДС за это время? Каков знак этой работы? Чему равен КПД цепи?
    Вычислите значения для 2 случаев: $r=0$ и $r = 30\,\text{Ом}$.
}
\answer{%
    \begin{align*}
    \eli_1 &= \frac{\ele}{R} = \frac{2\,\text{В}}{15\,\text{Ом}} = \frac2{15}\units{А} \approx 0{,}13\,\text{А},  \\
    \eli_2 &= \frac{\ele}{R + r} = \frac{2\,\text{В}}{15\,\text{Ом} + 30\,\text{Ом}} = \frac2{45}\units{А} \approx 0{,}04\,\text{А},  \\
    Q_1 &= \eli_1^2R\tau = \sqr{\frac{\ele}{R}} R \tau
            = \sqr{\frac{2\,\text{В}}{15\,\text{Ом}}} \cdot 15\,\text{Ом} \cdot 10\,\text{с} = \frac83\units{Дж} \approx 2{,}667\,\text{Дж},  \\
    Q_2 &= \eli_2^2R\tau = \sqr{\frac{\ele}{R + r}} R \tau
            = \sqr{\frac{2\,\text{В}}{15\,\text{Ом} + 30\,\text{Ом}}} \cdot 15\,\text{Ом} \cdot 10\,\text{с} = \frac8{27}\units{Дж} \approx 0{,}296\,\text{Дж},  \\
    A_1 &= q_1\ele = \eli_1\tau\ele = \frac{\ele}{R} \tau \ele
            = \frac{\ele^2 \tau}{R} = \frac{\sqr{2\,\text{В}} \cdot 10\,\text{с}}{15\,\text{Ом}}
            = \frac83\units{Дж} \approx 2{,}667\,\text{Дж}, \text{положительна},  \\
    A_2 &= q_2\ele = \eli_2\tau\ele = \frac{\ele}{R + r} \tau \ele
            = \frac{\ele^2 \tau}{R + r} = \frac{\sqr{2\,\text{В}} \cdot 10\,\text{с}}{15\,\text{Ом} + 30\,\text{Ом}}
            = \frac89\units{Дж} \approx 0{,}889\,\text{Дж}, \text{положительна},  \\
    \eta_1 &= \frac{Q_1}{A_1} = \ldots = \frac{R}{R} = 1,  \\
    \eta_2 &= \frac{Q_2}{A_2} = \ldots = \frac{R}{R + r} = \frac13 \approx 0{,}33.
    \end{align*}
}
\solutionspace{180pt}

\tasknumber{6}%
\task{%
    Лампочки, сопротивления которых $R_1 = 6\,\text{Ом}$ и $R_2 = 24\,\text{Ом}$, поочерёдно подключённные к некоторому источнику тока,
    потребляют одинаковую мощность.
    Найти внутреннее сопротивление источника и КПД цепи в каждом случае.
}
\answer{%
    \begin{align*}
        P_1 &= \sqr{\frac{\ele}{R_1 + r}}R_1,
        P_2  = \sqr{\frac{\ele}{R_2 + r}}R_2,
        P_1 = P_2 \implies  \\
        &\implies R_1 \sqr{R_2 + r} = R_2 \sqr{R_1 + r} \implies  \\
        &\implies R_1 R_2^2 + 2 R_1 R_2 r + R_1 r^2 =
                    R_2 R_1^2 + 2 R_2 R_1 r + R_2 r^2  \implies  \\
    &\implies r^2 (R_2 - R_1) = R_2^2 R_2 - R_1^2 R_2 \implies  \\
    &\implies r
            = \sqrt{R_1 R_2 \frac{R_2 - R_1}{R_2 - R_1}}
            = \sqrt{R_1 R_2}
            = \sqrt{6\,\text{Ом} \cdot 24\,\text{Ом}}
            = 12\,\text{Ом}.
            \\
    \eta_1
            &= \frac{R_1}{R_1 + r}
            = \frac{\sqrt{R_1}}{\sqrt{R_1} + \sqrt{R_2}}
            = 0{,}333,  \\
    \eta_2
            &= \frac{R_2}{R_2 + r}
            = \frac{\sqrt{R_2}}{\sqrt{R_2} + \sqrt{R_1}}
            = 0{,}667
    \end{align*}
}

\variantsplitter

\addpersonalvariant{Матвей Кузьмин}

\tasknumber{1}%
\task{%
    Напротив физических величин укажите их обозначения и единицы измерения в СИ:
    \begin{enumerate}
        \item разность потенциалов,
        \item мощность тока,
        \item ЭДС,
        \item внутреннее сопротивление полной цепи.
    \end{enumerate}
}
\solutionspace{20pt}

\tasknumber{2}%
\task{%
    Запишите физический закон или формулу:
    \begin{enumerate}
        \item правило Кирхгофа для узла цепи,
        \item сопротивление резистора через удельное сопротивление,
        \item закон Ома для неоднородного участка цепи.
    \end{enumerate}
}
\solutionspace{40pt}

\tasknumber{3}%
\task{%
    На резистор сопротивлением $R = 30\,\text{Ом}$ подали напряжение $U = 240\,\text{В}$.
    Определите ток, который потечёт через резистор, и мощность, выделяющуюся на нём.
}
\answer{%
    \begin{align*}
    \eli &= \frac{U}{R} = \frac{240\,\text{В}}{30\,\text{Ом}} = 8\,\text{А},  \\
    P &= \frac{U^2}{R} = \frac{\sqr{240\,\text{В}}}{30\,\text{Ом}} = 1920\,\text{Вт}
    \end{align*}
}
\solutionspace{60pt}

\tasknumber{4}%
\task{%
    Через резистор сопротивлением $R = 30\,\text{Ом}$ протекает электрический ток $\eli = 8\,\text{А}$.
    Определите, чему равны напряжение на резисторе и мощность, выделяющаяся на нём.
}
\answer{%
    \begin{align*}
    U &= \eli R = 8\,\text{А} \cdot 30\,\text{Ом} = 240\,\text{В},  \\
    P &= \eli^2R = \sqr{8\,\text{А}} \cdot 30\,\text{Ом} = 1920\,\text{Вт}
    \end{align*}
}
\solutionspace{60pt}

\tasknumber{5}%
\task{%
    Замкнутая электрическая цепь состоит из ЭДС $\ele = 1\,\text{В}$ и сопротивлением $r$
    и резистора $R = 15\,\text{Ом}$.
    Определите ток, протекающий в цепи.
    Какая тепловая энергия выделится на резисторе за время
    $\tau = 10\,\text{с}$? Какая работа будет совершена ЭДС за это время? Каков знак этой работы? Чему равен КПД цепи?
    Вычислите значения для 2 случаев: $r=0$ и $r = 20\,\text{Ом}$.
}
\answer{%
    \begin{align*}
    \eli_1 &= \frac{\ele}{R} = \frac{1\,\text{В}}{15\,\text{Ом}} = \frac1{15}\units{А} \approx 0{,}07\,\text{А},  \\
    \eli_2 &= \frac{\ele}{R + r} = \frac{1\,\text{В}}{15\,\text{Ом} + 20\,\text{Ом}} = \frac1{35}\units{А} \approx 0{,}03\,\text{А},  \\
    Q_1 &= \eli_1^2R\tau = \sqr{\frac{\ele}{R}} R \tau
            = \sqr{\frac{1\,\text{В}}{15\,\text{Ом}}} \cdot 15\,\text{Ом} \cdot 10\,\text{с} = \frac23\units{Дж} \approx 0{,}667\,\text{Дж},  \\
    Q_2 &= \eli_2^2R\tau = \sqr{\frac{\ele}{R + r}} R \tau
            = \sqr{\frac{1\,\text{В}}{15\,\text{Ом} + 20\,\text{Ом}}} \cdot 15\,\text{Ом} \cdot 10\,\text{с} = \frac6{49}\units{Дж} \approx 0{,}122\,\text{Дж},  \\
    A_1 &= q_1\ele = \eli_1\tau\ele = \frac{\ele}{R} \tau \ele
            = \frac{\ele^2 \tau}{R} = \frac{\sqr{1\,\text{В}} \cdot 10\,\text{с}}{15\,\text{Ом}}
            = \frac23\units{Дж} \approx 0{,}667\,\text{Дж}, \text{положительна},  \\
    A_2 &= q_2\ele = \eli_2\tau\ele = \frac{\ele}{R + r} \tau \ele
            = \frac{\ele^2 \tau}{R + r} = \frac{\sqr{1\,\text{В}} \cdot 10\,\text{с}}{15\,\text{Ом} + 20\,\text{Ом}}
            = \frac27\units{Дж} \approx 0{,}286\,\text{Дж}, \text{положительна},  \\
    \eta_1 &= \frac{Q_1}{A_1} = \ldots = \frac{R}{R} = 1,  \\
    \eta_2 &= \frac{Q_2}{A_2} = \ldots = \frac{R}{R + r} = \frac37 \approx 0{,}43.
    \end{align*}
}
\solutionspace{180pt}

\tasknumber{6}%
\task{%
    Лампочки, сопротивления которых $R_1 = 0{,}25\,\text{Ом}$ и $R_2 = 16\,\text{Ом}$, поочерёдно подключённные к некоторому источнику тока,
    потребляют одинаковую мощность.
    Найти внутреннее сопротивление источника и КПД цепи в каждом случае.
}
\answer{%
    \begin{align*}
        P_1 &= \sqr{\frac{\ele}{R_1 + r}}R_1,
        P_2  = \sqr{\frac{\ele}{R_2 + r}}R_2,
        P_1 = P_2 \implies  \\
        &\implies R_1 \sqr{R_2 + r} = R_2 \sqr{R_1 + r} \implies  \\
        &\implies R_1 R_2^2 + 2 R_1 R_2 r + R_1 r^2 =
                    R_2 R_1^2 + 2 R_2 R_1 r + R_2 r^2  \implies  \\
    &\implies r^2 (R_2 - R_1) = R_2^2 R_2 - R_1^2 R_2 \implies  \\
    &\implies r
            = \sqrt{R_1 R_2 \frac{R_2 - R_1}{R_2 - R_1}}
            = \sqrt{R_1 R_2}
            = \sqrt{0{,}25\,\text{Ом} \cdot 16\,\text{Ом}}
            = 2\,\text{Ом}.
            \\
    \eta_1
            &= \frac{R_1}{R_1 + r}
            = \frac{\sqrt{R_1}}{\sqrt{R_1} + \sqrt{R_2}}
            = 0{,}111,  \\
    \eta_2
            &= \frac{R_2}{R_2 + r}
            = \frac{\sqrt{R_2}}{\sqrt{R_2} + \sqrt{R_1}}
            = 0{,}889
    \end{align*}
}

\variantsplitter

\addpersonalvariant{Сергей Малышев}

\tasknumber{1}%
\task{%
    Напротив физических величин укажите их обозначения и единицы измерения в СИ:
    \begin{enumerate}
        \item напряжение,
        \item работа тока,
        \item удельное сопротивление,
        \item внутреннее сопротивление полной цепи.
    \end{enumerate}
}
\solutionspace{20pt}

\tasknumber{2}%
\task{%
    Запишите физический закон или формулу:
    \begin{enumerate}
        \item правило Кирхгофа для узла цепи,
        \item сопротивление резистора через удельное сопротивление,
        \item ЭДС (определение).
    \end{enumerate}
}
\solutionspace{40pt}

\tasknumber{3}%
\task{%
    На резистор сопротивлением $r = 5\,\text{Ом}$ подали напряжение $V = 240\,\text{В}$.
    Определите ток, который потечёт через резистор, и мощность, выделяющуюся на нём.
}
\answer{%
    \begin{align*}
    \eli &= \frac{V}{r} = \frac{240\,\text{В}}{5\,\text{Ом}} = 48\,\text{А},  \\
    P &= \frac{V^2}{r} = \frac{\sqr{240\,\text{В}}}{5\,\text{Ом}} = 11520\,\text{Вт}
    \end{align*}
}
\solutionspace{60pt}

\tasknumber{4}%
\task{%
    Через резистор сопротивлением $R = 18\,\text{Ом}$ протекает электрический ток $\eli = 3\,\text{А}$.
    Определите, чему равны напряжение на резисторе и мощность, выделяющаяся на нём.
}
\answer{%
    \begin{align*}
    U &= \eli R = 3\,\text{А} \cdot 18\,\text{Ом} = 54\,\text{В},  \\
    P &= \eli^2R = \sqr{3\,\text{А}} \cdot 18\,\text{Ом} = 162\,\text{Вт}
    \end{align*}
}
\solutionspace{60pt}

\tasknumber{5}%
\task{%
    Замкнутая электрическая цепь состоит из ЭДС $\ele = 4\,\text{В}$ и сопротивлением $r$
    и резистора $R = 24\,\text{Ом}$.
    Определите ток, протекающий в цепи.
    Какая тепловая энергия выделится на резисторе за время
    $\tau = 2\,\text{с}$? Какая работа будет совершена ЭДС за это время? Каков знак этой работы? Чему равен КПД цепи?
    Вычислите значения для 2 случаев: $r=0$ и $r = 20\,\text{Ом}$.
}
\answer{%
    \begin{align*}
    \eli_1 &= \frac{\ele}{R} = \frac{4\,\text{В}}{24\,\text{Ом}} = \frac16\units{А} \approx 0{,}17\,\text{А},  \\
    \eli_2 &= \frac{\ele}{R + r} = \frac{4\,\text{В}}{24\,\text{Ом} + 20\,\text{Ом}} = \frac1{11}\units{А} \approx 0{,}09\,\text{А},  \\
    Q_1 &= \eli_1^2R\tau = \sqr{\frac{\ele}{R}} R \tau
            = \sqr{\frac{4\,\text{В}}{24\,\text{Ом}}} \cdot 24\,\text{Ом} \cdot 2\,\text{с} = \frac43\units{Дж} \approx 1{,}333\,\text{Дж},  \\
    Q_2 &= \eli_2^2R\tau = \sqr{\frac{\ele}{R + r}} R \tau
            = \sqr{\frac{4\,\text{В}}{24\,\text{Ом} + 20\,\text{Ом}}} \cdot 24\,\text{Ом} \cdot 2\,\text{с} = \frac{48}{121}\units{Дж} \approx 0{,}397\,\text{Дж},  \\
    A_1 &= q_1\ele = \eli_1\tau\ele = \frac{\ele}{R} \tau \ele
            = \frac{\ele^2 \tau}{R} = \frac{\sqr{4\,\text{В}} \cdot 2\,\text{с}}{24\,\text{Ом}}
            = \frac43\units{Дж} \approx 1{,}333\,\text{Дж}, \text{положительна},  \\
    A_2 &= q_2\ele = \eli_2\tau\ele = \frac{\ele}{R + r} \tau \ele
            = \frac{\ele^2 \tau}{R + r} = \frac{\sqr{4\,\text{В}} \cdot 2\,\text{с}}{24\,\text{Ом} + 20\,\text{Ом}}
            = \frac8{11}\units{Дж} \approx 0{,}727\,\text{Дж}, \text{положительна},  \\
    \eta_1 &= \frac{Q_1}{A_1} = \ldots = \frac{R}{R} = 1,  \\
    \eta_2 &= \frac{Q_2}{A_2} = \ldots = \frac{R}{R + r} = \frac6{11} \approx 0{,}55.
    \end{align*}
}
\solutionspace{180pt}

\tasknumber{6}%
\task{%
    Лампочки, сопротивления которых $R_1 = 4\,\text{Ом}$ и $R_2 = 36\,\text{Ом}$, поочерёдно подключённные к некоторому источнику тока,
    потребляют одинаковую мощность.
    Найти внутреннее сопротивление источника и КПД цепи в каждом случае.
}
\answer{%
    \begin{align*}
        P_1 &= \sqr{\frac{\ele}{R_1 + r}}R_1,
        P_2  = \sqr{\frac{\ele}{R_2 + r}}R_2,
        P_1 = P_2 \implies  \\
        &\implies R_1 \sqr{R_2 + r} = R_2 \sqr{R_1 + r} \implies  \\
        &\implies R_1 R_2^2 + 2 R_1 R_2 r + R_1 r^2 =
                    R_2 R_1^2 + 2 R_2 R_1 r + R_2 r^2  \implies  \\
    &\implies r^2 (R_2 - R_1) = R_2^2 R_2 - R_1^2 R_2 \implies  \\
    &\implies r
            = \sqrt{R_1 R_2 \frac{R_2 - R_1}{R_2 - R_1}}
            = \sqrt{R_1 R_2}
            = \sqrt{4\,\text{Ом} \cdot 36\,\text{Ом}}
            = 12\,\text{Ом}.
            \\
    \eta_1
            &= \frac{R_1}{R_1 + r}
            = \frac{\sqrt{R_1}}{\sqrt{R_1} + \sqrt{R_2}}
            = 0{,}250,  \\
    \eta_2
            &= \frac{R_2}{R_2 + r}
            = \frac{\sqrt{R_2}}{\sqrt{R_2} + \sqrt{R_1}}
            = 0{,}750
    \end{align*}
}

\variantsplitter

\addpersonalvariant{Алина Полканова}

\tasknumber{1}%
\task{%
    Напротив физических величин укажите их обозначения и единицы измерения в СИ:
    \begin{enumerate}
        \item напряжение,
        \item работа тока,
        \item ЭДС,
        \item внутреннее сопротивление полной цепи.
    \end{enumerate}
}
\solutionspace{20pt}

\tasknumber{2}%
\task{%
    Запишите физический закон или формулу:
    \begin{enumerate}
        \item правило Кирхгофа для узла цепи,
        \item закон Ома для однородного участка цепи,
        \item ЭДС (определение).
    \end{enumerate}
}
\solutionspace{40pt}

\tasknumber{3}%
\task{%
    На резистор сопротивлением $R = 5\,\text{Ом}$ подали напряжение $V = 120\,\text{В}$.
    Определите ток, который потечёт через резистор, и мощность, выделяющуюся на нём.
}
\answer{%
    \begin{align*}
    \eli &= \frac{V}{R} = \frac{120\,\text{В}}{5\,\text{Ом}} = 24\,\text{А},  \\
    P &= \frac{V^2}{R} = \frac{\sqr{120\,\text{В}}}{5\,\text{Ом}} = 2880\,\text{Вт}
    \end{align*}
}
\solutionspace{60pt}

\tasknumber{4}%
\task{%
    Через резистор сопротивлением $R = 18\,\text{Ом}$ протекает электрический ток $\eli = 8\,\text{А}$.
    Определите, чему равны напряжение на резисторе и мощность, выделяющаяся на нём.
}
\answer{%
    \begin{align*}
    U &= \eli R = 8\,\text{А} \cdot 18\,\text{Ом} = 144\,\text{В},  \\
    P &= \eli^2R = \sqr{8\,\text{А}} \cdot 18\,\text{Ом} = 1152\,\text{Вт}
    \end{align*}
}
\solutionspace{60pt}

\tasknumber{5}%
\task{%
    Замкнутая электрическая цепь состоит из ЭДС $\ele = 3\,\text{В}$ и сопротивлением $r$
    и резистора $R = 15\,\text{Ом}$.
    Определите ток, протекающий в цепи.
    Какая тепловая энергия выделится на резисторе за время
    $\tau = 5\,\text{с}$? Какая работа будет совершена ЭДС за это время? Каков знак этой работы? Чему равен КПД цепи?
    Вычислите значения для 2 случаев: $r=0$ и $r = 60\,\text{Ом}$.
}
\answer{%
    \begin{align*}
    \eli_1 &= \frac{\ele}{R} = \frac{3\,\text{В}}{15\,\text{Ом}} = \frac15\units{А} \approx 0{,}20\,\text{А},  \\
    \eli_2 &= \frac{\ele}{R + r} = \frac{3\,\text{В}}{15\,\text{Ом} + 60\,\text{Ом}} = \frac1{25}\units{А} \approx 0{,}04\,\text{А},  \\
    Q_1 &= \eli_1^2R\tau = \sqr{\frac{\ele}{R}} R \tau
            = \sqr{\frac{3\,\text{В}}{15\,\text{Ом}}} \cdot 15\,\text{Ом} \cdot 5\,\text{с} = 3\units{Дж} \approx 3\,\text{Дж},  \\
    Q_2 &= \eli_2^2R\tau = \sqr{\frac{\ele}{R + r}} R \tau
            = \sqr{\frac{3\,\text{В}}{15\,\text{Ом} + 60\,\text{Ом}}} \cdot 15\,\text{Ом} \cdot 5\,\text{с} = \frac3{25}\units{Дж} \approx 0{,}120\,\text{Дж},  \\
    A_1 &= q_1\ele = \eli_1\tau\ele = \frac{\ele}{R} \tau \ele
            = \frac{\ele^2 \tau}{R} = \frac{\sqr{3\,\text{В}} \cdot 5\,\text{с}}{15\,\text{Ом}}
            = 3\units{Дж} \approx 3\,\text{Дж}, \text{положительна},  \\
    A_2 &= q_2\ele = \eli_2\tau\ele = \frac{\ele}{R + r} \tau \ele
            = \frac{\ele^2 \tau}{R + r} = \frac{\sqr{3\,\text{В}} \cdot 5\,\text{с}}{15\,\text{Ом} + 60\,\text{Ом}}
            = \frac35\units{Дж} \approx 0{,}600\,\text{Дж}, \text{положительна},  \\
    \eta_1 &= \frac{Q_1}{A_1} = \ldots = \frac{R}{R} = 1,  \\
    \eta_2 &= \frac{Q_2}{A_2} = \ldots = \frac{R}{R + r} = \frac15 \approx 0{,}20.
    \end{align*}
}
\solutionspace{180pt}

\tasknumber{6}%
\task{%
    Лампочки, сопротивления которых $R_1 = 0{,}25\,\text{Ом}$ и $R_2 = 16\,\text{Ом}$, поочерёдно подключённные к некоторому источнику тока,
    потребляют одинаковую мощность.
    Найти внутреннее сопротивление источника и КПД цепи в каждом случае.
}
\answer{%
    \begin{align*}
        P_1 &= \sqr{\frac{\ele}{R_1 + r}}R_1,
        P_2  = \sqr{\frac{\ele}{R_2 + r}}R_2,
        P_1 = P_2 \implies  \\
        &\implies R_1 \sqr{R_2 + r} = R_2 \sqr{R_1 + r} \implies  \\
        &\implies R_1 R_2^2 + 2 R_1 R_2 r + R_1 r^2 =
                    R_2 R_1^2 + 2 R_2 R_1 r + R_2 r^2  \implies  \\
    &\implies r^2 (R_2 - R_1) = R_2^2 R_2 - R_1^2 R_2 \implies  \\
    &\implies r
            = \sqrt{R_1 R_2 \frac{R_2 - R_1}{R_2 - R_1}}
            = \sqrt{R_1 R_2}
            = \sqrt{0{,}25\,\text{Ом} \cdot 16\,\text{Ом}}
            = 2\,\text{Ом}.
            \\
    \eta_1
            &= \frac{R_1}{R_1 + r}
            = \frac{\sqrt{R_1}}{\sqrt{R_1} + \sqrt{R_2}}
            = 0{,}111,  \\
    \eta_2
            &= \frac{R_2}{R_2 + r}
            = \frac{\sqrt{R_2}}{\sqrt{R_2} + \sqrt{R_1}}
            = 0{,}889
    \end{align*}
}

\variantsplitter

\addpersonalvariant{Сергей Пономарёв}

\tasknumber{1}%
\task{%
    Напротив физических величин укажите их обозначения и единицы измерения в СИ:
    \begin{enumerate}
        \item напряжение,
        \item работа тока,
        \item удельное сопротивление,
        \item внутреннее сопротивление полной цепи.
    \end{enumerate}
}
\solutionspace{20pt}

\tasknumber{2}%
\task{%
    Запишите физический закон или формулу:
    \begin{enumerate}
        \item правило Кирхгофа для замкнутого контура,
        \item сопротивление резистора через удельное сопротивление,
        \item ЭДС (определение).
    \end{enumerate}
}
\solutionspace{40pt}

\tasknumber{3}%
\task{%
    На резистор сопротивлением $r = 30\,\text{Ом}$ подали напряжение $U = 240\,\text{В}$.
    Определите ток, который потечёт через резистор, и мощность, выделяющуюся на нём.
}
\answer{%
    \begin{align*}
    \eli &= \frac{U}{r} = \frac{240\,\text{В}}{30\,\text{Ом}} = 8\,\text{А},  \\
    P &= \frac{U^2}{r} = \frac{\sqr{240\,\text{В}}}{30\,\text{Ом}} = 1920\,\text{Вт}
    \end{align*}
}
\solutionspace{60pt}

\tasknumber{4}%
\task{%
    Через резистор сопротивлением $r = 18\,\text{Ом}$ протекает электрический ток $\eli = 6\,\text{А}$.
    Определите, чему равны напряжение на резисторе и мощность, выделяющаяся на нём.
}
\answer{%
    \begin{align*}
    U &= \eli r = 6\,\text{А} \cdot 18\,\text{Ом} = 108\,\text{В},  \\
    P &= \eli^2r = \sqr{6\,\text{А}} \cdot 18\,\text{Ом} = 648\,\text{Вт}
    \end{align*}
}
\solutionspace{60pt}

\tasknumber{5}%
\task{%
    Замкнутая электрическая цепь состоит из ЭДС $\ele = 4\,\text{В}$ и сопротивлением $r$
    и резистора $R = 10\,\text{Ом}$.
    Определите ток, протекающий в цепи.
    Какая тепловая энергия выделится на резисторе за время
    $\tau = 5\,\text{с}$? Какая работа будет совершена ЭДС за это время? Каков знак этой работы? Чему равен КПД цепи?
    Вычислите значения для 2 случаев: $r=0$ и $r = 30\,\text{Ом}$.
}
\answer{%
    \begin{align*}
    \eli_1 &= \frac{\ele}{R} = \frac{4\,\text{В}}{10\,\text{Ом}} = \frac25\units{А} \approx 0{,}40\,\text{А},  \\
    \eli_2 &= \frac{\ele}{R + r} = \frac{4\,\text{В}}{10\,\text{Ом} + 30\,\text{Ом}} = \frac1{10}\units{А} \approx 0{,}10\,\text{А},  \\
    Q_1 &= \eli_1^2R\tau = \sqr{\frac{\ele}{R}} R \tau
            = \sqr{\frac{4\,\text{В}}{10\,\text{Ом}}} \cdot 10\,\text{Ом} \cdot 5\,\text{с} = 8\units{Дж} \approx 8\,\text{Дж},  \\
    Q_2 &= \eli_2^2R\tau = \sqr{\frac{\ele}{R + r}} R \tau
            = \sqr{\frac{4\,\text{В}}{10\,\text{Ом} + 30\,\text{Ом}}} \cdot 10\,\text{Ом} \cdot 5\,\text{с} = \frac12\units{Дж} \approx 0{,}500\,\text{Дж},  \\
    A_1 &= q_1\ele = \eli_1\tau\ele = \frac{\ele}{R} \tau \ele
            = \frac{\ele^2 \tau}{R} = \frac{\sqr{4\,\text{В}} \cdot 5\,\text{с}}{10\,\text{Ом}}
            = 8\units{Дж} \approx 8\,\text{Дж}, \text{положительна},  \\
    A_2 &= q_2\ele = \eli_2\tau\ele = \frac{\ele}{R + r} \tau \ele
            = \frac{\ele^2 \tau}{R + r} = \frac{\sqr{4\,\text{В}} \cdot 5\,\text{с}}{10\,\text{Ом} + 30\,\text{Ом}}
            = 2\units{Дж} \approx 2\,\text{Дж}, \text{положительна},  \\
    \eta_1 &= \frac{Q_1}{A_1} = \ldots = \frac{R}{R} = 1,  \\
    \eta_2 &= \frac{Q_2}{A_2} = \ldots = \frac{R}{R + r} = \frac14 \approx 0{,}25.
    \end{align*}
}
\solutionspace{180pt}

\tasknumber{6}%
\task{%
    Лампочки, сопротивления которых $R_1 = 6\,\text{Ом}$ и $R_2 = 54\,\text{Ом}$, поочерёдно подключённные к некоторому источнику тока,
    потребляют одинаковую мощность.
    Найти внутреннее сопротивление источника и КПД цепи в каждом случае.
}
\answer{%
    \begin{align*}
        P_1 &= \sqr{\frac{\ele}{R_1 + r}}R_1,
        P_2  = \sqr{\frac{\ele}{R_2 + r}}R_2,
        P_1 = P_2 \implies  \\
        &\implies R_1 \sqr{R_2 + r} = R_2 \sqr{R_1 + r} \implies  \\
        &\implies R_1 R_2^2 + 2 R_1 R_2 r + R_1 r^2 =
                    R_2 R_1^2 + 2 R_2 R_1 r + R_2 r^2  \implies  \\
    &\implies r^2 (R_2 - R_1) = R_2^2 R_2 - R_1^2 R_2 \implies  \\
    &\implies r
            = \sqrt{R_1 R_2 \frac{R_2 - R_1}{R_2 - R_1}}
            = \sqrt{R_1 R_2}
            = \sqrt{6\,\text{Ом} \cdot 54\,\text{Ом}}
            = 18\,\text{Ом}.
            \\
    \eta_1
            &= \frac{R_1}{R_1 + r}
            = \frac{\sqrt{R_1}}{\sqrt{R_1} + \sqrt{R_2}}
            = 0{,}250,  \\
    \eta_2
            &= \frac{R_2}{R_2 + r}
            = \frac{\sqrt{R_2}}{\sqrt{R_2} + \sqrt{R_1}}
            = 0{,}750
    \end{align*}
}

\variantsplitter

\addpersonalvariant{Егор Свистушкин}

\tasknumber{1}%
\task{%
    Напротив физических величин укажите их обозначения и единицы измерения в СИ:
    \begin{enumerate}
        \item сила тока,
        \item мощность тока,
        \item ЭДС,
        \item внутреннее сопротивление полной цепи.
    \end{enumerate}
}
\solutionspace{20pt}

\tasknumber{2}%
\task{%
    Запишите физический закон или формулу:
    \begin{enumerate}
        \item правило Кирхгофа для замкнутого контура,
        \item сопротивление резистора через удельное сопротивление,
        \item закон Ома для неоднородного участка цепи.
    \end{enumerate}
}
\solutionspace{40pt}

\tasknumber{3}%
\task{%
    На резистор сопротивлением $r = 18\,\text{Ом}$ подали напряжение $U = 120\,\text{В}$.
    Определите ток, который потечёт через резистор, и мощность, выделяющуюся на нём.
}
\answer{%
    \begin{align*}
    \eli &= \frac{U}{r} = \frac{120\,\text{В}}{18\,\text{Ом}} = 6{,}67\,\text{А},  \\
    P &= \frac{U^2}{r} = \frac{\sqr{120\,\text{В}}}{18\,\text{Ом}} = 800\,\text{Вт}
    \end{align*}
}
\solutionspace{60pt}

\tasknumber{4}%
\task{%
    Через резистор сопротивлением $R = 12\,\text{Ом}$ протекает электрический ток $\eli = 2\,\text{А}$.
    Определите, чему равны напряжение на резисторе и мощность, выделяющаяся на нём.
}
\answer{%
    \begin{align*}
    U &= \eli R = 2\,\text{А} \cdot 12\,\text{Ом} = 24\,\text{В},  \\
    P &= \eli^2R = \sqr{2\,\text{А}} \cdot 12\,\text{Ом} = 48\,\text{Вт}
    \end{align*}
}
\solutionspace{60pt}

\tasknumber{5}%
\task{%
    Замкнутая электрическая цепь состоит из ЭДС $\ele = 4\,\text{В}$ и сопротивлением $r$
    и резистора $R = 10\,\text{Ом}$.
    Определите ток, протекающий в цепи.
    Какая тепловая энергия выделится на резисторе за время
    $\tau = 2\,\text{с}$? Какая работа будет совершена ЭДС за это время? Каков знак этой работы? Чему равен КПД цепи?
    Вычислите значения для 2 случаев: $r=0$ и $r = 20\,\text{Ом}$.
}
\answer{%
    \begin{align*}
    \eli_1 &= \frac{\ele}{R} = \frac{4\,\text{В}}{10\,\text{Ом}} = \frac25\units{А} \approx 0{,}40\,\text{А},  \\
    \eli_2 &= \frac{\ele}{R + r} = \frac{4\,\text{В}}{10\,\text{Ом} + 20\,\text{Ом}} = \frac2{15}\units{А} \approx 0{,}13\,\text{А},  \\
    Q_1 &= \eli_1^2R\tau = \sqr{\frac{\ele}{R}} R \tau
            = \sqr{\frac{4\,\text{В}}{10\,\text{Ом}}} \cdot 10\,\text{Ом} \cdot 2\,\text{с} = \frac{16}5\units{Дж} \approx 3{,}200\,\text{Дж},  \\
    Q_2 &= \eli_2^2R\tau = \sqr{\frac{\ele}{R + r}} R \tau
            = \sqr{\frac{4\,\text{В}}{10\,\text{Ом} + 20\,\text{Ом}}} \cdot 10\,\text{Ом} \cdot 2\,\text{с} = \frac{16}{45}\units{Дж} \approx 0{,}356\,\text{Дж},  \\
    A_1 &= q_1\ele = \eli_1\tau\ele = \frac{\ele}{R} \tau \ele
            = \frac{\ele^2 \tau}{R} = \frac{\sqr{4\,\text{В}} \cdot 2\,\text{с}}{10\,\text{Ом}}
            = \frac{16}5\units{Дж} \approx 3{,}200\,\text{Дж}, \text{положительна},  \\
    A_2 &= q_2\ele = \eli_2\tau\ele = \frac{\ele}{R + r} \tau \ele
            = \frac{\ele^2 \tau}{R + r} = \frac{\sqr{4\,\text{В}} \cdot 2\,\text{с}}{10\,\text{Ом} + 20\,\text{Ом}}
            = \frac{16}{15}\units{Дж} \approx 1{,}067\,\text{Дж}, \text{положительна},  \\
    \eta_1 &= \frac{Q_1}{A_1} = \ldots = \frac{R}{R} = 1,  \\
    \eta_2 &= \frac{Q_2}{A_2} = \ldots = \frac{R}{R + r} = \frac13 \approx 0{,}33.
    \end{align*}
}
\solutionspace{180pt}

\tasknumber{6}%
\task{%
    Лампочки, сопротивления которых $R_1 = 3\,\text{Ом}$ и $R_2 = 48\,\text{Ом}$, поочерёдно подключённные к некоторому источнику тока,
    потребляют одинаковую мощность.
    Найти внутреннее сопротивление источника и КПД цепи в каждом случае.
}
\answer{%
    \begin{align*}
        P_1 &= \sqr{\frac{\ele}{R_1 + r}}R_1,
        P_2  = \sqr{\frac{\ele}{R_2 + r}}R_2,
        P_1 = P_2 \implies  \\
        &\implies R_1 \sqr{R_2 + r} = R_2 \sqr{R_1 + r} \implies  \\
        &\implies R_1 R_2^2 + 2 R_1 R_2 r + R_1 r^2 =
                    R_2 R_1^2 + 2 R_2 R_1 r + R_2 r^2  \implies  \\
    &\implies r^2 (R_2 - R_1) = R_2^2 R_2 - R_1^2 R_2 \implies  \\
    &\implies r
            = \sqrt{R_1 R_2 \frac{R_2 - R_1}{R_2 - R_1}}
            = \sqrt{R_1 R_2}
            = \sqrt{3\,\text{Ом} \cdot 48\,\text{Ом}}
            = 12\,\text{Ом}.
            \\
    \eta_1
            &= \frac{R_1}{R_1 + r}
            = \frac{\sqrt{R_1}}{\sqrt{R_1} + \sqrt{R_2}}
            = 0{,}200,  \\
    \eta_2
            &= \frac{R_2}{R_2 + r}
            = \frac{\sqrt{R_2}}{\sqrt{R_2} + \sqrt{R_1}}
            = 0{,}800
    \end{align*}
}

\variantsplitter

\addpersonalvariant{Дмитрий Соколов}

\tasknumber{1}%
\task{%
    Напротив физических величин укажите их обозначения и единицы измерения в СИ:
    \begin{enumerate}
        \item сила тока,
        \item мощность тока,
        \item ЭДС,
        \item внешнее сопротивление полной цепи.
    \end{enumerate}
}
\solutionspace{20pt}

\tasknumber{2}%
\task{%
    Запишите физический закон или формулу:
    \begin{enumerate}
        \item правило Кирхгофа для узла цепи,
        \item закон Ома для однородного участка цепи,
        \item ЭДС (определение).
    \end{enumerate}
}
\solutionspace{40pt}

\tasknumber{3}%
\task{%
    На резистор сопротивлением $r = 30\,\text{Ом}$ подали напряжение $V = 180\,\text{В}$.
    Определите ток, который потечёт через резистор, и мощность, выделяющуюся на нём.
}
\answer{%
    \begin{align*}
    \eli &= \frac{V}{r} = \frac{180\,\text{В}}{30\,\text{Ом}} = 6\,\text{А},  \\
    P &= \frac{V^2}{r} = \frac{\sqr{180\,\text{В}}}{30\,\text{Ом}} = 1080\,\text{Вт}
    \end{align*}
}
\solutionspace{60pt}

\tasknumber{4}%
\task{%
    Через резистор сопротивлением $R = 5\,\text{Ом}$ протекает электрический ток $\eli = 2\,\text{А}$.
    Определите, чему равны напряжение на резисторе и мощность, выделяющаяся на нём.
}
\answer{%
    \begin{align*}
    U &= \eli R = 2\,\text{А} \cdot 5\,\text{Ом} = 10\,\text{В},  \\
    P &= \eli^2R = \sqr{2\,\text{А}} \cdot 5\,\text{Ом} = 20\,\text{Вт}
    \end{align*}
}
\solutionspace{60pt}

\tasknumber{5}%
\task{%
    Замкнутая электрическая цепь состоит из ЭДС $\ele = 2\,\text{В}$ и сопротивлением $r$
    и резистора $R = 30\,\text{Ом}$.
    Определите ток, протекающий в цепи.
    Какая тепловая энергия выделится на резисторе за время
    $\tau = 10\,\text{с}$? Какая работа будет совершена ЭДС за это время? Каков знак этой работы? Чему равен КПД цепи?
    Вычислите значения для 2 случаев: $r=0$ и $r = 60\,\text{Ом}$.
}
\answer{%
    \begin{align*}
    \eli_1 &= \frac{\ele}{R} = \frac{2\,\text{В}}{30\,\text{Ом}} = \frac1{15}\units{А} \approx 0{,}07\,\text{А},  \\
    \eli_2 &= \frac{\ele}{R + r} = \frac{2\,\text{В}}{30\,\text{Ом} + 60\,\text{Ом}} = \frac1{45}\units{А} \approx 0{,}02\,\text{А},  \\
    Q_1 &= \eli_1^2R\tau = \sqr{\frac{\ele}{R}} R \tau
            = \sqr{\frac{2\,\text{В}}{30\,\text{Ом}}} \cdot 30\,\text{Ом} \cdot 10\,\text{с} = \frac43\units{Дж} \approx 1{,}333\,\text{Дж},  \\
    Q_2 &= \eli_2^2R\tau = \sqr{\frac{\ele}{R + r}} R \tau
            = \sqr{\frac{2\,\text{В}}{30\,\text{Ом} + 60\,\text{Ом}}} \cdot 30\,\text{Ом} \cdot 10\,\text{с} = \frac4{27}\units{Дж} \approx 0{,}148\,\text{Дж},  \\
    A_1 &= q_1\ele = \eli_1\tau\ele = \frac{\ele}{R} \tau \ele
            = \frac{\ele^2 \tau}{R} = \frac{\sqr{2\,\text{В}} \cdot 10\,\text{с}}{30\,\text{Ом}}
            = \frac43\units{Дж} \approx 1{,}333\,\text{Дж}, \text{положительна},  \\
    A_2 &= q_2\ele = \eli_2\tau\ele = \frac{\ele}{R + r} \tau \ele
            = \frac{\ele^2 \tau}{R + r} = \frac{\sqr{2\,\text{В}} \cdot 10\,\text{с}}{30\,\text{Ом} + 60\,\text{Ом}}
            = \frac49\units{Дж} \approx 0{,}444\,\text{Дж}, \text{положительна},  \\
    \eta_1 &= \frac{Q_1}{A_1} = \ldots = \frac{R}{R} = 1,  \\
    \eta_2 &= \frac{Q_2}{A_2} = \ldots = \frac{R}{R + r} = \frac13 \approx 0{,}33.
    \end{align*}
}
\solutionspace{180pt}

\tasknumber{6}%
\task{%
    Лампочки, сопротивления которых $R_1 = 0{,}50\,\text{Ом}$ и $R_2 = 2\,\text{Ом}$, поочерёдно подключённные к некоторому источнику тока,
    потребляют одинаковую мощность.
    Найти внутреннее сопротивление источника и КПД цепи в каждом случае.
}
\answer{%
    \begin{align*}
        P_1 &= \sqr{\frac{\ele}{R_1 + r}}R_1,
        P_2  = \sqr{\frac{\ele}{R_2 + r}}R_2,
        P_1 = P_2 \implies  \\
        &\implies R_1 \sqr{R_2 + r} = R_2 \sqr{R_1 + r} \implies  \\
        &\implies R_1 R_2^2 + 2 R_1 R_2 r + R_1 r^2 =
                    R_2 R_1^2 + 2 R_2 R_1 r + R_2 r^2  \implies  \\
    &\implies r^2 (R_2 - R_1) = R_2^2 R_2 - R_1^2 R_2 \implies  \\
    &\implies r
            = \sqrt{R_1 R_2 \frac{R_2 - R_1}{R_2 - R_1}}
            = \sqrt{R_1 R_2}
            = \sqrt{0{,}50\,\text{Ом} \cdot 2\,\text{Ом}}
            = 1\,\text{Ом}.
            \\
    \eta_1
            &= \frac{R_1}{R_1 + r}
            = \frac{\sqrt{R_1}}{\sqrt{R_1} + \sqrt{R_2}}
            = 0{,}333,  \\
    \eta_2
            &= \frac{R_2}{R_2 + r}
            = \frac{\sqrt{R_2}}{\sqrt{R_2} + \sqrt{R_1}}
            = 0{,}667
    \end{align*}
}

\variantsplitter

\addpersonalvariant{Арсений Трофимов}

\tasknumber{1}%
\task{%
    Напротив физических величин укажите их обозначения и единицы измерения в СИ:
    \begin{enumerate}
        \item сила тока,
        \item работа тока,
        \item удельное сопротивление,
        \item внешнее сопротивление полной цепи.
    \end{enumerate}
}
\solutionspace{20pt}

\tasknumber{2}%
\task{%
    Запишите физический закон или формулу:
    \begin{enumerate}
        \item правило Кирхгофа для узла цепи,
        \item сопротивление резистора через удельное сопротивление,
        \item ЭДС (определение).
    \end{enumerate}
}
\solutionspace{40pt}

\tasknumber{3}%
\task{%
    На резистор сопротивлением $R = 18\,\text{Ом}$ подали напряжение $U = 120\,\text{В}$.
    Определите ток, который потечёт через резистор, и мощность, выделяющуюся на нём.
}
\answer{%
    \begin{align*}
    \eli &= \frac{U}{R} = \frac{120\,\text{В}}{18\,\text{Ом}} = 6{,}67\,\text{А},  \\
    P &= \frac{U^2}{R} = \frac{\sqr{120\,\text{В}}}{18\,\text{Ом}} = 800\,\text{Вт}
    \end{align*}
}
\solutionspace{60pt}

\tasknumber{4}%
\task{%
    Через резистор сопротивлением $R = 30\,\text{Ом}$ протекает электрический ток $\eli = 2\,\text{А}$.
    Определите, чему равны напряжение на резисторе и мощность, выделяющаяся на нём.
}
\answer{%
    \begin{align*}
    U &= \eli R = 2\,\text{А} \cdot 30\,\text{Ом} = 60\,\text{В},  \\
    P &= \eli^2R = \sqr{2\,\text{А}} \cdot 30\,\text{Ом} = 120\,\text{Вт}
    \end{align*}
}
\solutionspace{60pt}

\tasknumber{5}%
\task{%
    Замкнутая электрическая цепь состоит из ЭДС $\ele = 4\,\text{В}$ и сопротивлением $r$
    и резистора $R = 24\,\text{Ом}$.
    Определите ток, протекающий в цепи.
    Какая тепловая энергия выделится на резисторе за время
    $\tau = 2\,\text{с}$? Какая работа будет совершена ЭДС за это время? Каков знак этой работы? Чему равен КПД цепи?
    Вычислите значения для 2 случаев: $r=0$ и $r = 10\,\text{Ом}$.
}
\answer{%
    \begin{align*}
    \eli_1 &= \frac{\ele}{R} = \frac{4\,\text{В}}{24\,\text{Ом}} = \frac16\units{А} \approx 0{,}17\,\text{А},  \\
    \eli_2 &= \frac{\ele}{R + r} = \frac{4\,\text{В}}{24\,\text{Ом} + 10\,\text{Ом}} = \frac2{17}\units{А} \approx 0{,}12\,\text{А},  \\
    Q_1 &= \eli_1^2R\tau = \sqr{\frac{\ele}{R}} R \tau
            = \sqr{\frac{4\,\text{В}}{24\,\text{Ом}}} \cdot 24\,\text{Ом} \cdot 2\,\text{с} = \frac43\units{Дж} \approx 1{,}333\,\text{Дж},  \\
    Q_2 &= \eli_2^2R\tau = \sqr{\frac{\ele}{R + r}} R \tau
            = \sqr{\frac{4\,\text{В}}{24\,\text{Ом} + 10\,\text{Ом}}} \cdot 24\,\text{Ом} \cdot 2\,\text{с} = \frac{192}{289}\units{Дж} \approx 0{,}664\,\text{Дж},  \\
    A_1 &= q_1\ele = \eli_1\tau\ele = \frac{\ele}{R} \tau \ele
            = \frac{\ele^2 \tau}{R} = \frac{\sqr{4\,\text{В}} \cdot 2\,\text{с}}{24\,\text{Ом}}
            = \frac43\units{Дж} \approx 1{,}333\,\text{Дж}, \text{положительна},  \\
    A_2 &= q_2\ele = \eli_2\tau\ele = \frac{\ele}{R + r} \tau \ele
            = \frac{\ele^2 \tau}{R + r} = \frac{\sqr{4\,\text{В}} \cdot 2\,\text{с}}{24\,\text{Ом} + 10\,\text{Ом}}
            = \frac{16}{17}\units{Дж} \approx 0{,}941\,\text{Дж}, \text{положительна},  \\
    \eta_1 &= \frac{Q_1}{A_1} = \ldots = \frac{R}{R} = 1,  \\
    \eta_2 &= \frac{Q_2}{A_2} = \ldots = \frac{R}{R + r} = \frac{12}{17} \approx 0{,}71.
    \end{align*}
}
\solutionspace{180pt}

\tasknumber{6}%
\task{%
    Лампочки, сопротивления которых $R_1 = 0{,}25\,\text{Ом}$ и $R_2 = 4\,\text{Ом}$, поочерёдно подключённные к некоторому источнику тока,
    потребляют одинаковую мощность.
    Найти внутреннее сопротивление источника и КПД цепи в каждом случае.
}
\answer{%
    \begin{align*}
        P_1 &= \sqr{\frac{\ele}{R_1 + r}}R_1,
        P_2  = \sqr{\frac{\ele}{R_2 + r}}R_2,
        P_1 = P_2 \implies  \\
        &\implies R_1 \sqr{R_2 + r} = R_2 \sqr{R_1 + r} \implies  \\
        &\implies R_1 R_2^2 + 2 R_1 R_2 r + R_1 r^2 =
                    R_2 R_1^2 + 2 R_2 R_1 r + R_2 r^2  \implies  \\
    &\implies r^2 (R_2 - R_1) = R_2^2 R_2 - R_1^2 R_2 \implies  \\
    &\implies r
            = \sqrt{R_1 R_2 \frac{R_2 - R_1}{R_2 - R_1}}
            = \sqrt{R_1 R_2}
            = \sqrt{0{,}25\,\text{Ом} \cdot 4\,\text{Ом}}
            = 1\,\text{Ом}.
            \\
    \eta_1
            &= \frac{R_1}{R_1 + r}
            = \frac{\sqrt{R_1}}{\sqrt{R_1} + \sqrt{R_2}}
            = 0{,}200,  \\
    \eta_2
            &= \frac{R_2}{R_2 + r}
            = \frac{\sqrt{R_2}}{\sqrt{R_2} + \sqrt{R_1}}
            = 0{,}800
    \end{align*}
}
% autogenerated
