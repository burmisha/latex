\setdate{23~марта~2021}
\setclass{10«АБ»}

\addpersonalvariant{Михаил Бурмистров}

\tasknumber{1}%
\task{%
    Определите КПД (оставив ответ точным в виде нескоратимой дроби) цикла 1231, рабочим телом которого является идеальный одноатомный газ, если
    \begin{itemize}
        \item 12 — изохорический нагрев в три раза,
        \item 23 — изобарическое расширение, при котором температура растёт в четыре раза,
        \item 31 — процесс, график которого в $PV$-координатах является отрезком прямой.
    \end{itemize}
    Бонус: замените цикл 1231 циклом, в котором 12 — изохорический нагрев в три раза, 23 — процесс, график которого в $PV$-координатах является отрезком прямой, 31 — изобарическое охлаждение, при котором температура падает в три раза.
}
\answer{%
    \begin{align*}
    A_{12} &= 0, \Delta U_{12} > 0, \implies Q_{12} = A_{12} + \Delta U_{12} > 0.
    \\
    A_{23} &> 0, \Delta U_{23} > 0, \implies Q_{23} = A_{23} + \Delta U_{23} > 0, \\
    A_{31} &= 0, \Delta U_{31} < 0, \implies Q_{31} = A_{31} + \Delta U_{31} < 0.
    \\
    P_1V_1 &= \nu R T_1, P_2V_2 = \nu R T_2, P_3V_3 = \nu R T_3 \text{ — уравнения состояния идеального газа}, \\
    &\text{Пусть $P_0$, $V_0$, $T_0$ — давление, объём и температура в точке 1 (минимальные во всём цикле):} \\
    P_1 &= P_0, P_2 = P_3, V_1 = V_2 = V_0, \text{остальные соотношения нужно считать} \\
    T_2 &= 3T_1 = 3T_0 \text{(по условию)} \implies \frac{P_2}{P_1} = \frac{P_2V_0}{P_1V_0} = \frac{P_2 V_2}{P_1 V_1}= \frac{\nu R T_2}{\nu R T_1} = \frac{T_2}{T_1} = 3 \implies P_2 = 3 P_1 = 3 P_0, \\
    T_3 &= 4T_2 = 12T_0 \text{(по условию)} \implies \frac{V_3}{V_2} = \frac{P_3V_3}{P_2V_2}= \frac{\nu R T_3}{\nu R T_2} = \frac{T_3}{T_2} = 4 \implies V_3 = 4 V_2 = 4 V_0.
    \\
    A_\text{цикл} &= \frac 12 (4P_0 - P_0)(3V_0 - V_0) = \frac 12 \cdot 6 \cdot P_0V_0, \\
    A_{23} &= 3P_0 \cdot (4V_0 - V_0) = 9P_0V_0, \\
    \Delta U_{23} &= \frac 32 \nu R T_3 - \frac 32 \nu R T_2 = \frac 32 P_3 V_3 - \frac 32 P_2 V_2 = \frac 32 \cdot 3 P_0 \cdot 4 V_0 -  \frac 32 \cdot 3 P_0 \cdot V_0 = \frac 32 \cdot 9 \cdot P_0V_0, \\
    \Delta U_{12} &= \frac 32 \nu R T_2 - \frac 32 \nu R T_1 = \frac 32 P_2 V_2 - \frac 32 P_1 V_1 = \frac 32 \cdot 3 P_0V_0 - \frac 32 P_0V_0 = \frac 32 \cdot 2 \cdot P_0V_0.
    \\
    \eta &= \frac{A_\text{цикл}}{Q_+} = \frac{A_\text{цикл}}{Q_{12} + Q_{23}}  = \frac{A_\text{цикл}}{A_{12} + \Delta U_{12} + A_{23} + \Delta U_{23}} =  \\
     &= \frac{\frac 12 \cdot 6 \cdot P_0V_0}{0 + \frac 32 \cdot 2 \cdot P_0V_0 + 9P_0V_0 + \frac 32 \cdot 9 \cdot P_0V_0} = \frac{\frac 12 \cdot 6}{\frac 32 \cdot 2 + 9 + \frac 32 \cdot 9} = \frac2{17} \approx 0.118.
    \end{align*}


        График процесса не в масштабе (эта часть пока не готова и сделать автоматически аккуратно сложно), но с верными подписями (а для решения этого достаточно):

        \begin{tikzpicture}[thick]
            \draw[-{Latex}] (0, 0) -- (0, 7) node[above left] {$P$};
            \draw[-{Latex}] (0, 0) -- (10, 0) node[right] {$V$};

            \draw[dashed] (0, 2) node[left] {$P_1 = P_0$} -| (3, 0) node[below] {$V_1 = V_2 = V_0$};
            \draw[dashed] (0, 6) node[left] {$P_2 = P_3 = 3P_0$} -| (9, 0) node[below] {$V_3 = 4V_0$};

            \draw (3, 2) node[above left]{1} node[below left]{$T_1 = T_0$}
                   (3, 6) node[below left]{2} node[above]{$T_2 = 3T_0$}
                   (9, 6) node[above right]{3} node[below right]{$T_3 = 12T_0$};
            \draw[midar] (3, 2) -- (3, 6);
            \draw[midar] (3, 6) -- (9, 6);
            \draw[midar] (9, 6) -- (3, 2);
        \end{tikzpicture}

        Решение бонуса:
        \begin{align*}
            A_{12} &= 0, \Delta U_{12} > 0, \implies Q_{12} = A_{12} + \Delta U_{12} > 0, \\
            A_{23} &> 0, \Delta U_{23} \text{ — ничего нельзя сказать, нужно исследовать отдельно}, \\
            A_{31} &< 0, \Delta U_{31} < 0, \implies Q_{31} = A_{31} + \Delta U_{31} < 0.
            \\
        \end{align*}

        Уравнения состояния идеального газа для точек 1, 2, 3: $P_1V_1 = \nu R T_1, P_2V_2 = \nu R T_2, P_3V_3 = \nu R T_3$.
        Пусть $P_0$, $V_0$, $T_0$ — давление, объём и температура в точке 1 (минимальные во всём цикле).

        12 --- изохора, $\frac{P_1V_1}{T_1} = \nu R = \frac{P_2V_2}{T_2}, V_2=V_1=V_0 \implies \frac{P_1}{T_1} =  \frac{P_2}{T_2} \implies P_2 = P_1 \frac{T_2}{T_1} = 3P_0$,

        31 --- изобара, $\frac{P_1V_1}{T_1} = \nu R = \frac{P_3V_3}{T_3}, P_3=P_1=P_0 \implies \frac{V_3}{T_3} =  \frac{V_1}{T_1} \implies V_3 = V_1 \frac{T_3}{T_1} = 3V_0$,

        Таким образом, используя новые обозначения, состояния газа в точках 1, 2 и 3 описываются макропараметрами $(P_0, V_0, T_0), (3P_0, V_0, 3T_0), (P_0, 3V_0, 3T_0)$ соответственно.

        \begin{tikzpicture}[thick]
            \draw[-{Latex}] (0, 0) -- (0, 7) node[above left] {$P$};
            \draw[-{Latex}] (0, 0) -- (10, 0) node[right] {$V$};

            \draw[dashed] (0, 2) node[left] {$P_1 = P_3 = P_0$} -| (9, 0) node[below] {$V_3 = 3V_0$};
            \draw[dashed] (0, 6) node[left] {$P_2 = 3P_0$} -| (3, 0) node[below] {$V_1 = V_2 = V_0$};

            \draw[dashed] (0, 5) node[left] {$P$} -| (4.5, 0) node[below] {$V$};
            \draw[dashed] (0, 4.6) node[left] {$P'$} -| (5.1, 0) node[below] {$V'$};

            \draw (3, 2) node[above left]{1} node[below left]{$T_1 = T_0$}
                   (3, 6) node[below left]{2} node[above]{$T_2 = 3T_0$}
                   (9, 2) node[above right]{3} node[below right]{$T_3 = 3T_0$};
            \draw[midar] (3, 2) -- (3, 6);
            \draw[midar] (3, 6) -- (9, 2);
            \draw[midar] (9, 2) -- (3, 2);
            \draw   (4.5, 5) node[above right]{$T$} (5.1, 4.6) node[above right]{$T'$};
        \end{tikzpicture}


        Теперь рассмотрим отдельно процесс 23, к остальному вернёмся позже.
        Уравнение этой прямой в $PV$-координатах: $P(V) = 4P_0 - \frac{P_0}{V_0} V$.
        Это значит, что при изменении объёма на $\Delta V$ давление изменится на $\Delta P = - \frac{P_0}{V_0} \Delta V$, обратите внимание на знак.

        Рассмотрим произвольную точку в процессе 23 и дадим процессу ещё немного свершиться, при этом объём изменится на $\Delta V$, давление — на $\Delta P$, температура (иначе бы была гипербола, а не прямая) — на $\Delta T$,
        т.е.
        из состояния $(P, V, T)$ мы перешли в $(P', V', T')$, причём  $P' = P + \Delta P, V' = V + \Delta V, T' = T + \Delta T$.

        При этом изменится внутренняя энергия:
        \begin{align*}
        \Delta U
            &= U' - U = \frac 32 \nu R T' - \frac 32 \nu R T = \frac 32 (P+\Delta P) (V+\Delta V) - \frac 32 PV\\
            &= \frac 32 ((P+\Delta P) (V+\Delta V) - PV) = \frac 32 (P\Delta V + V \Delta P + \Delta P \Delta V).
        \end{align*}

        Рассмотрим малые изменения объёма, тогда и изменение давления будем малым (т.к.
        $\Delta P = - \frac{P_0}{V_0} \Delta V$),
        а третьим слагаемым в выражении для $\Delta U$  можно пренебречь по сравнению с двумя другими:
        два первых это малые величины, а третье — произведение двух малых.
        Тогда $\Delta U = \frac 32 (P\Delta V + V \Delta P)$.

        Работа газа при этом малом изменении объёма — это площадь трапеции (тут ещё раз пренебрегли малым слагаемым):
        $$A = \frac{P + P'}2 \Delta V = \cbr{P + \frac{\Delta P}2} \Delta V = P \Delta V.$$

        Подведённое количество теплоты, используя первое начало термодинамики, будет равно
        \begin{align*}
        Q
            &= \frac 32 (P\Delta V + V \Delta P) + P \Delta V =  \frac 52 P\Delta V + \frac 32 V \Delta P = \\
            &= \frac 52 P\Delta V + \frac 32 V \cdot \cbr{- \frac{P_0}{V_0} \Delta V} = \frac{\Delta V}2 \cdot \cbr{5P - \frac{P_0}{V_0} V} = \\
            &= \frac{\Delta V}2 \cdot \cbr{5 \cdot \cbr{4P_0 - \frac{P_0}{V_0} V} - \frac{P_0}{V_0} V}
             = \frac{\Delta V \cdot P_0}2 \cdot \cbr{5 \cdot 4 - 8\frac V{V_0}}.
        \end{align*}

        Таком образом, знак количества теплоты $Q$ на участке 23 зависит от конкретного значения $V$:
        \begin{itemize}
            \item $\Delta V > 0$ на всём участке 23, поскольку газ расширяется,
            \item $P > 0$ — всегда, у нас идеальный газ, удары о стенки сосуда абсолютно упругие, а молекулы не взаимодействуют и поэтому давление только положительно,
            \item если $5 \cdot 4 - 8\frac V{V_0} > 0$ — тепло подводят, если же меньше нуля — отводят.
        \end{itemize}
        Решая последнее неравенство, получаем конкретное значение $V^*$: при $V < V^*$ тепло подводят, далее~— отводят.
        Тут *~--- некоторая точка между точками 2 и 3, конкретные значения надо досчитать:
        $$V^* = V_0 \cdot \frac{5 \cdot 4}8 = \frac52 \cdot V_0 \implies P^* = 4P_0 - \frac{P_0}{V_0} V^* = \ldots = \frac32 \cdot P_0.$$

        Т.е.
        чтобы вычислить $Q_+$, надо сложить количества теплоты на участке 12 и лишь части участка 23 — участке 2*,
        той его части где это количество теплоты положительно.
        Имеем: $Q_+ = Q_{12} + Q_{2*}$.

        Теперь возвращаемся к циклу целиком и получаем:
        \begin{align*}
        A_\text{цикл} &= \frac 12 \cdot (3P_0 - P_0) \cdot (3V_0 - V_0) = 2 \cdot P_0V_0, \\
        A_{2*} &= \frac{P^* + 3P_0}2 \cdot (V^* - V_0)
            = \frac{\frac32 \cdot P_0 + 3P_0}2 \cdot \cbr{\frac52 \cdot V_0 - V_0}
            = \ldots = \frac{27}8 \cdot P_0 V_0, \\
        \Delta U_{2*} &= \frac 32 \nu R T^* - \frac 32 \nu R T_2 = \frac 32 (P^*V^* - P_0 \cdot 3V_0)
            = \frac 32 \cbr{\frac32 \cdot P_0 \cdot \frac52 \cdot V_0 - P_0 \cdot 3V_0}
            = \frac98 \cdot P_0 V_0, \\
        \Delta U_{12} &= \frac 32 \nu R T_2 - \frac 32 \nu R T_1 = \frac 32 (3P_0V_0 - P_0V_0) = \ldots = 3 \cdot P_0 V_0, \\
        \eta &= \frac{A_\text{цикл}}{Q_+} = \frac{A_\text{цикл}}{Q_{12} + Q_{2*}}
            = \frac{A_\text{цикл}}{A_{12} + \Delta U_{12} + A_{2*} + \Delta U_{2*}} = \\
            &= \frac{2 \cdot P_0V_0}{0 + 3 \cdot P_0 V_0 + \frac{27}8 \cdot P_0 V_0 + \frac98 \cdot P_0 V_0}
             = \frac{A_bonus_cycle:LaTeX}{3 + \frac{27}8 + \frac98}
             = \frac4{15} \leftarrow \text{вжух и готово!}
        \end{align*}
}
\solutionspace{360pt}

\tasknumber{2}%
\task{%
    При температуре $25\celsius$ относительная влажность воздуха составляет $70\%$.
    \begin{itemize}
        \item Определите точку росы для этого воздуха.
        \item Какой станет относительная влажность этого воздуха, если нагреть его до $80\celsius$?
    \end{itemize}
}
\answer{%
    \begin{align*}
    &\text{Значения плотности насыщенного водяного пара определяем по таблице:} \\
    &\rho_{\text{нас.
    пара 25} \celsius} = 23{,}000\,\frac{\text{г}}{\text{м}^{3}}, \rho_{\text{нас.
    пара 80} \celsius} = 293{,}000\,\frac{\text{г}}{\text{м}^{3}}.
    \\
    \varphi_1 &= \frac{\rho_\text{пара}}{\rho_{\text{нас.
    пара 25} \celsius}} \implies {\rho_\text{пара}} = \rho_{\text{нас.
    пара 25} \celsius} \cdot \varphi_1 = 23{,}000\,\frac{\text{г}}{\text{м}^{3}} \cdot 0{,}70 = 16{,}100\,\frac{\text{г}}{\text{м}^{3}}.
    \\
    &\text{По таблице определяем, при какой температуре пар с такой плотностью станет насыщенным:}  \\
    t_\text{росы} &= 18{,}8\celsius, \\
    \varphi_2 &= \frac{\rho_\text{пара}}{\rho_{\text{нас.
    пара 80} \celsius}} = \frac{\rho_{\text{нас.
    пара 25} \celsius} \cdot \varphi_1}{\rho_{\text{нас.
    пара 80} \celsius}}= \varphi_1 \cdot \frac{\rho_{\text{нас.
    пара 25} \celsius}}{\rho_{\text{нас.
    пара 80} \celsius}} = 0{,}70 \cdot \frac{23{,}000\,\frac{\text{г}}{\text{м}^{3}}}{293{,}000\,\frac{\text{г}}{\text{м}^{3}}} = 0{,}055 \approx 5{,}5\%.
    \end{align*}
}
\solutionspace{80pt}

\tasknumber{3}%
\task{%
    Из уравнения состояния идеального газа выведите или выразите...
    \begin{enumerate}
        \item объём,
        \item молярную массу,
        \item концентрацию молекул газа.
    \end{enumerate}
}

\tasknumber{4}%
\task{%
    Запишите формулы и рядом с каждой физичической величиной укажите её название и единицы измерения в СИ:
    \begin{enumerate}
        \item первое начало термодинамики,
        \item внутренняя энергия идеального одноатомного газа.
    \end{enumerate}
}

\variantsplitter

\addpersonalvariant{Ирина Ан}

\tasknumber{1}%
\task{%
    Определите КПД (оставив ответ точным в виде нескоратимой дроби) цикла 1231, рабочим телом которого является идеальный одноатомный газ, если
    \begin{itemize}
        \item 12 — изохорический нагрев в два раза,
        \item 23 — изобарическое расширение, при котором температура растёт в шесть раз,
        \item 31 — процесс, график которого в $PV$-координатах является отрезком прямой.
    \end{itemize}
    Бонус: замените цикл 1231 циклом, в котором 12 — изохорический нагрев в два раза, 23 — процесс, график которого в $PV$-координатах является отрезком прямой, 31 — изобарическое охлаждение, при котором температура падает в два раза.
}
\answer{%
    \begin{align*}
    A_{12} &= 0, \Delta U_{12} > 0, \implies Q_{12} = A_{12} + \Delta U_{12} > 0.
    \\
    A_{23} &> 0, \Delta U_{23} > 0, \implies Q_{23} = A_{23} + \Delta U_{23} > 0, \\
    A_{31} &= 0, \Delta U_{31} < 0, \implies Q_{31} = A_{31} + \Delta U_{31} < 0.
    \\
    P_1V_1 &= \nu R T_1, P_2V_2 = \nu R T_2, P_3V_3 = \nu R T_3 \text{ — уравнения состояния идеального газа}, \\
    &\text{Пусть $P_0$, $V_0$, $T_0$ — давление, объём и температура в точке 1 (минимальные во всём цикле):} \\
    P_1 &= P_0, P_2 = P_3, V_1 = V_2 = V_0, \text{остальные соотношения нужно считать} \\
    T_2 &= 2T_1 = 2T_0 \text{(по условию)} \implies \frac{P_2}{P_1} = \frac{P_2V_0}{P_1V_0} = \frac{P_2 V_2}{P_1 V_1}= \frac{\nu R T_2}{\nu R T_1} = \frac{T_2}{T_1} = 2 \implies P_2 = 2 P_1 = 2 P_0, \\
    T_3 &= 6T_2 = 12T_0 \text{(по условию)} \implies \frac{V_3}{V_2} = \frac{P_3V_3}{P_2V_2}= \frac{\nu R T_3}{\nu R T_2} = \frac{T_3}{T_2} = 6 \implies V_3 = 6 V_2 = 6 V_0.
    \\
    A_\text{цикл} &= \frac 12 (6P_0 - P_0)(2V_0 - V_0) = \frac 12 \cdot 5 \cdot P_0V_0, \\
    A_{23} &= 2P_0 \cdot (6V_0 - V_0) = 10P_0V_0, \\
    \Delta U_{23} &= \frac 32 \nu R T_3 - \frac 32 \nu R T_2 = \frac 32 P_3 V_3 - \frac 32 P_2 V_2 = \frac 32 \cdot 2 P_0 \cdot 6 V_0 -  \frac 32 \cdot 2 P_0 \cdot V_0 = \frac 32 \cdot 10 \cdot P_0V_0, \\
    \Delta U_{12} &= \frac 32 \nu R T_2 - \frac 32 \nu R T_1 = \frac 32 P_2 V_2 - \frac 32 P_1 V_1 = \frac 32 \cdot 2 P_0V_0 - \frac 32 P_0V_0 = \frac 32 \cdot 1 \cdot P_0V_0.
    \\
    \eta &= \frac{A_\text{цикл}}{Q_+} = \frac{A_\text{цикл}}{Q_{12} + Q_{23}}  = \frac{A_\text{цикл}}{A_{12} + \Delta U_{12} + A_{23} + \Delta U_{23}} =  \\
     &= \frac{\frac 12 \cdot 5 \cdot P_0V_0}{0 + \frac 32 \cdot 1 \cdot P_0V_0 + 10P_0V_0 + \frac 32 \cdot 10 \cdot P_0V_0} = \frac{\frac 12 \cdot 5}{\frac 32 \cdot 1 + 10 + \frac 32 \cdot 10} = \frac5{53} \approx 0.094.
    \end{align*}


        График процесса не в масштабе (эта часть пока не готова и сделать автоматически аккуратно сложно), но с верными подписями (а для решения этого достаточно):

        \begin{tikzpicture}[thick]
            \draw[-{Latex}] (0, 0) -- (0, 7) node[above left] {$P$};
            \draw[-{Latex}] (0, 0) -- (10, 0) node[right] {$V$};

            \draw[dashed] (0, 2) node[left] {$P_1 = P_0$} -| (3, 0) node[below] {$V_1 = V_2 = V_0$};
            \draw[dashed] (0, 6) node[left] {$P_2 = P_3 = 2P_0$} -| (9, 0) node[below] {$V_3 = 6V_0$};

            \draw (3, 2) node[above left]{1} node[below left]{$T_1 = T_0$}
                   (3, 6) node[below left]{2} node[above]{$T_2 = 2T_0$}
                   (9, 6) node[above right]{3} node[below right]{$T_3 = 12T_0$};
            \draw[midar] (3, 2) -- (3, 6);
            \draw[midar] (3, 6) -- (9, 6);
            \draw[midar] (9, 6) -- (3, 2);
        \end{tikzpicture}

        Решение бонуса:
        \begin{align*}
            A_{12} &= 0, \Delta U_{12} > 0, \implies Q_{12} = A_{12} + \Delta U_{12} > 0, \\
            A_{23} &> 0, \Delta U_{23} \text{ — ничего нельзя сказать, нужно исследовать отдельно}, \\
            A_{31} &< 0, \Delta U_{31} < 0, \implies Q_{31} = A_{31} + \Delta U_{31} < 0.
            \\
        \end{align*}

        Уравнения состояния идеального газа для точек 1, 2, 3: $P_1V_1 = \nu R T_1, P_2V_2 = \nu R T_2, P_3V_3 = \nu R T_3$.
        Пусть $P_0$, $V_0$, $T_0$ — давление, объём и температура в точке 1 (минимальные во всём цикле).

        12 --- изохора, $\frac{P_1V_1}{T_1} = \nu R = \frac{P_2V_2}{T_2}, V_2=V_1=V_0 \implies \frac{P_1}{T_1} =  \frac{P_2}{T_2} \implies P_2 = P_1 \frac{T_2}{T_1} = 2P_0$,

        31 --- изобара, $\frac{P_1V_1}{T_1} = \nu R = \frac{P_3V_3}{T_3}, P_3=P_1=P_0 \implies \frac{V_3}{T_3} =  \frac{V_1}{T_1} \implies V_3 = V_1 \frac{T_3}{T_1} = 2V_0$,

        Таким образом, используя новые обозначения, состояния газа в точках 1, 2 и 3 описываются макропараметрами $(P_0, V_0, T_0), (2P_0, V_0, 2T_0), (P_0, 2V_0, 2T_0)$ соответственно.

        \begin{tikzpicture}[thick]
            \draw[-{Latex}] (0, 0) -- (0, 7) node[above left] {$P$};
            \draw[-{Latex}] (0, 0) -- (10, 0) node[right] {$V$};

            \draw[dashed] (0, 2) node[left] {$P_1 = P_3 = P_0$} -| (9, 0) node[below] {$V_3 = 2V_0$};
            \draw[dashed] (0, 6) node[left] {$P_2 = 2P_0$} -| (3, 0) node[below] {$V_1 = V_2 = V_0$};

            \draw[dashed] (0, 5) node[left] {$P$} -| (4.5, 0) node[below] {$V$};
            \draw[dashed] (0, 4.6) node[left] {$P'$} -| (5.1, 0) node[below] {$V'$};

            \draw (3, 2) node[above left]{1} node[below left]{$T_1 = T_0$}
                   (3, 6) node[below left]{2} node[above]{$T_2 = 2T_0$}
                   (9, 2) node[above right]{3} node[below right]{$T_3 = 2T_0$};
            \draw[midar] (3, 2) -- (3, 6);
            \draw[midar] (3, 6) -- (9, 2);
            \draw[midar] (9, 2) -- (3, 2);
            \draw   (4.5, 5) node[above right]{$T$} (5.1, 4.6) node[above right]{$T'$};
        \end{tikzpicture}


        Теперь рассмотрим отдельно процесс 23, к остальному вернёмся позже.
        Уравнение этой прямой в $PV$-координатах: $P(V) = 3P_0 - \frac{P_0}{V_0} V$.
        Это значит, что при изменении объёма на $\Delta V$ давление изменится на $\Delta P = - \frac{P_0}{V_0} \Delta V$, обратите внимание на знак.

        Рассмотрим произвольную точку в процессе 23 и дадим процессу ещё немного свершиться, при этом объём изменится на $\Delta V$, давление — на $\Delta P$, температура (иначе бы была гипербола, а не прямая) — на $\Delta T$,
        т.е.
        из состояния $(P, V, T)$ мы перешли в $(P', V', T')$, причём  $P' = P + \Delta P, V' = V + \Delta V, T' = T + \Delta T$.

        При этом изменится внутренняя энергия:
        \begin{align*}
        \Delta U
            &= U' - U = \frac 32 \nu R T' - \frac 32 \nu R T = \frac 32 (P+\Delta P) (V+\Delta V) - \frac 32 PV\\
            &= \frac 32 ((P+\Delta P) (V+\Delta V) - PV) = \frac 32 (P\Delta V + V \Delta P + \Delta P \Delta V).
        \end{align*}

        Рассмотрим малые изменения объёма, тогда и изменение давления будем малым (т.к.
        $\Delta P = - \frac{P_0}{V_0} \Delta V$),
        а третьим слагаемым в выражении для $\Delta U$  можно пренебречь по сравнению с двумя другими:
        два первых это малые величины, а третье — произведение двух малых.
        Тогда $\Delta U = \frac 32 (P\Delta V + V \Delta P)$.

        Работа газа при этом малом изменении объёма — это площадь трапеции (тут ещё раз пренебрегли малым слагаемым):
        $$A = \frac{P + P'}2 \Delta V = \cbr{P + \frac{\Delta P}2} \Delta V = P \Delta V.$$

        Подведённое количество теплоты, используя первое начало термодинамики, будет равно
        \begin{align*}
        Q
            &= \frac 32 (P\Delta V + V \Delta P) + P \Delta V =  \frac 52 P\Delta V + \frac 32 V \Delta P = \\
            &= \frac 52 P\Delta V + \frac 32 V \cdot \cbr{- \frac{P_0}{V_0} \Delta V} = \frac{\Delta V}2 \cdot \cbr{5P - \frac{P_0}{V_0} V} = \\
            &= \frac{\Delta V}2 \cdot \cbr{5 \cdot \cbr{3P_0 - \frac{P_0}{V_0} V} - \frac{P_0}{V_0} V}
             = \frac{\Delta V \cdot P_0}2 \cdot \cbr{5 \cdot 3 - 8\frac V{V_0}}.
        \end{align*}

        Таком образом, знак количества теплоты $Q$ на участке 23 зависит от конкретного значения $V$:
        \begin{itemize}
            \item $\Delta V > 0$ на всём участке 23, поскольку газ расширяется,
            \item $P > 0$ — всегда, у нас идеальный газ, удары о стенки сосуда абсолютно упругие, а молекулы не взаимодействуют и поэтому давление только положительно,
            \item если $5 \cdot 3 - 8\frac V{V_0} > 0$ — тепло подводят, если же меньше нуля — отводят.
        \end{itemize}
        Решая последнее неравенство, получаем конкретное значение $V^*$: при $V < V^*$ тепло подводят, далее~— отводят.
        Тут *~--- некоторая точка между точками 2 и 3, конкретные значения надо досчитать:
        $$V^* = V_0 \cdot \frac{5 \cdot 3}8 = \frac{15}8 \cdot V_0 \implies P^* = 3P_0 - \frac{P_0}{V_0} V^* = \ldots = \frac98 \cdot P_0.$$

        Т.е.
        чтобы вычислить $Q_+$, надо сложить количества теплоты на участке 12 и лишь части участка 23 — участке 2*,
        той его части где это количество теплоты положительно.
        Имеем: $Q_+ = Q_{12} + Q_{2*}$.

        Теперь возвращаемся к циклу целиком и получаем:
        \begin{align*}
        A_\text{цикл} &= \frac 12 \cdot (2P_0 - P_0) \cdot (2V_0 - V_0) = \frac12 \cdot P_0V_0, \\
        A_{2*} &= \frac{P^* + 2P_0}2 \cdot (V^* - V_0)
            = \frac{\frac98 \cdot P_0 + 2P_0}2 \cdot \cbr{\frac{15}8 \cdot V_0 - V_0}
            = \ldots = \frac{175}{128} \cdot P_0 V_0, \\
        \Delta U_{2*} &= \frac 32 \nu R T^* - \frac 32 \nu R T_2 = \frac 32 (P^*V^* - P_0 \cdot 2V_0)
            = \frac 32 \cbr{\frac98 \cdot P_0 \cdot \frac{15}8 \cdot V_0 - P_0 \cdot 2V_0}
            = \frac{21}{128} \cdot P_0 V_0, \\
        \Delta U_{12} &= \frac 32 \nu R T_2 - \frac 32 \nu R T_1 = \frac 32 (2P_0V_0 - P_0V_0) = \ldots = \frac32 \cdot P_0 V_0, \\
        \eta &= \frac{A_\text{цикл}}{Q_+} = \frac{A_\text{цикл}}{Q_{12} + Q_{2*}}
            = \frac{A_\text{цикл}}{A_{12} + \Delta U_{12} + A_{2*} + \Delta U_{2*}} = \\
            &= \frac{\frac12 \cdot P_0V_0}{0 + \frac32 \cdot P_0 V_0 + \frac{175}{128} \cdot P_0 V_0 + \frac{21}{128} \cdot P_0 V_0}
             = \frac{A_bonus_cycle:LaTeX}{\frac32 + \frac{175}{128} + \frac{21}{128}}
             = \frac{16}{97} \leftarrow \text{вжух и готово!}
        \end{align*}
}
\solutionspace{360pt}

\tasknumber{2}%
\task{%
    При температуре $30\celsius$ относительная влажность воздуха составляет $70\%$.
    \begin{itemize}
        \item Определите точку росы для этого воздуха.
        \item Какой станет относительная влажность этого воздуха, если нагреть его до $60\celsius$?
    \end{itemize}
}
\answer{%
    \begin{align*}
    &\text{Значения плотности насыщенного водяного пара определяем по таблице:} \\
    &\rho_{\text{нас.
    пара 30} \celsius} = 30{,}300\,\frac{\text{г}}{\text{м}^{3}}, \rho_{\text{нас.
    пара 60} \celsius} = 130{,}000\,\frac{\text{г}}{\text{м}^{3}}.
    \\
    \varphi_1 &= \frac{\rho_\text{пара}}{\rho_{\text{нас.
    пара 30} \celsius}} \implies {\rho_\text{пара}} = \rho_{\text{нас.
    пара 30} \celsius} \cdot \varphi_1 = 30{,}300\,\frac{\text{г}}{\text{м}^{3}} \cdot 0{,}70 = 21{,}210\,\frac{\text{г}}{\text{м}^{3}}.
    \\
    &\text{По таблице определяем, при какой температуре пар с такой плотностью станет насыщенным:}  \\
    t_\text{росы} &= 23{,}5\celsius, \\
    \varphi_2 &= \frac{\rho_\text{пара}}{\rho_{\text{нас.
    пара 60} \celsius}} = \frac{\rho_{\text{нас.
    пара 30} \celsius} \cdot \varphi_1}{\rho_{\text{нас.
    пара 60} \celsius}}= \varphi_1 \cdot \frac{\rho_{\text{нас.
    пара 30} \celsius}}{\rho_{\text{нас.
    пара 60} \celsius}} = 0{,}70 \cdot \frac{30{,}300\,\frac{\text{г}}{\text{м}^{3}}}{130{,}000\,\frac{\text{г}}{\text{м}^{3}}} = 0{,}163 \approx 16{,}3\%.
    \end{align*}
}
\solutionspace{80pt}

\tasknumber{3}%
\task{%
    Из уравнения состояния идеального газа выведите или выразите...
    \begin{enumerate}
        \item объём,
        \item молярную массу,
        \item концентрацию молекул газа.
    \end{enumerate}
}

\tasknumber{4}%
\task{%
    Запишите формулы и рядом с каждой физичической величиной укажите её название и единицы измерения в СИ:
    \begin{enumerate}
        \item первое начало термодинамики,
        \item внутренняя энергия идеального одноатомного газа.
    \end{enumerate}
}

\variantsplitter

\addpersonalvariant{Софья Андрианова}

\tasknumber{1}%
\task{%
    Определите КПД (оставив ответ точным в виде нескоратимой дроби) цикла 1231, рабочим телом которого является идеальный одноатомный газ, если
    \begin{itemize}
        \item 12 — изохорический нагрев в шесть раз,
        \item 23 — изобарическое расширение, при котором температура растёт в шесть раз,
        \item 31 — процесс, график которого в $PV$-координатах является отрезком прямой.
    \end{itemize}
    Бонус: замените цикл 1231 циклом, в котором 12 — изохорический нагрев в шесть раз, 23 — процесс, график которого в $PV$-координатах является отрезком прямой, 31 — изобарическое охлаждение, при котором температура падает в шесть раз.
}
\answer{%
    \begin{align*}
    A_{12} &= 0, \Delta U_{12} > 0, \implies Q_{12} = A_{12} + \Delta U_{12} > 0.
    \\
    A_{23} &> 0, \Delta U_{23} > 0, \implies Q_{23} = A_{23} + \Delta U_{23} > 0, \\
    A_{31} &= 0, \Delta U_{31} < 0, \implies Q_{31} = A_{31} + \Delta U_{31} < 0.
    \\
    P_1V_1 &= \nu R T_1, P_2V_2 = \nu R T_2, P_3V_3 = \nu R T_3 \text{ — уравнения состояния идеального газа}, \\
    &\text{Пусть $P_0$, $V_0$, $T_0$ — давление, объём и температура в точке 1 (минимальные во всём цикле):} \\
    P_1 &= P_0, P_2 = P_3, V_1 = V_2 = V_0, \text{остальные соотношения нужно считать} \\
    T_2 &= 6T_1 = 6T_0 \text{(по условию)} \implies \frac{P_2}{P_1} = \frac{P_2V_0}{P_1V_0} = \frac{P_2 V_2}{P_1 V_1}= \frac{\nu R T_2}{\nu R T_1} = \frac{T_2}{T_1} = 6 \implies P_2 = 6 P_1 = 6 P_0, \\
    T_3 &= 6T_2 = 36T_0 \text{(по условию)} \implies \frac{V_3}{V_2} = \frac{P_3V_3}{P_2V_2}= \frac{\nu R T_3}{\nu R T_2} = \frac{T_3}{T_2} = 6 \implies V_3 = 6 V_2 = 6 V_0.
    \\
    A_\text{цикл} &= \frac 12 (6P_0 - P_0)(6V_0 - V_0) = \frac 12 \cdot 25 \cdot P_0V_0, \\
    A_{23} &= 6P_0 \cdot (6V_0 - V_0) = 30P_0V_0, \\
    \Delta U_{23} &= \frac 32 \nu R T_3 - \frac 32 \nu R T_2 = \frac 32 P_3 V_3 - \frac 32 P_2 V_2 = \frac 32 \cdot 6 P_0 \cdot 6 V_0 -  \frac 32 \cdot 6 P_0 \cdot V_0 = \frac 32 \cdot 30 \cdot P_0V_0, \\
    \Delta U_{12} &= \frac 32 \nu R T_2 - \frac 32 \nu R T_1 = \frac 32 P_2 V_2 - \frac 32 P_1 V_1 = \frac 32 \cdot 6 P_0V_0 - \frac 32 P_0V_0 = \frac 32 \cdot 5 \cdot P_0V_0.
    \\
    \eta &= \frac{A_\text{цикл}}{Q_+} = \frac{A_\text{цикл}}{Q_{12} + Q_{23}}  = \frac{A_\text{цикл}}{A_{12} + \Delta U_{12} + A_{23} + \Delta U_{23}} =  \\
     &= \frac{\frac 12 \cdot 25 \cdot P_0V_0}{0 + \frac 32 \cdot 5 \cdot P_0V_0 + 30P_0V_0 + \frac 32 \cdot 30 \cdot P_0V_0} = \frac{\frac 12 \cdot 25}{\frac 32 \cdot 5 + 30 + \frac 32 \cdot 30} = \frac5{33} \approx 0.152.
    \end{align*}


        График процесса не в масштабе (эта часть пока не готова и сделать автоматически аккуратно сложно), но с верными подписями (а для решения этого достаточно):

        \begin{tikzpicture}[thick]
            \draw[-{Latex}] (0, 0) -- (0, 7) node[above left] {$P$};
            \draw[-{Latex}] (0, 0) -- (10, 0) node[right] {$V$};

            \draw[dashed] (0, 2) node[left] {$P_1 = P_0$} -| (3, 0) node[below] {$V_1 = V_2 = V_0$};
            \draw[dashed] (0, 6) node[left] {$P_2 = P_3 = 6P_0$} -| (9, 0) node[below] {$V_3 = 6V_0$};

            \draw (3, 2) node[above left]{1} node[below left]{$T_1 = T_0$}
                   (3, 6) node[below left]{2} node[above]{$T_2 = 6T_0$}
                   (9, 6) node[above right]{3} node[below right]{$T_3 = 36T_0$};
            \draw[midar] (3, 2) -- (3, 6);
            \draw[midar] (3, 6) -- (9, 6);
            \draw[midar] (9, 6) -- (3, 2);
        \end{tikzpicture}

        Решение бонуса:
        \begin{align*}
            A_{12} &= 0, \Delta U_{12} > 0, \implies Q_{12} = A_{12} + \Delta U_{12} > 0, \\
            A_{23} &> 0, \Delta U_{23} \text{ — ничего нельзя сказать, нужно исследовать отдельно}, \\
            A_{31} &< 0, \Delta U_{31} < 0, \implies Q_{31} = A_{31} + \Delta U_{31} < 0.
            \\
        \end{align*}

        Уравнения состояния идеального газа для точек 1, 2, 3: $P_1V_1 = \nu R T_1, P_2V_2 = \nu R T_2, P_3V_3 = \nu R T_3$.
        Пусть $P_0$, $V_0$, $T_0$ — давление, объём и температура в точке 1 (минимальные во всём цикле).

        12 --- изохора, $\frac{P_1V_1}{T_1} = \nu R = \frac{P_2V_2}{T_2}, V_2=V_1=V_0 \implies \frac{P_1}{T_1} =  \frac{P_2}{T_2} \implies P_2 = P_1 \frac{T_2}{T_1} = 6P_0$,

        31 --- изобара, $\frac{P_1V_1}{T_1} = \nu R = \frac{P_3V_3}{T_3}, P_3=P_1=P_0 \implies \frac{V_3}{T_3} =  \frac{V_1}{T_1} \implies V_3 = V_1 \frac{T_3}{T_1} = 6V_0$,

        Таким образом, используя новые обозначения, состояния газа в точках 1, 2 и 3 описываются макропараметрами $(P_0, V_0, T_0), (6P_0, V_0, 6T_0), (P_0, 6V_0, 6T_0)$ соответственно.

        \begin{tikzpicture}[thick]
            \draw[-{Latex}] (0, 0) -- (0, 7) node[above left] {$P$};
            \draw[-{Latex}] (0, 0) -- (10, 0) node[right] {$V$};

            \draw[dashed] (0, 2) node[left] {$P_1 = P_3 = P_0$} -| (9, 0) node[below] {$V_3 = 6V_0$};
            \draw[dashed] (0, 6) node[left] {$P_2 = 6P_0$} -| (3, 0) node[below] {$V_1 = V_2 = V_0$};

            \draw[dashed] (0, 5) node[left] {$P$} -| (4.5, 0) node[below] {$V$};
            \draw[dashed] (0, 4.6) node[left] {$P'$} -| (5.1, 0) node[below] {$V'$};

            \draw (3, 2) node[above left]{1} node[below left]{$T_1 = T_0$}
                   (3, 6) node[below left]{2} node[above]{$T_2 = 6T_0$}
                   (9, 2) node[above right]{3} node[below right]{$T_3 = 6T_0$};
            \draw[midar] (3, 2) -- (3, 6);
            \draw[midar] (3, 6) -- (9, 2);
            \draw[midar] (9, 2) -- (3, 2);
            \draw   (4.5, 5) node[above right]{$T$} (5.1, 4.6) node[above right]{$T'$};
        \end{tikzpicture}


        Теперь рассмотрим отдельно процесс 23, к остальному вернёмся позже.
        Уравнение этой прямой в $PV$-координатах: $P(V) = 7P_0 - \frac{P_0}{V_0} V$.
        Это значит, что при изменении объёма на $\Delta V$ давление изменится на $\Delta P = - \frac{P_0}{V_0} \Delta V$, обратите внимание на знак.

        Рассмотрим произвольную точку в процессе 23 и дадим процессу ещё немного свершиться, при этом объём изменится на $\Delta V$, давление — на $\Delta P$, температура (иначе бы была гипербола, а не прямая) — на $\Delta T$,
        т.е.
        из состояния $(P, V, T)$ мы перешли в $(P', V', T')$, причём  $P' = P + \Delta P, V' = V + \Delta V, T' = T + \Delta T$.

        При этом изменится внутренняя энергия:
        \begin{align*}
        \Delta U
            &= U' - U = \frac 32 \nu R T' - \frac 32 \nu R T = \frac 32 (P+\Delta P) (V+\Delta V) - \frac 32 PV\\
            &= \frac 32 ((P+\Delta P) (V+\Delta V) - PV) = \frac 32 (P\Delta V + V \Delta P + \Delta P \Delta V).
        \end{align*}

        Рассмотрим малые изменения объёма, тогда и изменение давления будем малым (т.к.
        $\Delta P = - \frac{P_0}{V_0} \Delta V$),
        а третьим слагаемым в выражении для $\Delta U$  можно пренебречь по сравнению с двумя другими:
        два первых это малые величины, а третье — произведение двух малых.
        Тогда $\Delta U = \frac 32 (P\Delta V + V \Delta P)$.

        Работа газа при этом малом изменении объёма — это площадь трапеции (тут ещё раз пренебрегли малым слагаемым):
        $$A = \frac{P + P'}2 \Delta V = \cbr{P + \frac{\Delta P}2} \Delta V = P \Delta V.$$

        Подведённое количество теплоты, используя первое начало термодинамики, будет равно
        \begin{align*}
        Q
            &= \frac 32 (P\Delta V + V \Delta P) + P \Delta V =  \frac 52 P\Delta V + \frac 32 V \Delta P = \\
            &= \frac 52 P\Delta V + \frac 32 V \cdot \cbr{- \frac{P_0}{V_0} \Delta V} = \frac{\Delta V}2 \cdot \cbr{5P - \frac{P_0}{V_0} V} = \\
            &= \frac{\Delta V}2 \cdot \cbr{5 \cdot \cbr{7P_0 - \frac{P_0}{V_0} V} - \frac{P_0}{V_0} V}
             = \frac{\Delta V \cdot P_0}2 \cdot \cbr{5 \cdot 7 - 8\frac V{V_0}}.
        \end{align*}

        Таком образом, знак количества теплоты $Q$ на участке 23 зависит от конкретного значения $V$:
        \begin{itemize}
            \item $\Delta V > 0$ на всём участке 23, поскольку газ расширяется,
            \item $P > 0$ — всегда, у нас идеальный газ, удары о стенки сосуда абсолютно упругие, а молекулы не взаимодействуют и поэтому давление только положительно,
            \item если $5 \cdot 7 - 8\frac V{V_0} > 0$ — тепло подводят, если же меньше нуля — отводят.
        \end{itemize}
        Решая последнее неравенство, получаем конкретное значение $V^*$: при $V < V^*$ тепло подводят, далее~— отводят.
        Тут *~--- некоторая точка между точками 2 и 3, конкретные значения надо досчитать:
        $$V^* = V_0 \cdot \frac{5 \cdot 7}8 = \frac{35}8 \cdot V_0 \implies P^* = 7P_0 - \frac{P_0}{V_0} V^* = \ldots = \frac{21}8 \cdot P_0.$$

        Т.е.
        чтобы вычислить $Q_+$, надо сложить количества теплоты на участке 12 и лишь части участка 23 — участке 2*,
        той его части где это количество теплоты положительно.
        Имеем: $Q_+ = Q_{12} + Q_{2*}$.

        Теперь возвращаемся к циклу целиком и получаем:
        \begin{align*}
        A_\text{цикл} &= \frac 12 \cdot (6P_0 - P_0) \cdot (6V_0 - V_0) = \frac{25}2 \cdot P_0V_0, \\
        A_{2*} &= \frac{P^* + 6P_0}2 \cdot (V^* - V_0)
            = \frac{\frac{21}8 \cdot P_0 + 6P_0}2 \cdot \cbr{\frac{35}8 \cdot V_0 - V_0}
            = \ldots = \frac{1863}{128} \cdot P_0 V_0, \\
        \Delta U_{2*} &= \frac 32 \nu R T^* - \frac 32 \nu R T_2 = \frac 32 (P^*V^* - P_0 \cdot 6V_0)
            = \frac 32 \cbr{\frac{21}8 \cdot P_0 \cdot \frac{35}8 \cdot V_0 - P_0 \cdot 6V_0}
            = \frac{1053}{128} \cdot P_0 V_0, \\
        \Delta U_{12} &= \frac 32 \nu R T_2 - \frac 32 \nu R T_1 = \frac 32 (6P_0V_0 - P_0V_0) = \ldots = \frac{15}2 \cdot P_0 V_0, \\
        \eta &= \frac{A_\text{цикл}}{Q_+} = \frac{A_\text{цикл}}{Q_{12} + Q_{2*}}
            = \frac{A_\text{цикл}}{A_{12} + \Delta U_{12} + A_{2*} + \Delta U_{2*}} = \\
            &= \frac{\frac{25}2 \cdot P_0V_0}{0 + \frac{15}2 \cdot P_0 V_0 + \frac{1863}{128} \cdot P_0 V_0 + \frac{1053}{128} \cdot P_0 V_0}
             = \frac{A_bonus_cycle:LaTeX}{\frac{15}2 + \frac{1863}{128} + \frac{1053}{128}}
             = \frac{400}{969} \leftarrow \text{вжух и готово!}
        \end{align*}
}
\solutionspace{360pt}

\tasknumber{2}%
\task{%
    При температуре $15\celsius$ относительная влажность воздуха составляет $75\%$.
    \begin{itemize}
        \item Определите точку росы для этого воздуха.
        \item Какой станет относительная влажность этого воздуха, если нагреть его до $70\celsius$?
    \end{itemize}
}
\answer{%
    \begin{align*}
    &\text{Значения плотности насыщенного водяного пара определяем по таблице:} \\
    &\rho_{\text{нас.
    пара 15} \celsius} = 12{,}800\,\frac{\text{г}}{\text{м}^{3}}, \rho_{\text{нас.
    пара 70} \celsius} = 198{,}000\,\frac{\text{г}}{\text{м}^{3}}.
    \\
    \varphi_1 &= \frac{\rho_\text{пара}}{\rho_{\text{нас.
    пара 15} \celsius}} \implies {\rho_\text{пара}} = \rho_{\text{нас.
    пара 15} \celsius} \cdot \varphi_1 = 12{,}800\,\frac{\text{г}}{\text{м}^{3}} \cdot 0{,}75 = 9{,}600\,\frac{\text{г}}{\text{м}^{3}}.
    \\
    &\text{По таблице определяем, при какой температуре пар с такой плотностью станет насыщенным:}  \\
    t_\text{росы} &= 10{,}3\celsius, \\
    \varphi_2 &= \frac{\rho_\text{пара}}{\rho_{\text{нас.
    пара 70} \celsius}} = \frac{\rho_{\text{нас.
    пара 15} \celsius} \cdot \varphi_1}{\rho_{\text{нас.
    пара 70} \celsius}}= \varphi_1 \cdot \frac{\rho_{\text{нас.
    пара 15} \celsius}}{\rho_{\text{нас.
    пара 70} \celsius}} = 0{,}75 \cdot \frac{12{,}800\,\frac{\text{г}}{\text{м}^{3}}}{198{,}000\,\frac{\text{г}}{\text{м}^{3}}} = 0{,}048 \approx 4{,}8\%.
    \end{align*}
}
\solutionspace{80pt}

\tasknumber{3}%
\task{%
    Из уравнения состояния идеального газа выведите или выразите...
    \begin{enumerate}
        \item давление,
        \item молярную массу,
        \item плотность газа.
    \end{enumerate}
}

\tasknumber{4}%
\task{%
    Запишите формулы и рядом с каждой физичической величиной укажите её название и единицы измерения в СИ:
    \begin{enumerate}
        \item первое начало термодинамики,
        \item внутренняя энергия идеального одноатомного газа.
    \end{enumerate}
}

\variantsplitter

\addpersonalvariant{Владимир Артемчук}

\tasknumber{1}%
\task{%
    Определите КПД (оставив ответ точным в виде нескоратимой дроби) цикла 1231, рабочим телом которого является идеальный одноатомный газ, если
    \begin{itemize}
        \item 12 — изохорический нагрев в пять раз,
        \item 23 — изобарическое расширение, при котором температура растёт в три раза,
        \item 31 — процесс, график которого в $PV$-координатах является отрезком прямой.
    \end{itemize}
    Бонус: замените цикл 1231 циклом, в котором 12 — изохорический нагрев в пять раз, 23 — процесс, график которого в $PV$-координатах является отрезком прямой, 31 — изобарическое охлаждение, при котором температура падает в пять раз.
}
\answer{%
    \begin{align*}
    A_{12} &= 0, \Delta U_{12} > 0, \implies Q_{12} = A_{12} + \Delta U_{12} > 0.
    \\
    A_{23} &> 0, \Delta U_{23} > 0, \implies Q_{23} = A_{23} + \Delta U_{23} > 0, \\
    A_{31} &= 0, \Delta U_{31} < 0, \implies Q_{31} = A_{31} + \Delta U_{31} < 0.
    \\
    P_1V_1 &= \nu R T_1, P_2V_2 = \nu R T_2, P_3V_3 = \nu R T_3 \text{ — уравнения состояния идеального газа}, \\
    &\text{Пусть $P_0$, $V_0$, $T_0$ — давление, объём и температура в точке 1 (минимальные во всём цикле):} \\
    P_1 &= P_0, P_2 = P_3, V_1 = V_2 = V_0, \text{остальные соотношения нужно считать} \\
    T_2 &= 5T_1 = 5T_0 \text{(по условию)} \implies \frac{P_2}{P_1} = \frac{P_2V_0}{P_1V_0} = \frac{P_2 V_2}{P_1 V_1}= \frac{\nu R T_2}{\nu R T_1} = \frac{T_2}{T_1} = 5 \implies P_2 = 5 P_1 = 5 P_0, \\
    T_3 &= 3T_2 = 15T_0 \text{(по условию)} \implies \frac{V_3}{V_2} = \frac{P_3V_3}{P_2V_2}= \frac{\nu R T_3}{\nu R T_2} = \frac{T_3}{T_2} = 3 \implies V_3 = 3 V_2 = 3 V_0.
    \\
    A_\text{цикл} &= \frac 12 (3P_0 - P_0)(5V_0 - V_0) = \frac 12 \cdot 8 \cdot P_0V_0, \\
    A_{23} &= 5P_0 \cdot (3V_0 - V_0) = 10P_0V_0, \\
    \Delta U_{23} &= \frac 32 \nu R T_3 - \frac 32 \nu R T_2 = \frac 32 P_3 V_3 - \frac 32 P_2 V_2 = \frac 32 \cdot 5 P_0 \cdot 3 V_0 -  \frac 32 \cdot 5 P_0 \cdot V_0 = \frac 32 \cdot 10 \cdot P_0V_0, \\
    \Delta U_{12} &= \frac 32 \nu R T_2 - \frac 32 \nu R T_1 = \frac 32 P_2 V_2 - \frac 32 P_1 V_1 = \frac 32 \cdot 5 P_0V_0 - \frac 32 P_0V_0 = \frac 32 \cdot 4 \cdot P_0V_0.
    \\
    \eta &= \frac{A_\text{цикл}}{Q_+} = \frac{A_\text{цикл}}{Q_{12} + Q_{23}}  = \frac{A_\text{цикл}}{A_{12} + \Delta U_{12} + A_{23} + \Delta U_{23}} =  \\
     &= \frac{\frac 12 \cdot 8 \cdot P_0V_0}{0 + \frac 32 \cdot 4 \cdot P_0V_0 + 10P_0V_0 + \frac 32 \cdot 10 \cdot P_0V_0} = \frac{\frac 12 \cdot 8}{\frac 32 \cdot 4 + 10 + \frac 32 \cdot 10} = \frac4{31} \approx 0.129.
    \end{align*}


        График процесса не в масштабе (эта часть пока не готова и сделать автоматически аккуратно сложно), но с верными подписями (а для решения этого достаточно):

        \begin{tikzpicture}[thick]
            \draw[-{Latex}] (0, 0) -- (0, 7) node[above left] {$P$};
            \draw[-{Latex}] (0, 0) -- (10, 0) node[right] {$V$};

            \draw[dashed] (0, 2) node[left] {$P_1 = P_0$} -| (3, 0) node[below] {$V_1 = V_2 = V_0$};
            \draw[dashed] (0, 6) node[left] {$P_2 = P_3 = 5P_0$} -| (9, 0) node[below] {$V_3 = 3V_0$};

            \draw (3, 2) node[above left]{1} node[below left]{$T_1 = T_0$}
                   (3, 6) node[below left]{2} node[above]{$T_2 = 5T_0$}
                   (9, 6) node[above right]{3} node[below right]{$T_3 = 15T_0$};
            \draw[midar] (3, 2) -- (3, 6);
            \draw[midar] (3, 6) -- (9, 6);
            \draw[midar] (9, 6) -- (3, 2);
        \end{tikzpicture}

        Решение бонуса:
        \begin{align*}
            A_{12} &= 0, \Delta U_{12} > 0, \implies Q_{12} = A_{12} + \Delta U_{12} > 0, \\
            A_{23} &> 0, \Delta U_{23} \text{ — ничего нельзя сказать, нужно исследовать отдельно}, \\
            A_{31} &< 0, \Delta U_{31} < 0, \implies Q_{31} = A_{31} + \Delta U_{31} < 0.
            \\
        \end{align*}

        Уравнения состояния идеального газа для точек 1, 2, 3: $P_1V_1 = \nu R T_1, P_2V_2 = \nu R T_2, P_3V_3 = \nu R T_3$.
        Пусть $P_0$, $V_0$, $T_0$ — давление, объём и температура в точке 1 (минимальные во всём цикле).

        12 --- изохора, $\frac{P_1V_1}{T_1} = \nu R = \frac{P_2V_2}{T_2}, V_2=V_1=V_0 \implies \frac{P_1}{T_1} =  \frac{P_2}{T_2} \implies P_2 = P_1 \frac{T_2}{T_1} = 5P_0$,

        31 --- изобара, $\frac{P_1V_1}{T_1} = \nu R = \frac{P_3V_3}{T_3}, P_3=P_1=P_0 \implies \frac{V_3}{T_3} =  \frac{V_1}{T_1} \implies V_3 = V_1 \frac{T_3}{T_1} = 5V_0$,

        Таким образом, используя новые обозначения, состояния газа в точках 1, 2 и 3 описываются макропараметрами $(P_0, V_0, T_0), (5P_0, V_0, 5T_0), (P_0, 5V_0, 5T_0)$ соответственно.

        \begin{tikzpicture}[thick]
            \draw[-{Latex}] (0, 0) -- (0, 7) node[above left] {$P$};
            \draw[-{Latex}] (0, 0) -- (10, 0) node[right] {$V$};

            \draw[dashed] (0, 2) node[left] {$P_1 = P_3 = P_0$} -| (9, 0) node[below] {$V_3 = 5V_0$};
            \draw[dashed] (0, 6) node[left] {$P_2 = 5P_0$} -| (3, 0) node[below] {$V_1 = V_2 = V_0$};

            \draw[dashed] (0, 5) node[left] {$P$} -| (4.5, 0) node[below] {$V$};
            \draw[dashed] (0, 4.6) node[left] {$P'$} -| (5.1, 0) node[below] {$V'$};

            \draw (3, 2) node[above left]{1} node[below left]{$T_1 = T_0$}
                   (3, 6) node[below left]{2} node[above]{$T_2 = 5T_0$}
                   (9, 2) node[above right]{3} node[below right]{$T_3 = 5T_0$};
            \draw[midar] (3, 2) -- (3, 6);
            \draw[midar] (3, 6) -- (9, 2);
            \draw[midar] (9, 2) -- (3, 2);
            \draw   (4.5, 5) node[above right]{$T$} (5.1, 4.6) node[above right]{$T'$};
        \end{tikzpicture}


        Теперь рассмотрим отдельно процесс 23, к остальному вернёмся позже.
        Уравнение этой прямой в $PV$-координатах: $P(V) = 6P_0 - \frac{P_0}{V_0} V$.
        Это значит, что при изменении объёма на $\Delta V$ давление изменится на $\Delta P = - \frac{P_0}{V_0} \Delta V$, обратите внимание на знак.

        Рассмотрим произвольную точку в процессе 23 и дадим процессу ещё немного свершиться, при этом объём изменится на $\Delta V$, давление — на $\Delta P$, температура (иначе бы была гипербола, а не прямая) — на $\Delta T$,
        т.е.
        из состояния $(P, V, T)$ мы перешли в $(P', V', T')$, причём  $P' = P + \Delta P, V' = V + \Delta V, T' = T + \Delta T$.

        При этом изменится внутренняя энергия:
        \begin{align*}
        \Delta U
            &= U' - U = \frac 32 \nu R T' - \frac 32 \nu R T = \frac 32 (P+\Delta P) (V+\Delta V) - \frac 32 PV\\
            &= \frac 32 ((P+\Delta P) (V+\Delta V) - PV) = \frac 32 (P\Delta V + V \Delta P + \Delta P \Delta V).
        \end{align*}

        Рассмотрим малые изменения объёма, тогда и изменение давления будем малым (т.к.
        $\Delta P = - \frac{P_0}{V_0} \Delta V$),
        а третьим слагаемым в выражении для $\Delta U$  можно пренебречь по сравнению с двумя другими:
        два первых это малые величины, а третье — произведение двух малых.
        Тогда $\Delta U = \frac 32 (P\Delta V + V \Delta P)$.

        Работа газа при этом малом изменении объёма — это площадь трапеции (тут ещё раз пренебрегли малым слагаемым):
        $$A = \frac{P + P'}2 \Delta V = \cbr{P + \frac{\Delta P}2} \Delta V = P \Delta V.$$

        Подведённое количество теплоты, используя первое начало термодинамики, будет равно
        \begin{align*}
        Q
            &= \frac 32 (P\Delta V + V \Delta P) + P \Delta V =  \frac 52 P\Delta V + \frac 32 V \Delta P = \\
            &= \frac 52 P\Delta V + \frac 32 V \cdot \cbr{- \frac{P_0}{V_0} \Delta V} = \frac{\Delta V}2 \cdot \cbr{5P - \frac{P_0}{V_0} V} = \\
            &= \frac{\Delta V}2 \cdot \cbr{5 \cdot \cbr{6P_0 - \frac{P_0}{V_0} V} - \frac{P_0}{V_0} V}
             = \frac{\Delta V \cdot P_0}2 \cdot \cbr{5 \cdot 6 - 8\frac V{V_0}}.
        \end{align*}

        Таком образом, знак количества теплоты $Q$ на участке 23 зависит от конкретного значения $V$:
        \begin{itemize}
            \item $\Delta V > 0$ на всём участке 23, поскольку газ расширяется,
            \item $P > 0$ — всегда, у нас идеальный газ, удары о стенки сосуда абсолютно упругие, а молекулы не взаимодействуют и поэтому давление только положительно,
            \item если $5 \cdot 6 - 8\frac V{V_0} > 0$ — тепло подводят, если же меньше нуля — отводят.
        \end{itemize}
        Решая последнее неравенство, получаем конкретное значение $V^*$: при $V < V^*$ тепло подводят, далее~— отводят.
        Тут *~--- некоторая точка между точками 2 и 3, конкретные значения надо досчитать:
        $$V^* = V_0 \cdot \frac{5 \cdot 6}8 = \frac{15}4 \cdot V_0 \implies P^* = 6P_0 - \frac{P_0}{V_0} V^* = \ldots = \frac94 \cdot P_0.$$

        Т.е.
        чтобы вычислить $Q_+$, надо сложить количества теплоты на участке 12 и лишь части участка 23 — участке 2*,
        той его части где это количество теплоты положительно.
        Имеем: $Q_+ = Q_{12} + Q_{2*}$.

        Теперь возвращаемся к циклу целиком и получаем:
        \begin{align*}
        A_\text{цикл} &= \frac 12 \cdot (5P_0 - P_0) \cdot (5V_0 - V_0) = 8 \cdot P_0V_0, \\
        A_{2*} &= \frac{P^* + 5P_0}2 \cdot (V^* - V_0)
            = \frac{\frac94 \cdot P_0 + 5P_0}2 \cdot \cbr{\frac{15}4 \cdot V_0 - V_0}
            = \ldots = \frac{319}{32} \cdot P_0 V_0, \\
        \Delta U_{2*} &= \frac 32 \nu R T^* - \frac 32 \nu R T_2 = \frac 32 (P^*V^* - P_0 \cdot 5V_0)
            = \frac 32 \cbr{\frac94 \cdot P_0 \cdot \frac{15}4 \cdot V_0 - P_0 \cdot 5V_0}
            = \frac{165}{32} \cdot P_0 V_0, \\
        \Delta U_{12} &= \frac 32 \nu R T_2 - \frac 32 \nu R T_1 = \frac 32 (5P_0V_0 - P_0V_0) = \ldots = 6 \cdot P_0 V_0, \\
        \eta &= \frac{A_\text{цикл}}{Q_+} = \frac{A_\text{цикл}}{Q_{12} + Q_{2*}}
            = \frac{A_\text{цикл}}{A_{12} + \Delta U_{12} + A_{2*} + \Delta U_{2*}} = \\
            &= \frac{8 \cdot P_0V_0}{0 + 6 \cdot P_0 V_0 + \frac{319}{32} \cdot P_0 V_0 + \frac{165}{32} \cdot P_0 V_0}
             = \frac{A_bonus_cycle:LaTeX}{6 + \frac{319}{32} + \frac{165}{32}}
             = \frac{64}{169} \leftarrow \text{вжух и готово!}
        \end{align*}
}
\solutionspace{360pt}

\tasknumber{2}%
\task{%
    При температуре $20\celsius$ относительная влажность воздуха составляет $60\%$.
    \begin{itemize}
        \item Определите точку росы для этого воздуха.
        \item Какой станет относительная влажность этого воздуха, если нагреть его до $40\celsius$?
    \end{itemize}
}
\answer{%
    \begin{align*}
    &\text{Значения плотности насыщенного водяного пара определяем по таблице:} \\
    &\rho_{\text{нас.
    пара 20} \celsius} = 17{,}300\,\frac{\text{г}}{\text{м}^{3}}, \rho_{\text{нас.
    пара 40} \celsius} = 51{,}200\,\frac{\text{г}}{\text{м}^{3}}.
    \\
    \varphi_1 &= \frac{\rho_\text{пара}}{\rho_{\text{нас.
    пара 20} \celsius}} \implies {\rho_\text{пара}} = \rho_{\text{нас.
    пара 20} \celsius} \cdot \varphi_1 = 17{,}300\,\frac{\text{г}}{\text{м}^{3}} \cdot 0{,}60 = 10{,}380\,\frac{\text{г}}{\text{м}^{3}}.
    \\
    &\text{По таблице определяем, при какой температуре пар с такой плотностью станет насыщенным:}  \\
    t_\text{росы} &= 11{,}5\celsius, \\
    \varphi_2 &= \frac{\rho_\text{пара}}{\rho_{\text{нас.
    пара 40} \celsius}} = \frac{\rho_{\text{нас.
    пара 20} \celsius} \cdot \varphi_1}{\rho_{\text{нас.
    пара 40} \celsius}}= \varphi_1 \cdot \frac{\rho_{\text{нас.
    пара 20} \celsius}}{\rho_{\text{нас.
    пара 40} \celsius}} = 0{,}60 \cdot \frac{17{,}300\,\frac{\text{г}}{\text{м}^{3}}}{51{,}200\,\frac{\text{г}}{\text{м}^{3}}} = 0{,}203 \approx 20{,}3\%.
    \end{align*}
}
\solutionspace{80pt}

\tasknumber{3}%
\task{%
    Из уравнения состояния идеального газа выведите или выразите...
    \begin{enumerate}
        \item объём,
        \item температуру,
        \item плотность газа.
    \end{enumerate}
}

\tasknumber{4}%
\task{%
    Запишите формулы и рядом с каждой физичической величиной укажите её название и единицы измерения в СИ:
    \begin{enumerate}
        \item первое начало термодинамики,
        \item внутренняя энергия идеального одноатомного газа.
    \end{enumerate}
}

\variantsplitter

\addpersonalvariant{Софья Белянкина}

\tasknumber{1}%
\task{%
    Определите КПД (оставив ответ точным в виде нескоратимой дроби) цикла 1231, рабочим телом которого является идеальный одноатомный газ, если
    \begin{itemize}
        \item 12 — изохорический нагрев в три раза,
        \item 23 — изобарическое расширение, при котором температура растёт в шесть раз,
        \item 31 — процесс, график которого в $PV$-координатах является отрезком прямой.
    \end{itemize}
    Бонус: замените цикл 1231 циклом, в котором 12 — изохорический нагрев в три раза, 23 — процесс, график которого в $PV$-координатах является отрезком прямой, 31 — изобарическое охлаждение, при котором температура падает в три раза.
}
\answer{%
    \begin{align*}
    A_{12} &= 0, \Delta U_{12} > 0, \implies Q_{12} = A_{12} + \Delta U_{12} > 0.
    \\
    A_{23} &> 0, \Delta U_{23} > 0, \implies Q_{23} = A_{23} + \Delta U_{23} > 0, \\
    A_{31} &= 0, \Delta U_{31} < 0, \implies Q_{31} = A_{31} + \Delta U_{31} < 0.
    \\
    P_1V_1 &= \nu R T_1, P_2V_2 = \nu R T_2, P_3V_3 = \nu R T_3 \text{ — уравнения состояния идеального газа}, \\
    &\text{Пусть $P_0$, $V_0$, $T_0$ — давление, объём и температура в точке 1 (минимальные во всём цикле):} \\
    P_1 &= P_0, P_2 = P_3, V_1 = V_2 = V_0, \text{остальные соотношения нужно считать} \\
    T_2 &= 3T_1 = 3T_0 \text{(по условию)} \implies \frac{P_2}{P_1} = \frac{P_2V_0}{P_1V_0} = \frac{P_2 V_2}{P_1 V_1}= \frac{\nu R T_2}{\nu R T_1} = \frac{T_2}{T_1} = 3 \implies P_2 = 3 P_1 = 3 P_0, \\
    T_3 &= 6T_2 = 18T_0 \text{(по условию)} \implies \frac{V_3}{V_2} = \frac{P_3V_3}{P_2V_2}= \frac{\nu R T_3}{\nu R T_2} = \frac{T_3}{T_2} = 6 \implies V_3 = 6 V_2 = 6 V_0.
    \\
    A_\text{цикл} &= \frac 12 (6P_0 - P_0)(3V_0 - V_0) = \frac 12 \cdot 10 \cdot P_0V_0, \\
    A_{23} &= 3P_0 \cdot (6V_0 - V_0) = 15P_0V_0, \\
    \Delta U_{23} &= \frac 32 \nu R T_3 - \frac 32 \nu R T_2 = \frac 32 P_3 V_3 - \frac 32 P_2 V_2 = \frac 32 \cdot 3 P_0 \cdot 6 V_0 -  \frac 32 \cdot 3 P_0 \cdot V_0 = \frac 32 \cdot 15 \cdot P_0V_0, \\
    \Delta U_{12} &= \frac 32 \nu R T_2 - \frac 32 \nu R T_1 = \frac 32 P_2 V_2 - \frac 32 P_1 V_1 = \frac 32 \cdot 3 P_0V_0 - \frac 32 P_0V_0 = \frac 32 \cdot 2 \cdot P_0V_0.
    \\
    \eta &= \frac{A_\text{цикл}}{Q_+} = \frac{A_\text{цикл}}{Q_{12} + Q_{23}}  = \frac{A_\text{цикл}}{A_{12} + \Delta U_{12} + A_{23} + \Delta U_{23}} =  \\
     &= \frac{\frac 12 \cdot 10 \cdot P_0V_0}{0 + \frac 32 \cdot 2 \cdot P_0V_0 + 15P_0V_0 + \frac 32 \cdot 15 \cdot P_0V_0} = \frac{\frac 12 \cdot 10}{\frac 32 \cdot 2 + 15 + \frac 32 \cdot 15} = \frac{10}{81} \approx 0.123.
    \end{align*}


        График процесса не в масштабе (эта часть пока не готова и сделать автоматически аккуратно сложно), но с верными подписями (а для решения этого достаточно):

        \begin{tikzpicture}[thick]
            \draw[-{Latex}] (0, 0) -- (0, 7) node[above left] {$P$};
            \draw[-{Latex}] (0, 0) -- (10, 0) node[right] {$V$};

            \draw[dashed] (0, 2) node[left] {$P_1 = P_0$} -| (3, 0) node[below] {$V_1 = V_2 = V_0$};
            \draw[dashed] (0, 6) node[left] {$P_2 = P_3 = 3P_0$} -| (9, 0) node[below] {$V_3 = 6V_0$};

            \draw (3, 2) node[above left]{1} node[below left]{$T_1 = T_0$}
                   (3, 6) node[below left]{2} node[above]{$T_2 = 3T_0$}
                   (9, 6) node[above right]{3} node[below right]{$T_3 = 18T_0$};
            \draw[midar] (3, 2) -- (3, 6);
            \draw[midar] (3, 6) -- (9, 6);
            \draw[midar] (9, 6) -- (3, 2);
        \end{tikzpicture}

        Решение бонуса:
        \begin{align*}
            A_{12} &= 0, \Delta U_{12} > 0, \implies Q_{12} = A_{12} + \Delta U_{12} > 0, \\
            A_{23} &> 0, \Delta U_{23} \text{ — ничего нельзя сказать, нужно исследовать отдельно}, \\
            A_{31} &< 0, \Delta U_{31} < 0, \implies Q_{31} = A_{31} + \Delta U_{31} < 0.
            \\
        \end{align*}

        Уравнения состояния идеального газа для точек 1, 2, 3: $P_1V_1 = \nu R T_1, P_2V_2 = \nu R T_2, P_3V_3 = \nu R T_3$.
        Пусть $P_0$, $V_0$, $T_0$ — давление, объём и температура в точке 1 (минимальные во всём цикле).

        12 --- изохора, $\frac{P_1V_1}{T_1} = \nu R = \frac{P_2V_2}{T_2}, V_2=V_1=V_0 \implies \frac{P_1}{T_1} =  \frac{P_2}{T_2} \implies P_2 = P_1 \frac{T_2}{T_1} = 3P_0$,

        31 --- изобара, $\frac{P_1V_1}{T_1} = \nu R = \frac{P_3V_3}{T_3}, P_3=P_1=P_0 \implies \frac{V_3}{T_3} =  \frac{V_1}{T_1} \implies V_3 = V_1 \frac{T_3}{T_1} = 3V_0$,

        Таким образом, используя новые обозначения, состояния газа в точках 1, 2 и 3 описываются макропараметрами $(P_0, V_0, T_0), (3P_0, V_0, 3T_0), (P_0, 3V_0, 3T_0)$ соответственно.

        \begin{tikzpicture}[thick]
            \draw[-{Latex}] (0, 0) -- (0, 7) node[above left] {$P$};
            \draw[-{Latex}] (0, 0) -- (10, 0) node[right] {$V$};

            \draw[dashed] (0, 2) node[left] {$P_1 = P_3 = P_0$} -| (9, 0) node[below] {$V_3 = 3V_0$};
            \draw[dashed] (0, 6) node[left] {$P_2 = 3P_0$} -| (3, 0) node[below] {$V_1 = V_2 = V_0$};

            \draw[dashed] (0, 5) node[left] {$P$} -| (4.5, 0) node[below] {$V$};
            \draw[dashed] (0, 4.6) node[left] {$P'$} -| (5.1, 0) node[below] {$V'$};

            \draw (3, 2) node[above left]{1} node[below left]{$T_1 = T_0$}
                   (3, 6) node[below left]{2} node[above]{$T_2 = 3T_0$}
                   (9, 2) node[above right]{3} node[below right]{$T_3 = 3T_0$};
            \draw[midar] (3, 2) -- (3, 6);
            \draw[midar] (3, 6) -- (9, 2);
            \draw[midar] (9, 2) -- (3, 2);
            \draw   (4.5, 5) node[above right]{$T$} (5.1, 4.6) node[above right]{$T'$};
        \end{tikzpicture}


        Теперь рассмотрим отдельно процесс 23, к остальному вернёмся позже.
        Уравнение этой прямой в $PV$-координатах: $P(V) = 4P_0 - \frac{P_0}{V_0} V$.
        Это значит, что при изменении объёма на $\Delta V$ давление изменится на $\Delta P = - \frac{P_0}{V_0} \Delta V$, обратите внимание на знак.

        Рассмотрим произвольную точку в процессе 23 и дадим процессу ещё немного свершиться, при этом объём изменится на $\Delta V$, давление — на $\Delta P$, температура (иначе бы была гипербола, а не прямая) — на $\Delta T$,
        т.е.
        из состояния $(P, V, T)$ мы перешли в $(P', V', T')$, причём  $P' = P + \Delta P, V' = V + \Delta V, T' = T + \Delta T$.

        При этом изменится внутренняя энергия:
        \begin{align*}
        \Delta U
            &= U' - U = \frac 32 \nu R T' - \frac 32 \nu R T = \frac 32 (P+\Delta P) (V+\Delta V) - \frac 32 PV\\
            &= \frac 32 ((P+\Delta P) (V+\Delta V) - PV) = \frac 32 (P\Delta V + V \Delta P + \Delta P \Delta V).
        \end{align*}

        Рассмотрим малые изменения объёма, тогда и изменение давления будем малым (т.к.
        $\Delta P = - \frac{P_0}{V_0} \Delta V$),
        а третьим слагаемым в выражении для $\Delta U$  можно пренебречь по сравнению с двумя другими:
        два первых это малые величины, а третье — произведение двух малых.
        Тогда $\Delta U = \frac 32 (P\Delta V + V \Delta P)$.

        Работа газа при этом малом изменении объёма — это площадь трапеции (тут ещё раз пренебрегли малым слагаемым):
        $$A = \frac{P + P'}2 \Delta V = \cbr{P + \frac{\Delta P}2} \Delta V = P \Delta V.$$

        Подведённое количество теплоты, используя первое начало термодинамики, будет равно
        \begin{align*}
        Q
            &= \frac 32 (P\Delta V + V \Delta P) + P \Delta V =  \frac 52 P\Delta V + \frac 32 V \Delta P = \\
            &= \frac 52 P\Delta V + \frac 32 V \cdot \cbr{- \frac{P_0}{V_0} \Delta V} = \frac{\Delta V}2 \cdot \cbr{5P - \frac{P_0}{V_0} V} = \\
            &= \frac{\Delta V}2 \cdot \cbr{5 \cdot \cbr{4P_0 - \frac{P_0}{V_0} V} - \frac{P_0}{V_0} V}
             = \frac{\Delta V \cdot P_0}2 \cdot \cbr{5 \cdot 4 - 8\frac V{V_0}}.
        \end{align*}

        Таком образом, знак количества теплоты $Q$ на участке 23 зависит от конкретного значения $V$:
        \begin{itemize}
            \item $\Delta V > 0$ на всём участке 23, поскольку газ расширяется,
            \item $P > 0$ — всегда, у нас идеальный газ, удары о стенки сосуда абсолютно упругие, а молекулы не взаимодействуют и поэтому давление только положительно,
            \item если $5 \cdot 4 - 8\frac V{V_0} > 0$ — тепло подводят, если же меньше нуля — отводят.
        \end{itemize}
        Решая последнее неравенство, получаем конкретное значение $V^*$: при $V < V^*$ тепло подводят, далее~— отводят.
        Тут *~--- некоторая точка между точками 2 и 3, конкретные значения надо досчитать:
        $$V^* = V_0 \cdot \frac{5 \cdot 4}8 = \frac52 \cdot V_0 \implies P^* = 4P_0 - \frac{P_0}{V_0} V^* = \ldots = \frac32 \cdot P_0.$$

        Т.е.
        чтобы вычислить $Q_+$, надо сложить количества теплоты на участке 12 и лишь части участка 23 — участке 2*,
        той его части где это количество теплоты положительно.
        Имеем: $Q_+ = Q_{12} + Q_{2*}$.

        Теперь возвращаемся к циклу целиком и получаем:
        \begin{align*}
        A_\text{цикл} &= \frac 12 \cdot (3P_0 - P_0) \cdot (3V_0 - V_0) = 2 \cdot P_0V_0, \\
        A_{2*} &= \frac{P^* + 3P_0}2 \cdot (V^* - V_0)
            = \frac{\frac32 \cdot P_0 + 3P_0}2 \cdot \cbr{\frac52 \cdot V_0 - V_0}
            = \ldots = \frac{27}8 \cdot P_0 V_0, \\
        \Delta U_{2*} &= \frac 32 \nu R T^* - \frac 32 \nu R T_2 = \frac 32 (P^*V^* - P_0 \cdot 3V_0)
            = \frac 32 \cbr{\frac32 \cdot P_0 \cdot \frac52 \cdot V_0 - P_0 \cdot 3V_0}
            = \frac98 \cdot P_0 V_0, \\
        \Delta U_{12} &= \frac 32 \nu R T_2 - \frac 32 \nu R T_1 = \frac 32 (3P_0V_0 - P_0V_0) = \ldots = 3 \cdot P_0 V_0, \\
        \eta &= \frac{A_\text{цикл}}{Q_+} = \frac{A_\text{цикл}}{Q_{12} + Q_{2*}}
            = \frac{A_\text{цикл}}{A_{12} + \Delta U_{12} + A_{2*} + \Delta U_{2*}} = \\
            &= \frac{2 \cdot P_0V_0}{0 + 3 \cdot P_0 V_0 + \frac{27}8 \cdot P_0 V_0 + \frac98 \cdot P_0 V_0}
             = \frac{A_bonus_cycle:LaTeX}{3 + \frac{27}8 + \frac98}
             = \frac4{15} \leftarrow \text{вжух и готово!}
        \end{align*}
}
\solutionspace{360pt}

\tasknumber{2}%
\task{%
    При температуре $25\celsius$ относительная влажность воздуха составляет $45\%$.
    \begin{itemize}
        \item Определите точку росы для этого воздуха.
        \item Какой станет относительная влажность этого воздуха, если нагреть его до $50\celsius$?
    \end{itemize}
}
\answer{%
    \begin{align*}
    &\text{Значения плотности насыщенного водяного пара определяем по таблице:} \\
    &\rho_{\text{нас.
    пара 25} \celsius} = 23{,}000\,\frac{\text{г}}{\text{м}^{3}}, \rho_{\text{нас.
    пара 50} \celsius} = 83{,}000\,\frac{\text{г}}{\text{м}^{3}}.
    \\
    \varphi_1 &= \frac{\rho_\text{пара}}{\rho_{\text{нас.
    пара 25} \celsius}} \implies {\rho_\text{пара}} = \rho_{\text{нас.
    пара 25} \celsius} \cdot \varphi_1 = 23{,}000\,\frac{\text{г}}{\text{м}^{3}} \cdot 0{,}45 = 10{,}350\,\frac{\text{г}}{\text{м}^{3}}.
    \\
    &\text{По таблице определяем, при какой температуре пар с такой плотностью станет насыщенным:}  \\
    t_\text{росы} &= 11{,}5\celsius, \\
    \varphi_2 &= \frac{\rho_\text{пара}}{\rho_{\text{нас.
    пара 50} \celsius}} = \frac{\rho_{\text{нас.
    пара 25} \celsius} \cdot \varphi_1}{\rho_{\text{нас.
    пара 50} \celsius}}= \varphi_1 \cdot \frac{\rho_{\text{нас.
    пара 25} \celsius}}{\rho_{\text{нас.
    пара 50} \celsius}} = 0{,}45 \cdot \frac{23{,}000\,\frac{\text{г}}{\text{м}^{3}}}{83{,}000\,\frac{\text{г}}{\text{м}^{3}}} = 0{,}125 \approx 12{,}5\%.
    \end{align*}
}
\solutionspace{80pt}

\tasknumber{3}%
\task{%
    Из уравнения состояния идеального газа выведите или выразите...
    \begin{enumerate}
        \item давление,
        \item молярную массу,
        \item плотность газа.
    \end{enumerate}
}

\tasknumber{4}%
\task{%
    Запишите формулы и рядом с каждой физичической величиной укажите её название и единицы измерения в СИ:
    \begin{enumerate}
        \item первое начало термодинамики,
        \item внутренняя энергия идеального одноатомного газа.
    \end{enumerate}
}

\variantsplitter

\addpersonalvariant{Варвара Егиазарян}

\tasknumber{1}%
\task{%
    Определите КПД (оставив ответ точным в виде нескоратимой дроби) цикла 1231, рабочим телом которого является идеальный одноатомный газ, если
    \begin{itemize}
        \item 12 — изохорический нагрев в два раза,
        \item 23 — изобарическое расширение, при котором температура растёт в шесть раз,
        \item 31 — процесс, график которого в $PV$-координатах является отрезком прямой.
    \end{itemize}
    Бонус: замените цикл 1231 циклом, в котором 12 — изохорический нагрев в два раза, 23 — процесс, график которого в $PV$-координатах является отрезком прямой, 31 — изобарическое охлаждение, при котором температура падает в два раза.
}
\answer{%
    \begin{align*}
    A_{12} &= 0, \Delta U_{12} > 0, \implies Q_{12} = A_{12} + \Delta U_{12} > 0.
    \\
    A_{23} &> 0, \Delta U_{23} > 0, \implies Q_{23} = A_{23} + \Delta U_{23} > 0, \\
    A_{31} &= 0, \Delta U_{31} < 0, \implies Q_{31} = A_{31} + \Delta U_{31} < 0.
    \\
    P_1V_1 &= \nu R T_1, P_2V_2 = \nu R T_2, P_3V_3 = \nu R T_3 \text{ — уравнения состояния идеального газа}, \\
    &\text{Пусть $P_0$, $V_0$, $T_0$ — давление, объём и температура в точке 1 (минимальные во всём цикле):} \\
    P_1 &= P_0, P_2 = P_3, V_1 = V_2 = V_0, \text{остальные соотношения нужно считать} \\
    T_2 &= 2T_1 = 2T_0 \text{(по условию)} \implies \frac{P_2}{P_1} = \frac{P_2V_0}{P_1V_0} = \frac{P_2 V_2}{P_1 V_1}= \frac{\nu R T_2}{\nu R T_1} = \frac{T_2}{T_1} = 2 \implies P_2 = 2 P_1 = 2 P_0, \\
    T_3 &= 6T_2 = 12T_0 \text{(по условию)} \implies \frac{V_3}{V_2} = \frac{P_3V_3}{P_2V_2}= \frac{\nu R T_3}{\nu R T_2} = \frac{T_3}{T_2} = 6 \implies V_3 = 6 V_2 = 6 V_0.
    \\
    A_\text{цикл} &= \frac 12 (6P_0 - P_0)(2V_0 - V_0) = \frac 12 \cdot 5 \cdot P_0V_0, \\
    A_{23} &= 2P_0 \cdot (6V_0 - V_0) = 10P_0V_0, \\
    \Delta U_{23} &= \frac 32 \nu R T_3 - \frac 32 \nu R T_2 = \frac 32 P_3 V_3 - \frac 32 P_2 V_2 = \frac 32 \cdot 2 P_0 \cdot 6 V_0 -  \frac 32 \cdot 2 P_0 \cdot V_0 = \frac 32 \cdot 10 \cdot P_0V_0, \\
    \Delta U_{12} &= \frac 32 \nu R T_2 - \frac 32 \nu R T_1 = \frac 32 P_2 V_2 - \frac 32 P_1 V_1 = \frac 32 \cdot 2 P_0V_0 - \frac 32 P_0V_0 = \frac 32 \cdot 1 \cdot P_0V_0.
    \\
    \eta &= \frac{A_\text{цикл}}{Q_+} = \frac{A_\text{цикл}}{Q_{12} + Q_{23}}  = \frac{A_\text{цикл}}{A_{12} + \Delta U_{12} + A_{23} + \Delta U_{23}} =  \\
     &= \frac{\frac 12 \cdot 5 \cdot P_0V_0}{0 + \frac 32 \cdot 1 \cdot P_0V_0 + 10P_0V_0 + \frac 32 \cdot 10 \cdot P_0V_0} = \frac{\frac 12 \cdot 5}{\frac 32 \cdot 1 + 10 + \frac 32 \cdot 10} = \frac5{53} \approx 0.094.
    \end{align*}


        График процесса не в масштабе (эта часть пока не готова и сделать автоматически аккуратно сложно), но с верными подписями (а для решения этого достаточно):

        \begin{tikzpicture}[thick]
            \draw[-{Latex}] (0, 0) -- (0, 7) node[above left] {$P$};
            \draw[-{Latex}] (0, 0) -- (10, 0) node[right] {$V$};

            \draw[dashed] (0, 2) node[left] {$P_1 = P_0$} -| (3, 0) node[below] {$V_1 = V_2 = V_0$};
            \draw[dashed] (0, 6) node[left] {$P_2 = P_3 = 2P_0$} -| (9, 0) node[below] {$V_3 = 6V_0$};

            \draw (3, 2) node[above left]{1} node[below left]{$T_1 = T_0$}
                   (3, 6) node[below left]{2} node[above]{$T_2 = 2T_0$}
                   (9, 6) node[above right]{3} node[below right]{$T_3 = 12T_0$};
            \draw[midar] (3, 2) -- (3, 6);
            \draw[midar] (3, 6) -- (9, 6);
            \draw[midar] (9, 6) -- (3, 2);
        \end{tikzpicture}

        Решение бонуса:
        \begin{align*}
            A_{12} &= 0, \Delta U_{12} > 0, \implies Q_{12} = A_{12} + \Delta U_{12} > 0, \\
            A_{23} &> 0, \Delta U_{23} \text{ — ничего нельзя сказать, нужно исследовать отдельно}, \\
            A_{31} &< 0, \Delta U_{31} < 0, \implies Q_{31} = A_{31} + \Delta U_{31} < 0.
            \\
        \end{align*}

        Уравнения состояния идеального газа для точек 1, 2, 3: $P_1V_1 = \nu R T_1, P_2V_2 = \nu R T_2, P_3V_3 = \nu R T_3$.
        Пусть $P_0$, $V_0$, $T_0$ — давление, объём и температура в точке 1 (минимальные во всём цикле).

        12 --- изохора, $\frac{P_1V_1}{T_1} = \nu R = \frac{P_2V_2}{T_2}, V_2=V_1=V_0 \implies \frac{P_1}{T_1} =  \frac{P_2}{T_2} \implies P_2 = P_1 \frac{T_2}{T_1} = 2P_0$,

        31 --- изобара, $\frac{P_1V_1}{T_1} = \nu R = \frac{P_3V_3}{T_3}, P_3=P_1=P_0 \implies \frac{V_3}{T_3} =  \frac{V_1}{T_1} \implies V_3 = V_1 \frac{T_3}{T_1} = 2V_0$,

        Таким образом, используя новые обозначения, состояния газа в точках 1, 2 и 3 описываются макропараметрами $(P_0, V_0, T_0), (2P_0, V_0, 2T_0), (P_0, 2V_0, 2T_0)$ соответственно.

        \begin{tikzpicture}[thick]
            \draw[-{Latex}] (0, 0) -- (0, 7) node[above left] {$P$};
            \draw[-{Latex}] (0, 0) -- (10, 0) node[right] {$V$};

            \draw[dashed] (0, 2) node[left] {$P_1 = P_3 = P_0$} -| (9, 0) node[below] {$V_3 = 2V_0$};
            \draw[dashed] (0, 6) node[left] {$P_2 = 2P_0$} -| (3, 0) node[below] {$V_1 = V_2 = V_0$};

            \draw[dashed] (0, 5) node[left] {$P$} -| (4.5, 0) node[below] {$V$};
            \draw[dashed] (0, 4.6) node[left] {$P'$} -| (5.1, 0) node[below] {$V'$};

            \draw (3, 2) node[above left]{1} node[below left]{$T_1 = T_0$}
                   (3, 6) node[below left]{2} node[above]{$T_2 = 2T_0$}
                   (9, 2) node[above right]{3} node[below right]{$T_3 = 2T_0$};
            \draw[midar] (3, 2) -- (3, 6);
            \draw[midar] (3, 6) -- (9, 2);
            \draw[midar] (9, 2) -- (3, 2);
            \draw   (4.5, 5) node[above right]{$T$} (5.1, 4.6) node[above right]{$T'$};
        \end{tikzpicture}


        Теперь рассмотрим отдельно процесс 23, к остальному вернёмся позже.
        Уравнение этой прямой в $PV$-координатах: $P(V) = 3P_0 - \frac{P_0}{V_0} V$.
        Это значит, что при изменении объёма на $\Delta V$ давление изменится на $\Delta P = - \frac{P_0}{V_0} \Delta V$, обратите внимание на знак.

        Рассмотрим произвольную точку в процессе 23 и дадим процессу ещё немного свершиться, при этом объём изменится на $\Delta V$, давление — на $\Delta P$, температура (иначе бы была гипербола, а не прямая) — на $\Delta T$,
        т.е.
        из состояния $(P, V, T)$ мы перешли в $(P', V', T')$, причём  $P' = P + \Delta P, V' = V + \Delta V, T' = T + \Delta T$.

        При этом изменится внутренняя энергия:
        \begin{align*}
        \Delta U
            &= U' - U = \frac 32 \nu R T' - \frac 32 \nu R T = \frac 32 (P+\Delta P) (V+\Delta V) - \frac 32 PV\\
            &= \frac 32 ((P+\Delta P) (V+\Delta V) - PV) = \frac 32 (P\Delta V + V \Delta P + \Delta P \Delta V).
        \end{align*}

        Рассмотрим малые изменения объёма, тогда и изменение давления будем малым (т.к.
        $\Delta P = - \frac{P_0}{V_0} \Delta V$),
        а третьим слагаемым в выражении для $\Delta U$  можно пренебречь по сравнению с двумя другими:
        два первых это малые величины, а третье — произведение двух малых.
        Тогда $\Delta U = \frac 32 (P\Delta V + V \Delta P)$.

        Работа газа при этом малом изменении объёма — это площадь трапеции (тут ещё раз пренебрегли малым слагаемым):
        $$A = \frac{P + P'}2 \Delta V = \cbr{P + \frac{\Delta P}2} \Delta V = P \Delta V.$$

        Подведённое количество теплоты, используя первое начало термодинамики, будет равно
        \begin{align*}
        Q
            &= \frac 32 (P\Delta V + V \Delta P) + P \Delta V =  \frac 52 P\Delta V + \frac 32 V \Delta P = \\
            &= \frac 52 P\Delta V + \frac 32 V \cdot \cbr{- \frac{P_0}{V_0} \Delta V} = \frac{\Delta V}2 \cdot \cbr{5P - \frac{P_0}{V_0} V} = \\
            &= \frac{\Delta V}2 \cdot \cbr{5 \cdot \cbr{3P_0 - \frac{P_0}{V_0} V} - \frac{P_0}{V_0} V}
             = \frac{\Delta V \cdot P_0}2 \cdot \cbr{5 \cdot 3 - 8\frac V{V_0}}.
        \end{align*}

        Таком образом, знак количества теплоты $Q$ на участке 23 зависит от конкретного значения $V$:
        \begin{itemize}
            \item $\Delta V > 0$ на всём участке 23, поскольку газ расширяется,
            \item $P > 0$ — всегда, у нас идеальный газ, удары о стенки сосуда абсолютно упругие, а молекулы не взаимодействуют и поэтому давление только положительно,
            \item если $5 \cdot 3 - 8\frac V{V_0} > 0$ — тепло подводят, если же меньше нуля — отводят.
        \end{itemize}
        Решая последнее неравенство, получаем конкретное значение $V^*$: при $V < V^*$ тепло подводят, далее~— отводят.
        Тут *~--- некоторая точка между точками 2 и 3, конкретные значения надо досчитать:
        $$V^* = V_0 \cdot \frac{5 \cdot 3}8 = \frac{15}8 \cdot V_0 \implies P^* = 3P_0 - \frac{P_0}{V_0} V^* = \ldots = \frac98 \cdot P_0.$$

        Т.е.
        чтобы вычислить $Q_+$, надо сложить количества теплоты на участке 12 и лишь части участка 23 — участке 2*,
        той его части где это количество теплоты положительно.
        Имеем: $Q_+ = Q_{12} + Q_{2*}$.

        Теперь возвращаемся к циклу целиком и получаем:
        \begin{align*}
        A_\text{цикл} &= \frac 12 \cdot (2P_0 - P_0) \cdot (2V_0 - V_0) = \frac12 \cdot P_0V_0, \\
        A_{2*} &= \frac{P^* + 2P_0}2 \cdot (V^* - V_0)
            = \frac{\frac98 \cdot P_0 + 2P_0}2 \cdot \cbr{\frac{15}8 \cdot V_0 - V_0}
            = \ldots = \frac{175}{128} \cdot P_0 V_0, \\
        \Delta U_{2*} &= \frac 32 \nu R T^* - \frac 32 \nu R T_2 = \frac 32 (P^*V^* - P_0 \cdot 2V_0)
            = \frac 32 \cbr{\frac98 \cdot P_0 \cdot \frac{15}8 \cdot V_0 - P_0 \cdot 2V_0}
            = \frac{21}{128} \cdot P_0 V_0, \\
        \Delta U_{12} &= \frac 32 \nu R T_2 - \frac 32 \nu R T_1 = \frac 32 (2P_0V_0 - P_0V_0) = \ldots = \frac32 \cdot P_0 V_0, \\
        \eta &= \frac{A_\text{цикл}}{Q_+} = \frac{A_\text{цикл}}{Q_{12} + Q_{2*}}
            = \frac{A_\text{цикл}}{A_{12} + \Delta U_{12} + A_{2*} + \Delta U_{2*}} = \\
            &= \frac{\frac12 \cdot P_0V_0}{0 + \frac32 \cdot P_0 V_0 + \frac{175}{128} \cdot P_0 V_0 + \frac{21}{128} \cdot P_0 V_0}
             = \frac{A_bonus_cycle:LaTeX}{\frac32 + \frac{175}{128} + \frac{21}{128}}
             = \frac{16}{97} \leftarrow \text{вжух и готово!}
        \end{align*}
}
\solutionspace{360pt}

\tasknumber{2}%
\task{%
    При температуре $20\celsius$ относительная влажность воздуха составляет $75\%$.
    \begin{itemize}
        \item Определите точку росы для этого воздуха.
        \item Какой станет относительная влажность этого воздуха, если нагреть его до $50\celsius$?
    \end{itemize}
}
\answer{%
    \begin{align*}
    &\text{Значения плотности насыщенного водяного пара определяем по таблице:} \\
    &\rho_{\text{нас.
    пара 20} \celsius} = 17{,}300\,\frac{\text{г}}{\text{м}^{3}}, \rho_{\text{нас.
    пара 50} \celsius} = 83{,}000\,\frac{\text{г}}{\text{м}^{3}}.
    \\
    \varphi_1 &= \frac{\rho_\text{пара}}{\rho_{\text{нас.
    пара 20} \celsius}} \implies {\rho_\text{пара}} = \rho_{\text{нас.
    пара 20} \celsius} \cdot \varphi_1 = 17{,}300\,\frac{\text{г}}{\text{м}^{3}} \cdot 0{,}75 = 12{,}975\,\frac{\text{г}}{\text{м}^{3}}.
    \\
    &\text{По таблице определяем, при какой температуре пар с такой плотностью станет насыщенным:}  \\
    t_\text{росы} &= 15{,}2\celsius, \\
    \varphi_2 &= \frac{\rho_\text{пара}}{\rho_{\text{нас.
    пара 50} \celsius}} = \frac{\rho_{\text{нас.
    пара 20} \celsius} \cdot \varphi_1}{\rho_{\text{нас.
    пара 50} \celsius}}= \varphi_1 \cdot \frac{\rho_{\text{нас.
    пара 20} \celsius}}{\rho_{\text{нас.
    пара 50} \celsius}} = 0{,}75 \cdot \frac{17{,}300\,\frac{\text{г}}{\text{м}^{3}}}{83{,}000\,\frac{\text{г}}{\text{м}^{3}}} = 0{,}156 \approx 15{,}6\%.
    \end{align*}
}
\solutionspace{80pt}

\tasknumber{3}%
\task{%
    Из уравнения состояния идеального газа выведите или выразите...
    \begin{enumerate}
        \item объём,
        \item молярную массу,
        \item концентрацию молекул газа.
    \end{enumerate}
}

\tasknumber{4}%
\task{%
    Запишите формулы и рядом с каждой физичической величиной укажите её название и единицы измерения в СИ:
    \begin{enumerate}
        \item первое начало термодинамики,
        \item внутренняя энергия идеального одноатомного газа.
    \end{enumerate}
}

\variantsplitter

\addpersonalvariant{Владислав Емелин}

\tasknumber{1}%
\task{%
    Определите КПД (оставив ответ точным в виде нескоратимой дроби) цикла 1231, рабочим телом которого является идеальный одноатомный газ, если
    \begin{itemize}
        \item 12 — изохорический нагрев в пять раз,
        \item 23 — изобарическое расширение, при котором температура растёт в шесть раз,
        \item 31 — процесс, график которого в $PV$-координатах является отрезком прямой.
    \end{itemize}
    Бонус: замените цикл 1231 циклом, в котором 12 — изохорический нагрев в пять раз, 23 — процесс, график которого в $PV$-координатах является отрезком прямой, 31 — изобарическое охлаждение, при котором температура падает в пять раз.
}
\answer{%
    \begin{align*}
    A_{12} &= 0, \Delta U_{12} > 0, \implies Q_{12} = A_{12} + \Delta U_{12} > 0.
    \\
    A_{23} &> 0, \Delta U_{23} > 0, \implies Q_{23} = A_{23} + \Delta U_{23} > 0, \\
    A_{31} &= 0, \Delta U_{31} < 0, \implies Q_{31} = A_{31} + \Delta U_{31} < 0.
    \\
    P_1V_1 &= \nu R T_1, P_2V_2 = \nu R T_2, P_3V_3 = \nu R T_3 \text{ — уравнения состояния идеального газа}, \\
    &\text{Пусть $P_0$, $V_0$, $T_0$ — давление, объём и температура в точке 1 (минимальные во всём цикле):} \\
    P_1 &= P_0, P_2 = P_3, V_1 = V_2 = V_0, \text{остальные соотношения нужно считать} \\
    T_2 &= 5T_1 = 5T_0 \text{(по условию)} \implies \frac{P_2}{P_1} = \frac{P_2V_0}{P_1V_0} = \frac{P_2 V_2}{P_1 V_1}= \frac{\nu R T_2}{\nu R T_1} = \frac{T_2}{T_1} = 5 \implies P_2 = 5 P_1 = 5 P_0, \\
    T_3 &= 6T_2 = 30T_0 \text{(по условию)} \implies \frac{V_3}{V_2} = \frac{P_3V_3}{P_2V_2}= \frac{\nu R T_3}{\nu R T_2} = \frac{T_3}{T_2} = 6 \implies V_3 = 6 V_2 = 6 V_0.
    \\
    A_\text{цикл} &= \frac 12 (6P_0 - P_0)(5V_0 - V_0) = \frac 12 \cdot 20 \cdot P_0V_0, \\
    A_{23} &= 5P_0 \cdot (6V_0 - V_0) = 25P_0V_0, \\
    \Delta U_{23} &= \frac 32 \nu R T_3 - \frac 32 \nu R T_2 = \frac 32 P_3 V_3 - \frac 32 P_2 V_2 = \frac 32 \cdot 5 P_0 \cdot 6 V_0 -  \frac 32 \cdot 5 P_0 \cdot V_0 = \frac 32 \cdot 25 \cdot P_0V_0, \\
    \Delta U_{12} &= \frac 32 \nu R T_2 - \frac 32 \nu R T_1 = \frac 32 P_2 V_2 - \frac 32 P_1 V_1 = \frac 32 \cdot 5 P_0V_0 - \frac 32 P_0V_0 = \frac 32 \cdot 4 \cdot P_0V_0.
    \\
    \eta &= \frac{A_\text{цикл}}{Q_+} = \frac{A_\text{цикл}}{Q_{12} + Q_{23}}  = \frac{A_\text{цикл}}{A_{12} + \Delta U_{12} + A_{23} + \Delta U_{23}} =  \\
     &= \frac{\frac 12 \cdot 20 \cdot P_0V_0}{0 + \frac 32 \cdot 4 \cdot P_0V_0 + 25P_0V_0 + \frac 32 \cdot 25 \cdot P_0V_0} = \frac{\frac 12 \cdot 20}{\frac 32 \cdot 4 + 25 + \frac 32 \cdot 25} = \frac{20}{137} \approx 0.146.
    \end{align*}


        График процесса не в масштабе (эта часть пока не готова и сделать автоматически аккуратно сложно), но с верными подписями (а для решения этого достаточно):

        \begin{tikzpicture}[thick]
            \draw[-{Latex}] (0, 0) -- (0, 7) node[above left] {$P$};
            \draw[-{Latex}] (0, 0) -- (10, 0) node[right] {$V$};

            \draw[dashed] (0, 2) node[left] {$P_1 = P_0$} -| (3, 0) node[below] {$V_1 = V_2 = V_0$};
            \draw[dashed] (0, 6) node[left] {$P_2 = P_3 = 5P_0$} -| (9, 0) node[below] {$V_3 = 6V_0$};

            \draw (3, 2) node[above left]{1} node[below left]{$T_1 = T_0$}
                   (3, 6) node[below left]{2} node[above]{$T_2 = 5T_0$}
                   (9, 6) node[above right]{3} node[below right]{$T_3 = 30T_0$};
            \draw[midar] (3, 2) -- (3, 6);
            \draw[midar] (3, 6) -- (9, 6);
            \draw[midar] (9, 6) -- (3, 2);
        \end{tikzpicture}

        Решение бонуса:
        \begin{align*}
            A_{12} &= 0, \Delta U_{12} > 0, \implies Q_{12} = A_{12} + \Delta U_{12} > 0, \\
            A_{23} &> 0, \Delta U_{23} \text{ — ничего нельзя сказать, нужно исследовать отдельно}, \\
            A_{31} &< 0, \Delta U_{31} < 0, \implies Q_{31} = A_{31} + \Delta U_{31} < 0.
            \\
        \end{align*}

        Уравнения состояния идеального газа для точек 1, 2, 3: $P_1V_1 = \nu R T_1, P_2V_2 = \nu R T_2, P_3V_3 = \nu R T_3$.
        Пусть $P_0$, $V_0$, $T_0$ — давление, объём и температура в точке 1 (минимальные во всём цикле).

        12 --- изохора, $\frac{P_1V_1}{T_1} = \nu R = \frac{P_2V_2}{T_2}, V_2=V_1=V_0 \implies \frac{P_1}{T_1} =  \frac{P_2}{T_2} \implies P_2 = P_1 \frac{T_2}{T_1} = 5P_0$,

        31 --- изобара, $\frac{P_1V_1}{T_1} = \nu R = \frac{P_3V_3}{T_3}, P_3=P_1=P_0 \implies \frac{V_3}{T_3} =  \frac{V_1}{T_1} \implies V_3 = V_1 \frac{T_3}{T_1} = 5V_0$,

        Таким образом, используя новые обозначения, состояния газа в точках 1, 2 и 3 описываются макропараметрами $(P_0, V_0, T_0), (5P_0, V_0, 5T_0), (P_0, 5V_0, 5T_0)$ соответственно.

        \begin{tikzpicture}[thick]
            \draw[-{Latex}] (0, 0) -- (0, 7) node[above left] {$P$};
            \draw[-{Latex}] (0, 0) -- (10, 0) node[right] {$V$};

            \draw[dashed] (0, 2) node[left] {$P_1 = P_3 = P_0$} -| (9, 0) node[below] {$V_3 = 5V_0$};
            \draw[dashed] (0, 6) node[left] {$P_2 = 5P_0$} -| (3, 0) node[below] {$V_1 = V_2 = V_0$};

            \draw[dashed] (0, 5) node[left] {$P$} -| (4.5, 0) node[below] {$V$};
            \draw[dashed] (0, 4.6) node[left] {$P'$} -| (5.1, 0) node[below] {$V'$};

            \draw (3, 2) node[above left]{1} node[below left]{$T_1 = T_0$}
                   (3, 6) node[below left]{2} node[above]{$T_2 = 5T_0$}
                   (9, 2) node[above right]{3} node[below right]{$T_3 = 5T_0$};
            \draw[midar] (3, 2) -- (3, 6);
            \draw[midar] (3, 6) -- (9, 2);
            \draw[midar] (9, 2) -- (3, 2);
            \draw   (4.5, 5) node[above right]{$T$} (5.1, 4.6) node[above right]{$T'$};
        \end{tikzpicture}


        Теперь рассмотрим отдельно процесс 23, к остальному вернёмся позже.
        Уравнение этой прямой в $PV$-координатах: $P(V) = 6P_0 - \frac{P_0}{V_0} V$.
        Это значит, что при изменении объёма на $\Delta V$ давление изменится на $\Delta P = - \frac{P_0}{V_0} \Delta V$, обратите внимание на знак.

        Рассмотрим произвольную точку в процессе 23 и дадим процессу ещё немного свершиться, при этом объём изменится на $\Delta V$, давление — на $\Delta P$, температура (иначе бы была гипербола, а не прямая) — на $\Delta T$,
        т.е.
        из состояния $(P, V, T)$ мы перешли в $(P', V', T')$, причём  $P' = P + \Delta P, V' = V + \Delta V, T' = T + \Delta T$.

        При этом изменится внутренняя энергия:
        \begin{align*}
        \Delta U
            &= U' - U = \frac 32 \nu R T' - \frac 32 \nu R T = \frac 32 (P+\Delta P) (V+\Delta V) - \frac 32 PV\\
            &= \frac 32 ((P+\Delta P) (V+\Delta V) - PV) = \frac 32 (P\Delta V + V \Delta P + \Delta P \Delta V).
        \end{align*}

        Рассмотрим малые изменения объёма, тогда и изменение давления будем малым (т.к.
        $\Delta P = - \frac{P_0}{V_0} \Delta V$),
        а третьим слагаемым в выражении для $\Delta U$  можно пренебречь по сравнению с двумя другими:
        два первых это малые величины, а третье — произведение двух малых.
        Тогда $\Delta U = \frac 32 (P\Delta V + V \Delta P)$.

        Работа газа при этом малом изменении объёма — это площадь трапеции (тут ещё раз пренебрегли малым слагаемым):
        $$A = \frac{P + P'}2 \Delta V = \cbr{P + \frac{\Delta P}2} \Delta V = P \Delta V.$$

        Подведённое количество теплоты, используя первое начало термодинамики, будет равно
        \begin{align*}
        Q
            &= \frac 32 (P\Delta V + V \Delta P) + P \Delta V =  \frac 52 P\Delta V + \frac 32 V \Delta P = \\
            &= \frac 52 P\Delta V + \frac 32 V \cdot \cbr{- \frac{P_0}{V_0} \Delta V} = \frac{\Delta V}2 \cdot \cbr{5P - \frac{P_0}{V_0} V} = \\
            &= \frac{\Delta V}2 \cdot \cbr{5 \cdot \cbr{6P_0 - \frac{P_0}{V_0} V} - \frac{P_0}{V_0} V}
             = \frac{\Delta V \cdot P_0}2 \cdot \cbr{5 \cdot 6 - 8\frac V{V_0}}.
        \end{align*}

        Таком образом, знак количества теплоты $Q$ на участке 23 зависит от конкретного значения $V$:
        \begin{itemize}
            \item $\Delta V > 0$ на всём участке 23, поскольку газ расширяется,
            \item $P > 0$ — всегда, у нас идеальный газ, удары о стенки сосуда абсолютно упругие, а молекулы не взаимодействуют и поэтому давление только положительно,
            \item если $5 \cdot 6 - 8\frac V{V_0} > 0$ — тепло подводят, если же меньше нуля — отводят.
        \end{itemize}
        Решая последнее неравенство, получаем конкретное значение $V^*$: при $V < V^*$ тепло подводят, далее~— отводят.
        Тут *~--- некоторая точка между точками 2 и 3, конкретные значения надо досчитать:
        $$V^* = V_0 \cdot \frac{5 \cdot 6}8 = \frac{15}4 \cdot V_0 \implies P^* = 6P_0 - \frac{P_0}{V_0} V^* = \ldots = \frac94 \cdot P_0.$$

        Т.е.
        чтобы вычислить $Q_+$, надо сложить количества теплоты на участке 12 и лишь части участка 23 — участке 2*,
        той его части где это количество теплоты положительно.
        Имеем: $Q_+ = Q_{12} + Q_{2*}$.

        Теперь возвращаемся к циклу целиком и получаем:
        \begin{align*}
        A_\text{цикл} &= \frac 12 \cdot (5P_0 - P_0) \cdot (5V_0 - V_0) = 8 \cdot P_0V_0, \\
        A_{2*} &= \frac{P^* + 5P_0}2 \cdot (V^* - V_0)
            = \frac{\frac94 \cdot P_0 + 5P_0}2 \cdot \cbr{\frac{15}4 \cdot V_0 - V_0}
            = \ldots = \frac{319}{32} \cdot P_0 V_0, \\
        \Delta U_{2*} &= \frac 32 \nu R T^* - \frac 32 \nu R T_2 = \frac 32 (P^*V^* - P_0 \cdot 5V_0)
            = \frac 32 \cbr{\frac94 \cdot P_0 \cdot \frac{15}4 \cdot V_0 - P_0 \cdot 5V_0}
            = \frac{165}{32} \cdot P_0 V_0, \\
        \Delta U_{12} &= \frac 32 \nu R T_2 - \frac 32 \nu R T_1 = \frac 32 (5P_0V_0 - P_0V_0) = \ldots = 6 \cdot P_0 V_0, \\
        \eta &= \frac{A_\text{цикл}}{Q_+} = \frac{A_\text{цикл}}{Q_{12} + Q_{2*}}
            = \frac{A_\text{цикл}}{A_{12} + \Delta U_{12} + A_{2*} + \Delta U_{2*}} = \\
            &= \frac{8 \cdot P_0V_0}{0 + 6 \cdot P_0 V_0 + \frac{319}{32} \cdot P_0 V_0 + \frac{165}{32} \cdot P_0 V_0}
             = \frac{A_bonus_cycle:LaTeX}{6 + \frac{319}{32} + \frac{165}{32}}
             = \frac{64}{169} \leftarrow \text{вжух и готово!}
        \end{align*}
}
\solutionspace{360pt}

\tasknumber{2}%
\task{%
    При температуре $20\celsius$ относительная влажность воздуха составляет $45\%$.
    \begin{itemize}
        \item Определите точку росы для этого воздуха.
        \item Какой станет относительная влажность этого воздуха, если нагреть его до $40\celsius$?
    \end{itemize}
}
\answer{%
    \begin{align*}
    &\text{Значения плотности насыщенного водяного пара определяем по таблице:} \\
    &\rho_{\text{нас.
    пара 20} \celsius} = 17{,}300\,\frac{\text{г}}{\text{м}^{3}}, \rho_{\text{нас.
    пара 40} \celsius} = 51{,}200\,\frac{\text{г}}{\text{м}^{3}}.
    \\
    \varphi_1 &= \frac{\rho_\text{пара}}{\rho_{\text{нас.
    пара 20} \celsius}} \implies {\rho_\text{пара}} = \rho_{\text{нас.
    пара 20} \celsius} \cdot \varphi_1 = 17{,}300\,\frac{\text{г}}{\text{м}^{3}} \cdot 0{,}45 = 7{,}785\,\frac{\text{г}}{\text{м}^{3}}.
    \\
    &\text{По таблице определяем, при какой температуре пар с такой плотностью станет насыщенным:}  \\
    t_\text{росы} &= 7{,}0\celsius, \\
    \varphi_2 &= \frac{\rho_\text{пара}}{\rho_{\text{нас.
    пара 40} \celsius}} = \frac{\rho_{\text{нас.
    пара 20} \celsius} \cdot \varphi_1}{\rho_{\text{нас.
    пара 40} \celsius}}= \varphi_1 \cdot \frac{\rho_{\text{нас.
    пара 20} \celsius}}{\rho_{\text{нас.
    пара 40} \celsius}} = 0{,}45 \cdot \frac{17{,}300\,\frac{\text{г}}{\text{м}^{3}}}{51{,}200\,\frac{\text{г}}{\text{м}^{3}}} = 0{,}152 \approx 15{,}2\%.
    \end{align*}
}
\solutionspace{80pt}

\tasknumber{3}%
\task{%
    Из уравнения состояния идеального газа выведите или выразите...
    \begin{enumerate}
        \item объём,
        \item температуру,
        \item плотность газа.
    \end{enumerate}
}

\tasknumber{4}%
\task{%
    Запишите формулы и рядом с каждой физичической величиной укажите её название и единицы измерения в СИ:
    \begin{enumerate}
        \item первое начало термодинамики,
        \item внутренняя энергия идеального одноатомного газа.
    \end{enumerate}
}

\variantsplitter

\addpersonalvariant{Артём Жичин}

\tasknumber{1}%
\task{%
    Определите КПД (оставив ответ точным в виде нескоратимой дроби) цикла 1231, рабочим телом которого является идеальный одноатомный газ, если
    \begin{itemize}
        \item 12 — изохорический нагрев в четыре раза,
        \item 23 — изобарическое расширение, при котором температура растёт в четыре раза,
        \item 31 — процесс, график которого в $PV$-координатах является отрезком прямой.
    \end{itemize}
    Бонус: замените цикл 1231 циклом, в котором 12 — изохорический нагрев в четыре раза, 23 — процесс, график которого в $PV$-координатах является отрезком прямой, 31 — изобарическое охлаждение, при котором температура падает в четыре раза.
}
\answer{%
    \begin{align*}
    A_{12} &= 0, \Delta U_{12} > 0, \implies Q_{12} = A_{12} + \Delta U_{12} > 0.
    \\
    A_{23} &> 0, \Delta U_{23} > 0, \implies Q_{23} = A_{23} + \Delta U_{23} > 0, \\
    A_{31} &= 0, \Delta U_{31} < 0, \implies Q_{31} = A_{31} + \Delta U_{31} < 0.
    \\
    P_1V_1 &= \nu R T_1, P_2V_2 = \nu R T_2, P_3V_3 = \nu R T_3 \text{ — уравнения состояния идеального газа}, \\
    &\text{Пусть $P_0$, $V_0$, $T_0$ — давление, объём и температура в точке 1 (минимальные во всём цикле):} \\
    P_1 &= P_0, P_2 = P_3, V_1 = V_2 = V_0, \text{остальные соотношения нужно считать} \\
    T_2 &= 4T_1 = 4T_0 \text{(по условию)} \implies \frac{P_2}{P_1} = \frac{P_2V_0}{P_1V_0} = \frac{P_2 V_2}{P_1 V_1}= \frac{\nu R T_2}{\nu R T_1} = \frac{T_2}{T_1} = 4 \implies P_2 = 4 P_1 = 4 P_0, \\
    T_3 &= 4T_2 = 16T_0 \text{(по условию)} \implies \frac{V_3}{V_2} = \frac{P_3V_3}{P_2V_2}= \frac{\nu R T_3}{\nu R T_2} = \frac{T_3}{T_2} = 4 \implies V_3 = 4 V_2 = 4 V_0.
    \\
    A_\text{цикл} &= \frac 12 (4P_0 - P_0)(4V_0 - V_0) = \frac 12 \cdot 9 \cdot P_0V_0, \\
    A_{23} &= 4P_0 \cdot (4V_0 - V_0) = 12P_0V_0, \\
    \Delta U_{23} &= \frac 32 \nu R T_3 - \frac 32 \nu R T_2 = \frac 32 P_3 V_3 - \frac 32 P_2 V_2 = \frac 32 \cdot 4 P_0 \cdot 4 V_0 -  \frac 32 \cdot 4 P_0 \cdot V_0 = \frac 32 \cdot 12 \cdot P_0V_0, \\
    \Delta U_{12} &= \frac 32 \nu R T_2 - \frac 32 \nu R T_1 = \frac 32 P_2 V_2 - \frac 32 P_1 V_1 = \frac 32 \cdot 4 P_0V_0 - \frac 32 P_0V_0 = \frac 32 \cdot 3 \cdot P_0V_0.
    \\
    \eta &= \frac{A_\text{цикл}}{Q_+} = \frac{A_\text{цикл}}{Q_{12} + Q_{23}}  = \frac{A_\text{цикл}}{A_{12} + \Delta U_{12} + A_{23} + \Delta U_{23}} =  \\
     &= \frac{\frac 12 \cdot 9 \cdot P_0V_0}{0 + \frac 32 \cdot 3 \cdot P_0V_0 + 12P_0V_0 + \frac 32 \cdot 12 \cdot P_0V_0} = \frac{\frac 12 \cdot 9}{\frac 32 \cdot 3 + 12 + \frac 32 \cdot 12} = \frac3{23} \approx 0.130.
    \end{align*}


        График процесса не в масштабе (эта часть пока не готова и сделать автоматически аккуратно сложно), но с верными подписями (а для решения этого достаточно):

        \begin{tikzpicture}[thick]
            \draw[-{Latex}] (0, 0) -- (0, 7) node[above left] {$P$};
            \draw[-{Latex}] (0, 0) -- (10, 0) node[right] {$V$};

            \draw[dashed] (0, 2) node[left] {$P_1 = P_0$} -| (3, 0) node[below] {$V_1 = V_2 = V_0$};
            \draw[dashed] (0, 6) node[left] {$P_2 = P_3 = 4P_0$} -| (9, 0) node[below] {$V_3 = 4V_0$};

            \draw (3, 2) node[above left]{1} node[below left]{$T_1 = T_0$}
                   (3, 6) node[below left]{2} node[above]{$T_2 = 4T_0$}
                   (9, 6) node[above right]{3} node[below right]{$T_3 = 16T_0$};
            \draw[midar] (3, 2) -- (3, 6);
            \draw[midar] (3, 6) -- (9, 6);
            \draw[midar] (9, 6) -- (3, 2);
        \end{tikzpicture}

        Решение бонуса:
        \begin{align*}
            A_{12} &= 0, \Delta U_{12} > 0, \implies Q_{12} = A_{12} + \Delta U_{12} > 0, \\
            A_{23} &> 0, \Delta U_{23} \text{ — ничего нельзя сказать, нужно исследовать отдельно}, \\
            A_{31} &< 0, \Delta U_{31} < 0, \implies Q_{31} = A_{31} + \Delta U_{31} < 0.
            \\
        \end{align*}

        Уравнения состояния идеального газа для точек 1, 2, 3: $P_1V_1 = \nu R T_1, P_2V_2 = \nu R T_2, P_3V_3 = \nu R T_3$.
        Пусть $P_0$, $V_0$, $T_0$ — давление, объём и температура в точке 1 (минимальные во всём цикле).

        12 --- изохора, $\frac{P_1V_1}{T_1} = \nu R = \frac{P_2V_2}{T_2}, V_2=V_1=V_0 \implies \frac{P_1}{T_1} =  \frac{P_2}{T_2} \implies P_2 = P_1 \frac{T_2}{T_1} = 4P_0$,

        31 --- изобара, $\frac{P_1V_1}{T_1} = \nu R = \frac{P_3V_3}{T_3}, P_3=P_1=P_0 \implies \frac{V_3}{T_3} =  \frac{V_1}{T_1} \implies V_3 = V_1 \frac{T_3}{T_1} = 4V_0$,

        Таким образом, используя новые обозначения, состояния газа в точках 1, 2 и 3 описываются макропараметрами $(P_0, V_0, T_0), (4P_0, V_0, 4T_0), (P_0, 4V_0, 4T_0)$ соответственно.

        \begin{tikzpicture}[thick]
            \draw[-{Latex}] (0, 0) -- (0, 7) node[above left] {$P$};
            \draw[-{Latex}] (0, 0) -- (10, 0) node[right] {$V$};

            \draw[dashed] (0, 2) node[left] {$P_1 = P_3 = P_0$} -| (9, 0) node[below] {$V_3 = 4V_0$};
            \draw[dashed] (0, 6) node[left] {$P_2 = 4P_0$} -| (3, 0) node[below] {$V_1 = V_2 = V_0$};

            \draw[dashed] (0, 5) node[left] {$P$} -| (4.5, 0) node[below] {$V$};
            \draw[dashed] (0, 4.6) node[left] {$P'$} -| (5.1, 0) node[below] {$V'$};

            \draw (3, 2) node[above left]{1} node[below left]{$T_1 = T_0$}
                   (3, 6) node[below left]{2} node[above]{$T_2 = 4T_0$}
                   (9, 2) node[above right]{3} node[below right]{$T_3 = 4T_0$};
            \draw[midar] (3, 2) -- (3, 6);
            \draw[midar] (3, 6) -- (9, 2);
            \draw[midar] (9, 2) -- (3, 2);
            \draw   (4.5, 5) node[above right]{$T$} (5.1, 4.6) node[above right]{$T'$};
        \end{tikzpicture}


        Теперь рассмотрим отдельно процесс 23, к остальному вернёмся позже.
        Уравнение этой прямой в $PV$-координатах: $P(V) = 5P_0 - \frac{P_0}{V_0} V$.
        Это значит, что при изменении объёма на $\Delta V$ давление изменится на $\Delta P = - \frac{P_0}{V_0} \Delta V$, обратите внимание на знак.

        Рассмотрим произвольную точку в процессе 23 и дадим процессу ещё немного свершиться, при этом объём изменится на $\Delta V$, давление — на $\Delta P$, температура (иначе бы была гипербола, а не прямая) — на $\Delta T$,
        т.е.
        из состояния $(P, V, T)$ мы перешли в $(P', V', T')$, причём  $P' = P + \Delta P, V' = V + \Delta V, T' = T + \Delta T$.

        При этом изменится внутренняя энергия:
        \begin{align*}
        \Delta U
            &= U' - U = \frac 32 \nu R T' - \frac 32 \nu R T = \frac 32 (P+\Delta P) (V+\Delta V) - \frac 32 PV\\
            &= \frac 32 ((P+\Delta P) (V+\Delta V) - PV) = \frac 32 (P\Delta V + V \Delta P + \Delta P \Delta V).
        \end{align*}

        Рассмотрим малые изменения объёма, тогда и изменение давления будем малым (т.к.
        $\Delta P = - \frac{P_0}{V_0} \Delta V$),
        а третьим слагаемым в выражении для $\Delta U$  можно пренебречь по сравнению с двумя другими:
        два первых это малые величины, а третье — произведение двух малых.
        Тогда $\Delta U = \frac 32 (P\Delta V + V \Delta P)$.

        Работа газа при этом малом изменении объёма — это площадь трапеции (тут ещё раз пренебрегли малым слагаемым):
        $$A = \frac{P + P'}2 \Delta V = \cbr{P + \frac{\Delta P}2} \Delta V = P \Delta V.$$

        Подведённое количество теплоты, используя первое начало термодинамики, будет равно
        \begin{align*}
        Q
            &= \frac 32 (P\Delta V + V \Delta P) + P \Delta V =  \frac 52 P\Delta V + \frac 32 V \Delta P = \\
            &= \frac 52 P\Delta V + \frac 32 V \cdot \cbr{- \frac{P_0}{V_0} \Delta V} = \frac{\Delta V}2 \cdot \cbr{5P - \frac{P_0}{V_0} V} = \\
            &= \frac{\Delta V}2 \cdot \cbr{5 \cdot \cbr{5P_0 - \frac{P_0}{V_0} V} - \frac{P_0}{V_0} V}
             = \frac{\Delta V \cdot P_0}2 \cdot \cbr{5 \cdot 5 - 8\frac V{V_0}}.
        \end{align*}

        Таком образом, знак количества теплоты $Q$ на участке 23 зависит от конкретного значения $V$:
        \begin{itemize}
            \item $\Delta V > 0$ на всём участке 23, поскольку газ расширяется,
            \item $P > 0$ — всегда, у нас идеальный газ, удары о стенки сосуда абсолютно упругие, а молекулы не взаимодействуют и поэтому давление только положительно,
            \item если $5 \cdot 5 - 8\frac V{V_0} > 0$ — тепло подводят, если же меньше нуля — отводят.
        \end{itemize}
        Решая последнее неравенство, получаем конкретное значение $V^*$: при $V < V^*$ тепло подводят, далее~— отводят.
        Тут *~--- некоторая точка между точками 2 и 3, конкретные значения надо досчитать:
        $$V^* = V_0 \cdot \frac{5 \cdot 5}8 = \frac{25}8 \cdot V_0 \implies P^* = 5P_0 - \frac{P_0}{V_0} V^* = \ldots = \frac{15}8 \cdot P_0.$$

        Т.е.
        чтобы вычислить $Q_+$, надо сложить количества теплоты на участке 12 и лишь части участка 23 — участке 2*,
        той его части где это количество теплоты положительно.
        Имеем: $Q_+ = Q_{12} + Q_{2*}$.

        Теперь возвращаемся к циклу целиком и получаем:
        \begin{align*}
        A_\text{цикл} &= \frac 12 \cdot (4P_0 - P_0) \cdot (4V_0 - V_0) = \frac92 \cdot P_0V_0, \\
        A_{2*} &= \frac{P^* + 4P_0}2 \cdot (V^* - V_0)
            = \frac{\frac{15}8 \cdot P_0 + 4P_0}2 \cdot \cbr{\frac{25}8 \cdot V_0 - V_0}
            = \ldots = \frac{799}{128} \cdot P_0 V_0, \\
        \Delta U_{2*} &= \frac 32 \nu R T^* - \frac 32 \nu R T_2 = \frac 32 (P^*V^* - P_0 \cdot 4V_0)
            = \frac 32 \cbr{\frac{15}8 \cdot P_0 \cdot \frac{25}8 \cdot V_0 - P_0 \cdot 4V_0}
            = \frac{357}{128} \cdot P_0 V_0, \\
        \Delta U_{12} &= \frac 32 \nu R T_2 - \frac 32 \nu R T_1 = \frac 32 (4P_0V_0 - P_0V_0) = \ldots = \frac92 \cdot P_0 V_0, \\
        \eta &= \frac{A_\text{цикл}}{Q_+} = \frac{A_\text{цикл}}{Q_{12} + Q_{2*}}
            = \frac{A_\text{цикл}}{A_{12} + \Delta U_{12} + A_{2*} + \Delta U_{2*}} = \\
            &= \frac{\frac92 \cdot P_0V_0}{0 + \frac92 \cdot P_0 V_0 + \frac{799}{128} \cdot P_0 V_0 + \frac{357}{128} \cdot P_0 V_0}
             = \frac{A_bonus_cycle:LaTeX}{\frac92 + \frac{799}{128} + \frac{357}{128}}
             = \frac{144}{433} \leftarrow \text{вжух и готово!}
        \end{align*}
}
\solutionspace{360pt}

\tasknumber{2}%
\task{%
    При температуре $20\celsius$ относительная влажность воздуха составляет $50\%$.
    \begin{itemize}
        \item Определите точку росы для этого воздуха.
        \item Какой станет относительная влажность этого воздуха, если нагреть его до $80\celsius$?
    \end{itemize}
}
\answer{%
    \begin{align*}
    &\text{Значения плотности насыщенного водяного пара определяем по таблице:} \\
    &\rho_{\text{нас.
    пара 20} \celsius} = 17{,}300\,\frac{\text{г}}{\text{м}^{3}}, \rho_{\text{нас.
    пара 80} \celsius} = 293{,}000\,\frac{\text{г}}{\text{м}^{3}}.
    \\
    \varphi_1 &= \frac{\rho_\text{пара}}{\rho_{\text{нас.
    пара 20} \celsius}} \implies {\rho_\text{пара}} = \rho_{\text{нас.
    пара 20} \celsius} \cdot \varphi_1 = 17{,}300\,\frac{\text{г}}{\text{м}^{3}} \cdot 0{,}50 = 8{,}650\,\frac{\text{г}}{\text{м}^{3}}.
    \\
    &\text{По таблице определяем, при какой температуре пар с такой плотностью станет насыщенным:}  \\
    t_\text{росы} &= 8{,}7\celsius, \\
    \varphi_2 &= \frac{\rho_\text{пара}}{\rho_{\text{нас.
    пара 80} \celsius}} = \frac{\rho_{\text{нас.
    пара 20} \celsius} \cdot \varphi_1}{\rho_{\text{нас.
    пара 80} \celsius}}= \varphi_1 \cdot \frac{\rho_{\text{нас.
    пара 20} \celsius}}{\rho_{\text{нас.
    пара 80} \celsius}} = 0{,}50 \cdot \frac{17{,}300\,\frac{\text{г}}{\text{м}^{3}}}{293{,}000\,\frac{\text{г}}{\text{м}^{3}}} = 0{,}030 \approx 3{,}0\%.
    \end{align*}
}
\solutionspace{80pt}

\tasknumber{3}%
\task{%
    Из уравнения состояния идеального газа выведите или выразите...
    \begin{enumerate}
        \item объём,
        \item молярную массу,
        \item концентрацию молекул газа.
    \end{enumerate}
}

\tasknumber{4}%
\task{%
    Запишите формулы и рядом с каждой физичической величиной укажите её название и единицы измерения в СИ:
    \begin{enumerate}
        \item первое начало термодинамики,
        \item внутренняя энергия идеального одноатомного газа.
    \end{enumerate}
}

\variantsplitter

\addpersonalvariant{Дарья Кошман}

\tasknumber{1}%
\task{%
    Определите КПД (оставив ответ точным в виде нескоратимой дроби) цикла 1231, рабочим телом которого является идеальный одноатомный газ, если
    \begin{itemize}
        \item 12 — изохорический нагрев в шесть раз,
        \item 23 — изобарическое расширение, при котором температура растёт в три раза,
        \item 31 — процесс, график которого в $PV$-координатах является отрезком прямой.
    \end{itemize}
    Бонус: замените цикл 1231 циклом, в котором 12 — изохорический нагрев в шесть раз, 23 — процесс, график которого в $PV$-координатах является отрезком прямой, 31 — изобарическое охлаждение, при котором температура падает в шесть раз.
}
\answer{%
    \begin{align*}
    A_{12} &= 0, \Delta U_{12} > 0, \implies Q_{12} = A_{12} + \Delta U_{12} > 0.
    \\
    A_{23} &> 0, \Delta U_{23} > 0, \implies Q_{23} = A_{23} + \Delta U_{23} > 0, \\
    A_{31} &= 0, \Delta U_{31} < 0, \implies Q_{31} = A_{31} + \Delta U_{31} < 0.
    \\
    P_1V_1 &= \nu R T_1, P_2V_2 = \nu R T_2, P_3V_3 = \nu R T_3 \text{ — уравнения состояния идеального газа}, \\
    &\text{Пусть $P_0$, $V_0$, $T_0$ — давление, объём и температура в точке 1 (минимальные во всём цикле):} \\
    P_1 &= P_0, P_2 = P_3, V_1 = V_2 = V_0, \text{остальные соотношения нужно считать} \\
    T_2 &= 6T_1 = 6T_0 \text{(по условию)} \implies \frac{P_2}{P_1} = \frac{P_2V_0}{P_1V_0} = \frac{P_2 V_2}{P_1 V_1}= \frac{\nu R T_2}{\nu R T_1} = \frac{T_2}{T_1} = 6 \implies P_2 = 6 P_1 = 6 P_0, \\
    T_3 &= 3T_2 = 18T_0 \text{(по условию)} \implies \frac{V_3}{V_2} = \frac{P_3V_3}{P_2V_2}= \frac{\nu R T_3}{\nu R T_2} = \frac{T_3}{T_2} = 3 \implies V_3 = 3 V_2 = 3 V_0.
    \\
    A_\text{цикл} &= \frac 12 (3P_0 - P_0)(6V_0 - V_0) = \frac 12 \cdot 10 \cdot P_0V_0, \\
    A_{23} &= 6P_0 \cdot (3V_0 - V_0) = 12P_0V_0, \\
    \Delta U_{23} &= \frac 32 \nu R T_3 - \frac 32 \nu R T_2 = \frac 32 P_3 V_3 - \frac 32 P_2 V_2 = \frac 32 \cdot 6 P_0 \cdot 3 V_0 -  \frac 32 \cdot 6 P_0 \cdot V_0 = \frac 32 \cdot 12 \cdot P_0V_0, \\
    \Delta U_{12} &= \frac 32 \nu R T_2 - \frac 32 \nu R T_1 = \frac 32 P_2 V_2 - \frac 32 P_1 V_1 = \frac 32 \cdot 6 P_0V_0 - \frac 32 P_0V_0 = \frac 32 \cdot 5 \cdot P_0V_0.
    \\
    \eta &= \frac{A_\text{цикл}}{Q_+} = \frac{A_\text{цикл}}{Q_{12} + Q_{23}}  = \frac{A_\text{цикл}}{A_{12} + \Delta U_{12} + A_{23} + \Delta U_{23}} =  \\
     &= \frac{\frac 12 \cdot 10 \cdot P_0V_0}{0 + \frac 32 \cdot 5 \cdot P_0V_0 + 12P_0V_0 + \frac 32 \cdot 12 \cdot P_0V_0} = \frac{\frac 12 \cdot 10}{\frac 32 \cdot 5 + 12 + \frac 32 \cdot 12} = \frac2{15} \approx 0.133.
    \end{align*}


        График процесса не в масштабе (эта часть пока не готова и сделать автоматически аккуратно сложно), но с верными подписями (а для решения этого достаточно):

        \begin{tikzpicture}[thick]
            \draw[-{Latex}] (0, 0) -- (0, 7) node[above left] {$P$};
            \draw[-{Latex}] (0, 0) -- (10, 0) node[right] {$V$};

            \draw[dashed] (0, 2) node[left] {$P_1 = P_0$} -| (3, 0) node[below] {$V_1 = V_2 = V_0$};
            \draw[dashed] (0, 6) node[left] {$P_2 = P_3 = 6P_0$} -| (9, 0) node[below] {$V_3 = 3V_0$};

            \draw (3, 2) node[above left]{1} node[below left]{$T_1 = T_0$}
                   (3, 6) node[below left]{2} node[above]{$T_2 = 6T_0$}
                   (9, 6) node[above right]{3} node[below right]{$T_3 = 18T_0$};
            \draw[midar] (3, 2) -- (3, 6);
            \draw[midar] (3, 6) -- (9, 6);
            \draw[midar] (9, 6) -- (3, 2);
        \end{tikzpicture}

        Решение бонуса:
        \begin{align*}
            A_{12} &= 0, \Delta U_{12} > 0, \implies Q_{12} = A_{12} + \Delta U_{12} > 0, \\
            A_{23} &> 0, \Delta U_{23} \text{ — ничего нельзя сказать, нужно исследовать отдельно}, \\
            A_{31} &< 0, \Delta U_{31} < 0, \implies Q_{31} = A_{31} + \Delta U_{31} < 0.
            \\
        \end{align*}

        Уравнения состояния идеального газа для точек 1, 2, 3: $P_1V_1 = \nu R T_1, P_2V_2 = \nu R T_2, P_3V_3 = \nu R T_3$.
        Пусть $P_0$, $V_0$, $T_0$ — давление, объём и температура в точке 1 (минимальные во всём цикле).

        12 --- изохора, $\frac{P_1V_1}{T_1} = \nu R = \frac{P_2V_2}{T_2}, V_2=V_1=V_0 \implies \frac{P_1}{T_1} =  \frac{P_2}{T_2} \implies P_2 = P_1 \frac{T_2}{T_1} = 6P_0$,

        31 --- изобара, $\frac{P_1V_1}{T_1} = \nu R = \frac{P_3V_3}{T_3}, P_3=P_1=P_0 \implies \frac{V_3}{T_3} =  \frac{V_1}{T_1} \implies V_3 = V_1 \frac{T_3}{T_1} = 6V_0$,

        Таким образом, используя новые обозначения, состояния газа в точках 1, 2 и 3 описываются макропараметрами $(P_0, V_0, T_0), (6P_0, V_0, 6T_0), (P_0, 6V_0, 6T_0)$ соответственно.

        \begin{tikzpicture}[thick]
            \draw[-{Latex}] (0, 0) -- (0, 7) node[above left] {$P$};
            \draw[-{Latex}] (0, 0) -- (10, 0) node[right] {$V$};

            \draw[dashed] (0, 2) node[left] {$P_1 = P_3 = P_0$} -| (9, 0) node[below] {$V_3 = 6V_0$};
            \draw[dashed] (0, 6) node[left] {$P_2 = 6P_0$} -| (3, 0) node[below] {$V_1 = V_2 = V_0$};

            \draw[dashed] (0, 5) node[left] {$P$} -| (4.5, 0) node[below] {$V$};
            \draw[dashed] (0, 4.6) node[left] {$P'$} -| (5.1, 0) node[below] {$V'$};

            \draw (3, 2) node[above left]{1} node[below left]{$T_1 = T_0$}
                   (3, 6) node[below left]{2} node[above]{$T_2 = 6T_0$}
                   (9, 2) node[above right]{3} node[below right]{$T_3 = 6T_0$};
            \draw[midar] (3, 2) -- (3, 6);
            \draw[midar] (3, 6) -- (9, 2);
            \draw[midar] (9, 2) -- (3, 2);
            \draw   (4.5, 5) node[above right]{$T$} (5.1, 4.6) node[above right]{$T'$};
        \end{tikzpicture}


        Теперь рассмотрим отдельно процесс 23, к остальному вернёмся позже.
        Уравнение этой прямой в $PV$-координатах: $P(V) = 7P_0 - \frac{P_0}{V_0} V$.
        Это значит, что при изменении объёма на $\Delta V$ давление изменится на $\Delta P = - \frac{P_0}{V_0} \Delta V$, обратите внимание на знак.

        Рассмотрим произвольную точку в процессе 23 и дадим процессу ещё немного свершиться, при этом объём изменится на $\Delta V$, давление — на $\Delta P$, температура (иначе бы была гипербола, а не прямая) — на $\Delta T$,
        т.е.
        из состояния $(P, V, T)$ мы перешли в $(P', V', T')$, причём  $P' = P + \Delta P, V' = V + \Delta V, T' = T + \Delta T$.

        При этом изменится внутренняя энергия:
        \begin{align*}
        \Delta U
            &= U' - U = \frac 32 \nu R T' - \frac 32 \nu R T = \frac 32 (P+\Delta P) (V+\Delta V) - \frac 32 PV\\
            &= \frac 32 ((P+\Delta P) (V+\Delta V) - PV) = \frac 32 (P\Delta V + V \Delta P + \Delta P \Delta V).
        \end{align*}

        Рассмотрим малые изменения объёма, тогда и изменение давления будем малым (т.к.
        $\Delta P = - \frac{P_0}{V_0} \Delta V$),
        а третьим слагаемым в выражении для $\Delta U$  можно пренебречь по сравнению с двумя другими:
        два первых это малые величины, а третье — произведение двух малых.
        Тогда $\Delta U = \frac 32 (P\Delta V + V \Delta P)$.

        Работа газа при этом малом изменении объёма — это площадь трапеции (тут ещё раз пренебрегли малым слагаемым):
        $$A = \frac{P + P'}2 \Delta V = \cbr{P + \frac{\Delta P}2} \Delta V = P \Delta V.$$

        Подведённое количество теплоты, используя первое начало термодинамики, будет равно
        \begin{align*}
        Q
            &= \frac 32 (P\Delta V + V \Delta P) + P \Delta V =  \frac 52 P\Delta V + \frac 32 V \Delta P = \\
            &= \frac 52 P\Delta V + \frac 32 V \cdot \cbr{- \frac{P_0}{V_0} \Delta V} = \frac{\Delta V}2 \cdot \cbr{5P - \frac{P_0}{V_0} V} = \\
            &= \frac{\Delta V}2 \cdot \cbr{5 \cdot \cbr{7P_0 - \frac{P_0}{V_0} V} - \frac{P_0}{V_0} V}
             = \frac{\Delta V \cdot P_0}2 \cdot \cbr{5 \cdot 7 - 8\frac V{V_0}}.
        \end{align*}

        Таком образом, знак количества теплоты $Q$ на участке 23 зависит от конкретного значения $V$:
        \begin{itemize}
            \item $\Delta V > 0$ на всём участке 23, поскольку газ расширяется,
            \item $P > 0$ — всегда, у нас идеальный газ, удары о стенки сосуда абсолютно упругие, а молекулы не взаимодействуют и поэтому давление только положительно,
            \item если $5 \cdot 7 - 8\frac V{V_0} > 0$ — тепло подводят, если же меньше нуля — отводят.
        \end{itemize}
        Решая последнее неравенство, получаем конкретное значение $V^*$: при $V < V^*$ тепло подводят, далее~— отводят.
        Тут *~--- некоторая точка между точками 2 и 3, конкретные значения надо досчитать:
        $$V^* = V_0 \cdot \frac{5 \cdot 7}8 = \frac{35}8 \cdot V_0 \implies P^* = 7P_0 - \frac{P_0}{V_0} V^* = \ldots = \frac{21}8 \cdot P_0.$$

        Т.е.
        чтобы вычислить $Q_+$, надо сложить количества теплоты на участке 12 и лишь части участка 23 — участке 2*,
        той его части где это количество теплоты положительно.
        Имеем: $Q_+ = Q_{12} + Q_{2*}$.

        Теперь возвращаемся к циклу целиком и получаем:
        \begin{align*}
        A_\text{цикл} &= \frac 12 \cdot (6P_0 - P_0) \cdot (6V_0 - V_0) = \frac{25}2 \cdot P_0V_0, \\
        A_{2*} &= \frac{P^* + 6P_0}2 \cdot (V^* - V_0)
            = \frac{\frac{21}8 \cdot P_0 + 6P_0}2 \cdot \cbr{\frac{35}8 \cdot V_0 - V_0}
            = \ldots = \frac{1863}{128} \cdot P_0 V_0, \\
        \Delta U_{2*} &= \frac 32 \nu R T^* - \frac 32 \nu R T_2 = \frac 32 (P^*V^* - P_0 \cdot 6V_0)
            = \frac 32 \cbr{\frac{21}8 \cdot P_0 \cdot \frac{35}8 \cdot V_0 - P_0 \cdot 6V_0}
            = \frac{1053}{128} \cdot P_0 V_0, \\
        \Delta U_{12} &= \frac 32 \nu R T_2 - \frac 32 \nu R T_1 = \frac 32 (6P_0V_0 - P_0V_0) = \ldots = \frac{15}2 \cdot P_0 V_0, \\
        \eta &= \frac{A_\text{цикл}}{Q_+} = \frac{A_\text{цикл}}{Q_{12} + Q_{2*}}
            = \frac{A_\text{цикл}}{A_{12} + \Delta U_{12} + A_{2*} + \Delta U_{2*}} = \\
            &= \frac{\frac{25}2 \cdot P_0V_0}{0 + \frac{15}2 \cdot P_0 V_0 + \frac{1863}{128} \cdot P_0 V_0 + \frac{1053}{128} \cdot P_0 V_0}
             = \frac{A_bonus_cycle:LaTeX}{\frac{15}2 + \frac{1863}{128} + \frac{1053}{128}}
             = \frac{400}{969} \leftarrow \text{вжух и готово!}
        \end{align*}
}
\solutionspace{360pt}

\tasknumber{2}%
\task{%
    При температуре $30\celsius$ относительная влажность воздуха составляет $65\%$.
    \begin{itemize}
        \item Определите точку росы для этого воздуха.
        \item Какой станет относительная влажность этого воздуха, если нагреть его до $80\celsius$?
    \end{itemize}
}
\answer{%
    \begin{align*}
    &\text{Значения плотности насыщенного водяного пара определяем по таблице:} \\
    &\rho_{\text{нас.
    пара 30} \celsius} = 30{,}300\,\frac{\text{г}}{\text{м}^{3}}, \rho_{\text{нас.
    пара 80} \celsius} = 293{,}000\,\frac{\text{г}}{\text{м}^{3}}.
    \\
    \varphi_1 &= \frac{\rho_\text{пара}}{\rho_{\text{нас.
    пара 30} \celsius}} \implies {\rho_\text{пара}} = \rho_{\text{нас.
    пара 30} \celsius} \cdot \varphi_1 = 30{,}300\,\frac{\text{г}}{\text{м}^{3}} \cdot 0{,}65 = 19{,}695\,\frac{\text{г}}{\text{м}^{3}}.
    \\
    &\text{По таблице определяем, при какой температуре пар с такой плотностью станет насыщенным:}  \\
    t_\text{росы} &= 22{,}2\celsius, \\
    \varphi_2 &= \frac{\rho_\text{пара}}{\rho_{\text{нас.
    пара 80} \celsius}} = \frac{\rho_{\text{нас.
    пара 30} \celsius} \cdot \varphi_1}{\rho_{\text{нас.
    пара 80} \celsius}}= \varphi_1 \cdot \frac{\rho_{\text{нас.
    пара 30} \celsius}}{\rho_{\text{нас.
    пара 80} \celsius}} = 0{,}65 \cdot \frac{30{,}300\,\frac{\text{г}}{\text{м}^{3}}}{293{,}000\,\frac{\text{г}}{\text{м}^{3}}} = 0{,}067 \approx 6{,}7\%.
    \end{align*}
}
\solutionspace{80pt}

\tasknumber{3}%
\task{%
    Из уравнения состояния идеального газа выведите или выразите...
    \begin{enumerate}
        \item объём,
        \item температуру,
        \item концентрацию молекул газа.
    \end{enumerate}
}

\tasknumber{4}%
\task{%
    Запишите формулы и рядом с каждой физичической величиной укажите её название и единицы измерения в СИ:
    \begin{enumerate}
        \item первое начало термодинамики,
        \item внутренняя энергия идеального одноатомного газа.
    \end{enumerate}
}

\variantsplitter

\addpersonalvariant{Анна Кузьмичёва}

\tasknumber{1}%
\task{%
    Определите КПД (оставив ответ точным в виде нескоратимой дроби) цикла 1231, рабочим телом которого является идеальный одноатомный газ, если
    \begin{itemize}
        \item 12 — изохорический нагрев в пять раз,
        \item 23 — изобарическое расширение, при котором температура растёт в шесть раз,
        \item 31 — процесс, график которого в $PV$-координатах является отрезком прямой.
    \end{itemize}
    Бонус: замените цикл 1231 циклом, в котором 12 — изохорический нагрев в пять раз, 23 — процесс, график которого в $PV$-координатах является отрезком прямой, 31 — изобарическое охлаждение, при котором температура падает в пять раз.
}
\answer{%
    \begin{align*}
    A_{12} &= 0, \Delta U_{12} > 0, \implies Q_{12} = A_{12} + \Delta U_{12} > 0.
    \\
    A_{23} &> 0, \Delta U_{23} > 0, \implies Q_{23} = A_{23} + \Delta U_{23} > 0, \\
    A_{31} &= 0, \Delta U_{31} < 0, \implies Q_{31} = A_{31} + \Delta U_{31} < 0.
    \\
    P_1V_1 &= \nu R T_1, P_2V_2 = \nu R T_2, P_3V_3 = \nu R T_3 \text{ — уравнения состояния идеального газа}, \\
    &\text{Пусть $P_0$, $V_0$, $T_0$ — давление, объём и температура в точке 1 (минимальные во всём цикле):} \\
    P_1 &= P_0, P_2 = P_3, V_1 = V_2 = V_0, \text{остальные соотношения нужно считать} \\
    T_2 &= 5T_1 = 5T_0 \text{(по условию)} \implies \frac{P_2}{P_1} = \frac{P_2V_0}{P_1V_0} = \frac{P_2 V_2}{P_1 V_1}= \frac{\nu R T_2}{\nu R T_1} = \frac{T_2}{T_1} = 5 \implies P_2 = 5 P_1 = 5 P_0, \\
    T_3 &= 6T_2 = 30T_0 \text{(по условию)} \implies \frac{V_3}{V_2} = \frac{P_3V_3}{P_2V_2}= \frac{\nu R T_3}{\nu R T_2} = \frac{T_3}{T_2} = 6 \implies V_3 = 6 V_2 = 6 V_0.
    \\
    A_\text{цикл} &= \frac 12 (6P_0 - P_0)(5V_0 - V_0) = \frac 12 \cdot 20 \cdot P_0V_0, \\
    A_{23} &= 5P_0 \cdot (6V_0 - V_0) = 25P_0V_0, \\
    \Delta U_{23} &= \frac 32 \nu R T_3 - \frac 32 \nu R T_2 = \frac 32 P_3 V_3 - \frac 32 P_2 V_2 = \frac 32 \cdot 5 P_0 \cdot 6 V_0 -  \frac 32 \cdot 5 P_0 \cdot V_0 = \frac 32 \cdot 25 \cdot P_0V_0, \\
    \Delta U_{12} &= \frac 32 \nu R T_2 - \frac 32 \nu R T_1 = \frac 32 P_2 V_2 - \frac 32 P_1 V_1 = \frac 32 \cdot 5 P_0V_0 - \frac 32 P_0V_0 = \frac 32 \cdot 4 \cdot P_0V_0.
    \\
    \eta &= \frac{A_\text{цикл}}{Q_+} = \frac{A_\text{цикл}}{Q_{12} + Q_{23}}  = \frac{A_\text{цикл}}{A_{12} + \Delta U_{12} + A_{23} + \Delta U_{23}} =  \\
     &= \frac{\frac 12 \cdot 20 \cdot P_0V_0}{0 + \frac 32 \cdot 4 \cdot P_0V_0 + 25P_0V_0 + \frac 32 \cdot 25 \cdot P_0V_0} = \frac{\frac 12 \cdot 20}{\frac 32 \cdot 4 + 25 + \frac 32 \cdot 25} = \frac{20}{137} \approx 0.146.
    \end{align*}


        График процесса не в масштабе (эта часть пока не готова и сделать автоматически аккуратно сложно), но с верными подписями (а для решения этого достаточно):

        \begin{tikzpicture}[thick]
            \draw[-{Latex}] (0, 0) -- (0, 7) node[above left] {$P$};
            \draw[-{Latex}] (0, 0) -- (10, 0) node[right] {$V$};

            \draw[dashed] (0, 2) node[left] {$P_1 = P_0$} -| (3, 0) node[below] {$V_1 = V_2 = V_0$};
            \draw[dashed] (0, 6) node[left] {$P_2 = P_3 = 5P_0$} -| (9, 0) node[below] {$V_3 = 6V_0$};

            \draw (3, 2) node[above left]{1} node[below left]{$T_1 = T_0$}
                   (3, 6) node[below left]{2} node[above]{$T_2 = 5T_0$}
                   (9, 6) node[above right]{3} node[below right]{$T_3 = 30T_0$};
            \draw[midar] (3, 2) -- (3, 6);
            \draw[midar] (3, 6) -- (9, 6);
            \draw[midar] (9, 6) -- (3, 2);
        \end{tikzpicture}

        Решение бонуса:
        \begin{align*}
            A_{12} &= 0, \Delta U_{12} > 0, \implies Q_{12} = A_{12} + \Delta U_{12} > 0, \\
            A_{23} &> 0, \Delta U_{23} \text{ — ничего нельзя сказать, нужно исследовать отдельно}, \\
            A_{31} &< 0, \Delta U_{31} < 0, \implies Q_{31} = A_{31} + \Delta U_{31} < 0.
            \\
        \end{align*}

        Уравнения состояния идеального газа для точек 1, 2, 3: $P_1V_1 = \nu R T_1, P_2V_2 = \nu R T_2, P_3V_3 = \nu R T_3$.
        Пусть $P_0$, $V_0$, $T_0$ — давление, объём и температура в точке 1 (минимальные во всём цикле).

        12 --- изохора, $\frac{P_1V_1}{T_1} = \nu R = \frac{P_2V_2}{T_2}, V_2=V_1=V_0 \implies \frac{P_1}{T_1} =  \frac{P_2}{T_2} \implies P_2 = P_1 \frac{T_2}{T_1} = 5P_0$,

        31 --- изобара, $\frac{P_1V_1}{T_1} = \nu R = \frac{P_3V_3}{T_3}, P_3=P_1=P_0 \implies \frac{V_3}{T_3} =  \frac{V_1}{T_1} \implies V_3 = V_1 \frac{T_3}{T_1} = 5V_0$,

        Таким образом, используя новые обозначения, состояния газа в точках 1, 2 и 3 описываются макропараметрами $(P_0, V_0, T_0), (5P_0, V_0, 5T_0), (P_0, 5V_0, 5T_0)$ соответственно.

        \begin{tikzpicture}[thick]
            \draw[-{Latex}] (0, 0) -- (0, 7) node[above left] {$P$};
            \draw[-{Latex}] (0, 0) -- (10, 0) node[right] {$V$};

            \draw[dashed] (0, 2) node[left] {$P_1 = P_3 = P_0$} -| (9, 0) node[below] {$V_3 = 5V_0$};
            \draw[dashed] (0, 6) node[left] {$P_2 = 5P_0$} -| (3, 0) node[below] {$V_1 = V_2 = V_0$};

            \draw[dashed] (0, 5) node[left] {$P$} -| (4.5, 0) node[below] {$V$};
            \draw[dashed] (0, 4.6) node[left] {$P'$} -| (5.1, 0) node[below] {$V'$};

            \draw (3, 2) node[above left]{1} node[below left]{$T_1 = T_0$}
                   (3, 6) node[below left]{2} node[above]{$T_2 = 5T_0$}
                   (9, 2) node[above right]{3} node[below right]{$T_3 = 5T_0$};
            \draw[midar] (3, 2) -- (3, 6);
            \draw[midar] (3, 6) -- (9, 2);
            \draw[midar] (9, 2) -- (3, 2);
            \draw   (4.5, 5) node[above right]{$T$} (5.1, 4.6) node[above right]{$T'$};
        \end{tikzpicture}


        Теперь рассмотрим отдельно процесс 23, к остальному вернёмся позже.
        Уравнение этой прямой в $PV$-координатах: $P(V) = 6P_0 - \frac{P_0}{V_0} V$.
        Это значит, что при изменении объёма на $\Delta V$ давление изменится на $\Delta P = - \frac{P_0}{V_0} \Delta V$, обратите внимание на знак.

        Рассмотрим произвольную точку в процессе 23 и дадим процессу ещё немного свершиться, при этом объём изменится на $\Delta V$, давление — на $\Delta P$, температура (иначе бы была гипербола, а не прямая) — на $\Delta T$,
        т.е.
        из состояния $(P, V, T)$ мы перешли в $(P', V', T')$, причём  $P' = P + \Delta P, V' = V + \Delta V, T' = T + \Delta T$.

        При этом изменится внутренняя энергия:
        \begin{align*}
        \Delta U
            &= U' - U = \frac 32 \nu R T' - \frac 32 \nu R T = \frac 32 (P+\Delta P) (V+\Delta V) - \frac 32 PV\\
            &= \frac 32 ((P+\Delta P) (V+\Delta V) - PV) = \frac 32 (P\Delta V + V \Delta P + \Delta P \Delta V).
        \end{align*}

        Рассмотрим малые изменения объёма, тогда и изменение давления будем малым (т.к.
        $\Delta P = - \frac{P_0}{V_0} \Delta V$),
        а третьим слагаемым в выражении для $\Delta U$  можно пренебречь по сравнению с двумя другими:
        два первых это малые величины, а третье — произведение двух малых.
        Тогда $\Delta U = \frac 32 (P\Delta V + V \Delta P)$.

        Работа газа при этом малом изменении объёма — это площадь трапеции (тут ещё раз пренебрегли малым слагаемым):
        $$A = \frac{P + P'}2 \Delta V = \cbr{P + \frac{\Delta P}2} \Delta V = P \Delta V.$$

        Подведённое количество теплоты, используя первое начало термодинамики, будет равно
        \begin{align*}
        Q
            &= \frac 32 (P\Delta V + V \Delta P) + P \Delta V =  \frac 52 P\Delta V + \frac 32 V \Delta P = \\
            &= \frac 52 P\Delta V + \frac 32 V \cdot \cbr{- \frac{P_0}{V_0} \Delta V} = \frac{\Delta V}2 \cdot \cbr{5P - \frac{P_0}{V_0} V} = \\
            &= \frac{\Delta V}2 \cdot \cbr{5 \cdot \cbr{6P_0 - \frac{P_0}{V_0} V} - \frac{P_0}{V_0} V}
             = \frac{\Delta V \cdot P_0}2 \cdot \cbr{5 \cdot 6 - 8\frac V{V_0}}.
        \end{align*}

        Таком образом, знак количества теплоты $Q$ на участке 23 зависит от конкретного значения $V$:
        \begin{itemize}
            \item $\Delta V > 0$ на всём участке 23, поскольку газ расширяется,
            \item $P > 0$ — всегда, у нас идеальный газ, удары о стенки сосуда абсолютно упругие, а молекулы не взаимодействуют и поэтому давление только положительно,
            \item если $5 \cdot 6 - 8\frac V{V_0} > 0$ — тепло подводят, если же меньше нуля — отводят.
        \end{itemize}
        Решая последнее неравенство, получаем конкретное значение $V^*$: при $V < V^*$ тепло подводят, далее~— отводят.
        Тут *~--- некоторая точка между точками 2 и 3, конкретные значения надо досчитать:
        $$V^* = V_0 \cdot \frac{5 \cdot 6}8 = \frac{15}4 \cdot V_0 \implies P^* = 6P_0 - \frac{P_0}{V_0} V^* = \ldots = \frac94 \cdot P_0.$$

        Т.е.
        чтобы вычислить $Q_+$, надо сложить количества теплоты на участке 12 и лишь части участка 23 — участке 2*,
        той его части где это количество теплоты положительно.
        Имеем: $Q_+ = Q_{12} + Q_{2*}$.

        Теперь возвращаемся к циклу целиком и получаем:
        \begin{align*}
        A_\text{цикл} &= \frac 12 \cdot (5P_0 - P_0) \cdot (5V_0 - V_0) = 8 \cdot P_0V_0, \\
        A_{2*} &= \frac{P^* + 5P_0}2 \cdot (V^* - V_0)
            = \frac{\frac94 \cdot P_0 + 5P_0}2 \cdot \cbr{\frac{15}4 \cdot V_0 - V_0}
            = \ldots = \frac{319}{32} \cdot P_0 V_0, \\
        \Delta U_{2*} &= \frac 32 \nu R T^* - \frac 32 \nu R T_2 = \frac 32 (P^*V^* - P_0 \cdot 5V_0)
            = \frac 32 \cbr{\frac94 \cdot P_0 \cdot \frac{15}4 \cdot V_0 - P_0 \cdot 5V_0}
            = \frac{165}{32} \cdot P_0 V_0, \\
        \Delta U_{12} &= \frac 32 \nu R T_2 - \frac 32 \nu R T_1 = \frac 32 (5P_0V_0 - P_0V_0) = \ldots = 6 \cdot P_0 V_0, \\
        \eta &= \frac{A_\text{цикл}}{Q_+} = \frac{A_\text{цикл}}{Q_{12} + Q_{2*}}
            = \frac{A_\text{цикл}}{A_{12} + \Delta U_{12} + A_{2*} + \Delta U_{2*}} = \\
            &= \frac{8 \cdot P_0V_0}{0 + 6 \cdot P_0 V_0 + \frac{319}{32} \cdot P_0 V_0 + \frac{165}{32} \cdot P_0 V_0}
             = \frac{A_bonus_cycle:LaTeX}{6 + \frac{319}{32} + \frac{165}{32}}
             = \frac{64}{169} \leftarrow \text{вжух и готово!}
        \end{align*}
}
\solutionspace{360pt}

\tasknumber{2}%
\task{%
    При температуре $20\celsius$ относительная влажность воздуха составляет $55\%$.
    \begin{itemize}
        \item Определите точку росы для этого воздуха.
        \item Какой станет относительная влажность этого воздуха, если нагреть его до $70\celsius$?
    \end{itemize}
}
\answer{%
    \begin{align*}
    &\text{Значения плотности насыщенного водяного пара определяем по таблице:} \\
    &\rho_{\text{нас.
    пара 20} \celsius} = 17{,}300\,\frac{\text{г}}{\text{м}^{3}}, \rho_{\text{нас.
    пара 70} \celsius} = 198{,}000\,\frac{\text{г}}{\text{м}^{3}}.
    \\
    \varphi_1 &= \frac{\rho_\text{пара}}{\rho_{\text{нас.
    пара 20} \celsius}} \implies {\rho_\text{пара}} = \rho_{\text{нас.
    пара 20} \celsius} \cdot \varphi_1 = 17{,}300\,\frac{\text{г}}{\text{м}^{3}} \cdot 0{,}55 = 9{,}515\,\frac{\text{г}}{\text{м}^{3}}.
    \\
    &\text{По таблице определяем, при какой температуре пар с такой плотностью станет насыщенным:}  \\
    t_\text{росы} &= 10{,}2\celsius, \\
    \varphi_2 &= \frac{\rho_\text{пара}}{\rho_{\text{нас.
    пара 70} \celsius}} = \frac{\rho_{\text{нас.
    пара 20} \celsius} \cdot \varphi_1}{\rho_{\text{нас.
    пара 70} \celsius}}= \varphi_1 \cdot \frac{\rho_{\text{нас.
    пара 20} \celsius}}{\rho_{\text{нас.
    пара 70} \celsius}} = 0{,}55 \cdot \frac{17{,}300\,\frac{\text{г}}{\text{м}^{3}}}{198{,}000\,\frac{\text{г}}{\text{м}^{3}}} = 0{,}048 \approx 4{,}8\%.
    \end{align*}
}
\solutionspace{80pt}

\tasknumber{3}%
\task{%
    Из уравнения состояния идеального газа выведите или выразите...
    \begin{enumerate}
        \item давление,
        \item температуру,
        \item плотность газа.
    \end{enumerate}
}

\tasknumber{4}%
\task{%
    Запишите формулы и рядом с каждой физичической величиной укажите её название и единицы измерения в СИ:
    \begin{enumerate}
        \item первое начало термодинамики,
        \item внутренняя энергия идеального одноатомного газа.
    \end{enumerate}
}

\variantsplitter

\addpersonalvariant{Алёна Куприянова}

\tasknumber{1}%
\task{%
    Определите КПД (оставив ответ точным в виде нескоратимой дроби) цикла 1231, рабочим телом которого является идеальный одноатомный газ, если
    \begin{itemize}
        \item 12 — изохорический нагрев в четыре раза,
        \item 23 — изобарическое расширение, при котором температура растёт в пять раз,
        \item 31 — процесс, график которого в $PV$-координатах является отрезком прямой.
    \end{itemize}
    Бонус: замените цикл 1231 циклом, в котором 12 — изохорический нагрев в четыре раза, 23 — процесс, график которого в $PV$-координатах является отрезком прямой, 31 — изобарическое охлаждение, при котором температура падает в четыре раза.
}
\answer{%
    \begin{align*}
    A_{12} &= 0, \Delta U_{12} > 0, \implies Q_{12} = A_{12} + \Delta U_{12} > 0.
    \\
    A_{23} &> 0, \Delta U_{23} > 0, \implies Q_{23} = A_{23} + \Delta U_{23} > 0, \\
    A_{31} &= 0, \Delta U_{31} < 0, \implies Q_{31} = A_{31} + \Delta U_{31} < 0.
    \\
    P_1V_1 &= \nu R T_1, P_2V_2 = \nu R T_2, P_3V_3 = \nu R T_3 \text{ — уравнения состояния идеального газа}, \\
    &\text{Пусть $P_0$, $V_0$, $T_0$ — давление, объём и температура в точке 1 (минимальные во всём цикле):} \\
    P_1 &= P_0, P_2 = P_3, V_1 = V_2 = V_0, \text{остальные соотношения нужно считать} \\
    T_2 &= 4T_1 = 4T_0 \text{(по условию)} \implies \frac{P_2}{P_1} = \frac{P_2V_0}{P_1V_0} = \frac{P_2 V_2}{P_1 V_1}= \frac{\nu R T_2}{\nu R T_1} = \frac{T_2}{T_1} = 4 \implies P_2 = 4 P_1 = 4 P_0, \\
    T_3 &= 5T_2 = 20T_0 \text{(по условию)} \implies \frac{V_3}{V_2} = \frac{P_3V_3}{P_2V_2}= \frac{\nu R T_3}{\nu R T_2} = \frac{T_3}{T_2} = 5 \implies V_3 = 5 V_2 = 5 V_0.
    \\
    A_\text{цикл} &= \frac 12 (5P_0 - P_0)(4V_0 - V_0) = \frac 12 \cdot 12 \cdot P_0V_0, \\
    A_{23} &= 4P_0 \cdot (5V_0 - V_0) = 16P_0V_0, \\
    \Delta U_{23} &= \frac 32 \nu R T_3 - \frac 32 \nu R T_2 = \frac 32 P_3 V_3 - \frac 32 P_2 V_2 = \frac 32 \cdot 4 P_0 \cdot 5 V_0 -  \frac 32 \cdot 4 P_0 \cdot V_0 = \frac 32 \cdot 16 \cdot P_0V_0, \\
    \Delta U_{12} &= \frac 32 \nu R T_2 - \frac 32 \nu R T_1 = \frac 32 P_2 V_2 - \frac 32 P_1 V_1 = \frac 32 \cdot 4 P_0V_0 - \frac 32 P_0V_0 = \frac 32 \cdot 3 \cdot P_0V_0.
    \\
    \eta &= \frac{A_\text{цикл}}{Q_+} = \frac{A_\text{цикл}}{Q_{12} + Q_{23}}  = \frac{A_\text{цикл}}{A_{12} + \Delta U_{12} + A_{23} + \Delta U_{23}} =  \\
     &= \frac{\frac 12 \cdot 12 \cdot P_0V_0}{0 + \frac 32 \cdot 3 \cdot P_0V_0 + 16P_0V_0 + \frac 32 \cdot 16 \cdot P_0V_0} = \frac{\frac 12 \cdot 12}{\frac 32 \cdot 3 + 16 + \frac 32 \cdot 16} = \frac{12}{89} \approx 0.135.
    \end{align*}


        График процесса не в масштабе (эта часть пока не готова и сделать автоматически аккуратно сложно), но с верными подписями (а для решения этого достаточно):

        \begin{tikzpicture}[thick]
            \draw[-{Latex}] (0, 0) -- (0, 7) node[above left] {$P$};
            \draw[-{Latex}] (0, 0) -- (10, 0) node[right] {$V$};

            \draw[dashed] (0, 2) node[left] {$P_1 = P_0$} -| (3, 0) node[below] {$V_1 = V_2 = V_0$};
            \draw[dashed] (0, 6) node[left] {$P_2 = P_3 = 4P_0$} -| (9, 0) node[below] {$V_3 = 5V_0$};

            \draw (3, 2) node[above left]{1} node[below left]{$T_1 = T_0$}
                   (3, 6) node[below left]{2} node[above]{$T_2 = 4T_0$}
                   (9, 6) node[above right]{3} node[below right]{$T_3 = 20T_0$};
            \draw[midar] (3, 2) -- (3, 6);
            \draw[midar] (3, 6) -- (9, 6);
            \draw[midar] (9, 6) -- (3, 2);
        \end{tikzpicture}

        Решение бонуса:
        \begin{align*}
            A_{12} &= 0, \Delta U_{12} > 0, \implies Q_{12} = A_{12} + \Delta U_{12} > 0, \\
            A_{23} &> 0, \Delta U_{23} \text{ — ничего нельзя сказать, нужно исследовать отдельно}, \\
            A_{31} &< 0, \Delta U_{31} < 0, \implies Q_{31} = A_{31} + \Delta U_{31} < 0.
            \\
        \end{align*}

        Уравнения состояния идеального газа для точек 1, 2, 3: $P_1V_1 = \nu R T_1, P_2V_2 = \nu R T_2, P_3V_3 = \nu R T_3$.
        Пусть $P_0$, $V_0$, $T_0$ — давление, объём и температура в точке 1 (минимальные во всём цикле).

        12 --- изохора, $\frac{P_1V_1}{T_1} = \nu R = \frac{P_2V_2}{T_2}, V_2=V_1=V_0 \implies \frac{P_1}{T_1} =  \frac{P_2}{T_2} \implies P_2 = P_1 \frac{T_2}{T_1} = 4P_0$,

        31 --- изобара, $\frac{P_1V_1}{T_1} = \nu R = \frac{P_3V_3}{T_3}, P_3=P_1=P_0 \implies \frac{V_3}{T_3} =  \frac{V_1}{T_1} \implies V_3 = V_1 \frac{T_3}{T_1} = 4V_0$,

        Таким образом, используя новые обозначения, состояния газа в точках 1, 2 и 3 описываются макропараметрами $(P_0, V_0, T_0), (4P_0, V_0, 4T_0), (P_0, 4V_0, 4T_0)$ соответственно.

        \begin{tikzpicture}[thick]
            \draw[-{Latex}] (0, 0) -- (0, 7) node[above left] {$P$};
            \draw[-{Latex}] (0, 0) -- (10, 0) node[right] {$V$};

            \draw[dashed] (0, 2) node[left] {$P_1 = P_3 = P_0$} -| (9, 0) node[below] {$V_3 = 4V_0$};
            \draw[dashed] (0, 6) node[left] {$P_2 = 4P_0$} -| (3, 0) node[below] {$V_1 = V_2 = V_0$};

            \draw[dashed] (0, 5) node[left] {$P$} -| (4.5, 0) node[below] {$V$};
            \draw[dashed] (0, 4.6) node[left] {$P'$} -| (5.1, 0) node[below] {$V'$};

            \draw (3, 2) node[above left]{1} node[below left]{$T_1 = T_0$}
                   (3, 6) node[below left]{2} node[above]{$T_2 = 4T_0$}
                   (9, 2) node[above right]{3} node[below right]{$T_3 = 4T_0$};
            \draw[midar] (3, 2) -- (3, 6);
            \draw[midar] (3, 6) -- (9, 2);
            \draw[midar] (9, 2) -- (3, 2);
            \draw   (4.5, 5) node[above right]{$T$} (5.1, 4.6) node[above right]{$T'$};
        \end{tikzpicture}


        Теперь рассмотрим отдельно процесс 23, к остальному вернёмся позже.
        Уравнение этой прямой в $PV$-координатах: $P(V) = 5P_0 - \frac{P_0}{V_0} V$.
        Это значит, что при изменении объёма на $\Delta V$ давление изменится на $\Delta P = - \frac{P_0}{V_0} \Delta V$, обратите внимание на знак.

        Рассмотрим произвольную точку в процессе 23 и дадим процессу ещё немного свершиться, при этом объём изменится на $\Delta V$, давление — на $\Delta P$, температура (иначе бы была гипербола, а не прямая) — на $\Delta T$,
        т.е.
        из состояния $(P, V, T)$ мы перешли в $(P', V', T')$, причём  $P' = P + \Delta P, V' = V + \Delta V, T' = T + \Delta T$.

        При этом изменится внутренняя энергия:
        \begin{align*}
        \Delta U
            &= U' - U = \frac 32 \nu R T' - \frac 32 \nu R T = \frac 32 (P+\Delta P) (V+\Delta V) - \frac 32 PV\\
            &= \frac 32 ((P+\Delta P) (V+\Delta V) - PV) = \frac 32 (P\Delta V + V \Delta P + \Delta P \Delta V).
        \end{align*}

        Рассмотрим малые изменения объёма, тогда и изменение давления будем малым (т.к.
        $\Delta P = - \frac{P_0}{V_0} \Delta V$),
        а третьим слагаемым в выражении для $\Delta U$  можно пренебречь по сравнению с двумя другими:
        два первых это малые величины, а третье — произведение двух малых.
        Тогда $\Delta U = \frac 32 (P\Delta V + V \Delta P)$.

        Работа газа при этом малом изменении объёма — это площадь трапеции (тут ещё раз пренебрегли малым слагаемым):
        $$A = \frac{P + P'}2 \Delta V = \cbr{P + \frac{\Delta P}2} \Delta V = P \Delta V.$$

        Подведённое количество теплоты, используя первое начало термодинамики, будет равно
        \begin{align*}
        Q
            &= \frac 32 (P\Delta V + V \Delta P) + P \Delta V =  \frac 52 P\Delta V + \frac 32 V \Delta P = \\
            &= \frac 52 P\Delta V + \frac 32 V \cdot \cbr{- \frac{P_0}{V_0} \Delta V} = \frac{\Delta V}2 \cdot \cbr{5P - \frac{P_0}{V_0} V} = \\
            &= \frac{\Delta V}2 \cdot \cbr{5 \cdot \cbr{5P_0 - \frac{P_0}{V_0} V} - \frac{P_0}{V_0} V}
             = \frac{\Delta V \cdot P_0}2 \cdot \cbr{5 \cdot 5 - 8\frac V{V_0}}.
        \end{align*}

        Таком образом, знак количества теплоты $Q$ на участке 23 зависит от конкретного значения $V$:
        \begin{itemize}
            \item $\Delta V > 0$ на всём участке 23, поскольку газ расширяется,
            \item $P > 0$ — всегда, у нас идеальный газ, удары о стенки сосуда абсолютно упругие, а молекулы не взаимодействуют и поэтому давление только положительно,
            \item если $5 \cdot 5 - 8\frac V{V_0} > 0$ — тепло подводят, если же меньше нуля — отводят.
        \end{itemize}
        Решая последнее неравенство, получаем конкретное значение $V^*$: при $V < V^*$ тепло подводят, далее~— отводят.
        Тут *~--- некоторая точка между точками 2 и 3, конкретные значения надо досчитать:
        $$V^* = V_0 \cdot \frac{5 \cdot 5}8 = \frac{25}8 \cdot V_0 \implies P^* = 5P_0 - \frac{P_0}{V_0} V^* = \ldots = \frac{15}8 \cdot P_0.$$

        Т.е.
        чтобы вычислить $Q_+$, надо сложить количества теплоты на участке 12 и лишь части участка 23 — участке 2*,
        той его части где это количество теплоты положительно.
        Имеем: $Q_+ = Q_{12} + Q_{2*}$.

        Теперь возвращаемся к циклу целиком и получаем:
        \begin{align*}
        A_\text{цикл} &= \frac 12 \cdot (4P_0 - P_0) \cdot (4V_0 - V_0) = \frac92 \cdot P_0V_0, \\
        A_{2*} &= \frac{P^* + 4P_0}2 \cdot (V^* - V_0)
            = \frac{\frac{15}8 \cdot P_0 + 4P_0}2 \cdot \cbr{\frac{25}8 \cdot V_0 - V_0}
            = \ldots = \frac{799}{128} \cdot P_0 V_0, \\
        \Delta U_{2*} &= \frac 32 \nu R T^* - \frac 32 \nu R T_2 = \frac 32 (P^*V^* - P_0 \cdot 4V_0)
            = \frac 32 \cbr{\frac{15}8 \cdot P_0 \cdot \frac{25}8 \cdot V_0 - P_0 \cdot 4V_0}
            = \frac{357}{128} \cdot P_0 V_0, \\
        \Delta U_{12} &= \frac 32 \nu R T_2 - \frac 32 \nu R T_1 = \frac 32 (4P_0V_0 - P_0V_0) = \ldots = \frac92 \cdot P_0 V_0, \\
        \eta &= \frac{A_\text{цикл}}{Q_+} = \frac{A_\text{цикл}}{Q_{12} + Q_{2*}}
            = \frac{A_\text{цикл}}{A_{12} + \Delta U_{12} + A_{2*} + \Delta U_{2*}} = \\
            &= \frac{\frac92 \cdot P_0V_0}{0 + \frac92 \cdot P_0 V_0 + \frac{799}{128} \cdot P_0 V_0 + \frac{357}{128} \cdot P_0 V_0}
             = \frac{A_bonus_cycle:LaTeX}{\frac92 + \frac{799}{128} + \frac{357}{128}}
             = \frac{144}{433} \leftarrow \text{вжух и готово!}
        \end{align*}
}
\solutionspace{360pt}

\tasknumber{2}%
\task{%
    При температуре $15\celsius$ относительная влажность воздуха составляет $50\%$.
    \begin{itemize}
        \item Определите точку росы для этого воздуха.
        \item Какой станет относительная влажность этого воздуха, если нагреть его до $70\celsius$?
    \end{itemize}
}
\answer{%
    \begin{align*}
    &\text{Значения плотности насыщенного водяного пара определяем по таблице:} \\
    &\rho_{\text{нас.
    пара 15} \celsius} = 12{,}800\,\frac{\text{г}}{\text{м}^{3}}, \rho_{\text{нас.
    пара 70} \celsius} = 198{,}000\,\frac{\text{г}}{\text{м}^{3}}.
    \\
    \varphi_1 &= \frac{\rho_\text{пара}}{\rho_{\text{нас.
    пара 15} \celsius}} \implies {\rho_\text{пара}} = \rho_{\text{нас.
    пара 15} \celsius} \cdot \varphi_1 = 12{,}800\,\frac{\text{г}}{\text{м}^{3}} \cdot 0{,}50 = 6{,}400\,\frac{\text{г}}{\text{м}^{3}}.
    \\
    &\text{По таблице определяем, при какой температуре пар с такой плотностью станет насыщенным:}  \\
    t_\text{росы} &= 4{,}0\celsius, \\
    \varphi_2 &= \frac{\rho_\text{пара}}{\rho_{\text{нас.
    пара 70} \celsius}} = \frac{\rho_{\text{нас.
    пара 15} \celsius} \cdot \varphi_1}{\rho_{\text{нас.
    пара 70} \celsius}}= \varphi_1 \cdot \frac{\rho_{\text{нас.
    пара 15} \celsius}}{\rho_{\text{нас.
    пара 70} \celsius}} = 0{,}50 \cdot \frac{12{,}800\,\frac{\text{г}}{\text{м}^{3}}}{198{,}000\,\frac{\text{г}}{\text{м}^{3}}} = 0{,}032 \approx 3{,}2\%.
    \end{align*}
}
\solutionspace{80pt}

\tasknumber{3}%
\task{%
    Из уравнения состояния идеального газа выведите или выразите...
    \begin{enumerate}
        \item объём,
        \item молярную массу,
        \item концентрацию молекул газа.
    \end{enumerate}
}

\tasknumber{4}%
\task{%
    Запишите формулы и рядом с каждой физичической величиной укажите её название и единицы измерения в СИ:
    \begin{enumerate}
        \item первое начало термодинамики,
        \item внутренняя энергия идеального одноатомного газа.
    \end{enumerate}
}

\variantsplitter

\addpersonalvariant{Ярослав Лавровский}

\tasknumber{1}%
\task{%
    Определите КПД (оставив ответ точным в виде нескоратимой дроби) цикла 1231, рабочим телом которого является идеальный одноатомный газ, если
    \begin{itemize}
        \item 12 — изохорический нагрев в три раза,
        \item 23 — изобарическое расширение, при котором температура растёт в четыре раза,
        \item 31 — процесс, график которого в $PV$-координатах является отрезком прямой.
    \end{itemize}
    Бонус: замените цикл 1231 циклом, в котором 12 — изохорический нагрев в три раза, 23 — процесс, график которого в $PV$-координатах является отрезком прямой, 31 — изобарическое охлаждение, при котором температура падает в три раза.
}
\answer{%
    \begin{align*}
    A_{12} &= 0, \Delta U_{12} > 0, \implies Q_{12} = A_{12} + \Delta U_{12} > 0.
    \\
    A_{23} &> 0, \Delta U_{23} > 0, \implies Q_{23} = A_{23} + \Delta U_{23} > 0, \\
    A_{31} &= 0, \Delta U_{31} < 0, \implies Q_{31} = A_{31} + \Delta U_{31} < 0.
    \\
    P_1V_1 &= \nu R T_1, P_2V_2 = \nu R T_2, P_3V_3 = \nu R T_3 \text{ — уравнения состояния идеального газа}, \\
    &\text{Пусть $P_0$, $V_0$, $T_0$ — давление, объём и температура в точке 1 (минимальные во всём цикле):} \\
    P_1 &= P_0, P_2 = P_3, V_1 = V_2 = V_0, \text{остальные соотношения нужно считать} \\
    T_2 &= 3T_1 = 3T_0 \text{(по условию)} \implies \frac{P_2}{P_1} = \frac{P_2V_0}{P_1V_0} = \frac{P_2 V_2}{P_1 V_1}= \frac{\nu R T_2}{\nu R T_1} = \frac{T_2}{T_1} = 3 \implies P_2 = 3 P_1 = 3 P_0, \\
    T_3 &= 4T_2 = 12T_0 \text{(по условию)} \implies \frac{V_3}{V_2} = \frac{P_3V_3}{P_2V_2}= \frac{\nu R T_3}{\nu R T_2} = \frac{T_3}{T_2} = 4 \implies V_3 = 4 V_2 = 4 V_0.
    \\
    A_\text{цикл} &= \frac 12 (4P_0 - P_0)(3V_0 - V_0) = \frac 12 \cdot 6 \cdot P_0V_0, \\
    A_{23} &= 3P_0 \cdot (4V_0 - V_0) = 9P_0V_0, \\
    \Delta U_{23} &= \frac 32 \nu R T_3 - \frac 32 \nu R T_2 = \frac 32 P_3 V_3 - \frac 32 P_2 V_2 = \frac 32 \cdot 3 P_0 \cdot 4 V_0 -  \frac 32 \cdot 3 P_0 \cdot V_0 = \frac 32 \cdot 9 \cdot P_0V_0, \\
    \Delta U_{12} &= \frac 32 \nu R T_2 - \frac 32 \nu R T_1 = \frac 32 P_2 V_2 - \frac 32 P_1 V_1 = \frac 32 \cdot 3 P_0V_0 - \frac 32 P_0V_0 = \frac 32 \cdot 2 \cdot P_0V_0.
    \\
    \eta &= \frac{A_\text{цикл}}{Q_+} = \frac{A_\text{цикл}}{Q_{12} + Q_{23}}  = \frac{A_\text{цикл}}{A_{12} + \Delta U_{12} + A_{23} + \Delta U_{23}} =  \\
     &= \frac{\frac 12 \cdot 6 \cdot P_0V_0}{0 + \frac 32 \cdot 2 \cdot P_0V_0 + 9P_0V_0 + \frac 32 \cdot 9 \cdot P_0V_0} = \frac{\frac 12 \cdot 6}{\frac 32 \cdot 2 + 9 + \frac 32 \cdot 9} = \frac2{17} \approx 0.118.
    \end{align*}


        График процесса не в масштабе (эта часть пока не готова и сделать автоматически аккуратно сложно), но с верными подписями (а для решения этого достаточно):

        \begin{tikzpicture}[thick]
            \draw[-{Latex}] (0, 0) -- (0, 7) node[above left] {$P$};
            \draw[-{Latex}] (0, 0) -- (10, 0) node[right] {$V$};

            \draw[dashed] (0, 2) node[left] {$P_1 = P_0$} -| (3, 0) node[below] {$V_1 = V_2 = V_0$};
            \draw[dashed] (0, 6) node[left] {$P_2 = P_3 = 3P_0$} -| (9, 0) node[below] {$V_3 = 4V_0$};

            \draw (3, 2) node[above left]{1} node[below left]{$T_1 = T_0$}
                   (3, 6) node[below left]{2} node[above]{$T_2 = 3T_0$}
                   (9, 6) node[above right]{3} node[below right]{$T_3 = 12T_0$};
            \draw[midar] (3, 2) -- (3, 6);
            \draw[midar] (3, 6) -- (9, 6);
            \draw[midar] (9, 6) -- (3, 2);
        \end{tikzpicture}

        Решение бонуса:
        \begin{align*}
            A_{12} &= 0, \Delta U_{12} > 0, \implies Q_{12} = A_{12} + \Delta U_{12} > 0, \\
            A_{23} &> 0, \Delta U_{23} \text{ — ничего нельзя сказать, нужно исследовать отдельно}, \\
            A_{31} &< 0, \Delta U_{31} < 0, \implies Q_{31} = A_{31} + \Delta U_{31} < 0.
            \\
        \end{align*}

        Уравнения состояния идеального газа для точек 1, 2, 3: $P_1V_1 = \nu R T_1, P_2V_2 = \nu R T_2, P_3V_3 = \nu R T_3$.
        Пусть $P_0$, $V_0$, $T_0$ — давление, объём и температура в точке 1 (минимальные во всём цикле).

        12 --- изохора, $\frac{P_1V_1}{T_1} = \nu R = \frac{P_2V_2}{T_2}, V_2=V_1=V_0 \implies \frac{P_1}{T_1} =  \frac{P_2}{T_2} \implies P_2 = P_1 \frac{T_2}{T_1} = 3P_0$,

        31 --- изобара, $\frac{P_1V_1}{T_1} = \nu R = \frac{P_3V_3}{T_3}, P_3=P_1=P_0 \implies \frac{V_3}{T_3} =  \frac{V_1}{T_1} \implies V_3 = V_1 \frac{T_3}{T_1} = 3V_0$,

        Таким образом, используя новые обозначения, состояния газа в точках 1, 2 и 3 описываются макропараметрами $(P_0, V_0, T_0), (3P_0, V_0, 3T_0), (P_0, 3V_0, 3T_0)$ соответственно.

        \begin{tikzpicture}[thick]
            \draw[-{Latex}] (0, 0) -- (0, 7) node[above left] {$P$};
            \draw[-{Latex}] (0, 0) -- (10, 0) node[right] {$V$};

            \draw[dashed] (0, 2) node[left] {$P_1 = P_3 = P_0$} -| (9, 0) node[below] {$V_3 = 3V_0$};
            \draw[dashed] (0, 6) node[left] {$P_2 = 3P_0$} -| (3, 0) node[below] {$V_1 = V_2 = V_0$};

            \draw[dashed] (0, 5) node[left] {$P$} -| (4.5, 0) node[below] {$V$};
            \draw[dashed] (0, 4.6) node[left] {$P'$} -| (5.1, 0) node[below] {$V'$};

            \draw (3, 2) node[above left]{1} node[below left]{$T_1 = T_0$}
                   (3, 6) node[below left]{2} node[above]{$T_2 = 3T_0$}
                   (9, 2) node[above right]{3} node[below right]{$T_3 = 3T_0$};
            \draw[midar] (3, 2) -- (3, 6);
            \draw[midar] (3, 6) -- (9, 2);
            \draw[midar] (9, 2) -- (3, 2);
            \draw   (4.5, 5) node[above right]{$T$} (5.1, 4.6) node[above right]{$T'$};
        \end{tikzpicture}


        Теперь рассмотрим отдельно процесс 23, к остальному вернёмся позже.
        Уравнение этой прямой в $PV$-координатах: $P(V) = 4P_0 - \frac{P_0}{V_0} V$.
        Это значит, что при изменении объёма на $\Delta V$ давление изменится на $\Delta P = - \frac{P_0}{V_0} \Delta V$, обратите внимание на знак.

        Рассмотрим произвольную точку в процессе 23 и дадим процессу ещё немного свершиться, при этом объём изменится на $\Delta V$, давление — на $\Delta P$, температура (иначе бы была гипербола, а не прямая) — на $\Delta T$,
        т.е.
        из состояния $(P, V, T)$ мы перешли в $(P', V', T')$, причём  $P' = P + \Delta P, V' = V + \Delta V, T' = T + \Delta T$.

        При этом изменится внутренняя энергия:
        \begin{align*}
        \Delta U
            &= U' - U = \frac 32 \nu R T' - \frac 32 \nu R T = \frac 32 (P+\Delta P) (V+\Delta V) - \frac 32 PV\\
            &= \frac 32 ((P+\Delta P) (V+\Delta V) - PV) = \frac 32 (P\Delta V + V \Delta P + \Delta P \Delta V).
        \end{align*}

        Рассмотрим малые изменения объёма, тогда и изменение давления будем малым (т.к.
        $\Delta P = - \frac{P_0}{V_0} \Delta V$),
        а третьим слагаемым в выражении для $\Delta U$  можно пренебречь по сравнению с двумя другими:
        два первых это малые величины, а третье — произведение двух малых.
        Тогда $\Delta U = \frac 32 (P\Delta V + V \Delta P)$.

        Работа газа при этом малом изменении объёма — это площадь трапеции (тут ещё раз пренебрегли малым слагаемым):
        $$A = \frac{P + P'}2 \Delta V = \cbr{P + \frac{\Delta P}2} \Delta V = P \Delta V.$$

        Подведённое количество теплоты, используя первое начало термодинамики, будет равно
        \begin{align*}
        Q
            &= \frac 32 (P\Delta V + V \Delta P) + P \Delta V =  \frac 52 P\Delta V + \frac 32 V \Delta P = \\
            &= \frac 52 P\Delta V + \frac 32 V \cdot \cbr{- \frac{P_0}{V_0} \Delta V} = \frac{\Delta V}2 \cdot \cbr{5P - \frac{P_0}{V_0} V} = \\
            &= \frac{\Delta V}2 \cdot \cbr{5 \cdot \cbr{4P_0 - \frac{P_0}{V_0} V} - \frac{P_0}{V_0} V}
             = \frac{\Delta V \cdot P_0}2 \cdot \cbr{5 \cdot 4 - 8\frac V{V_0}}.
        \end{align*}

        Таком образом, знак количества теплоты $Q$ на участке 23 зависит от конкретного значения $V$:
        \begin{itemize}
            \item $\Delta V > 0$ на всём участке 23, поскольку газ расширяется,
            \item $P > 0$ — всегда, у нас идеальный газ, удары о стенки сосуда абсолютно упругие, а молекулы не взаимодействуют и поэтому давление только положительно,
            \item если $5 \cdot 4 - 8\frac V{V_0} > 0$ — тепло подводят, если же меньше нуля — отводят.
        \end{itemize}
        Решая последнее неравенство, получаем конкретное значение $V^*$: при $V < V^*$ тепло подводят, далее~— отводят.
        Тут *~--- некоторая точка между точками 2 и 3, конкретные значения надо досчитать:
        $$V^* = V_0 \cdot \frac{5 \cdot 4}8 = \frac52 \cdot V_0 \implies P^* = 4P_0 - \frac{P_0}{V_0} V^* = \ldots = \frac32 \cdot P_0.$$

        Т.е.
        чтобы вычислить $Q_+$, надо сложить количества теплоты на участке 12 и лишь части участка 23 — участке 2*,
        той его части где это количество теплоты положительно.
        Имеем: $Q_+ = Q_{12} + Q_{2*}$.

        Теперь возвращаемся к циклу целиком и получаем:
        \begin{align*}
        A_\text{цикл} &= \frac 12 \cdot (3P_0 - P_0) \cdot (3V_0 - V_0) = 2 \cdot P_0V_0, \\
        A_{2*} &= \frac{P^* + 3P_0}2 \cdot (V^* - V_0)
            = \frac{\frac32 \cdot P_0 + 3P_0}2 \cdot \cbr{\frac52 \cdot V_0 - V_0}
            = \ldots = \frac{27}8 \cdot P_0 V_0, \\
        \Delta U_{2*} &= \frac 32 \nu R T^* - \frac 32 \nu R T_2 = \frac 32 (P^*V^* - P_0 \cdot 3V_0)
            = \frac 32 \cbr{\frac32 \cdot P_0 \cdot \frac52 \cdot V_0 - P_0 \cdot 3V_0}
            = \frac98 \cdot P_0 V_0, \\
        \Delta U_{12} &= \frac 32 \nu R T_2 - \frac 32 \nu R T_1 = \frac 32 (3P_0V_0 - P_0V_0) = \ldots = 3 \cdot P_0 V_0, \\
        \eta &= \frac{A_\text{цикл}}{Q_+} = \frac{A_\text{цикл}}{Q_{12} + Q_{2*}}
            = \frac{A_\text{цикл}}{A_{12} + \Delta U_{12} + A_{2*} + \Delta U_{2*}} = \\
            &= \frac{2 \cdot P_0V_0}{0 + 3 \cdot P_0 V_0 + \frac{27}8 \cdot P_0 V_0 + \frac98 \cdot P_0 V_0}
             = \frac{A_bonus_cycle:LaTeX}{3 + \frac{27}8 + \frac98}
             = \frac4{15} \leftarrow \text{вжух и готово!}
        \end{align*}
}
\solutionspace{360pt}

\tasknumber{2}%
\task{%
    При температуре $30\celsius$ относительная влажность воздуха составляет $65\%$.
    \begin{itemize}
        \item Определите точку росы для этого воздуха.
        \item Какой станет относительная влажность этого воздуха, если нагреть его до $50\celsius$?
    \end{itemize}
}
\answer{%
    \begin{align*}
    &\text{Значения плотности насыщенного водяного пара определяем по таблице:} \\
    &\rho_{\text{нас.
    пара 30} \celsius} = 30{,}300\,\frac{\text{г}}{\text{м}^{3}}, \rho_{\text{нас.
    пара 50} \celsius} = 83{,}000\,\frac{\text{г}}{\text{м}^{3}}.
    \\
    \varphi_1 &= \frac{\rho_\text{пара}}{\rho_{\text{нас.
    пара 30} \celsius}} \implies {\rho_\text{пара}} = \rho_{\text{нас.
    пара 30} \celsius} \cdot \varphi_1 = 30{,}300\,\frac{\text{г}}{\text{м}^{3}} \cdot 0{,}65 = 19{,}695\,\frac{\text{г}}{\text{м}^{3}}.
    \\
    &\text{По таблице определяем, при какой температуре пар с такой плотностью станет насыщенным:}  \\
    t_\text{росы} &= 22{,}2\celsius, \\
    \varphi_2 &= \frac{\rho_\text{пара}}{\rho_{\text{нас.
    пара 50} \celsius}} = \frac{\rho_{\text{нас.
    пара 30} \celsius} \cdot \varphi_1}{\rho_{\text{нас.
    пара 50} \celsius}}= \varphi_1 \cdot \frac{\rho_{\text{нас.
    пара 30} \celsius}}{\rho_{\text{нас.
    пара 50} \celsius}} = 0{,}65 \cdot \frac{30{,}300\,\frac{\text{г}}{\text{м}^{3}}}{83{,}000\,\frac{\text{г}}{\text{м}^{3}}} = 0{,}237 \approx 23{,}7\%.
    \end{align*}
}
\solutionspace{80pt}

\tasknumber{3}%
\task{%
    Из уравнения состояния идеального газа выведите или выразите...
    \begin{enumerate}
        \item давление,
        \item температуру,
        \item концентрацию молекул газа.
    \end{enumerate}
}

\tasknumber{4}%
\task{%
    Запишите формулы и рядом с каждой физичической величиной укажите её название и единицы измерения в СИ:
    \begin{enumerate}
        \item первое начало термодинамики,
        \item внутренняя энергия идеального одноатомного газа.
    \end{enumerate}
}

\variantsplitter

\addpersonalvariant{Анастасия Ламанова}

\tasknumber{1}%
\task{%
    Определите КПД (оставив ответ точным в виде нескоратимой дроби) цикла 1231, рабочим телом которого является идеальный одноатомный газ, если
    \begin{itemize}
        \item 12 — изохорический нагрев в четыре раза,
        \item 23 — изобарическое расширение, при котором температура растёт в четыре раза,
        \item 31 — процесс, график которого в $PV$-координатах является отрезком прямой.
    \end{itemize}
    Бонус: замените цикл 1231 циклом, в котором 12 — изохорический нагрев в четыре раза, 23 — процесс, график которого в $PV$-координатах является отрезком прямой, 31 — изобарическое охлаждение, при котором температура падает в четыре раза.
}
\answer{%
    \begin{align*}
    A_{12} &= 0, \Delta U_{12} > 0, \implies Q_{12} = A_{12} + \Delta U_{12} > 0.
    \\
    A_{23} &> 0, \Delta U_{23} > 0, \implies Q_{23} = A_{23} + \Delta U_{23} > 0, \\
    A_{31} &= 0, \Delta U_{31} < 0, \implies Q_{31} = A_{31} + \Delta U_{31} < 0.
    \\
    P_1V_1 &= \nu R T_1, P_2V_2 = \nu R T_2, P_3V_3 = \nu R T_3 \text{ — уравнения состояния идеального газа}, \\
    &\text{Пусть $P_0$, $V_0$, $T_0$ — давление, объём и температура в точке 1 (минимальные во всём цикле):} \\
    P_1 &= P_0, P_2 = P_3, V_1 = V_2 = V_0, \text{остальные соотношения нужно считать} \\
    T_2 &= 4T_1 = 4T_0 \text{(по условию)} \implies \frac{P_2}{P_1} = \frac{P_2V_0}{P_1V_0} = \frac{P_2 V_2}{P_1 V_1}= \frac{\nu R T_2}{\nu R T_1} = \frac{T_2}{T_1} = 4 \implies P_2 = 4 P_1 = 4 P_0, \\
    T_3 &= 4T_2 = 16T_0 \text{(по условию)} \implies \frac{V_3}{V_2} = \frac{P_3V_3}{P_2V_2}= \frac{\nu R T_3}{\nu R T_2} = \frac{T_3}{T_2} = 4 \implies V_3 = 4 V_2 = 4 V_0.
    \\
    A_\text{цикл} &= \frac 12 (4P_0 - P_0)(4V_0 - V_0) = \frac 12 \cdot 9 \cdot P_0V_0, \\
    A_{23} &= 4P_0 \cdot (4V_0 - V_0) = 12P_0V_0, \\
    \Delta U_{23} &= \frac 32 \nu R T_3 - \frac 32 \nu R T_2 = \frac 32 P_3 V_3 - \frac 32 P_2 V_2 = \frac 32 \cdot 4 P_0 \cdot 4 V_0 -  \frac 32 \cdot 4 P_0 \cdot V_0 = \frac 32 \cdot 12 \cdot P_0V_0, \\
    \Delta U_{12} &= \frac 32 \nu R T_2 - \frac 32 \nu R T_1 = \frac 32 P_2 V_2 - \frac 32 P_1 V_1 = \frac 32 \cdot 4 P_0V_0 - \frac 32 P_0V_0 = \frac 32 \cdot 3 \cdot P_0V_0.
    \\
    \eta &= \frac{A_\text{цикл}}{Q_+} = \frac{A_\text{цикл}}{Q_{12} + Q_{23}}  = \frac{A_\text{цикл}}{A_{12} + \Delta U_{12} + A_{23} + \Delta U_{23}} =  \\
     &= \frac{\frac 12 \cdot 9 \cdot P_0V_0}{0 + \frac 32 \cdot 3 \cdot P_0V_0 + 12P_0V_0 + \frac 32 \cdot 12 \cdot P_0V_0} = \frac{\frac 12 \cdot 9}{\frac 32 \cdot 3 + 12 + \frac 32 \cdot 12} = \frac3{23} \approx 0.130.
    \end{align*}


        График процесса не в масштабе (эта часть пока не готова и сделать автоматически аккуратно сложно), но с верными подписями (а для решения этого достаточно):

        \begin{tikzpicture}[thick]
            \draw[-{Latex}] (0, 0) -- (0, 7) node[above left] {$P$};
            \draw[-{Latex}] (0, 0) -- (10, 0) node[right] {$V$};

            \draw[dashed] (0, 2) node[left] {$P_1 = P_0$} -| (3, 0) node[below] {$V_1 = V_2 = V_0$};
            \draw[dashed] (0, 6) node[left] {$P_2 = P_3 = 4P_0$} -| (9, 0) node[below] {$V_3 = 4V_0$};

            \draw (3, 2) node[above left]{1} node[below left]{$T_1 = T_0$}
                   (3, 6) node[below left]{2} node[above]{$T_2 = 4T_0$}
                   (9, 6) node[above right]{3} node[below right]{$T_3 = 16T_0$};
            \draw[midar] (3, 2) -- (3, 6);
            \draw[midar] (3, 6) -- (9, 6);
            \draw[midar] (9, 6) -- (3, 2);
        \end{tikzpicture}

        Решение бонуса:
        \begin{align*}
            A_{12} &= 0, \Delta U_{12} > 0, \implies Q_{12} = A_{12} + \Delta U_{12} > 0, \\
            A_{23} &> 0, \Delta U_{23} \text{ — ничего нельзя сказать, нужно исследовать отдельно}, \\
            A_{31} &< 0, \Delta U_{31} < 0, \implies Q_{31} = A_{31} + \Delta U_{31} < 0.
            \\
        \end{align*}

        Уравнения состояния идеального газа для точек 1, 2, 3: $P_1V_1 = \nu R T_1, P_2V_2 = \nu R T_2, P_3V_3 = \nu R T_3$.
        Пусть $P_0$, $V_0$, $T_0$ — давление, объём и температура в точке 1 (минимальные во всём цикле).

        12 --- изохора, $\frac{P_1V_1}{T_1} = \nu R = \frac{P_2V_2}{T_2}, V_2=V_1=V_0 \implies \frac{P_1}{T_1} =  \frac{P_2}{T_2} \implies P_2 = P_1 \frac{T_2}{T_1} = 4P_0$,

        31 --- изобара, $\frac{P_1V_1}{T_1} = \nu R = \frac{P_3V_3}{T_3}, P_3=P_1=P_0 \implies \frac{V_3}{T_3} =  \frac{V_1}{T_1} \implies V_3 = V_1 \frac{T_3}{T_1} = 4V_0$,

        Таким образом, используя новые обозначения, состояния газа в точках 1, 2 и 3 описываются макропараметрами $(P_0, V_0, T_0), (4P_0, V_0, 4T_0), (P_0, 4V_0, 4T_0)$ соответственно.

        \begin{tikzpicture}[thick]
            \draw[-{Latex}] (0, 0) -- (0, 7) node[above left] {$P$};
            \draw[-{Latex}] (0, 0) -- (10, 0) node[right] {$V$};

            \draw[dashed] (0, 2) node[left] {$P_1 = P_3 = P_0$} -| (9, 0) node[below] {$V_3 = 4V_0$};
            \draw[dashed] (0, 6) node[left] {$P_2 = 4P_0$} -| (3, 0) node[below] {$V_1 = V_2 = V_0$};

            \draw[dashed] (0, 5) node[left] {$P$} -| (4.5, 0) node[below] {$V$};
            \draw[dashed] (0, 4.6) node[left] {$P'$} -| (5.1, 0) node[below] {$V'$};

            \draw (3, 2) node[above left]{1} node[below left]{$T_1 = T_0$}
                   (3, 6) node[below left]{2} node[above]{$T_2 = 4T_0$}
                   (9, 2) node[above right]{3} node[below right]{$T_3 = 4T_0$};
            \draw[midar] (3, 2) -- (3, 6);
            \draw[midar] (3, 6) -- (9, 2);
            \draw[midar] (9, 2) -- (3, 2);
            \draw   (4.5, 5) node[above right]{$T$} (5.1, 4.6) node[above right]{$T'$};
        \end{tikzpicture}


        Теперь рассмотрим отдельно процесс 23, к остальному вернёмся позже.
        Уравнение этой прямой в $PV$-координатах: $P(V) = 5P_0 - \frac{P_0}{V_0} V$.
        Это значит, что при изменении объёма на $\Delta V$ давление изменится на $\Delta P = - \frac{P_0}{V_0} \Delta V$, обратите внимание на знак.

        Рассмотрим произвольную точку в процессе 23 и дадим процессу ещё немного свершиться, при этом объём изменится на $\Delta V$, давление — на $\Delta P$, температура (иначе бы была гипербола, а не прямая) — на $\Delta T$,
        т.е.
        из состояния $(P, V, T)$ мы перешли в $(P', V', T')$, причём  $P' = P + \Delta P, V' = V + \Delta V, T' = T + \Delta T$.

        При этом изменится внутренняя энергия:
        \begin{align*}
        \Delta U
            &= U' - U = \frac 32 \nu R T' - \frac 32 \nu R T = \frac 32 (P+\Delta P) (V+\Delta V) - \frac 32 PV\\
            &= \frac 32 ((P+\Delta P) (V+\Delta V) - PV) = \frac 32 (P\Delta V + V \Delta P + \Delta P \Delta V).
        \end{align*}

        Рассмотрим малые изменения объёма, тогда и изменение давления будем малым (т.к.
        $\Delta P = - \frac{P_0}{V_0} \Delta V$),
        а третьим слагаемым в выражении для $\Delta U$  можно пренебречь по сравнению с двумя другими:
        два первых это малые величины, а третье — произведение двух малых.
        Тогда $\Delta U = \frac 32 (P\Delta V + V \Delta P)$.

        Работа газа при этом малом изменении объёма — это площадь трапеции (тут ещё раз пренебрегли малым слагаемым):
        $$A = \frac{P + P'}2 \Delta V = \cbr{P + \frac{\Delta P}2} \Delta V = P \Delta V.$$

        Подведённое количество теплоты, используя первое начало термодинамики, будет равно
        \begin{align*}
        Q
            &= \frac 32 (P\Delta V + V \Delta P) + P \Delta V =  \frac 52 P\Delta V + \frac 32 V \Delta P = \\
            &= \frac 52 P\Delta V + \frac 32 V \cdot \cbr{- \frac{P_0}{V_0} \Delta V} = \frac{\Delta V}2 \cdot \cbr{5P - \frac{P_0}{V_0} V} = \\
            &= \frac{\Delta V}2 \cdot \cbr{5 \cdot \cbr{5P_0 - \frac{P_0}{V_0} V} - \frac{P_0}{V_0} V}
             = \frac{\Delta V \cdot P_0}2 \cdot \cbr{5 \cdot 5 - 8\frac V{V_0}}.
        \end{align*}

        Таком образом, знак количества теплоты $Q$ на участке 23 зависит от конкретного значения $V$:
        \begin{itemize}
            \item $\Delta V > 0$ на всём участке 23, поскольку газ расширяется,
            \item $P > 0$ — всегда, у нас идеальный газ, удары о стенки сосуда абсолютно упругие, а молекулы не взаимодействуют и поэтому давление только положительно,
            \item если $5 \cdot 5 - 8\frac V{V_0} > 0$ — тепло подводят, если же меньше нуля — отводят.
        \end{itemize}
        Решая последнее неравенство, получаем конкретное значение $V^*$: при $V < V^*$ тепло подводят, далее~— отводят.
        Тут *~--- некоторая точка между точками 2 и 3, конкретные значения надо досчитать:
        $$V^* = V_0 \cdot \frac{5 \cdot 5}8 = \frac{25}8 \cdot V_0 \implies P^* = 5P_0 - \frac{P_0}{V_0} V^* = \ldots = \frac{15}8 \cdot P_0.$$

        Т.е.
        чтобы вычислить $Q_+$, надо сложить количества теплоты на участке 12 и лишь части участка 23 — участке 2*,
        той его части где это количество теплоты положительно.
        Имеем: $Q_+ = Q_{12} + Q_{2*}$.

        Теперь возвращаемся к циклу целиком и получаем:
        \begin{align*}
        A_\text{цикл} &= \frac 12 \cdot (4P_0 - P_0) \cdot (4V_0 - V_0) = \frac92 \cdot P_0V_0, \\
        A_{2*} &= \frac{P^* + 4P_0}2 \cdot (V^* - V_0)
            = \frac{\frac{15}8 \cdot P_0 + 4P_0}2 \cdot \cbr{\frac{25}8 \cdot V_0 - V_0}
            = \ldots = \frac{799}{128} \cdot P_0 V_0, \\
        \Delta U_{2*} &= \frac 32 \nu R T^* - \frac 32 \nu R T_2 = \frac 32 (P^*V^* - P_0 \cdot 4V_0)
            = \frac 32 \cbr{\frac{15}8 \cdot P_0 \cdot \frac{25}8 \cdot V_0 - P_0 \cdot 4V_0}
            = \frac{357}{128} \cdot P_0 V_0, \\
        \Delta U_{12} &= \frac 32 \nu R T_2 - \frac 32 \nu R T_1 = \frac 32 (4P_0V_0 - P_0V_0) = \ldots = \frac92 \cdot P_0 V_0, \\
        \eta &= \frac{A_\text{цикл}}{Q_+} = \frac{A_\text{цикл}}{Q_{12} + Q_{2*}}
            = \frac{A_\text{цикл}}{A_{12} + \Delta U_{12} + A_{2*} + \Delta U_{2*}} = \\
            &= \frac{\frac92 \cdot P_0V_0}{0 + \frac92 \cdot P_0 V_0 + \frac{799}{128} \cdot P_0 V_0 + \frac{357}{128} \cdot P_0 V_0}
             = \frac{A_bonus_cycle:LaTeX}{\frac92 + \frac{799}{128} + \frac{357}{128}}
             = \frac{144}{433} \leftarrow \text{вжух и готово!}
        \end{align*}
}
\solutionspace{360pt}

\tasknumber{2}%
\task{%
    При температуре $15\celsius$ относительная влажность воздуха составляет $60\%$.
    \begin{itemize}
        \item Определите точку росы для этого воздуха.
        \item Какой станет относительная влажность этого воздуха, если нагреть его до $50\celsius$?
    \end{itemize}
}
\answer{%
    \begin{align*}
    &\text{Значения плотности насыщенного водяного пара определяем по таблице:} \\
    &\rho_{\text{нас.
    пара 15} \celsius} = 12{,}800\,\frac{\text{г}}{\text{м}^{3}}, \rho_{\text{нас.
    пара 50} \celsius} = 83{,}000\,\frac{\text{г}}{\text{м}^{3}}.
    \\
    \varphi_1 &= \frac{\rho_\text{пара}}{\rho_{\text{нас.
    пара 15} \celsius}} \implies {\rho_\text{пара}} = \rho_{\text{нас.
    пара 15} \celsius} \cdot \varphi_1 = 12{,}800\,\frac{\text{г}}{\text{м}^{3}} \cdot 0{,}60 = 7{,}680\,\frac{\text{г}}{\text{м}^{3}}.
    \\
    &\text{По таблице определяем, при какой температуре пар с такой плотностью станет насыщенным:}  \\
    t_\text{росы} &= 6{,}8\celsius, \\
    \varphi_2 &= \frac{\rho_\text{пара}}{\rho_{\text{нас.
    пара 50} \celsius}} = \frac{\rho_{\text{нас.
    пара 15} \celsius} \cdot \varphi_1}{\rho_{\text{нас.
    пара 50} \celsius}}= \varphi_1 \cdot \frac{\rho_{\text{нас.
    пара 15} \celsius}}{\rho_{\text{нас.
    пара 50} \celsius}} = 0{,}60 \cdot \frac{12{,}800\,\frac{\text{г}}{\text{м}^{3}}}{83{,}000\,\frac{\text{г}}{\text{м}^{3}}} = 0{,}093 \approx 9{,}3\%.
    \end{align*}
}
\solutionspace{80pt}

\tasknumber{3}%
\task{%
    Из уравнения состояния идеального газа выведите или выразите...
    \begin{enumerate}
        \item давление,
        \item молярную массу,
        \item концентрацию молекул газа.
    \end{enumerate}
}

\tasknumber{4}%
\task{%
    Запишите формулы и рядом с каждой физичической величиной укажите её название и единицы измерения в СИ:
    \begin{enumerate}
        \item первое начало термодинамики,
        \item внутренняя энергия идеального одноатомного газа.
    \end{enumerate}
}

\variantsplitter

\addpersonalvariant{Виктория Легонькова}

\tasknumber{1}%
\task{%
    Определите КПД (оставив ответ точным в виде нескоратимой дроби) цикла 1231, рабочим телом которого является идеальный одноатомный газ, если
    \begin{itemize}
        \item 12 — изохорический нагрев в три раза,
        \item 23 — изобарическое расширение, при котором температура растёт в шесть раз,
        \item 31 — процесс, график которого в $PV$-координатах является отрезком прямой.
    \end{itemize}
    Бонус: замените цикл 1231 циклом, в котором 12 — изохорический нагрев в три раза, 23 — процесс, график которого в $PV$-координатах является отрезком прямой, 31 — изобарическое охлаждение, при котором температура падает в три раза.
}
\answer{%
    \begin{align*}
    A_{12} &= 0, \Delta U_{12} > 0, \implies Q_{12} = A_{12} + \Delta U_{12} > 0.
    \\
    A_{23} &> 0, \Delta U_{23} > 0, \implies Q_{23} = A_{23} + \Delta U_{23} > 0, \\
    A_{31} &= 0, \Delta U_{31} < 0, \implies Q_{31} = A_{31} + \Delta U_{31} < 0.
    \\
    P_1V_1 &= \nu R T_1, P_2V_2 = \nu R T_2, P_3V_3 = \nu R T_3 \text{ — уравнения состояния идеального газа}, \\
    &\text{Пусть $P_0$, $V_0$, $T_0$ — давление, объём и температура в точке 1 (минимальные во всём цикле):} \\
    P_1 &= P_0, P_2 = P_3, V_1 = V_2 = V_0, \text{остальные соотношения нужно считать} \\
    T_2 &= 3T_1 = 3T_0 \text{(по условию)} \implies \frac{P_2}{P_1} = \frac{P_2V_0}{P_1V_0} = \frac{P_2 V_2}{P_1 V_1}= \frac{\nu R T_2}{\nu R T_1} = \frac{T_2}{T_1} = 3 \implies P_2 = 3 P_1 = 3 P_0, \\
    T_3 &= 6T_2 = 18T_0 \text{(по условию)} \implies \frac{V_3}{V_2} = \frac{P_3V_3}{P_2V_2}= \frac{\nu R T_3}{\nu R T_2} = \frac{T_3}{T_2} = 6 \implies V_3 = 6 V_2 = 6 V_0.
    \\
    A_\text{цикл} &= \frac 12 (6P_0 - P_0)(3V_0 - V_0) = \frac 12 \cdot 10 \cdot P_0V_0, \\
    A_{23} &= 3P_0 \cdot (6V_0 - V_0) = 15P_0V_0, \\
    \Delta U_{23} &= \frac 32 \nu R T_3 - \frac 32 \nu R T_2 = \frac 32 P_3 V_3 - \frac 32 P_2 V_2 = \frac 32 \cdot 3 P_0 \cdot 6 V_0 -  \frac 32 \cdot 3 P_0 \cdot V_0 = \frac 32 \cdot 15 \cdot P_0V_0, \\
    \Delta U_{12} &= \frac 32 \nu R T_2 - \frac 32 \nu R T_1 = \frac 32 P_2 V_2 - \frac 32 P_1 V_1 = \frac 32 \cdot 3 P_0V_0 - \frac 32 P_0V_0 = \frac 32 \cdot 2 \cdot P_0V_0.
    \\
    \eta &= \frac{A_\text{цикл}}{Q_+} = \frac{A_\text{цикл}}{Q_{12} + Q_{23}}  = \frac{A_\text{цикл}}{A_{12} + \Delta U_{12} + A_{23} + \Delta U_{23}} =  \\
     &= \frac{\frac 12 \cdot 10 \cdot P_0V_0}{0 + \frac 32 \cdot 2 \cdot P_0V_0 + 15P_0V_0 + \frac 32 \cdot 15 \cdot P_0V_0} = \frac{\frac 12 \cdot 10}{\frac 32 \cdot 2 + 15 + \frac 32 \cdot 15} = \frac{10}{81} \approx 0.123.
    \end{align*}


        График процесса не в масштабе (эта часть пока не готова и сделать автоматически аккуратно сложно), но с верными подписями (а для решения этого достаточно):

        \begin{tikzpicture}[thick]
            \draw[-{Latex}] (0, 0) -- (0, 7) node[above left] {$P$};
            \draw[-{Latex}] (0, 0) -- (10, 0) node[right] {$V$};

            \draw[dashed] (0, 2) node[left] {$P_1 = P_0$} -| (3, 0) node[below] {$V_1 = V_2 = V_0$};
            \draw[dashed] (0, 6) node[left] {$P_2 = P_3 = 3P_0$} -| (9, 0) node[below] {$V_3 = 6V_0$};

            \draw (3, 2) node[above left]{1} node[below left]{$T_1 = T_0$}
                   (3, 6) node[below left]{2} node[above]{$T_2 = 3T_0$}
                   (9, 6) node[above right]{3} node[below right]{$T_3 = 18T_0$};
            \draw[midar] (3, 2) -- (3, 6);
            \draw[midar] (3, 6) -- (9, 6);
            \draw[midar] (9, 6) -- (3, 2);
        \end{tikzpicture}

        Решение бонуса:
        \begin{align*}
            A_{12} &= 0, \Delta U_{12} > 0, \implies Q_{12} = A_{12} + \Delta U_{12} > 0, \\
            A_{23} &> 0, \Delta U_{23} \text{ — ничего нельзя сказать, нужно исследовать отдельно}, \\
            A_{31} &< 0, \Delta U_{31} < 0, \implies Q_{31} = A_{31} + \Delta U_{31} < 0.
            \\
        \end{align*}

        Уравнения состояния идеального газа для точек 1, 2, 3: $P_1V_1 = \nu R T_1, P_2V_2 = \nu R T_2, P_3V_3 = \nu R T_3$.
        Пусть $P_0$, $V_0$, $T_0$ — давление, объём и температура в точке 1 (минимальные во всём цикле).

        12 --- изохора, $\frac{P_1V_1}{T_1} = \nu R = \frac{P_2V_2}{T_2}, V_2=V_1=V_0 \implies \frac{P_1}{T_1} =  \frac{P_2}{T_2} \implies P_2 = P_1 \frac{T_2}{T_1} = 3P_0$,

        31 --- изобара, $\frac{P_1V_1}{T_1} = \nu R = \frac{P_3V_3}{T_3}, P_3=P_1=P_0 \implies \frac{V_3}{T_3} =  \frac{V_1}{T_1} \implies V_3 = V_1 \frac{T_3}{T_1} = 3V_0$,

        Таким образом, используя новые обозначения, состояния газа в точках 1, 2 и 3 описываются макропараметрами $(P_0, V_0, T_0), (3P_0, V_0, 3T_0), (P_0, 3V_0, 3T_0)$ соответственно.

        \begin{tikzpicture}[thick]
            \draw[-{Latex}] (0, 0) -- (0, 7) node[above left] {$P$};
            \draw[-{Latex}] (0, 0) -- (10, 0) node[right] {$V$};

            \draw[dashed] (0, 2) node[left] {$P_1 = P_3 = P_0$} -| (9, 0) node[below] {$V_3 = 3V_0$};
            \draw[dashed] (0, 6) node[left] {$P_2 = 3P_0$} -| (3, 0) node[below] {$V_1 = V_2 = V_0$};

            \draw[dashed] (0, 5) node[left] {$P$} -| (4.5, 0) node[below] {$V$};
            \draw[dashed] (0, 4.6) node[left] {$P'$} -| (5.1, 0) node[below] {$V'$};

            \draw (3, 2) node[above left]{1} node[below left]{$T_1 = T_0$}
                   (3, 6) node[below left]{2} node[above]{$T_2 = 3T_0$}
                   (9, 2) node[above right]{3} node[below right]{$T_3 = 3T_0$};
            \draw[midar] (3, 2) -- (3, 6);
            \draw[midar] (3, 6) -- (9, 2);
            \draw[midar] (9, 2) -- (3, 2);
            \draw   (4.5, 5) node[above right]{$T$} (5.1, 4.6) node[above right]{$T'$};
        \end{tikzpicture}


        Теперь рассмотрим отдельно процесс 23, к остальному вернёмся позже.
        Уравнение этой прямой в $PV$-координатах: $P(V) = 4P_0 - \frac{P_0}{V_0} V$.
        Это значит, что при изменении объёма на $\Delta V$ давление изменится на $\Delta P = - \frac{P_0}{V_0} \Delta V$, обратите внимание на знак.

        Рассмотрим произвольную точку в процессе 23 и дадим процессу ещё немного свершиться, при этом объём изменится на $\Delta V$, давление — на $\Delta P$, температура (иначе бы была гипербола, а не прямая) — на $\Delta T$,
        т.е.
        из состояния $(P, V, T)$ мы перешли в $(P', V', T')$, причём  $P' = P + \Delta P, V' = V + \Delta V, T' = T + \Delta T$.

        При этом изменится внутренняя энергия:
        \begin{align*}
        \Delta U
            &= U' - U = \frac 32 \nu R T' - \frac 32 \nu R T = \frac 32 (P+\Delta P) (V+\Delta V) - \frac 32 PV\\
            &= \frac 32 ((P+\Delta P) (V+\Delta V) - PV) = \frac 32 (P\Delta V + V \Delta P + \Delta P \Delta V).
        \end{align*}

        Рассмотрим малые изменения объёма, тогда и изменение давления будем малым (т.к.
        $\Delta P = - \frac{P_0}{V_0} \Delta V$),
        а третьим слагаемым в выражении для $\Delta U$  можно пренебречь по сравнению с двумя другими:
        два первых это малые величины, а третье — произведение двух малых.
        Тогда $\Delta U = \frac 32 (P\Delta V + V \Delta P)$.

        Работа газа при этом малом изменении объёма — это площадь трапеции (тут ещё раз пренебрегли малым слагаемым):
        $$A = \frac{P + P'}2 \Delta V = \cbr{P + \frac{\Delta P}2} \Delta V = P \Delta V.$$

        Подведённое количество теплоты, используя первое начало термодинамики, будет равно
        \begin{align*}
        Q
            &= \frac 32 (P\Delta V + V \Delta P) + P \Delta V =  \frac 52 P\Delta V + \frac 32 V \Delta P = \\
            &= \frac 52 P\Delta V + \frac 32 V \cdot \cbr{- \frac{P_0}{V_0} \Delta V} = \frac{\Delta V}2 \cdot \cbr{5P - \frac{P_0}{V_0} V} = \\
            &= \frac{\Delta V}2 \cdot \cbr{5 \cdot \cbr{4P_0 - \frac{P_0}{V_0} V} - \frac{P_0}{V_0} V}
             = \frac{\Delta V \cdot P_0}2 \cdot \cbr{5 \cdot 4 - 8\frac V{V_0}}.
        \end{align*}

        Таком образом, знак количества теплоты $Q$ на участке 23 зависит от конкретного значения $V$:
        \begin{itemize}
            \item $\Delta V > 0$ на всём участке 23, поскольку газ расширяется,
            \item $P > 0$ — всегда, у нас идеальный газ, удары о стенки сосуда абсолютно упругие, а молекулы не взаимодействуют и поэтому давление только положительно,
            \item если $5 \cdot 4 - 8\frac V{V_0} > 0$ — тепло подводят, если же меньше нуля — отводят.
        \end{itemize}
        Решая последнее неравенство, получаем конкретное значение $V^*$: при $V < V^*$ тепло подводят, далее~— отводят.
        Тут *~--- некоторая точка между точками 2 и 3, конкретные значения надо досчитать:
        $$V^* = V_0 \cdot \frac{5 \cdot 4}8 = \frac52 \cdot V_0 \implies P^* = 4P_0 - \frac{P_0}{V_0} V^* = \ldots = \frac32 \cdot P_0.$$

        Т.е.
        чтобы вычислить $Q_+$, надо сложить количества теплоты на участке 12 и лишь части участка 23 — участке 2*,
        той его части где это количество теплоты положительно.
        Имеем: $Q_+ = Q_{12} + Q_{2*}$.

        Теперь возвращаемся к циклу целиком и получаем:
        \begin{align*}
        A_\text{цикл} &= \frac 12 \cdot (3P_0 - P_0) \cdot (3V_0 - V_0) = 2 \cdot P_0V_0, \\
        A_{2*} &= \frac{P^* + 3P_0}2 \cdot (V^* - V_0)
            = \frac{\frac32 \cdot P_0 + 3P_0}2 \cdot \cbr{\frac52 \cdot V_0 - V_0}
            = \ldots = \frac{27}8 \cdot P_0 V_0, \\
        \Delta U_{2*} &= \frac 32 \nu R T^* - \frac 32 \nu R T_2 = \frac 32 (P^*V^* - P_0 \cdot 3V_0)
            = \frac 32 \cbr{\frac32 \cdot P_0 \cdot \frac52 \cdot V_0 - P_0 \cdot 3V_0}
            = \frac98 \cdot P_0 V_0, \\
        \Delta U_{12} &= \frac 32 \nu R T_2 - \frac 32 \nu R T_1 = \frac 32 (3P_0V_0 - P_0V_0) = \ldots = 3 \cdot P_0 V_0, \\
        \eta &= \frac{A_\text{цикл}}{Q_+} = \frac{A_\text{цикл}}{Q_{12} + Q_{2*}}
            = \frac{A_\text{цикл}}{A_{12} + \Delta U_{12} + A_{2*} + \Delta U_{2*}} = \\
            &= \frac{2 \cdot P_0V_0}{0 + 3 \cdot P_0 V_0 + \frac{27}8 \cdot P_0 V_0 + \frac98 \cdot P_0 V_0}
             = \frac{A_bonus_cycle:LaTeX}{3 + \frac{27}8 + \frac98}
             = \frac4{15} \leftarrow \text{вжух и готово!}
        \end{align*}
}
\solutionspace{360pt}

\tasknumber{2}%
\task{%
    При температуре $25\celsius$ относительная влажность воздуха составляет $75\%$.
    \begin{itemize}
        \item Определите точку росы для этого воздуха.
        \item Какой станет относительная влажность этого воздуха, если нагреть его до $70\celsius$?
    \end{itemize}
}
\answer{%
    \begin{align*}
    &\text{Значения плотности насыщенного водяного пара определяем по таблице:} \\
    &\rho_{\text{нас.
    пара 25} \celsius} = 23{,}000\,\frac{\text{г}}{\text{м}^{3}}, \rho_{\text{нас.
    пара 70} \celsius} = 198{,}000\,\frac{\text{г}}{\text{м}^{3}}.
    \\
    \varphi_1 &= \frac{\rho_\text{пара}}{\rho_{\text{нас.
    пара 25} \celsius}} \implies {\rho_\text{пара}} = \rho_{\text{нас.
    пара 25} \celsius} \cdot \varphi_1 = 23{,}000\,\frac{\text{г}}{\text{м}^{3}} \cdot 0{,}75 = 17{,}250\,\frac{\text{г}}{\text{м}^{3}}.
    \\
    &\text{По таблице определяем, при какой температуре пар с такой плотностью станет насыщенным:}  \\
    t_\text{росы} &= 19{,}9\celsius, \\
    \varphi_2 &= \frac{\rho_\text{пара}}{\rho_{\text{нас.
    пара 70} \celsius}} = \frac{\rho_{\text{нас.
    пара 25} \celsius} \cdot \varphi_1}{\rho_{\text{нас.
    пара 70} \celsius}}= \varphi_1 \cdot \frac{\rho_{\text{нас.
    пара 25} \celsius}}{\rho_{\text{нас.
    пара 70} \celsius}} = 0{,}75 \cdot \frac{23{,}000\,\frac{\text{г}}{\text{м}^{3}}}{198{,}000\,\frac{\text{г}}{\text{м}^{3}}} = 0{,}087 \approx 8{,}7\%.
    \end{align*}
}
\solutionspace{80pt}

\tasknumber{3}%
\task{%
    Из уравнения состояния идеального газа выведите или выразите...
    \begin{enumerate}
        \item давление,
        \item молярную массу,
        \item концентрацию молекул газа.
    \end{enumerate}
}

\tasknumber{4}%
\task{%
    Запишите формулы и рядом с каждой физичической величиной укажите её название и единицы измерения в СИ:
    \begin{enumerate}
        \item первое начало термодинамики,
        \item внутренняя энергия идеального одноатомного газа.
    \end{enumerate}
}

\variantsplitter

\addpersonalvariant{Семён Мартынов}

\tasknumber{1}%
\task{%
    Определите КПД (оставив ответ точным в виде нескоратимой дроби) цикла 1231, рабочим телом которого является идеальный одноатомный газ, если
    \begin{itemize}
        \item 12 — изохорический нагрев в пять раз,
        \item 23 — изобарическое расширение, при котором температура растёт в пять раз,
        \item 31 — процесс, график которого в $PV$-координатах является отрезком прямой.
    \end{itemize}
    Бонус: замените цикл 1231 циклом, в котором 12 — изохорический нагрев в пять раз, 23 — процесс, график которого в $PV$-координатах является отрезком прямой, 31 — изобарическое охлаждение, при котором температура падает в пять раз.
}
\answer{%
    \begin{align*}
    A_{12} &= 0, \Delta U_{12} > 0, \implies Q_{12} = A_{12} + \Delta U_{12} > 0.
    \\
    A_{23} &> 0, \Delta U_{23} > 0, \implies Q_{23} = A_{23} + \Delta U_{23} > 0, \\
    A_{31} &= 0, \Delta U_{31} < 0, \implies Q_{31} = A_{31} + \Delta U_{31} < 0.
    \\
    P_1V_1 &= \nu R T_1, P_2V_2 = \nu R T_2, P_3V_3 = \nu R T_3 \text{ — уравнения состояния идеального газа}, \\
    &\text{Пусть $P_0$, $V_0$, $T_0$ — давление, объём и температура в точке 1 (минимальные во всём цикле):} \\
    P_1 &= P_0, P_2 = P_3, V_1 = V_2 = V_0, \text{остальные соотношения нужно считать} \\
    T_2 &= 5T_1 = 5T_0 \text{(по условию)} \implies \frac{P_2}{P_1} = \frac{P_2V_0}{P_1V_0} = \frac{P_2 V_2}{P_1 V_1}= \frac{\nu R T_2}{\nu R T_1} = \frac{T_2}{T_1} = 5 \implies P_2 = 5 P_1 = 5 P_0, \\
    T_3 &= 5T_2 = 25T_0 \text{(по условию)} \implies \frac{V_3}{V_2} = \frac{P_3V_3}{P_2V_2}= \frac{\nu R T_3}{\nu R T_2} = \frac{T_3}{T_2} = 5 \implies V_3 = 5 V_2 = 5 V_0.
    \\
    A_\text{цикл} &= \frac 12 (5P_0 - P_0)(5V_0 - V_0) = \frac 12 \cdot 16 \cdot P_0V_0, \\
    A_{23} &= 5P_0 \cdot (5V_0 - V_0) = 20P_0V_0, \\
    \Delta U_{23} &= \frac 32 \nu R T_3 - \frac 32 \nu R T_2 = \frac 32 P_3 V_3 - \frac 32 P_2 V_2 = \frac 32 \cdot 5 P_0 \cdot 5 V_0 -  \frac 32 \cdot 5 P_0 \cdot V_0 = \frac 32 \cdot 20 \cdot P_0V_0, \\
    \Delta U_{12} &= \frac 32 \nu R T_2 - \frac 32 \nu R T_1 = \frac 32 P_2 V_2 - \frac 32 P_1 V_1 = \frac 32 \cdot 5 P_0V_0 - \frac 32 P_0V_0 = \frac 32 \cdot 4 \cdot P_0V_0.
    \\
    \eta &= \frac{A_\text{цикл}}{Q_+} = \frac{A_\text{цикл}}{Q_{12} + Q_{23}}  = \frac{A_\text{цикл}}{A_{12} + \Delta U_{12} + A_{23} + \Delta U_{23}} =  \\
     &= \frac{\frac 12 \cdot 16 \cdot P_0V_0}{0 + \frac 32 \cdot 4 \cdot P_0V_0 + 20P_0V_0 + \frac 32 \cdot 20 \cdot P_0V_0} = \frac{\frac 12 \cdot 16}{\frac 32 \cdot 4 + 20 + \frac 32 \cdot 20} = \frac17 \approx 0.143.
    \end{align*}


        График процесса не в масштабе (эта часть пока не готова и сделать автоматически аккуратно сложно), но с верными подписями (а для решения этого достаточно):

        \begin{tikzpicture}[thick]
            \draw[-{Latex}] (0, 0) -- (0, 7) node[above left] {$P$};
            \draw[-{Latex}] (0, 0) -- (10, 0) node[right] {$V$};

            \draw[dashed] (0, 2) node[left] {$P_1 = P_0$} -| (3, 0) node[below] {$V_1 = V_2 = V_0$};
            \draw[dashed] (0, 6) node[left] {$P_2 = P_3 = 5P_0$} -| (9, 0) node[below] {$V_3 = 5V_0$};

            \draw (3, 2) node[above left]{1} node[below left]{$T_1 = T_0$}
                   (3, 6) node[below left]{2} node[above]{$T_2 = 5T_0$}
                   (9, 6) node[above right]{3} node[below right]{$T_3 = 25T_0$};
            \draw[midar] (3, 2) -- (3, 6);
            \draw[midar] (3, 6) -- (9, 6);
            \draw[midar] (9, 6) -- (3, 2);
        \end{tikzpicture}

        Решение бонуса:
        \begin{align*}
            A_{12} &= 0, \Delta U_{12} > 0, \implies Q_{12} = A_{12} + \Delta U_{12} > 0, \\
            A_{23} &> 0, \Delta U_{23} \text{ — ничего нельзя сказать, нужно исследовать отдельно}, \\
            A_{31} &< 0, \Delta U_{31} < 0, \implies Q_{31} = A_{31} + \Delta U_{31} < 0.
            \\
        \end{align*}

        Уравнения состояния идеального газа для точек 1, 2, 3: $P_1V_1 = \nu R T_1, P_2V_2 = \nu R T_2, P_3V_3 = \nu R T_3$.
        Пусть $P_0$, $V_0$, $T_0$ — давление, объём и температура в точке 1 (минимальные во всём цикле).

        12 --- изохора, $\frac{P_1V_1}{T_1} = \nu R = \frac{P_2V_2}{T_2}, V_2=V_1=V_0 \implies \frac{P_1}{T_1} =  \frac{P_2}{T_2} \implies P_2 = P_1 \frac{T_2}{T_1} = 5P_0$,

        31 --- изобара, $\frac{P_1V_1}{T_1} = \nu R = \frac{P_3V_3}{T_3}, P_3=P_1=P_0 \implies \frac{V_3}{T_3} =  \frac{V_1}{T_1} \implies V_3 = V_1 \frac{T_3}{T_1} = 5V_0$,

        Таким образом, используя новые обозначения, состояния газа в точках 1, 2 и 3 описываются макропараметрами $(P_0, V_0, T_0), (5P_0, V_0, 5T_0), (P_0, 5V_0, 5T_0)$ соответственно.

        \begin{tikzpicture}[thick]
            \draw[-{Latex}] (0, 0) -- (0, 7) node[above left] {$P$};
            \draw[-{Latex}] (0, 0) -- (10, 0) node[right] {$V$};

            \draw[dashed] (0, 2) node[left] {$P_1 = P_3 = P_0$} -| (9, 0) node[below] {$V_3 = 5V_0$};
            \draw[dashed] (0, 6) node[left] {$P_2 = 5P_0$} -| (3, 0) node[below] {$V_1 = V_2 = V_0$};

            \draw[dashed] (0, 5) node[left] {$P$} -| (4.5, 0) node[below] {$V$};
            \draw[dashed] (0, 4.6) node[left] {$P'$} -| (5.1, 0) node[below] {$V'$};

            \draw (3, 2) node[above left]{1} node[below left]{$T_1 = T_0$}
                   (3, 6) node[below left]{2} node[above]{$T_2 = 5T_0$}
                   (9, 2) node[above right]{3} node[below right]{$T_3 = 5T_0$};
            \draw[midar] (3, 2) -- (3, 6);
            \draw[midar] (3, 6) -- (9, 2);
            \draw[midar] (9, 2) -- (3, 2);
            \draw   (4.5, 5) node[above right]{$T$} (5.1, 4.6) node[above right]{$T'$};
        \end{tikzpicture}


        Теперь рассмотрим отдельно процесс 23, к остальному вернёмся позже.
        Уравнение этой прямой в $PV$-координатах: $P(V) = 6P_0 - \frac{P_0}{V_0} V$.
        Это значит, что при изменении объёма на $\Delta V$ давление изменится на $\Delta P = - \frac{P_0}{V_0} \Delta V$, обратите внимание на знак.

        Рассмотрим произвольную точку в процессе 23 и дадим процессу ещё немного свершиться, при этом объём изменится на $\Delta V$, давление — на $\Delta P$, температура (иначе бы была гипербола, а не прямая) — на $\Delta T$,
        т.е.
        из состояния $(P, V, T)$ мы перешли в $(P', V', T')$, причём  $P' = P + \Delta P, V' = V + \Delta V, T' = T + \Delta T$.

        При этом изменится внутренняя энергия:
        \begin{align*}
        \Delta U
            &= U' - U = \frac 32 \nu R T' - \frac 32 \nu R T = \frac 32 (P+\Delta P) (V+\Delta V) - \frac 32 PV\\
            &= \frac 32 ((P+\Delta P) (V+\Delta V) - PV) = \frac 32 (P\Delta V + V \Delta P + \Delta P \Delta V).
        \end{align*}

        Рассмотрим малые изменения объёма, тогда и изменение давления будем малым (т.к.
        $\Delta P = - \frac{P_0}{V_0} \Delta V$),
        а третьим слагаемым в выражении для $\Delta U$  можно пренебречь по сравнению с двумя другими:
        два первых это малые величины, а третье — произведение двух малых.
        Тогда $\Delta U = \frac 32 (P\Delta V + V \Delta P)$.

        Работа газа при этом малом изменении объёма — это площадь трапеции (тут ещё раз пренебрегли малым слагаемым):
        $$A = \frac{P + P'}2 \Delta V = \cbr{P + \frac{\Delta P}2} \Delta V = P \Delta V.$$

        Подведённое количество теплоты, используя первое начало термодинамики, будет равно
        \begin{align*}
        Q
            &= \frac 32 (P\Delta V + V \Delta P) + P \Delta V =  \frac 52 P\Delta V + \frac 32 V \Delta P = \\
            &= \frac 52 P\Delta V + \frac 32 V \cdot \cbr{- \frac{P_0}{V_0} \Delta V} = \frac{\Delta V}2 \cdot \cbr{5P - \frac{P_0}{V_0} V} = \\
            &= \frac{\Delta V}2 \cdot \cbr{5 \cdot \cbr{6P_0 - \frac{P_0}{V_0} V} - \frac{P_0}{V_0} V}
             = \frac{\Delta V \cdot P_0}2 \cdot \cbr{5 \cdot 6 - 8\frac V{V_0}}.
        \end{align*}

        Таком образом, знак количества теплоты $Q$ на участке 23 зависит от конкретного значения $V$:
        \begin{itemize}
            \item $\Delta V > 0$ на всём участке 23, поскольку газ расширяется,
            \item $P > 0$ — всегда, у нас идеальный газ, удары о стенки сосуда абсолютно упругие, а молекулы не взаимодействуют и поэтому давление только положительно,
            \item если $5 \cdot 6 - 8\frac V{V_0} > 0$ — тепло подводят, если же меньше нуля — отводят.
        \end{itemize}
        Решая последнее неравенство, получаем конкретное значение $V^*$: при $V < V^*$ тепло подводят, далее~— отводят.
        Тут *~--- некоторая точка между точками 2 и 3, конкретные значения надо досчитать:
        $$V^* = V_0 \cdot \frac{5 \cdot 6}8 = \frac{15}4 \cdot V_0 \implies P^* = 6P_0 - \frac{P_0}{V_0} V^* = \ldots = \frac94 \cdot P_0.$$

        Т.е.
        чтобы вычислить $Q_+$, надо сложить количества теплоты на участке 12 и лишь части участка 23 — участке 2*,
        той его части где это количество теплоты положительно.
        Имеем: $Q_+ = Q_{12} + Q_{2*}$.

        Теперь возвращаемся к циклу целиком и получаем:
        \begin{align*}
        A_\text{цикл} &= \frac 12 \cdot (5P_0 - P_0) \cdot (5V_0 - V_0) = 8 \cdot P_0V_0, \\
        A_{2*} &= \frac{P^* + 5P_0}2 \cdot (V^* - V_0)
            = \frac{\frac94 \cdot P_0 + 5P_0}2 \cdot \cbr{\frac{15}4 \cdot V_0 - V_0}
            = \ldots = \frac{319}{32} \cdot P_0 V_0, \\
        \Delta U_{2*} &= \frac 32 \nu R T^* - \frac 32 \nu R T_2 = \frac 32 (P^*V^* - P_0 \cdot 5V_0)
            = \frac 32 \cbr{\frac94 \cdot P_0 \cdot \frac{15}4 \cdot V_0 - P_0 \cdot 5V_0}
            = \frac{165}{32} \cdot P_0 V_0, \\
        \Delta U_{12} &= \frac 32 \nu R T_2 - \frac 32 \nu R T_1 = \frac 32 (5P_0V_0 - P_0V_0) = \ldots = 6 \cdot P_0 V_0, \\
        \eta &= \frac{A_\text{цикл}}{Q_+} = \frac{A_\text{цикл}}{Q_{12} + Q_{2*}}
            = \frac{A_\text{цикл}}{A_{12} + \Delta U_{12} + A_{2*} + \Delta U_{2*}} = \\
            &= \frac{8 \cdot P_0V_0}{0 + 6 \cdot P_0 V_0 + \frac{319}{32} \cdot P_0 V_0 + \frac{165}{32} \cdot P_0 V_0}
             = \frac{A_bonus_cycle:LaTeX}{6 + \frac{319}{32} + \frac{165}{32}}
             = \frac{64}{169} \leftarrow \text{вжух и готово!}
        \end{align*}
}
\solutionspace{360pt}

\tasknumber{2}%
\task{%
    При температуре $30\celsius$ относительная влажность воздуха составляет $55\%$.
    \begin{itemize}
        \item Определите точку росы для этого воздуха.
        \item Какой станет относительная влажность этого воздуха, если нагреть его до $70\celsius$?
    \end{itemize}
}
\answer{%
    \begin{align*}
    &\text{Значения плотности насыщенного водяного пара определяем по таблице:} \\
    &\rho_{\text{нас.
    пара 30} \celsius} = 30{,}300\,\frac{\text{г}}{\text{м}^{3}}, \rho_{\text{нас.
    пара 70} \celsius} = 198{,}000\,\frac{\text{г}}{\text{м}^{3}}.
    \\
    \varphi_1 &= \frac{\rho_\text{пара}}{\rho_{\text{нас.
    пара 30} \celsius}} \implies {\rho_\text{пара}} = \rho_{\text{нас.
    пара 30} \celsius} \cdot \varphi_1 = 30{,}300\,\frac{\text{г}}{\text{м}^{3}} \cdot 0{,}55 = 16{,}665\,\frac{\text{г}}{\text{м}^{3}}.
    \\
    &\text{По таблице определяем, при какой температуре пар с такой плотностью станет насыщенным:}  \\
    t_\text{росы} &= 19{,}4\celsius, \\
    \varphi_2 &= \frac{\rho_\text{пара}}{\rho_{\text{нас.
    пара 70} \celsius}} = \frac{\rho_{\text{нас.
    пара 30} \celsius} \cdot \varphi_1}{\rho_{\text{нас.
    пара 70} \celsius}}= \varphi_1 \cdot \frac{\rho_{\text{нас.
    пара 30} \celsius}}{\rho_{\text{нас.
    пара 70} \celsius}} = 0{,}55 \cdot \frac{30{,}300\,\frac{\text{г}}{\text{м}^{3}}}{198{,}000\,\frac{\text{г}}{\text{м}^{3}}} = 0{,}084 \approx 8{,}4\%.
    \end{align*}
}
\solutionspace{80pt}

\tasknumber{3}%
\task{%
    Из уравнения состояния идеального газа выведите или выразите...
    \begin{enumerate}
        \item объём,
        \item температуру,
        \item плотность газа.
    \end{enumerate}
}

\tasknumber{4}%
\task{%
    Запишите формулы и рядом с каждой физичической величиной укажите её название и единицы измерения в СИ:
    \begin{enumerate}
        \item первое начало термодинамики,
        \item внутренняя энергия идеального одноатомного газа.
    \end{enumerate}
}

\variantsplitter

\addpersonalvariant{Варвара Минаева}

\tasknumber{1}%
\task{%
    Определите КПД (оставив ответ точным в виде нескоратимой дроби) цикла 1231, рабочим телом которого является идеальный одноатомный газ, если
    \begin{itemize}
        \item 12 — изохорический нагрев в два раза,
        \item 23 — изобарическое расширение, при котором температура растёт в два раза,
        \item 31 — процесс, график которого в $PV$-координатах является отрезком прямой.
    \end{itemize}
    Бонус: замените цикл 1231 циклом, в котором 12 — изохорический нагрев в два раза, 23 — процесс, график которого в $PV$-координатах является отрезком прямой, 31 — изобарическое охлаждение, при котором температура падает в два раза.
}
\answer{%
    \begin{align*}
    A_{12} &= 0, \Delta U_{12} > 0, \implies Q_{12} = A_{12} + \Delta U_{12} > 0.
    \\
    A_{23} &> 0, \Delta U_{23} > 0, \implies Q_{23} = A_{23} + \Delta U_{23} > 0, \\
    A_{31} &= 0, \Delta U_{31} < 0, \implies Q_{31} = A_{31} + \Delta U_{31} < 0.
    \\
    P_1V_1 &= \nu R T_1, P_2V_2 = \nu R T_2, P_3V_3 = \nu R T_3 \text{ — уравнения состояния идеального газа}, \\
    &\text{Пусть $P_0$, $V_0$, $T_0$ — давление, объём и температура в точке 1 (минимальные во всём цикле):} \\
    P_1 &= P_0, P_2 = P_3, V_1 = V_2 = V_0, \text{остальные соотношения нужно считать} \\
    T_2 &= 2T_1 = 2T_0 \text{(по условию)} \implies \frac{P_2}{P_1} = \frac{P_2V_0}{P_1V_0} = \frac{P_2 V_2}{P_1 V_1}= \frac{\nu R T_2}{\nu R T_1} = \frac{T_2}{T_1} = 2 \implies P_2 = 2 P_1 = 2 P_0, \\
    T_3 &= 2T_2 = 4T_0 \text{(по условию)} \implies \frac{V_3}{V_2} = \frac{P_3V_3}{P_2V_2}= \frac{\nu R T_3}{\nu R T_2} = \frac{T_3}{T_2} = 2 \implies V_3 = 2 V_2 = 2 V_0.
    \\
    A_\text{цикл} &= \frac 12 (2P_0 - P_0)(2V_0 - V_0) = \frac 12 \cdot 1 \cdot P_0V_0, \\
    A_{23} &= 2P_0 \cdot (2V_0 - V_0) = 2P_0V_0, \\
    \Delta U_{23} &= \frac 32 \nu R T_3 - \frac 32 \nu R T_2 = \frac 32 P_3 V_3 - \frac 32 P_2 V_2 = \frac 32 \cdot 2 P_0 \cdot 2 V_0 -  \frac 32 \cdot 2 P_0 \cdot V_0 = \frac 32 \cdot 2 \cdot P_0V_0, \\
    \Delta U_{12} &= \frac 32 \nu R T_2 - \frac 32 \nu R T_1 = \frac 32 P_2 V_2 - \frac 32 P_1 V_1 = \frac 32 \cdot 2 P_0V_0 - \frac 32 P_0V_0 = \frac 32 \cdot 1 \cdot P_0V_0.
    \\
    \eta &= \frac{A_\text{цикл}}{Q_+} = \frac{A_\text{цикл}}{Q_{12} + Q_{23}}  = \frac{A_\text{цикл}}{A_{12} + \Delta U_{12} + A_{23} + \Delta U_{23}} =  \\
     &= \frac{\frac 12 \cdot 1 \cdot P_0V_0}{0 + \frac 32 \cdot 1 \cdot P_0V_0 + 2P_0V_0 + \frac 32 \cdot 2 \cdot P_0V_0} = \frac{\frac 12 \cdot 1}{\frac 32 \cdot 1 + 2 + \frac 32 \cdot 2} = \frac1{13} \approx 0.077.
    \end{align*}


        График процесса не в масштабе (эта часть пока не готова и сделать автоматически аккуратно сложно), но с верными подписями (а для решения этого достаточно):

        \begin{tikzpicture}[thick]
            \draw[-{Latex}] (0, 0) -- (0, 7) node[above left] {$P$};
            \draw[-{Latex}] (0, 0) -- (10, 0) node[right] {$V$};

            \draw[dashed] (0, 2) node[left] {$P_1 = P_0$} -| (3, 0) node[below] {$V_1 = V_2 = V_0$};
            \draw[dashed] (0, 6) node[left] {$P_2 = P_3 = 2P_0$} -| (9, 0) node[below] {$V_3 = 2V_0$};

            \draw (3, 2) node[above left]{1} node[below left]{$T_1 = T_0$}
                   (3, 6) node[below left]{2} node[above]{$T_2 = 2T_0$}
                   (9, 6) node[above right]{3} node[below right]{$T_3 = 4T_0$};
            \draw[midar] (3, 2) -- (3, 6);
            \draw[midar] (3, 6) -- (9, 6);
            \draw[midar] (9, 6) -- (3, 2);
        \end{tikzpicture}

        Решение бонуса:
        \begin{align*}
            A_{12} &= 0, \Delta U_{12} > 0, \implies Q_{12} = A_{12} + \Delta U_{12} > 0, \\
            A_{23} &> 0, \Delta U_{23} \text{ — ничего нельзя сказать, нужно исследовать отдельно}, \\
            A_{31} &< 0, \Delta U_{31} < 0, \implies Q_{31} = A_{31} + \Delta U_{31} < 0.
            \\
        \end{align*}

        Уравнения состояния идеального газа для точек 1, 2, 3: $P_1V_1 = \nu R T_1, P_2V_2 = \nu R T_2, P_3V_3 = \nu R T_3$.
        Пусть $P_0$, $V_0$, $T_0$ — давление, объём и температура в точке 1 (минимальные во всём цикле).

        12 --- изохора, $\frac{P_1V_1}{T_1} = \nu R = \frac{P_2V_2}{T_2}, V_2=V_1=V_0 \implies \frac{P_1}{T_1} =  \frac{P_2}{T_2} \implies P_2 = P_1 \frac{T_2}{T_1} = 2P_0$,

        31 --- изобара, $\frac{P_1V_1}{T_1} = \nu R = \frac{P_3V_3}{T_3}, P_3=P_1=P_0 \implies \frac{V_3}{T_3} =  \frac{V_1}{T_1} \implies V_3 = V_1 \frac{T_3}{T_1} = 2V_0$,

        Таким образом, используя новые обозначения, состояния газа в точках 1, 2 и 3 описываются макропараметрами $(P_0, V_0, T_0), (2P_0, V_0, 2T_0), (P_0, 2V_0, 2T_0)$ соответственно.

        \begin{tikzpicture}[thick]
            \draw[-{Latex}] (0, 0) -- (0, 7) node[above left] {$P$};
            \draw[-{Latex}] (0, 0) -- (10, 0) node[right] {$V$};

            \draw[dashed] (0, 2) node[left] {$P_1 = P_3 = P_0$} -| (9, 0) node[below] {$V_3 = 2V_0$};
            \draw[dashed] (0, 6) node[left] {$P_2 = 2P_0$} -| (3, 0) node[below] {$V_1 = V_2 = V_0$};

            \draw[dashed] (0, 5) node[left] {$P$} -| (4.5, 0) node[below] {$V$};
            \draw[dashed] (0, 4.6) node[left] {$P'$} -| (5.1, 0) node[below] {$V'$};

            \draw (3, 2) node[above left]{1} node[below left]{$T_1 = T_0$}
                   (3, 6) node[below left]{2} node[above]{$T_2 = 2T_0$}
                   (9, 2) node[above right]{3} node[below right]{$T_3 = 2T_0$};
            \draw[midar] (3, 2) -- (3, 6);
            \draw[midar] (3, 6) -- (9, 2);
            \draw[midar] (9, 2) -- (3, 2);
            \draw   (4.5, 5) node[above right]{$T$} (5.1, 4.6) node[above right]{$T'$};
        \end{tikzpicture}


        Теперь рассмотрим отдельно процесс 23, к остальному вернёмся позже.
        Уравнение этой прямой в $PV$-координатах: $P(V) = 3P_0 - \frac{P_0}{V_0} V$.
        Это значит, что при изменении объёма на $\Delta V$ давление изменится на $\Delta P = - \frac{P_0}{V_0} \Delta V$, обратите внимание на знак.

        Рассмотрим произвольную точку в процессе 23 и дадим процессу ещё немного свершиться, при этом объём изменится на $\Delta V$, давление — на $\Delta P$, температура (иначе бы была гипербола, а не прямая) — на $\Delta T$,
        т.е.
        из состояния $(P, V, T)$ мы перешли в $(P', V', T')$, причём  $P' = P + \Delta P, V' = V + \Delta V, T' = T + \Delta T$.

        При этом изменится внутренняя энергия:
        \begin{align*}
        \Delta U
            &= U' - U = \frac 32 \nu R T' - \frac 32 \nu R T = \frac 32 (P+\Delta P) (V+\Delta V) - \frac 32 PV\\
            &= \frac 32 ((P+\Delta P) (V+\Delta V) - PV) = \frac 32 (P\Delta V + V \Delta P + \Delta P \Delta V).
        \end{align*}

        Рассмотрим малые изменения объёма, тогда и изменение давления будем малым (т.к.
        $\Delta P = - \frac{P_0}{V_0} \Delta V$),
        а третьим слагаемым в выражении для $\Delta U$  можно пренебречь по сравнению с двумя другими:
        два первых это малые величины, а третье — произведение двух малых.
        Тогда $\Delta U = \frac 32 (P\Delta V + V \Delta P)$.

        Работа газа при этом малом изменении объёма — это площадь трапеции (тут ещё раз пренебрегли малым слагаемым):
        $$A = \frac{P + P'}2 \Delta V = \cbr{P + \frac{\Delta P}2} \Delta V = P \Delta V.$$

        Подведённое количество теплоты, используя первое начало термодинамики, будет равно
        \begin{align*}
        Q
            &= \frac 32 (P\Delta V + V \Delta P) + P \Delta V =  \frac 52 P\Delta V + \frac 32 V \Delta P = \\
            &= \frac 52 P\Delta V + \frac 32 V \cdot \cbr{- \frac{P_0}{V_0} \Delta V} = \frac{\Delta V}2 \cdot \cbr{5P - \frac{P_0}{V_0} V} = \\
            &= \frac{\Delta V}2 \cdot \cbr{5 \cdot \cbr{3P_0 - \frac{P_0}{V_0} V} - \frac{P_0}{V_0} V}
             = \frac{\Delta V \cdot P_0}2 \cdot \cbr{5 \cdot 3 - 8\frac V{V_0}}.
        \end{align*}

        Таком образом, знак количества теплоты $Q$ на участке 23 зависит от конкретного значения $V$:
        \begin{itemize}
            \item $\Delta V > 0$ на всём участке 23, поскольку газ расширяется,
            \item $P > 0$ — всегда, у нас идеальный газ, удары о стенки сосуда абсолютно упругие, а молекулы не взаимодействуют и поэтому давление только положительно,
            \item если $5 \cdot 3 - 8\frac V{V_0} > 0$ — тепло подводят, если же меньше нуля — отводят.
        \end{itemize}
        Решая последнее неравенство, получаем конкретное значение $V^*$: при $V < V^*$ тепло подводят, далее~— отводят.
        Тут *~--- некоторая точка между точками 2 и 3, конкретные значения надо досчитать:
        $$V^* = V_0 \cdot \frac{5 \cdot 3}8 = \frac{15}8 \cdot V_0 \implies P^* = 3P_0 - \frac{P_0}{V_0} V^* = \ldots = \frac98 \cdot P_0.$$

        Т.е.
        чтобы вычислить $Q_+$, надо сложить количества теплоты на участке 12 и лишь части участка 23 — участке 2*,
        той его части где это количество теплоты положительно.
        Имеем: $Q_+ = Q_{12} + Q_{2*}$.

        Теперь возвращаемся к циклу целиком и получаем:
        \begin{align*}
        A_\text{цикл} &= \frac 12 \cdot (2P_0 - P_0) \cdot (2V_0 - V_0) = \frac12 \cdot P_0V_0, \\
        A_{2*} &= \frac{P^* + 2P_0}2 \cdot (V^* - V_0)
            = \frac{\frac98 \cdot P_0 + 2P_0}2 \cdot \cbr{\frac{15}8 \cdot V_0 - V_0}
            = \ldots = \frac{175}{128} \cdot P_0 V_0, \\
        \Delta U_{2*} &= \frac 32 \nu R T^* - \frac 32 \nu R T_2 = \frac 32 (P^*V^* - P_0 \cdot 2V_0)
            = \frac 32 \cbr{\frac98 \cdot P_0 \cdot \frac{15}8 \cdot V_0 - P_0 \cdot 2V_0}
            = \frac{21}{128} \cdot P_0 V_0, \\
        \Delta U_{12} &= \frac 32 \nu R T_2 - \frac 32 \nu R T_1 = \frac 32 (2P_0V_0 - P_0V_0) = \ldots = \frac32 \cdot P_0 V_0, \\
        \eta &= \frac{A_\text{цикл}}{Q_+} = \frac{A_\text{цикл}}{Q_{12} + Q_{2*}}
            = \frac{A_\text{цикл}}{A_{12} + \Delta U_{12} + A_{2*} + \Delta U_{2*}} = \\
            &= \frac{\frac12 \cdot P_0V_0}{0 + \frac32 \cdot P_0 V_0 + \frac{175}{128} \cdot P_0 V_0 + \frac{21}{128} \cdot P_0 V_0}
             = \frac{A_bonus_cycle:LaTeX}{\frac32 + \frac{175}{128} + \frac{21}{128}}
             = \frac{16}{97} \leftarrow \text{вжух и готово!}
        \end{align*}
}
\solutionspace{360pt}

\tasknumber{2}%
\task{%
    При температуре $20\celsius$ относительная влажность воздуха составляет $55\%$.
    \begin{itemize}
        \item Определите точку росы для этого воздуха.
        \item Какой станет относительная влажность этого воздуха, если нагреть его до $70\celsius$?
    \end{itemize}
}
\answer{%
    \begin{align*}
    &\text{Значения плотности насыщенного водяного пара определяем по таблице:} \\
    &\rho_{\text{нас.
    пара 20} \celsius} = 17{,}300\,\frac{\text{г}}{\text{м}^{3}}, \rho_{\text{нас.
    пара 70} \celsius} = 198{,}000\,\frac{\text{г}}{\text{м}^{3}}.
    \\
    \varphi_1 &= \frac{\rho_\text{пара}}{\rho_{\text{нас.
    пара 20} \celsius}} \implies {\rho_\text{пара}} = \rho_{\text{нас.
    пара 20} \celsius} \cdot \varphi_1 = 17{,}300\,\frac{\text{г}}{\text{м}^{3}} \cdot 0{,}55 = 9{,}515\,\frac{\text{г}}{\text{м}^{3}}.
    \\
    &\text{По таблице определяем, при какой температуре пар с такой плотностью станет насыщенным:}  \\
    t_\text{росы} &= 10{,}2\celsius, \\
    \varphi_2 &= \frac{\rho_\text{пара}}{\rho_{\text{нас.
    пара 70} \celsius}} = \frac{\rho_{\text{нас.
    пара 20} \celsius} \cdot \varphi_1}{\rho_{\text{нас.
    пара 70} \celsius}}= \varphi_1 \cdot \frac{\rho_{\text{нас.
    пара 20} \celsius}}{\rho_{\text{нас.
    пара 70} \celsius}} = 0{,}55 \cdot \frac{17{,}300\,\frac{\text{г}}{\text{м}^{3}}}{198{,}000\,\frac{\text{г}}{\text{м}^{3}}} = 0{,}048 \approx 4{,}8\%.
    \end{align*}
}
\solutionspace{80pt}

\tasknumber{3}%
\task{%
    Из уравнения состояния идеального газа выведите или выразите...
    \begin{enumerate}
        \item объём,
        \item температуру,
        \item плотность газа.
    \end{enumerate}
}

\tasknumber{4}%
\task{%
    Запишите формулы и рядом с каждой физичической величиной укажите её название и единицы измерения в СИ:
    \begin{enumerate}
        \item первое начало термодинамики,
        \item внутренняя энергия идеального одноатомного газа.
    \end{enumerate}
}

\variantsplitter

\addpersonalvariant{Леонид Никитин}

\tasknumber{1}%
\task{%
    Определите КПД (оставив ответ точным в виде нескоратимой дроби) цикла 1231, рабочим телом которого является идеальный одноатомный газ, если
    \begin{itemize}
        \item 12 — изохорический нагрев в три раза,
        \item 23 — изобарическое расширение, при котором температура растёт в два раза,
        \item 31 — процесс, график которого в $PV$-координатах является отрезком прямой.
    \end{itemize}
    Бонус: замените цикл 1231 циклом, в котором 12 — изохорический нагрев в три раза, 23 — процесс, график которого в $PV$-координатах является отрезком прямой, 31 — изобарическое охлаждение, при котором температура падает в три раза.
}
\answer{%
    \begin{align*}
    A_{12} &= 0, \Delta U_{12} > 0, \implies Q_{12} = A_{12} + \Delta U_{12} > 0.
    \\
    A_{23} &> 0, \Delta U_{23} > 0, \implies Q_{23} = A_{23} + \Delta U_{23} > 0, \\
    A_{31} &= 0, \Delta U_{31} < 0, \implies Q_{31} = A_{31} + \Delta U_{31} < 0.
    \\
    P_1V_1 &= \nu R T_1, P_2V_2 = \nu R T_2, P_3V_3 = \nu R T_3 \text{ — уравнения состояния идеального газа}, \\
    &\text{Пусть $P_0$, $V_0$, $T_0$ — давление, объём и температура в точке 1 (минимальные во всём цикле):} \\
    P_1 &= P_0, P_2 = P_3, V_1 = V_2 = V_0, \text{остальные соотношения нужно считать} \\
    T_2 &= 3T_1 = 3T_0 \text{(по условию)} \implies \frac{P_2}{P_1} = \frac{P_2V_0}{P_1V_0} = \frac{P_2 V_2}{P_1 V_1}= \frac{\nu R T_2}{\nu R T_1} = \frac{T_2}{T_1} = 3 \implies P_2 = 3 P_1 = 3 P_0, \\
    T_3 &= 2T_2 = 6T_0 \text{(по условию)} \implies \frac{V_3}{V_2} = \frac{P_3V_3}{P_2V_2}= \frac{\nu R T_3}{\nu R T_2} = \frac{T_3}{T_2} = 2 \implies V_3 = 2 V_2 = 2 V_0.
    \\
    A_\text{цикл} &= \frac 12 (2P_0 - P_0)(3V_0 - V_0) = \frac 12 \cdot 2 \cdot P_0V_0, \\
    A_{23} &= 3P_0 \cdot (2V_0 - V_0) = 3P_0V_0, \\
    \Delta U_{23} &= \frac 32 \nu R T_3 - \frac 32 \nu R T_2 = \frac 32 P_3 V_3 - \frac 32 P_2 V_2 = \frac 32 \cdot 3 P_0 \cdot 2 V_0 -  \frac 32 \cdot 3 P_0 \cdot V_0 = \frac 32 \cdot 3 \cdot P_0V_0, \\
    \Delta U_{12} &= \frac 32 \nu R T_2 - \frac 32 \nu R T_1 = \frac 32 P_2 V_2 - \frac 32 P_1 V_1 = \frac 32 \cdot 3 P_0V_0 - \frac 32 P_0V_0 = \frac 32 \cdot 2 \cdot P_0V_0.
    \\
    \eta &= \frac{A_\text{цикл}}{Q_+} = \frac{A_\text{цикл}}{Q_{12} + Q_{23}}  = \frac{A_\text{цикл}}{A_{12} + \Delta U_{12} + A_{23} + \Delta U_{23}} =  \\
     &= \frac{\frac 12 \cdot 2 \cdot P_0V_0}{0 + \frac 32 \cdot 2 \cdot P_0V_0 + 3P_0V_0 + \frac 32 \cdot 3 \cdot P_0V_0} = \frac{\frac 12 \cdot 2}{\frac 32 \cdot 2 + 3 + \frac 32 \cdot 3} = \frac2{21} \approx 0.095.
    \end{align*}


        График процесса не в масштабе (эта часть пока не готова и сделать автоматически аккуратно сложно), но с верными подписями (а для решения этого достаточно):

        \begin{tikzpicture}[thick]
            \draw[-{Latex}] (0, 0) -- (0, 7) node[above left] {$P$};
            \draw[-{Latex}] (0, 0) -- (10, 0) node[right] {$V$};

            \draw[dashed] (0, 2) node[left] {$P_1 = P_0$} -| (3, 0) node[below] {$V_1 = V_2 = V_0$};
            \draw[dashed] (0, 6) node[left] {$P_2 = P_3 = 3P_0$} -| (9, 0) node[below] {$V_3 = 2V_0$};

            \draw (3, 2) node[above left]{1} node[below left]{$T_1 = T_0$}
                   (3, 6) node[below left]{2} node[above]{$T_2 = 3T_0$}
                   (9, 6) node[above right]{3} node[below right]{$T_3 = 6T_0$};
            \draw[midar] (3, 2) -- (3, 6);
            \draw[midar] (3, 6) -- (9, 6);
            \draw[midar] (9, 6) -- (3, 2);
        \end{tikzpicture}

        Решение бонуса:
        \begin{align*}
            A_{12} &= 0, \Delta U_{12} > 0, \implies Q_{12} = A_{12} + \Delta U_{12} > 0, \\
            A_{23} &> 0, \Delta U_{23} \text{ — ничего нельзя сказать, нужно исследовать отдельно}, \\
            A_{31} &< 0, \Delta U_{31} < 0, \implies Q_{31} = A_{31} + \Delta U_{31} < 0.
            \\
        \end{align*}

        Уравнения состояния идеального газа для точек 1, 2, 3: $P_1V_1 = \nu R T_1, P_2V_2 = \nu R T_2, P_3V_3 = \nu R T_3$.
        Пусть $P_0$, $V_0$, $T_0$ — давление, объём и температура в точке 1 (минимальные во всём цикле).

        12 --- изохора, $\frac{P_1V_1}{T_1} = \nu R = \frac{P_2V_2}{T_2}, V_2=V_1=V_0 \implies \frac{P_1}{T_1} =  \frac{P_2}{T_2} \implies P_2 = P_1 \frac{T_2}{T_1} = 3P_0$,

        31 --- изобара, $\frac{P_1V_1}{T_1} = \nu R = \frac{P_3V_3}{T_3}, P_3=P_1=P_0 \implies \frac{V_3}{T_3} =  \frac{V_1}{T_1} \implies V_3 = V_1 \frac{T_3}{T_1} = 3V_0$,

        Таким образом, используя новые обозначения, состояния газа в точках 1, 2 и 3 описываются макропараметрами $(P_0, V_0, T_0), (3P_0, V_0, 3T_0), (P_0, 3V_0, 3T_0)$ соответственно.

        \begin{tikzpicture}[thick]
            \draw[-{Latex}] (0, 0) -- (0, 7) node[above left] {$P$};
            \draw[-{Latex}] (0, 0) -- (10, 0) node[right] {$V$};

            \draw[dashed] (0, 2) node[left] {$P_1 = P_3 = P_0$} -| (9, 0) node[below] {$V_3 = 3V_0$};
            \draw[dashed] (0, 6) node[left] {$P_2 = 3P_0$} -| (3, 0) node[below] {$V_1 = V_2 = V_0$};

            \draw[dashed] (0, 5) node[left] {$P$} -| (4.5, 0) node[below] {$V$};
            \draw[dashed] (0, 4.6) node[left] {$P'$} -| (5.1, 0) node[below] {$V'$};

            \draw (3, 2) node[above left]{1} node[below left]{$T_1 = T_0$}
                   (3, 6) node[below left]{2} node[above]{$T_2 = 3T_0$}
                   (9, 2) node[above right]{3} node[below right]{$T_3 = 3T_0$};
            \draw[midar] (3, 2) -- (3, 6);
            \draw[midar] (3, 6) -- (9, 2);
            \draw[midar] (9, 2) -- (3, 2);
            \draw   (4.5, 5) node[above right]{$T$} (5.1, 4.6) node[above right]{$T'$};
        \end{tikzpicture}


        Теперь рассмотрим отдельно процесс 23, к остальному вернёмся позже.
        Уравнение этой прямой в $PV$-координатах: $P(V) = 4P_0 - \frac{P_0}{V_0} V$.
        Это значит, что при изменении объёма на $\Delta V$ давление изменится на $\Delta P = - \frac{P_0}{V_0} \Delta V$, обратите внимание на знак.

        Рассмотрим произвольную точку в процессе 23 и дадим процессу ещё немного свершиться, при этом объём изменится на $\Delta V$, давление — на $\Delta P$, температура (иначе бы была гипербола, а не прямая) — на $\Delta T$,
        т.е.
        из состояния $(P, V, T)$ мы перешли в $(P', V', T')$, причём  $P' = P + \Delta P, V' = V + \Delta V, T' = T + \Delta T$.

        При этом изменится внутренняя энергия:
        \begin{align*}
        \Delta U
            &= U' - U = \frac 32 \nu R T' - \frac 32 \nu R T = \frac 32 (P+\Delta P) (V+\Delta V) - \frac 32 PV\\
            &= \frac 32 ((P+\Delta P) (V+\Delta V) - PV) = \frac 32 (P\Delta V + V \Delta P + \Delta P \Delta V).
        \end{align*}

        Рассмотрим малые изменения объёма, тогда и изменение давления будем малым (т.к.
        $\Delta P = - \frac{P_0}{V_0} \Delta V$),
        а третьим слагаемым в выражении для $\Delta U$  можно пренебречь по сравнению с двумя другими:
        два первых это малые величины, а третье — произведение двух малых.
        Тогда $\Delta U = \frac 32 (P\Delta V + V \Delta P)$.

        Работа газа при этом малом изменении объёма — это площадь трапеции (тут ещё раз пренебрегли малым слагаемым):
        $$A = \frac{P + P'}2 \Delta V = \cbr{P + \frac{\Delta P}2} \Delta V = P \Delta V.$$

        Подведённое количество теплоты, используя первое начало термодинамики, будет равно
        \begin{align*}
        Q
            &= \frac 32 (P\Delta V + V \Delta P) + P \Delta V =  \frac 52 P\Delta V + \frac 32 V \Delta P = \\
            &= \frac 52 P\Delta V + \frac 32 V \cdot \cbr{- \frac{P_0}{V_0} \Delta V} = \frac{\Delta V}2 \cdot \cbr{5P - \frac{P_0}{V_0} V} = \\
            &= \frac{\Delta V}2 \cdot \cbr{5 \cdot \cbr{4P_0 - \frac{P_0}{V_0} V} - \frac{P_0}{V_0} V}
             = \frac{\Delta V \cdot P_0}2 \cdot \cbr{5 \cdot 4 - 8\frac V{V_0}}.
        \end{align*}

        Таком образом, знак количества теплоты $Q$ на участке 23 зависит от конкретного значения $V$:
        \begin{itemize}
            \item $\Delta V > 0$ на всём участке 23, поскольку газ расширяется,
            \item $P > 0$ — всегда, у нас идеальный газ, удары о стенки сосуда абсолютно упругие, а молекулы не взаимодействуют и поэтому давление только положительно,
            \item если $5 \cdot 4 - 8\frac V{V_0} > 0$ — тепло подводят, если же меньше нуля — отводят.
        \end{itemize}
        Решая последнее неравенство, получаем конкретное значение $V^*$: при $V < V^*$ тепло подводят, далее~— отводят.
        Тут *~--- некоторая точка между точками 2 и 3, конкретные значения надо досчитать:
        $$V^* = V_0 \cdot \frac{5 \cdot 4}8 = \frac52 \cdot V_0 \implies P^* = 4P_0 - \frac{P_0}{V_0} V^* = \ldots = \frac32 \cdot P_0.$$

        Т.е.
        чтобы вычислить $Q_+$, надо сложить количества теплоты на участке 12 и лишь части участка 23 — участке 2*,
        той его части где это количество теплоты положительно.
        Имеем: $Q_+ = Q_{12} + Q_{2*}$.

        Теперь возвращаемся к циклу целиком и получаем:
        \begin{align*}
        A_\text{цикл} &= \frac 12 \cdot (3P_0 - P_0) \cdot (3V_0 - V_0) = 2 \cdot P_0V_0, \\
        A_{2*} &= \frac{P^* + 3P_0}2 \cdot (V^* - V_0)
            = \frac{\frac32 \cdot P_0 + 3P_0}2 \cdot \cbr{\frac52 \cdot V_0 - V_0}
            = \ldots = \frac{27}8 \cdot P_0 V_0, \\
        \Delta U_{2*} &= \frac 32 \nu R T^* - \frac 32 \nu R T_2 = \frac 32 (P^*V^* - P_0 \cdot 3V_0)
            = \frac 32 \cbr{\frac32 \cdot P_0 \cdot \frac52 \cdot V_0 - P_0 \cdot 3V_0}
            = \frac98 \cdot P_0 V_0, \\
        \Delta U_{12} &= \frac 32 \nu R T_2 - \frac 32 \nu R T_1 = \frac 32 (3P_0V_0 - P_0V_0) = \ldots = 3 \cdot P_0 V_0, \\
        \eta &= \frac{A_\text{цикл}}{Q_+} = \frac{A_\text{цикл}}{Q_{12} + Q_{2*}}
            = \frac{A_\text{цикл}}{A_{12} + \Delta U_{12} + A_{2*} + \Delta U_{2*}} = \\
            &= \frac{2 \cdot P_0V_0}{0 + 3 \cdot P_0 V_0 + \frac{27}8 \cdot P_0 V_0 + \frac98 \cdot P_0 V_0}
             = \frac{A_bonus_cycle:LaTeX}{3 + \frac{27}8 + \frac98}
             = \frac4{15} \leftarrow \text{вжух и готово!}
        \end{align*}
}
\solutionspace{360pt}

\tasknumber{2}%
\task{%
    При температуре $15\celsius$ относительная влажность воздуха составляет $60\%$.
    \begin{itemize}
        \item Определите точку росы для этого воздуха.
        \item Какой станет относительная влажность этого воздуха, если нагреть его до $50\celsius$?
    \end{itemize}
}
\answer{%
    \begin{align*}
    &\text{Значения плотности насыщенного водяного пара определяем по таблице:} \\
    &\rho_{\text{нас.
    пара 15} \celsius} = 12{,}800\,\frac{\text{г}}{\text{м}^{3}}, \rho_{\text{нас.
    пара 50} \celsius} = 83{,}000\,\frac{\text{г}}{\text{м}^{3}}.
    \\
    \varphi_1 &= \frac{\rho_\text{пара}}{\rho_{\text{нас.
    пара 15} \celsius}} \implies {\rho_\text{пара}} = \rho_{\text{нас.
    пара 15} \celsius} \cdot \varphi_1 = 12{,}800\,\frac{\text{г}}{\text{м}^{3}} \cdot 0{,}60 = 7{,}680\,\frac{\text{г}}{\text{м}^{3}}.
    \\
    &\text{По таблице определяем, при какой температуре пар с такой плотностью станет насыщенным:}  \\
    t_\text{росы} &= 6{,}8\celsius, \\
    \varphi_2 &= \frac{\rho_\text{пара}}{\rho_{\text{нас.
    пара 50} \celsius}} = \frac{\rho_{\text{нас.
    пара 15} \celsius} \cdot \varphi_1}{\rho_{\text{нас.
    пара 50} \celsius}}= \varphi_1 \cdot \frac{\rho_{\text{нас.
    пара 15} \celsius}}{\rho_{\text{нас.
    пара 50} \celsius}} = 0{,}60 \cdot \frac{12{,}800\,\frac{\text{г}}{\text{м}^{3}}}{83{,}000\,\frac{\text{г}}{\text{м}^{3}}} = 0{,}093 \approx 9{,}3\%.
    \end{align*}
}
\solutionspace{80pt}

\tasknumber{3}%
\task{%
    Из уравнения состояния идеального газа выведите или выразите...
    \begin{enumerate}
        \item давление,
        \item температуру,
        \item концентрацию молекул газа.
    \end{enumerate}
}

\tasknumber{4}%
\task{%
    Запишите формулы и рядом с каждой физичической величиной укажите её название и единицы измерения в СИ:
    \begin{enumerate}
        \item первое начало термодинамики,
        \item внутренняя энергия идеального одноатомного газа.
    \end{enumerate}
}

\variantsplitter

\addpersonalvariant{Тимофей Полетаев}

\tasknumber{1}%
\task{%
    Определите КПД (оставив ответ точным в виде нескоратимой дроби) цикла 1231, рабочим телом которого является идеальный одноатомный газ, если
    \begin{itemize}
        \item 12 — изохорический нагрев в два раза,
        \item 23 — изобарическое расширение, при котором температура растёт в два раза,
        \item 31 — процесс, график которого в $PV$-координатах является отрезком прямой.
    \end{itemize}
    Бонус: замените цикл 1231 циклом, в котором 12 — изохорический нагрев в два раза, 23 — процесс, график которого в $PV$-координатах является отрезком прямой, 31 — изобарическое охлаждение, при котором температура падает в два раза.
}
\answer{%
    \begin{align*}
    A_{12} &= 0, \Delta U_{12} > 0, \implies Q_{12} = A_{12} + \Delta U_{12} > 0.
    \\
    A_{23} &> 0, \Delta U_{23} > 0, \implies Q_{23} = A_{23} + \Delta U_{23} > 0, \\
    A_{31} &= 0, \Delta U_{31} < 0, \implies Q_{31} = A_{31} + \Delta U_{31} < 0.
    \\
    P_1V_1 &= \nu R T_1, P_2V_2 = \nu R T_2, P_3V_3 = \nu R T_3 \text{ — уравнения состояния идеального газа}, \\
    &\text{Пусть $P_0$, $V_0$, $T_0$ — давление, объём и температура в точке 1 (минимальные во всём цикле):} \\
    P_1 &= P_0, P_2 = P_3, V_1 = V_2 = V_0, \text{остальные соотношения нужно считать} \\
    T_2 &= 2T_1 = 2T_0 \text{(по условию)} \implies \frac{P_2}{P_1} = \frac{P_2V_0}{P_1V_0} = \frac{P_2 V_2}{P_1 V_1}= \frac{\nu R T_2}{\nu R T_1} = \frac{T_2}{T_1} = 2 \implies P_2 = 2 P_1 = 2 P_0, \\
    T_3 &= 2T_2 = 4T_0 \text{(по условию)} \implies \frac{V_3}{V_2} = \frac{P_3V_3}{P_2V_2}= \frac{\nu R T_3}{\nu R T_2} = \frac{T_3}{T_2} = 2 \implies V_3 = 2 V_2 = 2 V_0.
    \\
    A_\text{цикл} &= \frac 12 (2P_0 - P_0)(2V_0 - V_0) = \frac 12 \cdot 1 \cdot P_0V_0, \\
    A_{23} &= 2P_0 \cdot (2V_0 - V_0) = 2P_0V_0, \\
    \Delta U_{23} &= \frac 32 \nu R T_3 - \frac 32 \nu R T_2 = \frac 32 P_3 V_3 - \frac 32 P_2 V_2 = \frac 32 \cdot 2 P_0 \cdot 2 V_0 -  \frac 32 \cdot 2 P_0 \cdot V_0 = \frac 32 \cdot 2 \cdot P_0V_0, \\
    \Delta U_{12} &= \frac 32 \nu R T_2 - \frac 32 \nu R T_1 = \frac 32 P_2 V_2 - \frac 32 P_1 V_1 = \frac 32 \cdot 2 P_0V_0 - \frac 32 P_0V_0 = \frac 32 \cdot 1 \cdot P_0V_0.
    \\
    \eta &= \frac{A_\text{цикл}}{Q_+} = \frac{A_\text{цикл}}{Q_{12} + Q_{23}}  = \frac{A_\text{цикл}}{A_{12} + \Delta U_{12} + A_{23} + \Delta U_{23}} =  \\
     &= \frac{\frac 12 \cdot 1 \cdot P_0V_0}{0 + \frac 32 \cdot 1 \cdot P_0V_0 + 2P_0V_0 + \frac 32 \cdot 2 \cdot P_0V_0} = \frac{\frac 12 \cdot 1}{\frac 32 \cdot 1 + 2 + \frac 32 \cdot 2} = \frac1{13} \approx 0.077.
    \end{align*}


        График процесса не в масштабе (эта часть пока не готова и сделать автоматически аккуратно сложно), но с верными подписями (а для решения этого достаточно):

        \begin{tikzpicture}[thick]
            \draw[-{Latex}] (0, 0) -- (0, 7) node[above left] {$P$};
            \draw[-{Latex}] (0, 0) -- (10, 0) node[right] {$V$};

            \draw[dashed] (0, 2) node[left] {$P_1 = P_0$} -| (3, 0) node[below] {$V_1 = V_2 = V_0$};
            \draw[dashed] (0, 6) node[left] {$P_2 = P_3 = 2P_0$} -| (9, 0) node[below] {$V_3 = 2V_0$};

            \draw (3, 2) node[above left]{1} node[below left]{$T_1 = T_0$}
                   (3, 6) node[below left]{2} node[above]{$T_2 = 2T_0$}
                   (9, 6) node[above right]{3} node[below right]{$T_3 = 4T_0$};
            \draw[midar] (3, 2) -- (3, 6);
            \draw[midar] (3, 6) -- (9, 6);
            \draw[midar] (9, 6) -- (3, 2);
        \end{tikzpicture}

        Решение бонуса:
        \begin{align*}
            A_{12} &= 0, \Delta U_{12} > 0, \implies Q_{12} = A_{12} + \Delta U_{12} > 0, \\
            A_{23} &> 0, \Delta U_{23} \text{ — ничего нельзя сказать, нужно исследовать отдельно}, \\
            A_{31} &< 0, \Delta U_{31} < 0, \implies Q_{31} = A_{31} + \Delta U_{31} < 0.
            \\
        \end{align*}

        Уравнения состояния идеального газа для точек 1, 2, 3: $P_1V_1 = \nu R T_1, P_2V_2 = \nu R T_2, P_3V_3 = \nu R T_3$.
        Пусть $P_0$, $V_0$, $T_0$ — давление, объём и температура в точке 1 (минимальные во всём цикле).

        12 --- изохора, $\frac{P_1V_1}{T_1} = \nu R = \frac{P_2V_2}{T_2}, V_2=V_1=V_0 \implies \frac{P_1}{T_1} =  \frac{P_2}{T_2} \implies P_2 = P_1 \frac{T_2}{T_1} = 2P_0$,

        31 --- изобара, $\frac{P_1V_1}{T_1} = \nu R = \frac{P_3V_3}{T_3}, P_3=P_1=P_0 \implies \frac{V_3}{T_3} =  \frac{V_1}{T_1} \implies V_3 = V_1 \frac{T_3}{T_1} = 2V_0$,

        Таким образом, используя новые обозначения, состояния газа в точках 1, 2 и 3 описываются макропараметрами $(P_0, V_0, T_0), (2P_0, V_0, 2T_0), (P_0, 2V_0, 2T_0)$ соответственно.

        \begin{tikzpicture}[thick]
            \draw[-{Latex}] (0, 0) -- (0, 7) node[above left] {$P$};
            \draw[-{Latex}] (0, 0) -- (10, 0) node[right] {$V$};

            \draw[dashed] (0, 2) node[left] {$P_1 = P_3 = P_0$} -| (9, 0) node[below] {$V_3 = 2V_0$};
            \draw[dashed] (0, 6) node[left] {$P_2 = 2P_0$} -| (3, 0) node[below] {$V_1 = V_2 = V_0$};

            \draw[dashed] (0, 5) node[left] {$P$} -| (4.5, 0) node[below] {$V$};
            \draw[dashed] (0, 4.6) node[left] {$P'$} -| (5.1, 0) node[below] {$V'$};

            \draw (3, 2) node[above left]{1} node[below left]{$T_1 = T_0$}
                   (3, 6) node[below left]{2} node[above]{$T_2 = 2T_0$}
                   (9, 2) node[above right]{3} node[below right]{$T_3 = 2T_0$};
            \draw[midar] (3, 2) -- (3, 6);
            \draw[midar] (3, 6) -- (9, 2);
            \draw[midar] (9, 2) -- (3, 2);
            \draw   (4.5, 5) node[above right]{$T$} (5.1, 4.6) node[above right]{$T'$};
        \end{tikzpicture}


        Теперь рассмотрим отдельно процесс 23, к остальному вернёмся позже.
        Уравнение этой прямой в $PV$-координатах: $P(V) = 3P_0 - \frac{P_0}{V_0} V$.
        Это значит, что при изменении объёма на $\Delta V$ давление изменится на $\Delta P = - \frac{P_0}{V_0} \Delta V$, обратите внимание на знак.

        Рассмотрим произвольную точку в процессе 23 и дадим процессу ещё немного свершиться, при этом объём изменится на $\Delta V$, давление — на $\Delta P$, температура (иначе бы была гипербола, а не прямая) — на $\Delta T$,
        т.е.
        из состояния $(P, V, T)$ мы перешли в $(P', V', T')$, причём  $P' = P + \Delta P, V' = V + \Delta V, T' = T + \Delta T$.

        При этом изменится внутренняя энергия:
        \begin{align*}
        \Delta U
            &= U' - U = \frac 32 \nu R T' - \frac 32 \nu R T = \frac 32 (P+\Delta P) (V+\Delta V) - \frac 32 PV\\
            &= \frac 32 ((P+\Delta P) (V+\Delta V) - PV) = \frac 32 (P\Delta V + V \Delta P + \Delta P \Delta V).
        \end{align*}

        Рассмотрим малые изменения объёма, тогда и изменение давления будем малым (т.к.
        $\Delta P = - \frac{P_0}{V_0} \Delta V$),
        а третьим слагаемым в выражении для $\Delta U$  можно пренебречь по сравнению с двумя другими:
        два первых это малые величины, а третье — произведение двух малых.
        Тогда $\Delta U = \frac 32 (P\Delta V + V \Delta P)$.

        Работа газа при этом малом изменении объёма — это площадь трапеции (тут ещё раз пренебрегли малым слагаемым):
        $$A = \frac{P + P'}2 \Delta V = \cbr{P + \frac{\Delta P}2} \Delta V = P \Delta V.$$

        Подведённое количество теплоты, используя первое начало термодинамики, будет равно
        \begin{align*}
        Q
            &= \frac 32 (P\Delta V + V \Delta P) + P \Delta V =  \frac 52 P\Delta V + \frac 32 V \Delta P = \\
            &= \frac 52 P\Delta V + \frac 32 V \cdot \cbr{- \frac{P_0}{V_0} \Delta V} = \frac{\Delta V}2 \cdot \cbr{5P - \frac{P_0}{V_0} V} = \\
            &= \frac{\Delta V}2 \cdot \cbr{5 \cdot \cbr{3P_0 - \frac{P_0}{V_0} V} - \frac{P_0}{V_0} V}
             = \frac{\Delta V \cdot P_0}2 \cdot \cbr{5 \cdot 3 - 8\frac V{V_0}}.
        \end{align*}

        Таком образом, знак количества теплоты $Q$ на участке 23 зависит от конкретного значения $V$:
        \begin{itemize}
            \item $\Delta V > 0$ на всём участке 23, поскольку газ расширяется,
            \item $P > 0$ — всегда, у нас идеальный газ, удары о стенки сосуда абсолютно упругие, а молекулы не взаимодействуют и поэтому давление только положительно,
            \item если $5 \cdot 3 - 8\frac V{V_0} > 0$ — тепло подводят, если же меньше нуля — отводят.
        \end{itemize}
        Решая последнее неравенство, получаем конкретное значение $V^*$: при $V < V^*$ тепло подводят, далее~— отводят.
        Тут *~--- некоторая точка между точками 2 и 3, конкретные значения надо досчитать:
        $$V^* = V_0 \cdot \frac{5 \cdot 3}8 = \frac{15}8 \cdot V_0 \implies P^* = 3P_0 - \frac{P_0}{V_0} V^* = \ldots = \frac98 \cdot P_0.$$

        Т.е.
        чтобы вычислить $Q_+$, надо сложить количества теплоты на участке 12 и лишь части участка 23 — участке 2*,
        той его части где это количество теплоты положительно.
        Имеем: $Q_+ = Q_{12} + Q_{2*}$.

        Теперь возвращаемся к циклу целиком и получаем:
        \begin{align*}
        A_\text{цикл} &= \frac 12 \cdot (2P_0 - P_0) \cdot (2V_0 - V_0) = \frac12 \cdot P_0V_0, \\
        A_{2*} &= \frac{P^* + 2P_0}2 \cdot (V^* - V_0)
            = \frac{\frac98 \cdot P_0 + 2P_0}2 \cdot \cbr{\frac{15}8 \cdot V_0 - V_0}
            = \ldots = \frac{175}{128} \cdot P_0 V_0, \\
        \Delta U_{2*} &= \frac 32 \nu R T^* - \frac 32 \nu R T_2 = \frac 32 (P^*V^* - P_0 \cdot 2V_0)
            = \frac 32 \cbr{\frac98 \cdot P_0 \cdot \frac{15}8 \cdot V_0 - P_0 \cdot 2V_0}
            = \frac{21}{128} \cdot P_0 V_0, \\
        \Delta U_{12} &= \frac 32 \nu R T_2 - \frac 32 \nu R T_1 = \frac 32 (2P_0V_0 - P_0V_0) = \ldots = \frac32 \cdot P_0 V_0, \\
        \eta &= \frac{A_\text{цикл}}{Q_+} = \frac{A_\text{цикл}}{Q_{12} + Q_{2*}}
            = \frac{A_\text{цикл}}{A_{12} + \Delta U_{12} + A_{2*} + \Delta U_{2*}} = \\
            &= \frac{\frac12 \cdot P_0V_0}{0 + \frac32 \cdot P_0 V_0 + \frac{175}{128} \cdot P_0 V_0 + \frac{21}{128} \cdot P_0 V_0}
             = \frac{A_bonus_cycle:LaTeX}{\frac32 + \frac{175}{128} + \frac{21}{128}}
             = \frac{16}{97} \leftarrow \text{вжух и готово!}
        \end{align*}
}
\solutionspace{360pt}

\tasknumber{2}%
\task{%
    При температуре $25\celsius$ относительная влажность воздуха составляет $65\%$.
    \begin{itemize}
        \item Определите точку росы для этого воздуха.
        \item Какой станет относительная влажность этого воздуха, если нагреть его до $40\celsius$?
    \end{itemize}
}
\answer{%
    \begin{align*}
    &\text{Значения плотности насыщенного водяного пара определяем по таблице:} \\
    &\rho_{\text{нас.
    пара 25} \celsius} = 23{,}000\,\frac{\text{г}}{\text{м}^{3}}, \rho_{\text{нас.
    пара 40} \celsius} = 51{,}200\,\frac{\text{г}}{\text{м}^{3}}.
    \\
    \varphi_1 &= \frac{\rho_\text{пара}}{\rho_{\text{нас.
    пара 25} \celsius}} \implies {\rho_\text{пара}} = \rho_{\text{нас.
    пара 25} \celsius} \cdot \varphi_1 = 23{,}000\,\frac{\text{г}}{\text{м}^{3}} \cdot 0{,}65 = 14{,}950\,\frac{\text{г}}{\text{м}^{3}}.
    \\
    &\text{По таблице определяем, при какой температуре пар с такой плотностью станет насыщенным:}  \\
    t_\text{росы} &= 17{,}5\celsius, \\
    \varphi_2 &= \frac{\rho_\text{пара}}{\rho_{\text{нас.
    пара 40} \celsius}} = \frac{\rho_{\text{нас.
    пара 25} \celsius} \cdot \varphi_1}{\rho_{\text{нас.
    пара 40} \celsius}}= \varphi_1 \cdot \frac{\rho_{\text{нас.
    пара 25} \celsius}}{\rho_{\text{нас.
    пара 40} \celsius}} = 0{,}65 \cdot \frac{23{,}000\,\frac{\text{г}}{\text{м}^{3}}}{51{,}200\,\frac{\text{г}}{\text{м}^{3}}} = 0{,}292 \approx 29{,}2\%.
    \end{align*}
}
\solutionspace{80pt}

\tasknumber{3}%
\task{%
    Из уравнения состояния идеального газа выведите или выразите...
    \begin{enumerate}
        \item объём,
        \item температуру,
        \item плотность газа.
    \end{enumerate}
}

\tasknumber{4}%
\task{%
    Запишите формулы и рядом с каждой физичической величиной укажите её название и единицы измерения в СИ:
    \begin{enumerate}
        \item первое начало термодинамики,
        \item внутренняя энергия идеального одноатомного газа.
    \end{enumerate}
}

\variantsplitter

\addpersonalvariant{Андрей Рожков}

\tasknumber{1}%
\task{%
    Определите КПД (оставив ответ точным в виде нескоратимой дроби) цикла 1231, рабочим телом которого является идеальный одноатомный газ, если
    \begin{itemize}
        \item 12 — изохорический нагрев в четыре раза,
        \item 23 — изобарическое расширение, при котором температура растёт в пять раз,
        \item 31 — процесс, график которого в $PV$-координатах является отрезком прямой.
    \end{itemize}
    Бонус: замените цикл 1231 циклом, в котором 12 — изохорический нагрев в четыре раза, 23 — процесс, график которого в $PV$-координатах является отрезком прямой, 31 — изобарическое охлаждение, при котором температура падает в четыре раза.
}
\answer{%
    \begin{align*}
    A_{12} &= 0, \Delta U_{12} > 0, \implies Q_{12} = A_{12} + \Delta U_{12} > 0.
    \\
    A_{23} &> 0, \Delta U_{23} > 0, \implies Q_{23} = A_{23} + \Delta U_{23} > 0, \\
    A_{31} &= 0, \Delta U_{31} < 0, \implies Q_{31} = A_{31} + \Delta U_{31} < 0.
    \\
    P_1V_1 &= \nu R T_1, P_2V_2 = \nu R T_2, P_3V_3 = \nu R T_3 \text{ — уравнения состояния идеального газа}, \\
    &\text{Пусть $P_0$, $V_0$, $T_0$ — давление, объём и температура в точке 1 (минимальные во всём цикле):} \\
    P_1 &= P_0, P_2 = P_3, V_1 = V_2 = V_0, \text{остальные соотношения нужно считать} \\
    T_2 &= 4T_1 = 4T_0 \text{(по условию)} \implies \frac{P_2}{P_1} = \frac{P_2V_0}{P_1V_0} = \frac{P_2 V_2}{P_1 V_1}= \frac{\nu R T_2}{\nu R T_1} = \frac{T_2}{T_1} = 4 \implies P_2 = 4 P_1 = 4 P_0, \\
    T_3 &= 5T_2 = 20T_0 \text{(по условию)} \implies \frac{V_3}{V_2} = \frac{P_3V_3}{P_2V_2}= \frac{\nu R T_3}{\nu R T_2} = \frac{T_3}{T_2} = 5 \implies V_3 = 5 V_2 = 5 V_0.
    \\
    A_\text{цикл} &= \frac 12 (5P_0 - P_0)(4V_0 - V_0) = \frac 12 \cdot 12 \cdot P_0V_0, \\
    A_{23} &= 4P_0 \cdot (5V_0 - V_0) = 16P_0V_0, \\
    \Delta U_{23} &= \frac 32 \nu R T_3 - \frac 32 \nu R T_2 = \frac 32 P_3 V_3 - \frac 32 P_2 V_2 = \frac 32 \cdot 4 P_0 \cdot 5 V_0 -  \frac 32 \cdot 4 P_0 \cdot V_0 = \frac 32 \cdot 16 \cdot P_0V_0, \\
    \Delta U_{12} &= \frac 32 \nu R T_2 - \frac 32 \nu R T_1 = \frac 32 P_2 V_2 - \frac 32 P_1 V_1 = \frac 32 \cdot 4 P_0V_0 - \frac 32 P_0V_0 = \frac 32 \cdot 3 \cdot P_0V_0.
    \\
    \eta &= \frac{A_\text{цикл}}{Q_+} = \frac{A_\text{цикл}}{Q_{12} + Q_{23}}  = \frac{A_\text{цикл}}{A_{12} + \Delta U_{12} + A_{23} + \Delta U_{23}} =  \\
     &= \frac{\frac 12 \cdot 12 \cdot P_0V_0}{0 + \frac 32 \cdot 3 \cdot P_0V_0 + 16P_0V_0 + \frac 32 \cdot 16 \cdot P_0V_0} = \frac{\frac 12 \cdot 12}{\frac 32 \cdot 3 + 16 + \frac 32 \cdot 16} = \frac{12}{89} \approx 0.135.
    \end{align*}


        График процесса не в масштабе (эта часть пока не готова и сделать автоматически аккуратно сложно), но с верными подписями (а для решения этого достаточно):

        \begin{tikzpicture}[thick]
            \draw[-{Latex}] (0, 0) -- (0, 7) node[above left] {$P$};
            \draw[-{Latex}] (0, 0) -- (10, 0) node[right] {$V$};

            \draw[dashed] (0, 2) node[left] {$P_1 = P_0$} -| (3, 0) node[below] {$V_1 = V_2 = V_0$};
            \draw[dashed] (0, 6) node[left] {$P_2 = P_3 = 4P_0$} -| (9, 0) node[below] {$V_3 = 5V_0$};

            \draw (3, 2) node[above left]{1} node[below left]{$T_1 = T_0$}
                   (3, 6) node[below left]{2} node[above]{$T_2 = 4T_0$}
                   (9, 6) node[above right]{3} node[below right]{$T_3 = 20T_0$};
            \draw[midar] (3, 2) -- (3, 6);
            \draw[midar] (3, 6) -- (9, 6);
            \draw[midar] (9, 6) -- (3, 2);
        \end{tikzpicture}

        Решение бонуса:
        \begin{align*}
            A_{12} &= 0, \Delta U_{12} > 0, \implies Q_{12} = A_{12} + \Delta U_{12} > 0, \\
            A_{23} &> 0, \Delta U_{23} \text{ — ничего нельзя сказать, нужно исследовать отдельно}, \\
            A_{31} &< 0, \Delta U_{31} < 0, \implies Q_{31} = A_{31} + \Delta U_{31} < 0.
            \\
        \end{align*}

        Уравнения состояния идеального газа для точек 1, 2, 3: $P_1V_1 = \nu R T_1, P_2V_2 = \nu R T_2, P_3V_3 = \nu R T_3$.
        Пусть $P_0$, $V_0$, $T_0$ — давление, объём и температура в точке 1 (минимальные во всём цикле).

        12 --- изохора, $\frac{P_1V_1}{T_1} = \nu R = \frac{P_2V_2}{T_2}, V_2=V_1=V_0 \implies \frac{P_1}{T_1} =  \frac{P_2}{T_2} \implies P_2 = P_1 \frac{T_2}{T_1} = 4P_0$,

        31 --- изобара, $\frac{P_1V_1}{T_1} = \nu R = \frac{P_3V_3}{T_3}, P_3=P_1=P_0 \implies \frac{V_3}{T_3} =  \frac{V_1}{T_1} \implies V_3 = V_1 \frac{T_3}{T_1} = 4V_0$,

        Таким образом, используя новые обозначения, состояния газа в точках 1, 2 и 3 описываются макропараметрами $(P_0, V_0, T_0), (4P_0, V_0, 4T_0), (P_0, 4V_0, 4T_0)$ соответственно.

        \begin{tikzpicture}[thick]
            \draw[-{Latex}] (0, 0) -- (0, 7) node[above left] {$P$};
            \draw[-{Latex}] (0, 0) -- (10, 0) node[right] {$V$};

            \draw[dashed] (0, 2) node[left] {$P_1 = P_3 = P_0$} -| (9, 0) node[below] {$V_3 = 4V_0$};
            \draw[dashed] (0, 6) node[left] {$P_2 = 4P_0$} -| (3, 0) node[below] {$V_1 = V_2 = V_0$};

            \draw[dashed] (0, 5) node[left] {$P$} -| (4.5, 0) node[below] {$V$};
            \draw[dashed] (0, 4.6) node[left] {$P'$} -| (5.1, 0) node[below] {$V'$};

            \draw (3, 2) node[above left]{1} node[below left]{$T_1 = T_0$}
                   (3, 6) node[below left]{2} node[above]{$T_2 = 4T_0$}
                   (9, 2) node[above right]{3} node[below right]{$T_3 = 4T_0$};
            \draw[midar] (3, 2) -- (3, 6);
            \draw[midar] (3, 6) -- (9, 2);
            \draw[midar] (9, 2) -- (3, 2);
            \draw   (4.5, 5) node[above right]{$T$} (5.1, 4.6) node[above right]{$T'$};
        \end{tikzpicture}


        Теперь рассмотрим отдельно процесс 23, к остальному вернёмся позже.
        Уравнение этой прямой в $PV$-координатах: $P(V) = 5P_0 - \frac{P_0}{V_0} V$.
        Это значит, что при изменении объёма на $\Delta V$ давление изменится на $\Delta P = - \frac{P_0}{V_0} \Delta V$, обратите внимание на знак.

        Рассмотрим произвольную точку в процессе 23 и дадим процессу ещё немного свершиться, при этом объём изменится на $\Delta V$, давление — на $\Delta P$, температура (иначе бы была гипербола, а не прямая) — на $\Delta T$,
        т.е.
        из состояния $(P, V, T)$ мы перешли в $(P', V', T')$, причём  $P' = P + \Delta P, V' = V + \Delta V, T' = T + \Delta T$.

        При этом изменится внутренняя энергия:
        \begin{align*}
        \Delta U
            &= U' - U = \frac 32 \nu R T' - \frac 32 \nu R T = \frac 32 (P+\Delta P) (V+\Delta V) - \frac 32 PV\\
            &= \frac 32 ((P+\Delta P) (V+\Delta V) - PV) = \frac 32 (P\Delta V + V \Delta P + \Delta P \Delta V).
        \end{align*}

        Рассмотрим малые изменения объёма, тогда и изменение давления будем малым (т.к.
        $\Delta P = - \frac{P_0}{V_0} \Delta V$),
        а третьим слагаемым в выражении для $\Delta U$  можно пренебречь по сравнению с двумя другими:
        два первых это малые величины, а третье — произведение двух малых.
        Тогда $\Delta U = \frac 32 (P\Delta V + V \Delta P)$.

        Работа газа при этом малом изменении объёма — это площадь трапеции (тут ещё раз пренебрегли малым слагаемым):
        $$A = \frac{P + P'}2 \Delta V = \cbr{P + \frac{\Delta P}2} \Delta V = P \Delta V.$$

        Подведённое количество теплоты, используя первое начало термодинамики, будет равно
        \begin{align*}
        Q
            &= \frac 32 (P\Delta V + V \Delta P) + P \Delta V =  \frac 52 P\Delta V + \frac 32 V \Delta P = \\
            &= \frac 52 P\Delta V + \frac 32 V \cdot \cbr{- \frac{P_0}{V_0} \Delta V} = \frac{\Delta V}2 \cdot \cbr{5P - \frac{P_0}{V_0} V} = \\
            &= \frac{\Delta V}2 \cdot \cbr{5 \cdot \cbr{5P_0 - \frac{P_0}{V_0} V} - \frac{P_0}{V_0} V}
             = \frac{\Delta V \cdot P_0}2 \cdot \cbr{5 \cdot 5 - 8\frac V{V_0}}.
        \end{align*}

        Таком образом, знак количества теплоты $Q$ на участке 23 зависит от конкретного значения $V$:
        \begin{itemize}
            \item $\Delta V > 0$ на всём участке 23, поскольку газ расширяется,
            \item $P > 0$ — всегда, у нас идеальный газ, удары о стенки сосуда абсолютно упругие, а молекулы не взаимодействуют и поэтому давление только положительно,
            \item если $5 \cdot 5 - 8\frac V{V_0} > 0$ — тепло подводят, если же меньше нуля — отводят.
        \end{itemize}
        Решая последнее неравенство, получаем конкретное значение $V^*$: при $V < V^*$ тепло подводят, далее~— отводят.
        Тут *~--- некоторая точка между точками 2 и 3, конкретные значения надо досчитать:
        $$V^* = V_0 \cdot \frac{5 \cdot 5}8 = \frac{25}8 \cdot V_0 \implies P^* = 5P_0 - \frac{P_0}{V_0} V^* = \ldots = \frac{15}8 \cdot P_0.$$

        Т.е.
        чтобы вычислить $Q_+$, надо сложить количества теплоты на участке 12 и лишь части участка 23 — участке 2*,
        той его части где это количество теплоты положительно.
        Имеем: $Q_+ = Q_{12} + Q_{2*}$.

        Теперь возвращаемся к циклу целиком и получаем:
        \begin{align*}
        A_\text{цикл} &= \frac 12 \cdot (4P_0 - P_0) \cdot (4V_0 - V_0) = \frac92 \cdot P_0V_0, \\
        A_{2*} &= \frac{P^* + 4P_0}2 \cdot (V^* - V_0)
            = \frac{\frac{15}8 \cdot P_0 + 4P_0}2 \cdot \cbr{\frac{25}8 \cdot V_0 - V_0}
            = \ldots = \frac{799}{128} \cdot P_0 V_0, \\
        \Delta U_{2*} &= \frac 32 \nu R T^* - \frac 32 \nu R T_2 = \frac 32 (P^*V^* - P_0 \cdot 4V_0)
            = \frac 32 \cbr{\frac{15}8 \cdot P_0 \cdot \frac{25}8 \cdot V_0 - P_0 \cdot 4V_0}
            = \frac{357}{128} \cdot P_0 V_0, \\
        \Delta U_{12} &= \frac 32 \nu R T_2 - \frac 32 \nu R T_1 = \frac 32 (4P_0V_0 - P_0V_0) = \ldots = \frac92 \cdot P_0 V_0, \\
        \eta &= \frac{A_\text{цикл}}{Q_+} = \frac{A_\text{цикл}}{Q_{12} + Q_{2*}}
            = \frac{A_\text{цикл}}{A_{12} + \Delta U_{12} + A_{2*} + \Delta U_{2*}} = \\
            &= \frac{\frac92 \cdot P_0V_0}{0 + \frac92 \cdot P_0 V_0 + \frac{799}{128} \cdot P_0 V_0 + \frac{357}{128} \cdot P_0 V_0}
             = \frac{A_bonus_cycle:LaTeX}{\frac92 + \frac{799}{128} + \frac{357}{128}}
             = \frac{144}{433} \leftarrow \text{вжух и готово!}
        \end{align*}
}
\solutionspace{360pt}

\tasknumber{2}%
\task{%
    При температуре $30\celsius$ относительная влажность воздуха составляет $60\%$.
    \begin{itemize}
        \item Определите точку росы для этого воздуха.
        \item Какой станет относительная влажность этого воздуха, если нагреть его до $70\celsius$?
    \end{itemize}
}
\answer{%
    \begin{align*}
    &\text{Значения плотности насыщенного водяного пара определяем по таблице:} \\
    &\rho_{\text{нас.
    пара 30} \celsius} = 30{,}300\,\frac{\text{г}}{\text{м}^{3}}, \rho_{\text{нас.
    пара 70} \celsius} = 198{,}000\,\frac{\text{г}}{\text{м}^{3}}.
    \\
    \varphi_1 &= \frac{\rho_\text{пара}}{\rho_{\text{нас.
    пара 30} \celsius}} \implies {\rho_\text{пара}} = \rho_{\text{нас.
    пара 30} \celsius} \cdot \varphi_1 = 30{,}300\,\frac{\text{г}}{\text{м}^{3}} \cdot 0{,}60 = 18{,}180\,\frac{\text{г}}{\text{м}^{3}}.
    \\
    &\text{По таблице определяем, при какой температуре пар с такой плотностью станет насыщенным:}  \\
    t_\text{росы} &= 20{,}9\celsius, \\
    \varphi_2 &= \frac{\rho_\text{пара}}{\rho_{\text{нас.
    пара 70} \celsius}} = \frac{\rho_{\text{нас.
    пара 30} \celsius} \cdot \varphi_1}{\rho_{\text{нас.
    пара 70} \celsius}}= \varphi_1 \cdot \frac{\rho_{\text{нас.
    пара 30} \celsius}}{\rho_{\text{нас.
    пара 70} \celsius}} = 0{,}60 \cdot \frac{30{,}300\,\frac{\text{г}}{\text{м}^{3}}}{198{,}000\,\frac{\text{г}}{\text{м}^{3}}} = 0{,}092 \approx 9{,}2\%.
    \end{align*}
}
\solutionspace{80pt}

\tasknumber{3}%
\task{%
    Из уравнения состояния идеального газа выведите или выразите...
    \begin{enumerate}
        \item давление,
        \item молярную массу,
        \item концентрацию молекул газа.
    \end{enumerate}
}

\tasknumber{4}%
\task{%
    Запишите формулы и рядом с каждой физичической величиной укажите её название и единицы измерения в СИ:
    \begin{enumerate}
        \item первое начало термодинамики,
        \item внутренняя энергия идеального одноатомного газа.
    \end{enumerate}
}

\variantsplitter

\addpersonalvariant{Рената Таржиманова}

\tasknumber{1}%
\task{%
    Определите КПД (оставив ответ точным в виде нескоратимой дроби) цикла 1231, рабочим телом которого является идеальный одноатомный газ, если
    \begin{itemize}
        \item 12 — изохорический нагрев в шесть раз,
        \item 23 — изобарическое расширение, при котором температура растёт в два раза,
        \item 31 — процесс, график которого в $PV$-координатах является отрезком прямой.
    \end{itemize}
    Бонус: замените цикл 1231 циклом, в котором 12 — изохорический нагрев в шесть раз, 23 — процесс, график которого в $PV$-координатах является отрезком прямой, 31 — изобарическое охлаждение, при котором температура падает в шесть раз.
}
\answer{%
    \begin{align*}
    A_{12} &= 0, \Delta U_{12} > 0, \implies Q_{12} = A_{12} + \Delta U_{12} > 0.
    \\
    A_{23} &> 0, \Delta U_{23} > 0, \implies Q_{23} = A_{23} + \Delta U_{23} > 0, \\
    A_{31} &= 0, \Delta U_{31} < 0, \implies Q_{31} = A_{31} + \Delta U_{31} < 0.
    \\
    P_1V_1 &= \nu R T_1, P_2V_2 = \nu R T_2, P_3V_3 = \nu R T_3 \text{ — уравнения состояния идеального газа}, \\
    &\text{Пусть $P_0$, $V_0$, $T_0$ — давление, объём и температура в точке 1 (минимальные во всём цикле):} \\
    P_1 &= P_0, P_2 = P_3, V_1 = V_2 = V_0, \text{остальные соотношения нужно считать} \\
    T_2 &= 6T_1 = 6T_0 \text{(по условию)} \implies \frac{P_2}{P_1} = \frac{P_2V_0}{P_1V_0} = \frac{P_2 V_2}{P_1 V_1}= \frac{\nu R T_2}{\nu R T_1} = \frac{T_2}{T_1} = 6 \implies P_2 = 6 P_1 = 6 P_0, \\
    T_3 &= 2T_2 = 12T_0 \text{(по условию)} \implies \frac{V_3}{V_2} = \frac{P_3V_3}{P_2V_2}= \frac{\nu R T_3}{\nu R T_2} = \frac{T_3}{T_2} = 2 \implies V_3 = 2 V_2 = 2 V_0.
    \\
    A_\text{цикл} &= \frac 12 (2P_0 - P_0)(6V_0 - V_0) = \frac 12 \cdot 5 \cdot P_0V_0, \\
    A_{23} &= 6P_0 \cdot (2V_0 - V_0) = 6P_0V_0, \\
    \Delta U_{23} &= \frac 32 \nu R T_3 - \frac 32 \nu R T_2 = \frac 32 P_3 V_3 - \frac 32 P_2 V_2 = \frac 32 \cdot 6 P_0 \cdot 2 V_0 -  \frac 32 \cdot 6 P_0 \cdot V_0 = \frac 32 \cdot 6 \cdot P_0V_0, \\
    \Delta U_{12} &= \frac 32 \nu R T_2 - \frac 32 \nu R T_1 = \frac 32 P_2 V_2 - \frac 32 P_1 V_1 = \frac 32 \cdot 6 P_0V_0 - \frac 32 P_0V_0 = \frac 32 \cdot 5 \cdot P_0V_0.
    \\
    \eta &= \frac{A_\text{цикл}}{Q_+} = \frac{A_\text{цикл}}{Q_{12} + Q_{23}}  = \frac{A_\text{цикл}}{A_{12} + \Delta U_{12} + A_{23} + \Delta U_{23}} =  \\
     &= \frac{\frac 12 \cdot 5 \cdot P_0V_0}{0 + \frac 32 \cdot 5 \cdot P_0V_0 + 6P_0V_0 + \frac 32 \cdot 6 \cdot P_0V_0} = \frac{\frac 12 \cdot 5}{\frac 32 \cdot 5 + 6 + \frac 32 \cdot 6} = \frac19 \approx 0.111.
    \end{align*}


        График процесса не в масштабе (эта часть пока не готова и сделать автоматически аккуратно сложно), но с верными подписями (а для решения этого достаточно):

        \begin{tikzpicture}[thick]
            \draw[-{Latex}] (0, 0) -- (0, 7) node[above left] {$P$};
            \draw[-{Latex}] (0, 0) -- (10, 0) node[right] {$V$};

            \draw[dashed] (0, 2) node[left] {$P_1 = P_0$} -| (3, 0) node[below] {$V_1 = V_2 = V_0$};
            \draw[dashed] (0, 6) node[left] {$P_2 = P_3 = 6P_0$} -| (9, 0) node[below] {$V_3 = 2V_0$};

            \draw (3, 2) node[above left]{1} node[below left]{$T_1 = T_0$}
                   (3, 6) node[below left]{2} node[above]{$T_2 = 6T_0$}
                   (9, 6) node[above right]{3} node[below right]{$T_3 = 12T_0$};
            \draw[midar] (3, 2) -- (3, 6);
            \draw[midar] (3, 6) -- (9, 6);
            \draw[midar] (9, 6) -- (3, 2);
        \end{tikzpicture}

        Решение бонуса:
        \begin{align*}
            A_{12} &= 0, \Delta U_{12} > 0, \implies Q_{12} = A_{12} + \Delta U_{12} > 0, \\
            A_{23} &> 0, \Delta U_{23} \text{ — ничего нельзя сказать, нужно исследовать отдельно}, \\
            A_{31} &< 0, \Delta U_{31} < 0, \implies Q_{31} = A_{31} + \Delta U_{31} < 0.
            \\
        \end{align*}

        Уравнения состояния идеального газа для точек 1, 2, 3: $P_1V_1 = \nu R T_1, P_2V_2 = \nu R T_2, P_3V_3 = \nu R T_3$.
        Пусть $P_0$, $V_0$, $T_0$ — давление, объём и температура в точке 1 (минимальные во всём цикле).

        12 --- изохора, $\frac{P_1V_1}{T_1} = \nu R = \frac{P_2V_2}{T_2}, V_2=V_1=V_0 \implies \frac{P_1}{T_1} =  \frac{P_2}{T_2} \implies P_2 = P_1 \frac{T_2}{T_1} = 6P_0$,

        31 --- изобара, $\frac{P_1V_1}{T_1} = \nu R = \frac{P_3V_3}{T_3}, P_3=P_1=P_0 \implies \frac{V_3}{T_3} =  \frac{V_1}{T_1} \implies V_3 = V_1 \frac{T_3}{T_1} = 6V_0$,

        Таким образом, используя новые обозначения, состояния газа в точках 1, 2 и 3 описываются макропараметрами $(P_0, V_0, T_0), (6P_0, V_0, 6T_0), (P_0, 6V_0, 6T_0)$ соответственно.

        \begin{tikzpicture}[thick]
            \draw[-{Latex}] (0, 0) -- (0, 7) node[above left] {$P$};
            \draw[-{Latex}] (0, 0) -- (10, 0) node[right] {$V$};

            \draw[dashed] (0, 2) node[left] {$P_1 = P_3 = P_0$} -| (9, 0) node[below] {$V_3 = 6V_0$};
            \draw[dashed] (0, 6) node[left] {$P_2 = 6P_0$} -| (3, 0) node[below] {$V_1 = V_2 = V_0$};

            \draw[dashed] (0, 5) node[left] {$P$} -| (4.5, 0) node[below] {$V$};
            \draw[dashed] (0, 4.6) node[left] {$P'$} -| (5.1, 0) node[below] {$V'$};

            \draw (3, 2) node[above left]{1} node[below left]{$T_1 = T_0$}
                   (3, 6) node[below left]{2} node[above]{$T_2 = 6T_0$}
                   (9, 2) node[above right]{3} node[below right]{$T_3 = 6T_0$};
            \draw[midar] (3, 2) -- (3, 6);
            \draw[midar] (3, 6) -- (9, 2);
            \draw[midar] (9, 2) -- (3, 2);
            \draw   (4.5, 5) node[above right]{$T$} (5.1, 4.6) node[above right]{$T'$};
        \end{tikzpicture}


        Теперь рассмотрим отдельно процесс 23, к остальному вернёмся позже.
        Уравнение этой прямой в $PV$-координатах: $P(V) = 7P_0 - \frac{P_0}{V_0} V$.
        Это значит, что при изменении объёма на $\Delta V$ давление изменится на $\Delta P = - \frac{P_0}{V_0} \Delta V$, обратите внимание на знак.

        Рассмотрим произвольную точку в процессе 23 и дадим процессу ещё немного свершиться, при этом объём изменится на $\Delta V$, давление — на $\Delta P$, температура (иначе бы была гипербола, а не прямая) — на $\Delta T$,
        т.е.
        из состояния $(P, V, T)$ мы перешли в $(P', V', T')$, причём  $P' = P + \Delta P, V' = V + \Delta V, T' = T + \Delta T$.

        При этом изменится внутренняя энергия:
        \begin{align*}
        \Delta U
            &= U' - U = \frac 32 \nu R T' - \frac 32 \nu R T = \frac 32 (P+\Delta P) (V+\Delta V) - \frac 32 PV\\
            &= \frac 32 ((P+\Delta P) (V+\Delta V) - PV) = \frac 32 (P\Delta V + V \Delta P + \Delta P \Delta V).
        \end{align*}

        Рассмотрим малые изменения объёма, тогда и изменение давления будем малым (т.к.
        $\Delta P = - \frac{P_0}{V_0} \Delta V$),
        а третьим слагаемым в выражении для $\Delta U$  можно пренебречь по сравнению с двумя другими:
        два первых это малые величины, а третье — произведение двух малых.
        Тогда $\Delta U = \frac 32 (P\Delta V + V \Delta P)$.

        Работа газа при этом малом изменении объёма — это площадь трапеции (тут ещё раз пренебрегли малым слагаемым):
        $$A = \frac{P + P'}2 \Delta V = \cbr{P + \frac{\Delta P}2} \Delta V = P \Delta V.$$

        Подведённое количество теплоты, используя первое начало термодинамики, будет равно
        \begin{align*}
        Q
            &= \frac 32 (P\Delta V + V \Delta P) + P \Delta V =  \frac 52 P\Delta V + \frac 32 V \Delta P = \\
            &= \frac 52 P\Delta V + \frac 32 V \cdot \cbr{- \frac{P_0}{V_0} \Delta V} = \frac{\Delta V}2 \cdot \cbr{5P - \frac{P_0}{V_0} V} = \\
            &= \frac{\Delta V}2 \cdot \cbr{5 \cdot \cbr{7P_0 - \frac{P_0}{V_0} V} - \frac{P_0}{V_0} V}
             = \frac{\Delta V \cdot P_0}2 \cdot \cbr{5 \cdot 7 - 8\frac V{V_0}}.
        \end{align*}

        Таком образом, знак количества теплоты $Q$ на участке 23 зависит от конкретного значения $V$:
        \begin{itemize}
            \item $\Delta V > 0$ на всём участке 23, поскольку газ расширяется,
            \item $P > 0$ — всегда, у нас идеальный газ, удары о стенки сосуда абсолютно упругие, а молекулы не взаимодействуют и поэтому давление только положительно,
            \item если $5 \cdot 7 - 8\frac V{V_0} > 0$ — тепло подводят, если же меньше нуля — отводят.
        \end{itemize}
        Решая последнее неравенство, получаем конкретное значение $V^*$: при $V < V^*$ тепло подводят, далее~— отводят.
        Тут *~--- некоторая точка между точками 2 и 3, конкретные значения надо досчитать:
        $$V^* = V_0 \cdot \frac{5 \cdot 7}8 = \frac{35}8 \cdot V_0 \implies P^* = 7P_0 - \frac{P_0}{V_0} V^* = \ldots = \frac{21}8 \cdot P_0.$$

        Т.е.
        чтобы вычислить $Q_+$, надо сложить количества теплоты на участке 12 и лишь части участка 23 — участке 2*,
        той его части где это количество теплоты положительно.
        Имеем: $Q_+ = Q_{12} + Q_{2*}$.

        Теперь возвращаемся к циклу целиком и получаем:
        \begin{align*}
        A_\text{цикл} &= \frac 12 \cdot (6P_0 - P_0) \cdot (6V_0 - V_0) = \frac{25}2 \cdot P_0V_0, \\
        A_{2*} &= \frac{P^* + 6P_0}2 \cdot (V^* - V_0)
            = \frac{\frac{21}8 \cdot P_0 + 6P_0}2 \cdot \cbr{\frac{35}8 \cdot V_0 - V_0}
            = \ldots = \frac{1863}{128} \cdot P_0 V_0, \\
        \Delta U_{2*} &= \frac 32 \nu R T^* - \frac 32 \nu R T_2 = \frac 32 (P^*V^* - P_0 \cdot 6V_0)
            = \frac 32 \cbr{\frac{21}8 \cdot P_0 \cdot \frac{35}8 \cdot V_0 - P_0 \cdot 6V_0}
            = \frac{1053}{128} \cdot P_0 V_0, \\
        \Delta U_{12} &= \frac 32 \nu R T_2 - \frac 32 \nu R T_1 = \frac 32 (6P_0V_0 - P_0V_0) = \ldots = \frac{15}2 \cdot P_0 V_0, \\
        \eta &= \frac{A_\text{цикл}}{Q_+} = \frac{A_\text{цикл}}{Q_{12} + Q_{2*}}
            = \frac{A_\text{цикл}}{A_{12} + \Delta U_{12} + A_{2*} + \Delta U_{2*}} = \\
            &= \frac{\frac{25}2 \cdot P_0V_0}{0 + \frac{15}2 \cdot P_0 V_0 + \frac{1863}{128} \cdot P_0 V_0 + \frac{1053}{128} \cdot P_0 V_0}
             = \frac{A_bonus_cycle:LaTeX}{\frac{15}2 + \frac{1863}{128} + \frac{1053}{128}}
             = \frac{400}{969} \leftarrow \text{вжух и готово!}
        \end{align*}
}
\solutionspace{360pt}

\tasknumber{2}%
\task{%
    При температуре $15\celsius$ относительная влажность воздуха составляет $70\%$.
    \begin{itemize}
        \item Определите точку росы для этого воздуха.
        \item Какой станет относительная влажность этого воздуха, если нагреть его до $70\celsius$?
    \end{itemize}
}
\answer{%
    \begin{align*}
    &\text{Значения плотности насыщенного водяного пара определяем по таблице:} \\
    &\rho_{\text{нас.
    пара 15} \celsius} = 12{,}800\,\frac{\text{г}}{\text{м}^{3}}, \rho_{\text{нас.
    пара 70} \celsius} = 198{,}000\,\frac{\text{г}}{\text{м}^{3}}.
    \\
    \varphi_1 &= \frac{\rho_\text{пара}}{\rho_{\text{нас.
    пара 15} \celsius}} \implies {\rho_\text{пара}} = \rho_{\text{нас.
    пара 15} \celsius} \cdot \varphi_1 = 12{,}800\,\frac{\text{г}}{\text{м}^{3}} \cdot 0{,}70 = 8{,}960\,\frac{\text{г}}{\text{м}^{3}}.
    \\
    &\text{По таблице определяем, при какой температуре пар с такой плотностью станет насыщенным:}  \\
    t_\text{росы} &= 9{,}3\celsius, \\
    \varphi_2 &= \frac{\rho_\text{пара}}{\rho_{\text{нас.
    пара 70} \celsius}} = \frac{\rho_{\text{нас.
    пара 15} \celsius} \cdot \varphi_1}{\rho_{\text{нас.
    пара 70} \celsius}}= \varphi_1 \cdot \frac{\rho_{\text{нас.
    пара 15} \celsius}}{\rho_{\text{нас.
    пара 70} \celsius}} = 0{,}70 \cdot \frac{12{,}800\,\frac{\text{г}}{\text{м}^{3}}}{198{,}000\,\frac{\text{г}}{\text{м}^{3}}} = 0{,}045 \approx 4{,}5\%.
    \end{align*}
}
\solutionspace{80pt}

\tasknumber{3}%
\task{%
    Из уравнения состояния идеального газа выведите или выразите...
    \begin{enumerate}
        \item давление,
        \item температуру,
        \item плотность газа.
    \end{enumerate}
}

\tasknumber{4}%
\task{%
    Запишите формулы и рядом с каждой физичической величиной укажите её название и единицы измерения в СИ:
    \begin{enumerate}
        \item первое начало термодинамики,
        \item внутренняя энергия идеального одноатомного газа.
    \end{enumerate}
}

\variantsplitter

\addpersonalvariant{Андрей Щербаков}

\tasknumber{1}%
\task{%
    Определите КПД (оставив ответ точным в виде нескоратимой дроби) цикла 1231, рабочим телом которого является идеальный одноатомный газ, если
    \begin{itemize}
        \item 12 — изохорический нагрев в два раза,
        \item 23 — изобарическое расширение, при котором температура растёт в два раза,
        \item 31 — процесс, график которого в $PV$-координатах является отрезком прямой.
    \end{itemize}
    Бонус: замените цикл 1231 циклом, в котором 12 — изохорический нагрев в два раза, 23 — процесс, график которого в $PV$-координатах является отрезком прямой, 31 — изобарическое охлаждение, при котором температура падает в два раза.
}
\answer{%
    \begin{align*}
    A_{12} &= 0, \Delta U_{12} > 0, \implies Q_{12} = A_{12} + \Delta U_{12} > 0.
    \\
    A_{23} &> 0, \Delta U_{23} > 0, \implies Q_{23} = A_{23} + \Delta U_{23} > 0, \\
    A_{31} &= 0, \Delta U_{31} < 0, \implies Q_{31} = A_{31} + \Delta U_{31} < 0.
    \\
    P_1V_1 &= \nu R T_1, P_2V_2 = \nu R T_2, P_3V_3 = \nu R T_3 \text{ — уравнения состояния идеального газа}, \\
    &\text{Пусть $P_0$, $V_0$, $T_0$ — давление, объём и температура в точке 1 (минимальные во всём цикле):} \\
    P_1 &= P_0, P_2 = P_3, V_1 = V_2 = V_0, \text{остальные соотношения нужно считать} \\
    T_2 &= 2T_1 = 2T_0 \text{(по условию)} \implies \frac{P_2}{P_1} = \frac{P_2V_0}{P_1V_0} = \frac{P_2 V_2}{P_1 V_1}= \frac{\nu R T_2}{\nu R T_1} = \frac{T_2}{T_1} = 2 \implies P_2 = 2 P_1 = 2 P_0, \\
    T_3 &= 2T_2 = 4T_0 \text{(по условию)} \implies \frac{V_3}{V_2} = \frac{P_3V_3}{P_2V_2}= \frac{\nu R T_3}{\nu R T_2} = \frac{T_3}{T_2} = 2 \implies V_3 = 2 V_2 = 2 V_0.
    \\
    A_\text{цикл} &= \frac 12 (2P_0 - P_0)(2V_0 - V_0) = \frac 12 \cdot 1 \cdot P_0V_0, \\
    A_{23} &= 2P_0 \cdot (2V_0 - V_0) = 2P_0V_0, \\
    \Delta U_{23} &= \frac 32 \nu R T_3 - \frac 32 \nu R T_2 = \frac 32 P_3 V_3 - \frac 32 P_2 V_2 = \frac 32 \cdot 2 P_0 \cdot 2 V_0 -  \frac 32 \cdot 2 P_0 \cdot V_0 = \frac 32 \cdot 2 \cdot P_0V_0, \\
    \Delta U_{12} &= \frac 32 \nu R T_2 - \frac 32 \nu R T_1 = \frac 32 P_2 V_2 - \frac 32 P_1 V_1 = \frac 32 \cdot 2 P_0V_0 - \frac 32 P_0V_0 = \frac 32 \cdot 1 \cdot P_0V_0.
    \\
    \eta &= \frac{A_\text{цикл}}{Q_+} = \frac{A_\text{цикл}}{Q_{12} + Q_{23}}  = \frac{A_\text{цикл}}{A_{12} + \Delta U_{12} + A_{23} + \Delta U_{23}} =  \\
     &= \frac{\frac 12 \cdot 1 \cdot P_0V_0}{0 + \frac 32 \cdot 1 \cdot P_0V_0 + 2P_0V_0 + \frac 32 \cdot 2 \cdot P_0V_0} = \frac{\frac 12 \cdot 1}{\frac 32 \cdot 1 + 2 + \frac 32 \cdot 2} = \frac1{13} \approx 0.077.
    \end{align*}


        График процесса не в масштабе (эта часть пока не готова и сделать автоматически аккуратно сложно), но с верными подписями (а для решения этого достаточно):

        \begin{tikzpicture}[thick]
            \draw[-{Latex}] (0, 0) -- (0, 7) node[above left] {$P$};
            \draw[-{Latex}] (0, 0) -- (10, 0) node[right] {$V$};

            \draw[dashed] (0, 2) node[left] {$P_1 = P_0$} -| (3, 0) node[below] {$V_1 = V_2 = V_0$};
            \draw[dashed] (0, 6) node[left] {$P_2 = P_3 = 2P_0$} -| (9, 0) node[below] {$V_3 = 2V_0$};

            \draw (3, 2) node[above left]{1} node[below left]{$T_1 = T_0$}
                   (3, 6) node[below left]{2} node[above]{$T_2 = 2T_0$}
                   (9, 6) node[above right]{3} node[below right]{$T_3 = 4T_0$};
            \draw[midar] (3, 2) -- (3, 6);
            \draw[midar] (3, 6) -- (9, 6);
            \draw[midar] (9, 6) -- (3, 2);
        \end{tikzpicture}

        Решение бонуса:
        \begin{align*}
            A_{12} &= 0, \Delta U_{12} > 0, \implies Q_{12} = A_{12} + \Delta U_{12} > 0, \\
            A_{23} &> 0, \Delta U_{23} \text{ — ничего нельзя сказать, нужно исследовать отдельно}, \\
            A_{31} &< 0, \Delta U_{31} < 0, \implies Q_{31} = A_{31} + \Delta U_{31} < 0.
            \\
        \end{align*}

        Уравнения состояния идеального газа для точек 1, 2, 3: $P_1V_1 = \nu R T_1, P_2V_2 = \nu R T_2, P_3V_3 = \nu R T_3$.
        Пусть $P_0$, $V_0$, $T_0$ — давление, объём и температура в точке 1 (минимальные во всём цикле).

        12 --- изохора, $\frac{P_1V_1}{T_1} = \nu R = \frac{P_2V_2}{T_2}, V_2=V_1=V_0 \implies \frac{P_1}{T_1} =  \frac{P_2}{T_2} \implies P_2 = P_1 \frac{T_2}{T_1} = 2P_0$,

        31 --- изобара, $\frac{P_1V_1}{T_1} = \nu R = \frac{P_3V_3}{T_3}, P_3=P_1=P_0 \implies \frac{V_3}{T_3} =  \frac{V_1}{T_1} \implies V_3 = V_1 \frac{T_3}{T_1} = 2V_0$,

        Таким образом, используя новые обозначения, состояния газа в точках 1, 2 и 3 описываются макропараметрами $(P_0, V_0, T_0), (2P_0, V_0, 2T_0), (P_0, 2V_0, 2T_0)$ соответственно.

        \begin{tikzpicture}[thick]
            \draw[-{Latex}] (0, 0) -- (0, 7) node[above left] {$P$};
            \draw[-{Latex}] (0, 0) -- (10, 0) node[right] {$V$};

            \draw[dashed] (0, 2) node[left] {$P_1 = P_3 = P_0$} -| (9, 0) node[below] {$V_3 = 2V_0$};
            \draw[dashed] (0, 6) node[left] {$P_2 = 2P_0$} -| (3, 0) node[below] {$V_1 = V_2 = V_0$};

            \draw[dashed] (0, 5) node[left] {$P$} -| (4.5, 0) node[below] {$V$};
            \draw[dashed] (0, 4.6) node[left] {$P'$} -| (5.1, 0) node[below] {$V'$};

            \draw (3, 2) node[above left]{1} node[below left]{$T_1 = T_0$}
                   (3, 6) node[below left]{2} node[above]{$T_2 = 2T_0$}
                   (9, 2) node[above right]{3} node[below right]{$T_3 = 2T_0$};
            \draw[midar] (3, 2) -- (3, 6);
            \draw[midar] (3, 6) -- (9, 2);
            \draw[midar] (9, 2) -- (3, 2);
            \draw   (4.5, 5) node[above right]{$T$} (5.1, 4.6) node[above right]{$T'$};
        \end{tikzpicture}


        Теперь рассмотрим отдельно процесс 23, к остальному вернёмся позже.
        Уравнение этой прямой в $PV$-координатах: $P(V) = 3P_0 - \frac{P_0}{V_0} V$.
        Это значит, что при изменении объёма на $\Delta V$ давление изменится на $\Delta P = - \frac{P_0}{V_0} \Delta V$, обратите внимание на знак.

        Рассмотрим произвольную точку в процессе 23 и дадим процессу ещё немного свершиться, при этом объём изменится на $\Delta V$, давление — на $\Delta P$, температура (иначе бы была гипербола, а не прямая) — на $\Delta T$,
        т.е.
        из состояния $(P, V, T)$ мы перешли в $(P', V', T')$, причём  $P' = P + \Delta P, V' = V + \Delta V, T' = T + \Delta T$.

        При этом изменится внутренняя энергия:
        \begin{align*}
        \Delta U
            &= U' - U = \frac 32 \nu R T' - \frac 32 \nu R T = \frac 32 (P+\Delta P) (V+\Delta V) - \frac 32 PV\\
            &= \frac 32 ((P+\Delta P) (V+\Delta V) - PV) = \frac 32 (P\Delta V + V \Delta P + \Delta P \Delta V).
        \end{align*}

        Рассмотрим малые изменения объёма, тогда и изменение давления будем малым (т.к.
        $\Delta P = - \frac{P_0}{V_0} \Delta V$),
        а третьим слагаемым в выражении для $\Delta U$  можно пренебречь по сравнению с двумя другими:
        два первых это малые величины, а третье — произведение двух малых.
        Тогда $\Delta U = \frac 32 (P\Delta V + V \Delta P)$.

        Работа газа при этом малом изменении объёма — это площадь трапеции (тут ещё раз пренебрегли малым слагаемым):
        $$A = \frac{P + P'}2 \Delta V = \cbr{P + \frac{\Delta P}2} \Delta V = P \Delta V.$$

        Подведённое количество теплоты, используя первое начало термодинамики, будет равно
        \begin{align*}
        Q
            &= \frac 32 (P\Delta V + V \Delta P) + P \Delta V =  \frac 52 P\Delta V + \frac 32 V \Delta P = \\
            &= \frac 52 P\Delta V + \frac 32 V \cdot \cbr{- \frac{P_0}{V_0} \Delta V} = \frac{\Delta V}2 \cdot \cbr{5P - \frac{P_0}{V_0} V} = \\
            &= \frac{\Delta V}2 \cdot \cbr{5 \cdot \cbr{3P_0 - \frac{P_0}{V_0} V} - \frac{P_0}{V_0} V}
             = \frac{\Delta V \cdot P_0}2 \cdot \cbr{5 \cdot 3 - 8\frac V{V_0}}.
        \end{align*}

        Таком образом, знак количества теплоты $Q$ на участке 23 зависит от конкретного значения $V$:
        \begin{itemize}
            \item $\Delta V > 0$ на всём участке 23, поскольку газ расширяется,
            \item $P > 0$ — всегда, у нас идеальный газ, удары о стенки сосуда абсолютно упругие, а молекулы не взаимодействуют и поэтому давление только положительно,
            \item если $5 \cdot 3 - 8\frac V{V_0} > 0$ — тепло подводят, если же меньше нуля — отводят.
        \end{itemize}
        Решая последнее неравенство, получаем конкретное значение $V^*$: при $V < V^*$ тепло подводят, далее~— отводят.
        Тут *~--- некоторая точка между точками 2 и 3, конкретные значения надо досчитать:
        $$V^* = V_0 \cdot \frac{5 \cdot 3}8 = \frac{15}8 \cdot V_0 \implies P^* = 3P_0 - \frac{P_0}{V_0} V^* = \ldots = \frac98 \cdot P_0.$$

        Т.е.
        чтобы вычислить $Q_+$, надо сложить количества теплоты на участке 12 и лишь части участка 23 — участке 2*,
        той его части где это количество теплоты положительно.
        Имеем: $Q_+ = Q_{12} + Q_{2*}$.

        Теперь возвращаемся к циклу целиком и получаем:
        \begin{align*}
        A_\text{цикл} &= \frac 12 \cdot (2P_0 - P_0) \cdot (2V_0 - V_0) = \frac12 \cdot P_0V_0, \\
        A_{2*} &= \frac{P^* + 2P_0}2 \cdot (V^* - V_0)
            = \frac{\frac98 \cdot P_0 + 2P_0}2 \cdot \cbr{\frac{15}8 \cdot V_0 - V_0}
            = \ldots = \frac{175}{128} \cdot P_0 V_0, \\
        \Delta U_{2*} &= \frac 32 \nu R T^* - \frac 32 \nu R T_2 = \frac 32 (P^*V^* - P_0 \cdot 2V_0)
            = \frac 32 \cbr{\frac98 \cdot P_0 \cdot \frac{15}8 \cdot V_0 - P_0 \cdot 2V_0}
            = \frac{21}{128} \cdot P_0 V_0, \\
        \Delta U_{12} &= \frac 32 \nu R T_2 - \frac 32 \nu R T_1 = \frac 32 (2P_0V_0 - P_0V_0) = \ldots = \frac32 \cdot P_0 V_0, \\
        \eta &= \frac{A_\text{цикл}}{Q_+} = \frac{A_\text{цикл}}{Q_{12} + Q_{2*}}
            = \frac{A_\text{цикл}}{A_{12} + \Delta U_{12} + A_{2*} + \Delta U_{2*}} = \\
            &= \frac{\frac12 \cdot P_0V_0}{0 + \frac32 \cdot P_0 V_0 + \frac{175}{128} \cdot P_0 V_0 + \frac{21}{128} \cdot P_0 V_0}
             = \frac{A_bonus_cycle:LaTeX}{\frac32 + \frac{175}{128} + \frac{21}{128}}
             = \frac{16}{97} \leftarrow \text{вжух и готово!}
        \end{align*}
}
\solutionspace{360pt}

\tasknumber{2}%
\task{%
    При температуре $25\celsius$ относительная влажность воздуха составляет $75\%$.
    \begin{itemize}
        \item Определите точку росы для этого воздуха.
        \item Какой станет относительная влажность этого воздуха, если нагреть его до $60\celsius$?
    \end{itemize}
}
\answer{%
    \begin{align*}
    &\text{Значения плотности насыщенного водяного пара определяем по таблице:} \\
    &\rho_{\text{нас.
    пара 25} \celsius} = 23{,}000\,\frac{\text{г}}{\text{м}^{3}}, \rho_{\text{нас.
    пара 60} \celsius} = 130{,}000\,\frac{\text{г}}{\text{м}^{3}}.
    \\
    \varphi_1 &= \frac{\rho_\text{пара}}{\rho_{\text{нас.
    пара 25} \celsius}} \implies {\rho_\text{пара}} = \rho_{\text{нас.
    пара 25} \celsius} \cdot \varphi_1 = 23{,}000\,\frac{\text{г}}{\text{м}^{3}} \cdot 0{,}75 = 17{,}250\,\frac{\text{г}}{\text{м}^{3}}.
    \\
    &\text{По таблице определяем, при какой температуре пар с такой плотностью станет насыщенным:}  \\
    t_\text{росы} &= 19{,}9\celsius, \\
    \varphi_2 &= \frac{\rho_\text{пара}}{\rho_{\text{нас.
    пара 60} \celsius}} = \frac{\rho_{\text{нас.
    пара 25} \celsius} \cdot \varphi_1}{\rho_{\text{нас.
    пара 60} \celsius}}= \varphi_1 \cdot \frac{\rho_{\text{нас.
    пара 25} \celsius}}{\rho_{\text{нас.
    пара 60} \celsius}} = 0{,}75 \cdot \frac{23{,}000\,\frac{\text{г}}{\text{м}^{3}}}{130{,}000\,\frac{\text{г}}{\text{м}^{3}}} = 0{,}133 \approx 13{,}3\%.
    \end{align*}
}
\solutionspace{80pt}

\tasknumber{3}%
\task{%
    Из уравнения состояния идеального газа выведите или выразите...
    \begin{enumerate}
        \item объём,
        \item температуру,
        \item плотность газа.
    \end{enumerate}
}

\tasknumber{4}%
\task{%
    Запишите формулы и рядом с каждой физичической величиной укажите её название и единицы измерения в СИ:
    \begin{enumerate}
        \item первое начало термодинамики,
        \item внутренняя энергия идеального одноатомного газа.
    \end{enumerate}
}

\variantsplitter

\addpersonalvariant{Михаил Ярошевский}

\tasknumber{1}%
\task{%
    Определите КПД (оставив ответ точным в виде нескоратимой дроби) цикла 1231, рабочим телом которого является идеальный одноатомный газ, если
    \begin{itemize}
        \item 12 — изохорический нагрев в два раза,
        \item 23 — изобарическое расширение, при котором температура растёт в три раза,
        \item 31 — процесс, график которого в $PV$-координатах является отрезком прямой.
    \end{itemize}
    Бонус: замените цикл 1231 циклом, в котором 12 — изохорический нагрев в два раза, 23 — процесс, график которого в $PV$-координатах является отрезком прямой, 31 — изобарическое охлаждение, при котором температура падает в два раза.
}
\answer{%
    \begin{align*}
    A_{12} &= 0, \Delta U_{12} > 0, \implies Q_{12} = A_{12} + \Delta U_{12} > 0.
    \\
    A_{23} &> 0, \Delta U_{23} > 0, \implies Q_{23} = A_{23} + \Delta U_{23} > 0, \\
    A_{31} &= 0, \Delta U_{31} < 0, \implies Q_{31} = A_{31} + \Delta U_{31} < 0.
    \\
    P_1V_1 &= \nu R T_1, P_2V_2 = \nu R T_2, P_3V_3 = \nu R T_3 \text{ — уравнения состояния идеального газа}, \\
    &\text{Пусть $P_0$, $V_0$, $T_0$ — давление, объём и температура в точке 1 (минимальные во всём цикле):} \\
    P_1 &= P_0, P_2 = P_3, V_1 = V_2 = V_0, \text{остальные соотношения нужно считать} \\
    T_2 &= 2T_1 = 2T_0 \text{(по условию)} \implies \frac{P_2}{P_1} = \frac{P_2V_0}{P_1V_0} = \frac{P_2 V_2}{P_1 V_1}= \frac{\nu R T_2}{\nu R T_1} = \frac{T_2}{T_1} = 2 \implies P_2 = 2 P_1 = 2 P_0, \\
    T_3 &= 3T_2 = 6T_0 \text{(по условию)} \implies \frac{V_3}{V_2} = \frac{P_3V_3}{P_2V_2}= \frac{\nu R T_3}{\nu R T_2} = \frac{T_3}{T_2} = 3 \implies V_3 = 3 V_2 = 3 V_0.
    \\
    A_\text{цикл} &= \frac 12 (3P_0 - P_0)(2V_0 - V_0) = \frac 12 \cdot 2 \cdot P_0V_0, \\
    A_{23} &= 2P_0 \cdot (3V_0 - V_0) = 4P_0V_0, \\
    \Delta U_{23} &= \frac 32 \nu R T_3 - \frac 32 \nu R T_2 = \frac 32 P_3 V_3 - \frac 32 P_2 V_2 = \frac 32 \cdot 2 P_0 \cdot 3 V_0 -  \frac 32 \cdot 2 P_0 \cdot V_0 = \frac 32 \cdot 4 \cdot P_0V_0, \\
    \Delta U_{12} &= \frac 32 \nu R T_2 - \frac 32 \nu R T_1 = \frac 32 P_2 V_2 - \frac 32 P_1 V_1 = \frac 32 \cdot 2 P_0V_0 - \frac 32 P_0V_0 = \frac 32 \cdot 1 \cdot P_0V_0.
    \\
    \eta &= \frac{A_\text{цикл}}{Q_+} = \frac{A_\text{цикл}}{Q_{12} + Q_{23}}  = \frac{A_\text{цикл}}{A_{12} + \Delta U_{12} + A_{23} + \Delta U_{23}} =  \\
     &= \frac{\frac 12 \cdot 2 \cdot P_0V_0}{0 + \frac 32 \cdot 1 \cdot P_0V_0 + 4P_0V_0 + \frac 32 \cdot 4 \cdot P_0V_0} = \frac{\frac 12 \cdot 2}{\frac 32 \cdot 1 + 4 + \frac 32 \cdot 4} = \frac2{23} \approx 0.087.
    \end{align*}


        График процесса не в масштабе (эта часть пока не готова и сделать автоматически аккуратно сложно), но с верными подписями (а для решения этого достаточно):

        \begin{tikzpicture}[thick]
            \draw[-{Latex}] (0, 0) -- (0, 7) node[above left] {$P$};
            \draw[-{Latex}] (0, 0) -- (10, 0) node[right] {$V$};

            \draw[dashed] (0, 2) node[left] {$P_1 = P_0$} -| (3, 0) node[below] {$V_1 = V_2 = V_0$};
            \draw[dashed] (0, 6) node[left] {$P_2 = P_3 = 2P_0$} -| (9, 0) node[below] {$V_3 = 3V_0$};

            \draw (3, 2) node[above left]{1} node[below left]{$T_1 = T_0$}
                   (3, 6) node[below left]{2} node[above]{$T_2 = 2T_0$}
                   (9, 6) node[above right]{3} node[below right]{$T_3 = 6T_0$};
            \draw[midar] (3, 2) -- (3, 6);
            \draw[midar] (3, 6) -- (9, 6);
            \draw[midar] (9, 6) -- (3, 2);
        \end{tikzpicture}

        Решение бонуса:
        \begin{align*}
            A_{12} &= 0, \Delta U_{12} > 0, \implies Q_{12} = A_{12} + \Delta U_{12} > 0, \\
            A_{23} &> 0, \Delta U_{23} \text{ — ничего нельзя сказать, нужно исследовать отдельно}, \\
            A_{31} &< 0, \Delta U_{31} < 0, \implies Q_{31} = A_{31} + \Delta U_{31} < 0.
            \\
        \end{align*}

        Уравнения состояния идеального газа для точек 1, 2, 3: $P_1V_1 = \nu R T_1, P_2V_2 = \nu R T_2, P_3V_3 = \nu R T_3$.
        Пусть $P_0$, $V_0$, $T_0$ — давление, объём и температура в точке 1 (минимальные во всём цикле).

        12 --- изохора, $\frac{P_1V_1}{T_1} = \nu R = \frac{P_2V_2}{T_2}, V_2=V_1=V_0 \implies \frac{P_1}{T_1} =  \frac{P_2}{T_2} \implies P_2 = P_1 \frac{T_2}{T_1} = 2P_0$,

        31 --- изобара, $\frac{P_1V_1}{T_1} = \nu R = \frac{P_3V_3}{T_3}, P_3=P_1=P_0 \implies \frac{V_3}{T_3} =  \frac{V_1}{T_1} \implies V_3 = V_1 \frac{T_3}{T_1} = 2V_0$,

        Таким образом, используя новые обозначения, состояния газа в точках 1, 2 и 3 описываются макропараметрами $(P_0, V_0, T_0), (2P_0, V_0, 2T_0), (P_0, 2V_0, 2T_0)$ соответственно.

        \begin{tikzpicture}[thick]
            \draw[-{Latex}] (0, 0) -- (0, 7) node[above left] {$P$};
            \draw[-{Latex}] (0, 0) -- (10, 0) node[right] {$V$};

            \draw[dashed] (0, 2) node[left] {$P_1 = P_3 = P_0$} -| (9, 0) node[below] {$V_3 = 2V_0$};
            \draw[dashed] (0, 6) node[left] {$P_2 = 2P_0$} -| (3, 0) node[below] {$V_1 = V_2 = V_0$};

            \draw[dashed] (0, 5) node[left] {$P$} -| (4.5, 0) node[below] {$V$};
            \draw[dashed] (0, 4.6) node[left] {$P'$} -| (5.1, 0) node[below] {$V'$};

            \draw (3, 2) node[above left]{1} node[below left]{$T_1 = T_0$}
                   (3, 6) node[below left]{2} node[above]{$T_2 = 2T_0$}
                   (9, 2) node[above right]{3} node[below right]{$T_3 = 2T_0$};
            \draw[midar] (3, 2) -- (3, 6);
            \draw[midar] (3, 6) -- (9, 2);
            \draw[midar] (9, 2) -- (3, 2);
            \draw   (4.5, 5) node[above right]{$T$} (5.1, 4.6) node[above right]{$T'$};
        \end{tikzpicture}


        Теперь рассмотрим отдельно процесс 23, к остальному вернёмся позже.
        Уравнение этой прямой в $PV$-координатах: $P(V) = 3P_0 - \frac{P_0}{V_0} V$.
        Это значит, что при изменении объёма на $\Delta V$ давление изменится на $\Delta P = - \frac{P_0}{V_0} \Delta V$, обратите внимание на знак.

        Рассмотрим произвольную точку в процессе 23 и дадим процессу ещё немного свершиться, при этом объём изменится на $\Delta V$, давление — на $\Delta P$, температура (иначе бы была гипербола, а не прямая) — на $\Delta T$,
        т.е.
        из состояния $(P, V, T)$ мы перешли в $(P', V', T')$, причём  $P' = P + \Delta P, V' = V + \Delta V, T' = T + \Delta T$.

        При этом изменится внутренняя энергия:
        \begin{align*}
        \Delta U
            &= U' - U = \frac 32 \nu R T' - \frac 32 \nu R T = \frac 32 (P+\Delta P) (V+\Delta V) - \frac 32 PV\\
            &= \frac 32 ((P+\Delta P) (V+\Delta V) - PV) = \frac 32 (P\Delta V + V \Delta P + \Delta P \Delta V).
        \end{align*}

        Рассмотрим малые изменения объёма, тогда и изменение давления будем малым (т.к.
        $\Delta P = - \frac{P_0}{V_0} \Delta V$),
        а третьим слагаемым в выражении для $\Delta U$  можно пренебречь по сравнению с двумя другими:
        два первых это малые величины, а третье — произведение двух малых.
        Тогда $\Delta U = \frac 32 (P\Delta V + V \Delta P)$.

        Работа газа при этом малом изменении объёма — это площадь трапеции (тут ещё раз пренебрегли малым слагаемым):
        $$A = \frac{P + P'}2 \Delta V = \cbr{P + \frac{\Delta P}2} \Delta V = P \Delta V.$$

        Подведённое количество теплоты, используя первое начало термодинамики, будет равно
        \begin{align*}
        Q
            &= \frac 32 (P\Delta V + V \Delta P) + P \Delta V =  \frac 52 P\Delta V + \frac 32 V \Delta P = \\
            &= \frac 52 P\Delta V + \frac 32 V \cdot \cbr{- \frac{P_0}{V_0} \Delta V} = \frac{\Delta V}2 \cdot \cbr{5P - \frac{P_0}{V_0} V} = \\
            &= \frac{\Delta V}2 \cdot \cbr{5 \cdot \cbr{3P_0 - \frac{P_0}{V_0} V} - \frac{P_0}{V_0} V}
             = \frac{\Delta V \cdot P_0}2 \cdot \cbr{5 \cdot 3 - 8\frac V{V_0}}.
        \end{align*}

        Таком образом, знак количества теплоты $Q$ на участке 23 зависит от конкретного значения $V$:
        \begin{itemize}
            \item $\Delta V > 0$ на всём участке 23, поскольку газ расширяется,
            \item $P > 0$ — всегда, у нас идеальный газ, удары о стенки сосуда абсолютно упругие, а молекулы не взаимодействуют и поэтому давление только положительно,
            \item если $5 \cdot 3 - 8\frac V{V_0} > 0$ — тепло подводят, если же меньше нуля — отводят.
        \end{itemize}
        Решая последнее неравенство, получаем конкретное значение $V^*$: при $V < V^*$ тепло подводят, далее~— отводят.
        Тут *~--- некоторая точка между точками 2 и 3, конкретные значения надо досчитать:
        $$V^* = V_0 \cdot \frac{5 \cdot 3}8 = \frac{15}8 \cdot V_0 \implies P^* = 3P_0 - \frac{P_0}{V_0} V^* = \ldots = \frac98 \cdot P_0.$$

        Т.е.
        чтобы вычислить $Q_+$, надо сложить количества теплоты на участке 12 и лишь части участка 23 — участке 2*,
        той его части где это количество теплоты положительно.
        Имеем: $Q_+ = Q_{12} + Q_{2*}$.

        Теперь возвращаемся к циклу целиком и получаем:
        \begin{align*}
        A_\text{цикл} &= \frac 12 \cdot (2P_0 - P_0) \cdot (2V_0 - V_0) = \frac12 \cdot P_0V_0, \\
        A_{2*} &= \frac{P^* + 2P_0}2 \cdot (V^* - V_0)
            = \frac{\frac98 \cdot P_0 + 2P_0}2 \cdot \cbr{\frac{15}8 \cdot V_0 - V_0}
            = \ldots = \frac{175}{128} \cdot P_0 V_0, \\
        \Delta U_{2*} &= \frac 32 \nu R T^* - \frac 32 \nu R T_2 = \frac 32 (P^*V^* - P_0 \cdot 2V_0)
            = \frac 32 \cbr{\frac98 \cdot P_0 \cdot \frac{15}8 \cdot V_0 - P_0 \cdot 2V_0}
            = \frac{21}{128} \cdot P_0 V_0, \\
        \Delta U_{12} &= \frac 32 \nu R T_2 - \frac 32 \nu R T_1 = \frac 32 (2P_0V_0 - P_0V_0) = \ldots = \frac32 \cdot P_0 V_0, \\
        \eta &= \frac{A_\text{цикл}}{Q_+} = \frac{A_\text{цикл}}{Q_{12} + Q_{2*}}
            = \frac{A_\text{цикл}}{A_{12} + \Delta U_{12} + A_{2*} + \Delta U_{2*}} = \\
            &= \frac{\frac12 \cdot P_0V_0}{0 + \frac32 \cdot P_0 V_0 + \frac{175}{128} \cdot P_0 V_0 + \frac{21}{128} \cdot P_0 V_0}
             = \frac{A_bonus_cycle:LaTeX}{\frac32 + \frac{175}{128} + \frac{21}{128}}
             = \frac{16}{97} \leftarrow \text{вжух и готово!}
        \end{align*}
}
\solutionspace{360pt}

\tasknumber{2}%
\task{%
    При температуре $15\celsius$ относительная влажность воздуха составляет $50\%$.
    \begin{itemize}
        \item Определите точку росы для этого воздуха.
        \item Какой станет относительная влажность этого воздуха, если нагреть его до $80\celsius$?
    \end{itemize}
}
\answer{%
    \begin{align*}
    &\text{Значения плотности насыщенного водяного пара определяем по таблице:} \\
    &\rho_{\text{нас.
    пара 15} \celsius} = 12{,}800\,\frac{\text{г}}{\text{м}^{3}}, \rho_{\text{нас.
    пара 80} \celsius} = 293{,}000\,\frac{\text{г}}{\text{м}^{3}}.
    \\
    \varphi_1 &= \frac{\rho_\text{пара}}{\rho_{\text{нас.
    пара 15} \celsius}} \implies {\rho_\text{пара}} = \rho_{\text{нас.
    пара 15} \celsius} \cdot \varphi_1 = 12{,}800\,\frac{\text{г}}{\text{м}^{3}} \cdot 0{,}50 = 6{,}400\,\frac{\text{г}}{\text{м}^{3}}.
    \\
    &\text{По таблице определяем, при какой температуре пар с такой плотностью станет насыщенным:}  \\
    t_\text{росы} &= 4{,}0\celsius, \\
    \varphi_2 &= \frac{\rho_\text{пара}}{\rho_{\text{нас.
    пара 80} \celsius}} = \frac{\rho_{\text{нас.
    пара 15} \celsius} \cdot \varphi_1}{\rho_{\text{нас.
    пара 80} \celsius}}= \varphi_1 \cdot \frac{\rho_{\text{нас.
    пара 15} \celsius}}{\rho_{\text{нас.
    пара 80} \celsius}} = 0{,}50 \cdot \frac{12{,}800\,\frac{\text{г}}{\text{м}^{3}}}{293{,}000\,\frac{\text{г}}{\text{м}^{3}}} = 0{,}022 \approx 2{,}2\%.
    \end{align*}
}
\solutionspace{80pt}

\tasknumber{3}%
\task{%
    Из уравнения состояния идеального газа выведите или выразите...
    \begin{enumerate}
        \item объём,
        \item температуру,
        \item концентрацию молекул газа.
    \end{enumerate}
}

\tasknumber{4}%
\task{%
    Запишите формулы и рядом с каждой физичической величиной укажите её название и единицы измерения в СИ:
    \begin{enumerate}
        \item первое начало термодинамики,
        \item внутренняя энергия идеального одноатомного газа.
    \end{enumerate}
}

\variantsplitter

\addpersonalvariant{Алексей Алимпиев}

\tasknumber{1}%
\task{%
    Определите КПД (оставив ответ точным в виде нескоратимой дроби) цикла 1231, рабочим телом которого является идеальный одноатомный газ, если
    \begin{itemize}
        \item 12 — изохорический нагрев в три раза,
        \item 23 — изобарическое расширение, при котором температура растёт в два раза,
        \item 31 — процесс, график которого в $PV$-координатах является отрезком прямой.
    \end{itemize}
    Бонус: замените цикл 1231 циклом, в котором 12 — изохорический нагрев в три раза, 23 — процесс, график которого в $PV$-координатах является отрезком прямой, 31 — изобарическое охлаждение, при котором температура падает в три раза.
}
\answer{%
    \begin{align*}
    A_{12} &= 0, \Delta U_{12} > 0, \implies Q_{12} = A_{12} + \Delta U_{12} > 0.
    \\
    A_{23} &> 0, \Delta U_{23} > 0, \implies Q_{23} = A_{23} + \Delta U_{23} > 0, \\
    A_{31} &= 0, \Delta U_{31} < 0, \implies Q_{31} = A_{31} + \Delta U_{31} < 0.
    \\
    P_1V_1 &= \nu R T_1, P_2V_2 = \nu R T_2, P_3V_3 = \nu R T_3 \text{ — уравнения состояния идеального газа}, \\
    &\text{Пусть $P_0$, $V_0$, $T_0$ — давление, объём и температура в точке 1 (минимальные во всём цикле):} \\
    P_1 &= P_0, P_2 = P_3, V_1 = V_2 = V_0, \text{остальные соотношения нужно считать} \\
    T_2 &= 3T_1 = 3T_0 \text{(по условию)} \implies \frac{P_2}{P_1} = \frac{P_2V_0}{P_1V_0} = \frac{P_2 V_2}{P_1 V_1}= \frac{\nu R T_2}{\nu R T_1} = \frac{T_2}{T_1} = 3 \implies P_2 = 3 P_1 = 3 P_0, \\
    T_3 &= 2T_2 = 6T_0 \text{(по условию)} \implies \frac{V_3}{V_2} = \frac{P_3V_3}{P_2V_2}= \frac{\nu R T_3}{\nu R T_2} = \frac{T_3}{T_2} = 2 \implies V_3 = 2 V_2 = 2 V_0.
    \\
    A_\text{цикл} &= \frac 12 (2P_0 - P_0)(3V_0 - V_0) = \frac 12 \cdot 2 \cdot P_0V_0, \\
    A_{23} &= 3P_0 \cdot (2V_0 - V_0) = 3P_0V_0, \\
    \Delta U_{23} &= \frac 32 \nu R T_3 - \frac 32 \nu R T_2 = \frac 32 P_3 V_3 - \frac 32 P_2 V_2 = \frac 32 \cdot 3 P_0 \cdot 2 V_0 -  \frac 32 \cdot 3 P_0 \cdot V_0 = \frac 32 \cdot 3 \cdot P_0V_0, \\
    \Delta U_{12} &= \frac 32 \nu R T_2 - \frac 32 \nu R T_1 = \frac 32 P_2 V_2 - \frac 32 P_1 V_1 = \frac 32 \cdot 3 P_0V_0 - \frac 32 P_0V_0 = \frac 32 \cdot 2 \cdot P_0V_0.
    \\
    \eta &= \frac{A_\text{цикл}}{Q_+} = \frac{A_\text{цикл}}{Q_{12} + Q_{23}}  = \frac{A_\text{цикл}}{A_{12} + \Delta U_{12} + A_{23} + \Delta U_{23}} =  \\
     &= \frac{\frac 12 \cdot 2 \cdot P_0V_0}{0 + \frac 32 \cdot 2 \cdot P_0V_0 + 3P_0V_0 + \frac 32 \cdot 3 \cdot P_0V_0} = \frac{\frac 12 \cdot 2}{\frac 32 \cdot 2 + 3 + \frac 32 \cdot 3} = \frac2{21} \approx 0.095.
    \end{align*}


        График процесса не в масштабе (эта часть пока не готова и сделать автоматически аккуратно сложно), но с верными подписями (а для решения этого достаточно):

        \begin{tikzpicture}[thick]
            \draw[-{Latex}] (0, 0) -- (0, 7) node[above left] {$P$};
            \draw[-{Latex}] (0, 0) -- (10, 0) node[right] {$V$};

            \draw[dashed] (0, 2) node[left] {$P_1 = P_0$} -| (3, 0) node[below] {$V_1 = V_2 = V_0$};
            \draw[dashed] (0, 6) node[left] {$P_2 = P_3 = 3P_0$} -| (9, 0) node[below] {$V_3 = 2V_0$};

            \draw (3, 2) node[above left]{1} node[below left]{$T_1 = T_0$}
                   (3, 6) node[below left]{2} node[above]{$T_2 = 3T_0$}
                   (9, 6) node[above right]{3} node[below right]{$T_3 = 6T_0$};
            \draw[midar] (3, 2) -- (3, 6);
            \draw[midar] (3, 6) -- (9, 6);
            \draw[midar] (9, 6) -- (3, 2);
        \end{tikzpicture}

        Решение бонуса:
        \begin{align*}
            A_{12} &= 0, \Delta U_{12} > 0, \implies Q_{12} = A_{12} + \Delta U_{12} > 0, \\
            A_{23} &> 0, \Delta U_{23} \text{ — ничего нельзя сказать, нужно исследовать отдельно}, \\
            A_{31} &< 0, \Delta U_{31} < 0, \implies Q_{31} = A_{31} + \Delta U_{31} < 0.
            \\
        \end{align*}

        Уравнения состояния идеального газа для точек 1, 2, 3: $P_1V_1 = \nu R T_1, P_2V_2 = \nu R T_2, P_3V_3 = \nu R T_3$.
        Пусть $P_0$, $V_0$, $T_0$ — давление, объём и температура в точке 1 (минимальные во всём цикле).

        12 --- изохора, $\frac{P_1V_1}{T_1} = \nu R = \frac{P_2V_2}{T_2}, V_2=V_1=V_0 \implies \frac{P_1}{T_1} =  \frac{P_2}{T_2} \implies P_2 = P_1 \frac{T_2}{T_1} = 3P_0$,

        31 --- изобара, $\frac{P_1V_1}{T_1} = \nu R = \frac{P_3V_3}{T_3}, P_3=P_1=P_0 \implies \frac{V_3}{T_3} =  \frac{V_1}{T_1} \implies V_3 = V_1 \frac{T_3}{T_1} = 3V_0$,

        Таким образом, используя новые обозначения, состояния газа в точках 1, 2 и 3 описываются макропараметрами $(P_0, V_0, T_0), (3P_0, V_0, 3T_0), (P_0, 3V_0, 3T_0)$ соответственно.

        \begin{tikzpicture}[thick]
            \draw[-{Latex}] (0, 0) -- (0, 7) node[above left] {$P$};
            \draw[-{Latex}] (0, 0) -- (10, 0) node[right] {$V$};

            \draw[dashed] (0, 2) node[left] {$P_1 = P_3 = P_0$} -| (9, 0) node[below] {$V_3 = 3V_0$};
            \draw[dashed] (0, 6) node[left] {$P_2 = 3P_0$} -| (3, 0) node[below] {$V_1 = V_2 = V_0$};

            \draw[dashed] (0, 5) node[left] {$P$} -| (4.5, 0) node[below] {$V$};
            \draw[dashed] (0, 4.6) node[left] {$P'$} -| (5.1, 0) node[below] {$V'$};

            \draw (3, 2) node[above left]{1} node[below left]{$T_1 = T_0$}
                   (3, 6) node[below left]{2} node[above]{$T_2 = 3T_0$}
                   (9, 2) node[above right]{3} node[below right]{$T_3 = 3T_0$};
            \draw[midar] (3, 2) -- (3, 6);
            \draw[midar] (3, 6) -- (9, 2);
            \draw[midar] (9, 2) -- (3, 2);
            \draw   (4.5, 5) node[above right]{$T$} (5.1, 4.6) node[above right]{$T'$};
        \end{tikzpicture}


        Теперь рассмотрим отдельно процесс 23, к остальному вернёмся позже.
        Уравнение этой прямой в $PV$-координатах: $P(V) = 4P_0 - \frac{P_0}{V_0} V$.
        Это значит, что при изменении объёма на $\Delta V$ давление изменится на $\Delta P = - \frac{P_0}{V_0} \Delta V$, обратите внимание на знак.

        Рассмотрим произвольную точку в процессе 23 и дадим процессу ещё немного свершиться, при этом объём изменится на $\Delta V$, давление — на $\Delta P$, температура (иначе бы была гипербола, а не прямая) — на $\Delta T$,
        т.е.
        из состояния $(P, V, T)$ мы перешли в $(P', V', T')$, причём  $P' = P + \Delta P, V' = V + \Delta V, T' = T + \Delta T$.

        При этом изменится внутренняя энергия:
        \begin{align*}
        \Delta U
            &= U' - U = \frac 32 \nu R T' - \frac 32 \nu R T = \frac 32 (P+\Delta P) (V+\Delta V) - \frac 32 PV\\
            &= \frac 32 ((P+\Delta P) (V+\Delta V) - PV) = \frac 32 (P\Delta V + V \Delta P + \Delta P \Delta V).
        \end{align*}

        Рассмотрим малые изменения объёма, тогда и изменение давления будем малым (т.к.
        $\Delta P = - \frac{P_0}{V_0} \Delta V$),
        а третьим слагаемым в выражении для $\Delta U$  можно пренебречь по сравнению с двумя другими:
        два первых это малые величины, а третье — произведение двух малых.
        Тогда $\Delta U = \frac 32 (P\Delta V + V \Delta P)$.

        Работа газа при этом малом изменении объёма — это площадь трапеции (тут ещё раз пренебрегли малым слагаемым):
        $$A = \frac{P + P'}2 \Delta V = \cbr{P + \frac{\Delta P}2} \Delta V = P \Delta V.$$

        Подведённое количество теплоты, используя первое начало термодинамики, будет равно
        \begin{align*}
        Q
            &= \frac 32 (P\Delta V + V \Delta P) + P \Delta V =  \frac 52 P\Delta V + \frac 32 V \Delta P = \\
            &= \frac 52 P\Delta V + \frac 32 V \cdot \cbr{- \frac{P_0}{V_0} \Delta V} = \frac{\Delta V}2 \cdot \cbr{5P - \frac{P_0}{V_0} V} = \\
            &= \frac{\Delta V}2 \cdot \cbr{5 \cdot \cbr{4P_0 - \frac{P_0}{V_0} V} - \frac{P_0}{V_0} V}
             = \frac{\Delta V \cdot P_0}2 \cdot \cbr{5 \cdot 4 - 8\frac V{V_0}}.
        \end{align*}

        Таком образом, знак количества теплоты $Q$ на участке 23 зависит от конкретного значения $V$:
        \begin{itemize}
            \item $\Delta V > 0$ на всём участке 23, поскольку газ расширяется,
            \item $P > 0$ — всегда, у нас идеальный газ, удары о стенки сосуда абсолютно упругие, а молекулы не взаимодействуют и поэтому давление только положительно,
            \item если $5 \cdot 4 - 8\frac V{V_0} > 0$ — тепло подводят, если же меньше нуля — отводят.
        \end{itemize}
        Решая последнее неравенство, получаем конкретное значение $V^*$: при $V < V^*$ тепло подводят, далее~— отводят.
        Тут *~--- некоторая точка между точками 2 и 3, конкретные значения надо досчитать:
        $$V^* = V_0 \cdot \frac{5 \cdot 4}8 = \frac52 \cdot V_0 \implies P^* = 4P_0 - \frac{P_0}{V_0} V^* = \ldots = \frac32 \cdot P_0.$$

        Т.е.
        чтобы вычислить $Q_+$, надо сложить количества теплоты на участке 12 и лишь части участка 23 — участке 2*,
        той его части где это количество теплоты положительно.
        Имеем: $Q_+ = Q_{12} + Q_{2*}$.

        Теперь возвращаемся к циклу целиком и получаем:
        \begin{align*}
        A_\text{цикл} &= \frac 12 \cdot (3P_0 - P_0) \cdot (3V_0 - V_0) = 2 \cdot P_0V_0, \\
        A_{2*} &= \frac{P^* + 3P_0}2 \cdot (V^* - V_0)
            = \frac{\frac32 \cdot P_0 + 3P_0}2 \cdot \cbr{\frac52 \cdot V_0 - V_0}
            = \ldots = \frac{27}8 \cdot P_0 V_0, \\
        \Delta U_{2*} &= \frac 32 \nu R T^* - \frac 32 \nu R T_2 = \frac 32 (P^*V^* - P_0 \cdot 3V_0)
            = \frac 32 \cbr{\frac32 \cdot P_0 \cdot \frac52 \cdot V_0 - P_0 \cdot 3V_0}
            = \frac98 \cdot P_0 V_0, \\
        \Delta U_{12} &= \frac 32 \nu R T_2 - \frac 32 \nu R T_1 = \frac 32 (3P_0V_0 - P_0V_0) = \ldots = 3 \cdot P_0 V_0, \\
        \eta &= \frac{A_\text{цикл}}{Q_+} = \frac{A_\text{цикл}}{Q_{12} + Q_{2*}}
            = \frac{A_\text{цикл}}{A_{12} + \Delta U_{12} + A_{2*} + \Delta U_{2*}} = \\
            &= \frac{2 \cdot P_0V_0}{0 + 3 \cdot P_0 V_0 + \frac{27}8 \cdot P_0 V_0 + \frac98 \cdot P_0 V_0}
             = \frac{A_bonus_cycle:LaTeX}{3 + \frac{27}8 + \frac98}
             = \frac4{15} \leftarrow \text{вжух и готово!}
        \end{align*}
}
\solutionspace{360pt}

\tasknumber{2}%
\task{%
    При температуре $30\celsius$ относительная влажность воздуха составляет $75\%$.
    \begin{itemize}
        \item Определите точку росы для этого воздуха.
        \item Какой станет относительная влажность этого воздуха, если нагреть его до $70\celsius$?
    \end{itemize}
}
\answer{%
    \begin{align*}
    &\text{Значения плотности насыщенного водяного пара определяем по таблице:} \\
    &\rho_{\text{нас.
    пара 30} \celsius} = 30{,}300\,\frac{\text{г}}{\text{м}^{3}}, \rho_{\text{нас.
    пара 70} \celsius} = 198{,}000\,\frac{\text{г}}{\text{м}^{3}}.
    \\
    \varphi_1 &= \frac{\rho_\text{пара}}{\rho_{\text{нас.
    пара 30} \celsius}} \implies {\rho_\text{пара}} = \rho_{\text{нас.
    пара 30} \celsius} \cdot \varphi_1 = 30{,}300\,\frac{\text{г}}{\text{м}^{3}} \cdot 0{,}75 = 22{,}725\,\frac{\text{г}}{\text{м}^{3}}.
    \\
    &\text{По таблице определяем, при какой температуре пар с такой плотностью станет насыщенным:}  \\
    t_\text{росы} &= 24{,}8\celsius, \\
    \varphi_2 &= \frac{\rho_\text{пара}}{\rho_{\text{нас.
    пара 70} \celsius}} = \frac{\rho_{\text{нас.
    пара 30} \celsius} \cdot \varphi_1}{\rho_{\text{нас.
    пара 70} \celsius}}= \varphi_1 \cdot \frac{\rho_{\text{нас.
    пара 30} \celsius}}{\rho_{\text{нас.
    пара 70} \celsius}} = 0{,}75 \cdot \frac{30{,}300\,\frac{\text{г}}{\text{м}^{3}}}{198{,}000\,\frac{\text{г}}{\text{м}^{3}}} = 0{,}115 \approx 11{,}5\%.
    \end{align*}
}
\solutionspace{80pt}

\tasknumber{3}%
\task{%
    Из уравнения состояния идеального газа выведите или выразите...
    \begin{enumerate}
        \item объём,
        \item температуру,
        \item плотность газа.
    \end{enumerate}
}

\tasknumber{4}%
\task{%
    Запишите формулы и рядом с каждой физичической величиной укажите её название и единицы измерения в СИ:
    \begin{enumerate}
        \item первое начало термодинамики,
        \item внутренняя энергия идеального одноатомного газа.
    \end{enumerate}
}

\variantsplitter

\addpersonalvariant{Евгений Васин}

\tasknumber{1}%
\task{%
    Определите КПД (оставив ответ точным в виде нескоратимой дроби) цикла 1231, рабочим телом которого является идеальный одноатомный газ, если
    \begin{itemize}
        \item 12 — изохорический нагрев в три раза,
        \item 23 — изобарическое расширение, при котором температура растёт в шесть раз,
        \item 31 — процесс, график которого в $PV$-координатах является отрезком прямой.
    \end{itemize}
    Бонус: замените цикл 1231 циклом, в котором 12 — изохорический нагрев в три раза, 23 — процесс, график которого в $PV$-координатах является отрезком прямой, 31 — изобарическое охлаждение, при котором температура падает в три раза.
}
\answer{%
    \begin{align*}
    A_{12} &= 0, \Delta U_{12} > 0, \implies Q_{12} = A_{12} + \Delta U_{12} > 0.
    \\
    A_{23} &> 0, \Delta U_{23} > 0, \implies Q_{23} = A_{23} + \Delta U_{23} > 0, \\
    A_{31} &= 0, \Delta U_{31} < 0, \implies Q_{31} = A_{31} + \Delta U_{31} < 0.
    \\
    P_1V_1 &= \nu R T_1, P_2V_2 = \nu R T_2, P_3V_3 = \nu R T_3 \text{ — уравнения состояния идеального газа}, \\
    &\text{Пусть $P_0$, $V_0$, $T_0$ — давление, объём и температура в точке 1 (минимальные во всём цикле):} \\
    P_1 &= P_0, P_2 = P_3, V_1 = V_2 = V_0, \text{остальные соотношения нужно считать} \\
    T_2 &= 3T_1 = 3T_0 \text{(по условию)} \implies \frac{P_2}{P_1} = \frac{P_2V_0}{P_1V_0} = \frac{P_2 V_2}{P_1 V_1}= \frac{\nu R T_2}{\nu R T_1} = \frac{T_2}{T_1} = 3 \implies P_2 = 3 P_1 = 3 P_0, \\
    T_3 &= 6T_2 = 18T_0 \text{(по условию)} \implies \frac{V_3}{V_2} = \frac{P_3V_3}{P_2V_2}= \frac{\nu R T_3}{\nu R T_2} = \frac{T_3}{T_2} = 6 \implies V_3 = 6 V_2 = 6 V_0.
    \\
    A_\text{цикл} &= \frac 12 (6P_0 - P_0)(3V_0 - V_0) = \frac 12 \cdot 10 \cdot P_0V_0, \\
    A_{23} &= 3P_0 \cdot (6V_0 - V_0) = 15P_0V_0, \\
    \Delta U_{23} &= \frac 32 \nu R T_3 - \frac 32 \nu R T_2 = \frac 32 P_3 V_3 - \frac 32 P_2 V_2 = \frac 32 \cdot 3 P_0 \cdot 6 V_0 -  \frac 32 \cdot 3 P_0 \cdot V_0 = \frac 32 \cdot 15 \cdot P_0V_0, \\
    \Delta U_{12} &= \frac 32 \nu R T_2 - \frac 32 \nu R T_1 = \frac 32 P_2 V_2 - \frac 32 P_1 V_1 = \frac 32 \cdot 3 P_0V_0 - \frac 32 P_0V_0 = \frac 32 \cdot 2 \cdot P_0V_0.
    \\
    \eta &= \frac{A_\text{цикл}}{Q_+} = \frac{A_\text{цикл}}{Q_{12} + Q_{23}}  = \frac{A_\text{цикл}}{A_{12} + \Delta U_{12} + A_{23} + \Delta U_{23}} =  \\
     &= \frac{\frac 12 \cdot 10 \cdot P_0V_0}{0 + \frac 32 \cdot 2 \cdot P_0V_0 + 15P_0V_0 + \frac 32 \cdot 15 \cdot P_0V_0} = \frac{\frac 12 \cdot 10}{\frac 32 \cdot 2 + 15 + \frac 32 \cdot 15} = \frac{10}{81} \approx 0.123.
    \end{align*}


        График процесса не в масштабе (эта часть пока не готова и сделать автоматически аккуратно сложно), но с верными подписями (а для решения этого достаточно):

        \begin{tikzpicture}[thick]
            \draw[-{Latex}] (0, 0) -- (0, 7) node[above left] {$P$};
            \draw[-{Latex}] (0, 0) -- (10, 0) node[right] {$V$};

            \draw[dashed] (0, 2) node[left] {$P_1 = P_0$} -| (3, 0) node[below] {$V_1 = V_2 = V_0$};
            \draw[dashed] (0, 6) node[left] {$P_2 = P_3 = 3P_0$} -| (9, 0) node[below] {$V_3 = 6V_0$};

            \draw (3, 2) node[above left]{1} node[below left]{$T_1 = T_0$}
                   (3, 6) node[below left]{2} node[above]{$T_2 = 3T_0$}
                   (9, 6) node[above right]{3} node[below right]{$T_3 = 18T_0$};
            \draw[midar] (3, 2) -- (3, 6);
            \draw[midar] (3, 6) -- (9, 6);
            \draw[midar] (9, 6) -- (3, 2);
        \end{tikzpicture}

        Решение бонуса:
        \begin{align*}
            A_{12} &= 0, \Delta U_{12} > 0, \implies Q_{12} = A_{12} + \Delta U_{12} > 0, \\
            A_{23} &> 0, \Delta U_{23} \text{ — ничего нельзя сказать, нужно исследовать отдельно}, \\
            A_{31} &< 0, \Delta U_{31} < 0, \implies Q_{31} = A_{31} + \Delta U_{31} < 0.
            \\
        \end{align*}

        Уравнения состояния идеального газа для точек 1, 2, 3: $P_1V_1 = \nu R T_1, P_2V_2 = \nu R T_2, P_3V_3 = \nu R T_3$.
        Пусть $P_0$, $V_0$, $T_0$ — давление, объём и температура в точке 1 (минимальные во всём цикле).

        12 --- изохора, $\frac{P_1V_1}{T_1} = \nu R = \frac{P_2V_2}{T_2}, V_2=V_1=V_0 \implies \frac{P_1}{T_1} =  \frac{P_2}{T_2} \implies P_2 = P_1 \frac{T_2}{T_1} = 3P_0$,

        31 --- изобара, $\frac{P_1V_1}{T_1} = \nu R = \frac{P_3V_3}{T_3}, P_3=P_1=P_0 \implies \frac{V_3}{T_3} =  \frac{V_1}{T_1} \implies V_3 = V_1 \frac{T_3}{T_1} = 3V_0$,

        Таким образом, используя новые обозначения, состояния газа в точках 1, 2 и 3 описываются макропараметрами $(P_0, V_0, T_0), (3P_0, V_0, 3T_0), (P_0, 3V_0, 3T_0)$ соответственно.

        \begin{tikzpicture}[thick]
            \draw[-{Latex}] (0, 0) -- (0, 7) node[above left] {$P$};
            \draw[-{Latex}] (0, 0) -- (10, 0) node[right] {$V$};

            \draw[dashed] (0, 2) node[left] {$P_1 = P_3 = P_0$} -| (9, 0) node[below] {$V_3 = 3V_0$};
            \draw[dashed] (0, 6) node[left] {$P_2 = 3P_0$} -| (3, 0) node[below] {$V_1 = V_2 = V_0$};

            \draw[dashed] (0, 5) node[left] {$P$} -| (4.5, 0) node[below] {$V$};
            \draw[dashed] (0, 4.6) node[left] {$P'$} -| (5.1, 0) node[below] {$V'$};

            \draw (3, 2) node[above left]{1} node[below left]{$T_1 = T_0$}
                   (3, 6) node[below left]{2} node[above]{$T_2 = 3T_0$}
                   (9, 2) node[above right]{3} node[below right]{$T_3 = 3T_0$};
            \draw[midar] (3, 2) -- (3, 6);
            \draw[midar] (3, 6) -- (9, 2);
            \draw[midar] (9, 2) -- (3, 2);
            \draw   (4.5, 5) node[above right]{$T$} (5.1, 4.6) node[above right]{$T'$};
        \end{tikzpicture}


        Теперь рассмотрим отдельно процесс 23, к остальному вернёмся позже.
        Уравнение этой прямой в $PV$-координатах: $P(V) = 4P_0 - \frac{P_0}{V_0} V$.
        Это значит, что при изменении объёма на $\Delta V$ давление изменится на $\Delta P = - \frac{P_0}{V_0} \Delta V$, обратите внимание на знак.

        Рассмотрим произвольную точку в процессе 23 и дадим процессу ещё немного свершиться, при этом объём изменится на $\Delta V$, давление — на $\Delta P$, температура (иначе бы была гипербола, а не прямая) — на $\Delta T$,
        т.е.
        из состояния $(P, V, T)$ мы перешли в $(P', V', T')$, причём  $P' = P + \Delta P, V' = V + \Delta V, T' = T + \Delta T$.

        При этом изменится внутренняя энергия:
        \begin{align*}
        \Delta U
            &= U' - U = \frac 32 \nu R T' - \frac 32 \nu R T = \frac 32 (P+\Delta P) (V+\Delta V) - \frac 32 PV\\
            &= \frac 32 ((P+\Delta P) (V+\Delta V) - PV) = \frac 32 (P\Delta V + V \Delta P + \Delta P \Delta V).
        \end{align*}

        Рассмотрим малые изменения объёма, тогда и изменение давления будем малым (т.к.
        $\Delta P = - \frac{P_0}{V_0} \Delta V$),
        а третьим слагаемым в выражении для $\Delta U$  можно пренебречь по сравнению с двумя другими:
        два первых это малые величины, а третье — произведение двух малых.
        Тогда $\Delta U = \frac 32 (P\Delta V + V \Delta P)$.

        Работа газа при этом малом изменении объёма — это площадь трапеции (тут ещё раз пренебрегли малым слагаемым):
        $$A = \frac{P + P'}2 \Delta V = \cbr{P + \frac{\Delta P}2} \Delta V = P \Delta V.$$

        Подведённое количество теплоты, используя первое начало термодинамики, будет равно
        \begin{align*}
        Q
            &= \frac 32 (P\Delta V + V \Delta P) + P \Delta V =  \frac 52 P\Delta V + \frac 32 V \Delta P = \\
            &= \frac 52 P\Delta V + \frac 32 V \cdot \cbr{- \frac{P_0}{V_0} \Delta V} = \frac{\Delta V}2 \cdot \cbr{5P - \frac{P_0}{V_0} V} = \\
            &= \frac{\Delta V}2 \cdot \cbr{5 \cdot \cbr{4P_0 - \frac{P_0}{V_0} V} - \frac{P_0}{V_0} V}
             = \frac{\Delta V \cdot P_0}2 \cdot \cbr{5 \cdot 4 - 8\frac V{V_0}}.
        \end{align*}

        Таком образом, знак количества теплоты $Q$ на участке 23 зависит от конкретного значения $V$:
        \begin{itemize}
            \item $\Delta V > 0$ на всём участке 23, поскольку газ расширяется,
            \item $P > 0$ — всегда, у нас идеальный газ, удары о стенки сосуда абсолютно упругие, а молекулы не взаимодействуют и поэтому давление только положительно,
            \item если $5 \cdot 4 - 8\frac V{V_0} > 0$ — тепло подводят, если же меньше нуля — отводят.
        \end{itemize}
        Решая последнее неравенство, получаем конкретное значение $V^*$: при $V < V^*$ тепло подводят, далее~— отводят.
        Тут *~--- некоторая точка между точками 2 и 3, конкретные значения надо досчитать:
        $$V^* = V_0 \cdot \frac{5 \cdot 4}8 = \frac52 \cdot V_0 \implies P^* = 4P_0 - \frac{P_0}{V_0} V^* = \ldots = \frac32 \cdot P_0.$$

        Т.е.
        чтобы вычислить $Q_+$, надо сложить количества теплоты на участке 12 и лишь части участка 23 — участке 2*,
        той его части где это количество теплоты положительно.
        Имеем: $Q_+ = Q_{12} + Q_{2*}$.

        Теперь возвращаемся к циклу целиком и получаем:
        \begin{align*}
        A_\text{цикл} &= \frac 12 \cdot (3P_0 - P_0) \cdot (3V_0 - V_0) = 2 \cdot P_0V_0, \\
        A_{2*} &= \frac{P^* + 3P_0}2 \cdot (V^* - V_0)
            = \frac{\frac32 \cdot P_0 + 3P_0}2 \cdot \cbr{\frac52 \cdot V_0 - V_0}
            = \ldots = \frac{27}8 \cdot P_0 V_0, \\
        \Delta U_{2*} &= \frac 32 \nu R T^* - \frac 32 \nu R T_2 = \frac 32 (P^*V^* - P_0 \cdot 3V_0)
            = \frac 32 \cbr{\frac32 \cdot P_0 \cdot \frac52 \cdot V_0 - P_0 \cdot 3V_0}
            = \frac98 \cdot P_0 V_0, \\
        \Delta U_{12} &= \frac 32 \nu R T_2 - \frac 32 \nu R T_1 = \frac 32 (3P_0V_0 - P_0V_0) = \ldots = 3 \cdot P_0 V_0, \\
        \eta &= \frac{A_\text{цикл}}{Q_+} = \frac{A_\text{цикл}}{Q_{12} + Q_{2*}}
            = \frac{A_\text{цикл}}{A_{12} + \Delta U_{12} + A_{2*} + \Delta U_{2*}} = \\
            &= \frac{2 \cdot P_0V_0}{0 + 3 \cdot P_0 V_0 + \frac{27}8 \cdot P_0 V_0 + \frac98 \cdot P_0 V_0}
             = \frac{A_bonus_cycle:LaTeX}{3 + \frac{27}8 + \frac98}
             = \frac4{15} \leftarrow \text{вжух и готово!}
        \end{align*}
}
\solutionspace{360pt}

\tasknumber{2}%
\task{%
    При температуре $30\celsius$ относительная влажность воздуха составляет $70\%$.
    \begin{itemize}
        \item Определите точку росы для этого воздуха.
        \item Какой станет относительная влажность этого воздуха, если нагреть его до $80\celsius$?
    \end{itemize}
}
\answer{%
    \begin{align*}
    &\text{Значения плотности насыщенного водяного пара определяем по таблице:} \\
    &\rho_{\text{нас.
    пара 30} \celsius} = 30{,}300\,\frac{\text{г}}{\text{м}^{3}}, \rho_{\text{нас.
    пара 80} \celsius} = 293{,}000\,\frac{\text{г}}{\text{м}^{3}}.
    \\
    \varphi_1 &= \frac{\rho_\text{пара}}{\rho_{\text{нас.
    пара 30} \celsius}} \implies {\rho_\text{пара}} = \rho_{\text{нас.
    пара 30} \celsius} \cdot \varphi_1 = 30{,}300\,\frac{\text{г}}{\text{м}^{3}} \cdot 0{,}70 = 21{,}210\,\frac{\text{г}}{\text{м}^{3}}.
    \\
    &\text{По таблице определяем, при какой температуре пар с такой плотностью станет насыщенным:}  \\
    t_\text{росы} &= 23{,}5\celsius, \\
    \varphi_2 &= \frac{\rho_\text{пара}}{\rho_{\text{нас.
    пара 80} \celsius}} = \frac{\rho_{\text{нас.
    пара 30} \celsius} \cdot \varphi_1}{\rho_{\text{нас.
    пара 80} \celsius}}= \varphi_1 \cdot \frac{\rho_{\text{нас.
    пара 30} \celsius}}{\rho_{\text{нас.
    пара 80} \celsius}} = 0{,}70 \cdot \frac{30{,}300\,\frac{\text{г}}{\text{м}^{3}}}{293{,}000\,\frac{\text{г}}{\text{м}^{3}}} = 0{,}072 \approx 7{,}2\%.
    \end{align*}
}
\solutionspace{80pt}

\tasknumber{3}%
\task{%
    Из уравнения состояния идеального газа выведите или выразите...
    \begin{enumerate}
        \item давление,
        \item молярную массу,
        \item плотность газа.
    \end{enumerate}
}

\tasknumber{4}%
\task{%
    Запишите формулы и рядом с каждой физичической величиной укажите её название и единицы измерения в СИ:
    \begin{enumerate}
        \item первое начало термодинамики,
        \item внутренняя энергия идеального одноатомного газа.
    \end{enumerate}
}

\variantsplitter

\addpersonalvariant{Вячеслав Волохов}

\tasknumber{1}%
\task{%
    Определите КПД (оставив ответ точным в виде нескоратимой дроби) цикла 1231, рабочим телом которого является идеальный одноатомный газ, если
    \begin{itemize}
        \item 12 — изохорический нагрев в пять раз,
        \item 23 — изобарическое расширение, при котором температура растёт в три раза,
        \item 31 — процесс, график которого в $PV$-координатах является отрезком прямой.
    \end{itemize}
    Бонус: замените цикл 1231 циклом, в котором 12 — изохорический нагрев в пять раз, 23 — процесс, график которого в $PV$-координатах является отрезком прямой, 31 — изобарическое охлаждение, при котором температура падает в пять раз.
}
\answer{%
    \begin{align*}
    A_{12} &= 0, \Delta U_{12} > 0, \implies Q_{12} = A_{12} + \Delta U_{12} > 0.
    \\
    A_{23} &> 0, \Delta U_{23} > 0, \implies Q_{23} = A_{23} + \Delta U_{23} > 0, \\
    A_{31} &= 0, \Delta U_{31} < 0, \implies Q_{31} = A_{31} + \Delta U_{31} < 0.
    \\
    P_1V_1 &= \nu R T_1, P_2V_2 = \nu R T_2, P_3V_3 = \nu R T_3 \text{ — уравнения состояния идеального газа}, \\
    &\text{Пусть $P_0$, $V_0$, $T_0$ — давление, объём и температура в точке 1 (минимальные во всём цикле):} \\
    P_1 &= P_0, P_2 = P_3, V_1 = V_2 = V_0, \text{остальные соотношения нужно считать} \\
    T_2 &= 5T_1 = 5T_0 \text{(по условию)} \implies \frac{P_2}{P_1} = \frac{P_2V_0}{P_1V_0} = \frac{P_2 V_2}{P_1 V_1}= \frac{\nu R T_2}{\nu R T_1} = \frac{T_2}{T_1} = 5 \implies P_2 = 5 P_1 = 5 P_0, \\
    T_3 &= 3T_2 = 15T_0 \text{(по условию)} \implies \frac{V_3}{V_2} = \frac{P_3V_3}{P_2V_2}= \frac{\nu R T_3}{\nu R T_2} = \frac{T_3}{T_2} = 3 \implies V_3 = 3 V_2 = 3 V_0.
    \\
    A_\text{цикл} &= \frac 12 (3P_0 - P_0)(5V_0 - V_0) = \frac 12 \cdot 8 \cdot P_0V_0, \\
    A_{23} &= 5P_0 \cdot (3V_0 - V_0) = 10P_0V_0, \\
    \Delta U_{23} &= \frac 32 \nu R T_3 - \frac 32 \nu R T_2 = \frac 32 P_3 V_3 - \frac 32 P_2 V_2 = \frac 32 \cdot 5 P_0 \cdot 3 V_0 -  \frac 32 \cdot 5 P_0 \cdot V_0 = \frac 32 \cdot 10 \cdot P_0V_0, \\
    \Delta U_{12} &= \frac 32 \nu R T_2 - \frac 32 \nu R T_1 = \frac 32 P_2 V_2 - \frac 32 P_1 V_1 = \frac 32 \cdot 5 P_0V_0 - \frac 32 P_0V_0 = \frac 32 \cdot 4 \cdot P_0V_0.
    \\
    \eta &= \frac{A_\text{цикл}}{Q_+} = \frac{A_\text{цикл}}{Q_{12} + Q_{23}}  = \frac{A_\text{цикл}}{A_{12} + \Delta U_{12} + A_{23} + \Delta U_{23}} =  \\
     &= \frac{\frac 12 \cdot 8 \cdot P_0V_0}{0 + \frac 32 \cdot 4 \cdot P_0V_0 + 10P_0V_0 + \frac 32 \cdot 10 \cdot P_0V_0} = \frac{\frac 12 \cdot 8}{\frac 32 \cdot 4 + 10 + \frac 32 \cdot 10} = \frac4{31} \approx 0.129.
    \end{align*}


        График процесса не в масштабе (эта часть пока не готова и сделать автоматически аккуратно сложно), но с верными подписями (а для решения этого достаточно):

        \begin{tikzpicture}[thick]
            \draw[-{Latex}] (0, 0) -- (0, 7) node[above left] {$P$};
            \draw[-{Latex}] (0, 0) -- (10, 0) node[right] {$V$};

            \draw[dashed] (0, 2) node[left] {$P_1 = P_0$} -| (3, 0) node[below] {$V_1 = V_2 = V_0$};
            \draw[dashed] (0, 6) node[left] {$P_2 = P_3 = 5P_0$} -| (9, 0) node[below] {$V_3 = 3V_0$};

            \draw (3, 2) node[above left]{1} node[below left]{$T_1 = T_0$}
                   (3, 6) node[below left]{2} node[above]{$T_2 = 5T_0$}
                   (9, 6) node[above right]{3} node[below right]{$T_3 = 15T_0$};
            \draw[midar] (3, 2) -- (3, 6);
            \draw[midar] (3, 6) -- (9, 6);
            \draw[midar] (9, 6) -- (3, 2);
        \end{tikzpicture}

        Решение бонуса:
        \begin{align*}
            A_{12} &= 0, \Delta U_{12} > 0, \implies Q_{12} = A_{12} + \Delta U_{12} > 0, \\
            A_{23} &> 0, \Delta U_{23} \text{ — ничего нельзя сказать, нужно исследовать отдельно}, \\
            A_{31} &< 0, \Delta U_{31} < 0, \implies Q_{31} = A_{31} + \Delta U_{31} < 0.
            \\
        \end{align*}

        Уравнения состояния идеального газа для точек 1, 2, 3: $P_1V_1 = \nu R T_1, P_2V_2 = \nu R T_2, P_3V_3 = \nu R T_3$.
        Пусть $P_0$, $V_0$, $T_0$ — давление, объём и температура в точке 1 (минимальные во всём цикле).

        12 --- изохора, $\frac{P_1V_1}{T_1} = \nu R = \frac{P_2V_2}{T_2}, V_2=V_1=V_0 \implies \frac{P_1}{T_1} =  \frac{P_2}{T_2} \implies P_2 = P_1 \frac{T_2}{T_1} = 5P_0$,

        31 --- изобара, $\frac{P_1V_1}{T_1} = \nu R = \frac{P_3V_3}{T_3}, P_3=P_1=P_0 \implies \frac{V_3}{T_3} =  \frac{V_1}{T_1} \implies V_3 = V_1 \frac{T_3}{T_1} = 5V_0$,

        Таким образом, используя новые обозначения, состояния газа в точках 1, 2 и 3 описываются макропараметрами $(P_0, V_0, T_0), (5P_0, V_0, 5T_0), (P_0, 5V_0, 5T_0)$ соответственно.

        \begin{tikzpicture}[thick]
            \draw[-{Latex}] (0, 0) -- (0, 7) node[above left] {$P$};
            \draw[-{Latex}] (0, 0) -- (10, 0) node[right] {$V$};

            \draw[dashed] (0, 2) node[left] {$P_1 = P_3 = P_0$} -| (9, 0) node[below] {$V_3 = 5V_0$};
            \draw[dashed] (0, 6) node[left] {$P_2 = 5P_0$} -| (3, 0) node[below] {$V_1 = V_2 = V_0$};

            \draw[dashed] (0, 5) node[left] {$P$} -| (4.5, 0) node[below] {$V$};
            \draw[dashed] (0, 4.6) node[left] {$P'$} -| (5.1, 0) node[below] {$V'$};

            \draw (3, 2) node[above left]{1} node[below left]{$T_1 = T_0$}
                   (3, 6) node[below left]{2} node[above]{$T_2 = 5T_0$}
                   (9, 2) node[above right]{3} node[below right]{$T_3 = 5T_0$};
            \draw[midar] (3, 2) -- (3, 6);
            \draw[midar] (3, 6) -- (9, 2);
            \draw[midar] (9, 2) -- (3, 2);
            \draw   (4.5, 5) node[above right]{$T$} (5.1, 4.6) node[above right]{$T'$};
        \end{tikzpicture}


        Теперь рассмотрим отдельно процесс 23, к остальному вернёмся позже.
        Уравнение этой прямой в $PV$-координатах: $P(V) = 6P_0 - \frac{P_0}{V_0} V$.
        Это значит, что при изменении объёма на $\Delta V$ давление изменится на $\Delta P = - \frac{P_0}{V_0} \Delta V$, обратите внимание на знак.

        Рассмотрим произвольную точку в процессе 23 и дадим процессу ещё немного свершиться, при этом объём изменится на $\Delta V$, давление — на $\Delta P$, температура (иначе бы была гипербола, а не прямая) — на $\Delta T$,
        т.е.
        из состояния $(P, V, T)$ мы перешли в $(P', V', T')$, причём  $P' = P + \Delta P, V' = V + \Delta V, T' = T + \Delta T$.

        При этом изменится внутренняя энергия:
        \begin{align*}
        \Delta U
            &= U' - U = \frac 32 \nu R T' - \frac 32 \nu R T = \frac 32 (P+\Delta P) (V+\Delta V) - \frac 32 PV\\
            &= \frac 32 ((P+\Delta P) (V+\Delta V) - PV) = \frac 32 (P\Delta V + V \Delta P + \Delta P \Delta V).
        \end{align*}

        Рассмотрим малые изменения объёма, тогда и изменение давления будем малым (т.к.
        $\Delta P = - \frac{P_0}{V_0} \Delta V$),
        а третьим слагаемым в выражении для $\Delta U$  можно пренебречь по сравнению с двумя другими:
        два первых это малые величины, а третье — произведение двух малых.
        Тогда $\Delta U = \frac 32 (P\Delta V + V \Delta P)$.

        Работа газа при этом малом изменении объёма — это площадь трапеции (тут ещё раз пренебрегли малым слагаемым):
        $$A = \frac{P + P'}2 \Delta V = \cbr{P + \frac{\Delta P}2} \Delta V = P \Delta V.$$

        Подведённое количество теплоты, используя первое начало термодинамики, будет равно
        \begin{align*}
        Q
            &= \frac 32 (P\Delta V + V \Delta P) + P \Delta V =  \frac 52 P\Delta V + \frac 32 V \Delta P = \\
            &= \frac 52 P\Delta V + \frac 32 V \cdot \cbr{- \frac{P_0}{V_0} \Delta V} = \frac{\Delta V}2 \cdot \cbr{5P - \frac{P_0}{V_0} V} = \\
            &= \frac{\Delta V}2 \cdot \cbr{5 \cdot \cbr{6P_0 - \frac{P_0}{V_0} V} - \frac{P_0}{V_0} V}
             = \frac{\Delta V \cdot P_0}2 \cdot \cbr{5 \cdot 6 - 8\frac V{V_0}}.
        \end{align*}

        Таком образом, знак количества теплоты $Q$ на участке 23 зависит от конкретного значения $V$:
        \begin{itemize}
            \item $\Delta V > 0$ на всём участке 23, поскольку газ расширяется,
            \item $P > 0$ — всегда, у нас идеальный газ, удары о стенки сосуда абсолютно упругие, а молекулы не взаимодействуют и поэтому давление только положительно,
            \item если $5 \cdot 6 - 8\frac V{V_0} > 0$ — тепло подводят, если же меньше нуля — отводят.
        \end{itemize}
        Решая последнее неравенство, получаем конкретное значение $V^*$: при $V < V^*$ тепло подводят, далее~— отводят.
        Тут *~--- некоторая точка между точками 2 и 3, конкретные значения надо досчитать:
        $$V^* = V_0 \cdot \frac{5 \cdot 6}8 = \frac{15}4 \cdot V_0 \implies P^* = 6P_0 - \frac{P_0}{V_0} V^* = \ldots = \frac94 \cdot P_0.$$

        Т.е.
        чтобы вычислить $Q_+$, надо сложить количества теплоты на участке 12 и лишь части участка 23 — участке 2*,
        той его части где это количество теплоты положительно.
        Имеем: $Q_+ = Q_{12} + Q_{2*}$.

        Теперь возвращаемся к циклу целиком и получаем:
        \begin{align*}
        A_\text{цикл} &= \frac 12 \cdot (5P_0 - P_0) \cdot (5V_0 - V_0) = 8 \cdot P_0V_0, \\
        A_{2*} &= \frac{P^* + 5P_0}2 \cdot (V^* - V_0)
            = \frac{\frac94 \cdot P_0 + 5P_0}2 \cdot \cbr{\frac{15}4 \cdot V_0 - V_0}
            = \ldots = \frac{319}{32} \cdot P_0 V_0, \\
        \Delta U_{2*} &= \frac 32 \nu R T^* - \frac 32 \nu R T_2 = \frac 32 (P^*V^* - P_0 \cdot 5V_0)
            = \frac 32 \cbr{\frac94 \cdot P_0 \cdot \frac{15}4 \cdot V_0 - P_0 \cdot 5V_0}
            = \frac{165}{32} \cdot P_0 V_0, \\
        \Delta U_{12} &= \frac 32 \nu R T_2 - \frac 32 \nu R T_1 = \frac 32 (5P_0V_0 - P_0V_0) = \ldots = 6 \cdot P_0 V_0, \\
        \eta &= \frac{A_\text{цикл}}{Q_+} = \frac{A_\text{цикл}}{Q_{12} + Q_{2*}}
            = \frac{A_\text{цикл}}{A_{12} + \Delta U_{12} + A_{2*} + \Delta U_{2*}} = \\
            &= \frac{8 \cdot P_0V_0}{0 + 6 \cdot P_0 V_0 + \frac{319}{32} \cdot P_0 V_0 + \frac{165}{32} \cdot P_0 V_0}
             = \frac{A_bonus_cycle:LaTeX}{6 + \frac{319}{32} + \frac{165}{32}}
             = \frac{64}{169} \leftarrow \text{вжух и готово!}
        \end{align*}
}
\solutionspace{360pt}

\tasknumber{2}%
\task{%
    При температуре $30\celsius$ относительная влажность воздуха составляет $60\%$.
    \begin{itemize}
        \item Определите точку росы для этого воздуха.
        \item Какой станет относительная влажность этого воздуха, если нагреть его до $40\celsius$?
    \end{itemize}
}
\answer{%
    \begin{align*}
    &\text{Значения плотности насыщенного водяного пара определяем по таблице:} \\
    &\rho_{\text{нас.
    пара 30} \celsius} = 30{,}300\,\frac{\text{г}}{\text{м}^{3}}, \rho_{\text{нас.
    пара 40} \celsius} = 51{,}200\,\frac{\text{г}}{\text{м}^{3}}.
    \\
    \varphi_1 &= \frac{\rho_\text{пара}}{\rho_{\text{нас.
    пара 30} \celsius}} \implies {\rho_\text{пара}} = \rho_{\text{нас.
    пара 30} \celsius} \cdot \varphi_1 = 30{,}300\,\frac{\text{г}}{\text{м}^{3}} \cdot 0{,}60 = 18{,}180\,\frac{\text{г}}{\text{м}^{3}}.
    \\
    &\text{По таблице определяем, при какой температуре пар с такой плотностью станет насыщенным:}  \\
    t_\text{росы} &= 20{,}9\celsius, \\
    \varphi_2 &= \frac{\rho_\text{пара}}{\rho_{\text{нас.
    пара 40} \celsius}} = \frac{\rho_{\text{нас.
    пара 30} \celsius} \cdot \varphi_1}{\rho_{\text{нас.
    пара 40} \celsius}}= \varphi_1 \cdot \frac{\rho_{\text{нас.
    пара 30} \celsius}}{\rho_{\text{нас.
    пара 40} \celsius}} = 0{,}60 \cdot \frac{30{,}300\,\frac{\text{г}}{\text{м}^{3}}}{51{,}200\,\frac{\text{г}}{\text{м}^{3}}} = 0{,}355 \approx 35{,}5\%.
    \end{align*}
}
\solutionspace{80pt}

\tasknumber{3}%
\task{%
    Из уравнения состояния идеального газа выведите или выразите...
    \begin{enumerate}
        \item объём,
        \item молярную массу,
        \item плотность газа.
    \end{enumerate}
}

\tasknumber{4}%
\task{%
    Запишите формулы и рядом с каждой физичической величиной укажите её название и единицы измерения в СИ:
    \begin{enumerate}
        \item первое начало термодинамики,
        \item внутренняя энергия идеального одноатомного газа.
    \end{enumerate}
}

\variantsplitter

\addpersonalvariant{Герман Говоров}

\tasknumber{1}%
\task{%
    Определите КПД (оставив ответ точным в виде нескоратимой дроби) цикла 1231, рабочим телом которого является идеальный одноатомный газ, если
    \begin{itemize}
        \item 12 — изохорический нагрев в два раза,
        \item 23 — изобарическое расширение, при котором температура растёт в четыре раза,
        \item 31 — процесс, график которого в $PV$-координатах является отрезком прямой.
    \end{itemize}
    Бонус: замените цикл 1231 циклом, в котором 12 — изохорический нагрев в два раза, 23 — процесс, график которого в $PV$-координатах является отрезком прямой, 31 — изобарическое охлаждение, при котором температура падает в два раза.
}
\answer{%
    \begin{align*}
    A_{12} &= 0, \Delta U_{12} > 0, \implies Q_{12} = A_{12} + \Delta U_{12} > 0.
    \\
    A_{23} &> 0, \Delta U_{23} > 0, \implies Q_{23} = A_{23} + \Delta U_{23} > 0, \\
    A_{31} &= 0, \Delta U_{31} < 0, \implies Q_{31} = A_{31} + \Delta U_{31} < 0.
    \\
    P_1V_1 &= \nu R T_1, P_2V_2 = \nu R T_2, P_3V_3 = \nu R T_3 \text{ — уравнения состояния идеального газа}, \\
    &\text{Пусть $P_0$, $V_0$, $T_0$ — давление, объём и температура в точке 1 (минимальные во всём цикле):} \\
    P_1 &= P_0, P_2 = P_3, V_1 = V_2 = V_0, \text{остальные соотношения нужно считать} \\
    T_2 &= 2T_1 = 2T_0 \text{(по условию)} \implies \frac{P_2}{P_1} = \frac{P_2V_0}{P_1V_0} = \frac{P_2 V_2}{P_1 V_1}= \frac{\nu R T_2}{\nu R T_1} = \frac{T_2}{T_1} = 2 \implies P_2 = 2 P_1 = 2 P_0, \\
    T_3 &= 4T_2 = 8T_0 \text{(по условию)} \implies \frac{V_3}{V_2} = \frac{P_3V_3}{P_2V_2}= \frac{\nu R T_3}{\nu R T_2} = \frac{T_3}{T_2} = 4 \implies V_3 = 4 V_2 = 4 V_0.
    \\
    A_\text{цикл} &= \frac 12 (4P_0 - P_0)(2V_0 - V_0) = \frac 12 \cdot 3 \cdot P_0V_0, \\
    A_{23} &= 2P_0 \cdot (4V_0 - V_0) = 6P_0V_0, \\
    \Delta U_{23} &= \frac 32 \nu R T_3 - \frac 32 \nu R T_2 = \frac 32 P_3 V_3 - \frac 32 P_2 V_2 = \frac 32 \cdot 2 P_0 \cdot 4 V_0 -  \frac 32 \cdot 2 P_0 \cdot V_0 = \frac 32 \cdot 6 \cdot P_0V_0, \\
    \Delta U_{12} &= \frac 32 \nu R T_2 - \frac 32 \nu R T_1 = \frac 32 P_2 V_2 - \frac 32 P_1 V_1 = \frac 32 \cdot 2 P_0V_0 - \frac 32 P_0V_0 = \frac 32 \cdot 1 \cdot P_0V_0.
    \\
    \eta &= \frac{A_\text{цикл}}{Q_+} = \frac{A_\text{цикл}}{Q_{12} + Q_{23}}  = \frac{A_\text{цикл}}{A_{12} + \Delta U_{12} + A_{23} + \Delta U_{23}} =  \\
     &= \frac{\frac 12 \cdot 3 \cdot P_0V_0}{0 + \frac 32 \cdot 1 \cdot P_0V_0 + 6P_0V_0 + \frac 32 \cdot 6 \cdot P_0V_0} = \frac{\frac 12 \cdot 3}{\frac 32 \cdot 1 + 6 + \frac 32 \cdot 6} = \frac1{11} \approx 0.091.
    \end{align*}


        График процесса не в масштабе (эта часть пока не готова и сделать автоматически аккуратно сложно), но с верными подписями (а для решения этого достаточно):

        \begin{tikzpicture}[thick]
            \draw[-{Latex}] (0, 0) -- (0, 7) node[above left] {$P$};
            \draw[-{Latex}] (0, 0) -- (10, 0) node[right] {$V$};

            \draw[dashed] (0, 2) node[left] {$P_1 = P_0$} -| (3, 0) node[below] {$V_1 = V_2 = V_0$};
            \draw[dashed] (0, 6) node[left] {$P_2 = P_3 = 2P_0$} -| (9, 0) node[below] {$V_3 = 4V_0$};

            \draw (3, 2) node[above left]{1} node[below left]{$T_1 = T_0$}
                   (3, 6) node[below left]{2} node[above]{$T_2 = 2T_0$}
                   (9, 6) node[above right]{3} node[below right]{$T_3 = 8T_0$};
            \draw[midar] (3, 2) -- (3, 6);
            \draw[midar] (3, 6) -- (9, 6);
            \draw[midar] (9, 6) -- (3, 2);
        \end{tikzpicture}

        Решение бонуса:
        \begin{align*}
            A_{12} &= 0, \Delta U_{12} > 0, \implies Q_{12} = A_{12} + \Delta U_{12} > 0, \\
            A_{23} &> 0, \Delta U_{23} \text{ — ничего нельзя сказать, нужно исследовать отдельно}, \\
            A_{31} &< 0, \Delta U_{31} < 0, \implies Q_{31} = A_{31} + \Delta U_{31} < 0.
            \\
        \end{align*}

        Уравнения состояния идеального газа для точек 1, 2, 3: $P_1V_1 = \nu R T_1, P_2V_2 = \nu R T_2, P_3V_3 = \nu R T_3$.
        Пусть $P_0$, $V_0$, $T_0$ — давление, объём и температура в точке 1 (минимальные во всём цикле).

        12 --- изохора, $\frac{P_1V_1}{T_1} = \nu R = \frac{P_2V_2}{T_2}, V_2=V_1=V_0 \implies \frac{P_1}{T_1} =  \frac{P_2}{T_2} \implies P_2 = P_1 \frac{T_2}{T_1} = 2P_0$,

        31 --- изобара, $\frac{P_1V_1}{T_1} = \nu R = \frac{P_3V_3}{T_3}, P_3=P_1=P_0 \implies \frac{V_3}{T_3} =  \frac{V_1}{T_1} \implies V_3 = V_1 \frac{T_3}{T_1} = 2V_0$,

        Таким образом, используя новые обозначения, состояния газа в точках 1, 2 и 3 описываются макропараметрами $(P_0, V_0, T_0), (2P_0, V_0, 2T_0), (P_0, 2V_0, 2T_0)$ соответственно.

        \begin{tikzpicture}[thick]
            \draw[-{Latex}] (0, 0) -- (0, 7) node[above left] {$P$};
            \draw[-{Latex}] (0, 0) -- (10, 0) node[right] {$V$};

            \draw[dashed] (0, 2) node[left] {$P_1 = P_3 = P_0$} -| (9, 0) node[below] {$V_3 = 2V_0$};
            \draw[dashed] (0, 6) node[left] {$P_2 = 2P_0$} -| (3, 0) node[below] {$V_1 = V_2 = V_0$};

            \draw[dashed] (0, 5) node[left] {$P$} -| (4.5, 0) node[below] {$V$};
            \draw[dashed] (0, 4.6) node[left] {$P'$} -| (5.1, 0) node[below] {$V'$};

            \draw (3, 2) node[above left]{1} node[below left]{$T_1 = T_0$}
                   (3, 6) node[below left]{2} node[above]{$T_2 = 2T_0$}
                   (9, 2) node[above right]{3} node[below right]{$T_3 = 2T_0$};
            \draw[midar] (3, 2) -- (3, 6);
            \draw[midar] (3, 6) -- (9, 2);
            \draw[midar] (9, 2) -- (3, 2);
            \draw   (4.5, 5) node[above right]{$T$} (5.1, 4.6) node[above right]{$T'$};
        \end{tikzpicture}


        Теперь рассмотрим отдельно процесс 23, к остальному вернёмся позже.
        Уравнение этой прямой в $PV$-координатах: $P(V) = 3P_0 - \frac{P_0}{V_0} V$.
        Это значит, что при изменении объёма на $\Delta V$ давление изменится на $\Delta P = - \frac{P_0}{V_0} \Delta V$, обратите внимание на знак.

        Рассмотрим произвольную точку в процессе 23 и дадим процессу ещё немного свершиться, при этом объём изменится на $\Delta V$, давление — на $\Delta P$, температура (иначе бы была гипербола, а не прямая) — на $\Delta T$,
        т.е.
        из состояния $(P, V, T)$ мы перешли в $(P', V', T')$, причём  $P' = P + \Delta P, V' = V + \Delta V, T' = T + \Delta T$.

        При этом изменится внутренняя энергия:
        \begin{align*}
        \Delta U
            &= U' - U = \frac 32 \nu R T' - \frac 32 \nu R T = \frac 32 (P+\Delta P) (V+\Delta V) - \frac 32 PV\\
            &= \frac 32 ((P+\Delta P) (V+\Delta V) - PV) = \frac 32 (P\Delta V + V \Delta P + \Delta P \Delta V).
        \end{align*}

        Рассмотрим малые изменения объёма, тогда и изменение давления будем малым (т.к.
        $\Delta P = - \frac{P_0}{V_0} \Delta V$),
        а третьим слагаемым в выражении для $\Delta U$  можно пренебречь по сравнению с двумя другими:
        два первых это малые величины, а третье — произведение двух малых.
        Тогда $\Delta U = \frac 32 (P\Delta V + V \Delta P)$.

        Работа газа при этом малом изменении объёма — это площадь трапеции (тут ещё раз пренебрегли малым слагаемым):
        $$A = \frac{P + P'}2 \Delta V = \cbr{P + \frac{\Delta P}2} \Delta V = P \Delta V.$$

        Подведённое количество теплоты, используя первое начало термодинамики, будет равно
        \begin{align*}
        Q
            &= \frac 32 (P\Delta V + V \Delta P) + P \Delta V =  \frac 52 P\Delta V + \frac 32 V \Delta P = \\
            &= \frac 52 P\Delta V + \frac 32 V \cdot \cbr{- \frac{P_0}{V_0} \Delta V} = \frac{\Delta V}2 \cdot \cbr{5P - \frac{P_0}{V_0} V} = \\
            &= \frac{\Delta V}2 \cdot \cbr{5 \cdot \cbr{3P_0 - \frac{P_0}{V_0} V} - \frac{P_0}{V_0} V}
             = \frac{\Delta V \cdot P_0}2 \cdot \cbr{5 \cdot 3 - 8\frac V{V_0}}.
        \end{align*}

        Таком образом, знак количества теплоты $Q$ на участке 23 зависит от конкретного значения $V$:
        \begin{itemize}
            \item $\Delta V > 0$ на всём участке 23, поскольку газ расширяется,
            \item $P > 0$ — всегда, у нас идеальный газ, удары о стенки сосуда абсолютно упругие, а молекулы не взаимодействуют и поэтому давление только положительно,
            \item если $5 \cdot 3 - 8\frac V{V_0} > 0$ — тепло подводят, если же меньше нуля — отводят.
        \end{itemize}
        Решая последнее неравенство, получаем конкретное значение $V^*$: при $V < V^*$ тепло подводят, далее~— отводят.
        Тут *~--- некоторая точка между точками 2 и 3, конкретные значения надо досчитать:
        $$V^* = V_0 \cdot \frac{5 \cdot 3}8 = \frac{15}8 \cdot V_0 \implies P^* = 3P_0 - \frac{P_0}{V_0} V^* = \ldots = \frac98 \cdot P_0.$$

        Т.е.
        чтобы вычислить $Q_+$, надо сложить количества теплоты на участке 12 и лишь части участка 23 — участке 2*,
        той его части где это количество теплоты положительно.
        Имеем: $Q_+ = Q_{12} + Q_{2*}$.

        Теперь возвращаемся к циклу целиком и получаем:
        \begin{align*}
        A_\text{цикл} &= \frac 12 \cdot (2P_0 - P_0) \cdot (2V_0 - V_0) = \frac12 \cdot P_0V_0, \\
        A_{2*} &= \frac{P^* + 2P_0}2 \cdot (V^* - V_0)
            = \frac{\frac98 \cdot P_0 + 2P_0}2 \cdot \cbr{\frac{15}8 \cdot V_0 - V_0}
            = \ldots = \frac{175}{128} \cdot P_0 V_0, \\
        \Delta U_{2*} &= \frac 32 \nu R T^* - \frac 32 \nu R T_2 = \frac 32 (P^*V^* - P_0 \cdot 2V_0)
            = \frac 32 \cbr{\frac98 \cdot P_0 \cdot \frac{15}8 \cdot V_0 - P_0 \cdot 2V_0}
            = \frac{21}{128} \cdot P_0 V_0, \\
        \Delta U_{12} &= \frac 32 \nu R T_2 - \frac 32 \nu R T_1 = \frac 32 (2P_0V_0 - P_0V_0) = \ldots = \frac32 \cdot P_0 V_0, \\
        \eta &= \frac{A_\text{цикл}}{Q_+} = \frac{A_\text{цикл}}{Q_{12} + Q_{2*}}
            = \frac{A_\text{цикл}}{A_{12} + \Delta U_{12} + A_{2*} + \Delta U_{2*}} = \\
            &= \frac{\frac12 \cdot P_0V_0}{0 + \frac32 \cdot P_0 V_0 + \frac{175}{128} \cdot P_0 V_0 + \frac{21}{128} \cdot P_0 V_0}
             = \frac{A_bonus_cycle:LaTeX}{\frac32 + \frac{175}{128} + \frac{21}{128}}
             = \frac{16}{97} \leftarrow \text{вжух и готово!}
        \end{align*}
}
\solutionspace{360pt}

\tasknumber{2}%
\task{%
    При температуре $15\celsius$ относительная влажность воздуха составляет $75\%$.
    \begin{itemize}
        \item Определите точку росы для этого воздуха.
        \item Какой станет относительная влажность этого воздуха, если нагреть его до $80\celsius$?
    \end{itemize}
}
\answer{%
    \begin{align*}
    &\text{Значения плотности насыщенного водяного пара определяем по таблице:} \\
    &\rho_{\text{нас.
    пара 15} \celsius} = 12{,}800\,\frac{\text{г}}{\text{м}^{3}}, \rho_{\text{нас.
    пара 80} \celsius} = 293{,}000\,\frac{\text{г}}{\text{м}^{3}}.
    \\
    \varphi_1 &= \frac{\rho_\text{пара}}{\rho_{\text{нас.
    пара 15} \celsius}} \implies {\rho_\text{пара}} = \rho_{\text{нас.
    пара 15} \celsius} \cdot \varphi_1 = 12{,}800\,\frac{\text{г}}{\text{м}^{3}} \cdot 0{,}75 = 9{,}600\,\frac{\text{г}}{\text{м}^{3}}.
    \\
    &\text{По таблице определяем, при какой температуре пар с такой плотностью станет насыщенным:}  \\
    t_\text{росы} &= 10{,}3\celsius, \\
    \varphi_2 &= \frac{\rho_\text{пара}}{\rho_{\text{нас.
    пара 80} \celsius}} = \frac{\rho_{\text{нас.
    пара 15} \celsius} \cdot \varphi_1}{\rho_{\text{нас.
    пара 80} \celsius}}= \varphi_1 \cdot \frac{\rho_{\text{нас.
    пара 15} \celsius}}{\rho_{\text{нас.
    пара 80} \celsius}} = 0{,}75 \cdot \frac{12{,}800\,\frac{\text{г}}{\text{м}^{3}}}{293{,}000\,\frac{\text{г}}{\text{м}^{3}}} = 0{,}033 \approx 3{,}3\%.
    \end{align*}
}
\solutionspace{80pt}

\tasknumber{3}%
\task{%
    Из уравнения состояния идеального газа выведите или выразите...
    \begin{enumerate}
        \item объём,
        \item молярную массу,
        \item плотность газа.
    \end{enumerate}
}

\tasknumber{4}%
\task{%
    Запишите формулы и рядом с каждой физичической величиной укажите её название и единицы измерения в СИ:
    \begin{enumerate}
        \item первое начало термодинамики,
        \item внутренняя энергия идеального одноатомного газа.
    \end{enumerate}
}

\variantsplitter

\addpersonalvariant{София Журавлёва}

\tasknumber{1}%
\task{%
    Определите КПД (оставив ответ точным в виде нескоратимой дроби) цикла 1231, рабочим телом которого является идеальный одноатомный газ, если
    \begin{itemize}
        \item 12 — изохорический нагрев в три раза,
        \item 23 — изобарическое расширение, при котором температура растёт в четыре раза,
        \item 31 — процесс, график которого в $PV$-координатах является отрезком прямой.
    \end{itemize}
    Бонус: замените цикл 1231 циклом, в котором 12 — изохорический нагрев в три раза, 23 — процесс, график которого в $PV$-координатах является отрезком прямой, 31 — изобарическое охлаждение, при котором температура падает в три раза.
}
\answer{%
    \begin{align*}
    A_{12} &= 0, \Delta U_{12} > 0, \implies Q_{12} = A_{12} + \Delta U_{12} > 0.
    \\
    A_{23} &> 0, \Delta U_{23} > 0, \implies Q_{23} = A_{23} + \Delta U_{23} > 0, \\
    A_{31} &= 0, \Delta U_{31} < 0, \implies Q_{31} = A_{31} + \Delta U_{31} < 0.
    \\
    P_1V_1 &= \nu R T_1, P_2V_2 = \nu R T_2, P_3V_3 = \nu R T_3 \text{ — уравнения состояния идеального газа}, \\
    &\text{Пусть $P_0$, $V_0$, $T_0$ — давление, объём и температура в точке 1 (минимальные во всём цикле):} \\
    P_1 &= P_0, P_2 = P_3, V_1 = V_2 = V_0, \text{остальные соотношения нужно считать} \\
    T_2 &= 3T_1 = 3T_0 \text{(по условию)} \implies \frac{P_2}{P_1} = \frac{P_2V_0}{P_1V_0} = \frac{P_2 V_2}{P_1 V_1}= \frac{\nu R T_2}{\nu R T_1} = \frac{T_2}{T_1} = 3 \implies P_2 = 3 P_1 = 3 P_0, \\
    T_3 &= 4T_2 = 12T_0 \text{(по условию)} \implies \frac{V_3}{V_2} = \frac{P_3V_3}{P_2V_2}= \frac{\nu R T_3}{\nu R T_2} = \frac{T_3}{T_2} = 4 \implies V_3 = 4 V_2 = 4 V_0.
    \\
    A_\text{цикл} &= \frac 12 (4P_0 - P_0)(3V_0 - V_0) = \frac 12 \cdot 6 \cdot P_0V_0, \\
    A_{23} &= 3P_0 \cdot (4V_0 - V_0) = 9P_0V_0, \\
    \Delta U_{23} &= \frac 32 \nu R T_3 - \frac 32 \nu R T_2 = \frac 32 P_3 V_3 - \frac 32 P_2 V_2 = \frac 32 \cdot 3 P_0 \cdot 4 V_0 -  \frac 32 \cdot 3 P_0 \cdot V_0 = \frac 32 \cdot 9 \cdot P_0V_0, \\
    \Delta U_{12} &= \frac 32 \nu R T_2 - \frac 32 \nu R T_1 = \frac 32 P_2 V_2 - \frac 32 P_1 V_1 = \frac 32 \cdot 3 P_0V_0 - \frac 32 P_0V_0 = \frac 32 \cdot 2 \cdot P_0V_0.
    \\
    \eta &= \frac{A_\text{цикл}}{Q_+} = \frac{A_\text{цикл}}{Q_{12} + Q_{23}}  = \frac{A_\text{цикл}}{A_{12} + \Delta U_{12} + A_{23} + \Delta U_{23}} =  \\
     &= \frac{\frac 12 \cdot 6 \cdot P_0V_0}{0 + \frac 32 \cdot 2 \cdot P_0V_0 + 9P_0V_0 + \frac 32 \cdot 9 \cdot P_0V_0} = \frac{\frac 12 \cdot 6}{\frac 32 \cdot 2 + 9 + \frac 32 \cdot 9} = \frac2{17} \approx 0.118.
    \end{align*}


        График процесса не в масштабе (эта часть пока не готова и сделать автоматически аккуратно сложно), но с верными подписями (а для решения этого достаточно):

        \begin{tikzpicture}[thick]
            \draw[-{Latex}] (0, 0) -- (0, 7) node[above left] {$P$};
            \draw[-{Latex}] (0, 0) -- (10, 0) node[right] {$V$};

            \draw[dashed] (0, 2) node[left] {$P_1 = P_0$} -| (3, 0) node[below] {$V_1 = V_2 = V_0$};
            \draw[dashed] (0, 6) node[left] {$P_2 = P_3 = 3P_0$} -| (9, 0) node[below] {$V_3 = 4V_0$};

            \draw (3, 2) node[above left]{1} node[below left]{$T_1 = T_0$}
                   (3, 6) node[below left]{2} node[above]{$T_2 = 3T_0$}
                   (9, 6) node[above right]{3} node[below right]{$T_3 = 12T_0$};
            \draw[midar] (3, 2) -- (3, 6);
            \draw[midar] (3, 6) -- (9, 6);
            \draw[midar] (9, 6) -- (3, 2);
        \end{tikzpicture}

        Решение бонуса:
        \begin{align*}
            A_{12} &= 0, \Delta U_{12} > 0, \implies Q_{12} = A_{12} + \Delta U_{12} > 0, \\
            A_{23} &> 0, \Delta U_{23} \text{ — ничего нельзя сказать, нужно исследовать отдельно}, \\
            A_{31} &< 0, \Delta U_{31} < 0, \implies Q_{31} = A_{31} + \Delta U_{31} < 0.
            \\
        \end{align*}

        Уравнения состояния идеального газа для точек 1, 2, 3: $P_1V_1 = \nu R T_1, P_2V_2 = \nu R T_2, P_3V_3 = \nu R T_3$.
        Пусть $P_0$, $V_0$, $T_0$ — давление, объём и температура в точке 1 (минимальные во всём цикле).

        12 --- изохора, $\frac{P_1V_1}{T_1} = \nu R = \frac{P_2V_2}{T_2}, V_2=V_1=V_0 \implies \frac{P_1}{T_1} =  \frac{P_2}{T_2} \implies P_2 = P_1 \frac{T_2}{T_1} = 3P_0$,

        31 --- изобара, $\frac{P_1V_1}{T_1} = \nu R = \frac{P_3V_3}{T_3}, P_3=P_1=P_0 \implies \frac{V_3}{T_3} =  \frac{V_1}{T_1} \implies V_3 = V_1 \frac{T_3}{T_1} = 3V_0$,

        Таким образом, используя новые обозначения, состояния газа в точках 1, 2 и 3 описываются макропараметрами $(P_0, V_0, T_0), (3P_0, V_0, 3T_0), (P_0, 3V_0, 3T_0)$ соответственно.

        \begin{tikzpicture}[thick]
            \draw[-{Latex}] (0, 0) -- (0, 7) node[above left] {$P$};
            \draw[-{Latex}] (0, 0) -- (10, 0) node[right] {$V$};

            \draw[dashed] (0, 2) node[left] {$P_1 = P_3 = P_0$} -| (9, 0) node[below] {$V_3 = 3V_0$};
            \draw[dashed] (0, 6) node[left] {$P_2 = 3P_0$} -| (3, 0) node[below] {$V_1 = V_2 = V_0$};

            \draw[dashed] (0, 5) node[left] {$P$} -| (4.5, 0) node[below] {$V$};
            \draw[dashed] (0, 4.6) node[left] {$P'$} -| (5.1, 0) node[below] {$V'$};

            \draw (3, 2) node[above left]{1} node[below left]{$T_1 = T_0$}
                   (3, 6) node[below left]{2} node[above]{$T_2 = 3T_0$}
                   (9, 2) node[above right]{3} node[below right]{$T_3 = 3T_0$};
            \draw[midar] (3, 2) -- (3, 6);
            \draw[midar] (3, 6) -- (9, 2);
            \draw[midar] (9, 2) -- (3, 2);
            \draw   (4.5, 5) node[above right]{$T$} (5.1, 4.6) node[above right]{$T'$};
        \end{tikzpicture}


        Теперь рассмотрим отдельно процесс 23, к остальному вернёмся позже.
        Уравнение этой прямой в $PV$-координатах: $P(V) = 4P_0 - \frac{P_0}{V_0} V$.
        Это значит, что при изменении объёма на $\Delta V$ давление изменится на $\Delta P = - \frac{P_0}{V_0} \Delta V$, обратите внимание на знак.

        Рассмотрим произвольную точку в процессе 23 и дадим процессу ещё немного свершиться, при этом объём изменится на $\Delta V$, давление — на $\Delta P$, температура (иначе бы была гипербола, а не прямая) — на $\Delta T$,
        т.е.
        из состояния $(P, V, T)$ мы перешли в $(P', V', T')$, причём  $P' = P + \Delta P, V' = V + \Delta V, T' = T + \Delta T$.

        При этом изменится внутренняя энергия:
        \begin{align*}
        \Delta U
            &= U' - U = \frac 32 \nu R T' - \frac 32 \nu R T = \frac 32 (P+\Delta P) (V+\Delta V) - \frac 32 PV\\
            &= \frac 32 ((P+\Delta P) (V+\Delta V) - PV) = \frac 32 (P\Delta V + V \Delta P + \Delta P \Delta V).
        \end{align*}

        Рассмотрим малые изменения объёма, тогда и изменение давления будем малым (т.к.
        $\Delta P = - \frac{P_0}{V_0} \Delta V$),
        а третьим слагаемым в выражении для $\Delta U$  можно пренебречь по сравнению с двумя другими:
        два первых это малые величины, а третье — произведение двух малых.
        Тогда $\Delta U = \frac 32 (P\Delta V + V \Delta P)$.

        Работа газа при этом малом изменении объёма — это площадь трапеции (тут ещё раз пренебрегли малым слагаемым):
        $$A = \frac{P + P'}2 \Delta V = \cbr{P + \frac{\Delta P}2} \Delta V = P \Delta V.$$

        Подведённое количество теплоты, используя первое начало термодинамики, будет равно
        \begin{align*}
        Q
            &= \frac 32 (P\Delta V + V \Delta P) + P \Delta V =  \frac 52 P\Delta V + \frac 32 V \Delta P = \\
            &= \frac 52 P\Delta V + \frac 32 V \cdot \cbr{- \frac{P_0}{V_0} \Delta V} = \frac{\Delta V}2 \cdot \cbr{5P - \frac{P_0}{V_0} V} = \\
            &= \frac{\Delta V}2 \cdot \cbr{5 \cdot \cbr{4P_0 - \frac{P_0}{V_0} V} - \frac{P_0}{V_0} V}
             = \frac{\Delta V \cdot P_0}2 \cdot \cbr{5 \cdot 4 - 8\frac V{V_0}}.
        \end{align*}

        Таком образом, знак количества теплоты $Q$ на участке 23 зависит от конкретного значения $V$:
        \begin{itemize}
            \item $\Delta V > 0$ на всём участке 23, поскольку газ расширяется,
            \item $P > 0$ — всегда, у нас идеальный газ, удары о стенки сосуда абсолютно упругие, а молекулы не взаимодействуют и поэтому давление только положительно,
            \item если $5 \cdot 4 - 8\frac V{V_0} > 0$ — тепло подводят, если же меньше нуля — отводят.
        \end{itemize}
        Решая последнее неравенство, получаем конкретное значение $V^*$: при $V < V^*$ тепло подводят, далее~— отводят.
        Тут *~--- некоторая точка между точками 2 и 3, конкретные значения надо досчитать:
        $$V^* = V_0 \cdot \frac{5 \cdot 4}8 = \frac52 \cdot V_0 \implies P^* = 4P_0 - \frac{P_0}{V_0} V^* = \ldots = \frac32 \cdot P_0.$$

        Т.е.
        чтобы вычислить $Q_+$, надо сложить количества теплоты на участке 12 и лишь части участка 23 — участке 2*,
        той его части где это количество теплоты положительно.
        Имеем: $Q_+ = Q_{12} + Q_{2*}$.

        Теперь возвращаемся к циклу целиком и получаем:
        \begin{align*}
        A_\text{цикл} &= \frac 12 \cdot (3P_0 - P_0) \cdot (3V_0 - V_0) = 2 \cdot P_0V_0, \\
        A_{2*} &= \frac{P^* + 3P_0}2 \cdot (V^* - V_0)
            = \frac{\frac32 \cdot P_0 + 3P_0}2 \cdot \cbr{\frac52 \cdot V_0 - V_0}
            = \ldots = \frac{27}8 \cdot P_0 V_0, \\
        \Delta U_{2*} &= \frac 32 \nu R T^* - \frac 32 \nu R T_2 = \frac 32 (P^*V^* - P_0 \cdot 3V_0)
            = \frac 32 \cbr{\frac32 \cdot P_0 \cdot \frac52 \cdot V_0 - P_0 \cdot 3V_0}
            = \frac98 \cdot P_0 V_0, \\
        \Delta U_{12} &= \frac 32 \nu R T_2 - \frac 32 \nu R T_1 = \frac 32 (3P_0V_0 - P_0V_0) = \ldots = 3 \cdot P_0 V_0, \\
        \eta &= \frac{A_\text{цикл}}{Q_+} = \frac{A_\text{цикл}}{Q_{12} + Q_{2*}}
            = \frac{A_\text{цикл}}{A_{12} + \Delta U_{12} + A_{2*} + \Delta U_{2*}} = \\
            &= \frac{2 \cdot P_0V_0}{0 + 3 \cdot P_0 V_0 + \frac{27}8 \cdot P_0 V_0 + \frac98 \cdot P_0 V_0}
             = \frac{A_bonus_cycle:LaTeX}{3 + \frac{27}8 + \frac98}
             = \frac4{15} \leftarrow \text{вжух и готово!}
        \end{align*}
}
\solutionspace{360pt}

\tasknumber{2}%
\task{%
    При температуре $20\celsius$ относительная влажность воздуха составляет $40\%$.
    \begin{itemize}
        \item Определите точку росы для этого воздуха.
        \item Какой станет относительная влажность этого воздуха, если нагреть его до $70\celsius$?
    \end{itemize}
}
\answer{%
    \begin{align*}
    &\text{Значения плотности насыщенного водяного пара определяем по таблице:} \\
    &\rho_{\text{нас.
    пара 20} \celsius} = 17{,}300\,\frac{\text{г}}{\text{м}^{3}}, \rho_{\text{нас.
    пара 70} \celsius} = 198{,}000\,\frac{\text{г}}{\text{м}^{3}}.
    \\
    \varphi_1 &= \frac{\rho_\text{пара}}{\rho_{\text{нас.
    пара 20} \celsius}} \implies {\rho_\text{пара}} = \rho_{\text{нас.
    пара 20} \celsius} \cdot \varphi_1 = 17{,}300\,\frac{\text{г}}{\text{м}^{3}} \cdot 0{,}40 = 6{,}920\,\frac{\text{г}}{\text{м}^{3}}.
    \\
    &\text{По таблице определяем, при какой температуре пар с такой плотностью станет насыщенным:}  \\
    t_\text{росы} &= 5{,}2\celsius, \\
    \varphi_2 &= \frac{\rho_\text{пара}}{\rho_{\text{нас.
    пара 70} \celsius}} = \frac{\rho_{\text{нас.
    пара 20} \celsius} \cdot \varphi_1}{\rho_{\text{нас.
    пара 70} \celsius}}= \varphi_1 \cdot \frac{\rho_{\text{нас.
    пара 20} \celsius}}{\rho_{\text{нас.
    пара 70} \celsius}} = 0{,}40 \cdot \frac{17{,}300\,\frac{\text{г}}{\text{м}^{3}}}{198{,}000\,\frac{\text{г}}{\text{м}^{3}}} = 0{,}035 \approx 3{,}5\%.
    \end{align*}
}
\solutionspace{80pt}

\tasknumber{3}%
\task{%
    Из уравнения состояния идеального газа выведите или выразите...
    \begin{enumerate}
        \item объём,
        \item молярную массу,
        \item плотность газа.
    \end{enumerate}
}

\tasknumber{4}%
\task{%
    Запишите формулы и рядом с каждой физичической величиной укажите её название и единицы измерения в СИ:
    \begin{enumerate}
        \item первое начало термодинамики,
        \item внутренняя энергия идеального одноатомного газа.
    \end{enumerate}
}

\variantsplitter

\addpersonalvariant{Константин Козлов}

\tasknumber{1}%
\task{%
    Определите КПД (оставив ответ точным в виде нескоратимой дроби) цикла 1231, рабочим телом которого является идеальный одноатомный газ, если
    \begin{itemize}
        \item 12 — изохорический нагрев в пять раз,
        \item 23 — изобарическое расширение, при котором температура растёт в пять раз,
        \item 31 — процесс, график которого в $PV$-координатах является отрезком прямой.
    \end{itemize}
    Бонус: замените цикл 1231 циклом, в котором 12 — изохорический нагрев в пять раз, 23 — процесс, график которого в $PV$-координатах является отрезком прямой, 31 — изобарическое охлаждение, при котором температура падает в пять раз.
}
\answer{%
    \begin{align*}
    A_{12} &= 0, \Delta U_{12} > 0, \implies Q_{12} = A_{12} + \Delta U_{12} > 0.
    \\
    A_{23} &> 0, \Delta U_{23} > 0, \implies Q_{23} = A_{23} + \Delta U_{23} > 0, \\
    A_{31} &= 0, \Delta U_{31} < 0, \implies Q_{31} = A_{31} + \Delta U_{31} < 0.
    \\
    P_1V_1 &= \nu R T_1, P_2V_2 = \nu R T_2, P_3V_3 = \nu R T_3 \text{ — уравнения состояния идеального газа}, \\
    &\text{Пусть $P_0$, $V_0$, $T_0$ — давление, объём и температура в точке 1 (минимальные во всём цикле):} \\
    P_1 &= P_0, P_2 = P_3, V_1 = V_2 = V_0, \text{остальные соотношения нужно считать} \\
    T_2 &= 5T_1 = 5T_0 \text{(по условию)} \implies \frac{P_2}{P_1} = \frac{P_2V_0}{P_1V_0} = \frac{P_2 V_2}{P_1 V_1}= \frac{\nu R T_2}{\nu R T_1} = \frac{T_2}{T_1} = 5 \implies P_2 = 5 P_1 = 5 P_0, \\
    T_3 &= 5T_2 = 25T_0 \text{(по условию)} \implies \frac{V_3}{V_2} = \frac{P_3V_3}{P_2V_2}= \frac{\nu R T_3}{\nu R T_2} = \frac{T_3}{T_2} = 5 \implies V_3 = 5 V_2 = 5 V_0.
    \\
    A_\text{цикл} &= \frac 12 (5P_0 - P_0)(5V_0 - V_0) = \frac 12 \cdot 16 \cdot P_0V_0, \\
    A_{23} &= 5P_0 \cdot (5V_0 - V_0) = 20P_0V_0, \\
    \Delta U_{23} &= \frac 32 \nu R T_3 - \frac 32 \nu R T_2 = \frac 32 P_3 V_3 - \frac 32 P_2 V_2 = \frac 32 \cdot 5 P_0 \cdot 5 V_0 -  \frac 32 \cdot 5 P_0 \cdot V_0 = \frac 32 \cdot 20 \cdot P_0V_0, \\
    \Delta U_{12} &= \frac 32 \nu R T_2 - \frac 32 \nu R T_1 = \frac 32 P_2 V_2 - \frac 32 P_1 V_1 = \frac 32 \cdot 5 P_0V_0 - \frac 32 P_0V_0 = \frac 32 \cdot 4 \cdot P_0V_0.
    \\
    \eta &= \frac{A_\text{цикл}}{Q_+} = \frac{A_\text{цикл}}{Q_{12} + Q_{23}}  = \frac{A_\text{цикл}}{A_{12} + \Delta U_{12} + A_{23} + \Delta U_{23}} =  \\
     &= \frac{\frac 12 \cdot 16 \cdot P_0V_0}{0 + \frac 32 \cdot 4 \cdot P_0V_0 + 20P_0V_0 + \frac 32 \cdot 20 \cdot P_0V_0} = \frac{\frac 12 \cdot 16}{\frac 32 \cdot 4 + 20 + \frac 32 \cdot 20} = \frac17 \approx 0.143.
    \end{align*}


        График процесса не в масштабе (эта часть пока не готова и сделать автоматически аккуратно сложно), но с верными подписями (а для решения этого достаточно):

        \begin{tikzpicture}[thick]
            \draw[-{Latex}] (0, 0) -- (0, 7) node[above left] {$P$};
            \draw[-{Latex}] (0, 0) -- (10, 0) node[right] {$V$};

            \draw[dashed] (0, 2) node[left] {$P_1 = P_0$} -| (3, 0) node[below] {$V_1 = V_2 = V_0$};
            \draw[dashed] (0, 6) node[left] {$P_2 = P_3 = 5P_0$} -| (9, 0) node[below] {$V_3 = 5V_0$};

            \draw (3, 2) node[above left]{1} node[below left]{$T_1 = T_0$}
                   (3, 6) node[below left]{2} node[above]{$T_2 = 5T_0$}
                   (9, 6) node[above right]{3} node[below right]{$T_3 = 25T_0$};
            \draw[midar] (3, 2) -- (3, 6);
            \draw[midar] (3, 6) -- (9, 6);
            \draw[midar] (9, 6) -- (3, 2);
        \end{tikzpicture}

        Решение бонуса:
        \begin{align*}
            A_{12} &= 0, \Delta U_{12} > 0, \implies Q_{12} = A_{12} + \Delta U_{12} > 0, \\
            A_{23} &> 0, \Delta U_{23} \text{ — ничего нельзя сказать, нужно исследовать отдельно}, \\
            A_{31} &< 0, \Delta U_{31} < 0, \implies Q_{31} = A_{31} + \Delta U_{31} < 0.
            \\
        \end{align*}

        Уравнения состояния идеального газа для точек 1, 2, 3: $P_1V_1 = \nu R T_1, P_2V_2 = \nu R T_2, P_3V_3 = \nu R T_3$.
        Пусть $P_0$, $V_0$, $T_0$ — давление, объём и температура в точке 1 (минимальные во всём цикле).

        12 --- изохора, $\frac{P_1V_1}{T_1} = \nu R = \frac{P_2V_2}{T_2}, V_2=V_1=V_0 \implies \frac{P_1}{T_1} =  \frac{P_2}{T_2} \implies P_2 = P_1 \frac{T_2}{T_1} = 5P_0$,

        31 --- изобара, $\frac{P_1V_1}{T_1} = \nu R = \frac{P_3V_3}{T_3}, P_3=P_1=P_0 \implies \frac{V_3}{T_3} =  \frac{V_1}{T_1} \implies V_3 = V_1 \frac{T_3}{T_1} = 5V_0$,

        Таким образом, используя новые обозначения, состояния газа в точках 1, 2 и 3 описываются макропараметрами $(P_0, V_0, T_0), (5P_0, V_0, 5T_0), (P_0, 5V_0, 5T_0)$ соответственно.

        \begin{tikzpicture}[thick]
            \draw[-{Latex}] (0, 0) -- (0, 7) node[above left] {$P$};
            \draw[-{Latex}] (0, 0) -- (10, 0) node[right] {$V$};

            \draw[dashed] (0, 2) node[left] {$P_1 = P_3 = P_0$} -| (9, 0) node[below] {$V_3 = 5V_0$};
            \draw[dashed] (0, 6) node[left] {$P_2 = 5P_0$} -| (3, 0) node[below] {$V_1 = V_2 = V_0$};

            \draw[dashed] (0, 5) node[left] {$P$} -| (4.5, 0) node[below] {$V$};
            \draw[dashed] (0, 4.6) node[left] {$P'$} -| (5.1, 0) node[below] {$V'$};

            \draw (3, 2) node[above left]{1} node[below left]{$T_1 = T_0$}
                   (3, 6) node[below left]{2} node[above]{$T_2 = 5T_0$}
                   (9, 2) node[above right]{3} node[below right]{$T_3 = 5T_0$};
            \draw[midar] (3, 2) -- (3, 6);
            \draw[midar] (3, 6) -- (9, 2);
            \draw[midar] (9, 2) -- (3, 2);
            \draw   (4.5, 5) node[above right]{$T$} (5.1, 4.6) node[above right]{$T'$};
        \end{tikzpicture}


        Теперь рассмотрим отдельно процесс 23, к остальному вернёмся позже.
        Уравнение этой прямой в $PV$-координатах: $P(V) = 6P_0 - \frac{P_0}{V_0} V$.
        Это значит, что при изменении объёма на $\Delta V$ давление изменится на $\Delta P = - \frac{P_0}{V_0} \Delta V$, обратите внимание на знак.

        Рассмотрим произвольную точку в процессе 23 и дадим процессу ещё немного свершиться, при этом объём изменится на $\Delta V$, давление — на $\Delta P$, температура (иначе бы была гипербола, а не прямая) — на $\Delta T$,
        т.е.
        из состояния $(P, V, T)$ мы перешли в $(P', V', T')$, причём  $P' = P + \Delta P, V' = V + \Delta V, T' = T + \Delta T$.

        При этом изменится внутренняя энергия:
        \begin{align*}
        \Delta U
            &= U' - U = \frac 32 \nu R T' - \frac 32 \nu R T = \frac 32 (P+\Delta P) (V+\Delta V) - \frac 32 PV\\
            &= \frac 32 ((P+\Delta P) (V+\Delta V) - PV) = \frac 32 (P\Delta V + V \Delta P + \Delta P \Delta V).
        \end{align*}

        Рассмотрим малые изменения объёма, тогда и изменение давления будем малым (т.к.
        $\Delta P = - \frac{P_0}{V_0} \Delta V$),
        а третьим слагаемым в выражении для $\Delta U$  можно пренебречь по сравнению с двумя другими:
        два первых это малые величины, а третье — произведение двух малых.
        Тогда $\Delta U = \frac 32 (P\Delta V + V \Delta P)$.

        Работа газа при этом малом изменении объёма — это площадь трапеции (тут ещё раз пренебрегли малым слагаемым):
        $$A = \frac{P + P'}2 \Delta V = \cbr{P + \frac{\Delta P}2} \Delta V = P \Delta V.$$

        Подведённое количество теплоты, используя первое начало термодинамики, будет равно
        \begin{align*}
        Q
            &= \frac 32 (P\Delta V + V \Delta P) + P \Delta V =  \frac 52 P\Delta V + \frac 32 V \Delta P = \\
            &= \frac 52 P\Delta V + \frac 32 V \cdot \cbr{- \frac{P_0}{V_0} \Delta V} = \frac{\Delta V}2 \cdot \cbr{5P - \frac{P_0}{V_0} V} = \\
            &= \frac{\Delta V}2 \cdot \cbr{5 \cdot \cbr{6P_0 - \frac{P_0}{V_0} V} - \frac{P_0}{V_0} V}
             = \frac{\Delta V \cdot P_0}2 \cdot \cbr{5 \cdot 6 - 8\frac V{V_0}}.
        \end{align*}

        Таком образом, знак количества теплоты $Q$ на участке 23 зависит от конкретного значения $V$:
        \begin{itemize}
            \item $\Delta V > 0$ на всём участке 23, поскольку газ расширяется,
            \item $P > 0$ — всегда, у нас идеальный газ, удары о стенки сосуда абсолютно упругие, а молекулы не взаимодействуют и поэтому давление только положительно,
            \item если $5 \cdot 6 - 8\frac V{V_0} > 0$ — тепло подводят, если же меньше нуля — отводят.
        \end{itemize}
        Решая последнее неравенство, получаем конкретное значение $V^*$: при $V < V^*$ тепло подводят, далее~— отводят.
        Тут *~--- некоторая точка между точками 2 и 3, конкретные значения надо досчитать:
        $$V^* = V_0 \cdot \frac{5 \cdot 6}8 = \frac{15}4 \cdot V_0 \implies P^* = 6P_0 - \frac{P_0}{V_0} V^* = \ldots = \frac94 \cdot P_0.$$

        Т.е.
        чтобы вычислить $Q_+$, надо сложить количества теплоты на участке 12 и лишь части участка 23 — участке 2*,
        той его части где это количество теплоты положительно.
        Имеем: $Q_+ = Q_{12} + Q_{2*}$.

        Теперь возвращаемся к циклу целиком и получаем:
        \begin{align*}
        A_\text{цикл} &= \frac 12 \cdot (5P_0 - P_0) \cdot (5V_0 - V_0) = 8 \cdot P_0V_0, \\
        A_{2*} &= \frac{P^* + 5P_0}2 \cdot (V^* - V_0)
            = \frac{\frac94 \cdot P_0 + 5P_0}2 \cdot \cbr{\frac{15}4 \cdot V_0 - V_0}
            = \ldots = \frac{319}{32} \cdot P_0 V_0, \\
        \Delta U_{2*} &= \frac 32 \nu R T^* - \frac 32 \nu R T_2 = \frac 32 (P^*V^* - P_0 \cdot 5V_0)
            = \frac 32 \cbr{\frac94 \cdot P_0 \cdot \frac{15}4 \cdot V_0 - P_0 \cdot 5V_0}
            = \frac{165}{32} \cdot P_0 V_0, \\
        \Delta U_{12} &= \frac 32 \nu R T_2 - \frac 32 \nu R T_1 = \frac 32 (5P_0V_0 - P_0V_0) = \ldots = 6 \cdot P_0 V_0, \\
        \eta &= \frac{A_\text{цикл}}{Q_+} = \frac{A_\text{цикл}}{Q_{12} + Q_{2*}}
            = \frac{A_\text{цикл}}{A_{12} + \Delta U_{12} + A_{2*} + \Delta U_{2*}} = \\
            &= \frac{8 \cdot P_0V_0}{0 + 6 \cdot P_0 V_0 + \frac{319}{32} \cdot P_0 V_0 + \frac{165}{32} \cdot P_0 V_0}
             = \frac{A_bonus_cycle:LaTeX}{6 + \frac{319}{32} + \frac{165}{32}}
             = \frac{64}{169} \leftarrow \text{вжух и готово!}
        \end{align*}
}
\solutionspace{360pt}

\tasknumber{2}%
\task{%
    При температуре $15\celsius$ относительная влажность воздуха составляет $45\%$.
    \begin{itemize}
        \item Определите точку росы для этого воздуха.
        \item Какой станет относительная влажность этого воздуха, если нагреть его до $60\celsius$?
    \end{itemize}
}
\answer{%
    \begin{align*}
    &\text{Значения плотности насыщенного водяного пара определяем по таблице:} \\
    &\rho_{\text{нас.
    пара 15} \celsius} = 12{,}800\,\frac{\text{г}}{\text{м}^{3}}, \rho_{\text{нас.
    пара 60} \celsius} = 130{,}000\,\frac{\text{г}}{\text{м}^{3}}.
    \\
    \varphi_1 &= \frac{\rho_\text{пара}}{\rho_{\text{нас.
    пара 15} \celsius}} \implies {\rho_\text{пара}} = \rho_{\text{нас.
    пара 15} \celsius} \cdot \varphi_1 = 12{,}800\,\frac{\text{г}}{\text{м}^{3}} \cdot 0{,}45 = 5{,}760\,\frac{\text{г}}{\text{м}^{3}}.
    \\
    &\text{По таблице определяем, при какой температуре пар с такой плотностью станет насыщенным:}  \\
    t_\text{росы} &= 2{,}4\celsius, \\
    \varphi_2 &= \frac{\rho_\text{пара}}{\rho_{\text{нас.
    пара 60} \celsius}} = \frac{\rho_{\text{нас.
    пара 15} \celsius} \cdot \varphi_1}{\rho_{\text{нас.
    пара 60} \celsius}}= \varphi_1 \cdot \frac{\rho_{\text{нас.
    пара 15} \celsius}}{\rho_{\text{нас.
    пара 60} \celsius}} = 0{,}45 \cdot \frac{12{,}800\,\frac{\text{г}}{\text{м}^{3}}}{130{,}000\,\frac{\text{г}}{\text{м}^{3}}} = 0{,}044 \approx 4{,}4\%.
    \end{align*}
}
\solutionspace{80pt}

\tasknumber{3}%
\task{%
    Из уравнения состояния идеального газа выведите или выразите...
    \begin{enumerate}
        \item объём,
        \item молярную массу,
        \item концентрацию молекул газа.
    \end{enumerate}
}

\tasknumber{4}%
\task{%
    Запишите формулы и рядом с каждой физичической величиной укажите её название и единицы измерения в СИ:
    \begin{enumerate}
        \item первое начало термодинамики,
        \item внутренняя энергия идеального одноатомного газа.
    \end{enumerate}
}

\variantsplitter

\addpersonalvariant{Наталья Кравченко}

\tasknumber{1}%
\task{%
    Определите КПД (оставив ответ точным в виде нескоратимой дроби) цикла 1231, рабочим телом которого является идеальный одноатомный газ, если
    \begin{itemize}
        \item 12 — изохорический нагрев в три раза,
        \item 23 — изобарическое расширение, при котором температура растёт в шесть раз,
        \item 31 — процесс, график которого в $PV$-координатах является отрезком прямой.
    \end{itemize}
    Бонус: замените цикл 1231 циклом, в котором 12 — изохорический нагрев в три раза, 23 — процесс, график которого в $PV$-координатах является отрезком прямой, 31 — изобарическое охлаждение, при котором температура падает в три раза.
}
\answer{%
    \begin{align*}
    A_{12} &= 0, \Delta U_{12} > 0, \implies Q_{12} = A_{12} + \Delta U_{12} > 0.
    \\
    A_{23} &> 0, \Delta U_{23} > 0, \implies Q_{23} = A_{23} + \Delta U_{23} > 0, \\
    A_{31} &= 0, \Delta U_{31} < 0, \implies Q_{31} = A_{31} + \Delta U_{31} < 0.
    \\
    P_1V_1 &= \nu R T_1, P_2V_2 = \nu R T_2, P_3V_3 = \nu R T_3 \text{ — уравнения состояния идеального газа}, \\
    &\text{Пусть $P_0$, $V_0$, $T_0$ — давление, объём и температура в точке 1 (минимальные во всём цикле):} \\
    P_1 &= P_0, P_2 = P_3, V_1 = V_2 = V_0, \text{остальные соотношения нужно считать} \\
    T_2 &= 3T_1 = 3T_0 \text{(по условию)} \implies \frac{P_2}{P_1} = \frac{P_2V_0}{P_1V_0} = \frac{P_2 V_2}{P_1 V_1}= \frac{\nu R T_2}{\nu R T_1} = \frac{T_2}{T_1} = 3 \implies P_2 = 3 P_1 = 3 P_0, \\
    T_3 &= 6T_2 = 18T_0 \text{(по условию)} \implies \frac{V_3}{V_2} = \frac{P_3V_3}{P_2V_2}= \frac{\nu R T_3}{\nu R T_2} = \frac{T_3}{T_2} = 6 \implies V_3 = 6 V_2 = 6 V_0.
    \\
    A_\text{цикл} &= \frac 12 (6P_0 - P_0)(3V_0 - V_0) = \frac 12 \cdot 10 \cdot P_0V_0, \\
    A_{23} &= 3P_0 \cdot (6V_0 - V_0) = 15P_0V_0, \\
    \Delta U_{23} &= \frac 32 \nu R T_3 - \frac 32 \nu R T_2 = \frac 32 P_3 V_3 - \frac 32 P_2 V_2 = \frac 32 \cdot 3 P_0 \cdot 6 V_0 -  \frac 32 \cdot 3 P_0 \cdot V_0 = \frac 32 \cdot 15 \cdot P_0V_0, \\
    \Delta U_{12} &= \frac 32 \nu R T_2 - \frac 32 \nu R T_1 = \frac 32 P_2 V_2 - \frac 32 P_1 V_1 = \frac 32 \cdot 3 P_0V_0 - \frac 32 P_0V_0 = \frac 32 \cdot 2 \cdot P_0V_0.
    \\
    \eta &= \frac{A_\text{цикл}}{Q_+} = \frac{A_\text{цикл}}{Q_{12} + Q_{23}}  = \frac{A_\text{цикл}}{A_{12} + \Delta U_{12} + A_{23} + \Delta U_{23}} =  \\
     &= \frac{\frac 12 \cdot 10 \cdot P_0V_0}{0 + \frac 32 \cdot 2 \cdot P_0V_0 + 15P_0V_0 + \frac 32 \cdot 15 \cdot P_0V_0} = \frac{\frac 12 \cdot 10}{\frac 32 \cdot 2 + 15 + \frac 32 \cdot 15} = \frac{10}{81} \approx 0.123.
    \end{align*}


        График процесса не в масштабе (эта часть пока не готова и сделать автоматически аккуратно сложно), но с верными подписями (а для решения этого достаточно):

        \begin{tikzpicture}[thick]
            \draw[-{Latex}] (0, 0) -- (0, 7) node[above left] {$P$};
            \draw[-{Latex}] (0, 0) -- (10, 0) node[right] {$V$};

            \draw[dashed] (0, 2) node[left] {$P_1 = P_0$} -| (3, 0) node[below] {$V_1 = V_2 = V_0$};
            \draw[dashed] (0, 6) node[left] {$P_2 = P_3 = 3P_0$} -| (9, 0) node[below] {$V_3 = 6V_0$};

            \draw (3, 2) node[above left]{1} node[below left]{$T_1 = T_0$}
                   (3, 6) node[below left]{2} node[above]{$T_2 = 3T_0$}
                   (9, 6) node[above right]{3} node[below right]{$T_3 = 18T_0$};
            \draw[midar] (3, 2) -- (3, 6);
            \draw[midar] (3, 6) -- (9, 6);
            \draw[midar] (9, 6) -- (3, 2);
        \end{tikzpicture}

        Решение бонуса:
        \begin{align*}
            A_{12} &= 0, \Delta U_{12} > 0, \implies Q_{12} = A_{12} + \Delta U_{12} > 0, \\
            A_{23} &> 0, \Delta U_{23} \text{ — ничего нельзя сказать, нужно исследовать отдельно}, \\
            A_{31} &< 0, \Delta U_{31} < 0, \implies Q_{31} = A_{31} + \Delta U_{31} < 0.
            \\
        \end{align*}

        Уравнения состояния идеального газа для точек 1, 2, 3: $P_1V_1 = \nu R T_1, P_2V_2 = \nu R T_2, P_3V_3 = \nu R T_3$.
        Пусть $P_0$, $V_0$, $T_0$ — давление, объём и температура в точке 1 (минимальные во всём цикле).

        12 --- изохора, $\frac{P_1V_1}{T_1} = \nu R = \frac{P_2V_2}{T_2}, V_2=V_1=V_0 \implies \frac{P_1}{T_1} =  \frac{P_2}{T_2} \implies P_2 = P_1 \frac{T_2}{T_1} = 3P_0$,

        31 --- изобара, $\frac{P_1V_1}{T_1} = \nu R = \frac{P_3V_3}{T_3}, P_3=P_1=P_0 \implies \frac{V_3}{T_3} =  \frac{V_1}{T_1} \implies V_3 = V_1 \frac{T_3}{T_1} = 3V_0$,

        Таким образом, используя новые обозначения, состояния газа в точках 1, 2 и 3 описываются макропараметрами $(P_0, V_0, T_0), (3P_0, V_0, 3T_0), (P_0, 3V_0, 3T_0)$ соответственно.

        \begin{tikzpicture}[thick]
            \draw[-{Latex}] (0, 0) -- (0, 7) node[above left] {$P$};
            \draw[-{Latex}] (0, 0) -- (10, 0) node[right] {$V$};

            \draw[dashed] (0, 2) node[left] {$P_1 = P_3 = P_0$} -| (9, 0) node[below] {$V_3 = 3V_0$};
            \draw[dashed] (0, 6) node[left] {$P_2 = 3P_0$} -| (3, 0) node[below] {$V_1 = V_2 = V_0$};

            \draw[dashed] (0, 5) node[left] {$P$} -| (4.5, 0) node[below] {$V$};
            \draw[dashed] (0, 4.6) node[left] {$P'$} -| (5.1, 0) node[below] {$V'$};

            \draw (3, 2) node[above left]{1} node[below left]{$T_1 = T_0$}
                   (3, 6) node[below left]{2} node[above]{$T_2 = 3T_0$}
                   (9, 2) node[above right]{3} node[below right]{$T_3 = 3T_0$};
            \draw[midar] (3, 2) -- (3, 6);
            \draw[midar] (3, 6) -- (9, 2);
            \draw[midar] (9, 2) -- (3, 2);
            \draw   (4.5, 5) node[above right]{$T$} (5.1, 4.6) node[above right]{$T'$};
        \end{tikzpicture}


        Теперь рассмотрим отдельно процесс 23, к остальному вернёмся позже.
        Уравнение этой прямой в $PV$-координатах: $P(V) = 4P_0 - \frac{P_0}{V_0} V$.
        Это значит, что при изменении объёма на $\Delta V$ давление изменится на $\Delta P = - \frac{P_0}{V_0} \Delta V$, обратите внимание на знак.

        Рассмотрим произвольную точку в процессе 23 и дадим процессу ещё немного свершиться, при этом объём изменится на $\Delta V$, давление — на $\Delta P$, температура (иначе бы была гипербола, а не прямая) — на $\Delta T$,
        т.е.
        из состояния $(P, V, T)$ мы перешли в $(P', V', T')$, причём  $P' = P + \Delta P, V' = V + \Delta V, T' = T + \Delta T$.

        При этом изменится внутренняя энергия:
        \begin{align*}
        \Delta U
            &= U' - U = \frac 32 \nu R T' - \frac 32 \nu R T = \frac 32 (P+\Delta P) (V+\Delta V) - \frac 32 PV\\
            &= \frac 32 ((P+\Delta P) (V+\Delta V) - PV) = \frac 32 (P\Delta V + V \Delta P + \Delta P \Delta V).
        \end{align*}

        Рассмотрим малые изменения объёма, тогда и изменение давления будем малым (т.к.
        $\Delta P = - \frac{P_0}{V_0} \Delta V$),
        а третьим слагаемым в выражении для $\Delta U$  можно пренебречь по сравнению с двумя другими:
        два первых это малые величины, а третье — произведение двух малых.
        Тогда $\Delta U = \frac 32 (P\Delta V + V \Delta P)$.

        Работа газа при этом малом изменении объёма — это площадь трапеции (тут ещё раз пренебрегли малым слагаемым):
        $$A = \frac{P + P'}2 \Delta V = \cbr{P + \frac{\Delta P}2} \Delta V = P \Delta V.$$

        Подведённое количество теплоты, используя первое начало термодинамики, будет равно
        \begin{align*}
        Q
            &= \frac 32 (P\Delta V + V \Delta P) + P \Delta V =  \frac 52 P\Delta V + \frac 32 V \Delta P = \\
            &= \frac 52 P\Delta V + \frac 32 V \cdot \cbr{- \frac{P_0}{V_0} \Delta V} = \frac{\Delta V}2 \cdot \cbr{5P - \frac{P_0}{V_0} V} = \\
            &= \frac{\Delta V}2 \cdot \cbr{5 \cdot \cbr{4P_0 - \frac{P_0}{V_0} V} - \frac{P_0}{V_0} V}
             = \frac{\Delta V \cdot P_0}2 \cdot \cbr{5 \cdot 4 - 8\frac V{V_0}}.
        \end{align*}

        Таком образом, знак количества теплоты $Q$ на участке 23 зависит от конкретного значения $V$:
        \begin{itemize}
            \item $\Delta V > 0$ на всём участке 23, поскольку газ расширяется,
            \item $P > 0$ — всегда, у нас идеальный газ, удары о стенки сосуда абсолютно упругие, а молекулы не взаимодействуют и поэтому давление только положительно,
            \item если $5 \cdot 4 - 8\frac V{V_0} > 0$ — тепло подводят, если же меньше нуля — отводят.
        \end{itemize}
        Решая последнее неравенство, получаем конкретное значение $V^*$: при $V < V^*$ тепло подводят, далее~— отводят.
        Тут *~--- некоторая точка между точками 2 и 3, конкретные значения надо досчитать:
        $$V^* = V_0 \cdot \frac{5 \cdot 4}8 = \frac52 \cdot V_0 \implies P^* = 4P_0 - \frac{P_0}{V_0} V^* = \ldots = \frac32 \cdot P_0.$$

        Т.е.
        чтобы вычислить $Q_+$, надо сложить количества теплоты на участке 12 и лишь части участка 23 — участке 2*,
        той его части где это количество теплоты положительно.
        Имеем: $Q_+ = Q_{12} + Q_{2*}$.

        Теперь возвращаемся к циклу целиком и получаем:
        \begin{align*}
        A_\text{цикл} &= \frac 12 \cdot (3P_0 - P_0) \cdot (3V_0 - V_0) = 2 \cdot P_0V_0, \\
        A_{2*} &= \frac{P^* + 3P_0}2 \cdot (V^* - V_0)
            = \frac{\frac32 \cdot P_0 + 3P_0}2 \cdot \cbr{\frac52 \cdot V_0 - V_0}
            = \ldots = \frac{27}8 \cdot P_0 V_0, \\
        \Delta U_{2*} &= \frac 32 \nu R T^* - \frac 32 \nu R T_2 = \frac 32 (P^*V^* - P_0 \cdot 3V_0)
            = \frac 32 \cbr{\frac32 \cdot P_0 \cdot \frac52 \cdot V_0 - P_0 \cdot 3V_0}
            = \frac98 \cdot P_0 V_0, \\
        \Delta U_{12} &= \frac 32 \nu R T_2 - \frac 32 \nu R T_1 = \frac 32 (3P_0V_0 - P_0V_0) = \ldots = 3 \cdot P_0 V_0, \\
        \eta &= \frac{A_\text{цикл}}{Q_+} = \frac{A_\text{цикл}}{Q_{12} + Q_{2*}}
            = \frac{A_\text{цикл}}{A_{12} + \Delta U_{12} + A_{2*} + \Delta U_{2*}} = \\
            &= \frac{2 \cdot P_0V_0}{0 + 3 \cdot P_0 V_0 + \frac{27}8 \cdot P_0 V_0 + \frac98 \cdot P_0 V_0}
             = \frac{A_bonus_cycle:LaTeX}{3 + \frac{27}8 + \frac98}
             = \frac4{15} \leftarrow \text{вжух и готово!}
        \end{align*}
}
\solutionspace{360pt}

\tasknumber{2}%
\task{%
    При температуре $25\celsius$ относительная влажность воздуха составляет $70\%$.
    \begin{itemize}
        \item Определите точку росы для этого воздуха.
        \item Какой станет относительная влажность этого воздуха, если нагреть его до $60\celsius$?
    \end{itemize}
}
\answer{%
    \begin{align*}
    &\text{Значения плотности насыщенного водяного пара определяем по таблице:} \\
    &\rho_{\text{нас.
    пара 25} \celsius} = 23{,}000\,\frac{\text{г}}{\text{м}^{3}}, \rho_{\text{нас.
    пара 60} \celsius} = 130{,}000\,\frac{\text{г}}{\text{м}^{3}}.
    \\
    \varphi_1 &= \frac{\rho_\text{пара}}{\rho_{\text{нас.
    пара 25} \celsius}} \implies {\rho_\text{пара}} = \rho_{\text{нас.
    пара 25} \celsius} \cdot \varphi_1 = 23{,}000\,\frac{\text{г}}{\text{м}^{3}} \cdot 0{,}70 = 16{,}100\,\frac{\text{г}}{\text{м}^{3}}.
    \\
    &\text{По таблице определяем, при какой температуре пар с такой плотностью станет насыщенным:}  \\
    t_\text{росы} &= 18{,}8\celsius, \\
    \varphi_2 &= \frac{\rho_\text{пара}}{\rho_{\text{нас.
    пара 60} \celsius}} = \frac{\rho_{\text{нас.
    пара 25} \celsius} \cdot \varphi_1}{\rho_{\text{нас.
    пара 60} \celsius}}= \varphi_1 \cdot \frac{\rho_{\text{нас.
    пара 25} \celsius}}{\rho_{\text{нас.
    пара 60} \celsius}} = 0{,}70 \cdot \frac{23{,}000\,\frac{\text{г}}{\text{м}^{3}}}{130{,}000\,\frac{\text{г}}{\text{м}^{3}}} = 0{,}124 \approx 12{,}4\%.
    \end{align*}
}
\solutionspace{80pt}

\tasknumber{3}%
\task{%
    Из уравнения состояния идеального газа выведите или выразите...
    \begin{enumerate}
        \item давление,
        \item молярную массу,
        \item плотность газа.
    \end{enumerate}
}

\tasknumber{4}%
\task{%
    Запишите формулы и рядом с каждой физичической величиной укажите её название и единицы измерения в СИ:
    \begin{enumerate}
        \item первое начало термодинамики,
        \item внутренняя энергия идеального одноатомного газа.
    \end{enumerate}
}

\variantsplitter

\addpersonalvariant{Матвей Кузьмин}

\tasknumber{1}%
\task{%
    Определите КПД (оставив ответ точным в виде нескоратимой дроби) цикла 1231, рабочим телом которого является идеальный одноатомный газ, если
    \begin{itemize}
        \item 12 — изохорический нагрев в четыре раза,
        \item 23 — изобарическое расширение, при котором температура растёт в два раза,
        \item 31 — процесс, график которого в $PV$-координатах является отрезком прямой.
    \end{itemize}
    Бонус: замените цикл 1231 циклом, в котором 12 — изохорический нагрев в четыре раза, 23 — процесс, график которого в $PV$-координатах является отрезком прямой, 31 — изобарическое охлаждение, при котором температура падает в четыре раза.
}
\answer{%
    \begin{align*}
    A_{12} &= 0, \Delta U_{12} > 0, \implies Q_{12} = A_{12} + \Delta U_{12} > 0.
    \\
    A_{23} &> 0, \Delta U_{23} > 0, \implies Q_{23} = A_{23} + \Delta U_{23} > 0, \\
    A_{31} &= 0, \Delta U_{31} < 0, \implies Q_{31} = A_{31} + \Delta U_{31} < 0.
    \\
    P_1V_1 &= \nu R T_1, P_2V_2 = \nu R T_2, P_3V_3 = \nu R T_3 \text{ — уравнения состояния идеального газа}, \\
    &\text{Пусть $P_0$, $V_0$, $T_0$ — давление, объём и температура в точке 1 (минимальные во всём цикле):} \\
    P_1 &= P_0, P_2 = P_3, V_1 = V_2 = V_0, \text{остальные соотношения нужно считать} \\
    T_2 &= 4T_1 = 4T_0 \text{(по условию)} \implies \frac{P_2}{P_1} = \frac{P_2V_0}{P_1V_0} = \frac{P_2 V_2}{P_1 V_1}= \frac{\nu R T_2}{\nu R T_1} = \frac{T_2}{T_1} = 4 \implies P_2 = 4 P_1 = 4 P_0, \\
    T_3 &= 2T_2 = 8T_0 \text{(по условию)} \implies \frac{V_3}{V_2} = \frac{P_3V_3}{P_2V_2}= \frac{\nu R T_3}{\nu R T_2} = \frac{T_3}{T_2} = 2 \implies V_3 = 2 V_2 = 2 V_0.
    \\
    A_\text{цикл} &= \frac 12 (2P_0 - P_0)(4V_0 - V_0) = \frac 12 \cdot 3 \cdot P_0V_0, \\
    A_{23} &= 4P_0 \cdot (2V_0 - V_0) = 4P_0V_0, \\
    \Delta U_{23} &= \frac 32 \nu R T_3 - \frac 32 \nu R T_2 = \frac 32 P_3 V_3 - \frac 32 P_2 V_2 = \frac 32 \cdot 4 P_0 \cdot 2 V_0 -  \frac 32 \cdot 4 P_0 \cdot V_0 = \frac 32 \cdot 4 \cdot P_0V_0, \\
    \Delta U_{12} &= \frac 32 \nu R T_2 - \frac 32 \nu R T_1 = \frac 32 P_2 V_2 - \frac 32 P_1 V_1 = \frac 32 \cdot 4 P_0V_0 - \frac 32 P_0V_0 = \frac 32 \cdot 3 \cdot P_0V_0.
    \\
    \eta &= \frac{A_\text{цикл}}{Q_+} = \frac{A_\text{цикл}}{Q_{12} + Q_{23}}  = \frac{A_\text{цикл}}{A_{12} + \Delta U_{12} + A_{23} + \Delta U_{23}} =  \\
     &= \frac{\frac 12 \cdot 3 \cdot P_0V_0}{0 + \frac 32 \cdot 3 \cdot P_0V_0 + 4P_0V_0 + \frac 32 \cdot 4 \cdot P_0V_0} = \frac{\frac 12 \cdot 3}{\frac 32 \cdot 3 + 4 + \frac 32 \cdot 4} = \frac3{29} \approx 0.103.
    \end{align*}


        График процесса не в масштабе (эта часть пока не готова и сделать автоматически аккуратно сложно), но с верными подписями (а для решения этого достаточно):

        \begin{tikzpicture}[thick]
            \draw[-{Latex}] (0, 0) -- (0, 7) node[above left] {$P$};
            \draw[-{Latex}] (0, 0) -- (10, 0) node[right] {$V$};

            \draw[dashed] (0, 2) node[left] {$P_1 = P_0$} -| (3, 0) node[below] {$V_1 = V_2 = V_0$};
            \draw[dashed] (0, 6) node[left] {$P_2 = P_3 = 4P_0$} -| (9, 0) node[below] {$V_3 = 2V_0$};

            \draw (3, 2) node[above left]{1} node[below left]{$T_1 = T_0$}
                   (3, 6) node[below left]{2} node[above]{$T_2 = 4T_0$}
                   (9, 6) node[above right]{3} node[below right]{$T_3 = 8T_0$};
            \draw[midar] (3, 2) -- (3, 6);
            \draw[midar] (3, 6) -- (9, 6);
            \draw[midar] (9, 6) -- (3, 2);
        \end{tikzpicture}

        Решение бонуса:
        \begin{align*}
            A_{12} &= 0, \Delta U_{12} > 0, \implies Q_{12} = A_{12} + \Delta U_{12} > 0, \\
            A_{23} &> 0, \Delta U_{23} \text{ — ничего нельзя сказать, нужно исследовать отдельно}, \\
            A_{31} &< 0, \Delta U_{31} < 0, \implies Q_{31} = A_{31} + \Delta U_{31} < 0.
            \\
        \end{align*}

        Уравнения состояния идеального газа для точек 1, 2, 3: $P_1V_1 = \nu R T_1, P_2V_2 = \nu R T_2, P_3V_3 = \nu R T_3$.
        Пусть $P_0$, $V_0$, $T_0$ — давление, объём и температура в точке 1 (минимальные во всём цикле).

        12 --- изохора, $\frac{P_1V_1}{T_1} = \nu R = \frac{P_2V_2}{T_2}, V_2=V_1=V_0 \implies \frac{P_1}{T_1} =  \frac{P_2}{T_2} \implies P_2 = P_1 \frac{T_2}{T_1} = 4P_0$,

        31 --- изобара, $\frac{P_1V_1}{T_1} = \nu R = \frac{P_3V_3}{T_3}, P_3=P_1=P_0 \implies \frac{V_3}{T_3} =  \frac{V_1}{T_1} \implies V_3 = V_1 \frac{T_3}{T_1} = 4V_0$,

        Таким образом, используя новые обозначения, состояния газа в точках 1, 2 и 3 описываются макропараметрами $(P_0, V_0, T_0), (4P_0, V_0, 4T_0), (P_0, 4V_0, 4T_0)$ соответственно.

        \begin{tikzpicture}[thick]
            \draw[-{Latex}] (0, 0) -- (0, 7) node[above left] {$P$};
            \draw[-{Latex}] (0, 0) -- (10, 0) node[right] {$V$};

            \draw[dashed] (0, 2) node[left] {$P_1 = P_3 = P_0$} -| (9, 0) node[below] {$V_3 = 4V_0$};
            \draw[dashed] (0, 6) node[left] {$P_2 = 4P_0$} -| (3, 0) node[below] {$V_1 = V_2 = V_0$};

            \draw[dashed] (0, 5) node[left] {$P$} -| (4.5, 0) node[below] {$V$};
            \draw[dashed] (0, 4.6) node[left] {$P'$} -| (5.1, 0) node[below] {$V'$};

            \draw (3, 2) node[above left]{1} node[below left]{$T_1 = T_0$}
                   (3, 6) node[below left]{2} node[above]{$T_2 = 4T_0$}
                   (9, 2) node[above right]{3} node[below right]{$T_3 = 4T_0$};
            \draw[midar] (3, 2) -- (3, 6);
            \draw[midar] (3, 6) -- (9, 2);
            \draw[midar] (9, 2) -- (3, 2);
            \draw   (4.5, 5) node[above right]{$T$} (5.1, 4.6) node[above right]{$T'$};
        \end{tikzpicture}


        Теперь рассмотрим отдельно процесс 23, к остальному вернёмся позже.
        Уравнение этой прямой в $PV$-координатах: $P(V) = 5P_0 - \frac{P_0}{V_0} V$.
        Это значит, что при изменении объёма на $\Delta V$ давление изменится на $\Delta P = - \frac{P_0}{V_0} \Delta V$, обратите внимание на знак.

        Рассмотрим произвольную точку в процессе 23 и дадим процессу ещё немного свершиться, при этом объём изменится на $\Delta V$, давление — на $\Delta P$, температура (иначе бы была гипербола, а не прямая) — на $\Delta T$,
        т.е.
        из состояния $(P, V, T)$ мы перешли в $(P', V', T')$, причём  $P' = P + \Delta P, V' = V + \Delta V, T' = T + \Delta T$.

        При этом изменится внутренняя энергия:
        \begin{align*}
        \Delta U
            &= U' - U = \frac 32 \nu R T' - \frac 32 \nu R T = \frac 32 (P+\Delta P) (V+\Delta V) - \frac 32 PV\\
            &= \frac 32 ((P+\Delta P) (V+\Delta V) - PV) = \frac 32 (P\Delta V + V \Delta P + \Delta P \Delta V).
        \end{align*}

        Рассмотрим малые изменения объёма, тогда и изменение давления будем малым (т.к.
        $\Delta P = - \frac{P_0}{V_0} \Delta V$),
        а третьим слагаемым в выражении для $\Delta U$  можно пренебречь по сравнению с двумя другими:
        два первых это малые величины, а третье — произведение двух малых.
        Тогда $\Delta U = \frac 32 (P\Delta V + V \Delta P)$.

        Работа газа при этом малом изменении объёма — это площадь трапеции (тут ещё раз пренебрегли малым слагаемым):
        $$A = \frac{P + P'}2 \Delta V = \cbr{P + \frac{\Delta P}2} \Delta V = P \Delta V.$$

        Подведённое количество теплоты, используя первое начало термодинамики, будет равно
        \begin{align*}
        Q
            &= \frac 32 (P\Delta V + V \Delta P) + P \Delta V =  \frac 52 P\Delta V + \frac 32 V \Delta P = \\
            &= \frac 52 P\Delta V + \frac 32 V \cdot \cbr{- \frac{P_0}{V_0} \Delta V} = \frac{\Delta V}2 \cdot \cbr{5P - \frac{P_0}{V_0} V} = \\
            &= \frac{\Delta V}2 \cdot \cbr{5 \cdot \cbr{5P_0 - \frac{P_0}{V_0} V} - \frac{P_0}{V_0} V}
             = \frac{\Delta V \cdot P_0}2 \cdot \cbr{5 \cdot 5 - 8\frac V{V_0}}.
        \end{align*}

        Таком образом, знак количества теплоты $Q$ на участке 23 зависит от конкретного значения $V$:
        \begin{itemize}
            \item $\Delta V > 0$ на всём участке 23, поскольку газ расширяется,
            \item $P > 0$ — всегда, у нас идеальный газ, удары о стенки сосуда абсолютно упругие, а молекулы не взаимодействуют и поэтому давление только положительно,
            \item если $5 \cdot 5 - 8\frac V{V_0} > 0$ — тепло подводят, если же меньше нуля — отводят.
        \end{itemize}
        Решая последнее неравенство, получаем конкретное значение $V^*$: при $V < V^*$ тепло подводят, далее~— отводят.
        Тут *~--- некоторая точка между точками 2 и 3, конкретные значения надо досчитать:
        $$V^* = V_0 \cdot \frac{5 \cdot 5}8 = \frac{25}8 \cdot V_0 \implies P^* = 5P_0 - \frac{P_0}{V_0} V^* = \ldots = \frac{15}8 \cdot P_0.$$

        Т.е.
        чтобы вычислить $Q_+$, надо сложить количества теплоты на участке 12 и лишь части участка 23 — участке 2*,
        той его части где это количество теплоты положительно.
        Имеем: $Q_+ = Q_{12} + Q_{2*}$.

        Теперь возвращаемся к циклу целиком и получаем:
        \begin{align*}
        A_\text{цикл} &= \frac 12 \cdot (4P_0 - P_0) \cdot (4V_0 - V_0) = \frac92 \cdot P_0V_0, \\
        A_{2*} &= \frac{P^* + 4P_0}2 \cdot (V^* - V_0)
            = \frac{\frac{15}8 \cdot P_0 + 4P_0}2 \cdot \cbr{\frac{25}8 \cdot V_0 - V_0}
            = \ldots = \frac{799}{128} \cdot P_0 V_0, \\
        \Delta U_{2*} &= \frac 32 \nu R T^* - \frac 32 \nu R T_2 = \frac 32 (P^*V^* - P_0 \cdot 4V_0)
            = \frac 32 \cbr{\frac{15}8 \cdot P_0 \cdot \frac{25}8 \cdot V_0 - P_0 \cdot 4V_0}
            = \frac{357}{128} \cdot P_0 V_0, \\
        \Delta U_{12} &= \frac 32 \nu R T_2 - \frac 32 \nu R T_1 = \frac 32 (4P_0V_0 - P_0V_0) = \ldots = \frac92 \cdot P_0 V_0, \\
        \eta &= \frac{A_\text{цикл}}{Q_+} = \frac{A_\text{цикл}}{Q_{12} + Q_{2*}}
            = \frac{A_\text{цикл}}{A_{12} + \Delta U_{12} + A_{2*} + \Delta U_{2*}} = \\
            &= \frac{\frac92 \cdot P_0V_0}{0 + \frac92 \cdot P_0 V_0 + \frac{799}{128} \cdot P_0 V_0 + \frac{357}{128} \cdot P_0 V_0}
             = \frac{A_bonus_cycle:LaTeX}{\frac92 + \frac{799}{128} + \frac{357}{128}}
             = \frac{144}{433} \leftarrow \text{вжух и готово!}
        \end{align*}
}
\solutionspace{360pt}

\tasknumber{2}%
\task{%
    При температуре $15\celsius$ относительная влажность воздуха составляет $40\%$.
    \begin{itemize}
        \item Определите точку росы для этого воздуха.
        \item Какой станет относительная влажность этого воздуха, если нагреть его до $70\celsius$?
    \end{itemize}
}
\answer{%
    \begin{align*}
    &\text{Значения плотности насыщенного водяного пара определяем по таблице:} \\
    &\rho_{\text{нас.
    пара 15} \celsius} = 12{,}800\,\frac{\text{г}}{\text{м}^{3}}, \rho_{\text{нас.
    пара 70} \celsius} = 198{,}000\,\frac{\text{г}}{\text{м}^{3}}.
    \\
    \varphi_1 &= \frac{\rho_\text{пара}}{\rho_{\text{нас.
    пара 15} \celsius}} \implies {\rho_\text{пара}} = \rho_{\text{нас.
    пара 15} \celsius} \cdot \varphi_1 = 12{,}800\,\frac{\text{г}}{\text{м}^{3}} \cdot 0{,}40 = 5{,}120\,\frac{\text{г}}{\text{м}^{3}}.
    \\
    &\text{По таблице определяем, при какой температуре пар с такой плотностью станет насыщенным:}  \\
    t_\text{росы} &= 0{,}7\celsius, \\
    \varphi_2 &= \frac{\rho_\text{пара}}{\rho_{\text{нас.
    пара 70} \celsius}} = \frac{\rho_{\text{нас.
    пара 15} \celsius} \cdot \varphi_1}{\rho_{\text{нас.
    пара 70} \celsius}}= \varphi_1 \cdot \frac{\rho_{\text{нас.
    пара 15} \celsius}}{\rho_{\text{нас.
    пара 70} \celsius}} = 0{,}40 \cdot \frac{12{,}800\,\frac{\text{г}}{\text{м}^{3}}}{198{,}000\,\frac{\text{г}}{\text{м}^{3}}} = 0{,}026 \approx 2{,}6\%.
    \end{align*}
}
\solutionspace{80pt}

\tasknumber{3}%
\task{%
    Из уравнения состояния идеального газа выведите или выразите...
    \begin{enumerate}
        \item давление,
        \item температуру,
        \item концентрацию молекул газа.
    \end{enumerate}
}

\tasknumber{4}%
\task{%
    Запишите формулы и рядом с каждой физичической величиной укажите её название и единицы измерения в СИ:
    \begin{enumerate}
        \item первое начало термодинамики,
        \item внутренняя энергия идеального одноатомного газа.
    \end{enumerate}
}

\variantsplitter

\addpersonalvariant{Сергей Малышев}

\tasknumber{1}%
\task{%
    Определите КПД (оставив ответ точным в виде нескоратимой дроби) цикла 1231, рабочим телом которого является идеальный одноатомный газ, если
    \begin{itemize}
        \item 12 — изохорический нагрев в три раза,
        \item 23 — изобарическое расширение, при котором температура растёт в два раза,
        \item 31 — процесс, график которого в $PV$-координатах является отрезком прямой.
    \end{itemize}
    Бонус: замените цикл 1231 циклом, в котором 12 — изохорический нагрев в три раза, 23 — процесс, график которого в $PV$-координатах является отрезком прямой, 31 — изобарическое охлаждение, при котором температура падает в три раза.
}
\answer{%
    \begin{align*}
    A_{12} &= 0, \Delta U_{12} > 0, \implies Q_{12} = A_{12} + \Delta U_{12} > 0.
    \\
    A_{23} &> 0, \Delta U_{23} > 0, \implies Q_{23} = A_{23} + \Delta U_{23} > 0, \\
    A_{31} &= 0, \Delta U_{31} < 0, \implies Q_{31} = A_{31} + \Delta U_{31} < 0.
    \\
    P_1V_1 &= \nu R T_1, P_2V_2 = \nu R T_2, P_3V_3 = \nu R T_3 \text{ — уравнения состояния идеального газа}, \\
    &\text{Пусть $P_0$, $V_0$, $T_0$ — давление, объём и температура в точке 1 (минимальные во всём цикле):} \\
    P_1 &= P_0, P_2 = P_3, V_1 = V_2 = V_0, \text{остальные соотношения нужно считать} \\
    T_2 &= 3T_1 = 3T_0 \text{(по условию)} \implies \frac{P_2}{P_1} = \frac{P_2V_0}{P_1V_0} = \frac{P_2 V_2}{P_1 V_1}= \frac{\nu R T_2}{\nu R T_1} = \frac{T_2}{T_1} = 3 \implies P_2 = 3 P_1 = 3 P_0, \\
    T_3 &= 2T_2 = 6T_0 \text{(по условию)} \implies \frac{V_3}{V_2} = \frac{P_3V_3}{P_2V_2}= \frac{\nu R T_3}{\nu R T_2} = \frac{T_3}{T_2} = 2 \implies V_3 = 2 V_2 = 2 V_0.
    \\
    A_\text{цикл} &= \frac 12 (2P_0 - P_0)(3V_0 - V_0) = \frac 12 \cdot 2 \cdot P_0V_0, \\
    A_{23} &= 3P_0 \cdot (2V_0 - V_0) = 3P_0V_0, \\
    \Delta U_{23} &= \frac 32 \nu R T_3 - \frac 32 \nu R T_2 = \frac 32 P_3 V_3 - \frac 32 P_2 V_2 = \frac 32 \cdot 3 P_0 \cdot 2 V_0 -  \frac 32 \cdot 3 P_0 \cdot V_0 = \frac 32 \cdot 3 \cdot P_0V_0, \\
    \Delta U_{12} &= \frac 32 \nu R T_2 - \frac 32 \nu R T_1 = \frac 32 P_2 V_2 - \frac 32 P_1 V_1 = \frac 32 \cdot 3 P_0V_0 - \frac 32 P_0V_0 = \frac 32 \cdot 2 \cdot P_0V_0.
    \\
    \eta &= \frac{A_\text{цикл}}{Q_+} = \frac{A_\text{цикл}}{Q_{12} + Q_{23}}  = \frac{A_\text{цикл}}{A_{12} + \Delta U_{12} + A_{23} + \Delta U_{23}} =  \\
     &= \frac{\frac 12 \cdot 2 \cdot P_0V_0}{0 + \frac 32 \cdot 2 \cdot P_0V_0 + 3P_0V_0 + \frac 32 \cdot 3 \cdot P_0V_0} = \frac{\frac 12 \cdot 2}{\frac 32 \cdot 2 + 3 + \frac 32 \cdot 3} = \frac2{21} \approx 0.095.
    \end{align*}


        График процесса не в масштабе (эта часть пока не готова и сделать автоматически аккуратно сложно), но с верными подписями (а для решения этого достаточно):

        \begin{tikzpicture}[thick]
            \draw[-{Latex}] (0, 0) -- (0, 7) node[above left] {$P$};
            \draw[-{Latex}] (0, 0) -- (10, 0) node[right] {$V$};

            \draw[dashed] (0, 2) node[left] {$P_1 = P_0$} -| (3, 0) node[below] {$V_1 = V_2 = V_0$};
            \draw[dashed] (0, 6) node[left] {$P_2 = P_3 = 3P_0$} -| (9, 0) node[below] {$V_3 = 2V_0$};

            \draw (3, 2) node[above left]{1} node[below left]{$T_1 = T_0$}
                   (3, 6) node[below left]{2} node[above]{$T_2 = 3T_0$}
                   (9, 6) node[above right]{3} node[below right]{$T_3 = 6T_0$};
            \draw[midar] (3, 2) -- (3, 6);
            \draw[midar] (3, 6) -- (9, 6);
            \draw[midar] (9, 6) -- (3, 2);
        \end{tikzpicture}

        Решение бонуса:
        \begin{align*}
            A_{12} &= 0, \Delta U_{12} > 0, \implies Q_{12} = A_{12} + \Delta U_{12} > 0, \\
            A_{23} &> 0, \Delta U_{23} \text{ — ничего нельзя сказать, нужно исследовать отдельно}, \\
            A_{31} &< 0, \Delta U_{31} < 0, \implies Q_{31} = A_{31} + \Delta U_{31} < 0.
            \\
        \end{align*}

        Уравнения состояния идеального газа для точек 1, 2, 3: $P_1V_1 = \nu R T_1, P_2V_2 = \nu R T_2, P_3V_3 = \nu R T_3$.
        Пусть $P_0$, $V_0$, $T_0$ — давление, объём и температура в точке 1 (минимальные во всём цикле).

        12 --- изохора, $\frac{P_1V_1}{T_1} = \nu R = \frac{P_2V_2}{T_2}, V_2=V_1=V_0 \implies \frac{P_1}{T_1} =  \frac{P_2}{T_2} \implies P_2 = P_1 \frac{T_2}{T_1} = 3P_0$,

        31 --- изобара, $\frac{P_1V_1}{T_1} = \nu R = \frac{P_3V_3}{T_3}, P_3=P_1=P_0 \implies \frac{V_3}{T_3} =  \frac{V_1}{T_1} \implies V_3 = V_1 \frac{T_3}{T_1} = 3V_0$,

        Таким образом, используя новые обозначения, состояния газа в точках 1, 2 и 3 описываются макропараметрами $(P_0, V_0, T_0), (3P_0, V_0, 3T_0), (P_0, 3V_0, 3T_0)$ соответственно.

        \begin{tikzpicture}[thick]
            \draw[-{Latex}] (0, 0) -- (0, 7) node[above left] {$P$};
            \draw[-{Latex}] (0, 0) -- (10, 0) node[right] {$V$};

            \draw[dashed] (0, 2) node[left] {$P_1 = P_3 = P_0$} -| (9, 0) node[below] {$V_3 = 3V_0$};
            \draw[dashed] (0, 6) node[left] {$P_2 = 3P_0$} -| (3, 0) node[below] {$V_1 = V_2 = V_0$};

            \draw[dashed] (0, 5) node[left] {$P$} -| (4.5, 0) node[below] {$V$};
            \draw[dashed] (0, 4.6) node[left] {$P'$} -| (5.1, 0) node[below] {$V'$};

            \draw (3, 2) node[above left]{1} node[below left]{$T_1 = T_0$}
                   (3, 6) node[below left]{2} node[above]{$T_2 = 3T_0$}
                   (9, 2) node[above right]{3} node[below right]{$T_3 = 3T_0$};
            \draw[midar] (3, 2) -- (3, 6);
            \draw[midar] (3, 6) -- (9, 2);
            \draw[midar] (9, 2) -- (3, 2);
            \draw   (4.5, 5) node[above right]{$T$} (5.1, 4.6) node[above right]{$T'$};
        \end{tikzpicture}


        Теперь рассмотрим отдельно процесс 23, к остальному вернёмся позже.
        Уравнение этой прямой в $PV$-координатах: $P(V) = 4P_0 - \frac{P_0}{V_0} V$.
        Это значит, что при изменении объёма на $\Delta V$ давление изменится на $\Delta P = - \frac{P_0}{V_0} \Delta V$, обратите внимание на знак.

        Рассмотрим произвольную точку в процессе 23 и дадим процессу ещё немного свершиться, при этом объём изменится на $\Delta V$, давление — на $\Delta P$, температура (иначе бы была гипербола, а не прямая) — на $\Delta T$,
        т.е.
        из состояния $(P, V, T)$ мы перешли в $(P', V', T')$, причём  $P' = P + \Delta P, V' = V + \Delta V, T' = T + \Delta T$.

        При этом изменится внутренняя энергия:
        \begin{align*}
        \Delta U
            &= U' - U = \frac 32 \nu R T' - \frac 32 \nu R T = \frac 32 (P+\Delta P) (V+\Delta V) - \frac 32 PV\\
            &= \frac 32 ((P+\Delta P) (V+\Delta V) - PV) = \frac 32 (P\Delta V + V \Delta P + \Delta P \Delta V).
        \end{align*}

        Рассмотрим малые изменения объёма, тогда и изменение давления будем малым (т.к.
        $\Delta P = - \frac{P_0}{V_0} \Delta V$),
        а третьим слагаемым в выражении для $\Delta U$  можно пренебречь по сравнению с двумя другими:
        два первых это малые величины, а третье — произведение двух малых.
        Тогда $\Delta U = \frac 32 (P\Delta V + V \Delta P)$.

        Работа газа при этом малом изменении объёма — это площадь трапеции (тут ещё раз пренебрегли малым слагаемым):
        $$A = \frac{P + P'}2 \Delta V = \cbr{P + \frac{\Delta P}2} \Delta V = P \Delta V.$$

        Подведённое количество теплоты, используя первое начало термодинамики, будет равно
        \begin{align*}
        Q
            &= \frac 32 (P\Delta V + V \Delta P) + P \Delta V =  \frac 52 P\Delta V + \frac 32 V \Delta P = \\
            &= \frac 52 P\Delta V + \frac 32 V \cdot \cbr{- \frac{P_0}{V_0} \Delta V} = \frac{\Delta V}2 \cdot \cbr{5P - \frac{P_0}{V_0} V} = \\
            &= \frac{\Delta V}2 \cdot \cbr{5 \cdot \cbr{4P_0 - \frac{P_0}{V_0} V} - \frac{P_0}{V_0} V}
             = \frac{\Delta V \cdot P_0}2 \cdot \cbr{5 \cdot 4 - 8\frac V{V_0}}.
        \end{align*}

        Таком образом, знак количества теплоты $Q$ на участке 23 зависит от конкретного значения $V$:
        \begin{itemize}
            \item $\Delta V > 0$ на всём участке 23, поскольку газ расширяется,
            \item $P > 0$ — всегда, у нас идеальный газ, удары о стенки сосуда абсолютно упругие, а молекулы не взаимодействуют и поэтому давление только положительно,
            \item если $5 \cdot 4 - 8\frac V{V_0} > 0$ — тепло подводят, если же меньше нуля — отводят.
        \end{itemize}
        Решая последнее неравенство, получаем конкретное значение $V^*$: при $V < V^*$ тепло подводят, далее~— отводят.
        Тут *~--- некоторая точка между точками 2 и 3, конкретные значения надо досчитать:
        $$V^* = V_0 \cdot \frac{5 \cdot 4}8 = \frac52 \cdot V_0 \implies P^* = 4P_0 - \frac{P_0}{V_0} V^* = \ldots = \frac32 \cdot P_0.$$

        Т.е.
        чтобы вычислить $Q_+$, надо сложить количества теплоты на участке 12 и лишь части участка 23 — участке 2*,
        той его части где это количество теплоты положительно.
        Имеем: $Q_+ = Q_{12} + Q_{2*}$.

        Теперь возвращаемся к циклу целиком и получаем:
        \begin{align*}
        A_\text{цикл} &= \frac 12 \cdot (3P_0 - P_0) \cdot (3V_0 - V_0) = 2 \cdot P_0V_0, \\
        A_{2*} &= \frac{P^* + 3P_0}2 \cdot (V^* - V_0)
            = \frac{\frac32 \cdot P_0 + 3P_0}2 \cdot \cbr{\frac52 \cdot V_0 - V_0}
            = \ldots = \frac{27}8 \cdot P_0 V_0, \\
        \Delta U_{2*} &= \frac 32 \nu R T^* - \frac 32 \nu R T_2 = \frac 32 (P^*V^* - P_0 \cdot 3V_0)
            = \frac 32 \cbr{\frac32 \cdot P_0 \cdot \frac52 \cdot V_0 - P_0 \cdot 3V_0}
            = \frac98 \cdot P_0 V_0, \\
        \Delta U_{12} &= \frac 32 \nu R T_2 - \frac 32 \nu R T_1 = \frac 32 (3P_0V_0 - P_0V_0) = \ldots = 3 \cdot P_0 V_0, \\
        \eta &= \frac{A_\text{цикл}}{Q_+} = \frac{A_\text{цикл}}{Q_{12} + Q_{2*}}
            = \frac{A_\text{цикл}}{A_{12} + \Delta U_{12} + A_{2*} + \Delta U_{2*}} = \\
            &= \frac{2 \cdot P_0V_0}{0 + 3 \cdot P_0 V_0 + \frac{27}8 \cdot P_0 V_0 + \frac98 \cdot P_0 V_0}
             = \frac{A_bonus_cycle:LaTeX}{3 + \frac{27}8 + \frac98}
             = \frac4{15} \leftarrow \text{вжух и готово!}
        \end{align*}
}
\solutionspace{360pt}

\tasknumber{2}%
\task{%
    При температуре $20\celsius$ относительная влажность воздуха составляет $75\%$.
    \begin{itemize}
        \item Определите точку росы для этого воздуха.
        \item Какой станет относительная влажность этого воздуха, если нагреть его до $50\celsius$?
    \end{itemize}
}
\answer{%
    \begin{align*}
    &\text{Значения плотности насыщенного водяного пара определяем по таблице:} \\
    &\rho_{\text{нас.
    пара 20} \celsius} = 17{,}300\,\frac{\text{г}}{\text{м}^{3}}, \rho_{\text{нас.
    пара 50} \celsius} = 83{,}000\,\frac{\text{г}}{\text{м}^{3}}.
    \\
    \varphi_1 &= \frac{\rho_\text{пара}}{\rho_{\text{нас.
    пара 20} \celsius}} \implies {\rho_\text{пара}} = \rho_{\text{нас.
    пара 20} \celsius} \cdot \varphi_1 = 17{,}300\,\frac{\text{г}}{\text{м}^{3}} \cdot 0{,}75 = 12{,}975\,\frac{\text{г}}{\text{м}^{3}}.
    \\
    &\text{По таблице определяем, при какой температуре пар с такой плотностью станет насыщенным:}  \\
    t_\text{росы} &= 15{,}2\celsius, \\
    \varphi_2 &= \frac{\rho_\text{пара}}{\rho_{\text{нас.
    пара 50} \celsius}} = \frac{\rho_{\text{нас.
    пара 20} \celsius} \cdot \varphi_1}{\rho_{\text{нас.
    пара 50} \celsius}}= \varphi_1 \cdot \frac{\rho_{\text{нас.
    пара 20} \celsius}}{\rho_{\text{нас.
    пара 50} \celsius}} = 0{,}75 \cdot \frac{17{,}300\,\frac{\text{г}}{\text{м}^{3}}}{83{,}000\,\frac{\text{г}}{\text{м}^{3}}} = 0{,}156 \approx 15{,}6\%.
    \end{align*}
}
\solutionspace{80pt}

\tasknumber{3}%
\task{%
    Из уравнения состояния идеального газа выведите или выразите...
    \begin{enumerate}
        \item объём,
        \item молярную массу,
        \item плотность газа.
    \end{enumerate}
}

\tasknumber{4}%
\task{%
    Запишите формулы и рядом с каждой физичической величиной укажите её название и единицы измерения в СИ:
    \begin{enumerate}
        \item первое начало термодинамики,
        \item внутренняя энергия идеального одноатомного газа.
    \end{enumerate}
}

\variantsplitter

\addpersonalvariant{Алина Полканова}

\tasknumber{1}%
\task{%
    Определите КПД (оставив ответ точным в виде нескоратимой дроби) цикла 1231, рабочим телом которого является идеальный одноатомный газ, если
    \begin{itemize}
        \item 12 — изохорический нагрев в два раза,
        \item 23 — изобарическое расширение, при котором температура растёт в пять раз,
        \item 31 — процесс, график которого в $PV$-координатах является отрезком прямой.
    \end{itemize}
    Бонус: замените цикл 1231 циклом, в котором 12 — изохорический нагрев в два раза, 23 — процесс, график которого в $PV$-координатах является отрезком прямой, 31 — изобарическое охлаждение, при котором температура падает в два раза.
}
\answer{%
    \begin{align*}
    A_{12} &= 0, \Delta U_{12} > 0, \implies Q_{12} = A_{12} + \Delta U_{12} > 0.
    \\
    A_{23} &> 0, \Delta U_{23} > 0, \implies Q_{23} = A_{23} + \Delta U_{23} > 0, \\
    A_{31} &= 0, \Delta U_{31} < 0, \implies Q_{31} = A_{31} + \Delta U_{31} < 0.
    \\
    P_1V_1 &= \nu R T_1, P_2V_2 = \nu R T_2, P_3V_3 = \nu R T_3 \text{ — уравнения состояния идеального газа}, \\
    &\text{Пусть $P_0$, $V_0$, $T_0$ — давление, объём и температура в точке 1 (минимальные во всём цикле):} \\
    P_1 &= P_0, P_2 = P_3, V_1 = V_2 = V_0, \text{остальные соотношения нужно считать} \\
    T_2 &= 2T_1 = 2T_0 \text{(по условию)} \implies \frac{P_2}{P_1} = \frac{P_2V_0}{P_1V_0} = \frac{P_2 V_2}{P_1 V_1}= \frac{\nu R T_2}{\nu R T_1} = \frac{T_2}{T_1} = 2 \implies P_2 = 2 P_1 = 2 P_0, \\
    T_3 &= 5T_2 = 10T_0 \text{(по условию)} \implies \frac{V_3}{V_2} = \frac{P_3V_3}{P_2V_2}= \frac{\nu R T_3}{\nu R T_2} = \frac{T_3}{T_2} = 5 \implies V_3 = 5 V_2 = 5 V_0.
    \\
    A_\text{цикл} &= \frac 12 (5P_0 - P_0)(2V_0 - V_0) = \frac 12 \cdot 4 \cdot P_0V_0, \\
    A_{23} &= 2P_0 \cdot (5V_0 - V_0) = 8P_0V_0, \\
    \Delta U_{23} &= \frac 32 \nu R T_3 - \frac 32 \nu R T_2 = \frac 32 P_3 V_3 - \frac 32 P_2 V_2 = \frac 32 \cdot 2 P_0 \cdot 5 V_0 -  \frac 32 \cdot 2 P_0 \cdot V_0 = \frac 32 \cdot 8 \cdot P_0V_0, \\
    \Delta U_{12} &= \frac 32 \nu R T_2 - \frac 32 \nu R T_1 = \frac 32 P_2 V_2 - \frac 32 P_1 V_1 = \frac 32 \cdot 2 P_0V_0 - \frac 32 P_0V_0 = \frac 32 \cdot 1 \cdot P_0V_0.
    \\
    \eta &= \frac{A_\text{цикл}}{Q_+} = \frac{A_\text{цикл}}{Q_{12} + Q_{23}}  = \frac{A_\text{цикл}}{A_{12} + \Delta U_{12} + A_{23} + \Delta U_{23}} =  \\
     &= \frac{\frac 12 \cdot 4 \cdot P_0V_0}{0 + \frac 32 \cdot 1 \cdot P_0V_0 + 8P_0V_0 + \frac 32 \cdot 8 \cdot P_0V_0} = \frac{\frac 12 \cdot 4}{\frac 32 \cdot 1 + 8 + \frac 32 \cdot 8} = \frac4{43} \approx 0.093.
    \end{align*}


        График процесса не в масштабе (эта часть пока не готова и сделать автоматически аккуратно сложно), но с верными подписями (а для решения этого достаточно):

        \begin{tikzpicture}[thick]
            \draw[-{Latex}] (0, 0) -- (0, 7) node[above left] {$P$};
            \draw[-{Latex}] (0, 0) -- (10, 0) node[right] {$V$};

            \draw[dashed] (0, 2) node[left] {$P_1 = P_0$} -| (3, 0) node[below] {$V_1 = V_2 = V_0$};
            \draw[dashed] (0, 6) node[left] {$P_2 = P_3 = 2P_0$} -| (9, 0) node[below] {$V_3 = 5V_0$};

            \draw (3, 2) node[above left]{1} node[below left]{$T_1 = T_0$}
                   (3, 6) node[below left]{2} node[above]{$T_2 = 2T_0$}
                   (9, 6) node[above right]{3} node[below right]{$T_3 = 10T_0$};
            \draw[midar] (3, 2) -- (3, 6);
            \draw[midar] (3, 6) -- (9, 6);
            \draw[midar] (9, 6) -- (3, 2);
        \end{tikzpicture}

        Решение бонуса:
        \begin{align*}
            A_{12} &= 0, \Delta U_{12} > 0, \implies Q_{12} = A_{12} + \Delta U_{12} > 0, \\
            A_{23} &> 0, \Delta U_{23} \text{ — ничего нельзя сказать, нужно исследовать отдельно}, \\
            A_{31} &< 0, \Delta U_{31} < 0, \implies Q_{31} = A_{31} + \Delta U_{31} < 0.
            \\
        \end{align*}

        Уравнения состояния идеального газа для точек 1, 2, 3: $P_1V_1 = \nu R T_1, P_2V_2 = \nu R T_2, P_3V_3 = \nu R T_3$.
        Пусть $P_0$, $V_0$, $T_0$ — давление, объём и температура в точке 1 (минимальные во всём цикле).

        12 --- изохора, $\frac{P_1V_1}{T_1} = \nu R = \frac{P_2V_2}{T_2}, V_2=V_1=V_0 \implies \frac{P_1}{T_1} =  \frac{P_2}{T_2} \implies P_2 = P_1 \frac{T_2}{T_1} = 2P_0$,

        31 --- изобара, $\frac{P_1V_1}{T_1} = \nu R = \frac{P_3V_3}{T_3}, P_3=P_1=P_0 \implies \frac{V_3}{T_3} =  \frac{V_1}{T_1} \implies V_3 = V_1 \frac{T_3}{T_1} = 2V_0$,

        Таким образом, используя новые обозначения, состояния газа в точках 1, 2 и 3 описываются макропараметрами $(P_0, V_0, T_0), (2P_0, V_0, 2T_0), (P_0, 2V_0, 2T_0)$ соответственно.

        \begin{tikzpicture}[thick]
            \draw[-{Latex}] (0, 0) -- (0, 7) node[above left] {$P$};
            \draw[-{Latex}] (0, 0) -- (10, 0) node[right] {$V$};

            \draw[dashed] (0, 2) node[left] {$P_1 = P_3 = P_0$} -| (9, 0) node[below] {$V_3 = 2V_0$};
            \draw[dashed] (0, 6) node[left] {$P_2 = 2P_0$} -| (3, 0) node[below] {$V_1 = V_2 = V_0$};

            \draw[dashed] (0, 5) node[left] {$P$} -| (4.5, 0) node[below] {$V$};
            \draw[dashed] (0, 4.6) node[left] {$P'$} -| (5.1, 0) node[below] {$V'$};

            \draw (3, 2) node[above left]{1} node[below left]{$T_1 = T_0$}
                   (3, 6) node[below left]{2} node[above]{$T_2 = 2T_0$}
                   (9, 2) node[above right]{3} node[below right]{$T_3 = 2T_0$};
            \draw[midar] (3, 2) -- (3, 6);
            \draw[midar] (3, 6) -- (9, 2);
            \draw[midar] (9, 2) -- (3, 2);
            \draw   (4.5, 5) node[above right]{$T$} (5.1, 4.6) node[above right]{$T'$};
        \end{tikzpicture}


        Теперь рассмотрим отдельно процесс 23, к остальному вернёмся позже.
        Уравнение этой прямой в $PV$-координатах: $P(V) = 3P_0 - \frac{P_0}{V_0} V$.
        Это значит, что при изменении объёма на $\Delta V$ давление изменится на $\Delta P = - \frac{P_0}{V_0} \Delta V$, обратите внимание на знак.

        Рассмотрим произвольную точку в процессе 23 и дадим процессу ещё немного свершиться, при этом объём изменится на $\Delta V$, давление — на $\Delta P$, температура (иначе бы была гипербола, а не прямая) — на $\Delta T$,
        т.е.
        из состояния $(P, V, T)$ мы перешли в $(P', V', T')$, причём  $P' = P + \Delta P, V' = V + \Delta V, T' = T + \Delta T$.

        При этом изменится внутренняя энергия:
        \begin{align*}
        \Delta U
            &= U' - U = \frac 32 \nu R T' - \frac 32 \nu R T = \frac 32 (P+\Delta P) (V+\Delta V) - \frac 32 PV\\
            &= \frac 32 ((P+\Delta P) (V+\Delta V) - PV) = \frac 32 (P\Delta V + V \Delta P + \Delta P \Delta V).
        \end{align*}

        Рассмотрим малые изменения объёма, тогда и изменение давления будем малым (т.к.
        $\Delta P = - \frac{P_0}{V_0} \Delta V$),
        а третьим слагаемым в выражении для $\Delta U$  можно пренебречь по сравнению с двумя другими:
        два первых это малые величины, а третье — произведение двух малых.
        Тогда $\Delta U = \frac 32 (P\Delta V + V \Delta P)$.

        Работа газа при этом малом изменении объёма — это площадь трапеции (тут ещё раз пренебрегли малым слагаемым):
        $$A = \frac{P + P'}2 \Delta V = \cbr{P + \frac{\Delta P}2} \Delta V = P \Delta V.$$

        Подведённое количество теплоты, используя первое начало термодинамики, будет равно
        \begin{align*}
        Q
            &= \frac 32 (P\Delta V + V \Delta P) + P \Delta V =  \frac 52 P\Delta V + \frac 32 V \Delta P = \\
            &= \frac 52 P\Delta V + \frac 32 V \cdot \cbr{- \frac{P_0}{V_0} \Delta V} = \frac{\Delta V}2 \cdot \cbr{5P - \frac{P_0}{V_0} V} = \\
            &= \frac{\Delta V}2 \cdot \cbr{5 \cdot \cbr{3P_0 - \frac{P_0}{V_0} V} - \frac{P_0}{V_0} V}
             = \frac{\Delta V \cdot P_0}2 \cdot \cbr{5 \cdot 3 - 8\frac V{V_0}}.
        \end{align*}

        Таком образом, знак количества теплоты $Q$ на участке 23 зависит от конкретного значения $V$:
        \begin{itemize}
            \item $\Delta V > 0$ на всём участке 23, поскольку газ расширяется,
            \item $P > 0$ — всегда, у нас идеальный газ, удары о стенки сосуда абсолютно упругие, а молекулы не взаимодействуют и поэтому давление только положительно,
            \item если $5 \cdot 3 - 8\frac V{V_0} > 0$ — тепло подводят, если же меньше нуля — отводят.
        \end{itemize}
        Решая последнее неравенство, получаем конкретное значение $V^*$: при $V < V^*$ тепло подводят, далее~— отводят.
        Тут *~--- некоторая точка между точками 2 и 3, конкретные значения надо досчитать:
        $$V^* = V_0 \cdot \frac{5 \cdot 3}8 = \frac{15}8 \cdot V_0 \implies P^* = 3P_0 - \frac{P_0}{V_0} V^* = \ldots = \frac98 \cdot P_0.$$

        Т.е.
        чтобы вычислить $Q_+$, надо сложить количества теплоты на участке 12 и лишь части участка 23 — участке 2*,
        той его части где это количество теплоты положительно.
        Имеем: $Q_+ = Q_{12} + Q_{2*}$.

        Теперь возвращаемся к циклу целиком и получаем:
        \begin{align*}
        A_\text{цикл} &= \frac 12 \cdot (2P_0 - P_0) \cdot (2V_0 - V_0) = \frac12 \cdot P_0V_0, \\
        A_{2*} &= \frac{P^* + 2P_0}2 \cdot (V^* - V_0)
            = \frac{\frac98 \cdot P_0 + 2P_0}2 \cdot \cbr{\frac{15}8 \cdot V_0 - V_0}
            = \ldots = \frac{175}{128} \cdot P_0 V_0, \\
        \Delta U_{2*} &= \frac 32 \nu R T^* - \frac 32 \nu R T_2 = \frac 32 (P^*V^* - P_0 \cdot 2V_0)
            = \frac 32 \cbr{\frac98 \cdot P_0 \cdot \frac{15}8 \cdot V_0 - P_0 \cdot 2V_0}
            = \frac{21}{128} \cdot P_0 V_0, \\
        \Delta U_{12} &= \frac 32 \nu R T_2 - \frac 32 \nu R T_1 = \frac 32 (2P_0V_0 - P_0V_0) = \ldots = \frac32 \cdot P_0 V_0, \\
        \eta &= \frac{A_\text{цикл}}{Q_+} = \frac{A_\text{цикл}}{Q_{12} + Q_{2*}}
            = \frac{A_\text{цикл}}{A_{12} + \Delta U_{12} + A_{2*} + \Delta U_{2*}} = \\
            &= \frac{\frac12 \cdot P_0V_0}{0 + \frac32 \cdot P_0 V_0 + \frac{175}{128} \cdot P_0 V_0 + \frac{21}{128} \cdot P_0 V_0}
             = \frac{A_bonus_cycle:LaTeX}{\frac32 + \frac{175}{128} + \frac{21}{128}}
             = \frac{16}{97} \leftarrow \text{вжух и готово!}
        \end{align*}
}
\solutionspace{360pt}

\tasknumber{2}%
\task{%
    При температуре $25\celsius$ относительная влажность воздуха составляет $75\%$.
    \begin{itemize}
        \item Определите точку росы для этого воздуха.
        \item Какой станет относительная влажность этого воздуха, если нагреть его до $60\celsius$?
    \end{itemize}
}
\answer{%
    \begin{align*}
    &\text{Значения плотности насыщенного водяного пара определяем по таблице:} \\
    &\rho_{\text{нас.
    пара 25} \celsius} = 23{,}000\,\frac{\text{г}}{\text{м}^{3}}, \rho_{\text{нас.
    пара 60} \celsius} = 130{,}000\,\frac{\text{г}}{\text{м}^{3}}.
    \\
    \varphi_1 &= \frac{\rho_\text{пара}}{\rho_{\text{нас.
    пара 25} \celsius}} \implies {\rho_\text{пара}} = \rho_{\text{нас.
    пара 25} \celsius} \cdot \varphi_1 = 23{,}000\,\frac{\text{г}}{\text{м}^{3}} \cdot 0{,}75 = 17{,}250\,\frac{\text{г}}{\text{м}^{3}}.
    \\
    &\text{По таблице определяем, при какой температуре пар с такой плотностью станет насыщенным:}  \\
    t_\text{росы} &= 19{,}9\celsius, \\
    \varphi_2 &= \frac{\rho_\text{пара}}{\rho_{\text{нас.
    пара 60} \celsius}} = \frac{\rho_{\text{нас.
    пара 25} \celsius} \cdot \varphi_1}{\rho_{\text{нас.
    пара 60} \celsius}}= \varphi_1 \cdot \frac{\rho_{\text{нас.
    пара 25} \celsius}}{\rho_{\text{нас.
    пара 60} \celsius}} = 0{,}75 \cdot \frac{23{,}000\,\frac{\text{г}}{\text{м}^{3}}}{130{,}000\,\frac{\text{г}}{\text{м}^{3}}} = 0{,}133 \approx 13{,}3\%.
    \end{align*}
}
\solutionspace{80pt}

\tasknumber{3}%
\task{%
    Из уравнения состояния идеального газа выведите или выразите...
    \begin{enumerate}
        \item давление,
        \item молярную массу,
        \item концентрацию молекул газа.
    \end{enumerate}
}

\tasknumber{4}%
\task{%
    Запишите формулы и рядом с каждой физичической величиной укажите её название и единицы измерения в СИ:
    \begin{enumerate}
        \item первое начало термодинамики,
        \item внутренняя энергия идеального одноатомного газа.
    \end{enumerate}
}

\variantsplitter

\addpersonalvariant{Сергей Пономарёв}

\tasknumber{1}%
\task{%
    Определите КПД (оставив ответ точным в виде нескоратимой дроби) цикла 1231, рабочим телом которого является идеальный одноатомный газ, если
    \begin{itemize}
        \item 12 — изохорический нагрев в пять раз,
        \item 23 — изобарическое расширение, при котором температура растёт в шесть раз,
        \item 31 — процесс, график которого в $PV$-координатах является отрезком прямой.
    \end{itemize}
    Бонус: замените цикл 1231 циклом, в котором 12 — изохорический нагрев в пять раз, 23 — процесс, график которого в $PV$-координатах является отрезком прямой, 31 — изобарическое охлаждение, при котором температура падает в пять раз.
}
\answer{%
    \begin{align*}
    A_{12} &= 0, \Delta U_{12} > 0, \implies Q_{12} = A_{12} + \Delta U_{12} > 0.
    \\
    A_{23} &> 0, \Delta U_{23} > 0, \implies Q_{23} = A_{23} + \Delta U_{23} > 0, \\
    A_{31} &= 0, \Delta U_{31} < 0, \implies Q_{31} = A_{31} + \Delta U_{31} < 0.
    \\
    P_1V_1 &= \nu R T_1, P_2V_2 = \nu R T_2, P_3V_3 = \nu R T_3 \text{ — уравнения состояния идеального газа}, \\
    &\text{Пусть $P_0$, $V_0$, $T_0$ — давление, объём и температура в точке 1 (минимальные во всём цикле):} \\
    P_1 &= P_0, P_2 = P_3, V_1 = V_2 = V_0, \text{остальные соотношения нужно считать} \\
    T_2 &= 5T_1 = 5T_0 \text{(по условию)} \implies \frac{P_2}{P_1} = \frac{P_2V_0}{P_1V_0} = \frac{P_2 V_2}{P_1 V_1}= \frac{\nu R T_2}{\nu R T_1} = \frac{T_2}{T_1} = 5 \implies P_2 = 5 P_1 = 5 P_0, \\
    T_3 &= 6T_2 = 30T_0 \text{(по условию)} \implies \frac{V_3}{V_2} = \frac{P_3V_3}{P_2V_2}= \frac{\nu R T_3}{\nu R T_2} = \frac{T_3}{T_2} = 6 \implies V_3 = 6 V_2 = 6 V_0.
    \\
    A_\text{цикл} &= \frac 12 (6P_0 - P_0)(5V_0 - V_0) = \frac 12 \cdot 20 \cdot P_0V_0, \\
    A_{23} &= 5P_0 \cdot (6V_0 - V_0) = 25P_0V_0, \\
    \Delta U_{23} &= \frac 32 \nu R T_3 - \frac 32 \nu R T_2 = \frac 32 P_3 V_3 - \frac 32 P_2 V_2 = \frac 32 \cdot 5 P_0 \cdot 6 V_0 -  \frac 32 \cdot 5 P_0 \cdot V_0 = \frac 32 \cdot 25 \cdot P_0V_0, \\
    \Delta U_{12} &= \frac 32 \nu R T_2 - \frac 32 \nu R T_1 = \frac 32 P_2 V_2 - \frac 32 P_1 V_1 = \frac 32 \cdot 5 P_0V_0 - \frac 32 P_0V_0 = \frac 32 \cdot 4 \cdot P_0V_0.
    \\
    \eta &= \frac{A_\text{цикл}}{Q_+} = \frac{A_\text{цикл}}{Q_{12} + Q_{23}}  = \frac{A_\text{цикл}}{A_{12} + \Delta U_{12} + A_{23} + \Delta U_{23}} =  \\
     &= \frac{\frac 12 \cdot 20 \cdot P_0V_0}{0 + \frac 32 \cdot 4 \cdot P_0V_0 + 25P_0V_0 + \frac 32 \cdot 25 \cdot P_0V_0} = \frac{\frac 12 \cdot 20}{\frac 32 \cdot 4 + 25 + \frac 32 \cdot 25} = \frac{20}{137} \approx 0.146.
    \end{align*}


        График процесса не в масштабе (эта часть пока не готова и сделать автоматически аккуратно сложно), но с верными подписями (а для решения этого достаточно):

        \begin{tikzpicture}[thick]
            \draw[-{Latex}] (0, 0) -- (0, 7) node[above left] {$P$};
            \draw[-{Latex}] (0, 0) -- (10, 0) node[right] {$V$};

            \draw[dashed] (0, 2) node[left] {$P_1 = P_0$} -| (3, 0) node[below] {$V_1 = V_2 = V_0$};
            \draw[dashed] (0, 6) node[left] {$P_2 = P_3 = 5P_0$} -| (9, 0) node[below] {$V_3 = 6V_0$};

            \draw (3, 2) node[above left]{1} node[below left]{$T_1 = T_0$}
                   (3, 6) node[below left]{2} node[above]{$T_2 = 5T_0$}
                   (9, 6) node[above right]{3} node[below right]{$T_3 = 30T_0$};
            \draw[midar] (3, 2) -- (3, 6);
            \draw[midar] (3, 6) -- (9, 6);
            \draw[midar] (9, 6) -- (3, 2);
        \end{tikzpicture}

        Решение бонуса:
        \begin{align*}
            A_{12} &= 0, \Delta U_{12} > 0, \implies Q_{12} = A_{12} + \Delta U_{12} > 0, \\
            A_{23} &> 0, \Delta U_{23} \text{ — ничего нельзя сказать, нужно исследовать отдельно}, \\
            A_{31} &< 0, \Delta U_{31} < 0, \implies Q_{31} = A_{31} + \Delta U_{31} < 0.
            \\
        \end{align*}

        Уравнения состояния идеального газа для точек 1, 2, 3: $P_1V_1 = \nu R T_1, P_2V_2 = \nu R T_2, P_3V_3 = \nu R T_3$.
        Пусть $P_0$, $V_0$, $T_0$ — давление, объём и температура в точке 1 (минимальные во всём цикле).

        12 --- изохора, $\frac{P_1V_1}{T_1} = \nu R = \frac{P_2V_2}{T_2}, V_2=V_1=V_0 \implies \frac{P_1}{T_1} =  \frac{P_2}{T_2} \implies P_2 = P_1 \frac{T_2}{T_1} = 5P_0$,

        31 --- изобара, $\frac{P_1V_1}{T_1} = \nu R = \frac{P_3V_3}{T_3}, P_3=P_1=P_0 \implies \frac{V_3}{T_3} =  \frac{V_1}{T_1} \implies V_3 = V_1 \frac{T_3}{T_1} = 5V_0$,

        Таким образом, используя новые обозначения, состояния газа в точках 1, 2 и 3 описываются макропараметрами $(P_0, V_0, T_0), (5P_0, V_0, 5T_0), (P_0, 5V_0, 5T_0)$ соответственно.

        \begin{tikzpicture}[thick]
            \draw[-{Latex}] (0, 0) -- (0, 7) node[above left] {$P$};
            \draw[-{Latex}] (0, 0) -- (10, 0) node[right] {$V$};

            \draw[dashed] (0, 2) node[left] {$P_1 = P_3 = P_0$} -| (9, 0) node[below] {$V_3 = 5V_0$};
            \draw[dashed] (0, 6) node[left] {$P_2 = 5P_0$} -| (3, 0) node[below] {$V_1 = V_2 = V_0$};

            \draw[dashed] (0, 5) node[left] {$P$} -| (4.5, 0) node[below] {$V$};
            \draw[dashed] (0, 4.6) node[left] {$P'$} -| (5.1, 0) node[below] {$V'$};

            \draw (3, 2) node[above left]{1} node[below left]{$T_1 = T_0$}
                   (3, 6) node[below left]{2} node[above]{$T_2 = 5T_0$}
                   (9, 2) node[above right]{3} node[below right]{$T_3 = 5T_0$};
            \draw[midar] (3, 2) -- (3, 6);
            \draw[midar] (3, 6) -- (9, 2);
            \draw[midar] (9, 2) -- (3, 2);
            \draw   (4.5, 5) node[above right]{$T$} (5.1, 4.6) node[above right]{$T'$};
        \end{tikzpicture}


        Теперь рассмотрим отдельно процесс 23, к остальному вернёмся позже.
        Уравнение этой прямой в $PV$-координатах: $P(V) = 6P_0 - \frac{P_0}{V_0} V$.
        Это значит, что при изменении объёма на $\Delta V$ давление изменится на $\Delta P = - \frac{P_0}{V_0} \Delta V$, обратите внимание на знак.

        Рассмотрим произвольную точку в процессе 23 и дадим процессу ещё немного свершиться, при этом объём изменится на $\Delta V$, давление — на $\Delta P$, температура (иначе бы была гипербола, а не прямая) — на $\Delta T$,
        т.е.
        из состояния $(P, V, T)$ мы перешли в $(P', V', T')$, причём  $P' = P + \Delta P, V' = V + \Delta V, T' = T + \Delta T$.

        При этом изменится внутренняя энергия:
        \begin{align*}
        \Delta U
            &= U' - U = \frac 32 \nu R T' - \frac 32 \nu R T = \frac 32 (P+\Delta P) (V+\Delta V) - \frac 32 PV\\
            &= \frac 32 ((P+\Delta P) (V+\Delta V) - PV) = \frac 32 (P\Delta V + V \Delta P + \Delta P \Delta V).
        \end{align*}

        Рассмотрим малые изменения объёма, тогда и изменение давления будем малым (т.к.
        $\Delta P = - \frac{P_0}{V_0} \Delta V$),
        а третьим слагаемым в выражении для $\Delta U$  можно пренебречь по сравнению с двумя другими:
        два первых это малые величины, а третье — произведение двух малых.
        Тогда $\Delta U = \frac 32 (P\Delta V + V \Delta P)$.

        Работа газа при этом малом изменении объёма — это площадь трапеции (тут ещё раз пренебрегли малым слагаемым):
        $$A = \frac{P + P'}2 \Delta V = \cbr{P + \frac{\Delta P}2} \Delta V = P \Delta V.$$

        Подведённое количество теплоты, используя первое начало термодинамики, будет равно
        \begin{align*}
        Q
            &= \frac 32 (P\Delta V + V \Delta P) + P \Delta V =  \frac 52 P\Delta V + \frac 32 V \Delta P = \\
            &= \frac 52 P\Delta V + \frac 32 V \cdot \cbr{- \frac{P_0}{V_0} \Delta V} = \frac{\Delta V}2 \cdot \cbr{5P - \frac{P_0}{V_0} V} = \\
            &= \frac{\Delta V}2 \cdot \cbr{5 \cdot \cbr{6P_0 - \frac{P_0}{V_0} V} - \frac{P_0}{V_0} V}
             = \frac{\Delta V \cdot P_0}2 \cdot \cbr{5 \cdot 6 - 8\frac V{V_0}}.
        \end{align*}

        Таком образом, знак количества теплоты $Q$ на участке 23 зависит от конкретного значения $V$:
        \begin{itemize}
            \item $\Delta V > 0$ на всём участке 23, поскольку газ расширяется,
            \item $P > 0$ — всегда, у нас идеальный газ, удары о стенки сосуда абсолютно упругие, а молекулы не взаимодействуют и поэтому давление только положительно,
            \item если $5 \cdot 6 - 8\frac V{V_0} > 0$ — тепло подводят, если же меньше нуля — отводят.
        \end{itemize}
        Решая последнее неравенство, получаем конкретное значение $V^*$: при $V < V^*$ тепло подводят, далее~— отводят.
        Тут *~--- некоторая точка между точками 2 и 3, конкретные значения надо досчитать:
        $$V^* = V_0 \cdot \frac{5 \cdot 6}8 = \frac{15}4 \cdot V_0 \implies P^* = 6P_0 - \frac{P_0}{V_0} V^* = \ldots = \frac94 \cdot P_0.$$

        Т.е.
        чтобы вычислить $Q_+$, надо сложить количества теплоты на участке 12 и лишь части участка 23 — участке 2*,
        той его части где это количество теплоты положительно.
        Имеем: $Q_+ = Q_{12} + Q_{2*}$.

        Теперь возвращаемся к циклу целиком и получаем:
        \begin{align*}
        A_\text{цикл} &= \frac 12 \cdot (5P_0 - P_0) \cdot (5V_0 - V_0) = 8 \cdot P_0V_0, \\
        A_{2*} &= \frac{P^* + 5P_0}2 \cdot (V^* - V_0)
            = \frac{\frac94 \cdot P_0 + 5P_0}2 \cdot \cbr{\frac{15}4 \cdot V_0 - V_0}
            = \ldots = \frac{319}{32} \cdot P_0 V_0, \\
        \Delta U_{2*} &= \frac 32 \nu R T^* - \frac 32 \nu R T_2 = \frac 32 (P^*V^* - P_0 \cdot 5V_0)
            = \frac 32 \cbr{\frac94 \cdot P_0 \cdot \frac{15}4 \cdot V_0 - P_0 \cdot 5V_0}
            = \frac{165}{32} \cdot P_0 V_0, \\
        \Delta U_{12} &= \frac 32 \nu R T_2 - \frac 32 \nu R T_1 = \frac 32 (5P_0V_0 - P_0V_0) = \ldots = 6 \cdot P_0 V_0, \\
        \eta &= \frac{A_\text{цикл}}{Q_+} = \frac{A_\text{цикл}}{Q_{12} + Q_{2*}}
            = \frac{A_\text{цикл}}{A_{12} + \Delta U_{12} + A_{2*} + \Delta U_{2*}} = \\
            &= \frac{8 \cdot P_0V_0}{0 + 6 \cdot P_0 V_0 + \frac{319}{32} \cdot P_0 V_0 + \frac{165}{32} \cdot P_0 V_0}
             = \frac{A_bonus_cycle:LaTeX}{6 + \frac{319}{32} + \frac{165}{32}}
             = \frac{64}{169} \leftarrow \text{вжух и готово!}
        \end{align*}
}
\solutionspace{360pt}

\tasknumber{2}%
\task{%
    При температуре $20\celsius$ относительная влажность воздуха составляет $50\%$.
    \begin{itemize}
        \item Определите точку росы для этого воздуха.
        \item Какой станет относительная влажность этого воздуха, если нагреть его до $50\celsius$?
    \end{itemize}
}
\answer{%
    \begin{align*}
    &\text{Значения плотности насыщенного водяного пара определяем по таблице:} \\
    &\rho_{\text{нас.
    пара 20} \celsius} = 17{,}300\,\frac{\text{г}}{\text{м}^{3}}, \rho_{\text{нас.
    пара 50} \celsius} = 83{,}000\,\frac{\text{г}}{\text{м}^{3}}.
    \\
    \varphi_1 &= \frac{\rho_\text{пара}}{\rho_{\text{нас.
    пара 20} \celsius}} \implies {\rho_\text{пара}} = \rho_{\text{нас.
    пара 20} \celsius} \cdot \varphi_1 = 17{,}300\,\frac{\text{г}}{\text{м}^{3}} \cdot 0{,}50 = 8{,}650\,\frac{\text{г}}{\text{м}^{3}}.
    \\
    &\text{По таблице определяем, при какой температуре пар с такой плотностью станет насыщенным:}  \\
    t_\text{росы} &= 8{,}7\celsius, \\
    \varphi_2 &= \frac{\rho_\text{пара}}{\rho_{\text{нас.
    пара 50} \celsius}} = \frac{\rho_{\text{нас.
    пара 20} \celsius} \cdot \varphi_1}{\rho_{\text{нас.
    пара 50} \celsius}}= \varphi_1 \cdot \frac{\rho_{\text{нас.
    пара 20} \celsius}}{\rho_{\text{нас.
    пара 50} \celsius}} = 0{,}50 \cdot \frac{17{,}300\,\frac{\text{г}}{\text{м}^{3}}}{83{,}000\,\frac{\text{г}}{\text{м}^{3}}} = 0{,}104 \approx 10{,}4\%.
    \end{align*}
}
\solutionspace{80pt}

\tasknumber{3}%
\task{%
    Из уравнения состояния идеального газа выведите или выразите...
    \begin{enumerate}
        \item давление,
        \item молярную массу,
        \item плотность газа.
    \end{enumerate}
}

\tasknumber{4}%
\task{%
    Запишите формулы и рядом с каждой физичической величиной укажите её название и единицы измерения в СИ:
    \begin{enumerate}
        \item первое начало термодинамики,
        \item внутренняя энергия идеального одноатомного газа.
    \end{enumerate}
}

\variantsplitter

\addpersonalvariant{Егор Свистушкин}

\tasknumber{1}%
\task{%
    Определите КПД (оставив ответ точным в виде нескоратимой дроби) цикла 1231, рабочим телом которого является идеальный одноатомный газ, если
    \begin{itemize}
        \item 12 — изохорический нагрев в два раза,
        \item 23 — изобарическое расширение, при котором температура растёт в два раза,
        \item 31 — процесс, график которого в $PV$-координатах является отрезком прямой.
    \end{itemize}
    Бонус: замените цикл 1231 циклом, в котором 12 — изохорический нагрев в два раза, 23 — процесс, график которого в $PV$-координатах является отрезком прямой, 31 — изобарическое охлаждение, при котором температура падает в два раза.
}
\answer{%
    \begin{align*}
    A_{12} &= 0, \Delta U_{12} > 0, \implies Q_{12} = A_{12} + \Delta U_{12} > 0.
    \\
    A_{23} &> 0, \Delta U_{23} > 0, \implies Q_{23} = A_{23} + \Delta U_{23} > 0, \\
    A_{31} &= 0, \Delta U_{31} < 0, \implies Q_{31} = A_{31} + \Delta U_{31} < 0.
    \\
    P_1V_1 &= \nu R T_1, P_2V_2 = \nu R T_2, P_3V_3 = \nu R T_3 \text{ — уравнения состояния идеального газа}, \\
    &\text{Пусть $P_0$, $V_0$, $T_0$ — давление, объём и температура в точке 1 (минимальные во всём цикле):} \\
    P_1 &= P_0, P_2 = P_3, V_1 = V_2 = V_0, \text{остальные соотношения нужно считать} \\
    T_2 &= 2T_1 = 2T_0 \text{(по условию)} \implies \frac{P_2}{P_1} = \frac{P_2V_0}{P_1V_0} = \frac{P_2 V_2}{P_1 V_1}= \frac{\nu R T_2}{\nu R T_1} = \frac{T_2}{T_1} = 2 \implies P_2 = 2 P_1 = 2 P_0, \\
    T_3 &= 2T_2 = 4T_0 \text{(по условию)} \implies \frac{V_3}{V_2} = \frac{P_3V_3}{P_2V_2}= \frac{\nu R T_3}{\nu R T_2} = \frac{T_3}{T_2} = 2 \implies V_3 = 2 V_2 = 2 V_0.
    \\
    A_\text{цикл} &= \frac 12 (2P_0 - P_0)(2V_0 - V_0) = \frac 12 \cdot 1 \cdot P_0V_0, \\
    A_{23} &= 2P_0 \cdot (2V_0 - V_0) = 2P_0V_0, \\
    \Delta U_{23} &= \frac 32 \nu R T_3 - \frac 32 \nu R T_2 = \frac 32 P_3 V_3 - \frac 32 P_2 V_2 = \frac 32 \cdot 2 P_0 \cdot 2 V_0 -  \frac 32 \cdot 2 P_0 \cdot V_0 = \frac 32 \cdot 2 \cdot P_0V_0, \\
    \Delta U_{12} &= \frac 32 \nu R T_2 - \frac 32 \nu R T_1 = \frac 32 P_2 V_2 - \frac 32 P_1 V_1 = \frac 32 \cdot 2 P_0V_0 - \frac 32 P_0V_0 = \frac 32 \cdot 1 \cdot P_0V_0.
    \\
    \eta &= \frac{A_\text{цикл}}{Q_+} = \frac{A_\text{цикл}}{Q_{12} + Q_{23}}  = \frac{A_\text{цикл}}{A_{12} + \Delta U_{12} + A_{23} + \Delta U_{23}} =  \\
     &= \frac{\frac 12 \cdot 1 \cdot P_0V_0}{0 + \frac 32 \cdot 1 \cdot P_0V_0 + 2P_0V_0 + \frac 32 \cdot 2 \cdot P_0V_0} = \frac{\frac 12 \cdot 1}{\frac 32 \cdot 1 + 2 + \frac 32 \cdot 2} = \frac1{13} \approx 0.077.
    \end{align*}


        График процесса не в масштабе (эта часть пока не готова и сделать автоматически аккуратно сложно), но с верными подписями (а для решения этого достаточно):

        \begin{tikzpicture}[thick]
            \draw[-{Latex}] (0, 0) -- (0, 7) node[above left] {$P$};
            \draw[-{Latex}] (0, 0) -- (10, 0) node[right] {$V$};

            \draw[dashed] (0, 2) node[left] {$P_1 = P_0$} -| (3, 0) node[below] {$V_1 = V_2 = V_0$};
            \draw[dashed] (0, 6) node[left] {$P_2 = P_3 = 2P_0$} -| (9, 0) node[below] {$V_3 = 2V_0$};

            \draw (3, 2) node[above left]{1} node[below left]{$T_1 = T_0$}
                   (3, 6) node[below left]{2} node[above]{$T_2 = 2T_0$}
                   (9, 6) node[above right]{3} node[below right]{$T_3 = 4T_0$};
            \draw[midar] (3, 2) -- (3, 6);
            \draw[midar] (3, 6) -- (9, 6);
            \draw[midar] (9, 6) -- (3, 2);
        \end{tikzpicture}

        Решение бонуса:
        \begin{align*}
            A_{12} &= 0, \Delta U_{12} > 0, \implies Q_{12} = A_{12} + \Delta U_{12} > 0, \\
            A_{23} &> 0, \Delta U_{23} \text{ — ничего нельзя сказать, нужно исследовать отдельно}, \\
            A_{31} &< 0, \Delta U_{31} < 0, \implies Q_{31} = A_{31} + \Delta U_{31} < 0.
            \\
        \end{align*}

        Уравнения состояния идеального газа для точек 1, 2, 3: $P_1V_1 = \nu R T_1, P_2V_2 = \nu R T_2, P_3V_3 = \nu R T_3$.
        Пусть $P_0$, $V_0$, $T_0$ — давление, объём и температура в точке 1 (минимальные во всём цикле).

        12 --- изохора, $\frac{P_1V_1}{T_1} = \nu R = \frac{P_2V_2}{T_2}, V_2=V_1=V_0 \implies \frac{P_1}{T_1} =  \frac{P_2}{T_2} \implies P_2 = P_1 \frac{T_2}{T_1} = 2P_0$,

        31 --- изобара, $\frac{P_1V_1}{T_1} = \nu R = \frac{P_3V_3}{T_3}, P_3=P_1=P_0 \implies \frac{V_3}{T_3} =  \frac{V_1}{T_1} \implies V_3 = V_1 \frac{T_3}{T_1} = 2V_0$,

        Таким образом, используя новые обозначения, состояния газа в точках 1, 2 и 3 описываются макропараметрами $(P_0, V_0, T_0), (2P_0, V_0, 2T_0), (P_0, 2V_0, 2T_0)$ соответственно.

        \begin{tikzpicture}[thick]
            \draw[-{Latex}] (0, 0) -- (0, 7) node[above left] {$P$};
            \draw[-{Latex}] (0, 0) -- (10, 0) node[right] {$V$};

            \draw[dashed] (0, 2) node[left] {$P_1 = P_3 = P_0$} -| (9, 0) node[below] {$V_3 = 2V_0$};
            \draw[dashed] (0, 6) node[left] {$P_2 = 2P_0$} -| (3, 0) node[below] {$V_1 = V_2 = V_0$};

            \draw[dashed] (0, 5) node[left] {$P$} -| (4.5, 0) node[below] {$V$};
            \draw[dashed] (0, 4.6) node[left] {$P'$} -| (5.1, 0) node[below] {$V'$};

            \draw (3, 2) node[above left]{1} node[below left]{$T_1 = T_0$}
                   (3, 6) node[below left]{2} node[above]{$T_2 = 2T_0$}
                   (9, 2) node[above right]{3} node[below right]{$T_3 = 2T_0$};
            \draw[midar] (3, 2) -- (3, 6);
            \draw[midar] (3, 6) -- (9, 2);
            \draw[midar] (9, 2) -- (3, 2);
            \draw   (4.5, 5) node[above right]{$T$} (5.1, 4.6) node[above right]{$T'$};
        \end{tikzpicture}


        Теперь рассмотрим отдельно процесс 23, к остальному вернёмся позже.
        Уравнение этой прямой в $PV$-координатах: $P(V) = 3P_0 - \frac{P_0}{V_0} V$.
        Это значит, что при изменении объёма на $\Delta V$ давление изменится на $\Delta P = - \frac{P_0}{V_0} \Delta V$, обратите внимание на знак.

        Рассмотрим произвольную точку в процессе 23 и дадим процессу ещё немного свершиться, при этом объём изменится на $\Delta V$, давление — на $\Delta P$, температура (иначе бы была гипербола, а не прямая) — на $\Delta T$,
        т.е.
        из состояния $(P, V, T)$ мы перешли в $(P', V', T')$, причём  $P' = P + \Delta P, V' = V + \Delta V, T' = T + \Delta T$.

        При этом изменится внутренняя энергия:
        \begin{align*}
        \Delta U
            &= U' - U = \frac 32 \nu R T' - \frac 32 \nu R T = \frac 32 (P+\Delta P) (V+\Delta V) - \frac 32 PV\\
            &= \frac 32 ((P+\Delta P) (V+\Delta V) - PV) = \frac 32 (P\Delta V + V \Delta P + \Delta P \Delta V).
        \end{align*}

        Рассмотрим малые изменения объёма, тогда и изменение давления будем малым (т.к.
        $\Delta P = - \frac{P_0}{V_0} \Delta V$),
        а третьим слагаемым в выражении для $\Delta U$  можно пренебречь по сравнению с двумя другими:
        два первых это малые величины, а третье — произведение двух малых.
        Тогда $\Delta U = \frac 32 (P\Delta V + V \Delta P)$.

        Работа газа при этом малом изменении объёма — это площадь трапеции (тут ещё раз пренебрегли малым слагаемым):
        $$A = \frac{P + P'}2 \Delta V = \cbr{P + \frac{\Delta P}2} \Delta V = P \Delta V.$$

        Подведённое количество теплоты, используя первое начало термодинамики, будет равно
        \begin{align*}
        Q
            &= \frac 32 (P\Delta V + V \Delta P) + P \Delta V =  \frac 52 P\Delta V + \frac 32 V \Delta P = \\
            &= \frac 52 P\Delta V + \frac 32 V \cdot \cbr{- \frac{P_0}{V_0} \Delta V} = \frac{\Delta V}2 \cdot \cbr{5P - \frac{P_0}{V_0} V} = \\
            &= \frac{\Delta V}2 \cdot \cbr{5 \cdot \cbr{3P_0 - \frac{P_0}{V_0} V} - \frac{P_0}{V_0} V}
             = \frac{\Delta V \cdot P_0}2 \cdot \cbr{5 \cdot 3 - 8\frac V{V_0}}.
        \end{align*}

        Таком образом, знак количества теплоты $Q$ на участке 23 зависит от конкретного значения $V$:
        \begin{itemize}
            \item $\Delta V > 0$ на всём участке 23, поскольку газ расширяется,
            \item $P > 0$ — всегда, у нас идеальный газ, удары о стенки сосуда абсолютно упругие, а молекулы не взаимодействуют и поэтому давление только положительно,
            \item если $5 \cdot 3 - 8\frac V{V_0} > 0$ — тепло подводят, если же меньше нуля — отводят.
        \end{itemize}
        Решая последнее неравенство, получаем конкретное значение $V^*$: при $V < V^*$ тепло подводят, далее~— отводят.
        Тут *~--- некоторая точка между точками 2 и 3, конкретные значения надо досчитать:
        $$V^* = V_0 \cdot \frac{5 \cdot 3}8 = \frac{15}8 \cdot V_0 \implies P^* = 3P_0 - \frac{P_0}{V_0} V^* = \ldots = \frac98 \cdot P_0.$$

        Т.е.
        чтобы вычислить $Q_+$, надо сложить количества теплоты на участке 12 и лишь части участка 23 — участке 2*,
        той его части где это количество теплоты положительно.
        Имеем: $Q_+ = Q_{12} + Q_{2*}$.

        Теперь возвращаемся к циклу целиком и получаем:
        \begin{align*}
        A_\text{цикл} &= \frac 12 \cdot (2P_0 - P_0) \cdot (2V_0 - V_0) = \frac12 \cdot P_0V_0, \\
        A_{2*} &= \frac{P^* + 2P_0}2 \cdot (V^* - V_0)
            = \frac{\frac98 \cdot P_0 + 2P_0}2 \cdot \cbr{\frac{15}8 \cdot V_0 - V_0}
            = \ldots = \frac{175}{128} \cdot P_0 V_0, \\
        \Delta U_{2*} &= \frac 32 \nu R T^* - \frac 32 \nu R T_2 = \frac 32 (P^*V^* - P_0 \cdot 2V_0)
            = \frac 32 \cbr{\frac98 \cdot P_0 \cdot \frac{15}8 \cdot V_0 - P_0 \cdot 2V_0}
            = \frac{21}{128} \cdot P_0 V_0, \\
        \Delta U_{12} &= \frac 32 \nu R T_2 - \frac 32 \nu R T_1 = \frac 32 (2P_0V_0 - P_0V_0) = \ldots = \frac32 \cdot P_0 V_0, \\
        \eta &= \frac{A_\text{цикл}}{Q_+} = \frac{A_\text{цикл}}{Q_{12} + Q_{2*}}
            = \frac{A_\text{цикл}}{A_{12} + \Delta U_{12} + A_{2*} + \Delta U_{2*}} = \\
            &= \frac{\frac12 \cdot P_0V_0}{0 + \frac32 \cdot P_0 V_0 + \frac{175}{128} \cdot P_0 V_0 + \frac{21}{128} \cdot P_0 V_0}
             = \frac{A_bonus_cycle:LaTeX}{\frac32 + \frac{175}{128} + \frac{21}{128}}
             = \frac{16}{97} \leftarrow \text{вжух и готово!}
        \end{align*}
}
\solutionspace{360pt}

\tasknumber{2}%
\task{%
    При температуре $30\celsius$ относительная влажность воздуха составляет $50\%$.
    \begin{itemize}
        \item Определите точку росы для этого воздуха.
        \item Какой станет относительная влажность этого воздуха, если нагреть его до $60\celsius$?
    \end{itemize}
}
\answer{%
    \begin{align*}
    &\text{Значения плотности насыщенного водяного пара определяем по таблице:} \\
    &\rho_{\text{нас.
    пара 30} \celsius} = 30{,}300\,\frac{\text{г}}{\text{м}^{3}}, \rho_{\text{нас.
    пара 60} \celsius} = 130{,}000\,\frac{\text{г}}{\text{м}^{3}}.
    \\
    \varphi_1 &= \frac{\rho_\text{пара}}{\rho_{\text{нас.
    пара 30} \celsius}} \implies {\rho_\text{пара}} = \rho_{\text{нас.
    пара 30} \celsius} \cdot \varphi_1 = 30{,}300\,\frac{\text{г}}{\text{м}^{3}} \cdot 0{,}50 = 15{,}150\,\frac{\text{г}}{\text{м}^{3}}.
    \\
    &\text{По таблице определяем, при какой температуре пар с такой плотностью станет насыщенным:}  \\
    t_\text{росы} &= 17{,}7\celsius, \\
    \varphi_2 &= \frac{\rho_\text{пара}}{\rho_{\text{нас.
    пара 60} \celsius}} = \frac{\rho_{\text{нас.
    пара 30} \celsius} \cdot \varphi_1}{\rho_{\text{нас.
    пара 60} \celsius}}= \varphi_1 \cdot \frac{\rho_{\text{нас.
    пара 30} \celsius}}{\rho_{\text{нас.
    пара 60} \celsius}} = 0{,}50 \cdot \frac{30{,}300\,\frac{\text{г}}{\text{м}^{3}}}{130{,}000\,\frac{\text{г}}{\text{м}^{3}}} = 0{,}117 \approx 11{,}7\%.
    \end{align*}
}
\solutionspace{80pt}

\tasknumber{3}%
\task{%
    Из уравнения состояния идеального газа выведите или выразите...
    \begin{enumerate}
        \item давление,
        \item температуру,
        \item концентрацию молекул газа.
    \end{enumerate}
}

\tasknumber{4}%
\task{%
    Запишите формулы и рядом с каждой физичической величиной укажите её название и единицы измерения в СИ:
    \begin{enumerate}
        \item первое начало термодинамики,
        \item внутренняя энергия идеального одноатомного газа.
    \end{enumerate}
}

\variantsplitter

\addpersonalvariant{Дмитрий Соколов}

\tasknumber{1}%
\task{%
    Определите КПД (оставив ответ точным в виде нескоратимой дроби) цикла 1231, рабочим телом которого является идеальный одноатомный газ, если
    \begin{itemize}
        \item 12 — изохорический нагрев в пять раз,
        \item 23 — изобарическое расширение, при котором температура растёт в шесть раз,
        \item 31 — процесс, график которого в $PV$-координатах является отрезком прямой.
    \end{itemize}
    Бонус: замените цикл 1231 циклом, в котором 12 — изохорический нагрев в пять раз, 23 — процесс, график которого в $PV$-координатах является отрезком прямой, 31 — изобарическое охлаждение, при котором температура падает в пять раз.
}
\answer{%
    \begin{align*}
    A_{12} &= 0, \Delta U_{12} > 0, \implies Q_{12} = A_{12} + \Delta U_{12} > 0.
    \\
    A_{23} &> 0, \Delta U_{23} > 0, \implies Q_{23} = A_{23} + \Delta U_{23} > 0, \\
    A_{31} &= 0, \Delta U_{31} < 0, \implies Q_{31} = A_{31} + \Delta U_{31} < 0.
    \\
    P_1V_1 &= \nu R T_1, P_2V_2 = \nu R T_2, P_3V_3 = \nu R T_3 \text{ — уравнения состояния идеального газа}, \\
    &\text{Пусть $P_0$, $V_0$, $T_0$ — давление, объём и температура в точке 1 (минимальные во всём цикле):} \\
    P_1 &= P_0, P_2 = P_3, V_1 = V_2 = V_0, \text{остальные соотношения нужно считать} \\
    T_2 &= 5T_1 = 5T_0 \text{(по условию)} \implies \frac{P_2}{P_1} = \frac{P_2V_0}{P_1V_0} = \frac{P_2 V_2}{P_1 V_1}= \frac{\nu R T_2}{\nu R T_1} = \frac{T_2}{T_1} = 5 \implies P_2 = 5 P_1 = 5 P_0, \\
    T_3 &= 6T_2 = 30T_0 \text{(по условию)} \implies \frac{V_3}{V_2} = \frac{P_3V_3}{P_2V_2}= \frac{\nu R T_3}{\nu R T_2} = \frac{T_3}{T_2} = 6 \implies V_3 = 6 V_2 = 6 V_0.
    \\
    A_\text{цикл} &= \frac 12 (6P_0 - P_0)(5V_0 - V_0) = \frac 12 \cdot 20 \cdot P_0V_0, \\
    A_{23} &= 5P_0 \cdot (6V_0 - V_0) = 25P_0V_0, \\
    \Delta U_{23} &= \frac 32 \nu R T_3 - \frac 32 \nu R T_2 = \frac 32 P_3 V_3 - \frac 32 P_2 V_2 = \frac 32 \cdot 5 P_0 \cdot 6 V_0 -  \frac 32 \cdot 5 P_0 \cdot V_0 = \frac 32 \cdot 25 \cdot P_0V_0, \\
    \Delta U_{12} &= \frac 32 \nu R T_2 - \frac 32 \nu R T_1 = \frac 32 P_2 V_2 - \frac 32 P_1 V_1 = \frac 32 \cdot 5 P_0V_0 - \frac 32 P_0V_0 = \frac 32 \cdot 4 \cdot P_0V_0.
    \\
    \eta &= \frac{A_\text{цикл}}{Q_+} = \frac{A_\text{цикл}}{Q_{12} + Q_{23}}  = \frac{A_\text{цикл}}{A_{12} + \Delta U_{12} + A_{23} + \Delta U_{23}} =  \\
     &= \frac{\frac 12 \cdot 20 \cdot P_0V_0}{0 + \frac 32 \cdot 4 \cdot P_0V_0 + 25P_0V_0 + \frac 32 \cdot 25 \cdot P_0V_0} = \frac{\frac 12 \cdot 20}{\frac 32 \cdot 4 + 25 + \frac 32 \cdot 25} = \frac{20}{137} \approx 0.146.
    \end{align*}


        График процесса не в масштабе (эта часть пока не готова и сделать автоматически аккуратно сложно), но с верными подписями (а для решения этого достаточно):

        \begin{tikzpicture}[thick]
            \draw[-{Latex}] (0, 0) -- (0, 7) node[above left] {$P$};
            \draw[-{Latex}] (0, 0) -- (10, 0) node[right] {$V$};

            \draw[dashed] (0, 2) node[left] {$P_1 = P_0$} -| (3, 0) node[below] {$V_1 = V_2 = V_0$};
            \draw[dashed] (0, 6) node[left] {$P_2 = P_3 = 5P_0$} -| (9, 0) node[below] {$V_3 = 6V_0$};

            \draw (3, 2) node[above left]{1} node[below left]{$T_1 = T_0$}
                   (3, 6) node[below left]{2} node[above]{$T_2 = 5T_0$}
                   (9, 6) node[above right]{3} node[below right]{$T_3 = 30T_0$};
            \draw[midar] (3, 2) -- (3, 6);
            \draw[midar] (3, 6) -- (9, 6);
            \draw[midar] (9, 6) -- (3, 2);
        \end{tikzpicture}

        Решение бонуса:
        \begin{align*}
            A_{12} &= 0, \Delta U_{12} > 0, \implies Q_{12} = A_{12} + \Delta U_{12} > 0, \\
            A_{23} &> 0, \Delta U_{23} \text{ — ничего нельзя сказать, нужно исследовать отдельно}, \\
            A_{31} &< 0, \Delta U_{31} < 0, \implies Q_{31} = A_{31} + \Delta U_{31} < 0.
            \\
        \end{align*}

        Уравнения состояния идеального газа для точек 1, 2, 3: $P_1V_1 = \nu R T_1, P_2V_2 = \nu R T_2, P_3V_3 = \nu R T_3$.
        Пусть $P_0$, $V_0$, $T_0$ — давление, объём и температура в точке 1 (минимальные во всём цикле).

        12 --- изохора, $\frac{P_1V_1}{T_1} = \nu R = \frac{P_2V_2}{T_2}, V_2=V_1=V_0 \implies \frac{P_1}{T_1} =  \frac{P_2}{T_2} \implies P_2 = P_1 \frac{T_2}{T_1} = 5P_0$,

        31 --- изобара, $\frac{P_1V_1}{T_1} = \nu R = \frac{P_3V_3}{T_3}, P_3=P_1=P_0 \implies \frac{V_3}{T_3} =  \frac{V_1}{T_1} \implies V_3 = V_1 \frac{T_3}{T_1} = 5V_0$,

        Таким образом, используя новые обозначения, состояния газа в точках 1, 2 и 3 описываются макропараметрами $(P_0, V_0, T_0), (5P_0, V_0, 5T_0), (P_0, 5V_0, 5T_0)$ соответственно.

        \begin{tikzpicture}[thick]
            \draw[-{Latex}] (0, 0) -- (0, 7) node[above left] {$P$};
            \draw[-{Latex}] (0, 0) -- (10, 0) node[right] {$V$};

            \draw[dashed] (0, 2) node[left] {$P_1 = P_3 = P_0$} -| (9, 0) node[below] {$V_3 = 5V_0$};
            \draw[dashed] (0, 6) node[left] {$P_2 = 5P_0$} -| (3, 0) node[below] {$V_1 = V_2 = V_0$};

            \draw[dashed] (0, 5) node[left] {$P$} -| (4.5, 0) node[below] {$V$};
            \draw[dashed] (0, 4.6) node[left] {$P'$} -| (5.1, 0) node[below] {$V'$};

            \draw (3, 2) node[above left]{1} node[below left]{$T_1 = T_0$}
                   (3, 6) node[below left]{2} node[above]{$T_2 = 5T_0$}
                   (9, 2) node[above right]{3} node[below right]{$T_3 = 5T_0$};
            \draw[midar] (3, 2) -- (3, 6);
            \draw[midar] (3, 6) -- (9, 2);
            \draw[midar] (9, 2) -- (3, 2);
            \draw   (4.5, 5) node[above right]{$T$} (5.1, 4.6) node[above right]{$T'$};
        \end{tikzpicture}


        Теперь рассмотрим отдельно процесс 23, к остальному вернёмся позже.
        Уравнение этой прямой в $PV$-координатах: $P(V) = 6P_0 - \frac{P_0}{V_0} V$.
        Это значит, что при изменении объёма на $\Delta V$ давление изменится на $\Delta P = - \frac{P_0}{V_0} \Delta V$, обратите внимание на знак.

        Рассмотрим произвольную точку в процессе 23 и дадим процессу ещё немного свершиться, при этом объём изменится на $\Delta V$, давление — на $\Delta P$, температура (иначе бы была гипербола, а не прямая) — на $\Delta T$,
        т.е.
        из состояния $(P, V, T)$ мы перешли в $(P', V', T')$, причём  $P' = P + \Delta P, V' = V + \Delta V, T' = T + \Delta T$.

        При этом изменится внутренняя энергия:
        \begin{align*}
        \Delta U
            &= U' - U = \frac 32 \nu R T' - \frac 32 \nu R T = \frac 32 (P+\Delta P) (V+\Delta V) - \frac 32 PV\\
            &= \frac 32 ((P+\Delta P) (V+\Delta V) - PV) = \frac 32 (P\Delta V + V \Delta P + \Delta P \Delta V).
        \end{align*}

        Рассмотрим малые изменения объёма, тогда и изменение давления будем малым (т.к.
        $\Delta P = - \frac{P_0}{V_0} \Delta V$),
        а третьим слагаемым в выражении для $\Delta U$  можно пренебречь по сравнению с двумя другими:
        два первых это малые величины, а третье — произведение двух малых.
        Тогда $\Delta U = \frac 32 (P\Delta V + V \Delta P)$.

        Работа газа при этом малом изменении объёма — это площадь трапеции (тут ещё раз пренебрегли малым слагаемым):
        $$A = \frac{P + P'}2 \Delta V = \cbr{P + \frac{\Delta P}2} \Delta V = P \Delta V.$$

        Подведённое количество теплоты, используя первое начало термодинамики, будет равно
        \begin{align*}
        Q
            &= \frac 32 (P\Delta V + V \Delta P) + P \Delta V =  \frac 52 P\Delta V + \frac 32 V \Delta P = \\
            &= \frac 52 P\Delta V + \frac 32 V \cdot \cbr{- \frac{P_0}{V_0} \Delta V} = \frac{\Delta V}2 \cdot \cbr{5P - \frac{P_0}{V_0} V} = \\
            &= \frac{\Delta V}2 \cdot \cbr{5 \cdot \cbr{6P_0 - \frac{P_0}{V_0} V} - \frac{P_0}{V_0} V}
             = \frac{\Delta V \cdot P_0}2 \cdot \cbr{5 \cdot 6 - 8\frac V{V_0}}.
        \end{align*}

        Таком образом, знак количества теплоты $Q$ на участке 23 зависит от конкретного значения $V$:
        \begin{itemize}
            \item $\Delta V > 0$ на всём участке 23, поскольку газ расширяется,
            \item $P > 0$ — всегда, у нас идеальный газ, удары о стенки сосуда абсолютно упругие, а молекулы не взаимодействуют и поэтому давление только положительно,
            \item если $5 \cdot 6 - 8\frac V{V_0} > 0$ — тепло подводят, если же меньше нуля — отводят.
        \end{itemize}
        Решая последнее неравенство, получаем конкретное значение $V^*$: при $V < V^*$ тепло подводят, далее~— отводят.
        Тут *~--- некоторая точка между точками 2 и 3, конкретные значения надо досчитать:
        $$V^* = V_0 \cdot \frac{5 \cdot 6}8 = \frac{15}4 \cdot V_0 \implies P^* = 6P_0 - \frac{P_0}{V_0} V^* = \ldots = \frac94 \cdot P_0.$$

        Т.е.
        чтобы вычислить $Q_+$, надо сложить количества теплоты на участке 12 и лишь части участка 23 — участке 2*,
        той его части где это количество теплоты положительно.
        Имеем: $Q_+ = Q_{12} + Q_{2*}$.

        Теперь возвращаемся к циклу целиком и получаем:
        \begin{align*}
        A_\text{цикл} &= \frac 12 \cdot (5P_0 - P_0) \cdot (5V_0 - V_0) = 8 \cdot P_0V_0, \\
        A_{2*} &= \frac{P^* + 5P_0}2 \cdot (V^* - V_0)
            = \frac{\frac94 \cdot P_0 + 5P_0}2 \cdot \cbr{\frac{15}4 \cdot V_0 - V_0}
            = \ldots = \frac{319}{32} \cdot P_0 V_0, \\
        \Delta U_{2*} &= \frac 32 \nu R T^* - \frac 32 \nu R T_2 = \frac 32 (P^*V^* - P_0 \cdot 5V_0)
            = \frac 32 \cbr{\frac94 \cdot P_0 \cdot \frac{15}4 \cdot V_0 - P_0 \cdot 5V_0}
            = \frac{165}{32} \cdot P_0 V_0, \\
        \Delta U_{12} &= \frac 32 \nu R T_2 - \frac 32 \nu R T_1 = \frac 32 (5P_0V_0 - P_0V_0) = \ldots = 6 \cdot P_0 V_0, \\
        \eta &= \frac{A_\text{цикл}}{Q_+} = \frac{A_\text{цикл}}{Q_{12} + Q_{2*}}
            = \frac{A_\text{цикл}}{A_{12} + \Delta U_{12} + A_{2*} + \Delta U_{2*}} = \\
            &= \frac{8 \cdot P_0V_0}{0 + 6 \cdot P_0 V_0 + \frac{319}{32} \cdot P_0 V_0 + \frac{165}{32} \cdot P_0 V_0}
             = \frac{A_bonus_cycle:LaTeX}{6 + \frac{319}{32} + \frac{165}{32}}
             = \frac{64}{169} \leftarrow \text{вжух и готово!}
        \end{align*}
}
\solutionspace{360pt}

\tasknumber{2}%
\task{%
    При температуре $30\celsius$ относительная влажность воздуха составляет $65\%$.
    \begin{itemize}
        \item Определите точку росы для этого воздуха.
        \item Какой станет относительная влажность этого воздуха, если нагреть его до $50\celsius$?
    \end{itemize}
}
\answer{%
    \begin{align*}
    &\text{Значения плотности насыщенного водяного пара определяем по таблице:} \\
    &\rho_{\text{нас.
    пара 30} \celsius} = 30{,}300\,\frac{\text{г}}{\text{м}^{3}}, \rho_{\text{нас.
    пара 50} \celsius} = 83{,}000\,\frac{\text{г}}{\text{м}^{3}}.
    \\
    \varphi_1 &= \frac{\rho_\text{пара}}{\rho_{\text{нас.
    пара 30} \celsius}} \implies {\rho_\text{пара}} = \rho_{\text{нас.
    пара 30} \celsius} \cdot \varphi_1 = 30{,}300\,\frac{\text{г}}{\text{м}^{3}} \cdot 0{,}65 = 19{,}695\,\frac{\text{г}}{\text{м}^{3}}.
    \\
    &\text{По таблице определяем, при какой температуре пар с такой плотностью станет насыщенным:}  \\
    t_\text{росы} &= 22{,}2\celsius, \\
    \varphi_2 &= \frac{\rho_\text{пара}}{\rho_{\text{нас.
    пара 50} \celsius}} = \frac{\rho_{\text{нас.
    пара 30} \celsius} \cdot \varphi_1}{\rho_{\text{нас.
    пара 50} \celsius}}= \varphi_1 \cdot \frac{\rho_{\text{нас.
    пара 30} \celsius}}{\rho_{\text{нас.
    пара 50} \celsius}} = 0{,}65 \cdot \frac{30{,}300\,\frac{\text{г}}{\text{м}^{3}}}{83{,}000\,\frac{\text{г}}{\text{м}^{3}}} = 0{,}237 \approx 23{,}7\%.
    \end{align*}
}
\solutionspace{80pt}

\tasknumber{3}%
\task{%
    Из уравнения состояния идеального газа выведите или выразите...
    \begin{enumerate}
        \item объём,
        \item молярную массу,
        \item концентрацию молекул газа.
    \end{enumerate}
}

\tasknumber{4}%
\task{%
    Запишите формулы и рядом с каждой физичической величиной укажите её название и единицы измерения в СИ:
    \begin{enumerate}
        \item первое начало термодинамики,
        \item внутренняя энергия идеального одноатомного газа.
    \end{enumerate}
}

\variantsplitter

\addpersonalvariant{Арсений Трофимов}

\tasknumber{1}%
\task{%
    Определите КПД (оставив ответ точным в виде нескоратимой дроби) цикла 1231, рабочим телом которого является идеальный одноатомный газ, если
    \begin{itemize}
        \item 12 — изохорический нагрев в пять раз,
        \item 23 — изобарическое расширение, при котором температура растёт в два раза,
        \item 31 — процесс, график которого в $PV$-координатах является отрезком прямой.
    \end{itemize}
    Бонус: замените цикл 1231 циклом, в котором 12 — изохорический нагрев в пять раз, 23 — процесс, график которого в $PV$-координатах является отрезком прямой, 31 — изобарическое охлаждение, при котором температура падает в пять раз.
}
\answer{%
    \begin{align*}
    A_{12} &= 0, \Delta U_{12} > 0, \implies Q_{12} = A_{12} + \Delta U_{12} > 0.
    \\
    A_{23} &> 0, \Delta U_{23} > 0, \implies Q_{23} = A_{23} + \Delta U_{23} > 0, \\
    A_{31} &= 0, \Delta U_{31} < 0, \implies Q_{31} = A_{31} + \Delta U_{31} < 0.
    \\
    P_1V_1 &= \nu R T_1, P_2V_2 = \nu R T_2, P_3V_3 = \nu R T_3 \text{ — уравнения состояния идеального газа}, \\
    &\text{Пусть $P_0$, $V_0$, $T_0$ — давление, объём и температура в точке 1 (минимальные во всём цикле):} \\
    P_1 &= P_0, P_2 = P_3, V_1 = V_2 = V_0, \text{остальные соотношения нужно считать} \\
    T_2 &= 5T_1 = 5T_0 \text{(по условию)} \implies \frac{P_2}{P_1} = \frac{P_2V_0}{P_1V_0} = \frac{P_2 V_2}{P_1 V_1}= \frac{\nu R T_2}{\nu R T_1} = \frac{T_2}{T_1} = 5 \implies P_2 = 5 P_1 = 5 P_0, \\
    T_3 &= 2T_2 = 10T_0 \text{(по условию)} \implies \frac{V_3}{V_2} = \frac{P_3V_3}{P_2V_2}= \frac{\nu R T_3}{\nu R T_2} = \frac{T_3}{T_2} = 2 \implies V_3 = 2 V_2 = 2 V_0.
    \\
    A_\text{цикл} &= \frac 12 (2P_0 - P_0)(5V_0 - V_0) = \frac 12 \cdot 4 \cdot P_0V_0, \\
    A_{23} &= 5P_0 \cdot (2V_0 - V_0) = 5P_0V_0, \\
    \Delta U_{23} &= \frac 32 \nu R T_3 - \frac 32 \nu R T_2 = \frac 32 P_3 V_3 - \frac 32 P_2 V_2 = \frac 32 \cdot 5 P_0 \cdot 2 V_0 -  \frac 32 \cdot 5 P_0 \cdot V_0 = \frac 32 \cdot 5 \cdot P_0V_0, \\
    \Delta U_{12} &= \frac 32 \nu R T_2 - \frac 32 \nu R T_1 = \frac 32 P_2 V_2 - \frac 32 P_1 V_1 = \frac 32 \cdot 5 P_0V_0 - \frac 32 P_0V_0 = \frac 32 \cdot 4 \cdot P_0V_0.
    \\
    \eta &= \frac{A_\text{цикл}}{Q_+} = \frac{A_\text{цикл}}{Q_{12} + Q_{23}}  = \frac{A_\text{цикл}}{A_{12} + \Delta U_{12} + A_{23} + \Delta U_{23}} =  \\
     &= \frac{\frac 12 \cdot 4 \cdot P_0V_0}{0 + \frac 32 \cdot 4 \cdot P_0V_0 + 5P_0V_0 + \frac 32 \cdot 5 \cdot P_0V_0} = \frac{\frac 12 \cdot 4}{\frac 32 \cdot 4 + 5 + \frac 32 \cdot 5} = \frac4{37} \approx 0.108.
    \end{align*}


        График процесса не в масштабе (эта часть пока не готова и сделать автоматически аккуратно сложно), но с верными подписями (а для решения этого достаточно):

        \begin{tikzpicture}[thick]
            \draw[-{Latex}] (0, 0) -- (0, 7) node[above left] {$P$};
            \draw[-{Latex}] (0, 0) -- (10, 0) node[right] {$V$};

            \draw[dashed] (0, 2) node[left] {$P_1 = P_0$} -| (3, 0) node[below] {$V_1 = V_2 = V_0$};
            \draw[dashed] (0, 6) node[left] {$P_2 = P_3 = 5P_0$} -| (9, 0) node[below] {$V_3 = 2V_0$};

            \draw (3, 2) node[above left]{1} node[below left]{$T_1 = T_0$}
                   (3, 6) node[below left]{2} node[above]{$T_2 = 5T_0$}
                   (9, 6) node[above right]{3} node[below right]{$T_3 = 10T_0$};
            \draw[midar] (3, 2) -- (3, 6);
            \draw[midar] (3, 6) -- (9, 6);
            \draw[midar] (9, 6) -- (3, 2);
        \end{tikzpicture}

        Решение бонуса:
        \begin{align*}
            A_{12} &= 0, \Delta U_{12} > 0, \implies Q_{12} = A_{12} + \Delta U_{12} > 0, \\
            A_{23} &> 0, \Delta U_{23} \text{ — ничего нельзя сказать, нужно исследовать отдельно}, \\
            A_{31} &< 0, \Delta U_{31} < 0, \implies Q_{31} = A_{31} + \Delta U_{31} < 0.
            \\
        \end{align*}

        Уравнения состояния идеального газа для точек 1, 2, 3: $P_1V_1 = \nu R T_1, P_2V_2 = \nu R T_2, P_3V_3 = \nu R T_3$.
        Пусть $P_0$, $V_0$, $T_0$ — давление, объём и температура в точке 1 (минимальные во всём цикле).

        12 --- изохора, $\frac{P_1V_1}{T_1} = \nu R = \frac{P_2V_2}{T_2}, V_2=V_1=V_0 \implies \frac{P_1}{T_1} =  \frac{P_2}{T_2} \implies P_2 = P_1 \frac{T_2}{T_1} = 5P_0$,

        31 --- изобара, $\frac{P_1V_1}{T_1} = \nu R = \frac{P_3V_3}{T_3}, P_3=P_1=P_0 \implies \frac{V_3}{T_3} =  \frac{V_1}{T_1} \implies V_3 = V_1 \frac{T_3}{T_1} = 5V_0$,

        Таким образом, используя новые обозначения, состояния газа в точках 1, 2 и 3 описываются макропараметрами $(P_0, V_0, T_0), (5P_0, V_0, 5T_0), (P_0, 5V_0, 5T_0)$ соответственно.

        \begin{tikzpicture}[thick]
            \draw[-{Latex}] (0, 0) -- (0, 7) node[above left] {$P$};
            \draw[-{Latex}] (0, 0) -- (10, 0) node[right] {$V$};

            \draw[dashed] (0, 2) node[left] {$P_1 = P_3 = P_0$} -| (9, 0) node[below] {$V_3 = 5V_0$};
            \draw[dashed] (0, 6) node[left] {$P_2 = 5P_0$} -| (3, 0) node[below] {$V_1 = V_2 = V_0$};

            \draw[dashed] (0, 5) node[left] {$P$} -| (4.5, 0) node[below] {$V$};
            \draw[dashed] (0, 4.6) node[left] {$P'$} -| (5.1, 0) node[below] {$V'$};

            \draw (3, 2) node[above left]{1} node[below left]{$T_1 = T_0$}
                   (3, 6) node[below left]{2} node[above]{$T_2 = 5T_0$}
                   (9, 2) node[above right]{3} node[below right]{$T_3 = 5T_0$};
            \draw[midar] (3, 2) -- (3, 6);
            \draw[midar] (3, 6) -- (9, 2);
            \draw[midar] (9, 2) -- (3, 2);
            \draw   (4.5, 5) node[above right]{$T$} (5.1, 4.6) node[above right]{$T'$};
        \end{tikzpicture}


        Теперь рассмотрим отдельно процесс 23, к остальному вернёмся позже.
        Уравнение этой прямой в $PV$-координатах: $P(V) = 6P_0 - \frac{P_0}{V_0} V$.
        Это значит, что при изменении объёма на $\Delta V$ давление изменится на $\Delta P = - \frac{P_0}{V_0} \Delta V$, обратите внимание на знак.

        Рассмотрим произвольную точку в процессе 23 и дадим процессу ещё немного свершиться, при этом объём изменится на $\Delta V$, давление — на $\Delta P$, температура (иначе бы была гипербола, а не прямая) — на $\Delta T$,
        т.е.
        из состояния $(P, V, T)$ мы перешли в $(P', V', T')$, причём  $P' = P + \Delta P, V' = V + \Delta V, T' = T + \Delta T$.

        При этом изменится внутренняя энергия:
        \begin{align*}
        \Delta U
            &= U' - U = \frac 32 \nu R T' - \frac 32 \nu R T = \frac 32 (P+\Delta P) (V+\Delta V) - \frac 32 PV\\
            &= \frac 32 ((P+\Delta P) (V+\Delta V) - PV) = \frac 32 (P\Delta V + V \Delta P + \Delta P \Delta V).
        \end{align*}

        Рассмотрим малые изменения объёма, тогда и изменение давления будем малым (т.к.
        $\Delta P = - \frac{P_0}{V_0} \Delta V$),
        а третьим слагаемым в выражении для $\Delta U$  можно пренебречь по сравнению с двумя другими:
        два первых это малые величины, а третье — произведение двух малых.
        Тогда $\Delta U = \frac 32 (P\Delta V + V \Delta P)$.

        Работа газа при этом малом изменении объёма — это площадь трапеции (тут ещё раз пренебрегли малым слагаемым):
        $$A = \frac{P + P'}2 \Delta V = \cbr{P + \frac{\Delta P}2} \Delta V = P \Delta V.$$

        Подведённое количество теплоты, используя первое начало термодинамики, будет равно
        \begin{align*}
        Q
            &= \frac 32 (P\Delta V + V \Delta P) + P \Delta V =  \frac 52 P\Delta V + \frac 32 V \Delta P = \\
            &= \frac 52 P\Delta V + \frac 32 V \cdot \cbr{- \frac{P_0}{V_0} \Delta V} = \frac{\Delta V}2 \cdot \cbr{5P - \frac{P_0}{V_0} V} = \\
            &= \frac{\Delta V}2 \cdot \cbr{5 \cdot \cbr{6P_0 - \frac{P_0}{V_0} V} - \frac{P_0}{V_0} V}
             = \frac{\Delta V \cdot P_0}2 \cdot \cbr{5 \cdot 6 - 8\frac V{V_0}}.
        \end{align*}

        Таком образом, знак количества теплоты $Q$ на участке 23 зависит от конкретного значения $V$:
        \begin{itemize}
            \item $\Delta V > 0$ на всём участке 23, поскольку газ расширяется,
            \item $P > 0$ — всегда, у нас идеальный газ, удары о стенки сосуда абсолютно упругие, а молекулы не взаимодействуют и поэтому давление только положительно,
            \item если $5 \cdot 6 - 8\frac V{V_0} > 0$ — тепло подводят, если же меньше нуля — отводят.
        \end{itemize}
        Решая последнее неравенство, получаем конкретное значение $V^*$: при $V < V^*$ тепло подводят, далее~— отводят.
        Тут *~--- некоторая точка между точками 2 и 3, конкретные значения надо досчитать:
        $$V^* = V_0 \cdot \frac{5 \cdot 6}8 = \frac{15}4 \cdot V_0 \implies P^* = 6P_0 - \frac{P_0}{V_0} V^* = \ldots = \frac94 \cdot P_0.$$

        Т.е.
        чтобы вычислить $Q_+$, надо сложить количества теплоты на участке 12 и лишь части участка 23 — участке 2*,
        той его части где это количество теплоты положительно.
        Имеем: $Q_+ = Q_{12} + Q_{2*}$.

        Теперь возвращаемся к циклу целиком и получаем:
        \begin{align*}
        A_\text{цикл} &= \frac 12 \cdot (5P_0 - P_0) \cdot (5V_0 - V_0) = 8 \cdot P_0V_0, \\
        A_{2*} &= \frac{P^* + 5P_0}2 \cdot (V^* - V_0)
            = \frac{\frac94 \cdot P_0 + 5P_0}2 \cdot \cbr{\frac{15}4 \cdot V_0 - V_0}
            = \ldots = \frac{319}{32} \cdot P_0 V_0, \\
        \Delta U_{2*} &= \frac 32 \nu R T^* - \frac 32 \nu R T_2 = \frac 32 (P^*V^* - P_0 \cdot 5V_0)
            = \frac 32 \cbr{\frac94 \cdot P_0 \cdot \frac{15}4 \cdot V_0 - P_0 \cdot 5V_0}
            = \frac{165}{32} \cdot P_0 V_0, \\
        \Delta U_{12} &= \frac 32 \nu R T_2 - \frac 32 \nu R T_1 = \frac 32 (5P_0V_0 - P_0V_0) = \ldots = 6 \cdot P_0 V_0, \\
        \eta &= \frac{A_\text{цикл}}{Q_+} = \frac{A_\text{цикл}}{Q_{12} + Q_{2*}}
            = \frac{A_\text{цикл}}{A_{12} + \Delta U_{12} + A_{2*} + \Delta U_{2*}} = \\
            &= \frac{8 \cdot P_0V_0}{0 + 6 \cdot P_0 V_0 + \frac{319}{32} \cdot P_0 V_0 + \frac{165}{32} \cdot P_0 V_0}
             = \frac{A_bonus_cycle:LaTeX}{6 + \frac{319}{32} + \frac{165}{32}}
             = \frac{64}{169} \leftarrow \text{вжух и готово!}
        \end{align*}
}
\solutionspace{360pt}

\tasknumber{2}%
\task{%
    При температуре $25\celsius$ относительная влажность воздуха составляет $70\%$.
    \begin{itemize}
        \item Определите точку росы для этого воздуха.
        \item Какой станет относительная влажность этого воздуха, если нагреть его до $70\celsius$?
    \end{itemize}
}
\answer{%
    \begin{align*}
    &\text{Значения плотности насыщенного водяного пара определяем по таблице:} \\
    &\rho_{\text{нас.
    пара 25} \celsius} = 23{,}000\,\frac{\text{г}}{\text{м}^{3}}, \rho_{\text{нас.
    пара 70} \celsius} = 198{,}000\,\frac{\text{г}}{\text{м}^{3}}.
    \\
    \varphi_1 &= \frac{\rho_\text{пара}}{\rho_{\text{нас.
    пара 25} \celsius}} \implies {\rho_\text{пара}} = \rho_{\text{нас.
    пара 25} \celsius} \cdot \varphi_1 = 23{,}000\,\frac{\text{г}}{\text{м}^{3}} \cdot 0{,}70 = 16{,}100\,\frac{\text{г}}{\text{м}^{3}}.
    \\
    &\text{По таблице определяем, при какой температуре пар с такой плотностью станет насыщенным:}  \\
    t_\text{росы} &= 18{,}8\celsius, \\
    \varphi_2 &= \frac{\rho_\text{пара}}{\rho_{\text{нас.
    пара 70} \celsius}} = \frac{\rho_{\text{нас.
    пара 25} \celsius} \cdot \varphi_1}{\rho_{\text{нас.
    пара 70} \celsius}}= \varphi_1 \cdot \frac{\rho_{\text{нас.
    пара 25} \celsius}}{\rho_{\text{нас.
    пара 70} \celsius}} = 0{,}70 \cdot \frac{23{,}000\,\frac{\text{г}}{\text{м}^{3}}}{198{,}000\,\frac{\text{г}}{\text{м}^{3}}} = 0{,}081 \approx 8{,}1\%.
    \end{align*}
}
\solutionspace{80pt}

\tasknumber{3}%
\task{%
    Из уравнения состояния идеального газа выведите или выразите...
    \begin{enumerate}
        \item объём,
        \item температуру,
        \item концентрацию молекул газа.
    \end{enumerate}
}

\tasknumber{4}%
\task{%
    Запишите формулы и рядом с каждой физичической величиной укажите её название и единицы измерения в СИ:
    \begin{enumerate}
        \item первое начало термодинамики,
        \item внутренняя энергия идеального одноатомного газа.
    \end{enumerate}
}
% autogenerated
