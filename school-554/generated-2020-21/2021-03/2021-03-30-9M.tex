\setdate{30~марта~2021}
\setclass{9«М»}

\addpersonalvariant{Михаил Бурмистров}

\tasknumber{1}%
\task{%
    Длина волны света в~вакууме $\lambda = 700\,\text{нм}$.
    Какова частота этой световой волны?
    Какова длина этой волны в среде с показателем преломления $n = 1{,}5$?
    Может ли человек увидеть такую волну света, и если да, то какой именно цвет соответствует этим волнам в вакууме и в этой среде?
}
\answer{%
    \begin{align*}
    \nu &= \frac 1T = \frac 1{\lambda/c} = \frac c\lambda = \frac{3 \cdot 10^{8}\,\frac{\text{м}}{\text{с}}}{700\,\text{нм}} \approx 4{,}29 \cdot 10^{14}\,\text{Гц}, \\
    \nu' = \nu &\cbr{\text{или } T' = T} \implies \lambda' = v'T' = \frac vn T = \frac{ vt }n = \frac \lambda n = \frac{700\,\text{нм}}{1{,}5} \approx 467 \cdot 10^{-9}\,\text{м}.
    \\
    &\text{380 нм---фиол---440---син---485---гол---500---зел---565---жёл---590---оранж---625---крас---780 нм}
    \end{align*}
}
\solutionspace{180pt}

\tasknumber{2}%
\task{%
    Напротив физических величин укажите их обозначения и единицы измерения в СИ, а в пункте «г)» запишите физический закон или формулу:
    \begin{enumerate}
        \item корость света в вакууме,
        \item длина волны,
        \item период колебаний напряжённости электрического поля в электромагнитной волне,
        \item относительный показатель преломления среды.
    \end{enumerate}
}
\solutionspace{20pt}

\tasknumber{3}%
\task{%
    Выразите (нужен вывод из базовых физических законов):
    \begin{enumerate}
        \item период колебаний через длину волны и скорость её распространения,
        \item энергию фотона через длину электромагнитной волны,
        \item скорость света в вакууме через скорость света в среде и её абсолютный показатель преломления.
    \end{enumerate}
}
\solutionspace{60pt}

\tasknumber{4}%
\task{%
    Укажите букву, соответствующую физическую величину (из текущего раздела), её едииницы измерения в СИ и выразите её из какого-либо уравнения:
    \begin{enumerate}
        \item «л'амбда»,
        \item «вэ»,
        \item «н'у»,
        \item «тэ».
    \end{enumerate}
}
\solutionspace{100pt}

\tasknumber{5}%
\task{%
    Напротив каждой приставки единиц СИ укажите её полное название и соответствующий множитель:
    \begin{enumerate}
        \item М,
        \item к,
        \item н,
        \item Т.
    \end{enumerate}
}

\variantsplitter

\addpersonalvariant{Артём Глембо}

\tasknumber{1}%
\task{%
    Длина волны света в~вакууме $\lambda = 400\,\text{нм}$.
    Какова частота этой световой волны?
    Какова длина этой волны в среде с показателем преломления $n = 1{,}6$?
    Может ли человек увидеть такую волну света, и если да, то какой именно цвет соответствует этим волнам в вакууме и в этой среде?
}
\answer{%
    \begin{align*}
    \nu &= \frac 1T = \frac 1{\lambda/c} = \frac c\lambda = \frac{3 \cdot 10^{8}\,\frac{\text{м}}{\text{с}}}{400\,\text{нм}} \approx 7{,}50 \cdot 10^{14}\,\text{Гц}, \\
    \nu' = \nu &\cbr{\text{или } T' = T} \implies \lambda' = v'T' = \frac vn T = \frac{ vt }n = \frac \lambda n = \frac{400\,\text{нм}}{1{,}6} \approx 250 \cdot 10^{-9}\,\text{м}.
    \\
    &\text{380 нм---фиол---440---син---485---гол---500---зел---565---жёл---590---оранж---625---крас---780 нм}
    \end{align*}
}
\solutionspace{180pt}

\tasknumber{2}%
\task{%
    Напротив физических величин укажите их обозначения и единицы измерения в СИ, а в пункте «г)» запишите физический закон или формулу:
    \begin{enumerate}
        \item корость света в вакууме,
        \item частоты волны,
        \item период колебаний индукции магнитного поля в электромагнитной волне,
        \item абсолютный показатель преломления среды.
    \end{enumerate}
}
\solutionspace{20pt}

\tasknumber{3}%
\task{%
    Выразите (нужен вывод из базовых физических законов):
    \begin{enumerate}
        \item период колебаний через длину волны и скорость её распространения,
        \item энергию фотона через длину электромагнитной волны,
        \item скорость света в вакууме через скорость света в среде и её абсолютный показатель преломления.
    \end{enumerate}
}
\solutionspace{60pt}

\tasknumber{4}%
\task{%
    Укажите букву, соответствующую физическую величину (из текущего раздела), её едииницы измерения в СИ и выразите её из какого-либо уравнения:
    \begin{enumerate}
        \item «л'амбда»,
        \item «цэ»,
        \item «аш»,
        \item «эн».
    \end{enumerate}
}
\solutionspace{100pt}

\tasknumber{5}%
\task{%
    Напротив каждой приставки единиц СИ укажите её полное название и соответствующий множитель:
    \begin{enumerate}
        \item М,
        \item мк,
        \item Г,
        \item с.
    \end{enumerate}
}

\variantsplitter

\addpersonalvariant{Наталья Гончарова}

\tasknumber{1}%
\task{%
    Длина волны света в~вакууме $\lambda = 500\,\text{нм}$.
    Какова частота этой световой волны?
    Какова длина этой волны в среде с показателем преломления $n = 1{,}3$?
    Может ли человек увидеть такую волну света, и если да, то какой именно цвет соответствует этим волнам в вакууме и в этой среде?
}
\answer{%
    \begin{align*}
    \nu &= \frac 1T = \frac 1{\lambda/c} = \frac c\lambda = \frac{3 \cdot 10^{8}\,\frac{\text{м}}{\text{с}}}{500\,\text{нм}} \approx 6 \cdot 10^{14}\,\text{Гц}, \\
    \nu' = \nu &\cbr{\text{или } T' = T} \implies \lambda' = v'T' = \frac vn T = \frac{ vt }n = \frac \lambda n = \frac{500\,\text{нм}}{1{,}3} \approx 385 \cdot 10^{-9}\,\text{м}.
    \\
    &\text{380 нм---фиол---440---син---485---гол---500---зел---565---жёл---590---оранж---625---крас---780 нм}
    \end{align*}
}
\solutionspace{180pt}

\tasknumber{2}%
\task{%
    Напротив физических величин укажите их обозначения и единицы измерения в СИ, а в пункте «г)» запишите физический закон или формулу:
    \begin{enumerate}
        \item скорость света в среде,
        \item длина волны,
        \item период колебаний индукции магнитного поля в электромагнитной волне,
        \item абсолютный показатель преломления среды.
    \end{enumerate}
}
\solutionspace{20pt}

\tasknumber{3}%
\task{%
    Выразите (нужен вывод из базовых физических законов):
    \begin{enumerate}
        \item период колебаний через длину волны и скорость её распространения,
        \item энергию фотона через период колебаний в электромагнитной волне,
        \item скорость света в вакууме через скорость света в среде и её абсолютный показатель преломления.
    \end{enumerate}
}
\solutionspace{60pt}

\tasknumber{4}%
\task{%
    Укажите букву, соответствующую физическую величину (из текущего раздела), её едииницы измерения в СИ и выразите её из какого-либо уравнения:
    \begin{enumerate}
        \item «йэ»,
        \item «вэ»,
        \item «аш»,
        \item «эн».
    \end{enumerate}
}
\solutionspace{100pt}

\tasknumber{5}%
\task{%
    Напротив каждой приставки единиц СИ укажите её полное название и соответствующий множитель:
    \begin{enumerate}
        \item м,
        \item к,
        \item Г,
        \item с.
    \end{enumerate}
}

\variantsplitter

\addpersonalvariant{Файёзбек Касымов}

\tasknumber{1}%
\task{%
    Длина волны света в~вакууме $\lambda = 400\,\text{нм}$.
    Какова частота этой световой волны?
    Какова длина этой волны в среде с показателем преломления $n = 1{,}5$?
    Может ли человек увидеть такую волну света, и если да, то какой именно цвет соответствует этим волнам в вакууме и в этой среде?
}
\answer{%
    \begin{align*}
    \nu &= \frac 1T = \frac 1{\lambda/c} = \frac c\lambda = \frac{3 \cdot 10^{8}\,\frac{\text{м}}{\text{с}}}{400\,\text{нм}} \approx 7{,}50 \cdot 10^{14}\,\text{Гц}, \\
    \nu' = \nu &\cbr{\text{или } T' = T} \implies \lambda' = v'T' = \frac vn T = \frac{ vt }n = \frac \lambda n = \frac{400\,\text{нм}}{1{,}5} \approx 267 \cdot 10^{-9}\,\text{м}.
    \\
    &\text{380 нм---фиол---440---син---485---гол---500---зел---565---жёл---590---оранж---625---крас---780 нм}
    \end{align*}
}
\solutionspace{180pt}

\tasknumber{2}%
\task{%
    Напротив физических величин укажите их обозначения и единицы измерения в СИ, а в пункте «г)» запишите физический закон или формулу:
    \begin{enumerate}
        \item скорость света в среде,
        \item длина волны,
        \item период колебаний напряжённости электрического поля в электромагнитной волне,
        \item относительный показатель преломления среды.
    \end{enumerate}
}
\solutionspace{20pt}

\tasknumber{3}%
\task{%
    Выразите (нужен вывод из базовых физических законов):
    \begin{enumerate}
        \item частоту колебаний через длину волны и скорость её распространения,
        \item энергию фотона через длину электромагнитной волны,
        \item скорость света в среде через её абсолютный показатель преломления и скорость света в вакууме.
    \end{enumerate}
}
\solutionspace{60pt}

\tasknumber{4}%
\task{%
    Укажите букву, соответствующую физическую величину (из текущего раздела), её едииницы измерения в СИ и выразите её из какого-либо уравнения:
    \begin{enumerate}
        \item «йэ»,
        \item «вэ»,
        \item «н'у»,
        \item «тэ».
    \end{enumerate}
}
\solutionspace{100pt}

\tasknumber{5}%
\task{%
    Напротив каждой приставки единиц СИ укажите её полное название и соответствующий множитель:
    \begin{enumerate}
        \item м,
        \item к,
        \item н,
        \item Т.
    \end{enumerate}
}

\variantsplitter

\addpersonalvariant{Александр Козинец}

\tasknumber{1}%
\task{%
    Длина волны света в~вакууме $\lambda = 700\,\text{нм}$.
    Какова частота этой световой волны?
    Какова длина этой волны в среде с показателем преломления $n = 1{,}7$?
    Может ли человек увидеть такую волну света, и если да, то какой именно цвет соответствует этим волнам в вакууме и в этой среде?
}
\answer{%
    \begin{align*}
    \nu &= \frac 1T = \frac 1{\lambda/c} = \frac c\lambda = \frac{3 \cdot 10^{8}\,\frac{\text{м}}{\text{с}}}{700\,\text{нм}} \approx 4{,}29 \cdot 10^{14}\,\text{Гц}, \\
    \nu' = \nu &\cbr{\text{или } T' = T} \implies \lambda' = v'T' = \frac vn T = \frac{ vt }n = \frac \lambda n = \frac{700\,\text{нм}}{1{,}7} \approx 412 \cdot 10^{-9}\,\text{м}.
    \\
    &\text{380 нм---фиол---440---син---485---гол---500---зел---565---жёл---590---оранж---625---крас---780 нм}
    \end{align*}
}
\solutionspace{180pt}

\tasknumber{2}%
\task{%
    Напротив физических величин укажите их обозначения и единицы измерения в СИ, а в пункте «г)» запишите физический закон или формулу:
    \begin{enumerate}
        \item корость света в вакууме,
        \item длина волны,
        \item период колебаний напряжённости электрического поля в электромагнитной волне,
        \item относительный показатель преломления среды.
    \end{enumerate}
}
\solutionspace{20pt}

\tasknumber{3}%
\task{%
    Выразите (нужен вывод из базовых физических законов):
    \begin{enumerate}
        \item период колебаний через длину волны и скорость её распространения,
        \item энергию фотона через период колебаний в электромагнитной волне,
        \item скорость света в среде через её абсолютный показатель преломления и скорость света в вакууме.
    \end{enumerate}
}
\solutionspace{60pt}

\tasknumber{4}%
\task{%
    Укажите букву, соответствующую физическую величину (из текущего раздела), её едииницы измерения в СИ и выразите её из какого-либо уравнения:
    \begin{enumerate}
        \item «л'амбда»,
        \item «вэ»,
        \item «н'у»,
        \item «тэ».
    \end{enumerate}
}
\solutionspace{100pt}

\tasknumber{5}%
\task{%
    Напротив каждой приставки единиц СИ укажите её полное название и соответствующий множитель:
    \begin{enumerate}
        \item М,
        \item к,
        \item н,
        \item Т.
    \end{enumerate}
}

\variantsplitter

\addpersonalvariant{Андрей Куликовский}

\tasknumber{1}%
\task{%
    Длина волны света в~вакууме $\lambda = 600\,\text{нм}$.
    Какова частота этой световой волны?
    Какова длина этой волны в среде с показателем преломления $n = 1{,}5$?
    Может ли человек увидеть такую волну света, и если да, то какой именно цвет соответствует этим волнам в вакууме и в этой среде?
}
\answer{%
    \begin{align*}
    \nu &= \frac 1T = \frac 1{\lambda/c} = \frac c\lambda = \frac{3 \cdot 10^{8}\,\frac{\text{м}}{\text{с}}}{600\,\text{нм}} \approx 5 \cdot 10^{14}\,\text{Гц}, \\
    \nu' = \nu &\cbr{\text{или } T' = T} \implies \lambda' = v'T' = \frac vn T = \frac{ vt }n = \frac \lambda n = \frac{600\,\text{нм}}{1{,}5} \approx 400 \cdot 10^{-9}\,\text{м}.
    \\
    &\text{380 нм---фиол---440---син---485---гол---500---зел---565---жёл---590---оранж---625---крас---780 нм}
    \end{align*}
}
\solutionspace{180pt}

\tasknumber{2}%
\task{%
    Напротив физических величин укажите их обозначения и единицы измерения в СИ, а в пункте «г)» запишите физический закон или формулу:
    \begin{enumerate}
        \item скорость света в среде,
        \item длина волны,
        \item период колебаний напряжённости электрического поля в электромагнитной волне,
        \item относительный показатель преломления среды.
    \end{enumerate}
}
\solutionspace{20pt}

\tasknumber{3}%
\task{%
    Выразите (нужен вывод из базовых физических законов):
    \begin{enumerate}
        \item частоту колебаний через длину волны и скорость её распространения,
        \item энергию фотона через длину электромагнитной волны,
        \item скорость света в среде через её абсолютный показатель преломления и скорость света в вакууме.
    \end{enumerate}
}
\solutionspace{60pt}

\tasknumber{4}%
\task{%
    Укажите букву, соответствующую физическую величину (из текущего раздела), её едииницы измерения в СИ и выразите её из какого-либо уравнения:
    \begin{enumerate}
        \item «йэ»,
        \item «вэ»,
        \item «н'у»,
        \item «тэ».
    \end{enumerate}
}
\solutionspace{100pt}

\tasknumber{5}%
\task{%
    Напротив каждой приставки единиц СИ укажите её полное название и соответствующий множитель:
    \begin{enumerate}
        \item м,
        \item к,
        \item н,
        \item Т.
    \end{enumerate}
}

\variantsplitter

\addpersonalvariant{Полина Лоткова}

\tasknumber{1}%
\task{%
    Длина волны света в~вакууме $\lambda = 600\,\text{нм}$.
    Какова частота этой световой волны?
    Какова длина этой волны в среде с показателем преломления $n = 1{,}3$?
    Может ли человек увидеть такую волну света, и если да, то какой именно цвет соответствует этим волнам в вакууме и в этой среде?
}
\answer{%
    \begin{align*}
    \nu &= \frac 1T = \frac 1{\lambda/c} = \frac c\lambda = \frac{3 \cdot 10^{8}\,\frac{\text{м}}{\text{с}}}{600\,\text{нм}} \approx 5 \cdot 10^{14}\,\text{Гц}, \\
    \nu' = \nu &\cbr{\text{или } T' = T} \implies \lambda' = v'T' = \frac vn T = \frac{ vt }n = \frac \lambda n = \frac{600\,\text{нм}}{1{,}3} \approx 462 \cdot 10^{-9}\,\text{м}.
    \\
    &\text{380 нм---фиол---440---син---485---гол---500---зел---565---жёл---590---оранж---625---крас---780 нм}
    \end{align*}
}
\solutionspace{180pt}

\tasknumber{2}%
\task{%
    Напротив физических величин укажите их обозначения и единицы измерения в СИ, а в пункте «г)» запишите физический закон или формулу:
    \begin{enumerate}
        \item корость света в вакууме,
        \item длина волны,
        \item период колебаний напряжённости электрического поля в электромагнитной волне,
        \item относительный показатель преломления среды.
    \end{enumerate}
}
\solutionspace{20pt}

\tasknumber{3}%
\task{%
    Выразите (нужен вывод из базовых физических законов):
    \begin{enumerate}
        \item период колебаний через длину волны и скорость её распространения,
        \item энергию фотона через длину электромагнитной волны,
        \item скорость света в среде через её абсолютный показатель преломления и скорость света в вакууме.
    \end{enumerate}
}
\solutionspace{60pt}

\tasknumber{4}%
\task{%
    Укажите букву, соответствующую физическую величину (из текущего раздела), её едииницы измерения в СИ и выразите её из какого-либо уравнения:
    \begin{enumerate}
        \item «л'амбда»,
        \item «вэ»,
        \item «н'у»,
        \item «тэ».
    \end{enumerate}
}
\solutionspace{100pt}

\tasknumber{5}%
\task{%
    Напротив каждой приставки единиц СИ укажите её полное название и соответствующий множитель:
    \begin{enumerate}
        \item М,
        \item к,
        \item н,
        \item Т.
    \end{enumerate}
}

\variantsplitter

\addpersonalvariant{Екатерина Медведева}

\tasknumber{1}%
\task{%
    Длина волны света в~вакууме $\lambda = 500\,\text{нм}$.
    Какова частота этой световой волны?
    Какова длина этой волны в среде с показателем преломления $n = 1{,}6$?
    Может ли человек увидеть такую волну света, и если да, то какой именно цвет соответствует этим волнам в вакууме и в этой среде?
}
\answer{%
    \begin{align*}
    \nu &= \frac 1T = \frac 1{\lambda/c} = \frac c\lambda = \frac{3 \cdot 10^{8}\,\frac{\text{м}}{\text{с}}}{500\,\text{нм}} \approx 6 \cdot 10^{14}\,\text{Гц}, \\
    \nu' = \nu &\cbr{\text{или } T' = T} \implies \lambda' = v'T' = \frac vn T = \frac{ vt }n = \frac \lambda n = \frac{500\,\text{нм}}{1{,}6} \approx 313 \cdot 10^{-9}\,\text{м}.
    \\
    &\text{380 нм---фиол---440---син---485---гол---500---зел---565---жёл---590---оранж---625---крас---780 нм}
    \end{align*}
}
\solutionspace{180pt}

\tasknumber{2}%
\task{%
    Напротив физических величин укажите их обозначения и единицы измерения в СИ, а в пункте «г)» запишите физический закон или формулу:
    \begin{enumerate}
        \item корость света в вакууме,
        \item длина волны,
        \item период колебаний индукции магнитного поля в электромагнитной волне,
        \item относительный показатель преломления среды.
    \end{enumerate}
}
\solutionspace{20pt}

\tasknumber{3}%
\task{%
    Выразите (нужен вывод из базовых физических законов):
    \begin{enumerate}
        \item период колебаний через длину волны и скорость её распространения,
        \item энергию фотона через период колебаний в электромагнитной волне,
        \item скорость света в вакууме через скорость света в среде и её абсолютный показатель преломления.
    \end{enumerate}
}
\solutionspace{60pt}

\tasknumber{4}%
\task{%
    Укажите букву, соответствующую физическую величину (из текущего раздела), её едииницы измерения в СИ и выразите её из какого-либо уравнения:
    \begin{enumerate}
        \item «л'амбда»,
        \item «вэ»,
        \item «аш»,
        \item «тэ».
    \end{enumerate}
}
\solutionspace{100pt}

\tasknumber{5}%
\task{%
    Напротив каждой приставки единиц СИ укажите её полное название и соответствующий множитель:
    \begin{enumerate}
        \item М,
        \item к,
        \item Г,
        \item Т.
    \end{enumerate}
}

\variantsplitter

\addpersonalvariant{Константин Мельник}

\tasknumber{1}%
\task{%
    Длина волны света в~вакууме $\lambda = 400\,\text{нм}$.
    Какова частота этой световой волны?
    Какова длина этой волны в среде с показателем преломления $n = 1{,}5$?
    Может ли человек увидеть такую волну света, и если да, то какой именно цвет соответствует этим волнам в вакууме и в этой среде?
}
\answer{%
    \begin{align*}
    \nu &= \frac 1T = \frac 1{\lambda/c} = \frac c\lambda = \frac{3 \cdot 10^{8}\,\frac{\text{м}}{\text{с}}}{400\,\text{нм}} \approx 7{,}50 \cdot 10^{14}\,\text{Гц}, \\
    \nu' = \nu &\cbr{\text{или } T' = T} \implies \lambda' = v'T' = \frac vn T = \frac{ vt }n = \frac \lambda n = \frac{400\,\text{нм}}{1{,}5} \approx 267 \cdot 10^{-9}\,\text{м}.
    \\
    &\text{380 нм---фиол---440---син---485---гол---500---зел---565---жёл---590---оранж---625---крас---780 нм}
    \end{align*}
}
\solutionspace{180pt}

\tasknumber{2}%
\task{%
    Напротив физических величин укажите их обозначения и единицы измерения в СИ, а в пункте «г)» запишите физический закон или формулу:
    \begin{enumerate}
        \item скорость света в среде,
        \item частоты волны,
        \item период колебаний напряжённости электрического поля в электромагнитной волне,
        \item относительный показатель преломления среды.
    \end{enumerate}
}
\solutionspace{20pt}

\tasknumber{3}%
\task{%
    Выразите (нужен вывод из базовых физических законов):
    \begin{enumerate}
        \item частоту колебаний через длину волны и скорость её распространения,
        \item энергию фотона через период колебаний в электромагнитной волне,
        \item скорость света в вакууме через скорость света в среде и её абсолютный показатель преломления.
    \end{enumerate}
}
\solutionspace{60pt}

\tasknumber{4}%
\task{%
    Укажите букву, соответствующую физическую величину (из текущего раздела), её едииницы измерения в СИ и выразите её из какого-либо уравнения:
    \begin{enumerate}
        \item «йэ»,
        \item «цэ»,
        \item «н'у»,
        \item «тэ».
    \end{enumerate}
}
\solutionspace{100pt}

\tasknumber{5}%
\task{%
    Напротив каждой приставки единиц СИ укажите её полное название и соответствующий множитель:
    \begin{enumerate}
        \item м,
        \item мк,
        \item н,
        \item Т.
    \end{enumerate}
}

\variantsplitter

\addpersonalvariant{Степан Небоваренков}

\tasknumber{1}%
\task{%
    Длина волны света в~вакууме $\lambda = 600\,\text{нм}$.
    Какова частота этой световой волны?
    Какова длина этой волны в среде с показателем преломления $n = 1{,}6$?
    Может ли человек увидеть такую волну света, и если да, то какой именно цвет соответствует этим волнам в вакууме и в этой среде?
}
\answer{%
    \begin{align*}
    \nu &= \frac 1T = \frac 1{\lambda/c} = \frac c\lambda = \frac{3 \cdot 10^{8}\,\frac{\text{м}}{\text{с}}}{600\,\text{нм}} \approx 5 \cdot 10^{14}\,\text{Гц}, \\
    \nu' = \nu &\cbr{\text{или } T' = T} \implies \lambda' = v'T' = \frac vn T = \frac{ vt }n = \frac \lambda n = \frac{600\,\text{нм}}{1{,}6} \approx 375 \cdot 10^{-9}\,\text{м}.
    \\
    &\text{380 нм---фиол---440---син---485---гол---500---зел---565---жёл---590---оранж---625---крас---780 нм}
    \end{align*}
}
\solutionspace{180pt}

\tasknumber{2}%
\task{%
    Напротив физических величин укажите их обозначения и единицы измерения в СИ, а в пункте «г)» запишите физический закон или формулу:
    \begin{enumerate}
        \item скорость света в среде,
        \item длина волны,
        \item период колебаний индукции магнитного поля в электромагнитной волне,
        \item относительный показатель преломления среды.
    \end{enumerate}
}
\solutionspace{20pt}

\tasknumber{3}%
\task{%
    Выразите (нужен вывод из базовых физических законов):
    \begin{enumerate}
        \item частоту колебаний через длину волны и скорость её распространения,
        \item энергию фотона через длину электромагнитной волны,
        \item скорость света в вакууме через скорость света в среде и её абсолютный показатель преломления.
    \end{enumerate}
}
\solutionspace{60pt}

\tasknumber{4}%
\task{%
    Укажите букву, соответствующую физическую величину (из текущего раздела), её едииницы измерения в СИ и выразите её из какого-либо уравнения:
    \begin{enumerate}
        \item «йэ»,
        \item «вэ»,
        \item «аш»,
        \item «тэ».
    \end{enumerate}
}
\solutionspace{100pt}

\tasknumber{5}%
\task{%
    Напротив каждой приставки единиц СИ укажите её полное название и соответствующий множитель:
    \begin{enumerate}
        \item м,
        \item к,
        \item Г,
        \item Т.
    \end{enumerate}
}

\variantsplitter

\addpersonalvariant{Матвей Неретин}

\tasknumber{1}%
\task{%
    Длина волны света в~вакууме $\lambda = 400\,\text{нм}$.
    Какова частота этой световой волны?
    Какова длина этой волны в среде с показателем преломления $n = 1{,}7$?
    Может ли человек увидеть такую волну света, и если да, то какой именно цвет соответствует этим волнам в вакууме и в этой среде?
}
\answer{%
    \begin{align*}
    \nu &= \frac 1T = \frac 1{\lambda/c} = \frac c\lambda = \frac{3 \cdot 10^{8}\,\frac{\text{м}}{\text{с}}}{400\,\text{нм}} \approx 7{,}50 \cdot 10^{14}\,\text{Гц}, \\
    \nu' = \nu &\cbr{\text{или } T' = T} \implies \lambda' = v'T' = \frac vn T = \frac{ vt }n = \frac \lambda n = \frac{400\,\text{нм}}{1{,}7} \approx 235 \cdot 10^{-9}\,\text{м}.
    \\
    &\text{380 нм---фиол---440---син---485---гол---500---зел---565---жёл---590---оранж---625---крас---780 нм}
    \end{align*}
}
\solutionspace{180pt}

\tasknumber{2}%
\task{%
    Напротив физических величин укажите их обозначения и единицы измерения в СИ, а в пункте «г)» запишите физический закон или формулу:
    \begin{enumerate}
        \item корость света в вакууме,
        \item частоты волны,
        \item период колебаний напряжённости электрического поля в электромагнитной волне,
        \item абсолютный показатель преломления среды.
    \end{enumerate}
}
\solutionspace{20pt}

\tasknumber{3}%
\task{%
    Выразите (нужен вывод из базовых физических законов):
    \begin{enumerate}
        \item частоту колебаний через длину волны и скорость её распространения,
        \item энергию фотона через период колебаний в электромагнитной волне,
        \item скорость света в вакууме через скорость света в среде и её абсолютный показатель преломления.
    \end{enumerate}
}
\solutionspace{60pt}

\tasknumber{4}%
\task{%
    Укажите букву, соответствующую физическую величину (из текущего раздела), её едииницы измерения в СИ и выразите её из какого-либо уравнения:
    \begin{enumerate}
        \item «л'амбда»,
        \item «цэ»,
        \item «н'у»,
        \item «эн».
    \end{enumerate}
}
\solutionspace{100pt}

\tasknumber{5}%
\task{%
    Напротив каждой приставки единиц СИ укажите её полное название и соответствующий множитель:
    \begin{enumerate}
        \item М,
        \item мк,
        \item н,
        \item с.
    \end{enumerate}
}

\variantsplitter

\addpersonalvariant{Мария Никонова}

\tasknumber{1}%
\task{%
    Длина волны света в~вакууме $\lambda = 400\,\text{нм}$.
    Какова частота этой световой волны?
    Какова длина этой волны в среде с показателем преломления $n = 1{,}4$?
    Может ли человек увидеть такую волну света, и если да, то какой именно цвет соответствует этим волнам в вакууме и в этой среде?
}
\answer{%
    \begin{align*}
    \nu &= \frac 1T = \frac 1{\lambda/c} = \frac c\lambda = \frac{3 \cdot 10^{8}\,\frac{\text{м}}{\text{с}}}{400\,\text{нм}} \approx 7{,}50 \cdot 10^{14}\,\text{Гц}, \\
    \nu' = \nu &\cbr{\text{или } T' = T} \implies \lambda' = v'T' = \frac vn T = \frac{ vt }n = \frac \lambda n = \frac{400\,\text{нм}}{1{,}4} \approx 286 \cdot 10^{-9}\,\text{м}.
    \\
    &\text{380 нм---фиол---440---син---485---гол---500---зел---565---жёл---590---оранж---625---крас---780 нм}
    \end{align*}
}
\solutionspace{180pt}

\tasknumber{2}%
\task{%
    Напротив физических величин укажите их обозначения и единицы измерения в СИ, а в пункте «г)» запишите физический закон или формулу:
    \begin{enumerate}
        \item скорость света в среде,
        \item длина волны,
        \item период колебаний индукции магнитного поля в электромагнитной волне,
        \item относительный показатель преломления среды.
    \end{enumerate}
}
\solutionspace{20pt}

\tasknumber{3}%
\task{%
    Выразите (нужен вывод из базовых физических законов):
    \begin{enumerate}
        \item период колебаний через длину волны и скорость её распространения,
        \item энергию фотона через длину электромагнитной волны,
        \item скорость света в среде через её абсолютный показатель преломления и скорость света в вакууме.
    \end{enumerate}
}
\solutionspace{60pt}

\tasknumber{4}%
\task{%
    Укажите букву, соответствующую физическую величину (из текущего раздела), её едииницы измерения в СИ и выразите её из какого-либо уравнения:
    \begin{enumerate}
        \item «йэ»,
        \item «вэ»,
        \item «аш»,
        \item «тэ».
    \end{enumerate}
}
\solutionspace{100pt}

\tasknumber{5}%
\task{%
    Напротив каждой приставки единиц СИ укажите её полное название и соответствующий множитель:
    \begin{enumerate}
        \item м,
        \item к,
        \item Г,
        \item Т.
    \end{enumerate}
}

\variantsplitter

\addpersonalvariant{Даниил Палаткин}

\tasknumber{1}%
\task{%
    Длина волны света в~вакууме $\lambda = 400\,\text{нм}$.
    Какова частота этой световой волны?
    Какова длина этой волны в среде с показателем преломления $n = 1{,}6$?
    Может ли человек увидеть такую волну света, и если да, то какой именно цвет соответствует этим волнам в вакууме и в этой среде?
}
\answer{%
    \begin{align*}
    \nu &= \frac 1T = \frac 1{\lambda/c} = \frac c\lambda = \frac{3 \cdot 10^{8}\,\frac{\text{м}}{\text{с}}}{400\,\text{нм}} \approx 7{,}50 \cdot 10^{14}\,\text{Гц}, \\
    \nu' = \nu &\cbr{\text{или } T' = T} \implies \lambda' = v'T' = \frac vn T = \frac{ vt }n = \frac \lambda n = \frac{400\,\text{нм}}{1{,}6} \approx 250 \cdot 10^{-9}\,\text{м}.
    \\
    &\text{380 нм---фиол---440---син---485---гол---500---зел---565---жёл---590---оранж---625---крас---780 нм}
    \end{align*}
}
\solutionspace{180pt}

\tasknumber{2}%
\task{%
    Напротив физических величин укажите их обозначения и единицы измерения в СИ, а в пункте «г)» запишите физический закон или формулу:
    \begin{enumerate}
        \item корость света в вакууме,
        \item частоты волны,
        \item период колебаний индукции магнитного поля в электромагнитной волне,
        \item абсолютный показатель преломления среды.
    \end{enumerate}
}
\solutionspace{20pt}

\tasknumber{3}%
\task{%
    Выразите (нужен вывод из базовых физических законов):
    \begin{enumerate}
        \item частоту колебаний через длину волны и скорость её распространения,
        \item энергию фотона через длину электромагнитной волны,
        \item скорость света в среде через её абсолютный показатель преломления и скорость света в вакууме.
    \end{enumerate}
}
\solutionspace{60pt}

\tasknumber{4}%
\task{%
    Укажите букву, соответствующую физическую величину (из текущего раздела), её едииницы измерения в СИ и выразите её из какого-либо уравнения:
    \begin{enumerate}
        \item «л'амбда»,
        \item «цэ»,
        \item «аш»,
        \item «эн».
    \end{enumerate}
}
\solutionspace{100pt}

\tasknumber{5}%
\task{%
    Напротив каждой приставки единиц СИ укажите её полное название и соответствующий множитель:
    \begin{enumerate}
        \item М,
        \item мк,
        \item Г,
        \item с.
    \end{enumerate}
}

\variantsplitter

\addpersonalvariant{Станислав Пикун}

\tasknumber{1}%
\task{%
    Длина волны света в~вакууме $\lambda = 600\,\text{нм}$.
    Какова частота этой световой волны?
    Какова длина этой волны в среде с показателем преломления $n = 1{,}4$?
    Может ли человек увидеть такую волну света, и если да, то какой именно цвет соответствует этим волнам в вакууме и в этой среде?
}
\answer{%
    \begin{align*}
    \nu &= \frac 1T = \frac 1{\lambda/c} = \frac c\lambda = \frac{3 \cdot 10^{8}\,\frac{\text{м}}{\text{с}}}{600\,\text{нм}} \approx 5 \cdot 10^{14}\,\text{Гц}, \\
    \nu' = \nu &\cbr{\text{или } T' = T} \implies \lambda' = v'T' = \frac vn T = \frac{ vt }n = \frac \lambda n = \frac{600\,\text{нм}}{1{,}4} \approx 429 \cdot 10^{-9}\,\text{м}.
    \\
    &\text{380 нм---фиол---440---син---485---гол---500---зел---565---жёл---590---оранж---625---крас---780 нм}
    \end{align*}
}
\solutionspace{180pt}

\tasknumber{2}%
\task{%
    Напротив физических величин укажите их обозначения и единицы измерения в СИ, а в пункте «г)» запишите физический закон или формулу:
    \begin{enumerate}
        \item корость света в вакууме,
        \item частоты волны,
        \item период колебаний напряжённости электрического поля в электромагнитной волне,
        \item абсолютный показатель преломления среды.
    \end{enumerate}
}
\solutionspace{20pt}

\tasknumber{3}%
\task{%
    Выразите (нужен вывод из базовых физических законов):
    \begin{enumerate}
        \item частоту колебаний через длину волны и скорость её распространения,
        \item энергию фотона через период колебаний в электромагнитной волне,
        \item скорость света в вакууме через скорость света в среде и её абсолютный показатель преломления.
    \end{enumerate}
}
\solutionspace{60pt}

\tasknumber{4}%
\task{%
    Укажите букву, соответствующую физическую величину (из текущего раздела), её едииницы измерения в СИ и выразите её из какого-либо уравнения:
    \begin{enumerate}
        \item «л'амбда»,
        \item «цэ»,
        \item «н'у»,
        \item «эн».
    \end{enumerate}
}
\solutionspace{100pt}

\tasknumber{5}%
\task{%
    Напротив каждой приставки единиц СИ укажите её полное название и соответствующий множитель:
    \begin{enumerate}
        \item М,
        \item мк,
        \item н,
        \item с.
    \end{enumerate}
}

\variantsplitter

\addpersonalvariant{Илья Пичугин}

\tasknumber{1}%
\task{%
    Длина волны света в~вакууме $\lambda = 700\,\text{нм}$.
    Какова частота этой световой волны?
    Какова длина этой волны в среде с показателем преломления $n = 1{,}5$?
    Может ли человек увидеть такую волну света, и если да, то какой именно цвет соответствует этим волнам в вакууме и в этой среде?
}
\answer{%
    \begin{align*}
    \nu &= \frac 1T = \frac 1{\lambda/c} = \frac c\lambda = \frac{3 \cdot 10^{8}\,\frac{\text{м}}{\text{с}}}{700\,\text{нм}} \approx 4{,}29 \cdot 10^{14}\,\text{Гц}, \\
    \nu' = \nu &\cbr{\text{или } T' = T} \implies \lambda' = v'T' = \frac vn T = \frac{ vt }n = \frac \lambda n = \frac{700\,\text{нм}}{1{,}5} \approx 467 \cdot 10^{-9}\,\text{м}.
    \\
    &\text{380 нм---фиол---440---син---485---гол---500---зел---565---жёл---590---оранж---625---крас---780 нм}
    \end{align*}
}
\solutionspace{180pt}

\tasknumber{2}%
\task{%
    Напротив физических величин укажите их обозначения и единицы измерения в СИ, а в пункте «г)» запишите физический закон или формулу:
    \begin{enumerate}
        \item корость света в вакууме,
        \item длина волны,
        \item период колебаний напряжённости электрического поля в электромагнитной волне,
        \item абсолютный показатель преломления среды.
    \end{enumerate}
}
\solutionspace{20pt}

\tasknumber{3}%
\task{%
    Выразите (нужен вывод из базовых физических законов):
    \begin{enumerate}
        \item период колебаний через длину волны и скорость её распространения,
        \item энергию фотона через длину электромагнитной волны,
        \item скорость света в среде через её абсолютный показатель преломления и скорость света в вакууме.
    \end{enumerate}
}
\solutionspace{60pt}

\tasknumber{4}%
\task{%
    Укажите букву, соответствующую физическую величину (из текущего раздела), её едииницы измерения в СИ и выразите её из какого-либо уравнения:
    \begin{enumerate}
        \item «л'амбда»,
        \item «вэ»,
        \item «н'у»,
        \item «эн».
    \end{enumerate}
}
\solutionspace{100pt}

\tasknumber{5}%
\task{%
    Напротив каждой приставки единиц СИ укажите её полное название и соответствующий множитель:
    \begin{enumerate}
        \item М,
        \item к,
        \item н,
        \item с.
    \end{enumerate}
}

\variantsplitter

\addpersonalvariant{Кирилл Севрюгин}

\tasknumber{1}%
\task{%
    Длина волны света в~вакууме $\lambda = 400\,\text{нм}$.
    Какова частота этой световой волны?
    Какова длина этой волны в среде с показателем преломления $n = 1{,}7$?
    Может ли человек увидеть такую волну света, и если да, то какой именно цвет соответствует этим волнам в вакууме и в этой среде?
}
\answer{%
    \begin{align*}
    \nu &= \frac 1T = \frac 1{\lambda/c} = \frac c\lambda = \frac{3 \cdot 10^{8}\,\frac{\text{м}}{\text{с}}}{400\,\text{нм}} \approx 7{,}50 \cdot 10^{14}\,\text{Гц}, \\
    \nu' = \nu &\cbr{\text{или } T' = T} \implies \lambda' = v'T' = \frac vn T = \frac{ vt }n = \frac \lambda n = \frac{400\,\text{нм}}{1{,}7} \approx 235 \cdot 10^{-9}\,\text{м}.
    \\
    &\text{380 нм---фиол---440---син---485---гол---500---зел---565---жёл---590---оранж---625---крас---780 нм}
    \end{align*}
}
\solutionspace{180pt}

\tasknumber{2}%
\task{%
    Напротив физических величин укажите их обозначения и единицы измерения в СИ, а в пункте «г)» запишите физический закон или формулу:
    \begin{enumerate}
        \item скорость света в среде,
        \item частоты волны,
        \item период колебаний напряжённости электрического поля в электромагнитной волне,
        \item относительный показатель преломления среды.
    \end{enumerate}
}
\solutionspace{20pt}

\tasknumber{3}%
\task{%
    Выразите (нужен вывод из базовых физических законов):
    \begin{enumerate}
        \item период колебаний через длину волны и скорость её распространения,
        \item энергию фотона через период колебаний в электромагнитной волне,
        \item скорость света в вакууме через скорость света в среде и её абсолютный показатель преломления.
    \end{enumerate}
}
\solutionspace{60pt}

\tasknumber{4}%
\task{%
    Укажите букву, соответствующую физическую величину (из текущего раздела), её едииницы измерения в СИ и выразите её из какого-либо уравнения:
    \begin{enumerate}
        \item «йэ»,
        \item «цэ»,
        \item «н'у»,
        \item «тэ».
    \end{enumerate}
}
\solutionspace{100pt}

\tasknumber{5}%
\task{%
    Напротив каждой приставки единиц СИ укажите её полное название и соответствующий множитель:
    \begin{enumerate}
        \item м,
        \item мк,
        \item н,
        \item Т.
    \end{enumerate}
}

\variantsplitter

\addpersonalvariant{Илья Стратонников}

\tasknumber{1}%
\task{%
    Длина волны света в~вакууме $\lambda = 600\,\text{нм}$.
    Какова частота этой световой волны?
    Какова длина этой волны в среде с показателем преломления $n = 1{,}6$?
    Может ли человек увидеть такую волну света, и если да, то какой именно цвет соответствует этим волнам в вакууме и в этой среде?
}
\answer{%
    \begin{align*}
    \nu &= \frac 1T = \frac 1{\lambda/c} = \frac c\lambda = \frac{3 \cdot 10^{8}\,\frac{\text{м}}{\text{с}}}{600\,\text{нм}} \approx 5 \cdot 10^{14}\,\text{Гц}, \\
    \nu' = \nu &\cbr{\text{или } T' = T} \implies \lambda' = v'T' = \frac vn T = \frac{ vt }n = \frac \lambda n = \frac{600\,\text{нм}}{1{,}6} \approx 375 \cdot 10^{-9}\,\text{м}.
    \\
    &\text{380 нм---фиол---440---син---485---гол---500---зел---565---жёл---590---оранж---625---крас---780 нм}
    \end{align*}
}
\solutionspace{180pt}

\tasknumber{2}%
\task{%
    Напротив физических величин укажите их обозначения и единицы измерения в СИ, а в пункте «г)» запишите физический закон или формулу:
    \begin{enumerate}
        \item скорость света в среде,
        \item частоты волны,
        \item период колебаний напряжённости электрического поля в электромагнитной волне,
        \item относительный показатель преломления среды.
    \end{enumerate}
}
\solutionspace{20pt}

\tasknumber{3}%
\task{%
    Выразите (нужен вывод из базовых физических законов):
    \begin{enumerate}
        \item период колебаний через длину волны и скорость её распространения,
        \item энергию фотона через длину электромагнитной волны,
        \item скорость света в вакууме через скорость света в среде и её абсолютный показатель преломления.
    \end{enumerate}
}
\solutionspace{60pt}

\tasknumber{4}%
\task{%
    Укажите букву, соответствующую физическую величину (из текущего раздела), её едииницы измерения в СИ и выразите её из какого-либо уравнения:
    \begin{enumerate}
        \item «йэ»,
        \item «цэ»,
        \item «н'у»,
        \item «тэ».
    \end{enumerate}
}
\solutionspace{100pt}

\tasknumber{5}%
\task{%
    Напротив каждой приставки единиц СИ укажите её полное название и соответствующий множитель:
    \begin{enumerate}
        \item м,
        \item мк,
        \item н,
        \item Т.
    \end{enumerate}
}

\variantsplitter

\addpersonalvariant{Иван Шустов}

\tasknumber{1}%
\task{%
    Длина волны света в~вакууме $\lambda = 600\,\text{нм}$.
    Какова частота этой световой волны?
    Какова длина этой волны в среде с показателем преломления $n = 1{,}3$?
    Может ли человек увидеть такую волну света, и если да, то какой именно цвет соответствует этим волнам в вакууме и в этой среде?
}
\answer{%
    \begin{align*}
    \nu &= \frac 1T = \frac 1{\lambda/c} = \frac c\lambda = \frac{3 \cdot 10^{8}\,\frac{\text{м}}{\text{с}}}{600\,\text{нм}} \approx 5 \cdot 10^{14}\,\text{Гц}, \\
    \nu' = \nu &\cbr{\text{или } T' = T} \implies \lambda' = v'T' = \frac vn T = \frac{ vt }n = \frac \lambda n = \frac{600\,\text{нм}}{1{,}3} \approx 462 \cdot 10^{-9}\,\text{м}.
    \\
    &\text{380 нм---фиол---440---син---485---гол---500---зел---565---жёл---590---оранж---625---крас---780 нм}
    \end{align*}
}
\solutionspace{180pt}

\tasknumber{2}%
\task{%
    Напротив физических величин укажите их обозначения и единицы измерения в СИ, а в пункте «г)» запишите физический закон или формулу:
    \begin{enumerate}
        \item скорость света в среде,
        \item частоты волны,
        \item период колебаний напряжённости электрического поля в электромагнитной волне,
        \item относительный показатель преломления среды.
    \end{enumerate}
}
\solutionspace{20pt}

\tasknumber{3}%
\task{%
    Выразите (нужен вывод из базовых физических законов):
    \begin{enumerate}
        \item период колебаний через длину волны и скорость её распространения,
        \item энергию фотона через длину электромагнитной волны,
        \item скорость света в среде через её абсолютный показатель преломления и скорость света в вакууме.
    \end{enumerate}
}
\solutionspace{60pt}

\tasknumber{4}%
\task{%
    Укажите букву, соответствующую физическую величину (из текущего раздела), её едииницы измерения в СИ и выразите её из какого-либо уравнения:
    \begin{enumerate}
        \item «йэ»,
        \item «цэ»,
        \item «н'у»,
        \item «тэ».
    \end{enumerate}
}
\solutionspace{100pt}

\tasknumber{5}%
\task{%
    Напротив каждой приставки единиц СИ укажите её полное название и соответствующий множитель:
    \begin{enumerate}
        \item м,
        \item мк,
        \item н,
        \item Т.
    \end{enumerate}
}
% autogenerated
