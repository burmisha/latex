\newcommand\rootpath{../../..}
\documentclass[12pt,a4paper]{amsart}%DVI-mode.
\usepackage{graphics,graphicx,epsfig}%DVI-mode.
%\documentclass[pdftex,12pt]{amsart} %PDF-mode.
%\usepackage[pdftex]{graphicx}       %PDF-mode.

%\usepackage{a4wide}                 % Fit the text to A4 page tightly.
\usepackage[utf8]{inputenc}
\usepackage[T2A]{fontenc}
\usepackage[english,russian]{babel} % Download Russian fonts.
\usepackage{amsmath,amsfonts,amssymb,amsthm,amscd,mathrsfs} % Use AMS symbols.
\usepackage{tikz}
\usetikzlibrary{circuits.ee.IEC}
\usetikzlibrary{shapes.geometric}
\usetikzlibrary{decorations.markings}
%\usetikzlibrary{dashs}
%\usetikzlibrary{info}


\textheight=29cm % высота текста
\textwidth=18cm % ширина текста
\topmargin=-2.5cm % отступ от верхнего края
\parskip=6pt % интервал между абзацами
\oddsidemargin=-1.5cm
\evensidemargin=-1.5cm 

% wide docs
% \oddsidemargin=0cm
% \evensidemargin=0cm 
% \textheight=29cm % высота текста
% \textwidth=15cm % ширина текста
% \topmargin=-1.5cm % отступ от верхнего края
% \parskip=18pt % интервал между абзацами


\parindent=0pt % абзацный отступ
\tolerance=500 % терпимость к "жидким" строкам
\binoppenalty=10000 % штраф за перенос формул - 10000 - абсолютный запрет
\relpenalty=10000
\flushbottom % выравнивание высоты страниц
\def\baselinestretch{1.00}
\pagenumbering{gobble}

\begin{document}
\newcommand\bivec[2]{\begin{pmatrix} #1 \\ #2 \end{pmatrix}}

\newcommand\ol[1]{\overline{#1}}

\newcommand\p[1]{\ensuremath{\Prob\!\left(#1\right)}}
\def\cond{\,|\,}
\newcommand\e[1]{\mathsf{E}\!\left(#1\right)}
\newcommand\disp[1]{\mathsf{D}\!\left(#1\right)}
%\newcommand\norm[2]{\mathcal{N}\!\cbr{#1,#2}}
\newcommand\sign{\text{ sign }}

\newcommand\al[1]{\begin{align*} #1 \end{align*}}
\newcommand\begcas[1]{\begin{cases}#1\end{cases}}
\newcommand\tab[2]{	\vspace{-#1pt}
						\begin{tabbing} 
						#2
						\end{tabbing}
					\vspace{-#1pt}
					}


\newcommand\maintext[1]{{\bfseries\sffamily{#1}}}
\newcommand\simpletitle[1]{\begin{center} \maintext{#1} \end{center}}

\def\le{\leqslant}
\def\ge{\geqslant}
\def\Ell{\mathcal{L}}
\def\eps{\varepsilon}
\def\x{\ensuremath{\textbf{x}}}
\def\y{\ensuremath{\textbf{y}}}
\def\Rn{\ensuremath{\mathbb{R}^n}}
\def\RSS{\mathsf{RSS}}

\newcommand\mb[1]{\ensuremath{\boldsymbol{\mathbf{#1}}}}
\newcommand\argmax[1]{\arg\underset{#1}\max\,} % \operatornamewithlimits
%\newcommand{\prodl}{\mathop{\textstyle\prod}\limits}
\newcommand{\prodl}{\prod\limits}
\newcommand{\suml}{\sum\limits}
\newcommand\foral[1]{\forall\,#1\:}
\newcommand\exist[1]{\exists\,#1\:\colon}

\newcommand\cbr[1]{\left(#1\right)} %circled brackets
\newcommand\fbr[1]{\left\{#1\right\}} %figure brackets
\newcommand\sbr[1]{\left[#1\right]} %square brackets
\newcommand\modul[1]{\left|#1\right|}
\newcommand\cdf[2]{\cdot\frac{#1}{#2}}
\newcommand\integr[3]{\int\limits_{#1}^{#2}{#3}}
\newcommand\obol[1]{O\!\cbr{#1}}
\newcommand\norm[1]{\ensuremath{\left\|{#1}\right\|}}

\newcommand\dd[2]{\frac{\partial#1}{\partial#2}}

\newcommand\addeps[2]{
	\begin{figure} [!ht] %lrp
		\centering
		\includegraphics[height=240px]{#1.eps}
		\vspace{-10pt}
		\caption{#2}
		\label{eps:#1}
	\end{figure}
}

\newcommand\addtikz[4]{
	\begin{figure} [!ht] %lrp
		\centering
		\begin{tikzpicture}[x=#2cm,y=#2cm,#3]
			\input{#1.tikz}
		\end{tikzpicture}
		\vspace{-10pt}
		\caption{#4}
		\label{tikz:#1}	
	\end{figure}
}



\newcommand\addepssize[3]{
	\begin{figure} [!ht] %lrp hp
		\centering
		\includegraphics[height=#3px]{#1.eps}
		\vspace{-10pt}
		\caption{#2}
		\label{eps:#1}
	\end{figure}
}

\def\algorithmicrequire{\textbf{Вход:}}
\def\algorithmicensure{\textbf{Выход:}}
\def\algorithmicif{\textbf{если}}
\def\algorithmicthen{\textbf{то}}
\def\algorithmicelse{\textbf{иначе}}
\def\algorithmicelsif{\textbf{иначе если}}
\def\algorithmicfor{\textbf{для}}
\def\algorithmicforall{\textbf{для всех}}
\def\algorithmicdo{}
\def\algorithmicwhile{\textbf{пока}}
\def\algorithmicrepeat{\textbf{повторять}}
\def\algorithmicuntil{\textbf{пока}}
\def\algorithmicloop{\textbf{цикл}}
% переопределение стиля комментариев
\def\algorithmiccomment#1{\quad// {\sl #1}}
%\raggedright
\classdate{7}{20 апреля 2018}

\task 1
Площадь большого поршня гидравлического домкрата $S_1 = 20\units{см}^2$, а малого $S_2 = 0{,}5\units{см}^2.$ Груз какой максимальной массы можно поднять этим домкратом, если на малый поршень давить с силой не более $F=200\units{Н}?$ Силой трения от стенки цилиндров пренебречь.

\task 2
В сосуд налита вода. Расстояние от поверхности воды до дна $H = 0{,}5\units{м},$ площадь дна $S = 0{,}1\units{м}^2.$ Найти гидростатическое давление $P_1$ и полное давление $P_2$ вблизи дна. Найти силу давления воды на дно. Плотность воды \rhowater

\task 3
На лёгкий поршень площадью $S=900\units{см}^2,$ касающийся поверхности воды, поставили гирю массы $m=3\units{кг}$. Высота слоя воды в сосуде с вертикальными стенками $H = 20\units{см}$. Определить давление жидкости вблизи дна, если плотность воды \rhowater

\task 4
Давление газов в конце сгорания в цилиндре дизельного двигателя трактора $P = 9\units{МПа}.$ Диаметр цилиндра $d = 130\units{мм}.$ С какой силой газы давят на поршень в цилиндре? Площадь круга диаметром $D$ равна $S = \cfrac{\pi D^2}4.$

\task 5
Площадь малого поршня гидравлического подъёмника $S_1 = 0{,}8\units{см}^2$, а большого $S_2 = 40\units{см}^2.$ Какую силу $F$ надо приложить к малому поршню, чтобы поднять груз весом $P = 8\units{кН}?$

\task 6
Герметичный сосуд полностью заполнен водой и стоит на столе. На небольшой поршень площадью $S$ давят рукой с силой $F$. Поршень находится ниже крышки сосуда на $H_1$, выше дна на $H_2$ и может свободно перемещаться. Плотность воды $\rho$, атмосферное давление $P_A$. Найти давления $P_1$ и $P_2$ в воде вблизи крышки и дна сосуда.
\\ \\
\classdate{7}{20 апреля 2018}

\task 1
Площадь большого поршня гидравлического домкрата $S_1 = 20\units{см}^2$, а малого $S_2 = 0{,}5\units{см}^2.$ Груз какой максимальной массы можно поднять этим домкратом, если на малый поршень давить с силой не более $F=200\units{Н}?$ Силой трения от стенки цилиндров пренебречь.

\task 2
В сосуд налита вода. Расстояние от поверхности воды до дна $H = 0{,}5\units{м},$ площадь дна $S = 0{,}1\units{м}^2.$ Найти гидростатическое давление $P_1$ и полное давление $P_2$ вблизи дна. Найти силу давления воды на дно. Плотность воды \rhowater

\task 3
На лёгкий поршень площадью $S=900\units{см}^2,$ касающийся поверхности воды, поставили гирю массы $m=3\units{кг}$. Высота слоя воды в сосуде с вертикальными стенками $H = 20\units{см}$. Определить давление жидкости вблизи дна, если плотность воды \rhowater

\task 4
Давление газов в конце сгорания в цилиндре дизельного двигателя трактора $P = 9\units{МПа}.$ Диаметр цилиндра $d = 130\units{мм}.$ С какой силой газы давят на поршень в цилиндре? Площадь круга диаметром $D$ равна $S = \cfrac{\pi D^2}4.$

\task 5
Площадь малого поршня гидравлического подъёмника $S_1 = 0{,}8\units{см}^2$, а большого $S_2 = 40\units{см}^2.$ Какую силу $F$ надо приложить к малому поршню, чтобы поднять груз весом $P = 8\units{кН}?$

\task 6
Герметичный сосуд полностью заполнен водой и стоит на столе. На небольшой поршень площадью $S$ давят рукой с силой $F$. Поршень находится ниже крышки сосуда на $H_1$, выше дна на $H_2$ и может свободно перемещаться. Плотность воды $\rho$, атмосферное давление $P_A$. Найти давления $P_1$ и $P_2$ в воде вблизи крышки и дна сосуда.

\newpage

\adddate{8 класс. 20 апреля 2018}

\task 1
Между точками $A$ и $B$ электрической цепи подключены последовательно резисторы $R_1 = 10\units{Ом}$ и $R_2 = 20\units{Ом}$ и параллельно им $R_3 = 30\units{Ом}.$ Найдите эквивалентное сопротивление $R_{AB}$ этого участка цепи.

\task 2
Электрическая цепь состоит из последовательности $N$ одинаковых звеньев, в которых каждый резистор имеет сопротивление $r$. Последнее звено замкнуто резистором сопротивлением $R$. При каком соотношении $\cfrac{R}{r}$ сопротивление цепи не зависит от числа звеньев?

\task 3
Для измерения сопротивления $R$ проводника собрана электрическая цепь. Вольтметр $V$ показывает напряжение $U_V = 5\units{В},$ показание амперметра $A$ равно $I_A = 25\units{мА}.$ Найдите величину $R$ сопротивления проводника. Внутреннее сопротивление вольтметра $R_V = 1{,}0\units{кОм},$ внутреннее сопротивление амперметра $R_A = 2{,}0\units{Ом}.$

\task 4
Шкала гальванометра имеет $N=100$ делений, цена деления $\delta = 1\units{мкА}$. Внутреннее сопротивление гальванометра $R_G = 1{,}0\units{кОм}.$ Как из этого прибора сделать вольтметр для измерения напряжений до $U = 100\units{В}$ или амперметр для измерения токов силой до $I = 1\units{А}?$

\\ \\ \\ \\ \\ \\ \\ \\
\adddate{8 класс. 20 апреля 2018}

\task 1
Между точками $A$ и $B$ электрической цепи подключены последовательно резисторы $R_1 = 10\units{Ом}$ и $R_2 = 20\units{Ом}$ и параллельно им $R_3 = 30\units{Ом}.$ Найдите эквивалентное сопротивление $R_{AB}$ этого участка цепи.

\task 2
Электрическая цепь состоит из последовательности $N$ одинаковых звеньев, в которых каждый резистор имеет сопротивление $r$. Последнее звено замкнуто резистором сопротивлением $R$. При каком соотношении $\cfrac{R}{r}$ сопротивление цепи не зависит от числа звеньев?

\task 3
Для измерения сопротивления $R$ проводника собрана электрическая цепь. Вольтметр $V$ показывает напряжение $U_V = 5\units{В},$ показание амперметра $A$ равно $I_A = 25\units{мА}.$ Найдите величину $R$ сопротивления проводника. Внутреннее сопротивление вольтметра $R_V = 1{,}0\units{кОм},$ внутреннее сопротивление амперметра $R_A = 2{,}0\units{Ом}.$

\task 4
Шкала гальванометра имеет $N=100$ делений, цена деления $\delta = 1\units{мкА}$. Внутреннее сопротивление гальванометра $R_G = 1{,}0\units{кОм}.$ Как из этого прибора сделать вольтметр для измерения напряжений до $U = 100\units{В}$ или амперметр для измерения токов силой до $I = 1\units{А}?$


% \begin{flushright}
\textsc{ГБОУ школа №554, 20 ноября 2018\,г.}
\end{flushright}

\begin{center}
\LARGE \textsc{Математический бой, 8 класс}
\end{center}

\problem{1} Есть тридцать карточек, на каждой написано по одному числу: на десяти карточках~–~$a$,  на десяти других~–~$b$ и на десяти оставшихся~–~$c$ (числа  различны). Известно, что к любым пяти карточкам можно подобрать ещё пять так, что сумма чисел на этих десяти карточках будет равна нулю. Докажите, что~одно из~чисел~$a, b, c$ равно нулю.

\problem{2} Вокруг стола стола пустили пакет с орешками. Первый взял один орешек, второй — 2, третий — 3 и так далее: каждый следующий брал на 1 орешек больше. Известно, что на втором круге было взято в сумме на 100 орешков больше, чем на первом. Сколько человек сидело за столом?

% \problem{2} Натуральное число разрешено увеличить на любое целое число процентов от 1 до 100, если при этом получаем натуральное число. Найдите наименьшее натуральное число, которое нельзя при помощи таких операций получить из~числа 1.

% \problem{3} Найти сумму $1^2 - 2^2 + 3^2 - 4^2 + 5^2 + \ldots - 2018^2$.

\problem{3} В кружке рукоделия, где занимается Валя, более 93\% участников~—~девочки. Какое наименьшее число участников может быть в таком кружке?

\problem{4} Произведение 2018 целых чисел равно 1. Может ли их сумма оказаться равной~0?

% \problem{4} Можно ли все натуральные числа от~1 до~9 записать в~клетки таблицы~$3\times3$ так, чтобы сумма в~любых двух соседних (по~вертикали или горизонтали) клетках равнялось простому числу?

\problem{5} На доске написано 2018 нулей и 2019 единиц. Женя стирает 2 числа и, если они были одинаковы, дописывает к оставшимся один ноль, а~если разные — единицу. Потом Женя повторяет эту операцию снова, потом ещё и~так далее. В~результате на~доске останется только одно число. Что это за~число?

\problem{6} Докажите, что в~любой компании людей найдутся 2~человека, имеющие равное число знакомых в этой компании (если $A$~знаком с~$B$, то~и $B$~знаком с~$A$).

\problem{7} Три колокола начинают бить одновременно. Интервалы между ударами колоколов соответственно составляют $\cfrac43$~секунды, $\cfrac53$~секунды и $2$~секунды. Совпавшие по времени удары воспринимаются за~один. Сколько ударов будет услышано за 1~минуту, включая первый и последний удары?

\problem{8} Восемь одинаковых момент расположены по кругу. Известно, что три из~них~— фальшивые, и они расположены рядом друг с~другом. Вес фальшивой монеты отличается от~веса настоящей. Все фальшивые монеты весят одинаково, но неизвестно, тяжелее или легче фальшивая монета настоящей. Покажите, что за~3~взвешивания на~чашечных весах без~гирь можно определить все фальшивые монеты.

\end{document}

\begin{document}

\setdate{20~марта~2021}
\setclass{10«АБ»}

\addpersonalvariant{Михаил Бурмистров}

\tasknumber{1}%
\task{%
    Определите КПД (оставив ответ точным в виде нескоратимой дроби) цикла 1231, рабочим телом которого является идеальный одноатомный газ, если
    \begin{itemize}
        \item 12 — изохорический нагрев в шесть раз,
        \item 23 — изобарическое расширение, при котором температура растёт в три раза,
        \item 31 — процесс, график которого в $PV$-координатах является отрезком прямой.
    \end{itemize}
    Бонус: замените цикл 1231 циклом, в котором 12 — изохорический нагрев в шесть раз, 23 — процесс, график которого в $PV$-координатах является отрезком прямой, 31 — изобарическое охлаждение, при котором температура падает в шесть раз.
}
\answer{%
    \begin{align*}
    A_{12} &= 0, \Delta U_{12} > 0, \implies Q_{12} = A_{12} + \Delta U_{12} > 0.
    \\
    A_{23} &> 0, \Delta U_{23} > 0, \implies Q_{23} = A_{23} + \Delta U_{23} > 0, \\
    A_{31} &= 0, \Delta U_{31} < 0, \implies Q_{31} = A_{31} + \Delta U_{31} < 0.
    \\
    P_1V_1 &= \nu R T_1, P_2V_2 = \nu R T_2, P_3V_3 = \nu R T_3 \text{ — уравнения состояния идеального газа}, \\
    &\text{Пусть $P_0$, $V_0$, $T_0$ — давление, объём и температура в точке 1 (минимальные во всём цикле):} \\
    P_1 &= P_0, P_2 = P_3, V_1 = V_2 = V_0, \text{остальные соотношения нужно считать} \\
    T_2 &= 6T_1 = 6T_0 \text{(по условию)} \implies \frac{P_2}{P_1} = \frac{P_2V_0}{P_1V_0} = \frac{P_2 V_2}{P_1 V_1}= \frac{\nu R T_2}{\nu R T_1} = \frac{T_2}{T_1} = 6 \implies P_2 = 6 P_1 = 6 P_0, \\
    T_3 &= 3T_2 = 18T_0 \text{(по условию)} \implies \frac{V_3}{V_2} = \frac{P_3V_3}{P_2V_2}= \frac{\nu R T_3}{\nu R T_2} = \frac{T_3}{T_2} = 3 \implies V_3 = 3 V_2 = 3 V_0.
    \\
    A_\text{цикл} &= \frac 12 (3P_0 - P_0)(6V_0 - V_0) = \frac 12 \cdot 10 \cdot P_0V_0, \\
    A_{23} &= 6P_0 \cdot (3V_0 - V_0) = 12P_0V_0, \\
    \Delta U_{23} &= \frac 32 \nu R T_3 - \frac 32 \nu R T_2 = \frac 32 P_3 V_3 - \frac 32 P_2 V_2 = \frac 32 \cdot 6 P_0 \cdot 3 V_0 -  \frac 32 \cdot 6 P_0 \cdot V_0 = \frac 32 \cdot 12 \cdot P_0V_0, \\
    \Delta U_{12} &= \frac 32 \nu R T_2 - \frac 32 \nu R T_1 = \frac 32 P_2 V_2 - \frac 32 P_1 V_1 = \frac 32 \cdot 6 P_0V_0 - \frac 32 P_0V_0 = \frac 32 \cdot 5 \cdot P_0V_0.
    \\
    \eta &= \frac{A_\text{цикл}}{Q_+} = \frac{A_\text{цикл}}{Q_{12} + Q_{23}}  = \frac{A_\text{цикл}}{A_{12} + \Delta U_{12} + A_{23} + \Delta U_{23}} =  \\
     &= \frac{\frac 12 \cdot 10 \cdot P_0V_0}{0 + \frac 32 \cdot 5 \cdot P_0V_0 + 12P_0V_0 + \frac 32 \cdot 12 \cdot P_0V_0} = \frac{\frac 12 \cdot 10}{\frac 32 \cdot 5 + 12 + \frac 32 \cdot 12} = \frac2{15} \approx 0.133.
    \end{align*}


        График процесса не в масштабе (эта часть пока не готова и сделать автоматически аккуратно сложно), но с верными подписями (а для решения этого достаточно):

        \begin{tikzpicture}[thick]
            \draw[-{Latex}] (0, 0) -- (0, 7) node[above left] {$P$};
            \draw[-{Latex}] (0, 0) -- (10, 0) node[right] {$V$};

            \draw[dashed] (0, 2) node[left] {$P_1 = P_0$} -| (3, 0) node[below] {$V_1 = V_2 = V_0$};
            \draw[dashed] (0, 6) node[left] {$P_2 = P_3 = 6P_0$} -| (9, 0) node[below] {$V_3 = 3V_0$};

            \draw (3, 2) node[above left]{1} node[below left]{$T_1 = T_0$}
                   (3, 6) node[below left]{2} node[above]{$T_2 = 6T_0$}
                   (9, 6) node[above right]{3} node[below right]{$T_3 = 18T_0$};
            \draw[midar] (3, 2) -- (3, 6);
            \draw[midar] (3, 6) -- (9, 6);
            \draw[midar] (9, 6) -- (3, 2);
        \end{tikzpicture}

        Решение бонуса:
        \begin{align*}
            A_{12} &= 0, \Delta U_{12} > 0, \implies Q_{12} = A_{12} + \Delta U_{12} > 0, \\
            A_{23} &> 0, \Delta U_{23} \text{ — ничего нельзя сказать, нужно исследовать отдельно}, \\
            A_{31} &< 0, \Delta U_{31} < 0, \implies Q_{31} = A_{31} + \Delta U_{31} < 0.
            \\
        \end{align*}

        Уравнения состояния идеального газа для точек 1, 2, 3: $P_1V_1 = \nu R T_1, P_2V_2 = \nu R T_2, P_3V_3 = \nu R T_3$.
        Пусть $P_0$, $V_0$, $T_0$ — давление, объём и температура в точке 1 (минимальные во всём цикле).

        12 --- изохора, $\frac{P_1V_1}{T_1} = \nu R = \frac{P_2V_2}{T_2}, V_2=V_1=V_0 \implies \frac{P_1}{T_1} =  \frac{P_2}{T_2} \implies P_2 = P_1 \frac{T_2}{T_1} = 6P_0$,

        31 --- изобара, $\frac{P_1V_1}{T_1} = \nu R = \frac{P_3V_3}{T_3}, P_3=P_1=P_0 \implies \frac{V_3}{T_3} =  \frac{V_1}{T_1} \implies V_3 = V_1 \frac{T_3}{T_1} = 6V_0$,

        Таким образом, используя новые обозначения, состояния газа в точках 1, 2 и 3 описываются макропараметрами $(P_0, V_0, T_0), (6P_0, V_0, 6T_0), (P_0, 6V_0, 6T_0)$ соответственно.

        \begin{tikzpicture}[thick]
            \draw[-{Latex}] (0, 0) -- (0, 7) node[above left] {$P$};
            \draw[-{Latex}] (0, 0) -- (10, 0) node[right] {$V$};

            \draw[dashed] (0, 2) node[left] {$P_1 = P_3 = P_0$} -| (9, 0) node[below] {$V_3 = 6V_0$};
            \draw[dashed] (0, 6) node[left] {$P_2 = 6P_0$} -| (3, 0) node[below] {$V_1 = V_2 = V_0$};

            \draw[dashed] (0, 5) node[left] {$P$} -| (4.5, 0) node[below] {$V$};
            \draw[dashed] (0, 4.6) node[left] {$P'$} -| (5.1, 0) node[below] {$V'$};

            \draw (3, 2) node[above left]{1} node[below left]{$T_1 = T_0$}
                   (3, 6) node[below left]{2} node[above]{$T_2 = 6T_0$}
                   (9, 2) node[above right]{3} node[below right]{$T_3 = 6T_0$};
            \draw[midar] (3, 2) -- (3, 6);
            \draw[midar] (3, 6) -- (9, 2);
            \draw[midar] (9, 2) -- (3, 2);
            \draw   (4.5, 5) node[above right]{$T$} (5.1, 4.6) node[above right]{$T'$};
        \end{tikzpicture}


        Теперь рассмотрим отдельно процесс 23, к остальному вернёмся позже.
        Уравнение этой прямой в $PV$-координатах: $P(V) = 7P_0 - \frac{P_0}{V_0} V$.
        Это значит, что при изменении объёма на $\Delta V$ давление изменится на $\Delta P = - \frac{P_0}{V_0} \Delta V$, обратите внимание на знак.

        Рассмотрим произвольную точку в процессе 23 и дадим процессу ещё немного свершиться, при этом объём изменится на $\Delta V$, давление — на $\Delta P$, температура (иначе бы была гипербола, а не прямая) — на $\Delta T$,
        т.е.
        из состояния $(P, V, T)$ мы перешли в $(P', V', T')$, причём  $P' = P + \Delta P, V' = V + \Delta V, T' = T + \Delta T$.

        При этом изменится внутренняя энергия:
        \begin{align*}
        \Delta U
            &= U' - U = \frac 32 \nu R T' - \frac 32 \nu R T = \frac 32 (P+\Delta P) (V+\Delta V) - \frac 32 PV\\
            &= \frac 32 ((P+\Delta P) (V+\Delta V) - PV) = \frac 32 (P\Delta V + V \Delta P + \Delta P \Delta V).
        \end{align*}

        Рассмотрим малые изменения объёма, тогда и изменение давления будем малым (т.к.
        $\Delta P = - \frac{P_0}{V_0} \Delta V$),
        а третьим слагаемым в выражении для $\Delta U$  можно пренебречь по сравнению с двумя другими:
        два первых это малые величины, а третье — произведение двух малых.
        Тогда $\Delta U = \frac 32 (P\Delta V + V \Delta P)$.

        Работа газа при этом малом изменении объёма — это площадь трапеции (тут ещё раз пренебрегли малым слагаемым):
        $$A = \frac{P + P'}2 \Delta V = \cbr{P + \frac{\Delta P}2} \Delta V = P \Delta V.$$

        Подведённое количество теплоты, используя первое начало термодинамики, будет равно
        \begin{align*}
        Q
            &= \frac 32 (P\Delta V + V \Delta P) + P \Delta V =  \frac 52 P\Delta V + \frac 32 V \Delta P = \\
            &= \frac 52 P\Delta V + \frac 32 V \cdot \cbr{- \frac{P_0}{V_0} \Delta V} = \frac{\Delta V}2 \cdot \cbr{5P - \frac{P_0}{V_0} V} = \\
            &= \frac{\Delta V}2 \cdot \cbr{5 \cdot \cbr{7P_0 - \frac{P_0}{V_0} V} - \frac{P_0}{V_0} V}
             = \frac{\Delta V \cdot P_0}2 \cdot \cbr{5 \cdot 7 - 8\frac V{V_0}}.
        \end{align*}

        Таком образом, знак количества теплоты $Q$ на участке 23 зависит от конкретного значения $V$:
        \begin{itemize}
            \item $\Delta V > 0$ на всём участке 23, поскольку газ расширяется,
            \item $P > 0$ — всегда, у нас идеальный газ, удары о стенки сосуда абсолютно упругие, а молекулы не взаимодействуют и поэтому давление только положительно,
            \item если $5 \cdot 7 - 8\frac V{V_0} > 0$ — тепло подводят, если же меньше нуля — отводят.
        \end{itemize}
        Решая последнее неравенство, получаем конкретное значение $V^*$: при $V < V^*$ тепло подводят, далее~— отводят.
        Тут *~--- некоторая точка между точками 2 и 3, конкретные значения надо досчитать:
        $$V^* = V_0 \cdot \frac{5 \cdot 7}8 = \frac{35}8 \cdot V_0 \implies P^* = 7P_0 - \frac{P_0}{V_0} V^* = \ldots = \frac{21}8 \cdot P_0.$$

        Т.е.
        чтобы вычислить $Q_+$, надо сложить количества теплоты на участке 12 и лишь части участка 23 — участке 2*,
        той его части где это количество теплоты положительно.
        Имеем: $Q_+ = Q_{12} + Q_{2*}$.

        Теперь возвращаемся к циклу целиком и получаем:
        \begin{align*}
        A_\text{цикл} &= \frac 12 \cdot (6P_0 - P_0) \cdot (6V_0 - V_0) = \frac{25}2 \cdot P_0V_0, \\
        A_{2*} &= \frac{P^* + 6P_0}2 \cdot (V^* - V_0)
            = \frac{\frac{21}8 \cdot P_0 + 6P_0}2 \cdot \cbr{\frac{35}8 \cdot V_0 - V_0}
            = \ldots = \frac{1863}{128} \cdot P_0 V_0, \\
        \Delta U_{2*} &= \frac 32 \nu R T^* - \frac 32 \nu R T_2 = \frac 32 (P^*V^* - P_0 \cdot 6V_0)
            = \frac 32 \cbr{\frac{21}8 \cdot P_0 \cdot \frac{35}8 \cdot V_0 - P_0 \cdot 6V_0}
            = \frac{1053}{128} \cdot P_0 V_0, \\
        \Delta U_{12} &= \frac 32 \nu R T_2 - \frac 32 \nu R T_1 = \frac 32 (6P_0V_0 - P_0V_0) = \ldots = \frac{15}2 \cdot P_0 V_0, \\
        \eta &= \frac{A_\text{цикл}}{Q_+} = \frac{A_\text{цикл}}{Q_{12} + Q_{2*}}
            = \frac{A_\text{цикл}}{A_{12} + \Delta U_{12} + A_{2*} + \Delta U_{2*}} = \\
            &= \frac{\frac{25}2 \cdot P_0V_0}{0 + \frac{15}2 \cdot P_0 V_0 + \frac{1863}{128} \cdot P_0 V_0 + \frac{1053}{128} \cdot P_0 V_0}
             = \frac{A_bonus_cycle:LaTeX}{\frac{15}2 + \frac{1863}{128} + \frac{1053}{128}}
             = \frac{400}{969} \leftarrow \text{вжух и готово!}
        \end{align*}
}
\solutionspace{360pt}

\tasknumber{2}%
\task{%
    Определите КПД (оставив ответ точным в виде нескоратимой дроби) цикла 1231, рабочим телом которого является идеальный одноатомный газ, если
    \begin{itemize}
        \item 12 — изобарическое расширение,
        \item 23 — процесс, график которого в $PV$-координатах является отрезком прямой, а объём уменьшается в три раза,
        \item 31 — изохорический нагрев с увеличением давления в шесть раз,
    \end{itemize}
}
\answer{%
    \begin{align*}
    A_{12} &> 0, \Delta U_{12} > 0, \implies Q_{12} = A_{12} + \Delta U_{12} > 0.
    \\
    A_{23} &< 0, \Delta U_{23} < 0, \implies Q_{23} = A_{23} + \Delta U_{23} < 0, \\
    A_{31} &= 0, \Delta U_{31} > 0, \implies Q_{31} = A_{31} + \Delta U_{31} > 0.
    \\
    P_1V_1 &= \nu R T_1, P_2V_2 = \nu R T_2, P_3V_3 = \nu R T_3 \text{ — уравнения состояния идеального газа}, \\
    &\text{Пусть $P_0$, $V_0$, $T_0$ — давление, объём и температура в точке 3 (минимальные во всём цикле):} \\
    P_3 &= P_0, P_1 = P_2 = 6P_0, V_1 = V_3 = V_0, V_2 = 3V_3 = 3V_0 \\
    A_\text{цикл} &= \frac 12 (P_2-P_1)(V_1-V_2) = \frac 12 (6P_0 - P_0)(3V_0 - V_0) = \frac 12 \cdot 10 \cdot P_0V_0, \\
    A_{12} &= 3P_0 \cdot (6V_0 - V_0) = 15P_0V_0, \\
    \Delta U_{12} &= \frac 32 \nu R T_2 - \frac 32 \nu R T_1 = \frac 32 P_2 V_2 - \frac 32 P_1 V_1 = \frac 32 \cdot 6 P_0 \cdot 3 V_0 -  \frac 32 \cdot 3 P_0 \cdot V_0 = \frac 32 \cdot 15 \cdot P_0V_0, \\
    \Delta U_{31} &= \frac 32 \nu R T_1 - \frac 32 \nu R T_3 = \frac 32 P_1 V_1 - \frac 32 P_3 V_3 = \frac 32 \cdot 6 P_0V_0 - \frac 32 P_0V_0 = \frac 32 \cdot 5 \cdot P_0V_0.
    \\
    \eta &= \frac{A_\text{цикл}}{Q_+} = \frac{A_\text{цикл}}{Q_{12} + Q_{31}}  = \frac{A_\text{цикл}}{A_{12} + \Delta U_{12} + A_{31} + \Delta U_{31}} =  \\
     &= \frac{\frac 12 \cdot 10 \cdot P_0V_0}{15P_0V_0 + \frac 32 \cdot 15 \cdot P_0V_0 + 0 + \frac 32 \cdot 5 \cdot P_0V_0} = \frac{\frac 12 \cdot 10}{15 + \frac 32 \cdot 15 + \frac 32 \cdot 5} = \frac19 \approx 0{,}111.
    \end{align*}
}

\end{document}
% autogenerated
