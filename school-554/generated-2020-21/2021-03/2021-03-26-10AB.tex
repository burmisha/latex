\setdate{26~марта~2021}
\setclass{10«АБ»}

\addpersonalvariant{Михаил Бурмистров}

\tasknumber{1}%
\task{%
    Сколько молекул водяного пара содержится в сосуде объёмом $3\,\text{л}$ при температуре $90\celsius$,
    и влажности воздуха $35\%$?
}
\answer{%
    Уравнение состояния идеального газа (и учтём, что $R = N_A \cdot k$,
    это чуть упростит выячичления, но вовсе не обязательно это делать):
    $$
        PV = \nu RT = \frac N{N_A} RT \implies N = PV \cdot \frac{N_A}{RT}=  \frac{PV}{kT}
    $$
    Плотность насыщенного водяного пара при $90\celsius$ ищем по таблице: $P_{\text{нас.
    пара 90} \celsius} = 70{,}100\,\text{кПа}.$

    Получаем плотность пара в сосуде $\varphi = \frac P{P_{\text{нас.
    пара 90} \celsius}} \implies P = \varphi P_{\text{нас.
    пара 90} \celsius}.$

    И подставляем в ответ (по сути, его можно было получить быстрее из формул $P = nkT, n = \frac NV$):
    $$
        N = \frac{\varphi \cdot P_{\text{нас.
        пара 90} \celsius} \cdot V}{kT}
         = \frac{0{,}35 \cdot 70{,}100\,\text{кПа} \cdot 3\,\text{л}}{1{,}38 \cdot 10^{-23}\,\frac{\text{Дж}}{\text{К}} \cdot 363\,\text{К}}
         \approx 147 \cdot 10^{20}.
    $$

    Другой вариант решения (через плотности) приводит в результату:
    $$
        N = N_A \nu = N_A \cdot \frac m{\mu}
          = N_A \frac{\rho V}{\mu}
          = N_A \frac{\varphi \cdot \rho_{\text{нас.
          пара 90} \celsius} \cdot V}{\mu}
          = 6{,}02 \cdot 10^{23}\,\frac{1}{\text{моль}} \cdot \frac{0{,}35 \cdot 424\,\frac{\text{г}}{\text{м}^{3}} \cdot 3\,\text{л}}{18\,\frac{\text{г}}{\text{моль}}}
          \approx 149 \cdot 10^{20}.
    $$
}
\solutionspace{160pt}

\tasknumber{2}%
\task{%
    В герметичном сосуде находится влажный воздух при температуре $25\celsius$ и относительной влажности $25\%$.
    \begin{enumerate}
        \item Чему равно парциальное давление насыщенного водяного пара при этой температуре?
        \item Чему равно парциальное давление водяного пара?
        \item Определите точку росы этого пара?
        \item Каким станет парциальное давление водяного пара, если сосуд нагреть до $80\celsius$?
        \item Чему будет равна относительная влажность воздуха, если сосуд нагреть до $80\celsius$?
        \item Получите ответ на предыдущий вопрос, используя плотности, а не давления.
    \end{enumerate}
}
\answer{%
    Парциальное давление насыщенного водяного пара при $25\celsius$ ищем по таблице: $$P_{\text{нас.
    пара 25} \celsius} = 3{,}170\,\text{кПа}.$$

    Парциальное давление водяного пара
    $$P_\text{пара 1} = \varphi_1 \cdot P_{\text{нас.
    пара 25} \celsius} = 0{,}25 \cdot 3{,}170\,\text{кПа} = 0{,}7925\,\text{кПа}.$$

    Точку росы определяем по таблице: при какой температуре пар с давлением $P_\text{пара 1} = 0{,}7925\,\text{кПа}$ станет насыщенным: $3{,}6\celsius$.

    После нагрева парциальное давление пара возрастёт:
    $$
        \frac{P_\text{пара 1} \cdot V}{T_1} = \nu R = \frac{P_\text{пара 2} \cdot V}{T_2}
        \implies P_\text{пара 2} = P_\text{пара 1} \cdot \frac{T_2}{T_1} = 0{,}7925\,\text{кПа} \cdot \frac{353\,\text{К}}{298\,\text{К}} \approx 0{,}9388\,\text{кПа}.
    $$

    Парциальное давление насыщенного водяного пара при $80\celsius$ ищем по таблице: $P_{\text{нас.
    пара 80} \celsius} = 47{,}300\,\text{кПа}$.
    Теперь определяем влажность:
    $$
        \varphi_2 = \frac{P_\text{пара 2}}{P_{\text{нас.
        пара 80} \celsius}} = \frac{0{,}9388\,\text{кПа}}{47{,}300\,\text{кПа}} \approx 0{,}020 = 2{,}0\%.
    $$

    Или же выражаем то же самое через плотности (плотность не изменяется при изохорном нагревании $\rho_1 =\rho_2 = \rho$, в отличие от давления):
    $$
        \varphi_2 = \frac{\rho}{\rho_{\text{нас.
        пара 80} \celsius}} = \frac{\varphi_1\rho_{\text{нас.
        пара 25} \celsius}}{\rho_{\text{нас.
        пара 80} \celsius}}
        = \frac{0{,}25 \cdot 23\,\frac{\text{г}}{\text{м}^{3}}}{293\,\frac{\text{г}}{\text{м}^{3}}} \approx 0{,}020 = 2{,}0\%.
    $$
    Сравните 2 последних результата.
}
\solutionspace{200pt}

\tasknumber{3}%
\task{%
    Закрытый сосуд объёмом $15\,\text{л}$ заполнен сухим воздухом при давлении $100\,\text{кПа}$ и температуре $30\celsius$.
    Каким станет давление в сосуде, если в него налить $10\,\text{г}$ воды и нагреть содержимое сосуда до $90\celsius$?
}
\answer{%
    Конечное давление газа в сосуде складывается по закону Дальтона из давления нагретого сухого воздуха $P'_\text{воздуха}$ и
    давления насыщенного пара $P_\text{пара}$:
    $$P' = P'_\text{воздуха} + P_\text{пара}.$$

    Сперва определим новое давление сухого воздуха из уравнения состояния идеального газа:
    $$
        \frac{P'_\text{воздуха} \cdot V}{T'} = \nu R = \frac{P \cdot V}{T}
        \implies P'_\text{воздуха} = P \cdot \frac{T'}{T} = 100\,\text{кПа} \cdot \frac{363\,\text{К}}{303\,\text{К}} \approx 120\,\text{кПа}.
    $$

    Чтобы найти давление пара, нужно понять, будет ли он насыщенным после нагрева или нет.

    Плотность насыщенного пара при температуре $90\celsius$ равна $424\,\frac{\text{г}}{\text{м}^{3}}$, тогда для того,
    чтобы весь сосуд был заполнен насыщенным водяным паром нужно
    $m_\text{н.
    п.} = \rho_\text{н.
    п.
    90 $\celsius$} \cdot V = 424\,\frac{\text{г}}{\text{м}^{3}} \cdot 15\,\text{л} \approx 6{,}4\,\text{г}$ воды.
    Сравнивая эту массу с массой воды из условия, получаем массу жидкости, которая испарится: $m_\text{пара} = 6{,}4\,\text{г}$.
    Осталось определить давление этого пара:
    $$P_\text{пара} = \frac{m_\text{пара}RT'}{\mu V} = \frac{6{,}4\,\text{г} \cdot 8{,}31\,\frac{\text{Дж}}{\text{моль}\cdot\text{К}} \cdot 363\,\text{К}}{18\,\frac{\text{г}}{\text{моль}} \cdot 15\,\text{л}} \approx 72\,\text{кПа}.$$

    Получаем ответ: $P'_\text{пара} = 191{,}3\,\text{кПа}$.

    Другой вариант решения для давления пара:
    Определим давление пара, если бы вся вода испарилась (что не факт):
    $$P_\text{max} = \frac{mRT'}{\mu V} = \frac{10\,\text{г} \cdot 8{,}31\,\frac{\text{Дж}}{\text{моль}\cdot\text{К}} \cdot 363\,\text{К}}{18\,\frac{\text{г}}{\text{моль}} \cdot 15\,\text{л}} \approx 112\,\text{кПа}.$$
    Сравниваем это давление с давлением насыщенного пара при этой температуре $P_\text{н.
    п.
    90 $\celsius$} = 70{,}100\,\text{кПа}$:
    если у нас получилось меньше табличного значения,
    то вся вода испарилась, если же больше — испарилась лишь часть, а пар является насыщенным.
    Отсюда сразу получаем давление пара: $P'_\text{пара} = 70{,}1\,\text{кПа}$.
    Сравните этот результат с первым вариантом решения.

    Тут получаем ответ: $P'_\text{пара} = 189{,}9\,\text{кПа}$.
}
\solutionspace{150pt}

\tasknumber{4}%
\task{%
    Напротив физических величин запишите определение, обозначение и единицы измерения в системе СИ (если есть):
    \begin{enumerate}
        \item относительная влажность,
        \item динамическое равновесие.
    \end{enumerate}
}

\variantsplitter

\addpersonalvariant{Ирина Ан}

\tasknumber{1}%
\task{%
    Сколько молекул водяного пара содержится в сосуде объёмом $3\,\text{л}$ при температуре $60\celsius$,
    и влажности воздуха $55\%$?
}
\answer{%
    Уравнение состояния идеального газа (и учтём, что $R = N_A \cdot k$,
    это чуть упростит выячичления, но вовсе не обязательно это делать):
    $$
        PV = \nu RT = \frac N{N_A} RT \implies N = PV \cdot \frac{N_A}{RT}=  \frac{PV}{kT}
    $$
    Плотность насыщенного водяного пара при $60\celsius$ ищем по таблице: $P_{\text{нас.
    пара 60} \celsius} = 19{,}900\,\text{кПа}.$

    Получаем плотность пара в сосуде $\varphi = \frac P{P_{\text{нас.
    пара 60} \celsius}} \implies P = \varphi P_{\text{нас.
    пара 60} \celsius}.$

    И подставляем в ответ (по сути, его можно было получить быстрее из формул $P = nkT, n = \frac NV$):
    $$
        N = \frac{\varphi \cdot P_{\text{нас.
        пара 60} \celsius} \cdot V}{kT}
         = \frac{0{,}55 \cdot 19{,}900\,\text{кПа} \cdot 3\,\text{л}}{1{,}38 \cdot 10^{-23}\,\frac{\text{Дж}}{\text{К}} \cdot 333\,\text{К}}
         \approx 71 \cdot 10^{20}.
    $$

    Другой вариант решения (через плотности) приводит в результату:
    $$
        N = N_A \nu = N_A \cdot \frac m{\mu}
          = N_A \frac{\rho V}{\mu}
          = N_A \frac{\varphi \cdot \rho_{\text{нас.
          пара 60} \celsius} \cdot V}{\mu}
          = 6{,}02 \cdot 10^{23}\,\frac{1}{\text{моль}} \cdot \frac{0{,}55 \cdot 130\,\frac{\text{г}}{\text{м}^{3}} \cdot 3\,\text{л}}{18\,\frac{\text{г}}{\text{моль}}}
          \approx 72 \cdot 10^{20}.
    $$
}
\solutionspace{160pt}

\tasknumber{2}%
\task{%
    В герметичном сосуде находится влажный воздух при температуре $20\celsius$ и относительной влажности $40\%$.
    \begin{enumerate}
        \item Чему равно парциальное давление насыщенного водяного пара при этой температуре?
        \item Чему равно парциальное давление водяного пара?
        \item Определите точку росы этого пара?
        \item Каким станет парциальное давление водяного пара, если сосуд нагреть до $70\celsius$?
        \item Чему будет равна относительная влажность воздуха, если сосуд нагреть до $70\celsius$?
        \item Получите ответ на предыдущий вопрос, используя плотности, а не давления.
    \end{enumerate}
}
\answer{%
    Парциальное давление насыщенного водяного пара при $20\celsius$ ищем по таблице: $$P_{\text{нас.
    пара 20} \celsius} = 2{,}340\,\text{кПа}.$$

    Парциальное давление водяного пара
    $$P_\text{пара 1} = \varphi_1 \cdot P_{\text{нас.
    пара 20} \celsius} = 0{,}40 \cdot 2{,}340\,\text{кПа} = 0{,}9360\,\text{кПа}.$$

    Точку росы определяем по таблице: при какой температуре пар с давлением $P_\text{пара 1} = 0{,}9360\,\text{кПа}$ станет насыщенным: $6{,}0\celsius$.

    После нагрева парциальное давление пара возрастёт:
    $$
        \frac{P_\text{пара 1} \cdot V}{T_1} = \nu R = \frac{P_\text{пара 2} \cdot V}{T_2}
        \implies P_\text{пара 2} = P_\text{пара 1} \cdot \frac{T_2}{T_1} = 0{,}9360\,\text{кПа} \cdot \frac{343\,\text{К}}{293\,\text{К}} \approx 1{,}0957\,\text{кПа}.
    $$

    Парциальное давление насыщенного водяного пара при $70\celsius$ ищем по таблице: $P_{\text{нас.
    пара 70} \celsius} = 31\,\text{кПа}$.
    Теперь определяем влажность:
    $$
        \varphi_2 = \frac{P_\text{пара 2}}{P_{\text{нас.
        пара 70} \celsius}} = \frac{1{,}0957\,\text{кПа}}{31\,\text{кПа}} \approx 0{,}035 = 3{,}5\%.
    $$

    Или же выражаем то же самое через плотности (плотность не изменяется при изохорном нагревании $\rho_1 =\rho_2 = \rho$, в отличие от давления):
    $$
        \varphi_2 = \frac{\rho}{\rho_{\text{нас.
        пара 70} \celsius}} = \frac{\varphi_1\rho_{\text{нас.
        пара 20} \celsius}}{\rho_{\text{нас.
        пара 70} \celsius}}
        = \frac{0{,}40 \cdot 17{,}30\,\frac{\text{г}}{\text{м}^{3}}}{198\,\frac{\text{г}}{\text{м}^{3}}} \approx 0{,}035 = 3{,}5\%.
    $$
    Сравните 2 последних результата.
}
\solutionspace{200pt}

\tasknumber{3}%
\task{%
    Закрытый сосуд объёмом $10\,\text{л}$ заполнен сухим воздухом при давлении $100\,\text{кПа}$ и температуре $30\celsius$.
    Каким станет давление в сосуде, если в него налить $20\,\text{г}$ воды и нагреть содержимое сосуда до $80\celsius$?
}
\answer{%
    Конечное давление газа в сосуде складывается по закону Дальтона из давления нагретого сухого воздуха $P'_\text{воздуха}$ и
    давления насыщенного пара $P_\text{пара}$:
    $$P' = P'_\text{воздуха} + P_\text{пара}.$$

    Сперва определим новое давление сухого воздуха из уравнения состояния идеального газа:
    $$
        \frac{P'_\text{воздуха} \cdot V}{T'} = \nu R = \frac{P \cdot V}{T}
        \implies P'_\text{воздуха} = P \cdot \frac{T'}{T} = 100\,\text{кПа} \cdot \frac{353\,\text{К}}{303\,\text{К}} \approx 117\,\text{кПа}.
    $$

    Чтобы найти давление пара, нужно понять, будет ли он насыщенным после нагрева или нет.

    Плотность насыщенного пара при температуре $80\celsius$ равна $293\,\frac{\text{г}}{\text{м}^{3}}$, тогда для того,
    чтобы весь сосуд был заполнен насыщенным водяным паром нужно
    $m_\text{н.
    п.} = \rho_\text{н.
    п.
    80 $\celsius$} \cdot V = 293\,\frac{\text{г}}{\text{м}^{3}} \cdot 10\,\text{л} \approx 2{,}9\,\text{г}$ воды.
    Сравнивая эту массу с массой воды из условия, получаем массу жидкости, которая испарится: $m_\text{пара} = 2{,}9\,\text{г}$.
    Осталось определить давление этого пара:
    $$P_\text{пара} = \frac{m_\text{пара}RT'}{\mu V} = \frac{2{,}9\,\text{г} \cdot 8{,}31\,\frac{\text{Дж}}{\text{моль}\cdot\text{К}} \cdot 353\,\text{К}}{18\,\frac{\text{г}}{\text{моль}} \cdot 10\,\text{л}} \approx 47\,\text{кПа}.$$

    Получаем ответ: $P'_\text{пара} = 163{,}8\,\text{кПа}$.

    Другой вариант решения для давления пара:
    Определим давление пара, если бы вся вода испарилась (что не факт):
    $$P_\text{max} = \frac{mRT'}{\mu V} = \frac{20\,\text{г} \cdot 8{,}31\,\frac{\text{Дж}}{\text{моль}\cdot\text{К}} \cdot 353\,\text{К}}{18\,\frac{\text{г}}{\text{моль}} \cdot 10\,\text{л}} \approx 330\,\text{кПа}.$$
    Сравниваем это давление с давлением насыщенного пара при этой температуре $P_\text{н.
    п.
    80 $\celsius$} = 47{,}300\,\text{кПа}$:
    если у нас получилось меньше табличного значения,
    то вся вода испарилась, если же больше — испарилась лишь часть, а пар является насыщенным.
    Отсюда сразу получаем давление пара: $P'_\text{пара} = 47{,}3\,\text{кПа}$.
    Сравните этот результат с первым вариантом решения.

    Тут получаем ответ: $P'_\text{пара} = 163{,}8\,\text{кПа}$.
}
\solutionspace{150pt}

\tasknumber{4}%
\task{%
    Напротив физических величин запишите определение, обозначение и единицы измерения в системе СИ (если есть):
    \begin{enumerate}
        \item относительная влажность,
        \item насыщенный пар.
    \end{enumerate}
}

\variantsplitter

\addpersonalvariant{Софья Андрианова}

\tasknumber{1}%
\task{%
    Сколько молекул водяного пара содержится в сосуде объёмом $6\,\text{л}$ при температуре $50\celsius$,
    и влажности воздуха $30\%$?
}
\answer{%
    Уравнение состояния идеального газа (и учтём, что $R = N_A \cdot k$,
    это чуть упростит выячичления, но вовсе не обязательно это делать):
    $$
        PV = \nu RT = \frac N{N_A} RT \implies N = PV \cdot \frac{N_A}{RT}=  \frac{PV}{kT}
    $$
    Плотность насыщенного водяного пара при $50\celsius$ ищем по таблице: $P_{\text{нас.
    пара 50} \celsius} = 12{,}300\,\text{кПа}.$

    Получаем плотность пара в сосуде $\varphi = \frac P{P_{\text{нас.
    пара 50} \celsius}} \implies P = \varphi P_{\text{нас.
    пара 50} \celsius}.$

    И подставляем в ответ (по сути, его можно было получить быстрее из формул $P = nkT, n = \frac NV$):
    $$
        N = \frac{\varphi \cdot P_{\text{нас.
        пара 50} \celsius} \cdot V}{kT}
         = \frac{0{,}30 \cdot 12{,}300\,\text{кПа} \cdot 6\,\text{л}}{1{,}38 \cdot 10^{-23}\,\frac{\text{Дж}}{\text{К}} \cdot 323\,\text{К}}
         \approx 50 \cdot 10^{20}.
    $$

    Другой вариант решения (через плотности) приводит в результату:
    $$
        N = N_A \nu = N_A \cdot \frac m{\mu}
          = N_A \frac{\rho V}{\mu}
          = N_A \frac{\varphi \cdot \rho_{\text{нас.
          пара 50} \celsius} \cdot V}{\mu}
          = 6{,}02 \cdot 10^{23}\,\frac{1}{\text{моль}} \cdot \frac{0{,}30 \cdot 83\,\frac{\text{г}}{\text{м}^{3}} \cdot 6\,\text{л}}{18\,\frac{\text{г}}{\text{моль}}}
          \approx 50 \cdot 10^{20}.
    $$
}
\solutionspace{160pt}

\tasknumber{2}%
\task{%
    В герметичном сосуде находится влажный воздух при температуре $30\celsius$ и относительной влажности $60\%$.
    \begin{enumerate}
        \item Чему равно парциальное давление насыщенного водяного пара при этой температуре?
        \item Чему равно парциальное давление водяного пара?
        \item Определите точку росы этого пара?
        \item Каким станет парциальное давление водяного пара, если сосуд нагреть до $80\celsius$?
        \item Чему будет равна относительная влажность воздуха, если сосуд нагреть до $80\celsius$?
        \item Получите ответ на предыдущий вопрос, используя плотности, а не давления.
    \end{enumerate}
}
\answer{%
    Парциальное давление насыщенного водяного пара при $30\celsius$ ищем по таблице: $$P_{\text{нас.
    пара 30} \celsius} = 4{,}240\,\text{кПа}.$$

    Парциальное давление водяного пара
    $$P_\text{пара 1} = \varphi_1 \cdot P_{\text{нас.
    пара 30} \celsius} = 0{,}60 \cdot 4{,}240\,\text{кПа} = 2{,}544\,\text{кПа}.$$

    Точку росы определяем по таблице: при какой температуре пар с давлением $P_\text{пара 1} = 2{,}544\,\text{кПа}$ станет насыщенным: $21{,}4\celsius$.

    После нагрева парциальное давление пара возрастёт:
    $$
        \frac{P_\text{пара 1} \cdot V}{T_1} = \nu R = \frac{P_\text{пара 2} \cdot V}{T_2}
        \implies P_\text{пара 2} = P_\text{пара 1} \cdot \frac{T_2}{T_1} = 2{,}544\,\text{кПа} \cdot \frac{353\,\text{К}}{303\,\text{К}} \approx 2{,}964\,\text{кПа}.
    $$

    Парциальное давление насыщенного водяного пара при $80\celsius$ ищем по таблице: $P_{\text{нас.
    пара 80} \celsius} = 47{,}300\,\text{кПа}$.
    Теперь определяем влажность:
    $$
        \varphi_2 = \frac{P_\text{пара 2}}{P_{\text{нас.
        пара 80} \celsius}} = \frac{2{,}964\,\text{кПа}}{47{,}300\,\text{кПа}} \approx 0{,}063 = 6{,}3\%.
    $$

    Или же выражаем то же самое через плотности (плотность не изменяется при изохорном нагревании $\rho_1 =\rho_2 = \rho$, в отличие от давления):
    $$
        \varphi_2 = \frac{\rho}{\rho_{\text{нас.
        пара 80} \celsius}} = \frac{\varphi_1\rho_{\text{нас.
        пара 30} \celsius}}{\rho_{\text{нас.
        пара 80} \celsius}}
        = \frac{0{,}60 \cdot 30{,}30\,\frac{\text{г}}{\text{м}^{3}}}{293\,\frac{\text{г}}{\text{м}^{3}}} \approx 0{,}062 = 6{,}2\%.
    $$
    Сравните 2 последних результата.
}
\solutionspace{200pt}

\tasknumber{3}%
\task{%
    Закрытый сосуд объёмом $10\,\text{л}$ заполнен сухим воздухом при давлении $100\,\text{кПа}$ и температуре $10\celsius$.
    Каким станет давление в сосуде, если в него налить $5\,\text{г}$ воды и нагреть содержимое сосуда до $90\celsius$?
}
\answer{%
    Конечное давление газа в сосуде складывается по закону Дальтона из давления нагретого сухого воздуха $P'_\text{воздуха}$ и
    давления насыщенного пара $P_\text{пара}$:
    $$P' = P'_\text{воздуха} + P_\text{пара}.$$

    Сперва определим новое давление сухого воздуха из уравнения состояния идеального газа:
    $$
        \frac{P'_\text{воздуха} \cdot V}{T'} = \nu R = \frac{P \cdot V}{T}
        \implies P'_\text{воздуха} = P \cdot \frac{T'}{T} = 100\,\text{кПа} \cdot \frac{363\,\text{К}}{283\,\text{К}} \approx 128\,\text{кПа}.
    $$

    Чтобы найти давление пара, нужно понять, будет ли он насыщенным после нагрева или нет.

    Плотность насыщенного пара при температуре $90\celsius$ равна $424\,\frac{\text{г}}{\text{м}^{3}}$, тогда для того,
    чтобы весь сосуд был заполнен насыщенным водяным паром нужно
    $m_\text{н.
    п.} = \rho_\text{н.
    п.
    90 $\celsius$} \cdot V = 424\,\frac{\text{г}}{\text{м}^{3}} \cdot 10\,\text{л} \approx 4{,}2\,\text{г}$ воды.
    Сравнивая эту массу с массой воды из условия, получаем массу жидкости, которая испарится: $m_\text{пара} = 4{,}2\,\text{г}$.
    Осталось определить давление этого пара:
    $$P_\text{пара} = \frac{m_\text{пара}RT'}{\mu V} = \frac{4{,}2\,\text{г} \cdot 8{,}31\,\frac{\text{Дж}}{\text{моль}\cdot\text{К}} \cdot 363\,\text{К}}{18\,\frac{\text{г}}{\text{моль}} \cdot 10\,\text{л}} \approx 70\,\text{кПа}.$$

    Получаем ответ: $P'_\text{пара} = 198{,}7\,\text{кПа}$.

    Другой вариант решения для давления пара:
    Определим давление пара, если бы вся вода испарилась (что не факт):
    $$P_\text{max} = \frac{mRT'}{\mu V} = \frac{5\,\text{г} \cdot 8{,}31\,\frac{\text{Дж}}{\text{моль}\cdot\text{К}} \cdot 363\,\text{К}}{18\,\frac{\text{г}}{\text{моль}} \cdot 10\,\text{л}} \approx 84\,\text{кПа}.$$
    Сравниваем это давление с давлением насыщенного пара при этой температуре $P_\text{н.
    п.
    90 $\celsius$} = 70{,}100\,\text{кПа}$:
    если у нас получилось меньше табличного значения,
    то вся вода испарилась, если же больше — испарилась лишь часть, а пар является насыщенным.
    Отсюда сразу получаем давление пара: $P'_\text{пара} = 70{,}1\,\text{кПа}$.
    Сравните этот результат с первым вариантом решения.

    Тут получаем ответ: $P'_\text{пара} = 198{,}4\,\text{кПа}$.
}
\solutionspace{150pt}

\tasknumber{4}%
\task{%
    Напротив физических величин запишите определение, обозначение и единицы измерения в системе СИ (если есть):
    \begin{enumerate}
        \item относительная влажность,
        \item насыщенный пар.
    \end{enumerate}
}

\variantsplitter

\addpersonalvariant{Владимир Артемчук}

\tasknumber{1}%
\task{%
    Сколько молекул водяного пара содержится в сосуде объёмом $15\,\text{л}$ при температуре $60\celsius$,
    и влажности воздуха $80\%$?
}
\answer{%
    Уравнение состояния идеального газа (и учтём, что $R = N_A \cdot k$,
    это чуть упростит выячичления, но вовсе не обязательно это делать):
    $$
        PV = \nu RT = \frac N{N_A} RT \implies N = PV \cdot \frac{N_A}{RT}=  \frac{PV}{kT}
    $$
    Плотность насыщенного водяного пара при $60\celsius$ ищем по таблице: $P_{\text{нас.
    пара 60} \celsius} = 19{,}900\,\text{кПа}.$

    Получаем плотность пара в сосуде $\varphi = \frac P{P_{\text{нас.
    пара 60} \celsius}} \implies P = \varphi P_{\text{нас.
    пара 60} \celsius}.$

    И подставляем в ответ (по сути, его можно было получить быстрее из формул $P = nkT, n = \frac NV$):
    $$
        N = \frac{\varphi \cdot P_{\text{нас.
        пара 60} \celsius} \cdot V}{kT}
         = \frac{0{,}80 \cdot 19{,}900\,\text{кПа} \cdot 15\,\text{л}}{1{,}38 \cdot 10^{-23}\,\frac{\text{Дж}}{\text{К}} \cdot 333\,\text{К}}
         \approx 520 \cdot 10^{20}.
    $$

    Другой вариант решения (через плотности) приводит в результату:
    $$
        N = N_A \nu = N_A \cdot \frac m{\mu}
          = N_A \frac{\rho V}{\mu}
          = N_A \frac{\varphi \cdot \rho_{\text{нас.
          пара 60} \celsius} \cdot V}{\mu}
          = 6{,}02 \cdot 10^{23}\,\frac{1}{\text{моль}} \cdot \frac{0{,}80 \cdot 130\,\frac{\text{г}}{\text{м}^{3}} \cdot 15\,\text{л}}{18\,\frac{\text{г}}{\text{моль}}}
          \approx 520 \cdot 10^{20}.
    $$
}
\solutionspace{160pt}

\tasknumber{2}%
\task{%
    В герметичном сосуде находится влажный воздух при температуре $25\celsius$ и относительной влажности $40\%$.
    \begin{enumerate}
        \item Чему равно парциальное давление насыщенного водяного пара при этой температуре?
        \item Чему равно парциальное давление водяного пара?
        \item Определите точку росы этого пара?
        \item Каким станет парциальное давление водяного пара, если сосуд нагреть до $90\celsius$?
        \item Чему будет равна относительная влажность воздуха, если сосуд нагреть до $90\celsius$?
        \item Получите ответ на предыдущий вопрос, используя плотности, а не давления.
    \end{enumerate}
}
\answer{%
    Парциальное давление насыщенного водяного пара при $25\celsius$ ищем по таблице: $$P_{\text{нас.
    пара 25} \celsius} = 3{,}170\,\text{кПа}.$$

    Парциальное давление водяного пара
    $$P_\text{пара 1} = \varphi_1 \cdot P_{\text{нас.
    пара 25} \celsius} = 0{,}40 \cdot 3{,}170\,\text{кПа} = 1{,}2680\,\text{кПа}.$$

    Точку росы определяем по таблице: при какой температуре пар с давлением $P_\text{пара 1} = 1{,}2680\,\text{кПа}$ станет насыщенным: $10{,}5\celsius$.

    После нагрева парциальное давление пара возрастёт:
    $$
        \frac{P_\text{пара 1} \cdot V}{T_1} = \nu R = \frac{P_\text{пара 2} \cdot V}{T_2}
        \implies P_\text{пара 2} = P_\text{пара 1} \cdot \frac{T_2}{T_1} = 1{,}2680\,\text{кПа} \cdot \frac{363\,\text{К}}{298\,\text{К}} \approx 1{,}5446\,\text{кПа}.
    $$

    Парциальное давление насыщенного водяного пара при $90\celsius$ ищем по таблице: $P_{\text{нас.
    пара 90} \celsius} = 70{,}100\,\text{кПа}$.
    Теперь определяем влажность:
    $$
        \varphi_2 = \frac{P_\text{пара 2}}{P_{\text{нас.
        пара 90} \celsius}} = \frac{1{,}5446\,\text{кПа}}{70{,}100\,\text{кПа}} \approx 0{,}022 = 2{,}2\%.
    $$

    Или же выражаем то же самое через плотности (плотность не изменяется при изохорном нагревании $\rho_1 =\rho_2 = \rho$, в отличие от давления):
    $$
        \varphi_2 = \frac{\rho}{\rho_{\text{нас.
        пара 90} \celsius}} = \frac{\varphi_1\rho_{\text{нас.
        пара 25} \celsius}}{\rho_{\text{нас.
        пара 90} \celsius}}
        = \frac{0{,}40 \cdot 23\,\frac{\text{г}}{\text{м}^{3}}}{424\,\frac{\text{г}}{\text{м}^{3}}} \approx 0{,}022 = 2{,}2\%.
    $$
    Сравните 2 последних результата.
}
\solutionspace{200pt}

\tasknumber{3}%
\task{%
    Закрытый сосуд объёмом $15\,\text{л}$ заполнен сухим воздухом при давлении $100\,\text{кПа}$ и температуре $20\celsius$.
    Каким станет давление в сосуде, если в него налить $30\,\text{г}$ воды и нагреть содержимое сосуда до $80\celsius$?
}
\answer{%
    Конечное давление газа в сосуде складывается по закону Дальтона из давления нагретого сухого воздуха $P'_\text{воздуха}$ и
    давления насыщенного пара $P_\text{пара}$:
    $$P' = P'_\text{воздуха} + P_\text{пара}.$$

    Сперва определим новое давление сухого воздуха из уравнения состояния идеального газа:
    $$
        \frac{P'_\text{воздуха} \cdot V}{T'} = \nu R = \frac{P \cdot V}{T}
        \implies P'_\text{воздуха} = P \cdot \frac{T'}{T} = 100\,\text{кПа} \cdot \frac{353\,\text{К}}{293\,\text{К}} \approx 120\,\text{кПа}.
    $$

    Чтобы найти давление пара, нужно понять, будет ли он насыщенным после нагрева или нет.

    Плотность насыщенного пара при температуре $80\celsius$ равна $293\,\frac{\text{г}}{\text{м}^{3}}$, тогда для того,
    чтобы весь сосуд был заполнен насыщенным водяным паром нужно
    $m_\text{н.
    п.} = \rho_\text{н.
    п.
    80 $\celsius$} \cdot V = 293\,\frac{\text{г}}{\text{м}^{3}} \cdot 15\,\text{л} \approx 4{,}4\,\text{г}$ воды.
    Сравнивая эту массу с массой воды из условия, получаем массу жидкости, которая испарится: $m_\text{пара} = 4{,}4\,\text{г}$.
    Осталось определить давление этого пара:
    $$P_\text{пара} = \frac{m_\text{пара}RT'}{\mu V} = \frac{4{,}4\,\text{г} \cdot 8{,}31\,\frac{\text{Дж}}{\text{моль}\cdot\text{К}} \cdot 353\,\text{К}}{18\,\frac{\text{г}}{\text{моль}} \cdot 15\,\text{л}} \approx 48\,\text{кПа}.$$

    Получаем ответ: $P'_\text{пара} = 168{,}3\,\text{кПа}$.

    Другой вариант решения для давления пара:
    Определим давление пара, если бы вся вода испарилась (что не факт):
    $$P_\text{max} = \frac{mRT'}{\mu V} = \frac{30\,\text{г} \cdot 8{,}31\,\frac{\text{Дж}}{\text{моль}\cdot\text{К}} \cdot 353\,\text{К}}{18\,\frac{\text{г}}{\text{моль}} \cdot 15\,\text{л}} \approx 330\,\text{кПа}.$$
    Сравниваем это давление с давлением насыщенного пара при этой температуре $P_\text{н.
    п.
    80 $\celsius$} = 47{,}300\,\text{кПа}$:
    если у нас получилось меньше табличного значения,
    то вся вода испарилась, если же больше — испарилась лишь часть, а пар является насыщенным.
    Отсюда сразу получаем давление пара: $P'_\text{пара} = 47{,}3\,\text{кПа}$.
    Сравните этот результат с первым вариантом решения.

    Тут получаем ответ: $P'_\text{пара} = 167{,}8\,\text{кПа}$.
}
\solutionspace{150pt}

\tasknumber{4}%
\task{%
    Напротив физических величин запишите определение, обозначение и единицы измерения в системе СИ (если есть):
    \begin{enumerate}
        \item абсолютная влажность,
        \item насыщенный пар.
    \end{enumerate}
}

\variantsplitter

\addpersonalvariant{Софья Белянкина}

\tasknumber{1}%
\task{%
    Сколько молекул водяного пара содержится в сосуде объёмом $3\,\text{л}$ при температуре $30\celsius$,
    и влажности воздуха $20\%$?
}
\answer{%
    Уравнение состояния идеального газа (и учтём, что $R = N_A \cdot k$,
    это чуть упростит выячичления, но вовсе не обязательно это делать):
    $$
        PV = \nu RT = \frac N{N_A} RT \implies N = PV \cdot \frac{N_A}{RT}=  \frac{PV}{kT}
    $$
    Плотность насыщенного водяного пара при $30\celsius$ ищем по таблице: $P_{\text{нас.
    пара 30} \celsius} = 4{,}240\,\text{кПа}.$

    Получаем плотность пара в сосуде $\varphi = \frac P{P_{\text{нас.
    пара 30} \celsius}} \implies P = \varphi P_{\text{нас.
    пара 30} \celsius}.$

    И подставляем в ответ (по сути, его можно было получить быстрее из формул $P = nkT, n = \frac NV$):
    $$
        N = \frac{\varphi \cdot P_{\text{нас.
        пара 30} \celsius} \cdot V}{kT}
         = \frac{0{,}20 \cdot 4{,}240\,\text{кПа} \cdot 3\,\text{л}}{1{,}38 \cdot 10^{-23}\,\frac{\text{Дж}}{\text{К}} \cdot 303\,\text{К}}
         \approx 6{,}1 \cdot 10^{20}.
    $$

    Другой вариант решения (через плотности) приводит в результату:
    $$
        N = N_A \nu = N_A \cdot \frac m{\mu}
          = N_A \frac{\rho V}{\mu}
          = N_A \frac{\varphi \cdot \rho_{\text{нас.
          пара 30} \celsius} \cdot V}{\mu}
          = 6{,}02 \cdot 10^{23}\,\frac{1}{\text{моль}} \cdot \frac{0{,}20 \cdot 30{,}30\,\frac{\text{г}}{\text{м}^{3}} \cdot 3\,\text{л}}{18\,\frac{\text{г}}{\text{моль}}}
          \approx 6{,}1 \cdot 10^{20}.
    $$
}
\solutionspace{160pt}

\tasknumber{2}%
\task{%
    В герметичном сосуде находится влажный воздух при температуре $20\celsius$ и относительной влажности $80\%$.
    \begin{enumerate}
        \item Чему равно парциальное давление насыщенного водяного пара при этой температуре?
        \item Чему равно парциальное давление водяного пара?
        \item Определите точку росы этого пара?
        \item Каким станет парциальное давление водяного пара, если сосуд нагреть до $80\celsius$?
        \item Чему будет равна относительная влажность воздуха, если сосуд нагреть до $80\celsius$?
        \item Получите ответ на предыдущий вопрос, используя плотности, а не давления.
    \end{enumerate}
}
\answer{%
    Парциальное давление насыщенного водяного пара при $20\celsius$ ищем по таблице: $$P_{\text{нас.
    пара 20} \celsius} = 2{,}340\,\text{кПа}.$$

    Парциальное давление водяного пара
    $$P_\text{пара 1} = \varphi_1 \cdot P_{\text{нас.
    пара 20} \celsius} = 0{,}80 \cdot 2{,}340\,\text{кПа} = 1{,}8720\,\text{кПа}.$$

    Точку росы определяем по таблице: при какой температуре пар с давлением $P_\text{пара 1} = 1{,}8720\,\text{кПа}$ станет насыщенным: $16{,}5\celsius$.

    После нагрева парциальное давление пара возрастёт:
    $$
        \frac{P_\text{пара 1} \cdot V}{T_1} = \nu R = \frac{P_\text{пара 2} \cdot V}{T_2}
        \implies P_\text{пара 2} = P_\text{пара 1} \cdot \frac{T_2}{T_1} = 1{,}8720\,\text{кПа} \cdot \frac{353\,\text{К}}{293\,\text{К}} \approx 2{,}255\,\text{кПа}.
    $$

    Парциальное давление насыщенного водяного пара при $80\celsius$ ищем по таблице: $P_{\text{нас.
    пара 80} \celsius} = 47{,}300\,\text{кПа}$.
    Теперь определяем влажность:
    $$
        \varphi_2 = \frac{P_\text{пара 2}}{P_{\text{нас.
        пара 80} \celsius}} = \frac{2{,}255\,\text{кПа}}{47{,}300\,\text{кПа}} \approx 0{,}048 = 4{,}8\%.
    $$

    Или же выражаем то же самое через плотности (плотность не изменяется при изохорном нагревании $\rho_1 =\rho_2 = \rho$, в отличие от давления):
    $$
        \varphi_2 = \frac{\rho}{\rho_{\text{нас.
        пара 80} \celsius}} = \frac{\varphi_1\rho_{\text{нас.
        пара 20} \celsius}}{\rho_{\text{нас.
        пара 80} \celsius}}
        = \frac{0{,}80 \cdot 17{,}30\,\frac{\text{г}}{\text{м}^{3}}}{293\,\frac{\text{г}}{\text{м}^{3}}} \approx 0{,}047 = 4{,}7\%.
    $$
    Сравните 2 последних результата.
}
\solutionspace{200pt}

\tasknumber{3}%
\task{%
    Закрытый сосуд объёмом $10\,\text{л}$ заполнен сухим воздухом при давлении $100\,\text{кПа}$ и температуре $10\celsius$.
    Каким станет давление в сосуде, если в него налить $10\,\text{г}$ воды и нагреть содержимое сосуда до $90\celsius$?
}
\answer{%
    Конечное давление газа в сосуде складывается по закону Дальтона из давления нагретого сухого воздуха $P'_\text{воздуха}$ и
    давления насыщенного пара $P_\text{пара}$:
    $$P' = P'_\text{воздуха} + P_\text{пара}.$$

    Сперва определим новое давление сухого воздуха из уравнения состояния идеального газа:
    $$
        \frac{P'_\text{воздуха} \cdot V}{T'} = \nu R = \frac{P \cdot V}{T}
        \implies P'_\text{воздуха} = P \cdot \frac{T'}{T} = 100\,\text{кПа} \cdot \frac{363\,\text{К}}{283\,\text{К}} \approx 128\,\text{кПа}.
    $$

    Чтобы найти давление пара, нужно понять, будет ли он насыщенным после нагрева или нет.

    Плотность насыщенного пара при температуре $90\celsius$ равна $424\,\frac{\text{г}}{\text{м}^{3}}$, тогда для того,
    чтобы весь сосуд был заполнен насыщенным водяным паром нужно
    $m_\text{н.
    п.} = \rho_\text{н.
    п.
    90 $\celsius$} \cdot V = 424\,\frac{\text{г}}{\text{м}^{3}} \cdot 10\,\text{л} \approx 4{,}2\,\text{г}$ воды.
    Сравнивая эту массу с массой воды из условия, получаем массу жидкости, которая испарится: $m_\text{пара} = 4{,}2\,\text{г}$.
    Осталось определить давление этого пара:
    $$P_\text{пара} = \frac{m_\text{пара}RT'}{\mu V} = \frac{4{,}2\,\text{г} \cdot 8{,}31\,\frac{\text{Дж}}{\text{моль}\cdot\text{К}} \cdot 363\,\text{К}}{18\,\frac{\text{г}}{\text{моль}} \cdot 10\,\text{л}} \approx 70\,\text{кПа}.$$

    Получаем ответ: $P'_\text{пара} = 198{,}7\,\text{кПа}$.

    Другой вариант решения для давления пара:
    Определим давление пара, если бы вся вода испарилась (что не факт):
    $$P_\text{max} = \frac{mRT'}{\mu V} = \frac{10\,\text{г} \cdot 8{,}31\,\frac{\text{Дж}}{\text{моль}\cdot\text{К}} \cdot 363\,\text{К}}{18\,\frac{\text{г}}{\text{моль}} \cdot 10\,\text{л}} \approx 168\,\text{кПа}.$$
    Сравниваем это давление с давлением насыщенного пара при этой температуре $P_\text{н.
    п.
    90 $\celsius$} = 70{,}100\,\text{кПа}$:
    если у нас получилось меньше табличного значения,
    то вся вода испарилась, если же больше — испарилась лишь часть, а пар является насыщенным.
    Отсюда сразу получаем давление пара: $P'_\text{пара} = 70{,}1\,\text{кПа}$.
    Сравните этот результат с первым вариантом решения.

    Тут получаем ответ: $P'_\text{пара} = 198{,}4\,\text{кПа}$.
}
\solutionspace{150pt}

\tasknumber{4}%
\task{%
    Напротив физических величин запишите определение, обозначение и единицы измерения в системе СИ (если есть):
    \begin{enumerate}
        \item абсолютная влажность,
        \item динамическое равновесие.
    \end{enumerate}
}

\variantsplitter

\addpersonalvariant{Варвара Егиазарян}

\tasknumber{1}%
\task{%
    Сколько молекул водяного пара содержится в сосуде объёмом $12\,\text{л}$ при температуре $100\celsius$,
    и влажности воздуха $60\%$?
}
\answer{%
    Уравнение состояния идеального газа (и учтём, что $R = N_A \cdot k$,
    это чуть упростит выячичления, но вовсе не обязательно это делать):
    $$
        PV = \nu RT = \frac N{N_A} RT \implies N = PV \cdot \frac{N_A}{RT}=  \frac{PV}{kT}
    $$
    Плотность насыщенного водяного пара при $100\celsius$ ищем по таблице: $P_{\text{нас.
    пара 100} \celsius} = 101{,}300\,\text{кПа}.$

    Получаем плотность пара в сосуде $\varphi = \frac P{P_{\text{нас.
    пара 100} \celsius}} \implies P = \varphi P_{\text{нас.
    пара 100} \celsius}.$

    И подставляем в ответ (по сути, его можно было получить быстрее из формул $P = nkT, n = \frac NV$):
    $$
        N = \frac{\varphi \cdot P_{\text{нас.
        пара 100} \celsius} \cdot V}{kT}
         = \frac{0{,}60 \cdot 101{,}300\,\text{кПа} \cdot 12\,\text{л}}{1{,}38 \cdot 10^{-23}\,\frac{\text{Дж}}{\text{К}} \cdot 373\,\text{К}}
         \approx 1420 \cdot 10^{20}.
    $$

    Другой вариант решения (через плотности) приводит в результату:
    $$
        N = N_A \nu = N_A \cdot \frac m{\mu}
          = N_A \frac{\rho V}{\mu}
          = N_A \frac{\varphi \cdot \rho_{\text{нас.
          пара 100} \celsius} \cdot V}{\mu}
          = 6{,}02 \cdot 10^{23}\,\frac{1}{\text{моль}} \cdot \frac{0{,}60 \cdot 598\,\frac{\text{г}}{\text{м}^{3}} \cdot 12\,\text{л}}{18\,\frac{\text{г}}{\text{моль}}}
          \approx 1440 \cdot 10^{20}.
    $$
}
\solutionspace{160pt}

\tasknumber{2}%
\task{%
    В герметичном сосуде находится влажный воздух при температуре $30\celsius$ и относительной влажности $30\%$.
    \begin{enumerate}
        \item Чему равно парциальное давление насыщенного водяного пара при этой температуре?
        \item Чему равно парциальное давление водяного пара?
        \item Определите точку росы этого пара?
        \item Каким станет парциальное давление водяного пара, если сосуд нагреть до $90\celsius$?
        \item Чему будет равна относительная влажность воздуха, если сосуд нагреть до $90\celsius$?
        \item Получите ответ на предыдущий вопрос, используя плотности, а не давления.
    \end{enumerate}
}
\answer{%
    Парциальное давление насыщенного водяного пара при $30\celsius$ ищем по таблице: $$P_{\text{нас.
    пара 30} \celsius} = 4{,}240\,\text{кПа}.$$

    Парциальное давление водяного пара
    $$P_\text{пара 1} = \varphi_1 \cdot P_{\text{нас.
    пара 30} \celsius} = 0{,}30 \cdot 4{,}240\,\text{кПа} = 1{,}2720\,\text{кПа}.$$

    Точку росы определяем по таблице: при какой температуре пар с давлением $P_\text{пара 1} = 1{,}2720\,\text{кПа}$ станет насыщенным: $10{,}5\celsius$.

    После нагрева парциальное давление пара возрастёт:
    $$
        \frac{P_\text{пара 1} \cdot V}{T_1} = \nu R = \frac{P_\text{пара 2} \cdot V}{T_2}
        \implies P_\text{пара 2} = P_\text{пара 1} \cdot \frac{T_2}{T_1} = 1{,}2720\,\text{кПа} \cdot \frac{363\,\text{К}}{303\,\text{К}} \approx 1{,}5239\,\text{кПа}.
    $$

    Парциальное давление насыщенного водяного пара при $90\celsius$ ищем по таблице: $P_{\text{нас.
    пара 90} \celsius} = 70{,}100\,\text{кПа}$.
    Теперь определяем влажность:
    $$
        \varphi_2 = \frac{P_\text{пара 2}}{P_{\text{нас.
        пара 90} \celsius}} = \frac{1{,}5239\,\text{кПа}}{70{,}100\,\text{кПа}} \approx 0{,}022 = 2{,}2\%.
    $$

    Или же выражаем то же самое через плотности (плотность не изменяется при изохорном нагревании $\rho_1 =\rho_2 = \rho$, в отличие от давления):
    $$
        \varphi_2 = \frac{\rho}{\rho_{\text{нас.
        пара 90} \celsius}} = \frac{\varphi_1\rho_{\text{нас.
        пара 30} \celsius}}{\rho_{\text{нас.
        пара 90} \celsius}}
        = \frac{0{,}30 \cdot 30{,}30\,\frac{\text{г}}{\text{м}^{3}}}{424\,\frac{\text{г}}{\text{м}^{3}}} \approx 0{,}021 = 2{,}1\%.
    $$
    Сравните 2 последних результата.
}
\solutionspace{200pt}

\tasknumber{3}%
\task{%
    Закрытый сосуд объёмом $20\,\text{л}$ заполнен сухим воздухом при давлении $100\,\text{кПа}$ и температуре $20\celsius$.
    Каким станет давление в сосуде, если в него налить $20\,\text{г}$ воды и нагреть содержимое сосуда до $90\celsius$?
}
\answer{%
    Конечное давление газа в сосуде складывается по закону Дальтона из давления нагретого сухого воздуха $P'_\text{воздуха}$ и
    давления насыщенного пара $P_\text{пара}$:
    $$P' = P'_\text{воздуха} + P_\text{пара}.$$

    Сперва определим новое давление сухого воздуха из уравнения состояния идеального газа:
    $$
        \frac{P'_\text{воздуха} \cdot V}{T'} = \nu R = \frac{P \cdot V}{T}
        \implies P'_\text{воздуха} = P \cdot \frac{T'}{T} = 100\,\text{кПа} \cdot \frac{363\,\text{К}}{293\,\text{К}} \approx 124\,\text{кПа}.
    $$

    Чтобы найти давление пара, нужно понять, будет ли он насыщенным после нагрева или нет.

    Плотность насыщенного пара при температуре $90\celsius$ равна $424\,\frac{\text{г}}{\text{м}^{3}}$, тогда для того,
    чтобы весь сосуд был заполнен насыщенным водяным паром нужно
    $m_\text{н.
    п.} = \rho_\text{н.
    п.
    90 $\celsius$} \cdot V = 424\,\frac{\text{г}}{\text{м}^{3}} \cdot 20\,\text{л} \approx 8{,}48\,\text{г}$ воды.
    Сравнивая эту массу с массой воды из условия, получаем массу жидкости, которая испарится: $m_\text{пара} = 8{,}5\,\text{г}$.
    Осталось определить давление этого пара:
    $$P_\text{пара} = \frac{m_\text{пара}RT'}{\mu V} = \frac{8{,}5\,\text{г} \cdot 8{,}31\,\frac{\text{Дж}}{\text{моль}\cdot\text{К}} \cdot 363\,\text{К}}{18\,\frac{\text{г}}{\text{моль}} \cdot 20\,\text{л}} \approx 71\,\text{кПа}.$$

    Получаем ответ: $P'_\text{пара} = 195{,}1\,\text{кПа}$.

    Другой вариант решения для давления пара:
    Определим давление пара, если бы вся вода испарилась (что не факт):
    $$P_\text{max} = \frac{mRT'}{\mu V} = \frac{20\,\text{г} \cdot 8{,}31\,\frac{\text{Дж}}{\text{моль}\cdot\text{К}} \cdot 363\,\text{К}}{18\,\frac{\text{г}}{\text{моль}} \cdot 20\,\text{л}} \approx 168\,\text{кПа}.$$
    Сравниваем это давление с давлением насыщенного пара при этой температуре $P_\text{н.
    п.
    90 $\celsius$} = 70{,}100\,\text{кПа}$:
    если у нас получилось меньше табличного значения,
    то вся вода испарилась, если же больше — испарилась лишь часть, а пар является насыщенным.
    Отсюда сразу получаем давление пара: $P'_\text{пара} = 70{,}1\,\text{кПа}$.
    Сравните этот результат с первым вариантом решения.

    Тут получаем ответ: $P'_\text{пара} = 194\,\text{кПа}$.
}
\solutionspace{150pt}

\tasknumber{4}%
\task{%
    Напротив физических величин запишите определение, обозначение и единицы измерения в системе СИ (если есть):
    \begin{enumerate}
        \item относительная влажность,
        \item насыщенный пар.
    \end{enumerate}
}

\variantsplitter

\addpersonalvariant{Владислав Емелин}

\tasknumber{1}%
\task{%
    Сколько молекул водяного пара содержится в сосуде объёмом $12\,\text{л}$ при температуре $100\celsius$,
    и влажности воздуха $25\%$?
}
\answer{%
    Уравнение состояния идеального газа (и учтём, что $R = N_A \cdot k$,
    это чуть упростит выячичления, но вовсе не обязательно это делать):
    $$
        PV = \nu RT = \frac N{N_A} RT \implies N = PV \cdot \frac{N_A}{RT}=  \frac{PV}{kT}
    $$
    Плотность насыщенного водяного пара при $100\celsius$ ищем по таблице: $P_{\text{нас.
    пара 100} \celsius} = 101{,}300\,\text{кПа}.$

    Получаем плотность пара в сосуде $\varphi = \frac P{P_{\text{нас.
    пара 100} \celsius}} \implies P = \varphi P_{\text{нас.
    пара 100} \celsius}.$

    И подставляем в ответ (по сути, его можно было получить быстрее из формул $P = nkT, n = \frac NV$):
    $$
        N = \frac{\varphi \cdot P_{\text{нас.
        пара 100} \celsius} \cdot V}{kT}
         = \frac{0{,}25 \cdot 101{,}300\,\text{кПа} \cdot 12\,\text{л}}{1{,}38 \cdot 10^{-23}\,\frac{\text{Дж}}{\text{К}} \cdot 373\,\text{К}}
         \approx 590 \cdot 10^{20}.
    $$

    Другой вариант решения (через плотности) приводит в результату:
    $$
        N = N_A \nu = N_A \cdot \frac m{\mu}
          = N_A \frac{\rho V}{\mu}
          = N_A \frac{\varphi \cdot \rho_{\text{нас.
          пара 100} \celsius} \cdot V}{\mu}
          = 6{,}02 \cdot 10^{23}\,\frac{1}{\text{моль}} \cdot \frac{0{,}25 \cdot 598\,\frac{\text{г}}{\text{м}^{3}} \cdot 12\,\text{л}}{18\,\frac{\text{г}}{\text{моль}}}
          \approx 600 \cdot 10^{20}.
    $$
}
\solutionspace{160pt}

\tasknumber{2}%
\task{%
    В герметичном сосуде находится влажный воздух при температуре $15\celsius$ и относительной влажности $65\%$.
    \begin{enumerate}
        \item Чему равно парциальное давление насыщенного водяного пара при этой температуре?
        \item Чему равно парциальное давление водяного пара?
        \item Определите точку росы этого пара?
        \item Каким станет парциальное давление водяного пара, если сосуд нагреть до $80\celsius$?
        \item Чему будет равна относительная влажность воздуха, если сосуд нагреть до $80\celsius$?
        \item Получите ответ на предыдущий вопрос, используя плотности, а не давления.
    \end{enumerate}
}
\answer{%
    Парциальное давление насыщенного водяного пара при $15\celsius$ ищем по таблице: $$P_{\text{нас.
    пара 15} \celsius} = 1{,}700\,\text{кПа}.$$

    Парциальное давление водяного пара
    $$P_\text{пара 1} = \varphi_1 \cdot P_{\text{нас.
    пара 15} \celsius} = 0{,}65 \cdot 1{,}700\,\text{кПа} = 1{,}105\,\text{кПа}.$$

    Точку росы определяем по таблице: при какой температуре пар с давлением $P_\text{пара 1} = 1{,}105\,\text{кПа}$ станет насыщенным: $8{,}4\celsius$.

    После нагрева парциальное давление пара возрастёт:
    $$
        \frac{P_\text{пара 1} \cdot V}{T_1} = \nu R = \frac{P_\text{пара 2} \cdot V}{T_2}
        \implies P_\text{пара 2} = P_\text{пара 1} \cdot \frac{T_2}{T_1} = 1{,}105\,\text{кПа} \cdot \frac{353\,\text{К}}{288\,\text{К}} \approx 1{,}354\,\text{кПа}.
    $$

    Парциальное давление насыщенного водяного пара при $80\celsius$ ищем по таблице: $P_{\text{нас.
    пара 80} \celsius} = 47{,}300\,\text{кПа}$.
    Теперь определяем влажность:
    $$
        \varphi_2 = \frac{P_\text{пара 2}}{P_{\text{нас.
        пара 80} \celsius}} = \frac{1{,}354\,\text{кПа}}{47{,}300\,\text{кПа}} \approx 0{,}029 = 2{,}9\%.
    $$

    Или же выражаем то же самое через плотности (плотность не изменяется при изохорном нагревании $\rho_1 =\rho_2 = \rho$, в отличие от давления):
    $$
        \varphi_2 = \frac{\rho}{\rho_{\text{нас.
        пара 80} \celsius}} = \frac{\varphi_1\rho_{\text{нас.
        пара 15} \celsius}}{\rho_{\text{нас.
        пара 80} \celsius}}
        = \frac{0{,}65 \cdot 12{,}80\,\frac{\text{г}}{\text{м}^{3}}}{293\,\frac{\text{г}}{\text{м}^{3}}} \approx 0{,}028 = 2{,}8\%.
    $$
    Сравните 2 последних результата.
}
\solutionspace{200pt}

\tasknumber{3}%
\task{%
    Закрытый сосуд объёмом $15\,\text{л}$ заполнен сухим воздухом при давлении $100\,\text{кПа}$ и температуре $30\celsius$.
    Каким станет давление в сосуде, если в него налить $20\,\text{г}$ воды и нагреть содержимое сосуда до $100\celsius$?
}
\answer{%
    Конечное давление газа в сосуде складывается по закону Дальтона из давления нагретого сухого воздуха $P'_\text{воздуха}$ и
    давления насыщенного пара $P_\text{пара}$:
    $$P' = P'_\text{воздуха} + P_\text{пара}.$$

    Сперва определим новое давление сухого воздуха из уравнения состояния идеального газа:
    $$
        \frac{P'_\text{воздуха} \cdot V}{T'} = \nu R = \frac{P \cdot V}{T}
        \implies P'_\text{воздуха} = P \cdot \frac{T'}{T} = 100\,\text{кПа} \cdot \frac{373\,\text{К}}{303\,\text{К}} \approx 123\,\text{кПа}.
    $$

    Чтобы найти давление пара, нужно понять, будет ли он насыщенным после нагрева или нет.

    Плотность насыщенного пара при температуре $100\celsius$ равна $598\,\frac{\text{г}}{\text{м}^{3}}$, тогда для того,
    чтобы весь сосуд был заполнен насыщенным водяным паром нужно
    $m_\text{н.
    п.} = \rho_\text{н.
    п.
    100 $\celsius$} \cdot V = 598\,\frac{\text{г}}{\text{м}^{3}} \cdot 15\,\text{л} \approx 9{,}0\,\text{г}$ воды.
    Сравнивая эту массу с массой воды из условия, получаем массу жидкости, которая испарится: $m_\text{пара} = 9\,\text{г}$.
    Осталось определить давление этого пара:
    $$P_\text{пара} = \frac{m_\text{пара}RT'}{\mu V} = \frac{9\,\text{г} \cdot 8{,}31\,\frac{\text{Дж}}{\text{моль}\cdot\text{К}} \cdot 373\,\text{К}}{18\,\frac{\text{г}}{\text{моль}} \cdot 15\,\text{л}} \approx 103\,\text{кПа}.$$

    Получаем ответ: $P'_\text{пара} = 226{,}4\,\text{кПа}$.

    Другой вариант решения для давления пара:
    Определим давление пара, если бы вся вода испарилась (что не факт):
    $$P_\text{max} = \frac{mRT'}{\mu V} = \frac{20\,\text{г} \cdot 8{,}31\,\frac{\text{Дж}}{\text{моль}\cdot\text{К}} \cdot 373\,\text{К}}{18\,\frac{\text{г}}{\text{моль}} \cdot 15\,\text{л}} \approx 230\,\text{кПа}.$$
    Сравниваем это давление с давлением насыщенного пара при этой температуре $P_\text{н.
    п.
    100 $\celsius$} = 101{,}300\,\text{кПа}$:
    если у нас получилось меньше табличного значения,
    то вся вода испарилась, если же больше — испарилась лишь часть, а пар является насыщенным.
    Отсюда сразу получаем давление пара: $P'_\text{пара} = 101{,}3\,\text{кПа}$.
    Сравните этот результат с первым вариантом решения.

    Тут получаем ответ: $P'_\text{пара} = 224{,}4\,\text{кПа}$.
}
\solutionspace{150pt}

\tasknumber{4}%
\task{%
    Напротив физических величин запишите определение, обозначение и единицы измерения в системе СИ (если есть):
    \begin{enumerate}
        \item относительная влажность,
        \item динамическое равновесие.
    \end{enumerate}
}

\variantsplitter

\addpersonalvariant{Артём Жичин}

\tasknumber{1}%
\task{%
    Сколько молекул водяного пара содержится в сосуде объёмом $15\,\text{л}$ при температуре $100\celsius$,
    и влажности воздуха $55\%$?
}
\answer{%
    Уравнение состояния идеального газа (и учтём, что $R = N_A \cdot k$,
    это чуть упростит выячичления, но вовсе не обязательно это делать):
    $$
        PV = \nu RT = \frac N{N_A} RT \implies N = PV \cdot \frac{N_A}{RT}=  \frac{PV}{kT}
    $$
    Плотность насыщенного водяного пара при $100\celsius$ ищем по таблице: $P_{\text{нас.
    пара 100} \celsius} = 101{,}300\,\text{кПа}.$

    Получаем плотность пара в сосуде $\varphi = \frac P{P_{\text{нас.
    пара 100} \celsius}} \implies P = \varphi P_{\text{нас.
    пара 100} \celsius}.$

    И подставляем в ответ (по сути, его можно было получить быстрее из формул $P = nkT, n = \frac NV$):
    $$
        N = \frac{\varphi \cdot P_{\text{нас.
        пара 100} \celsius} \cdot V}{kT}
         = \frac{0{,}55 \cdot 101{,}300\,\text{кПа} \cdot 15\,\text{л}}{1{,}38 \cdot 10^{-23}\,\frac{\text{Дж}}{\text{К}} \cdot 373\,\text{К}}
         \approx 1620 \cdot 10^{20}.
    $$

    Другой вариант решения (через плотности) приводит в результату:
    $$
        N = N_A \nu = N_A \cdot \frac m{\mu}
          = N_A \frac{\rho V}{\mu}
          = N_A \frac{\varphi \cdot \rho_{\text{нас.
          пара 100} \celsius} \cdot V}{\mu}
          = 6{,}02 \cdot 10^{23}\,\frac{1}{\text{моль}} \cdot \frac{0{,}55 \cdot 598\,\frac{\text{г}}{\text{м}^{3}} \cdot 15\,\text{л}}{18\,\frac{\text{г}}{\text{моль}}}
          \approx 1650 \cdot 10^{20}.
    $$
}
\solutionspace{160pt}

\tasknumber{2}%
\task{%
    В герметичном сосуде находится влажный воздух при температуре $20\celsius$ и относительной влажности $40\%$.
    \begin{enumerate}
        \item Чему равно парциальное давление насыщенного водяного пара при этой температуре?
        \item Чему равно парциальное давление водяного пара?
        \item Определите точку росы этого пара?
        \item Каким станет парциальное давление водяного пара, если сосуд нагреть до $90\celsius$?
        \item Чему будет равна относительная влажность воздуха, если сосуд нагреть до $90\celsius$?
        \item Получите ответ на предыдущий вопрос, используя плотности, а не давления.
    \end{enumerate}
}
\answer{%
    Парциальное давление насыщенного водяного пара при $20\celsius$ ищем по таблице: $$P_{\text{нас.
    пара 20} \celsius} = 2{,}340\,\text{кПа}.$$

    Парциальное давление водяного пара
    $$P_\text{пара 1} = \varphi_1 \cdot P_{\text{нас.
    пара 20} \celsius} = 0{,}40 \cdot 2{,}340\,\text{кПа} = 0{,}9360\,\text{кПа}.$$

    Точку росы определяем по таблице: при какой температуре пар с давлением $P_\text{пара 1} = 0{,}9360\,\text{кПа}$ станет насыщенным: $6{,}0\celsius$.

    После нагрева парциальное давление пара возрастёт:
    $$
        \frac{P_\text{пара 1} \cdot V}{T_1} = \nu R = \frac{P_\text{пара 2} \cdot V}{T_2}
        \implies P_\text{пара 2} = P_\text{пара 1} \cdot \frac{T_2}{T_1} = 0{,}9360\,\text{кПа} \cdot \frac{363\,\text{К}}{293\,\text{К}} \approx 1{,}1596\,\text{кПа}.
    $$

    Парциальное давление насыщенного водяного пара при $90\celsius$ ищем по таблице: $P_{\text{нас.
    пара 90} \celsius} = 70{,}100\,\text{кПа}$.
    Теперь определяем влажность:
    $$
        \varphi_2 = \frac{P_\text{пара 2}}{P_{\text{нас.
        пара 90} \celsius}} = \frac{1{,}1596\,\text{кПа}}{70{,}100\,\text{кПа}} \approx 0{,}017 = 1{,}7\%.
    $$

    Или же выражаем то же самое через плотности (плотность не изменяется при изохорном нагревании $\rho_1 =\rho_2 = \rho$, в отличие от давления):
    $$
        \varphi_2 = \frac{\rho}{\rho_{\text{нас.
        пара 90} \celsius}} = \frac{\varphi_1\rho_{\text{нас.
        пара 20} \celsius}}{\rho_{\text{нас.
        пара 90} \celsius}}
        = \frac{0{,}40 \cdot 17{,}30\,\frac{\text{г}}{\text{м}^{3}}}{424\,\frac{\text{г}}{\text{м}^{3}}} \approx 0{,}016 = 1{,}6\%.
    $$
    Сравните 2 последних результата.
}
\solutionspace{200pt}

\tasknumber{3}%
\task{%
    Закрытый сосуд объёмом $10\,\text{л}$ заполнен сухим воздухом при давлении $100\,\text{кПа}$ и температуре $20\celsius$.
    Каким станет давление в сосуде, если в него налить $20\,\text{г}$ воды и нагреть содержимое сосуда до $90\celsius$?
}
\answer{%
    Конечное давление газа в сосуде складывается по закону Дальтона из давления нагретого сухого воздуха $P'_\text{воздуха}$ и
    давления насыщенного пара $P_\text{пара}$:
    $$P' = P'_\text{воздуха} + P_\text{пара}.$$

    Сперва определим новое давление сухого воздуха из уравнения состояния идеального газа:
    $$
        \frac{P'_\text{воздуха} \cdot V}{T'} = \nu R = \frac{P \cdot V}{T}
        \implies P'_\text{воздуха} = P \cdot \frac{T'}{T} = 100\,\text{кПа} \cdot \frac{363\,\text{К}}{293\,\text{К}} \approx 124\,\text{кПа}.
    $$

    Чтобы найти давление пара, нужно понять, будет ли он насыщенным после нагрева или нет.

    Плотность насыщенного пара при температуре $90\celsius$ равна $424\,\frac{\text{г}}{\text{м}^{3}}$, тогда для того,
    чтобы весь сосуд был заполнен насыщенным водяным паром нужно
    $m_\text{н.
    п.} = \rho_\text{н.
    п.
    90 $\celsius$} \cdot V = 424\,\frac{\text{г}}{\text{м}^{3}} \cdot 10\,\text{л} \approx 4{,}2\,\text{г}$ воды.
    Сравнивая эту массу с массой воды из условия, получаем массу жидкости, которая испарится: $m_\text{пара} = 4{,}2\,\text{г}$.
    Осталось определить давление этого пара:
    $$P_\text{пара} = \frac{m_\text{пара}RT'}{\mu V} = \frac{4{,}2\,\text{г} \cdot 8{,}31\,\frac{\text{Дж}}{\text{моль}\cdot\text{К}} \cdot 363\,\text{К}}{18\,\frac{\text{г}}{\text{моль}} \cdot 10\,\text{л}} \approx 70\,\text{кПа}.$$

    Получаем ответ: $P'_\text{пара} = 194{,}3\,\text{кПа}$.

    Другой вариант решения для давления пара:
    Определим давление пара, если бы вся вода испарилась (что не факт):
    $$P_\text{max} = \frac{mRT'}{\mu V} = \frac{20\,\text{г} \cdot 8{,}31\,\frac{\text{Дж}}{\text{моль}\cdot\text{К}} \cdot 363\,\text{К}}{18\,\frac{\text{г}}{\text{моль}} \cdot 10\,\text{л}} \approx 340\,\text{кПа}.$$
    Сравниваем это давление с давлением насыщенного пара при этой температуре $P_\text{н.
    п.
    90 $\celsius$} = 70{,}100\,\text{кПа}$:
    если у нас получилось меньше табличного значения,
    то вся вода испарилась, если же больше — испарилась лишь часть, а пар является насыщенным.
    Отсюда сразу получаем давление пара: $P'_\text{пара} = 70{,}1\,\text{кПа}$.
    Сравните этот результат с первым вариантом решения.

    Тут получаем ответ: $P'_\text{пара} = 194\,\text{кПа}$.
}
\solutionspace{150pt}

\tasknumber{4}%
\task{%
    Напротив физических величин запишите определение, обозначение и единицы измерения в системе СИ (если есть):
    \begin{enumerate}
        \item абсолютная влажность,
        \item динамическое равновесие.
    \end{enumerate}
}

\variantsplitter

\addpersonalvariant{Дарья Кошман}

\tasknumber{1}%
\task{%
    Сколько молекул водяного пара содержится в сосуде объёмом $7\,\text{л}$ при температуре $90\celsius$,
    и влажности воздуха $60\%$?
}
\answer{%
    Уравнение состояния идеального газа (и учтём, что $R = N_A \cdot k$,
    это чуть упростит выячичления, но вовсе не обязательно это делать):
    $$
        PV = \nu RT = \frac N{N_A} RT \implies N = PV \cdot \frac{N_A}{RT}=  \frac{PV}{kT}
    $$
    Плотность насыщенного водяного пара при $90\celsius$ ищем по таблице: $P_{\text{нас.
    пара 90} \celsius} = 70{,}100\,\text{кПа}.$

    Получаем плотность пара в сосуде $\varphi = \frac P{P_{\text{нас.
    пара 90} \celsius}} \implies P = \varphi P_{\text{нас.
    пара 90} \celsius}.$

    И подставляем в ответ (по сути, его можно было получить быстрее из формул $P = nkT, n = \frac NV$):
    $$
        N = \frac{\varphi \cdot P_{\text{нас.
        пара 90} \celsius} \cdot V}{kT}
         = \frac{0{,}60 \cdot 70{,}100\,\text{кПа} \cdot 7\,\text{л}}{1{,}38 \cdot 10^{-23}\,\frac{\text{Дж}}{\text{К}} \cdot 363\,\text{К}}
         \approx 590 \cdot 10^{20}.
    $$

    Другой вариант решения (через плотности) приводит в результату:
    $$
        N = N_A \nu = N_A \cdot \frac m{\mu}
          = N_A \frac{\rho V}{\mu}
          = N_A \frac{\varphi \cdot \rho_{\text{нас.
          пара 90} \celsius} \cdot V}{\mu}
          = 6{,}02 \cdot 10^{23}\,\frac{1}{\text{моль}} \cdot \frac{0{,}60 \cdot 424\,\frac{\text{г}}{\text{м}^{3}} \cdot 7\,\text{л}}{18\,\frac{\text{г}}{\text{моль}}}
          \approx 600 \cdot 10^{20}.
    $$
}
\solutionspace{160pt}

\tasknumber{2}%
\task{%
    В герметичном сосуде находится влажный воздух при температуре $20\celsius$ и относительной влажности $20\%$.
    \begin{enumerate}
        \item Чему равно парциальное давление насыщенного водяного пара при этой температуре?
        \item Чему равно парциальное давление водяного пара?
        \item Определите точку росы этого пара?
        \item Каким станет парциальное давление водяного пара, если сосуд нагреть до $70\celsius$?
        \item Чему будет равна относительная влажность воздуха, если сосуд нагреть до $70\celsius$?
        \item Получите ответ на предыдущий вопрос, используя плотности, а не давления.
    \end{enumerate}
}
\answer{%
    Парциальное давление насыщенного водяного пара при $20\celsius$ ищем по таблице: $$P_{\text{нас.
    пара 20} \celsius} = 2{,}340\,\text{кПа}.$$

    Парциальное давление водяного пара
    $$P_\text{пара 1} = \varphi_1 \cdot P_{\text{нас.
    пара 20} \celsius} = 0{,}20 \cdot 2{,}340\,\text{кПа} = 0{,}4680\,\text{кПа}.$$

    Точку росы определяем по таблице: при какой температуре пар с давлением $P_\text{пара 1} = 0{,}4680\,\text{кПа}$ станет насыщенным: $0{,}0\celsius$.

    После нагрева парциальное давление пара возрастёт:
    $$
        \frac{P_\text{пара 1} \cdot V}{T_1} = \nu R = \frac{P_\text{пара 2} \cdot V}{T_2}
        \implies P_\text{пара 2} = P_\text{пара 1} \cdot \frac{T_2}{T_1} = 0{,}4680\,\text{кПа} \cdot \frac{343\,\text{К}}{293\,\text{К}} \approx 0{,}5479\,\text{кПа}.
    $$

    Парциальное давление насыщенного водяного пара при $70\celsius$ ищем по таблице: $P_{\text{нас.
    пара 70} \celsius} = 31\,\text{кПа}$.
    Теперь определяем влажность:
    $$
        \varphi_2 = \frac{P_\text{пара 2}}{P_{\text{нас.
        пара 70} \celsius}} = \frac{0{,}5479\,\text{кПа}}{31\,\text{кПа}} \approx 0{,}018 = 1{,}8\%.
    $$

    Или же выражаем то же самое через плотности (плотность не изменяется при изохорном нагревании $\rho_1 =\rho_2 = \rho$, в отличие от давления):
    $$
        \varphi_2 = \frac{\rho}{\rho_{\text{нас.
        пара 70} \celsius}} = \frac{\varphi_1\rho_{\text{нас.
        пара 20} \celsius}}{\rho_{\text{нас.
        пара 70} \celsius}}
        = \frac{0{,}20 \cdot 17{,}30\,\frac{\text{г}}{\text{м}^{3}}}{198\,\frac{\text{г}}{\text{м}^{3}}} \approx 0{,}017 = 1{,}7\%.
    $$
    Сравните 2 последних результата.
}
\solutionspace{200pt}

\tasknumber{3}%
\task{%
    Закрытый сосуд объёмом $20\,\text{л}$ заполнен сухим воздухом при давлении $100\,\text{кПа}$ и температуре $10\celsius$.
    Каким станет давление в сосуде, если в него налить $5\,\text{г}$ воды и нагреть содержимое сосуда до $80\celsius$?
}
\answer{%
    Конечное давление газа в сосуде складывается по закону Дальтона из давления нагретого сухого воздуха $P'_\text{воздуха}$ и
    давления насыщенного пара $P_\text{пара}$:
    $$P' = P'_\text{воздуха} + P_\text{пара}.$$

    Сперва определим новое давление сухого воздуха из уравнения состояния идеального газа:
    $$
        \frac{P'_\text{воздуха} \cdot V}{T'} = \nu R = \frac{P \cdot V}{T}
        \implies P'_\text{воздуха} = P \cdot \frac{T'}{T} = 100\,\text{кПа} \cdot \frac{353\,\text{К}}{283\,\text{К}} \approx 125\,\text{кПа}.
    $$

    Чтобы найти давление пара, нужно понять, будет ли он насыщенным после нагрева или нет.

    Плотность насыщенного пара при температуре $80\celsius$ равна $293\,\frac{\text{г}}{\text{м}^{3}}$, тогда для того,
    чтобы весь сосуд был заполнен насыщенным водяным паром нужно
    $m_\text{н.
    п.} = \rho_\text{н.
    п.
    80 $\celsius$} \cdot V = 293\,\frac{\text{г}}{\text{м}^{3}} \cdot 20\,\text{л} \approx 5{,}86\,\text{г}$ воды.
    Сравнивая эту массу с массой воды из условия, получаем массу жидкости, которая испарится: $m_\text{пара} = 5\,\text{г}$.
    Осталось определить давление этого пара:
    $$P_\text{пара} = \frac{m_\text{пара}RT'}{\mu V} = \frac{5\,\text{г} \cdot 8{,}31\,\frac{\text{Дж}}{\text{моль}\cdot\text{К}} \cdot 353\,\text{К}}{18\,\frac{\text{г}}{\text{моль}} \cdot 20\,\text{л}} \approx 41\,\text{кПа}.$$

    Получаем ответ: $P'_\text{пара} = 165{,}5\,\text{кПа}$.

    Другой вариант решения для давления пара:
    Определим давление пара, если бы вся вода испарилась (что не факт):
    $$P_\text{max} = \frac{mRT'}{\mu V} = \frac{5\,\text{г} \cdot 8{,}31\,\frac{\text{Дж}}{\text{моль}\cdot\text{К}} \cdot 353\,\text{К}}{18\,\frac{\text{г}}{\text{моль}} \cdot 20\,\text{л}} \approx 41\,\text{кПа}.$$
    Сравниваем это давление с давлением насыщенного пара при этой температуре $P_\text{н.
    п.
    80 $\celsius$} = 47{,}300\,\text{кПа}$:
    если у нас получилось меньше табличного значения,
    то вся вода испарилась, если же больше — испарилась лишь часть, а пар является насыщенным.
    Отсюда сразу получаем давление пара: $P'_\text{пара} = 40{,}7\,\text{кПа}$.
    Сравните этот результат с первым вариантом решения.

    Тут получаем ответ: $P'_\text{пара} = 165{,}5\,\text{кПа}$.
}
\solutionspace{150pt}

\tasknumber{4}%
\task{%
    Напротив физических величин запишите определение, обозначение и единицы измерения в системе СИ (если есть):
    \begin{enumerate}
        \item абсолютная влажность,
        \item динамическое равновесие.
    \end{enumerate}
}

\variantsplitter

\addpersonalvariant{Анна Кузьмичёва}

\tasknumber{1}%
\task{%
    Сколько молекул водяного пара содержится в сосуде объёмом $3\,\text{л}$ при температуре $50\celsius$,
    и влажности воздуха $25\%$?
}
\answer{%
    Уравнение состояния идеального газа (и учтём, что $R = N_A \cdot k$,
    это чуть упростит выячичления, но вовсе не обязательно это делать):
    $$
        PV = \nu RT = \frac N{N_A} RT \implies N = PV \cdot \frac{N_A}{RT}=  \frac{PV}{kT}
    $$
    Плотность насыщенного водяного пара при $50\celsius$ ищем по таблице: $P_{\text{нас.
    пара 50} \celsius} = 12{,}300\,\text{кПа}.$

    Получаем плотность пара в сосуде $\varphi = \frac P{P_{\text{нас.
    пара 50} \celsius}} \implies P = \varphi P_{\text{нас.
    пара 50} \celsius}.$

    И подставляем в ответ (по сути, его можно было получить быстрее из формул $P = nkT, n = \frac NV$):
    $$
        N = \frac{\varphi \cdot P_{\text{нас.
        пара 50} \celsius} \cdot V}{kT}
         = \frac{0{,}25 \cdot 12{,}300\,\text{кПа} \cdot 3\,\text{л}}{1{,}38 \cdot 10^{-23}\,\frac{\text{Дж}}{\text{К}} \cdot 323\,\text{К}}
         \approx 21 \cdot 10^{20}.
    $$

    Другой вариант решения (через плотности) приводит в результату:
    $$
        N = N_A \nu = N_A \cdot \frac m{\mu}
          = N_A \frac{\rho V}{\mu}
          = N_A \frac{\varphi \cdot \rho_{\text{нас.
          пара 50} \celsius} \cdot V}{\mu}
          = 6{,}02 \cdot 10^{23}\,\frac{1}{\text{моль}} \cdot \frac{0{,}25 \cdot 83\,\frac{\text{г}}{\text{м}^{3}} \cdot 3\,\text{л}}{18\,\frac{\text{г}}{\text{моль}}}
          \approx 21 \cdot 10^{20}.
    $$
}
\solutionspace{160pt}

\tasknumber{2}%
\task{%
    В герметичном сосуде находится влажный воздух при температуре $15\celsius$ и относительной влажности $35\%$.
    \begin{enumerate}
        \item Чему равно парциальное давление насыщенного водяного пара при этой температуре?
        \item Чему равно парциальное давление водяного пара?
        \item Определите точку росы этого пара?
        \item Каким станет парциальное давление водяного пара, если сосуд нагреть до $90\celsius$?
        \item Чему будет равна относительная влажность воздуха, если сосуд нагреть до $90\celsius$?
        \item Получите ответ на предыдущий вопрос, используя плотности, а не давления.
    \end{enumerate}
}
\answer{%
    Парциальное давление насыщенного водяного пара при $15\celsius$ ищем по таблице: $$P_{\text{нас.
    пара 15} \celsius} = 1{,}700\,\text{кПа}.$$

    Парциальное давление водяного пара
    $$P_\text{пара 1} = \varphi_1 \cdot P_{\text{нас.
    пара 15} \celsius} = 0{,}35 \cdot 1{,}700\,\text{кПа} = 0{,}595\,\text{кПа}.$$

    Точку росы определяем по таблице: при какой температуре пар с давлением $P_\text{пара 1} = 0{,}595\,\text{кПа}$ станет насыщенным: $0{,}0\celsius$.

    После нагрева парциальное давление пара возрастёт:
    $$
        \frac{P_\text{пара 1} \cdot V}{T_1} = \nu R = \frac{P_\text{пара 2} \cdot V}{T_2}
        \implies P_\text{пара 2} = P_\text{пара 1} \cdot \frac{T_2}{T_1} = 0{,}595\,\text{кПа} \cdot \frac{363\,\text{К}}{288\,\text{К}} \approx 0{,}750\,\text{кПа}.
    $$

    Парциальное давление насыщенного водяного пара при $90\celsius$ ищем по таблице: $P_{\text{нас.
    пара 90} \celsius} = 70{,}100\,\text{кПа}$.
    Теперь определяем влажность:
    $$
        \varphi_2 = \frac{P_\text{пара 2}}{P_{\text{нас.
        пара 90} \celsius}} = \frac{0{,}750\,\text{кПа}}{70{,}100\,\text{кПа}} \approx 0{,}011 = 1{,}1\%.
    $$

    Или же выражаем то же самое через плотности (плотность не изменяется при изохорном нагревании $\rho_1 =\rho_2 = \rho$, в отличие от давления):
    $$
        \varphi_2 = \frac{\rho}{\rho_{\text{нас.
        пара 90} \celsius}} = \frac{\varphi_1\rho_{\text{нас.
        пара 15} \celsius}}{\rho_{\text{нас.
        пара 90} \celsius}}
        = \frac{0{,}35 \cdot 12{,}80\,\frac{\text{г}}{\text{м}^{3}}}{424\,\frac{\text{г}}{\text{м}^{3}}} \approx 0{,}011 = 1{,}1\%.
    $$
    Сравните 2 последних результата.
}
\solutionspace{200pt}

\tasknumber{3}%
\task{%
    Закрытый сосуд объёмом $15\,\text{л}$ заполнен сухим воздухом при давлении $100\,\text{кПа}$ и температуре $20\celsius$.
    Каким станет давление в сосуде, если в него налить $30\,\text{г}$ воды и нагреть содержимое сосуда до $100\celsius$?
}
\answer{%
    Конечное давление газа в сосуде складывается по закону Дальтона из давления нагретого сухого воздуха $P'_\text{воздуха}$ и
    давления насыщенного пара $P_\text{пара}$:
    $$P' = P'_\text{воздуха} + P_\text{пара}.$$

    Сперва определим новое давление сухого воздуха из уравнения состояния идеального газа:
    $$
        \frac{P'_\text{воздуха} \cdot V}{T'} = \nu R = \frac{P \cdot V}{T}
        \implies P'_\text{воздуха} = P \cdot \frac{T'}{T} = 100\,\text{кПа} \cdot \frac{373\,\text{К}}{293\,\text{К}} \approx 127\,\text{кПа}.
    $$

    Чтобы найти давление пара, нужно понять, будет ли он насыщенным после нагрева или нет.

    Плотность насыщенного пара при температуре $100\celsius$ равна $598\,\frac{\text{г}}{\text{м}^{3}}$, тогда для того,
    чтобы весь сосуд был заполнен насыщенным водяным паром нужно
    $m_\text{н.
    п.} = \rho_\text{н.
    п.
    100 $\celsius$} \cdot V = 598\,\frac{\text{г}}{\text{м}^{3}} \cdot 15\,\text{л} \approx 9{,}0\,\text{г}$ воды.
    Сравнивая эту массу с массой воды из условия, получаем массу жидкости, которая испарится: $m_\text{пара} = 9\,\text{г}$.
    Осталось определить давление этого пара:
    $$P_\text{пара} = \frac{m_\text{пара}RT'}{\mu V} = \frac{9\,\text{г} \cdot 8{,}31\,\frac{\text{Дж}}{\text{моль}\cdot\text{К}} \cdot 373\,\text{К}}{18\,\frac{\text{г}}{\text{моль}} \cdot 15\,\text{л}} \approx 103\,\text{кПа}.$$

    Получаем ответ: $P'_\text{пара} = 230{,}6\,\text{кПа}$.

    Другой вариант решения для давления пара:
    Определим давление пара, если бы вся вода испарилась (что не факт):
    $$P_\text{max} = \frac{mRT'}{\mu V} = \frac{30\,\text{г} \cdot 8{,}31\,\frac{\text{Дж}}{\text{моль}\cdot\text{К}} \cdot 373\,\text{К}}{18\,\frac{\text{г}}{\text{моль}} \cdot 15\,\text{л}} \approx 340\,\text{кПа}.$$
    Сравниваем это давление с давлением насыщенного пара при этой температуре $P_\text{н.
    п.
    100 $\celsius$} = 101{,}300\,\text{кПа}$:
    если у нас получилось меньше табличного значения,
    то вся вода испарилась, если же больше — испарилась лишь часть, а пар является насыщенным.
    Отсюда сразу получаем давление пара: $P'_\text{пара} = 101{,}3\,\text{кПа}$.
    Сравните этот результат с первым вариантом решения.

    Тут получаем ответ: $P'_\text{пара} = 228{,}6\,\text{кПа}$.
}
\solutionspace{150pt}

\tasknumber{4}%
\task{%
    Напротив физических величин запишите определение, обозначение и единицы измерения в системе СИ (если есть):
    \begin{enumerate}
        \item абсолютная влажность,
        \item динамическое равновесие.
    \end{enumerate}
}

\variantsplitter

\addpersonalvariant{Алёна Куприянова}

\tasknumber{1}%
\task{%
    Сколько молекул водяного пара содержится в сосуде объёмом $15\,\text{л}$ при температуре $40\celsius$,
    и влажности воздуха $60\%$?
}
\answer{%
    Уравнение состояния идеального газа (и учтём, что $R = N_A \cdot k$,
    это чуть упростит выячичления, но вовсе не обязательно это делать):
    $$
        PV = \nu RT = \frac N{N_A} RT \implies N = PV \cdot \frac{N_A}{RT}=  \frac{PV}{kT}
    $$
    Плотность насыщенного водяного пара при $40\celsius$ ищем по таблице: $P_{\text{нас.
    пара 40} \celsius} = 7{,}370\,\text{кПа}.$

    Получаем плотность пара в сосуде $\varphi = \frac P{P_{\text{нас.
    пара 40} \celsius}} \implies P = \varphi P_{\text{нас.
    пара 40} \celsius}.$

    И подставляем в ответ (по сути, его можно было получить быстрее из формул $P = nkT, n = \frac NV$):
    $$
        N = \frac{\varphi \cdot P_{\text{нас.
        пара 40} \celsius} \cdot V}{kT}
         = \frac{0{,}60 \cdot 7{,}370\,\text{кПа} \cdot 15\,\text{л}}{1{,}38 \cdot 10^{-23}\,\frac{\text{Дж}}{\text{К}} \cdot 313\,\text{К}}
         \approx 154 \cdot 10^{20}.
    $$

    Другой вариант решения (через плотности) приводит в результату:
    $$
        N = N_A \nu = N_A \cdot \frac m{\mu}
          = N_A \frac{\rho V}{\mu}
          = N_A \frac{\varphi \cdot \rho_{\text{нас.
          пара 40} \celsius} \cdot V}{\mu}
          = 6{,}02 \cdot 10^{23}\,\frac{1}{\text{моль}} \cdot \frac{0{,}60 \cdot 51{,}20\,\frac{\text{г}}{\text{м}^{3}} \cdot 15\,\text{л}}{18\,\frac{\text{г}}{\text{моль}}}
          \approx 154 \cdot 10^{20}.
    $$
}
\solutionspace{160pt}

\tasknumber{2}%
\task{%
    В герметичном сосуде находится влажный воздух при температуре $25\celsius$ и относительной влажности $50\%$.
    \begin{enumerate}
        \item Чему равно парциальное давление насыщенного водяного пара при этой температуре?
        \item Чему равно парциальное давление водяного пара?
        \item Определите точку росы этого пара?
        \item Каким станет парциальное давление водяного пара, если сосуд нагреть до $90\celsius$?
        \item Чему будет равна относительная влажность воздуха, если сосуд нагреть до $90\celsius$?
        \item Получите ответ на предыдущий вопрос, используя плотности, а не давления.
    \end{enumerate}
}
\answer{%
    Парциальное давление насыщенного водяного пара при $25\celsius$ ищем по таблице: $$P_{\text{нас.
    пара 25} \celsius} = 3{,}170\,\text{кПа}.$$

    Парциальное давление водяного пара
    $$P_\text{пара 1} = \varphi_1 \cdot P_{\text{нас.
    пара 25} \celsius} = 0{,}50 \cdot 3{,}170\,\text{кПа} = 1{,}5850\,\text{кПа}.$$

    Точку росы определяем по таблице: при какой температуре пар с давлением $P_\text{пара 1} = 1{,}5850\,\text{кПа}$ станет насыщенным: $13{,}9\celsius$.

    После нагрева парциальное давление пара возрастёт:
    $$
        \frac{P_\text{пара 1} \cdot V}{T_1} = \nu R = \frac{P_\text{пара 2} \cdot V}{T_2}
        \implies P_\text{пара 2} = P_\text{пара 1} \cdot \frac{T_2}{T_1} = 1{,}5850\,\text{кПа} \cdot \frac{363\,\text{К}}{298\,\text{К}} \approx 1{,}9307\,\text{кПа}.
    $$

    Парциальное давление насыщенного водяного пара при $90\celsius$ ищем по таблице: $P_{\text{нас.
    пара 90} \celsius} = 70{,}100\,\text{кПа}$.
    Теперь определяем влажность:
    $$
        \varphi_2 = \frac{P_\text{пара 2}}{P_{\text{нас.
        пара 90} \celsius}} = \frac{1{,}9307\,\text{кПа}}{70{,}100\,\text{кПа}} \approx 0{,}028 = 2{,}8\%.
    $$

    Или же выражаем то же самое через плотности (плотность не изменяется при изохорном нагревании $\rho_1 =\rho_2 = \rho$, в отличие от давления):
    $$
        \varphi_2 = \frac{\rho}{\rho_{\text{нас.
        пара 90} \celsius}} = \frac{\varphi_1\rho_{\text{нас.
        пара 25} \celsius}}{\rho_{\text{нас.
        пара 90} \celsius}}
        = \frac{0{,}50 \cdot 23\,\frac{\text{г}}{\text{м}^{3}}}{424\,\frac{\text{г}}{\text{м}^{3}}} \approx 0{,}027 = 2{,}7\%.
    $$
    Сравните 2 последних результата.
}
\solutionspace{200pt}

\tasknumber{3}%
\task{%
    Закрытый сосуд объёмом $15\,\text{л}$ заполнен сухим воздухом при давлении $100\,\text{кПа}$ и температуре $10\celsius$.
    Каким станет давление в сосуде, если в него налить $5\,\text{г}$ воды и нагреть содержимое сосуда до $90\celsius$?
}
\answer{%
    Конечное давление газа в сосуде складывается по закону Дальтона из давления нагретого сухого воздуха $P'_\text{воздуха}$ и
    давления насыщенного пара $P_\text{пара}$:
    $$P' = P'_\text{воздуха} + P_\text{пара}.$$

    Сперва определим новое давление сухого воздуха из уравнения состояния идеального газа:
    $$
        \frac{P'_\text{воздуха} \cdot V}{T'} = \nu R = \frac{P \cdot V}{T}
        \implies P'_\text{воздуха} = P \cdot \frac{T'}{T} = 100\,\text{кПа} \cdot \frac{363\,\text{К}}{283\,\text{К}} \approx 128\,\text{кПа}.
    $$

    Чтобы найти давление пара, нужно понять, будет ли он насыщенным после нагрева или нет.

    Плотность насыщенного пара при температуре $90\celsius$ равна $424\,\frac{\text{г}}{\text{м}^{3}}$, тогда для того,
    чтобы весь сосуд был заполнен насыщенным водяным паром нужно
    $m_\text{н.
    п.} = \rho_\text{н.
    п.
    90 $\celsius$} \cdot V = 424\,\frac{\text{г}}{\text{м}^{3}} \cdot 15\,\text{л} \approx 6{,}4\,\text{г}$ воды.
    Сравнивая эту массу с массой воды из условия, получаем массу жидкости, которая испарится: $m_\text{пара} = 5\,\text{г}$.
    Осталось определить давление этого пара:
    $$P_\text{пара} = \frac{m_\text{пара}RT'}{\mu V} = \frac{5\,\text{г} \cdot 8{,}31\,\frac{\text{Дж}}{\text{моль}\cdot\text{К}} \cdot 363\,\text{К}}{18\,\frac{\text{г}}{\text{моль}} \cdot 15\,\text{л}} \approx 56\,\text{кПа}.$$

    Получаем ответ: $P'_\text{пара} = 184{,}1\,\text{кПа}$.

    Другой вариант решения для давления пара:
    Определим давление пара, если бы вся вода испарилась (что не факт):
    $$P_\text{max} = \frac{mRT'}{\mu V} = \frac{5\,\text{г} \cdot 8{,}31\,\frac{\text{Дж}}{\text{моль}\cdot\text{К}} \cdot 363\,\text{К}}{18\,\frac{\text{г}}{\text{моль}} \cdot 15\,\text{л}} \approx 56\,\text{кПа}.$$
    Сравниваем это давление с давлением насыщенного пара при этой температуре $P_\text{н.
    п.
    90 $\celsius$} = 70{,}100\,\text{кПа}$:
    если у нас получилось меньше табличного значения,
    то вся вода испарилась, если же больше — испарилась лишь часть, а пар является насыщенным.
    Отсюда сразу получаем давление пара: $P'_\text{пара} = 55{,}9\,\text{кПа}$.
    Сравните этот результат с первым вариантом решения.

    Тут получаем ответ: $P'_\text{пара} = 184{,}1\,\text{кПа}$.
}
\solutionspace{150pt}

\tasknumber{4}%
\task{%
    Напротив физических величин запишите определение, обозначение и единицы измерения в системе СИ (если есть):
    \begin{enumerate}
        \item относительная влажность,
        \item насыщенный пар.
    \end{enumerate}
}

\variantsplitter

\addpersonalvariant{Ярослав Лавровский}

\tasknumber{1}%
\task{%
    Сколько молекул водяного пара содержится в сосуде объёмом $7\,\text{л}$ при температуре $100\celsius$,
    и влажности воздуха $40\%$?
}
\answer{%
    Уравнение состояния идеального газа (и учтём, что $R = N_A \cdot k$,
    это чуть упростит выячичления, но вовсе не обязательно это делать):
    $$
        PV = \nu RT = \frac N{N_A} RT \implies N = PV \cdot \frac{N_A}{RT}=  \frac{PV}{kT}
    $$
    Плотность насыщенного водяного пара при $100\celsius$ ищем по таблице: $P_{\text{нас.
    пара 100} \celsius} = 101{,}300\,\text{кПа}.$

    Получаем плотность пара в сосуде $\varphi = \frac P{P_{\text{нас.
    пара 100} \celsius}} \implies P = \varphi P_{\text{нас.
    пара 100} \celsius}.$

    И подставляем в ответ (по сути, его можно было получить быстрее из формул $P = nkT, n = \frac NV$):
    $$
        N = \frac{\varphi \cdot P_{\text{нас.
        пара 100} \celsius} \cdot V}{kT}
         = \frac{0{,}40 \cdot 101{,}300\,\text{кПа} \cdot 7\,\text{л}}{1{,}38 \cdot 10^{-23}\,\frac{\text{Дж}}{\text{К}} \cdot 373\,\text{К}}
         \approx 550 \cdot 10^{20}.
    $$

    Другой вариант решения (через плотности) приводит в результату:
    $$
        N = N_A \nu = N_A \cdot \frac m{\mu}
          = N_A \frac{\rho V}{\mu}
          = N_A \frac{\varphi \cdot \rho_{\text{нас.
          пара 100} \celsius} \cdot V}{\mu}
          = 6{,}02 \cdot 10^{23}\,\frac{1}{\text{моль}} \cdot \frac{0{,}40 \cdot 598\,\frac{\text{г}}{\text{м}^{3}} \cdot 7\,\text{л}}{18\,\frac{\text{г}}{\text{моль}}}
          \approx 560 \cdot 10^{20}.
    $$
}
\solutionspace{160pt}

\tasknumber{2}%
\task{%
    В герметичном сосуде находится влажный воздух при температуре $15\celsius$ и относительной влажности $40\%$.
    \begin{enumerate}
        \item Чему равно парциальное давление насыщенного водяного пара при этой температуре?
        \item Чему равно парциальное давление водяного пара?
        \item Определите точку росы этого пара?
        \item Каким станет парциальное давление водяного пара, если сосуд нагреть до $90\celsius$?
        \item Чему будет равна относительная влажность воздуха, если сосуд нагреть до $90\celsius$?
        \item Получите ответ на предыдущий вопрос, используя плотности, а не давления.
    \end{enumerate}
}
\answer{%
    Парциальное давление насыщенного водяного пара при $15\celsius$ ищем по таблице: $$P_{\text{нас.
    пара 15} \celsius} = 1{,}700\,\text{кПа}.$$

    Парциальное давление водяного пара
    $$P_\text{пара 1} = \varphi_1 \cdot P_{\text{нас.
    пара 15} \celsius} = 0{,}40 \cdot 1{,}700\,\text{кПа} = 0{,}680\,\text{кПа}.$$

    Точку росы определяем по таблице: при какой температуре пар с давлением $P_\text{пара 1} = 0{,}680\,\text{кПа}$ станет насыщенным: $1{,}5\celsius$.

    После нагрева парциальное давление пара возрастёт:
    $$
        \frac{P_\text{пара 1} \cdot V}{T_1} = \nu R = \frac{P_\text{пара 2} \cdot V}{T_2}
        \implies P_\text{пара 2} = P_\text{пара 1} \cdot \frac{T_2}{T_1} = 0{,}680\,\text{кПа} \cdot \frac{363\,\text{К}}{288\,\text{К}} \approx 0{,}857\,\text{кПа}.
    $$

    Парциальное давление насыщенного водяного пара при $90\celsius$ ищем по таблице: $P_{\text{нас.
    пара 90} \celsius} = 70{,}100\,\text{кПа}$.
    Теперь определяем влажность:
    $$
        \varphi_2 = \frac{P_\text{пара 2}}{P_{\text{нас.
        пара 90} \celsius}} = \frac{0{,}857\,\text{кПа}}{70{,}100\,\text{кПа}} \approx 0{,}012 = 1{,}2\%.
    $$

    Или же выражаем то же самое через плотности (плотность не изменяется при изохорном нагревании $\rho_1 =\rho_2 = \rho$, в отличие от давления):
    $$
        \varphi_2 = \frac{\rho}{\rho_{\text{нас.
        пара 90} \celsius}} = \frac{\varphi_1\rho_{\text{нас.
        пара 15} \celsius}}{\rho_{\text{нас.
        пара 90} \celsius}}
        = \frac{0{,}40 \cdot 12{,}80\,\frac{\text{г}}{\text{м}^{3}}}{424\,\frac{\text{г}}{\text{м}^{3}}} \approx 0{,}012 = 1{,}2\%.
    $$
    Сравните 2 последних результата.
}
\solutionspace{200pt}

\tasknumber{3}%
\task{%
    Закрытый сосуд объёмом $15\,\text{л}$ заполнен сухим воздухом при давлении $100\,\text{кПа}$ и температуре $10\celsius$.
    Каким станет давление в сосуде, если в него налить $5\,\text{г}$ воды и нагреть содержимое сосуда до $80\celsius$?
}
\answer{%
    Конечное давление газа в сосуде складывается по закону Дальтона из давления нагретого сухого воздуха $P'_\text{воздуха}$ и
    давления насыщенного пара $P_\text{пара}$:
    $$P' = P'_\text{воздуха} + P_\text{пара}.$$

    Сперва определим новое давление сухого воздуха из уравнения состояния идеального газа:
    $$
        \frac{P'_\text{воздуха} \cdot V}{T'} = \nu R = \frac{P \cdot V}{T}
        \implies P'_\text{воздуха} = P \cdot \frac{T'}{T} = 100\,\text{кПа} \cdot \frac{353\,\text{К}}{283\,\text{К}} \approx 125\,\text{кПа}.
    $$

    Чтобы найти давление пара, нужно понять, будет ли он насыщенным после нагрева или нет.

    Плотность насыщенного пара при температуре $80\celsius$ равна $293\,\frac{\text{г}}{\text{м}^{3}}$, тогда для того,
    чтобы весь сосуд был заполнен насыщенным водяным паром нужно
    $m_\text{н.
    п.} = \rho_\text{н.
    п.
    80 $\celsius$} \cdot V = 293\,\frac{\text{г}}{\text{м}^{3}} \cdot 15\,\text{л} \approx 4{,}4\,\text{г}$ воды.
    Сравнивая эту массу с массой воды из условия, получаем массу жидкости, которая испарится: $m_\text{пара} = 4{,}4\,\text{г}$.
    Осталось определить давление этого пара:
    $$P_\text{пара} = \frac{m_\text{пара}RT'}{\mu V} = \frac{4{,}4\,\text{г} \cdot 8{,}31\,\frac{\text{Дж}}{\text{моль}\cdot\text{К}} \cdot 353\,\text{К}}{18\,\frac{\text{г}}{\text{моль}} \cdot 15\,\text{л}} \approx 48\,\text{кПа}.$$

    Получаем ответ: $P'_\text{пара} = 172{,}5\,\text{кПа}$.

    Другой вариант решения для давления пара:
    Определим давление пара, если бы вся вода испарилась (что не факт):
    $$P_\text{max} = \frac{mRT'}{\mu V} = \frac{5\,\text{г} \cdot 8{,}31\,\frac{\text{Дж}}{\text{моль}\cdot\text{К}} \cdot 353\,\text{К}}{18\,\frac{\text{г}}{\text{моль}} \cdot 15\,\text{л}} \approx 54\,\text{кПа}.$$
    Сравниваем это давление с давлением насыщенного пара при этой температуре $P_\text{н.
    п.
    80 $\celsius$} = 47{,}300\,\text{кПа}$:
    если у нас получилось меньше табличного значения,
    то вся вода испарилась, если же больше — испарилась лишь часть, а пар является насыщенным.
    Отсюда сразу получаем давление пара: $P'_\text{пара} = 47{,}3\,\text{кПа}$.
    Сравните этот результат с первым вариантом решения.

    Тут получаем ответ: $P'_\text{пара} = 172\,\text{кПа}$.
}
\solutionspace{150pt}

\tasknumber{4}%
\task{%
    Напротив физических величин запишите определение, обозначение и единицы измерения в системе СИ (если есть):
    \begin{enumerate}
        \item относительная влажность,
        \item насыщенный пар.
    \end{enumerate}
}

\variantsplitter

\addpersonalvariant{Анастасия Ламанова}

\tasknumber{1}%
\task{%
    Сколько молекул водяного пара содержится в сосуде объёмом $15\,\text{л}$ при температуре $15\celsius$,
    и влажности воздуха $35\%$?
}
\answer{%
    Уравнение состояния идеального газа (и учтём, что $R = N_A \cdot k$,
    это чуть упростит выячичления, но вовсе не обязательно это делать):
    $$
        PV = \nu RT = \frac N{N_A} RT \implies N = PV \cdot \frac{N_A}{RT}=  \frac{PV}{kT}
    $$
    Плотность насыщенного водяного пара при $15\celsius$ ищем по таблице: $P_{\text{нас.
    пара 15} \celsius} = 1{,}700\,\text{кПа}.$

    Получаем плотность пара в сосуде $\varphi = \frac P{P_{\text{нас.
    пара 15} \celsius}} \implies P = \varphi P_{\text{нас.
    пара 15} \celsius}.$

    И подставляем в ответ (по сути, его можно было получить быстрее из формул $P = nkT, n = \frac NV$):
    $$
        N = \frac{\varphi \cdot P_{\text{нас.
        пара 15} \celsius} \cdot V}{kT}
         = \frac{0{,}35 \cdot 1{,}700\,\text{кПа} \cdot 15\,\text{л}}{1{,}38 \cdot 10^{-23}\,\frac{\text{Дж}}{\text{К}} \cdot 288\,\text{К}}
         \approx 22 \cdot 10^{20}.
    $$

    Другой вариант решения (через плотности) приводит в результату:
    $$
        N = N_A \nu = N_A \cdot \frac m{\mu}
          = N_A \frac{\rho V}{\mu}
          = N_A \frac{\varphi \cdot \rho_{\text{нас.
          пара 15} \celsius} \cdot V}{\mu}
          = 6{,}02 \cdot 10^{23}\,\frac{1}{\text{моль}} \cdot \frac{0{,}35 \cdot 12{,}80\,\frac{\text{г}}{\text{м}^{3}} \cdot 15\,\text{л}}{18\,\frac{\text{г}}{\text{моль}}}
          \approx 22 \cdot 10^{20}.
    $$
}
\solutionspace{160pt}

\tasknumber{2}%
\task{%
    В герметичном сосуде находится влажный воздух при температуре $15\celsius$ и относительной влажности $35\%$.
    \begin{enumerate}
        \item Чему равно парциальное давление насыщенного водяного пара при этой температуре?
        \item Чему равно парциальное давление водяного пара?
        \item Определите точку росы этого пара?
        \item Каким станет парциальное давление водяного пара, если сосуд нагреть до $70\celsius$?
        \item Чему будет равна относительная влажность воздуха, если сосуд нагреть до $70\celsius$?
        \item Получите ответ на предыдущий вопрос, используя плотности, а не давления.
    \end{enumerate}
}
\answer{%
    Парциальное давление насыщенного водяного пара при $15\celsius$ ищем по таблице: $$P_{\text{нас.
    пара 15} \celsius} = 1{,}700\,\text{кПа}.$$

    Парциальное давление водяного пара
    $$P_\text{пара 1} = \varphi_1 \cdot P_{\text{нас.
    пара 15} \celsius} = 0{,}35 \cdot 1{,}700\,\text{кПа} = 0{,}595\,\text{кПа}.$$

    Точку росы определяем по таблице: при какой температуре пар с давлением $P_\text{пара 1} = 0{,}595\,\text{кПа}$ станет насыщенным: $0{,}0\celsius$.

    После нагрева парциальное давление пара возрастёт:
    $$
        \frac{P_\text{пара 1} \cdot V}{T_1} = \nu R = \frac{P_\text{пара 2} \cdot V}{T_2}
        \implies P_\text{пара 2} = P_\text{пара 1} \cdot \frac{T_2}{T_1} = 0{,}595\,\text{кПа} \cdot \frac{343\,\text{К}}{288\,\text{К}} \approx 0{,}709\,\text{кПа}.
    $$

    Парциальное давление насыщенного водяного пара при $70\celsius$ ищем по таблице: $P_{\text{нас.
    пара 70} \celsius} = 31\,\text{кПа}$.
    Теперь определяем влажность:
    $$
        \varphi_2 = \frac{P_\text{пара 2}}{P_{\text{нас.
        пара 70} \celsius}} = \frac{0{,}709\,\text{кПа}}{31\,\text{кПа}} \approx 0{,}023 = 2{,}3\%.
    $$

    Или же выражаем то же самое через плотности (плотность не изменяется при изохорном нагревании $\rho_1 =\rho_2 = \rho$, в отличие от давления):
    $$
        \varphi_2 = \frac{\rho}{\rho_{\text{нас.
        пара 70} \celsius}} = \frac{\varphi_1\rho_{\text{нас.
        пара 15} \celsius}}{\rho_{\text{нас.
        пара 70} \celsius}}
        = \frac{0{,}35 \cdot 12{,}80\,\frac{\text{г}}{\text{м}^{3}}}{198\,\frac{\text{г}}{\text{м}^{3}}} \approx 0{,}023 = 2{,}3\%.
    $$
    Сравните 2 последних результата.
}
\solutionspace{200pt}

\tasknumber{3}%
\task{%
    Закрытый сосуд объёмом $20\,\text{л}$ заполнен сухим воздухом при давлении $100\,\text{кПа}$ и температуре $30\celsius$.
    Каким станет давление в сосуде, если в него налить $30\,\text{г}$ воды и нагреть содержимое сосуда до $90\celsius$?
}
\answer{%
    Конечное давление газа в сосуде складывается по закону Дальтона из давления нагретого сухого воздуха $P'_\text{воздуха}$ и
    давления насыщенного пара $P_\text{пара}$:
    $$P' = P'_\text{воздуха} + P_\text{пара}.$$

    Сперва определим новое давление сухого воздуха из уравнения состояния идеального газа:
    $$
        \frac{P'_\text{воздуха} \cdot V}{T'} = \nu R = \frac{P \cdot V}{T}
        \implies P'_\text{воздуха} = P \cdot \frac{T'}{T} = 100\,\text{кПа} \cdot \frac{363\,\text{К}}{303\,\text{К}} \approx 120\,\text{кПа}.
    $$

    Чтобы найти давление пара, нужно понять, будет ли он насыщенным после нагрева или нет.

    Плотность насыщенного пара при температуре $90\celsius$ равна $424\,\frac{\text{г}}{\text{м}^{3}}$, тогда для того,
    чтобы весь сосуд был заполнен насыщенным водяным паром нужно
    $m_\text{н.
    п.} = \rho_\text{н.
    п.
    90 $\celsius$} \cdot V = 424\,\frac{\text{г}}{\text{м}^{3}} \cdot 20\,\text{л} \approx 8{,}48\,\text{г}$ воды.
    Сравнивая эту массу с массой воды из условия, получаем массу жидкости, которая испарится: $m_\text{пара} = 8{,}5\,\text{г}$.
    Осталось определить давление этого пара:
    $$P_\text{пара} = \frac{m_\text{пара}RT'}{\mu V} = \frac{8{,}5\,\text{г} \cdot 8{,}31\,\frac{\text{Дж}}{\text{моль}\cdot\text{К}} \cdot 363\,\text{К}}{18\,\frac{\text{г}}{\text{моль}} \cdot 20\,\text{л}} \approx 71\,\text{кПа}.$$

    Получаем ответ: $P'_\text{пара} = 191\,\text{кПа}$.

    Другой вариант решения для давления пара:
    Определим давление пара, если бы вся вода испарилась (что не факт):
    $$P_\text{max} = \frac{mRT'}{\mu V} = \frac{30\,\text{г} \cdot 8{,}31\,\frac{\text{Дж}}{\text{моль}\cdot\text{К}} \cdot 363\,\text{К}}{18\,\frac{\text{г}}{\text{моль}} \cdot 20\,\text{л}} \approx 250\,\text{кПа}.$$
    Сравниваем это давление с давлением насыщенного пара при этой температуре $P_\text{н.
    п.
    90 $\celsius$} = 70{,}100\,\text{кПа}$:
    если у нас получилось меньше табличного значения,
    то вся вода испарилась, если же больше — испарилась лишь часть, а пар является насыщенным.
    Отсюда сразу получаем давление пара: $P'_\text{пара} = 70{,}1\,\text{кПа}$.
    Сравните этот результат с первым вариантом решения.

    Тут получаем ответ: $P'_\text{пара} = 189{,}9\,\text{кПа}$.
}
\solutionspace{150pt}

\tasknumber{4}%
\task{%
    Напротив физических величин запишите определение, обозначение и единицы измерения в системе СИ (если есть):
    \begin{enumerate}
        \item относительная влажность,
        \item динамическое равновесие.
    \end{enumerate}
}

\variantsplitter

\addpersonalvariant{Виктория Легонькова}

\tasknumber{1}%
\task{%
    Сколько молекул водяного пара содержится в сосуде объёмом $6\,\text{л}$ при температуре $20\celsius$,
    и влажности воздуха $70\%$?
}
\answer{%
    Уравнение состояния идеального газа (и учтём, что $R = N_A \cdot k$,
    это чуть упростит выячичления, но вовсе не обязательно это делать):
    $$
        PV = \nu RT = \frac N{N_A} RT \implies N = PV \cdot \frac{N_A}{RT}=  \frac{PV}{kT}
    $$
    Плотность насыщенного водяного пара при $20\celsius$ ищем по таблице: $P_{\text{нас.
    пара 20} \celsius} = 2{,}340\,\text{кПа}.$

    Получаем плотность пара в сосуде $\varphi = \frac P{P_{\text{нас.
    пара 20} \celsius}} \implies P = \varphi P_{\text{нас.
    пара 20} \celsius}.$

    И подставляем в ответ (по сути, его можно было получить быстрее из формул $P = nkT, n = \frac NV$):
    $$
        N = \frac{\varphi \cdot P_{\text{нас.
        пара 20} \celsius} \cdot V}{kT}
         = \frac{0{,}70 \cdot 2{,}340\,\text{кПа} \cdot 6\,\text{л}}{1{,}38 \cdot 10^{-23}\,\frac{\text{Дж}}{\text{К}} \cdot 293\,\text{К}}
         \approx 24 \cdot 10^{20}.
    $$

    Другой вариант решения (через плотности) приводит в результату:
    $$
        N = N_A \nu = N_A \cdot \frac m{\mu}
          = N_A \frac{\rho V}{\mu}
          = N_A \frac{\varphi \cdot \rho_{\text{нас.
          пара 20} \celsius} \cdot V}{\mu}
          = 6{,}02 \cdot 10^{23}\,\frac{1}{\text{моль}} \cdot \frac{0{,}70 \cdot 17{,}30\,\frac{\text{г}}{\text{м}^{3}} \cdot 6\,\text{л}}{18\,\frac{\text{г}}{\text{моль}}}
          \approx 24 \cdot 10^{20}.
    $$
}
\solutionspace{160pt}

\tasknumber{2}%
\task{%
    В герметичном сосуде находится влажный воздух при температуре $30\celsius$ и относительной влажности $25\%$.
    \begin{enumerate}
        \item Чему равно парциальное давление насыщенного водяного пара при этой температуре?
        \item Чему равно парциальное давление водяного пара?
        \item Определите точку росы этого пара?
        \item Каким станет парциальное давление водяного пара, если сосуд нагреть до $90\celsius$?
        \item Чему будет равна относительная влажность воздуха, если сосуд нагреть до $90\celsius$?
        \item Получите ответ на предыдущий вопрос, используя плотности, а не давления.
    \end{enumerate}
}
\answer{%
    Парциальное давление насыщенного водяного пара при $30\celsius$ ищем по таблице: $$P_{\text{нас.
    пара 30} \celsius} = 4{,}240\,\text{кПа}.$$

    Парциальное давление водяного пара
    $$P_\text{пара 1} = \varphi_1 \cdot P_{\text{нас.
    пара 30} \celsius} = 0{,}25 \cdot 4{,}240\,\text{кПа} = 1{,}0600\,\text{кПа}.$$

    Точку росы определяем по таблице: при какой температуре пар с давлением $P_\text{пара 1} = 1{,}0600\,\text{кПа}$ станет насыщенным: $7{,}8\celsius$.

    После нагрева парциальное давление пара возрастёт:
    $$
        \frac{P_\text{пара 1} \cdot V}{T_1} = \nu R = \frac{P_\text{пара 2} \cdot V}{T_2}
        \implies P_\text{пара 2} = P_\text{пара 1} \cdot \frac{T_2}{T_1} = 1{,}0600\,\text{кПа} \cdot \frac{363\,\text{К}}{303\,\text{К}} \approx 1{,}2699\,\text{кПа}.
    $$

    Парциальное давление насыщенного водяного пара при $90\celsius$ ищем по таблице: $P_{\text{нас.
    пара 90} \celsius} = 70{,}100\,\text{кПа}$.
    Теперь определяем влажность:
    $$
        \varphi_2 = \frac{P_\text{пара 2}}{P_{\text{нас.
        пара 90} \celsius}} = \frac{1{,}2699\,\text{кПа}}{70{,}100\,\text{кПа}} \approx 0{,}018 = 1{,}8\%.
    $$

    Или же выражаем то же самое через плотности (плотность не изменяется при изохорном нагревании $\rho_1 =\rho_2 = \rho$, в отличие от давления):
    $$
        \varphi_2 = \frac{\rho}{\rho_{\text{нас.
        пара 90} \celsius}} = \frac{\varphi_1\rho_{\text{нас.
        пара 30} \celsius}}{\rho_{\text{нас.
        пара 90} \celsius}}
        = \frac{0{,}25 \cdot 30{,}30\,\frac{\text{г}}{\text{м}^{3}}}{424\,\frac{\text{г}}{\text{м}^{3}}} \approx 0{,}018 = 1{,}8\%.
    $$
    Сравните 2 последних результата.
}
\solutionspace{200pt}

\tasknumber{3}%
\task{%
    Закрытый сосуд объёмом $10\,\text{л}$ заполнен сухим воздухом при давлении $100\,\text{кПа}$ и температуре $20\celsius$.
    Каким станет давление в сосуде, если в него налить $30\,\text{г}$ воды и нагреть содержимое сосуда до $90\celsius$?
}
\answer{%
    Конечное давление газа в сосуде складывается по закону Дальтона из давления нагретого сухого воздуха $P'_\text{воздуха}$ и
    давления насыщенного пара $P_\text{пара}$:
    $$P' = P'_\text{воздуха} + P_\text{пара}.$$

    Сперва определим новое давление сухого воздуха из уравнения состояния идеального газа:
    $$
        \frac{P'_\text{воздуха} \cdot V}{T'} = \nu R = \frac{P \cdot V}{T}
        \implies P'_\text{воздуха} = P \cdot \frac{T'}{T} = 100\,\text{кПа} \cdot \frac{363\,\text{К}}{293\,\text{К}} \approx 124\,\text{кПа}.
    $$

    Чтобы найти давление пара, нужно понять, будет ли он насыщенным после нагрева или нет.

    Плотность насыщенного пара при температуре $90\celsius$ равна $424\,\frac{\text{г}}{\text{м}^{3}}$, тогда для того,
    чтобы весь сосуд был заполнен насыщенным водяным паром нужно
    $m_\text{н.
    п.} = \rho_\text{н.
    п.
    90 $\celsius$} \cdot V = 424\,\frac{\text{г}}{\text{м}^{3}} \cdot 10\,\text{л} \approx 4{,}2\,\text{г}$ воды.
    Сравнивая эту массу с массой воды из условия, получаем массу жидкости, которая испарится: $m_\text{пара} = 4{,}2\,\text{г}$.
    Осталось определить давление этого пара:
    $$P_\text{пара} = \frac{m_\text{пара}RT'}{\mu V} = \frac{4{,}2\,\text{г} \cdot 8{,}31\,\frac{\text{Дж}}{\text{моль}\cdot\text{К}} \cdot 363\,\text{К}}{18\,\frac{\text{г}}{\text{моль}} \cdot 10\,\text{л}} \approx 70\,\text{кПа}.$$

    Получаем ответ: $P'_\text{пара} = 194{,}3\,\text{кПа}$.

    Другой вариант решения для давления пара:
    Определим давление пара, если бы вся вода испарилась (что не факт):
    $$P_\text{max} = \frac{mRT'}{\mu V} = \frac{30\,\text{г} \cdot 8{,}31\,\frac{\text{Дж}}{\text{моль}\cdot\text{К}} \cdot 363\,\text{К}}{18\,\frac{\text{г}}{\text{моль}} \cdot 10\,\text{л}} \approx 500\,\text{кПа}.$$
    Сравниваем это давление с давлением насыщенного пара при этой температуре $P_\text{н.
    п.
    90 $\celsius$} = 70{,}100\,\text{кПа}$:
    если у нас получилось меньше табличного значения,
    то вся вода испарилась, если же больше — испарилась лишь часть, а пар является насыщенным.
    Отсюда сразу получаем давление пара: $P'_\text{пара} = 70{,}1\,\text{кПа}$.
    Сравните этот результат с первым вариантом решения.

    Тут получаем ответ: $P'_\text{пара} = 194\,\text{кПа}$.
}
\solutionspace{150pt}

\tasknumber{4}%
\task{%
    Напротив физических величин запишите определение, обозначение и единицы измерения в системе СИ (если есть):
    \begin{enumerate}
        \item относительная влажность,
        \item динамическое равновесие.
    \end{enumerate}
}

\variantsplitter

\addpersonalvariant{Семён Мартынов}

\tasknumber{1}%
\task{%
    Сколько молекул водяного пара содержится в сосуде объёмом $6\,\text{л}$ при температуре $100\celsius$,
    и влажности воздуха $35\%$?
}
\answer{%
    Уравнение состояния идеального газа (и учтём, что $R = N_A \cdot k$,
    это чуть упростит выячичления, но вовсе не обязательно это делать):
    $$
        PV = \nu RT = \frac N{N_A} RT \implies N = PV \cdot \frac{N_A}{RT}=  \frac{PV}{kT}
    $$
    Плотность насыщенного водяного пара при $100\celsius$ ищем по таблице: $P_{\text{нас.
    пара 100} \celsius} = 101{,}300\,\text{кПа}.$

    Получаем плотность пара в сосуде $\varphi = \frac P{P_{\text{нас.
    пара 100} \celsius}} \implies P = \varphi P_{\text{нас.
    пара 100} \celsius}.$

    И подставляем в ответ (по сути, его можно было получить быстрее из формул $P = nkT, n = \frac NV$):
    $$
        N = \frac{\varphi \cdot P_{\text{нас.
        пара 100} \celsius} \cdot V}{kT}
         = \frac{0{,}35 \cdot 101{,}300\,\text{кПа} \cdot 6\,\text{л}}{1{,}38 \cdot 10^{-23}\,\frac{\text{Дж}}{\text{К}} \cdot 373\,\text{К}}
         \approx 410 \cdot 10^{20}.
    $$

    Другой вариант решения (через плотности) приводит в результату:
    $$
        N = N_A \nu = N_A \cdot \frac m{\mu}
          = N_A \frac{\rho V}{\mu}
          = N_A \frac{\varphi \cdot \rho_{\text{нас.
          пара 100} \celsius} \cdot V}{\mu}
          = 6{,}02 \cdot 10^{23}\,\frac{1}{\text{моль}} \cdot \frac{0{,}35 \cdot 598\,\frac{\text{г}}{\text{м}^{3}} \cdot 6\,\text{л}}{18\,\frac{\text{г}}{\text{моль}}}
          \approx 420 \cdot 10^{20}.
    $$
}
\solutionspace{160pt}

\tasknumber{2}%
\task{%
    В герметичном сосуде находится влажный воздух при температуре $15\celsius$ и относительной влажности $65\%$.
    \begin{enumerate}
        \item Чему равно парциальное давление насыщенного водяного пара при этой температуре?
        \item Чему равно парциальное давление водяного пара?
        \item Определите точку росы этого пара?
        \item Каким станет парциальное давление водяного пара, если сосуд нагреть до $70\celsius$?
        \item Чему будет равна относительная влажность воздуха, если сосуд нагреть до $70\celsius$?
        \item Получите ответ на предыдущий вопрос, используя плотности, а не давления.
    \end{enumerate}
}
\answer{%
    Парциальное давление насыщенного водяного пара при $15\celsius$ ищем по таблице: $$P_{\text{нас.
    пара 15} \celsius} = 1{,}700\,\text{кПа}.$$

    Парциальное давление водяного пара
    $$P_\text{пара 1} = \varphi_1 \cdot P_{\text{нас.
    пара 15} \celsius} = 0{,}65 \cdot 1{,}700\,\text{кПа} = 1{,}105\,\text{кПа}.$$

    Точку росы определяем по таблице: при какой температуре пар с давлением $P_\text{пара 1} = 1{,}105\,\text{кПа}$ станет насыщенным: $8{,}4\celsius$.

    После нагрева парциальное давление пара возрастёт:
    $$
        \frac{P_\text{пара 1} \cdot V}{T_1} = \nu R = \frac{P_\text{пара 2} \cdot V}{T_2}
        \implies P_\text{пара 2} = P_\text{пара 1} \cdot \frac{T_2}{T_1} = 1{,}105\,\text{кПа} \cdot \frac{343\,\text{К}}{288\,\text{К}} \approx 1{,}316\,\text{кПа}.
    $$

    Парциальное давление насыщенного водяного пара при $70\celsius$ ищем по таблице: $P_{\text{нас.
    пара 70} \celsius} = 31\,\text{кПа}$.
    Теперь определяем влажность:
    $$
        \varphi_2 = \frac{P_\text{пара 2}}{P_{\text{нас.
        пара 70} \celsius}} = \frac{1{,}316\,\text{кПа}}{31\,\text{кПа}} \approx 0{,}042 = 4{,}2\%.
    $$

    Или же выражаем то же самое через плотности (плотность не изменяется при изохорном нагревании $\rho_1 =\rho_2 = \rho$, в отличие от давления):
    $$
        \varphi_2 = \frac{\rho}{\rho_{\text{нас.
        пара 70} \celsius}} = \frac{\varphi_1\rho_{\text{нас.
        пара 15} \celsius}}{\rho_{\text{нас.
        пара 70} \celsius}}
        = \frac{0{,}65 \cdot 12{,}80\,\frac{\text{г}}{\text{м}^{3}}}{198\,\frac{\text{г}}{\text{м}^{3}}} \approx 0{,}042 = 4{,}2\%.
    $$
    Сравните 2 последних результата.
}
\solutionspace{200pt}

\tasknumber{3}%
\task{%
    Закрытый сосуд объёмом $10\,\text{л}$ заполнен сухим воздухом при давлении $100\,\text{кПа}$ и температуре $20\celsius$.
    Каким станет давление в сосуде, если в него налить $20\,\text{г}$ воды и нагреть содержимое сосуда до $80\celsius$?
}
\answer{%
    Конечное давление газа в сосуде складывается по закону Дальтона из давления нагретого сухого воздуха $P'_\text{воздуха}$ и
    давления насыщенного пара $P_\text{пара}$:
    $$P' = P'_\text{воздуха} + P_\text{пара}.$$

    Сперва определим новое давление сухого воздуха из уравнения состояния идеального газа:
    $$
        \frac{P'_\text{воздуха} \cdot V}{T'} = \nu R = \frac{P \cdot V}{T}
        \implies P'_\text{воздуха} = P \cdot \frac{T'}{T} = 100\,\text{кПа} \cdot \frac{353\,\text{К}}{293\,\text{К}} \approx 120\,\text{кПа}.
    $$

    Чтобы найти давление пара, нужно понять, будет ли он насыщенным после нагрева или нет.

    Плотность насыщенного пара при температуре $80\celsius$ равна $293\,\frac{\text{г}}{\text{м}^{3}}$, тогда для того,
    чтобы весь сосуд был заполнен насыщенным водяным паром нужно
    $m_\text{н.
    п.} = \rho_\text{н.
    п.
    80 $\celsius$} \cdot V = 293\,\frac{\text{г}}{\text{м}^{3}} \cdot 10\,\text{л} \approx 2{,}9\,\text{г}$ воды.
    Сравнивая эту массу с массой воды из условия, получаем массу жидкости, которая испарится: $m_\text{пара} = 2{,}9\,\text{г}$.
    Осталось определить давление этого пара:
    $$P_\text{пара} = \frac{m_\text{пара}RT'}{\mu V} = \frac{2{,}9\,\text{г} \cdot 8{,}31\,\frac{\text{Дж}}{\text{моль}\cdot\text{К}} \cdot 353\,\text{К}}{18\,\frac{\text{г}}{\text{моль}} \cdot 10\,\text{л}} \approx 47\,\text{кПа}.$$

    Получаем ответ: $P'_\text{пара} = 167{,}7\,\text{кПа}$.

    Другой вариант решения для давления пара:
    Определим давление пара, если бы вся вода испарилась (что не факт):
    $$P_\text{max} = \frac{mRT'}{\mu V} = \frac{20\,\text{г} \cdot 8{,}31\,\frac{\text{Дж}}{\text{моль}\cdot\text{К}} \cdot 353\,\text{К}}{18\,\frac{\text{г}}{\text{моль}} \cdot 10\,\text{л}} \approx 330\,\text{кПа}.$$
    Сравниваем это давление с давлением насыщенного пара при этой температуре $P_\text{н.
    п.
    80 $\celsius$} = 47{,}300\,\text{кПа}$:
    если у нас получилось меньше табличного значения,
    то вся вода испарилась, если же больше — испарилась лишь часть, а пар является насыщенным.
    Отсюда сразу получаем давление пара: $P'_\text{пара} = 47{,}3\,\text{кПа}$.
    Сравните этот результат с первым вариантом решения.

    Тут получаем ответ: $P'_\text{пара} = 167{,}8\,\text{кПа}$.
}
\solutionspace{150pt}

\tasknumber{4}%
\task{%
    Напротив физических величин запишите определение, обозначение и единицы измерения в системе СИ (если есть):
    \begin{enumerate}
        \item относительная влажность,
        \item насыщенный пар.
    \end{enumerate}
}

\variantsplitter

\addpersonalvariant{Варвара Минаева}

\tasknumber{1}%
\task{%
    Сколько молекул водяного пара содержится в сосуде объёмом $7\,\text{л}$ при температуре $100\celsius$,
    и влажности воздуха $25\%$?
}
\answer{%
    Уравнение состояния идеального газа (и учтём, что $R = N_A \cdot k$,
    это чуть упростит выячичления, но вовсе не обязательно это делать):
    $$
        PV = \nu RT = \frac N{N_A} RT \implies N = PV \cdot \frac{N_A}{RT}=  \frac{PV}{kT}
    $$
    Плотность насыщенного водяного пара при $100\celsius$ ищем по таблице: $P_{\text{нас.
    пара 100} \celsius} = 101{,}300\,\text{кПа}.$

    Получаем плотность пара в сосуде $\varphi = \frac P{P_{\text{нас.
    пара 100} \celsius}} \implies P = \varphi P_{\text{нас.
    пара 100} \celsius}.$

    И подставляем в ответ (по сути, его можно было получить быстрее из формул $P = nkT, n = \frac NV$):
    $$
        N = \frac{\varphi \cdot P_{\text{нас.
        пара 100} \celsius} \cdot V}{kT}
         = \frac{0{,}25 \cdot 101{,}300\,\text{кПа} \cdot 7\,\text{л}}{1{,}38 \cdot 10^{-23}\,\frac{\text{Дж}}{\text{К}} \cdot 373\,\text{К}}
         \approx 340 \cdot 10^{20}.
    $$

    Другой вариант решения (через плотности) приводит в результату:
    $$
        N = N_A \nu = N_A \cdot \frac m{\mu}
          = N_A \frac{\rho V}{\mu}
          = N_A \frac{\varphi \cdot \rho_{\text{нас.
          пара 100} \celsius} \cdot V}{\mu}
          = 6{,}02 \cdot 10^{23}\,\frac{1}{\text{моль}} \cdot \frac{0{,}25 \cdot 598\,\frac{\text{г}}{\text{м}^{3}} \cdot 7\,\text{л}}{18\,\frac{\text{г}}{\text{моль}}}
          \approx 350 \cdot 10^{20}.
    $$
}
\solutionspace{160pt}

\tasknumber{2}%
\task{%
    В герметичном сосуде находится влажный воздух при температуре $25\celsius$ и относительной влажности $25\%$.
    \begin{enumerate}
        \item Чему равно парциальное давление насыщенного водяного пара при этой температуре?
        \item Чему равно парциальное давление водяного пара?
        \item Определите точку росы этого пара?
        \item Каким станет парциальное давление водяного пара, если сосуд нагреть до $90\celsius$?
        \item Чему будет равна относительная влажность воздуха, если сосуд нагреть до $90\celsius$?
        \item Получите ответ на предыдущий вопрос, используя плотности, а не давления.
    \end{enumerate}
}
\answer{%
    Парциальное давление насыщенного водяного пара при $25\celsius$ ищем по таблице: $$P_{\text{нас.
    пара 25} \celsius} = 3{,}170\,\text{кПа}.$$

    Парциальное давление водяного пара
    $$P_\text{пара 1} = \varphi_1 \cdot P_{\text{нас.
    пара 25} \celsius} = 0{,}25 \cdot 3{,}170\,\text{кПа} = 0{,}7925\,\text{кПа}.$$

    Точку росы определяем по таблице: при какой температуре пар с давлением $P_\text{пара 1} = 0{,}7925\,\text{кПа}$ станет насыщенным: $3{,}6\celsius$.

    После нагрева парциальное давление пара возрастёт:
    $$
        \frac{P_\text{пара 1} \cdot V}{T_1} = \nu R = \frac{P_\text{пара 2} \cdot V}{T_2}
        \implies P_\text{пара 2} = P_\text{пара 1} \cdot \frac{T_2}{T_1} = 0{,}7925\,\text{кПа} \cdot \frac{363\,\text{К}}{298\,\text{К}} \approx 0{,}9654\,\text{кПа}.
    $$

    Парциальное давление насыщенного водяного пара при $90\celsius$ ищем по таблице: $P_{\text{нас.
    пара 90} \celsius} = 70{,}100\,\text{кПа}$.
    Теперь определяем влажность:
    $$
        \varphi_2 = \frac{P_\text{пара 2}}{P_{\text{нас.
        пара 90} \celsius}} = \frac{0{,}9654\,\text{кПа}}{70{,}100\,\text{кПа}} \approx 0{,}014 = 1{,}4\%.
    $$

    Или же выражаем то же самое через плотности (плотность не изменяется при изохорном нагревании $\rho_1 =\rho_2 = \rho$, в отличие от давления):
    $$
        \varphi_2 = \frac{\rho}{\rho_{\text{нас.
        пара 90} \celsius}} = \frac{\varphi_1\rho_{\text{нас.
        пара 25} \celsius}}{\rho_{\text{нас.
        пара 90} \celsius}}
        = \frac{0{,}25 \cdot 23\,\frac{\text{г}}{\text{м}^{3}}}{424\,\frac{\text{г}}{\text{м}^{3}}} \approx 0{,}014 = 1{,}4\%.
    $$
    Сравните 2 последних результата.
}
\solutionspace{200pt}

\tasknumber{3}%
\task{%
    Закрытый сосуд объёмом $15\,\text{л}$ заполнен сухим воздухом при давлении $100\,\text{кПа}$ и температуре $10\celsius$.
    Каким станет давление в сосуде, если в него налить $5\,\text{г}$ воды и нагреть содержимое сосуда до $100\celsius$?
}
\answer{%
    Конечное давление газа в сосуде складывается по закону Дальтона из давления нагретого сухого воздуха $P'_\text{воздуха}$ и
    давления насыщенного пара $P_\text{пара}$:
    $$P' = P'_\text{воздуха} + P_\text{пара}.$$

    Сперва определим новое давление сухого воздуха из уравнения состояния идеального газа:
    $$
        \frac{P'_\text{воздуха} \cdot V}{T'} = \nu R = \frac{P \cdot V}{T}
        \implies P'_\text{воздуха} = P \cdot \frac{T'}{T} = 100\,\text{кПа} \cdot \frac{373\,\text{К}}{283\,\text{К}} \approx 132\,\text{кПа}.
    $$

    Чтобы найти давление пара, нужно понять, будет ли он насыщенным после нагрева или нет.

    Плотность насыщенного пара при температуре $100\celsius$ равна $598\,\frac{\text{г}}{\text{м}^{3}}$, тогда для того,
    чтобы весь сосуд был заполнен насыщенным водяным паром нужно
    $m_\text{н.
    п.} = \rho_\text{н.
    п.
    100 $\celsius$} \cdot V = 598\,\frac{\text{г}}{\text{м}^{3}} \cdot 15\,\text{л} \approx 9{,}0\,\text{г}$ воды.
    Сравнивая эту массу с массой воды из условия, получаем массу жидкости, которая испарится: $m_\text{пара} = 5\,\text{г}$.
    Осталось определить давление этого пара:
    $$P_\text{пара} = \frac{m_\text{пара}RT'}{\mu V} = \frac{5\,\text{г} \cdot 8{,}31\,\frac{\text{Дж}}{\text{моль}\cdot\text{К}} \cdot 373\,\text{К}}{18\,\frac{\text{г}}{\text{моль}} \cdot 15\,\text{л}} \approx 57\,\text{кПа}.$$

    Получаем ответ: $P'_\text{пара} = 189{,}2\,\text{кПа}$.

    Другой вариант решения для давления пара:
    Определим давление пара, если бы вся вода испарилась (что не факт):
    $$P_\text{max} = \frac{mRT'}{\mu V} = \frac{5\,\text{г} \cdot 8{,}31\,\frac{\text{Дж}}{\text{моль}\cdot\text{К}} \cdot 373\,\text{К}}{18\,\frac{\text{г}}{\text{моль}} \cdot 15\,\text{л}} \approx 57\,\text{кПа}.$$
    Сравниваем это давление с давлением насыщенного пара при этой температуре $P_\text{н.
    п.
    100 $\celsius$} = 101{,}300\,\text{кПа}$:
    если у нас получилось меньше табличного значения,
    то вся вода испарилась, если же больше — испарилась лишь часть, а пар является насыщенным.
    Отсюда сразу получаем давление пара: $P'_\text{пара} = 57{,}4\,\text{кПа}$.
    Сравните этот результат с первым вариантом решения.

    Тут получаем ответ: $P'_\text{пара} = 189{,}2\,\text{кПа}$.
}
\solutionspace{150pt}

\tasknumber{4}%
\task{%
    Напротив физических величин запишите определение, обозначение и единицы измерения в системе СИ (если есть):
    \begin{enumerate}
        \item абсолютная влажность,
        \item динамическое равновесие.
    \end{enumerate}
}

\variantsplitter

\addpersonalvariant{Леонид Никитин}

\tasknumber{1}%
\task{%
    Сколько молекул водяного пара содержится в сосуде объёмом $7\,\text{л}$ при температуре $80\celsius$,
    и влажности воздуха $40\%$?
}
\answer{%
    Уравнение состояния идеального газа (и учтём, что $R = N_A \cdot k$,
    это чуть упростит выячичления, но вовсе не обязательно это делать):
    $$
        PV = \nu RT = \frac N{N_A} RT \implies N = PV \cdot \frac{N_A}{RT}=  \frac{PV}{kT}
    $$
    Плотность насыщенного водяного пара при $80\celsius$ ищем по таблице: $P_{\text{нас.
    пара 80} \celsius} = 47{,}300\,\text{кПа}.$

    Получаем плотность пара в сосуде $\varphi = \frac P{P_{\text{нас.
    пара 80} \celsius}} \implies P = \varphi P_{\text{нас.
    пара 80} \celsius}.$

    И подставляем в ответ (по сути, его можно было получить быстрее из формул $P = nkT, n = \frac NV$):
    $$
        N = \frac{\varphi \cdot P_{\text{нас.
        пара 80} \celsius} \cdot V}{kT}
         = \frac{0{,}40 \cdot 47{,}300\,\text{кПа} \cdot 7\,\text{л}}{1{,}38 \cdot 10^{-23}\,\frac{\text{Дж}}{\text{К}} \cdot 353\,\text{К}}
         \approx 270 \cdot 10^{20}.
    $$

    Другой вариант решения (через плотности) приводит в результату:
    $$
        N = N_A \nu = N_A \cdot \frac m{\mu}
          = N_A \frac{\rho V}{\mu}
          = N_A \frac{\varphi \cdot \rho_{\text{нас.
          пара 80} \celsius} \cdot V}{\mu}
          = 6{,}02 \cdot 10^{23}\,\frac{1}{\text{моль}} \cdot \frac{0{,}40 \cdot 293\,\frac{\text{г}}{\text{м}^{3}} \cdot 7\,\text{л}}{18\,\frac{\text{г}}{\text{моль}}}
          \approx 270 \cdot 10^{20}.
    $$
}
\solutionspace{160pt}

\tasknumber{2}%
\task{%
    В герметичном сосуде находится влажный воздух при температуре $25\celsius$ и относительной влажности $20\%$.
    \begin{enumerate}
        \item Чему равно парциальное давление насыщенного водяного пара при этой температуре?
        \item Чему равно парциальное давление водяного пара?
        \item Определите точку росы этого пара?
        \item Каким станет парциальное давление водяного пара, если сосуд нагреть до $80\celsius$?
        \item Чему будет равна относительная влажность воздуха, если сосуд нагреть до $80\celsius$?
        \item Получите ответ на предыдущий вопрос, используя плотности, а не давления.
    \end{enumerate}
}
\answer{%
    Парциальное давление насыщенного водяного пара при $25\celsius$ ищем по таблице: $$P_{\text{нас.
    пара 25} \celsius} = 3{,}170\,\text{кПа}.$$

    Парциальное давление водяного пара
    $$P_\text{пара 1} = \varphi_1 \cdot P_{\text{нас.
    пара 25} \celsius} = 0{,}20 \cdot 3{,}170\,\text{кПа} = 0{,}6340\,\text{кПа}.$$

    Точку росы определяем по таблице: при какой температуре пар с давлением $P_\text{пара 1} = 0{,}6340\,\text{кПа}$ станет насыщенным: $0{,}5\celsius$.

    После нагрева парциальное давление пара возрастёт:
    $$
        \frac{P_\text{пара 1} \cdot V}{T_1} = \nu R = \frac{P_\text{пара 2} \cdot V}{T_2}
        \implies P_\text{пара 2} = P_\text{пара 1} \cdot \frac{T_2}{T_1} = 0{,}6340\,\text{кПа} \cdot \frac{353\,\text{К}}{298\,\text{К}} \approx 0{,}7510\,\text{кПа}.
    $$

    Парциальное давление насыщенного водяного пара при $80\celsius$ ищем по таблице: $P_{\text{нас.
    пара 80} \celsius} = 47{,}300\,\text{кПа}$.
    Теперь определяем влажность:
    $$
        \varphi_2 = \frac{P_\text{пара 2}}{P_{\text{нас.
        пара 80} \celsius}} = \frac{0{,}7510\,\text{кПа}}{47{,}300\,\text{кПа}} \approx 0{,}016 = 1{,}6\%.
    $$

    Или же выражаем то же самое через плотности (плотность не изменяется при изохорном нагревании $\rho_1 =\rho_2 = \rho$, в отличие от давления):
    $$
        \varphi_2 = \frac{\rho}{\rho_{\text{нас.
        пара 80} \celsius}} = \frac{\varphi_1\rho_{\text{нас.
        пара 25} \celsius}}{\rho_{\text{нас.
        пара 80} \celsius}}
        = \frac{0{,}20 \cdot 23\,\frac{\text{г}}{\text{м}^{3}}}{293\,\frac{\text{г}}{\text{м}^{3}}} \approx 0{,}016 = 1{,}6\%.
    $$
    Сравните 2 последних результата.
}
\solutionspace{200pt}

\tasknumber{3}%
\task{%
    Закрытый сосуд объёмом $10\,\text{л}$ заполнен сухим воздухом при давлении $100\,\text{кПа}$ и температуре $20\celsius$.
    Каким станет давление в сосуде, если в него налить $20\,\text{г}$ воды и нагреть содержимое сосуда до $90\celsius$?
}
\answer{%
    Конечное давление газа в сосуде складывается по закону Дальтона из давления нагретого сухого воздуха $P'_\text{воздуха}$ и
    давления насыщенного пара $P_\text{пара}$:
    $$P' = P'_\text{воздуха} + P_\text{пара}.$$

    Сперва определим новое давление сухого воздуха из уравнения состояния идеального газа:
    $$
        \frac{P'_\text{воздуха} \cdot V}{T'} = \nu R = \frac{P \cdot V}{T}
        \implies P'_\text{воздуха} = P \cdot \frac{T'}{T} = 100\,\text{кПа} \cdot \frac{363\,\text{К}}{293\,\text{К}} \approx 124\,\text{кПа}.
    $$

    Чтобы найти давление пара, нужно понять, будет ли он насыщенным после нагрева или нет.

    Плотность насыщенного пара при температуре $90\celsius$ равна $424\,\frac{\text{г}}{\text{м}^{3}}$, тогда для того,
    чтобы весь сосуд был заполнен насыщенным водяным паром нужно
    $m_\text{н.
    п.} = \rho_\text{н.
    п.
    90 $\celsius$} \cdot V = 424\,\frac{\text{г}}{\text{м}^{3}} \cdot 10\,\text{л} \approx 4{,}2\,\text{г}$ воды.
    Сравнивая эту массу с массой воды из условия, получаем массу жидкости, которая испарится: $m_\text{пара} = 4{,}2\,\text{г}$.
    Осталось определить давление этого пара:
    $$P_\text{пара} = \frac{m_\text{пара}RT'}{\mu V} = \frac{4{,}2\,\text{г} \cdot 8{,}31\,\frac{\text{Дж}}{\text{моль}\cdot\text{К}} \cdot 363\,\text{К}}{18\,\frac{\text{г}}{\text{моль}} \cdot 10\,\text{л}} \approx 70\,\text{кПа}.$$

    Получаем ответ: $P'_\text{пара} = 194{,}3\,\text{кПа}$.

    Другой вариант решения для давления пара:
    Определим давление пара, если бы вся вода испарилась (что не факт):
    $$P_\text{max} = \frac{mRT'}{\mu V} = \frac{20\,\text{г} \cdot 8{,}31\,\frac{\text{Дж}}{\text{моль}\cdot\text{К}} \cdot 363\,\text{К}}{18\,\frac{\text{г}}{\text{моль}} \cdot 10\,\text{л}} \approx 340\,\text{кПа}.$$
    Сравниваем это давление с давлением насыщенного пара при этой температуре $P_\text{н.
    п.
    90 $\celsius$} = 70{,}100\,\text{кПа}$:
    если у нас получилось меньше табличного значения,
    то вся вода испарилась, если же больше — испарилась лишь часть, а пар является насыщенным.
    Отсюда сразу получаем давление пара: $P'_\text{пара} = 70{,}1\,\text{кПа}$.
    Сравните этот результат с первым вариантом решения.

    Тут получаем ответ: $P'_\text{пара} = 194\,\text{кПа}$.
}
\solutionspace{150pt}

\tasknumber{4}%
\task{%
    Напротив физических величин запишите определение, обозначение и единицы измерения в системе СИ (если есть):
    \begin{enumerate}
        \item абсолютная влажность,
        \item динамическое равновесие.
    \end{enumerate}
}

\variantsplitter

\addpersonalvariant{Тимофей Полетаев}

\tasknumber{1}%
\task{%
    Сколько молекул водяного пара содержится в сосуде объёмом $15\,\text{л}$ при температуре $100\celsius$,
    и влажности воздуха $20\%$?
}
\answer{%
    Уравнение состояния идеального газа (и учтём, что $R = N_A \cdot k$,
    это чуть упростит выячичления, но вовсе не обязательно это делать):
    $$
        PV = \nu RT = \frac N{N_A} RT \implies N = PV \cdot \frac{N_A}{RT}=  \frac{PV}{kT}
    $$
    Плотность насыщенного водяного пара при $100\celsius$ ищем по таблице: $P_{\text{нас.
    пара 100} \celsius} = 101{,}300\,\text{кПа}.$

    Получаем плотность пара в сосуде $\varphi = \frac P{P_{\text{нас.
    пара 100} \celsius}} \implies P = \varphi P_{\text{нас.
    пара 100} \celsius}.$

    И подставляем в ответ (по сути, его можно было получить быстрее из формул $P = nkT, n = \frac NV$):
    $$
        N = \frac{\varphi \cdot P_{\text{нас.
        пара 100} \celsius} \cdot V}{kT}
         = \frac{0{,}20 \cdot 101{,}300\,\text{кПа} \cdot 15\,\text{л}}{1{,}38 \cdot 10^{-23}\,\frac{\text{Дж}}{\text{К}} \cdot 373\,\text{К}}
         \approx 590 \cdot 10^{20}.
    $$

    Другой вариант решения (через плотности) приводит в результату:
    $$
        N = N_A \nu = N_A \cdot \frac m{\mu}
          = N_A \frac{\rho V}{\mu}
          = N_A \frac{\varphi \cdot \rho_{\text{нас.
          пара 100} \celsius} \cdot V}{\mu}
          = 6{,}02 \cdot 10^{23}\,\frac{1}{\text{моль}} \cdot \frac{0{,}20 \cdot 598\,\frac{\text{г}}{\text{м}^{3}} \cdot 15\,\text{л}}{18\,\frac{\text{г}}{\text{моль}}}
          \approx 600 \cdot 10^{20}.
    $$
}
\solutionspace{160pt}

\tasknumber{2}%
\task{%
    В герметичном сосуде находится влажный воздух при температуре $15\celsius$ и относительной влажности $25\%$.
    \begin{enumerate}
        \item Чему равно парциальное давление насыщенного водяного пара при этой температуре?
        \item Чему равно парциальное давление водяного пара?
        \item Определите точку росы этого пара?
        \item Каким станет парциальное давление водяного пара, если сосуд нагреть до $70\celsius$?
        \item Чему будет равна относительная влажность воздуха, если сосуд нагреть до $70\celsius$?
        \item Получите ответ на предыдущий вопрос, используя плотности, а не давления.
    \end{enumerate}
}
\answer{%
    Парциальное давление насыщенного водяного пара при $15\celsius$ ищем по таблице: $$P_{\text{нас.
    пара 15} \celsius} = 1{,}700\,\text{кПа}.$$

    Парциальное давление водяного пара
    $$P_\text{пара 1} = \varphi_1 \cdot P_{\text{нас.
    пара 15} \celsius} = 0{,}25 \cdot 1{,}700\,\text{кПа} = 0{,}425\,\text{кПа}.$$

    Точку росы определяем по таблице: при какой температуре пар с давлением $P_\text{пара 1} = 0{,}425\,\text{кПа}$ станет насыщенным: $0{,}0\celsius$.

    После нагрева парциальное давление пара возрастёт:
    $$
        \frac{P_\text{пара 1} \cdot V}{T_1} = \nu R = \frac{P_\text{пара 2} \cdot V}{T_2}
        \implies P_\text{пара 2} = P_\text{пара 1} \cdot \frac{T_2}{T_1} = 0{,}425\,\text{кПа} \cdot \frac{343\,\text{К}}{288\,\text{К}} \approx 0{,}506\,\text{кПа}.
    $$

    Парциальное давление насыщенного водяного пара при $70\celsius$ ищем по таблице: $P_{\text{нас.
    пара 70} \celsius} = 31\,\text{кПа}$.
    Теперь определяем влажность:
    $$
        \varphi_2 = \frac{P_\text{пара 2}}{P_{\text{нас.
        пара 70} \celsius}} = \frac{0{,}506\,\text{кПа}}{31\,\text{кПа}} \approx 0{,}016 = 1{,}6\%.
    $$

    Или же выражаем то же самое через плотности (плотность не изменяется при изохорном нагревании $\rho_1 =\rho_2 = \rho$, в отличие от давления):
    $$
        \varphi_2 = \frac{\rho}{\rho_{\text{нас.
        пара 70} \celsius}} = \frac{\varphi_1\rho_{\text{нас.
        пара 15} \celsius}}{\rho_{\text{нас.
        пара 70} \celsius}}
        = \frac{0{,}25 \cdot 12{,}80\,\frac{\text{г}}{\text{м}^{3}}}{198\,\frac{\text{г}}{\text{м}^{3}}} \approx 0{,}016 = 1{,}6\%.
    $$
    Сравните 2 последних результата.
}
\solutionspace{200pt}

\tasknumber{3}%
\task{%
    Закрытый сосуд объёмом $15\,\text{л}$ заполнен сухим воздухом при давлении $100\,\text{кПа}$ и температуре $20\celsius$.
    Каким станет давление в сосуде, если в него налить $5\,\text{г}$ воды и нагреть содержимое сосуда до $90\celsius$?
}
\answer{%
    Конечное давление газа в сосуде складывается по закону Дальтона из давления нагретого сухого воздуха $P'_\text{воздуха}$ и
    давления насыщенного пара $P_\text{пара}$:
    $$P' = P'_\text{воздуха} + P_\text{пара}.$$

    Сперва определим новое давление сухого воздуха из уравнения состояния идеального газа:
    $$
        \frac{P'_\text{воздуха} \cdot V}{T'} = \nu R = \frac{P \cdot V}{T}
        \implies P'_\text{воздуха} = P \cdot \frac{T'}{T} = 100\,\text{кПа} \cdot \frac{363\,\text{К}}{293\,\text{К}} \approx 124\,\text{кПа}.
    $$

    Чтобы найти давление пара, нужно понять, будет ли он насыщенным после нагрева или нет.

    Плотность насыщенного пара при температуре $90\celsius$ равна $424\,\frac{\text{г}}{\text{м}^{3}}$, тогда для того,
    чтобы весь сосуд был заполнен насыщенным водяным паром нужно
    $m_\text{н.
    п.} = \rho_\text{н.
    п.
    90 $\celsius$} \cdot V = 424\,\frac{\text{г}}{\text{м}^{3}} \cdot 15\,\text{л} \approx 6{,}4\,\text{г}$ воды.
    Сравнивая эту массу с массой воды из условия, получаем массу жидкости, которая испарится: $m_\text{пара} = 5\,\text{г}$.
    Осталось определить давление этого пара:
    $$P_\text{пара} = \frac{m_\text{пара}RT'}{\mu V} = \frac{5\,\text{г} \cdot 8{,}31\,\frac{\text{Дж}}{\text{моль}\cdot\text{К}} \cdot 363\,\text{К}}{18\,\frac{\text{г}}{\text{моль}} \cdot 15\,\text{л}} \approx 56\,\text{кПа}.$$

    Получаем ответ: $P'_\text{пара} = 179{,}8\,\text{кПа}$.

    Другой вариант решения для давления пара:
    Определим давление пара, если бы вся вода испарилась (что не факт):
    $$P_\text{max} = \frac{mRT'}{\mu V} = \frac{5\,\text{г} \cdot 8{,}31\,\frac{\text{Дж}}{\text{моль}\cdot\text{К}} \cdot 363\,\text{К}}{18\,\frac{\text{г}}{\text{моль}} \cdot 15\,\text{л}} \approx 56\,\text{кПа}.$$
    Сравниваем это давление с давлением насыщенного пара при этой температуре $P_\text{н.
    п.
    90 $\celsius$} = 70{,}100\,\text{кПа}$:
    если у нас получилось меньше табличного значения,
    то вся вода испарилась, если же больше — испарилась лишь часть, а пар является насыщенным.
    Отсюда сразу получаем давление пара: $P'_\text{пара} = 55{,}9\,\text{кПа}$.
    Сравните этот результат с первым вариантом решения.

    Тут получаем ответ: $P'_\text{пара} = 179{,}8\,\text{кПа}$.
}
\solutionspace{150pt}

\tasknumber{4}%
\task{%
    Напротив физических величин запишите определение, обозначение и единицы измерения в системе СИ (если есть):
    \begin{enumerate}
        \item относительная влажность,
        \item динамическое равновесие.
    \end{enumerate}
}

\variantsplitter

\addpersonalvariant{Андрей Рожков}

\tasknumber{1}%
\task{%
    Сколько молекул водяного пара содержится в сосуде объёмом $12\,\text{л}$ при температуре $30\celsius$,
    и влажности воздуха $75\%$?
}
\answer{%
    Уравнение состояния идеального газа (и учтём, что $R = N_A \cdot k$,
    это чуть упростит выячичления, но вовсе не обязательно это делать):
    $$
        PV = \nu RT = \frac N{N_A} RT \implies N = PV \cdot \frac{N_A}{RT}=  \frac{PV}{kT}
    $$
    Плотность насыщенного водяного пара при $30\celsius$ ищем по таблице: $P_{\text{нас.
    пара 30} \celsius} = 4{,}240\,\text{кПа}.$

    Получаем плотность пара в сосуде $\varphi = \frac P{P_{\text{нас.
    пара 30} \celsius}} \implies P = \varphi P_{\text{нас.
    пара 30} \celsius}.$

    И подставляем в ответ (по сути, его можно было получить быстрее из формул $P = nkT, n = \frac NV$):
    $$
        N = \frac{\varphi \cdot P_{\text{нас.
        пара 30} \celsius} \cdot V}{kT}
         = \frac{0{,}75 \cdot 4{,}240\,\text{кПа} \cdot 12\,\text{л}}{1{,}38 \cdot 10^{-23}\,\frac{\text{Дж}}{\text{К}} \cdot 303\,\text{К}}
         \approx 91 \cdot 10^{20}.
    $$

    Другой вариант решения (через плотности) приводит в результату:
    $$
        N = N_A \nu = N_A \cdot \frac m{\mu}
          = N_A \frac{\rho V}{\mu}
          = N_A \frac{\varphi \cdot \rho_{\text{нас.
          пара 30} \celsius} \cdot V}{\mu}
          = 6{,}02 \cdot 10^{23}\,\frac{1}{\text{моль}} \cdot \frac{0{,}75 \cdot 30{,}30\,\frac{\text{г}}{\text{м}^{3}} \cdot 12\,\text{л}}{18\,\frac{\text{г}}{\text{моль}}}
          \approx 91 \cdot 10^{20}.
    $$
}
\solutionspace{160pt}

\tasknumber{2}%
\task{%
    В герметичном сосуде находится влажный воздух при температуре $40\celsius$ и относительной влажности $40\%$.
    \begin{enumerate}
        \item Чему равно парциальное давление насыщенного водяного пара при этой температуре?
        \item Чему равно парциальное давление водяного пара?
        \item Определите точку росы этого пара?
        \item Каким станет парциальное давление водяного пара, если сосуд нагреть до $90\celsius$?
        \item Чему будет равна относительная влажность воздуха, если сосуд нагреть до $90\celsius$?
        \item Получите ответ на предыдущий вопрос, используя плотности, а не давления.
    \end{enumerate}
}
\answer{%
    Парциальное давление насыщенного водяного пара при $40\celsius$ ищем по таблице: $$P_{\text{нас.
    пара 40} \celsius} = 7{,}370\,\text{кПа}.$$

    Парциальное давление водяного пара
    $$P_\text{пара 1} = \varphi_1 \cdot P_{\text{нас.
    пара 40} \celsius} = 0{,}40 \cdot 7{,}370\,\text{кПа} = 2{,}948\,\text{кПа}.$$

    Точку росы определяем по таблице: при какой температуре пар с давлением $P_\text{пара 1} = 2{,}948\,\text{кПа}$ станет насыщенным: $23{,}8\celsius$.

    После нагрева парциальное давление пара возрастёт:
    $$
        \frac{P_\text{пара 1} \cdot V}{T_1} = \nu R = \frac{P_\text{пара 2} \cdot V}{T_2}
        \implies P_\text{пара 2} = P_\text{пара 1} \cdot \frac{T_2}{T_1} = 2{,}948\,\text{кПа} \cdot \frac{363\,\text{К}}{313\,\text{К}} \approx 3{,}419\,\text{кПа}.
    $$

    Парциальное давление насыщенного водяного пара при $90\celsius$ ищем по таблице: $P_{\text{нас.
    пара 90} \celsius} = 70{,}100\,\text{кПа}$.
    Теперь определяем влажность:
    $$
        \varphi_2 = \frac{P_\text{пара 2}}{P_{\text{нас.
        пара 90} \celsius}} = \frac{3{,}419\,\text{кПа}}{70{,}100\,\text{кПа}} \approx 0{,}049 = 4{,}9\%.
    $$

    Или же выражаем то же самое через плотности (плотность не изменяется при изохорном нагревании $\rho_1 =\rho_2 = \rho$, в отличие от давления):
    $$
        \varphi_2 = \frac{\rho}{\rho_{\text{нас.
        пара 90} \celsius}} = \frac{\varphi_1\rho_{\text{нас.
        пара 40} \celsius}}{\rho_{\text{нас.
        пара 90} \celsius}}
        = \frac{0{,}40 \cdot 51{,}20\,\frac{\text{г}}{\text{м}^{3}}}{424\,\frac{\text{г}}{\text{м}^{3}}} \approx 0{,}048 = 4{,}8\%.
    $$
    Сравните 2 последних результата.
}
\solutionspace{200pt}

\tasknumber{3}%
\task{%
    Закрытый сосуд объёмом $20\,\text{л}$ заполнен сухим воздухом при давлении $100\,\text{кПа}$ и температуре $20\celsius$.
    Каким станет давление в сосуде, если в него налить $20\,\text{г}$ воды и нагреть содержимое сосуда до $80\celsius$?
}
\answer{%
    Конечное давление газа в сосуде складывается по закону Дальтона из давления нагретого сухого воздуха $P'_\text{воздуха}$ и
    давления насыщенного пара $P_\text{пара}$:
    $$P' = P'_\text{воздуха} + P_\text{пара}.$$

    Сперва определим новое давление сухого воздуха из уравнения состояния идеального газа:
    $$
        \frac{P'_\text{воздуха} \cdot V}{T'} = \nu R = \frac{P \cdot V}{T}
        \implies P'_\text{воздуха} = P \cdot \frac{T'}{T} = 100\,\text{кПа} \cdot \frac{353\,\text{К}}{293\,\text{К}} \approx 120\,\text{кПа}.
    $$

    Чтобы найти давление пара, нужно понять, будет ли он насыщенным после нагрева или нет.

    Плотность насыщенного пара при температуре $80\celsius$ равна $293\,\frac{\text{г}}{\text{м}^{3}}$, тогда для того,
    чтобы весь сосуд был заполнен насыщенным водяным паром нужно
    $m_\text{н.
    п.} = \rho_\text{н.
    п.
    80 $\celsius$} \cdot V = 293\,\frac{\text{г}}{\text{м}^{3}} \cdot 20\,\text{л} \approx 5{,}86\,\text{г}$ воды.
    Сравнивая эту массу с массой воды из условия, получаем массу жидкости, которая испарится: $m_\text{пара} = 5{,}9\,\text{г}$.
    Осталось определить давление этого пара:
    $$P_\text{пара} = \frac{m_\text{пара}RT'}{\mu V} = \frac{5{,}9\,\text{г} \cdot 8{,}31\,\frac{\text{Дж}}{\text{моль}\cdot\text{К}} \cdot 353\,\text{К}}{18\,\frac{\text{г}}{\text{моль}} \cdot 20\,\text{л}} \approx 48\,\text{кПа}.$$

    Получаем ответ: $P'_\text{пара} = 168{,}6\,\text{кПа}$.

    Другой вариант решения для давления пара:
    Определим давление пара, если бы вся вода испарилась (что не факт):
    $$P_\text{max} = \frac{mRT'}{\mu V} = \frac{20\,\text{г} \cdot 8{,}31\,\frac{\text{Дж}}{\text{моль}\cdot\text{К}} \cdot 353\,\text{К}}{18\,\frac{\text{г}}{\text{моль}} \cdot 20\,\text{л}} \approx 163\,\text{кПа}.$$
    Сравниваем это давление с давлением насыщенного пара при этой температуре $P_\text{н.
    п.
    80 $\celsius$} = 47{,}300\,\text{кПа}$:
    если у нас получилось меньше табличного значения,
    то вся вода испарилась, если же больше — испарилась лишь часть, а пар является насыщенным.
    Отсюда сразу получаем давление пара: $P'_\text{пара} = 47{,}3\,\text{кПа}$.
    Сравните этот результат с первым вариантом решения.

    Тут получаем ответ: $P'_\text{пара} = 167{,}8\,\text{кПа}$.
}
\solutionspace{150pt}

\tasknumber{4}%
\task{%
    Напротив физических величин запишите определение, обозначение и единицы измерения в системе СИ (если есть):
    \begin{enumerate}
        \item относительная влажность,
        \item динамическое равновесие.
    \end{enumerate}
}

\variantsplitter

\addpersonalvariant{Рената Таржиманова}

\tasknumber{1}%
\task{%
    Сколько молекул водяного пара содержится в сосуде объёмом $12\,\text{л}$ при температуре $20\celsius$,
    и влажности воздуха $35\%$?
}
\answer{%
    Уравнение состояния идеального газа (и учтём, что $R = N_A \cdot k$,
    это чуть упростит выячичления, но вовсе не обязательно это делать):
    $$
        PV = \nu RT = \frac N{N_A} RT \implies N = PV \cdot \frac{N_A}{RT}=  \frac{PV}{kT}
    $$
    Плотность насыщенного водяного пара при $20\celsius$ ищем по таблице: $P_{\text{нас.
    пара 20} \celsius} = 2{,}340\,\text{кПа}.$

    Получаем плотность пара в сосуде $\varphi = \frac P{P_{\text{нас.
    пара 20} \celsius}} \implies P = \varphi P_{\text{нас.
    пара 20} \celsius}.$

    И подставляем в ответ (по сути, его можно было получить быстрее из формул $P = nkT, n = \frac NV$):
    $$
        N = \frac{\varphi \cdot P_{\text{нас.
        пара 20} \celsius} \cdot V}{kT}
         = \frac{0{,}35 \cdot 2{,}340\,\text{кПа} \cdot 12\,\text{л}}{1{,}38 \cdot 10^{-23}\,\frac{\text{Дж}}{\text{К}} \cdot 293\,\text{К}}
         \approx 24 \cdot 10^{20}.
    $$

    Другой вариант решения (через плотности) приводит в результату:
    $$
        N = N_A \nu = N_A \cdot \frac m{\mu}
          = N_A \frac{\rho V}{\mu}
          = N_A \frac{\varphi \cdot \rho_{\text{нас.
          пара 20} \celsius} \cdot V}{\mu}
          = 6{,}02 \cdot 10^{23}\,\frac{1}{\text{моль}} \cdot \frac{0{,}35 \cdot 17{,}30\,\frac{\text{г}}{\text{м}^{3}} \cdot 12\,\text{л}}{18\,\frac{\text{г}}{\text{моль}}}
          \approx 24 \cdot 10^{20}.
    $$
}
\solutionspace{160pt}

\tasknumber{2}%
\task{%
    В герметичном сосуде находится влажный воздух при температуре $15\celsius$ и относительной влажности $55\%$.
    \begin{enumerate}
        \item Чему равно парциальное давление насыщенного водяного пара при этой температуре?
        \item Чему равно парциальное давление водяного пара?
        \item Определите точку росы этого пара?
        \item Каким станет парциальное давление водяного пара, если сосуд нагреть до $80\celsius$?
        \item Чему будет равна относительная влажность воздуха, если сосуд нагреть до $80\celsius$?
        \item Получите ответ на предыдущий вопрос, используя плотности, а не давления.
    \end{enumerate}
}
\answer{%
    Парциальное давление насыщенного водяного пара при $15\celsius$ ищем по таблице: $$P_{\text{нас.
    пара 15} \celsius} = 1{,}700\,\text{кПа}.$$

    Парциальное давление водяного пара
    $$P_\text{пара 1} = \varphi_1 \cdot P_{\text{нас.
    пара 15} \celsius} = 0{,}55 \cdot 1{,}700\,\text{кПа} = 0{,}935\,\text{кПа}.$$

    Точку росы определяем по таблице: при какой температуре пар с давлением $P_\text{пара 1} = 0{,}935\,\text{кПа}$ станет насыщенным: $6{,}0\celsius$.

    После нагрева парциальное давление пара возрастёт:
    $$
        \frac{P_\text{пара 1} \cdot V}{T_1} = \nu R = \frac{P_\text{пара 2} \cdot V}{T_2}
        \implies P_\text{пара 2} = P_\text{пара 1} \cdot \frac{T_2}{T_1} = 0{,}935\,\text{кПа} \cdot \frac{353\,\text{К}}{288\,\text{К}} \approx 1{,}146\,\text{кПа}.
    $$

    Парциальное давление насыщенного водяного пара при $80\celsius$ ищем по таблице: $P_{\text{нас.
    пара 80} \celsius} = 47{,}300\,\text{кПа}$.
    Теперь определяем влажность:
    $$
        \varphi_2 = \frac{P_\text{пара 2}}{P_{\text{нас.
        пара 80} \celsius}} = \frac{1{,}146\,\text{кПа}}{47{,}300\,\text{кПа}} \approx 0{,}024 = 2{,}4\%.
    $$

    Или же выражаем то же самое через плотности (плотность не изменяется при изохорном нагревании $\rho_1 =\rho_2 = \rho$, в отличие от давления):
    $$
        \varphi_2 = \frac{\rho}{\rho_{\text{нас.
        пара 80} \celsius}} = \frac{\varphi_1\rho_{\text{нас.
        пара 15} \celsius}}{\rho_{\text{нас.
        пара 80} \celsius}}
        = \frac{0{,}55 \cdot 12{,}80\,\frac{\text{г}}{\text{м}^{3}}}{293\,\frac{\text{г}}{\text{м}^{3}}} \approx 0{,}024 = 2{,}4\%.
    $$
    Сравните 2 последних результата.
}
\solutionspace{200pt}

\tasknumber{3}%
\task{%
    Закрытый сосуд объёмом $20\,\text{л}$ заполнен сухим воздухом при давлении $100\,\text{кПа}$ и температуре $20\celsius$.
    Каким станет давление в сосуде, если в него налить $10\,\text{г}$ воды и нагреть содержимое сосуда до $90\celsius$?
}
\answer{%
    Конечное давление газа в сосуде складывается по закону Дальтона из давления нагретого сухого воздуха $P'_\text{воздуха}$ и
    давления насыщенного пара $P_\text{пара}$:
    $$P' = P'_\text{воздуха} + P_\text{пара}.$$

    Сперва определим новое давление сухого воздуха из уравнения состояния идеального газа:
    $$
        \frac{P'_\text{воздуха} \cdot V}{T'} = \nu R = \frac{P \cdot V}{T}
        \implies P'_\text{воздуха} = P \cdot \frac{T'}{T} = 100\,\text{кПа} \cdot \frac{363\,\text{К}}{293\,\text{К}} \approx 124\,\text{кПа}.
    $$

    Чтобы найти давление пара, нужно понять, будет ли он насыщенным после нагрева или нет.

    Плотность насыщенного пара при температуре $90\celsius$ равна $424\,\frac{\text{г}}{\text{м}^{3}}$, тогда для того,
    чтобы весь сосуд был заполнен насыщенным водяным паром нужно
    $m_\text{н.
    п.} = \rho_\text{н.
    п.
    90 $\celsius$} \cdot V = 424\,\frac{\text{г}}{\text{м}^{3}} \cdot 20\,\text{л} \approx 8{,}48\,\text{г}$ воды.
    Сравнивая эту массу с массой воды из условия, получаем массу жидкости, которая испарится: $m_\text{пара} = 8{,}5\,\text{г}$.
    Осталось определить давление этого пара:
    $$P_\text{пара} = \frac{m_\text{пара}RT'}{\mu V} = \frac{8{,}5\,\text{г} \cdot 8{,}31\,\frac{\text{Дж}}{\text{моль}\cdot\text{К}} \cdot 363\,\text{К}}{18\,\frac{\text{г}}{\text{моль}} \cdot 20\,\text{л}} \approx 71\,\text{кПа}.$$

    Получаем ответ: $P'_\text{пара} = 195{,}1\,\text{кПа}$.

    Другой вариант решения для давления пара:
    Определим давление пара, если бы вся вода испарилась (что не факт):
    $$P_\text{max} = \frac{mRT'}{\mu V} = \frac{10\,\text{г} \cdot 8{,}31\,\frac{\text{Дж}}{\text{моль}\cdot\text{К}} \cdot 363\,\text{К}}{18\,\frac{\text{г}}{\text{моль}} \cdot 20\,\text{л}} \approx 84\,\text{кПа}.$$
    Сравниваем это давление с давлением насыщенного пара при этой температуре $P_\text{н.
    п.
    90 $\celsius$} = 70{,}100\,\text{кПа}$:
    если у нас получилось меньше табличного значения,
    то вся вода испарилась, если же больше — испарилась лишь часть, а пар является насыщенным.
    Отсюда сразу получаем давление пара: $P'_\text{пара} = 70{,}1\,\text{кПа}$.
    Сравните этот результат с первым вариантом решения.

    Тут получаем ответ: $P'_\text{пара} = 194\,\text{кПа}$.
}
\solutionspace{150pt}

\tasknumber{4}%
\task{%
    Напротив физических величин запишите определение, обозначение и единицы измерения в системе СИ (если есть):
    \begin{enumerate}
        \item абсолютная влажность,
        \item насыщенный пар.
    \end{enumerate}
}

\variantsplitter

\addpersonalvariant{Андрей Щербаков}

\tasknumber{1}%
\task{%
    Сколько молекул водяного пара содержится в сосуде объёмом $6\,\text{л}$ при температуре $20\celsius$,
    и влажности воздуха $65\%$?
}
\answer{%
    Уравнение состояния идеального газа (и учтём, что $R = N_A \cdot k$,
    это чуть упростит выячичления, но вовсе не обязательно это делать):
    $$
        PV = \nu RT = \frac N{N_A} RT \implies N = PV \cdot \frac{N_A}{RT}=  \frac{PV}{kT}
    $$
    Плотность насыщенного водяного пара при $20\celsius$ ищем по таблице: $P_{\text{нас.
    пара 20} \celsius} = 2{,}340\,\text{кПа}.$

    Получаем плотность пара в сосуде $\varphi = \frac P{P_{\text{нас.
    пара 20} \celsius}} \implies P = \varphi P_{\text{нас.
    пара 20} \celsius}.$

    И подставляем в ответ (по сути, его можно было получить быстрее из формул $P = nkT, n = \frac NV$):
    $$
        N = \frac{\varphi \cdot P_{\text{нас.
        пара 20} \celsius} \cdot V}{kT}
         = \frac{0{,}65 \cdot 2{,}340\,\text{кПа} \cdot 6\,\text{л}}{1{,}38 \cdot 10^{-23}\,\frac{\text{Дж}}{\text{К}} \cdot 293\,\text{К}}
         \approx 23 \cdot 10^{20}.
    $$

    Другой вариант решения (через плотности) приводит в результату:
    $$
        N = N_A \nu = N_A \cdot \frac m{\mu}
          = N_A \frac{\rho V}{\mu}
          = N_A \frac{\varphi \cdot \rho_{\text{нас.
          пара 20} \celsius} \cdot V}{\mu}
          = 6{,}02 \cdot 10^{23}\,\frac{1}{\text{моль}} \cdot \frac{0{,}65 \cdot 17{,}30\,\frac{\text{г}}{\text{м}^{3}} \cdot 6\,\text{л}}{18\,\frac{\text{г}}{\text{моль}}}
          \approx 23 \cdot 10^{20}.
    $$
}
\solutionspace{160pt}

\tasknumber{2}%
\task{%
    В герметичном сосуде находится влажный воздух при температуре $25\celsius$ и относительной влажности $40\%$.
    \begin{enumerate}
        \item Чему равно парциальное давление насыщенного водяного пара при этой температуре?
        \item Чему равно парциальное давление водяного пара?
        \item Определите точку росы этого пара?
        \item Каким станет парциальное давление водяного пара, если сосуд нагреть до $90\celsius$?
        \item Чему будет равна относительная влажность воздуха, если сосуд нагреть до $90\celsius$?
        \item Получите ответ на предыдущий вопрос, используя плотности, а не давления.
    \end{enumerate}
}
\answer{%
    Парциальное давление насыщенного водяного пара при $25\celsius$ ищем по таблице: $$P_{\text{нас.
    пара 25} \celsius} = 3{,}170\,\text{кПа}.$$

    Парциальное давление водяного пара
    $$P_\text{пара 1} = \varphi_1 \cdot P_{\text{нас.
    пара 25} \celsius} = 0{,}40 \cdot 3{,}170\,\text{кПа} = 1{,}2680\,\text{кПа}.$$

    Точку росы определяем по таблице: при какой температуре пар с давлением $P_\text{пара 1} = 1{,}2680\,\text{кПа}$ станет насыщенным: $10{,}5\celsius$.

    После нагрева парциальное давление пара возрастёт:
    $$
        \frac{P_\text{пара 1} \cdot V}{T_1} = \nu R = \frac{P_\text{пара 2} \cdot V}{T_2}
        \implies P_\text{пара 2} = P_\text{пара 1} \cdot \frac{T_2}{T_1} = 1{,}2680\,\text{кПа} \cdot \frac{363\,\text{К}}{298\,\text{К}} \approx 1{,}5446\,\text{кПа}.
    $$

    Парциальное давление насыщенного водяного пара при $90\celsius$ ищем по таблице: $P_{\text{нас.
    пара 90} \celsius} = 70{,}100\,\text{кПа}$.
    Теперь определяем влажность:
    $$
        \varphi_2 = \frac{P_\text{пара 2}}{P_{\text{нас.
        пара 90} \celsius}} = \frac{1{,}5446\,\text{кПа}}{70{,}100\,\text{кПа}} \approx 0{,}022 = 2{,}2\%.
    $$

    Или же выражаем то же самое через плотности (плотность не изменяется при изохорном нагревании $\rho_1 =\rho_2 = \rho$, в отличие от давления):
    $$
        \varphi_2 = \frac{\rho}{\rho_{\text{нас.
        пара 90} \celsius}} = \frac{\varphi_1\rho_{\text{нас.
        пара 25} \celsius}}{\rho_{\text{нас.
        пара 90} \celsius}}
        = \frac{0{,}40 \cdot 23\,\frac{\text{г}}{\text{м}^{3}}}{424\,\frac{\text{г}}{\text{м}^{3}}} \approx 0{,}022 = 2{,}2\%.
    $$
    Сравните 2 последних результата.
}
\solutionspace{200pt}

\tasknumber{3}%
\task{%
    Закрытый сосуд объёмом $15\,\text{л}$ заполнен сухим воздухом при давлении $100\,\text{кПа}$ и температуре $20\celsius$.
    Каким станет давление в сосуде, если в него налить $5\,\text{г}$ воды и нагреть содержимое сосуда до $90\celsius$?
}
\answer{%
    Конечное давление газа в сосуде складывается по закону Дальтона из давления нагретого сухого воздуха $P'_\text{воздуха}$ и
    давления насыщенного пара $P_\text{пара}$:
    $$P' = P'_\text{воздуха} + P_\text{пара}.$$

    Сперва определим новое давление сухого воздуха из уравнения состояния идеального газа:
    $$
        \frac{P'_\text{воздуха} \cdot V}{T'} = \nu R = \frac{P \cdot V}{T}
        \implies P'_\text{воздуха} = P \cdot \frac{T'}{T} = 100\,\text{кПа} \cdot \frac{363\,\text{К}}{293\,\text{К}} \approx 124\,\text{кПа}.
    $$

    Чтобы найти давление пара, нужно понять, будет ли он насыщенным после нагрева или нет.

    Плотность насыщенного пара при температуре $90\celsius$ равна $424\,\frac{\text{г}}{\text{м}^{3}}$, тогда для того,
    чтобы весь сосуд был заполнен насыщенным водяным паром нужно
    $m_\text{н.
    п.} = \rho_\text{н.
    п.
    90 $\celsius$} \cdot V = 424\,\frac{\text{г}}{\text{м}^{3}} \cdot 15\,\text{л} \approx 6{,}4\,\text{г}$ воды.
    Сравнивая эту массу с массой воды из условия, получаем массу жидкости, которая испарится: $m_\text{пара} = 5\,\text{г}$.
    Осталось определить давление этого пара:
    $$P_\text{пара} = \frac{m_\text{пара}RT'}{\mu V} = \frac{5\,\text{г} \cdot 8{,}31\,\frac{\text{Дж}}{\text{моль}\cdot\text{К}} \cdot 363\,\text{К}}{18\,\frac{\text{г}}{\text{моль}} \cdot 15\,\text{л}} \approx 56\,\text{кПа}.$$

    Получаем ответ: $P'_\text{пара} = 179{,}8\,\text{кПа}$.

    Другой вариант решения для давления пара:
    Определим давление пара, если бы вся вода испарилась (что не факт):
    $$P_\text{max} = \frac{mRT'}{\mu V} = \frac{5\,\text{г} \cdot 8{,}31\,\frac{\text{Дж}}{\text{моль}\cdot\text{К}} \cdot 363\,\text{К}}{18\,\frac{\text{г}}{\text{моль}} \cdot 15\,\text{л}} \approx 56\,\text{кПа}.$$
    Сравниваем это давление с давлением насыщенного пара при этой температуре $P_\text{н.
    п.
    90 $\celsius$} = 70{,}100\,\text{кПа}$:
    если у нас получилось меньше табличного значения,
    то вся вода испарилась, если же больше — испарилась лишь часть, а пар является насыщенным.
    Отсюда сразу получаем давление пара: $P'_\text{пара} = 55{,}9\,\text{кПа}$.
    Сравните этот результат с первым вариантом решения.

    Тут получаем ответ: $P'_\text{пара} = 179{,}8\,\text{кПа}$.
}
\solutionspace{150pt}

\tasknumber{4}%
\task{%
    Напротив физических величин запишите определение, обозначение и единицы измерения в системе СИ (если есть):
    \begin{enumerate}
        \item абсолютная влажность,
        \item насыщенный пар.
    \end{enumerate}
}

\variantsplitter

\addpersonalvariant{Михаил Ярошевский}

\tasknumber{1}%
\task{%
    Сколько молекул водяного пара содержится в сосуде объёмом $3\,\text{л}$ при температуре $60\celsius$,
    и влажности воздуха $25\%$?
}
\answer{%
    Уравнение состояния идеального газа (и учтём, что $R = N_A \cdot k$,
    это чуть упростит выячичления, но вовсе не обязательно это делать):
    $$
        PV = \nu RT = \frac N{N_A} RT \implies N = PV \cdot \frac{N_A}{RT}=  \frac{PV}{kT}
    $$
    Плотность насыщенного водяного пара при $60\celsius$ ищем по таблице: $P_{\text{нас.
    пара 60} \celsius} = 19{,}900\,\text{кПа}.$

    Получаем плотность пара в сосуде $\varphi = \frac P{P_{\text{нас.
    пара 60} \celsius}} \implies P = \varphi P_{\text{нас.
    пара 60} \celsius}.$

    И подставляем в ответ (по сути, его можно было получить быстрее из формул $P = nkT, n = \frac NV$):
    $$
        N = \frac{\varphi \cdot P_{\text{нас.
        пара 60} \celsius} \cdot V}{kT}
         = \frac{0{,}25 \cdot 19{,}900\,\text{кПа} \cdot 3\,\text{л}}{1{,}38 \cdot 10^{-23}\,\frac{\text{Дж}}{\text{К}} \cdot 333\,\text{К}}
         \approx 32 \cdot 10^{20}.
    $$

    Другой вариант решения (через плотности) приводит в результату:
    $$
        N = N_A \nu = N_A \cdot \frac m{\mu}
          = N_A \frac{\rho V}{\mu}
          = N_A \frac{\varphi \cdot \rho_{\text{нас.
          пара 60} \celsius} \cdot V}{\mu}
          = 6{,}02 \cdot 10^{23}\,\frac{1}{\text{моль}} \cdot \frac{0{,}25 \cdot 130\,\frac{\text{г}}{\text{м}^{3}} \cdot 3\,\text{л}}{18\,\frac{\text{г}}{\text{моль}}}
          \approx 33 \cdot 10^{20}.
    $$
}
\solutionspace{160pt}

\tasknumber{2}%
\task{%
    В герметичном сосуде находится влажный воздух при температуре $30\celsius$ и относительной влажности $55\%$.
    \begin{enumerate}
        \item Чему равно парциальное давление насыщенного водяного пара при этой температуре?
        \item Чему равно парциальное давление водяного пара?
        \item Определите точку росы этого пара?
        \item Каким станет парциальное давление водяного пара, если сосуд нагреть до $90\celsius$?
        \item Чему будет равна относительная влажность воздуха, если сосуд нагреть до $90\celsius$?
        \item Получите ответ на предыдущий вопрос, используя плотности, а не давления.
    \end{enumerate}
}
\answer{%
    Парциальное давление насыщенного водяного пара при $30\celsius$ ищем по таблице: $$P_{\text{нас.
    пара 30} \celsius} = 4{,}240\,\text{кПа}.$$

    Парциальное давление водяного пара
    $$P_\text{пара 1} = \varphi_1 \cdot P_{\text{нас.
    пара 30} \celsius} = 0{,}55 \cdot 4{,}240\,\text{кПа} = 2{,}332\,\text{кПа}.$$

    Точку росы определяем по таблице: при какой температуре пар с давлением $P_\text{пара 1} = 2{,}332\,\text{кПа}$ станет насыщенным: $19{,}9\celsius$.

    После нагрева парциальное давление пара возрастёт:
    $$
        \frac{P_\text{пара 1} \cdot V}{T_1} = \nu R = \frac{P_\text{пара 2} \cdot V}{T_2}
        \implies P_\text{пара 2} = P_\text{пара 1} \cdot \frac{T_2}{T_1} = 2{,}332\,\text{кПа} \cdot \frac{363\,\text{К}}{303\,\text{К}} \approx 2{,}794\,\text{кПа}.
    $$

    Парциальное давление насыщенного водяного пара при $90\celsius$ ищем по таблице: $P_{\text{нас.
    пара 90} \celsius} = 70{,}100\,\text{кПа}$.
    Теперь определяем влажность:
    $$
        \varphi_2 = \frac{P_\text{пара 2}}{P_{\text{нас.
        пара 90} \celsius}} = \frac{2{,}794\,\text{кПа}}{70{,}100\,\text{кПа}} \approx 0{,}040 = 4{,}0\%.
    $$

    Или же выражаем то же самое через плотности (плотность не изменяется при изохорном нагревании $\rho_1 =\rho_2 = \rho$, в отличие от давления):
    $$
        \varphi_2 = \frac{\rho}{\rho_{\text{нас.
        пара 90} \celsius}} = \frac{\varphi_1\rho_{\text{нас.
        пара 30} \celsius}}{\rho_{\text{нас.
        пара 90} \celsius}}
        = \frac{0{,}55 \cdot 30{,}30\,\frac{\text{г}}{\text{м}^{3}}}{424\,\frac{\text{г}}{\text{м}^{3}}} \approx 0{,}039 = 3{,}9\%.
    $$
    Сравните 2 последних результата.
}
\solutionspace{200pt}

\tasknumber{3}%
\task{%
    Закрытый сосуд объёмом $10\,\text{л}$ заполнен сухим воздухом при давлении $100\,\text{кПа}$ и температуре $10\celsius$.
    Каким станет давление в сосуде, если в него налить $30\,\text{г}$ воды и нагреть содержимое сосуда до $100\celsius$?
}
\answer{%
    Конечное давление газа в сосуде складывается по закону Дальтона из давления нагретого сухого воздуха $P'_\text{воздуха}$ и
    давления насыщенного пара $P_\text{пара}$:
    $$P' = P'_\text{воздуха} + P_\text{пара}.$$

    Сперва определим новое давление сухого воздуха из уравнения состояния идеального газа:
    $$
        \frac{P'_\text{воздуха} \cdot V}{T'} = \nu R = \frac{P \cdot V}{T}
        \implies P'_\text{воздуха} = P \cdot \frac{T'}{T} = 100\,\text{кПа} \cdot \frac{373\,\text{К}}{283\,\text{К}} \approx 132\,\text{кПа}.
    $$

    Чтобы найти давление пара, нужно понять, будет ли он насыщенным после нагрева или нет.

    Плотность насыщенного пара при температуре $100\celsius$ равна $598\,\frac{\text{г}}{\text{м}^{3}}$, тогда для того,
    чтобы весь сосуд был заполнен насыщенным водяным паром нужно
    $m_\text{н.
    п.} = \rho_\text{н.
    п.
    100 $\celsius$} \cdot V = 598\,\frac{\text{г}}{\text{м}^{3}} \cdot 10\,\text{л} \approx 6{,}0\,\text{г}$ воды.
    Сравнивая эту массу с массой воды из условия, получаем массу жидкости, которая испарится: $m_\text{пара} = 6\,\text{г}$.
    Осталось определить давление этого пара:
    $$P_\text{пара} = \frac{m_\text{пара}RT'}{\mu V} = \frac{6\,\text{г} \cdot 8{,}31\,\frac{\text{Дж}}{\text{моль}\cdot\text{К}} \cdot 373\,\text{К}}{18\,\frac{\text{г}}{\text{моль}} \cdot 10\,\text{л}} \approx 103\,\text{кПа}.$$

    Получаем ответ: $P'_\text{пара} = 235{,}1\,\text{кПа}$.

    Другой вариант решения для давления пара:
    Определим давление пара, если бы вся вода испарилась (что не факт):
    $$P_\text{max} = \frac{mRT'}{\mu V} = \frac{30\,\text{г} \cdot 8{,}31\,\frac{\text{Дж}}{\text{моль}\cdot\text{К}} \cdot 373\,\text{К}}{18\,\frac{\text{г}}{\text{моль}} \cdot 10\,\text{л}} \approx 520\,\text{кПа}.$$
    Сравниваем это давление с давлением насыщенного пара при этой температуре $P_\text{н.
    п.
    100 $\celsius$} = 101{,}300\,\text{кПа}$:
    если у нас получилось меньше табличного значения,
    то вся вода испарилась, если же больше — испарилась лишь часть, а пар является насыщенным.
    Отсюда сразу получаем давление пара: $P'_\text{пара} = 101{,}3\,\text{кПа}$.
    Сравните этот результат с первым вариантом решения.

    Тут получаем ответ: $P'_\text{пара} = 233{,}1\,\text{кПа}$.
}
\solutionspace{150pt}

\tasknumber{4}%
\task{%
    Напротив физических величин запишите определение, обозначение и единицы измерения в системе СИ (если есть):
    \begin{enumerate}
        \item абсолютная влажность,
        \item насыщенный пар.
    \end{enumerate}
}

\variantsplitter

\addpersonalvariant{Алексей Алимпиев}

\tasknumber{1}%
\task{%
    Сколько молекул водяного пара содержится в сосуде объёмом $12\,\text{л}$ при температуре $70\celsius$,
    и влажности воздуха $30\%$?
}
\answer{%
    Уравнение состояния идеального газа (и учтём, что $R = N_A \cdot k$,
    это чуть упростит выячичления, но вовсе не обязательно это делать):
    $$
        PV = \nu RT = \frac N{N_A} RT \implies N = PV \cdot \frac{N_A}{RT}=  \frac{PV}{kT}
    $$
    Плотность насыщенного водяного пара при $70\celsius$ ищем по таблице: $P_{\text{нас.
    пара 70} \celsius} = 31\,\text{кПа}.$

    Получаем плотность пара в сосуде $\varphi = \frac P{P_{\text{нас.
    пара 70} \celsius}} \implies P = \varphi P_{\text{нас.
    пара 70} \celsius}.$

    И подставляем в ответ (по сути, его можно было получить быстрее из формул $P = nkT, n = \frac NV$):
    $$
        N = \frac{\varphi \cdot P_{\text{нас.
        пара 70} \celsius} \cdot V}{kT}
         = \frac{0{,}30 \cdot 31\,\text{кПа} \cdot 12\,\text{л}}{1{,}38 \cdot 10^{-23}\,\frac{\text{Дж}}{\text{К}} \cdot 343\,\text{К}}
         \approx 240 \cdot 10^{20}.
    $$

    Другой вариант решения (через плотности) приводит в результату:
    $$
        N = N_A \nu = N_A \cdot \frac m{\mu}
          = N_A \frac{\rho V}{\mu}
          = N_A \frac{\varphi \cdot \rho_{\text{нас.
          пара 70} \celsius} \cdot V}{\mu}
          = 6{,}02 \cdot 10^{23}\,\frac{1}{\text{моль}} \cdot \frac{0{,}30 \cdot 198\,\frac{\text{г}}{\text{м}^{3}} \cdot 12\,\text{л}}{18\,\frac{\text{г}}{\text{моль}}}
          \approx 240 \cdot 10^{20}.
    $$
}
\solutionspace{160pt}

\tasknumber{2}%
\task{%
    В герметичном сосуде находится влажный воздух при температуре $25\celsius$ и относительной влажности $35\%$.
    \begin{enumerate}
        \item Чему равно парциальное давление насыщенного водяного пара при этой температуре?
        \item Чему равно парциальное давление водяного пара?
        \item Определите точку росы этого пара?
        \item Каким станет парциальное давление водяного пара, если сосуд нагреть до $70\celsius$?
        \item Чему будет равна относительная влажность воздуха, если сосуд нагреть до $70\celsius$?
        \item Получите ответ на предыдущий вопрос, используя плотности, а не давления.
    \end{enumerate}
}
\answer{%
    Парциальное давление насыщенного водяного пара при $25\celsius$ ищем по таблице: $$P_{\text{нас.
    пара 25} \celsius} = 3{,}170\,\text{кПа}.$$

    Парциальное давление водяного пара
    $$P_\text{пара 1} = \varphi_1 \cdot P_{\text{нас.
    пара 25} \celsius} = 0{,}35 \cdot 3{,}170\,\text{кПа} = 1{,}1095\,\text{кПа}.$$

    Точку росы определяем по таблице: при какой температуре пар с давлением $P_\text{пара 1} = 1{,}1095\,\text{кПа}$ станет насыщенным: $8{,}5\celsius$.

    После нагрева парциальное давление пара возрастёт:
    $$
        \frac{P_\text{пара 1} \cdot V}{T_1} = \nu R = \frac{P_\text{пара 2} \cdot V}{T_2}
        \implies P_\text{пара 2} = P_\text{пара 1} \cdot \frac{T_2}{T_1} = 1{,}1095\,\text{кПа} \cdot \frac{343\,\text{К}}{298\,\text{К}} \approx 1{,}2770\,\text{кПа}.
    $$

    Парциальное давление насыщенного водяного пара при $70\celsius$ ищем по таблице: $P_{\text{нас.
    пара 70} \celsius} = 31\,\text{кПа}$.
    Теперь определяем влажность:
    $$
        \varphi_2 = \frac{P_\text{пара 2}}{P_{\text{нас.
        пара 70} \celsius}} = \frac{1{,}2770\,\text{кПа}}{31\,\text{кПа}} \approx 0{,}041 = 4{,}1\%.
    $$

    Или же выражаем то же самое через плотности (плотность не изменяется при изохорном нагревании $\rho_1 =\rho_2 = \rho$, в отличие от давления):
    $$
        \varphi_2 = \frac{\rho}{\rho_{\text{нас.
        пара 70} \celsius}} = \frac{\varphi_1\rho_{\text{нас.
        пара 25} \celsius}}{\rho_{\text{нас.
        пара 70} \celsius}}
        = \frac{0{,}35 \cdot 23\,\frac{\text{г}}{\text{м}^{3}}}{198\,\frac{\text{г}}{\text{м}^{3}}} \approx 0{,}041 = 4{,}1\%.
    $$
    Сравните 2 последних результата.
}
\solutionspace{200pt}

\tasknumber{3}%
\task{%
    Закрытый сосуд объёмом $10\,\text{л}$ заполнен сухим воздухом при давлении $100\,\text{кПа}$ и температуре $20\celsius$.
    Каким станет давление в сосуде, если в него налить $20\,\text{г}$ воды и нагреть содержимое сосуда до $90\celsius$?
}
\answer{%
    Конечное давление газа в сосуде складывается по закону Дальтона из давления нагретого сухого воздуха $P'_\text{воздуха}$ и
    давления насыщенного пара $P_\text{пара}$:
    $$P' = P'_\text{воздуха} + P_\text{пара}.$$

    Сперва определим новое давление сухого воздуха из уравнения состояния идеального газа:
    $$
        \frac{P'_\text{воздуха} \cdot V}{T'} = \nu R = \frac{P \cdot V}{T}
        \implies P'_\text{воздуха} = P \cdot \frac{T'}{T} = 100\,\text{кПа} \cdot \frac{363\,\text{К}}{293\,\text{К}} \approx 124\,\text{кПа}.
    $$

    Чтобы найти давление пара, нужно понять, будет ли он насыщенным после нагрева или нет.

    Плотность насыщенного пара при температуре $90\celsius$ равна $424\,\frac{\text{г}}{\text{м}^{3}}$, тогда для того,
    чтобы весь сосуд был заполнен насыщенным водяным паром нужно
    $m_\text{н.
    п.} = \rho_\text{н.
    п.
    90 $\celsius$} \cdot V = 424\,\frac{\text{г}}{\text{м}^{3}} \cdot 10\,\text{л} \approx 4{,}2\,\text{г}$ воды.
    Сравнивая эту массу с массой воды из условия, получаем массу жидкости, которая испарится: $m_\text{пара} = 4{,}2\,\text{г}$.
    Осталось определить давление этого пара:
    $$P_\text{пара} = \frac{m_\text{пара}RT'}{\mu V} = \frac{4{,}2\,\text{г} \cdot 8{,}31\,\frac{\text{Дж}}{\text{моль}\cdot\text{К}} \cdot 363\,\text{К}}{18\,\frac{\text{г}}{\text{моль}} \cdot 10\,\text{л}} \approx 70\,\text{кПа}.$$

    Получаем ответ: $P'_\text{пара} = 194{,}3\,\text{кПа}$.

    Другой вариант решения для давления пара:
    Определим давление пара, если бы вся вода испарилась (что не факт):
    $$P_\text{max} = \frac{mRT'}{\mu V} = \frac{20\,\text{г} \cdot 8{,}31\,\frac{\text{Дж}}{\text{моль}\cdot\text{К}} \cdot 363\,\text{К}}{18\,\frac{\text{г}}{\text{моль}} \cdot 10\,\text{л}} \approx 340\,\text{кПа}.$$
    Сравниваем это давление с давлением насыщенного пара при этой температуре $P_\text{н.
    п.
    90 $\celsius$} = 70{,}100\,\text{кПа}$:
    если у нас получилось меньше табличного значения,
    то вся вода испарилась, если же больше — испарилась лишь часть, а пар является насыщенным.
    Отсюда сразу получаем давление пара: $P'_\text{пара} = 70{,}1\,\text{кПа}$.
    Сравните этот результат с первым вариантом решения.

    Тут получаем ответ: $P'_\text{пара} = 194\,\text{кПа}$.
}
\solutionspace{150pt}

\tasknumber{4}%
\task{%
    Напротив физических величин запишите определение, обозначение и единицы измерения в системе СИ (если есть):
    \begin{enumerate}
        \item относительная влажность,
        \item насыщенный пар.
    \end{enumerate}
}

\variantsplitter

\addpersonalvariant{Евгений Васин}

\tasknumber{1}%
\task{%
    Сколько молекул водяного пара содержится в сосуде объёмом $6\,\text{л}$ при температуре $40\celsius$,
    и влажности воздуха $75\%$?
}
\answer{%
    Уравнение состояния идеального газа (и учтём, что $R = N_A \cdot k$,
    это чуть упростит выячичления, но вовсе не обязательно это делать):
    $$
        PV = \nu RT = \frac N{N_A} RT \implies N = PV \cdot \frac{N_A}{RT}=  \frac{PV}{kT}
    $$
    Плотность насыщенного водяного пара при $40\celsius$ ищем по таблице: $P_{\text{нас.
    пара 40} \celsius} = 7{,}370\,\text{кПа}.$

    Получаем плотность пара в сосуде $\varphi = \frac P{P_{\text{нас.
    пара 40} \celsius}} \implies P = \varphi P_{\text{нас.
    пара 40} \celsius}.$

    И подставляем в ответ (по сути, его можно было получить быстрее из формул $P = nkT, n = \frac NV$):
    $$
        N = \frac{\varphi \cdot P_{\text{нас.
        пара 40} \celsius} \cdot V}{kT}
         = \frac{0{,}75 \cdot 7{,}370\,\text{кПа} \cdot 6\,\text{л}}{1{,}38 \cdot 10^{-23}\,\frac{\text{Дж}}{\text{К}} \cdot 313\,\text{К}}
         \approx 77 \cdot 10^{20}.
    $$

    Другой вариант решения (через плотности) приводит в результату:
    $$
        N = N_A \nu = N_A \cdot \frac m{\mu}
          = N_A \frac{\rho V}{\mu}
          = N_A \frac{\varphi \cdot \rho_{\text{нас.
          пара 40} \celsius} \cdot V}{\mu}
          = 6{,}02 \cdot 10^{23}\,\frac{1}{\text{моль}} \cdot \frac{0{,}75 \cdot 51{,}20\,\frac{\text{г}}{\text{м}^{3}} \cdot 6\,\text{л}}{18\,\frac{\text{г}}{\text{моль}}}
          \approx 77 \cdot 10^{20}.
    $$
}
\solutionspace{160pt}

\tasknumber{2}%
\task{%
    В герметичном сосуде находится влажный воздух при температуре $25\celsius$ и относительной влажности $45\%$.
    \begin{enumerate}
        \item Чему равно парциальное давление насыщенного водяного пара при этой температуре?
        \item Чему равно парциальное давление водяного пара?
        \item Определите точку росы этого пара?
        \item Каким станет парциальное давление водяного пара, если сосуд нагреть до $70\celsius$?
        \item Чему будет равна относительная влажность воздуха, если сосуд нагреть до $70\celsius$?
        \item Получите ответ на предыдущий вопрос, используя плотности, а не давления.
    \end{enumerate}
}
\answer{%
    Парциальное давление насыщенного водяного пара при $25\celsius$ ищем по таблице: $$P_{\text{нас.
    пара 25} \celsius} = 3{,}170\,\text{кПа}.$$

    Парциальное давление водяного пара
    $$P_\text{пара 1} = \varphi_1 \cdot P_{\text{нас.
    пара 25} \celsius} = 0{,}45 \cdot 3{,}170\,\text{кПа} = 1{,}4265\,\text{кПа}.$$

    Точку росы определяем по таблице: при какой температуре пар с давлением $P_\text{пара 1} = 1{,}4265\,\text{кПа}$ станет насыщенным: $12{,}3\celsius$.

    После нагрева парциальное давление пара возрастёт:
    $$
        \frac{P_\text{пара 1} \cdot V}{T_1} = \nu R = \frac{P_\text{пара 2} \cdot V}{T_2}
        \implies P_\text{пара 2} = P_\text{пара 1} \cdot \frac{T_2}{T_1} = 1{,}4265\,\text{кПа} \cdot \frac{343\,\text{К}}{298\,\text{К}} \approx 1{,}6419\,\text{кПа}.
    $$

    Парциальное давление насыщенного водяного пара при $70\celsius$ ищем по таблице: $P_{\text{нас.
    пара 70} \celsius} = 31\,\text{кПа}$.
    Теперь определяем влажность:
    $$
        \varphi_2 = \frac{P_\text{пара 2}}{P_{\text{нас.
        пара 70} \celsius}} = \frac{1{,}6419\,\text{кПа}}{31\,\text{кПа}} \approx 0{,}053 = 5{,}3\%.
    $$

    Или же выражаем то же самое через плотности (плотность не изменяется при изохорном нагревании $\rho_1 =\rho_2 = \rho$, в отличие от давления):
    $$
        \varphi_2 = \frac{\rho}{\rho_{\text{нас.
        пара 70} \celsius}} = \frac{\varphi_1\rho_{\text{нас.
        пара 25} \celsius}}{\rho_{\text{нас.
        пара 70} \celsius}}
        = \frac{0{,}45 \cdot 23\,\frac{\text{г}}{\text{м}^{3}}}{198\,\frac{\text{г}}{\text{м}^{3}}} \approx 0{,}052 = 5{,}2\%.
    $$
    Сравните 2 последних результата.
}
\solutionspace{200pt}

\tasknumber{3}%
\task{%
    Закрытый сосуд объёмом $10\,\text{л}$ заполнен сухим воздухом при давлении $100\,\text{кПа}$ и температуре $30\celsius$.
    Каким станет давление в сосуде, если в него налить $5\,\text{г}$ воды и нагреть содержимое сосуда до $80\celsius$?
}
\answer{%
    Конечное давление газа в сосуде складывается по закону Дальтона из давления нагретого сухого воздуха $P'_\text{воздуха}$ и
    давления насыщенного пара $P_\text{пара}$:
    $$P' = P'_\text{воздуха} + P_\text{пара}.$$

    Сперва определим новое давление сухого воздуха из уравнения состояния идеального газа:
    $$
        \frac{P'_\text{воздуха} \cdot V}{T'} = \nu R = \frac{P \cdot V}{T}
        \implies P'_\text{воздуха} = P \cdot \frac{T'}{T} = 100\,\text{кПа} \cdot \frac{353\,\text{К}}{303\,\text{К}} \approx 117\,\text{кПа}.
    $$

    Чтобы найти давление пара, нужно понять, будет ли он насыщенным после нагрева или нет.

    Плотность насыщенного пара при температуре $80\celsius$ равна $293\,\frac{\text{г}}{\text{м}^{3}}$, тогда для того,
    чтобы весь сосуд был заполнен насыщенным водяным паром нужно
    $m_\text{н.
    п.} = \rho_\text{н.
    п.
    80 $\celsius$} \cdot V = 293\,\frac{\text{г}}{\text{м}^{3}} \cdot 10\,\text{л} \approx 2{,}9\,\text{г}$ воды.
    Сравнивая эту массу с массой воды из условия, получаем массу жидкости, которая испарится: $m_\text{пара} = 2{,}9\,\text{г}$.
    Осталось определить давление этого пара:
    $$P_\text{пара} = \frac{m_\text{пара}RT'}{\mu V} = \frac{2{,}9\,\text{г} \cdot 8{,}31\,\frac{\text{Дж}}{\text{моль}\cdot\text{К}} \cdot 353\,\text{К}}{18\,\frac{\text{г}}{\text{моль}} \cdot 10\,\text{л}} \approx 47\,\text{кПа}.$$

    Получаем ответ: $P'_\text{пара} = 163{,}8\,\text{кПа}$.

    Другой вариант решения для давления пара:
    Определим давление пара, если бы вся вода испарилась (что не факт):
    $$P_\text{max} = \frac{mRT'}{\mu V} = \frac{5\,\text{г} \cdot 8{,}31\,\frac{\text{Дж}}{\text{моль}\cdot\text{К}} \cdot 353\,\text{К}}{18\,\frac{\text{г}}{\text{моль}} \cdot 10\,\text{л}} \approx 81\,\text{кПа}.$$
    Сравниваем это давление с давлением насыщенного пара при этой температуре $P_\text{н.
    п.
    80 $\celsius$} = 47{,}300\,\text{кПа}$:
    если у нас получилось меньше табличного значения,
    то вся вода испарилась, если же больше — испарилась лишь часть, а пар является насыщенным.
    Отсюда сразу получаем давление пара: $P'_\text{пара} = 47{,}3\,\text{кПа}$.
    Сравните этот результат с первым вариантом решения.

    Тут получаем ответ: $P'_\text{пара} = 163{,}8\,\text{кПа}$.
}
\solutionspace{150pt}

\tasknumber{4}%
\task{%
    Напротив физических величин запишите определение, обозначение и единицы измерения в системе СИ (если есть):
    \begin{enumerate}
        \item абсолютная влажность,
        \item динамическое равновесие.
    \end{enumerate}
}

\variantsplitter

\addpersonalvariant{Вячеслав Волохов}

\tasknumber{1}%
\task{%
    Сколько молекул водяного пара содержится в сосуде объёмом $12\,\text{л}$ при температуре $40\celsius$,
    и влажности воздуха $75\%$?
}
\answer{%
    Уравнение состояния идеального газа (и учтём, что $R = N_A \cdot k$,
    это чуть упростит выячичления, но вовсе не обязательно это делать):
    $$
        PV = \nu RT = \frac N{N_A} RT \implies N = PV \cdot \frac{N_A}{RT}=  \frac{PV}{kT}
    $$
    Плотность насыщенного водяного пара при $40\celsius$ ищем по таблице: $P_{\text{нас.
    пара 40} \celsius} = 7{,}370\,\text{кПа}.$

    Получаем плотность пара в сосуде $\varphi = \frac P{P_{\text{нас.
    пара 40} \celsius}} \implies P = \varphi P_{\text{нас.
    пара 40} \celsius}.$

    И подставляем в ответ (по сути, его можно было получить быстрее из формул $P = nkT, n = \frac NV$):
    $$
        N = \frac{\varphi \cdot P_{\text{нас.
        пара 40} \celsius} \cdot V}{kT}
         = \frac{0{,}75 \cdot 7{,}370\,\text{кПа} \cdot 12\,\text{л}}{1{,}38 \cdot 10^{-23}\,\frac{\text{Дж}}{\text{К}} \cdot 313\,\text{К}}
         \approx 154 \cdot 10^{20}.
    $$

    Другой вариант решения (через плотности) приводит в результату:
    $$
        N = N_A \nu = N_A \cdot \frac m{\mu}
          = N_A \frac{\rho V}{\mu}
          = N_A \frac{\varphi \cdot \rho_{\text{нас.
          пара 40} \celsius} \cdot V}{\mu}
          = 6{,}02 \cdot 10^{23}\,\frac{1}{\text{моль}} \cdot \frac{0{,}75 \cdot 51{,}20\,\frac{\text{г}}{\text{м}^{3}} \cdot 12\,\text{л}}{18\,\frac{\text{г}}{\text{моль}}}
          \approx 154 \cdot 10^{20}.
    $$
}
\solutionspace{160pt}

\tasknumber{2}%
\task{%
    В герметичном сосуде находится влажный воздух при температуре $15\celsius$ и относительной влажности $65\%$.
    \begin{enumerate}
        \item Чему равно парциальное давление насыщенного водяного пара при этой температуре?
        \item Чему равно парциальное давление водяного пара?
        \item Определите точку росы этого пара?
        \item Каким станет парциальное давление водяного пара, если сосуд нагреть до $90\celsius$?
        \item Чему будет равна относительная влажность воздуха, если сосуд нагреть до $90\celsius$?
        \item Получите ответ на предыдущий вопрос, используя плотности, а не давления.
    \end{enumerate}
}
\answer{%
    Парциальное давление насыщенного водяного пара при $15\celsius$ ищем по таблице: $$P_{\text{нас.
    пара 15} \celsius} = 1{,}700\,\text{кПа}.$$

    Парциальное давление водяного пара
    $$P_\text{пара 1} = \varphi_1 \cdot P_{\text{нас.
    пара 15} \celsius} = 0{,}65 \cdot 1{,}700\,\text{кПа} = 1{,}105\,\text{кПа}.$$

    Точку росы определяем по таблице: при какой температуре пар с давлением $P_\text{пара 1} = 1{,}105\,\text{кПа}$ станет насыщенным: $8{,}4\celsius$.

    После нагрева парциальное давление пара возрастёт:
    $$
        \frac{P_\text{пара 1} \cdot V}{T_1} = \nu R = \frac{P_\text{пара 2} \cdot V}{T_2}
        \implies P_\text{пара 2} = P_\text{пара 1} \cdot \frac{T_2}{T_1} = 1{,}105\,\text{кПа} \cdot \frac{363\,\text{К}}{288\,\text{К}} \approx 1{,}393\,\text{кПа}.
    $$

    Парциальное давление насыщенного водяного пара при $90\celsius$ ищем по таблице: $P_{\text{нас.
    пара 90} \celsius} = 70{,}100\,\text{кПа}$.
    Теперь определяем влажность:
    $$
        \varphi_2 = \frac{P_\text{пара 2}}{P_{\text{нас.
        пара 90} \celsius}} = \frac{1{,}393\,\text{кПа}}{70{,}100\,\text{кПа}} \approx 0{,}020 = 2{,}0\%.
    $$

    Или же выражаем то же самое через плотности (плотность не изменяется при изохорном нагревании $\rho_1 =\rho_2 = \rho$, в отличие от давления):
    $$
        \varphi_2 = \frac{\rho}{\rho_{\text{нас.
        пара 90} \celsius}} = \frac{\varphi_1\rho_{\text{нас.
        пара 15} \celsius}}{\rho_{\text{нас.
        пара 90} \celsius}}
        = \frac{0{,}65 \cdot 12{,}80\,\frac{\text{г}}{\text{м}^{3}}}{424\,\frac{\text{г}}{\text{м}^{3}}} \approx 0{,}020 = 2{,}0\%.
    $$
    Сравните 2 последних результата.
}
\solutionspace{200pt}

\tasknumber{3}%
\task{%
    Закрытый сосуд объёмом $10\,\text{л}$ заполнен сухим воздухом при давлении $100\,\text{кПа}$ и температуре $20\celsius$.
    Каким станет давление в сосуде, если в него налить $5\,\text{г}$ воды и нагреть содержимое сосуда до $80\celsius$?
}
\answer{%
    Конечное давление газа в сосуде складывается по закону Дальтона из давления нагретого сухого воздуха $P'_\text{воздуха}$ и
    давления насыщенного пара $P_\text{пара}$:
    $$P' = P'_\text{воздуха} + P_\text{пара}.$$

    Сперва определим новое давление сухого воздуха из уравнения состояния идеального газа:
    $$
        \frac{P'_\text{воздуха} \cdot V}{T'} = \nu R = \frac{P \cdot V}{T}
        \implies P'_\text{воздуха} = P \cdot \frac{T'}{T} = 100\,\text{кПа} \cdot \frac{353\,\text{К}}{293\,\text{К}} \approx 120\,\text{кПа}.
    $$

    Чтобы найти давление пара, нужно понять, будет ли он насыщенным после нагрева или нет.

    Плотность насыщенного пара при температуре $80\celsius$ равна $293\,\frac{\text{г}}{\text{м}^{3}}$, тогда для того,
    чтобы весь сосуд был заполнен насыщенным водяным паром нужно
    $m_\text{н.
    п.} = \rho_\text{н.
    п.
    80 $\celsius$} \cdot V = 293\,\frac{\text{г}}{\text{м}^{3}} \cdot 10\,\text{л} \approx 2{,}9\,\text{г}$ воды.
    Сравнивая эту массу с массой воды из условия, получаем массу жидкости, которая испарится: $m_\text{пара} = 2{,}9\,\text{г}$.
    Осталось определить давление этого пара:
    $$P_\text{пара} = \frac{m_\text{пара}RT'}{\mu V} = \frac{2{,}9\,\text{г} \cdot 8{,}31\,\frac{\text{Дж}}{\text{моль}\cdot\text{К}} \cdot 353\,\text{К}}{18\,\frac{\text{г}}{\text{моль}} \cdot 10\,\text{л}} \approx 47\,\text{кПа}.$$

    Получаем ответ: $P'_\text{пара} = 167{,}7\,\text{кПа}$.

    Другой вариант решения для давления пара:
    Определим давление пара, если бы вся вода испарилась (что не факт):
    $$P_\text{max} = \frac{mRT'}{\mu V} = \frac{5\,\text{г} \cdot 8{,}31\,\frac{\text{Дж}}{\text{моль}\cdot\text{К}} \cdot 353\,\text{К}}{18\,\frac{\text{г}}{\text{моль}} \cdot 10\,\text{л}} \approx 81\,\text{кПа}.$$
    Сравниваем это давление с давлением насыщенного пара при этой температуре $P_\text{н.
    п.
    80 $\celsius$} = 47{,}300\,\text{кПа}$:
    если у нас получилось меньше табличного значения,
    то вся вода испарилась, если же больше — испарилась лишь часть, а пар является насыщенным.
    Отсюда сразу получаем давление пара: $P'_\text{пара} = 47{,}3\,\text{кПа}$.
    Сравните этот результат с первым вариантом решения.

    Тут получаем ответ: $P'_\text{пара} = 167{,}8\,\text{кПа}$.
}
\solutionspace{150pt}

\tasknumber{4}%
\task{%
    Напротив физических величин запишите определение, обозначение и единицы измерения в системе СИ (если есть):
    \begin{enumerate}
        \item абсолютная влажность,
        \item насыщенный пар.
    \end{enumerate}
}

\variantsplitter

\addpersonalvariant{Герман Говоров}

\tasknumber{1}%
\task{%
    Сколько молекул водяного пара содержится в сосуде объёмом $12\,\text{л}$ при температуре $20\celsius$,
    и влажности воздуха $55\%$?
}
\answer{%
    Уравнение состояния идеального газа (и учтём, что $R = N_A \cdot k$,
    это чуть упростит выячичления, но вовсе не обязательно это делать):
    $$
        PV = \nu RT = \frac N{N_A} RT \implies N = PV \cdot \frac{N_A}{RT}=  \frac{PV}{kT}
    $$
    Плотность насыщенного водяного пара при $20\celsius$ ищем по таблице: $P_{\text{нас.
    пара 20} \celsius} = 2{,}340\,\text{кПа}.$

    Получаем плотность пара в сосуде $\varphi = \frac P{P_{\text{нас.
    пара 20} \celsius}} \implies P = \varphi P_{\text{нас.
    пара 20} \celsius}.$

    И подставляем в ответ (по сути, его можно было получить быстрее из формул $P = nkT, n = \frac NV$):
    $$
        N = \frac{\varphi \cdot P_{\text{нас.
        пара 20} \celsius} \cdot V}{kT}
         = \frac{0{,}55 \cdot 2{,}340\,\text{кПа} \cdot 12\,\text{л}}{1{,}38 \cdot 10^{-23}\,\frac{\text{Дж}}{\text{К}} \cdot 293\,\text{К}}
         \approx 38 \cdot 10^{20}.
    $$

    Другой вариант решения (через плотности) приводит в результату:
    $$
        N = N_A \nu = N_A \cdot \frac m{\mu}
          = N_A \frac{\rho V}{\mu}
          = N_A \frac{\varphi \cdot \rho_{\text{нас.
          пара 20} \celsius} \cdot V}{\mu}
          = 6{,}02 \cdot 10^{23}\,\frac{1}{\text{моль}} \cdot \frac{0{,}55 \cdot 17{,}30\,\frac{\text{г}}{\text{м}^{3}} \cdot 12\,\text{л}}{18\,\frac{\text{г}}{\text{моль}}}
          \approx 38 \cdot 10^{20}.
    $$
}
\solutionspace{160pt}

\tasknumber{2}%
\task{%
    В герметичном сосуде находится влажный воздух при температуре $30\celsius$ и относительной влажности $50\%$.
    \begin{enumerate}
        \item Чему равно парциальное давление насыщенного водяного пара при этой температуре?
        \item Чему равно парциальное давление водяного пара?
        \item Определите точку росы этого пара?
        \item Каким станет парциальное давление водяного пара, если сосуд нагреть до $80\celsius$?
        \item Чему будет равна относительная влажность воздуха, если сосуд нагреть до $80\celsius$?
        \item Получите ответ на предыдущий вопрос, используя плотности, а не давления.
    \end{enumerate}
}
\answer{%
    Парциальное давление насыщенного водяного пара при $30\celsius$ ищем по таблице: $$P_{\text{нас.
    пара 30} \celsius} = 4{,}240\,\text{кПа}.$$

    Парциальное давление водяного пара
    $$P_\text{пара 1} = \varphi_1 \cdot P_{\text{нас.
    пара 30} \celsius} = 0{,}50 \cdot 4{,}240\,\text{кПа} = 2{,}120\,\text{кПа}.$$

    Точку росы определяем по таблице: при какой температуре пар с давлением $P_\text{пара 1} = 2{,}120\,\text{кПа}$ станет насыщенным: $18{,}5\celsius$.

    После нагрева парциальное давление пара возрастёт:
    $$
        \frac{P_\text{пара 1} \cdot V}{T_1} = \nu R = \frac{P_\text{пара 2} \cdot V}{T_2}
        \implies P_\text{пара 2} = P_\text{пара 1} \cdot \frac{T_2}{T_1} = 2{,}120\,\text{кПа} \cdot \frac{353\,\text{К}}{303\,\text{К}} \approx 2{,}470\,\text{кПа}.
    $$

    Парциальное давление насыщенного водяного пара при $80\celsius$ ищем по таблице: $P_{\text{нас.
    пара 80} \celsius} = 47{,}300\,\text{кПа}$.
    Теперь определяем влажность:
    $$
        \varphi_2 = \frac{P_\text{пара 2}}{P_{\text{нас.
        пара 80} \celsius}} = \frac{2{,}470\,\text{кПа}}{47{,}300\,\text{кПа}} \approx 0{,}052 = 5{,}2\%.
    $$

    Или же выражаем то же самое через плотности (плотность не изменяется при изохорном нагревании $\rho_1 =\rho_2 = \rho$, в отличие от давления):
    $$
        \varphi_2 = \frac{\rho}{\rho_{\text{нас.
        пара 80} \celsius}} = \frac{\varphi_1\rho_{\text{нас.
        пара 30} \celsius}}{\rho_{\text{нас.
        пара 80} \celsius}}
        = \frac{0{,}50 \cdot 30{,}30\,\frac{\text{г}}{\text{м}^{3}}}{293\,\frac{\text{г}}{\text{м}^{3}}} \approx 0{,}052 = 5{,}2\%.
    $$
    Сравните 2 последних результата.
}
\solutionspace{200pt}

\tasknumber{3}%
\task{%
    Закрытый сосуд объёмом $10\,\text{л}$ заполнен сухим воздухом при давлении $100\,\text{кПа}$ и температуре $20\celsius$.
    Каким станет давление в сосуде, если в него налить $5\,\text{г}$ воды и нагреть содержимое сосуда до $90\celsius$?
}
\answer{%
    Конечное давление газа в сосуде складывается по закону Дальтона из давления нагретого сухого воздуха $P'_\text{воздуха}$ и
    давления насыщенного пара $P_\text{пара}$:
    $$P' = P'_\text{воздуха} + P_\text{пара}.$$

    Сперва определим новое давление сухого воздуха из уравнения состояния идеального газа:
    $$
        \frac{P'_\text{воздуха} \cdot V}{T'} = \nu R = \frac{P \cdot V}{T}
        \implies P'_\text{воздуха} = P \cdot \frac{T'}{T} = 100\,\text{кПа} \cdot \frac{363\,\text{К}}{293\,\text{К}} \approx 124\,\text{кПа}.
    $$

    Чтобы найти давление пара, нужно понять, будет ли он насыщенным после нагрева или нет.

    Плотность насыщенного пара при температуре $90\celsius$ равна $424\,\frac{\text{г}}{\text{м}^{3}}$, тогда для того,
    чтобы весь сосуд был заполнен насыщенным водяным паром нужно
    $m_\text{н.
    п.} = \rho_\text{н.
    п.
    90 $\celsius$} \cdot V = 424\,\frac{\text{г}}{\text{м}^{3}} \cdot 10\,\text{л} \approx 4{,}2\,\text{г}$ воды.
    Сравнивая эту массу с массой воды из условия, получаем массу жидкости, которая испарится: $m_\text{пара} = 4{,}2\,\text{г}$.
    Осталось определить давление этого пара:
    $$P_\text{пара} = \frac{m_\text{пара}RT'}{\mu V} = \frac{4{,}2\,\text{г} \cdot 8{,}31\,\frac{\text{Дж}}{\text{моль}\cdot\text{К}} \cdot 363\,\text{К}}{18\,\frac{\text{г}}{\text{моль}} \cdot 10\,\text{л}} \approx 70\,\text{кПа}.$$

    Получаем ответ: $P'_\text{пара} = 194{,}3\,\text{кПа}$.

    Другой вариант решения для давления пара:
    Определим давление пара, если бы вся вода испарилась (что не факт):
    $$P_\text{max} = \frac{mRT'}{\mu V} = \frac{5\,\text{г} \cdot 8{,}31\,\frac{\text{Дж}}{\text{моль}\cdot\text{К}} \cdot 363\,\text{К}}{18\,\frac{\text{г}}{\text{моль}} \cdot 10\,\text{л}} \approx 84\,\text{кПа}.$$
    Сравниваем это давление с давлением насыщенного пара при этой температуре $P_\text{н.
    п.
    90 $\celsius$} = 70{,}100\,\text{кПа}$:
    если у нас получилось меньше табличного значения,
    то вся вода испарилась, если же больше — испарилась лишь часть, а пар является насыщенным.
    Отсюда сразу получаем давление пара: $P'_\text{пара} = 70{,}1\,\text{кПа}$.
    Сравните этот результат с первым вариантом решения.

    Тут получаем ответ: $P'_\text{пара} = 194\,\text{кПа}$.
}
\solutionspace{150pt}

\tasknumber{4}%
\task{%
    Напротив физических величин запишите определение, обозначение и единицы измерения в системе СИ (если есть):
    \begin{enumerate}
        \item относительная влажность,
        \item насыщенный пар.
    \end{enumerate}
}

\variantsplitter

\addpersonalvariant{София Журавлёва}

\tasknumber{1}%
\task{%
    Сколько молекул водяного пара содержится в сосуде объёмом $12\,\text{л}$ при температуре $40\celsius$,
    и влажности воздуха $30\%$?
}
\answer{%
    Уравнение состояния идеального газа (и учтём, что $R = N_A \cdot k$,
    это чуть упростит выячичления, но вовсе не обязательно это делать):
    $$
        PV = \nu RT = \frac N{N_A} RT \implies N = PV \cdot \frac{N_A}{RT}=  \frac{PV}{kT}
    $$
    Плотность насыщенного водяного пара при $40\celsius$ ищем по таблице: $P_{\text{нас.
    пара 40} \celsius} = 7{,}370\,\text{кПа}.$

    Получаем плотность пара в сосуде $\varphi = \frac P{P_{\text{нас.
    пара 40} \celsius}} \implies P = \varphi P_{\text{нас.
    пара 40} \celsius}.$

    И подставляем в ответ (по сути, его можно было получить быстрее из формул $P = nkT, n = \frac NV$):
    $$
        N = \frac{\varphi \cdot P_{\text{нас.
        пара 40} \celsius} \cdot V}{kT}
         = \frac{0{,}30 \cdot 7{,}370\,\text{кПа} \cdot 12\,\text{л}}{1{,}38 \cdot 10^{-23}\,\frac{\text{Дж}}{\text{К}} \cdot 313\,\text{К}}
         \approx 61 \cdot 10^{20}.
    $$

    Другой вариант решения (через плотности) приводит в результату:
    $$
        N = N_A \nu = N_A \cdot \frac m{\mu}
          = N_A \frac{\rho V}{\mu}
          = N_A \frac{\varphi \cdot \rho_{\text{нас.
          пара 40} \celsius} \cdot V}{\mu}
          = 6{,}02 \cdot 10^{23}\,\frac{1}{\text{моль}} \cdot \frac{0{,}30 \cdot 51{,}20\,\frac{\text{г}}{\text{м}^{3}} \cdot 12\,\text{л}}{18\,\frac{\text{г}}{\text{моль}}}
          \approx 62 \cdot 10^{20}.
    $$
}
\solutionspace{160pt}

\tasknumber{2}%
\task{%
    В герметичном сосуде находится влажный воздух при температуре $25\celsius$ и относительной влажности $75\%$.
    \begin{enumerate}
        \item Чему равно парциальное давление насыщенного водяного пара при этой температуре?
        \item Чему равно парциальное давление водяного пара?
        \item Определите точку росы этого пара?
        \item Каким станет парциальное давление водяного пара, если сосуд нагреть до $70\celsius$?
        \item Чему будет равна относительная влажность воздуха, если сосуд нагреть до $70\celsius$?
        \item Получите ответ на предыдущий вопрос, используя плотности, а не давления.
    \end{enumerate}
}
\answer{%
    Парциальное давление насыщенного водяного пара при $25\celsius$ ищем по таблице: $$P_{\text{нас.
    пара 25} \celsius} = 3{,}170\,\text{кПа}.$$

    Парциальное давление водяного пара
    $$P_\text{пара 1} = \varphi_1 \cdot P_{\text{нас.
    пара 25} \celsius} = 0{,}75 \cdot 3{,}170\,\text{кПа} = 2{,}378\,\text{кПа}.$$

    Точку росы определяем по таблице: при какой температуре пар с давлением $P_\text{пара 1} = 2{,}378\,\text{кПа}$ станет насыщенным: $20{,}3\celsius$.

    После нагрева парциальное давление пара возрастёт:
    $$
        \frac{P_\text{пара 1} \cdot V}{T_1} = \nu R = \frac{P_\text{пара 2} \cdot V}{T_2}
        \implies P_\text{пара 2} = P_\text{пара 1} \cdot \frac{T_2}{T_1} = 2{,}378\,\text{кПа} \cdot \frac{343\,\text{К}}{298\,\text{К}} \approx 2{,}737\,\text{кПа}.
    $$

    Парциальное давление насыщенного водяного пара при $70\celsius$ ищем по таблице: $P_{\text{нас.
    пара 70} \celsius} = 31\,\text{кПа}$.
    Теперь определяем влажность:
    $$
        \varphi_2 = \frac{P_\text{пара 2}}{P_{\text{нас.
        пара 70} \celsius}} = \frac{2{,}737\,\text{кПа}}{31\,\text{кПа}} \approx 0{,}088 = 8{,}8\%.
    $$

    Или же выражаем то же самое через плотности (плотность не изменяется при изохорном нагревании $\rho_1 =\rho_2 = \rho$, в отличие от давления):
    $$
        \varphi_2 = \frac{\rho}{\rho_{\text{нас.
        пара 70} \celsius}} = \frac{\varphi_1\rho_{\text{нас.
        пара 25} \celsius}}{\rho_{\text{нас.
        пара 70} \celsius}}
        = \frac{0{,}75 \cdot 23\,\frac{\text{г}}{\text{м}^{3}}}{198\,\frac{\text{г}}{\text{м}^{3}}} \approx 0{,}087 = 8{,}7\%.
    $$
    Сравните 2 последних результата.
}
\solutionspace{200pt}

\tasknumber{3}%
\task{%
    Закрытый сосуд объёмом $15\,\text{л}$ заполнен сухим воздухом при давлении $100\,\text{кПа}$ и температуре $30\celsius$.
    Каким станет давление в сосуде, если в него налить $5\,\text{г}$ воды и нагреть содержимое сосуда до $100\celsius$?
}
\answer{%
    Конечное давление газа в сосуде складывается по закону Дальтона из давления нагретого сухого воздуха $P'_\text{воздуха}$ и
    давления насыщенного пара $P_\text{пара}$:
    $$P' = P'_\text{воздуха} + P_\text{пара}.$$

    Сперва определим новое давление сухого воздуха из уравнения состояния идеального газа:
    $$
        \frac{P'_\text{воздуха} \cdot V}{T'} = \nu R = \frac{P \cdot V}{T}
        \implies P'_\text{воздуха} = P \cdot \frac{T'}{T} = 100\,\text{кПа} \cdot \frac{373\,\text{К}}{303\,\text{К}} \approx 123\,\text{кПа}.
    $$

    Чтобы найти давление пара, нужно понять, будет ли он насыщенным после нагрева или нет.

    Плотность насыщенного пара при температуре $100\celsius$ равна $598\,\frac{\text{г}}{\text{м}^{3}}$, тогда для того,
    чтобы весь сосуд был заполнен насыщенным водяным паром нужно
    $m_\text{н.
    п.} = \rho_\text{н.
    п.
    100 $\celsius$} \cdot V = 598\,\frac{\text{г}}{\text{м}^{3}} \cdot 15\,\text{л} \approx 9{,}0\,\text{г}$ воды.
    Сравнивая эту массу с массой воды из условия, получаем массу жидкости, которая испарится: $m_\text{пара} = 5\,\text{г}$.
    Осталось определить давление этого пара:
    $$P_\text{пара} = \frac{m_\text{пара}RT'}{\mu V} = \frac{5\,\text{г} \cdot 8{,}31\,\frac{\text{Дж}}{\text{моль}\cdot\text{К}} \cdot 373\,\text{К}}{18\,\frac{\text{г}}{\text{моль}} \cdot 15\,\text{л}} \approx 57\,\text{кПа}.$$

    Получаем ответ: $P'_\text{пара} = 180{,}5\,\text{кПа}$.

    Другой вариант решения для давления пара:
    Определим давление пара, если бы вся вода испарилась (что не факт):
    $$P_\text{max} = \frac{mRT'}{\mu V} = \frac{5\,\text{г} \cdot 8{,}31\,\frac{\text{Дж}}{\text{моль}\cdot\text{К}} \cdot 373\,\text{К}}{18\,\frac{\text{г}}{\text{моль}} \cdot 15\,\text{л}} \approx 57\,\text{кПа}.$$
    Сравниваем это давление с давлением насыщенного пара при этой температуре $P_\text{н.
    п.
    100 $\celsius$} = 101{,}300\,\text{кПа}$:
    если у нас получилось меньше табличного значения,
    то вся вода испарилась, если же больше — испарилась лишь часть, а пар является насыщенным.
    Отсюда сразу получаем давление пара: $P'_\text{пара} = 57{,}4\,\text{кПа}$.
    Сравните этот результат с первым вариантом решения.

    Тут получаем ответ: $P'_\text{пара} = 180{,}5\,\text{кПа}$.
}
\solutionspace{150pt}

\tasknumber{4}%
\task{%
    Напротив физических величин запишите определение, обозначение и единицы измерения в системе СИ (если есть):
    \begin{enumerate}
        \item относительная влажность,
        \item динамическое равновесие.
    \end{enumerate}
}

\variantsplitter

\addpersonalvariant{Константин Козлов}

\tasknumber{1}%
\task{%
    Сколько молекул водяного пара содержится в сосуде объёмом $12\,\text{л}$ при температуре $70\celsius$,
    и влажности воздуха $55\%$?
}
\answer{%
    Уравнение состояния идеального газа (и учтём, что $R = N_A \cdot k$,
    это чуть упростит выячичления, но вовсе не обязательно это делать):
    $$
        PV = \nu RT = \frac N{N_A} RT \implies N = PV \cdot \frac{N_A}{RT}=  \frac{PV}{kT}
    $$
    Плотность насыщенного водяного пара при $70\celsius$ ищем по таблице: $P_{\text{нас.
    пара 70} \celsius} = 31\,\text{кПа}.$

    Получаем плотность пара в сосуде $\varphi = \frac P{P_{\text{нас.
    пара 70} \celsius}} \implies P = \varphi P_{\text{нас.
    пара 70} \celsius}.$

    И подставляем в ответ (по сути, его можно было получить быстрее из формул $P = nkT, n = \frac NV$):
    $$
        N = \frac{\varphi \cdot P_{\text{нас.
        пара 70} \celsius} \cdot V}{kT}
         = \frac{0{,}55 \cdot 31\,\text{кПа} \cdot 12\,\text{л}}{1{,}38 \cdot 10^{-23}\,\frac{\text{Дж}}{\text{К}} \cdot 343\,\text{К}}
         \approx 430 \cdot 10^{20}.
    $$

    Другой вариант решения (через плотности) приводит в результату:
    $$
        N = N_A \nu = N_A \cdot \frac m{\mu}
          = N_A \frac{\rho V}{\mu}
          = N_A \frac{\varphi \cdot \rho_{\text{нас.
          пара 70} \celsius} \cdot V}{\mu}
          = 6{,}02 \cdot 10^{23}\,\frac{1}{\text{моль}} \cdot \frac{0{,}55 \cdot 198\,\frac{\text{г}}{\text{м}^{3}} \cdot 12\,\text{л}}{18\,\frac{\text{г}}{\text{моль}}}
          \approx 440 \cdot 10^{20}.
    $$
}
\solutionspace{160pt}

\tasknumber{2}%
\task{%
    В герметичном сосуде находится влажный воздух при температуре $40\celsius$ и относительной влажности $75\%$.
    \begin{enumerate}
        \item Чему равно парциальное давление насыщенного водяного пара при этой температуре?
        \item Чему равно парциальное давление водяного пара?
        \item Определите точку росы этого пара?
        \item Каким станет парциальное давление водяного пара, если сосуд нагреть до $80\celsius$?
        \item Чему будет равна относительная влажность воздуха, если сосуд нагреть до $80\celsius$?
        \item Получите ответ на предыдущий вопрос, используя плотности, а не давления.
    \end{enumerate}
}
\answer{%
    Парциальное давление насыщенного водяного пара при $40\celsius$ ищем по таблице: $$P_{\text{нас.
    пара 40} \celsius} = 7{,}370\,\text{кПа}.$$

    Парциальное давление водяного пара
    $$P_\text{пара 1} = \varphi_1 \cdot P_{\text{нас.
    пара 40} \celsius} = 0{,}75 \cdot 7{,}370\,\text{кПа} = 5{,}528\,\text{кПа}.$$

    Точку росы определяем по таблице: при какой температуре пар с давлением $P_\text{пара 1} = 5{,}528\,\text{кПа}$ станет насыщенным: $34{,}1\celsius$.

    После нагрева парциальное давление пара возрастёт:
    $$
        \frac{P_\text{пара 1} \cdot V}{T_1} = \nu R = \frac{P_\text{пара 2} \cdot V}{T_2}
        \implies P_\text{пара 2} = P_\text{пара 1} \cdot \frac{T_2}{T_1} = 5{,}528\,\text{кПа} \cdot \frac{353\,\text{К}}{313\,\text{К}} \approx 6{,}234\,\text{кПа}.
    $$

    Парциальное давление насыщенного водяного пара при $80\celsius$ ищем по таблице: $P_{\text{нас.
    пара 80} \celsius} = 47{,}300\,\text{кПа}$.
    Теперь определяем влажность:
    $$
        \varphi_2 = \frac{P_\text{пара 2}}{P_{\text{нас.
        пара 80} \celsius}} = \frac{6{,}234\,\text{кПа}}{47{,}300\,\text{кПа}} \approx 0{,}132 = 13{,}2\%.
    $$

    Или же выражаем то же самое через плотности (плотность не изменяется при изохорном нагревании $\rho_1 =\rho_2 = \rho$, в отличие от давления):
    $$
        \varphi_2 = \frac{\rho}{\rho_{\text{нас.
        пара 80} \celsius}} = \frac{\varphi_1\rho_{\text{нас.
        пара 40} \celsius}}{\rho_{\text{нас.
        пара 80} \celsius}}
        = \frac{0{,}75 \cdot 51{,}20\,\frac{\text{г}}{\text{м}^{3}}}{293\,\frac{\text{г}}{\text{м}^{3}}} \approx 0{,}131 = 13{,}1\%.
    $$
    Сравните 2 последних результата.
}
\solutionspace{200pt}

\tasknumber{3}%
\task{%
    Закрытый сосуд объёмом $10\,\text{л}$ заполнен сухим воздухом при давлении $100\,\text{кПа}$ и температуре $20\celsius$.
    Каким станет давление в сосуде, если в него налить $30\,\text{г}$ воды и нагреть содержимое сосуда до $100\celsius$?
}
\answer{%
    Конечное давление газа в сосуде складывается по закону Дальтона из давления нагретого сухого воздуха $P'_\text{воздуха}$ и
    давления насыщенного пара $P_\text{пара}$:
    $$P' = P'_\text{воздуха} + P_\text{пара}.$$

    Сперва определим новое давление сухого воздуха из уравнения состояния идеального газа:
    $$
        \frac{P'_\text{воздуха} \cdot V}{T'} = \nu R = \frac{P \cdot V}{T}
        \implies P'_\text{воздуха} = P \cdot \frac{T'}{T} = 100\,\text{кПа} \cdot \frac{373\,\text{К}}{293\,\text{К}} \approx 127\,\text{кПа}.
    $$

    Чтобы найти давление пара, нужно понять, будет ли он насыщенным после нагрева или нет.

    Плотность насыщенного пара при температуре $100\celsius$ равна $598\,\frac{\text{г}}{\text{м}^{3}}$, тогда для того,
    чтобы весь сосуд был заполнен насыщенным водяным паром нужно
    $m_\text{н.
    п.} = \rho_\text{н.
    п.
    100 $\celsius$} \cdot V = 598\,\frac{\text{г}}{\text{м}^{3}} \cdot 10\,\text{л} \approx 6{,}0\,\text{г}$ воды.
    Сравнивая эту массу с массой воды из условия, получаем массу жидкости, которая испарится: $m_\text{пара} = 6\,\text{г}$.
    Осталось определить давление этого пара:
    $$P_\text{пара} = \frac{m_\text{пара}RT'}{\mu V} = \frac{6\,\text{г} \cdot 8{,}31\,\frac{\text{Дж}}{\text{моль}\cdot\text{К}} \cdot 373\,\text{К}}{18\,\frac{\text{г}}{\text{моль}} \cdot 10\,\text{л}} \approx 103\,\text{кПа}.$$

    Получаем ответ: $P'_\text{пара} = 230{,}6\,\text{кПа}$.

    Другой вариант решения для давления пара:
    Определим давление пара, если бы вся вода испарилась (что не факт):
    $$P_\text{max} = \frac{mRT'}{\mu V} = \frac{30\,\text{г} \cdot 8{,}31\,\frac{\text{Дж}}{\text{моль}\cdot\text{К}} \cdot 373\,\text{К}}{18\,\frac{\text{г}}{\text{моль}} \cdot 10\,\text{л}} \approx 520\,\text{кПа}.$$
    Сравниваем это давление с давлением насыщенного пара при этой температуре $P_\text{н.
    п.
    100 $\celsius$} = 101{,}300\,\text{кПа}$:
    если у нас получилось меньше табличного значения,
    то вся вода испарилась, если же больше — испарилась лишь часть, а пар является насыщенным.
    Отсюда сразу получаем давление пара: $P'_\text{пара} = 101{,}3\,\text{кПа}$.
    Сравните этот результат с первым вариантом решения.

    Тут получаем ответ: $P'_\text{пара} = 228{,}6\,\text{кПа}$.
}
\solutionspace{150pt}

\tasknumber{4}%
\task{%
    Напротив физических величин запишите определение, обозначение и единицы измерения в системе СИ (если есть):
    \begin{enumerate}
        \item абсолютная влажность,
        \item динамическое равновесие.
    \end{enumerate}
}

\variantsplitter

\addpersonalvariant{Наталья Кравченко}

\tasknumber{1}%
\task{%
    Сколько молекул водяного пара содержится в сосуде объёмом $15\,\text{л}$ при температуре $60\celsius$,
    и влажности воздуха $75\%$?
}
\answer{%
    Уравнение состояния идеального газа (и учтём, что $R = N_A \cdot k$,
    это чуть упростит выячичления, но вовсе не обязательно это делать):
    $$
        PV = \nu RT = \frac N{N_A} RT \implies N = PV \cdot \frac{N_A}{RT}=  \frac{PV}{kT}
    $$
    Плотность насыщенного водяного пара при $60\celsius$ ищем по таблице: $P_{\text{нас.
    пара 60} \celsius} = 19{,}900\,\text{кПа}.$

    Получаем плотность пара в сосуде $\varphi = \frac P{P_{\text{нас.
    пара 60} \celsius}} \implies P = \varphi P_{\text{нас.
    пара 60} \celsius}.$

    И подставляем в ответ (по сути, его можно было получить быстрее из формул $P = nkT, n = \frac NV$):
    $$
        N = \frac{\varphi \cdot P_{\text{нас.
        пара 60} \celsius} \cdot V}{kT}
         = \frac{0{,}75 \cdot 19{,}900\,\text{кПа} \cdot 15\,\text{л}}{1{,}38 \cdot 10^{-23}\,\frac{\text{Дж}}{\text{К}} \cdot 333\,\text{К}}
         \approx 490 \cdot 10^{20}.
    $$

    Другой вариант решения (через плотности) приводит в результату:
    $$
        N = N_A \nu = N_A \cdot \frac m{\mu}
          = N_A \frac{\rho V}{\mu}
          = N_A \frac{\varphi \cdot \rho_{\text{нас.
          пара 60} \celsius} \cdot V}{\mu}
          = 6{,}02 \cdot 10^{23}\,\frac{1}{\text{моль}} \cdot \frac{0{,}75 \cdot 130\,\frac{\text{г}}{\text{м}^{3}} \cdot 15\,\text{л}}{18\,\frac{\text{г}}{\text{моль}}}
          \approx 490 \cdot 10^{20}.
    $$
}
\solutionspace{160pt}

\tasknumber{2}%
\task{%
    В герметичном сосуде находится влажный воздух при температуре $40\celsius$ и относительной влажности $70\%$.
    \begin{enumerate}
        \item Чему равно парциальное давление насыщенного водяного пара при этой температуре?
        \item Чему равно парциальное давление водяного пара?
        \item Определите точку росы этого пара?
        \item Каким станет парциальное давление водяного пара, если сосуд нагреть до $90\celsius$?
        \item Чему будет равна относительная влажность воздуха, если сосуд нагреть до $90\celsius$?
        \item Получите ответ на предыдущий вопрос, используя плотности, а не давления.
    \end{enumerate}
}
\answer{%
    Парциальное давление насыщенного водяного пара при $40\celsius$ ищем по таблице: $$P_{\text{нас.
    пара 40} \celsius} = 7{,}370\,\text{кПа}.$$

    Парциальное давление водяного пара
    $$P_\text{пара 1} = \varphi_1 \cdot P_{\text{нас.
    пара 40} \celsius} = 0{,}70 \cdot 7{,}370\,\text{кПа} = 5{,}159\,\text{кПа}.$$

    Точку росы определяем по таблице: при какой температуре пар с давлением $P_\text{пара 1} = 5{,}159\,\text{кПа}$ станет насыщенным: $32{,}9\celsius$.

    После нагрева парциальное давление пара возрастёт:
    $$
        \frac{P_\text{пара 1} \cdot V}{T_1} = \nu R = \frac{P_\text{пара 2} \cdot V}{T_2}
        \implies P_\text{пара 2} = P_\text{пара 1} \cdot \frac{T_2}{T_1} = 5{,}159\,\text{кПа} \cdot \frac{363\,\text{К}}{313\,\text{К}} \approx 5{,}983\,\text{кПа}.
    $$

    Парциальное давление насыщенного водяного пара при $90\celsius$ ищем по таблице: $P_{\text{нас.
    пара 90} \celsius} = 70{,}100\,\text{кПа}$.
    Теперь определяем влажность:
    $$
        \varphi_2 = \frac{P_\text{пара 2}}{P_{\text{нас.
        пара 90} \celsius}} = \frac{5{,}983\,\text{кПа}}{70{,}100\,\text{кПа}} \approx 0{,}085 = 8{,}5\%.
    $$

    Или же выражаем то же самое через плотности (плотность не изменяется при изохорном нагревании $\rho_1 =\rho_2 = \rho$, в отличие от давления):
    $$
        \varphi_2 = \frac{\rho}{\rho_{\text{нас.
        пара 90} \celsius}} = \frac{\varphi_1\rho_{\text{нас.
        пара 40} \celsius}}{\rho_{\text{нас.
        пара 90} \celsius}}
        = \frac{0{,}70 \cdot 51{,}20\,\frac{\text{г}}{\text{м}^{3}}}{424\,\frac{\text{г}}{\text{м}^{3}}} \approx 0{,}085 = 8{,}5\%.
    $$
    Сравните 2 последних результата.
}
\solutionspace{200pt}

\tasknumber{3}%
\task{%
    Закрытый сосуд объёмом $15\,\text{л}$ заполнен сухим воздухом при давлении $100\,\text{кПа}$ и температуре $10\celsius$.
    Каким станет давление в сосуде, если в него налить $5\,\text{г}$ воды и нагреть содержимое сосуда до $80\celsius$?
}
\answer{%
    Конечное давление газа в сосуде складывается по закону Дальтона из давления нагретого сухого воздуха $P'_\text{воздуха}$ и
    давления насыщенного пара $P_\text{пара}$:
    $$P' = P'_\text{воздуха} + P_\text{пара}.$$

    Сперва определим новое давление сухого воздуха из уравнения состояния идеального газа:
    $$
        \frac{P'_\text{воздуха} \cdot V}{T'} = \nu R = \frac{P \cdot V}{T}
        \implies P'_\text{воздуха} = P \cdot \frac{T'}{T} = 100\,\text{кПа} \cdot \frac{353\,\text{К}}{283\,\text{К}} \approx 125\,\text{кПа}.
    $$

    Чтобы найти давление пара, нужно понять, будет ли он насыщенным после нагрева или нет.

    Плотность насыщенного пара при температуре $80\celsius$ равна $293\,\frac{\text{г}}{\text{м}^{3}}$, тогда для того,
    чтобы весь сосуд был заполнен насыщенным водяным паром нужно
    $m_\text{н.
    п.} = \rho_\text{н.
    п.
    80 $\celsius$} \cdot V = 293\,\frac{\text{г}}{\text{м}^{3}} \cdot 15\,\text{л} \approx 4{,}4\,\text{г}$ воды.
    Сравнивая эту массу с массой воды из условия, получаем массу жидкости, которая испарится: $m_\text{пара} = 4{,}4\,\text{г}$.
    Осталось определить давление этого пара:
    $$P_\text{пара} = \frac{m_\text{пара}RT'}{\mu V} = \frac{4{,}4\,\text{г} \cdot 8{,}31\,\frac{\text{Дж}}{\text{моль}\cdot\text{К}} \cdot 353\,\text{К}}{18\,\frac{\text{г}}{\text{моль}} \cdot 15\,\text{л}} \approx 48\,\text{кПа}.$$

    Получаем ответ: $P'_\text{пара} = 172{,}5\,\text{кПа}$.

    Другой вариант решения для давления пара:
    Определим давление пара, если бы вся вода испарилась (что не факт):
    $$P_\text{max} = \frac{mRT'}{\mu V} = \frac{5\,\text{г} \cdot 8{,}31\,\frac{\text{Дж}}{\text{моль}\cdot\text{К}} \cdot 353\,\text{К}}{18\,\frac{\text{г}}{\text{моль}} \cdot 15\,\text{л}} \approx 54\,\text{кПа}.$$
    Сравниваем это давление с давлением насыщенного пара при этой температуре $P_\text{н.
    п.
    80 $\celsius$} = 47{,}300\,\text{кПа}$:
    если у нас получилось меньше табличного значения,
    то вся вода испарилась, если же больше — испарилась лишь часть, а пар является насыщенным.
    Отсюда сразу получаем давление пара: $P'_\text{пара} = 47{,}3\,\text{кПа}$.
    Сравните этот результат с первым вариантом решения.

    Тут получаем ответ: $P'_\text{пара} = 172\,\text{кПа}$.
}
\solutionspace{150pt}

\tasknumber{4}%
\task{%
    Напротив физических величин запишите определение, обозначение и единицы измерения в системе СИ (если есть):
    \begin{enumerate}
        \item относительная влажность,
        \item насыщенный пар.
    \end{enumerate}
}

\variantsplitter

\addpersonalvariant{Матвей Кузьмин}

\tasknumber{1}%
\task{%
    Сколько молекул водяного пара содержится в сосуде объёмом $15\,\text{л}$ при температуре $20\celsius$,
    и влажности воздуха $20\%$?
}
\answer{%
    Уравнение состояния идеального газа (и учтём, что $R = N_A \cdot k$,
    это чуть упростит выячичления, но вовсе не обязательно это делать):
    $$
        PV = \nu RT = \frac N{N_A} RT \implies N = PV \cdot \frac{N_A}{RT}=  \frac{PV}{kT}
    $$
    Плотность насыщенного водяного пара при $20\celsius$ ищем по таблице: $P_{\text{нас.
    пара 20} \celsius} = 2{,}340\,\text{кПа}.$

    Получаем плотность пара в сосуде $\varphi = \frac P{P_{\text{нас.
    пара 20} \celsius}} \implies P = \varphi P_{\text{нас.
    пара 20} \celsius}.$

    И подставляем в ответ (по сути, его можно было получить быстрее из формул $P = nkT, n = \frac NV$):
    $$
        N = \frac{\varphi \cdot P_{\text{нас.
        пара 20} \celsius} \cdot V}{kT}
         = \frac{0{,}20 \cdot 2{,}340\,\text{кПа} \cdot 15\,\text{л}}{1{,}38 \cdot 10^{-23}\,\frac{\text{Дж}}{\text{К}} \cdot 293\,\text{К}}
         \approx 17{,}4 \cdot 10^{20}.
    $$

    Другой вариант решения (через плотности) приводит в результату:
    $$
        N = N_A \nu = N_A \cdot \frac m{\mu}
          = N_A \frac{\rho V}{\mu}
          = N_A \frac{\varphi \cdot \rho_{\text{нас.
          пара 20} \celsius} \cdot V}{\mu}
          = 6{,}02 \cdot 10^{23}\,\frac{1}{\text{моль}} \cdot \frac{0{,}20 \cdot 17{,}30\,\frac{\text{г}}{\text{м}^{3}} \cdot 15\,\text{л}}{18\,\frac{\text{г}}{\text{моль}}}
          \approx 17{,}4 \cdot 10^{20}.
    $$
}
\solutionspace{160pt}

\tasknumber{2}%
\task{%
    В герметичном сосуде находится влажный воздух при температуре $25\celsius$ и относительной влажности $45\%$.
    \begin{enumerate}
        \item Чему равно парциальное давление насыщенного водяного пара при этой температуре?
        \item Чему равно парциальное давление водяного пара?
        \item Определите точку росы этого пара?
        \item Каким станет парциальное давление водяного пара, если сосуд нагреть до $70\celsius$?
        \item Чему будет равна относительная влажность воздуха, если сосуд нагреть до $70\celsius$?
        \item Получите ответ на предыдущий вопрос, используя плотности, а не давления.
    \end{enumerate}
}
\answer{%
    Парциальное давление насыщенного водяного пара при $25\celsius$ ищем по таблице: $$P_{\text{нас.
    пара 25} \celsius} = 3{,}170\,\text{кПа}.$$

    Парциальное давление водяного пара
    $$P_\text{пара 1} = \varphi_1 \cdot P_{\text{нас.
    пара 25} \celsius} = 0{,}45 \cdot 3{,}170\,\text{кПа} = 1{,}4265\,\text{кПа}.$$

    Точку росы определяем по таблице: при какой температуре пар с давлением $P_\text{пара 1} = 1{,}4265\,\text{кПа}$ станет насыщенным: $12{,}3\celsius$.

    После нагрева парциальное давление пара возрастёт:
    $$
        \frac{P_\text{пара 1} \cdot V}{T_1} = \nu R = \frac{P_\text{пара 2} \cdot V}{T_2}
        \implies P_\text{пара 2} = P_\text{пара 1} \cdot \frac{T_2}{T_1} = 1{,}4265\,\text{кПа} \cdot \frac{343\,\text{К}}{298\,\text{К}} \approx 1{,}6419\,\text{кПа}.
    $$

    Парциальное давление насыщенного водяного пара при $70\celsius$ ищем по таблице: $P_{\text{нас.
    пара 70} \celsius} = 31\,\text{кПа}$.
    Теперь определяем влажность:
    $$
        \varphi_2 = \frac{P_\text{пара 2}}{P_{\text{нас.
        пара 70} \celsius}} = \frac{1{,}6419\,\text{кПа}}{31\,\text{кПа}} \approx 0{,}053 = 5{,}3\%.
    $$

    Или же выражаем то же самое через плотности (плотность не изменяется при изохорном нагревании $\rho_1 =\rho_2 = \rho$, в отличие от давления):
    $$
        \varphi_2 = \frac{\rho}{\rho_{\text{нас.
        пара 70} \celsius}} = \frac{\varphi_1\rho_{\text{нас.
        пара 25} \celsius}}{\rho_{\text{нас.
        пара 70} \celsius}}
        = \frac{0{,}45 \cdot 23\,\frac{\text{г}}{\text{м}^{3}}}{198\,\frac{\text{г}}{\text{м}^{3}}} \approx 0{,}052 = 5{,}2\%.
    $$
    Сравните 2 последних результата.
}
\solutionspace{200pt}

\tasknumber{3}%
\task{%
    Закрытый сосуд объёмом $10\,\text{л}$ заполнен сухим воздухом при давлении $100\,\text{кПа}$ и температуре $30\celsius$.
    Каким станет давление в сосуде, если в него налить $5\,\text{г}$ воды и нагреть содержимое сосуда до $90\celsius$?
}
\answer{%
    Конечное давление газа в сосуде складывается по закону Дальтона из давления нагретого сухого воздуха $P'_\text{воздуха}$ и
    давления насыщенного пара $P_\text{пара}$:
    $$P' = P'_\text{воздуха} + P_\text{пара}.$$

    Сперва определим новое давление сухого воздуха из уравнения состояния идеального газа:
    $$
        \frac{P'_\text{воздуха} \cdot V}{T'} = \nu R = \frac{P \cdot V}{T}
        \implies P'_\text{воздуха} = P \cdot \frac{T'}{T} = 100\,\text{кПа} \cdot \frac{363\,\text{К}}{303\,\text{К}} \approx 120\,\text{кПа}.
    $$

    Чтобы найти давление пара, нужно понять, будет ли он насыщенным после нагрева или нет.

    Плотность насыщенного пара при температуре $90\celsius$ равна $424\,\frac{\text{г}}{\text{м}^{3}}$, тогда для того,
    чтобы весь сосуд был заполнен насыщенным водяным паром нужно
    $m_\text{н.
    п.} = \rho_\text{н.
    п.
    90 $\celsius$} \cdot V = 424\,\frac{\text{г}}{\text{м}^{3}} \cdot 10\,\text{л} \approx 4{,}2\,\text{г}$ воды.
    Сравнивая эту массу с массой воды из условия, получаем массу жидкости, которая испарится: $m_\text{пара} = 4{,}2\,\text{г}$.
    Осталось определить давление этого пара:
    $$P_\text{пара} = \frac{m_\text{пара}RT'}{\mu V} = \frac{4{,}2\,\text{г} \cdot 8{,}31\,\frac{\text{Дж}}{\text{моль}\cdot\text{К}} \cdot 363\,\text{К}}{18\,\frac{\text{г}}{\text{моль}} \cdot 10\,\text{л}} \approx 70\,\text{кПа}.$$

    Получаем ответ: $P'_\text{пара} = 190{,}2\,\text{кПа}$.

    Другой вариант решения для давления пара:
    Определим давление пара, если бы вся вода испарилась (что не факт):
    $$P_\text{max} = \frac{mRT'}{\mu V} = \frac{5\,\text{г} \cdot 8{,}31\,\frac{\text{Дж}}{\text{моль}\cdot\text{К}} \cdot 363\,\text{К}}{18\,\frac{\text{г}}{\text{моль}} \cdot 10\,\text{л}} \approx 84\,\text{кПа}.$$
    Сравниваем это давление с давлением насыщенного пара при этой температуре $P_\text{н.
    п.
    90 $\celsius$} = 70{,}100\,\text{кПа}$:
    если у нас получилось меньше табличного значения,
    то вся вода испарилась, если же больше — испарилась лишь часть, а пар является насыщенным.
    Отсюда сразу получаем давление пара: $P'_\text{пара} = 70{,}1\,\text{кПа}$.
    Сравните этот результат с первым вариантом решения.

    Тут получаем ответ: $P'_\text{пара} = 189{,}9\,\text{кПа}$.
}
\solutionspace{150pt}

\tasknumber{4}%
\task{%
    Напротив физических величин запишите определение, обозначение и единицы измерения в системе СИ (если есть):
    \begin{enumerate}
        \item относительная влажность,
        \item динамическое равновесие.
    \end{enumerate}
}

\variantsplitter

\addpersonalvariant{Сергей Малышев}

\tasknumber{1}%
\task{%
    Сколько молекул водяного пара содержится в сосуде объёмом $3\,\text{л}$ при температуре $25\celsius$,
    и влажности воздуха $75\%$?
}
\answer{%
    Уравнение состояния идеального газа (и учтём, что $R = N_A \cdot k$,
    это чуть упростит выячичления, но вовсе не обязательно это делать):
    $$
        PV = \nu RT = \frac N{N_A} RT \implies N = PV \cdot \frac{N_A}{RT}=  \frac{PV}{kT}
    $$
    Плотность насыщенного водяного пара при $25\celsius$ ищем по таблице: $P_{\text{нас.
    пара 25} \celsius} = 3{,}170\,\text{кПа}.$

    Получаем плотность пара в сосуде $\varphi = \frac P{P_{\text{нас.
    пара 25} \celsius}} \implies P = \varphi P_{\text{нас.
    пара 25} \celsius}.$

    И подставляем в ответ (по сути, его можно было получить быстрее из формул $P = nkT, n = \frac NV$):
    $$
        N = \frac{\varphi \cdot P_{\text{нас.
        пара 25} \celsius} \cdot V}{kT}
         = \frac{0{,}75 \cdot 3{,}170\,\text{кПа} \cdot 3\,\text{л}}{1{,}38 \cdot 10^{-23}\,\frac{\text{Дж}}{\text{К}} \cdot 298\,\text{К}}
         \approx 17{,}3 \cdot 10^{20}.
    $$

    Другой вариант решения (через плотности) приводит в результату:
    $$
        N = N_A \nu = N_A \cdot \frac m{\mu}
          = N_A \frac{\rho V}{\mu}
          = N_A \frac{\varphi \cdot \rho_{\text{нас.
          пара 25} \celsius} \cdot V}{\mu}
          = 6{,}02 \cdot 10^{23}\,\frac{1}{\text{моль}} \cdot \frac{0{,}75 \cdot 23\,\frac{\text{г}}{\text{м}^{3}} \cdot 3\,\text{л}}{18\,\frac{\text{г}}{\text{моль}}}
          \approx 17{,}3 \cdot 10^{20}.
    $$
}
\solutionspace{160pt}

\tasknumber{2}%
\task{%
    В герметичном сосуде находится влажный воздух при температуре $20\celsius$ и относительной влажности $55\%$.
    \begin{enumerate}
        \item Чему равно парциальное давление насыщенного водяного пара при этой температуре?
        \item Чему равно парциальное давление водяного пара?
        \item Определите точку росы этого пара?
        \item Каким станет парциальное давление водяного пара, если сосуд нагреть до $80\celsius$?
        \item Чему будет равна относительная влажность воздуха, если сосуд нагреть до $80\celsius$?
        \item Получите ответ на предыдущий вопрос, используя плотности, а не давления.
    \end{enumerate}
}
\answer{%
    Парциальное давление насыщенного водяного пара при $20\celsius$ ищем по таблице: $$P_{\text{нас.
    пара 20} \celsius} = 2{,}340\,\text{кПа}.$$

    Парциальное давление водяного пара
    $$P_\text{пара 1} = \varphi_1 \cdot P_{\text{нас.
    пара 20} \celsius} = 0{,}55 \cdot 2{,}340\,\text{кПа} = 1{,}2870\,\text{кПа}.$$

    Точку росы определяем по таблице: при какой температуре пар с давлением $P_\text{пара 1} = 1{,}2870\,\text{кПа}$ станет насыщенным: $10{,}7\celsius$.

    После нагрева парциальное давление пара возрастёт:
    $$
        \frac{P_\text{пара 1} \cdot V}{T_1} = \nu R = \frac{P_\text{пара 2} \cdot V}{T_2}
        \implies P_\text{пара 2} = P_\text{пара 1} \cdot \frac{T_2}{T_1} = 1{,}2870\,\text{кПа} \cdot \frac{353\,\text{К}}{293\,\text{К}} \approx 1{,}5505\,\text{кПа}.
    $$

    Парциальное давление насыщенного водяного пара при $80\celsius$ ищем по таблице: $P_{\text{нас.
    пара 80} \celsius} = 47{,}300\,\text{кПа}$.
    Теперь определяем влажность:
    $$
        \varphi_2 = \frac{P_\text{пара 2}}{P_{\text{нас.
        пара 80} \celsius}} = \frac{1{,}5505\,\text{кПа}}{47{,}300\,\text{кПа}} \approx 0{,}033 = 3{,}3\%.
    $$

    Или же выражаем то же самое через плотности (плотность не изменяется при изохорном нагревании $\rho_1 =\rho_2 = \rho$, в отличие от давления):
    $$
        \varphi_2 = \frac{\rho}{\rho_{\text{нас.
        пара 80} \celsius}} = \frac{\varphi_1\rho_{\text{нас.
        пара 20} \celsius}}{\rho_{\text{нас.
        пара 80} \celsius}}
        = \frac{0{,}55 \cdot 17{,}30\,\frac{\text{г}}{\text{м}^{3}}}{293\,\frac{\text{г}}{\text{м}^{3}}} \approx 0{,}032 = 3{,}2\%.
    $$
    Сравните 2 последних результата.
}
\solutionspace{200pt}

\tasknumber{3}%
\task{%
    Закрытый сосуд объёмом $10\,\text{л}$ заполнен сухим воздухом при давлении $100\,\text{кПа}$ и температуре $20\celsius$.
    Каким станет давление в сосуде, если в него налить $30\,\text{г}$ воды и нагреть содержимое сосуда до $80\celsius$?
}
\answer{%
    Конечное давление газа в сосуде складывается по закону Дальтона из давления нагретого сухого воздуха $P'_\text{воздуха}$ и
    давления насыщенного пара $P_\text{пара}$:
    $$P' = P'_\text{воздуха} + P_\text{пара}.$$

    Сперва определим новое давление сухого воздуха из уравнения состояния идеального газа:
    $$
        \frac{P'_\text{воздуха} \cdot V}{T'} = \nu R = \frac{P \cdot V}{T}
        \implies P'_\text{воздуха} = P \cdot \frac{T'}{T} = 100\,\text{кПа} \cdot \frac{353\,\text{К}}{293\,\text{К}} \approx 120\,\text{кПа}.
    $$

    Чтобы найти давление пара, нужно понять, будет ли он насыщенным после нагрева или нет.

    Плотность насыщенного пара при температуре $80\celsius$ равна $293\,\frac{\text{г}}{\text{м}^{3}}$, тогда для того,
    чтобы весь сосуд был заполнен насыщенным водяным паром нужно
    $m_\text{н.
    п.} = \rho_\text{н.
    п.
    80 $\celsius$} \cdot V = 293\,\frac{\text{г}}{\text{м}^{3}} \cdot 10\,\text{л} \approx 2{,}9\,\text{г}$ воды.
    Сравнивая эту массу с массой воды из условия, получаем массу жидкости, которая испарится: $m_\text{пара} = 2{,}9\,\text{г}$.
    Осталось определить давление этого пара:
    $$P_\text{пара} = \frac{m_\text{пара}RT'}{\mu V} = \frac{2{,}9\,\text{г} \cdot 8{,}31\,\frac{\text{Дж}}{\text{моль}\cdot\text{К}} \cdot 353\,\text{К}}{18\,\frac{\text{г}}{\text{моль}} \cdot 10\,\text{л}} \approx 47\,\text{кПа}.$$

    Получаем ответ: $P'_\text{пара} = 167{,}7\,\text{кПа}$.

    Другой вариант решения для давления пара:
    Определим давление пара, если бы вся вода испарилась (что не факт):
    $$P_\text{max} = \frac{mRT'}{\mu V} = \frac{30\,\text{г} \cdot 8{,}31\,\frac{\text{Дж}}{\text{моль}\cdot\text{К}} \cdot 353\,\text{К}}{18\,\frac{\text{г}}{\text{моль}} \cdot 10\,\text{л}} \approx 490\,\text{кПа}.$$
    Сравниваем это давление с давлением насыщенного пара при этой температуре $P_\text{н.
    п.
    80 $\celsius$} = 47{,}300\,\text{кПа}$:
    если у нас получилось меньше табличного значения,
    то вся вода испарилась, если же больше — испарилась лишь часть, а пар является насыщенным.
    Отсюда сразу получаем давление пара: $P'_\text{пара} = 47{,}3\,\text{кПа}$.
    Сравните этот результат с первым вариантом решения.

    Тут получаем ответ: $P'_\text{пара} = 167{,}8\,\text{кПа}$.
}
\solutionspace{150pt}

\tasknumber{4}%
\task{%
    Напротив физических величин запишите определение, обозначение и единицы измерения в системе СИ (если есть):
    \begin{enumerate}
        \item абсолютная влажность,
        \item динамическое равновесие.
    \end{enumerate}
}

\variantsplitter

\addpersonalvariant{Алина Полканова}

\tasknumber{1}%
\task{%
    Сколько молекул водяного пара содержится в сосуде объёмом $3\,\text{л}$ при температуре $90\celsius$,
    и влажности воздуха $20\%$?
}
\answer{%
    Уравнение состояния идеального газа (и учтём, что $R = N_A \cdot k$,
    это чуть упростит выячичления, но вовсе не обязательно это делать):
    $$
        PV = \nu RT = \frac N{N_A} RT \implies N = PV \cdot \frac{N_A}{RT}=  \frac{PV}{kT}
    $$
    Плотность насыщенного водяного пара при $90\celsius$ ищем по таблице: $P_{\text{нас.
    пара 90} \celsius} = 70{,}100\,\text{кПа}.$

    Получаем плотность пара в сосуде $\varphi = \frac P{P_{\text{нас.
    пара 90} \celsius}} \implies P = \varphi P_{\text{нас.
    пара 90} \celsius}.$

    И подставляем в ответ (по сути, его можно было получить быстрее из формул $P = nkT, n = \frac NV$):
    $$
        N = \frac{\varphi \cdot P_{\text{нас.
        пара 90} \celsius} \cdot V}{kT}
         = \frac{0{,}20 \cdot 70{,}100\,\text{кПа} \cdot 3\,\text{л}}{1{,}38 \cdot 10^{-23}\,\frac{\text{Дж}}{\text{К}} \cdot 363\,\text{К}}
         \approx 84 \cdot 10^{20}.
    $$

    Другой вариант решения (через плотности) приводит в результату:
    $$
        N = N_A \nu = N_A \cdot \frac m{\mu}
          = N_A \frac{\rho V}{\mu}
          = N_A \frac{\varphi \cdot \rho_{\text{нас.
          пара 90} \celsius} \cdot V}{\mu}
          = 6{,}02 \cdot 10^{23}\,\frac{1}{\text{моль}} \cdot \frac{0{,}20 \cdot 424\,\frac{\text{г}}{\text{м}^{3}} \cdot 3\,\text{л}}{18\,\frac{\text{г}}{\text{моль}}}
          \approx 85 \cdot 10^{20}.
    $$
}
\solutionspace{160pt}

\tasknumber{2}%
\task{%
    В герметичном сосуде находится влажный воздух при температуре $25\celsius$ и относительной влажности $35\%$.
    \begin{enumerate}
        \item Чему равно парциальное давление насыщенного водяного пара при этой температуре?
        \item Чему равно парциальное давление водяного пара?
        \item Определите точку росы этого пара?
        \item Каким станет парциальное давление водяного пара, если сосуд нагреть до $80\celsius$?
        \item Чему будет равна относительная влажность воздуха, если сосуд нагреть до $80\celsius$?
        \item Получите ответ на предыдущий вопрос, используя плотности, а не давления.
    \end{enumerate}
}
\answer{%
    Парциальное давление насыщенного водяного пара при $25\celsius$ ищем по таблице: $$P_{\text{нас.
    пара 25} \celsius} = 3{,}170\,\text{кПа}.$$

    Парциальное давление водяного пара
    $$P_\text{пара 1} = \varphi_1 \cdot P_{\text{нас.
    пара 25} \celsius} = 0{,}35 \cdot 3{,}170\,\text{кПа} = 1{,}1095\,\text{кПа}.$$

    Точку росы определяем по таблице: при какой температуре пар с давлением $P_\text{пара 1} = 1{,}1095\,\text{кПа}$ станет насыщенным: $8{,}5\celsius$.

    После нагрева парциальное давление пара возрастёт:
    $$
        \frac{P_\text{пара 1} \cdot V}{T_1} = \nu R = \frac{P_\text{пара 2} \cdot V}{T_2}
        \implies P_\text{пара 2} = P_\text{пара 1} \cdot \frac{T_2}{T_1} = 1{,}1095\,\text{кПа} \cdot \frac{353\,\text{К}}{298\,\text{К}} \approx 1{,}3143\,\text{кПа}.
    $$

    Парциальное давление насыщенного водяного пара при $80\celsius$ ищем по таблице: $P_{\text{нас.
    пара 80} \celsius} = 47{,}300\,\text{кПа}$.
    Теперь определяем влажность:
    $$
        \varphi_2 = \frac{P_\text{пара 2}}{P_{\text{нас.
        пара 80} \celsius}} = \frac{1{,}3143\,\text{кПа}}{47{,}300\,\text{кПа}} \approx 0{,}028 = 2{,}8\%.
    $$

    Или же выражаем то же самое через плотности (плотность не изменяется при изохорном нагревании $\rho_1 =\rho_2 = \rho$, в отличие от давления):
    $$
        \varphi_2 = \frac{\rho}{\rho_{\text{нас.
        пара 80} \celsius}} = \frac{\varphi_1\rho_{\text{нас.
        пара 25} \celsius}}{\rho_{\text{нас.
        пара 80} \celsius}}
        = \frac{0{,}35 \cdot 23\,\frac{\text{г}}{\text{м}^{3}}}{293\,\frac{\text{г}}{\text{м}^{3}}} \approx 0{,}027 = 2{,}7\%.
    $$
    Сравните 2 последних результата.
}
\solutionspace{200pt}

\tasknumber{3}%
\task{%
    Закрытый сосуд объёмом $10\,\text{л}$ заполнен сухим воздухом при давлении $100\,\text{кПа}$ и температуре $10\celsius$.
    Каким станет давление в сосуде, если в него налить $5\,\text{г}$ воды и нагреть содержимое сосуда до $80\celsius$?
}
\answer{%
    Конечное давление газа в сосуде складывается по закону Дальтона из давления нагретого сухого воздуха $P'_\text{воздуха}$ и
    давления насыщенного пара $P_\text{пара}$:
    $$P' = P'_\text{воздуха} + P_\text{пара}.$$

    Сперва определим новое давление сухого воздуха из уравнения состояния идеального газа:
    $$
        \frac{P'_\text{воздуха} \cdot V}{T'} = \nu R = \frac{P \cdot V}{T}
        \implies P'_\text{воздуха} = P \cdot \frac{T'}{T} = 100\,\text{кПа} \cdot \frac{353\,\text{К}}{283\,\text{К}} \approx 125\,\text{кПа}.
    $$

    Чтобы найти давление пара, нужно понять, будет ли он насыщенным после нагрева или нет.

    Плотность насыщенного пара при температуре $80\celsius$ равна $293\,\frac{\text{г}}{\text{м}^{3}}$, тогда для того,
    чтобы весь сосуд был заполнен насыщенным водяным паром нужно
    $m_\text{н.
    п.} = \rho_\text{н.
    п.
    80 $\celsius$} \cdot V = 293\,\frac{\text{г}}{\text{м}^{3}} \cdot 10\,\text{л} \approx 2{,}9\,\text{г}$ воды.
    Сравнивая эту массу с массой воды из условия, получаем массу жидкости, которая испарится: $m_\text{пара} = 2{,}9\,\text{г}$.
    Осталось определить давление этого пара:
    $$P_\text{пара} = \frac{m_\text{пара}RT'}{\mu V} = \frac{2{,}9\,\text{г} \cdot 8{,}31\,\frac{\text{Дж}}{\text{моль}\cdot\text{К}} \cdot 353\,\text{К}}{18\,\frac{\text{г}}{\text{моль}} \cdot 10\,\text{л}} \approx 47\,\text{кПа}.$$

    Получаем ответ: $P'_\text{пара} = 172\,\text{кПа}$.

    Другой вариант решения для давления пара:
    Определим давление пара, если бы вся вода испарилась (что не факт):
    $$P_\text{max} = \frac{mRT'}{\mu V} = \frac{5\,\text{г} \cdot 8{,}31\,\frac{\text{Дж}}{\text{моль}\cdot\text{К}} \cdot 353\,\text{К}}{18\,\frac{\text{г}}{\text{моль}} \cdot 10\,\text{л}} \approx 81\,\text{кПа}.$$
    Сравниваем это давление с давлением насыщенного пара при этой температуре $P_\text{н.
    п.
    80 $\celsius$} = 47{,}300\,\text{кПа}$:
    если у нас получилось меньше табличного значения,
    то вся вода испарилась, если же больше — испарилась лишь часть, а пар является насыщенным.
    Отсюда сразу получаем давление пара: $P'_\text{пара} = 47{,}3\,\text{кПа}$.
    Сравните этот результат с первым вариантом решения.

    Тут получаем ответ: $P'_\text{пара} = 172\,\text{кПа}$.
}
\solutionspace{150pt}

\tasknumber{4}%
\task{%
    Напротив физических величин запишите определение, обозначение и единицы измерения в системе СИ (если есть):
    \begin{enumerate}
        \item абсолютная влажность,
        \item насыщенный пар.
    \end{enumerate}
}

\variantsplitter

\addpersonalvariant{Сергей Пономарёв}

\tasknumber{1}%
\task{%
    Сколько молекул водяного пара содержится в сосуде объёмом $15\,\text{л}$ при температуре $60\celsius$,
    и влажности воздуха $75\%$?
}
\answer{%
    Уравнение состояния идеального газа (и учтём, что $R = N_A \cdot k$,
    это чуть упростит выячичления, но вовсе не обязательно это делать):
    $$
        PV = \nu RT = \frac N{N_A} RT \implies N = PV \cdot \frac{N_A}{RT}=  \frac{PV}{kT}
    $$
    Плотность насыщенного водяного пара при $60\celsius$ ищем по таблице: $P_{\text{нас.
    пара 60} \celsius} = 19{,}900\,\text{кПа}.$

    Получаем плотность пара в сосуде $\varphi = \frac P{P_{\text{нас.
    пара 60} \celsius}} \implies P = \varphi P_{\text{нас.
    пара 60} \celsius}.$

    И подставляем в ответ (по сути, его можно было получить быстрее из формул $P = nkT, n = \frac NV$):
    $$
        N = \frac{\varphi \cdot P_{\text{нас.
        пара 60} \celsius} \cdot V}{kT}
         = \frac{0{,}75 \cdot 19{,}900\,\text{кПа} \cdot 15\,\text{л}}{1{,}38 \cdot 10^{-23}\,\frac{\text{Дж}}{\text{К}} \cdot 333\,\text{К}}
         \approx 490 \cdot 10^{20}.
    $$

    Другой вариант решения (через плотности) приводит в результату:
    $$
        N = N_A \nu = N_A \cdot \frac m{\mu}
          = N_A \frac{\rho V}{\mu}
          = N_A \frac{\varphi \cdot \rho_{\text{нас.
          пара 60} \celsius} \cdot V}{\mu}
          = 6{,}02 \cdot 10^{23}\,\frac{1}{\text{моль}} \cdot \frac{0{,}75 \cdot 130\,\frac{\text{г}}{\text{м}^{3}} \cdot 15\,\text{л}}{18\,\frac{\text{г}}{\text{моль}}}
          \approx 490 \cdot 10^{20}.
    $$
}
\solutionspace{160pt}

\tasknumber{2}%
\task{%
    В герметичном сосуде находится влажный воздух при температуре $30\celsius$ и относительной влажности $40\%$.
    \begin{enumerate}
        \item Чему равно парциальное давление насыщенного водяного пара при этой температуре?
        \item Чему равно парциальное давление водяного пара?
        \item Определите точку росы этого пара?
        \item Каким станет парциальное давление водяного пара, если сосуд нагреть до $90\celsius$?
        \item Чему будет равна относительная влажность воздуха, если сосуд нагреть до $90\celsius$?
        \item Получите ответ на предыдущий вопрос, используя плотности, а не давления.
    \end{enumerate}
}
\answer{%
    Парциальное давление насыщенного водяного пара при $30\celsius$ ищем по таблице: $$P_{\text{нас.
    пара 30} \celsius} = 4{,}240\,\text{кПа}.$$

    Парциальное давление водяного пара
    $$P_\text{пара 1} = \varphi_1 \cdot P_{\text{нас.
    пара 30} \celsius} = 0{,}40 \cdot 4{,}240\,\text{кПа} = 1{,}6960\,\text{кПа}.$$

    Точку росы определяем по таблице: при какой температуре пар с давлением $P_\text{пара 1} = 1{,}6960\,\text{кПа}$ станет насыщенным: $15{,}0\celsius$.

    После нагрева парциальное давление пара возрастёт:
    $$
        \frac{P_\text{пара 1} \cdot V}{T_1} = \nu R = \frac{P_\text{пара 2} \cdot V}{T_2}
        \implies P_\text{пара 2} = P_\text{пара 1} \cdot \frac{T_2}{T_1} = 1{,}6960\,\text{кПа} \cdot \frac{363\,\text{К}}{303\,\text{К}} \approx 2{,}032\,\text{кПа}.
    $$

    Парциальное давление насыщенного водяного пара при $90\celsius$ ищем по таблице: $P_{\text{нас.
    пара 90} \celsius} = 70{,}100\,\text{кПа}$.
    Теперь определяем влажность:
    $$
        \varphi_2 = \frac{P_\text{пара 2}}{P_{\text{нас.
        пара 90} \celsius}} = \frac{2{,}032\,\text{кПа}}{70{,}100\,\text{кПа}} \approx 0{,}029 = 2{,}9\%.
    $$

    Или же выражаем то же самое через плотности (плотность не изменяется при изохорном нагревании $\rho_1 =\rho_2 = \rho$, в отличие от давления):
    $$
        \varphi_2 = \frac{\rho}{\rho_{\text{нас.
        пара 90} \celsius}} = \frac{\varphi_1\rho_{\text{нас.
        пара 30} \celsius}}{\rho_{\text{нас.
        пара 90} \celsius}}
        = \frac{0{,}40 \cdot 30{,}30\,\frac{\text{г}}{\text{м}^{3}}}{424\,\frac{\text{г}}{\text{м}^{3}}} \approx 0{,}029 = 2{,}9\%.
    $$
    Сравните 2 последних результата.
}
\solutionspace{200pt}

\tasknumber{3}%
\task{%
    Закрытый сосуд объёмом $10\,\text{л}$ заполнен сухим воздухом при давлении $100\,\text{кПа}$ и температуре $20\celsius$.
    Каким станет давление в сосуде, если в него налить $10\,\text{г}$ воды и нагреть содержимое сосуда до $80\celsius$?
}
\answer{%
    Конечное давление газа в сосуде складывается по закону Дальтона из давления нагретого сухого воздуха $P'_\text{воздуха}$ и
    давления насыщенного пара $P_\text{пара}$:
    $$P' = P'_\text{воздуха} + P_\text{пара}.$$

    Сперва определим новое давление сухого воздуха из уравнения состояния идеального газа:
    $$
        \frac{P'_\text{воздуха} \cdot V}{T'} = \nu R = \frac{P \cdot V}{T}
        \implies P'_\text{воздуха} = P \cdot \frac{T'}{T} = 100\,\text{кПа} \cdot \frac{353\,\text{К}}{293\,\text{К}} \approx 120\,\text{кПа}.
    $$

    Чтобы найти давление пара, нужно понять, будет ли он насыщенным после нагрева или нет.

    Плотность насыщенного пара при температуре $80\celsius$ равна $293\,\frac{\text{г}}{\text{м}^{3}}$, тогда для того,
    чтобы весь сосуд был заполнен насыщенным водяным паром нужно
    $m_\text{н.
    п.} = \rho_\text{н.
    п.
    80 $\celsius$} \cdot V = 293\,\frac{\text{г}}{\text{м}^{3}} \cdot 10\,\text{л} \approx 2{,}9\,\text{г}$ воды.
    Сравнивая эту массу с массой воды из условия, получаем массу жидкости, которая испарится: $m_\text{пара} = 2{,}9\,\text{г}$.
    Осталось определить давление этого пара:
    $$P_\text{пара} = \frac{m_\text{пара}RT'}{\mu V} = \frac{2{,}9\,\text{г} \cdot 8{,}31\,\frac{\text{Дж}}{\text{моль}\cdot\text{К}} \cdot 353\,\text{К}}{18\,\frac{\text{г}}{\text{моль}} \cdot 10\,\text{л}} \approx 47\,\text{кПа}.$$

    Получаем ответ: $P'_\text{пара} = 167{,}7\,\text{кПа}$.

    Другой вариант решения для давления пара:
    Определим давление пара, если бы вся вода испарилась (что не факт):
    $$P_\text{max} = \frac{mRT'}{\mu V} = \frac{10\,\text{г} \cdot 8{,}31\,\frac{\text{Дж}}{\text{моль}\cdot\text{К}} \cdot 353\,\text{К}}{18\,\frac{\text{г}}{\text{моль}} \cdot 10\,\text{л}} \approx 163\,\text{кПа}.$$
    Сравниваем это давление с давлением насыщенного пара при этой температуре $P_\text{н.
    п.
    80 $\celsius$} = 47{,}300\,\text{кПа}$:
    если у нас получилось меньше табличного значения,
    то вся вода испарилась, если же больше — испарилась лишь часть, а пар является насыщенным.
    Отсюда сразу получаем давление пара: $P'_\text{пара} = 47{,}3\,\text{кПа}$.
    Сравните этот результат с первым вариантом решения.

    Тут получаем ответ: $P'_\text{пара} = 167{,}8\,\text{кПа}$.
}
\solutionspace{150pt}

\tasknumber{4}%
\task{%
    Напротив физических величин запишите определение, обозначение и единицы измерения в системе СИ (если есть):
    \begin{enumerate}
        \item относительная влажность,
        \item насыщенный пар.
    \end{enumerate}
}

\variantsplitter

\addpersonalvariant{Егор Свистушкин}

\tasknumber{1}%
\task{%
    Сколько молекул водяного пара содержится в сосуде объёмом $12\,\text{л}$ при температуре $80\celsius$,
    и влажности воздуха $25\%$?
}
\answer{%
    Уравнение состояния идеального газа (и учтём, что $R = N_A \cdot k$,
    это чуть упростит выячичления, но вовсе не обязательно это делать):
    $$
        PV = \nu RT = \frac N{N_A} RT \implies N = PV \cdot \frac{N_A}{RT}=  \frac{PV}{kT}
    $$
    Плотность насыщенного водяного пара при $80\celsius$ ищем по таблице: $P_{\text{нас.
    пара 80} \celsius} = 47{,}300\,\text{кПа}.$

    Получаем плотность пара в сосуде $\varphi = \frac P{P_{\text{нас.
    пара 80} \celsius}} \implies P = \varphi P_{\text{нас.
    пара 80} \celsius}.$

    И подставляем в ответ (по сути, его можно было получить быстрее из формул $P = nkT, n = \frac NV$):
    $$
        N = \frac{\varphi \cdot P_{\text{нас.
        пара 80} \celsius} \cdot V}{kT}
         = \frac{0{,}25 \cdot 47{,}300\,\text{кПа} \cdot 12\,\text{л}}{1{,}38 \cdot 10^{-23}\,\frac{\text{Дж}}{\text{К}} \cdot 353\,\text{К}}
         \approx 290 \cdot 10^{20}.
    $$

    Другой вариант решения (через плотности) приводит в результату:
    $$
        N = N_A \nu = N_A \cdot \frac m{\mu}
          = N_A \frac{\rho V}{\mu}
          = N_A \frac{\varphi \cdot \rho_{\text{нас.
          пара 80} \celsius} \cdot V}{\mu}
          = 6{,}02 \cdot 10^{23}\,\frac{1}{\text{моль}} \cdot \frac{0{,}25 \cdot 293\,\frac{\text{г}}{\text{м}^{3}} \cdot 12\,\text{л}}{18\,\frac{\text{г}}{\text{моль}}}
          \approx 290 \cdot 10^{20}.
    $$
}
\solutionspace{160pt}

\tasknumber{2}%
\task{%
    В герметичном сосуде находится влажный воздух при температуре $25\celsius$ и относительной влажности $45\%$.
    \begin{enumerate}
        \item Чему равно парциальное давление насыщенного водяного пара при этой температуре?
        \item Чему равно парциальное давление водяного пара?
        \item Определите точку росы этого пара?
        \item Каким станет парциальное давление водяного пара, если сосуд нагреть до $90\celsius$?
        \item Чему будет равна относительная влажность воздуха, если сосуд нагреть до $90\celsius$?
        \item Получите ответ на предыдущий вопрос, используя плотности, а не давления.
    \end{enumerate}
}
\answer{%
    Парциальное давление насыщенного водяного пара при $25\celsius$ ищем по таблице: $$P_{\text{нас.
    пара 25} \celsius} = 3{,}170\,\text{кПа}.$$

    Парциальное давление водяного пара
    $$P_\text{пара 1} = \varphi_1 \cdot P_{\text{нас.
    пара 25} \celsius} = 0{,}45 \cdot 3{,}170\,\text{кПа} = 1{,}4265\,\text{кПа}.$$

    Точку росы определяем по таблице: при какой температуре пар с давлением $P_\text{пара 1} = 1{,}4265\,\text{кПа}$ станет насыщенным: $12{,}3\celsius$.

    После нагрева парциальное давление пара возрастёт:
    $$
        \frac{P_\text{пара 1} \cdot V}{T_1} = \nu R = \frac{P_\text{пара 2} \cdot V}{T_2}
        \implies P_\text{пара 2} = P_\text{пара 1} \cdot \frac{T_2}{T_1} = 1{,}4265\,\text{кПа} \cdot \frac{363\,\text{К}}{298\,\text{К}} \approx 1{,}7376\,\text{кПа}.
    $$

    Парциальное давление насыщенного водяного пара при $90\celsius$ ищем по таблице: $P_{\text{нас.
    пара 90} \celsius} = 70{,}100\,\text{кПа}$.
    Теперь определяем влажность:
    $$
        \varphi_2 = \frac{P_\text{пара 2}}{P_{\text{нас.
        пара 90} \celsius}} = \frac{1{,}7376\,\text{кПа}}{70{,}100\,\text{кПа}} \approx 0{,}025 = 2{,}5\%.
    $$

    Или же выражаем то же самое через плотности (плотность не изменяется при изохорном нагревании $\rho_1 =\rho_2 = \rho$, в отличие от давления):
    $$
        \varphi_2 = \frac{\rho}{\rho_{\text{нас.
        пара 90} \celsius}} = \frac{\varphi_1\rho_{\text{нас.
        пара 25} \celsius}}{\rho_{\text{нас.
        пара 90} \celsius}}
        = \frac{0{,}45 \cdot 23\,\frac{\text{г}}{\text{м}^{3}}}{424\,\frac{\text{г}}{\text{м}^{3}}} \approx 0{,}024 = 2{,}4\%.
    $$
    Сравните 2 последних результата.
}
\solutionspace{200pt}

\tasknumber{3}%
\task{%
    Закрытый сосуд объёмом $15\,\text{л}$ заполнен сухим воздухом при давлении $100\,\text{кПа}$ и температуре $10\celsius$.
    Каким станет давление в сосуде, если в него налить $30\,\text{г}$ воды и нагреть содержимое сосуда до $100\celsius$?
}
\answer{%
    Конечное давление газа в сосуде складывается по закону Дальтона из давления нагретого сухого воздуха $P'_\text{воздуха}$ и
    давления насыщенного пара $P_\text{пара}$:
    $$P' = P'_\text{воздуха} + P_\text{пара}.$$

    Сперва определим новое давление сухого воздуха из уравнения состояния идеального газа:
    $$
        \frac{P'_\text{воздуха} \cdot V}{T'} = \nu R = \frac{P \cdot V}{T}
        \implies P'_\text{воздуха} = P \cdot \frac{T'}{T} = 100\,\text{кПа} \cdot \frac{373\,\text{К}}{283\,\text{К}} \approx 132\,\text{кПа}.
    $$

    Чтобы найти давление пара, нужно понять, будет ли он насыщенным после нагрева или нет.

    Плотность насыщенного пара при температуре $100\celsius$ равна $598\,\frac{\text{г}}{\text{м}^{3}}$, тогда для того,
    чтобы весь сосуд был заполнен насыщенным водяным паром нужно
    $m_\text{н.
    п.} = \rho_\text{н.
    п.
    100 $\celsius$} \cdot V = 598\,\frac{\text{г}}{\text{м}^{3}} \cdot 15\,\text{л} \approx 9{,}0\,\text{г}$ воды.
    Сравнивая эту массу с массой воды из условия, получаем массу жидкости, которая испарится: $m_\text{пара} = 9\,\text{г}$.
    Осталось определить давление этого пара:
    $$P_\text{пара} = \frac{m_\text{пара}RT'}{\mu V} = \frac{9\,\text{г} \cdot 8{,}31\,\frac{\text{Дж}}{\text{моль}\cdot\text{К}} \cdot 373\,\text{К}}{18\,\frac{\text{г}}{\text{моль}} \cdot 15\,\text{л}} \approx 103\,\text{кПа}.$$

    Получаем ответ: $P'_\text{пара} = 235{,}1\,\text{кПа}$.

    Другой вариант решения для давления пара:
    Определим давление пара, если бы вся вода испарилась (что не факт):
    $$P_\text{max} = \frac{mRT'}{\mu V} = \frac{30\,\text{г} \cdot 8{,}31\,\frac{\text{Дж}}{\text{моль}\cdot\text{К}} \cdot 373\,\text{К}}{18\,\frac{\text{г}}{\text{моль}} \cdot 15\,\text{л}} \approx 340\,\text{кПа}.$$
    Сравниваем это давление с давлением насыщенного пара при этой температуре $P_\text{н.
    п.
    100 $\celsius$} = 101{,}300\,\text{кПа}$:
    если у нас получилось меньше табличного значения,
    то вся вода испарилась, если же больше — испарилась лишь часть, а пар является насыщенным.
    Отсюда сразу получаем давление пара: $P'_\text{пара} = 101{,}3\,\text{кПа}$.
    Сравните этот результат с первым вариантом решения.

    Тут получаем ответ: $P'_\text{пара} = 233{,}1\,\text{кПа}$.
}
\solutionspace{150pt}

\tasknumber{4}%
\task{%
    Напротив физических величин запишите определение, обозначение и единицы измерения в системе СИ (если есть):
    \begin{enumerate}
        \item абсолютная влажность,
        \item насыщенный пар.
    \end{enumerate}
}

\variantsplitter

\addpersonalvariant{Дмитрий Соколов}

\tasknumber{1}%
\task{%
    Сколько молекул водяного пара содержится в сосуде объёмом $15\,\text{л}$ при температуре $25\celsius$,
    и влажности воздуха $55\%$?
}
\answer{%
    Уравнение состояния идеального газа (и учтём, что $R = N_A \cdot k$,
    это чуть упростит выячичления, но вовсе не обязательно это делать):
    $$
        PV = \nu RT = \frac N{N_A} RT \implies N = PV \cdot \frac{N_A}{RT}=  \frac{PV}{kT}
    $$
    Плотность насыщенного водяного пара при $25\celsius$ ищем по таблице: $P_{\text{нас.
    пара 25} \celsius} = 3{,}170\,\text{кПа}.$

    Получаем плотность пара в сосуде $\varphi = \frac P{P_{\text{нас.
    пара 25} \celsius}} \implies P = \varphi P_{\text{нас.
    пара 25} \celsius}.$

    И подставляем в ответ (по сути, его можно было получить быстрее из формул $P = nkT, n = \frac NV$):
    $$
        N = \frac{\varphi \cdot P_{\text{нас.
        пара 25} \celsius} \cdot V}{kT}
         = \frac{0{,}55 \cdot 3{,}170\,\text{кПа} \cdot 15\,\text{л}}{1{,}38 \cdot 10^{-23}\,\frac{\text{Дж}}{\text{К}} \cdot 298\,\text{К}}
         \approx 64 \cdot 10^{20}.
    $$

    Другой вариант решения (через плотности) приводит в результату:
    $$
        N = N_A \nu = N_A \cdot \frac m{\mu}
          = N_A \frac{\rho V}{\mu}
          = N_A \frac{\varphi \cdot \rho_{\text{нас.
          пара 25} \celsius} \cdot V}{\mu}
          = 6{,}02 \cdot 10^{23}\,\frac{1}{\text{моль}} \cdot \frac{0{,}55 \cdot 23\,\frac{\text{г}}{\text{м}^{3}} \cdot 15\,\text{л}}{18\,\frac{\text{г}}{\text{моль}}}
          \approx 63 \cdot 10^{20}.
    $$
}
\solutionspace{160pt}

\tasknumber{2}%
\task{%
    В герметичном сосуде находится влажный воздух при температуре $40\celsius$ и относительной влажности $45\%$.
    \begin{enumerate}
        \item Чему равно парциальное давление насыщенного водяного пара при этой температуре?
        \item Чему равно парциальное давление водяного пара?
        \item Определите точку росы этого пара?
        \item Каким станет парциальное давление водяного пара, если сосуд нагреть до $90\celsius$?
        \item Чему будет равна относительная влажность воздуха, если сосуд нагреть до $90\celsius$?
        \item Получите ответ на предыдущий вопрос, используя плотности, а не давления.
    \end{enumerate}
}
\answer{%
    Парциальное давление насыщенного водяного пара при $40\celsius$ ищем по таблице: $$P_{\text{нас.
    пара 40} \celsius} = 7{,}370\,\text{кПа}.$$

    Парциальное давление водяного пара
    $$P_\text{пара 1} = \varphi_1 \cdot P_{\text{нас.
    пара 40} \celsius} = 0{,}45 \cdot 7{,}370\,\text{кПа} = 3{,}317\,\text{кПа}.$$

    Точку росы определяем по таблице: при какой температуре пар с давлением $P_\text{пара 1} = 3{,}317\,\text{кПа}$ станет насыщенным: $25{,}7\celsius$.

    После нагрева парциальное давление пара возрастёт:
    $$
        \frac{P_\text{пара 1} \cdot V}{T_1} = \nu R = \frac{P_\text{пара 2} \cdot V}{T_2}
        \implies P_\text{пара 2} = P_\text{пара 1} \cdot \frac{T_2}{T_1} = 3{,}317\,\text{кПа} \cdot \frac{363\,\text{К}}{313\,\text{К}} \approx 3{,}846\,\text{кПа}.
    $$

    Парциальное давление насыщенного водяного пара при $90\celsius$ ищем по таблице: $P_{\text{нас.
    пара 90} \celsius} = 70{,}100\,\text{кПа}$.
    Теперь определяем влажность:
    $$
        \varphi_2 = \frac{P_\text{пара 2}}{P_{\text{нас.
        пара 90} \celsius}} = \frac{3{,}846\,\text{кПа}}{70{,}100\,\text{кПа}} \approx 0{,}055 = 5{,}5\%.
    $$

    Или же выражаем то же самое через плотности (плотность не изменяется при изохорном нагревании $\rho_1 =\rho_2 = \rho$, в отличие от давления):
    $$
        \varphi_2 = \frac{\rho}{\rho_{\text{нас.
        пара 90} \celsius}} = \frac{\varphi_1\rho_{\text{нас.
        пара 40} \celsius}}{\rho_{\text{нас.
        пара 90} \celsius}}
        = \frac{0{,}45 \cdot 51{,}20\,\frac{\text{г}}{\text{м}^{3}}}{424\,\frac{\text{г}}{\text{м}^{3}}} \approx 0{,}054 = 5{,}4\%.
    $$
    Сравните 2 последних результата.
}
\solutionspace{200pt}

\tasknumber{3}%
\task{%
    Закрытый сосуд объёмом $20\,\text{л}$ заполнен сухим воздухом при давлении $100\,\text{кПа}$ и температуре $20\celsius$.
    Каким станет давление в сосуде, если в него налить $10\,\text{г}$ воды и нагреть содержимое сосуда до $90\celsius$?
}
\answer{%
    Конечное давление газа в сосуде складывается по закону Дальтона из давления нагретого сухого воздуха $P'_\text{воздуха}$ и
    давления насыщенного пара $P_\text{пара}$:
    $$P' = P'_\text{воздуха} + P_\text{пара}.$$

    Сперва определим новое давление сухого воздуха из уравнения состояния идеального газа:
    $$
        \frac{P'_\text{воздуха} \cdot V}{T'} = \nu R = \frac{P \cdot V}{T}
        \implies P'_\text{воздуха} = P \cdot \frac{T'}{T} = 100\,\text{кПа} \cdot \frac{363\,\text{К}}{293\,\text{К}} \approx 124\,\text{кПа}.
    $$

    Чтобы найти давление пара, нужно понять, будет ли он насыщенным после нагрева или нет.

    Плотность насыщенного пара при температуре $90\celsius$ равна $424\,\frac{\text{г}}{\text{м}^{3}}$, тогда для того,
    чтобы весь сосуд был заполнен насыщенным водяным паром нужно
    $m_\text{н.
    п.} = \rho_\text{н.
    п.
    90 $\celsius$} \cdot V = 424\,\frac{\text{г}}{\text{м}^{3}} \cdot 20\,\text{л} \approx 8{,}48\,\text{г}$ воды.
    Сравнивая эту массу с массой воды из условия, получаем массу жидкости, которая испарится: $m_\text{пара} = 8{,}5\,\text{г}$.
    Осталось определить давление этого пара:
    $$P_\text{пара} = \frac{m_\text{пара}RT'}{\mu V} = \frac{8{,}5\,\text{г} \cdot 8{,}31\,\frac{\text{Дж}}{\text{моль}\cdot\text{К}} \cdot 363\,\text{К}}{18\,\frac{\text{г}}{\text{моль}} \cdot 20\,\text{л}} \approx 71\,\text{кПа}.$$

    Получаем ответ: $P'_\text{пара} = 195{,}1\,\text{кПа}$.

    Другой вариант решения для давления пара:
    Определим давление пара, если бы вся вода испарилась (что не факт):
    $$P_\text{max} = \frac{mRT'}{\mu V} = \frac{10\,\text{г} \cdot 8{,}31\,\frac{\text{Дж}}{\text{моль}\cdot\text{К}} \cdot 363\,\text{К}}{18\,\frac{\text{г}}{\text{моль}} \cdot 20\,\text{л}} \approx 84\,\text{кПа}.$$
    Сравниваем это давление с давлением насыщенного пара при этой температуре $P_\text{н.
    п.
    90 $\celsius$} = 70{,}100\,\text{кПа}$:
    если у нас получилось меньше табличного значения,
    то вся вода испарилась, если же больше — испарилась лишь часть, а пар является насыщенным.
    Отсюда сразу получаем давление пара: $P'_\text{пара} = 70{,}1\,\text{кПа}$.
    Сравните этот результат с первым вариантом решения.

    Тут получаем ответ: $P'_\text{пара} = 194\,\text{кПа}$.
}
\solutionspace{150pt}

\tasknumber{4}%
\task{%
    Напротив физических величин запишите определение, обозначение и единицы измерения в системе СИ (если есть):
    \begin{enumerate}
        \item относительная влажность,
        \item динамическое равновесие.
    \end{enumerate}
}

\variantsplitter

\addpersonalvariant{Арсений Трофимов}

\tasknumber{1}%
\task{%
    Сколько молекул водяного пара содержится в сосуде объёмом $7\,\text{л}$ при температуре $100\celsius$,
    и влажности воздуха $30\%$?
}
\answer{%
    Уравнение состояния идеального газа (и учтём, что $R = N_A \cdot k$,
    это чуть упростит выячичления, но вовсе не обязательно это делать):
    $$
        PV = \nu RT = \frac N{N_A} RT \implies N = PV \cdot \frac{N_A}{RT}=  \frac{PV}{kT}
    $$
    Плотность насыщенного водяного пара при $100\celsius$ ищем по таблице: $P_{\text{нас.
    пара 100} \celsius} = 101{,}300\,\text{кПа}.$

    Получаем плотность пара в сосуде $\varphi = \frac P{P_{\text{нас.
    пара 100} \celsius}} \implies P = \varphi P_{\text{нас.
    пара 100} \celsius}.$

    И подставляем в ответ (по сути, его можно было получить быстрее из формул $P = nkT, n = \frac NV$):
    $$
        N = \frac{\varphi \cdot P_{\text{нас.
        пара 100} \celsius} \cdot V}{kT}
         = \frac{0{,}30 \cdot 101{,}300\,\text{кПа} \cdot 7\,\text{л}}{1{,}38 \cdot 10^{-23}\,\frac{\text{Дж}}{\text{К}} \cdot 373\,\text{К}}
         \approx 410 \cdot 10^{20}.
    $$

    Другой вариант решения (через плотности) приводит в результату:
    $$
        N = N_A \nu = N_A \cdot \frac m{\mu}
          = N_A \frac{\rho V}{\mu}
          = N_A \frac{\varphi \cdot \rho_{\text{нас.
          пара 100} \celsius} \cdot V}{\mu}
          = 6{,}02 \cdot 10^{23}\,\frac{1}{\text{моль}} \cdot \frac{0{,}30 \cdot 598\,\frac{\text{г}}{\text{м}^{3}} \cdot 7\,\text{л}}{18\,\frac{\text{г}}{\text{моль}}}
          \approx 420 \cdot 10^{20}.
    $$
}
\solutionspace{160pt}

\tasknumber{2}%
\task{%
    В герметичном сосуде находится влажный воздух при температуре $25\celsius$ и относительной влажности $45\%$.
    \begin{enumerate}
        \item Чему равно парциальное давление насыщенного водяного пара при этой температуре?
        \item Чему равно парциальное давление водяного пара?
        \item Определите точку росы этого пара?
        \item Каким станет парциальное давление водяного пара, если сосуд нагреть до $70\celsius$?
        \item Чему будет равна относительная влажность воздуха, если сосуд нагреть до $70\celsius$?
        \item Получите ответ на предыдущий вопрос, используя плотности, а не давления.
    \end{enumerate}
}
\answer{%
    Парциальное давление насыщенного водяного пара при $25\celsius$ ищем по таблице: $$P_{\text{нас.
    пара 25} \celsius} = 3{,}170\,\text{кПа}.$$

    Парциальное давление водяного пара
    $$P_\text{пара 1} = \varphi_1 \cdot P_{\text{нас.
    пара 25} \celsius} = 0{,}45 \cdot 3{,}170\,\text{кПа} = 1{,}4265\,\text{кПа}.$$

    Точку росы определяем по таблице: при какой температуре пар с давлением $P_\text{пара 1} = 1{,}4265\,\text{кПа}$ станет насыщенным: $12{,}3\celsius$.

    После нагрева парциальное давление пара возрастёт:
    $$
        \frac{P_\text{пара 1} \cdot V}{T_1} = \nu R = \frac{P_\text{пара 2} \cdot V}{T_2}
        \implies P_\text{пара 2} = P_\text{пара 1} \cdot \frac{T_2}{T_1} = 1{,}4265\,\text{кПа} \cdot \frac{343\,\text{К}}{298\,\text{К}} \approx 1{,}6419\,\text{кПа}.
    $$

    Парциальное давление насыщенного водяного пара при $70\celsius$ ищем по таблице: $P_{\text{нас.
    пара 70} \celsius} = 31\,\text{кПа}$.
    Теперь определяем влажность:
    $$
        \varphi_2 = \frac{P_\text{пара 2}}{P_{\text{нас.
        пара 70} \celsius}} = \frac{1{,}6419\,\text{кПа}}{31\,\text{кПа}} \approx 0{,}053 = 5{,}3\%.
    $$

    Или же выражаем то же самое через плотности (плотность не изменяется при изохорном нагревании $\rho_1 =\rho_2 = \rho$, в отличие от давления):
    $$
        \varphi_2 = \frac{\rho}{\rho_{\text{нас.
        пара 70} \celsius}} = \frac{\varphi_1\rho_{\text{нас.
        пара 25} \celsius}}{\rho_{\text{нас.
        пара 70} \celsius}}
        = \frac{0{,}45 \cdot 23\,\frac{\text{г}}{\text{м}^{3}}}{198\,\frac{\text{г}}{\text{м}^{3}}} \approx 0{,}052 = 5{,}2\%.
    $$
    Сравните 2 последних результата.
}
\solutionspace{200pt}

\tasknumber{3}%
\task{%
    Закрытый сосуд объёмом $15\,\text{л}$ заполнен сухим воздухом при давлении $100\,\text{кПа}$ и температуре $20\celsius$.
    Каким станет давление в сосуде, если в него налить $10\,\text{г}$ воды и нагреть содержимое сосуда до $100\celsius$?
}
\answer{%
    Конечное давление газа в сосуде складывается по закону Дальтона из давления нагретого сухого воздуха $P'_\text{воздуха}$ и
    давления насыщенного пара $P_\text{пара}$:
    $$P' = P'_\text{воздуха} + P_\text{пара}.$$

    Сперва определим новое давление сухого воздуха из уравнения состояния идеального газа:
    $$
        \frac{P'_\text{воздуха} \cdot V}{T'} = \nu R = \frac{P \cdot V}{T}
        \implies P'_\text{воздуха} = P \cdot \frac{T'}{T} = 100\,\text{кПа} \cdot \frac{373\,\text{К}}{293\,\text{К}} \approx 127\,\text{кПа}.
    $$

    Чтобы найти давление пара, нужно понять, будет ли он насыщенным после нагрева или нет.

    Плотность насыщенного пара при температуре $100\celsius$ равна $598\,\frac{\text{г}}{\text{м}^{3}}$, тогда для того,
    чтобы весь сосуд был заполнен насыщенным водяным паром нужно
    $m_\text{н.
    п.} = \rho_\text{н.
    п.
    100 $\celsius$} \cdot V = 598\,\frac{\text{г}}{\text{м}^{3}} \cdot 15\,\text{л} \approx 9{,}0\,\text{г}$ воды.
    Сравнивая эту массу с массой воды из условия, получаем массу жидкости, которая испарится: $m_\text{пара} = 9\,\text{г}$.
    Осталось определить давление этого пара:
    $$P_\text{пара} = \frac{m_\text{пара}RT'}{\mu V} = \frac{9\,\text{г} \cdot 8{,}31\,\frac{\text{Дж}}{\text{моль}\cdot\text{К}} \cdot 373\,\text{К}}{18\,\frac{\text{г}}{\text{моль}} \cdot 15\,\text{л}} \approx 103\,\text{кПа}.$$

    Получаем ответ: $P'_\text{пара} = 230{,}6\,\text{кПа}$.

    Другой вариант решения для давления пара:
    Определим давление пара, если бы вся вода испарилась (что не факт):
    $$P_\text{max} = \frac{mRT'}{\mu V} = \frac{10\,\text{г} \cdot 8{,}31\,\frac{\text{Дж}}{\text{моль}\cdot\text{К}} \cdot 373\,\text{К}}{18\,\frac{\text{г}}{\text{моль}} \cdot 15\,\text{л}} \approx 115\,\text{кПа}.$$
    Сравниваем это давление с давлением насыщенного пара при этой температуре $P_\text{н.
    п.
    100 $\celsius$} = 101{,}300\,\text{кПа}$:
    если у нас получилось меньше табличного значения,
    то вся вода испарилась, если же больше — испарилась лишь часть, а пар является насыщенным.
    Отсюда сразу получаем давление пара: $P'_\text{пара} = 101{,}3\,\text{кПа}$.
    Сравните этот результат с первым вариантом решения.

    Тут получаем ответ: $P'_\text{пара} = 228{,}6\,\text{кПа}$.
}
\solutionspace{150pt}

\tasknumber{4}%
\task{%
    Напротив физических величин запишите определение, обозначение и единицы измерения в системе СИ (если есть):
    \begin{enumerate}
        \item абсолютная влажность,
        \item динамическое равновесие.
    \end{enumerate}
}
% autogenerated
