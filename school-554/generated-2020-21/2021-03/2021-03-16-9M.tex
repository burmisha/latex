\setdate{16~марта~2021}
\setclass{9«М»}

\addpersonalvariant{Михаил Бурмистров}

\tasknumber{1}%
\task{%
    Определите ёмкость конденсатора, если при его зарядке до напряжения
    $U = 20\,\text{кВ}$ он приобретает заряд $Q = 18\,\text{мКл}$.
    % Чему при этом равны заряды обкладок конденсатора (сделайте рисунок и укажите их)?
    Ответ выразите в нанофарадах.
}
\answer{%
    $
        Q = CU \implies
        C = \frac{Q}{U} = \frac{18\,\text{мКл}}{20\,\text{кВ}} = 900\,\text{нФ}.
        \text{ Заряды обкладок: $Q$ и $-Q$}
    $
}
\solutionspace{120pt}

\tasknumber{2}%
\task{%
    На конденсаторе указано: $C = 120\,\text{пФ}$, $V = 400\,\text{В}$.
    Удастся ли его использовать для накопления заряда $Q = 50\,\text{нКл}$?
    (в ответе укажите «да» или «нет»)
}
\answer{%
    $
        Q_{\text{max}} = CV = 120\,\text{пФ} \cdot 400\,\text{В} = 48\,\text{нКл}
        \implies Q_{\text{max}}  <  Q \implies \text{не удастся}
    $
}
\solutionspace{80pt}

\tasknumber{3}%
\task{%
    Как и во сколько раз изменится ёмкость плоского конденсатора
    при уменьшении площади пластин в 3 раз
    и уменьшении расстояния между ними в 4 раз?
    В ответе укажите простую дробь или число — отношение новой ёмкости к старой.
}
\answer{%
    $
        \frac{C'}C
            = \frac{\eps_0\eps \frac S3}{\frac d4} \Big/ \frac{\eps_0\eps S}d
            = \frac{4}{3} = > 1 \implies \text{увеличится в $\frac43$ раз}
    $
}
\solutionspace{80pt}

\tasknumber{4}%
\task{%
    Электрическая ёмкость конденсатора равна $C = 750\,\text{пФ}$,
    при этом ему сообщён заряд $q = 300\,\text{нКл}$.
    Какова энергия заряженного конденсатора?
    Ответ выразите в микроджоулях и округлите до целого.
}
\answer{%
    $
        W
        = \frac{q^2}{2C}
        = \frac{\sqr{300\,\text{нКл}}}{2 \cdot 750\,\text{пФ}}
        = 60\,\text{мкДж}
    $
}
\solutionspace{80pt}

\tasknumber{5}%
\task{%
    Напротив физических величин укажите их обозначения и единицы измерения в СИ:
    \begin{enumerate}
        \item ёмкость конденсатора,
        \item индуктивность катушки.
    \end{enumerate}
}
\solutionspace{40pt}

\tasknumber{6}%
\task{%
    Запишите формулы, выражающие:
    \begin{enumerate}
        \item заряд конденсатора через его ёмкость и поданное напряжение,
        \item энергию конденсатора через его ёмкость и заряд,
        \item частоту колебаний в электромагнитном контуре, состоящем из конденсатора и катушки индуктивности,
    \end{enumerate}
}

\variantsplitter

\addpersonalvariant{Артём Глембо}

\tasknumber{1}%
\task{%
    Определите ёмкость конденсатора, если при его зарядке до напряжения
    $V = 40\,\text{кВ}$ он приобретает заряд $q = 4\,\text{мКл}$.
    % Чему при этом равны заряды обкладок конденсатора (сделайте рисунок и укажите их)?
    Ответ выразите в нанофарадах.
}
\answer{%
    $
        q = CV \implies
        C = \frac{q}{V} = \frac{4\,\text{мКл}}{40\,\text{кВ}} = 100\,\text{нФ}.
        \text{ Заряды обкладок: $q$ и $-q$}
    $
}
\solutionspace{120pt}

\tasknumber{2}%
\task{%
    На конденсаторе указано: $C = 80\,\text{пФ}$, $U = 450\,\text{В}$.
    Удастся ли его использовать для накопления заряда $q = 60\,\text{нКл}$?
    (в ответе укажите «да» или «нет»)
}
\answer{%
    $
        q_{\text{max}} = CU = 80\,\text{пФ} \cdot 450\,\text{В} = 36\,\text{нКл}
        \implies q_{\text{max}}  <  q \implies \text{не удастся}
    $
}
\solutionspace{80pt}

\tasknumber{3}%
\task{%
    Как и во сколько раз изменится ёмкость плоского конденсатора
    при уменьшении площади пластин в 2 раз
    и уменьшении расстояния между ними в 2 раз?
    В ответе укажите простую дробь или число — отношение новой ёмкости к старой.
}
\answer{%
    $
        \frac{C'}C
            = \frac{\eps_0\eps \frac S2}{\frac d2} \Big/ \frac{\eps_0\eps S}d
            = \frac{2}{2} = = 1 \implies \text{не изменится в $1$ раз}
    $
}
\solutionspace{80pt}

\tasknumber{4}%
\task{%
    Электрическая ёмкость конденсатора равна $C = 750\,\text{пФ}$,
    при этом ему сообщён заряд $Q = 500\,\text{нКл}$.
    Какова энергия заряженного конденсатора?
    Ответ выразите в микроджоулях и округлите до целого.
}
\answer{%
    $
        W
        = \frac{Q^2}{2C}
        = \frac{\sqr{500\,\text{нКл}}}{2 \cdot 750\,\text{пФ}}
        = 166{,}67\,\text{мкДж}
    $
}
\solutionspace{80pt}

\tasknumber{5}%
\task{%
    Напротив физических величин укажите их обозначения и единицы измерения в СИ:
    \begin{enumerate}
        \item ёмкость конденсатора,
        \item индуктивность катушки.
    \end{enumerate}
}
\solutionspace{40pt}

\tasknumber{6}%
\task{%
    Запишите формулы, выражающие:
    \begin{enumerate}
        \item заряд конденсатора через его ёмкость и поданное напряжение,
        \item энергию конденсатора через его ёмкость и заряд,
        \item период колебаний в электромагнитном контуре, состоящем из конденсатора и катушки индуктивности,
    \end{enumerate}
}

\variantsplitter

\addpersonalvariant{Наталья Гончарова}

\tasknumber{1}%
\task{%
    Определите ёмкость конденсатора, если при его зарядке до напряжения
    $U = 50\,\text{кВ}$ он приобретает заряд $Q = 24\,\text{мКл}$.
    % Чему при этом равны заряды обкладок конденсатора (сделайте рисунок и укажите их)?
    Ответ выразите в нанофарадах.
}
\answer{%
    $
        Q = CU \implies
        C = \frac{Q}{U} = \frac{24\,\text{мКл}}{50\,\text{кВ}} = 480\,\text{нФ}.
        \text{ Заряды обкладок: $Q$ и $-Q$}
    $
}
\solutionspace{120pt}

\tasknumber{2}%
\task{%
    На конденсаторе указано: $C = 50\,\text{пФ}$, $V = 200\,\text{В}$.
    Удастся ли его использовать для накопления заряда $q = 60\,\text{нКл}$?
    (в ответе укажите «да» или «нет»)
}
\answer{%
    $
        q_{\text{max}} = CV = 50\,\text{пФ} \cdot 200\,\text{В} = 10\,\text{нКл}
        \implies q_{\text{max}}  <  q \implies \text{не удастся}
    $
}
\solutionspace{80pt}

\tasknumber{3}%
\task{%
    Как и во сколько раз изменится ёмкость плоского конденсатора
    при уменьшении площади пластин в 2 раз
    и уменьшении расстояния между ними в 7 раз?
    В ответе укажите простую дробь или число — отношение новой ёмкости к старой.
}
\answer{%
    $
        \frac{C'}C
            = \frac{\eps_0\eps \frac S2}{\frac d7} \Big/ \frac{\eps_0\eps S}d
            = \frac{7}{2} = > 1 \implies \text{увеличится в $\frac72$ раз}
    $
}
\solutionspace{80pt}

\tasknumber{4}%
\task{%
    Электрическая ёмкость конденсатора равна $C = 200\,\text{пФ}$,
    при этом ему сообщён заряд $q = 800\,\text{нКл}$.
    Какова энергия заряженного конденсатора?
    Ответ выразите в микроджоулях и округлите до целого.
}
\answer{%
    $
        W
        = \frac{q^2}{2C}
        = \frac{\sqr{800\,\text{нКл}}}{2 \cdot 200\,\text{пФ}}
        = 1600\,\text{мкДж}
    $
}
\solutionspace{80pt}

\tasknumber{5}%
\task{%
    Напротив физических величин укажите их обозначения и единицы измерения в СИ:
    \begin{enumerate}
        \item ёмкость конденсатора,
        \item индуктивность катушки.
    \end{enumerate}
}
\solutionspace{40pt}

\tasknumber{6}%
\task{%
    Запишите формулы, выражающие:
    \begin{enumerate}
        \item заряд конденсатора через его ёмкость и поданное напряжение,
        \item энергию конденсатора через его ёмкость и заряд,
        \item период колебаний в электромагнитном контуре, состоящем из конденсатора и катушки индуктивности,
    \end{enumerate}
}

\variantsplitter

\addpersonalvariant{Файёзбек Касымов}

\tasknumber{1}%
\task{%
    Определите ёмкость конденсатора, если при его зарядке до напряжения
    $V = 20\,\text{кВ}$ он приобретает заряд $q = 18\,\text{мКл}$.
    % Чему при этом равны заряды обкладок конденсатора (сделайте рисунок и укажите их)?
    Ответ выразите в нанофарадах.
}
\answer{%
    $
        q = CV \implies
        C = \frac{q}{V} = \frac{18\,\text{мКл}}{20\,\text{кВ}} = 900\,\text{нФ}.
        \text{ Заряды обкладок: $q$ и $-q$}
    $
}
\solutionspace{120pt}

\tasknumber{2}%
\task{%
    На конденсаторе указано: $C = 80\,\text{пФ}$, $U = 450\,\text{В}$.
    Удастся ли его использовать для накопления заряда $q = 30\,\text{нКл}$?
    (в ответе укажите «да» или «нет»)
}
\answer{%
    $
        q_{\text{max}} = CU = 80\,\text{пФ} \cdot 450\,\text{В} = 36\,\text{нКл}
        \implies q_{\text{max}} \ge q \implies \text{удастся}
    $
}
\solutionspace{80pt}

\tasknumber{3}%
\task{%
    Как и во сколько раз изменится ёмкость плоского конденсатора
    при уменьшении площади пластин в 7 раз
    и уменьшении расстояния между ними в 2 раз?
    В ответе укажите простую дробь или число — отношение новой ёмкости к старой.
}
\answer{%
    $
        \frac{C'}C
            = \frac{\eps_0\eps \frac S7}{\frac d2} \Big/ \frac{\eps_0\eps S}d
            = \frac{2}{7} = < 1 \implies \text{уменьшится в $\frac72$ раз}
    $
}
\solutionspace{80pt}

\tasknumber{4}%
\task{%
    Электрическая ёмкость конденсатора равна $C = 200\,\text{пФ}$,
    при этом ему сообщён заряд $Q = 300\,\text{нКл}$.
    Какова энергия заряженного конденсатора?
    Ответ выразите в микроджоулях и округлите до целого.
}
\answer{%
    $
        W
        = \frac{Q^2}{2C}
        = \frac{\sqr{300\,\text{нКл}}}{2 \cdot 200\,\text{пФ}}
        = 225\,\text{мкДж}
    $
}
\solutionspace{80pt}

\tasknumber{5}%
\task{%
    Напротив физических величин укажите их обозначения и единицы измерения в СИ:
    \begin{enumerate}
        \item ёмкость конденсатора,
        \item индуктивность катушки.
    \end{enumerate}
}
\solutionspace{40pt}

\tasknumber{6}%
\task{%
    Запишите формулы, выражающие:
    \begin{enumerate}
        \item заряд конденсатора через его ёмкость и поданное напряжение,
        \item энергию конденсатора через его ёмкость и поданное напряжение,
        \item частоту колебаний в электромагнитном контуре, состоящем из конденсатора и катушки индуктивности,
    \end{enumerate}
}

\variantsplitter

\addpersonalvariant{Александр Козинец}

\tasknumber{1}%
\task{%
    Определите ёмкость конденсатора, если при его зарядке до напряжения
    $U = 50\,\text{кВ}$ он приобретает заряд $Q = 24\,\text{мКл}$.
    % Чему при этом равны заряды обкладок конденсатора (сделайте рисунок и укажите их)?
    Ответ выразите в нанофарадах.
}
\answer{%
    $
        Q = CU \implies
        C = \frac{Q}{U} = \frac{24\,\text{мКл}}{50\,\text{кВ}} = 480\,\text{нФ}.
        \text{ Заряды обкладок: $Q$ и $-Q$}
    $
}
\solutionspace{120pt}

\tasknumber{2}%
\task{%
    На конденсаторе указано: $C = 50\,\text{пФ}$, $V = 400\,\text{В}$.
    Удастся ли его использовать для накопления заряда $q = 50\,\text{нКл}$?
    (в ответе укажите «да» или «нет»)
}
\answer{%
    $
        q_{\text{max}} = CV = 50\,\text{пФ} \cdot 400\,\text{В} = 20\,\text{нКл}
        \implies q_{\text{max}}  <  q \implies \text{не удастся}
    $
}
\solutionspace{80pt}

\tasknumber{3}%
\task{%
    Как и во сколько раз изменится ёмкость плоского конденсатора
    при уменьшении площади пластин в 4 раз
    и уменьшении расстояния между ними в 6 раз?
    В ответе укажите простую дробь или число — отношение новой ёмкости к старой.
}
\answer{%
    $
        \frac{C'}C
            = \frac{\eps_0\eps \frac S4}{\frac d6} \Big/ \frac{\eps_0\eps S}d
            = \frac{6}{4} = > 1 \implies \text{увеличится в $\frac32$ раз}
    $
}
\solutionspace{80pt}

\tasknumber{4}%
\task{%
    Электрическая ёмкость конденсатора равна $C = 400\,\text{пФ}$,
    при этом ему сообщён заряд $q = 500\,\text{нКл}$.
    Какова энергия заряженного конденсатора?
    Ответ выразите в микроджоулях и округлите до целого.
}
\answer{%
    $
        W
        = \frac{q^2}{2C}
        = \frac{\sqr{500\,\text{нКл}}}{2 \cdot 400\,\text{пФ}}
        = 312{,}50\,\text{мкДж}
    $
}
\solutionspace{80pt}

\tasknumber{5}%
\task{%
    Напротив физических величин укажите их обозначения и единицы измерения в СИ:
    \begin{enumerate}
        \item ёмкость конденсатора,
        \item индуктивность катушки.
    \end{enumerate}
}
\solutionspace{40pt}

\tasknumber{6}%
\task{%
    Запишите формулы, выражающие:
    \begin{enumerate}
        \item заряд конденсатора через его ёмкость и поданное напряжение,
        \item энергию конденсатора через его ёмкость и поданное напряжение,
        \item период колебаний в электромагнитном контуре, состоящем из конденсатора и катушки индуктивности,
    \end{enumerate}
}

\variantsplitter

\addpersonalvariant{Андрей Куликовский}

\tasknumber{1}%
\task{%
    Определите ёмкость конденсатора, если при его зарядке до напряжения
    $U = 50\,\text{кВ}$ он приобретает заряд $q = 25\,\text{мКл}$.
    % Чему при этом равны заряды обкладок конденсатора (сделайте рисунок и укажите их)?
    Ответ выразите в нанофарадах.
}
\answer{%
    $
        q = CU \implies
        C = \frac{q}{U} = \frac{25\,\text{мКл}}{50\,\text{кВ}} = 500\,\text{нФ}.
        \text{ Заряды обкладок: $q$ и $-q$}
    $
}
\solutionspace{120pt}

\tasknumber{2}%
\task{%
    На конденсаторе указано: $C = 120\,\text{пФ}$, $V = 400\,\text{В}$.
    Удастся ли его использовать для накопления заряда $Q = 50\,\text{нКл}$?
    (в ответе укажите «да» или «нет»)
}
\answer{%
    $
        Q_{\text{max}} = CV = 120\,\text{пФ} \cdot 400\,\text{В} = 48\,\text{нКл}
        \implies Q_{\text{max}}  <  Q \implies \text{не удастся}
    $
}
\solutionspace{80pt}

\tasknumber{3}%
\task{%
    Как и во сколько раз изменится ёмкость плоского конденсатора
    при уменьшении площади пластин в 4 раз
    и уменьшении расстояния между ними в 2 раз?
    В ответе укажите простую дробь или число — отношение новой ёмкости к старой.
}
\answer{%
    $
        \frac{C'}C
            = \frac{\eps_0\eps \frac S4}{\frac d2} \Big/ \frac{\eps_0\eps S}d
            = \frac{2}{4} = < 1 \implies \text{уменьшится в $2$ раз}
    $
}
\solutionspace{80pt}

\tasknumber{4}%
\task{%
    Электрическая ёмкость конденсатора равна $C = 400\,\text{пФ}$,
    при этом ему сообщён заряд $Q = 800\,\text{нКл}$.
    Какова энергия заряженного конденсатора?
    Ответ выразите в микроджоулях и округлите до целого.
}
\answer{%
    $
        W
        = \frac{Q^2}{2C}
        = \frac{\sqr{800\,\text{нКл}}}{2 \cdot 400\,\text{пФ}}
        = 800\,\text{мкДж}
    $
}
\solutionspace{80pt}

\tasknumber{5}%
\task{%
    Напротив физических величин укажите их обозначения и единицы измерения в СИ:
    \begin{enumerate}
        \item ёмкость конденсатора,
        \item индуктивность катушки.
    \end{enumerate}
}
\solutionspace{40pt}

\tasknumber{6}%
\task{%
    Запишите формулы, выражающие:
    \begin{enumerate}
        \item заряд конденсатора через его ёмкость и поданное напряжение,
        \item энергию конденсатора через его ёмкость и поданное напряжение,
        \item частоту колебаний в электромагнитном контуре, состоящем из конденсатора и катушки индуктивности,
    \end{enumerate}
}

\variantsplitter

\addpersonalvariant{Полина Лоткова}

\tasknumber{1}%
\task{%
    Определите ёмкость конденсатора, если при его зарядке до напряжения
    $V = 2\,\text{кВ}$ он приобретает заряд $Q = 6\,\text{мКл}$.
    % Чему при этом равны заряды обкладок конденсатора (сделайте рисунок и укажите их)?
    Ответ выразите в нанофарадах.
}
\answer{%
    $
        Q = CV \implies
        C = \frac{Q}{V} = \frac{6\,\text{мКл}}{2\,\text{кВ}} = 3000\,\text{нФ}.
        \text{ Заряды обкладок: $Q$ и $-Q$}
    $
}
\solutionspace{120pt}

\tasknumber{2}%
\task{%
    На конденсаторе указано: $C = 80\,\text{пФ}$, $U = 450\,\text{В}$.
    Удастся ли его использовать для накопления заряда $q = 50\,\text{нКл}$?
    (в ответе укажите «да» или «нет»)
}
\answer{%
    $
        q_{\text{max}} = CU = 80\,\text{пФ} \cdot 450\,\text{В} = 36\,\text{нКл}
        \implies q_{\text{max}}  <  q \implies \text{не удастся}
    $
}
\solutionspace{80pt}

\tasknumber{3}%
\task{%
    Как и во сколько раз изменится ёмкость плоского конденсатора
    при уменьшении площади пластин в 3 раз
    и уменьшении расстояния между ними в 2 раз?
    В ответе укажите простую дробь или число — отношение новой ёмкости к старой.
}
\answer{%
    $
        \frac{C'}C
            = \frac{\eps_0\eps \frac S3}{\frac d2} \Big/ \frac{\eps_0\eps S}d
            = \frac{2}{3} = < 1 \implies \text{уменьшится в $\frac32$ раз}
    $
}
\solutionspace{80pt}

\tasknumber{4}%
\task{%
    Электрическая ёмкость конденсатора равна $C = 200\,\text{пФ}$,
    при этом ему сообщён заряд $q = 300\,\text{нКл}$.
    Какова энергия заряженного конденсатора?
    Ответ выразите в микроджоулях и округлите до целого.
}
\answer{%
    $
        W
        = \frac{q^2}{2C}
        = \frac{\sqr{300\,\text{нКл}}}{2 \cdot 200\,\text{пФ}}
        = 225\,\text{мкДж}
    $
}
\solutionspace{80pt}

\tasknumber{5}%
\task{%
    Напротив физических величин укажите их обозначения и единицы измерения в СИ:
    \begin{enumerate}
        \item ёмкость конденсатора,
        \item индуктивность катушки.
    \end{enumerate}
}
\solutionspace{40pt}

\tasknumber{6}%
\task{%
    Запишите формулы, выражающие:
    \begin{enumerate}
        \item заряд конденсатора через его ёмкость и поданное напряжение,
        \item энергию конденсатора через его заряд и поданное напряжение,
        \item частоту колебаний в электромагнитном контуре, состоящем из конденсатора и катушки индуктивности,
    \end{enumerate}
}

\variantsplitter

\addpersonalvariant{Екатерина Медведева}

\tasknumber{1}%
\task{%
    Определите ёмкость конденсатора, если при его зарядке до напряжения
    $U = 40\,\text{кВ}$ он приобретает заряд $q = 6\,\text{мКл}$.
    % Чему при этом равны заряды обкладок конденсатора (сделайте рисунок и укажите их)?
    Ответ выразите в нанофарадах.
}
\answer{%
    $
        q = CU \implies
        C = \frac{q}{U} = \frac{6\,\text{мКл}}{40\,\text{кВ}} = 150\,\text{нФ}.
        \text{ Заряды обкладок: $q$ и $-q$}
    $
}
\solutionspace{120pt}

\tasknumber{2}%
\task{%
    На конденсаторе указано: $C = 50\,\text{пФ}$, $U = 400\,\text{В}$.
    Удастся ли его использовать для накопления заряда $Q = 30\,\text{нКл}$?
    (в ответе укажите «да» или «нет»)
}
\answer{%
    $
        Q_{\text{max}} = CU = 50\,\text{пФ} \cdot 400\,\text{В} = 20\,\text{нКл}
        \implies Q_{\text{max}}  <  Q \implies \text{не удастся}
    $
}
\solutionspace{80pt}

\tasknumber{3}%
\task{%
    Как и во сколько раз изменится ёмкость плоского конденсатора
    при уменьшении площади пластин в 4 раз
    и уменьшении расстояния между ними в 3 раз?
    В ответе укажите простую дробь или число — отношение новой ёмкости к старой.
}
\answer{%
    $
        \frac{C'}C
            = \frac{\eps_0\eps \frac S4}{\frac d3} \Big/ \frac{\eps_0\eps S}d
            = \frac{3}{4} = < 1 \implies \text{уменьшится в $\frac43$ раз}
    $
}
\solutionspace{80pt}

\tasknumber{4}%
\task{%
    Электрическая ёмкость конденсатора равна $C = 400\,\text{пФ}$,
    при этом ему сообщён заряд $Q = 800\,\text{нКл}$.
    Какова энергия заряженного конденсатора?
    Ответ выразите в микроджоулях и округлите до целого.
}
\answer{%
    $
        W
        = \frac{Q^2}{2C}
        = \frac{\sqr{800\,\text{нКл}}}{2 \cdot 400\,\text{пФ}}
        = 800\,\text{мкДж}
    $
}
\solutionspace{80pt}

\tasknumber{5}%
\task{%
    Напротив физических величин укажите их обозначения и единицы измерения в СИ:
    \begin{enumerate}
        \item ёмкость конденсатора,
        \item индуктивность катушки.
    \end{enumerate}
}
\solutionspace{40pt}

\tasknumber{6}%
\task{%
    Запишите формулы, выражающие:
    \begin{enumerate}
        \item заряд конденсатора через его ёмкость и поданное напряжение,
        \item энергию конденсатора через его ёмкость и поданное напряжение,
        \item частоту колебаний в электромагнитном контуре, состоящем из конденсатора и катушки индуктивности,
    \end{enumerate}
}

\variantsplitter

\addpersonalvariant{Константин Мельник}

\tasknumber{1}%
\task{%
    Определите ёмкость конденсатора, если при его зарядке до напряжения
    $U = 2\,\text{кВ}$ он приобретает заряд $Q = 4\,\text{мКл}$.
    % Чему при этом равны заряды обкладок конденсатора (сделайте рисунок и укажите их)?
    Ответ выразите в нанофарадах.
}
\answer{%
    $
        Q = CU \implies
        C = \frac{Q}{U} = \frac{4\,\text{мКл}}{2\,\text{кВ}} = 2000\,\text{нФ}.
        \text{ Заряды обкладок: $Q$ и $-Q$}
    $
}
\solutionspace{120pt}

\tasknumber{2}%
\task{%
    На конденсаторе указано: $C = 80\,\text{пФ}$, $V = 300\,\text{В}$.
    Удастся ли его использовать для накопления заряда $Q = 50\,\text{нКл}$?
    (в ответе укажите «да» или «нет»)
}
\answer{%
    $
        Q_{\text{max}} = CV = 80\,\text{пФ} \cdot 300\,\text{В} = 24\,\text{нКл}
        \implies Q_{\text{max}}  <  Q \implies \text{не удастся}
    $
}
\solutionspace{80pt}

\tasknumber{3}%
\task{%
    Как и во сколько раз изменится ёмкость плоского конденсатора
    при уменьшении площади пластин в 2 раз
    и уменьшении расстояния между ними в 6 раз?
    В ответе укажите простую дробь или число — отношение новой ёмкости к старой.
}
\answer{%
    $
        \frac{C'}C
            = \frac{\eps_0\eps \frac S2}{\frac d6} \Big/ \frac{\eps_0\eps S}d
            = \frac{6}{2} = > 1 \implies \text{увеличится в $3$ раз}
    $
}
\solutionspace{80pt}

\tasknumber{4}%
\task{%
    Электрическая ёмкость конденсатора равна $C = 750\,\text{пФ}$,
    при этом ему сообщён заряд $q = 900\,\text{нКл}$.
    Какова энергия заряженного конденсатора?
    Ответ выразите в микроджоулях и округлите до целого.
}
\answer{%
    $
        W
        = \frac{q^2}{2C}
        = \frac{\sqr{900\,\text{нКл}}}{2 \cdot 750\,\text{пФ}}
        = 540\,\text{мкДж}
    $
}
\solutionspace{80pt}

\tasknumber{5}%
\task{%
    Напротив физических величин укажите их обозначения и единицы измерения в СИ:
    \begin{enumerate}
        \item ёмкость конденсатора,
        \item индуктивность катушки.
    \end{enumerate}
}
\solutionspace{40pt}

\tasknumber{6}%
\task{%
    Запишите формулы, выражающие:
    \begin{enumerate}
        \item заряд конденсатора через его ёмкость и поданное напряжение,
        \item энергию конденсатора через его ёмкость и заряд,
        \item период колебаний в электромагнитном контуре, состоящем из конденсатора и катушки индуктивности,
    \end{enumerate}
}

\variantsplitter

\addpersonalvariant{Степан Небоваренков}

\tasknumber{1}%
\task{%
    Определите ёмкость конденсатора, если при его зарядке до напряжения
    $U = 4\,\text{кВ}$ он приобретает заряд $Q = 24\,\text{мКл}$.
    % Чему при этом равны заряды обкладок конденсатора (сделайте рисунок и укажите их)?
    Ответ выразите в нанофарадах.
}
\answer{%
    $
        Q = CU \implies
        C = \frac{Q}{U} = \frac{24\,\text{мКл}}{4\,\text{кВ}} = 6000\,\text{нФ}.
        \text{ Заряды обкладок: $Q$ и $-Q$}
    $
}
\solutionspace{120pt}

\tasknumber{2}%
\task{%
    На конденсаторе указано: $C = 50\,\text{пФ}$, $V = 450\,\text{В}$.
    Удастся ли его использовать для накопления заряда $Q = 60\,\text{нКл}$?
    (в ответе укажите «да» или «нет»)
}
\answer{%
    $
        Q_{\text{max}} = CV = 50\,\text{пФ} \cdot 450\,\text{В} = 22\,\text{нКл}
        \implies Q_{\text{max}}  <  Q \implies \text{не удастся}
    $
}
\solutionspace{80pt}

\tasknumber{3}%
\task{%
    Как и во сколько раз изменится ёмкость плоского конденсатора
    при уменьшении площади пластин в 6 раз
    и уменьшении расстояния между ними в 2 раз?
    В ответе укажите простую дробь или число — отношение новой ёмкости к старой.
}
\answer{%
    $
        \frac{C'}C
            = \frac{\eps_0\eps \frac S6}{\frac d2} \Big/ \frac{\eps_0\eps S}d
            = \frac{2}{6} = < 1 \implies \text{уменьшится в $3$ раз}
    $
}
\solutionspace{80pt}

\tasknumber{4}%
\task{%
    Электрическая ёмкость конденсатора равна $C = 600\,\text{пФ}$,
    при этом ему сообщён заряд $q = 900\,\text{нКл}$.
    Какова энергия заряженного конденсатора?
    Ответ выразите в микроджоулях и округлите до целого.
}
\answer{%
    $
        W
        = \frac{q^2}{2C}
        = \frac{\sqr{900\,\text{нКл}}}{2 \cdot 600\,\text{пФ}}
        = 675\,\text{мкДж}
    $
}
\solutionspace{80pt}

\tasknumber{5}%
\task{%
    Напротив физических величин укажите их обозначения и единицы измерения в СИ:
    \begin{enumerate}
        \item ёмкость конденсатора,
        \item индуктивность катушки.
    \end{enumerate}
}
\solutionspace{40pt}

\tasknumber{6}%
\task{%
    Запишите формулы, выражающие:
    \begin{enumerate}
        \item заряд конденсатора через его ёмкость и поданное напряжение,
        \item энергию конденсатора через его ёмкость и заряд,
        \item период колебаний в электромагнитном контуре, состоящем из конденсатора и катушки индуктивности,
    \end{enumerate}
}

\variantsplitter

\addpersonalvariant{Матвей Неретин}

\tasknumber{1}%
\task{%
    Определите ёмкость конденсатора, если при его зарядке до напряжения
    $U = 20\,\text{кВ}$ он приобретает заряд $q = 6\,\text{мКл}$.
    % Чему при этом равны заряды обкладок конденсатора (сделайте рисунок и укажите их)?
    Ответ выразите в нанофарадах.
}
\answer{%
    $
        q = CU \implies
        C = \frac{q}{U} = \frac{6\,\text{мКл}}{20\,\text{кВ}} = 300\,\text{нФ}.
        \text{ Заряды обкладок: $q$ и $-q$}
    $
}
\solutionspace{120pt}

\tasknumber{2}%
\task{%
    На конденсаторе указано: $C = 100\,\text{пФ}$, $V = 300\,\text{В}$.
    Удастся ли его использовать для накопления заряда $q = 50\,\text{нКл}$?
    (в ответе укажите «да» или «нет»)
}
\answer{%
    $
        q_{\text{max}} = CV = 100\,\text{пФ} \cdot 300\,\text{В} = 30\,\text{нКл}
        \implies q_{\text{max}}  <  q \implies \text{не удастся}
    $
}
\solutionspace{80pt}

\tasknumber{3}%
\task{%
    Как и во сколько раз изменится ёмкость плоского конденсатора
    при уменьшении площади пластин в 3 раз
    и уменьшении расстояния между ними в 3 раз?
    В ответе укажите простую дробь или число — отношение новой ёмкости к старой.
}
\answer{%
    $
        \frac{C'}C
            = \frac{\eps_0\eps \frac S3}{\frac d3} \Big/ \frac{\eps_0\eps S}d
            = \frac{3}{3} = = 1 \implies \text{не изменится в $1$ раз}
    $
}
\solutionspace{80pt}

\tasknumber{4}%
\task{%
    Электрическая ёмкость конденсатора равна $C = 750\,\text{пФ}$,
    при этом ему сообщён заряд $q = 900\,\text{нКл}$.
    Какова энергия заряженного конденсатора?
    Ответ выразите в микроджоулях и округлите до целого.
}
\answer{%
    $
        W
        = \frac{q^2}{2C}
        = \frac{\sqr{900\,\text{нКл}}}{2 \cdot 750\,\text{пФ}}
        = 540\,\text{мкДж}
    $
}
\solutionspace{80pt}

\tasknumber{5}%
\task{%
    Напротив физических величин укажите их обозначения и единицы измерения в СИ:
    \begin{enumerate}
        \item ёмкость конденсатора,
        \item индуктивность катушки.
    \end{enumerate}
}
\solutionspace{40pt}

\tasknumber{6}%
\task{%
    Запишите формулы, выражающие:
    \begin{enumerate}
        \item заряд конденсатора через его ёмкость и поданное напряжение,
        \item энергию конденсатора через его ёмкость и поданное напряжение,
        \item частоту колебаний в электромагнитном контуре, состоящем из конденсатора и катушки индуктивности,
    \end{enumerate}
}

\variantsplitter

\addpersonalvariant{Мария Никонова}

\tasknumber{1}%
\task{%
    Определите ёмкость конденсатора, если при его зарядке до напряжения
    $U = 2\,\text{кВ}$ он приобретает заряд $q = 15\,\text{мКл}$.
    % Чему при этом равны заряды обкладок конденсатора (сделайте рисунок и укажите их)?
    Ответ выразите в нанофарадах.
}
\answer{%
    $
        q = CU \implies
        C = \frac{q}{U} = \frac{15\,\text{мКл}}{2\,\text{кВ}} = 7500\,\text{нФ}.
        \text{ Заряды обкладок: $q$ и $-q$}
    $
}
\solutionspace{120pt}

\tasknumber{2}%
\task{%
    На конденсаторе указано: $C = 100\,\text{пФ}$, $V = 400\,\text{В}$.
    Удастся ли его использовать для накопления заряда $Q = 60\,\text{нКл}$?
    (в ответе укажите «да» или «нет»)
}
\answer{%
    $
        Q_{\text{max}} = CV = 100\,\text{пФ} \cdot 400\,\text{В} = 40\,\text{нКл}
        \implies Q_{\text{max}}  <  Q \implies \text{не удастся}
    $
}
\solutionspace{80pt}

\tasknumber{3}%
\task{%
    Как и во сколько раз изменится ёмкость плоского конденсатора
    при уменьшении площади пластин в 7 раз
    и уменьшении расстояния между ними в 2 раз?
    В ответе укажите простую дробь или число — отношение новой ёмкости к старой.
}
\answer{%
    $
        \frac{C'}C
            = \frac{\eps_0\eps \frac S7}{\frac d2} \Big/ \frac{\eps_0\eps S}d
            = \frac{2}{7} = < 1 \implies \text{уменьшится в $\frac72$ раз}
    $
}
\solutionspace{80pt}

\tasknumber{4}%
\task{%
    Электрическая ёмкость конденсатора равна $C = 750\,\text{пФ}$,
    при этом ему сообщён заряд $Q = 900\,\text{нКл}$.
    Какова энергия заряженного конденсатора?
    Ответ выразите в микроджоулях и округлите до целого.
}
\answer{%
    $
        W
        = \frac{Q^2}{2C}
        = \frac{\sqr{900\,\text{нКл}}}{2 \cdot 750\,\text{пФ}}
        = 540\,\text{мкДж}
    $
}
\solutionspace{80pt}

\tasknumber{5}%
\task{%
    Напротив физических величин укажите их обозначения и единицы измерения в СИ:
    \begin{enumerate}
        \item ёмкость конденсатора,
        \item индуктивность катушки.
    \end{enumerate}
}
\solutionspace{40pt}

\tasknumber{6}%
\task{%
    Запишите формулы, выражающие:
    \begin{enumerate}
        \item заряд конденсатора через его ёмкость и поданное напряжение,
        \item энергию конденсатора через его ёмкость и поданное напряжение,
        \item период колебаний в электромагнитном контуре, состоящем из конденсатора и катушки индуктивности,
    \end{enumerate}
}

\variantsplitter

\addpersonalvariant{Даниил Палаткин}

\tasknumber{1}%
\task{%
    Определите ёмкость конденсатора, если при его зарядке до напряжения
    $V = 4\,\text{кВ}$ он приобретает заряд $q = 15\,\text{мКл}$.
    % Чему при этом равны заряды обкладок конденсатора (сделайте рисунок и укажите их)?
    Ответ выразите в нанофарадах.
}
\answer{%
    $
        q = CV \implies
        C = \frac{q}{V} = \frac{15\,\text{мКл}}{4\,\text{кВ}} = 3750\,\text{нФ}.
        \text{ Заряды обкладок: $q$ и $-q$}
    $
}
\solutionspace{120pt}

\tasknumber{2}%
\task{%
    На конденсаторе указано: $C = 150\,\text{пФ}$, $U = 400\,\text{В}$.
    Удастся ли его использовать для накопления заряда $q = 60\,\text{нКл}$?
    (в ответе укажите «да» или «нет»)
}
\answer{%
    $
        q_{\text{max}} = CU = 150\,\text{пФ} \cdot 400\,\text{В} = 60\,\text{нКл}
        \implies q_{\text{max}} \ge q \implies \text{удастся}
    $
}
\solutionspace{80pt}

\tasknumber{3}%
\task{%
    Как и во сколько раз изменится ёмкость плоского конденсатора
    при уменьшении площади пластин в 8 раз
    и уменьшении расстояния между ними в 5 раз?
    В ответе укажите простую дробь или число — отношение новой ёмкости к старой.
}
\answer{%
    $
        \frac{C'}C
            = \frac{\eps_0\eps \frac S8}{\frac d5} \Big/ \frac{\eps_0\eps S}d
            = \frac{5}{8} = < 1 \implies \text{уменьшится в $\frac85$ раз}
    $
}
\solutionspace{80pt}

\tasknumber{4}%
\task{%
    Электрическая ёмкость конденсатора равна $C = 600\,\text{пФ}$,
    при этом ему сообщён заряд $Q = 900\,\text{нКл}$.
    Какова энергия заряженного конденсатора?
    Ответ выразите в микроджоулях и округлите до целого.
}
\answer{%
    $
        W
        = \frac{Q^2}{2C}
        = \frac{\sqr{900\,\text{нКл}}}{2 \cdot 600\,\text{пФ}}
        = 675\,\text{мкДж}
    $
}
\solutionspace{80pt}

\tasknumber{5}%
\task{%
    Напротив физических величин укажите их обозначения и единицы измерения в СИ:
    \begin{enumerate}
        \item ёмкость конденсатора,
        \item индуктивность катушки.
    \end{enumerate}
}
\solutionspace{40pt}

\tasknumber{6}%
\task{%
    Запишите формулы, выражающие:
    \begin{enumerate}
        \item заряд конденсатора через его ёмкость и поданное напряжение,
        \item энергию конденсатора через его заряд и поданное напряжение,
        \item период колебаний в электромагнитном контуре, состоящем из конденсатора и катушки индуктивности,
    \end{enumerate}
}

\variantsplitter

\addpersonalvariant{Станислав Пикун}

\tasknumber{1}%
\task{%
    Определите ёмкость конденсатора, если при его зарядке до напряжения
    $U = 50\,\text{кВ}$ он приобретает заряд $Q = 25\,\text{мКл}$.
    % Чему при этом равны заряды обкладок конденсатора (сделайте рисунок и укажите их)?
    Ответ выразите в нанофарадах.
}
\answer{%
    $
        Q = CU \implies
        C = \frac{Q}{U} = \frac{25\,\text{мКл}}{50\,\text{кВ}} = 500\,\text{нФ}.
        \text{ Заряды обкладок: $Q$ и $-Q$}
    $
}
\solutionspace{120pt}

\tasknumber{2}%
\task{%
    На конденсаторе указано: $C = 80\,\text{пФ}$, $U = 300\,\text{В}$.
    Удастся ли его использовать для накопления заряда $Q = 30\,\text{нКл}$?
    (в ответе укажите «да» или «нет»)
}
\answer{%
    $
        Q_{\text{max}} = CU = 80\,\text{пФ} \cdot 300\,\text{В} = 24\,\text{нКл}
        \implies Q_{\text{max}}  <  Q \implies \text{не удастся}
    $
}
\solutionspace{80pt}

\tasknumber{3}%
\task{%
    Как и во сколько раз изменится ёмкость плоского конденсатора
    при уменьшении площади пластин в 8 раз
    и уменьшении расстояния между ними в 4 раз?
    В ответе укажите простую дробь или число — отношение новой ёмкости к старой.
}
\answer{%
    $
        \frac{C'}C
            = \frac{\eps_0\eps \frac S8}{\frac d4} \Big/ \frac{\eps_0\eps S}d
            = \frac{4}{8} = < 1 \implies \text{уменьшится в $2$ раз}
    $
}
\solutionspace{80pt}

\tasknumber{4}%
\task{%
    Электрическая ёмкость конденсатора равна $C = 750\,\text{пФ}$,
    при этом ему сообщён заряд $Q = 900\,\text{нКл}$.
    Какова энергия заряженного конденсатора?
    Ответ выразите в микроджоулях и округлите до целого.
}
\answer{%
    $
        W
        = \frac{Q^2}{2C}
        = \frac{\sqr{900\,\text{нКл}}}{2 \cdot 750\,\text{пФ}}
        = 540\,\text{мкДж}
    $
}
\solutionspace{80pt}

\tasknumber{5}%
\task{%
    Напротив физических величин укажите их обозначения и единицы измерения в СИ:
    \begin{enumerate}
        \item ёмкость конденсатора,
        \item индуктивность катушки.
    \end{enumerate}
}
\solutionspace{40pt}

\tasknumber{6}%
\task{%
    Запишите формулы, выражающие:
    \begin{enumerate}
        \item заряд конденсатора через его ёмкость и поданное напряжение,
        \item энергию конденсатора через его ёмкость и заряд,
        \item период колебаний в электромагнитном контуре, состоящем из конденсатора и катушки индуктивности,
    \end{enumerate}
}

\variantsplitter

\addpersonalvariant{Илья Пичугин}

\tasknumber{1}%
\task{%
    Определите ёмкость конденсатора, если при его зарядке до напряжения
    $U = 40\,\text{кВ}$ он приобретает заряд $Q = 25\,\text{мКл}$.
    % Чему при этом равны заряды обкладок конденсатора (сделайте рисунок и укажите их)?
    Ответ выразите в нанофарадах.
}
\answer{%
    $
        Q = CU \implies
        C = \frac{Q}{U} = \frac{25\,\text{мКл}}{40\,\text{кВ}} = 625\,\text{нФ}.
        \text{ Заряды обкладок: $Q$ и $-Q$}
    $
}
\solutionspace{120pt}

\tasknumber{2}%
\task{%
    На конденсаторе указано: $C = 50\,\text{пФ}$, $V = 300\,\text{В}$.
    Удастся ли его использовать для накопления заряда $Q = 50\,\text{нКл}$?
    (в ответе укажите «да» или «нет»)
}
\answer{%
    $
        Q_{\text{max}} = CV = 50\,\text{пФ} \cdot 300\,\text{В} = 15\,\text{нКл}
        \implies Q_{\text{max}}  <  Q \implies \text{не удастся}
    $
}
\solutionspace{80pt}

\tasknumber{3}%
\task{%
    Как и во сколько раз изменится ёмкость плоского конденсатора
    при уменьшении площади пластин в 5 раз
    и уменьшении расстояния между ними в 3 раз?
    В ответе укажите простую дробь или число — отношение новой ёмкости к старой.
}
\answer{%
    $
        \frac{C'}C
            = \frac{\eps_0\eps \frac S5}{\frac d3} \Big/ \frac{\eps_0\eps S}d
            = \frac{3}{5} = < 1 \implies \text{уменьшится в $\frac53$ раз}
    $
}
\solutionspace{80pt}

\tasknumber{4}%
\task{%
    Электрическая ёмкость конденсатора равна $C = 600\,\text{пФ}$,
    при этом ему сообщён заряд $q = 300\,\text{нКл}$.
    Какова энергия заряженного конденсатора?
    Ответ выразите в микроджоулях и округлите до целого.
}
\answer{%
    $
        W
        = \frac{q^2}{2C}
        = \frac{\sqr{300\,\text{нКл}}}{2 \cdot 600\,\text{пФ}}
        = 75\,\text{мкДж}
    $
}
\solutionspace{80pt}

\tasknumber{5}%
\task{%
    Напротив физических величин укажите их обозначения и единицы измерения в СИ:
    \begin{enumerate}
        \item ёмкость конденсатора,
        \item индуктивность катушки.
    \end{enumerate}
}
\solutionspace{40pt}

\tasknumber{6}%
\task{%
    Запишите формулы, выражающие:
    \begin{enumerate}
        \item заряд конденсатора через его ёмкость и поданное напряжение,
        \item энергию конденсатора через его ёмкость и заряд,
        \item период колебаний в электромагнитном контуре, состоящем из конденсатора и катушки индуктивности,
    \end{enumerate}
}

\variantsplitter

\addpersonalvariant{Кирилл Севрюгин}

\tasknumber{1}%
\task{%
    Определите ёмкость конденсатора, если при его зарядке до напряжения
    $V = 40\,\text{кВ}$ он приобретает заряд $Q = 15\,\text{мКл}$.
    % Чему при этом равны заряды обкладок конденсатора (сделайте рисунок и укажите их)?
    Ответ выразите в нанофарадах.
}
\answer{%
    $
        Q = CV \implies
        C = \frac{Q}{V} = \frac{15\,\text{мКл}}{40\,\text{кВ}} = 375\,\text{нФ}.
        \text{ Заряды обкладок: $Q$ и $-Q$}
    $
}
\solutionspace{120pt}

\tasknumber{2}%
\task{%
    На конденсаторе указано: $C = 150\,\text{пФ}$, $V = 450\,\text{В}$.
    Удастся ли его использовать для накопления заряда $Q = 50\,\text{нКл}$?
    (в ответе укажите «да» или «нет»)
}
\answer{%
    $
        Q_{\text{max}} = CV = 150\,\text{пФ} \cdot 450\,\text{В} = 67\,\text{нКл}
        \implies Q_{\text{max}} \ge Q \implies \text{удастся}
    $
}
\solutionspace{80pt}

\tasknumber{3}%
\task{%
    Как и во сколько раз изменится ёмкость плоского конденсатора
    при уменьшении площади пластин в 4 раз
    и уменьшении расстояния между ними в 3 раз?
    В ответе укажите простую дробь или число — отношение новой ёмкости к старой.
}
\answer{%
    $
        \frac{C'}C
            = \frac{\eps_0\eps \frac S4}{\frac d3} \Big/ \frac{\eps_0\eps S}d
            = \frac{3}{4} = < 1 \implies \text{уменьшится в $\frac43$ раз}
    $
}
\solutionspace{80pt}

\tasknumber{4}%
\task{%
    Электрическая ёмкость конденсатора равна $C = 750\,\text{пФ}$,
    при этом ему сообщён заряд $Q = 900\,\text{нКл}$.
    Какова энергия заряженного конденсатора?
    Ответ выразите в микроджоулях и округлите до целого.
}
\answer{%
    $
        W
        = \frac{Q^2}{2C}
        = \frac{\sqr{900\,\text{нКл}}}{2 \cdot 750\,\text{пФ}}
        = 540\,\text{мкДж}
    $
}
\solutionspace{80pt}

\tasknumber{5}%
\task{%
    Напротив физических величин укажите их обозначения и единицы измерения в СИ:
    \begin{enumerate}
        \item ёмкость конденсатора,
        \item индуктивность катушки.
    \end{enumerate}
}
\solutionspace{40pt}

\tasknumber{6}%
\task{%
    Запишите формулы, выражающие:
    \begin{enumerate}
        \item заряд конденсатора через его ёмкость и поданное напряжение,
        \item энергию конденсатора через его заряд и поданное напряжение,
        \item частоту колебаний в электромагнитном контуре, состоящем из конденсатора и катушки индуктивности,
    \end{enumerate}
}

\variantsplitter

\addpersonalvariant{Илья Стратонников}

\tasknumber{1}%
\task{%
    Определите ёмкость конденсатора, если при его зарядке до напряжения
    $U = 20\,\text{кВ}$ он приобретает заряд $Q = 24\,\text{мКл}$.
    % Чему при этом равны заряды обкладок конденсатора (сделайте рисунок и укажите их)?
    Ответ выразите в нанофарадах.
}
\answer{%
    $
        Q = CU \implies
        C = \frac{Q}{U} = \frac{24\,\text{мКл}}{20\,\text{кВ}} = 1200\,\text{нФ}.
        \text{ Заряды обкладок: $Q$ и $-Q$}
    $
}
\solutionspace{120pt}

\tasknumber{2}%
\task{%
    На конденсаторе указано: $C = 100\,\text{пФ}$, $V = 300\,\text{В}$.
    Удастся ли его использовать для накопления заряда $q = 30\,\text{нКл}$?
    (в ответе укажите «да» или «нет»)
}
\answer{%
    $
        q_{\text{max}} = CV = 100\,\text{пФ} \cdot 300\,\text{В} = 30\,\text{нКл}
        \implies q_{\text{max}} \ge q \implies \text{удастся}
    $
}
\solutionspace{80pt}

\tasknumber{3}%
\task{%
    Как и во сколько раз изменится ёмкость плоского конденсатора
    при уменьшении площади пластин в 2 раз
    и уменьшении расстояния между ними в 5 раз?
    В ответе укажите простую дробь или число — отношение новой ёмкости к старой.
}
\answer{%
    $
        \frac{C'}C
            = \frac{\eps_0\eps \frac S2}{\frac d5} \Big/ \frac{\eps_0\eps S}d
            = \frac{5}{2} = > 1 \implies \text{увеличится в $\frac52$ раз}
    $
}
\solutionspace{80pt}

\tasknumber{4}%
\task{%
    Электрическая ёмкость конденсатора равна $C = 400\,\text{пФ}$,
    при этом ему сообщён заряд $q = 900\,\text{нКл}$.
    Какова энергия заряженного конденсатора?
    Ответ выразите в микроджоулях и округлите до целого.
}
\answer{%
    $
        W
        = \frac{q^2}{2C}
        = \frac{\sqr{900\,\text{нКл}}}{2 \cdot 400\,\text{пФ}}
        = 1012{,}50\,\text{мкДж}
    $
}
\solutionspace{80pt}

\tasknumber{5}%
\task{%
    Напротив физических величин укажите их обозначения и единицы измерения в СИ:
    \begin{enumerate}
        \item ёмкость конденсатора,
        \item индуктивность катушки.
    \end{enumerate}
}
\solutionspace{40pt}

\tasknumber{6}%
\task{%
    Запишите формулы, выражающие:
    \begin{enumerate}
        \item заряд конденсатора через его ёмкость и поданное напряжение,
        \item энергию конденсатора через его ёмкость и поданное напряжение,
        \item частоту колебаний в электромагнитном контуре, состоящем из конденсатора и катушки индуктивности,
    \end{enumerate}
}

\variantsplitter

\addpersonalvariant{Иван Шустов}

\tasknumber{1}%
\task{%
    Определите ёмкость конденсатора, если при его зарядке до напряжения
    $U = 20\,\text{кВ}$ он приобретает заряд $q = 15\,\text{мКл}$.
    % Чему при этом равны заряды обкладок конденсатора (сделайте рисунок и укажите их)?
    Ответ выразите в нанофарадах.
}
\answer{%
    $
        q = CU \implies
        C = \frac{q}{U} = \frac{15\,\text{мКл}}{20\,\text{кВ}} = 750\,\text{нФ}.
        \text{ Заряды обкладок: $q$ и $-q$}
    $
}
\solutionspace{120pt}

\tasknumber{2}%
\task{%
    На конденсаторе указано: $C = 120\,\text{пФ}$, $U = 300\,\text{В}$.
    Удастся ли его использовать для накопления заряда $q = 30\,\text{нКл}$?
    (в ответе укажите «да» или «нет»)
}
\answer{%
    $
        q_{\text{max}} = CU = 120\,\text{пФ} \cdot 300\,\text{В} = 36\,\text{нКл}
        \implies q_{\text{max}} \ge q \implies \text{удастся}
    $
}
\solutionspace{80pt}

\tasknumber{3}%
\task{%
    Как и во сколько раз изменится ёмкость плоского конденсатора
    при уменьшении площади пластин в 7 раз
    и уменьшении расстояния между ними в 2 раз?
    В ответе укажите простую дробь или число — отношение новой ёмкости к старой.
}
\answer{%
    $
        \frac{C'}C
            = \frac{\eps_0\eps \frac S7}{\frac d2} \Big/ \frac{\eps_0\eps S}d
            = \frac{2}{7} = < 1 \implies \text{уменьшится в $\frac72$ раз}
    $
}
\solutionspace{80pt}

\tasknumber{4}%
\task{%
    Электрическая ёмкость конденсатора равна $C = 400\,\text{пФ}$,
    при этом ему сообщён заряд $q = 800\,\text{нКл}$.
    Какова энергия заряженного конденсатора?
    Ответ выразите в микроджоулях и округлите до целого.
}
\answer{%
    $
        W
        = \frac{q^2}{2C}
        = \frac{\sqr{800\,\text{нКл}}}{2 \cdot 400\,\text{пФ}}
        = 800\,\text{мкДж}
    $
}
\solutionspace{80pt}

\tasknumber{5}%
\task{%
    Напротив физических величин укажите их обозначения и единицы измерения в СИ:
    \begin{enumerate}
        \item ёмкость конденсатора,
        \item индуктивность катушки.
    \end{enumerate}
}
\solutionspace{40pt}

\tasknumber{6}%
\task{%
    Запишите формулы, выражающие:
    \begin{enumerate}
        \item заряд конденсатора через его ёмкость и поданное напряжение,
        \item энергию конденсатора через его ёмкость и поданное напряжение,
        \item период колебаний в электромагнитном контуре, состоящем из конденсатора и катушки индуктивности,
    \end{enumerate}
}
% autogenerated
