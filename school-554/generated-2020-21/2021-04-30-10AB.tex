\setdate{30~апреля~2021}
\setclass{10«АБ»}

\addpersonalvariant{Михаил Бурмистров}

\tasknumber{1}%
\task{%
    Напротив физических величин укажите их обозначения и единицы измерения в СИ:
    \begin{enumerate}
        \item сила тока,
        \item работа тока,
        \item ЭДС,
        \item внутреннее сопротивление полной цепи.
    \end{enumerate}
}
\solutionspace{20pt}

\tasknumber{2}%
\task{%
    Запишите физический закон или формулу:
    \begin{enumerate}
        \item правило Кирхгофа для узла цепи,
        \item закон Ома для однородного участка цепи,
        \item ЭДС (определение).
    \end{enumerate}
}
\solutionspace{40pt}

\tasknumber{3}%
\task{%
    На резистор сопротивлением $R = 5\,\text{Ом}$ подали напряжение $V = 120\,\text{В}$.
    Определите ток, который потечёт через резистор, и мощность, выделяющуюся на нём.
}
\answer{%
    \begin{align*}
    \mathcal{I} &= \frac{ V }{ R } = \frac{ 120\,\text{В} }{ 5\,\text{Ом} } = 24{,}00\,\text{А},  \\
    P &= \frac{V^2}{ R } = \frac{ \sqr{ 120\,\text{В} } }{ 5\,\text{Ом} } = 2880{,}00\,\text{Вт}
    \end{align*}
}
\solutionspace{60pt}

\tasknumber{4}%
\task{%
    Через резистор сопротивлением $R = 12\,\text{Ом}$ протекает электрический ток $\mathcal{I} = 3{,}00\,\text{А}$.
    Определите, чему равны напряжение на резисторе и мощность, выделяющаяся на нём.
}
\answer{%
    \begin{align*}
    U &= \mathcal{I}R = 3{,}00\,\text{А} \cdot 12\,\text{Ом} = 36\,\text{В},  \\
    P &= \mathcal{I}^2R = \sqr{ 3{,}00\,\text{А} } \cdot 12\,\text{Ом} = 108\,\text{Вт}
    \end{align*}
}
\solutionspace{60pt}

\tasknumber{5}%
\task{%
    Замкнутая электрическая цепь состоит из ЭДС $\mathcal{E} = 2\,\text{В}$ и сопротивлением $r$
    и резистора $R = 24\,\text{Ом}$.
    Определите ток, протекающий в цепи.
    Какая тепловая энергия выделится на резисторе за время
    $\tau = 5\,\text{с}$? Какая работа будет совершена ЭДС за это время? Каков знак этой работы? Чему равен КПД цепи?
    Вычислите значения для 2 случаев: $r=0$ и $r = 10\,\text{Ом}$.
}
\answer{%
    \begin{align*}
    \mathcal{I}_1 &= \frac{ \mathcal{E} }{ R } = \frac{ 2\,\text{В} }{ 24\,\text{Ом} } = 0{,}08\,\text{А},  \\
    \mathcal{I}_2 &= \frac{ \mathcal{E} }{R + r} = \frac{ 2\,\text{В} }{24\,\text{Ом} + 10\,\text{Ом}} = 0{,}06\,\text{А},  \\
    Q_1 &= \mathcal{I}_1^2R\tau = \sqr{\frac{ \mathcal{E} }{ R }} R \tau
            = \sqr{\frac{ 2\,\text{В} }{ 24\,\text{Ом} }} \cdot 24\,\text{Ом} \cdot 5\,\text{с} = 0{,}768\,\text{Дж},  \\
    Q_2 &= \mathcal{I}_2^2R\tau = \sqr{\frac{ \mathcal{E} }{R + r}} R \tau
            = \sqr{\frac{ 2\,\text{В} }{24\,\text{Ом} + 10\,\text{Ом}}} \cdot 24\,\text{Ом} \cdot 5\,\text{с} = 0{,}432\,\text{Дж},  \\
    A_1 &= \mathcal{I}_1\tau\mathcal{E} = \frac{ \mathcal{E} }{R} \tau \mathcal{E}
            = \frac{\mathcal{E}^2 \tau}{ R } = \frac{\sqr{ 2\,\text{В} } \cdot 5\,\text{с}}{ 24\,\text{Ом} }
            = 0{,}800\,\text{Дж}, \text{положительна},  \\
    A_2 &= \mathcal{I}_2\tau\mathcal{E} = \frac{ \mathcal{E} }{R + r} \tau \mathcal{E}
            = \frac{\mathcal{E}^2 \tau}{R + r} = \frac{\sqr{ 2\,\text{В} } \cdot 5\,\text{с}}{24\,\text{Ом} + 10\,\text{Ом}}
            = 0{,}600\,\text{Дж}, \text{положительна},  \\
    \eta_1 &= \frac{ Q_1 }{ A_1 } = \ldots = \frac{ R }{ R } = 1,  \\
    \eta_2 &= \frac{ Q_2 }{ A_2 } = \ldots = \frac{ R }{R + r} = 0{,}72
    \end{align*}
}
\solutionspace{180pt}

\tasknumber{6}%
\task{%
    Лампочки, сопротивления которых $R_1 = 3{,}00\,\text{Ом}$ и $R_2 = 12{,}00\,\text{Ом}$, поочерёдно подключённные к некоторому источнику тока,
    потребляют одинаковую мощность.
    Найти внутреннее сопротивление источника и КПД цепи в каждом случае.
}
\answer{%
    \begin{align*}
        P_1 &= \sqr{\frac{ \mathcal{E} }{R_1 + r}}R_1,
        P_2  = \sqr{\frac{ \mathcal{E} }{R_2 + r}}R_2,
        P_1 = P_2 \implies  \\
        &\implies R_1 \sqr{R_2 + r} = R_2 \sqr{R_1 + r} \implies  \\
        &\implies R_1 R_2^2 + 2 R_1 R_2 r + R_1 r^2 =
                    R_2 R_1^2 + 2 R_2 R_1 r + R_2 r^2  \implies  \\
    &\implies r^2 (R_2 - R_1) = R_2^2 R_2 - R_1^2 R_2 \implies  \\
    &\implies r
            = \sqrt{R_1 R_2 \frac{R_2 - R_1}{R_2 - R_1}}
            = \sqrt{R_1 R_2}
            = \sqrt{3{,}00\,\text{Ом} \cdot 12{,}00\,\text{Ом}}
            = 6{,}0\,\text{Ом}.
            \\
    \eta_1
            &= \frac{ R_1 }{R_1 + r}
            = \frac{\sqrt{ R_1 }}{\sqrt{ R_1 } + \sqrt{ R_2 }}
            = 0{,}333,  \\
    \eta_2
            &= \frac{ R_2 }{R_2 + r}
            = \frac{ \sqrt{ R_2 } }{\sqrt{ R_2 } + \sqrt{ R_1 }}
            = 0{,}667
    \end{align*}
}

\variantsplitter

\addpersonalvariant{Ирина Ан}

\tasknumber{1}%
\task{%
    Напротив физических величин укажите их обозначения и единицы измерения в СИ:
    \begin{enumerate}
        \item разность потенциалов,
        \item работа тока,
        \item ЭДС,
        \item внутреннее сопротивление полной цепи.
    \end{enumerate}
}
\solutionspace{20pt}

\tasknumber{2}%
\task{%
    Запишите физический закон или формулу:
    \begin{enumerate}
        \item правило Кирхгофа для замкнутого контура,
        \item сопротивление резистора через удельное сопротивление,
        \item ЭДС (определение).
    \end{enumerate}
}
\solutionspace{40pt}

\tasknumber{3}%
\task{%
    На резистор сопротивлением $r = 18\,\text{Ом}$ подали напряжение $V = 120\,\text{В}$.
    Определите ток, который потечёт через резистор, и мощность, выделяющуюся на нём.
}
\answer{%
    \begin{align*}
    \mathcal{I} &= \frac{ V }{ r } = \frac{ 120\,\text{В} }{ 18\,\text{Ом} } = 6{,}67\,\text{А},  \\
    P &= \frac{V^2}{ r } = \frac{ \sqr{ 120\,\text{В} } }{ 18\,\text{Ом} } = 800{,}00\,\text{Вт}
    \end{align*}
}
\solutionspace{60pt}

\tasknumber{4}%
\task{%
    Через резистор сопротивлением $R = 5\,\text{Ом}$ протекает электрический ток $\mathcal{I} = 3{,}00\,\text{А}$.
    Определите, чему равны напряжение на резисторе и мощность, выделяющаяся на нём.
}
\answer{%
    \begin{align*}
    U &= \mathcal{I}R = 3{,}00\,\text{А} \cdot 5\,\text{Ом} = 15\,\text{В},  \\
    P &= \mathcal{I}^2R = \sqr{ 3{,}00\,\text{А} } \cdot 5\,\text{Ом} = 45\,\text{Вт}
    \end{align*}
}
\solutionspace{60pt}

\tasknumber{5}%
\task{%
    Замкнутая электрическая цепь состоит из ЭДС $\mathcal{E} = 2\,\text{В}$ и сопротивлением $r$
    и резистора $R = 30\,\text{Ом}$.
    Определите ток, протекающий в цепи.
    Какая тепловая энергия выделится на резисторе за время
    $\tau = 5\,\text{с}$? Какая работа будет совершена ЭДС за это время? Каков знак этой работы? Чему равен КПД цепи?
    Вычислите значения для 2 случаев: $r=0$ и $r = 30\,\text{Ом}$.
}
\answer{%
    \begin{align*}
    \mathcal{I}_1 &= \frac{ \mathcal{E} }{ R } = \frac{ 2\,\text{В} }{ 30\,\text{Ом} } = 0{,}07\,\text{А},  \\
    \mathcal{I}_2 &= \frac{ \mathcal{E} }{R + r} = \frac{ 2\,\text{В} }{30\,\text{Ом} + 30\,\text{Ом}} = 0{,}03\,\text{А},  \\
    Q_1 &= \mathcal{I}_1^2R\tau = \sqr{\frac{ \mathcal{E} }{ R }} R \tau
            = \sqr{\frac{ 2\,\text{В} }{ 30\,\text{Ом} }} \cdot 30\,\text{Ом} \cdot 5\,\text{с} = 0{,}735\,\text{Дж},  \\
    Q_2 &= \mathcal{I}_2^2R\tau = \sqr{\frac{ \mathcal{E} }{R + r}} R \tau
            = \sqr{\frac{ 2\,\text{В} }{30\,\text{Ом} + 30\,\text{Ом}}} \cdot 30\,\text{Ом} \cdot 5\,\text{с} = 0{,}135\,\text{Дж},  \\
    A_1 &= \mathcal{I}_1\tau\mathcal{E} = \frac{ \mathcal{E} }{R} \tau \mathcal{E}
            = \frac{\mathcal{E}^2 \tau}{ R } = \frac{\sqr{ 2\,\text{В} } \cdot 5\,\text{с}}{ 30\,\text{Ом} }
            = 0{,}700\,\text{Дж}, \text{положительна},  \\
    A_2 &= \mathcal{I}_2\tau\mathcal{E} = \frac{ \mathcal{E} }{R + r} \tau \mathcal{E}
            = \frac{\mathcal{E}^2 \tau}{R + r} = \frac{\sqr{ 2\,\text{В} } \cdot 5\,\text{с}}{30\,\text{Ом} + 30\,\text{Ом}}
            = 0{,}300\,\text{Дж}, \text{положительна},  \\
    \eta_1 &= \frac{ Q_1 }{ A_1 } = \ldots = \frac{ R }{ R } = 1,  \\
    \eta_2 &= \frac{ Q_2 }{ A_2 } = \ldots = \frac{ R }{R + r} = 0{,}45
    \end{align*}
}
\solutionspace{180pt}

\tasknumber{6}%
\task{%
    Лампочки, сопротивления которых $R_1 = 0{,}25\,\text{Ом}$ и $R_2 = 4{,}00\,\text{Ом}$, поочерёдно подключённные к некоторому источнику тока,
    потребляют одинаковую мощность.
    Найти внутреннее сопротивление источника и КПД цепи в каждом случае.
}
\answer{%
    \begin{align*}
        P_1 &= \sqr{\frac{ \mathcal{E} }{R_1 + r}}R_1,
        P_2  = \sqr{\frac{ \mathcal{E} }{R_2 + r}}R_2,
        P_1 = P_2 \implies  \\
        &\implies R_1 \sqr{R_2 + r} = R_2 \sqr{R_1 + r} \implies  \\
        &\implies R_1 R_2^2 + 2 R_1 R_2 r + R_1 r^2 =
                    R_2 R_1^2 + 2 R_2 R_1 r + R_2 r^2  \implies  \\
    &\implies r^2 (R_2 - R_1) = R_2^2 R_2 - R_1^2 R_2 \implies  \\
    &\implies r
            = \sqrt{R_1 R_2 \frac{R_2 - R_1}{R_2 - R_1}}
            = \sqrt{R_1 R_2}
            = \sqrt{0{,}25\,\text{Ом} \cdot 4{,}00\,\text{Ом}}
            = 1{,}0\,\text{Ом}.
            \\
    \eta_1
            &= \frac{ R_1 }{R_1 + r}
            = \frac{\sqrt{ R_1 }}{\sqrt{ R_1 } + \sqrt{ R_2 }}
            = 0{,}200,  \\
    \eta_2
            &= \frac{ R_2 }{R_2 + r}
            = \frac{ \sqrt{ R_2 } }{\sqrt{ R_2 } + \sqrt{ R_1 }}
            = 0{,}800
    \end{align*}
}

\variantsplitter

\addpersonalvariant{Софья Андрианова}

\tasknumber{1}%
\task{%
    Напротив физических величин укажите их обозначения и единицы измерения в СИ:
    \begin{enumerate}
        \item сила тока,
        \item мощность тока,
        \item ЭДС,
        \item внешнее сопротивление полной цепи.
    \end{enumerate}
}
\solutionspace{20pt}

\tasknumber{2}%
\task{%
    Запишите физический закон или формулу:
    \begin{enumerate}
        \item правило Кирхгофа для узла цепи,
        \item сопротивление резистора через удельное сопротивление,
        \item ЭДС (определение).
    \end{enumerate}
}
\solutionspace{40pt}

\tasknumber{3}%
\task{%
    На резистор сопротивлением $r = 12\,\text{Ом}$ подали напряжение $V = 150\,\text{В}$.
    Определите ток, который потечёт через резистор, и мощность, выделяющуюся на нём.
}
\answer{%
    \begin{align*}
    \mathcal{I} &= \frac{ V }{ r } = \frac{ 150\,\text{В} }{ 12\,\text{Ом} } = 12{,}50\,\text{А},  \\
    P &= \frac{V^2}{ r } = \frac{ \sqr{ 150\,\text{В} } }{ 12\,\text{Ом} } = 1875{,}00\,\text{Вт}
    \end{align*}
}
\solutionspace{60pt}

\tasknumber{4}%
\task{%
    Через резистор сопротивлением $R = 30\,\text{Ом}$ протекает электрический ток $\mathcal{I} = 4{,}00\,\text{А}$.
    Определите, чему равны напряжение на резисторе и мощность, выделяющаяся на нём.
}
\answer{%
    \begin{align*}
    U &= \mathcal{I}R = 4{,}00\,\text{А} \cdot 30\,\text{Ом} = 120\,\text{В},  \\
    P &= \mathcal{I}^2R = \sqr{ 4{,}00\,\text{А} } \cdot 30\,\text{Ом} = 480\,\text{Вт}
    \end{align*}
}
\solutionspace{60pt}

\tasknumber{5}%
\task{%
    Замкнутая электрическая цепь состоит из ЭДС $\mathcal{E} = 4\,\text{В}$ и сопротивлением $r$
    и резистора $R = 24\,\text{Ом}$.
    Определите ток, протекающий в цепи.
    Какая тепловая энергия выделится на резисторе за время
    $\tau = 10\,\text{с}$? Какая работа будет совершена ЭДС за это время? Каков знак этой работы? Чему равен КПД цепи?
    Вычислите значения для 2 случаев: $r=0$ и $r = 30\,\text{Ом}$.
}
\answer{%
    \begin{align*}
    \mathcal{I}_1 &= \frac{ \mathcal{E} }{ R } = \frac{ 4\,\text{В} }{ 24\,\text{Ом} } = 0{,}17\,\text{А},  \\
    \mathcal{I}_2 &= \frac{ \mathcal{E} }{R + r} = \frac{ 4\,\text{В} }{24\,\text{Ом} + 30\,\text{Ом}} = 0{,}07\,\text{А},  \\
    Q_1 &= \mathcal{I}_1^2R\tau = \sqr{\frac{ \mathcal{E} }{ R }} R \tau
            = \sqr{\frac{ 4\,\text{В} }{ 24\,\text{Ом} }} \cdot 24\,\text{Ом} \cdot 10\,\text{с} = 6{,}936\,\text{Дж},  \\
    Q_2 &= \mathcal{I}_2^2R\tau = \sqr{\frac{ \mathcal{E} }{R + r}} R \tau
            = \sqr{\frac{ 4\,\text{В} }{24\,\text{Ом} + 30\,\text{Ом}}} \cdot 24\,\text{Ом} \cdot 10\,\text{с} = 1{,}176\,\text{Дж},  \\
    A_1 &= \mathcal{I}_1\tau\mathcal{E} = \frac{ \mathcal{E} }{R} \tau \mathcal{E}
            = \frac{\mathcal{E}^2 \tau}{ R } = \frac{\sqr{ 4\,\text{В} } \cdot 10\,\text{с}}{ 24\,\text{Ом} }
            = 6{,}800\,\text{Дж}, \text{положительна},  \\
    A_2 &= \mathcal{I}_2\tau\mathcal{E} = \frac{ \mathcal{E} }{R + r} \tau \mathcal{E}
            = \frac{\mathcal{E}^2 \tau}{R + r} = \frac{\sqr{ 4\,\text{В} } \cdot 10\,\text{с}}{24\,\text{Ом} + 30\,\text{Ом}}
            = 2{,}800\,\text{Дж}, \text{положительна},  \\
    \eta_1 &= \frac{ Q_1 }{ A_1 } = \ldots = \frac{ R }{ R } = 1,  \\
    \eta_2 &= \frac{ Q_2 }{ A_2 } = \ldots = \frac{ R }{R + r} = 0{,}42
    \end{align*}
}
\solutionspace{180pt}

\tasknumber{6}%
\task{%
    Лампочки, сопротивления которых $R_1 = 4{,}00\,\text{Ом}$ и $R_2 = 100{,}00\,\text{Ом}$, поочерёдно подключённные к некоторому источнику тока,
    потребляют одинаковую мощность.
    Найти внутреннее сопротивление источника и КПД цепи в каждом случае.
}
\answer{%
    \begin{align*}
        P_1 &= \sqr{\frac{ \mathcal{E} }{R_1 + r}}R_1,
        P_2  = \sqr{\frac{ \mathcal{E} }{R_2 + r}}R_2,
        P_1 = P_2 \implies  \\
        &\implies R_1 \sqr{R_2 + r} = R_2 \sqr{R_1 + r} \implies  \\
        &\implies R_1 R_2^2 + 2 R_1 R_2 r + R_1 r^2 =
                    R_2 R_1^2 + 2 R_2 R_1 r + R_2 r^2  \implies  \\
    &\implies r^2 (R_2 - R_1) = R_2^2 R_2 - R_1^2 R_2 \implies  \\
    &\implies r
            = \sqrt{R_1 R_2 \frac{R_2 - R_1}{R_2 - R_1}}
            = \sqrt{R_1 R_2}
            = \sqrt{4{,}00\,\text{Ом} \cdot 100{,}00\,\text{Ом}}
            = 20{,}0\,\text{Ом}.
            \\
    \eta_1
            &= \frac{ R_1 }{R_1 + r}
            = \frac{\sqrt{ R_1 }}{\sqrt{ R_1 } + \sqrt{ R_2 }}
            = 0{,}167,  \\
    \eta_2
            &= \frac{ R_2 }{R_2 + r}
            = \frac{ \sqrt{ R_2 } }{\sqrt{ R_2 } + \sqrt{ R_1 }}
            = 0{,}833
    \end{align*}
}

\variantsplitter

\addpersonalvariant{Владимир Артемчук}

\tasknumber{1}%
\task{%
    Напротив физических величин укажите их обозначения и единицы измерения в СИ:
    \begin{enumerate}
        \item напряжение,
        \item мощность тока,
        \item удельное сопротивление,
        \item внутреннее сопротивление полной цепи.
    \end{enumerate}
}
\solutionspace{20pt}

\tasknumber{2}%
\task{%
    Запишите физический закон или формулу:
    \begin{enumerate}
        \item правило Кирхгофа для замкнутого контура,
        \item закон Ома для однородного участка цепи,
        \item ЭДС (определение).
    \end{enumerate}
}
\solutionspace{40pt}

\tasknumber{3}%
\task{%
    На резистор сопротивлением $r = 18\,\text{Ом}$ подали напряжение $U = 180\,\text{В}$.
    Определите ток, который потечёт через резистор, и мощность, выделяющуюся на нём.
}
\answer{%
    \begin{align*}
    \mathcal{I} &= \frac{ U }{ r } = \frac{ 180\,\text{В} }{ 18\,\text{Ом} } = 10{,}00\,\text{А},  \\
    P &= \frac{U^2}{ r } = \frac{ \sqr{ 180\,\text{В} } }{ 18\,\text{Ом} } = 1800{,}00\,\text{Вт}
    \end{align*}
}
\solutionspace{60pt}

\tasknumber{4}%
\task{%
    Через резистор сопротивлением $r = 12\,\text{Ом}$ протекает электрический ток $\mathcal{I} = 2{,}00\,\text{А}$.
    Определите, чему равны напряжение на резисторе и мощность, выделяющаяся на нём.
}
\answer{%
    \begin{align*}
    U &= \mathcal{I}r = 2{,}00\,\text{А} \cdot 12\,\text{Ом} = 24\,\text{В},  \\
    P &= \mathcal{I}^2r = \sqr{ 2{,}00\,\text{А} } \cdot 12\,\text{Ом} = 48\,\text{Вт}
    \end{align*}
}
\solutionspace{60pt}

\tasknumber{5}%
\task{%
    Замкнутая электрическая цепь состоит из ЭДС $\mathcal{E} = 1\,\text{В}$ и сопротивлением $r$
    и резистора $R = 30\,\text{Ом}$.
    Определите ток, протекающий в цепи.
    Какая тепловая энергия выделится на резисторе за время
    $\tau = 10\,\text{с}$? Какая работа будет совершена ЭДС за это время? Каков знак этой работы? Чему равен КПД цепи?
    Вычислите значения для 2 случаев: $r=0$ и $r = 30\,\text{Ом}$.
}
\answer{%
    \begin{align*}
    \mathcal{I}_1 &= \frac{ \mathcal{E} }{ R } = \frac{ 1\,\text{В} }{ 30\,\text{Ом} } = 0{,}03\,\text{А},  \\
    \mathcal{I}_2 &= \frac{ \mathcal{E} }{R + r} = \frac{ 1\,\text{В} }{30\,\text{Ом} + 30\,\text{Ом}} = 0{,}02\,\text{А},  \\
    Q_1 &= \mathcal{I}_1^2R\tau = \sqr{\frac{ \mathcal{E} }{ R }} R \tau
            = \sqr{\frac{ 1\,\text{В} }{ 30\,\text{Ом} }} \cdot 30\,\text{Ом} \cdot 10\,\text{с} = 0{,}270\,\text{Дж},  \\
    Q_2 &= \mathcal{I}_2^2R\tau = \sqr{\frac{ \mathcal{E} }{R + r}} R \tau
            = \sqr{\frac{ 1\,\text{В} }{30\,\text{Ом} + 30\,\text{Ом}}} \cdot 30\,\text{Ом} \cdot 10\,\text{с} = 0{,}120\,\text{Дж},  \\
    A_1 &= \mathcal{I}_1\tau\mathcal{E} = \frac{ \mathcal{E} }{R} \tau \mathcal{E}
            = \frac{\mathcal{E}^2 \tau}{ R } = \frac{\sqr{ 1\,\text{В} } \cdot 10\,\text{с}}{ 30\,\text{Ом} }
            = 0{,}300\,\text{Дж}, \text{положительна},  \\
    A_2 &= \mathcal{I}_2\tau\mathcal{E} = \frac{ \mathcal{E} }{R + r} \tau \mathcal{E}
            = \frac{\mathcal{E}^2 \tau}{R + r} = \frac{\sqr{ 1\,\text{В} } \cdot 10\,\text{с}}{30\,\text{Ом} + 30\,\text{Ом}}
            = 0{,}200\,\text{Дж}, \text{положительна},  \\
    \eta_1 &= \frac{ Q_1 }{ A_1 } = \ldots = \frac{ R }{ R } = 1,  \\
    \eta_2 &= \frac{ Q_2 }{ A_2 } = \ldots = \frac{ R }{R + r} = 0{,}60
    \end{align*}
}
\solutionspace{180pt}

\tasknumber{6}%
\task{%
    Лампочки, сопротивления которых $R_1 = 1{,}00\,\text{Ом}$ и $R_2 = 49{,}00\,\text{Ом}$, поочерёдно подключённные к некоторому источнику тока,
    потребляют одинаковую мощность.
    Найти внутреннее сопротивление источника и КПД цепи в каждом случае.
}
\answer{%
    \begin{align*}
        P_1 &= \sqr{\frac{ \mathcal{E} }{R_1 + r}}R_1,
        P_2  = \sqr{\frac{ \mathcal{E} }{R_2 + r}}R_2,
        P_1 = P_2 \implies  \\
        &\implies R_1 \sqr{R_2 + r} = R_2 \sqr{R_1 + r} \implies  \\
        &\implies R_1 R_2^2 + 2 R_1 R_2 r + R_1 r^2 =
                    R_2 R_1^2 + 2 R_2 R_1 r + R_2 r^2  \implies  \\
    &\implies r^2 (R_2 - R_1) = R_2^2 R_2 - R_1^2 R_2 \implies  \\
    &\implies r
            = \sqrt{R_1 R_2 \frac{R_2 - R_1}{R_2 - R_1}}
            = \sqrt{R_1 R_2}
            = \sqrt{1{,}00\,\text{Ом} \cdot 49{,}00\,\text{Ом}}
            = 7{,}0\,\text{Ом}.
            \\
    \eta_1
            &= \frac{ R_1 }{R_1 + r}
            = \frac{\sqrt{ R_1 }}{\sqrt{ R_1 } + \sqrt{ R_2 }}
            = 0{,}125,  \\
    \eta_2
            &= \frac{ R_2 }{R_2 + r}
            = \frac{ \sqrt{ R_2 } }{\sqrt{ R_2 } + \sqrt{ R_1 }}
            = 0{,}875
    \end{align*}
}

\variantsplitter

\addpersonalvariant{Софья Белянкина}

\tasknumber{1}%
\task{%
    Напротив физических величин укажите их обозначения и единицы измерения в СИ:
    \begin{enumerate}
        \item разность потенциалов,
        \item мощность тока,
        \item ЭДС,
        \item внутреннее сопротивление полной цепи.
    \end{enumerate}
}
\solutionspace{20pt}

\tasknumber{2}%
\task{%
    Запишите физический закон или формулу:
    \begin{enumerate}
        \item правило Кирхгофа для замкнутого контура,
        \item закон Ома для однородного участка цепи,
        \item закон Ома для неоднородного участка цепи.
    \end{enumerate}
}
\solutionspace{40pt}

\tasknumber{3}%
\task{%
    На резистор сопротивлением $R = 30\,\text{Ом}$ подали напряжение $V = 120\,\text{В}$.
    Определите ток, который потечёт через резистор, и мощность, выделяющуюся на нём.
}
\answer{%
    \begin{align*}
    \mathcal{I} &= \frac{ V }{ R } = \frac{ 120\,\text{В} }{ 30\,\text{Ом} } = 4{,}00\,\text{А},  \\
    P &= \frac{V^2}{ R } = \frac{ \sqr{ 120\,\text{В} } }{ 30\,\text{Ом} } = 480{,}00\,\text{Вт}
    \end{align*}
}
\solutionspace{60pt}

\tasknumber{4}%
\task{%
    Через резистор сопротивлением $R = 5\,\text{Ом}$ протекает электрический ток $\mathcal{I} = 4{,}00\,\text{А}$.
    Определите, чему равны напряжение на резисторе и мощность, выделяющаяся на нём.
}
\answer{%
    \begin{align*}
    U &= \mathcal{I}R = 4{,}00\,\text{А} \cdot 5\,\text{Ом} = 20\,\text{В},  \\
    P &= \mathcal{I}^2R = \sqr{ 4{,}00\,\text{А} } \cdot 5\,\text{Ом} = 80\,\text{Вт}
    \end{align*}
}
\solutionspace{60pt}

\tasknumber{5}%
\task{%
    Замкнутая электрическая цепь состоит из ЭДС $\mathcal{E} = 3\,\text{В}$ и сопротивлением $r$
    и резистора $R = 15\,\text{Ом}$.
    Определите ток, протекающий в цепи.
    Какая тепловая энергия выделится на резисторе за время
    $\tau = 10\,\text{с}$? Какая работа будет совершена ЭДС за это время? Каков знак этой работы? Чему равен КПД цепи?
    Вычислите значения для 2 случаев: $r=0$ и $r = 10\,\text{Ом}$.
}
\answer{%
    \begin{align*}
    \mathcal{I}_1 &= \frac{ \mathcal{E} }{ R } = \frac{ 3\,\text{В} }{ 15\,\text{Ом} } = 0{,}20\,\text{А},  \\
    \mathcal{I}_2 &= \frac{ \mathcal{E} }{R + r} = \frac{ 3\,\text{В} }{15\,\text{Ом} + 10\,\text{Ом}} = 0{,}12\,\text{А},  \\
    Q_1 &= \mathcal{I}_1^2R\tau = \sqr{\frac{ \mathcal{E} }{ R }} R \tau
            = \sqr{\frac{ 3\,\text{В} }{ 15\,\text{Ом} }} \cdot 15\,\text{Ом} \cdot 10\,\text{с} = 6{,}000\,\text{Дж},  \\
    Q_2 &= \mathcal{I}_2^2R\tau = \sqr{\frac{ \mathcal{E} }{R + r}} R \tau
            = \sqr{\frac{ 3\,\text{В} }{15\,\text{Ом} + 10\,\text{Ом}}} \cdot 15\,\text{Ом} \cdot 10\,\text{с} = 2{,}160\,\text{Дж},  \\
    A_1 &= \mathcal{I}_1\tau\mathcal{E} = \frac{ \mathcal{E} }{R} \tau \mathcal{E}
            = \frac{\mathcal{E}^2 \tau}{ R } = \frac{\sqr{ 3\,\text{В} } \cdot 10\,\text{с}}{ 15\,\text{Ом} }
            = 6{,}000\,\text{Дж}, \text{положительна},  \\
    A_2 &= \mathcal{I}_2\tau\mathcal{E} = \frac{ \mathcal{E} }{R + r} \tau \mathcal{E}
            = \frac{\mathcal{E}^2 \tau}{R + r} = \frac{\sqr{ 3\,\text{В} } \cdot 10\,\text{с}}{15\,\text{Ом} + 10\,\text{Ом}}
            = 3{,}600\,\text{Дж}, \text{положительна},  \\
    \eta_1 &= \frac{ Q_1 }{ A_1 } = \ldots = \frac{ R }{ R } = 1,  \\
    \eta_2 &= \frac{ Q_2 }{ A_2 } = \ldots = \frac{ R }{R + r} = 0{,}60
    \end{align*}
}
\solutionspace{180pt}

\tasknumber{6}%
\task{%
    Лампочки, сопротивления которых $R_1 = 3{,}00\,\text{Ом}$ и $R_2 = 48{,}00\,\text{Ом}$, поочерёдно подключённные к некоторому источнику тока,
    потребляют одинаковую мощность.
    Найти внутреннее сопротивление источника и КПД цепи в каждом случае.
}
\answer{%
    \begin{align*}
        P_1 &= \sqr{\frac{ \mathcal{E} }{R_1 + r}}R_1,
        P_2  = \sqr{\frac{ \mathcal{E} }{R_2 + r}}R_2,
        P_1 = P_2 \implies  \\
        &\implies R_1 \sqr{R_2 + r} = R_2 \sqr{R_1 + r} \implies  \\
        &\implies R_1 R_2^2 + 2 R_1 R_2 r + R_1 r^2 =
                    R_2 R_1^2 + 2 R_2 R_1 r + R_2 r^2  \implies  \\
    &\implies r^2 (R_2 - R_1) = R_2^2 R_2 - R_1^2 R_2 \implies  \\
    &\implies r
            = \sqrt{R_1 R_2 \frac{R_2 - R_1}{R_2 - R_1}}
            = \sqrt{R_1 R_2}
            = \sqrt{3{,}00\,\text{Ом} \cdot 48{,}00\,\text{Ом}}
            = 12{,}0\,\text{Ом}.
            \\
    \eta_1
            &= \frac{ R_1 }{R_1 + r}
            = \frac{\sqrt{ R_1 }}{\sqrt{ R_1 } + \sqrt{ R_2 }}
            = 0{,}200,  \\
    \eta_2
            &= \frac{ R_2 }{R_2 + r}
            = \frac{ \sqrt{ R_2 } }{\sqrt{ R_2 } + \sqrt{ R_1 }}
            = 0{,}800
    \end{align*}
}

\variantsplitter

\addpersonalvariant{Варвара Егиазарян}

\tasknumber{1}%
\task{%
    Напротив физических величин укажите их обозначения и единицы измерения в СИ:
    \begin{enumerate}
        \item сила тока,
        \item работа тока,
        \item удельное сопротивление,
        \item внутреннее сопротивление полной цепи.
    \end{enumerate}
}
\solutionspace{20pt}

\tasknumber{2}%
\task{%
    Запишите физический закон или формулу:
    \begin{enumerate}
        \item правило Кирхгофа для замкнутого контура,
        \item закон Ома для однородного участка цепи,
        \item закон Ома для неоднородного участка цепи.
    \end{enumerate}
}
\solutionspace{40pt}

\tasknumber{3}%
\task{%
    На резистор сопротивлением $r = 12\,\text{Ом}$ подали напряжение $U = 180\,\text{В}$.
    Определите ток, который потечёт через резистор, и мощность, выделяющуюся на нём.
}
\answer{%
    \begin{align*}
    \mathcal{I} &= \frac{ U }{ r } = \frac{ 180\,\text{В} }{ 12\,\text{Ом} } = 15{,}00\,\text{А},  \\
    P &= \frac{U^2}{ r } = \frac{ \sqr{ 180\,\text{В} } }{ 12\,\text{Ом} } = 2700{,}00\,\text{Вт}
    \end{align*}
}
\solutionspace{60pt}

\tasknumber{4}%
\task{%
    Через резистор сопротивлением $R = 18\,\text{Ом}$ протекает электрический ток $\mathcal{I} = 8{,}00\,\text{А}$.
    Определите, чему равны напряжение на резисторе и мощность, выделяющаяся на нём.
}
\answer{%
    \begin{align*}
    U &= \mathcal{I}R = 8{,}00\,\text{А} \cdot 18\,\text{Ом} = 144\,\text{В},  \\
    P &= \mathcal{I}^2R = \sqr{ 8{,}00\,\text{А} } \cdot 18\,\text{Ом} = 1152\,\text{Вт}
    \end{align*}
}
\solutionspace{60pt}

\tasknumber{5}%
\task{%
    Замкнутая электрическая цепь состоит из ЭДС $\mathcal{E} = 1\,\text{В}$ и сопротивлением $r$
    и резистора $R = 24\,\text{Ом}$.
    Определите ток, протекающий в цепи.
    Какая тепловая энергия выделится на резисторе за время
    $\tau = 2\,\text{с}$? Какая работа будет совершена ЭДС за это время? Каков знак этой работы? Чему равен КПД цепи?
    Вычислите значения для 2 случаев: $r=0$ и $r = 20\,\text{Ом}$.
}
\answer{%
    \begin{align*}
    \mathcal{I}_1 &= \frac{ \mathcal{E} }{ R } = \frac{ 1\,\text{В} }{ 24\,\text{Ом} } = 0{,}04\,\text{А},  \\
    \mathcal{I}_2 &= \frac{ \mathcal{E} }{R + r} = \frac{ 1\,\text{В} }{24\,\text{Ом} + 20\,\text{Ом}} = 0{,}02\,\text{А},  \\
    Q_1 &= \mathcal{I}_1^2R\tau = \sqr{\frac{ \mathcal{E} }{ R }} R \tau
            = \sqr{\frac{ 1\,\text{В} }{ 24\,\text{Ом} }} \cdot 24\,\text{Ом} \cdot 2\,\text{с} = 0{,}077\,\text{Дж},  \\
    Q_2 &= \mathcal{I}_2^2R\tau = \sqr{\frac{ \mathcal{E} }{R + r}} R \tau
            = \sqr{\frac{ 1\,\text{В} }{24\,\text{Ом} + 20\,\text{Ом}}} \cdot 24\,\text{Ом} \cdot 2\,\text{с} = 0{,}019\,\text{Дж},  \\
    A_1 &= \mathcal{I}_1\tau\mathcal{E} = \frac{ \mathcal{E} }{R} \tau \mathcal{E}
            = \frac{\mathcal{E}^2 \tau}{ R } = \frac{\sqr{ 1\,\text{В} } \cdot 2\,\text{с}}{ 24\,\text{Ом} }
            = 0{,}080\,\text{Дж}, \text{положительна},  \\
    A_2 &= \mathcal{I}_2\tau\mathcal{E} = \frac{ \mathcal{E} }{R + r} \tau \mathcal{E}
            = \frac{\mathcal{E}^2 \tau}{R + r} = \frac{\sqr{ 1\,\text{В} } \cdot 2\,\text{с}}{24\,\text{Ом} + 20\,\text{Ом}}
            = 0{,}040\,\text{Дж}, \text{положительна},  \\
    \eta_1 &= \frac{ Q_1 }{ A_1 } = \ldots = \frac{ R }{ R } = 1,  \\
    \eta_2 &= \frac{ Q_2 }{ A_2 } = \ldots = \frac{ R }{R + r} = 0{,}47
    \end{align*}
}
\solutionspace{180pt}

\tasknumber{6}%
\task{%
    Лампочки, сопротивления которых $R_1 = 5{,}00\,\text{Ом}$ и $R_2 = 80{,}00\,\text{Ом}$, поочерёдно подключённные к некоторому источнику тока,
    потребляют одинаковую мощность.
    Найти внутреннее сопротивление источника и КПД цепи в каждом случае.
}
\answer{%
    \begin{align*}
        P_1 &= \sqr{\frac{ \mathcal{E} }{R_1 + r}}R_1,
        P_2  = \sqr{\frac{ \mathcal{E} }{R_2 + r}}R_2,
        P_1 = P_2 \implies  \\
        &\implies R_1 \sqr{R_2 + r} = R_2 \sqr{R_1 + r} \implies  \\
        &\implies R_1 R_2^2 + 2 R_1 R_2 r + R_1 r^2 =
                    R_2 R_1^2 + 2 R_2 R_1 r + R_2 r^2  \implies  \\
    &\implies r^2 (R_2 - R_1) = R_2^2 R_2 - R_1^2 R_2 \implies  \\
    &\implies r
            = \sqrt{R_1 R_2 \frac{R_2 - R_1}{R_2 - R_1}}
            = \sqrt{R_1 R_2}
            = \sqrt{5{,}00\,\text{Ом} \cdot 80{,}00\,\text{Ом}}
            = 20{,}0\,\text{Ом}.
            \\
    \eta_1
            &= \frac{ R_1 }{R_1 + r}
            = \frac{\sqrt{ R_1 }}{\sqrt{ R_1 } + \sqrt{ R_2 }}
            = 0{,}200,  \\
    \eta_2
            &= \frac{ R_2 }{R_2 + r}
            = \frac{ \sqrt{ R_2 } }{\sqrt{ R_2 } + \sqrt{ R_1 }}
            = 0{,}800
    \end{align*}
}

\variantsplitter

\addpersonalvariant{Владислав Емелин}

\tasknumber{1}%
\task{%
    Напротив физических величин укажите их обозначения и единицы измерения в СИ:
    \begin{enumerate}
        \item напряжение,
        \item работа тока,
        \item ЭДС,
        \item внутреннее сопротивление полной цепи.
    \end{enumerate}
}
\solutionspace{20pt}

\tasknumber{2}%
\task{%
    Запишите физический закон или формулу:
    \begin{enumerate}
        \item правило Кирхгофа для узла цепи,
        \item сопротивление резистора через удельное сопротивление,
        \item закон Ома для неоднородного участка цепи.
    \end{enumerate}
}
\solutionspace{40pt}

\tasknumber{3}%
\task{%
    На резистор сопротивлением $r = 12\,\text{Ом}$ подали напряжение $V = 240\,\text{В}$.
    Определите ток, который потечёт через резистор, и мощность, выделяющуюся на нём.
}
\answer{%
    \begin{align*}
    \mathcal{I} &= \frac{ V }{ r } = \frac{ 240\,\text{В} }{ 12\,\text{Ом} } = 20{,}00\,\text{А},  \\
    P &= \frac{V^2}{ r } = \frac{ \sqr{ 240\,\text{В} } }{ 12\,\text{Ом} } = 4800{,}00\,\text{Вт}
    \end{align*}
}
\solutionspace{60pt}

\tasknumber{4}%
\task{%
    Через резистор сопротивлением $R = 18\,\text{Ом}$ протекает электрический ток $\mathcal{I} = 8{,}00\,\text{А}$.
    Определите, чему равны напряжение на резисторе и мощность, выделяющаяся на нём.
}
\answer{%
    \begin{align*}
    U &= \mathcal{I}R = 8{,}00\,\text{А} \cdot 18\,\text{Ом} = 144\,\text{В},  \\
    P &= \mathcal{I}^2R = \sqr{ 8{,}00\,\text{А} } \cdot 18\,\text{Ом} = 1152\,\text{Вт}
    \end{align*}
}
\solutionspace{60pt}

\tasknumber{5}%
\task{%
    Замкнутая электрическая цепь состоит из ЭДС $\mathcal{E} = 4\,\text{В}$ и сопротивлением $r$
    и резистора $R = 24\,\text{Ом}$.
    Определите ток, протекающий в цепи.
    Какая тепловая энергия выделится на резисторе за время
    $\tau = 2\,\text{с}$? Какая работа будет совершена ЭДС за это время? Каков знак этой работы? Чему равен КПД цепи?
    Вычислите значения для 2 случаев: $r=0$ и $r = 60\,\text{Ом}$.
}
\answer{%
    \begin{align*}
    \mathcal{I}_1 &= \frac{ \mathcal{E} }{ R } = \frac{ 4\,\text{В} }{ 24\,\text{Ом} } = 0{,}17\,\text{А},  \\
    \mathcal{I}_2 &= \frac{ \mathcal{E} }{R + r} = \frac{ 4\,\text{В} }{24\,\text{Ом} + 60\,\text{Ом}} = 0{,}05\,\text{А},  \\
    Q_1 &= \mathcal{I}_1^2R\tau = \sqr{\frac{ \mathcal{E} }{ R }} R \tau
            = \sqr{\frac{ 4\,\text{В} }{ 24\,\text{Ом} }} \cdot 24\,\text{Ом} \cdot 2\,\text{с} = 1{,}387\,\text{Дж},  \\
    Q_2 &= \mathcal{I}_2^2R\tau = \sqr{\frac{ \mathcal{E} }{R + r}} R \tau
            = \sqr{\frac{ 4\,\text{В} }{24\,\text{Ом} + 60\,\text{Ом}}} \cdot 24\,\text{Ом} \cdot 2\,\text{с} = 0{,}120\,\text{Дж},  \\
    A_1 &= \mathcal{I}_1\tau\mathcal{E} = \frac{ \mathcal{E} }{R} \tau \mathcal{E}
            = \frac{\mathcal{E}^2 \tau}{ R } = \frac{\sqr{ 4\,\text{В} } \cdot 2\,\text{с}}{ 24\,\text{Ом} }
            = 1{,}360\,\text{Дж}, \text{положительна},  \\
    A_2 &= \mathcal{I}_2\tau\mathcal{E} = \frac{ \mathcal{E} }{R + r} \tau \mathcal{E}
            = \frac{\mathcal{E}^2 \tau}{R + r} = \frac{\sqr{ 4\,\text{В} } \cdot 2\,\text{с}}{24\,\text{Ом} + 60\,\text{Ом}}
            = 0{,}400\,\text{Дж}, \text{положительна},  \\
    \eta_1 &= \frac{ Q_1 }{ A_1 } = \ldots = \frac{ R }{ R } = 1,  \\
    \eta_2 &= \frac{ Q_2 }{ A_2 } = \ldots = \frac{ R }{R + r} = 0{,}30
    \end{align*}
}
\solutionspace{180pt}

\tasknumber{6}%
\task{%
    Лампочки, сопротивления которых $R_1 = 0{,}50\,\text{Ом}$ и $R_2 = 2{,}00\,\text{Ом}$, поочерёдно подключённные к некоторому источнику тока,
    потребляют одинаковую мощность.
    Найти внутреннее сопротивление источника и КПД цепи в каждом случае.
}
\answer{%
    \begin{align*}
        P_1 &= \sqr{\frac{ \mathcal{E} }{R_1 + r}}R_1,
        P_2  = \sqr{\frac{ \mathcal{E} }{R_2 + r}}R_2,
        P_1 = P_2 \implies  \\
        &\implies R_1 \sqr{R_2 + r} = R_2 \sqr{R_1 + r} \implies  \\
        &\implies R_1 R_2^2 + 2 R_1 R_2 r + R_1 r^2 =
                    R_2 R_1^2 + 2 R_2 R_1 r + R_2 r^2  \implies  \\
    &\implies r^2 (R_2 - R_1) = R_2^2 R_2 - R_1^2 R_2 \implies  \\
    &\implies r
            = \sqrt{R_1 R_2 \frac{R_2 - R_1}{R_2 - R_1}}
            = \sqrt{R_1 R_2}
            = \sqrt{0{,}50\,\text{Ом} \cdot 2{,}00\,\text{Ом}}
            = 1{,}0\,\text{Ом}.
            \\
    \eta_1
            &= \frac{ R_1 }{R_1 + r}
            = \frac{\sqrt{ R_1 }}{\sqrt{ R_1 } + \sqrt{ R_2 }}
            = 0{,}333,  \\
    \eta_2
            &= \frac{ R_2 }{R_2 + r}
            = \frac{ \sqrt{ R_2 } }{\sqrt{ R_2 } + \sqrt{ R_1 }}
            = 0{,}667
    \end{align*}
}

\variantsplitter

\addpersonalvariant{Артём Жичин}

\tasknumber{1}%
\task{%
    Напротив физических величин укажите их обозначения и единицы измерения в СИ:
    \begin{enumerate}
        \item сила тока,
        \item работа тока,
        \item ЭДС,
        \item внутреннее сопротивление полной цепи.
    \end{enumerate}
}
\solutionspace{20pt}

\tasknumber{2}%
\task{%
    Запишите физический закон или формулу:
    \begin{enumerate}
        \item правило Кирхгофа для узла цепи,
        \item сопротивление резистора через удельное сопротивление,
        \item закон Ома для неоднородного участка цепи.
    \end{enumerate}
}
\solutionspace{40pt}

\tasknumber{3}%
\task{%
    На резистор сопротивлением $r = 30\,\text{Ом}$ подали напряжение $U = 120\,\text{В}$.
    Определите ток, который потечёт через резистор, и мощность, выделяющуюся на нём.
}
\answer{%
    \begin{align*}
    \mathcal{I} &= \frac{ U }{ r } = \frac{ 120\,\text{В} }{ 30\,\text{Ом} } = 4{,}00\,\text{А},  \\
    P &= \frac{U^2}{ r } = \frac{ \sqr{ 120\,\text{В} } }{ 30\,\text{Ом} } = 480{,}00\,\text{Вт}
    \end{align*}
}
\solutionspace{60pt}

\tasknumber{4}%
\task{%
    Через резистор сопротивлением $r = 18\,\text{Ом}$ протекает электрический ток $\mathcal{I} = 15{,}00\,\text{А}$.
    Определите, чему равны напряжение на резисторе и мощность, выделяющаяся на нём.
}
\answer{%
    \begin{align*}
    U &= \mathcal{I}r = 15{,}00\,\text{А} \cdot 18\,\text{Ом} = 270\,\text{В},  \\
    P &= \mathcal{I}^2r = \sqr{ 15{,}00\,\text{А} } \cdot 18\,\text{Ом} = 4050\,\text{Вт}
    \end{align*}
}
\solutionspace{60pt}

\tasknumber{5}%
\task{%
    Замкнутая электрическая цепь состоит из ЭДС $\mathcal{E} = 2\,\text{В}$ и сопротивлением $r$
    и резистора $R = 10\,\text{Ом}$.
    Определите ток, протекающий в цепи.
    Какая тепловая энергия выделится на резисторе за время
    $\tau = 5\,\text{с}$? Какая работа будет совершена ЭДС за это время? Каков знак этой работы? Чему равен КПД цепи?
    Вычислите значения для 2 случаев: $r=0$ и $r = 60\,\text{Ом}$.
}
\answer{%
    \begin{align*}
    \mathcal{I}_1 &= \frac{ \mathcal{E} }{ R } = \frac{ 2\,\text{В} }{ 10\,\text{Ом} } = 0{,}20\,\text{А},  \\
    \mathcal{I}_2 &= \frac{ \mathcal{E} }{R + r} = \frac{ 2\,\text{В} }{10\,\text{Ом} + 60\,\text{Ом}} = 0{,}03\,\text{А},  \\
    Q_1 &= \mathcal{I}_1^2R\tau = \sqr{\frac{ \mathcal{E} }{ R }} R \tau
            = \sqr{\frac{ 2\,\text{В} }{ 10\,\text{Ом} }} \cdot 10\,\text{Ом} \cdot 5\,\text{с} = 2{,}000\,\text{Дж},  \\
    Q_2 &= \mathcal{I}_2^2R\tau = \sqr{\frac{ \mathcal{E} }{R + r}} R \tau
            = \sqr{\frac{ 2\,\text{В} }{10\,\text{Ом} + 60\,\text{Ом}}} \cdot 10\,\text{Ом} \cdot 5\,\text{с} = 0{,}045\,\text{Дж},  \\
    A_1 &= \mathcal{I}_1\tau\mathcal{E} = \frac{ \mathcal{E} }{R} \tau \mathcal{E}
            = \frac{\mathcal{E}^2 \tau}{ R } = \frac{\sqr{ 2\,\text{В} } \cdot 5\,\text{с}}{ 10\,\text{Ом} }
            = 2{,}000\,\text{Дж}, \text{положительна},  \\
    A_2 &= \mathcal{I}_2\tau\mathcal{E} = \frac{ \mathcal{E} }{R + r} \tau \mathcal{E}
            = \frac{\mathcal{E}^2 \tau}{R + r} = \frac{\sqr{ 2\,\text{В} } \cdot 5\,\text{с}}{10\,\text{Ом} + 60\,\text{Ом}}
            = 0{,}300\,\text{Дж}, \text{положительна},  \\
    \eta_1 &= \frac{ Q_1 }{ A_1 } = \ldots = \frac{ R }{ R } = 1,  \\
    \eta_2 &= \frac{ Q_2 }{ A_2 } = \ldots = \frac{ R }{R + r} = 0{,}15
    \end{align*}
}
\solutionspace{180pt}

\tasknumber{6}%
\task{%
    Лампочки, сопротивления которых $R_1 = 1{,}00\,\text{Ом}$ и $R_2 = 49{,}00\,\text{Ом}$, поочерёдно подключённные к некоторому источнику тока,
    потребляют одинаковую мощность.
    Найти внутреннее сопротивление источника и КПД цепи в каждом случае.
}
\answer{%
    \begin{align*}
        P_1 &= \sqr{\frac{ \mathcal{E} }{R_1 + r}}R_1,
        P_2  = \sqr{\frac{ \mathcal{E} }{R_2 + r}}R_2,
        P_1 = P_2 \implies  \\
        &\implies R_1 \sqr{R_2 + r} = R_2 \sqr{R_1 + r} \implies  \\
        &\implies R_1 R_2^2 + 2 R_1 R_2 r + R_1 r^2 =
                    R_2 R_1^2 + 2 R_2 R_1 r + R_2 r^2  \implies  \\
    &\implies r^2 (R_2 - R_1) = R_2^2 R_2 - R_1^2 R_2 \implies  \\
    &\implies r
            = \sqrt{R_1 R_2 \frac{R_2 - R_1}{R_2 - R_1}}
            = \sqrt{R_1 R_2}
            = \sqrt{1{,}00\,\text{Ом} \cdot 49{,}00\,\text{Ом}}
            = 7{,}0\,\text{Ом}.
            \\
    \eta_1
            &= \frac{ R_1 }{R_1 + r}
            = \frac{\sqrt{ R_1 }}{\sqrt{ R_1 } + \sqrt{ R_2 }}
            = 0{,}125,  \\
    \eta_2
            &= \frac{ R_2 }{R_2 + r}
            = \frac{ \sqrt{ R_2 } }{\sqrt{ R_2 } + \sqrt{ R_1 }}
            = 0{,}875
    \end{align*}
}

\variantsplitter

\addpersonalvariant{Дарья Кошман}

\tasknumber{1}%
\task{%
    Напротив физических величин укажите их обозначения и единицы измерения в СИ:
    \begin{enumerate}
        \item напряжение,
        \item работа тока,
        \item ЭДС,
        \item внутреннее сопротивление полной цепи.
    \end{enumerate}
}
\solutionspace{20pt}

\tasknumber{2}%
\task{%
    Запишите физический закон или формулу:
    \begin{enumerate}
        \item правило Кирхгофа для замкнутого контура,
        \item закон Ома для однородного участка цепи,
        \item закон Ома для неоднородного участка цепи.
    \end{enumerate}
}
\solutionspace{40pt}

\tasknumber{3}%
\task{%
    На резистор сопротивлением $R = 18\,\text{Ом}$ подали напряжение $V = 240\,\text{В}$.
    Определите ток, который потечёт через резистор, и мощность, выделяющуюся на нём.
}
\answer{%
    \begin{align*}
    \mathcal{I} &= \frac{ V }{ R } = \frac{ 240\,\text{В} }{ 18\,\text{Ом} } = 13{,}33\,\text{А},  \\
    P &= \frac{V^2}{ R } = \frac{ \sqr{ 240\,\text{В} } }{ 18\,\text{Ом} } = 3200{,}00\,\text{Вт}
    \end{align*}
}
\solutionspace{60pt}

\tasknumber{4}%
\task{%
    Через резистор сопротивлением $r = 18\,\text{Ом}$ протекает электрический ток $\mathcal{I} = 6{,}00\,\text{А}$.
    Определите, чему равны напряжение на резисторе и мощность, выделяющаяся на нём.
}
\answer{%
    \begin{align*}
    U &= \mathcal{I}r = 6{,}00\,\text{А} \cdot 18\,\text{Ом} = 108\,\text{В},  \\
    P &= \mathcal{I}^2r = \sqr{ 6{,}00\,\text{А} } \cdot 18\,\text{Ом} = 648\,\text{Вт}
    \end{align*}
}
\solutionspace{60pt}

\tasknumber{5}%
\task{%
    Замкнутая электрическая цепь состоит из ЭДС $\mathcal{E} = 3\,\text{В}$ и сопротивлением $r$
    и резистора $R = 24\,\text{Ом}$.
    Определите ток, протекающий в цепи.
    Какая тепловая энергия выделится на резисторе за время
    $\tau = 10\,\text{с}$? Какая работа будет совершена ЭДС за это время? Каков знак этой работы? Чему равен КПД цепи?
    Вычислите значения для 2 случаев: $r=0$ и $r = 10\,\text{Ом}$.
}
\answer{%
    \begin{align*}
    \mathcal{I}_1 &= \frac{ \mathcal{E} }{ R } = \frac{ 3\,\text{В} }{ 24\,\text{Ом} } = 0{,}12\,\text{А},  \\
    \mathcal{I}_2 &= \frac{ \mathcal{E} }{R + r} = \frac{ 3\,\text{В} }{24\,\text{Ом} + 10\,\text{Ом}} = 0{,}09\,\text{А},  \\
    Q_1 &= \mathcal{I}_1^2R\tau = \sqr{\frac{ \mathcal{E} }{ R }} R \tau
            = \sqr{\frac{ 3\,\text{В} }{ 24\,\text{Ом} }} \cdot 24\,\text{Ом} \cdot 10\,\text{с} = 3{,}456\,\text{Дж},  \\
    Q_2 &= \mathcal{I}_2^2R\tau = \sqr{\frac{ \mathcal{E} }{R + r}} R \tau
            = \sqr{\frac{ 3\,\text{В} }{24\,\text{Ом} + 10\,\text{Ом}}} \cdot 24\,\text{Ом} \cdot 10\,\text{с} = 1{,}944\,\text{Дж},  \\
    A_1 &= \mathcal{I}_1\tau\mathcal{E} = \frac{ \mathcal{E} }{R} \tau \mathcal{E}
            = \frac{\mathcal{E}^2 \tau}{ R } = \frac{\sqr{ 3\,\text{В} } \cdot 10\,\text{с}}{ 24\,\text{Ом} }
            = 3{,}600\,\text{Дж}, \text{положительна},  \\
    A_2 &= \mathcal{I}_2\tau\mathcal{E} = \frac{ \mathcal{E} }{R + r} \tau \mathcal{E}
            = \frac{\mathcal{E}^2 \tau}{R + r} = \frac{\sqr{ 3\,\text{В} } \cdot 10\,\text{с}}{24\,\text{Ом} + 10\,\text{Ом}}
            = 2{,}700\,\text{Дж}, \text{положительна},  \\
    \eta_1 &= \frac{ Q_1 }{ A_1 } = \ldots = \frac{ R }{ R } = 1,  \\
    \eta_2 &= \frac{ Q_2 }{ A_2 } = \ldots = \frac{ R }{R + r} = 0{,}72
    \end{align*}
}
\solutionspace{180pt}

\tasknumber{6}%
\task{%
    Лампочки, сопротивления которых $R_1 = 0{,}25\,\text{Ом}$ и $R_2 = 16{,}00\,\text{Ом}$, поочерёдно подключённные к некоторому источнику тока,
    потребляют одинаковую мощность.
    Найти внутреннее сопротивление источника и КПД цепи в каждом случае.
}
\answer{%
    \begin{align*}
        P_1 &= \sqr{\frac{ \mathcal{E} }{R_1 + r}}R_1,
        P_2  = \sqr{\frac{ \mathcal{E} }{R_2 + r}}R_2,
        P_1 = P_2 \implies  \\
        &\implies R_1 \sqr{R_2 + r} = R_2 \sqr{R_1 + r} \implies  \\
        &\implies R_1 R_2^2 + 2 R_1 R_2 r + R_1 r^2 =
                    R_2 R_1^2 + 2 R_2 R_1 r + R_2 r^2  \implies  \\
    &\implies r^2 (R_2 - R_1) = R_2^2 R_2 - R_1^2 R_2 \implies  \\
    &\implies r
            = \sqrt{R_1 R_2 \frac{R_2 - R_1}{R_2 - R_1}}
            = \sqrt{R_1 R_2}
            = \sqrt{0{,}25\,\text{Ом} \cdot 16{,}00\,\text{Ом}}
            = 2{,}0\,\text{Ом}.
            \\
    \eta_1
            &= \frac{ R_1 }{R_1 + r}
            = \frac{\sqrt{ R_1 }}{\sqrt{ R_1 } + \sqrt{ R_2 }}
            = 0{,}111,  \\
    \eta_2
            &= \frac{ R_2 }{R_2 + r}
            = \frac{ \sqrt{ R_2 } }{\sqrt{ R_2 } + \sqrt{ R_1 }}
            = 0{,}889
    \end{align*}
}

\variantsplitter

\addpersonalvariant{Анна Кузьмичёва}

\tasknumber{1}%
\task{%
    Напротив физических величин укажите их обозначения и единицы измерения в СИ:
    \begin{enumerate}
        \item разность потенциалов,
        \item работа тока,
        \item удельное сопротивление,
        \item внутреннее сопротивление полной цепи.
    \end{enumerate}
}
\solutionspace{20pt}

\tasknumber{2}%
\task{%
    Запишите физический закон или формулу:
    \begin{enumerate}
        \item правило Кирхгофа для замкнутого контура,
        \item сопротивление резистора через удельное сопротивление,
        \item ЭДС (определение).
    \end{enumerate}
}
\solutionspace{40pt}

\tasknumber{3}%
\task{%
    На резистор сопротивлением $R = 12\,\text{Ом}$ подали напряжение $U = 150\,\text{В}$.
    Определите ток, который потечёт через резистор, и мощность, выделяющуюся на нём.
}
\answer{%
    \begin{align*}
    \mathcal{I} &= \frac{ U }{ R } = \frac{ 150\,\text{В} }{ 12\,\text{Ом} } = 12{,}50\,\text{А},  \\
    P &= \frac{U^2}{ R } = \frac{ \sqr{ 150\,\text{В} } }{ 12\,\text{Ом} } = 1875{,}00\,\text{Вт}
    \end{align*}
}
\solutionspace{60pt}

\tasknumber{4}%
\task{%
    Через резистор сопротивлением $R = 18\,\text{Ом}$ протекает электрический ток $\mathcal{I} = 5{,}00\,\text{А}$.
    Определите, чему равны напряжение на резисторе и мощность, выделяющаяся на нём.
}
\answer{%
    \begin{align*}
    U &= \mathcal{I}R = 5{,}00\,\text{А} \cdot 18\,\text{Ом} = 90\,\text{В},  \\
    P &= \mathcal{I}^2R = \sqr{ 5{,}00\,\text{А} } \cdot 18\,\text{Ом} = 450\,\text{Вт}
    \end{align*}
}
\solutionspace{60pt}

\tasknumber{5}%
\task{%
    Замкнутая электрическая цепь состоит из ЭДС $\mathcal{E} = 1\,\text{В}$ и сопротивлением $r$
    и резистора $R = 10\,\text{Ом}$.
    Определите ток, протекающий в цепи.
    Какая тепловая энергия выделится на резисторе за время
    $\tau = 10\,\text{с}$? Какая работа будет совершена ЭДС за это время? Каков знак этой работы? Чему равен КПД цепи?
    Вычислите значения для 2 случаев: $r=0$ и $r = 60\,\text{Ом}$.
}
\answer{%
    \begin{align*}
    \mathcal{I}_1 &= \frac{ \mathcal{E} }{ R } = \frac{ 1\,\text{В} }{ 10\,\text{Ом} } = 0{,}10\,\text{А},  \\
    \mathcal{I}_2 &= \frac{ \mathcal{E} }{R + r} = \frac{ 1\,\text{В} }{10\,\text{Ом} + 60\,\text{Ом}} = 0{,}010\,\text{А},  \\
    Q_1 &= \mathcal{I}_1^2R\tau = \sqr{\frac{ \mathcal{E} }{ R }} R \tau
            = \sqr{\frac{ 1\,\text{В} }{ 10\,\text{Ом} }} \cdot 10\,\text{Ом} \cdot 10\,\text{с} = 1{,}000\,\text{Дж},  \\
    Q_2 &= \mathcal{I}_2^2R\tau = \sqr{\frac{ \mathcal{E} }{R + r}} R \tau
            = \sqr{\frac{ 1\,\text{В} }{10\,\text{Ом} + 60\,\text{Ом}}} \cdot 10\,\text{Ом} \cdot 10\,\text{с} = 0{,}010\,\text{Дж},  \\
    A_1 &= \mathcal{I}_1\tau\mathcal{E} = \frac{ \mathcal{E} }{R} \tau \mathcal{E}
            = \frac{\mathcal{E}^2 \tau}{ R } = \frac{\sqr{ 1\,\text{В} } \cdot 10\,\text{с}}{ 10\,\text{Ом} }
            = 1{,}000\,\text{Дж}, \text{положительна},  \\
    A_2 &= \mathcal{I}_2\tau\mathcal{E} = \frac{ \mathcal{E} }{R + r} \tau \mathcal{E}
            = \frac{\mathcal{E}^2 \tau}{R + r} = \frac{\sqr{ 1\,\text{В} } \cdot 10\,\text{с}}{10\,\text{Ом} + 60\,\text{Ом}}
            = 0{,}100\,\text{Дж}, \text{положительна},  \\
    \eta_1 &= \frac{ Q_1 }{ A_1 } = \ldots = \frac{ R }{ R } = 1,  \\
    \eta_2 &= \frac{ Q_2 }{ A_2 } = \ldots = \frac{ R }{R + r} = 0{,}10
    \end{align*}
}
\solutionspace{180pt}

\tasknumber{6}%
\task{%
    Лампочки, сопротивления которых $R_1 = 6{,}00\,\text{Ом}$ и $R_2 = 54{,}00\,\text{Ом}$, поочерёдно подключённные к некоторому источнику тока,
    потребляют одинаковую мощность.
    Найти внутреннее сопротивление источника и КПД цепи в каждом случае.
}
\answer{%
    \begin{align*}
        P_1 &= \sqr{\frac{ \mathcal{E} }{R_1 + r}}R_1,
        P_2  = \sqr{\frac{ \mathcal{E} }{R_2 + r}}R_2,
        P_1 = P_2 \implies  \\
        &\implies R_1 \sqr{R_2 + r} = R_2 \sqr{R_1 + r} \implies  \\
        &\implies R_1 R_2^2 + 2 R_1 R_2 r + R_1 r^2 =
                    R_2 R_1^2 + 2 R_2 R_1 r + R_2 r^2  \implies  \\
    &\implies r^2 (R_2 - R_1) = R_2^2 R_2 - R_1^2 R_2 \implies  \\
    &\implies r
            = \sqrt{R_1 R_2 \frac{R_2 - R_1}{R_2 - R_1}}
            = \sqrt{R_1 R_2}
            = \sqrt{6{,}00\,\text{Ом} \cdot 54{,}00\,\text{Ом}}
            = 18{,}0\,\text{Ом}.
            \\
    \eta_1
            &= \frac{ R_1 }{R_1 + r}
            = \frac{\sqrt{ R_1 }}{\sqrt{ R_1 } + \sqrt{ R_2 }}
            = 0{,}250,  \\
    \eta_2
            &= \frac{ R_2 }{R_2 + r}
            = \frac{ \sqrt{ R_2 } }{\sqrt{ R_2 } + \sqrt{ R_1 }}
            = 0{,}750
    \end{align*}
}

\variantsplitter

\addpersonalvariant{Алёна Куприянова}

\tasknumber{1}%
\task{%
    Напротив физических величин укажите их обозначения и единицы измерения в СИ:
    \begin{enumerate}
        \item разность потенциалов,
        \item мощность тока,
        \item удельное сопротивление,
        \item внешнее сопротивление полной цепи.
    \end{enumerate}
}
\solutionspace{20pt}

\tasknumber{2}%
\task{%
    Запишите физический закон или формулу:
    \begin{enumerate}
        \item правило Кирхгофа для узла цепи,
        \item закон Ома для однородного участка цепи,
        \item закон Ома для неоднородного участка цепи.
    \end{enumerate}
}
\solutionspace{40pt}

\tasknumber{3}%
\task{%
    На резистор сопротивлением $R = 18\,\text{Ом}$ подали напряжение $V = 240\,\text{В}$.
    Определите ток, который потечёт через резистор, и мощность, выделяющуюся на нём.
}
\answer{%
    \begin{align*}
    \mathcal{I} &= \frac{ V }{ R } = \frac{ 240\,\text{В} }{ 18\,\text{Ом} } = 13{,}33\,\text{А},  \\
    P &= \frac{V^2}{ R } = \frac{ \sqr{ 240\,\text{В} } }{ 18\,\text{Ом} } = 3200{,}00\,\text{Вт}
    \end{align*}
}
\solutionspace{60pt}

\tasknumber{4}%
\task{%
    Через резистор сопротивлением $R = 18\,\text{Ом}$ протекает электрический ток $\mathcal{I} = 2{,}00\,\text{А}$.
    Определите, чему равны напряжение на резисторе и мощность, выделяющаяся на нём.
}
\answer{%
    \begin{align*}
    U &= \mathcal{I}R = 2{,}00\,\text{А} \cdot 18\,\text{Ом} = 36\,\text{В},  \\
    P &= \mathcal{I}^2R = \sqr{ 2{,}00\,\text{А} } \cdot 18\,\text{Ом} = 72\,\text{Вт}
    \end{align*}
}
\solutionspace{60pt}

\tasknumber{5}%
\task{%
    Замкнутая электрическая цепь состоит из ЭДС $\mathcal{E} = 4\,\text{В}$ и сопротивлением $r$
    и резистора $R = 15\,\text{Ом}$.
    Определите ток, протекающий в цепи.
    Какая тепловая энергия выделится на резисторе за время
    $\tau = 2\,\text{с}$? Какая работа будет совершена ЭДС за это время? Каков знак этой работы? Чему равен КПД цепи?
    Вычислите значения для 2 случаев: $r=0$ и $r = 60\,\text{Ом}$.
}
\answer{%
    \begin{align*}
    \mathcal{I}_1 &= \frac{ \mathcal{E} }{ R } = \frac{ 4\,\text{В} }{ 15\,\text{Ом} } = 0{,}27\,\text{А},  \\
    \mathcal{I}_2 &= \frac{ \mathcal{E} }{R + r} = \frac{ 4\,\text{В} }{15\,\text{Ом} + 60\,\text{Ом}} = 0{,}05\,\text{А},  \\
    Q_1 &= \mathcal{I}_1^2R\tau = \sqr{\frac{ \mathcal{E} }{ R }} R \tau
            = \sqr{\frac{ 4\,\text{В} }{ 15\,\text{Ом} }} \cdot 15\,\text{Ом} \cdot 2\,\text{с} = 2{,}187\,\text{Дж},  \\
    Q_2 &= \mathcal{I}_2^2R\tau = \sqr{\frac{ \mathcal{E} }{R + r}} R \tau
            = \sqr{\frac{ 4\,\text{В} }{15\,\text{Ом} + 60\,\text{Ом}}} \cdot 15\,\text{Ом} \cdot 2\,\text{с} = 0{,}075\,\text{Дж},  \\
    A_1 &= \mathcal{I}_1\tau\mathcal{E} = \frac{ \mathcal{E} }{R} \tau \mathcal{E}
            = \frac{\mathcal{E}^2 \tau}{ R } = \frac{\sqr{ 4\,\text{В} } \cdot 2\,\text{с}}{ 15\,\text{Ом} }
            = 2{,}160\,\text{Дж}, \text{положительна},  \\
    A_2 &= \mathcal{I}_2\tau\mathcal{E} = \frac{ \mathcal{E} }{R + r} \tau \mathcal{E}
            = \frac{\mathcal{E}^2 \tau}{R + r} = \frac{\sqr{ 4\,\text{В} } \cdot 2\,\text{с}}{15\,\text{Ом} + 60\,\text{Ом}}
            = 0{,}400\,\text{Дж}, \text{положительна},  \\
    \eta_1 &= \frac{ Q_1 }{ A_1 } = \ldots = \frac{ R }{ R } = 1,  \\
    \eta_2 &= \frac{ Q_2 }{ A_2 } = \ldots = \frac{ R }{R + r} = 0{,}19
    \end{align*}
}
\solutionspace{180pt}

\tasknumber{6}%
\task{%
    Лампочки, сопротивления которых $R_1 = 0{,}25\,\text{Ом}$ и $R_2 = 64{,}00\,\text{Ом}$, поочерёдно подключённные к некоторому источнику тока,
    потребляют одинаковую мощность.
    Найти внутреннее сопротивление источника и КПД цепи в каждом случае.
}
\answer{%
    \begin{align*}
        P_1 &= \sqr{\frac{ \mathcal{E} }{R_1 + r}}R_1,
        P_2  = \sqr{\frac{ \mathcal{E} }{R_2 + r}}R_2,
        P_1 = P_2 \implies  \\
        &\implies R_1 \sqr{R_2 + r} = R_2 \sqr{R_1 + r} \implies  \\
        &\implies R_1 R_2^2 + 2 R_1 R_2 r + R_1 r^2 =
                    R_2 R_1^2 + 2 R_2 R_1 r + R_2 r^2  \implies  \\
    &\implies r^2 (R_2 - R_1) = R_2^2 R_2 - R_1^2 R_2 \implies  \\
    &\implies r
            = \sqrt{R_1 R_2 \frac{R_2 - R_1}{R_2 - R_1}}
            = \sqrt{R_1 R_2}
            = \sqrt{0{,}25\,\text{Ом} \cdot 64{,}00\,\text{Ом}}
            = 4{,}0\,\text{Ом}.
            \\
    \eta_1
            &= \frac{ R_1 }{R_1 + r}
            = \frac{\sqrt{ R_1 }}{\sqrt{ R_1 } + \sqrt{ R_2 }}
            = 0{,}059,  \\
    \eta_2
            &= \frac{ R_2 }{R_2 + r}
            = \frac{ \sqrt{ R_2 } }{\sqrt{ R_2 } + \sqrt{ R_1 }}
            = 0{,}941
    \end{align*}
}

\variantsplitter

\addpersonalvariant{Ярослав Лавровский}

\tasknumber{1}%
\task{%
    Напротив физических величин укажите их обозначения и единицы измерения в СИ:
    \begin{enumerate}
        \item напряжение,
        \item мощность тока,
        \item удельное сопротивление,
        \item внешнее сопротивление полной цепи.
    \end{enumerate}
}
\solutionspace{20pt}

\tasknumber{2}%
\task{%
    Запишите физический закон или формулу:
    \begin{enumerate}
        \item правило Кирхгофа для замкнутого контура,
        \item сопротивление резистора через удельное сопротивление,
        \item ЭДС (определение).
    \end{enumerate}
}
\solutionspace{40pt}

\tasknumber{3}%
\task{%
    На резистор сопротивлением $r = 18\,\text{Ом}$ подали напряжение $U = 240\,\text{В}$.
    Определите ток, который потечёт через резистор, и мощность, выделяющуюся на нём.
}
\answer{%
    \begin{align*}
    \mathcal{I} &= \frac{ U }{ r } = \frac{ 240\,\text{В} }{ 18\,\text{Ом} } = 13{,}33\,\text{А},  \\
    P &= \frac{U^2}{ r } = \frac{ \sqr{ 240\,\text{В} } }{ 18\,\text{Ом} } = 3200{,}00\,\text{Вт}
    \end{align*}
}
\solutionspace{60pt}

\tasknumber{4}%
\task{%
    Через резистор сопротивлением $r = 18\,\text{Ом}$ протекает электрический ток $\mathcal{I} = 2{,}00\,\text{А}$.
    Определите, чему равны напряжение на резисторе и мощность, выделяющаяся на нём.
}
\answer{%
    \begin{align*}
    U &= \mathcal{I}r = 2{,}00\,\text{А} \cdot 18\,\text{Ом} = 36\,\text{В},  \\
    P &= \mathcal{I}^2r = \sqr{ 2{,}00\,\text{А} } \cdot 18\,\text{Ом} = 72\,\text{Вт}
    \end{align*}
}
\solutionspace{60pt}

\tasknumber{5}%
\task{%
    Замкнутая электрическая цепь состоит из ЭДС $\mathcal{E} = 1\,\text{В}$ и сопротивлением $r$
    и резистора $R = 24\,\text{Ом}$.
    Определите ток, протекающий в цепи.
    Какая тепловая энергия выделится на резисторе за время
    $\tau = 2\,\text{с}$? Какая работа будет совершена ЭДС за это время? Каков знак этой работы? Чему равен КПД цепи?
    Вычислите значения для 2 случаев: $r=0$ и $r = 30\,\text{Ом}$.
}
\answer{%
    \begin{align*}
    \mathcal{I}_1 &= \frac{ \mathcal{E} }{ R } = \frac{ 1\,\text{В} }{ 24\,\text{Ом} } = 0{,}04\,\text{А},  \\
    \mathcal{I}_2 &= \frac{ \mathcal{E} }{R + r} = \frac{ 1\,\text{В} }{24\,\text{Ом} + 30\,\text{Ом}} = 0{,}02\,\text{А},  \\
    Q_1 &= \mathcal{I}_1^2R\tau = \sqr{\frac{ \mathcal{E} }{ R }} R \tau
            = \sqr{\frac{ 1\,\text{В} }{ 24\,\text{Ом} }} \cdot 24\,\text{Ом} \cdot 2\,\text{с} = 0{,}077\,\text{Дж},  \\
    Q_2 &= \mathcal{I}_2^2R\tau = \sqr{\frac{ \mathcal{E} }{R + r}} R \tau
            = \sqr{\frac{ 1\,\text{В} }{24\,\text{Ом} + 30\,\text{Ом}}} \cdot 24\,\text{Ом} \cdot 2\,\text{с} = 0{,}019\,\text{Дж},  \\
    A_1 &= \mathcal{I}_1\tau\mathcal{E} = \frac{ \mathcal{E} }{R} \tau \mathcal{E}
            = \frac{\mathcal{E}^2 \tau}{ R } = \frac{\sqr{ 1\,\text{В} } \cdot 2\,\text{с}}{ 24\,\text{Ом} }
            = 0{,}080\,\text{Дж}, \text{положительна},  \\
    A_2 &= \mathcal{I}_2\tau\mathcal{E} = \frac{ \mathcal{E} }{R + r} \tau \mathcal{E}
            = \frac{\mathcal{E}^2 \tau}{R + r} = \frac{\sqr{ 1\,\text{В} } \cdot 2\,\text{с}}{24\,\text{Ом} + 30\,\text{Ом}}
            = 0{,}040\,\text{Дж}, \text{положительна},  \\
    \eta_1 &= \frac{ Q_1 }{ A_1 } = \ldots = \frac{ R }{ R } = 1,  \\
    \eta_2 &= \frac{ Q_2 }{ A_2 } = \ldots = \frac{ R }{R + r} = 0{,}47
    \end{align*}
}
\solutionspace{180pt}

\tasknumber{6}%
\task{%
    Лампочки, сопротивления которых $R_1 = 5{,}00\,\text{Ом}$ и $R_2 = 80{,}00\,\text{Ом}$, поочерёдно подключённные к некоторому источнику тока,
    потребляют одинаковую мощность.
    Найти внутреннее сопротивление источника и КПД цепи в каждом случае.
}
\answer{%
    \begin{align*}
        P_1 &= \sqr{\frac{ \mathcal{E} }{R_1 + r}}R_1,
        P_2  = \sqr{\frac{ \mathcal{E} }{R_2 + r}}R_2,
        P_1 = P_2 \implies  \\
        &\implies R_1 \sqr{R_2 + r} = R_2 \sqr{R_1 + r} \implies  \\
        &\implies R_1 R_2^2 + 2 R_1 R_2 r + R_1 r^2 =
                    R_2 R_1^2 + 2 R_2 R_1 r + R_2 r^2  \implies  \\
    &\implies r^2 (R_2 - R_1) = R_2^2 R_2 - R_1^2 R_2 \implies  \\
    &\implies r
            = \sqrt{R_1 R_2 \frac{R_2 - R_1}{R_2 - R_1}}
            = \sqrt{R_1 R_2}
            = \sqrt{5{,}00\,\text{Ом} \cdot 80{,}00\,\text{Ом}}
            = 20{,}0\,\text{Ом}.
            \\
    \eta_1
            &= \frac{ R_1 }{R_1 + r}
            = \frac{\sqrt{ R_1 }}{\sqrt{ R_1 } + \sqrt{ R_2 }}
            = 0{,}200,  \\
    \eta_2
            &= \frac{ R_2 }{R_2 + r}
            = \frac{ \sqrt{ R_2 } }{\sqrt{ R_2 } + \sqrt{ R_1 }}
            = 0{,}800
    \end{align*}
}

\variantsplitter

\addpersonalvariant{Анастасия Ламанова}

\tasknumber{1}%
\task{%
    Напротив физических величин укажите их обозначения и единицы измерения в СИ:
    \begin{enumerate}
        \item разность потенциалов,
        \item мощность тока,
        \item ЭДС,
        \item внутреннее сопротивление полной цепи.
    \end{enumerate}
}
\solutionspace{20pt}

\tasknumber{2}%
\task{%
    Запишите физический закон или формулу:
    \begin{enumerate}
        \item правило Кирхгофа для замкнутого контура,
        \item сопротивление резистора через удельное сопротивление,
        \item ЭДС (определение).
    \end{enumerate}
}
\solutionspace{40pt}

\tasknumber{3}%
\task{%
    На резистор сопротивлением $r = 18\,\text{Ом}$ подали напряжение $V = 150\,\text{В}$.
    Определите ток, который потечёт через резистор, и мощность, выделяющуюся на нём.
}
\answer{%
    \begin{align*}
    \mathcal{I} &= \frac{ V }{ r } = \frac{ 150\,\text{В} }{ 18\,\text{Ом} } = 8{,}33\,\text{А},  \\
    P &= \frac{V^2}{ r } = \frac{ \sqr{ 150\,\text{В} } }{ 18\,\text{Ом} } = 1250{,}00\,\text{Вт}
    \end{align*}
}
\solutionspace{60pt}

\tasknumber{4}%
\task{%
    Через резистор сопротивлением $r = 12\,\text{Ом}$ протекает электрический ток $\mathcal{I} = 3{,}00\,\text{А}$.
    Определите, чему равны напряжение на резисторе и мощность, выделяющаяся на нём.
}
\answer{%
    \begin{align*}
    U &= \mathcal{I}r = 3{,}00\,\text{А} \cdot 12\,\text{Ом} = 36\,\text{В},  \\
    P &= \mathcal{I}^2r = \sqr{ 3{,}00\,\text{А} } \cdot 12\,\text{Ом} = 108\,\text{Вт}
    \end{align*}
}
\solutionspace{60pt}

\tasknumber{5}%
\task{%
    Замкнутая электрическая цепь состоит из ЭДС $\mathcal{E} = 3\,\text{В}$ и сопротивлением $r$
    и резистора $R = 15\,\text{Ом}$.
    Определите ток, протекающий в цепи.
    Какая тепловая энергия выделится на резисторе за время
    $\tau = 10\,\text{с}$? Какая работа будет совершена ЭДС за это время? Каков знак этой работы? Чему равен КПД цепи?
    Вычислите значения для 2 случаев: $r=0$ и $r = 60\,\text{Ом}$.
}
\answer{%
    \begin{align*}
    \mathcal{I}_1 &= \frac{ \mathcal{E} }{ R } = \frac{ 3\,\text{В} }{ 15\,\text{Ом} } = 0{,}20\,\text{А},  \\
    \mathcal{I}_2 &= \frac{ \mathcal{E} }{R + r} = \frac{ 3\,\text{В} }{15\,\text{Ом} + 60\,\text{Ом}} = 0{,}04\,\text{А},  \\
    Q_1 &= \mathcal{I}_1^2R\tau = \sqr{\frac{ \mathcal{E} }{ R }} R \tau
            = \sqr{\frac{ 3\,\text{В} }{ 15\,\text{Ом} }} \cdot 15\,\text{Ом} \cdot 10\,\text{с} = 6{,}000\,\text{Дж},  \\
    Q_2 &= \mathcal{I}_2^2R\tau = \sqr{\frac{ \mathcal{E} }{R + r}} R \tau
            = \sqr{\frac{ 3\,\text{В} }{15\,\text{Ом} + 60\,\text{Ом}}} \cdot 15\,\text{Ом} \cdot 10\,\text{с} = 0{,}240\,\text{Дж},  \\
    A_1 &= \mathcal{I}_1\tau\mathcal{E} = \frac{ \mathcal{E} }{R} \tau \mathcal{E}
            = \frac{\mathcal{E}^2 \tau}{ R } = \frac{\sqr{ 3\,\text{В} } \cdot 10\,\text{с}}{ 15\,\text{Ом} }
            = 6{,}000\,\text{Дж}, \text{положительна},  \\
    A_2 &= \mathcal{I}_2\tau\mathcal{E} = \frac{ \mathcal{E} }{R + r} \tau \mathcal{E}
            = \frac{\mathcal{E}^2 \tau}{R + r} = \frac{\sqr{ 3\,\text{В} } \cdot 10\,\text{с}}{15\,\text{Ом} + 60\,\text{Ом}}
            = 1{,}200\,\text{Дж}, \text{положительна},  \\
    \eta_1 &= \frac{ Q_1 }{ A_1 } = \ldots = \frac{ R }{ R } = 1,  \\
    \eta_2 &= \frac{ Q_2 }{ A_2 } = \ldots = \frac{ R }{R + r} = 0{,}20
    \end{align*}
}
\solutionspace{180pt}

\tasknumber{6}%
\task{%
    Лампочки, сопротивления которых $R_1 = 5{,}00\,\text{Ом}$ и $R_2 = 80{,}00\,\text{Ом}$, поочерёдно подключённные к некоторому источнику тока,
    потребляют одинаковую мощность.
    Найти внутреннее сопротивление источника и КПД цепи в каждом случае.
}
\answer{%
    \begin{align*}
        P_1 &= \sqr{\frac{ \mathcal{E} }{R_1 + r}}R_1,
        P_2  = \sqr{\frac{ \mathcal{E} }{R_2 + r}}R_2,
        P_1 = P_2 \implies  \\
        &\implies R_1 \sqr{R_2 + r} = R_2 \sqr{R_1 + r} \implies  \\
        &\implies R_1 R_2^2 + 2 R_1 R_2 r + R_1 r^2 =
                    R_2 R_1^2 + 2 R_2 R_1 r + R_2 r^2  \implies  \\
    &\implies r^2 (R_2 - R_1) = R_2^2 R_2 - R_1^2 R_2 \implies  \\
    &\implies r
            = \sqrt{R_1 R_2 \frac{R_2 - R_1}{R_2 - R_1}}
            = \sqrt{R_1 R_2}
            = \sqrt{5{,}00\,\text{Ом} \cdot 80{,}00\,\text{Ом}}
            = 20{,}0\,\text{Ом}.
            \\
    \eta_1
            &= \frac{ R_1 }{R_1 + r}
            = \frac{\sqrt{ R_1 }}{\sqrt{ R_1 } + \sqrt{ R_2 }}
            = 0{,}200,  \\
    \eta_2
            &= \frac{ R_2 }{R_2 + r}
            = \frac{ \sqrt{ R_2 } }{\sqrt{ R_2 } + \sqrt{ R_1 }}
            = 0{,}800
    \end{align*}
}

\variantsplitter

\addpersonalvariant{Виктория Легонькова}

\tasknumber{1}%
\task{%
    Напротив физических величин укажите их обозначения и единицы измерения в СИ:
    \begin{enumerate}
        \item разность потенциалов,
        \item работа тока,
        \item ЭДС,
        \item внешнее сопротивление полной цепи.
    \end{enumerate}
}
\solutionspace{20pt}

\tasknumber{2}%
\task{%
    Запишите физический закон или формулу:
    \begin{enumerate}
        \item правило Кирхгофа для узла цепи,
        \item закон Ома для однородного участка цепи,
        \item закон Ома для неоднородного участка цепи.
    \end{enumerate}
}
\solutionspace{40pt}

\tasknumber{3}%
\task{%
    На резистор сопротивлением $R = 5\,\text{Ом}$ подали напряжение $U = 240\,\text{В}$.
    Определите ток, который потечёт через резистор, и мощность, выделяющуюся на нём.
}
\answer{%
    \begin{align*}
    \mathcal{I} &= \frac{ U }{ R } = \frac{ 240\,\text{В} }{ 5\,\text{Ом} } = 48{,}00\,\text{А},  \\
    P &= \frac{U^2}{ R } = \frac{ \sqr{ 240\,\text{В} } }{ 5\,\text{Ом} } = 11520{,}00\,\text{Вт}
    \end{align*}
}
\solutionspace{60pt}

\tasknumber{4}%
\task{%
    Через резистор сопротивлением $R = 5\,\text{Ом}$ протекает электрический ток $\mathcal{I} = 10{,}00\,\text{А}$.
    Определите, чему равны напряжение на резисторе и мощность, выделяющаяся на нём.
}
\answer{%
    \begin{align*}
    U &= \mathcal{I}R = 10{,}00\,\text{А} \cdot 5\,\text{Ом} = 50\,\text{В},  \\
    P &= \mathcal{I}^2R = \sqr{ 10{,}00\,\text{А} } \cdot 5\,\text{Ом} = 500\,\text{Вт}
    \end{align*}
}
\solutionspace{60pt}

\tasknumber{5}%
\task{%
    Замкнутая электрическая цепь состоит из ЭДС $\mathcal{E} = 2\,\text{В}$ и сопротивлением $r$
    и резистора $R = 24\,\text{Ом}$.
    Определите ток, протекающий в цепи.
    Какая тепловая энергия выделится на резисторе за время
    $\tau = 10\,\text{с}$? Какая работа будет совершена ЭДС за это время? Каков знак этой работы? Чему равен КПД цепи?
    Вычислите значения для 2 случаев: $r=0$ и $r = 60\,\text{Ом}$.
}
\answer{%
    \begin{align*}
    \mathcal{I}_1 &= \frac{ \mathcal{E} }{ R } = \frac{ 2\,\text{В} }{ 24\,\text{Ом} } = 0{,}08\,\text{А},  \\
    \mathcal{I}_2 &= \frac{ \mathcal{E} }{R + r} = \frac{ 2\,\text{В} }{24\,\text{Ом} + 60\,\text{Ом}} = 0{,}02\,\text{А},  \\
    Q_1 &= \mathcal{I}_1^2R\tau = \sqr{\frac{ \mathcal{E} }{ R }} R \tau
            = \sqr{\frac{ 2\,\text{В} }{ 24\,\text{Ом} }} \cdot 24\,\text{Ом} \cdot 10\,\text{с} = 1{,}536\,\text{Дж},  \\
    Q_2 &= \mathcal{I}_2^2R\tau = \sqr{\frac{ \mathcal{E} }{R + r}} R \tau
            = \sqr{\frac{ 2\,\text{В} }{24\,\text{Ом} + 60\,\text{Ом}}} \cdot 24\,\text{Ом} \cdot 10\,\text{с} = 0{,}096\,\text{Дж},  \\
    A_1 &= \mathcal{I}_1\tau\mathcal{E} = \frac{ \mathcal{E} }{R} \tau \mathcal{E}
            = \frac{\mathcal{E}^2 \tau}{ R } = \frac{\sqr{ 2\,\text{В} } \cdot 10\,\text{с}}{ 24\,\text{Ом} }
            = 1{,}600\,\text{Дж}, \text{положительна},  \\
    A_2 &= \mathcal{I}_2\tau\mathcal{E} = \frac{ \mathcal{E} }{R + r} \tau \mathcal{E}
            = \frac{\mathcal{E}^2 \tau}{R + r} = \frac{\sqr{ 2\,\text{В} } \cdot 10\,\text{с}}{24\,\text{Ом} + 60\,\text{Ом}}
            = 0{,}400\,\text{Дж}, \text{положительна},  \\
    \eta_1 &= \frac{ Q_1 }{ A_1 } = \ldots = \frac{ R }{ R } = 1,  \\
    \eta_2 &= \frac{ Q_2 }{ A_2 } = \ldots = \frac{ R }{R + r} = 0{,}24
    \end{align*}
}
\solutionspace{180pt}

\tasknumber{6}%
\task{%
    Лампочки, сопротивления которых $R_1 = 1{,}00\,\text{Ом}$ и $R_2 = 9{,}00\,\text{Ом}$, поочерёдно подключённные к некоторому источнику тока,
    потребляют одинаковую мощность.
    Найти внутреннее сопротивление источника и КПД цепи в каждом случае.
}
\answer{%
    \begin{align*}
        P_1 &= \sqr{\frac{ \mathcal{E} }{R_1 + r}}R_1,
        P_2  = \sqr{\frac{ \mathcal{E} }{R_2 + r}}R_2,
        P_1 = P_2 \implies  \\
        &\implies R_1 \sqr{R_2 + r} = R_2 \sqr{R_1 + r} \implies  \\
        &\implies R_1 R_2^2 + 2 R_1 R_2 r + R_1 r^2 =
                    R_2 R_1^2 + 2 R_2 R_1 r + R_2 r^2  \implies  \\
    &\implies r^2 (R_2 - R_1) = R_2^2 R_2 - R_1^2 R_2 \implies  \\
    &\implies r
            = \sqrt{R_1 R_2 \frac{R_2 - R_1}{R_2 - R_1}}
            = \sqrt{R_1 R_2}
            = \sqrt{1{,}00\,\text{Ом} \cdot 9{,}00\,\text{Ом}}
            = 3{,}0\,\text{Ом}.
            \\
    \eta_1
            &= \frac{ R_1 }{R_1 + r}
            = \frac{\sqrt{ R_1 }}{\sqrt{ R_1 } + \sqrt{ R_2 }}
            = 0{,}250,  \\
    \eta_2
            &= \frac{ R_2 }{R_2 + r}
            = \frac{ \sqrt{ R_2 } }{\sqrt{ R_2 } + \sqrt{ R_1 }}
            = 0{,}750
    \end{align*}
}

\variantsplitter

\addpersonalvariant{Семён Мартынов}

\tasknumber{1}%
\task{%
    Напротив физических величин укажите их обозначения и единицы измерения в СИ:
    \begin{enumerate}
        \item сила тока,
        \item мощность тока,
        \item удельное сопротивление,
        \item внешнее сопротивление полной цепи.
    \end{enumerate}
}
\solutionspace{20pt}

\tasknumber{2}%
\task{%
    Запишите физический закон или формулу:
    \begin{enumerate}
        \item правило Кирхгофа для узла цепи,
        \item сопротивление резистора через удельное сопротивление,
        \item закон Ома для неоднородного участка цепи.
    \end{enumerate}
}
\solutionspace{40pt}

\tasknumber{3}%
\task{%
    На резистор сопротивлением $R = 30\,\text{Ом}$ подали напряжение $U = 180\,\text{В}$.
    Определите ток, который потечёт через резистор, и мощность, выделяющуюся на нём.
}
\answer{%
    \begin{align*}
    \mathcal{I} &= \frac{ U }{ R } = \frac{ 180\,\text{В} }{ 30\,\text{Ом} } = 6{,}00\,\text{А},  \\
    P &= \frac{U^2}{ R } = \frac{ \sqr{ 180\,\text{В} } }{ 30\,\text{Ом} } = 1080{,}00\,\text{Вт}
    \end{align*}
}
\solutionspace{60pt}

\tasknumber{4}%
\task{%
    Через резистор сопротивлением $R = 18\,\text{Ом}$ протекает электрический ток $\mathcal{I} = 15{,}00\,\text{А}$.
    Определите, чему равны напряжение на резисторе и мощность, выделяющаяся на нём.
}
\answer{%
    \begin{align*}
    U &= \mathcal{I}R = 15{,}00\,\text{А} \cdot 18\,\text{Ом} = 270\,\text{В},  \\
    P &= \mathcal{I}^2R = \sqr{ 15{,}00\,\text{А} } \cdot 18\,\text{Ом} = 4050\,\text{Вт}
    \end{align*}
}
\solutionspace{60pt}

\tasknumber{5}%
\task{%
    Замкнутая электрическая цепь состоит из ЭДС $\mathcal{E} = 2\,\text{В}$ и сопротивлением $r$
    и резистора $R = 10\,\text{Ом}$.
    Определите ток, протекающий в цепи.
    Какая тепловая энергия выделится на резисторе за время
    $\tau = 2\,\text{с}$? Какая работа будет совершена ЭДС за это время? Каков знак этой работы? Чему равен КПД цепи?
    Вычислите значения для 2 случаев: $r=0$ и $r = 60\,\text{Ом}$.
}
\answer{%
    \begin{align*}
    \mathcal{I}_1 &= \frac{ \mathcal{E} }{ R } = \frac{ 2\,\text{В} }{ 10\,\text{Ом} } = 0{,}20\,\text{А},  \\
    \mathcal{I}_2 &= \frac{ \mathcal{E} }{R + r} = \frac{ 2\,\text{В} }{10\,\text{Ом} + 60\,\text{Ом}} = 0{,}03\,\text{А},  \\
    Q_1 &= \mathcal{I}_1^2R\tau = \sqr{\frac{ \mathcal{E} }{ R }} R \tau
            = \sqr{\frac{ 2\,\text{В} }{ 10\,\text{Ом} }} \cdot 10\,\text{Ом} \cdot 2\,\text{с} = 0{,}800\,\text{Дж},  \\
    Q_2 &= \mathcal{I}_2^2R\tau = \sqr{\frac{ \mathcal{E} }{R + r}} R \tau
            = \sqr{\frac{ 2\,\text{В} }{10\,\text{Ом} + 60\,\text{Ом}}} \cdot 10\,\text{Ом} \cdot 2\,\text{с} = 0{,}018\,\text{Дж},  \\
    A_1 &= \mathcal{I}_1\tau\mathcal{E} = \frac{ \mathcal{E} }{R} \tau \mathcal{E}
            = \frac{\mathcal{E}^2 \tau}{ R } = \frac{\sqr{ 2\,\text{В} } \cdot 2\,\text{с}}{ 10\,\text{Ом} }
            = 0{,}800\,\text{Дж}, \text{положительна},  \\
    A_2 &= \mathcal{I}_2\tau\mathcal{E} = \frac{ \mathcal{E} }{R + r} \tau \mathcal{E}
            = \frac{\mathcal{E}^2 \tau}{R + r} = \frac{\sqr{ 2\,\text{В} } \cdot 2\,\text{с}}{10\,\text{Ом} + 60\,\text{Ом}}
            = 0{,}120\,\text{Дж}, \text{положительна},  \\
    \eta_1 &= \frac{ Q_1 }{ A_1 } = \ldots = \frac{ R }{ R } = 1,  \\
    \eta_2 &= \frac{ Q_2 }{ A_2 } = \ldots = \frac{ R }{R + r} = 0{,}15
    \end{align*}
}
\solutionspace{180pt}

\tasknumber{6}%
\task{%
    Лампочки, сопротивления которых $R_1 = 0{,}50\,\text{Ом}$ и $R_2 = 18{,}00\,\text{Ом}$, поочерёдно подключённные к некоторому источнику тока,
    потребляют одинаковую мощность.
    Найти внутреннее сопротивление источника и КПД цепи в каждом случае.
}
\answer{%
    \begin{align*}
        P_1 &= \sqr{\frac{ \mathcal{E} }{R_1 + r}}R_1,
        P_2  = \sqr{\frac{ \mathcal{E} }{R_2 + r}}R_2,
        P_1 = P_2 \implies  \\
        &\implies R_1 \sqr{R_2 + r} = R_2 \sqr{R_1 + r} \implies  \\
        &\implies R_1 R_2^2 + 2 R_1 R_2 r + R_1 r^2 =
                    R_2 R_1^2 + 2 R_2 R_1 r + R_2 r^2  \implies  \\
    &\implies r^2 (R_2 - R_1) = R_2^2 R_2 - R_1^2 R_2 \implies  \\
    &\implies r
            = \sqrt{R_1 R_2 \frac{R_2 - R_1}{R_2 - R_1}}
            = \sqrt{R_1 R_2}
            = \sqrt{0{,}50\,\text{Ом} \cdot 18{,}00\,\text{Ом}}
            = 3{,}0\,\text{Ом}.
            \\
    \eta_1
            &= \frac{ R_1 }{R_1 + r}
            = \frac{\sqrt{ R_1 }}{\sqrt{ R_1 } + \sqrt{ R_2 }}
            = 0{,}143,  \\
    \eta_2
            &= \frac{ R_2 }{R_2 + r}
            = \frac{ \sqrt{ R_2 } }{\sqrt{ R_2 } + \sqrt{ R_1 }}
            = 0{,}857
    \end{align*}
}

\variantsplitter

\addpersonalvariant{Варвара Минаева}

\tasknumber{1}%
\task{%
    Напротив физических величин укажите их обозначения и единицы измерения в СИ:
    \begin{enumerate}
        \item напряжение,
        \item работа тока,
        \item удельное сопротивление,
        \item внутреннее сопротивление полной цепи.
    \end{enumerate}
}
\solutionspace{20pt}

\tasknumber{2}%
\task{%
    Запишите физический закон или формулу:
    \begin{enumerate}
        \item правило Кирхгофа для замкнутого контура,
        \item закон Ома для однородного участка цепи,
        \item закон Ома для неоднородного участка цепи.
    \end{enumerate}
}
\solutionspace{40pt}

\tasknumber{3}%
\task{%
    На резистор сопротивлением $r = 5\,\text{Ом}$ подали напряжение $V = 180\,\text{В}$.
    Определите ток, который потечёт через резистор, и мощность, выделяющуюся на нём.
}
\answer{%
    \begin{align*}
    \mathcal{I} &= \frac{ V }{ r } = \frac{ 180\,\text{В} }{ 5\,\text{Ом} } = 36{,}00\,\text{А},  \\
    P &= \frac{V^2}{ r } = \frac{ \sqr{ 180\,\text{В} } }{ 5\,\text{Ом} } = 6480{,}00\,\text{Вт}
    \end{align*}
}
\solutionspace{60pt}

\tasknumber{4}%
\task{%
    Через резистор сопротивлением $r = 30\,\text{Ом}$ протекает электрический ток $\mathcal{I} = 6{,}00\,\text{А}$.
    Определите, чему равны напряжение на резисторе и мощность, выделяющаяся на нём.
}
\answer{%
    \begin{align*}
    U &= \mathcal{I}r = 6{,}00\,\text{А} \cdot 30\,\text{Ом} = 180\,\text{В},  \\
    P &= \mathcal{I}^2r = \sqr{ 6{,}00\,\text{А} } \cdot 30\,\text{Ом} = 1080\,\text{Вт}
    \end{align*}
}
\solutionspace{60pt}

\tasknumber{5}%
\task{%
    Замкнутая электрическая цепь состоит из ЭДС $\mathcal{E} = 1\,\text{В}$ и сопротивлением $r$
    и резистора $R = 10\,\text{Ом}$.
    Определите ток, протекающий в цепи.
    Какая тепловая энергия выделится на резисторе за время
    $\tau = 5\,\text{с}$? Какая работа будет совершена ЭДС за это время? Каков знак этой работы? Чему равен КПД цепи?
    Вычислите значения для 2 случаев: $r=0$ и $r = 20\,\text{Ом}$.
}
\answer{%
    \begin{align*}
    \mathcal{I}_1 &= \frac{ \mathcal{E} }{ R } = \frac{ 1\,\text{В} }{ 10\,\text{Ом} } = 0{,}10\,\text{А},  \\
    \mathcal{I}_2 &= \frac{ \mathcal{E} }{R + r} = \frac{ 1\,\text{В} }{10\,\text{Ом} + 20\,\text{Ом}} = 0{,}03\,\text{А},  \\
    Q_1 &= \mathcal{I}_1^2R\tau = \sqr{\frac{ \mathcal{E} }{ R }} R \tau
            = \sqr{\frac{ 1\,\text{В} }{ 10\,\text{Ом} }} \cdot 10\,\text{Ом} \cdot 5\,\text{с} = 0{,}500\,\text{Дж},  \\
    Q_2 &= \mathcal{I}_2^2R\tau = \sqr{\frac{ \mathcal{E} }{R + r}} R \tau
            = \sqr{\frac{ 1\,\text{В} }{10\,\text{Ом} + 20\,\text{Ом}}} \cdot 10\,\text{Ом} \cdot 5\,\text{с} = 0{,}045\,\text{Дж},  \\
    A_1 &= \mathcal{I}_1\tau\mathcal{E} = \frac{ \mathcal{E} }{R} \tau \mathcal{E}
            = \frac{\mathcal{E}^2 \tau}{ R } = \frac{\sqr{ 1\,\text{В} } \cdot 5\,\text{с}}{ 10\,\text{Ом} }
            = 0{,}500\,\text{Дж}, \text{положительна},  \\
    A_2 &= \mathcal{I}_2\tau\mathcal{E} = \frac{ \mathcal{E} }{R + r} \tau \mathcal{E}
            = \frac{\mathcal{E}^2 \tau}{R + r} = \frac{\sqr{ 1\,\text{В} } \cdot 5\,\text{с}}{10\,\text{Ом} + 20\,\text{Ом}}
            = 0{,}150\,\text{Дж}, \text{положительна},  \\
    \eta_1 &= \frac{ Q_1 }{ A_1 } = \ldots = \frac{ R }{ R } = 1,  \\
    \eta_2 &= \frac{ Q_2 }{ A_2 } = \ldots = \frac{ R }{R + r} = 0{,}30
    \end{align*}
}
\solutionspace{180pt}

\tasknumber{6}%
\task{%
    Лампочки, сопротивления которых $R_1 = 0{,}25\,\text{Ом}$ и $R_2 = 4{,}00\,\text{Ом}$, поочерёдно подключённные к некоторому источнику тока,
    потребляют одинаковую мощность.
    Найти внутреннее сопротивление источника и КПД цепи в каждом случае.
}
\answer{%
    \begin{align*}
        P_1 &= \sqr{\frac{ \mathcal{E} }{R_1 + r}}R_1,
        P_2  = \sqr{\frac{ \mathcal{E} }{R_2 + r}}R_2,
        P_1 = P_2 \implies  \\
        &\implies R_1 \sqr{R_2 + r} = R_2 \sqr{R_1 + r} \implies  \\
        &\implies R_1 R_2^2 + 2 R_1 R_2 r + R_1 r^2 =
                    R_2 R_1^2 + 2 R_2 R_1 r + R_2 r^2  \implies  \\
    &\implies r^2 (R_2 - R_1) = R_2^2 R_2 - R_1^2 R_2 \implies  \\
    &\implies r
            = \sqrt{R_1 R_2 \frac{R_2 - R_1}{R_2 - R_1}}
            = \sqrt{R_1 R_2}
            = \sqrt{0{,}25\,\text{Ом} \cdot 4{,}00\,\text{Ом}}
            = 1{,}0\,\text{Ом}.
            \\
    \eta_1
            &= \frac{ R_1 }{R_1 + r}
            = \frac{\sqrt{ R_1 }}{\sqrt{ R_1 } + \sqrt{ R_2 }}
            = 0{,}200,  \\
    \eta_2
            &= \frac{ R_2 }{R_2 + r}
            = \frac{ \sqrt{ R_2 } }{\sqrt{ R_2 } + \sqrt{ R_1 }}
            = 0{,}800
    \end{align*}
}

\variantsplitter

\addpersonalvariant{Леонид Никитин}

\tasknumber{1}%
\task{%
    Напротив физических величин укажите их обозначения и единицы измерения в СИ:
    \begin{enumerate}
        \item разность потенциалов,
        \item мощность тока,
        \item удельное сопротивление,
        \item внешнее сопротивление полной цепи.
    \end{enumerate}
}
\solutionspace{20pt}

\tasknumber{2}%
\task{%
    Запишите физический закон или формулу:
    \begin{enumerate}
        \item правило Кирхгофа для узла цепи,
        \item закон Ома для однородного участка цепи,
        \item ЭДС (определение).
    \end{enumerate}
}
\solutionspace{40pt}

\tasknumber{3}%
\task{%
    На резистор сопротивлением $r = 12\,\text{Ом}$ подали напряжение $U = 240\,\text{В}$.
    Определите ток, который потечёт через резистор, и мощность, выделяющуюся на нём.
}
\answer{%
    \begin{align*}
    \mathcal{I} &= \frac{ U }{ r } = \frac{ 240\,\text{В} }{ 12\,\text{Ом} } = 20{,}00\,\text{А},  \\
    P &= \frac{U^2}{ r } = \frac{ \sqr{ 240\,\text{В} } }{ 12\,\text{Ом} } = 4800{,}00\,\text{Вт}
    \end{align*}
}
\solutionspace{60pt}

\tasknumber{4}%
\task{%
    Через резистор сопротивлением $r = 30\,\text{Ом}$ протекает электрический ток $\mathcal{I} = 6{,}00\,\text{А}$.
    Определите, чему равны напряжение на резисторе и мощность, выделяющаяся на нём.
}
\answer{%
    \begin{align*}
    U &= \mathcal{I}r = 6{,}00\,\text{А} \cdot 30\,\text{Ом} = 180\,\text{В},  \\
    P &= \mathcal{I}^2r = \sqr{ 6{,}00\,\text{А} } \cdot 30\,\text{Ом} = 1080\,\text{Вт}
    \end{align*}
}
\solutionspace{60pt}

\tasknumber{5}%
\task{%
    Замкнутая электрическая цепь состоит из ЭДС $\mathcal{E} = 4\,\text{В}$ и сопротивлением $r$
    и резистора $R = 15\,\text{Ом}$.
    Определите ток, протекающий в цепи.
    Какая тепловая энергия выделится на резисторе за время
    $\tau = 5\,\text{с}$? Какая работа будет совершена ЭДС за это время? Каков знак этой работы? Чему равен КПД цепи?
    Вычислите значения для 2 случаев: $r=0$ и $r = 10\,\text{Ом}$.
}
\answer{%
    \begin{align*}
    \mathcal{I}_1 &= \frac{ \mathcal{E} }{ R } = \frac{ 4\,\text{В} }{ 15\,\text{Ом} } = 0{,}27\,\text{А},  \\
    \mathcal{I}_2 &= \frac{ \mathcal{E} }{R + r} = \frac{ 4\,\text{В} }{15\,\text{Ом} + 10\,\text{Ом}} = 0{,}16\,\text{А},  \\
    Q_1 &= \mathcal{I}_1^2R\tau = \sqr{\frac{ \mathcal{E} }{ R }} R \tau
            = \sqr{\frac{ 4\,\text{В} }{ 15\,\text{Ом} }} \cdot 15\,\text{Ом} \cdot 5\,\text{с} = 5{,}468\,\text{Дж},  \\
    Q_2 &= \mathcal{I}_2^2R\tau = \sqr{\frac{ \mathcal{E} }{R + r}} R \tau
            = \sqr{\frac{ 4\,\text{В} }{15\,\text{Ом} + 10\,\text{Ом}}} \cdot 15\,\text{Ом} \cdot 5\,\text{с} = 1{,}920\,\text{Дж},  \\
    A_1 &= \mathcal{I}_1\tau\mathcal{E} = \frac{ \mathcal{E} }{R} \tau \mathcal{E}
            = \frac{\mathcal{E}^2 \tau}{ R } = \frac{\sqr{ 4\,\text{В} } \cdot 5\,\text{с}}{ 15\,\text{Ом} }
            = 5{,}400\,\text{Дж}, \text{положительна},  \\
    A_2 &= \mathcal{I}_2\tau\mathcal{E} = \frac{ \mathcal{E} }{R + r} \tau \mathcal{E}
            = \frac{\mathcal{E}^2 \tau}{R + r} = \frac{\sqr{ 4\,\text{В} } \cdot 5\,\text{с}}{15\,\text{Ом} + 10\,\text{Ом}}
            = 3{,}200\,\text{Дж}, \text{положительна},  \\
    \eta_1 &= \frac{ Q_1 }{ A_1 } = \ldots = \frac{ R }{ R } = 1,  \\
    \eta_2 &= \frac{ Q_2 }{ A_2 } = \ldots = \frac{ R }{R + r} = 0{,}60
    \end{align*}
}
\solutionspace{180pt}

\tasknumber{6}%
\task{%
    Лампочки, сопротивления которых $R_1 = 0{,}50\,\text{Ом}$ и $R_2 = 18{,}00\,\text{Ом}$, поочерёдно подключённные к некоторому источнику тока,
    потребляют одинаковую мощность.
    Найти внутреннее сопротивление источника и КПД цепи в каждом случае.
}
\answer{%
    \begin{align*}
        P_1 &= \sqr{\frac{ \mathcal{E} }{R_1 + r}}R_1,
        P_2  = \sqr{\frac{ \mathcal{E} }{R_2 + r}}R_2,
        P_1 = P_2 \implies  \\
        &\implies R_1 \sqr{R_2 + r} = R_2 \sqr{R_1 + r} \implies  \\
        &\implies R_1 R_2^2 + 2 R_1 R_2 r + R_1 r^2 =
                    R_2 R_1^2 + 2 R_2 R_1 r + R_2 r^2  \implies  \\
    &\implies r^2 (R_2 - R_1) = R_2^2 R_2 - R_1^2 R_2 \implies  \\
    &\implies r
            = \sqrt{R_1 R_2 \frac{R_2 - R_1}{R_2 - R_1}}
            = \sqrt{R_1 R_2}
            = \sqrt{0{,}50\,\text{Ом} \cdot 18{,}00\,\text{Ом}}
            = 3{,}0\,\text{Ом}.
            \\
    \eta_1
            &= \frac{ R_1 }{R_1 + r}
            = \frac{\sqrt{ R_1 }}{\sqrt{ R_1 } + \sqrt{ R_2 }}
            = 0{,}143,  \\
    \eta_2
            &= \frac{ R_2 }{R_2 + r}
            = \frac{ \sqrt{ R_2 } }{\sqrt{ R_2 } + \sqrt{ R_1 }}
            = 0{,}857
    \end{align*}
}

\variantsplitter

\addpersonalvariant{Тимофей Полетаев}

\tasknumber{1}%
\task{%
    Напротив физических величин укажите их обозначения и единицы измерения в СИ:
    \begin{enumerate}
        \item сила тока,
        \item мощность тока,
        \item ЭДС,
        \item внутреннее сопротивление полной цепи.
    \end{enumerate}
}
\solutionspace{20pt}

\tasknumber{2}%
\task{%
    Запишите физический закон или формулу:
    \begin{enumerate}
        \item правило Кирхгофа для замкнутого контура,
        \item закон Ома для однородного участка цепи,
        \item ЭДС (определение).
    \end{enumerate}
}
\solutionspace{40pt}

\tasknumber{3}%
\task{%
    На резистор сопротивлением $r = 12\,\text{Ом}$ подали напряжение $U = 180\,\text{В}$.
    Определите ток, который потечёт через резистор, и мощность, выделяющуюся на нём.
}
\answer{%
    \begin{align*}
    \mathcal{I} &= \frac{ U }{ r } = \frac{ 180\,\text{В} }{ 12\,\text{Ом} } = 15{,}00\,\text{А},  \\
    P &= \frac{U^2}{ r } = \frac{ \sqr{ 180\,\text{В} } }{ 12\,\text{Ом} } = 2700{,}00\,\text{Вт}
    \end{align*}
}
\solutionspace{60pt}

\tasknumber{4}%
\task{%
    Через резистор сопротивлением $R = 30\,\text{Ом}$ протекает электрический ток $\mathcal{I} = 5{,}00\,\text{А}$.
    Определите, чему равны напряжение на резисторе и мощность, выделяющаяся на нём.
}
\answer{%
    \begin{align*}
    U &= \mathcal{I}R = 5{,}00\,\text{А} \cdot 30\,\text{Ом} = 150\,\text{В},  \\
    P &= \mathcal{I}^2R = \sqr{ 5{,}00\,\text{А} } \cdot 30\,\text{Ом} = 750\,\text{Вт}
    \end{align*}
}
\solutionspace{60pt}

\tasknumber{5}%
\task{%
    Замкнутая электрическая цепь состоит из ЭДС $\mathcal{E} = 3\,\text{В}$ и сопротивлением $r$
    и резистора $R = 24\,\text{Ом}$.
    Определите ток, протекающий в цепи.
    Какая тепловая энергия выделится на резисторе за время
    $\tau = 2\,\text{с}$? Какая работа будет совершена ЭДС за это время? Каков знак этой работы? Чему равен КПД цепи?
    Вычислите значения для 2 случаев: $r=0$ и $r = 10\,\text{Ом}$.
}
\answer{%
    \begin{align*}
    \mathcal{I}_1 &= \frac{ \mathcal{E} }{ R } = \frac{ 3\,\text{В} }{ 24\,\text{Ом} } = 0{,}12\,\text{А},  \\
    \mathcal{I}_2 &= \frac{ \mathcal{E} }{R + r} = \frac{ 3\,\text{В} }{24\,\text{Ом} + 10\,\text{Ом}} = 0{,}09\,\text{А},  \\
    Q_1 &= \mathcal{I}_1^2R\tau = \sqr{\frac{ \mathcal{E} }{ R }} R \tau
            = \sqr{\frac{ 3\,\text{В} }{ 24\,\text{Ом} }} \cdot 24\,\text{Ом} \cdot 2\,\text{с} = 0{,}691\,\text{Дж},  \\
    Q_2 &= \mathcal{I}_2^2R\tau = \sqr{\frac{ \mathcal{E} }{R + r}} R \tau
            = \sqr{\frac{ 3\,\text{В} }{24\,\text{Ом} + 10\,\text{Ом}}} \cdot 24\,\text{Ом} \cdot 2\,\text{с} = 0{,}389\,\text{Дж},  \\
    A_1 &= \mathcal{I}_1\tau\mathcal{E} = \frac{ \mathcal{E} }{R} \tau \mathcal{E}
            = \frac{\mathcal{E}^2 \tau}{ R } = \frac{\sqr{ 3\,\text{В} } \cdot 2\,\text{с}}{ 24\,\text{Ом} }
            = 0{,}720\,\text{Дж}, \text{положительна},  \\
    A_2 &= \mathcal{I}_2\tau\mathcal{E} = \frac{ \mathcal{E} }{R + r} \tau \mathcal{E}
            = \frac{\mathcal{E}^2 \tau}{R + r} = \frac{\sqr{ 3\,\text{В} } \cdot 2\,\text{с}}{24\,\text{Ом} + 10\,\text{Ом}}
            = 0{,}540\,\text{Дж}, \text{положительна},  \\
    \eta_1 &= \frac{ Q_1 }{ A_1 } = \ldots = \frac{ R }{ R } = 1,  \\
    \eta_2 &= \frac{ Q_2 }{ A_2 } = \ldots = \frac{ R }{R + r} = 0{,}72
    \end{align*}
}
\solutionspace{180pt}

\tasknumber{6}%
\task{%
    Лампочки, сопротивления которых $R_1 = 4{,}00\,\text{Ом}$ и $R_2 = 36{,}00\,\text{Ом}$, поочерёдно подключённные к некоторому источнику тока,
    потребляют одинаковую мощность.
    Найти внутреннее сопротивление источника и КПД цепи в каждом случае.
}
\answer{%
    \begin{align*}
        P_1 &= \sqr{\frac{ \mathcal{E} }{R_1 + r}}R_1,
        P_2  = \sqr{\frac{ \mathcal{E} }{R_2 + r}}R_2,
        P_1 = P_2 \implies  \\
        &\implies R_1 \sqr{R_2 + r} = R_2 \sqr{R_1 + r} \implies  \\
        &\implies R_1 R_2^2 + 2 R_1 R_2 r + R_1 r^2 =
                    R_2 R_1^2 + 2 R_2 R_1 r + R_2 r^2  \implies  \\
    &\implies r^2 (R_2 - R_1) = R_2^2 R_2 - R_1^2 R_2 \implies  \\
    &\implies r
            = \sqrt{R_1 R_2 \frac{R_2 - R_1}{R_2 - R_1}}
            = \sqrt{R_1 R_2}
            = \sqrt{4{,}00\,\text{Ом} \cdot 36{,}00\,\text{Ом}}
            = 12{,}0\,\text{Ом}.
            \\
    \eta_1
            &= \frac{ R_1 }{R_1 + r}
            = \frac{\sqrt{ R_1 }}{\sqrt{ R_1 } + \sqrt{ R_2 }}
            = 0{,}250,  \\
    \eta_2
            &= \frac{ R_2 }{R_2 + r}
            = \frac{ \sqrt{ R_2 } }{\sqrt{ R_2 } + \sqrt{ R_1 }}
            = 0{,}750
    \end{align*}
}

\variantsplitter

\addpersonalvariant{Андрей Рожков}

\tasknumber{1}%
\task{%
    Напротив физических величин укажите их обозначения и единицы измерения в СИ:
    \begin{enumerate}
        \item разность потенциалов,
        \item мощность тока,
        \item ЭДС,
        \item внешнее сопротивление полной цепи.
    \end{enumerate}
}
\solutionspace{20pt}

\tasknumber{2}%
\task{%
    Запишите физический закон или формулу:
    \begin{enumerate}
        \item правило Кирхгофа для узла цепи,
        \item закон Ома для однородного участка цепи,
        \item закон Ома для неоднородного участка цепи.
    \end{enumerate}
}
\solutionspace{40pt}

\tasknumber{3}%
\task{%
    На резистор сопротивлением $r = 12\,\text{Ом}$ подали напряжение $U = 150\,\text{В}$.
    Определите ток, который потечёт через резистор, и мощность, выделяющуюся на нём.
}
\answer{%
    \begin{align*}
    \mathcal{I} &= \frac{ U }{ r } = \frac{ 150\,\text{В} }{ 12\,\text{Ом} } = 12{,}50\,\text{А},  \\
    P &= \frac{U^2}{ r } = \frac{ \sqr{ 150\,\text{В} } }{ 12\,\text{Ом} } = 1875{,}00\,\text{Вт}
    \end{align*}
}
\solutionspace{60pt}

\tasknumber{4}%
\task{%
    Через резистор сопротивлением $r = 5\,\text{Ом}$ протекает электрический ток $\mathcal{I} = 10{,}00\,\text{А}$.
    Определите, чему равны напряжение на резисторе и мощность, выделяющаяся на нём.
}
\answer{%
    \begin{align*}
    U &= \mathcal{I}r = 10{,}00\,\text{А} \cdot 5\,\text{Ом} = 50\,\text{В},  \\
    P &= \mathcal{I}^2r = \sqr{ 10{,}00\,\text{А} } \cdot 5\,\text{Ом} = 500\,\text{Вт}
    \end{align*}
}
\solutionspace{60pt}

\tasknumber{5}%
\task{%
    Замкнутая электрическая цепь состоит из ЭДС $\mathcal{E} = 3\,\text{В}$ и сопротивлением $r$
    и резистора $R = 24\,\text{Ом}$.
    Определите ток, протекающий в цепи.
    Какая тепловая энергия выделится на резисторе за время
    $\tau = 5\,\text{с}$? Какая работа будет совершена ЭДС за это время? Каков знак этой работы? Чему равен КПД цепи?
    Вычислите значения для 2 случаев: $r=0$ и $r = 30\,\text{Ом}$.
}
\answer{%
    \begin{align*}
    \mathcal{I}_1 &= \frac{ \mathcal{E} }{ R } = \frac{ 3\,\text{В} }{ 24\,\text{Ом} } = 0{,}12\,\text{А},  \\
    \mathcal{I}_2 &= \frac{ \mathcal{E} }{R + r} = \frac{ 3\,\text{В} }{24\,\text{Ом} + 30\,\text{Ом}} = 0{,}06\,\text{А},  \\
    Q_1 &= \mathcal{I}_1^2R\tau = \sqr{\frac{ \mathcal{E} }{ R }} R \tau
            = \sqr{\frac{ 3\,\text{В} }{ 24\,\text{Ом} }} \cdot 24\,\text{Ом} \cdot 5\,\text{с} = 1{,}728\,\text{Дж},  \\
    Q_2 &= \mathcal{I}_2^2R\tau = \sqr{\frac{ \mathcal{E} }{R + r}} R \tau
            = \sqr{\frac{ 3\,\text{В} }{24\,\text{Ом} + 30\,\text{Ом}}} \cdot 24\,\text{Ом} \cdot 5\,\text{с} = 0{,}432\,\text{Дж},  \\
    A_1 &= \mathcal{I}_1\tau\mathcal{E} = \frac{ \mathcal{E} }{R} \tau \mathcal{E}
            = \frac{\mathcal{E}^2 \tau}{ R } = \frac{\sqr{ 3\,\text{В} } \cdot 5\,\text{с}}{ 24\,\text{Ом} }
            = 1{,}800\,\text{Дж}, \text{положительна},  \\
    A_2 &= \mathcal{I}_2\tau\mathcal{E} = \frac{ \mathcal{E} }{R + r} \tau \mathcal{E}
            = \frac{\mathcal{E}^2 \tau}{R + r} = \frac{\sqr{ 3\,\text{В} } \cdot 5\,\text{с}}{24\,\text{Ом} + 30\,\text{Ом}}
            = 0{,}900\,\text{Дж}, \text{положительна},  \\
    \eta_1 &= \frac{ Q_1 }{ A_1 } = \ldots = \frac{ R }{ R } = 1,  \\
    \eta_2 &= \frac{ Q_2 }{ A_2 } = \ldots = \frac{ R }{R + r} = 0{,}48
    \end{align*}
}
\solutionspace{180pt}

\tasknumber{6}%
\task{%
    Лампочки, сопротивления которых $R_1 = 5{,}00\,\text{Ом}$ и $R_2 = 45{,}00\,\text{Ом}$, поочерёдно подключённные к некоторому источнику тока,
    потребляют одинаковую мощность.
    Найти внутреннее сопротивление источника и КПД цепи в каждом случае.
}
\answer{%
    \begin{align*}
        P_1 &= \sqr{\frac{ \mathcal{E} }{R_1 + r}}R_1,
        P_2  = \sqr{\frac{ \mathcal{E} }{R_2 + r}}R_2,
        P_1 = P_2 \implies  \\
        &\implies R_1 \sqr{R_2 + r} = R_2 \sqr{R_1 + r} \implies  \\
        &\implies R_1 R_2^2 + 2 R_1 R_2 r + R_1 r^2 =
                    R_2 R_1^2 + 2 R_2 R_1 r + R_2 r^2  \implies  \\
    &\implies r^2 (R_2 - R_1) = R_2^2 R_2 - R_1^2 R_2 \implies  \\
    &\implies r
            = \sqrt{R_1 R_2 \frac{R_2 - R_1}{R_2 - R_1}}
            = \sqrt{R_1 R_2}
            = \sqrt{5{,}00\,\text{Ом} \cdot 45{,}00\,\text{Ом}}
            = 15{,}0\,\text{Ом}.
            \\
    \eta_1
            &= \frac{ R_1 }{R_1 + r}
            = \frac{\sqrt{ R_1 }}{\sqrt{ R_1 } + \sqrt{ R_2 }}
            = 0{,}250,  \\
    \eta_2
            &= \frac{ R_2 }{R_2 + r}
            = \frac{ \sqrt{ R_2 } }{\sqrt{ R_2 } + \sqrt{ R_1 }}
            = 0{,}750
    \end{align*}
}

\variantsplitter

\addpersonalvariant{Рената Таржиманова}

\tasknumber{1}%
\task{%
    Напротив физических величин укажите их обозначения и единицы измерения в СИ:
    \begin{enumerate}
        \item напряжение,
        \item работа тока,
        \item ЭДС,
        \item внутреннее сопротивление полной цепи.
    \end{enumerate}
}
\solutionspace{20pt}

\tasknumber{2}%
\task{%
    Запишите физический закон или формулу:
    \begin{enumerate}
        \item правило Кирхгофа для замкнутого контура,
        \item закон Ома для однородного участка цепи,
        \item ЭДС (определение).
    \end{enumerate}
}
\solutionspace{40pt}

\tasknumber{3}%
\task{%
    На резистор сопротивлением $R = 12\,\text{Ом}$ подали напряжение $U = 150\,\text{В}$.
    Определите ток, который потечёт через резистор, и мощность, выделяющуюся на нём.
}
\answer{%
    \begin{align*}
    \mathcal{I} &= \frac{ U }{ R } = \frac{ 150\,\text{В} }{ 12\,\text{Ом} } = 12{,}50\,\text{А},  \\
    P &= \frac{U^2}{ R } = \frac{ \sqr{ 150\,\text{В} } }{ 12\,\text{Ом} } = 1875{,}00\,\text{Вт}
    \end{align*}
}
\solutionspace{60pt}

\tasknumber{4}%
\task{%
    Через резистор сопротивлением $R = 18\,\text{Ом}$ протекает электрический ток $\mathcal{I} = 5{,}00\,\text{А}$.
    Определите, чему равны напряжение на резисторе и мощность, выделяющаяся на нём.
}
\answer{%
    \begin{align*}
    U &= \mathcal{I}R = 5{,}00\,\text{А} \cdot 18\,\text{Ом} = 90\,\text{В},  \\
    P &= \mathcal{I}^2R = \sqr{ 5{,}00\,\text{А} } \cdot 18\,\text{Ом} = 450\,\text{Вт}
    \end{align*}
}
\solutionspace{60pt}

\tasknumber{5}%
\task{%
    Замкнутая электрическая цепь состоит из ЭДС $\mathcal{E} = 4\,\text{В}$ и сопротивлением $r$
    и резистора $R = 24\,\text{Ом}$.
    Определите ток, протекающий в цепи.
    Какая тепловая энергия выделится на резисторе за время
    $\tau = 2\,\text{с}$? Какая работа будет совершена ЭДС за это время? Каков знак этой работы? Чему равен КПД цепи?
    Вычислите значения для 2 случаев: $r=0$ и $r = 30\,\text{Ом}$.
}
\answer{%
    \begin{align*}
    \mathcal{I}_1 &= \frac{ \mathcal{E} }{ R } = \frac{ 4\,\text{В} }{ 24\,\text{Ом} } = 0{,}17\,\text{А},  \\
    \mathcal{I}_2 &= \frac{ \mathcal{E} }{R + r} = \frac{ 4\,\text{В} }{24\,\text{Ом} + 30\,\text{Ом}} = 0{,}07\,\text{А},  \\
    Q_1 &= \mathcal{I}_1^2R\tau = \sqr{\frac{ \mathcal{E} }{ R }} R \tau
            = \sqr{\frac{ 4\,\text{В} }{ 24\,\text{Ом} }} \cdot 24\,\text{Ом} \cdot 2\,\text{с} = 1{,}387\,\text{Дж},  \\
    Q_2 &= \mathcal{I}_2^2R\tau = \sqr{\frac{ \mathcal{E} }{R + r}} R \tau
            = \sqr{\frac{ 4\,\text{В} }{24\,\text{Ом} + 30\,\text{Ом}}} \cdot 24\,\text{Ом} \cdot 2\,\text{с} = 0{,}235\,\text{Дж},  \\
    A_1 &= \mathcal{I}_1\tau\mathcal{E} = \frac{ \mathcal{E} }{R} \tau \mathcal{E}
            = \frac{\mathcal{E}^2 \tau}{ R } = \frac{\sqr{ 4\,\text{В} } \cdot 2\,\text{с}}{ 24\,\text{Ом} }
            = 1{,}360\,\text{Дж}, \text{положительна},  \\
    A_2 &= \mathcal{I}_2\tau\mathcal{E} = \frac{ \mathcal{E} }{R + r} \tau \mathcal{E}
            = \frac{\mathcal{E}^2 \tau}{R + r} = \frac{\sqr{ 4\,\text{В} } \cdot 2\,\text{с}}{24\,\text{Ом} + 30\,\text{Ом}}
            = 0{,}560\,\text{Дж}, \text{положительна},  \\
    \eta_1 &= \frac{ Q_1 }{ A_1 } = \ldots = \frac{ R }{ R } = 1,  \\
    \eta_2 &= \frac{ Q_2 }{ A_2 } = \ldots = \frac{ R }{R + r} = 0{,}42
    \end{align*}
}
\solutionspace{180pt}

\tasknumber{6}%
\task{%
    Лампочки, сопротивления которых $R_1 = 4{,}00\,\text{Ом}$ и $R_2 = 36{,}00\,\text{Ом}$, поочерёдно подключённные к некоторому источнику тока,
    потребляют одинаковую мощность.
    Найти внутреннее сопротивление источника и КПД цепи в каждом случае.
}
\answer{%
    \begin{align*}
        P_1 &= \sqr{\frac{ \mathcal{E} }{R_1 + r}}R_1,
        P_2  = \sqr{\frac{ \mathcal{E} }{R_2 + r}}R_2,
        P_1 = P_2 \implies  \\
        &\implies R_1 \sqr{R_2 + r} = R_2 \sqr{R_1 + r} \implies  \\
        &\implies R_1 R_2^2 + 2 R_1 R_2 r + R_1 r^2 =
                    R_2 R_1^2 + 2 R_2 R_1 r + R_2 r^2  \implies  \\
    &\implies r^2 (R_2 - R_1) = R_2^2 R_2 - R_1^2 R_2 \implies  \\
    &\implies r
            = \sqrt{R_1 R_2 \frac{R_2 - R_1}{R_2 - R_1}}
            = \sqrt{R_1 R_2}
            = \sqrt{4{,}00\,\text{Ом} \cdot 36{,}00\,\text{Ом}}
            = 12{,}0\,\text{Ом}.
            \\
    \eta_1
            &= \frac{ R_1 }{R_1 + r}
            = \frac{\sqrt{ R_1 }}{\sqrt{ R_1 } + \sqrt{ R_2 }}
            = 0{,}250,  \\
    \eta_2
            &= \frac{ R_2 }{R_2 + r}
            = \frac{ \sqrt{ R_2 } }{\sqrt{ R_2 } + \sqrt{ R_1 }}
            = 0{,}750
    \end{align*}
}

\variantsplitter

\addpersonalvariant{Андрей Щербаков}

\tasknumber{1}%
\task{%
    Напротив физических величин укажите их обозначения и единицы измерения в СИ:
    \begin{enumerate}
        \item напряжение,
        \item мощность тока,
        \item ЭДС,
        \item внешнее сопротивление полной цепи.
    \end{enumerate}
}
\solutionspace{20pt}

\tasknumber{2}%
\task{%
    Запишите физический закон или формулу:
    \begin{enumerate}
        \item правило Кирхгофа для замкнутого контура,
        \item сопротивление резистора через удельное сопротивление,
        \item ЭДС (определение).
    \end{enumerate}
}
\solutionspace{40pt}

\tasknumber{3}%
\task{%
    На резистор сопротивлением $R = 5\,\text{Ом}$ подали напряжение $U = 180\,\text{В}$.
    Определите ток, который потечёт через резистор, и мощность, выделяющуюся на нём.
}
\answer{%
    \begin{align*}
    \mathcal{I} &= \frac{ U }{ R } = \frac{ 180\,\text{В} }{ 5\,\text{Ом} } = 36{,}00\,\text{А},  \\
    P &= \frac{U^2}{ R } = \frac{ \sqr{ 180\,\text{В} } }{ 5\,\text{Ом} } = 6480{,}00\,\text{Вт}
    \end{align*}
}
\solutionspace{60pt}

\tasknumber{4}%
\task{%
    Через резистор сопротивлением $r = 12\,\text{Ом}$ протекает электрический ток $\mathcal{I} = 5{,}00\,\text{А}$.
    Определите, чему равны напряжение на резисторе и мощность, выделяющаяся на нём.
}
\answer{%
    \begin{align*}
    U &= \mathcal{I}r = 5{,}00\,\text{А} \cdot 12\,\text{Ом} = 60\,\text{В},  \\
    P &= \mathcal{I}^2r = \sqr{ 5{,}00\,\text{А} } \cdot 12\,\text{Ом} = 300\,\text{Вт}
    \end{align*}
}
\solutionspace{60pt}

\tasknumber{5}%
\task{%
    Замкнутая электрическая цепь состоит из ЭДС $\mathcal{E} = 4\,\text{В}$ и сопротивлением $r$
    и резистора $R = 30\,\text{Ом}$.
    Определите ток, протекающий в цепи.
    Какая тепловая энергия выделится на резисторе за время
    $\tau = 2\,\text{с}$? Какая работа будет совершена ЭДС за это время? Каков знак этой работы? Чему равен КПД цепи?
    Вычислите значения для 2 случаев: $r=0$ и $r = 20\,\text{Ом}$.
}
\answer{%
    \begin{align*}
    \mathcal{I}_1 &= \frac{ \mathcal{E} }{ R } = \frac{ 4\,\text{В} }{ 30\,\text{Ом} } = 0{,}13\,\text{А},  \\
    \mathcal{I}_2 &= \frac{ \mathcal{E} }{R + r} = \frac{ 4\,\text{В} }{30\,\text{Ом} + 20\,\text{Ом}} = 0{,}08\,\text{А},  \\
    Q_1 &= \mathcal{I}_1^2R\tau = \sqr{\frac{ \mathcal{E} }{ R }} R \tau
            = \sqr{\frac{ 4\,\text{В} }{ 30\,\text{Ом} }} \cdot 30\,\text{Ом} \cdot 2\,\text{с} = 1{,}014\,\text{Дж},  \\
    Q_2 &= \mathcal{I}_2^2R\tau = \sqr{\frac{ \mathcal{E} }{R + r}} R \tau
            = \sqr{\frac{ 4\,\text{В} }{30\,\text{Ом} + 20\,\text{Ом}}} \cdot 30\,\text{Ом} \cdot 2\,\text{с} = 0{,}384\,\text{Дж},  \\
    A_1 &= \mathcal{I}_1\tau\mathcal{E} = \frac{ \mathcal{E} }{R} \tau \mathcal{E}
            = \frac{\mathcal{E}^2 \tau}{ R } = \frac{\sqr{ 4\,\text{В} } \cdot 2\,\text{с}}{ 30\,\text{Ом} }
            = 1{,}040\,\text{Дж}, \text{положительна},  \\
    A_2 &= \mathcal{I}_2\tau\mathcal{E} = \frac{ \mathcal{E} }{R + r} \tau \mathcal{E}
            = \frac{\mathcal{E}^2 \tau}{R + r} = \frac{\sqr{ 4\,\text{В} } \cdot 2\,\text{с}}{30\,\text{Ом} + 20\,\text{Ом}}
            = 0{,}640\,\text{Дж}, \text{положительна},  \\
    \eta_1 &= \frac{ Q_1 }{ A_1 } = \ldots = \frac{ R }{ R } = 1,  \\
    \eta_2 &= \frac{ Q_2 }{ A_2 } = \ldots = \frac{ R }{R + r} = 0{,}60
    \end{align*}
}
\solutionspace{180pt}

\tasknumber{6}%
\task{%
    Лампочки, сопротивления которых $R_1 = 0{,}50\,\text{Ом}$ и $R_2 = 4{,}50\,\text{Ом}$, поочерёдно подключённные к некоторому источнику тока,
    потребляют одинаковую мощность.
    Найти внутреннее сопротивление источника и КПД цепи в каждом случае.
}
\answer{%
    \begin{align*}
        P_1 &= \sqr{\frac{ \mathcal{E} }{R_1 + r}}R_1,
        P_2  = \sqr{\frac{ \mathcal{E} }{R_2 + r}}R_2,
        P_1 = P_2 \implies  \\
        &\implies R_1 \sqr{R_2 + r} = R_2 \sqr{R_1 + r} \implies  \\
        &\implies R_1 R_2^2 + 2 R_1 R_2 r + R_1 r^2 =
                    R_2 R_1^2 + 2 R_2 R_1 r + R_2 r^2  \implies  \\
    &\implies r^2 (R_2 - R_1) = R_2^2 R_2 - R_1^2 R_2 \implies  \\
    &\implies r
            = \sqrt{R_1 R_2 \frac{R_2 - R_1}{R_2 - R_1}}
            = \sqrt{R_1 R_2}
            = \sqrt{0{,}50\,\text{Ом} \cdot 4{,}50\,\text{Ом}}
            = 1{,}5\,\text{Ом}.
            \\
    \eta_1
            &= \frac{ R_1 }{R_1 + r}
            = \frac{\sqrt{ R_1 }}{\sqrt{ R_1 } + \sqrt{ R_2 }}
            = 0{,}250,  \\
    \eta_2
            &= \frac{ R_2 }{R_2 + r}
            = \frac{ \sqrt{ R_2 } }{\sqrt{ R_2 } + \sqrt{ R_1 }}
            = 0{,}750
    \end{align*}
}

\variantsplitter

\addpersonalvariant{Михаил Ярошевский}

\tasknumber{1}%
\task{%
    Напротив физических величин укажите их обозначения и единицы измерения в СИ:
    \begin{enumerate}
        \item сила тока,
        \item мощность тока,
        \item удельное сопротивление,
        \item внутреннее сопротивление полной цепи.
    \end{enumerate}
}
\solutionspace{20pt}

\tasknumber{2}%
\task{%
    Запишите физический закон или формулу:
    \begin{enumerate}
        \item правило Кирхгофа для замкнутого контура,
        \item сопротивление резистора через удельное сопротивление,
        \item закон Ома для неоднородного участка цепи.
    \end{enumerate}
}
\solutionspace{40pt}

\tasknumber{3}%
\task{%
    На резистор сопротивлением $r = 30\,\text{Ом}$ подали напряжение $V = 180\,\text{В}$.
    Определите ток, который потечёт через резистор, и мощность, выделяющуюся на нём.
}
\answer{%
    \begin{align*}
    \mathcal{I} &= \frac{ V }{ r } = \frac{ 180\,\text{В} }{ 30\,\text{Ом} } = 6{,}00\,\text{А},  \\
    P &= \frac{V^2}{ r } = \frac{ \sqr{ 180\,\text{В} } }{ 30\,\text{Ом} } = 1080{,}00\,\text{Вт}
    \end{align*}
}
\solutionspace{60pt}

\tasknumber{4}%
\task{%
    Через резистор сопротивлением $r = 12\,\text{Ом}$ протекает электрический ток $\mathcal{I} = 3{,}00\,\text{А}$.
    Определите, чему равны напряжение на резисторе и мощность, выделяющаяся на нём.
}
\answer{%
    \begin{align*}
    U &= \mathcal{I}r = 3{,}00\,\text{А} \cdot 12\,\text{Ом} = 36\,\text{В},  \\
    P &= \mathcal{I}^2r = \sqr{ 3{,}00\,\text{А} } \cdot 12\,\text{Ом} = 108\,\text{Вт}
    \end{align*}
}
\solutionspace{60pt}

\tasknumber{5}%
\task{%
    Замкнутая электрическая цепь состоит из ЭДС $\mathcal{E} = 4\,\text{В}$ и сопротивлением $r$
    и резистора $R = 15\,\text{Ом}$.
    Определите ток, протекающий в цепи.
    Какая тепловая энергия выделится на резисторе за время
    $\tau = 10\,\text{с}$? Какая работа будет совершена ЭДС за это время? Каков знак этой работы? Чему равен КПД цепи?
    Вычислите значения для 2 случаев: $r=0$ и $r = 30\,\text{Ом}$.
}
\answer{%
    \begin{align*}
    \mathcal{I}_1 &= \frac{ \mathcal{E} }{ R } = \frac{ 4\,\text{В} }{ 15\,\text{Ом} } = 0{,}27\,\text{А},  \\
    \mathcal{I}_2 &= \frac{ \mathcal{E} }{R + r} = \frac{ 4\,\text{В} }{15\,\text{Ом} + 30\,\text{Ом}} = 0{,}09\,\text{А},  \\
    Q_1 &= \mathcal{I}_1^2R\tau = \sqr{\frac{ \mathcal{E} }{ R }} R \tau
            = \sqr{\frac{ 4\,\text{В} }{ 15\,\text{Ом} }} \cdot 15\,\text{Ом} \cdot 10\,\text{с} = 10{,}935\,\text{Дж},  \\
    Q_2 &= \mathcal{I}_2^2R\tau = \sqr{\frac{ \mathcal{E} }{R + r}} R \tau
            = \sqr{\frac{ 4\,\text{В} }{15\,\text{Ом} + 30\,\text{Ом}}} \cdot 15\,\text{Ом} \cdot 10\,\text{с} = 1{,}215\,\text{Дж},  \\
    A_1 &= \mathcal{I}_1\tau\mathcal{E} = \frac{ \mathcal{E} }{R} \tau \mathcal{E}
            = \frac{\mathcal{E}^2 \tau}{ R } = \frac{\sqr{ 4\,\text{В} } \cdot 10\,\text{с}}{ 15\,\text{Ом} }
            = 10{,}800\,\text{Дж}, \text{положительна},  \\
    A_2 &= \mathcal{I}_2\tau\mathcal{E} = \frac{ \mathcal{E} }{R + r} \tau \mathcal{E}
            = \frac{\mathcal{E}^2 \tau}{R + r} = \frac{\sqr{ 4\,\text{В} } \cdot 10\,\text{с}}{15\,\text{Ом} + 30\,\text{Ом}}
            = 3{,}600\,\text{Дж}, \text{положительна},  \\
    \eta_1 &= \frac{ Q_1 }{ A_1 } = \ldots = \frac{ R }{ R } = 1,  \\
    \eta_2 &= \frac{ Q_2 }{ A_2 } = \ldots = \frac{ R }{R + r} = 0{,}34
    \end{align*}
}
\solutionspace{180pt}

\tasknumber{6}%
\task{%
    Лампочки, сопротивления которых $R_1 = 0{,}50\,\text{Ом}$ и $R_2 = 18{,}00\,\text{Ом}$, поочерёдно подключённные к некоторому источнику тока,
    потребляют одинаковую мощность.
    Найти внутреннее сопротивление источника и КПД цепи в каждом случае.
}
\answer{%
    \begin{align*}
        P_1 &= \sqr{\frac{ \mathcal{E} }{R_1 + r}}R_1,
        P_2  = \sqr{\frac{ \mathcal{E} }{R_2 + r}}R_2,
        P_1 = P_2 \implies  \\
        &\implies R_1 \sqr{R_2 + r} = R_2 \sqr{R_1 + r} \implies  \\
        &\implies R_1 R_2^2 + 2 R_1 R_2 r + R_1 r^2 =
                    R_2 R_1^2 + 2 R_2 R_1 r + R_2 r^2  \implies  \\
    &\implies r^2 (R_2 - R_1) = R_2^2 R_2 - R_1^2 R_2 \implies  \\
    &\implies r
            = \sqrt{R_1 R_2 \frac{R_2 - R_1}{R_2 - R_1}}
            = \sqrt{R_1 R_2}
            = \sqrt{0{,}50\,\text{Ом} \cdot 18{,}00\,\text{Ом}}
            = 3{,}0\,\text{Ом}.
            \\
    \eta_1
            &= \frac{ R_1 }{R_1 + r}
            = \frac{\sqrt{ R_1 }}{\sqrt{ R_1 } + \sqrt{ R_2 }}
            = 0{,}143,  \\
    \eta_2
            &= \frac{ R_2 }{R_2 + r}
            = \frac{ \sqrt{ R_2 } }{\sqrt{ R_2 } + \sqrt{ R_1 }}
            = 0{,}857
    \end{align*}
}

\variantsplitter

\addpersonalvariant{Алексей Алимпиев}

\tasknumber{1}%
\task{%
    Напротив физических величин укажите их обозначения и единицы измерения в СИ:
    \begin{enumerate}
        \item сила тока,
        \item работа тока,
        \item удельное сопротивление,
        \item внутреннее сопротивление полной цепи.
    \end{enumerate}
}
\solutionspace{20pt}

\tasknumber{2}%
\task{%
    Запишите физический закон или формулу:
    \begin{enumerate}
        \item правило Кирхгофа для узла цепи,
        \item закон Ома для однородного участка цепи,
        \item ЭДС (определение).
    \end{enumerate}
}
\solutionspace{40pt}

\tasknumber{3}%
\task{%
    На резистор сопротивлением $R = 18\,\text{Ом}$ подали напряжение $V = 150\,\text{В}$.
    Определите ток, который потечёт через резистор, и мощность, выделяющуюся на нём.
}
\answer{%
    \begin{align*}
    \mathcal{I} &= \frac{ V }{ R } = \frac{ 150\,\text{В} }{ 18\,\text{Ом} } = 8{,}33\,\text{А},  \\
    P &= \frac{V^2}{ R } = \frac{ \sqr{ 150\,\text{В} } }{ 18\,\text{Ом} } = 1250{,}00\,\text{Вт}
    \end{align*}
}
\solutionspace{60pt}

\tasknumber{4}%
\task{%
    Через резистор сопротивлением $r = 30\,\text{Ом}$ протекает электрический ток $\mathcal{I} = 5{,}00\,\text{А}$.
    Определите, чему равны напряжение на резисторе и мощность, выделяющаяся на нём.
}
\answer{%
    \begin{align*}
    U &= \mathcal{I}r = 5{,}00\,\text{А} \cdot 30\,\text{Ом} = 150\,\text{В},  \\
    P &= \mathcal{I}^2r = \sqr{ 5{,}00\,\text{А} } \cdot 30\,\text{Ом} = 750\,\text{Вт}
    \end{align*}
}
\solutionspace{60pt}

\tasknumber{5}%
\task{%
    Замкнутая электрическая цепь состоит из ЭДС $\mathcal{E} = 2\,\text{В}$ и сопротивлением $r$
    и резистора $R = 10\,\text{Ом}$.
    Определите ток, протекающий в цепи.
    Какая тепловая энергия выделится на резисторе за время
    $\tau = 10\,\text{с}$? Какая работа будет совершена ЭДС за это время? Каков знак этой работы? Чему равен КПД цепи?
    Вычислите значения для 2 случаев: $r=0$ и $r = 20\,\text{Ом}$.
}
\answer{%
    \begin{align*}
    \mathcal{I}_1 &= \frac{ \mathcal{E} }{ R } = \frac{ 2\,\text{В} }{ 10\,\text{Ом} } = 0{,}20\,\text{А},  \\
    \mathcal{I}_2 &= \frac{ \mathcal{E} }{R + r} = \frac{ 2\,\text{В} }{10\,\text{Ом} + 20\,\text{Ом}} = 0{,}07\,\text{А},  \\
    Q_1 &= \mathcal{I}_1^2R\tau = \sqr{\frac{ \mathcal{E} }{ R }} R \tau
            = \sqr{\frac{ 2\,\text{В} }{ 10\,\text{Ом} }} \cdot 10\,\text{Ом} \cdot 10\,\text{с} = 4{,}000\,\text{Дж},  \\
    Q_2 &= \mathcal{I}_2^2R\tau = \sqr{\frac{ \mathcal{E} }{R + r}} R \tau
            = \sqr{\frac{ 2\,\text{В} }{10\,\text{Ом} + 20\,\text{Ом}}} \cdot 10\,\text{Ом} \cdot 10\,\text{с} = 0{,}490\,\text{Дж},  \\
    A_1 &= \mathcal{I}_1\tau\mathcal{E} = \frac{ \mathcal{E} }{R} \tau \mathcal{E}
            = \frac{\mathcal{E}^2 \tau}{ R } = \frac{\sqr{ 2\,\text{В} } \cdot 10\,\text{с}}{ 10\,\text{Ом} }
            = 4{,}000\,\text{Дж}, \text{положительна},  \\
    A_2 &= \mathcal{I}_2\tau\mathcal{E} = \frac{ \mathcal{E} }{R + r} \tau \mathcal{E}
            = \frac{\mathcal{E}^2 \tau}{R + r} = \frac{\sqr{ 2\,\text{В} } \cdot 10\,\text{с}}{10\,\text{Ом} + 20\,\text{Ом}}
            = 1{,}400\,\text{Дж}, \text{положительна},  \\
    \eta_1 &= \frac{ Q_1 }{ A_1 } = \ldots = \frac{ R }{ R } = 1,  \\
    \eta_2 &= \frac{ Q_2 }{ A_2 } = \ldots = \frac{ R }{R + r} = 0{,}35
    \end{align*}
}
\solutionspace{180pt}

\tasknumber{6}%
\task{%
    Лампочки, сопротивления которых $R_1 = 0{,}50\,\text{Ом}$ и $R_2 = 2{,}00\,\text{Ом}$, поочерёдно подключённные к некоторому источнику тока,
    потребляют одинаковую мощность.
    Найти внутреннее сопротивление источника и КПД цепи в каждом случае.
}
\answer{%
    \begin{align*}
        P_1 &= \sqr{\frac{ \mathcal{E} }{R_1 + r}}R_1,
        P_2  = \sqr{\frac{ \mathcal{E} }{R_2 + r}}R_2,
        P_1 = P_2 \implies  \\
        &\implies R_1 \sqr{R_2 + r} = R_2 \sqr{R_1 + r} \implies  \\
        &\implies R_1 R_2^2 + 2 R_1 R_2 r + R_1 r^2 =
                    R_2 R_1^2 + 2 R_2 R_1 r + R_2 r^2  \implies  \\
    &\implies r^2 (R_2 - R_1) = R_2^2 R_2 - R_1^2 R_2 \implies  \\
    &\implies r
            = \sqrt{R_1 R_2 \frac{R_2 - R_1}{R_2 - R_1}}
            = \sqrt{R_1 R_2}
            = \sqrt{0{,}50\,\text{Ом} \cdot 2{,}00\,\text{Ом}}
            = 1{,}0\,\text{Ом}.
            \\
    \eta_1
            &= \frac{ R_1 }{R_1 + r}
            = \frac{\sqrt{ R_1 }}{\sqrt{ R_1 } + \sqrt{ R_2 }}
            = 0{,}333,  \\
    \eta_2
            &= \frac{ R_2 }{R_2 + r}
            = \frac{ \sqrt{ R_2 } }{\sqrt{ R_2 } + \sqrt{ R_1 }}
            = 0{,}667
    \end{align*}
}

\variantsplitter

\addpersonalvariant{Евгений Васин}

\tasknumber{1}%
\task{%
    Напротив физических величин укажите их обозначения и единицы измерения в СИ:
    \begin{enumerate}
        \item сила тока,
        \item работа тока,
        \item ЭДС,
        \item внутреннее сопротивление полной цепи.
    \end{enumerate}
}
\solutionspace{20pt}

\tasknumber{2}%
\task{%
    Запишите физический закон или формулу:
    \begin{enumerate}
        \item правило Кирхгофа для узла цепи,
        \item закон Ома для однородного участка цепи,
        \item закон Ома для неоднородного участка цепи.
    \end{enumerate}
}
\solutionspace{40pt}

\tasknumber{3}%
\task{%
    На резистор сопротивлением $R = 5\,\text{Ом}$ подали напряжение $U = 240\,\text{В}$.
    Определите ток, который потечёт через резистор, и мощность, выделяющуюся на нём.
}
\answer{%
    \begin{align*}
    \mathcal{I} &= \frac{ U }{ R } = \frac{ 240\,\text{В} }{ 5\,\text{Ом} } = 48{,}00\,\text{А},  \\
    P &= \frac{U^2}{ R } = \frac{ \sqr{ 240\,\text{В} } }{ 5\,\text{Ом} } = 11520{,}00\,\text{Вт}
    \end{align*}
}
\solutionspace{60pt}

\tasknumber{4}%
\task{%
    Через резистор сопротивлением $R = 30\,\text{Ом}$ протекает электрический ток $\mathcal{I} = 8{,}00\,\text{А}$.
    Определите, чему равны напряжение на резисторе и мощность, выделяющаяся на нём.
}
\answer{%
    \begin{align*}
    U &= \mathcal{I}R = 8{,}00\,\text{А} \cdot 30\,\text{Ом} = 240\,\text{В},  \\
    P &= \mathcal{I}^2R = \sqr{ 8{,}00\,\text{А} } \cdot 30\,\text{Ом} = 1920\,\text{Вт}
    \end{align*}
}
\solutionspace{60pt}

\tasknumber{5}%
\task{%
    Замкнутая электрическая цепь состоит из ЭДС $\mathcal{E} = 4\,\text{В}$ и сопротивлением $r$
    и резистора $R = 30\,\text{Ом}$.
    Определите ток, протекающий в цепи.
    Какая тепловая энергия выделится на резисторе за время
    $\tau = 5\,\text{с}$? Какая работа будет совершена ЭДС за это время? Каков знак этой работы? Чему равен КПД цепи?
    Вычислите значения для 2 случаев: $r=0$ и $r = 60\,\text{Ом}$.
}
\answer{%
    \begin{align*}
    \mathcal{I}_1 &= \frac{ \mathcal{E} }{ R } = \frac{ 4\,\text{В} }{ 30\,\text{Ом} } = 0{,}13\,\text{А},  \\
    \mathcal{I}_2 &= \frac{ \mathcal{E} }{R + r} = \frac{ 4\,\text{В} }{30\,\text{Ом} + 60\,\text{Ом}} = 0{,}04\,\text{А},  \\
    Q_1 &= \mathcal{I}_1^2R\tau = \sqr{\frac{ \mathcal{E} }{ R }} R \tau
            = \sqr{\frac{ 4\,\text{В} }{ 30\,\text{Ом} }} \cdot 30\,\text{Ом} \cdot 5\,\text{с} = 2{,}535\,\text{Дж},  \\
    Q_2 &= \mathcal{I}_2^2R\tau = \sqr{\frac{ \mathcal{E} }{R + r}} R \tau
            = \sqr{\frac{ 4\,\text{В} }{30\,\text{Ом} + 60\,\text{Ом}}} \cdot 30\,\text{Ом} \cdot 5\,\text{с} = 0{,}240\,\text{Дж},  \\
    A_1 &= \mathcal{I}_1\tau\mathcal{E} = \frac{ \mathcal{E} }{R} \tau \mathcal{E}
            = \frac{\mathcal{E}^2 \tau}{ R } = \frac{\sqr{ 4\,\text{В} } \cdot 5\,\text{с}}{ 30\,\text{Ом} }
            = 2{,}600\,\text{Дж}, \text{положительна},  \\
    A_2 &= \mathcal{I}_2\tau\mathcal{E} = \frac{ \mathcal{E} }{R + r} \tau \mathcal{E}
            = \frac{\mathcal{E}^2 \tau}{R + r} = \frac{\sqr{ 4\,\text{В} } \cdot 5\,\text{с}}{30\,\text{Ом} + 60\,\text{Ом}}
            = 0{,}800\,\text{Дж}, \text{положительна},  \\
    \eta_1 &= \frac{ Q_1 }{ A_1 } = \ldots = \frac{ R }{ R } = 1,  \\
    \eta_2 &= \frac{ Q_2 }{ A_2 } = \ldots = \frac{ R }{R + r} = 0{,}30
    \end{align*}
}
\solutionspace{180pt}

\tasknumber{6}%
\task{%
    Лампочки, сопротивления которых $R_1 = 0{,}50\,\text{Ом}$ и $R_2 = 4{,}50\,\text{Ом}$, поочерёдно подключённные к некоторому источнику тока,
    потребляют одинаковую мощность.
    Найти внутреннее сопротивление источника и КПД цепи в каждом случае.
}
\answer{%
    \begin{align*}
        P_1 &= \sqr{\frac{ \mathcal{E} }{R_1 + r}}R_1,
        P_2  = \sqr{\frac{ \mathcal{E} }{R_2 + r}}R_2,
        P_1 = P_2 \implies  \\
        &\implies R_1 \sqr{R_2 + r} = R_2 \sqr{R_1 + r} \implies  \\
        &\implies R_1 R_2^2 + 2 R_1 R_2 r + R_1 r^2 =
                    R_2 R_1^2 + 2 R_2 R_1 r + R_2 r^2  \implies  \\
    &\implies r^2 (R_2 - R_1) = R_2^2 R_2 - R_1^2 R_2 \implies  \\
    &\implies r
            = \sqrt{R_1 R_2 \frac{R_2 - R_1}{R_2 - R_1}}
            = \sqrt{R_1 R_2}
            = \sqrt{0{,}50\,\text{Ом} \cdot 4{,}50\,\text{Ом}}
            = 1{,}5\,\text{Ом}.
            \\
    \eta_1
            &= \frac{ R_1 }{R_1 + r}
            = \frac{\sqrt{ R_1 }}{\sqrt{ R_1 } + \sqrt{ R_2 }}
            = 0{,}250,  \\
    \eta_2
            &= \frac{ R_2 }{R_2 + r}
            = \frac{ \sqrt{ R_2 } }{\sqrt{ R_2 } + \sqrt{ R_1 }}
            = 0{,}750
    \end{align*}
}

\variantsplitter

\addpersonalvariant{Вячеслав Волохов}

\tasknumber{1}%
\task{%
    Напротив физических величин укажите их обозначения и единицы измерения в СИ:
    \begin{enumerate}
        \item разность потенциалов,
        \item мощность тока,
        \item удельное сопротивление,
        \item внутреннее сопротивление полной цепи.
    \end{enumerate}
}
\solutionspace{20pt}

\tasknumber{2}%
\task{%
    Запишите физический закон или формулу:
    \begin{enumerate}
        \item правило Кирхгофа для узла цепи,
        \item сопротивление резистора через удельное сопротивление,
        \item закон Ома для неоднородного участка цепи.
    \end{enumerate}
}
\solutionspace{40pt}

\tasknumber{3}%
\task{%
    На резистор сопротивлением $r = 30\,\text{Ом}$ подали напряжение $V = 120\,\text{В}$.
    Определите ток, который потечёт через резистор, и мощность, выделяющуюся на нём.
}
\answer{%
    \begin{align*}
    \mathcal{I} &= \frac{ V }{ r } = \frac{ 120\,\text{В} }{ 30\,\text{Ом} } = 4{,}00\,\text{А},  \\
    P &= \frac{V^2}{ r } = \frac{ \sqr{ 120\,\text{В} } }{ 30\,\text{Ом} } = 480{,}00\,\text{Вт}
    \end{align*}
}
\solutionspace{60pt}

\tasknumber{4}%
\task{%
    Через резистор сопротивлением $r = 5\,\text{Ом}$ протекает электрический ток $\mathcal{I} = 15{,}00\,\text{А}$.
    Определите, чему равны напряжение на резисторе и мощность, выделяющаяся на нём.
}
\answer{%
    \begin{align*}
    U &= \mathcal{I}r = 15{,}00\,\text{А} \cdot 5\,\text{Ом} = 75\,\text{В},  \\
    P &= \mathcal{I}^2r = \sqr{ 15{,}00\,\text{А} } \cdot 5\,\text{Ом} = 1125\,\text{Вт}
    \end{align*}
}
\solutionspace{60pt}

\tasknumber{5}%
\task{%
    Замкнутая электрическая цепь состоит из ЭДС $\mathcal{E} = 2\,\text{В}$ и сопротивлением $r$
    и резистора $R = 24\,\text{Ом}$.
    Определите ток, протекающий в цепи.
    Какая тепловая энергия выделится на резисторе за время
    $\tau = 5\,\text{с}$? Какая работа будет совершена ЭДС за это время? Каков знак этой работы? Чему равен КПД цепи?
    Вычислите значения для 2 случаев: $r=0$ и $r = 10\,\text{Ом}$.
}
\answer{%
    \begin{align*}
    \mathcal{I}_1 &= \frac{ \mathcal{E} }{ R } = \frac{ 2\,\text{В} }{ 24\,\text{Ом} } = 0{,}08\,\text{А},  \\
    \mathcal{I}_2 &= \frac{ \mathcal{E} }{R + r} = \frac{ 2\,\text{В} }{24\,\text{Ом} + 10\,\text{Ом}} = 0{,}06\,\text{А},  \\
    Q_1 &= \mathcal{I}_1^2R\tau = \sqr{\frac{ \mathcal{E} }{ R }} R \tau
            = \sqr{\frac{ 2\,\text{В} }{ 24\,\text{Ом} }} \cdot 24\,\text{Ом} \cdot 5\,\text{с} = 0{,}768\,\text{Дж},  \\
    Q_2 &= \mathcal{I}_2^2R\tau = \sqr{\frac{ \mathcal{E} }{R + r}} R \tau
            = \sqr{\frac{ 2\,\text{В} }{24\,\text{Ом} + 10\,\text{Ом}}} \cdot 24\,\text{Ом} \cdot 5\,\text{с} = 0{,}432\,\text{Дж},  \\
    A_1 &= \mathcal{I}_1\tau\mathcal{E} = \frac{ \mathcal{E} }{R} \tau \mathcal{E}
            = \frac{\mathcal{E}^2 \tau}{ R } = \frac{\sqr{ 2\,\text{В} } \cdot 5\,\text{с}}{ 24\,\text{Ом} }
            = 0{,}800\,\text{Дж}, \text{положительна},  \\
    A_2 &= \mathcal{I}_2\tau\mathcal{E} = \frac{ \mathcal{E} }{R + r} \tau \mathcal{E}
            = \frac{\mathcal{E}^2 \tau}{R + r} = \frac{\sqr{ 2\,\text{В} } \cdot 5\,\text{с}}{24\,\text{Ом} + 10\,\text{Ом}}
            = 0{,}600\,\text{Дж}, \text{положительна},  \\
    \eta_1 &= \frac{ Q_1 }{ A_1 } = \ldots = \frac{ R }{ R } = 1,  \\
    \eta_2 &= \frac{ Q_2 }{ A_2 } = \ldots = \frac{ R }{R + r} = 0{,}72
    \end{align*}
}
\solutionspace{180pt}

\tasknumber{6}%
\task{%
    Лампочки, сопротивления которых $R_1 = 0{,}50\,\text{Ом}$ и $R_2 = 2{,}00\,\text{Ом}$, поочерёдно подключённные к некоторому источнику тока,
    потребляют одинаковую мощность.
    Найти внутреннее сопротивление источника и КПД цепи в каждом случае.
}
\answer{%
    \begin{align*}
        P_1 &= \sqr{\frac{ \mathcal{E} }{R_1 + r}}R_1,
        P_2  = \sqr{\frac{ \mathcal{E} }{R_2 + r}}R_2,
        P_1 = P_2 \implies  \\
        &\implies R_1 \sqr{R_2 + r} = R_2 \sqr{R_1 + r} \implies  \\
        &\implies R_1 R_2^2 + 2 R_1 R_2 r + R_1 r^2 =
                    R_2 R_1^2 + 2 R_2 R_1 r + R_2 r^2  \implies  \\
    &\implies r^2 (R_2 - R_1) = R_2^2 R_2 - R_1^2 R_2 \implies  \\
    &\implies r
            = \sqrt{R_1 R_2 \frac{R_2 - R_1}{R_2 - R_1}}
            = \sqrt{R_1 R_2}
            = \sqrt{0{,}50\,\text{Ом} \cdot 2{,}00\,\text{Ом}}
            = 1{,}0\,\text{Ом}.
            \\
    \eta_1
            &= \frac{ R_1 }{R_1 + r}
            = \frac{\sqrt{ R_1 }}{\sqrt{ R_1 } + \sqrt{ R_2 }}
            = 0{,}333,  \\
    \eta_2
            &= \frac{ R_2 }{R_2 + r}
            = \frac{ \sqrt{ R_2 } }{\sqrt{ R_2 } + \sqrt{ R_1 }}
            = 0{,}667
    \end{align*}
}

\variantsplitter

\addpersonalvariant{Герман Говоров}

\tasknumber{1}%
\task{%
    Напротив физических величин укажите их обозначения и единицы измерения в СИ:
    \begin{enumerate}
        \item напряжение,
        \item работа тока,
        \item удельное сопротивление,
        \item внешнее сопротивление полной цепи.
    \end{enumerate}
}
\solutionspace{20pt}

\tasknumber{2}%
\task{%
    Запишите физический закон или формулу:
    \begin{enumerate}
        \item правило Кирхгофа для замкнутого контура,
        \item закон Ома для однородного участка цепи,
        \item ЭДС (определение).
    \end{enumerate}
}
\solutionspace{40pt}

\tasknumber{3}%
\task{%
    На резистор сопротивлением $r = 5\,\text{Ом}$ подали напряжение $V = 240\,\text{В}$.
    Определите ток, который потечёт через резистор, и мощность, выделяющуюся на нём.
}
\answer{%
    \begin{align*}
    \mathcal{I} &= \frac{ V }{ r } = \frac{ 240\,\text{В} }{ 5\,\text{Ом} } = 48{,}00\,\text{А},  \\
    P &= \frac{V^2}{ r } = \frac{ \sqr{ 240\,\text{В} } }{ 5\,\text{Ом} } = 11520{,}00\,\text{Вт}
    \end{align*}
}
\solutionspace{60pt}

\tasknumber{4}%
\task{%
    Через резистор сопротивлением $R = 30\,\text{Ом}$ протекает электрический ток $\mathcal{I} = 10{,}00\,\text{А}$.
    Определите, чему равны напряжение на резисторе и мощность, выделяющаяся на нём.
}
\answer{%
    \begin{align*}
    U &= \mathcal{I}R = 10{,}00\,\text{А} \cdot 30\,\text{Ом} = 300\,\text{В},  \\
    P &= \mathcal{I}^2R = \sqr{ 10{,}00\,\text{А} } \cdot 30\,\text{Ом} = 3000\,\text{Вт}
    \end{align*}
}
\solutionspace{60pt}

\tasknumber{5}%
\task{%
    Замкнутая электрическая цепь состоит из ЭДС $\mathcal{E} = 2\,\text{В}$ и сопротивлением $r$
    и резистора $R = 30\,\text{Ом}$.
    Определите ток, протекающий в цепи.
    Какая тепловая энергия выделится на резисторе за время
    $\tau = 2\,\text{с}$? Какая работа будет совершена ЭДС за это время? Каков знак этой работы? Чему равен КПД цепи?
    Вычислите значения для 2 случаев: $r=0$ и $r = 60\,\text{Ом}$.
}
\answer{%
    \begin{align*}
    \mathcal{I}_1 &= \frac{ \mathcal{E} }{ R } = \frac{ 2\,\text{В} }{ 30\,\text{Ом} } = 0{,}07\,\text{А},  \\
    \mathcal{I}_2 &= \frac{ \mathcal{E} }{R + r} = \frac{ 2\,\text{В} }{30\,\text{Ом} + 60\,\text{Ом}} = 0{,}02\,\text{А},  \\
    Q_1 &= \mathcal{I}_1^2R\tau = \sqr{\frac{ \mathcal{E} }{ R }} R \tau
            = \sqr{\frac{ 2\,\text{В} }{ 30\,\text{Ом} }} \cdot 30\,\text{Ом} \cdot 2\,\text{с} = 0{,}294\,\text{Дж},  \\
    Q_2 &= \mathcal{I}_2^2R\tau = \sqr{\frac{ \mathcal{E} }{R + r}} R \tau
            = \sqr{\frac{ 2\,\text{В} }{30\,\text{Ом} + 60\,\text{Ом}}} \cdot 30\,\text{Ом} \cdot 2\,\text{с} = 0{,}024\,\text{Дж},  \\
    A_1 &= \mathcal{I}_1\tau\mathcal{E} = \frac{ \mathcal{E} }{R} \tau \mathcal{E}
            = \frac{\mathcal{E}^2 \tau}{ R } = \frac{\sqr{ 2\,\text{В} } \cdot 2\,\text{с}}{ 30\,\text{Ом} }
            = 0{,}280\,\text{Дж}, \text{положительна},  \\
    A_2 &= \mathcal{I}_2\tau\mathcal{E} = \frac{ \mathcal{E} }{R + r} \tau \mathcal{E}
            = \frac{\mathcal{E}^2 \tau}{R + r} = \frac{\sqr{ 2\,\text{В} } \cdot 2\,\text{с}}{30\,\text{Ом} + 60\,\text{Ом}}
            = 0{,}080\,\text{Дж}, \text{положительна},  \\
    \eta_1 &= \frac{ Q_1 }{ A_1 } = \ldots = \frac{ R }{ R } = 1,  \\
    \eta_2 &= \frac{ Q_2 }{ A_2 } = \ldots = \frac{ R }{R + r} = 0{,}30
    \end{align*}
}
\solutionspace{180pt}

\tasknumber{6}%
\task{%
    Лампочки, сопротивления которых $R_1 = 0{,}50\,\text{Ом}$ и $R_2 = 18{,}00\,\text{Ом}$, поочерёдно подключённные к некоторому источнику тока,
    потребляют одинаковую мощность.
    Найти внутреннее сопротивление источника и КПД цепи в каждом случае.
}
\answer{%
    \begin{align*}
        P_1 &= \sqr{\frac{ \mathcal{E} }{R_1 + r}}R_1,
        P_2  = \sqr{\frac{ \mathcal{E} }{R_2 + r}}R_2,
        P_1 = P_2 \implies  \\
        &\implies R_1 \sqr{R_2 + r} = R_2 \sqr{R_1 + r} \implies  \\
        &\implies R_1 R_2^2 + 2 R_1 R_2 r + R_1 r^2 =
                    R_2 R_1^2 + 2 R_2 R_1 r + R_2 r^2  \implies  \\
    &\implies r^2 (R_2 - R_1) = R_2^2 R_2 - R_1^2 R_2 \implies  \\
    &\implies r
            = \sqrt{R_1 R_2 \frac{R_2 - R_1}{R_2 - R_1}}
            = \sqrt{R_1 R_2}
            = \sqrt{0{,}50\,\text{Ом} \cdot 18{,}00\,\text{Ом}}
            = 3{,}0\,\text{Ом}.
            \\
    \eta_1
            &= \frac{ R_1 }{R_1 + r}
            = \frac{\sqrt{ R_1 }}{\sqrt{ R_1 } + \sqrt{ R_2 }}
            = 0{,}143,  \\
    \eta_2
            &= \frac{ R_2 }{R_2 + r}
            = \frac{ \sqrt{ R_2 } }{\sqrt{ R_2 } + \sqrt{ R_1 }}
            = 0{,}857
    \end{align*}
}

\variantsplitter

\addpersonalvariant{София Журавлёва}

\tasknumber{1}%
\task{%
    Напротив физических величин укажите их обозначения и единицы измерения в СИ:
    \begin{enumerate}
        \item сила тока,
        \item мощность тока,
        \item ЭДС,
        \item внутреннее сопротивление полной цепи.
    \end{enumerate}
}
\solutionspace{20pt}

\tasknumber{2}%
\task{%
    Запишите физический закон или формулу:
    \begin{enumerate}
        \item правило Кирхгофа для узла цепи,
        \item сопротивление резистора через удельное сопротивление,
        \item ЭДС (определение).
    \end{enumerate}
}
\solutionspace{40pt}

\tasknumber{3}%
\task{%
    На резистор сопротивлением $r = 18\,\text{Ом}$ подали напряжение $U = 240\,\text{В}$.
    Определите ток, который потечёт через резистор, и мощность, выделяющуюся на нём.
}
\answer{%
    \begin{align*}
    \mathcal{I} &= \frac{ U }{ r } = \frac{ 240\,\text{В} }{ 18\,\text{Ом} } = 13{,}33\,\text{А},  \\
    P &= \frac{U^2}{ r } = \frac{ \sqr{ 240\,\text{В} } }{ 18\,\text{Ом} } = 3200{,}00\,\text{Вт}
    \end{align*}
}
\solutionspace{60pt}

\tasknumber{4}%
\task{%
    Через резистор сопротивлением $r = 30\,\text{Ом}$ протекает электрический ток $\mathcal{I} = 5{,}00\,\text{А}$.
    Определите, чему равны напряжение на резисторе и мощность, выделяющаяся на нём.
}
\answer{%
    \begin{align*}
    U &= \mathcal{I}r = 5{,}00\,\text{А} \cdot 30\,\text{Ом} = 150\,\text{В},  \\
    P &= \mathcal{I}^2r = \sqr{ 5{,}00\,\text{А} } \cdot 30\,\text{Ом} = 750\,\text{Вт}
    \end{align*}
}
\solutionspace{60pt}

\tasknumber{5}%
\task{%
    Замкнутая электрическая цепь состоит из ЭДС $\mathcal{E} = 4\,\text{В}$ и сопротивлением $r$
    и резистора $R = 24\,\text{Ом}$.
    Определите ток, протекающий в цепи.
    Какая тепловая энергия выделится на резисторе за время
    $\tau = 5\,\text{с}$? Какая работа будет совершена ЭДС за это время? Каков знак этой работы? Чему равен КПД цепи?
    Вычислите значения для 2 случаев: $r=0$ и $r = 30\,\text{Ом}$.
}
\answer{%
    \begin{align*}
    \mathcal{I}_1 &= \frac{ \mathcal{E} }{ R } = \frac{ 4\,\text{В} }{ 24\,\text{Ом} } = 0{,}17\,\text{А},  \\
    \mathcal{I}_2 &= \frac{ \mathcal{E} }{R + r} = \frac{ 4\,\text{В} }{24\,\text{Ом} + 30\,\text{Ом}} = 0{,}07\,\text{А},  \\
    Q_1 &= \mathcal{I}_1^2R\tau = \sqr{\frac{ \mathcal{E} }{ R }} R \tau
            = \sqr{\frac{ 4\,\text{В} }{ 24\,\text{Ом} }} \cdot 24\,\text{Ом} \cdot 5\,\text{с} = 3{,}468\,\text{Дж},  \\
    Q_2 &= \mathcal{I}_2^2R\tau = \sqr{\frac{ \mathcal{E} }{R + r}} R \tau
            = \sqr{\frac{ 4\,\text{В} }{24\,\text{Ом} + 30\,\text{Ом}}} \cdot 24\,\text{Ом} \cdot 5\,\text{с} = 0{,}588\,\text{Дж},  \\
    A_1 &= \mathcal{I}_1\tau\mathcal{E} = \frac{ \mathcal{E} }{R} \tau \mathcal{E}
            = \frac{\mathcal{E}^2 \tau}{ R } = \frac{\sqr{ 4\,\text{В} } \cdot 5\,\text{с}}{ 24\,\text{Ом} }
            = 3{,}400\,\text{Дж}, \text{положительна},  \\
    A_2 &= \mathcal{I}_2\tau\mathcal{E} = \frac{ \mathcal{E} }{R + r} \tau \mathcal{E}
            = \frac{\mathcal{E}^2 \tau}{R + r} = \frac{\sqr{ 4\,\text{В} } \cdot 5\,\text{с}}{24\,\text{Ом} + 30\,\text{Ом}}
            = 1{,}400\,\text{Дж}, \text{положительна},  \\
    \eta_1 &= \frac{ Q_1 }{ A_1 } = \ldots = \frac{ R }{ R } = 1,  \\
    \eta_2 &= \frac{ Q_2 }{ A_2 } = \ldots = \frac{ R }{R + r} = 0{,}42
    \end{align*}
}
\solutionspace{180pt}

\tasknumber{6}%
\task{%
    Лампочки, сопротивления которых $R_1 = 0{,}50\,\text{Ом}$ и $R_2 = 2{,}00\,\text{Ом}$, поочерёдно подключённные к некоторому источнику тока,
    потребляют одинаковую мощность.
    Найти внутреннее сопротивление источника и КПД цепи в каждом случае.
}
\answer{%
    \begin{align*}
        P_1 &= \sqr{\frac{ \mathcal{E} }{R_1 + r}}R_1,
        P_2  = \sqr{\frac{ \mathcal{E} }{R_2 + r}}R_2,
        P_1 = P_2 \implies  \\
        &\implies R_1 \sqr{R_2 + r} = R_2 \sqr{R_1 + r} \implies  \\
        &\implies R_1 R_2^2 + 2 R_1 R_2 r + R_1 r^2 =
                    R_2 R_1^2 + 2 R_2 R_1 r + R_2 r^2  \implies  \\
    &\implies r^2 (R_2 - R_1) = R_2^2 R_2 - R_1^2 R_2 \implies  \\
    &\implies r
            = \sqrt{R_1 R_2 \frac{R_2 - R_1}{R_2 - R_1}}
            = \sqrt{R_1 R_2}
            = \sqrt{0{,}50\,\text{Ом} \cdot 2{,}00\,\text{Ом}}
            = 1{,}0\,\text{Ом}.
            \\
    \eta_1
            &= \frac{ R_1 }{R_1 + r}
            = \frac{\sqrt{ R_1 }}{\sqrt{ R_1 } + \sqrt{ R_2 }}
            = 0{,}333,  \\
    \eta_2
            &= \frac{ R_2 }{R_2 + r}
            = \frac{ \sqrt{ R_2 } }{\sqrt{ R_2 } + \sqrt{ R_1 }}
            = 0{,}667
    \end{align*}
}

\variantsplitter

\addpersonalvariant{Константин Козлов}

\tasknumber{1}%
\task{%
    Напротив физических величин укажите их обозначения и единицы измерения в СИ:
    \begin{enumerate}
        \item разность потенциалов,
        \item мощность тока,
        \item удельное сопротивление,
        \item внешнее сопротивление полной цепи.
    \end{enumerate}
}
\solutionspace{20pt}

\tasknumber{2}%
\task{%
    Запишите физический закон или формулу:
    \begin{enumerate}
        \item правило Кирхгофа для узла цепи,
        \item сопротивление резистора через удельное сопротивление,
        \item закон Ома для неоднородного участка цепи.
    \end{enumerate}
}
\solutionspace{40pt}

\tasknumber{3}%
\task{%
    На резистор сопротивлением $r = 12\,\text{Ом}$ подали напряжение $U = 150\,\text{В}$.
    Определите ток, который потечёт через резистор, и мощность, выделяющуюся на нём.
}
\answer{%
    \begin{align*}
    \mathcal{I} &= \frac{ U }{ r } = \frac{ 150\,\text{В} }{ 12\,\text{Ом} } = 12{,}50\,\text{А},  \\
    P &= \frac{U^2}{ r } = \frac{ \sqr{ 150\,\text{В} } }{ 12\,\text{Ом} } = 1875{,}00\,\text{Вт}
    \end{align*}
}
\solutionspace{60pt}

\tasknumber{4}%
\task{%
    Через резистор сопротивлением $R = 30\,\text{Ом}$ протекает электрический ток $\mathcal{I} = 10{,}00\,\text{А}$.
    Определите, чему равны напряжение на резисторе и мощность, выделяющаяся на нём.
}
\answer{%
    \begin{align*}
    U &= \mathcal{I}R = 10{,}00\,\text{А} \cdot 30\,\text{Ом} = 300\,\text{В},  \\
    P &= \mathcal{I}^2R = \sqr{ 10{,}00\,\text{А} } \cdot 30\,\text{Ом} = 3000\,\text{Вт}
    \end{align*}
}
\solutionspace{60pt}

\tasknumber{5}%
\task{%
    Замкнутая электрическая цепь состоит из ЭДС $\mathcal{E} = 3\,\text{В}$ и сопротивлением $r$
    и резистора $R = 24\,\text{Ом}$.
    Определите ток, протекающий в цепи.
    Какая тепловая энергия выделится на резисторе за время
    $\tau = 2\,\text{с}$? Какая работа будет совершена ЭДС за это время? Каков знак этой работы? Чему равен КПД цепи?
    Вычислите значения для 2 случаев: $r=0$ и $r = 30\,\text{Ом}$.
}
\answer{%
    \begin{align*}
    \mathcal{I}_1 &= \frac{ \mathcal{E} }{ R } = \frac{ 3\,\text{В} }{ 24\,\text{Ом} } = 0{,}12\,\text{А},  \\
    \mathcal{I}_2 &= \frac{ \mathcal{E} }{R + r} = \frac{ 3\,\text{В} }{24\,\text{Ом} + 30\,\text{Ом}} = 0{,}06\,\text{А},  \\
    Q_1 &= \mathcal{I}_1^2R\tau = \sqr{\frac{ \mathcal{E} }{ R }} R \tau
            = \sqr{\frac{ 3\,\text{В} }{ 24\,\text{Ом} }} \cdot 24\,\text{Ом} \cdot 2\,\text{с} = 0{,}691\,\text{Дж},  \\
    Q_2 &= \mathcal{I}_2^2R\tau = \sqr{\frac{ \mathcal{E} }{R + r}} R \tau
            = \sqr{\frac{ 3\,\text{В} }{24\,\text{Ом} + 30\,\text{Ом}}} \cdot 24\,\text{Ом} \cdot 2\,\text{с} = 0{,}173\,\text{Дж},  \\
    A_1 &= \mathcal{I}_1\tau\mathcal{E} = \frac{ \mathcal{E} }{R} \tau \mathcal{E}
            = \frac{\mathcal{E}^2 \tau}{ R } = \frac{\sqr{ 3\,\text{В} } \cdot 2\,\text{с}}{ 24\,\text{Ом} }
            = 0{,}720\,\text{Дж}, \text{положительна},  \\
    A_2 &= \mathcal{I}_2\tau\mathcal{E} = \frac{ \mathcal{E} }{R + r} \tau \mathcal{E}
            = \frac{\mathcal{E}^2 \tau}{R + r} = \frac{\sqr{ 3\,\text{В} } \cdot 2\,\text{с}}{24\,\text{Ом} + 30\,\text{Ом}}
            = 0{,}360\,\text{Дж}, \text{положительна},  \\
    \eta_1 &= \frac{ Q_1 }{ A_1 } = \ldots = \frac{ R }{ R } = 1,  \\
    \eta_2 &= \frac{ Q_2 }{ A_2 } = \ldots = \frac{ R }{R + r} = 0{,}48
    \end{align*}
}
\solutionspace{180pt}

\tasknumber{6}%
\task{%
    Лампочки, сопротивления которых $R_1 = 4{,}00\,\text{Ом}$ и $R_2 = 36{,}00\,\text{Ом}$, поочерёдно подключённные к некоторому источнику тока,
    потребляют одинаковую мощность.
    Найти внутреннее сопротивление источника и КПД цепи в каждом случае.
}
\answer{%
    \begin{align*}
        P_1 &= \sqr{\frac{ \mathcal{E} }{R_1 + r}}R_1,
        P_2  = \sqr{\frac{ \mathcal{E} }{R_2 + r}}R_2,
        P_1 = P_2 \implies  \\
        &\implies R_1 \sqr{R_2 + r} = R_2 \sqr{R_1 + r} \implies  \\
        &\implies R_1 R_2^2 + 2 R_1 R_2 r + R_1 r^2 =
                    R_2 R_1^2 + 2 R_2 R_1 r + R_2 r^2  \implies  \\
    &\implies r^2 (R_2 - R_1) = R_2^2 R_2 - R_1^2 R_2 \implies  \\
    &\implies r
            = \sqrt{R_1 R_2 \frac{R_2 - R_1}{R_2 - R_1}}
            = \sqrt{R_1 R_2}
            = \sqrt{4{,}00\,\text{Ом} \cdot 36{,}00\,\text{Ом}}
            = 12{,}0\,\text{Ом}.
            \\
    \eta_1
            &= \frac{ R_1 }{R_1 + r}
            = \frac{\sqrt{ R_1 }}{\sqrt{ R_1 } + \sqrt{ R_2 }}
            = 0{,}250,  \\
    \eta_2
            &= \frac{ R_2 }{R_2 + r}
            = \frac{ \sqrt{ R_2 } }{\sqrt{ R_2 } + \sqrt{ R_1 }}
            = 0{,}750
    \end{align*}
}

\variantsplitter

\addpersonalvariant{Наталья Кравченко}

\tasknumber{1}%
\task{%
    Напротив физических величин укажите их обозначения и единицы измерения в СИ:
    \begin{enumerate}
        \item напряжение,
        \item работа тока,
        \item ЭДС,
        \item внутреннее сопротивление полной цепи.
    \end{enumerate}
}
\solutionspace{20pt}

\tasknumber{2}%
\task{%
    Запишите физический закон или формулу:
    \begin{enumerate}
        \item правило Кирхгофа для замкнутого контура,
        \item сопротивление резистора через удельное сопротивление,
        \item ЭДС (определение).
    \end{enumerate}
}
\solutionspace{40pt}

\tasknumber{3}%
\task{%
    На резистор сопротивлением $R = 30\,\text{Ом}$ подали напряжение $U = 180\,\text{В}$.
    Определите ток, который потечёт через резистор, и мощность, выделяющуюся на нём.
}
\answer{%
    \begin{align*}
    \mathcal{I} &= \frac{ U }{ R } = \frac{ 180\,\text{В} }{ 30\,\text{Ом} } = 6{,}00\,\text{А},  \\
    P &= \frac{U^2}{ R } = \frac{ \sqr{ 180\,\text{В} } }{ 30\,\text{Ом} } = 1080{,}00\,\text{Вт}
    \end{align*}
}
\solutionspace{60pt}

\tasknumber{4}%
\task{%
    Через резистор сопротивлением $R = 5\,\text{Ом}$ протекает электрический ток $\mathcal{I} = 5{,}00\,\text{А}$.
    Определите, чему равны напряжение на резисторе и мощность, выделяющаяся на нём.
}
\answer{%
    \begin{align*}
    U &= \mathcal{I}R = 5{,}00\,\text{А} \cdot 5\,\text{Ом} = 25\,\text{В},  \\
    P &= \mathcal{I}^2R = \sqr{ 5{,}00\,\text{А} } \cdot 5\,\text{Ом} = 125\,\text{Вт}
    \end{align*}
}
\solutionspace{60pt}

\tasknumber{5}%
\task{%
    Замкнутая электрическая цепь состоит из ЭДС $\mathcal{E} = 2\,\text{В}$ и сопротивлением $r$
    и резистора $R = 15\,\text{Ом}$.
    Определите ток, протекающий в цепи.
    Какая тепловая энергия выделится на резисторе за время
    $\tau = 10\,\text{с}$? Какая работа будет совершена ЭДС за это время? Каков знак этой работы? Чему равен КПД цепи?
    Вычислите значения для 2 случаев: $r=0$ и $r = 30\,\text{Ом}$.
}
\answer{%
    \begin{align*}
    \mathcal{I}_1 &= \frac{ \mathcal{E} }{ R } = \frac{ 2\,\text{В} }{ 15\,\text{Ом} } = 0{,}13\,\text{А},  \\
    \mathcal{I}_2 &= \frac{ \mathcal{E} }{R + r} = \frac{ 2\,\text{В} }{15\,\text{Ом} + 30\,\text{Ом}} = 0{,}04\,\text{А},  \\
    Q_1 &= \mathcal{I}_1^2R\tau = \sqr{\frac{ \mathcal{E} }{ R }} R \tau
            = \sqr{\frac{ 2\,\text{В} }{ 15\,\text{Ом} }} \cdot 15\,\text{Ом} \cdot 10\,\text{с} = 2{,}535\,\text{Дж},  \\
    Q_2 &= \mathcal{I}_2^2R\tau = \sqr{\frac{ \mathcal{E} }{R + r}} R \tau
            = \sqr{\frac{ 2\,\text{В} }{15\,\text{Ом} + 30\,\text{Ом}}} \cdot 15\,\text{Ом} \cdot 10\,\text{с} = 0{,}240\,\text{Дж},  \\
    A_1 &= \mathcal{I}_1\tau\mathcal{E} = \frac{ \mathcal{E} }{R} \tau \mathcal{E}
            = \frac{\mathcal{E}^2 \tau}{ R } = \frac{\sqr{ 2\,\text{В} } \cdot 10\,\text{с}}{ 15\,\text{Ом} }
            = 2{,}600\,\text{Дж}, \text{положительна},  \\
    A_2 &= \mathcal{I}_2\tau\mathcal{E} = \frac{ \mathcal{E} }{R + r} \tau \mathcal{E}
            = \frac{\mathcal{E}^2 \tau}{R + r} = \frac{\sqr{ 2\,\text{В} } \cdot 10\,\text{с}}{15\,\text{Ом} + 30\,\text{Ом}}
            = 0{,}800\,\text{Дж}, \text{положительна},  \\
    \eta_1 &= \frac{ Q_1 }{ A_1 } = \ldots = \frac{ R }{ R } = 1,  \\
    \eta_2 &= \frac{ Q_2 }{ A_2 } = \ldots = \frac{ R }{R + r} = 0{,}30
    \end{align*}
}
\solutionspace{180pt}

\tasknumber{6}%
\task{%
    Лампочки, сопротивления которых $R_1 = 6{,}00\,\text{Ом}$ и $R_2 = 24{,}00\,\text{Ом}$, поочерёдно подключённные к некоторому источнику тока,
    потребляют одинаковую мощность.
    Найти внутреннее сопротивление источника и КПД цепи в каждом случае.
}
\answer{%
    \begin{align*}
        P_1 &= \sqr{\frac{ \mathcal{E} }{R_1 + r}}R_1,
        P_2  = \sqr{\frac{ \mathcal{E} }{R_2 + r}}R_2,
        P_1 = P_2 \implies  \\
        &\implies R_1 \sqr{R_2 + r} = R_2 \sqr{R_1 + r} \implies  \\
        &\implies R_1 R_2^2 + 2 R_1 R_2 r + R_1 r^2 =
                    R_2 R_1^2 + 2 R_2 R_1 r + R_2 r^2  \implies  \\
    &\implies r^2 (R_2 - R_1) = R_2^2 R_2 - R_1^2 R_2 \implies  \\
    &\implies r
            = \sqrt{R_1 R_2 \frac{R_2 - R_1}{R_2 - R_1}}
            = \sqrt{R_1 R_2}
            = \sqrt{6{,}00\,\text{Ом} \cdot 24{,}00\,\text{Ом}}
            = 12{,}0\,\text{Ом}.
            \\
    \eta_1
            &= \frac{ R_1 }{R_1 + r}
            = \frac{\sqrt{ R_1 }}{\sqrt{ R_1 } + \sqrt{ R_2 }}
            = 0{,}333,  \\
    \eta_2
            &= \frac{ R_2 }{R_2 + r}
            = \frac{ \sqrt{ R_2 } }{\sqrt{ R_2 } + \sqrt{ R_1 }}
            = 0{,}667
    \end{align*}
}

\variantsplitter

\addpersonalvariant{Матвей Кузьмин}

\tasknumber{1}%
\task{%
    Напротив физических величин укажите их обозначения и единицы измерения в СИ:
    \begin{enumerate}
        \item разность потенциалов,
        \item мощность тока,
        \item ЭДС,
        \item внутреннее сопротивление полной цепи.
    \end{enumerate}
}
\solutionspace{20pt}

\tasknumber{2}%
\task{%
    Запишите физический закон или формулу:
    \begin{enumerate}
        \item правило Кирхгофа для узла цепи,
        \item сопротивление резистора через удельное сопротивление,
        \item закон Ома для неоднородного участка цепи.
    \end{enumerate}
}
\solutionspace{40pt}

\tasknumber{3}%
\task{%
    На резистор сопротивлением $R = 30\,\text{Ом}$ подали напряжение $U = 240\,\text{В}$.
    Определите ток, который потечёт через резистор, и мощность, выделяющуюся на нём.
}
\answer{%
    \begin{align*}
    \mathcal{I} &= \frac{ U }{ R } = \frac{ 240\,\text{В} }{ 30\,\text{Ом} } = 8{,}00\,\text{А},  \\
    P &= \frac{U^2}{ R } = \frac{ \sqr{ 240\,\text{В} } }{ 30\,\text{Ом} } = 1920{,}00\,\text{Вт}
    \end{align*}
}
\solutionspace{60pt}

\tasknumber{4}%
\task{%
    Через резистор сопротивлением $R = 30\,\text{Ом}$ протекает электрический ток $\mathcal{I} = 8{,}00\,\text{А}$.
    Определите, чему равны напряжение на резисторе и мощность, выделяющаяся на нём.
}
\answer{%
    \begin{align*}
    U &= \mathcal{I}R = 8{,}00\,\text{А} \cdot 30\,\text{Ом} = 240\,\text{В},  \\
    P &= \mathcal{I}^2R = \sqr{ 8{,}00\,\text{А} } \cdot 30\,\text{Ом} = 1920\,\text{Вт}
    \end{align*}
}
\solutionspace{60pt}

\tasknumber{5}%
\task{%
    Замкнутая электрическая цепь состоит из ЭДС $\mathcal{E} = 1\,\text{В}$ и сопротивлением $r$
    и резистора $R = 15\,\text{Ом}$.
    Определите ток, протекающий в цепи.
    Какая тепловая энергия выделится на резисторе за время
    $\tau = 10\,\text{с}$? Какая работа будет совершена ЭДС за это время? Каков знак этой работы? Чему равен КПД цепи?
    Вычислите значения для 2 случаев: $r=0$ и $r = 20\,\text{Ом}$.
}
\answer{%
    \begin{align*}
    \mathcal{I}_1 &= \frac{ \mathcal{E} }{ R } = \frac{ 1\,\text{В} }{ 15\,\text{Ом} } = 0{,}07\,\text{А},  \\
    \mathcal{I}_2 &= \frac{ \mathcal{E} }{R + r} = \frac{ 1\,\text{В} }{15\,\text{Ом} + 20\,\text{Ом}} = 0{,}03\,\text{А},  \\
    Q_1 &= \mathcal{I}_1^2R\tau = \sqr{\frac{ \mathcal{E} }{ R }} R \tau
            = \sqr{\frac{ 1\,\text{В} }{ 15\,\text{Ом} }} \cdot 15\,\text{Ом} \cdot 10\,\text{с} = 0{,}735\,\text{Дж},  \\
    Q_2 &= \mathcal{I}_2^2R\tau = \sqr{\frac{ \mathcal{E} }{R + r}} R \tau
            = \sqr{\frac{ 1\,\text{В} }{15\,\text{Ом} + 20\,\text{Ом}}} \cdot 15\,\text{Ом} \cdot 10\,\text{с} = 0{,}135\,\text{Дж},  \\
    A_1 &= \mathcal{I}_1\tau\mathcal{E} = \frac{ \mathcal{E} }{R} \tau \mathcal{E}
            = \frac{\mathcal{E}^2 \tau}{ R } = \frac{\sqr{ 1\,\text{В} } \cdot 10\,\text{с}}{ 15\,\text{Ом} }
            = 0{,}700\,\text{Дж}, \text{положительна},  \\
    A_2 &= \mathcal{I}_2\tau\mathcal{E} = \frac{ \mathcal{E} }{R + r} \tau \mathcal{E}
            = \frac{\mathcal{E}^2 \tau}{R + r} = \frac{\sqr{ 1\,\text{В} } \cdot 10\,\text{с}}{15\,\text{Ом} + 20\,\text{Ом}}
            = 0{,}300\,\text{Дж}, \text{положительна},  \\
    \eta_1 &= \frac{ Q_1 }{ A_1 } = \ldots = \frac{ R }{ R } = 1,  \\
    \eta_2 &= \frac{ Q_2 }{ A_2 } = \ldots = \frac{ R }{R + r} = 0{,}45
    \end{align*}
}
\solutionspace{180pt}

\tasknumber{6}%
\task{%
    Лампочки, сопротивления которых $R_1 = 0{,}25\,\text{Ом}$ и $R_2 = 16{,}00\,\text{Ом}$, поочерёдно подключённные к некоторому источнику тока,
    потребляют одинаковую мощность.
    Найти внутреннее сопротивление источника и КПД цепи в каждом случае.
}
\answer{%
    \begin{align*}
        P_1 &= \sqr{\frac{ \mathcal{E} }{R_1 + r}}R_1,
        P_2  = \sqr{\frac{ \mathcal{E} }{R_2 + r}}R_2,
        P_1 = P_2 \implies  \\
        &\implies R_1 \sqr{R_2 + r} = R_2 \sqr{R_1 + r} \implies  \\
        &\implies R_1 R_2^2 + 2 R_1 R_2 r + R_1 r^2 =
                    R_2 R_1^2 + 2 R_2 R_1 r + R_2 r^2  \implies  \\
    &\implies r^2 (R_2 - R_1) = R_2^2 R_2 - R_1^2 R_2 \implies  \\
    &\implies r
            = \sqrt{R_1 R_2 \frac{R_2 - R_1}{R_2 - R_1}}
            = \sqrt{R_1 R_2}
            = \sqrt{0{,}25\,\text{Ом} \cdot 16{,}00\,\text{Ом}}
            = 2{,}0\,\text{Ом}.
            \\
    \eta_1
            &= \frac{ R_1 }{R_1 + r}
            = \frac{\sqrt{ R_1 }}{\sqrt{ R_1 } + \sqrt{ R_2 }}
            = 0{,}111,  \\
    \eta_2
            &= \frac{ R_2 }{R_2 + r}
            = \frac{ \sqrt{ R_2 } }{\sqrt{ R_2 } + \sqrt{ R_1 }}
            = 0{,}889
    \end{align*}
}

\variantsplitter

\addpersonalvariant{Сергей Малышев}

\tasknumber{1}%
\task{%
    Напротив физических величин укажите их обозначения и единицы измерения в СИ:
    \begin{enumerate}
        \item напряжение,
        \item работа тока,
        \item удельное сопротивление,
        \item внутреннее сопротивление полной цепи.
    \end{enumerate}
}
\solutionspace{20pt}

\tasknumber{2}%
\task{%
    Запишите физический закон или формулу:
    \begin{enumerate}
        \item правило Кирхгофа для узла цепи,
        \item сопротивление резистора через удельное сопротивление,
        \item ЭДС (определение).
    \end{enumerate}
}
\solutionspace{40pt}

\tasknumber{3}%
\task{%
    На резистор сопротивлением $r = 5\,\text{Ом}$ подали напряжение $V = 240\,\text{В}$.
    Определите ток, который потечёт через резистор, и мощность, выделяющуюся на нём.
}
\answer{%
    \begin{align*}
    \mathcal{I} &= \frac{ V }{ r } = \frac{ 240\,\text{В} }{ 5\,\text{Ом} } = 48{,}00\,\text{А},  \\
    P &= \frac{V^2}{ r } = \frac{ \sqr{ 240\,\text{В} } }{ 5\,\text{Ом} } = 11520{,}00\,\text{Вт}
    \end{align*}
}
\solutionspace{60pt}

\tasknumber{4}%
\task{%
    Через резистор сопротивлением $R = 18\,\text{Ом}$ протекает электрический ток $\mathcal{I} = 3{,}00\,\text{А}$.
    Определите, чему равны напряжение на резисторе и мощность, выделяющаяся на нём.
}
\answer{%
    \begin{align*}
    U &= \mathcal{I}R = 3{,}00\,\text{А} \cdot 18\,\text{Ом} = 54\,\text{В},  \\
    P &= \mathcal{I}^2R = \sqr{ 3{,}00\,\text{А} } \cdot 18\,\text{Ом} = 162\,\text{Вт}
    \end{align*}
}
\solutionspace{60pt}

\tasknumber{5}%
\task{%
    Замкнутая электрическая цепь состоит из ЭДС $\mathcal{E} = 4\,\text{В}$ и сопротивлением $r$
    и резистора $R = 24\,\text{Ом}$.
    Определите ток, протекающий в цепи.
    Какая тепловая энергия выделится на резисторе за время
    $\tau = 2\,\text{с}$? Какая работа будет совершена ЭДС за это время? Каков знак этой работы? Чему равен КПД цепи?
    Вычислите значения для 2 случаев: $r=0$ и $r = 20\,\text{Ом}$.
}
\answer{%
    \begin{align*}
    \mathcal{I}_1 &= \frac{ \mathcal{E} }{ R } = \frac{ 4\,\text{В} }{ 24\,\text{Ом} } = 0{,}17\,\text{А},  \\
    \mathcal{I}_2 &= \frac{ \mathcal{E} }{R + r} = \frac{ 4\,\text{В} }{24\,\text{Ом} + 20\,\text{Ом}} = 0{,}09\,\text{А},  \\
    Q_1 &= \mathcal{I}_1^2R\tau = \sqr{\frac{ \mathcal{E} }{ R }} R \tau
            = \sqr{\frac{ 4\,\text{В} }{ 24\,\text{Ом} }} \cdot 24\,\text{Ом} \cdot 2\,\text{с} = 1{,}387\,\text{Дж},  \\
    Q_2 &= \mathcal{I}_2^2R\tau = \sqr{\frac{ \mathcal{E} }{R + r}} R \tau
            = \sqr{\frac{ 4\,\text{В} }{24\,\text{Ом} + 20\,\text{Ом}}} \cdot 24\,\text{Ом} \cdot 2\,\text{с} = 0{,}389\,\text{Дж},  \\
    A_1 &= \mathcal{I}_1\tau\mathcal{E} = \frac{ \mathcal{E} }{R} \tau \mathcal{E}
            = \frac{\mathcal{E}^2 \tau}{ R } = \frac{\sqr{ 4\,\text{В} } \cdot 2\,\text{с}}{ 24\,\text{Ом} }
            = 1{,}360\,\text{Дж}, \text{положительна},  \\
    A_2 &= \mathcal{I}_2\tau\mathcal{E} = \frac{ \mathcal{E} }{R + r} \tau \mathcal{E}
            = \frac{\mathcal{E}^2 \tau}{R + r} = \frac{\sqr{ 4\,\text{В} } \cdot 2\,\text{с}}{24\,\text{Ом} + 20\,\text{Ом}}
            = 0{,}720\,\text{Дж}, \text{положительна},  \\
    \eta_1 &= \frac{ Q_1 }{ A_1 } = \ldots = \frac{ R }{ R } = 1,  \\
    \eta_2 &= \frac{ Q_2 }{ A_2 } = \ldots = \frac{ R }{R + r} = 0{,}54
    \end{align*}
}
\solutionspace{180pt}

\tasknumber{6}%
\task{%
    Лампочки, сопротивления которых $R_1 = 4{,}00\,\text{Ом}$ и $R_2 = 36{,}00\,\text{Ом}$, поочерёдно подключённные к некоторому источнику тока,
    потребляют одинаковую мощность.
    Найти внутреннее сопротивление источника и КПД цепи в каждом случае.
}
\answer{%
    \begin{align*}
        P_1 &= \sqr{\frac{ \mathcal{E} }{R_1 + r}}R_1,
        P_2  = \sqr{\frac{ \mathcal{E} }{R_2 + r}}R_2,
        P_1 = P_2 \implies  \\
        &\implies R_1 \sqr{R_2 + r} = R_2 \sqr{R_1 + r} \implies  \\
        &\implies R_1 R_2^2 + 2 R_1 R_2 r + R_1 r^2 =
                    R_2 R_1^2 + 2 R_2 R_1 r + R_2 r^2  \implies  \\
    &\implies r^2 (R_2 - R_1) = R_2^2 R_2 - R_1^2 R_2 \implies  \\
    &\implies r
            = \sqrt{R_1 R_2 \frac{R_2 - R_1}{R_2 - R_1}}
            = \sqrt{R_1 R_2}
            = \sqrt{4{,}00\,\text{Ом} \cdot 36{,}00\,\text{Ом}}
            = 12{,}0\,\text{Ом}.
            \\
    \eta_1
            &= \frac{ R_1 }{R_1 + r}
            = \frac{\sqrt{ R_1 }}{\sqrt{ R_1 } + \sqrt{ R_2 }}
            = 0{,}250,  \\
    \eta_2
            &= \frac{ R_2 }{R_2 + r}
            = \frac{ \sqrt{ R_2 } }{\sqrt{ R_2 } + \sqrt{ R_1 }}
            = 0{,}750
    \end{align*}
}

\variantsplitter

\addpersonalvariant{Алина Полканова}

\tasknumber{1}%
\task{%
    Напротив физических величин укажите их обозначения и единицы измерения в СИ:
    \begin{enumerate}
        \item напряжение,
        \item работа тока,
        \item ЭДС,
        \item внутреннее сопротивление полной цепи.
    \end{enumerate}
}
\solutionspace{20pt}

\tasknumber{2}%
\task{%
    Запишите физический закон или формулу:
    \begin{enumerate}
        \item правило Кирхгофа для узла цепи,
        \item закон Ома для однородного участка цепи,
        \item ЭДС (определение).
    \end{enumerate}
}
\solutionspace{40pt}

\tasknumber{3}%
\task{%
    На резистор сопротивлением $R = 5\,\text{Ом}$ подали напряжение $V = 120\,\text{В}$.
    Определите ток, который потечёт через резистор, и мощность, выделяющуюся на нём.
}
\answer{%
    \begin{align*}
    \mathcal{I} &= \frac{ V }{ R } = \frac{ 120\,\text{В} }{ 5\,\text{Ом} } = 24{,}00\,\text{А},  \\
    P &= \frac{V^2}{ R } = \frac{ \sqr{ 120\,\text{В} } }{ 5\,\text{Ом} } = 2880{,}00\,\text{Вт}
    \end{align*}
}
\solutionspace{60pt}

\tasknumber{4}%
\task{%
    Через резистор сопротивлением $R = 18\,\text{Ом}$ протекает электрический ток $\mathcal{I} = 8{,}00\,\text{А}$.
    Определите, чему равны напряжение на резисторе и мощность, выделяющаяся на нём.
}
\answer{%
    \begin{align*}
    U &= \mathcal{I}R = 8{,}00\,\text{А} \cdot 18\,\text{Ом} = 144\,\text{В},  \\
    P &= \mathcal{I}^2R = \sqr{ 8{,}00\,\text{А} } \cdot 18\,\text{Ом} = 1152\,\text{Вт}
    \end{align*}
}
\solutionspace{60pt}

\tasknumber{5}%
\task{%
    Замкнутая электрическая цепь состоит из ЭДС $\mathcal{E} = 3\,\text{В}$ и сопротивлением $r$
    и резистора $R = 15\,\text{Ом}$.
    Определите ток, протекающий в цепи.
    Какая тепловая энергия выделится на резисторе за время
    $\tau = 5\,\text{с}$? Какая работа будет совершена ЭДС за это время? Каков знак этой работы? Чему равен КПД цепи?
    Вычислите значения для 2 случаев: $r=0$ и $r = 60\,\text{Ом}$.
}
\answer{%
    \begin{align*}
    \mathcal{I}_1 &= \frac{ \mathcal{E} }{ R } = \frac{ 3\,\text{В} }{ 15\,\text{Ом} } = 0{,}20\,\text{А},  \\
    \mathcal{I}_2 &= \frac{ \mathcal{E} }{R + r} = \frac{ 3\,\text{В} }{15\,\text{Ом} + 60\,\text{Ом}} = 0{,}04\,\text{А},  \\
    Q_1 &= \mathcal{I}_1^2R\tau = \sqr{\frac{ \mathcal{E} }{ R }} R \tau
            = \sqr{\frac{ 3\,\text{В} }{ 15\,\text{Ом} }} \cdot 15\,\text{Ом} \cdot 5\,\text{с} = 3{,}000\,\text{Дж},  \\
    Q_2 &= \mathcal{I}_2^2R\tau = \sqr{\frac{ \mathcal{E} }{R + r}} R \tau
            = \sqr{\frac{ 3\,\text{В} }{15\,\text{Ом} + 60\,\text{Ом}}} \cdot 15\,\text{Ом} \cdot 5\,\text{с} = 0{,}120\,\text{Дж},  \\
    A_1 &= \mathcal{I}_1\tau\mathcal{E} = \frac{ \mathcal{E} }{R} \tau \mathcal{E}
            = \frac{\mathcal{E}^2 \tau}{ R } = \frac{\sqr{ 3\,\text{В} } \cdot 5\,\text{с}}{ 15\,\text{Ом} }
            = 3{,}000\,\text{Дж}, \text{положительна},  \\
    A_2 &= \mathcal{I}_2\tau\mathcal{E} = \frac{ \mathcal{E} }{R + r} \tau \mathcal{E}
            = \frac{\mathcal{E}^2 \tau}{R + r} = \frac{\sqr{ 3\,\text{В} } \cdot 5\,\text{с}}{15\,\text{Ом} + 60\,\text{Ом}}
            = 0{,}600\,\text{Дж}, \text{положительна},  \\
    \eta_1 &= \frac{ Q_1 }{ A_1 } = \ldots = \frac{ R }{ R } = 1,  \\
    \eta_2 &= \frac{ Q_2 }{ A_2 } = \ldots = \frac{ R }{R + r} = 0{,}20
    \end{align*}
}
\solutionspace{180pt}

\tasknumber{6}%
\task{%
    Лампочки, сопротивления которых $R_1 = 0{,}25\,\text{Ом}$ и $R_2 = 16{,}00\,\text{Ом}$, поочерёдно подключённные к некоторому источнику тока,
    потребляют одинаковую мощность.
    Найти внутреннее сопротивление источника и КПД цепи в каждом случае.
}
\answer{%
    \begin{align*}
        P_1 &= \sqr{\frac{ \mathcal{E} }{R_1 + r}}R_1,
        P_2  = \sqr{\frac{ \mathcal{E} }{R_2 + r}}R_2,
        P_1 = P_2 \implies  \\
        &\implies R_1 \sqr{R_2 + r} = R_2 \sqr{R_1 + r} \implies  \\
        &\implies R_1 R_2^2 + 2 R_1 R_2 r + R_1 r^2 =
                    R_2 R_1^2 + 2 R_2 R_1 r + R_2 r^2  \implies  \\
    &\implies r^2 (R_2 - R_1) = R_2^2 R_2 - R_1^2 R_2 \implies  \\
    &\implies r
            = \sqrt{R_1 R_2 \frac{R_2 - R_1}{R_2 - R_1}}
            = \sqrt{R_1 R_2}
            = \sqrt{0{,}25\,\text{Ом} \cdot 16{,}00\,\text{Ом}}
            = 2{,}0\,\text{Ом}.
            \\
    \eta_1
            &= \frac{ R_1 }{R_1 + r}
            = \frac{\sqrt{ R_1 }}{\sqrt{ R_1 } + \sqrt{ R_2 }}
            = 0{,}111,  \\
    \eta_2
            &= \frac{ R_2 }{R_2 + r}
            = \frac{ \sqrt{ R_2 } }{\sqrt{ R_2 } + \sqrt{ R_1 }}
            = 0{,}889
    \end{align*}
}

\variantsplitter

\addpersonalvariant{Сергей Пономарёв}

\tasknumber{1}%
\task{%
    Напротив физических величин укажите их обозначения и единицы измерения в СИ:
    \begin{enumerate}
        \item напряжение,
        \item работа тока,
        \item удельное сопротивление,
        \item внутреннее сопротивление полной цепи.
    \end{enumerate}
}
\solutionspace{20pt}

\tasknumber{2}%
\task{%
    Запишите физический закон или формулу:
    \begin{enumerate}
        \item правило Кирхгофа для замкнутого контура,
        \item сопротивление резистора через удельное сопротивление,
        \item ЭДС (определение).
    \end{enumerate}
}
\solutionspace{40pt}

\tasknumber{3}%
\task{%
    На резистор сопротивлением $r = 30\,\text{Ом}$ подали напряжение $U = 240\,\text{В}$.
    Определите ток, который потечёт через резистор, и мощность, выделяющуюся на нём.
}
\answer{%
    \begin{align*}
    \mathcal{I} &= \frac{ U }{ r } = \frac{ 240\,\text{В} }{ 30\,\text{Ом} } = 8{,}00\,\text{А},  \\
    P &= \frac{U^2}{ r } = \frac{ \sqr{ 240\,\text{В} } }{ 30\,\text{Ом} } = 1920{,}00\,\text{Вт}
    \end{align*}
}
\solutionspace{60pt}

\tasknumber{4}%
\task{%
    Через резистор сопротивлением $r = 18\,\text{Ом}$ протекает электрический ток $\mathcal{I} = 6{,}00\,\text{А}$.
    Определите, чему равны напряжение на резисторе и мощность, выделяющаяся на нём.
}
\answer{%
    \begin{align*}
    U &= \mathcal{I}r = 6{,}00\,\text{А} \cdot 18\,\text{Ом} = 108\,\text{В},  \\
    P &= \mathcal{I}^2r = \sqr{ 6{,}00\,\text{А} } \cdot 18\,\text{Ом} = 648\,\text{Вт}
    \end{align*}
}
\solutionspace{60pt}

\tasknumber{5}%
\task{%
    Замкнутая электрическая цепь состоит из ЭДС $\mathcal{E} = 4\,\text{В}$ и сопротивлением $r$
    и резистора $R = 10\,\text{Ом}$.
    Определите ток, протекающий в цепи.
    Какая тепловая энергия выделится на резисторе за время
    $\tau = 5\,\text{с}$? Какая работа будет совершена ЭДС за это время? Каков знак этой работы? Чему равен КПД цепи?
    Вычислите значения для 2 случаев: $r=0$ и $r = 30\,\text{Ом}$.
}
\answer{%
    \begin{align*}
    \mathcal{I}_1 &= \frac{ \mathcal{E} }{ R } = \frac{ 4\,\text{В} }{ 10\,\text{Ом} } = 0{,}40\,\text{А},  \\
    \mathcal{I}_2 &= \frac{ \mathcal{E} }{R + r} = \frac{ 4\,\text{В} }{10\,\text{Ом} + 30\,\text{Ом}} = 0{,}10\,\text{А},  \\
    Q_1 &= \mathcal{I}_1^2R\tau = \sqr{\frac{ \mathcal{E} }{ R }} R \tau
            = \sqr{\frac{ 4\,\text{В} }{ 10\,\text{Ом} }} \cdot 10\,\text{Ом} \cdot 5\,\text{с} = 8{,}000\,\text{Дж},  \\
    Q_2 &= \mathcal{I}_2^2R\tau = \sqr{\frac{ \mathcal{E} }{R + r}} R \tau
            = \sqr{\frac{ 4\,\text{В} }{10\,\text{Ом} + 30\,\text{Ом}}} \cdot 10\,\text{Ом} \cdot 5\,\text{с} = 0{,}500\,\text{Дж},  \\
    A_1 &= \mathcal{I}_1\tau\mathcal{E} = \frac{ \mathcal{E} }{R} \tau \mathcal{E}
            = \frac{\mathcal{E}^2 \tau}{ R } = \frac{\sqr{ 4\,\text{В} } \cdot 5\,\text{с}}{ 10\,\text{Ом} }
            = 8{,}000\,\text{Дж}, \text{положительна},  \\
    A_2 &= \mathcal{I}_2\tau\mathcal{E} = \frac{ \mathcal{E} }{R + r} \tau \mathcal{E}
            = \frac{\mathcal{E}^2 \tau}{R + r} = \frac{\sqr{ 4\,\text{В} } \cdot 5\,\text{с}}{10\,\text{Ом} + 30\,\text{Ом}}
            = 2{,}000\,\text{Дж}, \text{положительна},  \\
    \eta_1 &= \frac{ Q_1 }{ A_1 } = \ldots = \frac{ R }{ R } = 1,  \\
    \eta_2 &= \frac{ Q_2 }{ A_2 } = \ldots = \frac{ R }{R + r} = 0{,}25
    \end{align*}
}
\solutionspace{180pt}

\tasknumber{6}%
\task{%
    Лампочки, сопротивления которых $R_1 = 6{,}00\,\text{Ом}$ и $R_2 = 54{,}00\,\text{Ом}$, поочерёдно подключённные к некоторому источнику тока,
    потребляют одинаковую мощность.
    Найти внутреннее сопротивление источника и КПД цепи в каждом случае.
}
\answer{%
    \begin{align*}
        P_1 &= \sqr{\frac{ \mathcal{E} }{R_1 + r}}R_1,
        P_2  = \sqr{\frac{ \mathcal{E} }{R_2 + r}}R_2,
        P_1 = P_2 \implies  \\
        &\implies R_1 \sqr{R_2 + r} = R_2 \sqr{R_1 + r} \implies  \\
        &\implies R_1 R_2^2 + 2 R_1 R_2 r + R_1 r^2 =
                    R_2 R_1^2 + 2 R_2 R_1 r + R_2 r^2  \implies  \\
    &\implies r^2 (R_2 - R_1) = R_2^2 R_2 - R_1^2 R_2 \implies  \\
    &\implies r
            = \sqrt{R_1 R_2 \frac{R_2 - R_1}{R_2 - R_1}}
            = \sqrt{R_1 R_2}
            = \sqrt{6{,}00\,\text{Ом} \cdot 54{,}00\,\text{Ом}}
            = 18{,}0\,\text{Ом}.
            \\
    \eta_1
            &= \frac{ R_1 }{R_1 + r}
            = \frac{\sqrt{ R_1 }}{\sqrt{ R_1 } + \sqrt{ R_2 }}
            = 0{,}250,  \\
    \eta_2
            &= \frac{ R_2 }{R_2 + r}
            = \frac{ \sqrt{ R_2 } }{\sqrt{ R_2 } + \sqrt{ R_1 }}
            = 0{,}750
    \end{align*}
}

\variantsplitter

\addpersonalvariant{Егор Свистушкин}

\tasknumber{1}%
\task{%
    Напротив физических величин укажите их обозначения и единицы измерения в СИ:
    \begin{enumerate}
        \item сила тока,
        \item мощность тока,
        \item ЭДС,
        \item внутреннее сопротивление полной цепи.
    \end{enumerate}
}
\solutionspace{20pt}

\tasknumber{2}%
\task{%
    Запишите физический закон или формулу:
    \begin{enumerate}
        \item правило Кирхгофа для замкнутого контура,
        \item сопротивление резистора через удельное сопротивление,
        \item закон Ома для неоднородного участка цепи.
    \end{enumerate}
}
\solutionspace{40pt}

\tasknumber{3}%
\task{%
    На резистор сопротивлением $r = 18\,\text{Ом}$ подали напряжение $U = 120\,\text{В}$.
    Определите ток, который потечёт через резистор, и мощность, выделяющуюся на нём.
}
\answer{%
    \begin{align*}
    \mathcal{I} &= \frac{ U }{ r } = \frac{ 120\,\text{В} }{ 18\,\text{Ом} } = 6{,}67\,\text{А},  \\
    P &= \frac{U^2}{ r } = \frac{ \sqr{ 120\,\text{В} } }{ 18\,\text{Ом} } = 800{,}00\,\text{Вт}
    \end{align*}
}
\solutionspace{60pt}

\tasknumber{4}%
\task{%
    Через резистор сопротивлением $R = 12\,\text{Ом}$ протекает электрический ток $\mathcal{I} = 2{,}00\,\text{А}$.
    Определите, чему равны напряжение на резисторе и мощность, выделяющаяся на нём.
}
\answer{%
    \begin{align*}
    U &= \mathcal{I}R = 2{,}00\,\text{А} \cdot 12\,\text{Ом} = 24\,\text{В},  \\
    P &= \mathcal{I}^2R = \sqr{ 2{,}00\,\text{А} } \cdot 12\,\text{Ом} = 48\,\text{Вт}
    \end{align*}
}
\solutionspace{60pt}

\tasknumber{5}%
\task{%
    Замкнутая электрическая цепь состоит из ЭДС $\mathcal{E} = 4\,\text{В}$ и сопротивлением $r$
    и резистора $R = 10\,\text{Ом}$.
    Определите ток, протекающий в цепи.
    Какая тепловая энергия выделится на резисторе за время
    $\tau = 2\,\text{с}$? Какая работа будет совершена ЭДС за это время? Каков знак этой работы? Чему равен КПД цепи?
    Вычислите значения для 2 случаев: $r=0$ и $r = 20\,\text{Ом}$.
}
\answer{%
    \begin{align*}
    \mathcal{I}_1 &= \frac{ \mathcal{E} }{ R } = \frac{ 4\,\text{В} }{ 10\,\text{Ом} } = 0{,}40\,\text{А},  \\
    \mathcal{I}_2 &= \frac{ \mathcal{E} }{R + r} = \frac{ 4\,\text{В} }{10\,\text{Ом} + 20\,\text{Ом}} = 0{,}13\,\text{А},  \\
    Q_1 &= \mathcal{I}_1^2R\tau = \sqr{\frac{ \mathcal{E} }{ R }} R \tau
            = \sqr{\frac{ 4\,\text{В} }{ 10\,\text{Ом} }} \cdot 10\,\text{Ом} \cdot 2\,\text{с} = 3{,}200\,\text{Дж},  \\
    Q_2 &= \mathcal{I}_2^2R\tau = \sqr{\frac{ \mathcal{E} }{R + r}} R \tau
            = \sqr{\frac{ 4\,\text{В} }{10\,\text{Ом} + 20\,\text{Ом}}} \cdot 10\,\text{Ом} \cdot 2\,\text{с} = 0{,}338\,\text{Дж},  \\
    A_1 &= \mathcal{I}_1\tau\mathcal{E} = \frac{ \mathcal{E} }{R} \tau \mathcal{E}
            = \frac{\mathcal{E}^2 \tau}{ R } = \frac{\sqr{ 4\,\text{В} } \cdot 2\,\text{с}}{ 10\,\text{Ом} }
            = 3{,}200\,\text{Дж}, \text{положительна},  \\
    A_2 &= \mathcal{I}_2\tau\mathcal{E} = \frac{ \mathcal{E} }{R + r} \tau \mathcal{E}
            = \frac{\mathcal{E}^2 \tau}{R + r} = \frac{\sqr{ 4\,\text{В} } \cdot 2\,\text{с}}{10\,\text{Ом} + 20\,\text{Ом}}
            = 1{,}040\,\text{Дж}, \text{положительна},  \\
    \eta_1 &= \frac{ Q_1 }{ A_1 } = \ldots = \frac{ R }{ R } = 1,  \\
    \eta_2 &= \frac{ Q_2 }{ A_2 } = \ldots = \frac{ R }{R + r} = 0{,}33
    \end{align*}
}
\solutionspace{180pt}

\tasknumber{6}%
\task{%
    Лампочки, сопротивления которых $R_1 = 3{,}00\,\text{Ом}$ и $R_2 = 48{,}00\,\text{Ом}$, поочерёдно подключённные к некоторому источнику тока,
    потребляют одинаковую мощность.
    Найти внутреннее сопротивление источника и КПД цепи в каждом случае.
}
\answer{%
    \begin{align*}
        P_1 &= \sqr{\frac{ \mathcal{E} }{R_1 + r}}R_1,
        P_2  = \sqr{\frac{ \mathcal{E} }{R_2 + r}}R_2,
        P_1 = P_2 \implies  \\
        &\implies R_1 \sqr{R_2 + r} = R_2 \sqr{R_1 + r} \implies  \\
        &\implies R_1 R_2^2 + 2 R_1 R_2 r + R_1 r^2 =
                    R_2 R_1^2 + 2 R_2 R_1 r + R_2 r^2  \implies  \\
    &\implies r^2 (R_2 - R_1) = R_2^2 R_2 - R_1^2 R_2 \implies  \\
    &\implies r
            = \sqrt{R_1 R_2 \frac{R_2 - R_1}{R_2 - R_1}}
            = \sqrt{R_1 R_2}
            = \sqrt{3{,}00\,\text{Ом} \cdot 48{,}00\,\text{Ом}}
            = 12{,}0\,\text{Ом}.
            \\
    \eta_1
            &= \frac{ R_1 }{R_1 + r}
            = \frac{\sqrt{ R_1 }}{\sqrt{ R_1 } + \sqrt{ R_2 }}
            = 0{,}200,  \\
    \eta_2
            &= \frac{ R_2 }{R_2 + r}
            = \frac{ \sqrt{ R_2 } }{\sqrt{ R_2 } + \sqrt{ R_1 }}
            = 0{,}800
    \end{align*}
}

\variantsplitter

\addpersonalvariant{Дмитрий Соколов}

\tasknumber{1}%
\task{%
    Напротив физических величин укажите их обозначения и единицы измерения в СИ:
    \begin{enumerate}
        \item сила тока,
        \item мощность тока,
        \item ЭДС,
        \item внешнее сопротивление полной цепи.
    \end{enumerate}
}
\solutionspace{20pt}

\tasknumber{2}%
\task{%
    Запишите физический закон или формулу:
    \begin{enumerate}
        \item правило Кирхгофа для узла цепи,
        \item закон Ома для однородного участка цепи,
        \item ЭДС (определение).
    \end{enumerate}
}
\solutionspace{40pt}

\tasknumber{3}%
\task{%
    На резистор сопротивлением $r = 30\,\text{Ом}$ подали напряжение $V = 180\,\text{В}$.
    Определите ток, который потечёт через резистор, и мощность, выделяющуюся на нём.
}
\answer{%
    \begin{align*}
    \mathcal{I} &= \frac{ V }{ r } = \frac{ 180\,\text{В} }{ 30\,\text{Ом} } = 6{,}00\,\text{А},  \\
    P &= \frac{V^2}{ r } = \frac{ \sqr{ 180\,\text{В} } }{ 30\,\text{Ом} } = 1080{,}00\,\text{Вт}
    \end{align*}
}
\solutionspace{60pt}

\tasknumber{4}%
\task{%
    Через резистор сопротивлением $R = 5\,\text{Ом}$ протекает электрический ток $\mathcal{I} = 2{,}00\,\text{А}$.
    Определите, чему равны напряжение на резисторе и мощность, выделяющаяся на нём.
}
\answer{%
    \begin{align*}
    U &= \mathcal{I}R = 2{,}00\,\text{А} \cdot 5\,\text{Ом} = 10\,\text{В},  \\
    P &= \mathcal{I}^2R = \sqr{ 2{,}00\,\text{А} } \cdot 5\,\text{Ом} = 20\,\text{Вт}
    \end{align*}
}
\solutionspace{60pt}

\tasknumber{5}%
\task{%
    Замкнутая электрическая цепь состоит из ЭДС $\mathcal{E} = 2\,\text{В}$ и сопротивлением $r$
    и резистора $R = 30\,\text{Ом}$.
    Определите ток, протекающий в цепи.
    Какая тепловая энергия выделится на резисторе за время
    $\tau = 10\,\text{с}$? Какая работа будет совершена ЭДС за это время? Каков знак этой работы? Чему равен КПД цепи?
    Вычислите значения для 2 случаев: $r=0$ и $r = 60\,\text{Ом}$.
}
\answer{%
    \begin{align*}
    \mathcal{I}_1 &= \frac{ \mathcal{E} }{ R } = \frac{ 2\,\text{В} }{ 30\,\text{Ом} } = 0{,}07\,\text{А},  \\
    \mathcal{I}_2 &= \frac{ \mathcal{E} }{R + r} = \frac{ 2\,\text{В} }{30\,\text{Ом} + 60\,\text{Ом}} = 0{,}02\,\text{А},  \\
    Q_1 &= \mathcal{I}_1^2R\tau = \sqr{\frac{ \mathcal{E} }{ R }} R \tau
            = \sqr{\frac{ 2\,\text{В} }{ 30\,\text{Ом} }} \cdot 30\,\text{Ом} \cdot 10\,\text{с} = 1{,}470\,\text{Дж},  \\
    Q_2 &= \mathcal{I}_2^2R\tau = \sqr{\frac{ \mathcal{E} }{R + r}} R \tau
            = \sqr{\frac{ 2\,\text{В} }{30\,\text{Ом} + 60\,\text{Ом}}} \cdot 30\,\text{Ом} \cdot 10\,\text{с} = 0{,}120\,\text{Дж},  \\
    A_1 &= \mathcal{I}_1\tau\mathcal{E} = \frac{ \mathcal{E} }{R} \tau \mathcal{E}
            = \frac{\mathcal{E}^2 \tau}{ R } = \frac{\sqr{ 2\,\text{В} } \cdot 10\,\text{с}}{ 30\,\text{Ом} }
            = 1{,}400\,\text{Дж}, \text{положительна},  \\
    A_2 &= \mathcal{I}_2\tau\mathcal{E} = \frac{ \mathcal{E} }{R + r} \tau \mathcal{E}
            = \frac{\mathcal{E}^2 \tau}{R + r} = \frac{\sqr{ 2\,\text{В} } \cdot 10\,\text{с}}{30\,\text{Ом} + 60\,\text{Ом}}
            = 0{,}400\,\text{Дж}, \text{положительна},  \\
    \eta_1 &= \frac{ Q_1 }{ A_1 } = \ldots = \frac{ R }{ R } = 1,  \\
    \eta_2 &= \frac{ Q_2 }{ A_2 } = \ldots = \frac{ R }{R + r} = 0{,}30
    \end{align*}
}
\solutionspace{180pt}

\tasknumber{6}%
\task{%
    Лампочки, сопротивления которых $R_1 = 0{,}50\,\text{Ом}$ и $R_2 = 2{,}00\,\text{Ом}$, поочерёдно подключённные к некоторому источнику тока,
    потребляют одинаковую мощность.
    Найти внутреннее сопротивление источника и КПД цепи в каждом случае.
}
\answer{%
    \begin{align*}
        P_1 &= \sqr{\frac{ \mathcal{E} }{R_1 + r}}R_1,
        P_2  = \sqr{\frac{ \mathcal{E} }{R_2 + r}}R_2,
        P_1 = P_2 \implies  \\
        &\implies R_1 \sqr{R_2 + r} = R_2 \sqr{R_1 + r} \implies  \\
        &\implies R_1 R_2^2 + 2 R_1 R_2 r + R_1 r^2 =
                    R_2 R_1^2 + 2 R_2 R_1 r + R_2 r^2  \implies  \\
    &\implies r^2 (R_2 - R_1) = R_2^2 R_2 - R_1^2 R_2 \implies  \\
    &\implies r
            = \sqrt{R_1 R_2 \frac{R_2 - R_1}{R_2 - R_1}}
            = \sqrt{R_1 R_2}
            = \sqrt{0{,}50\,\text{Ом} \cdot 2{,}00\,\text{Ом}}
            = 1{,}0\,\text{Ом}.
            \\
    \eta_1
            &= \frac{ R_1 }{R_1 + r}
            = \frac{\sqrt{ R_1 }}{\sqrt{ R_1 } + \sqrt{ R_2 }}
            = 0{,}333,  \\
    \eta_2
            &= \frac{ R_2 }{R_2 + r}
            = \frac{ \sqrt{ R_2 } }{\sqrt{ R_2 } + \sqrt{ R_1 }}
            = 0{,}667
    \end{align*}
}

\variantsplitter

\addpersonalvariant{Арсений Трофимов}

\tasknumber{1}%
\task{%
    Напротив физических величин укажите их обозначения и единицы измерения в СИ:
    \begin{enumerate}
        \item сила тока,
        \item работа тока,
        \item удельное сопротивление,
        \item внешнее сопротивление полной цепи.
    \end{enumerate}
}
\solutionspace{20pt}

\tasknumber{2}%
\task{%
    Запишите физический закон или формулу:
    \begin{enumerate}
        \item правило Кирхгофа для узла цепи,
        \item сопротивление резистора через удельное сопротивление,
        \item ЭДС (определение).
    \end{enumerate}
}
\solutionspace{40pt}

\tasknumber{3}%
\task{%
    На резистор сопротивлением $R = 18\,\text{Ом}$ подали напряжение $U = 120\,\text{В}$.
    Определите ток, который потечёт через резистор, и мощность, выделяющуюся на нём.
}
\answer{%
    \begin{align*}
    \mathcal{I} &= \frac{ U }{ R } = \frac{ 120\,\text{В} }{ 18\,\text{Ом} } = 6{,}67\,\text{А},  \\
    P &= \frac{U^2}{ R } = \frac{ \sqr{ 120\,\text{В} } }{ 18\,\text{Ом} } = 800{,}00\,\text{Вт}
    \end{align*}
}
\solutionspace{60pt}

\tasknumber{4}%
\task{%
    Через резистор сопротивлением $R = 30\,\text{Ом}$ протекает электрический ток $\mathcal{I} = 2{,}00\,\text{А}$.
    Определите, чему равны напряжение на резисторе и мощность, выделяющаяся на нём.
}
\answer{%
    \begin{align*}
    U &= \mathcal{I}R = 2{,}00\,\text{А} \cdot 30\,\text{Ом} = 60\,\text{В},  \\
    P &= \mathcal{I}^2R = \sqr{ 2{,}00\,\text{А} } \cdot 30\,\text{Ом} = 120\,\text{Вт}
    \end{align*}
}
\solutionspace{60pt}

\tasknumber{5}%
\task{%
    Замкнутая электрическая цепь состоит из ЭДС $\mathcal{E} = 4\,\text{В}$ и сопротивлением $r$
    и резистора $R = 24\,\text{Ом}$.
    Определите ток, протекающий в цепи.
    Какая тепловая энергия выделится на резисторе за время
    $\tau = 2\,\text{с}$? Какая работа будет совершена ЭДС за это время? Каков знак этой работы? Чему равен КПД цепи?
    Вычислите значения для 2 случаев: $r=0$ и $r = 10\,\text{Ом}$.
}
\answer{%
    \begin{align*}
    \mathcal{I}_1 &= \frac{ \mathcal{E} }{ R } = \frac{ 4\,\text{В} }{ 24\,\text{Ом} } = 0{,}17\,\text{А},  \\
    \mathcal{I}_2 &= \frac{ \mathcal{E} }{R + r} = \frac{ 4\,\text{В} }{24\,\text{Ом} + 10\,\text{Ом}} = 0{,}12\,\text{А},  \\
    Q_1 &= \mathcal{I}_1^2R\tau = \sqr{\frac{ \mathcal{E} }{ R }} R \tau
            = \sqr{\frac{ 4\,\text{В} }{ 24\,\text{Ом} }} \cdot 24\,\text{Ом} \cdot 2\,\text{с} = 1{,}387\,\text{Дж},  \\
    Q_2 &= \mathcal{I}_2^2R\tau = \sqr{\frac{ \mathcal{E} }{R + r}} R \tau
            = \sqr{\frac{ 4\,\text{В} }{24\,\text{Ом} + 10\,\text{Ом}}} \cdot 24\,\text{Ом} \cdot 2\,\text{с} = 0{,}691\,\text{Дж},  \\
    A_1 &= \mathcal{I}_1\tau\mathcal{E} = \frac{ \mathcal{E} }{R} \tau \mathcal{E}
            = \frac{\mathcal{E}^2 \tau}{ R } = \frac{\sqr{ 4\,\text{В} } \cdot 2\,\text{с}}{ 24\,\text{Ом} }
            = 1{,}360\,\text{Дж}, \text{положительна},  \\
    A_2 &= \mathcal{I}_2\tau\mathcal{E} = \frac{ \mathcal{E} }{R + r} \tau \mathcal{E}
            = \frac{\mathcal{E}^2 \tau}{R + r} = \frac{\sqr{ 4\,\text{В} } \cdot 2\,\text{с}}{24\,\text{Ом} + 10\,\text{Ом}}
            = 0{,}960\,\text{Дж}, \text{положительна},  \\
    \eta_1 &= \frac{ Q_1 }{ A_1 } = \ldots = \frac{ R }{ R } = 1,  \\
    \eta_2 &= \frac{ Q_2 }{ A_2 } = \ldots = \frac{ R }{R + r} = 0{,}72
    \end{align*}
}
\solutionspace{180pt}

\tasknumber{6}%
\task{%
    Лампочки, сопротивления которых $R_1 = 0{,}25\,\text{Ом}$ и $R_2 = 4{,}00\,\text{Ом}$, поочерёдно подключённные к некоторому источнику тока,
    потребляют одинаковую мощность.
    Найти внутреннее сопротивление источника и КПД цепи в каждом случае.
}
\answer{%
    \begin{align*}
        P_1 &= \sqr{\frac{ \mathcal{E} }{R_1 + r}}R_1,
        P_2  = \sqr{\frac{ \mathcal{E} }{R_2 + r}}R_2,
        P_1 = P_2 \implies  \\
        &\implies R_1 \sqr{R_2 + r} = R_2 \sqr{R_1 + r} \implies  \\
        &\implies R_1 R_2^2 + 2 R_1 R_2 r + R_1 r^2 =
                    R_2 R_1^2 + 2 R_2 R_1 r + R_2 r^2  \implies  \\
    &\implies r^2 (R_2 - R_1) = R_2^2 R_2 - R_1^2 R_2 \implies  \\
    &\implies r
            = \sqrt{R_1 R_2 \frac{R_2 - R_1}{R_2 - R_1}}
            = \sqrt{R_1 R_2}
            = \sqrt{0{,}25\,\text{Ом} \cdot 4{,}00\,\text{Ом}}
            = 1{,}0\,\text{Ом}.
            \\
    \eta_1
            &= \frac{ R_1 }{R_1 + r}
            = \frac{\sqrt{ R_1 }}{\sqrt{ R_1 } + \sqrt{ R_2 }}
            = 0{,}200,  \\
    \eta_2
            &= \frac{ R_2 }{R_2 + r}
            = \frac{ \sqrt{ R_2 } }{\sqrt{ R_2 } + \sqrt{ R_1 }}
            = 0{,}800
    \end{align*}
}
% autogenerated
