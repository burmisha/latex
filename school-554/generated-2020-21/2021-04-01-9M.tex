\setdate{1~апреля~2021}
\setclass{9«М»}

\addpersonalvariant{Михаил Бурмистров}

\tasknumber{1}%
\task{%
    При переходе электрона в атоме с одной стационарной орбиты на другую
    излучается фотон с энергией $4{,}04 \cdot 10^{-19}\,\text{Дж}$.
    Какова длина волны этой линии спектра?
    Постоянная Планка $h = 6{,}626 \cdot 10^{-34}\,\text{Дж}\cdot\text{с}$, скорость света $c = 3 \cdot 10^{8}\,\frac{\text{м}}{\text{с}}$.
}
\answer{%
    $
        E = h\nu = h \frac c\lambda
        \implies \lambda = \frac{hc}E
            = \frac{6{,}626 \cdot 10^{-34}\,\text{Дж}\cdot\text{с} \cdot {3 \cdot 10^{8}\,\frac{\text{м}}{\text{с}}}}{4{,}04 \cdot 10^{-19}\,\text{Дж}}
            = 492{,}03\,\text{нм}.
    $
}
\solutionspace{150pt}

\tasknumber{2}%
\task{%
    Излучение какой длины волны поглотил атом водорода, если полная энергия в атоме увеличилась на $4 \cdot 10^{-19}\,\text{Дж}$?
    Постоянная Планка $h = 6{,}626 \cdot 10^{-34}\,\text{Дж}\cdot\text{с}$, скорость света $c = 3 \cdot 10^{8}\,\frac{\text{м}}{\text{с}}$.
}
\answer{%
    $
        E = h\nu = h \frac c\lambda
        \implies \lambda = \frac{hc}E
            = \frac{6{,}626 \cdot 10^{-34}\,\text{Дж}\cdot\text{с} \cdot {3 \cdot 10^{8}\,\frac{\text{м}}{\text{с}}}}{4 \cdot 10^{-19}\,\text{Дж}}
            = 496{,}95\,\text{нм}.
    $
}
\solutionspace{150pt}

\tasknumber{3}%
\task{%
    Сделайте схематичный рисунок энергетических уровней атома водорода
    и отметьте на нём первый (основной) уровень и последующие.
    Сколько различных длин волн может испустить атом водорода,
    находящийся в 3-м возбуждённом состоянии?
    Отметьте все соответствующие переходы на рисунке и укажите,
    при каком переходе (среди отмеченных) частота излучённого фотона максимальна.
}
\answer{%
    $N = 3{,}0, \text{самая длинная линия}$
}

\variantsplitter

\addpersonalvariant{Артём Глембо}

\tasknumber{1}%
\task{%
    При переходе электрона в атоме с одной стационарной орбиты на другую
    излучается фотон с энергией $4{,}04 \cdot 10^{-19}\,\text{Дж}$.
    Какова длина волны этой линии спектра?
    Постоянная Планка $h = 6{,}626 \cdot 10^{-34}\,\text{Дж}\cdot\text{с}$, скорость света $c = 3 \cdot 10^{8}\,\frac{\text{м}}{\text{с}}$.
}
\answer{%
    $
        E = h\nu = h \frac c\lambda
        \implies \lambda = \frac{hc}E
            = \frac{6{,}626 \cdot 10^{-34}\,\text{Дж}\cdot\text{с} \cdot {3 \cdot 10^{8}\,\frac{\text{м}}{\text{с}}}}{4{,}04 \cdot 10^{-19}\,\text{Дж}}
            = 492{,}03\,\text{нм}.
    $
}
\solutionspace{150pt}

\tasknumber{2}%
\task{%
    Излучение какой длины волны поглотил атом водорода, если полная энергия в атоме увеличилась на $4 \cdot 10^{-19}\,\text{Дж}$?
    Постоянная Планка $h = 6{,}626 \cdot 10^{-34}\,\text{Дж}\cdot\text{с}$, скорость света $c = 3 \cdot 10^{8}\,\frac{\text{м}}{\text{с}}$.
}
\answer{%
    $
        E = h\nu = h \frac c\lambda
        \implies \lambda = \frac{hc}E
            = \frac{6{,}626 \cdot 10^{-34}\,\text{Дж}\cdot\text{с} \cdot {3 \cdot 10^{8}\,\frac{\text{м}}{\text{с}}}}{4 \cdot 10^{-19}\,\text{Дж}}
            = 496{,}95\,\text{нм}.
    $
}
\solutionspace{150pt}

\tasknumber{3}%
\task{%
    Сделайте схематичный рисунок энергетических уровней атома водорода
    и отметьте на нём первый (основной) уровень и последующие.
    Сколько различных длин волн может испустить атом водорода,
    находящийся в 3-м возбуждённом состоянии?
    Отметьте все соответствующие переходы на рисунке и укажите,
    при каком переходе (среди отмеченных) длина волны излучённого фотона минимальна.
}
\answer{%
    $N = 3{,}0, \text{самая длинная линия}$
}

\variantsplitter

\addpersonalvariant{Наталья Гончарова}

\tasknumber{1}%
\task{%
    При переходе электрона в атоме с одной стационарной орбиты на другую
    излучается фотон с энергией $4{,}04 \cdot 10^{-19}\,\text{Дж}$.
    Какова длина волны этой линии спектра?
    Постоянная Планка $h = 6{,}626 \cdot 10^{-34}\,\text{Дж}\cdot\text{с}$, скорость света $c = 3 \cdot 10^{8}\,\frac{\text{м}}{\text{с}}$.
}
\answer{%
    $
        E = h\nu = h \frac c\lambda
        \implies \lambda = \frac{hc}E
            = \frac{6{,}626 \cdot 10^{-34}\,\text{Дж}\cdot\text{с} \cdot {3 \cdot 10^{8}\,\frac{\text{м}}{\text{с}}}}{4{,}04 \cdot 10^{-19}\,\text{Дж}}
            = 492{,}03\,\text{нм}.
    $
}
\solutionspace{150pt}

\tasknumber{2}%
\task{%
    Излучение какой длины волны поглотил атом водорода, если полная энергия в атоме увеличилась на $4 \cdot 10^{-19}\,\text{Дж}$?
    Постоянная Планка $h = 6{,}626 \cdot 10^{-34}\,\text{Дж}\cdot\text{с}$, скорость света $c = 3 \cdot 10^{8}\,\frac{\text{м}}{\text{с}}$.
}
\answer{%
    $
        E = h\nu = h \frac c\lambda
        \implies \lambda = \frac{hc}E
            = \frac{6{,}626 \cdot 10^{-34}\,\text{Дж}\cdot\text{с} \cdot {3 \cdot 10^{8}\,\frac{\text{м}}{\text{с}}}}{4 \cdot 10^{-19}\,\text{Дж}}
            = 496{,}95\,\text{нм}.
    $
}
\solutionspace{150pt}

\tasknumber{3}%
\task{%
    Сделайте схематичный рисунок энергетических уровней атома водорода
    и отметьте на нём первый (основной) уровень и последующие.
    Сколько различных длин волн может испустить атом водорода,
    находящийся в 5-м возбуждённом состоянии?
    Отметьте все соответствующие переходы на рисунке и укажите,
    при каком переходе (среди отмеченных) длина волны излучённого фотона максимальна.
}
\answer{%
    $N = 10{,}0, \text{самая короткая линия}$
}

\variantsplitter

\addpersonalvariant{Файёзбек Касымов}

\tasknumber{1}%
\task{%
    При переходе электрона в атоме с одной стационарной орбиты на другую
    излучается фотон с энергией $5{,}05 \cdot 10^{-19}\,\text{Дж}$.
    Какова длина волны этой линии спектра?
    Постоянная Планка $h = 6{,}626 \cdot 10^{-34}\,\text{Дж}\cdot\text{с}$, скорость света $c = 3 \cdot 10^{8}\,\frac{\text{м}}{\text{с}}$.
}
\answer{%
    $
        E = h\nu = h \frac c\lambda
        \implies \lambda = \frac{hc}E
            = \frac{6{,}626 \cdot 10^{-34}\,\text{Дж}\cdot\text{с} \cdot {3 \cdot 10^{8}\,\frac{\text{м}}{\text{с}}}}{5{,}05 \cdot 10^{-19}\,\text{Дж}}
            = 393{,}62\,\text{нм}.
    $
}
\solutionspace{150pt}

\tasknumber{2}%
\task{%
    Излучение какой длины волны поглотил атом водорода, если полная энергия в атоме увеличилась на $3 \cdot 10^{-19}\,\text{Дж}$?
    Постоянная Планка $h = 6{,}626 \cdot 10^{-34}\,\text{Дж}\cdot\text{с}$, скорость света $c = 3 \cdot 10^{8}\,\frac{\text{м}}{\text{с}}$.
}
\answer{%
    $
        E = h\nu = h \frac c\lambda
        \implies \lambda = \frac{hc}E
            = \frac{6{,}626 \cdot 10^{-34}\,\text{Дж}\cdot\text{с} \cdot {3 \cdot 10^{8}\,\frac{\text{м}}{\text{с}}}}{3 \cdot 10^{-19}\,\text{Дж}}
            = 662{,}60\,\text{нм}.
    $
}
\solutionspace{150pt}

\tasknumber{3}%
\task{%
    Сделайте схематичный рисунок энергетических уровней атома водорода
    и отметьте на нём первый (основной) уровень и последующие.
    Сколько различных длин волн может испустить атом водорода,
    находящийся в 3-м возбуждённом состоянии?
    Отметьте все соответствующие переходы на рисунке и укажите,
    при каком переходе (среди отмеченных) длина волны излучённого фотона максимальна.
}
\answer{%
    $N = 3{,}0, \text{самая короткая линия}$
}

\variantsplitter

\addpersonalvariant{Александр Козинец}

\tasknumber{1}%
\task{%
    При переходе электрона в атоме с одной стационарной орбиты на другую
    излучается фотон с энергией $0{,}55 \cdot 10^{-19}\,\text{Дж}$.
    Какова длина волны этой линии спектра?
    Постоянная Планка $h = 6{,}626 \cdot 10^{-34}\,\text{Дж}\cdot\text{с}$, скорость света $c = 3 \cdot 10^{8}\,\frac{\text{м}}{\text{с}}$.
}
\answer{%
    $
        E = h\nu = h \frac c\lambda
        \implies \lambda = \frac{hc}E
            = \frac{6{,}626 \cdot 10^{-34}\,\text{Дж}\cdot\text{с} \cdot {3 \cdot 10^{8}\,\frac{\text{м}}{\text{с}}}}{0{,}55 \cdot 10^{-19}\,\text{Дж}}
            = 3614{,}18\,\text{нм}.
    $
}
\solutionspace{150pt}

\tasknumber{2}%
\task{%
    Излучение какой длины волны поглотил атом водорода, если полная энергия в атоме увеличилась на $3 \cdot 10^{-19}\,\text{Дж}$?
    Постоянная Планка $h = 6{,}626 \cdot 10^{-34}\,\text{Дж}\cdot\text{с}$, скорость света $c = 3 \cdot 10^{8}\,\frac{\text{м}}{\text{с}}$.
}
\answer{%
    $
        E = h\nu = h \frac c\lambda
        \implies \lambda = \frac{hc}E
            = \frac{6{,}626 \cdot 10^{-34}\,\text{Дж}\cdot\text{с} \cdot {3 \cdot 10^{8}\,\frac{\text{м}}{\text{с}}}}{3 \cdot 10^{-19}\,\text{Дж}}
            = 662{,}60\,\text{нм}.
    $
}
\solutionspace{150pt}

\tasknumber{3}%
\task{%
    Сделайте схематичный рисунок энергетических уровней атома водорода
    и отметьте на нём первый (основной) уровень и последующие.
    Сколько различных длин волн может испустить атом водорода,
    находящийся в 3-м возбуждённом состоянии?
    Отметьте все соответствующие переходы на рисунке и укажите,
    при каком переходе (среди отмеченных) частота излучённого фотона максимальна.
}
\answer{%
    $N = 3{,}0, \text{самая длинная линия}$
}

\variantsplitter

\addpersonalvariant{Андрей Куликовский}

\tasknumber{1}%
\task{%
    При переходе электрона в атоме с одной стационарной орбиты на другую
    излучается фотон с энергией $5{,}05 \cdot 10^{-19}\,\text{Дж}$.
    Какова длина волны этой линии спектра?
    Постоянная Планка $h = 6{,}626 \cdot 10^{-34}\,\text{Дж}\cdot\text{с}$, скорость света $c = 3 \cdot 10^{8}\,\frac{\text{м}}{\text{с}}$.
}
\answer{%
    $
        E = h\nu = h \frac c\lambda
        \implies \lambda = \frac{hc}E
            = \frac{6{,}626 \cdot 10^{-34}\,\text{Дж}\cdot\text{с} \cdot {3 \cdot 10^{8}\,\frac{\text{м}}{\text{с}}}}{5{,}05 \cdot 10^{-19}\,\text{Дж}}
            = 393{,}62\,\text{нм}.
    $
}
\solutionspace{150pt}

\tasknumber{2}%
\task{%
    Излучение какой длины волны поглотил атом водорода, если полная энергия в атоме увеличилась на $6 \cdot 10^{-19}\,\text{Дж}$?
    Постоянная Планка $h = 6{,}626 \cdot 10^{-34}\,\text{Дж}\cdot\text{с}$, скорость света $c = 3 \cdot 10^{8}\,\frac{\text{м}}{\text{с}}$.
}
\answer{%
    $
        E = h\nu = h \frac c\lambda
        \implies \lambda = \frac{hc}E
            = \frac{6{,}626 \cdot 10^{-34}\,\text{Дж}\cdot\text{с} \cdot {3 \cdot 10^{8}\,\frac{\text{м}}{\text{с}}}}{6 \cdot 10^{-19}\,\text{Дж}}
            = 331{,}30\,\text{нм}.
    $
}
\solutionspace{150pt}

\tasknumber{3}%
\task{%
    Сделайте схематичный рисунок энергетических уровней атома водорода
    и отметьте на нём первый (основной) уровень и последующие.
    Сколько различных длин волн может испустить атом водорода,
    находящийся в 4-м возбуждённом состоянии?
    Отметьте все соответствующие переходы на рисунке и укажите,
    при каком переходе (среди отмеченных) частота излучённого фотона максимальна.
}
\answer{%
    $N = 6{,}0, \text{самая длинная линия}$
}

\variantsplitter

\addpersonalvariant{Полина Лоткова}

\tasknumber{1}%
\task{%
    При переходе электрона в атоме с одной стационарной орбиты на другую
    излучается фотон с энергией $4{,}04 \cdot 10^{-19}\,\text{Дж}$.
    Какова длина волны этой линии спектра?
    Постоянная Планка $h = 6{,}626 \cdot 10^{-34}\,\text{Дж}\cdot\text{с}$, скорость света $c = 3 \cdot 10^{8}\,\frac{\text{м}}{\text{с}}$.
}
\answer{%
    $
        E = h\nu = h \frac c\lambda
        \implies \lambda = \frac{hc}E
            = \frac{6{,}626 \cdot 10^{-34}\,\text{Дж}\cdot\text{с} \cdot {3 \cdot 10^{8}\,\frac{\text{м}}{\text{с}}}}{4{,}04 \cdot 10^{-19}\,\text{Дж}}
            = 492{,}03\,\text{нм}.
    $
}
\solutionspace{150pt}

\tasknumber{2}%
\task{%
    Излучение какой длины волны поглотил атом водорода, если полная энергия в атоме увеличилась на $2 \cdot 10^{-19}\,\text{Дж}$?
    Постоянная Планка $h = 6{,}626 \cdot 10^{-34}\,\text{Дж}\cdot\text{с}$, скорость света $c = 3 \cdot 10^{8}\,\frac{\text{м}}{\text{с}}$.
}
\answer{%
    $
        E = h\nu = h \frac c\lambda
        \implies \lambda = \frac{hc}E
            = \frac{6{,}626 \cdot 10^{-34}\,\text{Дж}\cdot\text{с} \cdot {3 \cdot 10^{8}\,\frac{\text{м}}{\text{с}}}}{2 \cdot 10^{-19}\,\text{Дж}}
            = 993{,}90\,\text{нм}.
    $
}
\solutionspace{150pt}

\tasknumber{3}%
\task{%
    Сделайте схематичный рисунок энергетических уровней атома водорода
    и отметьте на нём первый (основной) уровень и последующие.
    Сколько различных длин волн может испустить атом водорода,
    находящийся в 3-м возбуждённом состоянии?
    Отметьте все соответствующие переходы на рисунке и укажите,
    при каком переходе (среди отмеченных) длина волны излучённого фотона минимальна.
}
\answer{%
    $N = 3{,}0, \text{самая длинная линия}$
}

\variantsplitter

\addpersonalvariant{Екатерина Медведева}

\tasknumber{1}%
\task{%
    При переходе электрона в атоме с одной стационарной орбиты на другую
    излучается фотон с энергией $1{,}01 \cdot 10^{-19}\,\text{Дж}$.
    Какова длина волны этой линии спектра?
    Постоянная Планка $h = 6{,}626 \cdot 10^{-34}\,\text{Дж}\cdot\text{с}$, скорость света $c = 3 \cdot 10^{8}\,\frac{\text{м}}{\text{с}}$.
}
\answer{%
    $
        E = h\nu = h \frac c\lambda
        \implies \lambda = \frac{hc}E
            = \frac{6{,}626 \cdot 10^{-34}\,\text{Дж}\cdot\text{с} \cdot {3 \cdot 10^{8}\,\frac{\text{м}}{\text{с}}}}{1{,}01 \cdot 10^{-19}\,\text{Дж}}
            = 1968{,}12\,\text{нм}.
    $
}
\solutionspace{150pt}

\tasknumber{2}%
\task{%
    Излучение какой длины волны поглотил атом водорода, если полная энергия в атоме увеличилась на $2 \cdot 10^{-19}\,\text{Дж}$?
    Постоянная Планка $h = 6{,}626 \cdot 10^{-34}\,\text{Дж}\cdot\text{с}$, скорость света $c = 3 \cdot 10^{8}\,\frac{\text{м}}{\text{с}}$.
}
\answer{%
    $
        E = h\nu = h \frac c\lambda
        \implies \lambda = \frac{hc}E
            = \frac{6{,}626 \cdot 10^{-34}\,\text{Дж}\cdot\text{с} \cdot {3 \cdot 10^{8}\,\frac{\text{м}}{\text{с}}}}{2 \cdot 10^{-19}\,\text{Дж}}
            = 993{,}90\,\text{нм}.
    $
}
\solutionspace{150pt}

\tasknumber{3}%
\task{%
    Сделайте схематичный рисунок энергетических уровней атома водорода
    и отметьте на нём первый (основной) уровень и последующие.
    Сколько различных длин волн может испустить атом водорода,
    находящийся в 3-м возбуждённом состоянии?
    Отметьте все соответствующие переходы на рисунке и укажите,
    при каком переходе (среди отмеченных) частота излучённого фотона минимальна.
}
\answer{%
    $N = 3{,}0, \text{самая короткая линия}$
}

\variantsplitter

\addpersonalvariant{Константин Мельник}

\tasknumber{1}%
\task{%
    При переходе электрона в атоме с одной стационарной орбиты на другую
    излучается фотон с энергией $7{,}07 \cdot 10^{-19}\,\text{Дж}$.
    Какова длина волны этой линии спектра?
    Постоянная Планка $h = 6{,}626 \cdot 10^{-34}\,\text{Дж}\cdot\text{с}$, скорость света $c = 3 \cdot 10^{8}\,\frac{\text{м}}{\text{с}}$.
}
\answer{%
    $
        E = h\nu = h \frac c\lambda
        \implies \lambda = \frac{hc}E
            = \frac{6{,}626 \cdot 10^{-34}\,\text{Дж}\cdot\text{с} \cdot {3 \cdot 10^{8}\,\frac{\text{м}}{\text{с}}}}{7{,}07 \cdot 10^{-19}\,\text{Дж}}
            = 281{,}16\,\text{нм}.
    $
}
\solutionspace{150pt}

\tasknumber{2}%
\task{%
    Излучение какой длины волны поглотил атом водорода, если полная энергия в атоме увеличилась на $6 \cdot 10^{-19}\,\text{Дж}$?
    Постоянная Планка $h = 6{,}626 \cdot 10^{-34}\,\text{Дж}\cdot\text{с}$, скорость света $c = 3 \cdot 10^{8}\,\frac{\text{м}}{\text{с}}$.
}
\answer{%
    $
        E = h\nu = h \frac c\lambda
        \implies \lambda = \frac{hc}E
            = \frac{6{,}626 \cdot 10^{-34}\,\text{Дж}\cdot\text{с} \cdot {3 \cdot 10^{8}\,\frac{\text{м}}{\text{с}}}}{6 \cdot 10^{-19}\,\text{Дж}}
            = 331{,}30\,\text{нм}.
    $
}
\solutionspace{150pt}

\tasknumber{3}%
\task{%
    Сделайте схематичный рисунок энергетических уровней атома водорода
    и отметьте на нём первый (основной) уровень и последующие.
    Сколько различных длин волн может испустить атом водорода,
    находящийся в 5-м возбуждённом состоянии?
    Отметьте все соответствующие переходы на рисунке и укажите,
    при каком переходе (среди отмеченных) длина волны излучённого фотона максимальна.
}
\answer{%
    $N = 10{,}0, \text{самая короткая линия}$
}

\variantsplitter

\addpersonalvariant{Степан Небоваренков}

\tasknumber{1}%
\task{%
    При переходе электрона в атоме с одной стационарной орбиты на другую
    излучается фотон с энергией $7{,}07 \cdot 10^{-19}\,\text{Дж}$.
    Какова длина волны этой линии спектра?
    Постоянная Планка $h = 6{,}626 \cdot 10^{-34}\,\text{Дж}\cdot\text{с}$, скорость света $c = 3 \cdot 10^{8}\,\frac{\text{м}}{\text{с}}$.
}
\answer{%
    $
        E = h\nu = h \frac c\lambda
        \implies \lambda = \frac{hc}E
            = \frac{6{,}626 \cdot 10^{-34}\,\text{Дж}\cdot\text{с} \cdot {3 \cdot 10^{8}\,\frac{\text{м}}{\text{с}}}}{7{,}07 \cdot 10^{-19}\,\text{Дж}}
            = 281{,}16\,\text{нм}.
    $
}
\solutionspace{150pt}

\tasknumber{2}%
\task{%
    Излучение какой длины волны поглотил атом водорода, если полная энергия в атоме увеличилась на $6 \cdot 10^{-19}\,\text{Дж}$?
    Постоянная Планка $h = 6{,}626 \cdot 10^{-34}\,\text{Дж}\cdot\text{с}$, скорость света $c = 3 \cdot 10^{8}\,\frac{\text{м}}{\text{с}}$.
}
\answer{%
    $
        E = h\nu = h \frac c\lambda
        \implies \lambda = \frac{hc}E
            = \frac{6{,}626 \cdot 10^{-34}\,\text{Дж}\cdot\text{с} \cdot {3 \cdot 10^{8}\,\frac{\text{м}}{\text{с}}}}{6 \cdot 10^{-19}\,\text{Дж}}
            = 331{,}30\,\text{нм}.
    $
}
\solutionspace{150pt}

\tasknumber{3}%
\task{%
    Сделайте схематичный рисунок энергетических уровней атома водорода
    и отметьте на нём первый (основной) уровень и последующие.
    Сколько различных длин волн может испустить атом водорода,
    находящийся в 4-м возбуждённом состоянии?
    Отметьте все соответствующие переходы на рисунке и укажите,
    при каком переходе (среди отмеченных) частота излучённого фотона минимальна.
}
\answer{%
    $N = 6{,}0, \text{самая короткая линия}$
}

\variantsplitter

\addpersonalvariant{Матвей Неретин}

\tasknumber{1}%
\task{%
    При переходе электрона в атоме с одной стационарной орбиты на другую
    излучается фотон с энергией $5{,}05 \cdot 10^{-19}\,\text{Дж}$.
    Какова длина волны этой линии спектра?
    Постоянная Планка $h = 6{,}626 \cdot 10^{-34}\,\text{Дж}\cdot\text{с}$, скорость света $c = 3 \cdot 10^{8}\,\frac{\text{м}}{\text{с}}$.
}
\answer{%
    $
        E = h\nu = h \frac c\lambda
        \implies \lambda = \frac{hc}E
            = \frac{6{,}626 \cdot 10^{-34}\,\text{Дж}\cdot\text{с} \cdot {3 \cdot 10^{8}\,\frac{\text{м}}{\text{с}}}}{5{,}05 \cdot 10^{-19}\,\text{Дж}}
            = 393{,}62\,\text{нм}.
    $
}
\solutionspace{150pt}

\tasknumber{2}%
\task{%
    Излучение какой длины волны поглотил атом водорода, если полная энергия в атоме увеличилась на $6 \cdot 10^{-19}\,\text{Дж}$?
    Постоянная Планка $h = 6{,}626 \cdot 10^{-34}\,\text{Дж}\cdot\text{с}$, скорость света $c = 3 \cdot 10^{8}\,\frac{\text{м}}{\text{с}}$.
}
\answer{%
    $
        E = h\nu = h \frac c\lambda
        \implies \lambda = \frac{hc}E
            = \frac{6{,}626 \cdot 10^{-34}\,\text{Дж}\cdot\text{с} \cdot {3 \cdot 10^{8}\,\frac{\text{м}}{\text{с}}}}{6 \cdot 10^{-19}\,\text{Дж}}
            = 331{,}30\,\text{нм}.
    $
}
\solutionspace{150pt}

\tasknumber{3}%
\task{%
    Сделайте схематичный рисунок энергетических уровней атома водорода
    и отметьте на нём первый (основной) уровень и последующие.
    Сколько различных длин волн может испустить атом водорода,
    находящийся в 4-м возбуждённом состоянии?
    Отметьте все соответствующие переходы на рисунке и укажите,
    при каком переходе (среди отмеченных) длина волны излучённого фотона минимальна.
}
\answer{%
    $N = 6{,}0, \text{самая длинная линия}$
}

\variantsplitter

\addpersonalvariant{Мария Никонова}

\tasknumber{1}%
\task{%
    При переходе электрона в атоме с одной стационарной орбиты на другую
    излучается фотон с энергией $7{,}07 \cdot 10^{-19}\,\text{Дж}$.
    Какова длина волны этой линии спектра?
    Постоянная Планка $h = 6{,}626 \cdot 10^{-34}\,\text{Дж}\cdot\text{с}$, скорость света $c = 3 \cdot 10^{8}\,\frac{\text{м}}{\text{с}}$.
}
\answer{%
    $
        E = h\nu = h \frac c\lambda
        \implies \lambda = \frac{hc}E
            = \frac{6{,}626 \cdot 10^{-34}\,\text{Дж}\cdot\text{с} \cdot {3 \cdot 10^{8}\,\frac{\text{м}}{\text{с}}}}{7{,}07 \cdot 10^{-19}\,\text{Дж}}
            = 281{,}16\,\text{нм}.
    $
}
\solutionspace{150pt}

\tasknumber{2}%
\task{%
    Излучение какой длины волны поглотил атом водорода, если полная энергия в атоме увеличилась на $3 \cdot 10^{-19}\,\text{Дж}$?
    Постоянная Планка $h = 6{,}626 \cdot 10^{-34}\,\text{Дж}\cdot\text{с}$, скорость света $c = 3 \cdot 10^{8}\,\frac{\text{м}}{\text{с}}$.
}
\answer{%
    $
        E = h\nu = h \frac c\lambda
        \implies \lambda = \frac{hc}E
            = \frac{6{,}626 \cdot 10^{-34}\,\text{Дж}\cdot\text{с} \cdot {3 \cdot 10^{8}\,\frac{\text{м}}{\text{с}}}}{3 \cdot 10^{-19}\,\text{Дж}}
            = 662{,}60\,\text{нм}.
    $
}
\solutionspace{150pt}

\tasknumber{3}%
\task{%
    Сделайте схематичный рисунок энергетических уровней атома водорода
    и отметьте на нём первый (основной) уровень и последующие.
    Сколько различных длин волн может испустить атом водорода,
    находящийся в 4-м возбуждённом состоянии?
    Отметьте все соответствующие переходы на рисунке и укажите,
    при каком переходе (среди отмеченных) энергия излучённого фотона максимальна.
}
\answer{%
    $N = 6{,}0, \text{самая длинная линия}$
}

\variantsplitter

\addpersonalvariant{Даниил Палаткин}

\tasknumber{1}%
\task{%
    При переходе электрона в атоме с одной стационарной орбиты на другую
    излучается фотон с энергией $1{,}01 \cdot 10^{-19}\,\text{Дж}$.
    Какова длина волны этой линии спектра?
    Постоянная Планка $h = 6{,}626 \cdot 10^{-34}\,\text{Дж}\cdot\text{с}$, скорость света $c = 3 \cdot 10^{8}\,\frac{\text{м}}{\text{с}}$.
}
\answer{%
    $
        E = h\nu = h \frac c\lambda
        \implies \lambda = \frac{hc}E
            = \frac{6{,}626 \cdot 10^{-34}\,\text{Дж}\cdot\text{с} \cdot {3 \cdot 10^{8}\,\frac{\text{м}}{\text{с}}}}{1{,}01 \cdot 10^{-19}\,\text{Дж}}
            = 1968{,}12\,\text{нм}.
    $
}
\solutionspace{150pt}

\tasknumber{2}%
\task{%
    Излучение какой длины волны поглотил атом водорода, если полная энергия в атоме увеличилась на $2 \cdot 10^{-19}\,\text{Дж}$?
    Постоянная Планка $h = 6{,}626 \cdot 10^{-34}\,\text{Дж}\cdot\text{с}$, скорость света $c = 3 \cdot 10^{8}\,\frac{\text{м}}{\text{с}}$.
}
\answer{%
    $
        E = h\nu = h \frac c\lambda
        \implies \lambda = \frac{hc}E
            = \frac{6{,}626 \cdot 10^{-34}\,\text{Дж}\cdot\text{с} \cdot {3 \cdot 10^{8}\,\frac{\text{м}}{\text{с}}}}{2 \cdot 10^{-19}\,\text{Дж}}
            = 993{,}90\,\text{нм}.
    $
}
\solutionspace{150pt}

\tasknumber{3}%
\task{%
    Сделайте схематичный рисунок энергетических уровней атома водорода
    и отметьте на нём первый (основной) уровень и последующие.
    Сколько различных длин волн может испустить атом водорода,
    находящийся в 5-м возбуждённом состоянии?
    Отметьте все соответствующие переходы на рисунке и укажите,
    при каком переходе (среди отмеченных) энергия излучённого фотона максимальна.
}
\answer{%
    $N = 10{,}0, \text{самая длинная линия}$
}

\variantsplitter

\addpersonalvariant{Станислав Пикун}

\tasknumber{1}%
\task{%
    При переходе электрона в атоме с одной стационарной орбиты на другую
    излучается фотон с энергией $0{,}55 \cdot 10^{-19}\,\text{Дж}$.
    Какова длина волны этой линии спектра?
    Постоянная Планка $h = 6{,}626 \cdot 10^{-34}\,\text{Дж}\cdot\text{с}$, скорость света $c = 3 \cdot 10^{8}\,\frac{\text{м}}{\text{с}}$.
}
\answer{%
    $
        E = h\nu = h \frac c\lambda
        \implies \lambda = \frac{hc}E
            = \frac{6{,}626 \cdot 10^{-34}\,\text{Дж}\cdot\text{с} \cdot {3 \cdot 10^{8}\,\frac{\text{м}}{\text{с}}}}{0{,}55 \cdot 10^{-19}\,\text{Дж}}
            = 3614{,}18\,\text{нм}.
    $
}
\solutionspace{150pt}

\tasknumber{2}%
\task{%
    Излучение какой длины волны поглотил атом водорода, если полная энергия в атоме увеличилась на $3 \cdot 10^{-19}\,\text{Дж}$?
    Постоянная Планка $h = 6{,}626 \cdot 10^{-34}\,\text{Дж}\cdot\text{с}$, скорость света $c = 3 \cdot 10^{8}\,\frac{\text{м}}{\text{с}}$.
}
\answer{%
    $
        E = h\nu = h \frac c\lambda
        \implies \lambda = \frac{hc}E
            = \frac{6{,}626 \cdot 10^{-34}\,\text{Дж}\cdot\text{с} \cdot {3 \cdot 10^{8}\,\frac{\text{м}}{\text{с}}}}{3 \cdot 10^{-19}\,\text{Дж}}
            = 662{,}60\,\text{нм}.
    $
}
\solutionspace{150pt}

\tasknumber{3}%
\task{%
    Сделайте схематичный рисунок энергетических уровней атома водорода
    и отметьте на нём первый (основной) уровень и последующие.
    Сколько различных длин волн может испустить атом водорода,
    находящийся в 5-м возбуждённом состоянии?
    Отметьте все соответствующие переходы на рисунке и укажите,
    при каком переходе (среди отмеченных) длина волны излучённого фотона минимальна.
}
\answer{%
    $N = 10{,}0, \text{самая длинная линия}$
}

\variantsplitter

\addpersonalvariant{Илья Пичугин}

\tasknumber{1}%
\task{%
    При переходе электрона в атоме с одной стационарной орбиты на другую
    излучается фотон с энергией $2{,}02 \cdot 10^{-19}\,\text{Дж}$.
    Какова длина волны этой линии спектра?
    Постоянная Планка $h = 6{,}626 \cdot 10^{-34}\,\text{Дж}\cdot\text{с}$, скорость света $c = 3 \cdot 10^{8}\,\frac{\text{м}}{\text{с}}$.
}
\answer{%
    $
        E = h\nu = h \frac c\lambda
        \implies \lambda = \frac{hc}E
            = \frac{6{,}626 \cdot 10^{-34}\,\text{Дж}\cdot\text{с} \cdot {3 \cdot 10^{8}\,\frac{\text{м}}{\text{с}}}}{2{,}02 \cdot 10^{-19}\,\text{Дж}}
            = 984{,}06\,\text{нм}.
    $
}
\solutionspace{150pt}

\tasknumber{2}%
\task{%
    Излучение какой длины волны поглотил атом водорода, если полная энергия в атоме увеличилась на $2 \cdot 10^{-19}\,\text{Дж}$?
    Постоянная Планка $h = 6{,}626 \cdot 10^{-34}\,\text{Дж}\cdot\text{с}$, скорость света $c = 3 \cdot 10^{8}\,\frac{\text{м}}{\text{с}}$.
}
\answer{%
    $
        E = h\nu = h \frac c\lambda
        \implies \lambda = \frac{hc}E
            = \frac{6{,}626 \cdot 10^{-34}\,\text{Дж}\cdot\text{с} \cdot {3 \cdot 10^{8}\,\frac{\text{м}}{\text{с}}}}{2 \cdot 10^{-19}\,\text{Дж}}
            = 993{,}90\,\text{нм}.
    $
}
\solutionspace{150pt}

\tasknumber{3}%
\task{%
    Сделайте схематичный рисунок энергетических уровней атома водорода
    и отметьте на нём первый (основной) уровень и последующие.
    Сколько различных длин волн может испустить атом водорода,
    находящийся в 5-м возбуждённом состоянии?
    Отметьте все соответствующие переходы на рисунке и укажите,
    при каком переходе (среди отмеченных) энергия излучённого фотона минимальна.
}
\answer{%
    $N = 10{,}0, \text{самая короткая линия}$
}

\variantsplitter

\addpersonalvariant{Кирилл Севрюгин}

\tasknumber{1}%
\task{%
    При переходе электрона в атоме с одной стационарной орбиты на другую
    излучается фотон с энергией $1{,}01 \cdot 10^{-19}\,\text{Дж}$.
    Какова длина волны этой линии спектра?
    Постоянная Планка $h = 6{,}626 \cdot 10^{-34}\,\text{Дж}\cdot\text{с}$, скорость света $c = 3 \cdot 10^{8}\,\frac{\text{м}}{\text{с}}$.
}
\answer{%
    $
        E = h\nu = h \frac c\lambda
        \implies \lambda = \frac{hc}E
            = \frac{6{,}626 \cdot 10^{-34}\,\text{Дж}\cdot\text{с} \cdot {3 \cdot 10^{8}\,\frac{\text{м}}{\text{с}}}}{1{,}01 \cdot 10^{-19}\,\text{Дж}}
            = 1968{,}12\,\text{нм}.
    $
}
\solutionspace{150pt}

\tasknumber{2}%
\task{%
    Излучение какой длины волны поглотил атом водорода, если полная энергия в атоме увеличилась на $4 \cdot 10^{-19}\,\text{Дж}$?
    Постоянная Планка $h = 6{,}626 \cdot 10^{-34}\,\text{Дж}\cdot\text{с}$, скорость света $c = 3 \cdot 10^{8}\,\frac{\text{м}}{\text{с}}$.
}
\answer{%
    $
        E = h\nu = h \frac c\lambda
        \implies \lambda = \frac{hc}E
            = \frac{6{,}626 \cdot 10^{-34}\,\text{Дж}\cdot\text{с} \cdot {3 \cdot 10^{8}\,\frac{\text{м}}{\text{с}}}}{4 \cdot 10^{-19}\,\text{Дж}}
            = 496{,}95\,\text{нм}.
    $
}
\solutionspace{150pt}

\tasknumber{3}%
\task{%
    Сделайте схематичный рисунок энергетических уровней атома водорода
    и отметьте на нём первый (основной) уровень и последующие.
    Сколько различных длин волн может испустить атом водорода,
    находящийся в 3-м возбуждённом состоянии?
    Отметьте все соответствующие переходы на рисунке и укажите,
    при каком переходе (среди отмеченных) частота излучённого фотона максимальна.
}
\answer{%
    $N = 3{,}0, \text{самая длинная линия}$
}

\variantsplitter

\addpersonalvariant{Илья Стратонников}

\tasknumber{1}%
\task{%
    При переходе электрона в атоме с одной стационарной орбиты на другую
    излучается фотон с энергией $5{,}05 \cdot 10^{-19}\,\text{Дж}$.
    Какова длина волны этой линии спектра?
    Постоянная Планка $h = 6{,}626 \cdot 10^{-34}\,\text{Дж}\cdot\text{с}$, скорость света $c = 3 \cdot 10^{8}\,\frac{\text{м}}{\text{с}}$.
}
\answer{%
    $
        E = h\nu = h \frac c\lambda
        \implies \lambda = \frac{hc}E
            = \frac{6{,}626 \cdot 10^{-34}\,\text{Дж}\cdot\text{с} \cdot {3 \cdot 10^{8}\,\frac{\text{м}}{\text{с}}}}{5{,}05 \cdot 10^{-19}\,\text{Дж}}
            = 393{,}62\,\text{нм}.
    $
}
\solutionspace{150pt}

\tasknumber{2}%
\task{%
    Излучение какой длины волны поглотил атом водорода, если полная энергия в атоме увеличилась на $6 \cdot 10^{-19}\,\text{Дж}$?
    Постоянная Планка $h = 6{,}626 \cdot 10^{-34}\,\text{Дж}\cdot\text{с}$, скорость света $c = 3 \cdot 10^{8}\,\frac{\text{м}}{\text{с}}$.
}
\answer{%
    $
        E = h\nu = h \frac c\lambda
        \implies \lambda = \frac{hc}E
            = \frac{6{,}626 \cdot 10^{-34}\,\text{Дж}\cdot\text{с} \cdot {3 \cdot 10^{8}\,\frac{\text{м}}{\text{с}}}}{6 \cdot 10^{-19}\,\text{Дж}}
            = 331{,}30\,\text{нм}.
    $
}
\solutionspace{150pt}

\tasknumber{3}%
\task{%
    Сделайте схематичный рисунок энергетических уровней атома водорода
    и отметьте на нём первый (основной) уровень и последующие.
    Сколько различных длин волн может испустить атом водорода,
    находящийся в 4-м возбуждённом состоянии?
    Отметьте все соответствующие переходы на рисунке и укажите,
    при каком переходе (среди отмеченных) длина волны излучённого фотона максимальна.
}
\answer{%
    $N = 6{,}0, \text{самая короткая линия}$
}

\variantsplitter

\addpersonalvariant{Иван Шустов}

\tasknumber{1}%
\task{%
    При переходе электрона в атоме с одной стационарной орбиты на другую
    излучается фотон с энергией $4{,}04 \cdot 10^{-19}\,\text{Дж}$.
    Какова длина волны этой линии спектра?
    Постоянная Планка $h = 6{,}626 \cdot 10^{-34}\,\text{Дж}\cdot\text{с}$, скорость света $c = 3 \cdot 10^{8}\,\frac{\text{м}}{\text{с}}$.
}
\answer{%
    $
        E = h\nu = h \frac c\lambda
        \implies \lambda = \frac{hc}E
            = \frac{6{,}626 \cdot 10^{-34}\,\text{Дж}\cdot\text{с} \cdot {3 \cdot 10^{8}\,\frac{\text{м}}{\text{с}}}}{4{,}04 \cdot 10^{-19}\,\text{Дж}}
            = 492{,}03\,\text{нм}.
    $
}
\solutionspace{150pt}

\tasknumber{2}%
\task{%
    Излучение какой длины волны поглотил атом водорода, если полная энергия в атоме увеличилась на $4 \cdot 10^{-19}\,\text{Дж}$?
    Постоянная Планка $h = 6{,}626 \cdot 10^{-34}\,\text{Дж}\cdot\text{с}$, скорость света $c = 3 \cdot 10^{8}\,\frac{\text{м}}{\text{с}}$.
}
\answer{%
    $
        E = h\nu = h \frac c\lambda
        \implies \lambda = \frac{hc}E
            = \frac{6{,}626 \cdot 10^{-34}\,\text{Дж}\cdot\text{с} \cdot {3 \cdot 10^{8}\,\frac{\text{м}}{\text{с}}}}{4 \cdot 10^{-19}\,\text{Дж}}
            = 496{,}95\,\text{нм}.
    $
}
\solutionspace{150pt}

\tasknumber{3}%
\task{%
    Сделайте схематичный рисунок энергетических уровней атома водорода
    и отметьте на нём первый (основной) уровень и последующие.
    Сколько различных длин волн может испустить атом водорода,
    находящийся в 3-м возбуждённом состоянии?
    Отметьте все соответствующие переходы на рисунке и укажите,
    при каком переходе (среди отмеченных) частота излучённого фотона минимальна.
}
\answer{%
    $N = 3{,}0, \text{самая короткая линия}$
}
% autogenerated
