\setdate{10~сентября~2020}
\setclass{9«М»}

\addpersonalvariant{Михаил Бурмистров}

\tasknumber{1}%
\task{%
    Запишите определения, формулы и физические законы (можно сокращать, но не упустите ключевое):
    \begin{enumerate}
        \item материальная точка,
        \item механическое движение,
        \item путь.
    \end{enumerate}
}
\solutionspace{120pt}

\tasknumber{2}%
\task{%
    Небольшой лёгкий самолёт взлетел из аэропорта, пролетел $30\,\text{км}$ строго на юг, потом повернул и пролетел $40\,\text{км}$ на запад,
    а после по прямой вернулся обратно в аэропорт.
    Определите путь и модуль перемещения самолёта, считая Землю плоской.
}
\solutionspace{120pt}

\tasknumber{3}%
\task{%
    Саша плавает в бассейне длиной $25\,\text{м}$: от одного бортика к другому и обратно.
    Определите её перемещение, если её путь к текущему моменту составил $290\,\text{м}$.
}
\answer{%
    $10\,\text{м}$
}

\variantsplitter

\addpersonalvariant{Артём Глембо}

\tasknumber{1}%
\task{%
    Запишите определения, формулы и физические законы (можно сокращать, но не упустите ключевое):
    \begin{enumerate}
        \item система отсчёта,
        \item поступательное движение,
        \item траектория.
    \end{enumerate}
}
\solutionspace{120pt}

\tasknumber{2}%
\task{%
    Небольшой лёгкий самолёт взлетел из аэропорта, пролетел $30\,\text{км}$ строго на север, потом повернул и пролетел $40\,\text{км}$ на восток,
    а после по прямой вернулся обратно в аэропорт.
    Определите путь и модуль перемещения самолёта, считая Землю плоской.
}
\solutionspace{120pt}

\tasknumber{3}%
\task{%
    Валя плавает в бассейне длиной $25\,\text{м}$: от одного бортика к другому и обратно.
    Определите его перемещение, если его путь к текущему моменту составил $290\,\text{м}$.
}
\answer{%
    $10\,\text{м}$
}

\variantsplitter

\addpersonalvariant{Наталья Гончарова}

\tasknumber{1}%
\task{%
    Запишите определения, формулы и физические законы (можно сокращать, но не упустите ключевое):
    \begin{enumerate}
        \item система отсчёта,
        \item поступательное движение,
        \item путь.
    \end{enumerate}
}
\solutionspace{120pt}

\tasknumber{2}%
\task{%
    Небольшой лёгкий самолёт взлетел из аэропорта, пролетел $30\,\text{км}$ строго на юг, потом повернул и пролетел $40\,\text{км}$ на запад,
    а после по прямой вернулся обратно в аэропорт.
    Определите путь и модуль перемещения самолёта, считая Землю плоской.
}
\solutionspace{120pt}

\tasknumber{3}%
\task{%
    Саша плавает в бассейне длиной $50\,\text{м}$: от одного бортика к другому и обратно.
    Определите её перемещение, если её путь к текущему моменту составил $310\,\text{м}$.
}
\answer{%
    $10\,\text{м}$
}

\variantsplitter

\addpersonalvariant{Файёзбек Касымов}

\tasknumber{1}%
\task{%
    Запишите определения, формулы и физические законы (можно сокращать, но не упустите ключевое):
    \begin{enumerate}
        \item материальная точка,
        \item поступательное движение,
        \item траектория.
    \end{enumerate}
}
\solutionspace{120pt}

\tasknumber{2}%
\task{%
    Небольшой лёгкий самолёт взлетел из аэропорта, пролетел $7\,\text{км}$ строго на север, потом повернул и пролетел $24\,\text{км}$ на восток,
    а после по прямой вернулся обратно в аэропорт.
    Определите путь и модуль перемещения самолёта, считая Землю плоской.
}
\solutionspace{120pt}

\tasknumber{3}%
\task{%
    Валя плавает в бассейне длиной $25\,\text{м}$: от одного бортика к другому и обратно.
    Определите её перемещение, если её путь к текущему моменту составил $250\,\text{м}$.
}
\answer{%
    $0\,\text{м}$
}

\variantsplitter

\addpersonalvariant{Александр Козинец}

\tasknumber{1}%
\task{%
    Запишите определения, формулы и физические законы (можно сокращать, но не упустите ключевое):
    \begin{enumerate}
        \item материальная точка,
        \item поступательное движение,
        \item перемещение.
    \end{enumerate}
}
\solutionspace{120pt}

\tasknumber{2}%
\task{%
    Небольшой лёгкий самолёт взлетел из аэропорта, пролетел $40\,\text{км}$ строго на север, потом повернул и пролетел $30\,\text{км}$ на восток,
    а после по прямой вернулся обратно в аэропорт.
    Определите путь и модуль перемещения самолёта, считая Землю плоской.
}
\solutionspace{120pt}

\tasknumber{3}%
\task{%
    Женя плавает в бассейне длиной $50\,\text{м}$: от одного бортика к другому и обратно.
    Определите её перемещение, если её путь к текущему моменту составил $150\,\text{м}$.
}
\answer{%
    $50\,\text{м}$
}

\variantsplitter

\addpersonalvariant{Андрей Куликовский}

\tasknumber{1}%
\task{%
    Запишите определения, формулы и физические законы (можно сокращать, но не упустите ключевое):
    \begin{enumerate}
        \item материальная точка,
        \item поступательное движение,
        \item путь.
    \end{enumerate}
}
\solutionspace{120pt}

\tasknumber{2}%
\task{%
    Небольшой лёгкий самолёт взлетел из аэропорта, пролетел $24\,\text{км}$ строго на север, потом повернул и пролетел $7\,\text{км}$ на восток,
    а после по прямой вернулся обратно в аэропорт.
    Определите путь и модуль перемещения самолёта, считая Землю плоской.
}
\solutionspace{120pt}

\tasknumber{3}%
\task{%
    Женя плавает в бассейне длиной $50\,\text{м}$: от одного бортика к другому и обратно.
    Определите его перемещение, если его путь к текущему моменту составил $290\,\text{м}$.
}
\answer{%
    $10\,\text{м}$
}

\variantsplitter

\addpersonalvariant{Полина Лоткова}

\tasknumber{1}%
\task{%
    Запишите определения, формулы и физические законы (можно сокращать, но не упустите ключевое):
    \begin{enumerate}
        \item материальная точка,
        \item поступательное движение,
        \item перемещение.
    \end{enumerate}
}
\solutionspace{120pt}

\tasknumber{2}%
\task{%
    Небольшой лёгкий самолёт взлетел из аэропорта, пролетел $7\,\text{км}$ строго на север, потом повернул и пролетел $24\,\text{км}$ на восток,
    а после по прямой вернулся обратно в аэропорт.
    Определите путь и модуль перемещения самолёта, считая Землю плоской.
}
\solutionspace{120pt}

\tasknumber{3}%
\task{%
    Валя плавает в бассейне длиной $50\,\text{м}$: от одного бортика к другому и обратно.
    Определите её перемещение, если её путь к текущему моменту составил $230\,\text{м}$.
}
\answer{%
    $30\,\text{м}$
}

\variantsplitter

\addpersonalvariant{Екатерина Медведева}

\tasknumber{1}%
\task{%
    Запишите определения, формулы и физические законы (можно сокращать, но не упустите ключевое):
    \begin{enumerate}
        \item материальная точка,
        \item поступательное движение,
        \item траектория.
    \end{enumerate}
}
\solutionspace{120pt}

\tasknumber{2}%
\task{%
    Небольшой лёгкий самолёт взлетел из аэропорта, пролетел $7\,\text{км}$ строго на юг, потом повернул и пролетел $24\,\text{км}$ на восток,
    а после по прямой вернулся обратно в аэропорт.
    Определите путь и модуль перемещения самолёта, считая Землю плоской.
}
\solutionspace{120pt}

\tasknumber{3}%
\task{%
    Валя плавает в бассейне длиной $50\,\text{м}$: от одного бортика к другому и обратно.
    Определите её перемещение, если её путь к текущему моменту составил $170\,\text{м}$.
}
\answer{%
    $30\,\text{м}$
}

\variantsplitter

\addpersonalvariant{Константин Мельник}

\tasknumber{1}%
\task{%
    Запишите определения, формулы и физические законы (можно сокращать, но не упустите ключевое):
    \begin{enumerate}
        \item система отсчёта,
        \item поступательное движение,
        \item траектория.
    \end{enumerate}
}
\solutionspace{120pt}

\tasknumber{2}%
\task{%
    Небольшой лёгкий самолёт взлетел из аэропорта, пролетел $7\,\text{км}$ строго на север, потом повернул и пролетел $24\,\text{км}$ на восток,
    а после по прямой вернулся обратно в аэропорт.
    Определите путь и модуль перемещения самолёта, считая Землю плоской.
}
\solutionspace{120pt}

\tasknumber{3}%
\task{%
    Женя плавает в бассейне длиной $50\,\text{м}$: от одного бортика к другому и обратно.
    Определите его перемещение, если его путь к текущему моменту составил $210\,\text{м}$.
}
\answer{%
    $10\,\text{м}$
}

\variantsplitter

\addpersonalvariant{Степан Небоваренков}

\tasknumber{1}%
\task{%
    Запишите определения, формулы и физические законы (можно сокращать, но не упустите ключевое):
    \begin{enumerate}
        \item материальная точка,
        \item механическое движение,
        \item перемещение.
    \end{enumerate}
}
\solutionspace{120pt}

\tasknumber{2}%
\task{%
    Небольшой лёгкий самолёт взлетел из аэропорта, пролетел $40\,\text{км}$ строго на север, потом повернул и пролетел $30\,\text{км}$ на запад,
    а после по прямой вернулся обратно в аэропорт.
    Определите путь и модуль перемещения самолёта, считая Землю плоской.
}
\solutionspace{120pt}

\tasknumber{3}%
\task{%
    Женя плавает в бассейне длиной $25\,\text{м}$: от одного бортика к другому и обратно.
    Определите её перемещение, если её путь к текущему моменту составил $250\,\text{м}$.
}
\answer{%
    $0\,\text{м}$
}

\variantsplitter

\addpersonalvariant{Матвей Неретин}

\tasknumber{1}%
\task{%
    Запишите определения, формулы и физические законы (можно сокращать, но не упустите ключевое):
    \begin{enumerate}
        \item основная задача механики,
        \item поступательное движение,
        \item перемещение.
    \end{enumerate}
}
\solutionspace{120pt}

\tasknumber{2}%
\task{%
    Небольшой лёгкий самолёт взлетел из аэропорта, пролетел $12\,\text{км}$ строго на юг, потом повернул и пролетел $5\,\text{км}$ на восток,
    а после по прямой вернулся обратно в аэропорт.
    Определите путь и модуль перемещения самолёта, считая Землю плоской.
}
\solutionspace{120pt}

\tasknumber{3}%
\task{%
    Женя плавает в бассейне длиной $50\,\text{м}$: от одного бортика к другому и обратно.
    Определите её перемещение, если её путь к текущему моменту составил $150\,\text{м}$.
}
\answer{%
    $50\,\text{м}$
}

\variantsplitter

\addpersonalvariant{Мария Никонова}

\tasknumber{1}%
\task{%
    Запишите определения, формулы и физические законы (можно сокращать, но не упустите ключевое):
    \begin{enumerate}
        \item система отсчёта,
        \item механическое движение,
        \item траектория.
    \end{enumerate}
}
\solutionspace{120pt}

\tasknumber{2}%
\task{%
    Небольшой лёгкий самолёт взлетел из аэропорта, пролетел $24\,\text{км}$ строго на север, потом повернул и пролетел $7\,\text{км}$ на восток,
    а после по прямой вернулся обратно в аэропорт.
    Определите путь и модуль перемещения самолёта, считая Землю плоской.
}
\solutionspace{120pt}

\tasknumber{3}%
\task{%
    Саша плавает в бассейне длиной $25\,\text{м}$: от одного бортика к другому и обратно.
    Определите её перемещение, если её путь к текущему моменту составил $310\,\text{м}$.
}
\answer{%
    $10\,\text{м}$
}

\variantsplitter

\addpersonalvariant{Даниил Палаткин}

\tasknumber{1}%
\task{%
    Запишите определения, формулы и физические законы (можно сокращать, но не упустите ключевое):
    \begin{enumerate}
        \item основная задача механики,
        \item поступательное движение,
        \item траектория.
    \end{enumerate}
}
\solutionspace{120pt}

\tasknumber{2}%
\task{%
    Небольшой лёгкий самолёт взлетел из аэропорта, пролетел $24\,\text{км}$ строго на юг, потом повернул и пролетел $7\,\text{км}$ на восток,
    а после по прямой вернулся обратно в аэропорт.
    Определите путь и модуль перемещения самолёта, считая Землю плоской.
}
\solutionspace{120pt}

\tasknumber{3}%
\task{%
    Валя плавает в бассейне длиной $25\,\text{м}$: от одного бортика к другому и обратно.
    Определите её перемещение, если её путь к текущему моменту составил $330\,\text{м}$.
}
\answer{%
    $20\,\text{м}$
}

\variantsplitter

\addpersonalvariant{Станислав Пикун}

\tasknumber{1}%
\task{%
    Запишите определения, формулы и физические законы (можно сокращать, но не упустите ключевое):
    \begin{enumerate}
        \item основная задача механики,
        \item поступательное движение,
        \item траектория.
    \end{enumerate}
}
\solutionspace{120pt}

\tasknumber{2}%
\task{%
    Небольшой лёгкий самолёт взлетел из аэропорта, пролетел $5\,\text{км}$ строго на север, потом повернул и пролетел $12\,\text{км}$ на запад,
    а после по прямой вернулся обратно в аэропорт.
    Определите путь и модуль перемещения самолёта, считая Землю плоской.
}
\solutionspace{120pt}

\tasknumber{3}%
\task{%
    Валя плавает в бассейне длиной $25\,\text{м}$: от одного бортика к другому и обратно.
    Определите его перемещение, если его путь к текущему моменту составил $310\,\text{м}$.
}
\answer{%
    $10\,\text{м}$
}

\variantsplitter

\addpersonalvariant{Илья Пичугин}

\tasknumber{1}%
\task{%
    Запишите определения, формулы и физические законы (можно сокращать, но не упустите ключевое):
    \begin{enumerate}
        \item система отсчёта,
        \item механическое движение,
        \item траектория.
    \end{enumerate}
}
\solutionspace{120pt}

\tasknumber{2}%
\task{%
    Небольшой лёгкий самолёт взлетел из аэропорта, пролетел $30\,\text{км}$ строго на юг, потом повернул и пролетел $40\,\text{км}$ на восток,
    а после по прямой вернулся обратно в аэропорт.
    Определите путь и модуль перемещения самолёта, считая Землю плоской.
}
\solutionspace{120pt}

\tasknumber{3}%
\task{%
    Валя плавает в бассейне длиной $50\,\text{м}$: от одного бортика к другому и обратно.
    Определите его перемещение, если его путь к текущему моменту составил $170\,\text{м}$.
}
\answer{%
    $30\,\text{м}$
}

\variantsplitter

\addpersonalvariant{Кирилл Севрюгин}

\tasknumber{1}%
\task{%
    Запишите определения, формулы и физические законы (можно сокращать, но не упустите ключевое):
    \begin{enumerate}
        \item материальная точка,
        \item механическое движение,
        \item перемещение.
    \end{enumerate}
}
\solutionspace{120pt}

\tasknumber{2}%
\task{%
    Небольшой лёгкий самолёт взлетел из аэропорта, пролетел $5\,\text{км}$ строго на север, потом повернул и пролетел $12\,\text{км}$ на восток,
    а после по прямой вернулся обратно в аэропорт.
    Определите путь и модуль перемещения самолёта, считая Землю плоской.
}
\solutionspace{120pt}

\tasknumber{3}%
\task{%
    Валя плавает в бассейне длиной $50\,\text{м}$: от одного бортика к другому и обратно.
    Определите его перемещение, если его путь к текущему моменту составил $330\,\text{м}$.
}
\answer{%
    $30\,\text{м}$
}

\variantsplitter

\addpersonalvariant{Илья Стратонников}

\tasknumber{1}%
\task{%
    Запишите определения, формулы и физические законы (можно сокращать, но не упустите ключевое):
    \begin{enumerate}
        \item материальная точка,
        \item механическое движение,
        \item перемещение.
    \end{enumerate}
}
\solutionspace{120pt}

\tasknumber{2}%
\task{%
    Небольшой лёгкий самолёт взлетел из аэропорта, пролетел $30\,\text{км}$ строго на север, потом повернул и пролетел $40\,\text{км}$ на запад,
    а после по прямой вернулся обратно в аэропорт.
    Определите путь и модуль перемещения самолёта, считая Землю плоской.
}
\solutionspace{120pt}

\tasknumber{3}%
\task{%
    Валя плавает в бассейне длиной $25\,\text{м}$: от одного бортика к другому и обратно.
    Определите его перемещение, если его путь к текущему моменту составил $210\,\text{м}$.
}
\answer{%
    $10\,\text{м}$
}

\variantsplitter

\addpersonalvariant{Иван Шустов}

\tasknumber{1}%
\task{%
    Запишите определения, формулы и физические законы (можно сокращать, но не упустите ключевое):
    \begin{enumerate}
        \item система отсчёта,
        \item механическое движение,
        \item путь.
    \end{enumerate}
}
\solutionspace{120pt}

\tasknumber{2}%
\task{%
    Небольшой лёгкий самолёт взлетел из аэропорта, пролетел $24\,\text{км}$ строго на юг, потом повернул и пролетел $7\,\text{км}$ на запад,
    а после по прямой вернулся обратно в аэропорт.
    Определите путь и модуль перемещения самолёта, считая Землю плоской.
}
\solutionspace{120pt}

\tasknumber{3}%
\task{%
    Саша плавает в бассейне длиной $25\,\text{м}$: от одного бортика к другому и обратно.
    Определите его перемещение, если его путь к текущему моменту составил $250\,\text{м}$.
}
\answer{%
    $0\,\text{м}$
}
% autogenerated
