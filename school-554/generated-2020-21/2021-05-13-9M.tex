\setdate{13~мая~2021}
\setclass{9«М»}

\addpersonalvariant{Михаил Бурмистров}

\tasknumber{1}%
\task{%
    Какая доля (от начального количества) радиоактивных ядер останется через время,
    равное двум периодам полураспада? Ответ выразить в процентах.
}
\answer{%
    \begin{align*}
    N &= N_0 \cdot 2^{- \frac t{T_{1/2}}} \implies
        \frac N{N_0} = 2^{- \frac t{T_{1/2}}}
        = 2^{-2} \approx 0{,}25 \approx 25\% \\
    N_\text{расп.} &= N_0 - N = N_0 - N_0 \cdot 2^{-\frac t{T_{1/2}}}
        = N_0\cbr{1 - 2^{-\frac t{T_{1/2}}}} \implies
        \frac{N_\text{расп.}}{N_0} = 1 - 2^{-\frac t{T_{1/2}}}
        = 1 - 2^{-2} \approx 0{,}75 \approx 75\%
    \end{align*}
}
\solutionspace{150pt}

\tasknumber{2}%
\task{%
    Сколько процентов ядер радиоактивного железа $\ce{^{59}Fe}$
    останется через $136{,}8\,\text{суток}$, если период его полураспада составляет $45{,}6\,\text{суток}$?
}
\answer{%
    \begin{align*}
    N &= N_0 \cdot 2^{-\frac t{T_{1/2}}}
        = 2^{-\frac{136{,}8\,\text{суток}}{45{,}6\,\text{суток}}}
        \approx 0{,}125 \approx 12{,}5\%
    \end{align*}
}
\solutionspace{150pt}

\tasknumber{3}%
\task{%
    За $2\,\text{суток}$ от начального количества ядер радиоизотопа осталась половина.
    Каков период полураспада этого изотопа (ответ приведите в сутках)?
    Какая ещё доля (также от начального количества) распадётся, если подождать ещё столько же?
}
\answer{%
    \begin{align*}
            N &= N_0 \cdot 2^{-\frac t{T_{1/2}}}
            \implies \frac N{N_0} = 2^{-\frac t{T_{1/2}}}
            \implies \frac 1{2} = 2^{-\frac {2\,\text{суток}}{T_{1/2}}}
            \implies 1 = \frac {2\,\text{суток}}{T_{1/2}}
            \implies T_{1/2} = \frac {2\,\text{суток}}1 \approx 2{,}0\,\text{суток}.
         \\
            \delta &= \frac{N(t)}{N_0} - \frac{N(2t)}{N_0}
            = 2^{-\frac t{T_{1/2}}} - 2^{-\frac {2t}{T_{1/2}}}
            = 2^{-\frac t{T_{1/2}}}\cbr{1 - 2^{-\frac t{T_{1/2}}}}
            = \frac 1{2} \cdot \cbr{1-\frac 1{2}} \approx 0{,}250
    \end{align*}
}
\solutionspace{150pt}

\tasknumber{4}%
\task{%
    Энергия связи ядра гелия \ce{^{3}_{2}He} равна $7{,}72\,\text{МэВ}$.
    Найти дефект массы этого ядра.
    Ответ выразите в а.е.м.
    и кг.
    Скорость света $c = 2{,}998 \cdot 10^{8}\,\frac{\text{м}}{\text{с}}$, элементарный заряд $e = 1{,}6 \cdot 10^{-19}\,\text{Кл}$.
}
\answer{%
    \begin{align*}
    E_\text{св.} &= \Delta m c^2 \implies \\
    \implies
            \Delta m &= \frac {E_\text{св.}}{c^2} = \frac{7{,}72\,\text{МэВ}}{\sqr{2{,}998 \cdot 10^{8}\,\frac{\text{м}}{\text{с}}}}
            = \frac{7{,}72 \cdot 10^6 \cdot 1{,}6 \cdot 10^{-19}\,\text{Дж}}{\sqr{2{,}998 \cdot 10^{8}\,\frac{\text{м}}{\text{с}}}}
            \approx 1{,}374 \cdot 10^{-29}\,\text{кг} \approx 0{,}00828\,\text{а.е.м.}
    \end{align*}
}

\variantsplitter

\addpersonalvariant{Артём Глембо}

\tasknumber{1}%
\task{%
    Какая доля (от начального количества) радиоактивных ядер распадётся через время,
    равное четырём периодам полураспада? Ответ выразить в процентах.
}
\answer{%
    \begin{align*}
    N &= N_0 \cdot 2^{- \frac t{T_{1/2}}} \implies
        \frac N{N_0} = 2^{- \frac t{T_{1/2}}}
        = 2^{-4} \approx 0{,}06 \approx 6\% \\
    N_\text{расп.} &= N_0 - N = N_0 - N_0 \cdot 2^{-\frac t{T_{1/2}}}
        = N_0\cbr{1 - 2^{-\frac t{T_{1/2}}}} \implies
        \frac{N_\text{расп.}}{N_0} = 1 - 2^{-\frac t{T_{1/2}}}
        = 1 - 2^{-4} \approx 0{,}94 \approx 94\%
    \end{align*}
}
\solutionspace{150pt}

\tasknumber{2}%
\task{%
    Сколько процентов ядер радиоактивного железа $\ce{^{59}Fe}$
    останется через $136{,}8\,\text{суток}$, если период его полураспада составляет $45{,}6\,\text{суток}$?
}
\answer{%
    \begin{align*}
    N &= N_0 \cdot 2^{-\frac t{T_{1/2}}}
        = 2^{-\frac{136{,}8\,\text{суток}}{45{,}6\,\text{суток}}}
        \approx 0{,}125 \approx 12{,}5\%
    \end{align*}
}
\solutionspace{150pt}

\tasknumber{3}%
\task{%
    За $2\,\text{суток}$ от начального количества ядер радиоизотопа осталась половина.
    Каков период полураспада этого изотопа (ответ приведите в сутках)?
    Какая ещё доля (также от начального количества) распадётся, если подождать ещё столько же?
}
\answer{%
    \begin{align*}
            N &= N_0 \cdot 2^{-\frac t{T_{1/2}}}
            \implies \frac N{N_0} = 2^{-\frac t{T_{1/2}}}
            \implies \frac 1{2} = 2^{-\frac {2\,\text{суток}}{T_{1/2}}}
            \implies 1 = \frac {2\,\text{суток}}{T_{1/2}}
            \implies T_{1/2} = \frac {2\,\text{суток}}1 \approx 2{,}0\,\text{суток}.
         \\
            \delta &= \frac{N(t)}{N_0} - \frac{N(2t)}{N_0}
            = 2^{-\frac t{T_{1/2}}} - 2^{-\frac {2t}{T_{1/2}}}
            = 2^{-\frac t{T_{1/2}}}\cbr{1 - 2^{-\frac t{T_{1/2}}}}
            = \frac 1{2} \cdot \cbr{1-\frac 1{2}} \approx 0{,}250
    \end{align*}
}
\solutionspace{150pt}

\tasknumber{4}%
\task{%
    Энергия связи ядра гелия \ce{^{3}_{2}He} равна $7{,}72\,\text{МэВ}$.
    Найти дефект массы этого ядра.
    Ответ выразите в а.е.м.
    и кг.
    Скорость света $c = 2{,}998 \cdot 10^{8}\,\frac{\text{м}}{\text{с}}$, элементарный заряд $e = 1{,}6 \cdot 10^{-19}\,\text{Кл}$.
}
\answer{%
    \begin{align*}
    E_\text{св.} &= \Delta m c^2 \implies \\
    \implies
            \Delta m &= \frac {E_\text{св.}}{c^2} = \frac{7{,}72\,\text{МэВ}}{\sqr{2{,}998 \cdot 10^{8}\,\frac{\text{м}}{\text{с}}}}
            = \frac{7{,}72 \cdot 10^6 \cdot 1{,}6 \cdot 10^{-19}\,\text{Дж}}{\sqr{2{,}998 \cdot 10^{8}\,\frac{\text{м}}{\text{с}}}}
            \approx 1{,}374 \cdot 10^{-29}\,\text{кг} \approx 0{,}00828\,\text{а.е.м.}
    \end{align*}
}

\variantsplitter

\addpersonalvariant{Наталья Гончарова}

\tasknumber{1}%
\task{%
    Какая доля (от начального количества) радиоактивных ядер распадётся через время,
    равное трём периодам полураспада? Ответ выразить в процентах.
}
\answer{%
    \begin{align*}
    N &= N_0 \cdot 2^{- \frac t{T_{1/2}}} \implies
        \frac N{N_0} = 2^{- \frac t{T_{1/2}}}
        = 2^{-3} \approx 0{,}12 \approx 12\% \\
    N_\text{расп.} &= N_0 - N = N_0 - N_0 \cdot 2^{-\frac t{T_{1/2}}}
        = N_0\cbr{1 - 2^{-\frac t{T_{1/2}}}} \implies
        \frac{N_\text{расп.}}{N_0} = 1 - 2^{-\frac t{T_{1/2}}}
        = 1 - 2^{-3} \approx 0{,}88 \approx 88\%
    \end{align*}
}
\solutionspace{150pt}

\tasknumber{2}%
\task{%
    Сколько процентов ядер радиоактивного железа $\ce{^{59}Fe}$
    останется через $136{,}8\,\text{суток}$, если период его полураспада составляет $45{,}6\,\text{суток}$?
}
\answer{%
    \begin{align*}
    N &= N_0 \cdot 2^{-\frac t{T_{1/2}}}
        = 2^{-\frac{136{,}8\,\text{суток}}{45{,}6\,\text{суток}}}
        \approx 0{,}125 \approx 12{,}5\%
    \end{align*}
}
\solutionspace{150pt}

\tasknumber{3}%
\task{%
    За $5\,\text{суток}$ от начального количества ядер радиоизотопа осталась одна шестнадцатая.
    Каков период полураспада этого изотопа (ответ приведите в сутках)?
    Какая ещё доля (также от начального количества) распадётся, если подождать ещё столько же?
}
\answer{%
    \begin{align*}
            N &= N_0 \cdot 2^{-\frac t{T_{1/2}}}
            \implies \frac N{N_0} = 2^{-\frac t{T_{1/2}}}
            \implies \frac 1{16} = 2^{-\frac {5\,\text{суток}}{T_{1/2}}}
            \implies 4 = \frac {5\,\text{суток}}{T_{1/2}}
            \implies T_{1/2} = \frac {5\,\text{суток}}4 \approx 1{,}2\,\text{суток}.
         \\
            \delta &= \frac{N(t)}{N_0} - \frac{N(2t)}{N_0}
            = 2^{-\frac t{T_{1/2}}} - 2^{-\frac {2t}{T_{1/2}}}
            = 2^{-\frac t{T_{1/2}}}\cbr{1 - 2^{-\frac t{T_{1/2}}}}
            = \frac 1{16} \cdot \cbr{1-\frac 1{16}} \approx 0{,}059
    \end{align*}
}
\solutionspace{150pt}

\tasknumber{4}%
\task{%
    Энергия связи ядра лития \ce{^{6}_{3}Li} равна $31{,}99\,\text{МэВ}$.
    Найти дефект массы этого ядра.
    Ответ выразите в а.е.м.
    и кг.
    Скорость света $c = 2{,}998 \cdot 10^{8}\,\frac{\text{м}}{\text{с}}$, элементарный заряд $e = 1{,}6 \cdot 10^{-19}\,\text{Кл}$.
}
\answer{%
    \begin{align*}
    E_\text{св.} &= \Delta m c^2 \implies \\
    \implies
            \Delta m &= \frac {E_\text{св.}}{c^2} = \frac{31{,}99\,\text{МэВ}}{\sqr{2{,}998 \cdot 10^{8}\,\frac{\text{м}}{\text{с}}}}
            = \frac{31{,}99 \cdot 10^6 \cdot 1{,}6 \cdot 10^{-19}\,\text{Дж}}{\sqr{2{,}998 \cdot 10^{8}\,\frac{\text{м}}{\text{с}}}}
            \approx 5{,}69 \cdot 10^{-29}\,\text{кг} \approx 0{,}0343\,\text{а.е.м.}
    \end{align*}
}

\variantsplitter

\addpersonalvariant{Файёзбек Касымов}

\tasknumber{1}%
\task{%
    Какая доля (от начального количества) радиоактивных ядер распадётся через время,
    равное двум периодам полураспада? Ответ выразить в процентах.
}
\answer{%
    \begin{align*}
    N &= N_0 \cdot 2^{- \frac t{T_{1/2}}} \implies
        \frac N{N_0} = 2^{- \frac t{T_{1/2}}}
        = 2^{-2} \approx 0{,}25 \approx 25\% \\
    N_\text{расп.} &= N_0 - N = N_0 - N_0 \cdot 2^{-\frac t{T_{1/2}}}
        = N_0\cbr{1 - 2^{-\frac t{T_{1/2}}}} \implies
        \frac{N_\text{расп.}}{N_0} = 1 - 2^{-\frac t{T_{1/2}}}
        = 1 - 2^{-2} \approx 0{,}75 \approx 75\%
    \end{align*}
}
\solutionspace{150pt}

\tasknumber{2}%
\task{%
    Сколько процентов ядер радиоактивного железа $\ce{^{59}Fe}$
    останется через $136{,}8\,\text{суток}$, если период его полураспада составляет $45{,}6\,\text{суток}$?
}
\answer{%
    \begin{align*}
    N &= N_0 \cdot 2^{-\frac t{T_{1/2}}}
        = 2^{-\frac{136{,}8\,\text{суток}}{45{,}6\,\text{суток}}}
        \approx 0{,}125 \approx 12{,}5\%
    \end{align*}
}
\solutionspace{150pt}

\tasknumber{3}%
\task{%
    За $4\,\text{суток}$ от начального количества ядер радиоизотопа осталась четверть.
    Каков период полураспада этого изотопа (ответ приведите в сутках)?
    Какая ещё доля (также от начального количества) распадётся, если подождать ещё столько же?
}
\answer{%
    \begin{align*}
            N &= N_0 \cdot 2^{-\frac t{T_{1/2}}}
            \implies \frac N{N_0} = 2^{-\frac t{T_{1/2}}}
            \implies \frac 1{4} = 2^{-\frac {4\,\text{суток}}{T_{1/2}}}
            \implies 2 = \frac {4\,\text{суток}}{T_{1/2}}
            \implies T_{1/2} = \frac {4\,\text{суток}}2 \approx 2{,}0\,\text{суток}.
         \\
            \delta &= \frac{N(t)}{N_0} - \frac{N(2t)}{N_0}
            = 2^{-\frac t{T_{1/2}}} - 2^{-\frac {2t}{T_{1/2}}}
            = 2^{-\frac t{T_{1/2}}}\cbr{1 - 2^{-\frac t{T_{1/2}}}}
            = \frac 1{4} \cdot \cbr{1-\frac 1{4}} \approx 0{,}188
    \end{align*}
}
\solutionspace{150pt}

\tasknumber{4}%
\task{%
    Энергия связи ядра дейтерия \ce{^{2}_{1}H} (D) равна $2{,}22\,\text{МэВ}$.
    Найти дефект массы этого ядра.
    Ответ выразите в а.е.м.
    и кг.
    Скорость света $c = 2{,}998 \cdot 10^{8}\,\frac{\text{м}}{\text{с}}$, элементарный заряд $e = 1{,}6 \cdot 10^{-19}\,\text{Кл}$.
}
\answer{%
    \begin{align*}
    E_\text{св.} &= \Delta m c^2 \implies \\
    \implies
            \Delta m &= \frac {E_\text{св.}}{c^2} = \frac{2{,}22\,\text{МэВ}}{\sqr{2{,}998 \cdot 10^{8}\,\frac{\text{м}}{\text{с}}}}
            = \frac{2{,}22 \cdot 10^6 \cdot 1{,}6 \cdot 10^{-19}\,\text{Дж}}{\sqr{2{,}998 \cdot 10^{8}\,\frac{\text{м}}{\text{с}}}}
            \approx 0{,}395 \cdot 10^{-29}\,\text{кг} \approx 0{,}00238\,\text{а.е.м.}
    \end{align*}
}

\variantsplitter

\addpersonalvariant{Александр Козинец}

\tasknumber{1}%
\task{%
    Какая доля (от начального количества) радиоактивных ядер распадётся через время,
    равное четырём периодам полураспада? Ответ выразить в процентах.
}
\answer{%
    \begin{align*}
    N &= N_0 \cdot 2^{- \frac t{T_{1/2}}} \implies
        \frac N{N_0} = 2^{- \frac t{T_{1/2}}}
        = 2^{-4} \approx 0{,}06 \approx 6\% \\
    N_\text{расп.} &= N_0 - N = N_0 - N_0 \cdot 2^{-\frac t{T_{1/2}}}
        = N_0\cbr{1 - 2^{-\frac t{T_{1/2}}}} \implies
        \frac{N_\text{расп.}}{N_0} = 1 - 2^{-\frac t{T_{1/2}}}
        = 1 - 2^{-4} \approx 0{,}94 \approx 94\%
    \end{align*}
}
\solutionspace{150pt}

\tasknumber{2}%
\task{%
    Сколько процентов ядер радиоактивного железа $\ce{^{59}Fe}$
    останется через $91{,}2\,\text{суток}$, если период его полураспада составляет $45{,}6\,\text{суток}$?
}
\answer{%
    \begin{align*}
    N &= N_0 \cdot 2^{-\frac t{T_{1/2}}}
        = 2^{-\frac{91{,}2\,\text{суток}}{45{,}6\,\text{суток}}}
        \approx 0{,}25 \approx 25{,}0\%
    \end{align*}
}
\solutionspace{150pt}

\tasknumber{3}%
\task{%
    За $2\,\text{суток}$ от начального количества ядер радиоизотопа осталась одна шестнадцатая.
    Каков период полураспада этого изотопа (ответ приведите в сутках)?
    Какая ещё доля (также от начального количества) распадётся, если подождать ещё столько же?
}
\answer{%
    \begin{align*}
            N &= N_0 \cdot 2^{-\frac t{T_{1/2}}}
            \implies \frac N{N_0} = 2^{-\frac t{T_{1/2}}}
            \implies \frac 1{16} = 2^{-\frac {2\,\text{суток}}{T_{1/2}}}
            \implies 4 = \frac {2\,\text{суток}}{T_{1/2}}
            \implies T_{1/2} = \frac {2\,\text{суток}}4 \approx 0{,}5\,\text{суток}.
         \\
            \delta &= \frac{N(t)}{N_0} - \frac{N(2t)}{N_0}
            = 2^{-\frac t{T_{1/2}}} - 2^{-\frac {2t}{T_{1/2}}}
            = 2^{-\frac t{T_{1/2}}}\cbr{1 - 2^{-\frac t{T_{1/2}}}}
            = \frac 1{16} \cdot \cbr{1-\frac 1{16}} \approx 0{,}059
    \end{align*}
}
\solutionspace{150pt}

\tasknumber{4}%
\task{%
    Энергия связи ядра кислорода \ce{^{17}_{8}O} равна $131{,}8\,\text{МэВ}$.
    Найти дефект массы этого ядра.
    Ответ выразите в а.е.м.
    и кг.
    Скорость света $c = 2{,}998 \cdot 10^{8}\,\frac{\text{м}}{\text{с}}$, элементарный заряд $e = 1{,}6 \cdot 10^{-19}\,\text{Кл}$.
}
\answer{%
    \begin{align*}
    E_\text{св.} &= \Delta m c^2 \implies \\
    \implies
            \Delta m &= \frac {E_\text{св.}}{c^2} = \frac{131{,}8\,\text{МэВ}}{\sqr{2{,}998 \cdot 10^{8}\,\frac{\text{м}}{\text{с}}}}
            = \frac{131{,}8 \cdot 10^6 \cdot 1{,}6 \cdot 10^{-19}\,\text{Дж}}{\sqr{2{,}998 \cdot 10^{8}\,\frac{\text{м}}{\text{с}}}}
            \approx 23{,}5 \cdot 10^{-29}\,\text{кг} \approx 0{,}1413\,\text{а.е.м.}
    \end{align*}
}

\variantsplitter

\addpersonalvariant{Андрей Куликовский}

\tasknumber{1}%
\task{%
    Какая доля (от начального количества) радиоактивных ядер останется через время,
    равное трём периодам полураспада? Ответ выразить в процентах.
}
\answer{%
    \begin{align*}
    N &= N_0 \cdot 2^{- \frac t{T_{1/2}}} \implies
        \frac N{N_0} = 2^{- \frac t{T_{1/2}}}
        = 2^{-3} \approx 0{,}12 \approx 12\% \\
    N_\text{расп.} &= N_0 - N = N_0 - N_0 \cdot 2^{-\frac t{T_{1/2}}}
        = N_0\cbr{1 - 2^{-\frac t{T_{1/2}}}} \implies
        \frac{N_\text{расп.}}{N_0} = 1 - 2^{-\frac t{T_{1/2}}}
        = 1 - 2^{-3} \approx 0{,}88 \approx 88\%
    \end{align*}
}
\solutionspace{150pt}

\tasknumber{2}%
\task{%
    Сколько процентов ядер радиоактивного железа $\ce{^{59}Fe}$
    останется через $136{,}8\,\text{суток}$, если период его полураспада составляет $45{,}6\,\text{суток}$?
}
\answer{%
    \begin{align*}
    N &= N_0 \cdot 2^{-\frac t{T_{1/2}}}
        = 2^{-\frac{136{,}8\,\text{суток}}{45{,}6\,\text{суток}}}
        \approx 0{,}125 \approx 12{,}5\%
    \end{align*}
}
\solutionspace{150pt}

\tasknumber{3}%
\task{%
    За $3\,\text{суток}$ от начального количества ядер радиоизотопа осталась одна восьмая.
    Каков период полураспада этого изотопа (ответ приведите в сутках)?
    Какая ещё доля (также от начального количества) распадётся, если подождать ещё столько же?
}
\answer{%
    \begin{align*}
            N &= N_0 \cdot 2^{-\frac t{T_{1/2}}}
            \implies \frac N{N_0} = 2^{-\frac t{T_{1/2}}}
            \implies \frac 1{8} = 2^{-\frac {3\,\text{суток}}{T_{1/2}}}
            \implies 3 = \frac {3\,\text{суток}}{T_{1/2}}
            \implies T_{1/2} = \frac {3\,\text{суток}}3 \approx 1{,}0\,\text{суток}.
         \\
            \delta &= \frac{N(t)}{N_0} - \frac{N(2t)}{N_0}
            = 2^{-\frac t{T_{1/2}}} - 2^{-\frac {2t}{T_{1/2}}}
            = 2^{-\frac t{T_{1/2}}}\cbr{1 - 2^{-\frac t{T_{1/2}}}}
            = \frac 1{8} \cdot \cbr{1-\frac 1{8}} \approx 0{,}109
    \end{align*}
}
\solutionspace{150pt}

\tasknumber{4}%
\task{%
    Энергия связи ядра бора \ce{^{10}_{5}B} равна $64{,}7\,\text{МэВ}$.
    Найти дефект массы этого ядра.
    Ответ выразите в а.е.м.
    и кг.
    Скорость света $c = 2{,}998 \cdot 10^{8}\,\frac{\text{м}}{\text{с}}$, элементарный заряд $e = 1{,}6 \cdot 10^{-19}\,\text{Кл}$.
}
\answer{%
    \begin{align*}
    E_\text{св.} &= \Delta m c^2 \implies \\
    \implies
            \Delta m &= \frac {E_\text{св.}}{c^2} = \frac{64{,}7\,\text{МэВ}}{\sqr{2{,}998 \cdot 10^{8}\,\frac{\text{м}}{\text{с}}}}
            = \frac{64{,}7 \cdot 10^6 \cdot 1{,}6 \cdot 10^{-19}\,\text{Дж}}{\sqr{2{,}998 \cdot 10^{8}\,\frac{\text{м}}{\text{с}}}}
            \approx 11{,}52 \cdot 10^{-29}\,\text{кг} \approx 0{,}0694\,\text{а.е.м.}
    \end{align*}
}

\variantsplitter

\addpersonalvariant{Полина Лоткова}

\tasknumber{1}%
\task{%
    Какая доля (от начального количества) радиоактивных ядер распадётся через время,
    равное двум периодам полураспада? Ответ выразить в процентах.
}
\answer{%
    \begin{align*}
    N &= N_0 \cdot 2^{- \frac t{T_{1/2}}} \implies
        \frac N{N_0} = 2^{- \frac t{T_{1/2}}}
        = 2^{-2} \approx 0{,}25 \approx 25\% \\
    N_\text{расп.} &= N_0 - N = N_0 - N_0 \cdot 2^{-\frac t{T_{1/2}}}
        = N_0\cbr{1 - 2^{-\frac t{T_{1/2}}}} \implies
        \frac{N_\text{расп.}}{N_0} = 1 - 2^{-\frac t{T_{1/2}}}
        = 1 - 2^{-2} \approx 0{,}75 \approx 75\%
    \end{align*}
}
\solutionspace{150pt}

\tasknumber{2}%
\task{%
    Сколько процентов ядер радиоактивного железа $\ce{^{59}Fe}$
    останется через $136{,}8\,\text{суток}$, если период его полураспада составляет $45{,}6\,\text{суток}$?
}
\answer{%
    \begin{align*}
    N &= N_0 \cdot 2^{-\frac t{T_{1/2}}}
        = 2^{-\frac{136{,}8\,\text{суток}}{45{,}6\,\text{суток}}}
        \approx 0{,}125 \approx 12{,}5\%
    \end{align*}
}
\solutionspace{150pt}

\tasknumber{3}%
\task{%
    За $4\,\text{суток}$ от начального количества ядер радиоизотопа осталась четверть.
    Каков период полураспада этого изотопа (ответ приведите в сутках)?
    Какая ещё доля (также от начального количества) распадётся, если подождать ещё столько же?
}
\answer{%
    \begin{align*}
            N &= N_0 \cdot 2^{-\frac t{T_{1/2}}}
            \implies \frac N{N_0} = 2^{-\frac t{T_{1/2}}}
            \implies \frac 1{4} = 2^{-\frac {4\,\text{суток}}{T_{1/2}}}
            \implies 2 = \frac {4\,\text{суток}}{T_{1/2}}
            \implies T_{1/2} = \frac {4\,\text{суток}}2 \approx 2{,}0\,\text{суток}.
         \\
            \delta &= \frac{N(t)}{N_0} - \frac{N(2t)}{N_0}
            = 2^{-\frac t{T_{1/2}}} - 2^{-\frac {2t}{T_{1/2}}}
            = 2^{-\frac t{T_{1/2}}}\cbr{1 - 2^{-\frac t{T_{1/2}}}}
            = \frac 1{4} \cdot \cbr{1-\frac 1{4}} \approx 0{,}188
    \end{align*}
}
\solutionspace{150pt}

\tasknumber{4}%
\task{%
    Энергия связи ядра азота \ce{^{14}_{7}N} равна $115{,}5\,\text{МэВ}$.
    Найти дефект массы этого ядра.
    Ответ выразите в а.е.м.
    и кг.
    Скорость света $c = 2{,}998 \cdot 10^{8}\,\frac{\text{м}}{\text{с}}$, элементарный заряд $e = 1{,}6 \cdot 10^{-19}\,\text{Кл}$.
}
\answer{%
    \begin{align*}
    E_\text{св.} &= \Delta m c^2 \implies \\
    \implies
            \Delta m &= \frac {E_\text{св.}}{c^2} = \frac{115{,}5\,\text{МэВ}}{\sqr{2{,}998 \cdot 10^{8}\,\frac{\text{м}}{\text{с}}}}
            = \frac{115{,}5 \cdot 10^6 \cdot 1{,}6 \cdot 10^{-19}\,\text{Дж}}{\sqr{2{,}998 \cdot 10^{8}\,\frac{\text{м}}{\text{с}}}}
            \approx 20{,}6 \cdot 10^{-29}\,\text{кг} \approx 0{,}1238\,\text{а.е.м.}
    \end{align*}
}

\variantsplitter

\addpersonalvariant{Екатерина Медведева}

\tasknumber{1}%
\task{%
    Какая доля (от начального количества) радиоактивных ядер останется через время,
    равное трём периодам полураспада? Ответ выразить в процентах.
}
\answer{%
    \begin{align*}
    N &= N_0 \cdot 2^{- \frac t{T_{1/2}}} \implies
        \frac N{N_0} = 2^{- \frac t{T_{1/2}}}
        = 2^{-3} \approx 0{,}12 \approx 12\% \\
    N_\text{расп.} &= N_0 - N = N_0 - N_0 \cdot 2^{-\frac t{T_{1/2}}}
        = N_0\cbr{1 - 2^{-\frac t{T_{1/2}}}} \implies
        \frac{N_\text{расп.}}{N_0} = 1 - 2^{-\frac t{T_{1/2}}}
        = 1 - 2^{-3} \approx 0{,}88 \approx 88\%
    \end{align*}
}
\solutionspace{150pt}

\tasknumber{2}%
\task{%
    Сколько процентов ядер радиоактивного железа $\ce{^{59}Fe}$
    останется через $182{,}4\,\text{суток}$, если период его полураспада составляет $45{,}6\,\text{суток}$?
}
\answer{%
    \begin{align*}
    N &= N_0 \cdot 2^{-\frac t{T_{1/2}}}
        = 2^{-\frac{182{,}4\,\text{суток}}{45{,}6\,\text{суток}}}
        \approx 0{,}0625 \approx 6{,}25\%
    \end{align*}
}
\solutionspace{150pt}

\tasknumber{3}%
\task{%
    За $4\,\text{суток}$ от начального количества ядер радиоизотопа осталась одна шестнадцатая.
    Каков период полураспада этого изотопа (ответ приведите в сутках)?
    Какая ещё доля (также от начального количества) распадётся, если подождать ещё столько же?
}
\answer{%
    \begin{align*}
            N &= N_0 \cdot 2^{-\frac t{T_{1/2}}}
            \implies \frac N{N_0} = 2^{-\frac t{T_{1/2}}}
            \implies \frac 1{16} = 2^{-\frac {4\,\text{суток}}{T_{1/2}}}
            \implies 4 = \frac {4\,\text{суток}}{T_{1/2}}
            \implies T_{1/2} = \frac {4\,\text{суток}}4 \approx 1{,}0\,\text{суток}.
         \\
            \delta &= \frac{N(t)}{N_0} - \frac{N(2t)}{N_0}
            = 2^{-\frac t{T_{1/2}}} - 2^{-\frac {2t}{T_{1/2}}}
            = 2^{-\frac t{T_{1/2}}}\cbr{1 - 2^{-\frac t{T_{1/2}}}}
            = \frac 1{16} \cdot \cbr{1-\frac 1{16}} \approx 0{,}059
    \end{align*}
}
\solutionspace{150pt}

\tasknumber{4}%
\task{%
    Энергия связи ядра лития \ce{^{7}_{3}Li} равна $39{,}2\,\text{МэВ}$.
    Найти дефект массы этого ядра.
    Ответ выразите в а.е.м.
    и кг.
    Скорость света $c = 2{,}998 \cdot 10^{8}\,\frac{\text{м}}{\text{с}}$, элементарный заряд $e = 1{,}6 \cdot 10^{-19}\,\text{Кл}$.
}
\answer{%
    \begin{align*}
    E_\text{св.} &= \Delta m c^2 \implies \\
    \implies
            \Delta m &= \frac {E_\text{св.}}{c^2} = \frac{39{,}2\,\text{МэВ}}{\sqr{2{,}998 \cdot 10^{8}\,\frac{\text{м}}{\text{с}}}}
            = \frac{39{,}2 \cdot 10^6 \cdot 1{,}6 \cdot 10^{-19}\,\text{Дж}}{\sqr{2{,}998 \cdot 10^{8}\,\frac{\text{м}}{\text{с}}}}
            \approx 6{,}98 \cdot 10^{-29}\,\text{кг} \approx 0{,}0420\,\text{а.е.м.}
    \end{align*}
}

\variantsplitter

\addpersonalvariant{Константин Мельник}

\tasknumber{1}%
\task{%
    Какая доля (от начального количества) радиоактивных ядер распадётся через время,
    равное четырём периодам полураспада? Ответ выразить в процентах.
}
\answer{%
    \begin{align*}
    N &= N_0 \cdot 2^{- \frac t{T_{1/2}}} \implies
        \frac N{N_0} = 2^{- \frac t{T_{1/2}}}
        = 2^{-4} \approx 0{,}06 \approx 6\% \\
    N_\text{расп.} &= N_0 - N = N_0 - N_0 \cdot 2^{-\frac t{T_{1/2}}}
        = N_0\cbr{1 - 2^{-\frac t{T_{1/2}}}} \implies
        \frac{N_\text{расп.}}{N_0} = 1 - 2^{-\frac t{T_{1/2}}}
        = 1 - 2^{-4} \approx 0{,}94 \approx 94\%
    \end{align*}
}
\solutionspace{150pt}

\tasknumber{2}%
\task{%
    Сколько процентов ядер радиоактивного железа $\ce{^{59}Fe}$
    останется через $136{,}8\,\text{суток}$, если период его полураспада составляет $45{,}6\,\text{суток}$?
}
\answer{%
    \begin{align*}
    N &= N_0 \cdot 2^{-\frac t{T_{1/2}}}
        = 2^{-\frac{136{,}8\,\text{суток}}{45{,}6\,\text{суток}}}
        \approx 0{,}125 \approx 12{,}5\%
    \end{align*}
}
\solutionspace{150pt}

\tasknumber{3}%
\task{%
    За $4\,\text{суток}$ от начального количества ядер радиоизотопа осталась одна шестнадцатая.
    Каков период полураспада этого изотопа (ответ приведите в сутках)?
    Какая ещё доля (также от начального количества) распадётся, если подождать ещё столько же?
}
\answer{%
    \begin{align*}
            N &= N_0 \cdot 2^{-\frac t{T_{1/2}}}
            \implies \frac N{N_0} = 2^{-\frac t{T_{1/2}}}
            \implies \frac 1{16} = 2^{-\frac {4\,\text{суток}}{T_{1/2}}}
            \implies 4 = \frac {4\,\text{суток}}{T_{1/2}}
            \implies T_{1/2} = \frac {4\,\text{суток}}4 \approx 1{,}0\,\text{суток}.
         \\
            \delta &= \frac{N(t)}{N_0} - \frac{N(2t)}{N_0}
            = 2^{-\frac t{T_{1/2}}} - 2^{-\frac {2t}{T_{1/2}}}
            = 2^{-\frac t{T_{1/2}}}\cbr{1 - 2^{-\frac t{T_{1/2}}}}
            = \frac 1{16} \cdot \cbr{1-\frac 1{16}} \approx 0{,}059
    \end{align*}
}
\solutionspace{150pt}

\tasknumber{4}%
\task{%
    Энергия связи ядра гелия \ce{^{3}_{2}He} равна $7{,}72\,\text{МэВ}$.
    Найти дефект массы этого ядра.
    Ответ выразите в а.е.м.
    и кг.
    Скорость света $c = 2{,}998 \cdot 10^{8}\,\frac{\text{м}}{\text{с}}$, элементарный заряд $e = 1{,}6 \cdot 10^{-19}\,\text{Кл}$.
}
\answer{%
    \begin{align*}
    E_\text{св.} &= \Delta m c^2 \implies \\
    \implies
            \Delta m &= \frac {E_\text{св.}}{c^2} = \frac{7{,}72\,\text{МэВ}}{\sqr{2{,}998 \cdot 10^{8}\,\frac{\text{м}}{\text{с}}}}
            = \frac{7{,}72 \cdot 10^6 \cdot 1{,}6 \cdot 10^{-19}\,\text{Дж}}{\sqr{2{,}998 \cdot 10^{8}\,\frac{\text{м}}{\text{с}}}}
            \approx 1{,}374 \cdot 10^{-29}\,\text{кг} \approx 0{,}00828\,\text{а.е.м.}
    \end{align*}
}

\variantsplitter

\addpersonalvariant{Степан Небоваренков}

\tasknumber{1}%
\task{%
    Какая доля (от начального количества) радиоактивных ядер распадётся через время,
    равное трём периодам полураспада? Ответ выразить в процентах.
}
\answer{%
    \begin{align*}
    N &= N_0 \cdot 2^{- \frac t{T_{1/2}}} \implies
        \frac N{N_0} = 2^{- \frac t{T_{1/2}}}
        = 2^{-3} \approx 0{,}12 \approx 12\% \\
    N_\text{расп.} &= N_0 - N = N_0 - N_0 \cdot 2^{-\frac t{T_{1/2}}}
        = N_0\cbr{1 - 2^{-\frac t{T_{1/2}}}} \implies
        \frac{N_\text{расп.}}{N_0} = 1 - 2^{-\frac t{T_{1/2}}}
        = 1 - 2^{-3} \approx 0{,}88 \approx 88\%
    \end{align*}
}
\solutionspace{150pt}

\tasknumber{2}%
\task{%
    Сколько процентов ядер радиоактивного железа $\ce{^{59}Fe}$
    останется через $91{,}2\,\text{суток}$, если период его полураспада составляет $45{,}6\,\text{суток}$?
}
\answer{%
    \begin{align*}
    N &= N_0 \cdot 2^{-\frac t{T_{1/2}}}
        = 2^{-\frac{91{,}2\,\text{суток}}{45{,}6\,\text{суток}}}
        \approx 0{,}25 \approx 25{,}0\%
    \end{align*}
}
\solutionspace{150pt}

\tasknumber{3}%
\task{%
    За $2\,\text{суток}$ от начального количества ядер радиоизотопа осталась четверть.
    Каков период полураспада этого изотопа (ответ приведите в сутках)?
    Какая ещё доля (также от начального количества) распадётся, если подождать ещё столько же?
}
\answer{%
    \begin{align*}
            N &= N_0 \cdot 2^{-\frac t{T_{1/2}}}
            \implies \frac N{N_0} = 2^{-\frac t{T_{1/2}}}
            \implies \frac 1{4} = 2^{-\frac {2\,\text{суток}}{T_{1/2}}}
            \implies 2 = \frac {2\,\text{суток}}{T_{1/2}}
            \implies T_{1/2} = \frac {2\,\text{суток}}2 \approx 1{,}0\,\text{суток}.
         \\
            \delta &= \frac{N(t)}{N_0} - \frac{N(2t)}{N_0}
            = 2^{-\frac t{T_{1/2}}} - 2^{-\frac {2t}{T_{1/2}}}
            = 2^{-\frac t{T_{1/2}}}\cbr{1 - 2^{-\frac t{T_{1/2}}}}
            = \frac 1{4} \cdot \cbr{1-\frac 1{4}} \approx 0{,}188
    \end{align*}
}
\solutionspace{150pt}

\tasknumber{4}%
\task{%
    Энергия связи ядра азота \ce{^{14}_{7}N} равна $104{,}7\,\text{МэВ}$.
    Найти дефект массы этого ядра.
    Ответ выразите в а.е.м.
    и кг.
    Скорость света $c = 2{,}998 \cdot 10^{8}\,\frac{\text{м}}{\text{с}}$, элементарный заряд $e = 1{,}6 \cdot 10^{-19}\,\text{Кл}$.
}
\answer{%
    \begin{align*}
    E_\text{св.} &= \Delta m c^2 \implies \\
    \implies
            \Delta m &= \frac {E_\text{св.}}{c^2} = \frac{104{,}7\,\text{МэВ}}{\sqr{2{,}998 \cdot 10^{8}\,\frac{\text{м}}{\text{с}}}}
            = \frac{104{,}7 \cdot 10^6 \cdot 1{,}6 \cdot 10^{-19}\,\text{Дж}}{\sqr{2{,}998 \cdot 10^{8}\,\frac{\text{м}}{\text{с}}}}
            \approx 18{,}64 \cdot 10^{-29}\,\text{кг} \approx 0{,}1122\,\text{а.е.м.}
    \end{align*}
}

\variantsplitter

\addpersonalvariant{Матвей Неретин}

\tasknumber{1}%
\task{%
    Какая доля (от начального количества) радиоактивных ядер распадётся через время,
    равное двум периодам полураспада? Ответ выразить в процентах.
}
\answer{%
    \begin{align*}
    N &= N_0 \cdot 2^{- \frac t{T_{1/2}}} \implies
        \frac N{N_0} = 2^{- \frac t{T_{1/2}}}
        = 2^{-2} \approx 0{,}25 \approx 25\% \\
    N_\text{расп.} &= N_0 - N = N_0 - N_0 \cdot 2^{-\frac t{T_{1/2}}}
        = N_0\cbr{1 - 2^{-\frac t{T_{1/2}}}} \implies
        \frac{N_\text{расп.}}{N_0} = 1 - 2^{-\frac t{T_{1/2}}}
        = 1 - 2^{-2} \approx 0{,}75 \approx 75\%
    \end{align*}
}
\solutionspace{150pt}

\tasknumber{2}%
\task{%
    Сколько процентов ядер радиоактивного железа $\ce{^{59}Fe}$
    останется через $182{,}4\,\text{суток}$, если период его полураспада составляет $45{,}6\,\text{суток}$?
}
\answer{%
    \begin{align*}
    N &= N_0 \cdot 2^{-\frac t{T_{1/2}}}
        = 2^{-\frac{182{,}4\,\text{суток}}{45{,}6\,\text{суток}}}
        \approx 0{,}0625 \approx 6{,}25\%
    \end{align*}
}
\solutionspace{150pt}

\tasknumber{3}%
\task{%
    За $5\,\text{суток}$ от начального количества ядер радиоизотопа осталась половина.
    Каков период полураспада этого изотопа (ответ приведите в сутках)?
    Какая ещё доля (также от начального количества) распадётся, если подождать ещё столько же?
}
\answer{%
    \begin{align*}
            N &= N_0 \cdot 2^{-\frac t{T_{1/2}}}
            \implies \frac N{N_0} = 2^{-\frac t{T_{1/2}}}
            \implies \frac 1{2} = 2^{-\frac {5\,\text{суток}}{T_{1/2}}}
            \implies 1 = \frac {5\,\text{суток}}{T_{1/2}}
            \implies T_{1/2} = \frac {5\,\text{суток}}1 \approx 5{,}0\,\text{суток}.
         \\
            \delta &= \frac{N(t)}{N_0} - \frac{N(2t)}{N_0}
            = 2^{-\frac t{T_{1/2}}} - 2^{-\frac {2t}{T_{1/2}}}
            = 2^{-\frac t{T_{1/2}}}\cbr{1 - 2^{-\frac t{T_{1/2}}}}
            = \frac 1{2} \cdot \cbr{1-\frac 1{2}} \approx 0{,}250
    \end{align*}
}
\solutionspace{150pt}

\tasknumber{4}%
\task{%
    Энергия связи ядра дейтерия \ce{^{2}_{1}H} (D) равна $2{,}22\,\text{МэВ}$.
    Найти дефект массы этого ядра.
    Ответ выразите в а.е.м.
    и кг.
    Скорость света $c = 2{,}998 \cdot 10^{8}\,\frac{\text{м}}{\text{с}}$, элементарный заряд $e = 1{,}6 \cdot 10^{-19}\,\text{Кл}$.
}
\answer{%
    \begin{align*}
    E_\text{св.} &= \Delta m c^2 \implies \\
    \implies
            \Delta m &= \frac {E_\text{св.}}{c^2} = \frac{2{,}22\,\text{МэВ}}{\sqr{2{,}998 \cdot 10^{8}\,\frac{\text{м}}{\text{с}}}}
            = \frac{2{,}22 \cdot 10^6 \cdot 1{,}6 \cdot 10^{-19}\,\text{Дж}}{\sqr{2{,}998 \cdot 10^{8}\,\frac{\text{м}}{\text{с}}}}
            \approx 0{,}395 \cdot 10^{-29}\,\text{кг} \approx 0{,}00238\,\text{а.е.м.}
    \end{align*}
}

\variantsplitter

\addpersonalvariant{Мария Никонова}

\tasknumber{1}%
\task{%
    Какая доля (от начального количества) радиоактивных ядер распадётся через время,
    равное трём периодам полураспада? Ответ выразить в процентах.
}
\answer{%
    \begin{align*}
    N &= N_0 \cdot 2^{- \frac t{T_{1/2}}} \implies
        \frac N{N_0} = 2^{- \frac t{T_{1/2}}}
        = 2^{-3} \approx 0{,}12 \approx 12\% \\
    N_\text{расп.} &= N_0 - N = N_0 - N_0 \cdot 2^{-\frac t{T_{1/2}}}
        = N_0\cbr{1 - 2^{-\frac t{T_{1/2}}}} \implies
        \frac{N_\text{расп.}}{N_0} = 1 - 2^{-\frac t{T_{1/2}}}
        = 1 - 2^{-3} \approx 0{,}88 \approx 88\%
    \end{align*}
}
\solutionspace{150pt}

\tasknumber{2}%
\task{%
    Сколько процентов ядер радиоактивного железа $\ce{^{59}Fe}$
    останется через $91{,}2\,\text{суток}$, если период его полураспада составляет $45{,}6\,\text{суток}$?
}
\answer{%
    \begin{align*}
    N &= N_0 \cdot 2^{-\frac t{T_{1/2}}}
        = 2^{-\frac{91{,}2\,\text{суток}}{45{,}6\,\text{суток}}}
        \approx 0{,}25 \approx 25{,}0\%
    \end{align*}
}
\solutionspace{150pt}

\tasknumber{3}%
\task{%
    За $3\,\text{суток}$ от начального количества ядер радиоизотопа осталась одна шестнадцатая.
    Каков период полураспада этого изотопа (ответ приведите в сутках)?
    Какая ещё доля (также от начального количества) распадётся, если подождать ещё столько же?
}
\answer{%
    \begin{align*}
            N &= N_0 \cdot 2^{-\frac t{T_{1/2}}}
            \implies \frac N{N_0} = 2^{-\frac t{T_{1/2}}}
            \implies \frac 1{16} = 2^{-\frac {3\,\text{суток}}{T_{1/2}}}
            \implies 4 = \frac {3\,\text{суток}}{T_{1/2}}
            \implies T_{1/2} = \frac {3\,\text{суток}}4 \approx 0{,}8\,\text{суток}.
         \\
            \delta &= \frac{N(t)}{N_0} - \frac{N(2t)}{N_0}
            = 2^{-\frac t{T_{1/2}}} - 2^{-\frac {2t}{T_{1/2}}}
            = 2^{-\frac t{T_{1/2}}}\cbr{1 - 2^{-\frac t{T_{1/2}}}}
            = \frac 1{16} \cdot \cbr{1-\frac 1{16}} \approx 0{,}059
    \end{align*}
}
\solutionspace{150pt}

\tasknumber{4}%
\task{%
    Энергия связи ядра лития \ce{^{7}_{3}Li} равна $39{,}2\,\text{МэВ}$.
    Найти дефект массы этого ядра.
    Ответ выразите в а.е.м.
    и кг.
    Скорость света $c = 2{,}998 \cdot 10^{8}\,\frac{\text{м}}{\text{с}}$, элементарный заряд $e = 1{,}6 \cdot 10^{-19}\,\text{Кл}$.
}
\answer{%
    \begin{align*}
    E_\text{св.} &= \Delta m c^2 \implies \\
    \implies
            \Delta m &= \frac {E_\text{св.}}{c^2} = \frac{39{,}2\,\text{МэВ}}{\sqr{2{,}998 \cdot 10^{8}\,\frac{\text{м}}{\text{с}}}}
            = \frac{39{,}2 \cdot 10^6 \cdot 1{,}6 \cdot 10^{-19}\,\text{Дж}}{\sqr{2{,}998 \cdot 10^{8}\,\frac{\text{м}}{\text{с}}}}
            \approx 6{,}98 \cdot 10^{-29}\,\text{кг} \approx 0{,}0420\,\text{а.е.м.}
    \end{align*}
}

\variantsplitter

\addpersonalvariant{Даниил Палаткин}

\tasknumber{1}%
\task{%
    Какая доля (от начального количества) радиоактивных ядер останется через время,
    равное двум периодам полураспада? Ответ выразить в процентах.
}
\answer{%
    \begin{align*}
    N &= N_0 \cdot 2^{- \frac t{T_{1/2}}} \implies
        \frac N{N_0} = 2^{- \frac t{T_{1/2}}}
        = 2^{-2} \approx 0{,}25 \approx 25\% \\
    N_\text{расп.} &= N_0 - N = N_0 - N_0 \cdot 2^{-\frac t{T_{1/2}}}
        = N_0\cbr{1 - 2^{-\frac t{T_{1/2}}}} \implies
        \frac{N_\text{расп.}}{N_0} = 1 - 2^{-\frac t{T_{1/2}}}
        = 1 - 2^{-2} \approx 0{,}75 \approx 75\%
    \end{align*}
}
\solutionspace{150pt}

\tasknumber{2}%
\task{%
    Сколько процентов ядер радиоактивного железа $\ce{^{59}Fe}$
    останется через $136{,}8\,\text{суток}$, если период его полураспада составляет $45{,}6\,\text{суток}$?
}
\answer{%
    \begin{align*}
    N &= N_0 \cdot 2^{-\frac t{T_{1/2}}}
        = 2^{-\frac{136{,}8\,\text{суток}}{45{,}6\,\text{суток}}}
        \approx 0{,}125 \approx 12{,}5\%
    \end{align*}
}
\solutionspace{150pt}

\tasknumber{3}%
\task{%
    За $5\,\text{суток}$ от начального количества ядер радиоизотопа осталась одна восьмая.
    Каков период полураспада этого изотопа (ответ приведите в сутках)?
    Какая ещё доля (также от начального количества) распадётся, если подождать ещё столько же?
}
\answer{%
    \begin{align*}
            N &= N_0 \cdot 2^{-\frac t{T_{1/2}}}
            \implies \frac N{N_0} = 2^{-\frac t{T_{1/2}}}
            \implies \frac 1{8} = 2^{-\frac {5\,\text{суток}}{T_{1/2}}}
            \implies 3 = \frac {5\,\text{суток}}{T_{1/2}}
            \implies T_{1/2} = \frac {5\,\text{суток}}3 \approx 1{,}7\,\text{суток}.
         \\
            \delta &= \frac{N(t)}{N_0} - \frac{N(2t)}{N_0}
            = 2^{-\frac t{T_{1/2}}} - 2^{-\frac {2t}{T_{1/2}}}
            = 2^{-\frac t{T_{1/2}}}\cbr{1 - 2^{-\frac t{T_{1/2}}}}
            = \frac 1{8} \cdot \cbr{1-\frac 1{8}} \approx 0{,}109
    \end{align*}
}
\solutionspace{150pt}

\tasknumber{4}%
\task{%
    Энергия связи ядра бора \ce{^{11}_{5}B} равна $76{,}2\,\text{МэВ}$.
    Найти дефект массы этого ядра.
    Ответ выразите в а.е.м.
    и кг.
    Скорость света $c = 2{,}998 \cdot 10^{8}\,\frac{\text{м}}{\text{с}}$, элементарный заряд $e = 1{,}6 \cdot 10^{-19}\,\text{Кл}$.
}
\answer{%
    \begin{align*}
    E_\text{св.} &= \Delta m c^2 \implies \\
    \implies
            \Delta m &= \frac {E_\text{св.}}{c^2} = \frac{76{,}2\,\text{МэВ}}{\sqr{2{,}998 \cdot 10^{8}\,\frac{\text{м}}{\text{с}}}}
            = \frac{76{,}2 \cdot 10^6 \cdot 1{,}6 \cdot 10^{-19}\,\text{Дж}}{\sqr{2{,}998 \cdot 10^{8}\,\frac{\text{м}}{\text{с}}}}
            \approx 13{,}56 \cdot 10^{-29}\,\text{кг} \approx 0{,}0817\,\text{а.е.м.}
    \end{align*}
}

\variantsplitter

\addpersonalvariant{Станислав Пикун}

\tasknumber{1}%
\task{%
    Какая доля (от начального количества) радиоактивных ядер останется через время,
    равное трём периодам полураспада? Ответ выразить в процентах.
}
\answer{%
    \begin{align*}
    N &= N_0 \cdot 2^{- \frac t{T_{1/2}}} \implies
        \frac N{N_0} = 2^{- \frac t{T_{1/2}}}
        = 2^{-3} \approx 0{,}12 \approx 12\% \\
    N_\text{расп.} &= N_0 - N = N_0 - N_0 \cdot 2^{-\frac t{T_{1/2}}}
        = N_0\cbr{1 - 2^{-\frac t{T_{1/2}}}} \implies
        \frac{N_\text{расп.}}{N_0} = 1 - 2^{-\frac t{T_{1/2}}}
        = 1 - 2^{-3} \approx 0{,}88 \approx 88\%
    \end{align*}
}
\solutionspace{150pt}

\tasknumber{2}%
\task{%
    Сколько процентов ядер радиоактивного железа $\ce{^{59}Fe}$
    останется через $136{,}8\,\text{суток}$, если период его полураспада составляет $45{,}6\,\text{суток}$?
}
\answer{%
    \begin{align*}
    N &= N_0 \cdot 2^{-\frac t{T_{1/2}}}
        = 2^{-\frac{136{,}8\,\text{суток}}{45{,}6\,\text{суток}}}
        \approx 0{,}125 \approx 12{,}5\%
    \end{align*}
}
\solutionspace{150pt}

\tasknumber{3}%
\task{%
    За $5\,\text{суток}$ от начального количества ядер радиоизотопа осталась одна шестнадцатая.
    Каков период полураспада этого изотопа (ответ приведите в сутках)?
    Какая ещё доля (также от начального количества) распадётся, если подождать ещё столько же?
}
\answer{%
    \begin{align*}
            N &= N_0 \cdot 2^{-\frac t{T_{1/2}}}
            \implies \frac N{N_0} = 2^{-\frac t{T_{1/2}}}
            \implies \frac 1{16} = 2^{-\frac {5\,\text{суток}}{T_{1/2}}}
            \implies 4 = \frac {5\,\text{суток}}{T_{1/2}}
            \implies T_{1/2} = \frac {5\,\text{суток}}4 \approx 1{,}2\,\text{суток}.
         \\
            \delta &= \frac{N(t)}{N_0} - \frac{N(2t)}{N_0}
            = 2^{-\frac t{T_{1/2}}} - 2^{-\frac {2t}{T_{1/2}}}
            = 2^{-\frac t{T_{1/2}}}\cbr{1 - 2^{-\frac t{T_{1/2}}}}
            = \frac 1{16} \cdot \cbr{1-\frac 1{16}} \approx 0{,}059
    \end{align*}
}
\solutionspace{150pt}

\tasknumber{4}%
\task{%
    Энергия связи ядра кислорода \ce{^{16}_{8}O} равна $127{,}6\,\text{МэВ}$.
    Найти дефект массы этого ядра.
    Ответ выразите в а.е.м.
    и кг.
    Скорость света $c = 2{,}998 \cdot 10^{8}\,\frac{\text{м}}{\text{с}}$, элементарный заряд $e = 1{,}6 \cdot 10^{-19}\,\text{Кл}$.
}
\answer{%
    \begin{align*}
    E_\text{св.} &= \Delta m c^2 \implies \\
    \implies
            \Delta m &= \frac {E_\text{св.}}{c^2} = \frac{127{,}6\,\text{МэВ}}{\sqr{2{,}998 \cdot 10^{8}\,\frac{\text{м}}{\text{с}}}}
            = \frac{127{,}6 \cdot 10^6 \cdot 1{,}6 \cdot 10^{-19}\,\text{Дж}}{\sqr{2{,}998 \cdot 10^{8}\,\frac{\text{м}}{\text{с}}}}
            \approx 22{,}7 \cdot 10^{-29}\,\text{кг} \approx 0{,}1368\,\text{а.е.м.}
    \end{align*}
}

\variantsplitter

\addpersonalvariant{Илья Пичугин}

\tasknumber{1}%
\task{%
    Какая доля (от начального количества) радиоактивных ядер останется через время,
    равное четырём периодам полураспада? Ответ выразить в процентах.
}
\answer{%
    \begin{align*}
    N &= N_0 \cdot 2^{- \frac t{T_{1/2}}} \implies
        \frac N{N_0} = 2^{- \frac t{T_{1/2}}}
        = 2^{-4} \approx 0{,}06 \approx 6\% \\
    N_\text{расп.} &= N_0 - N = N_0 - N_0 \cdot 2^{-\frac t{T_{1/2}}}
        = N_0\cbr{1 - 2^{-\frac t{T_{1/2}}}} \implies
        \frac{N_\text{расп.}}{N_0} = 1 - 2^{-\frac t{T_{1/2}}}
        = 1 - 2^{-4} \approx 0{,}94 \approx 94\%
    \end{align*}
}
\solutionspace{150pt}

\tasknumber{2}%
\task{%
    Сколько процентов ядер радиоактивного железа $\ce{^{59}Fe}$
    останется через $91{,}2\,\text{суток}$, если период его полураспада составляет $45{,}6\,\text{суток}$?
}
\answer{%
    \begin{align*}
    N &= N_0 \cdot 2^{-\frac t{T_{1/2}}}
        = 2^{-\frac{91{,}2\,\text{суток}}{45{,}6\,\text{суток}}}
        \approx 0{,}25 \approx 25{,}0\%
    \end{align*}
}
\solutionspace{150pt}

\tasknumber{3}%
\task{%
    За $3\,\text{суток}$ от начального количества ядер радиоизотопа осталась одна восьмая.
    Каков период полураспада этого изотопа (ответ приведите в сутках)?
    Какая ещё доля (также от начального количества) распадётся, если подождать ещё столько же?
}
\answer{%
    \begin{align*}
            N &= N_0 \cdot 2^{-\frac t{T_{1/2}}}
            \implies \frac N{N_0} = 2^{-\frac t{T_{1/2}}}
            \implies \frac 1{8} = 2^{-\frac {3\,\text{суток}}{T_{1/2}}}
            \implies 3 = \frac {3\,\text{суток}}{T_{1/2}}
            \implies T_{1/2} = \frac {3\,\text{суток}}3 \approx 1{,}0\,\text{суток}.
         \\
            \delta &= \frac{N(t)}{N_0} - \frac{N(2t)}{N_0}
            = 2^{-\frac t{T_{1/2}}} - 2^{-\frac {2t}{T_{1/2}}}
            = 2^{-\frac t{T_{1/2}}}\cbr{1 - 2^{-\frac t{T_{1/2}}}}
            = \frac 1{8} \cdot \cbr{1-\frac 1{8}} \approx 0{,}109
    \end{align*}
}
\solutionspace{150pt}

\tasknumber{4}%
\task{%
    Энергия связи ядра лития \ce{^{6}_{3}Li} равна $31{,}99\,\text{МэВ}$.
    Найти дефект массы этого ядра.
    Ответ выразите в а.е.м.
    и кг.
    Скорость света $c = 2{,}998 \cdot 10^{8}\,\frac{\text{м}}{\text{с}}$, элементарный заряд $e = 1{,}6 \cdot 10^{-19}\,\text{Кл}$.
}
\answer{%
    \begin{align*}
    E_\text{св.} &= \Delta m c^2 \implies \\
    \implies
            \Delta m &= \frac {E_\text{св.}}{c^2} = \frac{31{,}99\,\text{МэВ}}{\sqr{2{,}998 \cdot 10^{8}\,\frac{\text{м}}{\text{с}}}}
            = \frac{31{,}99 \cdot 10^6 \cdot 1{,}6 \cdot 10^{-19}\,\text{Дж}}{\sqr{2{,}998 \cdot 10^{8}\,\frac{\text{м}}{\text{с}}}}
            \approx 5{,}69 \cdot 10^{-29}\,\text{кг} \approx 0{,}0343\,\text{а.е.м.}
    \end{align*}
}

\variantsplitter

\addpersonalvariant{Кирилл Севрюгин}

\tasknumber{1}%
\task{%
    Какая доля (от начального количества) радиоактивных ядер останется через время,
    равное двум периодам полураспада? Ответ выразить в процентах.
}
\answer{%
    \begin{align*}
    N &= N_0 \cdot 2^{- \frac t{T_{1/2}}} \implies
        \frac N{N_0} = 2^{- \frac t{T_{1/2}}}
        = 2^{-2} \approx 0{,}25 \approx 25\% \\
    N_\text{расп.} &= N_0 - N = N_0 - N_0 \cdot 2^{-\frac t{T_{1/2}}}
        = N_0\cbr{1 - 2^{-\frac t{T_{1/2}}}} \implies
        \frac{N_\text{расп.}}{N_0} = 1 - 2^{-\frac t{T_{1/2}}}
        = 1 - 2^{-2} \approx 0{,}75 \approx 75\%
    \end{align*}
}
\solutionspace{150pt}

\tasknumber{2}%
\task{%
    Сколько процентов ядер радиоактивного железа $\ce{^{59}Fe}$
    останется через $136{,}8\,\text{суток}$, если период его полураспада составляет $45{,}6\,\text{суток}$?
}
\answer{%
    \begin{align*}
    N &= N_0 \cdot 2^{-\frac t{T_{1/2}}}
        = 2^{-\frac{136{,}8\,\text{суток}}{45{,}6\,\text{суток}}}
        \approx 0{,}125 \approx 12{,}5\%
    \end{align*}
}
\solutionspace{150pt}

\tasknumber{3}%
\task{%
    За $2\,\text{суток}$ от начального количества ядер радиоизотопа осталась половина.
    Каков период полураспада этого изотопа (ответ приведите в сутках)?
    Какая ещё доля (также от начального количества) распадётся, если подождать ещё столько же?
}
\answer{%
    \begin{align*}
            N &= N_0 \cdot 2^{-\frac t{T_{1/2}}}
            \implies \frac N{N_0} = 2^{-\frac t{T_{1/2}}}
            \implies \frac 1{2} = 2^{-\frac {2\,\text{суток}}{T_{1/2}}}
            \implies 1 = \frac {2\,\text{суток}}{T_{1/2}}
            \implies T_{1/2} = \frac {2\,\text{суток}}1 \approx 2{,}0\,\text{суток}.
         \\
            \delta &= \frac{N(t)}{N_0} - \frac{N(2t)}{N_0}
            = 2^{-\frac t{T_{1/2}}} - 2^{-\frac {2t}{T_{1/2}}}
            = 2^{-\frac t{T_{1/2}}}\cbr{1 - 2^{-\frac t{T_{1/2}}}}
            = \frac 1{2} \cdot \cbr{1-\frac 1{2}} \approx 0{,}250
    \end{align*}
}
\solutionspace{150pt}

\tasknumber{4}%
\task{%
    Энергия связи ядра углерода \ce{^{12}_{6}C} равна $92{,}2\,\text{МэВ}$.
    Найти дефект массы этого ядра.
    Ответ выразите в а.е.м.
    и кг.
    Скорость света $c = 2{,}998 \cdot 10^{8}\,\frac{\text{м}}{\text{с}}$, элементарный заряд $e = 1{,}6 \cdot 10^{-19}\,\text{Кл}$.
}
\answer{%
    \begin{align*}
    E_\text{св.} &= \Delta m c^2 \implies \\
    \implies
            \Delta m &= \frac {E_\text{св.}}{c^2} = \frac{92{,}2\,\text{МэВ}}{\sqr{2{,}998 \cdot 10^{8}\,\frac{\text{м}}{\text{с}}}}
            = \frac{92{,}2 \cdot 10^6 \cdot 1{,}6 \cdot 10^{-19}\,\text{Дж}}{\sqr{2{,}998 \cdot 10^{8}\,\frac{\text{м}}{\text{с}}}}
            \approx 16{,}41 \cdot 10^{-29}\,\text{кг} \approx 0{,}0988\,\text{а.е.м.}
    \end{align*}
}

\variantsplitter

\addpersonalvariant{Илья Стратонников}

\tasknumber{1}%
\task{%
    Какая доля (от начального количества) радиоактивных ядер распадётся через время,
    равное трём периодам полураспада? Ответ выразить в процентах.
}
\answer{%
    \begin{align*}
    N &= N_0 \cdot 2^{- \frac t{T_{1/2}}} \implies
        \frac N{N_0} = 2^{- \frac t{T_{1/2}}}
        = 2^{-3} \approx 0{,}12 \approx 12\% \\
    N_\text{расп.} &= N_0 - N = N_0 - N_0 \cdot 2^{-\frac t{T_{1/2}}}
        = N_0\cbr{1 - 2^{-\frac t{T_{1/2}}}} \implies
        \frac{N_\text{расп.}}{N_0} = 1 - 2^{-\frac t{T_{1/2}}}
        = 1 - 2^{-3} \approx 0{,}88 \approx 88\%
    \end{align*}
}
\solutionspace{150pt}

\tasknumber{2}%
\task{%
    Сколько процентов ядер радиоактивного железа $\ce{^{59}Fe}$
    останется через $91{,}2\,\text{суток}$, если период его полураспада составляет $45{,}6\,\text{суток}$?
}
\answer{%
    \begin{align*}
    N &= N_0 \cdot 2^{-\frac t{T_{1/2}}}
        = 2^{-\frac{91{,}2\,\text{суток}}{45{,}6\,\text{суток}}}
        \approx 0{,}25 \approx 25{,}0\%
    \end{align*}
}
\solutionspace{150pt}

\tasknumber{3}%
\task{%
    За $3\,\text{суток}$ от начального количества ядер радиоизотопа осталась половина.
    Каков период полураспада этого изотопа (ответ приведите в сутках)?
    Какая ещё доля (также от начального количества) распадётся, если подождать ещё столько же?
}
\answer{%
    \begin{align*}
            N &= N_0 \cdot 2^{-\frac t{T_{1/2}}}
            \implies \frac N{N_0} = 2^{-\frac t{T_{1/2}}}
            \implies \frac 1{2} = 2^{-\frac {3\,\text{суток}}{T_{1/2}}}
            \implies 1 = \frac {3\,\text{суток}}{T_{1/2}}
            \implies T_{1/2} = \frac {3\,\text{суток}}1 \approx 3{,}0\,\text{суток}.
         \\
            \delta &= \frac{N(t)}{N_0} - \frac{N(2t)}{N_0}
            = 2^{-\frac t{T_{1/2}}} - 2^{-\frac {2t}{T_{1/2}}}
            = 2^{-\frac t{T_{1/2}}}\cbr{1 - 2^{-\frac t{T_{1/2}}}}
            = \frac 1{2} \cdot \cbr{1-\frac 1{2}} \approx 0{,}250
    \end{align*}
}
\solutionspace{150pt}

\tasknumber{4}%
\task{%
    Энергия связи ядра трития \ce{^{3}_{1}H} (T) равна $8{,}48\,\text{МэВ}$.
    Найти дефект массы этого ядра.
    Ответ выразите в а.е.м.
    и кг.
    Скорость света $c = 2{,}998 \cdot 10^{8}\,\frac{\text{м}}{\text{с}}$, элементарный заряд $e = 1{,}6 \cdot 10^{-19}\,\text{Кл}$.
}
\answer{%
    \begin{align*}
    E_\text{св.} &= \Delta m c^2 \implies \\
    \implies
            \Delta m &= \frac {E_\text{св.}}{c^2} = \frac{8{,}48\,\text{МэВ}}{\sqr{2{,}998 \cdot 10^{8}\,\frac{\text{м}}{\text{с}}}}
            = \frac{8{,}48 \cdot 10^6 \cdot 1{,}6 \cdot 10^{-19}\,\text{Дж}}{\sqr{2{,}998 \cdot 10^{8}\,\frac{\text{м}}{\text{с}}}}
            \approx 1{,}510 \cdot 10^{-29}\,\text{кг} \approx 0{,}00909\,\text{а.е.м.}
    \end{align*}
}

\variantsplitter

\addpersonalvariant{Иван Шустов}

\tasknumber{1}%
\task{%
    Какая доля (от начального количества) радиоактивных ядер останется через время,
    равное трём периодам полураспада? Ответ выразить в процентах.
}
\answer{%
    \begin{align*}
    N &= N_0 \cdot 2^{- \frac t{T_{1/2}}} \implies
        \frac N{N_0} = 2^{- \frac t{T_{1/2}}}
        = 2^{-3} \approx 0{,}12 \approx 12\% \\
    N_\text{расп.} &= N_0 - N = N_0 - N_0 \cdot 2^{-\frac t{T_{1/2}}}
        = N_0\cbr{1 - 2^{-\frac t{T_{1/2}}}} \implies
        \frac{N_\text{расп.}}{N_0} = 1 - 2^{-\frac t{T_{1/2}}}
        = 1 - 2^{-3} \approx 0{,}88 \approx 88\%
    \end{align*}
}
\solutionspace{150pt}

\tasknumber{2}%
\task{%
    Сколько процентов ядер радиоактивного железа $\ce{^{59}Fe}$
    останется через $136{,}8\,\text{суток}$, если период его полураспада составляет $45{,}6\,\text{суток}$?
}
\answer{%
    \begin{align*}
    N &= N_0 \cdot 2^{-\frac t{T_{1/2}}}
        = 2^{-\frac{136{,}8\,\text{суток}}{45{,}6\,\text{суток}}}
        \approx 0{,}125 \approx 12{,}5\%
    \end{align*}
}
\solutionspace{150pt}

\tasknumber{3}%
\task{%
    За $4\,\text{суток}$ от начального количества ядер радиоизотопа осталась половина.
    Каков период полураспада этого изотопа (ответ приведите в сутках)?
    Какая ещё доля (также от начального количества) распадётся, если подождать ещё столько же?
}
\answer{%
    \begin{align*}
            N &= N_0 \cdot 2^{-\frac t{T_{1/2}}}
            \implies \frac N{N_0} = 2^{-\frac t{T_{1/2}}}
            \implies \frac 1{2} = 2^{-\frac {4\,\text{суток}}{T_{1/2}}}
            \implies 1 = \frac {4\,\text{суток}}{T_{1/2}}
            \implies T_{1/2} = \frac {4\,\text{суток}}1 \approx 4{,}0\,\text{суток}.
         \\
            \delta &= \frac{N(t)}{N_0} - \frac{N(2t)}{N_0}
            = 2^{-\frac t{T_{1/2}}} - 2^{-\frac {2t}{T_{1/2}}}
            = 2^{-\frac t{T_{1/2}}}\cbr{1 - 2^{-\frac t{T_{1/2}}}}
            = \frac 1{2} \cdot \cbr{1-\frac 1{2}} \approx 0{,}250
    \end{align*}
}
\solutionspace{150pt}

\tasknumber{4}%
\task{%
    Энергия связи ядра трития \ce{^{3}_{1}H} (T) равна $8{,}48\,\text{МэВ}$.
    Найти дефект массы этого ядра.
    Ответ выразите в а.е.м.
    и кг.
    Скорость света $c = 2{,}998 \cdot 10^{8}\,\frac{\text{м}}{\text{с}}$, элементарный заряд $e = 1{,}6 \cdot 10^{-19}\,\text{Кл}$.
}
\answer{%
    \begin{align*}
    E_\text{св.} &= \Delta m c^2 \implies \\
    \implies
            \Delta m &= \frac {E_\text{св.}}{c^2} = \frac{8{,}48\,\text{МэВ}}{\sqr{2{,}998 \cdot 10^{8}\,\frac{\text{м}}{\text{с}}}}
            = \frac{8{,}48 \cdot 10^6 \cdot 1{,}6 \cdot 10^{-19}\,\text{Дж}}{\sqr{2{,}998 \cdot 10^{8}\,\frac{\text{м}}{\text{с}}}}
            \approx 1{,}510 \cdot 10^{-29}\,\text{кг} \approx 0{,}00909\,\text{а.е.м.}
    \end{align*}
}
% autogenerated
