\setdate{22~января~2021}
\setclass{10«АБ»}

\addpersonalvariant{Михаил Бурмистров}

\tasknumber{1}%
\task{%
    Молекулы газа в некотором сосуде движутся со средней скоростью $150\,\frac{\text{м}}{\text{с}}$.
    Определите, какое расстояние в среднем проходит одна из таких молекул за $2\,\text{час}$.
}
\answer{%
    $s = v t = 150\,\frac{\text{м}}{\text{с}} \cdot 2\,\text{час} = 1{,}08 \cdot 10^{6}\,\text{м}.$
}
\solutionspace{40pt}

\tasknumber{2}%
\task{%
    Напротив каждой физической величины укажите её обозначение и единицы измерения в СИ:
    \begin{enumerate}
        \item масса,
        \item плотность,
        \item молярная масса.
    \end{enumerate}
}

\tasknumber{3}%
\task{%
    Ответьте на вопросы и запишите формулы:
    \begin{enumerate}
        \item запишите 3 основных положения МКТ,
        \item cвязь количества вещества, массы тела и молярной массы.
    \end{enumerate}
}
\solutionspace{60pt}

\tasknumber{4}%
\task{%
    Определите молярную массу веществ (не табличное значение, а вычислением по таблице Менделеева):
    \begin{enumerate}
        \item гелий,
        \item азот,
        \item озон.
    \end{enumerate}
}
\solutionspace{30pt}

\tasknumber{5}%
\task{%
    Укажите, верны ли утверждения («да» или «нет» слева от каждого утверждения):
    \begin{enumerate}
        \item В твёрдом состоянии вещества связи между молекулами наиболее сильны (в сравнении с жидким и газообразным состояниями).
        \item Любая частица (например, картошечка в супе) находится в броуновском движении, однако наблюдать его технически возможно только для малых частиц.
        \item Сжимаемость газов объясняется проникновением атомов молекул друг в друга и уменьшением межатомного расстояния внутри молекул.
        \item Броуновское движение частиц пыльцы в жидкости — следствие взаимодействия этих частиц пыльцы между собой.
        \item Если в двух телах одинаковое число молекул, то их массы с большой точностью будут равны.
        \item Если в двух телах одинаковое число протонов и нейтронов (между телами), то и массы тел с большой точностью окажутся равны.
        \item При определении размеров молекул мы зачастую пренебрегаем их формой, не различая радиус и диаметр, а то и вовсе считая их форму кубической.
        \item Диффузия вызвана тепловым движением молекул и может наблюдаться в твердых, жидких и газообразных веществах.
    \end{enumerate}
}
\answer{%
    $
        \text{да, да, нет, нет, нет, да, да, да}
    $
}

\tasknumber{6}%
\task{%
    Какое количество вещества содержит тело, состоящее из $9 \cdot 10^{25}$ молекул?
}
\answer{%
    $\nu = \frac N{N_A} = \frac{9 \cdot 10^{25}}{6{,}02 \cdot 10^{23}\,\frac{1}{\text{моль}}} = 149{,}50\,\text{моль}.$
}
\solutionspace{40pt}

\tasknumber{7}%
\task{%
    Какова масса $10\,\text{моль}$ (\ce{C4H10}) бутана? Молярная масса бутана $58\,\frac{\text{г}}{\text{моль}}$.
}
\answer{%
    $m = \mu\nu = 58\,\frac{\text{г}}{\text{моль}} \cdot 10\,\text{моль} = 580\,\text{г}.$
}
\solutionspace{40pt}

\tasknumber{8}%
\task{%
    Сколько молекул содержится в $50\,\text{г}$ декана? Молярная масса декана (\ce{C10H22}) $142\,\frac{\text{г}}{\text{моль}}$.
}
\answer{%
    $N = N_A\nu = N_A\frac{m}{\mu} = 6{,}02 \cdot 10^{23}\,\frac{1}{\text{моль}} \cdot \frac{50\,\text{г}}{142\,\frac{\text{г}}{\text{моль}}} = 210 \cdot 10^{21}.$
}

\variantsplitter

\addpersonalvariant{Ирина Ан}

\tasknumber{1}%
\task{%
    Молекулы газа в некотором сосуде движутся со средней скоростью $300\,\frac{\text{м}}{\text{с}}$.
    Определите, какое расстояние в среднем проходит одна из таких молекул за $4\,\text{час}$.
}
\answer{%
    $s = v t = 300\,\frac{\text{м}}{\text{с}} \cdot 4\,\text{час} = 4{,}3 \cdot 10^{6}\,\text{м}.$
}
\solutionspace{40pt}

\tasknumber{2}%
\task{%
    Напротив каждой физической величины укажите её обозначение и единицы измерения в СИ:
    \begin{enumerate}
        \item объём,
        \item плотность,
        \item молярная масса.
    \end{enumerate}
}

\tasknumber{3}%
\task{%
    Ответьте на вопросы и запишите формулы:
    \begin{enumerate}
        \item сформилируйте, что такое броуновское движение,
        \item cвязь количества вещества, числа частиц и числа Авогадро.
    \end{enumerate}
}
\solutionspace{60pt}

\tasknumber{4}%
\task{%
    Определите молярную массу веществ (не табличное значение, а вычислением по таблице Менделеева):
    \begin{enumerate}
        \item неон,
        \item азот,
        \item вода.
    \end{enumerate}
}
\solutionspace{30pt}

\tasknumber{5}%
\task{%
    Укажите, верны ли утверждения («да» или «нет» слева от каждого утверждения):
    \begin{enumerate}
        \item В твёрдом состоянии вещества связи между молекулами наиболее сильны (в сравнении с жидким и газообразным состояниями).
        \item Любая частица (например, картошечка в супе) находится в броуновском движении, однако наблюдать его технически возможно только для малых частиц.
        \item Сжимаемость газов объясняется проникновением атомов молекул друг в друга и уменьшением межатомного расстояния внутри молекул.
        \item Броуновское движение частиц пыльцы в жидкости — следствие взаимодействия этих частиц пыльцы между собой.
        \item Если в двух телах одинаковое число молекул, то их массы с большой точностью будут равны.
        \item Если в двух телах одинаковое число протонов и нейтронов (между телами), то и массы тел с большой точностью окажутся равны.
        \item При определении размеров молекул мы зачастую пренебрегаем их формой, не различая радиус и диаметр, а то и вовсе считая их форму кубической.
        \item Диффузия вызвана тепловым движением молекул и может наблюдаться в твердых, жидких и газообразных веществах.
    \end{enumerate}
}
\answer{%
    $
        \text{да, да, нет, нет, нет, да, да, да}
    $
}

\tasknumber{6}%
\task{%
    Какое количество вещества содержит тело, состоящее из $3 \cdot 10^{24}$ молекул?
}
\answer{%
    $\nu = \frac N{N_A} = \frac{3 \cdot 10^{24}}{6{,}02 \cdot 10^{23}\,\frac{1}{\text{моль}}} = 4{,}98\,\text{моль}.$
}
\solutionspace{40pt}

\tasknumber{7}%
\task{%
    Какова масса $4\,\text{моль}$ (\ce{C4H10}) бутана? Молярная масса бутана $58\,\frac{\text{г}}{\text{моль}}$.
}
\answer{%
    $m = \mu\nu = 58\,\frac{\text{г}}{\text{моль}} \cdot 4\,\text{моль} = 232\,\text{г}.$
}
\solutionspace{40pt}

\tasknumber{8}%
\task{%
    Сколько молекул содержится в $50\,\text{г}$ гексана? Молярная масса гексана (\ce{C6H14}) $86\,\frac{\text{г}}{\text{моль}}$.
}
\answer{%
    $N = N_A\nu = N_A\frac{m}{\mu} = 6{,}02 \cdot 10^{23}\,\frac{1}{\text{моль}} \cdot \frac{50\,\text{г}}{86\,\frac{\text{г}}{\text{моль}}} = 350 \cdot 10^{21}.$
}

\variantsplitter

\addpersonalvariant{Софья Андрианова}

\tasknumber{1}%
\task{%
    Молекулы газа в некотором сосуде движутся со средней скоростью $250\,\frac{\text{м}}{\text{с}}$.
    Определите, какое расстояние в среднем проходит одна из таких молекул за $4\,\text{сут}$.
}
\answer{%
    $s = v t = 250\,\frac{\text{м}}{\text{с}} \cdot 4\,\text{сут} = 86 \cdot 10^{6}\,\text{м}.$
}
\solutionspace{40pt}

\tasknumber{2}%
\task{%
    Напротив каждой физической величины укажите её обозначение и единицы измерения в СИ:
    \begin{enumerate}
        \item масса,
        \item количество вещества,
        \item количество молекул.
    \end{enumerate}
}

\tasknumber{3}%
\task{%
    Ответьте на вопросы и запишите формулы:
    \begin{enumerate}
        \item запишите 3 основных положения МКТ,
        \item cвязь количества вещества, массы тела и молярной массы.
    \end{enumerate}
}
\solutionspace{60pt}

\tasknumber{4}%
\task{%
    Определите молярную массу веществ (не табличное значение, а вычислением по таблице Менделеева):
    \begin{enumerate}
        \item гелий,
        \item кислород,
        \item вода.
    \end{enumerate}
}
\solutionspace{30pt}

\tasknumber{5}%
\task{%
    Укажите, верны ли утверждения («да» или «нет» слева от каждого утверждения):
    \begin{enumerate}
        \item В твёрдом состоянии вещества связи между молекулами наиболее сильны (в сравнении с жидким и газообразным состояниями).
        \item Любая частица (например, картошечка в супе) находится в броуновском движении, однако наблюдать его технически возможно только для малых частиц.
        \item Сжимаемость газов объясняется проникновением атомов молекул друг в друга и уменьшением межатомного расстояния внутри молекул.
        \item Броуновское движение частиц пыльцы в жидкости — следствие взаимодействия этих частиц пыльцы между собой.
        \item Если в двух телах одинаковое число молекул, то их массы с большой точностью будут равны.
        \item Если в двух телах одинаковое число протонов и нейтронов (между телами), то и массы тел с большой точностью окажутся равны.
        \item При определении размеров молекул мы зачастую пренебрегаем их формой, не различая радиус и диаметр, а то и вовсе считая их форму кубической.
        \item Диффузия вызвана тепловым движением молекул и может наблюдаться в твердых, жидких и газообразных веществах.
    \end{enumerate}
}
\answer{%
    $
        \text{да, да, нет, нет, нет, да, да, да}
    $
}

\tasknumber{6}%
\task{%
    Какое количество вещества содержит тело, состоящее из $3 \cdot 10^{24}$ молекул?
}
\answer{%
    $\nu = \frac N{N_A} = \frac{3 \cdot 10^{24}}{6{,}02 \cdot 10^{23}\,\frac{1}{\text{моль}}} = 4{,}98\,\text{моль}.$
}
\solutionspace{40pt}

\tasknumber{7}%
\task{%
    Какова масса $5\,\text{моль}$ (\ce{CH4}) метана? Молярная масса метана $16\,\frac{\text{г}}{\text{моль}}$.
}
\answer{%
    $m = \mu\nu = 16\,\frac{\text{г}}{\text{моль}} \cdot 5\,\text{моль} = 80\,\text{г}.$
}
\solutionspace{40pt}

\tasknumber{8}%
\task{%
    Сколько молекул содержится в $200\,\text{г}$ метана? Молярная масса метана (\ce{CH4}) $16\,\frac{\text{г}}{\text{моль}}$.
}
\answer{%
    $N = N_A\nu = N_A\frac{m}{\mu} = 6{,}02 \cdot 10^{23}\,\frac{1}{\text{моль}} \cdot \frac{200\,\text{г}}{16\,\frac{\text{г}}{\text{моль}}} = 8 \cdot 10^{24}.$
}

\variantsplitter

\addpersonalvariant{Владимир Артемчук}

\tasknumber{1}%
\task{%
    Молекулы газа в некотором сосуде движутся со средней скоростью $200\,\frac{\text{м}}{\text{с}}$.
    Определите, какое расстояние в среднем проходит одна из таких молекул за $5\,\text{сут}$.
}
\answer{%
    $s = v t = 200\,\frac{\text{м}}{\text{с}} \cdot 5\,\text{сут} = 86 \cdot 10^{6}\,\text{м}.$
}
\solutionspace{40pt}

\tasknumber{2}%
\task{%
    Напротив каждой физической величины укажите её обозначение и единицы измерения в СИ:
    \begin{enumerate}
        \item объём,
        \item количество вещества,
        \item молярная масса.
    \end{enumerate}
}

\tasknumber{3}%
\task{%
    Ответьте на вопросы и запишите формулы:
    \begin{enumerate}
        \item сформилируйте, что такое броуновское движение,
        \item cвязь количества вещества, массы тела и молярной массы.
    \end{enumerate}
}
\solutionspace{60pt}

\tasknumber{4}%
\task{%
    Определите молярную массу веществ (не табличное значение, а вычислением по таблице Менделеева):
    \begin{enumerate}
        \item неон,
        \item кислород,
        \item вода.
    \end{enumerate}
}
\solutionspace{30pt}

\tasknumber{5}%
\task{%
    Укажите, верны ли утверждения («да» или «нет» слева от каждого утверждения):
    \begin{enumerate}
        \item В твёрдом состоянии вещества связи между молекулами наиболее сильны (в сравнении с жидким и газообразным состояниями).
        \item Любая частица (например, картошечка в супе) находится в броуновском движении, однако наблюдать его технически возможно только для малых частиц.
        \item Сжимаемость газов объясняется проникновением атомов молекул друг в друга и уменьшением межатомного расстояния внутри молекул.
        \item Броуновское движение частиц пыльцы в жидкости — следствие взаимодействия этих частиц пыльцы между собой.
        \item Если в двух телах одинаковое число молекул, то их массы с большой точностью будут равны.
        \item Если в двух телах одинаковое число протонов и нейтронов (между телами), то и массы тел с большой точностью окажутся равны.
        \item При определении размеров молекул мы зачастую пренебрегаем их формой, не различая радиус и диаметр, а то и вовсе считая их форму кубической.
        \item Диффузия вызвана тепловым движением молекул и может наблюдаться в твердых, жидких и газообразных веществах.
    \end{enumerate}
}
\answer{%
    $
        \text{да, да, нет, нет, нет, да, да, да}
    $
}

\tasknumber{6}%
\task{%
    Какое количество вещества содержит тело, состоящее из $3 \cdot 10^{22}$ молекул?
}
\answer{%
    $\nu = \frac N{N_A} = \frac{3 \cdot 10^{22}}{6{,}02 \cdot 10^{23}\,\frac{1}{\text{моль}}} = 0{,}05\,\text{моль}.$
}
\solutionspace{40pt}

\tasknumber{7}%
\task{%
    Какова масса $50\,\text{моль}$ (\ce{C10H22}) декана? Молярная масса декана $142\,\frac{\text{г}}{\text{моль}}$.
}
\answer{%
    $m = \mu\nu = 142\,\frac{\text{г}}{\text{моль}} \cdot 50\,\text{моль} = 7100\,\text{г}.$
}
\solutionspace{40pt}

\tasknumber{8}%
\task{%
    Сколько молекул содержится в $20\,\text{г}$ декана? Молярная масса декана (\ce{C10H22}) $142\,\frac{\text{г}}{\text{моль}}$.
}
\answer{%
    $N = N_A\nu = N_A\frac{m}{\mu} = 6{,}02 \cdot 10^{23}\,\frac{1}{\text{моль}} \cdot \frac{20\,\text{г}}{142\,\frac{\text{г}}{\text{моль}}} = 85 \cdot 10^{21}.$
}

\variantsplitter

\addpersonalvariant{Софья Белянкина}

\tasknumber{1}%
\task{%
    Молекулы газа в некотором сосуде движутся со средней скоростью $150\,\frac{\text{м}}{\text{с}}$.
    Определите, какое расстояние в среднем проходит одна из таких молекул за $5\,\text{сут}$.
}
\answer{%
    $s = v t = 150\,\frac{\text{м}}{\text{с}} \cdot 5\,\text{сут} = 65 \cdot 10^{6}\,\text{м}.$
}
\solutionspace{40pt}

\tasknumber{2}%
\task{%
    Напротив каждой физической величины укажите её обозначение и единицы измерения в СИ:
    \begin{enumerate}
        \item масса,
        \item количество вещества,
        \item количество молекул.
    \end{enumerate}
}

\tasknumber{3}%
\task{%
    Ответьте на вопросы и запишите формулы:
    \begin{enumerate}
        \item сформилируйте, что такое броуновское движение,
        \item cвязь количества вещества, массы тела и молярной массы.
    \end{enumerate}
}
\solutionspace{60pt}

\tasknumber{4}%
\task{%
    Определите молярную массу веществ (не табличное значение, а вычислением по таблице Менделеева):
    \begin{enumerate}
        \item гелий,
        \item кислород,
        \item вода.
    \end{enumerate}
}
\solutionspace{30pt}

\tasknumber{5}%
\task{%
    Укажите, верны ли утверждения («да» или «нет» слева от каждого утверждения):
    \begin{enumerate}
        \item В твёрдом состоянии вещества связи между молекулами наиболее сильны (в сравнении с жидким и газообразным состояниями).
        \item Любая частица (например, картошечка в супе) находится в броуновском движении, однако наблюдать его технически возможно только для малых частиц.
        \item Сжимаемость газов объясняется проникновением атомов молекул друг в друга и уменьшением межатомного расстояния внутри молекул.
        \item Броуновское движение частиц пыльцы в жидкости — следствие взаимодействия этих частиц пыльцы между собой.
        \item Если в двух телах одинаковое число молекул, то их массы с большой точностью будут равны.
        \item Если в двух телах одинаковое число протонов и нейтронов (между телами), то и массы тел с большой точностью окажутся равны.
        \item При определении размеров молекул мы зачастую пренебрегаем их формой, не различая радиус и диаметр, а то и вовсе считая их форму кубической.
        \item Диффузия вызвана тепловым движением молекул и может наблюдаться в твердых, жидких и газообразных веществах.
    \end{enumerate}
}
\answer{%
    $
        \text{да, да, нет, нет, нет, да, да, да}
    $
}

\tasknumber{6}%
\task{%
    Какое количество вещества содержит тело, состоящее из $9 \cdot 10^{23}$ молекул?
}
\answer{%
    $\nu = \frac N{N_A} = \frac{9 \cdot 10^{23}}{6{,}02 \cdot 10^{23}\,\frac{1}{\text{моль}}} = 1{,}50\,\text{моль}.$
}
\solutionspace{40pt}

\tasknumber{7}%
\task{%
    Какова масса $5\,\text{моль}$ (\ce{C3H8}) пропана? Молярная масса пропана $44\,\frac{\text{г}}{\text{моль}}$.
}
\answer{%
    $m = \mu\nu = 44\,\frac{\text{г}}{\text{моль}} \cdot 5\,\text{моль} = 220\,\text{г}.$
}
\solutionspace{40pt}

\tasknumber{8}%
\task{%
    Сколько молекул содержится в $20\,\text{г}$ пропана? Молярная масса пропана (\ce{C3H8}) $44\,\frac{\text{г}}{\text{моль}}$.
}
\answer{%
    $N = N_A\nu = N_A\frac{m}{\mu} = 6{,}02 \cdot 10^{23}\,\frac{1}{\text{моль}} \cdot \frac{20\,\text{г}}{44\,\frac{\text{г}}{\text{моль}}} = 270 \cdot 10^{21}.$
}

\variantsplitter

\addpersonalvariant{Варвара Егиазарян}

\tasknumber{1}%
\task{%
    Молекулы газа в некотором сосуде движутся со средней скоростью $300\,\frac{\text{м}}{\text{с}}$.
    Определите, какое расстояние в среднем проходит одна из таких молекул за $4\,\text{сут}$.
}
\answer{%
    $s = v t = 300\,\frac{\text{м}}{\text{с}} \cdot 4\,\text{сут} = 104 \cdot 10^{6}\,\text{м}.$
}
\solutionspace{40pt}

\tasknumber{2}%
\task{%
    Напротив каждой физической величины укажите её обозначение и единицы измерения в СИ:
    \begin{enumerate}
        \item объём,
        \item количество вещества,
        \item молярная масса.
    \end{enumerate}
}

\tasknumber{3}%
\task{%
    Ответьте на вопросы и запишите формулы:
    \begin{enumerate}
        \item запишите 3 основных положения МКТ,
        \item cвязь количества вещества, массы тела и молярной массы.
    \end{enumerate}
}
\solutionspace{60pt}

\tasknumber{4}%
\task{%
    Определите молярную массу веществ (не табличное значение, а вычислением по таблице Менделеева):
    \begin{enumerate}
        \item неон,
        \item кислород,
        \item озон.
    \end{enumerate}
}
\solutionspace{30pt}

\tasknumber{5}%
\task{%
    Укажите, верны ли утверждения («да» или «нет» слева от каждого утверждения):
    \begin{enumerate}
        \item В твёрдом состоянии вещества связи между молекулами наиболее сильны (в сравнении с жидким и газообразным состояниями).
        \item Любая частица (например, картошечка в супе) находится в броуновском движении, однако наблюдать его технически возможно только для малых частиц.
        \item Сжимаемость газов объясняется проникновением атомов молекул друг в друга и уменьшением межатомного расстояния внутри молекул.
        \item Броуновское движение частиц пыльцы в жидкости — следствие взаимодействия этих частиц пыльцы между собой.
        \item Если в двух телах одинаковое число молекул, то их массы с большой точностью будут равны.
        \item Если в двух телах одинаковое число протонов и нейтронов (между телами), то и массы тел с большой точностью окажутся равны.
        \item При определении размеров молекул мы зачастую пренебрегаем их формой, не различая радиус и диаметр, а то и вовсе считая их форму кубической.
        \item Диффузия вызвана тепловым движением молекул и может наблюдаться в твердых, жидких и газообразных веществах.
    \end{enumerate}
}
\answer{%
    $
        \text{да, да, нет, нет, нет, да, да, да}
    $
}

\tasknumber{6}%
\task{%
    Какое количество вещества содержит тело, состоящее из $3 \cdot 10^{24}$ молекул?
}
\answer{%
    $\nu = \frac N{N_A} = \frac{3 \cdot 10^{24}}{6{,}02 \cdot 10^{23}\,\frac{1}{\text{моль}}} = 4{,}98\,\text{моль}.$
}
\solutionspace{40pt}

\tasknumber{7}%
\task{%
    Какова масса $10\,\text{моль}$ (\ce{C4H10}) бутана? Молярная масса бутана $58\,\frac{\text{г}}{\text{моль}}$.
}
\answer{%
    $m = \mu\nu = 58\,\frac{\text{г}}{\text{моль}} \cdot 10\,\text{моль} = 580\,\text{г}.$
}
\solutionspace{40pt}

\tasknumber{8}%
\task{%
    Сколько молекул содержится в $200\,\text{г}$ пентана? Молярная масса пентана (\ce{C5H12}) $72\,\frac{\text{г}}{\text{моль}}$.
}
\answer{%
    $N = N_A\nu = N_A\frac{m}{\mu} = 6{,}02 \cdot 10^{23}\,\frac{1}{\text{моль}} \cdot \frac{200\,\text{г}}{72\,\frac{\text{г}}{\text{моль}}} = 1{,}67 \cdot 10^{24}.$
}

\variantsplitter

\addpersonalvariant{Владислав Емелин}

\tasknumber{1}%
\task{%
    Молекулы газа в некотором сосуде движутся со средней скоростью $500\,\frac{\text{м}}{\text{с}}$.
    Определите, какое расстояние в среднем проходит одна из таких молекул за $2\,\text{час}$.
}
\answer{%
    $s = v t = 500\,\frac{\text{м}}{\text{с}} \cdot 2\,\text{час} = 3{,}6 \cdot 10^{6}\,\text{м}.$
}
\solutionspace{40pt}

\tasknumber{2}%
\task{%
    Напротив каждой физической величины укажите её обозначение и единицы измерения в СИ:
    \begin{enumerate}
        \item масса,
        \item количество вещества,
        \item молярная масса.
    \end{enumerate}
}

\tasknumber{3}%
\task{%
    Ответьте на вопросы и запишите формулы:
    \begin{enumerate}
        \item запишите 3 основных положения МКТ,
        \item cвязь количества вещества, массы тела и молярной массы.
    \end{enumerate}
}
\solutionspace{60pt}

\tasknumber{4}%
\task{%
    Определите молярную массу веществ (не табличное значение, а вычислением по таблице Менделеева):
    \begin{enumerate}
        \item гелий,
        \item кислород,
        \item озон.
    \end{enumerate}
}
\solutionspace{30pt}

\tasknumber{5}%
\task{%
    Укажите, верны ли утверждения («да» или «нет» слева от каждого утверждения):
    \begin{enumerate}
        \item В твёрдом состоянии вещества связи между молекулами наиболее сильны (в сравнении с жидким и газообразным состояниями).
        \item Любая частица (например, картошечка в супе) находится в броуновском движении, однако наблюдать его технически возможно только для малых частиц.
        \item Сжимаемость газов объясняется проникновением атомов молекул друг в друга и уменьшением межатомного расстояния внутри молекул.
        \item Броуновское движение частиц пыльцы в жидкости — следствие взаимодействия этих частиц пыльцы между собой.
        \item Если в двух телах одинаковое число молекул, то их массы с большой точностью будут равны.
        \item Если в двух телах одинаковое число протонов и нейтронов (между телами), то и массы тел с большой точностью окажутся равны.
        \item При определении размеров молекул мы зачастую пренебрегаем их формой, не различая радиус и диаметр, а то и вовсе считая их форму кубической.
        \item Диффузия вызвана тепловым движением молекул и может наблюдаться в твердых, жидких и газообразных веществах.
    \end{enumerate}
}
\answer{%
    $
        \text{да, да, нет, нет, нет, да, да, да}
    $
}

\tasknumber{6}%
\task{%
    Какое количество вещества содержит тело, состоящее из $12 \cdot 10^{24}$ молекул?
}
\answer{%
    $\nu = \frac N{N_A} = \frac{12 \cdot 10^{24}}{6{,}02 \cdot 10^{23}\,\frac{1}{\text{моль}}} = 19{,}93\,\text{моль}.$
}
\solutionspace{40pt}

\tasknumber{7}%
\task{%
    Какова масса $2\,\text{моль}$ (\ce{CH4}) метана? Молярная масса метана $16\,\frac{\text{г}}{\text{моль}}$.
}
\answer{%
    $m = \mu\nu = 16\,\frac{\text{г}}{\text{моль}} \cdot 2\,\text{моль} = 32\,\text{г}.$
}
\solutionspace{40pt}

\tasknumber{8}%
\task{%
    Сколько молекул содержится в $20\,\text{г}$ метана? Молярная масса метана (\ce{CH4}) $16\,\frac{\text{г}}{\text{моль}}$.
}
\answer{%
    $N = N_A\nu = N_A\frac{m}{\mu} = 6{,}02 \cdot 10^{23}\,\frac{1}{\text{моль}} \cdot \frac{20\,\text{г}}{16\,\frac{\text{г}}{\text{моль}}} = 800 \cdot 10^{21}.$
}

\variantsplitter

\addpersonalvariant{Артём Жичин}

\tasknumber{1}%
\task{%
    Молекулы газа в некотором сосуде движутся со средней скоростью $150\,\frac{\text{м}}{\text{с}}$.
    Определите, какое расстояние в среднем проходит одна из таких молекул за $5\,\text{сут}$.
}
\answer{%
    $s = v t = 150\,\frac{\text{м}}{\text{с}} \cdot 5\,\text{сут} = 65 \cdot 10^{6}\,\text{м}.$
}
\solutionspace{40pt}

\tasknumber{2}%
\task{%
    Напротив каждой физической величины укажите её обозначение и единицы измерения в СИ:
    \begin{enumerate}
        \item масса,
        \item плотность,
        \item количество молекул.
    \end{enumerate}
}

\tasknumber{3}%
\task{%
    Ответьте на вопросы и запишите формулы:
    \begin{enumerate}
        \item сформилируйте, что такое броуновское движение,
        \item cвязь количества вещества, числа частиц и числа Авогадро.
    \end{enumerate}
}
\solutionspace{60pt}

\tasknumber{4}%
\task{%
    Определите молярную массу веществ (не табличное значение, а вычислением по таблице Менделеева):
    \begin{enumerate}
        \item гелий,
        \item азот,
        \item озон.
    \end{enumerate}
}
\solutionspace{30pt}

\tasknumber{5}%
\task{%
    Укажите, верны ли утверждения («да» или «нет» слева от каждого утверждения):
    \begin{enumerate}
        \item В твёрдом состоянии вещества связи между молекулами наиболее сильны (в сравнении с жидким и газообразным состояниями).
        \item Любая частица (например, картошечка в супе) находится в броуновском движении, однако наблюдать его технически возможно только для малых частиц.
        \item Сжимаемость газов объясняется проникновением атомов молекул друг в друга и уменьшением межатомного расстояния внутри молекул.
        \item Броуновское движение частиц пыльцы в жидкости — следствие взаимодействия этих частиц пыльцы между собой.
        \item Если в двух телах одинаковое число молекул, то их массы с большой точностью будут равны.
        \item Если в двух телах одинаковое число протонов и нейтронов (между телами), то и массы тел с большой точностью окажутся равны.
        \item При определении размеров молекул мы зачастую пренебрегаем их формой, не различая радиус и диаметр, а то и вовсе считая их форму кубической.
        \item Диффузия вызвана тепловым движением молекул и может наблюдаться в твердых, жидких и газообразных веществах.
    \end{enumerate}
}
\answer{%
    $
        \text{да, да, нет, нет, нет, да, да, да}
    $
}

\tasknumber{6}%
\task{%
    Какое количество вещества содержит тело, состоящее из $9 \cdot 10^{25}$ молекул?
}
\answer{%
    $\nu = \frac N{N_A} = \frac{9 \cdot 10^{25}}{6{,}02 \cdot 10^{23}\,\frac{1}{\text{моль}}} = 149{,}50\,\text{моль}.$
}
\solutionspace{40pt}

\tasknumber{7}%
\task{%
    Какова масса $2\,\text{моль}$ (\ce{C7H16}) гептана? Молярная масса гептана $100\,\frac{\text{г}}{\text{моль}}$.
}
\answer{%
    $m = \mu\nu = 100\,\frac{\text{г}}{\text{моль}} \cdot 2\,\text{моль} = 200\,\text{г}.$
}
\solutionspace{40pt}

\tasknumber{8}%
\task{%
    Сколько молекул содержится в $200\,\text{г}$ гексана? Молярная масса гексана (\ce{C6H14}) $86\,\frac{\text{г}}{\text{моль}}$.
}
\answer{%
    $N = N_A\nu = N_A\frac{m}{\mu} = 6{,}02 \cdot 10^{23}\,\frac{1}{\text{моль}} \cdot \frac{200\,\text{г}}{86\,\frac{\text{г}}{\text{моль}}} = 1{,}40 \cdot 10^{24}.$
}

\variantsplitter

\addpersonalvariant{Дарья Кошман}

\tasknumber{1}%
\task{%
    Молекулы газа в некотором сосуде движутся со средней скоростью $300\,\frac{\text{м}}{\text{с}}$.
    Определите, какое расстояние в среднем проходит одна из таких молекул за $4\,\text{час}$.
}
\answer{%
    $s = v t = 300\,\frac{\text{м}}{\text{с}} \cdot 4\,\text{час} = 4{,}3 \cdot 10^{6}\,\text{м}.$
}
\solutionspace{40pt}

\tasknumber{2}%
\task{%
    Напротив каждой физической величины укажите её обозначение и единицы измерения в СИ:
    \begin{enumerate}
        \item объём,
        \item плотность,
        \item количество молекул.
    \end{enumerate}
}

\tasknumber{3}%
\task{%
    Ответьте на вопросы и запишите формулы:
    \begin{enumerate}
        \item сформилируйте, что такое броуновское движение,
        \item cвязь количества вещества, числа частиц и числа Авогадро.
    \end{enumerate}
}
\solutionspace{60pt}

\tasknumber{4}%
\task{%
    Определите молярную массу веществ (не табличное значение, а вычислением по таблице Менделеева):
    \begin{enumerate}
        \item неон,
        \item азот,
        \item озон.
    \end{enumerate}
}
\solutionspace{30pt}

\tasknumber{5}%
\task{%
    Укажите, верны ли утверждения («да» или «нет» слева от каждого утверждения):
    \begin{enumerate}
        \item В твёрдом состоянии вещества связи между молекулами наиболее сильны (в сравнении с жидким и газообразным состояниями).
        \item Любая частица (например, картошечка в супе) находится в броуновском движении, однако наблюдать его технически возможно только для малых частиц.
        \item Сжимаемость газов объясняется проникновением атомов молекул друг в друга и уменьшением межатомного расстояния внутри молекул.
        \item Броуновское движение частиц пыльцы в жидкости — следствие взаимодействия этих частиц пыльцы между собой.
        \item Если в двух телах одинаковое число молекул, то их массы с большой точностью будут равны.
        \item Если в двух телах одинаковое число протонов и нейтронов (между телами), то и массы тел с большой точностью окажутся равны.
        \item При определении размеров молекул мы зачастую пренебрегаем их формой, не различая радиус и диаметр, а то и вовсе считая их форму кубической.
        \item Диффузия вызвана тепловым движением молекул и может наблюдаться в твердых, жидких и газообразных веществах.
    \end{enumerate}
}
\answer{%
    $
        \text{да, да, нет, нет, нет, да, да, да}
    $
}

\tasknumber{6}%
\task{%
    Какое количество вещества содержит тело, состоящее из $12 \cdot 10^{22}$ молекул?
}
\answer{%
    $\nu = \frac N{N_A} = \frac{12 \cdot 10^{22}}{6{,}02 \cdot 10^{23}\,\frac{1}{\text{моль}}} = 0{,}20\,\text{моль}.$
}
\solutionspace{40pt}

\tasknumber{7}%
\task{%
    Какова масса $2\,\text{моль}$ (\ce{C8H18}) октана? Молярная масса октана $114\,\frac{\text{г}}{\text{моль}}$.
}
\answer{%
    $m = \mu\nu = 114\,\frac{\text{г}}{\text{моль}} \cdot 2\,\text{моль} = 228\,\text{г}.$
}
\solutionspace{40pt}

\tasknumber{8}%
\task{%
    Сколько молекул содержится в $20\,\text{г}$ декана? Молярная масса декана (\ce{C10H22}) $142\,\frac{\text{г}}{\text{моль}}$.
}
\answer{%
    $N = N_A\nu = N_A\frac{m}{\mu} = 6{,}02 \cdot 10^{23}\,\frac{1}{\text{моль}} \cdot \frac{20\,\text{г}}{142\,\frac{\text{г}}{\text{моль}}} = 85 \cdot 10^{21}.$
}

\variantsplitter

\addpersonalvariant{Анна Кузьмичёва}

\tasknumber{1}%
\task{%
    Молекулы газа в некотором сосуде движутся со средней скоростью $200\,\frac{\text{м}}{\text{с}}$.
    Определите, какое расстояние в среднем проходит одна из таких молекул за $3\,\text{сут}$.
}
\answer{%
    $s = v t = 200\,\frac{\text{м}}{\text{с}} \cdot 3\,\text{сут} = 52 \cdot 10^{6}\,\text{м}.$
}
\solutionspace{40pt}

\tasknumber{2}%
\task{%
    Напротив каждой физической величины укажите её обозначение и единицы измерения в СИ:
    \begin{enumerate}
        \item масса,
        \item количество вещества,
        \item молярная масса.
    \end{enumerate}
}

\tasknumber{3}%
\task{%
    Ответьте на вопросы и запишите формулы:
    \begin{enumerate}
        \item запишите 3 основных положения МКТ,
        \item cвязь количества вещества, числа частиц и числа Авогадро.
    \end{enumerate}
}
\solutionspace{60pt}

\tasknumber{4}%
\task{%
    Определите молярную массу веществ (не табличное значение, а вычислением по таблице Менделеева):
    \begin{enumerate}
        \item гелий,
        \item кислород,
        \item озон.
    \end{enumerate}
}
\solutionspace{30pt}

\tasknumber{5}%
\task{%
    Укажите, верны ли утверждения («да» или «нет» слева от каждого утверждения):
    \begin{enumerate}
        \item В твёрдом состоянии вещества связи между молекулами наиболее сильны (в сравнении с жидким и газообразным состояниями).
        \item Любая частица (например, картошечка в супе) находится в броуновском движении, однако наблюдать его технически возможно только для малых частиц.
        \item Сжимаемость газов объясняется проникновением атомов молекул друг в друга и уменьшением межатомного расстояния внутри молекул.
        \item Броуновское движение частиц пыльцы в жидкости — следствие взаимодействия этих частиц пыльцы между собой.
        \item Если в двух телах одинаковое число молекул, то их массы с большой точностью будут равны.
        \item Если в двух телах одинаковое число протонов и нейтронов (между телами), то и массы тел с большой точностью окажутся равны.
        \item При определении размеров молекул мы зачастую пренебрегаем их формой, не различая радиус и диаметр, а то и вовсе считая их форму кубической.
        \item Диффузия вызвана тепловым движением молекул и может наблюдаться в твердых, жидких и газообразных веществах.
    \end{enumerate}
}
\answer{%
    $
        \text{да, да, нет, нет, нет, да, да, да}
    $
}

\tasknumber{6}%
\task{%
    Какое количество вещества содержит тело, состоящее из $9 \cdot 10^{23}$ молекул?
}
\answer{%
    $\nu = \frac N{N_A} = \frac{9 \cdot 10^{23}}{6{,}02 \cdot 10^{23}\,\frac{1}{\text{моль}}} = 1{,}50\,\text{моль}.$
}
\solutionspace{40pt}

\tasknumber{7}%
\task{%
    Какова масса $25\,\text{моль}$ (\ce{C4H10}) бутана? Молярная масса бутана $58\,\frac{\text{г}}{\text{моль}}$.
}
\answer{%
    $m = \mu\nu = 58\,\frac{\text{г}}{\text{моль}} \cdot 25\,\text{моль} = 1450\,\text{г}.$
}
\solutionspace{40pt}

\tasknumber{8}%
\task{%
    Сколько молекул содержится в $50\,\text{г}$ пентана? Молярная масса пентана (\ce{C5H12}) $72\,\frac{\text{г}}{\text{моль}}$.
}
\answer{%
    $N = N_A\nu = N_A\frac{m}{\mu} = 6{,}02 \cdot 10^{23}\,\frac{1}{\text{моль}} \cdot \frac{50\,\text{г}}{72\,\frac{\text{г}}{\text{моль}}} = 420 \cdot 10^{21}.$
}

\variantsplitter

\addpersonalvariant{Алёна Куприянова}

\tasknumber{1}%
\task{%
    Молекулы газа в некотором сосуде движутся со средней скоростью $300\,\frac{\text{м}}{\text{с}}$.
    Определите, какое расстояние в среднем проходит одна из таких молекул за $5\,\text{час}$.
}
\answer{%
    $s = v t = 300\,\frac{\text{м}}{\text{с}} \cdot 5\,\text{час} = 5{,}4 \cdot 10^{6}\,\text{м}.$
}
\solutionspace{40pt}

\tasknumber{2}%
\task{%
    Напротив каждой физической величины укажите её обозначение и единицы измерения в СИ:
    \begin{enumerate}
        \item объём,
        \item количество вещества,
        \item молярная масса.
    \end{enumerate}
}

\tasknumber{3}%
\task{%
    Ответьте на вопросы и запишите формулы:
    \begin{enumerate}
        \item сформилируйте, что такое броуновское движение,
        \item cвязь количества вещества, массы тела и молярной массы.
    \end{enumerate}
}
\solutionspace{60pt}

\tasknumber{4}%
\task{%
    Определите молярную массу веществ (не табличное значение, а вычислением по таблице Менделеева):
    \begin{enumerate}
        \item неон,
        \item кислород,
        \item озон.
    \end{enumerate}
}
\solutionspace{30pt}

\tasknumber{5}%
\task{%
    Укажите, верны ли утверждения («да» или «нет» слева от каждого утверждения):
    \begin{enumerate}
        \item В твёрдом состоянии вещества связи между молекулами наиболее сильны (в сравнении с жидким и газообразным состояниями).
        \item Любая частица (например, картошечка в супе) находится в броуновском движении, однако наблюдать его технически возможно только для малых частиц.
        \item Сжимаемость газов объясняется проникновением атомов молекул друг в друга и уменьшением межатомного расстояния внутри молекул.
        \item Броуновское движение частиц пыльцы в жидкости — следствие взаимодействия этих частиц пыльцы между собой.
        \item Если в двух телах одинаковое число молекул, то их массы с большой точностью будут равны.
        \item Если в двух телах одинаковое число протонов и нейтронов (между телами), то и массы тел с большой точностью окажутся равны.
        \item При определении размеров молекул мы зачастую пренебрегаем их формой, не различая радиус и диаметр, а то и вовсе считая их форму кубической.
        \item Диффузия вызвана тепловым движением молекул и может наблюдаться в твердых, жидких и газообразных веществах.
    \end{enumerate}
}
\answer{%
    $
        \text{да, да, нет, нет, нет, да, да, да}
    $
}

\tasknumber{6}%
\task{%
    Какое количество вещества содержит тело, состоящее из $3 \cdot 10^{25}$ молекул?
}
\answer{%
    $\nu = \frac N{N_A} = \frac{3 \cdot 10^{25}}{6{,}02 \cdot 10^{23}\,\frac{1}{\text{моль}}} = 49{,}83\,\text{моль}.$
}
\solutionspace{40pt}

\tasknumber{7}%
\task{%
    Какова масса $50\,\text{моль}$ (\ce{C2H6}) этана? Молярная масса этана $30\,\frac{\text{г}}{\text{моль}}$.
}
\answer{%
    $m = \mu\nu = 30\,\frac{\text{г}}{\text{моль}} \cdot 50\,\text{моль} = 1500\,\text{г}.$
}
\solutionspace{40pt}

\tasknumber{8}%
\task{%
    Сколько молекул содержится в $200\,\text{г}$ пропана? Молярная масса пропана (\ce{C3H8}) $44\,\frac{\text{г}}{\text{моль}}$.
}
\answer{%
    $N = N_A\nu = N_A\frac{m}{\mu} = 6{,}02 \cdot 10^{23}\,\frac{1}{\text{моль}} \cdot \frac{200\,\text{г}}{44\,\frac{\text{г}}{\text{моль}}} = 2{,}7 \cdot 10^{24}.$
}

\variantsplitter

\addpersonalvariant{Ярослав Лавровский}

\tasknumber{1}%
\task{%
    Молекулы газа в некотором сосуде движутся со средней скоростью $150\,\frac{\text{м}}{\text{с}}$.
    Определите, какое расстояние в среднем проходит одна из таких молекул за $3\,\text{час}$.
}
\answer{%
    $s = v t = 150\,\frac{\text{м}}{\text{с}} \cdot 3\,\text{час} = 1{,}62 \cdot 10^{6}\,\text{м}.$
}
\solutionspace{40pt}

\tasknumber{2}%
\task{%
    Напротив каждой физической величины укажите её обозначение и единицы измерения в СИ:
    \begin{enumerate}
        \item масса,
        \item плотность,
        \item молярная масса.
    \end{enumerate}
}

\tasknumber{3}%
\task{%
    Ответьте на вопросы и запишите формулы:
    \begin{enumerate}
        \item запишите 3 основных положения МКТ,
        \item cвязь количества вещества, массы тела и молярной массы.
    \end{enumerate}
}
\solutionspace{60pt}

\tasknumber{4}%
\task{%
    Определите молярную массу веществ (не табличное значение, а вычислением по таблице Менделеева):
    \begin{enumerate}
        \item гелий,
        \item азот,
        \item углекислый газ.
    \end{enumerate}
}
\solutionspace{30pt}

\tasknumber{5}%
\task{%
    Укажите, верны ли утверждения («да» или «нет» слева от каждого утверждения):
    \begin{enumerate}
        \item В твёрдом состоянии вещества связи между молекулами наиболее сильны (в сравнении с жидким и газообразным состояниями).
        \item Любая частица (например, картошечка в супе) находится в броуновском движении, однако наблюдать его технически возможно только для малых частиц.
        \item Сжимаемость газов объясняется проникновением атомов молекул друг в друга и уменьшением межатомного расстояния внутри молекул.
        \item Броуновское движение частиц пыльцы в жидкости — следствие взаимодействия этих частиц пыльцы между собой.
        \item Если в двух телах одинаковое число молекул, то их массы с большой точностью будут равны.
        \item Если в двух телах одинаковое число протонов и нейтронов (между телами), то и массы тел с большой точностью окажутся равны.
        \item При определении размеров молекул мы зачастую пренебрегаем их формой, не различая радиус и диаметр, а то и вовсе считая их форму кубической.
        \item Диффузия вызвана тепловым движением молекул и может наблюдаться в твердых, жидких и газообразных веществах.
    \end{enumerate}
}
\answer{%
    $
        \text{да, да, нет, нет, нет, да, да, да}
    $
}

\tasknumber{6}%
\task{%
    Какое количество вещества содержит тело, состоящее из $9 \cdot 10^{22}$ молекул?
}
\answer{%
    $\nu = \frac N{N_A} = \frac{9 \cdot 10^{22}}{6{,}02 \cdot 10^{23}\,\frac{1}{\text{моль}}} = 0{,}15\,\text{моль}.$
}
\solutionspace{40pt}

\tasknumber{7}%
\task{%
    Какова масса $5\,\text{моль}$ (\ce{C2H6}) этана? Молярная масса этана $30\,\frac{\text{г}}{\text{моль}}$.
}
\answer{%
    $m = \mu\nu = 30\,\frac{\text{г}}{\text{моль}} \cdot 5\,\text{моль} = 150\,\text{г}.$
}
\solutionspace{40pt}

\tasknumber{8}%
\task{%
    Сколько молекул содержится в $500\,\text{г}$ декана? Молярная масса декана (\ce{C10H22}) $142\,\frac{\text{г}}{\text{моль}}$.
}
\answer{%
    $N = N_A\nu = N_A\frac{m}{\mu} = 6{,}02 \cdot 10^{23}\,\frac{1}{\text{моль}} \cdot \frac{500\,\text{г}}{142\,\frac{\text{г}}{\text{моль}}} = 2{,}1 \cdot 10^{24}.$
}

\variantsplitter

\addpersonalvariant{Анастасия Ламанова}

\tasknumber{1}%
\task{%
    Молекулы газа в некотором сосуде движутся со средней скоростью $250\,\frac{\text{м}}{\text{с}}$.
    Определите, какое расстояние в среднем проходит одна из таких молекул за $5\,\text{час}$.
}
\answer{%
    $s = v t = 250\,\frac{\text{м}}{\text{с}} \cdot 5\,\text{час} = 4{,}5 \cdot 10^{6}\,\text{м}.$
}
\solutionspace{40pt}

\tasknumber{2}%
\task{%
    Напротив каждой физической величины укажите её обозначение и единицы измерения в СИ:
    \begin{enumerate}
        \item объём,
        \item плотность,
        \item количество молекул.
    \end{enumerate}
}

\tasknumber{3}%
\task{%
    Ответьте на вопросы и запишите формулы:
    \begin{enumerate}
        \item запишите 3 основных положения МКТ,
        \item cвязь количества вещества, массы тела и молярной массы.
    \end{enumerate}
}
\solutionspace{60pt}

\tasknumber{4}%
\task{%
    Определите молярную массу веществ (не табличное значение, а вычислением по таблице Менделеева):
    \begin{enumerate}
        \item неон,
        \item азот,
        \item углекислый газ.
    \end{enumerate}
}
\solutionspace{30pt}

\tasknumber{5}%
\task{%
    Укажите, верны ли утверждения («да» или «нет» слева от каждого утверждения):
    \begin{enumerate}
        \item В твёрдом состоянии вещества связи между молекулами наиболее сильны (в сравнении с жидким и газообразным состояниями).
        \item Любая частица (например, картошечка в супе) находится в броуновском движении, однако наблюдать его технически возможно только для малых частиц.
        \item Сжимаемость газов объясняется проникновением атомов молекул друг в друга и уменьшением межатомного расстояния внутри молекул.
        \item Броуновское движение частиц пыльцы в жидкости — следствие взаимодействия этих частиц пыльцы между собой.
        \item Если в двух телах одинаковое число молекул, то их массы с большой точностью будут равны.
        \item Если в двух телах одинаковое число протонов и нейтронов (между телами), то и массы тел с большой точностью окажутся равны.
        \item При определении размеров молекул мы зачастую пренебрегаем их формой, не различая радиус и диаметр, а то и вовсе считая их форму кубической.
        \item Диффузия вызвана тепловым движением молекул и может наблюдаться в твердых, жидких и газообразных веществах.
    \end{enumerate}
}
\answer{%
    $
        \text{да, да, нет, нет, нет, да, да, да}
    $
}

\tasknumber{6}%
\task{%
    Какое количество вещества содержит тело, состоящее из $3 \cdot 10^{24}$ молекул?
}
\answer{%
    $\nu = \frac N{N_A} = \frac{3 \cdot 10^{24}}{6{,}02 \cdot 10^{23}\,\frac{1}{\text{моль}}} = 4{,}98\,\text{моль}.$
}
\solutionspace{40pt}

\tasknumber{7}%
\task{%
    Какова масса $15\,\text{моль}$ (\ce{C2H6}) этана? Молярная масса этана $30\,\frac{\text{г}}{\text{моль}}$.
}
\answer{%
    $m = \mu\nu = 30\,\frac{\text{г}}{\text{моль}} \cdot 15\,\text{моль} = 450\,\text{г}.$
}
\solutionspace{40pt}

\tasknumber{8}%
\task{%
    Сколько молекул содержится в $50\,\text{г}$ октана? Молярная масса октана (\ce{C8H18}) $114\,\frac{\text{г}}{\text{моль}}$.
}
\answer{%
    $N = N_A\nu = N_A\frac{m}{\mu} = 6{,}02 \cdot 10^{23}\,\frac{1}{\text{моль}} \cdot \frac{50\,\text{г}}{114\,\frac{\text{г}}{\text{моль}}} = 260 \cdot 10^{21}.$
}

\variantsplitter

\addpersonalvariant{Виктория Легонькова}

\tasknumber{1}%
\task{%
    Молекулы газа в некотором сосуде движутся со средней скоростью $300\,\frac{\text{м}}{\text{с}}$.
    Определите, какое расстояние в среднем проходит одна из таких молекул за $3\,\text{сут}$.
}
\answer{%
    $s = v t = 300\,\frac{\text{м}}{\text{с}} \cdot 3\,\text{сут} = 78 \cdot 10^{6}\,\text{м}.$
}
\solutionspace{40pt}

\tasknumber{2}%
\task{%
    Напротив каждой физической величины укажите её обозначение и единицы измерения в СИ:
    \begin{enumerate}
        \item объём,
        \item плотность,
        \item молярная масса.
    \end{enumerate}
}

\tasknumber{3}%
\task{%
    Ответьте на вопросы и запишите формулы:
    \begin{enumerate}
        \item сформилируйте, что такое броуновское движение,
        \item cвязь количества вещества, массы тела и молярной массы.
    \end{enumerate}
}
\solutionspace{60pt}

\tasknumber{4}%
\task{%
    Определите молярную массу веществ (не табличное значение, а вычислением по таблице Менделеева):
    \begin{enumerate}
        \item неон,
        \item азот,
        \item озон.
    \end{enumerate}
}
\solutionspace{30pt}

\tasknumber{5}%
\task{%
    Укажите, верны ли утверждения («да» или «нет» слева от каждого утверждения):
    \begin{enumerate}
        \item В твёрдом состоянии вещества связи между молекулами наиболее сильны (в сравнении с жидким и газообразным состояниями).
        \item Любая частица (например, картошечка в супе) находится в броуновском движении, однако наблюдать его технически возможно только для малых частиц.
        \item Сжимаемость газов объясняется проникновением атомов молекул друг в друга и уменьшением межатомного расстояния внутри молекул.
        \item Броуновское движение частиц пыльцы в жидкости — следствие взаимодействия этих частиц пыльцы между собой.
        \item Если в двух телах одинаковое число молекул, то их массы с большой точностью будут равны.
        \item Если в двух телах одинаковое число протонов и нейтронов (между телами), то и массы тел с большой точностью окажутся равны.
        \item При определении размеров молекул мы зачастую пренебрегаем их формой, не различая радиус и диаметр, а то и вовсе считая их форму кубической.
        \item Диффузия вызвана тепловым движением молекул и может наблюдаться в твердых, жидких и газообразных веществах.
    \end{enumerate}
}
\answer{%
    $
        \text{да, да, нет, нет, нет, да, да, да}
    $
}

\tasknumber{6}%
\task{%
    Какое количество вещества содержит тело, состоящее из $9 \cdot 10^{23}$ молекул?
}
\answer{%
    $\nu = \frac N{N_A} = \frac{9 \cdot 10^{23}}{6{,}02 \cdot 10^{23}\,\frac{1}{\text{моль}}} = 1{,}50\,\text{моль}.$
}
\solutionspace{40pt}

\tasknumber{7}%
\task{%
    Какова масса $20\,\text{моль}$ (\ce{C2H6}) этана? Молярная масса этана $30\,\frac{\text{г}}{\text{моль}}$.
}
\answer{%
    $m = \mu\nu = 30\,\frac{\text{г}}{\text{моль}} \cdot 20\,\text{моль} = 600\,\text{г}.$
}
\solutionspace{40pt}

\tasknumber{8}%
\task{%
    Сколько молекул содержится в $200\,\text{г}$ нонана? Молярная масса нонана (\ce{C9H20}) $128\,\frac{\text{г}}{\text{моль}}$.
}
\answer{%
    $N = N_A\nu = N_A\frac{m}{\mu} = 6{,}02 \cdot 10^{23}\,\frac{1}{\text{моль}} \cdot \frac{200\,\text{г}}{128\,\frac{\text{г}}{\text{моль}}} = 940 \cdot 10^{21}.$
}

\variantsplitter

\addpersonalvariant{Семён Мартынов}

\tasknumber{1}%
\task{%
    Молекулы газа в некотором сосуде движутся со средней скоростью $500\,\frac{\text{м}}{\text{с}}$.
    Определите, какое расстояние в среднем проходит одна из таких молекул за $2\,\text{сут}$.
}
\answer{%
    $s = v t = 500\,\frac{\text{м}}{\text{с}} \cdot 2\,\text{сут} = 86 \cdot 10^{6}\,\text{м}.$
}
\solutionspace{40pt}

\tasknumber{2}%
\task{%
    Напротив каждой физической величины укажите её обозначение и единицы измерения в СИ:
    \begin{enumerate}
        \item объём,
        \item плотность,
        \item количество молекул.
    \end{enumerate}
}

\tasknumber{3}%
\task{%
    Ответьте на вопросы и запишите формулы:
    \begin{enumerate}
        \item сформилируйте, что такое броуновское движение,
        \item cвязь количества вещества, числа частиц и числа Авогадро.
    \end{enumerate}
}
\solutionspace{60pt}

\tasknumber{4}%
\task{%
    Определите молярную массу веществ (не табличное значение, а вычислением по таблице Менделеева):
    \begin{enumerate}
        \item неон,
        \item азот,
        \item озон.
    \end{enumerate}
}
\solutionspace{30pt}

\tasknumber{5}%
\task{%
    Укажите, верны ли утверждения («да» или «нет» слева от каждого утверждения):
    \begin{enumerate}
        \item В твёрдом состоянии вещества связи между молекулами наиболее сильны (в сравнении с жидким и газообразным состояниями).
        \item Любая частица (например, картошечка в супе) находится в броуновском движении, однако наблюдать его технически возможно только для малых частиц.
        \item Сжимаемость газов объясняется проникновением атомов молекул друг в друга и уменьшением межатомного расстояния внутри молекул.
        \item Броуновское движение частиц пыльцы в жидкости — следствие взаимодействия этих частиц пыльцы между собой.
        \item Если в двух телах одинаковое число молекул, то их массы с большой точностью будут равны.
        \item Если в двух телах одинаковое число протонов и нейтронов (между телами), то и массы тел с большой точностью окажутся равны.
        \item При определении размеров молекул мы зачастую пренебрегаем их формой, не различая радиус и диаметр, а то и вовсе считая их форму кубической.
        \item Диффузия вызвана тепловым движением молекул и может наблюдаться в твердых, жидких и газообразных веществах.
    \end{enumerate}
}
\answer{%
    $
        \text{да, да, нет, нет, нет, да, да, да}
    $
}

\tasknumber{6}%
\task{%
    Какое количество вещества содержит тело, состоящее из $9 \cdot 10^{22}$ молекул?
}
\answer{%
    $\nu = \frac N{N_A} = \frac{9 \cdot 10^{22}}{6{,}02 \cdot 10^{23}\,\frac{1}{\text{моль}}} = 0{,}15\,\text{моль}.$
}
\solutionspace{40pt}

\tasknumber{7}%
\task{%
    Какова масса $5\,\text{моль}$ (\ce{C10H22}) декана? Молярная масса декана $142\,\frac{\text{г}}{\text{моль}}$.
}
\answer{%
    $m = \mu\nu = 142\,\frac{\text{г}}{\text{моль}} \cdot 5\,\text{моль} = 710\,\text{г}.$
}
\solutionspace{40pt}

\tasknumber{8}%
\task{%
    Сколько молекул содержится в $200\,\text{г}$ нонана? Молярная масса нонана (\ce{C9H20}) $128\,\frac{\text{г}}{\text{моль}}$.
}
\answer{%
    $N = N_A\nu = N_A\frac{m}{\mu} = 6{,}02 \cdot 10^{23}\,\frac{1}{\text{моль}} \cdot \frac{200\,\text{г}}{128\,\frac{\text{г}}{\text{моль}}} = 940 \cdot 10^{21}.$
}

\variantsplitter

\addpersonalvariant{Варвара Минаева}

\tasknumber{1}%
\task{%
    Молекулы газа в некотором сосуде движутся со средней скоростью $500\,\frac{\text{м}}{\text{с}}$.
    Определите, какое расстояние в среднем проходит одна из таких молекул за $3\,\text{сут}$.
}
\answer{%
    $s = v t = 500\,\frac{\text{м}}{\text{с}} \cdot 3\,\text{сут} = 130 \cdot 10^{6}\,\text{м}.$
}
\solutionspace{40pt}

\tasknumber{2}%
\task{%
    Напротив каждой физической величины укажите её обозначение и единицы измерения в СИ:
    \begin{enumerate}
        \item объём,
        \item количество вещества,
        \item молярная масса.
    \end{enumerate}
}

\tasknumber{3}%
\task{%
    Ответьте на вопросы и запишите формулы:
    \begin{enumerate}
        \item запишите 3 основных положения МКТ,
        \item cвязь количества вещества, массы тела и молярной массы.
    \end{enumerate}
}
\solutionspace{60pt}

\tasknumber{4}%
\task{%
    Определите молярную массу веществ (не табличное значение, а вычислением по таблице Менделеева):
    \begin{enumerate}
        \item неон,
        \item кислород,
        \item озон.
    \end{enumerate}
}
\solutionspace{30pt}

\tasknumber{5}%
\task{%
    Укажите, верны ли утверждения («да» или «нет» слева от каждого утверждения):
    \begin{enumerate}
        \item В твёрдом состоянии вещества связи между молекулами наиболее сильны (в сравнении с жидким и газообразным состояниями).
        \item Любая частица (например, картошечка в супе) находится в броуновском движении, однако наблюдать его технически возможно только для малых частиц.
        \item Сжимаемость газов объясняется проникновением атомов молекул друг в друга и уменьшением межатомного расстояния внутри молекул.
        \item Броуновское движение частиц пыльцы в жидкости — следствие взаимодействия этих частиц пыльцы между собой.
        \item Если в двух телах одинаковое число молекул, то их массы с большой точностью будут равны.
        \item Если в двух телах одинаковое число протонов и нейтронов (между телами), то и массы тел с большой точностью окажутся равны.
        \item При определении размеров молекул мы зачастую пренебрегаем их формой, не различая радиус и диаметр, а то и вовсе считая их форму кубической.
        \item Диффузия вызвана тепловым движением молекул и может наблюдаться в твердых, жидких и газообразных веществах.
    \end{enumerate}
}
\answer{%
    $
        \text{да, да, нет, нет, нет, да, да, да}
    $
}

\tasknumber{6}%
\task{%
    Какое количество вещества содержит тело, состоящее из $12 \cdot 10^{23}$ молекул?
}
\answer{%
    $\nu = \frac N{N_A} = \frac{12 \cdot 10^{23}}{6{,}02 \cdot 10^{23}\,\frac{1}{\text{моль}}} = 1{,}99\,\text{моль}.$
}
\solutionspace{40pt}

\tasknumber{7}%
\task{%
    Какова масса $15\,\text{моль}$ (\ce{C9H20}) нонана? Молярная масса нонана $128\,\frac{\text{г}}{\text{моль}}$.
}
\answer{%
    $m = \mu\nu = 128\,\frac{\text{г}}{\text{моль}} \cdot 15\,\text{моль} = 1920\,\text{г}.$
}
\solutionspace{40pt}

\tasknumber{8}%
\task{%
    Сколько молекул содержится в $20\,\text{г}$ декана? Молярная масса декана (\ce{C10H22}) $142\,\frac{\text{г}}{\text{моль}}$.
}
\answer{%
    $N = N_A\nu = N_A\frac{m}{\mu} = 6{,}02 \cdot 10^{23}\,\frac{1}{\text{моль}} \cdot \frac{20\,\text{г}}{142\,\frac{\text{г}}{\text{моль}}} = 85 \cdot 10^{21}.$
}

\variantsplitter

\addpersonalvariant{Леонид Никитин}

\tasknumber{1}%
\task{%
    Молекулы газа в некотором сосуде движутся со средней скоростью $200\,\frac{\text{м}}{\text{с}}$.
    Определите, какое расстояние в среднем проходит одна из таких молекул за $3\,\text{сут}$.
}
\answer{%
    $s = v t = 200\,\frac{\text{м}}{\text{с}} \cdot 3\,\text{сут} = 52 \cdot 10^{6}\,\text{м}.$
}
\solutionspace{40pt}

\tasknumber{2}%
\task{%
    Напротив каждой физической величины укажите её обозначение и единицы измерения в СИ:
    \begin{enumerate}
        \item объём,
        \item количество вещества,
        \item количество молекул.
    \end{enumerate}
}

\tasknumber{3}%
\task{%
    Ответьте на вопросы и запишите формулы:
    \begin{enumerate}
        \item запишите 3 основных положения МКТ,
        \item cвязь количества вещества, массы тела и молярной массы.
    \end{enumerate}
}
\solutionspace{60pt}

\tasknumber{4}%
\task{%
    Определите молярную массу веществ (не табличное значение, а вычислением по таблице Менделеева):
    \begin{enumerate}
        \item неон,
        \item кислород,
        \item вода.
    \end{enumerate}
}
\solutionspace{30pt}

\tasknumber{5}%
\task{%
    Укажите, верны ли утверждения («да» или «нет» слева от каждого утверждения):
    \begin{enumerate}
        \item В твёрдом состоянии вещества связи между молекулами наиболее сильны (в сравнении с жидким и газообразным состояниями).
        \item Любая частица (например, картошечка в супе) находится в броуновском движении, однако наблюдать его технически возможно только для малых частиц.
        \item Сжимаемость газов объясняется проникновением атомов молекул друг в друга и уменьшением межатомного расстояния внутри молекул.
        \item Броуновское движение частиц пыльцы в жидкости — следствие взаимодействия этих частиц пыльцы между собой.
        \item Если в двух телах одинаковое число молекул, то их массы с большой точностью будут равны.
        \item Если в двух телах одинаковое число протонов и нейтронов (между телами), то и массы тел с большой точностью окажутся равны.
        \item При определении размеров молекул мы зачастую пренебрегаем их формой, не различая радиус и диаметр, а то и вовсе считая их форму кубической.
        \item Диффузия вызвана тепловым движением молекул и может наблюдаться в твердых, жидких и газообразных веществах.
    \end{enumerate}
}
\answer{%
    $
        \text{да, да, нет, нет, нет, да, да, да}
    $
}

\tasknumber{6}%
\task{%
    Какое количество вещества содержит тело, состоящее из $12 \cdot 10^{22}$ молекул?
}
\answer{%
    $\nu = \frac N{N_A} = \frac{12 \cdot 10^{22}}{6{,}02 \cdot 10^{23}\,\frac{1}{\text{моль}}} = 0{,}20\,\text{моль}.$
}
\solutionspace{40pt}

\tasknumber{7}%
\task{%
    Какова масса $2\,\text{моль}$ (\ce{C7H16}) гептана? Молярная масса гептана $100\,\frac{\text{г}}{\text{моль}}$.
}
\answer{%
    $m = \mu\nu = 100\,\frac{\text{г}}{\text{моль}} \cdot 2\,\text{моль} = 200\,\text{г}.$
}
\solutionspace{40pt}

\tasknumber{8}%
\task{%
    Сколько молекул содержится в $500\,\text{г}$ пентана? Молярная масса пентана (\ce{C5H12}) $72\,\frac{\text{г}}{\text{моль}}$.
}
\answer{%
    $N = N_A\nu = N_A\frac{m}{\mu} = 6{,}02 \cdot 10^{23}\,\frac{1}{\text{моль}} \cdot \frac{500\,\text{г}}{72\,\frac{\text{г}}{\text{моль}}} = 4{,}2 \cdot 10^{24}.$
}

\variantsplitter

\addpersonalvariant{Тимофей Полетаев}

\tasknumber{1}%
\task{%
    Молекулы газа в некотором сосуде движутся со средней скоростью $500\,\frac{\text{м}}{\text{с}}$.
    Определите, какое расстояние в среднем проходит одна из таких молекул за $4\,\text{час}$.
}
\answer{%
    $s = v t = 500\,\frac{\text{м}}{\text{с}} \cdot 4\,\text{час} = 7{,}2 \cdot 10^{6}\,\text{м}.$
}
\solutionspace{40pt}

\tasknumber{2}%
\task{%
    Напротив каждой физической величины укажите её обозначение и единицы измерения в СИ:
    \begin{enumerate}
        \item масса,
        \item плотность,
        \item количество молекул.
    \end{enumerate}
}

\tasknumber{3}%
\task{%
    Ответьте на вопросы и запишите формулы:
    \begin{enumerate}
        \item запишите 3 основных положения МКТ,
        \item cвязь количества вещества, числа частиц и числа Авогадро.
    \end{enumerate}
}
\solutionspace{60pt}

\tasknumber{4}%
\task{%
    Определите молярную массу веществ (не табличное значение, а вычислением по таблице Менделеева):
    \begin{enumerate}
        \item гелий,
        \item азот,
        \item озон.
    \end{enumerate}
}
\solutionspace{30pt}

\tasknumber{5}%
\task{%
    Укажите, верны ли утверждения («да» или «нет» слева от каждого утверждения):
    \begin{enumerate}
        \item В твёрдом состоянии вещества связи между молекулами наиболее сильны (в сравнении с жидким и газообразным состояниями).
        \item Любая частица (например, картошечка в супе) находится в броуновском движении, однако наблюдать его технически возможно только для малых частиц.
        \item Сжимаемость газов объясняется проникновением атомов молекул друг в друга и уменьшением межатомного расстояния внутри молекул.
        \item Броуновское движение частиц пыльцы в жидкости — следствие взаимодействия этих частиц пыльцы между собой.
        \item Если в двух телах одинаковое число молекул, то их массы с большой точностью будут равны.
        \item Если в двух телах одинаковое число протонов и нейтронов (между телами), то и массы тел с большой точностью окажутся равны.
        \item При определении размеров молекул мы зачастую пренебрегаем их формой, не различая радиус и диаметр, а то и вовсе считая их форму кубической.
        \item Диффузия вызвана тепловым движением молекул и может наблюдаться в твердых, жидких и газообразных веществах.
    \end{enumerate}
}
\answer{%
    $
        \text{да, да, нет, нет, нет, да, да, да}
    $
}

\tasknumber{6}%
\task{%
    Какое количество вещества содержит тело, состоящее из $12 \cdot 10^{23}$ молекул?
}
\answer{%
    $\nu = \frac N{N_A} = \frac{12 \cdot 10^{23}}{6{,}02 \cdot 10^{23}\,\frac{1}{\text{моль}}} = 1{,}99\,\text{моль}.$
}
\solutionspace{40pt}

\tasknumber{7}%
\task{%
    Какова масса $4\,\text{моль}$ (\ce{C5H12}) пентана? Молярная масса пентана $72\,\frac{\text{г}}{\text{моль}}$.
}
\answer{%
    $m = \mu\nu = 72\,\frac{\text{г}}{\text{моль}} \cdot 4\,\text{моль} = 288\,\text{г}.$
}
\solutionspace{40pt}

\tasknumber{8}%
\task{%
    Сколько молекул содержится в $20\,\text{г}$ этана? Молярная масса этана (\ce{C2H6}) $30\,\frac{\text{г}}{\text{моль}}$.
}
\answer{%
    $N = N_A\nu = N_A\frac{m}{\mu} = 6{,}02 \cdot 10^{23}\,\frac{1}{\text{моль}} \cdot \frac{20\,\text{г}}{30\,\frac{\text{г}}{\text{моль}}} = 400 \cdot 10^{21}.$
}

\variantsplitter

\addpersonalvariant{Андрей Рожков}

\tasknumber{1}%
\task{%
    Молекулы газа в некотором сосуде движутся со средней скоростью $300\,\frac{\text{м}}{\text{с}}$.
    Определите, какое расстояние в среднем проходит одна из таких молекул за $2\,\text{час}$.
}
\answer{%
    $s = v t = 300\,\frac{\text{м}}{\text{с}} \cdot 2\,\text{час} = 2{,}2 \cdot 10^{6}\,\text{м}.$
}
\solutionspace{40pt}

\tasknumber{2}%
\task{%
    Напротив каждой физической величины укажите её обозначение и единицы измерения в СИ:
    \begin{enumerate}
        \item масса,
        \item количество вещества,
        \item количество молекул.
    \end{enumerate}
}

\tasknumber{3}%
\task{%
    Ответьте на вопросы и запишите формулы:
    \begin{enumerate}
        \item сформилируйте, что такое броуновское движение,
        \item cвязь количества вещества, числа частиц и числа Авогадро.
    \end{enumerate}
}
\solutionspace{60pt}

\tasknumber{4}%
\task{%
    Определите молярную массу веществ (не табличное значение, а вычислением по таблице Менделеева):
    \begin{enumerate}
        \item гелий,
        \item кислород,
        \item углекислый газ.
    \end{enumerate}
}
\solutionspace{30pt}

\tasknumber{5}%
\task{%
    Укажите, верны ли утверждения («да» или «нет» слева от каждого утверждения):
    \begin{enumerate}
        \item В твёрдом состоянии вещества связи между молекулами наиболее сильны (в сравнении с жидким и газообразным состояниями).
        \item Любая частица (например, картошечка в супе) находится в броуновском движении, однако наблюдать его технически возможно только для малых частиц.
        \item Сжимаемость газов объясняется проникновением атомов молекул друг в друга и уменьшением межатомного расстояния внутри молекул.
        \item Броуновское движение частиц пыльцы в жидкости — следствие взаимодействия этих частиц пыльцы между собой.
        \item Если в двух телах одинаковое число молекул, то их массы с большой точностью будут равны.
        \item Если в двух телах одинаковое число протонов и нейтронов (между телами), то и массы тел с большой точностью окажутся равны.
        \item При определении размеров молекул мы зачастую пренебрегаем их формой, не различая радиус и диаметр, а то и вовсе считая их форму кубической.
        \item Диффузия вызвана тепловым движением молекул и может наблюдаться в твердых, жидких и газообразных веществах.
    \end{enumerate}
}
\answer{%
    $
        \text{да, да, нет, нет, нет, да, да, да}
    $
}

\tasknumber{6}%
\task{%
    Какое количество вещества содержит тело, состоящее из $12 \cdot 10^{25}$ молекул?
}
\answer{%
    $\nu = \frac N{N_A} = \frac{12 \cdot 10^{25}}{6{,}02 \cdot 10^{23}\,\frac{1}{\text{моль}}} = 199{,}34\,\text{моль}.$
}
\solutionspace{40pt}

\tasknumber{7}%
\task{%
    Какова масса $15\,\text{моль}$ (\ce{C7H16}) гептана? Молярная масса гептана $100\,\frac{\text{г}}{\text{моль}}$.
}
\answer{%
    $m = \mu\nu = 100\,\frac{\text{г}}{\text{моль}} \cdot 15\,\text{моль} = 1500\,\text{г}.$
}
\solutionspace{40pt}

\tasknumber{8}%
\task{%
    Сколько молекул содержится в $20\,\text{г}$ нонана? Молярная масса нонана (\ce{C9H20}) $128\,\frac{\text{г}}{\text{моль}}$.
}
\answer{%
    $N = N_A\nu = N_A\frac{m}{\mu} = 6{,}02 \cdot 10^{23}\,\frac{1}{\text{моль}} \cdot \frac{20\,\text{г}}{128\,\frac{\text{г}}{\text{моль}}} = 94 \cdot 10^{21}.$
}

\variantsplitter

\addpersonalvariant{Рената Таржиманова}

\tasknumber{1}%
\task{%
    Молекулы газа в некотором сосуде движутся со средней скоростью $300\,\frac{\text{м}}{\text{с}}$.
    Определите, какое расстояние в среднем проходит одна из таких молекул за $3\,\text{час}$.
}
\answer{%
    $s = v t = 300\,\frac{\text{м}}{\text{с}} \cdot 3\,\text{час} = 3{,}2 \cdot 10^{6}\,\text{м}.$
}
\solutionspace{40pt}

\tasknumber{2}%
\task{%
    Напротив каждой физической величины укажите её обозначение и единицы измерения в СИ:
    \begin{enumerate}
        \item объём,
        \item плотность,
        \item молярная масса.
    \end{enumerate}
}

\tasknumber{3}%
\task{%
    Ответьте на вопросы и запишите формулы:
    \begin{enumerate}
        \item сформилируйте, что такое броуновское движение,
        \item cвязь количества вещества, массы тела и молярной массы.
    \end{enumerate}
}
\solutionspace{60pt}

\tasknumber{4}%
\task{%
    Определите молярную массу веществ (не табличное значение, а вычислением по таблице Менделеева):
    \begin{enumerate}
        \item неон,
        \item азот,
        \item углекислый газ.
    \end{enumerate}
}
\solutionspace{30pt}

\tasknumber{5}%
\task{%
    Укажите, верны ли утверждения («да» или «нет» слева от каждого утверждения):
    \begin{enumerate}
        \item В твёрдом состоянии вещества связи между молекулами наиболее сильны (в сравнении с жидким и газообразным состояниями).
        \item Любая частица (например, картошечка в супе) находится в броуновском движении, однако наблюдать его технически возможно только для малых частиц.
        \item Сжимаемость газов объясняется проникновением атомов молекул друг в друга и уменьшением межатомного расстояния внутри молекул.
        \item Броуновское движение частиц пыльцы в жидкости — следствие взаимодействия этих частиц пыльцы между собой.
        \item Если в двух телах одинаковое число молекул, то их массы с большой точностью будут равны.
        \item Если в двух телах одинаковое число протонов и нейтронов (между телами), то и массы тел с большой точностью окажутся равны.
        \item При определении размеров молекул мы зачастую пренебрегаем их формой, не различая радиус и диаметр, а то и вовсе считая их форму кубической.
        \item Диффузия вызвана тепловым движением молекул и может наблюдаться в твердых, жидких и газообразных веществах.
    \end{enumerate}
}
\answer{%
    $
        \text{да, да, нет, нет, нет, да, да, да}
    $
}

\tasknumber{6}%
\task{%
    Какое количество вещества содержит тело, состоящее из $3 \cdot 10^{23}$ молекул?
}
\answer{%
    $\nu = \frac N{N_A} = \frac{3 \cdot 10^{23}}{6{,}02 \cdot 10^{23}\,\frac{1}{\text{моль}}} = 0{,}50\,\text{моль}.$
}
\solutionspace{40pt}

\tasknumber{7}%
\task{%
    Какова масса $10\,\text{моль}$ (\ce{C8H18}) октана? Молярная масса октана $114\,\frac{\text{г}}{\text{моль}}$.
}
\answer{%
    $m = \mu\nu = 114\,\frac{\text{г}}{\text{моль}} \cdot 10\,\text{моль} = 1140\,\text{г}.$
}
\solutionspace{40pt}

\tasknumber{8}%
\task{%
    Сколько молекул содержится в $500\,\text{г}$ гексана? Молярная масса гексана (\ce{C6H14}) $86\,\frac{\text{г}}{\text{моль}}$.
}
\answer{%
    $N = N_A\nu = N_A\frac{m}{\mu} = 6{,}02 \cdot 10^{23}\,\frac{1}{\text{моль}} \cdot \frac{500\,\text{г}}{86\,\frac{\text{г}}{\text{моль}}} = 3{,}5 \cdot 10^{24}.$
}

\variantsplitter

\addpersonalvariant{Андрей Щербаков}

\tasknumber{1}%
\task{%
    Молекулы газа в некотором сосуде движутся со средней скоростью $300\,\frac{\text{м}}{\text{с}}$.
    Определите, какое расстояние в среднем проходит одна из таких молекул за $2\,\text{сут}$.
}
\answer{%
    $s = v t = 300\,\frac{\text{м}}{\text{с}} \cdot 2\,\text{сут} = 52 \cdot 10^{6}\,\text{м}.$
}
\solutionspace{40pt}

\tasknumber{2}%
\task{%
    Напротив каждой физической величины укажите её обозначение и единицы измерения в СИ:
    \begin{enumerate}
        \item объём,
        \item плотность,
        \item молярная масса.
    \end{enumerate}
}

\tasknumber{3}%
\task{%
    Ответьте на вопросы и запишите формулы:
    \begin{enumerate}
        \item запишите 3 основных положения МКТ,
        \item cвязь количества вещества, числа частиц и числа Авогадро.
    \end{enumerate}
}
\solutionspace{60pt}

\tasknumber{4}%
\task{%
    Определите молярную массу веществ (не табличное значение, а вычислением по таблице Менделеева):
    \begin{enumerate}
        \item неон,
        \item азот,
        \item озон.
    \end{enumerate}
}
\solutionspace{30pt}

\tasknumber{5}%
\task{%
    Укажите, верны ли утверждения («да» или «нет» слева от каждого утверждения):
    \begin{enumerate}
        \item В твёрдом состоянии вещества связи между молекулами наиболее сильны (в сравнении с жидким и газообразным состояниями).
        \item Любая частица (например, картошечка в супе) находится в броуновском движении, однако наблюдать его технически возможно только для малых частиц.
        \item Сжимаемость газов объясняется проникновением атомов молекул друг в друга и уменьшением межатомного расстояния внутри молекул.
        \item Броуновское движение частиц пыльцы в жидкости — следствие взаимодействия этих частиц пыльцы между собой.
        \item Если в двух телах одинаковое число молекул, то их массы с большой точностью будут равны.
        \item Если в двух телах одинаковое число протонов и нейтронов (между телами), то и массы тел с большой точностью окажутся равны.
        \item При определении размеров молекул мы зачастую пренебрегаем их формой, не различая радиус и диаметр, а то и вовсе считая их форму кубической.
        \item Диффузия вызвана тепловым движением молекул и может наблюдаться в твердых, жидких и газообразных веществах.
    \end{enumerate}
}
\answer{%
    $
        \text{да, да, нет, нет, нет, да, да, да}
    $
}

\tasknumber{6}%
\task{%
    Какое количество вещества содержит тело, состоящее из $12 \cdot 10^{24}$ молекул?
}
\answer{%
    $\nu = \frac N{N_A} = \frac{12 \cdot 10^{24}}{6{,}02 \cdot 10^{23}\,\frac{1}{\text{моль}}} = 19{,}93\,\text{моль}.$
}
\solutionspace{40pt}

\tasknumber{7}%
\task{%
    Какова масса $5\,\text{моль}$ (\ce{C9H20}) нонана? Молярная масса нонана $128\,\frac{\text{г}}{\text{моль}}$.
}
\answer{%
    $m = \mu\nu = 128\,\frac{\text{г}}{\text{моль}} \cdot 5\,\text{моль} = 640\,\text{г}.$
}
\solutionspace{40pt}

\tasknumber{8}%
\task{%
    Сколько молекул содержится в $500\,\text{г}$ гексана? Молярная масса гексана (\ce{C6H14}) $86\,\frac{\text{г}}{\text{моль}}$.
}
\answer{%
    $N = N_A\nu = N_A\frac{m}{\mu} = 6{,}02 \cdot 10^{23}\,\frac{1}{\text{моль}} \cdot \frac{500\,\text{г}}{86\,\frac{\text{г}}{\text{моль}}} = 3{,}5 \cdot 10^{24}.$
}

\variantsplitter

\addpersonalvariant{Михаил Ярошевский}

\tasknumber{1}%
\task{%
    Молекулы газа в некотором сосуде движутся со средней скоростью $200\,\frac{\text{м}}{\text{с}}$.
    Определите, какое расстояние в среднем проходит одна из таких молекул за $3\,\text{час}$.
}
\answer{%
    $s = v t = 200\,\frac{\text{м}}{\text{с}} \cdot 3\,\text{час} = 2{,}2 \cdot 10^{6}\,\text{м}.$
}
\solutionspace{40pt}

\tasknumber{2}%
\task{%
    Напротив каждой физической величины укажите её обозначение и единицы измерения в СИ:
    \begin{enumerate}
        \item объём,
        \item количество вещества,
        \item молярная масса.
    \end{enumerate}
}

\tasknumber{3}%
\task{%
    Ответьте на вопросы и запишите формулы:
    \begin{enumerate}
        \item запишите 3 основных положения МКТ,
        \item cвязь количества вещества, числа частиц и числа Авогадро.
    \end{enumerate}
}
\solutionspace{60pt}

\tasknumber{4}%
\task{%
    Определите молярную массу веществ (не табличное значение, а вычислением по таблице Менделеева):
    \begin{enumerate}
        \item неон,
        \item кислород,
        \item вода.
    \end{enumerate}
}
\solutionspace{30pt}

\tasknumber{5}%
\task{%
    Укажите, верны ли утверждения («да» или «нет» слева от каждого утверждения):
    \begin{enumerate}
        \item В твёрдом состоянии вещества связи между молекулами наиболее сильны (в сравнении с жидким и газообразным состояниями).
        \item Любая частица (например, картошечка в супе) находится в броуновском движении, однако наблюдать его технически возможно только для малых частиц.
        \item Сжимаемость газов объясняется проникновением атомов молекул друг в друга и уменьшением межатомного расстояния внутри молекул.
        \item Броуновское движение частиц пыльцы в жидкости — следствие взаимодействия этих частиц пыльцы между собой.
        \item Если в двух телах одинаковое число молекул, то их массы с большой точностью будут равны.
        \item Если в двух телах одинаковое число протонов и нейтронов (между телами), то и массы тел с большой точностью окажутся равны.
        \item При определении размеров молекул мы зачастую пренебрегаем их формой, не различая радиус и диаметр, а то и вовсе считая их форму кубической.
        \item Диффузия вызвана тепловым движением молекул и может наблюдаться в твердых, жидких и газообразных веществах.
    \end{enumerate}
}
\answer{%
    $
        \text{да, да, нет, нет, нет, да, да, да}
    $
}

\tasknumber{6}%
\task{%
    Какое количество вещества содержит тело, состоящее из $9 \cdot 10^{25}$ молекул?
}
\answer{%
    $\nu = \frac N{N_A} = \frac{9 \cdot 10^{25}}{6{,}02 \cdot 10^{23}\,\frac{1}{\text{моль}}} = 149{,}50\,\text{моль}.$
}
\solutionspace{40pt}

\tasknumber{7}%
\task{%
    Какова масса $2\,\text{моль}$ (\ce{C3H8}) пропана? Молярная масса пропана $44\,\frac{\text{г}}{\text{моль}}$.
}
\answer{%
    $m = \mu\nu = 44\,\frac{\text{г}}{\text{моль}} \cdot 2\,\text{моль} = 88\,\text{г}.$
}
\solutionspace{40pt}

\tasknumber{8}%
\task{%
    Сколько молекул содержится в $20\,\text{г}$ гексана? Молярная масса гексана (\ce{C6H14}) $86\,\frac{\text{г}}{\text{моль}}$.
}
\answer{%
    $N = N_A\nu = N_A\frac{m}{\mu} = 6{,}02 \cdot 10^{23}\,\frac{1}{\text{моль}} \cdot \frac{20\,\text{г}}{86\,\frac{\text{г}}{\text{моль}}} = 140 \cdot 10^{21}.$
}

\variantsplitter

\addpersonalvariant{Алексей Алимпиев}

\tasknumber{1}%
\task{%
    Молекулы газа в некотором сосуде движутся со средней скоростью $200\,\frac{\text{м}}{\text{с}}$.
    Определите, какое расстояние в среднем проходит одна из таких молекул за $5\,\text{час}$.
}
\answer{%
    $s = v t = 200\,\frac{\text{м}}{\text{с}} \cdot 5\,\text{час} = 3{,}6 \cdot 10^{6}\,\text{м}.$
}
\solutionspace{40pt}

\tasknumber{2}%
\task{%
    Напротив каждой физической величины укажите её обозначение и единицы измерения в СИ:
    \begin{enumerate}
        \item масса,
        \item плотность,
        \item молярная масса.
    \end{enumerate}
}

\tasknumber{3}%
\task{%
    Ответьте на вопросы и запишите формулы:
    \begin{enumerate}
        \item сформилируйте, что такое броуновское движение,
        \item cвязь количества вещества, числа частиц и числа Авогадро.
    \end{enumerate}
}
\solutionspace{60pt}

\tasknumber{4}%
\task{%
    Определите молярную массу веществ (не табличное значение, а вычислением по таблице Менделеева):
    \begin{enumerate}
        \item гелий,
        \item азот,
        \item углекислый газ.
    \end{enumerate}
}
\solutionspace{30pt}

\tasknumber{5}%
\task{%
    Укажите, верны ли утверждения («да» или «нет» слева от каждого утверждения):
    \begin{enumerate}
        \item В твёрдом состоянии вещества связи между молекулами наиболее сильны (в сравнении с жидким и газообразным состояниями).
        \item Любая частица (например, картошечка в супе) находится в броуновском движении, однако наблюдать его технически возможно только для малых частиц.
        \item Сжимаемость газов объясняется проникновением атомов молекул друг в друга и уменьшением межатомного расстояния внутри молекул.
        \item Броуновское движение частиц пыльцы в жидкости — следствие взаимодействия этих частиц пыльцы между собой.
        \item Если в двух телах одинаковое число молекул, то их массы с большой точностью будут равны.
        \item Если в двух телах одинаковое число протонов и нейтронов (между телами), то и массы тел с большой точностью окажутся равны.
        \item При определении размеров молекул мы зачастую пренебрегаем их формой, не различая радиус и диаметр, а то и вовсе считая их форму кубической.
        \item Диффузия вызвана тепловым движением молекул и может наблюдаться в твердых, жидких и газообразных веществах.
    \end{enumerate}
}
\answer{%
    $
        \text{да, да, нет, нет, нет, да, да, да}
    $
}

\tasknumber{6}%
\task{%
    Какое количество вещества содержит тело, состоящее из $9 \cdot 10^{25}$ молекул?
}
\answer{%
    $\nu = \frac N{N_A} = \frac{9 \cdot 10^{25}}{6{,}02 \cdot 10^{23}\,\frac{1}{\text{моль}}} = 149{,}50\,\text{моль}.$
}
\solutionspace{40pt}

\tasknumber{7}%
\task{%
    Какова масса $5\,\text{моль}$ (\ce{CH4}) метана? Молярная масса метана $16\,\frac{\text{г}}{\text{моль}}$.
}
\answer{%
    $m = \mu\nu = 16\,\frac{\text{г}}{\text{моль}} \cdot 5\,\text{моль} = 80\,\text{г}.$
}
\solutionspace{40pt}

\tasknumber{8}%
\task{%
    Сколько молекул содержится в $200\,\text{г}$ пропана? Молярная масса пропана (\ce{C3H8}) $44\,\frac{\text{г}}{\text{моль}}$.
}
\answer{%
    $N = N_A\nu = N_A\frac{m}{\mu} = 6{,}02 \cdot 10^{23}\,\frac{1}{\text{моль}} \cdot \frac{200\,\text{г}}{44\,\frac{\text{г}}{\text{моль}}} = 2{,}7 \cdot 10^{24}.$
}

\variantsplitter

\addpersonalvariant{Евгений Васин}

\tasknumber{1}%
\task{%
    Молекулы газа в некотором сосуде движутся со средней скоростью $300\,\frac{\text{м}}{\text{с}}$.
    Определите, какое расстояние в среднем проходит одна из таких молекул за $2\,\text{сут}$.
}
\answer{%
    $s = v t = 300\,\frac{\text{м}}{\text{с}} \cdot 2\,\text{сут} = 52 \cdot 10^{6}\,\text{м}.$
}
\solutionspace{40pt}

\tasknumber{2}%
\task{%
    Напротив каждой физической величины укажите её обозначение и единицы измерения в СИ:
    \begin{enumerate}
        \item объём,
        \item плотность,
        \item количество молекул.
    \end{enumerate}
}

\tasknumber{3}%
\task{%
    Ответьте на вопросы и запишите формулы:
    \begin{enumerate}
        \item запишите 3 основных положения МКТ,
        \item cвязь количества вещества, числа частиц и числа Авогадро.
    \end{enumerate}
}
\solutionspace{60pt}

\tasknumber{4}%
\task{%
    Определите молярную массу веществ (не табличное значение, а вычислением по таблице Менделеева):
    \begin{enumerate}
        \item неон,
        \item азот,
        \item углекислый газ.
    \end{enumerate}
}
\solutionspace{30pt}

\tasknumber{5}%
\task{%
    Укажите, верны ли утверждения («да» или «нет» слева от каждого утверждения):
    \begin{enumerate}
        \item В твёрдом состоянии вещества связи между молекулами наиболее сильны (в сравнении с жидким и газообразным состояниями).
        \item Любая частица (например, картошечка в супе) находится в броуновском движении, однако наблюдать его технически возможно только для малых частиц.
        \item Сжимаемость газов объясняется проникновением атомов молекул друг в друга и уменьшением межатомного расстояния внутри молекул.
        \item Броуновское движение частиц пыльцы в жидкости — следствие взаимодействия этих частиц пыльцы между собой.
        \item Если в двух телах одинаковое число молекул, то их массы с большой точностью будут равны.
        \item Если в двух телах одинаковое число протонов и нейтронов (между телами), то и массы тел с большой точностью окажутся равны.
        \item При определении размеров молекул мы зачастую пренебрегаем их формой, не различая радиус и диаметр, а то и вовсе считая их форму кубической.
        \item Диффузия вызвана тепловым движением молекул и может наблюдаться в твердых, жидких и газообразных веществах.
    \end{enumerate}
}
\answer{%
    $
        \text{да, да, нет, нет, нет, да, да, да}
    $
}

\tasknumber{6}%
\task{%
    Какое количество вещества содержит тело, состоящее из $3 \cdot 10^{22}$ молекул?
}
\answer{%
    $\nu = \frac N{N_A} = \frac{3 \cdot 10^{22}}{6{,}02 \cdot 10^{23}\,\frac{1}{\text{моль}}} = 0{,}05\,\text{моль}.$
}
\solutionspace{40pt}

\tasknumber{7}%
\task{%
    Какова масса $15\,\text{моль}$ (\ce{C3H8}) пропана? Молярная масса пропана $44\,\frac{\text{г}}{\text{моль}}$.
}
\answer{%
    $m = \mu\nu = 44\,\frac{\text{г}}{\text{моль}} \cdot 15\,\text{моль} = 660\,\text{г}.$
}
\solutionspace{40pt}

\tasknumber{8}%
\task{%
    Сколько молекул содержится в $500\,\text{г}$ декана? Молярная масса декана (\ce{C10H22}) $142\,\frac{\text{г}}{\text{моль}}$.
}
\answer{%
    $N = N_A\nu = N_A\frac{m}{\mu} = 6{,}02 \cdot 10^{23}\,\frac{1}{\text{моль}} \cdot \frac{500\,\text{г}}{142\,\frac{\text{г}}{\text{моль}}} = 2{,}1 \cdot 10^{24}.$
}

\variantsplitter

\addpersonalvariant{Вячеслав Волохов}

\tasknumber{1}%
\task{%
    Молекулы газа в некотором сосуде движутся со средней скоростью $300\,\frac{\text{м}}{\text{с}}$.
    Определите, какое расстояние в среднем проходит одна из таких молекул за $3\,\text{сут}$.
}
\answer{%
    $s = v t = 300\,\frac{\text{м}}{\text{с}} \cdot 3\,\text{сут} = 78 \cdot 10^{6}\,\text{м}.$
}
\solutionspace{40pt}

\tasknumber{2}%
\task{%
    Напротив каждой физической величины укажите её обозначение и единицы измерения в СИ:
    \begin{enumerate}
        \item объём,
        \item количество вещества,
        \item молярная масса.
    \end{enumerate}
}

\tasknumber{3}%
\task{%
    Ответьте на вопросы и запишите формулы:
    \begin{enumerate}
        \item сформилируйте, что такое броуновское движение,
        \item cвязь количества вещества, массы тела и молярной массы.
    \end{enumerate}
}
\solutionspace{60pt}

\tasknumber{4}%
\task{%
    Определите молярную массу веществ (не табличное значение, а вычислением по таблице Менделеева):
    \begin{enumerate}
        \item неон,
        \item кислород,
        \item озон.
    \end{enumerate}
}
\solutionspace{30pt}

\tasknumber{5}%
\task{%
    Укажите, верны ли утверждения («да» или «нет» слева от каждого утверждения):
    \begin{enumerate}
        \item В твёрдом состоянии вещества связи между молекулами наиболее сильны (в сравнении с жидким и газообразным состояниями).
        \item Любая частица (например, картошечка в супе) находится в броуновском движении, однако наблюдать его технически возможно только для малых частиц.
        \item Сжимаемость газов объясняется проникновением атомов молекул друг в друга и уменьшением межатомного расстояния внутри молекул.
        \item Броуновское движение частиц пыльцы в жидкости — следствие взаимодействия этих частиц пыльцы между собой.
        \item Если в двух телах одинаковое число молекул, то их массы с большой точностью будут равны.
        \item Если в двух телах одинаковое число протонов и нейтронов (между телами), то и массы тел с большой точностью окажутся равны.
        \item При определении размеров молекул мы зачастую пренебрегаем их формой, не различая радиус и диаметр, а то и вовсе считая их форму кубической.
        \item Диффузия вызвана тепловым движением молекул и может наблюдаться в твердых, жидких и газообразных веществах.
    \end{enumerate}
}
\answer{%
    $
        \text{да, да, нет, нет, нет, да, да, да}
    $
}

\tasknumber{6}%
\task{%
    Какое количество вещества содержит тело, состоящее из $12 \cdot 10^{23}$ молекул?
}
\answer{%
    $\nu = \frac N{N_A} = \frac{12 \cdot 10^{23}}{6{,}02 \cdot 10^{23}\,\frac{1}{\text{моль}}} = 1{,}99\,\text{моль}.$
}
\solutionspace{40pt}

\tasknumber{7}%
\task{%
    Какова масса $20\,\text{моль}$ (\ce{C7H16}) гептана? Молярная масса гептана $100\,\frac{\text{г}}{\text{моль}}$.
}
\answer{%
    $m = \mu\nu = 100\,\frac{\text{г}}{\text{моль}} \cdot 20\,\text{моль} = 2000\,\text{г}.$
}
\solutionspace{40pt}

\tasknumber{8}%
\task{%
    Сколько молекул содержится в $200\,\text{г}$ декана? Молярная масса декана (\ce{C10H22}) $142\,\frac{\text{г}}{\text{моль}}$.
}
\answer{%
    $N = N_A\nu = N_A\frac{m}{\mu} = 6{,}02 \cdot 10^{23}\,\frac{1}{\text{моль}} \cdot \frac{200\,\text{г}}{142\,\frac{\text{г}}{\text{моль}}} = 850 \cdot 10^{21}.$
}

\variantsplitter

\addpersonalvariant{Герман Говоров}

\tasknumber{1}%
\task{%
    Молекулы газа в некотором сосуде движутся со средней скоростью $300\,\frac{\text{м}}{\text{с}}$.
    Определите, какое расстояние в среднем проходит одна из таких молекул за $2\,\text{час}$.
}
\answer{%
    $s = v t = 300\,\frac{\text{м}}{\text{с}} \cdot 2\,\text{час} = 2{,}2 \cdot 10^{6}\,\text{м}.$
}
\solutionspace{40pt}

\tasknumber{2}%
\task{%
    Напротив каждой физической величины укажите её обозначение и единицы измерения в СИ:
    \begin{enumerate}
        \item объём,
        \item количество вещества,
        \item молярная масса.
    \end{enumerate}
}

\tasknumber{3}%
\task{%
    Ответьте на вопросы и запишите формулы:
    \begin{enumerate}
        \item запишите 3 основных положения МКТ,
        \item cвязь количества вещества, числа частиц и числа Авогадро.
    \end{enumerate}
}
\solutionspace{60pt}

\tasknumber{4}%
\task{%
    Определите молярную массу веществ (не табличное значение, а вычислением по таблице Менделеева):
    \begin{enumerate}
        \item неон,
        \item кислород,
        \item озон.
    \end{enumerate}
}
\solutionspace{30pt}

\tasknumber{5}%
\task{%
    Укажите, верны ли утверждения («да» или «нет» слева от каждого утверждения):
    \begin{enumerate}
        \item В твёрдом состоянии вещества связи между молекулами наиболее сильны (в сравнении с жидким и газообразным состояниями).
        \item Любая частица (например, картошечка в супе) находится в броуновском движении, однако наблюдать его технически возможно только для малых частиц.
        \item Сжимаемость газов объясняется проникновением атомов молекул друг в друга и уменьшением межатомного расстояния внутри молекул.
        \item Броуновское движение частиц пыльцы в жидкости — следствие взаимодействия этих частиц пыльцы между собой.
        \item Если в двух телах одинаковое число молекул, то их массы с большой точностью будут равны.
        \item Если в двух телах одинаковое число протонов и нейтронов (между телами), то и массы тел с большой точностью окажутся равны.
        \item При определении размеров молекул мы зачастую пренебрегаем их формой, не различая радиус и диаметр, а то и вовсе считая их форму кубической.
        \item Диффузия вызвана тепловым движением молекул и может наблюдаться в твердых, жидких и газообразных веществах.
    \end{enumerate}
}
\answer{%
    $
        \text{да, да, нет, нет, нет, да, да, да}
    $
}

\tasknumber{6}%
\task{%
    Какое количество вещества содержит тело, состоящее из $12 \cdot 10^{25}$ молекул?
}
\answer{%
    $\nu = \frac N{N_A} = \frac{12 \cdot 10^{25}}{6{,}02 \cdot 10^{23}\,\frac{1}{\text{моль}}} = 199{,}34\,\text{моль}.$
}
\solutionspace{40pt}

\tasknumber{7}%
\task{%
    Какова масса $10\,\text{моль}$ (\ce{C9H20}) нонана? Молярная масса нонана $128\,\frac{\text{г}}{\text{моль}}$.
}
\answer{%
    $m = \mu\nu = 128\,\frac{\text{г}}{\text{моль}} \cdot 10\,\text{моль} = 1280\,\text{г}.$
}
\solutionspace{40pt}

\tasknumber{8}%
\task{%
    Сколько молекул содержится в $500\,\text{г}$ октана? Молярная масса октана (\ce{C8H18}) $114\,\frac{\text{г}}{\text{моль}}$.
}
\answer{%
    $N = N_A\nu = N_A\frac{m}{\mu} = 6{,}02 \cdot 10^{23}\,\frac{1}{\text{моль}} \cdot \frac{500\,\text{г}}{114\,\frac{\text{г}}{\text{моль}}} = 2{,}6 \cdot 10^{24}.$
}

\variantsplitter

\addpersonalvariant{София Журавлёва}

\tasknumber{1}%
\task{%
    Молекулы газа в некотором сосуде движутся со средней скоростью $200\,\frac{\text{м}}{\text{с}}$.
    Определите, какое расстояние в среднем проходит одна из таких молекул за $5\,\text{час}$.
}
\answer{%
    $s = v t = 200\,\frac{\text{м}}{\text{с}} \cdot 5\,\text{час} = 3{,}6 \cdot 10^{6}\,\text{м}.$
}
\solutionspace{40pt}

\tasknumber{2}%
\task{%
    Напротив каждой физической величины укажите её обозначение и единицы измерения в СИ:
    \begin{enumerate}
        \item масса,
        \item плотность,
        \item количество молекул.
    \end{enumerate}
}

\tasknumber{3}%
\task{%
    Ответьте на вопросы и запишите формулы:
    \begin{enumerate}
        \item запишите 3 основных положения МКТ,
        \item cвязь количества вещества, массы тела и молярной массы.
    \end{enumerate}
}
\solutionspace{60pt}

\tasknumber{4}%
\task{%
    Определите молярную массу веществ (не табличное значение, а вычислением по таблице Менделеева):
    \begin{enumerate}
        \item гелий,
        \item азот,
        \item вода.
    \end{enumerate}
}
\solutionspace{30pt}

\tasknumber{5}%
\task{%
    Укажите, верны ли утверждения («да» или «нет» слева от каждого утверждения):
    \begin{enumerate}
        \item В твёрдом состоянии вещества связи между молекулами наиболее сильны (в сравнении с жидким и газообразным состояниями).
        \item Любая частица (например, картошечка в супе) находится в броуновском движении, однако наблюдать его технически возможно только для малых частиц.
        \item Сжимаемость газов объясняется проникновением атомов молекул друг в друга и уменьшением межатомного расстояния внутри молекул.
        \item Броуновское движение частиц пыльцы в жидкости — следствие взаимодействия этих частиц пыльцы между собой.
        \item Если в двух телах одинаковое число молекул, то их массы с большой точностью будут равны.
        \item Если в двух телах одинаковое число протонов и нейтронов (между телами), то и массы тел с большой точностью окажутся равны.
        \item При определении размеров молекул мы зачастую пренебрегаем их формой, не различая радиус и диаметр, а то и вовсе считая их форму кубической.
        \item Диффузия вызвана тепловым движением молекул и может наблюдаться в твердых, жидких и газообразных веществах.
    \end{enumerate}
}
\answer{%
    $
        \text{да, да, нет, нет, нет, да, да, да}
    $
}

\tasknumber{6}%
\task{%
    Какое количество вещества содержит тело, состоящее из $3 \cdot 10^{22}$ молекул?
}
\answer{%
    $\nu = \frac N{N_A} = \frac{3 \cdot 10^{22}}{6{,}02 \cdot 10^{23}\,\frac{1}{\text{моль}}} = 0{,}05\,\text{моль}.$
}
\solutionspace{40pt}

\tasknumber{7}%
\task{%
    Какова масса $50\,\text{моль}$ (\ce{C3H8}) пропана? Молярная масса пропана $44\,\frac{\text{г}}{\text{моль}}$.
}
\answer{%
    $m = \mu\nu = 44\,\frac{\text{г}}{\text{моль}} \cdot 50\,\text{моль} = 2200\,\text{г}.$
}
\solutionspace{40pt}

\tasknumber{8}%
\task{%
    Сколько молекул содержится в $500\,\text{г}$ этана? Молярная масса этана (\ce{C2H6}) $30\,\frac{\text{г}}{\text{моль}}$.
}
\answer{%
    $N = N_A\nu = N_A\frac{m}{\mu} = 6{,}02 \cdot 10^{23}\,\frac{1}{\text{моль}} \cdot \frac{500\,\text{г}}{30\,\frac{\text{г}}{\text{моль}}} = 10{,}0 \cdot 10^{24}.$
}

\variantsplitter

\addpersonalvariant{Константин Козлов}

\tasknumber{1}%
\task{%
    Молекулы газа в некотором сосуде движутся со средней скоростью $500\,\frac{\text{м}}{\text{с}}$.
    Определите, какое расстояние в среднем проходит одна из таких молекул за $4\,\text{час}$.
}
\answer{%
    $s = v t = 500\,\frac{\text{м}}{\text{с}} \cdot 4\,\text{час} = 7{,}2 \cdot 10^{6}\,\text{м}.$
}
\solutionspace{40pt}

\tasknumber{2}%
\task{%
    Напротив каждой физической величины укажите её обозначение и единицы измерения в СИ:
    \begin{enumerate}
        \item объём,
        \item плотность,
        \item количество молекул.
    \end{enumerate}
}

\tasknumber{3}%
\task{%
    Ответьте на вопросы и запишите формулы:
    \begin{enumerate}
        \item запишите 3 основных положения МКТ,
        \item cвязь количества вещества, массы тела и молярной массы.
    \end{enumerate}
}
\solutionspace{60pt}

\tasknumber{4}%
\task{%
    Определите молярную массу веществ (не табличное значение, а вычислением по таблице Менделеева):
    \begin{enumerate}
        \item неон,
        \item азот,
        \item вода.
    \end{enumerate}
}
\solutionspace{30pt}

\tasknumber{5}%
\task{%
    Укажите, верны ли утверждения («да» или «нет» слева от каждого утверждения):
    \begin{enumerate}
        \item В твёрдом состоянии вещества связи между молекулами наиболее сильны (в сравнении с жидким и газообразным состояниями).
        \item Любая частица (например, картошечка в супе) находится в броуновском движении, однако наблюдать его технически возможно только для малых частиц.
        \item Сжимаемость газов объясняется проникновением атомов молекул друг в друга и уменьшением межатомного расстояния внутри молекул.
        \item Броуновское движение частиц пыльцы в жидкости — следствие взаимодействия этих частиц пыльцы между собой.
        \item Если в двух телах одинаковое число молекул, то их массы с большой точностью будут равны.
        \item Если в двух телах одинаковое число протонов и нейтронов (между телами), то и массы тел с большой точностью окажутся равны.
        \item При определении размеров молекул мы зачастую пренебрегаем их формой, не различая радиус и диаметр, а то и вовсе считая их форму кубической.
        \item Диффузия вызвана тепловым движением молекул и может наблюдаться в твердых, жидких и газообразных веществах.
    \end{enumerate}
}
\answer{%
    $
        \text{да, да, нет, нет, нет, да, да, да}
    $
}

\tasknumber{6}%
\task{%
    Какое количество вещества содержит тело, состоящее из $9 \cdot 10^{22}$ молекул?
}
\answer{%
    $\nu = \frac N{N_A} = \frac{9 \cdot 10^{22}}{6{,}02 \cdot 10^{23}\,\frac{1}{\text{моль}}} = 0{,}15\,\text{моль}.$
}
\solutionspace{40pt}

\tasknumber{7}%
\task{%
    Какова масса $25\,\text{моль}$ (\ce{C3H8}) пропана? Молярная масса пропана $44\,\frac{\text{г}}{\text{моль}}$.
}
\answer{%
    $m = \mu\nu = 44\,\frac{\text{г}}{\text{моль}} \cdot 25\,\text{моль} = 1100\,\text{г}.$
}
\solutionspace{40pt}

\tasknumber{8}%
\task{%
    Сколько молекул содержится в $500\,\text{г}$ метана? Молярная масса метана (\ce{CH4}) $16\,\frac{\text{г}}{\text{моль}}$.
}
\answer{%
    $N = N_A\nu = N_A\frac{m}{\mu} = 6{,}02 \cdot 10^{23}\,\frac{1}{\text{моль}} \cdot \frac{500\,\text{г}}{16\,\frac{\text{г}}{\text{моль}}} = 19 \cdot 10^{24}.$
}

\variantsplitter

\addpersonalvariant{Наталья Кравченко}

\tasknumber{1}%
\task{%
    Молекулы газа в некотором сосуде движутся со средней скоростью $250\,\frac{\text{м}}{\text{с}}$.
    Определите, какое расстояние в среднем проходит одна из таких молекул за $4\,\text{час}$.
}
\answer{%
    $s = v t = 250\,\frac{\text{м}}{\text{с}} \cdot 4\,\text{час} = 3{,}6 \cdot 10^{6}\,\text{м}.$
}
\solutionspace{40pt}

\tasknumber{2}%
\task{%
    Напротив каждой физической величины укажите её обозначение и единицы измерения в СИ:
    \begin{enumerate}
        \item масса,
        \item плотность,
        \item количество молекул.
    \end{enumerate}
}

\tasknumber{3}%
\task{%
    Ответьте на вопросы и запишите формулы:
    \begin{enumerate}
        \item сформилируйте, что такое броуновское движение,
        \item cвязь количества вещества, массы тела и молярной массы.
    \end{enumerate}
}
\solutionspace{60pt}

\tasknumber{4}%
\task{%
    Определите молярную массу веществ (не табличное значение, а вычислением по таблице Менделеева):
    \begin{enumerate}
        \item гелий,
        \item азот,
        \item озон.
    \end{enumerate}
}
\solutionspace{30pt}

\tasknumber{5}%
\task{%
    Укажите, верны ли утверждения («да» или «нет» слева от каждого утверждения):
    \begin{enumerate}
        \item В твёрдом состоянии вещества связи между молекулами наиболее сильны (в сравнении с жидким и газообразным состояниями).
        \item Любая частица (например, картошечка в супе) находится в броуновском движении, однако наблюдать его технически возможно только для малых частиц.
        \item Сжимаемость газов объясняется проникновением атомов молекул друг в друга и уменьшением межатомного расстояния внутри молекул.
        \item Броуновское движение частиц пыльцы в жидкости — следствие взаимодействия этих частиц пыльцы между собой.
        \item Если в двух телах одинаковое число молекул, то их массы с большой точностью будут равны.
        \item Если в двух телах одинаковое число протонов и нейтронов (между телами), то и массы тел с большой точностью окажутся равны.
        \item При определении размеров молекул мы зачастую пренебрегаем их формой, не различая радиус и диаметр, а то и вовсе считая их форму кубической.
        \item Диффузия вызвана тепловым движением молекул и может наблюдаться в твердых, жидких и газообразных веществах.
    \end{enumerate}
}
\answer{%
    $
        \text{да, да, нет, нет, нет, да, да, да}
    $
}

\tasknumber{6}%
\task{%
    Какое количество вещества содержит тело, состоящее из $3 \cdot 10^{25}$ молекул?
}
\answer{%
    $\nu = \frac N{N_A} = \frac{3 \cdot 10^{25}}{6{,}02 \cdot 10^{23}\,\frac{1}{\text{моль}}} = 49{,}83\,\text{моль}.$
}
\solutionspace{40pt}

\tasknumber{7}%
\task{%
    Какова масса $50\,\text{моль}$ (\ce{CH4}) метана? Молярная масса метана $16\,\frac{\text{г}}{\text{моль}}$.
}
\answer{%
    $m = \mu\nu = 16\,\frac{\text{г}}{\text{моль}} \cdot 50\,\text{моль} = 800\,\text{г}.$
}
\solutionspace{40pt}

\tasknumber{8}%
\task{%
    Сколько молекул содержится в $20\,\text{г}$ пентана? Молярная масса пентана (\ce{C5H12}) $72\,\frac{\text{г}}{\text{моль}}$.
}
\answer{%
    $N = N_A\nu = N_A\frac{m}{\mu} = 6{,}02 \cdot 10^{23}\,\frac{1}{\text{моль}} \cdot \frac{20\,\text{г}}{72\,\frac{\text{г}}{\text{моль}}} = 167 \cdot 10^{21}.$
}

\variantsplitter

\addpersonalvariant{Матвей Кузьмин}

\tasknumber{1}%
\task{%
    Молекулы газа в некотором сосуде движутся со средней скоростью $250\,\frac{\text{м}}{\text{с}}$.
    Определите, какое расстояние в среднем проходит одна из таких молекул за $4\,\text{сут}$.
}
\answer{%
    $s = v t = 250\,\frac{\text{м}}{\text{с}} \cdot 4\,\text{сут} = 86 \cdot 10^{6}\,\text{м}.$
}
\solutionspace{40pt}

\tasknumber{2}%
\task{%
    Напротив каждой физической величины укажите её обозначение и единицы измерения в СИ:
    \begin{enumerate}
        \item масса,
        \item плотность,
        \item молярная масса.
    \end{enumerate}
}

\tasknumber{3}%
\task{%
    Ответьте на вопросы и запишите формулы:
    \begin{enumerate}
        \item сформилируйте, что такое броуновское движение,
        \item cвязь количества вещества, числа частиц и числа Авогадро.
    \end{enumerate}
}
\solutionspace{60pt}

\tasknumber{4}%
\task{%
    Определите молярную массу веществ (не табличное значение, а вычислением по таблице Менделеева):
    \begin{enumerate}
        \item гелий,
        \item азот,
        \item углекислый газ.
    \end{enumerate}
}
\solutionspace{30pt}

\tasknumber{5}%
\task{%
    Укажите, верны ли утверждения («да» или «нет» слева от каждого утверждения):
    \begin{enumerate}
        \item В твёрдом состоянии вещества связи между молекулами наиболее сильны (в сравнении с жидким и газообразным состояниями).
        \item Любая частица (например, картошечка в супе) находится в броуновском движении, однако наблюдать его технически возможно только для малых частиц.
        \item Сжимаемость газов объясняется проникновением атомов молекул друг в друга и уменьшением межатомного расстояния внутри молекул.
        \item Броуновское движение частиц пыльцы в жидкости — следствие взаимодействия этих частиц пыльцы между собой.
        \item Если в двух телах одинаковое число молекул, то их массы с большой точностью будут равны.
        \item Если в двух телах одинаковое число протонов и нейтронов (между телами), то и массы тел с большой точностью окажутся равны.
        \item При определении размеров молекул мы зачастую пренебрегаем их формой, не различая радиус и диаметр, а то и вовсе считая их форму кубической.
        \item Диффузия вызвана тепловым движением молекул и может наблюдаться в твердых, жидких и газообразных веществах.
    \end{enumerate}
}
\answer{%
    $
        \text{да, да, нет, нет, нет, да, да, да}
    $
}

\tasknumber{6}%
\task{%
    Какое количество вещества содержит тело, состоящее из $3 \cdot 10^{25}$ молекул?
}
\answer{%
    $\nu = \frac N{N_A} = \frac{3 \cdot 10^{25}}{6{,}02 \cdot 10^{23}\,\frac{1}{\text{моль}}} = 49{,}83\,\text{моль}.$
}
\solutionspace{40pt}

\tasknumber{7}%
\task{%
    Какова масса $20\,\text{моль}$ (\ce{C3H8}) пропана? Молярная масса пропана $44\,\frac{\text{г}}{\text{моль}}$.
}
\answer{%
    $m = \mu\nu = 44\,\frac{\text{г}}{\text{моль}} \cdot 20\,\text{моль} = 880\,\text{г}.$
}
\solutionspace{40pt}

\tasknumber{8}%
\task{%
    Сколько молекул содержится в $20\,\text{г}$ октана? Молярная масса октана (\ce{C8H18}) $114\,\frac{\text{г}}{\text{моль}}$.
}
\answer{%
    $N = N_A\nu = N_A\frac{m}{\mu} = 6{,}02 \cdot 10^{23}\,\frac{1}{\text{моль}} \cdot \frac{20\,\text{г}}{114\,\frac{\text{г}}{\text{моль}}} = 106 \cdot 10^{21}.$
}

\variantsplitter

\addpersonalvariant{Сергей Малышев}

\tasknumber{1}%
\task{%
    Молекулы газа в некотором сосуде движутся со средней скоростью $250\,\frac{\text{м}}{\text{с}}$.
    Определите, какое расстояние в среднем проходит одна из таких молекул за $4\,\text{сут}$.
}
\answer{%
    $s = v t = 250\,\frac{\text{м}}{\text{с}} \cdot 4\,\text{сут} = 86 \cdot 10^{6}\,\text{м}.$
}
\solutionspace{40pt}

\tasknumber{2}%
\task{%
    Напротив каждой физической величины укажите её обозначение и единицы измерения в СИ:
    \begin{enumerate}
        \item объём,
        \item плотность,
        \item молярная масса.
    \end{enumerate}
}

\tasknumber{3}%
\task{%
    Ответьте на вопросы и запишите формулы:
    \begin{enumerate}
        \item сформилируйте, что такое броуновское движение,
        \item cвязь количества вещества, массы тела и молярной массы.
    \end{enumerate}
}
\solutionspace{60pt}

\tasknumber{4}%
\task{%
    Определите молярную массу веществ (не табличное значение, а вычислением по таблице Менделеева):
    \begin{enumerate}
        \item неон,
        \item азот,
        \item вода.
    \end{enumerate}
}
\solutionspace{30pt}

\tasknumber{5}%
\task{%
    Укажите, верны ли утверждения («да» или «нет» слева от каждого утверждения):
    \begin{enumerate}
        \item В твёрдом состоянии вещества связи между молекулами наиболее сильны (в сравнении с жидким и газообразным состояниями).
        \item Любая частица (например, картошечка в супе) находится в броуновском движении, однако наблюдать его технически возможно только для малых частиц.
        \item Сжимаемость газов объясняется проникновением атомов молекул друг в друга и уменьшением межатомного расстояния внутри молекул.
        \item Броуновское движение частиц пыльцы в жидкости — следствие взаимодействия этих частиц пыльцы между собой.
        \item Если в двух телах одинаковое число молекул, то их массы с большой точностью будут равны.
        \item Если в двух телах одинаковое число протонов и нейтронов (между телами), то и массы тел с большой точностью окажутся равны.
        \item При определении размеров молекул мы зачастую пренебрегаем их формой, не различая радиус и диаметр, а то и вовсе считая их форму кубической.
        \item Диффузия вызвана тепловым движением молекул и может наблюдаться в твердых, жидких и газообразных веществах.
    \end{enumerate}
}
\answer{%
    $
        \text{да, да, нет, нет, нет, да, да, да}
    $
}

\tasknumber{6}%
\task{%
    Какое количество вещества содержит тело, состоящее из $12 \cdot 10^{22}$ молекул?
}
\answer{%
    $\nu = \frac N{N_A} = \frac{12 \cdot 10^{22}}{6{,}02 \cdot 10^{23}\,\frac{1}{\text{моль}}} = 0{,}20\,\text{моль}.$
}
\solutionspace{40pt}

\tasknumber{7}%
\task{%
    Какова масса $15\,\text{моль}$ (\ce{C3H8}) пропана? Молярная масса пропана $44\,\frac{\text{г}}{\text{моль}}$.
}
\answer{%
    $m = \mu\nu = 44\,\frac{\text{г}}{\text{моль}} \cdot 15\,\text{моль} = 660\,\text{г}.$
}
\solutionspace{40pt}

\tasknumber{8}%
\task{%
    Сколько молекул содержится в $200\,\text{г}$ октана? Молярная масса октана (\ce{C8H18}) $114\,\frac{\text{г}}{\text{моль}}$.
}
\answer{%
    $N = N_A\nu = N_A\frac{m}{\mu} = 6{,}02 \cdot 10^{23}\,\frac{1}{\text{моль}} \cdot \frac{200\,\text{г}}{114\,\frac{\text{г}}{\text{моль}}} = 1{,}06 \cdot 10^{24}.$
}

\variantsplitter

\addpersonalvariant{Алина Полканова}

\tasknumber{1}%
\task{%
    Молекулы газа в некотором сосуде движутся со средней скоростью $200\,\frac{\text{м}}{\text{с}}$.
    Определите, какое расстояние в среднем проходит одна из таких молекул за $4\,\text{сут}$.
}
\answer{%
    $s = v t = 200\,\frac{\text{м}}{\text{с}} \cdot 4\,\text{сут} = 69 \cdot 10^{6}\,\text{м}.$
}
\solutionspace{40pt}

\tasknumber{2}%
\task{%
    Напротив каждой физической величины укажите её обозначение и единицы измерения в СИ:
    \begin{enumerate}
        \item объём,
        \item количество вещества,
        \item количество молекул.
    \end{enumerate}
}

\tasknumber{3}%
\task{%
    Ответьте на вопросы и запишите формулы:
    \begin{enumerate}
        \item сформилируйте, что такое броуновское движение,
        \item cвязь количества вещества, числа частиц и числа Авогадро.
    \end{enumerate}
}
\solutionspace{60pt}

\tasknumber{4}%
\task{%
    Определите молярную массу веществ (не табличное значение, а вычислением по таблице Менделеева):
    \begin{enumerate}
        \item неон,
        \item кислород,
        \item вода.
    \end{enumerate}
}
\solutionspace{30pt}

\tasknumber{5}%
\task{%
    Укажите, верны ли утверждения («да» или «нет» слева от каждого утверждения):
    \begin{enumerate}
        \item В твёрдом состоянии вещества связи между молекулами наиболее сильны (в сравнении с жидким и газообразным состояниями).
        \item Любая частица (например, картошечка в супе) находится в броуновском движении, однако наблюдать его технически возможно только для малых частиц.
        \item Сжимаемость газов объясняется проникновением атомов молекул друг в друга и уменьшением межатомного расстояния внутри молекул.
        \item Броуновское движение частиц пыльцы в жидкости — следствие взаимодействия этих частиц пыльцы между собой.
        \item Если в двух телах одинаковое число молекул, то их массы с большой точностью будут равны.
        \item Если в двух телах одинаковое число протонов и нейтронов (между телами), то и массы тел с большой точностью окажутся равны.
        \item При определении размеров молекул мы зачастую пренебрегаем их формой, не различая радиус и диаметр, а то и вовсе считая их форму кубической.
        \item Диффузия вызвана тепловым движением молекул и может наблюдаться в твердых, жидких и газообразных веществах.
    \end{enumerate}
}
\answer{%
    $
        \text{да, да, нет, нет, нет, да, да, да}
    $
}

\tasknumber{6}%
\task{%
    Какое количество вещества содержит тело, состоящее из $3 \cdot 10^{24}$ молекул?
}
\answer{%
    $\nu = \frac N{N_A} = \frac{3 \cdot 10^{24}}{6{,}02 \cdot 10^{23}\,\frac{1}{\text{моль}}} = 4{,}98\,\text{моль}.$
}
\solutionspace{40pt}

\tasknumber{7}%
\task{%
    Какова масса $10\,\text{моль}$ (\ce{C8H18}) октана? Молярная масса октана $114\,\frac{\text{г}}{\text{моль}}$.
}
\answer{%
    $m = \mu\nu = 114\,\frac{\text{г}}{\text{моль}} \cdot 10\,\text{моль} = 1140\,\text{г}.$
}
\solutionspace{40pt}

\tasknumber{8}%
\task{%
    Сколько молекул содержится в $50\,\text{г}$ метана? Молярная масса метана (\ce{CH4}) $16\,\frac{\text{г}}{\text{моль}}$.
}
\answer{%
    $N = N_A\nu = N_A\frac{m}{\mu} = 6{,}02 \cdot 10^{23}\,\frac{1}{\text{моль}} \cdot \frac{50\,\text{г}}{16\,\frac{\text{г}}{\text{моль}}} = 1{,}9 \cdot 10^{24}.$
}

\variantsplitter

\addpersonalvariant{Сергей Пономарёв}

\tasknumber{1}%
\task{%
    Молекулы газа в некотором сосуде движутся со средней скоростью $250\,\frac{\text{м}}{\text{с}}$.
    Определите, какое расстояние в среднем проходит одна из таких молекул за $4\,\text{час}$.
}
\answer{%
    $s = v t = 250\,\frac{\text{м}}{\text{с}} \cdot 4\,\text{час} = 3{,}6 \cdot 10^{6}\,\text{м}.$
}
\solutionspace{40pt}

\tasknumber{2}%
\task{%
    Напротив каждой физической величины укажите её обозначение и единицы измерения в СИ:
    \begin{enumerate}
        \item масса,
        \item плотность,
        \item количество молекул.
    \end{enumerate}
}

\tasknumber{3}%
\task{%
    Ответьте на вопросы и запишите формулы:
    \begin{enumerate}
        \item сформилируйте, что такое броуновское движение,
        \item cвязь количества вещества, числа частиц и числа Авогадро.
    \end{enumerate}
}
\solutionspace{60pt}

\tasknumber{4}%
\task{%
    Определите молярную массу веществ (не табличное значение, а вычислением по таблице Менделеева):
    \begin{enumerate}
        \item гелий,
        \item азот,
        \item озон.
    \end{enumerate}
}
\solutionspace{30pt}

\tasknumber{5}%
\task{%
    Укажите, верны ли утверждения («да» или «нет» слева от каждого утверждения):
    \begin{enumerate}
        \item В твёрдом состоянии вещества связи между молекулами наиболее сильны (в сравнении с жидким и газообразным состояниями).
        \item Любая частица (например, картошечка в супе) находится в броуновском движении, однако наблюдать его технически возможно только для малых частиц.
        \item Сжимаемость газов объясняется проникновением атомов молекул друг в друга и уменьшением межатомного расстояния внутри молекул.
        \item Броуновское движение частиц пыльцы в жидкости — следствие взаимодействия этих частиц пыльцы между собой.
        \item Если в двух телах одинаковое число молекул, то их массы с большой точностью будут равны.
        \item Если в двух телах одинаковое число протонов и нейтронов (между телами), то и массы тел с большой точностью окажутся равны.
        \item При определении размеров молекул мы зачастую пренебрегаем их формой, не различая радиус и диаметр, а то и вовсе считая их форму кубической.
        \item Диффузия вызвана тепловым движением молекул и может наблюдаться в твердых, жидких и газообразных веществах.
    \end{enumerate}
}
\answer{%
    $
        \text{да, да, нет, нет, нет, да, да, да}
    $
}

\tasknumber{6}%
\task{%
    Какое количество вещества содержит тело, состоящее из $9 \cdot 10^{24}$ молекул?
}
\answer{%
    $\nu = \frac N{N_A} = \frac{9 \cdot 10^{24}}{6{,}02 \cdot 10^{23}\,\frac{1}{\text{моль}}} = 14{,}95\,\text{моль}.$
}
\solutionspace{40pt}

\tasknumber{7}%
\task{%
    Какова масса $4\,\text{моль}$ (\ce{C2H6}) этана? Молярная масса этана $30\,\frac{\text{г}}{\text{моль}}$.
}
\answer{%
    $m = \mu\nu = 30\,\frac{\text{г}}{\text{моль}} \cdot 4\,\text{моль} = 120\,\text{г}.$
}
\solutionspace{40pt}

\tasknumber{8}%
\task{%
    Сколько молекул содержится в $200\,\text{г}$ декана? Молярная масса декана (\ce{C10H22}) $142\,\frac{\text{г}}{\text{моль}}$.
}
\answer{%
    $N = N_A\nu = N_A\frac{m}{\mu} = 6{,}02 \cdot 10^{23}\,\frac{1}{\text{моль}} \cdot \frac{200\,\text{г}}{142\,\frac{\text{г}}{\text{моль}}} = 850 \cdot 10^{21}.$
}

\variantsplitter

\addpersonalvariant{Егор Свистушкин}

\tasknumber{1}%
\task{%
    Молекулы газа в некотором сосуде движутся со средней скоростью $150\,\frac{\text{м}}{\text{с}}$.
    Определите, какое расстояние в среднем проходит одна из таких молекул за $4\,\text{сут}$.
}
\answer{%
    $s = v t = 150\,\frac{\text{м}}{\text{с}} \cdot 4\,\text{сут} = 52 \cdot 10^{6}\,\text{м}.$
}
\solutionspace{40pt}

\tasknumber{2}%
\task{%
    Напротив каждой физической величины укажите её обозначение и единицы измерения в СИ:
    \begin{enumerate}
        \item объём,
        \item количество вещества,
        \item количество молекул.
    \end{enumerate}
}

\tasknumber{3}%
\task{%
    Ответьте на вопросы и запишите формулы:
    \begin{enumerate}
        \item сформилируйте, что такое броуновское движение,
        \item cвязь количества вещества, числа частиц и числа Авогадро.
    \end{enumerate}
}
\solutionspace{60pt}

\tasknumber{4}%
\task{%
    Определите молярную массу веществ (не табличное значение, а вычислением по таблице Менделеева):
    \begin{enumerate}
        \item неон,
        \item кислород,
        \item вода.
    \end{enumerate}
}
\solutionspace{30pt}

\tasknumber{5}%
\task{%
    Укажите, верны ли утверждения («да» или «нет» слева от каждого утверждения):
    \begin{enumerate}
        \item В твёрдом состоянии вещества связи между молекулами наиболее сильны (в сравнении с жидким и газообразным состояниями).
        \item Любая частица (например, картошечка в супе) находится в броуновском движении, однако наблюдать его технически возможно только для малых частиц.
        \item Сжимаемость газов объясняется проникновением атомов молекул друг в друга и уменьшением межатомного расстояния внутри молекул.
        \item Броуновское движение частиц пыльцы в жидкости — следствие взаимодействия этих частиц пыльцы между собой.
        \item Если в двух телах одинаковое число молекул, то их массы с большой точностью будут равны.
        \item Если в двух телах одинаковое число протонов и нейтронов (между телами), то и массы тел с большой точностью окажутся равны.
        \item При определении размеров молекул мы зачастую пренебрегаем их формой, не различая радиус и диаметр, а то и вовсе считая их форму кубической.
        \item Диффузия вызвана тепловым движением молекул и может наблюдаться в твердых, жидких и газообразных веществах.
    \end{enumerate}
}
\answer{%
    $
        \text{да, да, нет, нет, нет, да, да, да}
    $
}

\tasknumber{6}%
\task{%
    Какое количество вещества содержит тело, состоящее из $9 \cdot 10^{25}$ молекул?
}
\answer{%
    $\nu = \frac N{N_A} = \frac{9 \cdot 10^{25}}{6{,}02 \cdot 10^{23}\,\frac{1}{\text{моль}}} = 149{,}50\,\text{моль}.$
}
\solutionspace{40pt}

\tasknumber{7}%
\task{%
    Какова масса $25\,\text{моль}$ (\ce{C8H18}) октана? Молярная масса октана $114\,\frac{\text{г}}{\text{моль}}$.
}
\answer{%
    $m = \mu\nu = 114\,\frac{\text{г}}{\text{моль}} \cdot 25\,\text{моль} = 2850\,\text{г}.$
}
\solutionspace{40pt}

\tasknumber{8}%
\task{%
    Сколько молекул содержится в $500\,\text{г}$ пропана? Молярная масса пропана (\ce{C3H8}) $44\,\frac{\text{г}}{\text{моль}}$.
}
\answer{%
    $N = N_A\nu = N_A\frac{m}{\mu} = 6{,}02 \cdot 10^{23}\,\frac{1}{\text{моль}} \cdot \frac{500\,\text{г}}{44\,\frac{\text{г}}{\text{моль}}} = 6{,}8 \cdot 10^{24}.$
}

\variantsplitter

\addpersonalvariant{Дмитрий Соколов}

\tasknumber{1}%
\task{%
    Молекулы газа в некотором сосуде движутся со средней скоростью $300\,\frac{\text{м}}{\text{с}}$.
    Определите, какое расстояние в среднем проходит одна из таких молекул за $2\,\text{сут}$.
}
\answer{%
    $s = v t = 300\,\frac{\text{м}}{\text{с}} \cdot 2\,\text{сут} = 52 \cdot 10^{6}\,\text{м}.$
}
\solutionspace{40pt}

\tasknumber{2}%
\task{%
    Напротив каждой физической величины укажите её обозначение и единицы измерения в СИ:
    \begin{enumerate}
        \item масса,
        \item количество вещества,
        \item количество молекул.
    \end{enumerate}
}

\tasknumber{3}%
\task{%
    Ответьте на вопросы и запишите формулы:
    \begin{enumerate}
        \item запишите 3 основных положения МКТ,
        \item cвязь количества вещества, числа частиц и числа Авогадро.
    \end{enumerate}
}
\solutionspace{60pt}

\tasknumber{4}%
\task{%
    Определите молярную массу веществ (не табличное значение, а вычислением по таблице Менделеева):
    \begin{enumerate}
        \item гелий,
        \item кислород,
        \item вода.
    \end{enumerate}
}
\solutionspace{30pt}

\tasknumber{5}%
\task{%
    Укажите, верны ли утверждения («да» или «нет» слева от каждого утверждения):
    \begin{enumerate}
        \item В твёрдом состоянии вещества связи между молекулами наиболее сильны (в сравнении с жидким и газообразным состояниями).
        \item Любая частица (например, картошечка в супе) находится в броуновском движении, однако наблюдать его технически возможно только для малых частиц.
        \item Сжимаемость газов объясняется проникновением атомов молекул друг в друга и уменьшением межатомного расстояния внутри молекул.
        \item Броуновское движение частиц пыльцы в жидкости — следствие взаимодействия этих частиц пыльцы между собой.
        \item Если в двух телах одинаковое число молекул, то их массы с большой точностью будут равны.
        \item Если в двух телах одинаковое число протонов и нейтронов (между телами), то и массы тел с большой точностью окажутся равны.
        \item При определении размеров молекул мы зачастую пренебрегаем их формой, не различая радиус и диаметр, а то и вовсе считая их форму кубической.
        \item Диффузия вызвана тепловым движением молекул и может наблюдаться в твердых, жидких и газообразных веществах.
    \end{enumerate}
}
\answer{%
    $
        \text{да, да, нет, нет, нет, да, да, да}
    $
}

\tasknumber{6}%
\task{%
    Какое количество вещества содержит тело, состоящее из $3 \cdot 10^{24}$ молекул?
}
\answer{%
    $\nu = \frac N{N_A} = \frac{3 \cdot 10^{24}}{6{,}02 \cdot 10^{23}\,\frac{1}{\text{моль}}} = 4{,}98\,\text{моль}.$
}
\solutionspace{40pt}

\tasknumber{7}%
\task{%
    Какова масса $50\,\text{моль}$ (\ce{CH4}) метана? Молярная масса метана $16\,\frac{\text{г}}{\text{моль}}$.
}
\answer{%
    $m = \mu\nu = 16\,\frac{\text{г}}{\text{моль}} \cdot 50\,\text{моль} = 800\,\text{г}.$
}
\solutionspace{40pt}

\tasknumber{8}%
\task{%
    Сколько молекул содержится в $50\,\text{г}$ этана? Молярная масса этана (\ce{C2H6}) $30\,\frac{\text{г}}{\text{моль}}$.
}
\answer{%
    $N = N_A\nu = N_A\frac{m}{\mu} = 6{,}02 \cdot 10^{23}\,\frac{1}{\text{моль}} \cdot \frac{50\,\text{г}}{30\,\frac{\text{г}}{\text{моль}}} = 1{,}00 \cdot 10^{24}.$
}

\variantsplitter

\addpersonalvariant{Арсений Трофимов}

\tasknumber{1}%
\task{%
    Молекулы газа в некотором сосуде движутся со средней скоростью $150\,\frac{\text{м}}{\text{с}}$.
    Определите, какое расстояние в среднем проходит одна из таких молекул за $4\,\text{час}$.
}
\answer{%
    $s = v t = 150\,\frac{\text{м}}{\text{с}} \cdot 4\,\text{час} = 2{,}2 \cdot 10^{6}\,\text{м}.$
}
\solutionspace{40pt}

\tasknumber{2}%
\task{%
    Напротив каждой физической величины укажите её обозначение и единицы измерения в СИ:
    \begin{enumerate}
        \item масса,
        \item количество вещества,
        \item молярная масса.
    \end{enumerate}
}

\tasknumber{3}%
\task{%
    Ответьте на вопросы и запишите формулы:
    \begin{enumerate}
        \item запишите 3 основных положения МКТ,
        \item cвязь количества вещества, числа частиц и числа Авогадро.
    \end{enumerate}
}
\solutionspace{60pt}

\tasknumber{4}%
\task{%
    Определите молярную массу веществ (не табличное значение, а вычислением по таблице Менделеева):
    \begin{enumerate}
        \item гелий,
        \item кислород,
        \item вода.
    \end{enumerate}
}
\solutionspace{30pt}

\tasknumber{5}%
\task{%
    Укажите, верны ли утверждения («да» или «нет» слева от каждого утверждения):
    \begin{enumerate}
        \item В твёрдом состоянии вещества связи между молекулами наиболее сильны (в сравнении с жидким и газообразным состояниями).
        \item Любая частица (например, картошечка в супе) находится в броуновском движении, однако наблюдать его технически возможно только для малых частиц.
        \item Сжимаемость газов объясняется проникновением атомов молекул друг в друга и уменьшением межатомного расстояния внутри молекул.
        \item Броуновское движение частиц пыльцы в жидкости — следствие взаимодействия этих частиц пыльцы между собой.
        \item Если в двух телах одинаковое число молекул, то их массы с большой точностью будут равны.
        \item Если в двух телах одинаковое число протонов и нейтронов (между телами), то и массы тел с большой точностью окажутся равны.
        \item При определении размеров молекул мы зачастую пренебрегаем их формой, не различая радиус и диаметр, а то и вовсе считая их форму кубической.
        \item Диффузия вызвана тепловым движением молекул и может наблюдаться в твердых, жидких и газообразных веществах.
    \end{enumerate}
}
\answer{%
    $
        \text{да, да, нет, нет, нет, да, да, да}
    $
}

\tasknumber{6}%
\task{%
    Какое количество вещества содержит тело, состоящее из $12 \cdot 10^{23}$ молекул?
}
\answer{%
    $\nu = \frac N{N_A} = \frac{12 \cdot 10^{23}}{6{,}02 \cdot 10^{23}\,\frac{1}{\text{моль}}} = 1{,}99\,\text{моль}.$
}
\solutionspace{40pt}

\tasknumber{7}%
\task{%
    Какова масса $20\,\text{моль}$ (\ce{C10H22}) декана? Молярная масса декана $142\,\frac{\text{г}}{\text{моль}}$.
}
\answer{%
    $m = \mu\nu = 142\,\frac{\text{г}}{\text{моль}} \cdot 20\,\text{моль} = 2840\,\text{г}.$
}
\solutionspace{40pt}

\tasknumber{8}%
\task{%
    Сколько молекул содержится в $50\,\text{г}$ октана? Молярная масса октана (\ce{C8H18}) $114\,\frac{\text{г}}{\text{моль}}$.
}
\answer{%
    $N = N_A\nu = N_A\frac{m}{\mu} = 6{,}02 \cdot 10^{23}\,\frac{1}{\text{моль}} \cdot \frac{50\,\text{г}}{114\,\frac{\text{г}}{\text{моль}}} = 260 \cdot 10^{21}.$
}
% autogenerated
