\setdate{13~мая~2021}
\setclass{10«АБ»}

\addpersonalvariant{Михаил Бурмистров}

\tasknumber{1}%
\task{%
    Определите ток $\eli_2$, протекающий через резистор $R_2$ (см.
    рис.),
    направление этого тока и разность потенциалов $U_2$ на этом резисторе,
    если $R_1 = 3\,\text{Ом}$, $R_2 = 8\,\text{Ом}$, $R_3 = 10\,\text{Ом}$, $\ele_1 = 5\,\text{В}$, $\ele_2 = 3\,\text{В}$, $\ele_3 = 2\,\text{В}$.
    Внутренним сопротивлением всех трёх ЭДС пренебречь.
    Ответы получите в виде несократимых дробей, а также определите приближённые значения.

    \begin{tikzpicture}[circuit ee IEC, thick]
        \foreach \contact/\x in {1/0, 2/3, 3/6}
        {
            \node [contact] (top contact \contact) at (\x, 0) {};
            \node [contact] (bottom contact \contact) at (\x, 4) {};
       }
        \draw  (bottom contact 1) -- (bottom contact 2) -- (bottom contact 3);
        \draw  (top contact 1) -- (top contact 2) -- (top contact 3);
        \draw  (bottom contact 1) to [resistor={near start, info=$R_1$}, battery={near end, info=$\ele_1$}] (top contact 1);
        \draw  (bottom contact 2) to [resistor={near start, info=$R_2$}, battery={near end, info=$\ele_2$}] (top contact 2);
        \draw  (bottom contact 3) to [resistor={near start, info=$R_3$}, battery={near end, info=$\ele_3$}] (top contact 3);
    \end{tikzpicture}
}
\answer{%
    План:
    \begin{itemize}
        \item отметим на рисунке произвольно направления токов (если получим отрицательный ответ, значит не угадали направление и только),
        \item выберем и обозначим на рисунке контуры (здесь всего 3, значит будет нужно $3-1=2$), для них запишем законы Кирхгофа,
        \item выберем и выделим на рисунке нетривиальные узлы (здесь всего 2, значит будет нужно $2-1=1$), для него запишем закон Кирхгофа,
        \item попытаемся решить получившуюся систему.
        В конкретном решении мы пытались первым делом найти $\eli_2$, но, возможно, в вашем варианте будет быстрее решать систему в другом порядке.
        Мы всё же проделаем всё в лоб, подробно и целиком.
    \end{itemize}


    \begin{tikzpicture}[circuit ee IEC, thick]
        \foreach \contact/\x in {1/0, 2/3, 3/6}
        {
            \node [contact] (top contact \contact) at (\x, 0) {};
            \node [contact] (bottom contact \contact) at (\x, 4) {};
       }
        \draw  (bottom contact 1) -- (bottom contact 2) -- (bottom contact 3);
        \draw  (top contact 1) -- (top contact 2) -- (top contact 3);
        \draw  (bottom contact 1) to [resistor={near start, info=$R_1$}, current direction'={midway, info=$\eli_1$}, battery={near end, info=$\ele_1$}] (top contact 1);
        \draw  (bottom contact 2) to [resistor={near start, info=$R_2$}, current direction'={midway, info=$\eli_2$}, battery={near end, info=$\ele_2$}] (top contact 2);
        \draw  (bottom contact 3) to [resistor={near start, info=$R_3$}, current direction'={midway, info=$\eli_3$}, battery={near end, info=$\ele_3$}] (top contact 3);
        \draw [-{Latex},color=red] (1.2, 2.5) arc [start angle = 135, end angle = -160, radius = 0.6];
        \draw [-{Latex},color=blue] (4.2, 2.5) arc [start angle = 135, end angle = -160, radius = 0.6];
        \node [contact,color=green!71!black] (bottomc) at (bottom contact 2) {};
    \end{tikzpicture}

    \begin{align*}
        &\begin{cases}
            {\color{red} \eli_1R_1 - \eli_2R_2 = \ele_1 - \ele_2}, \\
            {\color{blue} \eli_2R_2 - \eli_3R_3 = \ele_2 - \ele_3}, \\
            {\color{green!71!black} \eli_1 + \eli_2 + \eli_3 = 0};
        \end{cases}
        \qquad \implies \qquad
        \begin{cases}
            \eli_1 = \frac{\ele_1 - \ele_2 + \eli_2R_2}{R_1}, \\
            \eli_3 = \frac{\eli_2R_2 - \ele_2 + \ele_3}{R_3}, \\
            \eli_1 + \eli_2 + \eli_3 = 0, \\
        \end{cases} \implies \\
        \implies
            &\eli_2 + \frac{\ele_1 - \ele_2 + \eli_2R_2}{R_1} + \frac{\eli_2R_2 - \ele_2 + \ele_3}{R_3} = 0, \\
        &   \eli_2\cbr{1 + \frac{ R_2 }{ R_1 } + \frac{ R_2 }{ R_3 }} + \frac{\ele_1 - \ele_2}{ R_1 } + \frac{\ele_3 - \ele_2}{ R_3 } = 0, \\
        &   \eli_2 = \cfrac{\cfrac{\ele_2 - \ele_1}{ R_1 } + \cfrac{\ele_2 - \ele_3}{ R_3 }}{1 + \cfrac{ R_2 }{ R_1 } + \cfrac{ R_2 }{ R_3 }}
            = \cfrac{\cfrac{3\,\text{В} - 5\,\text{В}}{ 3\,\text{Ом} } + \cfrac{3\,\text{В} - 2\,\text{В}}{ 10\,\text{Ом} }}{1 + \cfrac{ 8\,\text{Ом} }{ 3\,\text{Ом} } + \cfrac{ 8\,\text{Ом} }{ 10\,\text{Ом} }}
            = -\frac{17}{134}\units{А} \approx -0{,}13000\,\text{А}, \\
        &   U_2 = \eli_2R_2 = \cfrac{\cfrac{\ele_2 - \ele_1}{ R_1 } + \cfrac{\ele_2 - \ele_3}{ R_3 }}{1 + \cfrac{ R_2 }{ R_1 } + \cfrac{ R_2 }{ R_3 }} \cdot R_2
            = \cfrac{\cfrac{3\,\text{В} - 5\,\text{В}}{ 3\,\text{Ом} } + \cfrac{3\,\text{В} - 2\,\text{В}}{ 10\,\text{Ом} }}{1 + \cfrac{ 8\,\text{Ом} }{ 3\,\text{Ом} } + \cfrac{ 8\,\text{Ом} }{ 10\,\text{Ом} }} \cdot 8\,\text{Ом}
            = -\frac{17}{134}\units{А} \cdot 8\,\text{Ом} = -\frac{68}{67}\units{В} \approx -1{,}0100\,\text{В}.
    \end{align*}

    Одну пару силы тока и напряжения получили.
    Для некоторых вариантов это уже ответ, но не у всех.
    Для упрощения записи преобразуем (чтобы избавитсья от 4-этажной дроби) и подставим в уже полученные уравнения:

    \begin{align*}
    \eli_2
        &=
        \frac{\frac{\ele_2 - \ele_1}{ R_1 } + \frac{\ele_2 - \ele_3}{ R_3 }}{1 + \frac{ R_2 }{ R_1 } + \frac{ R_2 }{ R_3 }}
        =
        \frac{(\ele_2 - \ele_1)R_3 + (\ele_2 - \ele_3)R_1}{R_1R_3 + R_2R_3 + R_2R_1},
        \\
    \eli_1
        &=  \frac{\ele_1 - \ele_2 + \eli_2R_2}{R_1}
        =   \frac{\ele_1 - \ele_2 + \cfrac{(\ele_2 - \ele_1)R_3 + (\ele_2 - \ele_3)R_1}{R_1R_3 + R_2R_3 + R_2R_1} \cdot R_2}{R_1} = \\
        &=  \frac{
            \ele_1R_1R_3 + \ele_1R_2R_3 + \ele_1R_2R_1
            - \ele_2R_1R_3 - \ele_2R_2R_3 - \ele_2R_2R_1
            + \ele_2R_3R_2 - \ele_1R_3R_2 + \ele_2R_1R_2 - \ele_3R_1R_2
       }{R_1 \cdot \cbr{R_1R_3 + R_2R_3 + R_2R_1}}
        = \\ &=
        \frac{
            \ele_1\cbr{R_1R_3 + R_2R_3 + R_2R_1 - R_3R_2}
            + \ele_2\cbr{- R_1R_3 - R_2R_3 - R_2R_1 + R_3R_2 + R_1R_2}
            - \ele_3R_1R_2
       }{R_1 \cdot \cbr{R_1R_3 + R_2R_3 + R_2R_1}}
        = \\ &=
        \frac{
            \ele_1\cbr{R_1R_3 + R_2R_1}
            + \ele_2\cbr{- R_1R_3}
            - \ele_3R_1R_2
       }{R_1 \cdot \cbr{R_1R_3 + R_2R_3 + R_2R_1}}
        =
        \frac{
            \ele_1\cbr{R_3 + R_2} - \ele_2R_3 - \ele_3R_2
       }{R_1R_3 + R_2R_3 + R_2R_1}
        = \\ &=
        \frac{
            (\ele_1 - \ele_3)R_2 + (\ele_1 - \ele_2)R_3
       }{R_1R_3 + R_2R_3 + R_2R_1}
        =
        \frac{
            \cfrac{\ele_1 - \ele_3}{ R_3 } + \cfrac{\ele_1 - \ele_2}{ R_2 }
       }{\cfrac{ R_1 }{ R_2 } + 1 + \cfrac{ R_1 }{ R_3 }}
        =
        \frac{
            \cfrac{5\,\text{В} - 2\,\text{В}}{ 10\,\text{Ом} } + \cfrac{5\,\text{В} - 3\,\text{В}}{ 8\,\text{Ом} }
       }{\cfrac{ 3\,\text{Ом} }{ 8\,\text{Ом} } + 1 + \cfrac{ 3\,\text{Ом} }{ 10\,\text{Ом} }}
        = \frac{22}{67}\units{А} \approx 0{,}33\,\text{А}.
        \\
    U_1
        &=
        \eli_1R_1
        =
        \frac{
            \cfrac{\ele_1 - \ele_3}{ R_3 } + \cfrac{\ele_1 - \ele_2}{ R_2 }
       }{\cfrac{ R_1 }{ R_2 } + 1 + \cfrac{ R_1 }{ R_3 }} \cdot R_1
        =
        \frac{22}{67}\units{А} \cdot 3\,\text{Ом} = \frac{66}{67}\units{В} \approx 0{,}99\,\text{В}.
    \end{align*}

    Если вы проделали все эти вычисления выше вместе со мной, то
    \begin{itemize}
        \item вы совершили ошибку, выбрав неверный путь решения:
        слишком длинное решение, очень легко ошибиться в индексах, дробях, знаках или потерять какой-то множитель,
        \item можно было выразить из исходной системы другие токи и получить сразу нажный вам,
        а не какой-то 2-й,
        \item можно было сэкономить: все три резистора и ЭДС соединены одинаково,
        поэтому ответ для 1-го резистора должен отличаться лишь перестановкой индексов (этот факт крайне полезен при проверке ответа, у нас всё сошлось),
        я специально подгонял выражение для $\eli_1$ к этому виду, вынося за скобки и преобразуя дробь,
        \item вы молодец, потому что не побоялись и получили верный ответ грамотным способом,
    \end{itemize}
    так что переходим к третьему резистору.
    Будет похоже, но кого это когда останавливало...

    \begin{align*}
    \eli_3
        &=  \frac{\eli_2R_2 - \ele_2 + \ele_3}{ R_3 }
        =
        \cfrac{
            \cfrac{
                (\ele_2 - \ele_1)R_3 + (\ele_2 - \ele_3)R_1
           }{
                R_1R_3 + R_2R_3 + R_2R_1
           } \cdot R_2 - \ele_2 + \ele_3}{ R_3 }
        = \\ &=
        \frac{
            \ele_2R_3R_2 - \ele_1R_3R_2 + \ele_2R_1R_2 - \ele_3R_1R_2
            - \ele_2R_1R_3 - \ele_2R_2R_3 - \ele_2R_2R_1
            + \ele_3R_1R_3 + \ele_3R_2R_3 + \ele_3R_2R_1
       }{\cbr{R_1R_3 + R_2R_3 + R_2R_1} \cdot R_3}
        = \\ &=
        \frac{
            - \ele_1R_3R_2 - \ele_2R_1R_3 + \ele_3R_1R_3 + \ele_3R_2R_3
       }{\cbr{R_1R_3 + R_2R_3 + R_2R_1} \cdot R_3}
        =
        \frac{
            - \ele_1R_2 - \ele_2R_1 + \ele_3R_1 + \ele_3R_2
       }{R_1R_3 + R_2R_3 + R_2R_1}
        = \\ &=
        \frac{
            R_1(\ele_3 - \ele_2) + R_2(\ele_3 - \ele_1)
       }{R_1R_3 + R_2R_3 + R_2R_1}
        =
        \frac{
            \cfrac{\ele_3 - \ele_2}{ R_2 } + \cfrac{\ele_3 - \ele_1}{ R_1 }
       }{\cfrac{ R_3 }{ R_2 } + \cfrac{ R_3 }{ R_1 } + 1}
        =
        \frac{
            \cfrac{2\,\text{В} - 3\,\text{В}}{ 8\,\text{Ом} } + \cfrac{2\,\text{В} - 5\,\text{В}}{ 3\,\text{Ом} }
       }{\cfrac{ 10\,\text{Ом} }{ 8\,\text{Ом} } + \cfrac{ 10\,\text{Ом} }{ 3\,\text{Ом} } + 1}
        = -\frac{27}{134}\units{А} \approx -0{,}2000\,\text{А}.
        \\
    U_3
        &=
        \eli_3R_3
        =
        \frac{
            \cfrac{\ele_3 - \ele_2}{ R_2 } + \cfrac{\ele_3 - \ele_1}{ R_1 }
       }{\cfrac{ R_3 }{ R_2 } + \cfrac{ R_3 }{ R_1 } + 1} \cdot R_3
        =
        -\frac{27}{134}\units{А} \cdot 10\,\text{Ом} = -\frac{135}{67}\units{В} \approx -2{,}010\,\text{В}.
    \end{align*}

    Положительные ответы говорят, что мы угадали на рисунке направление тока (тут нет нашей заслуги, повезло),
    отрицательные — что не угадали (и в этом нет ошибки), и ток течёт в противоположную сторону.
    Напомним, что направление тока — это направление движения положительных зарядов,
    а в металлах носители заряда — электроны, которые заряжены отрицательно.
}

\variantsplitter

\addpersonalvariant{Ирина Ан}

\tasknumber{1}%
\task{%
    Определите ток $\eli_3$, протекающий через резистор $R_3$ (см.
    рис.),
    направление этого тока и разность потенциалов $U_3$ на этом резисторе,
    если $R_1 = 4\,\text{Ом}$, $R_2 = 8\,\text{Ом}$, $R_3 = 15\,\text{Ом}$, $\ele_1 = 4\,\text{В}$, $\ele_2 = 6\,\text{В}$, $\ele_3 = 2\,\text{В}$.
    Внутренним сопротивлением всех трёх ЭДС пренебречь.
    Ответы получите в виде несократимых дробей, а также определите приближённые значения.

    \begin{tikzpicture}[circuit ee IEC, thick]
        \foreach \contact/\x in {1/0, 2/3, 3/6}
        {
            \node [contact] (top contact \contact) at (\x, 0) {};
            \node [contact] (bottom contact \contact) at (\x, 4) {};
       }
        \draw  (bottom contact 1) -- (bottom contact 2) -- (bottom contact 3);
        \draw  (top contact 1) -- (top contact 2) -- (top contact 3);
        \draw  (bottom contact 1) to [resistor={near start, info=$R_1$}, battery={near end, info=$\ele_1$}] (top contact 1);
        \draw  (bottom contact 2) to [resistor={near start, info=$R_2$}, battery={near end, info=$\ele_2$}] (top contact 2);
        \draw  (bottom contact 3) to [resistor={near start, info=$R_3$}, battery={near end, info=$\ele_3$}] (top contact 3);
    \end{tikzpicture}
}
\answer{%
    План:
    \begin{itemize}
        \item отметим на рисунке произвольно направления токов (если получим отрицательный ответ, значит не угадали направление и только),
        \item выберем и обозначим на рисунке контуры (здесь всего 3, значит будет нужно $3-1=2$), для них запишем законы Кирхгофа,
        \item выберем и выделим на рисунке нетривиальные узлы (здесь всего 2, значит будет нужно $2-1=1$), для него запишем закон Кирхгофа,
        \item попытаемся решить получившуюся систему.
        В конкретном решении мы пытались первым делом найти $\eli_2$, но, возможно, в вашем варианте будет быстрее решать систему в другом порядке.
        Мы всё же проделаем всё в лоб, подробно и целиком.
    \end{itemize}


    \begin{tikzpicture}[circuit ee IEC, thick]
        \foreach \contact/\x in {1/0, 2/3, 3/6}
        {
            \node [contact] (top contact \contact) at (\x, 0) {};
            \node [contact] (bottom contact \contact) at (\x, 4) {};
       }
        \draw  (bottom contact 1) -- (bottom contact 2) -- (bottom contact 3);
        \draw  (top contact 1) -- (top contact 2) -- (top contact 3);
        \draw  (bottom contact 1) to [resistor={near start, info=$R_1$}, current direction'={midway, info=$\eli_1$}, battery={near end, info=$\ele_1$}] (top contact 1);
        \draw  (bottom contact 2) to [resistor={near start, info=$R_2$}, current direction'={midway, info=$\eli_2$}, battery={near end, info=$\ele_2$}] (top contact 2);
        \draw  (bottom contact 3) to [resistor={near start, info=$R_3$}, current direction'={midway, info=$\eli_3$}, battery={near end, info=$\ele_3$}] (top contact 3);
        \draw [-{Latex},color=red] (1.2, 2.5) arc [start angle = 135, end angle = -160, radius = 0.6];
        \draw [-{Latex},color=blue] (4.2, 2.5) arc [start angle = 135, end angle = -160, radius = 0.6];
        \node [contact,color=green!71!black] (bottomc) at (bottom contact 2) {};
    \end{tikzpicture}

    \begin{align*}
        &\begin{cases}
            {\color{red} \eli_1R_1 - \eli_2R_2 = \ele_1 - \ele_2}, \\
            {\color{blue} \eli_2R_2 - \eli_3R_3 = \ele_2 - \ele_3}, \\
            {\color{green!71!black} \eli_1 + \eli_2 + \eli_3 = 0};
        \end{cases}
        \qquad \implies \qquad
        \begin{cases}
            \eli_1 = \frac{\ele_1 - \ele_2 + \eli_2R_2}{R_1}, \\
            \eli_3 = \frac{\eli_2R_2 - \ele_2 + \ele_3}{R_3}, \\
            \eli_1 + \eli_2 + \eli_3 = 0, \\
        \end{cases} \implies \\
        \implies
            &\eli_2 + \frac{\ele_1 - \ele_2 + \eli_2R_2}{R_1} + \frac{\eli_2R_2 - \ele_2 + \ele_3}{R_3} = 0, \\
        &   \eli_2\cbr{1 + \frac{ R_2 }{ R_1 } + \frac{ R_2 }{ R_3 }} + \frac{\ele_1 - \ele_2}{ R_1 } + \frac{\ele_3 - \ele_2}{ R_3 } = 0, \\
        &   \eli_2 = \cfrac{\cfrac{\ele_2 - \ele_1}{ R_1 } + \cfrac{\ele_2 - \ele_3}{ R_3 }}{1 + \cfrac{ R_2 }{ R_1 } + \cfrac{ R_2 }{ R_3 }}
            = \cfrac{\cfrac{6\,\text{В} - 4\,\text{В}}{ 4\,\text{Ом} } + \cfrac{6\,\text{В} - 2\,\text{В}}{ 15\,\text{Ом} }}{1 + \cfrac{ 8\,\text{Ом} }{ 4\,\text{Ом} } + \cfrac{ 8\,\text{Ом} }{ 15\,\text{Ом} }}
            = \frac{23}{106}\units{А} \approx 0{,}22\,\text{А}, \\
        &   U_2 = \eli_2R_2 = \cfrac{\cfrac{\ele_2 - \ele_1}{ R_1 } + \cfrac{\ele_2 - \ele_3}{ R_3 }}{1 + \cfrac{ R_2 }{ R_1 } + \cfrac{ R_2 }{ R_3 }} \cdot R_2
            = \cfrac{\cfrac{6\,\text{В} - 4\,\text{В}}{ 4\,\text{Ом} } + \cfrac{6\,\text{В} - 2\,\text{В}}{ 15\,\text{Ом} }}{1 + \cfrac{ 8\,\text{Ом} }{ 4\,\text{Ом} } + \cfrac{ 8\,\text{Ом} }{ 15\,\text{Ом} }} \cdot 8\,\text{Ом}
            = \frac{23}{106}\units{А} \cdot 8\,\text{Ом} = \frac{92}{53}\units{В} \approx 1{,}74\,\text{В}.
    \end{align*}

    Одну пару силы тока и напряжения получили.
    Для некоторых вариантов это уже ответ, но не у всех.
    Для упрощения записи преобразуем (чтобы избавитсья от 4-этажной дроби) и подставим в уже полученные уравнения:

    \begin{align*}
    \eli_2
        &=
        \frac{\frac{\ele_2 - \ele_1}{ R_1 } + \frac{\ele_2 - \ele_3}{ R_3 }}{1 + \frac{ R_2 }{ R_1 } + \frac{ R_2 }{ R_3 }}
        =
        \frac{(\ele_2 - \ele_1)R_3 + (\ele_2 - \ele_3)R_1}{R_1R_3 + R_2R_3 + R_2R_1},
        \\
    \eli_1
        &=  \frac{\ele_1 - \ele_2 + \eli_2R_2}{R_1}
        =   \frac{\ele_1 - \ele_2 + \cfrac{(\ele_2 - \ele_1)R_3 + (\ele_2 - \ele_3)R_1}{R_1R_3 + R_2R_3 + R_2R_1} \cdot R_2}{R_1} = \\
        &=  \frac{
            \ele_1R_1R_3 + \ele_1R_2R_3 + \ele_1R_2R_1
            - \ele_2R_1R_3 - \ele_2R_2R_3 - \ele_2R_2R_1
            + \ele_2R_3R_2 - \ele_1R_3R_2 + \ele_2R_1R_2 - \ele_3R_1R_2
       }{R_1 \cdot \cbr{R_1R_3 + R_2R_3 + R_2R_1}}
        = \\ &=
        \frac{
            \ele_1\cbr{R_1R_3 + R_2R_3 + R_2R_1 - R_3R_2}
            + \ele_2\cbr{- R_1R_3 - R_2R_3 - R_2R_1 + R_3R_2 + R_1R_2}
            - \ele_3R_1R_2
       }{R_1 \cdot \cbr{R_1R_3 + R_2R_3 + R_2R_1}}
        = \\ &=
        \frac{
            \ele_1\cbr{R_1R_3 + R_2R_1}
            + \ele_2\cbr{- R_1R_3}
            - \ele_3R_1R_2
       }{R_1 \cdot \cbr{R_1R_3 + R_2R_3 + R_2R_1}}
        =
        \frac{
            \ele_1\cbr{R_3 + R_2} - \ele_2R_3 - \ele_3R_2
       }{R_1R_3 + R_2R_3 + R_2R_1}
        = \\ &=
        \frac{
            (\ele_1 - \ele_3)R_2 + (\ele_1 - \ele_2)R_3
       }{R_1R_3 + R_2R_3 + R_2R_1}
        =
        \frac{
            \cfrac{\ele_1 - \ele_3}{ R_3 } + \cfrac{\ele_1 - \ele_2}{ R_2 }
       }{\cfrac{ R_1 }{ R_2 } + 1 + \cfrac{ R_1 }{ R_3 }}
        =
        \frac{
            \cfrac{4\,\text{В} - 2\,\text{В}}{ 15\,\text{Ом} } + \cfrac{4\,\text{В} - 6\,\text{В}}{ 8\,\text{Ом} }
       }{\cfrac{ 4\,\text{Ом} }{ 8\,\text{Ом} } + 1 + \cfrac{ 4\,\text{Ом} }{ 15\,\text{Ом} }}
        = -\frac7{106}\units{А} \approx -0{,}07000\,\text{А}.
        \\
    U_1
        &=
        \eli_1R_1
        =
        \frac{
            \cfrac{\ele_1 - \ele_3}{ R_3 } + \cfrac{\ele_1 - \ele_2}{ R_2 }
       }{\cfrac{ R_1 }{ R_2 } + 1 + \cfrac{ R_1 }{ R_3 }} \cdot R_1
        =
        -\frac7{106}\units{А} \cdot 4\,\text{Ом} = -\frac{14}{53}\units{В} \approx -0{,}2600\,\text{В}.
    \end{align*}

    Если вы проделали все эти вычисления выше вместе со мной, то
    \begin{itemize}
        \item вы совершили ошибку, выбрав неверный путь решения:
        слишком длинное решение, очень легко ошибиться в индексах, дробях, знаках или потерять какой-то множитель,
        \item можно было выразить из исходной системы другие токи и получить сразу нажный вам,
        а не какой-то 2-й,
        \item можно было сэкономить: все три резистора и ЭДС соединены одинаково,
        поэтому ответ для 1-го резистора должен отличаться лишь перестановкой индексов (этот факт крайне полезен при проверке ответа, у нас всё сошлось),
        я специально подгонял выражение для $\eli_1$ к этому виду, вынося за скобки и преобразуя дробь,
        \item вы молодец, потому что не побоялись и получили верный ответ грамотным способом,
    \end{itemize}
    так что переходим к третьему резистору.
    Будет похоже, но кого это когда останавливало...

    \begin{align*}
    \eli_3
        &=  \frac{\eli_2R_2 - \ele_2 + \ele_3}{ R_3 }
        =
        \cfrac{
            \cfrac{
                (\ele_2 - \ele_1)R_3 + (\ele_2 - \ele_3)R_1
           }{
                R_1R_3 + R_2R_3 + R_2R_1
           } \cdot R_2 - \ele_2 + \ele_3}{ R_3 }
        = \\ &=
        \frac{
            \ele_2R_3R_2 - \ele_1R_3R_2 + \ele_2R_1R_2 - \ele_3R_1R_2
            - \ele_2R_1R_3 - \ele_2R_2R_3 - \ele_2R_2R_1
            + \ele_3R_1R_3 + \ele_3R_2R_3 + \ele_3R_2R_1
       }{\cbr{R_1R_3 + R_2R_3 + R_2R_1} \cdot R_3}
        = \\ &=
        \frac{
            - \ele_1R_3R_2 - \ele_2R_1R_3 + \ele_3R_1R_3 + \ele_3R_2R_3
       }{\cbr{R_1R_3 + R_2R_3 + R_2R_1} \cdot R_3}
        =
        \frac{
            - \ele_1R_2 - \ele_2R_1 + \ele_3R_1 + \ele_3R_2
       }{R_1R_3 + R_2R_3 + R_2R_1}
        = \\ &=
        \frac{
            R_1(\ele_3 - \ele_2) + R_2(\ele_3 - \ele_1)
       }{R_1R_3 + R_2R_3 + R_2R_1}
        =
        \frac{
            \cfrac{\ele_3 - \ele_2}{ R_2 } + \cfrac{\ele_3 - \ele_1}{ R_1 }
       }{\cfrac{ R_3 }{ R_2 } + \cfrac{ R_3 }{ R_1 } + 1}
        =
        \frac{
            \cfrac{2\,\text{В} - 6\,\text{В}}{ 8\,\text{Ом} } + \cfrac{2\,\text{В} - 4\,\text{В}}{ 4\,\text{Ом} }
       }{\cfrac{ 15\,\text{Ом} }{ 8\,\text{Ом} } + \cfrac{ 15\,\text{Ом} }{ 4\,\text{Ом} } + 1}
        = -\frac8{53}\units{А} \approx -0{,}15000\,\text{А}.
        \\
    U_3
        &=
        \eli_3R_3
        =
        \frac{
            \cfrac{\ele_3 - \ele_2}{ R_2 } + \cfrac{\ele_3 - \ele_1}{ R_1 }
       }{\cfrac{ R_3 }{ R_2 } + \cfrac{ R_3 }{ R_1 } + 1} \cdot R_3
        =
        -\frac8{53}\units{А} \cdot 15\,\text{Ом} = -\frac{120}{53}\units{В} \approx -2{,}260\,\text{В}.
    \end{align*}

    Положительные ответы говорят, что мы угадали на рисунке направление тока (тут нет нашей заслуги, повезло),
    отрицательные — что не угадали (и в этом нет ошибки), и ток течёт в противоположную сторону.
    Напомним, что направление тока — это направление движения положительных зарядов,
    а в металлах носители заряда — электроны, которые заряжены отрицательно.
}

\variantsplitter

\addpersonalvariant{Софья Андрианова}

\tasknumber{1}%
\task{%
    Определите ток $\eli_1$, протекающий через резистор $R_1$ (см.
    рис.),
    направление этого тока и разность потенциалов $U_1$ на этом резисторе,
    если $R_1 = 3\,\text{Ом}$, $R_2 = 6\,\text{Ом}$, $R_3 = 15\,\text{Ом}$, $\ele_1 = 4\,\text{В}$, $\ele_2 = 6\,\text{В}$, $\ele_3 = 8\,\text{В}$.
    Внутренним сопротивлением всех трёх ЭДС пренебречь.
    Ответы получите в виде несократимых дробей, а также определите приближённые значения.

    \begin{tikzpicture}[circuit ee IEC, thick]
        \foreach \contact/\x in {1/0, 2/3, 3/6}
        {
            \node [contact] (top contact \contact) at (\x, 0) {};
            \node [contact] (bottom contact \contact) at (\x, 4) {};
       }
        \draw  (bottom contact 1) -- (bottom contact 2) -- (bottom contact 3);
        \draw  (top contact 1) -- (top contact 2) -- (top contact 3);
        \draw  (bottom contact 1) to [resistor={near start, info=$R_1$}, battery={near end, info=$\ele_1$}] (top contact 1);
        \draw  (bottom contact 2) to [resistor={near start, info=$R_2$}, battery={near end, info=$\ele_2$}] (top contact 2);
        \draw  (bottom contact 3) to [resistor={near start, info=$R_3$}, battery={near end, info=$\ele_3$}] (top contact 3);
    \end{tikzpicture}
}
\answer{%
    План:
    \begin{itemize}
        \item отметим на рисунке произвольно направления токов (если получим отрицательный ответ, значит не угадали направление и только),
        \item выберем и обозначим на рисунке контуры (здесь всего 3, значит будет нужно $3-1=2$), для них запишем законы Кирхгофа,
        \item выберем и выделим на рисунке нетривиальные узлы (здесь всего 2, значит будет нужно $2-1=1$), для него запишем закон Кирхгофа,
        \item попытаемся решить получившуюся систему.
        В конкретном решении мы пытались первым делом найти $\eli_2$, но, возможно, в вашем варианте будет быстрее решать систему в другом порядке.
        Мы всё же проделаем всё в лоб, подробно и целиком.
    \end{itemize}


    \begin{tikzpicture}[circuit ee IEC, thick]
        \foreach \contact/\x in {1/0, 2/3, 3/6}
        {
            \node [contact] (top contact \contact) at (\x, 0) {};
            \node [contact] (bottom contact \contact) at (\x, 4) {};
       }
        \draw  (bottom contact 1) -- (bottom contact 2) -- (bottom contact 3);
        \draw  (top contact 1) -- (top contact 2) -- (top contact 3);
        \draw  (bottom contact 1) to [resistor={near start, info=$R_1$}, current direction'={midway, info=$\eli_1$}, battery={near end, info=$\ele_1$}] (top contact 1);
        \draw  (bottom contact 2) to [resistor={near start, info=$R_2$}, current direction'={midway, info=$\eli_2$}, battery={near end, info=$\ele_2$}] (top contact 2);
        \draw  (bottom contact 3) to [resistor={near start, info=$R_3$}, current direction'={midway, info=$\eli_3$}, battery={near end, info=$\ele_3$}] (top contact 3);
        \draw [-{Latex},color=red] (1.2, 2.5) arc [start angle = 135, end angle = -160, radius = 0.6];
        \draw [-{Latex},color=blue] (4.2, 2.5) arc [start angle = 135, end angle = -160, radius = 0.6];
        \node [contact,color=green!71!black] (bottomc) at (bottom contact 2) {};
    \end{tikzpicture}

    \begin{align*}
        &\begin{cases}
            {\color{red} \eli_1R_1 - \eli_2R_2 = \ele_1 - \ele_2}, \\
            {\color{blue} \eli_2R_2 - \eli_3R_3 = \ele_2 - \ele_3}, \\
            {\color{green!71!black} \eli_1 + \eli_2 + \eli_3 = 0};
        \end{cases}
        \qquad \implies \qquad
        \begin{cases}
            \eli_1 = \frac{\ele_1 - \ele_2 + \eli_2R_2}{R_1}, \\
            \eli_3 = \frac{\eli_2R_2 - \ele_2 + \ele_3}{R_3}, \\
            \eli_1 + \eli_2 + \eli_3 = 0, \\
        \end{cases} \implies \\
        \implies
            &\eli_2 + \frac{\ele_1 - \ele_2 + \eli_2R_2}{R_1} + \frac{\eli_2R_2 - \ele_2 + \ele_3}{R_3} = 0, \\
        &   \eli_2\cbr{1 + \frac{ R_2 }{ R_1 } + \frac{ R_2 }{ R_3 }} + \frac{\ele_1 - \ele_2}{ R_1 } + \frac{\ele_3 - \ele_2}{ R_3 } = 0, \\
        &   \eli_2 = \cfrac{\cfrac{\ele_2 - \ele_1}{ R_1 } + \cfrac{\ele_2 - \ele_3}{ R_3 }}{1 + \cfrac{ R_2 }{ R_1 } + \cfrac{ R_2 }{ R_3 }}
            = \cfrac{\cfrac{6\,\text{В} - 4\,\text{В}}{ 3\,\text{Ом} } + \cfrac{6\,\text{В} - 8\,\text{В}}{ 15\,\text{Ом} }}{1 + \cfrac{ 6\,\text{Ом} }{ 3\,\text{Ом} } + \cfrac{ 6\,\text{Ом} }{ 15\,\text{Ом} }}
            = \frac8{51}\units{А} \approx 0{,}16\,\text{А}, \\
        &   U_2 = \eli_2R_2 = \cfrac{\cfrac{\ele_2 - \ele_1}{ R_1 } + \cfrac{\ele_2 - \ele_3}{ R_3 }}{1 + \cfrac{ R_2 }{ R_1 } + \cfrac{ R_2 }{ R_3 }} \cdot R_2
            = \cfrac{\cfrac{6\,\text{В} - 4\,\text{В}}{ 3\,\text{Ом} } + \cfrac{6\,\text{В} - 8\,\text{В}}{ 15\,\text{Ом} }}{1 + \cfrac{ 6\,\text{Ом} }{ 3\,\text{Ом} } + \cfrac{ 6\,\text{Ом} }{ 15\,\text{Ом} }} \cdot 6\,\text{Ом}
            = \frac8{51}\units{А} \cdot 6\,\text{Ом} = \frac{16}{17}\units{В} \approx 0{,}94\,\text{В}.
    \end{align*}

    Одну пару силы тока и напряжения получили.
    Для некоторых вариантов это уже ответ, но не у всех.
    Для упрощения записи преобразуем (чтобы избавитсья от 4-этажной дроби) и подставим в уже полученные уравнения:

    \begin{align*}
    \eli_2
        &=
        \frac{\frac{\ele_2 - \ele_1}{ R_1 } + \frac{\ele_2 - \ele_3}{ R_3 }}{1 + \frac{ R_2 }{ R_1 } + \frac{ R_2 }{ R_3 }}
        =
        \frac{(\ele_2 - \ele_1)R_3 + (\ele_2 - \ele_3)R_1}{R_1R_3 + R_2R_3 + R_2R_1},
        \\
    \eli_1
        &=  \frac{\ele_1 - \ele_2 + \eli_2R_2}{R_1}
        =   \frac{\ele_1 - \ele_2 + \cfrac{(\ele_2 - \ele_1)R_3 + (\ele_2 - \ele_3)R_1}{R_1R_3 + R_2R_3 + R_2R_1} \cdot R_2}{R_1} = \\
        &=  \frac{
            \ele_1R_1R_3 + \ele_1R_2R_3 + \ele_1R_2R_1
            - \ele_2R_1R_3 - \ele_2R_2R_3 - \ele_2R_2R_1
            + \ele_2R_3R_2 - \ele_1R_3R_2 + \ele_2R_1R_2 - \ele_3R_1R_2
       }{R_1 \cdot \cbr{R_1R_3 + R_2R_3 + R_2R_1}}
        = \\ &=
        \frac{
            \ele_1\cbr{R_1R_3 + R_2R_3 + R_2R_1 - R_3R_2}
            + \ele_2\cbr{- R_1R_3 - R_2R_3 - R_2R_1 + R_3R_2 + R_1R_2}
            - \ele_3R_1R_2
       }{R_1 \cdot \cbr{R_1R_3 + R_2R_3 + R_2R_1}}
        = \\ &=
        \frac{
            \ele_1\cbr{R_1R_3 + R_2R_1}
            + \ele_2\cbr{- R_1R_3}
            - \ele_3R_1R_2
       }{R_1 \cdot \cbr{R_1R_3 + R_2R_3 + R_2R_1}}
        =
        \frac{
            \ele_1\cbr{R_3 + R_2} - \ele_2R_3 - \ele_3R_2
       }{R_1R_3 + R_2R_3 + R_2R_1}
        = \\ &=
        \frac{
            (\ele_1 - \ele_3)R_2 + (\ele_1 - \ele_2)R_3
       }{R_1R_3 + R_2R_3 + R_2R_1}
        =
        \frac{
            \cfrac{\ele_1 - \ele_3}{ R_3 } + \cfrac{\ele_1 - \ele_2}{ R_2 }
       }{\cfrac{ R_1 }{ R_2 } + 1 + \cfrac{ R_1 }{ R_3 }}
        =
        \frac{
            \cfrac{4\,\text{В} - 8\,\text{В}}{ 15\,\text{Ом} } + \cfrac{4\,\text{В} - 6\,\text{В}}{ 6\,\text{Ом} }
       }{\cfrac{ 3\,\text{Ом} }{ 6\,\text{Ом} } + 1 + \cfrac{ 3\,\text{Ом} }{ 15\,\text{Ом} }}
        = -\frac6{17}\units{А} \approx -0{,}3500\,\text{А}.
        \\
    U_1
        &=
        \eli_1R_1
        =
        \frac{
            \cfrac{\ele_1 - \ele_3}{ R_3 } + \cfrac{\ele_1 - \ele_2}{ R_2 }
       }{\cfrac{ R_1 }{ R_2 } + 1 + \cfrac{ R_1 }{ R_3 }} \cdot R_1
        =
        -\frac6{17}\units{А} \cdot 3\,\text{Ом} = -\frac{18}{17}\units{В} \approx -1{,}0600\,\text{В}.
    \end{align*}

    Если вы проделали все эти вычисления выше вместе со мной, то
    \begin{itemize}
        \item вы совершили ошибку, выбрав неверный путь решения:
        слишком длинное решение, очень легко ошибиться в индексах, дробях, знаках или потерять какой-то множитель,
        \item можно было выразить из исходной системы другие токи и получить сразу нажный вам,
        а не какой-то 2-й,
        \item можно было сэкономить: все три резистора и ЭДС соединены одинаково,
        поэтому ответ для 1-го резистора должен отличаться лишь перестановкой индексов (этот факт крайне полезен при проверке ответа, у нас всё сошлось),
        я специально подгонял выражение для $\eli_1$ к этому виду, вынося за скобки и преобразуя дробь,
        \item вы молодец, потому что не побоялись и получили верный ответ грамотным способом,
    \end{itemize}
    так что переходим к третьему резистору.
    Будет похоже, но кого это когда останавливало...

    \begin{align*}
    \eli_3
        &=  \frac{\eli_2R_2 - \ele_2 + \ele_3}{ R_3 }
        =
        \cfrac{
            \cfrac{
                (\ele_2 - \ele_1)R_3 + (\ele_2 - \ele_3)R_1
           }{
                R_1R_3 + R_2R_3 + R_2R_1
           } \cdot R_2 - \ele_2 + \ele_3}{ R_3 }
        = \\ &=
        \frac{
            \ele_2R_3R_2 - \ele_1R_3R_2 + \ele_2R_1R_2 - \ele_3R_1R_2
            - \ele_2R_1R_3 - \ele_2R_2R_3 - \ele_2R_2R_1
            + \ele_3R_1R_3 + \ele_3R_2R_3 + \ele_3R_2R_1
       }{\cbr{R_1R_3 + R_2R_3 + R_2R_1} \cdot R_3}
        = \\ &=
        \frac{
            - \ele_1R_3R_2 - \ele_2R_1R_3 + \ele_3R_1R_3 + \ele_3R_2R_3
       }{\cbr{R_1R_3 + R_2R_3 + R_2R_1} \cdot R_3}
        =
        \frac{
            - \ele_1R_2 - \ele_2R_1 + \ele_3R_1 + \ele_3R_2
       }{R_1R_3 + R_2R_3 + R_2R_1}
        = \\ &=
        \frac{
            R_1(\ele_3 - \ele_2) + R_2(\ele_3 - \ele_1)
       }{R_1R_3 + R_2R_3 + R_2R_1}
        =
        \frac{
            \cfrac{\ele_3 - \ele_2}{ R_2 } + \cfrac{\ele_3 - \ele_1}{ R_1 }
       }{\cfrac{ R_3 }{ R_2 } + \cfrac{ R_3 }{ R_1 } + 1}
        =
        \frac{
            \cfrac{8\,\text{В} - 6\,\text{В}}{ 6\,\text{Ом} } + \cfrac{8\,\text{В} - 4\,\text{В}}{ 3\,\text{Ом} }
       }{\cfrac{ 15\,\text{Ом} }{ 6\,\text{Ом} } + \cfrac{ 15\,\text{Ом} }{ 3\,\text{Ом} } + 1}
        = \frac{10}{51}\units{А} \approx 0{,}20\,\text{А}.
        \\
    U_3
        &=
        \eli_3R_3
        =
        \frac{
            \cfrac{\ele_3 - \ele_2}{ R_2 } + \cfrac{\ele_3 - \ele_1}{ R_1 }
       }{\cfrac{ R_3 }{ R_2 } + \cfrac{ R_3 }{ R_1 } + 1} \cdot R_3
        =
        \frac{10}{51}\units{А} \cdot 15\,\text{Ом} = \frac{50}{17}\units{В} \approx 2{,}94\,\text{В}.
    \end{align*}

    Положительные ответы говорят, что мы угадали на рисунке направление тока (тут нет нашей заслуги, повезло),
    отрицательные — что не угадали (и в этом нет ошибки), и ток течёт в противоположную сторону.
    Напомним, что направление тока — это направление движения положительных зарядов,
    а в металлах носители заряда — электроны, которые заряжены отрицательно.
}

\variantsplitter

\addpersonalvariant{Владимир Артемчук}

\tasknumber{1}%
\task{%
    Определите ток $\eli_2$, протекающий через резистор $R_2$ (см.
    рис.),
    направление этого тока и разность потенциалов $U_2$ на этом резисторе,
    если $R_1 = 3\,\text{Ом}$, $R_2 = 5\,\text{Ом}$, $R_3 = 10\,\text{Ом}$, $\ele_1 = 4\,\text{В}$, $\ele_2 = 3\,\text{В}$, $\ele_3 = 2\,\text{В}$.
    Внутренним сопротивлением всех трёх ЭДС пренебречь.
    Ответы получите в виде несократимых дробей, а также определите приближённые значения.

    \begin{tikzpicture}[circuit ee IEC, thick]
        \foreach \contact/\x in {1/0, 2/3, 3/6}
        {
            \node [contact] (top contact \contact) at (\x, 0) {};
            \node [contact] (bottom contact \contact) at (\x, 4) {};
       }
        \draw  (bottom contact 1) -- (bottom contact 2) -- (bottom contact 3);
        \draw  (top contact 1) -- (top contact 2) -- (top contact 3);
        \draw  (bottom contact 1) to [resistor={near start, info=$R_1$}, battery={near end, info=$\ele_1$}] (top contact 1);
        \draw  (bottom contact 2) to [resistor={near start, info=$R_2$}, battery={near end, info=$\ele_2$}] (top contact 2);
        \draw  (bottom contact 3) to [resistor={near start, info=$R_3$}, battery={near end, info=$\ele_3$}] (top contact 3);
    \end{tikzpicture}
}
\answer{%
    План:
    \begin{itemize}
        \item отметим на рисунке произвольно направления токов (если получим отрицательный ответ, значит не угадали направление и только),
        \item выберем и обозначим на рисунке контуры (здесь всего 3, значит будет нужно $3-1=2$), для них запишем законы Кирхгофа,
        \item выберем и выделим на рисунке нетривиальные узлы (здесь всего 2, значит будет нужно $2-1=1$), для него запишем закон Кирхгофа,
        \item попытаемся решить получившуюся систему.
        В конкретном решении мы пытались первым делом найти $\eli_2$, но, возможно, в вашем варианте будет быстрее решать систему в другом порядке.
        Мы всё же проделаем всё в лоб, подробно и целиком.
    \end{itemize}


    \begin{tikzpicture}[circuit ee IEC, thick]
        \foreach \contact/\x in {1/0, 2/3, 3/6}
        {
            \node [contact] (top contact \contact) at (\x, 0) {};
            \node [contact] (bottom contact \contact) at (\x, 4) {};
       }
        \draw  (bottom contact 1) -- (bottom contact 2) -- (bottom contact 3);
        \draw  (top contact 1) -- (top contact 2) -- (top contact 3);
        \draw  (bottom contact 1) to [resistor={near start, info=$R_1$}, current direction'={midway, info=$\eli_1$}, battery={near end, info=$\ele_1$}] (top contact 1);
        \draw  (bottom contact 2) to [resistor={near start, info=$R_2$}, current direction'={midway, info=$\eli_2$}, battery={near end, info=$\ele_2$}] (top contact 2);
        \draw  (bottom contact 3) to [resistor={near start, info=$R_3$}, current direction'={midway, info=$\eli_3$}, battery={near end, info=$\ele_3$}] (top contact 3);
        \draw [-{Latex},color=red] (1.2, 2.5) arc [start angle = 135, end angle = -160, radius = 0.6];
        \draw [-{Latex},color=blue] (4.2, 2.5) arc [start angle = 135, end angle = -160, radius = 0.6];
        \node [contact,color=green!71!black] (bottomc) at (bottom contact 2) {};
    \end{tikzpicture}

    \begin{align*}
        &\begin{cases}
            {\color{red} \eli_1R_1 - \eli_2R_2 = \ele_1 - \ele_2}, \\
            {\color{blue} \eli_2R_2 - \eli_3R_3 = \ele_2 - \ele_3}, \\
            {\color{green!71!black} \eli_1 + \eli_2 + \eli_3 = 0};
        \end{cases}
        \qquad \implies \qquad
        \begin{cases}
            \eli_1 = \frac{\ele_1 - \ele_2 + \eli_2R_2}{R_1}, \\
            \eli_3 = \frac{\eli_2R_2 - \ele_2 + \ele_3}{R_3}, \\
            \eli_1 + \eli_2 + \eli_3 = 0, \\
        \end{cases} \implies \\
        \implies
            &\eli_2 + \frac{\ele_1 - \ele_2 + \eli_2R_2}{R_1} + \frac{\eli_2R_2 - \ele_2 + \ele_3}{R_3} = 0, \\
        &   \eli_2\cbr{1 + \frac{ R_2 }{ R_1 } + \frac{ R_2 }{ R_3 }} + \frac{\ele_1 - \ele_2}{ R_1 } + \frac{\ele_3 - \ele_2}{ R_3 } = 0, \\
        &   \eli_2 = \cfrac{\cfrac{\ele_2 - \ele_1}{ R_1 } + \cfrac{\ele_2 - \ele_3}{ R_3 }}{1 + \cfrac{ R_2 }{ R_1 } + \cfrac{ R_2 }{ R_3 }}
            = \cfrac{\cfrac{3\,\text{В} - 4\,\text{В}}{ 3\,\text{Ом} } + \cfrac{3\,\text{В} - 2\,\text{В}}{ 10\,\text{Ом} }}{1 + \cfrac{ 5\,\text{Ом} }{ 3\,\text{Ом} } + \cfrac{ 5\,\text{Ом} }{ 10\,\text{Ом} }}
            = -\frac7{95}\units{А} \approx -0{,}07000\,\text{А}, \\
        &   U_2 = \eli_2R_2 = \cfrac{\cfrac{\ele_2 - \ele_1}{ R_1 } + \cfrac{\ele_2 - \ele_3}{ R_3 }}{1 + \cfrac{ R_2 }{ R_1 } + \cfrac{ R_2 }{ R_3 }} \cdot R_2
            = \cfrac{\cfrac{3\,\text{В} - 4\,\text{В}}{ 3\,\text{Ом} } + \cfrac{3\,\text{В} - 2\,\text{В}}{ 10\,\text{Ом} }}{1 + \cfrac{ 5\,\text{Ом} }{ 3\,\text{Ом} } + \cfrac{ 5\,\text{Ом} }{ 10\,\text{Ом} }} \cdot 5\,\text{Ом}
            = -\frac7{95}\units{А} \cdot 5\,\text{Ом} = -\frac7{19}\units{В} \approx -0{,}3700\,\text{В}.
    \end{align*}

    Одну пару силы тока и напряжения получили.
    Для некоторых вариантов это уже ответ, но не у всех.
    Для упрощения записи преобразуем (чтобы избавитсья от 4-этажной дроби) и подставим в уже полученные уравнения:

    \begin{align*}
    \eli_2
        &=
        \frac{\frac{\ele_2 - \ele_1}{ R_1 } + \frac{\ele_2 - \ele_3}{ R_3 }}{1 + \frac{ R_2 }{ R_1 } + \frac{ R_2 }{ R_3 }}
        =
        \frac{(\ele_2 - \ele_1)R_3 + (\ele_2 - \ele_3)R_1}{R_1R_3 + R_2R_3 + R_2R_1},
        \\
    \eli_1
        &=  \frac{\ele_1 - \ele_2 + \eli_2R_2}{R_1}
        =   \frac{\ele_1 - \ele_2 + \cfrac{(\ele_2 - \ele_1)R_3 + (\ele_2 - \ele_3)R_1}{R_1R_3 + R_2R_3 + R_2R_1} \cdot R_2}{R_1} = \\
        &=  \frac{
            \ele_1R_1R_3 + \ele_1R_2R_3 + \ele_1R_2R_1
            - \ele_2R_1R_3 - \ele_2R_2R_3 - \ele_2R_2R_1
            + \ele_2R_3R_2 - \ele_1R_3R_2 + \ele_2R_1R_2 - \ele_3R_1R_2
       }{R_1 \cdot \cbr{R_1R_3 + R_2R_3 + R_2R_1}}
        = \\ &=
        \frac{
            \ele_1\cbr{R_1R_3 + R_2R_3 + R_2R_1 - R_3R_2}
            + \ele_2\cbr{- R_1R_3 - R_2R_3 - R_2R_1 + R_3R_2 + R_1R_2}
            - \ele_3R_1R_2
       }{R_1 \cdot \cbr{R_1R_3 + R_2R_3 + R_2R_1}}
        = \\ &=
        \frac{
            \ele_1\cbr{R_1R_3 + R_2R_1}
            + \ele_2\cbr{- R_1R_3}
            - \ele_3R_1R_2
       }{R_1 \cdot \cbr{R_1R_3 + R_2R_3 + R_2R_1}}
        =
        \frac{
            \ele_1\cbr{R_3 + R_2} - \ele_2R_3 - \ele_3R_2
       }{R_1R_3 + R_2R_3 + R_2R_1}
        = \\ &=
        \frac{
            (\ele_1 - \ele_3)R_2 + (\ele_1 - \ele_2)R_3
       }{R_1R_3 + R_2R_3 + R_2R_1}
        =
        \frac{
            \cfrac{\ele_1 - \ele_3}{ R_3 } + \cfrac{\ele_1 - \ele_2}{ R_2 }
       }{\cfrac{ R_1 }{ R_2 } + 1 + \cfrac{ R_1 }{ R_3 }}
        =
        \frac{
            \cfrac{4\,\text{В} - 2\,\text{В}}{ 10\,\text{Ом} } + \cfrac{4\,\text{В} - 3\,\text{В}}{ 5\,\text{Ом} }
       }{\cfrac{ 3\,\text{Ом} }{ 5\,\text{Ом} } + 1 + \cfrac{ 3\,\text{Ом} }{ 10\,\text{Ом} }}
        = \frac4{19}\units{А} \approx 0{,}21\,\text{А}.
        \\
    U_1
        &=
        \eli_1R_1
        =
        \frac{
            \cfrac{\ele_1 - \ele_3}{ R_3 } + \cfrac{\ele_1 - \ele_2}{ R_2 }
       }{\cfrac{ R_1 }{ R_2 } + 1 + \cfrac{ R_1 }{ R_3 }} \cdot R_1
        =
        \frac4{19}\units{А} \cdot 3\,\text{Ом} = \frac{12}{19}\units{В} \approx 0{,}63\,\text{В}.
    \end{align*}

    Если вы проделали все эти вычисления выше вместе со мной, то
    \begin{itemize}
        \item вы совершили ошибку, выбрав неверный путь решения:
        слишком длинное решение, очень легко ошибиться в индексах, дробях, знаках или потерять какой-то множитель,
        \item можно было выразить из исходной системы другие токи и получить сразу нажный вам,
        а не какой-то 2-й,
        \item можно было сэкономить: все три резистора и ЭДС соединены одинаково,
        поэтому ответ для 1-го резистора должен отличаться лишь перестановкой индексов (этот факт крайне полезен при проверке ответа, у нас всё сошлось),
        я специально подгонял выражение для $\eli_1$ к этому виду, вынося за скобки и преобразуя дробь,
        \item вы молодец, потому что не побоялись и получили верный ответ грамотным способом,
    \end{itemize}
    так что переходим к третьему резистору.
    Будет похоже, но кого это когда останавливало...

    \begin{align*}
    \eli_3
        &=  \frac{\eli_2R_2 - \ele_2 + \ele_3}{ R_3 }
        =
        \cfrac{
            \cfrac{
                (\ele_2 - \ele_1)R_3 + (\ele_2 - \ele_3)R_1
           }{
                R_1R_3 + R_2R_3 + R_2R_1
           } \cdot R_2 - \ele_2 + \ele_3}{ R_3 }
        = \\ &=
        \frac{
            \ele_2R_3R_2 - \ele_1R_3R_2 + \ele_2R_1R_2 - \ele_3R_1R_2
            - \ele_2R_1R_3 - \ele_2R_2R_3 - \ele_2R_2R_1
            + \ele_3R_1R_3 + \ele_3R_2R_3 + \ele_3R_2R_1
       }{\cbr{R_1R_3 + R_2R_3 + R_2R_1} \cdot R_3}
        = \\ &=
        \frac{
            - \ele_1R_3R_2 - \ele_2R_1R_3 + \ele_3R_1R_3 + \ele_3R_2R_3
       }{\cbr{R_1R_3 + R_2R_3 + R_2R_1} \cdot R_3}
        =
        \frac{
            - \ele_1R_2 - \ele_2R_1 + \ele_3R_1 + \ele_3R_2
       }{R_1R_3 + R_2R_3 + R_2R_1}
        = \\ &=
        \frac{
            R_1(\ele_3 - \ele_2) + R_2(\ele_3 - \ele_1)
       }{R_1R_3 + R_2R_3 + R_2R_1}
        =
        \frac{
            \cfrac{\ele_3 - \ele_2}{ R_2 } + \cfrac{\ele_3 - \ele_1}{ R_1 }
       }{\cfrac{ R_3 }{ R_2 } + \cfrac{ R_3 }{ R_1 } + 1}
        =
        \frac{
            \cfrac{2\,\text{В} - 3\,\text{В}}{ 5\,\text{Ом} } + \cfrac{2\,\text{В} - 4\,\text{В}}{ 3\,\text{Ом} }
       }{\cfrac{ 10\,\text{Ом} }{ 5\,\text{Ом} } + \cfrac{ 10\,\text{Ом} }{ 3\,\text{Ом} } + 1}
        = -\frac{13}{95}\units{А} \approx -0{,}14000\,\text{А}.
        \\
    U_3
        &=
        \eli_3R_3
        =
        \frac{
            \cfrac{\ele_3 - \ele_2}{ R_2 } + \cfrac{\ele_3 - \ele_1}{ R_1 }
       }{\cfrac{ R_3 }{ R_2 } + \cfrac{ R_3 }{ R_1 } + 1} \cdot R_3
        =
        -\frac{13}{95}\units{А} \cdot 10\,\text{Ом} = -\frac{26}{19}\units{В} \approx -1{,}3700\,\text{В}.
    \end{align*}

    Положительные ответы говорят, что мы угадали на рисунке направление тока (тут нет нашей заслуги, повезло),
    отрицательные — что не угадали (и в этом нет ошибки), и ток течёт в противоположную сторону.
    Напомним, что направление тока — это направление движения положительных зарядов,
    а в металлах носители заряда — электроны, которые заряжены отрицательно.
}

\variantsplitter

\addpersonalvariant{Софья Белянкина}

\tasknumber{1}%
\task{%
    Определите ток $\eli_3$, протекающий через резистор $R_3$ (см.
    рис.),
    направление этого тока и разность потенциалов $U_3$ на этом резисторе,
    если $R_1 = 4\,\text{Ом}$, $R_2 = 6\,\text{Ом}$, $R_3 = 15\,\text{Ом}$, $\ele_1 = 4\,\text{В}$, $\ele_2 = 3\,\text{В}$, $\ele_3 = 8\,\text{В}$.
    Внутренним сопротивлением всех трёх ЭДС пренебречь.
    Ответы получите в виде несократимых дробей, а также определите приближённые значения.

    \begin{tikzpicture}[circuit ee IEC, thick]
        \foreach \contact/\x in {1/0, 2/3, 3/6}
        {
            \node [contact] (top contact \contact) at (\x, 0) {};
            \node [contact] (bottom contact \contact) at (\x, 4) {};
       }
        \draw  (bottom contact 1) -- (bottom contact 2) -- (bottom contact 3);
        \draw  (top contact 1) -- (top contact 2) -- (top contact 3);
        \draw  (bottom contact 1) to [resistor={near start, info=$R_1$}, battery={near end, info=$\ele_1$}] (top contact 1);
        \draw  (bottom contact 2) to [resistor={near start, info=$R_2$}, battery={near end, info=$\ele_2$}] (top contact 2);
        \draw  (bottom contact 3) to [resistor={near start, info=$R_3$}, battery={near end, info=$\ele_3$}] (top contact 3);
    \end{tikzpicture}
}
\answer{%
    План:
    \begin{itemize}
        \item отметим на рисунке произвольно направления токов (если получим отрицательный ответ, значит не угадали направление и только),
        \item выберем и обозначим на рисунке контуры (здесь всего 3, значит будет нужно $3-1=2$), для них запишем законы Кирхгофа,
        \item выберем и выделим на рисунке нетривиальные узлы (здесь всего 2, значит будет нужно $2-1=1$), для него запишем закон Кирхгофа,
        \item попытаемся решить получившуюся систему.
        В конкретном решении мы пытались первым делом найти $\eli_2$, но, возможно, в вашем варианте будет быстрее решать систему в другом порядке.
        Мы всё же проделаем всё в лоб, подробно и целиком.
    \end{itemize}


    \begin{tikzpicture}[circuit ee IEC, thick]
        \foreach \contact/\x in {1/0, 2/3, 3/6}
        {
            \node [contact] (top contact \contact) at (\x, 0) {};
            \node [contact] (bottom contact \contact) at (\x, 4) {};
       }
        \draw  (bottom contact 1) -- (bottom contact 2) -- (bottom contact 3);
        \draw  (top contact 1) -- (top contact 2) -- (top contact 3);
        \draw  (bottom contact 1) to [resistor={near start, info=$R_1$}, current direction'={midway, info=$\eli_1$}, battery={near end, info=$\ele_1$}] (top contact 1);
        \draw  (bottom contact 2) to [resistor={near start, info=$R_2$}, current direction'={midway, info=$\eli_2$}, battery={near end, info=$\ele_2$}] (top contact 2);
        \draw  (bottom contact 3) to [resistor={near start, info=$R_3$}, current direction'={midway, info=$\eli_3$}, battery={near end, info=$\ele_3$}] (top contact 3);
        \draw [-{Latex},color=red] (1.2, 2.5) arc [start angle = 135, end angle = -160, radius = 0.6];
        \draw [-{Latex},color=blue] (4.2, 2.5) arc [start angle = 135, end angle = -160, radius = 0.6];
        \node [contact,color=green!71!black] (bottomc) at (bottom contact 2) {};
    \end{tikzpicture}

    \begin{align*}
        &\begin{cases}
            {\color{red} \eli_1R_1 - \eli_2R_2 = \ele_1 - \ele_2}, \\
            {\color{blue} \eli_2R_2 - \eli_3R_3 = \ele_2 - \ele_3}, \\
            {\color{green!71!black} \eli_1 + \eli_2 + \eli_3 = 0};
        \end{cases}
        \qquad \implies \qquad
        \begin{cases}
            \eli_1 = \frac{\ele_1 - \ele_2 + \eli_2R_2}{R_1}, \\
            \eli_3 = \frac{\eli_2R_2 - \ele_2 + \ele_3}{R_3}, \\
            \eli_1 + \eli_2 + \eli_3 = 0, \\
        \end{cases} \implies \\
        \implies
            &\eli_2 + \frac{\ele_1 - \ele_2 + \eli_2R_2}{R_1} + \frac{\eli_2R_2 - \ele_2 + \ele_3}{R_3} = 0, \\
        &   \eli_2\cbr{1 + \frac{ R_2 }{ R_1 } + \frac{ R_2 }{ R_3 }} + \frac{\ele_1 - \ele_2}{ R_1 } + \frac{\ele_3 - \ele_2}{ R_3 } = 0, \\
        &   \eli_2 = \cfrac{\cfrac{\ele_2 - \ele_1}{ R_1 } + \cfrac{\ele_2 - \ele_3}{ R_3 }}{1 + \cfrac{ R_2 }{ R_1 } + \cfrac{ R_2 }{ R_3 }}
            = \cfrac{\cfrac{3\,\text{В} - 4\,\text{В}}{ 4\,\text{Ом} } + \cfrac{3\,\text{В} - 8\,\text{В}}{ 15\,\text{Ом} }}{1 + \cfrac{ 6\,\text{Ом} }{ 4\,\text{Ом} } + \cfrac{ 6\,\text{Ом} }{ 15\,\text{Ом} }}
            = -\frac{35}{174}\units{А} \approx -0{,}2000\,\text{А}, \\
        &   U_2 = \eli_2R_2 = \cfrac{\cfrac{\ele_2 - \ele_1}{ R_1 } + \cfrac{\ele_2 - \ele_3}{ R_3 }}{1 + \cfrac{ R_2 }{ R_1 } + \cfrac{ R_2 }{ R_3 }} \cdot R_2
            = \cfrac{\cfrac{3\,\text{В} - 4\,\text{В}}{ 4\,\text{Ом} } + \cfrac{3\,\text{В} - 8\,\text{В}}{ 15\,\text{Ом} }}{1 + \cfrac{ 6\,\text{Ом} }{ 4\,\text{Ом} } + \cfrac{ 6\,\text{Ом} }{ 15\,\text{Ом} }} \cdot 6\,\text{Ом}
            = -\frac{35}{174}\units{А} \cdot 6\,\text{Ом} = -\frac{35}{29}\units{В} \approx -1{,}2100\,\text{В}.
    \end{align*}

    Одну пару силы тока и напряжения получили.
    Для некоторых вариантов это уже ответ, но не у всех.
    Для упрощения записи преобразуем (чтобы избавитсья от 4-этажной дроби) и подставим в уже полученные уравнения:

    \begin{align*}
    \eli_2
        &=
        \frac{\frac{\ele_2 - \ele_1}{ R_1 } + \frac{\ele_2 - \ele_3}{ R_3 }}{1 + \frac{ R_2 }{ R_1 } + \frac{ R_2 }{ R_3 }}
        =
        \frac{(\ele_2 - \ele_1)R_3 + (\ele_2 - \ele_3)R_1}{R_1R_3 + R_2R_3 + R_2R_1},
        \\
    \eli_1
        &=  \frac{\ele_1 - \ele_2 + \eli_2R_2}{R_1}
        =   \frac{\ele_1 - \ele_2 + \cfrac{(\ele_2 - \ele_1)R_3 + (\ele_2 - \ele_3)R_1}{R_1R_3 + R_2R_3 + R_2R_1} \cdot R_2}{R_1} = \\
        &=  \frac{
            \ele_1R_1R_3 + \ele_1R_2R_3 + \ele_1R_2R_1
            - \ele_2R_1R_3 - \ele_2R_2R_3 - \ele_2R_2R_1
            + \ele_2R_3R_2 - \ele_1R_3R_2 + \ele_2R_1R_2 - \ele_3R_1R_2
       }{R_1 \cdot \cbr{R_1R_3 + R_2R_3 + R_2R_1}}
        = \\ &=
        \frac{
            \ele_1\cbr{R_1R_3 + R_2R_3 + R_2R_1 - R_3R_2}
            + \ele_2\cbr{- R_1R_3 - R_2R_3 - R_2R_1 + R_3R_2 + R_1R_2}
            - \ele_3R_1R_2
       }{R_1 \cdot \cbr{R_1R_3 + R_2R_3 + R_2R_1}}
        = \\ &=
        \frac{
            \ele_1\cbr{R_1R_3 + R_2R_1}
            + \ele_2\cbr{- R_1R_3}
            - \ele_3R_1R_2
       }{R_1 \cdot \cbr{R_1R_3 + R_2R_3 + R_2R_1}}
        =
        \frac{
            \ele_1\cbr{R_3 + R_2} - \ele_2R_3 - \ele_3R_2
       }{R_1R_3 + R_2R_3 + R_2R_1}
        = \\ &=
        \frac{
            (\ele_1 - \ele_3)R_2 + (\ele_1 - \ele_2)R_3
       }{R_1R_3 + R_2R_3 + R_2R_1}
        =
        \frac{
            \cfrac{\ele_1 - \ele_3}{ R_3 } + \cfrac{\ele_1 - \ele_2}{ R_2 }
       }{\cfrac{ R_1 }{ R_2 } + 1 + \cfrac{ R_1 }{ R_3 }}
        =
        \frac{
            \cfrac{4\,\text{В} - 8\,\text{В}}{ 15\,\text{Ом} } + \cfrac{4\,\text{В} - 3\,\text{В}}{ 6\,\text{Ом} }
       }{\cfrac{ 4\,\text{Ом} }{ 6\,\text{Ом} } + 1 + \cfrac{ 4\,\text{Ом} }{ 15\,\text{Ом} }}
        = -\frac3{58}\units{А} \approx -0{,}05000\,\text{А}.
        \\
    U_1
        &=
        \eli_1R_1
        =
        \frac{
            \cfrac{\ele_1 - \ele_3}{ R_3 } + \cfrac{\ele_1 - \ele_2}{ R_2 }
       }{\cfrac{ R_1 }{ R_2 } + 1 + \cfrac{ R_1 }{ R_3 }} \cdot R_1
        =
        -\frac3{58}\units{А} \cdot 4\,\text{Ом} = -\frac6{29}\units{В} \approx -0{,}2100\,\text{В}.
    \end{align*}

    Если вы проделали все эти вычисления выше вместе со мной, то
    \begin{itemize}
        \item вы совершили ошибку, выбрав неверный путь решения:
        слишком длинное решение, очень легко ошибиться в индексах, дробях, знаках или потерять какой-то множитель,
        \item можно было выразить из исходной системы другие токи и получить сразу нажный вам,
        а не какой-то 2-й,
        \item можно было сэкономить: все три резистора и ЭДС соединены одинаково,
        поэтому ответ для 1-го резистора должен отличаться лишь перестановкой индексов (этот факт крайне полезен при проверке ответа, у нас всё сошлось),
        я специально подгонял выражение для $\eli_1$ к этому виду, вынося за скобки и преобразуя дробь,
        \item вы молодец, потому что не побоялись и получили верный ответ грамотным способом,
    \end{itemize}
    так что переходим к третьему резистору.
    Будет похоже, но кого это когда останавливало...

    \begin{align*}
    \eli_3
        &=  \frac{\eli_2R_2 - \ele_2 + \ele_3}{ R_3 }
        =
        \cfrac{
            \cfrac{
                (\ele_2 - \ele_1)R_3 + (\ele_2 - \ele_3)R_1
           }{
                R_1R_3 + R_2R_3 + R_2R_1
           } \cdot R_2 - \ele_2 + \ele_3}{ R_3 }
        = \\ &=
        \frac{
            \ele_2R_3R_2 - \ele_1R_3R_2 + \ele_2R_1R_2 - \ele_3R_1R_2
            - \ele_2R_1R_3 - \ele_2R_2R_3 - \ele_2R_2R_1
            + \ele_3R_1R_3 + \ele_3R_2R_3 + \ele_3R_2R_1
       }{\cbr{R_1R_3 + R_2R_3 + R_2R_1} \cdot R_3}
        = \\ &=
        \frac{
            - \ele_1R_3R_2 - \ele_2R_1R_3 + \ele_3R_1R_3 + \ele_3R_2R_3
       }{\cbr{R_1R_3 + R_2R_3 + R_2R_1} \cdot R_3}
        =
        \frac{
            - \ele_1R_2 - \ele_2R_1 + \ele_3R_1 + \ele_3R_2
       }{R_1R_3 + R_2R_3 + R_2R_1}
        = \\ &=
        \frac{
            R_1(\ele_3 - \ele_2) + R_2(\ele_3 - \ele_1)
       }{R_1R_3 + R_2R_3 + R_2R_1}
        =
        \frac{
            \cfrac{\ele_3 - \ele_2}{ R_2 } + \cfrac{\ele_3 - \ele_1}{ R_1 }
       }{\cfrac{ R_3 }{ R_2 } + \cfrac{ R_3 }{ R_1 } + 1}
        =
        \frac{
            \cfrac{8\,\text{В} - 3\,\text{В}}{ 6\,\text{Ом} } + \cfrac{8\,\text{В} - 4\,\text{В}}{ 4\,\text{Ом} }
       }{\cfrac{ 15\,\text{Ом} }{ 6\,\text{Ом} } + \cfrac{ 15\,\text{Ом} }{ 4\,\text{Ом} } + 1}
        = \frac{22}{87}\units{А} \approx 0{,}25\,\text{А}.
        \\
    U_3
        &=
        \eli_3R_3
        =
        \frac{
            \cfrac{\ele_3 - \ele_2}{ R_2 } + \cfrac{\ele_3 - \ele_1}{ R_1 }
       }{\cfrac{ R_3 }{ R_2 } + \cfrac{ R_3 }{ R_1 } + 1} \cdot R_3
        =
        \frac{22}{87}\units{А} \cdot 15\,\text{Ом} = \frac{110}{29}\units{В} \approx 3{,}79\,\text{В}.
    \end{align*}

    Положительные ответы говорят, что мы угадали на рисунке направление тока (тут нет нашей заслуги, повезло),
    отрицательные — что не угадали (и в этом нет ошибки), и ток течёт в противоположную сторону.
    Напомним, что направление тока — это направление движения положительных зарядов,
    а в металлах носители заряда — электроны, которые заряжены отрицательно.
}

\variantsplitter

\addpersonalvariant{Варвара Егиазарян}

\tasknumber{1}%
\task{%
    Определите ток $\eli_1$, протекающий через резистор $R_1$ (см.
    рис.),
    направление этого тока и разность потенциалов $U_1$ на этом резисторе,
    если $R_1 = 3\,\text{Ом}$, $R_2 = 5\,\text{Ом}$, $R_3 = 12\,\text{Ом}$, $\ele_1 = 5\,\text{В}$, $\ele_2 = 6\,\text{В}$, $\ele_3 = 2\,\text{В}$.
    Внутренним сопротивлением всех трёх ЭДС пренебречь.
    Ответы получите в виде несократимых дробей, а также определите приближённые значения.

    \begin{tikzpicture}[circuit ee IEC, thick]
        \foreach \contact/\x in {1/0, 2/3, 3/6}
        {
            \node [contact] (top contact \contact) at (\x, 0) {};
            \node [contact] (bottom contact \contact) at (\x, 4) {};
       }
        \draw  (bottom contact 1) -- (bottom contact 2) -- (bottom contact 3);
        \draw  (top contact 1) -- (top contact 2) -- (top contact 3);
        \draw  (bottom contact 1) to [resistor={near start, info=$R_1$}, battery={near end, info=$\ele_1$}] (top contact 1);
        \draw  (bottom contact 2) to [resistor={near start, info=$R_2$}, battery={near end, info=$\ele_2$}] (top contact 2);
        \draw  (bottom contact 3) to [resistor={near start, info=$R_3$}, battery={near end, info=$\ele_3$}] (top contact 3);
    \end{tikzpicture}
}
\answer{%
    План:
    \begin{itemize}
        \item отметим на рисунке произвольно направления токов (если получим отрицательный ответ, значит не угадали направление и только),
        \item выберем и обозначим на рисунке контуры (здесь всего 3, значит будет нужно $3-1=2$), для них запишем законы Кирхгофа,
        \item выберем и выделим на рисунке нетривиальные узлы (здесь всего 2, значит будет нужно $2-1=1$), для него запишем закон Кирхгофа,
        \item попытаемся решить получившуюся систему.
        В конкретном решении мы пытались первым делом найти $\eli_2$, но, возможно, в вашем варианте будет быстрее решать систему в другом порядке.
        Мы всё же проделаем всё в лоб, подробно и целиком.
    \end{itemize}


    \begin{tikzpicture}[circuit ee IEC, thick]
        \foreach \contact/\x in {1/0, 2/3, 3/6}
        {
            \node [contact] (top contact \contact) at (\x, 0) {};
            \node [contact] (bottom contact \contact) at (\x, 4) {};
       }
        \draw  (bottom contact 1) -- (bottom contact 2) -- (bottom contact 3);
        \draw  (top contact 1) -- (top contact 2) -- (top contact 3);
        \draw  (bottom contact 1) to [resistor={near start, info=$R_1$}, current direction'={midway, info=$\eli_1$}, battery={near end, info=$\ele_1$}] (top contact 1);
        \draw  (bottom contact 2) to [resistor={near start, info=$R_2$}, current direction'={midway, info=$\eli_2$}, battery={near end, info=$\ele_2$}] (top contact 2);
        \draw  (bottom contact 3) to [resistor={near start, info=$R_3$}, current direction'={midway, info=$\eli_3$}, battery={near end, info=$\ele_3$}] (top contact 3);
        \draw [-{Latex},color=red] (1.2, 2.5) arc [start angle = 135, end angle = -160, radius = 0.6];
        \draw [-{Latex},color=blue] (4.2, 2.5) arc [start angle = 135, end angle = -160, radius = 0.6];
        \node [contact,color=green!71!black] (bottomc) at (bottom contact 2) {};
    \end{tikzpicture}

    \begin{align*}
        &\begin{cases}
            {\color{red} \eli_1R_1 - \eli_2R_2 = \ele_1 - \ele_2}, \\
            {\color{blue} \eli_2R_2 - \eli_3R_3 = \ele_2 - \ele_3}, \\
            {\color{green!71!black} \eli_1 + \eli_2 + \eli_3 = 0};
        \end{cases}
        \qquad \implies \qquad
        \begin{cases}
            \eli_1 = \frac{\ele_1 - \ele_2 + \eli_2R_2}{R_1}, \\
            \eli_3 = \frac{\eli_2R_2 - \ele_2 + \ele_3}{R_3}, \\
            \eli_1 + \eli_2 + \eli_3 = 0, \\
        \end{cases} \implies \\
        \implies
            &\eli_2 + \frac{\ele_1 - \ele_2 + \eli_2R_2}{R_1} + \frac{\eli_2R_2 - \ele_2 + \ele_3}{R_3} = 0, \\
        &   \eli_2\cbr{1 + \frac{ R_2 }{ R_1 } + \frac{ R_2 }{ R_3 }} + \frac{\ele_1 - \ele_2}{ R_1 } + \frac{\ele_3 - \ele_2}{ R_3 } = 0, \\
        &   \eli_2 = \cfrac{\cfrac{\ele_2 - \ele_1}{ R_1 } + \cfrac{\ele_2 - \ele_3}{ R_3 }}{1 + \cfrac{ R_2 }{ R_1 } + \cfrac{ R_2 }{ R_3 }}
            = \cfrac{\cfrac{6\,\text{В} - 5\,\text{В}}{ 3\,\text{Ом} } + \cfrac{6\,\text{В} - 2\,\text{В}}{ 12\,\text{Ом} }}{1 + \cfrac{ 5\,\text{Ом} }{ 3\,\text{Ом} } + \cfrac{ 5\,\text{Ом} }{ 12\,\text{Ом} }}
            = \frac8{37}\units{А} \approx 0{,}22\,\text{А}, \\
        &   U_2 = \eli_2R_2 = \cfrac{\cfrac{\ele_2 - \ele_1}{ R_1 } + \cfrac{\ele_2 - \ele_3}{ R_3 }}{1 + \cfrac{ R_2 }{ R_1 } + \cfrac{ R_2 }{ R_3 }} \cdot R_2
            = \cfrac{\cfrac{6\,\text{В} - 5\,\text{В}}{ 3\,\text{Ом} } + \cfrac{6\,\text{В} - 2\,\text{В}}{ 12\,\text{Ом} }}{1 + \cfrac{ 5\,\text{Ом} }{ 3\,\text{Ом} } + \cfrac{ 5\,\text{Ом} }{ 12\,\text{Ом} }} \cdot 5\,\text{Ом}
            = \frac8{37}\units{А} \cdot 5\,\text{Ом} = \frac{40}{37}\units{В} \approx 1{,}08\,\text{В}.
    \end{align*}

    Одну пару силы тока и напряжения получили.
    Для некоторых вариантов это уже ответ, но не у всех.
    Для упрощения записи преобразуем (чтобы избавитсья от 4-этажной дроби) и подставим в уже полученные уравнения:

    \begin{align*}
    \eli_2
        &=
        \frac{\frac{\ele_2 - \ele_1}{ R_1 } + \frac{\ele_2 - \ele_3}{ R_3 }}{1 + \frac{ R_2 }{ R_1 } + \frac{ R_2 }{ R_3 }}
        =
        \frac{(\ele_2 - \ele_1)R_3 + (\ele_2 - \ele_3)R_1}{R_1R_3 + R_2R_3 + R_2R_1},
        \\
    \eli_1
        &=  \frac{\ele_1 - \ele_2 + \eli_2R_2}{R_1}
        =   \frac{\ele_1 - \ele_2 + \cfrac{(\ele_2 - \ele_1)R_3 + (\ele_2 - \ele_3)R_1}{R_1R_3 + R_2R_3 + R_2R_1} \cdot R_2}{R_1} = \\
        &=  \frac{
            \ele_1R_1R_3 + \ele_1R_2R_3 + \ele_1R_2R_1
            - \ele_2R_1R_3 - \ele_2R_2R_3 - \ele_2R_2R_1
            + \ele_2R_3R_2 - \ele_1R_3R_2 + \ele_2R_1R_2 - \ele_3R_1R_2
       }{R_1 \cdot \cbr{R_1R_3 + R_2R_3 + R_2R_1}}
        = \\ &=
        \frac{
            \ele_1\cbr{R_1R_3 + R_2R_3 + R_2R_1 - R_3R_2}
            + \ele_2\cbr{- R_1R_3 - R_2R_3 - R_2R_1 + R_3R_2 + R_1R_2}
            - \ele_3R_1R_2
       }{R_1 \cdot \cbr{R_1R_3 + R_2R_3 + R_2R_1}}
        = \\ &=
        \frac{
            \ele_1\cbr{R_1R_3 + R_2R_1}
            + \ele_2\cbr{- R_1R_3}
            - \ele_3R_1R_2
       }{R_1 \cdot \cbr{R_1R_3 + R_2R_3 + R_2R_1}}
        =
        \frac{
            \ele_1\cbr{R_3 + R_2} - \ele_2R_3 - \ele_3R_2
       }{R_1R_3 + R_2R_3 + R_2R_1}
        = \\ &=
        \frac{
            (\ele_1 - \ele_3)R_2 + (\ele_1 - \ele_2)R_3
       }{R_1R_3 + R_2R_3 + R_2R_1}
        =
        \frac{
            \cfrac{\ele_1 - \ele_3}{ R_3 } + \cfrac{\ele_1 - \ele_2}{ R_2 }
       }{\cfrac{ R_1 }{ R_2 } + 1 + \cfrac{ R_1 }{ R_3 }}
        =
        \frac{
            \cfrac{5\,\text{В} - 2\,\text{В}}{ 12\,\text{Ом} } + \cfrac{5\,\text{В} - 6\,\text{В}}{ 5\,\text{Ом} }
       }{\cfrac{ 3\,\text{Ом} }{ 5\,\text{Ом} } + 1 + \cfrac{ 3\,\text{Ом} }{ 12\,\text{Ом} }}
        = \frac1{37}\units{А} \approx 0{,}03\,\text{А}.
        \\
    U_1
        &=
        \eli_1R_1
        =
        \frac{
            \cfrac{\ele_1 - \ele_3}{ R_3 } + \cfrac{\ele_1 - \ele_2}{ R_2 }
       }{\cfrac{ R_1 }{ R_2 } + 1 + \cfrac{ R_1 }{ R_3 }} \cdot R_1
        =
        \frac1{37}\units{А} \cdot 3\,\text{Ом} = \frac3{37}\units{В} \approx 0{,}08\,\text{В}.
    \end{align*}

    Если вы проделали все эти вычисления выше вместе со мной, то
    \begin{itemize}
        \item вы совершили ошибку, выбрав неверный путь решения:
        слишком длинное решение, очень легко ошибиться в индексах, дробях, знаках или потерять какой-то множитель,
        \item можно было выразить из исходной системы другие токи и получить сразу нажный вам,
        а не какой-то 2-й,
        \item можно было сэкономить: все три резистора и ЭДС соединены одинаково,
        поэтому ответ для 1-го резистора должен отличаться лишь перестановкой индексов (этот факт крайне полезен при проверке ответа, у нас всё сошлось),
        я специально подгонял выражение для $\eli_1$ к этому виду, вынося за скобки и преобразуя дробь,
        \item вы молодец, потому что не побоялись и получили верный ответ грамотным способом,
    \end{itemize}
    так что переходим к третьему резистору.
    Будет похоже, но кого это когда останавливало...

    \begin{align*}
    \eli_3
        &=  \frac{\eli_2R_2 - \ele_2 + \ele_3}{ R_3 }
        =
        \cfrac{
            \cfrac{
                (\ele_2 - \ele_1)R_3 + (\ele_2 - \ele_3)R_1
           }{
                R_1R_3 + R_2R_3 + R_2R_1
           } \cdot R_2 - \ele_2 + \ele_3}{ R_3 }
        = \\ &=
        \frac{
            \ele_2R_3R_2 - \ele_1R_3R_2 + \ele_2R_1R_2 - \ele_3R_1R_2
            - \ele_2R_1R_3 - \ele_2R_2R_3 - \ele_2R_2R_1
            + \ele_3R_1R_3 + \ele_3R_2R_3 + \ele_3R_2R_1
       }{\cbr{R_1R_3 + R_2R_3 + R_2R_1} \cdot R_3}
        = \\ &=
        \frac{
            - \ele_1R_3R_2 - \ele_2R_1R_3 + \ele_3R_1R_3 + \ele_3R_2R_3
       }{\cbr{R_1R_3 + R_2R_3 + R_2R_1} \cdot R_3}
        =
        \frac{
            - \ele_1R_2 - \ele_2R_1 + \ele_3R_1 + \ele_3R_2
       }{R_1R_3 + R_2R_3 + R_2R_1}
        = \\ &=
        \frac{
            R_1(\ele_3 - \ele_2) + R_2(\ele_3 - \ele_1)
       }{R_1R_3 + R_2R_3 + R_2R_1}
        =
        \frac{
            \cfrac{\ele_3 - \ele_2}{ R_2 } + \cfrac{\ele_3 - \ele_1}{ R_1 }
       }{\cfrac{ R_3 }{ R_2 } + \cfrac{ R_3 }{ R_1 } + 1}
        =
        \frac{
            \cfrac{2\,\text{В} - 6\,\text{В}}{ 5\,\text{Ом} } + \cfrac{2\,\text{В} - 5\,\text{В}}{ 3\,\text{Ом} }
       }{\cfrac{ 12\,\text{Ом} }{ 5\,\text{Ом} } + \cfrac{ 12\,\text{Ом} }{ 3\,\text{Ом} } + 1}
        = -\frac9{37}\units{А} \approx -0{,}2400\,\text{А}.
        \\
    U_3
        &=
        \eli_3R_3
        =
        \frac{
            \cfrac{\ele_3 - \ele_2}{ R_2 } + \cfrac{\ele_3 - \ele_1}{ R_1 }
       }{\cfrac{ R_3 }{ R_2 } + \cfrac{ R_3 }{ R_1 } + 1} \cdot R_3
        =
        -\frac9{37}\units{А} \cdot 12\,\text{Ом} = -\frac{108}{37}\units{В} \approx -2{,}920\,\text{В}.
    \end{align*}

    Положительные ответы говорят, что мы угадали на рисунке направление тока (тут нет нашей заслуги, повезло),
    отрицательные — что не угадали (и в этом нет ошибки), и ток течёт в противоположную сторону.
    Напомним, что направление тока — это направление движения положительных зарядов,
    а в металлах носители заряда — электроны, которые заряжены отрицательно.
}

\variantsplitter

\addpersonalvariant{Владислав Емелин}

\tasknumber{1}%
\task{%
    Определите ток $\eli_3$, протекающий через резистор $R_3$ (см.
    рис.),
    направление этого тока и разность потенциалов $U_3$ на этом резисторе,
    если $R_1 = 3\,\text{Ом}$, $R_2 = 5\,\text{Ом}$, $R_3 = 15\,\text{Ом}$, $\ele_1 = 5\,\text{В}$, $\ele_2 = 3\,\text{В}$, $\ele_3 = 2\,\text{В}$.
    Внутренним сопротивлением всех трёх ЭДС пренебречь.
    Ответы получите в виде несократимых дробей, а также определите приближённые значения.

    \begin{tikzpicture}[circuit ee IEC, thick]
        \foreach \contact/\x in {1/0, 2/3, 3/6}
        {
            \node [contact] (top contact \contact) at (\x, 0) {};
            \node [contact] (bottom contact \contact) at (\x, 4) {};
       }
        \draw  (bottom contact 1) -- (bottom contact 2) -- (bottom contact 3);
        \draw  (top contact 1) -- (top contact 2) -- (top contact 3);
        \draw  (bottom contact 1) to [resistor={near start, info=$R_1$}, battery={near end, info=$\ele_1$}] (top contact 1);
        \draw  (bottom contact 2) to [resistor={near start, info=$R_2$}, battery={near end, info=$\ele_2$}] (top contact 2);
        \draw  (bottom contact 3) to [resistor={near start, info=$R_3$}, battery={near end, info=$\ele_3$}] (top contact 3);
    \end{tikzpicture}
}
\answer{%
    План:
    \begin{itemize}
        \item отметим на рисунке произвольно направления токов (если получим отрицательный ответ, значит не угадали направление и только),
        \item выберем и обозначим на рисунке контуры (здесь всего 3, значит будет нужно $3-1=2$), для них запишем законы Кирхгофа,
        \item выберем и выделим на рисунке нетривиальные узлы (здесь всего 2, значит будет нужно $2-1=1$), для него запишем закон Кирхгофа,
        \item попытаемся решить получившуюся систему.
        В конкретном решении мы пытались первым делом найти $\eli_2$, но, возможно, в вашем варианте будет быстрее решать систему в другом порядке.
        Мы всё же проделаем всё в лоб, подробно и целиком.
    \end{itemize}


    \begin{tikzpicture}[circuit ee IEC, thick]
        \foreach \contact/\x in {1/0, 2/3, 3/6}
        {
            \node [contact] (top contact \contact) at (\x, 0) {};
            \node [contact] (bottom contact \contact) at (\x, 4) {};
       }
        \draw  (bottom contact 1) -- (bottom contact 2) -- (bottom contact 3);
        \draw  (top contact 1) -- (top contact 2) -- (top contact 3);
        \draw  (bottom contact 1) to [resistor={near start, info=$R_1$}, current direction'={midway, info=$\eli_1$}, battery={near end, info=$\ele_1$}] (top contact 1);
        \draw  (bottom contact 2) to [resistor={near start, info=$R_2$}, current direction'={midway, info=$\eli_2$}, battery={near end, info=$\ele_2$}] (top contact 2);
        \draw  (bottom contact 3) to [resistor={near start, info=$R_3$}, current direction'={midway, info=$\eli_3$}, battery={near end, info=$\ele_3$}] (top contact 3);
        \draw [-{Latex},color=red] (1.2, 2.5) arc [start angle = 135, end angle = -160, radius = 0.6];
        \draw [-{Latex},color=blue] (4.2, 2.5) arc [start angle = 135, end angle = -160, radius = 0.6];
        \node [contact,color=green!71!black] (bottomc) at (bottom contact 2) {};
    \end{tikzpicture}

    \begin{align*}
        &\begin{cases}
            {\color{red} \eli_1R_1 - \eli_2R_2 = \ele_1 - \ele_2}, \\
            {\color{blue} \eli_2R_2 - \eli_3R_3 = \ele_2 - \ele_3}, \\
            {\color{green!71!black} \eli_1 + \eli_2 + \eli_3 = 0};
        \end{cases}
        \qquad \implies \qquad
        \begin{cases}
            \eli_1 = \frac{\ele_1 - \ele_2 + \eli_2R_2}{R_1}, \\
            \eli_3 = \frac{\eli_2R_2 - \ele_2 + \ele_3}{R_3}, \\
            \eli_1 + \eli_2 + \eli_3 = 0, \\
        \end{cases} \implies \\
        \implies
            &\eli_2 + \frac{\ele_1 - \ele_2 + \eli_2R_2}{R_1} + \frac{\eli_2R_2 - \ele_2 + \ele_3}{R_3} = 0, \\
        &   \eli_2\cbr{1 + \frac{ R_2 }{ R_1 } + \frac{ R_2 }{ R_3 }} + \frac{\ele_1 - \ele_2}{ R_1 } + \frac{\ele_3 - \ele_2}{ R_3 } = 0, \\
        &   \eli_2 = \cfrac{\cfrac{\ele_2 - \ele_1}{ R_1 } + \cfrac{\ele_2 - \ele_3}{ R_3 }}{1 + \cfrac{ R_2 }{ R_1 } + \cfrac{ R_2 }{ R_3 }}
            = \cfrac{\cfrac{3\,\text{В} - 5\,\text{В}}{ 3\,\text{Ом} } + \cfrac{3\,\text{В} - 2\,\text{В}}{ 15\,\text{Ом} }}{1 + \cfrac{ 5\,\text{Ом} }{ 3\,\text{Ом} } + \cfrac{ 5\,\text{Ом} }{ 15\,\text{Ом} }}
            = -\frac15\units{А} \approx -0{,}2000\,\text{А}, \\
        &   U_2 = \eli_2R_2 = \cfrac{\cfrac{\ele_2 - \ele_1}{ R_1 } + \cfrac{\ele_2 - \ele_3}{ R_3 }}{1 + \cfrac{ R_2 }{ R_1 } + \cfrac{ R_2 }{ R_3 }} \cdot R_2
            = \cfrac{\cfrac{3\,\text{В} - 5\,\text{В}}{ 3\,\text{Ом} } + \cfrac{3\,\text{В} - 2\,\text{В}}{ 15\,\text{Ом} }}{1 + \cfrac{ 5\,\text{Ом} }{ 3\,\text{Ом} } + \cfrac{ 5\,\text{Ом} }{ 15\,\text{Ом} }} \cdot 5\,\text{Ом}
            = -\frac15\units{А} \cdot 5\,\text{Ом} = -1\units{В} \approx -1{,}0000\,\text{В}.
    \end{align*}

    Одну пару силы тока и напряжения получили.
    Для некоторых вариантов это уже ответ, но не у всех.
    Для упрощения записи преобразуем (чтобы избавитсья от 4-этажной дроби) и подставим в уже полученные уравнения:

    \begin{align*}
    \eli_2
        &=
        \frac{\frac{\ele_2 - \ele_1}{ R_1 } + \frac{\ele_2 - \ele_3}{ R_3 }}{1 + \frac{ R_2 }{ R_1 } + \frac{ R_2 }{ R_3 }}
        =
        \frac{(\ele_2 - \ele_1)R_3 + (\ele_2 - \ele_3)R_1}{R_1R_3 + R_2R_3 + R_2R_1},
        \\
    \eli_1
        &=  \frac{\ele_1 - \ele_2 + \eli_2R_2}{R_1}
        =   \frac{\ele_1 - \ele_2 + \cfrac{(\ele_2 - \ele_1)R_3 + (\ele_2 - \ele_3)R_1}{R_1R_3 + R_2R_3 + R_2R_1} \cdot R_2}{R_1} = \\
        &=  \frac{
            \ele_1R_1R_3 + \ele_1R_2R_3 + \ele_1R_2R_1
            - \ele_2R_1R_3 - \ele_2R_2R_3 - \ele_2R_2R_1
            + \ele_2R_3R_2 - \ele_1R_3R_2 + \ele_2R_1R_2 - \ele_3R_1R_2
       }{R_1 \cdot \cbr{R_1R_3 + R_2R_3 + R_2R_1}}
        = \\ &=
        \frac{
            \ele_1\cbr{R_1R_3 + R_2R_3 + R_2R_1 - R_3R_2}
            + \ele_2\cbr{- R_1R_3 - R_2R_3 - R_2R_1 + R_3R_2 + R_1R_2}
            - \ele_3R_1R_2
       }{R_1 \cdot \cbr{R_1R_3 + R_2R_3 + R_2R_1}}
        = \\ &=
        \frac{
            \ele_1\cbr{R_1R_3 + R_2R_1}
            + \ele_2\cbr{- R_1R_3}
            - \ele_3R_1R_2
       }{R_1 \cdot \cbr{R_1R_3 + R_2R_3 + R_2R_1}}
        =
        \frac{
            \ele_1\cbr{R_3 + R_2} - \ele_2R_3 - \ele_3R_2
       }{R_1R_3 + R_2R_3 + R_2R_1}
        = \\ &=
        \frac{
            (\ele_1 - \ele_3)R_2 + (\ele_1 - \ele_2)R_3
       }{R_1R_3 + R_2R_3 + R_2R_1}
        =
        \frac{
            \cfrac{\ele_1 - \ele_3}{ R_3 } + \cfrac{\ele_1 - \ele_2}{ R_2 }
       }{\cfrac{ R_1 }{ R_2 } + 1 + \cfrac{ R_1 }{ R_3 }}
        =
        \frac{
            \cfrac{5\,\text{В} - 2\,\text{В}}{ 15\,\text{Ом} } + \cfrac{5\,\text{В} - 3\,\text{В}}{ 5\,\text{Ом} }
       }{\cfrac{ 3\,\text{Ом} }{ 5\,\text{Ом} } + 1 + \cfrac{ 3\,\text{Ом} }{ 15\,\text{Ом} }}
        = \frac13\units{А} \approx 0{,}33\,\text{А}.
        \\
    U_1
        &=
        \eli_1R_1
        =
        \frac{
            \cfrac{\ele_1 - \ele_3}{ R_3 } + \cfrac{\ele_1 - \ele_2}{ R_2 }
       }{\cfrac{ R_1 }{ R_2 } + 1 + \cfrac{ R_1 }{ R_3 }} \cdot R_1
        =
        \frac13\units{А} \cdot 3\,\text{Ом} = 1\units{В} \approx 1{,}00\,\text{В}.
    \end{align*}

    Если вы проделали все эти вычисления выше вместе со мной, то
    \begin{itemize}
        \item вы совершили ошибку, выбрав неверный путь решения:
        слишком длинное решение, очень легко ошибиться в индексах, дробях, знаках или потерять какой-то множитель,
        \item можно было выразить из исходной системы другие токи и получить сразу нажный вам,
        а не какой-то 2-й,
        \item можно было сэкономить: все три резистора и ЭДС соединены одинаково,
        поэтому ответ для 1-го резистора должен отличаться лишь перестановкой индексов (этот факт крайне полезен при проверке ответа, у нас всё сошлось),
        я специально подгонял выражение для $\eli_1$ к этому виду, вынося за скобки и преобразуя дробь,
        \item вы молодец, потому что не побоялись и получили верный ответ грамотным способом,
    \end{itemize}
    так что переходим к третьему резистору.
    Будет похоже, но кого это когда останавливало...

    \begin{align*}
    \eli_3
        &=  \frac{\eli_2R_2 - \ele_2 + \ele_3}{ R_3 }
        =
        \cfrac{
            \cfrac{
                (\ele_2 - \ele_1)R_3 + (\ele_2 - \ele_3)R_1
           }{
                R_1R_3 + R_2R_3 + R_2R_1
           } \cdot R_2 - \ele_2 + \ele_3}{ R_3 }
        = \\ &=
        \frac{
            \ele_2R_3R_2 - \ele_1R_3R_2 + \ele_2R_1R_2 - \ele_3R_1R_2
            - \ele_2R_1R_3 - \ele_2R_2R_3 - \ele_2R_2R_1
            + \ele_3R_1R_3 + \ele_3R_2R_3 + \ele_3R_2R_1
       }{\cbr{R_1R_3 + R_2R_3 + R_2R_1} \cdot R_3}
        = \\ &=
        \frac{
            - \ele_1R_3R_2 - \ele_2R_1R_3 + \ele_3R_1R_3 + \ele_3R_2R_3
       }{\cbr{R_1R_3 + R_2R_3 + R_2R_1} \cdot R_3}
        =
        \frac{
            - \ele_1R_2 - \ele_2R_1 + \ele_3R_1 + \ele_3R_2
       }{R_1R_3 + R_2R_3 + R_2R_1}
        = \\ &=
        \frac{
            R_1(\ele_3 - \ele_2) + R_2(\ele_3 - \ele_1)
       }{R_1R_3 + R_2R_3 + R_2R_1}
        =
        \frac{
            \cfrac{\ele_3 - \ele_2}{ R_2 } + \cfrac{\ele_3 - \ele_1}{ R_1 }
       }{\cfrac{ R_3 }{ R_2 } + \cfrac{ R_3 }{ R_1 } + 1}
        =
        \frac{
            \cfrac{2\,\text{В} - 3\,\text{В}}{ 5\,\text{Ом} } + \cfrac{2\,\text{В} - 5\,\text{В}}{ 3\,\text{Ом} }
       }{\cfrac{ 15\,\text{Ом} }{ 5\,\text{Ом} } + \cfrac{ 15\,\text{Ом} }{ 3\,\text{Ом} } + 1}
        = -\frac2{15}\units{А} \approx -0{,}13000\,\text{А}.
        \\
    U_3
        &=
        \eli_3R_3
        =
        \frac{
            \cfrac{\ele_3 - \ele_2}{ R_2 } + \cfrac{\ele_3 - \ele_1}{ R_1 }
       }{\cfrac{ R_3 }{ R_2 } + \cfrac{ R_3 }{ R_1 } + 1} \cdot R_3
        =
        -\frac2{15}\units{А} \cdot 15\,\text{Ом} = -2\units{В} \approx -2{,}000\,\text{В}.
    \end{align*}

    Положительные ответы говорят, что мы угадали на рисунке направление тока (тут нет нашей заслуги, повезло),
    отрицательные — что не угадали (и в этом нет ошибки), и ток течёт в противоположную сторону.
    Напомним, что направление тока — это направление движения положительных зарядов,
    а в металлах носители заряда — электроны, которые заряжены отрицательно.
}

\variantsplitter

\addpersonalvariant{Артём Жичин}

\tasknumber{1}%
\task{%
    Определите ток $\eli_2$, протекающий через резистор $R_2$ (см.
    рис.),
    направление этого тока и разность потенциалов $U_2$ на этом резисторе,
    если $R_1 = 4\,\text{Ом}$, $R_2 = 6\,\text{Ом}$, $R_3 = 10\,\text{Ом}$, $\ele_1 = 4\,\text{В}$, $\ele_2 = 6\,\text{В}$, $\ele_3 = 2\,\text{В}$.
    Внутренним сопротивлением всех трёх ЭДС пренебречь.
    Ответы получите в виде несократимых дробей, а также определите приближённые значения.

    \begin{tikzpicture}[circuit ee IEC, thick]
        \foreach \contact/\x in {1/0, 2/3, 3/6}
        {
            \node [contact] (top contact \contact) at (\x, 0) {};
            \node [contact] (bottom contact \contact) at (\x, 4) {};
       }
        \draw  (bottom contact 1) -- (bottom contact 2) -- (bottom contact 3);
        \draw  (top contact 1) -- (top contact 2) -- (top contact 3);
        \draw  (bottom contact 1) to [resistor={near start, info=$R_1$}, battery={near end, info=$\ele_1$}] (top contact 1);
        \draw  (bottom contact 2) to [resistor={near start, info=$R_2$}, battery={near end, info=$\ele_2$}] (top contact 2);
        \draw  (bottom contact 3) to [resistor={near start, info=$R_3$}, battery={near end, info=$\ele_3$}] (top contact 3);
    \end{tikzpicture}
}
\answer{%
    План:
    \begin{itemize}
        \item отметим на рисунке произвольно направления токов (если получим отрицательный ответ, значит не угадали направление и только),
        \item выберем и обозначим на рисунке контуры (здесь всего 3, значит будет нужно $3-1=2$), для них запишем законы Кирхгофа,
        \item выберем и выделим на рисунке нетривиальные узлы (здесь всего 2, значит будет нужно $2-1=1$), для него запишем закон Кирхгофа,
        \item попытаемся решить получившуюся систему.
        В конкретном решении мы пытались первым делом найти $\eli_2$, но, возможно, в вашем варианте будет быстрее решать систему в другом порядке.
        Мы всё же проделаем всё в лоб, подробно и целиком.
    \end{itemize}


    \begin{tikzpicture}[circuit ee IEC, thick]
        \foreach \contact/\x in {1/0, 2/3, 3/6}
        {
            \node [contact] (top contact \contact) at (\x, 0) {};
            \node [contact] (bottom contact \contact) at (\x, 4) {};
       }
        \draw  (bottom contact 1) -- (bottom contact 2) -- (bottom contact 3);
        \draw  (top contact 1) -- (top contact 2) -- (top contact 3);
        \draw  (bottom contact 1) to [resistor={near start, info=$R_1$}, current direction'={midway, info=$\eli_1$}, battery={near end, info=$\ele_1$}] (top contact 1);
        \draw  (bottom contact 2) to [resistor={near start, info=$R_2$}, current direction'={midway, info=$\eli_2$}, battery={near end, info=$\ele_2$}] (top contact 2);
        \draw  (bottom contact 3) to [resistor={near start, info=$R_3$}, current direction'={midway, info=$\eli_3$}, battery={near end, info=$\ele_3$}] (top contact 3);
        \draw [-{Latex},color=red] (1.2, 2.5) arc [start angle = 135, end angle = -160, radius = 0.6];
        \draw [-{Latex},color=blue] (4.2, 2.5) arc [start angle = 135, end angle = -160, radius = 0.6];
        \node [contact,color=green!71!black] (bottomc) at (bottom contact 2) {};
    \end{tikzpicture}

    \begin{align*}
        &\begin{cases}
            {\color{red} \eli_1R_1 - \eli_2R_2 = \ele_1 - \ele_2}, \\
            {\color{blue} \eli_2R_2 - \eli_3R_3 = \ele_2 - \ele_3}, \\
            {\color{green!71!black} \eli_1 + \eli_2 + \eli_3 = 0};
        \end{cases}
        \qquad \implies \qquad
        \begin{cases}
            \eli_1 = \frac{\ele_1 - \ele_2 + \eli_2R_2}{R_1}, \\
            \eli_3 = \frac{\eli_2R_2 - \ele_2 + \ele_3}{R_3}, \\
            \eli_1 + \eli_2 + \eli_3 = 0, \\
        \end{cases} \implies \\
        \implies
            &\eli_2 + \frac{\ele_1 - \ele_2 + \eli_2R_2}{R_1} + \frac{\eli_2R_2 - \ele_2 + \ele_3}{R_3} = 0, \\
        &   \eli_2\cbr{1 + \frac{ R_2 }{ R_1 } + \frac{ R_2 }{ R_3 }} + \frac{\ele_1 - \ele_2}{ R_1 } + \frac{\ele_3 - \ele_2}{ R_3 } = 0, \\
        &   \eli_2 = \cfrac{\cfrac{\ele_2 - \ele_1}{ R_1 } + \cfrac{\ele_2 - \ele_3}{ R_3 }}{1 + \cfrac{ R_2 }{ R_1 } + \cfrac{ R_2 }{ R_3 }}
            = \cfrac{\cfrac{6\,\text{В} - 4\,\text{В}}{ 4\,\text{Ом} } + \cfrac{6\,\text{В} - 2\,\text{В}}{ 10\,\text{Ом} }}{1 + \cfrac{ 6\,\text{Ом} }{ 4\,\text{Ом} } + \cfrac{ 6\,\text{Ом} }{ 10\,\text{Ом} }}
            = \frac9{31}\units{А} \approx 0{,}29\,\text{А}, \\
        &   U_2 = \eli_2R_2 = \cfrac{\cfrac{\ele_2 - \ele_1}{ R_1 } + \cfrac{\ele_2 - \ele_3}{ R_3 }}{1 + \cfrac{ R_2 }{ R_1 } + \cfrac{ R_2 }{ R_3 }} \cdot R_2
            = \cfrac{\cfrac{6\,\text{В} - 4\,\text{В}}{ 4\,\text{Ом} } + \cfrac{6\,\text{В} - 2\,\text{В}}{ 10\,\text{Ом} }}{1 + \cfrac{ 6\,\text{Ом} }{ 4\,\text{Ом} } + \cfrac{ 6\,\text{Ом} }{ 10\,\text{Ом} }} \cdot 6\,\text{Ом}
            = \frac9{31}\units{А} \cdot 6\,\text{Ом} = \frac{54}{31}\units{В} \approx 1{,}74\,\text{В}.
    \end{align*}

    Одну пару силы тока и напряжения получили.
    Для некоторых вариантов это уже ответ, но не у всех.
    Для упрощения записи преобразуем (чтобы избавитсья от 4-этажной дроби) и подставим в уже полученные уравнения:

    \begin{align*}
    \eli_2
        &=
        \frac{\frac{\ele_2 - \ele_1}{ R_1 } + \frac{\ele_2 - \ele_3}{ R_3 }}{1 + \frac{ R_2 }{ R_1 } + \frac{ R_2 }{ R_3 }}
        =
        \frac{(\ele_2 - \ele_1)R_3 + (\ele_2 - \ele_3)R_1}{R_1R_3 + R_2R_3 + R_2R_1},
        \\
    \eli_1
        &=  \frac{\ele_1 - \ele_2 + \eli_2R_2}{R_1}
        =   \frac{\ele_1 - \ele_2 + \cfrac{(\ele_2 - \ele_1)R_3 + (\ele_2 - \ele_3)R_1}{R_1R_3 + R_2R_3 + R_2R_1} \cdot R_2}{R_1} = \\
        &=  \frac{
            \ele_1R_1R_3 + \ele_1R_2R_3 + \ele_1R_2R_1
            - \ele_2R_1R_3 - \ele_2R_2R_3 - \ele_2R_2R_1
            + \ele_2R_3R_2 - \ele_1R_3R_2 + \ele_2R_1R_2 - \ele_3R_1R_2
       }{R_1 \cdot \cbr{R_1R_3 + R_2R_3 + R_2R_1}}
        = \\ &=
        \frac{
            \ele_1\cbr{R_1R_3 + R_2R_3 + R_2R_1 - R_3R_2}
            + \ele_2\cbr{- R_1R_3 - R_2R_3 - R_2R_1 + R_3R_2 + R_1R_2}
            - \ele_3R_1R_2
       }{R_1 \cdot \cbr{R_1R_3 + R_2R_3 + R_2R_1}}
        = \\ &=
        \frac{
            \ele_1\cbr{R_1R_3 + R_2R_1}
            + \ele_2\cbr{- R_1R_3}
            - \ele_3R_1R_2
       }{R_1 \cdot \cbr{R_1R_3 + R_2R_3 + R_2R_1}}
        =
        \frac{
            \ele_1\cbr{R_3 + R_2} - \ele_2R_3 - \ele_3R_2
       }{R_1R_3 + R_2R_3 + R_2R_1}
        = \\ &=
        \frac{
            (\ele_1 - \ele_3)R_2 + (\ele_1 - \ele_2)R_3
       }{R_1R_3 + R_2R_3 + R_2R_1}
        =
        \frac{
            \cfrac{\ele_1 - \ele_3}{ R_3 } + \cfrac{\ele_1 - \ele_2}{ R_2 }
       }{\cfrac{ R_1 }{ R_2 } + 1 + \cfrac{ R_1 }{ R_3 }}
        =
        \frac{
            \cfrac{4\,\text{В} - 2\,\text{В}}{ 10\,\text{Ом} } + \cfrac{4\,\text{В} - 6\,\text{В}}{ 6\,\text{Ом} }
       }{\cfrac{ 4\,\text{Ом} }{ 6\,\text{Ом} } + 1 + \cfrac{ 4\,\text{Ом} }{ 10\,\text{Ом} }}
        = -\frac2{31}\units{А} \approx -0{,}06000\,\text{А}.
        \\
    U_1
        &=
        \eli_1R_1
        =
        \frac{
            \cfrac{\ele_1 - \ele_3}{ R_3 } + \cfrac{\ele_1 - \ele_2}{ R_2 }
       }{\cfrac{ R_1 }{ R_2 } + 1 + \cfrac{ R_1 }{ R_3 }} \cdot R_1
        =
        -\frac2{31}\units{А} \cdot 4\,\text{Ом} = -\frac8{31}\units{В} \approx -0{,}2600\,\text{В}.
    \end{align*}

    Если вы проделали все эти вычисления выше вместе со мной, то
    \begin{itemize}
        \item вы совершили ошибку, выбрав неверный путь решения:
        слишком длинное решение, очень легко ошибиться в индексах, дробях, знаках или потерять какой-то множитель,
        \item можно было выразить из исходной системы другие токи и получить сразу нажный вам,
        а не какой-то 2-й,
        \item можно было сэкономить: все три резистора и ЭДС соединены одинаково,
        поэтому ответ для 1-го резистора должен отличаться лишь перестановкой индексов (этот факт крайне полезен при проверке ответа, у нас всё сошлось),
        я специально подгонял выражение для $\eli_1$ к этому виду, вынося за скобки и преобразуя дробь,
        \item вы молодец, потому что не побоялись и получили верный ответ грамотным способом,
    \end{itemize}
    так что переходим к третьему резистору.
    Будет похоже, но кого это когда останавливало...

    \begin{align*}
    \eli_3
        &=  \frac{\eli_2R_2 - \ele_2 + \ele_3}{ R_3 }
        =
        \cfrac{
            \cfrac{
                (\ele_2 - \ele_1)R_3 + (\ele_2 - \ele_3)R_1
           }{
                R_1R_3 + R_2R_3 + R_2R_1
           } \cdot R_2 - \ele_2 + \ele_3}{ R_3 }
        = \\ &=
        \frac{
            \ele_2R_3R_2 - \ele_1R_3R_2 + \ele_2R_1R_2 - \ele_3R_1R_2
            - \ele_2R_1R_3 - \ele_2R_2R_3 - \ele_2R_2R_1
            + \ele_3R_1R_3 + \ele_3R_2R_3 + \ele_3R_2R_1
       }{\cbr{R_1R_3 + R_2R_3 + R_2R_1} \cdot R_3}
        = \\ &=
        \frac{
            - \ele_1R_3R_2 - \ele_2R_1R_3 + \ele_3R_1R_3 + \ele_3R_2R_3
       }{\cbr{R_1R_3 + R_2R_3 + R_2R_1} \cdot R_3}
        =
        \frac{
            - \ele_1R_2 - \ele_2R_1 + \ele_3R_1 + \ele_3R_2
       }{R_1R_3 + R_2R_3 + R_2R_1}
        = \\ &=
        \frac{
            R_1(\ele_3 - \ele_2) + R_2(\ele_3 - \ele_1)
       }{R_1R_3 + R_2R_3 + R_2R_1}
        =
        \frac{
            \cfrac{\ele_3 - \ele_2}{ R_2 } + \cfrac{\ele_3 - \ele_1}{ R_1 }
       }{\cfrac{ R_3 }{ R_2 } + \cfrac{ R_3 }{ R_1 } + 1}
        =
        \frac{
            \cfrac{2\,\text{В} - 6\,\text{В}}{ 6\,\text{Ом} } + \cfrac{2\,\text{В} - 4\,\text{В}}{ 4\,\text{Ом} }
       }{\cfrac{ 10\,\text{Ом} }{ 6\,\text{Ом} } + \cfrac{ 10\,\text{Ом} }{ 4\,\text{Ом} } + 1}
        = -\frac7{31}\units{А} \approx -0{,}2300\,\text{А}.
        \\
    U_3
        &=
        \eli_3R_3
        =
        \frac{
            \cfrac{\ele_3 - \ele_2}{ R_2 } + \cfrac{\ele_3 - \ele_1}{ R_1 }
       }{\cfrac{ R_3 }{ R_2 } + \cfrac{ R_3 }{ R_1 } + 1} \cdot R_3
        =
        -\frac7{31}\units{А} \cdot 10\,\text{Ом} = -\frac{70}{31}\units{В} \approx -2{,}260\,\text{В}.
    \end{align*}

    Положительные ответы говорят, что мы угадали на рисунке направление тока (тут нет нашей заслуги, повезло),
    отрицательные — что не угадали (и в этом нет ошибки), и ток течёт в противоположную сторону.
    Напомним, что направление тока — это направление движения положительных зарядов,
    а в металлах носители заряда — электроны, которые заряжены отрицательно.
}

\variantsplitter

\addpersonalvariant{Дарья Кошман}

\tasknumber{1}%
\task{%
    Определите ток $\eli_2$, протекающий через резистор $R_2$ (см.
    рис.),
    направление этого тока и разность потенциалов $U_2$ на этом резисторе,
    если $R_1 = 4\,\text{Ом}$, $R_2 = 6\,\text{Ом}$, $R_3 = 15\,\text{Ом}$, $\ele_1 = 5\,\text{В}$, $\ele_2 = 3\,\text{В}$, $\ele_3 = 8\,\text{В}$.
    Внутренним сопротивлением всех трёх ЭДС пренебречь.
    Ответы получите в виде несократимых дробей, а также определите приближённые значения.

    \begin{tikzpicture}[circuit ee IEC, thick]
        \foreach \contact/\x in {1/0, 2/3, 3/6}
        {
            \node [contact] (top contact \contact) at (\x, 0) {};
            \node [contact] (bottom contact \contact) at (\x, 4) {};
       }
        \draw  (bottom contact 1) -- (bottom contact 2) -- (bottom contact 3);
        \draw  (top contact 1) -- (top contact 2) -- (top contact 3);
        \draw  (bottom contact 1) to [resistor={near start, info=$R_1$}, battery={near end, info=$\ele_1$}] (top contact 1);
        \draw  (bottom contact 2) to [resistor={near start, info=$R_2$}, battery={near end, info=$\ele_2$}] (top contact 2);
        \draw  (bottom contact 3) to [resistor={near start, info=$R_3$}, battery={near end, info=$\ele_3$}] (top contact 3);
    \end{tikzpicture}
}
\answer{%
    План:
    \begin{itemize}
        \item отметим на рисунке произвольно направления токов (если получим отрицательный ответ, значит не угадали направление и только),
        \item выберем и обозначим на рисунке контуры (здесь всего 3, значит будет нужно $3-1=2$), для них запишем законы Кирхгофа,
        \item выберем и выделим на рисунке нетривиальные узлы (здесь всего 2, значит будет нужно $2-1=1$), для него запишем закон Кирхгофа,
        \item попытаемся решить получившуюся систему.
        В конкретном решении мы пытались первым делом найти $\eli_2$, но, возможно, в вашем варианте будет быстрее решать систему в другом порядке.
        Мы всё же проделаем всё в лоб, подробно и целиком.
    \end{itemize}


    \begin{tikzpicture}[circuit ee IEC, thick]
        \foreach \contact/\x in {1/0, 2/3, 3/6}
        {
            \node [contact] (top contact \contact) at (\x, 0) {};
            \node [contact] (bottom contact \contact) at (\x, 4) {};
       }
        \draw  (bottom contact 1) -- (bottom contact 2) -- (bottom contact 3);
        \draw  (top contact 1) -- (top contact 2) -- (top contact 3);
        \draw  (bottom contact 1) to [resistor={near start, info=$R_1$}, current direction'={midway, info=$\eli_1$}, battery={near end, info=$\ele_1$}] (top contact 1);
        \draw  (bottom contact 2) to [resistor={near start, info=$R_2$}, current direction'={midway, info=$\eli_2$}, battery={near end, info=$\ele_2$}] (top contact 2);
        \draw  (bottom contact 3) to [resistor={near start, info=$R_3$}, current direction'={midway, info=$\eli_3$}, battery={near end, info=$\ele_3$}] (top contact 3);
        \draw [-{Latex},color=red] (1.2, 2.5) arc [start angle = 135, end angle = -160, radius = 0.6];
        \draw [-{Latex},color=blue] (4.2, 2.5) arc [start angle = 135, end angle = -160, radius = 0.6];
        \node [contact,color=green!71!black] (bottomc) at (bottom contact 2) {};
    \end{tikzpicture}

    \begin{align*}
        &\begin{cases}
            {\color{red} \eli_1R_1 - \eli_2R_2 = \ele_1 - \ele_2}, \\
            {\color{blue} \eli_2R_2 - \eli_3R_3 = \ele_2 - \ele_3}, \\
            {\color{green!71!black} \eli_1 + \eli_2 + \eli_3 = 0};
        \end{cases}
        \qquad \implies \qquad
        \begin{cases}
            \eli_1 = \frac{\ele_1 - \ele_2 + \eli_2R_2}{R_1}, \\
            \eli_3 = \frac{\eli_2R_2 - \ele_2 + \ele_3}{R_3}, \\
            \eli_1 + \eli_2 + \eli_3 = 0, \\
        \end{cases} \implies \\
        \implies
            &\eli_2 + \frac{\ele_1 - \ele_2 + \eli_2R_2}{R_1} + \frac{\eli_2R_2 - \ele_2 + \ele_3}{R_3} = 0, \\
        &   \eli_2\cbr{1 + \frac{ R_2 }{ R_1 } + \frac{ R_2 }{ R_3 }} + \frac{\ele_1 - \ele_2}{ R_1 } + \frac{\ele_3 - \ele_2}{ R_3 } = 0, \\
        &   \eli_2 = \cfrac{\cfrac{\ele_2 - \ele_1}{ R_1 } + \cfrac{\ele_2 - \ele_3}{ R_3 }}{1 + \cfrac{ R_2 }{ R_1 } + \cfrac{ R_2 }{ R_3 }}
            = \cfrac{\cfrac{3\,\text{В} - 5\,\text{В}}{ 4\,\text{Ом} } + \cfrac{3\,\text{В} - 8\,\text{В}}{ 15\,\text{Ом} }}{1 + \cfrac{ 6\,\text{Ом} }{ 4\,\text{Ом} } + \cfrac{ 6\,\text{Ом} }{ 15\,\text{Ом} }}
            = -\frac{25}{87}\units{А} \approx -0{,}2900\,\text{А}, \\
        &   U_2 = \eli_2R_2 = \cfrac{\cfrac{\ele_2 - \ele_1}{ R_1 } + \cfrac{\ele_2 - \ele_3}{ R_3 }}{1 + \cfrac{ R_2 }{ R_1 } + \cfrac{ R_2 }{ R_3 }} \cdot R_2
            = \cfrac{\cfrac{3\,\text{В} - 5\,\text{В}}{ 4\,\text{Ом} } + \cfrac{3\,\text{В} - 8\,\text{В}}{ 15\,\text{Ом} }}{1 + \cfrac{ 6\,\text{Ом} }{ 4\,\text{Ом} } + \cfrac{ 6\,\text{Ом} }{ 15\,\text{Ом} }} \cdot 6\,\text{Ом}
            = -\frac{25}{87}\units{А} \cdot 6\,\text{Ом} = -\frac{50}{29}\units{В} \approx -1{,}7200\,\text{В}.
    \end{align*}

    Одну пару силы тока и напряжения получили.
    Для некоторых вариантов это уже ответ, но не у всех.
    Для упрощения записи преобразуем (чтобы избавитсья от 4-этажной дроби) и подставим в уже полученные уравнения:

    \begin{align*}
    \eli_2
        &=
        \frac{\frac{\ele_2 - \ele_1}{ R_1 } + \frac{\ele_2 - \ele_3}{ R_3 }}{1 + \frac{ R_2 }{ R_1 } + \frac{ R_2 }{ R_3 }}
        =
        \frac{(\ele_2 - \ele_1)R_3 + (\ele_2 - \ele_3)R_1}{R_1R_3 + R_2R_3 + R_2R_1},
        \\
    \eli_1
        &=  \frac{\ele_1 - \ele_2 + \eli_2R_2}{R_1}
        =   \frac{\ele_1 - \ele_2 + \cfrac{(\ele_2 - \ele_1)R_3 + (\ele_2 - \ele_3)R_1}{R_1R_3 + R_2R_3 + R_2R_1} \cdot R_2}{R_1} = \\
        &=  \frac{
            \ele_1R_1R_3 + \ele_1R_2R_3 + \ele_1R_2R_1
            - \ele_2R_1R_3 - \ele_2R_2R_3 - \ele_2R_2R_1
            + \ele_2R_3R_2 - \ele_1R_3R_2 + \ele_2R_1R_2 - \ele_3R_1R_2
       }{R_1 \cdot \cbr{R_1R_3 + R_2R_3 + R_2R_1}}
        = \\ &=
        \frac{
            \ele_1\cbr{R_1R_3 + R_2R_3 + R_2R_1 - R_3R_2}
            + \ele_2\cbr{- R_1R_3 - R_2R_3 - R_2R_1 + R_3R_2 + R_1R_2}
            - \ele_3R_1R_2
       }{R_1 \cdot \cbr{R_1R_3 + R_2R_3 + R_2R_1}}
        = \\ &=
        \frac{
            \ele_1\cbr{R_1R_3 + R_2R_1}
            + \ele_2\cbr{- R_1R_3}
            - \ele_3R_1R_2
       }{R_1 \cdot \cbr{R_1R_3 + R_2R_3 + R_2R_1}}
        =
        \frac{
            \ele_1\cbr{R_3 + R_2} - \ele_2R_3 - \ele_3R_2
       }{R_1R_3 + R_2R_3 + R_2R_1}
        = \\ &=
        \frac{
            (\ele_1 - \ele_3)R_2 + (\ele_1 - \ele_2)R_3
       }{R_1R_3 + R_2R_3 + R_2R_1}
        =
        \frac{
            \cfrac{\ele_1 - \ele_3}{ R_3 } + \cfrac{\ele_1 - \ele_2}{ R_2 }
       }{\cfrac{ R_1 }{ R_2 } + 1 + \cfrac{ R_1 }{ R_3 }}
        =
        \frac{
            \cfrac{5\,\text{В} - 8\,\text{В}}{ 15\,\text{Ом} } + \cfrac{5\,\text{В} - 3\,\text{В}}{ 6\,\text{Ом} }
       }{\cfrac{ 4\,\text{Ом} }{ 6\,\text{Ом} } + 1 + \cfrac{ 4\,\text{Ом} }{ 15\,\text{Ом} }}
        = \frac2{29}\units{А} \approx 0{,}07\,\text{А}.
        \\
    U_1
        &=
        \eli_1R_1
        =
        \frac{
            \cfrac{\ele_1 - \ele_3}{ R_3 } + \cfrac{\ele_1 - \ele_2}{ R_2 }
       }{\cfrac{ R_1 }{ R_2 } + 1 + \cfrac{ R_1 }{ R_3 }} \cdot R_1
        =
        \frac2{29}\units{А} \cdot 4\,\text{Ом} = \frac8{29}\units{В} \approx 0{,}28\,\text{В}.
    \end{align*}

    Если вы проделали все эти вычисления выше вместе со мной, то
    \begin{itemize}
        \item вы совершили ошибку, выбрав неверный путь решения:
        слишком длинное решение, очень легко ошибиться в индексах, дробях, знаках или потерять какой-то множитель,
        \item можно было выразить из исходной системы другие токи и получить сразу нажный вам,
        а не какой-то 2-й,
        \item можно было сэкономить: все три резистора и ЭДС соединены одинаково,
        поэтому ответ для 1-го резистора должен отличаться лишь перестановкой индексов (этот факт крайне полезен при проверке ответа, у нас всё сошлось),
        я специально подгонял выражение для $\eli_1$ к этому виду, вынося за скобки и преобразуя дробь,
        \item вы молодец, потому что не побоялись и получили верный ответ грамотным способом,
    \end{itemize}
    так что переходим к третьему резистору.
    Будет похоже, но кого это когда останавливало...

    \begin{align*}
    \eli_3
        &=  \frac{\eli_2R_2 - \ele_2 + \ele_3}{ R_3 }
        =
        \cfrac{
            \cfrac{
                (\ele_2 - \ele_1)R_3 + (\ele_2 - \ele_3)R_1
           }{
                R_1R_3 + R_2R_3 + R_2R_1
           } \cdot R_2 - \ele_2 + \ele_3}{ R_3 }
        = \\ &=
        \frac{
            \ele_2R_3R_2 - \ele_1R_3R_2 + \ele_2R_1R_2 - \ele_3R_1R_2
            - \ele_2R_1R_3 - \ele_2R_2R_3 - \ele_2R_2R_1
            + \ele_3R_1R_3 + \ele_3R_2R_3 + \ele_3R_2R_1
       }{\cbr{R_1R_3 + R_2R_3 + R_2R_1} \cdot R_3}
        = \\ &=
        \frac{
            - \ele_1R_3R_2 - \ele_2R_1R_3 + \ele_3R_1R_3 + \ele_3R_2R_3
       }{\cbr{R_1R_3 + R_2R_3 + R_2R_1} \cdot R_3}
        =
        \frac{
            - \ele_1R_2 - \ele_2R_1 + \ele_3R_1 + \ele_3R_2
       }{R_1R_3 + R_2R_3 + R_2R_1}
        = \\ &=
        \frac{
            R_1(\ele_3 - \ele_2) + R_2(\ele_3 - \ele_1)
       }{R_1R_3 + R_2R_3 + R_2R_1}
        =
        \frac{
            \cfrac{\ele_3 - \ele_2}{ R_2 } + \cfrac{\ele_3 - \ele_1}{ R_1 }
       }{\cfrac{ R_3 }{ R_2 } + \cfrac{ R_3 }{ R_1 } + 1}
        =
        \frac{
            \cfrac{8\,\text{В} - 3\,\text{В}}{ 6\,\text{Ом} } + \cfrac{8\,\text{В} - 5\,\text{В}}{ 4\,\text{Ом} }
       }{\cfrac{ 15\,\text{Ом} }{ 6\,\text{Ом} } + \cfrac{ 15\,\text{Ом} }{ 4\,\text{Ом} } + 1}
        = \frac{19}{87}\units{А} \approx 0{,}22\,\text{А}.
        \\
    U_3
        &=
        \eli_3R_3
        =
        \frac{
            \cfrac{\ele_3 - \ele_2}{ R_2 } + \cfrac{\ele_3 - \ele_1}{ R_1 }
       }{\cfrac{ R_3 }{ R_2 } + \cfrac{ R_3 }{ R_1 } + 1} \cdot R_3
        =
        \frac{19}{87}\units{А} \cdot 15\,\text{Ом} = \frac{95}{29}\units{В} \approx 3{,}28\,\text{В}.
    \end{align*}

    Положительные ответы говорят, что мы угадали на рисунке направление тока (тут нет нашей заслуги, повезло),
    отрицательные — что не угадали (и в этом нет ошибки), и ток течёт в противоположную сторону.
    Напомним, что направление тока — это направление движения положительных зарядов,
    а в металлах носители заряда — электроны, которые заряжены отрицательно.
}

\variantsplitter

\addpersonalvariant{Анна Кузьмичёва}

\tasknumber{1}%
\task{%
    Определите ток $\eli_3$, протекающий через резистор $R_3$ (см.
    рис.),
    направление этого тока и разность потенциалов $U_3$ на этом резисторе,
    если $R_1 = 2\,\text{Ом}$, $R_2 = 8\,\text{Ом}$, $R_3 = 15\,\text{Ом}$, $\ele_1 = 4\,\text{В}$, $\ele_2 = 3\,\text{В}$, $\ele_3 = 2\,\text{В}$.
    Внутренним сопротивлением всех трёх ЭДС пренебречь.
    Ответы получите в виде несократимых дробей, а также определите приближённые значения.

    \begin{tikzpicture}[circuit ee IEC, thick]
        \foreach \contact/\x in {1/0, 2/3, 3/6}
        {
            \node [contact] (top contact \contact) at (\x, 0) {};
            \node [contact] (bottom contact \contact) at (\x, 4) {};
       }
        \draw  (bottom contact 1) -- (bottom contact 2) -- (bottom contact 3);
        \draw  (top contact 1) -- (top contact 2) -- (top contact 3);
        \draw  (bottom contact 1) to [resistor={near start, info=$R_1$}, battery={near end, info=$\ele_1$}] (top contact 1);
        \draw  (bottom contact 2) to [resistor={near start, info=$R_2$}, battery={near end, info=$\ele_2$}] (top contact 2);
        \draw  (bottom contact 3) to [resistor={near start, info=$R_3$}, battery={near end, info=$\ele_3$}] (top contact 3);
    \end{tikzpicture}
}
\answer{%
    План:
    \begin{itemize}
        \item отметим на рисунке произвольно направления токов (если получим отрицательный ответ, значит не угадали направление и только),
        \item выберем и обозначим на рисунке контуры (здесь всего 3, значит будет нужно $3-1=2$), для них запишем законы Кирхгофа,
        \item выберем и выделим на рисунке нетривиальные узлы (здесь всего 2, значит будет нужно $2-1=1$), для него запишем закон Кирхгофа,
        \item попытаемся решить получившуюся систему.
        В конкретном решении мы пытались первым делом найти $\eli_2$, но, возможно, в вашем варианте будет быстрее решать систему в другом порядке.
        Мы всё же проделаем всё в лоб, подробно и целиком.
    \end{itemize}


    \begin{tikzpicture}[circuit ee IEC, thick]
        \foreach \contact/\x in {1/0, 2/3, 3/6}
        {
            \node [contact] (top contact \contact) at (\x, 0) {};
            \node [contact] (bottom contact \contact) at (\x, 4) {};
       }
        \draw  (bottom contact 1) -- (bottom contact 2) -- (bottom contact 3);
        \draw  (top contact 1) -- (top contact 2) -- (top contact 3);
        \draw  (bottom contact 1) to [resistor={near start, info=$R_1$}, current direction'={midway, info=$\eli_1$}, battery={near end, info=$\ele_1$}] (top contact 1);
        \draw  (bottom contact 2) to [resistor={near start, info=$R_2$}, current direction'={midway, info=$\eli_2$}, battery={near end, info=$\ele_2$}] (top contact 2);
        \draw  (bottom contact 3) to [resistor={near start, info=$R_3$}, current direction'={midway, info=$\eli_3$}, battery={near end, info=$\ele_3$}] (top contact 3);
        \draw [-{Latex},color=red] (1.2, 2.5) arc [start angle = 135, end angle = -160, radius = 0.6];
        \draw [-{Latex},color=blue] (4.2, 2.5) arc [start angle = 135, end angle = -160, radius = 0.6];
        \node [contact,color=green!71!black] (bottomc) at (bottom contact 2) {};
    \end{tikzpicture}

    \begin{align*}
        &\begin{cases}
            {\color{red} \eli_1R_1 - \eli_2R_2 = \ele_1 - \ele_2}, \\
            {\color{blue} \eli_2R_2 - \eli_3R_3 = \ele_2 - \ele_3}, \\
            {\color{green!71!black} \eli_1 + \eli_2 + \eli_3 = 0};
        \end{cases}
        \qquad \implies \qquad
        \begin{cases}
            \eli_1 = \frac{\ele_1 - \ele_2 + \eli_2R_2}{R_1}, \\
            \eli_3 = \frac{\eli_2R_2 - \ele_2 + \ele_3}{R_3}, \\
            \eli_1 + \eli_2 + \eli_3 = 0, \\
        \end{cases} \implies \\
        \implies
            &\eli_2 + \frac{\ele_1 - \ele_2 + \eli_2R_2}{R_1} + \frac{\eli_2R_2 - \ele_2 + \ele_3}{R_3} = 0, \\
        &   \eli_2\cbr{1 + \frac{ R_2 }{ R_1 } + \frac{ R_2 }{ R_3 }} + \frac{\ele_1 - \ele_2}{ R_1 } + \frac{\ele_3 - \ele_2}{ R_3 } = 0, \\
        &   \eli_2 = \cfrac{\cfrac{\ele_2 - \ele_1}{ R_1 } + \cfrac{\ele_2 - \ele_3}{ R_3 }}{1 + \cfrac{ R_2 }{ R_1 } + \cfrac{ R_2 }{ R_3 }}
            = \cfrac{\cfrac{3\,\text{В} - 4\,\text{В}}{ 2\,\text{Ом} } + \cfrac{3\,\text{В} - 2\,\text{В}}{ 15\,\text{Ом} }}{1 + \cfrac{ 8\,\text{Ом} }{ 2\,\text{Ом} } + \cfrac{ 8\,\text{Ом} }{ 15\,\text{Ом} }}
            = -\frac{13}{166}\units{А} \approx -0{,}08000\,\text{А}, \\
        &   U_2 = \eli_2R_2 = \cfrac{\cfrac{\ele_2 - \ele_1}{ R_1 } + \cfrac{\ele_2 - \ele_3}{ R_3 }}{1 + \cfrac{ R_2 }{ R_1 } + \cfrac{ R_2 }{ R_3 }} \cdot R_2
            = \cfrac{\cfrac{3\,\text{В} - 4\,\text{В}}{ 2\,\text{Ом} } + \cfrac{3\,\text{В} - 2\,\text{В}}{ 15\,\text{Ом} }}{1 + \cfrac{ 8\,\text{Ом} }{ 2\,\text{Ом} } + \cfrac{ 8\,\text{Ом} }{ 15\,\text{Ом} }} \cdot 8\,\text{Ом}
            = -\frac{13}{166}\units{А} \cdot 8\,\text{Ом} = -\frac{52}{83}\units{В} \approx -0{,}6300\,\text{В}.
    \end{align*}

    Одну пару силы тока и напряжения получили.
    Для некоторых вариантов это уже ответ, но не у всех.
    Для упрощения записи преобразуем (чтобы избавитсья от 4-этажной дроби) и подставим в уже полученные уравнения:

    \begin{align*}
    \eli_2
        &=
        \frac{\frac{\ele_2 - \ele_1}{ R_1 } + \frac{\ele_2 - \ele_3}{ R_3 }}{1 + \frac{ R_2 }{ R_1 } + \frac{ R_2 }{ R_3 }}
        =
        \frac{(\ele_2 - \ele_1)R_3 + (\ele_2 - \ele_3)R_1}{R_1R_3 + R_2R_3 + R_2R_1},
        \\
    \eli_1
        &=  \frac{\ele_1 - \ele_2 + \eli_2R_2}{R_1}
        =   \frac{\ele_1 - \ele_2 + \cfrac{(\ele_2 - \ele_1)R_3 + (\ele_2 - \ele_3)R_1}{R_1R_3 + R_2R_3 + R_2R_1} \cdot R_2}{R_1} = \\
        &=  \frac{
            \ele_1R_1R_3 + \ele_1R_2R_3 + \ele_1R_2R_1
            - \ele_2R_1R_3 - \ele_2R_2R_3 - \ele_2R_2R_1
            + \ele_2R_3R_2 - \ele_1R_3R_2 + \ele_2R_1R_2 - \ele_3R_1R_2
       }{R_1 \cdot \cbr{R_1R_3 + R_2R_3 + R_2R_1}}
        = \\ &=
        \frac{
            \ele_1\cbr{R_1R_3 + R_2R_3 + R_2R_1 - R_3R_2}
            + \ele_2\cbr{- R_1R_3 - R_2R_3 - R_2R_1 + R_3R_2 + R_1R_2}
            - \ele_3R_1R_2
       }{R_1 \cdot \cbr{R_1R_3 + R_2R_3 + R_2R_1}}
        = \\ &=
        \frac{
            \ele_1\cbr{R_1R_3 + R_2R_1}
            + \ele_2\cbr{- R_1R_3}
            - \ele_3R_1R_2
       }{R_1 \cdot \cbr{R_1R_3 + R_2R_3 + R_2R_1}}
        =
        \frac{
            \ele_1\cbr{R_3 + R_2} - \ele_2R_3 - \ele_3R_2
       }{R_1R_3 + R_2R_3 + R_2R_1}
        = \\ &=
        \frac{
            (\ele_1 - \ele_3)R_2 + (\ele_1 - \ele_2)R_3
       }{R_1R_3 + R_2R_3 + R_2R_1}
        =
        \frac{
            \cfrac{\ele_1 - \ele_3}{ R_3 } + \cfrac{\ele_1 - \ele_2}{ R_2 }
       }{\cfrac{ R_1 }{ R_2 } + 1 + \cfrac{ R_1 }{ R_3 }}
        =
        \frac{
            \cfrac{4\,\text{В} - 2\,\text{В}}{ 15\,\text{Ом} } + \cfrac{4\,\text{В} - 3\,\text{В}}{ 8\,\text{Ом} }
       }{\cfrac{ 2\,\text{Ом} }{ 8\,\text{Ом} } + 1 + \cfrac{ 2\,\text{Ом} }{ 15\,\text{Ом} }}
        = \frac{31}{166}\units{А} \approx 0{,}19\,\text{А}.
        \\
    U_1
        &=
        \eli_1R_1
        =
        \frac{
            \cfrac{\ele_1 - \ele_3}{ R_3 } + \cfrac{\ele_1 - \ele_2}{ R_2 }
       }{\cfrac{ R_1 }{ R_2 } + 1 + \cfrac{ R_1 }{ R_3 }} \cdot R_1
        =
        \frac{31}{166}\units{А} \cdot 2\,\text{Ом} = \frac{31}{83}\units{В} \approx 0{,}37\,\text{В}.
    \end{align*}

    Если вы проделали все эти вычисления выше вместе со мной, то
    \begin{itemize}
        \item вы совершили ошибку, выбрав неверный путь решения:
        слишком длинное решение, очень легко ошибиться в индексах, дробях, знаках или потерять какой-то множитель,
        \item можно было выразить из исходной системы другие токи и получить сразу нажный вам,
        а не какой-то 2-й,
        \item можно было сэкономить: все три резистора и ЭДС соединены одинаково,
        поэтому ответ для 1-го резистора должен отличаться лишь перестановкой индексов (этот факт крайне полезен при проверке ответа, у нас всё сошлось),
        я специально подгонял выражение для $\eli_1$ к этому виду, вынося за скобки и преобразуя дробь,
        \item вы молодец, потому что не побоялись и получили верный ответ грамотным способом,
    \end{itemize}
    так что переходим к третьему резистору.
    Будет похоже, но кого это когда останавливало...

    \begin{align*}
    \eli_3
        &=  \frac{\eli_2R_2 - \ele_2 + \ele_3}{ R_3 }
        =
        \cfrac{
            \cfrac{
                (\ele_2 - \ele_1)R_3 + (\ele_2 - \ele_3)R_1
           }{
                R_1R_3 + R_2R_3 + R_2R_1
           } \cdot R_2 - \ele_2 + \ele_3}{ R_3 }
        = \\ &=
        \frac{
            \ele_2R_3R_2 - \ele_1R_3R_2 + \ele_2R_1R_2 - \ele_3R_1R_2
            - \ele_2R_1R_3 - \ele_2R_2R_3 - \ele_2R_2R_1
            + \ele_3R_1R_3 + \ele_3R_2R_3 + \ele_3R_2R_1
       }{\cbr{R_1R_3 + R_2R_3 + R_2R_1} \cdot R_3}
        = \\ &=
        \frac{
            - \ele_1R_3R_2 - \ele_2R_1R_3 + \ele_3R_1R_3 + \ele_3R_2R_3
       }{\cbr{R_1R_3 + R_2R_3 + R_2R_1} \cdot R_3}
        =
        \frac{
            - \ele_1R_2 - \ele_2R_1 + \ele_3R_1 + \ele_3R_2
       }{R_1R_3 + R_2R_3 + R_2R_1}
        = \\ &=
        \frac{
            R_1(\ele_3 - \ele_2) + R_2(\ele_3 - \ele_1)
       }{R_1R_3 + R_2R_3 + R_2R_1}
        =
        \frac{
            \cfrac{\ele_3 - \ele_2}{ R_2 } + \cfrac{\ele_3 - \ele_1}{ R_1 }
       }{\cfrac{ R_3 }{ R_2 } + \cfrac{ R_3 }{ R_1 } + 1}
        =
        \frac{
            \cfrac{2\,\text{В} - 3\,\text{В}}{ 8\,\text{Ом} } + \cfrac{2\,\text{В} - 4\,\text{В}}{ 2\,\text{Ом} }
       }{\cfrac{ 15\,\text{Ом} }{ 8\,\text{Ом} } + \cfrac{ 15\,\text{Ом} }{ 2\,\text{Ом} } + 1}
        = -\frac9{83}\units{А} \approx -0{,}11000\,\text{А}.
        \\
    U_3
        &=
        \eli_3R_3
        =
        \frac{
            \cfrac{\ele_3 - \ele_2}{ R_2 } + \cfrac{\ele_3 - \ele_1}{ R_1 }
       }{\cfrac{ R_3 }{ R_2 } + \cfrac{ R_3 }{ R_1 } + 1} \cdot R_3
        =
        -\frac9{83}\units{А} \cdot 15\,\text{Ом} = -\frac{135}{83}\units{В} \approx -1{,}6300\,\text{В}.
    \end{align*}

    Положительные ответы говорят, что мы угадали на рисунке направление тока (тут нет нашей заслуги, повезло),
    отрицательные — что не угадали (и в этом нет ошибки), и ток течёт в противоположную сторону.
    Напомним, что направление тока — это направление движения положительных зарядов,
    а в металлах носители заряда — электроны, которые заряжены отрицательно.
}

\variantsplitter

\addpersonalvariant{Алёна Куприянова}

\tasknumber{1}%
\task{%
    Определите ток $\eli_2$, протекающий через резистор $R_2$ (см.
    рис.),
    направление этого тока и разность потенциалов $U_2$ на этом резисторе,
    если $R_1 = 2\,\text{Ом}$, $R_2 = 6\,\text{Ом}$, $R_3 = 10\,\text{Ом}$, $\ele_1 = 5\,\text{В}$, $\ele_2 = 6\,\text{В}$, $\ele_3 = 8\,\text{В}$.
    Внутренним сопротивлением всех трёх ЭДС пренебречь.
    Ответы получите в виде несократимых дробей, а также определите приближённые значения.

    \begin{tikzpicture}[circuit ee IEC, thick]
        \foreach \contact/\x in {1/0, 2/3, 3/6}
        {
            \node [contact] (top contact \contact) at (\x, 0) {};
            \node [contact] (bottom contact \contact) at (\x, 4) {};
       }
        \draw  (bottom contact 1) -- (bottom contact 2) -- (bottom contact 3);
        \draw  (top contact 1) -- (top contact 2) -- (top contact 3);
        \draw  (bottom contact 1) to [resistor={near start, info=$R_1$}, battery={near end, info=$\ele_1$}] (top contact 1);
        \draw  (bottom contact 2) to [resistor={near start, info=$R_2$}, battery={near end, info=$\ele_2$}] (top contact 2);
        \draw  (bottom contact 3) to [resistor={near start, info=$R_3$}, battery={near end, info=$\ele_3$}] (top contact 3);
    \end{tikzpicture}
}
\answer{%
    План:
    \begin{itemize}
        \item отметим на рисунке произвольно направления токов (если получим отрицательный ответ, значит не угадали направление и только),
        \item выберем и обозначим на рисунке контуры (здесь всего 3, значит будет нужно $3-1=2$), для них запишем законы Кирхгофа,
        \item выберем и выделим на рисунке нетривиальные узлы (здесь всего 2, значит будет нужно $2-1=1$), для него запишем закон Кирхгофа,
        \item попытаемся решить получившуюся систему.
        В конкретном решении мы пытались первым делом найти $\eli_2$, но, возможно, в вашем варианте будет быстрее решать систему в другом порядке.
        Мы всё же проделаем всё в лоб, подробно и целиком.
    \end{itemize}


    \begin{tikzpicture}[circuit ee IEC, thick]
        \foreach \contact/\x in {1/0, 2/3, 3/6}
        {
            \node [contact] (top contact \contact) at (\x, 0) {};
            \node [contact] (bottom contact \contact) at (\x, 4) {};
       }
        \draw  (bottom contact 1) -- (bottom contact 2) -- (bottom contact 3);
        \draw  (top contact 1) -- (top contact 2) -- (top contact 3);
        \draw  (bottom contact 1) to [resistor={near start, info=$R_1$}, current direction'={midway, info=$\eli_1$}, battery={near end, info=$\ele_1$}] (top contact 1);
        \draw  (bottom contact 2) to [resistor={near start, info=$R_2$}, current direction'={midway, info=$\eli_2$}, battery={near end, info=$\ele_2$}] (top contact 2);
        \draw  (bottom contact 3) to [resistor={near start, info=$R_3$}, current direction'={midway, info=$\eli_3$}, battery={near end, info=$\ele_3$}] (top contact 3);
        \draw [-{Latex},color=red] (1.2, 2.5) arc [start angle = 135, end angle = -160, radius = 0.6];
        \draw [-{Latex},color=blue] (4.2, 2.5) arc [start angle = 135, end angle = -160, radius = 0.6];
        \node [contact,color=green!71!black] (bottomc) at (bottom contact 2) {};
    \end{tikzpicture}

    \begin{align*}
        &\begin{cases}
            {\color{red} \eli_1R_1 - \eli_2R_2 = \ele_1 - \ele_2}, \\
            {\color{blue} \eli_2R_2 - \eli_3R_3 = \ele_2 - \ele_3}, \\
            {\color{green!71!black} \eli_1 + \eli_2 + \eli_3 = 0};
        \end{cases}
        \qquad \implies \qquad
        \begin{cases}
            \eli_1 = \frac{\ele_1 - \ele_2 + \eli_2R_2}{R_1}, \\
            \eli_3 = \frac{\eli_2R_2 - \ele_2 + \ele_3}{R_3}, \\
            \eli_1 + \eli_2 + \eli_3 = 0, \\
        \end{cases} \implies \\
        \implies
            &\eli_2 + \frac{\ele_1 - \ele_2 + \eli_2R_2}{R_1} + \frac{\eli_2R_2 - \ele_2 + \ele_3}{R_3} = 0, \\
        &   \eli_2\cbr{1 + \frac{ R_2 }{ R_1 } + \frac{ R_2 }{ R_3 }} + \frac{\ele_1 - \ele_2}{ R_1 } + \frac{\ele_3 - \ele_2}{ R_3 } = 0, \\
        &   \eli_2 = \cfrac{\cfrac{\ele_2 - \ele_1}{ R_1 } + \cfrac{\ele_2 - \ele_3}{ R_3 }}{1 + \cfrac{ R_2 }{ R_1 } + \cfrac{ R_2 }{ R_3 }}
            = \cfrac{\cfrac{6\,\text{В} - 5\,\text{В}}{ 2\,\text{Ом} } + \cfrac{6\,\text{В} - 8\,\text{В}}{ 10\,\text{Ом} }}{1 + \cfrac{ 6\,\text{Ом} }{ 2\,\text{Ом} } + \cfrac{ 6\,\text{Ом} }{ 10\,\text{Ом} }}
            = \frac3{46}\units{А} \approx 0{,}07\,\text{А}, \\
        &   U_2 = \eli_2R_2 = \cfrac{\cfrac{\ele_2 - \ele_1}{ R_1 } + \cfrac{\ele_2 - \ele_3}{ R_3 }}{1 + \cfrac{ R_2 }{ R_1 } + \cfrac{ R_2 }{ R_3 }} \cdot R_2
            = \cfrac{\cfrac{6\,\text{В} - 5\,\text{В}}{ 2\,\text{Ом} } + \cfrac{6\,\text{В} - 8\,\text{В}}{ 10\,\text{Ом} }}{1 + \cfrac{ 6\,\text{Ом} }{ 2\,\text{Ом} } + \cfrac{ 6\,\text{Ом} }{ 10\,\text{Ом} }} \cdot 6\,\text{Ом}
            = \frac3{46}\units{А} \cdot 6\,\text{Ом} = \frac9{23}\units{В} \approx 0{,}39\,\text{В}.
    \end{align*}

    Одну пару силы тока и напряжения получили.
    Для некоторых вариантов это уже ответ, но не у всех.
    Для упрощения записи преобразуем (чтобы избавитсья от 4-этажной дроби) и подставим в уже полученные уравнения:

    \begin{align*}
    \eli_2
        &=
        \frac{\frac{\ele_2 - \ele_1}{ R_1 } + \frac{\ele_2 - \ele_3}{ R_3 }}{1 + \frac{ R_2 }{ R_1 } + \frac{ R_2 }{ R_3 }}
        =
        \frac{(\ele_2 - \ele_1)R_3 + (\ele_2 - \ele_3)R_1}{R_1R_3 + R_2R_3 + R_2R_1},
        \\
    \eli_1
        &=  \frac{\ele_1 - \ele_2 + \eli_2R_2}{R_1}
        =   \frac{\ele_1 - \ele_2 + \cfrac{(\ele_2 - \ele_1)R_3 + (\ele_2 - \ele_3)R_1}{R_1R_3 + R_2R_3 + R_2R_1} \cdot R_2}{R_1} = \\
        &=  \frac{
            \ele_1R_1R_3 + \ele_1R_2R_3 + \ele_1R_2R_1
            - \ele_2R_1R_3 - \ele_2R_2R_3 - \ele_2R_2R_1
            + \ele_2R_3R_2 - \ele_1R_3R_2 + \ele_2R_1R_2 - \ele_3R_1R_2
       }{R_1 \cdot \cbr{R_1R_3 + R_2R_3 + R_2R_1}}
        = \\ &=
        \frac{
            \ele_1\cbr{R_1R_3 + R_2R_3 + R_2R_1 - R_3R_2}
            + \ele_2\cbr{- R_1R_3 - R_2R_3 - R_2R_1 + R_3R_2 + R_1R_2}
            - \ele_3R_1R_2
       }{R_1 \cdot \cbr{R_1R_3 + R_2R_3 + R_2R_1}}
        = \\ &=
        \frac{
            \ele_1\cbr{R_1R_3 + R_2R_1}
            + \ele_2\cbr{- R_1R_3}
            - \ele_3R_1R_2
       }{R_1 \cdot \cbr{R_1R_3 + R_2R_3 + R_2R_1}}
        =
        \frac{
            \ele_1\cbr{R_3 + R_2} - \ele_2R_3 - \ele_3R_2
       }{R_1R_3 + R_2R_3 + R_2R_1}
        = \\ &=
        \frac{
            (\ele_1 - \ele_3)R_2 + (\ele_1 - \ele_2)R_3
       }{R_1R_3 + R_2R_3 + R_2R_1}
        =
        \frac{
            \cfrac{\ele_1 - \ele_3}{ R_3 } + \cfrac{\ele_1 - \ele_2}{ R_2 }
       }{\cfrac{ R_1 }{ R_2 } + 1 + \cfrac{ R_1 }{ R_3 }}
        =
        \frac{
            \cfrac{5\,\text{В} - 8\,\text{В}}{ 10\,\text{Ом} } + \cfrac{5\,\text{В} - 6\,\text{В}}{ 6\,\text{Ом} }
       }{\cfrac{ 2\,\text{Ом} }{ 6\,\text{Ом} } + 1 + \cfrac{ 2\,\text{Ом} }{ 10\,\text{Ом} }}
        = -\frac7{23}\units{А} \approx -0{,}3000\,\text{А}.
        \\
    U_1
        &=
        \eli_1R_1
        =
        \frac{
            \cfrac{\ele_1 - \ele_3}{ R_3 } + \cfrac{\ele_1 - \ele_2}{ R_2 }
       }{\cfrac{ R_1 }{ R_2 } + 1 + \cfrac{ R_1 }{ R_3 }} \cdot R_1
        =
        -\frac7{23}\units{А} \cdot 2\,\text{Ом} = -\frac{14}{23}\units{В} \approx -0{,}6100\,\text{В}.
    \end{align*}

    Если вы проделали все эти вычисления выше вместе со мной, то
    \begin{itemize}
        \item вы совершили ошибку, выбрав неверный путь решения:
        слишком длинное решение, очень легко ошибиться в индексах, дробях, знаках или потерять какой-то множитель,
        \item можно было выразить из исходной системы другие токи и получить сразу нажный вам,
        а не какой-то 2-й,
        \item можно было сэкономить: все три резистора и ЭДС соединены одинаково,
        поэтому ответ для 1-го резистора должен отличаться лишь перестановкой индексов (этот факт крайне полезен при проверке ответа, у нас всё сошлось),
        я специально подгонял выражение для $\eli_1$ к этому виду, вынося за скобки и преобразуя дробь,
        \item вы молодец, потому что не побоялись и получили верный ответ грамотным способом,
    \end{itemize}
    так что переходим к третьему резистору.
    Будет похоже, но кого это когда останавливало...

    \begin{align*}
    \eli_3
        &=  \frac{\eli_2R_2 - \ele_2 + \ele_3}{ R_3 }
        =
        \cfrac{
            \cfrac{
                (\ele_2 - \ele_1)R_3 + (\ele_2 - \ele_3)R_1
           }{
                R_1R_3 + R_2R_3 + R_2R_1
           } \cdot R_2 - \ele_2 + \ele_3}{ R_3 }
        = \\ &=
        \frac{
            \ele_2R_3R_2 - \ele_1R_3R_2 + \ele_2R_1R_2 - \ele_3R_1R_2
            - \ele_2R_1R_3 - \ele_2R_2R_3 - \ele_2R_2R_1
            + \ele_3R_1R_3 + \ele_3R_2R_3 + \ele_3R_2R_1
       }{\cbr{R_1R_3 + R_2R_3 + R_2R_1} \cdot R_3}
        = \\ &=
        \frac{
            - \ele_1R_3R_2 - \ele_2R_1R_3 + \ele_3R_1R_3 + \ele_3R_2R_3
       }{\cbr{R_1R_3 + R_2R_3 + R_2R_1} \cdot R_3}
        =
        \frac{
            - \ele_1R_2 - \ele_2R_1 + \ele_3R_1 + \ele_3R_2
       }{R_1R_3 + R_2R_3 + R_2R_1}
        = \\ &=
        \frac{
            R_1(\ele_3 - \ele_2) + R_2(\ele_3 - \ele_1)
       }{R_1R_3 + R_2R_3 + R_2R_1}
        =
        \frac{
            \cfrac{\ele_3 - \ele_2}{ R_2 } + \cfrac{\ele_3 - \ele_1}{ R_1 }
       }{\cfrac{ R_3 }{ R_2 } + \cfrac{ R_3 }{ R_1 } + 1}
        =
        \frac{
            \cfrac{8\,\text{В} - 6\,\text{В}}{ 6\,\text{Ом} } + \cfrac{8\,\text{В} - 5\,\text{В}}{ 2\,\text{Ом} }
       }{\cfrac{ 10\,\text{Ом} }{ 6\,\text{Ом} } + \cfrac{ 10\,\text{Ом} }{ 2\,\text{Ом} } + 1}
        = \frac{11}{46}\units{А} \approx 0{,}24\,\text{А}.
        \\
    U_3
        &=
        \eli_3R_3
        =
        \frac{
            \cfrac{\ele_3 - \ele_2}{ R_2 } + \cfrac{\ele_3 - \ele_1}{ R_1 }
       }{\cfrac{ R_3 }{ R_2 } + \cfrac{ R_3 }{ R_1 } + 1} \cdot R_3
        =
        \frac{11}{46}\units{А} \cdot 10\,\text{Ом} = \frac{55}{23}\units{В} \approx 2{,}39\,\text{В}.
    \end{align*}

    Положительные ответы говорят, что мы угадали на рисунке направление тока (тут нет нашей заслуги, повезло),
    отрицательные — что не угадали (и в этом нет ошибки), и ток течёт в противоположную сторону.
    Напомним, что направление тока — это направление движения положительных зарядов,
    а в металлах носители заряда — электроны, которые заряжены отрицательно.
}

\variantsplitter

\addpersonalvariant{Ярослав Лавровский}

\tasknumber{1}%
\task{%
    Определите ток $\eli_2$, протекающий через резистор $R_2$ (см.
    рис.),
    направление этого тока и разность потенциалов $U_2$ на этом резисторе,
    если $R_1 = 2\,\text{Ом}$, $R_2 = 5\,\text{Ом}$, $R_3 = 15\,\text{Ом}$, $\ele_1 = 5\,\text{В}$, $\ele_2 = 3\,\text{В}$, $\ele_3 = 2\,\text{В}$.
    Внутренним сопротивлением всех трёх ЭДС пренебречь.
    Ответы получите в виде несократимых дробей, а также определите приближённые значения.

    \begin{tikzpicture}[circuit ee IEC, thick]
        \foreach \contact/\x in {1/0, 2/3, 3/6}
        {
            \node [contact] (top contact \contact) at (\x, 0) {};
            \node [contact] (bottom contact \contact) at (\x, 4) {};
       }
        \draw  (bottom contact 1) -- (bottom contact 2) -- (bottom contact 3);
        \draw  (top contact 1) -- (top contact 2) -- (top contact 3);
        \draw  (bottom contact 1) to [resistor={near start, info=$R_1$}, battery={near end, info=$\ele_1$}] (top contact 1);
        \draw  (bottom contact 2) to [resistor={near start, info=$R_2$}, battery={near end, info=$\ele_2$}] (top contact 2);
        \draw  (bottom contact 3) to [resistor={near start, info=$R_3$}, battery={near end, info=$\ele_3$}] (top contact 3);
    \end{tikzpicture}
}
\answer{%
    План:
    \begin{itemize}
        \item отметим на рисунке произвольно направления токов (если получим отрицательный ответ, значит не угадали направление и только),
        \item выберем и обозначим на рисунке контуры (здесь всего 3, значит будет нужно $3-1=2$), для них запишем законы Кирхгофа,
        \item выберем и выделим на рисунке нетривиальные узлы (здесь всего 2, значит будет нужно $2-1=1$), для него запишем закон Кирхгофа,
        \item попытаемся решить получившуюся систему.
        В конкретном решении мы пытались первым делом найти $\eli_2$, но, возможно, в вашем варианте будет быстрее решать систему в другом порядке.
        Мы всё же проделаем всё в лоб, подробно и целиком.
    \end{itemize}


    \begin{tikzpicture}[circuit ee IEC, thick]
        \foreach \contact/\x in {1/0, 2/3, 3/6}
        {
            \node [contact] (top contact \contact) at (\x, 0) {};
            \node [contact] (bottom contact \contact) at (\x, 4) {};
       }
        \draw  (bottom contact 1) -- (bottom contact 2) -- (bottom contact 3);
        \draw  (top contact 1) -- (top contact 2) -- (top contact 3);
        \draw  (bottom contact 1) to [resistor={near start, info=$R_1$}, current direction'={midway, info=$\eli_1$}, battery={near end, info=$\ele_1$}] (top contact 1);
        \draw  (bottom contact 2) to [resistor={near start, info=$R_2$}, current direction'={midway, info=$\eli_2$}, battery={near end, info=$\ele_2$}] (top contact 2);
        \draw  (bottom contact 3) to [resistor={near start, info=$R_3$}, current direction'={midway, info=$\eli_3$}, battery={near end, info=$\ele_3$}] (top contact 3);
        \draw [-{Latex},color=red] (1.2, 2.5) arc [start angle = 135, end angle = -160, radius = 0.6];
        \draw [-{Latex},color=blue] (4.2, 2.5) arc [start angle = 135, end angle = -160, radius = 0.6];
        \node [contact,color=green!71!black] (bottomc) at (bottom contact 2) {};
    \end{tikzpicture}

    \begin{align*}
        &\begin{cases}
            {\color{red} \eli_1R_1 - \eli_2R_2 = \ele_1 - \ele_2}, \\
            {\color{blue} \eli_2R_2 - \eli_3R_3 = \ele_2 - \ele_3}, \\
            {\color{green!71!black} \eli_1 + \eli_2 + \eli_3 = 0};
        \end{cases}
        \qquad \implies \qquad
        \begin{cases}
            \eli_1 = \frac{\ele_1 - \ele_2 + \eli_2R_2}{R_1}, \\
            \eli_3 = \frac{\eli_2R_2 - \ele_2 + \ele_3}{R_3}, \\
            \eli_1 + \eli_2 + \eli_3 = 0, \\
        \end{cases} \implies \\
        \implies
            &\eli_2 + \frac{\ele_1 - \ele_2 + \eli_2R_2}{R_1} + \frac{\eli_2R_2 - \ele_2 + \ele_3}{R_3} = 0, \\
        &   \eli_2\cbr{1 + \frac{ R_2 }{ R_1 } + \frac{ R_2 }{ R_3 }} + \frac{\ele_1 - \ele_2}{ R_1 } + \frac{\ele_3 - \ele_2}{ R_3 } = 0, \\
        &   \eli_2 = \cfrac{\cfrac{\ele_2 - \ele_1}{ R_1 } + \cfrac{\ele_2 - \ele_3}{ R_3 }}{1 + \cfrac{ R_2 }{ R_1 } + \cfrac{ R_2 }{ R_3 }}
            = \cfrac{\cfrac{3\,\text{В} - 5\,\text{В}}{ 2\,\text{Ом} } + \cfrac{3\,\text{В} - 2\,\text{В}}{ 15\,\text{Ом} }}{1 + \cfrac{ 5\,\text{Ом} }{ 2\,\text{Ом} } + \cfrac{ 5\,\text{Ом} }{ 15\,\text{Ом} }}
            = -\frac{28}{115}\units{А} \approx -0{,}2400\,\text{А}, \\
        &   U_2 = \eli_2R_2 = \cfrac{\cfrac{\ele_2 - \ele_1}{ R_1 } + \cfrac{\ele_2 - \ele_3}{ R_3 }}{1 + \cfrac{ R_2 }{ R_1 } + \cfrac{ R_2 }{ R_3 }} \cdot R_2
            = \cfrac{\cfrac{3\,\text{В} - 5\,\text{В}}{ 2\,\text{Ом} } + \cfrac{3\,\text{В} - 2\,\text{В}}{ 15\,\text{Ом} }}{1 + \cfrac{ 5\,\text{Ом} }{ 2\,\text{Ом} } + \cfrac{ 5\,\text{Ом} }{ 15\,\text{Ом} }} \cdot 5\,\text{Ом}
            = -\frac{28}{115}\units{А} \cdot 5\,\text{Ом} = -\frac{28}{23}\units{В} \approx -1{,}2200\,\text{В}.
    \end{align*}

    Одну пару силы тока и напряжения получили.
    Для некоторых вариантов это уже ответ, но не у всех.
    Для упрощения записи преобразуем (чтобы избавитсья от 4-этажной дроби) и подставим в уже полученные уравнения:

    \begin{align*}
    \eli_2
        &=
        \frac{\frac{\ele_2 - \ele_1}{ R_1 } + \frac{\ele_2 - \ele_3}{ R_3 }}{1 + \frac{ R_2 }{ R_1 } + \frac{ R_2 }{ R_3 }}
        =
        \frac{(\ele_2 - \ele_1)R_3 + (\ele_2 - \ele_3)R_1}{R_1R_3 + R_2R_3 + R_2R_1},
        \\
    \eli_1
        &=  \frac{\ele_1 - \ele_2 + \eli_2R_2}{R_1}
        =   \frac{\ele_1 - \ele_2 + \cfrac{(\ele_2 - \ele_1)R_3 + (\ele_2 - \ele_3)R_1}{R_1R_3 + R_2R_3 + R_2R_1} \cdot R_2}{R_1} = \\
        &=  \frac{
            \ele_1R_1R_3 + \ele_1R_2R_3 + \ele_1R_2R_1
            - \ele_2R_1R_3 - \ele_2R_2R_3 - \ele_2R_2R_1
            + \ele_2R_3R_2 - \ele_1R_3R_2 + \ele_2R_1R_2 - \ele_3R_1R_2
       }{R_1 \cdot \cbr{R_1R_3 + R_2R_3 + R_2R_1}}
        = \\ &=
        \frac{
            \ele_1\cbr{R_1R_3 + R_2R_3 + R_2R_1 - R_3R_2}
            + \ele_2\cbr{- R_1R_3 - R_2R_3 - R_2R_1 + R_3R_2 + R_1R_2}
            - \ele_3R_1R_2
       }{R_1 \cdot \cbr{R_1R_3 + R_2R_3 + R_2R_1}}
        = \\ &=
        \frac{
            \ele_1\cbr{R_1R_3 + R_2R_1}
            + \ele_2\cbr{- R_1R_3}
            - \ele_3R_1R_2
       }{R_1 \cdot \cbr{R_1R_3 + R_2R_3 + R_2R_1}}
        =
        \frac{
            \ele_1\cbr{R_3 + R_2} - \ele_2R_3 - \ele_3R_2
       }{R_1R_3 + R_2R_3 + R_2R_1}
        = \\ &=
        \frac{
            (\ele_1 - \ele_3)R_2 + (\ele_1 - \ele_2)R_3
       }{R_1R_3 + R_2R_3 + R_2R_1}
        =
        \frac{
            \cfrac{\ele_1 - \ele_3}{ R_3 } + \cfrac{\ele_1 - \ele_2}{ R_2 }
       }{\cfrac{ R_1 }{ R_2 } + 1 + \cfrac{ R_1 }{ R_3 }}
        =
        \frac{
            \cfrac{5\,\text{В} - 2\,\text{В}}{ 15\,\text{Ом} } + \cfrac{5\,\text{В} - 3\,\text{В}}{ 5\,\text{Ом} }
       }{\cfrac{ 2\,\text{Ом} }{ 5\,\text{Ом} } + 1 + \cfrac{ 2\,\text{Ом} }{ 15\,\text{Ом} }}
        = \frac9{23}\units{А} \approx 0{,}39\,\text{А}.
        \\
    U_1
        &=
        \eli_1R_1
        =
        \frac{
            \cfrac{\ele_1 - \ele_3}{ R_3 } + \cfrac{\ele_1 - \ele_2}{ R_2 }
       }{\cfrac{ R_1 }{ R_2 } + 1 + \cfrac{ R_1 }{ R_3 }} \cdot R_1
        =
        \frac9{23}\units{А} \cdot 2\,\text{Ом} = \frac{18}{23}\units{В} \approx 0{,}78\,\text{В}.
    \end{align*}

    Если вы проделали все эти вычисления выше вместе со мной, то
    \begin{itemize}
        \item вы совершили ошибку, выбрав неверный путь решения:
        слишком длинное решение, очень легко ошибиться в индексах, дробях, знаках или потерять какой-то множитель,
        \item можно было выразить из исходной системы другие токи и получить сразу нажный вам,
        а не какой-то 2-й,
        \item можно было сэкономить: все три резистора и ЭДС соединены одинаково,
        поэтому ответ для 1-го резистора должен отличаться лишь перестановкой индексов (этот факт крайне полезен при проверке ответа, у нас всё сошлось),
        я специально подгонял выражение для $\eli_1$ к этому виду, вынося за скобки и преобразуя дробь,
        \item вы молодец, потому что не побоялись и получили верный ответ грамотным способом,
    \end{itemize}
    так что переходим к третьему резистору.
    Будет похоже, но кого это когда останавливало...

    \begin{align*}
    \eli_3
        &=  \frac{\eli_2R_2 - \ele_2 + \ele_3}{ R_3 }
        =
        \cfrac{
            \cfrac{
                (\ele_2 - \ele_1)R_3 + (\ele_2 - \ele_3)R_1
           }{
                R_1R_3 + R_2R_3 + R_2R_1
           } \cdot R_2 - \ele_2 + \ele_3}{ R_3 }
        = \\ &=
        \frac{
            \ele_2R_3R_2 - \ele_1R_3R_2 + \ele_2R_1R_2 - \ele_3R_1R_2
            - \ele_2R_1R_3 - \ele_2R_2R_3 - \ele_2R_2R_1
            + \ele_3R_1R_3 + \ele_3R_2R_3 + \ele_3R_2R_1
       }{\cbr{R_1R_3 + R_2R_3 + R_2R_1} \cdot R_3}
        = \\ &=
        \frac{
            - \ele_1R_3R_2 - \ele_2R_1R_3 + \ele_3R_1R_3 + \ele_3R_2R_3
       }{\cbr{R_1R_3 + R_2R_3 + R_2R_1} \cdot R_3}
        =
        \frac{
            - \ele_1R_2 - \ele_2R_1 + \ele_3R_1 + \ele_3R_2
       }{R_1R_3 + R_2R_3 + R_2R_1}
        = \\ &=
        \frac{
            R_1(\ele_3 - \ele_2) + R_2(\ele_3 - \ele_1)
       }{R_1R_3 + R_2R_3 + R_2R_1}
        =
        \frac{
            \cfrac{\ele_3 - \ele_2}{ R_2 } + \cfrac{\ele_3 - \ele_1}{ R_1 }
       }{\cfrac{ R_3 }{ R_2 } + \cfrac{ R_3 }{ R_1 } + 1}
        =
        \frac{
            \cfrac{2\,\text{В} - 3\,\text{В}}{ 5\,\text{Ом} } + \cfrac{2\,\text{В} - 5\,\text{В}}{ 2\,\text{Ом} }
       }{\cfrac{ 15\,\text{Ом} }{ 5\,\text{Ом} } + \cfrac{ 15\,\text{Ом} }{ 2\,\text{Ом} } + 1}
        = -\frac{17}{115}\units{А} \approx -0{,}15000\,\text{А}.
        \\
    U_3
        &=
        \eli_3R_3
        =
        \frac{
            \cfrac{\ele_3 - \ele_2}{ R_2 } + \cfrac{\ele_3 - \ele_1}{ R_1 }
       }{\cfrac{ R_3 }{ R_2 } + \cfrac{ R_3 }{ R_1 } + 1} \cdot R_3
        =
        -\frac{17}{115}\units{А} \cdot 15\,\text{Ом} = -\frac{51}{23}\units{В} \approx -2{,}220\,\text{В}.
    \end{align*}

    Положительные ответы говорят, что мы угадали на рисунке направление тока (тут нет нашей заслуги, повезло),
    отрицательные — что не угадали (и в этом нет ошибки), и ток течёт в противоположную сторону.
    Напомним, что направление тока — это направление движения положительных зарядов,
    а в металлах носители заряда — электроны, которые заряжены отрицательно.
}

\variantsplitter

\addpersonalvariant{Анастасия Ламанова}

\tasknumber{1}%
\task{%
    Определите ток $\eli_3$, протекающий через резистор $R_3$ (см.
    рис.),
    направление этого тока и разность потенциалов $U_3$ на этом резисторе,
    если $R_1 = 4\,\text{Ом}$, $R_2 = 8\,\text{Ом}$, $R_3 = 10\,\text{Ом}$, $\ele_1 = 4\,\text{В}$, $\ele_2 = 6\,\text{В}$, $\ele_3 = 8\,\text{В}$.
    Внутренним сопротивлением всех трёх ЭДС пренебречь.
    Ответы получите в виде несократимых дробей, а также определите приближённые значения.

    \begin{tikzpicture}[circuit ee IEC, thick]
        \foreach \contact/\x in {1/0, 2/3, 3/6}
        {
            \node [contact] (top contact \contact) at (\x, 0) {};
            \node [contact] (bottom contact \contact) at (\x, 4) {};
       }
        \draw  (bottom contact 1) -- (bottom contact 2) -- (bottom contact 3);
        \draw  (top contact 1) -- (top contact 2) -- (top contact 3);
        \draw  (bottom contact 1) to [resistor={near start, info=$R_1$}, battery={near end, info=$\ele_1$}] (top contact 1);
        \draw  (bottom contact 2) to [resistor={near start, info=$R_2$}, battery={near end, info=$\ele_2$}] (top contact 2);
        \draw  (bottom contact 3) to [resistor={near start, info=$R_3$}, battery={near end, info=$\ele_3$}] (top contact 3);
    \end{tikzpicture}
}
\answer{%
    План:
    \begin{itemize}
        \item отметим на рисунке произвольно направления токов (если получим отрицательный ответ, значит не угадали направление и только),
        \item выберем и обозначим на рисунке контуры (здесь всего 3, значит будет нужно $3-1=2$), для них запишем законы Кирхгофа,
        \item выберем и выделим на рисунке нетривиальные узлы (здесь всего 2, значит будет нужно $2-1=1$), для него запишем закон Кирхгофа,
        \item попытаемся решить получившуюся систему.
        В конкретном решении мы пытались первым делом найти $\eli_2$, но, возможно, в вашем варианте будет быстрее решать систему в другом порядке.
        Мы всё же проделаем всё в лоб, подробно и целиком.
    \end{itemize}


    \begin{tikzpicture}[circuit ee IEC, thick]
        \foreach \contact/\x in {1/0, 2/3, 3/6}
        {
            \node [contact] (top contact \contact) at (\x, 0) {};
            \node [contact] (bottom contact \contact) at (\x, 4) {};
       }
        \draw  (bottom contact 1) -- (bottom contact 2) -- (bottom contact 3);
        \draw  (top contact 1) -- (top contact 2) -- (top contact 3);
        \draw  (bottom contact 1) to [resistor={near start, info=$R_1$}, current direction'={midway, info=$\eli_1$}, battery={near end, info=$\ele_1$}] (top contact 1);
        \draw  (bottom contact 2) to [resistor={near start, info=$R_2$}, current direction'={midway, info=$\eli_2$}, battery={near end, info=$\ele_2$}] (top contact 2);
        \draw  (bottom contact 3) to [resistor={near start, info=$R_3$}, current direction'={midway, info=$\eli_3$}, battery={near end, info=$\ele_3$}] (top contact 3);
        \draw [-{Latex},color=red] (1.2, 2.5) arc [start angle = 135, end angle = -160, radius = 0.6];
        \draw [-{Latex},color=blue] (4.2, 2.5) arc [start angle = 135, end angle = -160, radius = 0.6];
        \node [contact,color=green!71!black] (bottomc) at (bottom contact 2) {};
    \end{tikzpicture}

    \begin{align*}
        &\begin{cases}
            {\color{red} \eli_1R_1 - \eli_2R_2 = \ele_1 - \ele_2}, \\
            {\color{blue} \eli_2R_2 - \eli_3R_3 = \ele_2 - \ele_3}, \\
            {\color{green!71!black} \eli_1 + \eli_2 + \eli_3 = 0};
        \end{cases}
        \qquad \implies \qquad
        \begin{cases}
            \eli_1 = \frac{\ele_1 - \ele_2 + \eli_2R_2}{R_1}, \\
            \eli_3 = \frac{\eli_2R_2 - \ele_2 + \ele_3}{R_3}, \\
            \eli_1 + \eli_2 + \eli_3 = 0, \\
        \end{cases} \implies \\
        \implies
            &\eli_2 + \frac{\ele_1 - \ele_2 + \eli_2R_2}{R_1} + \frac{\eli_2R_2 - \ele_2 + \ele_3}{R_3} = 0, \\
        &   \eli_2\cbr{1 + \frac{ R_2 }{ R_1 } + \frac{ R_2 }{ R_3 }} + \frac{\ele_1 - \ele_2}{ R_1 } + \frac{\ele_3 - \ele_2}{ R_3 } = 0, \\
        &   \eli_2 = \cfrac{\cfrac{\ele_2 - \ele_1}{ R_1 } + \cfrac{\ele_2 - \ele_3}{ R_3 }}{1 + \cfrac{ R_2 }{ R_1 } + \cfrac{ R_2 }{ R_3 }}
            = \cfrac{\cfrac{6\,\text{В} - 4\,\text{В}}{ 4\,\text{Ом} } + \cfrac{6\,\text{В} - 8\,\text{В}}{ 10\,\text{Ом} }}{1 + \cfrac{ 8\,\text{Ом} }{ 4\,\text{Ом} } + \cfrac{ 8\,\text{Ом} }{ 10\,\text{Ом} }}
            = \frac3{38}\units{А} \approx 0{,}08\,\text{А}, \\
        &   U_2 = \eli_2R_2 = \cfrac{\cfrac{\ele_2 - \ele_1}{ R_1 } + \cfrac{\ele_2 - \ele_3}{ R_3 }}{1 + \cfrac{ R_2 }{ R_1 } + \cfrac{ R_2 }{ R_3 }} \cdot R_2
            = \cfrac{\cfrac{6\,\text{В} - 4\,\text{В}}{ 4\,\text{Ом} } + \cfrac{6\,\text{В} - 8\,\text{В}}{ 10\,\text{Ом} }}{1 + \cfrac{ 8\,\text{Ом} }{ 4\,\text{Ом} } + \cfrac{ 8\,\text{Ом} }{ 10\,\text{Ом} }} \cdot 8\,\text{Ом}
            = \frac3{38}\units{А} \cdot 8\,\text{Ом} = \frac{12}{19}\units{В} \approx 0{,}63\,\text{В}.
    \end{align*}

    Одну пару силы тока и напряжения получили.
    Для некоторых вариантов это уже ответ, но не у всех.
    Для упрощения записи преобразуем (чтобы избавитсья от 4-этажной дроби) и подставим в уже полученные уравнения:

    \begin{align*}
    \eli_2
        &=
        \frac{\frac{\ele_2 - \ele_1}{ R_1 } + \frac{\ele_2 - \ele_3}{ R_3 }}{1 + \frac{ R_2 }{ R_1 } + \frac{ R_2 }{ R_3 }}
        =
        \frac{(\ele_2 - \ele_1)R_3 + (\ele_2 - \ele_3)R_1}{R_1R_3 + R_2R_3 + R_2R_1},
        \\
    \eli_1
        &=  \frac{\ele_1 - \ele_2 + \eli_2R_2}{R_1}
        =   \frac{\ele_1 - \ele_2 + \cfrac{(\ele_2 - \ele_1)R_3 + (\ele_2 - \ele_3)R_1}{R_1R_3 + R_2R_3 + R_2R_1} \cdot R_2}{R_1} = \\
        &=  \frac{
            \ele_1R_1R_3 + \ele_1R_2R_3 + \ele_1R_2R_1
            - \ele_2R_1R_3 - \ele_2R_2R_3 - \ele_2R_2R_1
            + \ele_2R_3R_2 - \ele_1R_3R_2 + \ele_2R_1R_2 - \ele_3R_1R_2
       }{R_1 \cdot \cbr{R_1R_3 + R_2R_3 + R_2R_1}}
        = \\ &=
        \frac{
            \ele_1\cbr{R_1R_3 + R_2R_3 + R_2R_1 - R_3R_2}
            + \ele_2\cbr{- R_1R_3 - R_2R_3 - R_2R_1 + R_3R_2 + R_1R_2}
            - \ele_3R_1R_2
       }{R_1 \cdot \cbr{R_1R_3 + R_2R_3 + R_2R_1}}
        = \\ &=
        \frac{
            \ele_1\cbr{R_1R_3 + R_2R_1}
            + \ele_2\cbr{- R_1R_3}
            - \ele_3R_1R_2
       }{R_1 \cdot \cbr{R_1R_3 + R_2R_3 + R_2R_1}}
        =
        \frac{
            \ele_1\cbr{R_3 + R_2} - \ele_2R_3 - \ele_3R_2
       }{R_1R_3 + R_2R_3 + R_2R_1}
        = \\ &=
        \frac{
            (\ele_1 - \ele_3)R_2 + (\ele_1 - \ele_2)R_3
       }{R_1R_3 + R_2R_3 + R_2R_1}
        =
        \frac{
            \cfrac{\ele_1 - \ele_3}{ R_3 } + \cfrac{\ele_1 - \ele_2}{ R_2 }
       }{\cfrac{ R_1 }{ R_2 } + 1 + \cfrac{ R_1 }{ R_3 }}
        =
        \frac{
            \cfrac{4\,\text{В} - 8\,\text{В}}{ 10\,\text{Ом} } + \cfrac{4\,\text{В} - 6\,\text{В}}{ 8\,\text{Ом} }
       }{\cfrac{ 4\,\text{Ом} }{ 8\,\text{Ом} } + 1 + \cfrac{ 4\,\text{Ом} }{ 10\,\text{Ом} }}
        = -\frac{13}{38}\units{А} \approx -0{,}3400\,\text{А}.
        \\
    U_1
        &=
        \eli_1R_1
        =
        \frac{
            \cfrac{\ele_1 - \ele_3}{ R_3 } + \cfrac{\ele_1 - \ele_2}{ R_2 }
       }{\cfrac{ R_1 }{ R_2 } + 1 + \cfrac{ R_1 }{ R_3 }} \cdot R_1
        =
        -\frac{13}{38}\units{А} \cdot 4\,\text{Ом} = -\frac{26}{19}\units{В} \approx -1{,}3700\,\text{В}.
    \end{align*}

    Если вы проделали все эти вычисления выше вместе со мной, то
    \begin{itemize}
        \item вы совершили ошибку, выбрав неверный путь решения:
        слишком длинное решение, очень легко ошибиться в индексах, дробях, знаках или потерять какой-то множитель,
        \item можно было выразить из исходной системы другие токи и получить сразу нажный вам,
        а не какой-то 2-й,
        \item можно было сэкономить: все три резистора и ЭДС соединены одинаково,
        поэтому ответ для 1-го резистора должен отличаться лишь перестановкой индексов (этот факт крайне полезен при проверке ответа, у нас всё сошлось),
        я специально подгонял выражение для $\eli_1$ к этому виду, вынося за скобки и преобразуя дробь,
        \item вы молодец, потому что не побоялись и получили верный ответ грамотным способом,
    \end{itemize}
    так что переходим к третьему резистору.
    Будет похоже, но кого это когда останавливало...

    \begin{align*}
    \eli_3
        &=  \frac{\eli_2R_2 - \ele_2 + \ele_3}{ R_3 }
        =
        \cfrac{
            \cfrac{
                (\ele_2 - \ele_1)R_3 + (\ele_2 - \ele_3)R_1
           }{
                R_1R_3 + R_2R_3 + R_2R_1
           } \cdot R_2 - \ele_2 + \ele_3}{ R_3 }
        = \\ &=
        \frac{
            \ele_2R_3R_2 - \ele_1R_3R_2 + \ele_2R_1R_2 - \ele_3R_1R_2
            - \ele_2R_1R_3 - \ele_2R_2R_3 - \ele_2R_2R_1
            + \ele_3R_1R_3 + \ele_3R_2R_3 + \ele_3R_2R_1
       }{\cbr{R_1R_3 + R_2R_3 + R_2R_1} \cdot R_3}
        = \\ &=
        \frac{
            - \ele_1R_3R_2 - \ele_2R_1R_3 + \ele_3R_1R_3 + \ele_3R_2R_3
       }{\cbr{R_1R_3 + R_2R_3 + R_2R_1} \cdot R_3}
        =
        \frac{
            - \ele_1R_2 - \ele_2R_1 + \ele_3R_1 + \ele_3R_2
       }{R_1R_3 + R_2R_3 + R_2R_1}
        = \\ &=
        \frac{
            R_1(\ele_3 - \ele_2) + R_2(\ele_3 - \ele_1)
       }{R_1R_3 + R_2R_3 + R_2R_1}
        =
        \frac{
            \cfrac{\ele_3 - \ele_2}{ R_2 } + \cfrac{\ele_3 - \ele_1}{ R_1 }
       }{\cfrac{ R_3 }{ R_2 } + \cfrac{ R_3 }{ R_1 } + 1}
        =
        \frac{
            \cfrac{8\,\text{В} - 6\,\text{В}}{ 8\,\text{Ом} } + \cfrac{8\,\text{В} - 4\,\text{В}}{ 4\,\text{Ом} }
       }{\cfrac{ 10\,\text{Ом} }{ 8\,\text{Ом} } + \cfrac{ 10\,\text{Ом} }{ 4\,\text{Ом} } + 1}
        = \frac5{19}\units{А} \approx 0{,}26\,\text{А}.
        \\
    U_3
        &=
        \eli_3R_3
        =
        \frac{
            \cfrac{\ele_3 - \ele_2}{ R_2 } + \cfrac{\ele_3 - \ele_1}{ R_1 }
       }{\cfrac{ R_3 }{ R_2 } + \cfrac{ R_3 }{ R_1 } + 1} \cdot R_3
        =
        \frac5{19}\units{А} \cdot 10\,\text{Ом} = \frac{50}{19}\units{В} \approx 2{,}63\,\text{В}.
    \end{align*}

    Положительные ответы говорят, что мы угадали на рисунке направление тока (тут нет нашей заслуги, повезло),
    отрицательные — что не угадали (и в этом нет ошибки), и ток течёт в противоположную сторону.
    Напомним, что направление тока — это направление движения положительных зарядов,
    а в металлах носители заряда — электроны, которые заряжены отрицательно.
}

\variantsplitter

\addpersonalvariant{Виктория Легонькова}

\tasknumber{1}%
\task{%
    Определите ток $\eli_1$, протекающий через резистор $R_1$ (см.
    рис.),
    направление этого тока и разность потенциалов $U_1$ на этом резисторе,
    если $R_1 = 2\,\text{Ом}$, $R_2 = 5\,\text{Ом}$, $R_3 = 15\,\text{Ом}$, $\ele_1 = 5\,\text{В}$, $\ele_2 = 6\,\text{В}$, $\ele_3 = 8\,\text{В}$.
    Внутренним сопротивлением всех трёх ЭДС пренебречь.
    Ответы получите в виде несократимых дробей, а также определите приближённые значения.

    \begin{tikzpicture}[circuit ee IEC, thick]
        \foreach \contact/\x in {1/0, 2/3, 3/6}
        {
            \node [contact] (top contact \contact) at (\x, 0) {};
            \node [contact] (bottom contact \contact) at (\x, 4) {};
       }
        \draw  (bottom contact 1) -- (bottom contact 2) -- (bottom contact 3);
        \draw  (top contact 1) -- (top contact 2) -- (top contact 3);
        \draw  (bottom contact 1) to [resistor={near start, info=$R_1$}, battery={near end, info=$\ele_1$}] (top contact 1);
        \draw  (bottom contact 2) to [resistor={near start, info=$R_2$}, battery={near end, info=$\ele_2$}] (top contact 2);
        \draw  (bottom contact 3) to [resistor={near start, info=$R_3$}, battery={near end, info=$\ele_3$}] (top contact 3);
    \end{tikzpicture}
}
\answer{%
    План:
    \begin{itemize}
        \item отметим на рисунке произвольно направления токов (если получим отрицательный ответ, значит не угадали направление и только),
        \item выберем и обозначим на рисунке контуры (здесь всего 3, значит будет нужно $3-1=2$), для них запишем законы Кирхгофа,
        \item выберем и выделим на рисунке нетривиальные узлы (здесь всего 2, значит будет нужно $2-1=1$), для него запишем закон Кирхгофа,
        \item попытаемся решить получившуюся систему.
        В конкретном решении мы пытались первым делом найти $\eli_2$, но, возможно, в вашем варианте будет быстрее решать систему в другом порядке.
        Мы всё же проделаем всё в лоб, подробно и целиком.
    \end{itemize}


    \begin{tikzpicture}[circuit ee IEC, thick]
        \foreach \contact/\x in {1/0, 2/3, 3/6}
        {
            \node [contact] (top contact \contact) at (\x, 0) {};
            \node [contact] (bottom contact \contact) at (\x, 4) {};
       }
        \draw  (bottom contact 1) -- (bottom contact 2) -- (bottom contact 3);
        \draw  (top contact 1) -- (top contact 2) -- (top contact 3);
        \draw  (bottom contact 1) to [resistor={near start, info=$R_1$}, current direction'={midway, info=$\eli_1$}, battery={near end, info=$\ele_1$}] (top contact 1);
        \draw  (bottom contact 2) to [resistor={near start, info=$R_2$}, current direction'={midway, info=$\eli_2$}, battery={near end, info=$\ele_2$}] (top contact 2);
        \draw  (bottom contact 3) to [resistor={near start, info=$R_3$}, current direction'={midway, info=$\eli_3$}, battery={near end, info=$\ele_3$}] (top contact 3);
        \draw [-{Latex},color=red] (1.2, 2.5) arc [start angle = 135, end angle = -160, radius = 0.6];
        \draw [-{Latex},color=blue] (4.2, 2.5) arc [start angle = 135, end angle = -160, radius = 0.6];
        \node [contact,color=green!71!black] (bottomc) at (bottom contact 2) {};
    \end{tikzpicture}

    \begin{align*}
        &\begin{cases}
            {\color{red} \eli_1R_1 - \eli_2R_2 = \ele_1 - \ele_2}, \\
            {\color{blue} \eli_2R_2 - \eli_3R_3 = \ele_2 - \ele_3}, \\
            {\color{green!71!black} \eli_1 + \eli_2 + \eli_3 = 0};
        \end{cases}
        \qquad \implies \qquad
        \begin{cases}
            \eli_1 = \frac{\ele_1 - \ele_2 + \eli_2R_2}{R_1}, \\
            \eli_3 = \frac{\eli_2R_2 - \ele_2 + \ele_3}{R_3}, \\
            \eli_1 + \eli_2 + \eli_3 = 0, \\
        \end{cases} \implies \\
        \implies
            &\eli_2 + \frac{\ele_1 - \ele_2 + \eli_2R_2}{R_1} + \frac{\eli_2R_2 - \ele_2 + \ele_3}{R_3} = 0, \\
        &   \eli_2\cbr{1 + \frac{ R_2 }{ R_1 } + \frac{ R_2 }{ R_3 }} + \frac{\ele_1 - \ele_2}{ R_1 } + \frac{\ele_3 - \ele_2}{ R_3 } = 0, \\
        &   \eli_2 = \cfrac{\cfrac{\ele_2 - \ele_1}{ R_1 } + \cfrac{\ele_2 - \ele_3}{ R_3 }}{1 + \cfrac{ R_2 }{ R_1 } + \cfrac{ R_2 }{ R_3 }}
            = \cfrac{\cfrac{6\,\text{В} - 5\,\text{В}}{ 2\,\text{Ом} } + \cfrac{6\,\text{В} - 8\,\text{В}}{ 15\,\text{Ом} }}{1 + \cfrac{ 5\,\text{Ом} }{ 2\,\text{Ом} } + \cfrac{ 5\,\text{Ом} }{ 15\,\text{Ом} }}
            = \frac{11}{115}\units{А} \approx 0{,}10\,\text{А}, \\
        &   U_2 = \eli_2R_2 = \cfrac{\cfrac{\ele_2 - \ele_1}{ R_1 } + \cfrac{\ele_2 - \ele_3}{ R_3 }}{1 + \cfrac{ R_2 }{ R_1 } + \cfrac{ R_2 }{ R_3 }} \cdot R_2
            = \cfrac{\cfrac{6\,\text{В} - 5\,\text{В}}{ 2\,\text{Ом} } + \cfrac{6\,\text{В} - 8\,\text{В}}{ 15\,\text{Ом} }}{1 + \cfrac{ 5\,\text{Ом} }{ 2\,\text{Ом} } + \cfrac{ 5\,\text{Ом} }{ 15\,\text{Ом} }} \cdot 5\,\text{Ом}
            = \frac{11}{115}\units{А} \cdot 5\,\text{Ом} = \frac{11}{23}\units{В} \approx 0{,}48\,\text{В}.
    \end{align*}

    Одну пару силы тока и напряжения получили.
    Для некоторых вариантов это уже ответ, но не у всех.
    Для упрощения записи преобразуем (чтобы избавитсья от 4-этажной дроби) и подставим в уже полученные уравнения:

    \begin{align*}
    \eli_2
        &=
        \frac{\frac{\ele_2 - \ele_1}{ R_1 } + \frac{\ele_2 - \ele_3}{ R_3 }}{1 + \frac{ R_2 }{ R_1 } + \frac{ R_2 }{ R_3 }}
        =
        \frac{(\ele_2 - \ele_1)R_3 + (\ele_2 - \ele_3)R_1}{R_1R_3 + R_2R_3 + R_2R_1},
        \\
    \eli_1
        &=  \frac{\ele_1 - \ele_2 + \eli_2R_2}{R_1}
        =   \frac{\ele_1 - \ele_2 + \cfrac{(\ele_2 - \ele_1)R_3 + (\ele_2 - \ele_3)R_1}{R_1R_3 + R_2R_3 + R_2R_1} \cdot R_2}{R_1} = \\
        &=  \frac{
            \ele_1R_1R_3 + \ele_1R_2R_3 + \ele_1R_2R_1
            - \ele_2R_1R_3 - \ele_2R_2R_3 - \ele_2R_2R_1
            + \ele_2R_3R_2 - \ele_1R_3R_2 + \ele_2R_1R_2 - \ele_3R_1R_2
       }{R_1 \cdot \cbr{R_1R_3 + R_2R_3 + R_2R_1}}
        = \\ &=
        \frac{
            \ele_1\cbr{R_1R_3 + R_2R_3 + R_2R_1 - R_3R_2}
            + \ele_2\cbr{- R_1R_3 - R_2R_3 - R_2R_1 + R_3R_2 + R_1R_2}
            - \ele_3R_1R_2
       }{R_1 \cdot \cbr{R_1R_3 + R_2R_3 + R_2R_1}}
        = \\ &=
        \frac{
            \ele_1\cbr{R_1R_3 + R_2R_1}
            + \ele_2\cbr{- R_1R_3}
            - \ele_3R_1R_2
       }{R_1 \cdot \cbr{R_1R_3 + R_2R_3 + R_2R_1}}
        =
        \frac{
            \ele_1\cbr{R_3 + R_2} - \ele_2R_3 - \ele_3R_2
       }{R_1R_3 + R_2R_3 + R_2R_1}
        = \\ &=
        \frac{
            (\ele_1 - \ele_3)R_2 + (\ele_1 - \ele_2)R_3
       }{R_1R_3 + R_2R_3 + R_2R_1}
        =
        \frac{
            \cfrac{\ele_1 - \ele_3}{ R_3 } + \cfrac{\ele_1 - \ele_2}{ R_2 }
       }{\cfrac{ R_1 }{ R_2 } + 1 + \cfrac{ R_1 }{ R_3 }}
        =
        \frac{
            \cfrac{5\,\text{В} - 8\,\text{В}}{ 15\,\text{Ом} } + \cfrac{5\,\text{В} - 6\,\text{В}}{ 5\,\text{Ом} }
       }{\cfrac{ 2\,\text{Ом} }{ 5\,\text{Ом} } + 1 + \cfrac{ 2\,\text{Ом} }{ 15\,\text{Ом} }}
        = -\frac6{23}\units{А} \approx -0{,}2600\,\text{А}.
        \\
    U_1
        &=
        \eli_1R_1
        =
        \frac{
            \cfrac{\ele_1 - \ele_3}{ R_3 } + \cfrac{\ele_1 - \ele_2}{ R_2 }
       }{\cfrac{ R_1 }{ R_2 } + 1 + \cfrac{ R_1 }{ R_3 }} \cdot R_1
        =
        -\frac6{23}\units{А} \cdot 2\,\text{Ом} = -\frac{12}{23}\units{В} \approx -0{,}5200\,\text{В}.
    \end{align*}

    Если вы проделали все эти вычисления выше вместе со мной, то
    \begin{itemize}
        \item вы совершили ошибку, выбрав неверный путь решения:
        слишком длинное решение, очень легко ошибиться в индексах, дробях, знаках или потерять какой-то множитель,
        \item можно было выразить из исходной системы другие токи и получить сразу нажный вам,
        а не какой-то 2-й,
        \item можно было сэкономить: все три резистора и ЭДС соединены одинаково,
        поэтому ответ для 1-го резистора должен отличаться лишь перестановкой индексов (этот факт крайне полезен при проверке ответа, у нас всё сошлось),
        я специально подгонял выражение для $\eli_1$ к этому виду, вынося за скобки и преобразуя дробь,
        \item вы молодец, потому что не побоялись и получили верный ответ грамотным способом,
    \end{itemize}
    так что переходим к третьему резистору.
    Будет похоже, но кого это когда останавливало...

    \begin{align*}
    \eli_3
        &=  \frac{\eli_2R_2 - \ele_2 + \ele_3}{ R_3 }
        =
        \cfrac{
            \cfrac{
                (\ele_2 - \ele_1)R_3 + (\ele_2 - \ele_3)R_1
           }{
                R_1R_3 + R_2R_3 + R_2R_1
           } \cdot R_2 - \ele_2 + \ele_3}{ R_3 }
        = \\ &=
        \frac{
            \ele_2R_3R_2 - \ele_1R_3R_2 + \ele_2R_1R_2 - \ele_3R_1R_2
            - \ele_2R_1R_3 - \ele_2R_2R_3 - \ele_2R_2R_1
            + \ele_3R_1R_3 + \ele_3R_2R_3 + \ele_3R_2R_1
       }{\cbr{R_1R_3 + R_2R_3 + R_2R_1} \cdot R_3}
        = \\ &=
        \frac{
            - \ele_1R_3R_2 - \ele_2R_1R_3 + \ele_3R_1R_3 + \ele_3R_2R_3
       }{\cbr{R_1R_3 + R_2R_3 + R_2R_1} \cdot R_3}
        =
        \frac{
            - \ele_1R_2 - \ele_2R_1 + \ele_3R_1 + \ele_3R_2
       }{R_1R_3 + R_2R_3 + R_2R_1}
        = \\ &=
        \frac{
            R_1(\ele_3 - \ele_2) + R_2(\ele_3 - \ele_1)
       }{R_1R_3 + R_2R_3 + R_2R_1}
        =
        \frac{
            \cfrac{\ele_3 - \ele_2}{ R_2 } + \cfrac{\ele_3 - \ele_1}{ R_1 }
       }{\cfrac{ R_3 }{ R_2 } + \cfrac{ R_3 }{ R_1 } + 1}
        =
        \frac{
            \cfrac{8\,\text{В} - 6\,\text{В}}{ 5\,\text{Ом} } + \cfrac{8\,\text{В} - 5\,\text{В}}{ 2\,\text{Ом} }
       }{\cfrac{ 15\,\text{Ом} }{ 5\,\text{Ом} } + \cfrac{ 15\,\text{Ом} }{ 2\,\text{Ом} } + 1}
        = \frac{19}{115}\units{А} \approx 0{,}17\,\text{А}.
        \\
    U_3
        &=
        \eli_3R_3
        =
        \frac{
            \cfrac{\ele_3 - \ele_2}{ R_2 } + \cfrac{\ele_3 - \ele_1}{ R_1 }
       }{\cfrac{ R_3 }{ R_2 } + \cfrac{ R_3 }{ R_1 } + 1} \cdot R_3
        =
        \frac{19}{115}\units{А} \cdot 15\,\text{Ом} = \frac{57}{23}\units{В} \approx 2{,}48\,\text{В}.
    \end{align*}

    Положительные ответы говорят, что мы угадали на рисунке направление тока (тут нет нашей заслуги, повезло),
    отрицательные — что не угадали (и в этом нет ошибки), и ток течёт в противоположную сторону.
    Напомним, что направление тока — это направление движения положительных зарядов,
    а в металлах носители заряда — электроны, которые заряжены отрицательно.
}

\variantsplitter

\addpersonalvariant{Семён Мартынов}

\tasknumber{1}%
\task{%
    Определите ток $\eli_1$, протекающий через резистор $R_1$ (см.
    рис.),
    направление этого тока и разность потенциалов $U_1$ на этом резисторе,
    если $R_1 = 3\,\text{Ом}$, $R_2 = 5\,\text{Ом}$, $R_3 = 10\,\text{Ом}$, $\ele_1 = 5\,\text{В}$, $\ele_2 = 6\,\text{В}$, $\ele_3 = 2\,\text{В}$.
    Внутренним сопротивлением всех трёх ЭДС пренебречь.
    Ответы получите в виде несократимых дробей, а также определите приближённые значения.

    \begin{tikzpicture}[circuit ee IEC, thick]
        \foreach \contact/\x in {1/0, 2/3, 3/6}
        {
            \node [contact] (top contact \contact) at (\x, 0) {};
            \node [contact] (bottom contact \contact) at (\x, 4) {};
       }
        \draw  (bottom contact 1) -- (bottom contact 2) -- (bottom contact 3);
        \draw  (top contact 1) -- (top contact 2) -- (top contact 3);
        \draw  (bottom contact 1) to [resistor={near start, info=$R_1$}, battery={near end, info=$\ele_1$}] (top contact 1);
        \draw  (bottom contact 2) to [resistor={near start, info=$R_2$}, battery={near end, info=$\ele_2$}] (top contact 2);
        \draw  (bottom contact 3) to [resistor={near start, info=$R_3$}, battery={near end, info=$\ele_3$}] (top contact 3);
    \end{tikzpicture}
}
\answer{%
    План:
    \begin{itemize}
        \item отметим на рисунке произвольно направления токов (если получим отрицательный ответ, значит не угадали направление и только),
        \item выберем и обозначим на рисунке контуры (здесь всего 3, значит будет нужно $3-1=2$), для них запишем законы Кирхгофа,
        \item выберем и выделим на рисунке нетривиальные узлы (здесь всего 2, значит будет нужно $2-1=1$), для него запишем закон Кирхгофа,
        \item попытаемся решить получившуюся систему.
        В конкретном решении мы пытались первым делом найти $\eli_2$, но, возможно, в вашем варианте будет быстрее решать систему в другом порядке.
        Мы всё же проделаем всё в лоб, подробно и целиком.
    \end{itemize}


    \begin{tikzpicture}[circuit ee IEC, thick]
        \foreach \contact/\x in {1/0, 2/3, 3/6}
        {
            \node [contact] (top contact \contact) at (\x, 0) {};
            \node [contact] (bottom contact \contact) at (\x, 4) {};
       }
        \draw  (bottom contact 1) -- (bottom contact 2) -- (bottom contact 3);
        \draw  (top contact 1) -- (top contact 2) -- (top contact 3);
        \draw  (bottom contact 1) to [resistor={near start, info=$R_1$}, current direction'={midway, info=$\eli_1$}, battery={near end, info=$\ele_1$}] (top contact 1);
        \draw  (bottom contact 2) to [resistor={near start, info=$R_2$}, current direction'={midway, info=$\eli_2$}, battery={near end, info=$\ele_2$}] (top contact 2);
        \draw  (bottom contact 3) to [resistor={near start, info=$R_3$}, current direction'={midway, info=$\eli_3$}, battery={near end, info=$\ele_3$}] (top contact 3);
        \draw [-{Latex},color=red] (1.2, 2.5) arc [start angle = 135, end angle = -160, radius = 0.6];
        \draw [-{Latex},color=blue] (4.2, 2.5) arc [start angle = 135, end angle = -160, radius = 0.6];
        \node [contact,color=green!71!black] (bottomc) at (bottom contact 2) {};
    \end{tikzpicture}

    \begin{align*}
        &\begin{cases}
            {\color{red} \eli_1R_1 - \eli_2R_2 = \ele_1 - \ele_2}, \\
            {\color{blue} \eli_2R_2 - \eli_3R_3 = \ele_2 - \ele_3}, \\
            {\color{green!71!black} \eli_1 + \eli_2 + \eli_3 = 0};
        \end{cases}
        \qquad \implies \qquad
        \begin{cases}
            \eli_1 = \frac{\ele_1 - \ele_2 + \eli_2R_2}{R_1}, \\
            \eli_3 = \frac{\eli_2R_2 - \ele_2 + \ele_3}{R_3}, \\
            \eli_1 + \eli_2 + \eli_3 = 0, \\
        \end{cases} \implies \\
        \implies
            &\eli_2 + \frac{\ele_1 - \ele_2 + \eli_2R_2}{R_1} + \frac{\eli_2R_2 - \ele_2 + \ele_3}{R_3} = 0, \\
        &   \eli_2\cbr{1 + \frac{ R_2 }{ R_1 } + \frac{ R_2 }{ R_3 }} + \frac{\ele_1 - \ele_2}{ R_1 } + \frac{\ele_3 - \ele_2}{ R_3 } = 0, \\
        &   \eli_2 = \cfrac{\cfrac{\ele_2 - \ele_1}{ R_1 } + \cfrac{\ele_2 - \ele_3}{ R_3 }}{1 + \cfrac{ R_2 }{ R_1 } + \cfrac{ R_2 }{ R_3 }}
            = \cfrac{\cfrac{6\,\text{В} - 5\,\text{В}}{ 3\,\text{Ом} } + \cfrac{6\,\text{В} - 2\,\text{В}}{ 10\,\text{Ом} }}{1 + \cfrac{ 5\,\text{Ом} }{ 3\,\text{Ом} } + \cfrac{ 5\,\text{Ом} }{ 10\,\text{Ом} }}
            = \frac{22}{95}\units{А} \approx 0{,}23\,\text{А}, \\
        &   U_2 = \eli_2R_2 = \cfrac{\cfrac{\ele_2 - \ele_1}{ R_1 } + \cfrac{\ele_2 - \ele_3}{ R_3 }}{1 + \cfrac{ R_2 }{ R_1 } + \cfrac{ R_2 }{ R_3 }} \cdot R_2
            = \cfrac{\cfrac{6\,\text{В} - 5\,\text{В}}{ 3\,\text{Ом} } + \cfrac{6\,\text{В} - 2\,\text{В}}{ 10\,\text{Ом} }}{1 + \cfrac{ 5\,\text{Ом} }{ 3\,\text{Ом} } + \cfrac{ 5\,\text{Ом} }{ 10\,\text{Ом} }} \cdot 5\,\text{Ом}
            = \frac{22}{95}\units{А} \cdot 5\,\text{Ом} = \frac{22}{19}\units{В} \approx 1{,}16\,\text{В}.
    \end{align*}

    Одну пару силы тока и напряжения получили.
    Для некоторых вариантов это уже ответ, но не у всех.
    Для упрощения записи преобразуем (чтобы избавитсья от 4-этажной дроби) и подставим в уже полученные уравнения:

    \begin{align*}
    \eli_2
        &=
        \frac{\frac{\ele_2 - \ele_1}{ R_1 } + \frac{\ele_2 - \ele_3}{ R_3 }}{1 + \frac{ R_2 }{ R_1 } + \frac{ R_2 }{ R_3 }}
        =
        \frac{(\ele_2 - \ele_1)R_3 + (\ele_2 - \ele_3)R_1}{R_1R_3 + R_2R_3 + R_2R_1},
        \\
    \eli_1
        &=  \frac{\ele_1 - \ele_2 + \eli_2R_2}{R_1}
        =   \frac{\ele_1 - \ele_2 + \cfrac{(\ele_2 - \ele_1)R_3 + (\ele_2 - \ele_3)R_1}{R_1R_3 + R_2R_3 + R_2R_1} \cdot R_2}{R_1} = \\
        &=  \frac{
            \ele_1R_1R_3 + \ele_1R_2R_3 + \ele_1R_2R_1
            - \ele_2R_1R_3 - \ele_2R_2R_3 - \ele_2R_2R_1
            + \ele_2R_3R_2 - \ele_1R_3R_2 + \ele_2R_1R_2 - \ele_3R_1R_2
       }{R_1 \cdot \cbr{R_1R_3 + R_2R_3 + R_2R_1}}
        = \\ &=
        \frac{
            \ele_1\cbr{R_1R_3 + R_2R_3 + R_2R_1 - R_3R_2}
            + \ele_2\cbr{- R_1R_3 - R_2R_3 - R_2R_1 + R_3R_2 + R_1R_2}
            - \ele_3R_1R_2
       }{R_1 \cdot \cbr{R_1R_3 + R_2R_3 + R_2R_1}}
        = \\ &=
        \frac{
            \ele_1\cbr{R_1R_3 + R_2R_1}
            + \ele_2\cbr{- R_1R_3}
            - \ele_3R_1R_2
       }{R_1 \cdot \cbr{R_1R_3 + R_2R_3 + R_2R_1}}
        =
        \frac{
            \ele_1\cbr{R_3 + R_2} - \ele_2R_3 - \ele_3R_2
       }{R_1R_3 + R_2R_3 + R_2R_1}
        = \\ &=
        \frac{
            (\ele_1 - \ele_3)R_2 + (\ele_1 - \ele_2)R_3
       }{R_1R_3 + R_2R_3 + R_2R_1}
        =
        \frac{
            \cfrac{\ele_1 - \ele_3}{ R_3 } + \cfrac{\ele_1 - \ele_2}{ R_2 }
       }{\cfrac{ R_1 }{ R_2 } + 1 + \cfrac{ R_1 }{ R_3 }}
        =
        \frac{
            \cfrac{5\,\text{В} - 2\,\text{В}}{ 10\,\text{Ом} } + \cfrac{5\,\text{В} - 6\,\text{В}}{ 5\,\text{Ом} }
       }{\cfrac{ 3\,\text{Ом} }{ 5\,\text{Ом} } + 1 + \cfrac{ 3\,\text{Ом} }{ 10\,\text{Ом} }}
        = \frac1{19}\units{А} \approx 0{,}05\,\text{А}.
        \\
    U_1
        &=
        \eli_1R_1
        =
        \frac{
            \cfrac{\ele_1 - \ele_3}{ R_3 } + \cfrac{\ele_1 - \ele_2}{ R_2 }
       }{\cfrac{ R_1 }{ R_2 } + 1 + \cfrac{ R_1 }{ R_3 }} \cdot R_1
        =
        \frac1{19}\units{А} \cdot 3\,\text{Ом} = \frac3{19}\units{В} \approx 0{,}16\,\text{В}.
    \end{align*}

    Если вы проделали все эти вычисления выше вместе со мной, то
    \begin{itemize}
        \item вы совершили ошибку, выбрав неверный путь решения:
        слишком длинное решение, очень легко ошибиться в индексах, дробях, знаках или потерять какой-то множитель,
        \item можно было выразить из исходной системы другие токи и получить сразу нажный вам,
        а не какой-то 2-й,
        \item можно было сэкономить: все три резистора и ЭДС соединены одинаково,
        поэтому ответ для 1-го резистора должен отличаться лишь перестановкой индексов (этот факт крайне полезен при проверке ответа, у нас всё сошлось),
        я специально подгонял выражение для $\eli_1$ к этому виду, вынося за скобки и преобразуя дробь,
        \item вы молодец, потому что не побоялись и получили верный ответ грамотным способом,
    \end{itemize}
    так что переходим к третьему резистору.
    Будет похоже, но кого это когда останавливало...

    \begin{align*}
    \eli_3
        &=  \frac{\eli_2R_2 - \ele_2 + \ele_3}{ R_3 }
        =
        \cfrac{
            \cfrac{
                (\ele_2 - \ele_1)R_3 + (\ele_2 - \ele_3)R_1
           }{
                R_1R_3 + R_2R_3 + R_2R_1
           } \cdot R_2 - \ele_2 + \ele_3}{ R_3 }
        = \\ &=
        \frac{
            \ele_2R_3R_2 - \ele_1R_3R_2 + \ele_2R_1R_2 - \ele_3R_1R_2
            - \ele_2R_1R_3 - \ele_2R_2R_3 - \ele_2R_2R_1
            + \ele_3R_1R_3 + \ele_3R_2R_3 + \ele_3R_2R_1
       }{\cbr{R_1R_3 + R_2R_3 + R_2R_1} \cdot R_3}
        = \\ &=
        \frac{
            - \ele_1R_3R_2 - \ele_2R_1R_3 + \ele_3R_1R_3 + \ele_3R_2R_3
       }{\cbr{R_1R_3 + R_2R_3 + R_2R_1} \cdot R_3}
        =
        \frac{
            - \ele_1R_2 - \ele_2R_1 + \ele_3R_1 + \ele_3R_2
       }{R_1R_3 + R_2R_3 + R_2R_1}
        = \\ &=
        \frac{
            R_1(\ele_3 - \ele_2) + R_2(\ele_3 - \ele_1)
       }{R_1R_3 + R_2R_3 + R_2R_1}
        =
        \frac{
            \cfrac{\ele_3 - \ele_2}{ R_2 } + \cfrac{\ele_3 - \ele_1}{ R_1 }
       }{\cfrac{ R_3 }{ R_2 } + \cfrac{ R_3 }{ R_1 } + 1}
        =
        \frac{
            \cfrac{2\,\text{В} - 6\,\text{В}}{ 5\,\text{Ом} } + \cfrac{2\,\text{В} - 5\,\text{В}}{ 3\,\text{Ом} }
       }{\cfrac{ 10\,\text{Ом} }{ 5\,\text{Ом} } + \cfrac{ 10\,\text{Ом} }{ 3\,\text{Ом} } + 1}
        = -\frac{27}{95}\units{А} \approx -0{,}2800\,\text{А}.
        \\
    U_3
        &=
        \eli_3R_3
        =
        \frac{
            \cfrac{\ele_3 - \ele_2}{ R_2 } + \cfrac{\ele_3 - \ele_1}{ R_1 }
       }{\cfrac{ R_3 }{ R_2 } + \cfrac{ R_3 }{ R_1 } + 1} \cdot R_3
        =
        -\frac{27}{95}\units{А} \cdot 10\,\text{Ом} = -\frac{54}{19}\units{В} \approx -2{,}840\,\text{В}.
    \end{align*}

    Положительные ответы говорят, что мы угадали на рисунке направление тока (тут нет нашей заслуги, повезло),
    отрицательные — что не угадали (и в этом нет ошибки), и ток течёт в противоположную сторону.
    Напомним, что направление тока — это направление движения положительных зарядов,
    а в металлах носители заряда — электроны, которые заряжены отрицательно.
}

\variantsplitter

\addpersonalvariant{Варвара Минаева}

\tasknumber{1}%
\task{%
    Определите ток $\eli_3$, протекающий через резистор $R_3$ (см.
    рис.),
    направление этого тока и разность потенциалов $U_3$ на этом резисторе,
    если $R_1 = 4\,\text{Ом}$, $R_2 = 5\,\text{Ом}$, $R_3 = 15\,\text{Ом}$, $\ele_1 = 5\,\text{В}$, $\ele_2 = 3\,\text{В}$, $\ele_3 = 2\,\text{В}$.
    Внутренним сопротивлением всех трёх ЭДС пренебречь.
    Ответы получите в виде несократимых дробей, а также определите приближённые значения.

    \begin{tikzpicture}[circuit ee IEC, thick]
        \foreach \contact/\x in {1/0, 2/3, 3/6}
        {
            \node [contact] (top contact \contact) at (\x, 0) {};
            \node [contact] (bottom contact \contact) at (\x, 4) {};
       }
        \draw  (bottom contact 1) -- (bottom contact 2) -- (bottom contact 3);
        \draw  (top contact 1) -- (top contact 2) -- (top contact 3);
        \draw  (bottom contact 1) to [resistor={near start, info=$R_1$}, battery={near end, info=$\ele_1$}] (top contact 1);
        \draw  (bottom contact 2) to [resistor={near start, info=$R_2$}, battery={near end, info=$\ele_2$}] (top contact 2);
        \draw  (bottom contact 3) to [resistor={near start, info=$R_3$}, battery={near end, info=$\ele_3$}] (top contact 3);
    \end{tikzpicture}
}
\answer{%
    План:
    \begin{itemize}
        \item отметим на рисунке произвольно направления токов (если получим отрицательный ответ, значит не угадали направление и только),
        \item выберем и обозначим на рисунке контуры (здесь всего 3, значит будет нужно $3-1=2$), для них запишем законы Кирхгофа,
        \item выберем и выделим на рисунке нетривиальные узлы (здесь всего 2, значит будет нужно $2-1=1$), для него запишем закон Кирхгофа,
        \item попытаемся решить получившуюся систему.
        В конкретном решении мы пытались первым делом найти $\eli_2$, но, возможно, в вашем варианте будет быстрее решать систему в другом порядке.
        Мы всё же проделаем всё в лоб, подробно и целиком.
    \end{itemize}


    \begin{tikzpicture}[circuit ee IEC, thick]
        \foreach \contact/\x in {1/0, 2/3, 3/6}
        {
            \node [contact] (top contact \contact) at (\x, 0) {};
            \node [contact] (bottom contact \contact) at (\x, 4) {};
       }
        \draw  (bottom contact 1) -- (bottom contact 2) -- (bottom contact 3);
        \draw  (top contact 1) -- (top contact 2) -- (top contact 3);
        \draw  (bottom contact 1) to [resistor={near start, info=$R_1$}, current direction'={midway, info=$\eli_1$}, battery={near end, info=$\ele_1$}] (top contact 1);
        \draw  (bottom contact 2) to [resistor={near start, info=$R_2$}, current direction'={midway, info=$\eli_2$}, battery={near end, info=$\ele_2$}] (top contact 2);
        \draw  (bottom contact 3) to [resistor={near start, info=$R_3$}, current direction'={midway, info=$\eli_3$}, battery={near end, info=$\ele_3$}] (top contact 3);
        \draw [-{Latex},color=red] (1.2, 2.5) arc [start angle = 135, end angle = -160, radius = 0.6];
        \draw [-{Latex},color=blue] (4.2, 2.5) arc [start angle = 135, end angle = -160, radius = 0.6];
        \node [contact,color=green!71!black] (bottomc) at (bottom contact 2) {};
    \end{tikzpicture}

    \begin{align*}
        &\begin{cases}
            {\color{red} \eli_1R_1 - \eli_2R_2 = \ele_1 - \ele_2}, \\
            {\color{blue} \eli_2R_2 - \eli_3R_3 = \ele_2 - \ele_3}, \\
            {\color{green!71!black} \eli_1 + \eli_2 + \eli_3 = 0};
        \end{cases}
        \qquad \implies \qquad
        \begin{cases}
            \eli_1 = \frac{\ele_1 - \ele_2 + \eli_2R_2}{R_1}, \\
            \eli_3 = \frac{\eli_2R_2 - \ele_2 + \ele_3}{R_3}, \\
            \eli_1 + \eli_2 + \eli_3 = 0, \\
        \end{cases} \implies \\
        \implies
            &\eli_2 + \frac{\ele_1 - \ele_2 + \eli_2R_2}{R_1} + \frac{\eli_2R_2 - \ele_2 + \ele_3}{R_3} = 0, \\
        &   \eli_2\cbr{1 + \frac{ R_2 }{ R_1 } + \frac{ R_2 }{ R_3 }} + \frac{\ele_1 - \ele_2}{ R_1 } + \frac{\ele_3 - \ele_2}{ R_3 } = 0, \\
        &   \eli_2 = \cfrac{\cfrac{\ele_2 - \ele_1}{ R_1 } + \cfrac{\ele_2 - \ele_3}{ R_3 }}{1 + \cfrac{ R_2 }{ R_1 } + \cfrac{ R_2 }{ R_3 }}
            = \cfrac{\cfrac{3\,\text{В} - 5\,\text{В}}{ 4\,\text{Ом} } + \cfrac{3\,\text{В} - 2\,\text{В}}{ 15\,\text{Ом} }}{1 + \cfrac{ 5\,\text{Ом} }{ 4\,\text{Ом} } + \cfrac{ 5\,\text{Ом} }{ 15\,\text{Ом} }}
            = -\frac{26}{155}\units{А} \approx -0{,}17000\,\text{А}, \\
        &   U_2 = \eli_2R_2 = \cfrac{\cfrac{\ele_2 - \ele_1}{ R_1 } + \cfrac{\ele_2 - \ele_3}{ R_3 }}{1 + \cfrac{ R_2 }{ R_1 } + \cfrac{ R_2 }{ R_3 }} \cdot R_2
            = \cfrac{\cfrac{3\,\text{В} - 5\,\text{В}}{ 4\,\text{Ом} } + \cfrac{3\,\text{В} - 2\,\text{В}}{ 15\,\text{Ом} }}{1 + \cfrac{ 5\,\text{Ом} }{ 4\,\text{Ом} } + \cfrac{ 5\,\text{Ом} }{ 15\,\text{Ом} }} \cdot 5\,\text{Ом}
            = -\frac{26}{155}\units{А} \cdot 5\,\text{Ом} = -\frac{26}{31}\units{В} \approx -0{,}8400\,\text{В}.
    \end{align*}

    Одну пару силы тока и напряжения получили.
    Для некоторых вариантов это уже ответ, но не у всех.
    Для упрощения записи преобразуем (чтобы избавитсья от 4-этажной дроби) и подставим в уже полученные уравнения:

    \begin{align*}
    \eli_2
        &=
        \frac{\frac{\ele_2 - \ele_1}{ R_1 } + \frac{\ele_2 - \ele_3}{ R_3 }}{1 + \frac{ R_2 }{ R_1 } + \frac{ R_2 }{ R_3 }}
        =
        \frac{(\ele_2 - \ele_1)R_3 + (\ele_2 - \ele_3)R_1}{R_1R_3 + R_2R_3 + R_2R_1},
        \\
    \eli_1
        &=  \frac{\ele_1 - \ele_2 + \eli_2R_2}{R_1}
        =   \frac{\ele_1 - \ele_2 + \cfrac{(\ele_2 - \ele_1)R_3 + (\ele_2 - \ele_3)R_1}{R_1R_3 + R_2R_3 + R_2R_1} \cdot R_2}{R_1} = \\
        &=  \frac{
            \ele_1R_1R_3 + \ele_1R_2R_3 + \ele_1R_2R_1
            - \ele_2R_1R_3 - \ele_2R_2R_3 - \ele_2R_2R_1
            + \ele_2R_3R_2 - \ele_1R_3R_2 + \ele_2R_1R_2 - \ele_3R_1R_2
       }{R_1 \cdot \cbr{R_1R_3 + R_2R_3 + R_2R_1}}
        = \\ &=
        \frac{
            \ele_1\cbr{R_1R_3 + R_2R_3 + R_2R_1 - R_3R_2}
            + \ele_2\cbr{- R_1R_3 - R_2R_3 - R_2R_1 + R_3R_2 + R_1R_2}
            - \ele_3R_1R_2
       }{R_1 \cdot \cbr{R_1R_3 + R_2R_3 + R_2R_1}}
        = \\ &=
        \frac{
            \ele_1\cbr{R_1R_3 + R_2R_1}
            + \ele_2\cbr{- R_1R_3}
            - \ele_3R_1R_2
       }{R_1 \cdot \cbr{R_1R_3 + R_2R_3 + R_2R_1}}
        =
        \frac{
            \ele_1\cbr{R_3 + R_2} - \ele_2R_3 - \ele_3R_2
       }{R_1R_3 + R_2R_3 + R_2R_1}
        = \\ &=
        \frac{
            (\ele_1 - \ele_3)R_2 + (\ele_1 - \ele_2)R_3
       }{R_1R_3 + R_2R_3 + R_2R_1}
        =
        \frac{
            \cfrac{\ele_1 - \ele_3}{ R_3 } + \cfrac{\ele_1 - \ele_2}{ R_2 }
       }{\cfrac{ R_1 }{ R_2 } + 1 + \cfrac{ R_1 }{ R_3 }}
        =
        \frac{
            \cfrac{5\,\text{В} - 2\,\text{В}}{ 15\,\text{Ом} } + \cfrac{5\,\text{В} - 3\,\text{В}}{ 5\,\text{Ом} }
       }{\cfrac{ 4\,\text{Ом} }{ 5\,\text{Ом} } + 1 + \cfrac{ 4\,\text{Ом} }{ 15\,\text{Ом} }}
        = \frac9{31}\units{А} \approx 0{,}29\,\text{А}.
        \\
    U_1
        &=
        \eli_1R_1
        =
        \frac{
            \cfrac{\ele_1 - \ele_3}{ R_3 } + \cfrac{\ele_1 - \ele_2}{ R_2 }
       }{\cfrac{ R_1 }{ R_2 } + 1 + \cfrac{ R_1 }{ R_3 }} \cdot R_1
        =
        \frac9{31}\units{А} \cdot 4\,\text{Ом} = \frac{36}{31}\units{В} \approx 1{,}16\,\text{В}.
    \end{align*}

    Если вы проделали все эти вычисления выше вместе со мной, то
    \begin{itemize}
        \item вы совершили ошибку, выбрав неверный путь решения:
        слишком длинное решение, очень легко ошибиться в индексах, дробях, знаках или потерять какой-то множитель,
        \item можно было выразить из исходной системы другие токи и получить сразу нажный вам,
        а не какой-то 2-й,
        \item можно было сэкономить: все три резистора и ЭДС соединены одинаково,
        поэтому ответ для 1-го резистора должен отличаться лишь перестановкой индексов (этот факт крайне полезен при проверке ответа, у нас всё сошлось),
        я специально подгонял выражение для $\eli_1$ к этому виду, вынося за скобки и преобразуя дробь,
        \item вы молодец, потому что не побоялись и получили верный ответ грамотным способом,
    \end{itemize}
    так что переходим к третьему резистору.
    Будет похоже, но кого это когда останавливало...

    \begin{align*}
    \eli_3
        &=  \frac{\eli_2R_2 - \ele_2 + \ele_3}{ R_3 }
        =
        \cfrac{
            \cfrac{
                (\ele_2 - \ele_1)R_3 + (\ele_2 - \ele_3)R_1
           }{
                R_1R_3 + R_2R_3 + R_2R_1
           } \cdot R_2 - \ele_2 + \ele_3}{ R_3 }
        = \\ &=
        \frac{
            \ele_2R_3R_2 - \ele_1R_3R_2 + \ele_2R_1R_2 - \ele_3R_1R_2
            - \ele_2R_1R_3 - \ele_2R_2R_3 - \ele_2R_2R_1
            + \ele_3R_1R_3 + \ele_3R_2R_3 + \ele_3R_2R_1
       }{\cbr{R_1R_3 + R_2R_3 + R_2R_1} \cdot R_3}
        = \\ &=
        \frac{
            - \ele_1R_3R_2 - \ele_2R_1R_3 + \ele_3R_1R_3 + \ele_3R_2R_3
       }{\cbr{R_1R_3 + R_2R_3 + R_2R_1} \cdot R_3}
        =
        \frac{
            - \ele_1R_2 - \ele_2R_1 + \ele_3R_1 + \ele_3R_2
       }{R_1R_3 + R_2R_3 + R_2R_1}
        = \\ &=
        \frac{
            R_1(\ele_3 - \ele_2) + R_2(\ele_3 - \ele_1)
       }{R_1R_3 + R_2R_3 + R_2R_1}
        =
        \frac{
            \cfrac{\ele_3 - \ele_2}{ R_2 } + \cfrac{\ele_3 - \ele_1}{ R_1 }
       }{\cfrac{ R_3 }{ R_2 } + \cfrac{ R_3 }{ R_1 } + 1}
        =
        \frac{
            \cfrac{2\,\text{В} - 3\,\text{В}}{ 5\,\text{Ом} } + \cfrac{2\,\text{В} - 5\,\text{В}}{ 4\,\text{Ом} }
       }{\cfrac{ 15\,\text{Ом} }{ 5\,\text{Ом} } + \cfrac{ 15\,\text{Ом} }{ 4\,\text{Ом} } + 1}
        = -\frac{19}{155}\units{А} \approx -0{,}12000\,\text{А}.
        \\
    U_3
        &=
        \eli_3R_3
        =
        \frac{
            \cfrac{\ele_3 - \ele_2}{ R_2 } + \cfrac{\ele_3 - \ele_1}{ R_1 }
       }{\cfrac{ R_3 }{ R_2 } + \cfrac{ R_3 }{ R_1 } + 1} \cdot R_3
        =
        -\frac{19}{155}\units{А} \cdot 15\,\text{Ом} = -\frac{57}{31}\units{В} \approx -1{,}8400\,\text{В}.
    \end{align*}

    Положительные ответы говорят, что мы угадали на рисунке направление тока (тут нет нашей заслуги, повезло),
    отрицательные — что не угадали (и в этом нет ошибки), и ток течёт в противоположную сторону.
    Напомним, что направление тока — это направление движения положительных зарядов,
    а в металлах носители заряда — электроны, которые заряжены отрицательно.
}

\variantsplitter

\addpersonalvariant{Леонид Никитин}

\tasknumber{1}%
\task{%
    Определите ток $\eli_3$, протекающий через резистор $R_3$ (см.
    рис.),
    направление этого тока и разность потенциалов $U_3$ на этом резисторе,
    если $R_1 = 2\,\text{Ом}$, $R_2 = 8\,\text{Ом}$, $R_3 = 15\,\text{Ом}$, $\ele_1 = 5\,\text{В}$, $\ele_2 = 6\,\text{В}$, $\ele_3 = 2\,\text{В}$.
    Внутренним сопротивлением всех трёх ЭДС пренебречь.
    Ответы получите в виде несократимых дробей, а также определите приближённые значения.

    \begin{tikzpicture}[circuit ee IEC, thick]
        \foreach \contact/\x in {1/0, 2/3, 3/6}
        {
            \node [contact] (top contact \contact) at (\x, 0) {};
            \node [contact] (bottom contact \contact) at (\x, 4) {};
       }
        \draw  (bottom contact 1) -- (bottom contact 2) -- (bottom contact 3);
        \draw  (top contact 1) -- (top contact 2) -- (top contact 3);
        \draw  (bottom contact 1) to [resistor={near start, info=$R_1$}, battery={near end, info=$\ele_1$}] (top contact 1);
        \draw  (bottom contact 2) to [resistor={near start, info=$R_2$}, battery={near end, info=$\ele_2$}] (top contact 2);
        \draw  (bottom contact 3) to [resistor={near start, info=$R_3$}, battery={near end, info=$\ele_3$}] (top contact 3);
    \end{tikzpicture}
}
\answer{%
    План:
    \begin{itemize}
        \item отметим на рисунке произвольно направления токов (если получим отрицательный ответ, значит не угадали направление и только),
        \item выберем и обозначим на рисунке контуры (здесь всего 3, значит будет нужно $3-1=2$), для них запишем законы Кирхгофа,
        \item выберем и выделим на рисунке нетривиальные узлы (здесь всего 2, значит будет нужно $2-1=1$), для него запишем закон Кирхгофа,
        \item попытаемся решить получившуюся систему.
        В конкретном решении мы пытались первым делом найти $\eli_2$, но, возможно, в вашем варианте будет быстрее решать систему в другом порядке.
        Мы всё же проделаем всё в лоб, подробно и целиком.
    \end{itemize}


    \begin{tikzpicture}[circuit ee IEC, thick]
        \foreach \contact/\x in {1/0, 2/3, 3/6}
        {
            \node [contact] (top contact \contact) at (\x, 0) {};
            \node [contact] (bottom contact \contact) at (\x, 4) {};
       }
        \draw  (bottom contact 1) -- (bottom contact 2) -- (bottom contact 3);
        \draw  (top contact 1) -- (top contact 2) -- (top contact 3);
        \draw  (bottom contact 1) to [resistor={near start, info=$R_1$}, current direction'={midway, info=$\eli_1$}, battery={near end, info=$\ele_1$}] (top contact 1);
        \draw  (bottom contact 2) to [resistor={near start, info=$R_2$}, current direction'={midway, info=$\eli_2$}, battery={near end, info=$\ele_2$}] (top contact 2);
        \draw  (bottom contact 3) to [resistor={near start, info=$R_3$}, current direction'={midway, info=$\eli_3$}, battery={near end, info=$\ele_3$}] (top contact 3);
        \draw [-{Latex},color=red] (1.2, 2.5) arc [start angle = 135, end angle = -160, radius = 0.6];
        \draw [-{Latex},color=blue] (4.2, 2.5) arc [start angle = 135, end angle = -160, radius = 0.6];
        \node [contact,color=green!71!black] (bottomc) at (bottom contact 2) {};
    \end{tikzpicture}

    \begin{align*}
        &\begin{cases}
            {\color{red} \eli_1R_1 - \eli_2R_2 = \ele_1 - \ele_2}, \\
            {\color{blue} \eli_2R_2 - \eli_3R_3 = \ele_2 - \ele_3}, \\
            {\color{green!71!black} \eli_1 + \eli_2 + \eli_3 = 0};
        \end{cases}
        \qquad \implies \qquad
        \begin{cases}
            \eli_1 = \frac{\ele_1 - \ele_2 + \eli_2R_2}{R_1}, \\
            \eli_3 = \frac{\eli_2R_2 - \ele_2 + \ele_3}{R_3}, \\
            \eli_1 + \eli_2 + \eli_3 = 0, \\
        \end{cases} \implies \\
        \implies
            &\eli_2 + \frac{\ele_1 - \ele_2 + \eli_2R_2}{R_1} + \frac{\eli_2R_2 - \ele_2 + \ele_3}{R_3} = 0, \\
        &   \eli_2\cbr{1 + \frac{ R_2 }{ R_1 } + \frac{ R_2 }{ R_3 }} + \frac{\ele_1 - \ele_2}{ R_1 } + \frac{\ele_3 - \ele_2}{ R_3 } = 0, \\
        &   \eli_2 = \cfrac{\cfrac{\ele_2 - \ele_1}{ R_1 } + \cfrac{\ele_2 - \ele_3}{ R_3 }}{1 + \cfrac{ R_2 }{ R_1 } + \cfrac{ R_2 }{ R_3 }}
            = \cfrac{\cfrac{6\,\text{В} - 5\,\text{В}}{ 2\,\text{Ом} } + \cfrac{6\,\text{В} - 2\,\text{В}}{ 15\,\text{Ом} }}{1 + \cfrac{ 8\,\text{Ом} }{ 2\,\text{Ом} } + \cfrac{ 8\,\text{Ом} }{ 15\,\text{Ом} }}
            = \frac{23}{166}\units{А} \approx 0{,}14\,\text{А}, \\
        &   U_2 = \eli_2R_2 = \cfrac{\cfrac{\ele_2 - \ele_1}{ R_1 } + \cfrac{\ele_2 - \ele_3}{ R_3 }}{1 + \cfrac{ R_2 }{ R_1 } + \cfrac{ R_2 }{ R_3 }} \cdot R_2
            = \cfrac{\cfrac{6\,\text{В} - 5\,\text{В}}{ 2\,\text{Ом} } + \cfrac{6\,\text{В} - 2\,\text{В}}{ 15\,\text{Ом} }}{1 + \cfrac{ 8\,\text{Ом} }{ 2\,\text{Ом} } + \cfrac{ 8\,\text{Ом} }{ 15\,\text{Ом} }} \cdot 8\,\text{Ом}
            = \frac{23}{166}\units{А} \cdot 8\,\text{Ом} = \frac{92}{83}\units{В} \approx 1{,}11\,\text{В}.
    \end{align*}

    Одну пару силы тока и напряжения получили.
    Для некоторых вариантов это уже ответ, но не у всех.
    Для упрощения записи преобразуем (чтобы избавитсья от 4-этажной дроби) и подставим в уже полученные уравнения:

    \begin{align*}
    \eli_2
        &=
        \frac{\frac{\ele_2 - \ele_1}{ R_1 } + \frac{\ele_2 - \ele_3}{ R_3 }}{1 + \frac{ R_2 }{ R_1 } + \frac{ R_2 }{ R_3 }}
        =
        \frac{(\ele_2 - \ele_1)R_3 + (\ele_2 - \ele_3)R_1}{R_1R_3 + R_2R_3 + R_2R_1},
        \\
    \eli_1
        &=  \frac{\ele_1 - \ele_2 + \eli_2R_2}{R_1}
        =   \frac{\ele_1 - \ele_2 + \cfrac{(\ele_2 - \ele_1)R_3 + (\ele_2 - \ele_3)R_1}{R_1R_3 + R_2R_3 + R_2R_1} \cdot R_2}{R_1} = \\
        &=  \frac{
            \ele_1R_1R_3 + \ele_1R_2R_3 + \ele_1R_2R_1
            - \ele_2R_1R_3 - \ele_2R_2R_3 - \ele_2R_2R_1
            + \ele_2R_3R_2 - \ele_1R_3R_2 + \ele_2R_1R_2 - \ele_3R_1R_2
       }{R_1 \cdot \cbr{R_1R_3 + R_2R_3 + R_2R_1}}
        = \\ &=
        \frac{
            \ele_1\cbr{R_1R_3 + R_2R_3 + R_2R_1 - R_3R_2}
            + \ele_2\cbr{- R_1R_3 - R_2R_3 - R_2R_1 + R_3R_2 + R_1R_2}
            - \ele_3R_1R_2
       }{R_1 \cdot \cbr{R_1R_3 + R_2R_3 + R_2R_1}}
        = \\ &=
        \frac{
            \ele_1\cbr{R_1R_3 + R_2R_1}
            + \ele_2\cbr{- R_1R_3}
            - \ele_3R_1R_2
       }{R_1 \cdot \cbr{R_1R_3 + R_2R_3 + R_2R_1}}
        =
        \frac{
            \ele_1\cbr{R_3 + R_2} - \ele_2R_3 - \ele_3R_2
       }{R_1R_3 + R_2R_3 + R_2R_1}
        = \\ &=
        \frac{
            (\ele_1 - \ele_3)R_2 + (\ele_1 - \ele_2)R_3
       }{R_1R_3 + R_2R_3 + R_2R_1}
        =
        \frac{
            \cfrac{\ele_1 - \ele_3}{ R_3 } + \cfrac{\ele_1 - \ele_2}{ R_2 }
       }{\cfrac{ R_1 }{ R_2 } + 1 + \cfrac{ R_1 }{ R_3 }}
        =
        \frac{
            \cfrac{5\,\text{В} - 2\,\text{В}}{ 15\,\text{Ом} } + \cfrac{5\,\text{В} - 6\,\text{В}}{ 8\,\text{Ом} }
       }{\cfrac{ 2\,\text{Ом} }{ 8\,\text{Ом} } + 1 + \cfrac{ 2\,\text{Ом} }{ 15\,\text{Ом} }}
        = \frac9{166}\units{А} \approx 0{,}05\,\text{А}.
        \\
    U_1
        &=
        \eli_1R_1
        =
        \frac{
            \cfrac{\ele_1 - \ele_3}{ R_3 } + \cfrac{\ele_1 - \ele_2}{ R_2 }
       }{\cfrac{ R_1 }{ R_2 } + 1 + \cfrac{ R_1 }{ R_3 }} \cdot R_1
        =
        \frac9{166}\units{А} \cdot 2\,\text{Ом} = \frac9{83}\units{В} \approx 0{,}11\,\text{В}.
    \end{align*}

    Если вы проделали все эти вычисления выше вместе со мной, то
    \begin{itemize}
        \item вы совершили ошибку, выбрав неверный путь решения:
        слишком длинное решение, очень легко ошибиться в индексах, дробях, знаках или потерять какой-то множитель,
        \item можно было выразить из исходной системы другие токи и получить сразу нажный вам,
        а не какой-то 2-й,
        \item можно было сэкономить: все три резистора и ЭДС соединены одинаково,
        поэтому ответ для 1-го резистора должен отличаться лишь перестановкой индексов (этот факт крайне полезен при проверке ответа, у нас всё сошлось),
        я специально подгонял выражение для $\eli_1$ к этому виду, вынося за скобки и преобразуя дробь,
        \item вы молодец, потому что не побоялись и получили верный ответ грамотным способом,
    \end{itemize}
    так что переходим к третьему резистору.
    Будет похоже, но кого это когда останавливало...

    \begin{align*}
    \eli_3
        &=  \frac{\eli_2R_2 - \ele_2 + \ele_3}{ R_3 }
        =
        \cfrac{
            \cfrac{
                (\ele_2 - \ele_1)R_3 + (\ele_2 - \ele_3)R_1
           }{
                R_1R_3 + R_2R_3 + R_2R_1
           } \cdot R_2 - \ele_2 + \ele_3}{ R_3 }
        = \\ &=
        \frac{
            \ele_2R_3R_2 - \ele_1R_3R_2 + \ele_2R_1R_2 - \ele_3R_1R_2
            - \ele_2R_1R_3 - \ele_2R_2R_3 - \ele_2R_2R_1
            + \ele_3R_1R_3 + \ele_3R_2R_3 + \ele_3R_2R_1
       }{\cbr{R_1R_3 + R_2R_3 + R_2R_1} \cdot R_3}
        = \\ &=
        \frac{
            - \ele_1R_3R_2 - \ele_2R_1R_3 + \ele_3R_1R_3 + \ele_3R_2R_3
       }{\cbr{R_1R_3 + R_2R_3 + R_2R_1} \cdot R_3}
        =
        \frac{
            - \ele_1R_2 - \ele_2R_1 + \ele_3R_1 + \ele_3R_2
       }{R_1R_3 + R_2R_3 + R_2R_1}
        = \\ &=
        \frac{
            R_1(\ele_3 - \ele_2) + R_2(\ele_3 - \ele_1)
       }{R_1R_3 + R_2R_3 + R_2R_1}
        =
        \frac{
            \cfrac{\ele_3 - \ele_2}{ R_2 } + \cfrac{\ele_3 - \ele_1}{ R_1 }
       }{\cfrac{ R_3 }{ R_2 } + \cfrac{ R_3 }{ R_1 } + 1}
        =
        \frac{
            \cfrac{2\,\text{В} - 6\,\text{В}}{ 8\,\text{Ом} } + \cfrac{2\,\text{В} - 5\,\text{В}}{ 2\,\text{Ом} }
       }{\cfrac{ 15\,\text{Ом} }{ 8\,\text{Ом} } + \cfrac{ 15\,\text{Ом} }{ 2\,\text{Ом} } + 1}
        = -\frac{16}{83}\units{А} \approx -0{,}19000\,\text{А}.
        \\
    U_3
        &=
        \eli_3R_3
        =
        \frac{
            \cfrac{\ele_3 - \ele_2}{ R_2 } + \cfrac{\ele_3 - \ele_1}{ R_1 }
       }{\cfrac{ R_3 }{ R_2 } + \cfrac{ R_3 }{ R_1 } + 1} \cdot R_3
        =
        -\frac{16}{83}\units{А} \cdot 15\,\text{Ом} = -\frac{240}{83}\units{В} \approx -2{,}890\,\text{В}.
    \end{align*}

    Положительные ответы говорят, что мы угадали на рисунке направление тока (тут нет нашей заслуги, повезло),
    отрицательные — что не угадали (и в этом нет ошибки), и ток течёт в противоположную сторону.
    Напомним, что направление тока — это направление движения положительных зарядов,
    а в металлах носители заряда — электроны, которые заряжены отрицательно.
}

\variantsplitter

\addpersonalvariant{Тимофей Полетаев}

\tasknumber{1}%
\task{%
    Определите ток $\eli_3$, протекающий через резистор $R_3$ (см.
    рис.),
    направление этого тока и разность потенциалов $U_3$ на этом резисторе,
    если $R_1 = 2\,\text{Ом}$, $R_2 = 5\,\text{Ом}$, $R_3 = 15\,\text{Ом}$, $\ele_1 = 4\,\text{В}$, $\ele_2 = 6\,\text{В}$, $\ele_3 = 8\,\text{В}$.
    Внутренним сопротивлением всех трёх ЭДС пренебречь.
    Ответы получите в виде несократимых дробей, а также определите приближённые значения.

    \begin{tikzpicture}[circuit ee IEC, thick]
        \foreach \contact/\x in {1/0, 2/3, 3/6}
        {
            \node [contact] (top contact \contact) at (\x, 0) {};
            \node [contact] (bottom contact \contact) at (\x, 4) {};
       }
        \draw  (bottom contact 1) -- (bottom contact 2) -- (bottom contact 3);
        \draw  (top contact 1) -- (top contact 2) -- (top contact 3);
        \draw  (bottom contact 1) to [resistor={near start, info=$R_1$}, battery={near end, info=$\ele_1$}] (top contact 1);
        \draw  (bottom contact 2) to [resistor={near start, info=$R_2$}, battery={near end, info=$\ele_2$}] (top contact 2);
        \draw  (bottom contact 3) to [resistor={near start, info=$R_3$}, battery={near end, info=$\ele_3$}] (top contact 3);
    \end{tikzpicture}
}
\answer{%
    План:
    \begin{itemize}
        \item отметим на рисунке произвольно направления токов (если получим отрицательный ответ, значит не угадали направление и только),
        \item выберем и обозначим на рисунке контуры (здесь всего 3, значит будет нужно $3-1=2$), для них запишем законы Кирхгофа,
        \item выберем и выделим на рисунке нетривиальные узлы (здесь всего 2, значит будет нужно $2-1=1$), для него запишем закон Кирхгофа,
        \item попытаемся решить получившуюся систему.
        В конкретном решении мы пытались первым делом найти $\eli_2$, но, возможно, в вашем варианте будет быстрее решать систему в другом порядке.
        Мы всё же проделаем всё в лоб, подробно и целиком.
    \end{itemize}


    \begin{tikzpicture}[circuit ee IEC, thick]
        \foreach \contact/\x in {1/0, 2/3, 3/6}
        {
            \node [contact] (top contact \contact) at (\x, 0) {};
            \node [contact] (bottom contact \contact) at (\x, 4) {};
       }
        \draw  (bottom contact 1) -- (bottom contact 2) -- (bottom contact 3);
        \draw  (top contact 1) -- (top contact 2) -- (top contact 3);
        \draw  (bottom contact 1) to [resistor={near start, info=$R_1$}, current direction'={midway, info=$\eli_1$}, battery={near end, info=$\ele_1$}] (top contact 1);
        \draw  (bottom contact 2) to [resistor={near start, info=$R_2$}, current direction'={midway, info=$\eli_2$}, battery={near end, info=$\ele_2$}] (top contact 2);
        \draw  (bottom contact 3) to [resistor={near start, info=$R_3$}, current direction'={midway, info=$\eli_3$}, battery={near end, info=$\ele_3$}] (top contact 3);
        \draw [-{Latex},color=red] (1.2, 2.5) arc [start angle = 135, end angle = -160, radius = 0.6];
        \draw [-{Latex},color=blue] (4.2, 2.5) arc [start angle = 135, end angle = -160, radius = 0.6];
        \node [contact,color=green!71!black] (bottomc) at (bottom contact 2) {};
    \end{tikzpicture}

    \begin{align*}
        &\begin{cases}
            {\color{red} \eli_1R_1 - \eli_2R_2 = \ele_1 - \ele_2}, \\
            {\color{blue} \eli_2R_2 - \eli_3R_3 = \ele_2 - \ele_3}, \\
            {\color{green!71!black} \eli_1 + \eli_2 + \eli_3 = 0};
        \end{cases}
        \qquad \implies \qquad
        \begin{cases}
            \eli_1 = \frac{\ele_1 - \ele_2 + \eli_2R_2}{R_1}, \\
            \eli_3 = \frac{\eli_2R_2 - \ele_2 + \ele_3}{R_3}, \\
            \eli_1 + \eli_2 + \eli_3 = 0, \\
        \end{cases} \implies \\
        \implies
            &\eli_2 + \frac{\ele_1 - \ele_2 + \eli_2R_2}{R_1} + \frac{\eli_2R_2 - \ele_2 + \ele_3}{R_3} = 0, \\
        &   \eli_2\cbr{1 + \frac{ R_2 }{ R_1 } + \frac{ R_2 }{ R_3 }} + \frac{\ele_1 - \ele_2}{ R_1 } + \frac{\ele_3 - \ele_2}{ R_3 } = 0, \\
        &   \eli_2 = \cfrac{\cfrac{\ele_2 - \ele_1}{ R_1 } + \cfrac{\ele_2 - \ele_3}{ R_3 }}{1 + \cfrac{ R_2 }{ R_1 } + \cfrac{ R_2 }{ R_3 }}
            = \cfrac{\cfrac{6\,\text{В} - 4\,\text{В}}{ 2\,\text{Ом} } + \cfrac{6\,\text{В} - 8\,\text{В}}{ 15\,\text{Ом} }}{1 + \cfrac{ 5\,\text{Ом} }{ 2\,\text{Ом} } + \cfrac{ 5\,\text{Ом} }{ 15\,\text{Ом} }}
            = \frac{26}{115}\units{А} \approx 0{,}23\,\text{А}, \\
        &   U_2 = \eli_2R_2 = \cfrac{\cfrac{\ele_2 - \ele_1}{ R_1 } + \cfrac{\ele_2 - \ele_3}{ R_3 }}{1 + \cfrac{ R_2 }{ R_1 } + \cfrac{ R_2 }{ R_3 }} \cdot R_2
            = \cfrac{\cfrac{6\,\text{В} - 4\,\text{В}}{ 2\,\text{Ом} } + \cfrac{6\,\text{В} - 8\,\text{В}}{ 15\,\text{Ом} }}{1 + \cfrac{ 5\,\text{Ом} }{ 2\,\text{Ом} } + \cfrac{ 5\,\text{Ом} }{ 15\,\text{Ом} }} \cdot 5\,\text{Ом}
            = \frac{26}{115}\units{А} \cdot 5\,\text{Ом} = \frac{26}{23}\units{В} \approx 1{,}13\,\text{В}.
    \end{align*}

    Одну пару силы тока и напряжения получили.
    Для некоторых вариантов это уже ответ, но не у всех.
    Для упрощения записи преобразуем (чтобы избавитсья от 4-этажной дроби) и подставим в уже полученные уравнения:

    \begin{align*}
    \eli_2
        &=
        \frac{\frac{\ele_2 - \ele_1}{ R_1 } + \frac{\ele_2 - \ele_3}{ R_3 }}{1 + \frac{ R_2 }{ R_1 } + \frac{ R_2 }{ R_3 }}
        =
        \frac{(\ele_2 - \ele_1)R_3 + (\ele_2 - \ele_3)R_1}{R_1R_3 + R_2R_3 + R_2R_1},
        \\
    \eli_1
        &=  \frac{\ele_1 - \ele_2 + \eli_2R_2}{R_1}
        =   \frac{\ele_1 - \ele_2 + \cfrac{(\ele_2 - \ele_1)R_3 + (\ele_2 - \ele_3)R_1}{R_1R_3 + R_2R_3 + R_2R_1} \cdot R_2}{R_1} = \\
        &=  \frac{
            \ele_1R_1R_3 + \ele_1R_2R_3 + \ele_1R_2R_1
            - \ele_2R_1R_3 - \ele_2R_2R_3 - \ele_2R_2R_1
            + \ele_2R_3R_2 - \ele_1R_3R_2 + \ele_2R_1R_2 - \ele_3R_1R_2
       }{R_1 \cdot \cbr{R_1R_3 + R_2R_3 + R_2R_1}}
        = \\ &=
        \frac{
            \ele_1\cbr{R_1R_3 + R_2R_3 + R_2R_1 - R_3R_2}
            + \ele_2\cbr{- R_1R_3 - R_2R_3 - R_2R_1 + R_3R_2 + R_1R_2}
            - \ele_3R_1R_2
       }{R_1 \cdot \cbr{R_1R_3 + R_2R_3 + R_2R_1}}
        = \\ &=
        \frac{
            \ele_1\cbr{R_1R_3 + R_2R_1}
            + \ele_2\cbr{- R_1R_3}
            - \ele_3R_1R_2
       }{R_1 \cdot \cbr{R_1R_3 + R_2R_3 + R_2R_1}}
        =
        \frac{
            \ele_1\cbr{R_3 + R_2} - \ele_2R_3 - \ele_3R_2
       }{R_1R_3 + R_2R_3 + R_2R_1}
        = \\ &=
        \frac{
            (\ele_1 - \ele_3)R_2 + (\ele_1 - \ele_2)R_3
       }{R_1R_3 + R_2R_3 + R_2R_1}
        =
        \frac{
            \cfrac{\ele_1 - \ele_3}{ R_3 } + \cfrac{\ele_1 - \ele_2}{ R_2 }
       }{\cfrac{ R_1 }{ R_2 } + 1 + \cfrac{ R_1 }{ R_3 }}
        =
        \frac{
            \cfrac{4\,\text{В} - 8\,\text{В}}{ 15\,\text{Ом} } + \cfrac{4\,\text{В} - 6\,\text{В}}{ 5\,\text{Ом} }
       }{\cfrac{ 2\,\text{Ом} }{ 5\,\text{Ом} } + 1 + \cfrac{ 2\,\text{Ом} }{ 15\,\text{Ом} }}
        = -\frac{10}{23}\units{А} \approx -0{,}4300\,\text{А}.
        \\
    U_1
        &=
        \eli_1R_1
        =
        \frac{
            \cfrac{\ele_1 - \ele_3}{ R_3 } + \cfrac{\ele_1 - \ele_2}{ R_2 }
       }{\cfrac{ R_1 }{ R_2 } + 1 + \cfrac{ R_1 }{ R_3 }} \cdot R_1
        =
        -\frac{10}{23}\units{А} \cdot 2\,\text{Ом} = -\frac{20}{23}\units{В} \approx -0{,}8700\,\text{В}.
    \end{align*}

    Если вы проделали все эти вычисления выше вместе со мной, то
    \begin{itemize}
        \item вы совершили ошибку, выбрав неверный путь решения:
        слишком длинное решение, очень легко ошибиться в индексах, дробях, знаках или потерять какой-то множитель,
        \item можно было выразить из исходной системы другие токи и получить сразу нажный вам,
        а не какой-то 2-й,
        \item можно было сэкономить: все три резистора и ЭДС соединены одинаково,
        поэтому ответ для 1-го резистора должен отличаться лишь перестановкой индексов (этот факт крайне полезен при проверке ответа, у нас всё сошлось),
        я специально подгонял выражение для $\eli_1$ к этому виду, вынося за скобки и преобразуя дробь,
        \item вы молодец, потому что не побоялись и получили верный ответ грамотным способом,
    \end{itemize}
    так что переходим к третьему резистору.
    Будет похоже, но кого это когда останавливало...

    \begin{align*}
    \eli_3
        &=  \frac{\eli_2R_2 - \ele_2 + \ele_3}{ R_3 }
        =
        \cfrac{
            \cfrac{
                (\ele_2 - \ele_1)R_3 + (\ele_2 - \ele_3)R_1
           }{
                R_1R_3 + R_2R_3 + R_2R_1
           } \cdot R_2 - \ele_2 + \ele_3}{ R_3 }
        = \\ &=
        \frac{
            \ele_2R_3R_2 - \ele_1R_3R_2 + \ele_2R_1R_2 - \ele_3R_1R_2
            - \ele_2R_1R_3 - \ele_2R_2R_3 - \ele_2R_2R_1
            + \ele_3R_1R_3 + \ele_3R_2R_3 + \ele_3R_2R_1
       }{\cbr{R_1R_3 + R_2R_3 + R_2R_1} \cdot R_3}
        = \\ &=
        \frac{
            - \ele_1R_3R_2 - \ele_2R_1R_3 + \ele_3R_1R_3 + \ele_3R_2R_3
       }{\cbr{R_1R_3 + R_2R_3 + R_2R_1} \cdot R_3}
        =
        \frac{
            - \ele_1R_2 - \ele_2R_1 + \ele_3R_1 + \ele_3R_2
       }{R_1R_3 + R_2R_3 + R_2R_1}
        = \\ &=
        \frac{
            R_1(\ele_3 - \ele_2) + R_2(\ele_3 - \ele_1)
       }{R_1R_3 + R_2R_3 + R_2R_1}
        =
        \frac{
            \cfrac{\ele_3 - \ele_2}{ R_2 } + \cfrac{\ele_3 - \ele_1}{ R_1 }
       }{\cfrac{ R_3 }{ R_2 } + \cfrac{ R_3 }{ R_1 } + 1}
        =
        \frac{
            \cfrac{8\,\text{В} - 6\,\text{В}}{ 5\,\text{Ом} } + \cfrac{8\,\text{В} - 4\,\text{В}}{ 2\,\text{Ом} }
       }{\cfrac{ 15\,\text{Ом} }{ 5\,\text{Ом} } + \cfrac{ 15\,\text{Ом} }{ 2\,\text{Ом} } + 1}
        = \frac{24}{115}\units{А} \approx 0{,}21\,\text{А}.
        \\
    U_3
        &=
        \eli_3R_3
        =
        \frac{
            \cfrac{\ele_3 - \ele_2}{ R_2 } + \cfrac{\ele_3 - \ele_1}{ R_1 }
       }{\cfrac{ R_3 }{ R_2 } + \cfrac{ R_3 }{ R_1 } + 1} \cdot R_3
        =
        \frac{24}{115}\units{А} \cdot 15\,\text{Ом} = \frac{72}{23}\units{В} \approx 3{,}13\,\text{В}.
    \end{align*}

    Положительные ответы говорят, что мы угадали на рисунке направление тока (тут нет нашей заслуги, повезло),
    отрицательные — что не угадали (и в этом нет ошибки), и ток течёт в противоположную сторону.
    Напомним, что направление тока — это направление движения положительных зарядов,
    а в металлах носители заряда — электроны, которые заряжены отрицательно.
}

\variantsplitter

\addpersonalvariant{Андрей Рожков}

\tasknumber{1}%
\task{%
    Определите ток $\eli_3$, протекающий через резистор $R_3$ (см.
    рис.),
    направление этого тока и разность потенциалов $U_3$ на этом резисторе,
    если $R_1 = 4\,\text{Ом}$, $R_2 = 5\,\text{Ом}$, $R_3 = 10\,\text{Ом}$, $\ele_1 = 5\,\text{В}$, $\ele_2 = 3\,\text{В}$, $\ele_3 = 8\,\text{В}$.
    Внутренним сопротивлением всех трёх ЭДС пренебречь.
    Ответы получите в виде несократимых дробей, а также определите приближённые значения.

    \begin{tikzpicture}[circuit ee IEC, thick]
        \foreach \contact/\x in {1/0, 2/3, 3/6}
        {
            \node [contact] (top contact \contact) at (\x, 0) {};
            \node [contact] (bottom contact \contact) at (\x, 4) {};
       }
        \draw  (bottom contact 1) -- (bottom contact 2) -- (bottom contact 3);
        \draw  (top contact 1) -- (top contact 2) -- (top contact 3);
        \draw  (bottom contact 1) to [resistor={near start, info=$R_1$}, battery={near end, info=$\ele_1$}] (top contact 1);
        \draw  (bottom contact 2) to [resistor={near start, info=$R_2$}, battery={near end, info=$\ele_2$}] (top contact 2);
        \draw  (bottom contact 3) to [resistor={near start, info=$R_3$}, battery={near end, info=$\ele_3$}] (top contact 3);
    \end{tikzpicture}
}
\answer{%
    План:
    \begin{itemize}
        \item отметим на рисунке произвольно направления токов (если получим отрицательный ответ, значит не угадали направление и только),
        \item выберем и обозначим на рисунке контуры (здесь всего 3, значит будет нужно $3-1=2$), для них запишем законы Кирхгофа,
        \item выберем и выделим на рисунке нетривиальные узлы (здесь всего 2, значит будет нужно $2-1=1$), для него запишем закон Кирхгофа,
        \item попытаемся решить получившуюся систему.
        В конкретном решении мы пытались первым делом найти $\eli_2$, но, возможно, в вашем варианте будет быстрее решать систему в другом порядке.
        Мы всё же проделаем всё в лоб, подробно и целиком.
    \end{itemize}


    \begin{tikzpicture}[circuit ee IEC, thick]
        \foreach \contact/\x in {1/0, 2/3, 3/6}
        {
            \node [contact] (top contact \contact) at (\x, 0) {};
            \node [contact] (bottom contact \contact) at (\x, 4) {};
       }
        \draw  (bottom contact 1) -- (bottom contact 2) -- (bottom contact 3);
        \draw  (top contact 1) -- (top contact 2) -- (top contact 3);
        \draw  (bottom contact 1) to [resistor={near start, info=$R_1$}, current direction'={midway, info=$\eli_1$}, battery={near end, info=$\ele_1$}] (top contact 1);
        \draw  (bottom contact 2) to [resistor={near start, info=$R_2$}, current direction'={midway, info=$\eli_2$}, battery={near end, info=$\ele_2$}] (top contact 2);
        \draw  (bottom contact 3) to [resistor={near start, info=$R_3$}, current direction'={midway, info=$\eli_3$}, battery={near end, info=$\ele_3$}] (top contact 3);
        \draw [-{Latex},color=red] (1.2, 2.5) arc [start angle = 135, end angle = -160, radius = 0.6];
        \draw [-{Latex},color=blue] (4.2, 2.5) arc [start angle = 135, end angle = -160, radius = 0.6];
        \node [contact,color=green!71!black] (bottomc) at (bottom contact 2) {};
    \end{tikzpicture}

    \begin{align*}
        &\begin{cases}
            {\color{red} \eli_1R_1 - \eli_2R_2 = \ele_1 - \ele_2}, \\
            {\color{blue} \eli_2R_2 - \eli_3R_3 = \ele_2 - \ele_3}, \\
            {\color{green!71!black} \eli_1 + \eli_2 + \eli_3 = 0};
        \end{cases}
        \qquad \implies \qquad
        \begin{cases}
            \eli_1 = \frac{\ele_1 - \ele_2 + \eli_2R_2}{R_1}, \\
            \eli_3 = \frac{\eli_2R_2 - \ele_2 + \ele_3}{R_3}, \\
            \eli_1 + \eli_2 + \eli_3 = 0, \\
        \end{cases} \implies \\
        \implies
            &\eli_2 + \frac{\ele_1 - \ele_2 + \eli_2R_2}{R_1} + \frac{\eli_2R_2 - \ele_2 + \ele_3}{R_3} = 0, \\
        &   \eli_2\cbr{1 + \frac{ R_2 }{ R_1 } + \frac{ R_2 }{ R_3 }} + \frac{\ele_1 - \ele_2}{ R_1 } + \frac{\ele_3 - \ele_2}{ R_3 } = 0, \\
        &   \eli_2 = \cfrac{\cfrac{\ele_2 - \ele_1}{ R_1 } + \cfrac{\ele_2 - \ele_3}{ R_3 }}{1 + \cfrac{ R_2 }{ R_1 } + \cfrac{ R_2 }{ R_3 }}
            = \cfrac{\cfrac{3\,\text{В} - 5\,\text{В}}{ 4\,\text{Ом} } + \cfrac{3\,\text{В} - 8\,\text{В}}{ 10\,\text{Ом} }}{1 + \cfrac{ 5\,\text{Ом} }{ 4\,\text{Ом} } + \cfrac{ 5\,\text{Ом} }{ 10\,\text{Ом} }}
            = -\frac4{11}\units{А} \approx -0{,}3600\,\text{А}, \\
        &   U_2 = \eli_2R_2 = \cfrac{\cfrac{\ele_2 - \ele_1}{ R_1 } + \cfrac{\ele_2 - \ele_3}{ R_3 }}{1 + \cfrac{ R_2 }{ R_1 } + \cfrac{ R_2 }{ R_3 }} \cdot R_2
            = \cfrac{\cfrac{3\,\text{В} - 5\,\text{В}}{ 4\,\text{Ом} } + \cfrac{3\,\text{В} - 8\,\text{В}}{ 10\,\text{Ом} }}{1 + \cfrac{ 5\,\text{Ом} }{ 4\,\text{Ом} } + \cfrac{ 5\,\text{Ом} }{ 10\,\text{Ом} }} \cdot 5\,\text{Ом}
            = -\frac4{11}\units{А} \cdot 5\,\text{Ом} = -\frac{20}{11}\units{В} \approx -1{,}8200\,\text{В}.
    \end{align*}

    Одну пару силы тока и напряжения получили.
    Для некоторых вариантов это уже ответ, но не у всех.
    Для упрощения записи преобразуем (чтобы избавитсья от 4-этажной дроби) и подставим в уже полученные уравнения:

    \begin{align*}
    \eli_2
        &=
        \frac{\frac{\ele_2 - \ele_1}{ R_1 } + \frac{\ele_2 - \ele_3}{ R_3 }}{1 + \frac{ R_2 }{ R_1 } + \frac{ R_2 }{ R_3 }}
        =
        \frac{(\ele_2 - \ele_1)R_3 + (\ele_2 - \ele_3)R_1}{R_1R_3 + R_2R_3 + R_2R_1},
        \\
    \eli_1
        &=  \frac{\ele_1 - \ele_2 + \eli_2R_2}{R_1}
        =   \frac{\ele_1 - \ele_2 + \cfrac{(\ele_2 - \ele_1)R_3 + (\ele_2 - \ele_3)R_1}{R_1R_3 + R_2R_3 + R_2R_1} \cdot R_2}{R_1} = \\
        &=  \frac{
            \ele_1R_1R_3 + \ele_1R_2R_3 + \ele_1R_2R_1
            - \ele_2R_1R_3 - \ele_2R_2R_3 - \ele_2R_2R_1
            + \ele_2R_3R_2 - \ele_1R_3R_2 + \ele_2R_1R_2 - \ele_3R_1R_2
       }{R_1 \cdot \cbr{R_1R_3 + R_2R_3 + R_2R_1}}
        = \\ &=
        \frac{
            \ele_1\cbr{R_1R_3 + R_2R_3 + R_2R_1 - R_3R_2}
            + \ele_2\cbr{- R_1R_3 - R_2R_3 - R_2R_1 + R_3R_2 + R_1R_2}
            - \ele_3R_1R_2
       }{R_1 \cdot \cbr{R_1R_3 + R_2R_3 + R_2R_1}}
        = \\ &=
        \frac{
            \ele_1\cbr{R_1R_3 + R_2R_1}
            + \ele_2\cbr{- R_1R_3}
            - \ele_3R_1R_2
       }{R_1 \cdot \cbr{R_1R_3 + R_2R_3 + R_2R_1}}
        =
        \frac{
            \ele_1\cbr{R_3 + R_2} - \ele_2R_3 - \ele_3R_2
       }{R_1R_3 + R_2R_3 + R_2R_1}
        = \\ &=
        \frac{
            (\ele_1 - \ele_3)R_2 + (\ele_1 - \ele_2)R_3
       }{R_1R_3 + R_2R_3 + R_2R_1}
        =
        \frac{
            \cfrac{\ele_1 - \ele_3}{ R_3 } + \cfrac{\ele_1 - \ele_2}{ R_2 }
       }{\cfrac{ R_1 }{ R_2 } + 1 + \cfrac{ R_1 }{ R_3 }}
        =
        \frac{
            \cfrac{5\,\text{В} - 8\,\text{В}}{ 10\,\text{Ом} } + \cfrac{5\,\text{В} - 3\,\text{В}}{ 5\,\text{Ом} }
       }{\cfrac{ 4\,\text{Ом} }{ 5\,\text{Ом} } + 1 + \cfrac{ 4\,\text{Ом} }{ 10\,\text{Ом} }}
        = \frac1{22}\units{А} \approx 0{,}05\,\text{А}.
        \\
    U_1
        &=
        \eli_1R_1
        =
        \frac{
            \cfrac{\ele_1 - \ele_3}{ R_3 } + \cfrac{\ele_1 - \ele_2}{ R_2 }
       }{\cfrac{ R_1 }{ R_2 } + 1 + \cfrac{ R_1 }{ R_3 }} \cdot R_1
        =
        \frac1{22}\units{А} \cdot 4\,\text{Ом} = \frac2{11}\units{В} \approx 0{,}18\,\text{В}.
    \end{align*}

    Если вы проделали все эти вычисления выше вместе со мной, то
    \begin{itemize}
        \item вы совершили ошибку, выбрав неверный путь решения:
        слишком длинное решение, очень легко ошибиться в индексах, дробях, знаках или потерять какой-то множитель,
        \item можно было выразить из исходной системы другие токи и получить сразу нажный вам,
        а не какой-то 2-й,
        \item можно было сэкономить: все три резистора и ЭДС соединены одинаково,
        поэтому ответ для 1-го резистора должен отличаться лишь перестановкой индексов (этот факт крайне полезен при проверке ответа, у нас всё сошлось),
        я специально подгонял выражение для $\eli_1$ к этому виду, вынося за скобки и преобразуя дробь,
        \item вы молодец, потому что не побоялись и получили верный ответ грамотным способом,
    \end{itemize}
    так что переходим к третьему резистору.
    Будет похоже, но кого это когда останавливало...

    \begin{align*}
    \eli_3
        &=  \frac{\eli_2R_2 - \ele_2 + \ele_3}{ R_3 }
        =
        \cfrac{
            \cfrac{
                (\ele_2 - \ele_1)R_3 + (\ele_2 - \ele_3)R_1
           }{
                R_1R_3 + R_2R_3 + R_2R_1
           } \cdot R_2 - \ele_2 + \ele_3}{ R_3 }
        = \\ &=
        \frac{
            \ele_2R_3R_2 - \ele_1R_3R_2 + \ele_2R_1R_2 - \ele_3R_1R_2
            - \ele_2R_1R_3 - \ele_2R_2R_3 - \ele_2R_2R_1
            + \ele_3R_1R_3 + \ele_3R_2R_3 + \ele_3R_2R_1
       }{\cbr{R_1R_3 + R_2R_3 + R_2R_1} \cdot R_3}
        = \\ &=
        \frac{
            - \ele_1R_3R_2 - \ele_2R_1R_3 + \ele_3R_1R_3 + \ele_3R_2R_3
       }{\cbr{R_1R_3 + R_2R_3 + R_2R_1} \cdot R_3}
        =
        \frac{
            - \ele_1R_2 - \ele_2R_1 + \ele_3R_1 + \ele_3R_2
       }{R_1R_3 + R_2R_3 + R_2R_1}
        = \\ &=
        \frac{
            R_1(\ele_3 - \ele_2) + R_2(\ele_3 - \ele_1)
       }{R_1R_3 + R_2R_3 + R_2R_1}
        =
        \frac{
            \cfrac{\ele_3 - \ele_2}{ R_2 } + \cfrac{\ele_3 - \ele_1}{ R_1 }
       }{\cfrac{ R_3 }{ R_2 } + \cfrac{ R_3 }{ R_1 } + 1}
        =
        \frac{
            \cfrac{8\,\text{В} - 3\,\text{В}}{ 5\,\text{Ом} } + \cfrac{8\,\text{В} - 5\,\text{В}}{ 4\,\text{Ом} }
       }{\cfrac{ 10\,\text{Ом} }{ 5\,\text{Ом} } + \cfrac{ 10\,\text{Ом} }{ 4\,\text{Ом} } + 1}
        = \frac7{22}\units{А} \approx 0{,}32\,\text{А}.
        \\
    U_3
        &=
        \eli_3R_3
        =
        \frac{
            \cfrac{\ele_3 - \ele_2}{ R_2 } + \cfrac{\ele_3 - \ele_1}{ R_1 }
       }{\cfrac{ R_3 }{ R_2 } + \cfrac{ R_3 }{ R_1 } + 1} \cdot R_3
        =
        \frac7{22}\units{А} \cdot 10\,\text{Ом} = \frac{35}{11}\units{В} \approx 3{,}18\,\text{В}.
    \end{align*}

    Положительные ответы говорят, что мы угадали на рисунке направление тока (тут нет нашей заслуги, повезло),
    отрицательные — что не угадали (и в этом нет ошибки), и ток течёт в противоположную сторону.
    Напомним, что направление тока — это направление движения положительных зарядов,
    а в металлах носители заряда — электроны, которые заряжены отрицательно.
}

\variantsplitter

\addpersonalvariant{Рената Таржиманова}

\tasknumber{1}%
\task{%
    Определите ток $\eli_3$, протекающий через резистор $R_3$ (см.
    рис.),
    направление этого тока и разность потенциалов $U_3$ на этом резисторе,
    если $R_1 = 3\,\text{Ом}$, $R_2 = 8\,\text{Ом}$, $R_3 = 10\,\text{Ом}$, $\ele_1 = 5\,\text{В}$, $\ele_2 = 3\,\text{В}$, $\ele_3 = 2\,\text{В}$.
    Внутренним сопротивлением всех трёх ЭДС пренебречь.
    Ответы получите в виде несократимых дробей, а также определите приближённые значения.

    \begin{tikzpicture}[circuit ee IEC, thick]
        \foreach \contact/\x in {1/0, 2/3, 3/6}
        {
            \node [contact] (top contact \contact) at (\x, 0) {};
            \node [contact] (bottom contact \contact) at (\x, 4) {};
       }
        \draw  (bottom contact 1) -- (bottom contact 2) -- (bottom contact 3);
        \draw  (top contact 1) -- (top contact 2) -- (top contact 3);
        \draw  (bottom contact 1) to [resistor={near start, info=$R_1$}, battery={near end, info=$\ele_1$}] (top contact 1);
        \draw  (bottom contact 2) to [resistor={near start, info=$R_2$}, battery={near end, info=$\ele_2$}] (top contact 2);
        \draw  (bottom contact 3) to [resistor={near start, info=$R_3$}, battery={near end, info=$\ele_3$}] (top contact 3);
    \end{tikzpicture}
}
\answer{%
    План:
    \begin{itemize}
        \item отметим на рисунке произвольно направления токов (если получим отрицательный ответ, значит не угадали направление и только),
        \item выберем и обозначим на рисунке контуры (здесь всего 3, значит будет нужно $3-1=2$), для них запишем законы Кирхгофа,
        \item выберем и выделим на рисунке нетривиальные узлы (здесь всего 2, значит будет нужно $2-1=1$), для него запишем закон Кирхгофа,
        \item попытаемся решить получившуюся систему.
        В конкретном решении мы пытались первым делом найти $\eli_2$, но, возможно, в вашем варианте будет быстрее решать систему в другом порядке.
        Мы всё же проделаем всё в лоб, подробно и целиком.
    \end{itemize}


    \begin{tikzpicture}[circuit ee IEC, thick]
        \foreach \contact/\x in {1/0, 2/3, 3/6}
        {
            \node [contact] (top contact \contact) at (\x, 0) {};
            \node [contact] (bottom contact \contact) at (\x, 4) {};
       }
        \draw  (bottom contact 1) -- (bottom contact 2) -- (bottom contact 3);
        \draw  (top contact 1) -- (top contact 2) -- (top contact 3);
        \draw  (bottom contact 1) to [resistor={near start, info=$R_1$}, current direction'={midway, info=$\eli_1$}, battery={near end, info=$\ele_1$}] (top contact 1);
        \draw  (bottom contact 2) to [resistor={near start, info=$R_2$}, current direction'={midway, info=$\eli_2$}, battery={near end, info=$\ele_2$}] (top contact 2);
        \draw  (bottom contact 3) to [resistor={near start, info=$R_3$}, current direction'={midway, info=$\eli_3$}, battery={near end, info=$\ele_3$}] (top contact 3);
        \draw [-{Latex},color=red] (1.2, 2.5) arc [start angle = 135, end angle = -160, radius = 0.6];
        \draw [-{Latex},color=blue] (4.2, 2.5) arc [start angle = 135, end angle = -160, radius = 0.6];
        \node [contact,color=green!71!black] (bottomc) at (bottom contact 2) {};
    \end{tikzpicture}

    \begin{align*}
        &\begin{cases}
            {\color{red} \eli_1R_1 - \eli_2R_2 = \ele_1 - \ele_2}, \\
            {\color{blue} \eli_2R_2 - \eli_3R_3 = \ele_2 - \ele_3}, \\
            {\color{green!71!black} \eli_1 + \eli_2 + \eli_3 = 0};
        \end{cases}
        \qquad \implies \qquad
        \begin{cases}
            \eli_1 = \frac{\ele_1 - \ele_2 + \eli_2R_2}{R_1}, \\
            \eli_3 = \frac{\eli_2R_2 - \ele_2 + \ele_3}{R_3}, \\
            \eli_1 + \eli_2 + \eli_3 = 0, \\
        \end{cases} \implies \\
        \implies
            &\eli_2 + \frac{\ele_1 - \ele_2 + \eli_2R_2}{R_1} + \frac{\eli_2R_2 - \ele_2 + \ele_3}{R_3} = 0, \\
        &   \eli_2\cbr{1 + \frac{ R_2 }{ R_1 } + \frac{ R_2 }{ R_3 }} + \frac{\ele_1 - \ele_2}{ R_1 } + \frac{\ele_3 - \ele_2}{ R_3 } = 0, \\
        &   \eli_2 = \cfrac{\cfrac{\ele_2 - \ele_1}{ R_1 } + \cfrac{\ele_2 - \ele_3}{ R_3 }}{1 + \cfrac{ R_2 }{ R_1 } + \cfrac{ R_2 }{ R_3 }}
            = \cfrac{\cfrac{3\,\text{В} - 5\,\text{В}}{ 3\,\text{Ом} } + \cfrac{3\,\text{В} - 2\,\text{В}}{ 10\,\text{Ом} }}{1 + \cfrac{ 8\,\text{Ом} }{ 3\,\text{Ом} } + \cfrac{ 8\,\text{Ом} }{ 10\,\text{Ом} }}
            = -\frac{17}{134}\units{А} \approx -0{,}13000\,\text{А}, \\
        &   U_2 = \eli_2R_2 = \cfrac{\cfrac{\ele_2 - \ele_1}{ R_1 } + \cfrac{\ele_2 - \ele_3}{ R_3 }}{1 + \cfrac{ R_2 }{ R_1 } + \cfrac{ R_2 }{ R_3 }} \cdot R_2
            = \cfrac{\cfrac{3\,\text{В} - 5\,\text{В}}{ 3\,\text{Ом} } + \cfrac{3\,\text{В} - 2\,\text{В}}{ 10\,\text{Ом} }}{1 + \cfrac{ 8\,\text{Ом} }{ 3\,\text{Ом} } + \cfrac{ 8\,\text{Ом} }{ 10\,\text{Ом} }} \cdot 8\,\text{Ом}
            = -\frac{17}{134}\units{А} \cdot 8\,\text{Ом} = -\frac{68}{67}\units{В} \approx -1{,}0100\,\text{В}.
    \end{align*}

    Одну пару силы тока и напряжения получили.
    Для некоторых вариантов это уже ответ, но не у всех.
    Для упрощения записи преобразуем (чтобы избавитсья от 4-этажной дроби) и подставим в уже полученные уравнения:

    \begin{align*}
    \eli_2
        &=
        \frac{\frac{\ele_2 - \ele_1}{ R_1 } + \frac{\ele_2 - \ele_3}{ R_3 }}{1 + \frac{ R_2 }{ R_1 } + \frac{ R_2 }{ R_3 }}
        =
        \frac{(\ele_2 - \ele_1)R_3 + (\ele_2 - \ele_3)R_1}{R_1R_3 + R_2R_3 + R_2R_1},
        \\
    \eli_1
        &=  \frac{\ele_1 - \ele_2 + \eli_2R_2}{R_1}
        =   \frac{\ele_1 - \ele_2 + \cfrac{(\ele_2 - \ele_1)R_3 + (\ele_2 - \ele_3)R_1}{R_1R_3 + R_2R_3 + R_2R_1} \cdot R_2}{R_1} = \\
        &=  \frac{
            \ele_1R_1R_3 + \ele_1R_2R_3 + \ele_1R_2R_1
            - \ele_2R_1R_3 - \ele_2R_2R_3 - \ele_2R_2R_1
            + \ele_2R_3R_2 - \ele_1R_3R_2 + \ele_2R_1R_2 - \ele_3R_1R_2
       }{R_1 \cdot \cbr{R_1R_3 + R_2R_3 + R_2R_1}}
        = \\ &=
        \frac{
            \ele_1\cbr{R_1R_3 + R_2R_3 + R_2R_1 - R_3R_2}
            + \ele_2\cbr{- R_1R_3 - R_2R_3 - R_2R_1 + R_3R_2 + R_1R_2}
            - \ele_3R_1R_2
       }{R_1 \cdot \cbr{R_1R_3 + R_2R_3 + R_2R_1}}
        = \\ &=
        \frac{
            \ele_1\cbr{R_1R_3 + R_2R_1}
            + \ele_2\cbr{- R_1R_3}
            - \ele_3R_1R_2
       }{R_1 \cdot \cbr{R_1R_3 + R_2R_3 + R_2R_1}}
        =
        \frac{
            \ele_1\cbr{R_3 + R_2} - \ele_2R_3 - \ele_3R_2
       }{R_1R_3 + R_2R_3 + R_2R_1}
        = \\ &=
        \frac{
            (\ele_1 - \ele_3)R_2 + (\ele_1 - \ele_2)R_3
       }{R_1R_3 + R_2R_3 + R_2R_1}
        =
        \frac{
            \cfrac{\ele_1 - \ele_3}{ R_3 } + \cfrac{\ele_1 - \ele_2}{ R_2 }
       }{\cfrac{ R_1 }{ R_2 } + 1 + \cfrac{ R_1 }{ R_3 }}
        =
        \frac{
            \cfrac{5\,\text{В} - 2\,\text{В}}{ 10\,\text{Ом} } + \cfrac{5\,\text{В} - 3\,\text{В}}{ 8\,\text{Ом} }
       }{\cfrac{ 3\,\text{Ом} }{ 8\,\text{Ом} } + 1 + \cfrac{ 3\,\text{Ом} }{ 10\,\text{Ом} }}
        = \frac{22}{67}\units{А} \approx 0{,}33\,\text{А}.
        \\
    U_1
        &=
        \eli_1R_1
        =
        \frac{
            \cfrac{\ele_1 - \ele_3}{ R_3 } + \cfrac{\ele_1 - \ele_2}{ R_2 }
       }{\cfrac{ R_1 }{ R_2 } + 1 + \cfrac{ R_1 }{ R_3 }} \cdot R_1
        =
        \frac{22}{67}\units{А} \cdot 3\,\text{Ом} = \frac{66}{67}\units{В} \approx 0{,}99\,\text{В}.
    \end{align*}

    Если вы проделали все эти вычисления выше вместе со мной, то
    \begin{itemize}
        \item вы совершили ошибку, выбрав неверный путь решения:
        слишком длинное решение, очень легко ошибиться в индексах, дробях, знаках или потерять какой-то множитель,
        \item можно было выразить из исходной системы другие токи и получить сразу нажный вам,
        а не какой-то 2-й,
        \item можно было сэкономить: все три резистора и ЭДС соединены одинаково,
        поэтому ответ для 1-го резистора должен отличаться лишь перестановкой индексов (этот факт крайне полезен при проверке ответа, у нас всё сошлось),
        я специально подгонял выражение для $\eli_1$ к этому виду, вынося за скобки и преобразуя дробь,
        \item вы молодец, потому что не побоялись и получили верный ответ грамотным способом,
    \end{itemize}
    так что переходим к третьему резистору.
    Будет похоже, но кого это когда останавливало...

    \begin{align*}
    \eli_3
        &=  \frac{\eli_2R_2 - \ele_2 + \ele_3}{ R_3 }
        =
        \cfrac{
            \cfrac{
                (\ele_2 - \ele_1)R_3 + (\ele_2 - \ele_3)R_1
           }{
                R_1R_3 + R_2R_3 + R_2R_1
           } \cdot R_2 - \ele_2 + \ele_3}{ R_3 }
        = \\ &=
        \frac{
            \ele_2R_3R_2 - \ele_1R_3R_2 + \ele_2R_1R_2 - \ele_3R_1R_2
            - \ele_2R_1R_3 - \ele_2R_2R_3 - \ele_2R_2R_1
            + \ele_3R_1R_3 + \ele_3R_2R_3 + \ele_3R_2R_1
       }{\cbr{R_1R_3 + R_2R_3 + R_2R_1} \cdot R_3}
        = \\ &=
        \frac{
            - \ele_1R_3R_2 - \ele_2R_1R_3 + \ele_3R_1R_3 + \ele_3R_2R_3
       }{\cbr{R_1R_3 + R_2R_3 + R_2R_1} \cdot R_3}
        =
        \frac{
            - \ele_1R_2 - \ele_2R_1 + \ele_3R_1 + \ele_3R_2
       }{R_1R_3 + R_2R_3 + R_2R_1}
        = \\ &=
        \frac{
            R_1(\ele_3 - \ele_2) + R_2(\ele_3 - \ele_1)
       }{R_1R_3 + R_2R_3 + R_2R_1}
        =
        \frac{
            \cfrac{\ele_3 - \ele_2}{ R_2 } + \cfrac{\ele_3 - \ele_1}{ R_1 }
       }{\cfrac{ R_3 }{ R_2 } + \cfrac{ R_3 }{ R_1 } + 1}
        =
        \frac{
            \cfrac{2\,\text{В} - 3\,\text{В}}{ 8\,\text{Ом} } + \cfrac{2\,\text{В} - 5\,\text{В}}{ 3\,\text{Ом} }
       }{\cfrac{ 10\,\text{Ом} }{ 8\,\text{Ом} } + \cfrac{ 10\,\text{Ом} }{ 3\,\text{Ом} } + 1}
        = -\frac{27}{134}\units{А} \approx -0{,}2000\,\text{А}.
        \\
    U_3
        &=
        \eli_3R_3
        =
        \frac{
            \cfrac{\ele_3 - \ele_2}{ R_2 } + \cfrac{\ele_3 - \ele_1}{ R_1 }
       }{\cfrac{ R_3 }{ R_2 } + \cfrac{ R_3 }{ R_1 } + 1} \cdot R_3
        =
        -\frac{27}{134}\units{А} \cdot 10\,\text{Ом} = -\frac{135}{67}\units{В} \approx -2{,}010\,\text{В}.
    \end{align*}

    Положительные ответы говорят, что мы угадали на рисунке направление тока (тут нет нашей заслуги, повезло),
    отрицательные — что не угадали (и в этом нет ошибки), и ток течёт в противоположную сторону.
    Напомним, что направление тока — это направление движения положительных зарядов,
    а в металлах носители заряда — электроны, которые заряжены отрицательно.
}

\variantsplitter

\addpersonalvariant{Андрей Щербаков}

\tasknumber{1}%
\task{%
    Определите ток $\eli_3$, протекающий через резистор $R_3$ (см.
    рис.),
    направление этого тока и разность потенциалов $U_3$ на этом резисторе,
    если $R_1 = 2\,\text{Ом}$, $R_2 = 5\,\text{Ом}$, $R_3 = 12\,\text{Ом}$, $\ele_1 = 5\,\text{В}$, $\ele_2 = 3\,\text{В}$, $\ele_3 = 8\,\text{В}$.
    Внутренним сопротивлением всех трёх ЭДС пренебречь.
    Ответы получите в виде несократимых дробей, а также определите приближённые значения.

    \begin{tikzpicture}[circuit ee IEC, thick]
        \foreach \contact/\x in {1/0, 2/3, 3/6}
        {
            \node [contact] (top contact \contact) at (\x, 0) {};
            \node [contact] (bottom contact \contact) at (\x, 4) {};
       }
        \draw  (bottom contact 1) -- (bottom contact 2) -- (bottom contact 3);
        \draw  (top contact 1) -- (top contact 2) -- (top contact 3);
        \draw  (bottom contact 1) to [resistor={near start, info=$R_1$}, battery={near end, info=$\ele_1$}] (top contact 1);
        \draw  (bottom contact 2) to [resistor={near start, info=$R_2$}, battery={near end, info=$\ele_2$}] (top contact 2);
        \draw  (bottom contact 3) to [resistor={near start, info=$R_3$}, battery={near end, info=$\ele_3$}] (top contact 3);
    \end{tikzpicture}
}
\answer{%
    План:
    \begin{itemize}
        \item отметим на рисунке произвольно направления токов (если получим отрицательный ответ, значит не угадали направление и только),
        \item выберем и обозначим на рисунке контуры (здесь всего 3, значит будет нужно $3-1=2$), для них запишем законы Кирхгофа,
        \item выберем и выделим на рисунке нетривиальные узлы (здесь всего 2, значит будет нужно $2-1=1$), для него запишем закон Кирхгофа,
        \item попытаемся решить получившуюся систему.
        В конкретном решении мы пытались первым делом найти $\eli_2$, но, возможно, в вашем варианте будет быстрее решать систему в другом порядке.
        Мы всё же проделаем всё в лоб, подробно и целиком.
    \end{itemize}


    \begin{tikzpicture}[circuit ee IEC, thick]
        \foreach \contact/\x in {1/0, 2/3, 3/6}
        {
            \node [contact] (top contact \contact) at (\x, 0) {};
            \node [contact] (bottom contact \contact) at (\x, 4) {};
       }
        \draw  (bottom contact 1) -- (bottom contact 2) -- (bottom contact 3);
        \draw  (top contact 1) -- (top contact 2) -- (top contact 3);
        \draw  (bottom contact 1) to [resistor={near start, info=$R_1$}, current direction'={midway, info=$\eli_1$}, battery={near end, info=$\ele_1$}] (top contact 1);
        \draw  (bottom contact 2) to [resistor={near start, info=$R_2$}, current direction'={midway, info=$\eli_2$}, battery={near end, info=$\ele_2$}] (top contact 2);
        \draw  (bottom contact 3) to [resistor={near start, info=$R_3$}, current direction'={midway, info=$\eli_3$}, battery={near end, info=$\ele_3$}] (top contact 3);
        \draw [-{Latex},color=red] (1.2, 2.5) arc [start angle = 135, end angle = -160, radius = 0.6];
        \draw [-{Latex},color=blue] (4.2, 2.5) arc [start angle = 135, end angle = -160, radius = 0.6];
        \node [contact,color=green!71!black] (bottomc) at (bottom contact 2) {};
    \end{tikzpicture}

    \begin{align*}
        &\begin{cases}
            {\color{red} \eli_1R_1 - \eli_2R_2 = \ele_1 - \ele_2}, \\
            {\color{blue} \eli_2R_2 - \eli_3R_3 = \ele_2 - \ele_3}, \\
            {\color{green!71!black} \eli_1 + \eli_2 + \eli_3 = 0};
        \end{cases}
        \qquad \implies \qquad
        \begin{cases}
            \eli_1 = \frac{\ele_1 - \ele_2 + \eli_2R_2}{R_1}, \\
            \eli_3 = \frac{\eli_2R_2 - \ele_2 + \ele_3}{R_3}, \\
            \eli_1 + \eli_2 + \eli_3 = 0, \\
        \end{cases} \implies \\
        \implies
            &\eli_2 + \frac{\ele_1 - \ele_2 + \eli_2R_2}{R_1} + \frac{\eli_2R_2 - \ele_2 + \ele_3}{R_3} = 0, \\
        &   \eli_2\cbr{1 + \frac{ R_2 }{ R_1 } + \frac{ R_2 }{ R_3 }} + \frac{\ele_1 - \ele_2}{ R_1 } + \frac{\ele_3 - \ele_2}{ R_3 } = 0, \\
        &   \eli_2 = \cfrac{\cfrac{\ele_2 - \ele_1}{ R_1 } + \cfrac{\ele_2 - \ele_3}{ R_3 }}{1 + \cfrac{ R_2 }{ R_1 } + \cfrac{ R_2 }{ R_3 }}
            = \cfrac{\cfrac{3\,\text{В} - 5\,\text{В}}{ 2\,\text{Ом} } + \cfrac{3\,\text{В} - 8\,\text{В}}{ 12\,\text{Ом} }}{1 + \cfrac{ 5\,\text{Ом} }{ 2\,\text{Ом} } + \cfrac{ 5\,\text{Ом} }{ 12\,\text{Ом} }}
            = -\frac{17}{47}\units{А} \approx -0{,}3600\,\text{А}, \\
        &   U_2 = \eli_2R_2 = \cfrac{\cfrac{\ele_2 - \ele_1}{ R_1 } + \cfrac{\ele_2 - \ele_3}{ R_3 }}{1 + \cfrac{ R_2 }{ R_1 } + \cfrac{ R_2 }{ R_3 }} \cdot R_2
            = \cfrac{\cfrac{3\,\text{В} - 5\,\text{В}}{ 2\,\text{Ом} } + \cfrac{3\,\text{В} - 8\,\text{В}}{ 12\,\text{Ом} }}{1 + \cfrac{ 5\,\text{Ом} }{ 2\,\text{Ом} } + \cfrac{ 5\,\text{Ом} }{ 12\,\text{Ом} }} \cdot 5\,\text{Ом}
            = -\frac{17}{47}\units{А} \cdot 5\,\text{Ом} = -\frac{85}{47}\units{В} \approx -1{,}8100\,\text{В}.
    \end{align*}

    Одну пару силы тока и напряжения получили.
    Для некоторых вариантов это уже ответ, но не у всех.
    Для упрощения записи преобразуем (чтобы избавитсья от 4-этажной дроби) и подставим в уже полученные уравнения:

    \begin{align*}
    \eli_2
        &=
        \frac{\frac{\ele_2 - \ele_1}{ R_1 } + \frac{\ele_2 - \ele_3}{ R_3 }}{1 + \frac{ R_2 }{ R_1 } + \frac{ R_2 }{ R_3 }}
        =
        \frac{(\ele_2 - \ele_1)R_3 + (\ele_2 - \ele_3)R_1}{R_1R_3 + R_2R_3 + R_2R_1},
        \\
    \eli_1
        &=  \frac{\ele_1 - \ele_2 + \eli_2R_2}{R_1}
        =   \frac{\ele_1 - \ele_2 + \cfrac{(\ele_2 - \ele_1)R_3 + (\ele_2 - \ele_3)R_1}{R_1R_3 + R_2R_3 + R_2R_1} \cdot R_2}{R_1} = \\
        &=  \frac{
            \ele_1R_1R_3 + \ele_1R_2R_3 + \ele_1R_2R_1
            - \ele_2R_1R_3 - \ele_2R_2R_3 - \ele_2R_2R_1
            + \ele_2R_3R_2 - \ele_1R_3R_2 + \ele_2R_1R_2 - \ele_3R_1R_2
       }{R_1 \cdot \cbr{R_1R_3 + R_2R_3 + R_2R_1}}
        = \\ &=
        \frac{
            \ele_1\cbr{R_1R_3 + R_2R_3 + R_2R_1 - R_3R_2}
            + \ele_2\cbr{- R_1R_3 - R_2R_3 - R_2R_1 + R_3R_2 + R_1R_2}
            - \ele_3R_1R_2
       }{R_1 \cdot \cbr{R_1R_3 + R_2R_3 + R_2R_1}}
        = \\ &=
        \frac{
            \ele_1\cbr{R_1R_3 + R_2R_1}
            + \ele_2\cbr{- R_1R_3}
            - \ele_3R_1R_2
       }{R_1 \cdot \cbr{R_1R_3 + R_2R_3 + R_2R_1}}
        =
        \frac{
            \ele_1\cbr{R_3 + R_2} - \ele_2R_3 - \ele_3R_2
       }{R_1R_3 + R_2R_3 + R_2R_1}
        = \\ &=
        \frac{
            (\ele_1 - \ele_3)R_2 + (\ele_1 - \ele_2)R_3
       }{R_1R_3 + R_2R_3 + R_2R_1}
        =
        \frac{
            \cfrac{\ele_1 - \ele_3}{ R_3 } + \cfrac{\ele_1 - \ele_2}{ R_2 }
       }{\cfrac{ R_1 }{ R_2 } + 1 + \cfrac{ R_1 }{ R_3 }}
        =
        \frac{
            \cfrac{5\,\text{В} - 8\,\text{В}}{ 12\,\text{Ом} } + \cfrac{5\,\text{В} - 3\,\text{В}}{ 5\,\text{Ом} }
       }{\cfrac{ 2\,\text{Ом} }{ 5\,\text{Ом} } + 1 + \cfrac{ 2\,\text{Ом} }{ 12\,\text{Ом} }}
        = \frac9{94}\units{А} \approx 0{,}10\,\text{А}.
        \\
    U_1
        &=
        \eli_1R_1
        =
        \frac{
            \cfrac{\ele_1 - \ele_3}{ R_3 } + \cfrac{\ele_1 - \ele_2}{ R_2 }
       }{\cfrac{ R_1 }{ R_2 } + 1 + \cfrac{ R_1 }{ R_3 }} \cdot R_1
        =
        \frac9{94}\units{А} \cdot 2\,\text{Ом} = \frac9{47}\units{В} \approx 0{,}19\,\text{В}.
    \end{align*}

    Если вы проделали все эти вычисления выше вместе со мной, то
    \begin{itemize}
        \item вы совершили ошибку, выбрав неверный путь решения:
        слишком длинное решение, очень легко ошибиться в индексах, дробях, знаках или потерять какой-то множитель,
        \item можно было выразить из исходной системы другие токи и получить сразу нажный вам,
        а не какой-то 2-й,
        \item можно было сэкономить: все три резистора и ЭДС соединены одинаково,
        поэтому ответ для 1-го резистора должен отличаться лишь перестановкой индексов (этот факт крайне полезен при проверке ответа, у нас всё сошлось),
        я специально подгонял выражение для $\eli_1$ к этому виду, вынося за скобки и преобразуя дробь,
        \item вы молодец, потому что не побоялись и получили верный ответ грамотным способом,
    \end{itemize}
    так что переходим к третьему резистору.
    Будет похоже, но кого это когда останавливало...

    \begin{align*}
    \eli_3
        &=  \frac{\eli_2R_2 - \ele_2 + \ele_3}{ R_3 }
        =
        \cfrac{
            \cfrac{
                (\ele_2 - \ele_1)R_3 + (\ele_2 - \ele_3)R_1
           }{
                R_1R_3 + R_2R_3 + R_2R_1
           } \cdot R_2 - \ele_2 + \ele_3}{ R_3 }
        = \\ &=
        \frac{
            \ele_2R_3R_2 - \ele_1R_3R_2 + \ele_2R_1R_2 - \ele_3R_1R_2
            - \ele_2R_1R_3 - \ele_2R_2R_3 - \ele_2R_2R_1
            + \ele_3R_1R_3 + \ele_3R_2R_3 + \ele_3R_2R_1
       }{\cbr{R_1R_3 + R_2R_3 + R_2R_1} \cdot R_3}
        = \\ &=
        \frac{
            - \ele_1R_3R_2 - \ele_2R_1R_3 + \ele_3R_1R_3 + \ele_3R_2R_3
       }{\cbr{R_1R_3 + R_2R_3 + R_2R_1} \cdot R_3}
        =
        \frac{
            - \ele_1R_2 - \ele_2R_1 + \ele_3R_1 + \ele_3R_2
       }{R_1R_3 + R_2R_3 + R_2R_1}
        = \\ &=
        \frac{
            R_1(\ele_3 - \ele_2) + R_2(\ele_3 - \ele_1)
       }{R_1R_3 + R_2R_3 + R_2R_1}
        =
        \frac{
            \cfrac{\ele_3 - \ele_2}{ R_2 } + \cfrac{\ele_3 - \ele_1}{ R_1 }
       }{\cfrac{ R_3 }{ R_2 } + \cfrac{ R_3 }{ R_1 } + 1}
        =
        \frac{
            \cfrac{8\,\text{В} - 3\,\text{В}}{ 5\,\text{Ом} } + \cfrac{8\,\text{В} - 5\,\text{В}}{ 2\,\text{Ом} }
       }{\cfrac{ 12\,\text{Ом} }{ 5\,\text{Ом} } + \cfrac{ 12\,\text{Ом} }{ 2\,\text{Ом} } + 1}
        = \frac{25}{94}\units{А} \approx 0{,}27\,\text{А}.
        \\
    U_3
        &=
        \eli_3R_3
        =
        \frac{
            \cfrac{\ele_3 - \ele_2}{ R_2 } + \cfrac{\ele_3 - \ele_1}{ R_1 }
       }{\cfrac{ R_3 }{ R_2 } + \cfrac{ R_3 }{ R_1 } + 1} \cdot R_3
        =
        \frac{25}{94}\units{А} \cdot 12\,\text{Ом} = \frac{150}{47}\units{В} \approx 3{,}19\,\text{В}.
    \end{align*}

    Положительные ответы говорят, что мы угадали на рисунке направление тока (тут нет нашей заслуги, повезло),
    отрицательные — что не угадали (и в этом нет ошибки), и ток течёт в противоположную сторону.
    Напомним, что направление тока — это направление движения положительных зарядов,
    а в металлах носители заряда — электроны, которые заряжены отрицательно.
}

\variantsplitter

\addpersonalvariant{Михаил Ярошевский}

\tasknumber{1}%
\task{%
    Определите ток $\eli_1$, протекающий через резистор $R_1$ (см.
    рис.),
    направление этого тока и разность потенциалов $U_1$ на этом резисторе,
    если $R_1 = 4\,\text{Ом}$, $R_2 = 6\,\text{Ом}$, $R_3 = 10\,\text{Ом}$, $\ele_1 = 5\,\text{В}$, $\ele_2 = 6\,\text{В}$, $\ele_3 = 8\,\text{В}$.
    Внутренним сопротивлением всех трёх ЭДС пренебречь.
    Ответы получите в виде несократимых дробей, а также определите приближённые значения.

    \begin{tikzpicture}[circuit ee IEC, thick]
        \foreach \contact/\x in {1/0, 2/3, 3/6}
        {
            \node [contact] (top contact \contact) at (\x, 0) {};
            \node [contact] (bottom contact \contact) at (\x, 4) {};
       }
        \draw  (bottom contact 1) -- (bottom contact 2) -- (bottom contact 3);
        \draw  (top contact 1) -- (top contact 2) -- (top contact 3);
        \draw  (bottom contact 1) to [resistor={near start, info=$R_1$}, battery={near end, info=$\ele_1$}] (top contact 1);
        \draw  (bottom contact 2) to [resistor={near start, info=$R_2$}, battery={near end, info=$\ele_2$}] (top contact 2);
        \draw  (bottom contact 3) to [resistor={near start, info=$R_3$}, battery={near end, info=$\ele_3$}] (top contact 3);
    \end{tikzpicture}
}
\answer{%
    План:
    \begin{itemize}
        \item отметим на рисунке произвольно направления токов (если получим отрицательный ответ, значит не угадали направление и только),
        \item выберем и обозначим на рисунке контуры (здесь всего 3, значит будет нужно $3-1=2$), для них запишем законы Кирхгофа,
        \item выберем и выделим на рисунке нетривиальные узлы (здесь всего 2, значит будет нужно $2-1=1$), для него запишем закон Кирхгофа,
        \item попытаемся решить получившуюся систему.
        В конкретном решении мы пытались первым делом найти $\eli_2$, но, возможно, в вашем варианте будет быстрее решать систему в другом порядке.
        Мы всё же проделаем всё в лоб, подробно и целиком.
    \end{itemize}


    \begin{tikzpicture}[circuit ee IEC, thick]
        \foreach \contact/\x in {1/0, 2/3, 3/6}
        {
            \node [contact] (top contact \contact) at (\x, 0) {};
            \node [contact] (bottom contact \contact) at (\x, 4) {};
       }
        \draw  (bottom contact 1) -- (bottom contact 2) -- (bottom contact 3);
        \draw  (top contact 1) -- (top contact 2) -- (top contact 3);
        \draw  (bottom contact 1) to [resistor={near start, info=$R_1$}, current direction'={midway, info=$\eli_1$}, battery={near end, info=$\ele_1$}] (top contact 1);
        \draw  (bottom contact 2) to [resistor={near start, info=$R_2$}, current direction'={midway, info=$\eli_2$}, battery={near end, info=$\ele_2$}] (top contact 2);
        \draw  (bottom contact 3) to [resistor={near start, info=$R_3$}, current direction'={midway, info=$\eli_3$}, battery={near end, info=$\ele_3$}] (top contact 3);
        \draw [-{Latex},color=red] (1.2, 2.5) arc [start angle = 135, end angle = -160, radius = 0.6];
        \draw [-{Latex},color=blue] (4.2, 2.5) arc [start angle = 135, end angle = -160, radius = 0.6];
        \node [contact,color=green!71!black] (bottomc) at (bottom contact 2) {};
    \end{tikzpicture}

    \begin{align*}
        &\begin{cases}
            {\color{red} \eli_1R_1 - \eli_2R_2 = \ele_1 - \ele_2}, \\
            {\color{blue} \eli_2R_2 - \eli_3R_3 = \ele_2 - \ele_3}, \\
            {\color{green!71!black} \eli_1 + \eli_2 + \eli_3 = 0};
        \end{cases}
        \qquad \implies \qquad
        \begin{cases}
            \eli_1 = \frac{\ele_1 - \ele_2 + \eli_2R_2}{R_1}, \\
            \eli_3 = \frac{\eli_2R_2 - \ele_2 + \ele_3}{R_3}, \\
            \eli_1 + \eli_2 + \eli_3 = 0, \\
        \end{cases} \implies \\
        \implies
            &\eli_2 + \frac{\ele_1 - \ele_2 + \eli_2R_2}{R_1} + \frac{\eli_2R_2 - \ele_2 + \ele_3}{R_3} = 0, \\
        &   \eli_2\cbr{1 + \frac{ R_2 }{ R_1 } + \frac{ R_2 }{ R_3 }} + \frac{\ele_1 - \ele_2}{ R_1 } + \frac{\ele_3 - \ele_2}{ R_3 } = 0, \\
        &   \eli_2 = \cfrac{\cfrac{\ele_2 - \ele_1}{ R_1 } + \cfrac{\ele_2 - \ele_3}{ R_3 }}{1 + \cfrac{ R_2 }{ R_1 } + \cfrac{ R_2 }{ R_3 }}
            = \cfrac{\cfrac{6\,\text{В} - 5\,\text{В}}{ 4\,\text{Ом} } + \cfrac{6\,\text{В} - 8\,\text{В}}{ 10\,\text{Ом} }}{1 + \cfrac{ 6\,\text{Ом} }{ 4\,\text{Ом} } + \cfrac{ 6\,\text{Ом} }{ 10\,\text{Ом} }}
            = \frac1{62}\units{А} \approx 0{,}02\,\text{А}, \\
        &   U_2 = \eli_2R_2 = \cfrac{\cfrac{\ele_2 - \ele_1}{ R_1 } + \cfrac{\ele_2 - \ele_3}{ R_3 }}{1 + \cfrac{ R_2 }{ R_1 } + \cfrac{ R_2 }{ R_3 }} \cdot R_2
            = \cfrac{\cfrac{6\,\text{В} - 5\,\text{В}}{ 4\,\text{Ом} } + \cfrac{6\,\text{В} - 8\,\text{В}}{ 10\,\text{Ом} }}{1 + \cfrac{ 6\,\text{Ом} }{ 4\,\text{Ом} } + \cfrac{ 6\,\text{Ом} }{ 10\,\text{Ом} }} \cdot 6\,\text{Ом}
            = \frac1{62}\units{А} \cdot 6\,\text{Ом} = \frac3{31}\units{В} \approx 0{,}10\,\text{В}.
    \end{align*}

    Одну пару силы тока и напряжения получили.
    Для некоторых вариантов это уже ответ, но не у всех.
    Для упрощения записи преобразуем (чтобы избавитсья от 4-этажной дроби) и подставим в уже полученные уравнения:

    \begin{align*}
    \eli_2
        &=
        \frac{\frac{\ele_2 - \ele_1}{ R_1 } + \frac{\ele_2 - \ele_3}{ R_3 }}{1 + \frac{ R_2 }{ R_1 } + \frac{ R_2 }{ R_3 }}
        =
        \frac{(\ele_2 - \ele_1)R_3 + (\ele_2 - \ele_3)R_1}{R_1R_3 + R_2R_3 + R_2R_1},
        \\
    \eli_1
        &=  \frac{\ele_1 - \ele_2 + \eli_2R_2}{R_1}
        =   \frac{\ele_1 - \ele_2 + \cfrac{(\ele_2 - \ele_1)R_3 + (\ele_2 - \ele_3)R_1}{R_1R_3 + R_2R_3 + R_2R_1} \cdot R_2}{R_1} = \\
        &=  \frac{
            \ele_1R_1R_3 + \ele_1R_2R_3 + \ele_1R_2R_1
            - \ele_2R_1R_3 - \ele_2R_2R_3 - \ele_2R_2R_1
            + \ele_2R_3R_2 - \ele_1R_3R_2 + \ele_2R_1R_2 - \ele_3R_1R_2
       }{R_1 \cdot \cbr{R_1R_3 + R_2R_3 + R_2R_1}}
        = \\ &=
        \frac{
            \ele_1\cbr{R_1R_3 + R_2R_3 + R_2R_1 - R_3R_2}
            + \ele_2\cbr{- R_1R_3 - R_2R_3 - R_2R_1 + R_3R_2 + R_1R_2}
            - \ele_3R_1R_2
       }{R_1 \cdot \cbr{R_1R_3 + R_2R_3 + R_2R_1}}
        = \\ &=
        \frac{
            \ele_1\cbr{R_1R_3 + R_2R_1}
            + \ele_2\cbr{- R_1R_3}
            - \ele_3R_1R_2
       }{R_1 \cdot \cbr{R_1R_3 + R_2R_3 + R_2R_1}}
        =
        \frac{
            \ele_1\cbr{R_3 + R_2} - \ele_2R_3 - \ele_3R_2
       }{R_1R_3 + R_2R_3 + R_2R_1}
        = \\ &=
        \frac{
            (\ele_1 - \ele_3)R_2 + (\ele_1 - \ele_2)R_3
       }{R_1R_3 + R_2R_3 + R_2R_1}
        =
        \frac{
            \cfrac{\ele_1 - \ele_3}{ R_3 } + \cfrac{\ele_1 - \ele_2}{ R_2 }
       }{\cfrac{ R_1 }{ R_2 } + 1 + \cfrac{ R_1 }{ R_3 }}
        =
        \frac{
            \cfrac{5\,\text{В} - 8\,\text{В}}{ 10\,\text{Ом} } + \cfrac{5\,\text{В} - 6\,\text{В}}{ 6\,\text{Ом} }
       }{\cfrac{ 4\,\text{Ом} }{ 6\,\text{Ом} } + 1 + \cfrac{ 4\,\text{Ом} }{ 10\,\text{Ом} }}
        = -\frac7{31}\units{А} \approx -0{,}2300\,\text{А}.
        \\
    U_1
        &=
        \eli_1R_1
        =
        \frac{
            \cfrac{\ele_1 - \ele_3}{ R_3 } + \cfrac{\ele_1 - \ele_2}{ R_2 }
       }{\cfrac{ R_1 }{ R_2 } + 1 + \cfrac{ R_1 }{ R_3 }} \cdot R_1
        =
        -\frac7{31}\units{А} \cdot 4\,\text{Ом} = -\frac{28}{31}\units{В} \approx -0{,}9000\,\text{В}.
    \end{align*}

    Если вы проделали все эти вычисления выше вместе со мной, то
    \begin{itemize}
        \item вы совершили ошибку, выбрав неверный путь решения:
        слишком длинное решение, очень легко ошибиться в индексах, дробях, знаках или потерять какой-то множитель,
        \item можно было выразить из исходной системы другие токи и получить сразу нажный вам,
        а не какой-то 2-й,
        \item можно было сэкономить: все три резистора и ЭДС соединены одинаково,
        поэтому ответ для 1-го резистора должен отличаться лишь перестановкой индексов (этот факт крайне полезен при проверке ответа, у нас всё сошлось),
        я специально подгонял выражение для $\eli_1$ к этому виду, вынося за скобки и преобразуя дробь,
        \item вы молодец, потому что не побоялись и получили верный ответ грамотным способом,
    \end{itemize}
    так что переходим к третьему резистору.
    Будет похоже, но кого это когда останавливало...

    \begin{align*}
    \eli_3
        &=  \frac{\eli_2R_2 - \ele_2 + \ele_3}{ R_3 }
        =
        \cfrac{
            \cfrac{
                (\ele_2 - \ele_1)R_3 + (\ele_2 - \ele_3)R_1
           }{
                R_1R_3 + R_2R_3 + R_2R_1
           } \cdot R_2 - \ele_2 + \ele_3}{ R_3 }
        = \\ &=
        \frac{
            \ele_2R_3R_2 - \ele_1R_3R_2 + \ele_2R_1R_2 - \ele_3R_1R_2
            - \ele_2R_1R_3 - \ele_2R_2R_3 - \ele_2R_2R_1
            + \ele_3R_1R_3 + \ele_3R_2R_3 + \ele_3R_2R_1
       }{\cbr{R_1R_3 + R_2R_3 + R_2R_1} \cdot R_3}
        = \\ &=
        \frac{
            - \ele_1R_3R_2 - \ele_2R_1R_3 + \ele_3R_1R_3 + \ele_3R_2R_3
       }{\cbr{R_1R_3 + R_2R_3 + R_2R_1} \cdot R_3}
        =
        \frac{
            - \ele_1R_2 - \ele_2R_1 + \ele_3R_1 + \ele_3R_2
       }{R_1R_3 + R_2R_3 + R_2R_1}
        = \\ &=
        \frac{
            R_1(\ele_3 - \ele_2) + R_2(\ele_3 - \ele_1)
       }{R_1R_3 + R_2R_3 + R_2R_1}
        =
        \frac{
            \cfrac{\ele_3 - \ele_2}{ R_2 } + \cfrac{\ele_3 - \ele_1}{ R_1 }
       }{\cfrac{ R_3 }{ R_2 } + \cfrac{ R_3 }{ R_1 } + 1}
        =
        \frac{
            \cfrac{8\,\text{В} - 6\,\text{В}}{ 6\,\text{Ом} } + \cfrac{8\,\text{В} - 5\,\text{В}}{ 4\,\text{Ом} }
       }{\cfrac{ 10\,\text{Ом} }{ 6\,\text{Ом} } + \cfrac{ 10\,\text{Ом} }{ 4\,\text{Ом} } + 1}
        = \frac{13}{62}\units{А} \approx 0{,}21\,\text{А}.
        \\
    U_3
        &=
        \eli_3R_3
        =
        \frac{
            \cfrac{\ele_3 - \ele_2}{ R_2 } + \cfrac{\ele_3 - \ele_1}{ R_1 }
       }{\cfrac{ R_3 }{ R_2 } + \cfrac{ R_3 }{ R_1 } + 1} \cdot R_3
        =
        \frac{13}{62}\units{А} \cdot 10\,\text{Ом} = \frac{65}{31}\units{В} \approx 2{,}10\,\text{В}.
    \end{align*}

    Положительные ответы говорят, что мы угадали на рисунке направление тока (тут нет нашей заслуги, повезло),
    отрицательные — что не угадали (и в этом нет ошибки), и ток течёт в противоположную сторону.
    Напомним, что направление тока — это направление движения положительных зарядов,
    а в металлах носители заряда — электроны, которые заряжены отрицательно.
}

\variantsplitter

\addpersonalvariant{Алексей Алимпиев}

\tasknumber{1}%
\task{%
    Определите ток $\eli_3$, протекающий через резистор $R_3$ (см.
    рис.),
    направление этого тока и разность потенциалов $U_3$ на этом резисторе,
    если $R_1 = 4\,\text{Ом}$, $R_2 = 8\,\text{Ом}$, $R_3 = 12\,\text{Ом}$, $\ele_1 = 5\,\text{В}$, $\ele_2 = 6\,\text{В}$, $\ele_3 = 8\,\text{В}$.
    Внутренним сопротивлением всех трёх ЭДС пренебречь.
    Ответы получите в виде несократимых дробей, а также определите приближённые значения.

    \begin{tikzpicture}[circuit ee IEC, thick]
        \foreach \contact/\x in {1/0, 2/3, 3/6}
        {
            \node [contact] (top contact \contact) at (\x, 0) {};
            \node [contact] (bottom contact \contact) at (\x, 4) {};
       }
        \draw  (bottom contact 1) -- (bottom contact 2) -- (bottom contact 3);
        \draw  (top contact 1) -- (top contact 2) -- (top contact 3);
        \draw  (bottom contact 1) to [resistor={near start, info=$R_1$}, battery={near end, info=$\ele_1$}] (top contact 1);
        \draw  (bottom contact 2) to [resistor={near start, info=$R_2$}, battery={near end, info=$\ele_2$}] (top contact 2);
        \draw  (bottom contact 3) to [resistor={near start, info=$R_3$}, battery={near end, info=$\ele_3$}] (top contact 3);
    \end{tikzpicture}
}
\answer{%
    План:
    \begin{itemize}
        \item отметим на рисунке произвольно направления токов (если получим отрицательный ответ, значит не угадали направление и только),
        \item выберем и обозначим на рисунке контуры (здесь всего 3, значит будет нужно $3-1=2$), для них запишем законы Кирхгофа,
        \item выберем и выделим на рисунке нетривиальные узлы (здесь всего 2, значит будет нужно $2-1=1$), для него запишем закон Кирхгофа,
        \item попытаемся решить получившуюся систему.
        В конкретном решении мы пытались первым делом найти $\eli_2$, но, возможно, в вашем варианте будет быстрее решать систему в другом порядке.
        Мы всё же проделаем всё в лоб, подробно и целиком.
    \end{itemize}


    \begin{tikzpicture}[circuit ee IEC, thick]
        \foreach \contact/\x in {1/0, 2/3, 3/6}
        {
            \node [contact] (top contact \contact) at (\x, 0) {};
            \node [contact] (bottom contact \contact) at (\x, 4) {};
       }
        \draw  (bottom contact 1) -- (bottom contact 2) -- (bottom contact 3);
        \draw  (top contact 1) -- (top contact 2) -- (top contact 3);
        \draw  (bottom contact 1) to [resistor={near start, info=$R_1$}, current direction'={midway, info=$\eli_1$}, battery={near end, info=$\ele_1$}] (top contact 1);
        \draw  (bottom contact 2) to [resistor={near start, info=$R_2$}, current direction'={midway, info=$\eli_2$}, battery={near end, info=$\ele_2$}] (top contact 2);
        \draw  (bottom contact 3) to [resistor={near start, info=$R_3$}, current direction'={midway, info=$\eli_3$}, battery={near end, info=$\ele_3$}] (top contact 3);
        \draw [-{Latex},color=red] (1.2, 2.5) arc [start angle = 135, end angle = -160, radius = 0.6];
        \draw [-{Latex},color=blue] (4.2, 2.5) arc [start angle = 135, end angle = -160, radius = 0.6];
        \node [contact,color=green!71!black] (bottomc) at (bottom contact 2) {};
    \end{tikzpicture}

    \begin{align*}
        &\begin{cases}
            {\color{red} \eli_1R_1 - \eli_2R_2 = \ele_1 - \ele_2}, \\
            {\color{blue} \eli_2R_2 - \eli_3R_3 = \ele_2 - \ele_3}, \\
            {\color{green!71!black} \eli_1 + \eli_2 + \eli_3 = 0};
        \end{cases}
        \qquad \implies \qquad
        \begin{cases}
            \eli_1 = \frac{\ele_1 - \ele_2 + \eli_2R_2}{R_1}, \\
            \eli_3 = \frac{\eli_2R_2 - \ele_2 + \ele_3}{R_3}, \\
            \eli_1 + \eli_2 + \eli_3 = 0, \\
        \end{cases} \implies \\
        \implies
            &\eli_2 + \frac{\ele_1 - \ele_2 + \eli_2R_2}{R_1} + \frac{\eli_2R_2 - \ele_2 + \ele_3}{R_3} = 0, \\
        &   \eli_2\cbr{1 + \frac{ R_2 }{ R_1 } + \frac{ R_2 }{ R_3 }} + \frac{\ele_1 - \ele_2}{ R_1 } + \frac{\ele_3 - \ele_2}{ R_3 } = 0, \\
        &   \eli_2 = \cfrac{\cfrac{\ele_2 - \ele_1}{ R_1 } + \cfrac{\ele_2 - \ele_3}{ R_3 }}{1 + \cfrac{ R_2 }{ R_1 } + \cfrac{ R_2 }{ R_3 }}
            = \cfrac{\cfrac{6\,\text{В} - 5\,\text{В}}{ 4\,\text{Ом} } + \cfrac{6\,\text{В} - 8\,\text{В}}{ 12\,\text{Ом} }}{1 + \cfrac{ 8\,\text{Ом} }{ 4\,\text{Ом} } + \cfrac{ 8\,\text{Ом} }{ 12\,\text{Ом} }}
            = \frac1{44}\units{А} \approx 0{,}02\,\text{А}, \\
        &   U_2 = \eli_2R_2 = \cfrac{\cfrac{\ele_2 - \ele_1}{ R_1 } + \cfrac{\ele_2 - \ele_3}{ R_3 }}{1 + \cfrac{ R_2 }{ R_1 } + \cfrac{ R_2 }{ R_3 }} \cdot R_2
            = \cfrac{\cfrac{6\,\text{В} - 5\,\text{В}}{ 4\,\text{Ом} } + \cfrac{6\,\text{В} - 8\,\text{В}}{ 12\,\text{Ом} }}{1 + \cfrac{ 8\,\text{Ом} }{ 4\,\text{Ом} } + \cfrac{ 8\,\text{Ом} }{ 12\,\text{Ом} }} \cdot 8\,\text{Ом}
            = \frac1{44}\units{А} \cdot 8\,\text{Ом} = \frac2{11}\units{В} \approx 0{,}18\,\text{В}.
    \end{align*}

    Одну пару силы тока и напряжения получили.
    Для некоторых вариантов это уже ответ, но не у всех.
    Для упрощения записи преобразуем (чтобы избавитсья от 4-этажной дроби) и подставим в уже полученные уравнения:

    \begin{align*}
    \eli_2
        &=
        \frac{\frac{\ele_2 - \ele_1}{ R_1 } + \frac{\ele_2 - \ele_3}{ R_3 }}{1 + \frac{ R_2 }{ R_1 } + \frac{ R_2 }{ R_3 }}
        =
        \frac{(\ele_2 - \ele_1)R_3 + (\ele_2 - \ele_3)R_1}{R_1R_3 + R_2R_3 + R_2R_1},
        \\
    \eli_1
        &=  \frac{\ele_1 - \ele_2 + \eli_2R_2}{R_1}
        =   \frac{\ele_1 - \ele_2 + \cfrac{(\ele_2 - \ele_1)R_3 + (\ele_2 - \ele_3)R_1}{R_1R_3 + R_2R_3 + R_2R_1} \cdot R_2}{R_1} = \\
        &=  \frac{
            \ele_1R_1R_3 + \ele_1R_2R_3 + \ele_1R_2R_1
            - \ele_2R_1R_3 - \ele_2R_2R_3 - \ele_2R_2R_1
            + \ele_2R_3R_2 - \ele_1R_3R_2 + \ele_2R_1R_2 - \ele_3R_1R_2
       }{R_1 \cdot \cbr{R_1R_3 + R_2R_3 + R_2R_1}}
        = \\ &=
        \frac{
            \ele_1\cbr{R_1R_3 + R_2R_3 + R_2R_1 - R_3R_2}
            + \ele_2\cbr{- R_1R_3 - R_2R_3 - R_2R_1 + R_3R_2 + R_1R_2}
            - \ele_3R_1R_2
       }{R_1 \cdot \cbr{R_1R_3 + R_2R_3 + R_2R_1}}
        = \\ &=
        \frac{
            \ele_1\cbr{R_1R_3 + R_2R_1}
            + \ele_2\cbr{- R_1R_3}
            - \ele_3R_1R_2
       }{R_1 \cdot \cbr{R_1R_3 + R_2R_3 + R_2R_1}}
        =
        \frac{
            \ele_1\cbr{R_3 + R_2} - \ele_2R_3 - \ele_3R_2
       }{R_1R_3 + R_2R_3 + R_2R_1}
        = \\ &=
        \frac{
            (\ele_1 - \ele_3)R_2 + (\ele_1 - \ele_2)R_3
       }{R_1R_3 + R_2R_3 + R_2R_1}
        =
        \frac{
            \cfrac{\ele_1 - \ele_3}{ R_3 } + \cfrac{\ele_1 - \ele_2}{ R_2 }
       }{\cfrac{ R_1 }{ R_2 } + 1 + \cfrac{ R_1 }{ R_3 }}
        =
        \frac{
            \cfrac{5\,\text{В} - 8\,\text{В}}{ 12\,\text{Ом} } + \cfrac{5\,\text{В} - 6\,\text{В}}{ 8\,\text{Ом} }
       }{\cfrac{ 4\,\text{Ом} }{ 8\,\text{Ом} } + 1 + \cfrac{ 4\,\text{Ом} }{ 12\,\text{Ом} }}
        = -\frac9{44}\units{А} \approx -0{,}2000\,\text{А}.
        \\
    U_1
        &=
        \eli_1R_1
        =
        \frac{
            \cfrac{\ele_1 - \ele_3}{ R_3 } + \cfrac{\ele_1 - \ele_2}{ R_2 }
       }{\cfrac{ R_1 }{ R_2 } + 1 + \cfrac{ R_1 }{ R_3 }} \cdot R_1
        =
        -\frac9{44}\units{А} \cdot 4\,\text{Ом} = -\frac9{11}\units{В} \approx -0{,}8200\,\text{В}.
    \end{align*}

    Если вы проделали все эти вычисления выше вместе со мной, то
    \begin{itemize}
        \item вы совершили ошибку, выбрав неверный путь решения:
        слишком длинное решение, очень легко ошибиться в индексах, дробях, знаках или потерять какой-то множитель,
        \item можно было выразить из исходной системы другие токи и получить сразу нажный вам,
        а не какой-то 2-й,
        \item можно было сэкономить: все три резистора и ЭДС соединены одинаково,
        поэтому ответ для 1-го резистора должен отличаться лишь перестановкой индексов (этот факт крайне полезен при проверке ответа, у нас всё сошлось),
        я специально подгонял выражение для $\eli_1$ к этому виду, вынося за скобки и преобразуя дробь,
        \item вы молодец, потому что не побоялись и получили верный ответ грамотным способом,
    \end{itemize}
    так что переходим к третьему резистору.
    Будет похоже, но кого это когда останавливало...

    \begin{align*}
    \eli_3
        &=  \frac{\eli_2R_2 - \ele_2 + \ele_3}{ R_3 }
        =
        \cfrac{
            \cfrac{
                (\ele_2 - \ele_1)R_3 + (\ele_2 - \ele_3)R_1
           }{
                R_1R_3 + R_2R_3 + R_2R_1
           } \cdot R_2 - \ele_2 + \ele_3}{ R_3 }
        = \\ &=
        \frac{
            \ele_2R_3R_2 - \ele_1R_3R_2 + \ele_2R_1R_2 - \ele_3R_1R_2
            - \ele_2R_1R_3 - \ele_2R_2R_3 - \ele_2R_2R_1
            + \ele_3R_1R_3 + \ele_3R_2R_3 + \ele_3R_2R_1
       }{\cbr{R_1R_3 + R_2R_3 + R_2R_1} \cdot R_3}
        = \\ &=
        \frac{
            - \ele_1R_3R_2 - \ele_2R_1R_3 + \ele_3R_1R_3 + \ele_3R_2R_3
       }{\cbr{R_1R_3 + R_2R_3 + R_2R_1} \cdot R_3}
        =
        \frac{
            - \ele_1R_2 - \ele_2R_1 + \ele_3R_1 + \ele_3R_2
       }{R_1R_3 + R_2R_3 + R_2R_1}
        = \\ &=
        \frac{
            R_1(\ele_3 - \ele_2) + R_2(\ele_3 - \ele_1)
       }{R_1R_3 + R_2R_3 + R_2R_1}
        =
        \frac{
            \cfrac{\ele_3 - \ele_2}{ R_2 } + \cfrac{\ele_3 - \ele_1}{ R_1 }
       }{\cfrac{ R_3 }{ R_2 } + \cfrac{ R_3 }{ R_1 } + 1}
        =
        \frac{
            \cfrac{8\,\text{В} - 6\,\text{В}}{ 8\,\text{Ом} } + \cfrac{8\,\text{В} - 5\,\text{В}}{ 4\,\text{Ом} }
       }{\cfrac{ 12\,\text{Ом} }{ 8\,\text{Ом} } + \cfrac{ 12\,\text{Ом} }{ 4\,\text{Ом} } + 1}
        = \frac2{11}\units{А} \approx 0{,}18\,\text{А}.
        \\
    U_3
        &=
        \eli_3R_3
        =
        \frac{
            \cfrac{\ele_3 - \ele_2}{ R_2 } + \cfrac{\ele_3 - \ele_1}{ R_1 }
       }{\cfrac{ R_3 }{ R_2 } + \cfrac{ R_3 }{ R_1 } + 1} \cdot R_3
        =
        \frac2{11}\units{А} \cdot 12\,\text{Ом} = \frac{24}{11}\units{В} \approx 2{,}18\,\text{В}.
    \end{align*}

    Положительные ответы говорят, что мы угадали на рисунке направление тока (тут нет нашей заслуги, повезло),
    отрицательные — что не угадали (и в этом нет ошибки), и ток течёт в противоположную сторону.
    Напомним, что направление тока — это направление движения положительных зарядов,
    а в металлах носители заряда — электроны, которые заряжены отрицательно.
}

\variantsplitter

\addpersonalvariant{Евгений Васин}

\tasknumber{1}%
\task{%
    Определите ток $\eli_1$, протекающий через резистор $R_1$ (см.
    рис.),
    направление этого тока и разность потенциалов $U_1$ на этом резисторе,
    если $R_1 = 3\,\text{Ом}$, $R_2 = 5\,\text{Ом}$, $R_3 = 10\,\text{Ом}$, $\ele_1 = 4\,\text{В}$, $\ele_2 = 6\,\text{В}$, $\ele_3 = 8\,\text{В}$.
    Внутренним сопротивлением всех трёх ЭДС пренебречь.
    Ответы получите в виде несократимых дробей, а также определите приближённые значения.

    \begin{tikzpicture}[circuit ee IEC, thick]
        \foreach \contact/\x in {1/0, 2/3, 3/6}
        {
            \node [contact] (top contact \contact) at (\x, 0) {};
            \node [contact] (bottom contact \contact) at (\x, 4) {};
       }
        \draw  (bottom contact 1) -- (bottom contact 2) -- (bottom contact 3);
        \draw  (top contact 1) -- (top contact 2) -- (top contact 3);
        \draw  (bottom contact 1) to [resistor={near start, info=$R_1$}, battery={near end, info=$\ele_1$}] (top contact 1);
        \draw  (bottom contact 2) to [resistor={near start, info=$R_2$}, battery={near end, info=$\ele_2$}] (top contact 2);
        \draw  (bottom contact 3) to [resistor={near start, info=$R_3$}, battery={near end, info=$\ele_3$}] (top contact 3);
    \end{tikzpicture}
}
\answer{%
    План:
    \begin{itemize}
        \item отметим на рисунке произвольно направления токов (если получим отрицательный ответ, значит не угадали направление и только),
        \item выберем и обозначим на рисунке контуры (здесь всего 3, значит будет нужно $3-1=2$), для них запишем законы Кирхгофа,
        \item выберем и выделим на рисунке нетривиальные узлы (здесь всего 2, значит будет нужно $2-1=1$), для него запишем закон Кирхгофа,
        \item попытаемся решить получившуюся систему.
        В конкретном решении мы пытались первым делом найти $\eli_2$, но, возможно, в вашем варианте будет быстрее решать систему в другом порядке.
        Мы всё же проделаем всё в лоб, подробно и целиком.
    \end{itemize}


    \begin{tikzpicture}[circuit ee IEC, thick]
        \foreach \contact/\x in {1/0, 2/3, 3/6}
        {
            \node [contact] (top contact \contact) at (\x, 0) {};
            \node [contact] (bottom contact \contact) at (\x, 4) {};
       }
        \draw  (bottom contact 1) -- (bottom contact 2) -- (bottom contact 3);
        \draw  (top contact 1) -- (top contact 2) -- (top contact 3);
        \draw  (bottom contact 1) to [resistor={near start, info=$R_1$}, current direction'={midway, info=$\eli_1$}, battery={near end, info=$\ele_1$}] (top contact 1);
        \draw  (bottom contact 2) to [resistor={near start, info=$R_2$}, current direction'={midway, info=$\eli_2$}, battery={near end, info=$\ele_2$}] (top contact 2);
        \draw  (bottom contact 3) to [resistor={near start, info=$R_3$}, current direction'={midway, info=$\eli_3$}, battery={near end, info=$\ele_3$}] (top contact 3);
        \draw [-{Latex},color=red] (1.2, 2.5) arc [start angle = 135, end angle = -160, radius = 0.6];
        \draw [-{Latex},color=blue] (4.2, 2.5) arc [start angle = 135, end angle = -160, radius = 0.6];
        \node [contact,color=green!71!black] (bottomc) at (bottom contact 2) {};
    \end{tikzpicture}

    \begin{align*}
        &\begin{cases}
            {\color{red} \eli_1R_1 - \eli_2R_2 = \ele_1 - \ele_2}, \\
            {\color{blue} \eli_2R_2 - \eli_3R_3 = \ele_2 - \ele_3}, \\
            {\color{green!71!black} \eli_1 + \eli_2 + \eli_3 = 0};
        \end{cases}
        \qquad \implies \qquad
        \begin{cases}
            \eli_1 = \frac{\ele_1 - \ele_2 + \eli_2R_2}{R_1}, \\
            \eli_3 = \frac{\eli_2R_2 - \ele_2 + \ele_3}{R_3}, \\
            \eli_1 + \eli_2 + \eli_3 = 0, \\
        \end{cases} \implies \\
        \implies
            &\eli_2 + \frac{\ele_1 - \ele_2 + \eli_2R_2}{R_1} + \frac{\eli_2R_2 - \ele_2 + \ele_3}{R_3} = 0, \\
        &   \eli_2\cbr{1 + \frac{ R_2 }{ R_1 } + \frac{ R_2 }{ R_3 }} + \frac{\ele_1 - \ele_2}{ R_1 } + \frac{\ele_3 - \ele_2}{ R_3 } = 0, \\
        &   \eli_2 = \cfrac{\cfrac{\ele_2 - \ele_1}{ R_1 } + \cfrac{\ele_2 - \ele_3}{ R_3 }}{1 + \cfrac{ R_2 }{ R_1 } + \cfrac{ R_2 }{ R_3 }}
            = \cfrac{\cfrac{6\,\text{В} - 4\,\text{В}}{ 3\,\text{Ом} } + \cfrac{6\,\text{В} - 8\,\text{В}}{ 10\,\text{Ом} }}{1 + \cfrac{ 5\,\text{Ом} }{ 3\,\text{Ом} } + \cfrac{ 5\,\text{Ом} }{ 10\,\text{Ом} }}
            = \frac{14}{95}\units{А} \approx 0{,}15\,\text{А}, \\
        &   U_2 = \eli_2R_2 = \cfrac{\cfrac{\ele_2 - \ele_1}{ R_1 } + \cfrac{\ele_2 - \ele_3}{ R_3 }}{1 + \cfrac{ R_2 }{ R_1 } + \cfrac{ R_2 }{ R_3 }} \cdot R_2
            = \cfrac{\cfrac{6\,\text{В} - 4\,\text{В}}{ 3\,\text{Ом} } + \cfrac{6\,\text{В} - 8\,\text{В}}{ 10\,\text{Ом} }}{1 + \cfrac{ 5\,\text{Ом} }{ 3\,\text{Ом} } + \cfrac{ 5\,\text{Ом} }{ 10\,\text{Ом} }} \cdot 5\,\text{Ом}
            = \frac{14}{95}\units{А} \cdot 5\,\text{Ом} = \frac{14}{19}\units{В} \approx 0{,}74\,\text{В}.
    \end{align*}

    Одну пару силы тока и напряжения получили.
    Для некоторых вариантов это уже ответ, но не у всех.
    Для упрощения записи преобразуем (чтобы избавитсья от 4-этажной дроби) и подставим в уже полученные уравнения:

    \begin{align*}
    \eli_2
        &=
        \frac{\frac{\ele_2 - \ele_1}{ R_1 } + \frac{\ele_2 - \ele_3}{ R_3 }}{1 + \frac{ R_2 }{ R_1 } + \frac{ R_2 }{ R_3 }}
        =
        \frac{(\ele_2 - \ele_1)R_3 + (\ele_2 - \ele_3)R_1}{R_1R_3 + R_2R_3 + R_2R_1},
        \\
    \eli_1
        &=  \frac{\ele_1 - \ele_2 + \eli_2R_2}{R_1}
        =   \frac{\ele_1 - \ele_2 + \cfrac{(\ele_2 - \ele_1)R_3 + (\ele_2 - \ele_3)R_1}{R_1R_3 + R_2R_3 + R_2R_1} \cdot R_2}{R_1} = \\
        &=  \frac{
            \ele_1R_1R_3 + \ele_1R_2R_3 + \ele_1R_2R_1
            - \ele_2R_1R_3 - \ele_2R_2R_3 - \ele_2R_2R_1
            + \ele_2R_3R_2 - \ele_1R_3R_2 + \ele_2R_1R_2 - \ele_3R_1R_2
       }{R_1 \cdot \cbr{R_1R_3 + R_2R_3 + R_2R_1}}
        = \\ &=
        \frac{
            \ele_1\cbr{R_1R_3 + R_2R_3 + R_2R_1 - R_3R_2}
            + \ele_2\cbr{- R_1R_3 - R_2R_3 - R_2R_1 + R_3R_2 + R_1R_2}
            - \ele_3R_1R_2
       }{R_1 \cdot \cbr{R_1R_3 + R_2R_3 + R_2R_1}}
        = \\ &=
        \frac{
            \ele_1\cbr{R_1R_3 + R_2R_1}
            + \ele_2\cbr{- R_1R_3}
            - \ele_3R_1R_2
       }{R_1 \cdot \cbr{R_1R_3 + R_2R_3 + R_2R_1}}
        =
        \frac{
            \ele_1\cbr{R_3 + R_2} - \ele_2R_3 - \ele_3R_2
       }{R_1R_3 + R_2R_3 + R_2R_1}
        = \\ &=
        \frac{
            (\ele_1 - \ele_3)R_2 + (\ele_1 - \ele_2)R_3
       }{R_1R_3 + R_2R_3 + R_2R_1}
        =
        \frac{
            \cfrac{\ele_1 - \ele_3}{ R_3 } + \cfrac{\ele_1 - \ele_2}{ R_2 }
       }{\cfrac{ R_1 }{ R_2 } + 1 + \cfrac{ R_1 }{ R_3 }}
        =
        \frac{
            \cfrac{4\,\text{В} - 8\,\text{В}}{ 10\,\text{Ом} } + \cfrac{4\,\text{В} - 6\,\text{В}}{ 5\,\text{Ом} }
       }{\cfrac{ 3\,\text{Ом} }{ 5\,\text{Ом} } + 1 + \cfrac{ 3\,\text{Ом} }{ 10\,\text{Ом} }}
        = -\frac8{19}\units{А} \approx -0{,}4200\,\text{А}.
        \\
    U_1
        &=
        \eli_1R_1
        =
        \frac{
            \cfrac{\ele_1 - \ele_3}{ R_3 } + \cfrac{\ele_1 - \ele_2}{ R_2 }
       }{\cfrac{ R_1 }{ R_2 } + 1 + \cfrac{ R_1 }{ R_3 }} \cdot R_1
        =
        -\frac8{19}\units{А} \cdot 3\,\text{Ом} = -\frac{24}{19}\units{В} \approx -1{,}2600\,\text{В}.
    \end{align*}

    Если вы проделали все эти вычисления выше вместе со мной, то
    \begin{itemize}
        \item вы совершили ошибку, выбрав неверный путь решения:
        слишком длинное решение, очень легко ошибиться в индексах, дробях, знаках или потерять какой-то множитель,
        \item можно было выразить из исходной системы другие токи и получить сразу нажный вам,
        а не какой-то 2-й,
        \item можно было сэкономить: все три резистора и ЭДС соединены одинаково,
        поэтому ответ для 1-го резистора должен отличаться лишь перестановкой индексов (этот факт крайне полезен при проверке ответа, у нас всё сошлось),
        я специально подгонял выражение для $\eli_1$ к этому виду, вынося за скобки и преобразуя дробь,
        \item вы молодец, потому что не побоялись и получили верный ответ грамотным способом,
    \end{itemize}
    так что переходим к третьему резистору.
    Будет похоже, но кого это когда останавливало...

    \begin{align*}
    \eli_3
        &=  \frac{\eli_2R_2 - \ele_2 + \ele_3}{ R_3 }
        =
        \cfrac{
            \cfrac{
                (\ele_2 - \ele_1)R_3 + (\ele_2 - \ele_3)R_1
           }{
                R_1R_3 + R_2R_3 + R_2R_1
           } \cdot R_2 - \ele_2 + \ele_3}{ R_3 }
        = \\ &=
        \frac{
            \ele_2R_3R_2 - \ele_1R_3R_2 + \ele_2R_1R_2 - \ele_3R_1R_2
            - \ele_2R_1R_3 - \ele_2R_2R_3 - \ele_2R_2R_1
            + \ele_3R_1R_3 + \ele_3R_2R_3 + \ele_3R_2R_1
       }{\cbr{R_1R_3 + R_2R_3 + R_2R_1} \cdot R_3}
        = \\ &=
        \frac{
            - \ele_1R_3R_2 - \ele_2R_1R_3 + \ele_3R_1R_3 + \ele_3R_2R_3
       }{\cbr{R_1R_3 + R_2R_3 + R_2R_1} \cdot R_3}
        =
        \frac{
            - \ele_1R_2 - \ele_2R_1 + \ele_3R_1 + \ele_3R_2
       }{R_1R_3 + R_2R_3 + R_2R_1}
        = \\ &=
        \frac{
            R_1(\ele_3 - \ele_2) + R_2(\ele_3 - \ele_1)
       }{R_1R_3 + R_2R_3 + R_2R_1}
        =
        \frac{
            \cfrac{\ele_3 - \ele_2}{ R_2 } + \cfrac{\ele_3 - \ele_1}{ R_1 }
       }{\cfrac{ R_3 }{ R_2 } + \cfrac{ R_3 }{ R_1 } + 1}
        =
        \frac{
            \cfrac{8\,\text{В} - 6\,\text{В}}{ 5\,\text{Ом} } + \cfrac{8\,\text{В} - 4\,\text{В}}{ 3\,\text{Ом} }
       }{\cfrac{ 10\,\text{Ом} }{ 5\,\text{Ом} } + \cfrac{ 10\,\text{Ом} }{ 3\,\text{Ом} } + 1}
        = \frac{26}{95}\units{А} \approx 0{,}27\,\text{А}.
        \\
    U_3
        &=
        \eli_3R_3
        =
        \frac{
            \cfrac{\ele_3 - \ele_2}{ R_2 } + \cfrac{\ele_3 - \ele_1}{ R_1 }
       }{\cfrac{ R_3 }{ R_2 } + \cfrac{ R_3 }{ R_1 } + 1} \cdot R_3
        =
        \frac{26}{95}\units{А} \cdot 10\,\text{Ом} = \frac{52}{19}\units{В} \approx 2{,}74\,\text{В}.
    \end{align*}

    Положительные ответы говорят, что мы угадали на рисунке направление тока (тут нет нашей заслуги, повезло),
    отрицательные — что не угадали (и в этом нет ошибки), и ток течёт в противоположную сторону.
    Напомним, что направление тока — это направление движения положительных зарядов,
    а в металлах носители заряда — электроны, которые заряжены отрицательно.
}

\variantsplitter

\addpersonalvariant{Вячеслав Волохов}

\tasknumber{1}%
\task{%
    Определите ток $\eli_3$, протекающий через резистор $R_3$ (см.
    рис.),
    направление этого тока и разность потенциалов $U_3$ на этом резисторе,
    если $R_1 = 4\,\text{Ом}$, $R_2 = 6\,\text{Ом}$, $R_3 = 12\,\text{Ом}$, $\ele_1 = 5\,\text{В}$, $\ele_2 = 3\,\text{В}$, $\ele_3 = 8\,\text{В}$.
    Внутренним сопротивлением всех трёх ЭДС пренебречь.
    Ответы получите в виде несократимых дробей, а также определите приближённые значения.

    \begin{tikzpicture}[circuit ee IEC, thick]
        \foreach \contact/\x in {1/0, 2/3, 3/6}
        {
            \node [contact] (top contact \contact) at (\x, 0) {};
            \node [contact] (bottom contact \contact) at (\x, 4) {};
       }
        \draw  (bottom contact 1) -- (bottom contact 2) -- (bottom contact 3);
        \draw  (top contact 1) -- (top contact 2) -- (top contact 3);
        \draw  (bottom contact 1) to [resistor={near start, info=$R_1$}, battery={near end, info=$\ele_1$}] (top contact 1);
        \draw  (bottom contact 2) to [resistor={near start, info=$R_2$}, battery={near end, info=$\ele_2$}] (top contact 2);
        \draw  (bottom contact 3) to [resistor={near start, info=$R_3$}, battery={near end, info=$\ele_3$}] (top contact 3);
    \end{tikzpicture}
}
\answer{%
    План:
    \begin{itemize}
        \item отметим на рисунке произвольно направления токов (если получим отрицательный ответ, значит не угадали направление и только),
        \item выберем и обозначим на рисунке контуры (здесь всего 3, значит будет нужно $3-1=2$), для них запишем законы Кирхгофа,
        \item выберем и выделим на рисунке нетривиальные узлы (здесь всего 2, значит будет нужно $2-1=1$), для него запишем закон Кирхгофа,
        \item попытаемся решить получившуюся систему.
        В конкретном решении мы пытались первым делом найти $\eli_2$, но, возможно, в вашем варианте будет быстрее решать систему в другом порядке.
        Мы всё же проделаем всё в лоб, подробно и целиком.
    \end{itemize}


    \begin{tikzpicture}[circuit ee IEC, thick]
        \foreach \contact/\x in {1/0, 2/3, 3/6}
        {
            \node [contact] (top contact \contact) at (\x, 0) {};
            \node [contact] (bottom contact \contact) at (\x, 4) {};
       }
        \draw  (bottom contact 1) -- (bottom contact 2) -- (bottom contact 3);
        \draw  (top contact 1) -- (top contact 2) -- (top contact 3);
        \draw  (bottom contact 1) to [resistor={near start, info=$R_1$}, current direction'={midway, info=$\eli_1$}, battery={near end, info=$\ele_1$}] (top contact 1);
        \draw  (bottom contact 2) to [resistor={near start, info=$R_2$}, current direction'={midway, info=$\eli_2$}, battery={near end, info=$\ele_2$}] (top contact 2);
        \draw  (bottom contact 3) to [resistor={near start, info=$R_3$}, current direction'={midway, info=$\eli_3$}, battery={near end, info=$\ele_3$}] (top contact 3);
        \draw [-{Latex},color=red] (1.2, 2.5) arc [start angle = 135, end angle = -160, radius = 0.6];
        \draw [-{Latex},color=blue] (4.2, 2.5) arc [start angle = 135, end angle = -160, radius = 0.6];
        \node [contact,color=green!71!black] (bottomc) at (bottom contact 2) {};
    \end{tikzpicture}

    \begin{align*}
        &\begin{cases}
            {\color{red} \eli_1R_1 - \eli_2R_2 = \ele_1 - \ele_2}, \\
            {\color{blue} \eli_2R_2 - \eli_3R_3 = \ele_2 - \ele_3}, \\
            {\color{green!71!black} \eli_1 + \eli_2 + \eli_3 = 0};
        \end{cases}
        \qquad \implies \qquad
        \begin{cases}
            \eli_1 = \frac{\ele_1 - \ele_2 + \eli_2R_2}{R_1}, \\
            \eli_3 = \frac{\eli_2R_2 - \ele_2 + \ele_3}{R_3}, \\
            \eli_1 + \eli_2 + \eli_3 = 0, \\
        \end{cases} \implies \\
        \implies
            &\eli_2 + \frac{\ele_1 - \ele_2 + \eli_2R_2}{R_1} + \frac{\eli_2R_2 - \ele_2 + \ele_3}{R_3} = 0, \\
        &   \eli_2\cbr{1 + \frac{ R_2 }{ R_1 } + \frac{ R_2 }{ R_3 }} + \frac{\ele_1 - \ele_2}{ R_1 } + \frac{\ele_3 - \ele_2}{ R_3 } = 0, \\
        &   \eli_2 = \cfrac{\cfrac{\ele_2 - \ele_1}{ R_1 } + \cfrac{\ele_2 - \ele_3}{ R_3 }}{1 + \cfrac{ R_2 }{ R_1 } + \cfrac{ R_2 }{ R_3 }}
            = \cfrac{\cfrac{3\,\text{В} - 5\,\text{В}}{ 4\,\text{Ом} } + \cfrac{3\,\text{В} - 8\,\text{В}}{ 12\,\text{Ом} }}{1 + \cfrac{ 6\,\text{Ом} }{ 4\,\text{Ом} } + \cfrac{ 6\,\text{Ом} }{ 12\,\text{Ом} }}
            = -\frac{11}{36}\units{А} \approx -0{,}3100\,\text{А}, \\
        &   U_2 = \eli_2R_2 = \cfrac{\cfrac{\ele_2 - \ele_1}{ R_1 } + \cfrac{\ele_2 - \ele_3}{ R_3 }}{1 + \cfrac{ R_2 }{ R_1 } + \cfrac{ R_2 }{ R_3 }} \cdot R_2
            = \cfrac{\cfrac{3\,\text{В} - 5\,\text{В}}{ 4\,\text{Ом} } + \cfrac{3\,\text{В} - 8\,\text{В}}{ 12\,\text{Ом} }}{1 + \cfrac{ 6\,\text{Ом} }{ 4\,\text{Ом} } + \cfrac{ 6\,\text{Ом} }{ 12\,\text{Ом} }} \cdot 6\,\text{Ом}
            = -\frac{11}{36}\units{А} \cdot 6\,\text{Ом} = -\frac{11}6\units{В} \approx -1{,}8300\,\text{В}.
    \end{align*}

    Одну пару силы тока и напряжения получили.
    Для некоторых вариантов это уже ответ, но не у всех.
    Для упрощения записи преобразуем (чтобы избавитсья от 4-этажной дроби) и подставим в уже полученные уравнения:

    \begin{align*}
    \eli_2
        &=
        \frac{\frac{\ele_2 - \ele_1}{ R_1 } + \frac{\ele_2 - \ele_3}{ R_3 }}{1 + \frac{ R_2 }{ R_1 } + \frac{ R_2 }{ R_3 }}
        =
        \frac{(\ele_2 - \ele_1)R_3 + (\ele_2 - \ele_3)R_1}{R_1R_3 + R_2R_3 + R_2R_1},
        \\
    \eli_1
        &=  \frac{\ele_1 - \ele_2 + \eli_2R_2}{R_1}
        =   \frac{\ele_1 - \ele_2 + \cfrac{(\ele_2 - \ele_1)R_3 + (\ele_2 - \ele_3)R_1}{R_1R_3 + R_2R_3 + R_2R_1} \cdot R_2}{R_1} = \\
        &=  \frac{
            \ele_1R_1R_3 + \ele_1R_2R_3 + \ele_1R_2R_1
            - \ele_2R_1R_3 - \ele_2R_2R_3 - \ele_2R_2R_1
            + \ele_2R_3R_2 - \ele_1R_3R_2 + \ele_2R_1R_2 - \ele_3R_1R_2
       }{R_1 \cdot \cbr{R_1R_3 + R_2R_3 + R_2R_1}}
        = \\ &=
        \frac{
            \ele_1\cbr{R_1R_3 + R_2R_3 + R_2R_1 - R_3R_2}
            + \ele_2\cbr{- R_1R_3 - R_2R_3 - R_2R_1 + R_3R_2 + R_1R_2}
            - \ele_3R_1R_2
       }{R_1 \cdot \cbr{R_1R_3 + R_2R_3 + R_2R_1}}
        = \\ &=
        \frac{
            \ele_1\cbr{R_1R_3 + R_2R_1}
            + \ele_2\cbr{- R_1R_3}
            - \ele_3R_1R_2
       }{R_1 \cdot \cbr{R_1R_3 + R_2R_3 + R_2R_1}}
        =
        \frac{
            \ele_1\cbr{R_3 + R_2} - \ele_2R_3 - \ele_3R_2
       }{R_1R_3 + R_2R_3 + R_2R_1}
        = \\ &=
        \frac{
            (\ele_1 - \ele_3)R_2 + (\ele_1 - \ele_2)R_3
       }{R_1R_3 + R_2R_3 + R_2R_1}
        =
        \frac{
            \cfrac{\ele_1 - \ele_3}{ R_3 } + \cfrac{\ele_1 - \ele_2}{ R_2 }
       }{\cfrac{ R_1 }{ R_2 } + 1 + \cfrac{ R_1 }{ R_3 }}
        =
        \frac{
            \cfrac{5\,\text{В} - 8\,\text{В}}{ 12\,\text{Ом} } + \cfrac{5\,\text{В} - 3\,\text{В}}{ 6\,\text{Ом} }
       }{\cfrac{ 4\,\text{Ом} }{ 6\,\text{Ом} } + 1 + \cfrac{ 4\,\text{Ом} }{ 12\,\text{Ом} }}
        = \frac1{24}\units{А} \approx 0{,}04\,\text{А}.
        \\
    U_1
        &=
        \eli_1R_1
        =
        \frac{
            \cfrac{\ele_1 - \ele_3}{ R_3 } + \cfrac{\ele_1 - \ele_2}{ R_2 }
       }{\cfrac{ R_1 }{ R_2 } + 1 + \cfrac{ R_1 }{ R_3 }} \cdot R_1
        =
        \frac1{24}\units{А} \cdot 4\,\text{Ом} = \frac16\units{В} \approx 0{,}17\,\text{В}.
    \end{align*}

    Если вы проделали все эти вычисления выше вместе со мной, то
    \begin{itemize}
        \item вы совершили ошибку, выбрав неверный путь решения:
        слишком длинное решение, очень легко ошибиться в индексах, дробях, знаках или потерять какой-то множитель,
        \item можно было выразить из исходной системы другие токи и получить сразу нажный вам,
        а не какой-то 2-й,
        \item можно было сэкономить: все три резистора и ЭДС соединены одинаково,
        поэтому ответ для 1-го резистора должен отличаться лишь перестановкой индексов (этот факт крайне полезен при проверке ответа, у нас всё сошлось),
        я специально подгонял выражение для $\eli_1$ к этому виду, вынося за скобки и преобразуя дробь,
        \item вы молодец, потому что не побоялись и получили верный ответ грамотным способом,
    \end{itemize}
    так что переходим к третьему резистору.
    Будет похоже, но кого это когда останавливало...

    \begin{align*}
    \eli_3
        &=  \frac{\eli_2R_2 - \ele_2 + \ele_3}{ R_3 }
        =
        \cfrac{
            \cfrac{
                (\ele_2 - \ele_1)R_3 + (\ele_2 - \ele_3)R_1
           }{
                R_1R_3 + R_2R_3 + R_2R_1
           } \cdot R_2 - \ele_2 + \ele_3}{ R_3 }
        = \\ &=
        \frac{
            \ele_2R_3R_2 - \ele_1R_3R_2 + \ele_2R_1R_2 - \ele_3R_1R_2
            - \ele_2R_1R_3 - \ele_2R_2R_3 - \ele_2R_2R_1
            + \ele_3R_1R_3 + \ele_3R_2R_3 + \ele_3R_2R_1
       }{\cbr{R_1R_3 + R_2R_3 + R_2R_1} \cdot R_3}
        = \\ &=
        \frac{
            - \ele_1R_3R_2 - \ele_2R_1R_3 + \ele_3R_1R_3 + \ele_3R_2R_3
       }{\cbr{R_1R_3 + R_2R_3 + R_2R_1} \cdot R_3}
        =
        \frac{
            - \ele_1R_2 - \ele_2R_1 + \ele_3R_1 + \ele_3R_2
       }{R_1R_3 + R_2R_3 + R_2R_1}
        = \\ &=
        \frac{
            R_1(\ele_3 - \ele_2) + R_2(\ele_3 - \ele_1)
       }{R_1R_3 + R_2R_3 + R_2R_1}
        =
        \frac{
            \cfrac{\ele_3 - \ele_2}{ R_2 } + \cfrac{\ele_3 - \ele_1}{ R_1 }
       }{\cfrac{ R_3 }{ R_2 } + \cfrac{ R_3 }{ R_1 } + 1}
        =
        \frac{
            \cfrac{8\,\text{В} - 3\,\text{В}}{ 6\,\text{Ом} } + \cfrac{8\,\text{В} - 5\,\text{В}}{ 4\,\text{Ом} }
       }{\cfrac{ 12\,\text{Ом} }{ 6\,\text{Ом} } + \cfrac{ 12\,\text{Ом} }{ 4\,\text{Ом} } + 1}
        = \frac{19}{72}\units{А} \approx 0{,}26\,\text{А}.
        \\
    U_3
        &=
        \eli_3R_3
        =
        \frac{
            \cfrac{\ele_3 - \ele_2}{ R_2 } + \cfrac{\ele_3 - \ele_1}{ R_1 }
       }{\cfrac{ R_3 }{ R_2 } + \cfrac{ R_3 }{ R_1 } + 1} \cdot R_3
        =
        \frac{19}{72}\units{А} \cdot 12\,\text{Ом} = \frac{19}6\units{В} \approx 3{,}17\,\text{В}.
    \end{align*}

    Положительные ответы говорят, что мы угадали на рисунке направление тока (тут нет нашей заслуги, повезло),
    отрицательные — что не угадали (и в этом нет ошибки), и ток течёт в противоположную сторону.
    Напомним, что направление тока — это направление движения положительных зарядов,
    а в металлах носители заряда — электроны, которые заряжены отрицательно.
}

\variantsplitter

\addpersonalvariant{Герман Говоров}

\tasknumber{1}%
\task{%
    Определите ток $\eli_1$, протекающий через резистор $R_1$ (см.
    рис.),
    направление этого тока и разность потенциалов $U_1$ на этом резисторе,
    если $R_1 = 3\,\text{Ом}$, $R_2 = 8\,\text{Ом}$, $R_3 = 10\,\text{Ом}$, $\ele_1 = 4\,\text{В}$, $\ele_2 = 6\,\text{В}$, $\ele_3 = 8\,\text{В}$.
    Внутренним сопротивлением всех трёх ЭДС пренебречь.
    Ответы получите в виде несократимых дробей, а также определите приближённые значения.

    \begin{tikzpicture}[circuit ee IEC, thick]
        \foreach \contact/\x in {1/0, 2/3, 3/6}
        {
            \node [contact] (top contact \contact) at (\x, 0) {};
            \node [contact] (bottom contact \contact) at (\x, 4) {};
       }
        \draw  (bottom contact 1) -- (bottom contact 2) -- (bottom contact 3);
        \draw  (top contact 1) -- (top contact 2) -- (top contact 3);
        \draw  (bottom contact 1) to [resistor={near start, info=$R_1$}, battery={near end, info=$\ele_1$}] (top contact 1);
        \draw  (bottom contact 2) to [resistor={near start, info=$R_2$}, battery={near end, info=$\ele_2$}] (top contact 2);
        \draw  (bottom contact 3) to [resistor={near start, info=$R_3$}, battery={near end, info=$\ele_3$}] (top contact 3);
    \end{tikzpicture}
}
\answer{%
    План:
    \begin{itemize}
        \item отметим на рисунке произвольно направления токов (если получим отрицательный ответ, значит не угадали направление и только),
        \item выберем и обозначим на рисунке контуры (здесь всего 3, значит будет нужно $3-1=2$), для них запишем законы Кирхгофа,
        \item выберем и выделим на рисунке нетривиальные узлы (здесь всего 2, значит будет нужно $2-1=1$), для него запишем закон Кирхгофа,
        \item попытаемся решить получившуюся систему.
        В конкретном решении мы пытались первым делом найти $\eli_2$, но, возможно, в вашем варианте будет быстрее решать систему в другом порядке.
        Мы всё же проделаем всё в лоб, подробно и целиком.
    \end{itemize}


    \begin{tikzpicture}[circuit ee IEC, thick]
        \foreach \contact/\x in {1/0, 2/3, 3/6}
        {
            \node [contact] (top contact \contact) at (\x, 0) {};
            \node [contact] (bottom contact \contact) at (\x, 4) {};
       }
        \draw  (bottom contact 1) -- (bottom contact 2) -- (bottom contact 3);
        \draw  (top contact 1) -- (top contact 2) -- (top contact 3);
        \draw  (bottom contact 1) to [resistor={near start, info=$R_1$}, current direction'={midway, info=$\eli_1$}, battery={near end, info=$\ele_1$}] (top contact 1);
        \draw  (bottom contact 2) to [resistor={near start, info=$R_2$}, current direction'={midway, info=$\eli_2$}, battery={near end, info=$\ele_2$}] (top contact 2);
        \draw  (bottom contact 3) to [resistor={near start, info=$R_3$}, current direction'={midway, info=$\eli_3$}, battery={near end, info=$\ele_3$}] (top contact 3);
        \draw [-{Latex},color=red] (1.2, 2.5) arc [start angle = 135, end angle = -160, radius = 0.6];
        \draw [-{Latex},color=blue] (4.2, 2.5) arc [start angle = 135, end angle = -160, radius = 0.6];
        \node [contact,color=green!71!black] (bottomc) at (bottom contact 2) {};
    \end{tikzpicture}

    \begin{align*}
        &\begin{cases}
            {\color{red} \eli_1R_1 - \eli_2R_2 = \ele_1 - \ele_2}, \\
            {\color{blue} \eli_2R_2 - \eli_3R_3 = \ele_2 - \ele_3}, \\
            {\color{green!71!black} \eli_1 + \eli_2 + \eli_3 = 0};
        \end{cases}
        \qquad \implies \qquad
        \begin{cases}
            \eli_1 = \frac{\ele_1 - \ele_2 + \eli_2R_2}{R_1}, \\
            \eli_3 = \frac{\eli_2R_2 - \ele_2 + \ele_3}{R_3}, \\
            \eli_1 + \eli_2 + \eli_3 = 0, \\
        \end{cases} \implies \\
        \implies
            &\eli_2 + \frac{\ele_1 - \ele_2 + \eli_2R_2}{R_1} + \frac{\eli_2R_2 - \ele_2 + \ele_3}{R_3} = 0, \\
        &   \eli_2\cbr{1 + \frac{ R_2 }{ R_1 } + \frac{ R_2 }{ R_3 }} + \frac{\ele_1 - \ele_2}{ R_1 } + \frac{\ele_3 - \ele_2}{ R_3 } = 0, \\
        &   \eli_2 = \cfrac{\cfrac{\ele_2 - \ele_1}{ R_1 } + \cfrac{\ele_2 - \ele_3}{ R_3 }}{1 + \cfrac{ R_2 }{ R_1 } + \cfrac{ R_2 }{ R_3 }}
            = \cfrac{\cfrac{6\,\text{В} - 4\,\text{В}}{ 3\,\text{Ом} } + \cfrac{6\,\text{В} - 8\,\text{В}}{ 10\,\text{Ом} }}{1 + \cfrac{ 8\,\text{Ом} }{ 3\,\text{Ом} } + \cfrac{ 8\,\text{Ом} }{ 10\,\text{Ом} }}
            = \frac7{67}\units{А} \approx 0{,}10\,\text{А}, \\
        &   U_2 = \eli_2R_2 = \cfrac{\cfrac{\ele_2 - \ele_1}{ R_1 } + \cfrac{\ele_2 - \ele_3}{ R_3 }}{1 + \cfrac{ R_2 }{ R_1 } + \cfrac{ R_2 }{ R_3 }} \cdot R_2
            = \cfrac{\cfrac{6\,\text{В} - 4\,\text{В}}{ 3\,\text{Ом} } + \cfrac{6\,\text{В} - 8\,\text{В}}{ 10\,\text{Ом} }}{1 + \cfrac{ 8\,\text{Ом} }{ 3\,\text{Ом} } + \cfrac{ 8\,\text{Ом} }{ 10\,\text{Ом} }} \cdot 8\,\text{Ом}
            = \frac7{67}\units{А} \cdot 8\,\text{Ом} = \frac{56}{67}\units{В} \approx 0{,}84\,\text{В}.
    \end{align*}

    Одну пару силы тока и напряжения получили.
    Для некоторых вариантов это уже ответ, но не у всех.
    Для упрощения записи преобразуем (чтобы избавитсья от 4-этажной дроби) и подставим в уже полученные уравнения:

    \begin{align*}
    \eli_2
        &=
        \frac{\frac{\ele_2 - \ele_1}{ R_1 } + \frac{\ele_2 - \ele_3}{ R_3 }}{1 + \frac{ R_2 }{ R_1 } + \frac{ R_2 }{ R_3 }}
        =
        \frac{(\ele_2 - \ele_1)R_3 + (\ele_2 - \ele_3)R_1}{R_1R_3 + R_2R_3 + R_2R_1},
        \\
    \eli_1
        &=  \frac{\ele_1 - \ele_2 + \eli_2R_2}{R_1}
        =   \frac{\ele_1 - \ele_2 + \cfrac{(\ele_2 - \ele_1)R_3 + (\ele_2 - \ele_3)R_1}{R_1R_3 + R_2R_3 + R_2R_1} \cdot R_2}{R_1} = \\
        &=  \frac{
            \ele_1R_1R_3 + \ele_1R_2R_3 + \ele_1R_2R_1
            - \ele_2R_1R_3 - \ele_2R_2R_3 - \ele_2R_2R_1
            + \ele_2R_3R_2 - \ele_1R_3R_2 + \ele_2R_1R_2 - \ele_3R_1R_2
       }{R_1 \cdot \cbr{R_1R_3 + R_2R_3 + R_2R_1}}
        = \\ &=
        \frac{
            \ele_1\cbr{R_1R_3 + R_2R_3 + R_2R_1 - R_3R_2}
            + \ele_2\cbr{- R_1R_3 - R_2R_3 - R_2R_1 + R_3R_2 + R_1R_2}
            - \ele_3R_1R_2
       }{R_1 \cdot \cbr{R_1R_3 + R_2R_3 + R_2R_1}}
        = \\ &=
        \frac{
            \ele_1\cbr{R_1R_3 + R_2R_1}
            + \ele_2\cbr{- R_1R_3}
            - \ele_3R_1R_2
       }{R_1 \cdot \cbr{R_1R_3 + R_2R_3 + R_2R_1}}
        =
        \frac{
            \ele_1\cbr{R_3 + R_2} - \ele_2R_3 - \ele_3R_2
       }{R_1R_3 + R_2R_3 + R_2R_1}
        = \\ &=
        \frac{
            (\ele_1 - \ele_3)R_2 + (\ele_1 - \ele_2)R_3
       }{R_1R_3 + R_2R_3 + R_2R_1}
        =
        \frac{
            \cfrac{\ele_1 - \ele_3}{ R_3 } + \cfrac{\ele_1 - \ele_2}{ R_2 }
       }{\cfrac{ R_1 }{ R_2 } + 1 + \cfrac{ R_1 }{ R_3 }}
        =
        \frac{
            \cfrac{4\,\text{В} - 8\,\text{В}}{ 10\,\text{Ом} } + \cfrac{4\,\text{В} - 6\,\text{В}}{ 8\,\text{Ом} }
       }{\cfrac{ 3\,\text{Ом} }{ 8\,\text{Ом} } + 1 + \cfrac{ 3\,\text{Ом} }{ 10\,\text{Ом} }}
        = -\frac{26}{67}\units{А} \approx -0{,}3900\,\text{А}.
        \\
    U_1
        &=
        \eli_1R_1
        =
        \frac{
            \cfrac{\ele_1 - \ele_3}{ R_3 } + \cfrac{\ele_1 - \ele_2}{ R_2 }
       }{\cfrac{ R_1 }{ R_2 } + 1 + \cfrac{ R_1 }{ R_3 }} \cdot R_1
        =
        -\frac{26}{67}\units{А} \cdot 3\,\text{Ом} = -\frac{78}{67}\units{В} \approx -1{,}1600\,\text{В}.
    \end{align*}

    Если вы проделали все эти вычисления выше вместе со мной, то
    \begin{itemize}
        \item вы совершили ошибку, выбрав неверный путь решения:
        слишком длинное решение, очень легко ошибиться в индексах, дробях, знаках или потерять какой-то множитель,
        \item можно было выразить из исходной системы другие токи и получить сразу нажный вам,
        а не какой-то 2-й,
        \item можно было сэкономить: все три резистора и ЭДС соединены одинаково,
        поэтому ответ для 1-го резистора должен отличаться лишь перестановкой индексов (этот факт крайне полезен при проверке ответа, у нас всё сошлось),
        я специально подгонял выражение для $\eli_1$ к этому виду, вынося за скобки и преобразуя дробь,
        \item вы молодец, потому что не побоялись и получили верный ответ грамотным способом,
    \end{itemize}
    так что переходим к третьему резистору.
    Будет похоже, но кого это когда останавливало...

    \begin{align*}
    \eli_3
        &=  \frac{\eli_2R_2 - \ele_2 + \ele_3}{ R_3 }
        =
        \cfrac{
            \cfrac{
                (\ele_2 - \ele_1)R_3 + (\ele_2 - \ele_3)R_1
           }{
                R_1R_3 + R_2R_3 + R_2R_1
           } \cdot R_2 - \ele_2 + \ele_3}{ R_3 }
        = \\ &=
        \frac{
            \ele_2R_3R_2 - \ele_1R_3R_2 + \ele_2R_1R_2 - \ele_3R_1R_2
            - \ele_2R_1R_3 - \ele_2R_2R_3 - \ele_2R_2R_1
            + \ele_3R_1R_3 + \ele_3R_2R_3 + \ele_3R_2R_1
       }{\cbr{R_1R_3 + R_2R_3 + R_2R_1} \cdot R_3}
        = \\ &=
        \frac{
            - \ele_1R_3R_2 - \ele_2R_1R_3 + \ele_3R_1R_3 + \ele_3R_2R_3
       }{\cbr{R_1R_3 + R_2R_3 + R_2R_1} \cdot R_3}
        =
        \frac{
            - \ele_1R_2 - \ele_2R_1 + \ele_3R_1 + \ele_3R_2
       }{R_1R_3 + R_2R_3 + R_2R_1}
        = \\ &=
        \frac{
            R_1(\ele_3 - \ele_2) + R_2(\ele_3 - \ele_1)
       }{R_1R_3 + R_2R_3 + R_2R_1}
        =
        \frac{
            \cfrac{\ele_3 - \ele_2}{ R_2 } + \cfrac{\ele_3 - \ele_1}{ R_1 }
       }{\cfrac{ R_3 }{ R_2 } + \cfrac{ R_3 }{ R_1 } + 1}
        =
        \frac{
            \cfrac{8\,\text{В} - 6\,\text{В}}{ 8\,\text{Ом} } + \cfrac{8\,\text{В} - 4\,\text{В}}{ 3\,\text{Ом} }
       }{\cfrac{ 10\,\text{Ом} }{ 8\,\text{Ом} } + \cfrac{ 10\,\text{Ом} }{ 3\,\text{Ом} } + 1}
        = \frac{19}{67}\units{А} \approx 0{,}28\,\text{А}.
        \\
    U_3
        &=
        \eli_3R_3
        =
        \frac{
            \cfrac{\ele_3 - \ele_2}{ R_2 } + \cfrac{\ele_3 - \ele_1}{ R_1 }
       }{\cfrac{ R_3 }{ R_2 } + \cfrac{ R_3 }{ R_1 } + 1} \cdot R_3
        =
        \frac{19}{67}\units{А} \cdot 10\,\text{Ом} = \frac{190}{67}\units{В} \approx 2{,}84\,\text{В}.
    \end{align*}

    Положительные ответы говорят, что мы угадали на рисунке направление тока (тут нет нашей заслуги, повезло),
    отрицательные — что не угадали (и в этом нет ошибки), и ток течёт в противоположную сторону.
    Напомним, что направление тока — это направление движения положительных зарядов,
    а в металлах носители заряда — электроны, которые заряжены отрицательно.
}

\variantsplitter

\addpersonalvariant{София Журавлёва}

\tasknumber{1}%
\task{%
    Определите ток $\eli_1$, протекающий через резистор $R_1$ (см.
    рис.),
    направление этого тока и разность потенциалов $U_1$ на этом резисторе,
    если $R_1 = 4\,\text{Ом}$, $R_2 = 6\,\text{Ом}$, $R_3 = 15\,\text{Ом}$, $\ele_1 = 5\,\text{В}$, $\ele_2 = 3\,\text{В}$, $\ele_3 = 2\,\text{В}$.
    Внутренним сопротивлением всех трёх ЭДС пренебречь.
    Ответы получите в виде несократимых дробей, а также определите приближённые значения.

    \begin{tikzpicture}[circuit ee IEC, thick]
        \foreach \contact/\x in {1/0, 2/3, 3/6}
        {
            \node [contact] (top contact \contact) at (\x, 0) {};
            \node [contact] (bottom contact \contact) at (\x, 4) {};
       }
        \draw  (bottom contact 1) -- (bottom contact 2) -- (bottom contact 3);
        \draw  (top contact 1) -- (top contact 2) -- (top contact 3);
        \draw  (bottom contact 1) to [resistor={near start, info=$R_1$}, battery={near end, info=$\ele_1$}] (top contact 1);
        \draw  (bottom contact 2) to [resistor={near start, info=$R_2$}, battery={near end, info=$\ele_2$}] (top contact 2);
        \draw  (bottom contact 3) to [resistor={near start, info=$R_3$}, battery={near end, info=$\ele_3$}] (top contact 3);
    \end{tikzpicture}
}
\answer{%
    План:
    \begin{itemize}
        \item отметим на рисунке произвольно направления токов (если получим отрицательный ответ, значит не угадали направление и только),
        \item выберем и обозначим на рисунке контуры (здесь всего 3, значит будет нужно $3-1=2$), для них запишем законы Кирхгофа,
        \item выберем и выделим на рисунке нетривиальные узлы (здесь всего 2, значит будет нужно $2-1=1$), для него запишем закон Кирхгофа,
        \item попытаемся решить получившуюся систему.
        В конкретном решении мы пытались первым делом найти $\eli_2$, но, возможно, в вашем варианте будет быстрее решать систему в другом порядке.
        Мы всё же проделаем всё в лоб, подробно и целиком.
    \end{itemize}


    \begin{tikzpicture}[circuit ee IEC, thick]
        \foreach \contact/\x in {1/0, 2/3, 3/6}
        {
            \node [contact] (top contact \contact) at (\x, 0) {};
            \node [contact] (bottom contact \contact) at (\x, 4) {};
       }
        \draw  (bottom contact 1) -- (bottom contact 2) -- (bottom contact 3);
        \draw  (top contact 1) -- (top contact 2) -- (top contact 3);
        \draw  (bottom contact 1) to [resistor={near start, info=$R_1$}, current direction'={midway, info=$\eli_1$}, battery={near end, info=$\ele_1$}] (top contact 1);
        \draw  (bottom contact 2) to [resistor={near start, info=$R_2$}, current direction'={midway, info=$\eli_2$}, battery={near end, info=$\ele_2$}] (top contact 2);
        \draw  (bottom contact 3) to [resistor={near start, info=$R_3$}, current direction'={midway, info=$\eli_3$}, battery={near end, info=$\ele_3$}] (top contact 3);
        \draw [-{Latex},color=red] (1.2, 2.5) arc [start angle = 135, end angle = -160, radius = 0.6];
        \draw [-{Latex},color=blue] (4.2, 2.5) arc [start angle = 135, end angle = -160, radius = 0.6];
        \node [contact,color=green!71!black] (bottomc) at (bottom contact 2) {};
    \end{tikzpicture}

    \begin{align*}
        &\begin{cases}
            {\color{red} \eli_1R_1 - \eli_2R_2 = \ele_1 - \ele_2}, \\
            {\color{blue} \eli_2R_2 - \eli_3R_3 = \ele_2 - \ele_3}, \\
            {\color{green!71!black} \eli_1 + \eli_2 + \eli_3 = 0};
        \end{cases}
        \qquad \implies \qquad
        \begin{cases}
            \eli_1 = \frac{\ele_1 - \ele_2 + \eli_2R_2}{R_1}, \\
            \eli_3 = \frac{\eli_2R_2 - \ele_2 + \ele_3}{R_3}, \\
            \eli_1 + \eli_2 + \eli_3 = 0, \\
        \end{cases} \implies \\
        \implies
            &\eli_2 + \frac{\ele_1 - \ele_2 + \eli_2R_2}{R_1} + \frac{\eli_2R_2 - \ele_2 + \ele_3}{R_3} = 0, \\
        &   \eli_2\cbr{1 + \frac{ R_2 }{ R_1 } + \frac{ R_2 }{ R_3 }} + \frac{\ele_1 - \ele_2}{ R_1 } + \frac{\ele_3 - \ele_2}{ R_3 } = 0, \\
        &   \eli_2 = \cfrac{\cfrac{\ele_2 - \ele_1}{ R_1 } + \cfrac{\ele_2 - \ele_3}{ R_3 }}{1 + \cfrac{ R_2 }{ R_1 } + \cfrac{ R_2 }{ R_3 }}
            = \cfrac{\cfrac{3\,\text{В} - 5\,\text{В}}{ 4\,\text{Ом} } + \cfrac{3\,\text{В} - 2\,\text{В}}{ 15\,\text{Ом} }}{1 + \cfrac{ 6\,\text{Ом} }{ 4\,\text{Ом} } + \cfrac{ 6\,\text{Ом} }{ 15\,\text{Ом} }}
            = -\frac{13}{87}\units{А} \approx -0{,}15000\,\text{А}, \\
        &   U_2 = \eli_2R_2 = \cfrac{\cfrac{\ele_2 - \ele_1}{ R_1 } + \cfrac{\ele_2 - \ele_3}{ R_3 }}{1 + \cfrac{ R_2 }{ R_1 } + \cfrac{ R_2 }{ R_3 }} \cdot R_2
            = \cfrac{\cfrac{3\,\text{В} - 5\,\text{В}}{ 4\,\text{Ом} } + \cfrac{3\,\text{В} - 2\,\text{В}}{ 15\,\text{Ом} }}{1 + \cfrac{ 6\,\text{Ом} }{ 4\,\text{Ом} } + \cfrac{ 6\,\text{Ом} }{ 15\,\text{Ом} }} \cdot 6\,\text{Ом}
            = -\frac{13}{87}\units{А} \cdot 6\,\text{Ом} = -\frac{26}{29}\units{В} \approx -0{,}9000\,\text{В}.
    \end{align*}

    Одну пару силы тока и напряжения получили.
    Для некоторых вариантов это уже ответ, но не у всех.
    Для упрощения записи преобразуем (чтобы избавитсья от 4-этажной дроби) и подставим в уже полученные уравнения:

    \begin{align*}
    \eli_2
        &=
        \frac{\frac{\ele_2 - \ele_1}{ R_1 } + \frac{\ele_2 - \ele_3}{ R_3 }}{1 + \frac{ R_2 }{ R_1 } + \frac{ R_2 }{ R_3 }}
        =
        \frac{(\ele_2 - \ele_1)R_3 + (\ele_2 - \ele_3)R_1}{R_1R_3 + R_2R_3 + R_2R_1},
        \\
    \eli_1
        &=  \frac{\ele_1 - \ele_2 + \eli_2R_2}{R_1}
        =   \frac{\ele_1 - \ele_2 + \cfrac{(\ele_2 - \ele_1)R_3 + (\ele_2 - \ele_3)R_1}{R_1R_3 + R_2R_3 + R_2R_1} \cdot R_2}{R_1} = \\
        &=  \frac{
            \ele_1R_1R_3 + \ele_1R_2R_3 + \ele_1R_2R_1
            - \ele_2R_1R_3 - \ele_2R_2R_3 - \ele_2R_2R_1
            + \ele_2R_3R_2 - \ele_1R_3R_2 + \ele_2R_1R_2 - \ele_3R_1R_2
       }{R_1 \cdot \cbr{R_1R_3 + R_2R_3 + R_2R_1}}
        = \\ &=
        \frac{
            \ele_1\cbr{R_1R_3 + R_2R_3 + R_2R_1 - R_3R_2}
            + \ele_2\cbr{- R_1R_3 - R_2R_3 - R_2R_1 + R_3R_2 + R_1R_2}
            - \ele_3R_1R_2
       }{R_1 \cdot \cbr{R_1R_3 + R_2R_3 + R_2R_1}}
        = \\ &=
        \frac{
            \ele_1\cbr{R_1R_3 + R_2R_1}
            + \ele_2\cbr{- R_1R_3}
            - \ele_3R_1R_2
       }{R_1 \cdot \cbr{R_1R_3 + R_2R_3 + R_2R_1}}
        =
        \frac{
            \ele_1\cbr{R_3 + R_2} - \ele_2R_3 - \ele_3R_2
       }{R_1R_3 + R_2R_3 + R_2R_1}
        = \\ &=
        \frac{
            (\ele_1 - \ele_3)R_2 + (\ele_1 - \ele_2)R_3
       }{R_1R_3 + R_2R_3 + R_2R_1}
        =
        \frac{
            \cfrac{\ele_1 - \ele_3}{ R_3 } + \cfrac{\ele_1 - \ele_2}{ R_2 }
       }{\cfrac{ R_1 }{ R_2 } + 1 + \cfrac{ R_1 }{ R_3 }}
        =
        \frac{
            \cfrac{5\,\text{В} - 2\,\text{В}}{ 15\,\text{Ом} } + \cfrac{5\,\text{В} - 3\,\text{В}}{ 6\,\text{Ом} }
       }{\cfrac{ 4\,\text{Ом} }{ 6\,\text{Ом} } + 1 + \cfrac{ 4\,\text{Ом} }{ 15\,\text{Ом} }}
        = \frac8{29}\units{А} \approx 0{,}28\,\text{А}.
        \\
    U_1
        &=
        \eli_1R_1
        =
        \frac{
            \cfrac{\ele_1 - \ele_3}{ R_3 } + \cfrac{\ele_1 - \ele_2}{ R_2 }
       }{\cfrac{ R_1 }{ R_2 } + 1 + \cfrac{ R_1 }{ R_3 }} \cdot R_1
        =
        \frac8{29}\units{А} \cdot 4\,\text{Ом} = \frac{32}{29}\units{В} \approx 1{,}10\,\text{В}.
    \end{align*}

    Если вы проделали все эти вычисления выше вместе со мной, то
    \begin{itemize}
        \item вы совершили ошибку, выбрав неверный путь решения:
        слишком длинное решение, очень легко ошибиться в индексах, дробях, знаках или потерять какой-то множитель,
        \item можно было выразить из исходной системы другие токи и получить сразу нажный вам,
        а не какой-то 2-й,
        \item можно было сэкономить: все три резистора и ЭДС соединены одинаково,
        поэтому ответ для 1-го резистора должен отличаться лишь перестановкой индексов (этот факт крайне полезен при проверке ответа, у нас всё сошлось),
        я специально подгонял выражение для $\eli_1$ к этому виду, вынося за скобки и преобразуя дробь,
        \item вы молодец, потому что не побоялись и получили верный ответ грамотным способом,
    \end{itemize}
    так что переходим к третьему резистору.
    Будет похоже, но кого это когда останавливало...

    \begin{align*}
    \eli_3
        &=  \frac{\eli_2R_2 - \ele_2 + \ele_3}{ R_3 }
        =
        \cfrac{
            \cfrac{
                (\ele_2 - \ele_1)R_3 + (\ele_2 - \ele_3)R_1
           }{
                R_1R_3 + R_2R_3 + R_2R_1
           } \cdot R_2 - \ele_2 + \ele_3}{ R_3 }
        = \\ &=
        \frac{
            \ele_2R_3R_2 - \ele_1R_3R_2 + \ele_2R_1R_2 - \ele_3R_1R_2
            - \ele_2R_1R_3 - \ele_2R_2R_3 - \ele_2R_2R_1
            + \ele_3R_1R_3 + \ele_3R_2R_3 + \ele_3R_2R_1
       }{\cbr{R_1R_3 + R_2R_3 + R_2R_1} \cdot R_3}
        = \\ &=
        \frac{
            - \ele_1R_3R_2 - \ele_2R_1R_3 + \ele_3R_1R_3 + \ele_3R_2R_3
       }{\cbr{R_1R_3 + R_2R_3 + R_2R_1} \cdot R_3}
        =
        \frac{
            - \ele_1R_2 - \ele_2R_1 + \ele_3R_1 + \ele_3R_2
       }{R_1R_3 + R_2R_3 + R_2R_1}
        = \\ &=
        \frac{
            R_1(\ele_3 - \ele_2) + R_2(\ele_3 - \ele_1)
       }{R_1R_3 + R_2R_3 + R_2R_1}
        =
        \frac{
            \cfrac{\ele_3 - \ele_2}{ R_2 } + \cfrac{\ele_3 - \ele_1}{ R_1 }
       }{\cfrac{ R_3 }{ R_2 } + \cfrac{ R_3 }{ R_1 } + 1}
        =
        \frac{
            \cfrac{2\,\text{В} - 3\,\text{В}}{ 6\,\text{Ом} } + \cfrac{2\,\text{В} - 5\,\text{В}}{ 4\,\text{Ом} }
       }{\cfrac{ 15\,\text{Ом} }{ 6\,\text{Ом} } + \cfrac{ 15\,\text{Ом} }{ 4\,\text{Ом} } + 1}
        = -\frac{11}{87}\units{А} \approx -0{,}13000\,\text{А}.
        \\
    U_3
        &=
        \eli_3R_3
        =
        \frac{
            \cfrac{\ele_3 - \ele_2}{ R_2 } + \cfrac{\ele_3 - \ele_1}{ R_1 }
       }{\cfrac{ R_3 }{ R_2 } + \cfrac{ R_3 }{ R_1 } + 1} \cdot R_3
        =
        -\frac{11}{87}\units{А} \cdot 15\,\text{Ом} = -\frac{55}{29}\units{В} \approx -1{,}9000\,\text{В}.
    \end{align*}

    Положительные ответы говорят, что мы угадали на рисунке направление тока (тут нет нашей заслуги, повезло),
    отрицательные — что не угадали (и в этом нет ошибки), и ток течёт в противоположную сторону.
    Напомним, что направление тока — это направление движения положительных зарядов,
    а в металлах носители заряда — электроны, которые заряжены отрицательно.
}

\variantsplitter

\addpersonalvariant{Константин Козлов}

\tasknumber{1}%
\task{%
    Определите ток $\eli_2$, протекающий через резистор $R_2$ (см.
    рис.),
    направление этого тока и разность потенциалов $U_2$ на этом резисторе,
    если $R_1 = 2\,\text{Ом}$, $R_2 = 5\,\text{Ом}$, $R_3 = 12\,\text{Ом}$, $\ele_1 = 4\,\text{В}$, $\ele_2 = 6\,\text{В}$, $\ele_3 = 8\,\text{В}$.
    Внутренним сопротивлением всех трёх ЭДС пренебречь.
    Ответы получите в виде несократимых дробей, а также определите приближённые значения.

    \begin{tikzpicture}[circuit ee IEC, thick]
        \foreach \contact/\x in {1/0, 2/3, 3/6}
        {
            \node [contact] (top contact \contact) at (\x, 0) {};
            \node [contact] (bottom contact \contact) at (\x, 4) {};
       }
        \draw  (bottom contact 1) -- (bottom contact 2) -- (bottom contact 3);
        \draw  (top contact 1) -- (top contact 2) -- (top contact 3);
        \draw  (bottom contact 1) to [resistor={near start, info=$R_1$}, battery={near end, info=$\ele_1$}] (top contact 1);
        \draw  (bottom contact 2) to [resistor={near start, info=$R_2$}, battery={near end, info=$\ele_2$}] (top contact 2);
        \draw  (bottom contact 3) to [resistor={near start, info=$R_3$}, battery={near end, info=$\ele_3$}] (top contact 3);
    \end{tikzpicture}
}
\answer{%
    План:
    \begin{itemize}
        \item отметим на рисунке произвольно направления токов (если получим отрицательный ответ, значит не угадали направление и только),
        \item выберем и обозначим на рисунке контуры (здесь всего 3, значит будет нужно $3-1=2$), для них запишем законы Кирхгофа,
        \item выберем и выделим на рисунке нетривиальные узлы (здесь всего 2, значит будет нужно $2-1=1$), для него запишем закон Кирхгофа,
        \item попытаемся решить получившуюся систему.
        В конкретном решении мы пытались первым делом найти $\eli_2$, но, возможно, в вашем варианте будет быстрее решать систему в другом порядке.
        Мы всё же проделаем всё в лоб, подробно и целиком.
    \end{itemize}


    \begin{tikzpicture}[circuit ee IEC, thick]
        \foreach \contact/\x in {1/0, 2/3, 3/6}
        {
            \node [contact] (top contact \contact) at (\x, 0) {};
            \node [contact] (bottom contact \contact) at (\x, 4) {};
       }
        \draw  (bottom contact 1) -- (bottom contact 2) -- (bottom contact 3);
        \draw  (top contact 1) -- (top contact 2) -- (top contact 3);
        \draw  (bottom contact 1) to [resistor={near start, info=$R_1$}, current direction'={midway, info=$\eli_1$}, battery={near end, info=$\ele_1$}] (top contact 1);
        \draw  (bottom contact 2) to [resistor={near start, info=$R_2$}, current direction'={midway, info=$\eli_2$}, battery={near end, info=$\ele_2$}] (top contact 2);
        \draw  (bottom contact 3) to [resistor={near start, info=$R_3$}, current direction'={midway, info=$\eli_3$}, battery={near end, info=$\ele_3$}] (top contact 3);
        \draw [-{Latex},color=red] (1.2, 2.5) arc [start angle = 135, end angle = -160, radius = 0.6];
        \draw [-{Latex},color=blue] (4.2, 2.5) arc [start angle = 135, end angle = -160, radius = 0.6];
        \node [contact,color=green!71!black] (bottomc) at (bottom contact 2) {};
    \end{tikzpicture}

    \begin{align*}
        &\begin{cases}
            {\color{red} \eli_1R_1 - \eli_2R_2 = \ele_1 - \ele_2}, \\
            {\color{blue} \eli_2R_2 - \eli_3R_3 = \ele_2 - \ele_3}, \\
            {\color{green!71!black} \eli_1 + \eli_2 + \eli_3 = 0};
        \end{cases}
        \qquad \implies \qquad
        \begin{cases}
            \eli_1 = \frac{\ele_1 - \ele_2 + \eli_2R_2}{R_1}, \\
            \eli_3 = \frac{\eli_2R_2 - \ele_2 + \ele_3}{R_3}, \\
            \eli_1 + \eli_2 + \eli_3 = 0, \\
        \end{cases} \implies \\
        \implies
            &\eli_2 + \frac{\ele_1 - \ele_2 + \eli_2R_2}{R_1} + \frac{\eli_2R_2 - \ele_2 + \ele_3}{R_3} = 0, \\
        &   \eli_2\cbr{1 + \frac{ R_2 }{ R_1 } + \frac{ R_2 }{ R_3 }} + \frac{\ele_1 - \ele_2}{ R_1 } + \frac{\ele_3 - \ele_2}{ R_3 } = 0, \\
        &   \eli_2 = \cfrac{\cfrac{\ele_2 - \ele_1}{ R_1 } + \cfrac{\ele_2 - \ele_3}{ R_3 }}{1 + \cfrac{ R_2 }{ R_1 } + \cfrac{ R_2 }{ R_3 }}
            = \cfrac{\cfrac{6\,\text{В} - 4\,\text{В}}{ 2\,\text{Ом} } + \cfrac{6\,\text{В} - 8\,\text{В}}{ 12\,\text{Ом} }}{1 + \cfrac{ 5\,\text{Ом} }{ 2\,\text{Ом} } + \cfrac{ 5\,\text{Ом} }{ 12\,\text{Ом} }}
            = \frac{10}{47}\units{А} \approx 0{,}21\,\text{А}, \\
        &   U_2 = \eli_2R_2 = \cfrac{\cfrac{\ele_2 - \ele_1}{ R_1 } + \cfrac{\ele_2 - \ele_3}{ R_3 }}{1 + \cfrac{ R_2 }{ R_1 } + \cfrac{ R_2 }{ R_3 }} \cdot R_2
            = \cfrac{\cfrac{6\,\text{В} - 4\,\text{В}}{ 2\,\text{Ом} } + \cfrac{6\,\text{В} - 8\,\text{В}}{ 12\,\text{Ом} }}{1 + \cfrac{ 5\,\text{Ом} }{ 2\,\text{Ом} } + \cfrac{ 5\,\text{Ом} }{ 12\,\text{Ом} }} \cdot 5\,\text{Ом}
            = \frac{10}{47}\units{А} \cdot 5\,\text{Ом} = \frac{50}{47}\units{В} \approx 1{,}06\,\text{В}.
    \end{align*}

    Одну пару силы тока и напряжения получили.
    Для некоторых вариантов это уже ответ, но не у всех.
    Для упрощения записи преобразуем (чтобы избавитсья от 4-этажной дроби) и подставим в уже полученные уравнения:

    \begin{align*}
    \eli_2
        &=
        \frac{\frac{\ele_2 - \ele_1}{ R_1 } + \frac{\ele_2 - \ele_3}{ R_3 }}{1 + \frac{ R_2 }{ R_1 } + \frac{ R_2 }{ R_3 }}
        =
        \frac{(\ele_2 - \ele_1)R_3 + (\ele_2 - \ele_3)R_1}{R_1R_3 + R_2R_3 + R_2R_1},
        \\
    \eli_1
        &=  \frac{\ele_1 - \ele_2 + \eli_2R_2}{R_1}
        =   \frac{\ele_1 - \ele_2 + \cfrac{(\ele_2 - \ele_1)R_3 + (\ele_2 - \ele_3)R_1}{R_1R_3 + R_2R_3 + R_2R_1} \cdot R_2}{R_1} = \\
        &=  \frac{
            \ele_1R_1R_3 + \ele_1R_2R_3 + \ele_1R_2R_1
            - \ele_2R_1R_3 - \ele_2R_2R_3 - \ele_2R_2R_1
            + \ele_2R_3R_2 - \ele_1R_3R_2 + \ele_2R_1R_2 - \ele_3R_1R_2
       }{R_1 \cdot \cbr{R_1R_3 + R_2R_3 + R_2R_1}}
        = \\ &=
        \frac{
            \ele_1\cbr{R_1R_3 + R_2R_3 + R_2R_1 - R_3R_2}
            + \ele_2\cbr{- R_1R_3 - R_2R_3 - R_2R_1 + R_3R_2 + R_1R_2}
            - \ele_3R_1R_2
       }{R_1 \cdot \cbr{R_1R_3 + R_2R_3 + R_2R_1}}
        = \\ &=
        \frac{
            \ele_1\cbr{R_1R_3 + R_2R_1}
            + \ele_2\cbr{- R_1R_3}
            - \ele_3R_1R_2
       }{R_1 \cdot \cbr{R_1R_3 + R_2R_3 + R_2R_1}}
        =
        \frac{
            \ele_1\cbr{R_3 + R_2} - \ele_2R_3 - \ele_3R_2
       }{R_1R_3 + R_2R_3 + R_2R_1}
        = \\ &=
        \frac{
            (\ele_1 - \ele_3)R_2 + (\ele_1 - \ele_2)R_3
       }{R_1R_3 + R_2R_3 + R_2R_1}
        =
        \frac{
            \cfrac{\ele_1 - \ele_3}{ R_3 } + \cfrac{\ele_1 - \ele_2}{ R_2 }
       }{\cfrac{ R_1 }{ R_2 } + 1 + \cfrac{ R_1 }{ R_3 }}
        =
        \frac{
            \cfrac{4\,\text{В} - 8\,\text{В}}{ 12\,\text{Ом} } + \cfrac{4\,\text{В} - 6\,\text{В}}{ 5\,\text{Ом} }
       }{\cfrac{ 2\,\text{Ом} }{ 5\,\text{Ом} } + 1 + \cfrac{ 2\,\text{Ом} }{ 12\,\text{Ом} }}
        = -\frac{22}{47}\units{А} \approx -0{,}4700\,\text{А}.
        \\
    U_1
        &=
        \eli_1R_1
        =
        \frac{
            \cfrac{\ele_1 - \ele_3}{ R_3 } + \cfrac{\ele_1 - \ele_2}{ R_2 }
       }{\cfrac{ R_1 }{ R_2 } + 1 + \cfrac{ R_1 }{ R_3 }} \cdot R_1
        =
        -\frac{22}{47}\units{А} \cdot 2\,\text{Ом} = -\frac{44}{47}\units{В} \approx -0{,}9400\,\text{В}.
    \end{align*}

    Если вы проделали все эти вычисления выше вместе со мной, то
    \begin{itemize}
        \item вы совершили ошибку, выбрав неверный путь решения:
        слишком длинное решение, очень легко ошибиться в индексах, дробях, знаках или потерять какой-то множитель,
        \item можно было выразить из исходной системы другие токи и получить сразу нажный вам,
        а не какой-то 2-й,
        \item можно было сэкономить: все три резистора и ЭДС соединены одинаково,
        поэтому ответ для 1-го резистора должен отличаться лишь перестановкой индексов (этот факт крайне полезен при проверке ответа, у нас всё сошлось),
        я специально подгонял выражение для $\eli_1$ к этому виду, вынося за скобки и преобразуя дробь,
        \item вы молодец, потому что не побоялись и получили верный ответ грамотным способом,
    \end{itemize}
    так что переходим к третьему резистору.
    Будет похоже, но кого это когда останавливало...

    \begin{align*}
    \eli_3
        &=  \frac{\eli_2R_2 - \ele_2 + \ele_3}{ R_3 }
        =
        \cfrac{
            \cfrac{
                (\ele_2 - \ele_1)R_3 + (\ele_2 - \ele_3)R_1
           }{
                R_1R_3 + R_2R_3 + R_2R_1
           } \cdot R_2 - \ele_2 + \ele_3}{ R_3 }
        = \\ &=
        \frac{
            \ele_2R_3R_2 - \ele_1R_3R_2 + \ele_2R_1R_2 - \ele_3R_1R_2
            - \ele_2R_1R_3 - \ele_2R_2R_3 - \ele_2R_2R_1
            + \ele_3R_1R_3 + \ele_3R_2R_3 + \ele_3R_2R_1
       }{\cbr{R_1R_3 + R_2R_3 + R_2R_1} \cdot R_3}
        = \\ &=
        \frac{
            - \ele_1R_3R_2 - \ele_2R_1R_3 + \ele_3R_1R_3 + \ele_3R_2R_3
       }{\cbr{R_1R_3 + R_2R_3 + R_2R_1} \cdot R_3}
        =
        \frac{
            - \ele_1R_2 - \ele_2R_1 + \ele_3R_1 + \ele_3R_2
       }{R_1R_3 + R_2R_3 + R_2R_1}
        = \\ &=
        \frac{
            R_1(\ele_3 - \ele_2) + R_2(\ele_3 - \ele_1)
       }{R_1R_3 + R_2R_3 + R_2R_1}
        =
        \frac{
            \cfrac{\ele_3 - \ele_2}{ R_2 } + \cfrac{\ele_3 - \ele_1}{ R_1 }
       }{\cfrac{ R_3 }{ R_2 } + \cfrac{ R_3 }{ R_1 } + 1}
        =
        \frac{
            \cfrac{8\,\text{В} - 6\,\text{В}}{ 5\,\text{Ом} } + \cfrac{8\,\text{В} - 4\,\text{В}}{ 2\,\text{Ом} }
       }{\cfrac{ 12\,\text{Ом} }{ 5\,\text{Ом} } + \cfrac{ 12\,\text{Ом} }{ 2\,\text{Ом} } + 1}
        = \frac{12}{47}\units{А} \approx 0{,}26\,\text{А}.
        \\
    U_3
        &=
        \eli_3R_3
        =
        \frac{
            \cfrac{\ele_3 - \ele_2}{ R_2 } + \cfrac{\ele_3 - \ele_1}{ R_1 }
       }{\cfrac{ R_3 }{ R_2 } + \cfrac{ R_3 }{ R_1 } + 1} \cdot R_3
        =
        \frac{12}{47}\units{А} \cdot 12\,\text{Ом} = \frac{144}{47}\units{В} \approx 3{,}06\,\text{В}.
    \end{align*}

    Положительные ответы говорят, что мы угадали на рисунке направление тока (тут нет нашей заслуги, повезло),
    отрицательные — что не угадали (и в этом нет ошибки), и ток течёт в противоположную сторону.
    Напомним, что направление тока — это направление движения положительных зарядов,
    а в металлах носители заряда — электроны, которые заряжены отрицательно.
}

\variantsplitter

\addpersonalvariant{Наталья Кравченко}

\tasknumber{1}%
\task{%
    Определите ток $\eli_1$, протекающий через резистор $R_1$ (см.
    рис.),
    направление этого тока и разность потенциалов $U_1$ на этом резисторе,
    если $R_1 = 2\,\text{Ом}$, $R_2 = 6\,\text{Ом}$, $R_3 = 10\,\text{Ом}$, $\ele_1 = 4\,\text{В}$, $\ele_2 = 3\,\text{В}$, $\ele_3 = 8\,\text{В}$.
    Внутренним сопротивлением всех трёх ЭДС пренебречь.
    Ответы получите в виде несократимых дробей, а также определите приближённые значения.

    \begin{tikzpicture}[circuit ee IEC, thick]
        \foreach \contact/\x in {1/0, 2/3, 3/6}
        {
            \node [contact] (top contact \contact) at (\x, 0) {};
            \node [contact] (bottom contact \contact) at (\x, 4) {};
       }
        \draw  (bottom contact 1) -- (bottom contact 2) -- (bottom contact 3);
        \draw  (top contact 1) -- (top contact 2) -- (top contact 3);
        \draw  (bottom contact 1) to [resistor={near start, info=$R_1$}, battery={near end, info=$\ele_1$}] (top contact 1);
        \draw  (bottom contact 2) to [resistor={near start, info=$R_2$}, battery={near end, info=$\ele_2$}] (top contact 2);
        \draw  (bottom contact 3) to [resistor={near start, info=$R_3$}, battery={near end, info=$\ele_3$}] (top contact 3);
    \end{tikzpicture}
}
\answer{%
    План:
    \begin{itemize}
        \item отметим на рисунке произвольно направления токов (если получим отрицательный ответ, значит не угадали направление и только),
        \item выберем и обозначим на рисунке контуры (здесь всего 3, значит будет нужно $3-1=2$), для них запишем законы Кирхгофа,
        \item выберем и выделим на рисунке нетривиальные узлы (здесь всего 2, значит будет нужно $2-1=1$), для него запишем закон Кирхгофа,
        \item попытаемся решить получившуюся систему.
        В конкретном решении мы пытались первым делом найти $\eli_2$, но, возможно, в вашем варианте будет быстрее решать систему в другом порядке.
        Мы всё же проделаем всё в лоб, подробно и целиком.
    \end{itemize}


    \begin{tikzpicture}[circuit ee IEC, thick]
        \foreach \contact/\x in {1/0, 2/3, 3/6}
        {
            \node [contact] (top contact \contact) at (\x, 0) {};
            \node [contact] (bottom contact \contact) at (\x, 4) {};
       }
        \draw  (bottom contact 1) -- (bottom contact 2) -- (bottom contact 3);
        \draw  (top contact 1) -- (top contact 2) -- (top contact 3);
        \draw  (bottom contact 1) to [resistor={near start, info=$R_1$}, current direction'={midway, info=$\eli_1$}, battery={near end, info=$\ele_1$}] (top contact 1);
        \draw  (bottom contact 2) to [resistor={near start, info=$R_2$}, current direction'={midway, info=$\eli_2$}, battery={near end, info=$\ele_2$}] (top contact 2);
        \draw  (bottom contact 3) to [resistor={near start, info=$R_3$}, current direction'={midway, info=$\eli_3$}, battery={near end, info=$\ele_3$}] (top contact 3);
        \draw [-{Latex},color=red] (1.2, 2.5) arc [start angle = 135, end angle = -160, radius = 0.6];
        \draw [-{Latex},color=blue] (4.2, 2.5) arc [start angle = 135, end angle = -160, radius = 0.6];
        \node [contact,color=green!71!black] (bottomc) at (bottom contact 2) {};
    \end{tikzpicture}

    \begin{align*}
        &\begin{cases}
            {\color{red} \eli_1R_1 - \eli_2R_2 = \ele_1 - \ele_2}, \\
            {\color{blue} \eli_2R_2 - \eli_3R_3 = \ele_2 - \ele_3}, \\
            {\color{green!71!black} \eli_1 + \eli_2 + \eli_3 = 0};
        \end{cases}
        \qquad \implies \qquad
        \begin{cases}
            \eli_1 = \frac{\ele_1 - \ele_2 + \eli_2R_2}{R_1}, \\
            \eli_3 = \frac{\eli_2R_2 - \ele_2 + \ele_3}{R_3}, \\
            \eli_1 + \eli_2 + \eli_3 = 0, \\
        \end{cases} \implies \\
        \implies
            &\eli_2 + \frac{\ele_1 - \ele_2 + \eli_2R_2}{R_1} + \frac{\eli_2R_2 - \ele_2 + \ele_3}{R_3} = 0, \\
        &   \eli_2\cbr{1 + \frac{ R_2 }{ R_1 } + \frac{ R_2 }{ R_3 }} + \frac{\ele_1 - \ele_2}{ R_1 } + \frac{\ele_3 - \ele_2}{ R_3 } = 0, \\
        &   \eli_2 = \cfrac{\cfrac{\ele_2 - \ele_1}{ R_1 } + \cfrac{\ele_2 - \ele_3}{ R_3 }}{1 + \cfrac{ R_2 }{ R_1 } + \cfrac{ R_2 }{ R_3 }}
            = \cfrac{\cfrac{3\,\text{В} - 4\,\text{В}}{ 2\,\text{Ом} } + \cfrac{3\,\text{В} - 8\,\text{В}}{ 10\,\text{Ом} }}{1 + \cfrac{ 6\,\text{Ом} }{ 2\,\text{Ом} } + \cfrac{ 6\,\text{Ом} }{ 10\,\text{Ом} }}
            = -\frac5{23}\units{А} \approx -0{,}2200\,\text{А}, \\
        &   U_2 = \eli_2R_2 = \cfrac{\cfrac{\ele_2 - \ele_1}{ R_1 } + \cfrac{\ele_2 - \ele_3}{ R_3 }}{1 + \cfrac{ R_2 }{ R_1 } + \cfrac{ R_2 }{ R_3 }} \cdot R_2
            = \cfrac{\cfrac{3\,\text{В} - 4\,\text{В}}{ 2\,\text{Ом} } + \cfrac{3\,\text{В} - 8\,\text{В}}{ 10\,\text{Ом} }}{1 + \cfrac{ 6\,\text{Ом} }{ 2\,\text{Ом} } + \cfrac{ 6\,\text{Ом} }{ 10\,\text{Ом} }} \cdot 6\,\text{Ом}
            = -\frac5{23}\units{А} \cdot 6\,\text{Ом} = -\frac{30}{23}\units{В} \approx -1{,}3000\,\text{В}.
    \end{align*}

    Одну пару силы тока и напряжения получили.
    Для некоторых вариантов это уже ответ, но не у всех.
    Для упрощения записи преобразуем (чтобы избавитсья от 4-этажной дроби) и подставим в уже полученные уравнения:

    \begin{align*}
    \eli_2
        &=
        \frac{\frac{\ele_2 - \ele_1}{ R_1 } + \frac{\ele_2 - \ele_3}{ R_3 }}{1 + \frac{ R_2 }{ R_1 } + \frac{ R_2 }{ R_3 }}
        =
        \frac{(\ele_2 - \ele_1)R_3 + (\ele_2 - \ele_3)R_1}{R_1R_3 + R_2R_3 + R_2R_1},
        \\
    \eli_1
        &=  \frac{\ele_1 - \ele_2 + \eli_2R_2}{R_1}
        =   \frac{\ele_1 - \ele_2 + \cfrac{(\ele_2 - \ele_1)R_3 + (\ele_2 - \ele_3)R_1}{R_1R_3 + R_2R_3 + R_2R_1} \cdot R_2}{R_1} = \\
        &=  \frac{
            \ele_1R_1R_3 + \ele_1R_2R_3 + \ele_1R_2R_1
            - \ele_2R_1R_3 - \ele_2R_2R_3 - \ele_2R_2R_1
            + \ele_2R_3R_2 - \ele_1R_3R_2 + \ele_2R_1R_2 - \ele_3R_1R_2
       }{R_1 \cdot \cbr{R_1R_3 + R_2R_3 + R_2R_1}}
        = \\ &=
        \frac{
            \ele_1\cbr{R_1R_3 + R_2R_3 + R_2R_1 - R_3R_2}
            + \ele_2\cbr{- R_1R_3 - R_2R_3 - R_2R_1 + R_3R_2 + R_1R_2}
            - \ele_3R_1R_2
       }{R_1 \cdot \cbr{R_1R_3 + R_2R_3 + R_2R_1}}
        = \\ &=
        \frac{
            \ele_1\cbr{R_1R_3 + R_2R_1}
            + \ele_2\cbr{- R_1R_3}
            - \ele_3R_1R_2
       }{R_1 \cdot \cbr{R_1R_3 + R_2R_3 + R_2R_1}}
        =
        \frac{
            \ele_1\cbr{R_3 + R_2} - \ele_2R_3 - \ele_3R_2
       }{R_1R_3 + R_2R_3 + R_2R_1}
        = \\ &=
        \frac{
            (\ele_1 - \ele_3)R_2 + (\ele_1 - \ele_2)R_3
       }{R_1R_3 + R_2R_3 + R_2R_1}
        =
        \frac{
            \cfrac{\ele_1 - \ele_3}{ R_3 } + \cfrac{\ele_1 - \ele_2}{ R_2 }
       }{\cfrac{ R_1 }{ R_2 } + 1 + \cfrac{ R_1 }{ R_3 }}
        =
        \frac{
            \cfrac{4\,\text{В} - 8\,\text{В}}{ 10\,\text{Ом} } + \cfrac{4\,\text{В} - 3\,\text{В}}{ 6\,\text{Ом} }
       }{\cfrac{ 2\,\text{Ом} }{ 6\,\text{Ом} } + 1 + \cfrac{ 2\,\text{Ом} }{ 10\,\text{Ом} }}
        = -\frac7{46}\units{А} \approx -0{,}15000\,\text{А}.
        \\
    U_1
        &=
        \eli_1R_1
        =
        \frac{
            \cfrac{\ele_1 - \ele_3}{ R_3 } + \cfrac{\ele_1 - \ele_2}{ R_2 }
       }{\cfrac{ R_1 }{ R_2 } + 1 + \cfrac{ R_1 }{ R_3 }} \cdot R_1
        =
        -\frac7{46}\units{А} \cdot 2\,\text{Ом} = -\frac7{23}\units{В} \approx -0{,}3000\,\text{В}.
    \end{align*}

    Если вы проделали все эти вычисления выше вместе со мной, то
    \begin{itemize}
        \item вы совершили ошибку, выбрав неверный путь решения:
        слишком длинное решение, очень легко ошибиться в индексах, дробях, знаках или потерять какой-то множитель,
        \item можно было выразить из исходной системы другие токи и получить сразу нажный вам,
        а не какой-то 2-й,
        \item можно было сэкономить: все три резистора и ЭДС соединены одинаково,
        поэтому ответ для 1-го резистора должен отличаться лишь перестановкой индексов (этот факт крайне полезен при проверке ответа, у нас всё сошлось),
        я специально подгонял выражение для $\eli_1$ к этому виду, вынося за скобки и преобразуя дробь,
        \item вы молодец, потому что не побоялись и получили верный ответ грамотным способом,
    \end{itemize}
    так что переходим к третьему резистору.
    Будет похоже, но кого это когда останавливало...

    \begin{align*}
    \eli_3
        &=  \frac{\eli_2R_2 - \ele_2 + \ele_3}{ R_3 }
        =
        \cfrac{
            \cfrac{
                (\ele_2 - \ele_1)R_3 + (\ele_2 - \ele_3)R_1
           }{
                R_1R_3 + R_2R_3 + R_2R_1
           } \cdot R_2 - \ele_2 + \ele_3}{ R_3 }
        = \\ &=
        \frac{
            \ele_2R_3R_2 - \ele_1R_3R_2 + \ele_2R_1R_2 - \ele_3R_1R_2
            - \ele_2R_1R_3 - \ele_2R_2R_3 - \ele_2R_2R_1
            + \ele_3R_1R_3 + \ele_3R_2R_3 + \ele_3R_2R_1
       }{\cbr{R_1R_3 + R_2R_3 + R_2R_1} \cdot R_3}
        = \\ &=
        \frac{
            - \ele_1R_3R_2 - \ele_2R_1R_3 + \ele_3R_1R_3 + \ele_3R_2R_3
       }{\cbr{R_1R_3 + R_2R_3 + R_2R_1} \cdot R_3}
        =
        \frac{
            - \ele_1R_2 - \ele_2R_1 + \ele_3R_1 + \ele_3R_2
       }{R_1R_3 + R_2R_3 + R_2R_1}
        = \\ &=
        \frac{
            R_1(\ele_3 - \ele_2) + R_2(\ele_3 - \ele_1)
       }{R_1R_3 + R_2R_3 + R_2R_1}
        =
        \frac{
            \cfrac{\ele_3 - \ele_2}{ R_2 } + \cfrac{\ele_3 - \ele_1}{ R_1 }
       }{\cfrac{ R_3 }{ R_2 } + \cfrac{ R_3 }{ R_1 } + 1}
        =
        \frac{
            \cfrac{8\,\text{В} - 3\,\text{В}}{ 6\,\text{Ом} } + \cfrac{8\,\text{В} - 4\,\text{В}}{ 2\,\text{Ом} }
       }{\cfrac{ 10\,\text{Ом} }{ 6\,\text{Ом} } + \cfrac{ 10\,\text{Ом} }{ 2\,\text{Ом} } + 1}
        = \frac{17}{46}\units{А} \approx 0{,}37\,\text{А}.
        \\
    U_3
        &=
        \eli_3R_3
        =
        \frac{
            \cfrac{\ele_3 - \ele_2}{ R_2 } + \cfrac{\ele_3 - \ele_1}{ R_1 }
       }{\cfrac{ R_3 }{ R_2 } + \cfrac{ R_3 }{ R_1 } + 1} \cdot R_3
        =
        \frac{17}{46}\units{А} \cdot 10\,\text{Ом} = \frac{85}{23}\units{В} \approx 3{,}70\,\text{В}.
    \end{align*}

    Положительные ответы говорят, что мы угадали на рисунке направление тока (тут нет нашей заслуги, повезло),
    отрицательные — что не угадали (и в этом нет ошибки), и ток течёт в противоположную сторону.
    Напомним, что направление тока — это направление движения положительных зарядов,
    а в металлах носители заряда — электроны, которые заряжены отрицательно.
}

\variantsplitter

\addpersonalvariant{Матвей Кузьмин}

\tasknumber{1}%
\task{%
    Определите ток $\eli_2$, протекающий через резистор $R_2$ (см.
    рис.),
    направление этого тока и разность потенциалов $U_2$ на этом резисторе,
    если $R_1 = 3\,\text{Ом}$, $R_2 = 8\,\text{Ом}$, $R_3 = 15\,\text{Ом}$, $\ele_1 = 5\,\text{В}$, $\ele_2 = 3\,\text{В}$, $\ele_3 = 2\,\text{В}$.
    Внутренним сопротивлением всех трёх ЭДС пренебречь.
    Ответы получите в виде несократимых дробей, а также определите приближённые значения.

    \begin{tikzpicture}[circuit ee IEC, thick]
        \foreach \contact/\x in {1/0, 2/3, 3/6}
        {
            \node [contact] (top contact \contact) at (\x, 0) {};
            \node [contact] (bottom contact \contact) at (\x, 4) {};
       }
        \draw  (bottom contact 1) -- (bottom contact 2) -- (bottom contact 3);
        \draw  (top contact 1) -- (top contact 2) -- (top contact 3);
        \draw  (bottom contact 1) to [resistor={near start, info=$R_1$}, battery={near end, info=$\ele_1$}] (top contact 1);
        \draw  (bottom contact 2) to [resistor={near start, info=$R_2$}, battery={near end, info=$\ele_2$}] (top contact 2);
        \draw  (bottom contact 3) to [resistor={near start, info=$R_3$}, battery={near end, info=$\ele_3$}] (top contact 3);
    \end{tikzpicture}
}
\answer{%
    План:
    \begin{itemize}
        \item отметим на рисунке произвольно направления токов (если получим отрицательный ответ, значит не угадали направление и только),
        \item выберем и обозначим на рисунке контуры (здесь всего 3, значит будет нужно $3-1=2$), для них запишем законы Кирхгофа,
        \item выберем и выделим на рисунке нетривиальные узлы (здесь всего 2, значит будет нужно $2-1=1$), для него запишем закон Кирхгофа,
        \item попытаемся решить получившуюся систему.
        В конкретном решении мы пытались первым делом найти $\eli_2$, но, возможно, в вашем варианте будет быстрее решать систему в другом порядке.
        Мы всё же проделаем всё в лоб, подробно и целиком.
    \end{itemize}


    \begin{tikzpicture}[circuit ee IEC, thick]
        \foreach \contact/\x in {1/0, 2/3, 3/6}
        {
            \node [contact] (top contact \contact) at (\x, 0) {};
            \node [contact] (bottom contact \contact) at (\x, 4) {};
       }
        \draw  (bottom contact 1) -- (bottom contact 2) -- (bottom contact 3);
        \draw  (top contact 1) -- (top contact 2) -- (top contact 3);
        \draw  (bottom contact 1) to [resistor={near start, info=$R_1$}, current direction'={midway, info=$\eli_1$}, battery={near end, info=$\ele_1$}] (top contact 1);
        \draw  (bottom contact 2) to [resistor={near start, info=$R_2$}, current direction'={midway, info=$\eli_2$}, battery={near end, info=$\ele_2$}] (top contact 2);
        \draw  (bottom contact 3) to [resistor={near start, info=$R_3$}, current direction'={midway, info=$\eli_3$}, battery={near end, info=$\ele_3$}] (top contact 3);
        \draw [-{Latex},color=red] (1.2, 2.5) arc [start angle = 135, end angle = -160, radius = 0.6];
        \draw [-{Latex},color=blue] (4.2, 2.5) arc [start angle = 135, end angle = -160, radius = 0.6];
        \node [contact,color=green!71!black] (bottomc) at (bottom contact 2) {};
    \end{tikzpicture}

    \begin{align*}
        &\begin{cases}
            {\color{red} \eli_1R_1 - \eli_2R_2 = \ele_1 - \ele_2}, \\
            {\color{blue} \eli_2R_2 - \eli_3R_3 = \ele_2 - \ele_3}, \\
            {\color{green!71!black} \eli_1 + \eli_2 + \eli_3 = 0};
        \end{cases}
        \qquad \implies \qquad
        \begin{cases}
            \eli_1 = \frac{\ele_1 - \ele_2 + \eli_2R_2}{R_1}, \\
            \eli_3 = \frac{\eli_2R_2 - \ele_2 + \ele_3}{R_3}, \\
            \eli_1 + \eli_2 + \eli_3 = 0, \\
        \end{cases} \implies \\
        \implies
            &\eli_2 + \frac{\ele_1 - \ele_2 + \eli_2R_2}{R_1} + \frac{\eli_2R_2 - \ele_2 + \ele_3}{R_3} = 0, \\
        &   \eli_2\cbr{1 + \frac{ R_2 }{ R_1 } + \frac{ R_2 }{ R_3 }} + \frac{\ele_1 - \ele_2}{ R_1 } + \frac{\ele_3 - \ele_2}{ R_3 } = 0, \\
        &   \eli_2 = \cfrac{\cfrac{\ele_2 - \ele_1}{ R_1 } + \cfrac{\ele_2 - \ele_3}{ R_3 }}{1 + \cfrac{ R_2 }{ R_1 } + \cfrac{ R_2 }{ R_3 }}
            = \cfrac{\cfrac{3\,\text{В} - 5\,\text{В}}{ 3\,\text{Ом} } + \cfrac{3\,\text{В} - 2\,\text{В}}{ 15\,\text{Ом} }}{1 + \cfrac{ 8\,\text{Ом} }{ 3\,\text{Ом} } + \cfrac{ 8\,\text{Ом} }{ 15\,\text{Ом} }}
            = -\frac17\units{А} \approx -0{,}14000\,\text{А}, \\
        &   U_2 = \eli_2R_2 = \cfrac{\cfrac{\ele_2 - \ele_1}{ R_1 } + \cfrac{\ele_2 - \ele_3}{ R_3 }}{1 + \cfrac{ R_2 }{ R_1 } + \cfrac{ R_2 }{ R_3 }} \cdot R_2
            = \cfrac{\cfrac{3\,\text{В} - 5\,\text{В}}{ 3\,\text{Ом} } + \cfrac{3\,\text{В} - 2\,\text{В}}{ 15\,\text{Ом} }}{1 + \cfrac{ 8\,\text{Ом} }{ 3\,\text{Ом} } + \cfrac{ 8\,\text{Ом} }{ 15\,\text{Ом} }} \cdot 8\,\text{Ом}
            = -\frac17\units{А} \cdot 8\,\text{Ом} = -\frac87\units{В} \approx -1{,}1400\,\text{В}.
    \end{align*}

    Одну пару силы тока и напряжения получили.
    Для некоторых вариантов это уже ответ, но не у всех.
    Для упрощения записи преобразуем (чтобы избавитсья от 4-этажной дроби) и подставим в уже полученные уравнения:

    \begin{align*}
    \eli_2
        &=
        \frac{\frac{\ele_2 - \ele_1}{ R_1 } + \frac{\ele_2 - \ele_3}{ R_3 }}{1 + \frac{ R_2 }{ R_1 } + \frac{ R_2 }{ R_3 }}
        =
        \frac{(\ele_2 - \ele_1)R_3 + (\ele_2 - \ele_3)R_1}{R_1R_3 + R_2R_3 + R_2R_1},
        \\
    \eli_1
        &=  \frac{\ele_1 - \ele_2 + \eli_2R_2}{R_1}
        =   \frac{\ele_1 - \ele_2 + \cfrac{(\ele_2 - \ele_1)R_3 + (\ele_2 - \ele_3)R_1}{R_1R_3 + R_2R_3 + R_2R_1} \cdot R_2}{R_1} = \\
        &=  \frac{
            \ele_1R_1R_3 + \ele_1R_2R_3 + \ele_1R_2R_1
            - \ele_2R_1R_3 - \ele_2R_2R_3 - \ele_2R_2R_1
            + \ele_2R_3R_2 - \ele_1R_3R_2 + \ele_2R_1R_2 - \ele_3R_1R_2
       }{R_1 \cdot \cbr{R_1R_3 + R_2R_3 + R_2R_1}}
        = \\ &=
        \frac{
            \ele_1\cbr{R_1R_3 + R_2R_3 + R_2R_1 - R_3R_2}
            + \ele_2\cbr{- R_1R_3 - R_2R_3 - R_2R_1 + R_3R_2 + R_1R_2}
            - \ele_3R_1R_2
       }{R_1 \cdot \cbr{R_1R_3 + R_2R_3 + R_2R_1}}
        = \\ &=
        \frac{
            \ele_1\cbr{R_1R_3 + R_2R_1}
            + \ele_2\cbr{- R_1R_3}
            - \ele_3R_1R_2
       }{R_1 \cdot \cbr{R_1R_3 + R_2R_3 + R_2R_1}}
        =
        \frac{
            \ele_1\cbr{R_3 + R_2} - \ele_2R_3 - \ele_3R_2
       }{R_1R_3 + R_2R_3 + R_2R_1}
        = \\ &=
        \frac{
            (\ele_1 - \ele_3)R_2 + (\ele_1 - \ele_2)R_3
       }{R_1R_3 + R_2R_3 + R_2R_1}
        =
        \frac{
            \cfrac{\ele_1 - \ele_3}{ R_3 } + \cfrac{\ele_1 - \ele_2}{ R_2 }
       }{\cfrac{ R_1 }{ R_2 } + 1 + \cfrac{ R_1 }{ R_3 }}
        =
        \frac{
            \cfrac{5\,\text{В} - 2\,\text{В}}{ 15\,\text{Ом} } + \cfrac{5\,\text{В} - 3\,\text{В}}{ 8\,\text{Ом} }
       }{\cfrac{ 3\,\text{Ом} }{ 8\,\text{Ом} } + 1 + \cfrac{ 3\,\text{Ом} }{ 15\,\text{Ом} }}
        = \frac27\units{А} \approx 0{,}29\,\text{А}.
        \\
    U_1
        &=
        \eli_1R_1
        =
        \frac{
            \cfrac{\ele_1 - \ele_3}{ R_3 } + \cfrac{\ele_1 - \ele_2}{ R_2 }
       }{\cfrac{ R_1 }{ R_2 } + 1 + \cfrac{ R_1 }{ R_3 }} \cdot R_1
        =
        \frac27\units{А} \cdot 3\,\text{Ом} = \frac67\units{В} \approx 0{,}86\,\text{В}.
    \end{align*}

    Если вы проделали все эти вычисления выше вместе со мной, то
    \begin{itemize}
        \item вы совершили ошибку, выбрав неверный путь решения:
        слишком длинное решение, очень легко ошибиться в индексах, дробях, знаках или потерять какой-то множитель,
        \item можно было выразить из исходной системы другие токи и получить сразу нажный вам,
        а не какой-то 2-й,
        \item можно было сэкономить: все три резистора и ЭДС соединены одинаково,
        поэтому ответ для 1-го резистора должен отличаться лишь перестановкой индексов (этот факт крайне полезен при проверке ответа, у нас всё сошлось),
        я специально подгонял выражение для $\eli_1$ к этому виду, вынося за скобки и преобразуя дробь,
        \item вы молодец, потому что не побоялись и получили верный ответ грамотным способом,
    \end{itemize}
    так что переходим к третьему резистору.
    Будет похоже, но кого это когда останавливало...

    \begin{align*}
    \eli_3
        &=  \frac{\eli_2R_2 - \ele_2 + \ele_3}{ R_3 }
        =
        \cfrac{
            \cfrac{
                (\ele_2 - \ele_1)R_3 + (\ele_2 - \ele_3)R_1
           }{
                R_1R_3 + R_2R_3 + R_2R_1
           } \cdot R_2 - \ele_2 + \ele_3}{ R_3 }
        = \\ &=
        \frac{
            \ele_2R_3R_2 - \ele_1R_3R_2 + \ele_2R_1R_2 - \ele_3R_1R_2
            - \ele_2R_1R_3 - \ele_2R_2R_3 - \ele_2R_2R_1
            + \ele_3R_1R_3 + \ele_3R_2R_3 + \ele_3R_2R_1
       }{\cbr{R_1R_3 + R_2R_3 + R_2R_1} \cdot R_3}
        = \\ &=
        \frac{
            - \ele_1R_3R_2 - \ele_2R_1R_3 + \ele_3R_1R_3 + \ele_3R_2R_3
       }{\cbr{R_1R_3 + R_2R_3 + R_2R_1} \cdot R_3}
        =
        \frac{
            - \ele_1R_2 - \ele_2R_1 + \ele_3R_1 + \ele_3R_2
       }{R_1R_3 + R_2R_3 + R_2R_1}
        = \\ &=
        \frac{
            R_1(\ele_3 - \ele_2) + R_2(\ele_3 - \ele_1)
       }{R_1R_3 + R_2R_3 + R_2R_1}
        =
        \frac{
            \cfrac{\ele_3 - \ele_2}{ R_2 } + \cfrac{\ele_3 - \ele_1}{ R_1 }
       }{\cfrac{ R_3 }{ R_2 } + \cfrac{ R_3 }{ R_1 } + 1}
        =
        \frac{
            \cfrac{2\,\text{В} - 3\,\text{В}}{ 8\,\text{Ом} } + \cfrac{2\,\text{В} - 5\,\text{В}}{ 3\,\text{Ом} }
       }{\cfrac{ 15\,\text{Ом} }{ 8\,\text{Ом} } + \cfrac{ 15\,\text{Ом} }{ 3\,\text{Ом} } + 1}
        = -\frac17\units{А} \approx -0{,}14000\,\text{А}.
        \\
    U_3
        &=
        \eli_3R_3
        =
        \frac{
            \cfrac{\ele_3 - \ele_2}{ R_2 } + \cfrac{\ele_3 - \ele_1}{ R_1 }
       }{\cfrac{ R_3 }{ R_2 } + \cfrac{ R_3 }{ R_1 } + 1} \cdot R_3
        =
        -\frac17\units{А} \cdot 15\,\text{Ом} = -\frac{15}7\units{В} \approx -2{,}140\,\text{В}.
    \end{align*}

    Положительные ответы говорят, что мы угадали на рисунке направление тока (тут нет нашей заслуги, повезло),
    отрицательные — что не угадали (и в этом нет ошибки), и ток течёт в противоположную сторону.
    Напомним, что направление тока — это направление движения положительных зарядов,
    а в металлах носители заряда — электроны, которые заряжены отрицательно.
}

\variantsplitter

\addpersonalvariant{Сергей Малышев}

\tasknumber{1}%
\task{%
    Определите ток $\eli_3$, протекающий через резистор $R_3$ (см.
    рис.),
    направление этого тока и разность потенциалов $U_3$ на этом резисторе,
    если $R_1 = 2\,\text{Ом}$, $R_2 = 5\,\text{Ом}$, $R_3 = 15\,\text{Ом}$, $\ele_1 = 4\,\text{В}$, $\ele_2 = 3\,\text{В}$, $\ele_3 = 2\,\text{В}$.
    Внутренним сопротивлением всех трёх ЭДС пренебречь.
    Ответы получите в виде несократимых дробей, а также определите приближённые значения.

    \begin{tikzpicture}[circuit ee IEC, thick]
        \foreach \contact/\x in {1/0, 2/3, 3/6}
        {
            \node [contact] (top contact \contact) at (\x, 0) {};
            \node [contact] (bottom contact \contact) at (\x, 4) {};
       }
        \draw  (bottom contact 1) -- (bottom contact 2) -- (bottom contact 3);
        \draw  (top contact 1) -- (top contact 2) -- (top contact 3);
        \draw  (bottom contact 1) to [resistor={near start, info=$R_1$}, battery={near end, info=$\ele_1$}] (top contact 1);
        \draw  (bottom contact 2) to [resistor={near start, info=$R_2$}, battery={near end, info=$\ele_2$}] (top contact 2);
        \draw  (bottom contact 3) to [resistor={near start, info=$R_3$}, battery={near end, info=$\ele_3$}] (top contact 3);
    \end{tikzpicture}
}
\answer{%
    План:
    \begin{itemize}
        \item отметим на рисунке произвольно направления токов (если получим отрицательный ответ, значит не угадали направление и только),
        \item выберем и обозначим на рисунке контуры (здесь всего 3, значит будет нужно $3-1=2$), для них запишем законы Кирхгофа,
        \item выберем и выделим на рисунке нетривиальные узлы (здесь всего 2, значит будет нужно $2-1=1$), для него запишем закон Кирхгофа,
        \item попытаемся решить получившуюся систему.
        В конкретном решении мы пытались первым делом найти $\eli_2$, но, возможно, в вашем варианте будет быстрее решать систему в другом порядке.
        Мы всё же проделаем всё в лоб, подробно и целиком.
    \end{itemize}


    \begin{tikzpicture}[circuit ee IEC, thick]
        \foreach \contact/\x in {1/0, 2/3, 3/6}
        {
            \node [contact] (top contact \contact) at (\x, 0) {};
            \node [contact] (bottom contact \contact) at (\x, 4) {};
       }
        \draw  (bottom contact 1) -- (bottom contact 2) -- (bottom contact 3);
        \draw  (top contact 1) -- (top contact 2) -- (top contact 3);
        \draw  (bottom contact 1) to [resistor={near start, info=$R_1$}, current direction'={midway, info=$\eli_1$}, battery={near end, info=$\ele_1$}] (top contact 1);
        \draw  (bottom contact 2) to [resistor={near start, info=$R_2$}, current direction'={midway, info=$\eli_2$}, battery={near end, info=$\ele_2$}] (top contact 2);
        \draw  (bottom contact 3) to [resistor={near start, info=$R_3$}, current direction'={midway, info=$\eli_3$}, battery={near end, info=$\ele_3$}] (top contact 3);
        \draw [-{Latex},color=red] (1.2, 2.5) arc [start angle = 135, end angle = -160, radius = 0.6];
        \draw [-{Latex},color=blue] (4.2, 2.5) arc [start angle = 135, end angle = -160, radius = 0.6];
        \node [contact,color=green!71!black] (bottomc) at (bottom contact 2) {};
    \end{tikzpicture}

    \begin{align*}
        &\begin{cases}
            {\color{red} \eli_1R_1 - \eli_2R_2 = \ele_1 - \ele_2}, \\
            {\color{blue} \eli_2R_2 - \eli_3R_3 = \ele_2 - \ele_3}, \\
            {\color{green!71!black} \eli_1 + \eli_2 + \eli_3 = 0};
        \end{cases}
        \qquad \implies \qquad
        \begin{cases}
            \eli_1 = \frac{\ele_1 - \ele_2 + \eli_2R_2}{R_1}, \\
            \eli_3 = \frac{\eli_2R_2 - \ele_2 + \ele_3}{R_3}, \\
            \eli_1 + \eli_2 + \eli_3 = 0, \\
        \end{cases} \implies \\
        \implies
            &\eli_2 + \frac{\ele_1 - \ele_2 + \eli_2R_2}{R_1} + \frac{\eli_2R_2 - \ele_2 + \ele_3}{R_3} = 0, \\
        &   \eli_2\cbr{1 + \frac{ R_2 }{ R_1 } + \frac{ R_2 }{ R_3 }} + \frac{\ele_1 - \ele_2}{ R_1 } + \frac{\ele_3 - \ele_2}{ R_3 } = 0, \\
        &   \eli_2 = \cfrac{\cfrac{\ele_2 - \ele_1}{ R_1 } + \cfrac{\ele_2 - \ele_3}{ R_3 }}{1 + \cfrac{ R_2 }{ R_1 } + \cfrac{ R_2 }{ R_3 }}
            = \cfrac{\cfrac{3\,\text{В} - 4\,\text{В}}{ 2\,\text{Ом} } + \cfrac{3\,\text{В} - 2\,\text{В}}{ 15\,\text{Ом} }}{1 + \cfrac{ 5\,\text{Ом} }{ 2\,\text{Ом} } + \cfrac{ 5\,\text{Ом} }{ 15\,\text{Ом} }}
            = -\frac{13}{115}\units{А} \approx -0{,}11000\,\text{А}, \\
        &   U_2 = \eli_2R_2 = \cfrac{\cfrac{\ele_2 - \ele_1}{ R_1 } + \cfrac{\ele_2 - \ele_3}{ R_3 }}{1 + \cfrac{ R_2 }{ R_1 } + \cfrac{ R_2 }{ R_3 }} \cdot R_2
            = \cfrac{\cfrac{3\,\text{В} - 4\,\text{В}}{ 2\,\text{Ом} } + \cfrac{3\,\text{В} - 2\,\text{В}}{ 15\,\text{Ом} }}{1 + \cfrac{ 5\,\text{Ом} }{ 2\,\text{Ом} } + \cfrac{ 5\,\text{Ом} }{ 15\,\text{Ом} }} \cdot 5\,\text{Ом}
            = -\frac{13}{115}\units{А} \cdot 5\,\text{Ом} = -\frac{13}{23}\units{В} \approx -0{,}5700\,\text{В}.
    \end{align*}

    Одну пару силы тока и напряжения получили.
    Для некоторых вариантов это уже ответ, но не у всех.
    Для упрощения записи преобразуем (чтобы избавитсья от 4-этажной дроби) и подставим в уже полученные уравнения:

    \begin{align*}
    \eli_2
        &=
        \frac{\frac{\ele_2 - \ele_1}{ R_1 } + \frac{\ele_2 - \ele_3}{ R_3 }}{1 + \frac{ R_2 }{ R_1 } + \frac{ R_2 }{ R_3 }}
        =
        \frac{(\ele_2 - \ele_1)R_3 + (\ele_2 - \ele_3)R_1}{R_1R_3 + R_2R_3 + R_2R_1},
        \\
    \eli_1
        &=  \frac{\ele_1 - \ele_2 + \eli_2R_2}{R_1}
        =   \frac{\ele_1 - \ele_2 + \cfrac{(\ele_2 - \ele_1)R_3 + (\ele_2 - \ele_3)R_1}{R_1R_3 + R_2R_3 + R_2R_1} \cdot R_2}{R_1} = \\
        &=  \frac{
            \ele_1R_1R_3 + \ele_1R_2R_3 + \ele_1R_2R_1
            - \ele_2R_1R_3 - \ele_2R_2R_3 - \ele_2R_2R_1
            + \ele_2R_3R_2 - \ele_1R_3R_2 + \ele_2R_1R_2 - \ele_3R_1R_2
       }{R_1 \cdot \cbr{R_1R_3 + R_2R_3 + R_2R_1}}
        = \\ &=
        \frac{
            \ele_1\cbr{R_1R_3 + R_2R_3 + R_2R_1 - R_3R_2}
            + \ele_2\cbr{- R_1R_3 - R_2R_3 - R_2R_1 + R_3R_2 + R_1R_2}
            - \ele_3R_1R_2
       }{R_1 \cdot \cbr{R_1R_3 + R_2R_3 + R_2R_1}}
        = \\ &=
        \frac{
            \ele_1\cbr{R_1R_3 + R_2R_1}
            + \ele_2\cbr{- R_1R_3}
            - \ele_3R_1R_2
       }{R_1 \cdot \cbr{R_1R_3 + R_2R_3 + R_2R_1}}
        =
        \frac{
            \ele_1\cbr{R_3 + R_2} - \ele_2R_3 - \ele_3R_2
       }{R_1R_3 + R_2R_3 + R_2R_1}
        = \\ &=
        \frac{
            (\ele_1 - \ele_3)R_2 + (\ele_1 - \ele_2)R_3
       }{R_1R_3 + R_2R_3 + R_2R_1}
        =
        \frac{
            \cfrac{\ele_1 - \ele_3}{ R_3 } + \cfrac{\ele_1 - \ele_2}{ R_2 }
       }{\cfrac{ R_1 }{ R_2 } + 1 + \cfrac{ R_1 }{ R_3 }}
        =
        \frac{
            \cfrac{4\,\text{В} - 2\,\text{В}}{ 15\,\text{Ом} } + \cfrac{4\,\text{В} - 3\,\text{В}}{ 5\,\text{Ом} }
       }{\cfrac{ 2\,\text{Ом} }{ 5\,\text{Ом} } + 1 + \cfrac{ 2\,\text{Ом} }{ 15\,\text{Ом} }}
        = \frac5{23}\units{А} \approx 0{,}22\,\text{А}.
        \\
    U_1
        &=
        \eli_1R_1
        =
        \frac{
            \cfrac{\ele_1 - \ele_3}{ R_3 } + \cfrac{\ele_1 - \ele_2}{ R_2 }
       }{\cfrac{ R_1 }{ R_2 } + 1 + \cfrac{ R_1 }{ R_3 }} \cdot R_1
        =
        \frac5{23}\units{А} \cdot 2\,\text{Ом} = \frac{10}{23}\units{В} \approx 0{,}43\,\text{В}.
    \end{align*}

    Если вы проделали все эти вычисления выше вместе со мной, то
    \begin{itemize}
        \item вы совершили ошибку, выбрав неверный путь решения:
        слишком длинное решение, очень легко ошибиться в индексах, дробях, знаках или потерять какой-то множитель,
        \item можно было выразить из исходной системы другие токи и получить сразу нажный вам,
        а не какой-то 2-й,
        \item можно было сэкономить: все три резистора и ЭДС соединены одинаково,
        поэтому ответ для 1-го резистора должен отличаться лишь перестановкой индексов (этот факт крайне полезен при проверке ответа, у нас всё сошлось),
        я специально подгонял выражение для $\eli_1$ к этому виду, вынося за скобки и преобразуя дробь,
        \item вы молодец, потому что не побоялись и получили верный ответ грамотным способом,
    \end{itemize}
    так что переходим к третьему резистору.
    Будет похоже, но кого это когда останавливало...

    \begin{align*}
    \eli_3
        &=  \frac{\eli_2R_2 - \ele_2 + \ele_3}{ R_3 }
        =
        \cfrac{
            \cfrac{
                (\ele_2 - \ele_1)R_3 + (\ele_2 - \ele_3)R_1
           }{
                R_1R_3 + R_2R_3 + R_2R_1
           } \cdot R_2 - \ele_2 + \ele_3}{ R_3 }
        = \\ &=
        \frac{
            \ele_2R_3R_2 - \ele_1R_3R_2 + \ele_2R_1R_2 - \ele_3R_1R_2
            - \ele_2R_1R_3 - \ele_2R_2R_3 - \ele_2R_2R_1
            + \ele_3R_1R_3 + \ele_3R_2R_3 + \ele_3R_2R_1
       }{\cbr{R_1R_3 + R_2R_3 + R_2R_1} \cdot R_3}
        = \\ &=
        \frac{
            - \ele_1R_3R_2 - \ele_2R_1R_3 + \ele_3R_1R_3 + \ele_3R_2R_3
       }{\cbr{R_1R_3 + R_2R_3 + R_2R_1} \cdot R_3}
        =
        \frac{
            - \ele_1R_2 - \ele_2R_1 + \ele_3R_1 + \ele_3R_2
       }{R_1R_3 + R_2R_3 + R_2R_1}
        = \\ &=
        \frac{
            R_1(\ele_3 - \ele_2) + R_2(\ele_3 - \ele_1)
       }{R_1R_3 + R_2R_3 + R_2R_1}
        =
        \frac{
            \cfrac{\ele_3 - \ele_2}{ R_2 } + \cfrac{\ele_3 - \ele_1}{ R_1 }
       }{\cfrac{ R_3 }{ R_2 } + \cfrac{ R_3 }{ R_1 } + 1}
        =
        \frac{
            \cfrac{2\,\text{В} - 3\,\text{В}}{ 5\,\text{Ом} } + \cfrac{2\,\text{В} - 4\,\text{В}}{ 2\,\text{Ом} }
       }{\cfrac{ 15\,\text{Ом} }{ 5\,\text{Ом} } + \cfrac{ 15\,\text{Ом} }{ 2\,\text{Ом} } + 1}
        = -\frac{12}{115}\units{А} \approx -0{,}10000\,\text{А}.
        \\
    U_3
        &=
        \eli_3R_3
        =
        \frac{
            \cfrac{\ele_3 - \ele_2}{ R_2 } + \cfrac{\ele_3 - \ele_1}{ R_1 }
       }{\cfrac{ R_3 }{ R_2 } + \cfrac{ R_3 }{ R_1 } + 1} \cdot R_3
        =
        -\frac{12}{115}\units{А} \cdot 15\,\text{Ом} = -\frac{36}{23}\units{В} \approx -1{,}5700\,\text{В}.
    \end{align*}

    Положительные ответы говорят, что мы угадали на рисунке направление тока (тут нет нашей заслуги, повезло),
    отрицательные — что не угадали (и в этом нет ошибки), и ток течёт в противоположную сторону.
    Напомним, что направление тока — это направление движения положительных зарядов,
    а в металлах носители заряда — электроны, которые заряжены отрицательно.
}

\variantsplitter

\addpersonalvariant{Алина Полканова}

\tasknumber{1}%
\task{%
    Определите ток $\eli_2$, протекающий через резистор $R_2$ (см.
    рис.),
    направление этого тока и разность потенциалов $U_2$ на этом резисторе,
    если $R_1 = 4\,\text{Ом}$, $R_2 = 8\,\text{Ом}$, $R_3 = 10\,\text{Ом}$, $\ele_1 = 5\,\text{В}$, $\ele_2 = 3\,\text{В}$, $\ele_3 = 8\,\text{В}$.
    Внутренним сопротивлением всех трёх ЭДС пренебречь.
    Ответы получите в виде несократимых дробей, а также определите приближённые значения.

    \begin{tikzpicture}[circuit ee IEC, thick]
        \foreach \contact/\x in {1/0, 2/3, 3/6}
        {
            \node [contact] (top contact \contact) at (\x, 0) {};
            \node [contact] (bottom contact \contact) at (\x, 4) {};
       }
        \draw  (bottom contact 1) -- (bottom contact 2) -- (bottom contact 3);
        \draw  (top contact 1) -- (top contact 2) -- (top contact 3);
        \draw  (bottom contact 1) to [resistor={near start, info=$R_1$}, battery={near end, info=$\ele_1$}] (top contact 1);
        \draw  (bottom contact 2) to [resistor={near start, info=$R_2$}, battery={near end, info=$\ele_2$}] (top contact 2);
        \draw  (bottom contact 3) to [resistor={near start, info=$R_3$}, battery={near end, info=$\ele_3$}] (top contact 3);
    \end{tikzpicture}
}
\answer{%
    План:
    \begin{itemize}
        \item отметим на рисунке произвольно направления токов (если получим отрицательный ответ, значит не угадали направление и только),
        \item выберем и обозначим на рисунке контуры (здесь всего 3, значит будет нужно $3-1=2$), для них запишем законы Кирхгофа,
        \item выберем и выделим на рисунке нетривиальные узлы (здесь всего 2, значит будет нужно $2-1=1$), для него запишем закон Кирхгофа,
        \item попытаемся решить получившуюся систему.
        В конкретном решении мы пытались первым делом найти $\eli_2$, но, возможно, в вашем варианте будет быстрее решать систему в другом порядке.
        Мы всё же проделаем всё в лоб, подробно и целиком.
    \end{itemize}


    \begin{tikzpicture}[circuit ee IEC, thick]
        \foreach \contact/\x in {1/0, 2/3, 3/6}
        {
            \node [contact] (top contact \contact) at (\x, 0) {};
            \node [contact] (bottom contact \contact) at (\x, 4) {};
       }
        \draw  (bottom contact 1) -- (bottom contact 2) -- (bottom contact 3);
        \draw  (top contact 1) -- (top contact 2) -- (top contact 3);
        \draw  (bottom contact 1) to [resistor={near start, info=$R_1$}, current direction'={midway, info=$\eli_1$}, battery={near end, info=$\ele_1$}] (top contact 1);
        \draw  (bottom contact 2) to [resistor={near start, info=$R_2$}, current direction'={midway, info=$\eli_2$}, battery={near end, info=$\ele_2$}] (top contact 2);
        \draw  (bottom contact 3) to [resistor={near start, info=$R_3$}, current direction'={midway, info=$\eli_3$}, battery={near end, info=$\ele_3$}] (top contact 3);
        \draw [-{Latex},color=red] (1.2, 2.5) arc [start angle = 135, end angle = -160, radius = 0.6];
        \draw [-{Latex},color=blue] (4.2, 2.5) arc [start angle = 135, end angle = -160, radius = 0.6];
        \node [contact,color=green!71!black] (bottomc) at (bottom contact 2) {};
    \end{tikzpicture}

    \begin{align*}
        &\begin{cases}
            {\color{red} \eli_1R_1 - \eli_2R_2 = \ele_1 - \ele_2}, \\
            {\color{blue} \eli_2R_2 - \eli_3R_3 = \ele_2 - \ele_3}, \\
            {\color{green!71!black} \eli_1 + \eli_2 + \eli_3 = 0};
        \end{cases}
        \qquad \implies \qquad
        \begin{cases}
            \eli_1 = \frac{\ele_1 - \ele_2 + \eli_2R_2}{R_1}, \\
            \eli_3 = \frac{\eli_2R_2 - \ele_2 + \ele_3}{R_3}, \\
            \eli_1 + \eli_2 + \eli_3 = 0, \\
        \end{cases} \implies \\
        \implies
            &\eli_2 + \frac{\ele_1 - \ele_2 + \eli_2R_2}{R_1} + \frac{\eli_2R_2 - \ele_2 + \ele_3}{R_3} = 0, \\
        &   \eli_2\cbr{1 + \frac{ R_2 }{ R_1 } + \frac{ R_2 }{ R_3 }} + \frac{\ele_1 - \ele_2}{ R_1 } + \frac{\ele_3 - \ele_2}{ R_3 } = 0, \\
        &   \eli_2 = \cfrac{\cfrac{\ele_2 - \ele_1}{ R_1 } + \cfrac{\ele_2 - \ele_3}{ R_3 }}{1 + \cfrac{ R_2 }{ R_1 } + \cfrac{ R_2 }{ R_3 }}
            = \cfrac{\cfrac{3\,\text{В} - 5\,\text{В}}{ 4\,\text{Ом} } + \cfrac{3\,\text{В} - 8\,\text{В}}{ 10\,\text{Ом} }}{1 + \cfrac{ 8\,\text{Ом} }{ 4\,\text{Ом} } + \cfrac{ 8\,\text{Ом} }{ 10\,\text{Ом} }}
            = -\frac5{19}\units{А} \approx -0{,}2600\,\text{А}, \\
        &   U_2 = \eli_2R_2 = \cfrac{\cfrac{\ele_2 - \ele_1}{ R_1 } + \cfrac{\ele_2 - \ele_3}{ R_3 }}{1 + \cfrac{ R_2 }{ R_1 } + \cfrac{ R_2 }{ R_3 }} \cdot R_2
            = \cfrac{\cfrac{3\,\text{В} - 5\,\text{В}}{ 4\,\text{Ом} } + \cfrac{3\,\text{В} - 8\,\text{В}}{ 10\,\text{Ом} }}{1 + \cfrac{ 8\,\text{Ом} }{ 4\,\text{Ом} } + \cfrac{ 8\,\text{Ом} }{ 10\,\text{Ом} }} \cdot 8\,\text{Ом}
            = -\frac5{19}\units{А} \cdot 8\,\text{Ом} = -\frac{40}{19}\units{В} \approx -2{,}110\,\text{В}.
    \end{align*}

    Одну пару силы тока и напряжения получили.
    Для некоторых вариантов это уже ответ, но не у всех.
    Для упрощения записи преобразуем (чтобы избавитсья от 4-этажной дроби) и подставим в уже полученные уравнения:

    \begin{align*}
    \eli_2
        &=
        \frac{\frac{\ele_2 - \ele_1}{ R_1 } + \frac{\ele_2 - \ele_3}{ R_3 }}{1 + \frac{ R_2 }{ R_1 } + \frac{ R_2 }{ R_3 }}
        =
        \frac{(\ele_2 - \ele_1)R_3 + (\ele_2 - \ele_3)R_1}{R_1R_3 + R_2R_3 + R_2R_1},
        \\
    \eli_1
        &=  \frac{\ele_1 - \ele_2 + \eli_2R_2}{R_1}
        =   \frac{\ele_1 - \ele_2 + \cfrac{(\ele_2 - \ele_1)R_3 + (\ele_2 - \ele_3)R_1}{R_1R_3 + R_2R_3 + R_2R_1} \cdot R_2}{R_1} = \\
        &=  \frac{
            \ele_1R_1R_3 + \ele_1R_2R_3 + \ele_1R_2R_1
            - \ele_2R_1R_3 - \ele_2R_2R_3 - \ele_2R_2R_1
            + \ele_2R_3R_2 - \ele_1R_3R_2 + \ele_2R_1R_2 - \ele_3R_1R_2
       }{R_1 \cdot \cbr{R_1R_3 + R_2R_3 + R_2R_1}}
        = \\ &=
        \frac{
            \ele_1\cbr{R_1R_3 + R_2R_3 + R_2R_1 - R_3R_2}
            + \ele_2\cbr{- R_1R_3 - R_2R_3 - R_2R_1 + R_3R_2 + R_1R_2}
            - \ele_3R_1R_2
       }{R_1 \cdot \cbr{R_1R_3 + R_2R_3 + R_2R_1}}
        = \\ &=
        \frac{
            \ele_1\cbr{R_1R_3 + R_2R_1}
            + \ele_2\cbr{- R_1R_3}
            - \ele_3R_1R_2
       }{R_1 \cdot \cbr{R_1R_3 + R_2R_3 + R_2R_1}}
        =
        \frac{
            \ele_1\cbr{R_3 + R_2} - \ele_2R_3 - \ele_3R_2
       }{R_1R_3 + R_2R_3 + R_2R_1}
        = \\ &=
        \frac{
            (\ele_1 - \ele_3)R_2 + (\ele_1 - \ele_2)R_3
       }{R_1R_3 + R_2R_3 + R_2R_1}
        =
        \frac{
            \cfrac{\ele_1 - \ele_3}{ R_3 } + \cfrac{\ele_1 - \ele_2}{ R_2 }
       }{\cfrac{ R_1 }{ R_2 } + 1 + \cfrac{ R_1 }{ R_3 }}
        =
        \frac{
            \cfrac{5\,\text{В} - 8\,\text{В}}{ 10\,\text{Ом} } + \cfrac{5\,\text{В} - 3\,\text{В}}{ 8\,\text{Ом} }
       }{\cfrac{ 4\,\text{Ом} }{ 8\,\text{Ом} } + 1 + \cfrac{ 4\,\text{Ом} }{ 10\,\text{Ом} }}
        = -\frac1{38}\units{А} \approx -0{,}03000\,\text{А}.
        \\
    U_1
        &=
        \eli_1R_1
        =
        \frac{
            \cfrac{\ele_1 - \ele_3}{ R_3 } + \cfrac{\ele_1 - \ele_2}{ R_2 }
       }{\cfrac{ R_1 }{ R_2 } + 1 + \cfrac{ R_1 }{ R_3 }} \cdot R_1
        =
        -\frac1{38}\units{А} \cdot 4\,\text{Ом} = -\frac2{19}\units{В} \approx -0{,}11000\,\text{В}.
    \end{align*}

    Если вы проделали все эти вычисления выше вместе со мной, то
    \begin{itemize}
        \item вы совершили ошибку, выбрав неверный путь решения:
        слишком длинное решение, очень легко ошибиться в индексах, дробях, знаках или потерять какой-то множитель,
        \item можно было выразить из исходной системы другие токи и получить сразу нажный вам,
        а не какой-то 2-й,
        \item можно было сэкономить: все три резистора и ЭДС соединены одинаково,
        поэтому ответ для 1-го резистора должен отличаться лишь перестановкой индексов (этот факт крайне полезен при проверке ответа, у нас всё сошлось),
        я специально подгонял выражение для $\eli_1$ к этому виду, вынося за скобки и преобразуя дробь,
        \item вы молодец, потому что не побоялись и получили верный ответ грамотным способом,
    \end{itemize}
    так что переходим к третьему резистору.
    Будет похоже, но кого это когда останавливало...

    \begin{align*}
    \eli_3
        &=  \frac{\eli_2R_2 - \ele_2 + \ele_3}{ R_3 }
        =
        \cfrac{
            \cfrac{
                (\ele_2 - \ele_1)R_3 + (\ele_2 - \ele_3)R_1
           }{
                R_1R_3 + R_2R_3 + R_2R_1
           } \cdot R_2 - \ele_2 + \ele_3}{ R_3 }
        = \\ &=
        \frac{
            \ele_2R_3R_2 - \ele_1R_3R_2 + \ele_2R_1R_2 - \ele_3R_1R_2
            - \ele_2R_1R_3 - \ele_2R_2R_3 - \ele_2R_2R_1
            + \ele_3R_1R_3 + \ele_3R_2R_3 + \ele_3R_2R_1
       }{\cbr{R_1R_3 + R_2R_3 + R_2R_1} \cdot R_3}
        = \\ &=
        \frac{
            - \ele_1R_3R_2 - \ele_2R_1R_3 + \ele_3R_1R_3 + \ele_3R_2R_3
       }{\cbr{R_1R_3 + R_2R_3 + R_2R_1} \cdot R_3}
        =
        \frac{
            - \ele_1R_2 - \ele_2R_1 + \ele_3R_1 + \ele_3R_2
       }{R_1R_3 + R_2R_3 + R_2R_1}
        = \\ &=
        \frac{
            R_1(\ele_3 - \ele_2) + R_2(\ele_3 - \ele_1)
       }{R_1R_3 + R_2R_3 + R_2R_1}
        =
        \frac{
            \cfrac{\ele_3 - \ele_2}{ R_2 } + \cfrac{\ele_3 - \ele_1}{ R_1 }
       }{\cfrac{ R_3 }{ R_2 } + \cfrac{ R_3 }{ R_1 } + 1}
        =
        \frac{
            \cfrac{8\,\text{В} - 3\,\text{В}}{ 8\,\text{Ом} } + \cfrac{8\,\text{В} - 5\,\text{В}}{ 4\,\text{Ом} }
       }{\cfrac{ 10\,\text{Ом} }{ 8\,\text{Ом} } + \cfrac{ 10\,\text{Ом} }{ 4\,\text{Ом} } + 1}
        = \frac{11}{38}\units{А} \approx 0{,}29\,\text{А}.
        \\
    U_3
        &=
        \eli_3R_3
        =
        \frac{
            \cfrac{\ele_3 - \ele_2}{ R_2 } + \cfrac{\ele_3 - \ele_1}{ R_1 }
       }{\cfrac{ R_3 }{ R_2 } + \cfrac{ R_3 }{ R_1 } + 1} \cdot R_3
        =
        \frac{11}{38}\units{А} \cdot 10\,\text{Ом} = \frac{55}{19}\units{В} \approx 2{,}89\,\text{В}.
    \end{align*}

    Положительные ответы говорят, что мы угадали на рисунке направление тока (тут нет нашей заслуги, повезло),
    отрицательные — что не угадали (и в этом нет ошибки), и ток течёт в противоположную сторону.
    Напомним, что направление тока — это направление движения положительных зарядов,
    а в металлах носители заряда — электроны, которые заряжены отрицательно.
}

\variantsplitter

\addpersonalvariant{Сергей Пономарёв}

\tasknumber{1}%
\task{%
    Определите ток $\eli_1$, протекающий через резистор $R_1$ (см.
    рис.),
    направление этого тока и разность потенциалов $U_1$ на этом резисторе,
    если $R_1 = 2\,\text{Ом}$, $R_2 = 5\,\text{Ом}$, $R_3 = 15\,\text{Ом}$, $\ele_1 = 5\,\text{В}$, $\ele_2 = 3\,\text{В}$, $\ele_3 = 2\,\text{В}$.
    Внутренним сопротивлением всех трёх ЭДС пренебречь.
    Ответы получите в виде несократимых дробей, а также определите приближённые значения.

    \begin{tikzpicture}[circuit ee IEC, thick]
        \foreach \contact/\x in {1/0, 2/3, 3/6}
        {
            \node [contact] (top contact \contact) at (\x, 0) {};
            \node [contact] (bottom contact \contact) at (\x, 4) {};
       }
        \draw  (bottom contact 1) -- (bottom contact 2) -- (bottom contact 3);
        \draw  (top contact 1) -- (top contact 2) -- (top contact 3);
        \draw  (bottom contact 1) to [resistor={near start, info=$R_1$}, battery={near end, info=$\ele_1$}] (top contact 1);
        \draw  (bottom contact 2) to [resistor={near start, info=$R_2$}, battery={near end, info=$\ele_2$}] (top contact 2);
        \draw  (bottom contact 3) to [resistor={near start, info=$R_3$}, battery={near end, info=$\ele_3$}] (top contact 3);
    \end{tikzpicture}
}
\answer{%
    План:
    \begin{itemize}
        \item отметим на рисунке произвольно направления токов (если получим отрицательный ответ, значит не угадали направление и только),
        \item выберем и обозначим на рисунке контуры (здесь всего 3, значит будет нужно $3-1=2$), для них запишем законы Кирхгофа,
        \item выберем и выделим на рисунке нетривиальные узлы (здесь всего 2, значит будет нужно $2-1=1$), для него запишем закон Кирхгофа,
        \item попытаемся решить получившуюся систему.
        В конкретном решении мы пытались первым делом найти $\eli_2$, но, возможно, в вашем варианте будет быстрее решать систему в другом порядке.
        Мы всё же проделаем всё в лоб, подробно и целиком.
    \end{itemize}


    \begin{tikzpicture}[circuit ee IEC, thick]
        \foreach \contact/\x in {1/0, 2/3, 3/6}
        {
            \node [contact] (top contact \contact) at (\x, 0) {};
            \node [contact] (bottom contact \contact) at (\x, 4) {};
       }
        \draw  (bottom contact 1) -- (bottom contact 2) -- (bottom contact 3);
        \draw  (top contact 1) -- (top contact 2) -- (top contact 3);
        \draw  (bottom contact 1) to [resistor={near start, info=$R_1$}, current direction'={midway, info=$\eli_1$}, battery={near end, info=$\ele_1$}] (top contact 1);
        \draw  (bottom contact 2) to [resistor={near start, info=$R_2$}, current direction'={midway, info=$\eli_2$}, battery={near end, info=$\ele_2$}] (top contact 2);
        \draw  (bottom contact 3) to [resistor={near start, info=$R_3$}, current direction'={midway, info=$\eli_3$}, battery={near end, info=$\ele_3$}] (top contact 3);
        \draw [-{Latex},color=red] (1.2, 2.5) arc [start angle = 135, end angle = -160, radius = 0.6];
        \draw [-{Latex},color=blue] (4.2, 2.5) arc [start angle = 135, end angle = -160, radius = 0.6];
        \node [contact,color=green!71!black] (bottomc) at (bottom contact 2) {};
    \end{tikzpicture}

    \begin{align*}
        &\begin{cases}
            {\color{red} \eli_1R_1 - \eli_2R_2 = \ele_1 - \ele_2}, \\
            {\color{blue} \eli_2R_2 - \eli_3R_3 = \ele_2 - \ele_3}, \\
            {\color{green!71!black} \eli_1 + \eli_2 + \eli_3 = 0};
        \end{cases}
        \qquad \implies \qquad
        \begin{cases}
            \eli_1 = \frac{\ele_1 - \ele_2 + \eli_2R_2}{R_1}, \\
            \eli_3 = \frac{\eli_2R_2 - \ele_2 + \ele_3}{R_3}, \\
            \eli_1 + \eli_2 + \eli_3 = 0, \\
        \end{cases} \implies \\
        \implies
            &\eli_2 + \frac{\ele_1 - \ele_2 + \eli_2R_2}{R_1} + \frac{\eli_2R_2 - \ele_2 + \ele_3}{R_3} = 0, \\
        &   \eli_2\cbr{1 + \frac{ R_2 }{ R_1 } + \frac{ R_2 }{ R_3 }} + \frac{\ele_1 - \ele_2}{ R_1 } + \frac{\ele_3 - \ele_2}{ R_3 } = 0, \\
        &   \eli_2 = \cfrac{\cfrac{\ele_2 - \ele_1}{ R_1 } + \cfrac{\ele_2 - \ele_3}{ R_3 }}{1 + \cfrac{ R_2 }{ R_1 } + \cfrac{ R_2 }{ R_3 }}
            = \cfrac{\cfrac{3\,\text{В} - 5\,\text{В}}{ 2\,\text{Ом} } + \cfrac{3\,\text{В} - 2\,\text{В}}{ 15\,\text{Ом} }}{1 + \cfrac{ 5\,\text{Ом} }{ 2\,\text{Ом} } + \cfrac{ 5\,\text{Ом} }{ 15\,\text{Ом} }}
            = -\frac{28}{115}\units{А} \approx -0{,}2400\,\text{А}, \\
        &   U_2 = \eli_2R_2 = \cfrac{\cfrac{\ele_2 - \ele_1}{ R_1 } + \cfrac{\ele_2 - \ele_3}{ R_3 }}{1 + \cfrac{ R_2 }{ R_1 } + \cfrac{ R_2 }{ R_3 }} \cdot R_2
            = \cfrac{\cfrac{3\,\text{В} - 5\,\text{В}}{ 2\,\text{Ом} } + \cfrac{3\,\text{В} - 2\,\text{В}}{ 15\,\text{Ом} }}{1 + \cfrac{ 5\,\text{Ом} }{ 2\,\text{Ом} } + \cfrac{ 5\,\text{Ом} }{ 15\,\text{Ом} }} \cdot 5\,\text{Ом}
            = -\frac{28}{115}\units{А} \cdot 5\,\text{Ом} = -\frac{28}{23}\units{В} \approx -1{,}2200\,\text{В}.
    \end{align*}

    Одну пару силы тока и напряжения получили.
    Для некоторых вариантов это уже ответ, но не у всех.
    Для упрощения записи преобразуем (чтобы избавитсья от 4-этажной дроби) и подставим в уже полученные уравнения:

    \begin{align*}
    \eli_2
        &=
        \frac{\frac{\ele_2 - \ele_1}{ R_1 } + \frac{\ele_2 - \ele_3}{ R_3 }}{1 + \frac{ R_2 }{ R_1 } + \frac{ R_2 }{ R_3 }}
        =
        \frac{(\ele_2 - \ele_1)R_3 + (\ele_2 - \ele_3)R_1}{R_1R_3 + R_2R_3 + R_2R_1},
        \\
    \eli_1
        &=  \frac{\ele_1 - \ele_2 + \eli_2R_2}{R_1}
        =   \frac{\ele_1 - \ele_2 + \cfrac{(\ele_2 - \ele_1)R_3 + (\ele_2 - \ele_3)R_1}{R_1R_3 + R_2R_3 + R_2R_1} \cdot R_2}{R_1} = \\
        &=  \frac{
            \ele_1R_1R_3 + \ele_1R_2R_3 + \ele_1R_2R_1
            - \ele_2R_1R_3 - \ele_2R_2R_3 - \ele_2R_2R_1
            + \ele_2R_3R_2 - \ele_1R_3R_2 + \ele_2R_1R_2 - \ele_3R_1R_2
       }{R_1 \cdot \cbr{R_1R_3 + R_2R_3 + R_2R_1}}
        = \\ &=
        \frac{
            \ele_1\cbr{R_1R_3 + R_2R_3 + R_2R_1 - R_3R_2}
            + \ele_2\cbr{- R_1R_3 - R_2R_3 - R_2R_1 + R_3R_2 + R_1R_2}
            - \ele_3R_1R_2
       }{R_1 \cdot \cbr{R_1R_3 + R_2R_3 + R_2R_1}}
        = \\ &=
        \frac{
            \ele_1\cbr{R_1R_3 + R_2R_1}
            + \ele_2\cbr{- R_1R_3}
            - \ele_3R_1R_2
       }{R_1 \cdot \cbr{R_1R_3 + R_2R_3 + R_2R_1}}
        =
        \frac{
            \ele_1\cbr{R_3 + R_2} - \ele_2R_3 - \ele_3R_2
       }{R_1R_3 + R_2R_3 + R_2R_1}
        = \\ &=
        \frac{
            (\ele_1 - \ele_3)R_2 + (\ele_1 - \ele_2)R_3
       }{R_1R_3 + R_2R_3 + R_2R_1}
        =
        \frac{
            \cfrac{\ele_1 - \ele_3}{ R_3 } + \cfrac{\ele_1 - \ele_2}{ R_2 }
       }{\cfrac{ R_1 }{ R_2 } + 1 + \cfrac{ R_1 }{ R_3 }}
        =
        \frac{
            \cfrac{5\,\text{В} - 2\,\text{В}}{ 15\,\text{Ом} } + \cfrac{5\,\text{В} - 3\,\text{В}}{ 5\,\text{Ом} }
       }{\cfrac{ 2\,\text{Ом} }{ 5\,\text{Ом} } + 1 + \cfrac{ 2\,\text{Ом} }{ 15\,\text{Ом} }}
        = \frac9{23}\units{А} \approx 0{,}39\,\text{А}.
        \\
    U_1
        &=
        \eli_1R_1
        =
        \frac{
            \cfrac{\ele_1 - \ele_3}{ R_3 } + \cfrac{\ele_1 - \ele_2}{ R_2 }
       }{\cfrac{ R_1 }{ R_2 } + 1 + \cfrac{ R_1 }{ R_3 }} \cdot R_1
        =
        \frac9{23}\units{А} \cdot 2\,\text{Ом} = \frac{18}{23}\units{В} \approx 0{,}78\,\text{В}.
    \end{align*}

    Если вы проделали все эти вычисления выше вместе со мной, то
    \begin{itemize}
        \item вы совершили ошибку, выбрав неверный путь решения:
        слишком длинное решение, очень легко ошибиться в индексах, дробях, знаках или потерять какой-то множитель,
        \item можно было выразить из исходной системы другие токи и получить сразу нажный вам,
        а не какой-то 2-й,
        \item можно было сэкономить: все три резистора и ЭДС соединены одинаково,
        поэтому ответ для 1-го резистора должен отличаться лишь перестановкой индексов (этот факт крайне полезен при проверке ответа, у нас всё сошлось),
        я специально подгонял выражение для $\eli_1$ к этому виду, вынося за скобки и преобразуя дробь,
        \item вы молодец, потому что не побоялись и получили верный ответ грамотным способом,
    \end{itemize}
    так что переходим к третьему резистору.
    Будет похоже, но кого это когда останавливало...

    \begin{align*}
    \eli_3
        &=  \frac{\eli_2R_2 - \ele_2 + \ele_3}{ R_3 }
        =
        \cfrac{
            \cfrac{
                (\ele_2 - \ele_1)R_3 + (\ele_2 - \ele_3)R_1
           }{
                R_1R_3 + R_2R_3 + R_2R_1
           } \cdot R_2 - \ele_2 + \ele_3}{ R_3 }
        = \\ &=
        \frac{
            \ele_2R_3R_2 - \ele_1R_3R_2 + \ele_2R_1R_2 - \ele_3R_1R_2
            - \ele_2R_1R_3 - \ele_2R_2R_3 - \ele_2R_2R_1
            + \ele_3R_1R_3 + \ele_3R_2R_3 + \ele_3R_2R_1
       }{\cbr{R_1R_3 + R_2R_3 + R_2R_1} \cdot R_3}
        = \\ &=
        \frac{
            - \ele_1R_3R_2 - \ele_2R_1R_3 + \ele_3R_1R_3 + \ele_3R_2R_3
       }{\cbr{R_1R_3 + R_2R_3 + R_2R_1} \cdot R_3}
        =
        \frac{
            - \ele_1R_2 - \ele_2R_1 + \ele_3R_1 + \ele_3R_2
       }{R_1R_3 + R_2R_3 + R_2R_1}
        = \\ &=
        \frac{
            R_1(\ele_3 - \ele_2) + R_2(\ele_3 - \ele_1)
       }{R_1R_3 + R_2R_3 + R_2R_1}
        =
        \frac{
            \cfrac{\ele_3 - \ele_2}{ R_2 } + \cfrac{\ele_3 - \ele_1}{ R_1 }
       }{\cfrac{ R_3 }{ R_2 } + \cfrac{ R_3 }{ R_1 } + 1}
        =
        \frac{
            \cfrac{2\,\text{В} - 3\,\text{В}}{ 5\,\text{Ом} } + \cfrac{2\,\text{В} - 5\,\text{В}}{ 2\,\text{Ом} }
       }{\cfrac{ 15\,\text{Ом} }{ 5\,\text{Ом} } + \cfrac{ 15\,\text{Ом} }{ 2\,\text{Ом} } + 1}
        = -\frac{17}{115}\units{А} \approx -0{,}15000\,\text{А}.
        \\
    U_3
        &=
        \eli_3R_3
        =
        \frac{
            \cfrac{\ele_3 - \ele_2}{ R_2 } + \cfrac{\ele_3 - \ele_1}{ R_1 }
       }{\cfrac{ R_3 }{ R_2 } + \cfrac{ R_3 }{ R_1 } + 1} \cdot R_3
        =
        -\frac{17}{115}\units{А} \cdot 15\,\text{Ом} = -\frac{51}{23}\units{В} \approx -2{,}220\,\text{В}.
    \end{align*}

    Положительные ответы говорят, что мы угадали на рисунке направление тока (тут нет нашей заслуги, повезло),
    отрицательные — что не угадали (и в этом нет ошибки), и ток течёт в противоположную сторону.
    Напомним, что направление тока — это направление движения положительных зарядов,
    а в металлах носители заряда — электроны, которые заряжены отрицательно.
}

\variantsplitter

\addpersonalvariant{Егор Свистушкин}

\tasknumber{1}%
\task{%
    Определите ток $\eli_2$, протекающий через резистор $R_2$ (см.
    рис.),
    направление этого тока и разность потенциалов $U_2$ на этом резисторе,
    если $R_1 = 4\,\text{Ом}$, $R_2 = 5\,\text{Ом}$, $R_3 = 10\,\text{Ом}$, $\ele_1 = 5\,\text{В}$, $\ele_2 = 3\,\text{В}$, $\ele_3 = 2\,\text{В}$.
    Внутренним сопротивлением всех трёх ЭДС пренебречь.
    Ответы получите в виде несократимых дробей, а также определите приближённые значения.

    \begin{tikzpicture}[circuit ee IEC, thick]
        \foreach \contact/\x in {1/0, 2/3, 3/6}
        {
            \node [contact] (top contact \contact) at (\x, 0) {};
            \node [contact] (bottom contact \contact) at (\x, 4) {};
       }
        \draw  (bottom contact 1) -- (bottom contact 2) -- (bottom contact 3);
        \draw  (top contact 1) -- (top contact 2) -- (top contact 3);
        \draw  (bottom contact 1) to [resistor={near start, info=$R_1$}, battery={near end, info=$\ele_1$}] (top contact 1);
        \draw  (bottom contact 2) to [resistor={near start, info=$R_2$}, battery={near end, info=$\ele_2$}] (top contact 2);
        \draw  (bottom contact 3) to [resistor={near start, info=$R_3$}, battery={near end, info=$\ele_3$}] (top contact 3);
    \end{tikzpicture}
}
\answer{%
    План:
    \begin{itemize}
        \item отметим на рисунке произвольно направления токов (если получим отрицательный ответ, значит не угадали направление и только),
        \item выберем и обозначим на рисунке контуры (здесь всего 3, значит будет нужно $3-1=2$), для них запишем законы Кирхгофа,
        \item выберем и выделим на рисунке нетривиальные узлы (здесь всего 2, значит будет нужно $2-1=1$), для него запишем закон Кирхгофа,
        \item попытаемся решить получившуюся систему.
        В конкретном решении мы пытались первым делом найти $\eli_2$, но, возможно, в вашем варианте будет быстрее решать систему в другом порядке.
        Мы всё же проделаем всё в лоб, подробно и целиком.
    \end{itemize}


    \begin{tikzpicture}[circuit ee IEC, thick]
        \foreach \contact/\x in {1/0, 2/3, 3/6}
        {
            \node [contact] (top contact \contact) at (\x, 0) {};
            \node [contact] (bottom contact \contact) at (\x, 4) {};
       }
        \draw  (bottom contact 1) -- (bottom contact 2) -- (bottom contact 3);
        \draw  (top contact 1) -- (top contact 2) -- (top contact 3);
        \draw  (bottom contact 1) to [resistor={near start, info=$R_1$}, current direction'={midway, info=$\eli_1$}, battery={near end, info=$\ele_1$}] (top contact 1);
        \draw  (bottom contact 2) to [resistor={near start, info=$R_2$}, current direction'={midway, info=$\eli_2$}, battery={near end, info=$\ele_2$}] (top contact 2);
        \draw  (bottom contact 3) to [resistor={near start, info=$R_3$}, current direction'={midway, info=$\eli_3$}, battery={near end, info=$\ele_3$}] (top contact 3);
        \draw [-{Latex},color=red] (1.2, 2.5) arc [start angle = 135, end angle = -160, radius = 0.6];
        \draw [-{Latex},color=blue] (4.2, 2.5) arc [start angle = 135, end angle = -160, radius = 0.6];
        \node [contact,color=green!71!black] (bottomc) at (bottom contact 2) {};
    \end{tikzpicture}

    \begin{align*}
        &\begin{cases}
            {\color{red} \eli_1R_1 - \eli_2R_2 = \ele_1 - \ele_2}, \\
            {\color{blue} \eli_2R_2 - \eli_3R_3 = \ele_2 - \ele_3}, \\
            {\color{green!71!black} \eli_1 + \eli_2 + \eli_3 = 0};
        \end{cases}
        \qquad \implies \qquad
        \begin{cases}
            \eli_1 = \frac{\ele_1 - \ele_2 + \eli_2R_2}{R_1}, \\
            \eli_3 = \frac{\eli_2R_2 - \ele_2 + \ele_3}{R_3}, \\
            \eli_1 + \eli_2 + \eli_3 = 0, \\
        \end{cases} \implies \\
        \implies
            &\eli_2 + \frac{\ele_1 - \ele_2 + \eli_2R_2}{R_1} + \frac{\eli_2R_2 - \ele_2 + \ele_3}{R_3} = 0, \\
        &   \eli_2\cbr{1 + \frac{ R_2 }{ R_1 } + \frac{ R_2 }{ R_3 }} + \frac{\ele_1 - \ele_2}{ R_1 } + \frac{\ele_3 - \ele_2}{ R_3 } = 0, \\
        &   \eli_2 = \cfrac{\cfrac{\ele_2 - \ele_1}{ R_1 } + \cfrac{\ele_2 - \ele_3}{ R_3 }}{1 + \cfrac{ R_2 }{ R_1 } + \cfrac{ R_2 }{ R_3 }}
            = \cfrac{\cfrac{3\,\text{В} - 5\,\text{В}}{ 4\,\text{Ом} } + \cfrac{3\,\text{В} - 2\,\text{В}}{ 10\,\text{Ом} }}{1 + \cfrac{ 5\,\text{Ом} }{ 4\,\text{Ом} } + \cfrac{ 5\,\text{Ом} }{ 10\,\text{Ом} }}
            = -\frac8{55}\units{А} \approx -0{,}15000\,\text{А}, \\
        &   U_2 = \eli_2R_2 = \cfrac{\cfrac{\ele_2 - \ele_1}{ R_1 } + \cfrac{\ele_2 - \ele_3}{ R_3 }}{1 + \cfrac{ R_2 }{ R_1 } + \cfrac{ R_2 }{ R_3 }} \cdot R_2
            = \cfrac{\cfrac{3\,\text{В} - 5\,\text{В}}{ 4\,\text{Ом} } + \cfrac{3\,\text{В} - 2\,\text{В}}{ 10\,\text{Ом} }}{1 + \cfrac{ 5\,\text{Ом} }{ 4\,\text{Ом} } + \cfrac{ 5\,\text{Ом} }{ 10\,\text{Ом} }} \cdot 5\,\text{Ом}
            = -\frac8{55}\units{А} \cdot 5\,\text{Ом} = -\frac8{11}\units{В} \approx -0{,}7300\,\text{В}.
    \end{align*}

    Одну пару силы тока и напряжения получили.
    Для некоторых вариантов это уже ответ, но не у всех.
    Для упрощения записи преобразуем (чтобы избавитсья от 4-этажной дроби) и подставим в уже полученные уравнения:

    \begin{align*}
    \eli_2
        &=
        \frac{\frac{\ele_2 - \ele_1}{ R_1 } + \frac{\ele_2 - \ele_3}{ R_3 }}{1 + \frac{ R_2 }{ R_1 } + \frac{ R_2 }{ R_3 }}
        =
        \frac{(\ele_2 - \ele_1)R_3 + (\ele_2 - \ele_3)R_1}{R_1R_3 + R_2R_3 + R_2R_1},
        \\
    \eli_1
        &=  \frac{\ele_1 - \ele_2 + \eli_2R_2}{R_1}
        =   \frac{\ele_1 - \ele_2 + \cfrac{(\ele_2 - \ele_1)R_3 + (\ele_2 - \ele_3)R_1}{R_1R_3 + R_2R_3 + R_2R_1} \cdot R_2}{R_1} = \\
        &=  \frac{
            \ele_1R_1R_3 + \ele_1R_2R_3 + \ele_1R_2R_1
            - \ele_2R_1R_3 - \ele_2R_2R_3 - \ele_2R_2R_1
            + \ele_2R_3R_2 - \ele_1R_3R_2 + \ele_2R_1R_2 - \ele_3R_1R_2
       }{R_1 \cdot \cbr{R_1R_3 + R_2R_3 + R_2R_1}}
        = \\ &=
        \frac{
            \ele_1\cbr{R_1R_3 + R_2R_3 + R_2R_1 - R_3R_2}
            + \ele_2\cbr{- R_1R_3 - R_2R_3 - R_2R_1 + R_3R_2 + R_1R_2}
            - \ele_3R_1R_2
       }{R_1 \cdot \cbr{R_1R_3 + R_2R_3 + R_2R_1}}
        = \\ &=
        \frac{
            \ele_1\cbr{R_1R_3 + R_2R_1}
            + \ele_2\cbr{- R_1R_3}
            - \ele_3R_1R_2
       }{R_1 \cdot \cbr{R_1R_3 + R_2R_3 + R_2R_1}}
        =
        \frac{
            \ele_1\cbr{R_3 + R_2} - \ele_2R_3 - \ele_3R_2
       }{R_1R_3 + R_2R_3 + R_2R_1}
        = \\ &=
        \frac{
            (\ele_1 - \ele_3)R_2 + (\ele_1 - \ele_2)R_3
       }{R_1R_3 + R_2R_3 + R_2R_1}
        =
        \frac{
            \cfrac{\ele_1 - \ele_3}{ R_3 } + \cfrac{\ele_1 - \ele_2}{ R_2 }
       }{\cfrac{ R_1 }{ R_2 } + 1 + \cfrac{ R_1 }{ R_3 }}
        =
        \frac{
            \cfrac{5\,\text{В} - 2\,\text{В}}{ 10\,\text{Ом} } + \cfrac{5\,\text{В} - 3\,\text{В}}{ 5\,\text{Ом} }
       }{\cfrac{ 4\,\text{Ом} }{ 5\,\text{Ом} } + 1 + \cfrac{ 4\,\text{Ом} }{ 10\,\text{Ом} }}
        = \frac7{22}\units{А} \approx 0{,}32\,\text{А}.
        \\
    U_1
        &=
        \eli_1R_1
        =
        \frac{
            \cfrac{\ele_1 - \ele_3}{ R_3 } + \cfrac{\ele_1 - \ele_2}{ R_2 }
       }{\cfrac{ R_1 }{ R_2 } + 1 + \cfrac{ R_1 }{ R_3 }} \cdot R_1
        =
        \frac7{22}\units{А} \cdot 4\,\text{Ом} = \frac{14}{11}\units{В} \approx 1{,}27\,\text{В}.
    \end{align*}

    Если вы проделали все эти вычисления выше вместе со мной, то
    \begin{itemize}
        \item вы совершили ошибку, выбрав неверный путь решения:
        слишком длинное решение, очень легко ошибиться в индексах, дробях, знаках или потерять какой-то множитель,
        \item можно было выразить из исходной системы другие токи и получить сразу нажный вам,
        а не какой-то 2-й,
        \item можно было сэкономить: все три резистора и ЭДС соединены одинаково,
        поэтому ответ для 1-го резистора должен отличаться лишь перестановкой индексов (этот факт крайне полезен при проверке ответа, у нас всё сошлось),
        я специально подгонял выражение для $\eli_1$ к этому виду, вынося за скобки и преобразуя дробь,
        \item вы молодец, потому что не побоялись и получили верный ответ грамотным способом,
    \end{itemize}
    так что переходим к третьему резистору.
    Будет похоже, но кого это когда останавливало...

    \begin{align*}
    \eli_3
        &=  \frac{\eli_2R_2 - \ele_2 + \ele_3}{ R_3 }
        =
        \cfrac{
            \cfrac{
                (\ele_2 - \ele_1)R_3 + (\ele_2 - \ele_3)R_1
           }{
                R_1R_3 + R_2R_3 + R_2R_1
           } \cdot R_2 - \ele_2 + \ele_3}{ R_3 }
        = \\ &=
        \frac{
            \ele_2R_3R_2 - \ele_1R_3R_2 + \ele_2R_1R_2 - \ele_3R_1R_2
            - \ele_2R_1R_3 - \ele_2R_2R_3 - \ele_2R_2R_1
            + \ele_3R_1R_3 + \ele_3R_2R_3 + \ele_3R_2R_1
       }{\cbr{R_1R_3 + R_2R_3 + R_2R_1} \cdot R_3}
        = \\ &=
        \frac{
            - \ele_1R_3R_2 - \ele_2R_1R_3 + \ele_3R_1R_3 + \ele_3R_2R_3
       }{\cbr{R_1R_3 + R_2R_3 + R_2R_1} \cdot R_3}
        =
        \frac{
            - \ele_1R_2 - \ele_2R_1 + \ele_3R_1 + \ele_3R_2
       }{R_1R_3 + R_2R_3 + R_2R_1}
        = \\ &=
        \frac{
            R_1(\ele_3 - \ele_2) + R_2(\ele_3 - \ele_1)
       }{R_1R_3 + R_2R_3 + R_2R_1}
        =
        \frac{
            \cfrac{\ele_3 - \ele_2}{ R_2 } + \cfrac{\ele_3 - \ele_1}{ R_1 }
       }{\cfrac{ R_3 }{ R_2 } + \cfrac{ R_3 }{ R_1 } + 1}
        =
        \frac{
            \cfrac{2\,\text{В} - 3\,\text{В}}{ 5\,\text{Ом} } + \cfrac{2\,\text{В} - 5\,\text{В}}{ 4\,\text{Ом} }
       }{\cfrac{ 10\,\text{Ом} }{ 5\,\text{Ом} } + \cfrac{ 10\,\text{Ом} }{ 4\,\text{Ом} } + 1}
        = -\frac{19}{110}\units{А} \approx -0{,}17000\,\text{А}.
        \\
    U_3
        &=
        \eli_3R_3
        =
        \frac{
            \cfrac{\ele_3 - \ele_2}{ R_2 } + \cfrac{\ele_3 - \ele_1}{ R_1 }
       }{\cfrac{ R_3 }{ R_2 } + \cfrac{ R_3 }{ R_1 } + 1} \cdot R_3
        =
        -\frac{19}{110}\units{А} \cdot 10\,\text{Ом} = -\frac{19}{11}\units{В} \approx -1{,}7300\,\text{В}.
    \end{align*}

    Положительные ответы говорят, что мы угадали на рисунке направление тока (тут нет нашей заслуги, повезло),
    отрицательные — что не угадали (и в этом нет ошибки), и ток течёт в противоположную сторону.
    Напомним, что направление тока — это направление движения положительных зарядов,
    а в металлах носители заряда — электроны, которые заряжены отрицательно.
}

\variantsplitter

\addpersonalvariant{Дмитрий Соколов}

\tasknumber{1}%
\task{%
    Определите ток $\eli_1$, протекающий через резистор $R_1$ (см.
    рис.),
    направление этого тока и разность потенциалов $U_1$ на этом резисторе,
    если $R_1 = 3\,\text{Ом}$, $R_2 = 8\,\text{Ом}$, $R_3 = 10\,\text{Ом}$, $\ele_1 = 4\,\text{В}$, $\ele_2 = 3\,\text{В}$, $\ele_3 = 2\,\text{В}$.
    Внутренним сопротивлением всех трёх ЭДС пренебречь.
    Ответы получите в виде несократимых дробей, а также определите приближённые значения.

    \begin{tikzpicture}[circuit ee IEC, thick]
        \foreach \contact/\x in {1/0, 2/3, 3/6}
        {
            \node [contact] (top contact \contact) at (\x, 0) {};
            \node [contact] (bottom contact \contact) at (\x, 4) {};
       }
        \draw  (bottom contact 1) -- (bottom contact 2) -- (bottom contact 3);
        \draw  (top contact 1) -- (top contact 2) -- (top contact 3);
        \draw  (bottom contact 1) to [resistor={near start, info=$R_1$}, battery={near end, info=$\ele_1$}] (top contact 1);
        \draw  (bottom contact 2) to [resistor={near start, info=$R_2$}, battery={near end, info=$\ele_2$}] (top contact 2);
        \draw  (bottom contact 3) to [resistor={near start, info=$R_3$}, battery={near end, info=$\ele_3$}] (top contact 3);
    \end{tikzpicture}
}
\answer{%
    План:
    \begin{itemize}
        \item отметим на рисунке произвольно направления токов (если получим отрицательный ответ, значит не угадали направление и только),
        \item выберем и обозначим на рисунке контуры (здесь всего 3, значит будет нужно $3-1=2$), для них запишем законы Кирхгофа,
        \item выберем и выделим на рисунке нетривиальные узлы (здесь всего 2, значит будет нужно $2-1=1$), для него запишем закон Кирхгофа,
        \item попытаемся решить получившуюся систему.
        В конкретном решении мы пытались первым делом найти $\eli_2$, но, возможно, в вашем варианте будет быстрее решать систему в другом порядке.
        Мы всё же проделаем всё в лоб, подробно и целиком.
    \end{itemize}


    \begin{tikzpicture}[circuit ee IEC, thick]
        \foreach \contact/\x in {1/0, 2/3, 3/6}
        {
            \node [contact] (top contact \contact) at (\x, 0) {};
            \node [contact] (bottom contact \contact) at (\x, 4) {};
       }
        \draw  (bottom contact 1) -- (bottom contact 2) -- (bottom contact 3);
        \draw  (top contact 1) -- (top contact 2) -- (top contact 3);
        \draw  (bottom contact 1) to [resistor={near start, info=$R_1$}, current direction'={midway, info=$\eli_1$}, battery={near end, info=$\ele_1$}] (top contact 1);
        \draw  (bottom contact 2) to [resistor={near start, info=$R_2$}, current direction'={midway, info=$\eli_2$}, battery={near end, info=$\ele_2$}] (top contact 2);
        \draw  (bottom contact 3) to [resistor={near start, info=$R_3$}, current direction'={midway, info=$\eli_3$}, battery={near end, info=$\ele_3$}] (top contact 3);
        \draw [-{Latex},color=red] (1.2, 2.5) arc [start angle = 135, end angle = -160, radius = 0.6];
        \draw [-{Latex},color=blue] (4.2, 2.5) arc [start angle = 135, end angle = -160, radius = 0.6];
        \node [contact,color=green!71!black] (bottomc) at (bottom contact 2) {};
    \end{tikzpicture}

    \begin{align*}
        &\begin{cases}
            {\color{red} \eli_1R_1 - \eli_2R_2 = \ele_1 - \ele_2}, \\
            {\color{blue} \eli_2R_2 - \eli_3R_3 = \ele_2 - \ele_3}, \\
            {\color{green!71!black} \eli_1 + \eli_2 + \eli_3 = 0};
        \end{cases}
        \qquad \implies \qquad
        \begin{cases}
            \eli_1 = \frac{\ele_1 - \ele_2 + \eli_2R_2}{R_1}, \\
            \eli_3 = \frac{\eli_2R_2 - \ele_2 + \ele_3}{R_3}, \\
            \eli_1 + \eli_2 + \eli_3 = 0, \\
        \end{cases} \implies \\
        \implies
            &\eli_2 + \frac{\ele_1 - \ele_2 + \eli_2R_2}{R_1} + \frac{\eli_2R_2 - \ele_2 + \ele_3}{R_3} = 0, \\
        &   \eli_2\cbr{1 + \frac{ R_2 }{ R_1 } + \frac{ R_2 }{ R_3 }} + \frac{\ele_1 - \ele_2}{ R_1 } + \frac{\ele_3 - \ele_2}{ R_3 } = 0, \\
        &   \eli_2 = \cfrac{\cfrac{\ele_2 - \ele_1}{ R_1 } + \cfrac{\ele_2 - \ele_3}{ R_3 }}{1 + \cfrac{ R_2 }{ R_1 } + \cfrac{ R_2 }{ R_3 }}
            = \cfrac{\cfrac{3\,\text{В} - 4\,\text{В}}{ 3\,\text{Ом} } + \cfrac{3\,\text{В} - 2\,\text{В}}{ 10\,\text{Ом} }}{1 + \cfrac{ 8\,\text{Ом} }{ 3\,\text{Ом} } + \cfrac{ 8\,\text{Ом} }{ 10\,\text{Ом} }}
            = -\frac7{134}\units{А} \approx -0{,}05000\,\text{А}, \\
        &   U_2 = \eli_2R_2 = \cfrac{\cfrac{\ele_2 - \ele_1}{ R_1 } + \cfrac{\ele_2 - \ele_3}{ R_3 }}{1 + \cfrac{ R_2 }{ R_1 } + \cfrac{ R_2 }{ R_3 }} \cdot R_2
            = \cfrac{\cfrac{3\,\text{В} - 4\,\text{В}}{ 3\,\text{Ом} } + \cfrac{3\,\text{В} - 2\,\text{В}}{ 10\,\text{Ом} }}{1 + \cfrac{ 8\,\text{Ом} }{ 3\,\text{Ом} } + \cfrac{ 8\,\text{Ом} }{ 10\,\text{Ом} }} \cdot 8\,\text{Ом}
            = -\frac7{134}\units{А} \cdot 8\,\text{Ом} = -\frac{28}{67}\units{В} \approx -0{,}4200\,\text{В}.
    \end{align*}

    Одну пару силы тока и напряжения получили.
    Для некоторых вариантов это уже ответ, но не у всех.
    Для упрощения записи преобразуем (чтобы избавитсья от 4-этажной дроби) и подставим в уже полученные уравнения:

    \begin{align*}
    \eli_2
        &=
        \frac{\frac{\ele_2 - \ele_1}{ R_1 } + \frac{\ele_2 - \ele_3}{ R_3 }}{1 + \frac{ R_2 }{ R_1 } + \frac{ R_2 }{ R_3 }}
        =
        \frac{(\ele_2 - \ele_1)R_3 + (\ele_2 - \ele_3)R_1}{R_1R_3 + R_2R_3 + R_2R_1},
        \\
    \eli_1
        &=  \frac{\ele_1 - \ele_2 + \eli_2R_2}{R_1}
        =   \frac{\ele_1 - \ele_2 + \cfrac{(\ele_2 - \ele_1)R_3 + (\ele_2 - \ele_3)R_1}{R_1R_3 + R_2R_3 + R_2R_1} \cdot R_2}{R_1} = \\
        &=  \frac{
            \ele_1R_1R_3 + \ele_1R_2R_3 + \ele_1R_2R_1
            - \ele_2R_1R_3 - \ele_2R_2R_3 - \ele_2R_2R_1
            + \ele_2R_3R_2 - \ele_1R_3R_2 + \ele_2R_1R_2 - \ele_3R_1R_2
       }{R_1 \cdot \cbr{R_1R_3 + R_2R_3 + R_2R_1}}
        = \\ &=
        \frac{
            \ele_1\cbr{R_1R_3 + R_2R_3 + R_2R_1 - R_3R_2}
            + \ele_2\cbr{- R_1R_3 - R_2R_3 - R_2R_1 + R_3R_2 + R_1R_2}
            - \ele_3R_1R_2
       }{R_1 \cdot \cbr{R_1R_3 + R_2R_3 + R_2R_1}}
        = \\ &=
        \frac{
            \ele_1\cbr{R_1R_3 + R_2R_1}
            + \ele_2\cbr{- R_1R_3}
            - \ele_3R_1R_2
       }{R_1 \cdot \cbr{R_1R_3 + R_2R_3 + R_2R_1}}
        =
        \frac{
            \ele_1\cbr{R_3 + R_2} - \ele_2R_3 - \ele_3R_2
       }{R_1R_3 + R_2R_3 + R_2R_1}
        = \\ &=
        \frac{
            (\ele_1 - \ele_3)R_2 + (\ele_1 - \ele_2)R_3
       }{R_1R_3 + R_2R_3 + R_2R_1}
        =
        \frac{
            \cfrac{\ele_1 - \ele_3}{ R_3 } + \cfrac{\ele_1 - \ele_2}{ R_2 }
       }{\cfrac{ R_1 }{ R_2 } + 1 + \cfrac{ R_1 }{ R_3 }}
        =
        \frac{
            \cfrac{4\,\text{В} - 2\,\text{В}}{ 10\,\text{Ом} } + \cfrac{4\,\text{В} - 3\,\text{В}}{ 8\,\text{Ом} }
       }{\cfrac{ 3\,\text{Ом} }{ 8\,\text{Ом} } + 1 + \cfrac{ 3\,\text{Ом} }{ 10\,\text{Ом} }}
        = \frac{13}{67}\units{А} \approx 0{,}19\,\text{А}.
        \\
    U_1
        &=
        \eli_1R_1
        =
        \frac{
            \cfrac{\ele_1 - \ele_3}{ R_3 } + \cfrac{\ele_1 - \ele_2}{ R_2 }
       }{\cfrac{ R_1 }{ R_2 } + 1 + \cfrac{ R_1 }{ R_3 }} \cdot R_1
        =
        \frac{13}{67}\units{А} \cdot 3\,\text{Ом} = \frac{39}{67}\units{В} \approx 0{,}58\,\text{В}.
    \end{align*}

    Если вы проделали все эти вычисления выше вместе со мной, то
    \begin{itemize}
        \item вы совершили ошибку, выбрав неверный путь решения:
        слишком длинное решение, очень легко ошибиться в индексах, дробях, знаках или потерять какой-то множитель,
        \item можно было выразить из исходной системы другие токи и получить сразу нажный вам,
        а не какой-то 2-й,
        \item можно было сэкономить: все три резистора и ЭДС соединены одинаково,
        поэтому ответ для 1-го резистора должен отличаться лишь перестановкой индексов (этот факт крайне полезен при проверке ответа, у нас всё сошлось),
        я специально подгонял выражение для $\eli_1$ к этому виду, вынося за скобки и преобразуя дробь,
        \item вы молодец, потому что не побоялись и получили верный ответ грамотным способом,
    \end{itemize}
    так что переходим к третьему резистору.
    Будет похоже, но кого это когда останавливало...

    \begin{align*}
    \eli_3
        &=  \frac{\eli_2R_2 - \ele_2 + \ele_3}{ R_3 }
        =
        \cfrac{
            \cfrac{
                (\ele_2 - \ele_1)R_3 + (\ele_2 - \ele_3)R_1
           }{
                R_1R_3 + R_2R_3 + R_2R_1
           } \cdot R_2 - \ele_2 + \ele_3}{ R_3 }
        = \\ &=
        \frac{
            \ele_2R_3R_2 - \ele_1R_3R_2 + \ele_2R_1R_2 - \ele_3R_1R_2
            - \ele_2R_1R_3 - \ele_2R_2R_3 - \ele_2R_2R_1
            + \ele_3R_1R_3 + \ele_3R_2R_3 + \ele_3R_2R_1
       }{\cbr{R_1R_3 + R_2R_3 + R_2R_1} \cdot R_3}
        = \\ &=
        \frac{
            - \ele_1R_3R_2 - \ele_2R_1R_3 + \ele_3R_1R_3 + \ele_3R_2R_3
       }{\cbr{R_1R_3 + R_2R_3 + R_2R_1} \cdot R_3}
        =
        \frac{
            - \ele_1R_2 - \ele_2R_1 + \ele_3R_1 + \ele_3R_2
       }{R_1R_3 + R_2R_3 + R_2R_1}
        = \\ &=
        \frac{
            R_1(\ele_3 - \ele_2) + R_2(\ele_3 - \ele_1)
       }{R_1R_3 + R_2R_3 + R_2R_1}
        =
        \frac{
            \cfrac{\ele_3 - \ele_2}{ R_2 } + \cfrac{\ele_3 - \ele_1}{ R_1 }
       }{\cfrac{ R_3 }{ R_2 } + \cfrac{ R_3 }{ R_1 } + 1}
        =
        \frac{
            \cfrac{2\,\text{В} - 3\,\text{В}}{ 8\,\text{Ом} } + \cfrac{2\,\text{В} - 4\,\text{В}}{ 3\,\text{Ом} }
       }{\cfrac{ 10\,\text{Ом} }{ 8\,\text{Ом} } + \cfrac{ 10\,\text{Ом} }{ 3\,\text{Ом} } + 1}
        = -\frac{19}{134}\units{А} \approx -0{,}14000\,\text{А}.
        \\
    U_3
        &=
        \eli_3R_3
        =
        \frac{
            \cfrac{\ele_3 - \ele_2}{ R_2 } + \cfrac{\ele_3 - \ele_1}{ R_1 }
       }{\cfrac{ R_3 }{ R_2 } + \cfrac{ R_3 }{ R_1 } + 1} \cdot R_3
        =
        -\frac{19}{134}\units{А} \cdot 10\,\text{Ом} = -\frac{95}{67}\units{В} \approx -1{,}4200\,\text{В}.
    \end{align*}

    Положительные ответы говорят, что мы угадали на рисунке направление тока (тут нет нашей заслуги, повезло),
    отрицательные — что не угадали (и в этом нет ошибки), и ток течёт в противоположную сторону.
    Напомним, что направление тока — это направление движения положительных зарядов,
    а в металлах носители заряда — электроны, которые заряжены отрицательно.
}

\variantsplitter

\addpersonalvariant{Арсений Трофимов}

\tasknumber{1}%
\task{%
    Определите ток $\eli_1$, протекающий через резистор $R_1$ (см.
    рис.),
    направление этого тока и разность потенциалов $U_1$ на этом резисторе,
    если $R_1 = 4\,\text{Ом}$, $R_2 = 6\,\text{Ом}$, $R_3 = 15\,\text{Ом}$, $\ele_1 = 4\,\text{В}$, $\ele_2 = 6\,\text{В}$, $\ele_3 = 2\,\text{В}$.
    Внутренним сопротивлением всех трёх ЭДС пренебречь.
    Ответы получите в виде несократимых дробей, а также определите приближённые значения.

    \begin{tikzpicture}[circuit ee IEC, thick]
        \foreach \contact/\x in {1/0, 2/3, 3/6}
        {
            \node [contact] (top contact \contact) at (\x, 0) {};
            \node [contact] (bottom contact \contact) at (\x, 4) {};
       }
        \draw  (bottom contact 1) -- (bottom contact 2) -- (bottom contact 3);
        \draw  (top contact 1) -- (top contact 2) -- (top contact 3);
        \draw  (bottom contact 1) to [resistor={near start, info=$R_1$}, battery={near end, info=$\ele_1$}] (top contact 1);
        \draw  (bottom contact 2) to [resistor={near start, info=$R_2$}, battery={near end, info=$\ele_2$}] (top contact 2);
        \draw  (bottom contact 3) to [resistor={near start, info=$R_3$}, battery={near end, info=$\ele_3$}] (top contact 3);
    \end{tikzpicture}
}
\answer{%
    План:
    \begin{itemize}
        \item отметим на рисунке произвольно направления токов (если получим отрицательный ответ, значит не угадали направление и только),
        \item выберем и обозначим на рисунке контуры (здесь всего 3, значит будет нужно $3-1=2$), для них запишем законы Кирхгофа,
        \item выберем и выделим на рисунке нетривиальные узлы (здесь всего 2, значит будет нужно $2-1=1$), для него запишем закон Кирхгофа,
        \item попытаемся решить получившуюся систему.
        В конкретном решении мы пытались первым делом найти $\eli_2$, но, возможно, в вашем варианте будет быстрее решать систему в другом порядке.
        Мы всё же проделаем всё в лоб, подробно и целиком.
    \end{itemize}


    \begin{tikzpicture}[circuit ee IEC, thick]
        \foreach \contact/\x in {1/0, 2/3, 3/6}
        {
            \node [contact] (top contact \contact) at (\x, 0) {};
            \node [contact] (bottom contact \contact) at (\x, 4) {};
       }
        \draw  (bottom contact 1) -- (bottom contact 2) -- (bottom contact 3);
        \draw  (top contact 1) -- (top contact 2) -- (top contact 3);
        \draw  (bottom contact 1) to [resistor={near start, info=$R_1$}, current direction'={midway, info=$\eli_1$}, battery={near end, info=$\ele_1$}] (top contact 1);
        \draw  (bottom contact 2) to [resistor={near start, info=$R_2$}, current direction'={midway, info=$\eli_2$}, battery={near end, info=$\ele_2$}] (top contact 2);
        \draw  (bottom contact 3) to [resistor={near start, info=$R_3$}, current direction'={midway, info=$\eli_3$}, battery={near end, info=$\ele_3$}] (top contact 3);
        \draw [-{Latex},color=red] (1.2, 2.5) arc [start angle = 135, end angle = -160, radius = 0.6];
        \draw [-{Latex},color=blue] (4.2, 2.5) arc [start angle = 135, end angle = -160, radius = 0.6];
        \node [contact,color=green!71!black] (bottomc) at (bottom contact 2) {};
    \end{tikzpicture}

    \begin{align*}
        &\begin{cases}
            {\color{red} \eli_1R_1 - \eli_2R_2 = \ele_1 - \ele_2}, \\
            {\color{blue} \eli_2R_2 - \eli_3R_3 = \ele_2 - \ele_3}, \\
            {\color{green!71!black} \eli_1 + \eli_2 + \eli_3 = 0};
        \end{cases}
        \qquad \implies \qquad
        \begin{cases}
            \eli_1 = \frac{\ele_1 - \ele_2 + \eli_2R_2}{R_1}, \\
            \eli_3 = \frac{\eli_2R_2 - \ele_2 + \ele_3}{R_3}, \\
            \eli_1 + \eli_2 + \eli_3 = 0, \\
        \end{cases} \implies \\
        \implies
            &\eli_2 + \frac{\ele_1 - \ele_2 + \eli_2R_2}{R_1} + \frac{\eli_2R_2 - \ele_2 + \ele_3}{R_3} = 0, \\
        &   \eli_2\cbr{1 + \frac{ R_2 }{ R_1 } + \frac{ R_2 }{ R_3 }} + \frac{\ele_1 - \ele_2}{ R_1 } + \frac{\ele_3 - \ele_2}{ R_3 } = 0, \\
        &   \eli_2 = \cfrac{\cfrac{\ele_2 - \ele_1}{ R_1 } + \cfrac{\ele_2 - \ele_3}{ R_3 }}{1 + \cfrac{ R_2 }{ R_1 } + \cfrac{ R_2 }{ R_3 }}
            = \cfrac{\cfrac{6\,\text{В} - 4\,\text{В}}{ 4\,\text{Ом} } + \cfrac{6\,\text{В} - 2\,\text{В}}{ 15\,\text{Ом} }}{1 + \cfrac{ 6\,\text{Ом} }{ 4\,\text{Ом} } + \cfrac{ 6\,\text{Ом} }{ 15\,\text{Ом} }}
            = \frac{23}{87}\units{А} \approx 0{,}26\,\text{А}, \\
        &   U_2 = \eli_2R_2 = \cfrac{\cfrac{\ele_2 - \ele_1}{ R_1 } + \cfrac{\ele_2 - \ele_3}{ R_3 }}{1 + \cfrac{ R_2 }{ R_1 } + \cfrac{ R_2 }{ R_3 }} \cdot R_2
            = \cfrac{\cfrac{6\,\text{В} - 4\,\text{В}}{ 4\,\text{Ом} } + \cfrac{6\,\text{В} - 2\,\text{В}}{ 15\,\text{Ом} }}{1 + \cfrac{ 6\,\text{Ом} }{ 4\,\text{Ом} } + \cfrac{ 6\,\text{Ом} }{ 15\,\text{Ом} }} \cdot 6\,\text{Ом}
            = \frac{23}{87}\units{А} \cdot 6\,\text{Ом} = \frac{46}{29}\units{В} \approx 1{,}59\,\text{В}.
    \end{align*}

    Одну пару силы тока и напряжения получили.
    Для некоторых вариантов это уже ответ, но не у всех.
    Для упрощения записи преобразуем (чтобы избавитсья от 4-этажной дроби) и подставим в уже полученные уравнения:

    \begin{align*}
    \eli_2
        &=
        \frac{\frac{\ele_2 - \ele_1}{ R_1 } + \frac{\ele_2 - \ele_3}{ R_3 }}{1 + \frac{ R_2 }{ R_1 } + \frac{ R_2 }{ R_3 }}
        =
        \frac{(\ele_2 - \ele_1)R_3 + (\ele_2 - \ele_3)R_1}{R_1R_3 + R_2R_3 + R_2R_1},
        \\
    \eli_1
        &=  \frac{\ele_1 - \ele_2 + \eli_2R_2}{R_1}
        =   \frac{\ele_1 - \ele_2 + \cfrac{(\ele_2 - \ele_1)R_3 + (\ele_2 - \ele_3)R_1}{R_1R_3 + R_2R_3 + R_2R_1} \cdot R_2}{R_1} = \\
        &=  \frac{
            \ele_1R_1R_3 + \ele_1R_2R_3 + \ele_1R_2R_1
            - \ele_2R_1R_3 - \ele_2R_2R_3 - \ele_2R_2R_1
            + \ele_2R_3R_2 - \ele_1R_3R_2 + \ele_2R_1R_2 - \ele_3R_1R_2
       }{R_1 \cdot \cbr{R_1R_3 + R_2R_3 + R_2R_1}}
        = \\ &=
        \frac{
            \ele_1\cbr{R_1R_3 + R_2R_3 + R_2R_1 - R_3R_2}
            + \ele_2\cbr{- R_1R_3 - R_2R_3 - R_2R_1 + R_3R_2 + R_1R_2}
            - \ele_3R_1R_2
       }{R_1 \cdot \cbr{R_1R_3 + R_2R_3 + R_2R_1}}
        = \\ &=
        \frac{
            \ele_1\cbr{R_1R_3 + R_2R_1}
            + \ele_2\cbr{- R_1R_3}
            - \ele_3R_1R_2
       }{R_1 \cdot \cbr{R_1R_3 + R_2R_3 + R_2R_1}}
        =
        \frac{
            \ele_1\cbr{R_3 + R_2} - \ele_2R_3 - \ele_3R_2
       }{R_1R_3 + R_2R_3 + R_2R_1}
        = \\ &=
        \frac{
            (\ele_1 - \ele_3)R_2 + (\ele_1 - \ele_2)R_3
       }{R_1R_3 + R_2R_3 + R_2R_1}
        =
        \frac{
            \cfrac{\ele_1 - \ele_3}{ R_3 } + \cfrac{\ele_1 - \ele_2}{ R_2 }
       }{\cfrac{ R_1 }{ R_2 } + 1 + \cfrac{ R_1 }{ R_3 }}
        =
        \frac{
            \cfrac{4\,\text{В} - 2\,\text{В}}{ 15\,\text{Ом} } + \cfrac{4\,\text{В} - 6\,\text{В}}{ 6\,\text{Ом} }
       }{\cfrac{ 4\,\text{Ом} }{ 6\,\text{Ом} } + 1 + \cfrac{ 4\,\text{Ом} }{ 15\,\text{Ом} }}
        = -\frac3{29}\units{А} \approx -0{,}10000\,\text{А}.
        \\
    U_1
        &=
        \eli_1R_1
        =
        \frac{
            \cfrac{\ele_1 - \ele_3}{ R_3 } + \cfrac{\ele_1 - \ele_2}{ R_2 }
       }{\cfrac{ R_1 }{ R_2 } + 1 + \cfrac{ R_1 }{ R_3 }} \cdot R_1
        =
        -\frac3{29}\units{А} \cdot 4\,\text{Ом} = -\frac{12}{29}\units{В} \approx -0{,}4100\,\text{В}.
    \end{align*}

    Если вы проделали все эти вычисления выше вместе со мной, то
    \begin{itemize}
        \item вы совершили ошибку, выбрав неверный путь решения:
        слишком длинное решение, очень легко ошибиться в индексах, дробях, знаках или потерять какой-то множитель,
        \item можно было выразить из исходной системы другие токи и получить сразу нажный вам,
        а не какой-то 2-й,
        \item можно было сэкономить: все три резистора и ЭДС соединены одинаково,
        поэтому ответ для 1-го резистора должен отличаться лишь перестановкой индексов (этот факт крайне полезен при проверке ответа, у нас всё сошлось),
        я специально подгонял выражение для $\eli_1$ к этому виду, вынося за скобки и преобразуя дробь,
        \item вы молодец, потому что не побоялись и получили верный ответ грамотным способом,
    \end{itemize}
    так что переходим к третьему резистору.
    Будет похоже, но кого это когда останавливало...

    \begin{align*}
    \eli_3
        &=  \frac{\eli_2R_2 - \ele_2 + \ele_3}{ R_3 }
        =
        \cfrac{
            \cfrac{
                (\ele_2 - \ele_1)R_3 + (\ele_2 - \ele_3)R_1
           }{
                R_1R_3 + R_2R_3 + R_2R_1
           } \cdot R_2 - \ele_2 + \ele_3}{ R_3 }
        = \\ &=
        \frac{
            \ele_2R_3R_2 - \ele_1R_3R_2 + \ele_2R_1R_2 - \ele_3R_1R_2
            - \ele_2R_1R_3 - \ele_2R_2R_3 - \ele_2R_2R_1
            + \ele_3R_1R_3 + \ele_3R_2R_3 + \ele_3R_2R_1
       }{\cbr{R_1R_3 + R_2R_3 + R_2R_1} \cdot R_3}
        = \\ &=
        \frac{
            - \ele_1R_3R_2 - \ele_2R_1R_3 + \ele_3R_1R_3 + \ele_3R_2R_3
       }{\cbr{R_1R_3 + R_2R_3 + R_2R_1} \cdot R_3}
        =
        \frac{
            - \ele_1R_2 - \ele_2R_1 + \ele_3R_1 + \ele_3R_2
       }{R_1R_3 + R_2R_3 + R_2R_1}
        = \\ &=
        \frac{
            R_1(\ele_3 - \ele_2) + R_2(\ele_3 - \ele_1)
       }{R_1R_3 + R_2R_3 + R_2R_1}
        =
        \frac{
            \cfrac{\ele_3 - \ele_2}{ R_2 } + \cfrac{\ele_3 - \ele_1}{ R_1 }
       }{\cfrac{ R_3 }{ R_2 } + \cfrac{ R_3 }{ R_1 } + 1}
        =
        \frac{
            \cfrac{2\,\text{В} - 6\,\text{В}}{ 6\,\text{Ом} } + \cfrac{2\,\text{В} - 4\,\text{В}}{ 4\,\text{Ом} }
       }{\cfrac{ 15\,\text{Ом} }{ 6\,\text{Ом} } + \cfrac{ 15\,\text{Ом} }{ 4\,\text{Ом} } + 1}
        = -\frac{14}{87}\units{А} \approx -0{,}16000\,\text{А}.
        \\
    U_3
        &=
        \eli_3R_3
        =
        \frac{
            \cfrac{\ele_3 - \ele_2}{ R_2 } + \cfrac{\ele_3 - \ele_1}{ R_1 }
       }{\cfrac{ R_3 }{ R_2 } + \cfrac{ R_3 }{ R_1 } + 1} \cdot R_3
        =
        -\frac{14}{87}\units{А} \cdot 15\,\text{Ом} = -\frac{70}{29}\units{В} \approx -2{,}410\,\text{В}.
    \end{align*}

    Положительные ответы говорят, что мы угадали на рисунке направление тока (тут нет нашей заслуги, повезло),
    отрицательные — что не угадали (и в этом нет ошибки), и ток течёт в противоположную сторону.
    Напомним, что направление тока — это направление движения положительных зарядов,
    а в металлах носители заряда — электроны, которые заряжены отрицательно.
}
% autogenerated
