\setdate{11~марта~2021}
\setclass{10«АБ»}

\addpersonalvariant{Михаил Бурмистров}

\tasknumber{1}%
\task{%
    Напротив физических величин укажите их обозначения и единицы измерения в СИ, а в пункте «г)» запишите физический закон или формулу:
    \begin{enumerate}
        \item количество теплоты,
        \item работа внешних сил,
        \item удельная теплоёмкость,
        \item первое начало термодинамики.
    \end{enumerate}
}
\solutionspace{20pt}

\tasknumber{2}%
\task{%
    Определите объём идеального одноатомного газа,
    если его внутренняя энергия при давлении $2\,\text{атм}$ составляет $500\,\text{кДж}$.
    $p_{\text{aтм}} = 100\,\text{кПа}$.
}
\answer{%
    $U = \frac 32 \nu R T = \frac 32 PV \implies V = \frac 23 \cdot \frac UP= \frac 23 \cdot \frac{500\,\text{кДж}}{2\,\text{атм}} \approx 1{,}67\,\text{м}^{3}.$
}
\solutionspace{40pt}

\tasknumber{3}%
\task{%
    Газ расширился от $250\,\text{л}$ до $650\,\text{л}$.
    Давление газа при этом оставалось постоянным и равным $1{,}2\,\text{атм}$.
    Определите работу газа, ответ выразите в килоджоулях.
    $p_{\text{aтм}} = 100\,\text{кПа}$.
}
\answer{%
    $A = P\Delta V = P(V_2 - V_1) = 1{,}2\,\text{атм} \cdot \cbr{650\,\text{л} - 250\,\text{л}} = 48{,}0\,\text{кДж}.$
}
\solutionspace{40pt}

\tasknumber{4}%
\task{%
    $40\,\text{моль}$ идеального одноатомного газа в результате адиабатического процесса нагрелись на $80\,\text{К}$.
    Определите работу газа.
    Кто совершил положительную работу: газ или внешние силы?
    Универсальная газовая постоянная $R = 8{,}31\,\frac{\text{Дж}}{\text{моль}\cdot\text{К}}$.
}
\answer{%
    \begin{align*}
    Q &= 0, Q = \Delta U + A_\text{газа} \implies \\
    \implies A_\text{газа} &= - \Delta U = - \frac 32 \nu R \Delta T = - \frac 32 \cdot 40\,\text{моль} \cdot 8{,}31\,\frac{\text{Дж}}{\text{моль}\cdot\text{К}} \cdot 80\,\text{К}= -39{,}90\,\text{кДж}, \text{внешние силы.}
    \end{align*}
}
\solutionspace{40pt}

\tasknumber{5}%
\task{%
    Как изменилась внутренняя энергия одноатомного идеального газа при переходе из состояния 1 в состояние 2?
    $P_1 = 2\,\text{МПа}$, $V_1 = 3\,\text{л}$, $P_2 = 1{,}5\,\text{МПа}$, $V_2 = 8\,\text{л}$.
    Как изменилась при этом температура газа?
}
\answer{%
    \begin{align*}
    P_1V_1 &= \nu R T_1, P_2V_2 = \nu R T_2, \\
    \Delta U &= U_2-U_1 = \frac 32 \nu R T_2- \frac 32 \nu R T_1 = \frac 32 P_2 V_2 - \frac 32 P_1 V_1= \frac 32 \cdot \cbr{1{,}5\,\text{МПа} \cdot 8\,\text{л} - 2\,\text{МПа} \cdot 3\,\text{л}} = 9000\,\text{Дж}.
    \\
    \frac{T_2}{T_1} &= \frac{\frac{P_2V_2}{\nu R}}{\frac{P_1V_1}{\nu R}} = \frac{P_2V_2}{P_1V_1}= \frac{1{,}5\,\text{МПа} \cdot 8\,\text{л}}{2\,\text{МПа} \cdot 3\,\text{л}} \approx 2{,}00.
    \end{align*}
}
\solutionspace{80pt}

\tasknumber{6}%
\task{%
    $3\,\text{моль}$ идеального одноатомного газа нагрели на $20\,\text{К}$.
    Определите изменение внутренней энергии газа.
    Увеличилась она или уменьшилась?
    Универсальная газовая постоянная $R = 8{,}31\,\frac{\text{Дж}}{\text{моль}\cdot\text{К}}$.
}
\answer{%
    $
        \Delta U = \frac 32 \nu R \Delta T
            =  \frac 32 \cdot 3\,\text{моль} \cdot 8{,}31\,\frac{\text{Дж}}{\text{моль}\cdot\text{К}} \cdot 20\,\text{К}
            = 747\,\text{Дж}.
            \text{Увеличилась.}
    $
}
\solutionspace{40pt}

\tasknumber{7}%
\task{%
    Газу сообщили некоторое количество теплоты,
    при этом половину его он потратил на совершение работы,
    одновременно увеличив свою внутреннюю энергию на $1200\,\text{Дж}$.
    Определите количество теплоты, сообщённое газу.
}
\answer{%
    \begin{align*}
    Q &= A' + \Delta U, A' = \frac 12 Q \implies Q \cdot \cbr{1 - \frac 12} = \Delta U \implies Q = \frac{\Delta U}{1 - \frac 12} = \frac{1200\,\text{Дж}}{1 - \frac 12} \approx 2400\,\text{Дж}.
    \\
    A' &= \frac 12 Q
        = \frac 12 \cdot \frac{\Delta U}{1 - \frac 12}
        = \frac{\Delta U}{2 - 1}
        = \frac{1200\,\text{Дж}}{2 - 1} \approx 1200\,\text{Дж}.
    \end{align*}
}
\solutionspace{60pt}

\tasknumber{8}%
\task{%
    В некотором процессе внешние силы совершили над газом работу $300\,\text{Дж}$,
    при этом его внутренняя энергия увеличилась на $250\,\text{Дж}$.
    Определите количество тепла, переданное при этом процессе газу.
    Явно пропишите, подводили газу тепло или же отводили.
}
\answer{%
    $
        Q = A_\text{газа} + \Delta U, A_\text{газа} = -A_\text{внешняя}
        \implies Q = A_\text{газа} + \Delta U = - 300\,\text{Дж} +  250\,\text{Дж} = -50\,\text{Дж}.
        \text{ Отводили.}
    $
}

\variantsplitter

\addpersonalvariant{Ирина Ан}

\tasknumber{1}%
\task{%
    Напротив физических величин укажите их обозначения и единицы измерения в СИ, а в пункте «г)» запишите физический закон или формулу:
    \begin{enumerate}
        \item изменение внутренней энергии,
        \item работа газа,
        \item молярная теплоёмкость,
        \item внутренняя энергия идеального одноатомного газа.
    \end{enumerate}
}
\solutionspace{20pt}

\tasknumber{2}%
\task{%
    Определите объём идеального одноатомного газа,
    если его внутренняя энергия при давлении $2\,\text{атм}$ составляет $500\,\text{кДж}$.
    $p_{\text{aтм}} = 100\,\text{кПа}$.
}
\answer{%
    $U = \frac 32 \nu R T = \frac 32 PV \implies V = \frac 23 \cdot \frac UP= \frac 23 \cdot \frac{500\,\text{кДж}}{2\,\text{атм}} \approx 1{,}67\,\text{м}^{3}.$
}
\solutionspace{40pt}

\tasknumber{3}%
\task{%
    Газ расширился от $150\,\text{л}$ до $650\,\text{л}$.
    Давление газа при этом оставалось постоянным и равным $1{,}5\,\text{атм}$.
    Определите работу газа, ответ выразите в килоджоулях.
    $p_{\text{aтм}} = 100\,\text{кПа}$.
}
\answer{%
    $A = P\Delta V = P(V_2 - V_1) = 1{,}5\,\text{атм} \cdot \cbr{650\,\text{л} - 150\,\text{л}} = 75{,}0\,\text{кДж}.$
}
\solutionspace{40pt}

\tasknumber{4}%
\task{%
    $50\,\text{моль}$ идеального одноатомного газа в результате адиабатического процесса нагрелись на $80\,\text{К}$.
    Определите работу газа.
    Кто совершил положительную работу: газ или внешние силы?
    Универсальная газовая постоянная $R = 8{,}31\,\frac{\text{Дж}}{\text{моль}\cdot\text{К}}$.
}
\answer{%
    \begin{align*}
    Q &= 0, Q = \Delta U + A_\text{газа} \implies \\
    \implies A_\text{газа} &= - \Delta U = - \frac 32 \nu R \Delta T = - \frac 32 \cdot 50\,\text{моль} \cdot 8{,}31\,\frac{\text{Дж}}{\text{моль}\cdot\text{К}} \cdot 80\,\text{К}= -49{,}90\,\text{кДж}, \text{внешние силы.}
    \end{align*}
}
\solutionspace{40pt}

\tasknumber{5}%
\task{%
    Как изменилась внутренняя энергия одноатомного идеального газа при переходе из состояния 1 в состояние 2?
    $P_1 = 4\,\text{МПа}$, $V_1 = 3\,\text{л}$, $P_2 = 1{,}5\,\text{МПа}$, $V_2 = 8\,\text{л}$.
    Как изменилась при этом температура газа?
}
\answer{%
    \begin{align*}
    P_1V_1 &= \nu R T_1, P_2V_2 = \nu R T_2, \\
    \Delta U &= U_2-U_1 = \frac 32 \nu R T_2- \frac 32 \nu R T_1 = \frac 32 P_2 V_2 - \frac 32 P_1 V_1= \frac 32 \cdot \cbr{1{,}5\,\text{МПа} \cdot 8\,\text{л} - 4\,\text{МПа} \cdot 3\,\text{л}} = 0\,\text{Дж}.
    \\
    \frac{T_2}{T_1} &= \frac{\frac{P_2V_2}{\nu R}}{\frac{P_1V_1}{\nu R}} = \frac{P_2V_2}{P_1V_1}= \frac{1{,}5\,\text{МПа} \cdot 8\,\text{л}}{4\,\text{МПа} \cdot 3\,\text{л}} \approx 1{,}00.
    \end{align*}
}
\solutionspace{80pt}

\tasknumber{6}%
\task{%
    $4\,\text{моль}$ идеального одноатомного газа нагрели на $20\,\text{К}$.
    Определите изменение внутренней энергии газа.
    Увеличилась она или уменьшилась?
    Универсальная газовая постоянная $R = 8{,}31\,\frac{\text{Дж}}{\text{моль}\cdot\text{К}}$.
}
\answer{%
    $
        \Delta U = \frac 32 \nu R \Delta T
            =  \frac 32 \cdot 4\,\text{моль} \cdot 8{,}31\,\frac{\text{Дж}}{\text{моль}\cdot\text{К}} \cdot 20\,\text{К}
            = 997\,\text{Дж}.
            \text{Увеличилась.}
    $
}
\solutionspace{40pt}

\tasknumber{7}%
\task{%
    Газу сообщили некоторое количество теплоты,
    при этом треть его он потратил на совершение работы,
    одновременно увеличив свою внутреннюю энергию на $3000\,\text{Дж}$.
    Определите количество теплоты, сообщённое газу.
}
\answer{%
    \begin{align*}
    Q &= A' + \Delta U, A' = \frac 13 Q \implies Q \cdot \cbr{1 - \frac 13} = \Delta U \implies Q = \frac{\Delta U}{1 - \frac 13} = \frac{3000\,\text{Дж}}{1 - \frac 13} \approx 4500\,\text{Дж}.
    \\
    A' &= \frac 13 Q
        = \frac 13 \cdot \frac{\Delta U}{1 - \frac 13}
        = \frac{\Delta U}{3 - 1}
        = \frac{3000\,\text{Дж}}{3 - 1} \approx 1500\,\text{Дж}.
    \end{align*}
}
\solutionspace{60pt}

\tasknumber{8}%
\task{%
    В некотором процессе внешние силы совершили над газом работу $100\,\text{Дж}$,
    при этом его внутренняя энергия увеличилась на $450\,\text{Дж}$.
    Определите количество тепла, переданное при этом процессе газу.
    Явно пропишите, подводили газу тепло или же отводили.
}
\answer{%
    $
        Q = A_\text{газа} + \Delta U, A_\text{газа} = -A_\text{внешняя}
        \implies Q = A_\text{газа} + \Delta U = - 100\,\text{Дж} +  450\,\text{Дж} = 350\,\text{Дж}.
        \text{ Подводили.}
    $
}

\variantsplitter

\addpersonalvariant{Софья Андрианова}

\tasknumber{1}%
\task{%
    Напротив физических величин укажите их обозначения и единицы измерения в СИ, а в пункте «г)» запишите физический закон или формулу:
    \begin{enumerate}
        \item изменение внутренней энергии,
        \item работа газа,
        \item удельная теплоёмкость,
        \item первое начало термодинамики.
    \end{enumerate}
}
\solutionspace{20pt}

\tasknumber{2}%
\task{%
    Определите объём идеального одноатомного газа,
    если его внутренняя энергия при давлении $3\,\text{атм}$ составляет $250\,\text{кДж}$.
    $p_{\text{aтм}} = 100\,\text{кПа}$.
}
\answer{%
    $U = \frac 32 \nu R T = \frac 32 PV \implies V = \frac 23 \cdot \frac UP= \frac 23 \cdot \frac{250\,\text{кДж}}{3\,\text{атм}} \approx 0{,}56\,\text{м}^{3}.$
}
\solutionspace{40pt}

\tasknumber{3}%
\task{%
    Газ расширился от $250\,\text{л}$ до $650\,\text{л}$.
    Давление газа при этом оставалось постоянным и равным $1{,}5\,\text{атм}$.
    Определите работу газа, ответ выразите в килоджоулях.
    $p_{\text{aтм}} = 100\,\text{кПа}$.
}
\answer{%
    $A = P\Delta V = P(V_2 - V_1) = 1{,}5\,\text{атм} \cdot \cbr{650\,\text{л} - 250\,\text{л}} = 60{,}0\,\text{кДж}.$
}
\solutionspace{40pt}

\tasknumber{4}%
\task{%
    $40\,\text{моль}$ идеального одноатомного газа в результате адиабатического процесса нагрелись на $45\,\text{К}$.
    Определите работу газа.
    Кто совершил положительную работу: газ или внешние силы?
    Универсальная газовая постоянная $R = 8{,}31\,\frac{\text{Дж}}{\text{моль}\cdot\text{К}}$.
}
\answer{%
    \begin{align*}
    Q &= 0, Q = \Delta U + A_\text{газа} \implies \\
    \implies A_\text{газа} &= - \Delta U = - \frac 32 \nu R \Delta T = - \frac 32 \cdot 40\,\text{моль} \cdot 8{,}31\,\frac{\text{Дж}}{\text{моль}\cdot\text{К}} \cdot 45\,\text{К}= -22{,}40\,\text{кДж}, \text{внешние силы.}
    \end{align*}
}
\solutionspace{40pt}

\tasknumber{5}%
\task{%
    Как изменилась внутренняя энергия одноатомного идеального газа при переходе из состояния 1 в состояние 2?
    $P_1 = 3\,\text{МПа}$, $V_1 = 7\,\text{л}$, $P_2 = 2{,}5\,\text{МПа}$, $V_2 = 8\,\text{л}$.
    Как изменилась при этом температура газа?
}
\answer{%
    \begin{align*}
    P_1V_1 &= \nu R T_1, P_2V_2 = \nu R T_2, \\
    \Delta U &= U_2-U_1 = \frac 32 \nu R T_2- \frac 32 \nu R T_1 = \frac 32 P_2 V_2 - \frac 32 P_1 V_1= \frac 32 \cdot \cbr{2{,}5\,\text{МПа} \cdot 8\,\text{л} - 3\,\text{МПа} \cdot 7\,\text{л}} = -1500\,\text{Дж}.
    \\
    \frac{T_2}{T_1} &= \frac{\frac{P_2V_2}{\nu R}}{\frac{P_1V_1}{\nu R}} = \frac{P_2V_2}{P_1V_1}= \frac{2{,}5\,\text{МПа} \cdot 8\,\text{л}}{3\,\text{МПа} \cdot 7\,\text{л}} \approx 0{,}95.
    \end{align*}
}
\solutionspace{80pt}

\tasknumber{6}%
\task{%
    $3\,\text{моль}$ идеального одноатомного газа нагрели на $30\,\text{К}$.
    Определите изменение внутренней энергии газа.
    Увеличилась она или уменьшилась?
    Универсальная газовая постоянная $R = 8{,}31\,\frac{\text{Дж}}{\text{моль}\cdot\text{К}}$.
}
\answer{%
    $
        \Delta U = \frac 32 \nu R \Delta T
            =  \frac 32 \cdot 3\,\text{моль} \cdot 8{,}31\,\frac{\text{Дж}}{\text{моль}\cdot\text{К}} \cdot 30\,\text{К}
            = 1121\,\text{Дж}.
            \text{Увеличилась.}
    $
}
\solutionspace{40pt}

\tasknumber{7}%
\task{%
    Газу сообщили некоторое количество теплоты,
    при этом треть его он потратил на совершение работы,
    одновременно увеличив свою внутреннюю энергию на $3000\,\text{Дж}$.
    Определите количество теплоты, сообщённое газу.
}
\answer{%
    \begin{align*}
    Q &= A' + \Delta U, A' = \frac 13 Q \implies Q \cdot \cbr{1 - \frac 13} = \Delta U \implies Q = \frac{\Delta U}{1 - \frac 13} = \frac{3000\,\text{Дж}}{1 - \frac 13} \approx 4500\,\text{Дж}.
    \\
    A' &= \frac 13 Q
        = \frac 13 \cdot \frac{\Delta U}{1 - \frac 13}
        = \frac{\Delta U}{3 - 1}
        = \frac{3000\,\text{Дж}}{3 - 1} \approx 1500\,\text{Дж}.
    \end{align*}
}
\solutionspace{60pt}

\tasknumber{8}%
\task{%
    В некотором процессе внешние силы совершили над газом работу $300\,\text{Дж}$,
    при этом его внутренняя энергия увеличилась на $350\,\text{Дж}$.
    Определите количество тепла, переданное при этом процессе газу.
    Явно пропишите, подводили газу тепло или же отводили.
}
\answer{%
    $
        Q = A_\text{газа} + \Delta U, A_\text{газа} = -A_\text{внешняя}
        \implies Q = A_\text{газа} + \Delta U = - 300\,\text{Дж} +  350\,\text{Дж} = 50\,\text{Дж}.
        \text{ Подводили.}
    $
}

\variantsplitter

\addpersonalvariant{Владимир Артемчук}

\tasknumber{1}%
\task{%
    Напротив физических величин укажите их обозначения и единицы измерения в СИ, а в пункте «г)» запишите физический закон или формулу:
    \begin{enumerate}
        \item количество теплоты,
        \item работа газа,
        \item молярная теплоёмкость,
        \item первое начало термодинамики.
    \end{enumerate}
}
\solutionspace{20pt}

\tasknumber{2}%
\task{%
    Определите объём идеального одноатомного газа,
    если его внутренняя энергия при давлении $5\,\text{атм}$ составляет $250\,\text{кДж}$.
    $p_{\text{aтм}} = 100\,\text{кПа}$.
}
\answer{%
    $U = \frac 32 \nu R T = \frac 32 PV \implies V = \frac 23 \cdot \frac UP= \frac 23 \cdot \frac{250\,\text{кДж}}{5\,\text{атм}} \approx 0{,}33\,\text{м}^{3}.$
}
\solutionspace{40pt}

\tasknumber{3}%
\task{%
    Газ расширился от $150\,\text{л}$ до $550\,\text{л}$.
    Давление газа при этом оставалось постоянным и равным $2{,}5\,\text{атм}$.
    Определите работу газа, ответ выразите в килоджоулях.
    $p_{\text{aтм}} = 100\,\text{кПа}$.
}
\answer{%
    $A = P\Delta V = P(V_2 - V_1) = 2{,}5\,\text{атм} \cdot \cbr{550\,\text{л} - 150\,\text{л}} = 100{,}0\,\text{кДж}.$
}
\solutionspace{40pt}

\tasknumber{4}%
\task{%
    $40\,\text{моль}$ идеального одноатомного газа в результате адиабатического процесса нагрелись на $25\,\text{К}$.
    Определите работу газа.
    Кто совершил положительную работу: газ или внешние силы?
    Универсальная газовая постоянная $R = 8{,}31\,\frac{\text{Дж}}{\text{моль}\cdot\text{К}}$.
}
\answer{%
    \begin{align*}
    Q &= 0, Q = \Delta U + A_\text{газа} \implies \\
    \implies A_\text{газа} &= - \Delta U = - \frac 32 \nu R \Delta T = - \frac 32 \cdot 40\,\text{моль} \cdot 8{,}31\,\frac{\text{Дж}}{\text{моль}\cdot\text{К}} \cdot 25\,\text{К}= -12{,}500\,\text{кДж}, \text{внешние силы.}
    \end{align*}
}
\solutionspace{40pt}

\tasknumber{5}%
\task{%
    Как изменилась внутренняя энергия одноатомного идеального газа при переходе из состояния 1 в состояние 2?
    $P_1 = 4\,\text{МПа}$, $V_1 = 7\,\text{л}$, $P_2 = 3{,}5\,\text{МПа}$, $V_2 = 4\,\text{л}$.
    Как изменилась при этом температура газа?
}
\answer{%
    \begin{align*}
    P_1V_1 &= \nu R T_1, P_2V_2 = \nu R T_2, \\
    \Delta U &= U_2-U_1 = \frac 32 \nu R T_2- \frac 32 \nu R T_1 = \frac 32 P_2 V_2 - \frac 32 P_1 V_1= \frac 32 \cdot \cbr{3{,}5\,\text{МПа} \cdot 4\,\text{л} - 4\,\text{МПа} \cdot 7\,\text{л}} = -21000\,\text{Дж}.
    \\
    \frac{T_2}{T_1} &= \frac{\frac{P_2V_2}{\nu R}}{\frac{P_1V_1}{\nu R}} = \frac{P_2V_2}{P_1V_1}= \frac{3{,}5\,\text{МПа} \cdot 4\,\text{л}}{4\,\text{МПа} \cdot 7\,\text{л}} \approx 0{,}50.
    \end{align*}
}
\solutionspace{80pt}

\tasknumber{6}%
\task{%
    $3\,\text{моль}$ идеального одноатомного газа нагрели на $20\,\text{К}$.
    Определите изменение внутренней энергии газа.
    Увеличилась она или уменьшилась?
    Универсальная газовая постоянная $R = 8{,}31\,\frac{\text{Дж}}{\text{моль}\cdot\text{К}}$.
}
\answer{%
    $
        \Delta U = \frac 32 \nu R \Delta T
            =  \frac 32 \cdot 3\,\text{моль} \cdot 8{,}31\,\frac{\text{Дж}}{\text{моль}\cdot\text{К}} \cdot 20\,\text{К}
            = 747\,\text{Дж}.
            \text{Увеличилась.}
    $
}
\solutionspace{40pt}

\tasknumber{7}%
\task{%
    Газу сообщили некоторое количество теплоты,
    при этом половину его он потратил на совершение работы,
    одновременно увеличив свою внутреннюю энергию на $1200\,\text{Дж}$.
    Определите работу, совершённую газом.
}
\answer{%
    \begin{align*}
    Q &= A' + \Delta U, A' = \frac 12 Q \implies Q \cdot \cbr{1 - \frac 12} = \Delta U \implies Q = \frac{\Delta U}{1 - \frac 12} = \frac{1200\,\text{Дж}}{1 - \frac 12} \approx 2400\,\text{Дж}.
    \\
    A' &= \frac 12 Q
        = \frac 12 \cdot \frac{\Delta U}{1 - \frac 12}
        = \frac{\Delta U}{2 - 1}
        = \frac{1200\,\text{Дж}}{2 - 1} \approx 1200\,\text{Дж}.
    \end{align*}
}
\solutionspace{60pt}

\tasknumber{8}%
\task{%
    В некотором процессе газ совершил работу $200\,\text{Дж}$,
    при этом его внутренняя энергия увеличилась на $350\,\text{Дж}$.
    Определите количество тепла, переданное при этом процессе газу.
    Явно пропишите, подводили газу тепло или же отводили.
}
\answer{%
    $
        Q = A_\text{газа} + \Delta U, A_\text{газа} = -A_\text{внешняя}
        \implies Q = A_\text{газа} + \Delta U =  200\,\text{Дж} +  350\,\text{Дж} = 550\,\text{Дж}.
        \text{ Подводили.}
    $
}

\variantsplitter

\addpersonalvariant{Софья Белянкина}

\tasknumber{1}%
\task{%
    Напротив физических величин укажите их обозначения и единицы измерения в СИ, а в пункте «г)» запишите физический закон или формулу:
    \begin{enumerate}
        \item количество теплоты,
        \item работа внешних сил,
        \item удельная теплоёмкость,
        \item первое начало термодинамики.
    \end{enumerate}
}
\solutionspace{20pt}

\tasknumber{2}%
\task{%
    Определите объём идеального одноатомного газа,
    если его внутренняя энергия при давлении $3\,\text{атм}$ составляет $250\,\text{кДж}$.
    $p_{\text{aтм}} = 100\,\text{кПа}$.
}
\answer{%
    $U = \frac 32 \nu R T = \frac 32 PV \implies V = \frac 23 \cdot \frac UP= \frac 23 \cdot \frac{250\,\text{кДж}}{3\,\text{атм}} \approx 0{,}56\,\text{м}^{3}.$
}
\solutionspace{40pt}

\tasknumber{3}%
\task{%
    Газ расширился от $250\,\text{л}$ до $650\,\text{л}$.
    Давление газа при этом оставалось постоянным и равным $2{,}5\,\text{атм}$.
    Определите работу газа, ответ выразите в килоджоулях.
    $p_{\text{aтм}} = 100\,\text{кПа}$.
}
\answer{%
    $A = P\Delta V = P(V_2 - V_1) = 2{,}5\,\text{атм} \cdot \cbr{650\,\text{л} - 250\,\text{л}} = 100{,}0\,\text{кДж}.$
}
\solutionspace{40pt}

\tasknumber{4}%
\task{%
    $40\,\text{моль}$ идеального одноатомного газа в результате адиабатического процесса остыли на $60\,\text{К}$.
    Определите работу газа.
    Кто совершил положительную работу: газ или внешние силы?
    Универсальная газовая постоянная $R = 8{,}31\,\frac{\text{Дж}}{\text{моль}\cdot\text{К}}$.
}
\answer{%
    \begin{align*}
    Q &= 0, Q = \Delta U + A_\text{газа} \implies \\
    \implies A_\text{газа} &= - \Delta U = - \frac 32 \nu R \Delta T =  \frac 32 \cdot 40\,\text{моль} \cdot 8{,}31\,\frac{\text{Дж}}{\text{моль}\cdot\text{К}} \cdot 60\,\text{К}= 29{,}9\,\text{кДж}, \text{газ.}
    \end{align*}
}
\solutionspace{40pt}

\tasknumber{5}%
\task{%
    Как изменилась внутренняя энергия одноатомного идеального газа при переходе из состояния 1 в состояние 2?
    $P_1 = 4\,\text{МПа}$, $V_1 = 7\,\text{л}$, $P_2 = 2{,}5\,\text{МПа}$, $V_2 = 6\,\text{л}$.
    Как изменилась при этом температура газа?
}
\answer{%
    \begin{align*}
    P_1V_1 &= \nu R T_1, P_2V_2 = \nu R T_2, \\
    \Delta U &= U_2-U_1 = \frac 32 \nu R T_2- \frac 32 \nu R T_1 = \frac 32 P_2 V_2 - \frac 32 P_1 V_1= \frac 32 \cdot \cbr{2{,}5\,\text{МПа} \cdot 6\,\text{л} - 4\,\text{МПа} \cdot 7\,\text{л}} = -19500\,\text{Дж}.
    \\
    \frac{T_2}{T_1} &= \frac{\frac{P_2V_2}{\nu R}}{\frac{P_1V_1}{\nu R}} = \frac{P_2V_2}{P_1V_1}= \frac{2{,}5\,\text{МПа} \cdot 6\,\text{л}}{4\,\text{МПа} \cdot 7\,\text{л}} \approx 0{,}54.
    \end{align*}
}
\solutionspace{80pt}

\tasknumber{6}%
\task{%
    $3\,\text{моль}$ идеального одноатомного газа охладили на $10\,\text{К}$.
    Определите изменение внутренней энергии газа.
    Увеличилась она или уменьшилась?
    Универсальная газовая постоянная $R = 8{,}31\,\frac{\text{Дж}}{\text{моль}\cdot\text{К}}$.
}
\answer{%
    $
        \Delta U = \frac 32 \nu R \Delta T
            = - \frac 32 \cdot 3\,\text{моль} \cdot 8{,}31\,\frac{\text{Дж}}{\text{моль}\cdot\text{К}} \cdot 10\,\text{К}
            = -373\,\text{Дж}.
            \text{Уменьшилась.}
    $
}
\solutionspace{40pt}

\tasknumber{7}%
\task{%
    Газу сообщили некоторое количество теплоты,
    при этом четверть его он потратил на совершение работы,
    одновременно увеличив свою внутреннюю энергию на $2400\,\text{Дж}$.
    Определите работу, совершённую газом.
}
\answer{%
    \begin{align*}
    Q &= A' + \Delta U, A' = \frac 14 Q \implies Q \cdot \cbr{1 - \frac 14} = \Delta U \implies Q = \frac{\Delta U}{1 - \frac 14} = \frac{2400\,\text{Дж}}{1 - \frac 14} \approx 3200\,\text{Дж}.
    \\
    A' &= \frac 14 Q
        = \frac 14 \cdot \frac{\Delta U}{1 - \frac 14}
        = \frac{\Delta U}{4 - 1}
        = \frac{2400\,\text{Дж}}{4 - 1} \approx 800\,\text{Дж}.
    \end{align*}
}
\solutionspace{60pt}

\tasknumber{8}%
\task{%
    В некотором процессе газ совершил работу $100\,\text{Дж}$,
    при этом его внутренняя энергия увеличилась на $450\,\text{Дж}$.
    Определите количество тепла, переданное при этом процессе газу.
    Явно пропишите, подводили газу тепло или же отводили.
}
\answer{%
    $
        Q = A_\text{газа} + \Delta U, A_\text{газа} = -A_\text{внешняя}
        \implies Q = A_\text{газа} + \Delta U =  100\,\text{Дж} +  450\,\text{Дж} = 550\,\text{Дж}.
        \text{ Подводили.}
    $
}

\variantsplitter

\addpersonalvariant{Варвара Егиазарян}

\tasknumber{1}%
\task{%
    Напротив физических величин укажите их обозначения и единицы измерения в СИ, а в пункте «г)» запишите физический закон или формулу:
    \begin{enumerate}
        \item количество теплоты,
        \item работа газа,
        \item молярная теплоёмкость,
        \item первое начало термодинамики.
    \end{enumerate}
}
\solutionspace{20pt}

\tasknumber{2}%
\task{%
    Определите объём идеального одноатомного газа,
    если его внутренняя энергия при давлении $2\,\text{атм}$ составляет $250\,\text{кДж}$.
    $p_{\text{aтм}} = 100\,\text{кПа}$.
}
\answer{%
    $U = \frac 32 \nu R T = \frac 32 PV \implies V = \frac 23 \cdot \frac UP= \frac 23 \cdot \frac{250\,\text{кДж}}{2\,\text{атм}} \approx 0{,}83\,\text{м}^{3}.$
}
\solutionspace{40pt}

\tasknumber{3}%
\task{%
    Газ расширился от $200\,\text{л}$ до $550\,\text{л}$.
    Давление газа при этом оставалось постоянным и равным $1{,}5\,\text{атм}$.
    Определите работу газа, ответ выразите в килоджоулях.
    $p_{\text{aтм}} = 100\,\text{кПа}$.
}
\answer{%
    $A = P\Delta V = P(V_2 - V_1) = 1{,}5\,\text{атм} \cdot \cbr{550\,\text{л} - 200\,\text{л}} = 52{,}5\,\text{кДж}.$
}
\solutionspace{40pt}

\tasknumber{4}%
\task{%
    $60\,\text{моль}$ идеального одноатомного газа в результате адиабатического процесса нагрелись на $25\,\text{К}$.
    Определите работу газа.
    Кто совершил положительную работу: газ или внешние силы?
    Универсальная газовая постоянная $R = 8{,}31\,\frac{\text{Дж}}{\text{моль}\cdot\text{К}}$.
}
\answer{%
    \begin{align*}
    Q &= 0, Q = \Delta U + A_\text{газа} \implies \\
    \implies A_\text{газа} &= - \Delta U = - \frac 32 \nu R \Delta T = - \frac 32 \cdot 60\,\text{моль} \cdot 8{,}31\,\frac{\text{Дж}}{\text{моль}\cdot\text{К}} \cdot 25\,\text{К}= -18{,}700\,\text{кДж}, \text{внешние силы.}
    \end{align*}
}
\solutionspace{40pt}

\tasknumber{5}%
\task{%
    Как изменилась внутренняя энергия одноатомного идеального газа при переходе из состояния 1 в состояние 2?
    $P_1 = 2\,\text{МПа}$, $V_1 = 5\,\text{л}$, $P_2 = 1{,}5\,\text{МПа}$, $V_2 = 2\,\text{л}$.
    Как изменилась при этом температура газа?
}
\answer{%
    \begin{align*}
    P_1V_1 &= \nu R T_1, P_2V_2 = \nu R T_2, \\
    \Delta U &= U_2-U_1 = \frac 32 \nu R T_2- \frac 32 \nu R T_1 = \frac 32 P_2 V_2 - \frac 32 P_1 V_1= \frac 32 \cdot \cbr{1{,}5\,\text{МПа} \cdot 2\,\text{л} - 2\,\text{МПа} \cdot 5\,\text{л}} = -10500\,\text{Дж}.
    \\
    \frac{T_2}{T_1} &= \frac{\frac{P_2V_2}{\nu R}}{\frac{P_1V_1}{\nu R}} = \frac{P_2V_2}{P_1V_1}= \frac{1{,}5\,\text{МПа} \cdot 2\,\text{л}}{2\,\text{МПа} \cdot 5\,\text{л}} \approx 0{,}30.
    \end{align*}
}
\solutionspace{80pt}

\tasknumber{6}%
\task{%
    $5\,\text{моль}$ идеального одноатомного газа нагрели на $20\,\text{К}$.
    Определите изменение внутренней энергии газа.
    Увеличилась она или уменьшилась?
    Универсальная газовая постоянная $R = 8{,}31\,\frac{\text{Дж}}{\text{моль}\cdot\text{К}}$.
}
\answer{%
    $
        \Delta U = \frac 32 \nu R \Delta T
            =  \frac 32 \cdot 5\,\text{моль} \cdot 8{,}31\,\frac{\text{Дж}}{\text{моль}\cdot\text{К}} \cdot 20\,\text{К}
            = 1246\,\text{Дж}.
            \text{Увеличилась.}
    $
}
\solutionspace{40pt}

\tasknumber{7}%
\task{%
    Газу сообщили некоторое количество теплоты,
    при этом половину его он потратил на совершение работы,
    одновременно увеличив свою внутреннюю энергию на $2400\,\text{Дж}$.
    Определите работу, совершённую газом.
}
\answer{%
    \begin{align*}
    Q &= A' + \Delta U, A' = \frac 12 Q \implies Q \cdot \cbr{1 - \frac 12} = \Delta U \implies Q = \frac{\Delta U}{1 - \frac 12} = \frac{2400\,\text{Дж}}{1 - \frac 12} \approx 4800\,\text{Дж}.
    \\
    A' &= \frac 12 Q
        = \frac 12 \cdot \frac{\Delta U}{1 - \frac 12}
        = \frac{\Delta U}{2 - 1}
        = \frac{2400\,\text{Дж}}{2 - 1} \approx 2400\,\text{Дж}.
    \end{align*}
}
\solutionspace{60pt}

\tasknumber{8}%
\task{%
    В некотором процессе газ совершил работу $100\,\text{Дж}$,
    при этом его внутренняя энергия уменьшилась на $250\,\text{Дж}$.
    Определите количество тепла, переданное при этом процессе газу.
    Явно пропишите, подводили газу тепло или же отводили.
}
\answer{%
    $
        Q = A_\text{газа} + \Delta U, A_\text{газа} = -A_\text{внешняя}
        \implies Q = A_\text{газа} + \Delta U =  100\,\text{Дж} - 250\,\text{Дж} = -150\,\text{Дж}.
        \text{ Отводили.}
    $
}

\variantsplitter

\addpersonalvariant{Владислав Емелин}

\tasknumber{1}%
\task{%
    Напротив физических величин укажите их обозначения и единицы измерения в СИ, а в пункте «г)» запишите физический закон или формулу:
    \begin{enumerate}
        \item количество теплоты,
        \item работа внешних сил,
        \item удельная теплоёмкость,
        \item первое начало термодинамики.
    \end{enumerate}
}
\solutionspace{20pt}

\tasknumber{2}%
\task{%
    Определите объём идеального одноатомного газа,
    если его внутренняя энергия при давлении $5\,\text{атм}$ составляет $400\,\text{кДж}$.
    $p_{\text{aтм}} = 100\,\text{кПа}$.
}
\answer{%
    $U = \frac 32 \nu R T = \frac 32 PV \implies V = \frac 23 \cdot \frac UP= \frac 23 \cdot \frac{400\,\text{кДж}}{5\,\text{атм}} \approx 0{,}53\,\text{м}^{3}.$
}
\solutionspace{40pt}

\tasknumber{3}%
\task{%
    Газ расширился от $200\,\text{л}$ до $450\,\text{л}$.
    Давление газа при этом оставалось постоянным и равным $1{,}5\,\text{атм}$.
    Определите работу газа, ответ выразите в килоджоулях.
    $p_{\text{aтм}} = 100\,\text{кПа}$.
}
\answer{%
    $A = P\Delta V = P(V_2 - V_1) = 1{,}5\,\text{атм} \cdot \cbr{450\,\text{л} - 200\,\text{л}} = 37{,}5\,\text{кДж}.$
}
\solutionspace{40pt}

\tasknumber{4}%
\task{%
    $60\,\text{моль}$ идеального одноатомного газа в результате адиабатического процесса нагрелись на $120\,\text{К}$.
    Определите работу газа.
    Кто совершил положительную работу: газ или внешние силы?
    Универсальная газовая постоянная $R = 8{,}31\,\frac{\text{Дж}}{\text{моль}\cdot\text{К}}$.
}
\answer{%
    \begin{align*}
    Q &= 0, Q = \Delta U + A_\text{газа} \implies \\
    \implies A_\text{газа} &= - \Delta U = - \frac 32 \nu R \Delta T = - \frac 32 \cdot 60\,\text{моль} \cdot 8{,}31\,\frac{\text{Дж}}{\text{моль}\cdot\text{К}} \cdot 120\,\text{К}= -89{,}70\,\text{кДж}, \text{внешние силы.}
    \end{align*}
}
\solutionspace{40pt}

\tasknumber{5}%
\task{%
    Как изменилась внутренняя энергия одноатомного идеального газа при переходе из состояния 1 в состояние 2?
    $P_1 = 4\,\text{МПа}$, $V_1 = 5\,\text{л}$, $P_2 = 3{,}5\,\text{МПа}$, $V_2 = 6\,\text{л}$.
    Как изменилась при этом температура газа?
}
\answer{%
    \begin{align*}
    P_1V_1 &= \nu R T_1, P_2V_2 = \nu R T_2, \\
    \Delta U &= U_2-U_1 = \frac 32 \nu R T_2- \frac 32 \nu R T_1 = \frac 32 P_2 V_2 - \frac 32 P_1 V_1= \frac 32 \cdot \cbr{3{,}5\,\text{МПа} \cdot 6\,\text{л} - 4\,\text{МПа} \cdot 5\,\text{л}} = 1500\,\text{Дж}.
    \\
    \frac{T_2}{T_1} &= \frac{\frac{P_2V_2}{\nu R}}{\frac{P_1V_1}{\nu R}} = \frac{P_2V_2}{P_1V_1}= \frac{3{,}5\,\text{МПа} \cdot 6\,\text{л}}{4\,\text{МПа} \cdot 5\,\text{л}} \approx 1{,}05.
    \end{align*}
}
\solutionspace{80pt}

\tasknumber{6}%
\task{%
    $5\,\text{моль}$ идеального одноатомного газа нагрели на $30\,\text{К}$.
    Определите изменение внутренней энергии газа.
    Увеличилась она или уменьшилась?
    Универсальная газовая постоянная $R = 8{,}31\,\frac{\text{Дж}}{\text{моль}\cdot\text{К}}$.
}
\answer{%
    $
        \Delta U = \frac 32 \nu R \Delta T
            =  \frac 32 \cdot 5\,\text{моль} \cdot 8{,}31\,\frac{\text{Дж}}{\text{моль}\cdot\text{К}} \cdot 30\,\text{К}
            = 1869\,\text{Дж}.
            \text{Увеличилась.}
    $
}
\solutionspace{40pt}

\tasknumber{7}%
\task{%
    Газу сообщили некоторое количество теплоты,
    при этом четверть его он потратил на совершение работы,
    одновременно увеличив свою внутреннюю энергию на $2400\,\text{Дж}$.
    Определите количество теплоты, сообщённое газу.
}
\answer{%
    \begin{align*}
    Q &= A' + \Delta U, A' = \frac 14 Q \implies Q \cdot \cbr{1 - \frac 14} = \Delta U \implies Q = \frac{\Delta U}{1 - \frac 14} = \frac{2400\,\text{Дж}}{1 - \frac 14} \approx 3200\,\text{Дж}.
    \\
    A' &= \frac 14 Q
        = \frac 14 \cdot \frac{\Delta U}{1 - \frac 14}
        = \frac{\Delta U}{4 - 1}
        = \frac{2400\,\text{Дж}}{4 - 1} \approx 800\,\text{Дж}.
    \end{align*}
}
\solutionspace{60pt}

\tasknumber{8}%
\task{%
    В некотором процессе газ совершил работу $300\,\text{Дж}$,
    при этом его внутренняя энергия уменьшилась на $350\,\text{Дж}$.
    Определите количество тепла, переданное при этом процессе газу.
    Явно пропишите, подводили газу тепло или же отводили.
}
\answer{%
    $
        Q = A_\text{газа} + \Delta U, A_\text{газа} = -A_\text{внешняя}
        \implies Q = A_\text{газа} + \Delta U =  300\,\text{Дж} - 350\,\text{Дж} = -50\,\text{Дж}.
        \text{ Отводили.}
    $
}

\variantsplitter

\addpersonalvariant{Артём Жичин}

\tasknumber{1}%
\task{%
    Напротив физических величин укажите их обозначения и единицы измерения в СИ, а в пункте «г)» запишите физический закон или формулу:
    \begin{enumerate}
        \item изменение внутренней энергии,
        \item работа газа,
        \item молярная теплоёмкость,
        \item внутренняя энергия идеального одноатомного газа.
    \end{enumerate}
}
\solutionspace{20pt}

\tasknumber{2}%
\task{%
    Определите объём идеального одноатомного газа,
    если его внутренняя энергия при давлении $4\,\text{атм}$ составляет $300\,\text{кДж}$.
    $p_{\text{aтм}} = 100\,\text{кПа}$.
}
\answer{%
    $U = \frac 32 \nu R T = \frac 32 PV \implies V = \frac 23 \cdot \frac UP= \frac 23 \cdot \frac{300\,\text{кДж}}{4\,\text{атм}} \approx 0{,}50\,\text{м}^{3}.$
}
\solutionspace{40pt}

\tasknumber{3}%
\task{%
    Газ расширился от $350\,\text{л}$ до $450\,\text{л}$.
    Давление газа при этом оставалось постоянным и равным $1{,}8\,\text{атм}$.
    Определите работу газа, ответ выразите в килоджоулях.
    $p_{\text{aтм}} = 100\,\text{кПа}$.
}
\answer{%
    $A = P\Delta V = P(V_2 - V_1) = 1{,}8\,\text{атм} \cdot \cbr{450\,\text{л} - 350\,\text{л}} = 18{,}0\,\text{кДж}.$
}
\solutionspace{40pt}

\tasknumber{4}%
\task{%
    $60\,\text{моль}$ идеального одноатомного газа в результате адиабатического процесса остыли на $120\,\text{К}$.
    Определите работу газа.
    Кто совершил положительную работу: газ или внешние силы?
    Универсальная газовая постоянная $R = 8{,}31\,\frac{\text{Дж}}{\text{моль}\cdot\text{К}}$.
}
\answer{%
    \begin{align*}
    Q &= 0, Q = \Delta U + A_\text{газа} \implies \\
    \implies A_\text{газа} &= - \Delta U = - \frac 32 \nu R \Delta T =  \frac 32 \cdot 60\,\text{моль} \cdot 8{,}31\,\frac{\text{Дж}}{\text{моль}\cdot\text{К}} \cdot 120\,\text{К}= 89{,}7\,\text{кДж}, \text{газ.}
    \end{align*}
}
\solutionspace{40pt}

\tasknumber{5}%
\task{%
    Как изменилась внутренняя энергия одноатомного идеального газа при переходе из состояния 1 в состояние 2?
    $P_1 = 2\,\text{МПа}$, $V_1 = 3\,\text{л}$, $P_2 = 3{,}5\,\text{МПа}$, $V_2 = 4\,\text{л}$.
    Как изменилась при этом температура газа?
}
\answer{%
    \begin{align*}
    P_1V_1 &= \nu R T_1, P_2V_2 = \nu R T_2, \\
    \Delta U &= U_2-U_1 = \frac 32 \nu R T_2- \frac 32 \nu R T_1 = \frac 32 P_2 V_2 - \frac 32 P_1 V_1= \frac 32 \cdot \cbr{3{,}5\,\text{МПа} \cdot 4\,\text{л} - 2\,\text{МПа} \cdot 3\,\text{л}} = 12000\,\text{Дж}.
    \\
    \frac{T_2}{T_1} &= \frac{\frac{P_2V_2}{\nu R}}{\frac{P_1V_1}{\nu R}} = \frac{P_2V_2}{P_1V_1}= \frac{3{,}5\,\text{МПа} \cdot 4\,\text{л}}{2\,\text{МПа} \cdot 3\,\text{л}} \approx 2{,}33.
    \end{align*}
}
\solutionspace{80pt}

\tasknumber{6}%
\task{%
    $5\,\text{моль}$ идеального одноатомного газа охладили на $30\,\text{К}$.
    Определите изменение внутренней энергии газа.
    Увеличилась она или уменьшилась?
    Универсальная газовая постоянная $R = 8{,}31\,\frac{\text{Дж}}{\text{моль}\cdot\text{К}}$.
}
\answer{%
    $
        \Delta U = \frac 32 \nu R \Delta T
            = - \frac 32 \cdot 5\,\text{моль} \cdot 8{,}31\,\frac{\text{Дж}}{\text{моль}\cdot\text{К}} \cdot 30\,\text{К}
            = -1869\,\text{Дж}.
            \text{Уменьшилась.}
    $
}
\solutionspace{40pt}

\tasknumber{7}%
\task{%
    Газу сообщили некоторое количество теплоты,
    при этом треть его он потратил на совершение работы,
    одновременно увеличив свою внутреннюю энергию на $1200\,\text{Дж}$.
    Определите количество теплоты, сообщённое газу.
}
\answer{%
    \begin{align*}
    Q &= A' + \Delta U, A' = \frac 13 Q \implies Q \cdot \cbr{1 - \frac 13} = \Delta U \implies Q = \frac{\Delta U}{1 - \frac 13} = \frac{1200\,\text{Дж}}{1 - \frac 13} \approx 1800\,\text{Дж}.
    \\
    A' &= \frac 13 Q
        = \frac 13 \cdot \frac{\Delta U}{1 - \frac 13}
        = \frac{\Delta U}{3 - 1}
        = \frac{1200\,\text{Дж}}{3 - 1} \approx 600\,\text{Дж}.
    \end{align*}
}
\solutionspace{60pt}

\tasknumber{8}%
\task{%
    В некотором процессе газ совершил работу $200\,\text{Дж}$,
    при этом его внутренняя энергия уменьшилась на $250\,\text{Дж}$.
    Определите количество тепла, переданное при этом процессе газу.
    Явно пропишите, подводили газу тепло или же отводили.
}
\answer{%
    $
        Q = A_\text{газа} + \Delta U, A_\text{газа} = -A_\text{внешняя}
        \implies Q = A_\text{газа} + \Delta U =  200\,\text{Дж} - 250\,\text{Дж} = -50\,\text{Дж}.
        \text{ Отводили.}
    $
}

\variantsplitter

\addpersonalvariant{Дарья Кошман}

\tasknumber{1}%
\task{%
    Напротив физических величин укажите их обозначения и единицы измерения в СИ, а в пункте «г)» запишите физический закон или формулу:
    \begin{enumerate}
        \item изменение внутренней энергии,
        \item работа внешних сил,
        \item молярная теплоёмкость,
        \item внутренняя энергия идеального одноатомного газа.
    \end{enumerate}
}
\solutionspace{20pt}

\tasknumber{2}%
\task{%
    Определите объём идеального одноатомного газа,
    если его внутренняя энергия при давлении $4\,\text{атм}$ составляет $300\,\text{кДж}$.
    $p_{\text{aтм}} = 100\,\text{кПа}$.
}
\answer{%
    $U = \frac 32 \nu R T = \frac 32 PV \implies V = \frac 23 \cdot \frac UP= \frac 23 \cdot \frac{300\,\text{кДж}}{4\,\text{атм}} \approx 0{,}50\,\text{м}^{3}.$
}
\solutionspace{40pt}

\tasknumber{3}%
\task{%
    Газ расширился от $200\,\text{л}$ до $550\,\text{л}$.
    Давление газа при этом оставалось постоянным и равным $1{,}2\,\text{атм}$.
    Определите работу газа, ответ выразите в килоджоулях.
    $p_{\text{aтм}} = 100\,\text{кПа}$.
}
\answer{%
    $A = P\Delta V = P(V_2 - V_1) = 1{,}2\,\text{атм} \cdot \cbr{550\,\text{л} - 200\,\text{л}} = 42{,}0\,\text{кДж}.$
}
\solutionspace{40pt}

\tasknumber{4}%
\task{%
    $50\,\text{моль}$ идеального одноатомного газа в результате адиабатического процесса нагрелись на $25\,\text{К}$.
    Определите работу газа.
    Кто совершил положительную работу: газ или внешние силы?
    Универсальная газовая постоянная $R = 8{,}31\,\frac{\text{Дж}}{\text{моль}\cdot\text{К}}$.
}
\answer{%
    \begin{align*}
    Q &= 0, Q = \Delta U + A_\text{газа} \implies \\
    \implies A_\text{газа} &= - \Delta U = - \frac 32 \nu R \Delta T = - \frac 32 \cdot 50\,\text{моль} \cdot 8{,}31\,\frac{\text{Дж}}{\text{моль}\cdot\text{К}} \cdot 25\,\text{К}= -15{,}600\,\text{кДж}, \text{внешние силы.}
    \end{align*}
}
\solutionspace{40pt}

\tasknumber{5}%
\task{%
    Как изменилась внутренняя энергия одноатомного идеального газа при переходе из состояния 1 в состояние 2?
    $P_1 = 4\,\text{МПа}$, $V_1 = 7\,\text{л}$, $P_2 = 2{,}5\,\text{МПа}$, $V_2 = 4\,\text{л}$.
    Как изменилась при этом температура газа?
}
\answer{%
    \begin{align*}
    P_1V_1 &= \nu R T_1, P_2V_2 = \nu R T_2, \\
    \Delta U &= U_2-U_1 = \frac 32 \nu R T_2- \frac 32 \nu R T_1 = \frac 32 P_2 V_2 - \frac 32 P_1 V_1= \frac 32 \cdot \cbr{2{,}5\,\text{МПа} \cdot 4\,\text{л} - 4\,\text{МПа} \cdot 7\,\text{л}} = -27000\,\text{Дж}.
    \\
    \frac{T_2}{T_1} &= \frac{\frac{P_2V_2}{\nu R}}{\frac{P_1V_1}{\nu R}} = \frac{P_2V_2}{P_1V_1}= \frac{2{,}5\,\text{МПа} \cdot 4\,\text{л}}{4\,\text{МПа} \cdot 7\,\text{л}} \approx 0{,}36.
    \end{align*}
}
\solutionspace{80pt}

\tasknumber{6}%
\task{%
    $4\,\text{моль}$ идеального одноатомного газа нагрели на $20\,\text{К}$.
    Определите изменение внутренней энергии газа.
    Увеличилась она или уменьшилась?
    Универсальная газовая постоянная $R = 8{,}31\,\frac{\text{Дж}}{\text{моль}\cdot\text{К}}$.
}
\answer{%
    $
        \Delta U = \frac 32 \nu R \Delta T
            =  \frac 32 \cdot 4\,\text{моль} \cdot 8{,}31\,\frac{\text{Дж}}{\text{моль}\cdot\text{К}} \cdot 20\,\text{К}
            = 997\,\text{Дж}.
            \text{Увеличилась.}
    $
}
\solutionspace{40pt}

\tasknumber{7}%
\task{%
    Газу сообщили некоторое количество теплоты,
    при этом половину его он потратил на совершение работы,
    одновременно увеличив свою внутреннюю энергию на $1500\,\text{Дж}$.
    Определите количество теплоты, сообщённое газу.
}
\answer{%
    \begin{align*}
    Q &= A' + \Delta U, A' = \frac 12 Q \implies Q \cdot \cbr{1 - \frac 12} = \Delta U \implies Q = \frac{\Delta U}{1 - \frac 12} = \frac{1500\,\text{Дж}}{1 - \frac 12} \approx 3000\,\text{Дж}.
    \\
    A' &= \frac 12 Q
        = \frac 12 \cdot \frac{\Delta U}{1 - \frac 12}
        = \frac{\Delta U}{2 - 1}
        = \frac{1500\,\text{Дж}}{2 - 1} \approx 1500\,\text{Дж}.
    \end{align*}
}
\solutionspace{60pt}

\tasknumber{8}%
\task{%
    В некотором процессе газ совершил работу $100\,\text{Дж}$,
    при этом его внутренняя энергия уменьшилась на $450\,\text{Дж}$.
    Определите количество тепла, переданное при этом процессе газу.
    Явно пропишите, подводили газу тепло или же отводили.
}
\answer{%
    $
        Q = A_\text{газа} + \Delta U, A_\text{газа} = -A_\text{внешняя}
        \implies Q = A_\text{газа} + \Delta U =  100\,\text{Дж} - 450\,\text{Дж} = -350\,\text{Дж}.
        \text{ Отводили.}
    $
}

\variantsplitter

\addpersonalvariant{Анна Кузьмичёва}

\tasknumber{1}%
\task{%
    Напротив физических величин укажите их обозначения и единицы измерения в СИ, а в пункте «г)» запишите физический закон или формулу:
    \begin{enumerate}
        \item изменение внутренней энергии,
        \item работа внешних сил,
        \item молярная теплоёмкость,
        \item внутренняя энергия идеального одноатомного газа.
    \end{enumerate}
}
\solutionspace{20pt}

\tasknumber{2}%
\task{%
    Определите объём идеального одноатомного газа,
    если его внутренняя энергия при давлении $6\,\text{атм}$ составляет $300\,\text{кДж}$.
    $p_{\text{aтм}} = 100\,\text{кПа}$.
}
\answer{%
    $U = \frac 32 \nu R T = \frac 32 PV \implies V = \frac 23 \cdot \frac UP= \frac 23 \cdot \frac{300\,\text{кДж}}{6\,\text{атм}} \approx 0{,}33\,\text{м}^{3}.$
}
\solutionspace{40pt}

\tasknumber{3}%
\task{%
    Газ расширился от $150\,\text{л}$ до $650\,\text{л}$.
    Давление газа при этом оставалось постоянным и равным $1{,}2\,\text{атм}$.
    Определите работу газа, ответ выразите в килоджоулях.
    $p_{\text{aтм}} = 100\,\text{кПа}$.
}
\answer{%
    $A = P\Delta V = P(V_2 - V_1) = 1{,}2\,\text{атм} \cdot \cbr{650\,\text{л} - 150\,\text{л}} = 60{,}0\,\text{кДж}.$
}
\solutionspace{40pt}

\tasknumber{4}%
\task{%
    $60\,\text{моль}$ идеального одноатомного газа в результате адиабатического процесса остыли на $25\,\text{К}$.
    Определите работу газа.
    Кто совершил положительную работу: газ или внешние силы?
    Универсальная газовая постоянная $R = 8{,}31\,\frac{\text{Дж}}{\text{моль}\cdot\text{К}}$.
}
\answer{%
    \begin{align*}
    Q &= 0, Q = \Delta U + A_\text{газа} \implies \\
    \implies A_\text{газа} &= - \Delta U = - \frac 32 \nu R \Delta T =  \frac 32 \cdot 60\,\text{моль} \cdot 8{,}31\,\frac{\text{Дж}}{\text{моль}\cdot\text{К}} \cdot 25\,\text{К}= 18{,}7\,\text{кДж}, \text{газ.}
    \end{align*}
}
\solutionspace{40pt}

\tasknumber{5}%
\task{%
    Как изменилась внутренняя энергия одноатомного идеального газа при переходе из состояния 1 в состояние 2?
    $P_1 = 4\,\text{МПа}$, $V_1 = 7\,\text{л}$, $P_2 = 4{,}5\,\text{МПа}$, $V_2 = 2\,\text{л}$.
    Как изменилась при этом температура газа?
}
\answer{%
    \begin{align*}
    P_1V_1 &= \nu R T_1, P_2V_2 = \nu R T_2, \\
    \Delta U &= U_2-U_1 = \frac 32 \nu R T_2- \frac 32 \nu R T_1 = \frac 32 P_2 V_2 - \frac 32 P_1 V_1= \frac 32 \cdot \cbr{4{,}5\,\text{МПа} \cdot 2\,\text{л} - 4\,\text{МПа} \cdot 7\,\text{л}} = -28500\,\text{Дж}.
    \\
    \frac{T_2}{T_1} &= \frac{\frac{P_2V_2}{\nu R}}{\frac{P_1V_1}{\nu R}} = \frac{P_2V_2}{P_1V_1}= \frac{4{,}5\,\text{МПа} \cdot 2\,\text{л}}{4\,\text{МПа} \cdot 7\,\text{л}} \approx 0{,}32.
    \end{align*}
}
\solutionspace{80pt}

\tasknumber{6}%
\task{%
    $5\,\text{моль}$ идеального одноатомного газа охладили на $20\,\text{К}$.
    Определите изменение внутренней энергии газа.
    Увеличилась она или уменьшилась?
    Универсальная газовая постоянная $R = 8{,}31\,\frac{\text{Дж}}{\text{моль}\cdot\text{К}}$.
}
\answer{%
    $
        \Delta U = \frac 32 \nu R \Delta T
            = - \frac 32 \cdot 5\,\text{моль} \cdot 8{,}31\,\frac{\text{Дж}}{\text{моль}\cdot\text{К}} \cdot 20\,\text{К}
            = -1246\,\text{Дж}.
            \text{Уменьшилась.}
    $
}
\solutionspace{40pt}

\tasknumber{7}%
\task{%
    Газу сообщили некоторое количество теплоты,
    при этом треть его он потратил на совершение работы,
    одновременно увеличив свою внутреннюю энергию на $2400\,\text{Дж}$.
    Определите количество теплоты, сообщённое газу.
}
\answer{%
    \begin{align*}
    Q &= A' + \Delta U, A' = \frac 13 Q \implies Q \cdot \cbr{1 - \frac 13} = \Delta U \implies Q = \frac{\Delta U}{1 - \frac 13} = \frac{2400\,\text{Дж}}{1 - \frac 13} \approx 3600\,\text{Дж}.
    \\
    A' &= \frac 13 Q
        = \frac 13 \cdot \frac{\Delta U}{1 - \frac 13}
        = \frac{\Delta U}{3 - 1}
        = \frac{2400\,\text{Дж}}{3 - 1} \approx 1200\,\text{Дж}.
    \end{align*}
}
\solutionspace{60pt}

\tasknumber{8}%
\task{%
    В некотором процессе внешние силы совершили над газом работу $200\,\text{Дж}$,
    при этом его внутренняя энергия увеличилась на $150\,\text{Дж}$.
    Определите количество тепла, переданное при этом процессе газу.
    Явно пропишите, подводили газу тепло или же отводили.
}
\answer{%
    $
        Q = A_\text{газа} + \Delta U, A_\text{газа} = -A_\text{внешняя}
        \implies Q = A_\text{газа} + \Delta U = - 200\,\text{Дж} +  150\,\text{Дж} = -50\,\text{Дж}.
        \text{ Отводили.}
    $
}

\variantsplitter

\addpersonalvariant{Алёна Куприянова}

\tasknumber{1}%
\task{%
    Напротив физических величин укажите их обозначения и единицы измерения в СИ, а в пункте «г)» запишите физический закон или формулу:
    \begin{enumerate}
        \item изменение внутренней энергии,
        \item работа внешних сил,
        \item удельная теплоёмкость,
        \item внутренняя энергия идеального одноатомного газа.
    \end{enumerate}
}
\solutionspace{20pt}

\tasknumber{2}%
\task{%
    Определите объём идеального одноатомного газа,
    если его внутренняя энергия при давлении $3\,\text{атм}$ составляет $250\,\text{кДж}$.
    $p_{\text{aтм}} = 100\,\text{кПа}$.
}
\answer{%
    $U = \frac 32 \nu R T = \frac 32 PV \implies V = \frac 23 \cdot \frac UP= \frac 23 \cdot \frac{250\,\text{кДж}}{3\,\text{атм}} \approx 0{,}56\,\text{м}^{3}.$
}
\solutionspace{40pt}

\tasknumber{3}%
\task{%
    Газ расширился от $200\,\text{л}$ до $650\,\text{л}$.
    Давление газа при этом оставалось постоянным и равным $1{,}5\,\text{атм}$.
    Определите работу газа, ответ выразите в килоджоулях.
    $p_{\text{aтм}} = 100\,\text{кПа}$.
}
\answer{%
    $A = P\Delta V = P(V_2 - V_1) = 1{,}5\,\text{атм} \cdot \cbr{650\,\text{л} - 200\,\text{л}} = 67{,}5\,\text{кДж}.$
}
\solutionspace{40pt}

\tasknumber{4}%
\task{%
    $40\,\text{моль}$ идеального одноатомного газа в результате адиабатического процесса остыли на $120\,\text{К}$.
    Определите работу газа.
    Кто совершил положительную работу: газ или внешние силы?
    Универсальная газовая постоянная $R = 8{,}31\,\frac{\text{Дж}}{\text{моль}\cdot\text{К}}$.
}
\answer{%
    \begin{align*}
    Q &= 0, Q = \Delta U + A_\text{газа} \implies \\
    \implies A_\text{газа} &= - \Delta U = - \frac 32 \nu R \Delta T =  \frac 32 \cdot 40\,\text{моль} \cdot 8{,}31\,\frac{\text{Дж}}{\text{моль}\cdot\text{К}} \cdot 120\,\text{К}= 59{,}8\,\text{кДж}, \text{газ.}
    \end{align*}
}
\solutionspace{40pt}

\tasknumber{5}%
\task{%
    Как изменилась внутренняя энергия одноатомного идеального газа при переходе из состояния 1 в состояние 2?
    $P_1 = 2\,\text{МПа}$, $V_1 = 5\,\text{л}$, $P_2 = 4{,}5\,\text{МПа}$, $V_2 = 6\,\text{л}$.
    Как изменилась при этом температура газа?
}
\answer{%
    \begin{align*}
    P_1V_1 &= \nu R T_1, P_2V_2 = \nu R T_2, \\
    \Delta U &= U_2-U_1 = \frac 32 \nu R T_2- \frac 32 \nu R T_1 = \frac 32 P_2 V_2 - \frac 32 P_1 V_1= \frac 32 \cdot \cbr{4{,}5\,\text{МПа} \cdot 6\,\text{л} - 2\,\text{МПа} \cdot 5\,\text{л}} = 25500\,\text{Дж}.
    \\
    \frac{T_2}{T_1} &= \frac{\frac{P_2V_2}{\nu R}}{\frac{P_1V_1}{\nu R}} = \frac{P_2V_2}{P_1V_1}= \frac{4{,}5\,\text{МПа} \cdot 6\,\text{л}}{2\,\text{МПа} \cdot 5\,\text{л}} \approx 2{,}70.
    \end{align*}
}
\solutionspace{80pt}

\tasknumber{6}%
\task{%
    $3\,\text{моль}$ идеального одноатомного газа охладили на $30\,\text{К}$.
    Определите изменение внутренней энергии газа.
    Увеличилась она или уменьшилась?
    Универсальная газовая постоянная $R = 8{,}31\,\frac{\text{Дж}}{\text{моль}\cdot\text{К}}$.
}
\answer{%
    $
        \Delta U = \frac 32 \nu R \Delta T
            = - \frac 32 \cdot 3\,\text{моль} \cdot 8{,}31\,\frac{\text{Дж}}{\text{моль}\cdot\text{К}} \cdot 30\,\text{К}
            = -1121\,\text{Дж}.
            \text{Уменьшилась.}
    $
}
\solutionspace{40pt}

\tasknumber{7}%
\task{%
    Газу сообщили некоторое количество теплоты,
    при этом четверть его он потратил на совершение работы,
    одновременно увеличив свою внутреннюю энергию на $1500\,\text{Дж}$.
    Определите количество теплоты, сообщённое газу.
}
\answer{%
    \begin{align*}
    Q &= A' + \Delta U, A' = \frac 14 Q \implies Q \cdot \cbr{1 - \frac 14} = \Delta U \implies Q = \frac{\Delta U}{1 - \frac 14} = \frac{1500\,\text{Дж}}{1 - \frac 14} \approx 2000\,\text{Дж}.
    \\
    A' &= \frac 14 Q
        = \frac 14 \cdot \frac{\Delta U}{1 - \frac 14}
        = \frac{\Delta U}{4 - 1}
        = \frac{1500\,\text{Дж}}{4 - 1} \approx 500\,\text{Дж}.
    \end{align*}
}
\solutionspace{60pt}

\tasknumber{8}%
\task{%
    В некотором процессе внешние силы совершили над газом работу $100\,\text{Дж}$,
    при этом его внутренняя энергия увеличилась на $150\,\text{Дж}$.
    Определите количество тепла, переданное при этом процессе газу.
    Явно пропишите, подводили газу тепло или же отводили.
}
\answer{%
    $
        Q = A_\text{газа} + \Delta U, A_\text{газа} = -A_\text{внешняя}
        \implies Q = A_\text{газа} + \Delta U = - 100\,\text{Дж} +  150\,\text{Дж} = 50\,\text{Дж}.
        \text{ Подводили.}
    $
}

\variantsplitter

\addpersonalvariant{Ярослав Лавровский}

\tasknumber{1}%
\task{%
    Напротив физических величин укажите их обозначения и единицы измерения в СИ, а в пункте «г)» запишите физический закон или формулу:
    \begin{enumerate}
        \item количество теплоты,
        \item работа внешних сил,
        \item молярная теплоёмкость,
        \item внутренняя энергия идеального одноатомного газа.
    \end{enumerate}
}
\solutionspace{20pt}

\tasknumber{2}%
\task{%
    Определите объём идеального одноатомного газа,
    если его внутренняя энергия при давлении $2\,\text{атм}$ составляет $250\,\text{кДж}$.
    $p_{\text{aтм}} = 100\,\text{кПа}$.
}
\answer{%
    $U = \frac 32 \nu R T = \frac 32 PV \implies V = \frac 23 \cdot \frac UP= \frac 23 \cdot \frac{250\,\text{кДж}}{2\,\text{атм}} \approx 0{,}83\,\text{м}^{3}.$
}
\solutionspace{40pt}

\tasknumber{3}%
\task{%
    Газ расширился от $150\,\text{л}$ до $550\,\text{л}$.
    Давление газа при этом оставалось постоянным и равным $1{,}5\,\text{атм}$.
    Определите работу газа, ответ выразите в килоджоулях.
    $p_{\text{aтм}} = 100\,\text{кПа}$.
}
\answer{%
    $A = P\Delta V = P(V_2 - V_1) = 1{,}5\,\text{атм} \cdot \cbr{550\,\text{л} - 150\,\text{л}} = 60{,}0\,\text{кДж}.$
}
\solutionspace{40pt}

\tasknumber{4}%
\task{%
    $40\,\text{моль}$ идеального одноатомного газа в результате адиабатического процесса нагрелись на $25\,\text{К}$.
    Определите работу газа.
    Кто совершил положительную работу: газ или внешние силы?
    Универсальная газовая постоянная $R = 8{,}31\,\frac{\text{Дж}}{\text{моль}\cdot\text{К}}$.
}
\answer{%
    \begin{align*}
    Q &= 0, Q = \Delta U + A_\text{газа} \implies \\
    \implies A_\text{газа} &= - \Delta U = - \frac 32 \nu R \Delta T = - \frac 32 \cdot 40\,\text{моль} \cdot 8{,}31\,\frac{\text{Дж}}{\text{моль}\cdot\text{К}} \cdot 25\,\text{К}= -12{,}500\,\text{кДж}, \text{внешние силы.}
    \end{align*}
}
\solutionspace{40pt}

\tasknumber{5}%
\task{%
    Как изменилась внутренняя энергия одноатомного идеального газа при переходе из состояния 1 в состояние 2?
    $P_1 = 2\,\text{МПа}$, $V_1 = 3\,\text{л}$, $P_2 = 4{,}5\,\text{МПа}$, $V_2 = 4\,\text{л}$.
    Как изменилась при этом температура газа?
}
\answer{%
    \begin{align*}
    P_1V_1 &= \nu R T_1, P_2V_2 = \nu R T_2, \\
    \Delta U &= U_2-U_1 = \frac 32 \nu R T_2- \frac 32 \nu R T_1 = \frac 32 P_2 V_2 - \frac 32 P_1 V_1= \frac 32 \cdot \cbr{4{,}5\,\text{МПа} \cdot 4\,\text{л} - 2\,\text{МПа} \cdot 3\,\text{л}} = 18000\,\text{Дж}.
    \\
    \frac{T_2}{T_1} &= \frac{\frac{P_2V_2}{\nu R}}{\frac{P_1V_1}{\nu R}} = \frac{P_2V_2}{P_1V_1}= \frac{4{,}5\,\text{МПа} \cdot 4\,\text{л}}{2\,\text{МПа} \cdot 3\,\text{л}} \approx 3{,}00.
    \end{align*}
}
\solutionspace{80pt}

\tasknumber{6}%
\task{%
    $3\,\text{моль}$ идеального одноатомного газа нагрели на $20\,\text{К}$.
    Определите изменение внутренней энергии газа.
    Увеличилась она или уменьшилась?
    Универсальная газовая постоянная $R = 8{,}31\,\frac{\text{Дж}}{\text{моль}\cdot\text{К}}$.
}
\answer{%
    $
        \Delta U = \frac 32 \nu R \Delta T
            =  \frac 32 \cdot 3\,\text{моль} \cdot 8{,}31\,\frac{\text{Дж}}{\text{моль}\cdot\text{К}} \cdot 20\,\text{К}
            = 747\,\text{Дж}.
            \text{Увеличилась.}
    $
}
\solutionspace{40pt}

\tasknumber{7}%
\task{%
    Газу сообщили некоторое количество теплоты,
    при этом половину его он потратил на совершение работы,
    одновременно увеличив свою внутреннюю энергию на $1200\,\text{Дж}$.
    Определите работу, совершённую газом.
}
\answer{%
    \begin{align*}
    Q &= A' + \Delta U, A' = \frac 12 Q \implies Q \cdot \cbr{1 - \frac 12} = \Delta U \implies Q = \frac{\Delta U}{1 - \frac 12} = \frac{1200\,\text{Дж}}{1 - \frac 12} \approx 2400\,\text{Дж}.
    \\
    A' &= \frac 12 Q
        = \frac 12 \cdot \frac{\Delta U}{1 - \frac 12}
        = \frac{\Delta U}{2 - 1}
        = \frac{1200\,\text{Дж}}{2 - 1} \approx 1200\,\text{Дж}.
    \end{align*}
}
\solutionspace{60pt}

\tasknumber{8}%
\task{%
    В некотором процессе газ совершил работу $200\,\text{Дж}$,
    при этом его внутренняя энергия уменьшилась на $350\,\text{Дж}$.
    Определите количество тепла, переданное при этом процессе газу.
    Явно пропишите, подводили газу тепло или же отводили.
}
\answer{%
    $
        Q = A_\text{газа} + \Delta U, A_\text{газа} = -A_\text{внешняя}
        \implies Q = A_\text{газа} + \Delta U =  200\,\text{Дж} - 350\,\text{Дж} = -150\,\text{Дж}.
        \text{ Отводили.}
    $
}

\variantsplitter

\addpersonalvariant{Анастасия Ламанова}

\tasknumber{1}%
\task{%
    Напротив физических величин укажите их обозначения и единицы измерения в СИ, а в пункте «г)» запишите физический закон или формулу:
    \begin{enumerate}
        \item количество теплоты,
        \item работа газа,
        \item молярная теплоёмкость,
        \item первое начало термодинамики.
    \end{enumerate}
}
\solutionspace{20pt}

\tasknumber{2}%
\task{%
    Определите объём идеального одноатомного газа,
    если его внутренняя энергия при давлении $2\,\text{атм}$ составляет $400\,\text{кДж}$.
    $p_{\text{aтм}} = 100\,\text{кПа}$.
}
\answer{%
    $U = \frac 32 \nu R T = \frac 32 PV \implies V = \frac 23 \cdot \frac UP= \frac 23 \cdot \frac{400\,\text{кДж}}{2\,\text{атм}} \approx 1{,}33\,\text{м}^{3}.$
}
\solutionspace{40pt}

\tasknumber{3}%
\task{%
    Газ расширился от $250\,\text{л}$ до $650\,\text{л}$.
    Давление газа при этом оставалось постоянным и равным $1{,}5\,\text{атм}$.
    Определите работу газа, ответ выразите в килоджоулях.
    $p_{\text{aтм}} = 100\,\text{кПа}$.
}
\answer{%
    $A = P\Delta V = P(V_2 - V_1) = 1{,}5\,\text{атм} \cdot \cbr{650\,\text{л} - 250\,\text{л}} = 60{,}0\,\text{кДж}.$
}
\solutionspace{40pt}

\tasknumber{4}%
\task{%
    $40\,\text{моль}$ идеального одноатомного газа в результате адиабатического процесса нагрелись на $80\,\text{К}$.
    Определите работу газа.
    Кто совершил положительную работу: газ или внешние силы?
    Универсальная газовая постоянная $R = 8{,}31\,\frac{\text{Дж}}{\text{моль}\cdot\text{К}}$.
}
\answer{%
    \begin{align*}
    Q &= 0, Q = \Delta U + A_\text{газа} \implies \\
    \implies A_\text{газа} &= - \Delta U = - \frac 32 \nu R \Delta T = - \frac 32 \cdot 40\,\text{моль} \cdot 8{,}31\,\frac{\text{Дж}}{\text{моль}\cdot\text{К}} \cdot 80\,\text{К}= -39{,}90\,\text{кДж}, \text{внешние силы.}
    \end{align*}
}
\solutionspace{40pt}

\tasknumber{5}%
\task{%
    Как изменилась внутренняя энергия одноатомного идеального газа при переходе из состояния 1 в состояние 2?
    $P_1 = 2\,\text{МПа}$, $V_1 = 5\,\text{л}$, $P_2 = 1{,}5\,\text{МПа}$, $V_2 = 2\,\text{л}$.
    Как изменилась при этом температура газа?
}
\answer{%
    \begin{align*}
    P_1V_1 &= \nu R T_1, P_2V_2 = \nu R T_2, \\
    \Delta U &= U_2-U_1 = \frac 32 \nu R T_2- \frac 32 \nu R T_1 = \frac 32 P_2 V_2 - \frac 32 P_1 V_1= \frac 32 \cdot \cbr{1{,}5\,\text{МПа} \cdot 2\,\text{л} - 2\,\text{МПа} \cdot 5\,\text{л}} = -10500\,\text{Дж}.
    \\
    \frac{T_2}{T_1} &= \frac{\frac{P_2V_2}{\nu R}}{\frac{P_1V_1}{\nu R}} = \frac{P_2V_2}{P_1V_1}= \frac{1{,}5\,\text{МПа} \cdot 2\,\text{л}}{2\,\text{МПа} \cdot 5\,\text{л}} \approx 0{,}30.
    \end{align*}
}
\solutionspace{80pt}

\tasknumber{6}%
\task{%
    $3\,\text{моль}$ идеального одноатомного газа нагрели на $20\,\text{К}$.
    Определите изменение внутренней энергии газа.
    Увеличилась она или уменьшилась?
    Универсальная газовая постоянная $R = 8{,}31\,\frac{\text{Дж}}{\text{моль}\cdot\text{К}}$.
}
\answer{%
    $
        \Delta U = \frac 32 \nu R \Delta T
            =  \frac 32 \cdot 3\,\text{моль} \cdot 8{,}31\,\frac{\text{Дж}}{\text{моль}\cdot\text{К}} \cdot 20\,\text{К}
            = 747\,\text{Дж}.
            \text{Увеличилась.}
    $
}
\solutionspace{40pt}

\tasknumber{7}%
\task{%
    Газу сообщили некоторое количество теплоты,
    при этом треть его он потратил на совершение работы,
    одновременно увеличив свою внутреннюю энергию на $2400\,\text{Дж}$.
    Определите количество теплоты, сообщённое газу.
}
\answer{%
    \begin{align*}
    Q &= A' + \Delta U, A' = \frac 13 Q \implies Q \cdot \cbr{1 - \frac 13} = \Delta U \implies Q = \frac{\Delta U}{1 - \frac 13} = \frac{2400\,\text{Дж}}{1 - \frac 13} \approx 3600\,\text{Дж}.
    \\
    A' &= \frac 13 Q
        = \frac 13 \cdot \frac{\Delta U}{1 - \frac 13}
        = \frac{\Delta U}{3 - 1}
        = \frac{2400\,\text{Дж}}{3 - 1} \approx 1200\,\text{Дж}.
    \end{align*}
}
\solutionspace{60pt}

\tasknumber{8}%
\task{%
    В некотором процессе внешние силы совершили над газом работу $100\,\text{Дж}$,
    при этом его внутренняя энергия уменьшилась на $350\,\text{Дж}$.
    Определите количество тепла, переданное при этом процессе газу.
    Явно пропишите, подводили газу тепло или же отводили.
}
\answer{%
    $
        Q = A_\text{газа} + \Delta U, A_\text{газа} = -A_\text{внешняя}
        \implies Q = A_\text{газа} + \Delta U = - 100\,\text{Дж} - 350\,\text{Дж} = -450\,\text{Дж}.
        \text{ Отводили.}
    $
}

\variantsplitter

\addpersonalvariant{Виктория Легонькова}

\tasknumber{1}%
\task{%
    Напротив физических величин укажите их обозначения и единицы измерения в СИ, а в пункте «г)» запишите физический закон или формулу:
    \begin{enumerate}
        \item изменение внутренней энергии,
        \item работа газа,
        \item удельная теплоёмкость,
        \item внутренняя энергия идеального одноатомного газа.
    \end{enumerate}
}
\solutionspace{20pt}

\tasknumber{2}%
\task{%
    Определите объём идеального одноатомного газа,
    если его внутренняя энергия при давлении $3\,\text{атм}$ составляет $300\,\text{кДж}$.
    $p_{\text{aтм}} = 100\,\text{кПа}$.
}
\answer{%
    $U = \frac 32 \nu R T = \frac 32 PV \implies V = \frac 23 \cdot \frac UP= \frac 23 \cdot \frac{300\,\text{кДж}}{3\,\text{атм}} \approx 0{,}67\,\text{м}^{3}.$
}
\solutionspace{40pt}

\tasknumber{3}%
\task{%
    Газ расширился от $200\,\text{л}$ до $450\,\text{л}$.
    Давление газа при этом оставалось постоянным и равным $1{,}5\,\text{атм}$.
    Определите работу газа, ответ выразите в килоджоулях.
    $p_{\text{aтм}} = 100\,\text{кПа}$.
}
\answer{%
    $A = P\Delta V = P(V_2 - V_1) = 1{,}5\,\text{атм} \cdot \cbr{450\,\text{л} - 200\,\text{л}} = 37{,}5\,\text{кДж}.$
}
\solutionspace{40pt}

\tasknumber{4}%
\task{%
    $40\,\text{моль}$ идеального одноатомного газа в результате адиабатического процесса нагрелись на $45\,\text{К}$.
    Определите работу газа.
    Кто совершил положительную работу: газ или внешние силы?
    Универсальная газовая постоянная $R = 8{,}31\,\frac{\text{Дж}}{\text{моль}\cdot\text{К}}$.
}
\answer{%
    \begin{align*}
    Q &= 0, Q = \Delta U + A_\text{газа} \implies \\
    \implies A_\text{газа} &= - \Delta U = - \frac 32 \nu R \Delta T = - \frac 32 \cdot 40\,\text{моль} \cdot 8{,}31\,\frac{\text{Дж}}{\text{моль}\cdot\text{К}} \cdot 45\,\text{К}= -22{,}40\,\text{кДж}, \text{внешние силы.}
    \end{align*}
}
\solutionspace{40pt}

\tasknumber{5}%
\task{%
    Как изменилась внутренняя энергия одноатомного идеального газа при переходе из состояния 1 в состояние 2?
    $P_1 = 4\,\text{МПа}$, $V_1 = 3\,\text{л}$, $P_2 = 4{,}5\,\text{МПа}$, $V_2 = 8\,\text{л}$.
    Как изменилась при этом температура газа?
}
\answer{%
    \begin{align*}
    P_1V_1 &= \nu R T_1, P_2V_2 = \nu R T_2, \\
    \Delta U &= U_2-U_1 = \frac 32 \nu R T_2- \frac 32 \nu R T_1 = \frac 32 P_2 V_2 - \frac 32 P_1 V_1= \frac 32 \cdot \cbr{4{,}5\,\text{МПа} \cdot 8\,\text{л} - 4\,\text{МПа} \cdot 3\,\text{л}} = 36000\,\text{Дж}.
    \\
    \frac{T_2}{T_1} &= \frac{\frac{P_2V_2}{\nu R}}{\frac{P_1V_1}{\nu R}} = \frac{P_2V_2}{P_1V_1}= \frac{4{,}5\,\text{МПа} \cdot 8\,\text{л}}{4\,\text{МПа} \cdot 3\,\text{л}} \approx 3{,}00.
    \end{align*}
}
\solutionspace{80pt}

\tasknumber{6}%
\task{%
    $3\,\text{моль}$ идеального одноатомного газа нагрели на $30\,\text{К}$.
    Определите изменение внутренней энергии газа.
    Увеличилась она или уменьшилась?
    Универсальная газовая постоянная $R = 8{,}31\,\frac{\text{Дж}}{\text{моль}\cdot\text{К}}$.
}
\answer{%
    $
        \Delta U = \frac 32 \nu R \Delta T
            =  \frac 32 \cdot 3\,\text{моль} \cdot 8{,}31\,\frac{\text{Дж}}{\text{моль}\cdot\text{К}} \cdot 30\,\text{К}
            = 1121\,\text{Дж}.
            \text{Увеличилась.}
    $
}
\solutionspace{40pt}

\tasknumber{7}%
\task{%
    Газу сообщили некоторое количество теплоты,
    при этом половину его он потратил на совершение работы,
    одновременно увеличив свою внутреннюю энергию на $3000\,\text{Дж}$.
    Определите работу, совершённую газом.
}
\answer{%
    \begin{align*}
    Q &= A' + \Delta U, A' = \frac 12 Q \implies Q \cdot \cbr{1 - \frac 12} = \Delta U \implies Q = \frac{\Delta U}{1 - \frac 12} = \frac{3000\,\text{Дж}}{1 - \frac 12} \approx 6000\,\text{Дж}.
    \\
    A' &= \frac 12 Q
        = \frac 12 \cdot \frac{\Delta U}{1 - \frac 12}
        = \frac{\Delta U}{2 - 1}
        = \frac{3000\,\text{Дж}}{2 - 1} \approx 3000\,\text{Дж}.
    \end{align*}
}
\solutionspace{60pt}

\tasknumber{8}%
\task{%
    В некотором процессе газ совершил работу $200\,\text{Дж}$,
    при этом его внутренняя энергия увеличилась на $350\,\text{Дж}$.
    Определите количество тепла, переданное при этом процессе газу.
    Явно пропишите, подводили газу тепло или же отводили.
}
\answer{%
    $
        Q = A_\text{газа} + \Delta U, A_\text{газа} = -A_\text{внешняя}
        \implies Q = A_\text{газа} + \Delta U =  200\,\text{Дж} +  350\,\text{Дж} = 550\,\text{Дж}.
        \text{ Подводили.}
    $
}

\variantsplitter

\addpersonalvariant{Семён Мартынов}

\tasknumber{1}%
\task{%
    Напротив физических величин укажите их обозначения и единицы измерения в СИ, а в пункте «г)» запишите физический закон или формулу:
    \begin{enumerate}
        \item изменение внутренней энергии,
        \item работа внешних сил,
        \item молярная теплоёмкость,
        \item первое начало термодинамики.
    \end{enumerate}
}
\solutionspace{20pt}

\tasknumber{2}%
\task{%
    Определите объём идеального одноатомного газа,
    если его внутренняя энергия при давлении $5\,\text{атм}$ составляет $300\,\text{кДж}$.
    $p_{\text{aтм}} = 100\,\text{кПа}$.
}
\answer{%
    $U = \frac 32 \nu R T = \frac 32 PV \implies V = \frac 23 \cdot \frac UP= \frac 23 \cdot \frac{300\,\text{кДж}}{5\,\text{атм}} \approx 0{,}40\,\text{м}^{3}.$
}
\solutionspace{40pt}

\tasknumber{3}%
\task{%
    Газ расширился от $200\,\text{л}$ до $650\,\text{л}$.
    Давление газа при этом оставалось постоянным и равным $1{,}2\,\text{атм}$.
    Определите работу газа, ответ выразите в килоджоулях.
    $p_{\text{aтм}} = 100\,\text{кПа}$.
}
\answer{%
    $A = P\Delta V = P(V_2 - V_1) = 1{,}2\,\text{атм} \cdot \cbr{650\,\text{л} - 200\,\text{л}} = 54{,}0\,\text{кДж}.$
}
\solutionspace{40pt}

\tasknumber{4}%
\task{%
    $50\,\text{моль}$ идеального одноатомного газа в результате адиабатического процесса нагрелись на $60\,\text{К}$.
    Определите работу газа.
    Кто совершил положительную работу: газ или внешние силы?
    Универсальная газовая постоянная $R = 8{,}31\,\frac{\text{Дж}}{\text{моль}\cdot\text{К}}$.
}
\answer{%
    \begin{align*}
    Q &= 0, Q = \Delta U + A_\text{газа} \implies \\
    \implies A_\text{газа} &= - \Delta U = - \frac 32 \nu R \Delta T = - \frac 32 \cdot 50\,\text{моль} \cdot 8{,}31\,\frac{\text{Дж}}{\text{моль}\cdot\text{К}} \cdot 60\,\text{К}= -37{,}40\,\text{кДж}, \text{внешние силы.}
    \end{align*}
}
\solutionspace{40pt}

\tasknumber{5}%
\task{%
    Как изменилась внутренняя энергия одноатомного идеального газа при переходе из состояния 1 в состояние 2?
    $P_1 = 2\,\text{МПа}$, $V_1 = 7\,\text{л}$, $P_2 = 4{,}5\,\text{МПа}$, $V_2 = 6\,\text{л}$.
    Как изменилась при этом температура газа?
}
\answer{%
    \begin{align*}
    P_1V_1 &= \nu R T_1, P_2V_2 = \nu R T_2, \\
    \Delta U &= U_2-U_1 = \frac 32 \nu R T_2- \frac 32 \nu R T_1 = \frac 32 P_2 V_2 - \frac 32 P_1 V_1= \frac 32 \cdot \cbr{4{,}5\,\text{МПа} \cdot 6\,\text{л} - 2\,\text{МПа} \cdot 7\,\text{л}} = 19500\,\text{Дж}.
    \\
    \frac{T_2}{T_1} &= \frac{\frac{P_2V_2}{\nu R}}{\frac{P_1V_1}{\nu R}} = \frac{P_2V_2}{P_1V_1}= \frac{4{,}5\,\text{МПа} \cdot 6\,\text{л}}{2\,\text{МПа} \cdot 7\,\text{л}} \approx 1{,}93.
    \end{align*}
}
\solutionspace{80pt}

\tasknumber{6}%
\task{%
    $4\,\text{моль}$ идеального одноатомного газа нагрели на $10\,\text{К}$.
    Определите изменение внутренней энергии газа.
    Увеличилась она или уменьшилась?
    Универсальная газовая постоянная $R = 8{,}31\,\frac{\text{Дж}}{\text{моль}\cdot\text{К}}$.
}
\answer{%
    $
        \Delta U = \frac 32 \nu R \Delta T
            =  \frac 32 \cdot 4\,\text{моль} \cdot 8{,}31\,\frac{\text{Дж}}{\text{моль}\cdot\text{К}} \cdot 10\,\text{К}
            = 498\,\text{Дж}.
            \text{Увеличилась.}
    $
}
\solutionspace{40pt}

\tasknumber{7}%
\task{%
    Газу сообщили некоторое количество теплоты,
    при этом четверть его он потратил на совершение работы,
    одновременно увеличив свою внутреннюю энергию на $1200\,\text{Дж}$.
    Определите работу, совершённую газом.
}
\answer{%
    \begin{align*}
    Q &= A' + \Delta U, A' = \frac 14 Q \implies Q \cdot \cbr{1 - \frac 14} = \Delta U \implies Q = \frac{\Delta U}{1 - \frac 14} = \frac{1200\,\text{Дж}}{1 - \frac 14} \approx 1600\,\text{Дж}.
    \\
    A' &= \frac 14 Q
        = \frac 14 \cdot \frac{\Delta U}{1 - \frac 14}
        = \frac{\Delta U}{4 - 1}
        = \frac{1200\,\text{Дж}}{4 - 1} \approx 400\,\text{Дж}.
    \end{align*}
}
\solutionspace{60pt}

\tasknumber{8}%
\task{%
    В некотором процессе внешние силы совершили над газом работу $200\,\text{Дж}$,
    при этом его внутренняя энергия уменьшилась на $250\,\text{Дж}$.
    Определите количество тепла, переданное при этом процессе газу.
    Явно пропишите, подводили газу тепло или же отводили.
}
\answer{%
    $
        Q = A_\text{газа} + \Delta U, A_\text{газа} = -A_\text{внешняя}
        \implies Q = A_\text{газа} + \Delta U = - 200\,\text{Дж} - 250\,\text{Дж} = -450\,\text{Дж}.
        \text{ Отводили.}
    $
}

\variantsplitter

\addpersonalvariant{Варвара Минаева}

\tasknumber{1}%
\task{%
    Напротив физических величин укажите их обозначения и единицы измерения в СИ, а в пункте «г)» запишите физический закон или формулу:
    \begin{enumerate}
        \item количество теплоты,
        \item работа внешних сил,
        \item молярная теплоёмкость,
        \item первое начало термодинамики.
    \end{enumerate}
}
\solutionspace{20pt}

\tasknumber{2}%
\task{%
    Определите объём идеального одноатомного газа,
    если его внутренняя энергия при давлении $3\,\text{атм}$ составляет $500\,\text{кДж}$.
    $p_{\text{aтм}} = 100\,\text{кПа}$.
}
\answer{%
    $U = \frac 32 \nu R T = \frac 32 PV \implies V = \frac 23 \cdot \frac UP= \frac 23 \cdot \frac{500\,\text{кДж}}{3\,\text{атм}} \approx 1{,}11\,\text{м}^{3}.$
}
\solutionspace{40pt}

\tasknumber{3}%
\task{%
    Газ расширился от $200\,\text{л}$ до $550\,\text{л}$.
    Давление газа при этом оставалось постоянным и равным $2{,}5\,\text{атм}$.
    Определите работу газа, ответ выразите в килоджоулях.
    $p_{\text{aтм}} = 100\,\text{кПа}$.
}
\answer{%
    $A = P\Delta V = P(V_2 - V_1) = 2{,}5\,\text{атм} \cdot \cbr{550\,\text{л} - 200\,\text{л}} = 87{,}5\,\text{кДж}.$
}
\solutionspace{40pt}

\tasknumber{4}%
\task{%
    $50\,\text{моль}$ идеального одноатомного газа в результате адиабатического процесса остыли на $80\,\text{К}$.
    Определите работу газа.
    Кто совершил положительную работу: газ или внешние силы?
    Универсальная газовая постоянная $R = 8{,}31\,\frac{\text{Дж}}{\text{моль}\cdot\text{К}}$.
}
\answer{%
    \begin{align*}
    Q &= 0, Q = \Delta U + A_\text{газа} \implies \\
    \implies A_\text{газа} &= - \Delta U = - \frac 32 \nu R \Delta T =  \frac 32 \cdot 50\,\text{моль} \cdot 8{,}31\,\frac{\text{Дж}}{\text{моль}\cdot\text{К}} \cdot 80\,\text{К}= 49{,}9\,\text{кДж}, \text{газ.}
    \end{align*}
}
\solutionspace{40pt}

\tasknumber{5}%
\task{%
    Как изменилась внутренняя энергия одноатомного идеального газа при переходе из состояния 1 в состояние 2?
    $P_1 = 4\,\text{МПа}$, $V_1 = 7\,\text{л}$, $P_2 = 3{,}5\,\text{МПа}$, $V_2 = 4\,\text{л}$.
    Как изменилась при этом температура газа?
}
\answer{%
    \begin{align*}
    P_1V_1 &= \nu R T_1, P_2V_2 = \nu R T_2, \\
    \Delta U &= U_2-U_1 = \frac 32 \nu R T_2- \frac 32 \nu R T_1 = \frac 32 P_2 V_2 - \frac 32 P_1 V_1= \frac 32 \cdot \cbr{3{,}5\,\text{МПа} \cdot 4\,\text{л} - 4\,\text{МПа} \cdot 7\,\text{л}} = -21000\,\text{Дж}.
    \\
    \frac{T_2}{T_1} &= \frac{\frac{P_2V_2}{\nu R}}{\frac{P_1V_1}{\nu R}} = \frac{P_2V_2}{P_1V_1}= \frac{3{,}5\,\text{МПа} \cdot 4\,\text{л}}{4\,\text{МПа} \cdot 7\,\text{л}} \approx 0{,}50.
    \end{align*}
}
\solutionspace{80pt}

\tasknumber{6}%
\task{%
    $4\,\text{моль}$ идеального одноатомного газа охладили на $20\,\text{К}$.
    Определите изменение внутренней энергии газа.
    Увеличилась она или уменьшилась?
    Универсальная газовая постоянная $R = 8{,}31\,\frac{\text{Дж}}{\text{моль}\cdot\text{К}}$.
}
\answer{%
    $
        \Delta U = \frac 32 \nu R \Delta T
            = - \frac 32 \cdot 4\,\text{моль} \cdot 8{,}31\,\frac{\text{Дж}}{\text{моль}\cdot\text{К}} \cdot 20\,\text{К}
            = -997\,\text{Дж}.
            \text{Уменьшилась.}
    $
}
\solutionspace{40pt}

\tasknumber{7}%
\task{%
    Газу сообщили некоторое количество теплоты,
    при этом половину его он потратил на совершение работы,
    одновременно увеличив свою внутреннюю энергию на $2400\,\text{Дж}$.
    Определите количество теплоты, сообщённое газу.
}
\answer{%
    \begin{align*}
    Q &= A' + \Delta U, A' = \frac 12 Q \implies Q \cdot \cbr{1 - \frac 12} = \Delta U \implies Q = \frac{\Delta U}{1 - \frac 12} = \frac{2400\,\text{Дж}}{1 - \frac 12} \approx 4800\,\text{Дж}.
    \\
    A' &= \frac 12 Q
        = \frac 12 \cdot \frac{\Delta U}{1 - \frac 12}
        = \frac{\Delta U}{2 - 1}
        = \frac{2400\,\text{Дж}}{2 - 1} \approx 2400\,\text{Дж}.
    \end{align*}
}
\solutionspace{60pt}

\tasknumber{8}%
\task{%
    В некотором процессе внешние силы совершили над газом работу $200\,\text{Дж}$,
    при этом его внутренняя энергия уменьшилась на $150\,\text{Дж}$.
    Определите количество тепла, переданное при этом процессе газу.
    Явно пропишите, подводили газу тепло или же отводили.
}
\answer{%
    $
        Q = A_\text{газа} + \Delta U, A_\text{газа} = -A_\text{внешняя}
        \implies Q = A_\text{газа} + \Delta U = - 200\,\text{Дж} - 150\,\text{Дж} = -350\,\text{Дж}.
        \text{ Отводили.}
    $
}

\variantsplitter

\addpersonalvariant{Леонид Никитин}

\tasknumber{1}%
\task{%
    Напротив физических величин укажите их обозначения и единицы измерения в СИ, а в пункте «г)» запишите физический закон или формулу:
    \begin{enumerate}
        \item количество теплоты,
        \item работа внешних сил,
        \item удельная теплоёмкость,
        \item внутренняя энергия идеального одноатомного газа.
    \end{enumerate}
}
\solutionspace{20pt}

\tasknumber{2}%
\task{%
    Определите объём идеального одноатомного газа,
    если его внутренняя энергия при давлении $3\,\text{атм}$ составляет $250\,\text{кДж}$.
    $p_{\text{aтм}} = 100\,\text{кПа}$.
}
\answer{%
    $U = \frac 32 \nu R T = \frac 32 PV \implies V = \frac 23 \cdot \frac UP= \frac 23 \cdot \frac{250\,\text{кДж}}{3\,\text{атм}} \approx 0{,}56\,\text{м}^{3}.$
}
\solutionspace{40pt}

\tasknumber{3}%
\task{%
    Газ расширился от $350\,\text{л}$ до $550\,\text{л}$.
    Давление газа при этом оставалось постоянным и равным $3{,}5\,\text{атм}$.
    Определите работу газа, ответ выразите в килоджоулях.
    $p_{\text{aтм}} = 100\,\text{кПа}$.
}
\answer{%
    $A = P\Delta V = P(V_2 - V_1) = 3{,}5\,\text{атм} \cdot \cbr{550\,\text{л} - 350\,\text{л}} = 70{,}0\,\text{кДж}.$
}
\solutionspace{40pt}

\tasknumber{4}%
\task{%
    $60\,\text{моль}$ идеального одноатомного газа в результате адиабатического процесса остыли на $120\,\text{К}$.
    Определите работу газа.
    Кто совершил положительную работу: газ или внешние силы?
    Универсальная газовая постоянная $R = 8{,}31\,\frac{\text{Дж}}{\text{моль}\cdot\text{К}}$.
}
\answer{%
    \begin{align*}
    Q &= 0, Q = \Delta U + A_\text{газа} \implies \\
    \implies A_\text{газа} &= - \Delta U = - \frac 32 \nu R \Delta T =  \frac 32 \cdot 60\,\text{моль} \cdot 8{,}31\,\frac{\text{Дж}}{\text{моль}\cdot\text{К}} \cdot 120\,\text{К}= 89{,}7\,\text{кДж}, \text{газ.}
    \end{align*}
}
\solutionspace{40pt}

\tasknumber{5}%
\task{%
    Как изменилась внутренняя энергия одноатомного идеального газа при переходе из состояния 1 в состояние 2?
    $P_1 = 4\,\text{МПа}$, $V_1 = 3\,\text{л}$, $P_2 = 4{,}5\,\text{МПа}$, $V_2 = 2\,\text{л}$.
    Как изменилась при этом температура газа?
}
\answer{%
    \begin{align*}
    P_1V_1 &= \nu R T_1, P_2V_2 = \nu R T_2, \\
    \Delta U &= U_2-U_1 = \frac 32 \nu R T_2- \frac 32 \nu R T_1 = \frac 32 P_2 V_2 - \frac 32 P_1 V_1= \frac 32 \cdot \cbr{4{,}5\,\text{МПа} \cdot 2\,\text{л} - 4\,\text{МПа} \cdot 3\,\text{л}} = -4500\,\text{Дж}.
    \\
    \frac{T_2}{T_1} &= \frac{\frac{P_2V_2}{\nu R}}{\frac{P_1V_1}{\nu R}} = \frac{P_2V_2}{P_1V_1}= \frac{4{,}5\,\text{МПа} \cdot 2\,\text{л}}{4\,\text{МПа} \cdot 3\,\text{л}} \approx 0{,}75.
    \end{align*}
}
\solutionspace{80pt}

\tasknumber{6}%
\task{%
    $5\,\text{моль}$ идеального одноатомного газа охладили на $30\,\text{К}$.
    Определите изменение внутренней энергии газа.
    Увеличилась она или уменьшилась?
    Универсальная газовая постоянная $R = 8{,}31\,\frac{\text{Дж}}{\text{моль}\cdot\text{К}}$.
}
\answer{%
    $
        \Delta U = \frac 32 \nu R \Delta T
            = - \frac 32 \cdot 5\,\text{моль} \cdot 8{,}31\,\frac{\text{Дж}}{\text{моль}\cdot\text{К}} \cdot 30\,\text{К}
            = -1869\,\text{Дж}.
            \text{Уменьшилась.}
    $
}
\solutionspace{40pt}

\tasknumber{7}%
\task{%
    Газу сообщили некоторое количество теплоты,
    при этом четверть его он потратил на совершение работы,
    одновременно увеличив свою внутреннюю энергию на $1200\,\text{Дж}$.
    Определите работу, совершённую газом.
}
\answer{%
    \begin{align*}
    Q &= A' + \Delta U, A' = \frac 14 Q \implies Q \cdot \cbr{1 - \frac 14} = \Delta U \implies Q = \frac{\Delta U}{1 - \frac 14} = \frac{1200\,\text{Дж}}{1 - \frac 14} \approx 1600\,\text{Дж}.
    \\
    A' &= \frac 14 Q
        = \frac 14 \cdot \frac{\Delta U}{1 - \frac 14}
        = \frac{\Delta U}{4 - 1}
        = \frac{1200\,\text{Дж}}{4 - 1} \approx 400\,\text{Дж}.
    \end{align*}
}
\solutionspace{60pt}

\tasknumber{8}%
\task{%
    В некотором процессе газ совершил работу $100\,\text{Дж}$,
    при этом его внутренняя энергия увеличилась на $450\,\text{Дж}$.
    Определите количество тепла, переданное при этом процессе газу.
    Явно пропишите, подводили газу тепло или же отводили.
}
\answer{%
    $
        Q = A_\text{газа} + \Delta U, A_\text{газа} = -A_\text{внешняя}
        \implies Q = A_\text{газа} + \Delta U =  100\,\text{Дж} +  450\,\text{Дж} = 550\,\text{Дж}.
        \text{ Подводили.}
    $
}

\variantsplitter

\addpersonalvariant{Тимофей Полетаев}

\tasknumber{1}%
\task{%
    Напротив физических величин укажите их обозначения и единицы измерения в СИ, а в пункте «г)» запишите физический закон или формулу:
    \begin{enumerate}
        \item количество теплоты,
        \item работа газа,
        \item удельная теплоёмкость,
        \item внутренняя энергия идеального одноатомного газа.
    \end{enumerate}
}
\solutionspace{20pt}

\tasknumber{2}%
\task{%
    Определите объём идеального одноатомного газа,
    если его внутренняя энергия при давлении $6\,\text{атм}$ составляет $300\,\text{кДж}$.
    $p_{\text{aтм}} = 100\,\text{кПа}$.
}
\answer{%
    $U = \frac 32 \nu R T = \frac 32 PV \implies V = \frac 23 \cdot \frac UP= \frac 23 \cdot \frac{300\,\text{кДж}}{6\,\text{атм}} \approx 0{,}33\,\text{м}^{3}.$
}
\solutionspace{40pt}

\tasknumber{3}%
\task{%
    Газ расширился от $350\,\text{л}$ до $550\,\text{л}$.
    Давление газа при этом оставалось постоянным и равным $1{,}5\,\text{атм}$.
    Определите работу газа, ответ выразите в килоджоулях.
    $p_{\text{aтм}} = 100\,\text{кПа}$.
}
\answer{%
    $A = P\Delta V = P(V_2 - V_1) = 1{,}5\,\text{атм} \cdot \cbr{550\,\text{л} - 350\,\text{л}} = 30{,}0\,\text{кДж}.$
}
\solutionspace{40pt}

\tasknumber{4}%
\task{%
    $30\,\text{моль}$ идеального одноатомного газа в результате адиабатического процесса нагрелись на $80\,\text{К}$.
    Определите работу газа.
    Кто совершил положительную работу: газ или внешние силы?
    Универсальная газовая постоянная $R = 8{,}31\,\frac{\text{Дж}}{\text{моль}\cdot\text{К}}$.
}
\answer{%
    \begin{align*}
    Q &= 0, Q = \Delta U + A_\text{газа} \implies \\
    \implies A_\text{газа} &= - \Delta U = - \frac 32 \nu R \Delta T = - \frac 32 \cdot 30\,\text{моль} \cdot 8{,}31\,\frac{\text{Дж}}{\text{моль}\cdot\text{К}} \cdot 80\,\text{К}= -29{,}90\,\text{кДж}, \text{внешние силы.}
    \end{align*}
}
\solutionspace{40pt}

\tasknumber{5}%
\task{%
    Как изменилась внутренняя энергия одноатомного идеального газа при переходе из состояния 1 в состояние 2?
    $P_1 = 3\,\text{МПа}$, $V_1 = 3\,\text{л}$, $P_2 = 2{,}5\,\text{МПа}$, $V_2 = 6\,\text{л}$.
    Как изменилась при этом температура газа?
}
\answer{%
    \begin{align*}
    P_1V_1 &= \nu R T_1, P_2V_2 = \nu R T_2, \\
    \Delta U &= U_2-U_1 = \frac 32 \nu R T_2- \frac 32 \nu R T_1 = \frac 32 P_2 V_2 - \frac 32 P_1 V_1= \frac 32 \cdot \cbr{2{,}5\,\text{МПа} \cdot 6\,\text{л} - 3\,\text{МПа} \cdot 3\,\text{л}} = 9000\,\text{Дж}.
    \\
    \frac{T_2}{T_1} &= \frac{\frac{P_2V_2}{\nu R}}{\frac{P_1V_1}{\nu R}} = \frac{P_2V_2}{P_1V_1}= \frac{2{,}5\,\text{МПа} \cdot 6\,\text{л}}{3\,\text{МПа} \cdot 3\,\text{л}} \approx 1{,}67.
    \end{align*}
}
\solutionspace{80pt}

\tasknumber{6}%
\task{%
    $2\,\text{моль}$ идеального одноатомного газа нагрели на $20\,\text{К}$.
    Определите изменение внутренней энергии газа.
    Увеличилась она или уменьшилась?
    Универсальная газовая постоянная $R = 8{,}31\,\frac{\text{Дж}}{\text{моль}\cdot\text{К}}$.
}
\answer{%
    $
        \Delta U = \frac 32 \nu R \Delta T
            =  \frac 32 \cdot 2\,\text{моль} \cdot 8{,}31\,\frac{\text{Дж}}{\text{моль}\cdot\text{К}} \cdot 20\,\text{К}
            = 498\,\text{Дж}.
            \text{Увеличилась.}
    $
}
\solutionspace{40pt}

\tasknumber{7}%
\task{%
    Газу сообщили некоторое количество теплоты,
    при этом треть его он потратил на совершение работы,
    одновременно увеличив свою внутреннюю энергию на $2400\,\text{Дж}$.
    Определите работу, совершённую газом.
}
\answer{%
    \begin{align*}
    Q &= A' + \Delta U, A' = \frac 13 Q \implies Q \cdot \cbr{1 - \frac 13} = \Delta U \implies Q = \frac{\Delta U}{1 - \frac 13} = \frac{2400\,\text{Дж}}{1 - \frac 13} \approx 3600\,\text{Дж}.
    \\
    A' &= \frac 13 Q
        = \frac 13 \cdot \frac{\Delta U}{1 - \frac 13}
        = \frac{\Delta U}{3 - 1}
        = \frac{2400\,\text{Дж}}{3 - 1} \approx 1200\,\text{Дж}.
    \end{align*}
}
\solutionspace{60pt}

\tasknumber{8}%
\task{%
    В некотором процессе внешние силы совершили над газом работу $200\,\text{Дж}$,
    при этом его внутренняя энергия уменьшилась на $350\,\text{Дж}$.
    Определите количество тепла, переданное при этом процессе газу.
    Явно пропишите, подводили газу тепло или же отводили.
}
\answer{%
    $
        Q = A_\text{газа} + \Delta U, A_\text{газа} = -A_\text{внешняя}
        \implies Q = A_\text{газа} + \Delta U = - 200\,\text{Дж} - 350\,\text{Дж} = -550\,\text{Дж}.
        \text{ Отводили.}
    $
}

\variantsplitter

\addpersonalvariant{Андрей Рожков}

\tasknumber{1}%
\task{%
    Напротив физических величин укажите их обозначения и единицы измерения в СИ, а в пункте «г)» запишите физический закон или формулу:
    \begin{enumerate}
        \item изменение внутренней энергии,
        \item работа газа,
        \item удельная теплоёмкость,
        \item первое начало термодинамики.
    \end{enumerate}
}
\solutionspace{20pt}

\tasknumber{2}%
\task{%
    Определите объём идеального одноатомного газа,
    если его внутренняя энергия при давлении $2\,\text{атм}$ составляет $300\,\text{кДж}$.
    $p_{\text{aтм}} = 100\,\text{кПа}$.
}
\answer{%
    $U = \frac 32 \nu R T = \frac 32 PV \implies V = \frac 23 \cdot \frac UP= \frac 23 \cdot \frac{300\,\text{кДж}}{2\,\text{атм}} \approx 1{,}00\,\text{м}^{3}.$
}
\solutionspace{40pt}

\tasknumber{3}%
\task{%
    Газ расширился от $200\,\text{л}$ до $650\,\text{л}$.
    Давление газа при этом оставалось постоянным и равным $1{,}5\,\text{атм}$.
    Определите работу газа, ответ выразите в килоджоулях.
    $p_{\text{aтм}} = 100\,\text{кПа}$.
}
\answer{%
    $A = P\Delta V = P(V_2 - V_1) = 1{,}5\,\text{атм} \cdot \cbr{650\,\text{л} - 200\,\text{л}} = 67{,}5\,\text{кДж}.$
}
\solutionspace{40pt}

\tasknumber{4}%
\task{%
    $30\,\text{моль}$ идеального одноатомного газа в результате адиабатического процесса остыли на $60\,\text{К}$.
    Определите работу газа.
    Кто совершил положительную работу: газ или внешние силы?
    Универсальная газовая постоянная $R = 8{,}31\,\frac{\text{Дж}}{\text{моль}\cdot\text{К}}$.
}
\answer{%
    \begin{align*}
    Q &= 0, Q = \Delta U + A_\text{газа} \implies \\
    \implies A_\text{газа} &= - \Delta U = - \frac 32 \nu R \Delta T =  \frac 32 \cdot 30\,\text{моль} \cdot 8{,}31\,\frac{\text{Дж}}{\text{моль}\cdot\text{К}} \cdot 60\,\text{К}= 22{,}4\,\text{кДж}, \text{газ.}
    \end{align*}
}
\solutionspace{40pt}

\tasknumber{5}%
\task{%
    Как изменилась внутренняя энергия одноатомного идеального газа при переходе из состояния 1 в состояние 2?
    $P_1 = 2\,\text{МПа}$, $V_1 = 5\,\text{л}$, $P_2 = 4{,}5\,\text{МПа}$, $V_2 = 6\,\text{л}$.
    Как изменилась при этом температура газа?
}
\answer{%
    \begin{align*}
    P_1V_1 &= \nu R T_1, P_2V_2 = \nu R T_2, \\
    \Delta U &= U_2-U_1 = \frac 32 \nu R T_2- \frac 32 \nu R T_1 = \frac 32 P_2 V_2 - \frac 32 P_1 V_1= \frac 32 \cdot \cbr{4{,}5\,\text{МПа} \cdot 6\,\text{л} - 2\,\text{МПа} \cdot 5\,\text{л}} = 25500\,\text{Дж}.
    \\
    \frac{T_2}{T_1} &= \frac{\frac{P_2V_2}{\nu R}}{\frac{P_1V_1}{\nu R}} = \frac{P_2V_2}{P_1V_1}= \frac{4{,}5\,\text{МПа} \cdot 6\,\text{л}}{2\,\text{МПа} \cdot 5\,\text{л}} \approx 2{,}70.
    \end{align*}
}
\solutionspace{80pt}

\tasknumber{6}%
\task{%
    $2\,\text{моль}$ идеального одноатомного газа охладили на $10\,\text{К}$.
    Определите изменение внутренней энергии газа.
    Увеличилась она или уменьшилась?
    Универсальная газовая постоянная $R = 8{,}31\,\frac{\text{Дж}}{\text{моль}\cdot\text{К}}$.
}
\answer{%
    $
        \Delta U = \frac 32 \nu R \Delta T
            = - \frac 32 \cdot 2\,\text{моль} \cdot 8{,}31\,\frac{\text{Дж}}{\text{моль}\cdot\text{К}} \cdot 10\,\text{К}
            = -249\,\text{Дж}.
            \text{Уменьшилась.}
    $
}
\solutionspace{40pt}

\tasknumber{7}%
\task{%
    Газу сообщили некоторое количество теплоты,
    при этом треть его он потратил на совершение работы,
    одновременно увеличив свою внутреннюю энергию на $3000\,\text{Дж}$.
    Определите количество теплоты, сообщённое газу.
}
\answer{%
    \begin{align*}
    Q &= A' + \Delta U, A' = \frac 13 Q \implies Q \cdot \cbr{1 - \frac 13} = \Delta U \implies Q = \frac{\Delta U}{1 - \frac 13} = \frac{3000\,\text{Дж}}{1 - \frac 13} \approx 4500\,\text{Дж}.
    \\
    A' &= \frac 13 Q
        = \frac 13 \cdot \frac{\Delta U}{1 - \frac 13}
        = \frac{\Delta U}{3 - 1}
        = \frac{3000\,\text{Дж}}{3 - 1} \approx 1500\,\text{Дж}.
    \end{align*}
}
\solutionspace{60pt}

\tasknumber{8}%
\task{%
    В некотором процессе газ совершил работу $200\,\text{Дж}$,
    при этом его внутренняя энергия увеличилась на $250\,\text{Дж}$.
    Определите количество тепла, переданное при этом процессе газу.
    Явно пропишите, подводили газу тепло или же отводили.
}
\answer{%
    $
        Q = A_\text{газа} + \Delta U, A_\text{газа} = -A_\text{внешняя}
        \implies Q = A_\text{газа} + \Delta U =  200\,\text{Дж} +  250\,\text{Дж} = 450\,\text{Дж}.
        \text{ Подводили.}
    $
}

\variantsplitter

\addpersonalvariant{Рената Таржиманова}

\tasknumber{1}%
\task{%
    Напротив физических величин укажите их обозначения и единицы измерения в СИ, а в пункте «г)» запишите физический закон или формулу:
    \begin{enumerate}
        \item количество теплоты,
        \item работа газа,
        \item удельная теплоёмкость,
        \item первое начало термодинамики.
    \end{enumerate}
}
\solutionspace{20pt}

\tasknumber{2}%
\task{%
    Определите объём идеального одноатомного газа,
    если его внутренняя энергия при давлении $5\,\text{атм}$ составляет $300\,\text{кДж}$.
    $p_{\text{aтм}} = 100\,\text{кПа}$.
}
\answer{%
    $U = \frac 32 \nu R T = \frac 32 PV \implies V = \frac 23 \cdot \frac UP= \frac 23 \cdot \frac{300\,\text{кДж}}{5\,\text{атм}} \approx 0{,}40\,\text{м}^{3}.$
}
\solutionspace{40pt}

\tasknumber{3}%
\task{%
    Газ расширился от $350\,\text{л}$ до $650\,\text{л}$.
    Давление газа при этом оставалось постоянным и равным $1{,}5\,\text{атм}$.
    Определите работу газа, ответ выразите в килоджоулях.
    $p_{\text{aтм}} = 100\,\text{кПа}$.
}
\answer{%
    $A = P\Delta V = P(V_2 - V_1) = 1{,}5\,\text{атм} \cdot \cbr{650\,\text{л} - 350\,\text{л}} = 45{,}0\,\text{кДж}.$
}
\solutionspace{40pt}

\tasknumber{4}%
\task{%
    $40\,\text{моль}$ идеального одноатомного газа в результате адиабатического процесса остыли на $80\,\text{К}$.
    Определите работу газа.
    Кто совершил положительную работу: газ или внешние силы?
    Универсальная газовая постоянная $R = 8{,}31\,\frac{\text{Дж}}{\text{моль}\cdot\text{К}}$.
}
\answer{%
    \begin{align*}
    Q &= 0, Q = \Delta U + A_\text{газа} \implies \\
    \implies A_\text{газа} &= - \Delta U = - \frac 32 \nu R \Delta T =  \frac 32 \cdot 40\,\text{моль} \cdot 8{,}31\,\frac{\text{Дж}}{\text{моль}\cdot\text{К}} \cdot 80\,\text{К}= 39{,}9\,\text{кДж}, \text{газ.}
    \end{align*}
}
\solutionspace{40pt}

\tasknumber{5}%
\task{%
    Как изменилась внутренняя энергия одноатомного идеального газа при переходе из состояния 1 в состояние 2?
    $P_1 = 4\,\text{МПа}$, $V_1 = 7\,\text{л}$, $P_2 = 3{,}5\,\text{МПа}$, $V_2 = 6\,\text{л}$.
    Как изменилась при этом температура газа?
}
\answer{%
    \begin{align*}
    P_1V_1 &= \nu R T_1, P_2V_2 = \nu R T_2, \\
    \Delta U &= U_2-U_1 = \frac 32 \nu R T_2- \frac 32 \nu R T_1 = \frac 32 P_2 V_2 - \frac 32 P_1 V_1= \frac 32 \cdot \cbr{3{,}5\,\text{МПа} \cdot 6\,\text{л} - 4\,\text{МПа} \cdot 7\,\text{л}} = -10500\,\text{Дж}.
    \\
    \frac{T_2}{T_1} &= \frac{\frac{P_2V_2}{\nu R}}{\frac{P_1V_1}{\nu R}} = \frac{P_2V_2}{P_1V_1}= \frac{3{,}5\,\text{МПа} \cdot 6\,\text{л}}{4\,\text{МПа} \cdot 7\,\text{л}} \approx 0{,}75.
    \end{align*}
}
\solutionspace{80pt}

\tasknumber{6}%
\task{%
    $3\,\text{моль}$ идеального одноатомного газа охладили на $20\,\text{К}$.
    Определите изменение внутренней энергии газа.
    Увеличилась она или уменьшилась?
    Универсальная газовая постоянная $R = 8{,}31\,\frac{\text{Дж}}{\text{моль}\cdot\text{К}}$.
}
\answer{%
    $
        \Delta U = \frac 32 \nu R \Delta T
            = - \frac 32 \cdot 3\,\text{моль} \cdot 8{,}31\,\frac{\text{Дж}}{\text{моль}\cdot\text{К}} \cdot 20\,\text{К}
            = -747\,\text{Дж}.
            \text{Уменьшилась.}
    $
}
\solutionspace{40pt}

\tasknumber{7}%
\task{%
    Газу сообщили некоторое количество теплоты,
    при этом треть его он потратил на совершение работы,
    одновременно увеличив свою внутреннюю энергию на $1200\,\text{Дж}$.
    Определите количество теплоты, сообщённое газу.
}
\answer{%
    \begin{align*}
    Q &= A' + \Delta U, A' = \frac 13 Q \implies Q \cdot \cbr{1 - \frac 13} = \Delta U \implies Q = \frac{\Delta U}{1 - \frac 13} = \frac{1200\,\text{Дж}}{1 - \frac 13} \approx 1800\,\text{Дж}.
    \\
    A' &= \frac 13 Q
        = \frac 13 \cdot \frac{\Delta U}{1 - \frac 13}
        = \frac{\Delta U}{3 - 1}
        = \frac{1200\,\text{Дж}}{3 - 1} \approx 600\,\text{Дж}.
    \end{align*}
}
\solutionspace{60pt}

\tasknumber{8}%
\task{%
    В некотором процессе внешние силы совершили над газом работу $100\,\text{Дж}$,
    при этом его внутренняя энергия уменьшилась на $250\,\text{Дж}$.
    Определите количество тепла, переданное при этом процессе газу.
    Явно пропишите, подводили газу тепло или же отводили.
}
\answer{%
    $
        Q = A_\text{газа} + \Delta U, A_\text{газа} = -A_\text{внешняя}
        \implies Q = A_\text{газа} + \Delta U = - 100\,\text{Дж} - 250\,\text{Дж} = -350\,\text{Дж}.
        \text{ Отводили.}
    $
}

\variantsplitter

\addpersonalvariant{Андрей Щербаков}

\tasknumber{1}%
\task{%
    Напротив физических величин укажите их обозначения и единицы измерения в СИ, а в пункте «г)» запишите физический закон или формулу:
    \begin{enumerate}
        \item изменение внутренней энергии,
        \item работа внешних сил,
        \item удельная теплоёмкость,
        \item первое начало термодинамики.
    \end{enumerate}
}
\solutionspace{20pt}

\tasknumber{2}%
\task{%
    Определите объём идеального одноатомного газа,
    если его внутренняя энергия при давлении $5\,\text{атм}$ составляет $300\,\text{кДж}$.
    $p_{\text{aтм}} = 100\,\text{кПа}$.
}
\answer{%
    $U = \frac 32 \nu R T = \frac 32 PV \implies V = \frac 23 \cdot \frac UP= \frac 23 \cdot \frac{300\,\text{кДж}}{5\,\text{атм}} \approx 0{,}40\,\text{м}^{3}.$
}
\solutionspace{40pt}

\tasknumber{3}%
\task{%
    Газ расширился от $250\,\text{л}$ до $450\,\text{л}$.
    Давление газа при этом оставалось постоянным и равным $1{,}5\,\text{атм}$.
    Определите работу газа, ответ выразите в килоджоулях.
    $p_{\text{aтм}} = 100\,\text{кПа}$.
}
\answer{%
    $A = P\Delta V = P(V_2 - V_1) = 1{,}5\,\text{атм} \cdot \cbr{450\,\text{л} - 250\,\text{л}} = 30{,}0\,\text{кДж}.$
}
\solutionspace{40pt}

\tasknumber{4}%
\task{%
    $60\,\text{моль}$ идеального одноатомного газа в результате адиабатического процесса нагрелись на $80\,\text{К}$.
    Определите работу газа.
    Кто совершил положительную работу: газ или внешние силы?
    Универсальная газовая постоянная $R = 8{,}31\,\frac{\text{Дж}}{\text{моль}\cdot\text{К}}$.
}
\answer{%
    \begin{align*}
    Q &= 0, Q = \Delta U + A_\text{газа} \implies \\
    \implies A_\text{газа} &= - \Delta U = - \frac 32 \nu R \Delta T = - \frac 32 \cdot 60\,\text{моль} \cdot 8{,}31\,\frac{\text{Дж}}{\text{моль}\cdot\text{К}} \cdot 80\,\text{К}= -59{,}80\,\text{кДж}, \text{внешние силы.}
    \end{align*}
}
\solutionspace{40pt}

\tasknumber{5}%
\task{%
    Как изменилась внутренняя энергия одноатомного идеального газа при переходе из состояния 1 в состояние 2?
    $P_1 = 4\,\text{МПа}$, $V_1 = 3\,\text{л}$, $P_2 = 2{,}5\,\text{МПа}$, $V_2 = 6\,\text{л}$.
    Как изменилась при этом температура газа?
}
\answer{%
    \begin{align*}
    P_1V_1 &= \nu R T_1, P_2V_2 = \nu R T_2, \\
    \Delta U &= U_2-U_1 = \frac 32 \nu R T_2- \frac 32 \nu R T_1 = \frac 32 P_2 V_2 - \frac 32 P_1 V_1= \frac 32 \cdot \cbr{2{,}5\,\text{МПа} \cdot 6\,\text{л} - 4\,\text{МПа} \cdot 3\,\text{л}} = 4500\,\text{Дж}.
    \\
    \frac{T_2}{T_1} &= \frac{\frac{P_2V_2}{\nu R}}{\frac{P_1V_1}{\nu R}} = \frac{P_2V_2}{P_1V_1}= \frac{2{,}5\,\text{МПа} \cdot 6\,\text{л}}{4\,\text{МПа} \cdot 3\,\text{л}} \approx 1{,}25.
    \end{align*}
}
\solutionspace{80pt}

\tasknumber{6}%
\task{%
    $5\,\text{моль}$ идеального одноатомного газа нагрели на $20\,\text{К}$.
    Определите изменение внутренней энергии газа.
    Увеличилась она или уменьшилась?
    Универсальная газовая постоянная $R = 8{,}31\,\frac{\text{Дж}}{\text{моль}\cdot\text{К}}$.
}
\answer{%
    $
        \Delta U = \frac 32 \nu R \Delta T
            =  \frac 32 \cdot 5\,\text{моль} \cdot 8{,}31\,\frac{\text{Дж}}{\text{моль}\cdot\text{К}} \cdot 20\,\text{К}
            = 1246\,\text{Дж}.
            \text{Увеличилась.}
    $
}
\solutionspace{40pt}

\tasknumber{7}%
\task{%
    Газу сообщили некоторое количество теплоты,
    при этом четверть его он потратил на совершение работы,
    одновременно увеличив свою внутреннюю энергию на $3000\,\text{Дж}$.
    Определите работу, совершённую газом.
}
\answer{%
    \begin{align*}
    Q &= A' + \Delta U, A' = \frac 14 Q \implies Q \cdot \cbr{1 - \frac 14} = \Delta U \implies Q = \frac{\Delta U}{1 - \frac 14} = \frac{3000\,\text{Дж}}{1 - \frac 14} \approx 4000\,\text{Дж}.
    \\
    A' &= \frac 14 Q
        = \frac 14 \cdot \frac{\Delta U}{1 - \frac 14}
        = \frac{\Delta U}{4 - 1}
        = \frac{3000\,\text{Дж}}{4 - 1} \approx 1000\,\text{Дж}.
    \end{align*}
}
\solutionspace{60pt}

\tasknumber{8}%
\task{%
    В некотором процессе газ совершил работу $100\,\text{Дж}$,
    при этом его внутренняя энергия уменьшилась на $250\,\text{Дж}$.
    Определите количество тепла, переданное при этом процессе газу.
    Явно пропишите, подводили газу тепло или же отводили.
}
\answer{%
    $
        Q = A_\text{газа} + \Delta U, A_\text{газа} = -A_\text{внешняя}
        \implies Q = A_\text{газа} + \Delta U =  100\,\text{Дж} - 250\,\text{Дж} = -150\,\text{Дж}.
        \text{ Отводили.}
    $
}

\variantsplitter

\addpersonalvariant{Михаил Ярошевский}

\tasknumber{1}%
\task{%
    Напротив физических величин укажите их обозначения и единицы измерения в СИ, а в пункте «г)» запишите физический закон или формулу:
    \begin{enumerate}
        \item количество теплоты,
        \item работа газа,
        \item молярная теплоёмкость,
        \item первое начало термодинамики.
    \end{enumerate}
}
\solutionspace{20pt}

\tasknumber{2}%
\task{%
    Определите объём идеального одноатомного газа,
    если его внутренняя энергия при давлении $2\,\text{атм}$ составляет $400\,\text{кДж}$.
    $p_{\text{aтм}} = 100\,\text{кПа}$.
}
\answer{%
    $U = \frac 32 \nu R T = \frac 32 PV \implies V = \frac 23 \cdot \frac UP= \frac 23 \cdot \frac{400\,\text{кДж}}{2\,\text{атм}} \approx 1{,}33\,\text{м}^{3}.$
}
\solutionspace{40pt}

\tasknumber{3}%
\task{%
    Газ расширился от $150\,\text{л}$ до $550\,\text{л}$.
    Давление газа при этом оставалось постоянным и равным $3{,}5\,\text{атм}$.
    Определите работу газа, ответ выразите в килоджоулях.
    $p_{\text{aтм}} = 100\,\text{кПа}$.
}
\answer{%
    $A = P\Delta V = P(V_2 - V_1) = 3{,}5\,\text{атм} \cdot \cbr{550\,\text{л} - 150\,\text{л}} = 140{,}0\,\text{кДж}.$
}
\solutionspace{40pt}

\tasknumber{4}%
\task{%
    $40\,\text{моль}$ идеального одноатомного газа в результате адиабатического процесса нагрелись на $80\,\text{К}$.
    Определите работу газа.
    Кто совершил положительную работу: газ или внешние силы?
    Универсальная газовая постоянная $R = 8{,}31\,\frac{\text{Дж}}{\text{моль}\cdot\text{К}}$.
}
\answer{%
    \begin{align*}
    Q &= 0, Q = \Delta U + A_\text{газа} \implies \\
    \implies A_\text{газа} &= - \Delta U = - \frac 32 \nu R \Delta T = - \frac 32 \cdot 40\,\text{моль} \cdot 8{,}31\,\frac{\text{Дж}}{\text{моль}\cdot\text{К}} \cdot 80\,\text{К}= -39{,}90\,\text{кДж}, \text{внешние силы.}
    \end{align*}
}
\solutionspace{40pt}

\tasknumber{5}%
\task{%
    Как изменилась внутренняя энергия одноатомного идеального газа при переходе из состояния 1 в состояние 2?
    $P_1 = 3\,\text{МПа}$, $V_1 = 3\,\text{л}$, $P_2 = 3{,}5\,\text{МПа}$, $V_2 = 8\,\text{л}$.
    Как изменилась при этом температура газа?
}
\answer{%
    \begin{align*}
    P_1V_1 &= \nu R T_1, P_2V_2 = \nu R T_2, \\
    \Delta U &= U_2-U_1 = \frac 32 \nu R T_2- \frac 32 \nu R T_1 = \frac 32 P_2 V_2 - \frac 32 P_1 V_1= \frac 32 \cdot \cbr{3{,}5\,\text{МПа} \cdot 8\,\text{л} - 3\,\text{МПа} \cdot 3\,\text{л}} = 28500\,\text{Дж}.
    \\
    \frac{T_2}{T_1} &= \frac{\frac{P_2V_2}{\nu R}}{\frac{P_1V_1}{\nu R}} = \frac{P_2V_2}{P_1V_1}= \frac{3{,}5\,\text{МПа} \cdot 8\,\text{л}}{3\,\text{МПа} \cdot 3\,\text{л}} \approx 3{,}11.
    \end{align*}
}
\solutionspace{80pt}

\tasknumber{6}%
\task{%
    $3\,\text{моль}$ идеального одноатомного газа нагрели на $20\,\text{К}$.
    Определите изменение внутренней энергии газа.
    Увеличилась она или уменьшилась?
    Универсальная газовая постоянная $R = 8{,}31\,\frac{\text{Дж}}{\text{моль}\cdot\text{К}}$.
}
\answer{%
    $
        \Delta U = \frac 32 \nu R \Delta T
            =  \frac 32 \cdot 3\,\text{моль} \cdot 8{,}31\,\frac{\text{Дж}}{\text{моль}\cdot\text{К}} \cdot 20\,\text{К}
            = 747\,\text{Дж}.
            \text{Увеличилась.}
    $
}
\solutionspace{40pt}

\tasknumber{7}%
\task{%
    Газу сообщили некоторое количество теплоты,
    при этом половину его он потратил на совершение работы,
    одновременно увеличив свою внутреннюю энергию на $1500\,\text{Дж}$.
    Определите работу, совершённую газом.
}
\answer{%
    \begin{align*}
    Q &= A' + \Delta U, A' = \frac 12 Q \implies Q \cdot \cbr{1 - \frac 12} = \Delta U \implies Q = \frac{\Delta U}{1 - \frac 12} = \frac{1500\,\text{Дж}}{1 - \frac 12} \approx 3000\,\text{Дж}.
    \\
    A' &= \frac 12 Q
        = \frac 12 \cdot \frac{\Delta U}{1 - \frac 12}
        = \frac{\Delta U}{2 - 1}
        = \frac{1500\,\text{Дж}}{2 - 1} \approx 1500\,\text{Дж}.
    \end{align*}
}
\solutionspace{60pt}

\tasknumber{8}%
\task{%
    В некотором процессе газ совершил работу $200\,\text{Дж}$,
    при этом его внутренняя энергия увеличилась на $350\,\text{Дж}$.
    Определите количество тепла, переданное при этом процессе газу.
    Явно пропишите, подводили газу тепло или же отводили.
}
\answer{%
    $
        Q = A_\text{газа} + \Delta U, A_\text{газа} = -A_\text{внешняя}
        \implies Q = A_\text{газа} + \Delta U =  200\,\text{Дж} +  350\,\text{Дж} = 550\,\text{Дж}.
        \text{ Подводили.}
    $
}

\variantsplitter

\addpersonalvariant{Алексей Алимпиев}

\tasknumber{1}%
\task{%
    Напротив физических величин укажите их обозначения и единицы измерения в СИ, а в пункте «г)» запишите физический закон или формулу:
    \begin{enumerate}
        \item количество теплоты,
        \item работа внешних сил,
        \item молярная теплоёмкость,
        \item первое начало термодинамики.
    \end{enumerate}
}
\solutionspace{20pt}

\tasknumber{2}%
\task{%
    Определите объём идеального одноатомного газа,
    если его внутренняя энергия при давлении $5\,\text{атм}$ составляет $400\,\text{кДж}$.
    $p_{\text{aтм}} = 100\,\text{кПа}$.
}
\answer{%
    $U = \frac 32 \nu R T = \frac 32 PV \implies V = \frac 23 \cdot \frac UP= \frac 23 \cdot \frac{400\,\text{кДж}}{5\,\text{атм}} \approx 0{,}53\,\text{м}^{3}.$
}
\solutionspace{40pt}

\tasknumber{3}%
\task{%
    Газ расширился от $200\,\text{л}$ до $550\,\text{л}$.
    Давление газа при этом оставалось постоянным и равным $1{,}2\,\text{атм}$.
    Определите работу газа, ответ выразите в килоджоулях.
    $p_{\text{aтм}} = 100\,\text{кПа}$.
}
\answer{%
    $A = P\Delta V = P(V_2 - V_1) = 1{,}2\,\text{атм} \cdot \cbr{550\,\text{л} - 200\,\text{л}} = 42{,}0\,\text{кДж}.$
}
\solutionspace{40pt}

\tasknumber{4}%
\task{%
    $50\,\text{моль}$ идеального одноатомного газа в результате адиабатического процесса нагрелись на $15\,\text{К}$.
    Определите работу газа.
    Кто совершил положительную работу: газ или внешние силы?
    Универсальная газовая постоянная $R = 8{,}31\,\frac{\text{Дж}}{\text{моль}\cdot\text{К}}$.
}
\answer{%
    \begin{align*}
    Q &= 0, Q = \Delta U + A_\text{газа} \implies \\
    \implies A_\text{газа} &= - \Delta U = - \frac 32 \nu R \Delta T = - \frac 32 \cdot 50\,\text{моль} \cdot 8{,}31\,\frac{\text{Дж}}{\text{моль}\cdot\text{К}} \cdot 15\,\text{К}= -9{,}30\,\text{кДж}, \text{внешние силы.}
    \end{align*}
}
\solutionspace{40pt}

\tasknumber{5}%
\task{%
    Как изменилась внутренняя энергия одноатомного идеального газа при переходе из состояния 1 в состояние 2?
    $P_1 = 2\,\text{МПа}$, $V_1 = 7\,\text{л}$, $P_2 = 1{,}5\,\text{МПа}$, $V_2 = 2\,\text{л}$.
    Как изменилась при этом температура газа?
}
\answer{%
    \begin{align*}
    P_1V_1 &= \nu R T_1, P_2V_2 = \nu R T_2, \\
    \Delta U &= U_2-U_1 = \frac 32 \nu R T_2- \frac 32 \nu R T_1 = \frac 32 P_2 V_2 - \frac 32 P_1 V_1= \frac 32 \cdot \cbr{1{,}5\,\text{МПа} \cdot 2\,\text{л} - 2\,\text{МПа} \cdot 7\,\text{л}} = -16500\,\text{Дж}.
    \\
    \frac{T_2}{T_1} &= \frac{\frac{P_2V_2}{\nu R}}{\frac{P_1V_1}{\nu R}} = \frac{P_2V_2}{P_1V_1}= \frac{1{,}5\,\text{МПа} \cdot 2\,\text{л}}{2\,\text{МПа} \cdot 7\,\text{л}} \approx 0{,}21.
    \end{align*}
}
\solutionspace{80pt}

\tasknumber{6}%
\task{%
    $4\,\text{моль}$ идеального одноатомного газа нагрели на $10\,\text{К}$.
    Определите изменение внутренней энергии газа.
    Увеличилась она или уменьшилась?
    Универсальная газовая постоянная $R = 8{,}31\,\frac{\text{Дж}}{\text{моль}\cdot\text{К}}$.
}
\answer{%
    $
        \Delta U = \frac 32 \nu R \Delta T
            =  \frac 32 \cdot 4\,\text{моль} \cdot 8{,}31\,\frac{\text{Дж}}{\text{моль}\cdot\text{К}} \cdot 10\,\text{К}
            = 498\,\text{Дж}.
            \text{Увеличилась.}
    $
}
\solutionspace{40pt}

\tasknumber{7}%
\task{%
    Газу сообщили некоторое количество теплоты,
    при этом четверть его он потратил на совершение работы,
    одновременно увеличив свою внутреннюю энергию на $2400\,\text{Дж}$.
    Определите количество теплоты, сообщённое газу.
}
\answer{%
    \begin{align*}
    Q &= A' + \Delta U, A' = \frac 14 Q \implies Q \cdot \cbr{1 - \frac 14} = \Delta U \implies Q = \frac{\Delta U}{1 - \frac 14} = \frac{2400\,\text{Дж}}{1 - \frac 14} \approx 3200\,\text{Дж}.
    \\
    A' &= \frac 14 Q
        = \frac 14 \cdot \frac{\Delta U}{1 - \frac 14}
        = \frac{\Delta U}{4 - 1}
        = \frac{2400\,\text{Дж}}{4 - 1} \approx 800\,\text{Дж}.
    \end{align*}
}
\solutionspace{60pt}

\tasknumber{8}%
\task{%
    В некотором процессе внешние силы совершили над газом работу $200\,\text{Дж}$,
    при этом его внутренняя энергия увеличилась на $450\,\text{Дж}$.
    Определите количество тепла, переданное при этом процессе газу.
    Явно пропишите, подводили газу тепло или же отводили.
}
\answer{%
    $
        Q = A_\text{газа} + \Delta U, A_\text{газа} = -A_\text{внешняя}
        \implies Q = A_\text{газа} + \Delta U = - 200\,\text{Дж} +  450\,\text{Дж} = 250\,\text{Дж}.
        \text{ Подводили.}
    $
}

\variantsplitter

\addpersonalvariant{Евгений Васин}

\tasknumber{1}%
\task{%
    Напротив физических величин укажите их обозначения и единицы измерения в СИ, а в пункте «г)» запишите физический закон или формулу:
    \begin{enumerate}
        \item изменение внутренней энергии,
        \item работа газа,
        \item удельная теплоёмкость,
        \item внутренняя энергия идеального одноатомного газа.
    \end{enumerate}
}
\solutionspace{20pt}

\tasknumber{2}%
\task{%
    Определите объём идеального одноатомного газа,
    если его внутренняя энергия при давлении $2\,\text{атм}$ составляет $300\,\text{кДж}$.
    $p_{\text{aтм}} = 100\,\text{кПа}$.
}
\answer{%
    $U = \frac 32 \nu R T = \frac 32 PV \implies V = \frac 23 \cdot \frac UP= \frac 23 \cdot \frac{300\,\text{кДж}}{2\,\text{атм}} \approx 1{,}00\,\text{м}^{3}.$
}
\solutionspace{40pt}

\tasknumber{3}%
\task{%
    Газ расширился от $350\,\text{л}$ до $550\,\text{л}$.
    Давление газа при этом оставалось постоянным и равным $3{,}5\,\text{атм}$.
    Определите работу газа, ответ выразите в килоджоулях.
    $p_{\text{aтм}} = 100\,\text{кПа}$.
}
\answer{%
    $A = P\Delta V = P(V_2 - V_1) = 3{,}5\,\text{атм} \cdot \cbr{550\,\text{л} - 350\,\text{л}} = 70{,}0\,\text{кДж}.$
}
\solutionspace{40pt}

\tasknumber{4}%
\task{%
    $40\,\text{моль}$ идеального одноатомного газа в результате адиабатического процесса нагрелись на $25\,\text{К}$.
    Определите работу газа.
    Кто совершил положительную работу: газ или внешние силы?
    Универсальная газовая постоянная $R = 8{,}31\,\frac{\text{Дж}}{\text{моль}\cdot\text{К}}$.
}
\answer{%
    \begin{align*}
    Q &= 0, Q = \Delta U + A_\text{газа} \implies \\
    \implies A_\text{газа} &= - \Delta U = - \frac 32 \nu R \Delta T = - \frac 32 \cdot 40\,\text{моль} \cdot 8{,}31\,\frac{\text{Дж}}{\text{моль}\cdot\text{К}} \cdot 25\,\text{К}= -12{,}500\,\text{кДж}, \text{внешние силы.}
    \end{align*}
}
\solutionspace{40pt}

\tasknumber{5}%
\task{%
    Как изменилась внутренняя энергия одноатомного идеального газа при переходе из состояния 1 в состояние 2?
    $P_1 = 2\,\text{МПа}$, $V_1 = 7\,\text{л}$, $P_2 = 3{,}5\,\text{МПа}$, $V_2 = 2\,\text{л}$.
    Как изменилась при этом температура газа?
}
\answer{%
    \begin{align*}
    P_1V_1 &= \nu R T_1, P_2V_2 = \nu R T_2, \\
    \Delta U &= U_2-U_1 = \frac 32 \nu R T_2- \frac 32 \nu R T_1 = \frac 32 P_2 V_2 - \frac 32 P_1 V_1= \frac 32 \cdot \cbr{3{,}5\,\text{МПа} \cdot 2\,\text{л} - 2\,\text{МПа} \cdot 7\,\text{л}} = -10500\,\text{Дж}.
    \\
    \frac{T_2}{T_1} &= \frac{\frac{P_2V_2}{\nu R}}{\frac{P_1V_1}{\nu R}} = \frac{P_2V_2}{P_1V_1}= \frac{3{,}5\,\text{МПа} \cdot 2\,\text{л}}{2\,\text{МПа} \cdot 7\,\text{л}} \approx 0{,}50.
    \end{align*}
}
\solutionspace{80pt}

\tasknumber{6}%
\task{%
    $3\,\text{моль}$ идеального одноатомного газа нагрели на $20\,\text{К}$.
    Определите изменение внутренней энергии газа.
    Увеличилась она или уменьшилась?
    Универсальная газовая постоянная $R = 8{,}31\,\frac{\text{Дж}}{\text{моль}\cdot\text{К}}$.
}
\answer{%
    $
        \Delta U = \frac 32 \nu R \Delta T
            =  \frac 32 \cdot 3\,\text{моль} \cdot 8{,}31\,\frac{\text{Дж}}{\text{моль}\cdot\text{К}} \cdot 20\,\text{К}
            = 747\,\text{Дж}.
            \text{Увеличилась.}
    $
}
\solutionspace{40pt}

\tasknumber{7}%
\task{%
    Газу сообщили некоторое количество теплоты,
    при этом треть его он потратил на совершение работы,
    одновременно увеличив свою внутреннюю энергию на $1200\,\text{Дж}$.
    Определите работу, совершённую газом.
}
\answer{%
    \begin{align*}
    Q &= A' + \Delta U, A' = \frac 13 Q \implies Q \cdot \cbr{1 - \frac 13} = \Delta U \implies Q = \frac{\Delta U}{1 - \frac 13} = \frac{1200\,\text{Дж}}{1 - \frac 13} \approx 1800\,\text{Дж}.
    \\
    A' &= \frac 13 Q
        = \frac 13 \cdot \frac{\Delta U}{1 - \frac 13}
        = \frac{\Delta U}{3 - 1}
        = \frac{1200\,\text{Дж}}{3 - 1} \approx 600\,\text{Дж}.
    \end{align*}
}
\solutionspace{60pt}

\tasknumber{8}%
\task{%
    В некотором процессе внешние силы совершили над газом работу $200\,\text{Дж}$,
    при этом его внутренняя энергия уменьшилась на $250\,\text{Дж}$.
    Определите количество тепла, переданное при этом процессе газу.
    Явно пропишите, подводили газу тепло или же отводили.
}
\answer{%
    $
        Q = A_\text{газа} + \Delta U, A_\text{газа} = -A_\text{внешняя}
        \implies Q = A_\text{газа} + \Delta U = - 200\,\text{Дж} - 250\,\text{Дж} = -450\,\text{Дж}.
        \text{ Отводили.}
    $
}

\variantsplitter

\addpersonalvariant{Вячеслав Волохов}

\tasknumber{1}%
\task{%
    Напротив физических величин укажите их обозначения и единицы измерения в СИ, а в пункте «г)» запишите физический закон или формулу:
    \begin{enumerate}
        \item количество теплоты,
        \item работа внешних сил,
        \item молярная теплоёмкость,
        \item внутренняя энергия идеального одноатомного газа.
    \end{enumerate}
}
\solutionspace{20pt}

\tasknumber{2}%
\task{%
    Определите объём идеального одноатомного газа,
    если его внутренняя энергия при давлении $6\,\text{атм}$ составляет $250\,\text{кДж}$.
    $p_{\text{aтм}} = 100\,\text{кПа}$.
}
\answer{%
    $U = \frac 32 \nu R T = \frac 32 PV \implies V = \frac 23 \cdot \frac UP= \frac 23 \cdot \frac{250\,\text{кДж}}{6\,\text{атм}} \approx 0{,}28\,\text{м}^{3}.$
}
\solutionspace{40pt}

\tasknumber{3}%
\task{%
    Газ расширился от $250\,\text{л}$ до $550\,\text{л}$.
    Давление газа при этом оставалось постоянным и равным $1{,}2\,\text{атм}$.
    Определите работу газа, ответ выразите в килоджоулях.
    $p_{\text{aтм}} = 100\,\text{кПа}$.
}
\answer{%
    $A = P\Delta V = P(V_2 - V_1) = 1{,}2\,\text{атм} \cdot \cbr{550\,\text{л} - 250\,\text{л}} = 36{,}0\,\text{кДж}.$
}
\solutionspace{40pt}

\tasknumber{4}%
\task{%
    $40\,\text{моль}$ идеального одноатомного газа в результате адиабатического процесса нагрелись на $45\,\text{К}$.
    Определите работу газа.
    Кто совершил положительную работу: газ или внешние силы?
    Универсальная газовая постоянная $R = 8{,}31\,\frac{\text{Дж}}{\text{моль}\cdot\text{К}}$.
}
\answer{%
    \begin{align*}
    Q &= 0, Q = \Delta U + A_\text{газа} \implies \\
    \implies A_\text{газа} &= - \Delta U = - \frac 32 \nu R \Delta T = - \frac 32 \cdot 40\,\text{моль} \cdot 8{,}31\,\frac{\text{Дж}}{\text{моль}\cdot\text{К}} \cdot 45\,\text{К}= -22{,}40\,\text{кДж}, \text{внешние силы.}
    \end{align*}
}
\solutionspace{40pt}

\tasknumber{5}%
\task{%
    Как изменилась внутренняя энергия одноатомного идеального газа при переходе из состояния 1 в состояние 2?
    $P_1 = 2\,\text{МПа}$, $V_1 = 5\,\text{л}$, $P_2 = 3{,}5\,\text{МПа}$, $V_2 = 2\,\text{л}$.
    Как изменилась при этом температура газа?
}
\answer{%
    \begin{align*}
    P_1V_1 &= \nu R T_1, P_2V_2 = \nu R T_2, \\
    \Delta U &= U_2-U_1 = \frac 32 \nu R T_2- \frac 32 \nu R T_1 = \frac 32 P_2 V_2 - \frac 32 P_1 V_1= \frac 32 \cdot \cbr{3{,}5\,\text{МПа} \cdot 2\,\text{л} - 2\,\text{МПа} \cdot 5\,\text{л}} = -4500\,\text{Дж}.
    \\
    \frac{T_2}{T_1} &= \frac{\frac{P_2V_2}{\nu R}}{\frac{P_1V_1}{\nu R}} = \frac{P_2V_2}{P_1V_1}= \frac{3{,}5\,\text{МПа} \cdot 2\,\text{л}}{2\,\text{МПа} \cdot 5\,\text{л}} \approx 0{,}70.
    \end{align*}
}
\solutionspace{80pt}

\tasknumber{6}%
\task{%
    $3\,\text{моль}$ идеального одноатомного газа нагрели на $30\,\text{К}$.
    Определите изменение внутренней энергии газа.
    Увеличилась она или уменьшилась?
    Универсальная газовая постоянная $R = 8{,}31\,\frac{\text{Дж}}{\text{моль}\cdot\text{К}}$.
}
\answer{%
    $
        \Delta U = \frac 32 \nu R \Delta T
            =  \frac 32 \cdot 3\,\text{моль} \cdot 8{,}31\,\frac{\text{Дж}}{\text{моль}\cdot\text{К}} \cdot 30\,\text{К}
            = 1121\,\text{Дж}.
            \text{Увеличилась.}
    $
}
\solutionspace{40pt}

\tasknumber{7}%
\task{%
    Газу сообщили некоторое количество теплоты,
    при этом половину его он потратил на совершение работы,
    одновременно увеличив свою внутреннюю энергию на $3000\,\text{Дж}$.
    Определите работу, совершённую газом.
}
\answer{%
    \begin{align*}
    Q &= A' + \Delta U, A' = \frac 12 Q \implies Q \cdot \cbr{1 - \frac 12} = \Delta U \implies Q = \frac{\Delta U}{1 - \frac 12} = \frac{3000\,\text{Дж}}{1 - \frac 12} \approx 6000\,\text{Дж}.
    \\
    A' &= \frac 12 Q
        = \frac 12 \cdot \frac{\Delta U}{1 - \frac 12}
        = \frac{\Delta U}{2 - 1}
        = \frac{3000\,\text{Дж}}{2 - 1} \approx 3000\,\text{Дж}.
    \end{align*}
}
\solutionspace{60pt}

\tasknumber{8}%
\task{%
    В некотором процессе газ совершил работу $200\,\text{Дж}$,
    при этом его внутренняя энергия увеличилась на $150\,\text{Дж}$.
    Определите количество тепла, переданное при этом процессе газу.
    Явно пропишите, подводили газу тепло или же отводили.
}
\answer{%
    $
        Q = A_\text{газа} + \Delta U, A_\text{газа} = -A_\text{внешняя}
        \implies Q = A_\text{газа} + \Delta U =  200\,\text{Дж} +  150\,\text{Дж} = 350\,\text{Дж}.
        \text{ Подводили.}
    $
}

\variantsplitter

\addpersonalvariant{Герман Говоров}

\tasknumber{1}%
\task{%
    Напротив физических величин укажите их обозначения и единицы измерения в СИ, а в пункте «г)» запишите физический закон или формулу:
    \begin{enumerate}
        \item изменение внутренней энергии,
        \item работа внешних сил,
        \item удельная теплоёмкость,
        \item первое начало термодинамики.
    \end{enumerate}
}
\solutionspace{20pt}

\tasknumber{2}%
\task{%
    Определите объём идеального одноатомного газа,
    если его внутренняя энергия при давлении $4\,\text{атм}$ составляет $500\,\text{кДж}$.
    $p_{\text{aтм}} = 100\,\text{кПа}$.
}
\answer{%
    $U = \frac 32 \nu R T = \frac 32 PV \implies V = \frac 23 \cdot \frac UP= \frac 23 \cdot \frac{500\,\text{кДж}}{4\,\text{атм}} \approx 0{,}83\,\text{м}^{3}.$
}
\solutionspace{40pt}

\tasknumber{3}%
\task{%
    Газ расширился от $150\,\text{л}$ до $550\,\text{л}$.
    Давление газа при этом оставалось постоянным и равным $1{,}2\,\text{атм}$.
    Определите работу газа, ответ выразите в килоджоулях.
    $p_{\text{aтм}} = 100\,\text{кПа}$.
}
\answer{%
    $A = P\Delta V = P(V_2 - V_1) = 1{,}2\,\text{атм} \cdot \cbr{550\,\text{л} - 150\,\text{л}} = 48{,}0\,\text{кДж}.$
}
\solutionspace{40pt}

\tasknumber{4}%
\task{%
    $60\,\text{моль}$ идеального одноатомного газа в результате адиабатического процесса нагрелись на $45\,\text{К}$.
    Определите работу газа.
    Кто совершил положительную работу: газ или внешние силы?
    Универсальная газовая постоянная $R = 8{,}31\,\frac{\text{Дж}}{\text{моль}\cdot\text{К}}$.
}
\answer{%
    \begin{align*}
    Q &= 0, Q = \Delta U + A_\text{газа} \implies \\
    \implies A_\text{газа} &= - \Delta U = - \frac 32 \nu R \Delta T = - \frac 32 \cdot 60\,\text{моль} \cdot 8{,}31\,\frac{\text{Дж}}{\text{моль}\cdot\text{К}} \cdot 45\,\text{К}= -33{,}70\,\text{кДж}, \text{внешние силы.}
    \end{align*}
}
\solutionspace{40pt}

\tasknumber{5}%
\task{%
    Как изменилась внутренняя энергия одноатомного идеального газа при переходе из состояния 1 в состояние 2?
    $P_1 = 3\,\text{МПа}$, $V_1 = 3\,\text{л}$, $P_2 = 2{,}5\,\text{МПа}$, $V_2 = 4\,\text{л}$.
    Как изменилась при этом температура газа?
}
\answer{%
    \begin{align*}
    P_1V_1 &= \nu R T_1, P_2V_2 = \nu R T_2, \\
    \Delta U &= U_2-U_1 = \frac 32 \nu R T_2- \frac 32 \nu R T_1 = \frac 32 P_2 V_2 - \frac 32 P_1 V_1= \frac 32 \cdot \cbr{2{,}5\,\text{МПа} \cdot 4\,\text{л} - 3\,\text{МПа} \cdot 3\,\text{л}} = 1500\,\text{Дж}.
    \\
    \frac{T_2}{T_1} &= \frac{\frac{P_2V_2}{\nu R}}{\frac{P_1V_1}{\nu R}} = \frac{P_2V_2}{P_1V_1}= \frac{2{,}5\,\text{МПа} \cdot 4\,\text{л}}{3\,\text{МПа} \cdot 3\,\text{л}} \approx 1{,}11.
    \end{align*}
}
\solutionspace{80pt}

\tasknumber{6}%
\task{%
    $5\,\text{моль}$ идеального одноатомного газа нагрели на $30\,\text{К}$.
    Определите изменение внутренней энергии газа.
    Увеличилась она или уменьшилась?
    Универсальная газовая постоянная $R = 8{,}31\,\frac{\text{Дж}}{\text{моль}\cdot\text{К}}$.
}
\answer{%
    $
        \Delta U = \frac 32 \nu R \Delta T
            =  \frac 32 \cdot 5\,\text{моль} \cdot 8{,}31\,\frac{\text{Дж}}{\text{моль}\cdot\text{К}} \cdot 30\,\text{К}
            = 1869\,\text{Дж}.
            \text{Увеличилась.}
    $
}
\solutionspace{40pt}

\tasknumber{7}%
\task{%
    Газу сообщили некоторое количество теплоты,
    при этом четверть его он потратил на совершение работы,
    одновременно увеличив свою внутреннюю энергию на $1500\,\text{Дж}$.
    Определите количество теплоты, сообщённое газу.
}
\answer{%
    \begin{align*}
    Q &= A' + \Delta U, A' = \frac 14 Q \implies Q \cdot \cbr{1 - \frac 14} = \Delta U \implies Q = \frac{\Delta U}{1 - \frac 14} = \frac{1500\,\text{Дж}}{1 - \frac 14} \approx 2000\,\text{Дж}.
    \\
    A' &= \frac 14 Q
        = \frac 14 \cdot \frac{\Delta U}{1 - \frac 14}
        = \frac{\Delta U}{4 - 1}
        = \frac{1500\,\text{Дж}}{4 - 1} \approx 500\,\text{Дж}.
    \end{align*}
}
\solutionspace{60pt}

\tasknumber{8}%
\task{%
    В некотором процессе внешние силы совершили над газом работу $100\,\text{Дж}$,
    при этом его внутренняя энергия уменьшилась на $250\,\text{Дж}$.
    Определите количество тепла, переданное при этом процессе газу.
    Явно пропишите, подводили газу тепло или же отводили.
}
\answer{%
    $
        Q = A_\text{газа} + \Delta U, A_\text{газа} = -A_\text{внешняя}
        \implies Q = A_\text{газа} + \Delta U = - 100\,\text{Дж} - 250\,\text{Дж} = -350\,\text{Дж}.
        \text{ Отводили.}
    $
}

\variantsplitter

\addpersonalvariant{София Журавлёва}

\tasknumber{1}%
\task{%
    Напротив физических величин укажите их обозначения и единицы измерения в СИ, а в пункте «г)» запишите физический закон или формулу:
    \begin{enumerate}
        \item изменение внутренней энергии,
        \item работа газа,
        \item молярная теплоёмкость,
        \item внутренняя энергия идеального одноатомного газа.
    \end{enumerate}
}
\solutionspace{20pt}

\tasknumber{2}%
\task{%
    Определите объём идеального одноатомного газа,
    если его внутренняя энергия при давлении $5\,\text{атм}$ составляет $400\,\text{кДж}$.
    $p_{\text{aтм}} = 100\,\text{кПа}$.
}
\answer{%
    $U = \frac 32 \nu R T = \frac 32 PV \implies V = \frac 23 \cdot \frac UP= \frac 23 \cdot \frac{400\,\text{кДж}}{5\,\text{атм}} \approx 0{,}53\,\text{м}^{3}.$
}
\solutionspace{40pt}

\tasknumber{3}%
\task{%
    Газ расширился от $350\,\text{л}$ до $650\,\text{л}$.
    Давление газа при этом оставалось постоянным и равным $1{,}2\,\text{атм}$.
    Определите работу газа, ответ выразите в килоджоулях.
    $p_{\text{aтм}} = 100\,\text{кПа}$.
}
\answer{%
    $A = P\Delta V = P(V_2 - V_1) = 1{,}2\,\text{атм} \cdot \cbr{650\,\text{л} - 350\,\text{л}} = 36{,}0\,\text{кДж}.$
}
\solutionspace{40pt}

\tasknumber{4}%
\task{%
    $40\,\text{моль}$ идеального одноатомного газа в результате адиабатического процесса нагрелись на $80\,\text{К}$.
    Определите работу газа.
    Кто совершил положительную работу: газ или внешние силы?
    Универсальная газовая постоянная $R = 8{,}31\,\frac{\text{Дж}}{\text{моль}\cdot\text{К}}$.
}
\answer{%
    \begin{align*}
    Q &= 0, Q = \Delta U + A_\text{газа} \implies \\
    \implies A_\text{газа} &= - \Delta U = - \frac 32 \nu R \Delta T = - \frac 32 \cdot 40\,\text{моль} \cdot 8{,}31\,\frac{\text{Дж}}{\text{моль}\cdot\text{К}} \cdot 80\,\text{К}= -39{,}90\,\text{кДж}, \text{внешние силы.}
    \end{align*}
}
\solutionspace{40pt}

\tasknumber{5}%
\task{%
    Как изменилась внутренняя энергия одноатомного идеального газа при переходе из состояния 1 в состояние 2?
    $P_1 = 4\,\text{МПа}$, $V_1 = 3\,\text{л}$, $P_2 = 2{,}5\,\text{МПа}$, $V_2 = 6\,\text{л}$.
    Как изменилась при этом температура газа?
}
\answer{%
    \begin{align*}
    P_1V_1 &= \nu R T_1, P_2V_2 = \nu R T_2, \\
    \Delta U &= U_2-U_1 = \frac 32 \nu R T_2- \frac 32 \nu R T_1 = \frac 32 P_2 V_2 - \frac 32 P_1 V_1= \frac 32 \cdot \cbr{2{,}5\,\text{МПа} \cdot 6\,\text{л} - 4\,\text{МПа} \cdot 3\,\text{л}} = 4500\,\text{Дж}.
    \\
    \frac{T_2}{T_1} &= \frac{\frac{P_2V_2}{\nu R}}{\frac{P_1V_1}{\nu R}} = \frac{P_2V_2}{P_1V_1}= \frac{2{,}5\,\text{МПа} \cdot 6\,\text{л}}{4\,\text{МПа} \cdot 3\,\text{л}} \approx 1{,}25.
    \end{align*}
}
\solutionspace{80pt}

\tasknumber{6}%
\task{%
    $3\,\text{моль}$ идеального одноатомного газа нагрели на $20\,\text{К}$.
    Определите изменение внутренней энергии газа.
    Увеличилась она или уменьшилась?
    Универсальная газовая постоянная $R = 8{,}31\,\frac{\text{Дж}}{\text{моль}\cdot\text{К}}$.
}
\answer{%
    $
        \Delta U = \frac 32 \nu R \Delta T
            =  \frac 32 \cdot 3\,\text{моль} \cdot 8{,}31\,\frac{\text{Дж}}{\text{моль}\cdot\text{К}} \cdot 20\,\text{К}
            = 747\,\text{Дж}.
            \text{Увеличилась.}
    $
}
\solutionspace{40pt}

\tasknumber{7}%
\task{%
    Газу сообщили некоторое количество теплоты,
    при этом четверть его он потратил на совершение работы,
    одновременно увеличив свою внутреннюю энергию на $1500\,\text{Дж}$.
    Определите работу, совершённую газом.
}
\answer{%
    \begin{align*}
    Q &= A' + \Delta U, A' = \frac 14 Q \implies Q \cdot \cbr{1 - \frac 14} = \Delta U \implies Q = \frac{\Delta U}{1 - \frac 14} = \frac{1500\,\text{Дж}}{1 - \frac 14} \approx 2000\,\text{Дж}.
    \\
    A' &= \frac 14 Q
        = \frac 14 \cdot \frac{\Delta U}{1 - \frac 14}
        = \frac{\Delta U}{4 - 1}
        = \frac{1500\,\text{Дж}}{4 - 1} \approx 500\,\text{Дж}.
    \end{align*}
}
\solutionspace{60pt}

\tasknumber{8}%
\task{%
    В некотором процессе внешние силы совершили над газом работу $300\,\text{Дж}$,
    при этом его внутренняя энергия уменьшилась на $350\,\text{Дж}$.
    Определите количество тепла, переданное при этом процессе газу.
    Явно пропишите, подводили газу тепло или же отводили.
}
\answer{%
    $
        Q = A_\text{газа} + \Delta U, A_\text{газа} = -A_\text{внешняя}
        \implies Q = A_\text{газа} + \Delta U = - 300\,\text{Дж} - 350\,\text{Дж} = -650\,\text{Дж}.
        \text{ Отводили.}
    $
}

\variantsplitter

\addpersonalvariant{Константин Козлов}

\tasknumber{1}%
\task{%
    Напротив физических величин укажите их обозначения и единицы измерения в СИ, а в пункте «г)» запишите физический закон или формулу:
    \begin{enumerate}
        \item изменение внутренней энергии,
        \item работа газа,
        \item молярная теплоёмкость,
        \item внутренняя энергия идеального одноатомного газа.
    \end{enumerate}
}
\solutionspace{20pt}

\tasknumber{2}%
\task{%
    Определите объём идеального одноатомного газа,
    если его внутренняя энергия при давлении $5\,\text{атм}$ составляет $500\,\text{кДж}$.
    $p_{\text{aтм}} = 100\,\text{кПа}$.
}
\answer{%
    $U = \frac 32 \nu R T = \frac 32 PV \implies V = \frac 23 \cdot \frac UP= \frac 23 \cdot \frac{500\,\text{кДж}}{5\,\text{атм}} \approx 0{,}67\,\text{м}^{3}.$
}
\solutionspace{40pt}

\tasknumber{3}%
\task{%
    Газ расширился от $350\,\text{л}$ до $450\,\text{л}$.
    Давление газа при этом оставалось постоянным и равным $1{,}8\,\text{атм}$.
    Определите работу газа, ответ выразите в килоджоулях.
    $p_{\text{aтм}} = 100\,\text{кПа}$.
}
\answer{%
    $A = P\Delta V = P(V_2 - V_1) = 1{,}8\,\text{атм} \cdot \cbr{450\,\text{л} - 350\,\text{л}} = 18{,}0\,\text{кДж}.$
}
\solutionspace{40pt}

\tasknumber{4}%
\task{%
    $30\,\text{моль}$ идеального одноатомного газа в результате адиабатического процесса нагрелись на $15\,\text{К}$.
    Определите работу газа.
    Кто совершил положительную работу: газ или внешние силы?
    Универсальная газовая постоянная $R = 8{,}31\,\frac{\text{Дж}}{\text{моль}\cdot\text{К}}$.
}
\answer{%
    \begin{align*}
    Q &= 0, Q = \Delta U + A_\text{газа} \implies \\
    \implies A_\text{газа} &= - \Delta U = - \frac 32 \nu R \Delta T = - \frac 32 \cdot 30\,\text{моль} \cdot 8{,}31\,\frac{\text{Дж}}{\text{моль}\cdot\text{К}} \cdot 15\,\text{К}= -5{,}60\,\text{кДж}, \text{внешние силы.}
    \end{align*}
}
\solutionspace{40pt}

\tasknumber{5}%
\task{%
    Как изменилась внутренняя энергия одноатомного идеального газа при переходе из состояния 1 в состояние 2?
    $P_1 = 4\,\text{МПа}$, $V_1 = 3\,\text{л}$, $P_2 = 1{,}5\,\text{МПа}$, $V_2 = 8\,\text{л}$.
    Как изменилась при этом температура газа?
}
\answer{%
    \begin{align*}
    P_1V_1 &= \nu R T_1, P_2V_2 = \nu R T_2, \\
    \Delta U &= U_2-U_1 = \frac 32 \nu R T_2- \frac 32 \nu R T_1 = \frac 32 P_2 V_2 - \frac 32 P_1 V_1= \frac 32 \cdot \cbr{1{,}5\,\text{МПа} \cdot 8\,\text{л} - 4\,\text{МПа} \cdot 3\,\text{л}} = 0\,\text{Дж}.
    \\
    \frac{T_2}{T_1} &= \frac{\frac{P_2V_2}{\nu R}}{\frac{P_1V_1}{\nu R}} = \frac{P_2V_2}{P_1V_1}= \frac{1{,}5\,\text{МПа} \cdot 8\,\text{л}}{4\,\text{МПа} \cdot 3\,\text{л}} \approx 1{,}00.
    \end{align*}
}
\solutionspace{80pt}

\tasknumber{6}%
\task{%
    $2\,\text{моль}$ идеального одноатомного газа нагрели на $10\,\text{К}$.
    Определите изменение внутренней энергии газа.
    Увеличилась она или уменьшилась?
    Универсальная газовая постоянная $R = 8{,}31\,\frac{\text{Дж}}{\text{моль}\cdot\text{К}}$.
}
\answer{%
    $
        \Delta U = \frac 32 \nu R \Delta T
            =  \frac 32 \cdot 2\,\text{моль} \cdot 8{,}31\,\frac{\text{Дж}}{\text{моль}\cdot\text{К}} \cdot 10\,\text{К}
            = 249\,\text{Дж}.
            \text{Увеличилась.}
    $
}
\solutionspace{40pt}

\tasknumber{7}%
\task{%
    Газу сообщили некоторое количество теплоты,
    при этом половину его он потратил на совершение работы,
    одновременно увеличив свою внутреннюю энергию на $1500\,\text{Дж}$.
    Определите количество теплоты, сообщённое газу.
}
\answer{%
    \begin{align*}
    Q &= A' + \Delta U, A' = \frac 12 Q \implies Q \cdot \cbr{1 - \frac 12} = \Delta U \implies Q = \frac{\Delta U}{1 - \frac 12} = \frac{1500\,\text{Дж}}{1 - \frac 12} \approx 3000\,\text{Дж}.
    \\
    A' &= \frac 12 Q
        = \frac 12 \cdot \frac{\Delta U}{1 - \frac 12}
        = \frac{\Delta U}{2 - 1}
        = \frac{1500\,\text{Дж}}{2 - 1} \approx 1500\,\text{Дж}.
    \end{align*}
}
\solutionspace{60pt}

\tasknumber{8}%
\task{%
    В некотором процессе внешние силы совершили над газом работу $200\,\text{Дж}$,
    при этом его внутренняя энергия увеличилась на $250\,\text{Дж}$.
    Определите количество тепла, переданное при этом процессе газу.
    Явно пропишите, подводили газу тепло или же отводили.
}
\answer{%
    $
        Q = A_\text{газа} + \Delta U, A_\text{газа} = -A_\text{внешняя}
        \implies Q = A_\text{газа} + \Delta U = - 200\,\text{Дж} +  250\,\text{Дж} = 50\,\text{Дж}.
        \text{ Подводили.}
    $
}

\variantsplitter

\addpersonalvariant{Наталья Кравченко}

\tasknumber{1}%
\task{%
    Напротив физических величин укажите их обозначения и единицы измерения в СИ, а в пункте «г)» запишите физический закон или формулу:
    \begin{enumerate}
        \item количество теплоты,
        \item работа внешних сил,
        \item удельная теплоёмкость,
        \item внутренняя энергия идеального одноатомного газа.
    \end{enumerate}
}
\solutionspace{20pt}

\tasknumber{2}%
\task{%
    Определите объём идеального одноатомного газа,
    если его внутренняя энергия при давлении $3\,\text{атм}$ составляет $400\,\text{кДж}$.
    $p_{\text{aтм}} = 100\,\text{кПа}$.
}
\answer{%
    $U = \frac 32 \nu R T = \frac 32 PV \implies V = \frac 23 \cdot \frac UP= \frac 23 \cdot \frac{400\,\text{кДж}}{3\,\text{атм}} \approx 0{,}89\,\text{м}^{3}.$
}
\solutionspace{40pt}

\tasknumber{3}%
\task{%
    Газ расширился от $350\,\text{л}$ до $450\,\text{л}$.
    Давление газа при этом оставалось постоянным и равным $3{,}5\,\text{атм}$.
    Определите работу газа, ответ выразите в килоджоулях.
    $p_{\text{aтм}} = 100\,\text{кПа}$.
}
\answer{%
    $A = P\Delta V = P(V_2 - V_1) = 3{,}5\,\text{атм} \cdot \cbr{450\,\text{л} - 350\,\text{л}} = 35{,}0\,\text{кДж}.$
}
\solutionspace{40pt}

\tasknumber{4}%
\task{%
    $30\,\text{моль}$ идеального одноатомного газа в результате адиабатического процесса нагрелись на $25\,\text{К}$.
    Определите работу газа.
    Кто совершил положительную работу: газ или внешние силы?
    Универсальная газовая постоянная $R = 8{,}31\,\frac{\text{Дж}}{\text{моль}\cdot\text{К}}$.
}
\answer{%
    \begin{align*}
    Q &= 0, Q = \Delta U + A_\text{газа} \implies \\
    \implies A_\text{газа} &= - \Delta U = - \frac 32 \nu R \Delta T = - \frac 32 \cdot 30\,\text{моль} \cdot 8{,}31\,\frac{\text{Дж}}{\text{моль}\cdot\text{К}} \cdot 25\,\text{К}= -9{,}30\,\text{кДж}, \text{внешние силы.}
    \end{align*}
}
\solutionspace{40pt}

\tasknumber{5}%
\task{%
    Как изменилась внутренняя энергия одноатомного идеального газа при переходе из состояния 1 в состояние 2?
    $P_1 = 4\,\text{МПа}$, $V_1 = 7\,\text{л}$, $P_2 = 2{,}5\,\text{МПа}$, $V_2 = 4\,\text{л}$.
    Как изменилась при этом температура газа?
}
\answer{%
    \begin{align*}
    P_1V_1 &= \nu R T_1, P_2V_2 = \nu R T_2, \\
    \Delta U &= U_2-U_1 = \frac 32 \nu R T_2- \frac 32 \nu R T_1 = \frac 32 P_2 V_2 - \frac 32 P_1 V_1= \frac 32 \cdot \cbr{2{,}5\,\text{МПа} \cdot 4\,\text{л} - 4\,\text{МПа} \cdot 7\,\text{л}} = -27000\,\text{Дж}.
    \\
    \frac{T_2}{T_1} &= \frac{\frac{P_2V_2}{\nu R}}{\frac{P_1V_1}{\nu R}} = \frac{P_2V_2}{P_1V_1}= \frac{2{,}5\,\text{МПа} \cdot 4\,\text{л}}{4\,\text{МПа} \cdot 7\,\text{л}} \approx 0{,}36.
    \end{align*}
}
\solutionspace{80pt}

\tasknumber{6}%
\task{%
    $2\,\text{моль}$ идеального одноатомного газа нагрели на $20\,\text{К}$.
    Определите изменение внутренней энергии газа.
    Увеличилась она или уменьшилась?
    Универсальная газовая постоянная $R = 8{,}31\,\frac{\text{Дж}}{\text{моль}\cdot\text{К}}$.
}
\answer{%
    $
        \Delta U = \frac 32 \nu R \Delta T
            =  \frac 32 \cdot 2\,\text{моль} \cdot 8{,}31\,\frac{\text{Дж}}{\text{моль}\cdot\text{К}} \cdot 20\,\text{К}
            = 498\,\text{Дж}.
            \text{Увеличилась.}
    $
}
\solutionspace{40pt}

\tasknumber{7}%
\task{%
    Газу сообщили некоторое количество теплоты,
    при этом четверть его он потратил на совершение работы,
    одновременно увеличив свою внутреннюю энергию на $2400\,\text{Дж}$.
    Определите количество теплоты, сообщённое газу.
}
\answer{%
    \begin{align*}
    Q &= A' + \Delta U, A' = \frac 14 Q \implies Q \cdot \cbr{1 - \frac 14} = \Delta U \implies Q = \frac{\Delta U}{1 - \frac 14} = \frac{2400\,\text{Дж}}{1 - \frac 14} \approx 3200\,\text{Дж}.
    \\
    A' &= \frac 14 Q
        = \frac 14 \cdot \frac{\Delta U}{1 - \frac 14}
        = \frac{\Delta U}{4 - 1}
        = \frac{2400\,\text{Дж}}{4 - 1} \approx 800\,\text{Дж}.
    \end{align*}
}
\solutionspace{60pt}

\tasknumber{8}%
\task{%
    В некотором процессе внешние силы совершили над газом работу $300\,\text{Дж}$,
    при этом его внутренняя энергия уменьшилась на $450\,\text{Дж}$.
    Определите количество тепла, переданное при этом процессе газу.
    Явно пропишите, подводили газу тепло или же отводили.
}
\answer{%
    $
        Q = A_\text{газа} + \Delta U, A_\text{газа} = -A_\text{внешняя}
        \implies Q = A_\text{газа} + \Delta U = - 300\,\text{Дж} - 450\,\text{Дж} = -750\,\text{Дж}.
        \text{ Отводили.}
    $
}

\variantsplitter

\addpersonalvariant{Матвей Кузьмин}

\tasknumber{1}%
\task{%
    Напротив физических величин укажите их обозначения и единицы измерения в СИ, а в пункте «г)» запишите физический закон или формулу:
    \begin{enumerate}
        \item изменение внутренней энергии,
        \item работа внешних сил,
        \item молярная теплоёмкость,
        \item внутренняя энергия идеального одноатомного газа.
    \end{enumerate}
}
\solutionspace{20pt}

\tasknumber{2}%
\task{%
    Определите объём идеального одноатомного газа,
    если его внутренняя энергия при давлении $6\,\text{атм}$ составляет $300\,\text{кДж}$.
    $p_{\text{aтм}} = 100\,\text{кПа}$.
}
\answer{%
    $U = \frac 32 \nu R T = \frac 32 PV \implies V = \frac 23 \cdot \frac UP= \frac 23 \cdot \frac{300\,\text{кДж}}{6\,\text{атм}} \approx 0{,}33\,\text{м}^{3}.$
}
\solutionspace{40pt}

\tasknumber{3}%
\task{%
    Газ расширился от $200\,\text{л}$ до $450\,\text{л}$.
    Давление газа при этом оставалось постоянным и равным $2{,}5\,\text{атм}$.
    Определите работу газа, ответ выразите в килоджоулях.
    $p_{\text{aтм}} = 100\,\text{кПа}$.
}
\answer{%
    $A = P\Delta V = P(V_2 - V_1) = 2{,}5\,\text{атм} \cdot \cbr{450\,\text{л} - 200\,\text{л}} = 62{,}5\,\text{кДж}.$
}
\solutionspace{40pt}

\tasknumber{4}%
\task{%
    $30\,\text{моль}$ идеального одноатомного газа в результате адиабатического процесса остыли на $80\,\text{К}$.
    Определите работу газа.
    Кто совершил положительную работу: газ или внешние силы?
    Универсальная газовая постоянная $R = 8{,}31\,\frac{\text{Дж}}{\text{моль}\cdot\text{К}}$.
}
\answer{%
    \begin{align*}
    Q &= 0, Q = \Delta U + A_\text{газа} \implies \\
    \implies A_\text{газа} &= - \Delta U = - \frac 32 \nu R \Delta T =  \frac 32 \cdot 30\,\text{моль} \cdot 8{,}31\,\frac{\text{Дж}}{\text{моль}\cdot\text{К}} \cdot 80\,\text{К}= 29{,}9\,\text{кДж}, \text{газ.}
    \end{align*}
}
\solutionspace{40pt}

\tasknumber{5}%
\task{%
    Как изменилась внутренняя энергия одноатомного идеального газа при переходе из состояния 1 в состояние 2?
    $P_1 = 4\,\text{МПа}$, $V_1 = 3\,\text{л}$, $P_2 = 3{,}5\,\text{МПа}$, $V_2 = 2\,\text{л}$.
    Как изменилась при этом температура газа?
}
\answer{%
    \begin{align*}
    P_1V_1 &= \nu R T_1, P_2V_2 = \nu R T_2, \\
    \Delta U &= U_2-U_1 = \frac 32 \nu R T_2- \frac 32 \nu R T_1 = \frac 32 P_2 V_2 - \frac 32 P_1 V_1= \frac 32 \cdot \cbr{3{,}5\,\text{МПа} \cdot 2\,\text{л} - 4\,\text{МПа} \cdot 3\,\text{л}} = -7500\,\text{Дж}.
    \\
    \frac{T_2}{T_1} &= \frac{\frac{P_2V_2}{\nu R}}{\frac{P_1V_1}{\nu R}} = \frac{P_2V_2}{P_1V_1}= \frac{3{,}5\,\text{МПа} \cdot 2\,\text{л}}{4\,\text{МПа} \cdot 3\,\text{л}} \approx 0{,}58.
    \end{align*}
}
\solutionspace{80pt}

\tasknumber{6}%
\task{%
    $2\,\text{моль}$ идеального одноатомного газа охладили на $20\,\text{К}$.
    Определите изменение внутренней энергии газа.
    Увеличилась она или уменьшилась?
    Универсальная газовая постоянная $R = 8{,}31\,\frac{\text{Дж}}{\text{моль}\cdot\text{К}}$.
}
\answer{%
    $
        \Delta U = \frac 32 \nu R \Delta T
            = - \frac 32 \cdot 2\,\text{моль} \cdot 8{,}31\,\frac{\text{Дж}}{\text{моль}\cdot\text{К}} \cdot 20\,\text{К}
            = -498\,\text{Дж}.
            \text{Уменьшилась.}
    $
}
\solutionspace{40pt}

\tasknumber{7}%
\task{%
    Газу сообщили некоторое количество теплоты,
    при этом треть его он потратил на совершение работы,
    одновременно увеличив свою внутреннюю энергию на $1200\,\text{Дж}$.
    Определите количество теплоты, сообщённое газу.
}
\answer{%
    \begin{align*}
    Q &= A' + \Delta U, A' = \frac 13 Q \implies Q \cdot \cbr{1 - \frac 13} = \Delta U \implies Q = \frac{\Delta U}{1 - \frac 13} = \frac{1200\,\text{Дж}}{1 - \frac 13} \approx 1800\,\text{Дж}.
    \\
    A' &= \frac 13 Q
        = \frac 13 \cdot \frac{\Delta U}{1 - \frac 13}
        = \frac{\Delta U}{3 - 1}
        = \frac{1200\,\text{Дж}}{3 - 1} \approx 600\,\text{Дж}.
    \end{align*}
}
\solutionspace{60pt}

\tasknumber{8}%
\task{%
    В некотором процессе газ совершил работу $200\,\text{Дж}$,
    при этом его внутренняя энергия увеличилась на $450\,\text{Дж}$.
    Определите количество тепла, переданное при этом процессе газу.
    Явно пропишите, подводили газу тепло или же отводили.
}
\answer{%
    $
        Q = A_\text{газа} + \Delta U, A_\text{газа} = -A_\text{внешняя}
        \implies Q = A_\text{газа} + \Delta U =  200\,\text{Дж} +  450\,\text{Дж} = 650\,\text{Дж}.
        \text{ Подводили.}
    $
}

\variantsplitter

\addpersonalvariant{Сергей Малышев}

\tasknumber{1}%
\task{%
    Напротив физических величин укажите их обозначения и единицы измерения в СИ, а в пункте «г)» запишите физический закон или формулу:
    \begin{enumerate}
        \item изменение внутренней энергии,
        \item работа внешних сил,
        \item удельная теплоёмкость,
        \item первое начало термодинамики.
    \end{enumerate}
}
\solutionspace{20pt}

\tasknumber{2}%
\task{%
    Определите объём идеального одноатомного газа,
    если его внутренняя энергия при давлении $2\,\text{атм}$ составляет $250\,\text{кДж}$.
    $p_{\text{aтм}} = 100\,\text{кПа}$.
}
\answer{%
    $U = \frac 32 \nu R T = \frac 32 PV \implies V = \frac 23 \cdot \frac UP= \frac 23 \cdot \frac{250\,\text{кДж}}{2\,\text{атм}} \approx 0{,}83\,\text{м}^{3}.$
}
\solutionspace{40pt}

\tasknumber{3}%
\task{%
    Газ расширился от $350\,\text{л}$ до $550\,\text{л}$.
    Давление газа при этом оставалось постоянным и равным $1{,}8\,\text{атм}$.
    Определите работу газа, ответ выразите в килоджоулях.
    $p_{\text{aтм}} = 100\,\text{кПа}$.
}
\answer{%
    $A = P\Delta V = P(V_2 - V_1) = 1{,}8\,\text{атм} \cdot \cbr{550\,\text{л} - 350\,\text{л}} = 36{,}0\,\text{кДж}.$
}
\solutionspace{40pt}

\tasknumber{4}%
\task{%
    $40\,\text{моль}$ идеального одноатомного газа в результате адиабатического процесса нагрелись на $15\,\text{К}$.
    Определите работу газа.
    Кто совершил положительную работу: газ или внешние силы?
    Универсальная газовая постоянная $R = 8{,}31\,\frac{\text{Дж}}{\text{моль}\cdot\text{К}}$.
}
\answer{%
    \begin{align*}
    Q &= 0, Q = \Delta U + A_\text{газа} \implies \\
    \implies A_\text{газа} &= - \Delta U = - \frac 32 \nu R \Delta T = - \frac 32 \cdot 40\,\text{моль} \cdot 8{,}31\,\frac{\text{Дж}}{\text{моль}\cdot\text{К}} \cdot 15\,\text{К}= -7{,}50\,\text{кДж}, \text{внешние силы.}
    \end{align*}
}
\solutionspace{40pt}

\tasknumber{5}%
\task{%
    Как изменилась внутренняя энергия одноатомного идеального газа при переходе из состояния 1 в состояние 2?
    $P_1 = 4\,\text{МПа}$, $V_1 = 3\,\text{л}$, $P_2 = 2{,}5\,\text{МПа}$, $V_2 = 2\,\text{л}$.
    Как изменилась при этом температура газа?
}
\answer{%
    \begin{align*}
    P_1V_1 &= \nu R T_1, P_2V_2 = \nu R T_2, \\
    \Delta U &= U_2-U_1 = \frac 32 \nu R T_2- \frac 32 \nu R T_1 = \frac 32 P_2 V_2 - \frac 32 P_1 V_1= \frac 32 \cdot \cbr{2{,}5\,\text{МПа} \cdot 2\,\text{л} - 4\,\text{МПа} \cdot 3\,\text{л}} = -10500\,\text{Дж}.
    \\
    \frac{T_2}{T_1} &= \frac{\frac{P_2V_2}{\nu R}}{\frac{P_1V_1}{\nu R}} = \frac{P_2V_2}{P_1V_1}= \frac{2{,}5\,\text{МПа} \cdot 2\,\text{л}}{4\,\text{МПа} \cdot 3\,\text{л}} \approx 0{,}42.
    \end{align*}
}
\solutionspace{80pt}

\tasknumber{6}%
\task{%
    $3\,\text{моль}$ идеального одноатомного газа нагрели на $10\,\text{К}$.
    Определите изменение внутренней энергии газа.
    Увеличилась она или уменьшилась?
    Универсальная газовая постоянная $R = 8{,}31\,\frac{\text{Дж}}{\text{моль}\cdot\text{К}}$.
}
\answer{%
    $
        \Delta U = \frac 32 \nu R \Delta T
            =  \frac 32 \cdot 3\,\text{моль} \cdot 8{,}31\,\frac{\text{Дж}}{\text{моль}\cdot\text{К}} \cdot 10\,\text{К}
            = 373\,\text{Дж}.
            \text{Увеличилась.}
    $
}
\solutionspace{40pt}

\tasknumber{7}%
\task{%
    Газу сообщили некоторое количество теплоты,
    при этом четверть его он потратил на совершение работы,
    одновременно увеличив свою внутреннюю энергию на $3000\,\text{Дж}$.
    Определите количество теплоты, сообщённое газу.
}
\answer{%
    \begin{align*}
    Q &= A' + \Delta U, A' = \frac 14 Q \implies Q \cdot \cbr{1 - \frac 14} = \Delta U \implies Q = \frac{\Delta U}{1 - \frac 14} = \frac{3000\,\text{Дж}}{1 - \frac 14} \approx 4000\,\text{Дж}.
    \\
    A' &= \frac 14 Q
        = \frac 14 \cdot \frac{\Delta U}{1 - \frac 14}
        = \frac{\Delta U}{4 - 1}
        = \frac{3000\,\text{Дж}}{4 - 1} \approx 1000\,\text{Дж}.
    \end{align*}
}
\solutionspace{60pt}

\tasknumber{8}%
\task{%
    В некотором процессе газ совершил работу $100\,\text{Дж}$,
    при этом его внутренняя энергия увеличилась на $350\,\text{Дж}$.
    Определите количество тепла, переданное при этом процессе газу.
    Явно пропишите, подводили газу тепло или же отводили.
}
\answer{%
    $
        Q = A_\text{газа} + \Delta U, A_\text{газа} = -A_\text{внешняя}
        \implies Q = A_\text{газа} + \Delta U =  100\,\text{Дж} +  350\,\text{Дж} = 450\,\text{Дж}.
        \text{ Подводили.}
    $
}

\variantsplitter

\addpersonalvariant{Алина Полканова}

\tasknumber{1}%
\task{%
    Напротив физических величин укажите их обозначения и единицы измерения в СИ, а в пункте «г)» запишите физический закон или формулу:
    \begin{enumerate}
        \item количество теплоты,
        \item работа внешних сил,
        \item удельная теплоёмкость,
        \item внутренняя энергия идеального одноатомного газа.
    \end{enumerate}
}
\solutionspace{20pt}

\tasknumber{2}%
\task{%
    Определите объём идеального одноатомного газа,
    если его внутренняя энергия при давлении $4\,\text{атм}$ составляет $300\,\text{кДж}$.
    $p_{\text{aтм}} = 100\,\text{кПа}$.
}
\answer{%
    $U = \frac 32 \nu R T = \frac 32 PV \implies V = \frac 23 \cdot \frac UP= \frac 23 \cdot \frac{300\,\text{кДж}}{4\,\text{атм}} \approx 0{,}50\,\text{м}^{3}.$
}
\solutionspace{40pt}

\tasknumber{3}%
\task{%
    Газ расширился от $200\,\text{л}$ до $550\,\text{л}$.
    Давление газа при этом оставалось постоянным и равным $1{,}2\,\text{атм}$.
    Определите работу газа, ответ выразите в килоджоулях.
    $p_{\text{aтм}} = 100\,\text{кПа}$.
}
\answer{%
    $A = P\Delta V = P(V_2 - V_1) = 1{,}2\,\text{атм} \cdot \cbr{550\,\text{л} - 200\,\text{л}} = 42{,}0\,\text{кДж}.$
}
\solutionspace{40pt}

\tasknumber{4}%
\task{%
    $50\,\text{моль}$ идеального одноатомного газа в результате адиабатического процесса нагрелись на $25\,\text{К}$.
    Определите работу газа.
    Кто совершил положительную работу: газ или внешние силы?
    Универсальная газовая постоянная $R = 8{,}31\,\frac{\text{Дж}}{\text{моль}\cdot\text{К}}$.
}
\answer{%
    \begin{align*}
    Q &= 0, Q = \Delta U + A_\text{газа} \implies \\
    \implies A_\text{газа} &= - \Delta U = - \frac 32 \nu R \Delta T = - \frac 32 \cdot 50\,\text{моль} \cdot 8{,}31\,\frac{\text{Дж}}{\text{моль}\cdot\text{К}} \cdot 25\,\text{К}= -15{,}600\,\text{кДж}, \text{внешние силы.}
    \end{align*}
}
\solutionspace{40pt}

\tasknumber{5}%
\task{%
    Как изменилась внутренняя энергия одноатомного идеального газа при переходе из состояния 1 в состояние 2?
    $P_1 = 2\,\text{МПа}$, $V_1 = 3\,\text{л}$, $P_2 = 4{,}5\,\text{МПа}$, $V_2 = 6\,\text{л}$.
    Как изменилась при этом температура газа?
}
\answer{%
    \begin{align*}
    P_1V_1 &= \nu R T_1, P_2V_2 = \nu R T_2, \\
    \Delta U &= U_2-U_1 = \frac 32 \nu R T_2- \frac 32 \nu R T_1 = \frac 32 P_2 V_2 - \frac 32 P_1 V_1= \frac 32 \cdot \cbr{4{,}5\,\text{МПа} \cdot 6\,\text{л} - 2\,\text{МПа} \cdot 3\,\text{л}} = 31500\,\text{Дж}.
    \\
    \frac{T_2}{T_1} &= \frac{\frac{P_2V_2}{\nu R}}{\frac{P_1V_1}{\nu R}} = \frac{P_2V_2}{P_1V_1}= \frac{4{,}5\,\text{МПа} \cdot 6\,\text{л}}{2\,\text{МПа} \cdot 3\,\text{л}} \approx 4{,}50.
    \end{align*}
}
\solutionspace{80pt}

\tasknumber{6}%
\task{%
    $4\,\text{моль}$ идеального одноатомного газа нагрели на $20\,\text{К}$.
    Определите изменение внутренней энергии газа.
    Увеличилась она или уменьшилась?
    Универсальная газовая постоянная $R = 8{,}31\,\frac{\text{Дж}}{\text{моль}\cdot\text{К}}$.
}
\answer{%
    $
        \Delta U = \frac 32 \nu R \Delta T
            =  \frac 32 \cdot 4\,\text{моль} \cdot 8{,}31\,\frac{\text{Дж}}{\text{моль}\cdot\text{К}} \cdot 20\,\text{К}
            = 997\,\text{Дж}.
            \text{Увеличилась.}
    $
}
\solutionspace{40pt}

\tasknumber{7}%
\task{%
    Газу сообщили некоторое количество теплоты,
    при этом половину его он потратил на совершение работы,
    одновременно увеличив свою внутреннюю энергию на $1500\,\text{Дж}$.
    Определите работу, совершённую газом.
}
\answer{%
    \begin{align*}
    Q &= A' + \Delta U, A' = \frac 12 Q \implies Q \cdot \cbr{1 - \frac 12} = \Delta U \implies Q = \frac{\Delta U}{1 - \frac 12} = \frac{1500\,\text{Дж}}{1 - \frac 12} \approx 3000\,\text{Дж}.
    \\
    A' &= \frac 12 Q
        = \frac 12 \cdot \frac{\Delta U}{1 - \frac 12}
        = \frac{\Delta U}{2 - 1}
        = \frac{1500\,\text{Дж}}{2 - 1} \approx 1500\,\text{Дж}.
    \end{align*}
}
\solutionspace{60pt}

\tasknumber{8}%
\task{%
    В некотором процессе внешние силы совершили над газом работу $300\,\text{Дж}$,
    при этом его внутренняя энергия увеличилась на $350\,\text{Дж}$.
    Определите количество тепла, переданное при этом процессе газу.
    Явно пропишите, подводили газу тепло или же отводили.
}
\answer{%
    $
        Q = A_\text{газа} + \Delta U, A_\text{газа} = -A_\text{внешняя}
        \implies Q = A_\text{газа} + \Delta U = - 300\,\text{Дж} +  350\,\text{Дж} = 50\,\text{Дж}.
        \text{ Подводили.}
    $
}

\variantsplitter

\addpersonalvariant{Сергей Пономарёв}

\tasknumber{1}%
\task{%
    Напротив физических величин укажите их обозначения и единицы измерения в СИ, а в пункте «г)» запишите физический закон или формулу:
    \begin{enumerate}
        \item количество теплоты,
        \item работа внешних сил,
        \item удельная теплоёмкость,
        \item первое начало термодинамики.
    \end{enumerate}
}
\solutionspace{20pt}

\tasknumber{2}%
\task{%
    Определите объём идеального одноатомного газа,
    если его внутренняя энергия при давлении $4\,\text{атм}$ составляет $400\,\text{кДж}$.
    $p_{\text{aтм}} = 100\,\text{кПа}$.
}
\answer{%
    $U = \frac 32 \nu R T = \frac 32 PV \implies V = \frac 23 \cdot \frac UP= \frac 23 \cdot \frac{400\,\text{кДж}}{4\,\text{атм}} \approx 0{,}67\,\text{м}^{3}.$
}
\solutionspace{40pt}

\tasknumber{3}%
\task{%
    Газ расширился от $250\,\text{л}$ до $550\,\text{л}$.
    Давление газа при этом оставалось постоянным и равным $1{,}8\,\text{атм}$.
    Определите работу газа, ответ выразите в килоджоулях.
    $p_{\text{aтм}} = 100\,\text{кПа}$.
}
\answer{%
    $A = P\Delta V = P(V_2 - V_1) = 1{,}8\,\text{атм} \cdot \cbr{550\,\text{л} - 250\,\text{л}} = 54{,}0\,\text{кДж}.$
}
\solutionspace{40pt}

\tasknumber{4}%
\task{%
    $60\,\text{моль}$ идеального одноатомного газа в результате адиабатического процесса остыли на $15\,\text{К}$.
    Определите работу газа.
    Кто совершил положительную работу: газ или внешние силы?
    Универсальная газовая постоянная $R = 8{,}31\,\frac{\text{Дж}}{\text{моль}\cdot\text{К}}$.
}
\answer{%
    \begin{align*}
    Q &= 0, Q = \Delta U + A_\text{газа} \implies \\
    \implies A_\text{газа} &= - \Delta U = - \frac 32 \nu R \Delta T =  \frac 32 \cdot 60\,\text{моль} \cdot 8{,}31\,\frac{\text{Дж}}{\text{моль}\cdot\text{К}} \cdot 15\,\text{К}= 11{,}2\,\text{кДж}, \text{газ.}
    \end{align*}
}
\solutionspace{40pt}

\tasknumber{5}%
\task{%
    Как изменилась внутренняя энергия одноатомного идеального газа при переходе из состояния 1 в состояние 2?
    $P_1 = 2\,\text{МПа}$, $V_1 = 5\,\text{л}$, $P_2 = 1{,}5\,\text{МПа}$, $V_2 = 2\,\text{л}$.
    Как изменилась при этом температура газа?
}
\answer{%
    \begin{align*}
    P_1V_1 &= \nu R T_1, P_2V_2 = \nu R T_2, \\
    \Delta U &= U_2-U_1 = \frac 32 \nu R T_2- \frac 32 \nu R T_1 = \frac 32 P_2 V_2 - \frac 32 P_1 V_1= \frac 32 \cdot \cbr{1{,}5\,\text{МПа} \cdot 2\,\text{л} - 2\,\text{МПа} \cdot 5\,\text{л}} = -10500\,\text{Дж}.
    \\
    \frac{T_2}{T_1} &= \frac{\frac{P_2V_2}{\nu R}}{\frac{P_1V_1}{\nu R}} = \frac{P_2V_2}{P_1V_1}= \frac{1{,}5\,\text{МПа} \cdot 2\,\text{л}}{2\,\text{МПа} \cdot 5\,\text{л}} \approx 0{,}30.
    \end{align*}
}
\solutionspace{80pt}

\tasknumber{6}%
\task{%
    $5\,\text{моль}$ идеального одноатомного газа охладили на $10\,\text{К}$.
    Определите изменение внутренней энергии газа.
    Увеличилась она или уменьшилась?
    Универсальная газовая постоянная $R = 8{,}31\,\frac{\text{Дж}}{\text{моль}\cdot\text{К}}$.
}
\answer{%
    $
        \Delta U = \frac 32 \nu R \Delta T
            = - \frac 32 \cdot 5\,\text{моль} \cdot 8{,}31\,\frac{\text{Дж}}{\text{моль}\cdot\text{К}} \cdot 10\,\text{К}
            = -623\,\text{Дж}.
            \text{Уменьшилась.}
    $
}
\solutionspace{40pt}

\tasknumber{7}%
\task{%
    Газу сообщили некоторое количество теплоты,
    при этом треть его он потратил на совершение работы,
    одновременно увеличив свою внутреннюю энергию на $1200\,\text{Дж}$.
    Определите количество теплоты, сообщённое газу.
}
\answer{%
    \begin{align*}
    Q &= A' + \Delta U, A' = \frac 13 Q \implies Q \cdot \cbr{1 - \frac 13} = \Delta U \implies Q = \frac{\Delta U}{1 - \frac 13} = \frac{1200\,\text{Дж}}{1 - \frac 13} \approx 1800\,\text{Дж}.
    \\
    A' &= \frac 13 Q
        = \frac 13 \cdot \frac{\Delta U}{1 - \frac 13}
        = \frac{\Delta U}{3 - 1}
        = \frac{1200\,\text{Дж}}{3 - 1} \approx 600\,\text{Дж}.
    \end{align*}
}
\solutionspace{60pt}

\tasknumber{8}%
\task{%
    В некотором процессе газ совершил работу $100\,\text{Дж}$,
    при этом его внутренняя энергия увеличилась на $250\,\text{Дж}$.
    Определите количество тепла, переданное при этом процессе газу.
    Явно пропишите, подводили газу тепло или же отводили.
}
\answer{%
    $
        Q = A_\text{газа} + \Delta U, A_\text{газа} = -A_\text{внешняя}
        \implies Q = A_\text{газа} + \Delta U =  100\,\text{Дж} +  250\,\text{Дж} = 350\,\text{Дж}.
        \text{ Подводили.}
    $
}

\variantsplitter

\addpersonalvariant{Егор Свистушкин}

\tasknumber{1}%
\task{%
    Напротив физических величин укажите их обозначения и единицы измерения в СИ, а в пункте «г)» запишите физический закон или формулу:
    \begin{enumerate}
        \item количество теплоты,
        \item работа газа,
        \item удельная теплоёмкость,
        \item внутренняя энергия идеального одноатомного газа.
    \end{enumerate}
}
\solutionspace{20pt}

\tasknumber{2}%
\task{%
    Определите объём идеального одноатомного газа,
    если его внутренняя энергия при давлении $6\,\text{атм}$ составляет $300\,\text{кДж}$.
    $p_{\text{aтм}} = 100\,\text{кПа}$.
}
\answer{%
    $U = \frac 32 \nu R T = \frac 32 PV \implies V = \frac 23 \cdot \frac UP= \frac 23 \cdot \frac{300\,\text{кДж}}{6\,\text{атм}} \approx 0{,}33\,\text{м}^{3}.$
}
\solutionspace{40pt}

\tasknumber{3}%
\task{%
    Газ расширился от $350\,\text{л}$ до $650\,\text{л}$.
    Давление газа при этом оставалось постоянным и равным $1{,}2\,\text{атм}$.
    Определите работу газа, ответ выразите в килоджоулях.
    $p_{\text{aтм}} = 100\,\text{кПа}$.
}
\answer{%
    $A = P\Delta V = P(V_2 - V_1) = 1{,}2\,\text{атм} \cdot \cbr{650\,\text{л} - 350\,\text{л}} = 36{,}0\,\text{кДж}.$
}
\solutionspace{40pt}

\tasknumber{4}%
\task{%
    $30\,\text{моль}$ идеального одноатомного газа в результате адиабатического процесса остыли на $120\,\text{К}$.
    Определите работу газа.
    Кто совершил положительную работу: газ или внешние силы?
    Универсальная газовая постоянная $R = 8{,}31\,\frac{\text{Дж}}{\text{моль}\cdot\text{К}}$.
}
\answer{%
    \begin{align*}
    Q &= 0, Q = \Delta U + A_\text{газа} \implies \\
    \implies A_\text{газа} &= - \Delta U = - \frac 32 \nu R \Delta T =  \frac 32 \cdot 30\,\text{моль} \cdot 8{,}31\,\frac{\text{Дж}}{\text{моль}\cdot\text{К}} \cdot 120\,\text{К}= 44{,}9\,\text{кДж}, \text{газ.}
    \end{align*}
}
\solutionspace{40pt}

\tasknumber{5}%
\task{%
    Как изменилась внутренняя энергия одноатомного идеального газа при переходе из состояния 1 в состояние 2?
    $P_1 = 2\,\text{МПа}$, $V_1 = 5\,\text{л}$, $P_2 = 2{,}5\,\text{МПа}$, $V_2 = 8\,\text{л}$.
    Как изменилась при этом температура газа?
}
\answer{%
    \begin{align*}
    P_1V_1 &= \nu R T_1, P_2V_2 = \nu R T_2, \\
    \Delta U &= U_2-U_1 = \frac 32 \nu R T_2- \frac 32 \nu R T_1 = \frac 32 P_2 V_2 - \frac 32 P_1 V_1= \frac 32 \cdot \cbr{2{,}5\,\text{МПа} \cdot 8\,\text{л} - 2\,\text{МПа} \cdot 5\,\text{л}} = 15000\,\text{Дж}.
    \\
    \frac{T_2}{T_1} &= \frac{\frac{P_2V_2}{\nu R}}{\frac{P_1V_1}{\nu R}} = \frac{P_2V_2}{P_1V_1}= \frac{2{,}5\,\text{МПа} \cdot 8\,\text{л}}{2\,\text{МПа} \cdot 5\,\text{л}} \approx 2{,}00.
    \end{align*}
}
\solutionspace{80pt}

\tasknumber{6}%
\task{%
    $2\,\text{моль}$ идеального одноатомного газа охладили на $30\,\text{К}$.
    Определите изменение внутренней энергии газа.
    Увеличилась она или уменьшилась?
    Универсальная газовая постоянная $R = 8{,}31\,\frac{\text{Дж}}{\text{моль}\cdot\text{К}}$.
}
\answer{%
    $
        \Delta U = \frac 32 \nu R \Delta T
            = - \frac 32 \cdot 2\,\text{моль} \cdot 8{,}31\,\frac{\text{Дж}}{\text{моль}\cdot\text{К}} \cdot 30\,\text{К}
            = -747\,\text{Дж}.
            \text{Уменьшилась.}
    $
}
\solutionspace{40pt}

\tasknumber{7}%
\task{%
    Газу сообщили некоторое количество теплоты,
    при этом половину его он потратил на совершение работы,
    одновременно увеличив свою внутреннюю энергию на $1500\,\text{Дж}$.
    Определите количество теплоты, сообщённое газу.
}
\answer{%
    \begin{align*}
    Q &= A' + \Delta U, A' = \frac 12 Q \implies Q \cdot \cbr{1 - \frac 12} = \Delta U \implies Q = \frac{\Delta U}{1 - \frac 12} = \frac{1500\,\text{Дж}}{1 - \frac 12} \approx 3000\,\text{Дж}.
    \\
    A' &= \frac 12 Q
        = \frac 12 \cdot \frac{\Delta U}{1 - \frac 12}
        = \frac{\Delta U}{2 - 1}
        = \frac{1500\,\text{Дж}}{2 - 1} \approx 1500\,\text{Дж}.
    \end{align*}
}
\solutionspace{60pt}

\tasknumber{8}%
\task{%
    В некотором процессе внешние силы совершили над газом работу $100\,\text{Дж}$,
    при этом его внутренняя энергия увеличилась на $150\,\text{Дж}$.
    Определите количество тепла, переданное при этом процессе газу.
    Явно пропишите, подводили газу тепло или же отводили.
}
\answer{%
    $
        Q = A_\text{газа} + \Delta U, A_\text{газа} = -A_\text{внешняя}
        \implies Q = A_\text{газа} + \Delta U = - 100\,\text{Дж} +  150\,\text{Дж} = 50\,\text{Дж}.
        \text{ Подводили.}
    $
}

\variantsplitter

\addpersonalvariant{Дмитрий Соколов}

\tasknumber{1}%
\task{%
    Напротив физических величин укажите их обозначения и единицы измерения в СИ, а в пункте «г)» запишите физический закон или формулу:
    \begin{enumerate}
        \item изменение внутренней энергии,
        \item работа газа,
        \item удельная теплоёмкость,
        \item внутренняя энергия идеального одноатомного газа.
    \end{enumerate}
}
\solutionspace{20pt}

\tasknumber{2}%
\task{%
    Определите объём идеального одноатомного газа,
    если его внутренняя энергия при давлении $4\,\text{атм}$ составляет $300\,\text{кДж}$.
    $p_{\text{aтм}} = 100\,\text{кПа}$.
}
\answer{%
    $U = \frac 32 \nu R T = \frac 32 PV \implies V = \frac 23 \cdot \frac UP= \frac 23 \cdot \frac{300\,\text{кДж}}{4\,\text{атм}} \approx 0{,}50\,\text{м}^{3}.$
}
\solutionspace{40pt}

\tasknumber{3}%
\task{%
    Газ расширился от $250\,\text{л}$ до $450\,\text{л}$.
    Давление газа при этом оставалось постоянным и равным $2{,}5\,\text{атм}$.
    Определите работу газа, ответ выразите в килоджоулях.
    $p_{\text{aтм}} = 100\,\text{кПа}$.
}
\answer{%
    $A = P\Delta V = P(V_2 - V_1) = 2{,}5\,\text{атм} \cdot \cbr{450\,\text{л} - 250\,\text{л}} = 50{,}0\,\text{кДж}.$
}
\solutionspace{40pt}

\tasknumber{4}%
\task{%
    $40\,\text{моль}$ идеального одноатомного газа в результате адиабатического процесса нагрелись на $15\,\text{К}$.
    Определите работу газа.
    Кто совершил положительную работу: газ или внешние силы?
    Универсальная газовая постоянная $R = 8{,}31\,\frac{\text{Дж}}{\text{моль}\cdot\text{К}}$.
}
\answer{%
    \begin{align*}
    Q &= 0, Q = \Delta U + A_\text{газа} \implies \\
    \implies A_\text{газа} &= - \Delta U = - \frac 32 \nu R \Delta T = - \frac 32 \cdot 40\,\text{моль} \cdot 8{,}31\,\frac{\text{Дж}}{\text{моль}\cdot\text{К}} \cdot 15\,\text{К}= -7{,}50\,\text{кДж}, \text{внешние силы.}
    \end{align*}
}
\solutionspace{40pt}

\tasknumber{5}%
\task{%
    Как изменилась внутренняя энергия одноатомного идеального газа при переходе из состояния 1 в состояние 2?
    $P_1 = 2\,\text{МПа}$, $V_1 = 3\,\text{л}$, $P_2 = 2{,}5\,\text{МПа}$, $V_2 = 2\,\text{л}$.
    Как изменилась при этом температура газа?
}
\answer{%
    \begin{align*}
    P_1V_1 &= \nu R T_1, P_2V_2 = \nu R T_2, \\
    \Delta U &= U_2-U_1 = \frac 32 \nu R T_2- \frac 32 \nu R T_1 = \frac 32 P_2 V_2 - \frac 32 P_1 V_1= \frac 32 \cdot \cbr{2{,}5\,\text{МПа} \cdot 2\,\text{л} - 2\,\text{МПа} \cdot 3\,\text{л}} = -1500\,\text{Дж}.
    \\
    \frac{T_2}{T_1} &= \frac{\frac{P_2V_2}{\nu R}}{\frac{P_1V_1}{\nu R}} = \frac{P_2V_2}{P_1V_1}= \frac{2{,}5\,\text{МПа} \cdot 2\,\text{л}}{2\,\text{МПа} \cdot 3\,\text{л}} \approx 0{,}83.
    \end{align*}
}
\solutionspace{80pt}

\tasknumber{6}%
\task{%
    $3\,\text{моль}$ идеального одноатомного газа нагрели на $10\,\text{К}$.
    Определите изменение внутренней энергии газа.
    Увеличилась она или уменьшилась?
    Универсальная газовая постоянная $R = 8{,}31\,\frac{\text{Дж}}{\text{моль}\cdot\text{К}}$.
}
\answer{%
    $
        \Delta U = \frac 32 \nu R \Delta T
            =  \frac 32 \cdot 3\,\text{моль} \cdot 8{,}31\,\frac{\text{Дж}}{\text{моль}\cdot\text{К}} \cdot 10\,\text{К}
            = 373\,\text{Дж}.
            \text{Увеличилась.}
    $
}
\solutionspace{40pt}

\tasknumber{7}%
\task{%
    Газу сообщили некоторое количество теплоты,
    при этом четверть его он потратил на совершение работы,
    одновременно увеличив свою внутреннюю энергию на $2400\,\text{Дж}$.
    Определите работу, совершённую газом.
}
\answer{%
    \begin{align*}
    Q &= A' + \Delta U, A' = \frac 14 Q \implies Q \cdot \cbr{1 - \frac 14} = \Delta U \implies Q = \frac{\Delta U}{1 - \frac 14} = \frac{2400\,\text{Дж}}{1 - \frac 14} \approx 3200\,\text{Дж}.
    \\
    A' &= \frac 14 Q
        = \frac 14 \cdot \frac{\Delta U}{1 - \frac 14}
        = \frac{\Delta U}{4 - 1}
        = \frac{2400\,\text{Дж}}{4 - 1} \approx 800\,\text{Дж}.
    \end{align*}
}
\solutionspace{60pt}

\tasknumber{8}%
\task{%
    В некотором процессе внешние силы совершили над газом работу $100\,\text{Дж}$,
    при этом его внутренняя энергия увеличилась на $350\,\text{Дж}$.
    Определите количество тепла, переданное при этом процессе газу.
    Явно пропишите, подводили газу тепло или же отводили.
}
\answer{%
    $
        Q = A_\text{газа} + \Delta U, A_\text{газа} = -A_\text{внешняя}
        \implies Q = A_\text{газа} + \Delta U = - 100\,\text{Дж} +  350\,\text{Дж} = 250\,\text{Дж}.
        \text{ Подводили.}
    $
}

\variantsplitter

\addpersonalvariant{Арсений Трофимов}

\tasknumber{1}%
\task{%
    Напротив физических величин укажите их обозначения и единицы измерения в СИ, а в пункте «г)» запишите физический закон или формулу:
    \begin{enumerate}
        \item изменение внутренней энергии,
        \item работа внешних сил,
        \item молярная теплоёмкость,
        \item внутренняя энергия идеального одноатомного газа.
    \end{enumerate}
}
\solutionspace{20pt}

\tasknumber{2}%
\task{%
    Определите объём идеального одноатомного газа,
    если его внутренняя энергия при давлении $6\,\text{атм}$ составляет $400\,\text{кДж}$.
    $p_{\text{aтм}} = 100\,\text{кПа}$.
}
\answer{%
    $U = \frac 32 \nu R T = \frac 32 PV \implies V = \frac 23 \cdot \frac UP= \frac 23 \cdot \frac{400\,\text{кДж}}{6\,\text{атм}} \approx 0{,}44\,\text{м}^{3}.$
}
\solutionspace{40pt}

\tasknumber{3}%
\task{%
    Газ расширился от $200\,\text{л}$ до $450\,\text{л}$.
    Давление газа при этом оставалось постоянным и равным $3{,}5\,\text{атм}$.
    Определите работу газа, ответ выразите в килоджоулях.
    $p_{\text{aтм}} = 100\,\text{кПа}$.
}
\answer{%
    $A = P\Delta V = P(V_2 - V_1) = 3{,}5\,\text{атм} \cdot \cbr{450\,\text{л} - 200\,\text{л}} = 87{,}5\,\text{кДж}.$
}
\solutionspace{40pt}

\tasknumber{4}%
\task{%
    $50\,\text{моль}$ идеального одноатомного газа в результате адиабатического процесса остыли на $80\,\text{К}$.
    Определите работу газа.
    Кто совершил положительную работу: газ или внешние силы?
    Универсальная газовая постоянная $R = 8{,}31\,\frac{\text{Дж}}{\text{моль}\cdot\text{К}}$.
}
\answer{%
    \begin{align*}
    Q &= 0, Q = \Delta U + A_\text{газа} \implies \\
    \implies A_\text{газа} &= - \Delta U = - \frac 32 \nu R \Delta T =  \frac 32 \cdot 50\,\text{моль} \cdot 8{,}31\,\frac{\text{Дж}}{\text{моль}\cdot\text{К}} \cdot 80\,\text{К}= 49{,}9\,\text{кДж}, \text{газ.}
    \end{align*}
}
\solutionspace{40pt}

\tasknumber{5}%
\task{%
    Как изменилась внутренняя энергия одноатомного идеального газа при переходе из состояния 1 в состояние 2?
    $P_1 = 2\,\text{МПа}$, $V_1 = 3\,\text{л}$, $P_2 = 3{,}5\,\text{МПа}$, $V_2 = 6\,\text{л}$.
    Как изменилась при этом температура газа?
}
\answer{%
    \begin{align*}
    P_1V_1 &= \nu R T_1, P_2V_2 = \nu R T_2, \\
    \Delta U &= U_2-U_1 = \frac 32 \nu R T_2- \frac 32 \nu R T_1 = \frac 32 P_2 V_2 - \frac 32 P_1 V_1= \frac 32 \cdot \cbr{3{,}5\,\text{МПа} \cdot 6\,\text{л} - 2\,\text{МПа} \cdot 3\,\text{л}} = 22500\,\text{Дж}.
    \\
    \frac{T_2}{T_1} &= \frac{\frac{P_2V_2}{\nu R}}{\frac{P_1V_1}{\nu R}} = \frac{P_2V_2}{P_1V_1}= \frac{3{,}5\,\text{МПа} \cdot 6\,\text{л}}{2\,\text{МПа} \cdot 3\,\text{л}} \approx 3{,}50.
    \end{align*}
}
\solutionspace{80pt}

\tasknumber{6}%
\task{%
    $4\,\text{моль}$ идеального одноатомного газа охладили на $20\,\text{К}$.
    Определите изменение внутренней энергии газа.
    Увеличилась она или уменьшилась?
    Универсальная газовая постоянная $R = 8{,}31\,\frac{\text{Дж}}{\text{моль}\cdot\text{К}}$.
}
\answer{%
    $
        \Delta U = \frac 32 \nu R \Delta T
            = - \frac 32 \cdot 4\,\text{моль} \cdot 8{,}31\,\frac{\text{Дж}}{\text{моль}\cdot\text{К}} \cdot 20\,\text{К}
            = -997\,\text{Дж}.
            \text{Уменьшилась.}
    $
}
\solutionspace{40pt}

\tasknumber{7}%
\task{%
    Газу сообщили некоторое количество теплоты,
    при этом половину его он потратил на совершение работы,
    одновременно увеличив свою внутреннюю энергию на $1200\,\text{Дж}$.
    Определите работу, совершённую газом.
}
\answer{%
    \begin{align*}
    Q &= A' + \Delta U, A' = \frac 12 Q \implies Q \cdot \cbr{1 - \frac 12} = \Delta U \implies Q = \frac{\Delta U}{1 - \frac 12} = \frac{1200\,\text{Дж}}{1 - \frac 12} \approx 2400\,\text{Дж}.
    \\
    A' &= \frac 12 Q
        = \frac 12 \cdot \frac{\Delta U}{1 - \frac 12}
        = \frac{\Delta U}{2 - 1}
        = \frac{1200\,\text{Дж}}{2 - 1} \approx 1200\,\text{Дж}.
    \end{align*}
}
\solutionspace{60pt}

\tasknumber{8}%
\task{%
    В некотором процессе газ совершил работу $300\,\text{Дж}$,
    при этом его внутренняя энергия уменьшилась на $150\,\text{Дж}$.
    Определите количество тепла, переданное при этом процессе газу.
    Явно пропишите, подводили газу тепло или же отводили.
}
\answer{%
    $
        Q = A_\text{газа} + \Delta U, A_\text{газа} = -A_\text{внешняя}
        \implies Q = A_\text{газа} + \Delta U =  300\,\text{Дж} - 150\,\text{Дж} = 150\,\text{Дж}.
        \text{ Подводили.}
    $
}
% autogenerated
