\setdate{2~апреля~2021}
\setclass{10«АБ»}

\addpersonalvariant{Михаил Бурмистров}

\tasknumber{1}%
\task{%
    Позитрон $e^+$ вылетает из точки, потенциал которой $\varphi = 600\,\text{В}$,
    со скоростью $v = 6000000\,\frac{\text{м}}{\text{с}}$ параллельно линиям напряжённости однородного электрического поля.
    % Будет поле его ускорять или тормозить?
    В некоторой точке частица остановилась.
    Каков потенциал этой точки?
    Вдоль и против поля влетела изначально частица?
}
\answer{%
    \begin{align*}
    A_\text{внешних сил} &= \Delta E_\text{кин.} \implies A_\text{эл.
    поля} = 0 - \frac{mv^2}2.
    \\
    A_\text{эл.
    поля} &= q(\varphi_1 - \varphi_2) \implies\varphi_2 = \varphi_1 - \frac{A_\text{эл.
    поля}}q = \varphi_1 - \frac{- \frac{mv^2}2}q = \varphi_1 + \frac{mv^2}{2q} =  \\
    &= 600\,\text{В} + \frac{9{,}1 \cdot 10^{-31}\,\text{кг} \cdot \sqr{ 6000000\,\frac{\text{м}}{\text{с}} }}{2  \cdot 1{,}6 \cdot 10^{-19}\,\text{Кл}} \approx 702{,}4\,\text{В}.
    \end{align*}
}
\solutionspace{120pt}

\tasknumber{2}%
\task{%
    Три одинаковых положительных точечных заряда по $q$ каждый находятся
    на одной прямой так, что расстояние между каждыми двумя соседними равно $2d$.
    Какую минимальную работу необходимо совершить, чтобы перевести эти заряды в положение,
    при котором они образуют равносторонний треугольник со стороной $d$? Сделайте рисунки и получите ответ (формулой).
}
\solutionspace{120pt}

\tasknumber{3}%
\task{%
    На рисунке показано расположение трёх металлических пластин и указаны их потенциалы.
    Размеры пластин кораздо больше расстояния между ними.
    Отмечены также ось и начало координат.
    Дорисуйте на рисунке электрическое поле и постройте графики зависимости от координаты $x$:
    \begin{enumerate}
        \item проекции напряжённости электрического поля,
        \item потенциала электрического поля.
    \end{enumerate}
    \begin{tikzpicture}
        \draw[-{Latex}] (0, 0) -- (0, 3.5) node[below right] {$x$};
        \draw[thick]
            (-0.05, 0.5) -- (0.05, 0.5)     (0, 0.5) node[left] {$-2\,\text{см}$}     (0.5, 0.5) -- (4, 0.5) node[right] {$-30\,\text{В}$}
            (-0.05, 1.5) -- (0.05, 1.5)     (0, 1.5) node[left] {$0$}         (0.5, 1.5) -- (4, 1.5) node[right] {$0\,\text{В}$}
            (-0.05, 2.5) -- (0.05, 2.5)     (0, 2.5) node[left] {$2\,\text{см}$}    (0.5, 2.5) -- (4, 2.5) node[right] {$120\,\text{В}$};
    \end{tikzpicture}
}
\solutionspace{90pt}

\tasknumber{4}%
\task{%
    \begin{enumerate}
        \item Запишите закон Кулона (в диэлектрике).
        \item Из теоремы Гаусса выведите (нужен рисунок, применение и результат) формулу для напряженности электростатического поля снаружи равномерно заряженной сферы.
        \item Зарисуйте электрическое поле точечного отрицательного электрического заряда.
        \item Запишите формулу для вычисления потенциала электрического поля точечного заряда в диэлектрике.
        \item Запишите принцип суперпозиции (правило сложения) напряжённостей.
    \end{enumerate}
}

\variantsplitter

\addpersonalvariant{Ирина Ан}

\tasknumber{1}%
\task{%
    Позитрон $e^+$ вылетает из точки, потенциал которой $\varphi = 600\,\text{В}$,
    со скоростью $v = 3000000\,\frac{\text{м}}{\text{с}}$ параллельно линиям напряжённости однородного электрического поля.
    % Будет поле его ускорять или тормозить?
    В некоторой точке частица остановилась.
    Каков потенциал этой точки?
    Вдоль и против поля влетела изначально частица?
}
\answer{%
    \begin{align*}
    A_\text{внешних сил} &= \Delta E_\text{кин.} \implies A_\text{эл.
    поля} = 0 - \frac{mv^2}2.
    \\
    A_\text{эл.
    поля} &= q(\varphi_1 - \varphi_2) \implies\varphi_2 = \varphi_1 - \frac{A_\text{эл.
    поля}}q = \varphi_1 - \frac{- \frac{mv^2}2}q = \varphi_1 + \frac{mv^2}{2q} =  \\
    &= 600\,\text{В} + \frac{9{,}1 \cdot 10^{-31}\,\text{кг} \cdot \sqr{ 3000000\,\frac{\text{м}}{\text{с}} }}{2  \cdot 1{,}6 \cdot 10^{-19}\,\text{Кл}} \approx 625{,}6\,\text{В}.
    \end{align*}
}
\solutionspace{120pt}

\tasknumber{2}%
\task{%
    Три одинаковых положительных точечных заряда по $q$ каждый находятся
    на одной прямой так, что расстояние между каждыми двумя соседними равно $2d$.
    Какую минимальную работу необходимо совершить, чтобы перевести эти заряды в положение,
    при котором они образуют прямоугольный равнобедренный треугольник с гипотенузой $d$? Сделайте рисунки и получите ответ (формулой).
}
\solutionspace{120pt}

\tasknumber{3}%
\task{%
    На рисунке показано расположение трёх металлических пластин и указаны их потенциалы.
    Размеры пластин кораздо больше расстояния между ними.
    Отмечены также ось и начало координат.
    Дорисуйте на рисунке электрическое поле и постройте графики зависимости от координаты $x$:
    \begin{enumerate}
        \item проекции напряжённости электрического поля,
        \item потенциала электрического поля.
    \end{enumerate}
    \begin{tikzpicture}
        \draw[-{Latex}] (0, 0) -- (0, 3.5) node[below right] {$x$};
        \draw[thick]
            (-0.05, 0.5) -- (0.05, 0.5)     (0, 0.5) node[left] {$-2\,\text{см}$}     (0.5, 0.5) -- (4, 0.5) node[right] {$30\,\text{В}$}
            (-0.05, 1.5) -- (0.05, 1.5)     (0, 1.5) node[left] {$0$}         (0.5, 1.5) -- (4, 1.5) node[right] {$0\,\text{В}$}
            (-0.05, 2.5) -- (0.05, 2.5)     (0, 2.5) node[left] {$2\,\text{см}$}    (0.5, 2.5) -- (4, 2.5) node[right] {$-60\,\text{В}$};
    \end{tikzpicture}
}
\solutionspace{90pt}

\tasknumber{4}%
\task{%
    \begin{enumerate}
        \item Запишите закон Кулона (в диэлектрике).
        \item Из теоремы Гаусса выведите (нужен рисунок, применение и результат) формулу для напряженности электростатического поля около равномерно заряженной бесконечной плоскости.
        \item Зарисуйте электрическое поле точечного отрицательного электрического заряда.
        \item Запишите формулу для вычисления напряжённости электрического поля точечного заряда в диэлектрике.
        \item Запишите принцип суперпозиции (правило сложения) потенциалов.
    \end{enumerate}
}

\variantsplitter

\addpersonalvariant{Софья Андрианова}

\tasknumber{1}%
\task{%
    Позитрон $e^+$ вылетает из точки, потенциал которой $\varphi = 1000\,\text{В}$,
    со скоростью $v = 6000000\,\frac{\text{м}}{\text{с}}$ параллельно линиям напряжённости однородного электрического поля.
    % Будет поле его ускорять или тормозить?
    В некоторой точке частица остановилась.
    Каков потенциал этой точки?
    Вдоль и против поля влетела изначально частица?
}
\answer{%
    \begin{align*}
    A_\text{внешних сил} &= \Delta E_\text{кин.} \implies A_\text{эл.
    поля} = 0 - \frac{mv^2}2.
    \\
    A_\text{эл.
    поля} &= q(\varphi_1 - \varphi_2) \implies\varphi_2 = \varphi_1 - \frac{A_\text{эл.
    поля}}q = \varphi_1 - \frac{- \frac{mv^2}2}q = \varphi_1 + \frac{mv^2}{2q} =  \\
    &= 1000\,\text{В} + \frac{9{,}1 \cdot 10^{-31}\,\text{кг} \cdot \sqr{ 6000000\,\frac{\text{м}}{\text{с}} }}{2  \cdot 1{,}6 \cdot 10^{-19}\,\text{Кл}} \approx 1102{,}4\,\text{В}.
    \end{align*}
}
\solutionspace{120pt}

\tasknumber{2}%
\task{%
    Три одинаковых положительных точечных заряда по $q$ каждый находятся
    на одной прямой так, что расстояние между каждыми двумя соседними равно $2r$.
    Какую минимальную работу необходимо совершить, чтобы перевести эти заряды в положение,
    при котором они образуют прямоугольный равнобедренный треугольник с катетом $r$? Сделайте рисунки и получите ответ (формулой).
}
\solutionspace{120pt}

\tasknumber{3}%
\task{%
    На рисунке показано расположение трёх металлических пластин и указаны их потенциалы.
    Размеры пластин кораздо больше расстояния между ними.
    Отмечены также ось и начало координат.
    Дорисуйте на рисунке электрическое поле и постройте графики зависимости от координаты $x$:
    \begin{enumerate}
        \item проекции напряжённости электрического поля,
        \item потенциала электрического поля.
    \end{enumerate}
    \begin{tikzpicture}
        \draw[-{Latex}] (0, 0) -- (0, 3.5) node[below right] {$x$};
        \draw[thick]
            (-0.05, 0.5) -- (0.05, 0.5)     (0, 0.5) node[left] {$-3\,\text{см}$}     (0.5, 0.5) -- (4, 0.5) node[right] {$30\,\text{В}$}
            (-0.05, 1.5) -- (0.05, 1.5)     (0, 1.5) node[left] {$0$}         (0.5, 1.5) -- (4, 1.5) node[right] {$0\,\text{В}$}
            (-0.05, 2.5) -- (0.05, 2.5)     (0, 2.5) node[left] {$3\,\text{см}$}    (0.5, 2.5) -- (4, 2.5) node[right] {$-120\,\text{В}$};
    \end{tikzpicture}
}
\solutionspace{90pt}

\tasknumber{4}%
\task{%
    \begin{enumerate}
        \item Запишите закон сохранения электрического заряда.
        \item Из теоремы Гаусса выведите (нужен рисунок, применение и результат) формулу для напряженности электростатического поля около равномерно заряженной бесконечной плоскости.
        \item Зарисуйте электрическое поле точечного положительного электрического заряда.
        \item Запишите формулу для вычисления напряжённости электрического поля точечного заряда в диэлектрике.
        \item Запишите принцип суперпозиции (правило сложения) напряжённостей.
    \end{enumerate}
}

\variantsplitter

\addpersonalvariant{Владимир Артемчук}

\tasknumber{1}%
\task{%
    Электрон $e^-$ вылетает из точки, потенциал которой $\varphi = 800\,\text{В}$,
    со скоростью $v = 3000000\,\frac{\text{м}}{\text{с}}$ параллельно линиям напряжённости однородного электрического поля.
    % Будет поле его ускорять или тормозить?
    В некоторой точке частица остановилась.
    Каков потенциал этой точки?
    Вдоль и против поля влетела изначально частица?
}
\answer{%
    \begin{align*}
    A_\text{внешних сил} &= \Delta E_\text{кин.} \implies A_\text{эл.
    поля} = 0 - \frac{mv^2}2.
    \\
    A_\text{эл.
    поля} &= q(\varphi_1 - \varphi_2) \implies\varphi_2 = \varphi_1 - \frac{A_\text{эл.
    поля}}q = \varphi_1 - \frac{- \frac{mv^2}2}q = \varphi_1 + \frac{mv^2}{2q} =  \\
    &= 800\,\text{В} + \frac{9{,}1 \cdot 10^{-31}\,\text{кг} \cdot \sqr{ 3000000\,\frac{\text{м}}{\text{с}} }}{2  * (-1)  \cdot 1{,}6 \cdot 10^{-19}\,\text{Кл}} \approx 774{,}4\,\text{В}.
    \end{align*}
}
\solutionspace{120pt}

\tasknumber{2}%
\task{%
    Три одинаковых положительных точечных заряда по $q$ каждый находятся
    на одной прямой так, что расстояние между каждыми двумя соседними равно $3r$.
    Какую минимальную работу необходимо совершить, чтобы перевести эти заряды в положение,
    при котором они образуют равносторонний треугольник со стороной $r$? Сделайте рисунки и получите ответ (формулой).
}
\solutionspace{120pt}

\tasknumber{3}%
\task{%
    На рисунке показано расположение трёх металлических пластин и указаны их потенциалы.
    Размеры пластин кораздо больше расстояния между ними.
    Отмечены также ось и начало координат.
    Дорисуйте на рисунке электрическое поле и постройте графики зависимости от координаты $x$:
    \begin{enumerate}
        \item проекции напряжённости электрического поля,
        \item потенциала электрического поля.
    \end{enumerate}
    \begin{tikzpicture}
        \draw[-{Latex}] (0, 0) -- (0, 3.5) node[below right] {$x$};
        \draw[thick]
            (-0.05, 0.5) -- (0.05, 0.5)     (0, 0.5) node[left] {$-2\,\text{см}$}     (0.5, 0.5) -- (4, 0.5) node[right] {$-90\,\text{В}$}
            (-0.05, 1.5) -- (0.05, 1.5)     (0, 1.5) node[left] {$0$}         (0.5, 1.5) -- (4, 1.5) node[right] {$0\,\text{В}$}
            (-0.05, 2.5) -- (0.05, 2.5)     (0, 2.5) node[left] {$2\,\text{см}$}    (0.5, 2.5) -- (4, 2.5) node[right] {$-60\,\text{В}$};
    \end{tikzpicture}
}
\solutionspace{90pt}

\tasknumber{4}%
\task{%
    \begin{enumerate}
        \item Запишите закон сохранения электрического заряда.
        \item Из теоремы Гаусса выведите (нужен рисунок, применение и результат) формулу для напряженности электростатического поля около равномерно заряженной бесконечной плоскости.
        \item Зарисуйте электрическое поле точечного отрицательного электрического заряда.
        \item Запишите формулу для вычисления напряжённости электрического поля точечного заряда в диэлектрике.
        \item Запишите принцип суперпозиции (правило сложения) потенциалов.
    \end{enumerate}
}

\variantsplitter

\addpersonalvariant{Софья Белянкина}

\tasknumber{1}%
\task{%
    Позитрон $e^+$ вылетает из точки, потенциал которой $\varphi = 200\,\text{В}$,
    со скоростью $v = 6000000\,\frac{\text{м}}{\text{с}}$ параллельно линиям напряжённости однородного электрического поля.
    % Будет поле его ускорять или тормозить?
    В некоторой точке частица остановилась.
    Каков потенциал этой точки?
    Вдоль и против поля влетела изначально частица?
}
\answer{%
    \begin{align*}
    A_\text{внешних сил} &= \Delta E_\text{кин.} \implies A_\text{эл.
    поля} = 0 - \frac{mv^2}2.
    \\
    A_\text{эл.
    поля} &= q(\varphi_1 - \varphi_2) \implies\varphi_2 = \varphi_1 - \frac{A_\text{эл.
    поля}}q = \varphi_1 - \frac{- \frac{mv^2}2}q = \varphi_1 + \frac{mv^2}{2q} =  \\
    &= 200\,\text{В} + \frac{9{,}1 \cdot 10^{-31}\,\text{кг} \cdot \sqr{ 6000000\,\frac{\text{м}}{\text{с}} }}{2  \cdot 1{,}6 \cdot 10^{-19}\,\text{Кл}} \approx 302{,}4\,\text{В}.
    \end{align*}
}
\solutionspace{120pt}

\tasknumber{2}%
\task{%
    Три одинаковых положительных точечных заряда по $Q$ каждый находятся
    на одной прямой так, что расстояние между каждыми двумя соседними равно $3d$.
    Какую минимальную работу необходимо совершить, чтобы перевести эти заряды в положение,
    при котором они образуют прямоугольный равнобедренный треугольник с гипотенузой $d$? Сделайте рисунки и получите ответ (формулой).
}
\solutionspace{120pt}

\tasknumber{3}%
\task{%
    На рисунке показано расположение трёх металлических пластин и указаны их потенциалы.
    Размеры пластин кораздо больше расстояния между ними.
    Отмечены также ось и начало координат.
    Дорисуйте на рисунке электрическое поле и постройте графики зависимости от координаты $x$:
    \begin{enumerate}
        \item проекции напряжённости электрического поля,
        \item потенциала электрического поля.
    \end{enumerate}
    \begin{tikzpicture}
        \draw[-{Latex}] (0, 0) -- (0, 3.5) node[below right] {$x$};
        \draw[thick]
            (-0.05, 0.5) -- (0.05, 0.5)     (0, 0.5) node[left] {$-2\,\text{см}$}     (0.5, 0.5) -- (4, 0.5) node[right] {$150\,\text{В}$}
            (-0.05, 1.5) -- (0.05, 1.5)     (0, 1.5) node[left] {$0$}         (0.5, 1.5) -- (4, 1.5) node[right] {$0\,\text{В}$}
            (-0.05, 2.5) -- (0.05, 2.5)     (0, 2.5) node[left] {$2\,\text{см}$}    (0.5, 2.5) -- (4, 2.5) node[right] {$-60\,\text{В}$};
    \end{tikzpicture}
}
\solutionspace{90pt}

\tasknumber{4}%
\task{%
    \begin{enumerate}
        \item Запишите закон Кулона (в диэлектрике).
        \item Из теоремы Гаусса выведите (нужен рисунок, применение и результат) формулу для напряженности электростатического поля снаружи равномерно заряженной сферы.
        \item Зарисуйте электрическое поле точечного положительного электрического заряда.
        \item Запишите формулу для вычисления потенциала электрического поля точечного заряда в диэлектрике.
        \item Запишите принцип суперпозиции (правило сложения) потенциалов.
    \end{enumerate}
}

\variantsplitter

\addpersonalvariant{Варвара Егиазарян}

\tasknumber{1}%
\task{%
    Электрон $e^-$ вылетает из точки, потенциал которой $\varphi = 600\,\text{В}$,
    со скоростью $v = 3000000\,\frac{\text{м}}{\text{с}}$ параллельно линиям напряжённости однородного электрического поля.
    % Будет поле его ускорять или тормозить?
    В некоторой точке частица остановилась.
    Каков потенциал этой точки?
    Вдоль и против поля влетела изначально частица?
}
\answer{%
    \begin{align*}
    A_\text{внешних сил} &= \Delta E_\text{кин.} \implies A_\text{эл.
    поля} = 0 - \frac{mv^2}2.
    \\
    A_\text{эл.
    поля} &= q(\varphi_1 - \varphi_2) \implies\varphi_2 = \varphi_1 - \frac{A_\text{эл.
    поля}}q = \varphi_1 - \frac{- \frac{mv^2}2}q = \varphi_1 + \frac{mv^2}{2q} =  \\
    &= 600\,\text{В} + \frac{9{,}1 \cdot 10^{-31}\,\text{кг} \cdot \sqr{ 3000000\,\frac{\text{м}}{\text{с}} }}{2  * (-1)  \cdot 1{,}6 \cdot 10^{-19}\,\text{Кл}} \approx 574{,}4\,\text{В}.
    \end{align*}
}
\solutionspace{120pt}

\tasknumber{2}%
\task{%
    Три одинаковых положительных точечных заряда по $Q$ каждый находятся
    на одной прямой так, что расстояние между каждыми двумя соседними равно $2l$.
    Какую минимальную работу необходимо совершить, чтобы перевести эти заряды в положение,
    при котором они образуют прямоугольный равнобедренный треугольник с гипотенузой $l$? Сделайте рисунки и получите ответ (формулой).
}
\solutionspace{120pt}

\tasknumber{3}%
\task{%
    На рисунке показано расположение трёх металлических пластин и указаны их потенциалы.
    Размеры пластин кораздо больше расстояния между ними.
    Отмечены также ось и начало координат.
    Дорисуйте на рисунке электрическое поле и постройте графики зависимости от координаты $x$:
    \begin{enumerate}
        \item проекции напряжённости электрического поля,
        \item потенциала электрического поля.
    \end{enumerate}
    \begin{tikzpicture}
        \draw[-{Latex}] (0, 0) -- (0, 3.5) node[below right] {$x$};
        \draw[thick]
            (-0.05, 0.5) -- (0.05, 0.5)     (0, 0.5) node[left] {$-3\,\text{см}$}     (0.5, 0.5) -- (4, 0.5) node[right] {$-30\,\text{В}$}
            (-0.05, 1.5) -- (0.05, 1.5)     (0, 1.5) node[left] {$0$}         (0.5, 1.5) -- (4, 1.5) node[right] {$0\,\text{В}$}
            (-0.05, 2.5) -- (0.05, 2.5)     (0, 2.5) node[left] {$3\,\text{см}$}    (0.5, 2.5) -- (4, 2.5) node[right] {$120\,\text{В}$};
    \end{tikzpicture}
}
\solutionspace{90pt}

\tasknumber{4}%
\task{%
    \begin{enumerate}
        \item Запишите закон сохранения электрического заряда.
        \item Из теоремы Гаусса выведите (нужен рисунок, применение и результат) формулу для напряженности электростатического поля снаружи равномерно заряженной сферы.
        \item Зарисуйте электрическое поле точечного положительного электрического заряда.
        \item Запишите формулу для вычисления напряжённости электрического поля точечного заряда в диэлектрике.
        \item Запишите принцип суперпозиции (правило сложения) потенциалов.
    \end{enumerate}
}

\variantsplitter

\addpersonalvariant{Владислав Емелин}

\tasknumber{1}%
\task{%
    Позитрон $e^+$ вылетает из точки, потенциал которой $\varphi = 200\,\text{В}$,
    со скоростью $v = 3000000\,\frac{\text{м}}{\text{с}}$ параллельно линиям напряжённости однородного электрического поля.
    % Будет поле его ускорять или тормозить?
    В некоторой точке частица остановилась.
    Каков потенциал этой точки?
    Вдоль и против поля влетела изначально частица?
}
\answer{%
    \begin{align*}
    A_\text{внешних сил} &= \Delta E_\text{кин.} \implies A_\text{эл.
    поля} = 0 - \frac{mv^2}2.
    \\
    A_\text{эл.
    поля} &= q(\varphi_1 - \varphi_2) \implies\varphi_2 = \varphi_1 - \frac{A_\text{эл.
    поля}}q = \varphi_1 - \frac{- \frac{mv^2}2}q = \varphi_1 + \frac{mv^2}{2q} =  \\
    &= 200\,\text{В} + \frac{9{,}1 \cdot 10^{-31}\,\text{кг} \cdot \sqr{ 3000000\,\frac{\text{м}}{\text{с}} }}{2  \cdot 1{,}6 \cdot 10^{-19}\,\text{Кл}} \approx 225{,}6\,\text{В}.
    \end{align*}
}
\solutionspace{120pt}

\tasknumber{2}%
\task{%
    Три одинаковых положительных точечных заряда по $Q$ каждый находятся
    на одной прямой так, что расстояние между каждыми двумя соседними равно $3r$.
    Какую минимальную работу необходимо совершить, чтобы перевести эти заряды в положение,
    при котором они образуют равносторонний треугольник со стороной $r$? Сделайте рисунки и получите ответ (формулой).
}
\solutionspace{120pt}

\tasknumber{3}%
\task{%
    На рисунке показано расположение трёх металлических пластин и указаны их потенциалы.
    Размеры пластин кораздо больше расстояния между ними.
    Отмечены также ось и начало координат.
    Дорисуйте на рисунке электрическое поле и постройте графики зависимости от координаты $x$:
    \begin{enumerate}
        \item проекции напряжённости электрического поля,
        \item потенциала электрического поля.
    \end{enumerate}
    \begin{tikzpicture}
        \draw[-{Latex}] (0, 0) -- (0, 3.5) node[below right] {$x$};
        \draw[thick]
            (-0.05, 0.5) -- (0.05, 0.5)     (0, 0.5) node[left] {$-2\,\text{см}$}     (0.5, 0.5) -- (4, 0.5) node[right] {$30\,\text{В}$}
            (-0.05, 1.5) -- (0.05, 1.5)     (0, 1.5) node[left] {$0$}         (0.5, 1.5) -- (4, 1.5) node[right] {$0\,\text{В}$}
            (-0.05, 2.5) -- (0.05, 2.5)     (0, 2.5) node[left] {$2\,\text{см}$}    (0.5, 2.5) -- (4, 2.5) node[right] {$120\,\text{В}$};
    \end{tikzpicture}
}
\solutionspace{90pt}

\tasknumber{4}%
\task{%
    \begin{enumerate}
        \item Запишите закон сохранения электрического заряда.
        \item Из теоремы Гаусса выведите (нужен рисунок, применение и результат) формулу для напряженности электростатического поля снаружи равномерно заряженной сферы.
        \item Зарисуйте электрическое поле точечного отрицательного электрического заряда.
        \item Запишите формулу для вычисления потенциала электрического поля точечного заряда в диэлектрике.
        \item Запишите принцип суперпозиции (правило сложения) напряжённостей.
    \end{enumerate}
}

\variantsplitter

\addpersonalvariant{Артём Жичин}

\tasknumber{1}%
\task{%
    Электрон $e^-$ вылетает из точки, потенциал которой $\varphi = 200\,\text{В}$,
    со скоростью $v = 10000000\,\frac{\text{м}}{\text{с}}$ параллельно линиям напряжённости однородного электрического поля.
    % Будет поле его ускорять или тормозить?
    В некоторой точке частица остановилась.
    Каков потенциал этой точки?
    Вдоль и против поля влетела изначально частица?
}
\answer{%
    \begin{align*}
    A_\text{внешних сил} &= \Delta E_\text{кин.} \implies A_\text{эл.
    поля} = 0 - \frac{mv^2}2.
    \\
    A_\text{эл.
    поля} &= q(\varphi_1 - \varphi_2) \implies\varphi_2 = \varphi_1 - \frac{A_\text{эл.
    поля}}q = \varphi_1 - \frac{- \frac{mv^2}2}q = \varphi_1 + \frac{mv^2}{2q} =  \\
    &= 200\,\text{В} + \frac{9{,}1 \cdot 10^{-31}\,\text{кг} \cdot \sqr{ 10000000\,\frac{\text{м}}{\text{с}} }}{2  * (-1)  \cdot 1{,}6 \cdot 10^{-19}\,\text{Кл}} \approx -84{,}40\,\text{В}.
    \end{align*}
}
\solutionspace{120pt}

\tasknumber{2}%
\task{%
    Три одинаковых положительных точечных заряда по $q$ каждый находятся
    на одной прямой так, что расстояние между каждыми двумя соседними равно $2l$.
    Какую минимальную работу необходимо совершить, чтобы перевести эти заряды в положение,
    при котором они образуют прямоугольный равнобедренный треугольник с катетом $l$? Сделайте рисунки и получите ответ (формулой).
}
\solutionspace{120pt}

\tasknumber{3}%
\task{%
    На рисунке показано расположение трёх металлических пластин и указаны их потенциалы.
    Размеры пластин кораздо больше расстояния между ними.
    Отмечены также ось и начало координат.
    Дорисуйте на рисунке электрическое поле и постройте графики зависимости от координаты $x$:
    \begin{enumerate}
        \item проекции напряжённости электрического поля,
        \item потенциала электрического поля.
    \end{enumerate}
    \begin{tikzpicture}
        \draw[-{Latex}] (0, 0) -- (0, 3.5) node[below right] {$x$};
        \draw[thick]
            (-0.05, 0.5) -- (0.05, 0.5)     (0, 0.5) node[left] {$-2\,\text{см}$}     (0.5, 0.5) -- (4, 0.5) node[right] {$-90\,\text{В}$}
            (-0.05, 1.5) -- (0.05, 1.5)     (0, 1.5) node[left] {$0$}         (0.5, 1.5) -- (4, 1.5) node[right] {$0\,\text{В}$}
            (-0.05, 2.5) -- (0.05, 2.5)     (0, 2.5) node[left] {$2\,\text{см}$}    (0.5, 2.5) -- (4, 2.5) node[right] {$-60\,\text{В}$};
    \end{tikzpicture}
}
\solutionspace{90pt}

\tasknumber{4}%
\task{%
    \begin{enumerate}
        \item Запишите теорему Гаусса.
        \item Из теоремы Гаусса выведите (нужен рисунок, применение и результат) формулу для напряженности электростатического поля внутри равномерно заряженной сферы.
        \item Зарисуйте электрическое поле точечного положительного электрического заряда.
        \item Запишите формулу для вычисления потенциала электрического поля точечного заряда в диэлектрике.
        \item Запишите принцип суперпозиции (правило сложения) напряжённостей.
    \end{enumerate}
}

\variantsplitter

\addpersonalvariant{Дарья Кошман}

\tasknumber{1}%
\task{%
    Позитрон $e^+$ вылетает из точки, потенциал которой $\varphi = 400\,\text{В}$,
    со скоростью $v = 6000000\,\frac{\text{м}}{\text{с}}$ параллельно линиям напряжённости однородного электрического поля.
    % Будет поле его ускорять или тормозить?
    В некоторой точке частица остановилась.
    Каков потенциал этой точки?
    Вдоль и против поля влетела изначально частица?
}
\answer{%
    \begin{align*}
    A_\text{внешних сил} &= \Delta E_\text{кин.} \implies A_\text{эл.
    поля} = 0 - \frac{mv^2}2.
    \\
    A_\text{эл.
    поля} &= q(\varphi_1 - \varphi_2) \implies\varphi_2 = \varphi_1 - \frac{A_\text{эл.
    поля}}q = \varphi_1 - \frac{- \frac{mv^2}2}q = \varphi_1 + \frac{mv^2}{2q} =  \\
    &= 400\,\text{В} + \frac{9{,}1 \cdot 10^{-31}\,\text{кг} \cdot \sqr{ 6000000\,\frac{\text{м}}{\text{с}} }}{2  \cdot 1{,}6 \cdot 10^{-19}\,\text{Кл}} \approx 502{,}4\,\text{В}.
    \end{align*}
}
\solutionspace{120pt}

\tasknumber{2}%
\task{%
    Три одинаковых положительных точечных заряда по $q$ каждый находятся
    на одной прямой так, что расстояние между каждыми двумя соседними равно $3l$.
    Какую минимальную работу необходимо совершить, чтобы перевести эти заряды в положение,
    при котором они образуют равносторонний треугольник со стороной $l$? Сделайте рисунки и получите ответ (формулой).
}
\solutionspace{120pt}

\tasknumber{3}%
\task{%
    На рисунке показано расположение трёх металлических пластин и указаны их потенциалы.
    Размеры пластин кораздо больше расстояния между ними.
    Отмечены также ось и начало координат.
    Дорисуйте на рисунке электрическое поле и постройте графики зависимости от координаты $x$:
    \begin{enumerate}
        \item проекции напряжённости электрического поля,
        \item потенциала электрического поля.
    \end{enumerate}
    \begin{tikzpicture}
        \draw[-{Latex}] (0, 0) -- (0, 3.5) node[below right] {$x$};
        \draw[thick]
            (-0.05, 0.5) -- (0.05, 0.5)     (0, 0.5) node[left] {$-2\,\text{см}$}     (0.5, 0.5) -- (4, 0.5) node[right] {$90\,\text{В}$}
            (-0.05, 1.5) -- (0.05, 1.5)     (0, 1.5) node[left] {$0$}         (0.5, 1.5) -- (4, 1.5) node[right] {$0\,\text{В}$}
            (-0.05, 2.5) -- (0.05, 2.5)     (0, 2.5) node[left] {$2\,\text{см}$}    (0.5, 2.5) -- (4, 2.5) node[right] {$60\,\text{В}$};
    \end{tikzpicture}
}
\solutionspace{90pt}

\tasknumber{4}%
\task{%
    \begin{enumerate}
        \item Запишите теорему Гаусса.
        \item Из теоремы Гаусса выведите (нужен рисунок, применение и результат) формулу для напряженности электростатического поля около равномерно заряженной бесконечной плоскости.
        \item Зарисуйте электрическое поле точечного отрицательного электрического заряда.
        \item Запишите формулу для вычисления потенциала электрического поля точечного заряда в диэлектрике.
        \item Запишите принцип суперпозиции (правило сложения) напряжённостей.
    \end{enumerate}
}

\variantsplitter

\addpersonalvariant{Анна Кузьмичёва}

\tasknumber{1}%
\task{%
    Позитрон $e^+$ вылетает из точки, потенциал которой $\varphi = 800\,\text{В}$,
    со скоростью $v = 3000000\,\frac{\text{м}}{\text{с}}$ параллельно линиям напряжённости однородного электрического поля.
    % Будет поле его ускорять или тормозить?
    В некоторой точке частица остановилась.
    Каков потенциал этой точки?
    Вдоль и против поля влетела изначально частица?
}
\answer{%
    \begin{align*}
    A_\text{внешних сил} &= \Delta E_\text{кин.} \implies A_\text{эл.
    поля} = 0 - \frac{mv^2}2.
    \\
    A_\text{эл.
    поля} &= q(\varphi_1 - \varphi_2) \implies\varphi_2 = \varphi_1 - \frac{A_\text{эл.
    поля}}q = \varphi_1 - \frac{- \frac{mv^2}2}q = \varphi_1 + \frac{mv^2}{2q} =  \\
    &= 800\,\text{В} + \frac{9{,}1 \cdot 10^{-31}\,\text{кг} \cdot \sqr{ 3000000\,\frac{\text{м}}{\text{с}} }}{2  \cdot 1{,}6 \cdot 10^{-19}\,\text{Кл}} \approx 825{,}6\,\text{В}.
    \end{align*}
}
\solutionspace{120pt}

\tasknumber{2}%
\task{%
    Три одинаковых положительных точечных заряда по $Q$ каждый находятся
    на одной прямой так, что расстояние между каждыми двумя соседними равно $3d$.
    Какую минимальную работу необходимо совершить, чтобы перевести эти заряды в положение,
    при котором они образуют прямоугольный равнобедренный треугольник с катетом $d$? Сделайте рисунки и получите ответ (формулой).
}
\solutionspace{120pt}

\tasknumber{3}%
\task{%
    На рисунке показано расположение трёх металлических пластин и указаны их потенциалы.
    Размеры пластин кораздо больше расстояния между ними.
    Отмечены также ось и начало координат.
    Дорисуйте на рисунке электрическое поле и постройте графики зависимости от координаты $x$:
    \begin{enumerate}
        \item проекции напряжённости электрического поля,
        \item потенциала электрического поля.
    \end{enumerate}
    \begin{tikzpicture}
        \draw[-{Latex}] (0, 0) -- (0, 3.5) node[below right] {$x$};
        \draw[thick]
            (-0.05, 0.5) -- (0.05, 0.5)     (0, 0.5) node[left] {$-3\,\text{см}$}     (0.5, 0.5) -- (4, 0.5) node[right] {$-30\,\text{В}$}
            (-0.05, 1.5) -- (0.05, 1.5)     (0, 1.5) node[left] {$0$}         (0.5, 1.5) -- (4, 1.5) node[right] {$0\,\text{В}$}
            (-0.05, 2.5) -- (0.05, 2.5)     (0, 2.5) node[left] {$3\,\text{см}$}    (0.5, 2.5) -- (4, 2.5) node[right] {$120\,\text{В}$};
    \end{tikzpicture}
}
\solutionspace{90pt}

\tasknumber{4}%
\task{%
    \begin{enumerate}
        \item Запишите закон Кулона (в диэлектрике).
        \item Из теоремы Гаусса выведите (нужен рисунок, применение и результат) формулу для напряженности электростатического поля снаружи равномерно заряженной сферы.
        \item Зарисуйте электрическое поле точечного положительного электрического заряда.
        \item Запишите формулу для вычисления потенциала электрического поля точечного заряда в диэлектрике.
        \item Запишите принцип суперпозиции (правило сложения) потенциалов.
    \end{enumerate}
}

\variantsplitter

\addpersonalvariant{Алёна Куприянова}

\tasknumber{1}%
\task{%
    Электрон $e^-$ вылетает из точки, потенциал которой $\varphi = 600\,\text{В}$,
    со скоростью $v = 4000000\,\frac{\text{м}}{\text{с}}$ параллельно линиям напряжённости однородного электрического поля.
    % Будет поле его ускорять или тормозить?
    В некоторой точке частица остановилась.
    Каков потенциал этой точки?
    Вдоль и против поля влетела изначально частица?
}
\answer{%
    \begin{align*}
    A_\text{внешних сил} &= \Delta E_\text{кин.} \implies A_\text{эл.
    поля} = 0 - \frac{mv^2}2.
    \\
    A_\text{эл.
    поля} &= q(\varphi_1 - \varphi_2) \implies\varphi_2 = \varphi_1 - \frac{A_\text{эл.
    поля}}q = \varphi_1 - \frac{- \frac{mv^2}2}q = \varphi_1 + \frac{mv^2}{2q} =  \\
    &= 600\,\text{В} + \frac{9{,}1 \cdot 10^{-31}\,\text{кг} \cdot \sqr{ 4000000\,\frac{\text{м}}{\text{с}} }}{2  * (-1)  \cdot 1{,}6 \cdot 10^{-19}\,\text{Кл}} \approx 554{,}5\,\text{В}.
    \end{align*}
}
\solutionspace{120pt}

\tasknumber{2}%
\task{%
    Три одинаковых положительных точечных заряда по $Q$ каждый находятся
    на одной прямой так, что расстояние между каждыми двумя соседними равно $2l$.
    Какую минимальную работу необходимо совершить, чтобы перевести эти заряды в положение,
    при котором они образуют прямоугольный равнобедренный треугольник с катетом $l$? Сделайте рисунки и получите ответ (формулой).
}
\solutionspace{120pt}

\tasknumber{3}%
\task{%
    На рисунке показано расположение трёх металлических пластин и указаны их потенциалы.
    Размеры пластин кораздо больше расстояния между ними.
    Отмечены также ось и начало координат.
    Дорисуйте на рисунке электрическое поле и постройте графики зависимости от координаты $x$:
    \begin{enumerate}
        \item проекции напряжённости электрического поля,
        \item потенциала электрического поля.
    \end{enumerate}
    \begin{tikzpicture}
        \draw[-{Latex}] (0, 0) -- (0, 3.5) node[below right] {$x$};
        \draw[thick]
            (-0.05, 0.5) -- (0.05, 0.5)     (0, 0.5) node[left] {$-3\,\text{см}$}     (0.5, 0.5) -- (4, 0.5) node[right] {$90\,\text{В}$}
            (-0.05, 1.5) -- (0.05, 1.5)     (0, 1.5) node[left] {$0$}         (0.5, 1.5) -- (4, 1.5) node[right] {$0\,\text{В}$}
            (-0.05, 2.5) -- (0.05, 2.5)     (0, 2.5) node[left] {$3\,\text{см}$}    (0.5, 2.5) -- (4, 2.5) node[right] {$-120\,\text{В}$};
    \end{tikzpicture}
}
\solutionspace{90pt}

\tasknumber{4}%
\task{%
    \begin{enumerate}
        \item Запишите теорему Гаусса.
        \item Из теоремы Гаусса выведите (нужен рисунок, применение и результат) формулу для напряженности электростатического поля внутри равномерно заряженной сферы.
        \item Зарисуйте электрическое поле точечного отрицательного электрического заряда.
        \item Запишите формулу для вычисления потенциала электрического поля точечного заряда в диэлектрике.
        \item Запишите принцип суперпозиции (правило сложения) напряжённостей.
    \end{enumerate}
}

\variantsplitter

\addpersonalvariant{Ярослав Лавровский}

\tasknumber{1}%
\task{%
    Электрон $e^-$ вылетает из точки, потенциал которой $\varphi = 600\,\text{В}$,
    со скоростью $v = 4000000\,\frac{\text{м}}{\text{с}}$ параллельно линиям напряжённости однородного электрического поля.
    % Будет поле его ускорять или тормозить?
    В некоторой точке частица остановилась.
    Каков потенциал этой точки?
    Вдоль и против поля влетела изначально частица?
}
\answer{%
    \begin{align*}
    A_\text{внешних сил} &= \Delta E_\text{кин.} \implies A_\text{эл.
    поля} = 0 - \frac{mv^2}2.
    \\
    A_\text{эл.
    поля} &= q(\varphi_1 - \varphi_2) \implies\varphi_2 = \varphi_1 - \frac{A_\text{эл.
    поля}}q = \varphi_1 - \frac{- \frac{mv^2}2}q = \varphi_1 + \frac{mv^2}{2q} =  \\
    &= 600\,\text{В} + \frac{9{,}1 \cdot 10^{-31}\,\text{кг} \cdot \sqr{ 4000000\,\frac{\text{м}}{\text{с}} }}{2  * (-1)  \cdot 1{,}6 \cdot 10^{-19}\,\text{Кл}} \approx 554{,}5\,\text{В}.
    \end{align*}
}
\solutionspace{120pt}

\tasknumber{2}%
\task{%
    Три одинаковых положительных точечных заряда по $Q$ каждый находятся
    на одной прямой так, что расстояние между каждыми двумя соседними равно $3d$.
    Какую минимальную работу необходимо совершить, чтобы перевести эти заряды в положение,
    при котором они образуют прямоугольный равнобедренный треугольник с гипотенузой $d$? Сделайте рисунки и получите ответ (формулой).
}
\solutionspace{120pt}

\tasknumber{3}%
\task{%
    На рисунке показано расположение трёх металлических пластин и указаны их потенциалы.
    Размеры пластин кораздо больше расстояния между ними.
    Отмечены также ось и начало координат.
    Дорисуйте на рисунке электрическое поле и постройте графики зависимости от координаты $x$:
    \begin{enumerate}
        \item проекции напряжённости электрического поля,
        \item потенциала электрического поля.
    \end{enumerate}
    \begin{tikzpicture}
        \draw[-{Latex}] (0, 0) -- (0, 3.5) node[below right] {$x$};
        \draw[thick]
            (-0.05, 0.5) -- (0.05, 0.5)     (0, 0.5) node[left] {$-3\,\text{см}$}     (0.5, 0.5) -- (4, 0.5) node[right] {$90\,\text{В}$}
            (-0.05, 1.5) -- (0.05, 1.5)     (0, 1.5) node[left] {$0$}         (0.5, 1.5) -- (4, 1.5) node[right] {$0\,\text{В}$}
            (-0.05, 2.5) -- (0.05, 2.5)     (0, 2.5) node[left] {$3\,\text{см}$}    (0.5, 2.5) -- (4, 2.5) node[right] {$-120\,\text{В}$};
    \end{tikzpicture}
}
\solutionspace{90pt}

\tasknumber{4}%
\task{%
    \begin{enumerate}
        \item Запишите закон сохранения электрического заряда.
        \item Из теоремы Гаусса выведите (нужен рисунок, применение и результат) формулу для напряженности электростатического поля снаружи равномерно заряженной сферы.
        \item Зарисуйте электрическое поле точечного положительного электрического заряда.
        \item Запишите формулу для вычисления потенциала электрического поля точечного заряда в диэлектрике.
        \item Запишите принцип суперпозиции (правило сложения) напряжённостей.
    \end{enumerate}
}

\variantsplitter

\addpersonalvariant{Анастасия Ламанова}

\tasknumber{1}%
\task{%
    Электрон $e^-$ вылетает из точки, потенциал которой $\varphi = 200\,\text{В}$,
    со скоростью $v = 3000000\,\frac{\text{м}}{\text{с}}$ параллельно линиям напряжённости однородного электрического поля.
    % Будет поле его ускорять или тормозить?
    В некоторой точке частица остановилась.
    Каков потенциал этой точки?
    Вдоль и против поля влетела изначально частица?
}
\answer{%
    \begin{align*}
    A_\text{внешних сил} &= \Delta E_\text{кин.} \implies A_\text{эл.
    поля} = 0 - \frac{mv^2}2.
    \\
    A_\text{эл.
    поля} &= q(\varphi_1 - \varphi_2) \implies\varphi_2 = \varphi_1 - \frac{A_\text{эл.
    поля}}q = \varphi_1 - \frac{- \frac{mv^2}2}q = \varphi_1 + \frac{mv^2}{2q} =  \\
    &= 200\,\text{В} + \frac{9{,}1 \cdot 10^{-31}\,\text{кг} \cdot \sqr{ 3000000\,\frac{\text{м}}{\text{с}} }}{2  * (-1)  \cdot 1{,}6 \cdot 10^{-19}\,\text{Кл}} \approx 174{,}4\,\text{В}.
    \end{align*}
}
\solutionspace{120pt}

\tasknumber{2}%
\task{%
    Три одинаковых положительных точечных заряда по $q$ каждый находятся
    на одной прямой так, что расстояние между каждыми двумя соседними равно $2d$.
    Какую минимальную работу необходимо совершить, чтобы перевести эти заряды в положение,
    при котором они образуют прямоугольный равнобедренный треугольник с гипотенузой $d$? Сделайте рисунки и получите ответ (формулой).
}
\solutionspace{120pt}

\tasknumber{3}%
\task{%
    На рисунке показано расположение трёх металлических пластин и указаны их потенциалы.
    Размеры пластин кораздо больше расстояния между ними.
    Отмечены также ось и начало координат.
    Дорисуйте на рисунке электрическое поле и постройте графики зависимости от координаты $x$:
    \begin{enumerate}
        \item проекции напряжённости электрического поля,
        \item потенциала электрического поля.
    \end{enumerate}
    \begin{tikzpicture}
        \draw[-{Latex}] (0, 0) -- (0, 3.5) node[below right] {$x$};
        \draw[thick]
            (-0.05, 0.5) -- (0.05, 0.5)     (0, 0.5) node[left] {$-2\,\text{см}$}     (0.5, 0.5) -- (4, 0.5) node[right] {$30\,\text{В}$}
            (-0.05, 1.5) -- (0.05, 1.5)     (0, 1.5) node[left] {$0$}         (0.5, 1.5) -- (4, 1.5) node[right] {$0\,\text{В}$}
            (-0.05, 2.5) -- (0.05, 2.5)     (0, 2.5) node[left] {$2\,\text{см}$}    (0.5, 2.5) -- (4, 2.5) node[right] {$120\,\text{В}$};
    \end{tikzpicture}
}
\solutionspace{90pt}

\tasknumber{4}%
\task{%
    \begin{enumerate}
        \item Запишите закон сохранения электрического заряда.
        \item Из теоремы Гаусса выведите (нужен рисунок, применение и результат) формулу для напряженности электростатического поля внутри равномерно заряженной сферы.
        \item Зарисуйте электрическое поле точечного отрицательного электрического заряда.
        \item Запишите формулу для вычисления потенциала электрического поля точечного заряда в диэлектрике.
        \item Запишите принцип суперпозиции (правило сложения) напряжённостей.
    \end{enumerate}
}

\variantsplitter

\addpersonalvariant{Виктория Легонькова}

\tasknumber{1}%
\task{%
    Электрон $e^-$ вылетает из точки, потенциал которой $\varphi = 800\,\text{В}$,
    со скоростью $v = 10000000\,\frac{\text{м}}{\text{с}}$ параллельно линиям напряжённости однородного электрического поля.
    % Будет поле его ускорять или тормозить?
    В некоторой точке частица остановилась.
    Каков потенциал этой точки?
    Вдоль и против поля влетела изначально частица?
}
\answer{%
    \begin{align*}
    A_\text{внешних сил} &= \Delta E_\text{кин.} \implies A_\text{эл.
    поля} = 0 - \frac{mv^2}2.
    \\
    A_\text{эл.
    поля} &= q(\varphi_1 - \varphi_2) \implies\varphi_2 = \varphi_1 - \frac{A_\text{эл.
    поля}}q = \varphi_1 - \frac{- \frac{mv^2}2}q = \varphi_1 + \frac{mv^2}{2q} =  \\
    &= 800\,\text{В} + \frac{9{,}1 \cdot 10^{-31}\,\text{кг} \cdot \sqr{ 10000000\,\frac{\text{м}}{\text{с}} }}{2  * (-1)  \cdot 1{,}6 \cdot 10^{-19}\,\text{Кл}} \approx 515{,}6\,\text{В}.
    \end{align*}
}
\solutionspace{120pt}

\tasknumber{2}%
\task{%
    Три одинаковых положительных точечных заряда по $Q$ каждый находятся
    на одной прямой так, что расстояние между каждыми двумя соседними равно $3r$.
    Какую минимальную работу необходимо совершить, чтобы перевести эти заряды в положение,
    при котором они образуют прямоугольный равнобедренный треугольник с катетом $r$? Сделайте рисунки и получите ответ (формулой).
}
\solutionspace{120pt}

\tasknumber{3}%
\task{%
    На рисунке показано расположение трёх металлических пластин и указаны их потенциалы.
    Размеры пластин кораздо больше расстояния между ними.
    Отмечены также ось и начало координат.
    Дорисуйте на рисунке электрическое поле и постройте графики зависимости от координаты $x$:
    \begin{enumerate}
        \item проекции напряжённости электрического поля,
        \item потенциала электрического поля.
    \end{enumerate}
    \begin{tikzpicture}
        \draw[-{Latex}] (0, 0) -- (0, 3.5) node[below right] {$x$};
        \draw[thick]
            (-0.05, 0.5) -- (0.05, 0.5)     (0, 0.5) node[left] {$-3\,\text{см}$}     (0.5, 0.5) -- (4, 0.5) node[right] {$30\,\text{В}$}
            (-0.05, 1.5) -- (0.05, 1.5)     (0, 1.5) node[left] {$0$}         (0.5, 1.5) -- (4, 1.5) node[right] {$0\,\text{В}$}
            (-0.05, 2.5) -- (0.05, 2.5)     (0, 2.5) node[left] {$3\,\text{см}$}    (0.5, 2.5) -- (4, 2.5) node[right] {$120\,\text{В}$};
    \end{tikzpicture}
}
\solutionspace{90pt}

\tasknumber{4}%
\task{%
    \begin{enumerate}
        \item Запишите теорему Гаусса.
        \item Из теоремы Гаусса выведите (нужен рисунок, применение и результат) формулу для напряженности электростатического поля снаружи равномерно заряженной сферы.
        \item Зарисуйте электрическое поле точечного отрицательного электрического заряда.
        \item Запишите формулу для вычисления напряжённости электрического поля точечного заряда в диэлектрике.
        \item Запишите принцип суперпозиции (правило сложения) напряжённостей.
    \end{enumerate}
}

\variantsplitter

\addpersonalvariant{Семён Мартынов}

\tasknumber{1}%
\task{%
    Электрон $e^-$ вылетает из точки, потенциал которой $\varphi = 400\,\text{В}$,
    со скоростью $v = 12000000\,\frac{\text{м}}{\text{с}}$ параллельно линиям напряжённости однородного электрического поля.
    % Будет поле его ускорять или тормозить?
    В некоторой точке частица остановилась.
    Каков потенциал этой точки?
    Вдоль и против поля влетела изначально частица?
}
\answer{%
    \begin{align*}
    A_\text{внешних сил} &= \Delta E_\text{кин.} \implies A_\text{эл.
    поля} = 0 - \frac{mv^2}2.
    \\
    A_\text{эл.
    поля} &= q(\varphi_1 - \varphi_2) \implies\varphi_2 = \varphi_1 - \frac{A_\text{эл.
    поля}}q = \varphi_1 - \frac{- \frac{mv^2}2}q = \varphi_1 + \frac{mv^2}{2q} =  \\
    &= 400\,\text{В} + \frac{9{,}1 \cdot 10^{-31}\,\text{кг} \cdot \sqr{ 12000000\,\frac{\text{м}}{\text{с}} }}{2  * (-1)  \cdot 1{,}6 \cdot 10^{-19}\,\text{Кл}} \approx -9{,}50\,\text{В}.
    \end{align*}
}
\solutionspace{120pt}

\tasknumber{2}%
\task{%
    Три одинаковых положительных точечных заряда по $q$ каждый находятся
    на одной прямой так, что расстояние между каждыми двумя соседними равно $2r$.
    Какую минимальную работу необходимо совершить, чтобы перевести эти заряды в положение,
    при котором они образуют равносторонний треугольник со стороной $r$? Сделайте рисунки и получите ответ (формулой).
}
\solutionspace{120pt}

\tasknumber{3}%
\task{%
    На рисунке показано расположение трёх металлических пластин и указаны их потенциалы.
    Размеры пластин кораздо больше расстояния между ними.
    Отмечены также ось и начало координат.
    Дорисуйте на рисунке электрическое поле и постройте графики зависимости от координаты $x$:
    \begin{enumerate}
        \item проекции напряжённости электрического поля,
        \item потенциала электрического поля.
    \end{enumerate}
    \begin{tikzpicture}
        \draw[-{Latex}] (0, 0) -- (0, 3.5) node[below right] {$x$};
        \draw[thick]
            (-0.05, 0.5) -- (0.05, 0.5)     (0, 0.5) node[left] {$-2\,\text{см}$}     (0.5, 0.5) -- (4, 0.5) node[right] {$-90\,\text{В}$}
            (-0.05, 1.5) -- (0.05, 1.5)     (0, 1.5) node[left] {$0$}         (0.5, 1.5) -- (4, 1.5) node[right] {$0\,\text{В}$}
            (-0.05, 2.5) -- (0.05, 2.5)     (0, 2.5) node[left] {$2\,\text{см}$}    (0.5, 2.5) -- (4, 2.5) node[right] {$120\,\text{В}$};
    \end{tikzpicture}
}
\solutionspace{90pt}

\tasknumber{4}%
\task{%
    \begin{enumerate}
        \item Запишите теорему Гаусса.
        \item Из теоремы Гаусса выведите (нужен рисунок, применение и результат) формулу для напряженности электростатического поля около равномерно заряженной бесконечной плоскости.
        \item Зарисуйте электрическое поле точечного отрицательного электрического заряда.
        \item Запишите формулу для вычисления потенциала электрического поля точечного заряда в диэлектрике.
        \item Запишите принцип суперпозиции (правило сложения) напряжённостей.
    \end{enumerate}
}

\variantsplitter

\addpersonalvariant{Варвара Минаева}

\tasknumber{1}%
\task{%
    Позитрон $e^+$ вылетает из точки, потенциал которой $\varphi = 400\,\text{В}$,
    со скоростью $v = 3000000\,\frac{\text{м}}{\text{с}}$ параллельно линиям напряжённости однородного электрического поля.
    % Будет поле его ускорять или тормозить?
    В некоторой точке частица остановилась.
    Каков потенциал этой точки?
    Вдоль и против поля влетела изначально частица?
}
\answer{%
    \begin{align*}
    A_\text{внешних сил} &= \Delta E_\text{кин.} \implies A_\text{эл.
    поля} = 0 - \frac{mv^2}2.
    \\
    A_\text{эл.
    поля} &= q(\varphi_1 - \varphi_2) \implies\varphi_2 = \varphi_1 - \frac{A_\text{эл.
    поля}}q = \varphi_1 - \frac{- \frac{mv^2}2}q = \varphi_1 + \frac{mv^2}{2q} =  \\
    &= 400\,\text{В} + \frac{9{,}1 \cdot 10^{-31}\,\text{кг} \cdot \sqr{ 3000000\,\frac{\text{м}}{\text{с}} }}{2  \cdot 1{,}6 \cdot 10^{-19}\,\text{Кл}} \approx 425{,}6\,\text{В}.
    \end{align*}
}
\solutionspace{120pt}

\tasknumber{2}%
\task{%
    Три одинаковых положительных точечных заряда по $q$ каждый находятся
    на одной прямой так, что расстояние между каждыми двумя соседними равно $2d$.
    Какую минимальную работу необходимо совершить, чтобы перевести эти заряды в положение,
    при котором они образуют прямоугольный равнобедренный треугольник с гипотенузой $d$? Сделайте рисунки и получите ответ (формулой).
}
\solutionspace{120pt}

\tasknumber{3}%
\task{%
    На рисунке показано расположение трёх металлических пластин и указаны их потенциалы.
    Размеры пластин кораздо больше расстояния между ними.
    Отмечены также ось и начало координат.
    Дорисуйте на рисунке электрическое поле и постройте графики зависимости от координаты $x$:
    \begin{enumerate}
        \item проекции напряжённости электрического поля,
        \item потенциала электрического поля.
    \end{enumerate}
    \begin{tikzpicture}
        \draw[-{Latex}] (0, 0) -- (0, 3.5) node[below right] {$x$};
        \draw[thick]
            (-0.05, 0.5) -- (0.05, 0.5)     (0, 0.5) node[left] {$-3\,\text{см}$}     (0.5, 0.5) -- (4, 0.5) node[right] {$150\,\text{В}$}
            (-0.05, 1.5) -- (0.05, 1.5)     (0, 1.5) node[left] {$0$}         (0.5, 1.5) -- (4, 1.5) node[right] {$0\,\text{В}$}
            (-0.05, 2.5) -- (0.05, 2.5)     (0, 2.5) node[left] {$3\,\text{см}$}    (0.5, 2.5) -- (4, 2.5) node[right] {$-120\,\text{В}$};
    \end{tikzpicture}
}
\solutionspace{90pt}

\tasknumber{4}%
\task{%
    \begin{enumerate}
        \item Запишите закон Кулона (в диэлектрике).
        \item Из теоремы Гаусса выведите (нужен рисунок, применение и результат) формулу для напряженности электростатического поля снаружи равномерно заряженной сферы.
        \item Зарисуйте электрическое поле точечного положительного электрического заряда.
        \item Запишите формулу для вычисления потенциала электрического поля точечного заряда в диэлектрике.
        \item Запишите принцип суперпозиции (правило сложения) напряжённостей.
    \end{enumerate}
}

\variantsplitter

\addpersonalvariant{Леонид Никитин}

\tasknumber{1}%
\task{%
    Электрон $e^-$ вылетает из точки, потенциал которой $\varphi = 200\,\text{В}$,
    со скоростью $v = 4000000\,\frac{\text{м}}{\text{с}}$ параллельно линиям напряжённости однородного электрического поля.
    % Будет поле его ускорять или тормозить?
    В некоторой точке частица остановилась.
    Каков потенциал этой точки?
    Вдоль и против поля влетела изначально частица?
}
\answer{%
    \begin{align*}
    A_\text{внешних сил} &= \Delta E_\text{кин.} \implies A_\text{эл.
    поля} = 0 - \frac{mv^2}2.
    \\
    A_\text{эл.
    поля} &= q(\varphi_1 - \varphi_2) \implies\varphi_2 = \varphi_1 - \frac{A_\text{эл.
    поля}}q = \varphi_1 - \frac{- \frac{mv^2}2}q = \varphi_1 + \frac{mv^2}{2q} =  \\
    &= 200\,\text{В} + \frac{9{,}1 \cdot 10^{-31}\,\text{кг} \cdot \sqr{ 4000000\,\frac{\text{м}}{\text{с}} }}{2  * (-1)  \cdot 1{,}6 \cdot 10^{-19}\,\text{Кл}} \approx 154{,}5\,\text{В}.
    \end{align*}
}
\solutionspace{120pt}

\tasknumber{2}%
\task{%
    Три одинаковых положительных точечных заряда по $Q$ каждый находятся
    на одной прямой так, что расстояние между каждыми двумя соседними равно $3l$.
    Какую минимальную работу необходимо совершить, чтобы перевести эти заряды в положение,
    при котором они образуют прямоугольный равнобедренный треугольник с гипотенузой $l$? Сделайте рисунки и получите ответ (формулой).
}
\solutionspace{120pt}

\tasknumber{3}%
\task{%
    На рисунке показано расположение трёх металлических пластин и указаны их потенциалы.
    Размеры пластин кораздо больше расстояния между ними.
    Отмечены также ось и начало координат.
    Дорисуйте на рисунке электрическое поле и постройте графики зависимости от координаты $x$:
    \begin{enumerate}
        \item проекции напряжённости электрического поля,
        \item потенциала электрического поля.
    \end{enumerate}
    \begin{tikzpicture}
        \draw[-{Latex}] (0, 0) -- (0, 3.5) node[below right] {$x$};
        \draw[thick]
            (-0.05, 0.5) -- (0.05, 0.5)     (0, 0.5) node[left] {$-2\,\text{см}$}     (0.5, 0.5) -- (4, 0.5) node[right] {$-30\,\text{В}$}
            (-0.05, 1.5) -- (0.05, 1.5)     (0, 1.5) node[left] {$0$}         (0.5, 1.5) -- (4, 1.5) node[right] {$0\,\text{В}$}
            (-0.05, 2.5) -- (0.05, 2.5)     (0, 2.5) node[left] {$2\,\text{см}$}    (0.5, 2.5) -- (4, 2.5) node[right] {$-120\,\text{В}$};
    \end{tikzpicture}
}
\solutionspace{90pt}

\tasknumber{4}%
\task{%
    \begin{enumerate}
        \item Запишите теорему Гаусса.
        \item Из теоремы Гаусса выведите (нужен рисунок, применение и результат) формулу для напряженности электростатического поля снаружи равномерно заряженной сферы.
        \item Зарисуйте электрическое поле точечного отрицательного электрического заряда.
        \item Запишите формулу для вычисления напряжённости электрического поля точечного заряда в диэлектрике.
        \item Запишите принцип суперпозиции (правило сложения) напряжённостей.
    \end{enumerate}
}

\variantsplitter

\addpersonalvariant{Тимофей Полетаев}

\tasknumber{1}%
\task{%
    Электрон $e^-$ вылетает из точки, потенциал которой $\varphi = 200\,\text{В}$,
    со скоростью $v = 10000000\,\frac{\text{м}}{\text{с}}$ параллельно линиям напряжённости однородного электрического поля.
    % Будет поле его ускорять или тормозить?
    В некоторой точке частица остановилась.
    Каков потенциал этой точки?
    Вдоль и против поля влетела изначально частица?
}
\answer{%
    \begin{align*}
    A_\text{внешних сил} &= \Delta E_\text{кин.} \implies A_\text{эл.
    поля} = 0 - \frac{mv^2}2.
    \\
    A_\text{эл.
    поля} &= q(\varphi_1 - \varphi_2) \implies\varphi_2 = \varphi_1 - \frac{A_\text{эл.
    поля}}q = \varphi_1 - \frac{- \frac{mv^2}2}q = \varphi_1 + \frac{mv^2}{2q} =  \\
    &= 200\,\text{В} + \frac{9{,}1 \cdot 10^{-31}\,\text{кг} \cdot \sqr{ 10000000\,\frac{\text{м}}{\text{с}} }}{2  * (-1)  \cdot 1{,}6 \cdot 10^{-19}\,\text{Кл}} \approx -84{,}40\,\text{В}.
    \end{align*}
}
\solutionspace{120pt}

\tasknumber{2}%
\task{%
    Три одинаковых положительных точечных заряда по $Q$ каждый находятся
    на одной прямой так, что расстояние между каждыми двумя соседними равно $2r$.
    Какую минимальную работу необходимо совершить, чтобы перевести эти заряды в положение,
    при котором они образуют прямоугольный равнобедренный треугольник с катетом $r$? Сделайте рисунки и получите ответ (формулой).
}
\solutionspace{120pt}

\tasknumber{3}%
\task{%
    На рисунке показано расположение трёх металлических пластин и указаны их потенциалы.
    Размеры пластин кораздо больше расстояния между ними.
    Отмечены также ось и начало координат.
    Дорисуйте на рисунке электрическое поле и постройте графики зависимости от координаты $x$:
    \begin{enumerate}
        \item проекции напряжённости электрического поля,
        \item потенциала электрического поля.
    \end{enumerate}
    \begin{tikzpicture}
        \draw[-{Latex}] (0, 0) -- (0, 3.5) node[below right] {$x$};
        \draw[thick]
            (-0.05, 0.5) -- (0.05, 0.5)     (0, 0.5) node[left] {$-3\,\text{см}$}     (0.5, 0.5) -- (4, 0.5) node[right] {$30\,\text{В}$}
            (-0.05, 1.5) -- (0.05, 1.5)     (0, 1.5) node[left] {$0$}         (0.5, 1.5) -- (4, 1.5) node[right] {$0\,\text{В}$}
            (-0.05, 2.5) -- (0.05, 2.5)     (0, 2.5) node[left] {$3\,\text{см}$}    (0.5, 2.5) -- (4, 2.5) node[right] {$120\,\text{В}$};
    \end{tikzpicture}
}
\solutionspace{90pt}

\tasknumber{4}%
\task{%
    \begin{enumerate}
        \item Запишите теорему Гаусса.
        \item Из теоремы Гаусса выведите (нужен рисунок, применение и результат) формулу для напряженности электростатического поля внутри равномерно заряженной сферы.
        \item Зарисуйте электрическое поле точечного отрицательного электрического заряда.
        \item Запишите формулу для вычисления потенциала электрического поля точечного заряда в диэлектрике.
        \item Запишите принцип суперпозиции (правило сложения) напряжённостей.
    \end{enumerate}
}

\variantsplitter

\addpersonalvariant{Андрей Рожков}

\tasknumber{1}%
\task{%
    Позитрон $e^+$ вылетает из точки, потенциал которой $\varphi = 200\,\text{В}$,
    со скоростью $v = 3000000\,\frac{\text{м}}{\text{с}}$ параллельно линиям напряжённости однородного электрического поля.
    % Будет поле его ускорять или тормозить?
    В некоторой точке частица остановилась.
    Каков потенциал этой точки?
    Вдоль и против поля влетела изначально частица?
}
\answer{%
    \begin{align*}
    A_\text{внешних сил} &= \Delta E_\text{кин.} \implies A_\text{эл.
    поля} = 0 - \frac{mv^2}2.
    \\
    A_\text{эл.
    поля} &= q(\varphi_1 - \varphi_2) \implies\varphi_2 = \varphi_1 - \frac{A_\text{эл.
    поля}}q = \varphi_1 - \frac{- \frac{mv^2}2}q = \varphi_1 + \frac{mv^2}{2q} =  \\
    &= 200\,\text{В} + \frac{9{,}1 \cdot 10^{-31}\,\text{кг} \cdot \sqr{ 3000000\,\frac{\text{м}}{\text{с}} }}{2  \cdot 1{,}6 \cdot 10^{-19}\,\text{Кл}} \approx 225{,}6\,\text{В}.
    \end{align*}
}
\solutionspace{120pt}

\tasknumber{2}%
\task{%
    Три одинаковых положительных точечных заряда по $Q$ каждый находятся
    на одной прямой так, что расстояние между каждыми двумя соседними равно $3d$.
    Какую минимальную работу необходимо совершить, чтобы перевести эти заряды в положение,
    при котором они образуют прямоугольный равнобедренный треугольник с катетом $d$? Сделайте рисунки и получите ответ (формулой).
}
\solutionspace{120pt}

\tasknumber{3}%
\task{%
    На рисунке показано расположение трёх металлических пластин и указаны их потенциалы.
    Размеры пластин кораздо больше расстояния между ними.
    Отмечены также ось и начало координат.
    Дорисуйте на рисунке электрическое поле и постройте графики зависимости от координаты $x$:
    \begin{enumerate}
        \item проекции напряжённости электрического поля,
        \item потенциала электрического поля.
    \end{enumerate}
    \begin{tikzpicture}
        \draw[-{Latex}] (0, 0) -- (0, 3.5) node[below right] {$x$};
        \draw[thick]
            (-0.05, 0.5) -- (0.05, 0.5)     (0, 0.5) node[left] {$-2\,\text{см}$}     (0.5, 0.5) -- (4, 0.5) node[right] {$-90\,\text{В}$}
            (-0.05, 1.5) -- (0.05, 1.5)     (0, 1.5) node[left] {$0$}         (0.5, 1.5) -- (4, 1.5) node[right] {$0\,\text{В}$}
            (-0.05, 2.5) -- (0.05, 2.5)     (0, 2.5) node[left] {$2\,\text{см}$}    (0.5, 2.5) -- (4, 2.5) node[right] {$60\,\text{В}$};
    \end{tikzpicture}
}
\solutionspace{90pt}

\tasknumber{4}%
\task{%
    \begin{enumerate}
        \item Запишите закон сохранения электрического заряда.
        \item Из теоремы Гаусса выведите (нужен рисунок, применение и результат) формулу для напряженности электростатического поля внутри равномерно заряженной сферы.
        \item Зарисуйте электрическое поле точечного положительного электрического заряда.
        \item Запишите формулу для вычисления потенциала электрического поля точечного заряда в диэлектрике.
        \item Запишите принцип суперпозиции (правило сложения) потенциалов.
    \end{enumerate}
}

\variantsplitter

\addpersonalvariant{Рената Таржиманова}

\tasknumber{1}%
\task{%
    Позитрон $e^+$ вылетает из точки, потенциал которой $\varphi = 400\,\text{В}$,
    со скоростью $v = 12000000\,\frac{\text{м}}{\text{с}}$ параллельно линиям напряжённости однородного электрического поля.
    % Будет поле его ускорять или тормозить?
    В некоторой точке частица остановилась.
    Каков потенциал этой точки?
    Вдоль и против поля влетела изначально частица?
}
\answer{%
    \begin{align*}
    A_\text{внешних сил} &= \Delta E_\text{кин.} \implies A_\text{эл.
    поля} = 0 - \frac{mv^2}2.
    \\
    A_\text{эл.
    поля} &= q(\varphi_1 - \varphi_2) \implies\varphi_2 = \varphi_1 - \frac{A_\text{эл.
    поля}}q = \varphi_1 - \frac{- \frac{mv^2}2}q = \varphi_1 + \frac{mv^2}{2q} =  \\
    &= 400\,\text{В} + \frac{9{,}1 \cdot 10^{-31}\,\text{кг} \cdot \sqr{ 12000000\,\frac{\text{м}}{\text{с}} }}{2  \cdot 1{,}6 \cdot 10^{-19}\,\text{Кл}} \approx 809{,}5\,\text{В}.
    \end{align*}
}
\solutionspace{120pt}

\tasknumber{2}%
\task{%
    Три одинаковых положительных точечных заряда по $q$ каждый находятся
    на одной прямой так, что расстояние между каждыми двумя соседними равно $3l$.
    Какую минимальную работу необходимо совершить, чтобы перевести эти заряды в положение,
    при котором они образуют прямоугольный равнобедренный треугольник с гипотенузой $l$? Сделайте рисунки и получите ответ (формулой).
}
\solutionspace{120pt}

\tasknumber{3}%
\task{%
    На рисунке показано расположение трёх металлических пластин и указаны их потенциалы.
    Размеры пластин кораздо больше расстояния между ними.
    Отмечены также ось и начало координат.
    Дорисуйте на рисунке электрическое поле и постройте графики зависимости от координаты $x$:
    \begin{enumerate}
        \item проекции напряжённости электрического поля,
        \item потенциала электрического поля.
    \end{enumerate}
    \begin{tikzpicture}
        \draw[-{Latex}] (0, 0) -- (0, 3.5) node[below right] {$x$};
        \draw[thick]
            (-0.05, 0.5) -- (0.05, 0.5)     (0, 0.5) node[left] {$-3\,\text{см}$}     (0.5, 0.5) -- (4, 0.5) node[right] {$90\,\text{В}$}
            (-0.05, 1.5) -- (0.05, 1.5)     (0, 1.5) node[left] {$0$}         (0.5, 1.5) -- (4, 1.5) node[right] {$0\,\text{В}$}
            (-0.05, 2.5) -- (0.05, 2.5)     (0, 2.5) node[left] {$3\,\text{см}$}    (0.5, 2.5) -- (4, 2.5) node[right] {$-60\,\text{В}$};
    \end{tikzpicture}
}
\solutionspace{90pt}

\tasknumber{4}%
\task{%
    \begin{enumerate}
        \item Запишите закон сохранения электрического заряда.
        \item Из теоремы Гаусса выведите (нужен рисунок, применение и результат) формулу для напряженности электростатического поля около равномерно заряженной бесконечной плоскости.
        \item Зарисуйте электрическое поле точечного отрицательного электрического заряда.
        \item Запишите формулу для вычисления напряжённости электрического поля точечного заряда в диэлектрике.
        \item Запишите принцип суперпозиции (правило сложения) потенциалов.
    \end{enumerate}
}

\variantsplitter

\addpersonalvariant{Андрей Щербаков}

\tasknumber{1}%
\task{%
    Позитрон $e^+$ вылетает из точки, потенциал которой $\varphi = 600\,\text{В}$,
    со скоростью $v = 10000000\,\frac{\text{м}}{\text{с}}$ параллельно линиям напряжённости однородного электрического поля.
    % Будет поле его ускорять или тормозить?
    В некоторой точке частица остановилась.
    Каков потенциал этой точки?
    Вдоль и против поля влетела изначально частица?
}
\answer{%
    \begin{align*}
    A_\text{внешних сил} &= \Delta E_\text{кин.} \implies A_\text{эл.
    поля} = 0 - \frac{mv^2}2.
    \\
    A_\text{эл.
    поля} &= q(\varphi_1 - \varphi_2) \implies\varphi_2 = \varphi_1 - \frac{A_\text{эл.
    поля}}q = \varphi_1 - \frac{- \frac{mv^2}2}q = \varphi_1 + \frac{mv^2}{2q} =  \\
    &= 600\,\text{В} + \frac{9{,}1 \cdot 10^{-31}\,\text{кг} \cdot \sqr{ 10000000\,\frac{\text{м}}{\text{с}} }}{2  \cdot 1{,}6 \cdot 10^{-19}\,\text{Кл}} \approx 884{,}4\,\text{В}.
    \end{align*}
}
\solutionspace{120pt}

\tasknumber{2}%
\task{%
    Три одинаковых положительных точечных заряда по $Q$ каждый находятся
    на одной прямой так, что расстояние между каждыми двумя соседними равно $3l$.
    Какую минимальную работу необходимо совершить, чтобы перевести эти заряды в положение,
    при котором они образуют равносторонний треугольник со стороной $l$? Сделайте рисунки и получите ответ (формулой).
}
\solutionspace{120pt}

\tasknumber{3}%
\task{%
    На рисунке показано расположение трёх металлических пластин и указаны их потенциалы.
    Размеры пластин кораздо больше расстояния между ними.
    Отмечены также ось и начало координат.
    Дорисуйте на рисунке электрическое поле и постройте графики зависимости от координаты $x$:
    \begin{enumerate}
        \item проекции напряжённости электрического поля,
        \item потенциала электрического поля.
    \end{enumerate}
    \begin{tikzpicture}
        \draw[-{Latex}] (0, 0) -- (0, 3.5) node[below right] {$x$};
        \draw[thick]
            (-0.05, 0.5) -- (0.05, 0.5)     (0, 0.5) node[left] {$-2\,\text{см}$}     (0.5, 0.5) -- (4, 0.5) node[right] {$-90\,\text{В}$}
            (-0.05, 1.5) -- (0.05, 1.5)     (0, 1.5) node[left] {$0$}         (0.5, 1.5) -- (4, 1.5) node[right] {$0\,\text{В}$}
            (-0.05, 2.5) -- (0.05, 2.5)     (0, 2.5) node[left] {$2\,\text{см}$}    (0.5, 2.5) -- (4, 2.5) node[right] {$-60\,\text{В}$};
    \end{tikzpicture}
}
\solutionspace{90pt}

\tasknumber{4}%
\task{%
    \begin{enumerate}
        \item Запишите закон сохранения электрического заряда.
        \item Из теоремы Гаусса выведите (нужен рисунок, применение и результат) формулу для напряженности электростатического поля снаружи равномерно заряженной сферы.
        \item Зарисуйте электрическое поле точечного положительного электрического заряда.
        \item Запишите формулу для вычисления потенциала электрического поля точечного заряда в диэлектрике.
        \item Запишите принцип суперпозиции (правило сложения) напряжённостей.
    \end{enumerate}
}

\variantsplitter

\addpersonalvariant{Михаил Ярошевский}

\tasknumber{1}%
\task{%
    Электрон $e^-$ вылетает из точки, потенциал которой $\varphi = 800\,\text{В}$,
    со скоростью $v = 3000000\,\frac{\text{м}}{\text{с}}$ параллельно линиям напряжённости однородного электрического поля.
    % Будет поле его ускорять или тормозить?
    В некоторой точке частица остановилась.
    Каков потенциал этой точки?
    Вдоль и против поля влетела изначально частица?
}
\answer{%
    \begin{align*}
    A_\text{внешних сил} &= \Delta E_\text{кин.} \implies A_\text{эл.
    поля} = 0 - \frac{mv^2}2.
    \\
    A_\text{эл.
    поля} &= q(\varphi_1 - \varphi_2) \implies\varphi_2 = \varphi_1 - \frac{A_\text{эл.
    поля}}q = \varphi_1 - \frac{- \frac{mv^2}2}q = \varphi_1 + \frac{mv^2}{2q} =  \\
    &= 800\,\text{В} + \frac{9{,}1 \cdot 10^{-31}\,\text{кг} \cdot \sqr{ 3000000\,\frac{\text{м}}{\text{с}} }}{2  * (-1)  \cdot 1{,}6 \cdot 10^{-19}\,\text{Кл}} \approx 774{,}4\,\text{В}.
    \end{align*}
}
\solutionspace{120pt}

\tasknumber{2}%
\task{%
    Три одинаковых положительных точечных заряда по $q$ каждый находятся
    на одной прямой так, что расстояние между каждыми двумя соседними равно $3d$.
    Какую минимальную работу необходимо совершить, чтобы перевести эти заряды в положение,
    при котором они образуют прямоугольный равнобедренный треугольник с гипотенузой $d$? Сделайте рисунки и получите ответ (формулой).
}
\solutionspace{120pt}

\tasknumber{3}%
\task{%
    На рисунке показано расположение трёх металлических пластин и указаны их потенциалы.
    Размеры пластин кораздо больше расстояния между ними.
    Отмечены также ось и начало координат.
    Дорисуйте на рисунке электрическое поле и постройте графики зависимости от координаты $x$:
    \begin{enumerate}
        \item проекции напряжённости электрического поля,
        \item потенциала электрического поля.
    \end{enumerate}
    \begin{tikzpicture}
        \draw[-{Latex}] (0, 0) -- (0, 3.5) node[below right] {$x$};
        \draw[thick]
            (-0.05, 0.5) -- (0.05, 0.5)     (0, 0.5) node[left] {$-2\,\text{см}$}     (0.5, 0.5) -- (4, 0.5) node[right] {$30\,\text{В}$}
            (-0.05, 1.5) -- (0.05, 1.5)     (0, 1.5) node[left] {$0$}         (0.5, 1.5) -- (4, 1.5) node[right] {$0\,\text{В}$}
            (-0.05, 2.5) -- (0.05, 2.5)     (0, 2.5) node[left] {$2\,\text{см}$}    (0.5, 2.5) -- (4, 2.5) node[right] {$-60\,\text{В}$};
    \end{tikzpicture}
}
\solutionspace{90pt}

\tasknumber{4}%
\task{%
    \begin{enumerate}
        \item Запишите закон Кулона (в диэлектрике).
        \item Из теоремы Гаусса выведите (нужен рисунок, применение и результат) формулу для напряженности электростатического поля снаружи равномерно заряженной сферы.
        \item Зарисуйте электрическое поле точечного отрицательного электрического заряда.
        \item Запишите формулу для вычисления потенциала электрического поля точечного заряда в диэлектрике.
        \item Запишите принцип суперпозиции (правило сложения) напряжённостей.
    \end{enumerate}
}

\variantsplitter

\addpersonalvariant{Алексей Алимпиев}

\tasknumber{1}%
\task{%
    Электрон $e^-$ вылетает из точки, потенциал которой $\varphi = 800\,\text{В}$,
    со скоростью $v = 12000000\,\frac{\text{м}}{\text{с}}$ параллельно линиям напряжённости однородного электрического поля.
    % Будет поле его ускорять или тормозить?
    В некоторой точке частица остановилась.
    Каков потенциал этой точки?
    Вдоль и против поля влетела изначально частица?
}
\answer{%
    \begin{align*}
    A_\text{внешних сил} &= \Delta E_\text{кин.} \implies A_\text{эл.
    поля} = 0 - \frac{mv^2}2.
    \\
    A_\text{эл.
    поля} &= q(\varphi_1 - \varphi_2) \implies\varphi_2 = \varphi_1 - \frac{A_\text{эл.
    поля}}q = \varphi_1 - \frac{- \frac{mv^2}2}q = \varphi_1 + \frac{mv^2}{2q} =  \\
    &= 800\,\text{В} + \frac{9{,}1 \cdot 10^{-31}\,\text{кг} \cdot \sqr{ 12000000\,\frac{\text{м}}{\text{с}} }}{2  * (-1)  \cdot 1{,}6 \cdot 10^{-19}\,\text{Кл}} \approx 390{,}5\,\text{В}.
    \end{align*}
}
\solutionspace{120pt}

\tasknumber{2}%
\task{%
    Три одинаковых положительных точечных заряда по $Q$ каждый находятся
    на одной прямой так, что расстояние между каждыми двумя соседними равно $3d$.
    Какую минимальную работу необходимо совершить, чтобы перевести эти заряды в положение,
    при котором они образуют прямоугольный равнобедренный треугольник с катетом $d$? Сделайте рисунки и получите ответ (формулой).
}
\solutionspace{120pt}

\tasknumber{3}%
\task{%
    На рисунке показано расположение трёх металлических пластин и указаны их потенциалы.
    Размеры пластин кораздо больше расстояния между ними.
    Отмечены также ось и начало координат.
    Дорисуйте на рисунке электрическое поле и постройте графики зависимости от координаты $x$:
    \begin{enumerate}
        \item проекции напряжённости электрического поля,
        \item потенциала электрического поля.
    \end{enumerate}
    \begin{tikzpicture}
        \draw[-{Latex}] (0, 0) -- (0, 3.5) node[below right] {$x$};
        \draw[thick]
            (-0.05, 0.5) -- (0.05, 0.5)     (0, 0.5) node[left] {$-3\,\text{см}$}     (0.5, 0.5) -- (4, 0.5) node[right] {$90\,\text{В}$}
            (-0.05, 1.5) -- (0.05, 1.5)     (0, 1.5) node[left] {$0$}         (0.5, 1.5) -- (4, 1.5) node[right] {$0\,\text{В}$}
            (-0.05, 2.5) -- (0.05, 2.5)     (0, 2.5) node[left] {$3\,\text{см}$}    (0.5, 2.5) -- (4, 2.5) node[right] {$-60\,\text{В}$};
    \end{tikzpicture}
}
\solutionspace{90pt}

\tasknumber{4}%
\task{%
    \begin{enumerate}
        \item Запишите теорему Гаусса.
        \item Из теоремы Гаусса выведите (нужен рисунок, применение и результат) формулу для напряженности электростатического поля внутри равномерно заряженной сферы.
        \item Зарисуйте электрическое поле точечного положительного электрического заряда.
        \item Запишите формулу для вычисления потенциала электрического поля точечного заряда в диэлектрике.
        \item Запишите принцип суперпозиции (правило сложения) потенциалов.
    \end{enumerate}
}

\variantsplitter

\addpersonalvariant{Евгений Васин}

\tasknumber{1}%
\task{%
    Электрон $e^-$ вылетает из точки, потенциал которой $\varphi = 600\,\text{В}$,
    со скоростью $v = 3000000\,\frac{\text{м}}{\text{с}}$ параллельно линиям напряжённости однородного электрического поля.
    % Будет поле его ускорять или тормозить?
    В некоторой точке частица остановилась.
    Каков потенциал этой точки?
    Вдоль и против поля влетела изначально частица?
}
\answer{%
    \begin{align*}
    A_\text{внешних сил} &= \Delta E_\text{кин.} \implies A_\text{эл.
    поля} = 0 - \frac{mv^2}2.
    \\
    A_\text{эл.
    поля} &= q(\varphi_1 - \varphi_2) \implies\varphi_2 = \varphi_1 - \frac{A_\text{эл.
    поля}}q = \varphi_1 - \frac{- \frac{mv^2}2}q = \varphi_1 + \frac{mv^2}{2q} =  \\
    &= 600\,\text{В} + \frac{9{,}1 \cdot 10^{-31}\,\text{кг} \cdot \sqr{ 3000000\,\frac{\text{м}}{\text{с}} }}{2  * (-1)  \cdot 1{,}6 \cdot 10^{-19}\,\text{Кл}} \approx 574{,}4\,\text{В}.
    \end{align*}
}
\solutionspace{120pt}

\tasknumber{2}%
\task{%
    Три одинаковых положительных точечных заряда по $q$ каждый находятся
    на одной прямой так, что расстояние между каждыми двумя соседними равно $2l$.
    Какую минимальную работу необходимо совершить, чтобы перевести эти заряды в положение,
    при котором они образуют прямоугольный равнобедренный треугольник с гипотенузой $l$? Сделайте рисунки и получите ответ (формулой).
}
\solutionspace{120pt}

\tasknumber{3}%
\task{%
    На рисунке показано расположение трёх металлических пластин и указаны их потенциалы.
    Размеры пластин кораздо больше расстояния между ними.
    Отмечены также ось и начало координат.
    Дорисуйте на рисунке электрическое поле и постройте графики зависимости от координаты $x$:
    \begin{enumerate}
        \item проекции напряжённости электрического поля,
        \item потенциала электрического поля.
    \end{enumerate}
    \begin{tikzpicture}
        \draw[-{Latex}] (0, 0) -- (0, 3.5) node[below right] {$x$};
        \draw[thick]
            (-0.05, 0.5) -- (0.05, 0.5)     (0, 0.5) node[left] {$-2\,\text{см}$}     (0.5, 0.5) -- (4, 0.5) node[right] {$90\,\text{В}$}
            (-0.05, 1.5) -- (0.05, 1.5)     (0, 1.5) node[left] {$0$}         (0.5, 1.5) -- (4, 1.5) node[right] {$0\,\text{В}$}
            (-0.05, 2.5) -- (0.05, 2.5)     (0, 2.5) node[left] {$2\,\text{см}$}    (0.5, 2.5) -- (4, 2.5) node[right] {$120\,\text{В}$};
    \end{tikzpicture}
}
\solutionspace{90pt}

\tasknumber{4}%
\task{%
    \begin{enumerate}
        \item Запишите закон Кулона (в диэлектрике).
        \item Из теоремы Гаусса выведите (нужен рисунок, применение и результат) формулу для напряженности электростатического поля внутри равномерно заряженной сферы.
        \item Зарисуйте электрическое поле точечного отрицательного электрического заряда.
        \item Запишите формулу для вычисления потенциала электрического поля точечного заряда в диэлектрике.
        \item Запишите принцип суперпозиции (правило сложения) напряжённостей.
    \end{enumerate}
}

\variantsplitter

\addpersonalvariant{Вячеслав Волохов}

\tasknumber{1}%
\task{%
    Электрон $e^-$ вылетает из точки, потенциал которой $\varphi = 1000\,\text{В}$,
    со скоростью $v = 3000000\,\frac{\text{м}}{\text{с}}$ параллельно линиям напряжённости однородного электрического поля.
    % Будет поле его ускорять или тормозить?
    В некоторой точке частица остановилась.
    Каков потенциал этой точки?
    Вдоль и против поля влетела изначально частица?
}
\answer{%
    \begin{align*}
    A_\text{внешних сил} &= \Delta E_\text{кин.} \implies A_\text{эл.
    поля} = 0 - \frac{mv^2}2.
    \\
    A_\text{эл.
    поля} &= q(\varphi_1 - \varphi_2) \implies\varphi_2 = \varphi_1 - \frac{A_\text{эл.
    поля}}q = \varphi_1 - \frac{- \frac{mv^2}2}q = \varphi_1 + \frac{mv^2}{2q} =  \\
    &= 1000\,\text{В} + \frac{9{,}1 \cdot 10^{-31}\,\text{кг} \cdot \sqr{ 3000000\,\frac{\text{м}}{\text{с}} }}{2  * (-1)  \cdot 1{,}6 \cdot 10^{-19}\,\text{Кл}} \approx 974{,}4\,\text{В}.
    \end{align*}
}
\solutionspace{120pt}

\tasknumber{2}%
\task{%
    Три одинаковых положительных точечных заряда по $q$ каждый находятся
    на одной прямой так, что расстояние между каждыми двумя соседними равно $2l$.
    Какую минимальную работу необходимо совершить, чтобы перевести эти заряды в положение,
    при котором они образуют равносторонний треугольник со стороной $l$? Сделайте рисунки и получите ответ (формулой).
}
\solutionspace{120pt}

\tasknumber{3}%
\task{%
    На рисунке показано расположение трёх металлических пластин и указаны их потенциалы.
    Размеры пластин кораздо больше расстояния между ними.
    Отмечены также ось и начало координат.
    Дорисуйте на рисунке электрическое поле и постройте графики зависимости от координаты $x$:
    \begin{enumerate}
        \item проекции напряжённости электрического поля,
        \item потенциала электрического поля.
    \end{enumerate}
    \begin{tikzpicture}
        \draw[-{Latex}] (0, 0) -- (0, 3.5) node[below right] {$x$};
        \draw[thick]
            (-0.05, 0.5) -- (0.05, 0.5)     (0, 0.5) node[left] {$-2\,\text{см}$}     (0.5, 0.5) -- (4, 0.5) node[right] {$-30\,\text{В}$}
            (-0.05, 1.5) -- (0.05, 1.5)     (0, 1.5) node[left] {$0$}         (0.5, 1.5) -- (4, 1.5) node[right] {$0\,\text{В}$}
            (-0.05, 2.5) -- (0.05, 2.5)     (0, 2.5) node[left] {$2\,\text{см}$}    (0.5, 2.5) -- (4, 2.5) node[right] {$-120\,\text{В}$};
    \end{tikzpicture}
}
\solutionspace{90pt}

\tasknumber{4}%
\task{%
    \begin{enumerate}
        \item Запишите закон Кулона (в диэлектрике).
        \item Из теоремы Гаусса выведите (нужен рисунок, применение и результат) формулу для напряженности электростатического поля внутри равномерно заряженной сферы.
        \item Зарисуйте электрическое поле точечного отрицательного электрического заряда.
        \item Запишите формулу для вычисления напряжённости электрического поля точечного заряда в диэлектрике.
        \item Запишите принцип суперпозиции (правило сложения) напряжённостей.
    \end{enumerate}
}

\variantsplitter

\addpersonalvariant{Герман Говоров}

\tasknumber{1}%
\task{%
    Электрон $e^-$ вылетает из точки, потенциал которой $\varphi = 200\,\text{В}$,
    со скоростью $v = 4000000\,\frac{\text{м}}{\text{с}}$ параллельно линиям напряжённости однородного электрического поля.
    % Будет поле его ускорять или тормозить?
    В некоторой точке частица остановилась.
    Каков потенциал этой точки?
    Вдоль и против поля влетела изначально частица?
}
\answer{%
    \begin{align*}
    A_\text{внешних сил} &= \Delta E_\text{кин.} \implies A_\text{эл.
    поля} = 0 - \frac{mv^2}2.
    \\
    A_\text{эл.
    поля} &= q(\varphi_1 - \varphi_2) \implies\varphi_2 = \varphi_1 - \frac{A_\text{эл.
    поля}}q = \varphi_1 - \frac{- \frac{mv^2}2}q = \varphi_1 + \frac{mv^2}{2q} =  \\
    &= 200\,\text{В} + \frac{9{,}1 \cdot 10^{-31}\,\text{кг} \cdot \sqr{ 4000000\,\frac{\text{м}}{\text{с}} }}{2  * (-1)  \cdot 1{,}6 \cdot 10^{-19}\,\text{Кл}} \approx 154{,}5\,\text{В}.
    \end{align*}
}
\solutionspace{120pt}

\tasknumber{2}%
\task{%
    Три одинаковых положительных точечных заряда по $q$ каждый находятся
    на одной прямой так, что расстояние между каждыми двумя соседними равно $2d$.
    Какую минимальную работу необходимо совершить, чтобы перевести эти заряды в положение,
    при котором они образуют прямоугольный равнобедренный треугольник с гипотенузой $d$? Сделайте рисунки и получите ответ (формулой).
}
\solutionspace{120pt}

\tasknumber{3}%
\task{%
    На рисунке показано расположение трёх металлических пластин и указаны их потенциалы.
    Размеры пластин кораздо больше расстояния между ними.
    Отмечены также ось и начало координат.
    Дорисуйте на рисунке электрическое поле и постройте графики зависимости от координаты $x$:
    \begin{enumerate}
        \item проекции напряжённости электрического поля,
        \item потенциала электрического поля.
    \end{enumerate}
    \begin{tikzpicture}
        \draw[-{Latex}] (0, 0) -- (0, 3.5) node[below right] {$x$};
        \draw[thick]
            (-0.05, 0.5) -- (0.05, 0.5)     (0, 0.5) node[left] {$-2\,\text{см}$}     (0.5, 0.5) -- (4, 0.5) node[right] {$90\,\text{В}$}
            (-0.05, 1.5) -- (0.05, 1.5)     (0, 1.5) node[left] {$0$}         (0.5, 1.5) -- (4, 1.5) node[right] {$0\,\text{В}$}
            (-0.05, 2.5) -- (0.05, 2.5)     (0, 2.5) node[left] {$2\,\text{см}$}    (0.5, 2.5) -- (4, 2.5) node[right] {$120\,\text{В}$};
    \end{tikzpicture}
}
\solutionspace{90pt}

\tasknumber{4}%
\task{%
    \begin{enumerate}
        \item Запишите закон Кулона (в диэлектрике).
        \item Из теоремы Гаусса выведите (нужен рисунок, применение и результат) формулу для напряженности электростатического поля снаружи равномерно заряженной сферы.
        \item Зарисуйте электрическое поле точечного отрицательного электрического заряда.
        \item Запишите формулу для вычисления потенциала электрического поля точечного заряда в диэлектрике.
        \item Запишите принцип суперпозиции (правило сложения) напряжённостей.
    \end{enumerate}
}

\variantsplitter

\addpersonalvariant{София Журавлёва}

\tasknumber{1}%
\task{%
    Позитрон $e^+$ вылетает из точки, потенциал которой $\varphi = 600\,\text{В}$,
    со скоростью $v = 4000000\,\frac{\text{м}}{\text{с}}$ параллельно линиям напряжённости однородного электрического поля.
    % Будет поле его ускорять или тормозить?
    В некоторой точке частица остановилась.
    Каков потенциал этой точки?
    Вдоль и против поля влетела изначально частица?
}
\answer{%
    \begin{align*}
    A_\text{внешних сил} &= \Delta E_\text{кин.} \implies A_\text{эл.
    поля} = 0 - \frac{mv^2}2.
    \\
    A_\text{эл.
    поля} &= q(\varphi_1 - \varphi_2) \implies\varphi_2 = \varphi_1 - \frac{A_\text{эл.
    поля}}q = \varphi_1 - \frac{- \frac{mv^2}2}q = \varphi_1 + \frac{mv^2}{2q} =  \\
    &= 600\,\text{В} + \frac{9{,}1 \cdot 10^{-31}\,\text{кг} \cdot \sqr{ 4000000\,\frac{\text{м}}{\text{с}} }}{2  \cdot 1{,}6 \cdot 10^{-19}\,\text{Кл}} \approx 645{,}5\,\text{В}.
    \end{align*}
}
\solutionspace{120pt}

\tasknumber{2}%
\task{%
    Три одинаковых положительных точечных заряда по $q$ каждый находятся
    на одной прямой так, что расстояние между каждыми двумя соседними равно $2r$.
    Какую минимальную работу необходимо совершить, чтобы перевести эти заряды в положение,
    при котором они образуют прямоугольный равнобедренный треугольник с гипотенузой $r$? Сделайте рисунки и получите ответ (формулой).
}
\solutionspace{120pt}

\tasknumber{3}%
\task{%
    На рисунке показано расположение трёх металлических пластин и указаны их потенциалы.
    Размеры пластин кораздо больше расстояния между ними.
    Отмечены также ось и начало координат.
    Дорисуйте на рисунке электрическое поле и постройте графики зависимости от координаты $x$:
    \begin{enumerate}
        \item проекции напряжённости электрического поля,
        \item потенциала электрического поля.
    \end{enumerate}
    \begin{tikzpicture}
        \draw[-{Latex}] (0, 0) -- (0, 3.5) node[below right] {$x$};
        \draw[thick]
            (-0.05, 0.5) -- (0.05, 0.5)     (0, 0.5) node[left] {$-3\,\text{см}$}     (0.5, 0.5) -- (4, 0.5) node[right] {$150\,\text{В}$}
            (-0.05, 1.5) -- (0.05, 1.5)     (0, 1.5) node[left] {$0$}         (0.5, 1.5) -- (4, 1.5) node[right] {$0\,\text{В}$}
            (-0.05, 2.5) -- (0.05, 2.5)     (0, 2.5) node[left] {$3\,\text{см}$}    (0.5, 2.5) -- (4, 2.5) node[right] {$-60\,\text{В}$};
    \end{tikzpicture}
}
\solutionspace{90pt}

\tasknumber{4}%
\task{%
    \begin{enumerate}
        \item Запишите закон сохранения электрического заряда.
        \item Из теоремы Гаусса выведите (нужен рисунок, применение и результат) формулу для напряженности электростатического поля около равномерно заряженной бесконечной плоскости.
        \item Зарисуйте электрическое поле точечного положительного электрического заряда.
        \item Запишите формулу для вычисления потенциала электрического поля точечного заряда в диэлектрике.
        \item Запишите принцип суперпозиции (правило сложения) потенциалов.
    \end{enumerate}
}

\variantsplitter

\addpersonalvariant{Константин Козлов}

\tasknumber{1}%
\task{%
    Позитрон $e^+$ вылетает из точки, потенциал которой $\varphi = 200\,\text{В}$,
    со скоростью $v = 10000000\,\frac{\text{м}}{\text{с}}$ параллельно линиям напряжённости однородного электрического поля.
    % Будет поле его ускорять или тормозить?
    В некоторой точке частица остановилась.
    Каков потенциал этой точки?
    Вдоль и против поля влетела изначально частица?
}
\answer{%
    \begin{align*}
    A_\text{внешних сил} &= \Delta E_\text{кин.} \implies A_\text{эл.
    поля} = 0 - \frac{mv^2}2.
    \\
    A_\text{эл.
    поля} &= q(\varphi_1 - \varphi_2) \implies\varphi_2 = \varphi_1 - \frac{A_\text{эл.
    поля}}q = \varphi_1 - \frac{- \frac{mv^2}2}q = \varphi_1 + \frac{mv^2}{2q} =  \\
    &= 200\,\text{В} + \frac{9{,}1 \cdot 10^{-31}\,\text{кг} \cdot \sqr{ 10000000\,\frac{\text{м}}{\text{с}} }}{2  \cdot 1{,}6 \cdot 10^{-19}\,\text{Кл}} \approx 484{,}4\,\text{В}.
    \end{align*}
}
\solutionspace{120pt}

\tasknumber{2}%
\task{%
    Три одинаковых положительных точечных заряда по $Q$ каждый находятся
    на одной прямой так, что расстояние между каждыми двумя соседними равно $3l$.
    Какую минимальную работу необходимо совершить, чтобы перевести эти заряды в положение,
    при котором они образуют равносторонний треугольник со стороной $l$? Сделайте рисунки и получите ответ (формулой).
}
\solutionspace{120pt}

\tasknumber{3}%
\task{%
    На рисунке показано расположение трёх металлических пластин и указаны их потенциалы.
    Размеры пластин кораздо больше расстояния между ними.
    Отмечены также ось и начало координат.
    Дорисуйте на рисунке электрическое поле и постройте графики зависимости от координаты $x$:
    \begin{enumerate}
        \item проекции напряжённости электрического поля,
        \item потенциала электрического поля.
    \end{enumerate}
    \begin{tikzpicture}
        \draw[-{Latex}] (0, 0) -- (0, 3.5) node[below right] {$x$};
        \draw[thick]
            (-0.05, 0.5) -- (0.05, 0.5)     (0, 0.5) node[left] {$-2\,\text{см}$}     (0.5, 0.5) -- (4, 0.5) node[right] {$-90\,\text{В}$}
            (-0.05, 1.5) -- (0.05, 1.5)     (0, 1.5) node[left] {$0$}         (0.5, 1.5) -- (4, 1.5) node[right] {$0\,\text{В}$}
            (-0.05, 2.5) -- (0.05, 2.5)     (0, 2.5) node[left] {$2\,\text{см}$}    (0.5, 2.5) -- (4, 2.5) node[right] {$120\,\text{В}$};
    \end{tikzpicture}
}
\solutionspace{90pt}

\tasknumber{4}%
\task{%
    \begin{enumerate}
        \item Запишите закон Кулона (в диэлектрике).
        \item Из теоремы Гаусса выведите (нужен рисунок, применение и результат) формулу для напряженности электростатического поля около равномерно заряженной бесконечной плоскости.
        \item Зарисуйте электрическое поле точечного отрицательного электрического заряда.
        \item Запишите формулу для вычисления напряжённости электрического поля точечного заряда в диэлектрике.
        \item Запишите принцип суперпозиции (правило сложения) напряжённостей.
    \end{enumerate}
}

\variantsplitter

\addpersonalvariant{Наталья Кравченко}

\tasknumber{1}%
\task{%
    Электрон $e^-$ вылетает из точки, потенциал которой $\varphi = 1000\,\text{В}$,
    со скоростью $v = 6000000\,\frac{\text{м}}{\text{с}}$ параллельно линиям напряжённости однородного электрического поля.
    % Будет поле его ускорять или тормозить?
    В некоторой точке частица остановилась.
    Каков потенциал этой точки?
    Вдоль и против поля влетела изначально частица?
}
\answer{%
    \begin{align*}
    A_\text{внешних сил} &= \Delta E_\text{кин.} \implies A_\text{эл.
    поля} = 0 - \frac{mv^2}2.
    \\
    A_\text{эл.
    поля} &= q(\varphi_1 - \varphi_2) \implies\varphi_2 = \varphi_1 - \frac{A_\text{эл.
    поля}}q = \varphi_1 - \frac{- \frac{mv^2}2}q = \varphi_1 + \frac{mv^2}{2q} =  \\
    &= 1000\,\text{В} + \frac{9{,}1 \cdot 10^{-31}\,\text{кг} \cdot \sqr{ 6000000\,\frac{\text{м}}{\text{с}} }}{2  * (-1)  \cdot 1{,}6 \cdot 10^{-19}\,\text{Кл}} \approx 897{,}6\,\text{В}.
    \end{align*}
}
\solutionspace{120pt}

\tasknumber{2}%
\task{%
    Три одинаковых положительных точечных заряда по $Q$ каждый находятся
    на одной прямой так, что расстояние между каждыми двумя соседними равно $2l$.
    Какую минимальную работу необходимо совершить, чтобы перевести эти заряды в положение,
    при котором они образуют прямоугольный равнобедренный треугольник с гипотенузой $l$? Сделайте рисунки и получите ответ (формулой).
}
\solutionspace{120pt}

\tasknumber{3}%
\task{%
    На рисунке показано расположение трёх металлических пластин и указаны их потенциалы.
    Размеры пластин кораздо больше расстояния между ними.
    Отмечены также ось и начало координат.
    Дорисуйте на рисунке электрическое поле и постройте графики зависимости от координаты $x$:
    \begin{enumerate}
        \item проекции напряжённости электрического поля,
        \item потенциала электрического поля.
    \end{enumerate}
    \begin{tikzpicture}
        \draw[-{Latex}] (0, 0) -- (0, 3.5) node[below right] {$x$};
        \draw[thick]
            (-0.05, 0.5) -- (0.05, 0.5)     (0, 0.5) node[left] {$-2\,\text{см}$}     (0.5, 0.5) -- (4, 0.5) node[right] {$-90\,\text{В}$}
            (-0.05, 1.5) -- (0.05, 1.5)     (0, 1.5) node[left] {$0$}         (0.5, 1.5) -- (4, 1.5) node[right] {$0\,\text{В}$}
            (-0.05, 2.5) -- (0.05, 2.5)     (0, 2.5) node[left] {$2\,\text{см}$}    (0.5, 2.5) -- (4, 2.5) node[right] {$-60\,\text{В}$};
    \end{tikzpicture}
}
\solutionspace{90pt}

\tasknumber{4}%
\task{%
    \begin{enumerate}
        \item Запишите закон сохранения электрического заряда.
        \item Из теоремы Гаусса выведите (нужен рисунок, применение и результат) формулу для напряженности электростатического поля снаружи равномерно заряженной сферы.
        \item Зарисуйте электрическое поле точечного положительного электрического заряда.
        \item Запишите формулу для вычисления напряжённости электрического поля точечного заряда в диэлектрике.
        \item Запишите принцип суперпозиции (правило сложения) напряжённостей.
    \end{enumerate}
}

\variantsplitter

\addpersonalvariant{Матвей Кузьмин}

\tasknumber{1}%
\task{%
    Позитрон $e^+$ вылетает из точки, потенциал которой $\varphi = 200\,\text{В}$,
    со скоростью $v = 12000000\,\frac{\text{м}}{\text{с}}$ параллельно линиям напряжённости однородного электрического поля.
    % Будет поле его ускорять или тормозить?
    В некоторой точке частица остановилась.
    Каков потенциал этой точки?
    Вдоль и против поля влетела изначально частица?
}
\answer{%
    \begin{align*}
    A_\text{внешних сил} &= \Delta E_\text{кин.} \implies A_\text{эл.
    поля} = 0 - \frac{mv^2}2.
    \\
    A_\text{эл.
    поля} &= q(\varphi_1 - \varphi_2) \implies\varphi_2 = \varphi_1 - \frac{A_\text{эл.
    поля}}q = \varphi_1 - \frac{- \frac{mv^2}2}q = \varphi_1 + \frac{mv^2}{2q} =  \\
    &= 200\,\text{В} + \frac{9{,}1 \cdot 10^{-31}\,\text{кг} \cdot \sqr{ 12000000\,\frac{\text{м}}{\text{с}} }}{2  \cdot 1{,}6 \cdot 10^{-19}\,\text{Кл}} \approx 609{,}5\,\text{В}.
    \end{align*}
}
\solutionspace{120pt}

\tasknumber{2}%
\task{%
    Три одинаковых положительных точечных заряда по $q$ каждый находятся
    на одной прямой так, что расстояние между каждыми двумя соседними равно $2d$.
    Какую минимальную работу необходимо совершить, чтобы перевести эти заряды в положение,
    при котором они образуют прямоугольный равнобедренный треугольник с катетом $d$? Сделайте рисунки и получите ответ (формулой).
}
\solutionspace{120pt}

\tasknumber{3}%
\task{%
    На рисунке показано расположение трёх металлических пластин и указаны их потенциалы.
    Размеры пластин кораздо больше расстояния между ними.
    Отмечены также ось и начало координат.
    Дорисуйте на рисунке электрическое поле и постройте графики зависимости от координаты $x$:
    \begin{enumerate}
        \item проекции напряжённости электрического поля,
        \item потенциала электрического поля.
    \end{enumerate}
    \begin{tikzpicture}
        \draw[-{Latex}] (0, 0) -- (0, 3.5) node[below right] {$x$};
        \draw[thick]
            (-0.05, 0.5) -- (0.05, 0.5)     (0, 0.5) node[left] {$-2\,\text{см}$}     (0.5, 0.5) -- (4, 0.5) node[right] {$90\,\text{В}$}
            (-0.05, 1.5) -- (0.05, 1.5)     (0, 1.5) node[left] {$0$}         (0.5, 1.5) -- (4, 1.5) node[right] {$0\,\text{В}$}
            (-0.05, 2.5) -- (0.05, 2.5)     (0, 2.5) node[left] {$2\,\text{см}$}    (0.5, 2.5) -- (4, 2.5) node[right] {$60\,\text{В}$};
    \end{tikzpicture}
}
\solutionspace{90pt}

\tasknumber{4}%
\task{%
    \begin{enumerate}
        \item Запишите закон сохранения электрического заряда.
        \item Из теоремы Гаусса выведите (нужен рисунок, применение и результат) формулу для напряженности электростатического поля внутри равномерно заряженной сферы.
        \item Зарисуйте электрическое поле точечного положительного электрического заряда.
        \item Запишите формулу для вычисления напряжённости электрического поля точечного заряда в диэлектрике.
        \item Запишите принцип суперпозиции (правило сложения) потенциалов.
    \end{enumerate}
}

\variantsplitter

\addpersonalvariant{Сергей Малышев}

\tasknumber{1}%
\task{%
    Электрон $e^-$ вылетает из точки, потенциал которой $\varphi = 800\,\text{В}$,
    со скоростью $v = 6000000\,\frac{\text{м}}{\text{с}}$ параллельно линиям напряжённости однородного электрического поля.
    % Будет поле его ускорять или тормозить?
    В некоторой точке частица остановилась.
    Каков потенциал этой точки?
    Вдоль и против поля влетела изначально частица?
}
\answer{%
    \begin{align*}
    A_\text{внешних сил} &= \Delta E_\text{кин.} \implies A_\text{эл.
    поля} = 0 - \frac{mv^2}2.
    \\
    A_\text{эл.
    поля} &= q(\varphi_1 - \varphi_2) \implies\varphi_2 = \varphi_1 - \frac{A_\text{эл.
    поля}}q = \varphi_1 - \frac{- \frac{mv^2}2}q = \varphi_1 + \frac{mv^2}{2q} =  \\
    &= 800\,\text{В} + \frac{9{,}1 \cdot 10^{-31}\,\text{кг} \cdot \sqr{ 6000000\,\frac{\text{м}}{\text{с}} }}{2  * (-1)  \cdot 1{,}6 \cdot 10^{-19}\,\text{Кл}} \approx 697{,}6\,\text{В}.
    \end{align*}
}
\solutionspace{120pt}

\tasknumber{2}%
\task{%
    Три одинаковых положительных точечных заряда по $Q$ каждый находятся
    на одной прямой так, что расстояние между каждыми двумя соседними равно $2d$.
    Какую минимальную работу необходимо совершить, чтобы перевести эти заряды в положение,
    при котором они образуют прямоугольный равнобедренный треугольник с катетом $d$? Сделайте рисунки и получите ответ (формулой).
}
\solutionspace{120pt}

\tasknumber{3}%
\task{%
    На рисунке показано расположение трёх металлических пластин и указаны их потенциалы.
    Размеры пластин кораздо больше расстояния между ними.
    Отмечены также ось и начало координат.
    Дорисуйте на рисунке электрическое поле и постройте графики зависимости от координаты $x$:
    \begin{enumerate}
        \item проекции напряжённости электрического поля,
        \item потенциала электрического поля.
    \end{enumerate}
    \begin{tikzpicture}
        \draw[-{Latex}] (0, 0) -- (0, 3.5) node[below right] {$x$};
        \draw[thick]
            (-0.05, 0.5) -- (0.05, 0.5)     (0, 0.5) node[left] {$-2\,\text{см}$}     (0.5, 0.5) -- (4, 0.5) node[right] {$90\,\text{В}$}
            (-0.05, 1.5) -- (0.05, 1.5)     (0, 1.5) node[left] {$0$}         (0.5, 1.5) -- (4, 1.5) node[right] {$0\,\text{В}$}
            (-0.05, 2.5) -- (0.05, 2.5)     (0, 2.5) node[left] {$2\,\text{см}$}    (0.5, 2.5) -- (4, 2.5) node[right] {$120\,\text{В}$};
    \end{tikzpicture}
}
\solutionspace{90pt}

\tasknumber{4}%
\task{%
    \begin{enumerate}
        \item Запишите закон Кулона (в диэлектрике).
        \item Из теоремы Гаусса выведите (нужен рисунок, применение и результат) формулу для напряженности электростатического поля внутри равномерно заряженной сферы.
        \item Зарисуйте электрическое поле точечного положительного электрического заряда.
        \item Запишите формулу для вычисления напряжённости электрического поля точечного заряда в диэлектрике.
        \item Запишите принцип суперпозиции (правило сложения) напряжённостей.
    \end{enumerate}
}

\variantsplitter

\addpersonalvariant{Алина Полканова}

\tasknumber{1}%
\task{%
    Позитрон $e^+$ вылетает из точки, потенциал которой $\varphi = 1000\,\text{В}$,
    со скоростью $v = 3000000\,\frac{\text{м}}{\text{с}}$ параллельно линиям напряжённости однородного электрического поля.
    % Будет поле его ускорять или тормозить?
    В некоторой точке частица остановилась.
    Каков потенциал этой точки?
    Вдоль и против поля влетела изначально частица?
}
\answer{%
    \begin{align*}
    A_\text{внешних сил} &= \Delta E_\text{кин.} \implies A_\text{эл.
    поля} = 0 - \frac{mv^2}2.
    \\
    A_\text{эл.
    поля} &= q(\varphi_1 - \varphi_2) \implies\varphi_2 = \varphi_1 - \frac{A_\text{эл.
    поля}}q = \varphi_1 - \frac{- \frac{mv^2}2}q = \varphi_1 + \frac{mv^2}{2q} =  \\
    &= 1000\,\text{В} + \frac{9{,}1 \cdot 10^{-31}\,\text{кг} \cdot \sqr{ 3000000\,\frac{\text{м}}{\text{с}} }}{2  \cdot 1{,}6 \cdot 10^{-19}\,\text{Кл}} \approx 1025{,}6\,\text{В}.
    \end{align*}
}
\solutionspace{120pt}

\tasknumber{2}%
\task{%
    Три одинаковых положительных точечных заряда по $Q$ каждый находятся
    на одной прямой так, что расстояние между каждыми двумя соседними равно $2d$.
    Какую минимальную работу необходимо совершить, чтобы перевести эти заряды в положение,
    при котором они образуют прямоугольный равнобедренный треугольник с катетом $d$? Сделайте рисунки и получите ответ (формулой).
}
\solutionspace{120pt}

\tasknumber{3}%
\task{%
    На рисунке показано расположение трёх металлических пластин и указаны их потенциалы.
    Размеры пластин кораздо больше расстояния между ними.
    Отмечены также ось и начало координат.
    Дорисуйте на рисунке электрическое поле и постройте графики зависимости от координаты $x$:
    \begin{enumerate}
        \item проекции напряжённости электрического поля,
        \item потенциала электрического поля.
    \end{enumerate}
    \begin{tikzpicture}
        \draw[-{Latex}] (0, 0) -- (0, 3.5) node[below right] {$x$};
        \draw[thick]
            (-0.05, 0.5) -- (0.05, 0.5)     (0, 0.5) node[left] {$-3\,\text{см}$}     (0.5, 0.5) -- (4, 0.5) node[right] {$-90\,\text{В}$}
            (-0.05, 1.5) -- (0.05, 1.5)     (0, 1.5) node[left] {$0$}         (0.5, 1.5) -- (4, 1.5) node[right] {$0\,\text{В}$}
            (-0.05, 2.5) -- (0.05, 2.5)     (0, 2.5) node[left] {$3\,\text{см}$}    (0.5, 2.5) -- (4, 2.5) node[right] {$-60\,\text{В}$};
    \end{tikzpicture}
}
\solutionspace{90pt}

\tasknumber{4}%
\task{%
    \begin{enumerate}
        \item Запишите теорему Гаусса.
        \item Из теоремы Гаусса выведите (нужен рисунок, применение и результат) формулу для напряженности электростатического поля внутри равномерно заряженной сферы.
        \item Зарисуйте электрическое поле точечного положительного электрического заряда.
        \item Запишите формулу для вычисления потенциала электрического поля точечного заряда в диэлектрике.
        \item Запишите принцип суперпозиции (правило сложения) напряжённостей.
    \end{enumerate}
}

\variantsplitter

\addpersonalvariant{Сергей Пономарёв}

\tasknumber{1}%
\task{%
    Электрон $e^-$ вылетает из точки, потенциал которой $\varphi = 400\,\text{В}$,
    со скоростью $v = 6000000\,\frac{\text{м}}{\text{с}}$ параллельно линиям напряжённости однородного электрического поля.
    % Будет поле его ускорять или тормозить?
    В некоторой точке частица остановилась.
    Каков потенциал этой точки?
    Вдоль и против поля влетела изначально частица?
}
\answer{%
    \begin{align*}
    A_\text{внешних сил} &= \Delta E_\text{кин.} \implies A_\text{эл.
    поля} = 0 - \frac{mv^2}2.
    \\
    A_\text{эл.
    поля} &= q(\varphi_1 - \varphi_2) \implies\varphi_2 = \varphi_1 - \frac{A_\text{эл.
    поля}}q = \varphi_1 - \frac{- \frac{mv^2}2}q = \varphi_1 + \frac{mv^2}{2q} =  \\
    &= 400\,\text{В} + \frac{9{,}1 \cdot 10^{-31}\,\text{кг} \cdot \sqr{ 6000000\,\frac{\text{м}}{\text{с}} }}{2  * (-1)  \cdot 1{,}6 \cdot 10^{-19}\,\text{Кл}} \approx 297{,}6\,\text{В}.
    \end{align*}
}
\solutionspace{120pt}

\tasknumber{2}%
\task{%
    Три одинаковых положительных точечных заряда по $q$ каждый находятся
    на одной прямой так, что расстояние между каждыми двумя соседними равно $3l$.
    Какую минимальную работу необходимо совершить, чтобы перевести эти заряды в положение,
    при котором они образуют равносторонний треугольник со стороной $l$? Сделайте рисунки и получите ответ (формулой).
}
\solutionspace{120pt}

\tasknumber{3}%
\task{%
    На рисунке показано расположение трёх металлических пластин и указаны их потенциалы.
    Размеры пластин кораздо больше расстояния между ними.
    Отмечены также ось и начало координат.
    Дорисуйте на рисунке электрическое поле и постройте графики зависимости от координаты $x$:
    \begin{enumerate}
        \item проекции напряжённости электрического поля,
        \item потенциала электрического поля.
    \end{enumerate}
    \begin{tikzpicture}
        \draw[-{Latex}] (0, 0) -- (0, 3.5) node[below right] {$x$};
        \draw[thick]
            (-0.05, 0.5) -- (0.05, 0.5)     (0, 0.5) node[left] {$-3\,\text{см}$}     (0.5, 0.5) -- (4, 0.5) node[right] {$150\,\text{В}$}
            (-0.05, 1.5) -- (0.05, 1.5)     (0, 1.5) node[left] {$0$}         (0.5, 1.5) -- (4, 1.5) node[right] {$0\,\text{В}$}
            (-0.05, 2.5) -- (0.05, 2.5)     (0, 2.5) node[left] {$3\,\text{см}$}    (0.5, 2.5) -- (4, 2.5) node[right] {$-120\,\text{В}$};
    \end{tikzpicture}
}
\solutionspace{90pt}

\tasknumber{4}%
\task{%
    \begin{enumerate}
        \item Запишите закон сохранения электрического заряда.
        \item Из теоремы Гаусса выведите (нужен рисунок, применение и результат) формулу для напряженности электростатического поля около равномерно заряженной бесконечной плоскости.
        \item Зарисуйте электрическое поле точечного положительного электрического заряда.
        \item Запишите формулу для вычисления потенциала электрического поля точечного заряда в диэлектрике.
        \item Запишите принцип суперпозиции (правило сложения) потенциалов.
    \end{enumerate}
}

\variantsplitter

\addpersonalvariant{Егор Свистушкин}

\tasknumber{1}%
\task{%
    Позитрон $e^+$ вылетает из точки, потенциал которой $\varphi = 400\,\text{В}$,
    со скоростью $v = 12000000\,\frac{\text{м}}{\text{с}}$ параллельно линиям напряжённости однородного электрического поля.
    % Будет поле его ускорять или тормозить?
    В некоторой точке частица остановилась.
    Каков потенциал этой точки?
    Вдоль и против поля влетела изначально частица?
}
\answer{%
    \begin{align*}
    A_\text{внешних сил} &= \Delta E_\text{кин.} \implies A_\text{эл.
    поля} = 0 - \frac{mv^2}2.
    \\
    A_\text{эл.
    поля} &= q(\varphi_1 - \varphi_2) \implies\varphi_2 = \varphi_1 - \frac{A_\text{эл.
    поля}}q = \varphi_1 - \frac{- \frac{mv^2}2}q = \varphi_1 + \frac{mv^2}{2q} =  \\
    &= 400\,\text{В} + \frac{9{,}1 \cdot 10^{-31}\,\text{кг} \cdot \sqr{ 12000000\,\frac{\text{м}}{\text{с}} }}{2  \cdot 1{,}6 \cdot 10^{-19}\,\text{Кл}} \approx 809{,}5\,\text{В}.
    \end{align*}
}
\solutionspace{120pt}

\tasknumber{2}%
\task{%
    Три одинаковых положительных точечных заряда по $Q$ каждый находятся
    на одной прямой так, что расстояние между каждыми двумя соседними равно $3d$.
    Какую минимальную работу необходимо совершить, чтобы перевести эти заряды в положение,
    при котором они образуют прямоугольный равнобедренный треугольник с гипотенузой $d$? Сделайте рисунки и получите ответ (формулой).
}
\solutionspace{120pt}

\tasknumber{3}%
\task{%
    На рисунке показано расположение трёх металлических пластин и указаны их потенциалы.
    Размеры пластин кораздо больше расстояния между ними.
    Отмечены также ось и начало координат.
    Дорисуйте на рисунке электрическое поле и постройте графики зависимости от координаты $x$:
    \begin{enumerate}
        \item проекции напряжённости электрического поля,
        \item потенциала электрического поля.
    \end{enumerate}
    \begin{tikzpicture}
        \draw[-{Latex}] (0, 0) -- (0, 3.5) node[below right] {$x$};
        \draw[thick]
            (-0.05, 0.5) -- (0.05, 0.5)     (0, 0.5) node[left] {$-3\,\text{см}$}     (0.5, 0.5) -- (4, 0.5) node[right] {$90\,\text{В}$}
            (-0.05, 1.5) -- (0.05, 1.5)     (0, 1.5) node[left] {$0$}         (0.5, 1.5) -- (4, 1.5) node[right] {$0\,\text{В}$}
            (-0.05, 2.5) -- (0.05, 2.5)     (0, 2.5) node[left] {$3\,\text{см}$}    (0.5, 2.5) -- (4, 2.5) node[right] {$60\,\text{В}$};
    \end{tikzpicture}
}
\solutionspace{90pt}

\tasknumber{4}%
\task{%
    \begin{enumerate}
        \item Запишите теорему Гаусса.
        \item Из теоремы Гаусса выведите (нужен рисунок, применение и результат) формулу для напряженности электростатического поля снаружи равномерно заряженной сферы.
        \item Зарисуйте электрическое поле точечного положительного электрического заряда.
        \item Запишите формулу для вычисления напряжённости электрического поля точечного заряда в диэлектрике.
        \item Запишите принцип суперпозиции (правило сложения) потенциалов.
    \end{enumerate}
}

\variantsplitter

\addpersonalvariant{Дмитрий Соколов}

\tasknumber{1}%
\task{%
    Позитрон $e^+$ вылетает из точки, потенциал которой $\varphi = 400\,\text{В}$,
    со скоростью $v = 10000000\,\frac{\text{м}}{\text{с}}$ параллельно линиям напряжённости однородного электрического поля.
    % Будет поле его ускорять или тормозить?
    В некоторой точке частица остановилась.
    Каков потенциал этой точки?
    Вдоль и против поля влетела изначально частица?
}
\answer{%
    \begin{align*}
    A_\text{внешних сил} &= \Delta E_\text{кин.} \implies A_\text{эл.
    поля} = 0 - \frac{mv^2}2.
    \\
    A_\text{эл.
    поля} &= q(\varphi_1 - \varphi_2) \implies\varphi_2 = \varphi_1 - \frac{A_\text{эл.
    поля}}q = \varphi_1 - \frac{- \frac{mv^2}2}q = \varphi_1 + \frac{mv^2}{2q} =  \\
    &= 400\,\text{В} + \frac{9{,}1 \cdot 10^{-31}\,\text{кг} \cdot \sqr{ 10000000\,\frac{\text{м}}{\text{с}} }}{2  \cdot 1{,}6 \cdot 10^{-19}\,\text{Кл}} \approx 684{,}4\,\text{В}.
    \end{align*}
}
\solutionspace{120pt}

\tasknumber{2}%
\task{%
    Три одинаковых положительных точечных заряда по $q$ каждый находятся
    на одной прямой так, что расстояние между каждыми двумя соседними равно $3d$.
    Какую минимальную работу необходимо совершить, чтобы перевести эти заряды в положение,
    при котором они образуют прямоугольный равнобедренный треугольник с гипотенузой $d$? Сделайте рисунки и получите ответ (формулой).
}
\solutionspace{120pt}

\tasknumber{3}%
\task{%
    На рисунке показано расположение трёх металлических пластин и указаны их потенциалы.
    Размеры пластин кораздо больше расстояния между ними.
    Отмечены также ось и начало координат.
    Дорисуйте на рисунке электрическое поле и постройте графики зависимости от координаты $x$:
    \begin{enumerate}
        \item проекции напряжённости электрического поля,
        \item потенциала электрического поля.
    \end{enumerate}
    \begin{tikzpicture}
        \draw[-{Latex}] (0, 0) -- (0, 3.5) node[below right] {$x$};
        \draw[thick]
            (-0.05, 0.5) -- (0.05, 0.5)     (0, 0.5) node[left] {$-3\,\text{см}$}     (0.5, 0.5) -- (4, 0.5) node[right] {$-30\,\text{В}$}
            (-0.05, 1.5) -- (0.05, 1.5)     (0, 1.5) node[left] {$0$}         (0.5, 1.5) -- (4, 1.5) node[right] {$0\,\text{В}$}
            (-0.05, 2.5) -- (0.05, 2.5)     (0, 2.5) node[left] {$3\,\text{см}$}    (0.5, 2.5) -- (4, 2.5) node[right] {$-60\,\text{В}$};
    \end{tikzpicture}
}
\solutionspace{90pt}

\tasknumber{4}%
\task{%
    \begin{enumerate}
        \item Запишите теорему Гаусса.
        \item Из теоремы Гаусса выведите (нужен рисунок, применение и результат) формулу для напряженности электростатического поля снаружи равномерно заряженной сферы.
        \item Зарисуйте электрическое поле точечного положительного электрического заряда.
        \item Запишите формулу для вычисления потенциала электрического поля точечного заряда в диэлектрике.
        \item Запишите принцип суперпозиции (правило сложения) потенциалов.
    \end{enumerate}
}

\variantsplitter

\addpersonalvariant{Арсений Трофимов}

\tasknumber{1}%
\task{%
    Позитрон $e^+$ вылетает из точки, потенциал которой $\varphi = 200\,\text{В}$,
    со скоростью $v = 10000000\,\frac{\text{м}}{\text{с}}$ параллельно линиям напряжённости однородного электрического поля.
    % Будет поле его ускорять или тормозить?
    В некоторой точке частица остановилась.
    Каков потенциал этой точки?
    Вдоль и против поля влетела изначально частица?
}
\answer{%
    \begin{align*}
    A_\text{внешних сил} &= \Delta E_\text{кин.} \implies A_\text{эл.
    поля} = 0 - \frac{mv^2}2.
    \\
    A_\text{эл.
    поля} &= q(\varphi_1 - \varphi_2) \implies\varphi_2 = \varphi_1 - \frac{A_\text{эл.
    поля}}q = \varphi_1 - \frac{- \frac{mv^2}2}q = \varphi_1 + \frac{mv^2}{2q} =  \\
    &= 200\,\text{В} + \frac{9{,}1 \cdot 10^{-31}\,\text{кг} \cdot \sqr{ 10000000\,\frac{\text{м}}{\text{с}} }}{2  \cdot 1{,}6 \cdot 10^{-19}\,\text{Кл}} \approx 484{,}4\,\text{В}.
    \end{align*}
}
\solutionspace{120pt}

\tasknumber{2}%
\task{%
    Три одинаковых положительных точечных заряда по $Q$ каждый находятся
    на одной прямой так, что расстояние между каждыми двумя соседними равно $2l$.
    Какую минимальную работу необходимо совершить, чтобы перевести эти заряды в положение,
    при котором они образуют прямоугольный равнобедренный треугольник с гипотенузой $l$? Сделайте рисунки и получите ответ (формулой).
}
\solutionspace{120pt}

\tasknumber{3}%
\task{%
    На рисунке показано расположение трёх металлических пластин и указаны их потенциалы.
    Размеры пластин кораздо больше расстояния между ними.
    Отмечены также ось и начало координат.
    Дорисуйте на рисунке электрическое поле и постройте графики зависимости от координаты $x$:
    \begin{enumerate}
        \item проекции напряжённости электрического поля,
        \item потенциала электрического поля.
    \end{enumerate}
    \begin{tikzpicture}
        \draw[-{Latex}] (0, 0) -- (0, 3.5) node[below right] {$x$};
        \draw[thick]
            (-0.05, 0.5) -- (0.05, 0.5)     (0, 0.5) node[left] {$-2\,\text{см}$}     (0.5, 0.5) -- (4, 0.5) node[right] {$30\,\text{В}$}
            (-0.05, 1.5) -- (0.05, 1.5)     (0, 1.5) node[left] {$0$}         (0.5, 1.5) -- (4, 1.5) node[right] {$0\,\text{В}$}
            (-0.05, 2.5) -- (0.05, 2.5)     (0, 2.5) node[left] {$2\,\text{см}$}    (0.5, 2.5) -- (4, 2.5) node[right] {$-120\,\text{В}$};
    \end{tikzpicture}
}
\solutionspace{90pt}

\tasknumber{4}%
\task{%
    \begin{enumerate}
        \item Запишите закон сохранения электрического заряда.
        \item Из теоремы Гаусса выведите (нужен рисунок, применение и результат) формулу для напряженности электростатического поля снаружи равномерно заряженной сферы.
        \item Зарисуйте электрическое поле точечного отрицательного электрического заряда.
        \item Запишите формулу для вычисления потенциала электрического поля точечного заряда в диэлектрике.
        \item Запишите принцип суперпозиции (правило сложения) напряжённостей.
    \end{enumerate}
}
% autogenerated
