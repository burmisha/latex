\setdate{18~мая~2021}
\setclass{10«АБ»}

\addpersonalvariant{Михаил Бурмистров}

\tasknumber{1}%
\task{%
    Электрон летит прямолинейно из точки $A$ в точку $B$, за ним при этом наблюдает экспериментатор Глюк.
    Глюк заметил, что первую четверть пути электрон равномерно двигался со скоростью $2 \cdot 10^{5}\,\frac{\text{км}}{\text{ч}}$,
    затем его практически мгновенно ускорило электрическое поле
    и остаток пути электрон вновь равномерно двигался со скоростью $3 \cdot 10^{5}\,\frac{\text{км}}{\text{ч}}$.
    Определите среднюю скорость электрона.
    Ответ выразите в м/с и округлите до тысяч.
}
\solutionspace{120pt}

\tasknumber{2}%
\task{%
    Саша стартует на велосипеде и в течение $4\,\text{c}$ двигается с постоянным ускорением $0{,}5\,\frac{\text{м}}{\text{с}^{2}}$.
    Определите
    \begin{itemize}
        \item какую скорость при этом удастся достичь,
        \item какой путь за это время будет пройден,
        \item среднюю скорость за всё время движения, если после начального ускорения продолжить движение равномерно ещё в течение времени $nt$
    \end{itemize}
}
\solutionspace{120pt}

\tasknumber{3}%
\task{%
    Какой путь тело пройдёт за вторую секунду после начала свободного падения?
    Какую скорость в начале этой секунды оно имеет?
}
\solutionspace{120pt}

\tasknumber{4}%
\task{%
    Карусель диаметром $3\,\text{v}$ равномерно совершает 6 оборотов в минуту.
    Определите
    \begin{itemize}
        \item период и частоту её обращения,
        \item скорость и ускорение крайних её точек.
    \end{itemize}
}
\solutionspace{120pt}

\tasknumber{5}%
\task{%
    Паша стоит на обрыве над рекой и методично и строго горизонтально кидает в неё камушки.
    За этим всем наблюдает экспериментатор Глюк, который уже выяснил, что камушки падают в реку спустя $1{,}6\,\text{с}$ после броска,
    а вот дальность полёта оценить сложнее: придётся лезть в воду.
    Выручите Глюка и определите:
    \begin{itemize}
        \item высоту обрыва (вместе с ростом Паши).
        \item дальность полёта камушков (по горизонтали) и их скорость при падении, приняв начальную скорость броска равной $v = 18\,\frac{\text{м}}{\text{с}}$.
    \end{itemize}
    Сопротивлением воздуха пренебречь, $g = 10\,\frac{\text{м}}{\text{с}^{2}}$.
}
\solutionspace{120pt}

\tasknumber{6}%
\task{%
    Четыре одинаковых брусков лежат на гладком горизонтальном столе.
    Масса каждого бруска равна $3\,\text{кг}$,
    причём они пронумерованы от 1 до 4 и последовательно связаны между собой невесомыми
    нерастяжимыми нитями: 1 со 2, 2 с 3 (ну и с 1) и т.д.
    Экспериментатор Глюк прикладывает постоянную горизонтальную силу $90\,\text{Н}$ к бруску с наибольшим номером.
    С каким ускорением двигается система? Чему равна сила натяжения нити, связывающей бруски 1 и 2?
}
\solutionspace{120pt}

\tasknumber{7}%
\task{%
    Два бруска связаны лёгкой нерастяжимой нитью и перекинуты через неподвижный блок (см.
    рис.).
    Определите силу натяжения нити и ускорения брусков.
    Силами трения пренебречь, массы брусков
    равны $m_1 = 5\,\text{кг}$ и $m_2 = 14\,\text{кг}$.

    \begin{tikzpicture}[x=1.5cm,y=1.5cm,thick]
        \draw
            (-0.4, 0) rectangle (-0.2, 1.2)
            (0.15, 0.5) rectangle (0.45, 1)
            (0, 2) circle [radius=0.3] -- ++(up:0.5)
            (-0.3, 1.2) -- ++(up:0.8)
            (0.3, 1) -- ++(up:1)
            (-0.7, 2.5) -- (0.7, 2.5)
            ;
        \draw[pattern={Lines[angle=51,distance=3pt]},pattern color=black,draw=none] (-0.7, 2.5) rectangle (0.7, 2.75);
        \node [left] (left) at (-0.4, 0.6) { $m_1$ };
        \node [right] (right) at (0.4, 0.75) { $m_2$ };
    \end{tikzpicture}
}

\variantsplitter

\addpersonalvariant{Ирина Ан}

\tasknumber{1}%
\task{%
    Электрон летит прямолинейно из точки $A$ в точку $B$, за ним при этом наблюдает экспериментатор Глюк.
    Глюк заметил, что первую треть времени электрон равномерно двигался со скоростью $2 \cdot 10^{5}\,\frac{\text{км}}{\text{ч}}$,
    затем его практически мгновенно ускорило электрическое поле
    и остаток времени электрон вновь равномерно двигался со скоростью $6 \cdot 10^{5}\,\frac{\text{км}}{\text{ч}}$.
    Определите среднюю скорость электрона.
    Ответ выразите в м/с и округлите до тысяч.
}
\solutionspace{120pt}

\tasknumber{2}%
\task{%
    Валя стартует на мотоцикле и в течение $2\,\text{c}$ двигается с постоянным ускорением $2\,\frac{\text{м}}{\text{с}^{2}}$.
    Определите
    \begin{itemize}
        \item какую скорость при этом удастся достичь,
        \item какой путь за это время будет пройден,
        \item среднюю скорость за всё время движения, если после начального ускорения продолжить движение равномерно ещё в течение времени $nt$
    \end{itemize}
}
\solutionspace{120pt}

\tasknumber{3}%
\task{%
    Какой путь тело пройдёт за пятую секунду после начала свободного падения?
    Какую скорость в конце этой секунды оно имеет?
}
\solutionspace{120pt}

\tasknumber{4}%
\task{%
    Карусель диаметром $2\,\text{v}$ равномерно совершает 10 оборотов в минуту.
    Определите
    \begin{itemize}
        \item период и частоту её обращения,
        \item скорость и ускорение крайних её точек.
    \end{itemize}
}
\solutionspace{120pt}

\tasknumber{5}%
\task{%
    Миша стоит на обрыве над рекой и методично и строго горизонтально кидает в неё камушки.
    За этим всем наблюдает экспериментатор Глюк, который уже выяснил, что камушки падают в реку спустя $1{,}2\,\text{с}$ после броска,
    а вот дальность полёта оценить сложнее: придётся лезть в воду.
    Выручите Глюка и определите:
    \begin{itemize}
        \item высоту обрыва (вместе с ростом Миши).
        \item дальность полёта камушков (по горизонтали) и их скорость при падении, приняв начальную скорость броска равной $v = 18\,\frac{\text{м}}{\text{с}}$.
    \end{itemize}
    Сопротивлением воздуха пренебречь, $g = 10\,\frac{\text{м}}{\text{с}^{2}}$.
}
\solutionspace{120pt}

\tasknumber{6}%
\task{%
    Четыре одинаковых брусков лежат на гладком горизонтальном столе.
    Масса каждого бруска равна $2\,\text{кг}$,
    причём они пронумерованы от 1 до 4 и последовательно связаны между собой невесомыми
    нерастяжимыми нитями: 1 со 2, 2 с 3 (ну и с 1) и т.д.
    Экспериментатор Глюк прикладывает постоянную горизонтальную силу $90\,\text{Н}$ к бруску с наибольшим номером.
    С каким ускорением двигается система? Чему равна сила натяжения нити, связывающей бруски 3 и 4?
}
\solutionspace{120pt}

\tasknumber{7}%
\task{%
    Два бруска связаны лёгкой нерастяжимой нитью и перекинуты через неподвижный блок (см.
    рис.).
    Определите силу натяжения нити и ускорения брусков.
    Силами трения пренебречь, массы брусков
    равны $m_1 = 5\,\text{кг}$ и $m_2 = 10\,\text{кг}$.

    \begin{tikzpicture}[x=1.5cm,y=1.5cm,thick]
        \draw
            (-0.4, 0) rectangle (-0.2, 1.2)
            (0.15, 0.5) rectangle (0.45, 1)
            (0, 2) circle [radius=0.3] -- ++(up:0.5)
            (-0.3, 1.2) -- ++(up:0.8)
            (0.3, 1) -- ++(up:1)
            (-0.7, 2.5) -- (0.7, 2.5)
            ;
        \draw[pattern={Lines[angle=51,distance=3pt]},pattern color=black,draw=none] (-0.7, 2.5) rectangle (0.7, 2.75);
        \node [left] (left) at (-0.4, 0.6) { $m_1$ };
        \node [right] (right) at (0.4, 0.75) { $m_2$ };
    \end{tikzpicture}
}

\variantsplitter

\addpersonalvariant{Софья Андрианова}

\tasknumber{1}%
\task{%
    Электрон летит прямолинейно из точки $A$ в точку $B$, за ним при этом наблюдает экспериментатор Глюк.
    Глюк заметил, что первую треть времени электрон равномерно двигался со скоростью $2 \cdot 10^{5}\,\frac{\text{км}}{\text{ч}}$,
    затем его практически мгновенно ускорило электрическое поле
    и остаток времени электрон вновь равномерно двигался со скоростью $6 \cdot 10^{5}\,\frac{\text{км}}{\text{ч}}$.
    Определите среднюю скорость электрона.
    Ответ выразите в м/с и округлите до тысяч.
}
\solutionspace{120pt}

\tasknumber{2}%
\task{%
    Саша стартует на мотоцикле и в течение $4\,\text{c}$ двигается с постоянным ускорением $2{,}5\,\frac{\text{м}}{\text{с}^{2}}$.
    Определите
    \begin{itemize}
        \item какую скорость при этом удастся достичь,
        \item какой путь за это время будет пройден,
        \item среднюю скорость за всё время движения, если после начального ускорения продолжить движение равномерно ещё в течение времени $nt$
    \end{itemize}
}
\solutionspace{120pt}

\tasknumber{3}%
\task{%
    Какой путь тело пройдёт за четвёртую секунду после начала свободного падения?
    Какую скорость в начале этой секунды оно имеет?
}
\solutionspace{120pt}

\tasknumber{4}%
\task{%
    Карусель диаметром $3\,\text{v}$ равномерно совершает 6 оборотов в минуту.
    Определите
    \begin{itemize}
        \item период и частоту её обращения,
        \item скорость и ускорение крайних её точек.
    \end{itemize}
}
\solutionspace{120pt}

\tasknumber{5}%
\task{%
    Даша стоит на обрыве над рекой и методично и строго горизонтально кидает в неё камушки.
    За этим всем наблюдает экспериментатор Глюк, который уже выяснил, что камушки падают в реку спустя $1{,}3\,\text{с}$ после броска,
    а вот дальность полёта оценить сложнее: придётся лезть в воду.
    Выручите Глюка и определите:
    \begin{itemize}
        \item высоту обрыва (вместе с ростом Даши).
        \item дальность полёта камушков (по горизонтали) и их скорость при падении, приняв начальную скорость броска равной $v = 18\,\frac{\text{м}}{\text{с}}$.
    \end{itemize}
    Сопротивлением воздуха пренебречь, $g = 10\,\frac{\text{м}}{\text{с}^{2}}$.
}
\solutionspace{120pt}

\tasknumber{6}%
\task{%
    Шесть одинаковых брусков лежат на гладком горизонтальном столе.
    Масса каждого бруска равна $3\,\text{кг}$,
    причём они пронумерованы от 1 до 6 и последовательно связаны между собой невесомыми
    нерастяжимыми нитями: 1 со 2, 2 с 3 (ну и с 1) и т.д.
    Экспериментатор Глюк прикладывает постоянную горизонтальную силу $120\,\text{Н}$ к бруску с наименьшим номером.
    С каким ускорением двигается система? Чему равна сила натяжения нити, связывающей бруски 1 и 2?
}
\solutionspace{120pt}

\tasknumber{7}%
\task{%
    Два бруска связаны лёгкой нерастяжимой нитью и перекинуты через неподвижный блок (см.
    рис.).
    Определите силу натяжения нити и ускорения брусков.
    Силами трения пренебречь, массы брусков
    равны $m_1 = 11\,\text{кг}$ и $m_2 = 10\,\text{кг}$.

    \begin{tikzpicture}[x=1.5cm,y=1.5cm,thick]
        \draw
            (-0.4, 0) rectangle (-0.2, 1.2)
            (0.15, 0.5) rectangle (0.45, 1)
            (0, 2) circle [radius=0.3] -- ++(up:0.5)
            (-0.3, 1.2) -- ++(up:0.8)
            (0.3, 1) -- ++(up:1)
            (-0.7, 2.5) -- (0.7, 2.5)
            ;
        \draw[pattern={Lines[angle=51,distance=3pt]},pattern color=black,draw=none] (-0.7, 2.5) rectangle (0.7, 2.75);
        \node [left] (left) at (-0.4, 0.6) { $m_1$ };
        \node [right] (right) at (0.4, 0.75) { $m_2$ };
    \end{tikzpicture}
}

\variantsplitter

\addpersonalvariant{Владимир Артемчук}

\tasknumber{1}%
\task{%
    Электрон летит прямолинейно из точки $A$ в точку $B$, за ним при этом наблюдает экспериментатор Глюк.
    Глюк заметил, что первую четверть пути электрон равномерно двигался со скоростью $2 \cdot 10^{5}\,\frac{\text{км}}{\text{ч}}$,
    затем его практически мгновенно ускорило электрическое поле
    и остаток пути электрон вновь равномерно двигался со скоростью $6 \cdot 10^{5}\,\frac{\text{км}}{\text{ч}}$.
    Определите среднюю скорость электрона.
    Ответ выразите в м/с и округлите до тысяч.
}
\solutionspace{120pt}

\tasknumber{2}%
\task{%
    Женя стартует на мотоцикле и в течение $3\,\text{c}$ двигается с постоянным ускорением $0{,}5\,\frac{\text{м}}{\text{с}^{2}}$.
    Определите
    \begin{itemize}
        \item какую скорость при этом удастся достичь,
        \item какой путь за это время будет пройден,
        \item среднюю скорость за всё время движения, если после начального ускорения продолжить движение равномерно ещё в течение времени $nt$
    \end{itemize}
}
\solutionspace{120pt}

\tasknumber{3}%
\task{%
    Какой путь тело пройдёт за третью секунду после начала свободного падения?
    Какую скорость в начале этой секунды оно имеет?
}
\solutionspace{120pt}

\tasknumber{4}%
\task{%
    Карусель диаметром $3\,\text{v}$ равномерно совершает 10 оборотов в минуту.
    Определите
    \begin{itemize}
        \item период и частоту её обращения,
        \item скорость и ускорение крайних её точек.
    \end{itemize}
}
\solutionspace{120pt}

\tasknumber{5}%
\task{%
    Миша стоит на обрыве над рекой и методично и строго горизонтально кидает в неё камушки.
    За этим всем наблюдает экспериментатор Глюк, который уже выяснил, что камушки падают в реку спустя $1{,}3\,\text{с}$ после броска,
    а вот дальность полёта оценить сложнее: придётся лезть в воду.
    Выручите Глюка и определите:
    \begin{itemize}
        \item высоту обрыва (вместе с ростом Миши).
        \item дальность полёта камушков (по горизонтали) и их скорость при падении, приняв начальную скорость броска равной $v = 17\,\frac{\text{м}}{\text{с}}$.
    \end{itemize}
    Сопротивлением воздуха пренебречь, $g = 10\,\frac{\text{м}}{\text{с}^{2}}$.
}
\solutionspace{120pt}

\tasknumber{6}%
\task{%
    Четыре одинаковых брусков лежат на гладком горизонтальном столе.
    Масса каждого бруска равна $2\,\text{кг}$,
    причём они пронумерованы от 1 до 4 и последовательно связаны между собой невесомыми
    нерастяжимыми нитями: 1 со 2, 2 с 3 (ну и с 1) и т.д.
    Экспериментатор Глюк прикладывает постоянную горизонтальную силу $90\,\text{Н}$ к бруску с наименьшим номером.
    С каким ускорением двигается система? Чему равна сила натяжения нити, связывающей бруски 3 и 4?
}
\solutionspace{120pt}

\tasknumber{7}%
\task{%
    Два бруска связаны лёгкой нерастяжимой нитью и перекинуты через неподвижный блок (см.
    рис.).
    Определите силу натяжения нити и ускорения брусков.
    Силами трения пренебречь, массы брусков
    равны $m_1 = 8\,\text{кг}$ и $m_2 = 4\,\text{кг}$.

    \begin{tikzpicture}[x=1.5cm,y=1.5cm,thick]
        \draw
            (-0.4, 0) rectangle (-0.2, 1.2)
            (0.15, 0.5) rectangle (0.45, 1)
            (0, 2) circle [radius=0.3] -- ++(up:0.5)
            (-0.3, 1.2) -- ++(up:0.8)
            (0.3, 1) -- ++(up:1)
            (-0.7, 2.5) -- (0.7, 2.5)
            ;
        \draw[pattern={Lines[angle=51,distance=3pt]},pattern color=black,draw=none] (-0.7, 2.5) rectangle (0.7, 2.75);
        \node [left] (left) at (-0.4, 0.6) { $m_1$ };
        \node [right] (right) at (0.4, 0.75) { $m_2$ };
    \end{tikzpicture}
}

\variantsplitter

\addpersonalvariant{Софья Белянкина}

\tasknumber{1}%
\task{%
    Электрон летит прямолинейно из точки $A$ в точку $B$, за ним при этом наблюдает экспериментатор Глюк.
    Глюк заметил, что первую четверть пути электрон равномерно двигался со скоростью $4 \cdot 10^{5}\,\frac{\text{км}}{\text{ч}}$,
    затем его практически мгновенно ускорило электрическое поле
    и остаток пути электрон вновь равномерно двигался со скоростью $6 \cdot 10^{5}\,\frac{\text{км}}{\text{ч}}$.
    Определите среднюю скорость электрона.
    Ответ выразите в м/с и округлите до тысяч.
}
\solutionspace{120pt}

\tasknumber{2}%
\task{%
    Саша стартует на мотоцикле и в течение $2\,\text{c}$ двигается с постоянным ускорением $2\,\frac{\text{м}}{\text{с}^{2}}$.
    Определите
    \begin{itemize}
        \item какую скорость при этом удастся достичь,
        \item какой путь за это время будет пройден,
        \item среднюю скорость за всё время движения, если после начального ускорения продолжить движение равномерно ещё в течение времени $nt$
    \end{itemize}
}
\solutionspace{120pt}

\tasknumber{3}%
\task{%
    Какой путь тело пройдёт за третью секунду после начала свободного падения?
    Какую скорость в начале этой секунды оно имеет?
}
\solutionspace{120pt}

\tasknumber{4}%
\task{%
    Карусель диаметром $4\,\text{v}$ равномерно совершает 10 оборотов в минуту.
    Определите
    \begin{itemize}
        \item период и частоту её обращения,
        \item скорость и ускорение крайних её точек.
    \end{itemize}
}
\solutionspace{120pt}

\tasknumber{5}%
\task{%
    Миша стоит на обрыве над рекой и методично и строго горизонтально кидает в неё камушки.
    За этим всем наблюдает экспериментатор Глюк, который уже выяснил, что камушки падают в реку спустя $1{,}5\,\text{с}$ после броска,
    а вот дальность полёта оценить сложнее: придётся лезть в воду.
    Выручите Глюка и определите:
    \begin{itemize}
        \item высоту обрыва (вместе с ростом Миши).
        \item дальность полёта камушков (по горизонтали) и их скорость при падении, приняв начальную скорость броска равной $v = 15\,\frac{\text{м}}{\text{с}}$.
    \end{itemize}
    Сопротивлением воздуха пренебречь, $g = 10\,\frac{\text{м}}{\text{с}^{2}}$.
}
\solutionspace{120pt}

\tasknumber{6}%
\task{%
    Пять одинаковых брусков лежат на гладком горизонтальном столе.
    Масса каждого бруска равна $3\,\text{кг}$,
    причём они пронумерованы от 1 до 5 и последовательно связаны между собой невесомыми
    нерастяжимыми нитями: 1 со 2, 2 с 3 (ну и с 1) и т.д.
    Экспериментатор Глюк прикладывает постоянную горизонтальную силу $90\,\text{Н}$ к бруску с наибольшим номером.
    С каким ускорением двигается система? Чему равна сила натяжения нити, связывающей бруски 2 и 3?
}
\solutionspace{120pt}

\tasknumber{7}%
\task{%
    Два бруска связаны лёгкой нерастяжимой нитью и перекинуты через неподвижный блок (см.
    рис.).
    Определите силу натяжения нити и ускорения брусков.
    Силами трения пренебречь, массы брусков
    равны $m_1 = 8\,\text{кг}$ и $m_2 = 14\,\text{кг}$.

    \begin{tikzpicture}[x=1.5cm,y=1.5cm,thick]
        \draw
            (-0.4, 0) rectangle (-0.2, 1.2)
            (0.15, 0.5) rectangle (0.45, 1)
            (0, 2) circle [radius=0.3] -- ++(up:0.5)
            (-0.3, 1.2) -- ++(up:0.8)
            (0.3, 1) -- ++(up:1)
            (-0.7, 2.5) -- (0.7, 2.5)
            ;
        \draw[pattern={Lines[angle=51,distance=3pt]},pattern color=black,draw=none] (-0.7, 2.5) rectangle (0.7, 2.75);
        \node [left] (left) at (-0.4, 0.6) { $m_1$ };
        \node [right] (right) at (0.4, 0.75) { $m_2$ };
    \end{tikzpicture}
}

\variantsplitter

\addpersonalvariant{Варвара Егиазарян}

\tasknumber{1}%
\task{%
    Электрон летит прямолинейно из точки $A$ в точку $B$, за ним при этом наблюдает экспериментатор Глюк.
    Глюк заметил, что первую четверть пути электрон равномерно двигался со скоростью $2 \cdot 10^{5}\,\frac{\text{км}}{\text{ч}}$,
    затем его практически мгновенно ускорило электрическое поле
    и остаток пути электрон вновь равномерно двигался со скоростью $3 \cdot 10^{5}\,\frac{\text{км}}{\text{ч}}$.
    Определите среднюю скорость электрона.
    Ответ выразите в м/с и округлите до тысяч.
}
\solutionspace{120pt}

\tasknumber{2}%
\task{%
    Саша стартует на мотоцикле и в течение $10\,\text{c}$ двигается с постоянным ускорением $1{,}5\,\frac{\text{м}}{\text{с}^{2}}$.
    Определите
    \begin{itemize}
        \item какую скорость при этом удастся достичь,
        \item какой путь за это время будет пройден,
        \item среднюю скорость за всё время движения, если после начального ускорения продолжить движение равномерно ещё в течение времени $nt$
    \end{itemize}
}
\solutionspace{120pt}

\tasknumber{3}%
\task{%
    Какой путь тело пройдёт за пятую секунду после начала свободного падения?
    Какую скорость в конце этой секунды оно имеет?
}
\solutionspace{120pt}

\tasknumber{4}%
\task{%
    Карусель диаметром $5\,\text{v}$ равномерно совершает 10 оборотов в минуту.
    Определите
    \begin{itemize}
        \item период и частоту её обращения,
        \item скорость и ускорение крайних её точек.
    \end{itemize}
}
\solutionspace{120pt}

\tasknumber{5}%
\task{%
    Маша стоит на обрыве над рекой и методично и строго горизонтально кидает в неё камушки.
    За этим всем наблюдает экспериментатор Глюк, который уже выяснил, что камушки падают в реку спустя $1{,}2\,\text{с}$ после броска,
    а вот дальность полёта оценить сложнее: придётся лезть в воду.
    Выручите Глюка и определите:
    \begin{itemize}
        \item высоту обрыва (вместе с ростом Маши).
        \item дальность полёта камушков (по горизонтали) и их скорость при падении, приняв начальную скорость броска равной $v = 14\,\frac{\text{м}}{\text{с}}$.
    \end{itemize}
    Сопротивлением воздуха пренебречь, $g = 10\,\frac{\text{м}}{\text{с}^{2}}$.
}
\solutionspace{120pt}

\tasknumber{6}%
\task{%
    Шесть одинаковых брусков лежат на гладком горизонтальном столе.
    Масса каждого бруска равна $2\,\text{кг}$,
    причём они пронумерованы от 1 до 6 и последовательно связаны между собой невесомыми
    нерастяжимыми нитями: 1 со 2, 2 с 3 (ну и с 1) и т.д.
    Экспериментатор Глюк прикладывает постоянную горизонтальную силу $60\,\text{Н}$ к бруску с наименьшим номером.
    С каким ускорением двигается система? Чему равна сила натяжения нити, связывающей бруски 2 и 3?
}
\solutionspace{120pt}

\tasknumber{7}%
\task{%
    Два бруска связаны лёгкой нерастяжимой нитью и перекинуты через неподвижный блок (см.
    рис.).
    Определите силу натяжения нити и ускорения брусков.
    Силами трения пренебречь, массы брусков
    равны $m_1 = 8\,\text{кг}$ и $m_2 = 6\,\text{кг}$.

    \begin{tikzpicture}[x=1.5cm,y=1.5cm,thick]
        \draw
            (-0.4, 0) rectangle (-0.2, 1.2)
            (0.15, 0.5) rectangle (0.45, 1)
            (0, 2) circle [radius=0.3] -- ++(up:0.5)
            (-0.3, 1.2) -- ++(up:0.8)
            (0.3, 1) -- ++(up:1)
            (-0.7, 2.5) -- (0.7, 2.5)
            ;
        \draw[pattern={Lines[angle=51,distance=3pt]},pattern color=black,draw=none] (-0.7, 2.5) rectangle (0.7, 2.75);
        \node [left] (left) at (-0.4, 0.6) { $m_1$ };
        \node [right] (right) at (0.4, 0.75) { $m_2$ };
    \end{tikzpicture}
}

\variantsplitter

\addpersonalvariant{Владислав Емелин}

\tasknumber{1}%
\task{%
    Электрон летит прямолинейно из точки $A$ в точку $B$, за ним при этом наблюдает экспериментатор Глюк.
    Глюк заметил, что первую четверть времени электрон равномерно двигался со скоростью $4 \cdot 10^{5}\,\frac{\text{км}}{\text{ч}}$,
    затем его практически мгновенно ускорило электрическое поле
    и остаток времени электрон вновь равномерно двигался со скоростью $3 \cdot 10^{5}\,\frac{\text{км}}{\text{ч}}$.
    Определите среднюю скорость электрона.
    Ответ выразите в м/с и округлите до тысяч.
}
\solutionspace{120pt}

\tasknumber{2}%
\task{%
    Валя стартует на велосипеде и в течение $4\,\text{c}$ двигается с постоянным ускорением $2\,\frac{\text{м}}{\text{с}^{2}}$.
    Определите
    \begin{itemize}
        \item какую скорость при этом удастся достичь,
        \item какой путь за это время будет пройден,
        \item среднюю скорость за всё время движения, если после начального ускорения продолжить движение равномерно ещё в течение времени $nt$
    \end{itemize}
}
\solutionspace{120pt}

\tasknumber{3}%
\task{%
    Какой путь тело пройдёт за четвёртую секунду после начала свободного падения?
    Какую скорость в начале этой секунды оно имеет?
}
\solutionspace{120pt}

\tasknumber{4}%
\task{%
    Карусель радиусом $2\,\text{v}$ равномерно совершает 6 оборотов в минуту.
    Определите
    \begin{itemize}
        \item период и частоту её обращения,
        \item скорость и ускорение крайних её точек.
    \end{itemize}
}
\solutionspace{120pt}

\tasknumber{5}%
\task{%
    Паша стоит на обрыве над рекой и методично и строго горизонтально кидает в неё камушки.
    За этим всем наблюдает экспериментатор Глюк, который уже выяснил, что камушки падают в реку спустя $1{,}6\,\text{с}$ после броска,
    а вот дальность полёта оценить сложнее: придётся лезть в воду.
    Выручите Глюка и определите:
    \begin{itemize}
        \item высоту обрыва (вместе с ростом Паши).
        \item дальность полёта камушков (по горизонтали) и их скорость при падении, приняв начальную скорость броска равной $v = 16\,\frac{\text{м}}{\text{с}}$.
    \end{itemize}
    Сопротивлением воздуха пренебречь, $g = 10\,\frac{\text{м}}{\text{с}^{2}}$.
}
\solutionspace{120pt}

\tasknumber{6}%
\task{%
    Четыре одинаковых брусков лежат на гладком горизонтальном столе.
    Масса каждого бруска равна $2\,\text{кг}$,
    причём они пронумерованы от 1 до 4 и последовательно связаны между собой невесомыми
    нерастяжимыми нитями: 1 со 2, 2 с 3 (ну и с 1) и т.д.
    Экспериментатор Глюк прикладывает постоянную горизонтальную силу $90\,\text{Н}$ к бруску с наибольшим номером.
    С каким ускорением двигается система? Чему равна сила натяжения нити, связывающей бруски 3 и 4?
}
\solutionspace{120pt}

\tasknumber{7}%
\task{%
    Два бруска связаны лёгкой нерастяжимой нитью и перекинуты через неподвижный блок (см.
    рис.).
    Определите силу натяжения нити и ускорения брусков.
    Силами трения пренебречь, массы брусков
    равны $m_1 = 8\,\text{кг}$ и $m_2 = 14\,\text{кг}$.

    \begin{tikzpicture}[x=1.5cm,y=1.5cm,thick]
        \draw
            (-0.4, 0) rectangle (-0.2, 1.2)
            (0.15, 0.5) rectangle (0.45, 1)
            (0, 2) circle [radius=0.3] -- ++(up:0.5)
            (-0.3, 1.2) -- ++(up:0.8)
            (0.3, 1) -- ++(up:1)
            (-0.7, 2.5) -- (0.7, 2.5)
            ;
        \draw[pattern={Lines[angle=51,distance=3pt]},pattern color=black,draw=none] (-0.7, 2.5) rectangle (0.7, 2.75);
        \node [left] (left) at (-0.4, 0.6) { $m_1$ };
        \node [right] (right) at (0.4, 0.75) { $m_2$ };
    \end{tikzpicture}
}

\variantsplitter

\addpersonalvariant{Артём Жичин}

\tasknumber{1}%
\task{%
    Электрон летит прямолинейно из точки $A$ в точку $B$, за ним при этом наблюдает экспериментатор Глюк.
    Глюк заметил, что первую четверть пути электрон равномерно двигался со скоростью $4 \cdot 10^{5}\,\frac{\text{км}}{\text{ч}}$,
    затем его практически мгновенно ускорило электрическое поле
    и остаток пути электрон вновь равномерно двигался со скоростью $6 \cdot 10^{5}\,\frac{\text{км}}{\text{ч}}$.
    Определите среднюю скорость электрона.
    Ответ выразите в м/с и округлите до тысяч.
}
\solutionspace{120pt}

\tasknumber{2}%
\task{%
    Саша стартует на лошади и в течение $10\,\text{c}$ двигается с постоянным ускорением $0{,}5\,\frac{\text{м}}{\text{с}^{2}}$.
    Определите
    \begin{itemize}
        \item какую скорость при этом удастся достичь,
        \item какой путь за это время будет пройден,
        \item среднюю скорость за всё время движения, если после начального ускорения продолжить движение равномерно ещё в течение времени $nt$
    \end{itemize}
}
\solutionspace{120pt}

\tasknumber{3}%
\task{%
    Какой путь тело пройдёт за четвёртую секунду после начала свободного падения?
    Какую скорость в начале этой секунды оно имеет?
}
\solutionspace{120pt}

\tasknumber{4}%
\task{%
    Карусель радиусом $4\,\text{v}$ равномерно совершает 6 оборотов в минуту.
    Определите
    \begin{itemize}
        \item период и частоту её обращения,
        \item скорость и ускорение крайних её точек.
    \end{itemize}
}
\solutionspace{120pt}

\tasknumber{5}%
\task{%
    Паша стоит на обрыве над рекой и методично и строго горизонтально кидает в неё камушки.
    За этим всем наблюдает экспериментатор Глюк, который уже выяснил, что камушки падают в реку спустя $1{,}3\,\text{с}$ после броска,
    а вот дальность полёта оценить сложнее: придётся лезть в воду.
    Выручите Глюка и определите:
    \begin{itemize}
        \item высоту обрыва (вместе с ростом Паши).
        \item дальность полёта камушков (по горизонтали) и их скорость при падении, приняв начальную скорость броска равной $v = 13\,\frac{\text{м}}{\text{с}}$.
    \end{itemize}
    Сопротивлением воздуха пренебречь, $g = 10\,\frac{\text{м}}{\text{с}^{2}}$.
}
\solutionspace{120pt}

\tasknumber{6}%
\task{%
    Шесть одинаковых брусков лежат на гладком горизонтальном столе.
    Масса каждого бруска равна $3\,\text{кг}$,
    причём они пронумерованы от 1 до 6 и последовательно связаны между собой невесомыми
    нерастяжимыми нитями: 1 со 2, 2 с 3 (ну и с 1) и т.д.
    Экспериментатор Глюк прикладывает постоянную горизонтальную силу $60\,\text{Н}$ к бруску с наибольшим номером.
    С каким ускорением двигается система? Чему равна сила натяжения нити, связывающей бруски 2 и 3?
}
\solutionspace{120pt}

\tasknumber{7}%
\task{%
    Два бруска связаны лёгкой нерастяжимой нитью и перекинуты через неподвижный блок (см.
    рис.).
    Определите силу натяжения нити и ускорения брусков.
    Силами трения пренебречь, массы брусков
    равны $m_1 = 11\,\text{кг}$ и $m_2 = 10\,\text{кг}$.

    \begin{tikzpicture}[x=1.5cm,y=1.5cm,thick]
        \draw
            (-0.4, 0) rectangle (-0.2, 1.2)
            (0.15, 0.5) rectangle (0.45, 1)
            (0, 2) circle [radius=0.3] -- ++(up:0.5)
            (-0.3, 1.2) -- ++(up:0.8)
            (0.3, 1) -- ++(up:1)
            (-0.7, 2.5) -- (0.7, 2.5)
            ;
        \draw[pattern={Lines[angle=51,distance=3pt]},pattern color=black,draw=none] (-0.7, 2.5) rectangle (0.7, 2.75);
        \node [left] (left) at (-0.4, 0.6) { $m_1$ };
        \node [right] (right) at (0.4, 0.75) { $m_2$ };
    \end{tikzpicture}
}

\variantsplitter

\addpersonalvariant{Дарья Кошман}

\tasknumber{1}%
\task{%
    Электрон летит прямолинейно из точки $A$ в точку $B$, за ним при этом наблюдает экспериментатор Глюк.
    Глюк заметил, что первую половину времени электрон равномерно двигался со скоростью $2 \cdot 10^{5}\,\frac{\text{км}}{\text{ч}}$,
    затем его практически мгновенно ускорило электрическое поле
    и остаток времени электрон вновь равномерно двигался со скоростью $3 \cdot 10^{5}\,\frac{\text{км}}{\text{ч}}$.
    Определите среднюю скорость электрона.
    Ответ выразите в м/с и округлите до тысяч.
}
\solutionspace{120pt}

\tasknumber{2}%
\task{%
    Саша стартует на велосипеде и в течение $5\,\text{c}$ двигается с постоянным ускорением $2{,}5\,\frac{\text{м}}{\text{с}^{2}}$.
    Определите
    \begin{itemize}
        \item какую скорость при этом удастся достичь,
        \item какой путь за это время будет пройден,
        \item среднюю скорость за всё время движения, если после начального ускорения продолжить движение равномерно ещё в течение времени $nt$
    \end{itemize}
}
\solutionspace{120pt}

\tasknumber{3}%
\task{%
    Какой путь тело пройдёт за шестую секунду после начала свободного падения?
    Какую скорость в начале этой секунды оно имеет?
}
\solutionspace{120pt}

\tasknumber{4}%
\task{%
    Карусель радиусом $4\,\text{v}$ равномерно совершает 10 оборотов в минуту.
    Определите
    \begin{itemize}
        \item период и частоту её обращения,
        \item скорость и ускорение крайних её точек.
    \end{itemize}
}
\solutionspace{120pt}

\tasknumber{5}%
\task{%
    Миша стоит на обрыве над рекой и методично и строго горизонтально кидает в неё камушки.
    За этим всем наблюдает экспериментатор Глюк, который уже выяснил, что камушки падают в реку спустя $1{,}2\,\text{с}$ после броска,
    а вот дальность полёта оценить сложнее: придётся лезть в воду.
    Выручите Глюка и определите:
    \begin{itemize}
        \item высоту обрыва (вместе с ростом Миши).
        \item дальность полёта камушков (по горизонтали) и их скорость при падении, приняв начальную скорость броска равной $v = 12\,\frac{\text{м}}{\text{с}}$.
    \end{itemize}
    Сопротивлением воздуха пренебречь, $g = 10\,\frac{\text{м}}{\text{с}^{2}}$.
}
\solutionspace{120pt}

\tasknumber{6}%
\task{%
    Четыре одинаковых брусков лежат на гладком горизонтальном столе.
    Масса каждого бруска равна $3\,\text{кг}$,
    причём они пронумерованы от 1 до 4 и последовательно связаны между собой невесомыми
    нерастяжимыми нитями: 1 со 2, 2 с 3 (ну и с 1) и т.д.
    Экспериментатор Глюк прикладывает постоянную горизонтальную силу $90\,\text{Н}$ к бруску с наибольшим номером.
    С каким ускорением двигается система? Чему равна сила натяжения нити, связывающей бруски 1 и 2?
}
\solutionspace{120pt}

\tasknumber{7}%
\task{%
    Два бруска связаны лёгкой нерастяжимой нитью и перекинуты через неподвижный блок (см.
    рис.).
    Определите силу натяжения нити и ускорения брусков.
    Силами трения пренебречь, массы брусков
    равны $m_1 = 8\,\text{кг}$ и $m_2 = 6\,\text{кг}$.

    \begin{tikzpicture}[x=1.5cm,y=1.5cm,thick]
        \draw
            (-0.4, 0) rectangle (-0.2, 1.2)
            (0.15, 0.5) rectangle (0.45, 1)
            (0, 2) circle [radius=0.3] -- ++(up:0.5)
            (-0.3, 1.2) -- ++(up:0.8)
            (0.3, 1) -- ++(up:1)
            (-0.7, 2.5) -- (0.7, 2.5)
            ;
        \draw[pattern={Lines[angle=51,distance=3pt]},pattern color=black,draw=none] (-0.7, 2.5) rectangle (0.7, 2.75);
        \node [left] (left) at (-0.4, 0.6) { $m_1$ };
        \node [right] (right) at (0.4, 0.75) { $m_2$ };
    \end{tikzpicture}
}

\variantsplitter

\addpersonalvariant{Анна Кузьмичёва}

\tasknumber{1}%
\task{%
    Электрон летит прямолинейно из точки $A$ в точку $B$, за ним при этом наблюдает экспериментатор Глюк.
    Глюк заметил, что первую треть пути электрон равномерно двигался со скоростью $4 \cdot 10^{5}\,\frac{\text{км}}{\text{ч}}$,
    затем его практически мгновенно ускорило электрическое поле
    и остаток пути электрон вновь равномерно двигался со скоростью $3 \cdot 10^{5}\,\frac{\text{км}}{\text{ч}}$.
    Определите среднюю скорость электрона.
    Ответ выразите в м/с и округлите до тысяч.
}
\solutionspace{120pt}

\tasknumber{2}%
\task{%
    Саша стартует на велосипеде и в течение $3\,\text{c}$ двигается с постоянным ускорением $2\,\frac{\text{м}}{\text{с}^{2}}$.
    Определите
    \begin{itemize}
        \item какую скорость при этом удастся достичь,
        \item какой путь за это время будет пройден,
        \item среднюю скорость за всё время движения, если после начального ускорения продолжить движение равномерно ещё в течение времени $nt$
    \end{itemize}
}
\solutionspace{120pt}

\tasknumber{3}%
\task{%
    Какой путь тело пройдёт за шестую секунду после начала свободного падения?
    Какую скорость в начале этой секунды оно имеет?
}
\solutionspace{120pt}

\tasknumber{4}%
\task{%
    Карусель диаметром $5\,\text{v}$ равномерно совершает 5 оборотов в минуту.
    Определите
    \begin{itemize}
        \item период и частоту её обращения,
        \item скорость и ускорение крайних её точек.
    \end{itemize}
}
\solutionspace{120pt}

\tasknumber{5}%
\task{%
    Даша стоит на обрыве над рекой и методично и строго горизонтально кидает в неё камушки.
    За этим всем наблюдает экспериментатор Глюк, который уже выяснил, что камушки падают в реку спустя $1{,}2\,\text{с}$ после броска,
    а вот дальность полёта оценить сложнее: придётся лезть в воду.
    Выручите Глюка и определите:
    \begin{itemize}
        \item высоту обрыва (вместе с ростом Даши).
        \item дальность полёта камушков (по горизонтали) и их скорость при падении, приняв начальную скорость броска равной $v = 18\,\frac{\text{м}}{\text{с}}$.
    \end{itemize}
    Сопротивлением воздуха пренебречь, $g = 10\,\frac{\text{м}}{\text{с}^{2}}$.
}
\solutionspace{120pt}

\tasknumber{6}%
\task{%
    Пять одинаковых брусков лежат на гладком горизонтальном столе.
    Масса каждого бруска равна $2\,\text{кг}$,
    причём они пронумерованы от 1 до 5 и последовательно связаны между собой невесомыми
    нерастяжимыми нитями: 1 со 2, 2 с 3 (ну и с 1) и т.д.
    Экспериментатор Глюк прикладывает постоянную горизонтальную силу $90\,\text{Н}$ к бруску с наименьшим номером.
    С каким ускорением двигается система? Чему равна сила натяжения нити, связывающей бруски 1 и 2?
}
\solutionspace{120pt}

\tasknumber{7}%
\task{%
    Два бруска связаны лёгкой нерастяжимой нитью и перекинуты через неподвижный блок (см.
    рис.).
    Определите силу натяжения нити и ускорения брусков.
    Силами трения пренебречь, массы брусков
    равны $m_1 = 5\,\text{кг}$ и $m_2 = 4\,\text{кг}$.

    \begin{tikzpicture}[x=1.5cm,y=1.5cm,thick]
        \draw
            (-0.4, 0) rectangle (-0.2, 1.2)
            (0.15, 0.5) rectangle (0.45, 1)
            (0, 2) circle [radius=0.3] -- ++(up:0.5)
            (-0.3, 1.2) -- ++(up:0.8)
            (0.3, 1) -- ++(up:1)
            (-0.7, 2.5) -- (0.7, 2.5)
            ;
        \draw[pattern={Lines[angle=51,distance=3pt]},pattern color=black,draw=none] (-0.7, 2.5) rectangle (0.7, 2.75);
        \node [left] (left) at (-0.4, 0.6) { $m_1$ };
        \node [right] (right) at (0.4, 0.75) { $m_2$ };
    \end{tikzpicture}
}

\variantsplitter

\addpersonalvariant{Алёна Куприянова}

\tasknumber{1}%
\task{%
    Электрон летит прямолинейно из точки $A$ в точку $B$, за ним при этом наблюдает экспериментатор Глюк.
    Глюк заметил, что первую четверть времени электрон равномерно двигался со скоростью $2 \cdot 10^{5}\,\frac{\text{км}}{\text{ч}}$,
    затем его практически мгновенно ускорило электрическое поле
    и остаток времени электрон вновь равномерно двигался со скоростью $3 \cdot 10^{5}\,\frac{\text{км}}{\text{ч}}$.
    Определите среднюю скорость электрона.
    Ответ выразите в м/с и округлите до тысяч.
}
\solutionspace{120pt}

\tasknumber{2}%
\task{%
    Валя стартует на мотоцикле и в течение $10\,\text{c}$ двигается с постоянным ускорением $1{,}5\,\frac{\text{м}}{\text{с}^{2}}$.
    Определите
    \begin{itemize}
        \item какую скорость при этом удастся достичь,
        \item какой путь за это время будет пройден,
        \item среднюю скорость за всё время движения, если после начального ускорения продолжить движение равномерно ещё в течение времени $nt$
    \end{itemize}
}
\solutionspace{120pt}

\tasknumber{3}%
\task{%
    Какой путь тело пройдёт за вторую секунду после начала свободного падения?
    Какую скорость в конце этой секунды оно имеет?
}
\solutionspace{120pt}

\tasknumber{4}%
\task{%
    Карусель диаметром $3\,\text{v}$ равномерно совершает 5 оборотов в минуту.
    Определите
    \begin{itemize}
        \item период и частоту её обращения,
        \item скорость и ускорение крайних её точек.
    \end{itemize}
}
\solutionspace{120pt}

\tasknumber{5}%
\task{%
    Даша стоит на обрыве над рекой и методично и строго горизонтально кидает в неё камушки.
    За этим всем наблюдает экспериментатор Глюк, который уже выяснил, что камушки падают в реку спустя $1{,}2\,\text{с}$ после броска,
    а вот дальность полёта оценить сложнее: придётся лезть в воду.
    Выручите Глюка и определите:
    \begin{itemize}
        \item высоту обрыва (вместе с ростом Даши).
        \item дальность полёта камушков (по горизонтали) и их скорость при падении, приняв начальную скорость броска равной $v = 12\,\frac{\text{м}}{\text{с}}$.
    \end{itemize}
    Сопротивлением воздуха пренебречь, $g = 10\,\frac{\text{м}}{\text{с}^{2}}$.
}
\solutionspace{120pt}

\tasknumber{6}%
\task{%
    Четыре одинаковых брусков лежат на гладком горизонтальном столе.
    Масса каждого бруска равна $2\,\text{кг}$,
    причём они пронумерованы от 1 до 4 и последовательно связаны между собой невесомыми
    нерастяжимыми нитями: 1 со 2, 2 с 3 (ну и с 1) и т.д.
    Экспериментатор Глюк прикладывает постоянную горизонтальную силу $90\,\text{Н}$ к бруску с наибольшим номером.
    С каким ускорением двигается система? Чему равна сила натяжения нити, связывающей бруски 2 и 3?
}
\solutionspace{120pt}

\tasknumber{7}%
\task{%
    Два бруска связаны лёгкой нерастяжимой нитью и перекинуты через неподвижный блок (см.
    рис.).
    Определите силу натяжения нити и ускорения брусков.
    Силами трения пренебречь, массы брусков
    равны $m_1 = 11\,\text{кг}$ и $m_2 = 6\,\text{кг}$.

    \begin{tikzpicture}[x=1.5cm,y=1.5cm,thick]
        \draw
            (-0.4, 0) rectangle (-0.2, 1.2)
            (0.15, 0.5) rectangle (0.45, 1)
            (0, 2) circle [radius=0.3] -- ++(up:0.5)
            (-0.3, 1.2) -- ++(up:0.8)
            (0.3, 1) -- ++(up:1)
            (-0.7, 2.5) -- (0.7, 2.5)
            ;
        \draw[pattern={Lines[angle=51,distance=3pt]},pattern color=black,draw=none] (-0.7, 2.5) rectangle (0.7, 2.75);
        \node [left] (left) at (-0.4, 0.6) { $m_1$ };
        \node [right] (right) at (0.4, 0.75) { $m_2$ };
    \end{tikzpicture}
}

\variantsplitter

\addpersonalvariant{Ярослав Лавровский}

\tasknumber{1}%
\task{%
    Электрон летит прямолинейно из точки $A$ в точку $B$, за ним при этом наблюдает экспериментатор Глюк.
    Глюк заметил, что первую половину времени электрон равномерно двигался со скоростью $4 \cdot 10^{5}\,\frac{\text{км}}{\text{ч}}$,
    затем его практически мгновенно ускорило электрическое поле
    и остаток времени электрон вновь равномерно двигался со скоростью $6 \cdot 10^{5}\,\frac{\text{км}}{\text{ч}}$.
    Определите среднюю скорость электрона.
    Ответ выразите в м/с и округлите до тысяч.
}
\solutionspace{120pt}

\tasknumber{2}%
\task{%
    Валя стартует на мотоцикле и в течение $10\,\text{c}$ двигается с постоянным ускорением $2\,\frac{\text{м}}{\text{с}^{2}}$.
    Определите
    \begin{itemize}
        \item какую скорость при этом удастся достичь,
        \item какой путь за это время будет пройден,
        \item среднюю скорость за всё время движения, если после начального ускорения продолжить движение равномерно ещё в течение времени $nt$
    \end{itemize}
}
\solutionspace{120pt}

\tasknumber{3}%
\task{%
    Какой путь тело пройдёт за третью секунду после начала свободного падения?
    Какую скорость в начале этой секунды оно имеет?
}
\solutionspace{120pt}

\tasknumber{4}%
\task{%
    Карусель диаметром $5\,\text{v}$ равномерно совершает 10 оборотов в минуту.
    Определите
    \begin{itemize}
        \item период и частоту её обращения,
        \item скорость и ускорение крайних её точек.
    \end{itemize}
}
\solutionspace{120pt}

\tasknumber{5}%
\task{%
    Миша стоит на обрыве над рекой и методично и строго горизонтально кидает в неё камушки.
    За этим всем наблюдает экспериментатор Глюк, который уже выяснил, что камушки падают в реку спустя $1{,}7\,\text{с}$ после броска,
    а вот дальность полёта оценить сложнее: придётся лезть в воду.
    Выручите Глюка и определите:
    \begin{itemize}
        \item высоту обрыва (вместе с ростом Миши).
        \item дальность полёта камушков (по горизонтали) и их скорость при падении, приняв начальную скорость броска равной $v = 17\,\frac{\text{м}}{\text{с}}$.
    \end{itemize}
    Сопротивлением воздуха пренебречь, $g = 10\,\frac{\text{м}}{\text{с}^{2}}$.
}
\solutionspace{120pt}

\tasknumber{6}%
\task{%
    Пять одинаковых брусков лежат на гладком горизонтальном столе.
    Масса каждого бруска равна $3\,\text{кг}$,
    причём они пронумерованы от 1 до 5 и последовательно связаны между собой невесомыми
    нерастяжимыми нитями: 1 со 2, 2 с 3 (ну и с 1) и т.д.
    Экспериментатор Глюк прикладывает постоянную горизонтальную силу $90\,\text{Н}$ к бруску с наименьшим номером.
    С каким ускорением двигается система? Чему равна сила натяжения нити, связывающей бруски 2 и 3?
}
\solutionspace{120pt}

\tasknumber{7}%
\task{%
    Два бруска связаны лёгкой нерастяжимой нитью и перекинуты через неподвижный блок (см.
    рис.).
    Определите силу натяжения нити и ускорения брусков.
    Силами трения пренебречь, массы брусков
    равны $m_1 = 5\,\text{кг}$ и $m_2 = 6\,\text{кг}$.

    \begin{tikzpicture}[x=1.5cm,y=1.5cm,thick]
        \draw
            (-0.4, 0) rectangle (-0.2, 1.2)
            (0.15, 0.5) rectangle (0.45, 1)
            (0, 2) circle [radius=0.3] -- ++(up:0.5)
            (-0.3, 1.2) -- ++(up:0.8)
            (0.3, 1) -- ++(up:1)
            (-0.7, 2.5) -- (0.7, 2.5)
            ;
        \draw[pattern={Lines[angle=51,distance=3pt]},pattern color=black,draw=none] (-0.7, 2.5) rectangle (0.7, 2.75);
        \node [left] (left) at (-0.4, 0.6) { $m_1$ };
        \node [right] (right) at (0.4, 0.75) { $m_2$ };
    \end{tikzpicture}
}

\variantsplitter

\addpersonalvariant{Анастасия Ламанова}

\tasknumber{1}%
\task{%
    Электрон летит прямолинейно из точки $A$ в точку $B$, за ним при этом наблюдает экспериментатор Глюк.
    Глюк заметил, что первую треть времени электрон равномерно двигался со скоростью $4 \cdot 10^{5}\,\frac{\text{км}}{\text{ч}}$,
    затем его практически мгновенно ускорило электрическое поле
    и остаток времени электрон вновь равномерно двигался со скоростью $3 \cdot 10^{5}\,\frac{\text{км}}{\text{ч}}$.
    Определите среднюю скорость электрона.
    Ответ выразите в м/с и округлите до тысяч.
}
\solutionspace{120pt}

\tasknumber{2}%
\task{%
    Саша стартует на лошади и в течение $5\,\text{c}$ двигается с постоянным ускорением $2{,}5\,\frac{\text{м}}{\text{с}^{2}}$.
    Определите
    \begin{itemize}
        \item какую скорость при этом удастся достичь,
        \item какой путь за это время будет пройден,
        \item среднюю скорость за всё время движения, если после начального ускорения продолжить движение равномерно ещё в течение времени $nt$
    \end{itemize}
}
\solutionspace{120pt}

\tasknumber{3}%
\task{%
    Какой путь тело пройдёт за вторую секунду после начала свободного падения?
    Какую скорость в начале этой секунды оно имеет?
}
\solutionspace{120pt}

\tasknumber{4}%
\task{%
    Карусель радиусом $2\,\text{v}$ равномерно совершает 10 оборотов в минуту.
    Определите
    \begin{itemize}
        \item период и частоту её обращения,
        \item скорость и ускорение крайних её точек.
    \end{itemize}
}
\solutionspace{120pt}

\tasknumber{5}%
\task{%
    Даша стоит на обрыве над рекой и методично и строго горизонтально кидает в неё камушки.
    За этим всем наблюдает экспериментатор Глюк, который уже выяснил, что камушки падают в реку спустя $1{,}2\,\text{с}$ после броска,
    а вот дальность полёта оценить сложнее: придётся лезть в воду.
    Выручите Глюка и определите:
    \begin{itemize}
        \item высоту обрыва (вместе с ростом Даши).
        \item дальность полёта камушков (по горизонтали) и их скорость при падении, приняв начальную скорость броска равной $v = 17\,\frac{\text{м}}{\text{с}}$.
    \end{itemize}
    Сопротивлением воздуха пренебречь, $g = 10\,\frac{\text{м}}{\text{с}^{2}}$.
}
\solutionspace{120pt}

\tasknumber{6}%
\task{%
    Пять одинаковых брусков лежат на гладком горизонтальном столе.
    Масса каждого бруска равна $3\,\text{кг}$,
    причём они пронумерованы от 1 до 5 и последовательно связаны между собой невесомыми
    нерастяжимыми нитями: 1 со 2, 2 с 3 (ну и с 1) и т.д.
    Экспериментатор Глюк прикладывает постоянную горизонтальную силу $120\,\text{Н}$ к бруску с наибольшим номером.
    С каким ускорением двигается система? Чему равна сила натяжения нити, связывающей бруски 1 и 2?
}
\solutionspace{120pt}

\tasknumber{7}%
\task{%
    Два бруска связаны лёгкой нерастяжимой нитью и перекинуты через неподвижный блок (см.
    рис.).
    Определите силу натяжения нити и ускорения брусков.
    Силами трения пренебречь, массы брусков
    равны $m_1 = 11\,\text{кг}$ и $m_2 = 14\,\text{кг}$.

    \begin{tikzpicture}[x=1.5cm,y=1.5cm,thick]
        \draw
            (-0.4, 0) rectangle (-0.2, 1.2)
            (0.15, 0.5) rectangle (0.45, 1)
            (0, 2) circle [radius=0.3] -- ++(up:0.5)
            (-0.3, 1.2) -- ++(up:0.8)
            (0.3, 1) -- ++(up:1)
            (-0.7, 2.5) -- (0.7, 2.5)
            ;
        \draw[pattern={Lines[angle=51,distance=3pt]},pattern color=black,draw=none] (-0.7, 2.5) rectangle (0.7, 2.75);
        \node [left] (left) at (-0.4, 0.6) { $m_1$ };
        \node [right] (right) at (0.4, 0.75) { $m_2$ };
    \end{tikzpicture}
}

\variantsplitter

\addpersonalvariant{Виктория Легонькова}

\tasknumber{1}%
\task{%
    Электрон летит прямолинейно из точки $A$ в точку $B$, за ним при этом наблюдает экспериментатор Глюк.
    Глюк заметил, что первую треть времени электрон равномерно двигался со скоростью $4 \cdot 10^{5}\,\frac{\text{км}}{\text{ч}}$,
    затем его практически мгновенно ускорило электрическое поле
    и остаток времени электрон вновь равномерно двигался со скоростью $3 \cdot 10^{5}\,\frac{\text{км}}{\text{ч}}$.
    Определите среднюю скорость электрона.
    Ответ выразите в м/с и округлите до тысяч.
}
\solutionspace{120pt}

\tasknumber{2}%
\task{%
    Женя стартует на мотоцикле и в течение $3\,\text{c}$ двигается с постоянным ускорением $1{,}5\,\frac{\text{м}}{\text{с}^{2}}$.
    Определите
    \begin{itemize}
        \item какую скорость при этом удастся достичь,
        \item какой путь за это время будет пройден,
        \item среднюю скорость за всё время движения, если после начального ускорения продолжить движение равномерно ещё в течение времени $nt$
    \end{itemize}
}
\solutionspace{120pt}

\tasknumber{3}%
\task{%
    Какой путь тело пройдёт за вторую секунду после начала свободного падения?
    Какую скорость в конце этой секунды оно имеет?
}
\solutionspace{120pt}

\tasknumber{4}%
\task{%
    Карусель радиусом $4\,\text{v}$ равномерно совершает 10 оборотов в минуту.
    Определите
    \begin{itemize}
        \item период и частоту её обращения,
        \item скорость и ускорение крайних её точек.
    \end{itemize}
}
\solutionspace{120pt}

\tasknumber{5}%
\task{%
    Даша стоит на обрыве над рекой и методично и строго горизонтально кидает в неё камушки.
    За этим всем наблюдает экспериментатор Глюк, который уже выяснил, что камушки падают в реку спустя $1{,}3\,\text{с}$ после броска,
    а вот дальность полёта оценить сложнее: придётся лезть в воду.
    Выручите Глюка и определите:
    \begin{itemize}
        \item высоту обрыва (вместе с ростом Даши).
        \item дальность полёта камушков (по горизонтали) и их скорость при падении, приняв начальную скорость броска равной $v = 13\,\frac{\text{м}}{\text{с}}$.
    \end{itemize}
    Сопротивлением воздуха пренебречь, $g = 10\,\frac{\text{м}}{\text{с}^{2}}$.
}
\solutionspace{120pt}

\tasknumber{6}%
\task{%
    Пять одинаковых брусков лежат на гладком горизонтальном столе.
    Масса каждого бруска равна $3\,\text{кг}$,
    причём они пронумерованы от 1 до 5 и последовательно связаны между собой невесомыми
    нерастяжимыми нитями: 1 со 2, 2 с 3 (ну и с 1) и т.д.
    Экспериментатор Глюк прикладывает постоянную горизонтальную силу $90\,\text{Н}$ к бруску с наименьшим номером.
    С каким ускорением двигается система? Чему равна сила натяжения нити, связывающей бруски 1 и 2?
}
\solutionspace{120pt}

\tasknumber{7}%
\task{%
    Два бруска связаны лёгкой нерастяжимой нитью и перекинуты через неподвижный блок (см.
    рис.).
    Определите силу натяжения нити и ускорения брусков.
    Силами трения пренебречь, массы брусков
    равны $m_1 = 5\,\text{кг}$ и $m_2 = 10\,\text{кг}$.

    \begin{tikzpicture}[x=1.5cm,y=1.5cm,thick]
        \draw
            (-0.4, 0) rectangle (-0.2, 1.2)
            (0.15, 0.5) rectangle (0.45, 1)
            (0, 2) circle [radius=0.3] -- ++(up:0.5)
            (-0.3, 1.2) -- ++(up:0.8)
            (0.3, 1) -- ++(up:1)
            (-0.7, 2.5) -- (0.7, 2.5)
            ;
        \draw[pattern={Lines[angle=51,distance=3pt]},pattern color=black,draw=none] (-0.7, 2.5) rectangle (0.7, 2.75);
        \node [left] (left) at (-0.4, 0.6) { $m_1$ };
        \node [right] (right) at (0.4, 0.75) { $m_2$ };
    \end{tikzpicture}
}

\variantsplitter

\addpersonalvariant{Семён Мартынов}

\tasknumber{1}%
\task{%
    Электрон летит прямолинейно из точки $A$ в точку $B$, за ним при этом наблюдает экспериментатор Глюк.
    Глюк заметил, что первую четверть времени электрон равномерно двигался со скоростью $4 \cdot 10^{5}\,\frac{\text{км}}{\text{ч}}$,
    затем его практически мгновенно ускорило электрическое поле
    и остаток времени электрон вновь равномерно двигался со скоростью $6 \cdot 10^{5}\,\frac{\text{км}}{\text{ч}}$.
    Определите среднюю скорость электрона.
    Ответ выразите в м/с и округлите до тысяч.
}
\solutionspace{120pt}

\tasknumber{2}%
\task{%
    Женя стартует на мотоцикле и в течение $10\,\text{c}$ двигается с постоянным ускорением $1{,}5\,\frac{\text{м}}{\text{с}^{2}}$.
    Определите
    \begin{itemize}
        \item какую скорость при этом удастся достичь,
        \item какой путь за это время будет пройден,
        \item среднюю скорость за всё время движения, если после начального ускорения продолжить движение равномерно ещё в течение времени $nt$
    \end{itemize}
}
\solutionspace{120pt}

\tasknumber{3}%
\task{%
    Какой путь тело пройдёт за третью секунду после начала свободного падения?
    Какую скорость в начале этой секунды оно имеет?
}
\solutionspace{120pt}

\tasknumber{4}%
\task{%
    Карусель диаметром $5\,\text{v}$ равномерно совершает 10 оборотов в минуту.
    Определите
    \begin{itemize}
        \item период и частоту её обращения,
        \item скорость и ускорение крайних её точек.
    \end{itemize}
}
\solutionspace{120pt}

\tasknumber{5}%
\task{%
    Паша стоит на обрыве над рекой и методично и строго горизонтально кидает в неё камушки.
    За этим всем наблюдает экспериментатор Глюк, который уже выяснил, что камушки падают в реку спустя $1{,}3\,\text{с}$ после броска,
    а вот дальность полёта оценить сложнее: придётся лезть в воду.
    Выручите Глюка и определите:
    \begin{itemize}
        \item высоту обрыва (вместе с ростом Паши).
        \item дальность полёта камушков (по горизонтали) и их скорость при падении, приняв начальную скорость броска равной $v = 18\,\frac{\text{м}}{\text{с}}$.
    \end{itemize}
    Сопротивлением воздуха пренебречь, $g = 10\,\frac{\text{м}}{\text{с}^{2}}$.
}
\solutionspace{120pt}

\tasknumber{6}%
\task{%
    Пять одинаковых брусков лежат на гладком горизонтальном столе.
    Масса каждого бруска равна $2\,\text{кг}$,
    причём они пронумерованы от 1 до 5 и последовательно связаны между собой невесомыми
    нерастяжимыми нитями: 1 со 2, 2 с 3 (ну и с 1) и т.д.
    Экспериментатор Глюк прикладывает постоянную горизонтальную силу $90\,\text{Н}$ к бруску с наименьшим номером.
    С каким ускорением двигается система? Чему равна сила натяжения нити, связывающей бруски 1 и 2?
}
\solutionspace{120pt}

\tasknumber{7}%
\task{%
    Два бруска связаны лёгкой нерастяжимой нитью и перекинуты через неподвижный блок (см.
    рис.).
    Определите силу натяжения нити и ускорения брусков.
    Силами трения пренебречь, массы брусков
    равны $m_1 = 8\,\text{кг}$ и $m_2 = 10\,\text{кг}$.

    \begin{tikzpicture}[x=1.5cm,y=1.5cm,thick]
        \draw
            (-0.4, 0) rectangle (-0.2, 1.2)
            (0.15, 0.5) rectangle (0.45, 1)
            (0, 2) circle [radius=0.3] -- ++(up:0.5)
            (-0.3, 1.2) -- ++(up:0.8)
            (0.3, 1) -- ++(up:1)
            (-0.7, 2.5) -- (0.7, 2.5)
            ;
        \draw[pattern={Lines[angle=51,distance=3pt]},pattern color=black,draw=none] (-0.7, 2.5) rectangle (0.7, 2.75);
        \node [left] (left) at (-0.4, 0.6) { $m_1$ };
        \node [right] (right) at (0.4, 0.75) { $m_2$ };
    \end{tikzpicture}
}

\variantsplitter

\addpersonalvariant{Варвара Минаева}

\tasknumber{1}%
\task{%
    Электрон летит прямолинейно из точки $A$ в точку $B$, за ним при этом наблюдает экспериментатор Глюк.
    Глюк заметил, что первую треть времени электрон равномерно двигался со скоростью $4 \cdot 10^{5}\,\frac{\text{км}}{\text{ч}}$,
    затем его практически мгновенно ускорило электрическое поле
    и остаток времени электрон вновь равномерно двигался со скоростью $6 \cdot 10^{5}\,\frac{\text{км}}{\text{ч}}$.
    Определите среднюю скорость электрона.
    Ответ выразите в м/с и округлите до тысяч.
}
\solutionspace{120pt}

\tasknumber{2}%
\task{%
    Валя стартует на лошади и в течение $4\,\text{c}$ двигается с постоянным ускорением $0{,}5\,\frac{\text{м}}{\text{с}^{2}}$.
    Определите
    \begin{itemize}
        \item какую скорость при этом удастся достичь,
        \item какой путь за это время будет пройден,
        \item среднюю скорость за всё время движения, если после начального ускорения продолжить движение равномерно ещё в течение времени $nt$
    \end{itemize}
}
\solutionspace{120pt}

\tasknumber{3}%
\task{%
    Какой путь тело пройдёт за вторую секунду после начала свободного падения?
    Какую скорость в начале этой секунды оно имеет?
}
\solutionspace{120pt}

\tasknumber{4}%
\task{%
    Карусель диаметром $4\,\text{v}$ равномерно совершает 10 оборотов в минуту.
    Определите
    \begin{itemize}
        \item период и частоту её обращения,
        \item скорость и ускорение крайних её точек.
    \end{itemize}
}
\solutionspace{120pt}

\tasknumber{5}%
\task{%
    Миша стоит на обрыве над рекой и методично и строго горизонтально кидает в неё камушки.
    За этим всем наблюдает экспериментатор Глюк, который уже выяснил, что камушки падают в реку спустя $1{,}5\,\text{с}$ после броска,
    а вот дальность полёта оценить сложнее: придётся лезть в воду.
    Выручите Глюка и определите:
    \begin{itemize}
        \item высоту обрыва (вместе с ростом Миши).
        \item дальность полёта камушков (по горизонтали) и их скорость при падении, приняв начальную скорость броска равной $v = 16\,\frac{\text{м}}{\text{с}}$.
    \end{itemize}
    Сопротивлением воздуха пренебречь, $g = 10\,\frac{\text{м}}{\text{с}^{2}}$.
}
\solutionspace{120pt}

\tasknumber{6}%
\task{%
    Четыре одинаковых брусков лежат на гладком горизонтальном столе.
    Масса каждого бруска равна $3\,\text{кг}$,
    причём они пронумерованы от 1 до 4 и последовательно связаны между собой невесомыми
    нерастяжимыми нитями: 1 со 2, 2 с 3 (ну и с 1) и т.д.
    Экспериментатор Глюк прикладывает постоянную горизонтальную силу $60\,\text{Н}$ к бруску с наименьшим номером.
    С каким ускорением двигается система? Чему равна сила натяжения нити, связывающей бруски 3 и 4?
}
\solutionspace{120pt}

\tasknumber{7}%
\task{%
    Два бруска связаны лёгкой нерастяжимой нитью и перекинуты через неподвижный блок (см.
    рис.).
    Определите силу натяжения нити и ускорения брусков.
    Силами трения пренебречь, массы брусков
    равны $m_1 = 8\,\text{кг}$ и $m_2 = 10\,\text{кг}$.

    \begin{tikzpicture}[x=1.5cm,y=1.5cm,thick]
        \draw
            (-0.4, 0) rectangle (-0.2, 1.2)
            (0.15, 0.5) rectangle (0.45, 1)
            (0, 2) circle [radius=0.3] -- ++(up:0.5)
            (-0.3, 1.2) -- ++(up:0.8)
            (0.3, 1) -- ++(up:1)
            (-0.7, 2.5) -- (0.7, 2.5)
            ;
        \draw[pattern={Lines[angle=51,distance=3pt]},pattern color=black,draw=none] (-0.7, 2.5) rectangle (0.7, 2.75);
        \node [left] (left) at (-0.4, 0.6) { $m_1$ };
        \node [right] (right) at (0.4, 0.75) { $m_2$ };
    \end{tikzpicture}
}

\variantsplitter

\addpersonalvariant{Леонид Никитин}

\tasknumber{1}%
\task{%
    Электрон летит прямолинейно из точки $A$ в точку $B$, за ним при этом наблюдает экспериментатор Глюк.
    Глюк заметил, что первую половину пути электрон равномерно двигался со скоростью $2 \cdot 10^{5}\,\frac{\text{км}}{\text{ч}}$,
    затем его практически мгновенно ускорило электрическое поле
    и остаток пути электрон вновь равномерно двигался со скоростью $6 \cdot 10^{5}\,\frac{\text{км}}{\text{ч}}$.
    Определите среднюю скорость электрона.
    Ответ выразите в м/с и округлите до тысяч.
}
\solutionspace{120pt}

\tasknumber{2}%
\task{%
    Саша стартует на велосипеде и в течение $4\,\text{c}$ двигается с постоянным ускорением $2{,}5\,\frac{\text{м}}{\text{с}^{2}}$.
    Определите
    \begin{itemize}
        \item какую скорость при этом удастся достичь,
        \item какой путь за это время будет пройден,
        \item среднюю скорость за всё время движения, если после начального ускорения продолжить движение равномерно ещё в течение времени $nt$
    \end{itemize}
}
\solutionspace{120pt}

\tasknumber{3}%
\task{%
    Какой путь тело пройдёт за третью секунду после начала свободного падения?
    Какую скорость в конце этой секунды оно имеет?
}
\solutionspace{120pt}

\tasknumber{4}%
\task{%
    Карусель радиусом $5\,\text{v}$ равномерно совершает 6 оборотов в минуту.
    Определите
    \begin{itemize}
        \item период и частоту её обращения,
        \item скорость и ускорение крайних её точек.
    \end{itemize}
}
\solutionspace{120pt}

\tasknumber{5}%
\task{%
    Маша стоит на обрыве над рекой и методично и строго горизонтально кидает в неё камушки.
    За этим всем наблюдает экспериментатор Глюк, который уже выяснил, что камушки падают в реку спустя $1{,}5\,\text{с}$ после броска,
    а вот дальность полёта оценить сложнее: придётся лезть в воду.
    Выручите Глюка и определите:
    \begin{itemize}
        \item высоту обрыва (вместе с ростом Маши).
        \item дальность полёта камушков (по горизонтали) и их скорость при падении, приняв начальную скорость броска равной $v = 18\,\frac{\text{м}}{\text{с}}$.
    \end{itemize}
    Сопротивлением воздуха пренебречь, $g = 10\,\frac{\text{м}}{\text{с}^{2}}$.
}
\solutionspace{120pt}

\tasknumber{6}%
\task{%
    Шесть одинаковых брусков лежат на гладком горизонтальном столе.
    Масса каждого бруска равна $2\,\text{кг}$,
    причём они пронумерованы от 1 до 6 и последовательно связаны между собой невесомыми
    нерастяжимыми нитями: 1 со 2, 2 с 3 (ну и с 1) и т.д.
    Экспериментатор Глюк прикладывает постоянную горизонтальную силу $90\,\text{Н}$ к бруску с наименьшим номером.
    С каким ускорением двигается система? Чему равна сила натяжения нити, связывающей бруски 2 и 3?
}
\solutionspace{120pt}

\tasknumber{7}%
\task{%
    Два бруска связаны лёгкой нерастяжимой нитью и перекинуты через неподвижный блок (см.
    рис.).
    Определите силу натяжения нити и ускорения брусков.
    Силами трения пренебречь, массы брусков
    равны $m_1 = 11\,\text{кг}$ и $m_2 = 4\,\text{кг}$.

    \begin{tikzpicture}[x=1.5cm,y=1.5cm,thick]
        \draw
            (-0.4, 0) rectangle (-0.2, 1.2)
            (0.15, 0.5) rectangle (0.45, 1)
            (0, 2) circle [radius=0.3] -- ++(up:0.5)
            (-0.3, 1.2) -- ++(up:0.8)
            (0.3, 1) -- ++(up:1)
            (-0.7, 2.5) -- (0.7, 2.5)
            ;
        \draw[pattern={Lines[angle=51,distance=3pt]},pattern color=black,draw=none] (-0.7, 2.5) rectangle (0.7, 2.75);
        \node [left] (left) at (-0.4, 0.6) { $m_1$ };
        \node [right] (right) at (0.4, 0.75) { $m_2$ };
    \end{tikzpicture}
}

\variantsplitter

\addpersonalvariant{Тимофей Полетаев}

\tasknumber{1}%
\task{%
    Электрон летит прямолинейно из точки $A$ в точку $B$, за ним при этом наблюдает экспериментатор Глюк.
    Глюк заметил, что первую треть пути электрон равномерно двигался со скоростью $4 \cdot 10^{5}\,\frac{\text{км}}{\text{ч}}$,
    затем его практически мгновенно ускорило электрическое поле
    и остаток пути электрон вновь равномерно двигался со скоростью $3 \cdot 10^{5}\,\frac{\text{км}}{\text{ч}}$.
    Определите среднюю скорость электрона.
    Ответ выразите в м/с и округлите до тысяч.
}
\solutionspace{120pt}

\tasknumber{2}%
\task{%
    Саша стартует на велосипеде и в течение $4\,\text{c}$ двигается с постоянным ускорением $2{,}5\,\frac{\text{м}}{\text{с}^{2}}$.
    Определите
    \begin{itemize}
        \item какую скорость при этом удастся достичь,
        \item какой путь за это время будет пройден,
        \item среднюю скорость за всё время движения, если после начального ускорения продолжить движение равномерно ещё в течение времени $nt$
    \end{itemize}
}
\solutionspace{120pt}

\tasknumber{3}%
\task{%
    Какой путь тело пройдёт за вторую секунду после начала свободного падения?
    Какую скорость в конце этой секунды оно имеет?
}
\solutionspace{120pt}

\tasknumber{4}%
\task{%
    Карусель радиусом $3\,\text{v}$ равномерно совершает 10 оборотов в минуту.
    Определите
    \begin{itemize}
        \item период и частоту её обращения,
        \item скорость и ускорение крайних её точек.
    \end{itemize}
}
\solutionspace{120pt}

\tasknumber{5}%
\task{%
    Маша стоит на обрыве над рекой и методично и строго горизонтально кидает в неё камушки.
    За этим всем наблюдает экспериментатор Глюк, который уже выяснил, что камушки падают в реку спустя $1{,}3\,\text{с}$ после броска,
    а вот дальность полёта оценить сложнее: придётся лезть в воду.
    Выручите Глюка и определите:
    \begin{itemize}
        \item высоту обрыва (вместе с ростом Маши).
        \item дальность полёта камушков (по горизонтали) и их скорость при падении, приняв начальную скорость броска равной $v = 13\,\frac{\text{м}}{\text{с}}$.
    \end{itemize}
    Сопротивлением воздуха пренебречь, $g = 10\,\frac{\text{м}}{\text{с}^{2}}$.
}
\solutionspace{120pt}

\tasknumber{6}%
\task{%
    Шесть одинаковых брусков лежат на гладком горизонтальном столе.
    Масса каждого бруска равна $2\,\text{кг}$,
    причём они пронумерованы от 1 до 6 и последовательно связаны между собой невесомыми
    нерастяжимыми нитями: 1 со 2, 2 с 3 (ну и с 1) и т.д.
    Экспериментатор Глюк прикладывает постоянную горизонтальную силу $90\,\text{Н}$ к бруску с наибольшим номером.
    С каким ускорением двигается система? Чему равна сила натяжения нити, связывающей бруски 1 и 2?
}
\solutionspace{120pt}

\tasknumber{7}%
\task{%
    Два бруска связаны лёгкой нерастяжимой нитью и перекинуты через неподвижный блок (см.
    рис.).
    Определите силу натяжения нити и ускорения брусков.
    Силами трения пренебречь, массы брусков
    равны $m_1 = 11\,\text{кг}$ и $m_2 = 10\,\text{кг}$.

    \begin{tikzpicture}[x=1.5cm,y=1.5cm,thick]
        \draw
            (-0.4, 0) rectangle (-0.2, 1.2)
            (0.15, 0.5) rectangle (0.45, 1)
            (0, 2) circle [radius=0.3] -- ++(up:0.5)
            (-0.3, 1.2) -- ++(up:0.8)
            (0.3, 1) -- ++(up:1)
            (-0.7, 2.5) -- (0.7, 2.5)
            ;
        \draw[pattern={Lines[angle=51,distance=3pt]},pattern color=black,draw=none] (-0.7, 2.5) rectangle (0.7, 2.75);
        \node [left] (left) at (-0.4, 0.6) { $m_1$ };
        \node [right] (right) at (0.4, 0.75) { $m_2$ };
    \end{tikzpicture}
}

\variantsplitter

\addpersonalvariant{Андрей Рожков}

\tasknumber{1}%
\task{%
    Электрон летит прямолинейно из точки $A$ в точку $B$, за ним при этом наблюдает экспериментатор Глюк.
    Глюк заметил, что первую треть пути электрон равномерно двигался со скоростью $4 \cdot 10^{5}\,\frac{\text{км}}{\text{ч}}$,
    затем его практически мгновенно ускорило электрическое поле
    и остаток пути электрон вновь равномерно двигался со скоростью $6 \cdot 10^{5}\,\frac{\text{км}}{\text{ч}}$.
    Определите среднюю скорость электрона.
    Ответ выразите в м/с и округлите до тысяч.
}
\solutionspace{120pt}

\tasknumber{2}%
\task{%
    Валя стартует на велосипеде и в течение $4\,\text{c}$ двигается с постоянным ускорением $1{,}5\,\frac{\text{м}}{\text{с}^{2}}$.
    Определите
    \begin{itemize}
        \item какую скорость при этом удастся достичь,
        \item какой путь за это время будет пройден,
        \item среднюю скорость за всё время движения, если после начального ускорения продолжить движение равномерно ещё в течение времени $nt$
    \end{itemize}
}
\solutionspace{120pt}

\tasknumber{3}%
\task{%
    Какой путь тело пройдёт за третью секунду после начала свободного падения?
    Какую скорость в конце этой секунды оно имеет?
}
\solutionspace{120pt}

\tasknumber{4}%
\task{%
    Карусель радиусом $2\,\text{v}$ равномерно совершает 10 оборотов в минуту.
    Определите
    \begin{itemize}
        \item период и частоту её обращения,
        \item скорость и ускорение крайних её точек.
    \end{itemize}
}
\solutionspace{120pt}

\tasknumber{5}%
\task{%
    Даша стоит на обрыве над рекой и методично и строго горизонтально кидает в неё камушки.
    За этим всем наблюдает экспериментатор Глюк, который уже выяснил, что камушки падают в реку спустя $1{,}5\,\text{с}$ после броска,
    а вот дальность полёта оценить сложнее: придётся лезть в воду.
    Выручите Глюка и определите:
    \begin{itemize}
        \item высоту обрыва (вместе с ростом Даши).
        \item дальность полёта камушков (по горизонтали) и их скорость при падении, приняв начальную скорость броска равной $v = 18\,\frac{\text{м}}{\text{с}}$.
    \end{itemize}
    Сопротивлением воздуха пренебречь, $g = 10\,\frac{\text{м}}{\text{с}^{2}}$.
}
\solutionspace{120pt}

\tasknumber{6}%
\task{%
    Пять одинаковых брусков лежат на гладком горизонтальном столе.
    Масса каждого бруска равна $3\,\text{кг}$,
    причём они пронумерованы от 1 до 5 и последовательно связаны между собой невесомыми
    нерастяжимыми нитями: 1 со 2, 2 с 3 (ну и с 1) и т.д.
    Экспериментатор Глюк прикладывает постоянную горизонтальную силу $90\,\text{Н}$ к бруску с наименьшим номером.
    С каким ускорением двигается система? Чему равна сила натяжения нити, связывающей бруски 3 и 4?
}
\solutionspace{120pt}

\tasknumber{7}%
\task{%
    Два бруска связаны лёгкой нерастяжимой нитью и перекинуты через неподвижный блок (см.
    рис.).
    Определите силу натяжения нити и ускорения брусков.
    Силами трения пренебречь, массы брусков
    равны $m_1 = 8\,\text{кг}$ и $m_2 = 14\,\text{кг}$.

    \begin{tikzpicture}[x=1.5cm,y=1.5cm,thick]
        \draw
            (-0.4, 0) rectangle (-0.2, 1.2)
            (0.15, 0.5) rectangle (0.45, 1)
            (0, 2) circle [radius=0.3] -- ++(up:0.5)
            (-0.3, 1.2) -- ++(up:0.8)
            (0.3, 1) -- ++(up:1)
            (-0.7, 2.5) -- (0.7, 2.5)
            ;
        \draw[pattern={Lines[angle=51,distance=3pt]},pattern color=black,draw=none] (-0.7, 2.5) rectangle (0.7, 2.75);
        \node [left] (left) at (-0.4, 0.6) { $m_1$ };
        \node [right] (right) at (0.4, 0.75) { $m_2$ };
    \end{tikzpicture}
}

\variantsplitter

\addpersonalvariant{Рената Таржиманова}

\tasknumber{1}%
\task{%
    Электрон летит прямолинейно из точки $A$ в точку $B$, за ним при этом наблюдает экспериментатор Глюк.
    Глюк заметил, что первую четверть времени электрон равномерно двигался со скоростью $4 \cdot 10^{5}\,\frac{\text{км}}{\text{ч}}$,
    затем его практически мгновенно ускорило электрическое поле
    и остаток времени электрон вновь равномерно двигался со скоростью $3 \cdot 10^{5}\,\frac{\text{км}}{\text{ч}}$.
    Определите среднюю скорость электрона.
    Ответ выразите в м/с и округлите до тысяч.
}
\solutionspace{120pt}

\tasknumber{2}%
\task{%
    Женя стартует на мотоцикле и в течение $10\,\text{c}$ двигается с постоянным ускорением $2\,\frac{\text{м}}{\text{с}^{2}}$.
    Определите
    \begin{itemize}
        \item какую скорость при этом удастся достичь,
        \item какой путь за это время будет пройден,
        \item среднюю скорость за всё время движения, если после начального ускорения продолжить движение равномерно ещё в течение времени $nt$
    \end{itemize}
}
\solutionspace{120pt}

\tasknumber{3}%
\task{%
    Какой путь тело пройдёт за шестую секунду после начала свободного падения?
    Какую скорость в конце этой секунды оно имеет?
}
\solutionspace{120pt}

\tasknumber{4}%
\task{%
    Карусель диаметром $5\,\text{v}$ равномерно совершает 6 оборотов в минуту.
    Определите
    \begin{itemize}
        \item период и частоту её обращения,
        \item скорость и ускорение крайних её точек.
    \end{itemize}
}
\solutionspace{120pt}

\tasknumber{5}%
\task{%
    Миша стоит на обрыве над рекой и методично и строго горизонтально кидает в неё камушки.
    За этим всем наблюдает экспериментатор Глюк, который уже выяснил, что камушки падают в реку спустя $1{,}7\,\text{с}$ после броска,
    а вот дальность полёта оценить сложнее: придётся лезть в воду.
    Выручите Глюка и определите:
    \begin{itemize}
        \item высоту обрыва (вместе с ростом Миши).
        \item дальность полёта камушков (по горизонтали) и их скорость при падении, приняв начальную скорость броска равной $v = 12\,\frac{\text{м}}{\text{с}}$.
    \end{itemize}
    Сопротивлением воздуха пренебречь, $g = 10\,\frac{\text{м}}{\text{с}^{2}}$.
}
\solutionspace{120pt}

\tasknumber{6}%
\task{%
    Четыре одинаковых брусков лежат на гладком горизонтальном столе.
    Масса каждого бруска равна $3\,\text{кг}$,
    причём они пронумерованы от 1 до 4 и последовательно связаны между собой невесомыми
    нерастяжимыми нитями: 1 со 2, 2 с 3 (ну и с 1) и т.д.
    Экспериментатор Глюк прикладывает постоянную горизонтальную силу $120\,\text{Н}$ к бруску с наименьшим номером.
    С каким ускорением двигается система? Чему равна сила натяжения нити, связывающей бруски 3 и 4?
}
\solutionspace{120pt}

\tasknumber{7}%
\task{%
    Два бруска связаны лёгкой нерастяжимой нитью и перекинуты через неподвижный блок (см.
    рис.).
    Определите силу натяжения нити и ускорения брусков.
    Силами трения пренебречь, массы брусков
    равны $m_1 = 5\,\text{кг}$ и $m_2 = 14\,\text{кг}$.

    \begin{tikzpicture}[x=1.5cm,y=1.5cm,thick]
        \draw
            (-0.4, 0) rectangle (-0.2, 1.2)
            (0.15, 0.5) rectangle (0.45, 1)
            (0, 2) circle [radius=0.3] -- ++(up:0.5)
            (-0.3, 1.2) -- ++(up:0.8)
            (0.3, 1) -- ++(up:1)
            (-0.7, 2.5) -- (0.7, 2.5)
            ;
        \draw[pattern={Lines[angle=51,distance=3pt]},pattern color=black,draw=none] (-0.7, 2.5) rectangle (0.7, 2.75);
        \node [left] (left) at (-0.4, 0.6) { $m_1$ };
        \node [right] (right) at (0.4, 0.75) { $m_2$ };
    \end{tikzpicture}
}

\variantsplitter

\addpersonalvariant{Андрей Щербаков}

\tasknumber{1}%
\task{%
    Электрон летит прямолинейно из точки $A$ в точку $B$, за ним при этом наблюдает экспериментатор Глюк.
    Глюк заметил, что первую четверть времени электрон равномерно двигался со скоростью $4 \cdot 10^{5}\,\frac{\text{км}}{\text{ч}}$,
    затем его практически мгновенно ускорило электрическое поле
    и остаток времени электрон вновь равномерно двигался со скоростью $3 \cdot 10^{5}\,\frac{\text{км}}{\text{ч}}$.
    Определите среднюю скорость электрона.
    Ответ выразите в м/с и округлите до тысяч.
}
\solutionspace{120pt}

\tasknumber{2}%
\task{%
    Женя стартует на мотоцикле и в течение $10\,\text{c}$ двигается с постоянным ускорением $2\,\frac{\text{м}}{\text{с}^{2}}$.
    Определите
    \begin{itemize}
        \item какую скорость при этом удастся достичь,
        \item какой путь за это время будет пройден,
        \item среднюю скорость за всё время движения, если после начального ускорения продолжить движение равномерно ещё в течение времени $nt$
    \end{itemize}
}
\solutionspace{120pt}

\tasknumber{3}%
\task{%
    Какой путь тело пройдёт за пятую секунду после начала свободного падения?
    Какую скорость в начале этой секунды оно имеет?
}
\solutionspace{120pt}

\tasknumber{4}%
\task{%
    Карусель радиусом $4\,\text{v}$ равномерно совершает 5 оборотов в минуту.
    Определите
    \begin{itemize}
        \item период и частоту её обращения,
        \item скорость и ускорение крайних её точек.
    \end{itemize}
}
\solutionspace{120pt}

\tasknumber{5}%
\task{%
    Паша стоит на обрыве над рекой и методично и строго горизонтально кидает в неё камушки.
    За этим всем наблюдает экспериментатор Глюк, который уже выяснил, что камушки падают в реку спустя $1{,}2\,\text{с}$ после броска,
    а вот дальность полёта оценить сложнее: придётся лезть в воду.
    Выручите Глюка и определите:
    \begin{itemize}
        \item высоту обрыва (вместе с ростом Паши).
        \item дальность полёта камушков (по горизонтали) и их скорость при падении, приняв начальную скорость броска равной $v = 15\,\frac{\text{м}}{\text{с}}$.
    \end{itemize}
    Сопротивлением воздуха пренебречь, $g = 10\,\frac{\text{м}}{\text{с}^{2}}$.
}
\solutionspace{120pt}

\tasknumber{6}%
\task{%
    Шесть одинаковых брусков лежат на гладком горизонтальном столе.
    Масса каждого бруска равна $2\,\text{кг}$,
    причём они пронумерованы от 1 до 6 и последовательно связаны между собой невесомыми
    нерастяжимыми нитями: 1 со 2, 2 с 3 (ну и с 1) и т.д.
    Экспериментатор Глюк прикладывает постоянную горизонтальную силу $60\,\text{Н}$ к бруску с наименьшим номером.
    С каким ускорением двигается система? Чему равна сила натяжения нити, связывающей бруски 2 и 3?
}
\solutionspace{120pt}

\tasknumber{7}%
\task{%
    Два бруска связаны лёгкой нерастяжимой нитью и перекинуты через неподвижный блок (см.
    рис.).
    Определите силу натяжения нити и ускорения брусков.
    Силами трения пренебречь, массы брусков
    равны $m_1 = 8\,\text{кг}$ и $m_2 = 10\,\text{кг}$.

    \begin{tikzpicture}[x=1.5cm,y=1.5cm,thick]
        \draw
            (-0.4, 0) rectangle (-0.2, 1.2)
            (0.15, 0.5) rectangle (0.45, 1)
            (0, 2) circle [radius=0.3] -- ++(up:0.5)
            (-0.3, 1.2) -- ++(up:0.8)
            (0.3, 1) -- ++(up:1)
            (-0.7, 2.5) -- (0.7, 2.5)
            ;
        \draw[pattern={Lines[angle=51,distance=3pt]},pattern color=black,draw=none] (-0.7, 2.5) rectangle (0.7, 2.75);
        \node [left] (left) at (-0.4, 0.6) { $m_1$ };
        \node [right] (right) at (0.4, 0.75) { $m_2$ };
    \end{tikzpicture}
}

\variantsplitter

\addpersonalvariant{Михаил Ярошевский}

\tasknumber{1}%
\task{%
    Электрон летит прямолинейно из точки $A$ в точку $B$, за ним при этом наблюдает экспериментатор Глюк.
    Глюк заметил, что первую половину пути электрон равномерно двигался со скоростью $4 \cdot 10^{5}\,\frac{\text{км}}{\text{ч}}$,
    затем его практически мгновенно ускорило электрическое поле
    и остаток пути электрон вновь равномерно двигался со скоростью $3 \cdot 10^{5}\,\frac{\text{км}}{\text{ч}}$.
    Определите среднюю скорость электрона.
    Ответ выразите в м/с и округлите до тысяч.
}
\solutionspace{120pt}

\tasknumber{2}%
\task{%
    Саша стартует на лошади и в течение $10\,\text{c}$ двигается с постоянным ускорением $1{,}5\,\frac{\text{м}}{\text{с}^{2}}$.
    Определите
    \begin{itemize}
        \item какую скорость при этом удастся достичь,
        \item какой путь за это время будет пройден,
        \item среднюю скорость за всё время движения, если после начального ускорения продолжить движение равномерно ещё в течение времени $nt$
    \end{itemize}
}
\solutionspace{120pt}

\tasknumber{3}%
\task{%
    Какой путь тело пройдёт за четвёртую секунду после начала свободного падения?
    Какую скорость в конце этой секунды оно имеет?
}
\solutionspace{120pt}

\tasknumber{4}%
\task{%
    Карусель диаметром $3\,\text{v}$ равномерно совершает 10 оборотов в минуту.
    Определите
    \begin{itemize}
        \item период и частоту её обращения,
        \item скорость и ускорение крайних её точек.
    \end{itemize}
}
\solutionspace{120pt}

\tasknumber{5}%
\task{%
    Миша стоит на обрыве над рекой и методично и строго горизонтально кидает в неё камушки.
    За этим всем наблюдает экспериментатор Глюк, который уже выяснил, что камушки падают в реку спустя $1{,}6\,\text{с}$ после броска,
    а вот дальность полёта оценить сложнее: придётся лезть в воду.
    Выручите Глюка и определите:
    \begin{itemize}
        \item высоту обрыва (вместе с ростом Миши).
        \item дальность полёта камушков (по горизонтали) и их скорость при падении, приняв начальную скорость броска равной $v = 14\,\frac{\text{м}}{\text{с}}$.
    \end{itemize}
    Сопротивлением воздуха пренебречь, $g = 10\,\frac{\text{м}}{\text{с}^{2}}$.
}
\solutionspace{120pt}

\tasknumber{6}%
\task{%
    Шесть одинаковых брусков лежат на гладком горизонтальном столе.
    Масса каждого бруска равна $2\,\text{кг}$,
    причём они пронумерованы от 1 до 6 и последовательно связаны между собой невесомыми
    нерастяжимыми нитями: 1 со 2, 2 с 3 (ну и с 1) и т.д.
    Экспериментатор Глюк прикладывает постоянную горизонтальную силу $90\,\text{Н}$ к бруску с наибольшим номером.
    С каким ускорением двигается система? Чему равна сила натяжения нити, связывающей бруски 3 и 4?
}
\solutionspace{120pt}

\tasknumber{7}%
\task{%
    Два бруска связаны лёгкой нерастяжимой нитью и перекинуты через неподвижный блок (см.
    рис.).
    Определите силу натяжения нити и ускорения брусков.
    Силами трения пренебречь, массы брусков
    равны $m_1 = 11\,\text{кг}$ и $m_2 = 10\,\text{кг}$.

    \begin{tikzpicture}[x=1.5cm,y=1.5cm,thick]
        \draw
            (-0.4, 0) rectangle (-0.2, 1.2)
            (0.15, 0.5) rectangle (0.45, 1)
            (0, 2) circle [radius=0.3] -- ++(up:0.5)
            (-0.3, 1.2) -- ++(up:0.8)
            (0.3, 1) -- ++(up:1)
            (-0.7, 2.5) -- (0.7, 2.5)
            ;
        \draw[pattern={Lines[angle=51,distance=3pt]},pattern color=black,draw=none] (-0.7, 2.5) rectangle (0.7, 2.75);
        \node [left] (left) at (-0.4, 0.6) { $m_1$ };
        \node [right] (right) at (0.4, 0.75) { $m_2$ };
    \end{tikzpicture}
}

\variantsplitter

\addpersonalvariant{Алексей Алимпиев}

\tasknumber{1}%
\task{%
    Электрон летит прямолинейно из точки $A$ в точку $B$, за ним при этом наблюдает экспериментатор Глюк.
    Глюк заметил, что первую половину пути электрон равномерно двигался со скоростью $2 \cdot 10^{5}\,\frac{\text{км}}{\text{ч}}$,
    затем его практически мгновенно ускорило электрическое поле
    и остаток пути электрон вновь равномерно двигался со скоростью $6 \cdot 10^{5}\,\frac{\text{км}}{\text{ч}}$.
    Определите среднюю скорость электрона.
    Ответ выразите в м/с и округлите до тысяч.
}
\solutionspace{120pt}

\tasknumber{2}%
\task{%
    Женя стартует на мотоцикле и в течение $3\,\text{c}$ двигается с постоянным ускорением $2{,}5\,\frac{\text{м}}{\text{с}^{2}}$.
    Определите
    \begin{itemize}
        \item какую скорость при этом удастся достичь,
        \item какой путь за это время будет пройден,
        \item среднюю скорость за всё время движения, если после начального ускорения продолжить движение равномерно ещё в течение времени $nt$
    \end{itemize}
}
\solutionspace{120pt}

\tasknumber{3}%
\task{%
    Какой путь тело пройдёт за вторую секунду после начала свободного падения?
    Какую скорость в начале этой секунды оно имеет?
}
\solutionspace{120pt}

\tasknumber{4}%
\task{%
    Карусель диаметром $4\,\text{v}$ равномерно совершает 6 оборотов в минуту.
    Определите
    \begin{itemize}
        \item период и частоту её обращения,
        \item скорость и ускорение крайних её точек.
    \end{itemize}
}
\solutionspace{120pt}

\tasknumber{5}%
\task{%
    Даша стоит на обрыве над рекой и методично и строго горизонтально кидает в неё камушки.
    За этим всем наблюдает экспериментатор Глюк, который уже выяснил, что камушки падают в реку спустя $1{,}3\,\text{с}$ после броска,
    а вот дальность полёта оценить сложнее: придётся лезть в воду.
    Выручите Глюка и определите:
    \begin{itemize}
        \item высоту обрыва (вместе с ростом Даши).
        \item дальность полёта камушков (по горизонтали) и их скорость при падении, приняв начальную скорость броска равной $v = 13\,\frac{\text{м}}{\text{с}}$.
    \end{itemize}
    Сопротивлением воздуха пренебречь, $g = 10\,\frac{\text{м}}{\text{с}^{2}}$.
}
\solutionspace{120pt}

\tasknumber{6}%
\task{%
    Пять одинаковых брусков лежат на гладком горизонтальном столе.
    Масса каждого бруска равна $2\,\text{кг}$,
    причём они пронумерованы от 1 до 5 и последовательно связаны между собой невесомыми
    нерастяжимыми нитями: 1 со 2, 2 с 3 (ну и с 1) и т.д.
    Экспериментатор Глюк прикладывает постоянную горизонтальную силу $90\,\text{Н}$ к бруску с наибольшим номером.
    С каким ускорением двигается система? Чему равна сила натяжения нити, связывающей бруски 3 и 4?
}
\solutionspace{120pt}

\tasknumber{7}%
\task{%
    Два бруска связаны лёгкой нерастяжимой нитью и перекинуты через неподвижный блок (см.
    рис.).
    Определите силу натяжения нити и ускорения брусков.
    Силами трения пренебречь, массы брусков
    равны $m_1 = 8\,\text{кг}$ и $m_2 = 6\,\text{кг}$.

    \begin{tikzpicture}[x=1.5cm,y=1.5cm,thick]
        \draw
            (-0.4, 0) rectangle (-0.2, 1.2)
            (0.15, 0.5) rectangle (0.45, 1)
            (0, 2) circle [radius=0.3] -- ++(up:0.5)
            (-0.3, 1.2) -- ++(up:0.8)
            (0.3, 1) -- ++(up:1)
            (-0.7, 2.5) -- (0.7, 2.5)
            ;
        \draw[pattern={Lines[angle=51,distance=3pt]},pattern color=black,draw=none] (-0.7, 2.5) rectangle (0.7, 2.75);
        \node [left] (left) at (-0.4, 0.6) { $m_1$ };
        \node [right] (right) at (0.4, 0.75) { $m_2$ };
    \end{tikzpicture}
}

\variantsplitter

\addpersonalvariant{Евгений Васин}

\tasknumber{1}%
\task{%
    Электрон летит прямолинейно из точки $A$ в точку $B$, за ним при этом наблюдает экспериментатор Глюк.
    Глюк заметил, что первую четверть времени электрон равномерно двигался со скоростью $2 \cdot 10^{5}\,\frac{\text{км}}{\text{ч}}$,
    затем его практически мгновенно ускорило электрическое поле
    и остаток времени электрон вновь равномерно двигался со скоростью $6 \cdot 10^{5}\,\frac{\text{км}}{\text{ч}}$.
    Определите среднюю скорость электрона.
    Ответ выразите в м/с и округлите до тысяч.
}
\solutionspace{120pt}

\tasknumber{2}%
\task{%
    Женя стартует на велосипеде и в течение $5\,\text{c}$ двигается с постоянным ускорением $0{,}5\,\frac{\text{м}}{\text{с}^{2}}$.
    Определите
    \begin{itemize}
        \item какую скорость при этом удастся достичь,
        \item какой путь за это время будет пройден,
        \item среднюю скорость за всё время движения, если после начального ускорения продолжить движение равномерно ещё в течение времени $nt$
    \end{itemize}
}
\solutionspace{120pt}

\tasknumber{3}%
\task{%
    Какой путь тело пройдёт за шестую секунду после начала свободного падения?
    Какую скорость в начале этой секунды оно имеет?
}
\solutionspace{120pt}

\tasknumber{4}%
\task{%
    Карусель радиусом $5\,\text{v}$ равномерно совершает 6 оборотов в минуту.
    Определите
    \begin{itemize}
        \item период и частоту её обращения,
        \item скорость и ускорение крайних её точек.
    \end{itemize}
}
\solutionspace{120pt}

\tasknumber{5}%
\task{%
    Маша стоит на обрыве над рекой и методично и строго горизонтально кидает в неё камушки.
    За этим всем наблюдает экспериментатор Глюк, который уже выяснил, что камушки падают в реку спустя $1{,}7\,\text{с}$ после броска,
    а вот дальность полёта оценить сложнее: придётся лезть в воду.
    Выручите Глюка и определите:
    \begin{itemize}
        \item высоту обрыва (вместе с ростом Маши).
        \item дальность полёта камушков (по горизонтали) и их скорость при падении, приняв начальную скорость броска равной $v = 15\,\frac{\text{м}}{\text{с}}$.
    \end{itemize}
    Сопротивлением воздуха пренебречь, $g = 10\,\frac{\text{м}}{\text{с}^{2}}$.
}
\solutionspace{120pt}

\tasknumber{6}%
\task{%
    Шесть одинаковых брусков лежат на гладком горизонтальном столе.
    Масса каждого бруска равна $3\,\text{кг}$,
    причём они пронумерованы от 1 до 6 и последовательно связаны между собой невесомыми
    нерастяжимыми нитями: 1 со 2, 2 с 3 (ну и с 1) и т.д.
    Экспериментатор Глюк прикладывает постоянную горизонтальную силу $120\,\text{Н}$ к бруску с наибольшим номером.
    С каким ускорением двигается система? Чему равна сила натяжения нити, связывающей бруски 1 и 2?
}
\solutionspace{120pt}

\tasknumber{7}%
\task{%
    Два бруска связаны лёгкой нерастяжимой нитью и перекинуты через неподвижный блок (см.
    рис.).
    Определите силу натяжения нити и ускорения брусков.
    Силами трения пренебречь, массы брусков
    равны $m_1 = 5\,\text{кг}$ и $m_2 = 14\,\text{кг}$.

    \begin{tikzpicture}[x=1.5cm,y=1.5cm,thick]
        \draw
            (-0.4, 0) rectangle (-0.2, 1.2)
            (0.15, 0.5) rectangle (0.45, 1)
            (0, 2) circle [radius=0.3] -- ++(up:0.5)
            (-0.3, 1.2) -- ++(up:0.8)
            (0.3, 1) -- ++(up:1)
            (-0.7, 2.5) -- (0.7, 2.5)
            ;
        \draw[pattern={Lines[angle=51,distance=3pt]},pattern color=black,draw=none] (-0.7, 2.5) rectangle (0.7, 2.75);
        \node [left] (left) at (-0.4, 0.6) { $m_1$ };
        \node [right] (right) at (0.4, 0.75) { $m_2$ };
    \end{tikzpicture}
}

\variantsplitter

\addpersonalvariant{Вячеслав Волохов}

\tasknumber{1}%
\task{%
    Электрон летит прямолинейно из точки $A$ в точку $B$, за ним при этом наблюдает экспериментатор Глюк.
    Глюк заметил, что первую четверть пути электрон равномерно двигался со скоростью $4 \cdot 10^{5}\,\frac{\text{км}}{\text{ч}}$,
    затем его практически мгновенно ускорило электрическое поле
    и остаток пути электрон вновь равномерно двигался со скоростью $6 \cdot 10^{5}\,\frac{\text{км}}{\text{ч}}$.
    Определите среднюю скорость электрона.
    Ответ выразите в м/с и округлите до тысяч.
}
\solutionspace{120pt}

\tasknumber{2}%
\task{%
    Саша стартует на лошади и в течение $2\,\text{c}$ двигается с постоянным ускорением $1{,}5\,\frac{\text{м}}{\text{с}^{2}}$.
    Определите
    \begin{itemize}
        \item какую скорость при этом удастся достичь,
        \item какой путь за это время будет пройден,
        \item среднюю скорость за всё время движения, если после начального ускорения продолжить движение равномерно ещё в течение времени $nt$
    \end{itemize}
}
\solutionspace{120pt}

\tasknumber{3}%
\task{%
    Какой путь тело пройдёт за пятую секунду после начала свободного падения?
    Какую скорость в конце этой секунды оно имеет?
}
\solutionspace{120pt}

\tasknumber{4}%
\task{%
    Карусель радиусом $3\,\text{v}$ равномерно совершает 6 оборотов в минуту.
    Определите
    \begin{itemize}
        \item период и частоту её обращения,
        \item скорость и ускорение крайних её точек.
    \end{itemize}
}
\solutionspace{120pt}

\tasknumber{5}%
\task{%
    Миша стоит на обрыве над рекой и методично и строго горизонтально кидает в неё камушки.
    За этим всем наблюдает экспериментатор Глюк, который уже выяснил, что камушки падают в реку спустя $1{,}2\,\text{с}$ после броска,
    а вот дальность полёта оценить сложнее: придётся лезть в воду.
    Выручите Глюка и определите:
    \begin{itemize}
        \item высоту обрыва (вместе с ростом Миши).
        \item дальность полёта камушков (по горизонтали) и их скорость при падении, приняв начальную скорость броска равной $v = 12\,\frac{\text{м}}{\text{с}}$.
    \end{itemize}
    Сопротивлением воздуха пренебречь, $g = 10\,\frac{\text{м}}{\text{с}^{2}}$.
}
\solutionspace{120pt}

\tasknumber{6}%
\task{%
    Четыре одинаковых брусков лежат на гладком горизонтальном столе.
    Масса каждого бруска равна $3\,\text{кг}$,
    причём они пронумерованы от 1 до 4 и последовательно связаны между собой невесомыми
    нерастяжимыми нитями: 1 со 2, 2 с 3 (ну и с 1) и т.д.
    Экспериментатор Глюк прикладывает постоянную горизонтальную силу $90\,\text{Н}$ к бруску с наименьшим номером.
    С каким ускорением двигается система? Чему равна сила натяжения нити, связывающей бруски 1 и 2?
}
\solutionspace{120pt}

\tasknumber{7}%
\task{%
    Два бруска связаны лёгкой нерастяжимой нитью и перекинуты через неподвижный блок (см.
    рис.).
    Определите силу натяжения нити и ускорения брусков.
    Силами трения пренебречь, массы брусков
    равны $m_1 = 5\,\text{кг}$ и $m_2 = 6\,\text{кг}$.

    \begin{tikzpicture}[x=1.5cm,y=1.5cm,thick]
        \draw
            (-0.4, 0) rectangle (-0.2, 1.2)
            (0.15, 0.5) rectangle (0.45, 1)
            (0, 2) circle [radius=0.3] -- ++(up:0.5)
            (-0.3, 1.2) -- ++(up:0.8)
            (0.3, 1) -- ++(up:1)
            (-0.7, 2.5) -- (0.7, 2.5)
            ;
        \draw[pattern={Lines[angle=51,distance=3pt]},pattern color=black,draw=none] (-0.7, 2.5) rectangle (0.7, 2.75);
        \node [left] (left) at (-0.4, 0.6) { $m_1$ };
        \node [right] (right) at (0.4, 0.75) { $m_2$ };
    \end{tikzpicture}
}

\variantsplitter

\addpersonalvariant{Герман Говоров}

\tasknumber{1}%
\task{%
    Электрон летит прямолинейно из точки $A$ в точку $B$, за ним при этом наблюдает экспериментатор Глюк.
    Глюк заметил, что первую треть времени электрон равномерно двигался со скоростью $2 \cdot 10^{5}\,\frac{\text{км}}{\text{ч}}$,
    затем его практически мгновенно ускорило электрическое поле
    и остаток времени электрон вновь равномерно двигался со скоростью $3 \cdot 10^{5}\,\frac{\text{км}}{\text{ч}}$.
    Определите среднюю скорость электрона.
    Ответ выразите в м/с и округлите до тысяч.
}
\solutionspace{120pt}

\tasknumber{2}%
\task{%
    Женя стартует на лошади и в течение $3\,\text{c}$ двигается с постоянным ускорением $1{,}5\,\frac{\text{м}}{\text{с}^{2}}$.
    Определите
    \begin{itemize}
        \item какую скорость при этом удастся достичь,
        \item какой путь за это время будет пройден,
        \item среднюю скорость за всё время движения, если после начального ускорения продолжить движение равномерно ещё в течение времени $nt$
    \end{itemize}
}
\solutionspace{120pt}

\tasknumber{3}%
\task{%
    Какой путь тело пройдёт за шестую секунду после начала свободного падения?
    Какую скорость в начале этой секунды оно имеет?
}
\solutionspace{120pt}

\tasknumber{4}%
\task{%
    Карусель диаметром $3\,\text{v}$ равномерно совершает 5 оборотов в минуту.
    Определите
    \begin{itemize}
        \item период и частоту её обращения,
        \item скорость и ускорение крайних её точек.
    \end{itemize}
}
\solutionspace{120pt}

\tasknumber{5}%
\task{%
    Маша стоит на обрыве над рекой и методично и строго горизонтально кидает в неё камушки.
    За этим всем наблюдает экспериментатор Глюк, который уже выяснил, что камушки падают в реку спустя $1{,}7\,\text{с}$ после броска,
    а вот дальность полёта оценить сложнее: придётся лезть в воду.
    Выручите Глюка и определите:
    \begin{itemize}
        \item высоту обрыва (вместе с ростом Маши).
        \item дальность полёта камушков (по горизонтали) и их скорость при падении, приняв начальную скорость броска равной $v = 14\,\frac{\text{м}}{\text{с}}$.
    \end{itemize}
    Сопротивлением воздуха пренебречь, $g = 10\,\frac{\text{м}}{\text{с}^{2}}$.
}
\solutionspace{120pt}

\tasknumber{6}%
\task{%
    Четыре одинаковых брусков лежат на гладком горизонтальном столе.
    Масса каждого бруска равна $2\,\text{кг}$,
    причём они пронумерованы от 1 до 4 и последовательно связаны между собой невесомыми
    нерастяжимыми нитями: 1 со 2, 2 с 3 (ну и с 1) и т.д.
    Экспериментатор Глюк прикладывает постоянную горизонтальную силу $90\,\text{Н}$ к бруску с наибольшим номером.
    С каким ускорением двигается система? Чему равна сила натяжения нити, связывающей бруски 3 и 4?
}
\solutionspace{120pt}

\tasknumber{7}%
\task{%
    Два бруска связаны лёгкой нерастяжимой нитью и перекинуты через неподвижный блок (см.
    рис.).
    Определите силу натяжения нити и ускорения брусков.
    Силами трения пренебречь, массы брусков
    равны $m_1 = 5\,\text{кг}$ и $m_2 = 6\,\text{кг}$.

    \begin{tikzpicture}[x=1.5cm,y=1.5cm,thick]
        \draw
            (-0.4, 0) rectangle (-0.2, 1.2)
            (0.15, 0.5) rectangle (0.45, 1)
            (0, 2) circle [radius=0.3] -- ++(up:0.5)
            (-0.3, 1.2) -- ++(up:0.8)
            (0.3, 1) -- ++(up:1)
            (-0.7, 2.5) -- (0.7, 2.5)
            ;
        \draw[pattern={Lines[angle=51,distance=3pt]},pattern color=black,draw=none] (-0.7, 2.5) rectangle (0.7, 2.75);
        \node [left] (left) at (-0.4, 0.6) { $m_1$ };
        \node [right] (right) at (0.4, 0.75) { $m_2$ };
    \end{tikzpicture}
}

\variantsplitter

\addpersonalvariant{София Журавлёва}

\tasknumber{1}%
\task{%
    Электрон летит прямолинейно из точки $A$ в точку $B$, за ним при этом наблюдает экспериментатор Глюк.
    Глюк заметил, что первую половину пути электрон равномерно двигался со скоростью $4 \cdot 10^{5}\,\frac{\text{км}}{\text{ч}}$,
    затем его практически мгновенно ускорило электрическое поле
    и остаток пути электрон вновь равномерно двигался со скоростью $6 \cdot 10^{5}\,\frac{\text{км}}{\text{ч}}$.
    Определите среднюю скорость электрона.
    Ответ выразите в м/с и округлите до тысяч.
}
\solutionspace{120pt}

\tasknumber{2}%
\task{%
    Саша стартует на лошади и в течение $5\,\text{c}$ двигается с постоянным ускорением $2{,}5\,\frac{\text{м}}{\text{с}^{2}}$.
    Определите
    \begin{itemize}
        \item какую скорость при этом удастся достичь,
        \item какой путь за это время будет пройден,
        \item среднюю скорость за всё время движения, если после начального ускорения продолжить движение равномерно ещё в течение времени $nt$
    \end{itemize}
}
\solutionspace{120pt}

\tasknumber{3}%
\task{%
    Какой путь тело пройдёт за третью секунду после начала свободного падения?
    Какую скорость в конце этой секунды оно имеет?
}
\solutionspace{120pt}

\tasknumber{4}%
\task{%
    Карусель диаметром $2\,\text{v}$ равномерно совершает 10 оборотов в минуту.
    Определите
    \begin{itemize}
        \item период и частоту её обращения,
        \item скорость и ускорение крайних её точек.
    \end{itemize}
}
\solutionspace{120pt}

\tasknumber{5}%
\task{%
    Паша стоит на обрыве над рекой и методично и строго горизонтально кидает в неё камушки.
    За этим всем наблюдает экспериментатор Глюк, который уже выяснил, что камушки падают в реку спустя $1{,}4\,\text{с}$ после броска,
    а вот дальность полёта оценить сложнее: придётся лезть в воду.
    Выручите Глюка и определите:
    \begin{itemize}
        \item высоту обрыва (вместе с ростом Паши).
        \item дальность полёта камушков (по горизонтали) и их скорость при падении, приняв начальную скорость броска равной $v = 16\,\frac{\text{м}}{\text{с}}$.
    \end{itemize}
    Сопротивлением воздуха пренебречь, $g = 10\,\frac{\text{м}}{\text{с}^{2}}$.
}
\solutionspace{120pt}

\tasknumber{6}%
\task{%
    Шесть одинаковых брусков лежат на гладком горизонтальном столе.
    Масса каждого бруска равна $2\,\text{кг}$,
    причём они пронумерованы от 1 до 6 и последовательно связаны между собой невесомыми
    нерастяжимыми нитями: 1 со 2, 2 с 3 (ну и с 1) и т.д.
    Экспериментатор Глюк прикладывает постоянную горизонтальную силу $60\,\text{Н}$ к бруску с наименьшим номером.
    С каким ускорением двигается система? Чему равна сила натяжения нити, связывающей бруски 1 и 2?
}
\solutionspace{120pt}

\tasknumber{7}%
\task{%
    Два бруска связаны лёгкой нерастяжимой нитью и перекинуты через неподвижный блок (см.
    рис.).
    Определите силу натяжения нити и ускорения брусков.
    Силами трения пренебречь, массы брусков
    равны $m_1 = 11\,\text{кг}$ и $m_2 = 10\,\text{кг}$.

    \begin{tikzpicture}[x=1.5cm,y=1.5cm,thick]
        \draw
            (-0.4, 0) rectangle (-0.2, 1.2)
            (0.15, 0.5) rectangle (0.45, 1)
            (0, 2) circle [radius=0.3] -- ++(up:0.5)
            (-0.3, 1.2) -- ++(up:0.8)
            (0.3, 1) -- ++(up:1)
            (-0.7, 2.5) -- (0.7, 2.5)
            ;
        \draw[pattern={Lines[angle=51,distance=3pt]},pattern color=black,draw=none] (-0.7, 2.5) rectangle (0.7, 2.75);
        \node [left] (left) at (-0.4, 0.6) { $m_1$ };
        \node [right] (right) at (0.4, 0.75) { $m_2$ };
    \end{tikzpicture}
}

\variantsplitter

\addpersonalvariant{Константин Козлов}

\tasknumber{1}%
\task{%
    Электрон летит прямолинейно из точки $A$ в точку $B$, за ним при этом наблюдает экспериментатор Глюк.
    Глюк заметил, что первую треть времени электрон равномерно двигался со скоростью $2 \cdot 10^{5}\,\frac{\text{км}}{\text{ч}}$,
    затем его практически мгновенно ускорило электрическое поле
    и остаток времени электрон вновь равномерно двигался со скоростью $6 \cdot 10^{5}\,\frac{\text{км}}{\text{ч}}$.
    Определите среднюю скорость электрона.
    Ответ выразите в м/с и округлите до тысяч.
}
\solutionspace{120pt}

\tasknumber{2}%
\task{%
    Женя стартует на велосипеде и в течение $3\,\text{c}$ двигается с постоянным ускорением $2{,}5\,\frac{\text{м}}{\text{с}^{2}}$.
    Определите
    \begin{itemize}
        \item какую скорость при этом удастся достичь,
        \item какой путь за это время будет пройден,
        \item среднюю скорость за всё время движения, если после начального ускорения продолжить движение равномерно ещё в течение времени $nt$
    \end{itemize}
}
\solutionspace{120pt}

\tasknumber{3}%
\task{%
    Какой путь тело пройдёт за четвёртую секунду после начала свободного падения?
    Какую скорость в начале этой секунды оно имеет?
}
\solutionspace{120pt}

\tasknumber{4}%
\task{%
    Карусель диаметром $2\,\text{v}$ равномерно совершает 6 оборотов в минуту.
    Определите
    \begin{itemize}
        \item период и частоту её обращения,
        \item скорость и ускорение крайних её точек.
    \end{itemize}
}
\solutionspace{120pt}

\tasknumber{5}%
\task{%
    Миша стоит на обрыве над рекой и методично и строго горизонтально кидает в неё камушки.
    За этим всем наблюдает экспериментатор Глюк, который уже выяснил, что камушки падают в реку спустя $1{,}6\,\text{с}$ после броска,
    а вот дальность полёта оценить сложнее: придётся лезть в воду.
    Выручите Глюка и определите:
    \begin{itemize}
        \item высоту обрыва (вместе с ростом Миши).
        \item дальность полёта камушков (по горизонтали) и их скорость при падении, приняв начальную скорость броска равной $v = 16\,\frac{\text{м}}{\text{с}}$.
    \end{itemize}
    Сопротивлением воздуха пренебречь, $g = 10\,\frac{\text{м}}{\text{с}^{2}}$.
}
\solutionspace{120pt}

\tasknumber{6}%
\task{%
    Шесть одинаковых брусков лежат на гладком горизонтальном столе.
    Масса каждого бруска равна $2\,\text{кг}$,
    причём они пронумерованы от 1 до 6 и последовательно связаны между собой невесомыми
    нерастяжимыми нитями: 1 со 2, 2 с 3 (ну и с 1) и т.д.
    Экспериментатор Глюк прикладывает постоянную горизонтальную силу $90\,\text{Н}$ к бруску с наименьшим номером.
    С каким ускорением двигается система? Чему равна сила натяжения нити, связывающей бруски 3 и 4?
}
\solutionspace{120pt}

\tasknumber{7}%
\task{%
    Два бруска связаны лёгкой нерастяжимой нитью и перекинуты через неподвижный блок (см.
    рис.).
    Определите силу натяжения нити и ускорения брусков.
    Силами трения пренебречь, массы брусков
    равны $m_1 = 11\,\text{кг}$ и $m_2 = 10\,\text{кг}$.

    \begin{tikzpicture}[x=1.5cm,y=1.5cm,thick]
        \draw
            (-0.4, 0) rectangle (-0.2, 1.2)
            (0.15, 0.5) rectangle (0.45, 1)
            (0, 2) circle [radius=0.3] -- ++(up:0.5)
            (-0.3, 1.2) -- ++(up:0.8)
            (0.3, 1) -- ++(up:1)
            (-0.7, 2.5) -- (0.7, 2.5)
            ;
        \draw[pattern={Lines[angle=51,distance=3pt]},pattern color=black,draw=none] (-0.7, 2.5) rectangle (0.7, 2.75);
        \node [left] (left) at (-0.4, 0.6) { $m_1$ };
        \node [right] (right) at (0.4, 0.75) { $m_2$ };
    \end{tikzpicture}
}

\variantsplitter

\addpersonalvariant{Наталья Кравченко}

\tasknumber{1}%
\task{%
    Электрон летит прямолинейно из точки $A$ в точку $B$, за ним при этом наблюдает экспериментатор Глюк.
    Глюк заметил, что первую четверть пути электрон равномерно двигался со скоростью $4 \cdot 10^{5}\,\frac{\text{км}}{\text{ч}}$,
    затем его практически мгновенно ускорило электрическое поле
    и остаток пути электрон вновь равномерно двигался со скоростью $6 \cdot 10^{5}\,\frac{\text{км}}{\text{ч}}$.
    Определите среднюю скорость электрона.
    Ответ выразите в м/с и округлите до тысяч.
}
\solutionspace{120pt}

\tasknumber{2}%
\task{%
    Саша стартует на лошади и в течение $3\,\text{c}$ двигается с постоянным ускорением $0{,}5\,\frac{\text{м}}{\text{с}^{2}}$.
    Определите
    \begin{itemize}
        \item какую скорость при этом удастся достичь,
        \item какой путь за это время будет пройден,
        \item среднюю скорость за всё время движения, если после начального ускорения продолжить движение равномерно ещё в течение времени $nt$
    \end{itemize}
}
\solutionspace{120pt}

\tasknumber{3}%
\task{%
    Какой путь тело пройдёт за шестую секунду после начала свободного падения?
    Какую скорость в конце этой секунды оно имеет?
}
\solutionspace{120pt}

\tasknumber{4}%
\task{%
    Карусель диаметром $2\,\text{v}$ равномерно совершает 6 оборотов в минуту.
    Определите
    \begin{itemize}
        \item период и частоту её обращения,
        \item скорость и ускорение крайних её точек.
    \end{itemize}
}
\solutionspace{120pt}

\tasknumber{5}%
\task{%
    Маша стоит на обрыве над рекой и методично и строго горизонтально кидает в неё камушки.
    За этим всем наблюдает экспериментатор Глюк, который уже выяснил, что камушки падают в реку спустя $1{,}2\,\text{с}$ после броска,
    а вот дальность полёта оценить сложнее: придётся лезть в воду.
    Выручите Глюка и определите:
    \begin{itemize}
        \item высоту обрыва (вместе с ростом Маши).
        \item дальность полёта камушков (по горизонтали) и их скорость при падении, приняв начальную скорость броска равной $v = 15\,\frac{\text{м}}{\text{с}}$.
    \end{itemize}
    Сопротивлением воздуха пренебречь, $g = 10\,\frac{\text{м}}{\text{с}^{2}}$.
}
\solutionspace{120pt}

\tasknumber{6}%
\task{%
    Шесть одинаковых брусков лежат на гладком горизонтальном столе.
    Масса каждого бруска равна $2\,\text{кг}$,
    причём они пронумерованы от 1 до 6 и последовательно связаны между собой невесомыми
    нерастяжимыми нитями: 1 со 2, 2 с 3 (ну и с 1) и т.д.
    Экспериментатор Глюк прикладывает постоянную горизонтальную силу $60\,\text{Н}$ к бруску с наибольшим номером.
    С каким ускорением двигается система? Чему равна сила натяжения нити, связывающей бруски 3 и 4?
}
\solutionspace{120pt}

\tasknumber{7}%
\task{%
    Два бруска связаны лёгкой нерастяжимой нитью и перекинуты через неподвижный блок (см.
    рис.).
    Определите силу натяжения нити и ускорения брусков.
    Силами трения пренебречь, массы брусков
    равны $m_1 = 8\,\text{кг}$ и $m_2 = 6\,\text{кг}$.

    \begin{tikzpicture}[x=1.5cm,y=1.5cm,thick]
        \draw
            (-0.4, 0) rectangle (-0.2, 1.2)
            (0.15, 0.5) rectangle (0.45, 1)
            (0, 2) circle [radius=0.3] -- ++(up:0.5)
            (-0.3, 1.2) -- ++(up:0.8)
            (0.3, 1) -- ++(up:1)
            (-0.7, 2.5) -- (0.7, 2.5)
            ;
        \draw[pattern={Lines[angle=51,distance=3pt]},pattern color=black,draw=none] (-0.7, 2.5) rectangle (0.7, 2.75);
        \node [left] (left) at (-0.4, 0.6) { $m_1$ };
        \node [right] (right) at (0.4, 0.75) { $m_2$ };
    \end{tikzpicture}
}

\variantsplitter

\addpersonalvariant{Матвей Кузьмин}

\tasknumber{1}%
\task{%
    Электрон летит прямолинейно из точки $A$ в точку $B$, за ним при этом наблюдает экспериментатор Глюк.
    Глюк заметил, что первую четверть пути электрон равномерно двигался со скоростью $4 \cdot 10^{5}\,\frac{\text{км}}{\text{ч}}$,
    затем его практически мгновенно ускорило электрическое поле
    и остаток пути электрон вновь равномерно двигался со скоростью $6 \cdot 10^{5}\,\frac{\text{км}}{\text{ч}}$.
    Определите среднюю скорость электрона.
    Ответ выразите в м/с и округлите до тысяч.
}
\solutionspace{120pt}

\tasknumber{2}%
\task{%
    Саша стартует на лошади и в течение $10\,\text{c}$ двигается с постоянным ускорением $1{,}5\,\frac{\text{м}}{\text{с}^{2}}$.
    Определите
    \begin{itemize}
        \item какую скорость при этом удастся достичь,
        \item какой путь за это время будет пройден,
        \item среднюю скорость за всё время движения, если после начального ускорения продолжить движение равномерно ещё в течение времени $nt$
    \end{itemize}
}
\solutionspace{120pt}

\tasknumber{3}%
\task{%
    Какой путь тело пройдёт за вторую секунду после начала свободного падения?
    Какую скорость в конце этой секунды оно имеет?
}
\solutionspace{120pt}

\tasknumber{4}%
\task{%
    Карусель диаметром $3\,\text{v}$ равномерно совершает 10 оборотов в минуту.
    Определите
    \begin{itemize}
        \item период и частоту её обращения,
        \item скорость и ускорение крайних её точек.
    \end{itemize}
}
\solutionspace{120pt}

\tasknumber{5}%
\task{%
    Даша стоит на обрыве над рекой и методично и строго горизонтально кидает в неё камушки.
    За этим всем наблюдает экспериментатор Глюк, который уже выяснил, что камушки падают в реку спустя $1{,}6\,\text{с}$ после броска,
    а вот дальность полёта оценить сложнее: придётся лезть в воду.
    Выручите Глюка и определите:
    \begin{itemize}
        \item высоту обрыва (вместе с ростом Даши).
        \item дальность полёта камушков (по горизонтали) и их скорость при падении, приняв начальную скорость броска равной $v = 13\,\frac{\text{м}}{\text{с}}$.
    \end{itemize}
    Сопротивлением воздуха пренебречь, $g = 10\,\frac{\text{м}}{\text{с}^{2}}$.
}
\solutionspace{120pt}

\tasknumber{6}%
\task{%
    Пять одинаковых брусков лежат на гладком горизонтальном столе.
    Масса каждого бруска равна $2\,\text{кг}$,
    причём они пронумерованы от 1 до 5 и последовательно связаны между собой невесомыми
    нерастяжимыми нитями: 1 со 2, 2 с 3 (ну и с 1) и т.д.
    Экспериментатор Глюк прикладывает постоянную горизонтальную силу $120\,\text{Н}$ к бруску с наибольшим номером.
    С каким ускорением двигается система? Чему равна сила натяжения нити, связывающей бруски 2 и 3?
}
\solutionspace{120pt}

\tasknumber{7}%
\task{%
    Два бруска связаны лёгкой нерастяжимой нитью и перекинуты через неподвижный блок (см.
    рис.).
    Определите силу натяжения нити и ускорения брусков.
    Силами трения пренебречь, массы брусков
    равны $m_1 = 11\,\text{кг}$ и $m_2 = 10\,\text{кг}$.

    \begin{tikzpicture}[x=1.5cm,y=1.5cm,thick]
        \draw
            (-0.4, 0) rectangle (-0.2, 1.2)
            (0.15, 0.5) rectangle (0.45, 1)
            (0, 2) circle [radius=0.3] -- ++(up:0.5)
            (-0.3, 1.2) -- ++(up:0.8)
            (0.3, 1) -- ++(up:1)
            (-0.7, 2.5) -- (0.7, 2.5)
            ;
        \draw[pattern={Lines[angle=51,distance=3pt]},pattern color=black,draw=none] (-0.7, 2.5) rectangle (0.7, 2.75);
        \node [left] (left) at (-0.4, 0.6) { $m_1$ };
        \node [right] (right) at (0.4, 0.75) { $m_2$ };
    \end{tikzpicture}
}

\variantsplitter

\addpersonalvariant{Сергей Малышев}

\tasknumber{1}%
\task{%
    Электрон летит прямолинейно из точки $A$ в точку $B$, за ним при этом наблюдает экспериментатор Глюк.
    Глюк заметил, что первую половину пути электрон равномерно двигался со скоростью $4 \cdot 10^{5}\,\frac{\text{км}}{\text{ч}}$,
    затем его практически мгновенно ускорило электрическое поле
    и остаток пути электрон вновь равномерно двигался со скоростью $3 \cdot 10^{5}\,\frac{\text{км}}{\text{ч}}$.
    Определите среднюю скорость электрона.
    Ответ выразите в м/с и округлите до тысяч.
}
\solutionspace{120pt}

\tasknumber{2}%
\task{%
    Валя стартует на лошади и в течение $4\,\text{c}$ двигается с постоянным ускорением $2{,}5\,\frac{\text{м}}{\text{с}^{2}}$.
    Определите
    \begin{itemize}
        \item какую скорость при этом удастся достичь,
        \item какой путь за это время будет пройден,
        \item среднюю скорость за всё время движения, если после начального ускорения продолжить движение равномерно ещё в течение времени $nt$
    \end{itemize}
}
\solutionspace{120pt}

\tasknumber{3}%
\task{%
    Какой путь тело пройдёт за пятую секунду после начала свободного падения?
    Какую скорость в начале этой секунды оно имеет?
}
\solutionspace{120pt}

\tasknumber{4}%
\task{%
    Карусель диаметром $2\,\text{v}$ равномерно совершает 6 оборотов в минуту.
    Определите
    \begin{itemize}
        \item период и частоту её обращения,
        \item скорость и ускорение крайних её точек.
    \end{itemize}
}
\solutionspace{120pt}

\tasknumber{5}%
\task{%
    Даша стоит на обрыве над рекой и методично и строго горизонтально кидает в неё камушки.
    За этим всем наблюдает экспериментатор Глюк, который уже выяснил, что камушки падают в реку спустя $1{,}6\,\text{с}$ после броска,
    а вот дальность полёта оценить сложнее: придётся лезть в воду.
    Выручите Глюка и определите:
    \begin{itemize}
        \item высоту обрыва (вместе с ростом Даши).
        \item дальность полёта камушков (по горизонтали) и их скорость при падении, приняв начальную скорость броска равной $v = 14\,\frac{\text{м}}{\text{с}}$.
    \end{itemize}
    Сопротивлением воздуха пренебречь, $g = 10\,\frac{\text{м}}{\text{с}^{2}}$.
}
\solutionspace{120pt}

\tasknumber{6}%
\task{%
    Четыре одинаковых брусков лежат на гладком горизонтальном столе.
    Масса каждого бруска равна $2\,\text{кг}$,
    причём они пронумерованы от 1 до 4 и последовательно связаны между собой невесомыми
    нерастяжимыми нитями: 1 со 2, 2 с 3 (ну и с 1) и т.д.
    Экспериментатор Глюк прикладывает постоянную горизонтальную силу $60\,\text{Н}$ к бруску с наибольшим номером.
    С каким ускорением двигается система? Чему равна сила натяжения нити, связывающей бруски 1 и 2?
}
\solutionspace{120pt}

\tasknumber{7}%
\task{%
    Два бруска связаны лёгкой нерастяжимой нитью и перекинуты через неподвижный блок (см.
    рис.).
    Определите силу натяжения нити и ускорения брусков.
    Силами трения пренебречь, массы брусков
    равны $m_1 = 8\,\text{кг}$ и $m_2 = 14\,\text{кг}$.

    \begin{tikzpicture}[x=1.5cm,y=1.5cm,thick]
        \draw
            (-0.4, 0) rectangle (-0.2, 1.2)
            (0.15, 0.5) rectangle (0.45, 1)
            (0, 2) circle [radius=0.3] -- ++(up:0.5)
            (-0.3, 1.2) -- ++(up:0.8)
            (0.3, 1) -- ++(up:1)
            (-0.7, 2.5) -- (0.7, 2.5)
            ;
        \draw[pattern={Lines[angle=51,distance=3pt]},pattern color=black,draw=none] (-0.7, 2.5) rectangle (0.7, 2.75);
        \node [left] (left) at (-0.4, 0.6) { $m_1$ };
        \node [right] (right) at (0.4, 0.75) { $m_2$ };
    \end{tikzpicture}
}

\variantsplitter

\addpersonalvariant{Алина Полканова}

\tasknumber{1}%
\task{%
    Электрон летит прямолинейно из точки $A$ в точку $B$, за ним при этом наблюдает экспериментатор Глюк.
    Глюк заметил, что первую треть времени электрон равномерно двигался со скоростью $2 \cdot 10^{5}\,\frac{\text{км}}{\text{ч}}$,
    затем его практически мгновенно ускорило электрическое поле
    и остаток времени электрон вновь равномерно двигался со скоростью $3 \cdot 10^{5}\,\frac{\text{км}}{\text{ч}}$.
    Определите среднюю скорость электрона.
    Ответ выразите в м/с и округлите до тысяч.
}
\solutionspace{120pt}

\tasknumber{2}%
\task{%
    Женя стартует на лошади и в течение $4\,\text{c}$ двигается с постоянным ускорением $2\,\frac{\text{м}}{\text{с}^{2}}$.
    Определите
    \begin{itemize}
        \item какую скорость при этом удастся достичь,
        \item какой путь за это время будет пройден,
        \item среднюю скорость за всё время движения, если после начального ускорения продолжить движение равномерно ещё в течение времени $nt$
    \end{itemize}
}
\solutionspace{120pt}

\tasknumber{3}%
\task{%
    Какой путь тело пройдёт за третью секунду после начала свободного падения?
    Какую скорость в начале этой секунды оно имеет?
}
\solutionspace{120pt}

\tasknumber{4}%
\task{%
    Карусель радиусом $4\,\text{v}$ равномерно совершает 10 оборотов в минуту.
    Определите
    \begin{itemize}
        \item период и частоту её обращения,
        \item скорость и ускорение крайних её точек.
    \end{itemize}
}
\solutionspace{120pt}

\tasknumber{5}%
\task{%
    Маша стоит на обрыве над рекой и методично и строго горизонтально кидает в неё камушки.
    За этим всем наблюдает экспериментатор Глюк, который уже выяснил, что камушки падают в реку спустя $1{,}6\,\text{с}$ после броска,
    а вот дальность полёта оценить сложнее: придётся лезть в воду.
    Выручите Глюка и определите:
    \begin{itemize}
        \item высоту обрыва (вместе с ростом Маши).
        \item дальность полёта камушков (по горизонтали) и их скорость при падении, приняв начальную скорость броска равной $v = 12\,\frac{\text{м}}{\text{с}}$.
    \end{itemize}
    Сопротивлением воздуха пренебречь, $g = 10\,\frac{\text{м}}{\text{с}^{2}}$.
}
\solutionspace{120pt}

\tasknumber{6}%
\task{%
    Шесть одинаковых брусков лежат на гладком горизонтальном столе.
    Масса каждого бруска равна $2\,\text{кг}$,
    причём они пронумерованы от 1 до 6 и последовательно связаны между собой невесомыми
    нерастяжимыми нитями: 1 со 2, 2 с 3 (ну и с 1) и т.д.
    Экспериментатор Глюк прикладывает постоянную горизонтальную силу $120\,\text{Н}$ к бруску с наименьшим номером.
    С каким ускорением двигается система? Чему равна сила натяжения нити, связывающей бруски 1 и 2?
}
\solutionspace{120pt}

\tasknumber{7}%
\task{%
    Два бруска связаны лёгкой нерастяжимой нитью и перекинуты через неподвижный блок (см.
    рис.).
    Определите силу натяжения нити и ускорения брусков.
    Силами трения пренебречь, массы брусков
    равны $m_1 = 5\,\text{кг}$ и $m_2 = 14\,\text{кг}$.

    \begin{tikzpicture}[x=1.5cm,y=1.5cm,thick]
        \draw
            (-0.4, 0) rectangle (-0.2, 1.2)
            (0.15, 0.5) rectangle (0.45, 1)
            (0, 2) circle [radius=0.3] -- ++(up:0.5)
            (-0.3, 1.2) -- ++(up:0.8)
            (0.3, 1) -- ++(up:1)
            (-0.7, 2.5) -- (0.7, 2.5)
            ;
        \draw[pattern={Lines[angle=51,distance=3pt]},pattern color=black,draw=none] (-0.7, 2.5) rectangle (0.7, 2.75);
        \node [left] (left) at (-0.4, 0.6) { $m_1$ };
        \node [right] (right) at (0.4, 0.75) { $m_2$ };
    \end{tikzpicture}
}

\variantsplitter

\addpersonalvariant{Сергей Пономарёв}

\tasknumber{1}%
\task{%
    Электрон летит прямолинейно из точки $A$ в точку $B$, за ним при этом наблюдает экспериментатор Глюк.
    Глюк заметил, что первую четверть пути электрон равномерно двигался со скоростью $2 \cdot 10^{5}\,\frac{\text{км}}{\text{ч}}$,
    затем его практически мгновенно ускорило электрическое поле
    и остаток пути электрон вновь равномерно двигался со скоростью $6 \cdot 10^{5}\,\frac{\text{км}}{\text{ч}}$.
    Определите среднюю скорость электрона.
    Ответ выразите в м/с и округлите до тысяч.
}
\solutionspace{120pt}

\tasknumber{2}%
\task{%
    Валя стартует на велосипеде и в течение $2\,\text{c}$ двигается с постоянным ускорением $0{,}5\,\frac{\text{м}}{\text{с}^{2}}$.
    Определите
    \begin{itemize}
        \item какую скорость при этом удастся достичь,
        \item какой путь за это время будет пройден,
        \item среднюю скорость за всё время движения, если после начального ускорения продолжить движение равномерно ещё в течение времени $nt$
    \end{itemize}
}
\solutionspace{120pt}

\tasknumber{3}%
\task{%
    Какой путь тело пройдёт за вторую секунду после начала свободного падения?
    Какую скорость в начале этой секунды оно имеет?
}
\solutionspace{120pt}

\tasknumber{4}%
\task{%
    Карусель диаметром $5\,\text{v}$ равномерно совершает 6 оборотов в минуту.
    Определите
    \begin{itemize}
        \item период и частоту её обращения,
        \item скорость и ускорение крайних её точек.
    \end{itemize}
}
\solutionspace{120pt}

\tasknumber{5}%
\task{%
    Даша стоит на обрыве над рекой и методично и строго горизонтально кидает в неё камушки.
    За этим всем наблюдает экспериментатор Глюк, который уже выяснил, что камушки падают в реку спустя $1{,}3\,\text{с}$ после броска,
    а вот дальность полёта оценить сложнее: придётся лезть в воду.
    Выручите Глюка и определите:
    \begin{itemize}
        \item высоту обрыва (вместе с ростом Даши).
        \item дальность полёта камушков (по горизонтали) и их скорость при падении, приняв начальную скорость броска равной $v = 15\,\frac{\text{м}}{\text{с}}$.
    \end{itemize}
    Сопротивлением воздуха пренебречь, $g = 10\,\frac{\text{м}}{\text{с}^{2}}$.
}
\solutionspace{120pt}

\tasknumber{6}%
\task{%
    Шесть одинаковых брусков лежат на гладком горизонтальном столе.
    Масса каждого бруска равна $2\,\text{кг}$,
    причём они пронумерованы от 1 до 6 и последовательно связаны между собой невесомыми
    нерастяжимыми нитями: 1 со 2, 2 с 3 (ну и с 1) и т.д.
    Экспериментатор Глюк прикладывает постоянную горизонтальную силу $120\,\text{Н}$ к бруску с наибольшим номером.
    С каким ускорением двигается система? Чему равна сила натяжения нити, связывающей бруски 3 и 4?
}
\solutionspace{120pt}

\tasknumber{7}%
\task{%
    Два бруска связаны лёгкой нерастяжимой нитью и перекинуты через неподвижный блок (см.
    рис.).
    Определите силу натяжения нити и ускорения брусков.
    Силами трения пренебречь, массы брусков
    равны $m_1 = 5\,\text{кг}$ и $m_2 = 10\,\text{кг}$.

    \begin{tikzpicture}[x=1.5cm,y=1.5cm,thick]
        \draw
            (-0.4, 0) rectangle (-0.2, 1.2)
            (0.15, 0.5) rectangle (0.45, 1)
            (0, 2) circle [radius=0.3] -- ++(up:0.5)
            (-0.3, 1.2) -- ++(up:0.8)
            (0.3, 1) -- ++(up:1)
            (-0.7, 2.5) -- (0.7, 2.5)
            ;
        \draw[pattern={Lines[angle=51,distance=3pt]},pattern color=black,draw=none] (-0.7, 2.5) rectangle (0.7, 2.75);
        \node [left] (left) at (-0.4, 0.6) { $m_1$ };
        \node [right] (right) at (0.4, 0.75) { $m_2$ };
    \end{tikzpicture}
}

\variantsplitter

\addpersonalvariant{Егор Свистушкин}

\tasknumber{1}%
\task{%
    Электрон летит прямолинейно из точки $A$ в точку $B$, за ним при этом наблюдает экспериментатор Глюк.
    Глюк заметил, что первую половину пути электрон равномерно двигался со скоростью $2 \cdot 10^{5}\,\frac{\text{км}}{\text{ч}}$,
    затем его практически мгновенно ускорило электрическое поле
    и остаток пути электрон вновь равномерно двигался со скоростью $6 \cdot 10^{5}\,\frac{\text{км}}{\text{ч}}$.
    Определите среднюю скорость электрона.
    Ответ выразите в м/с и округлите до тысяч.
}
\solutionspace{120pt}

\tasknumber{2}%
\task{%
    Саша стартует на лошади и в течение $10\,\text{c}$ двигается с постоянным ускорением $2\,\frac{\text{м}}{\text{с}^{2}}$.
    Определите
    \begin{itemize}
        \item какую скорость при этом удастся достичь,
        \item какой путь за это время будет пройден,
        \item среднюю скорость за всё время движения, если после начального ускорения продолжить движение равномерно ещё в течение времени $nt$
    \end{itemize}
}
\solutionspace{120pt}

\tasknumber{3}%
\task{%
    Какой путь тело пройдёт за пятую секунду после начала свободного падения?
    Какую скорость в конце этой секунды оно имеет?
}
\solutionspace{120pt}

\tasknumber{4}%
\task{%
    Карусель радиусом $3\,\text{v}$ равномерно совершает 6 оборотов в минуту.
    Определите
    \begin{itemize}
        \item период и частоту её обращения,
        \item скорость и ускорение крайних её точек.
    \end{itemize}
}
\solutionspace{120pt}

\tasknumber{5}%
\task{%
    Миша стоит на обрыве над рекой и методично и строго горизонтально кидает в неё камушки.
    За этим всем наблюдает экспериментатор Глюк, который уже выяснил, что камушки падают в реку спустя $1{,}7\,\text{с}$ после броска,
    а вот дальность полёта оценить сложнее: придётся лезть в воду.
    Выручите Глюка и определите:
    \begin{itemize}
        \item высоту обрыва (вместе с ростом Миши).
        \item дальность полёта камушков (по горизонтали) и их скорость при падении, приняв начальную скорость броска равной $v = 17\,\frac{\text{м}}{\text{с}}$.
    \end{itemize}
    Сопротивлением воздуха пренебречь, $g = 10\,\frac{\text{м}}{\text{с}^{2}}$.
}
\solutionspace{120pt}

\tasknumber{6}%
\task{%
    Пять одинаковых брусков лежат на гладком горизонтальном столе.
    Масса каждого бруска равна $2\,\text{кг}$,
    причём они пронумерованы от 1 до 5 и последовательно связаны между собой невесомыми
    нерастяжимыми нитями: 1 со 2, 2 с 3 (ну и с 1) и т.д.
    Экспериментатор Глюк прикладывает постоянную горизонтальную силу $60\,\text{Н}$ к бруску с наибольшим номером.
    С каким ускорением двигается система? Чему равна сила натяжения нити, связывающей бруски 2 и 3?
}
\solutionspace{120pt}

\tasknumber{7}%
\task{%
    Два бруска связаны лёгкой нерастяжимой нитью и перекинуты через неподвижный блок (см.
    рис.).
    Определите силу натяжения нити и ускорения брусков.
    Силами трения пренебречь, массы брусков
    равны $m_1 = 11\,\text{кг}$ и $m_2 = 10\,\text{кг}$.

    \begin{tikzpicture}[x=1.5cm,y=1.5cm,thick]
        \draw
            (-0.4, 0) rectangle (-0.2, 1.2)
            (0.15, 0.5) rectangle (0.45, 1)
            (0, 2) circle [radius=0.3] -- ++(up:0.5)
            (-0.3, 1.2) -- ++(up:0.8)
            (0.3, 1) -- ++(up:1)
            (-0.7, 2.5) -- (0.7, 2.5)
            ;
        \draw[pattern={Lines[angle=51,distance=3pt]},pattern color=black,draw=none] (-0.7, 2.5) rectangle (0.7, 2.75);
        \node [left] (left) at (-0.4, 0.6) { $m_1$ };
        \node [right] (right) at (0.4, 0.75) { $m_2$ };
    \end{tikzpicture}
}

\variantsplitter

\addpersonalvariant{Дмитрий Соколов}

\tasknumber{1}%
\task{%
    Электрон летит прямолинейно из точки $A$ в точку $B$, за ним при этом наблюдает экспериментатор Глюк.
    Глюк заметил, что первую четверть пути электрон равномерно двигался со скоростью $2 \cdot 10^{5}\,\frac{\text{км}}{\text{ч}}$,
    затем его практически мгновенно ускорило электрическое поле
    и остаток пути электрон вновь равномерно двигался со скоростью $6 \cdot 10^{5}\,\frac{\text{км}}{\text{ч}}$.
    Определите среднюю скорость электрона.
    Ответ выразите в м/с и округлите до тысяч.
}
\solutionspace{120pt}

\tasknumber{2}%
\task{%
    Валя стартует на лошади и в течение $5\,\text{c}$ двигается с постоянным ускорением $1{,}5\,\frac{\text{м}}{\text{с}^{2}}$.
    Определите
    \begin{itemize}
        \item какую скорость при этом удастся достичь,
        \item какой путь за это время будет пройден,
        \item среднюю скорость за всё время движения, если после начального ускорения продолжить движение равномерно ещё в течение времени $nt$
    \end{itemize}
}
\solutionspace{120pt}

\tasknumber{3}%
\task{%
    Какой путь тело пройдёт за третью секунду после начала свободного падения?
    Какую скорость в конце этой секунды оно имеет?
}
\solutionspace{120pt}

\tasknumber{4}%
\task{%
    Карусель диаметром $2\,\text{v}$ равномерно совершает 6 оборотов в минуту.
    Определите
    \begin{itemize}
        \item период и частоту её обращения,
        \item скорость и ускорение крайних её точек.
    \end{itemize}
}
\solutionspace{120pt}

\tasknumber{5}%
\task{%
    Паша стоит на обрыве над рекой и методично и строго горизонтально кидает в неё камушки.
    За этим всем наблюдает экспериментатор Глюк, который уже выяснил, что камушки падают в реку спустя $1{,}7\,\text{с}$ после броска,
    а вот дальность полёта оценить сложнее: придётся лезть в воду.
    Выручите Глюка и определите:
    \begin{itemize}
        \item высоту обрыва (вместе с ростом Паши).
        \item дальность полёта камушков (по горизонтали) и их скорость при падении, приняв начальную скорость броска равной $v = 17\,\frac{\text{м}}{\text{с}}$.
    \end{itemize}
    Сопротивлением воздуха пренебречь, $g = 10\,\frac{\text{м}}{\text{с}^{2}}$.
}
\solutionspace{120pt}

\tasknumber{6}%
\task{%
    Шесть одинаковых брусков лежат на гладком горизонтальном столе.
    Масса каждого бруска равна $2\,\text{кг}$,
    причём они пронумерованы от 1 до 6 и последовательно связаны между собой невесомыми
    нерастяжимыми нитями: 1 со 2, 2 с 3 (ну и с 1) и т.д.
    Экспериментатор Глюк прикладывает постоянную горизонтальную силу $90\,\text{Н}$ к бруску с наибольшим номером.
    С каким ускорением двигается система? Чему равна сила натяжения нити, связывающей бруски 2 и 3?
}
\solutionspace{120pt}

\tasknumber{7}%
\task{%
    Два бруска связаны лёгкой нерастяжимой нитью и перекинуты через неподвижный блок (см.
    рис.).
    Определите силу натяжения нити и ускорения брусков.
    Силами трения пренебречь, массы брусков
    равны $m_1 = 11\,\text{кг}$ и $m_2 = 14\,\text{кг}$.

    \begin{tikzpicture}[x=1.5cm,y=1.5cm,thick]
        \draw
            (-0.4, 0) rectangle (-0.2, 1.2)
            (0.15, 0.5) rectangle (0.45, 1)
            (0, 2) circle [radius=0.3] -- ++(up:0.5)
            (-0.3, 1.2) -- ++(up:0.8)
            (0.3, 1) -- ++(up:1)
            (-0.7, 2.5) -- (0.7, 2.5)
            ;
        \draw[pattern={Lines[angle=51,distance=3pt]},pattern color=black,draw=none] (-0.7, 2.5) rectangle (0.7, 2.75);
        \node [left] (left) at (-0.4, 0.6) { $m_1$ };
        \node [right] (right) at (0.4, 0.75) { $m_2$ };
    \end{tikzpicture}
}

\variantsplitter

\addpersonalvariant{Арсений Трофимов}

\tasknumber{1}%
\task{%
    Электрон летит прямолинейно из точки $A$ в точку $B$, за ним при этом наблюдает экспериментатор Глюк.
    Глюк заметил, что первую четверть времени электрон равномерно двигался со скоростью $2 \cdot 10^{5}\,\frac{\text{км}}{\text{ч}}$,
    затем его практически мгновенно ускорило электрическое поле
    и остаток времени электрон вновь равномерно двигался со скоростью $3 \cdot 10^{5}\,\frac{\text{км}}{\text{ч}}$.
    Определите среднюю скорость электрона.
    Ответ выразите в м/с и округлите до тысяч.
}
\solutionspace{120pt}

\tasknumber{2}%
\task{%
    Валя стартует на велосипеде и в течение $3\,\text{c}$ двигается с постоянным ускорением $2{,}5\,\frac{\text{м}}{\text{с}^{2}}$.
    Определите
    \begin{itemize}
        \item какую скорость при этом удастся достичь,
        \item какой путь за это время будет пройден,
        \item среднюю скорость за всё время движения, если после начального ускорения продолжить движение равномерно ещё в течение времени $nt$
    \end{itemize}
}
\solutionspace{120pt}

\tasknumber{3}%
\task{%
    Какой путь тело пройдёт за четвёртую секунду после начала свободного падения?
    Какую скорость в начале этой секунды оно имеет?
}
\solutionspace{120pt}

\tasknumber{4}%
\task{%
    Карусель радиусом $4\,\text{v}$ равномерно совершает 5 оборотов в минуту.
    Определите
    \begin{itemize}
        \item период и частоту её обращения,
        \item скорость и ускорение крайних её точек.
    \end{itemize}
}
\solutionspace{120pt}

\tasknumber{5}%
\task{%
    Миша стоит на обрыве над рекой и методично и строго горизонтально кидает в неё камушки.
    За этим всем наблюдает экспериментатор Глюк, который уже выяснил, что камушки падают в реку спустя $1{,}3\,\text{с}$ после броска,
    а вот дальность полёта оценить сложнее: придётся лезть в воду.
    Выручите Глюка и определите:
    \begin{itemize}
        \item высоту обрыва (вместе с ростом Миши).
        \item дальность полёта камушков (по горизонтали) и их скорость при падении, приняв начальную скорость броска равной $v = 15\,\frac{\text{м}}{\text{с}}$.
    \end{itemize}
    Сопротивлением воздуха пренебречь, $g = 10\,\frac{\text{м}}{\text{с}^{2}}$.
}
\solutionspace{120pt}

\tasknumber{6}%
\task{%
    Четыре одинаковых брусков лежат на гладком горизонтальном столе.
    Масса каждого бруска равна $3\,\text{кг}$,
    причём они пронумерованы от 1 до 4 и последовательно связаны между собой невесомыми
    нерастяжимыми нитями: 1 со 2, 2 с 3 (ну и с 1) и т.д.
    Экспериментатор Глюк прикладывает постоянную горизонтальную силу $90\,\text{Н}$ к бруску с наибольшим номером.
    С каким ускорением двигается система? Чему равна сила натяжения нити, связывающей бруски 2 и 3?
}
\solutionspace{120pt}

\tasknumber{7}%
\task{%
    Два бруска связаны лёгкой нерастяжимой нитью и перекинуты через неподвижный блок (см.
    рис.).
    Определите силу натяжения нити и ускорения брусков.
    Силами трения пренебречь, массы брусков
    равны $m_1 = 5\,\text{кг}$ и $m_2 = 4\,\text{кг}$.

    \begin{tikzpicture}[x=1.5cm,y=1.5cm,thick]
        \draw
            (-0.4, 0) rectangle (-0.2, 1.2)
            (0.15, 0.5) rectangle (0.45, 1)
            (0, 2) circle [radius=0.3] -- ++(up:0.5)
            (-0.3, 1.2) -- ++(up:0.8)
            (0.3, 1) -- ++(up:1)
            (-0.7, 2.5) -- (0.7, 2.5)
            ;
        \draw[pattern={Lines[angle=51,distance=3pt]},pattern color=black,draw=none] (-0.7, 2.5) rectangle (0.7, 2.75);
        \node [left] (left) at (-0.4, 0.6) { $m_1$ };
        \node [right] (right) at (0.4, 0.75) { $m_2$ };
    \end{tikzpicture}
}
% autogenerated
