\setdate{18~мая~2021}
\setclass{10«АБ»}

\addpersonalvariant{Михаил Бурмистров}

\tasknumber{1}%
\task{%
    Саша стартует на велосипеде и в течение $t = 4\,\text{c}$ двигается с постоянным ускорением $0{,}5\,\frac{\text{м}}{\text{с}^{2}}$.
    Определите
    \begin{itemize}
        \item какую скорость при этом удастся достичь,
        \item какой путь за это время будет пройден,
        \item среднюю скорость за всё время движения, если после начального ускорения продолжить движение равномерно ещё в течение времени $2t$
    \end{itemize}
}
\solutionspace{120pt}

\tasknumber{2}%
\task{%
    Какой путь тело пройдёт за вторую секунду после начала свободного падения?
    Какую скорость в начале этой секунды оно имеет?
}
\solutionspace{120pt}

\tasknumber{3}%
\task{%
    Карусель диаметром $3\,\text{м}$ равномерно совершает 6 оборотов в минуту.
    Определите
    \begin{itemize}
        \item период и частоту её обращения,
        \item скорость и ускорение крайних её точек.
    \end{itemize}
}
\solutionspace{80pt}

\tasknumber{4}%
\task{%
    Паша стоит на обрыве над рекой и методично и строго горизонтально кидает в неё камушки.
    За этим всем наблюдает экспериментатор Глюк, который уже выяснил, что камушки падают в реку спустя $1{,}6\,\text{с}$ после броска,
    а вот дальность полёта оценить сложнее: придётся лезть в воду.
    Выручите Глюка и определите:
    \begin{itemize}
        \item высоту обрыва (вместе с ростом Паши).
        \item дальность полёта камушков (по горизонтали) и их скорость при падении, приняв начальную скорость броска равной $v = 18\,\frac{\text{м}}{\text{с}}$.
    \end{itemize}
    Сопротивлением воздуха пренебречь.
}
\solutionspace{120pt}

\tasknumber{5}%
\task{%
    Четыре одинаковых брусков массой $3\,\text{кг}$ каждый лежат на гладком горизонтальном столе.
    Бруски пронумерованы от 1 до 4 и последовательно связаны между собой
    невесомыми нерастяжимыми нитями: 1 со 2, 2 с 3 (ну и с 1) и т.д.
    Экспериментатор Глюк прикладывает постоянную горизонтальную силу $90\,\text{Н}$ к бруску с наибольшим номером.
    С каким ускорением двигается система? Чему равна сила натяжения нити, связывающей бруски 1 и 2?
}
\solutionspace{120pt}

\tasknumber{6}%
\task{%
    Два бруска связаны лёгкой нерастяжимой нитью и перекинуты через неподвижный блок (см.
    рис.).
    Определите силу натяжения нити и ускорения брусков.
    Силами трения пренебречь, массы брусков
    равны $m_1 = 5\,\text{кг}$ и $m_2 = 14\,\text{кг}$.
    % $g = 10\,\frac{\text{м}}{\text{с}^{2}}$.

    \begin{tikzpicture}[x=1.5cm,y=1.5cm,thick]
        \draw
            (-0.4, 0) rectangle (-0.2, 1.2)
            (0.15, 0.5) rectangle (0.45, 1)
            (0, 2) circle [radius=0.3] -- ++(up:0.5)
            (-0.3, 1.2) -- ++(up:0.8)
            (0.3, 1) -- ++(up:1)
            (-0.7, 2.5) -- (0.7, 2.5)
            ;
        \draw[pattern={Lines[angle=51,distance=3pt]},pattern color=black,draw=none] (-0.7, 2.5) rectangle (0.7, 2.75);
        \node [left] (left) at (-0.4, 0.6) { $m_1$ };
        \node [right] (right) at (0.4, 0.75) { $m_2$ };
    \end{tikzpicture}
}
\solutionspace{80pt}

\tasknumber{7}%
\task{%
    Тело массой $1{,}4\,\text{кг}$ лежит на горизонтальной поверхности.
    Коэффициент трения между поверхностью и телом $0{,}25$.
    К телу приложена горизонтальная сила $5{,}5\,\text{Н}$.
    Определите силу трения, действующую на тело, и ускорение тела.
    % $g = 10\,\frac{\text{м}}{\text{с}^{2}}$.
}
\solutionspace{120pt}

\tasknumber{8}%
\task{%
    Определите плотность неизвестного вещества, если известно, что опускании тела из него
    в подсолнечное масло оно будет плавать и на треть выступать над поверхностью жидкости.
}
\solutionspace{120pt}

\tasknumber{9}%
\task{%
    	Определите силу, действующую на левую опору однородного горизонтального стержня длиной $l = 7\,\text{м}$
    	и массой $M = 5\,\text{кг}$, к которому подвешен груз массой $m = 2\,\text{кг}$ на расстоянии $4\,\text{м}$ от правого конца (см.
    рис.).

        \begin{tikzpicture}[thick]
            \draw
                (-2, -0.1) rectangle (2, 0.1)
                (-0.5, -0.1) -- (-0.5, -1)
                (-0.7, -1) rectangle (-0.3, -1.3)
           		(-2, -0.1) -- +(0.15,-0.9) -- +(-0.15,-0.9) -- cycle
            	(2, -0.1) -- +(0.15,-0.9) -- +(-0.15,-0.9) -- cycle
            ;
            \draw[pattern={Lines[angle=51,distance=2pt]},pattern color=black,draw=none]
            	(-2.15, -1.15) rectangle +(0.3, 0.15)
            	(2.15, -1.15) rectangle +(-0.3, 0.15)
            ;
            \node [right] (m_small) at (-0.3, -1.15) { $m$ };
            \node [above] (M_big) at (0, 0.1) { $M$ };
        \end{tikzpicture}
}
\solutionspace{80pt}

\tasknumber{10}%
\task{%
    Тонкий однородный лом длиной $1\,\text{м}$ и массой $10\,\text{кг}$ лежит на горизонтальной поверхности.
    \begin{itemize}
        \item Какую минимальную силу надо приложить к одному из его концов, чтобы оторвать его от этой поверхности?
        \item Какую минимальную работу надо совершить, чтобы поставить его на землю в вертикальное положение?
    \end{itemize}
    % Примите $g = 10\,\frac{\text{м}}{\text{с}^{2}}$.
}
\answer{%
    $A = mg\frac l2 = 50\,\text{Дж}$
}
\solutionspace{120pt}

\tasknumber{11}%
\task{%
    Определите работу силы, которая обеспечит спуск тела массой $2\,\text{кг}$ на высоту $5\,\text{м}$ с постоянным ускорением $3\,\frac{\text{м}}{\text{c}^{2}}$.
    % Примите $g = 10\,\frac{\text{м}}{\text{с}^{2}}$.
}
\answer{%
    \begin{align*}
    &\text{Для подъёма:} A = Fh = (mg + ma) h = m(g+a)h, \\
    &\text{Для спуска:} A = -Fh = -(mg - ma) h = -m(g-a)h, \\
    &\text{В результате получаем:} -70\,\text{Дж}.
    \end{align*}
}
\solutionspace{60pt}

\tasknumber{12}%
\task{%
    Тело бросили вертикально вверх со скоростью $14\,\frac{\text{м}}{\text{c}}$.
    На какой высоте кинетическая энергия тела составит половину от потенциальной?
}
\solutionspace{100pt}

\tasknumber{13}%
\task{%
    Плотность воздуха при нормальных условиях равна $1{,}3\,\frac{\text{кг}}{\text{м}^{3}}$.
    Чему равна плотность воздуха
    при температуре $200\celsius$ и давлении $50\,\text{кПа}$?
}
\solutionspace{120pt}

\tasknumber{14}%
\task{%
    Небольшую цилиндрическую пробирку с воздухом погружают на некоторую глубину в глубокое пресное озеро,
    после чего воздух занимает в ней лишь пятую часть от общего объема.
    Определите глубину, на которую погрузили пробирку.
    Температуру считать постоянной $T = 280\,\text{К}$, давлением паров воды пренебречь,
    атмосферное давление принять равным $p_{\text{aтм}} = 100\,\text{кПа}$.
}
\answer{%
    \begin{align*}
    T\text{— const} &\implies P_1V_1 = \nu RT = P_2V_2.
    \\
    V_2 = \frac 15 V_1 &\implies P_1V_1 = P_2 \cdot \frac 15V_1 \implies P_2 = 5P_1 = 5p_{\text{aтм}}.
    \\
    P_2 = p_{\text{aтм}} + \rho_{\text{в}} g h \implies h = \frac{P_2 - p_{\text{aтм}}}{\rho_{\text{в}} g} &= \frac{5p_{\text{aтм}} - p_{\text{aтм}}}{\rho_{\text{в}} g} = \frac{4 \cdot p_{\text{aтм}}}{\rho_{\text{в}} g} =  \\
     &= \frac{4 \cdot 100\,\text{кПа}}{1000\,\frac{\text{кг}}{\text{м}^{3}} \cdot  10\,\frac{\text{м}}{\text{с}^{2}}} \approx 40\,\text{м}.
    \end{align*}
}
\solutionspace{120pt}

\tasknumber{15}%
\task{%
    Газу сообщили некоторое количество теплоты,
    при этом треть его он потратил на совершение работы,
    одновременно увеличив свою внутреннюю энергию на $1200\,\text{Дж}$.
    Определите работу, совершённую газом.
}
\answer{%
    \begin{align*}
    Q &= A' + \Delta U, A' = \frac 13 Q \implies Q \cdot \cbr{1 - \frac 13} = \Delta U \implies Q = \frac{\Delta U}{1 - \frac 13} = \frac{ 1200\,\text{Дж} }{1 - \frac 13} \approx 1800\,\text{Дж}.
    \\
    A' &= \frac 13 Q
        = \frac 13 \cdot \frac{\Delta U}{1 - \frac 13}
        = \frac{\Delta U}{3 - 1}
        = \frac{ 1200\,\text{Дж} }{3 - 1} \approx 600\,\text{Дж}.
    \end{align*}
}
\solutionspace{60pt}

\tasknumber{16}%
\task{%
    Два конденсатора ёмкостей $C_1 = 40\,\text{нФ}$ и $C_2 = 60\,\text{нФ}$ последовательно подключают
    к источнику напряжения $V = 400\,\text{В}$ (см.
    рис.).
    % Определите заряды каждого из конденсаторов.
    Определите заряд второго конденсатора.

    \begin{tikzpicture}[circuit ee IEC, semithick]
        \draw  (0, 0) to [capacitor={info={$C_1$}}] (1, 0)
                       to [capacitor={info={$C_2$}}] (2, 0)
        ;
        % \draw [-o] (0, 0) -- ++(-0.5, 0) node[left] {$-$};
        % \draw [-o] (2, 0) -- ++(0.5, 0) node[right] {$+$};
        \draw [-o] (0, 0) -- ++(-0.5, 0) node[left] {};
        \draw [-o] (2, 0) -- ++(0.5, 0) node[right] {};
    \end{tikzpicture}
}
\answer{%
    $
        Q_1
            = Q_2
            = CV
            = \frac{ V }{\frac1{C_1} + \frac1{C_2}}
            = \frac{C_1C_2V}{C_1 + C_2}
            = \frac{
                40\,\text{нФ} \cdot 60\,\text{нФ} \cdot 400\,\text{В}
            }{
                40\,\text{нФ} + 60\,\text{нФ}
            }
            = 9600{,}00\,\text{нКл}
    $
}
\solutionspace{120pt}

\tasknumber{17}%
\task{%
    В вакууме вдоль одной прямой расположены четыре отрицательных заряда так,
    что расстояние между соседними зарядами равно $r$.
    Сделайте рисунок,
    и определите силу, действующую на крайний заряд.
    Модули всех зарядов равны $q$ ($q > 0$).
}
\solutionspace{80pt}

\tasknumber{18}%
\task{%
    Юлия проводит эксперименты c 2 кусками одинаковой алюминиевой проволки, причём второй кусок в четыре раза длиннее первого.
    В одном из экспериментов Юлия подаёт на первый кусок проволки напряжение в шесть раз раз больше, чем на второй.
    Определите отношения в двух проволках в этом эксперименте (второй к первой):
    \begin{itemize}
        \item отношение сил тока,
        \item отношение выделяющихся мощностей.
    \end{itemize}
}
\answer{%
    $\eli_2 / \eli_1 = \frac1{24}, \P_2 / \P_1 = \frac1{24}, $
}

\variantsplitter

\addpersonalvariant{Ирина Ан}

\tasknumber{1}%
\task{%
    Валя стартует на мотоцикле и в течение $t = 2\,\text{c}$ двигается с постоянным ускорением $2\,\frac{\text{м}}{\text{с}^{2}}$.
    Определите
    \begin{itemize}
        \item какую скорость при этом удастся достичь,
        \item какой путь за это время будет пройден,
        \item среднюю скорость за всё время движения, если после начального ускорения продолжить движение равномерно ещё в течение времени $2t$
    \end{itemize}
}
\solutionspace{120pt}

\tasknumber{2}%
\task{%
    Какой путь тело пройдёт за пятую секунду после начала свободного падения?
    Какую скорость в конце этой секунды оно имеет?
}
\solutionspace{120pt}

\tasknumber{3}%
\task{%
    Карусель диаметром $2\,\text{м}$ равномерно совершает 10 оборотов в минуту.
    Определите
    \begin{itemize}
        \item период и частоту её обращения,
        \item скорость и ускорение крайних её точек.
    \end{itemize}
}
\solutionspace{80pt}

\tasknumber{4}%
\task{%
    Миша стоит на обрыве над рекой и методично и строго горизонтально кидает в неё камушки.
    За этим всем наблюдает экспериментатор Глюк, который уже выяснил, что камушки падают в реку спустя $1{,}2\,\text{с}$ после броска,
    а вот дальность полёта оценить сложнее: придётся лезть в воду.
    Выручите Глюка и определите:
    \begin{itemize}
        \item высоту обрыва (вместе с ростом Миши).
        \item дальность полёта камушков (по горизонтали) и их скорость при падении, приняв начальную скорость броска равной $v = 18\,\frac{\text{м}}{\text{с}}$.
    \end{itemize}
    Сопротивлением воздуха пренебречь.
}
\solutionspace{120pt}

\tasknumber{5}%
\task{%
    Четыре одинаковых брусков массой $2\,\text{кг}$ каждый лежат на гладком горизонтальном столе.
    Бруски пронумерованы от 1 до 4 и последовательно связаны между собой
    невесомыми нерастяжимыми нитями: 1 со 2, 2 с 3 (ну и с 1) и т.д.
    Экспериментатор Глюк прикладывает постоянную горизонтальную силу $90\,\text{Н}$ к бруску с наибольшим номером.
    С каким ускорением двигается система? Чему равна сила натяжения нити, связывающей бруски 3 и 4?
}
\solutionspace{120pt}

\tasknumber{6}%
\task{%
    Два бруска связаны лёгкой нерастяжимой нитью и перекинуты через неподвижный блок (см.
    рис.).
    Определите силу натяжения нити и ускорения брусков.
    Силами трения пренебречь, массы брусков
    равны $m_1 = 5\,\text{кг}$ и $m_2 = 10\,\text{кг}$.
    % $g = 10\,\frac{\text{м}}{\text{с}^{2}}$.

    \begin{tikzpicture}[x=1.5cm,y=1.5cm,thick]
        \draw
            (-0.4, 0) rectangle (-0.2, 1.2)
            (0.15, 0.5) rectangle (0.45, 1)
            (0, 2) circle [radius=0.3] -- ++(up:0.5)
            (-0.3, 1.2) -- ++(up:0.8)
            (0.3, 1) -- ++(up:1)
            (-0.7, 2.5) -- (0.7, 2.5)
            ;
        \draw[pattern={Lines[angle=51,distance=3pt]},pattern color=black,draw=none] (-0.7, 2.5) rectangle (0.7, 2.75);
        \node [left] (left) at (-0.4, 0.6) { $m_1$ };
        \node [right] (right) at (0.4, 0.75) { $m_2$ };
    \end{tikzpicture}
}
\solutionspace{80pt}

\tasknumber{7}%
\task{%
    Тело массой $2\,\text{кг}$ лежит на горизонтальной поверхности.
    Коэффициент трения между поверхностью и телом $0{,}2$.
    К телу приложена горизонтальная сила $5{,}5\,\text{Н}$.
    Определите силу трения, действующую на тело, и ускорение тела.
    % $g = 10\,\frac{\text{м}}{\text{с}^{2}}$.
}
\solutionspace{120pt}

\tasknumber{8}%
\task{%
    Определите плотность неизвестного вещества, если известно, что опускании тела из него
    в керосин оно будет плавать и на половину выступать над поверхностью жидкости.
}
\solutionspace{120pt}

\tasknumber{9}%
\task{%
    	Определите силу, действующую на левую опору однородного горизонтального стержня длиной $l = 3\,\text{м}$
    	и массой $M = 1\,\text{кг}$, к которому подвешен груз массой $m = 4\,\text{кг}$ на расстоянии $2\,\text{м}$ от правого конца (см.
    рис.).

        \begin{tikzpicture}[thick]
            \draw
                (-2, -0.1) rectangle (2, 0.1)
                (-0.5, -0.1) -- (-0.5, -1)
                (-0.7, -1) rectangle (-0.3, -1.3)
           		(-2, -0.1) -- +(0.15,-0.9) -- +(-0.15,-0.9) -- cycle
            	(2, -0.1) -- +(0.15,-0.9) -- +(-0.15,-0.9) -- cycle
            ;
            \draw[pattern={Lines[angle=51,distance=2pt]},pattern color=black,draw=none]
            	(-2.15, -1.15) rectangle +(0.3, 0.15)
            	(2.15, -1.15) rectangle +(-0.3, 0.15)
            ;
            \node [right] (m_small) at (-0.3, -1.15) { $m$ };
            \node [above] (M_big) at (0, 0.1) { $M$ };
        \end{tikzpicture}
}
\solutionspace{80pt}

\tasknumber{10}%
\task{%
    Тонкий однородный кусок арматуры длиной $2\,\text{м}$ и массой $20\,\text{кг}$ лежит на горизонтальной поверхности.
    \begin{itemize}
        \item Какую минимальную силу надо приложить к одному из его концов, чтобы оторвать его от этой поверхности?
        \item Какую минимальную работу надо совершить, чтобы поставить его на землю в вертикальное положение?
    \end{itemize}
    % Примите $g = 10\,\frac{\text{м}}{\text{с}^{2}}$.
}
\answer{%
    $A = mg\frac l2 = 200\,\text{Дж}$
}
\solutionspace{120pt}

\tasknumber{11}%
\task{%
    Определите работу силы, которая обеспечит спуск тела массой $5\,\text{кг}$ на высоту $2\,\text{м}$ с постоянным ускорением $2\,\frac{\text{м}}{\text{c}^{2}}$.
    % Примите $g = 10\,\frac{\text{м}}{\text{с}^{2}}$.
}
\answer{%
    \begin{align*}
    &\text{Для подъёма:} A = Fh = (mg + ma) h = m(g+a)h, \\
    &\text{Для спуска:} A = -Fh = -(mg - ma) h = -m(g-a)h, \\
    &\text{В результате получаем:} -80\,\text{Дж}.
    \end{align*}
}
\solutionspace{60pt}

\tasknumber{12}%
\task{%
    Тело бросили вертикально вверх со скоростью $10\,\frac{\text{м}}{\text{c}}$.
    На какой высоте кинетическая энергия тела составит половину от потенциальной?
}
\solutionspace{100pt}

\tasknumber{13}%
\task{%
    Плотность воздуха при нормальных условиях равна $1{,}3\,\frac{\text{кг}}{\text{м}^{3}}$.
    Чему равна плотность воздуха
    при температуре $100\celsius$ и давлении $150\,\text{кПа}$?
}
\solutionspace{120pt}

\tasknumber{14}%
\task{%
    Небольшую цилиндрическую пробирку с воздухом погружают на некоторую глубину в глубокое пресное озеро,
    после чего воздух занимает в ней лишь пятую часть от общего объема.
    Определите глубину, на которую погрузили пробирку.
    Температуру считать постоянной $T = 280\,\text{К}$, давлением паров воды пренебречь,
    атмосферное давление принять равным $p_{\text{aтм}} = 100\,\text{кПа}$.
}
\answer{%
    \begin{align*}
    T\text{— const} &\implies P_1V_1 = \nu RT = P_2V_2.
    \\
    V_2 = \frac 15 V_1 &\implies P_1V_1 = P_2 \cdot \frac 15V_1 \implies P_2 = 5P_1 = 5p_{\text{aтм}}.
    \\
    P_2 = p_{\text{aтм}} + \rho_{\text{в}} g h \implies h = \frac{P_2 - p_{\text{aтм}}}{\rho_{\text{в}} g} &= \frac{5p_{\text{aтм}} - p_{\text{aтм}}}{\rho_{\text{в}} g} = \frac{4 \cdot p_{\text{aтм}}}{\rho_{\text{в}} g} =  \\
     &= \frac{4 \cdot 100\,\text{кПа}}{1000\,\frac{\text{кг}}{\text{м}^{3}} \cdot  10\,\frac{\text{м}}{\text{с}^{2}}} \approx 40\,\text{м}.
    \end{align*}
}
\solutionspace{120pt}

\tasknumber{15}%
\task{%
    Газу сообщили некоторое количество теплоты,
    при этом четверть его он потратил на совершение работы,
    одновременно увеличив свою внутреннюю энергию на $3000\,\text{Дж}$.
    Определите количество теплоты, сообщённое газу.
}
\answer{%
    \begin{align*}
    Q &= A' + \Delta U, A' = \frac 14 Q \implies Q \cdot \cbr{1 - \frac 14} = \Delta U \implies Q = \frac{\Delta U}{1 - \frac 14} = \frac{ 3000\,\text{Дж} }{1 - \frac 14} \approx 4000\,\text{Дж}.
    \\
    A' &= \frac 14 Q
        = \frac 14 \cdot \frac{\Delta U}{1 - \frac 14}
        = \frac{\Delta U}{4 - 1}
        = \frac{ 3000\,\text{Дж} }{4 - 1} \approx 1000\,\text{Дж}.
    \end{align*}
}
\solutionspace{60pt}

\tasknumber{16}%
\task{%
    Два конденсатора ёмкостей $C_1 = 20\,\text{нФ}$ и $C_2 = 30\,\text{нФ}$ последовательно подключают
    к источнику напряжения $V = 300\,\text{В}$ (см.
    рис.).
    % Определите заряды каждого из конденсаторов.
    Определите заряд второго конденсатора.

    \begin{tikzpicture}[circuit ee IEC, semithick]
        \draw  (0, 0) to [capacitor={info={$C_1$}}] (1, 0)
                       to [capacitor={info={$C_2$}}] (2, 0)
        ;
        % \draw [-o] (0, 0) -- ++(-0.5, 0) node[left] {$-$};
        % \draw [-o] (2, 0) -- ++(0.5, 0) node[right] {$+$};
        \draw [-o] (0, 0) -- ++(-0.5, 0) node[left] {};
        \draw [-o] (2, 0) -- ++(0.5, 0) node[right] {};
    \end{tikzpicture}
}
\answer{%
    $
        Q_1
            = Q_2
            = CV
            = \frac{ V }{\frac1{C_1} + \frac1{C_2}}
            = \frac{C_1C_2V}{C_1 + C_2}
            = \frac{
                20\,\text{нФ} \cdot 30\,\text{нФ} \cdot 300\,\text{В}
            }{
                20\,\text{нФ} + 30\,\text{нФ}
            }
            = 3600{,}00\,\text{нКл}
    $
}
\solutionspace{120pt}

\tasknumber{17}%
\task{%
    В вакууме вдоль одной прямой расположены три положительных заряда так,
    что расстояние между соседними зарядами равно $a$.
    Сделайте рисунок,
    и определите силу, действующую на крайний заряд.
    Модули всех зарядов равны $Q$ ($Q > 0$).
}
\solutionspace{80pt}

\tasknumber{18}%
\task{%
    Юлия проводит эксперименты c 2 кусками одинаковой алюминиевой проволки, причём второй кусок в пять раз длиннее первого.
    В одном из экспериментов Юлия подаёт на первый кусок проволки напряжение в десять раз раз больше, чем на второй.
    Определите отношения в двух проволках в этом эксперименте (второй к первой):
    \begin{itemize}
        \item отношение сил тока,
        \item отношение выделяющихся мощностей.
    \end{itemize}
}
\answer{%
    $\eli_2 / \eli_1 = \frac1{50}, \P_2 / \P_1 = \frac1{50}, $
}

\variantsplitter

\addpersonalvariant{Софья Андрианова}

\tasknumber{1}%
\task{%
    Саша стартует на мотоцикле и в течение $t = 4\,\text{c}$ двигается с постоянным ускорением $2{,}5\,\frac{\text{м}}{\text{с}^{2}}$.
    Определите
    \begin{itemize}
        \item какую скорость при этом удастся достичь,
        \item какой путь за это время будет пройден,
        \item среднюю скорость за всё время движения, если после начального ускорения продолжить движение равномерно ещё в течение времени $3t$
    \end{itemize}
}
\solutionspace{120pt}

\tasknumber{2}%
\task{%
    Какой путь тело пройдёт за четвёртую секунду после начала свободного падения?
    Какую скорость в начале этой секунды оно имеет?
}
\solutionspace{120pt}

\tasknumber{3}%
\task{%
    Карусель диаметром $3\,\text{м}$ равномерно совершает 6 оборотов в минуту.
    Определите
    \begin{itemize}
        \item период и частоту её обращения,
        \item скорость и ускорение крайних её точек.
    \end{itemize}
}
\solutionspace{80pt}

\tasknumber{4}%
\task{%
    Даша стоит на обрыве над рекой и методично и строго горизонтально кидает в неё камушки.
    За этим всем наблюдает экспериментатор Глюк, который уже выяснил, что камушки падают в реку спустя $1{,}3\,\text{с}$ после броска,
    а вот дальность полёта оценить сложнее: придётся лезть в воду.
    Выручите Глюка и определите:
    \begin{itemize}
        \item высоту обрыва (вместе с ростом Даши).
        \item дальность полёта камушков (по горизонтали) и их скорость при падении, приняв начальную скорость броска равной $v = 18\,\frac{\text{м}}{\text{с}}$.
    \end{itemize}
    Сопротивлением воздуха пренебречь.
}
\solutionspace{120pt}

\tasknumber{5}%
\task{%
    Шесть одинаковых брусков массой $3\,\text{кг}$ каждый лежат на гладком горизонтальном столе.
    Бруски пронумерованы от 1 до 6 и последовательно связаны между собой
    невесомыми нерастяжимыми нитями: 1 со 2, 2 с 3 (ну и с 1) и т.д.
    Экспериментатор Глюк прикладывает постоянную горизонтальную силу $120\,\text{Н}$ к бруску с наименьшим номером.
    С каким ускорением двигается система? Чему равна сила натяжения нити, связывающей бруски 1 и 2?
}
\solutionspace{120pt}

\tasknumber{6}%
\task{%
    Два бруска связаны лёгкой нерастяжимой нитью и перекинуты через неподвижный блок (см.
    рис.).
    Определите силу натяжения нити и ускорения брусков.
    Силами трения пренебречь, массы брусков
    равны $m_1 = 11\,\text{кг}$ и $m_2 = 10\,\text{кг}$.
    % $g = 10\,\frac{\text{м}}{\text{с}^{2}}$.

    \begin{tikzpicture}[x=1.5cm,y=1.5cm,thick]
        \draw
            (-0.4, 0) rectangle (-0.2, 1.2)
            (0.15, 0.5) rectangle (0.45, 1)
            (0, 2) circle [radius=0.3] -- ++(up:0.5)
            (-0.3, 1.2) -- ++(up:0.8)
            (0.3, 1) -- ++(up:1)
            (-0.7, 2.5) -- (0.7, 2.5)
            ;
        \draw[pattern={Lines[angle=51,distance=3pt]},pattern color=black,draw=none] (-0.7, 2.5) rectangle (0.7, 2.75);
        \node [left] (left) at (-0.4, 0.6) { $m_1$ };
        \node [right] (right) at (0.4, 0.75) { $m_2$ };
    \end{tikzpicture}
}
\solutionspace{80pt}

\tasknumber{7}%
\task{%
    Тело массой $2\,\text{кг}$ лежит на горизонтальной поверхности.
    Коэффициент трения между поверхностью и телом $0{,}2$.
    К телу приложена горизонтальная сила $3{,}5\,\text{Н}$.
    Определите силу трения, действующую на тело, и ускорение тела.
    % $g = 10\,\frac{\text{м}}{\text{с}^{2}}$.
}
\solutionspace{120pt}

\tasknumber{8}%
\task{%
    Определите плотность неизвестного вещества, если известно, что опускании тела из него
    в подсолнечное масло оно будет плавать и на половину выступать над поверхностью жидкости.
}
\solutionspace{120pt}

\tasknumber{9}%
\task{%
    	Определите силу, действующую на левую опору однородного горизонтального стержня длиной $l = 7\,\text{м}$
    	и массой $M = 5\,\text{кг}$, к которому подвешен груз массой $m = 3\,\text{кг}$ на расстоянии $4\,\text{м}$ от правого конца (см.
    рис.).

        \begin{tikzpicture}[thick]
            \draw
                (-2, -0.1) rectangle (2, 0.1)
                (-0.5, -0.1) -- (-0.5, -1)
                (-0.7, -1) rectangle (-0.3, -1.3)
           		(-2, -0.1) -- +(0.15,-0.9) -- +(-0.15,-0.9) -- cycle
            	(2, -0.1) -- +(0.15,-0.9) -- +(-0.15,-0.9) -- cycle
            ;
            \draw[pattern={Lines[angle=51,distance=2pt]},pattern color=black,draw=none]
            	(-2.15, -1.15) rectangle +(0.3, 0.15)
            	(2.15, -1.15) rectangle +(-0.3, 0.15)
            ;
            \node [right] (m_small) at (-0.3, -1.15) { $m$ };
            \node [above] (M_big) at (0, 0.1) { $M$ };
        \end{tikzpicture}
}
\solutionspace{80pt}

\tasknumber{10}%
\task{%
    Тонкий однородный лом длиной $1\,\text{м}$ и массой $10\,\text{кг}$ лежит на горизонтальной поверхности.
    \begin{itemize}
        \item Какую минимальную силу надо приложить к одному из его концов, чтобы оторвать его от этой поверхности?
        \item Какую минимальную работу надо совершить, чтобы поставить его на землю в вертикальное положение?
    \end{itemize}
    % Примите $g = 10\,\frac{\text{м}}{\text{с}^{2}}$.
}
\answer{%
    $A = mg\frac l2 = 50\,\text{Дж}$
}
\solutionspace{120pt}

\tasknumber{11}%
\task{%
    Определите работу силы, которая обеспечит подъём тела массой $3\,\text{кг}$ на высоту $10\,\text{м}$ с постоянным ускорением $6\,\frac{\text{м}}{\text{c}^{2}}$.
    % Примите $g = 10\,\frac{\text{м}}{\text{с}^{2}}$.
}
\answer{%
    \begin{align*}
    &\text{Для подъёма:} A = Fh = (mg + ma) h = m(g+a)h, \\
    &\text{Для спуска:} A = -Fh = -(mg - ma) h = -m(g-a)h, \\
    &\text{В результате получаем:} 480\,\text{Дж}.
    \end{align*}
}
\solutionspace{60pt}

\tasknumber{12}%
\task{%
    Тело бросили вертикально вверх со скоростью $10\,\frac{\text{м}}{\text{c}}$.
    На какой высоте кинетическая энергия тела составит треть от потенциальной?
}
\solutionspace{100pt}

\tasknumber{13}%
\task{%
    Плотность воздуха при нормальных условиях равна $1{,}3\,\frac{\text{кг}}{\text{м}^{3}}$.
    Чему равна плотность воздуха
    при температуре $200\celsius$ и давлении $120\,\text{кПа}$?
}
\solutionspace{120pt}

\tasknumber{14}%
\task{%
    Небольшую цилиндрическую пробирку с воздухом погружают на некоторую глубину в глубокое пресное озеро,
    после чего воздух занимает в ней лишь пятую часть от общего объема.
    Определите глубину, на которую погрузили пробирку.
    Температуру считать постоянной $T = 292\,\text{К}$, давлением паров воды пренебречь,
    атмосферное давление принять равным $p_{\text{aтм}} = 100\,\text{кПа}$.
}
\answer{%
    \begin{align*}
    T\text{— const} &\implies P_1V_1 = \nu RT = P_2V_2.
    \\
    V_2 = \frac 15 V_1 &\implies P_1V_1 = P_2 \cdot \frac 15V_1 \implies P_2 = 5P_1 = 5p_{\text{aтм}}.
    \\
    P_2 = p_{\text{aтм}} + \rho_{\text{в}} g h \implies h = \frac{P_2 - p_{\text{aтм}}}{\rho_{\text{в}} g} &= \frac{5p_{\text{aтм}} - p_{\text{aтм}}}{\rho_{\text{в}} g} = \frac{4 \cdot p_{\text{aтм}}}{\rho_{\text{в}} g} =  \\
     &= \frac{4 \cdot 100\,\text{кПа}}{1000\,\frac{\text{кг}}{\text{м}^{3}} \cdot  10\,\frac{\text{м}}{\text{с}^{2}}} \approx 40\,\text{м}.
    \end{align*}
}
\solutionspace{120pt}

\tasknumber{15}%
\task{%
    Газу сообщили некоторое количество теплоты,
    при этом треть его он потратил на совершение работы,
    одновременно увеличив свою внутреннюю энергию на $2400\,\text{Дж}$.
    Определите работу, совершённую газом.
}
\answer{%
    \begin{align*}
    Q &= A' + \Delta U, A' = \frac 13 Q \implies Q \cdot \cbr{1 - \frac 13} = \Delta U \implies Q = \frac{\Delta U}{1 - \frac 13} = \frac{ 2400\,\text{Дж} }{1 - \frac 13} \approx 3600\,\text{Дж}.
    \\
    A' &= \frac 13 Q
        = \frac 13 \cdot \frac{\Delta U}{1 - \frac 13}
        = \frac{\Delta U}{3 - 1}
        = \frac{ 2400\,\text{Дж} }{3 - 1} \approx 1200\,\text{Дж}.
    \end{align*}
}
\solutionspace{60pt}

\tasknumber{16}%
\task{%
    Два конденсатора ёмкостей $C_1 = 60\,\text{нФ}$ и $C_2 = 40\,\text{нФ}$ последовательно подключают
    к источнику напряжения $V = 450\,\text{В}$ (см.
    рис.).
    % Определите заряды каждого из конденсаторов.
    Определите заряд первого конденсатора.

    \begin{tikzpicture}[circuit ee IEC, semithick]
        \draw  (0, 0) to [capacitor={info={$C_1$}}] (1, 0)
                       to [capacitor={info={$C_2$}}] (2, 0)
        ;
        % \draw [-o] (0, 0) -- ++(-0.5, 0) node[left] {$-$};
        % \draw [-o] (2, 0) -- ++(0.5, 0) node[right] {$+$};
        \draw [-o] (0, 0) -- ++(-0.5, 0) node[left] {};
        \draw [-o] (2, 0) -- ++(0.5, 0) node[right] {};
    \end{tikzpicture}
}
\answer{%
    $
        Q_1
            = Q_2
            = CV
            = \frac{ V }{\frac1{C_1} + \frac1{C_2}}
            = \frac{C_1C_2V}{C_1 + C_2}
            = \frac{
                60\,\text{нФ} \cdot 40\,\text{нФ} \cdot 450\,\text{В}
            }{
                60\,\text{нФ} + 40\,\text{нФ}
            }
            = 10800{,}00\,\text{нКл}
    $
}
\solutionspace{120pt}

\tasknumber{17}%
\task{%
    В вакууме вдоль одной прямой расположены три положительных заряда так,
    что расстояние между соседними зарядами равно $l$.
    Сделайте рисунок,
    и определите силу, действующую на крайний заряд.
    Модули всех зарядов равны $Q$ ($Q > 0$).
}
\solutionspace{80pt}

\tasknumber{18}%
\task{%
    Юлия проводит эксперименты c 2 кусками одинаковой медной проволки, причём второй кусок в семь раз длиннее первого.
    В одном из экспериментов Юлия подаёт на первый кусок проволки напряжение в три раза раз больше, чем на второй.
    Определите отношения в двух проволках в этом эксперименте (второй к первой):
    \begin{itemize}
        \item отношение сил тока,
        \item отношение выделяющихся мощностей.
    \end{itemize}
}
\answer{%
    $\eli_2 / \eli_1 = \frac1{21}, \P_2 / \P_1 = \frac1{21}, $
}

\variantsplitter

\addpersonalvariant{Владимир Артемчук}

\tasknumber{1}%
\task{%
    Женя стартует на мотоцикле и в течение $t = 3\,\text{c}$ двигается с постоянным ускорением $0{,}5\,\frac{\text{м}}{\text{с}^{2}}$.
    Определите
    \begin{itemize}
        \item какую скорость при этом удастся достичь,
        \item какой путь за это время будет пройден,
        \item среднюю скорость за всё время движения, если после начального ускорения продолжить движение равномерно ещё в течение времени $2t$
    \end{itemize}
}
\solutionspace{120pt}

\tasknumber{2}%
\task{%
    Какой путь тело пройдёт за третью секунду после начала свободного падения?
    Какую скорость в начале этой секунды оно имеет?
}
\solutionspace{120pt}

\tasknumber{3}%
\task{%
    Карусель диаметром $3\,\text{м}$ равномерно совершает 10 оборотов в минуту.
    Определите
    \begin{itemize}
        \item период и частоту её обращения,
        \item скорость и ускорение крайних её точек.
    \end{itemize}
}
\solutionspace{80pt}

\tasknumber{4}%
\task{%
    Миша стоит на обрыве над рекой и методично и строго горизонтально кидает в неё камушки.
    За этим всем наблюдает экспериментатор Глюк, который уже выяснил, что камушки падают в реку спустя $1{,}3\,\text{с}$ после броска,
    а вот дальность полёта оценить сложнее: придётся лезть в воду.
    Выручите Глюка и определите:
    \begin{itemize}
        \item высоту обрыва (вместе с ростом Миши).
        \item дальность полёта камушков (по горизонтали) и их скорость при падении, приняв начальную скорость броска равной $v = 17\,\frac{\text{м}}{\text{с}}$.
    \end{itemize}
    Сопротивлением воздуха пренебречь.
}
\solutionspace{120pt}

\tasknumber{5}%
\task{%
    Четыре одинаковых брусков массой $2\,\text{кг}$ каждый лежат на гладком горизонтальном столе.
    Бруски пронумерованы от 1 до 4 и последовательно связаны между собой
    невесомыми нерастяжимыми нитями: 1 со 2, 2 с 3 (ну и с 1) и т.д.
    Экспериментатор Глюк прикладывает постоянную горизонтальную силу $90\,\text{Н}$ к бруску с наименьшим номером.
    С каким ускорением двигается система? Чему равна сила натяжения нити, связывающей бруски 3 и 4?
}
\solutionspace{120pt}

\tasknumber{6}%
\task{%
    Два бруска связаны лёгкой нерастяжимой нитью и перекинуты через неподвижный блок (см.
    рис.).
    Определите силу натяжения нити и ускорения брусков.
    Силами трения пренебречь, массы брусков
    равны $m_1 = 8\,\text{кг}$ и $m_2 = 4\,\text{кг}$.
    % $g = 10\,\frac{\text{м}}{\text{с}^{2}}$.

    \begin{tikzpicture}[x=1.5cm,y=1.5cm,thick]
        \draw
            (-0.4, 0) rectangle (-0.2, 1.2)
            (0.15, 0.5) rectangle (0.45, 1)
            (0, 2) circle [radius=0.3] -- ++(up:0.5)
            (-0.3, 1.2) -- ++(up:0.8)
            (0.3, 1) -- ++(up:1)
            (-0.7, 2.5) -- (0.7, 2.5)
            ;
        \draw[pattern={Lines[angle=51,distance=3pt]},pattern color=black,draw=none] (-0.7, 2.5) rectangle (0.7, 2.75);
        \node [left] (left) at (-0.4, 0.6) { $m_1$ };
        \node [right] (right) at (0.4, 0.75) { $m_2$ };
    \end{tikzpicture}
}
\solutionspace{80pt}

\tasknumber{7}%
\task{%
    Тело массой $2{,}7\,\text{кг}$ лежит на горизонтальной поверхности.
    Коэффициент трения между поверхностью и телом $0{,}15$.
    К телу приложена горизонтальная сила $4{,}5\,\text{Н}$.
    Определите силу трения, действующую на тело, и ускорение тела.
    % $g = 10\,\frac{\text{м}}{\text{с}^{2}}$.
}
\solutionspace{120pt}

\tasknumber{8}%
\task{%
    Определите плотность неизвестного вещества, если известно, что опускании тела из него
    в керосин оно будет плавать и на четверть выступать над поверхностью жидкости.
}
\solutionspace{120pt}

\tasknumber{9}%
\task{%
    	Определите силу, действующую на левую опору однородного горизонтального стержня длиной $l = 7\,\text{м}$
    	и массой $M = 1\,\text{кг}$, к которому подвешен груз массой $m = 4\,\text{кг}$ на расстоянии $2\,\text{м}$ от правого конца (см.
    рис.).

        \begin{tikzpicture}[thick]
            \draw
                (-2, -0.1) rectangle (2, 0.1)
                (-0.5, -0.1) -- (-0.5, -1)
                (-0.7, -1) rectangle (-0.3, -1.3)
           		(-2, -0.1) -- +(0.15,-0.9) -- +(-0.15,-0.9) -- cycle
            	(2, -0.1) -- +(0.15,-0.9) -- +(-0.15,-0.9) -- cycle
            ;
            \draw[pattern={Lines[angle=51,distance=2pt]},pattern color=black,draw=none]
            	(-2.15, -1.15) rectangle +(0.3, 0.15)
            	(2.15, -1.15) rectangle +(-0.3, 0.15)
            ;
            \node [right] (m_small) at (-0.3, -1.15) { $m$ };
            \node [above] (M_big) at (0, 0.1) { $M$ };
        \end{tikzpicture}
}
\solutionspace{80pt}

\tasknumber{10}%
\task{%
    Тонкий однородный лом длиной $3\,\text{м}$ и массой $10\,\text{кг}$ лежит на горизонтальной поверхности.
    \begin{itemize}
        \item Какую минимальную силу надо приложить к одному из его концов, чтобы оторвать его от этой поверхности?
        \item Какую минимальную работу надо совершить, чтобы поставить его на землю в вертикальное положение?
    \end{itemize}
    % Примите $g = 10\,\frac{\text{м}}{\text{с}^{2}}$.
}
\answer{%
    $A = mg\frac l2 = 150\,\text{Дж}$
}
\solutionspace{120pt}

\tasknumber{11}%
\task{%
    Определите работу силы, которая обеспечит подъём тела массой $5\,\text{кг}$ на высоту $5\,\text{м}$ с постоянным ускорением $3\,\frac{\text{м}}{\text{c}^{2}}$.
    % Примите $g = 10\,\frac{\text{м}}{\text{с}^{2}}$.
}
\answer{%
    \begin{align*}
    &\text{Для подъёма:} A = Fh = (mg + ma) h = m(g+a)h, \\
    &\text{Для спуска:} A = -Fh = -(mg - ma) h = -m(g-a)h, \\
    &\text{В результате получаем:} 325\,\text{Дж}.
    \end{align*}
}
\solutionspace{60pt}

\tasknumber{12}%
\task{%
    Тело бросили вертикально вверх со скоростью $20\,\frac{\text{м}}{\text{c}}$.
    На какой высоте кинетическая энергия тела составит половину от потенциальной?
}
\solutionspace{100pt}

\tasknumber{13}%
\task{%
    Плотность воздуха при нормальных условиях равна $1{,}3\,\frac{\text{кг}}{\text{м}^{3}}$.
    Чему равна плотность воздуха
    при температуре $50\celsius$ и давлении $50\,\text{кПа}$?
}
\solutionspace{120pt}

\tasknumber{14}%
\task{%
    Небольшую цилиндрическую пробирку с воздухом погружают на некоторую глубину в глубокое пресное озеро,
    после чего воздух занимает в ней лишь пятую часть от общего объема.
    Определите глубину, на которую погрузили пробирку.
    Температуру считать постоянной $T = 281\,\text{К}$, давлением паров воды пренебречь,
    атмосферное давление принять равным $p_{\text{aтм}} = 100\,\text{кПа}$.
}
\answer{%
    \begin{align*}
    T\text{— const} &\implies P_1V_1 = \nu RT = P_2V_2.
    \\
    V_2 = \frac 15 V_1 &\implies P_1V_1 = P_2 \cdot \frac 15V_1 \implies P_2 = 5P_1 = 5p_{\text{aтм}}.
    \\
    P_2 = p_{\text{aтм}} + \rho_{\text{в}} g h \implies h = \frac{P_2 - p_{\text{aтм}}}{\rho_{\text{в}} g} &= \frac{5p_{\text{aтм}} - p_{\text{aтм}}}{\rho_{\text{в}} g} = \frac{4 \cdot p_{\text{aтм}}}{\rho_{\text{в}} g} =  \\
     &= \frac{4 \cdot 100\,\text{кПа}}{1000\,\frac{\text{кг}}{\text{м}^{3}} \cdot  10\,\frac{\text{м}}{\text{с}^{2}}} \approx 40\,\text{м}.
    \end{align*}
}
\solutionspace{120pt}

\tasknumber{15}%
\task{%
    Газу сообщили некоторое количество теплоты,
    при этом половину его он потратил на совершение работы,
    одновременно увеличив свою внутреннюю энергию на $1500\,\text{Дж}$.
    Определите работу, совершённую газом.
}
\answer{%
    \begin{align*}
    Q &= A' + \Delta U, A' = \frac 12 Q \implies Q \cdot \cbr{1 - \frac 12} = \Delta U \implies Q = \frac{\Delta U}{1 - \frac 12} = \frac{ 1500\,\text{Дж} }{1 - \frac 12} \approx 3000\,\text{Дж}.
    \\
    A' &= \frac 12 Q
        = \frac 12 \cdot \frac{\Delta U}{1 - \frac 12}
        = \frac{\Delta U}{2 - 1}
        = \frac{ 1500\,\text{Дж} }{2 - 1} \approx 1500\,\text{Дж}.
    \end{align*}
}
\solutionspace{60pt}

\tasknumber{16}%
\task{%
    Два конденсатора ёмкостей $C_1 = 60\,\text{нФ}$ и $C_2 = 30\,\text{нФ}$ последовательно подключают
    к источнику напряжения $V = 450\,\text{В}$ (см.
    рис.).
    % Определите заряды каждого из конденсаторов.
    Определите заряд второго конденсатора.

    \begin{tikzpicture}[circuit ee IEC, semithick]
        \draw  (0, 0) to [capacitor={info={$C_1$}}] (1, 0)
                       to [capacitor={info={$C_2$}}] (2, 0)
        ;
        % \draw [-o] (0, 0) -- ++(-0.5, 0) node[left] {$-$};
        % \draw [-o] (2, 0) -- ++(0.5, 0) node[right] {$+$};
        \draw [-o] (0, 0) -- ++(-0.5, 0) node[left] {};
        \draw [-o] (2, 0) -- ++(0.5, 0) node[right] {};
    \end{tikzpicture}
}
\answer{%
    $
        Q_1
            = Q_2
            = CV
            = \frac{ V }{\frac1{C_1} + \frac1{C_2}}
            = \frac{C_1C_2V}{C_1 + C_2}
            = \frac{
                60\,\text{нФ} \cdot 30\,\text{нФ} \cdot 450\,\text{В}
            }{
                60\,\text{нФ} + 30\,\text{нФ}
            }
            = 9000{,}00\,\text{нКл}
    $
}
\solutionspace{120pt}

\tasknumber{17}%
\task{%
    В вакууме вдоль одной прямой расположены четыре отрицательных заряда так,
    что расстояние между соседними зарядами равно $l$.
    Сделайте рисунок,
    и определите силу, действующую на крайний заряд.
    Модули всех зарядов равны $q$ ($q > 0$).
}
\solutionspace{80pt}

\tasknumber{18}%
\task{%
    Юлия проводит эксперименты c 2 кусками одинаковой алюминиевой проволки, причём второй кусок в девять раз длиннее первого.
    В одном из экспериментов Юлия подаёт на первый кусок проволки напряжение в восемь раз раз больше, чем на второй.
    Определите отношения в двух проволках в этом эксперименте (второй к первой):
    \begin{itemize}
        \item отношение сил тока,
        \item отношение выделяющихся мощностей.
    \end{itemize}
}
\answer{%
    $\eli_2 / \eli_1 = \frac1{72}, \P_2 / \P_1 = \frac1{72}, $
}

\variantsplitter

\addpersonalvariant{Софья Белянкина}

\tasknumber{1}%
\task{%
    Саша стартует на мотоцикле и в течение $t = 2\,\text{c}$ двигается с постоянным ускорением $2\,\frac{\text{м}}{\text{с}^{2}}$.
    Определите
    \begin{itemize}
        \item какую скорость при этом удастся достичь,
        \item какой путь за это время будет пройден,
        \item среднюю скорость за всё время движения, если после начального ускорения продолжить движение равномерно ещё в течение времени $2t$
    \end{itemize}
}
\solutionspace{120pt}

\tasknumber{2}%
\task{%
    Какой путь тело пройдёт за третью секунду после начала свободного падения?
    Какую скорость в начале этой секунды оно имеет?
}
\solutionspace{120pt}

\tasknumber{3}%
\task{%
    Карусель диаметром $4\,\text{м}$ равномерно совершает 10 оборотов в минуту.
    Определите
    \begin{itemize}
        \item период и частоту её обращения,
        \item скорость и ускорение крайних её точек.
    \end{itemize}
}
\solutionspace{80pt}

\tasknumber{4}%
\task{%
    Миша стоит на обрыве над рекой и методично и строго горизонтально кидает в неё камушки.
    За этим всем наблюдает экспериментатор Глюк, который уже выяснил, что камушки падают в реку спустя $1{,}5\,\text{с}$ после броска,
    а вот дальность полёта оценить сложнее: придётся лезть в воду.
    Выручите Глюка и определите:
    \begin{itemize}
        \item высоту обрыва (вместе с ростом Миши).
        \item дальность полёта камушков (по горизонтали) и их скорость при падении, приняв начальную скорость броска равной $v = 15\,\frac{\text{м}}{\text{с}}$.
    \end{itemize}
    Сопротивлением воздуха пренебречь.
}
\solutionspace{120pt}

\tasknumber{5}%
\task{%
    Пять одинаковых брусков массой $3\,\text{кг}$ каждый лежат на гладком горизонтальном столе.
    Бруски пронумерованы от 1 до 5 и последовательно связаны между собой
    невесомыми нерастяжимыми нитями: 1 со 2, 2 с 3 (ну и с 1) и т.д.
    Экспериментатор Глюк прикладывает постоянную горизонтальную силу $90\,\text{Н}$ к бруску с наибольшим номером.
    С каким ускорением двигается система? Чему равна сила натяжения нити, связывающей бруски 2 и 3?
}
\solutionspace{120pt}

\tasknumber{6}%
\task{%
    Два бруска связаны лёгкой нерастяжимой нитью и перекинуты через неподвижный блок (см.
    рис.).
    Определите силу натяжения нити и ускорения брусков.
    Силами трения пренебречь, массы брусков
    равны $m_1 = 8\,\text{кг}$ и $m_2 = 14\,\text{кг}$.
    % $g = 10\,\frac{\text{м}}{\text{с}^{2}}$.

    \begin{tikzpicture}[x=1.5cm,y=1.5cm,thick]
        \draw
            (-0.4, 0) rectangle (-0.2, 1.2)
            (0.15, 0.5) rectangle (0.45, 1)
            (0, 2) circle [radius=0.3] -- ++(up:0.5)
            (-0.3, 1.2) -- ++(up:0.8)
            (0.3, 1) -- ++(up:1)
            (-0.7, 2.5) -- (0.7, 2.5)
            ;
        \draw[pattern={Lines[angle=51,distance=3pt]},pattern color=black,draw=none] (-0.7, 2.5) rectangle (0.7, 2.75);
        \node [left] (left) at (-0.4, 0.6) { $m_1$ };
        \node [right] (right) at (0.4, 0.75) { $m_2$ };
    \end{tikzpicture}
}
\solutionspace{80pt}

\tasknumber{7}%
\task{%
    Тело массой $2{,}7\,\text{кг}$ лежит на горизонтальной поверхности.
    Коэффициент трения между поверхностью и телом $0{,}15$.
    К телу приложена горизонтальная сила $5{,}5\,\text{Н}$.
    Определите силу трения, действующую на тело, и ускорение тела.
    % $g = 10\,\frac{\text{м}}{\text{с}^{2}}$.
}
\solutionspace{120pt}

\tasknumber{8}%
\task{%
    Определите плотность неизвестного вещества, если известно, что опускании тела из него
    в подсолнечное масло оно будет плавать и на четверть выступать над поверхностью жидкости.
}
\solutionspace{120pt}

\tasknumber{9}%
\task{%
    	Определите силу, действующую на левую опору однородного горизонтального стержня длиной $l = 7\,\text{м}$
    	и массой $M = 5\,\text{кг}$, к которому подвешен груз массой $m = 2\,\text{кг}$ на расстоянии $4\,\text{м}$ от правого конца (см.
    рис.).

        \begin{tikzpicture}[thick]
            \draw
                (-2, -0.1) rectangle (2, 0.1)
                (-0.5, -0.1) -- (-0.5, -1)
                (-0.7, -1) rectangle (-0.3, -1.3)
           		(-2, -0.1) -- +(0.15,-0.9) -- +(-0.15,-0.9) -- cycle
            	(2, -0.1) -- +(0.15,-0.9) -- +(-0.15,-0.9) -- cycle
            ;
            \draw[pattern={Lines[angle=51,distance=2pt]},pattern color=black,draw=none]
            	(-2.15, -1.15) rectangle +(0.3, 0.15)
            	(2.15, -1.15) rectangle +(-0.3, 0.15)
            ;
            \node [right] (m_small) at (-0.3, -1.15) { $m$ };
            \node [above] (M_big) at (0, 0.1) { $M$ };
        \end{tikzpicture}
}
\solutionspace{80pt}

\tasknumber{10}%
\task{%
    Тонкий однородный кусок арматуры длиной $3\,\text{м}$ и массой $30\,\text{кг}$ лежит на горизонтальной поверхности.
    \begin{itemize}
        \item Какую минимальную силу надо приложить к одному из его концов, чтобы оторвать его от этой поверхности?
        \item Какую минимальную работу надо совершить, чтобы поставить его на землю в вертикальное положение?
    \end{itemize}
    % Примите $g = 10\,\frac{\text{м}}{\text{с}^{2}}$.
}
\answer{%
    $A = mg\frac l2 = 450\,\text{Дж}$
}
\solutionspace{120pt}

\tasknumber{11}%
\task{%
    Определите работу силы, которая обеспечит спуск тела массой $2\,\text{кг}$ на высоту $5\,\text{м}$ с постоянным ускорением $4\,\frac{\text{м}}{\text{c}^{2}}$.
    % Примите $g = 10\,\frac{\text{м}}{\text{с}^{2}}$.
}
\answer{%
    \begin{align*}
    &\text{Для подъёма:} A = Fh = (mg + ma) h = m(g+a)h, \\
    &\text{Для спуска:} A = -Fh = -(mg - ma) h = -m(g-a)h, \\
    &\text{В результате получаем:} -60\,\text{Дж}.
    \end{align*}
}
\solutionspace{60pt}

\tasknumber{12}%
\task{%
    Тело бросили вертикально вверх со скоростью $10\,\frac{\text{м}}{\text{c}}$.
    На какой высоте кинетическая энергия тела составит треть от потенциальной?
}
\solutionspace{100pt}

\tasknumber{13}%
\task{%
    Плотность воздуха при нормальных условиях равна $1{,}3\,\frac{\text{кг}}{\text{м}^{3}}$.
    Чему равна плотность воздуха
    при температуре $50\celsius$ и давлении $120\,\text{кПа}$?
}
\solutionspace{120pt}

\tasknumber{14}%
\task{%
    Небольшую цилиндрическую пробирку с воздухом погружают на некоторую глубину в глубокое пресное озеро,
    после чего воздух занимает в ней лишь третью часть от общего объема.
    Определите глубину, на которую погрузили пробирку.
    Температуру считать постоянной $T = 286\,\text{К}$, давлением паров воды пренебречь,
    атмосферное давление принять равным $p_{\text{aтм}} = 100\,\text{кПа}$.
}
\answer{%
    \begin{align*}
    T\text{— const} &\implies P_1V_1 = \nu RT = P_2V_2.
    \\
    V_2 = \frac 13 V_1 &\implies P_1V_1 = P_2 \cdot \frac 13V_1 \implies P_2 = 3P_1 = 3p_{\text{aтм}}.
    \\
    P_2 = p_{\text{aтм}} + \rho_{\text{в}} g h \implies h = \frac{P_2 - p_{\text{aтм}}}{\rho_{\text{в}} g} &= \frac{3p_{\text{aтм}} - p_{\text{aтм}}}{\rho_{\text{в}} g} = \frac{2 \cdot p_{\text{aтм}}}{\rho_{\text{в}} g} =  \\
     &= \frac{2 \cdot 100\,\text{кПа}}{1000\,\frac{\text{кг}}{\text{м}^{3}} \cdot  10\,\frac{\text{м}}{\text{с}^{2}}} \approx 20\,\text{м}.
    \end{align*}
}
\solutionspace{120pt}

\tasknumber{15}%
\task{%
    Газу сообщили некоторое количество теплоты,
    при этом четверть его он потратил на совершение работы,
    одновременно увеличив свою внутреннюю энергию на $1500\,\text{Дж}$.
    Определите количество теплоты, сообщённое газу.
}
\answer{%
    \begin{align*}
    Q &= A' + \Delta U, A' = \frac 14 Q \implies Q \cdot \cbr{1 - \frac 14} = \Delta U \implies Q = \frac{\Delta U}{1 - \frac 14} = \frac{ 1500\,\text{Дж} }{1 - \frac 14} \approx 2000\,\text{Дж}.
    \\
    A' &= \frac 14 Q
        = \frac 14 \cdot \frac{\Delta U}{1 - \frac 14}
        = \frac{\Delta U}{4 - 1}
        = \frac{ 1500\,\text{Дж} }{4 - 1} \approx 500\,\text{Дж}.
    \end{align*}
}
\solutionspace{60pt}

\tasknumber{16}%
\task{%
    Два конденсатора ёмкостей $C_1 = 60\,\text{нФ}$ и $C_2 = 40\,\text{нФ}$ последовательно подключают
    к источнику напряжения $V = 450\,\text{В}$ (см.
    рис.).
    % Определите заряды каждого из конденсаторов.
    Определите заряд первого конденсатора.

    \begin{tikzpicture}[circuit ee IEC, semithick]
        \draw  (0, 0) to [capacitor={info={$C_1$}}] (1, 0)
                       to [capacitor={info={$C_2$}}] (2, 0)
        ;
        % \draw [-o] (0, 0) -- ++(-0.5, 0) node[left] {$-$};
        % \draw [-o] (2, 0) -- ++(0.5, 0) node[right] {$+$};
        \draw [-o] (0, 0) -- ++(-0.5, 0) node[left] {};
        \draw [-o] (2, 0) -- ++(0.5, 0) node[right] {};
    \end{tikzpicture}
}
\answer{%
    $
        Q_1
            = Q_2
            = CV
            = \frac{ V }{\frac1{C_1} + \frac1{C_2}}
            = \frac{C_1C_2V}{C_1 + C_2}
            = \frac{
                60\,\text{нФ} \cdot 40\,\text{нФ} \cdot 450\,\text{В}
            }{
                60\,\text{нФ} + 40\,\text{нФ}
            }
            = 10800{,}00\,\text{нКл}
    $
}
\solutionspace{120pt}

\tasknumber{17}%
\task{%
    В вакууме вдоль одной прямой расположены три отрицательных заряда так,
    что расстояние между соседними зарядами равно $a$.
    Сделайте рисунок,
    и определите силу, действующую на крайний заряд.
    Модули всех зарядов равны $q$ ($q > 0$).
}
\solutionspace{80pt}

\tasknumber{18}%
\task{%
    Юлия проводит эксперименты c 2 кусками одинаковой стальной проволки, причём второй кусок в семь раз длиннее первого.
    В одном из экспериментов Юлия подаёт на первый кусок проволки напряжение в пять раз раз больше, чем на второй.
    Определите отношения в двух проволках в этом эксперименте (второй к первой):
    \begin{itemize}
        \item отношение сил тока,
        \item отношение выделяющихся мощностей.
    \end{itemize}
}
\answer{%
    $\eli_2 / \eli_1 = \frac1{35}, \P_2 / \P_1 = \frac1{35}, $
}

\variantsplitter

\addpersonalvariant{Варвара Егиазарян}

\tasknumber{1}%
\task{%
    Саша стартует на мотоцикле и в течение $t = 10\,\text{c}$ двигается с постоянным ускорением $1{,}5\,\frac{\text{м}}{\text{с}^{2}}$.
    Определите
    \begin{itemize}
        \item какую скорость при этом удастся достичь,
        \item какой путь за это время будет пройден,
        \item среднюю скорость за всё время движения, если после начального ускорения продолжить движение равномерно ещё в течение времени $3t$
    \end{itemize}
}
\solutionspace{120pt}

\tasknumber{2}%
\task{%
    Какой путь тело пройдёт за пятую секунду после начала свободного падения?
    Какую скорость в конце этой секунды оно имеет?
}
\solutionspace{120pt}

\tasknumber{3}%
\task{%
    Карусель диаметром $5\,\text{м}$ равномерно совершает 10 оборотов в минуту.
    Определите
    \begin{itemize}
        \item период и частоту её обращения,
        \item скорость и ускорение крайних её точек.
    \end{itemize}
}
\solutionspace{80pt}

\tasknumber{4}%
\task{%
    Маша стоит на обрыве над рекой и методично и строго горизонтально кидает в неё камушки.
    За этим всем наблюдает экспериментатор Глюк, который уже выяснил, что камушки падают в реку спустя $1{,}2\,\text{с}$ после броска,
    а вот дальность полёта оценить сложнее: придётся лезть в воду.
    Выручите Глюка и определите:
    \begin{itemize}
        \item высоту обрыва (вместе с ростом Маши).
        \item дальность полёта камушков (по горизонтали) и их скорость при падении, приняв начальную скорость броска равной $v = 14\,\frac{\text{м}}{\text{с}}$.
    \end{itemize}
    Сопротивлением воздуха пренебречь.
}
\solutionspace{120pt}

\tasknumber{5}%
\task{%
    Шесть одинаковых брусков массой $2\,\text{кг}$ каждый лежат на гладком горизонтальном столе.
    Бруски пронумерованы от 1 до 6 и последовательно связаны между собой
    невесомыми нерастяжимыми нитями: 1 со 2, 2 с 3 (ну и с 1) и т.д.
    Экспериментатор Глюк прикладывает постоянную горизонтальную силу $60\,\text{Н}$ к бруску с наименьшим номером.
    С каким ускорением двигается система? Чему равна сила натяжения нити, связывающей бруски 2 и 3?
}
\solutionspace{120pt}

\tasknumber{6}%
\task{%
    Два бруска связаны лёгкой нерастяжимой нитью и перекинуты через неподвижный блок (см.
    рис.).
    Определите силу натяжения нити и ускорения брусков.
    Силами трения пренебречь, массы брусков
    равны $m_1 = 8\,\text{кг}$ и $m_2 = 6\,\text{кг}$.
    % $g = 10\,\frac{\text{м}}{\text{с}^{2}}$.

    \begin{tikzpicture}[x=1.5cm,y=1.5cm,thick]
        \draw
            (-0.4, 0) rectangle (-0.2, 1.2)
            (0.15, 0.5) rectangle (0.45, 1)
            (0, 2) circle [radius=0.3] -- ++(up:0.5)
            (-0.3, 1.2) -- ++(up:0.8)
            (0.3, 1) -- ++(up:1)
            (-0.7, 2.5) -- (0.7, 2.5)
            ;
        \draw[pattern={Lines[angle=51,distance=3pt]},pattern color=black,draw=none] (-0.7, 2.5) rectangle (0.7, 2.75);
        \node [left] (left) at (-0.4, 0.6) { $m_1$ };
        \node [right] (right) at (0.4, 0.75) { $m_2$ };
    \end{tikzpicture}
}
\solutionspace{80pt}

\tasknumber{7}%
\task{%
    Тело массой $1{,}4\,\text{кг}$ лежит на горизонтальной поверхности.
    Коэффициент трения между поверхностью и телом $0{,}15$.
    К телу приложена горизонтальная сила $2{,}5\,\text{Н}$.
    Определите силу трения, действующую на тело, и ускорение тела.
    % $g = 10\,\frac{\text{м}}{\text{с}^{2}}$.
}
\solutionspace{120pt}

\tasknumber{8}%
\task{%
    Определите плотность неизвестного вещества, если известно, что опускании тела из него
    в керосин оно будет плавать и на половину выступать над поверхностью жидкости.
}
\solutionspace{120pt}

\tasknumber{9}%
\task{%
    	Определите силу, действующую на правую опору однородного горизонтального стержня длиной $l = 3\,\text{м}$
    	и массой $M = 1\,\text{кг}$, к которому подвешен груз массой $m = 3\,\text{кг}$ на расстоянии $2\,\text{м}$ от правого конца (см.
    рис.).

        \begin{tikzpicture}[thick]
            \draw
                (-2, -0.1) rectangle (2, 0.1)
                (-0.5, -0.1) -- (-0.5, -1)
                (-0.7, -1) rectangle (-0.3, -1.3)
           		(-2, -0.1) -- +(0.15,-0.9) -- +(-0.15,-0.9) -- cycle
            	(2, -0.1) -- +(0.15,-0.9) -- +(-0.15,-0.9) -- cycle
            ;
            \draw[pattern={Lines[angle=51,distance=2pt]},pattern color=black,draw=none]
            	(-2.15, -1.15) rectangle +(0.3, 0.15)
            	(2.15, -1.15) rectangle +(-0.3, 0.15)
            ;
            \node [right] (m_small) at (-0.3, -1.15) { $m$ };
            \node [above] (M_big) at (0, 0.1) { $M$ };
        \end{tikzpicture}
}
\solutionspace{80pt}

\tasknumber{10}%
\task{%
    Тонкий однородный кусок арматуры длиной $2\,\text{м}$ и массой $10\,\text{кг}$ лежит на горизонтальной поверхности.
    \begin{itemize}
        \item Какую минимальную силу надо приложить к одному из его концов, чтобы оторвать его от этой поверхности?
        \item Какую минимальную работу надо совершить, чтобы поставить его на землю в вертикальное положение?
    \end{itemize}
    % Примите $g = 10\,\frac{\text{м}}{\text{с}^{2}}$.
}
\answer{%
    $A = mg\frac l2 = 100\,\text{Дж}$
}
\solutionspace{120pt}

\tasknumber{11}%
\task{%
    Определите работу силы, которая обеспечит подъём тела массой $2\,\text{кг}$ на высоту $5\,\text{м}$ с постоянным ускорением $2\,\frac{\text{м}}{\text{c}^{2}}$.
    % Примите $g = 10\,\frac{\text{м}}{\text{с}^{2}}$.
}
\answer{%
    \begin{align*}
    &\text{Для подъёма:} A = Fh = (mg + ma) h = m(g+a)h, \\
    &\text{Для спуска:} A = -Fh = -(mg - ma) h = -m(g-a)h, \\
    &\text{В результате получаем:} 120\,\text{Дж}.
    \end{align*}
}
\solutionspace{60pt}

\tasknumber{12}%
\task{%
    Тело бросили вертикально вверх со скоростью $14\,\frac{\text{м}}{\text{c}}$.
    На какой высоте кинетическая энергия тела составит треть от потенциальной?
}
\solutionspace{100pt}

\tasknumber{13}%
\task{%
    Плотность воздуха при нормальных условиях равна $1{,}3\,\frac{\text{кг}}{\text{м}^{3}}$.
    Чему равна плотность воздуха
    при температуре $100\celsius$ и давлении $50\,\text{кПа}$?
}
\solutionspace{120pt}

\tasknumber{14}%
\task{%
    Небольшую цилиндрическую пробирку с воздухом погружают на некоторую глубину в глубокое пресное озеро,
    после чего воздух занимает в ней лишь третью часть от общего объема.
    Определите глубину, на которую погрузили пробирку.
    Температуру считать постоянной $T = 278\,\text{К}$, давлением паров воды пренебречь,
    атмосферное давление принять равным $p_{\text{aтм}} = 100\,\text{кПа}$.
}
\answer{%
    \begin{align*}
    T\text{— const} &\implies P_1V_1 = \nu RT = P_2V_2.
    \\
    V_2 = \frac 13 V_1 &\implies P_1V_1 = P_2 \cdot \frac 13V_1 \implies P_2 = 3P_1 = 3p_{\text{aтм}}.
    \\
    P_2 = p_{\text{aтм}} + \rho_{\text{в}} g h \implies h = \frac{P_2 - p_{\text{aтм}}}{\rho_{\text{в}} g} &= \frac{3p_{\text{aтм}} - p_{\text{aтм}}}{\rho_{\text{в}} g} = \frac{2 \cdot p_{\text{aтм}}}{\rho_{\text{в}} g} =  \\
     &= \frac{2 \cdot 100\,\text{кПа}}{1000\,\frac{\text{кг}}{\text{м}^{3}} \cdot  10\,\frac{\text{м}}{\text{с}^{2}}} \approx 20\,\text{м}.
    \end{align*}
}
\solutionspace{120pt}

\tasknumber{15}%
\task{%
    Газу сообщили некоторое количество теплоты,
    при этом половину его он потратил на совершение работы,
    одновременно увеличив свою внутреннюю энергию на $1500\,\text{Дж}$.
    Определите количество теплоты, сообщённое газу.
}
\answer{%
    \begin{align*}
    Q &= A' + \Delta U, A' = \frac 12 Q \implies Q \cdot \cbr{1 - \frac 12} = \Delta U \implies Q = \frac{\Delta U}{1 - \frac 12} = \frac{ 1500\,\text{Дж} }{1 - \frac 12} \approx 3000\,\text{Дж}.
    \\
    A' &= \frac 12 Q
        = \frac 12 \cdot \frac{\Delta U}{1 - \frac 12}
        = \frac{\Delta U}{2 - 1}
        = \frac{ 1500\,\text{Дж} }{2 - 1} \approx 1500\,\text{Дж}.
    \end{align*}
}
\solutionspace{60pt}

\tasknumber{16}%
\task{%
    Два конденсатора ёмкостей $C_1 = 30\,\text{нФ}$ и $C_2 = 60\,\text{нФ}$ последовательно подключают
    к источнику напряжения $V = 200\,\text{В}$ (см.
    рис.).
    % Определите заряды каждого из конденсаторов.
    Определите заряд второго конденсатора.

    \begin{tikzpicture}[circuit ee IEC, semithick]
        \draw  (0, 0) to [capacitor={info={$C_1$}}] (1, 0)
                       to [capacitor={info={$C_2$}}] (2, 0)
        ;
        % \draw [-o] (0, 0) -- ++(-0.5, 0) node[left] {$-$};
        % \draw [-o] (2, 0) -- ++(0.5, 0) node[right] {$+$};
        \draw [-o] (0, 0) -- ++(-0.5, 0) node[left] {};
        \draw [-o] (2, 0) -- ++(0.5, 0) node[right] {};
    \end{tikzpicture}
}
\answer{%
    $
        Q_1
            = Q_2
            = CV
            = \frac{ V }{\frac1{C_1} + \frac1{C_2}}
            = \frac{C_1C_2V}{C_1 + C_2}
            = \frac{
                30\,\text{нФ} \cdot 60\,\text{нФ} \cdot 200\,\text{В}
            }{
                30\,\text{нФ} + 60\,\text{нФ}
            }
            = 4000{,}00\,\text{нКл}
    $
}
\solutionspace{120pt}

\tasknumber{17}%
\task{%
    В вакууме вдоль одной прямой расположены четыре положительных заряда так,
    что расстояние между соседними зарядами равно $a$.
    Сделайте рисунок,
    и определите силу, действующую на крайний заряд.
    Модули всех зарядов равны $q$ ($q > 0$).
}
\solutionspace{80pt}

\tasknumber{18}%
\task{%
    Юлия проводит эксперименты c 2 кусками одинаковой стальной проволки, причём второй кусок в восемь раз длиннее первого.
    В одном из экспериментов Юлия подаёт на первый кусок проволки напряжение в восемь раз раз больше, чем на второй.
    Определите отношения в двух проволках в этом эксперименте (второй к первой):
    \begin{itemize}
        \item отношение сил тока,
        \item отношение выделяющихся мощностей.
    \end{itemize}
}
\answer{%
    $\eli_2 / \eli_1 = \frac1{64}, \P_2 / \P_1 = \frac1{64}, $
}

\variantsplitter

\addpersonalvariant{Владислав Емелин}

\tasknumber{1}%
\task{%
    Валя стартует на велосипеде и в течение $t = 4\,\text{c}$ двигается с постоянным ускорением $2\,\frac{\text{м}}{\text{с}^{2}}$.
    Определите
    \begin{itemize}
        \item какую скорость при этом удастся достичь,
        \item какой путь за это время будет пройден,
        \item среднюю скорость за всё время движения, если после начального ускорения продолжить движение равномерно ещё в течение времени $2t$
    \end{itemize}
}
\solutionspace{120pt}

\tasknumber{2}%
\task{%
    Какой путь тело пройдёт за четвёртую секунду после начала свободного падения?
    Какую скорость в начале этой секунды оно имеет?
}
\solutionspace{120pt}

\tasknumber{3}%
\task{%
    Карусель радиусом $2\,\text{м}$ равномерно совершает 6 оборотов в минуту.
    Определите
    \begin{itemize}
        \item период и частоту её обращения,
        \item скорость и ускорение крайних её точек.
    \end{itemize}
}
\solutionspace{80pt}

\tasknumber{4}%
\task{%
    Паша стоит на обрыве над рекой и методично и строго горизонтально кидает в неё камушки.
    За этим всем наблюдает экспериментатор Глюк, который уже выяснил, что камушки падают в реку спустя $1{,}6\,\text{с}$ после броска,
    а вот дальность полёта оценить сложнее: придётся лезть в воду.
    Выручите Глюка и определите:
    \begin{itemize}
        \item высоту обрыва (вместе с ростом Паши).
        \item дальность полёта камушков (по горизонтали) и их скорость при падении, приняв начальную скорость броска равной $v = 16\,\frac{\text{м}}{\text{с}}$.
    \end{itemize}
    Сопротивлением воздуха пренебречь.
}
\solutionspace{120pt}

\tasknumber{5}%
\task{%
    Четыре одинаковых брусков массой $2\,\text{кг}$ каждый лежат на гладком горизонтальном столе.
    Бруски пронумерованы от 1 до 4 и последовательно связаны между собой
    невесомыми нерастяжимыми нитями: 1 со 2, 2 с 3 (ну и с 1) и т.д.
    Экспериментатор Глюк прикладывает постоянную горизонтальную силу $90\,\text{Н}$ к бруску с наибольшим номером.
    С каким ускорением двигается система? Чему равна сила натяжения нити, связывающей бруски 3 и 4?
}
\solutionspace{120pt}

\tasknumber{6}%
\task{%
    Два бруска связаны лёгкой нерастяжимой нитью и перекинуты через неподвижный блок (см.
    рис.).
    Определите силу натяжения нити и ускорения брусков.
    Силами трения пренебречь, массы брусков
    равны $m_1 = 8\,\text{кг}$ и $m_2 = 14\,\text{кг}$.
    % $g = 10\,\frac{\text{м}}{\text{с}^{2}}$.

    \begin{tikzpicture}[x=1.5cm,y=1.5cm,thick]
        \draw
            (-0.4, 0) rectangle (-0.2, 1.2)
            (0.15, 0.5) rectangle (0.45, 1)
            (0, 2) circle [radius=0.3] -- ++(up:0.5)
            (-0.3, 1.2) -- ++(up:0.8)
            (0.3, 1) -- ++(up:1)
            (-0.7, 2.5) -- (0.7, 2.5)
            ;
        \draw[pattern={Lines[angle=51,distance=3pt]},pattern color=black,draw=none] (-0.7, 2.5) rectangle (0.7, 2.75);
        \node [left] (left) at (-0.4, 0.6) { $m_1$ };
        \node [right] (right) at (0.4, 0.75) { $m_2$ };
    \end{tikzpicture}
}
\solutionspace{80pt}

\tasknumber{7}%
\task{%
    Тело массой $2{,}7\,\text{кг}$ лежит на горизонтальной поверхности.
    Коэффициент трения между поверхностью и телом $0{,}25$.
    К телу приложена горизонтальная сила $3{,}5\,\text{Н}$.
    Определите силу трения, действующую на тело, и ускорение тела.
    % $g = 10\,\frac{\text{м}}{\text{с}^{2}}$.
}
\solutionspace{120pt}

\tasknumber{8}%
\task{%
    Определите плотность неизвестного вещества, если известно, что опускании тела из него
    в подсолнечное масло оно будет плавать и на половину выступать над поверхностью жидкости.
}
\solutionspace{120pt}

\tasknumber{9}%
\task{%
    	Определите силу, действующую на левую опору однородного горизонтального стержня длиной $l = 3\,\text{м}$
    	и массой $M = 1\,\text{кг}$, к которому подвешен груз массой $m = 4\,\text{кг}$ на расстоянии $2\,\text{м}$ от правого конца (см.
    рис.).

        \begin{tikzpicture}[thick]
            \draw
                (-2, -0.1) rectangle (2, 0.1)
                (-0.5, -0.1) -- (-0.5, -1)
                (-0.7, -1) rectangle (-0.3, -1.3)
           		(-2, -0.1) -- +(0.15,-0.9) -- +(-0.15,-0.9) -- cycle
            	(2, -0.1) -- +(0.15,-0.9) -- +(-0.15,-0.9) -- cycle
            ;
            \draw[pattern={Lines[angle=51,distance=2pt]},pattern color=black,draw=none]
            	(-2.15, -1.15) rectangle +(0.3, 0.15)
            	(2.15, -1.15) rectangle +(-0.3, 0.15)
            ;
            \node [right] (m_small) at (-0.3, -1.15) { $m$ };
            \node [above] (M_big) at (0, 0.1) { $M$ };
        \end{tikzpicture}
}
\solutionspace{80pt}

\tasknumber{10}%
\task{%
    Тонкий однородный шест длиной $1\,\text{м}$ и массой $10\,\text{кг}$ лежит на горизонтальной поверхности.
    \begin{itemize}
        \item Какую минимальную силу надо приложить к одному из его концов, чтобы оторвать его от этой поверхности?
        \item Какую минимальную работу надо совершить, чтобы поставить его на землю в вертикальное положение?
    \end{itemize}
    % Примите $g = 10\,\frac{\text{м}}{\text{с}^{2}}$.
}
\answer{%
    $A = mg\frac l2 = 50\,\text{Дж}$
}
\solutionspace{120pt}

\tasknumber{11}%
\task{%
    Определите работу силы, которая обеспечит подъём тела массой $2\,\text{кг}$ на высоту $5\,\text{м}$ с постоянным ускорением $2\,\frac{\text{м}}{\text{c}^{2}}$.
    % Примите $g = 10\,\frac{\text{м}}{\text{с}^{2}}$.
}
\answer{%
    \begin{align*}
    &\text{Для подъёма:} A = Fh = (mg + ma) h = m(g+a)h, \\
    &\text{Для спуска:} A = -Fh = -(mg - ma) h = -m(g-a)h, \\
    &\text{В результате получаем:} 120\,\text{Дж}.
    \end{align*}
}
\solutionspace{60pt}

\tasknumber{12}%
\task{%
    Тело бросили вертикально вверх со скоростью $20\,\frac{\text{м}}{\text{c}}$.
    На какой высоте кинетическая энергия тела составит половину от потенциальной?
}
\solutionspace{100pt}

\tasknumber{13}%
\task{%
    Плотность воздуха при нормальных условиях равна $1{,}3\,\frac{\text{кг}}{\text{м}^{3}}$.
    Чему равна плотность воздуха
    при температуре $200\celsius$ и давлении $50\,\text{кПа}$?
}
\solutionspace{120pt}

\tasknumber{14}%
\task{%
    Небольшую цилиндрическую пробирку с воздухом погружают на некоторую глубину в глубокое пресное озеро,
    после чего воздух занимает в ней лишь третью часть от общего объема.
    Определите глубину, на которую погрузили пробирку.
    Температуру считать постоянной $T = 279\,\text{К}$, давлением паров воды пренебречь,
    атмосферное давление принять равным $p_{\text{aтм}} = 100\,\text{кПа}$.
}
\answer{%
    \begin{align*}
    T\text{— const} &\implies P_1V_1 = \nu RT = P_2V_2.
    \\
    V_2 = \frac 13 V_1 &\implies P_1V_1 = P_2 \cdot \frac 13V_1 \implies P_2 = 3P_1 = 3p_{\text{aтм}}.
    \\
    P_2 = p_{\text{aтм}} + \rho_{\text{в}} g h \implies h = \frac{P_2 - p_{\text{aтм}}}{\rho_{\text{в}} g} &= \frac{3p_{\text{aтм}} - p_{\text{aтм}}}{\rho_{\text{в}} g} = \frac{2 \cdot p_{\text{aтм}}}{\rho_{\text{в}} g} =  \\
     &= \frac{2 \cdot 100\,\text{кПа}}{1000\,\frac{\text{кг}}{\text{м}^{3}} \cdot  10\,\frac{\text{м}}{\text{с}^{2}}} \approx 20\,\text{м}.
    \end{align*}
}
\solutionspace{120pt}

\tasknumber{15}%
\task{%
    Газу сообщили некоторое количество теплоты,
    при этом четверть его он потратил на совершение работы,
    одновременно увеличив свою внутреннюю энергию на $3000\,\text{Дж}$.
    Определите работу, совершённую газом.
}
\answer{%
    \begin{align*}
    Q &= A' + \Delta U, A' = \frac 14 Q \implies Q \cdot \cbr{1 - \frac 14} = \Delta U \implies Q = \frac{\Delta U}{1 - \frac 14} = \frac{ 3000\,\text{Дж} }{1 - \frac 14} \approx 4000\,\text{Дж}.
    \\
    A' &= \frac 14 Q
        = \frac 14 \cdot \frac{\Delta U}{1 - \frac 14}
        = \frac{\Delta U}{4 - 1}
        = \frac{ 3000\,\text{Дж} }{4 - 1} \approx 1000\,\text{Дж}.
    \end{align*}
}
\solutionspace{60pt}

\tasknumber{16}%
\task{%
    Два конденсатора ёмкостей $C_1 = 30\,\text{нФ}$ и $C_2 = 60\,\text{нФ}$ последовательно подключают
    к источнику напряжения $U = 300\,\text{В}$ (см.
    рис.).
    % Определите заряды каждого из конденсаторов.
    Определите заряд второго конденсатора.

    \begin{tikzpicture}[circuit ee IEC, semithick]
        \draw  (0, 0) to [capacitor={info={$C_1$}}] (1, 0)
                       to [capacitor={info={$C_2$}}] (2, 0)
        ;
        % \draw [-o] (0, 0) -- ++(-0.5, 0) node[left] {$-$};
        % \draw [-o] (2, 0) -- ++(0.5, 0) node[right] {$+$};
        \draw [-o] (0, 0) -- ++(-0.5, 0) node[left] {};
        \draw [-o] (2, 0) -- ++(0.5, 0) node[right] {};
    \end{tikzpicture}
}
\answer{%
    $
        Q_1
            = Q_2
            = CU
            = \frac{ U }{\frac1{C_1} + \frac1{C_2}}
            = \frac{C_1C_2U}{C_1 + C_2}
            = \frac{
                30\,\text{нФ} \cdot 60\,\text{нФ} \cdot 300\,\text{В}
            }{
                30\,\text{нФ} + 60\,\text{нФ}
            }
            = 6000{,}00\,\text{нКл}
    $
}
\solutionspace{120pt}

\tasknumber{17}%
\task{%
    В вакууме вдоль одной прямой расположены четыре положительных заряда так,
    что расстояние между соседними зарядами равно $l$.
    Сделайте рисунок,
    и определите силу, действующую на крайний заряд.
    Модули всех зарядов равны $Q$ ($Q > 0$).
}
\solutionspace{80pt}

\tasknumber{18}%
\task{%
    Юлия проводит эксперименты c 2 кусками одинаковой медной проволки, причём второй кусок в семь раз длиннее первого.
    В одном из экспериментов Юлия подаёт на первый кусок проволки напряжение в два раза раз больше, чем на второй.
    Определите отношения в двух проволках в этом эксперименте (второй к первой):
    \begin{itemize}
        \item отношение сил тока,
        \item отношение выделяющихся мощностей.
    \end{itemize}
}
\answer{%
    $\eli_2 / \eli_1 = \frac1{14}, \P_2 / \P_1 = \frac1{14}, $
}

\variantsplitter

\addpersonalvariant{Артём Жичин}

\tasknumber{1}%
\task{%
    Саша стартует на лошади и в течение $t = 10\,\text{c}$ двигается с постоянным ускорением $0{,}5\,\frac{\text{м}}{\text{с}^{2}}$.
    Определите
    \begin{itemize}
        \item какую скорость при этом удастся достичь,
        \item какой путь за это время будет пройден,
        \item среднюю скорость за всё время движения, если после начального ускорения продолжить движение равномерно ещё в течение времени $3t$
    \end{itemize}
}
\solutionspace{120pt}

\tasknumber{2}%
\task{%
    Какой путь тело пройдёт за четвёртую секунду после начала свободного падения?
    Какую скорость в начале этой секунды оно имеет?
}
\solutionspace{120pt}

\tasknumber{3}%
\task{%
    Карусель радиусом $4\,\text{м}$ равномерно совершает 6 оборотов в минуту.
    Определите
    \begin{itemize}
        \item период и частоту её обращения,
        \item скорость и ускорение крайних её точек.
    \end{itemize}
}
\solutionspace{80pt}

\tasknumber{4}%
\task{%
    Паша стоит на обрыве над рекой и методично и строго горизонтально кидает в неё камушки.
    За этим всем наблюдает экспериментатор Глюк, который уже выяснил, что камушки падают в реку спустя $1{,}3\,\text{с}$ после броска,
    а вот дальность полёта оценить сложнее: придётся лезть в воду.
    Выручите Глюка и определите:
    \begin{itemize}
        \item высоту обрыва (вместе с ростом Паши).
        \item дальность полёта камушков (по горизонтали) и их скорость при падении, приняв начальную скорость броска равной $v = 13\,\frac{\text{м}}{\text{с}}$.
    \end{itemize}
    Сопротивлением воздуха пренебречь.
}
\solutionspace{120pt}

\tasknumber{5}%
\task{%
    Шесть одинаковых брусков массой $3\,\text{кг}$ каждый лежат на гладком горизонтальном столе.
    Бруски пронумерованы от 1 до 6 и последовательно связаны между собой
    невесомыми нерастяжимыми нитями: 1 со 2, 2 с 3 (ну и с 1) и т.д.
    Экспериментатор Глюк прикладывает постоянную горизонтальную силу $60\,\text{Н}$ к бруску с наибольшим номером.
    С каким ускорением двигается система? Чему равна сила натяжения нити, связывающей бруски 2 и 3?
}
\solutionspace{120pt}

\tasknumber{6}%
\task{%
    Два бруска связаны лёгкой нерастяжимой нитью и перекинуты через неподвижный блок (см.
    рис.).
    Определите силу натяжения нити и ускорения брусков.
    Силами трения пренебречь, массы брусков
    равны $m_1 = 11\,\text{кг}$ и $m_2 = 10\,\text{кг}$.
    % $g = 10\,\frac{\text{м}}{\text{с}^{2}}$.

    \begin{tikzpicture}[x=1.5cm,y=1.5cm,thick]
        \draw
            (-0.4, 0) rectangle (-0.2, 1.2)
            (0.15, 0.5) rectangle (0.45, 1)
            (0, 2) circle [radius=0.3] -- ++(up:0.5)
            (-0.3, 1.2) -- ++(up:0.8)
            (0.3, 1) -- ++(up:1)
            (-0.7, 2.5) -- (0.7, 2.5)
            ;
        \draw[pattern={Lines[angle=51,distance=3pt]},pattern color=black,draw=none] (-0.7, 2.5) rectangle (0.7, 2.75);
        \node [left] (left) at (-0.4, 0.6) { $m_1$ };
        \node [right] (right) at (0.4, 0.75) { $m_2$ };
    \end{tikzpicture}
}
\solutionspace{80pt}

\tasknumber{7}%
\task{%
    Тело массой $2\,\text{кг}$ лежит на горизонтальной поверхности.
    Коэффициент трения между поверхностью и телом $0{,}2$.
    К телу приложена горизонтальная сила $4{,}5\,\text{Н}$.
    Определите силу трения, действующую на тело, и ускорение тела.
    % $g = 10\,\frac{\text{м}}{\text{с}^{2}}$.
}
\solutionspace{120pt}

\tasknumber{8}%
\task{%
    Определите плотность неизвестного вещества, если известно, что опускании тела из него
    в керосин оно будет плавать и на половину выступать над поверхностью жидкости.
}
\solutionspace{120pt}

\tasknumber{9}%
\task{%
    	Определите силу, действующую на правую опору однородного горизонтального стержня длиной $l = 7\,\text{м}$
    	и массой $M = 1\,\text{кг}$, к которому подвешен груз массой $m = 2\,\text{кг}$ на расстоянии $2\,\text{м}$ от правого конца (см.
    рис.).

        \begin{tikzpicture}[thick]
            \draw
                (-2, -0.1) rectangle (2, 0.1)
                (-0.5, -0.1) -- (-0.5, -1)
                (-0.7, -1) rectangle (-0.3, -1.3)
           		(-2, -0.1) -- +(0.15,-0.9) -- +(-0.15,-0.9) -- cycle
            	(2, -0.1) -- +(0.15,-0.9) -- +(-0.15,-0.9) -- cycle
            ;
            \draw[pattern={Lines[angle=51,distance=2pt]},pattern color=black,draw=none]
            	(-2.15, -1.15) rectangle +(0.3, 0.15)
            	(2.15, -1.15) rectangle +(-0.3, 0.15)
            ;
            \node [right] (m_small) at (-0.3, -1.15) { $m$ };
            \node [above] (M_big) at (0, 0.1) { $M$ };
        \end{tikzpicture}
}
\solutionspace{80pt}

\tasknumber{10}%
\task{%
    Тонкий однородный кусок арматуры длиной $3\,\text{м}$ и массой $10\,\text{кг}$ лежит на горизонтальной поверхности.
    \begin{itemize}
        \item Какую минимальную силу надо приложить к одному из его концов, чтобы оторвать его от этой поверхности?
        \item Какую минимальную работу надо совершить, чтобы поставить его на землю в вертикальное положение?
    \end{itemize}
    % Примите $g = 10\,\frac{\text{м}}{\text{с}^{2}}$.
}
\answer{%
    $A = mg\frac l2 = 150\,\text{Дж}$
}
\solutionspace{120pt}

\tasknumber{11}%
\task{%
    Определите работу силы, которая обеспечит подъём тела массой $3\,\text{кг}$ на высоту $2\,\text{м}$ с постоянным ускорением $4\,\frac{\text{м}}{\text{c}^{2}}$.
    % Примите $g = 10\,\frac{\text{м}}{\text{с}^{2}}$.
}
\answer{%
    \begin{align*}
    &\text{Для подъёма:} A = Fh = (mg + ma) h = m(g+a)h, \\
    &\text{Для спуска:} A = -Fh = -(mg - ma) h = -m(g-a)h, \\
    &\text{В результате получаем:} 84\,\text{Дж}.
    \end{align*}
}
\solutionspace{60pt}

\tasknumber{12}%
\task{%
    Тело бросили вертикально вверх со скоростью $20\,\frac{\text{м}}{\text{c}}$.
    На какой высоте кинетическая энергия тела составит половину от потенциальной?
}
\solutionspace{100pt}

\tasknumber{13}%
\task{%
    Плотность воздуха при нормальных условиях равна $1{,}3\,\frac{\text{кг}}{\text{м}^{3}}$.
    Чему равна плотность воздуха
    при температуре $100\celsius$ и давлении $80\,\text{кПа}$?
}
\solutionspace{120pt}

\tasknumber{14}%
\task{%
    Небольшую цилиндрическую пробирку с воздухом погружают на некоторую глубину в глубокое пресное озеро,
    после чего воздух занимает в ней лишь третью часть от общего объема.
    Определите глубину, на которую погрузили пробирку.
    Температуру считать постоянной $T = 280\,\text{К}$, давлением паров воды пренебречь,
    атмосферное давление принять равным $p_{\text{aтм}} = 100\,\text{кПа}$.
}
\answer{%
    \begin{align*}
    T\text{— const} &\implies P_1V_1 = \nu RT = P_2V_2.
    \\
    V_2 = \frac 13 V_1 &\implies P_1V_1 = P_2 \cdot \frac 13V_1 \implies P_2 = 3P_1 = 3p_{\text{aтм}}.
    \\
    P_2 = p_{\text{aтм}} + \rho_{\text{в}} g h \implies h = \frac{P_2 - p_{\text{aтм}}}{\rho_{\text{в}} g} &= \frac{3p_{\text{aтм}} - p_{\text{aтм}}}{\rho_{\text{в}} g} = \frac{2 \cdot p_{\text{aтм}}}{\rho_{\text{в}} g} =  \\
     &= \frac{2 \cdot 100\,\text{кПа}}{1000\,\frac{\text{кг}}{\text{м}^{3}} \cdot  10\,\frac{\text{м}}{\text{с}^{2}}} \approx 20\,\text{м}.
    \end{align*}
}
\solutionspace{120pt}

\tasknumber{15}%
\task{%
    Газу сообщили некоторое количество теплоты,
    при этом треть его он потратил на совершение работы,
    одновременно увеличив свою внутреннюю энергию на $3000\,\text{Дж}$.
    Определите работу, совершённую газом.
}
\answer{%
    \begin{align*}
    Q &= A' + \Delta U, A' = \frac 13 Q \implies Q \cdot \cbr{1 - \frac 13} = \Delta U \implies Q = \frac{\Delta U}{1 - \frac 13} = \frac{ 3000\,\text{Дж} }{1 - \frac 13} \approx 4500\,\text{Дж}.
    \\
    A' &= \frac 13 Q
        = \frac 13 \cdot \frac{\Delta U}{1 - \frac 13}
        = \frac{\Delta U}{3 - 1}
        = \frac{ 3000\,\text{Дж} }{3 - 1} \approx 1500\,\text{Дж}.
    \end{align*}
}
\solutionspace{60pt}

\tasknumber{16}%
\task{%
    Два конденсатора ёмкостей $C_1 = 30\,\text{нФ}$ и $C_2 = 60\,\text{нФ}$ последовательно подключают
    к источнику напряжения $U = 300\,\text{В}$ (см.
    рис.).
    % Определите заряды каждого из конденсаторов.
    Определите заряд второго конденсатора.

    \begin{tikzpicture}[circuit ee IEC, semithick]
        \draw  (0, 0) to [capacitor={info={$C_1$}}] (1, 0)
                       to [capacitor={info={$C_2$}}] (2, 0)
        ;
        % \draw [-o] (0, 0) -- ++(-0.5, 0) node[left] {$-$};
        % \draw [-o] (2, 0) -- ++(0.5, 0) node[right] {$+$};
        \draw [-o] (0, 0) -- ++(-0.5, 0) node[left] {};
        \draw [-o] (2, 0) -- ++(0.5, 0) node[right] {};
    \end{tikzpicture}
}
\answer{%
    $
        Q_1
            = Q_2
            = CU
            = \frac{ U }{\frac1{C_1} + \frac1{C_2}}
            = \frac{C_1C_2U}{C_1 + C_2}
            = \frac{
                30\,\text{нФ} \cdot 60\,\text{нФ} \cdot 300\,\text{В}
            }{
                30\,\text{нФ} + 60\,\text{нФ}
            }
            = 6000{,}00\,\text{нКл}
    $
}
\solutionspace{120pt}

\tasknumber{17}%
\task{%
    В вакууме вдоль одной прямой расположены четыре отрицательных заряда так,
    что расстояние между соседними зарядами равно $a$.
    Сделайте рисунок,
    и определите силу, действующую на крайний заряд.
    Модули всех зарядов равны $q$ ($q > 0$).
}
\solutionspace{80pt}

\tasknumber{18}%
\task{%
    Юлия проводит эксперименты c 2 кусками одинаковой стальной проволки, причём второй кусок в семь раз длиннее первого.
    В одном из экспериментов Юлия подаёт на первый кусок проволки напряжение в три раза раз больше, чем на второй.
    Определите отношения в двух проволках в этом эксперименте (второй к первой):
    \begin{itemize}
        \item отношение сил тока,
        \item отношение выделяющихся мощностей.
    \end{itemize}
}
\answer{%
    $\eli_2 / \eli_1 = \frac1{21}, \P_2 / \P_1 = \frac1{21}, $
}

\variantsplitter

\addpersonalvariant{Дарья Кошман}

\tasknumber{1}%
\task{%
    Саша стартует на велосипеде и в течение $t = 5\,\text{c}$ двигается с постоянным ускорением $2{,}5\,\frac{\text{м}}{\text{с}^{2}}$.
    Определите
    \begin{itemize}
        \item какую скорость при этом удастся достичь,
        \item какой путь за это время будет пройден,
        \item среднюю скорость за всё время движения, если после начального ускорения продолжить движение равномерно ещё в течение времени $3t$
    \end{itemize}
}
\solutionspace{120pt}

\tasknumber{2}%
\task{%
    Какой путь тело пройдёт за шестую секунду после начала свободного падения?
    Какую скорость в начале этой секунды оно имеет?
}
\solutionspace{120pt}

\tasknumber{3}%
\task{%
    Карусель радиусом $4\,\text{м}$ равномерно совершает 10 оборотов в минуту.
    Определите
    \begin{itemize}
        \item период и частоту её обращения,
        \item скорость и ускорение крайних её точек.
    \end{itemize}
}
\solutionspace{80pt}

\tasknumber{4}%
\task{%
    Миша стоит на обрыве над рекой и методично и строго горизонтально кидает в неё камушки.
    За этим всем наблюдает экспериментатор Глюк, который уже выяснил, что камушки падают в реку спустя $1{,}2\,\text{с}$ после броска,
    а вот дальность полёта оценить сложнее: придётся лезть в воду.
    Выручите Глюка и определите:
    \begin{itemize}
        \item высоту обрыва (вместе с ростом Миши).
        \item дальность полёта камушков (по горизонтали) и их скорость при падении, приняв начальную скорость броска равной $v = 12\,\frac{\text{м}}{\text{с}}$.
    \end{itemize}
    Сопротивлением воздуха пренебречь.
}
\solutionspace{120pt}

\tasknumber{5}%
\task{%
    Четыре одинаковых брусков массой $3\,\text{кг}$ каждый лежат на гладком горизонтальном столе.
    Бруски пронумерованы от 1 до 4 и последовательно связаны между собой
    невесомыми нерастяжимыми нитями: 1 со 2, 2 с 3 (ну и с 1) и т.д.
    Экспериментатор Глюк прикладывает постоянную горизонтальную силу $90\,\text{Н}$ к бруску с наибольшим номером.
    С каким ускорением двигается система? Чему равна сила натяжения нити, связывающей бруски 1 и 2?
}
\solutionspace{120pt}

\tasknumber{6}%
\task{%
    Два бруска связаны лёгкой нерастяжимой нитью и перекинуты через неподвижный блок (см.
    рис.).
    Определите силу натяжения нити и ускорения брусков.
    Силами трения пренебречь, массы брусков
    равны $m_1 = 8\,\text{кг}$ и $m_2 = 6\,\text{кг}$.
    % $g = 10\,\frac{\text{м}}{\text{с}^{2}}$.

    \begin{tikzpicture}[x=1.5cm,y=1.5cm,thick]
        \draw
            (-0.4, 0) rectangle (-0.2, 1.2)
            (0.15, 0.5) rectangle (0.45, 1)
            (0, 2) circle [radius=0.3] -- ++(up:0.5)
            (-0.3, 1.2) -- ++(up:0.8)
            (0.3, 1) -- ++(up:1)
            (-0.7, 2.5) -- (0.7, 2.5)
            ;
        \draw[pattern={Lines[angle=51,distance=3pt]},pattern color=black,draw=none] (-0.7, 2.5) rectangle (0.7, 2.75);
        \node [left] (left) at (-0.4, 0.6) { $m_1$ };
        \node [right] (right) at (0.4, 0.75) { $m_2$ };
    \end{tikzpicture}
}
\solutionspace{80pt}

\tasknumber{7}%
\task{%
    Тело массой $2\,\text{кг}$ лежит на горизонтальной поверхности.
    Коэффициент трения между поверхностью и телом $0{,}15$.
    К телу приложена горизонтальная сила $2{,}5\,\text{Н}$.
    Определите силу трения, действующую на тело, и ускорение тела.
    % $g = 10\,\frac{\text{м}}{\text{с}^{2}}$.
}
\solutionspace{120pt}

\tasknumber{8}%
\task{%
    Определите плотность неизвестного вещества, если известно, что опускании тела из него
    в керосин оно будет плавать и на половину выступать над поверхностью жидкости.
}
\solutionspace{120pt}

\tasknumber{9}%
\task{%
    	Определите силу, действующую на левую опору однородного горизонтального стержня длиной $l = 7\,\text{м}$
    	и массой $M = 5\,\text{кг}$, к которому подвешен груз массой $m = 4\,\text{кг}$ на расстоянии $4\,\text{м}$ от правого конца (см.
    рис.).

        \begin{tikzpicture}[thick]
            \draw
                (-2, -0.1) rectangle (2, 0.1)
                (-0.5, -0.1) -- (-0.5, -1)
                (-0.7, -1) rectangle (-0.3, -1.3)
           		(-2, -0.1) -- +(0.15,-0.9) -- +(-0.15,-0.9) -- cycle
            	(2, -0.1) -- +(0.15,-0.9) -- +(-0.15,-0.9) -- cycle
            ;
            \draw[pattern={Lines[angle=51,distance=2pt]},pattern color=black,draw=none]
            	(-2.15, -1.15) rectangle +(0.3, 0.15)
            	(2.15, -1.15) rectangle +(-0.3, 0.15)
            ;
            \node [right] (m_small) at (-0.3, -1.15) { $m$ };
            \node [above] (M_big) at (0, 0.1) { $M$ };
        \end{tikzpicture}
}
\solutionspace{80pt}

\tasknumber{10}%
\task{%
    Тонкий однородный лом длиной $3\,\text{м}$ и массой $20\,\text{кг}$ лежит на горизонтальной поверхности.
    \begin{itemize}
        \item Какую минимальную силу надо приложить к одному из его концов, чтобы оторвать его от этой поверхности?
        \item Какую минимальную работу надо совершить, чтобы поставить его на землю в вертикальное положение?
    \end{itemize}
    % Примите $g = 10\,\frac{\text{м}}{\text{с}^{2}}$.
}
\answer{%
    $A = mg\frac l2 = 300\,\text{Дж}$
}
\solutionspace{120pt}

\tasknumber{11}%
\task{%
    Определите работу силы, которая обеспечит спуск тела массой $2\,\text{кг}$ на высоту $5\,\text{м}$ с постоянным ускорением $3\,\frac{\text{м}}{\text{c}^{2}}$.
    % Примите $g = 10\,\frac{\text{м}}{\text{с}^{2}}$.
}
\answer{%
    \begin{align*}
    &\text{Для подъёма:} A = Fh = (mg + ma) h = m(g+a)h, \\
    &\text{Для спуска:} A = -Fh = -(mg - ma) h = -m(g-a)h, \\
    &\text{В результате получаем:} -70\,\text{Дж}.
    \end{align*}
}
\solutionspace{60pt}

\tasknumber{12}%
\task{%
    Тело бросили вертикально вверх со скоростью $14\,\frac{\text{м}}{\text{c}}$.
    На какой высоте кинетическая энергия тела составит треть от потенциальной?
}
\solutionspace{100pt}

\tasknumber{13}%
\task{%
    Плотность воздуха при нормальных условиях равна $1{,}3\,\frac{\text{кг}}{\text{м}^{3}}$.
    Чему равна плотность воздуха
    при температуре $150\celsius$ и давлении $80\,\text{кПа}$?
}
\solutionspace{120pt}

\tasknumber{14}%
\task{%
    Небольшую цилиндрическую пробирку с воздухом погружают на некоторую глубину в глубокое пресное озеро,
    после чего воздух занимает в ней лишь третью часть от общего объема.
    Определите глубину, на которую погрузили пробирку.
    Температуру считать постоянной $T = 280\,\text{К}$, давлением паров воды пренебречь,
    атмосферное давление принять равным $p_{\text{aтм}} = 100\,\text{кПа}$.
}
\answer{%
    \begin{align*}
    T\text{— const} &\implies P_1V_1 = \nu RT = P_2V_2.
    \\
    V_2 = \frac 13 V_1 &\implies P_1V_1 = P_2 \cdot \frac 13V_1 \implies P_2 = 3P_1 = 3p_{\text{aтм}}.
    \\
    P_2 = p_{\text{aтм}} + \rho_{\text{в}} g h \implies h = \frac{P_2 - p_{\text{aтм}}}{\rho_{\text{в}} g} &= \frac{3p_{\text{aтм}} - p_{\text{aтм}}}{\rho_{\text{в}} g} = \frac{2 \cdot p_{\text{aтм}}}{\rho_{\text{в}} g} =  \\
     &= \frac{2 \cdot 100\,\text{кПа}}{1000\,\frac{\text{кг}}{\text{м}^{3}} \cdot  10\,\frac{\text{м}}{\text{с}^{2}}} \approx 20\,\text{м}.
    \end{align*}
}
\solutionspace{120pt}

\tasknumber{15}%
\task{%
    Газу сообщили некоторое количество теплоты,
    при этом половину его он потратил на совершение работы,
    одновременно увеличив свою внутреннюю энергию на $1200\,\text{Дж}$.
    Определите работу, совершённую газом.
}
\answer{%
    \begin{align*}
    Q &= A' + \Delta U, A' = \frac 12 Q \implies Q \cdot \cbr{1 - \frac 12} = \Delta U \implies Q = \frac{\Delta U}{1 - \frac 12} = \frac{ 1200\,\text{Дж} }{1 - \frac 12} \approx 2400\,\text{Дж}.
    \\
    A' &= \frac 12 Q
        = \frac 12 \cdot \frac{\Delta U}{1 - \frac 12}
        = \frac{\Delta U}{2 - 1}
        = \frac{ 1200\,\text{Дж} }{2 - 1} \approx 1200\,\text{Дж}.
    \end{align*}
}
\solutionspace{60pt}

\tasknumber{16}%
\task{%
    Два конденсатора ёмкостей $C_1 = 20\,\text{нФ}$ и $C_2 = 60\,\text{нФ}$ последовательно подключают
    к источнику напряжения $V = 450\,\text{В}$ (см.
    рис.).
    % Определите заряды каждого из конденсаторов.
    Определите заряд первого конденсатора.

    \begin{tikzpicture}[circuit ee IEC, semithick]
        \draw  (0, 0) to [capacitor={info={$C_1$}}] (1, 0)
                       to [capacitor={info={$C_2$}}] (2, 0)
        ;
        % \draw [-o] (0, 0) -- ++(-0.5, 0) node[left] {$-$};
        % \draw [-o] (2, 0) -- ++(0.5, 0) node[right] {$+$};
        \draw [-o] (0, 0) -- ++(-0.5, 0) node[left] {};
        \draw [-o] (2, 0) -- ++(0.5, 0) node[right] {};
    \end{tikzpicture}
}
\answer{%
    $
        Q_1
            = Q_2
            = CV
            = \frac{ V }{\frac1{C_1} + \frac1{C_2}}
            = \frac{C_1C_2V}{C_1 + C_2}
            = \frac{
                20\,\text{нФ} \cdot 60\,\text{нФ} \cdot 450\,\text{В}
            }{
                20\,\text{нФ} + 60\,\text{нФ}
            }
            = 6750{,}00\,\text{нКл}
    $
}
\solutionspace{120pt}

\tasknumber{17}%
\task{%
    В вакууме вдоль одной прямой расположены три отрицательных заряда так,
    что расстояние между соседними зарядами равно $r$.
    Сделайте рисунок,
    и определите силу, действующую на крайний заряд.
    Модули всех зарядов равны $q$ ($q > 0$).
}
\solutionspace{80pt}

\tasknumber{18}%
\task{%
    Юлия проводит эксперименты c 2 кусками одинаковой медной проволки, причём второй кусок в десять раз длиннее первого.
    В одном из экспериментов Юлия подаёт на первый кусок проволки напряжение в семь раз раз больше, чем на второй.
    Определите отношения в двух проволках в этом эксперименте (второй к первой):
    \begin{itemize}
        \item отношение сил тока,
        \item отношение выделяющихся мощностей.
    \end{itemize}
}
\answer{%
    $\eli_2 / \eli_1 = \frac1{70}, \P_2 / \P_1 = \frac1{70}, $
}

\variantsplitter

\addpersonalvariant{Анна Кузьмичёва}

\tasknumber{1}%
\task{%
    Саша стартует на велосипеде и в течение $t = 3\,\text{c}$ двигается с постоянным ускорением $2\,\frac{\text{м}}{\text{с}^{2}}$.
    Определите
    \begin{itemize}
        \item какую скорость при этом удастся достичь,
        \item какой путь за это время будет пройден,
        \item среднюю скорость за всё время движения, если после начального ускорения продолжить движение равномерно ещё в течение времени $3t$
    \end{itemize}
}
\solutionspace{120pt}

\tasknumber{2}%
\task{%
    Какой путь тело пройдёт за шестую секунду после начала свободного падения?
    Какую скорость в начале этой секунды оно имеет?
}
\solutionspace{120pt}

\tasknumber{3}%
\task{%
    Карусель диаметром $5\,\text{м}$ равномерно совершает 5 оборотов в минуту.
    Определите
    \begin{itemize}
        \item период и частоту её обращения,
        \item скорость и ускорение крайних её точек.
    \end{itemize}
}
\solutionspace{80pt}

\tasknumber{4}%
\task{%
    Даша стоит на обрыве над рекой и методично и строго горизонтально кидает в неё камушки.
    За этим всем наблюдает экспериментатор Глюк, который уже выяснил, что камушки падают в реку спустя $1{,}2\,\text{с}$ после броска,
    а вот дальность полёта оценить сложнее: придётся лезть в воду.
    Выручите Глюка и определите:
    \begin{itemize}
        \item высоту обрыва (вместе с ростом Даши).
        \item дальность полёта камушков (по горизонтали) и их скорость при падении, приняв начальную скорость броска равной $v = 18\,\frac{\text{м}}{\text{с}}$.
    \end{itemize}
    Сопротивлением воздуха пренебречь.
}
\solutionspace{120pt}

\tasknumber{5}%
\task{%
    Пять одинаковых брусков массой $2\,\text{кг}$ каждый лежат на гладком горизонтальном столе.
    Бруски пронумерованы от 1 до 5 и последовательно связаны между собой
    невесомыми нерастяжимыми нитями: 1 со 2, 2 с 3 (ну и с 1) и т.д.
    Экспериментатор Глюк прикладывает постоянную горизонтальную силу $90\,\text{Н}$ к бруску с наименьшим номером.
    С каким ускорением двигается система? Чему равна сила натяжения нити, связывающей бруски 1 и 2?
}
\solutionspace{120pt}

\tasknumber{6}%
\task{%
    Два бруска связаны лёгкой нерастяжимой нитью и перекинуты через неподвижный блок (см.
    рис.).
    Определите силу натяжения нити и ускорения брусков.
    Силами трения пренебречь, массы брусков
    равны $m_1 = 5\,\text{кг}$ и $m_2 = 4\,\text{кг}$.
    % $g = 10\,\frac{\text{м}}{\text{с}^{2}}$.

    \begin{tikzpicture}[x=1.5cm,y=1.5cm,thick]
        \draw
            (-0.4, 0) rectangle (-0.2, 1.2)
            (0.15, 0.5) rectangle (0.45, 1)
            (0, 2) circle [radius=0.3] -- ++(up:0.5)
            (-0.3, 1.2) -- ++(up:0.8)
            (0.3, 1) -- ++(up:1)
            (-0.7, 2.5) -- (0.7, 2.5)
            ;
        \draw[pattern={Lines[angle=51,distance=3pt]},pattern color=black,draw=none] (-0.7, 2.5) rectangle (0.7, 2.75);
        \node [left] (left) at (-0.4, 0.6) { $m_1$ };
        \node [right] (right) at (0.4, 0.75) { $m_2$ };
    \end{tikzpicture}
}
\solutionspace{80pt}

\tasknumber{7}%
\task{%
    Тело массой $2\,\text{кг}$ лежит на горизонтальной поверхности.
    Коэффициент трения между поверхностью и телом $0{,}25$.
    К телу приложена горизонтальная сила $3{,}5\,\text{Н}$.
    Определите силу трения, действующую на тело, и ускорение тела.
    % $g = 10\,\frac{\text{м}}{\text{с}^{2}}$.
}
\solutionspace{120pt}

\tasknumber{8}%
\task{%
    Определите плотность неизвестного вещества, если известно, что опускании тела из него
    в керосин оно будет плавать и на четверть выступать над поверхностью жидкости.
}
\solutionspace{120pt}

\tasknumber{9}%
\task{%
    	Определите силу, действующую на правую опору однородного горизонтального стержня длиной $l = 5\,\text{м}$
    	и массой $M = 5\,\text{кг}$, к которому подвешен груз массой $m = 3\,\text{кг}$ на расстоянии $4\,\text{м}$ от правого конца (см.
    рис.).

        \begin{tikzpicture}[thick]
            \draw
                (-2, -0.1) rectangle (2, 0.1)
                (-0.5, -0.1) -- (-0.5, -1)
                (-0.7, -1) rectangle (-0.3, -1.3)
           		(-2, -0.1) -- +(0.15,-0.9) -- +(-0.15,-0.9) -- cycle
            	(2, -0.1) -- +(0.15,-0.9) -- +(-0.15,-0.9) -- cycle
            ;
            \draw[pattern={Lines[angle=51,distance=2pt]},pattern color=black,draw=none]
            	(-2.15, -1.15) rectangle +(0.3, 0.15)
            	(2.15, -1.15) rectangle +(-0.3, 0.15)
            ;
            \node [right] (m_small) at (-0.3, -1.15) { $m$ };
            \node [above] (M_big) at (0, 0.1) { $M$ };
        \end{tikzpicture}
}
\solutionspace{80pt}

\tasknumber{10}%
\task{%
    Тонкий однородный кусок арматуры длиной $2\,\text{м}$ и массой $10\,\text{кг}$ лежит на горизонтальной поверхности.
    \begin{itemize}
        \item Какую минимальную силу надо приложить к одному из его концов, чтобы оторвать его от этой поверхности?
        \item Какую минимальную работу надо совершить, чтобы поставить его на землю в вертикальное положение?
    \end{itemize}
    % Примите $g = 10\,\frac{\text{м}}{\text{с}^{2}}$.
}
\answer{%
    $A = mg\frac l2 = 100\,\text{Дж}$
}
\solutionspace{120pt}

\tasknumber{11}%
\task{%
    Определите работу силы, которая обеспечит подъём тела массой $3\,\text{кг}$ на высоту $10\,\text{м}$ с постоянным ускорением $3\,\frac{\text{м}}{\text{c}^{2}}$.
    % Примите $g = 10\,\frac{\text{м}}{\text{с}^{2}}$.
}
\answer{%
    \begin{align*}
    &\text{Для подъёма:} A = Fh = (mg + ma) h = m(g+a)h, \\
    &\text{Для спуска:} A = -Fh = -(mg - ma) h = -m(g-a)h, \\
    &\text{В результате получаем:} 390\,\text{Дж}.
    \end{align*}
}
\solutionspace{60pt}

\tasknumber{12}%
\task{%
    Тело бросили вертикально вверх со скоростью $20\,\frac{\text{м}}{\text{c}}$.
    На какой высоте кинетическая энергия тела составит половину от потенциальной?
}
\solutionspace{100pt}

\tasknumber{13}%
\task{%
    Плотность воздуха при нормальных условиях равна $1{,}3\,\frac{\text{кг}}{\text{м}^{3}}$.
    Чему равна плотность воздуха
    при температуре $100\celsius$ и давлении $120\,\text{кПа}$?
}
\solutionspace{120pt}

\tasknumber{14}%
\task{%
    Небольшую цилиндрическую пробирку с воздухом погружают на некоторую глубину в глубокое пресное озеро,
    после чего воздух занимает в ней лишь шестую часть от общего объема.
    Определите глубину, на которую погрузили пробирку.
    Температуру считать постоянной $T = 286\,\text{К}$, давлением паров воды пренебречь,
    атмосферное давление принять равным $p_{\text{aтм}} = 100\,\text{кПа}$.
}
\answer{%
    \begin{align*}
    T\text{— const} &\implies P_1V_1 = \nu RT = P_2V_2.
    \\
    V_2 = \frac 16 V_1 &\implies P_1V_1 = P_2 \cdot \frac 16V_1 \implies P_2 = 6P_1 = 6p_{\text{aтм}}.
    \\
    P_2 = p_{\text{aтм}} + \rho_{\text{в}} g h \implies h = \frac{P_2 - p_{\text{aтм}}}{\rho_{\text{в}} g} &= \frac{6p_{\text{aтм}} - p_{\text{aтм}}}{\rho_{\text{в}} g} = \frac{5 \cdot p_{\text{aтм}}}{\rho_{\text{в}} g} =  \\
     &= \frac{5 \cdot 100\,\text{кПа}}{1000\,\frac{\text{кг}}{\text{м}^{3}} \cdot  10\,\frac{\text{м}}{\text{с}^{2}}} \approx 50\,\text{м}.
    \end{align*}
}
\solutionspace{120pt}

\tasknumber{15}%
\task{%
    Газу сообщили некоторое количество теплоты,
    при этом половину его он потратил на совершение работы,
    одновременно увеличив свою внутреннюю энергию на $1500\,\text{Дж}$.
    Определите количество теплоты, сообщённое газу.
}
\answer{%
    \begin{align*}
    Q &= A' + \Delta U, A' = \frac 12 Q \implies Q \cdot \cbr{1 - \frac 12} = \Delta U \implies Q = \frac{\Delta U}{1 - \frac 12} = \frac{ 1500\,\text{Дж} }{1 - \frac 12} \approx 3000\,\text{Дж}.
    \\
    A' &= \frac 12 Q
        = \frac 12 \cdot \frac{\Delta U}{1 - \frac 12}
        = \frac{\Delta U}{2 - 1}
        = \frac{ 1500\,\text{Дж} }{2 - 1} \approx 1500\,\text{Дж}.
    \end{align*}
}
\solutionspace{60pt}

\tasknumber{16}%
\task{%
    Два конденсатора ёмкостей $C_1 = 30\,\text{нФ}$ и $C_2 = 20\,\text{нФ}$ последовательно подключают
    к источнику напряжения $U = 300\,\text{В}$ (см.
    рис.).
    % Определите заряды каждого из конденсаторов.
    Определите заряд второго конденсатора.

    \begin{tikzpicture}[circuit ee IEC, semithick]
        \draw  (0, 0) to [capacitor={info={$C_1$}}] (1, 0)
                       to [capacitor={info={$C_2$}}] (2, 0)
        ;
        % \draw [-o] (0, 0) -- ++(-0.5, 0) node[left] {$-$};
        % \draw [-o] (2, 0) -- ++(0.5, 0) node[right] {$+$};
        \draw [-o] (0, 0) -- ++(-0.5, 0) node[left] {};
        \draw [-o] (2, 0) -- ++(0.5, 0) node[right] {};
    \end{tikzpicture}
}
\answer{%
    $
        Q_1
            = Q_2
            = CU
            = \frac{ U }{\frac1{C_1} + \frac1{C_2}}
            = \frac{C_1C_2U}{C_1 + C_2}
            = \frac{
                30\,\text{нФ} \cdot 20\,\text{нФ} \cdot 300\,\text{В}
            }{
                30\,\text{нФ} + 20\,\text{нФ}
            }
            = 3600{,}00\,\text{нКл}
    $
}
\solutionspace{120pt}

\tasknumber{17}%
\task{%
    В вакууме вдоль одной прямой расположены четыре положительных заряда так,
    что расстояние между соседними зарядами равно $a$.
    Сделайте рисунок,
    и определите силу, действующую на крайний заряд.
    Модули всех зарядов равны $Q$ ($Q > 0$).
}
\solutionspace{80pt}

\tasknumber{18}%
\task{%
    Юлия проводит эксперименты c 2 кусками одинаковой медной проволки, причём второй кусок в два раза длиннее первого.
    В одном из экспериментов Юлия подаёт на первый кусок проволки напряжение в девять раз раз больше, чем на второй.
    Определите отношения в двух проволках в этом эксперименте (второй к первой):
    \begin{itemize}
        \item отношение сил тока,
        \item отношение выделяющихся мощностей.
    \end{itemize}
}
\answer{%
    $\eli_2 / \eli_1 = \frac1{18}, \P_2 / \P_1 = \frac1{18}, $
}

\variantsplitter

\addpersonalvariant{Алёна Куприянова}

\tasknumber{1}%
\task{%
    Валя стартует на мотоцикле и в течение $t = 10\,\text{c}$ двигается с постоянным ускорением $1{,}5\,\frac{\text{м}}{\text{с}^{2}}$.
    Определите
    \begin{itemize}
        \item какую скорость при этом удастся достичь,
        \item какой путь за это время будет пройден,
        \item среднюю скорость за всё время движения, если после начального ускорения продолжить движение равномерно ещё в течение времени $2t$
    \end{itemize}
}
\solutionspace{120pt}

\tasknumber{2}%
\task{%
    Какой путь тело пройдёт за вторую секунду после начала свободного падения?
    Какую скорость в конце этой секунды оно имеет?
}
\solutionspace{120pt}

\tasknumber{3}%
\task{%
    Карусель диаметром $3\,\text{м}$ равномерно совершает 5 оборотов в минуту.
    Определите
    \begin{itemize}
        \item период и частоту её обращения,
        \item скорость и ускорение крайних её точек.
    \end{itemize}
}
\solutionspace{80pt}

\tasknumber{4}%
\task{%
    Даша стоит на обрыве над рекой и методично и строго горизонтально кидает в неё камушки.
    За этим всем наблюдает экспериментатор Глюк, который уже выяснил, что камушки падают в реку спустя $1{,}2\,\text{с}$ после броска,
    а вот дальность полёта оценить сложнее: придётся лезть в воду.
    Выручите Глюка и определите:
    \begin{itemize}
        \item высоту обрыва (вместе с ростом Даши).
        \item дальность полёта камушков (по горизонтали) и их скорость при падении, приняв начальную скорость броска равной $v = 12\,\frac{\text{м}}{\text{с}}$.
    \end{itemize}
    Сопротивлением воздуха пренебречь.
}
\solutionspace{120pt}

\tasknumber{5}%
\task{%
    Четыре одинаковых брусков массой $2\,\text{кг}$ каждый лежат на гладком горизонтальном столе.
    Бруски пронумерованы от 1 до 4 и последовательно связаны между собой
    невесомыми нерастяжимыми нитями: 1 со 2, 2 с 3 (ну и с 1) и т.д.
    Экспериментатор Глюк прикладывает постоянную горизонтальную силу $90\,\text{Н}$ к бруску с наибольшим номером.
    С каким ускорением двигается система? Чему равна сила натяжения нити, связывающей бруски 2 и 3?
}
\solutionspace{120pt}

\tasknumber{6}%
\task{%
    Два бруска связаны лёгкой нерастяжимой нитью и перекинуты через неподвижный блок (см.
    рис.).
    Определите силу натяжения нити и ускорения брусков.
    Силами трения пренебречь, массы брусков
    равны $m_1 = 11\,\text{кг}$ и $m_2 = 6\,\text{кг}$.
    % $g = 10\,\frac{\text{м}}{\text{с}^{2}}$.

    \begin{tikzpicture}[x=1.5cm,y=1.5cm,thick]
        \draw
            (-0.4, 0) rectangle (-0.2, 1.2)
            (0.15, 0.5) rectangle (0.45, 1)
            (0, 2) circle [radius=0.3] -- ++(up:0.5)
            (-0.3, 1.2) -- ++(up:0.8)
            (0.3, 1) -- ++(up:1)
            (-0.7, 2.5) -- (0.7, 2.5)
            ;
        \draw[pattern={Lines[angle=51,distance=3pt]},pattern color=black,draw=none] (-0.7, 2.5) rectangle (0.7, 2.75);
        \node [left] (left) at (-0.4, 0.6) { $m_1$ };
        \node [right] (right) at (0.4, 0.75) { $m_2$ };
    \end{tikzpicture}
}
\solutionspace{80pt}

\tasknumber{7}%
\task{%
    Тело массой $2{,}7\,\text{кг}$ лежит на горизонтальной поверхности.
    Коэффициент трения между поверхностью и телом $0{,}15$.
    К телу приложена горизонтальная сила $4{,}5\,\text{Н}$.
    Определите силу трения, действующую на тело, и ускорение тела.
    % $g = 10\,\frac{\text{м}}{\text{с}^{2}}$.
}
\solutionspace{120pt}

\tasknumber{8}%
\task{%
    Определите плотность неизвестного вещества, если известно, что опускании тела из него
    в керосин оно будет плавать и на половину выступать над поверхностью жидкости.
}
\solutionspace{120pt}

\tasknumber{9}%
\task{%
    	Определите силу, действующую на правую опору однородного горизонтального стержня длиной $l = 9\,\text{м}$
    	и массой $M = 1\,\text{кг}$, к которому подвешен груз массой $m = 2\,\text{кг}$ на расстоянии $4\,\text{м}$ от правого конца (см.
    рис.).

        \begin{tikzpicture}[thick]
            \draw
                (-2, -0.1) rectangle (2, 0.1)
                (-0.5, -0.1) -- (-0.5, -1)
                (-0.7, -1) rectangle (-0.3, -1.3)
           		(-2, -0.1) -- +(0.15,-0.9) -- +(-0.15,-0.9) -- cycle
            	(2, -0.1) -- +(0.15,-0.9) -- +(-0.15,-0.9) -- cycle
            ;
            \draw[pattern={Lines[angle=51,distance=2pt]},pattern color=black,draw=none]
            	(-2.15, -1.15) rectangle +(0.3, 0.15)
            	(2.15, -1.15) rectangle +(-0.3, 0.15)
            ;
            \node [right] (m_small) at (-0.3, -1.15) { $m$ };
            \node [above] (M_big) at (0, 0.1) { $M$ };
        \end{tikzpicture}
}
\solutionspace{80pt}

\tasknumber{10}%
\task{%
    Тонкий однородный кусок арматуры длиной $3\,\text{м}$ и массой $30\,\text{кг}$ лежит на горизонтальной поверхности.
    \begin{itemize}
        \item Какую минимальную силу надо приложить к одному из его концов, чтобы оторвать его от этой поверхности?
        \item Какую минимальную работу надо совершить, чтобы поставить его на землю в вертикальное положение?
    \end{itemize}
    % Примите $g = 10\,\frac{\text{м}}{\text{с}^{2}}$.
}
\answer{%
    $A = mg\frac l2 = 450\,\text{Дж}$
}
\solutionspace{120pt}

\tasknumber{11}%
\task{%
    Определите работу силы, которая обеспечит спуск тела массой $3\,\text{кг}$ на высоту $5\,\text{м}$ с постоянным ускорением $2\,\frac{\text{м}}{\text{c}^{2}}$.
    % Примите $g = 10\,\frac{\text{м}}{\text{с}^{2}}$.
}
\answer{%
    \begin{align*}
    &\text{Для подъёма:} A = Fh = (mg + ma) h = m(g+a)h, \\
    &\text{Для спуска:} A = -Fh = -(mg - ma) h = -m(g-a)h, \\
    &\text{В результате получаем:} -120\,\text{Дж}.
    \end{align*}
}
\solutionspace{60pt}

\tasknumber{12}%
\task{%
    Тело бросили вертикально вверх со скоростью $14\,\frac{\text{м}}{\text{c}}$.
    На какой высоте кинетическая энергия тела составит треть от потенциальной?
}
\solutionspace{100pt}

\tasknumber{13}%
\task{%
    Плотность воздуха при нормальных условиях равна $1{,}3\,\frac{\text{кг}}{\text{м}^{3}}$.
    Чему равна плотность воздуха
    при температуре $100\celsius$ и давлении $150\,\text{кПа}$?
}
\solutionspace{120pt}

\tasknumber{14}%
\task{%
    Небольшую цилиндрическую пробирку с воздухом погружают на некоторую глубину в глубокое пресное озеро,
    после чего воздух занимает в ней лишь третью часть от общего объема.
    Определите глубину, на которую погрузили пробирку.
    Температуру считать постоянной $T = 289\,\text{К}$, давлением паров воды пренебречь,
    атмосферное давление принять равным $p_{\text{aтм}} = 100\,\text{кПа}$.
}
\answer{%
    \begin{align*}
    T\text{— const} &\implies P_1V_1 = \nu RT = P_2V_2.
    \\
    V_2 = \frac 13 V_1 &\implies P_1V_1 = P_2 \cdot \frac 13V_1 \implies P_2 = 3P_1 = 3p_{\text{aтм}}.
    \\
    P_2 = p_{\text{aтм}} + \rho_{\text{в}} g h \implies h = \frac{P_2 - p_{\text{aтм}}}{\rho_{\text{в}} g} &= \frac{3p_{\text{aтм}} - p_{\text{aтм}}}{\rho_{\text{в}} g} = \frac{2 \cdot p_{\text{aтм}}}{\rho_{\text{в}} g} =  \\
     &= \frac{2 \cdot 100\,\text{кПа}}{1000\,\frac{\text{кг}}{\text{м}^{3}} \cdot  10\,\frac{\text{м}}{\text{с}^{2}}} \approx 20\,\text{м}.
    \end{align*}
}
\solutionspace{120pt}

\tasknumber{15}%
\task{%
    Газу сообщили некоторое количество теплоты,
    при этом четверть его он потратил на совершение работы,
    одновременно увеличив свою внутреннюю энергию на $1200\,\text{Дж}$.
    Определите работу, совершённую газом.
}
\answer{%
    \begin{align*}
    Q &= A' + \Delta U, A' = \frac 14 Q \implies Q \cdot \cbr{1 - \frac 14} = \Delta U \implies Q = \frac{\Delta U}{1 - \frac 14} = \frac{ 1200\,\text{Дж} }{1 - \frac 14} \approx 1600\,\text{Дж}.
    \\
    A' &= \frac 14 Q
        = \frac 14 \cdot \frac{\Delta U}{1 - \frac 14}
        = \frac{\Delta U}{4 - 1}
        = \frac{ 1200\,\text{Дж} }{4 - 1} \approx 400\,\text{Дж}.
    \end{align*}
}
\solutionspace{60pt}

\tasknumber{16}%
\task{%
    Два конденсатора ёмкостей $C_1 = 20\,\text{нФ}$ и $C_2 = 40\,\text{нФ}$ последовательно подключают
    к источнику напряжения $U = 200\,\text{В}$ (см.
    рис.).
    % Определите заряды каждого из конденсаторов.
    Определите заряд второго конденсатора.

    \begin{tikzpicture}[circuit ee IEC, semithick]
        \draw  (0, 0) to [capacitor={info={$C_1$}}] (1, 0)
                       to [capacitor={info={$C_2$}}] (2, 0)
        ;
        % \draw [-o] (0, 0) -- ++(-0.5, 0) node[left] {$-$};
        % \draw [-o] (2, 0) -- ++(0.5, 0) node[right] {$+$};
        \draw [-o] (0, 0) -- ++(-0.5, 0) node[left] {};
        \draw [-o] (2, 0) -- ++(0.5, 0) node[right] {};
    \end{tikzpicture}
}
\answer{%
    $
        Q_1
            = Q_2
            = CU
            = \frac{ U }{\frac1{C_1} + \frac1{C_2}}
            = \frac{C_1C_2U}{C_1 + C_2}
            = \frac{
                20\,\text{нФ} \cdot 40\,\text{нФ} \cdot 200\,\text{В}
            }{
                20\,\text{нФ} + 40\,\text{нФ}
            }
            = 2666{,}67\,\text{нКл}
    $
}
\solutionspace{120pt}

\tasknumber{17}%
\task{%
    В вакууме вдоль одной прямой расположены четыре положительных заряда так,
    что расстояние между соседними зарядами равно $a$.
    Сделайте рисунок,
    и определите силу, действующую на крайний заряд.
    Модули всех зарядов равны $Q$ ($Q > 0$).
}
\solutionspace{80pt}

\tasknumber{18}%
\task{%
    Юлия проводит эксперименты c 2 кусками одинаковой медной проволки, причём второй кусок в девять раз длиннее первого.
    В одном из экспериментов Юлия подаёт на первый кусок проволки напряжение в два раза раз больше, чем на второй.
    Определите отношения в двух проволках в этом эксперименте (второй к первой):
    \begin{itemize}
        \item отношение сил тока,
        \item отношение выделяющихся мощностей.
    \end{itemize}
}
\answer{%
    $\eli_2 / \eli_1 = \frac1{18}, \P_2 / \P_1 = \frac1{18}, $
}

\variantsplitter

\addpersonalvariant{Ярослав Лавровский}

\tasknumber{1}%
\task{%
    Валя стартует на мотоцикле и в течение $t = 10\,\text{c}$ двигается с постоянным ускорением $2\,\frac{\text{м}}{\text{с}^{2}}$.
    Определите
    \begin{itemize}
        \item какую скорость при этом удастся достичь,
        \item какой путь за это время будет пройден,
        \item среднюю скорость за всё время движения, если после начального ускорения продолжить движение равномерно ещё в течение времени $3t$
    \end{itemize}
}
\solutionspace{120pt}

\tasknumber{2}%
\task{%
    Какой путь тело пройдёт за третью секунду после начала свободного падения?
    Какую скорость в начале этой секунды оно имеет?
}
\solutionspace{120pt}

\tasknumber{3}%
\task{%
    Карусель диаметром $5\,\text{м}$ равномерно совершает 10 оборотов в минуту.
    Определите
    \begin{itemize}
        \item период и частоту её обращения,
        \item скорость и ускорение крайних её точек.
    \end{itemize}
}
\solutionspace{80pt}

\tasknumber{4}%
\task{%
    Миша стоит на обрыве над рекой и методично и строго горизонтально кидает в неё камушки.
    За этим всем наблюдает экспериментатор Глюк, который уже выяснил, что камушки падают в реку спустя $1{,}7\,\text{с}$ после броска,
    а вот дальность полёта оценить сложнее: придётся лезть в воду.
    Выручите Глюка и определите:
    \begin{itemize}
        \item высоту обрыва (вместе с ростом Миши).
        \item дальность полёта камушков (по горизонтали) и их скорость при падении, приняв начальную скорость броска равной $v = 17\,\frac{\text{м}}{\text{с}}$.
    \end{itemize}
    Сопротивлением воздуха пренебречь.
}
\solutionspace{120pt}

\tasknumber{5}%
\task{%
    Пять одинаковых брусков массой $3\,\text{кг}$ каждый лежат на гладком горизонтальном столе.
    Бруски пронумерованы от 1 до 5 и последовательно связаны между собой
    невесомыми нерастяжимыми нитями: 1 со 2, 2 с 3 (ну и с 1) и т.д.
    Экспериментатор Глюк прикладывает постоянную горизонтальную силу $90\,\text{Н}$ к бруску с наименьшим номером.
    С каким ускорением двигается система? Чему равна сила натяжения нити, связывающей бруски 2 и 3?
}
\solutionspace{120pt}

\tasknumber{6}%
\task{%
    Два бруска связаны лёгкой нерастяжимой нитью и перекинуты через неподвижный блок (см.
    рис.).
    Определите силу натяжения нити и ускорения брусков.
    Силами трения пренебречь, массы брусков
    равны $m_1 = 5\,\text{кг}$ и $m_2 = 6\,\text{кг}$.
    % $g = 10\,\frac{\text{м}}{\text{с}^{2}}$.

    \begin{tikzpicture}[x=1.5cm,y=1.5cm,thick]
        \draw
            (-0.4, 0) rectangle (-0.2, 1.2)
            (0.15, 0.5) rectangle (0.45, 1)
            (0, 2) circle [radius=0.3] -- ++(up:0.5)
            (-0.3, 1.2) -- ++(up:0.8)
            (0.3, 1) -- ++(up:1)
            (-0.7, 2.5) -- (0.7, 2.5)
            ;
        \draw[pattern={Lines[angle=51,distance=3pt]},pattern color=black,draw=none] (-0.7, 2.5) rectangle (0.7, 2.75);
        \node [left] (left) at (-0.4, 0.6) { $m_1$ };
        \node [right] (right) at (0.4, 0.75) { $m_2$ };
    \end{tikzpicture}
}
\solutionspace{80pt}

\tasknumber{7}%
\task{%
    Тело массой $1{,}4\,\text{кг}$ лежит на горизонтальной поверхности.
    Коэффициент трения между поверхностью и телом $0{,}15$.
    К телу приложена горизонтальная сила $5{,}5\,\text{Н}$.
    Определите силу трения, действующую на тело, и ускорение тела.
    % $g = 10\,\frac{\text{м}}{\text{с}^{2}}$.
}
\solutionspace{120pt}

\tasknumber{8}%
\task{%
    Определите плотность неизвестного вещества, если известно, что опускании тела из него
    в керосин оно будет плавать и на половину выступать над поверхностью жидкости.
}
\solutionspace{120pt}

\tasknumber{9}%
\task{%
    	Определите силу, действующую на левую опору однородного горизонтального стержня длиной $l = 5\,\text{м}$
    	и массой $M = 1\,\text{кг}$, к которому подвешен груз массой $m = 3\,\text{кг}$ на расстоянии $2\,\text{м}$ от правого конца (см.
    рис.).

        \begin{tikzpicture}[thick]
            \draw
                (-2, -0.1) rectangle (2, 0.1)
                (-0.5, -0.1) -- (-0.5, -1)
                (-0.7, -1) rectangle (-0.3, -1.3)
           		(-2, -0.1) -- +(0.15,-0.9) -- +(-0.15,-0.9) -- cycle
            	(2, -0.1) -- +(0.15,-0.9) -- +(-0.15,-0.9) -- cycle
            ;
            \draw[pattern={Lines[angle=51,distance=2pt]},pattern color=black,draw=none]
            	(-2.15, -1.15) rectangle +(0.3, 0.15)
            	(2.15, -1.15) rectangle +(-0.3, 0.15)
            ;
            \node [right] (m_small) at (-0.3, -1.15) { $m$ };
            \node [above] (M_big) at (0, 0.1) { $M$ };
        \end{tikzpicture}
}
\solutionspace{80pt}

\tasknumber{10}%
\task{%
    Тонкий однородный лом длиной $3\,\text{м}$ и массой $30\,\text{кг}$ лежит на горизонтальной поверхности.
    \begin{itemize}
        \item Какую минимальную силу надо приложить к одному из его концов, чтобы оторвать его от этой поверхности?
        \item Какую минимальную работу надо совершить, чтобы поставить его на землю в вертикальное положение?
    \end{itemize}
    % Примите $g = 10\,\frac{\text{м}}{\text{с}^{2}}$.
}
\answer{%
    $A = mg\frac l2 = 450\,\text{Дж}$
}
\solutionspace{120pt}

\tasknumber{11}%
\task{%
    Определите работу силы, которая обеспечит подъём тела массой $5\,\text{кг}$ на высоту $10\,\text{м}$ с постоянным ускорением $3\,\frac{\text{м}}{\text{c}^{2}}$.
    % Примите $g = 10\,\frac{\text{м}}{\text{с}^{2}}$.
}
\answer{%
    \begin{align*}
    &\text{Для подъёма:} A = Fh = (mg + ma) h = m(g+a)h, \\
    &\text{Для спуска:} A = -Fh = -(mg - ma) h = -m(g-a)h, \\
    &\text{В результате получаем:} 650\,\text{Дж}.
    \end{align*}
}
\solutionspace{60pt}

\tasknumber{12}%
\task{%
    Тело бросили вертикально вверх со скоростью $20\,\frac{\text{м}}{\text{c}}$.
    На какой высоте кинетическая энергия тела составит половину от потенциальной?
}
\solutionspace{100pt}

\tasknumber{13}%
\task{%
    Плотность воздуха при нормальных условиях равна $1{,}3\,\frac{\text{кг}}{\text{м}^{3}}$.
    Чему равна плотность воздуха
    при температуре $150\celsius$ и давлении $80\,\text{кПа}$?
}
\solutionspace{120pt}

\tasknumber{14}%
\task{%
    Небольшую цилиндрическую пробирку с воздухом погружают на некоторую глубину в глубокое пресное озеро,
    после чего воздух занимает в ней лишь пятую часть от общего объема.
    Определите глубину, на которую погрузили пробирку.
    Температуру считать постоянной $T = 279\,\text{К}$, давлением паров воды пренебречь,
    атмосферное давление принять равным $p_{\text{aтм}} = 100\,\text{кПа}$.
}
\answer{%
    \begin{align*}
    T\text{— const} &\implies P_1V_1 = \nu RT = P_2V_2.
    \\
    V_2 = \frac 15 V_1 &\implies P_1V_1 = P_2 \cdot \frac 15V_1 \implies P_2 = 5P_1 = 5p_{\text{aтм}}.
    \\
    P_2 = p_{\text{aтм}} + \rho_{\text{в}} g h \implies h = \frac{P_2 - p_{\text{aтм}}}{\rho_{\text{в}} g} &= \frac{5p_{\text{aтм}} - p_{\text{aтм}}}{\rho_{\text{в}} g} = \frac{4 \cdot p_{\text{aтм}}}{\rho_{\text{в}} g} =  \\
     &= \frac{4 \cdot 100\,\text{кПа}}{1000\,\frac{\text{кг}}{\text{м}^{3}} \cdot  10\,\frac{\text{м}}{\text{с}^{2}}} \approx 40\,\text{м}.
    \end{align*}
}
\solutionspace{120pt}

\tasknumber{15}%
\task{%
    Газу сообщили некоторое количество теплоты,
    при этом четверть его он потратил на совершение работы,
    одновременно увеличив свою внутреннюю энергию на $1200\,\text{Дж}$.
    Определите работу, совершённую газом.
}
\answer{%
    \begin{align*}
    Q &= A' + \Delta U, A' = \frac 14 Q \implies Q \cdot \cbr{1 - \frac 14} = \Delta U \implies Q = \frac{\Delta U}{1 - \frac 14} = \frac{ 1200\,\text{Дж} }{1 - \frac 14} \approx 1600\,\text{Дж}.
    \\
    A' &= \frac 14 Q
        = \frac 14 \cdot \frac{\Delta U}{1 - \frac 14}
        = \frac{\Delta U}{4 - 1}
        = \frac{ 1200\,\text{Дж} }{4 - 1} \approx 400\,\text{Дж}.
    \end{align*}
}
\solutionspace{60pt}

\tasknumber{16}%
\task{%
    Два конденсатора ёмкостей $C_1 = 40\,\text{нФ}$ и $C_2 = 60\,\text{нФ}$ последовательно подключают
    к источнику напряжения $U = 300\,\text{В}$ (см.
    рис.).
    % Определите заряды каждого из конденсаторов.
    Определите заряд второго конденсатора.

    \begin{tikzpicture}[circuit ee IEC, semithick]
        \draw  (0, 0) to [capacitor={info={$C_1$}}] (1, 0)
                       to [capacitor={info={$C_2$}}] (2, 0)
        ;
        % \draw [-o] (0, 0) -- ++(-0.5, 0) node[left] {$-$};
        % \draw [-o] (2, 0) -- ++(0.5, 0) node[right] {$+$};
        \draw [-o] (0, 0) -- ++(-0.5, 0) node[left] {};
        \draw [-o] (2, 0) -- ++(0.5, 0) node[right] {};
    \end{tikzpicture}
}
\answer{%
    $
        Q_1
            = Q_2
            = CU
            = \frac{ U }{\frac1{C_1} + \frac1{C_2}}
            = \frac{C_1C_2U}{C_1 + C_2}
            = \frac{
                40\,\text{нФ} \cdot 60\,\text{нФ} \cdot 300\,\text{В}
            }{
                40\,\text{нФ} + 60\,\text{нФ}
            }
            = 7200{,}00\,\text{нКл}
    $
}
\solutionspace{120pt}

\tasknumber{17}%
\task{%
    В вакууме вдоль одной прямой расположены три отрицательных заряда так,
    что расстояние между соседними зарядами равно $d$.
    Сделайте рисунок,
    и определите силу, действующую на крайний заряд.
    Модули всех зарядов равны $Q$ ($Q > 0$).
}
\solutionspace{80pt}

\tasknumber{18}%
\task{%
    Юлия проводит эксперименты c 2 кусками одинаковой стальной проволки, причём второй кусок в два раза длиннее первого.
    В одном из экспериментов Юлия подаёт на первый кусок проволки напряжение в пять раз раз больше, чем на второй.
    Определите отношения в двух проволках в этом эксперименте (второй к первой):
    \begin{itemize}
        \item отношение сил тока,
        \item отношение выделяющихся мощностей.
    \end{itemize}
}
\answer{%
    $\eli_2 / \eli_1 = \frac1{10}, \P_2 / \P_1 = \frac1{10}, $
}

\variantsplitter

\addpersonalvariant{Анастасия Ламанова}

\tasknumber{1}%
\task{%
    Саша стартует на лошади и в течение $t = 5\,\text{c}$ двигается с постоянным ускорением $2{,}5\,\frac{\text{м}}{\text{с}^{2}}$.
    Определите
    \begin{itemize}
        \item какую скорость при этом удастся достичь,
        \item какой путь за это время будет пройден,
        \item среднюю скорость за всё время движения, если после начального ускорения продолжить движение равномерно ещё в течение времени $3t$
    \end{itemize}
}
\solutionspace{120pt}

\tasknumber{2}%
\task{%
    Какой путь тело пройдёт за вторую секунду после начала свободного падения?
    Какую скорость в начале этой секунды оно имеет?
}
\solutionspace{120pt}

\tasknumber{3}%
\task{%
    Карусель радиусом $2\,\text{м}$ равномерно совершает 10 оборотов в минуту.
    Определите
    \begin{itemize}
        \item период и частоту её обращения,
        \item скорость и ускорение крайних её точек.
    \end{itemize}
}
\solutionspace{80pt}

\tasknumber{4}%
\task{%
    Даша стоит на обрыве над рекой и методично и строго горизонтально кидает в неё камушки.
    За этим всем наблюдает экспериментатор Глюк, который уже выяснил, что камушки падают в реку спустя $1{,}2\,\text{с}$ после броска,
    а вот дальность полёта оценить сложнее: придётся лезть в воду.
    Выручите Глюка и определите:
    \begin{itemize}
        \item высоту обрыва (вместе с ростом Даши).
        \item дальность полёта камушков (по горизонтали) и их скорость при падении, приняв начальную скорость броска равной $v = 17\,\frac{\text{м}}{\text{с}}$.
    \end{itemize}
    Сопротивлением воздуха пренебречь.
}
\solutionspace{120pt}

\tasknumber{5}%
\task{%
    Пять одинаковых брусков массой $3\,\text{кг}$ каждый лежат на гладком горизонтальном столе.
    Бруски пронумерованы от 1 до 5 и последовательно связаны между собой
    невесомыми нерастяжимыми нитями: 1 со 2, 2 с 3 (ну и с 1) и т.д.
    Экспериментатор Глюк прикладывает постоянную горизонтальную силу $120\,\text{Н}$ к бруску с наибольшим номером.
    С каким ускорением двигается система? Чему равна сила натяжения нити, связывающей бруски 1 и 2?
}
\solutionspace{120pt}

\tasknumber{6}%
\task{%
    Два бруска связаны лёгкой нерастяжимой нитью и перекинуты через неподвижный блок (см.
    рис.).
    Определите силу натяжения нити и ускорения брусков.
    Силами трения пренебречь, массы брусков
    равны $m_1 = 11\,\text{кг}$ и $m_2 = 14\,\text{кг}$.
    % $g = 10\,\frac{\text{м}}{\text{с}^{2}}$.

    \begin{tikzpicture}[x=1.5cm,y=1.5cm,thick]
        \draw
            (-0.4, 0) rectangle (-0.2, 1.2)
            (0.15, 0.5) rectangle (0.45, 1)
            (0, 2) circle [radius=0.3] -- ++(up:0.5)
            (-0.3, 1.2) -- ++(up:0.8)
            (0.3, 1) -- ++(up:1)
            (-0.7, 2.5) -- (0.7, 2.5)
            ;
        \draw[pattern={Lines[angle=51,distance=3pt]},pattern color=black,draw=none] (-0.7, 2.5) rectangle (0.7, 2.75);
        \node [left] (left) at (-0.4, 0.6) { $m_1$ };
        \node [right] (right) at (0.4, 0.75) { $m_2$ };
    \end{tikzpicture}
}
\solutionspace{80pt}

\tasknumber{7}%
\task{%
    Тело массой $2\,\text{кг}$ лежит на горизонтальной поверхности.
    Коэффициент трения между поверхностью и телом $0{,}25$.
    К телу приложена горизонтальная сила $3{,}5\,\text{Н}$.
    Определите силу трения, действующую на тело, и ускорение тела.
    % $g = 10\,\frac{\text{м}}{\text{с}^{2}}$.
}
\solutionspace{120pt}

\tasknumber{8}%
\task{%
    Определите плотность неизвестного вещества, если известно, что опускании тела из него
    в керосин оно будет плавать и на треть выступать над поверхностью жидкости.
}
\solutionspace{120pt}

\tasknumber{9}%
\task{%
    	Определите силу, действующую на правую опору однородного горизонтального стержня длиной $l = 5\,\text{м}$
    	и массой $M = 1\,\text{кг}$, к которому подвешен груз массой $m = 4\,\text{кг}$ на расстоянии $2\,\text{м}$ от правого конца (см.
    рис.).

        \begin{tikzpicture}[thick]
            \draw
                (-2, -0.1) rectangle (2, 0.1)
                (-0.5, -0.1) -- (-0.5, -1)
                (-0.7, -1) rectangle (-0.3, -1.3)
           		(-2, -0.1) -- +(0.15,-0.9) -- +(-0.15,-0.9) -- cycle
            	(2, -0.1) -- +(0.15,-0.9) -- +(-0.15,-0.9) -- cycle
            ;
            \draw[pattern={Lines[angle=51,distance=2pt]},pattern color=black,draw=none]
            	(-2.15, -1.15) rectangle +(0.3, 0.15)
            	(2.15, -1.15) rectangle +(-0.3, 0.15)
            ;
            \node [right] (m_small) at (-0.3, -1.15) { $m$ };
            \node [above] (M_big) at (0, 0.1) { $M$ };
        \end{tikzpicture}
}
\solutionspace{80pt}

\tasknumber{10}%
\task{%
    Тонкий однородный шест длиной $2\,\text{м}$ и массой $10\,\text{кг}$ лежит на горизонтальной поверхности.
    \begin{itemize}
        \item Какую минимальную силу надо приложить к одному из его концов, чтобы оторвать его от этой поверхности?
        \item Какую минимальную работу надо совершить, чтобы поставить его на землю в вертикальное положение?
    \end{itemize}
    % Примите $g = 10\,\frac{\text{м}}{\text{с}^{2}}$.
}
\answer{%
    $A = mg\frac l2 = 100\,\text{Дж}$
}
\solutionspace{120pt}

\tasknumber{11}%
\task{%
    Определите работу силы, которая обеспечит подъём тела массой $5\,\text{кг}$ на высоту $10\,\text{м}$ с постоянным ускорением $3\,\frac{\text{м}}{\text{c}^{2}}$.
    % Примите $g = 10\,\frac{\text{м}}{\text{с}^{2}}$.
}
\answer{%
    \begin{align*}
    &\text{Для подъёма:} A = Fh = (mg + ma) h = m(g+a)h, \\
    &\text{Для спуска:} A = -Fh = -(mg - ma) h = -m(g-a)h, \\
    &\text{В результате получаем:} 650\,\text{Дж}.
    \end{align*}
}
\solutionspace{60pt}

\tasknumber{12}%
\task{%
    Тело бросили вертикально вверх со скоростью $14\,\frac{\text{м}}{\text{c}}$.
    На какой высоте кинетическая энергия тела составит половину от потенциальной?
}
\solutionspace{100pt}

\tasknumber{13}%
\task{%
    Плотность воздуха при нормальных условиях равна $1{,}3\,\frac{\text{кг}}{\text{м}^{3}}$.
    Чему равна плотность воздуха
    при температуре $50\celsius$ и давлении $120\,\text{кПа}$?
}
\solutionspace{120pt}

\tasknumber{14}%
\task{%
    Небольшую цилиндрическую пробирку с воздухом погружают на некоторую глубину в глубокое пресное озеро,
    после чего воздух занимает в ней лишь пятую часть от общего объема.
    Определите глубину, на которую погрузили пробирку.
    Температуру считать постоянной $T = 284\,\text{К}$, давлением паров воды пренебречь,
    атмосферное давление принять равным $p_{\text{aтм}} = 100\,\text{кПа}$.
}
\answer{%
    \begin{align*}
    T\text{— const} &\implies P_1V_1 = \nu RT = P_2V_2.
    \\
    V_2 = \frac 15 V_1 &\implies P_1V_1 = P_2 \cdot \frac 15V_1 \implies P_2 = 5P_1 = 5p_{\text{aтм}}.
    \\
    P_2 = p_{\text{aтм}} + \rho_{\text{в}} g h \implies h = \frac{P_2 - p_{\text{aтм}}}{\rho_{\text{в}} g} &= \frac{5p_{\text{aтм}} - p_{\text{aтм}}}{\rho_{\text{в}} g} = \frac{4 \cdot p_{\text{aтм}}}{\rho_{\text{в}} g} =  \\
     &= \frac{4 \cdot 100\,\text{кПа}}{1000\,\frac{\text{кг}}{\text{м}^{3}} \cdot  10\,\frac{\text{м}}{\text{с}^{2}}} \approx 40\,\text{м}.
    \end{align*}
}
\solutionspace{120pt}

\tasknumber{15}%
\task{%
    Газу сообщили некоторое количество теплоты,
    при этом половину его он потратил на совершение работы,
    одновременно увеличив свою внутреннюю энергию на $3000\,\text{Дж}$.
    Определите работу, совершённую газом.
}
\answer{%
    \begin{align*}
    Q &= A' + \Delta U, A' = \frac 12 Q \implies Q \cdot \cbr{1 - \frac 12} = \Delta U \implies Q = \frac{\Delta U}{1 - \frac 12} = \frac{ 3000\,\text{Дж} }{1 - \frac 12} \approx 6000\,\text{Дж}.
    \\
    A' &= \frac 12 Q
        = \frac 12 \cdot \frac{\Delta U}{1 - \frac 12}
        = \frac{\Delta U}{2 - 1}
        = \frac{ 3000\,\text{Дж} }{2 - 1} \approx 3000\,\text{Дж}.
    \end{align*}
}
\solutionspace{60pt}

\tasknumber{16}%
\task{%
    Два конденсатора ёмкостей $C_1 = 30\,\text{нФ}$ и $C_2 = 40\,\text{нФ}$ последовательно подключают
    к источнику напряжения $U = 400\,\text{В}$ (см.
    рис.).
    % Определите заряды каждого из конденсаторов.
    Определите заряд второго конденсатора.

    \begin{tikzpicture}[circuit ee IEC, semithick]
        \draw  (0, 0) to [capacitor={info={$C_1$}}] (1, 0)
                       to [capacitor={info={$C_2$}}] (2, 0)
        ;
        % \draw [-o] (0, 0) -- ++(-0.5, 0) node[left] {$-$};
        % \draw [-o] (2, 0) -- ++(0.5, 0) node[right] {$+$};
        \draw [-o] (0, 0) -- ++(-0.5, 0) node[left] {};
        \draw [-o] (2, 0) -- ++(0.5, 0) node[right] {};
    \end{tikzpicture}
}
\answer{%
    $
        Q_1
            = Q_2
            = CU
            = \frac{ U }{\frac1{C_1} + \frac1{C_2}}
            = \frac{C_1C_2U}{C_1 + C_2}
            = \frac{
                30\,\text{нФ} \cdot 40\,\text{нФ} \cdot 400\,\text{В}
            }{
                30\,\text{нФ} + 40\,\text{нФ}
            }
            = 6857{,}14\,\text{нКл}
    $
}
\solutionspace{120pt}

\tasknumber{17}%
\task{%
    В вакууме вдоль одной прямой расположены четыре отрицательных заряда так,
    что расстояние между соседними зарядами равно $l$.
    Сделайте рисунок,
    и определите силу, действующую на крайний заряд.
    Модули всех зарядов равны $Q$ ($Q > 0$).
}
\solutionspace{80pt}

\tasknumber{18}%
\task{%
    Юлия проводит эксперименты c 2 кусками одинаковой медной проволки, причём второй кусок в два раза длиннее первого.
    В одном из экспериментов Юлия подаёт на первый кусок проволки напряжение в три раза раз больше, чем на второй.
    Определите отношения в двух проволках в этом эксперименте (второй к первой):
    \begin{itemize}
        \item отношение сил тока,
        \item отношение выделяющихся мощностей.
    \end{itemize}
}
\answer{%
    $\eli_2 / \eli_1 = \frac16, \P_2 / \P_1 = \frac16, $
}

\variantsplitter

\addpersonalvariant{Виктория Легонькова}

\tasknumber{1}%
\task{%
    Женя стартует на мотоцикле и в течение $t = 3\,\text{c}$ двигается с постоянным ускорением $1{,}5\,\frac{\text{м}}{\text{с}^{2}}$.
    Определите
    \begin{itemize}
        \item какую скорость при этом удастся достичь,
        \item какой путь за это время будет пройден,
        \item среднюю скорость за всё время движения, если после начального ускорения продолжить движение равномерно ещё в течение времени $2t$
    \end{itemize}
}
\solutionspace{120pt}

\tasknumber{2}%
\task{%
    Какой путь тело пройдёт за вторую секунду после начала свободного падения?
    Какую скорость в конце этой секунды оно имеет?
}
\solutionspace{120pt}

\tasknumber{3}%
\task{%
    Карусель радиусом $4\,\text{м}$ равномерно совершает 10 оборотов в минуту.
    Определите
    \begin{itemize}
        \item период и частоту её обращения,
        \item скорость и ускорение крайних её точек.
    \end{itemize}
}
\solutionspace{80pt}

\tasknumber{4}%
\task{%
    Даша стоит на обрыве над рекой и методично и строго горизонтально кидает в неё камушки.
    За этим всем наблюдает экспериментатор Глюк, который уже выяснил, что камушки падают в реку спустя $1{,}3\,\text{с}$ после броска,
    а вот дальность полёта оценить сложнее: придётся лезть в воду.
    Выручите Глюка и определите:
    \begin{itemize}
        \item высоту обрыва (вместе с ростом Даши).
        \item дальность полёта камушков (по горизонтали) и их скорость при падении, приняв начальную скорость броска равной $v = 13\,\frac{\text{м}}{\text{с}}$.
    \end{itemize}
    Сопротивлением воздуха пренебречь.
}
\solutionspace{120pt}

\tasknumber{5}%
\task{%
    Пять одинаковых брусков массой $3\,\text{кг}$ каждый лежат на гладком горизонтальном столе.
    Бруски пронумерованы от 1 до 5 и последовательно связаны между собой
    невесомыми нерастяжимыми нитями: 1 со 2, 2 с 3 (ну и с 1) и т.д.
    Экспериментатор Глюк прикладывает постоянную горизонтальную силу $90\,\text{Н}$ к бруску с наименьшим номером.
    С каким ускорением двигается система? Чему равна сила натяжения нити, связывающей бруски 1 и 2?
}
\solutionspace{120pt}

\tasknumber{6}%
\task{%
    Два бруска связаны лёгкой нерастяжимой нитью и перекинуты через неподвижный блок (см.
    рис.).
    Определите силу натяжения нити и ускорения брусков.
    Силами трения пренебречь, массы брусков
    равны $m_1 = 5\,\text{кг}$ и $m_2 = 10\,\text{кг}$.
    % $g = 10\,\frac{\text{м}}{\text{с}^{2}}$.

    \begin{tikzpicture}[x=1.5cm,y=1.5cm,thick]
        \draw
            (-0.4, 0) rectangle (-0.2, 1.2)
            (0.15, 0.5) rectangle (0.45, 1)
            (0, 2) circle [radius=0.3] -- ++(up:0.5)
            (-0.3, 1.2) -- ++(up:0.8)
            (0.3, 1) -- ++(up:1)
            (-0.7, 2.5) -- (0.7, 2.5)
            ;
        \draw[pattern={Lines[angle=51,distance=3pt]},pattern color=black,draw=none] (-0.7, 2.5) rectangle (0.7, 2.75);
        \node [left] (left) at (-0.4, 0.6) { $m_1$ };
        \node [right] (right) at (0.4, 0.75) { $m_2$ };
    \end{tikzpicture}
}
\solutionspace{80pt}

\tasknumber{7}%
\task{%
    Тело массой $1{,}4\,\text{кг}$ лежит на горизонтальной поверхности.
    Коэффициент трения между поверхностью и телом $0{,}25$.
    К телу приложена горизонтальная сила $3{,}5\,\text{Н}$.
    Определите силу трения, действующую на тело, и ускорение тела.
    % $g = 10\,\frac{\text{м}}{\text{с}^{2}}$.
}
\solutionspace{120pt}

\tasknumber{8}%
\task{%
    Определите плотность неизвестного вещества, если известно, что опускании тела из него
    в подсолнечное масло оно будет плавать и на половину выступать над поверхностью жидкости.
}
\solutionspace{120pt}

\tasknumber{9}%
\task{%
    	Определите силу, действующую на правую опору однородного горизонтального стержня длиной $l = 7\,\text{м}$
    	и массой $M = 5\,\text{кг}$, к которому подвешен груз массой $m = 4\,\text{кг}$ на расстоянии $4\,\text{м}$ от правого конца (см.
    рис.).

        \begin{tikzpicture}[thick]
            \draw
                (-2, -0.1) rectangle (2, 0.1)
                (-0.5, -0.1) -- (-0.5, -1)
                (-0.7, -1) rectangle (-0.3, -1.3)
           		(-2, -0.1) -- +(0.15,-0.9) -- +(-0.15,-0.9) -- cycle
            	(2, -0.1) -- +(0.15,-0.9) -- +(-0.15,-0.9) -- cycle
            ;
            \draw[pattern={Lines[angle=51,distance=2pt]},pattern color=black,draw=none]
            	(-2.15, -1.15) rectangle +(0.3, 0.15)
            	(2.15, -1.15) rectangle +(-0.3, 0.15)
            ;
            \node [right] (m_small) at (-0.3, -1.15) { $m$ };
            \node [above] (M_big) at (0, 0.1) { $M$ };
        \end{tikzpicture}
}
\solutionspace{80pt}

\tasknumber{10}%
\task{%
    Тонкий однородный шест длиной $2\,\text{м}$ и массой $20\,\text{кг}$ лежит на горизонтальной поверхности.
    \begin{itemize}
        \item Какую минимальную силу надо приложить к одному из его концов, чтобы оторвать его от этой поверхности?
        \item Какую минимальную работу надо совершить, чтобы поставить его на землю в вертикальное положение?
    \end{itemize}
    % Примите $g = 10\,\frac{\text{м}}{\text{с}^{2}}$.
}
\answer{%
    $A = mg\frac l2 = 200\,\text{Дж}$
}
\solutionspace{120pt}

\tasknumber{11}%
\task{%
    Определите работу силы, которая обеспечит спуск тела массой $2\,\text{кг}$ на высоту $5\,\text{м}$ с постоянным ускорением $2\,\frac{\text{м}}{\text{c}^{2}}$.
    % Примите $g = 10\,\frac{\text{м}}{\text{с}^{2}}$.
}
\answer{%
    \begin{align*}
    &\text{Для подъёма:} A = Fh = (mg + ma) h = m(g+a)h, \\
    &\text{Для спуска:} A = -Fh = -(mg - ma) h = -m(g-a)h, \\
    &\text{В результате получаем:} -80\,\text{Дж}.
    \end{align*}
}
\solutionspace{60pt}

\tasknumber{12}%
\task{%
    Тело бросили вертикально вверх со скоростью $10\,\frac{\text{м}}{\text{c}}$.
    На какой высоте кинетическая энергия тела составит треть от потенциальной?
}
\solutionspace{100pt}

\tasknumber{13}%
\task{%
    Плотность воздуха при нормальных условиях равна $1{,}3\,\frac{\text{кг}}{\text{м}^{3}}$.
    Чему равна плотность воздуха
    при температуре $50\celsius$ и давлении $50\,\text{кПа}$?
}
\solutionspace{120pt}

\tasknumber{14}%
\task{%
    Небольшую цилиндрическую пробирку с воздухом погружают на некоторую глубину в глубокое пресное озеро,
    после чего воздух занимает в ней лишь шестую часть от общего объема.
    Определите глубину, на которую погрузили пробирку.
    Температуру считать постоянной $T = 291\,\text{К}$, давлением паров воды пренебречь,
    атмосферное давление принять равным $p_{\text{aтм}} = 100\,\text{кПа}$.
}
\answer{%
    \begin{align*}
    T\text{— const} &\implies P_1V_1 = \nu RT = P_2V_2.
    \\
    V_2 = \frac 16 V_1 &\implies P_1V_1 = P_2 \cdot \frac 16V_1 \implies P_2 = 6P_1 = 6p_{\text{aтм}}.
    \\
    P_2 = p_{\text{aтм}} + \rho_{\text{в}} g h \implies h = \frac{P_2 - p_{\text{aтм}}}{\rho_{\text{в}} g} &= \frac{6p_{\text{aтм}} - p_{\text{aтм}}}{\rho_{\text{в}} g} = \frac{5 \cdot p_{\text{aтм}}}{\rho_{\text{в}} g} =  \\
     &= \frac{5 \cdot 100\,\text{кПа}}{1000\,\frac{\text{кг}}{\text{м}^{3}} \cdot  10\,\frac{\text{м}}{\text{с}^{2}}} \approx 50\,\text{м}.
    \end{align*}
}
\solutionspace{120pt}

\tasknumber{15}%
\task{%
    Газу сообщили некоторое количество теплоты,
    при этом треть его он потратил на совершение работы,
    одновременно увеличив свою внутреннюю энергию на $1500\,\text{Дж}$.
    Определите работу, совершённую газом.
}
\answer{%
    \begin{align*}
    Q &= A' + \Delta U, A' = \frac 13 Q \implies Q \cdot \cbr{1 - \frac 13} = \Delta U \implies Q = \frac{\Delta U}{1 - \frac 13} = \frac{ 1500\,\text{Дж} }{1 - \frac 13} \approx 2250\,\text{Дж}.
    \\
    A' &= \frac 13 Q
        = \frac 13 \cdot \frac{\Delta U}{1 - \frac 13}
        = \frac{\Delta U}{3 - 1}
        = \frac{ 1500\,\text{Дж} }{3 - 1} \approx 750\,\text{Дж}.
    \end{align*}
}
\solutionspace{60pt}

\tasknumber{16}%
\task{%
    Два конденсатора ёмкостей $C_1 = 20\,\text{нФ}$ и $C_2 = 40\,\text{нФ}$ последовательно подключают
    к источнику напряжения $V = 300\,\text{В}$ (см.
    рис.).
    % Определите заряды каждого из конденсаторов.
    Определите заряд первого конденсатора.

    \begin{tikzpicture}[circuit ee IEC, semithick]
        \draw  (0, 0) to [capacitor={info={$C_1$}}] (1, 0)
                       to [capacitor={info={$C_2$}}] (2, 0)
        ;
        % \draw [-o] (0, 0) -- ++(-0.5, 0) node[left] {$-$};
        % \draw [-o] (2, 0) -- ++(0.5, 0) node[right] {$+$};
        \draw [-o] (0, 0) -- ++(-0.5, 0) node[left] {};
        \draw [-o] (2, 0) -- ++(0.5, 0) node[right] {};
    \end{tikzpicture}
}
\answer{%
    $
        Q_1
            = Q_2
            = CV
            = \frac{ V }{\frac1{C_1} + \frac1{C_2}}
            = \frac{C_1C_2V}{C_1 + C_2}
            = \frac{
                20\,\text{нФ} \cdot 40\,\text{нФ} \cdot 300\,\text{В}
            }{
                20\,\text{нФ} + 40\,\text{нФ}
            }
            = 4000{,}00\,\text{нКл}
    $
}
\solutionspace{120pt}

\tasknumber{17}%
\task{%
    В вакууме вдоль одной прямой расположены три отрицательных заряда так,
    что расстояние между соседними зарядами равно $d$.
    Сделайте рисунок,
    и определите силу, действующую на крайний заряд.
    Модули всех зарядов равны $Q$ ($Q > 0$).
}
\solutionspace{80pt}

\tasknumber{18}%
\task{%
    Юлия проводит эксперименты c 2 кусками одинаковой стальной проволки, причём второй кусок в пять раз длиннее первого.
    В одном из экспериментов Юлия подаёт на первый кусок проволки напряжение в три раза раз больше, чем на второй.
    Определите отношения в двух проволках в этом эксперименте (второй к первой):
    \begin{itemize}
        \item отношение сил тока,
        \item отношение выделяющихся мощностей.
    \end{itemize}
}
\answer{%
    $\eli_2 / \eli_1 = \frac1{15}, \P_2 / \P_1 = \frac1{15}, $
}

\variantsplitter

\addpersonalvariant{Семён Мартынов}

\tasknumber{1}%
\task{%
    Женя стартует на мотоцикле и в течение $t = 10\,\text{c}$ двигается с постоянным ускорением $1{,}5\,\frac{\text{м}}{\text{с}^{2}}$.
    Определите
    \begin{itemize}
        \item какую скорость при этом удастся достичь,
        \item какой путь за это время будет пройден,
        \item среднюю скорость за всё время движения, если после начального ускорения продолжить движение равномерно ещё в течение времени $2t$
    \end{itemize}
}
\solutionspace{120pt}

\tasknumber{2}%
\task{%
    Какой путь тело пройдёт за третью секунду после начала свободного падения?
    Какую скорость в начале этой секунды оно имеет?
}
\solutionspace{120pt}

\tasknumber{3}%
\task{%
    Карусель диаметром $5\,\text{м}$ равномерно совершает 10 оборотов в минуту.
    Определите
    \begin{itemize}
        \item период и частоту её обращения,
        \item скорость и ускорение крайних её точек.
    \end{itemize}
}
\solutionspace{80pt}

\tasknumber{4}%
\task{%
    Паша стоит на обрыве над рекой и методично и строго горизонтально кидает в неё камушки.
    За этим всем наблюдает экспериментатор Глюк, который уже выяснил, что камушки падают в реку спустя $1{,}3\,\text{с}$ после броска,
    а вот дальность полёта оценить сложнее: придётся лезть в воду.
    Выручите Глюка и определите:
    \begin{itemize}
        \item высоту обрыва (вместе с ростом Паши).
        \item дальность полёта камушков (по горизонтали) и их скорость при падении, приняв начальную скорость броска равной $v = 18\,\frac{\text{м}}{\text{с}}$.
    \end{itemize}
    Сопротивлением воздуха пренебречь.
}
\solutionspace{120pt}

\tasknumber{5}%
\task{%
    Пять одинаковых брусков массой $2\,\text{кг}$ каждый лежат на гладком горизонтальном столе.
    Бруски пронумерованы от 1 до 5 и последовательно связаны между собой
    невесомыми нерастяжимыми нитями: 1 со 2, 2 с 3 (ну и с 1) и т.д.
    Экспериментатор Глюк прикладывает постоянную горизонтальную силу $90\,\text{Н}$ к бруску с наименьшим номером.
    С каким ускорением двигается система? Чему равна сила натяжения нити, связывающей бруски 1 и 2?
}
\solutionspace{120pt}

\tasknumber{6}%
\task{%
    Два бруска связаны лёгкой нерастяжимой нитью и перекинуты через неподвижный блок (см.
    рис.).
    Определите силу натяжения нити и ускорения брусков.
    Силами трения пренебречь, массы брусков
    равны $m_1 = 8\,\text{кг}$ и $m_2 = 10\,\text{кг}$.
    % $g = 10\,\frac{\text{м}}{\text{с}^{2}}$.

    \begin{tikzpicture}[x=1.5cm,y=1.5cm,thick]
        \draw
            (-0.4, 0) rectangle (-0.2, 1.2)
            (0.15, 0.5) rectangle (0.45, 1)
            (0, 2) circle [radius=0.3] -- ++(up:0.5)
            (-0.3, 1.2) -- ++(up:0.8)
            (0.3, 1) -- ++(up:1)
            (-0.7, 2.5) -- (0.7, 2.5)
            ;
        \draw[pattern={Lines[angle=51,distance=3pt]},pattern color=black,draw=none] (-0.7, 2.5) rectangle (0.7, 2.75);
        \node [left] (left) at (-0.4, 0.6) { $m_1$ };
        \node [right] (right) at (0.4, 0.75) { $m_2$ };
    \end{tikzpicture}
}
\solutionspace{80pt}

\tasknumber{7}%
\task{%
    Тело массой $1{,}4\,\text{кг}$ лежит на горизонтальной поверхности.
    Коэффициент трения между поверхностью и телом $0{,}2$.
    К телу приложена горизонтальная сила $4{,}5\,\text{Н}$.
    Определите силу трения, действующую на тело, и ускорение тела.
    % $g = 10\,\frac{\text{м}}{\text{с}^{2}}$.
}
\solutionspace{120pt}

\tasknumber{8}%
\task{%
    Определите плотность неизвестного вещества, если известно, что опускании тела из него
    в подсолнечное масло оно будет плавать и на четверть выступать над поверхностью жидкости.
}
\solutionspace{120pt}

\tasknumber{9}%
\task{%
    	Определите силу, действующую на правую опору однородного горизонтального стержня длиной $l = 9\,\text{м}$
    	и массой $M = 1\,\text{кг}$, к которому подвешен груз массой $m = 4\,\text{кг}$ на расстоянии $4\,\text{м}$ от правого конца (см.
    рис.).

        \begin{tikzpicture}[thick]
            \draw
                (-2, -0.1) rectangle (2, 0.1)
                (-0.5, -0.1) -- (-0.5, -1)
                (-0.7, -1) rectangle (-0.3, -1.3)
           		(-2, -0.1) -- +(0.15,-0.9) -- +(-0.15,-0.9) -- cycle
            	(2, -0.1) -- +(0.15,-0.9) -- +(-0.15,-0.9) -- cycle
            ;
            \draw[pattern={Lines[angle=51,distance=2pt]},pattern color=black,draw=none]
            	(-2.15, -1.15) rectangle +(0.3, 0.15)
            	(2.15, -1.15) rectangle +(-0.3, 0.15)
            ;
            \node [right] (m_small) at (-0.3, -1.15) { $m$ };
            \node [above] (M_big) at (0, 0.1) { $M$ };
        \end{tikzpicture}
}
\solutionspace{80pt}

\tasknumber{10}%
\task{%
    Тонкий однородный лом длиной $2\,\text{м}$ и массой $10\,\text{кг}$ лежит на горизонтальной поверхности.
    \begin{itemize}
        \item Какую минимальную силу надо приложить к одному из его концов, чтобы оторвать его от этой поверхности?
        \item Какую минимальную работу надо совершить, чтобы поставить его на землю в вертикальное положение?
    \end{itemize}
    % Примите $g = 10\,\frac{\text{м}}{\text{с}^{2}}$.
}
\answer{%
    $A = mg\frac l2 = 100\,\text{Дж}$
}
\solutionspace{120pt}

\tasknumber{11}%
\task{%
    Определите работу силы, которая обеспечит спуск тела массой $5\,\text{кг}$ на высоту $10\,\text{м}$ с постоянным ускорением $3\,\frac{\text{м}}{\text{c}^{2}}$.
    % Примите $g = 10\,\frac{\text{м}}{\text{с}^{2}}$.
}
\answer{%
    \begin{align*}
    &\text{Для подъёма:} A = Fh = (mg + ma) h = m(g+a)h, \\
    &\text{Для спуска:} A = -Fh = -(mg - ma) h = -m(g-a)h, \\
    &\text{В результате получаем:} -350\,\text{Дж}.
    \end{align*}
}
\solutionspace{60pt}

\tasknumber{12}%
\task{%
    Тело бросили вертикально вверх со скоростью $14\,\frac{\text{м}}{\text{c}}$.
    На какой высоте кинетическая энергия тела составит треть от потенциальной?
}
\solutionspace{100pt}

\tasknumber{13}%
\task{%
    Плотность воздуха при нормальных условиях равна $1{,}3\,\frac{\text{кг}}{\text{м}^{3}}$.
    Чему равна плотность воздуха
    при температуре $150\celsius$ и давлении $80\,\text{кПа}$?
}
\solutionspace{120pt}

\tasknumber{14}%
\task{%
    Небольшую цилиндрическую пробирку с воздухом погружают на некоторую глубину в глубокое пресное озеро,
    после чего воздух занимает в ней лишь третью часть от общего объема.
    Определите глубину, на которую погрузили пробирку.
    Температуру считать постоянной $T = 290\,\text{К}$, давлением паров воды пренебречь,
    атмосферное давление принять равным $p_{\text{aтм}} = 100\,\text{кПа}$.
}
\answer{%
    \begin{align*}
    T\text{— const} &\implies P_1V_1 = \nu RT = P_2V_2.
    \\
    V_2 = \frac 13 V_1 &\implies P_1V_1 = P_2 \cdot \frac 13V_1 \implies P_2 = 3P_1 = 3p_{\text{aтм}}.
    \\
    P_2 = p_{\text{aтм}} + \rho_{\text{в}} g h \implies h = \frac{P_2 - p_{\text{aтм}}}{\rho_{\text{в}} g} &= \frac{3p_{\text{aтм}} - p_{\text{aтм}}}{\rho_{\text{в}} g} = \frac{2 \cdot p_{\text{aтм}}}{\rho_{\text{в}} g} =  \\
     &= \frac{2 \cdot 100\,\text{кПа}}{1000\,\frac{\text{кг}}{\text{м}^{3}} \cdot  10\,\frac{\text{м}}{\text{с}^{2}}} \approx 20\,\text{м}.
    \end{align*}
}
\solutionspace{120pt}

\tasknumber{15}%
\task{%
    Газу сообщили некоторое количество теплоты,
    при этом четверть его он потратил на совершение работы,
    одновременно увеличив свою внутреннюю энергию на $3000\,\text{Дж}$.
    Определите работу, совершённую газом.
}
\answer{%
    \begin{align*}
    Q &= A' + \Delta U, A' = \frac 14 Q \implies Q \cdot \cbr{1 - \frac 14} = \Delta U \implies Q = \frac{\Delta U}{1 - \frac 14} = \frac{ 3000\,\text{Дж} }{1 - \frac 14} \approx 4000\,\text{Дж}.
    \\
    A' &= \frac 14 Q
        = \frac 14 \cdot \frac{\Delta U}{1 - \frac 14}
        = \frac{\Delta U}{4 - 1}
        = \frac{ 3000\,\text{Дж} }{4 - 1} \approx 1000\,\text{Дж}.
    \end{align*}
}
\solutionspace{60pt}

\tasknumber{16}%
\task{%
    Два конденсатора ёмкостей $C_1 = 40\,\text{нФ}$ и $C_2 = 60\,\text{нФ}$ последовательно подключают
    к источнику напряжения $V = 200\,\text{В}$ (см.
    рис.).
    % Определите заряды каждого из конденсаторов.
    Определите заряд первого конденсатора.

    \begin{tikzpicture}[circuit ee IEC, semithick]
        \draw  (0, 0) to [capacitor={info={$C_1$}}] (1, 0)
                       to [capacitor={info={$C_2$}}] (2, 0)
        ;
        % \draw [-o] (0, 0) -- ++(-0.5, 0) node[left] {$-$};
        % \draw [-o] (2, 0) -- ++(0.5, 0) node[right] {$+$};
        \draw [-o] (0, 0) -- ++(-0.5, 0) node[left] {};
        \draw [-o] (2, 0) -- ++(0.5, 0) node[right] {};
    \end{tikzpicture}
}
\answer{%
    $
        Q_1
            = Q_2
            = CV
            = \frac{ V }{\frac1{C_1} + \frac1{C_2}}
            = \frac{C_1C_2V}{C_1 + C_2}
            = \frac{
                40\,\text{нФ} \cdot 60\,\text{нФ} \cdot 200\,\text{В}
            }{
                40\,\text{нФ} + 60\,\text{нФ}
            }
            = 4800{,}00\,\text{нКл}
    $
}
\solutionspace{120pt}

\tasknumber{17}%
\task{%
    В вакууме вдоль одной прямой расположены три отрицательных заряда так,
    что расстояние между соседними зарядами равно $l$.
    Сделайте рисунок,
    и определите силу, действующую на крайний заряд.
    Модули всех зарядов равны $Q$ ($Q > 0$).
}
\solutionspace{80pt}

\tasknumber{18}%
\task{%
    Юлия проводит эксперименты c 2 кусками одинаковой алюминиевой проволки, причём второй кусок в семь раз длиннее первого.
    В одном из экспериментов Юлия подаёт на первый кусок проволки напряжение в восемь раз раз больше, чем на второй.
    Определите отношения в двух проволках в этом эксперименте (второй к первой):
    \begin{itemize}
        \item отношение сил тока,
        \item отношение выделяющихся мощностей.
    \end{itemize}
}
\answer{%
    $\eli_2 / \eli_1 = \frac1{56}, \P_2 / \P_1 = \frac1{56}, $
}

\variantsplitter

\addpersonalvariant{Варвара Минаева}

\tasknumber{1}%
\task{%
    Валя стартует на лошади и в течение $t = 4\,\text{c}$ двигается с постоянным ускорением $0{,}5\,\frac{\text{м}}{\text{с}^{2}}$.
    Определите
    \begin{itemize}
        \item какую скорость при этом удастся достичь,
        \item какой путь за это время будет пройден,
        \item среднюю скорость за всё время движения, если после начального ускорения продолжить движение равномерно ещё в течение времени $3t$
    \end{itemize}
}
\solutionspace{120pt}

\tasknumber{2}%
\task{%
    Какой путь тело пройдёт за вторую секунду после начала свободного падения?
    Какую скорость в начале этой секунды оно имеет?
}
\solutionspace{120pt}

\tasknumber{3}%
\task{%
    Карусель диаметром $4\,\text{м}$ равномерно совершает 10 оборотов в минуту.
    Определите
    \begin{itemize}
        \item период и частоту её обращения,
        \item скорость и ускорение крайних её точек.
    \end{itemize}
}
\solutionspace{80pt}

\tasknumber{4}%
\task{%
    Миша стоит на обрыве над рекой и методично и строго горизонтально кидает в неё камушки.
    За этим всем наблюдает экспериментатор Глюк, который уже выяснил, что камушки падают в реку спустя $1{,}5\,\text{с}$ после броска,
    а вот дальность полёта оценить сложнее: придётся лезть в воду.
    Выручите Глюка и определите:
    \begin{itemize}
        \item высоту обрыва (вместе с ростом Миши).
        \item дальность полёта камушков (по горизонтали) и их скорость при падении, приняв начальную скорость броска равной $v = 16\,\frac{\text{м}}{\text{с}}$.
    \end{itemize}
    Сопротивлением воздуха пренебречь.
}
\solutionspace{120pt}

\tasknumber{5}%
\task{%
    Четыре одинаковых брусков массой $3\,\text{кг}$ каждый лежат на гладком горизонтальном столе.
    Бруски пронумерованы от 1 до 4 и последовательно связаны между собой
    невесомыми нерастяжимыми нитями: 1 со 2, 2 с 3 (ну и с 1) и т.д.
    Экспериментатор Глюк прикладывает постоянную горизонтальную силу $60\,\text{Н}$ к бруску с наименьшим номером.
    С каким ускорением двигается система? Чему равна сила натяжения нити, связывающей бруски 3 и 4?
}
\solutionspace{120pt}

\tasknumber{6}%
\task{%
    Два бруска связаны лёгкой нерастяжимой нитью и перекинуты через неподвижный блок (см.
    рис.).
    Определите силу натяжения нити и ускорения брусков.
    Силами трения пренебречь, массы брусков
    равны $m_1 = 8\,\text{кг}$ и $m_2 = 10\,\text{кг}$.
    % $g = 10\,\frac{\text{м}}{\text{с}^{2}}$.

    \begin{tikzpicture}[x=1.5cm,y=1.5cm,thick]
        \draw
            (-0.4, 0) rectangle (-0.2, 1.2)
            (0.15, 0.5) rectangle (0.45, 1)
            (0, 2) circle [radius=0.3] -- ++(up:0.5)
            (-0.3, 1.2) -- ++(up:0.8)
            (0.3, 1) -- ++(up:1)
            (-0.7, 2.5) -- (0.7, 2.5)
            ;
        \draw[pattern={Lines[angle=51,distance=3pt]},pattern color=black,draw=none] (-0.7, 2.5) rectangle (0.7, 2.75);
        \node [left] (left) at (-0.4, 0.6) { $m_1$ };
        \node [right] (right) at (0.4, 0.75) { $m_2$ };
    \end{tikzpicture}
}
\solutionspace{80pt}

\tasknumber{7}%
\task{%
    Тело массой $2{,}7\,\text{кг}$ лежит на горизонтальной поверхности.
    Коэффициент трения между поверхностью и телом $0{,}15$.
    К телу приложена горизонтальная сила $3{,}5\,\text{Н}$.
    Определите силу трения, действующую на тело, и ускорение тела.
    % $g = 10\,\frac{\text{м}}{\text{с}^{2}}$.
}
\solutionspace{120pt}

\tasknumber{8}%
\task{%
    Определите плотность неизвестного вещества, если известно, что опускании тела из него
    в подсолнечное масло оно будет плавать и на четверть выступать над поверхностью жидкости.
}
\solutionspace{120pt}

\tasknumber{9}%
\task{%
    	Определите силу, действующую на правую опору однородного горизонтального стержня длиной $l = 5\,\text{м}$
    	и массой $M = 5\,\text{кг}$, к которому подвешен груз массой $m = 2\,\text{кг}$ на расстоянии $4\,\text{м}$ от правого конца (см.
    рис.).

        \begin{tikzpicture}[thick]
            \draw
                (-2, -0.1) rectangle (2, 0.1)
                (-0.5, -0.1) -- (-0.5, -1)
                (-0.7, -1) rectangle (-0.3, -1.3)
           		(-2, -0.1) -- +(0.15,-0.9) -- +(-0.15,-0.9) -- cycle
            	(2, -0.1) -- +(0.15,-0.9) -- +(-0.15,-0.9) -- cycle
            ;
            \draw[pattern={Lines[angle=51,distance=2pt]},pattern color=black,draw=none]
            	(-2.15, -1.15) rectangle +(0.3, 0.15)
            	(2.15, -1.15) rectangle +(-0.3, 0.15)
            ;
            \node [right] (m_small) at (-0.3, -1.15) { $m$ };
            \node [above] (M_big) at (0, 0.1) { $M$ };
        \end{tikzpicture}
}
\solutionspace{80pt}

\tasknumber{10}%
\task{%
    Тонкий однородный шест длиной $3\,\text{м}$ и массой $10\,\text{кг}$ лежит на горизонтальной поверхности.
    \begin{itemize}
        \item Какую минимальную силу надо приложить к одному из его концов, чтобы оторвать его от этой поверхности?
        \item Какую минимальную работу надо совершить, чтобы поставить его на землю в вертикальное положение?
    \end{itemize}
    % Примите $g = 10\,\frac{\text{м}}{\text{с}^{2}}$.
}
\answer{%
    $A = mg\frac l2 = 150\,\text{Дж}$
}
\solutionspace{120pt}

\tasknumber{11}%
\task{%
    Определите работу силы, которая обеспечит спуск тела массой $2\,\text{кг}$ на высоту $10\,\text{м}$ с постоянным ускорением $4\,\frac{\text{м}}{\text{c}^{2}}$.
    % Примите $g = 10\,\frac{\text{м}}{\text{с}^{2}}$.
}
\answer{%
    \begin{align*}
    &\text{Для подъёма:} A = Fh = (mg + ma) h = m(g+a)h, \\
    &\text{Для спуска:} A = -Fh = -(mg - ma) h = -m(g-a)h, \\
    &\text{В результате получаем:} -120\,\text{Дж}.
    \end{align*}
}
\solutionspace{60pt}

\tasknumber{12}%
\task{%
    Тело бросили вертикально вверх со скоростью $14\,\frac{\text{м}}{\text{c}}$.
    На какой высоте кинетическая энергия тела составит треть от потенциальной?
}
\solutionspace{100pt}

\tasknumber{13}%
\task{%
    Плотность воздуха при нормальных условиях равна $1{,}3\,\frac{\text{кг}}{\text{м}^{3}}$.
    Чему равна плотность воздуха
    при температуре $200\celsius$ и давлении $120\,\text{кПа}$?
}
\solutionspace{120pt}

\tasknumber{14}%
\task{%
    Небольшую цилиндрическую пробирку с воздухом погружают на некоторую глубину в глубокое пресное озеро,
    после чего воздух занимает в ней лишь пятую часть от общего объема.
    Определите глубину, на которую погрузили пробирку.
    Температуру считать постоянной $T = 278\,\text{К}$, давлением паров воды пренебречь,
    атмосферное давление принять равным $p_{\text{aтм}} = 100\,\text{кПа}$.
}
\answer{%
    \begin{align*}
    T\text{— const} &\implies P_1V_1 = \nu RT = P_2V_2.
    \\
    V_2 = \frac 15 V_1 &\implies P_1V_1 = P_2 \cdot \frac 15V_1 \implies P_2 = 5P_1 = 5p_{\text{aтм}}.
    \\
    P_2 = p_{\text{aтм}} + \rho_{\text{в}} g h \implies h = \frac{P_2 - p_{\text{aтм}}}{\rho_{\text{в}} g} &= \frac{5p_{\text{aтм}} - p_{\text{aтм}}}{\rho_{\text{в}} g} = \frac{4 \cdot p_{\text{aтм}}}{\rho_{\text{в}} g} =  \\
     &= \frac{4 \cdot 100\,\text{кПа}}{1000\,\frac{\text{кг}}{\text{м}^{3}} \cdot  10\,\frac{\text{м}}{\text{с}^{2}}} \approx 40\,\text{м}.
    \end{align*}
}
\solutionspace{120pt}

\tasknumber{15}%
\task{%
    Газу сообщили некоторое количество теплоты,
    при этом половину его он потратил на совершение работы,
    одновременно увеличив свою внутреннюю энергию на $1500\,\text{Дж}$.
    Определите работу, совершённую газом.
}
\answer{%
    \begin{align*}
    Q &= A' + \Delta U, A' = \frac 12 Q \implies Q \cdot \cbr{1 - \frac 12} = \Delta U \implies Q = \frac{\Delta U}{1 - \frac 12} = \frac{ 1500\,\text{Дж} }{1 - \frac 12} \approx 3000\,\text{Дж}.
    \\
    A' &= \frac 12 Q
        = \frac 12 \cdot \frac{\Delta U}{1 - \frac 12}
        = \frac{\Delta U}{2 - 1}
        = \frac{ 1500\,\text{Дж} }{2 - 1} \approx 1500\,\text{Дж}.
    \end{align*}
}
\solutionspace{60pt}

\tasknumber{16}%
\task{%
    Два конденсатора ёмкостей $C_1 = 60\,\text{нФ}$ и $C_2 = 30\,\text{нФ}$ последовательно подключают
    к источнику напряжения $V = 400\,\text{В}$ (см.
    рис.).
    % Определите заряды каждого из конденсаторов.
    Определите заряд первого конденсатора.

    \begin{tikzpicture}[circuit ee IEC, semithick]
        \draw  (0, 0) to [capacitor={info={$C_1$}}] (1, 0)
                       to [capacitor={info={$C_2$}}] (2, 0)
        ;
        % \draw [-o] (0, 0) -- ++(-0.5, 0) node[left] {$-$};
        % \draw [-o] (2, 0) -- ++(0.5, 0) node[right] {$+$};
        \draw [-o] (0, 0) -- ++(-0.5, 0) node[left] {};
        \draw [-o] (2, 0) -- ++(0.5, 0) node[right] {};
    \end{tikzpicture}
}
\answer{%
    $
        Q_1
            = Q_2
            = CV
            = \frac{ V }{\frac1{C_1} + \frac1{C_2}}
            = \frac{C_1C_2V}{C_1 + C_2}
            = \frac{
                60\,\text{нФ} \cdot 30\,\text{нФ} \cdot 400\,\text{В}
            }{
                60\,\text{нФ} + 30\,\text{нФ}
            }
            = 8000{,}00\,\text{нКл}
    $
}
\solutionspace{120pt}

\tasknumber{17}%
\task{%
    В вакууме вдоль одной прямой расположены три положительных заряда так,
    что расстояние между соседними зарядами равно $d$.
    Сделайте рисунок,
    и определите силу, действующую на крайний заряд.
    Модули всех зарядов равны $q$ ($q > 0$).
}
\solutionspace{80pt}

\tasknumber{18}%
\task{%
    Юлия проводит эксперименты c 2 кусками одинаковой алюминиевой проволки, причём второй кусок в восемь раз длиннее первого.
    В одном из экспериментов Юлия подаёт на первый кусок проволки напряжение в четыре раза раз больше, чем на второй.
    Определите отношения в двух проволках в этом эксперименте (второй к первой):
    \begin{itemize}
        \item отношение сил тока,
        \item отношение выделяющихся мощностей.
    \end{itemize}
}
\answer{%
    $\eli_2 / \eli_1 = \frac1{32}, \P_2 / \P_1 = \frac1{32}, $
}

\variantsplitter

\addpersonalvariant{Леонид Никитин}

\tasknumber{1}%
\task{%
    Саша стартует на велосипеде и в течение $t = 4\,\text{c}$ двигается с постоянным ускорением $2{,}5\,\frac{\text{м}}{\text{с}^{2}}$.
    Определите
    \begin{itemize}
        \item какую скорость при этом удастся достичь,
        \item какой путь за это время будет пройден,
        \item среднюю скорость за всё время движения, если после начального ускорения продолжить движение равномерно ещё в течение времени $2t$
    \end{itemize}
}
\solutionspace{120pt}

\tasknumber{2}%
\task{%
    Какой путь тело пройдёт за третью секунду после начала свободного падения?
    Какую скорость в конце этой секунды оно имеет?
}
\solutionspace{120pt}

\tasknumber{3}%
\task{%
    Карусель радиусом $5\,\text{м}$ равномерно совершает 6 оборотов в минуту.
    Определите
    \begin{itemize}
        \item период и частоту её обращения,
        \item скорость и ускорение крайних её точек.
    \end{itemize}
}
\solutionspace{80pt}

\tasknumber{4}%
\task{%
    Маша стоит на обрыве над рекой и методично и строго горизонтально кидает в неё камушки.
    За этим всем наблюдает экспериментатор Глюк, который уже выяснил, что камушки падают в реку спустя $1{,}5\,\text{с}$ после броска,
    а вот дальность полёта оценить сложнее: придётся лезть в воду.
    Выручите Глюка и определите:
    \begin{itemize}
        \item высоту обрыва (вместе с ростом Маши).
        \item дальность полёта камушков (по горизонтали) и их скорость при падении, приняв начальную скорость броска равной $v = 18\,\frac{\text{м}}{\text{с}}$.
    \end{itemize}
    Сопротивлением воздуха пренебречь.
}
\solutionspace{120pt}

\tasknumber{5}%
\task{%
    Шесть одинаковых брусков массой $2\,\text{кг}$ каждый лежат на гладком горизонтальном столе.
    Бруски пронумерованы от 1 до 6 и последовательно связаны между собой
    невесомыми нерастяжимыми нитями: 1 со 2, 2 с 3 (ну и с 1) и т.д.
    Экспериментатор Глюк прикладывает постоянную горизонтальную силу $90\,\text{Н}$ к бруску с наименьшим номером.
    С каким ускорением двигается система? Чему равна сила натяжения нити, связывающей бруски 2 и 3?
}
\solutionspace{120pt}

\tasknumber{6}%
\task{%
    Два бруска связаны лёгкой нерастяжимой нитью и перекинуты через неподвижный блок (см.
    рис.).
    Определите силу натяжения нити и ускорения брусков.
    Силами трения пренебречь, массы брусков
    равны $m_1 = 11\,\text{кг}$ и $m_2 = 4\,\text{кг}$.
    % $g = 10\,\frac{\text{м}}{\text{с}^{2}}$.

    \begin{tikzpicture}[x=1.5cm,y=1.5cm,thick]
        \draw
            (-0.4, 0) rectangle (-0.2, 1.2)
            (0.15, 0.5) rectangle (0.45, 1)
            (0, 2) circle [radius=0.3] -- ++(up:0.5)
            (-0.3, 1.2) -- ++(up:0.8)
            (0.3, 1) -- ++(up:1)
            (-0.7, 2.5) -- (0.7, 2.5)
            ;
        \draw[pattern={Lines[angle=51,distance=3pt]},pattern color=black,draw=none] (-0.7, 2.5) rectangle (0.7, 2.75);
        \node [left] (left) at (-0.4, 0.6) { $m_1$ };
        \node [right] (right) at (0.4, 0.75) { $m_2$ };
    \end{tikzpicture}
}
\solutionspace{80pt}

\tasknumber{7}%
\task{%
    Тело массой $1{,}4\,\text{кг}$ лежит на горизонтальной поверхности.
    Коэффициент трения между поверхностью и телом $0{,}2$.
    К телу приложена горизонтальная сила $3{,}5\,\text{Н}$.
    Определите силу трения, действующую на тело, и ускорение тела.
    % $g = 10\,\frac{\text{м}}{\text{с}^{2}}$.
}
\solutionspace{120pt}

\tasknumber{8}%
\task{%
    Определите плотность неизвестного вещества, если известно, что опускании тела из него
    в подсолнечное масло оно будет плавать и на четверть выступать над поверхностью жидкости.
}
\solutionspace{120pt}

\tasknumber{9}%
\task{%
    	Определите силу, действующую на правую опору однородного горизонтального стержня длиной $l = 9\,\text{м}$
    	и массой $M = 1\,\text{кг}$, к которому подвешен груз массой $m = 3\,\text{кг}$ на расстоянии $4\,\text{м}$ от правого конца (см.
    рис.).

        \begin{tikzpicture}[thick]
            \draw
                (-2, -0.1) rectangle (2, 0.1)
                (-0.5, -0.1) -- (-0.5, -1)
                (-0.7, -1) rectangle (-0.3, -1.3)
           		(-2, -0.1) -- +(0.15,-0.9) -- +(-0.15,-0.9) -- cycle
            	(2, -0.1) -- +(0.15,-0.9) -- +(-0.15,-0.9) -- cycle
            ;
            \draw[pattern={Lines[angle=51,distance=2pt]},pattern color=black,draw=none]
            	(-2.15, -1.15) rectangle +(0.3, 0.15)
            	(2.15, -1.15) rectangle +(-0.3, 0.15)
            ;
            \node [right] (m_small) at (-0.3, -1.15) { $m$ };
            \node [above] (M_big) at (0, 0.1) { $M$ };
        \end{tikzpicture}
}
\solutionspace{80pt}

\tasknumber{10}%
\task{%
    Тонкий однородный лом длиной $3\,\text{м}$ и массой $20\,\text{кг}$ лежит на горизонтальной поверхности.
    \begin{itemize}
        \item Какую минимальную силу надо приложить к одному из его концов, чтобы оторвать его от этой поверхности?
        \item Какую минимальную работу надо совершить, чтобы поставить его на землю в вертикальное положение?
    \end{itemize}
    % Примите $g = 10\,\frac{\text{м}}{\text{с}^{2}}$.
}
\answer{%
    $A = mg\frac l2 = 300\,\text{Дж}$
}
\solutionspace{120pt}

\tasknumber{11}%
\task{%
    Определите работу силы, которая обеспечит спуск тела массой $5\,\text{кг}$ на высоту $10\,\text{м}$ с постоянным ускорением $6\,\frac{\text{м}}{\text{c}^{2}}$.
    % Примите $g = 10\,\frac{\text{м}}{\text{с}^{2}}$.
}
\answer{%
    \begin{align*}
    &\text{Для подъёма:} A = Fh = (mg + ma) h = m(g+a)h, \\
    &\text{Для спуска:} A = -Fh = -(mg - ma) h = -m(g-a)h, \\
    &\text{В результате получаем:} -200\,\text{Дж}.
    \end{align*}
}
\solutionspace{60pt}

\tasknumber{12}%
\task{%
    Тело бросили вертикально вверх со скоростью $20\,\frac{\text{м}}{\text{c}}$.
    На какой высоте кинетическая энергия тела составит треть от потенциальной?
}
\solutionspace{100pt}

\tasknumber{13}%
\task{%
    Плотность воздуха при нормальных условиях равна $1{,}3\,\frac{\text{кг}}{\text{м}^{3}}$.
    Чему равна плотность воздуха
    при температуре $100\celsius$ и давлении $120\,\text{кПа}$?
}
\solutionspace{120pt}

\tasknumber{14}%
\task{%
    Небольшую цилиндрическую пробирку с воздухом погружают на некоторую глубину в глубокое пресное озеро,
    после чего воздух занимает в ней лишь четвертую часть от общего объема.
    Определите глубину, на которую погрузили пробирку.
    Температуру считать постоянной $T = 288\,\text{К}$, давлением паров воды пренебречь,
    атмосферное давление принять равным $p_{\text{aтм}} = 100\,\text{кПа}$.
}
\answer{%
    \begin{align*}
    T\text{— const} &\implies P_1V_1 = \nu RT = P_2V_2.
    \\
    V_2 = \frac 14 V_1 &\implies P_1V_1 = P_2 \cdot \frac 14V_1 \implies P_2 = 4P_1 = 4p_{\text{aтм}}.
    \\
    P_2 = p_{\text{aтм}} + \rho_{\text{в}} g h \implies h = \frac{P_2 - p_{\text{aтм}}}{\rho_{\text{в}} g} &= \frac{4p_{\text{aтм}} - p_{\text{aтм}}}{\rho_{\text{в}} g} = \frac{3 \cdot p_{\text{aтм}}}{\rho_{\text{в}} g} =  \\
     &= \frac{3 \cdot 100\,\text{кПа}}{1000\,\frac{\text{кг}}{\text{м}^{3}} \cdot  10\,\frac{\text{м}}{\text{с}^{2}}} \approx 30\,\text{м}.
    \end{align*}
}
\solutionspace{120pt}

\tasknumber{15}%
\task{%
    Газу сообщили некоторое количество теплоты,
    при этом половину его он потратил на совершение работы,
    одновременно увеличив свою внутреннюю энергию на $1200\,\text{Дж}$.
    Определите работу, совершённую газом.
}
\answer{%
    \begin{align*}
    Q &= A' + \Delta U, A' = \frac 12 Q \implies Q \cdot \cbr{1 - \frac 12} = \Delta U \implies Q = \frac{\Delta U}{1 - \frac 12} = \frac{ 1200\,\text{Дж} }{1 - \frac 12} \approx 2400\,\text{Дж}.
    \\
    A' &= \frac 12 Q
        = \frac 12 \cdot \frac{\Delta U}{1 - \frac 12}
        = \frac{\Delta U}{2 - 1}
        = \frac{ 1200\,\text{Дж} }{2 - 1} \approx 1200\,\text{Дж}.
    \end{align*}
}
\solutionspace{60pt}

\tasknumber{16}%
\task{%
    Два конденсатора ёмкостей $C_1 = 20\,\text{нФ}$ и $C_2 = 60\,\text{нФ}$ последовательно подключают
    к источнику напряжения $U = 200\,\text{В}$ (см.
    рис.).
    % Определите заряды каждого из конденсаторов.
    Определите заряд второго конденсатора.

    \begin{tikzpicture}[circuit ee IEC, semithick]
        \draw  (0, 0) to [capacitor={info={$C_1$}}] (1, 0)
                       to [capacitor={info={$C_2$}}] (2, 0)
        ;
        % \draw [-o] (0, 0) -- ++(-0.5, 0) node[left] {$-$};
        % \draw [-o] (2, 0) -- ++(0.5, 0) node[right] {$+$};
        \draw [-o] (0, 0) -- ++(-0.5, 0) node[left] {};
        \draw [-o] (2, 0) -- ++(0.5, 0) node[right] {};
    \end{tikzpicture}
}
\answer{%
    $
        Q_1
            = Q_2
            = CU
            = \frac{ U }{\frac1{C_1} + \frac1{C_2}}
            = \frac{C_1C_2U}{C_1 + C_2}
            = \frac{
                20\,\text{нФ} \cdot 60\,\text{нФ} \cdot 200\,\text{В}
            }{
                20\,\text{нФ} + 60\,\text{нФ}
            }
            = 3000{,}00\,\text{нКл}
    $
}
\solutionspace{120pt}

\tasknumber{17}%
\task{%
    В вакууме вдоль одной прямой расположены три положительных заряда так,
    что расстояние между соседними зарядами равно $d$.
    Сделайте рисунок,
    и определите силу, действующую на крайний заряд.
    Модули всех зарядов равны $Q$ ($Q > 0$).
}
\solutionspace{80pt}

\tasknumber{18}%
\task{%
    Юлия проводит эксперименты c 2 кусками одинаковой медной проволки, причём второй кусок в шесть раз длиннее первого.
    В одном из экспериментов Юлия подаёт на первый кусок проволки напряжение в два раза раз больше, чем на второй.
    Определите отношения в двух проволках в этом эксперименте (второй к первой):
    \begin{itemize}
        \item отношение сил тока,
        \item отношение выделяющихся мощностей.
    \end{itemize}
}
\answer{%
    $\eli_2 / \eli_1 = \frac1{12}, \P_2 / \P_1 = \frac1{12}, $
}

\variantsplitter

\addpersonalvariant{Тимофей Полетаев}

\tasknumber{1}%
\task{%
    Саша стартует на велосипеде и в течение $t = 4\,\text{c}$ двигается с постоянным ускорением $2{,}5\,\frac{\text{м}}{\text{с}^{2}}$.
    Определите
    \begin{itemize}
        \item какую скорость при этом удастся достичь,
        \item какой путь за это время будет пройден,
        \item среднюю скорость за всё время движения, если после начального ускорения продолжить движение равномерно ещё в течение времени $3t$
    \end{itemize}
}
\solutionspace{120pt}

\tasknumber{2}%
\task{%
    Какой путь тело пройдёт за вторую секунду после начала свободного падения?
    Какую скорость в конце этой секунды оно имеет?
}
\solutionspace{120pt}

\tasknumber{3}%
\task{%
    Карусель радиусом $3\,\text{м}$ равномерно совершает 10 оборотов в минуту.
    Определите
    \begin{itemize}
        \item период и частоту её обращения,
        \item скорость и ускорение крайних её точек.
    \end{itemize}
}
\solutionspace{80pt}

\tasknumber{4}%
\task{%
    Маша стоит на обрыве над рекой и методично и строго горизонтально кидает в неё камушки.
    За этим всем наблюдает экспериментатор Глюк, который уже выяснил, что камушки падают в реку спустя $1{,}3\,\text{с}$ после броска,
    а вот дальность полёта оценить сложнее: придётся лезть в воду.
    Выручите Глюка и определите:
    \begin{itemize}
        \item высоту обрыва (вместе с ростом Маши).
        \item дальность полёта камушков (по горизонтали) и их скорость при падении, приняв начальную скорость броска равной $v = 13\,\frac{\text{м}}{\text{с}}$.
    \end{itemize}
    Сопротивлением воздуха пренебречь.
}
\solutionspace{120pt}

\tasknumber{5}%
\task{%
    Шесть одинаковых брусков массой $2\,\text{кг}$ каждый лежат на гладком горизонтальном столе.
    Бруски пронумерованы от 1 до 6 и последовательно связаны между собой
    невесомыми нерастяжимыми нитями: 1 со 2, 2 с 3 (ну и с 1) и т.д.
    Экспериментатор Глюк прикладывает постоянную горизонтальную силу $90\,\text{Н}$ к бруску с наибольшим номером.
    С каким ускорением двигается система? Чему равна сила натяжения нити, связывающей бруски 1 и 2?
}
\solutionspace{120pt}

\tasknumber{6}%
\task{%
    Два бруска связаны лёгкой нерастяжимой нитью и перекинуты через неподвижный блок (см.
    рис.).
    Определите силу натяжения нити и ускорения брусков.
    Силами трения пренебречь, массы брусков
    равны $m_1 = 11\,\text{кг}$ и $m_2 = 10\,\text{кг}$.
    % $g = 10\,\frac{\text{м}}{\text{с}^{2}}$.

    \begin{tikzpicture}[x=1.5cm,y=1.5cm,thick]
        \draw
            (-0.4, 0) rectangle (-0.2, 1.2)
            (0.15, 0.5) rectangle (0.45, 1)
            (0, 2) circle [radius=0.3] -- ++(up:0.5)
            (-0.3, 1.2) -- ++(up:0.8)
            (0.3, 1) -- ++(up:1)
            (-0.7, 2.5) -- (0.7, 2.5)
            ;
        \draw[pattern={Lines[angle=51,distance=3pt]},pattern color=black,draw=none] (-0.7, 2.5) rectangle (0.7, 2.75);
        \node [left] (left) at (-0.4, 0.6) { $m_1$ };
        \node [right] (right) at (0.4, 0.75) { $m_2$ };
    \end{tikzpicture}
}
\solutionspace{80pt}

\tasknumber{7}%
\task{%
    Тело массой $1{,}4\,\text{кг}$ лежит на горизонтальной поверхности.
    Коэффициент трения между поверхностью и телом $0{,}25$.
    К телу приложена горизонтальная сила $2{,}5\,\text{Н}$.
    Определите силу трения, действующую на тело, и ускорение тела.
    % $g = 10\,\frac{\text{м}}{\text{с}^{2}}$.
}
\solutionspace{120pt}

\tasknumber{8}%
\task{%
    Определите плотность неизвестного вещества, если известно, что опускании тела из него
    в подсолнечное масло оно будет плавать и на четверть выступать над поверхностью жидкости.
}
\solutionspace{120pt}

\tasknumber{9}%
\task{%
    	Определите силу, действующую на левую опору однородного горизонтального стержня длиной $l = 7\,\text{м}$
    	и массой $M = 1\,\text{кг}$, к которому подвешен груз массой $m = 3\,\text{кг}$ на расстоянии $4\,\text{м}$ от правого конца (см.
    рис.).

        \begin{tikzpicture}[thick]
            \draw
                (-2, -0.1) rectangle (2, 0.1)
                (-0.5, -0.1) -- (-0.5, -1)
                (-0.7, -1) rectangle (-0.3, -1.3)
           		(-2, -0.1) -- +(0.15,-0.9) -- +(-0.15,-0.9) -- cycle
            	(2, -0.1) -- +(0.15,-0.9) -- +(-0.15,-0.9) -- cycle
            ;
            \draw[pattern={Lines[angle=51,distance=2pt]},pattern color=black,draw=none]
            	(-2.15, -1.15) rectangle +(0.3, 0.15)
            	(2.15, -1.15) rectangle +(-0.3, 0.15)
            ;
            \node [right] (m_small) at (-0.3, -1.15) { $m$ };
            \node [above] (M_big) at (0, 0.1) { $M$ };
        \end{tikzpicture}
}
\solutionspace{80pt}

\tasknumber{10}%
\task{%
    Тонкий однородный шест длиной $3\,\text{м}$ и массой $30\,\text{кг}$ лежит на горизонтальной поверхности.
    \begin{itemize}
        \item Какую минимальную силу надо приложить к одному из его концов, чтобы оторвать его от этой поверхности?
        \item Какую минимальную работу надо совершить, чтобы поставить его на землю в вертикальное положение?
    \end{itemize}
    % Примите $g = 10\,\frac{\text{м}}{\text{с}^{2}}$.
}
\answer{%
    $A = mg\frac l2 = 450\,\text{Дж}$
}
\solutionspace{120pt}

\tasknumber{11}%
\task{%
    Определите работу силы, которая обеспечит подъём тела массой $2\,\text{кг}$ на высоту $5\,\text{м}$ с постоянным ускорением $3\,\frac{\text{м}}{\text{c}^{2}}$.
    % Примите $g = 10\,\frac{\text{м}}{\text{с}^{2}}$.
}
\answer{%
    \begin{align*}
    &\text{Для подъёма:} A = Fh = (mg + ma) h = m(g+a)h, \\
    &\text{Для спуска:} A = -Fh = -(mg - ma) h = -m(g-a)h, \\
    &\text{В результате получаем:} 130\,\text{Дж}.
    \end{align*}
}
\solutionspace{60pt}

\tasknumber{12}%
\task{%
    Тело бросили вертикально вверх со скоростью $14\,\frac{\text{м}}{\text{c}}$.
    На какой высоте кинетическая энергия тела составит треть от потенциальной?
}
\solutionspace{100pt}

\tasknumber{13}%
\task{%
    Плотность воздуха при нормальных условиях равна $1{,}3\,\frac{\text{кг}}{\text{м}^{3}}$.
    Чему равна плотность воздуха
    при температуре $50\celsius$ и давлении $50\,\text{кПа}$?
}
\solutionspace{120pt}

\tasknumber{14}%
\task{%
    Небольшую цилиндрическую пробирку с воздухом погружают на некоторую глубину в глубокое пресное озеро,
    после чего воздух занимает в ней лишь пятую часть от общего объема.
    Определите глубину, на которую погрузили пробирку.
    Температуру считать постоянной $T = 286\,\text{К}$, давлением паров воды пренебречь,
    атмосферное давление принять равным $p_{\text{aтм}} = 100\,\text{кПа}$.
}
\answer{%
    \begin{align*}
    T\text{— const} &\implies P_1V_1 = \nu RT = P_2V_2.
    \\
    V_2 = \frac 15 V_1 &\implies P_1V_1 = P_2 \cdot \frac 15V_1 \implies P_2 = 5P_1 = 5p_{\text{aтм}}.
    \\
    P_2 = p_{\text{aтм}} + \rho_{\text{в}} g h \implies h = \frac{P_2 - p_{\text{aтм}}}{\rho_{\text{в}} g} &= \frac{5p_{\text{aтм}} - p_{\text{aтм}}}{\rho_{\text{в}} g} = \frac{4 \cdot p_{\text{aтм}}}{\rho_{\text{в}} g} =  \\
     &= \frac{4 \cdot 100\,\text{кПа}}{1000\,\frac{\text{кг}}{\text{м}^{3}} \cdot  10\,\frac{\text{м}}{\text{с}^{2}}} \approx 40\,\text{м}.
    \end{align*}
}
\solutionspace{120pt}

\tasknumber{15}%
\task{%
    Газу сообщили некоторое количество теплоты,
    при этом четверть его он потратил на совершение работы,
    одновременно увеличив свою внутреннюю энергию на $2400\,\text{Дж}$.
    Определите работу, совершённую газом.
}
\answer{%
    \begin{align*}
    Q &= A' + \Delta U, A' = \frac 14 Q \implies Q \cdot \cbr{1 - \frac 14} = \Delta U \implies Q = \frac{\Delta U}{1 - \frac 14} = \frac{ 2400\,\text{Дж} }{1 - \frac 14} \approx 3200\,\text{Дж}.
    \\
    A' &= \frac 14 Q
        = \frac 14 \cdot \frac{\Delta U}{1 - \frac 14}
        = \frac{\Delta U}{4 - 1}
        = \frac{ 2400\,\text{Дж} }{4 - 1} \approx 800\,\text{Дж}.
    \end{align*}
}
\solutionspace{60pt}

\tasknumber{16}%
\task{%
    Два конденсатора ёмкостей $C_1 = 60\,\text{нФ}$ и $C_2 = 30\,\text{нФ}$ последовательно подключают
    к источнику напряжения $U = 200\,\text{В}$ (см.
    рис.).
    % Определите заряды каждого из конденсаторов.
    Определите заряд первого конденсатора.

    \begin{tikzpicture}[circuit ee IEC, semithick]
        \draw  (0, 0) to [capacitor={info={$C_1$}}] (1, 0)
                       to [capacitor={info={$C_2$}}] (2, 0)
        ;
        % \draw [-o] (0, 0) -- ++(-0.5, 0) node[left] {$-$};
        % \draw [-o] (2, 0) -- ++(0.5, 0) node[right] {$+$};
        \draw [-o] (0, 0) -- ++(-0.5, 0) node[left] {};
        \draw [-o] (2, 0) -- ++(0.5, 0) node[right] {};
    \end{tikzpicture}
}
\answer{%
    $
        Q_1
            = Q_2
            = CU
            = \frac{ U }{\frac1{C_1} + \frac1{C_2}}
            = \frac{C_1C_2U}{C_1 + C_2}
            = \frac{
                60\,\text{нФ} \cdot 30\,\text{нФ} \cdot 200\,\text{В}
            }{
                60\,\text{нФ} + 30\,\text{нФ}
            }
            = 4000{,}00\,\text{нКл}
    $
}
\solutionspace{120pt}

\tasknumber{17}%
\task{%
    В вакууме вдоль одной прямой расположены три отрицательных заряда так,
    что расстояние между соседними зарядами равно $r$.
    Сделайте рисунок,
    и определите силу, действующую на крайний заряд.
    Модули всех зарядов равны $q$ ($q > 0$).
}
\solutionspace{80pt}

\tasknumber{18}%
\task{%
    Юлия проводит эксперименты c 2 кусками одинаковой стальной проволки, причём второй кусок в пять раз длиннее первого.
    В одном из экспериментов Юлия подаёт на первый кусок проволки напряжение в два раза раз больше, чем на второй.
    Определите отношения в двух проволках в этом эксперименте (второй к первой):
    \begin{itemize}
        \item отношение сил тока,
        \item отношение выделяющихся мощностей.
    \end{itemize}
}
\answer{%
    $\eli_2 / \eli_1 = \frac1{10}, \P_2 / \P_1 = \frac1{10}, $
}

\variantsplitter

\addpersonalvariant{Андрей Рожков}

\tasknumber{1}%
\task{%
    Валя стартует на велосипеде и в течение $t = 4\,\text{c}$ двигается с постоянным ускорением $1{,}5\,\frac{\text{м}}{\text{с}^{2}}$.
    Определите
    \begin{itemize}
        \item какую скорость при этом удастся достичь,
        \item какой путь за это время будет пройден,
        \item среднюю скорость за всё время движения, если после начального ускорения продолжить движение равномерно ещё в течение времени $3t$
    \end{itemize}
}
\solutionspace{120pt}

\tasknumber{2}%
\task{%
    Какой путь тело пройдёт за третью секунду после начала свободного падения?
    Какую скорость в конце этой секунды оно имеет?
}
\solutionspace{120pt}

\tasknumber{3}%
\task{%
    Карусель радиусом $2\,\text{м}$ равномерно совершает 10 оборотов в минуту.
    Определите
    \begin{itemize}
        \item период и частоту её обращения,
        \item скорость и ускорение крайних её точек.
    \end{itemize}
}
\solutionspace{80pt}

\tasknumber{4}%
\task{%
    Даша стоит на обрыве над рекой и методично и строго горизонтально кидает в неё камушки.
    За этим всем наблюдает экспериментатор Глюк, который уже выяснил, что камушки падают в реку спустя $1{,}5\,\text{с}$ после броска,
    а вот дальность полёта оценить сложнее: придётся лезть в воду.
    Выручите Глюка и определите:
    \begin{itemize}
        \item высоту обрыва (вместе с ростом Даши).
        \item дальность полёта камушков (по горизонтали) и их скорость при падении, приняв начальную скорость броска равной $v = 18\,\frac{\text{м}}{\text{с}}$.
    \end{itemize}
    Сопротивлением воздуха пренебречь.
}
\solutionspace{120pt}

\tasknumber{5}%
\task{%
    Пять одинаковых брусков массой $3\,\text{кг}$ каждый лежат на гладком горизонтальном столе.
    Бруски пронумерованы от 1 до 5 и последовательно связаны между собой
    невесомыми нерастяжимыми нитями: 1 со 2, 2 с 3 (ну и с 1) и т.д.
    Экспериментатор Глюк прикладывает постоянную горизонтальную силу $90\,\text{Н}$ к бруску с наименьшим номером.
    С каким ускорением двигается система? Чему равна сила натяжения нити, связывающей бруски 3 и 4?
}
\solutionspace{120pt}

\tasknumber{6}%
\task{%
    Два бруска связаны лёгкой нерастяжимой нитью и перекинуты через неподвижный блок (см.
    рис.).
    Определите силу натяжения нити и ускорения брусков.
    Силами трения пренебречь, массы брусков
    равны $m_1 = 8\,\text{кг}$ и $m_2 = 14\,\text{кг}$.
    % $g = 10\,\frac{\text{м}}{\text{с}^{2}}$.

    \begin{tikzpicture}[x=1.5cm,y=1.5cm,thick]
        \draw
            (-0.4, 0) rectangle (-0.2, 1.2)
            (0.15, 0.5) rectangle (0.45, 1)
            (0, 2) circle [radius=0.3] -- ++(up:0.5)
            (-0.3, 1.2) -- ++(up:0.8)
            (0.3, 1) -- ++(up:1)
            (-0.7, 2.5) -- (0.7, 2.5)
            ;
        \draw[pattern={Lines[angle=51,distance=3pt]},pattern color=black,draw=none] (-0.7, 2.5) rectangle (0.7, 2.75);
        \node [left] (left) at (-0.4, 0.6) { $m_1$ };
        \node [right] (right) at (0.4, 0.75) { $m_2$ };
    \end{tikzpicture}
}
\solutionspace{80pt}

\tasknumber{7}%
\task{%
    Тело массой $1{,}4\,\text{кг}$ лежит на горизонтальной поверхности.
    Коэффициент трения между поверхностью и телом $0{,}2$.
    К телу приложена горизонтальная сила $4{,}5\,\text{Н}$.
    Определите силу трения, действующую на тело, и ускорение тела.
    % $g = 10\,\frac{\text{м}}{\text{с}^{2}}$.
}
\solutionspace{120pt}

\tasknumber{8}%
\task{%
    Определите плотность неизвестного вещества, если известно, что опускании тела из него
    в керосин оно будет плавать и на четверть выступать над поверхностью жидкости.
}
\solutionspace{120pt}

\tasknumber{9}%
\task{%
    	Определите силу, действующую на левую опору однородного горизонтального стержня длиной $l = 5\,\text{м}$
    	и массой $M = 1\,\text{кг}$, к которому подвешен груз массой $m = 2\,\text{кг}$ на расстоянии $2\,\text{м}$ от правого конца (см.
    рис.).

        \begin{tikzpicture}[thick]
            \draw
                (-2, -0.1) rectangle (2, 0.1)
                (-0.5, -0.1) -- (-0.5, -1)
                (-0.7, -1) rectangle (-0.3, -1.3)
           		(-2, -0.1) -- +(0.15,-0.9) -- +(-0.15,-0.9) -- cycle
            	(2, -0.1) -- +(0.15,-0.9) -- +(-0.15,-0.9) -- cycle
            ;
            \draw[pattern={Lines[angle=51,distance=2pt]},pattern color=black,draw=none]
            	(-2.15, -1.15) rectangle +(0.3, 0.15)
            	(2.15, -1.15) rectangle +(-0.3, 0.15)
            ;
            \node [right] (m_small) at (-0.3, -1.15) { $m$ };
            \node [above] (M_big) at (0, 0.1) { $M$ };
        \end{tikzpicture}
}
\solutionspace{80pt}

\tasknumber{10}%
\task{%
    Тонкий однородный лом длиной $3\,\text{м}$ и массой $30\,\text{кг}$ лежит на горизонтальной поверхности.
    \begin{itemize}
        \item Какую минимальную силу надо приложить к одному из его концов, чтобы оторвать его от этой поверхности?
        \item Какую минимальную работу надо совершить, чтобы поставить его на землю в вертикальное положение?
    \end{itemize}
    % Примите $g = 10\,\frac{\text{м}}{\text{с}^{2}}$.
}
\answer{%
    $A = mg\frac l2 = 450\,\text{Дж}$
}
\solutionspace{120pt}

\tasknumber{11}%
\task{%
    Определите работу силы, которая обеспечит спуск тела массой $5\,\text{кг}$ на высоту $10\,\text{м}$ с постоянным ускорением $4\,\frac{\text{м}}{\text{c}^{2}}$.
    % Примите $g = 10\,\frac{\text{м}}{\text{с}^{2}}$.
}
\answer{%
    \begin{align*}
    &\text{Для подъёма:} A = Fh = (mg + ma) h = m(g+a)h, \\
    &\text{Для спуска:} A = -Fh = -(mg - ma) h = -m(g-a)h, \\
    &\text{В результате получаем:} -300\,\text{Дж}.
    \end{align*}
}
\solutionspace{60pt}

\tasknumber{12}%
\task{%
    Тело бросили вертикально вверх со скоростью $14\,\frac{\text{м}}{\text{c}}$.
    На какой высоте кинетическая энергия тела составит треть от потенциальной?
}
\solutionspace{100pt}

\tasknumber{13}%
\task{%
    Плотность воздуха при нормальных условиях равна $1{,}3\,\frac{\text{кг}}{\text{м}^{3}}$.
    Чему равна плотность воздуха
    при температуре $150\celsius$ и давлении $150\,\text{кПа}$?
}
\solutionspace{120pt}

\tasknumber{14}%
\task{%
    Небольшую цилиндрическую пробирку с воздухом погружают на некоторую глубину в глубокое пресное озеро,
    после чего воздух занимает в ней лишь пятую часть от общего объема.
    Определите глубину, на которую погрузили пробирку.
    Температуру считать постоянной $T = 292\,\text{К}$, давлением паров воды пренебречь,
    атмосферное давление принять равным $p_{\text{aтм}} = 100\,\text{кПа}$.
}
\answer{%
    \begin{align*}
    T\text{— const} &\implies P_1V_1 = \nu RT = P_2V_2.
    \\
    V_2 = \frac 15 V_1 &\implies P_1V_1 = P_2 \cdot \frac 15V_1 \implies P_2 = 5P_1 = 5p_{\text{aтм}}.
    \\
    P_2 = p_{\text{aтм}} + \rho_{\text{в}} g h \implies h = \frac{P_2 - p_{\text{aтм}}}{\rho_{\text{в}} g} &= \frac{5p_{\text{aтм}} - p_{\text{aтм}}}{\rho_{\text{в}} g} = \frac{4 \cdot p_{\text{aтм}}}{\rho_{\text{в}} g} =  \\
     &= \frac{4 \cdot 100\,\text{кПа}}{1000\,\frac{\text{кг}}{\text{м}^{3}} \cdot  10\,\frac{\text{м}}{\text{с}^{2}}} \approx 40\,\text{м}.
    \end{align*}
}
\solutionspace{120pt}

\tasknumber{15}%
\task{%
    Газу сообщили некоторое количество теплоты,
    при этом половину его он потратил на совершение работы,
    одновременно увеличив свою внутреннюю энергию на $3000\,\text{Дж}$.
    Определите работу, совершённую газом.
}
\answer{%
    \begin{align*}
    Q &= A' + \Delta U, A' = \frac 12 Q \implies Q \cdot \cbr{1 - \frac 12} = \Delta U \implies Q = \frac{\Delta U}{1 - \frac 12} = \frac{ 3000\,\text{Дж} }{1 - \frac 12} \approx 6000\,\text{Дж}.
    \\
    A' &= \frac 12 Q
        = \frac 12 \cdot \frac{\Delta U}{1 - \frac 12}
        = \frac{\Delta U}{2 - 1}
        = \frac{ 3000\,\text{Дж} }{2 - 1} \approx 3000\,\text{Дж}.
    \end{align*}
}
\solutionspace{60pt}

\tasknumber{16}%
\task{%
    Два конденсатора ёмкостей $C_1 = 60\,\text{нФ}$ и $C_2 = 30\,\text{нФ}$ последовательно подключают
    к источнику напряжения $U = 300\,\text{В}$ (см.
    рис.).
    % Определите заряды каждого из конденсаторов.
    Определите заряд второго конденсатора.

    \begin{tikzpicture}[circuit ee IEC, semithick]
        \draw  (0, 0) to [capacitor={info={$C_1$}}] (1, 0)
                       to [capacitor={info={$C_2$}}] (2, 0)
        ;
        % \draw [-o] (0, 0) -- ++(-0.5, 0) node[left] {$-$};
        % \draw [-o] (2, 0) -- ++(0.5, 0) node[right] {$+$};
        \draw [-o] (0, 0) -- ++(-0.5, 0) node[left] {};
        \draw [-o] (2, 0) -- ++(0.5, 0) node[right] {};
    \end{tikzpicture}
}
\answer{%
    $
        Q_1
            = Q_2
            = CU
            = \frac{ U }{\frac1{C_1} + \frac1{C_2}}
            = \frac{C_1C_2U}{C_1 + C_2}
            = \frac{
                60\,\text{нФ} \cdot 30\,\text{нФ} \cdot 300\,\text{В}
            }{
                60\,\text{нФ} + 30\,\text{нФ}
            }
            = 6000{,}00\,\text{нКл}
    $
}
\solutionspace{120pt}

\tasknumber{17}%
\task{%
    В вакууме вдоль одной прямой расположены четыре отрицательных заряда так,
    что расстояние между соседними зарядами равно $d$.
    Сделайте рисунок,
    и определите силу, действующую на крайний заряд.
    Модули всех зарядов равны $Q$ ($Q > 0$).
}
\solutionspace{80pt}

\tasknumber{18}%
\task{%
    Юлия проводит эксперименты c 2 кусками одинаковой медной проволки, причём второй кусок в два раза длиннее первого.
    В одном из экспериментов Юлия подаёт на первый кусок проволки напряжение в десять раз раз больше, чем на второй.
    Определите отношения в двух проволках в этом эксперименте (второй к первой):
    \begin{itemize}
        \item отношение сил тока,
        \item отношение выделяющихся мощностей.
    \end{itemize}
}
\answer{%
    $\eli_2 / \eli_1 = \frac1{20}, \P_2 / \P_1 = \frac1{20}, $
}

\variantsplitter

\addpersonalvariant{Рената Таржиманова}

\tasknumber{1}%
\task{%
    Женя стартует на мотоцикле и в течение $t = 10\,\text{c}$ двигается с постоянным ускорением $2\,\frac{\text{м}}{\text{с}^{2}}$.
    Определите
    \begin{itemize}
        \item какую скорость при этом удастся достичь,
        \item какой путь за это время будет пройден,
        \item среднюю скорость за всё время движения, если после начального ускорения продолжить движение равномерно ещё в течение времени $2t$
    \end{itemize}
}
\solutionspace{120pt}

\tasknumber{2}%
\task{%
    Какой путь тело пройдёт за шестую секунду после начала свободного падения?
    Какую скорость в конце этой секунды оно имеет?
}
\solutionspace{120pt}

\tasknumber{3}%
\task{%
    Карусель диаметром $5\,\text{м}$ равномерно совершает 6 оборотов в минуту.
    Определите
    \begin{itemize}
        \item период и частоту её обращения,
        \item скорость и ускорение крайних её точек.
    \end{itemize}
}
\solutionspace{80pt}

\tasknumber{4}%
\task{%
    Миша стоит на обрыве над рекой и методично и строго горизонтально кидает в неё камушки.
    За этим всем наблюдает экспериментатор Глюк, который уже выяснил, что камушки падают в реку спустя $1{,}7\,\text{с}$ после броска,
    а вот дальность полёта оценить сложнее: придётся лезть в воду.
    Выручите Глюка и определите:
    \begin{itemize}
        \item высоту обрыва (вместе с ростом Миши).
        \item дальность полёта камушков (по горизонтали) и их скорость при падении, приняв начальную скорость броска равной $v = 12\,\frac{\text{м}}{\text{с}}$.
    \end{itemize}
    Сопротивлением воздуха пренебречь.
}
\solutionspace{120pt}

\tasknumber{5}%
\task{%
    Четыре одинаковых брусков массой $3\,\text{кг}$ каждый лежат на гладком горизонтальном столе.
    Бруски пронумерованы от 1 до 4 и последовательно связаны между собой
    невесомыми нерастяжимыми нитями: 1 со 2, 2 с 3 (ну и с 1) и т.д.
    Экспериментатор Глюк прикладывает постоянную горизонтальную силу $120\,\text{Н}$ к бруску с наименьшим номером.
    С каким ускорением двигается система? Чему равна сила натяжения нити, связывающей бруски 3 и 4?
}
\solutionspace{120pt}

\tasknumber{6}%
\task{%
    Два бруска связаны лёгкой нерастяжимой нитью и перекинуты через неподвижный блок (см.
    рис.).
    Определите силу натяжения нити и ускорения брусков.
    Силами трения пренебречь, массы брусков
    равны $m_1 = 5\,\text{кг}$ и $m_2 = 14\,\text{кг}$.
    % $g = 10\,\frac{\text{м}}{\text{с}^{2}}$.

    \begin{tikzpicture}[x=1.5cm,y=1.5cm,thick]
        \draw
            (-0.4, 0) rectangle (-0.2, 1.2)
            (0.15, 0.5) rectangle (0.45, 1)
            (0, 2) circle [radius=0.3] -- ++(up:0.5)
            (-0.3, 1.2) -- ++(up:0.8)
            (0.3, 1) -- ++(up:1)
            (-0.7, 2.5) -- (0.7, 2.5)
            ;
        \draw[pattern={Lines[angle=51,distance=3pt]},pattern color=black,draw=none] (-0.7, 2.5) rectangle (0.7, 2.75);
        \node [left] (left) at (-0.4, 0.6) { $m_1$ };
        \node [right] (right) at (0.4, 0.75) { $m_2$ };
    \end{tikzpicture}
}
\solutionspace{80pt}

\tasknumber{7}%
\task{%
    Тело массой $2{,}7\,\text{кг}$ лежит на горизонтальной поверхности.
    Коэффициент трения между поверхностью и телом $0{,}2$.
    К телу приложена горизонтальная сила $4{,}5\,\text{Н}$.
    Определите силу трения, действующую на тело, и ускорение тела.
    % $g = 10\,\frac{\text{м}}{\text{с}^{2}}$.
}
\solutionspace{120pt}

\tasknumber{8}%
\task{%
    Определите плотность неизвестного вещества, если известно, что опускании тела из него
    в подсолнечное масло оно будет плавать и на половину выступать над поверхностью жидкости.
}
\solutionspace{120pt}

\tasknumber{9}%
\task{%
    	Определите силу, действующую на левую опору однородного горизонтального стержня длиной $l = 5\,\text{м}$
    	и массой $M = 5\,\text{кг}$, к которому подвешен груз массой $m = 4\,\text{кг}$ на расстоянии $2\,\text{м}$ от правого конца (см.
    рис.).

        \begin{tikzpicture}[thick]
            \draw
                (-2, -0.1) rectangle (2, 0.1)
                (-0.5, -0.1) -- (-0.5, -1)
                (-0.7, -1) rectangle (-0.3, -1.3)
           		(-2, -0.1) -- +(0.15,-0.9) -- +(-0.15,-0.9) -- cycle
            	(2, -0.1) -- +(0.15,-0.9) -- +(-0.15,-0.9) -- cycle
            ;
            \draw[pattern={Lines[angle=51,distance=2pt]},pattern color=black,draw=none]
            	(-2.15, -1.15) rectangle +(0.3, 0.15)
            	(2.15, -1.15) rectangle +(-0.3, 0.15)
            ;
            \node [right] (m_small) at (-0.3, -1.15) { $m$ };
            \node [above] (M_big) at (0, 0.1) { $M$ };
        \end{tikzpicture}
}
\solutionspace{80pt}

\tasknumber{10}%
\task{%
    Тонкий однородный шест длиной $3\,\text{м}$ и массой $30\,\text{кг}$ лежит на горизонтальной поверхности.
    \begin{itemize}
        \item Какую минимальную силу надо приложить к одному из его концов, чтобы оторвать его от этой поверхности?
        \item Какую минимальную работу надо совершить, чтобы поставить его на землю в вертикальное положение?
    \end{itemize}
    % Примите $g = 10\,\frac{\text{м}}{\text{с}^{2}}$.
}
\answer{%
    $A = mg\frac l2 = 450\,\text{Дж}$
}
\solutionspace{120pt}

\tasknumber{11}%
\task{%
    Определите работу силы, которая обеспечит подъём тела массой $5\,\text{кг}$ на высоту $10\,\text{м}$ с постоянным ускорением $4\,\frac{\text{м}}{\text{c}^{2}}$.
    % Примите $g = 10\,\frac{\text{м}}{\text{с}^{2}}$.
}
\answer{%
    \begin{align*}
    &\text{Для подъёма:} A = Fh = (mg + ma) h = m(g+a)h, \\
    &\text{Для спуска:} A = -Fh = -(mg - ma) h = -m(g-a)h, \\
    &\text{В результате получаем:} 700\,\text{Дж}.
    \end{align*}
}
\solutionspace{60pt}

\tasknumber{12}%
\task{%
    Тело бросили вертикально вверх со скоростью $14\,\frac{\text{м}}{\text{c}}$.
    На какой высоте кинетическая энергия тела составит треть от потенциальной?
}
\solutionspace{100pt}

\tasknumber{13}%
\task{%
    Плотность воздуха при нормальных условиях равна $1{,}3\,\frac{\text{кг}}{\text{м}^{3}}$.
    Чему равна плотность воздуха
    при температуре $100\celsius$ и давлении $50\,\text{кПа}$?
}
\solutionspace{120pt}

\tasknumber{14}%
\task{%
    Небольшую цилиндрическую пробирку с воздухом погружают на некоторую глубину в глубокое пресное озеро,
    после чего воздух занимает в ней лишь третью часть от общего объема.
    Определите глубину, на которую погрузили пробирку.
    Температуру считать постоянной $T = 289\,\text{К}$, давлением паров воды пренебречь,
    атмосферное давление принять равным $p_{\text{aтм}} = 100\,\text{кПа}$.
}
\answer{%
    \begin{align*}
    T\text{— const} &\implies P_1V_1 = \nu RT = P_2V_2.
    \\
    V_2 = \frac 13 V_1 &\implies P_1V_1 = P_2 \cdot \frac 13V_1 \implies P_2 = 3P_1 = 3p_{\text{aтм}}.
    \\
    P_2 = p_{\text{aтм}} + \rho_{\text{в}} g h \implies h = \frac{P_2 - p_{\text{aтм}}}{\rho_{\text{в}} g} &= \frac{3p_{\text{aтм}} - p_{\text{aтм}}}{\rho_{\text{в}} g} = \frac{2 \cdot p_{\text{aтм}}}{\rho_{\text{в}} g} =  \\
     &= \frac{2 \cdot 100\,\text{кПа}}{1000\,\frac{\text{кг}}{\text{м}^{3}} \cdot  10\,\frac{\text{м}}{\text{с}^{2}}} \approx 20\,\text{м}.
    \end{align*}
}
\solutionspace{120pt}

\tasknumber{15}%
\task{%
    Газу сообщили некоторое количество теплоты,
    при этом треть его он потратил на совершение работы,
    одновременно увеличив свою внутреннюю энергию на $2400\,\text{Дж}$.
    Определите работу, совершённую газом.
}
\answer{%
    \begin{align*}
    Q &= A' + \Delta U, A' = \frac 13 Q \implies Q \cdot \cbr{1 - \frac 13} = \Delta U \implies Q = \frac{\Delta U}{1 - \frac 13} = \frac{ 2400\,\text{Дж} }{1 - \frac 13} \approx 3600\,\text{Дж}.
    \\
    A' &= \frac 13 Q
        = \frac 13 \cdot \frac{\Delta U}{1 - \frac 13}
        = \frac{\Delta U}{3 - 1}
        = \frac{ 2400\,\text{Дж} }{3 - 1} \approx 1200\,\text{Дж}.
    \end{align*}
}
\solutionspace{60pt}

\tasknumber{16}%
\task{%
    Два конденсатора ёмкостей $C_1 = 60\,\text{нФ}$ и $C_2 = 20\,\text{нФ}$ последовательно подключают
    к источнику напряжения $U = 450\,\text{В}$ (см.
    рис.).
    % Определите заряды каждого из конденсаторов.
    Определите заряд второго конденсатора.

    \begin{tikzpicture}[circuit ee IEC, semithick]
        \draw  (0, 0) to [capacitor={info={$C_1$}}] (1, 0)
                       to [capacitor={info={$C_2$}}] (2, 0)
        ;
        % \draw [-o] (0, 0) -- ++(-0.5, 0) node[left] {$-$};
        % \draw [-o] (2, 0) -- ++(0.5, 0) node[right] {$+$};
        \draw [-o] (0, 0) -- ++(-0.5, 0) node[left] {};
        \draw [-o] (2, 0) -- ++(0.5, 0) node[right] {};
    \end{tikzpicture}
}
\answer{%
    $
        Q_1
            = Q_2
            = CU
            = \frac{ U }{\frac1{C_1} + \frac1{C_2}}
            = \frac{C_1C_2U}{C_1 + C_2}
            = \frac{
                60\,\text{нФ} \cdot 20\,\text{нФ} \cdot 450\,\text{В}
            }{
                60\,\text{нФ} + 20\,\text{нФ}
            }
            = 6750{,}00\,\text{нКл}
    $
}
\solutionspace{120pt}

\tasknumber{17}%
\task{%
    В вакууме вдоль одной прямой расположены четыре положительных заряда так,
    что расстояние между соседними зарядами равно $r$.
    Сделайте рисунок,
    и определите силу, действующую на крайний заряд.
    Модули всех зарядов равны $Q$ ($Q > 0$).
}
\solutionspace{80pt}

\tasknumber{18}%
\task{%
    Юлия проводит эксперименты c 2 кусками одинаковой стальной проволки, причём второй кусок в четыре раза длиннее первого.
    В одном из экспериментов Юлия подаёт на первый кусок проволки напряжение в семь раз раз больше, чем на второй.
    Определите отношения в двух проволках в этом эксперименте (второй к первой):
    \begin{itemize}
        \item отношение сил тока,
        \item отношение выделяющихся мощностей.
    \end{itemize}
}
\answer{%
    $\eli_2 / \eli_1 = \frac1{28}, \P_2 / \P_1 = \frac1{28}, $
}

\variantsplitter

\addpersonalvariant{Андрей Щербаков}

\tasknumber{1}%
\task{%
    Женя стартует на мотоцикле и в течение $t = 10\,\text{c}$ двигается с постоянным ускорением $2\,\frac{\text{м}}{\text{с}^{2}}$.
    Определите
    \begin{itemize}
        \item какую скорость при этом удастся достичь,
        \item какой путь за это время будет пройден,
        \item среднюю скорость за всё время движения, если после начального ускорения продолжить движение равномерно ещё в течение времени $2t$
    \end{itemize}
}
\solutionspace{120pt}

\tasknumber{2}%
\task{%
    Какой путь тело пройдёт за пятую секунду после начала свободного падения?
    Какую скорость в начале этой секунды оно имеет?
}
\solutionspace{120pt}

\tasknumber{3}%
\task{%
    Карусель радиусом $4\,\text{м}$ равномерно совершает 5 оборотов в минуту.
    Определите
    \begin{itemize}
        \item период и частоту её обращения,
        \item скорость и ускорение крайних её точек.
    \end{itemize}
}
\solutionspace{80pt}

\tasknumber{4}%
\task{%
    Паша стоит на обрыве над рекой и методично и строго горизонтально кидает в неё камушки.
    За этим всем наблюдает экспериментатор Глюк, который уже выяснил, что камушки падают в реку спустя $1{,}2\,\text{с}$ после броска,
    а вот дальность полёта оценить сложнее: придётся лезть в воду.
    Выручите Глюка и определите:
    \begin{itemize}
        \item высоту обрыва (вместе с ростом Паши).
        \item дальность полёта камушков (по горизонтали) и их скорость при падении, приняв начальную скорость броска равной $v = 15\,\frac{\text{м}}{\text{с}}$.
    \end{itemize}
    Сопротивлением воздуха пренебречь.
}
\solutionspace{120pt}

\tasknumber{5}%
\task{%
    Шесть одинаковых брусков массой $2\,\text{кг}$ каждый лежат на гладком горизонтальном столе.
    Бруски пронумерованы от 1 до 6 и последовательно связаны между собой
    невесомыми нерастяжимыми нитями: 1 со 2, 2 с 3 (ну и с 1) и т.д.
    Экспериментатор Глюк прикладывает постоянную горизонтальную силу $60\,\text{Н}$ к бруску с наименьшим номером.
    С каким ускорением двигается система? Чему равна сила натяжения нити, связывающей бруски 2 и 3?
}
\solutionspace{120pt}

\tasknumber{6}%
\task{%
    Два бруска связаны лёгкой нерастяжимой нитью и перекинуты через неподвижный блок (см.
    рис.).
    Определите силу натяжения нити и ускорения брусков.
    Силами трения пренебречь, массы брусков
    равны $m_1 = 8\,\text{кг}$ и $m_2 = 10\,\text{кг}$.
    % $g = 10\,\frac{\text{м}}{\text{с}^{2}}$.

    \begin{tikzpicture}[x=1.5cm,y=1.5cm,thick]
        \draw
            (-0.4, 0) rectangle (-0.2, 1.2)
            (0.15, 0.5) rectangle (0.45, 1)
            (0, 2) circle [radius=0.3] -- ++(up:0.5)
            (-0.3, 1.2) -- ++(up:0.8)
            (0.3, 1) -- ++(up:1)
            (-0.7, 2.5) -- (0.7, 2.5)
            ;
        \draw[pattern={Lines[angle=51,distance=3pt]},pattern color=black,draw=none] (-0.7, 2.5) rectangle (0.7, 2.75);
        \node [left] (left) at (-0.4, 0.6) { $m_1$ };
        \node [right] (right) at (0.4, 0.75) { $m_2$ };
    \end{tikzpicture}
}
\solutionspace{80pt}

\tasknumber{7}%
\task{%
    Тело массой $1{,}4\,\text{кг}$ лежит на горизонтальной поверхности.
    Коэффициент трения между поверхностью и телом $0{,}2$.
    К телу приложена горизонтальная сила $3{,}5\,\text{Н}$.
    Определите силу трения, действующую на тело, и ускорение тела.
    % $g = 10\,\frac{\text{м}}{\text{с}^{2}}$.
}
\solutionspace{120pt}

\tasknumber{8}%
\task{%
    Определите плотность неизвестного вещества, если известно, что опускании тела из него
    в керосин оно будет плавать и на треть выступать над поверхностью жидкости.
}
\solutionspace{120pt}

\tasknumber{9}%
\task{%
    	Определите силу, действующую на левую опору однородного горизонтального стержня длиной $l = 9\,\text{м}$
    	и массой $M = 5\,\text{кг}$, к которому подвешен груз массой $m = 3\,\text{кг}$ на расстоянии $4\,\text{м}$ от правого конца (см.
    рис.).

        \begin{tikzpicture}[thick]
            \draw
                (-2, -0.1) rectangle (2, 0.1)
                (-0.5, -0.1) -- (-0.5, -1)
                (-0.7, -1) rectangle (-0.3, -1.3)
           		(-2, -0.1) -- +(0.15,-0.9) -- +(-0.15,-0.9) -- cycle
            	(2, -0.1) -- +(0.15,-0.9) -- +(-0.15,-0.9) -- cycle
            ;
            \draw[pattern={Lines[angle=51,distance=2pt]},pattern color=black,draw=none]
            	(-2.15, -1.15) rectangle +(0.3, 0.15)
            	(2.15, -1.15) rectangle +(-0.3, 0.15)
            ;
            \node [right] (m_small) at (-0.3, -1.15) { $m$ };
            \node [above] (M_big) at (0, 0.1) { $M$ };
        \end{tikzpicture}
}
\solutionspace{80pt}

\tasknumber{10}%
\task{%
    Тонкий однородный кусок арматуры длиной $1\,\text{м}$ и массой $30\,\text{кг}$ лежит на горизонтальной поверхности.
    \begin{itemize}
        \item Какую минимальную силу надо приложить к одному из его концов, чтобы оторвать его от этой поверхности?
        \item Какую минимальную работу надо совершить, чтобы поставить его на землю в вертикальное положение?
    \end{itemize}
    % Примите $g = 10\,\frac{\text{м}}{\text{с}^{2}}$.
}
\answer{%
    $A = mg\frac l2 = 150\,\text{Дж}$
}
\solutionspace{120pt}

\tasknumber{11}%
\task{%
    Определите работу силы, которая обеспечит спуск тела массой $5\,\text{кг}$ на высоту $10\,\text{м}$ с постоянным ускорением $3\,\frac{\text{м}}{\text{c}^{2}}$.
    % Примите $g = 10\,\frac{\text{м}}{\text{с}^{2}}$.
}
\answer{%
    \begin{align*}
    &\text{Для подъёма:} A = Fh = (mg + ma) h = m(g+a)h, \\
    &\text{Для спуска:} A = -Fh = -(mg - ma) h = -m(g-a)h, \\
    &\text{В результате получаем:} -350\,\text{Дж}.
    \end{align*}
}
\solutionspace{60pt}

\tasknumber{12}%
\task{%
    Тело бросили вертикально вверх со скоростью $10\,\frac{\text{м}}{\text{c}}$.
    На какой высоте кинетическая энергия тела составит треть от потенциальной?
}
\solutionspace{100pt}

\tasknumber{13}%
\task{%
    Плотность воздуха при нормальных условиях равна $1{,}3\,\frac{\text{кг}}{\text{м}^{3}}$.
    Чему равна плотность воздуха
    при температуре $200\celsius$ и давлении $50\,\text{кПа}$?
}
\solutionspace{120pt}

\tasknumber{14}%
\task{%
    Небольшую цилиндрическую пробирку с воздухом погружают на некоторую глубину в глубокое пресное озеро,
    после чего воздух занимает в ней лишь третью часть от общего объема.
    Определите глубину, на которую погрузили пробирку.
    Температуру считать постоянной $T = 292\,\text{К}$, давлением паров воды пренебречь,
    атмосферное давление принять равным $p_{\text{aтм}} = 100\,\text{кПа}$.
}
\answer{%
    \begin{align*}
    T\text{— const} &\implies P_1V_1 = \nu RT = P_2V_2.
    \\
    V_2 = \frac 13 V_1 &\implies P_1V_1 = P_2 \cdot \frac 13V_1 \implies P_2 = 3P_1 = 3p_{\text{aтм}}.
    \\
    P_2 = p_{\text{aтм}} + \rho_{\text{в}} g h \implies h = \frac{P_2 - p_{\text{aтм}}}{\rho_{\text{в}} g} &= \frac{3p_{\text{aтм}} - p_{\text{aтм}}}{\rho_{\text{в}} g} = \frac{2 \cdot p_{\text{aтм}}}{\rho_{\text{в}} g} =  \\
     &= \frac{2 \cdot 100\,\text{кПа}}{1000\,\frac{\text{кг}}{\text{м}^{3}} \cdot  10\,\frac{\text{м}}{\text{с}^{2}}} \approx 20\,\text{м}.
    \end{align*}
}
\solutionspace{120pt}

\tasknumber{15}%
\task{%
    Газу сообщили некоторое количество теплоты,
    при этом четверть его он потратил на совершение работы,
    одновременно увеличив свою внутреннюю энергию на $1200\,\text{Дж}$.
    Определите количество теплоты, сообщённое газу.
}
\answer{%
    \begin{align*}
    Q &= A' + \Delta U, A' = \frac 14 Q \implies Q \cdot \cbr{1 - \frac 14} = \Delta U \implies Q = \frac{\Delta U}{1 - \frac 14} = \frac{ 1200\,\text{Дж} }{1 - \frac 14} \approx 1600\,\text{Дж}.
    \\
    A' &= \frac 14 Q
        = \frac 14 \cdot \frac{\Delta U}{1 - \frac 14}
        = \frac{\Delta U}{4 - 1}
        = \frac{ 1200\,\text{Дж} }{4 - 1} \approx 400\,\text{Дж}.
    \end{align*}
}
\solutionspace{60pt}

\tasknumber{16}%
\task{%
    Два конденсатора ёмкостей $C_1 = 30\,\text{нФ}$ и $C_2 = 20\,\text{нФ}$ последовательно подключают
    к источнику напряжения $V = 450\,\text{В}$ (см.
    рис.).
    % Определите заряды каждого из конденсаторов.
    Определите заряд первого конденсатора.

    \begin{tikzpicture}[circuit ee IEC, semithick]
        \draw  (0, 0) to [capacitor={info={$C_1$}}] (1, 0)
                       to [capacitor={info={$C_2$}}] (2, 0)
        ;
        % \draw [-o] (0, 0) -- ++(-0.5, 0) node[left] {$-$};
        % \draw [-o] (2, 0) -- ++(0.5, 0) node[right] {$+$};
        \draw [-o] (0, 0) -- ++(-0.5, 0) node[left] {};
        \draw [-o] (2, 0) -- ++(0.5, 0) node[right] {};
    \end{tikzpicture}
}
\answer{%
    $
        Q_1
            = Q_2
            = CV
            = \frac{ V }{\frac1{C_1} + \frac1{C_2}}
            = \frac{C_1C_2V}{C_1 + C_2}
            = \frac{
                30\,\text{нФ} \cdot 20\,\text{нФ} \cdot 450\,\text{В}
            }{
                30\,\text{нФ} + 20\,\text{нФ}
            }
            = 5400{,}00\,\text{нКл}
    $
}
\solutionspace{120pt}

\tasknumber{17}%
\task{%
    В вакууме вдоль одной прямой расположены четыре отрицательных заряда так,
    что расстояние между соседними зарядами равно $l$.
    Сделайте рисунок,
    и определите силу, действующую на крайний заряд.
    Модули всех зарядов равны $Q$ ($Q > 0$).
}
\solutionspace{80pt}

\tasknumber{18}%
\task{%
    Юлия проводит эксперименты c 2 кусками одинаковой стальной проволки, причём второй кусок в пять раз длиннее первого.
    В одном из экспериментов Юлия подаёт на первый кусок проволки напряжение в восемь раз раз больше, чем на второй.
    Определите отношения в двух проволках в этом эксперименте (второй к первой):
    \begin{itemize}
        \item отношение сил тока,
        \item отношение выделяющихся мощностей.
    \end{itemize}
}
\answer{%
    $\eli_2 / \eli_1 = \frac1{40}, \P_2 / \P_1 = \frac1{40}, $
}

\variantsplitter

\addpersonalvariant{Михаил Ярошевский}

\tasknumber{1}%
\task{%
    Саша стартует на лошади и в течение $t = 10\,\text{c}$ двигается с постоянным ускорением $1{,}5\,\frac{\text{м}}{\text{с}^{2}}$.
    Определите
    \begin{itemize}
        \item какую скорость при этом удастся достичь,
        \item какой путь за это время будет пройден,
        \item среднюю скорость за всё время движения, если после начального ускорения продолжить движение равномерно ещё в течение времени $3t$
    \end{itemize}
}
\solutionspace{120pt}

\tasknumber{2}%
\task{%
    Какой путь тело пройдёт за четвёртую секунду после начала свободного падения?
    Какую скорость в конце этой секунды оно имеет?
}
\solutionspace{120pt}

\tasknumber{3}%
\task{%
    Карусель диаметром $3\,\text{м}$ равномерно совершает 10 оборотов в минуту.
    Определите
    \begin{itemize}
        \item период и частоту её обращения,
        \item скорость и ускорение крайних её точек.
    \end{itemize}
}
\solutionspace{80pt}

\tasknumber{4}%
\task{%
    Миша стоит на обрыве над рекой и методично и строго горизонтально кидает в неё камушки.
    За этим всем наблюдает экспериментатор Глюк, который уже выяснил, что камушки падают в реку спустя $1{,}6\,\text{с}$ после броска,
    а вот дальность полёта оценить сложнее: придётся лезть в воду.
    Выручите Глюка и определите:
    \begin{itemize}
        \item высоту обрыва (вместе с ростом Миши).
        \item дальность полёта камушков (по горизонтали) и их скорость при падении, приняв начальную скорость броска равной $v = 14\,\frac{\text{м}}{\text{с}}$.
    \end{itemize}
    Сопротивлением воздуха пренебречь.
}
\solutionspace{120pt}

\tasknumber{5}%
\task{%
    Шесть одинаковых брусков массой $2\,\text{кг}$ каждый лежат на гладком горизонтальном столе.
    Бруски пронумерованы от 1 до 6 и последовательно связаны между собой
    невесомыми нерастяжимыми нитями: 1 со 2, 2 с 3 (ну и с 1) и т.д.
    Экспериментатор Глюк прикладывает постоянную горизонтальную силу $90\,\text{Н}$ к бруску с наибольшим номером.
    С каким ускорением двигается система? Чему равна сила натяжения нити, связывающей бруски 3 и 4?
}
\solutionspace{120pt}

\tasknumber{6}%
\task{%
    Два бруска связаны лёгкой нерастяжимой нитью и перекинуты через неподвижный блок (см.
    рис.).
    Определите силу натяжения нити и ускорения брусков.
    Силами трения пренебречь, массы брусков
    равны $m_1 = 11\,\text{кг}$ и $m_2 = 10\,\text{кг}$.
    % $g = 10\,\frac{\text{м}}{\text{с}^{2}}$.

    \begin{tikzpicture}[x=1.5cm,y=1.5cm,thick]
        \draw
            (-0.4, 0) rectangle (-0.2, 1.2)
            (0.15, 0.5) rectangle (0.45, 1)
            (0, 2) circle [radius=0.3] -- ++(up:0.5)
            (-0.3, 1.2) -- ++(up:0.8)
            (0.3, 1) -- ++(up:1)
            (-0.7, 2.5) -- (0.7, 2.5)
            ;
        \draw[pattern={Lines[angle=51,distance=3pt]},pattern color=black,draw=none] (-0.7, 2.5) rectangle (0.7, 2.75);
        \node [left] (left) at (-0.4, 0.6) { $m_1$ };
        \node [right] (right) at (0.4, 0.75) { $m_2$ };
    \end{tikzpicture}
}
\solutionspace{80pt}

\tasknumber{7}%
\task{%
    Тело массой $2\,\text{кг}$ лежит на горизонтальной поверхности.
    Коэффициент трения между поверхностью и телом $0{,}15$.
    К телу приложена горизонтальная сила $5{,}5\,\text{Н}$.
    Определите силу трения, действующую на тело, и ускорение тела.
    % $g = 10\,\frac{\text{м}}{\text{с}^{2}}$.
}
\solutionspace{120pt}

\tasknumber{8}%
\task{%
    Определите плотность неизвестного вещества, если известно, что опускании тела из него
    в подсолнечное масло оно будет плавать и на треть выступать над поверхностью жидкости.
}
\solutionspace{120pt}

\tasknumber{9}%
\task{%
    	Определите силу, действующую на левую опору однородного горизонтального стержня длиной $l = 5\,\text{м}$
    	и массой $M = 1\,\text{кг}$, к которому подвешен груз массой $m = 2\,\text{кг}$ на расстоянии $2\,\text{м}$ от правого конца (см.
    рис.).

        \begin{tikzpicture}[thick]
            \draw
                (-2, -0.1) rectangle (2, 0.1)
                (-0.5, -0.1) -- (-0.5, -1)
                (-0.7, -1) rectangle (-0.3, -1.3)
           		(-2, -0.1) -- +(0.15,-0.9) -- +(-0.15,-0.9) -- cycle
            	(2, -0.1) -- +(0.15,-0.9) -- +(-0.15,-0.9) -- cycle
            ;
            \draw[pattern={Lines[angle=51,distance=2pt]},pattern color=black,draw=none]
            	(-2.15, -1.15) rectangle +(0.3, 0.15)
            	(2.15, -1.15) rectangle +(-0.3, 0.15)
            ;
            \node [right] (m_small) at (-0.3, -1.15) { $m$ };
            \node [above] (M_big) at (0, 0.1) { $M$ };
        \end{tikzpicture}
}
\solutionspace{80pt}

\tasknumber{10}%
\task{%
    Тонкий однородный лом длиной $1\,\text{м}$ и массой $30\,\text{кг}$ лежит на горизонтальной поверхности.
    \begin{itemize}
        \item Какую минимальную силу надо приложить к одному из его концов, чтобы оторвать его от этой поверхности?
        \item Какую минимальную работу надо совершить, чтобы поставить его на землю в вертикальное положение?
    \end{itemize}
    % Примите $g = 10\,\frac{\text{м}}{\text{с}^{2}}$.
}
\answer{%
    $A = mg\frac l2 = 150\,\text{Дж}$
}
\solutionspace{120pt}

\tasknumber{11}%
\task{%
    Определите работу силы, которая обеспечит спуск тела массой $3\,\text{кг}$ на высоту $10\,\text{м}$ с постоянным ускорением $6\,\frac{\text{м}}{\text{c}^{2}}$.
    % Примите $g = 10\,\frac{\text{м}}{\text{с}^{2}}$.
}
\answer{%
    \begin{align*}
    &\text{Для подъёма:} A = Fh = (mg + ma) h = m(g+a)h, \\
    &\text{Для спуска:} A = -Fh = -(mg - ma) h = -m(g-a)h, \\
    &\text{В результате получаем:} -120\,\text{Дж}.
    \end{align*}
}
\solutionspace{60pt}

\tasknumber{12}%
\task{%
    Тело бросили вертикально вверх со скоростью $14\,\frac{\text{м}}{\text{c}}$.
    На какой высоте кинетическая энергия тела составит половину от потенциальной?
}
\solutionspace{100pt}

\tasknumber{13}%
\task{%
    Плотность воздуха при нормальных условиях равна $1{,}3\,\frac{\text{кг}}{\text{м}^{3}}$.
    Чему равна плотность воздуха
    при температуре $100\celsius$ и давлении $50\,\text{кПа}$?
}
\solutionspace{120pt}

\tasknumber{14}%
\task{%
    Небольшую цилиндрическую пробирку с воздухом погружают на некоторую глубину в глубокое пресное озеро,
    после чего воздух занимает в ней лишь четвертую часть от общего объема.
    Определите глубину, на которую погрузили пробирку.
    Температуру считать постоянной $T = 287\,\text{К}$, давлением паров воды пренебречь,
    атмосферное давление принять равным $p_{\text{aтм}} = 100\,\text{кПа}$.
}
\answer{%
    \begin{align*}
    T\text{— const} &\implies P_1V_1 = \nu RT = P_2V_2.
    \\
    V_2 = \frac 14 V_1 &\implies P_1V_1 = P_2 \cdot \frac 14V_1 \implies P_2 = 4P_1 = 4p_{\text{aтм}}.
    \\
    P_2 = p_{\text{aтм}} + \rho_{\text{в}} g h \implies h = \frac{P_2 - p_{\text{aтм}}}{\rho_{\text{в}} g} &= \frac{4p_{\text{aтм}} - p_{\text{aтм}}}{\rho_{\text{в}} g} = \frac{3 \cdot p_{\text{aтм}}}{\rho_{\text{в}} g} =  \\
     &= \frac{3 \cdot 100\,\text{кПа}}{1000\,\frac{\text{кг}}{\text{м}^{3}} \cdot  10\,\frac{\text{м}}{\text{с}^{2}}} \approx 30\,\text{м}.
    \end{align*}
}
\solutionspace{120pt}

\tasknumber{15}%
\task{%
    Газу сообщили некоторое количество теплоты,
    при этом половину его он потратил на совершение работы,
    одновременно увеличив свою внутреннюю энергию на $1500\,\text{Дж}$.
    Определите работу, совершённую газом.
}
\answer{%
    \begin{align*}
    Q &= A' + \Delta U, A' = \frac 12 Q \implies Q \cdot \cbr{1 - \frac 12} = \Delta U \implies Q = \frac{\Delta U}{1 - \frac 12} = \frac{ 1500\,\text{Дж} }{1 - \frac 12} \approx 3000\,\text{Дж}.
    \\
    A' &= \frac 12 Q
        = \frac 12 \cdot \frac{\Delta U}{1 - \frac 12}
        = \frac{\Delta U}{2 - 1}
        = \frac{ 1500\,\text{Дж} }{2 - 1} \approx 1500\,\text{Дж}.
    \end{align*}
}
\solutionspace{60pt}

\tasknumber{16}%
\task{%
    Два конденсатора ёмкостей $C_1 = 20\,\text{нФ}$ и $C_2 = 30\,\text{нФ}$ последовательно подключают
    к источнику напряжения $V = 200\,\text{В}$ (см.
    рис.).
    % Определите заряды каждого из конденсаторов.
    Определите заряд второго конденсатора.

    \begin{tikzpicture}[circuit ee IEC, semithick]
        \draw  (0, 0) to [capacitor={info={$C_1$}}] (1, 0)
                       to [capacitor={info={$C_2$}}] (2, 0)
        ;
        % \draw [-o] (0, 0) -- ++(-0.5, 0) node[left] {$-$};
        % \draw [-o] (2, 0) -- ++(0.5, 0) node[right] {$+$};
        \draw [-o] (0, 0) -- ++(-0.5, 0) node[left] {};
        \draw [-o] (2, 0) -- ++(0.5, 0) node[right] {};
    \end{tikzpicture}
}
\answer{%
    $
        Q_1
            = Q_2
            = CV
            = \frac{ V }{\frac1{C_1} + \frac1{C_2}}
            = \frac{C_1C_2V}{C_1 + C_2}
            = \frac{
                20\,\text{нФ} \cdot 30\,\text{нФ} \cdot 200\,\text{В}
            }{
                20\,\text{нФ} + 30\,\text{нФ}
            }
            = 2400{,}00\,\text{нКл}
    $
}
\solutionspace{120pt}

\tasknumber{17}%
\task{%
    В вакууме вдоль одной прямой расположены три отрицательных заряда так,
    что расстояние между соседними зарядами равно $r$.
    Сделайте рисунок,
    и определите силу, действующую на крайний заряд.
    Модули всех зарядов равны $Q$ ($Q > 0$).
}
\solutionspace{80pt}

\tasknumber{18}%
\task{%
    Юлия проводит эксперименты c 2 кусками одинаковой стальной проволки, причём второй кусок в четыре раза длиннее первого.
    В одном из экспериментов Юлия подаёт на первый кусок проволки напряжение в три раза раз больше, чем на второй.
    Определите отношения в двух проволках в этом эксперименте (второй к первой):
    \begin{itemize}
        \item отношение сил тока,
        \item отношение выделяющихся мощностей.
    \end{itemize}
}
\answer{%
    $\eli_2 / \eli_1 = \frac1{12}, \P_2 / \P_1 = \frac1{12}, $
}

\variantsplitter

\addpersonalvariant{Алексей Алимпиев}

\tasknumber{1}%
\task{%
    Женя стартует на мотоцикле и в течение $t = 3\,\text{c}$ двигается с постоянным ускорением $2{,}5\,\frac{\text{м}}{\text{с}^{2}}$.
    Определите
    \begin{itemize}
        \item какую скорость при этом удастся достичь,
        \item какой путь за это время будет пройден,
        \item среднюю скорость за всё время движения, если после начального ускорения продолжить движение равномерно ещё в течение времени $3t$
    \end{itemize}
}
\solutionspace{120pt}

\tasknumber{2}%
\task{%
    Какой путь тело пройдёт за вторую секунду после начала свободного падения?
    Какую скорость в начале этой секунды оно имеет?
}
\solutionspace{120pt}

\tasknumber{3}%
\task{%
    Карусель диаметром $4\,\text{м}$ равномерно совершает 6 оборотов в минуту.
    Определите
    \begin{itemize}
        \item период и частоту её обращения,
        \item скорость и ускорение крайних её точек.
    \end{itemize}
}
\solutionspace{80pt}

\tasknumber{4}%
\task{%
    Даша стоит на обрыве над рекой и методично и строго горизонтально кидает в неё камушки.
    За этим всем наблюдает экспериментатор Глюк, который уже выяснил, что камушки падают в реку спустя $1{,}3\,\text{с}$ после броска,
    а вот дальность полёта оценить сложнее: придётся лезть в воду.
    Выручите Глюка и определите:
    \begin{itemize}
        \item высоту обрыва (вместе с ростом Даши).
        \item дальность полёта камушков (по горизонтали) и их скорость при падении, приняв начальную скорость броска равной $v = 13\,\frac{\text{м}}{\text{с}}$.
    \end{itemize}
    Сопротивлением воздуха пренебречь.
}
\solutionspace{120pt}

\tasknumber{5}%
\task{%
    Пять одинаковых брусков массой $2\,\text{кг}$ каждый лежат на гладком горизонтальном столе.
    Бруски пронумерованы от 1 до 5 и последовательно связаны между собой
    невесомыми нерастяжимыми нитями: 1 со 2, 2 с 3 (ну и с 1) и т.д.
    Экспериментатор Глюк прикладывает постоянную горизонтальную силу $90\,\text{Н}$ к бруску с наибольшим номером.
    С каким ускорением двигается система? Чему равна сила натяжения нити, связывающей бруски 3 и 4?
}
\solutionspace{120pt}

\tasknumber{6}%
\task{%
    Два бруска связаны лёгкой нерастяжимой нитью и перекинуты через неподвижный блок (см.
    рис.).
    Определите силу натяжения нити и ускорения брусков.
    Силами трения пренебречь, массы брусков
    равны $m_1 = 8\,\text{кг}$ и $m_2 = 6\,\text{кг}$.
    % $g = 10\,\frac{\text{м}}{\text{с}^{2}}$.

    \begin{tikzpicture}[x=1.5cm,y=1.5cm,thick]
        \draw
            (-0.4, 0) rectangle (-0.2, 1.2)
            (0.15, 0.5) rectangle (0.45, 1)
            (0, 2) circle [radius=0.3] -- ++(up:0.5)
            (-0.3, 1.2) -- ++(up:0.8)
            (0.3, 1) -- ++(up:1)
            (-0.7, 2.5) -- (0.7, 2.5)
            ;
        \draw[pattern={Lines[angle=51,distance=3pt]},pattern color=black,draw=none] (-0.7, 2.5) rectangle (0.7, 2.75);
        \node [left] (left) at (-0.4, 0.6) { $m_1$ };
        \node [right] (right) at (0.4, 0.75) { $m_2$ };
    \end{tikzpicture}
}
\solutionspace{80pt}

\tasknumber{7}%
\task{%
    Тело массой $2{,}7\,\text{кг}$ лежит на горизонтальной поверхности.
    Коэффициент трения между поверхностью и телом $0{,}15$.
    К телу приложена горизонтальная сила $5{,}5\,\text{Н}$.
    Определите силу трения, действующую на тело, и ускорение тела.
    % $g = 10\,\frac{\text{м}}{\text{с}^{2}}$.
}
\solutionspace{120pt}

\tasknumber{8}%
\task{%
    Определите плотность неизвестного вещества, если известно, что опускании тела из него
    в подсолнечное масло оно будет плавать и на четверть выступать над поверхностью жидкости.
}
\solutionspace{120pt}

\tasknumber{9}%
\task{%
    	Определите силу, действующую на левую опору однородного горизонтального стержня длиной $l = 9\,\text{м}$
    	и массой $M = 5\,\text{кг}$, к которому подвешен груз массой $m = 3\,\text{кг}$ на расстоянии $4\,\text{м}$ от правого конца (см.
    рис.).

        \begin{tikzpicture}[thick]
            \draw
                (-2, -0.1) rectangle (2, 0.1)
                (-0.5, -0.1) -- (-0.5, -1)
                (-0.7, -1) rectangle (-0.3, -1.3)
           		(-2, -0.1) -- +(0.15,-0.9) -- +(-0.15,-0.9) -- cycle
            	(2, -0.1) -- +(0.15,-0.9) -- +(-0.15,-0.9) -- cycle
            ;
            \draw[pattern={Lines[angle=51,distance=2pt]},pattern color=black,draw=none]
            	(-2.15, -1.15) rectangle +(0.3, 0.15)
            	(2.15, -1.15) rectangle +(-0.3, 0.15)
            ;
            \node [right] (m_small) at (-0.3, -1.15) { $m$ };
            \node [above] (M_big) at (0, 0.1) { $M$ };
        \end{tikzpicture}
}
\solutionspace{80pt}

\tasknumber{10}%
\task{%
    Тонкий однородный кусок арматуры длиной $1\,\text{м}$ и массой $10\,\text{кг}$ лежит на горизонтальной поверхности.
    \begin{itemize}
        \item Какую минимальную силу надо приложить к одному из его концов, чтобы оторвать его от этой поверхности?
        \item Какую минимальную работу надо совершить, чтобы поставить его на землю в вертикальное положение?
    \end{itemize}
    % Примите $g = 10\,\frac{\text{м}}{\text{с}^{2}}$.
}
\answer{%
    $A = mg\frac l2 = 50\,\text{Дж}$
}
\solutionspace{120pt}

\tasknumber{11}%
\task{%
    Определите работу силы, которая обеспечит подъём тела массой $2\,\text{кг}$ на высоту $2\,\text{м}$ с постоянным ускорением $4\,\frac{\text{м}}{\text{c}^{2}}$.
    % Примите $g = 10\,\frac{\text{м}}{\text{с}^{2}}$.
}
\answer{%
    \begin{align*}
    &\text{Для подъёма:} A = Fh = (mg + ma) h = m(g+a)h, \\
    &\text{Для спуска:} A = -Fh = -(mg - ma) h = -m(g-a)h, \\
    &\text{В результате получаем:} 56\,\text{Дж}.
    \end{align*}
}
\solutionspace{60pt}

\tasknumber{12}%
\task{%
    Тело бросили вертикально вверх со скоростью $14\,\frac{\text{м}}{\text{c}}$.
    На какой высоте кинетическая энергия тела составит треть от потенциальной?
}
\solutionspace{100pt}

\tasknumber{13}%
\task{%
    Плотность воздуха при нормальных условиях равна $1{,}3\,\frac{\text{кг}}{\text{м}^{3}}$.
    Чему равна плотность воздуха
    при температуре $100\celsius$ и давлении $80\,\text{кПа}$?
}
\solutionspace{120pt}

\tasknumber{14}%
\task{%
    Небольшую цилиндрическую пробирку с воздухом погружают на некоторую глубину в глубокое пресное озеро,
    после чего воздух занимает в ней лишь шестую часть от общего объема.
    Определите глубину, на которую погрузили пробирку.
    Температуру считать постоянной $T = 287\,\text{К}$, давлением паров воды пренебречь,
    атмосферное давление принять равным $p_{\text{aтм}} = 100\,\text{кПа}$.
}
\answer{%
    \begin{align*}
    T\text{— const} &\implies P_1V_1 = \nu RT = P_2V_2.
    \\
    V_2 = \frac 16 V_1 &\implies P_1V_1 = P_2 \cdot \frac 16V_1 \implies P_2 = 6P_1 = 6p_{\text{aтм}}.
    \\
    P_2 = p_{\text{aтм}} + \rho_{\text{в}} g h \implies h = \frac{P_2 - p_{\text{aтм}}}{\rho_{\text{в}} g} &= \frac{6p_{\text{aтм}} - p_{\text{aтм}}}{\rho_{\text{в}} g} = \frac{5 \cdot p_{\text{aтм}}}{\rho_{\text{в}} g} =  \\
     &= \frac{5 \cdot 100\,\text{кПа}}{1000\,\frac{\text{кг}}{\text{м}^{3}} \cdot  10\,\frac{\text{м}}{\text{с}^{2}}} \approx 50\,\text{м}.
    \end{align*}
}
\solutionspace{120pt}

\tasknumber{15}%
\task{%
    Газу сообщили некоторое количество теплоты,
    при этом четверть его он потратил на совершение работы,
    одновременно увеличив свою внутреннюю энергию на $1500\,\text{Дж}$.
    Определите количество теплоты, сообщённое газу.
}
\answer{%
    \begin{align*}
    Q &= A' + \Delta U, A' = \frac 14 Q \implies Q \cdot \cbr{1 - \frac 14} = \Delta U \implies Q = \frac{\Delta U}{1 - \frac 14} = \frac{ 1500\,\text{Дж} }{1 - \frac 14} \approx 2000\,\text{Дж}.
    \\
    A' &= \frac 14 Q
        = \frac 14 \cdot \frac{\Delta U}{1 - \frac 14}
        = \frac{\Delta U}{4 - 1}
        = \frac{ 1500\,\text{Дж} }{4 - 1} \approx 500\,\text{Дж}.
    \end{align*}
}
\solutionspace{60pt}

\tasknumber{16}%
\task{%
    Два конденсатора ёмкостей $C_1 = 60\,\text{нФ}$ и $C_2 = 30\,\text{нФ}$ последовательно подключают
    к источнику напряжения $U = 200\,\text{В}$ (см.
    рис.).
    % Определите заряды каждого из конденсаторов.
    Определите заряд первого конденсатора.

    \begin{tikzpicture}[circuit ee IEC, semithick]
        \draw  (0, 0) to [capacitor={info={$C_1$}}] (1, 0)
                       to [capacitor={info={$C_2$}}] (2, 0)
        ;
        % \draw [-o] (0, 0) -- ++(-0.5, 0) node[left] {$-$};
        % \draw [-o] (2, 0) -- ++(0.5, 0) node[right] {$+$};
        \draw [-o] (0, 0) -- ++(-0.5, 0) node[left] {};
        \draw [-o] (2, 0) -- ++(0.5, 0) node[right] {};
    \end{tikzpicture}
}
\answer{%
    $
        Q_1
            = Q_2
            = CU
            = \frac{ U }{\frac1{C_1} + \frac1{C_2}}
            = \frac{C_1C_2U}{C_1 + C_2}
            = \frac{
                60\,\text{нФ} \cdot 30\,\text{нФ} \cdot 200\,\text{В}
            }{
                60\,\text{нФ} + 30\,\text{нФ}
            }
            = 4000{,}00\,\text{нКл}
    $
}
\solutionspace{120pt}

\tasknumber{17}%
\task{%
    В вакууме вдоль одной прямой расположены три отрицательных заряда так,
    что расстояние между соседними зарядами равно $a$.
    Сделайте рисунок,
    и определите силу, действующую на крайний заряд.
    Модули всех зарядов равны $q$ ($q > 0$).
}
\solutionspace{80pt}

\tasknumber{18}%
\task{%
    Юлия проводит эксперименты c 2 кусками одинаковой алюминиевой проволки, причём второй кусок в восемь раз длиннее первого.
    В одном из экспериментов Юлия подаёт на первый кусок проволки напряжение в три раза раз больше, чем на второй.
    Определите отношения в двух проволках в этом эксперименте (второй к первой):
    \begin{itemize}
        \item отношение сил тока,
        \item отношение выделяющихся мощностей.
    \end{itemize}
}
\answer{%
    $\eli_2 / \eli_1 = \frac1{24}, \P_2 / \P_1 = \frac1{24}, $
}

\variantsplitter

\addpersonalvariant{Евгений Васин}

\tasknumber{1}%
\task{%
    Женя стартует на велосипеде и в течение $t = 5\,\text{c}$ двигается с постоянным ускорением $0{,}5\,\frac{\text{м}}{\text{с}^{2}}$.
    Определите
    \begin{itemize}
        \item какую скорость при этом удастся достичь,
        \item какой путь за это время будет пройден,
        \item среднюю скорость за всё время движения, если после начального ускорения продолжить движение равномерно ещё в течение времени $3t$
    \end{itemize}
}
\solutionspace{120pt}

\tasknumber{2}%
\task{%
    Какой путь тело пройдёт за шестую секунду после начала свободного падения?
    Какую скорость в начале этой секунды оно имеет?
}
\solutionspace{120pt}

\tasknumber{3}%
\task{%
    Карусель радиусом $5\,\text{м}$ равномерно совершает 6 оборотов в минуту.
    Определите
    \begin{itemize}
        \item период и частоту её обращения,
        \item скорость и ускорение крайних её точек.
    \end{itemize}
}
\solutionspace{80pt}

\tasknumber{4}%
\task{%
    Маша стоит на обрыве над рекой и методично и строго горизонтально кидает в неё камушки.
    За этим всем наблюдает экспериментатор Глюк, который уже выяснил, что камушки падают в реку спустя $1{,}7\,\text{с}$ после броска,
    а вот дальность полёта оценить сложнее: придётся лезть в воду.
    Выручите Глюка и определите:
    \begin{itemize}
        \item высоту обрыва (вместе с ростом Маши).
        \item дальность полёта камушков (по горизонтали) и их скорость при падении, приняв начальную скорость броска равной $v = 15\,\frac{\text{м}}{\text{с}}$.
    \end{itemize}
    Сопротивлением воздуха пренебречь.
}
\solutionspace{120pt}

\tasknumber{5}%
\task{%
    Шесть одинаковых брусков массой $3\,\text{кг}$ каждый лежат на гладком горизонтальном столе.
    Бруски пронумерованы от 1 до 6 и последовательно связаны между собой
    невесомыми нерастяжимыми нитями: 1 со 2, 2 с 3 (ну и с 1) и т.д.
    Экспериментатор Глюк прикладывает постоянную горизонтальную силу $120\,\text{Н}$ к бруску с наибольшим номером.
    С каким ускорением двигается система? Чему равна сила натяжения нити, связывающей бруски 1 и 2?
}
\solutionspace{120pt}

\tasknumber{6}%
\task{%
    Два бруска связаны лёгкой нерастяжимой нитью и перекинуты через неподвижный блок (см.
    рис.).
    Определите силу натяжения нити и ускорения брусков.
    Силами трения пренебречь, массы брусков
    равны $m_1 = 5\,\text{кг}$ и $m_2 = 14\,\text{кг}$.
    % $g = 10\,\frac{\text{м}}{\text{с}^{2}}$.

    \begin{tikzpicture}[x=1.5cm,y=1.5cm,thick]
        \draw
            (-0.4, 0) rectangle (-0.2, 1.2)
            (0.15, 0.5) rectangle (0.45, 1)
            (0, 2) circle [radius=0.3] -- ++(up:0.5)
            (-0.3, 1.2) -- ++(up:0.8)
            (0.3, 1) -- ++(up:1)
            (-0.7, 2.5) -- (0.7, 2.5)
            ;
        \draw[pattern={Lines[angle=51,distance=3pt]},pattern color=black,draw=none] (-0.7, 2.5) rectangle (0.7, 2.75);
        \node [left] (left) at (-0.4, 0.6) { $m_1$ };
        \node [right] (right) at (0.4, 0.75) { $m_2$ };
    \end{tikzpicture}
}
\solutionspace{80pt}

\tasknumber{7}%
\task{%
    Тело массой $1{,}4\,\text{кг}$ лежит на горизонтальной поверхности.
    Коэффициент трения между поверхностью и телом $0{,}25$.
    К телу приложена горизонтальная сила $2{,}5\,\text{Н}$.
    Определите силу трения, действующую на тело, и ускорение тела.
    % $g = 10\,\frac{\text{м}}{\text{с}^{2}}$.
}
\solutionspace{120pt}

\tasknumber{8}%
\task{%
    Определите плотность неизвестного вещества, если известно, что опускании тела из него
    в керосин оно будет плавать и на половину выступать над поверхностью жидкости.
}
\solutionspace{120pt}

\tasknumber{9}%
\task{%
    	Определите силу, действующую на левую опору однородного горизонтального стержня длиной $l = 5\,\text{м}$
    	и массой $M = 5\,\text{кг}$, к которому подвешен груз массой $m = 2\,\text{кг}$ на расстоянии $4\,\text{м}$ от правого конца (см.
    рис.).

        \begin{tikzpicture}[thick]
            \draw
                (-2, -0.1) rectangle (2, 0.1)
                (-0.5, -0.1) -- (-0.5, -1)
                (-0.7, -1) rectangle (-0.3, -1.3)
           		(-2, -0.1) -- +(0.15,-0.9) -- +(-0.15,-0.9) -- cycle
            	(2, -0.1) -- +(0.15,-0.9) -- +(-0.15,-0.9) -- cycle
            ;
            \draw[pattern={Lines[angle=51,distance=2pt]},pattern color=black,draw=none]
            	(-2.15, -1.15) rectangle +(0.3, 0.15)
            	(2.15, -1.15) rectangle +(-0.3, 0.15)
            ;
            \node [right] (m_small) at (-0.3, -1.15) { $m$ };
            \node [above] (M_big) at (0, 0.1) { $M$ };
        \end{tikzpicture}
}
\solutionspace{80pt}

\tasknumber{10}%
\task{%
    Тонкий однородный шест длиной $2\,\text{м}$ и массой $10\,\text{кг}$ лежит на горизонтальной поверхности.
    \begin{itemize}
        \item Какую минимальную силу надо приложить к одному из его концов, чтобы оторвать его от этой поверхности?
        \item Какую минимальную работу надо совершить, чтобы поставить его на землю в вертикальное положение?
    \end{itemize}
    % Примите $g = 10\,\frac{\text{м}}{\text{с}^{2}}$.
}
\answer{%
    $A = mg\frac l2 = 100\,\text{Дж}$
}
\solutionspace{120pt}

\tasknumber{11}%
\task{%
    Определите работу силы, которая обеспечит подъём тела массой $5\,\text{кг}$ на высоту $5\,\text{м}$ с постоянным ускорением $4\,\frac{\text{м}}{\text{c}^{2}}$.
    % Примите $g = 10\,\frac{\text{м}}{\text{с}^{2}}$.
}
\answer{%
    \begin{align*}
    &\text{Для подъёма:} A = Fh = (mg + ma) h = m(g+a)h, \\
    &\text{Для спуска:} A = -Fh = -(mg - ma) h = -m(g-a)h, \\
    &\text{В результате получаем:} 350\,\text{Дж}.
    \end{align*}
}
\solutionspace{60pt}

\tasknumber{12}%
\task{%
    Тело бросили вертикально вверх со скоростью $20\,\frac{\text{м}}{\text{c}}$.
    На какой высоте кинетическая энергия тела составит половину от потенциальной?
}
\solutionspace{100pt}

\tasknumber{13}%
\task{%
    Плотность воздуха при нормальных условиях равна $1{,}3\,\frac{\text{кг}}{\text{м}^{3}}$.
    Чему равна плотность воздуха
    при температуре $100\celsius$ и давлении $80\,\text{кПа}$?
}
\solutionspace{120pt}

\tasknumber{14}%
\task{%
    Небольшую цилиндрическую пробирку с воздухом погружают на некоторую глубину в глубокое пресное озеро,
    после чего воздух занимает в ней лишь третью часть от общего объема.
    Определите глубину, на которую погрузили пробирку.
    Температуру считать постоянной $T = 290\,\text{К}$, давлением паров воды пренебречь,
    атмосферное давление принять равным $p_{\text{aтм}} = 100\,\text{кПа}$.
}
\answer{%
    \begin{align*}
    T\text{— const} &\implies P_1V_1 = \nu RT = P_2V_2.
    \\
    V_2 = \frac 13 V_1 &\implies P_1V_1 = P_2 \cdot \frac 13V_1 \implies P_2 = 3P_1 = 3p_{\text{aтм}}.
    \\
    P_2 = p_{\text{aтм}} + \rho_{\text{в}} g h \implies h = \frac{P_2 - p_{\text{aтм}}}{\rho_{\text{в}} g} &= \frac{3p_{\text{aтм}} - p_{\text{aтм}}}{\rho_{\text{в}} g} = \frac{2 \cdot p_{\text{aтм}}}{\rho_{\text{в}} g} =  \\
     &= \frac{2 \cdot 100\,\text{кПа}}{1000\,\frac{\text{кг}}{\text{м}^{3}} \cdot  10\,\frac{\text{м}}{\text{с}^{2}}} \approx 20\,\text{м}.
    \end{align*}
}
\solutionspace{120pt}

\tasknumber{15}%
\task{%
    Газу сообщили некоторое количество теплоты,
    при этом половину его он потратил на совершение работы,
    одновременно увеличив свою внутреннюю энергию на $1500\,\text{Дж}$.
    Определите работу, совершённую газом.
}
\answer{%
    \begin{align*}
    Q &= A' + \Delta U, A' = \frac 12 Q \implies Q \cdot \cbr{1 - \frac 12} = \Delta U \implies Q = \frac{\Delta U}{1 - \frac 12} = \frac{ 1500\,\text{Дж} }{1 - \frac 12} \approx 3000\,\text{Дж}.
    \\
    A' &= \frac 12 Q
        = \frac 12 \cdot \frac{\Delta U}{1 - \frac 12}
        = \frac{\Delta U}{2 - 1}
        = \frac{ 1500\,\text{Дж} }{2 - 1} \approx 1500\,\text{Дж}.
    \end{align*}
}
\solutionspace{60pt}

\tasknumber{16}%
\task{%
    Два конденсатора ёмкостей $C_1 = 60\,\text{нФ}$ и $C_2 = 20\,\text{нФ}$ последовательно подключают
    к источнику напряжения $U = 450\,\text{В}$ (см.
    рис.).
    % Определите заряды каждого из конденсаторов.
    Определите заряд второго конденсатора.

    \begin{tikzpicture}[circuit ee IEC, semithick]
        \draw  (0, 0) to [capacitor={info={$C_1$}}] (1, 0)
                       to [capacitor={info={$C_2$}}] (2, 0)
        ;
        % \draw [-o] (0, 0) -- ++(-0.5, 0) node[left] {$-$};
        % \draw [-o] (2, 0) -- ++(0.5, 0) node[right] {$+$};
        \draw [-o] (0, 0) -- ++(-0.5, 0) node[left] {};
        \draw [-o] (2, 0) -- ++(0.5, 0) node[right] {};
    \end{tikzpicture}
}
\answer{%
    $
        Q_1
            = Q_2
            = CU
            = \frac{ U }{\frac1{C_1} + \frac1{C_2}}
            = \frac{C_1C_2U}{C_1 + C_2}
            = \frac{
                60\,\text{нФ} \cdot 20\,\text{нФ} \cdot 450\,\text{В}
            }{
                60\,\text{нФ} + 20\,\text{нФ}
            }
            = 6750{,}00\,\text{нКл}
    $
}
\solutionspace{120pt}

\tasknumber{17}%
\task{%
    В вакууме вдоль одной прямой расположены три положительных заряда так,
    что расстояние между соседними зарядами равно $l$.
    Сделайте рисунок,
    и определите силу, действующую на крайний заряд.
    Модули всех зарядов равны $Q$ ($Q > 0$).
}
\solutionspace{80pt}

\tasknumber{18}%
\task{%
    Юлия проводит эксперименты c 2 кусками одинаковой стальной проволки, причём второй кусок в десять раз длиннее первого.
    В одном из экспериментов Юлия подаёт на первый кусок проволки напряжение в два раза раз больше, чем на второй.
    Определите отношения в двух проволках в этом эксперименте (второй к первой):
    \begin{itemize}
        \item отношение сил тока,
        \item отношение выделяющихся мощностей.
    \end{itemize}
}
\answer{%
    $\eli_2 / \eli_1 = \frac1{20}, \P_2 / \P_1 = \frac1{20}, $
}

\variantsplitter

\addpersonalvariant{Вячеслав Волохов}

\tasknumber{1}%
\task{%
    Саша стартует на лошади и в течение $t = 2\,\text{c}$ двигается с постоянным ускорением $1{,}5\,\frac{\text{м}}{\text{с}^{2}}$.
    Определите
    \begin{itemize}
        \item какую скорость при этом удастся достичь,
        \item какой путь за это время будет пройден,
        \item среднюю скорость за всё время движения, если после начального ускорения продолжить движение равномерно ещё в течение времени $3t$
    \end{itemize}
}
\solutionspace{120pt}

\tasknumber{2}%
\task{%
    Какой путь тело пройдёт за пятую секунду после начала свободного падения?
    Какую скорость в конце этой секунды оно имеет?
}
\solutionspace{120pt}

\tasknumber{3}%
\task{%
    Карусель радиусом $3\,\text{м}$ равномерно совершает 6 оборотов в минуту.
    Определите
    \begin{itemize}
        \item период и частоту её обращения,
        \item скорость и ускорение крайних её точек.
    \end{itemize}
}
\solutionspace{80pt}

\tasknumber{4}%
\task{%
    Миша стоит на обрыве над рекой и методично и строго горизонтально кидает в неё камушки.
    За этим всем наблюдает экспериментатор Глюк, который уже выяснил, что камушки падают в реку спустя $1{,}2\,\text{с}$ после броска,
    а вот дальность полёта оценить сложнее: придётся лезть в воду.
    Выручите Глюка и определите:
    \begin{itemize}
        \item высоту обрыва (вместе с ростом Миши).
        \item дальность полёта камушков (по горизонтали) и их скорость при падении, приняв начальную скорость броска равной $v = 12\,\frac{\text{м}}{\text{с}}$.
    \end{itemize}
    Сопротивлением воздуха пренебречь.
}
\solutionspace{120pt}

\tasknumber{5}%
\task{%
    Четыре одинаковых брусков массой $3\,\text{кг}$ каждый лежат на гладком горизонтальном столе.
    Бруски пронумерованы от 1 до 4 и последовательно связаны между собой
    невесомыми нерастяжимыми нитями: 1 со 2, 2 с 3 (ну и с 1) и т.д.
    Экспериментатор Глюк прикладывает постоянную горизонтальную силу $90\,\text{Н}$ к бруску с наименьшим номером.
    С каким ускорением двигается система? Чему равна сила натяжения нити, связывающей бруски 1 и 2?
}
\solutionspace{120pt}

\tasknumber{6}%
\task{%
    Два бруска связаны лёгкой нерастяжимой нитью и перекинуты через неподвижный блок (см.
    рис.).
    Определите силу натяжения нити и ускорения брусков.
    Силами трения пренебречь, массы брусков
    равны $m_1 = 5\,\text{кг}$ и $m_2 = 6\,\text{кг}$.
    % $g = 10\,\frac{\text{м}}{\text{с}^{2}}$.

    \begin{tikzpicture}[x=1.5cm,y=1.5cm,thick]
        \draw
            (-0.4, 0) rectangle (-0.2, 1.2)
            (0.15, 0.5) rectangle (0.45, 1)
            (0, 2) circle [radius=0.3] -- ++(up:0.5)
            (-0.3, 1.2) -- ++(up:0.8)
            (0.3, 1) -- ++(up:1)
            (-0.7, 2.5) -- (0.7, 2.5)
            ;
        \draw[pattern={Lines[angle=51,distance=3pt]},pattern color=black,draw=none] (-0.7, 2.5) rectangle (0.7, 2.75);
        \node [left] (left) at (-0.4, 0.6) { $m_1$ };
        \node [right] (right) at (0.4, 0.75) { $m_2$ };
    \end{tikzpicture}
}
\solutionspace{80pt}

\tasknumber{7}%
\task{%
    Тело массой $1{,}4\,\text{кг}$ лежит на горизонтальной поверхности.
    Коэффициент трения между поверхностью и телом $0{,}25$.
    К телу приложена горизонтальная сила $4{,}5\,\text{Н}$.
    Определите силу трения, действующую на тело, и ускорение тела.
    % $g = 10\,\frac{\text{м}}{\text{с}^{2}}$.
}
\solutionspace{120pt}

\tasknumber{8}%
\task{%
    Определите плотность неизвестного вещества, если известно, что опускании тела из него
    в подсолнечное масло оно будет плавать и на треть выступать над поверхностью жидкости.
}
\solutionspace{120pt}

\tasknumber{9}%
\task{%
    	Определите силу, действующую на правую опору однородного горизонтального стержня длиной $l = 9\,\text{м}$
    	и массой $M = 1\,\text{кг}$, к которому подвешен груз массой $m = 3\,\text{кг}$ на расстоянии $4\,\text{м}$ от правого конца (см.
    рис.).

        \begin{tikzpicture}[thick]
            \draw
                (-2, -0.1) rectangle (2, 0.1)
                (-0.5, -0.1) -- (-0.5, -1)
                (-0.7, -1) rectangle (-0.3, -1.3)
           		(-2, -0.1) -- +(0.15,-0.9) -- +(-0.15,-0.9) -- cycle
            	(2, -0.1) -- +(0.15,-0.9) -- +(-0.15,-0.9) -- cycle
            ;
            \draw[pattern={Lines[angle=51,distance=2pt]},pattern color=black,draw=none]
            	(-2.15, -1.15) rectangle +(0.3, 0.15)
            	(2.15, -1.15) rectangle +(-0.3, 0.15)
            ;
            \node [right] (m_small) at (-0.3, -1.15) { $m$ };
            \node [above] (M_big) at (0, 0.1) { $M$ };
        \end{tikzpicture}
}
\solutionspace{80pt}

\tasknumber{10}%
\task{%
    Тонкий однородный лом длиной $3\,\text{м}$ и массой $10\,\text{кг}$ лежит на горизонтальной поверхности.
    \begin{itemize}
        \item Какую минимальную силу надо приложить к одному из его концов, чтобы оторвать его от этой поверхности?
        \item Какую минимальную работу надо совершить, чтобы поставить его на землю в вертикальное положение?
    \end{itemize}
    % Примите $g = 10\,\frac{\text{м}}{\text{с}^{2}}$.
}
\answer{%
    $A = mg\frac l2 = 150\,\text{Дж}$
}
\solutionspace{120pt}

\tasknumber{11}%
\task{%
    Определите работу силы, которая обеспечит подъём тела массой $5\,\text{кг}$ на высоту $10\,\text{м}$ с постоянным ускорением $2\,\frac{\text{м}}{\text{c}^{2}}$.
    % Примите $g = 10\,\frac{\text{м}}{\text{с}^{2}}$.
}
\answer{%
    \begin{align*}
    &\text{Для подъёма:} A = Fh = (mg + ma) h = m(g+a)h, \\
    &\text{Для спуска:} A = -Fh = -(mg - ma) h = -m(g-a)h, \\
    &\text{В результате получаем:} 600\,\text{Дж}.
    \end{align*}
}
\solutionspace{60pt}

\tasknumber{12}%
\task{%
    Тело бросили вертикально вверх со скоростью $10\,\frac{\text{м}}{\text{c}}$.
    На какой высоте кинетическая энергия тела составит треть от потенциальной?
}
\solutionspace{100pt}

\tasknumber{13}%
\task{%
    Плотность воздуха при нормальных условиях равна $1{,}3\,\frac{\text{кг}}{\text{м}^{3}}$.
    Чему равна плотность воздуха
    при температуре $150\celsius$ и давлении $120\,\text{кПа}$?
}
\solutionspace{120pt}

\tasknumber{14}%
\task{%
    Небольшую цилиндрическую пробирку с воздухом погружают на некоторую глубину в глубокое пресное озеро,
    после чего воздух занимает в ней лишь четвертую часть от общего объема.
    Определите глубину, на которую погрузили пробирку.
    Температуру считать постоянной $T = 289\,\text{К}$, давлением паров воды пренебречь,
    атмосферное давление принять равным $p_{\text{aтм}} = 100\,\text{кПа}$.
}
\answer{%
    \begin{align*}
    T\text{— const} &\implies P_1V_1 = \nu RT = P_2V_2.
    \\
    V_2 = \frac 14 V_1 &\implies P_1V_1 = P_2 \cdot \frac 14V_1 \implies P_2 = 4P_1 = 4p_{\text{aтм}}.
    \\
    P_2 = p_{\text{aтм}} + \rho_{\text{в}} g h \implies h = \frac{P_2 - p_{\text{aтм}}}{\rho_{\text{в}} g} &= \frac{4p_{\text{aтм}} - p_{\text{aтм}}}{\rho_{\text{в}} g} = \frac{3 \cdot p_{\text{aтм}}}{\rho_{\text{в}} g} =  \\
     &= \frac{3 \cdot 100\,\text{кПа}}{1000\,\frac{\text{кг}}{\text{м}^{3}} \cdot  10\,\frac{\text{м}}{\text{с}^{2}}} \approx 30\,\text{м}.
    \end{align*}
}
\solutionspace{120pt}

\tasknumber{15}%
\task{%
    Газу сообщили некоторое количество теплоты,
    при этом четверть его он потратил на совершение работы,
    одновременно увеличив свою внутреннюю энергию на $1500\,\text{Дж}$.
    Определите количество теплоты, сообщённое газу.
}
\answer{%
    \begin{align*}
    Q &= A' + \Delta U, A' = \frac 14 Q \implies Q \cdot \cbr{1 - \frac 14} = \Delta U \implies Q = \frac{\Delta U}{1 - \frac 14} = \frac{ 1500\,\text{Дж} }{1 - \frac 14} \approx 2000\,\text{Дж}.
    \\
    A' &= \frac 14 Q
        = \frac 14 \cdot \frac{\Delta U}{1 - \frac 14}
        = \frac{\Delta U}{4 - 1}
        = \frac{ 1500\,\text{Дж} }{4 - 1} \approx 500\,\text{Дж}.
    \end{align*}
}
\solutionspace{60pt}

\tasknumber{16}%
\task{%
    Два конденсатора ёмкостей $C_1 = 20\,\text{нФ}$ и $C_2 = 30\,\text{нФ}$ последовательно подключают
    к источнику напряжения $U = 200\,\text{В}$ (см.
    рис.).
    % Определите заряды каждого из конденсаторов.
    Определите заряд второго конденсатора.

    \begin{tikzpicture}[circuit ee IEC, semithick]
        \draw  (0, 0) to [capacitor={info={$C_1$}}] (1, 0)
                       to [capacitor={info={$C_2$}}] (2, 0)
        ;
        % \draw [-o] (0, 0) -- ++(-0.5, 0) node[left] {$-$};
        % \draw [-o] (2, 0) -- ++(0.5, 0) node[right] {$+$};
        \draw [-o] (0, 0) -- ++(-0.5, 0) node[left] {};
        \draw [-o] (2, 0) -- ++(0.5, 0) node[right] {};
    \end{tikzpicture}
}
\answer{%
    $
        Q_1
            = Q_2
            = CU
            = \frac{ U }{\frac1{C_1} + \frac1{C_2}}
            = \frac{C_1C_2U}{C_1 + C_2}
            = \frac{
                20\,\text{нФ} \cdot 30\,\text{нФ} \cdot 200\,\text{В}
            }{
                20\,\text{нФ} + 30\,\text{нФ}
            }
            = 2400{,}00\,\text{нКл}
    $
}
\solutionspace{120pt}

\tasknumber{17}%
\task{%
    В вакууме вдоль одной прямой расположены четыре положительных заряда так,
    что расстояние между соседними зарядами равно $d$.
    Сделайте рисунок,
    и определите силу, действующую на крайний заряд.
    Модули всех зарядов равны $Q$ ($Q > 0$).
}
\solutionspace{80pt}

\tasknumber{18}%
\task{%
    Юлия проводит эксперименты c 2 кусками одинаковой стальной проволки, причём второй кусок в девять раз длиннее первого.
    В одном из экспериментов Юлия подаёт на первый кусок проволки напряжение в четыре раза раз больше, чем на второй.
    Определите отношения в двух проволках в этом эксперименте (второй к первой):
    \begin{itemize}
        \item отношение сил тока,
        \item отношение выделяющихся мощностей.
    \end{itemize}
}
\answer{%
    $\eli_2 / \eli_1 = \frac1{36}, \P_2 / \P_1 = \frac1{36}, $
}

\variantsplitter

\addpersonalvariant{Герман Говоров}

\tasknumber{1}%
\task{%
    Женя стартует на лошади и в течение $t = 3\,\text{c}$ двигается с постоянным ускорением $1{,}5\,\frac{\text{м}}{\text{с}^{2}}$.
    Определите
    \begin{itemize}
        \item какую скорость при этом удастся достичь,
        \item какой путь за это время будет пройден,
        \item среднюю скорость за всё время движения, если после начального ускорения продолжить движение равномерно ещё в течение времени $2t$
    \end{itemize}
}
\solutionspace{120pt}

\tasknumber{2}%
\task{%
    Какой путь тело пройдёт за шестую секунду после начала свободного падения?
    Какую скорость в начале этой секунды оно имеет?
}
\solutionspace{120pt}

\tasknumber{3}%
\task{%
    Карусель диаметром $3\,\text{м}$ равномерно совершает 5 оборотов в минуту.
    Определите
    \begin{itemize}
        \item период и частоту её обращения,
        \item скорость и ускорение крайних её точек.
    \end{itemize}
}
\solutionspace{80pt}

\tasknumber{4}%
\task{%
    Маша стоит на обрыве над рекой и методично и строго горизонтально кидает в неё камушки.
    За этим всем наблюдает экспериментатор Глюк, который уже выяснил, что камушки падают в реку спустя $1{,}7\,\text{с}$ после броска,
    а вот дальность полёта оценить сложнее: придётся лезть в воду.
    Выручите Глюка и определите:
    \begin{itemize}
        \item высоту обрыва (вместе с ростом Маши).
        \item дальность полёта камушков (по горизонтали) и их скорость при падении, приняв начальную скорость броска равной $v = 14\,\frac{\text{м}}{\text{с}}$.
    \end{itemize}
    Сопротивлением воздуха пренебречь.
}
\solutionspace{120pt}

\tasknumber{5}%
\task{%
    Четыре одинаковых брусков массой $2\,\text{кг}$ каждый лежат на гладком горизонтальном столе.
    Бруски пронумерованы от 1 до 4 и последовательно связаны между собой
    невесомыми нерастяжимыми нитями: 1 со 2, 2 с 3 (ну и с 1) и т.д.
    Экспериментатор Глюк прикладывает постоянную горизонтальную силу $90\,\text{Н}$ к бруску с наибольшим номером.
    С каким ускорением двигается система? Чему равна сила натяжения нити, связывающей бруски 3 и 4?
}
\solutionspace{120pt}

\tasknumber{6}%
\task{%
    Два бруска связаны лёгкой нерастяжимой нитью и перекинуты через неподвижный блок (см.
    рис.).
    Определите силу натяжения нити и ускорения брусков.
    Силами трения пренебречь, массы брусков
    равны $m_1 = 5\,\text{кг}$ и $m_2 = 6\,\text{кг}$.
    % $g = 10\,\frac{\text{м}}{\text{с}^{2}}$.

    \begin{tikzpicture}[x=1.5cm,y=1.5cm,thick]
        \draw
            (-0.4, 0) rectangle (-0.2, 1.2)
            (0.15, 0.5) rectangle (0.45, 1)
            (0, 2) circle [radius=0.3] -- ++(up:0.5)
            (-0.3, 1.2) -- ++(up:0.8)
            (0.3, 1) -- ++(up:1)
            (-0.7, 2.5) -- (0.7, 2.5)
            ;
        \draw[pattern={Lines[angle=51,distance=3pt]},pattern color=black,draw=none] (-0.7, 2.5) rectangle (0.7, 2.75);
        \node [left] (left) at (-0.4, 0.6) { $m_1$ };
        \node [right] (right) at (0.4, 0.75) { $m_2$ };
    \end{tikzpicture}
}
\solutionspace{80pt}

\tasknumber{7}%
\task{%
    Тело массой $2{,}7\,\text{кг}$ лежит на горизонтальной поверхности.
    Коэффициент трения между поверхностью и телом $0{,}15$.
    К телу приложена горизонтальная сила $5{,}5\,\text{Н}$.
    Определите силу трения, действующую на тело, и ускорение тела.
    % $g = 10\,\frac{\text{м}}{\text{с}^{2}}$.
}
\solutionspace{120pt}

\tasknumber{8}%
\task{%
    Определите плотность неизвестного вещества, если известно, что опускании тела из него
    в подсолнечное масло оно будет плавать и на треть выступать над поверхностью жидкости.
}
\solutionspace{120pt}

\tasknumber{9}%
\task{%
    	Определите силу, действующую на правую опору однородного горизонтального стержня длиной $l = 9\,\text{м}$
    	и массой $M = 1\,\text{кг}$, к которому подвешен груз массой $m = 4\,\text{кг}$ на расстоянии $4\,\text{м}$ от правого конца (см.
    рис.).

        \begin{tikzpicture}[thick]
            \draw
                (-2, -0.1) rectangle (2, 0.1)
                (-0.5, -0.1) -- (-0.5, -1)
                (-0.7, -1) rectangle (-0.3, -1.3)
           		(-2, -0.1) -- +(0.15,-0.9) -- +(-0.15,-0.9) -- cycle
            	(2, -0.1) -- +(0.15,-0.9) -- +(-0.15,-0.9) -- cycle
            ;
            \draw[pattern={Lines[angle=51,distance=2pt]},pattern color=black,draw=none]
            	(-2.15, -1.15) rectangle +(0.3, 0.15)
            	(2.15, -1.15) rectangle +(-0.3, 0.15)
            ;
            \node [right] (m_small) at (-0.3, -1.15) { $m$ };
            \node [above] (M_big) at (0, 0.1) { $M$ };
        \end{tikzpicture}
}
\solutionspace{80pt}

\tasknumber{10}%
\task{%
    Тонкий однородный шест длиной $2\,\text{м}$ и массой $30\,\text{кг}$ лежит на горизонтальной поверхности.
    \begin{itemize}
        \item Какую минимальную силу надо приложить к одному из его концов, чтобы оторвать его от этой поверхности?
        \item Какую минимальную работу надо совершить, чтобы поставить его на землю в вертикальное положение?
    \end{itemize}
    % Примите $g = 10\,\frac{\text{м}}{\text{с}^{2}}$.
}
\answer{%
    $A = mg\frac l2 = 300\,\text{Дж}$
}
\solutionspace{120pt}

\tasknumber{11}%
\task{%
    Определите работу силы, которая обеспечит подъём тела массой $5\,\text{кг}$ на высоту $10\,\text{м}$ с постоянным ускорением $4\,\frac{\text{м}}{\text{c}^{2}}$.
    % Примите $g = 10\,\frac{\text{м}}{\text{с}^{2}}$.
}
\answer{%
    \begin{align*}
    &\text{Для подъёма:} A = Fh = (mg + ma) h = m(g+a)h, \\
    &\text{Для спуска:} A = -Fh = -(mg - ma) h = -m(g-a)h, \\
    &\text{В результате получаем:} 700\,\text{Дж}.
    \end{align*}
}
\solutionspace{60pt}

\tasknumber{12}%
\task{%
    Тело бросили вертикально вверх со скоростью $20\,\frac{\text{м}}{\text{c}}$.
    На какой высоте кинетическая энергия тела составит половину от потенциальной?
}
\solutionspace{100pt}

\tasknumber{13}%
\task{%
    Плотность воздуха при нормальных условиях равна $1{,}3\,\frac{\text{кг}}{\text{м}^{3}}$.
    Чему равна плотность воздуха
    при температуре $150\celsius$ и давлении $120\,\text{кПа}$?
}
\solutionspace{120pt}

\tasknumber{14}%
\task{%
    Небольшую цилиндрическую пробирку с воздухом погружают на некоторую глубину в глубокое пресное озеро,
    после чего воздух занимает в ней лишь третью часть от общего объема.
    Определите глубину, на которую погрузили пробирку.
    Температуру считать постоянной $T = 280\,\text{К}$, давлением паров воды пренебречь,
    атмосферное давление принять равным $p_{\text{aтм}} = 100\,\text{кПа}$.
}
\answer{%
    \begin{align*}
    T\text{— const} &\implies P_1V_1 = \nu RT = P_2V_2.
    \\
    V_2 = \frac 13 V_1 &\implies P_1V_1 = P_2 \cdot \frac 13V_1 \implies P_2 = 3P_1 = 3p_{\text{aтм}}.
    \\
    P_2 = p_{\text{aтм}} + \rho_{\text{в}} g h \implies h = \frac{P_2 - p_{\text{aтм}}}{\rho_{\text{в}} g} &= \frac{3p_{\text{aтм}} - p_{\text{aтм}}}{\rho_{\text{в}} g} = \frac{2 \cdot p_{\text{aтм}}}{\rho_{\text{в}} g} =  \\
     &= \frac{2 \cdot 100\,\text{кПа}}{1000\,\frac{\text{кг}}{\text{м}^{3}} \cdot  10\,\frac{\text{м}}{\text{с}^{2}}} \approx 20\,\text{м}.
    \end{align*}
}
\solutionspace{120pt}

\tasknumber{15}%
\task{%
    Газу сообщили некоторое количество теплоты,
    при этом четверть его он потратил на совершение работы,
    одновременно увеличив свою внутреннюю энергию на $3000\,\text{Дж}$.
    Определите работу, совершённую газом.
}
\answer{%
    \begin{align*}
    Q &= A' + \Delta U, A' = \frac 14 Q \implies Q \cdot \cbr{1 - \frac 14} = \Delta U \implies Q = \frac{\Delta U}{1 - \frac 14} = \frac{ 3000\,\text{Дж} }{1 - \frac 14} \approx 4000\,\text{Дж}.
    \\
    A' &= \frac 14 Q
        = \frac 14 \cdot \frac{\Delta U}{1 - \frac 14}
        = \frac{\Delta U}{4 - 1}
        = \frac{ 3000\,\text{Дж} }{4 - 1} \approx 1000\,\text{Дж}.
    \end{align*}
}
\solutionspace{60pt}

\tasknumber{16}%
\task{%
    Два конденсатора ёмкостей $C_1 = 20\,\text{нФ}$ и $C_2 = 60\,\text{нФ}$ последовательно подключают
    к источнику напряжения $V = 200\,\text{В}$ (см.
    рис.).
    % Определите заряды каждого из конденсаторов.
    Определите заряд первого конденсатора.

    \begin{tikzpicture}[circuit ee IEC, semithick]
        \draw  (0, 0) to [capacitor={info={$C_1$}}] (1, 0)
                       to [capacitor={info={$C_2$}}] (2, 0)
        ;
        % \draw [-o] (0, 0) -- ++(-0.5, 0) node[left] {$-$};
        % \draw [-o] (2, 0) -- ++(0.5, 0) node[right] {$+$};
        \draw [-o] (0, 0) -- ++(-0.5, 0) node[left] {};
        \draw [-o] (2, 0) -- ++(0.5, 0) node[right] {};
    \end{tikzpicture}
}
\answer{%
    $
        Q_1
            = Q_2
            = CV
            = \frac{ V }{\frac1{C_1} + \frac1{C_2}}
            = \frac{C_1C_2V}{C_1 + C_2}
            = \frac{
                20\,\text{нФ} \cdot 60\,\text{нФ} \cdot 200\,\text{В}
            }{
                20\,\text{нФ} + 60\,\text{нФ}
            }
            = 3000{,}00\,\text{нКл}
    $
}
\solutionspace{120pt}

\tasknumber{17}%
\task{%
    В вакууме вдоль одной прямой расположены три положительных заряда так,
    что расстояние между соседними зарядами равно $l$.
    Сделайте рисунок,
    и определите силу, действующую на крайний заряд.
    Модули всех зарядов равны $q$ ($q > 0$).
}
\solutionspace{80pt}

\tasknumber{18}%
\task{%
    Юлия проводит эксперименты c 2 кусками одинаковой стальной проволки, причём второй кусок в три раза длиннее первого.
    В одном из экспериментов Юлия подаёт на первый кусок проволки напряжение в два раза раз больше, чем на второй.
    Определите отношения в двух проволках в этом эксперименте (второй к первой):
    \begin{itemize}
        \item отношение сил тока,
        \item отношение выделяющихся мощностей.
    \end{itemize}
}
\answer{%
    $\eli_2 / \eli_1 = \frac16, \P_2 / \P_1 = \frac16, $
}

\variantsplitter

\addpersonalvariant{София Журавлёва}

\tasknumber{1}%
\task{%
    Саша стартует на лошади и в течение $t = 5\,\text{c}$ двигается с постоянным ускорением $2{,}5\,\frac{\text{м}}{\text{с}^{2}}$.
    Определите
    \begin{itemize}
        \item какую скорость при этом удастся достичь,
        \item какой путь за это время будет пройден,
        \item среднюю скорость за всё время движения, если после начального ускорения продолжить движение равномерно ещё в течение времени $3t$
    \end{itemize}
}
\solutionspace{120pt}

\tasknumber{2}%
\task{%
    Какой путь тело пройдёт за третью секунду после начала свободного падения?
    Какую скорость в конце этой секунды оно имеет?
}
\solutionspace{120pt}

\tasknumber{3}%
\task{%
    Карусель диаметром $2\,\text{м}$ равномерно совершает 10 оборотов в минуту.
    Определите
    \begin{itemize}
        \item период и частоту её обращения,
        \item скорость и ускорение крайних её точек.
    \end{itemize}
}
\solutionspace{80pt}

\tasknumber{4}%
\task{%
    Паша стоит на обрыве над рекой и методично и строго горизонтально кидает в неё камушки.
    За этим всем наблюдает экспериментатор Глюк, который уже выяснил, что камушки падают в реку спустя $1{,}4\,\text{с}$ после броска,
    а вот дальность полёта оценить сложнее: придётся лезть в воду.
    Выручите Глюка и определите:
    \begin{itemize}
        \item высоту обрыва (вместе с ростом Паши).
        \item дальность полёта камушков (по горизонтали) и их скорость при падении, приняв начальную скорость броска равной $v = 16\,\frac{\text{м}}{\text{с}}$.
    \end{itemize}
    Сопротивлением воздуха пренебречь.
}
\solutionspace{120pt}

\tasknumber{5}%
\task{%
    Шесть одинаковых брусков массой $2\,\text{кг}$ каждый лежат на гладком горизонтальном столе.
    Бруски пронумерованы от 1 до 6 и последовательно связаны между собой
    невесомыми нерастяжимыми нитями: 1 со 2, 2 с 3 (ну и с 1) и т.д.
    Экспериментатор Глюк прикладывает постоянную горизонтальную силу $60\,\text{Н}$ к бруску с наименьшим номером.
    С каким ускорением двигается система? Чему равна сила натяжения нити, связывающей бруски 1 и 2?
}
\solutionspace{120pt}

\tasknumber{6}%
\task{%
    Два бруска связаны лёгкой нерастяжимой нитью и перекинуты через неподвижный блок (см.
    рис.).
    Определите силу натяжения нити и ускорения брусков.
    Силами трения пренебречь, массы брусков
    равны $m_1 = 11\,\text{кг}$ и $m_2 = 10\,\text{кг}$.
    % $g = 10\,\frac{\text{м}}{\text{с}^{2}}$.

    \begin{tikzpicture}[x=1.5cm,y=1.5cm,thick]
        \draw
            (-0.4, 0) rectangle (-0.2, 1.2)
            (0.15, 0.5) rectangle (0.45, 1)
            (0, 2) circle [radius=0.3] -- ++(up:0.5)
            (-0.3, 1.2) -- ++(up:0.8)
            (0.3, 1) -- ++(up:1)
            (-0.7, 2.5) -- (0.7, 2.5)
            ;
        \draw[pattern={Lines[angle=51,distance=3pt]},pattern color=black,draw=none] (-0.7, 2.5) rectangle (0.7, 2.75);
        \node [left] (left) at (-0.4, 0.6) { $m_1$ };
        \node [right] (right) at (0.4, 0.75) { $m_2$ };
    \end{tikzpicture}
}
\solutionspace{80pt}

\tasknumber{7}%
\task{%
    Тело массой $2\,\text{кг}$ лежит на горизонтальной поверхности.
    Коэффициент трения между поверхностью и телом $0{,}2$.
    К телу приложена горизонтальная сила $3{,}5\,\text{Н}$.
    Определите силу трения, действующую на тело, и ускорение тела.
    % $g = 10\,\frac{\text{м}}{\text{с}^{2}}$.
}
\solutionspace{120pt}

\tasknumber{8}%
\task{%
    Определите плотность неизвестного вещества, если известно, что опускании тела из него
    в подсолнечное масло оно будет плавать и на половину выступать над поверхностью жидкости.
}
\solutionspace{120pt}

\tasknumber{9}%
\task{%
    	Определите силу, действующую на правую опору однородного горизонтального стержня длиной $l = 5\,\text{м}$
    	и массой $M = 1\,\text{кг}$, к которому подвешен груз массой $m = 4\,\text{кг}$ на расстоянии $2\,\text{м}$ от правого конца (см.
    рис.).

        \begin{tikzpicture}[thick]
            \draw
                (-2, -0.1) rectangle (2, 0.1)
                (-0.5, -0.1) -- (-0.5, -1)
                (-0.7, -1) rectangle (-0.3, -1.3)
           		(-2, -0.1) -- +(0.15,-0.9) -- +(-0.15,-0.9) -- cycle
            	(2, -0.1) -- +(0.15,-0.9) -- +(-0.15,-0.9) -- cycle
            ;
            \draw[pattern={Lines[angle=51,distance=2pt]},pattern color=black,draw=none]
            	(-2.15, -1.15) rectangle +(0.3, 0.15)
            	(2.15, -1.15) rectangle +(-0.3, 0.15)
            ;
            \node [right] (m_small) at (-0.3, -1.15) { $m$ };
            \node [above] (M_big) at (0, 0.1) { $M$ };
        \end{tikzpicture}
}
\solutionspace{80pt}

\tasknumber{10}%
\task{%
    Тонкий однородный кусок арматуры длиной $2\,\text{м}$ и массой $30\,\text{кг}$ лежит на горизонтальной поверхности.
    \begin{itemize}
        \item Какую минимальную силу надо приложить к одному из его концов, чтобы оторвать его от этой поверхности?
        \item Какую минимальную работу надо совершить, чтобы поставить его на землю в вертикальное положение?
    \end{itemize}
    % Примите $g = 10\,\frac{\text{м}}{\text{с}^{2}}$.
}
\answer{%
    $A = mg\frac l2 = 300\,\text{Дж}$
}
\solutionspace{120pt}

\tasknumber{11}%
\task{%
    Определите работу силы, которая обеспечит подъём тела массой $5\,\text{кг}$ на высоту $2\,\text{м}$ с постоянным ускорением $2\,\frac{\text{м}}{\text{c}^{2}}$.
    % Примите $g = 10\,\frac{\text{м}}{\text{с}^{2}}$.
}
\answer{%
    \begin{align*}
    &\text{Для подъёма:} A = Fh = (mg + ma) h = m(g+a)h, \\
    &\text{Для спуска:} A = -Fh = -(mg - ma) h = -m(g-a)h, \\
    &\text{В результате получаем:} 120\,\text{Дж}.
    \end{align*}
}
\solutionspace{60pt}

\tasknumber{12}%
\task{%
    Тело бросили вертикально вверх со скоростью $14\,\frac{\text{м}}{\text{c}}$.
    На какой высоте кинетическая энергия тела составит треть от потенциальной?
}
\solutionspace{100pt}

\tasknumber{13}%
\task{%
    Плотность воздуха при нормальных условиях равна $1{,}3\,\frac{\text{кг}}{\text{м}^{3}}$.
    Чему равна плотность воздуха
    при температуре $100\celsius$ и давлении $120\,\text{кПа}$?
}
\solutionspace{120pt}

\tasknumber{14}%
\task{%
    Небольшую цилиндрическую пробирку с воздухом погружают на некоторую глубину в глубокое пресное озеро,
    после чего воздух занимает в ней лишь пятую часть от общего объема.
    Определите глубину, на которую погрузили пробирку.
    Температуру считать постоянной $T = 281\,\text{К}$, давлением паров воды пренебречь,
    атмосферное давление принять равным $p_{\text{aтм}} = 100\,\text{кПа}$.
}
\answer{%
    \begin{align*}
    T\text{— const} &\implies P_1V_1 = \nu RT = P_2V_2.
    \\
    V_2 = \frac 15 V_1 &\implies P_1V_1 = P_2 \cdot \frac 15V_1 \implies P_2 = 5P_1 = 5p_{\text{aтм}}.
    \\
    P_2 = p_{\text{aтм}} + \rho_{\text{в}} g h \implies h = \frac{P_2 - p_{\text{aтм}}}{\rho_{\text{в}} g} &= \frac{5p_{\text{aтм}} - p_{\text{aтм}}}{\rho_{\text{в}} g} = \frac{4 \cdot p_{\text{aтм}}}{\rho_{\text{в}} g} =  \\
     &= \frac{4 \cdot 100\,\text{кПа}}{1000\,\frac{\text{кг}}{\text{м}^{3}} \cdot  10\,\frac{\text{м}}{\text{с}^{2}}} \approx 40\,\text{м}.
    \end{align*}
}
\solutionspace{120pt}

\tasknumber{15}%
\task{%
    Газу сообщили некоторое количество теплоты,
    при этом четверть его он потратил на совершение работы,
    одновременно увеличив свою внутреннюю энергию на $1500\,\text{Дж}$.
    Определите количество теплоты, сообщённое газу.
}
\answer{%
    \begin{align*}
    Q &= A' + \Delta U, A' = \frac 14 Q \implies Q \cdot \cbr{1 - \frac 14} = \Delta U \implies Q = \frac{\Delta U}{1 - \frac 14} = \frac{ 1500\,\text{Дж} }{1 - \frac 14} \approx 2000\,\text{Дж}.
    \\
    A' &= \frac 14 Q
        = \frac 14 \cdot \frac{\Delta U}{1 - \frac 14}
        = \frac{\Delta U}{4 - 1}
        = \frac{ 1500\,\text{Дж} }{4 - 1} \approx 500\,\text{Дж}.
    \end{align*}
}
\solutionspace{60pt}

\tasknumber{16}%
\task{%
    Два конденсатора ёмкостей $C_1 = 20\,\text{нФ}$ и $C_2 = 30\,\text{нФ}$ последовательно подключают
    к источнику напряжения $U = 150\,\text{В}$ (см.
    рис.).
    % Определите заряды каждого из конденсаторов.
    Определите заряд второго конденсатора.

    \begin{tikzpicture}[circuit ee IEC, semithick]
        \draw  (0, 0) to [capacitor={info={$C_1$}}] (1, 0)
                       to [capacitor={info={$C_2$}}] (2, 0)
        ;
        % \draw [-o] (0, 0) -- ++(-0.5, 0) node[left] {$-$};
        % \draw [-o] (2, 0) -- ++(0.5, 0) node[right] {$+$};
        \draw [-o] (0, 0) -- ++(-0.5, 0) node[left] {};
        \draw [-o] (2, 0) -- ++(0.5, 0) node[right] {};
    \end{tikzpicture}
}
\answer{%
    $
        Q_1
            = Q_2
            = CU
            = \frac{ U }{\frac1{C_1} + \frac1{C_2}}
            = \frac{C_1C_2U}{C_1 + C_2}
            = \frac{
                20\,\text{нФ} \cdot 30\,\text{нФ} \cdot 150\,\text{В}
            }{
                20\,\text{нФ} + 30\,\text{нФ}
            }
            = 1800{,}00\,\text{нКл}
    $
}
\solutionspace{120pt}

\tasknumber{17}%
\task{%
    В вакууме вдоль одной прямой расположены три отрицательных заряда так,
    что расстояние между соседними зарядами равно $d$.
    Сделайте рисунок,
    и определите силу, действующую на крайний заряд.
    Модули всех зарядов равны $Q$ ($Q > 0$).
}
\solutionspace{80pt}

\tasknumber{18}%
\task{%
    Юлия проводит эксперименты c 2 кусками одинаковой медной проволки, причём второй кусок в семь раз длиннее первого.
    В одном из экспериментов Юлия подаёт на первый кусок проволки напряжение в три раза раз больше, чем на второй.
    Определите отношения в двух проволках в этом эксперименте (второй к первой):
    \begin{itemize}
        \item отношение сил тока,
        \item отношение выделяющихся мощностей.
    \end{itemize}
}
\answer{%
    $\eli_2 / \eli_1 = \frac1{21}, \P_2 / \P_1 = \frac1{21}, $
}

\variantsplitter

\addpersonalvariant{Константин Козлов}

\tasknumber{1}%
\task{%
    Женя стартует на велосипеде и в течение $t = 3\,\text{c}$ двигается с постоянным ускорением $2{,}5\,\frac{\text{м}}{\text{с}^{2}}$.
    Определите
    \begin{itemize}
        \item какую скорость при этом удастся достичь,
        \item какой путь за это время будет пройден,
        \item среднюю скорость за всё время движения, если после начального ускорения продолжить движение равномерно ещё в течение времени $3t$
    \end{itemize}
}
\solutionspace{120pt}

\tasknumber{2}%
\task{%
    Какой путь тело пройдёт за четвёртую секунду после начала свободного падения?
    Какую скорость в начале этой секунды оно имеет?
}
\solutionspace{120pt}

\tasknumber{3}%
\task{%
    Карусель диаметром $2\,\text{м}$ равномерно совершает 6 оборотов в минуту.
    Определите
    \begin{itemize}
        \item период и частоту её обращения,
        \item скорость и ускорение крайних её точек.
    \end{itemize}
}
\solutionspace{80pt}

\tasknumber{4}%
\task{%
    Миша стоит на обрыве над рекой и методично и строго горизонтально кидает в неё камушки.
    За этим всем наблюдает экспериментатор Глюк, который уже выяснил, что камушки падают в реку спустя $1{,}6\,\text{с}$ после броска,
    а вот дальность полёта оценить сложнее: придётся лезть в воду.
    Выручите Глюка и определите:
    \begin{itemize}
        \item высоту обрыва (вместе с ростом Миши).
        \item дальность полёта камушков (по горизонтали) и их скорость при падении, приняв начальную скорость броска равной $v = 16\,\frac{\text{м}}{\text{с}}$.
    \end{itemize}
    Сопротивлением воздуха пренебречь.
}
\solutionspace{120pt}

\tasknumber{5}%
\task{%
    Шесть одинаковых брусков массой $2\,\text{кг}$ каждый лежат на гладком горизонтальном столе.
    Бруски пронумерованы от 1 до 6 и последовательно связаны между собой
    невесомыми нерастяжимыми нитями: 1 со 2, 2 с 3 (ну и с 1) и т.д.
    Экспериментатор Глюк прикладывает постоянную горизонтальную силу $90\,\text{Н}$ к бруску с наименьшим номером.
    С каким ускорением двигается система? Чему равна сила натяжения нити, связывающей бруски 3 и 4?
}
\solutionspace{120pt}

\tasknumber{6}%
\task{%
    Два бруска связаны лёгкой нерастяжимой нитью и перекинуты через неподвижный блок (см.
    рис.).
    Определите силу натяжения нити и ускорения брусков.
    Силами трения пренебречь, массы брусков
    равны $m_1 = 11\,\text{кг}$ и $m_2 = 10\,\text{кг}$.
    % $g = 10\,\frac{\text{м}}{\text{с}^{2}}$.

    \begin{tikzpicture}[x=1.5cm,y=1.5cm,thick]
        \draw
            (-0.4, 0) rectangle (-0.2, 1.2)
            (0.15, 0.5) rectangle (0.45, 1)
            (0, 2) circle [radius=0.3] -- ++(up:0.5)
            (-0.3, 1.2) -- ++(up:0.8)
            (0.3, 1) -- ++(up:1)
            (-0.7, 2.5) -- (0.7, 2.5)
            ;
        \draw[pattern={Lines[angle=51,distance=3pt]},pattern color=black,draw=none] (-0.7, 2.5) rectangle (0.7, 2.75);
        \node [left] (left) at (-0.4, 0.6) { $m_1$ };
        \node [right] (right) at (0.4, 0.75) { $m_2$ };
    \end{tikzpicture}
}
\solutionspace{80pt}

\tasknumber{7}%
\task{%
    Тело массой $2\,\text{кг}$ лежит на горизонтальной поверхности.
    Коэффициент трения между поверхностью и телом $0{,}25$.
    К телу приложена горизонтальная сила $5{,}5\,\text{Н}$.
    Определите силу трения, действующую на тело, и ускорение тела.
    % $g = 10\,\frac{\text{м}}{\text{с}^{2}}$.
}
\solutionspace{120pt}

\tasknumber{8}%
\task{%
    Определите плотность неизвестного вещества, если известно, что опускании тела из него
    в подсолнечное масло оно будет плавать и на половину выступать над поверхностью жидкости.
}
\solutionspace{120pt}

\tasknumber{9}%
\task{%
    	Определите силу, действующую на правую опору однородного горизонтального стержня длиной $l = 9\,\text{м}$
    	и массой $M = 1\,\text{кг}$, к которому подвешен груз массой $m = 3\,\text{кг}$ на расстоянии $4\,\text{м}$ от правого конца (см.
    рис.).

        \begin{tikzpicture}[thick]
            \draw
                (-2, -0.1) rectangle (2, 0.1)
                (-0.5, -0.1) -- (-0.5, -1)
                (-0.7, -1) rectangle (-0.3, -1.3)
           		(-2, -0.1) -- +(0.15,-0.9) -- +(-0.15,-0.9) -- cycle
            	(2, -0.1) -- +(0.15,-0.9) -- +(-0.15,-0.9) -- cycle
            ;
            \draw[pattern={Lines[angle=51,distance=2pt]},pattern color=black,draw=none]
            	(-2.15, -1.15) rectangle +(0.3, 0.15)
            	(2.15, -1.15) rectangle +(-0.3, 0.15)
            ;
            \node [right] (m_small) at (-0.3, -1.15) { $m$ };
            \node [above] (M_big) at (0, 0.1) { $M$ };
        \end{tikzpicture}
}
\solutionspace{80pt}

\tasknumber{10}%
\task{%
    Тонкий однородный кусок арматуры длиной $3\,\text{м}$ и массой $10\,\text{кг}$ лежит на горизонтальной поверхности.
    \begin{itemize}
        \item Какую минимальную силу надо приложить к одному из его концов, чтобы оторвать его от этой поверхности?
        \item Какую минимальную работу надо совершить, чтобы поставить его на землю в вертикальное положение?
    \end{itemize}
    % Примите $g = 10\,\frac{\text{м}}{\text{с}^{2}}$.
}
\answer{%
    $A = mg\frac l2 = 150\,\text{Дж}$
}
\solutionspace{120pt}

\tasknumber{11}%
\task{%
    Определите работу силы, которая обеспечит спуск тела массой $2\,\text{кг}$ на высоту $5\,\text{м}$ с постоянным ускорением $6\,\frac{\text{м}}{\text{c}^{2}}$.
    % Примите $g = 10\,\frac{\text{м}}{\text{с}^{2}}$.
}
\answer{%
    \begin{align*}
    &\text{Для подъёма:} A = Fh = (mg + ma) h = m(g+a)h, \\
    &\text{Для спуска:} A = -Fh = -(mg - ma) h = -m(g-a)h, \\
    &\text{В результате получаем:} -40\,\text{Дж}.
    \end{align*}
}
\solutionspace{60pt}

\tasknumber{12}%
\task{%
    Тело бросили вертикально вверх со скоростью $20\,\frac{\text{м}}{\text{c}}$.
    На какой высоте кинетическая энергия тела составит половину от потенциальной?
}
\solutionspace{100pt}

\tasknumber{13}%
\task{%
    Плотность воздуха при нормальных условиях равна $1{,}3\,\frac{\text{кг}}{\text{м}^{3}}$.
    Чему равна плотность воздуха
    при температуре $200\celsius$ и давлении $150\,\text{кПа}$?
}
\solutionspace{120pt}

\tasknumber{14}%
\task{%
    Небольшую цилиндрическую пробирку с воздухом погружают на некоторую глубину в глубокое пресное озеро,
    после чего воздух занимает в ней лишь пятую часть от общего объема.
    Определите глубину, на которую погрузили пробирку.
    Температуру считать постоянной $T = 292\,\text{К}$, давлением паров воды пренебречь,
    атмосферное давление принять равным $p_{\text{aтм}} = 100\,\text{кПа}$.
}
\answer{%
    \begin{align*}
    T\text{— const} &\implies P_1V_1 = \nu RT = P_2V_2.
    \\
    V_2 = \frac 15 V_1 &\implies P_1V_1 = P_2 \cdot \frac 15V_1 \implies P_2 = 5P_1 = 5p_{\text{aтм}}.
    \\
    P_2 = p_{\text{aтм}} + \rho_{\text{в}} g h \implies h = \frac{P_2 - p_{\text{aтм}}}{\rho_{\text{в}} g} &= \frac{5p_{\text{aтм}} - p_{\text{aтм}}}{\rho_{\text{в}} g} = \frac{4 \cdot p_{\text{aтм}}}{\rho_{\text{в}} g} =  \\
     &= \frac{4 \cdot 100\,\text{кПа}}{1000\,\frac{\text{кг}}{\text{м}^{3}} \cdot  10\,\frac{\text{м}}{\text{с}^{2}}} \approx 40\,\text{м}.
    \end{align*}
}
\solutionspace{120pt}

\tasknumber{15}%
\task{%
    Газу сообщили некоторое количество теплоты,
    при этом четверть его он потратил на совершение работы,
    одновременно увеличив свою внутреннюю энергию на $3000\,\text{Дж}$.
    Определите количество теплоты, сообщённое газу.
}
\answer{%
    \begin{align*}
    Q &= A' + \Delta U, A' = \frac 14 Q \implies Q \cdot \cbr{1 - \frac 14} = \Delta U \implies Q = \frac{\Delta U}{1 - \frac 14} = \frac{ 3000\,\text{Дж} }{1 - \frac 14} \approx 4000\,\text{Дж}.
    \\
    A' &= \frac 14 Q
        = \frac 14 \cdot \frac{\Delta U}{1 - \frac 14}
        = \frac{\Delta U}{4 - 1}
        = \frac{ 3000\,\text{Дж} }{4 - 1} \approx 1000\,\text{Дж}.
    \end{align*}
}
\solutionspace{60pt}

\tasknumber{16}%
\task{%
    Два конденсатора ёмкостей $C_1 = 60\,\text{нФ}$ и $C_2 = 30\,\text{нФ}$ последовательно подключают
    к источнику напряжения $U = 300\,\text{В}$ (см.
    рис.).
    % Определите заряды каждого из конденсаторов.
    Определите заряд первого конденсатора.

    \begin{tikzpicture}[circuit ee IEC, semithick]
        \draw  (0, 0) to [capacitor={info={$C_1$}}] (1, 0)
                       to [capacitor={info={$C_2$}}] (2, 0)
        ;
        % \draw [-o] (0, 0) -- ++(-0.5, 0) node[left] {$-$};
        % \draw [-o] (2, 0) -- ++(0.5, 0) node[right] {$+$};
        \draw [-o] (0, 0) -- ++(-0.5, 0) node[left] {};
        \draw [-o] (2, 0) -- ++(0.5, 0) node[right] {};
    \end{tikzpicture}
}
\answer{%
    $
        Q_1
            = Q_2
            = CU
            = \frac{ U }{\frac1{C_1} + \frac1{C_2}}
            = \frac{C_1C_2U}{C_1 + C_2}
            = \frac{
                60\,\text{нФ} \cdot 30\,\text{нФ} \cdot 300\,\text{В}
            }{
                60\,\text{нФ} + 30\,\text{нФ}
            }
            = 6000{,}00\,\text{нКл}
    $
}
\solutionspace{120pt}

\tasknumber{17}%
\task{%
    В вакууме вдоль одной прямой расположены четыре отрицательных заряда так,
    что расстояние между соседними зарядами равно $r$.
    Сделайте рисунок,
    и определите силу, действующую на крайний заряд.
    Модули всех зарядов равны $Q$ ($Q > 0$).
}
\solutionspace{80pt}

\tasknumber{18}%
\task{%
    Юлия проводит эксперименты c 2 кусками одинаковой стальной проволки, причём второй кусок в семь раз длиннее первого.
    В одном из экспериментов Юлия подаёт на первый кусок проволки напряжение в десять раз раз больше, чем на второй.
    Определите отношения в двух проволках в этом эксперименте (второй к первой):
    \begin{itemize}
        \item отношение сил тока,
        \item отношение выделяющихся мощностей.
    \end{itemize}
}
\answer{%
    $\eli_2 / \eli_1 = \frac1{70}, \P_2 / \P_1 = \frac1{70}, $
}

\variantsplitter

\addpersonalvariant{Наталья Кравченко}

\tasknumber{1}%
\task{%
    Саша стартует на лошади и в течение $t = 3\,\text{c}$ двигается с постоянным ускорением $0{,}5\,\frac{\text{м}}{\text{с}^{2}}$.
    Определите
    \begin{itemize}
        \item какую скорость при этом удастся достичь,
        \item какой путь за это время будет пройден,
        \item среднюю скорость за всё время движения, если после начального ускорения продолжить движение равномерно ещё в течение времени $2t$
    \end{itemize}
}
\solutionspace{120pt}

\tasknumber{2}%
\task{%
    Какой путь тело пройдёт за шестую секунду после начала свободного падения?
    Какую скорость в конце этой секунды оно имеет?
}
\solutionspace{120pt}

\tasknumber{3}%
\task{%
    Карусель диаметром $2\,\text{м}$ равномерно совершает 6 оборотов в минуту.
    Определите
    \begin{itemize}
        \item период и частоту её обращения,
        \item скорость и ускорение крайних её точек.
    \end{itemize}
}
\solutionspace{80pt}

\tasknumber{4}%
\task{%
    Маша стоит на обрыве над рекой и методично и строго горизонтально кидает в неё камушки.
    За этим всем наблюдает экспериментатор Глюк, который уже выяснил, что камушки падают в реку спустя $1{,}2\,\text{с}$ после броска,
    а вот дальность полёта оценить сложнее: придётся лезть в воду.
    Выручите Глюка и определите:
    \begin{itemize}
        \item высоту обрыва (вместе с ростом Маши).
        \item дальность полёта камушков (по горизонтали) и их скорость при падении, приняв начальную скорость броска равной $v = 15\,\frac{\text{м}}{\text{с}}$.
    \end{itemize}
    Сопротивлением воздуха пренебречь.
}
\solutionspace{120pt}

\tasknumber{5}%
\task{%
    Шесть одинаковых брусков массой $2\,\text{кг}$ каждый лежат на гладком горизонтальном столе.
    Бруски пронумерованы от 1 до 6 и последовательно связаны между собой
    невесомыми нерастяжимыми нитями: 1 со 2, 2 с 3 (ну и с 1) и т.д.
    Экспериментатор Глюк прикладывает постоянную горизонтальную силу $60\,\text{Н}$ к бруску с наибольшим номером.
    С каким ускорением двигается система? Чему равна сила натяжения нити, связывающей бруски 3 и 4?
}
\solutionspace{120pt}

\tasknumber{6}%
\task{%
    Два бруска связаны лёгкой нерастяжимой нитью и перекинуты через неподвижный блок (см.
    рис.).
    Определите силу натяжения нити и ускорения брусков.
    Силами трения пренебречь, массы брусков
    равны $m_1 = 8\,\text{кг}$ и $m_2 = 6\,\text{кг}$.
    % $g = 10\,\frac{\text{м}}{\text{с}^{2}}$.

    \begin{tikzpicture}[x=1.5cm,y=1.5cm,thick]
        \draw
            (-0.4, 0) rectangle (-0.2, 1.2)
            (0.15, 0.5) rectangle (0.45, 1)
            (0, 2) circle [radius=0.3] -- ++(up:0.5)
            (-0.3, 1.2) -- ++(up:0.8)
            (0.3, 1) -- ++(up:1)
            (-0.7, 2.5) -- (0.7, 2.5)
            ;
        \draw[pattern={Lines[angle=51,distance=3pt]},pattern color=black,draw=none] (-0.7, 2.5) rectangle (0.7, 2.75);
        \node [left] (left) at (-0.4, 0.6) { $m_1$ };
        \node [right] (right) at (0.4, 0.75) { $m_2$ };
    \end{tikzpicture}
}
\solutionspace{80pt}

\tasknumber{7}%
\task{%
    Тело массой $2\,\text{кг}$ лежит на горизонтальной поверхности.
    Коэффициент трения между поверхностью и телом $0{,}2$.
    К телу приложена горизонтальная сила $3{,}5\,\text{Н}$.
    Определите силу трения, действующую на тело, и ускорение тела.
    % $g = 10\,\frac{\text{м}}{\text{с}^{2}}$.
}
\solutionspace{120pt}

\tasknumber{8}%
\task{%
    Определите плотность неизвестного вещества, если известно, что опускании тела из него
    в керосин оно будет плавать и на треть выступать над поверхностью жидкости.
}
\solutionspace{120pt}

\tasknumber{9}%
\task{%
    	Определите силу, действующую на левую опору однородного горизонтального стержня длиной $l = 9\,\text{м}$
    	и массой $M = 5\,\text{кг}$, к которому подвешен груз массой $m = 2\,\text{кг}$ на расстоянии $4\,\text{м}$ от правого конца (см.
    рис.).

        \begin{tikzpicture}[thick]
            \draw
                (-2, -0.1) rectangle (2, 0.1)
                (-0.5, -0.1) -- (-0.5, -1)
                (-0.7, -1) rectangle (-0.3, -1.3)
           		(-2, -0.1) -- +(0.15,-0.9) -- +(-0.15,-0.9) -- cycle
            	(2, -0.1) -- +(0.15,-0.9) -- +(-0.15,-0.9) -- cycle
            ;
            \draw[pattern={Lines[angle=51,distance=2pt]},pattern color=black,draw=none]
            	(-2.15, -1.15) rectangle +(0.3, 0.15)
            	(2.15, -1.15) rectangle +(-0.3, 0.15)
            ;
            \node [right] (m_small) at (-0.3, -1.15) { $m$ };
            \node [above] (M_big) at (0, 0.1) { $M$ };
        \end{tikzpicture}
}
\solutionspace{80pt}

\tasknumber{10}%
\task{%
    Тонкий однородный лом длиной $1\,\text{м}$ и массой $20\,\text{кг}$ лежит на горизонтальной поверхности.
    \begin{itemize}
        \item Какую минимальную силу надо приложить к одному из его концов, чтобы оторвать его от этой поверхности?
        \item Какую минимальную работу надо совершить, чтобы поставить его на землю в вертикальное положение?
    \end{itemize}
    % Примите $g = 10\,\frac{\text{м}}{\text{с}^{2}}$.
}
\answer{%
    $A = mg\frac l2 = 100\,\text{Дж}$
}
\solutionspace{120pt}

\tasknumber{11}%
\task{%
    Определите работу силы, которая обеспечит спуск тела массой $2\,\text{кг}$ на высоту $2\,\text{м}$ с постоянным ускорением $4\,\frac{\text{м}}{\text{c}^{2}}$.
    % Примите $g = 10\,\frac{\text{м}}{\text{с}^{2}}$.
}
\answer{%
    \begin{align*}
    &\text{Для подъёма:} A = Fh = (mg + ma) h = m(g+a)h, \\
    &\text{Для спуска:} A = -Fh = -(mg - ma) h = -m(g-a)h, \\
    &\text{В результате получаем:} -24\,\text{Дж}.
    \end{align*}
}
\solutionspace{60pt}

\tasknumber{12}%
\task{%
    Тело бросили вертикально вверх со скоростью $10\,\frac{\text{м}}{\text{c}}$.
    На какой высоте кинетическая энергия тела составит треть от потенциальной?
}
\solutionspace{100pt}

\tasknumber{13}%
\task{%
    Плотность воздуха при нормальных условиях равна $1{,}3\,\frac{\text{кг}}{\text{м}^{3}}$.
    Чему равна плотность воздуха
    при температуре $150\celsius$ и давлении $120\,\text{кПа}$?
}
\solutionspace{120pt}

\tasknumber{14}%
\task{%
    Небольшую цилиндрическую пробирку с воздухом погружают на некоторую глубину в глубокое пресное озеро,
    после чего воздух занимает в ней лишь третью часть от общего объема.
    Определите глубину, на которую погрузили пробирку.
    Температуру считать постоянной $T = 291\,\text{К}$, давлением паров воды пренебречь,
    атмосферное давление принять равным $p_{\text{aтм}} = 100\,\text{кПа}$.
}
\answer{%
    \begin{align*}
    T\text{— const} &\implies P_1V_1 = \nu RT = P_2V_2.
    \\
    V_2 = \frac 13 V_1 &\implies P_1V_1 = P_2 \cdot \frac 13V_1 \implies P_2 = 3P_1 = 3p_{\text{aтм}}.
    \\
    P_2 = p_{\text{aтм}} + \rho_{\text{в}} g h \implies h = \frac{P_2 - p_{\text{aтм}}}{\rho_{\text{в}} g} &= \frac{3p_{\text{aтм}} - p_{\text{aтм}}}{\rho_{\text{в}} g} = \frac{2 \cdot p_{\text{aтм}}}{\rho_{\text{в}} g} =  \\
     &= \frac{2 \cdot 100\,\text{кПа}}{1000\,\frac{\text{кг}}{\text{м}^{3}} \cdot  10\,\frac{\text{м}}{\text{с}^{2}}} \approx 20\,\text{м}.
    \end{align*}
}
\solutionspace{120pt}

\tasknumber{15}%
\task{%
    Газу сообщили некоторое количество теплоты,
    при этом половину его он потратил на совершение работы,
    одновременно увеличив свою внутреннюю энергию на $1200\,\text{Дж}$.
    Определите количество теплоты, сообщённое газу.
}
\answer{%
    \begin{align*}
    Q &= A' + \Delta U, A' = \frac 12 Q \implies Q \cdot \cbr{1 - \frac 12} = \Delta U \implies Q = \frac{\Delta U}{1 - \frac 12} = \frac{ 1200\,\text{Дж} }{1 - \frac 12} \approx 2400\,\text{Дж}.
    \\
    A' &= \frac 12 Q
        = \frac 12 \cdot \frac{\Delta U}{1 - \frac 12}
        = \frac{\Delta U}{2 - 1}
        = \frac{ 1200\,\text{Дж} }{2 - 1} \approx 1200\,\text{Дж}.
    \end{align*}
}
\solutionspace{60pt}

\tasknumber{16}%
\task{%
    Два конденсатора ёмкостей $C_1 = 20\,\text{нФ}$ и $C_2 = 40\,\text{нФ}$ последовательно подключают
    к источнику напряжения $U = 300\,\text{В}$ (см.
    рис.).
    % Определите заряды каждого из конденсаторов.
    Определите заряд второго конденсатора.

    \begin{tikzpicture}[circuit ee IEC, semithick]
        \draw  (0, 0) to [capacitor={info={$C_1$}}] (1, 0)
                       to [capacitor={info={$C_2$}}] (2, 0)
        ;
        % \draw [-o] (0, 0) -- ++(-0.5, 0) node[left] {$-$};
        % \draw [-o] (2, 0) -- ++(0.5, 0) node[right] {$+$};
        \draw [-o] (0, 0) -- ++(-0.5, 0) node[left] {};
        \draw [-o] (2, 0) -- ++(0.5, 0) node[right] {};
    \end{tikzpicture}
}
\answer{%
    $
        Q_1
            = Q_2
            = CU
            = \frac{ U }{\frac1{C_1} + \frac1{C_2}}
            = \frac{C_1C_2U}{C_1 + C_2}
            = \frac{
                20\,\text{нФ} \cdot 40\,\text{нФ} \cdot 300\,\text{В}
            }{
                20\,\text{нФ} + 40\,\text{нФ}
            }
            = 4000{,}00\,\text{нКл}
    $
}
\solutionspace{120pt}

\tasknumber{17}%
\task{%
    В вакууме вдоль одной прямой расположены три положительных заряда так,
    что расстояние между соседними зарядами равно $r$.
    Сделайте рисунок,
    и определите силу, действующую на крайний заряд.
    Модули всех зарядов равны $q$ ($q > 0$).
}
\solutionspace{80pt}

\tasknumber{18}%
\task{%
    Юлия проводит эксперименты c 2 кусками одинаковой алюминиевой проволки, причём второй кусок в шесть раз длиннее первого.
    В одном из экспериментов Юлия подаёт на первый кусок проволки напряжение в пять раз раз больше, чем на второй.
    Определите отношения в двух проволках в этом эксперименте (второй к первой):
    \begin{itemize}
        \item отношение сил тока,
        \item отношение выделяющихся мощностей.
    \end{itemize}
}
\answer{%
    $\eli_2 / \eli_1 = \frac1{30}, \P_2 / \P_1 = \frac1{30}, $
}

\variantsplitter

\addpersonalvariant{Матвей Кузьмин}

\tasknumber{1}%
\task{%
    Саша стартует на лошади и в течение $t = 10\,\text{c}$ двигается с постоянным ускорением $1{,}5\,\frac{\text{м}}{\text{с}^{2}}$.
    Определите
    \begin{itemize}
        \item какую скорость при этом удастся достичь,
        \item какой путь за это время будет пройден,
        \item среднюю скорость за всё время движения, если после начального ускорения продолжить движение равномерно ещё в течение времени $3t$
    \end{itemize}
}
\solutionspace{120pt}

\tasknumber{2}%
\task{%
    Какой путь тело пройдёт за вторую секунду после начала свободного падения?
    Какую скорость в конце этой секунды оно имеет?
}
\solutionspace{120pt}

\tasknumber{3}%
\task{%
    Карусель диаметром $3\,\text{м}$ равномерно совершает 10 оборотов в минуту.
    Определите
    \begin{itemize}
        \item период и частоту её обращения,
        \item скорость и ускорение крайних её точек.
    \end{itemize}
}
\solutionspace{80pt}

\tasknumber{4}%
\task{%
    Даша стоит на обрыве над рекой и методично и строго горизонтально кидает в неё камушки.
    За этим всем наблюдает экспериментатор Глюк, который уже выяснил, что камушки падают в реку спустя $1{,}6\,\text{с}$ после броска,
    а вот дальность полёта оценить сложнее: придётся лезть в воду.
    Выручите Глюка и определите:
    \begin{itemize}
        \item высоту обрыва (вместе с ростом Даши).
        \item дальность полёта камушков (по горизонтали) и их скорость при падении, приняв начальную скорость броска равной $v = 13\,\frac{\text{м}}{\text{с}}$.
    \end{itemize}
    Сопротивлением воздуха пренебречь.
}
\solutionspace{120pt}

\tasknumber{5}%
\task{%
    Пять одинаковых брусков массой $2\,\text{кг}$ каждый лежат на гладком горизонтальном столе.
    Бруски пронумерованы от 1 до 5 и последовательно связаны между собой
    невесомыми нерастяжимыми нитями: 1 со 2, 2 с 3 (ну и с 1) и т.д.
    Экспериментатор Глюк прикладывает постоянную горизонтальную силу $120\,\text{Н}$ к бруску с наибольшим номером.
    С каким ускорением двигается система? Чему равна сила натяжения нити, связывающей бруски 2 и 3?
}
\solutionspace{120pt}

\tasknumber{6}%
\task{%
    Два бруска связаны лёгкой нерастяжимой нитью и перекинуты через неподвижный блок (см.
    рис.).
    Определите силу натяжения нити и ускорения брусков.
    Силами трения пренебречь, массы брусков
    равны $m_1 = 11\,\text{кг}$ и $m_2 = 10\,\text{кг}$.
    % $g = 10\,\frac{\text{м}}{\text{с}^{2}}$.

    \begin{tikzpicture}[x=1.5cm,y=1.5cm,thick]
        \draw
            (-0.4, 0) rectangle (-0.2, 1.2)
            (0.15, 0.5) rectangle (0.45, 1)
            (0, 2) circle [radius=0.3] -- ++(up:0.5)
            (-0.3, 1.2) -- ++(up:0.8)
            (0.3, 1) -- ++(up:1)
            (-0.7, 2.5) -- (0.7, 2.5)
            ;
        \draw[pattern={Lines[angle=51,distance=3pt]},pattern color=black,draw=none] (-0.7, 2.5) rectangle (0.7, 2.75);
        \node [left] (left) at (-0.4, 0.6) { $m_1$ };
        \node [right] (right) at (0.4, 0.75) { $m_2$ };
    \end{tikzpicture}
}
\solutionspace{80pt}

\tasknumber{7}%
\task{%
    Тело массой $2{,}7\,\text{кг}$ лежит на горизонтальной поверхности.
    Коэффициент трения между поверхностью и телом $0{,}25$.
    К телу приложена горизонтальная сила $4{,}5\,\text{Н}$.
    Определите силу трения, действующую на тело, и ускорение тела.
    % $g = 10\,\frac{\text{м}}{\text{с}^{2}}$.
}
\solutionspace{120pt}

\tasknumber{8}%
\task{%
    Определите плотность неизвестного вещества, если известно, что опускании тела из него
    в керосин оно будет плавать и на половину выступать над поверхностью жидкости.
}
\solutionspace{120pt}

\tasknumber{9}%
\task{%
    	Определите силу, действующую на правую опору однородного горизонтального стержня длиной $l = 9\,\text{м}$
    	и массой $M = 1\,\text{кг}$, к которому подвешен груз массой $m = 2\,\text{кг}$ на расстоянии $4\,\text{м}$ от правого конца (см.
    рис.).

        \begin{tikzpicture}[thick]
            \draw
                (-2, -0.1) rectangle (2, 0.1)
                (-0.5, -0.1) -- (-0.5, -1)
                (-0.7, -1) rectangle (-0.3, -1.3)
           		(-2, -0.1) -- +(0.15,-0.9) -- +(-0.15,-0.9) -- cycle
            	(2, -0.1) -- +(0.15,-0.9) -- +(-0.15,-0.9) -- cycle
            ;
            \draw[pattern={Lines[angle=51,distance=2pt]},pattern color=black,draw=none]
            	(-2.15, -1.15) rectangle +(0.3, 0.15)
            	(2.15, -1.15) rectangle +(-0.3, 0.15)
            ;
            \node [right] (m_small) at (-0.3, -1.15) { $m$ };
            \node [above] (M_big) at (0, 0.1) { $M$ };
        \end{tikzpicture}
}
\solutionspace{80pt}

\tasknumber{10}%
\task{%
    Тонкий однородный шест длиной $2\,\text{м}$ и массой $10\,\text{кг}$ лежит на горизонтальной поверхности.
    \begin{itemize}
        \item Какую минимальную силу надо приложить к одному из его концов, чтобы оторвать его от этой поверхности?
        \item Какую минимальную работу надо совершить, чтобы поставить его на землю в вертикальное положение?
    \end{itemize}
    % Примите $g = 10\,\frac{\text{м}}{\text{с}^{2}}$.
}
\answer{%
    $A = mg\frac l2 = 100\,\text{Дж}$
}
\solutionspace{120pt}

\tasknumber{11}%
\task{%
    Определите работу силы, которая обеспечит спуск тела массой $5\,\text{кг}$ на высоту $5\,\text{м}$ с постоянным ускорением $4\,\frac{\text{м}}{\text{c}^{2}}$.
    % Примите $g = 10\,\frac{\text{м}}{\text{с}^{2}}$.
}
\answer{%
    \begin{align*}
    &\text{Для подъёма:} A = Fh = (mg + ma) h = m(g+a)h, \\
    &\text{Для спуска:} A = -Fh = -(mg - ma) h = -m(g-a)h, \\
    &\text{В результате получаем:} -150\,\text{Дж}.
    \end{align*}
}
\solutionspace{60pt}

\tasknumber{12}%
\task{%
    Тело бросили вертикально вверх со скоростью $10\,\frac{\text{м}}{\text{c}}$.
    На какой высоте кинетическая энергия тела составит треть от потенциальной?
}
\solutionspace{100pt}

\tasknumber{13}%
\task{%
    Плотность воздуха при нормальных условиях равна $1{,}3\,\frac{\text{кг}}{\text{м}^{3}}$.
    Чему равна плотность воздуха
    при температуре $50\celsius$ и давлении $80\,\text{кПа}$?
}
\solutionspace{120pt}

\tasknumber{14}%
\task{%
    Небольшую цилиндрическую пробирку с воздухом погружают на некоторую глубину в глубокое пресное озеро,
    после чего воздух занимает в ней лишь третью часть от общего объема.
    Определите глубину, на которую погрузили пробирку.
    Температуру считать постоянной $T = 287\,\text{К}$, давлением паров воды пренебречь,
    атмосферное давление принять равным $p_{\text{aтм}} = 100\,\text{кПа}$.
}
\answer{%
    \begin{align*}
    T\text{— const} &\implies P_1V_1 = \nu RT = P_2V_2.
    \\
    V_2 = \frac 13 V_1 &\implies P_1V_1 = P_2 \cdot \frac 13V_1 \implies P_2 = 3P_1 = 3p_{\text{aтм}}.
    \\
    P_2 = p_{\text{aтм}} + \rho_{\text{в}} g h \implies h = \frac{P_2 - p_{\text{aтм}}}{\rho_{\text{в}} g} &= \frac{3p_{\text{aтм}} - p_{\text{aтм}}}{\rho_{\text{в}} g} = \frac{2 \cdot p_{\text{aтм}}}{\rho_{\text{в}} g} =  \\
     &= \frac{2 \cdot 100\,\text{кПа}}{1000\,\frac{\text{кг}}{\text{м}^{3}} \cdot  10\,\frac{\text{м}}{\text{с}^{2}}} \approx 20\,\text{м}.
    \end{align*}
}
\solutionspace{120pt}

\tasknumber{15}%
\task{%
    Газу сообщили некоторое количество теплоты,
    при этом половину его он потратил на совершение работы,
    одновременно увеличив свою внутреннюю энергию на $1200\,\text{Дж}$.
    Определите количество теплоты, сообщённое газу.
}
\answer{%
    \begin{align*}
    Q &= A' + \Delta U, A' = \frac 12 Q \implies Q \cdot \cbr{1 - \frac 12} = \Delta U \implies Q = \frac{\Delta U}{1 - \frac 12} = \frac{ 1200\,\text{Дж} }{1 - \frac 12} \approx 2400\,\text{Дж}.
    \\
    A' &= \frac 12 Q
        = \frac 12 \cdot \frac{\Delta U}{1 - \frac 12}
        = \frac{\Delta U}{2 - 1}
        = \frac{ 1200\,\text{Дж} }{2 - 1} \approx 1200\,\text{Дж}.
    \end{align*}
}
\solutionspace{60pt}

\tasknumber{16}%
\task{%
    Два конденсатора ёмкостей $C_1 = 30\,\text{нФ}$ и $C_2 = 60\,\text{нФ}$ последовательно подключают
    к источнику напряжения $U = 450\,\text{В}$ (см.
    рис.).
    % Определите заряды каждого из конденсаторов.
    Определите заряд первого конденсатора.

    \begin{tikzpicture}[circuit ee IEC, semithick]
        \draw  (0, 0) to [capacitor={info={$C_1$}}] (1, 0)
                       to [capacitor={info={$C_2$}}] (2, 0)
        ;
        % \draw [-o] (0, 0) -- ++(-0.5, 0) node[left] {$-$};
        % \draw [-o] (2, 0) -- ++(0.5, 0) node[right] {$+$};
        \draw [-o] (0, 0) -- ++(-0.5, 0) node[left] {};
        \draw [-o] (2, 0) -- ++(0.5, 0) node[right] {};
    \end{tikzpicture}
}
\answer{%
    $
        Q_1
            = Q_2
            = CU
            = \frac{ U }{\frac1{C_1} + \frac1{C_2}}
            = \frac{C_1C_2U}{C_1 + C_2}
            = \frac{
                30\,\text{нФ} \cdot 60\,\text{нФ} \cdot 450\,\text{В}
            }{
                30\,\text{нФ} + 60\,\text{нФ}
            }
            = 9000{,}00\,\text{нКл}
    $
}
\solutionspace{120pt}

\tasknumber{17}%
\task{%
    В вакууме вдоль одной прямой расположены четыре положительных заряда так,
    что расстояние между соседними зарядами равно $r$.
    Сделайте рисунок,
    и определите силу, действующую на крайний заряд.
    Модули всех зарядов равны $q$ ($q > 0$).
}
\solutionspace{80pt}

\tasknumber{18}%
\task{%
    Юлия проводит эксперименты c 2 кусками одинаковой стальной проволки, причём второй кусок в семь раз длиннее первого.
    В одном из экспериментов Юлия подаёт на первый кусок проволки напряжение в два раза раз больше, чем на второй.
    Определите отношения в двух проволках в этом эксперименте (второй к первой):
    \begin{itemize}
        \item отношение сил тока,
        \item отношение выделяющихся мощностей.
    \end{itemize}
}
\answer{%
    $\eli_2 / \eli_1 = \frac1{14}, \P_2 / \P_1 = \frac1{14}, $
}

\variantsplitter

\addpersonalvariant{Сергей Малышев}

\tasknumber{1}%
\task{%
    Валя стартует на лошади и в течение $t = 4\,\text{c}$ двигается с постоянным ускорением $2{,}5\,\frac{\text{м}}{\text{с}^{2}}$.
    Определите
    \begin{itemize}
        \item какую скорость при этом удастся достичь,
        \item какой путь за это время будет пройден,
        \item среднюю скорость за всё время движения, если после начального ускорения продолжить движение равномерно ещё в течение времени $3t$
    \end{itemize}
}
\solutionspace{120pt}

\tasknumber{2}%
\task{%
    Какой путь тело пройдёт за пятую секунду после начала свободного падения?
    Какую скорость в начале этой секунды оно имеет?
}
\solutionspace{120pt}

\tasknumber{3}%
\task{%
    Карусель диаметром $2\,\text{м}$ равномерно совершает 6 оборотов в минуту.
    Определите
    \begin{itemize}
        \item период и частоту её обращения,
        \item скорость и ускорение крайних её точек.
    \end{itemize}
}
\solutionspace{80pt}

\tasknumber{4}%
\task{%
    Даша стоит на обрыве над рекой и методично и строго горизонтально кидает в неё камушки.
    За этим всем наблюдает экспериментатор Глюк, который уже выяснил, что камушки падают в реку спустя $1{,}6\,\text{с}$ после броска,
    а вот дальность полёта оценить сложнее: придётся лезть в воду.
    Выручите Глюка и определите:
    \begin{itemize}
        \item высоту обрыва (вместе с ростом Даши).
        \item дальность полёта камушков (по горизонтали) и их скорость при падении, приняв начальную скорость броска равной $v = 14\,\frac{\text{м}}{\text{с}}$.
    \end{itemize}
    Сопротивлением воздуха пренебречь.
}
\solutionspace{120pt}

\tasknumber{5}%
\task{%
    Четыре одинаковых брусков массой $2\,\text{кг}$ каждый лежат на гладком горизонтальном столе.
    Бруски пронумерованы от 1 до 4 и последовательно связаны между собой
    невесомыми нерастяжимыми нитями: 1 со 2, 2 с 3 (ну и с 1) и т.д.
    Экспериментатор Глюк прикладывает постоянную горизонтальную силу $60\,\text{Н}$ к бруску с наибольшим номером.
    С каким ускорением двигается система? Чему равна сила натяжения нити, связывающей бруски 1 и 2?
}
\solutionspace{120pt}

\tasknumber{6}%
\task{%
    Два бруска связаны лёгкой нерастяжимой нитью и перекинуты через неподвижный блок (см.
    рис.).
    Определите силу натяжения нити и ускорения брусков.
    Силами трения пренебречь, массы брусков
    равны $m_1 = 8\,\text{кг}$ и $m_2 = 14\,\text{кг}$.
    % $g = 10\,\frac{\text{м}}{\text{с}^{2}}$.

    \begin{tikzpicture}[x=1.5cm,y=1.5cm,thick]
        \draw
            (-0.4, 0) rectangle (-0.2, 1.2)
            (0.15, 0.5) rectangle (0.45, 1)
            (0, 2) circle [radius=0.3] -- ++(up:0.5)
            (-0.3, 1.2) -- ++(up:0.8)
            (0.3, 1) -- ++(up:1)
            (-0.7, 2.5) -- (0.7, 2.5)
            ;
        \draw[pattern={Lines[angle=51,distance=3pt]},pattern color=black,draw=none] (-0.7, 2.5) rectangle (0.7, 2.75);
        \node [left] (left) at (-0.4, 0.6) { $m_1$ };
        \node [right] (right) at (0.4, 0.75) { $m_2$ };
    \end{tikzpicture}
}
\solutionspace{80pt}

\tasknumber{7}%
\task{%
    Тело массой $2\,\text{кг}$ лежит на горизонтальной поверхности.
    Коэффициент трения между поверхностью и телом $0{,}25$.
    К телу приложена горизонтальная сила $5{,}5\,\text{Н}$.
    Определите силу трения, действующую на тело, и ускорение тела.
    % $g = 10\,\frac{\text{м}}{\text{с}^{2}}$.
}
\solutionspace{120pt}

\tasknumber{8}%
\task{%
    Определите плотность неизвестного вещества, если известно, что опускании тела из него
    в подсолнечное масло оно будет плавать и на четверть выступать над поверхностью жидкости.
}
\solutionspace{120pt}

\tasknumber{9}%
\task{%
    	Определите силу, действующую на левую опору однородного горизонтального стержня длиной $l = 5\,\text{м}$
    	и массой $M = 1\,\text{кг}$, к которому подвешен груз массой $m = 4\,\text{кг}$ на расстоянии $2\,\text{м}$ от правого конца (см.
    рис.).

        \begin{tikzpicture}[thick]
            \draw
                (-2, -0.1) rectangle (2, 0.1)
                (-0.5, -0.1) -- (-0.5, -1)
                (-0.7, -1) rectangle (-0.3, -1.3)
           		(-2, -0.1) -- +(0.15,-0.9) -- +(-0.15,-0.9) -- cycle
            	(2, -0.1) -- +(0.15,-0.9) -- +(-0.15,-0.9) -- cycle
            ;
            \draw[pattern={Lines[angle=51,distance=2pt]},pattern color=black,draw=none]
            	(-2.15, -1.15) rectangle +(0.3, 0.15)
            	(2.15, -1.15) rectangle +(-0.3, 0.15)
            ;
            \node [right] (m_small) at (-0.3, -1.15) { $m$ };
            \node [above] (M_big) at (0, 0.1) { $M$ };
        \end{tikzpicture}
}
\solutionspace{80pt}

\tasknumber{10}%
\task{%
    Тонкий однородный кусок арматуры длиной $2\,\text{м}$ и массой $10\,\text{кг}$ лежит на горизонтальной поверхности.
    \begin{itemize}
        \item Какую минимальную силу надо приложить к одному из его концов, чтобы оторвать его от этой поверхности?
        \item Какую минимальную работу надо совершить, чтобы поставить его на землю в вертикальное положение?
    \end{itemize}
    % Примите $g = 10\,\frac{\text{м}}{\text{с}^{2}}$.
}
\answer{%
    $A = mg\frac l2 = 100\,\text{Дж}$
}
\solutionspace{120pt}

\tasknumber{11}%
\task{%
    Определите работу силы, которая обеспечит спуск тела массой $3\,\text{кг}$ на высоту $2\,\text{м}$ с постоянным ускорением $4\,\frac{\text{м}}{\text{c}^{2}}$.
    % Примите $g = 10\,\frac{\text{м}}{\text{с}^{2}}$.
}
\answer{%
    \begin{align*}
    &\text{Для подъёма:} A = Fh = (mg + ma) h = m(g+a)h, \\
    &\text{Для спуска:} A = -Fh = -(mg - ma) h = -m(g-a)h, \\
    &\text{В результате получаем:} -36\,\text{Дж}.
    \end{align*}
}
\solutionspace{60pt}

\tasknumber{12}%
\task{%
    Тело бросили вертикально вверх со скоростью $10\,\frac{\text{м}}{\text{c}}$.
    На какой высоте кинетическая энергия тела составит половину от потенциальной?
}
\solutionspace{100pt}

\tasknumber{13}%
\task{%
    Плотность воздуха при нормальных условиях равна $1{,}3\,\frac{\text{кг}}{\text{м}^{3}}$.
    Чему равна плотность воздуха
    при температуре $200\celsius$ и давлении $80\,\text{кПа}$?
}
\solutionspace{120pt}

\tasknumber{14}%
\task{%
    Небольшую цилиндрическую пробирку с воздухом погружают на некоторую глубину в глубокое пресное озеро,
    после чего воздух занимает в ней лишь пятую часть от общего объема.
    Определите глубину, на которую погрузили пробирку.
    Температуру считать постоянной $T = 291\,\text{К}$, давлением паров воды пренебречь,
    атмосферное давление принять равным $p_{\text{aтм}} = 100\,\text{кПа}$.
}
\answer{%
    \begin{align*}
    T\text{— const} &\implies P_1V_1 = \nu RT = P_2V_2.
    \\
    V_2 = \frac 15 V_1 &\implies P_1V_1 = P_2 \cdot \frac 15V_1 \implies P_2 = 5P_1 = 5p_{\text{aтм}}.
    \\
    P_2 = p_{\text{aтм}} + \rho_{\text{в}} g h \implies h = \frac{P_2 - p_{\text{aтм}}}{\rho_{\text{в}} g} &= \frac{5p_{\text{aтм}} - p_{\text{aтм}}}{\rho_{\text{в}} g} = \frac{4 \cdot p_{\text{aтм}}}{\rho_{\text{в}} g} =  \\
     &= \frac{4 \cdot 100\,\text{кПа}}{1000\,\frac{\text{кг}}{\text{м}^{3}} \cdot  10\,\frac{\text{м}}{\text{с}^{2}}} \approx 40\,\text{м}.
    \end{align*}
}
\solutionspace{120pt}

\tasknumber{15}%
\task{%
    Газу сообщили некоторое количество теплоты,
    при этом треть его он потратил на совершение работы,
    одновременно увеличив свою внутреннюю энергию на $1200\,\text{Дж}$.
    Определите количество теплоты, сообщённое газу.
}
\answer{%
    \begin{align*}
    Q &= A' + \Delta U, A' = \frac 13 Q \implies Q \cdot \cbr{1 - \frac 13} = \Delta U \implies Q = \frac{\Delta U}{1 - \frac 13} = \frac{ 1200\,\text{Дж} }{1 - \frac 13} \approx 1800\,\text{Дж}.
    \\
    A' &= \frac 13 Q
        = \frac 13 \cdot \frac{\Delta U}{1 - \frac 13}
        = \frac{\Delta U}{3 - 1}
        = \frac{ 1200\,\text{Дж} }{3 - 1} \approx 600\,\text{Дж}.
    \end{align*}
}
\solutionspace{60pt}

\tasknumber{16}%
\task{%
    Два конденсатора ёмкостей $C_1 = 60\,\text{нФ}$ и $C_2 = 40\,\text{нФ}$ последовательно подключают
    к источнику напряжения $U = 150\,\text{В}$ (см.
    рис.).
    % Определите заряды каждого из конденсаторов.
    Определите заряд первого конденсатора.

    \begin{tikzpicture}[circuit ee IEC, semithick]
        \draw  (0, 0) to [capacitor={info={$C_1$}}] (1, 0)
                       to [capacitor={info={$C_2$}}] (2, 0)
        ;
        % \draw [-o] (0, 0) -- ++(-0.5, 0) node[left] {$-$};
        % \draw [-o] (2, 0) -- ++(0.5, 0) node[right] {$+$};
        \draw [-o] (0, 0) -- ++(-0.5, 0) node[left] {};
        \draw [-o] (2, 0) -- ++(0.5, 0) node[right] {};
    \end{tikzpicture}
}
\answer{%
    $
        Q_1
            = Q_2
            = CU
            = \frac{ U }{\frac1{C_1} + \frac1{C_2}}
            = \frac{C_1C_2U}{C_1 + C_2}
            = \frac{
                60\,\text{нФ} \cdot 40\,\text{нФ} \cdot 150\,\text{В}
            }{
                60\,\text{нФ} + 40\,\text{нФ}
            }
            = 3600{,}00\,\text{нКл}
    $
}
\solutionspace{120pt}

\tasknumber{17}%
\task{%
    В вакууме вдоль одной прямой расположены четыре отрицательных заряда так,
    что расстояние между соседними зарядами равно $a$.
    Сделайте рисунок,
    и определите силу, действующую на крайний заряд.
    Модули всех зарядов равны $q$ ($q > 0$).
}
\solutionspace{80pt}

\tasknumber{18}%
\task{%
    Юлия проводит эксперименты c 2 кусками одинаковой алюминиевой проволки, причём второй кусок в семь раз длиннее первого.
    В одном из экспериментов Юлия подаёт на первый кусок проволки напряжение в три раза раз больше, чем на второй.
    Определите отношения в двух проволках в этом эксперименте (второй к первой):
    \begin{itemize}
        \item отношение сил тока,
        \item отношение выделяющихся мощностей.
    \end{itemize}
}
\answer{%
    $\eli_2 / \eli_1 = \frac1{21}, \P_2 / \P_1 = \frac1{21}, $
}

\variantsplitter

\addpersonalvariant{Алина Полканова}

\tasknumber{1}%
\task{%
    Женя стартует на лошади и в течение $t = 4\,\text{c}$ двигается с постоянным ускорением $2\,\frac{\text{м}}{\text{с}^{2}}$.
    Определите
    \begin{itemize}
        \item какую скорость при этом удастся достичь,
        \item какой путь за это время будет пройден,
        \item среднюю скорость за всё время движения, если после начального ускорения продолжить движение равномерно ещё в течение времени $2t$
    \end{itemize}
}
\solutionspace{120pt}

\tasknumber{2}%
\task{%
    Какой путь тело пройдёт за третью секунду после начала свободного падения?
    Какую скорость в начале этой секунды оно имеет?
}
\solutionspace{120pt}

\tasknumber{3}%
\task{%
    Карусель радиусом $4\,\text{м}$ равномерно совершает 10 оборотов в минуту.
    Определите
    \begin{itemize}
        \item период и частоту её обращения,
        \item скорость и ускорение крайних её точек.
    \end{itemize}
}
\solutionspace{80pt}

\tasknumber{4}%
\task{%
    Маша стоит на обрыве над рекой и методично и строго горизонтально кидает в неё камушки.
    За этим всем наблюдает экспериментатор Глюк, который уже выяснил, что камушки падают в реку спустя $1{,}6\,\text{с}$ после броска,
    а вот дальность полёта оценить сложнее: придётся лезть в воду.
    Выручите Глюка и определите:
    \begin{itemize}
        \item высоту обрыва (вместе с ростом Маши).
        \item дальность полёта камушков (по горизонтали) и их скорость при падении, приняв начальную скорость броска равной $v = 12\,\frac{\text{м}}{\text{с}}$.
    \end{itemize}
    Сопротивлением воздуха пренебречь.
}
\solutionspace{120pt}

\tasknumber{5}%
\task{%
    Шесть одинаковых брусков массой $2\,\text{кг}$ каждый лежат на гладком горизонтальном столе.
    Бруски пронумерованы от 1 до 6 и последовательно связаны между собой
    невесомыми нерастяжимыми нитями: 1 со 2, 2 с 3 (ну и с 1) и т.д.
    Экспериментатор Глюк прикладывает постоянную горизонтальную силу $120\,\text{Н}$ к бруску с наименьшим номером.
    С каким ускорением двигается система? Чему равна сила натяжения нити, связывающей бруски 1 и 2?
}
\solutionspace{120pt}

\tasknumber{6}%
\task{%
    Два бруска связаны лёгкой нерастяжимой нитью и перекинуты через неподвижный блок (см.
    рис.).
    Определите силу натяжения нити и ускорения брусков.
    Силами трения пренебречь, массы брусков
    равны $m_1 = 5\,\text{кг}$ и $m_2 = 14\,\text{кг}$.
    % $g = 10\,\frac{\text{м}}{\text{с}^{2}}$.

    \begin{tikzpicture}[x=1.5cm,y=1.5cm,thick]
        \draw
            (-0.4, 0) rectangle (-0.2, 1.2)
            (0.15, 0.5) rectangle (0.45, 1)
            (0, 2) circle [radius=0.3] -- ++(up:0.5)
            (-0.3, 1.2) -- ++(up:0.8)
            (0.3, 1) -- ++(up:1)
            (-0.7, 2.5) -- (0.7, 2.5)
            ;
        \draw[pattern={Lines[angle=51,distance=3pt]},pattern color=black,draw=none] (-0.7, 2.5) rectangle (0.7, 2.75);
        \node [left] (left) at (-0.4, 0.6) { $m_1$ };
        \node [right] (right) at (0.4, 0.75) { $m_2$ };
    \end{tikzpicture}
}
\solutionspace{80pt}

\tasknumber{7}%
\task{%
    Тело массой $2{,}7\,\text{кг}$ лежит на горизонтальной поверхности.
    Коэффициент трения между поверхностью и телом $0{,}25$.
    К телу приложена горизонтальная сила $4{,}5\,\text{Н}$.
    Определите силу трения, действующую на тело, и ускорение тела.
    % $g = 10\,\frac{\text{м}}{\text{с}^{2}}$.
}
\solutionspace{120pt}

\tasknumber{8}%
\task{%
    Определите плотность неизвестного вещества, если известно, что опускании тела из него
    в керосин оно будет плавать и на четверть выступать над поверхностью жидкости.
}
\solutionspace{120pt}

\tasknumber{9}%
\task{%
    	Определите силу, действующую на левую опору однородного горизонтального стержня длиной $l = 9\,\text{м}$
    	и массой $M = 1\,\text{кг}$, к которому подвешен груз массой $m = 4\,\text{кг}$ на расстоянии $4\,\text{м}$ от правого конца (см.
    рис.).

        \begin{tikzpicture}[thick]
            \draw
                (-2, -0.1) rectangle (2, 0.1)
                (-0.5, -0.1) -- (-0.5, -1)
                (-0.7, -1) rectangle (-0.3, -1.3)
           		(-2, -0.1) -- +(0.15,-0.9) -- +(-0.15,-0.9) -- cycle
            	(2, -0.1) -- +(0.15,-0.9) -- +(-0.15,-0.9) -- cycle
            ;
            \draw[pattern={Lines[angle=51,distance=2pt]},pattern color=black,draw=none]
            	(-2.15, -1.15) rectangle +(0.3, 0.15)
            	(2.15, -1.15) rectangle +(-0.3, 0.15)
            ;
            \node [right] (m_small) at (-0.3, -1.15) { $m$ };
            \node [above] (M_big) at (0, 0.1) { $M$ };
        \end{tikzpicture}
}
\solutionspace{80pt}

\tasknumber{10}%
\task{%
    Тонкий однородный шест длиной $3\,\text{м}$ и массой $10\,\text{кг}$ лежит на горизонтальной поверхности.
    \begin{itemize}
        \item Какую минимальную силу надо приложить к одному из его концов, чтобы оторвать его от этой поверхности?
        \item Какую минимальную работу надо совершить, чтобы поставить его на землю в вертикальное положение?
    \end{itemize}
    % Примите $g = 10\,\frac{\text{м}}{\text{с}^{2}}$.
}
\answer{%
    $A = mg\frac l2 = 150\,\text{Дж}$
}
\solutionspace{120pt}

\tasknumber{11}%
\task{%
    Определите работу силы, которая обеспечит подъём тела массой $2\,\text{кг}$ на высоту $5\,\text{м}$ с постоянным ускорением $2\,\frac{\text{м}}{\text{c}^{2}}$.
    % Примите $g = 10\,\frac{\text{м}}{\text{с}^{2}}$.
}
\answer{%
    \begin{align*}
    &\text{Для подъёма:} A = Fh = (mg + ma) h = m(g+a)h, \\
    &\text{Для спуска:} A = -Fh = -(mg - ma) h = -m(g-a)h, \\
    &\text{В результате получаем:} 120\,\text{Дж}.
    \end{align*}
}
\solutionspace{60pt}

\tasknumber{12}%
\task{%
    Тело бросили вертикально вверх со скоростью $20\,\frac{\text{м}}{\text{c}}$.
    На какой высоте кинетическая энергия тела составит половину от потенциальной?
}
\solutionspace{100pt}

\tasknumber{13}%
\task{%
    Плотность воздуха при нормальных условиях равна $1{,}3\,\frac{\text{кг}}{\text{м}^{3}}$.
    Чему равна плотность воздуха
    при температуре $200\celsius$ и давлении $50\,\text{кПа}$?
}
\solutionspace{120pt}

\tasknumber{14}%
\task{%
    Небольшую цилиндрическую пробирку с воздухом погружают на некоторую глубину в глубокое пресное озеро,
    после чего воздух занимает в ней лишь четвертую часть от общего объема.
    Определите глубину, на которую погрузили пробирку.
    Температуру считать постоянной $T = 288\,\text{К}$, давлением паров воды пренебречь,
    атмосферное давление принять равным $p_{\text{aтм}} = 100\,\text{кПа}$.
}
\answer{%
    \begin{align*}
    T\text{— const} &\implies P_1V_1 = \nu RT = P_2V_2.
    \\
    V_2 = \frac 14 V_1 &\implies P_1V_1 = P_2 \cdot \frac 14V_1 \implies P_2 = 4P_1 = 4p_{\text{aтм}}.
    \\
    P_2 = p_{\text{aтм}} + \rho_{\text{в}} g h \implies h = \frac{P_2 - p_{\text{aтм}}}{\rho_{\text{в}} g} &= \frac{4p_{\text{aтм}} - p_{\text{aтм}}}{\rho_{\text{в}} g} = \frac{3 \cdot p_{\text{aтм}}}{\rho_{\text{в}} g} =  \\
     &= \frac{3 \cdot 100\,\text{кПа}}{1000\,\frac{\text{кг}}{\text{м}^{3}} \cdot  10\,\frac{\text{м}}{\text{с}^{2}}} \approx 30\,\text{м}.
    \end{align*}
}
\solutionspace{120pt}

\tasknumber{15}%
\task{%
    Газу сообщили некоторое количество теплоты,
    при этом треть его он потратил на совершение работы,
    одновременно увеличив свою внутреннюю энергию на $2400\,\text{Дж}$.
    Определите количество теплоты, сообщённое газу.
}
\answer{%
    \begin{align*}
    Q &= A' + \Delta U, A' = \frac 13 Q \implies Q \cdot \cbr{1 - \frac 13} = \Delta U \implies Q = \frac{\Delta U}{1 - \frac 13} = \frac{ 2400\,\text{Дж} }{1 - \frac 13} \approx 3600\,\text{Дж}.
    \\
    A' &= \frac 13 Q
        = \frac 13 \cdot \frac{\Delta U}{1 - \frac 13}
        = \frac{\Delta U}{3 - 1}
        = \frac{ 2400\,\text{Дж} }{3 - 1} \approx 1200\,\text{Дж}.
    \end{align*}
}
\solutionspace{60pt}

\tasknumber{16}%
\task{%
    Два конденсатора ёмкостей $C_1 = 40\,\text{нФ}$ и $C_2 = 60\,\text{нФ}$ последовательно подключают
    к источнику напряжения $V = 150\,\text{В}$ (см.
    рис.).
    % Определите заряды каждого из конденсаторов.
    Определите заряд первого конденсатора.

    \begin{tikzpicture}[circuit ee IEC, semithick]
        \draw  (0, 0) to [capacitor={info={$C_1$}}] (1, 0)
                       to [capacitor={info={$C_2$}}] (2, 0)
        ;
        % \draw [-o] (0, 0) -- ++(-0.5, 0) node[left] {$-$};
        % \draw [-o] (2, 0) -- ++(0.5, 0) node[right] {$+$};
        \draw [-o] (0, 0) -- ++(-0.5, 0) node[left] {};
        \draw [-o] (2, 0) -- ++(0.5, 0) node[right] {};
    \end{tikzpicture}
}
\answer{%
    $
        Q_1
            = Q_2
            = CV
            = \frac{ V }{\frac1{C_1} + \frac1{C_2}}
            = \frac{C_1C_2V}{C_1 + C_2}
            = \frac{
                40\,\text{нФ} \cdot 60\,\text{нФ} \cdot 150\,\text{В}
            }{
                40\,\text{нФ} + 60\,\text{нФ}
            }
            = 3600{,}00\,\text{нКл}
    $
}
\solutionspace{120pt}

\tasknumber{17}%
\task{%
    В вакууме вдоль одной прямой расположены три отрицательных заряда так,
    что расстояние между соседними зарядами равно $l$.
    Сделайте рисунок,
    и определите силу, действующую на крайний заряд.
    Модули всех зарядов равны $q$ ($q > 0$).
}
\solutionspace{80pt}

\tasknumber{18}%
\task{%
    Юлия проводит эксперименты c 2 кусками одинаковой алюминиевой проволки, причём второй кусок в три раза длиннее первого.
    В одном из экспериментов Юлия подаёт на первый кусок проволки напряжение в два раза раз больше, чем на второй.
    Определите отношения в двух проволках в этом эксперименте (второй к первой):
    \begin{itemize}
        \item отношение сил тока,
        \item отношение выделяющихся мощностей.
    \end{itemize}
}
\answer{%
    $\eli_2 / \eli_1 = \frac16, \P_2 / \P_1 = \frac16, $
}

\variantsplitter

\addpersonalvariant{Сергей Пономарёв}

\tasknumber{1}%
\task{%
    Валя стартует на велосипеде и в течение $t = 2\,\text{c}$ двигается с постоянным ускорением $0{,}5\,\frac{\text{м}}{\text{с}^{2}}$.
    Определите
    \begin{itemize}
        \item какую скорость при этом удастся достичь,
        \item какой путь за это время будет пройден,
        \item среднюю скорость за всё время движения, если после начального ускорения продолжить движение равномерно ещё в течение времени $3t$
    \end{itemize}
}
\solutionspace{120pt}

\tasknumber{2}%
\task{%
    Какой путь тело пройдёт за вторую секунду после начала свободного падения?
    Какую скорость в начале этой секунды оно имеет?
}
\solutionspace{120pt}

\tasknumber{3}%
\task{%
    Карусель диаметром $5\,\text{м}$ равномерно совершает 6 оборотов в минуту.
    Определите
    \begin{itemize}
        \item период и частоту её обращения,
        \item скорость и ускорение крайних её точек.
    \end{itemize}
}
\solutionspace{80pt}

\tasknumber{4}%
\task{%
    Даша стоит на обрыве над рекой и методично и строго горизонтально кидает в неё камушки.
    За этим всем наблюдает экспериментатор Глюк, который уже выяснил, что камушки падают в реку спустя $1{,}3\,\text{с}$ после броска,
    а вот дальность полёта оценить сложнее: придётся лезть в воду.
    Выручите Глюка и определите:
    \begin{itemize}
        \item высоту обрыва (вместе с ростом Даши).
        \item дальность полёта камушков (по горизонтали) и их скорость при падении, приняв начальную скорость броска равной $v = 15\,\frac{\text{м}}{\text{с}}$.
    \end{itemize}
    Сопротивлением воздуха пренебречь.
}
\solutionspace{120pt}

\tasknumber{5}%
\task{%
    Шесть одинаковых брусков массой $2\,\text{кг}$ каждый лежат на гладком горизонтальном столе.
    Бруски пронумерованы от 1 до 6 и последовательно связаны между собой
    невесомыми нерастяжимыми нитями: 1 со 2, 2 с 3 (ну и с 1) и т.д.
    Экспериментатор Глюк прикладывает постоянную горизонтальную силу $120\,\text{Н}$ к бруску с наибольшим номером.
    С каким ускорением двигается система? Чему равна сила натяжения нити, связывающей бруски 3 и 4?
}
\solutionspace{120pt}

\tasknumber{6}%
\task{%
    Два бруска связаны лёгкой нерастяжимой нитью и перекинуты через неподвижный блок (см.
    рис.).
    Определите силу натяжения нити и ускорения брусков.
    Силами трения пренебречь, массы брусков
    равны $m_1 = 5\,\text{кг}$ и $m_2 = 10\,\text{кг}$.
    % $g = 10\,\frac{\text{м}}{\text{с}^{2}}$.

    \begin{tikzpicture}[x=1.5cm,y=1.5cm,thick]
        \draw
            (-0.4, 0) rectangle (-0.2, 1.2)
            (0.15, 0.5) rectangle (0.45, 1)
            (0, 2) circle [radius=0.3] -- ++(up:0.5)
            (-0.3, 1.2) -- ++(up:0.8)
            (0.3, 1) -- ++(up:1)
            (-0.7, 2.5) -- (0.7, 2.5)
            ;
        \draw[pattern={Lines[angle=51,distance=3pt]},pattern color=black,draw=none] (-0.7, 2.5) rectangle (0.7, 2.75);
        \node [left] (left) at (-0.4, 0.6) { $m_1$ };
        \node [right] (right) at (0.4, 0.75) { $m_2$ };
    \end{tikzpicture}
}
\solutionspace{80pt}

\tasknumber{7}%
\task{%
    Тело массой $2{,}7\,\text{кг}$ лежит на горизонтальной поверхности.
    Коэффициент трения между поверхностью и телом $0{,}25$.
    К телу приложена горизонтальная сила $3{,}5\,\text{Н}$.
    Определите силу трения, действующую на тело, и ускорение тела.
    % $g = 10\,\frac{\text{м}}{\text{с}^{2}}$.
}
\solutionspace{120pt}

\tasknumber{8}%
\task{%
    Определите плотность неизвестного вещества, если известно, что опускании тела из него
    в керосин оно будет плавать и на четверть выступать над поверхностью жидкости.
}
\solutionspace{120pt}

\tasknumber{9}%
\task{%
    	Определите силу, действующую на правую опору однородного горизонтального стержня длиной $l = 3\,\text{м}$
    	и массой $M = 1\,\text{кг}$, к которому подвешен груз массой $m = 3\,\text{кг}$ на расстоянии $2\,\text{м}$ от правого конца (см.
    рис.).

        \begin{tikzpicture}[thick]
            \draw
                (-2, -0.1) rectangle (2, 0.1)
                (-0.5, -0.1) -- (-0.5, -1)
                (-0.7, -1) rectangle (-0.3, -1.3)
           		(-2, -0.1) -- +(0.15,-0.9) -- +(-0.15,-0.9) -- cycle
            	(2, -0.1) -- +(0.15,-0.9) -- +(-0.15,-0.9) -- cycle
            ;
            \draw[pattern={Lines[angle=51,distance=2pt]},pattern color=black,draw=none]
            	(-2.15, -1.15) rectangle +(0.3, 0.15)
            	(2.15, -1.15) rectangle +(-0.3, 0.15)
            ;
            \node [right] (m_small) at (-0.3, -1.15) { $m$ };
            \node [above] (M_big) at (0, 0.1) { $M$ };
        \end{tikzpicture}
}
\solutionspace{80pt}

\tasknumber{10}%
\task{%
    Тонкий однородный шест длиной $3\,\text{м}$ и массой $30\,\text{кг}$ лежит на горизонтальной поверхности.
    \begin{itemize}
        \item Какую минимальную силу надо приложить к одному из его концов, чтобы оторвать его от этой поверхности?
        \item Какую минимальную работу надо совершить, чтобы поставить его на землю в вертикальное положение?
    \end{itemize}
    % Примите $g = 10\,\frac{\text{м}}{\text{с}^{2}}$.
}
\answer{%
    $A = mg\frac l2 = 450\,\text{Дж}$
}
\solutionspace{120pt}

\tasknumber{11}%
\task{%
    Определите работу силы, которая обеспечит спуск тела массой $5\,\text{кг}$ на высоту $10\,\text{м}$ с постоянным ускорением $2\,\frac{\text{м}}{\text{c}^{2}}$.
    % Примите $g = 10\,\frac{\text{м}}{\text{с}^{2}}$.
}
\answer{%
    \begin{align*}
    &\text{Для подъёма:} A = Fh = (mg + ma) h = m(g+a)h, \\
    &\text{Для спуска:} A = -Fh = -(mg - ma) h = -m(g-a)h, \\
    &\text{В результате получаем:} -400\,\text{Дж}.
    \end{align*}
}
\solutionspace{60pt}

\tasknumber{12}%
\task{%
    Тело бросили вертикально вверх со скоростью $14\,\frac{\text{м}}{\text{c}}$.
    На какой высоте кинетическая энергия тела составит половину от потенциальной?
}
\solutionspace{100pt}

\tasknumber{13}%
\task{%
    Плотность воздуха при нормальных условиях равна $1{,}3\,\frac{\text{кг}}{\text{м}^{3}}$.
    Чему равна плотность воздуха
    при температуре $150\celsius$ и давлении $120\,\text{кПа}$?
}
\solutionspace{120pt}

\tasknumber{14}%
\task{%
    Небольшую цилиндрическую пробирку с воздухом погружают на некоторую глубину в глубокое пресное озеро,
    после чего воздух занимает в ней лишь четвертую часть от общего объема.
    Определите глубину, на которую погрузили пробирку.
    Температуру считать постоянной $T = 279\,\text{К}$, давлением паров воды пренебречь,
    атмосферное давление принять равным $p_{\text{aтм}} = 100\,\text{кПа}$.
}
\answer{%
    \begin{align*}
    T\text{— const} &\implies P_1V_1 = \nu RT = P_2V_2.
    \\
    V_2 = \frac 14 V_1 &\implies P_1V_1 = P_2 \cdot \frac 14V_1 \implies P_2 = 4P_1 = 4p_{\text{aтм}}.
    \\
    P_2 = p_{\text{aтм}} + \rho_{\text{в}} g h \implies h = \frac{P_2 - p_{\text{aтм}}}{\rho_{\text{в}} g} &= \frac{4p_{\text{aтм}} - p_{\text{aтм}}}{\rho_{\text{в}} g} = \frac{3 \cdot p_{\text{aтм}}}{\rho_{\text{в}} g} =  \\
     &= \frac{3 \cdot 100\,\text{кПа}}{1000\,\frac{\text{кг}}{\text{м}^{3}} \cdot  10\,\frac{\text{м}}{\text{с}^{2}}} \approx 30\,\text{м}.
    \end{align*}
}
\solutionspace{120pt}

\tasknumber{15}%
\task{%
    Газу сообщили некоторое количество теплоты,
    при этом половину его он потратил на совершение работы,
    одновременно увеличив свою внутреннюю энергию на $1500\,\text{Дж}$.
    Определите количество теплоты, сообщённое газу.
}
\answer{%
    \begin{align*}
    Q &= A' + \Delta U, A' = \frac 12 Q \implies Q \cdot \cbr{1 - \frac 12} = \Delta U \implies Q = \frac{\Delta U}{1 - \frac 12} = \frac{ 1500\,\text{Дж} }{1 - \frac 12} \approx 3000\,\text{Дж}.
    \\
    A' &= \frac 12 Q
        = \frac 12 \cdot \frac{\Delta U}{1 - \frac 12}
        = \frac{\Delta U}{2 - 1}
        = \frac{ 1500\,\text{Дж} }{2 - 1} \approx 1500\,\text{Дж}.
    \end{align*}
}
\solutionspace{60pt}

\tasknumber{16}%
\task{%
    Два конденсатора ёмкостей $C_1 = 30\,\text{нФ}$ и $C_2 = 40\,\text{нФ}$ последовательно подключают
    к источнику напряжения $V = 450\,\text{В}$ (см.
    рис.).
    % Определите заряды каждого из конденсаторов.
    Определите заряд первого конденсатора.

    \begin{tikzpicture}[circuit ee IEC, semithick]
        \draw  (0, 0) to [capacitor={info={$C_1$}}] (1, 0)
                       to [capacitor={info={$C_2$}}] (2, 0)
        ;
        % \draw [-o] (0, 0) -- ++(-0.5, 0) node[left] {$-$};
        % \draw [-o] (2, 0) -- ++(0.5, 0) node[right] {$+$};
        \draw [-o] (0, 0) -- ++(-0.5, 0) node[left] {};
        \draw [-o] (2, 0) -- ++(0.5, 0) node[right] {};
    \end{tikzpicture}
}
\answer{%
    $
        Q_1
            = Q_2
            = CV
            = \frac{ V }{\frac1{C_1} + \frac1{C_2}}
            = \frac{C_1C_2V}{C_1 + C_2}
            = \frac{
                30\,\text{нФ} \cdot 40\,\text{нФ} \cdot 450\,\text{В}
            }{
                30\,\text{нФ} + 40\,\text{нФ}
            }
            = 7714{,}29\,\text{нКл}
    $
}
\solutionspace{120pt}

\tasknumber{17}%
\task{%
    В вакууме вдоль одной прямой расположены четыре отрицательных заряда так,
    что расстояние между соседними зарядами равно $l$.
    Сделайте рисунок,
    и определите силу, действующую на крайний заряд.
    Модули всех зарядов равны $Q$ ($Q > 0$).
}
\solutionspace{80pt}

\tasknumber{18}%
\task{%
    Юлия проводит эксперименты c 2 кусками одинаковой медной проволки, причём второй кусок в шесть раз длиннее первого.
    В одном из экспериментов Юлия подаёт на первый кусок проволки напряжение в два раза раз больше, чем на второй.
    Определите отношения в двух проволках в этом эксперименте (второй к первой):
    \begin{itemize}
        \item отношение сил тока,
        \item отношение выделяющихся мощностей.
    \end{itemize}
}
\answer{%
    $\eli_2 / \eli_1 = \frac1{12}, \P_2 / \P_1 = \frac1{12}, $
}

\variantsplitter

\addpersonalvariant{Егор Свистушкин}

\tasknumber{1}%
\task{%
    Саша стартует на лошади и в течение $t = 10\,\text{c}$ двигается с постоянным ускорением $2\,\frac{\text{м}}{\text{с}^{2}}$.
    Определите
    \begin{itemize}
        \item какую скорость при этом удастся достичь,
        \item какой путь за это время будет пройден,
        \item среднюю скорость за всё время движения, если после начального ускорения продолжить движение равномерно ещё в течение времени $3t$
    \end{itemize}
}
\solutionspace{120pt}

\tasknumber{2}%
\task{%
    Какой путь тело пройдёт за пятую секунду после начала свободного падения?
    Какую скорость в конце этой секунды оно имеет?
}
\solutionspace{120pt}

\tasknumber{3}%
\task{%
    Карусель радиусом $3\,\text{м}$ равномерно совершает 6 оборотов в минуту.
    Определите
    \begin{itemize}
        \item период и частоту её обращения,
        \item скорость и ускорение крайних её точек.
    \end{itemize}
}
\solutionspace{80pt}

\tasknumber{4}%
\task{%
    Миша стоит на обрыве над рекой и методично и строго горизонтально кидает в неё камушки.
    За этим всем наблюдает экспериментатор Глюк, который уже выяснил, что камушки падают в реку спустя $1{,}7\,\text{с}$ после броска,
    а вот дальность полёта оценить сложнее: придётся лезть в воду.
    Выручите Глюка и определите:
    \begin{itemize}
        \item высоту обрыва (вместе с ростом Миши).
        \item дальность полёта камушков (по горизонтали) и их скорость при падении, приняв начальную скорость броска равной $v = 17\,\frac{\text{м}}{\text{с}}$.
    \end{itemize}
    Сопротивлением воздуха пренебречь.
}
\solutionspace{120pt}

\tasknumber{5}%
\task{%
    Пять одинаковых брусков массой $2\,\text{кг}$ каждый лежат на гладком горизонтальном столе.
    Бруски пронумерованы от 1 до 5 и последовательно связаны между собой
    невесомыми нерастяжимыми нитями: 1 со 2, 2 с 3 (ну и с 1) и т.д.
    Экспериментатор Глюк прикладывает постоянную горизонтальную силу $60\,\text{Н}$ к бруску с наибольшим номером.
    С каким ускорением двигается система? Чему равна сила натяжения нити, связывающей бруски 2 и 3?
}
\solutionspace{120pt}

\tasknumber{6}%
\task{%
    Два бруска связаны лёгкой нерастяжимой нитью и перекинуты через неподвижный блок (см.
    рис.).
    Определите силу натяжения нити и ускорения брусков.
    Силами трения пренебречь, массы брусков
    равны $m_1 = 11\,\text{кг}$ и $m_2 = 10\,\text{кг}$.
    % $g = 10\,\frac{\text{м}}{\text{с}^{2}}$.

    \begin{tikzpicture}[x=1.5cm,y=1.5cm,thick]
        \draw
            (-0.4, 0) rectangle (-0.2, 1.2)
            (0.15, 0.5) rectangle (0.45, 1)
            (0, 2) circle [radius=0.3] -- ++(up:0.5)
            (-0.3, 1.2) -- ++(up:0.8)
            (0.3, 1) -- ++(up:1)
            (-0.7, 2.5) -- (0.7, 2.5)
            ;
        \draw[pattern={Lines[angle=51,distance=3pt]},pattern color=black,draw=none] (-0.7, 2.5) rectangle (0.7, 2.75);
        \node [left] (left) at (-0.4, 0.6) { $m_1$ };
        \node [right] (right) at (0.4, 0.75) { $m_2$ };
    \end{tikzpicture}
}
\solutionspace{80pt}

\tasknumber{7}%
\task{%
    Тело массой $1{,}4\,\text{кг}$ лежит на горизонтальной поверхности.
    Коэффициент трения между поверхностью и телом $0{,}15$.
    К телу приложена горизонтальная сила $3{,}5\,\text{Н}$.
    Определите силу трения, действующую на тело, и ускорение тела.
    % $g = 10\,\frac{\text{м}}{\text{с}^{2}}$.
}
\solutionspace{120pt}

\tasknumber{8}%
\task{%
    Определите плотность неизвестного вещества, если известно, что опускании тела из него
    в керосин оно будет плавать и на половину выступать над поверхностью жидкости.
}
\solutionspace{120pt}

\tasknumber{9}%
\task{%
    	Определите силу, действующую на правую опору однородного горизонтального стержня длиной $l = 5\,\text{м}$
    	и массой $M = 1\,\text{кг}$, к которому подвешен груз массой $m = 2\,\text{кг}$ на расстоянии $2\,\text{м}$ от правого конца (см.
    рис.).

        \begin{tikzpicture}[thick]
            \draw
                (-2, -0.1) rectangle (2, 0.1)
                (-0.5, -0.1) -- (-0.5, -1)
                (-0.7, -1) rectangle (-0.3, -1.3)
           		(-2, -0.1) -- +(0.15,-0.9) -- +(-0.15,-0.9) -- cycle
            	(2, -0.1) -- +(0.15,-0.9) -- +(-0.15,-0.9) -- cycle
            ;
            \draw[pattern={Lines[angle=51,distance=2pt]},pattern color=black,draw=none]
            	(-2.15, -1.15) rectangle +(0.3, 0.15)
            	(2.15, -1.15) rectangle +(-0.3, 0.15)
            ;
            \node [right] (m_small) at (-0.3, -1.15) { $m$ };
            \node [above] (M_big) at (0, 0.1) { $M$ };
        \end{tikzpicture}
}
\solutionspace{80pt}

\tasknumber{10}%
\task{%
    Тонкий однородный шест длиной $1\,\text{м}$ и массой $10\,\text{кг}$ лежит на горизонтальной поверхности.
    \begin{itemize}
        \item Какую минимальную силу надо приложить к одному из его концов, чтобы оторвать его от этой поверхности?
        \item Какую минимальную работу надо совершить, чтобы поставить его на землю в вертикальное положение?
    \end{itemize}
    % Примите $g = 10\,\frac{\text{м}}{\text{с}^{2}}$.
}
\answer{%
    $A = mg\frac l2 = 50\,\text{Дж}$
}
\solutionspace{120pt}

\tasknumber{11}%
\task{%
    Определите работу силы, которая обеспечит спуск тела массой $5\,\text{кг}$ на высоту $10\,\text{м}$ с постоянным ускорением $6\,\frac{\text{м}}{\text{c}^{2}}$.
    % Примите $g = 10\,\frac{\text{м}}{\text{с}^{2}}$.
}
\answer{%
    \begin{align*}
    &\text{Для подъёма:} A = Fh = (mg + ma) h = m(g+a)h, \\
    &\text{Для спуска:} A = -Fh = -(mg - ma) h = -m(g-a)h, \\
    &\text{В результате получаем:} -200\,\text{Дж}.
    \end{align*}
}
\solutionspace{60pt}

\tasknumber{12}%
\task{%
    Тело бросили вертикально вверх со скоростью $10\,\frac{\text{м}}{\text{c}}$.
    На какой высоте кинетическая энергия тела составит треть от потенциальной?
}
\solutionspace{100pt}

\tasknumber{13}%
\task{%
    Плотность воздуха при нормальных условиях равна $1{,}3\,\frac{\text{кг}}{\text{м}^{3}}$.
    Чему равна плотность воздуха
    при температуре $200\celsius$ и давлении $150\,\text{кПа}$?
}
\solutionspace{120pt}

\tasknumber{14}%
\task{%
    Небольшую цилиндрическую пробирку с воздухом погружают на некоторую глубину в глубокое пресное озеро,
    после чего воздух занимает в ней лишь третью часть от общего объема.
    Определите глубину, на которую погрузили пробирку.
    Температуру считать постоянной $T = 282\,\text{К}$, давлением паров воды пренебречь,
    атмосферное давление принять равным $p_{\text{aтм}} = 100\,\text{кПа}$.
}
\answer{%
    \begin{align*}
    T\text{— const} &\implies P_1V_1 = \nu RT = P_2V_2.
    \\
    V_2 = \frac 13 V_1 &\implies P_1V_1 = P_2 \cdot \frac 13V_1 \implies P_2 = 3P_1 = 3p_{\text{aтм}}.
    \\
    P_2 = p_{\text{aтм}} + \rho_{\text{в}} g h \implies h = \frac{P_2 - p_{\text{aтм}}}{\rho_{\text{в}} g} &= \frac{3p_{\text{aтм}} - p_{\text{aтм}}}{\rho_{\text{в}} g} = \frac{2 \cdot p_{\text{aтм}}}{\rho_{\text{в}} g} =  \\
     &= \frac{2 \cdot 100\,\text{кПа}}{1000\,\frac{\text{кг}}{\text{м}^{3}} \cdot  10\,\frac{\text{м}}{\text{с}^{2}}} \approx 20\,\text{м}.
    \end{align*}
}
\solutionspace{120pt}

\tasknumber{15}%
\task{%
    Газу сообщили некоторое количество теплоты,
    при этом половину его он потратил на совершение работы,
    одновременно увеличив свою внутреннюю энергию на $1200\,\text{Дж}$.
    Определите количество теплоты, сообщённое газу.
}
\answer{%
    \begin{align*}
    Q &= A' + \Delta U, A' = \frac 12 Q \implies Q \cdot \cbr{1 - \frac 12} = \Delta U \implies Q = \frac{\Delta U}{1 - \frac 12} = \frac{ 1200\,\text{Дж} }{1 - \frac 12} \approx 2400\,\text{Дж}.
    \\
    A' &= \frac 12 Q
        = \frac 12 \cdot \frac{\Delta U}{1 - \frac 12}
        = \frac{\Delta U}{2 - 1}
        = \frac{ 1200\,\text{Дж} }{2 - 1} \approx 1200\,\text{Дж}.
    \end{align*}
}
\solutionspace{60pt}

\tasknumber{16}%
\task{%
    Два конденсатора ёмкостей $C_1 = 30\,\text{нФ}$ и $C_2 = 20\,\text{нФ}$ последовательно подключают
    к источнику напряжения $U = 200\,\text{В}$ (см.
    рис.).
    % Определите заряды каждого из конденсаторов.
    Определите заряд второго конденсатора.

    \begin{tikzpicture}[circuit ee IEC, semithick]
        \draw  (0, 0) to [capacitor={info={$C_1$}}] (1, 0)
                       to [capacitor={info={$C_2$}}] (2, 0)
        ;
        % \draw [-o] (0, 0) -- ++(-0.5, 0) node[left] {$-$};
        % \draw [-o] (2, 0) -- ++(0.5, 0) node[right] {$+$};
        \draw [-o] (0, 0) -- ++(-0.5, 0) node[left] {};
        \draw [-o] (2, 0) -- ++(0.5, 0) node[right] {};
    \end{tikzpicture}
}
\answer{%
    $
        Q_1
            = Q_2
            = CU
            = \frac{ U }{\frac1{C_1} + \frac1{C_2}}
            = \frac{C_1C_2U}{C_1 + C_2}
            = \frac{
                30\,\text{нФ} \cdot 20\,\text{нФ} \cdot 200\,\text{В}
            }{
                30\,\text{нФ} + 20\,\text{нФ}
            }
            = 2400{,}00\,\text{нКл}
    $
}
\solutionspace{120pt}

\tasknumber{17}%
\task{%
    В вакууме вдоль одной прямой расположены три отрицательных заряда так,
    что расстояние между соседними зарядами равно $l$.
    Сделайте рисунок,
    и определите силу, действующую на крайний заряд.
    Модули всех зарядов равны $Q$ ($Q > 0$).
}
\solutionspace{80pt}

\tasknumber{18}%
\task{%
    Юлия проводит эксперименты c 2 кусками одинаковой медной проволки, причём второй кусок в четыре раза длиннее первого.
    В одном из экспериментов Юлия подаёт на первый кусок проволки напряжение в три раза раз больше, чем на второй.
    Определите отношения в двух проволках в этом эксперименте (второй к первой):
    \begin{itemize}
        \item отношение сил тока,
        \item отношение выделяющихся мощностей.
    \end{itemize}
}
\answer{%
    $\eli_2 / \eli_1 = \frac1{12}, \P_2 / \P_1 = \frac1{12}, $
}

\variantsplitter

\addpersonalvariant{Дмитрий Соколов}

\tasknumber{1}%
\task{%
    Валя стартует на лошади и в течение $t = 5\,\text{c}$ двигается с постоянным ускорением $1{,}5\,\frac{\text{м}}{\text{с}^{2}}$.
    Определите
    \begin{itemize}
        \item какую скорость при этом удастся достичь,
        \item какой путь за это время будет пройден,
        \item среднюю скорость за всё время движения, если после начального ускорения продолжить движение равномерно ещё в течение времени $3t$
    \end{itemize}
}
\solutionspace{120pt}

\tasknumber{2}%
\task{%
    Какой путь тело пройдёт за третью секунду после начала свободного падения?
    Какую скорость в конце этой секунды оно имеет?
}
\solutionspace{120pt}

\tasknumber{3}%
\task{%
    Карусель диаметром $2\,\text{м}$ равномерно совершает 6 оборотов в минуту.
    Определите
    \begin{itemize}
        \item период и частоту её обращения,
        \item скорость и ускорение крайних её точек.
    \end{itemize}
}
\solutionspace{80pt}

\tasknumber{4}%
\task{%
    Паша стоит на обрыве над рекой и методично и строго горизонтально кидает в неё камушки.
    За этим всем наблюдает экспериментатор Глюк, который уже выяснил, что камушки падают в реку спустя $1{,}7\,\text{с}$ после броска,
    а вот дальность полёта оценить сложнее: придётся лезть в воду.
    Выручите Глюка и определите:
    \begin{itemize}
        \item высоту обрыва (вместе с ростом Паши).
        \item дальность полёта камушков (по горизонтали) и их скорость при падении, приняв начальную скорость броска равной $v = 17\,\frac{\text{м}}{\text{с}}$.
    \end{itemize}
    Сопротивлением воздуха пренебречь.
}
\solutionspace{120pt}

\tasknumber{5}%
\task{%
    Шесть одинаковых брусков массой $2\,\text{кг}$ каждый лежат на гладком горизонтальном столе.
    Бруски пронумерованы от 1 до 6 и последовательно связаны между собой
    невесомыми нерастяжимыми нитями: 1 со 2, 2 с 3 (ну и с 1) и т.д.
    Экспериментатор Глюк прикладывает постоянную горизонтальную силу $90\,\text{Н}$ к бруску с наибольшим номером.
    С каким ускорением двигается система? Чему равна сила натяжения нити, связывающей бруски 2 и 3?
}
\solutionspace{120pt}

\tasknumber{6}%
\task{%
    Два бруска связаны лёгкой нерастяжимой нитью и перекинуты через неподвижный блок (см.
    рис.).
    Определите силу натяжения нити и ускорения брусков.
    Силами трения пренебречь, массы брусков
    равны $m_1 = 11\,\text{кг}$ и $m_2 = 14\,\text{кг}$.
    % $g = 10\,\frac{\text{м}}{\text{с}^{2}}$.

    \begin{tikzpicture}[x=1.5cm,y=1.5cm,thick]
        \draw
            (-0.4, 0) rectangle (-0.2, 1.2)
            (0.15, 0.5) rectangle (0.45, 1)
            (0, 2) circle [radius=0.3] -- ++(up:0.5)
            (-0.3, 1.2) -- ++(up:0.8)
            (0.3, 1) -- ++(up:1)
            (-0.7, 2.5) -- (0.7, 2.5)
            ;
        \draw[pattern={Lines[angle=51,distance=3pt]},pattern color=black,draw=none] (-0.7, 2.5) rectangle (0.7, 2.75);
        \node [left] (left) at (-0.4, 0.6) { $m_1$ };
        \node [right] (right) at (0.4, 0.75) { $m_2$ };
    \end{tikzpicture}
}
\solutionspace{80pt}

\tasknumber{7}%
\task{%
    Тело массой $1{,}4\,\text{кг}$ лежит на горизонтальной поверхности.
    Коэффициент трения между поверхностью и телом $0{,}25$.
    К телу приложена горизонтальная сила $3{,}5\,\text{Н}$.
    Определите силу трения, действующую на тело, и ускорение тела.
    % $g = 10\,\frac{\text{м}}{\text{с}^{2}}$.
}
\solutionspace{120pt}

\tasknumber{8}%
\task{%
    Определите плотность неизвестного вещества, если известно, что опускании тела из него
    в подсолнечное масло оно будет плавать и на треть выступать над поверхностью жидкости.
}
\solutionspace{120pt}

\tasknumber{9}%
\task{%
    	Определите силу, действующую на левую опору однородного горизонтального стержня длиной $l = 5\,\text{м}$
    	и массой $M = 5\,\text{кг}$, к которому подвешен груз массой $m = 2\,\text{кг}$ на расстоянии $4\,\text{м}$ от правого конца (см.
    рис.).

        \begin{tikzpicture}[thick]
            \draw
                (-2, -0.1) rectangle (2, 0.1)
                (-0.5, -0.1) -- (-0.5, -1)
                (-0.7, -1) rectangle (-0.3, -1.3)
           		(-2, -0.1) -- +(0.15,-0.9) -- +(-0.15,-0.9) -- cycle
            	(2, -0.1) -- +(0.15,-0.9) -- +(-0.15,-0.9) -- cycle
            ;
            \draw[pattern={Lines[angle=51,distance=2pt]},pattern color=black,draw=none]
            	(-2.15, -1.15) rectangle +(0.3, 0.15)
            	(2.15, -1.15) rectangle +(-0.3, 0.15)
            ;
            \node [right] (m_small) at (-0.3, -1.15) { $m$ };
            \node [above] (M_big) at (0, 0.1) { $M$ };
        \end{tikzpicture}
}
\solutionspace{80pt}

\tasknumber{10}%
\task{%
    Тонкий однородный лом длиной $3\,\text{м}$ и массой $10\,\text{кг}$ лежит на горизонтальной поверхности.
    \begin{itemize}
        \item Какую минимальную силу надо приложить к одному из его концов, чтобы оторвать его от этой поверхности?
        \item Какую минимальную работу надо совершить, чтобы поставить его на землю в вертикальное положение?
    \end{itemize}
    % Примите $g = 10\,\frac{\text{м}}{\text{с}^{2}}$.
}
\answer{%
    $A = mg\frac l2 = 150\,\text{Дж}$
}
\solutionspace{120pt}

\tasknumber{11}%
\task{%
    Определите работу силы, которая обеспечит подъём тела массой $5\,\text{кг}$ на высоту $5\,\text{м}$ с постоянным ускорением $3\,\frac{\text{м}}{\text{c}^{2}}$.
    % Примите $g = 10\,\frac{\text{м}}{\text{с}^{2}}$.
}
\answer{%
    \begin{align*}
    &\text{Для подъёма:} A = Fh = (mg + ma) h = m(g+a)h, \\
    &\text{Для спуска:} A = -Fh = -(mg - ma) h = -m(g-a)h, \\
    &\text{В результате получаем:} 325\,\text{Дж}.
    \end{align*}
}
\solutionspace{60pt}

\tasknumber{12}%
\task{%
    Тело бросили вертикально вверх со скоростью $14\,\frac{\text{м}}{\text{c}}$.
    На какой высоте кинетическая энергия тела составит треть от потенциальной?
}
\solutionspace{100pt}

\tasknumber{13}%
\task{%
    Плотность воздуха при нормальных условиях равна $1{,}3\,\frac{\text{кг}}{\text{м}^{3}}$.
    Чему равна плотность воздуха
    при температуре $150\celsius$ и давлении $150\,\text{кПа}$?
}
\solutionspace{120pt}

\tasknumber{14}%
\task{%
    Небольшую цилиндрическую пробирку с воздухом погружают на некоторую глубину в глубокое пресное озеро,
    после чего воздух занимает в ней лишь шестую часть от общего объема.
    Определите глубину, на которую погрузили пробирку.
    Температуру считать постоянной $T = 279\,\text{К}$, давлением паров воды пренебречь,
    атмосферное давление принять равным $p_{\text{aтм}} = 100\,\text{кПа}$.
}
\answer{%
    \begin{align*}
    T\text{— const} &\implies P_1V_1 = \nu RT = P_2V_2.
    \\
    V_2 = \frac 16 V_1 &\implies P_1V_1 = P_2 \cdot \frac 16V_1 \implies P_2 = 6P_1 = 6p_{\text{aтм}}.
    \\
    P_2 = p_{\text{aтм}} + \rho_{\text{в}} g h \implies h = \frac{P_2 - p_{\text{aтм}}}{\rho_{\text{в}} g} &= \frac{6p_{\text{aтм}} - p_{\text{aтм}}}{\rho_{\text{в}} g} = \frac{5 \cdot p_{\text{aтм}}}{\rho_{\text{в}} g} =  \\
     &= \frac{5 \cdot 100\,\text{кПа}}{1000\,\frac{\text{кг}}{\text{м}^{3}} \cdot  10\,\frac{\text{м}}{\text{с}^{2}}} \approx 50\,\text{м}.
    \end{align*}
}
\solutionspace{120pt}

\tasknumber{15}%
\task{%
    Газу сообщили некоторое количество теплоты,
    при этом треть его он потратил на совершение работы,
    одновременно увеличив свою внутреннюю энергию на $1500\,\text{Дж}$.
    Определите работу, совершённую газом.
}
\answer{%
    \begin{align*}
    Q &= A' + \Delta U, A' = \frac 13 Q \implies Q \cdot \cbr{1 - \frac 13} = \Delta U \implies Q = \frac{\Delta U}{1 - \frac 13} = \frac{ 1500\,\text{Дж} }{1 - \frac 13} \approx 2250\,\text{Дж}.
    \\
    A' &= \frac 13 Q
        = \frac 13 \cdot \frac{\Delta U}{1 - \frac 13}
        = \frac{\Delta U}{3 - 1}
        = \frac{ 1500\,\text{Дж} }{3 - 1} \approx 750\,\text{Дж}.
    \end{align*}
}
\solutionspace{60pt}

\tasknumber{16}%
\task{%
    Два конденсатора ёмкостей $C_1 = 60\,\text{нФ}$ и $C_2 = 40\,\text{нФ}$ последовательно подключают
    к источнику напряжения $U = 200\,\text{В}$ (см.
    рис.).
    % Определите заряды каждого из конденсаторов.
    Определите заряд второго конденсатора.

    \begin{tikzpicture}[circuit ee IEC, semithick]
        \draw  (0, 0) to [capacitor={info={$C_1$}}] (1, 0)
                       to [capacitor={info={$C_2$}}] (2, 0)
        ;
        % \draw [-o] (0, 0) -- ++(-0.5, 0) node[left] {$-$};
        % \draw [-o] (2, 0) -- ++(0.5, 0) node[right] {$+$};
        \draw [-o] (0, 0) -- ++(-0.5, 0) node[left] {};
        \draw [-o] (2, 0) -- ++(0.5, 0) node[right] {};
    \end{tikzpicture}
}
\answer{%
    $
        Q_1
            = Q_2
            = CU
            = \frac{ U }{\frac1{C_1} + \frac1{C_2}}
            = \frac{C_1C_2U}{C_1 + C_2}
            = \frac{
                60\,\text{нФ} \cdot 40\,\text{нФ} \cdot 200\,\text{В}
            }{
                60\,\text{нФ} + 40\,\text{нФ}
            }
            = 4800{,}00\,\text{нКл}
    $
}
\solutionspace{120pt}

\tasknumber{17}%
\task{%
    В вакууме вдоль одной прямой расположены три положительных заряда так,
    что расстояние между соседними зарядами равно $d$.
    Сделайте рисунок,
    и определите силу, действующую на крайний заряд.
    Модули всех зарядов равны $q$ ($q > 0$).
}
\solutionspace{80pt}

\tasknumber{18}%
\task{%
    Юлия проводит эксперименты c 2 кусками одинаковой медной проволки, причём второй кусок в два раза длиннее первого.
    В одном из экспериментов Юлия подаёт на первый кусок проволки напряжение в семь раз раз больше, чем на второй.
    Определите отношения в двух проволках в этом эксперименте (второй к первой):
    \begin{itemize}
        \item отношение сил тока,
        \item отношение выделяющихся мощностей.
    \end{itemize}
}
\answer{%
    $\eli_2 / \eli_1 = \frac1{14}, \P_2 / \P_1 = \frac1{14}, $
}

\variantsplitter

\addpersonalvariant{Арсений Трофимов}

\tasknumber{1}%
\task{%
    Валя стартует на велосипеде и в течение $t = 3\,\text{c}$ двигается с постоянным ускорением $2{,}5\,\frac{\text{м}}{\text{с}^{2}}$.
    Определите
    \begin{itemize}
        \item какую скорость при этом удастся достичь,
        \item какой путь за это время будет пройден,
        \item среднюю скорость за всё время движения, если после начального ускорения продолжить движение равномерно ещё в течение времени $3t$
    \end{itemize}
}
\solutionspace{120pt}

\tasknumber{2}%
\task{%
    Какой путь тело пройдёт за четвёртую секунду после начала свободного падения?
    Какую скорость в начале этой секунды оно имеет?
}
\solutionspace{120pt}

\tasknumber{3}%
\task{%
    Карусель радиусом $4\,\text{м}$ равномерно совершает 5 оборотов в минуту.
    Определите
    \begin{itemize}
        \item период и частоту её обращения,
        \item скорость и ускорение крайних её точек.
    \end{itemize}
}
\solutionspace{80pt}

\tasknumber{4}%
\task{%
    Миша стоит на обрыве над рекой и методично и строго горизонтально кидает в неё камушки.
    За этим всем наблюдает экспериментатор Глюк, который уже выяснил, что камушки падают в реку спустя $1{,}3\,\text{с}$ после броска,
    а вот дальность полёта оценить сложнее: придётся лезть в воду.
    Выручите Глюка и определите:
    \begin{itemize}
        \item высоту обрыва (вместе с ростом Миши).
        \item дальность полёта камушков (по горизонтали) и их скорость при падении, приняв начальную скорость броска равной $v = 15\,\frac{\text{м}}{\text{с}}$.
    \end{itemize}
    Сопротивлением воздуха пренебречь.
}
\solutionspace{120pt}

\tasknumber{5}%
\task{%
    Четыре одинаковых брусков массой $3\,\text{кг}$ каждый лежат на гладком горизонтальном столе.
    Бруски пронумерованы от 1 до 4 и последовательно связаны между собой
    невесомыми нерастяжимыми нитями: 1 со 2, 2 с 3 (ну и с 1) и т.д.
    Экспериментатор Глюк прикладывает постоянную горизонтальную силу $90\,\text{Н}$ к бруску с наибольшим номером.
    С каким ускорением двигается система? Чему равна сила натяжения нити, связывающей бруски 2 и 3?
}
\solutionspace{120pt}

\tasknumber{6}%
\task{%
    Два бруска связаны лёгкой нерастяжимой нитью и перекинуты через неподвижный блок (см.
    рис.).
    Определите силу натяжения нити и ускорения брусков.
    Силами трения пренебречь, массы брусков
    равны $m_1 = 5\,\text{кг}$ и $m_2 = 4\,\text{кг}$.
    % $g = 10\,\frac{\text{м}}{\text{с}^{2}}$.

    \begin{tikzpicture}[x=1.5cm,y=1.5cm,thick]
        \draw
            (-0.4, 0) rectangle (-0.2, 1.2)
            (0.15, 0.5) rectangle (0.45, 1)
            (0, 2) circle [radius=0.3] -- ++(up:0.5)
            (-0.3, 1.2) -- ++(up:0.8)
            (0.3, 1) -- ++(up:1)
            (-0.7, 2.5) -- (0.7, 2.5)
            ;
        \draw[pattern={Lines[angle=51,distance=3pt]},pattern color=black,draw=none] (-0.7, 2.5) rectangle (0.7, 2.75);
        \node [left] (left) at (-0.4, 0.6) { $m_1$ };
        \node [right] (right) at (0.4, 0.75) { $m_2$ };
    \end{tikzpicture}
}
\solutionspace{80pt}

\tasknumber{7}%
\task{%
    Тело массой $2{,}7\,\text{кг}$ лежит на горизонтальной поверхности.
    Коэффициент трения между поверхностью и телом $0{,}15$.
    К телу приложена горизонтальная сила $4{,}5\,\text{Н}$.
    Определите силу трения, действующую на тело, и ускорение тела.
    % $g = 10\,\frac{\text{м}}{\text{с}^{2}}$.
}
\solutionspace{120pt}

\tasknumber{8}%
\task{%
    Определите плотность неизвестного вещества, если известно, что опускании тела из него
    в керосин оно будет плавать и на треть выступать над поверхностью жидкости.
}
\solutionspace{120pt}

\tasknumber{9}%
\task{%
    	Определите силу, действующую на левую опору однородного горизонтального стержня длиной $l = 5\,\text{м}$
    	и массой $M = 5\,\text{кг}$, к которому подвешен груз массой $m = 4\,\text{кг}$ на расстоянии $2\,\text{м}$ от правого конца (см.
    рис.).

        \begin{tikzpicture}[thick]
            \draw
                (-2, -0.1) rectangle (2, 0.1)
                (-0.5, -0.1) -- (-0.5, -1)
                (-0.7, -1) rectangle (-0.3, -1.3)
           		(-2, -0.1) -- +(0.15,-0.9) -- +(-0.15,-0.9) -- cycle
            	(2, -0.1) -- +(0.15,-0.9) -- +(-0.15,-0.9) -- cycle
            ;
            \draw[pattern={Lines[angle=51,distance=2pt]},pattern color=black,draw=none]
            	(-2.15, -1.15) rectangle +(0.3, 0.15)
            	(2.15, -1.15) rectangle +(-0.3, 0.15)
            ;
            \node [right] (m_small) at (-0.3, -1.15) { $m$ };
            \node [above] (M_big) at (0, 0.1) { $M$ };
        \end{tikzpicture}
}
\solutionspace{80pt}

\tasknumber{10}%
\task{%
    Тонкий однородный шест длиной $2\,\text{м}$ и массой $20\,\text{кг}$ лежит на горизонтальной поверхности.
    \begin{itemize}
        \item Какую минимальную силу надо приложить к одному из его концов, чтобы оторвать его от этой поверхности?
        \item Какую минимальную работу надо совершить, чтобы поставить его на землю в вертикальное положение?
    \end{itemize}
    % Примите $g = 10\,\frac{\text{м}}{\text{с}^{2}}$.
}
\answer{%
    $A = mg\frac l2 = 200\,\text{Дж}$
}
\solutionspace{120pt}

\tasknumber{11}%
\task{%
    Определите работу силы, которая обеспечит спуск тела массой $3\,\text{кг}$ на высоту $2\,\text{м}$ с постоянным ускорением $6\,\frac{\text{м}}{\text{c}^{2}}$.
    % Примите $g = 10\,\frac{\text{м}}{\text{с}^{2}}$.
}
\answer{%
    \begin{align*}
    &\text{Для подъёма:} A = Fh = (mg + ma) h = m(g+a)h, \\
    &\text{Для спуска:} A = -Fh = -(mg - ma) h = -m(g-a)h, \\
    &\text{В результате получаем:} -24\,\text{Дж}.
    \end{align*}
}
\solutionspace{60pt}

\tasknumber{12}%
\task{%
    Тело бросили вертикально вверх со скоростью $14\,\frac{\text{м}}{\text{c}}$.
    На какой высоте кинетическая энергия тела составит треть от потенциальной?
}
\solutionspace{100pt}

\tasknumber{13}%
\task{%
    Плотность воздуха при нормальных условиях равна $1{,}3\,\frac{\text{кг}}{\text{м}^{3}}$.
    Чему равна плотность воздуха
    при температуре $100\celsius$ и давлении $80\,\text{кПа}$?
}
\solutionspace{120pt}

\tasknumber{14}%
\task{%
    Небольшую цилиндрическую пробирку с воздухом погружают на некоторую глубину в глубокое пресное озеро,
    после чего воздух занимает в ней лишь третью часть от общего объема.
    Определите глубину, на которую погрузили пробирку.
    Температуру считать постоянной $T = 288\,\text{К}$, давлением паров воды пренебречь,
    атмосферное давление принять равным $p_{\text{aтм}} = 100\,\text{кПа}$.
}
\answer{%
    \begin{align*}
    T\text{— const} &\implies P_1V_1 = \nu RT = P_2V_2.
    \\
    V_2 = \frac 13 V_1 &\implies P_1V_1 = P_2 \cdot \frac 13V_1 \implies P_2 = 3P_1 = 3p_{\text{aтм}}.
    \\
    P_2 = p_{\text{aтм}} + \rho_{\text{в}} g h \implies h = \frac{P_2 - p_{\text{aтм}}}{\rho_{\text{в}} g} &= \frac{3p_{\text{aтм}} - p_{\text{aтм}}}{\rho_{\text{в}} g} = \frac{2 \cdot p_{\text{aтм}}}{\rho_{\text{в}} g} =  \\
     &= \frac{2 \cdot 100\,\text{кПа}}{1000\,\frac{\text{кг}}{\text{м}^{3}} \cdot  10\,\frac{\text{м}}{\text{с}^{2}}} \approx 20\,\text{м}.
    \end{align*}
}
\solutionspace{120pt}

\tasknumber{15}%
\task{%
    Газу сообщили некоторое количество теплоты,
    при этом четверть его он потратил на совершение работы,
    одновременно увеличив свою внутреннюю энергию на $1500\,\text{Дж}$.
    Определите работу, совершённую газом.
}
\answer{%
    \begin{align*}
    Q &= A' + \Delta U, A' = \frac 14 Q \implies Q \cdot \cbr{1 - \frac 14} = \Delta U \implies Q = \frac{\Delta U}{1 - \frac 14} = \frac{ 1500\,\text{Дж} }{1 - \frac 14} \approx 2000\,\text{Дж}.
    \\
    A' &= \frac 14 Q
        = \frac 14 \cdot \frac{\Delta U}{1 - \frac 14}
        = \frac{\Delta U}{4 - 1}
        = \frac{ 1500\,\text{Дж} }{4 - 1} \approx 500\,\text{Дж}.
    \end{align*}
}
\solutionspace{60pt}

\tasknumber{16}%
\task{%
    Два конденсатора ёмкостей $C_1 = 40\,\text{нФ}$ и $C_2 = 60\,\text{нФ}$ последовательно подключают
    к источнику напряжения $U = 450\,\text{В}$ (см.
    рис.).
    % Определите заряды каждого из конденсаторов.
    Определите заряд первого конденсатора.

    \begin{tikzpicture}[circuit ee IEC, semithick]
        \draw  (0, 0) to [capacitor={info={$C_1$}}] (1, 0)
                       to [capacitor={info={$C_2$}}] (2, 0)
        ;
        % \draw [-o] (0, 0) -- ++(-0.5, 0) node[left] {$-$};
        % \draw [-o] (2, 0) -- ++(0.5, 0) node[right] {$+$};
        \draw [-o] (0, 0) -- ++(-0.5, 0) node[left] {};
        \draw [-o] (2, 0) -- ++(0.5, 0) node[right] {};
    \end{tikzpicture}
}
\answer{%
    $
        Q_1
            = Q_2
            = CU
            = \frac{ U }{\frac1{C_1} + \frac1{C_2}}
            = \frac{C_1C_2U}{C_1 + C_2}
            = \frac{
                40\,\text{нФ} \cdot 60\,\text{нФ} \cdot 450\,\text{В}
            }{
                40\,\text{нФ} + 60\,\text{нФ}
            }
            = 10800{,}00\,\text{нКл}
    $
}
\solutionspace{120pt}

\tasknumber{17}%
\task{%
    В вакууме вдоль одной прямой расположены три отрицательных заряда так,
    что расстояние между соседними зарядами равно $l$.
    Сделайте рисунок,
    и определите силу, действующую на крайний заряд.
    Модули всех зарядов равны $q$ ($q > 0$).
}
\solutionspace{80pt}

\tasknumber{18}%
\task{%
    Юлия проводит эксперименты c 2 кусками одинаковой стальной проволки, причём второй кусок в девять раз длиннее первого.
    В одном из экспериментов Юлия подаёт на первый кусок проволки напряжение в девять раз раз больше, чем на второй.
    Определите отношения в двух проволках в этом эксперименте (второй к первой):
    \begin{itemize}
        \item отношение сил тока,
        \item отношение выделяющихся мощностей.
    \end{itemize}
}
\answer{%
    $\eli_2 / \eli_1 = \frac1{81}, \P_2 / \P_1 = \frac1{81}, $
}
% autogenerated
