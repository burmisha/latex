\setdate{18~мая~2021}
\setclass{10«АБ»}

\addpersonalvariant{Михаил Бурмистров}

\tasknumber{1}%
\task{%
    Саша стартует на велосипеде и в течение $t = 4\,\text{c}$ двигается с постоянным ускорением $0{,}5\,\frac{\text{м}}{\text{с}^{2}}$.
    Определите
    \begin{itemize}
        \item какую скорость при этом удастся достичь,
        \item какой путь за это время будет пройден,
        \item среднюю скорость за всё время движения, если после начального ускорения продолжить движение равномерно ещё в течение времени $2t$
    \end{itemize}
}
\answer{%
    \begin{align*}
    v &= v_0 + a t = at = 0{,}5\,\frac{\text{м}}{\text{с}^{2}} \cdot 4\,\text{c} = 2{,}0\,\frac{\text{м}}{\text{с}}, \\
    s_x &= v_0t + \frac{a t^2}2 = \frac{a t^2}2 = \frac{0{,}5\,\frac{\text{м}}{\text{с}^{2}} \cdot \sqr{ 4\,\text{c} }}2 = 4{,}0\,\text{м}, \\
    v_\text{сред.} &= \frac{s_\text{общ}}{t_\text{общ.}} = \frac{s_x + v \cdot 2t}{t + 2t} = \frac{\frac{a t^2}2 + at \cdot 2t}{t (1 + 2)} = \\
    &= at \cdot \frac{\frac 12 + 2}{1 + 2} = 0{,}5\,\frac{\text{м}}{\text{с}^{2}} \cdot 4\,\text{c} \cdot \frac{\frac 12 + 2}{1 + 2} \approx 1{,}67\,\frac{\text{м}}{\text{c}}.
    \end{align*}
}
\solutionspace{120pt}

\tasknumber{2}%
\task{%
    Какой путь тело пройдёт за вторую секунду после начала свободного падения?
    Какую скорость в начале этой секунды оно имеет?
}
\answer{%
    \begin{align*}
    s &= -s_y = -(y_2-y_1) = y_1 - y_2 = \cbr{y_{0y} + v_{0y}t_1 - \frac{gt_1^2}2} - \cbr{y_{0y} + v_{0y}t_2 - \frac{gt_2^2}2} = \\
    &= \frac{gt_2^2}2 - \frac{gt_1^2}2 = \frac g2\cbr{t_2^2 - t_1^2} = 15{,}0\,\text{м}, \\
    v_y &= v_{0y} - gt = -gt = 10\,\frac{\text{м}}{\text{с}^{2}} \cdot 1\,\text{с} = -10\,\frac{\text{м}}{\text{с}}.
    \end{align*}
}
\solutionspace{120pt}

\tasknumber{3}%
\task{%
    Карусель диаметром $3\,\text{м}$ равномерно совершает 6 оборотов в минуту.
    Определите
    \begin{itemize}
        \item период и частоту её обращения,
        \item скорость и ускорение крайних её точек.
    \end{itemize}
}
\answer{%
    \begin{align*}
    t &= 60\,\text{с}, r = 1{,}5\,\text{м}, n = 6\units{оборотов}, \\
    T &= \frac tN = \frac{ 60\,\text{с} }{6} \approx 10{,}00\,\text{c}, \\
    \nu &= \frac 1T = \frac{6}{ 60\,\text{с} } \approx 0{,}10\,\text{Гц}, \\
    v &= \frac{2 \pi r}{T} = \frac{2 \pi r}{T} =  \frac{2 \pi r n}{t} \approx 0{,}94\,\frac{\text{м}}{\text{c}}, \\
    a &= \frac{v^2}{r} =  \frac{4 \pi^2 r n^2}{t^2} \approx 0{,}59\,\frac{\text{м}}{\text{с}^{2}}.
    \end{align*}
}
\solutionspace{80pt}

\tasknumber{4}%
\task{%
    Паша стоит на обрыве над рекой и методично и строго горизонтально кидает в неё камушки.
    За этим всем наблюдает экспериментатор Глюк, который уже выяснил, что камушки падают в реку спустя $1{,}6\,\text{с}$ после броска,
    а вот дальность полёта оценить сложнее: придётся лезть в воду.
    Выручите Глюка и определите:
    \begin{itemize}
        \item высоту обрыва (вместе с ростом Паши).
        \item дальность полёта камушков (по горизонтали) и их скорость при падении, приняв начальную скорость броска равной $v_0 = 18\,\frac{\text{м}}{\text{с}}$.
    \end{itemize}
    Сопротивлением воздуха пренебречь.
}
\answer{%
    \begin{align*}
    y &= y_0 + v_{0y}t - \frac{gt^2}2 = h - \frac{gt^2}2, \qquad y(\tau) = 0 \implies h - \frac{g\tau^2}2 = 0 \implies h = \frac{g\tau^2}2 \approx 12{,}8\,\text{м}.
    \\
    x &= x_0 + v_{0x}t = v_0t \implies L = v_0\tau \approx 28{,}8\,\text{м}.
    \\
    &v = \sqrt{v_x^2 + v_y^2} = \sqrt{v_{0x}^2 + \sqr{v_{0y} - g\tau}} = \sqrt{v_0^2 + \sqr{g\tau}} \approx 24{,}1\,\frac{\text{м}}{\text{c}}.
    \end{align*}
}
\solutionspace{120pt}

\tasknumber{5}%
\task{%
    Четыре одинаковых брусков массой $3\,\text{кг}$ каждый лежат на гладком горизонтальном столе.
    Бруски пронумерованы от 1 до 4 и последовательно связаны между собой
    невесомыми нерастяжимыми нитями: 1 со 2, 2 с 3 (ну и с 1) и т.д.
    Экспериментатор Глюк прикладывает постоянную горизонтальную силу $90\,\text{Н}$ к бруску с наибольшим номером.
    С каким ускорением двигается система? Чему равна сила натяжения нити, связывающей бруски 1 и 2?
}
\answer{%
    \begin{align*}
    a &= \frac{F}{4 m} = \frac{90\,\text{Н}}{4 \cdot 3\,\text{кг}} \approx 7{,}5\,\frac{\text{м}}{\text{c}^{2}}, \\
    T &= m'a = 1m \cdot \frac{F}{4 m} = \frac{1}{4} F \approx 22{,}5\,\text{Н}.
    \end{align*}
}
\solutionspace{120pt}

\tasknumber{6}%
\task{%
    Два бруска связаны лёгкой нерастяжимой нитью и перекинуты через неподвижный блок (см.
    рис.).
    Определите силу натяжения нити и ускорения брусков.
    Силами трения пренебречь, массы брусков
    равны $m_1 = 5\,\text{кг}$ и $m_2 = 14\,\text{кг}$.
    % $g = 10\,\frac{\text{м}}{\text{с}^{2}}$.

    \begin{tikzpicture}[x=1.5cm,y=1.5cm,thick]
        \draw
            (-0.4, 0) rectangle (-0.2, 1.2)
            (0.15, 0.5) rectangle (0.45, 1)
            (0, 2) circle [radius=0.3] -- ++(up:0.5)
            (-0.3, 1.2) -- ++(up:0.8)
            (0.3, 1) -- ++(up:1)
            (-0.7, 2.5) -- (0.7, 2.5)
            ;
        \draw[pattern={Lines[angle=51,distance=3pt]},pattern color=black,draw=none] (-0.7, 2.5) rectangle (0.7, 2.75);
        \node [left] (left) at (-0.4, 0.6) { $m_1$ };
        \node [right] (right) at (0.4, 0.75) { $m_2$ };
    \end{tikzpicture}
}
\answer{%
    Предположим, что левый брусок ускоряется вверх, тогда правый ускоряется вниз (с тем же ускорением).
    Запишем 2-й закон Ньютона 2 раза (для обоих тел) в проекции на вертикальную оси, направив её вверх.
    \begin{align*}
        &\begin{cases}
            T - m_1g = m_1a, \\
            T - m_2g = -m_2a,
        \end{cases} \\
        &\begin{cases}
            m_2g - m_1g = m_1a + m_2a, \\
            T = m_1a + m_1g, \\
        \end{cases} \\
        a &= \frac{m_2 - m_1}{m_1 + m_2} \cdot g = \frac{14\,\text{кг} - 5\,\text{кг}}{5\,\text{кг} + 14\,\text{кг}} \cdot 10\,\frac{\text{м}}{\text{с}^{2}} \approx 4{,}74\,\frac{\text{м}}{\text{c}^{2}}, \\
        T &= m_1(a + g) = m_1 \cdot g \cdot \cbr{\frac{m_2 - m_1}{m_1 + m_2} + 1} = m_1 \cdot g \cdot \frac{2m_2}{m_1 + m_2} = \\
            &= \frac{2 m_2 m_1 g}{m_1 + m_2} = \frac{2 \cdot 14\,\text{кг} \cdot 5\,\text{кг} \cdot 10\,\frac{\text{м}}{\text{с}^{2}}}{5\,\text{кг} + 14\,\text{кг}} \approx 73{,}7\,\text{Н}.
    \end{align*}
    Отрицательный ответ говорит, что мы лишь не угадали с направлением ускорений.
    Сила же всегда положительна.
}
\solutionspace{80pt}

\tasknumber{7}%
\task{%
    Тело массой $1{,}4\,\text{кг}$ лежит на горизонтальной поверхности.
    Коэффициент трения между поверхностью и телом $0{,}25$.
    К телу приложена горизонтальная сила $5{,}5\,\text{Н}$.
    Определите силу трения, действующую на тело, и ускорение тела.
    % $g = 10\,\frac{\text{м}}{\text{с}^{2}}$.
}
\answer{%
    \begin{align*}
    &F_\text{ трения покоя $\max$ } = \mu N = \mu m g = 0{,}25 \cdot 1{,}4\,\text{кг} \cdot 10\,\frac{\text{м}}{\text{с}^{2}} = 3{,}50\,\text{Н}, \\
    &F_\text{ трения покоя $\max$ } \le F \implies F_\text{ трения } = 3{,}50\,\text{Н}, a = \frac{F - F_\text{ трения }}{ m } = 1{,}43\,\frac{\text{м}}{\text{c}^{2}}, \\
    &\text{при равенстве возможны оба варианта: и едет, и не едет, но на ответы это не влияет.}
    \end{align*}
}
\solutionspace{120pt}

\tasknumber{8}%
\task{%
    Определите плотность неизвестного вещества, если известно, что опускании тела из него
    в подсолнечное масло оно будет плавать и на треть выступать над поверхностью жидкости.
}
\answer{%
    $F_\text{Арх.} = F_\text{тяж.} \implies \rho_\text{ж.} g V_\text{погр.} = m g \implies\rho_\text{ж.} g \cbr{V -\frac V3} = \rho V g \implies \rho = \rho_\text{ж.}\cbr{1 -\frac 13} \approx 600\,\frac{\text{кг}}{\text{м}^{3}}$
}
\solutionspace{120pt}

\tasknumber{9}%
\task{%
    	Определите силу, действующую на левую опору однородного горизонтального стержня длиной $l = 7\,\text{м}$
    	и массой $M = 5\,\text{кг}$, к которому подвешен груз массой $m = 2\,\text{кг}$ на расстоянии $4\,\text{м}$ от правого конца (см.
    рис.).

        \begin{tikzpicture}[thick]
            \draw
                (-2, -0.1) rectangle (2, 0.1)
                (-0.5, -0.1) -- (-0.5, -1)
                (-0.7, -1) rectangle (-0.3, -1.3)
           		(-2, -0.1) -- +(0.15,-0.9) -- +(-0.15,-0.9) -- cycle
            	(2, -0.1) -- +(0.15,-0.9) -- +(-0.15,-0.9) -- cycle
            ;
            \draw[pattern={Lines[angle=51,distance=2pt]},pattern color=black,draw=none]
            	(-2.15, -1.15) rectangle +(0.3, 0.15)
            	(2.15, -1.15) rectangle +(-0.3, 0.15)
            ;
            \node [right] (m_small) at (-0.3, -1.15) { $m$ };
            \node [above] (M_big) at (0, 0.1) { $M$ };
        \end{tikzpicture}
}
\answer{%
    \begin{align*}
        &\begin{cases}
            F_1 + F_2 - mg - Mg= 0, \\
            F_1 \cdot 0 - mg \cdot a - Mg \cdot \frac l2 + F_2 \cdot l = 0,
        \end{cases} \\
        F_2 &= \frac{mga + Mg\frac l2}l = \frac al \cdot mg + \frac{Mg}2 \approx 33{,}6\,\text{Н}, \\
        F_1 &= mg + Mg - F_2 = mg + Mg - \frac al \cdot mg - \frac{Mg}2 = \frac bl \cdot mg + \frac{Mg}2 \approx 36{,}4\,\text{Н}.
    \end{align*}
}
\solutionspace{80pt}

\tasknumber{10}%
\task{%
    Тонкий однородный лом длиной $1\,\text{м}$ и массой $10\,\text{кг}$ лежит на горизонтальной поверхности.
    \begin{itemize}
        \item Какую минимальную силу надо приложить к одному из его концов, чтобы оторвать его от этой поверхности?
        \item Какую минимальную работу надо совершить, чтобы поставить его на землю в вертикальное положение?
    \end{itemize}
    % Примите $g = 10\,\frac{\text{м}}{\text{с}^{2}}$.
}
\answer{%
    $F = \frac{mg}2 \approx 100\,\text{Н}, A = mg\frac l2 = 50\,\text{Дж}$
}
\solutionspace{120pt}

\tasknumber{11}%
\task{%
    Определите работу силы, которая обеспечит спуск тела массой $2\,\text{кг}$ на высоту $5\,\text{м}$ с постоянным ускорением $3\,\frac{\text{м}}{\text{c}^{2}}$.
    % Примите $g = 10\,\frac{\text{м}}{\text{с}^{2}}$.
}
\answer{%
    \begin{align*}
    &\text{Для подъёма:} A = Fh = (mg + ma) h = m(g+a)h, \\
    &\text{Для спуска:} A = -Fh = -(mg - ma) h = -m(g-a)h, \\
    &\text{В результате получаем:} -70\,\text{Дж}.
    \end{align*}
}
\solutionspace{60pt}

\tasknumber{12}%
\task{%
    Тело бросили вертикально вверх со скоростью $14\,\frac{\text{м}}{\text{c}}$.
    На какой высоте кинетическая энергия тела составит половину от потенциальной?
}
\answer{%
    \begin{align*}
    &0 + \frac{mv_0^2}2 = E_p + E_k, E_k = \frac 12 E_p \implies \\
    &\implies \frac{mv_0^2}2 = E_p + \frac 12 E_p = E_p\cbr{1 + \frac 12} = mgh\cbr{1 + \frac 12} \implies \\
    &\implies h = \frac{\frac{mv_0^2}2}{mg\cbr{1 + \frac 12}} = \frac{v_0^2}{2g} \cdot \frac 1{1 + \frac 12} \approx 6{,}5\,\text{м}.
    \end{align*}
}
\solutionspace{100pt}

\tasknumber{13}%
\task{%
    Плотность воздуха при нормальных условиях равна $1{,}3\,\frac{\text{кг}}{\text{м}^{3}}$.
    Чему равна плотность воздуха
    при температуре $200\celsius$ и давлении $50\,\text{кПа}$?
}
\answer{%
    \begin{align*}
    &\text{В общем случае:} PV = \frac m{\mu} RT \implies \rho = \frac mV = \frac m{\frac{\frac m{\mu} RT}P} = \frac{P\mu}{RT}, \\
    &\text{У нас 2 состояния:} \rho_1 = \frac{P_1\mu}{RT_1}, \rho_2 = \frac{P_2\mu}{RT_2} \implies \frac{\rho_2}{\rho_1} = \frac{\frac{P_2\mu}{RT_2}}{\frac{P_1\mu}{RT_1}} = \frac{P_2T_1}{P_1T_2} \implies \\
    &\implies \rho_2 = \rho_1 \cdot  \frac{P_2T_1}{P_1T_2} = 1{,}3\,\frac{\text{кг}}{\text{м}^{3}} \cdot \frac{50\,\text{кПа} \cdot 273\units{К}}{100\,\text{кПа} \cdot 473\units{К}} \approx 0{,}38\,\frac{\text{кг}}{\text{м}^{3}}.
    \end{align*}
}
\solutionspace{120pt}

\tasknumber{14}%
\task{%
    Небольшую цилиндрическую пробирку с воздухом погружают на некоторую глубину в глубокое пресное озеро,
    после чего воздух занимает в ней лишь пятую часть от общего объема.
    Определите глубину, на которую погрузили пробирку.
    Температуру считать постоянной $T = 280\,\text{К}$, давлением паров воды пренебречь,
    атмосферное давление принять равным $p_{\text{aтм}} = 100\,\text{кПа}$.
}
\answer{%
    \begin{align*}
    T\text{— const} &\implies P_1V_1 = \nu RT = P_2V_2.
    \\
    V_2 = \frac 15 V_1 &\implies P_1V_1 = P_2 \cdot \frac 15V_1 \implies P_2 = 5P_1 = 5p_{\text{aтм}}.
    \\
    P_2 = p_{\text{aтм}} + \rho_{\text{в}} g h \implies h = \frac{P_2 - p_{\text{aтм}}}{\rho_{\text{в}} g} &= \frac{5p_{\text{aтм}} - p_{\text{aтм}}}{\rho_{\text{в}} g} = \frac{4 \cdot p_{\text{aтм}}}{\rho_{\text{в}} g} =  \\
     &= \frac{4 \cdot 100\,\text{кПа}}{1000\,\frac{\text{кг}}{\text{м}^{3}} \cdot  10\,\frac{\text{м}}{\text{с}^{2}}} \approx 40\,\text{м}.
    \end{align*}
}
\solutionspace{120pt}

\tasknumber{15}%
\task{%
    Газу сообщили некоторое количество теплоты,
    при этом треть его он потратил на совершение работы,
    одновременно увеличив свою внутреннюю энергию на $1200\,\text{Дж}$.
    Определите работу, совершённую газом.
}
\answer{%
    \begin{align*}
    Q &= A' + \Delta U, A' = \frac 13 Q \implies Q \cdot \cbr{1 - \frac 13} = \Delta U \implies Q = \frac{\Delta U}{1 - \frac 13} = \frac{ 1200\,\text{Дж} }{1 - \frac 13} \approx 1800\,\text{Дж}.
    \\
    A' &= \frac 13 Q
        = \frac 13 \cdot \frac{\Delta U}{1 - \frac 13}
        = \frac{\Delta U}{3 - 1}
        = \frac{ 1200\,\text{Дж} }{3 - 1} \approx 600\,\text{Дж}.
    \end{align*}
}
\solutionspace{60pt}

\tasknumber{16}%
\task{%
    Два конденсатора ёмкостей $C_1 = 40\,\text{нФ}$ и $C_2 = 60\,\text{нФ}$ последовательно подключают
    к источнику напряжения $V = 400\,\text{В}$ (см.
    рис.).
    % Определите заряды каждого из конденсаторов.
    Определите заряд второго конденсатора.

    \begin{tikzpicture}[circuit ee IEC, semithick]
        \draw  (0, 0) to [capacitor={info={$C_1$}}] (1, 0)
                       to [capacitor={info={$C_2$}}] (2, 0)
        ;
        % \draw [-o] (0, 0) -- ++(-0.5, 0) node[left] {$-$};
        % \draw [-o] (2, 0) -- ++(0.5, 0) node[right] {$+$};
        \draw [-o] (0, 0) -- ++(-0.5, 0) node[left] {};
        \draw [-o] (2, 0) -- ++(0.5, 0) node[right] {};
    \end{tikzpicture}
}
\answer{%
    $
        Q_1
            = Q_2
            = CV
            = \frac{ V }{\frac1{C_1} + \frac1{C_2}}
            = \frac{C_1C_2V}{C_1 + C_2}
            = \frac{
                40\,\text{нФ} \cdot 60\,\text{нФ} \cdot 400\,\text{В}
            }{
                40\,\text{нФ} + 60\,\text{нФ}
            }
            = 9{,}60\,\text{мкКл}
    $
}
\solutionspace{120pt}

\tasknumber{17}%
\task{%
    В вакууме вдоль одной прямой расположены четыре отрицательных заряда так,
    что расстояние между соседними зарядами равно $r$.
    Сделайте рисунок,
    и определите силу, действующую на крайний заряд.
    Модули всех зарядов равны $q$ ($q > 0$).
}
\answer{%
    $F = \sum_i F_i = \ldots = \frac{49}{36} \frac{kq^2}{r^2}.$
}
\solutionspace{80pt}

\tasknumber{18}%
\task{%
    Юлия проводит эксперименты c 2 кусками одинаковой алюминиевой проволки, причём второй кусок в четыре раза длиннее первого.
    В одном из экспериментов Юлия подаёт на первый кусок проволки напряжение в шесть раз раз больше, чем на второй.
    Определите отношения в двух проволках в этом эксперименте (второй к первой):
    \begin{itemize}
        \item отношение сил тока,
        \item отношение выделяющихся мощностей.
    \end{itemize}
}
\answer{%
    $R_2 = 4R_1, U_1 = 6U_2 \implies  \eli_2 / \eli_1 = \frac{U_2 / R_2}{U_1 / R_1} = \frac{U_2}{U_1} \cdot \frac{R_1}{R_2} = \frac1{24}, P_2 / P_1 = \frac{U_2^2 / R_2}{U_1^2 / R_1} = \sqr{\frac{U_2}{U_1}} \cdot \frac{R_1}{R_2} = \frac1{144}.$
}

\variantsplitter

\addpersonalvariant{Ирина Ан}

\tasknumber{1}%
\task{%
    Валя стартует на мотоцикле и в течение $t = 2\,\text{c}$ двигается с постоянным ускорением $2\,\frac{\text{м}}{\text{с}^{2}}$.
    Определите
    \begin{itemize}
        \item какую скорость при этом удастся достичь,
        \item какой путь за это время будет пройден,
        \item среднюю скорость за всё время движения, если после начального ускорения продолжить движение равномерно ещё в течение времени $2t$
    \end{itemize}
}
\answer{%
    \begin{align*}
    v &= v_0 + a t = at = 2\,\frac{\text{м}}{\text{с}^{2}} \cdot 2\,\text{c} = 4{,}0\,\frac{\text{м}}{\text{с}}, \\
    s_x &= v_0t + \frac{a t^2}2 = \frac{a t^2}2 = \frac{2\,\frac{\text{м}}{\text{с}^{2}} \cdot \sqr{ 2\,\text{c} }}2 = 4{,}0\,\text{м}, \\
    v_\text{сред.} &= \frac{s_\text{общ}}{t_\text{общ.}} = \frac{s_x + v \cdot 2t}{t + 2t} = \frac{\frac{a t^2}2 + at \cdot 2t}{t (1 + 2)} = \\
    &= at \cdot \frac{\frac 12 + 2}{1 + 2} = 2\,\frac{\text{м}}{\text{с}^{2}} \cdot 2\,\text{c} \cdot \frac{\frac 12 + 2}{1 + 2} \approx 3{,}33\,\frac{\text{м}}{\text{c}}.
    \end{align*}
}
\solutionspace{120pt}

\tasknumber{2}%
\task{%
    Какой путь тело пройдёт за пятую секунду после начала свободного падения?
    Какую скорость в конце этой секунды оно имеет?
}
\answer{%
    \begin{align*}
    s &= -s_y = -(y_2-y_1) = y_1 - y_2 = \cbr{y_{0y} + v_{0y}t_1 - \frac{gt_1^2}2} - \cbr{y_{0y} + v_{0y}t_2 - \frac{gt_2^2}2} = \\
    &= \frac{gt_2^2}2 - \frac{gt_1^2}2 = \frac g2\cbr{t_2^2 - t_1^2} = 45{,}0\,\text{м}, \\
    v_y &= v_{0y} - gt = -gt = 10\,\frac{\text{м}}{\text{с}^{2}} \cdot 5\,\text{с} = -50\,\frac{\text{м}}{\text{с}}.
    \end{align*}
}
\solutionspace{120pt}

\tasknumber{3}%
\task{%
    Карусель диаметром $2\,\text{м}$ равномерно совершает 10 оборотов в минуту.
    Определите
    \begin{itemize}
        \item период и частоту её обращения,
        \item скорость и ускорение крайних её точек.
    \end{itemize}
}
\answer{%
    \begin{align*}
    t &= 60\,\text{с}, r = 1{,}0\,\text{м}, n = 10\units{оборотов}, \\
    T &= \frac tN = \frac{ 60\,\text{с} }{10} \approx 6{,}00\,\text{c}, \\
    \nu &= \frac 1T = \frac{10}{ 60\,\text{с} } \approx 0{,}17\,\text{Гц}, \\
    v &= \frac{2 \pi r}{T} = \frac{2 \pi r}{T} =  \frac{2 \pi r n}{t} \approx 1{,}05\,\frac{\text{м}}{\text{c}}, \\
    a &= \frac{v^2}{r} =  \frac{4 \pi^2 r n^2}{t^2} \approx 1{,}10\,\frac{\text{м}}{\text{с}^{2}}.
    \end{align*}
}
\solutionspace{80pt}

\tasknumber{4}%
\task{%
    Миша стоит на обрыве над рекой и методично и строго горизонтально кидает в неё камушки.
    За этим всем наблюдает экспериментатор Глюк, который уже выяснил, что камушки падают в реку спустя $1{,}2\,\text{с}$ после броска,
    а вот дальность полёта оценить сложнее: придётся лезть в воду.
    Выручите Глюка и определите:
    \begin{itemize}
        \item высоту обрыва (вместе с ростом Миши).
        \item дальность полёта камушков (по горизонтали) и их скорость при падении, приняв начальную скорость броска равной $v_0 = 18\,\frac{\text{м}}{\text{с}}$.
    \end{itemize}
    Сопротивлением воздуха пренебречь.
}
\answer{%
    \begin{align*}
    y &= y_0 + v_{0y}t - \frac{gt^2}2 = h - \frac{gt^2}2, \qquad y(\tau) = 0 \implies h - \frac{g\tau^2}2 = 0 \implies h = \frac{g\tau^2}2 \approx 7{,}2\,\text{м}.
    \\
    x &= x_0 + v_{0x}t = v_0t \implies L = v_0\tau \approx 21{,}6\,\text{м}.
    \\
    &v = \sqrt{v_x^2 + v_y^2} = \sqrt{v_{0x}^2 + \sqr{v_{0y} - g\tau}} = \sqrt{v_0^2 + \sqr{g\tau}} \approx 21{,}6\,\frac{\text{м}}{\text{c}}.
    \end{align*}
}
\solutionspace{120pt}

\tasknumber{5}%
\task{%
    Четыре одинаковых брусков массой $2\,\text{кг}$ каждый лежат на гладком горизонтальном столе.
    Бруски пронумерованы от 1 до 4 и последовательно связаны между собой
    невесомыми нерастяжимыми нитями: 1 со 2, 2 с 3 (ну и с 1) и т.д.
    Экспериментатор Глюк прикладывает постоянную горизонтальную силу $90\,\text{Н}$ к бруску с наибольшим номером.
    С каким ускорением двигается система? Чему равна сила натяжения нити, связывающей бруски 3 и 4?
}
\answer{%
    \begin{align*}
    a &= \frac{F}{4 m} = \frac{90\,\text{Н}}{4 \cdot 2\,\text{кг}} \approx 11{,}2\,\frac{\text{м}}{\text{c}^{2}}, \\
    T &= m'a = 3m \cdot \frac{F}{4 m} = \frac{3}{4} F \approx 67{,}5\,\text{Н}.
    \end{align*}
}
\solutionspace{120pt}

\tasknumber{6}%
\task{%
    Два бруска связаны лёгкой нерастяжимой нитью и перекинуты через неподвижный блок (см.
    рис.).
    Определите силу натяжения нити и ускорения брусков.
    Силами трения пренебречь, массы брусков
    равны $m_1 = 5\,\text{кг}$ и $m_2 = 10\,\text{кг}$.
    % $g = 10\,\frac{\text{м}}{\text{с}^{2}}$.

    \begin{tikzpicture}[x=1.5cm,y=1.5cm,thick]
        \draw
            (-0.4, 0) rectangle (-0.2, 1.2)
            (0.15, 0.5) rectangle (0.45, 1)
            (0, 2) circle [radius=0.3] -- ++(up:0.5)
            (-0.3, 1.2) -- ++(up:0.8)
            (0.3, 1) -- ++(up:1)
            (-0.7, 2.5) -- (0.7, 2.5)
            ;
        \draw[pattern={Lines[angle=51,distance=3pt]},pattern color=black,draw=none] (-0.7, 2.5) rectangle (0.7, 2.75);
        \node [left] (left) at (-0.4, 0.6) { $m_1$ };
        \node [right] (right) at (0.4, 0.75) { $m_2$ };
    \end{tikzpicture}
}
\answer{%
    Предположим, что левый брусок ускоряется вверх, тогда правый ускоряется вниз (с тем же ускорением).
    Запишем 2-й закон Ньютона 2 раза (для обоих тел) в проекции на вертикальную оси, направив её вверх.
    \begin{align*}
        &\begin{cases}
            T - m_1g = m_1a, \\
            T - m_2g = -m_2a,
        \end{cases} \\
        &\begin{cases}
            m_2g - m_1g = m_1a + m_2a, \\
            T = m_1a + m_1g, \\
        \end{cases} \\
        a &= \frac{m_2 - m_1}{m_1 + m_2} \cdot g = \frac{10\,\text{кг} - 5\,\text{кг}}{5\,\text{кг} + 10\,\text{кг}} \cdot 10\,\frac{\text{м}}{\text{с}^{2}} \approx 3{,}33\,\frac{\text{м}}{\text{c}^{2}}, \\
        T &= m_1(a + g) = m_1 \cdot g \cdot \cbr{\frac{m_2 - m_1}{m_1 + m_2} + 1} = m_1 \cdot g \cdot \frac{2m_2}{m_1 + m_2} = \\
            &= \frac{2 m_2 m_1 g}{m_1 + m_2} = \frac{2 \cdot 10\,\text{кг} \cdot 5\,\text{кг} \cdot 10\,\frac{\text{м}}{\text{с}^{2}}}{5\,\text{кг} + 10\,\text{кг}} \approx 66{,}7\,\text{Н}.
    \end{align*}
    Отрицательный ответ говорит, что мы лишь не угадали с направлением ускорений.
    Сила же всегда положительна.
}
\solutionspace{80pt}

\tasknumber{7}%
\task{%
    Тело массой $2\,\text{кг}$ лежит на горизонтальной поверхности.
    Коэффициент трения между поверхностью и телом $0{,}2$.
    К телу приложена горизонтальная сила $5{,}5\,\text{Н}$.
    Определите силу трения, действующую на тело, и ускорение тела.
    % $g = 10\,\frac{\text{м}}{\text{с}^{2}}$.
}
\answer{%
    \begin{align*}
    &F_\text{ трения покоя $\max$ } = \mu N = \mu m g = 0{,}2 \cdot 2\,\text{кг} \cdot 10\,\frac{\text{м}}{\text{с}^{2}} = 4{,}00\,\text{Н}, \\
    &F_\text{ трения покоя $\max$ } \le F \implies F_\text{ трения } = 4{,}00\,\text{Н}, a = \frac{F - F_\text{ трения }}{ m } = 0{,}75\,\frac{\text{м}}{\text{c}^{2}}, \\
    &\text{при равенстве возможны оба варианта: и едет, и не едет, но на ответы это не влияет.}
    \end{align*}
}
\solutionspace{120pt}

\tasknumber{8}%
\task{%
    Определите плотность неизвестного вещества, если известно, что опускании тела из него
    в керосин оно будет плавать и на половину выступать над поверхностью жидкости.
}
\answer{%
    $F_\text{Арх.} = F_\text{тяж.} \implies \rho_\text{ж.} g V_\text{погр.} = m g \implies\rho_\text{ж.} g \cbr{V -\frac V2} = \rho V g \implies \rho = \rho_\text{ж.}\cbr{1 -\frac 12} \approx 400\,\frac{\text{кг}}{\text{м}^{3}}$
}
\solutionspace{120pt}

\tasknumber{9}%
\task{%
    	Определите силу, действующую на левую опору однородного горизонтального стержня длиной $l = 3\,\text{м}$
    	и массой $M = 1\,\text{кг}$, к которому подвешен груз массой $m = 4\,\text{кг}$ на расстоянии $2\,\text{м}$ от правого конца (см.
    рис.).

        \begin{tikzpicture}[thick]
            \draw
                (-2, -0.1) rectangle (2, 0.1)
                (-0.5, -0.1) -- (-0.5, -1)
                (-0.7, -1) rectangle (-0.3, -1.3)
           		(-2, -0.1) -- +(0.15,-0.9) -- +(-0.15,-0.9) -- cycle
            	(2, -0.1) -- +(0.15,-0.9) -- +(-0.15,-0.9) -- cycle
            ;
            \draw[pattern={Lines[angle=51,distance=2pt]},pattern color=black,draw=none]
            	(-2.15, -1.15) rectangle +(0.3, 0.15)
            	(2.15, -1.15) rectangle +(-0.3, 0.15)
            ;
            \node [right] (m_small) at (-0.3, -1.15) { $m$ };
            \node [above] (M_big) at (0, 0.1) { $M$ };
        \end{tikzpicture}
}
\answer{%
    \begin{align*}
        &\begin{cases}
            F_1 + F_2 - mg - Mg= 0, \\
            F_1 \cdot 0 - mg \cdot a - Mg \cdot \frac l2 + F_2 \cdot l = 0,
        \end{cases} \\
        F_2 &= \frac{mga + Mg\frac l2}l = \frac al \cdot mg + \frac{Mg}2 \approx 18{,}3\,\text{Н}, \\
        F_1 &= mg + Mg - F_2 = mg + Mg - \frac al \cdot mg - \frac{Mg}2 = \frac bl \cdot mg + \frac{Mg}2 \approx 31{,}7\,\text{Н}.
    \end{align*}
}
\solutionspace{80pt}

\tasknumber{10}%
\task{%
    Тонкий однородный кусок арматуры длиной $2\,\text{м}$ и массой $20\,\text{кг}$ лежит на горизонтальной поверхности.
    \begin{itemize}
        \item Какую минимальную силу надо приложить к одному из его концов, чтобы оторвать его от этой поверхности?
        \item Какую минимальную работу надо совершить, чтобы поставить его на землю в вертикальное положение?
    \end{itemize}
    % Примите $g = 10\,\frac{\text{м}}{\text{с}^{2}}$.
}
\answer{%
    $F = \frac{mg}2 \approx 200\,\text{Н}, A = mg\frac l2 = 200\,\text{Дж}$
}
\solutionspace{120pt}

\tasknumber{11}%
\task{%
    Определите работу силы, которая обеспечит спуск тела массой $5\,\text{кг}$ на высоту $2\,\text{м}$ с постоянным ускорением $2\,\frac{\text{м}}{\text{c}^{2}}$.
    % Примите $g = 10\,\frac{\text{м}}{\text{с}^{2}}$.
}
\answer{%
    \begin{align*}
    &\text{Для подъёма:} A = Fh = (mg + ma) h = m(g+a)h, \\
    &\text{Для спуска:} A = -Fh = -(mg - ma) h = -m(g-a)h, \\
    &\text{В результате получаем:} -80\,\text{Дж}.
    \end{align*}
}
\solutionspace{60pt}

\tasknumber{12}%
\task{%
    Тело бросили вертикально вверх со скоростью $10\,\frac{\text{м}}{\text{c}}$.
    На какой высоте кинетическая энергия тела составит половину от потенциальной?
}
\answer{%
    \begin{align*}
    &0 + \frac{mv_0^2}2 = E_p + E_k, E_k = \frac 12 E_p \implies \\
    &\implies \frac{mv_0^2}2 = E_p + \frac 12 E_p = E_p\cbr{1 + \frac 12} = mgh\cbr{1 + \frac 12} \implies \\
    &\implies h = \frac{\frac{mv_0^2}2}{mg\cbr{1 + \frac 12}} = \frac{v_0^2}{2g} \cdot \frac 1{1 + \frac 12} \approx 3{,}3\,\text{м}.
    \end{align*}
}
\solutionspace{100pt}

\tasknumber{13}%
\task{%
    Плотность воздуха при нормальных условиях равна $1{,}3\,\frac{\text{кг}}{\text{м}^{3}}$.
    Чему равна плотность воздуха
    при температуре $100\celsius$ и давлении $150\,\text{кПа}$?
}
\answer{%
    \begin{align*}
    &\text{В общем случае:} PV = \frac m{\mu} RT \implies \rho = \frac mV = \frac m{\frac{\frac m{\mu} RT}P} = \frac{P\mu}{RT}, \\
    &\text{У нас 2 состояния:} \rho_1 = \frac{P_1\mu}{RT_1}, \rho_2 = \frac{P_2\mu}{RT_2} \implies \frac{\rho_2}{\rho_1} = \frac{\frac{P_2\mu}{RT_2}}{\frac{P_1\mu}{RT_1}} = \frac{P_2T_1}{P_1T_2} \implies \\
    &\implies \rho_2 = \rho_1 \cdot  \frac{P_2T_1}{P_1T_2} = 1{,}3\,\frac{\text{кг}}{\text{м}^{3}} \cdot \frac{150\,\text{кПа} \cdot 273\units{К}}{100\,\text{кПа} \cdot 373\units{К}} \approx 1{,}43\,\frac{\text{кг}}{\text{м}^{3}}.
    \end{align*}
}
\solutionspace{120pt}

\tasknumber{14}%
\task{%
    Небольшую цилиндрическую пробирку с воздухом погружают на некоторую глубину в глубокое пресное озеро,
    после чего воздух занимает в ней лишь пятую часть от общего объема.
    Определите глубину, на которую погрузили пробирку.
    Температуру считать постоянной $T = 280\,\text{К}$, давлением паров воды пренебречь,
    атмосферное давление принять равным $p_{\text{aтм}} = 100\,\text{кПа}$.
}
\answer{%
    \begin{align*}
    T\text{— const} &\implies P_1V_1 = \nu RT = P_2V_2.
    \\
    V_2 = \frac 15 V_1 &\implies P_1V_1 = P_2 \cdot \frac 15V_1 \implies P_2 = 5P_1 = 5p_{\text{aтм}}.
    \\
    P_2 = p_{\text{aтм}} + \rho_{\text{в}} g h \implies h = \frac{P_2 - p_{\text{aтм}}}{\rho_{\text{в}} g} &= \frac{5p_{\text{aтм}} - p_{\text{aтм}}}{\rho_{\text{в}} g} = \frac{4 \cdot p_{\text{aтм}}}{\rho_{\text{в}} g} =  \\
     &= \frac{4 \cdot 100\,\text{кПа}}{1000\,\frac{\text{кг}}{\text{м}^{3}} \cdot  10\,\frac{\text{м}}{\text{с}^{2}}} \approx 40\,\text{м}.
    \end{align*}
}
\solutionspace{120pt}

\tasknumber{15}%
\task{%
    Газу сообщили некоторое количество теплоты,
    при этом четверть его он потратил на совершение работы,
    одновременно увеличив свою внутреннюю энергию на $3000\,\text{Дж}$.
    Определите количество теплоты, сообщённое газу.
}
\answer{%
    \begin{align*}
    Q &= A' + \Delta U, A' = \frac 14 Q \implies Q \cdot \cbr{1 - \frac 14} = \Delta U \implies Q = \frac{\Delta U}{1 - \frac 14} = \frac{ 3000\,\text{Дж} }{1 - \frac 14} \approx 4000\,\text{Дж}.
    \\
    A' &= \frac 14 Q
        = \frac 14 \cdot \frac{\Delta U}{1 - \frac 14}
        = \frac{\Delta U}{4 - 1}
        = \frac{ 3000\,\text{Дж} }{4 - 1} \approx 1000\,\text{Дж}.
    \end{align*}
}
\solutionspace{60pt}

\tasknumber{16}%
\task{%
    Два конденсатора ёмкостей $C_1 = 20\,\text{нФ}$ и $C_2 = 30\,\text{нФ}$ последовательно подключают
    к источнику напряжения $V = 300\,\text{В}$ (см.
    рис.).
    % Определите заряды каждого из конденсаторов.
    Определите заряд второго конденсатора.

    \begin{tikzpicture}[circuit ee IEC, semithick]
        \draw  (0, 0) to [capacitor={info={$C_1$}}] (1, 0)
                       to [capacitor={info={$C_2$}}] (2, 0)
        ;
        % \draw [-o] (0, 0) -- ++(-0.5, 0) node[left] {$-$};
        % \draw [-o] (2, 0) -- ++(0.5, 0) node[right] {$+$};
        \draw [-o] (0, 0) -- ++(-0.5, 0) node[left] {};
        \draw [-o] (2, 0) -- ++(0.5, 0) node[right] {};
    \end{tikzpicture}
}
\answer{%
    $
        Q_1
            = Q_2
            = CV
            = \frac{ V }{\frac1{C_1} + \frac1{C_2}}
            = \frac{C_1C_2V}{C_1 + C_2}
            = \frac{
                20\,\text{нФ} \cdot 30\,\text{нФ} \cdot 300\,\text{В}
            }{
                20\,\text{нФ} + 30\,\text{нФ}
            }
            = 3{,}60\,\text{мкКл}
    $
}
\solutionspace{120pt}

\tasknumber{17}%
\task{%
    В вакууме вдоль одной прямой расположены три положительных заряда так,
    что расстояние между соседними зарядами равно $a$.
    Сделайте рисунок,
    и определите силу, действующую на крайний заряд.
    Модули всех зарядов равны $Q$ ($Q > 0$).
}
\answer{%
    $F = \sum_i F_i = \ldots = \frac54 \frac{kQ^2}{a^2}.$
}
\solutionspace{80pt}

\tasknumber{18}%
\task{%
    Юлия проводит эксперименты c 2 кусками одинаковой алюминиевой проволки, причём второй кусок в пять раз длиннее первого.
    В одном из экспериментов Юлия подаёт на первый кусок проволки напряжение в десять раз раз больше, чем на второй.
    Определите отношения в двух проволках в этом эксперименте (второй к первой):
    \begin{itemize}
        \item отношение сил тока,
        \item отношение выделяющихся мощностей.
    \end{itemize}
}
\answer{%
    $R_2 = 5R_1, U_1 = 10U_2 \implies  \eli_2 / \eli_1 = \frac{U_2 / R_2}{U_1 / R_1} = \frac{U_2}{U_1} \cdot \frac{R_1}{R_2} = \frac1{50}, P_2 / P_1 = \frac{U_2^2 / R_2}{U_1^2 / R_1} = \sqr{\frac{U_2}{U_1}} \cdot \frac{R_1}{R_2} = \frac1{500}.$
}

\variantsplitter

\addpersonalvariant{Софья Андрианова}

\tasknumber{1}%
\task{%
    Саша стартует на мотоцикле и в течение $t = 4\,\text{c}$ двигается с постоянным ускорением $2{,}5\,\frac{\text{м}}{\text{с}^{2}}$.
    Определите
    \begin{itemize}
        \item какую скорость при этом удастся достичь,
        \item какой путь за это время будет пройден,
        \item среднюю скорость за всё время движения, если после начального ускорения продолжить движение равномерно ещё в течение времени $3t$
    \end{itemize}
}
\answer{%
    \begin{align*}
    v &= v_0 + a t = at = 2{,}5\,\frac{\text{м}}{\text{с}^{2}} \cdot 4\,\text{c} = 10{,}0\,\frac{\text{м}}{\text{с}}, \\
    s_x &= v_0t + \frac{a t^2}2 = \frac{a t^2}2 = \frac{2{,}5\,\frac{\text{м}}{\text{с}^{2}} \cdot \sqr{ 4\,\text{c} }}2 = 20{,}0\,\text{м}, \\
    v_\text{сред.} &= \frac{s_\text{общ}}{t_\text{общ.}} = \frac{s_x + v \cdot 3t}{t + 3t} = \frac{\frac{a t^2}2 + at \cdot 3t}{t (1 + 3)} = \\
    &= at \cdot \frac{\frac 12 + 3}{1 + 3} = 2{,}5\,\frac{\text{м}}{\text{с}^{2}} \cdot 4\,\text{c} \cdot \frac{\frac 12 + 3}{1 + 3} \approx 8{,}75\,\frac{\text{м}}{\text{c}}.
    \end{align*}
}
\solutionspace{120pt}

\tasknumber{2}%
\task{%
    Какой путь тело пройдёт за четвёртую секунду после начала свободного падения?
    Какую скорость в начале этой секунды оно имеет?
}
\answer{%
    \begin{align*}
    s &= -s_y = -(y_2-y_1) = y_1 - y_2 = \cbr{y_{0y} + v_{0y}t_1 - \frac{gt_1^2}2} - \cbr{y_{0y} + v_{0y}t_2 - \frac{gt_2^2}2} = \\
    &= \frac{gt_2^2}2 - \frac{gt_1^2}2 = \frac g2\cbr{t_2^2 - t_1^2} = 35{,}0\,\text{м}, \\
    v_y &= v_{0y} - gt = -gt = 10\,\frac{\text{м}}{\text{с}^{2}} \cdot 3\,\text{с} = -30\,\frac{\text{м}}{\text{с}}.
    \end{align*}
}
\solutionspace{120pt}

\tasknumber{3}%
\task{%
    Карусель диаметром $3\,\text{м}$ равномерно совершает 6 оборотов в минуту.
    Определите
    \begin{itemize}
        \item период и частоту её обращения,
        \item скорость и ускорение крайних её точек.
    \end{itemize}
}
\answer{%
    \begin{align*}
    t &= 60\,\text{с}, r = 1{,}5\,\text{м}, n = 6\units{оборотов}, \\
    T &= \frac tN = \frac{ 60\,\text{с} }{6} \approx 10{,}00\,\text{c}, \\
    \nu &= \frac 1T = \frac{6}{ 60\,\text{с} } \approx 0{,}10\,\text{Гц}, \\
    v &= \frac{2 \pi r}{T} = \frac{2 \pi r}{T} =  \frac{2 \pi r n}{t} \approx 0{,}94\,\frac{\text{м}}{\text{c}}, \\
    a &= \frac{v^2}{r} =  \frac{4 \pi^2 r n^2}{t^2} \approx 0{,}59\,\frac{\text{м}}{\text{с}^{2}}.
    \end{align*}
}
\solutionspace{80pt}

\tasknumber{4}%
\task{%
    Даша стоит на обрыве над рекой и методично и строго горизонтально кидает в неё камушки.
    За этим всем наблюдает экспериментатор Глюк, который уже выяснил, что камушки падают в реку спустя $1{,}3\,\text{с}$ после броска,
    а вот дальность полёта оценить сложнее: придётся лезть в воду.
    Выручите Глюка и определите:
    \begin{itemize}
        \item высоту обрыва (вместе с ростом Даши).
        \item дальность полёта камушков (по горизонтали) и их скорость при падении, приняв начальную скорость броска равной $v_0 = 18\,\frac{\text{м}}{\text{с}}$.
    \end{itemize}
    Сопротивлением воздуха пренебречь.
}
\answer{%
    \begin{align*}
    y &= y_0 + v_{0y}t - \frac{gt^2}2 = h - \frac{gt^2}2, \qquad y(\tau) = 0 \implies h - \frac{g\tau^2}2 = 0 \implies h = \frac{g\tau^2}2 \approx 8{,}5\,\text{м}.
    \\
    x &= x_0 + v_{0x}t = v_0t \implies L = v_0\tau \approx 23{,}4\,\text{м}.
    \\
    &v = \sqrt{v_x^2 + v_y^2} = \sqrt{v_{0x}^2 + \sqr{v_{0y} - g\tau}} = \sqrt{v_0^2 + \sqr{g\tau}} \approx 22{,}2\,\frac{\text{м}}{\text{c}}.
    \end{align*}
}
\solutionspace{120pt}

\tasknumber{5}%
\task{%
    Шесть одинаковых брусков массой $3\,\text{кг}$ каждый лежат на гладком горизонтальном столе.
    Бруски пронумерованы от 1 до 6 и последовательно связаны между собой
    невесомыми нерастяжимыми нитями: 1 со 2, 2 с 3 (ну и с 1) и т.д.
    Экспериментатор Глюк прикладывает постоянную горизонтальную силу $120\,\text{Н}$ к бруску с наименьшим номером.
    С каким ускорением двигается система? Чему равна сила натяжения нити, связывающей бруски 1 и 2?
}
\answer{%
    \begin{align*}
    a &= \frac{F}{6 m} = \frac{120\,\text{Н}}{6 \cdot 3\,\text{кг}} \approx 6{,}7\,\frac{\text{м}}{\text{c}^{2}}, \\
    T &= m'a = 5m \cdot \frac{F}{6 m} = \frac{5}{6} F \approx 100{,}0\,\text{Н}.
    \end{align*}
}
\solutionspace{120pt}

\tasknumber{6}%
\task{%
    Два бруска связаны лёгкой нерастяжимой нитью и перекинуты через неподвижный блок (см.
    рис.).
    Определите силу натяжения нити и ускорения брусков.
    Силами трения пренебречь, массы брусков
    равны $m_1 = 11\,\text{кг}$ и $m_2 = 10\,\text{кг}$.
    % $g = 10\,\frac{\text{м}}{\text{с}^{2}}$.

    \begin{tikzpicture}[x=1.5cm,y=1.5cm,thick]
        \draw
            (-0.4, 0) rectangle (-0.2, 1.2)
            (0.15, 0.5) rectangle (0.45, 1)
            (0, 2) circle [radius=0.3] -- ++(up:0.5)
            (-0.3, 1.2) -- ++(up:0.8)
            (0.3, 1) -- ++(up:1)
            (-0.7, 2.5) -- (0.7, 2.5)
            ;
        \draw[pattern={Lines[angle=51,distance=3pt]},pattern color=black,draw=none] (-0.7, 2.5) rectangle (0.7, 2.75);
        \node [left] (left) at (-0.4, 0.6) { $m_1$ };
        \node [right] (right) at (0.4, 0.75) { $m_2$ };
    \end{tikzpicture}
}
\answer{%
    Предположим, что левый брусок ускоряется вверх, тогда правый ускоряется вниз (с тем же ускорением).
    Запишем 2-й закон Ньютона 2 раза (для обоих тел) в проекции на вертикальную оси, направив её вверх.
    \begin{align*}
        &\begin{cases}
            T - m_1g = m_1a, \\
            T - m_2g = -m_2a,
        \end{cases} \\
        &\begin{cases}
            m_2g - m_1g = m_1a + m_2a, \\
            T = m_1a + m_1g, \\
        \end{cases} \\
        a &= \frac{m_2 - m_1}{m_1 + m_2} \cdot g = \frac{10\,\text{кг} - 11\,\text{кг}}{11\,\text{кг} + 10\,\text{кг}} \cdot 10\,\frac{\text{м}}{\text{с}^{2}} \approx -0{,}4800\,\frac{\text{м}}{\text{c}^{2}}, \\
        T &= m_1(a + g) = m_1 \cdot g \cdot \cbr{\frac{m_2 - m_1}{m_1 + m_2} + 1} = m_1 \cdot g \cdot \frac{2m_2}{m_1 + m_2} = \\
            &= \frac{2 m_2 m_1 g}{m_1 + m_2} = \frac{2 \cdot 10\,\text{кг} \cdot 11\,\text{кг} \cdot 10\,\frac{\text{м}}{\text{с}^{2}}}{11\,\text{кг} + 10\,\text{кг}} \approx 104{,}8\,\text{Н}.
    \end{align*}
    Отрицательный ответ говорит, что мы лишь не угадали с направлением ускорений.
    Сила же всегда положительна.
}
\solutionspace{80pt}

\tasknumber{7}%
\task{%
    Тело массой $2\,\text{кг}$ лежит на горизонтальной поверхности.
    Коэффициент трения между поверхностью и телом $0{,}2$.
    К телу приложена горизонтальная сила $3{,}5\,\text{Н}$.
    Определите силу трения, действующую на тело, и ускорение тела.
    % $g = 10\,\frac{\text{м}}{\text{с}^{2}}$.
}
\answer{%
    \begin{align*}
    &F_\text{ трения покоя $\max$ } = \mu N = \mu m g = 0{,}2 \cdot 2\,\text{кг} \cdot 10\,\frac{\text{м}}{\text{с}^{2}} = 4{,}00\,\text{Н}, \\
    &F_\text{ трения покоя $\max$ } > F \implies F_\text{ трения } = 3{,}50\,\text{Н}, a = \frac{F - F_\text{ трения }}{ m } = 0\,\frac{\text{м}}{\text{c}^{2}}, \\
    &\text{при равенстве возможны оба варианта: и едет, и не едет, но на ответы это не влияет.}
    \end{align*}
}
\solutionspace{120pt}

\tasknumber{8}%
\task{%
    Определите плотность неизвестного вещества, если известно, что опускании тела из него
    в подсолнечное масло оно будет плавать и на половину выступать над поверхностью жидкости.
}
\answer{%
    $F_\text{Арх.} = F_\text{тяж.} \implies \rho_\text{ж.} g V_\text{погр.} = m g \implies\rho_\text{ж.} g \cbr{V -\frac V2} = \rho V g \implies \rho = \rho_\text{ж.}\cbr{1 -\frac 12} \approx 450\,\frac{\text{кг}}{\text{м}^{3}}$
}
\solutionspace{120pt}

\tasknumber{9}%
\task{%
    	Определите силу, действующую на левую опору однородного горизонтального стержня длиной $l = 7\,\text{м}$
    	и массой $M = 5\,\text{кг}$, к которому подвешен груз массой $m = 3\,\text{кг}$ на расстоянии $4\,\text{м}$ от правого конца (см.
    рис.).

        \begin{tikzpicture}[thick]
            \draw
                (-2, -0.1) rectangle (2, 0.1)
                (-0.5, -0.1) -- (-0.5, -1)
                (-0.7, -1) rectangle (-0.3, -1.3)
           		(-2, -0.1) -- +(0.15,-0.9) -- +(-0.15,-0.9) -- cycle
            	(2, -0.1) -- +(0.15,-0.9) -- +(-0.15,-0.9) -- cycle
            ;
            \draw[pattern={Lines[angle=51,distance=2pt]},pattern color=black,draw=none]
            	(-2.15, -1.15) rectangle +(0.3, 0.15)
            	(2.15, -1.15) rectangle +(-0.3, 0.15)
            ;
            \node [right] (m_small) at (-0.3, -1.15) { $m$ };
            \node [above] (M_big) at (0, 0.1) { $M$ };
        \end{tikzpicture}
}
\answer{%
    \begin{align*}
        &\begin{cases}
            F_1 + F_2 - mg - Mg= 0, \\
            F_1 \cdot 0 - mg \cdot a - Mg \cdot \frac l2 + F_2 \cdot l = 0,
        \end{cases} \\
        F_2 &= \frac{mga + Mg\frac l2}l = \frac al \cdot mg + \frac{Mg}2 \approx 37{,}9\,\text{Н}, \\
        F_1 &= mg + Mg - F_2 = mg + Mg - \frac al \cdot mg - \frac{Mg}2 = \frac bl \cdot mg + \frac{Mg}2 \approx 42{,}1\,\text{Н}.
    \end{align*}
}
\solutionspace{80pt}

\tasknumber{10}%
\task{%
    Тонкий однородный лом длиной $1\,\text{м}$ и массой $10\,\text{кг}$ лежит на горизонтальной поверхности.
    \begin{itemize}
        \item Какую минимальную силу надо приложить к одному из его концов, чтобы оторвать его от этой поверхности?
        \item Какую минимальную работу надо совершить, чтобы поставить его на землю в вертикальное положение?
    \end{itemize}
    % Примите $g = 10\,\frac{\text{м}}{\text{с}^{2}}$.
}
\answer{%
    $F = \frac{mg}2 \approx 100\,\text{Н}, A = mg\frac l2 = 50\,\text{Дж}$
}
\solutionspace{120pt}

\tasknumber{11}%
\task{%
    Определите работу силы, которая обеспечит подъём тела массой $3\,\text{кг}$ на высоту $10\,\text{м}$ с постоянным ускорением $6\,\frac{\text{м}}{\text{c}^{2}}$.
    % Примите $g = 10\,\frac{\text{м}}{\text{с}^{2}}$.
}
\answer{%
    \begin{align*}
    &\text{Для подъёма:} A = Fh = (mg + ma) h = m(g+a)h, \\
    &\text{Для спуска:} A = -Fh = -(mg - ma) h = -m(g-a)h, \\
    &\text{В результате получаем:} 480\,\text{Дж}.
    \end{align*}
}
\solutionspace{60pt}

\tasknumber{12}%
\task{%
    Тело бросили вертикально вверх со скоростью $10\,\frac{\text{м}}{\text{c}}$.
    На какой высоте кинетическая энергия тела составит треть от потенциальной?
}
\answer{%
    \begin{align*}
    &0 + \frac{mv_0^2}2 = E_p + E_k, E_k = \frac 13 E_p \implies \\
    &\implies \frac{mv_0^2}2 = E_p + \frac 13 E_p = E_p\cbr{1 + \frac 13} = mgh\cbr{1 + \frac 13} \implies \\
    &\implies h = \frac{\frac{mv_0^2}2}{mg\cbr{1 + \frac 13}} = \frac{v_0^2}{2g} \cdot \frac 1{1 + \frac 13} \approx 3{,}8\,\text{м}.
    \end{align*}
}
\solutionspace{100pt}

\tasknumber{13}%
\task{%
    Плотность воздуха при нормальных условиях равна $1{,}3\,\frac{\text{кг}}{\text{м}^{3}}$.
    Чему равна плотность воздуха
    при температуре $200\celsius$ и давлении $120\,\text{кПа}$?
}
\answer{%
    \begin{align*}
    &\text{В общем случае:} PV = \frac m{\mu} RT \implies \rho = \frac mV = \frac m{\frac{\frac m{\mu} RT}P} = \frac{P\mu}{RT}, \\
    &\text{У нас 2 состояния:} \rho_1 = \frac{P_1\mu}{RT_1}, \rho_2 = \frac{P_2\mu}{RT_2} \implies \frac{\rho_2}{\rho_1} = \frac{\frac{P_2\mu}{RT_2}}{\frac{P_1\mu}{RT_1}} = \frac{P_2T_1}{P_1T_2} \implies \\
    &\implies \rho_2 = \rho_1 \cdot  \frac{P_2T_1}{P_1T_2} = 1{,}3\,\frac{\text{кг}}{\text{м}^{3}} \cdot \frac{120\,\text{кПа} \cdot 273\units{К}}{100\,\text{кПа} \cdot 473\units{К}} \approx 0{,}90\,\frac{\text{кг}}{\text{м}^{3}}.
    \end{align*}
}
\solutionspace{120pt}

\tasknumber{14}%
\task{%
    Небольшую цилиндрическую пробирку с воздухом погружают на некоторую глубину в глубокое пресное озеро,
    после чего воздух занимает в ней лишь пятую часть от общего объема.
    Определите глубину, на которую погрузили пробирку.
    Температуру считать постоянной $T = 292\,\text{К}$, давлением паров воды пренебречь,
    атмосферное давление принять равным $p_{\text{aтм}} = 100\,\text{кПа}$.
}
\answer{%
    \begin{align*}
    T\text{— const} &\implies P_1V_1 = \nu RT = P_2V_2.
    \\
    V_2 = \frac 15 V_1 &\implies P_1V_1 = P_2 \cdot \frac 15V_1 \implies P_2 = 5P_1 = 5p_{\text{aтм}}.
    \\
    P_2 = p_{\text{aтм}} + \rho_{\text{в}} g h \implies h = \frac{P_2 - p_{\text{aтм}}}{\rho_{\text{в}} g} &= \frac{5p_{\text{aтм}} - p_{\text{aтм}}}{\rho_{\text{в}} g} = \frac{4 \cdot p_{\text{aтм}}}{\rho_{\text{в}} g} =  \\
     &= \frac{4 \cdot 100\,\text{кПа}}{1000\,\frac{\text{кг}}{\text{м}^{3}} \cdot  10\,\frac{\text{м}}{\text{с}^{2}}} \approx 40\,\text{м}.
    \end{align*}
}
\solutionspace{120pt}

\tasknumber{15}%
\task{%
    Газу сообщили некоторое количество теплоты,
    при этом треть его он потратил на совершение работы,
    одновременно увеличив свою внутреннюю энергию на $2400\,\text{Дж}$.
    Определите работу, совершённую газом.
}
\answer{%
    \begin{align*}
    Q &= A' + \Delta U, A' = \frac 13 Q \implies Q \cdot \cbr{1 - \frac 13} = \Delta U \implies Q = \frac{\Delta U}{1 - \frac 13} = \frac{ 2400\,\text{Дж} }{1 - \frac 13} \approx 3600\,\text{Дж}.
    \\
    A' &= \frac 13 Q
        = \frac 13 \cdot \frac{\Delta U}{1 - \frac 13}
        = \frac{\Delta U}{3 - 1}
        = \frac{ 2400\,\text{Дж} }{3 - 1} \approx 1200\,\text{Дж}.
    \end{align*}
}
\solutionspace{60pt}

\tasknumber{16}%
\task{%
    Два конденсатора ёмкостей $C_1 = 60\,\text{нФ}$ и $C_2 = 40\,\text{нФ}$ последовательно подключают
    к источнику напряжения $V = 450\,\text{В}$ (см.
    рис.).
    % Определите заряды каждого из конденсаторов.
    Определите заряд первого конденсатора.

    \begin{tikzpicture}[circuit ee IEC, semithick]
        \draw  (0, 0) to [capacitor={info={$C_1$}}] (1, 0)
                       to [capacitor={info={$C_2$}}] (2, 0)
        ;
        % \draw [-o] (0, 0) -- ++(-0.5, 0) node[left] {$-$};
        % \draw [-o] (2, 0) -- ++(0.5, 0) node[right] {$+$};
        \draw [-o] (0, 0) -- ++(-0.5, 0) node[left] {};
        \draw [-o] (2, 0) -- ++(0.5, 0) node[right] {};
    \end{tikzpicture}
}
\answer{%
    $
        Q_1
            = Q_2
            = CV
            = \frac{ V }{\frac1{C_1} + \frac1{C_2}}
            = \frac{C_1C_2V}{C_1 + C_2}
            = \frac{
                60\,\text{нФ} \cdot 40\,\text{нФ} \cdot 450\,\text{В}
            }{
                60\,\text{нФ} + 40\,\text{нФ}
            }
            = 10{,}80\,\text{мкКл}
    $
}
\solutionspace{120pt}

\tasknumber{17}%
\task{%
    В вакууме вдоль одной прямой расположены три положительных заряда так,
    что расстояние между соседними зарядами равно $l$.
    Сделайте рисунок,
    и определите силу, действующую на крайний заряд.
    Модули всех зарядов равны $Q$ ($Q > 0$).
}
\answer{%
    $F = \sum_i F_i = \ldots = \frac54 \frac{kQ^2}{l^2}.$
}
\solutionspace{80pt}

\tasknumber{18}%
\task{%
    Юлия проводит эксперименты c 2 кусками одинаковой медной проволки, причём второй кусок в семь раз длиннее первого.
    В одном из экспериментов Юлия подаёт на первый кусок проволки напряжение в три раза раз больше, чем на второй.
    Определите отношения в двух проволках в этом эксперименте (второй к первой):
    \begin{itemize}
        \item отношение сил тока,
        \item отношение выделяющихся мощностей.
    \end{itemize}
}
\answer{%
    $R_2 = 7R_1, U_1 = 3U_2 \implies  \eli_2 / \eli_1 = \frac{U_2 / R_2}{U_1 / R_1} = \frac{U_2}{U_1} \cdot \frac{R_1}{R_2} = \frac1{21}, P_2 / P_1 = \frac{U_2^2 / R_2}{U_1^2 / R_1} = \sqr{\frac{U_2}{U_1}} \cdot \frac{R_1}{R_2} = \frac1{63}.$
}

\variantsplitter

\addpersonalvariant{Владимир Артемчук}

\tasknumber{1}%
\task{%
    Женя стартует на мотоцикле и в течение $t = 3\,\text{c}$ двигается с постоянным ускорением $0{,}5\,\frac{\text{м}}{\text{с}^{2}}$.
    Определите
    \begin{itemize}
        \item какую скорость при этом удастся достичь,
        \item какой путь за это время будет пройден,
        \item среднюю скорость за всё время движения, если после начального ускорения продолжить движение равномерно ещё в течение времени $2t$
    \end{itemize}
}
\answer{%
    \begin{align*}
    v &= v_0 + a t = at = 0{,}5\,\frac{\text{м}}{\text{с}^{2}} \cdot 3\,\text{c} = 1{,}5\,\frac{\text{м}}{\text{с}}, \\
    s_x &= v_0t + \frac{a t^2}2 = \frac{a t^2}2 = \frac{0{,}5\,\frac{\text{м}}{\text{с}^{2}} \cdot \sqr{ 3\,\text{c} }}2 = 2{,}2\,\text{м}, \\
    v_\text{сред.} &= \frac{s_\text{общ}}{t_\text{общ.}} = \frac{s_x + v \cdot 2t}{t + 2t} = \frac{\frac{a t^2}2 + at \cdot 2t}{t (1 + 2)} = \\
    &= at \cdot \frac{\frac 12 + 2}{1 + 2} = 0{,}5\,\frac{\text{м}}{\text{с}^{2}} \cdot 3\,\text{c} \cdot \frac{\frac 12 + 2}{1 + 2} \approx 1{,}25\,\frac{\text{м}}{\text{c}}.
    \end{align*}
}
\solutionspace{120pt}

\tasknumber{2}%
\task{%
    Какой путь тело пройдёт за третью секунду после начала свободного падения?
    Какую скорость в начале этой секунды оно имеет?
}
\answer{%
    \begin{align*}
    s &= -s_y = -(y_2-y_1) = y_1 - y_2 = \cbr{y_{0y} + v_{0y}t_1 - \frac{gt_1^2}2} - \cbr{y_{0y} + v_{0y}t_2 - \frac{gt_2^2}2} = \\
    &= \frac{gt_2^2}2 - \frac{gt_1^2}2 = \frac g2\cbr{t_2^2 - t_1^2} = 25{,}0\,\text{м}, \\
    v_y &= v_{0y} - gt = -gt = 10\,\frac{\text{м}}{\text{с}^{2}} \cdot 2\,\text{с} = -20\,\frac{\text{м}}{\text{с}}.
    \end{align*}
}
\solutionspace{120pt}

\tasknumber{3}%
\task{%
    Карусель диаметром $3\,\text{м}$ равномерно совершает 10 оборотов в минуту.
    Определите
    \begin{itemize}
        \item период и частоту её обращения,
        \item скорость и ускорение крайних её точек.
    \end{itemize}
}
\answer{%
    \begin{align*}
    t &= 60\,\text{с}, r = 1{,}5\,\text{м}, n = 10\units{оборотов}, \\
    T &= \frac tN = \frac{ 60\,\text{с} }{10} \approx 6{,}00\,\text{c}, \\
    \nu &= \frac 1T = \frac{10}{ 60\,\text{с} } \approx 0{,}17\,\text{Гц}, \\
    v &= \frac{2 \pi r}{T} = \frac{2 \pi r}{T} =  \frac{2 \pi r n}{t} \approx 1{,}57\,\frac{\text{м}}{\text{c}}, \\
    a &= \frac{v^2}{r} =  \frac{4 \pi^2 r n^2}{t^2} \approx 1{,}64\,\frac{\text{м}}{\text{с}^{2}}.
    \end{align*}
}
\solutionspace{80pt}

\tasknumber{4}%
\task{%
    Миша стоит на обрыве над рекой и методично и строго горизонтально кидает в неё камушки.
    За этим всем наблюдает экспериментатор Глюк, который уже выяснил, что камушки падают в реку спустя $1{,}3\,\text{с}$ после броска,
    а вот дальность полёта оценить сложнее: придётся лезть в воду.
    Выручите Глюка и определите:
    \begin{itemize}
        \item высоту обрыва (вместе с ростом Миши).
        \item дальность полёта камушков (по горизонтали) и их скорость при падении, приняв начальную скорость броска равной $v_0 = 17\,\frac{\text{м}}{\text{с}}$.
    \end{itemize}
    Сопротивлением воздуха пренебречь.
}
\answer{%
    \begin{align*}
    y &= y_0 + v_{0y}t - \frac{gt^2}2 = h - \frac{gt^2}2, \qquad y(\tau) = 0 \implies h - \frac{g\tau^2}2 = 0 \implies h = \frac{g\tau^2}2 \approx 8{,}5\,\text{м}.
    \\
    x &= x_0 + v_{0x}t = v_0t \implies L = v_0\tau \approx 22{,}1\,\text{м}.
    \\
    &v = \sqrt{v_x^2 + v_y^2} = \sqrt{v_{0x}^2 + \sqr{v_{0y} - g\tau}} = \sqrt{v_0^2 + \sqr{g\tau}} \approx 21{,}4\,\frac{\text{м}}{\text{c}}.
    \end{align*}
}
\solutionspace{120pt}

\tasknumber{5}%
\task{%
    Четыре одинаковых брусков массой $2\,\text{кг}$ каждый лежат на гладком горизонтальном столе.
    Бруски пронумерованы от 1 до 4 и последовательно связаны между собой
    невесомыми нерастяжимыми нитями: 1 со 2, 2 с 3 (ну и с 1) и т.д.
    Экспериментатор Глюк прикладывает постоянную горизонтальную силу $90\,\text{Н}$ к бруску с наименьшим номером.
    С каким ускорением двигается система? Чему равна сила натяжения нити, связывающей бруски 3 и 4?
}
\answer{%
    \begin{align*}
    a &= \frac{F}{4 m} = \frac{90\,\text{Н}}{4 \cdot 2\,\text{кг}} \approx 11{,}2\,\frac{\text{м}}{\text{c}^{2}}, \\
    T &= m'a = 1m \cdot \frac{F}{4 m} = \frac{1}{4} F \approx 22{,}5\,\text{Н}.
    \end{align*}
}
\solutionspace{120pt}

\tasknumber{6}%
\task{%
    Два бруска связаны лёгкой нерастяжимой нитью и перекинуты через неподвижный блок (см.
    рис.).
    Определите силу натяжения нити и ускорения брусков.
    Силами трения пренебречь, массы брусков
    равны $m_1 = 8\,\text{кг}$ и $m_2 = 4\,\text{кг}$.
    % $g = 10\,\frac{\text{м}}{\text{с}^{2}}$.

    \begin{tikzpicture}[x=1.5cm,y=1.5cm,thick]
        \draw
            (-0.4, 0) rectangle (-0.2, 1.2)
            (0.15, 0.5) rectangle (0.45, 1)
            (0, 2) circle [radius=0.3] -- ++(up:0.5)
            (-0.3, 1.2) -- ++(up:0.8)
            (0.3, 1) -- ++(up:1)
            (-0.7, 2.5) -- (0.7, 2.5)
            ;
        \draw[pattern={Lines[angle=51,distance=3pt]},pattern color=black,draw=none] (-0.7, 2.5) rectangle (0.7, 2.75);
        \node [left] (left) at (-0.4, 0.6) { $m_1$ };
        \node [right] (right) at (0.4, 0.75) { $m_2$ };
    \end{tikzpicture}
}
\answer{%
    Предположим, что левый брусок ускоряется вверх, тогда правый ускоряется вниз (с тем же ускорением).
    Запишем 2-й закон Ньютона 2 раза (для обоих тел) в проекции на вертикальную оси, направив её вверх.
    \begin{align*}
        &\begin{cases}
            T - m_1g = m_1a, \\
            T - m_2g = -m_2a,
        \end{cases} \\
        &\begin{cases}
            m_2g - m_1g = m_1a + m_2a, \\
            T = m_1a + m_1g, \\
        \end{cases} \\
        a &= \frac{m_2 - m_1}{m_1 + m_2} \cdot g = \frac{4\,\text{кг} - 8\,\text{кг}}{8\,\text{кг} + 4\,\text{кг}} \cdot 10\,\frac{\text{м}}{\text{с}^{2}} \approx -3{,}330\,\frac{\text{м}}{\text{c}^{2}}, \\
        T &= m_1(a + g) = m_1 \cdot g \cdot \cbr{\frac{m_2 - m_1}{m_1 + m_2} + 1} = m_1 \cdot g \cdot \frac{2m_2}{m_1 + m_2} = \\
            &= \frac{2 m_2 m_1 g}{m_1 + m_2} = \frac{2 \cdot 4\,\text{кг} \cdot 8\,\text{кг} \cdot 10\,\frac{\text{м}}{\text{с}^{2}}}{8\,\text{кг} + 4\,\text{кг}} \approx 53{,}3\,\text{Н}.
    \end{align*}
    Отрицательный ответ говорит, что мы лишь не угадали с направлением ускорений.
    Сила же всегда положительна.
}
\solutionspace{80pt}

\tasknumber{7}%
\task{%
    Тело массой $2{,}7\,\text{кг}$ лежит на горизонтальной поверхности.
    Коэффициент трения между поверхностью и телом $0{,}15$.
    К телу приложена горизонтальная сила $4{,}5\,\text{Н}$.
    Определите силу трения, действующую на тело, и ускорение тела.
    % $g = 10\,\frac{\text{м}}{\text{с}^{2}}$.
}
\answer{%
    \begin{align*}
    &F_\text{ трения покоя $\max$ } = \mu N = \mu m g = 0{,}15 \cdot 2{,}7\,\text{кг} \cdot 10\,\frac{\text{м}}{\text{с}^{2}} = 4{,}05\,\text{Н}, \\
    &F_\text{ трения покоя $\max$ } \le F \implies F_\text{ трения } = 4{,}05\,\text{Н}, a = \frac{F - F_\text{ трения }}{ m } = 0{,}17\,\frac{\text{м}}{\text{c}^{2}}, \\
    &\text{при равенстве возможны оба варианта: и едет, и не едет, но на ответы это не влияет.}
    \end{align*}
}
\solutionspace{120pt}

\tasknumber{8}%
\task{%
    Определите плотность неизвестного вещества, если известно, что опускании тела из него
    в керосин оно будет плавать и на четверть выступать над поверхностью жидкости.
}
\answer{%
    $F_\text{Арх.} = F_\text{тяж.} \implies \rho_\text{ж.} g V_\text{погр.} = m g \implies\rho_\text{ж.} g \cbr{V -\frac V4} = \rho V g \implies \rho = \rho_\text{ж.}\cbr{1 -\frac 14} \approx 600\,\frac{\text{кг}}{\text{м}^{3}}$
}
\solutionspace{120pt}

\tasknumber{9}%
\task{%
    	Определите силу, действующую на левую опору однородного горизонтального стержня длиной $l = 7\,\text{м}$
    	и массой $M = 1\,\text{кг}$, к которому подвешен груз массой $m = 4\,\text{кг}$ на расстоянии $2\,\text{м}$ от правого конца (см.
    рис.).

        \begin{tikzpicture}[thick]
            \draw
                (-2, -0.1) rectangle (2, 0.1)
                (-0.5, -0.1) -- (-0.5, -1)
                (-0.7, -1) rectangle (-0.3, -1.3)
           		(-2, -0.1) -- +(0.15,-0.9) -- +(-0.15,-0.9) -- cycle
            	(2, -0.1) -- +(0.15,-0.9) -- +(-0.15,-0.9) -- cycle
            ;
            \draw[pattern={Lines[angle=51,distance=2pt]},pattern color=black,draw=none]
            	(-2.15, -1.15) rectangle +(0.3, 0.15)
            	(2.15, -1.15) rectangle +(-0.3, 0.15)
            ;
            \node [right] (m_small) at (-0.3, -1.15) { $m$ };
            \node [above] (M_big) at (0, 0.1) { $M$ };
        \end{tikzpicture}
}
\answer{%
    \begin{align*}
        &\begin{cases}
            F_1 + F_2 - mg - Mg= 0, \\
            F_1 \cdot 0 - mg \cdot a - Mg \cdot \frac l2 + F_2 \cdot l = 0,
        \end{cases} \\
        F_2 &= \frac{mga + Mg\frac l2}l = \frac al \cdot mg + \frac{Mg}2 \approx 33{,}6\,\text{Н}, \\
        F_1 &= mg + Mg - F_2 = mg + Mg - \frac al \cdot mg - \frac{Mg}2 = \frac bl \cdot mg + \frac{Mg}2 \approx 16{,}4\,\text{Н}.
    \end{align*}
}
\solutionspace{80pt}

\tasknumber{10}%
\task{%
    Тонкий однородный лом длиной $3\,\text{м}$ и массой $10\,\text{кг}$ лежит на горизонтальной поверхности.
    \begin{itemize}
        \item Какую минимальную силу надо приложить к одному из его концов, чтобы оторвать его от этой поверхности?
        \item Какую минимальную работу надо совершить, чтобы поставить его на землю в вертикальное положение?
    \end{itemize}
    % Примите $g = 10\,\frac{\text{м}}{\text{с}^{2}}$.
}
\answer{%
    $F = \frac{mg}2 \approx 100\,\text{Н}, A = mg\frac l2 = 150\,\text{Дж}$
}
\solutionspace{120pt}

\tasknumber{11}%
\task{%
    Определите работу силы, которая обеспечит подъём тела массой $5\,\text{кг}$ на высоту $5\,\text{м}$ с постоянным ускорением $3\,\frac{\text{м}}{\text{c}^{2}}$.
    % Примите $g = 10\,\frac{\text{м}}{\text{с}^{2}}$.
}
\answer{%
    \begin{align*}
    &\text{Для подъёма:} A = Fh = (mg + ma) h = m(g+a)h, \\
    &\text{Для спуска:} A = -Fh = -(mg - ma) h = -m(g-a)h, \\
    &\text{В результате получаем:} 325\,\text{Дж}.
    \end{align*}
}
\solutionspace{60pt}

\tasknumber{12}%
\task{%
    Тело бросили вертикально вверх со скоростью $20\,\frac{\text{м}}{\text{c}}$.
    На какой высоте кинетическая энергия тела составит половину от потенциальной?
}
\answer{%
    \begin{align*}
    &0 + \frac{mv_0^2}2 = E_p + E_k, E_k = \frac 12 E_p \implies \\
    &\implies \frac{mv_0^2}2 = E_p + \frac 12 E_p = E_p\cbr{1 + \frac 12} = mgh\cbr{1 + \frac 12} \implies \\
    &\implies h = \frac{\frac{mv_0^2}2}{mg\cbr{1 + \frac 12}} = \frac{v_0^2}{2g} \cdot \frac 1{1 + \frac 12} \approx 13{,}3\,\text{м}.
    \end{align*}
}
\solutionspace{100pt}

\tasknumber{13}%
\task{%
    Плотность воздуха при нормальных условиях равна $1{,}3\,\frac{\text{кг}}{\text{м}^{3}}$.
    Чему равна плотность воздуха
    при температуре $50\celsius$ и давлении $50\,\text{кПа}$?
}
\answer{%
    \begin{align*}
    &\text{В общем случае:} PV = \frac m{\mu} RT \implies \rho = \frac mV = \frac m{\frac{\frac m{\mu} RT}P} = \frac{P\mu}{RT}, \\
    &\text{У нас 2 состояния:} \rho_1 = \frac{P_1\mu}{RT_1}, \rho_2 = \frac{P_2\mu}{RT_2} \implies \frac{\rho_2}{\rho_1} = \frac{\frac{P_2\mu}{RT_2}}{\frac{P_1\mu}{RT_1}} = \frac{P_2T_1}{P_1T_2} \implies \\
    &\implies \rho_2 = \rho_1 \cdot  \frac{P_2T_1}{P_1T_2} = 1{,}3\,\frac{\text{кг}}{\text{м}^{3}} \cdot \frac{50\,\text{кПа} \cdot 273\units{К}}{100\,\text{кПа} \cdot 323\units{К}} \approx 0{,}55\,\frac{\text{кг}}{\text{м}^{3}}.
    \end{align*}
}
\solutionspace{120pt}

\tasknumber{14}%
\task{%
    Небольшую цилиндрическую пробирку с воздухом погружают на некоторую глубину в глубокое пресное озеро,
    после чего воздух занимает в ней лишь пятую часть от общего объема.
    Определите глубину, на которую погрузили пробирку.
    Температуру считать постоянной $T = 281\,\text{К}$, давлением паров воды пренебречь,
    атмосферное давление принять равным $p_{\text{aтм}} = 100\,\text{кПа}$.
}
\answer{%
    \begin{align*}
    T\text{— const} &\implies P_1V_1 = \nu RT = P_2V_2.
    \\
    V_2 = \frac 15 V_1 &\implies P_1V_1 = P_2 \cdot \frac 15V_1 \implies P_2 = 5P_1 = 5p_{\text{aтм}}.
    \\
    P_2 = p_{\text{aтм}} + \rho_{\text{в}} g h \implies h = \frac{P_2 - p_{\text{aтм}}}{\rho_{\text{в}} g} &= \frac{5p_{\text{aтм}} - p_{\text{aтм}}}{\rho_{\text{в}} g} = \frac{4 \cdot p_{\text{aтм}}}{\rho_{\text{в}} g} =  \\
     &= \frac{4 \cdot 100\,\text{кПа}}{1000\,\frac{\text{кг}}{\text{м}^{3}} \cdot  10\,\frac{\text{м}}{\text{с}^{2}}} \approx 40\,\text{м}.
    \end{align*}
}
\solutionspace{120pt}

\tasknumber{15}%
\task{%
    Газу сообщили некоторое количество теплоты,
    при этом половину его он потратил на совершение работы,
    одновременно увеличив свою внутреннюю энергию на $1500\,\text{Дж}$.
    Определите работу, совершённую газом.
}
\answer{%
    \begin{align*}
    Q &= A' + \Delta U, A' = \frac 12 Q \implies Q \cdot \cbr{1 - \frac 12} = \Delta U \implies Q = \frac{\Delta U}{1 - \frac 12} = \frac{ 1500\,\text{Дж} }{1 - \frac 12} \approx 3000\,\text{Дж}.
    \\
    A' &= \frac 12 Q
        = \frac 12 \cdot \frac{\Delta U}{1 - \frac 12}
        = \frac{\Delta U}{2 - 1}
        = \frac{ 1500\,\text{Дж} }{2 - 1} \approx 1500\,\text{Дж}.
    \end{align*}
}
\solutionspace{60pt}

\tasknumber{16}%
\task{%
    Два конденсатора ёмкостей $C_1 = 60\,\text{нФ}$ и $C_2 = 30\,\text{нФ}$ последовательно подключают
    к источнику напряжения $V = 450\,\text{В}$ (см.
    рис.).
    % Определите заряды каждого из конденсаторов.
    Определите заряд второго конденсатора.

    \begin{tikzpicture}[circuit ee IEC, semithick]
        \draw  (0, 0) to [capacitor={info={$C_1$}}] (1, 0)
                       to [capacitor={info={$C_2$}}] (2, 0)
        ;
        % \draw [-o] (0, 0) -- ++(-0.5, 0) node[left] {$-$};
        % \draw [-o] (2, 0) -- ++(0.5, 0) node[right] {$+$};
        \draw [-o] (0, 0) -- ++(-0.5, 0) node[left] {};
        \draw [-o] (2, 0) -- ++(0.5, 0) node[right] {};
    \end{tikzpicture}
}
\answer{%
    $
        Q_1
            = Q_2
            = CV
            = \frac{ V }{\frac1{C_1} + \frac1{C_2}}
            = \frac{C_1C_2V}{C_1 + C_2}
            = \frac{
                60\,\text{нФ} \cdot 30\,\text{нФ} \cdot 450\,\text{В}
            }{
                60\,\text{нФ} + 30\,\text{нФ}
            }
            = 9{,}00\,\text{мкКл}
    $
}
\solutionspace{120pt}

\tasknumber{17}%
\task{%
    В вакууме вдоль одной прямой расположены четыре отрицательных заряда так,
    что расстояние между соседними зарядами равно $l$.
    Сделайте рисунок,
    и определите силу, действующую на крайний заряд.
    Модули всех зарядов равны $q$ ($q > 0$).
}
\answer{%
    $F = \sum_i F_i = \ldots = \frac{49}{36} \frac{kq^2}{l^2}.$
}
\solutionspace{80pt}

\tasknumber{18}%
\task{%
    Юлия проводит эксперименты c 2 кусками одинаковой алюминиевой проволки, причём второй кусок в девять раз длиннее первого.
    В одном из экспериментов Юлия подаёт на первый кусок проволки напряжение в восемь раз раз больше, чем на второй.
    Определите отношения в двух проволках в этом эксперименте (второй к первой):
    \begin{itemize}
        \item отношение сил тока,
        \item отношение выделяющихся мощностей.
    \end{itemize}
}
\answer{%
    $R_2 = 9R_1, U_1 = 8U_2 \implies  \eli_2 / \eli_1 = \frac{U_2 / R_2}{U_1 / R_1} = \frac{U_2}{U_1} \cdot \frac{R_1}{R_2} = \frac1{72}, P_2 / P_1 = \frac{U_2^2 / R_2}{U_1^2 / R_1} = \sqr{\frac{U_2}{U_1}} \cdot \frac{R_1}{R_2} = \frac1{576}.$
}

\variantsplitter

\addpersonalvariant{Софья Белянкина}

\tasknumber{1}%
\task{%
    Саша стартует на мотоцикле и в течение $t = 2\,\text{c}$ двигается с постоянным ускорением $2\,\frac{\text{м}}{\text{с}^{2}}$.
    Определите
    \begin{itemize}
        \item какую скорость при этом удастся достичь,
        \item какой путь за это время будет пройден,
        \item среднюю скорость за всё время движения, если после начального ускорения продолжить движение равномерно ещё в течение времени $2t$
    \end{itemize}
}
\answer{%
    \begin{align*}
    v &= v_0 + a t = at = 2\,\frac{\text{м}}{\text{с}^{2}} \cdot 2\,\text{c} = 4{,}0\,\frac{\text{м}}{\text{с}}, \\
    s_x &= v_0t + \frac{a t^2}2 = \frac{a t^2}2 = \frac{2\,\frac{\text{м}}{\text{с}^{2}} \cdot \sqr{ 2\,\text{c} }}2 = 4{,}0\,\text{м}, \\
    v_\text{сред.} &= \frac{s_\text{общ}}{t_\text{общ.}} = \frac{s_x + v \cdot 2t}{t + 2t} = \frac{\frac{a t^2}2 + at \cdot 2t}{t (1 + 2)} = \\
    &= at \cdot \frac{\frac 12 + 2}{1 + 2} = 2\,\frac{\text{м}}{\text{с}^{2}} \cdot 2\,\text{c} \cdot \frac{\frac 12 + 2}{1 + 2} \approx 3{,}33\,\frac{\text{м}}{\text{c}}.
    \end{align*}
}
\solutionspace{120pt}

\tasknumber{2}%
\task{%
    Какой путь тело пройдёт за третью секунду после начала свободного падения?
    Какую скорость в начале этой секунды оно имеет?
}
\answer{%
    \begin{align*}
    s &= -s_y = -(y_2-y_1) = y_1 - y_2 = \cbr{y_{0y} + v_{0y}t_1 - \frac{gt_1^2}2} - \cbr{y_{0y} + v_{0y}t_2 - \frac{gt_2^2}2} = \\
    &= \frac{gt_2^2}2 - \frac{gt_1^2}2 = \frac g2\cbr{t_2^2 - t_1^2} = 25{,}0\,\text{м}, \\
    v_y &= v_{0y} - gt = -gt = 10\,\frac{\text{м}}{\text{с}^{2}} \cdot 2\,\text{с} = -20\,\frac{\text{м}}{\text{с}}.
    \end{align*}
}
\solutionspace{120pt}

\tasknumber{3}%
\task{%
    Карусель диаметром $4\,\text{м}$ равномерно совершает 10 оборотов в минуту.
    Определите
    \begin{itemize}
        \item период и частоту её обращения,
        \item скорость и ускорение крайних её точек.
    \end{itemize}
}
\answer{%
    \begin{align*}
    t &= 60\,\text{с}, r = 2{,}0\,\text{м}, n = 10\units{оборотов}, \\
    T &= \frac tN = \frac{ 60\,\text{с} }{10} \approx 6{,}00\,\text{c}, \\
    \nu &= \frac 1T = \frac{10}{ 60\,\text{с} } \approx 0{,}17\,\text{Гц}, \\
    v &= \frac{2 \pi r}{T} = \frac{2 \pi r}{T} =  \frac{2 \pi r n}{t} \approx 2{,}09\,\frac{\text{м}}{\text{c}}, \\
    a &= \frac{v^2}{r} =  \frac{4 \pi^2 r n^2}{t^2} \approx 2{,}19\,\frac{\text{м}}{\text{с}^{2}}.
    \end{align*}
}
\solutionspace{80pt}

\tasknumber{4}%
\task{%
    Миша стоит на обрыве над рекой и методично и строго горизонтально кидает в неё камушки.
    За этим всем наблюдает экспериментатор Глюк, который уже выяснил, что камушки падают в реку спустя $1{,}5\,\text{с}$ после броска,
    а вот дальность полёта оценить сложнее: придётся лезть в воду.
    Выручите Глюка и определите:
    \begin{itemize}
        \item высоту обрыва (вместе с ростом Миши).
        \item дальность полёта камушков (по горизонтали) и их скорость при падении, приняв начальную скорость броска равной $v_0 = 15\,\frac{\text{м}}{\text{с}}$.
    \end{itemize}
    Сопротивлением воздуха пренебречь.
}
\answer{%
    \begin{align*}
    y &= y_0 + v_{0y}t - \frac{gt^2}2 = h - \frac{gt^2}2, \qquad y(\tau) = 0 \implies h - \frac{g\tau^2}2 = 0 \implies h = \frac{g\tau^2}2 \approx 11{,}2\,\text{м}.
    \\
    x &= x_0 + v_{0x}t = v_0t \implies L = v_0\tau \approx 22{,}5\,\text{м}.
    \\
    &v = \sqrt{v_x^2 + v_y^2} = \sqrt{v_{0x}^2 + \sqr{v_{0y} - g\tau}} = \sqrt{v_0^2 + \sqr{g\tau}} \approx 21{,}2\,\frac{\text{м}}{\text{c}}.
    \end{align*}
}
\solutionspace{120pt}

\tasknumber{5}%
\task{%
    Пять одинаковых брусков массой $3\,\text{кг}$ каждый лежат на гладком горизонтальном столе.
    Бруски пронумерованы от 1 до 5 и последовательно связаны между собой
    невесомыми нерастяжимыми нитями: 1 со 2, 2 с 3 (ну и с 1) и т.д.
    Экспериментатор Глюк прикладывает постоянную горизонтальную силу $90\,\text{Н}$ к бруску с наибольшим номером.
    С каким ускорением двигается система? Чему равна сила натяжения нити, связывающей бруски 2 и 3?
}
\answer{%
    \begin{align*}
    a &= \frac{F}{5 m} = \frac{90\,\text{Н}}{5 \cdot 3\,\text{кг}} \approx 6{,}0\,\frac{\text{м}}{\text{c}^{2}}, \\
    T &= m'a = 2m \cdot \frac{F}{5 m} = \frac{2}{5} F \approx 36{,}0\,\text{Н}.
    \end{align*}
}
\solutionspace{120pt}

\tasknumber{6}%
\task{%
    Два бруска связаны лёгкой нерастяжимой нитью и перекинуты через неподвижный блок (см.
    рис.).
    Определите силу натяжения нити и ускорения брусков.
    Силами трения пренебречь, массы брусков
    равны $m_1 = 8\,\text{кг}$ и $m_2 = 14\,\text{кг}$.
    % $g = 10\,\frac{\text{м}}{\text{с}^{2}}$.

    \begin{tikzpicture}[x=1.5cm,y=1.5cm,thick]
        \draw
            (-0.4, 0) rectangle (-0.2, 1.2)
            (0.15, 0.5) rectangle (0.45, 1)
            (0, 2) circle [radius=0.3] -- ++(up:0.5)
            (-0.3, 1.2) -- ++(up:0.8)
            (0.3, 1) -- ++(up:1)
            (-0.7, 2.5) -- (0.7, 2.5)
            ;
        \draw[pattern={Lines[angle=51,distance=3pt]},pattern color=black,draw=none] (-0.7, 2.5) rectangle (0.7, 2.75);
        \node [left] (left) at (-0.4, 0.6) { $m_1$ };
        \node [right] (right) at (0.4, 0.75) { $m_2$ };
    \end{tikzpicture}
}
\answer{%
    Предположим, что левый брусок ускоряется вверх, тогда правый ускоряется вниз (с тем же ускорением).
    Запишем 2-й закон Ньютона 2 раза (для обоих тел) в проекции на вертикальную оси, направив её вверх.
    \begin{align*}
        &\begin{cases}
            T - m_1g = m_1a, \\
            T - m_2g = -m_2a,
        \end{cases} \\
        &\begin{cases}
            m_2g - m_1g = m_1a + m_2a, \\
            T = m_1a + m_1g, \\
        \end{cases} \\
        a &= \frac{m_2 - m_1}{m_1 + m_2} \cdot g = \frac{14\,\text{кг} - 8\,\text{кг}}{8\,\text{кг} + 14\,\text{кг}} \cdot 10\,\frac{\text{м}}{\text{с}^{2}} \approx 2{,}73\,\frac{\text{м}}{\text{c}^{2}}, \\
        T &= m_1(a + g) = m_1 \cdot g \cdot \cbr{\frac{m_2 - m_1}{m_1 + m_2} + 1} = m_1 \cdot g \cdot \frac{2m_2}{m_1 + m_2} = \\
            &= \frac{2 m_2 m_1 g}{m_1 + m_2} = \frac{2 \cdot 14\,\text{кг} \cdot 8\,\text{кг} \cdot 10\,\frac{\text{м}}{\text{с}^{2}}}{8\,\text{кг} + 14\,\text{кг}} \approx 101{,}8\,\text{Н}.
    \end{align*}
    Отрицательный ответ говорит, что мы лишь не угадали с направлением ускорений.
    Сила же всегда положительна.
}
\solutionspace{80pt}

\tasknumber{7}%
\task{%
    Тело массой $2{,}7\,\text{кг}$ лежит на горизонтальной поверхности.
    Коэффициент трения между поверхностью и телом $0{,}15$.
    К телу приложена горизонтальная сила $5{,}5\,\text{Н}$.
    Определите силу трения, действующую на тело, и ускорение тела.
    % $g = 10\,\frac{\text{м}}{\text{с}^{2}}$.
}
\answer{%
    \begin{align*}
    &F_\text{ трения покоя $\max$ } = \mu N = \mu m g = 0{,}15 \cdot 2{,}7\,\text{кг} \cdot 10\,\frac{\text{м}}{\text{с}^{2}} = 4{,}05\,\text{Н}, \\
    &F_\text{ трения покоя $\max$ } \le F \implies F_\text{ трения } = 4{,}05\,\text{Н}, a = \frac{F - F_\text{ трения }}{ m } = 0{,}54\,\frac{\text{м}}{\text{c}^{2}}, \\
    &\text{при равенстве возможны оба варианта: и едет, и не едет, но на ответы это не влияет.}
    \end{align*}
}
\solutionspace{120pt}

\tasknumber{8}%
\task{%
    Определите плотность неизвестного вещества, если известно, что опускании тела из него
    в подсолнечное масло оно будет плавать и на четверть выступать над поверхностью жидкости.
}
\answer{%
    $F_\text{Арх.} = F_\text{тяж.} \implies \rho_\text{ж.} g V_\text{погр.} = m g \implies\rho_\text{ж.} g \cbr{V -\frac V4} = \rho V g \implies \rho = \rho_\text{ж.}\cbr{1 -\frac 14} \approx 675\,\frac{\text{кг}}{\text{м}^{3}}$
}
\solutionspace{120pt}

\tasknumber{9}%
\task{%
    	Определите силу, действующую на левую опору однородного горизонтального стержня длиной $l = 7\,\text{м}$
    	и массой $M = 5\,\text{кг}$, к которому подвешен груз массой $m = 2\,\text{кг}$ на расстоянии $4\,\text{м}$ от правого конца (см.
    рис.).

        \begin{tikzpicture}[thick]
            \draw
                (-2, -0.1) rectangle (2, 0.1)
                (-0.5, -0.1) -- (-0.5, -1)
                (-0.7, -1) rectangle (-0.3, -1.3)
           		(-2, -0.1) -- +(0.15,-0.9) -- +(-0.15,-0.9) -- cycle
            	(2, -0.1) -- +(0.15,-0.9) -- +(-0.15,-0.9) -- cycle
            ;
            \draw[pattern={Lines[angle=51,distance=2pt]},pattern color=black,draw=none]
            	(-2.15, -1.15) rectangle +(0.3, 0.15)
            	(2.15, -1.15) rectangle +(-0.3, 0.15)
            ;
            \node [right] (m_small) at (-0.3, -1.15) { $m$ };
            \node [above] (M_big) at (0, 0.1) { $M$ };
        \end{tikzpicture}
}
\answer{%
    \begin{align*}
        &\begin{cases}
            F_1 + F_2 - mg - Mg= 0, \\
            F_1 \cdot 0 - mg \cdot a - Mg \cdot \frac l2 + F_2 \cdot l = 0,
        \end{cases} \\
        F_2 &= \frac{mga + Mg\frac l2}l = \frac al \cdot mg + \frac{Mg}2 \approx 33{,}6\,\text{Н}, \\
        F_1 &= mg + Mg - F_2 = mg + Mg - \frac al \cdot mg - \frac{Mg}2 = \frac bl \cdot mg + \frac{Mg}2 \approx 36{,}4\,\text{Н}.
    \end{align*}
}
\solutionspace{80pt}

\tasknumber{10}%
\task{%
    Тонкий однородный кусок арматуры длиной $3\,\text{м}$ и массой $30\,\text{кг}$ лежит на горизонтальной поверхности.
    \begin{itemize}
        \item Какую минимальную силу надо приложить к одному из его концов, чтобы оторвать его от этой поверхности?
        \item Какую минимальную работу надо совершить, чтобы поставить его на землю в вертикальное положение?
    \end{itemize}
    % Примите $g = 10\,\frac{\text{м}}{\text{с}^{2}}$.
}
\answer{%
    $F = \frac{mg}2 \approx 300\,\text{Н}, A = mg\frac l2 = 450\,\text{Дж}$
}
\solutionspace{120pt}

\tasknumber{11}%
\task{%
    Определите работу силы, которая обеспечит спуск тела массой $2\,\text{кг}$ на высоту $5\,\text{м}$ с постоянным ускорением $4\,\frac{\text{м}}{\text{c}^{2}}$.
    % Примите $g = 10\,\frac{\text{м}}{\text{с}^{2}}$.
}
\answer{%
    \begin{align*}
    &\text{Для подъёма:} A = Fh = (mg + ma) h = m(g+a)h, \\
    &\text{Для спуска:} A = -Fh = -(mg - ma) h = -m(g-a)h, \\
    &\text{В результате получаем:} -60\,\text{Дж}.
    \end{align*}
}
\solutionspace{60pt}

\tasknumber{12}%
\task{%
    Тело бросили вертикально вверх со скоростью $10\,\frac{\text{м}}{\text{c}}$.
    На какой высоте кинетическая энергия тела составит треть от потенциальной?
}
\answer{%
    \begin{align*}
    &0 + \frac{mv_0^2}2 = E_p + E_k, E_k = \frac 13 E_p \implies \\
    &\implies \frac{mv_0^2}2 = E_p + \frac 13 E_p = E_p\cbr{1 + \frac 13} = mgh\cbr{1 + \frac 13} \implies \\
    &\implies h = \frac{\frac{mv_0^2}2}{mg\cbr{1 + \frac 13}} = \frac{v_0^2}{2g} \cdot \frac 1{1 + \frac 13} \approx 3{,}8\,\text{м}.
    \end{align*}
}
\solutionspace{100pt}

\tasknumber{13}%
\task{%
    Плотность воздуха при нормальных условиях равна $1{,}3\,\frac{\text{кг}}{\text{м}^{3}}$.
    Чему равна плотность воздуха
    при температуре $50\celsius$ и давлении $120\,\text{кПа}$?
}
\answer{%
    \begin{align*}
    &\text{В общем случае:} PV = \frac m{\mu} RT \implies \rho = \frac mV = \frac m{\frac{\frac m{\mu} RT}P} = \frac{P\mu}{RT}, \\
    &\text{У нас 2 состояния:} \rho_1 = \frac{P_1\mu}{RT_1}, \rho_2 = \frac{P_2\mu}{RT_2} \implies \frac{\rho_2}{\rho_1} = \frac{\frac{P_2\mu}{RT_2}}{\frac{P_1\mu}{RT_1}} = \frac{P_2T_1}{P_1T_2} \implies \\
    &\implies \rho_2 = \rho_1 \cdot  \frac{P_2T_1}{P_1T_2} = 1{,}3\,\frac{\text{кг}}{\text{м}^{3}} \cdot \frac{120\,\text{кПа} \cdot 273\units{К}}{100\,\text{кПа} \cdot 323\units{К}} \approx 1{,}32\,\frac{\text{кг}}{\text{м}^{3}}.
    \end{align*}
}
\solutionspace{120pt}

\tasknumber{14}%
\task{%
    Небольшую цилиндрическую пробирку с воздухом погружают на некоторую глубину в глубокое пресное озеро,
    после чего воздух занимает в ней лишь третью часть от общего объема.
    Определите глубину, на которую погрузили пробирку.
    Температуру считать постоянной $T = 286\,\text{К}$, давлением паров воды пренебречь,
    атмосферное давление принять равным $p_{\text{aтм}} = 100\,\text{кПа}$.
}
\answer{%
    \begin{align*}
    T\text{— const} &\implies P_1V_1 = \nu RT = P_2V_2.
    \\
    V_2 = \frac 13 V_1 &\implies P_1V_1 = P_2 \cdot \frac 13V_1 \implies P_2 = 3P_1 = 3p_{\text{aтм}}.
    \\
    P_2 = p_{\text{aтм}} + \rho_{\text{в}} g h \implies h = \frac{P_2 - p_{\text{aтм}}}{\rho_{\text{в}} g} &= \frac{3p_{\text{aтм}} - p_{\text{aтм}}}{\rho_{\text{в}} g} = \frac{2 \cdot p_{\text{aтм}}}{\rho_{\text{в}} g} =  \\
     &= \frac{2 \cdot 100\,\text{кПа}}{1000\,\frac{\text{кг}}{\text{м}^{3}} \cdot  10\,\frac{\text{м}}{\text{с}^{2}}} \approx 20\,\text{м}.
    \end{align*}
}
\solutionspace{120pt}

\tasknumber{15}%
\task{%
    Газу сообщили некоторое количество теплоты,
    при этом четверть его он потратил на совершение работы,
    одновременно увеличив свою внутреннюю энергию на $1500\,\text{Дж}$.
    Определите количество теплоты, сообщённое газу.
}
\answer{%
    \begin{align*}
    Q &= A' + \Delta U, A' = \frac 14 Q \implies Q \cdot \cbr{1 - \frac 14} = \Delta U \implies Q = \frac{\Delta U}{1 - \frac 14} = \frac{ 1500\,\text{Дж} }{1 - \frac 14} \approx 2000\,\text{Дж}.
    \\
    A' &= \frac 14 Q
        = \frac 14 \cdot \frac{\Delta U}{1 - \frac 14}
        = \frac{\Delta U}{4 - 1}
        = \frac{ 1500\,\text{Дж} }{4 - 1} \approx 500\,\text{Дж}.
    \end{align*}
}
\solutionspace{60pt}

\tasknumber{16}%
\task{%
    Два конденсатора ёмкостей $C_1 = 60\,\text{нФ}$ и $C_2 = 40\,\text{нФ}$ последовательно подключают
    к источнику напряжения $V = 450\,\text{В}$ (см.
    рис.).
    % Определите заряды каждого из конденсаторов.
    Определите заряд первого конденсатора.

    \begin{tikzpicture}[circuit ee IEC, semithick]
        \draw  (0, 0) to [capacitor={info={$C_1$}}] (1, 0)
                       to [capacitor={info={$C_2$}}] (2, 0)
        ;
        % \draw [-o] (0, 0) -- ++(-0.5, 0) node[left] {$-$};
        % \draw [-o] (2, 0) -- ++(0.5, 0) node[right] {$+$};
        \draw [-o] (0, 0) -- ++(-0.5, 0) node[left] {};
        \draw [-o] (2, 0) -- ++(0.5, 0) node[right] {};
    \end{tikzpicture}
}
\answer{%
    $
        Q_1
            = Q_2
            = CV
            = \frac{ V }{\frac1{C_1} + \frac1{C_2}}
            = \frac{C_1C_2V}{C_1 + C_2}
            = \frac{
                60\,\text{нФ} \cdot 40\,\text{нФ} \cdot 450\,\text{В}
            }{
                60\,\text{нФ} + 40\,\text{нФ}
            }
            = 10{,}80\,\text{мкКл}
    $
}
\solutionspace{120pt}

\tasknumber{17}%
\task{%
    В вакууме вдоль одной прямой расположены три отрицательных заряда так,
    что расстояние между соседними зарядами равно $a$.
    Сделайте рисунок,
    и определите силу, действующую на крайний заряд.
    Модули всех зарядов равны $q$ ($q > 0$).
}
\answer{%
    $F = \sum_i F_i = \ldots = \frac54 \frac{kq^2}{a^2}.$
}
\solutionspace{80pt}

\tasknumber{18}%
\task{%
    Юлия проводит эксперименты c 2 кусками одинаковой стальной проволки, причём второй кусок в семь раз длиннее первого.
    В одном из экспериментов Юлия подаёт на первый кусок проволки напряжение в пять раз раз больше, чем на второй.
    Определите отношения в двух проволках в этом эксперименте (второй к первой):
    \begin{itemize}
        \item отношение сил тока,
        \item отношение выделяющихся мощностей.
    \end{itemize}
}
\answer{%
    $R_2 = 7R_1, U_1 = 5U_2 \implies  \eli_2 / \eli_1 = \frac{U_2 / R_2}{U_1 / R_1} = \frac{U_2}{U_1} \cdot \frac{R_1}{R_2} = \frac1{35}, P_2 / P_1 = \frac{U_2^2 / R_2}{U_1^2 / R_1} = \sqr{\frac{U_2}{U_1}} \cdot \frac{R_1}{R_2} = \frac1{175}.$
}

\variantsplitter

\addpersonalvariant{Варвара Егиазарян}

\tasknumber{1}%
\task{%
    Саша стартует на мотоцикле и в течение $t = 10\,\text{c}$ двигается с постоянным ускорением $1{,}5\,\frac{\text{м}}{\text{с}^{2}}$.
    Определите
    \begin{itemize}
        \item какую скорость при этом удастся достичь,
        \item какой путь за это время будет пройден,
        \item среднюю скорость за всё время движения, если после начального ускорения продолжить движение равномерно ещё в течение времени $3t$
    \end{itemize}
}
\answer{%
    \begin{align*}
    v &= v_0 + a t = at = 1{,}5\,\frac{\text{м}}{\text{с}^{2}} \cdot 10\,\text{c} = 15{,}0\,\frac{\text{м}}{\text{с}}, \\
    s_x &= v_0t + \frac{a t^2}2 = \frac{a t^2}2 = \frac{1{,}5\,\frac{\text{м}}{\text{с}^{2}} \cdot \sqr{ 10\,\text{c} }}2 = 75{,}0\,\text{м}, \\
    v_\text{сред.} &= \frac{s_\text{общ}}{t_\text{общ.}} = \frac{s_x + v \cdot 3t}{t + 3t} = \frac{\frac{a t^2}2 + at \cdot 3t}{t (1 + 3)} = \\
    &= at \cdot \frac{\frac 12 + 3}{1 + 3} = 1{,}5\,\frac{\text{м}}{\text{с}^{2}} \cdot 10\,\text{c} \cdot \frac{\frac 12 + 3}{1 + 3} \approx 13{,}12\,\frac{\text{м}}{\text{c}}.
    \end{align*}
}
\solutionspace{120pt}

\tasknumber{2}%
\task{%
    Какой путь тело пройдёт за пятую секунду после начала свободного падения?
    Какую скорость в конце этой секунды оно имеет?
}
\answer{%
    \begin{align*}
    s &= -s_y = -(y_2-y_1) = y_1 - y_2 = \cbr{y_{0y} + v_{0y}t_1 - \frac{gt_1^2}2} - \cbr{y_{0y} + v_{0y}t_2 - \frac{gt_2^2}2} = \\
    &= \frac{gt_2^2}2 - \frac{gt_1^2}2 = \frac g2\cbr{t_2^2 - t_1^2} = 45{,}0\,\text{м}, \\
    v_y &= v_{0y} - gt = -gt = 10\,\frac{\text{м}}{\text{с}^{2}} \cdot 5\,\text{с} = -50\,\frac{\text{м}}{\text{с}}.
    \end{align*}
}
\solutionspace{120pt}

\tasknumber{3}%
\task{%
    Карусель диаметром $5\,\text{м}$ равномерно совершает 10 оборотов в минуту.
    Определите
    \begin{itemize}
        \item период и частоту её обращения,
        \item скорость и ускорение крайних её точек.
    \end{itemize}
}
\answer{%
    \begin{align*}
    t &= 60\,\text{с}, r = 2{,}5\,\text{м}, n = 10\units{оборотов}, \\
    T &= \frac tN = \frac{ 60\,\text{с} }{10} \approx 6{,}00\,\text{c}, \\
    \nu &= \frac 1T = \frac{10}{ 60\,\text{с} } \approx 0{,}17\,\text{Гц}, \\
    v &= \frac{2 \pi r}{T} = \frac{2 \pi r}{T} =  \frac{2 \pi r n}{t} \approx 2{,}62\,\frac{\text{м}}{\text{c}}, \\
    a &= \frac{v^2}{r} =  \frac{4 \pi^2 r n^2}{t^2} \approx 2{,}74\,\frac{\text{м}}{\text{с}^{2}}.
    \end{align*}
}
\solutionspace{80pt}

\tasknumber{4}%
\task{%
    Маша стоит на обрыве над рекой и методично и строго горизонтально кидает в неё камушки.
    За этим всем наблюдает экспериментатор Глюк, который уже выяснил, что камушки падают в реку спустя $1{,}2\,\text{с}$ после броска,
    а вот дальность полёта оценить сложнее: придётся лезть в воду.
    Выручите Глюка и определите:
    \begin{itemize}
        \item высоту обрыва (вместе с ростом Маши).
        \item дальность полёта камушков (по горизонтали) и их скорость при падении, приняв начальную скорость броска равной $v_0 = 14\,\frac{\text{м}}{\text{с}}$.
    \end{itemize}
    Сопротивлением воздуха пренебречь.
}
\answer{%
    \begin{align*}
    y &= y_0 + v_{0y}t - \frac{gt^2}2 = h - \frac{gt^2}2, \qquad y(\tau) = 0 \implies h - \frac{g\tau^2}2 = 0 \implies h = \frac{g\tau^2}2 \approx 7{,}2\,\text{м}.
    \\
    x &= x_0 + v_{0x}t = v_0t \implies L = v_0\tau \approx 16{,}8\,\text{м}.
    \\
    &v = \sqrt{v_x^2 + v_y^2} = \sqrt{v_{0x}^2 + \sqr{v_{0y} - g\tau}} = \sqrt{v_0^2 + \sqr{g\tau}} \approx 18{,}4\,\frac{\text{м}}{\text{c}}.
    \end{align*}
}
\solutionspace{120pt}

\tasknumber{5}%
\task{%
    Шесть одинаковых брусков массой $2\,\text{кг}$ каждый лежат на гладком горизонтальном столе.
    Бруски пронумерованы от 1 до 6 и последовательно связаны между собой
    невесомыми нерастяжимыми нитями: 1 со 2, 2 с 3 (ну и с 1) и т.д.
    Экспериментатор Глюк прикладывает постоянную горизонтальную силу $60\,\text{Н}$ к бруску с наименьшим номером.
    С каким ускорением двигается система? Чему равна сила натяжения нити, связывающей бруски 2 и 3?
}
\answer{%
    \begin{align*}
    a &= \frac{F}{6 m} = \frac{60\,\text{Н}}{6 \cdot 2\,\text{кг}} \approx 5{,}0\,\frac{\text{м}}{\text{c}^{2}}, \\
    T &= m'a = 4m \cdot \frac{F}{6 m} = \frac{4}{6} F \approx 40{,}0\,\text{Н}.
    \end{align*}
}
\solutionspace{120pt}

\tasknumber{6}%
\task{%
    Два бруска связаны лёгкой нерастяжимой нитью и перекинуты через неподвижный блок (см.
    рис.).
    Определите силу натяжения нити и ускорения брусков.
    Силами трения пренебречь, массы брусков
    равны $m_1 = 8\,\text{кг}$ и $m_2 = 6\,\text{кг}$.
    % $g = 10\,\frac{\text{м}}{\text{с}^{2}}$.

    \begin{tikzpicture}[x=1.5cm,y=1.5cm,thick]
        \draw
            (-0.4, 0) rectangle (-0.2, 1.2)
            (0.15, 0.5) rectangle (0.45, 1)
            (0, 2) circle [radius=0.3] -- ++(up:0.5)
            (-0.3, 1.2) -- ++(up:0.8)
            (0.3, 1) -- ++(up:1)
            (-0.7, 2.5) -- (0.7, 2.5)
            ;
        \draw[pattern={Lines[angle=51,distance=3pt]},pattern color=black,draw=none] (-0.7, 2.5) rectangle (0.7, 2.75);
        \node [left] (left) at (-0.4, 0.6) { $m_1$ };
        \node [right] (right) at (0.4, 0.75) { $m_2$ };
    \end{tikzpicture}
}
\answer{%
    Предположим, что левый брусок ускоряется вверх, тогда правый ускоряется вниз (с тем же ускорением).
    Запишем 2-й закон Ньютона 2 раза (для обоих тел) в проекции на вертикальную оси, направив её вверх.
    \begin{align*}
        &\begin{cases}
            T - m_1g = m_1a, \\
            T - m_2g = -m_2a,
        \end{cases} \\
        &\begin{cases}
            m_2g - m_1g = m_1a + m_2a, \\
            T = m_1a + m_1g, \\
        \end{cases} \\
        a &= \frac{m_2 - m_1}{m_1 + m_2} \cdot g = \frac{6\,\text{кг} - 8\,\text{кг}}{8\,\text{кг} + 6\,\text{кг}} \cdot 10\,\frac{\text{м}}{\text{с}^{2}} \approx -1{,}4300\,\frac{\text{м}}{\text{c}^{2}}, \\
        T &= m_1(a + g) = m_1 \cdot g \cdot \cbr{\frac{m_2 - m_1}{m_1 + m_2} + 1} = m_1 \cdot g \cdot \frac{2m_2}{m_1 + m_2} = \\
            &= \frac{2 m_2 m_1 g}{m_1 + m_2} = \frac{2 \cdot 6\,\text{кг} \cdot 8\,\text{кг} \cdot 10\,\frac{\text{м}}{\text{с}^{2}}}{8\,\text{кг} + 6\,\text{кг}} \approx 68{,}6\,\text{Н}.
    \end{align*}
    Отрицательный ответ говорит, что мы лишь не угадали с направлением ускорений.
    Сила же всегда положительна.
}
\solutionspace{80pt}

\tasknumber{7}%
\task{%
    Тело массой $1{,}4\,\text{кг}$ лежит на горизонтальной поверхности.
    Коэффициент трения между поверхностью и телом $0{,}15$.
    К телу приложена горизонтальная сила $2{,}5\,\text{Н}$.
    Определите силу трения, действующую на тело, и ускорение тела.
    % $g = 10\,\frac{\text{м}}{\text{с}^{2}}$.
}
\answer{%
    \begin{align*}
    &F_\text{ трения покоя $\max$ } = \mu N = \mu m g = 0{,}15 \cdot 1{,}4\,\text{кг} \cdot 10\,\frac{\text{м}}{\text{с}^{2}} = 2{,}10\,\text{Н}, \\
    &F_\text{ трения покоя $\max$ } \le F \implies F_\text{ трения } = 2{,}10\,\text{Н}, a = \frac{F - F_\text{ трения }}{ m } = 0{,}29\,\frac{\text{м}}{\text{c}^{2}}, \\
    &\text{при равенстве возможны оба варианта: и едет, и не едет, но на ответы это не влияет.}
    \end{align*}
}
\solutionspace{120pt}

\tasknumber{8}%
\task{%
    Определите плотность неизвестного вещества, если известно, что опускании тела из него
    в керосин оно будет плавать и на половину выступать над поверхностью жидкости.
}
\answer{%
    $F_\text{Арх.} = F_\text{тяж.} \implies \rho_\text{ж.} g V_\text{погр.} = m g \implies\rho_\text{ж.} g \cbr{V -\frac V2} = \rho V g \implies \rho = \rho_\text{ж.}\cbr{1 -\frac 12} \approx 400\,\frac{\text{кг}}{\text{м}^{3}}$
}
\solutionspace{120pt}

\tasknumber{9}%
\task{%
    	Определите силу, действующую на правую опору однородного горизонтального стержня длиной $l = 3\,\text{м}$
    	и массой $M = 1\,\text{кг}$, к которому подвешен груз массой $m = 3\,\text{кг}$ на расстоянии $2\,\text{м}$ от правого конца (см.
    рис.).

        \begin{tikzpicture}[thick]
            \draw
                (-2, -0.1) rectangle (2, 0.1)
                (-0.5, -0.1) -- (-0.5, -1)
                (-0.7, -1) rectangle (-0.3, -1.3)
           		(-2, -0.1) -- +(0.15,-0.9) -- +(-0.15,-0.9) -- cycle
            	(2, -0.1) -- +(0.15,-0.9) -- +(-0.15,-0.9) -- cycle
            ;
            \draw[pattern={Lines[angle=51,distance=2pt]},pattern color=black,draw=none]
            	(-2.15, -1.15) rectangle +(0.3, 0.15)
            	(2.15, -1.15) rectangle +(-0.3, 0.15)
            ;
            \node [right] (m_small) at (-0.3, -1.15) { $m$ };
            \node [above] (M_big) at (0, 0.1) { $M$ };
        \end{tikzpicture}
}
\answer{%
    \begin{align*}
        &\begin{cases}
            F_1 + F_2 - mg - Mg= 0, \\
            F_1 \cdot 0 - mg \cdot a - Mg \cdot \frac l2 + F_2 \cdot l = 0,
        \end{cases} \\
        F_2 &= \frac{mga + Mg\frac l2}l = \frac al \cdot mg + \frac{Mg}2 \approx 15{,}0\,\text{Н}, \\
        F_1 &= mg + Mg - F_2 = mg + Mg - \frac al \cdot mg - \frac{Mg}2 = \frac bl \cdot mg + \frac{Mg}2 \approx 25{,}0\,\text{Н}.
    \end{align*}
}
\solutionspace{80pt}

\tasknumber{10}%
\task{%
    Тонкий однородный кусок арматуры длиной $2\,\text{м}$ и массой $10\,\text{кг}$ лежит на горизонтальной поверхности.
    \begin{itemize}
        \item Какую минимальную силу надо приложить к одному из его концов, чтобы оторвать его от этой поверхности?
        \item Какую минимальную работу надо совершить, чтобы поставить его на землю в вертикальное положение?
    \end{itemize}
    % Примите $g = 10\,\frac{\text{м}}{\text{с}^{2}}$.
}
\answer{%
    $F = \frac{mg}2 \approx 100\,\text{Н}, A = mg\frac l2 = 100\,\text{Дж}$
}
\solutionspace{120pt}

\tasknumber{11}%
\task{%
    Определите работу силы, которая обеспечит подъём тела массой $2\,\text{кг}$ на высоту $5\,\text{м}$ с постоянным ускорением $2\,\frac{\text{м}}{\text{c}^{2}}$.
    % Примите $g = 10\,\frac{\text{м}}{\text{с}^{2}}$.
}
\answer{%
    \begin{align*}
    &\text{Для подъёма:} A = Fh = (mg + ma) h = m(g+a)h, \\
    &\text{Для спуска:} A = -Fh = -(mg - ma) h = -m(g-a)h, \\
    &\text{В результате получаем:} 120\,\text{Дж}.
    \end{align*}
}
\solutionspace{60pt}

\tasknumber{12}%
\task{%
    Тело бросили вертикально вверх со скоростью $14\,\frac{\text{м}}{\text{c}}$.
    На какой высоте кинетическая энергия тела составит треть от потенциальной?
}
\answer{%
    \begin{align*}
    &0 + \frac{mv_0^2}2 = E_p + E_k, E_k = \frac 13 E_p \implies \\
    &\implies \frac{mv_0^2}2 = E_p + \frac 13 E_p = E_p\cbr{1 + \frac 13} = mgh\cbr{1 + \frac 13} \implies \\
    &\implies h = \frac{\frac{mv_0^2}2}{mg\cbr{1 + \frac 13}} = \frac{v_0^2}{2g} \cdot \frac 1{1 + \frac 13} \approx 7{,}4\,\text{м}.
    \end{align*}
}
\solutionspace{100pt}

\tasknumber{13}%
\task{%
    Плотность воздуха при нормальных условиях равна $1{,}3\,\frac{\text{кг}}{\text{м}^{3}}$.
    Чему равна плотность воздуха
    при температуре $100\celsius$ и давлении $50\,\text{кПа}$?
}
\answer{%
    \begin{align*}
    &\text{В общем случае:} PV = \frac m{\mu} RT \implies \rho = \frac mV = \frac m{\frac{\frac m{\mu} RT}P} = \frac{P\mu}{RT}, \\
    &\text{У нас 2 состояния:} \rho_1 = \frac{P_1\mu}{RT_1}, \rho_2 = \frac{P_2\mu}{RT_2} \implies \frac{\rho_2}{\rho_1} = \frac{\frac{P_2\mu}{RT_2}}{\frac{P_1\mu}{RT_1}} = \frac{P_2T_1}{P_1T_2} \implies \\
    &\implies \rho_2 = \rho_1 \cdot  \frac{P_2T_1}{P_1T_2} = 1{,}3\,\frac{\text{кг}}{\text{м}^{3}} \cdot \frac{50\,\text{кПа} \cdot 273\units{К}}{100\,\text{кПа} \cdot 373\units{К}} \approx 0{,}48\,\frac{\text{кг}}{\text{м}^{3}}.
    \end{align*}
}
\solutionspace{120pt}

\tasknumber{14}%
\task{%
    Небольшую цилиндрическую пробирку с воздухом погружают на некоторую глубину в глубокое пресное озеро,
    после чего воздух занимает в ней лишь третью часть от общего объема.
    Определите глубину, на которую погрузили пробирку.
    Температуру считать постоянной $T = 278\,\text{К}$, давлением паров воды пренебречь,
    атмосферное давление принять равным $p_{\text{aтм}} = 100\,\text{кПа}$.
}
\answer{%
    \begin{align*}
    T\text{— const} &\implies P_1V_1 = \nu RT = P_2V_2.
    \\
    V_2 = \frac 13 V_1 &\implies P_1V_1 = P_2 \cdot \frac 13V_1 \implies P_2 = 3P_1 = 3p_{\text{aтм}}.
    \\
    P_2 = p_{\text{aтм}} + \rho_{\text{в}} g h \implies h = \frac{P_2 - p_{\text{aтм}}}{\rho_{\text{в}} g} &= \frac{3p_{\text{aтм}} - p_{\text{aтм}}}{\rho_{\text{в}} g} = \frac{2 \cdot p_{\text{aтм}}}{\rho_{\text{в}} g} =  \\
     &= \frac{2 \cdot 100\,\text{кПа}}{1000\,\frac{\text{кг}}{\text{м}^{3}} \cdot  10\,\frac{\text{м}}{\text{с}^{2}}} \approx 20\,\text{м}.
    \end{align*}
}
\solutionspace{120pt}

\tasknumber{15}%
\task{%
    Газу сообщили некоторое количество теплоты,
    при этом половину его он потратил на совершение работы,
    одновременно увеличив свою внутреннюю энергию на $1500\,\text{Дж}$.
    Определите количество теплоты, сообщённое газу.
}
\answer{%
    \begin{align*}
    Q &= A' + \Delta U, A' = \frac 12 Q \implies Q \cdot \cbr{1 - \frac 12} = \Delta U \implies Q = \frac{\Delta U}{1 - \frac 12} = \frac{ 1500\,\text{Дж} }{1 - \frac 12} \approx 3000\,\text{Дж}.
    \\
    A' &= \frac 12 Q
        = \frac 12 \cdot \frac{\Delta U}{1 - \frac 12}
        = \frac{\Delta U}{2 - 1}
        = \frac{ 1500\,\text{Дж} }{2 - 1} \approx 1500\,\text{Дж}.
    \end{align*}
}
\solutionspace{60pt}

\tasknumber{16}%
\task{%
    Два конденсатора ёмкостей $C_1 = 30\,\text{нФ}$ и $C_2 = 60\,\text{нФ}$ последовательно подключают
    к источнику напряжения $V = 200\,\text{В}$ (см.
    рис.).
    % Определите заряды каждого из конденсаторов.
    Определите заряд второго конденсатора.

    \begin{tikzpicture}[circuit ee IEC, semithick]
        \draw  (0, 0) to [capacitor={info={$C_1$}}] (1, 0)
                       to [capacitor={info={$C_2$}}] (2, 0)
        ;
        % \draw [-o] (0, 0) -- ++(-0.5, 0) node[left] {$-$};
        % \draw [-o] (2, 0) -- ++(0.5, 0) node[right] {$+$};
        \draw [-o] (0, 0) -- ++(-0.5, 0) node[left] {};
        \draw [-o] (2, 0) -- ++(0.5, 0) node[right] {};
    \end{tikzpicture}
}
\answer{%
    $
        Q_1
            = Q_2
            = CV
            = \frac{ V }{\frac1{C_1} + \frac1{C_2}}
            = \frac{C_1C_2V}{C_1 + C_2}
            = \frac{
                30\,\text{нФ} \cdot 60\,\text{нФ} \cdot 200\,\text{В}
            }{
                30\,\text{нФ} + 60\,\text{нФ}
            }
            = 4{,}00\,\text{мкКл}
    $
}
\solutionspace{120pt}

\tasknumber{17}%
\task{%
    В вакууме вдоль одной прямой расположены четыре положительных заряда так,
    что расстояние между соседними зарядами равно $a$.
    Сделайте рисунок,
    и определите силу, действующую на крайний заряд.
    Модули всех зарядов равны $q$ ($q > 0$).
}
\answer{%
    $F = \sum_i F_i = \ldots = \frac{49}{36} \frac{kq^2}{a^2}.$
}
\solutionspace{80pt}

\tasknumber{18}%
\task{%
    Юлия проводит эксперименты c 2 кусками одинаковой стальной проволки, причём второй кусок в восемь раз длиннее первого.
    В одном из экспериментов Юлия подаёт на первый кусок проволки напряжение в восемь раз раз больше, чем на второй.
    Определите отношения в двух проволках в этом эксперименте (второй к первой):
    \begin{itemize}
        \item отношение сил тока,
        \item отношение выделяющихся мощностей.
    \end{itemize}
}
\answer{%
    $R_2 = 8R_1, U_1 = 8U_2 \implies  \eli_2 / \eli_1 = \frac{U_2 / R_2}{U_1 / R_1} = \frac{U_2}{U_1} \cdot \frac{R_1}{R_2} = \frac1{64}, P_2 / P_1 = \frac{U_2^2 / R_2}{U_1^2 / R_1} = \sqr{\frac{U_2}{U_1}} \cdot \frac{R_1}{R_2} = \frac1{512}.$
}

\variantsplitter

\addpersonalvariant{Владислав Емелин}

\tasknumber{1}%
\task{%
    Валя стартует на велосипеде и в течение $t = 4\,\text{c}$ двигается с постоянным ускорением $2\,\frac{\text{м}}{\text{с}^{2}}$.
    Определите
    \begin{itemize}
        \item какую скорость при этом удастся достичь,
        \item какой путь за это время будет пройден,
        \item среднюю скорость за всё время движения, если после начального ускорения продолжить движение равномерно ещё в течение времени $2t$
    \end{itemize}
}
\answer{%
    \begin{align*}
    v &= v_0 + a t = at = 2\,\frac{\text{м}}{\text{с}^{2}} \cdot 4\,\text{c} = 8{,}0\,\frac{\text{м}}{\text{с}}, \\
    s_x &= v_0t + \frac{a t^2}2 = \frac{a t^2}2 = \frac{2\,\frac{\text{м}}{\text{с}^{2}} \cdot \sqr{ 4\,\text{c} }}2 = 16{,}0\,\text{м}, \\
    v_\text{сред.} &= \frac{s_\text{общ}}{t_\text{общ.}} = \frac{s_x + v \cdot 2t}{t + 2t} = \frac{\frac{a t^2}2 + at \cdot 2t}{t (1 + 2)} = \\
    &= at \cdot \frac{\frac 12 + 2}{1 + 2} = 2\,\frac{\text{м}}{\text{с}^{2}} \cdot 4\,\text{c} \cdot \frac{\frac 12 + 2}{1 + 2} \approx 6{,}67\,\frac{\text{м}}{\text{c}}.
    \end{align*}
}
\solutionspace{120pt}

\tasknumber{2}%
\task{%
    Какой путь тело пройдёт за четвёртую секунду после начала свободного падения?
    Какую скорость в начале этой секунды оно имеет?
}
\answer{%
    \begin{align*}
    s &= -s_y = -(y_2-y_1) = y_1 - y_2 = \cbr{y_{0y} + v_{0y}t_1 - \frac{gt_1^2}2} - \cbr{y_{0y} + v_{0y}t_2 - \frac{gt_2^2}2} = \\
    &= \frac{gt_2^2}2 - \frac{gt_1^2}2 = \frac g2\cbr{t_2^2 - t_1^2} = 35{,}0\,\text{м}, \\
    v_y &= v_{0y} - gt = -gt = 10\,\frac{\text{м}}{\text{с}^{2}} \cdot 3\,\text{с} = -30\,\frac{\text{м}}{\text{с}}.
    \end{align*}
}
\solutionspace{120pt}

\tasknumber{3}%
\task{%
    Карусель радиусом $2\,\text{м}$ равномерно совершает 6 оборотов в минуту.
    Определите
    \begin{itemize}
        \item период и частоту её обращения,
        \item скорость и ускорение крайних её точек.
    \end{itemize}
}
\answer{%
    \begin{align*}
    t &= 60\,\text{с}, r = 2{,}0\,\text{м}, n = 6\units{оборотов}, \\
    T &= \frac tN = \frac{ 60\,\text{с} }{6} \approx 10{,}00\,\text{c}, \\
    \nu &= \frac 1T = \frac{6}{ 60\,\text{с} } \approx 0{,}10\,\text{Гц}, \\
    v &= \frac{2 \pi r}{T} = \frac{2 \pi r}{T} =  \frac{2 \pi r n}{t} \approx 1{,}26\,\frac{\text{м}}{\text{c}}, \\
    a &= \frac{v^2}{r} =  \frac{4 \pi^2 r n^2}{t^2} \approx 0{,}79\,\frac{\text{м}}{\text{с}^{2}}.
    \end{align*}
}
\solutionspace{80pt}

\tasknumber{4}%
\task{%
    Паша стоит на обрыве над рекой и методично и строго горизонтально кидает в неё камушки.
    За этим всем наблюдает экспериментатор Глюк, который уже выяснил, что камушки падают в реку спустя $1{,}6\,\text{с}$ после броска,
    а вот дальность полёта оценить сложнее: придётся лезть в воду.
    Выручите Глюка и определите:
    \begin{itemize}
        \item высоту обрыва (вместе с ростом Паши).
        \item дальность полёта камушков (по горизонтали) и их скорость при падении, приняв начальную скорость броска равной $v_0 = 16\,\frac{\text{м}}{\text{с}}$.
    \end{itemize}
    Сопротивлением воздуха пренебречь.
}
\answer{%
    \begin{align*}
    y &= y_0 + v_{0y}t - \frac{gt^2}2 = h - \frac{gt^2}2, \qquad y(\tau) = 0 \implies h - \frac{g\tau^2}2 = 0 \implies h = \frac{g\tau^2}2 \approx 12{,}8\,\text{м}.
    \\
    x &= x_0 + v_{0x}t = v_0t \implies L = v_0\tau \approx 25{,}6\,\text{м}.
    \\
    &v = \sqrt{v_x^2 + v_y^2} = \sqrt{v_{0x}^2 + \sqr{v_{0y} - g\tau}} = \sqrt{v_0^2 + \sqr{g\tau}} \approx 22{,}6\,\frac{\text{м}}{\text{c}}.
    \end{align*}
}
\solutionspace{120pt}

\tasknumber{5}%
\task{%
    Четыре одинаковых брусков массой $2\,\text{кг}$ каждый лежат на гладком горизонтальном столе.
    Бруски пронумерованы от 1 до 4 и последовательно связаны между собой
    невесомыми нерастяжимыми нитями: 1 со 2, 2 с 3 (ну и с 1) и т.д.
    Экспериментатор Глюк прикладывает постоянную горизонтальную силу $90\,\text{Н}$ к бруску с наибольшим номером.
    С каким ускорением двигается система? Чему равна сила натяжения нити, связывающей бруски 3 и 4?
}
\answer{%
    \begin{align*}
    a &= \frac{F}{4 m} = \frac{90\,\text{Н}}{4 \cdot 2\,\text{кг}} \approx 11{,}2\,\frac{\text{м}}{\text{c}^{2}}, \\
    T &= m'a = 3m \cdot \frac{F}{4 m} = \frac{3}{4} F \approx 67{,}5\,\text{Н}.
    \end{align*}
}
\solutionspace{120pt}

\tasknumber{6}%
\task{%
    Два бруска связаны лёгкой нерастяжимой нитью и перекинуты через неподвижный блок (см.
    рис.).
    Определите силу натяжения нити и ускорения брусков.
    Силами трения пренебречь, массы брусков
    равны $m_1 = 8\,\text{кг}$ и $m_2 = 14\,\text{кг}$.
    % $g = 10\,\frac{\text{м}}{\text{с}^{2}}$.

    \begin{tikzpicture}[x=1.5cm,y=1.5cm,thick]
        \draw
            (-0.4, 0) rectangle (-0.2, 1.2)
            (0.15, 0.5) rectangle (0.45, 1)
            (0, 2) circle [radius=0.3] -- ++(up:0.5)
            (-0.3, 1.2) -- ++(up:0.8)
            (0.3, 1) -- ++(up:1)
            (-0.7, 2.5) -- (0.7, 2.5)
            ;
        \draw[pattern={Lines[angle=51,distance=3pt]},pattern color=black,draw=none] (-0.7, 2.5) rectangle (0.7, 2.75);
        \node [left] (left) at (-0.4, 0.6) { $m_1$ };
        \node [right] (right) at (0.4, 0.75) { $m_2$ };
    \end{tikzpicture}
}
\answer{%
    Предположим, что левый брусок ускоряется вверх, тогда правый ускоряется вниз (с тем же ускорением).
    Запишем 2-й закон Ньютона 2 раза (для обоих тел) в проекции на вертикальную оси, направив её вверх.
    \begin{align*}
        &\begin{cases}
            T - m_1g = m_1a, \\
            T - m_2g = -m_2a,
        \end{cases} \\
        &\begin{cases}
            m_2g - m_1g = m_1a + m_2a, \\
            T = m_1a + m_1g, \\
        \end{cases} \\
        a &= \frac{m_2 - m_1}{m_1 + m_2} \cdot g = \frac{14\,\text{кг} - 8\,\text{кг}}{8\,\text{кг} + 14\,\text{кг}} \cdot 10\,\frac{\text{м}}{\text{с}^{2}} \approx 2{,}73\,\frac{\text{м}}{\text{c}^{2}}, \\
        T &= m_1(a + g) = m_1 \cdot g \cdot \cbr{\frac{m_2 - m_1}{m_1 + m_2} + 1} = m_1 \cdot g \cdot \frac{2m_2}{m_1 + m_2} = \\
            &= \frac{2 m_2 m_1 g}{m_1 + m_2} = \frac{2 \cdot 14\,\text{кг} \cdot 8\,\text{кг} \cdot 10\,\frac{\text{м}}{\text{с}^{2}}}{8\,\text{кг} + 14\,\text{кг}} \approx 101{,}8\,\text{Н}.
    \end{align*}
    Отрицательный ответ говорит, что мы лишь не угадали с направлением ускорений.
    Сила же всегда положительна.
}
\solutionspace{80pt}

\tasknumber{7}%
\task{%
    Тело массой $2{,}7\,\text{кг}$ лежит на горизонтальной поверхности.
    Коэффициент трения между поверхностью и телом $0{,}25$.
    К телу приложена горизонтальная сила $3{,}5\,\text{Н}$.
    Определите силу трения, действующую на тело, и ускорение тела.
    % $g = 10\,\frac{\text{м}}{\text{с}^{2}}$.
}
\answer{%
    \begin{align*}
    &F_\text{ трения покоя $\max$ } = \mu N = \mu m g = 0{,}25 \cdot 2{,}7\,\text{кг} \cdot 10\,\frac{\text{м}}{\text{с}^{2}} = 6{,}75\,\text{Н}, \\
    &F_\text{ трения покоя $\max$ } > F \implies F_\text{ трения } = 3{,}50\,\text{Н}, a = \frac{F - F_\text{ трения }}{ m } = 0\,\frac{\text{м}}{\text{c}^{2}}, \\
    &\text{при равенстве возможны оба варианта: и едет, и не едет, но на ответы это не влияет.}
    \end{align*}
}
\solutionspace{120pt}

\tasknumber{8}%
\task{%
    Определите плотность неизвестного вещества, если известно, что опускании тела из него
    в подсолнечное масло оно будет плавать и на половину выступать над поверхностью жидкости.
}
\answer{%
    $F_\text{Арх.} = F_\text{тяж.} \implies \rho_\text{ж.} g V_\text{погр.} = m g \implies\rho_\text{ж.} g \cbr{V -\frac V2} = \rho V g \implies \rho = \rho_\text{ж.}\cbr{1 -\frac 12} \approx 450\,\frac{\text{кг}}{\text{м}^{3}}$
}
\solutionspace{120pt}

\tasknumber{9}%
\task{%
    	Определите силу, действующую на левую опору однородного горизонтального стержня длиной $l = 3\,\text{м}$
    	и массой $M = 1\,\text{кг}$, к которому подвешен груз массой $m = 4\,\text{кг}$ на расстоянии $2\,\text{м}$ от правого конца (см.
    рис.).

        \begin{tikzpicture}[thick]
            \draw
                (-2, -0.1) rectangle (2, 0.1)
                (-0.5, -0.1) -- (-0.5, -1)
                (-0.7, -1) rectangle (-0.3, -1.3)
           		(-2, -0.1) -- +(0.15,-0.9) -- +(-0.15,-0.9) -- cycle
            	(2, -0.1) -- +(0.15,-0.9) -- +(-0.15,-0.9) -- cycle
            ;
            \draw[pattern={Lines[angle=51,distance=2pt]},pattern color=black,draw=none]
            	(-2.15, -1.15) rectangle +(0.3, 0.15)
            	(2.15, -1.15) rectangle +(-0.3, 0.15)
            ;
            \node [right] (m_small) at (-0.3, -1.15) { $m$ };
            \node [above] (M_big) at (0, 0.1) { $M$ };
        \end{tikzpicture}
}
\answer{%
    \begin{align*}
        &\begin{cases}
            F_1 + F_2 - mg - Mg= 0, \\
            F_1 \cdot 0 - mg \cdot a - Mg \cdot \frac l2 + F_2 \cdot l = 0,
        \end{cases} \\
        F_2 &= \frac{mga + Mg\frac l2}l = \frac al \cdot mg + \frac{Mg}2 \approx 18{,}3\,\text{Н}, \\
        F_1 &= mg + Mg - F_2 = mg + Mg - \frac al \cdot mg - \frac{Mg}2 = \frac bl \cdot mg + \frac{Mg}2 \approx 31{,}7\,\text{Н}.
    \end{align*}
}
\solutionspace{80pt}

\tasknumber{10}%
\task{%
    Тонкий однородный шест длиной $1\,\text{м}$ и массой $10\,\text{кг}$ лежит на горизонтальной поверхности.
    \begin{itemize}
        \item Какую минимальную силу надо приложить к одному из его концов, чтобы оторвать его от этой поверхности?
        \item Какую минимальную работу надо совершить, чтобы поставить его на землю в вертикальное положение?
    \end{itemize}
    % Примите $g = 10\,\frac{\text{м}}{\text{с}^{2}}$.
}
\answer{%
    $F = \frac{mg}2 \approx 100\,\text{Н}, A = mg\frac l2 = 50\,\text{Дж}$
}
\solutionspace{120pt}

\tasknumber{11}%
\task{%
    Определите работу силы, которая обеспечит подъём тела массой $2\,\text{кг}$ на высоту $5\,\text{м}$ с постоянным ускорением $2\,\frac{\text{м}}{\text{c}^{2}}$.
    % Примите $g = 10\,\frac{\text{м}}{\text{с}^{2}}$.
}
\answer{%
    \begin{align*}
    &\text{Для подъёма:} A = Fh = (mg + ma) h = m(g+a)h, \\
    &\text{Для спуска:} A = -Fh = -(mg - ma) h = -m(g-a)h, \\
    &\text{В результате получаем:} 120\,\text{Дж}.
    \end{align*}
}
\solutionspace{60pt}

\tasknumber{12}%
\task{%
    Тело бросили вертикально вверх со скоростью $20\,\frac{\text{м}}{\text{c}}$.
    На какой высоте кинетическая энергия тела составит половину от потенциальной?
}
\answer{%
    \begin{align*}
    &0 + \frac{mv_0^2}2 = E_p + E_k, E_k = \frac 12 E_p \implies \\
    &\implies \frac{mv_0^2}2 = E_p + \frac 12 E_p = E_p\cbr{1 + \frac 12} = mgh\cbr{1 + \frac 12} \implies \\
    &\implies h = \frac{\frac{mv_0^2}2}{mg\cbr{1 + \frac 12}} = \frac{v_0^2}{2g} \cdot \frac 1{1 + \frac 12} \approx 13{,}3\,\text{м}.
    \end{align*}
}
\solutionspace{100pt}

\tasknumber{13}%
\task{%
    Плотность воздуха при нормальных условиях равна $1{,}3\,\frac{\text{кг}}{\text{м}^{3}}$.
    Чему равна плотность воздуха
    при температуре $200\celsius$ и давлении $50\,\text{кПа}$?
}
\answer{%
    \begin{align*}
    &\text{В общем случае:} PV = \frac m{\mu} RT \implies \rho = \frac mV = \frac m{\frac{\frac m{\mu} RT}P} = \frac{P\mu}{RT}, \\
    &\text{У нас 2 состояния:} \rho_1 = \frac{P_1\mu}{RT_1}, \rho_2 = \frac{P_2\mu}{RT_2} \implies \frac{\rho_2}{\rho_1} = \frac{\frac{P_2\mu}{RT_2}}{\frac{P_1\mu}{RT_1}} = \frac{P_2T_1}{P_1T_2} \implies \\
    &\implies \rho_2 = \rho_1 \cdot  \frac{P_2T_1}{P_1T_2} = 1{,}3\,\frac{\text{кг}}{\text{м}^{3}} \cdot \frac{50\,\text{кПа} \cdot 273\units{К}}{100\,\text{кПа} \cdot 473\units{К}} \approx 0{,}38\,\frac{\text{кг}}{\text{м}^{3}}.
    \end{align*}
}
\solutionspace{120pt}

\tasknumber{14}%
\task{%
    Небольшую цилиндрическую пробирку с воздухом погружают на некоторую глубину в глубокое пресное озеро,
    после чего воздух занимает в ней лишь третью часть от общего объема.
    Определите глубину, на которую погрузили пробирку.
    Температуру считать постоянной $T = 279\,\text{К}$, давлением паров воды пренебречь,
    атмосферное давление принять равным $p_{\text{aтм}} = 100\,\text{кПа}$.
}
\answer{%
    \begin{align*}
    T\text{— const} &\implies P_1V_1 = \nu RT = P_2V_2.
    \\
    V_2 = \frac 13 V_1 &\implies P_1V_1 = P_2 \cdot \frac 13V_1 \implies P_2 = 3P_1 = 3p_{\text{aтм}}.
    \\
    P_2 = p_{\text{aтм}} + \rho_{\text{в}} g h \implies h = \frac{P_2 - p_{\text{aтм}}}{\rho_{\text{в}} g} &= \frac{3p_{\text{aтм}} - p_{\text{aтм}}}{\rho_{\text{в}} g} = \frac{2 \cdot p_{\text{aтм}}}{\rho_{\text{в}} g} =  \\
     &= \frac{2 \cdot 100\,\text{кПа}}{1000\,\frac{\text{кг}}{\text{м}^{3}} \cdot  10\,\frac{\text{м}}{\text{с}^{2}}} \approx 20\,\text{м}.
    \end{align*}
}
\solutionspace{120pt}

\tasknumber{15}%
\task{%
    Газу сообщили некоторое количество теплоты,
    при этом четверть его он потратил на совершение работы,
    одновременно увеличив свою внутреннюю энергию на $3000\,\text{Дж}$.
    Определите работу, совершённую газом.
}
\answer{%
    \begin{align*}
    Q &= A' + \Delta U, A' = \frac 14 Q \implies Q \cdot \cbr{1 - \frac 14} = \Delta U \implies Q = \frac{\Delta U}{1 - \frac 14} = \frac{ 3000\,\text{Дж} }{1 - \frac 14} \approx 4000\,\text{Дж}.
    \\
    A' &= \frac 14 Q
        = \frac 14 \cdot \frac{\Delta U}{1 - \frac 14}
        = \frac{\Delta U}{4 - 1}
        = \frac{ 3000\,\text{Дж} }{4 - 1} \approx 1000\,\text{Дж}.
    \end{align*}
}
\solutionspace{60pt}

\tasknumber{16}%
\task{%
    Два конденсатора ёмкостей $C_1 = 30\,\text{нФ}$ и $C_2 = 60\,\text{нФ}$ последовательно подключают
    к источнику напряжения $U = 300\,\text{В}$ (см.
    рис.).
    % Определите заряды каждого из конденсаторов.
    Определите заряд второго конденсатора.

    \begin{tikzpicture}[circuit ee IEC, semithick]
        \draw  (0, 0) to [capacitor={info={$C_1$}}] (1, 0)
                       to [capacitor={info={$C_2$}}] (2, 0)
        ;
        % \draw [-o] (0, 0) -- ++(-0.5, 0) node[left] {$-$};
        % \draw [-o] (2, 0) -- ++(0.5, 0) node[right] {$+$};
        \draw [-o] (0, 0) -- ++(-0.5, 0) node[left] {};
        \draw [-o] (2, 0) -- ++(0.5, 0) node[right] {};
    \end{tikzpicture}
}
\answer{%
    $
        Q_1
            = Q_2
            = CU
            = \frac{ U }{\frac1{C_1} + \frac1{C_2}}
            = \frac{C_1C_2U}{C_1 + C_2}
            = \frac{
                30\,\text{нФ} \cdot 60\,\text{нФ} \cdot 300\,\text{В}
            }{
                30\,\text{нФ} + 60\,\text{нФ}
            }
            = 6{,}00\,\text{мкКл}
    $
}
\solutionspace{120pt}

\tasknumber{17}%
\task{%
    В вакууме вдоль одной прямой расположены четыре положительных заряда так,
    что расстояние между соседними зарядами равно $l$.
    Сделайте рисунок,
    и определите силу, действующую на крайний заряд.
    Модули всех зарядов равны $Q$ ($Q > 0$).
}
\answer{%
    $F = \sum_i F_i = \ldots = \frac{49}{36} \frac{kQ^2}{l^2}.$
}
\solutionspace{80pt}

\tasknumber{18}%
\task{%
    Юлия проводит эксперименты c 2 кусками одинаковой медной проволки, причём второй кусок в семь раз длиннее первого.
    В одном из экспериментов Юлия подаёт на первый кусок проволки напряжение в два раза раз больше, чем на второй.
    Определите отношения в двух проволках в этом эксперименте (второй к первой):
    \begin{itemize}
        \item отношение сил тока,
        \item отношение выделяющихся мощностей.
    \end{itemize}
}
\answer{%
    $R_2 = 7R_1, U_1 = 2U_2 \implies  \eli_2 / \eli_1 = \frac{U_2 / R_2}{U_1 / R_1} = \frac{U_2}{U_1} \cdot \frac{R_1}{R_2} = \frac1{14}, P_2 / P_1 = \frac{U_2^2 / R_2}{U_1^2 / R_1} = \sqr{\frac{U_2}{U_1}} \cdot \frac{R_1}{R_2} = \frac1{28}.$
}

\variantsplitter

\addpersonalvariant{Артём Жичин}

\tasknumber{1}%
\task{%
    Саша стартует на лошади и в течение $t = 10\,\text{c}$ двигается с постоянным ускорением $0{,}5\,\frac{\text{м}}{\text{с}^{2}}$.
    Определите
    \begin{itemize}
        \item какую скорость при этом удастся достичь,
        \item какой путь за это время будет пройден,
        \item среднюю скорость за всё время движения, если после начального ускорения продолжить движение равномерно ещё в течение времени $3t$
    \end{itemize}
}
\answer{%
    \begin{align*}
    v &= v_0 + a t = at = 0{,}5\,\frac{\text{м}}{\text{с}^{2}} \cdot 10\,\text{c} = 5{,}0\,\frac{\text{м}}{\text{с}}, \\
    s_x &= v_0t + \frac{a t^2}2 = \frac{a t^2}2 = \frac{0{,}5\,\frac{\text{м}}{\text{с}^{2}} \cdot \sqr{ 10\,\text{c} }}2 = 25{,}0\,\text{м}, \\
    v_\text{сред.} &= \frac{s_\text{общ}}{t_\text{общ.}} = \frac{s_x + v \cdot 3t}{t + 3t} = \frac{\frac{a t^2}2 + at \cdot 3t}{t (1 + 3)} = \\
    &= at \cdot \frac{\frac 12 + 3}{1 + 3} = 0{,}5\,\frac{\text{м}}{\text{с}^{2}} \cdot 10\,\text{c} \cdot \frac{\frac 12 + 3}{1 + 3} \approx 4{,}38\,\frac{\text{м}}{\text{c}}.
    \end{align*}
}
\solutionspace{120pt}

\tasknumber{2}%
\task{%
    Какой путь тело пройдёт за четвёртую секунду после начала свободного падения?
    Какую скорость в начале этой секунды оно имеет?
}
\answer{%
    \begin{align*}
    s &= -s_y = -(y_2-y_1) = y_1 - y_2 = \cbr{y_{0y} + v_{0y}t_1 - \frac{gt_1^2}2} - \cbr{y_{0y} + v_{0y}t_2 - \frac{gt_2^2}2} = \\
    &= \frac{gt_2^2}2 - \frac{gt_1^2}2 = \frac g2\cbr{t_2^2 - t_1^2} = 35{,}0\,\text{м}, \\
    v_y &= v_{0y} - gt = -gt = 10\,\frac{\text{м}}{\text{с}^{2}} \cdot 3\,\text{с} = -30\,\frac{\text{м}}{\text{с}}.
    \end{align*}
}
\solutionspace{120pt}

\tasknumber{3}%
\task{%
    Карусель радиусом $4\,\text{м}$ равномерно совершает 6 оборотов в минуту.
    Определите
    \begin{itemize}
        \item период и частоту её обращения,
        \item скорость и ускорение крайних её точек.
    \end{itemize}
}
\answer{%
    \begin{align*}
    t &= 60\,\text{с}, r = 4{,}0\,\text{м}, n = 6\units{оборотов}, \\
    T &= \frac tN = \frac{ 60\,\text{с} }{6} \approx 10{,}00\,\text{c}, \\
    \nu &= \frac 1T = \frac{6}{ 60\,\text{с} } \approx 0{,}10\,\text{Гц}, \\
    v &= \frac{2 \pi r}{T} = \frac{2 \pi r}{T} =  \frac{2 \pi r n}{t} \approx 2{,}51\,\frac{\text{м}}{\text{c}}, \\
    a &= \frac{v^2}{r} =  \frac{4 \pi^2 r n^2}{t^2} \approx 1{,}58\,\frac{\text{м}}{\text{с}^{2}}.
    \end{align*}
}
\solutionspace{80pt}

\tasknumber{4}%
\task{%
    Паша стоит на обрыве над рекой и методично и строго горизонтально кидает в неё камушки.
    За этим всем наблюдает экспериментатор Глюк, который уже выяснил, что камушки падают в реку спустя $1{,}3\,\text{с}$ после броска,
    а вот дальность полёта оценить сложнее: придётся лезть в воду.
    Выручите Глюка и определите:
    \begin{itemize}
        \item высоту обрыва (вместе с ростом Паши).
        \item дальность полёта камушков (по горизонтали) и их скорость при падении, приняв начальную скорость броска равной $v_0 = 13\,\frac{\text{м}}{\text{с}}$.
    \end{itemize}
    Сопротивлением воздуха пренебречь.
}
\answer{%
    \begin{align*}
    y &= y_0 + v_{0y}t - \frac{gt^2}2 = h - \frac{gt^2}2, \qquad y(\tau) = 0 \implies h - \frac{g\tau^2}2 = 0 \implies h = \frac{g\tau^2}2 \approx 8{,}5\,\text{м}.
    \\
    x &= x_0 + v_{0x}t = v_0t \implies L = v_0\tau \approx 16{,}9\,\text{м}.
    \\
    &v = \sqrt{v_x^2 + v_y^2} = \sqrt{v_{0x}^2 + \sqr{v_{0y} - g\tau}} = \sqrt{v_0^2 + \sqr{g\tau}} \approx 18{,}4\,\frac{\text{м}}{\text{c}}.
    \end{align*}
}
\solutionspace{120pt}

\tasknumber{5}%
\task{%
    Шесть одинаковых брусков массой $3\,\text{кг}$ каждый лежат на гладком горизонтальном столе.
    Бруски пронумерованы от 1 до 6 и последовательно связаны между собой
    невесомыми нерастяжимыми нитями: 1 со 2, 2 с 3 (ну и с 1) и т.д.
    Экспериментатор Глюк прикладывает постоянную горизонтальную силу $60\,\text{Н}$ к бруску с наибольшим номером.
    С каким ускорением двигается система? Чему равна сила натяжения нити, связывающей бруски 2 и 3?
}
\answer{%
    \begin{align*}
    a &= \frac{F}{6 m} = \frac{60\,\text{Н}}{6 \cdot 3\,\text{кг}} \approx 3{,}3\,\frac{\text{м}}{\text{c}^{2}}, \\
    T &= m'a = 2m \cdot \frac{F}{6 m} = \frac{2}{6} F \approx 20{,}0\,\text{Н}.
    \end{align*}
}
\solutionspace{120pt}

\tasknumber{6}%
\task{%
    Два бруска связаны лёгкой нерастяжимой нитью и перекинуты через неподвижный блок (см.
    рис.).
    Определите силу натяжения нити и ускорения брусков.
    Силами трения пренебречь, массы брусков
    равны $m_1 = 11\,\text{кг}$ и $m_2 = 10\,\text{кг}$.
    % $g = 10\,\frac{\text{м}}{\text{с}^{2}}$.

    \begin{tikzpicture}[x=1.5cm,y=1.5cm,thick]
        \draw
            (-0.4, 0) rectangle (-0.2, 1.2)
            (0.15, 0.5) rectangle (0.45, 1)
            (0, 2) circle [radius=0.3] -- ++(up:0.5)
            (-0.3, 1.2) -- ++(up:0.8)
            (0.3, 1) -- ++(up:1)
            (-0.7, 2.5) -- (0.7, 2.5)
            ;
        \draw[pattern={Lines[angle=51,distance=3pt]},pattern color=black,draw=none] (-0.7, 2.5) rectangle (0.7, 2.75);
        \node [left] (left) at (-0.4, 0.6) { $m_1$ };
        \node [right] (right) at (0.4, 0.75) { $m_2$ };
    \end{tikzpicture}
}
\answer{%
    Предположим, что левый брусок ускоряется вверх, тогда правый ускоряется вниз (с тем же ускорением).
    Запишем 2-й закон Ньютона 2 раза (для обоих тел) в проекции на вертикальную оси, направив её вверх.
    \begin{align*}
        &\begin{cases}
            T - m_1g = m_1a, \\
            T - m_2g = -m_2a,
        \end{cases} \\
        &\begin{cases}
            m_2g - m_1g = m_1a + m_2a, \\
            T = m_1a + m_1g, \\
        \end{cases} \\
        a &= \frac{m_2 - m_1}{m_1 + m_2} \cdot g = \frac{10\,\text{кг} - 11\,\text{кг}}{11\,\text{кг} + 10\,\text{кг}} \cdot 10\,\frac{\text{м}}{\text{с}^{2}} \approx -0{,}4800\,\frac{\text{м}}{\text{c}^{2}}, \\
        T &= m_1(a + g) = m_1 \cdot g \cdot \cbr{\frac{m_2 - m_1}{m_1 + m_2} + 1} = m_1 \cdot g \cdot \frac{2m_2}{m_1 + m_2} = \\
            &= \frac{2 m_2 m_1 g}{m_1 + m_2} = \frac{2 \cdot 10\,\text{кг} \cdot 11\,\text{кг} \cdot 10\,\frac{\text{м}}{\text{с}^{2}}}{11\,\text{кг} + 10\,\text{кг}} \approx 104{,}8\,\text{Н}.
    \end{align*}
    Отрицательный ответ говорит, что мы лишь не угадали с направлением ускорений.
    Сила же всегда положительна.
}
\solutionspace{80pt}

\tasknumber{7}%
\task{%
    Тело массой $2\,\text{кг}$ лежит на горизонтальной поверхности.
    Коэффициент трения между поверхностью и телом $0{,}2$.
    К телу приложена горизонтальная сила $4{,}5\,\text{Н}$.
    Определите силу трения, действующую на тело, и ускорение тела.
    % $g = 10\,\frac{\text{м}}{\text{с}^{2}}$.
}
\answer{%
    \begin{align*}
    &F_\text{ трения покоя $\max$ } = \mu N = \mu m g = 0{,}2 \cdot 2\,\text{кг} \cdot 10\,\frac{\text{м}}{\text{с}^{2}} = 4{,}00\,\text{Н}, \\
    &F_\text{ трения покоя $\max$ } \le F \implies F_\text{ трения } = 4{,}00\,\text{Н}, a = \frac{F - F_\text{ трения }}{ m } = 0{,}25\,\frac{\text{м}}{\text{c}^{2}}, \\
    &\text{при равенстве возможны оба варианта: и едет, и не едет, но на ответы это не влияет.}
    \end{align*}
}
\solutionspace{120pt}

\tasknumber{8}%
\task{%
    Определите плотность неизвестного вещества, если известно, что опускании тела из него
    в керосин оно будет плавать и на половину выступать над поверхностью жидкости.
}
\answer{%
    $F_\text{Арх.} = F_\text{тяж.} \implies \rho_\text{ж.} g V_\text{погр.} = m g \implies\rho_\text{ж.} g \cbr{V -\frac V2} = \rho V g \implies \rho = \rho_\text{ж.}\cbr{1 -\frac 12} \approx 400\,\frac{\text{кг}}{\text{м}^{3}}$
}
\solutionspace{120pt}

\tasknumber{9}%
\task{%
    	Определите силу, действующую на правую опору однородного горизонтального стержня длиной $l = 7\,\text{м}$
    	и массой $M = 1\,\text{кг}$, к которому подвешен груз массой $m = 2\,\text{кг}$ на расстоянии $2\,\text{м}$ от правого конца (см.
    рис.).

        \begin{tikzpicture}[thick]
            \draw
                (-2, -0.1) rectangle (2, 0.1)
                (-0.5, -0.1) -- (-0.5, -1)
                (-0.7, -1) rectangle (-0.3, -1.3)
           		(-2, -0.1) -- +(0.15,-0.9) -- +(-0.15,-0.9) -- cycle
            	(2, -0.1) -- +(0.15,-0.9) -- +(-0.15,-0.9) -- cycle
            ;
            \draw[pattern={Lines[angle=51,distance=2pt]},pattern color=black,draw=none]
            	(-2.15, -1.15) rectangle +(0.3, 0.15)
            	(2.15, -1.15) rectangle +(-0.3, 0.15)
            ;
            \node [right] (m_small) at (-0.3, -1.15) { $m$ };
            \node [above] (M_big) at (0, 0.1) { $M$ };
        \end{tikzpicture}
}
\answer{%
    \begin{align*}
        &\begin{cases}
            F_1 + F_2 - mg - Mg= 0, \\
            F_1 \cdot 0 - mg \cdot a - Mg \cdot \frac l2 + F_2 \cdot l = 0,
        \end{cases} \\
        F_2 &= \frac{mga + Mg\frac l2}l = \frac al \cdot mg + \frac{Mg}2 \approx 19{,}3\,\text{Н}, \\
        F_1 &= mg + Mg - F_2 = mg + Mg - \frac al \cdot mg - \frac{Mg}2 = \frac bl \cdot mg + \frac{Mg}2 \approx 10{,}7\,\text{Н}.
    \end{align*}
}
\solutionspace{80pt}

\tasknumber{10}%
\task{%
    Тонкий однородный кусок арматуры длиной $3\,\text{м}$ и массой $10\,\text{кг}$ лежит на горизонтальной поверхности.
    \begin{itemize}
        \item Какую минимальную силу надо приложить к одному из его концов, чтобы оторвать его от этой поверхности?
        \item Какую минимальную работу надо совершить, чтобы поставить его на землю в вертикальное положение?
    \end{itemize}
    % Примите $g = 10\,\frac{\text{м}}{\text{с}^{2}}$.
}
\answer{%
    $F = \frac{mg}2 \approx 100\,\text{Н}, A = mg\frac l2 = 150\,\text{Дж}$
}
\solutionspace{120pt}

\tasknumber{11}%
\task{%
    Определите работу силы, которая обеспечит подъём тела массой $3\,\text{кг}$ на высоту $2\,\text{м}$ с постоянным ускорением $4\,\frac{\text{м}}{\text{c}^{2}}$.
    % Примите $g = 10\,\frac{\text{м}}{\text{с}^{2}}$.
}
\answer{%
    \begin{align*}
    &\text{Для подъёма:} A = Fh = (mg + ma) h = m(g+a)h, \\
    &\text{Для спуска:} A = -Fh = -(mg - ma) h = -m(g-a)h, \\
    &\text{В результате получаем:} 84\,\text{Дж}.
    \end{align*}
}
\solutionspace{60pt}

\tasknumber{12}%
\task{%
    Тело бросили вертикально вверх со скоростью $20\,\frac{\text{м}}{\text{c}}$.
    На какой высоте кинетическая энергия тела составит половину от потенциальной?
}
\answer{%
    \begin{align*}
    &0 + \frac{mv_0^2}2 = E_p + E_k, E_k = \frac 12 E_p \implies \\
    &\implies \frac{mv_0^2}2 = E_p + \frac 12 E_p = E_p\cbr{1 + \frac 12} = mgh\cbr{1 + \frac 12} \implies \\
    &\implies h = \frac{\frac{mv_0^2}2}{mg\cbr{1 + \frac 12}} = \frac{v_0^2}{2g} \cdot \frac 1{1 + \frac 12} \approx 13{,}3\,\text{м}.
    \end{align*}
}
\solutionspace{100pt}

\tasknumber{13}%
\task{%
    Плотность воздуха при нормальных условиях равна $1{,}3\,\frac{\text{кг}}{\text{м}^{3}}$.
    Чему равна плотность воздуха
    при температуре $100\celsius$ и давлении $80\,\text{кПа}$?
}
\answer{%
    \begin{align*}
    &\text{В общем случае:} PV = \frac m{\mu} RT \implies \rho = \frac mV = \frac m{\frac{\frac m{\mu} RT}P} = \frac{P\mu}{RT}, \\
    &\text{У нас 2 состояния:} \rho_1 = \frac{P_1\mu}{RT_1}, \rho_2 = \frac{P_2\mu}{RT_2} \implies \frac{\rho_2}{\rho_1} = \frac{\frac{P_2\mu}{RT_2}}{\frac{P_1\mu}{RT_1}} = \frac{P_2T_1}{P_1T_2} \implies \\
    &\implies \rho_2 = \rho_1 \cdot  \frac{P_2T_1}{P_1T_2} = 1{,}3\,\frac{\text{кг}}{\text{м}^{3}} \cdot \frac{80\,\text{кПа} \cdot 273\units{К}}{100\,\text{кПа} \cdot 373\units{К}} \approx 0{,}76\,\frac{\text{кг}}{\text{м}^{3}}.
    \end{align*}
}
\solutionspace{120pt}

\tasknumber{14}%
\task{%
    Небольшую цилиндрическую пробирку с воздухом погружают на некоторую глубину в глубокое пресное озеро,
    после чего воздух занимает в ней лишь третью часть от общего объема.
    Определите глубину, на которую погрузили пробирку.
    Температуру считать постоянной $T = 280\,\text{К}$, давлением паров воды пренебречь,
    атмосферное давление принять равным $p_{\text{aтм}} = 100\,\text{кПа}$.
}
\answer{%
    \begin{align*}
    T\text{— const} &\implies P_1V_1 = \nu RT = P_2V_2.
    \\
    V_2 = \frac 13 V_1 &\implies P_1V_1 = P_2 \cdot \frac 13V_1 \implies P_2 = 3P_1 = 3p_{\text{aтм}}.
    \\
    P_2 = p_{\text{aтм}} + \rho_{\text{в}} g h \implies h = \frac{P_2 - p_{\text{aтм}}}{\rho_{\text{в}} g} &= \frac{3p_{\text{aтм}} - p_{\text{aтм}}}{\rho_{\text{в}} g} = \frac{2 \cdot p_{\text{aтм}}}{\rho_{\text{в}} g} =  \\
     &= \frac{2 \cdot 100\,\text{кПа}}{1000\,\frac{\text{кг}}{\text{м}^{3}} \cdot  10\,\frac{\text{м}}{\text{с}^{2}}} \approx 20\,\text{м}.
    \end{align*}
}
\solutionspace{120pt}

\tasknumber{15}%
\task{%
    Газу сообщили некоторое количество теплоты,
    при этом треть его он потратил на совершение работы,
    одновременно увеличив свою внутреннюю энергию на $3000\,\text{Дж}$.
    Определите работу, совершённую газом.
}
\answer{%
    \begin{align*}
    Q &= A' + \Delta U, A' = \frac 13 Q \implies Q \cdot \cbr{1 - \frac 13} = \Delta U \implies Q = \frac{\Delta U}{1 - \frac 13} = \frac{ 3000\,\text{Дж} }{1 - \frac 13} \approx 4500\,\text{Дж}.
    \\
    A' &= \frac 13 Q
        = \frac 13 \cdot \frac{\Delta U}{1 - \frac 13}
        = \frac{\Delta U}{3 - 1}
        = \frac{ 3000\,\text{Дж} }{3 - 1} \approx 1500\,\text{Дж}.
    \end{align*}
}
\solutionspace{60pt}

\tasknumber{16}%
\task{%
    Два конденсатора ёмкостей $C_1 = 30\,\text{нФ}$ и $C_2 = 60\,\text{нФ}$ последовательно подключают
    к источнику напряжения $U = 300\,\text{В}$ (см.
    рис.).
    % Определите заряды каждого из конденсаторов.
    Определите заряд второго конденсатора.

    \begin{tikzpicture}[circuit ee IEC, semithick]
        \draw  (0, 0) to [capacitor={info={$C_1$}}] (1, 0)
                       to [capacitor={info={$C_2$}}] (2, 0)
        ;
        % \draw [-o] (0, 0) -- ++(-0.5, 0) node[left] {$-$};
        % \draw [-o] (2, 0) -- ++(0.5, 0) node[right] {$+$};
        \draw [-o] (0, 0) -- ++(-0.5, 0) node[left] {};
        \draw [-o] (2, 0) -- ++(0.5, 0) node[right] {};
    \end{tikzpicture}
}
\answer{%
    $
        Q_1
            = Q_2
            = CU
            = \frac{ U }{\frac1{C_1} + \frac1{C_2}}
            = \frac{C_1C_2U}{C_1 + C_2}
            = \frac{
                30\,\text{нФ} \cdot 60\,\text{нФ} \cdot 300\,\text{В}
            }{
                30\,\text{нФ} + 60\,\text{нФ}
            }
            = 6{,}00\,\text{мкКл}
    $
}
\solutionspace{120pt}

\tasknumber{17}%
\task{%
    В вакууме вдоль одной прямой расположены четыре отрицательных заряда так,
    что расстояние между соседними зарядами равно $a$.
    Сделайте рисунок,
    и определите силу, действующую на крайний заряд.
    Модули всех зарядов равны $q$ ($q > 0$).
}
\answer{%
    $F = \sum_i F_i = \ldots = \frac{49}{36} \frac{kq^2}{a^2}.$
}
\solutionspace{80pt}

\tasknumber{18}%
\task{%
    Юлия проводит эксперименты c 2 кусками одинаковой стальной проволки, причём второй кусок в семь раз длиннее первого.
    В одном из экспериментов Юлия подаёт на первый кусок проволки напряжение в три раза раз больше, чем на второй.
    Определите отношения в двух проволках в этом эксперименте (второй к первой):
    \begin{itemize}
        \item отношение сил тока,
        \item отношение выделяющихся мощностей.
    \end{itemize}
}
\answer{%
    $R_2 = 7R_1, U_1 = 3U_2 \implies  \eli_2 / \eli_1 = \frac{U_2 / R_2}{U_1 / R_1} = \frac{U_2}{U_1} \cdot \frac{R_1}{R_2} = \frac1{21}, P_2 / P_1 = \frac{U_2^2 / R_2}{U_1^2 / R_1} = \sqr{\frac{U_2}{U_1}} \cdot \frac{R_1}{R_2} = \frac1{63}.$
}

\variantsplitter

\addpersonalvariant{Дарья Кошман}

\tasknumber{1}%
\task{%
    Саша стартует на велосипеде и в течение $t = 5\,\text{c}$ двигается с постоянным ускорением $2{,}5\,\frac{\text{м}}{\text{с}^{2}}$.
    Определите
    \begin{itemize}
        \item какую скорость при этом удастся достичь,
        \item какой путь за это время будет пройден,
        \item среднюю скорость за всё время движения, если после начального ускорения продолжить движение равномерно ещё в течение времени $3t$
    \end{itemize}
}
\answer{%
    \begin{align*}
    v &= v_0 + a t = at = 2{,}5\,\frac{\text{м}}{\text{с}^{2}} \cdot 5\,\text{c} = 12{,}5\,\frac{\text{м}}{\text{с}}, \\
    s_x &= v_0t + \frac{a t^2}2 = \frac{a t^2}2 = \frac{2{,}5\,\frac{\text{м}}{\text{с}^{2}} \cdot \sqr{ 5\,\text{c} }}2 = 31{,}2\,\text{м}, \\
    v_\text{сред.} &= \frac{s_\text{общ}}{t_\text{общ.}} = \frac{s_x + v \cdot 3t}{t + 3t} = \frac{\frac{a t^2}2 + at \cdot 3t}{t (1 + 3)} = \\
    &= at \cdot \frac{\frac 12 + 3}{1 + 3} = 2{,}5\,\frac{\text{м}}{\text{с}^{2}} \cdot 5\,\text{c} \cdot \frac{\frac 12 + 3}{1 + 3} \approx 10{,}94\,\frac{\text{м}}{\text{c}}.
    \end{align*}
}
\solutionspace{120pt}

\tasknumber{2}%
\task{%
    Какой путь тело пройдёт за шестую секунду после начала свободного падения?
    Какую скорость в начале этой секунды оно имеет?
}
\answer{%
    \begin{align*}
    s &= -s_y = -(y_2-y_1) = y_1 - y_2 = \cbr{y_{0y} + v_{0y}t_1 - \frac{gt_1^2}2} - \cbr{y_{0y} + v_{0y}t_2 - \frac{gt_2^2}2} = \\
    &= \frac{gt_2^2}2 - \frac{gt_1^2}2 = \frac g2\cbr{t_2^2 - t_1^2} = 55{,}0\,\text{м}, \\
    v_y &= v_{0y} - gt = -gt = 10\,\frac{\text{м}}{\text{с}^{2}} \cdot 5\,\text{с} = -50\,\frac{\text{м}}{\text{с}}.
    \end{align*}
}
\solutionspace{120pt}

\tasknumber{3}%
\task{%
    Карусель радиусом $4\,\text{м}$ равномерно совершает 10 оборотов в минуту.
    Определите
    \begin{itemize}
        \item период и частоту её обращения,
        \item скорость и ускорение крайних её точек.
    \end{itemize}
}
\answer{%
    \begin{align*}
    t &= 60\,\text{с}, r = 4{,}0\,\text{м}, n = 10\units{оборотов}, \\
    T &= \frac tN = \frac{ 60\,\text{с} }{10} \approx 6{,}00\,\text{c}, \\
    \nu &= \frac 1T = \frac{10}{ 60\,\text{с} } \approx 0{,}17\,\text{Гц}, \\
    v &= \frac{2 \pi r}{T} = \frac{2 \pi r}{T} =  \frac{2 \pi r n}{t} \approx 4{,}19\,\frac{\text{м}}{\text{c}}, \\
    a &= \frac{v^2}{r} =  \frac{4 \pi^2 r n^2}{t^2} \approx 4{,}39\,\frac{\text{м}}{\text{с}^{2}}.
    \end{align*}
}
\solutionspace{80pt}

\tasknumber{4}%
\task{%
    Миша стоит на обрыве над рекой и методично и строго горизонтально кидает в неё камушки.
    За этим всем наблюдает экспериментатор Глюк, который уже выяснил, что камушки падают в реку спустя $1{,}2\,\text{с}$ после броска,
    а вот дальность полёта оценить сложнее: придётся лезть в воду.
    Выручите Глюка и определите:
    \begin{itemize}
        \item высоту обрыва (вместе с ростом Миши).
        \item дальность полёта камушков (по горизонтали) и их скорость при падении, приняв начальную скорость броска равной $v_0 = 12\,\frac{\text{м}}{\text{с}}$.
    \end{itemize}
    Сопротивлением воздуха пренебречь.
}
\answer{%
    \begin{align*}
    y &= y_0 + v_{0y}t - \frac{gt^2}2 = h - \frac{gt^2}2, \qquad y(\tau) = 0 \implies h - \frac{g\tau^2}2 = 0 \implies h = \frac{g\tau^2}2 \approx 7{,}2\,\text{м}.
    \\
    x &= x_0 + v_{0x}t = v_0t \implies L = v_0\tau \approx 14{,}4\,\text{м}.
    \\
    &v = \sqrt{v_x^2 + v_y^2} = \sqrt{v_{0x}^2 + \sqr{v_{0y} - g\tau}} = \sqrt{v_0^2 + \sqr{g\tau}} \approx 17{,}0\,\frac{\text{м}}{\text{c}}.
    \end{align*}
}
\solutionspace{120pt}

\tasknumber{5}%
\task{%
    Четыре одинаковых брусков массой $3\,\text{кг}$ каждый лежат на гладком горизонтальном столе.
    Бруски пронумерованы от 1 до 4 и последовательно связаны между собой
    невесомыми нерастяжимыми нитями: 1 со 2, 2 с 3 (ну и с 1) и т.д.
    Экспериментатор Глюк прикладывает постоянную горизонтальную силу $90\,\text{Н}$ к бруску с наибольшим номером.
    С каким ускорением двигается система? Чему равна сила натяжения нити, связывающей бруски 1 и 2?
}
\answer{%
    \begin{align*}
    a &= \frac{F}{4 m} = \frac{90\,\text{Н}}{4 \cdot 3\,\text{кг}} \approx 7{,}5\,\frac{\text{м}}{\text{c}^{2}}, \\
    T &= m'a = 1m \cdot \frac{F}{4 m} = \frac{1}{4} F \approx 22{,}5\,\text{Н}.
    \end{align*}
}
\solutionspace{120pt}

\tasknumber{6}%
\task{%
    Два бруска связаны лёгкой нерастяжимой нитью и перекинуты через неподвижный блок (см.
    рис.).
    Определите силу натяжения нити и ускорения брусков.
    Силами трения пренебречь, массы брусков
    равны $m_1 = 8\,\text{кг}$ и $m_2 = 6\,\text{кг}$.
    % $g = 10\,\frac{\text{м}}{\text{с}^{2}}$.

    \begin{tikzpicture}[x=1.5cm,y=1.5cm,thick]
        \draw
            (-0.4, 0) rectangle (-0.2, 1.2)
            (0.15, 0.5) rectangle (0.45, 1)
            (0, 2) circle [radius=0.3] -- ++(up:0.5)
            (-0.3, 1.2) -- ++(up:0.8)
            (0.3, 1) -- ++(up:1)
            (-0.7, 2.5) -- (0.7, 2.5)
            ;
        \draw[pattern={Lines[angle=51,distance=3pt]},pattern color=black,draw=none] (-0.7, 2.5) rectangle (0.7, 2.75);
        \node [left] (left) at (-0.4, 0.6) { $m_1$ };
        \node [right] (right) at (0.4, 0.75) { $m_2$ };
    \end{tikzpicture}
}
\answer{%
    Предположим, что левый брусок ускоряется вверх, тогда правый ускоряется вниз (с тем же ускорением).
    Запишем 2-й закон Ньютона 2 раза (для обоих тел) в проекции на вертикальную оси, направив её вверх.
    \begin{align*}
        &\begin{cases}
            T - m_1g = m_1a, \\
            T - m_2g = -m_2a,
        \end{cases} \\
        &\begin{cases}
            m_2g - m_1g = m_1a + m_2a, \\
            T = m_1a + m_1g, \\
        \end{cases} \\
        a &= \frac{m_2 - m_1}{m_1 + m_2} \cdot g = \frac{6\,\text{кг} - 8\,\text{кг}}{8\,\text{кг} + 6\,\text{кг}} \cdot 10\,\frac{\text{м}}{\text{с}^{2}} \approx -1{,}4300\,\frac{\text{м}}{\text{c}^{2}}, \\
        T &= m_1(a + g) = m_1 \cdot g \cdot \cbr{\frac{m_2 - m_1}{m_1 + m_2} + 1} = m_1 \cdot g \cdot \frac{2m_2}{m_1 + m_2} = \\
            &= \frac{2 m_2 m_1 g}{m_1 + m_2} = \frac{2 \cdot 6\,\text{кг} \cdot 8\,\text{кг} \cdot 10\,\frac{\text{м}}{\text{с}^{2}}}{8\,\text{кг} + 6\,\text{кг}} \approx 68{,}6\,\text{Н}.
    \end{align*}
    Отрицательный ответ говорит, что мы лишь не угадали с направлением ускорений.
    Сила же всегда положительна.
}
\solutionspace{80pt}

\tasknumber{7}%
\task{%
    Тело массой $2\,\text{кг}$ лежит на горизонтальной поверхности.
    Коэффициент трения между поверхностью и телом $0{,}15$.
    К телу приложена горизонтальная сила $2{,}5\,\text{Н}$.
    Определите силу трения, действующую на тело, и ускорение тела.
    % $g = 10\,\frac{\text{м}}{\text{с}^{2}}$.
}
\answer{%
    \begin{align*}
    &F_\text{ трения покоя $\max$ } = \mu N = \mu m g = 0{,}15 \cdot 2\,\text{кг} \cdot 10\,\frac{\text{м}}{\text{с}^{2}} = 3{,}00\,\text{Н}, \\
    &F_\text{ трения покоя $\max$ } > F \implies F_\text{ трения } = 2{,}50\,\text{Н}, a = \frac{F - F_\text{ трения }}{ m } = 0\,\frac{\text{м}}{\text{c}^{2}}, \\
    &\text{при равенстве возможны оба варианта: и едет, и не едет, но на ответы это не влияет.}
    \end{align*}
}
\solutionspace{120pt}

\tasknumber{8}%
\task{%
    Определите плотность неизвестного вещества, если известно, что опускании тела из него
    в керосин оно будет плавать и на половину выступать над поверхностью жидкости.
}
\answer{%
    $F_\text{Арх.} = F_\text{тяж.} \implies \rho_\text{ж.} g V_\text{погр.} = m g \implies\rho_\text{ж.} g \cbr{V -\frac V2} = \rho V g \implies \rho = \rho_\text{ж.}\cbr{1 -\frac 12} \approx 400\,\frac{\text{кг}}{\text{м}^{3}}$
}
\solutionspace{120pt}

\tasknumber{9}%
\task{%
    	Определите силу, действующую на левую опору однородного горизонтального стержня длиной $l = 7\,\text{м}$
    	и массой $M = 5\,\text{кг}$, к которому подвешен груз массой $m = 4\,\text{кг}$ на расстоянии $4\,\text{м}$ от правого конца (см.
    рис.).

        \begin{tikzpicture}[thick]
            \draw
                (-2, -0.1) rectangle (2, 0.1)
                (-0.5, -0.1) -- (-0.5, -1)
                (-0.7, -1) rectangle (-0.3, -1.3)
           		(-2, -0.1) -- +(0.15,-0.9) -- +(-0.15,-0.9) -- cycle
            	(2, -0.1) -- +(0.15,-0.9) -- +(-0.15,-0.9) -- cycle
            ;
            \draw[pattern={Lines[angle=51,distance=2pt]},pattern color=black,draw=none]
            	(-2.15, -1.15) rectangle +(0.3, 0.15)
            	(2.15, -1.15) rectangle +(-0.3, 0.15)
            ;
            \node [right] (m_small) at (-0.3, -1.15) { $m$ };
            \node [above] (M_big) at (0, 0.1) { $M$ };
        \end{tikzpicture}
}
\answer{%
    \begin{align*}
        &\begin{cases}
            F_1 + F_2 - mg - Mg= 0, \\
            F_1 \cdot 0 - mg \cdot a - Mg \cdot \frac l2 + F_2 \cdot l = 0,
        \end{cases} \\
        F_2 &= \frac{mga + Mg\frac l2}l = \frac al \cdot mg + \frac{Mg}2 \approx 42{,}1\,\text{Н}, \\
        F_1 &= mg + Mg - F_2 = mg + Mg - \frac al \cdot mg - \frac{Mg}2 = \frac bl \cdot mg + \frac{Mg}2 \approx 47{,}9\,\text{Н}.
    \end{align*}
}
\solutionspace{80pt}

\tasknumber{10}%
\task{%
    Тонкий однородный лом длиной $3\,\text{м}$ и массой $20\,\text{кг}$ лежит на горизонтальной поверхности.
    \begin{itemize}
        \item Какую минимальную силу надо приложить к одному из его концов, чтобы оторвать его от этой поверхности?
        \item Какую минимальную работу надо совершить, чтобы поставить его на землю в вертикальное положение?
    \end{itemize}
    % Примите $g = 10\,\frac{\text{м}}{\text{с}^{2}}$.
}
\answer{%
    $F = \frac{mg}2 \approx 200\,\text{Н}, A = mg\frac l2 = 300\,\text{Дж}$
}
\solutionspace{120pt}

\tasknumber{11}%
\task{%
    Определите работу силы, которая обеспечит спуск тела массой $2\,\text{кг}$ на высоту $5\,\text{м}$ с постоянным ускорением $3\,\frac{\text{м}}{\text{c}^{2}}$.
    % Примите $g = 10\,\frac{\text{м}}{\text{с}^{2}}$.
}
\answer{%
    \begin{align*}
    &\text{Для подъёма:} A = Fh = (mg + ma) h = m(g+a)h, \\
    &\text{Для спуска:} A = -Fh = -(mg - ma) h = -m(g-a)h, \\
    &\text{В результате получаем:} -70\,\text{Дж}.
    \end{align*}
}
\solutionspace{60pt}

\tasknumber{12}%
\task{%
    Тело бросили вертикально вверх со скоростью $14\,\frac{\text{м}}{\text{c}}$.
    На какой высоте кинетическая энергия тела составит треть от потенциальной?
}
\answer{%
    \begin{align*}
    &0 + \frac{mv_0^2}2 = E_p + E_k, E_k = \frac 13 E_p \implies \\
    &\implies \frac{mv_0^2}2 = E_p + \frac 13 E_p = E_p\cbr{1 + \frac 13} = mgh\cbr{1 + \frac 13} \implies \\
    &\implies h = \frac{\frac{mv_0^2}2}{mg\cbr{1 + \frac 13}} = \frac{v_0^2}{2g} \cdot \frac 1{1 + \frac 13} \approx 7{,}4\,\text{м}.
    \end{align*}
}
\solutionspace{100pt}

\tasknumber{13}%
\task{%
    Плотность воздуха при нормальных условиях равна $1{,}3\,\frac{\text{кг}}{\text{м}^{3}}$.
    Чему равна плотность воздуха
    при температуре $150\celsius$ и давлении $80\,\text{кПа}$?
}
\answer{%
    \begin{align*}
    &\text{В общем случае:} PV = \frac m{\mu} RT \implies \rho = \frac mV = \frac m{\frac{\frac m{\mu} RT}P} = \frac{P\mu}{RT}, \\
    &\text{У нас 2 состояния:} \rho_1 = \frac{P_1\mu}{RT_1}, \rho_2 = \frac{P_2\mu}{RT_2} \implies \frac{\rho_2}{\rho_1} = \frac{\frac{P_2\mu}{RT_2}}{\frac{P_1\mu}{RT_1}} = \frac{P_2T_1}{P_1T_2} \implies \\
    &\implies \rho_2 = \rho_1 \cdot  \frac{P_2T_1}{P_1T_2} = 1{,}3\,\frac{\text{кг}}{\text{м}^{3}} \cdot \frac{80\,\text{кПа} \cdot 273\units{К}}{100\,\text{кПа} \cdot 423\units{К}} \approx 0{,}67\,\frac{\text{кг}}{\text{м}^{3}}.
    \end{align*}
}
\solutionspace{120pt}

\tasknumber{14}%
\task{%
    Небольшую цилиндрическую пробирку с воздухом погружают на некоторую глубину в глубокое пресное озеро,
    после чего воздух занимает в ней лишь третью часть от общего объема.
    Определите глубину, на которую погрузили пробирку.
    Температуру считать постоянной $T = 280\,\text{К}$, давлением паров воды пренебречь,
    атмосферное давление принять равным $p_{\text{aтм}} = 100\,\text{кПа}$.
}
\answer{%
    \begin{align*}
    T\text{— const} &\implies P_1V_1 = \nu RT = P_2V_2.
    \\
    V_2 = \frac 13 V_1 &\implies P_1V_1 = P_2 \cdot \frac 13V_1 \implies P_2 = 3P_1 = 3p_{\text{aтм}}.
    \\
    P_2 = p_{\text{aтм}} + \rho_{\text{в}} g h \implies h = \frac{P_2 - p_{\text{aтм}}}{\rho_{\text{в}} g} &= \frac{3p_{\text{aтм}} - p_{\text{aтм}}}{\rho_{\text{в}} g} = \frac{2 \cdot p_{\text{aтм}}}{\rho_{\text{в}} g} =  \\
     &= \frac{2 \cdot 100\,\text{кПа}}{1000\,\frac{\text{кг}}{\text{м}^{3}} \cdot  10\,\frac{\text{м}}{\text{с}^{2}}} \approx 20\,\text{м}.
    \end{align*}
}
\solutionspace{120pt}

\tasknumber{15}%
\task{%
    Газу сообщили некоторое количество теплоты,
    при этом половину его он потратил на совершение работы,
    одновременно увеличив свою внутреннюю энергию на $1200\,\text{Дж}$.
    Определите работу, совершённую газом.
}
\answer{%
    \begin{align*}
    Q &= A' + \Delta U, A' = \frac 12 Q \implies Q \cdot \cbr{1 - \frac 12} = \Delta U \implies Q = \frac{\Delta U}{1 - \frac 12} = \frac{ 1200\,\text{Дж} }{1 - \frac 12} \approx 2400\,\text{Дж}.
    \\
    A' &= \frac 12 Q
        = \frac 12 \cdot \frac{\Delta U}{1 - \frac 12}
        = \frac{\Delta U}{2 - 1}
        = \frac{ 1200\,\text{Дж} }{2 - 1} \approx 1200\,\text{Дж}.
    \end{align*}
}
\solutionspace{60pt}

\tasknumber{16}%
\task{%
    Два конденсатора ёмкостей $C_1 = 20\,\text{нФ}$ и $C_2 = 60\,\text{нФ}$ последовательно подключают
    к источнику напряжения $V = 450\,\text{В}$ (см.
    рис.).
    % Определите заряды каждого из конденсаторов.
    Определите заряд первого конденсатора.

    \begin{tikzpicture}[circuit ee IEC, semithick]
        \draw  (0, 0) to [capacitor={info={$C_1$}}] (1, 0)
                       to [capacitor={info={$C_2$}}] (2, 0)
        ;
        % \draw [-o] (0, 0) -- ++(-0.5, 0) node[left] {$-$};
        % \draw [-o] (2, 0) -- ++(0.5, 0) node[right] {$+$};
        \draw [-o] (0, 0) -- ++(-0.5, 0) node[left] {};
        \draw [-o] (2, 0) -- ++(0.5, 0) node[right] {};
    \end{tikzpicture}
}
\answer{%
    $
        Q_1
            = Q_2
            = CV
            = \frac{ V }{\frac1{C_1} + \frac1{C_2}}
            = \frac{C_1C_2V}{C_1 + C_2}
            = \frac{
                20\,\text{нФ} \cdot 60\,\text{нФ} \cdot 450\,\text{В}
            }{
                20\,\text{нФ} + 60\,\text{нФ}
            }
            = 6{,}75\,\text{мкКл}
    $
}
\solutionspace{120pt}

\tasknumber{17}%
\task{%
    В вакууме вдоль одной прямой расположены три отрицательных заряда так,
    что расстояние между соседними зарядами равно $r$.
    Сделайте рисунок,
    и определите силу, действующую на крайний заряд.
    Модули всех зарядов равны $q$ ($q > 0$).
}
\answer{%
    $F = \sum_i F_i = \ldots = \frac54 \frac{kq^2}{r^2}.$
}
\solutionspace{80pt}

\tasknumber{18}%
\task{%
    Юлия проводит эксперименты c 2 кусками одинаковой медной проволки, причём второй кусок в десять раз длиннее первого.
    В одном из экспериментов Юлия подаёт на первый кусок проволки напряжение в семь раз раз больше, чем на второй.
    Определите отношения в двух проволках в этом эксперименте (второй к первой):
    \begin{itemize}
        \item отношение сил тока,
        \item отношение выделяющихся мощностей.
    \end{itemize}
}
\answer{%
    $R_2 = 10R_1, U_1 = 7U_2 \implies  \eli_2 / \eli_1 = \frac{U_2 / R_2}{U_1 / R_1} = \frac{U_2}{U_1} \cdot \frac{R_1}{R_2} = \frac1{70}, P_2 / P_1 = \frac{U_2^2 / R_2}{U_1^2 / R_1} = \sqr{\frac{U_2}{U_1}} \cdot \frac{R_1}{R_2} = \frac1{490}.$
}

\variantsplitter

\addpersonalvariant{Анна Кузьмичёва}

\tasknumber{1}%
\task{%
    Саша стартует на велосипеде и в течение $t = 3\,\text{c}$ двигается с постоянным ускорением $2\,\frac{\text{м}}{\text{с}^{2}}$.
    Определите
    \begin{itemize}
        \item какую скорость при этом удастся достичь,
        \item какой путь за это время будет пройден,
        \item среднюю скорость за всё время движения, если после начального ускорения продолжить движение равномерно ещё в течение времени $3t$
    \end{itemize}
}
\answer{%
    \begin{align*}
    v &= v_0 + a t = at = 2\,\frac{\text{м}}{\text{с}^{2}} \cdot 3\,\text{c} = 6{,}0\,\frac{\text{м}}{\text{с}}, \\
    s_x &= v_0t + \frac{a t^2}2 = \frac{a t^2}2 = \frac{2\,\frac{\text{м}}{\text{с}^{2}} \cdot \sqr{ 3\,\text{c} }}2 = 9{,}0\,\text{м}, \\
    v_\text{сред.} &= \frac{s_\text{общ}}{t_\text{общ.}} = \frac{s_x + v \cdot 3t}{t + 3t} = \frac{\frac{a t^2}2 + at \cdot 3t}{t (1 + 3)} = \\
    &= at \cdot \frac{\frac 12 + 3}{1 + 3} = 2\,\frac{\text{м}}{\text{с}^{2}} \cdot 3\,\text{c} \cdot \frac{\frac 12 + 3}{1 + 3} \approx 5{,}25\,\frac{\text{м}}{\text{c}}.
    \end{align*}
}
\solutionspace{120pt}

\tasknumber{2}%
\task{%
    Какой путь тело пройдёт за шестую секунду после начала свободного падения?
    Какую скорость в начале этой секунды оно имеет?
}
\answer{%
    \begin{align*}
    s &= -s_y = -(y_2-y_1) = y_1 - y_2 = \cbr{y_{0y} + v_{0y}t_1 - \frac{gt_1^2}2} - \cbr{y_{0y} + v_{0y}t_2 - \frac{gt_2^2}2} = \\
    &= \frac{gt_2^2}2 - \frac{gt_1^2}2 = \frac g2\cbr{t_2^2 - t_1^2} = 55{,}0\,\text{м}, \\
    v_y &= v_{0y} - gt = -gt = 10\,\frac{\text{м}}{\text{с}^{2}} \cdot 5\,\text{с} = -50\,\frac{\text{м}}{\text{с}}.
    \end{align*}
}
\solutionspace{120pt}

\tasknumber{3}%
\task{%
    Карусель диаметром $5\,\text{м}$ равномерно совершает 5 оборотов в минуту.
    Определите
    \begin{itemize}
        \item период и частоту её обращения,
        \item скорость и ускорение крайних её точек.
    \end{itemize}
}
\answer{%
    \begin{align*}
    t &= 60\,\text{с}, r = 2{,}5\,\text{м}, n = 5\units{оборотов}, \\
    T &= \frac tN = \frac{ 60\,\text{с} }{5} \approx 12{,}00\,\text{c}, \\
    \nu &= \frac 1T = \frac{5}{ 60\,\text{с} } \approx 0{,}08\,\text{Гц}, \\
    v &= \frac{2 \pi r}{T} = \frac{2 \pi r}{T} =  \frac{2 \pi r n}{t} \approx 1{,}31\,\frac{\text{м}}{\text{c}}, \\
    a &= \frac{v^2}{r} =  \frac{4 \pi^2 r n^2}{t^2} \approx 0{,}69\,\frac{\text{м}}{\text{с}^{2}}.
    \end{align*}
}
\solutionspace{80pt}

\tasknumber{4}%
\task{%
    Даша стоит на обрыве над рекой и методично и строго горизонтально кидает в неё камушки.
    За этим всем наблюдает экспериментатор Глюк, который уже выяснил, что камушки падают в реку спустя $1{,}2\,\text{с}$ после броска,
    а вот дальность полёта оценить сложнее: придётся лезть в воду.
    Выручите Глюка и определите:
    \begin{itemize}
        \item высоту обрыва (вместе с ростом Даши).
        \item дальность полёта камушков (по горизонтали) и их скорость при падении, приняв начальную скорость броска равной $v_0 = 18\,\frac{\text{м}}{\text{с}}$.
    \end{itemize}
    Сопротивлением воздуха пренебречь.
}
\answer{%
    \begin{align*}
    y &= y_0 + v_{0y}t - \frac{gt^2}2 = h - \frac{gt^2}2, \qquad y(\tau) = 0 \implies h - \frac{g\tau^2}2 = 0 \implies h = \frac{g\tau^2}2 \approx 7{,}2\,\text{м}.
    \\
    x &= x_0 + v_{0x}t = v_0t \implies L = v_0\tau \approx 21{,}6\,\text{м}.
    \\
    &v = \sqrt{v_x^2 + v_y^2} = \sqrt{v_{0x}^2 + \sqr{v_{0y} - g\tau}} = \sqrt{v_0^2 + \sqr{g\tau}} \approx 21{,}6\,\frac{\text{м}}{\text{c}}.
    \end{align*}
}
\solutionspace{120pt}

\tasknumber{5}%
\task{%
    Пять одинаковых брусков массой $2\,\text{кг}$ каждый лежат на гладком горизонтальном столе.
    Бруски пронумерованы от 1 до 5 и последовательно связаны между собой
    невесомыми нерастяжимыми нитями: 1 со 2, 2 с 3 (ну и с 1) и т.д.
    Экспериментатор Глюк прикладывает постоянную горизонтальную силу $90\,\text{Н}$ к бруску с наименьшим номером.
    С каким ускорением двигается система? Чему равна сила натяжения нити, связывающей бруски 1 и 2?
}
\answer{%
    \begin{align*}
    a &= \frac{F}{5 m} = \frac{90\,\text{Н}}{5 \cdot 2\,\text{кг}} \approx 9{,}0\,\frac{\text{м}}{\text{c}^{2}}, \\
    T &= m'a = 4m \cdot \frac{F}{5 m} = \frac{4}{5} F \approx 72{,}0\,\text{Н}.
    \end{align*}
}
\solutionspace{120pt}

\tasknumber{6}%
\task{%
    Два бруска связаны лёгкой нерастяжимой нитью и перекинуты через неподвижный блок (см.
    рис.).
    Определите силу натяжения нити и ускорения брусков.
    Силами трения пренебречь, массы брусков
    равны $m_1 = 5\,\text{кг}$ и $m_2 = 4\,\text{кг}$.
    % $g = 10\,\frac{\text{м}}{\text{с}^{2}}$.

    \begin{tikzpicture}[x=1.5cm,y=1.5cm,thick]
        \draw
            (-0.4, 0) rectangle (-0.2, 1.2)
            (0.15, 0.5) rectangle (0.45, 1)
            (0, 2) circle [radius=0.3] -- ++(up:0.5)
            (-0.3, 1.2) -- ++(up:0.8)
            (0.3, 1) -- ++(up:1)
            (-0.7, 2.5) -- (0.7, 2.5)
            ;
        \draw[pattern={Lines[angle=51,distance=3pt]},pattern color=black,draw=none] (-0.7, 2.5) rectangle (0.7, 2.75);
        \node [left] (left) at (-0.4, 0.6) { $m_1$ };
        \node [right] (right) at (0.4, 0.75) { $m_2$ };
    \end{tikzpicture}
}
\answer{%
    Предположим, что левый брусок ускоряется вверх, тогда правый ускоряется вниз (с тем же ускорением).
    Запишем 2-й закон Ньютона 2 раза (для обоих тел) в проекции на вертикальную оси, направив её вверх.
    \begin{align*}
        &\begin{cases}
            T - m_1g = m_1a, \\
            T - m_2g = -m_2a,
        \end{cases} \\
        &\begin{cases}
            m_2g - m_1g = m_1a + m_2a, \\
            T = m_1a + m_1g, \\
        \end{cases} \\
        a &= \frac{m_2 - m_1}{m_1 + m_2} \cdot g = \frac{4\,\text{кг} - 5\,\text{кг}}{5\,\text{кг} + 4\,\text{кг}} \cdot 10\,\frac{\text{м}}{\text{с}^{2}} \approx -1{,}1100\,\frac{\text{м}}{\text{c}^{2}}, \\
        T &= m_1(a + g) = m_1 \cdot g \cdot \cbr{\frac{m_2 - m_1}{m_1 + m_2} + 1} = m_1 \cdot g \cdot \frac{2m_2}{m_1 + m_2} = \\
            &= \frac{2 m_2 m_1 g}{m_1 + m_2} = \frac{2 \cdot 4\,\text{кг} \cdot 5\,\text{кг} \cdot 10\,\frac{\text{м}}{\text{с}^{2}}}{5\,\text{кг} + 4\,\text{кг}} \approx 44{,}4\,\text{Н}.
    \end{align*}
    Отрицательный ответ говорит, что мы лишь не угадали с направлением ускорений.
    Сила же всегда положительна.
}
\solutionspace{80pt}

\tasknumber{7}%
\task{%
    Тело массой $2\,\text{кг}$ лежит на горизонтальной поверхности.
    Коэффициент трения между поверхностью и телом $0{,}25$.
    К телу приложена горизонтальная сила $3{,}5\,\text{Н}$.
    Определите силу трения, действующую на тело, и ускорение тела.
    % $g = 10\,\frac{\text{м}}{\text{с}^{2}}$.
}
\answer{%
    \begin{align*}
    &F_\text{ трения покоя $\max$ } = \mu N = \mu m g = 0{,}25 \cdot 2\,\text{кг} \cdot 10\,\frac{\text{м}}{\text{с}^{2}} = 5{,}00\,\text{Н}, \\
    &F_\text{ трения покоя $\max$ } > F \implies F_\text{ трения } = 3{,}50\,\text{Н}, a = \frac{F - F_\text{ трения }}{ m } = 0\,\frac{\text{м}}{\text{c}^{2}}, \\
    &\text{при равенстве возможны оба варианта: и едет, и не едет, но на ответы это не влияет.}
    \end{align*}
}
\solutionspace{120pt}

\tasknumber{8}%
\task{%
    Определите плотность неизвестного вещества, если известно, что опускании тела из него
    в керосин оно будет плавать и на четверть выступать над поверхностью жидкости.
}
\answer{%
    $F_\text{Арх.} = F_\text{тяж.} \implies \rho_\text{ж.} g V_\text{погр.} = m g \implies\rho_\text{ж.} g \cbr{V -\frac V4} = \rho V g \implies \rho = \rho_\text{ж.}\cbr{1 -\frac 14} \approx 600\,\frac{\text{кг}}{\text{м}^{3}}$
}
\solutionspace{120pt}

\tasknumber{9}%
\task{%
    	Определите силу, действующую на правую опору однородного горизонтального стержня длиной $l = 5\,\text{м}$
    	и массой $M = 5\,\text{кг}$, к которому подвешен груз массой $m = 3\,\text{кг}$ на расстоянии $4\,\text{м}$ от правого конца (см.
    рис.).

        \begin{tikzpicture}[thick]
            \draw
                (-2, -0.1) rectangle (2, 0.1)
                (-0.5, -0.1) -- (-0.5, -1)
                (-0.7, -1) rectangle (-0.3, -1.3)
           		(-2, -0.1) -- +(0.15,-0.9) -- +(-0.15,-0.9) -- cycle
            	(2, -0.1) -- +(0.15,-0.9) -- +(-0.15,-0.9) -- cycle
            ;
            \draw[pattern={Lines[angle=51,distance=2pt]},pattern color=black,draw=none]
            	(-2.15, -1.15) rectangle +(0.3, 0.15)
            	(2.15, -1.15) rectangle +(-0.3, 0.15)
            ;
            \node [right] (m_small) at (-0.3, -1.15) { $m$ };
            \node [above] (M_big) at (0, 0.1) { $M$ };
        \end{tikzpicture}
}
\answer{%
    \begin{align*}
        &\begin{cases}
            F_1 + F_2 - mg - Mg= 0, \\
            F_1 \cdot 0 - mg \cdot a - Mg \cdot \frac l2 + F_2 \cdot l = 0,
        \end{cases} \\
        F_2 &= \frac{mga + Mg\frac l2}l = \frac al \cdot mg + \frac{Mg}2 \approx 31{,}0\,\text{Н}, \\
        F_1 &= mg + Mg - F_2 = mg + Mg - \frac al \cdot mg - \frac{Mg}2 = \frac bl \cdot mg + \frac{Mg}2 \approx 49{,}0\,\text{Н}.
    \end{align*}
}
\solutionspace{80pt}

\tasknumber{10}%
\task{%
    Тонкий однородный кусок арматуры длиной $2\,\text{м}$ и массой $10\,\text{кг}$ лежит на горизонтальной поверхности.
    \begin{itemize}
        \item Какую минимальную силу надо приложить к одному из его концов, чтобы оторвать его от этой поверхности?
        \item Какую минимальную работу надо совершить, чтобы поставить его на землю в вертикальное положение?
    \end{itemize}
    % Примите $g = 10\,\frac{\text{м}}{\text{с}^{2}}$.
}
\answer{%
    $F = \frac{mg}2 \approx 100\,\text{Н}, A = mg\frac l2 = 100\,\text{Дж}$
}
\solutionspace{120pt}

\tasknumber{11}%
\task{%
    Определите работу силы, которая обеспечит подъём тела массой $3\,\text{кг}$ на высоту $10\,\text{м}$ с постоянным ускорением $3\,\frac{\text{м}}{\text{c}^{2}}$.
    % Примите $g = 10\,\frac{\text{м}}{\text{с}^{2}}$.
}
\answer{%
    \begin{align*}
    &\text{Для подъёма:} A = Fh = (mg + ma) h = m(g+a)h, \\
    &\text{Для спуска:} A = -Fh = -(mg - ma) h = -m(g-a)h, \\
    &\text{В результате получаем:} 390\,\text{Дж}.
    \end{align*}
}
\solutionspace{60pt}

\tasknumber{12}%
\task{%
    Тело бросили вертикально вверх со скоростью $20\,\frac{\text{м}}{\text{c}}$.
    На какой высоте кинетическая энергия тела составит половину от потенциальной?
}
\answer{%
    \begin{align*}
    &0 + \frac{mv_0^2}2 = E_p + E_k, E_k = \frac 12 E_p \implies \\
    &\implies \frac{mv_0^2}2 = E_p + \frac 12 E_p = E_p\cbr{1 + \frac 12} = mgh\cbr{1 + \frac 12} \implies \\
    &\implies h = \frac{\frac{mv_0^2}2}{mg\cbr{1 + \frac 12}} = \frac{v_0^2}{2g} \cdot \frac 1{1 + \frac 12} \approx 13{,}3\,\text{м}.
    \end{align*}
}
\solutionspace{100pt}

\tasknumber{13}%
\task{%
    Плотность воздуха при нормальных условиях равна $1{,}3\,\frac{\text{кг}}{\text{м}^{3}}$.
    Чему равна плотность воздуха
    при температуре $100\celsius$ и давлении $120\,\text{кПа}$?
}
\answer{%
    \begin{align*}
    &\text{В общем случае:} PV = \frac m{\mu} RT \implies \rho = \frac mV = \frac m{\frac{\frac m{\mu} RT}P} = \frac{P\mu}{RT}, \\
    &\text{У нас 2 состояния:} \rho_1 = \frac{P_1\mu}{RT_1}, \rho_2 = \frac{P_2\mu}{RT_2} \implies \frac{\rho_2}{\rho_1} = \frac{\frac{P_2\mu}{RT_2}}{\frac{P_1\mu}{RT_1}} = \frac{P_2T_1}{P_1T_2} \implies \\
    &\implies \rho_2 = \rho_1 \cdot  \frac{P_2T_1}{P_1T_2} = 1{,}3\,\frac{\text{кг}}{\text{м}^{3}} \cdot \frac{120\,\text{кПа} \cdot 273\units{К}}{100\,\text{кПа} \cdot 373\units{К}} \approx 1{,}14\,\frac{\text{кг}}{\text{м}^{3}}.
    \end{align*}
}
\solutionspace{120pt}

\tasknumber{14}%
\task{%
    Небольшую цилиндрическую пробирку с воздухом погружают на некоторую глубину в глубокое пресное озеро,
    после чего воздух занимает в ней лишь шестую часть от общего объема.
    Определите глубину, на которую погрузили пробирку.
    Температуру считать постоянной $T = 286\,\text{К}$, давлением паров воды пренебречь,
    атмосферное давление принять равным $p_{\text{aтм}} = 100\,\text{кПа}$.
}
\answer{%
    \begin{align*}
    T\text{— const} &\implies P_1V_1 = \nu RT = P_2V_2.
    \\
    V_2 = \frac 16 V_1 &\implies P_1V_1 = P_2 \cdot \frac 16V_1 \implies P_2 = 6P_1 = 6p_{\text{aтм}}.
    \\
    P_2 = p_{\text{aтм}} + \rho_{\text{в}} g h \implies h = \frac{P_2 - p_{\text{aтм}}}{\rho_{\text{в}} g} &= \frac{6p_{\text{aтм}} - p_{\text{aтм}}}{\rho_{\text{в}} g} = \frac{5 \cdot p_{\text{aтм}}}{\rho_{\text{в}} g} =  \\
     &= \frac{5 \cdot 100\,\text{кПа}}{1000\,\frac{\text{кг}}{\text{м}^{3}} \cdot  10\,\frac{\text{м}}{\text{с}^{2}}} \approx 50\,\text{м}.
    \end{align*}
}
\solutionspace{120pt}

\tasknumber{15}%
\task{%
    Газу сообщили некоторое количество теплоты,
    при этом половину его он потратил на совершение работы,
    одновременно увеличив свою внутреннюю энергию на $1500\,\text{Дж}$.
    Определите количество теплоты, сообщённое газу.
}
\answer{%
    \begin{align*}
    Q &= A' + \Delta U, A' = \frac 12 Q \implies Q \cdot \cbr{1 - \frac 12} = \Delta U \implies Q = \frac{\Delta U}{1 - \frac 12} = \frac{ 1500\,\text{Дж} }{1 - \frac 12} \approx 3000\,\text{Дж}.
    \\
    A' &= \frac 12 Q
        = \frac 12 \cdot \frac{\Delta U}{1 - \frac 12}
        = \frac{\Delta U}{2 - 1}
        = \frac{ 1500\,\text{Дж} }{2 - 1} \approx 1500\,\text{Дж}.
    \end{align*}
}
\solutionspace{60pt}

\tasknumber{16}%
\task{%
    Два конденсатора ёмкостей $C_1 = 30\,\text{нФ}$ и $C_2 = 20\,\text{нФ}$ последовательно подключают
    к источнику напряжения $U = 300\,\text{В}$ (см.
    рис.).
    % Определите заряды каждого из конденсаторов.
    Определите заряд второго конденсатора.

    \begin{tikzpicture}[circuit ee IEC, semithick]
        \draw  (0, 0) to [capacitor={info={$C_1$}}] (1, 0)
                       to [capacitor={info={$C_2$}}] (2, 0)
        ;
        % \draw [-o] (0, 0) -- ++(-0.5, 0) node[left] {$-$};
        % \draw [-o] (2, 0) -- ++(0.5, 0) node[right] {$+$};
        \draw [-o] (0, 0) -- ++(-0.5, 0) node[left] {};
        \draw [-o] (2, 0) -- ++(0.5, 0) node[right] {};
    \end{tikzpicture}
}
\answer{%
    $
        Q_1
            = Q_2
            = CU
            = \frac{ U }{\frac1{C_1} + \frac1{C_2}}
            = \frac{C_1C_2U}{C_1 + C_2}
            = \frac{
                30\,\text{нФ} \cdot 20\,\text{нФ} \cdot 300\,\text{В}
            }{
                30\,\text{нФ} + 20\,\text{нФ}
            }
            = 3{,}60\,\text{мкКл}
    $
}
\solutionspace{120pt}

\tasknumber{17}%
\task{%
    В вакууме вдоль одной прямой расположены четыре положительных заряда так,
    что расстояние между соседними зарядами равно $a$.
    Сделайте рисунок,
    и определите силу, действующую на крайний заряд.
    Модули всех зарядов равны $Q$ ($Q > 0$).
}
\answer{%
    $F = \sum_i F_i = \ldots = \frac{49}{36} \frac{kQ^2}{a^2}.$
}
\solutionspace{80pt}

\tasknumber{18}%
\task{%
    Юлия проводит эксперименты c 2 кусками одинаковой медной проволки, причём второй кусок в два раза длиннее первого.
    В одном из экспериментов Юлия подаёт на первый кусок проволки напряжение в девять раз раз больше, чем на второй.
    Определите отношения в двух проволках в этом эксперименте (второй к первой):
    \begin{itemize}
        \item отношение сил тока,
        \item отношение выделяющихся мощностей.
    \end{itemize}
}
\answer{%
    $R_2 = 2R_1, U_1 = 9U_2 \implies  \eli_2 / \eli_1 = \frac{U_2 / R_2}{U_1 / R_1} = \frac{U_2}{U_1} \cdot \frac{R_1}{R_2} = \frac1{18}, P_2 / P_1 = \frac{U_2^2 / R_2}{U_1^2 / R_1} = \sqr{\frac{U_2}{U_1}} \cdot \frac{R_1}{R_2} = \frac1{162}.$
}

\variantsplitter

\addpersonalvariant{Алёна Куприянова}

\tasknumber{1}%
\task{%
    Валя стартует на мотоцикле и в течение $t = 10\,\text{c}$ двигается с постоянным ускорением $1{,}5\,\frac{\text{м}}{\text{с}^{2}}$.
    Определите
    \begin{itemize}
        \item какую скорость при этом удастся достичь,
        \item какой путь за это время будет пройден,
        \item среднюю скорость за всё время движения, если после начального ускорения продолжить движение равномерно ещё в течение времени $2t$
    \end{itemize}
}
\answer{%
    \begin{align*}
    v &= v_0 + a t = at = 1{,}5\,\frac{\text{м}}{\text{с}^{2}} \cdot 10\,\text{c} = 15{,}0\,\frac{\text{м}}{\text{с}}, \\
    s_x &= v_0t + \frac{a t^2}2 = \frac{a t^2}2 = \frac{1{,}5\,\frac{\text{м}}{\text{с}^{2}} \cdot \sqr{ 10\,\text{c} }}2 = 75{,}0\,\text{м}, \\
    v_\text{сред.} &= \frac{s_\text{общ}}{t_\text{общ.}} = \frac{s_x + v \cdot 2t}{t + 2t} = \frac{\frac{a t^2}2 + at \cdot 2t}{t (1 + 2)} = \\
    &= at \cdot \frac{\frac 12 + 2}{1 + 2} = 1{,}5\,\frac{\text{м}}{\text{с}^{2}} \cdot 10\,\text{c} \cdot \frac{\frac 12 + 2}{1 + 2} \approx 12{,}50\,\frac{\text{м}}{\text{c}}.
    \end{align*}
}
\solutionspace{120pt}

\tasknumber{2}%
\task{%
    Какой путь тело пройдёт за вторую секунду после начала свободного падения?
    Какую скорость в конце этой секунды оно имеет?
}
\answer{%
    \begin{align*}
    s &= -s_y = -(y_2-y_1) = y_1 - y_2 = \cbr{y_{0y} + v_{0y}t_1 - \frac{gt_1^2}2} - \cbr{y_{0y} + v_{0y}t_2 - \frac{gt_2^2}2} = \\
    &= \frac{gt_2^2}2 - \frac{gt_1^2}2 = \frac g2\cbr{t_2^2 - t_1^2} = 15{,}0\,\text{м}, \\
    v_y &= v_{0y} - gt = -gt = 10\,\frac{\text{м}}{\text{с}^{2}} \cdot 2\,\text{с} = -20\,\frac{\text{м}}{\text{с}}.
    \end{align*}
}
\solutionspace{120pt}

\tasknumber{3}%
\task{%
    Карусель диаметром $3\,\text{м}$ равномерно совершает 5 оборотов в минуту.
    Определите
    \begin{itemize}
        \item период и частоту её обращения,
        \item скорость и ускорение крайних её точек.
    \end{itemize}
}
\answer{%
    \begin{align*}
    t &= 60\,\text{с}, r = 1{,}5\,\text{м}, n = 5\units{оборотов}, \\
    T &= \frac tN = \frac{ 60\,\text{с} }{5} \approx 12{,}00\,\text{c}, \\
    \nu &= \frac 1T = \frac{5}{ 60\,\text{с} } \approx 0{,}08\,\text{Гц}, \\
    v &= \frac{2 \pi r}{T} = \frac{2 \pi r}{T} =  \frac{2 \pi r n}{t} \approx 0{,}79\,\frac{\text{м}}{\text{c}}, \\
    a &= \frac{v^2}{r} =  \frac{4 \pi^2 r n^2}{t^2} \approx 0{,}41\,\frac{\text{м}}{\text{с}^{2}}.
    \end{align*}
}
\solutionspace{80pt}

\tasknumber{4}%
\task{%
    Даша стоит на обрыве над рекой и методично и строго горизонтально кидает в неё камушки.
    За этим всем наблюдает экспериментатор Глюк, который уже выяснил, что камушки падают в реку спустя $1{,}2\,\text{с}$ после броска,
    а вот дальность полёта оценить сложнее: придётся лезть в воду.
    Выручите Глюка и определите:
    \begin{itemize}
        \item высоту обрыва (вместе с ростом Даши).
        \item дальность полёта камушков (по горизонтали) и их скорость при падении, приняв начальную скорость броска равной $v_0 = 12\,\frac{\text{м}}{\text{с}}$.
    \end{itemize}
    Сопротивлением воздуха пренебречь.
}
\answer{%
    \begin{align*}
    y &= y_0 + v_{0y}t - \frac{gt^2}2 = h - \frac{gt^2}2, \qquad y(\tau) = 0 \implies h - \frac{g\tau^2}2 = 0 \implies h = \frac{g\tau^2}2 \approx 7{,}2\,\text{м}.
    \\
    x &= x_0 + v_{0x}t = v_0t \implies L = v_0\tau \approx 14{,}4\,\text{м}.
    \\
    &v = \sqrt{v_x^2 + v_y^2} = \sqrt{v_{0x}^2 + \sqr{v_{0y} - g\tau}} = \sqrt{v_0^2 + \sqr{g\tau}} \approx 17{,}0\,\frac{\text{м}}{\text{c}}.
    \end{align*}
}
\solutionspace{120pt}

\tasknumber{5}%
\task{%
    Четыре одинаковых брусков массой $2\,\text{кг}$ каждый лежат на гладком горизонтальном столе.
    Бруски пронумерованы от 1 до 4 и последовательно связаны между собой
    невесомыми нерастяжимыми нитями: 1 со 2, 2 с 3 (ну и с 1) и т.д.
    Экспериментатор Глюк прикладывает постоянную горизонтальную силу $90\,\text{Н}$ к бруску с наибольшим номером.
    С каким ускорением двигается система? Чему равна сила натяжения нити, связывающей бруски 2 и 3?
}
\answer{%
    \begin{align*}
    a &= \frac{F}{4 m} = \frac{90\,\text{Н}}{4 \cdot 2\,\text{кг}} \approx 11{,}2\,\frac{\text{м}}{\text{c}^{2}}, \\
    T &= m'a = 2m \cdot \frac{F}{4 m} = \frac{2}{4} F \approx 45{,}0\,\text{Н}.
    \end{align*}
}
\solutionspace{120pt}

\tasknumber{6}%
\task{%
    Два бруска связаны лёгкой нерастяжимой нитью и перекинуты через неподвижный блок (см.
    рис.).
    Определите силу натяжения нити и ускорения брусков.
    Силами трения пренебречь, массы брусков
    равны $m_1 = 11\,\text{кг}$ и $m_2 = 6\,\text{кг}$.
    % $g = 10\,\frac{\text{м}}{\text{с}^{2}}$.

    \begin{tikzpicture}[x=1.5cm,y=1.5cm,thick]
        \draw
            (-0.4, 0) rectangle (-0.2, 1.2)
            (0.15, 0.5) rectangle (0.45, 1)
            (0, 2) circle [radius=0.3] -- ++(up:0.5)
            (-0.3, 1.2) -- ++(up:0.8)
            (0.3, 1) -- ++(up:1)
            (-0.7, 2.5) -- (0.7, 2.5)
            ;
        \draw[pattern={Lines[angle=51,distance=3pt]},pattern color=black,draw=none] (-0.7, 2.5) rectangle (0.7, 2.75);
        \node [left] (left) at (-0.4, 0.6) { $m_1$ };
        \node [right] (right) at (0.4, 0.75) { $m_2$ };
    \end{tikzpicture}
}
\answer{%
    Предположим, что левый брусок ускоряется вверх, тогда правый ускоряется вниз (с тем же ускорением).
    Запишем 2-й закон Ньютона 2 раза (для обоих тел) в проекции на вертикальную оси, направив её вверх.
    \begin{align*}
        &\begin{cases}
            T - m_1g = m_1a, \\
            T - m_2g = -m_2a,
        \end{cases} \\
        &\begin{cases}
            m_2g - m_1g = m_1a + m_2a, \\
            T = m_1a + m_1g, \\
        \end{cases} \\
        a &= \frac{m_2 - m_1}{m_1 + m_2} \cdot g = \frac{6\,\text{кг} - 11\,\text{кг}}{11\,\text{кг} + 6\,\text{кг}} \cdot 10\,\frac{\text{м}}{\text{с}^{2}} \approx -2{,}940\,\frac{\text{м}}{\text{c}^{2}}, \\
        T &= m_1(a + g) = m_1 \cdot g \cdot \cbr{\frac{m_2 - m_1}{m_1 + m_2} + 1} = m_1 \cdot g \cdot \frac{2m_2}{m_1 + m_2} = \\
            &= \frac{2 m_2 m_1 g}{m_1 + m_2} = \frac{2 \cdot 6\,\text{кг} \cdot 11\,\text{кг} \cdot 10\,\frac{\text{м}}{\text{с}^{2}}}{11\,\text{кг} + 6\,\text{кг}} \approx 77{,}6\,\text{Н}.
    \end{align*}
    Отрицательный ответ говорит, что мы лишь не угадали с направлением ускорений.
    Сила же всегда положительна.
}
\solutionspace{80pt}

\tasknumber{7}%
\task{%
    Тело массой $2{,}7\,\text{кг}$ лежит на горизонтальной поверхности.
    Коэффициент трения между поверхностью и телом $0{,}15$.
    К телу приложена горизонтальная сила $4{,}5\,\text{Н}$.
    Определите силу трения, действующую на тело, и ускорение тела.
    % $g = 10\,\frac{\text{м}}{\text{с}^{2}}$.
}
\answer{%
    \begin{align*}
    &F_\text{ трения покоя $\max$ } = \mu N = \mu m g = 0{,}15 \cdot 2{,}7\,\text{кг} \cdot 10\,\frac{\text{м}}{\text{с}^{2}} = 4{,}05\,\text{Н}, \\
    &F_\text{ трения покоя $\max$ } \le F \implies F_\text{ трения } = 4{,}05\,\text{Н}, a = \frac{F - F_\text{ трения }}{ m } = 0{,}17\,\frac{\text{м}}{\text{c}^{2}}, \\
    &\text{при равенстве возможны оба варианта: и едет, и не едет, но на ответы это не влияет.}
    \end{align*}
}
\solutionspace{120pt}

\tasknumber{8}%
\task{%
    Определите плотность неизвестного вещества, если известно, что опускании тела из него
    в керосин оно будет плавать и на половину выступать над поверхностью жидкости.
}
\answer{%
    $F_\text{Арх.} = F_\text{тяж.} \implies \rho_\text{ж.} g V_\text{погр.} = m g \implies\rho_\text{ж.} g \cbr{V -\frac V2} = \rho V g \implies \rho = \rho_\text{ж.}\cbr{1 -\frac 12} \approx 400\,\frac{\text{кг}}{\text{м}^{3}}$
}
\solutionspace{120pt}

\tasknumber{9}%
\task{%
    	Определите силу, действующую на правую опору однородного горизонтального стержня длиной $l = 9\,\text{м}$
    	и массой $M = 1\,\text{кг}$, к которому подвешен груз массой $m = 2\,\text{кг}$ на расстоянии $4\,\text{м}$ от правого конца (см.
    рис.).

        \begin{tikzpicture}[thick]
            \draw
                (-2, -0.1) rectangle (2, 0.1)
                (-0.5, -0.1) -- (-0.5, -1)
                (-0.7, -1) rectangle (-0.3, -1.3)
           		(-2, -0.1) -- +(0.15,-0.9) -- +(-0.15,-0.9) -- cycle
            	(2, -0.1) -- +(0.15,-0.9) -- +(-0.15,-0.9) -- cycle
            ;
            \draw[pattern={Lines[angle=51,distance=2pt]},pattern color=black,draw=none]
            	(-2.15, -1.15) rectangle +(0.3, 0.15)
            	(2.15, -1.15) rectangle +(-0.3, 0.15)
            ;
            \node [right] (m_small) at (-0.3, -1.15) { $m$ };
            \node [above] (M_big) at (0, 0.1) { $M$ };
        \end{tikzpicture}
}
\answer{%
    \begin{align*}
        &\begin{cases}
            F_1 + F_2 - mg - Mg= 0, \\
            F_1 \cdot 0 - mg \cdot a - Mg \cdot \frac l2 + F_2 \cdot l = 0,
        \end{cases} \\
        F_2 &= \frac{mga + Mg\frac l2}l = \frac al \cdot mg + \frac{Mg}2 \approx 16{,}1\,\text{Н}, \\
        F_1 &= mg + Mg - F_2 = mg + Mg - \frac al \cdot mg - \frac{Mg}2 = \frac bl \cdot mg + \frac{Mg}2 \approx 13{,}9\,\text{Н}.
    \end{align*}
}
\solutionspace{80pt}

\tasknumber{10}%
\task{%
    Тонкий однородный кусок арматуры длиной $3\,\text{м}$ и массой $30\,\text{кг}$ лежит на горизонтальной поверхности.
    \begin{itemize}
        \item Какую минимальную силу надо приложить к одному из его концов, чтобы оторвать его от этой поверхности?
        \item Какую минимальную работу надо совершить, чтобы поставить его на землю в вертикальное положение?
    \end{itemize}
    % Примите $g = 10\,\frac{\text{м}}{\text{с}^{2}}$.
}
\answer{%
    $F = \frac{mg}2 \approx 300\,\text{Н}, A = mg\frac l2 = 450\,\text{Дж}$
}
\solutionspace{120pt}

\tasknumber{11}%
\task{%
    Определите работу силы, которая обеспечит спуск тела массой $3\,\text{кг}$ на высоту $5\,\text{м}$ с постоянным ускорением $2\,\frac{\text{м}}{\text{c}^{2}}$.
    % Примите $g = 10\,\frac{\text{м}}{\text{с}^{2}}$.
}
\answer{%
    \begin{align*}
    &\text{Для подъёма:} A = Fh = (mg + ma) h = m(g+a)h, \\
    &\text{Для спуска:} A = -Fh = -(mg - ma) h = -m(g-a)h, \\
    &\text{В результате получаем:} -120\,\text{Дж}.
    \end{align*}
}
\solutionspace{60pt}

\tasknumber{12}%
\task{%
    Тело бросили вертикально вверх со скоростью $14\,\frac{\text{м}}{\text{c}}$.
    На какой высоте кинетическая энергия тела составит треть от потенциальной?
}
\answer{%
    \begin{align*}
    &0 + \frac{mv_0^2}2 = E_p + E_k, E_k = \frac 13 E_p \implies \\
    &\implies \frac{mv_0^2}2 = E_p + \frac 13 E_p = E_p\cbr{1 + \frac 13} = mgh\cbr{1 + \frac 13} \implies \\
    &\implies h = \frac{\frac{mv_0^2}2}{mg\cbr{1 + \frac 13}} = \frac{v_0^2}{2g} \cdot \frac 1{1 + \frac 13} \approx 7{,}4\,\text{м}.
    \end{align*}
}
\solutionspace{100pt}

\tasknumber{13}%
\task{%
    Плотность воздуха при нормальных условиях равна $1{,}3\,\frac{\text{кг}}{\text{м}^{3}}$.
    Чему равна плотность воздуха
    при температуре $100\celsius$ и давлении $150\,\text{кПа}$?
}
\answer{%
    \begin{align*}
    &\text{В общем случае:} PV = \frac m{\mu} RT \implies \rho = \frac mV = \frac m{\frac{\frac m{\mu} RT}P} = \frac{P\mu}{RT}, \\
    &\text{У нас 2 состояния:} \rho_1 = \frac{P_1\mu}{RT_1}, \rho_2 = \frac{P_2\mu}{RT_2} \implies \frac{\rho_2}{\rho_1} = \frac{\frac{P_2\mu}{RT_2}}{\frac{P_1\mu}{RT_1}} = \frac{P_2T_1}{P_1T_2} \implies \\
    &\implies \rho_2 = \rho_1 \cdot  \frac{P_2T_1}{P_1T_2} = 1{,}3\,\frac{\text{кг}}{\text{м}^{3}} \cdot \frac{150\,\text{кПа} \cdot 273\units{К}}{100\,\text{кПа} \cdot 373\units{К}} \approx 1{,}43\,\frac{\text{кг}}{\text{м}^{3}}.
    \end{align*}
}
\solutionspace{120pt}

\tasknumber{14}%
\task{%
    Небольшую цилиндрическую пробирку с воздухом погружают на некоторую глубину в глубокое пресное озеро,
    после чего воздух занимает в ней лишь третью часть от общего объема.
    Определите глубину, на которую погрузили пробирку.
    Температуру считать постоянной $T = 289\,\text{К}$, давлением паров воды пренебречь,
    атмосферное давление принять равным $p_{\text{aтм}} = 100\,\text{кПа}$.
}
\answer{%
    \begin{align*}
    T\text{— const} &\implies P_1V_1 = \nu RT = P_2V_2.
    \\
    V_2 = \frac 13 V_1 &\implies P_1V_1 = P_2 \cdot \frac 13V_1 \implies P_2 = 3P_1 = 3p_{\text{aтм}}.
    \\
    P_2 = p_{\text{aтм}} + \rho_{\text{в}} g h \implies h = \frac{P_2 - p_{\text{aтм}}}{\rho_{\text{в}} g} &= \frac{3p_{\text{aтм}} - p_{\text{aтм}}}{\rho_{\text{в}} g} = \frac{2 \cdot p_{\text{aтм}}}{\rho_{\text{в}} g} =  \\
     &= \frac{2 \cdot 100\,\text{кПа}}{1000\,\frac{\text{кг}}{\text{м}^{3}} \cdot  10\,\frac{\text{м}}{\text{с}^{2}}} \approx 20\,\text{м}.
    \end{align*}
}
\solutionspace{120pt}

\tasknumber{15}%
\task{%
    Газу сообщили некоторое количество теплоты,
    при этом четверть его он потратил на совершение работы,
    одновременно увеличив свою внутреннюю энергию на $1200\,\text{Дж}$.
    Определите работу, совершённую газом.
}
\answer{%
    \begin{align*}
    Q &= A' + \Delta U, A' = \frac 14 Q \implies Q \cdot \cbr{1 - \frac 14} = \Delta U \implies Q = \frac{\Delta U}{1 - \frac 14} = \frac{ 1200\,\text{Дж} }{1 - \frac 14} \approx 1600\,\text{Дж}.
    \\
    A' &= \frac 14 Q
        = \frac 14 \cdot \frac{\Delta U}{1 - \frac 14}
        = \frac{\Delta U}{4 - 1}
        = \frac{ 1200\,\text{Дж} }{4 - 1} \approx 400\,\text{Дж}.
    \end{align*}
}
\solutionspace{60pt}

\tasknumber{16}%
\task{%
    Два конденсатора ёмкостей $C_1 = 20\,\text{нФ}$ и $C_2 = 40\,\text{нФ}$ последовательно подключают
    к источнику напряжения $U = 200\,\text{В}$ (см.
    рис.).
    % Определите заряды каждого из конденсаторов.
    Определите заряд второго конденсатора.

    \begin{tikzpicture}[circuit ee IEC, semithick]
        \draw  (0, 0) to [capacitor={info={$C_1$}}] (1, 0)
                       to [capacitor={info={$C_2$}}] (2, 0)
        ;
        % \draw [-o] (0, 0) -- ++(-0.5, 0) node[left] {$-$};
        % \draw [-o] (2, 0) -- ++(0.5, 0) node[right] {$+$};
        \draw [-o] (0, 0) -- ++(-0.5, 0) node[left] {};
        \draw [-o] (2, 0) -- ++(0.5, 0) node[right] {};
    \end{tikzpicture}
}
\answer{%
    $
        Q_1
            = Q_2
            = CU
            = \frac{ U }{\frac1{C_1} + \frac1{C_2}}
            = \frac{C_1C_2U}{C_1 + C_2}
            = \frac{
                20\,\text{нФ} \cdot 40\,\text{нФ} \cdot 200\,\text{В}
            }{
                20\,\text{нФ} + 40\,\text{нФ}
            }
            = 2{,}67\,\text{мкКл}
    $
}
\solutionspace{120pt}

\tasknumber{17}%
\task{%
    В вакууме вдоль одной прямой расположены четыре положительных заряда так,
    что расстояние между соседними зарядами равно $a$.
    Сделайте рисунок,
    и определите силу, действующую на крайний заряд.
    Модули всех зарядов равны $Q$ ($Q > 0$).
}
\answer{%
    $F = \sum_i F_i = \ldots = \frac{49}{36} \frac{kQ^2}{a^2}.$
}
\solutionspace{80pt}

\tasknumber{18}%
\task{%
    Юлия проводит эксперименты c 2 кусками одинаковой медной проволки, причём второй кусок в девять раз длиннее первого.
    В одном из экспериментов Юлия подаёт на первый кусок проволки напряжение в два раза раз больше, чем на второй.
    Определите отношения в двух проволках в этом эксперименте (второй к первой):
    \begin{itemize}
        \item отношение сил тока,
        \item отношение выделяющихся мощностей.
    \end{itemize}
}
\answer{%
    $R_2 = 9R_1, U_1 = 2U_2 \implies  \eli_2 / \eli_1 = \frac{U_2 / R_2}{U_1 / R_1} = \frac{U_2}{U_1} \cdot \frac{R_1}{R_2} = \frac1{18}, P_2 / P_1 = \frac{U_2^2 / R_2}{U_1^2 / R_1} = \sqr{\frac{U_2}{U_1}} \cdot \frac{R_1}{R_2} = \frac1{36}.$
}

\variantsplitter

\addpersonalvariant{Ярослав Лавровский}

\tasknumber{1}%
\task{%
    Валя стартует на мотоцикле и в течение $t = 10\,\text{c}$ двигается с постоянным ускорением $2\,\frac{\text{м}}{\text{с}^{2}}$.
    Определите
    \begin{itemize}
        \item какую скорость при этом удастся достичь,
        \item какой путь за это время будет пройден,
        \item среднюю скорость за всё время движения, если после начального ускорения продолжить движение равномерно ещё в течение времени $3t$
    \end{itemize}
}
\answer{%
    \begin{align*}
    v &= v_0 + a t = at = 2\,\frac{\text{м}}{\text{с}^{2}} \cdot 10\,\text{c} = 20{,}0\,\frac{\text{м}}{\text{с}}, \\
    s_x &= v_0t + \frac{a t^2}2 = \frac{a t^2}2 = \frac{2\,\frac{\text{м}}{\text{с}^{2}} \cdot \sqr{ 10\,\text{c} }}2 = 100{,}0\,\text{м}, \\
    v_\text{сред.} &= \frac{s_\text{общ}}{t_\text{общ.}} = \frac{s_x + v \cdot 3t}{t + 3t} = \frac{\frac{a t^2}2 + at \cdot 3t}{t (1 + 3)} = \\
    &= at \cdot \frac{\frac 12 + 3}{1 + 3} = 2\,\frac{\text{м}}{\text{с}^{2}} \cdot 10\,\text{c} \cdot \frac{\frac 12 + 3}{1 + 3} \approx 17{,}50\,\frac{\text{м}}{\text{c}}.
    \end{align*}
}
\solutionspace{120pt}

\tasknumber{2}%
\task{%
    Какой путь тело пройдёт за третью секунду после начала свободного падения?
    Какую скорость в начале этой секунды оно имеет?
}
\answer{%
    \begin{align*}
    s &= -s_y = -(y_2-y_1) = y_1 - y_2 = \cbr{y_{0y} + v_{0y}t_1 - \frac{gt_1^2}2} - \cbr{y_{0y} + v_{0y}t_2 - \frac{gt_2^2}2} = \\
    &= \frac{gt_2^2}2 - \frac{gt_1^2}2 = \frac g2\cbr{t_2^2 - t_1^2} = 25{,}0\,\text{м}, \\
    v_y &= v_{0y} - gt = -gt = 10\,\frac{\text{м}}{\text{с}^{2}} \cdot 2\,\text{с} = -20\,\frac{\text{м}}{\text{с}}.
    \end{align*}
}
\solutionspace{120pt}

\tasknumber{3}%
\task{%
    Карусель диаметром $5\,\text{м}$ равномерно совершает 10 оборотов в минуту.
    Определите
    \begin{itemize}
        \item период и частоту её обращения,
        \item скорость и ускорение крайних её точек.
    \end{itemize}
}
\answer{%
    \begin{align*}
    t &= 60\,\text{с}, r = 2{,}5\,\text{м}, n = 10\units{оборотов}, \\
    T &= \frac tN = \frac{ 60\,\text{с} }{10} \approx 6{,}00\,\text{c}, \\
    \nu &= \frac 1T = \frac{10}{ 60\,\text{с} } \approx 0{,}17\,\text{Гц}, \\
    v &= \frac{2 \pi r}{T} = \frac{2 \pi r}{T} =  \frac{2 \pi r n}{t} \approx 2{,}62\,\frac{\text{м}}{\text{c}}, \\
    a &= \frac{v^2}{r} =  \frac{4 \pi^2 r n^2}{t^2} \approx 2{,}74\,\frac{\text{м}}{\text{с}^{2}}.
    \end{align*}
}
\solutionspace{80pt}

\tasknumber{4}%
\task{%
    Миша стоит на обрыве над рекой и методично и строго горизонтально кидает в неё камушки.
    За этим всем наблюдает экспериментатор Глюк, который уже выяснил, что камушки падают в реку спустя $1{,}7\,\text{с}$ после броска,
    а вот дальность полёта оценить сложнее: придётся лезть в воду.
    Выручите Глюка и определите:
    \begin{itemize}
        \item высоту обрыва (вместе с ростом Миши).
        \item дальность полёта камушков (по горизонтали) и их скорость при падении, приняв начальную скорость броска равной $v_0 = 17\,\frac{\text{м}}{\text{с}}$.
    \end{itemize}
    Сопротивлением воздуха пренебречь.
}
\answer{%
    \begin{align*}
    y &= y_0 + v_{0y}t - \frac{gt^2}2 = h - \frac{gt^2}2, \qquad y(\tau) = 0 \implies h - \frac{g\tau^2}2 = 0 \implies h = \frac{g\tau^2}2 \approx 14{,}4\,\text{м}.
    \\
    x &= x_0 + v_{0x}t = v_0t \implies L = v_0\tau \approx 28{,}9\,\text{м}.
    \\
    &v = \sqrt{v_x^2 + v_y^2} = \sqrt{v_{0x}^2 + \sqr{v_{0y} - g\tau}} = \sqrt{v_0^2 + \sqr{g\tau}} \approx 24{,}0\,\frac{\text{м}}{\text{c}}.
    \end{align*}
}
\solutionspace{120pt}

\tasknumber{5}%
\task{%
    Пять одинаковых брусков массой $3\,\text{кг}$ каждый лежат на гладком горизонтальном столе.
    Бруски пронумерованы от 1 до 5 и последовательно связаны между собой
    невесомыми нерастяжимыми нитями: 1 со 2, 2 с 3 (ну и с 1) и т.д.
    Экспериментатор Глюк прикладывает постоянную горизонтальную силу $90\,\text{Н}$ к бруску с наименьшим номером.
    С каким ускорением двигается система? Чему равна сила натяжения нити, связывающей бруски 2 и 3?
}
\answer{%
    \begin{align*}
    a &= \frac{F}{5 m} = \frac{90\,\text{Н}}{5 \cdot 3\,\text{кг}} \approx 6{,}0\,\frac{\text{м}}{\text{c}^{2}}, \\
    T &= m'a = 3m \cdot \frac{F}{5 m} = \frac{3}{5} F \approx 54{,}0\,\text{Н}.
    \end{align*}
}
\solutionspace{120pt}

\tasknumber{6}%
\task{%
    Два бруска связаны лёгкой нерастяжимой нитью и перекинуты через неподвижный блок (см.
    рис.).
    Определите силу натяжения нити и ускорения брусков.
    Силами трения пренебречь, массы брусков
    равны $m_1 = 5\,\text{кг}$ и $m_2 = 6\,\text{кг}$.
    % $g = 10\,\frac{\text{м}}{\text{с}^{2}}$.

    \begin{tikzpicture}[x=1.5cm,y=1.5cm,thick]
        \draw
            (-0.4, 0) rectangle (-0.2, 1.2)
            (0.15, 0.5) rectangle (0.45, 1)
            (0, 2) circle [radius=0.3] -- ++(up:0.5)
            (-0.3, 1.2) -- ++(up:0.8)
            (0.3, 1) -- ++(up:1)
            (-0.7, 2.5) -- (0.7, 2.5)
            ;
        \draw[pattern={Lines[angle=51,distance=3pt]},pattern color=black,draw=none] (-0.7, 2.5) rectangle (0.7, 2.75);
        \node [left] (left) at (-0.4, 0.6) { $m_1$ };
        \node [right] (right) at (0.4, 0.75) { $m_2$ };
    \end{tikzpicture}
}
\answer{%
    Предположим, что левый брусок ускоряется вверх, тогда правый ускоряется вниз (с тем же ускорением).
    Запишем 2-й закон Ньютона 2 раза (для обоих тел) в проекции на вертикальную оси, направив её вверх.
    \begin{align*}
        &\begin{cases}
            T - m_1g = m_1a, \\
            T - m_2g = -m_2a,
        \end{cases} \\
        &\begin{cases}
            m_2g - m_1g = m_1a + m_2a, \\
            T = m_1a + m_1g, \\
        \end{cases} \\
        a &= \frac{m_2 - m_1}{m_1 + m_2} \cdot g = \frac{6\,\text{кг} - 5\,\text{кг}}{5\,\text{кг} + 6\,\text{кг}} \cdot 10\,\frac{\text{м}}{\text{с}^{2}} \approx 0{,}91\,\frac{\text{м}}{\text{c}^{2}}, \\
        T &= m_1(a + g) = m_1 \cdot g \cdot \cbr{\frac{m_2 - m_1}{m_1 + m_2} + 1} = m_1 \cdot g \cdot \frac{2m_2}{m_1 + m_2} = \\
            &= \frac{2 m_2 m_1 g}{m_1 + m_2} = \frac{2 \cdot 6\,\text{кг} \cdot 5\,\text{кг} \cdot 10\,\frac{\text{м}}{\text{с}^{2}}}{5\,\text{кг} + 6\,\text{кг}} \approx 54{,}5\,\text{Н}.
    \end{align*}
    Отрицательный ответ говорит, что мы лишь не угадали с направлением ускорений.
    Сила же всегда положительна.
}
\solutionspace{80pt}

\tasknumber{7}%
\task{%
    Тело массой $1{,}4\,\text{кг}$ лежит на горизонтальной поверхности.
    Коэффициент трения между поверхностью и телом $0{,}15$.
    К телу приложена горизонтальная сила $5{,}5\,\text{Н}$.
    Определите силу трения, действующую на тело, и ускорение тела.
    % $g = 10\,\frac{\text{м}}{\text{с}^{2}}$.
}
\answer{%
    \begin{align*}
    &F_\text{ трения покоя $\max$ } = \mu N = \mu m g = 0{,}15 \cdot 1{,}4\,\text{кг} \cdot 10\,\frac{\text{м}}{\text{с}^{2}} = 2{,}10\,\text{Н}, \\
    &F_\text{ трения покоя $\max$ } \le F \implies F_\text{ трения } = 2{,}10\,\text{Н}, a = \frac{F - F_\text{ трения }}{ m } = 2{,}43\,\frac{\text{м}}{\text{c}^{2}}, \\
    &\text{при равенстве возможны оба варианта: и едет, и не едет, но на ответы это не влияет.}
    \end{align*}
}
\solutionspace{120pt}

\tasknumber{8}%
\task{%
    Определите плотность неизвестного вещества, если известно, что опускании тела из него
    в керосин оно будет плавать и на половину выступать над поверхностью жидкости.
}
\answer{%
    $F_\text{Арх.} = F_\text{тяж.} \implies \rho_\text{ж.} g V_\text{погр.} = m g \implies\rho_\text{ж.} g \cbr{V -\frac V2} = \rho V g \implies \rho = \rho_\text{ж.}\cbr{1 -\frac 12} \approx 400\,\frac{\text{кг}}{\text{м}^{3}}$
}
\solutionspace{120pt}

\tasknumber{9}%
\task{%
    	Определите силу, действующую на левую опору однородного горизонтального стержня длиной $l = 5\,\text{м}$
    	и массой $M = 1\,\text{кг}$, к которому подвешен груз массой $m = 3\,\text{кг}$ на расстоянии $2\,\text{м}$ от правого конца (см.
    рис.).

        \begin{tikzpicture}[thick]
            \draw
                (-2, -0.1) rectangle (2, 0.1)
                (-0.5, -0.1) -- (-0.5, -1)
                (-0.7, -1) rectangle (-0.3, -1.3)
           		(-2, -0.1) -- +(0.15,-0.9) -- +(-0.15,-0.9) -- cycle
            	(2, -0.1) -- +(0.15,-0.9) -- +(-0.15,-0.9) -- cycle
            ;
            \draw[pattern={Lines[angle=51,distance=2pt]},pattern color=black,draw=none]
            	(-2.15, -1.15) rectangle +(0.3, 0.15)
            	(2.15, -1.15) rectangle +(-0.3, 0.15)
            ;
            \node [right] (m_small) at (-0.3, -1.15) { $m$ };
            \node [above] (M_big) at (0, 0.1) { $M$ };
        \end{tikzpicture}
}
\answer{%
    \begin{align*}
        &\begin{cases}
            F_1 + F_2 - mg - Mg= 0, \\
            F_1 \cdot 0 - mg \cdot a - Mg \cdot \frac l2 + F_2 \cdot l = 0,
        \end{cases} \\
        F_2 &= \frac{mga + Mg\frac l2}l = \frac al \cdot mg + \frac{Mg}2 \approx 23{,}0\,\text{Н}, \\
        F_1 &= mg + Mg - F_2 = mg + Mg - \frac al \cdot mg - \frac{Mg}2 = \frac bl \cdot mg + \frac{Mg}2 \approx 17{,}0\,\text{Н}.
    \end{align*}
}
\solutionspace{80pt}

\tasknumber{10}%
\task{%
    Тонкий однородный лом длиной $3\,\text{м}$ и массой $30\,\text{кг}$ лежит на горизонтальной поверхности.
    \begin{itemize}
        \item Какую минимальную силу надо приложить к одному из его концов, чтобы оторвать его от этой поверхности?
        \item Какую минимальную работу надо совершить, чтобы поставить его на землю в вертикальное положение?
    \end{itemize}
    % Примите $g = 10\,\frac{\text{м}}{\text{с}^{2}}$.
}
\answer{%
    $F = \frac{mg}2 \approx 300\,\text{Н}, A = mg\frac l2 = 450\,\text{Дж}$
}
\solutionspace{120pt}

\tasknumber{11}%
\task{%
    Определите работу силы, которая обеспечит подъём тела массой $5\,\text{кг}$ на высоту $10\,\text{м}$ с постоянным ускорением $3\,\frac{\text{м}}{\text{c}^{2}}$.
    % Примите $g = 10\,\frac{\text{м}}{\text{с}^{2}}$.
}
\answer{%
    \begin{align*}
    &\text{Для подъёма:} A = Fh = (mg + ma) h = m(g+a)h, \\
    &\text{Для спуска:} A = -Fh = -(mg - ma) h = -m(g-a)h, \\
    &\text{В результате получаем:} 650\,\text{Дж}.
    \end{align*}
}
\solutionspace{60pt}

\tasknumber{12}%
\task{%
    Тело бросили вертикально вверх со скоростью $20\,\frac{\text{м}}{\text{c}}$.
    На какой высоте кинетическая энергия тела составит половину от потенциальной?
}
\answer{%
    \begin{align*}
    &0 + \frac{mv_0^2}2 = E_p + E_k, E_k = \frac 12 E_p \implies \\
    &\implies \frac{mv_0^2}2 = E_p + \frac 12 E_p = E_p\cbr{1 + \frac 12} = mgh\cbr{1 + \frac 12} \implies \\
    &\implies h = \frac{\frac{mv_0^2}2}{mg\cbr{1 + \frac 12}} = \frac{v_0^2}{2g} \cdot \frac 1{1 + \frac 12} \approx 13{,}3\,\text{м}.
    \end{align*}
}
\solutionspace{100pt}

\tasknumber{13}%
\task{%
    Плотность воздуха при нормальных условиях равна $1{,}3\,\frac{\text{кг}}{\text{м}^{3}}$.
    Чему равна плотность воздуха
    при температуре $150\celsius$ и давлении $80\,\text{кПа}$?
}
\answer{%
    \begin{align*}
    &\text{В общем случае:} PV = \frac m{\mu} RT \implies \rho = \frac mV = \frac m{\frac{\frac m{\mu} RT}P} = \frac{P\mu}{RT}, \\
    &\text{У нас 2 состояния:} \rho_1 = \frac{P_1\mu}{RT_1}, \rho_2 = \frac{P_2\mu}{RT_2} \implies \frac{\rho_2}{\rho_1} = \frac{\frac{P_2\mu}{RT_2}}{\frac{P_1\mu}{RT_1}} = \frac{P_2T_1}{P_1T_2} \implies \\
    &\implies \rho_2 = \rho_1 \cdot  \frac{P_2T_1}{P_1T_2} = 1{,}3\,\frac{\text{кг}}{\text{м}^{3}} \cdot \frac{80\,\text{кПа} \cdot 273\units{К}}{100\,\text{кПа} \cdot 423\units{К}} \approx 0{,}67\,\frac{\text{кг}}{\text{м}^{3}}.
    \end{align*}
}
\solutionspace{120pt}

\tasknumber{14}%
\task{%
    Небольшую цилиндрическую пробирку с воздухом погружают на некоторую глубину в глубокое пресное озеро,
    после чего воздух занимает в ней лишь пятую часть от общего объема.
    Определите глубину, на которую погрузили пробирку.
    Температуру считать постоянной $T = 279\,\text{К}$, давлением паров воды пренебречь,
    атмосферное давление принять равным $p_{\text{aтм}} = 100\,\text{кПа}$.
}
\answer{%
    \begin{align*}
    T\text{— const} &\implies P_1V_1 = \nu RT = P_2V_2.
    \\
    V_2 = \frac 15 V_1 &\implies P_1V_1 = P_2 \cdot \frac 15V_1 \implies P_2 = 5P_1 = 5p_{\text{aтм}}.
    \\
    P_2 = p_{\text{aтм}} + \rho_{\text{в}} g h \implies h = \frac{P_2 - p_{\text{aтм}}}{\rho_{\text{в}} g} &= \frac{5p_{\text{aтм}} - p_{\text{aтм}}}{\rho_{\text{в}} g} = \frac{4 \cdot p_{\text{aтм}}}{\rho_{\text{в}} g} =  \\
     &= \frac{4 \cdot 100\,\text{кПа}}{1000\,\frac{\text{кг}}{\text{м}^{3}} \cdot  10\,\frac{\text{м}}{\text{с}^{2}}} \approx 40\,\text{м}.
    \end{align*}
}
\solutionspace{120pt}

\tasknumber{15}%
\task{%
    Газу сообщили некоторое количество теплоты,
    при этом четверть его он потратил на совершение работы,
    одновременно увеличив свою внутреннюю энергию на $1200\,\text{Дж}$.
    Определите работу, совершённую газом.
}
\answer{%
    \begin{align*}
    Q &= A' + \Delta U, A' = \frac 14 Q \implies Q \cdot \cbr{1 - \frac 14} = \Delta U \implies Q = \frac{\Delta U}{1 - \frac 14} = \frac{ 1200\,\text{Дж} }{1 - \frac 14} \approx 1600\,\text{Дж}.
    \\
    A' &= \frac 14 Q
        = \frac 14 \cdot \frac{\Delta U}{1 - \frac 14}
        = \frac{\Delta U}{4 - 1}
        = \frac{ 1200\,\text{Дж} }{4 - 1} \approx 400\,\text{Дж}.
    \end{align*}
}
\solutionspace{60pt}

\tasknumber{16}%
\task{%
    Два конденсатора ёмкостей $C_1 = 40\,\text{нФ}$ и $C_2 = 60\,\text{нФ}$ последовательно подключают
    к источнику напряжения $U = 300\,\text{В}$ (см.
    рис.).
    % Определите заряды каждого из конденсаторов.
    Определите заряд второго конденсатора.

    \begin{tikzpicture}[circuit ee IEC, semithick]
        \draw  (0, 0) to [capacitor={info={$C_1$}}] (1, 0)
                       to [capacitor={info={$C_2$}}] (2, 0)
        ;
        % \draw [-o] (0, 0) -- ++(-0.5, 0) node[left] {$-$};
        % \draw [-o] (2, 0) -- ++(0.5, 0) node[right] {$+$};
        \draw [-o] (0, 0) -- ++(-0.5, 0) node[left] {};
        \draw [-o] (2, 0) -- ++(0.5, 0) node[right] {};
    \end{tikzpicture}
}
\answer{%
    $
        Q_1
            = Q_2
            = CU
            = \frac{ U }{\frac1{C_1} + \frac1{C_2}}
            = \frac{C_1C_2U}{C_1 + C_2}
            = \frac{
                40\,\text{нФ} \cdot 60\,\text{нФ} \cdot 300\,\text{В}
            }{
                40\,\text{нФ} + 60\,\text{нФ}
            }
            = 7{,}20\,\text{мкКл}
    $
}
\solutionspace{120pt}

\tasknumber{17}%
\task{%
    В вакууме вдоль одной прямой расположены три отрицательных заряда так,
    что расстояние между соседними зарядами равно $d$.
    Сделайте рисунок,
    и определите силу, действующую на крайний заряд.
    Модули всех зарядов равны $Q$ ($Q > 0$).
}
\answer{%
    $F = \sum_i F_i = \ldots = \frac54 \frac{kQ^2}{d^2}.$
}
\solutionspace{80pt}

\tasknumber{18}%
\task{%
    Юлия проводит эксперименты c 2 кусками одинаковой стальной проволки, причём второй кусок в два раза длиннее первого.
    В одном из экспериментов Юлия подаёт на первый кусок проволки напряжение в пять раз раз больше, чем на второй.
    Определите отношения в двух проволках в этом эксперименте (второй к первой):
    \begin{itemize}
        \item отношение сил тока,
        \item отношение выделяющихся мощностей.
    \end{itemize}
}
\answer{%
    $R_2 = 2R_1, U_1 = 5U_2 \implies  \eli_2 / \eli_1 = \frac{U_2 / R_2}{U_1 / R_1} = \frac{U_2}{U_1} \cdot \frac{R_1}{R_2} = \frac1{10}, P_2 / P_1 = \frac{U_2^2 / R_2}{U_1^2 / R_1} = \sqr{\frac{U_2}{U_1}} \cdot \frac{R_1}{R_2} = \frac1{50}.$
}

\variantsplitter

\addpersonalvariant{Анастасия Ламанова}

\tasknumber{1}%
\task{%
    Саша стартует на лошади и в течение $t = 5\,\text{c}$ двигается с постоянным ускорением $2{,}5\,\frac{\text{м}}{\text{с}^{2}}$.
    Определите
    \begin{itemize}
        \item какую скорость при этом удастся достичь,
        \item какой путь за это время будет пройден,
        \item среднюю скорость за всё время движения, если после начального ускорения продолжить движение равномерно ещё в течение времени $3t$
    \end{itemize}
}
\answer{%
    \begin{align*}
    v &= v_0 + a t = at = 2{,}5\,\frac{\text{м}}{\text{с}^{2}} \cdot 5\,\text{c} = 12{,}5\,\frac{\text{м}}{\text{с}}, \\
    s_x &= v_0t + \frac{a t^2}2 = \frac{a t^2}2 = \frac{2{,}5\,\frac{\text{м}}{\text{с}^{2}} \cdot \sqr{ 5\,\text{c} }}2 = 31{,}2\,\text{м}, \\
    v_\text{сред.} &= \frac{s_\text{общ}}{t_\text{общ.}} = \frac{s_x + v \cdot 3t}{t + 3t} = \frac{\frac{a t^2}2 + at \cdot 3t}{t (1 + 3)} = \\
    &= at \cdot \frac{\frac 12 + 3}{1 + 3} = 2{,}5\,\frac{\text{м}}{\text{с}^{2}} \cdot 5\,\text{c} \cdot \frac{\frac 12 + 3}{1 + 3} \approx 10{,}94\,\frac{\text{м}}{\text{c}}.
    \end{align*}
}
\solutionspace{120pt}

\tasknumber{2}%
\task{%
    Какой путь тело пройдёт за вторую секунду после начала свободного падения?
    Какую скорость в начале этой секунды оно имеет?
}
\answer{%
    \begin{align*}
    s &= -s_y = -(y_2-y_1) = y_1 - y_2 = \cbr{y_{0y} + v_{0y}t_1 - \frac{gt_1^2}2} - \cbr{y_{0y} + v_{0y}t_2 - \frac{gt_2^2}2} = \\
    &= \frac{gt_2^2}2 - \frac{gt_1^2}2 = \frac g2\cbr{t_2^2 - t_1^2} = 15{,}0\,\text{м}, \\
    v_y &= v_{0y} - gt = -gt = 10\,\frac{\text{м}}{\text{с}^{2}} \cdot 1\,\text{с} = -10\,\frac{\text{м}}{\text{с}}.
    \end{align*}
}
\solutionspace{120pt}

\tasknumber{3}%
\task{%
    Карусель радиусом $2\,\text{м}$ равномерно совершает 10 оборотов в минуту.
    Определите
    \begin{itemize}
        \item период и частоту её обращения,
        \item скорость и ускорение крайних её точек.
    \end{itemize}
}
\answer{%
    \begin{align*}
    t &= 60\,\text{с}, r = 2{,}0\,\text{м}, n = 10\units{оборотов}, \\
    T &= \frac tN = \frac{ 60\,\text{с} }{10} \approx 6{,}00\,\text{c}, \\
    \nu &= \frac 1T = \frac{10}{ 60\,\text{с} } \approx 0{,}17\,\text{Гц}, \\
    v &= \frac{2 \pi r}{T} = \frac{2 \pi r}{T} =  \frac{2 \pi r n}{t} \approx 2{,}09\,\frac{\text{м}}{\text{c}}, \\
    a &= \frac{v^2}{r} =  \frac{4 \pi^2 r n^2}{t^2} \approx 2{,}19\,\frac{\text{м}}{\text{с}^{2}}.
    \end{align*}
}
\solutionspace{80pt}

\tasknumber{4}%
\task{%
    Даша стоит на обрыве над рекой и методично и строго горизонтально кидает в неё камушки.
    За этим всем наблюдает экспериментатор Глюк, который уже выяснил, что камушки падают в реку спустя $1{,}2\,\text{с}$ после броска,
    а вот дальность полёта оценить сложнее: придётся лезть в воду.
    Выручите Глюка и определите:
    \begin{itemize}
        \item высоту обрыва (вместе с ростом Даши).
        \item дальность полёта камушков (по горизонтали) и их скорость при падении, приняв начальную скорость броска равной $v_0 = 17\,\frac{\text{м}}{\text{с}}$.
    \end{itemize}
    Сопротивлением воздуха пренебречь.
}
\answer{%
    \begin{align*}
    y &= y_0 + v_{0y}t - \frac{gt^2}2 = h - \frac{gt^2}2, \qquad y(\tau) = 0 \implies h - \frac{g\tau^2}2 = 0 \implies h = \frac{g\tau^2}2 \approx 7{,}2\,\text{м}.
    \\
    x &= x_0 + v_{0x}t = v_0t \implies L = v_0\tau \approx 20{,}4\,\text{м}.
    \\
    &v = \sqrt{v_x^2 + v_y^2} = \sqrt{v_{0x}^2 + \sqr{v_{0y} - g\tau}} = \sqrt{v_0^2 + \sqr{g\tau}} \approx 20{,}8\,\frac{\text{м}}{\text{c}}.
    \end{align*}
}
\solutionspace{120pt}

\tasknumber{5}%
\task{%
    Пять одинаковых брусков массой $3\,\text{кг}$ каждый лежат на гладком горизонтальном столе.
    Бруски пронумерованы от 1 до 5 и последовательно связаны между собой
    невесомыми нерастяжимыми нитями: 1 со 2, 2 с 3 (ну и с 1) и т.д.
    Экспериментатор Глюк прикладывает постоянную горизонтальную силу $120\,\text{Н}$ к бруску с наибольшим номером.
    С каким ускорением двигается система? Чему равна сила натяжения нити, связывающей бруски 1 и 2?
}
\answer{%
    \begin{align*}
    a &= \frac{F}{5 m} = \frac{120\,\text{Н}}{5 \cdot 3\,\text{кг}} \approx 8{,}0\,\frac{\text{м}}{\text{c}^{2}}, \\
    T &= m'a = 1m \cdot \frac{F}{5 m} = \frac{1}{5} F \approx 24{,}0\,\text{Н}.
    \end{align*}
}
\solutionspace{120pt}

\tasknumber{6}%
\task{%
    Два бруска связаны лёгкой нерастяжимой нитью и перекинуты через неподвижный блок (см.
    рис.).
    Определите силу натяжения нити и ускорения брусков.
    Силами трения пренебречь, массы брусков
    равны $m_1 = 11\,\text{кг}$ и $m_2 = 14\,\text{кг}$.
    % $g = 10\,\frac{\text{м}}{\text{с}^{2}}$.

    \begin{tikzpicture}[x=1.5cm,y=1.5cm,thick]
        \draw
            (-0.4, 0) rectangle (-0.2, 1.2)
            (0.15, 0.5) rectangle (0.45, 1)
            (0, 2) circle [radius=0.3] -- ++(up:0.5)
            (-0.3, 1.2) -- ++(up:0.8)
            (0.3, 1) -- ++(up:1)
            (-0.7, 2.5) -- (0.7, 2.5)
            ;
        \draw[pattern={Lines[angle=51,distance=3pt]},pattern color=black,draw=none] (-0.7, 2.5) rectangle (0.7, 2.75);
        \node [left] (left) at (-0.4, 0.6) { $m_1$ };
        \node [right] (right) at (0.4, 0.75) { $m_2$ };
    \end{tikzpicture}
}
\answer{%
    Предположим, что левый брусок ускоряется вверх, тогда правый ускоряется вниз (с тем же ускорением).
    Запишем 2-й закон Ньютона 2 раза (для обоих тел) в проекции на вертикальную оси, направив её вверх.
    \begin{align*}
        &\begin{cases}
            T - m_1g = m_1a, \\
            T - m_2g = -m_2a,
        \end{cases} \\
        &\begin{cases}
            m_2g - m_1g = m_1a + m_2a, \\
            T = m_1a + m_1g, \\
        \end{cases} \\
        a &= \frac{m_2 - m_1}{m_1 + m_2} \cdot g = \frac{14\,\text{кг} - 11\,\text{кг}}{11\,\text{кг} + 14\,\text{кг}} \cdot 10\,\frac{\text{м}}{\text{с}^{2}} \approx 1{,}20\,\frac{\text{м}}{\text{c}^{2}}, \\
        T &= m_1(a + g) = m_1 \cdot g \cdot \cbr{\frac{m_2 - m_1}{m_1 + m_2} + 1} = m_1 \cdot g \cdot \frac{2m_2}{m_1 + m_2} = \\
            &= \frac{2 m_2 m_1 g}{m_1 + m_2} = \frac{2 \cdot 14\,\text{кг} \cdot 11\,\text{кг} \cdot 10\,\frac{\text{м}}{\text{с}^{2}}}{11\,\text{кг} + 14\,\text{кг}} \approx 123{,}2\,\text{Н}.
    \end{align*}
    Отрицательный ответ говорит, что мы лишь не угадали с направлением ускорений.
    Сила же всегда положительна.
}
\solutionspace{80pt}

\tasknumber{7}%
\task{%
    Тело массой $2\,\text{кг}$ лежит на горизонтальной поверхности.
    Коэффициент трения между поверхностью и телом $0{,}25$.
    К телу приложена горизонтальная сила $3{,}5\,\text{Н}$.
    Определите силу трения, действующую на тело, и ускорение тела.
    % $g = 10\,\frac{\text{м}}{\text{с}^{2}}$.
}
\answer{%
    \begin{align*}
    &F_\text{ трения покоя $\max$ } = \mu N = \mu m g = 0{,}25 \cdot 2\,\text{кг} \cdot 10\,\frac{\text{м}}{\text{с}^{2}} = 5{,}00\,\text{Н}, \\
    &F_\text{ трения покоя $\max$ } > F \implies F_\text{ трения } = 3{,}50\,\text{Н}, a = \frac{F - F_\text{ трения }}{ m } = 0\,\frac{\text{м}}{\text{c}^{2}}, \\
    &\text{при равенстве возможны оба варианта: и едет, и не едет, но на ответы это не влияет.}
    \end{align*}
}
\solutionspace{120pt}

\tasknumber{8}%
\task{%
    Определите плотность неизвестного вещества, если известно, что опускании тела из него
    в керосин оно будет плавать и на треть выступать над поверхностью жидкости.
}
\answer{%
    $F_\text{Арх.} = F_\text{тяж.} \implies \rho_\text{ж.} g V_\text{погр.} = m g \implies\rho_\text{ж.} g \cbr{V -\frac V3} = \rho V g \implies \rho = \rho_\text{ж.}\cbr{1 -\frac 13} \approx 533\,\frac{\text{кг}}{\text{м}^{3}}$
}
\solutionspace{120pt}

\tasknumber{9}%
\task{%
    	Определите силу, действующую на правую опору однородного горизонтального стержня длиной $l = 5\,\text{м}$
    	и массой $M = 1\,\text{кг}$, к которому подвешен груз массой $m = 4\,\text{кг}$ на расстоянии $2\,\text{м}$ от правого конца (см.
    рис.).

        \begin{tikzpicture}[thick]
            \draw
                (-2, -0.1) rectangle (2, 0.1)
                (-0.5, -0.1) -- (-0.5, -1)
                (-0.7, -1) rectangle (-0.3, -1.3)
           		(-2, -0.1) -- +(0.15,-0.9) -- +(-0.15,-0.9) -- cycle
            	(2, -0.1) -- +(0.15,-0.9) -- +(-0.15,-0.9) -- cycle
            ;
            \draw[pattern={Lines[angle=51,distance=2pt]},pattern color=black,draw=none]
            	(-2.15, -1.15) rectangle +(0.3, 0.15)
            	(2.15, -1.15) rectangle +(-0.3, 0.15)
            ;
            \node [right] (m_small) at (-0.3, -1.15) { $m$ };
            \node [above] (M_big) at (0, 0.1) { $M$ };
        \end{tikzpicture}
}
\answer{%
    \begin{align*}
        &\begin{cases}
            F_1 + F_2 - mg - Mg= 0, \\
            F_1 \cdot 0 - mg \cdot a - Mg \cdot \frac l2 + F_2 \cdot l = 0,
        \end{cases} \\
        F_2 &= \frac{mga + Mg\frac l2}l = \frac al \cdot mg + \frac{Mg}2 \approx 29{,}0\,\text{Н}, \\
        F_1 &= mg + Mg - F_2 = mg + Mg - \frac al \cdot mg - \frac{Mg}2 = \frac bl \cdot mg + \frac{Mg}2 \approx 21{,}0\,\text{Н}.
    \end{align*}
}
\solutionspace{80pt}

\tasknumber{10}%
\task{%
    Тонкий однородный шест длиной $2\,\text{м}$ и массой $10\,\text{кг}$ лежит на горизонтальной поверхности.
    \begin{itemize}
        \item Какую минимальную силу надо приложить к одному из его концов, чтобы оторвать его от этой поверхности?
        \item Какую минимальную работу надо совершить, чтобы поставить его на землю в вертикальное положение?
    \end{itemize}
    % Примите $g = 10\,\frac{\text{м}}{\text{с}^{2}}$.
}
\answer{%
    $F = \frac{mg}2 \approx 100\,\text{Н}, A = mg\frac l2 = 100\,\text{Дж}$
}
\solutionspace{120pt}

\tasknumber{11}%
\task{%
    Определите работу силы, которая обеспечит подъём тела массой $5\,\text{кг}$ на высоту $10\,\text{м}$ с постоянным ускорением $3\,\frac{\text{м}}{\text{c}^{2}}$.
    % Примите $g = 10\,\frac{\text{м}}{\text{с}^{2}}$.
}
\answer{%
    \begin{align*}
    &\text{Для подъёма:} A = Fh = (mg + ma) h = m(g+a)h, \\
    &\text{Для спуска:} A = -Fh = -(mg - ma) h = -m(g-a)h, \\
    &\text{В результате получаем:} 650\,\text{Дж}.
    \end{align*}
}
\solutionspace{60pt}

\tasknumber{12}%
\task{%
    Тело бросили вертикально вверх со скоростью $14\,\frac{\text{м}}{\text{c}}$.
    На какой высоте кинетическая энергия тела составит половину от потенциальной?
}
\answer{%
    \begin{align*}
    &0 + \frac{mv_0^2}2 = E_p + E_k, E_k = \frac 12 E_p \implies \\
    &\implies \frac{mv_0^2}2 = E_p + \frac 12 E_p = E_p\cbr{1 + \frac 12} = mgh\cbr{1 + \frac 12} \implies \\
    &\implies h = \frac{\frac{mv_0^2}2}{mg\cbr{1 + \frac 12}} = \frac{v_0^2}{2g} \cdot \frac 1{1 + \frac 12} \approx 6{,}5\,\text{м}.
    \end{align*}
}
\solutionspace{100pt}

\tasknumber{13}%
\task{%
    Плотность воздуха при нормальных условиях равна $1{,}3\,\frac{\text{кг}}{\text{м}^{3}}$.
    Чему равна плотность воздуха
    при температуре $50\celsius$ и давлении $120\,\text{кПа}$?
}
\answer{%
    \begin{align*}
    &\text{В общем случае:} PV = \frac m{\mu} RT \implies \rho = \frac mV = \frac m{\frac{\frac m{\mu} RT}P} = \frac{P\mu}{RT}, \\
    &\text{У нас 2 состояния:} \rho_1 = \frac{P_1\mu}{RT_1}, \rho_2 = \frac{P_2\mu}{RT_2} \implies \frac{\rho_2}{\rho_1} = \frac{\frac{P_2\mu}{RT_2}}{\frac{P_1\mu}{RT_1}} = \frac{P_2T_1}{P_1T_2} \implies \\
    &\implies \rho_2 = \rho_1 \cdot  \frac{P_2T_1}{P_1T_2} = 1{,}3\,\frac{\text{кг}}{\text{м}^{3}} \cdot \frac{120\,\text{кПа} \cdot 273\units{К}}{100\,\text{кПа} \cdot 323\units{К}} \approx 1{,}32\,\frac{\text{кг}}{\text{м}^{3}}.
    \end{align*}
}
\solutionspace{120pt}

\tasknumber{14}%
\task{%
    Небольшую цилиндрическую пробирку с воздухом погружают на некоторую глубину в глубокое пресное озеро,
    после чего воздух занимает в ней лишь пятую часть от общего объема.
    Определите глубину, на которую погрузили пробирку.
    Температуру считать постоянной $T = 284\,\text{К}$, давлением паров воды пренебречь,
    атмосферное давление принять равным $p_{\text{aтм}} = 100\,\text{кПа}$.
}
\answer{%
    \begin{align*}
    T\text{— const} &\implies P_1V_1 = \nu RT = P_2V_2.
    \\
    V_2 = \frac 15 V_1 &\implies P_1V_1 = P_2 \cdot \frac 15V_1 \implies P_2 = 5P_1 = 5p_{\text{aтм}}.
    \\
    P_2 = p_{\text{aтм}} + \rho_{\text{в}} g h \implies h = \frac{P_2 - p_{\text{aтм}}}{\rho_{\text{в}} g} &= \frac{5p_{\text{aтм}} - p_{\text{aтм}}}{\rho_{\text{в}} g} = \frac{4 \cdot p_{\text{aтм}}}{\rho_{\text{в}} g} =  \\
     &= \frac{4 \cdot 100\,\text{кПа}}{1000\,\frac{\text{кг}}{\text{м}^{3}} \cdot  10\,\frac{\text{м}}{\text{с}^{2}}} \approx 40\,\text{м}.
    \end{align*}
}
\solutionspace{120pt}

\tasknumber{15}%
\task{%
    Газу сообщили некоторое количество теплоты,
    при этом половину его он потратил на совершение работы,
    одновременно увеличив свою внутреннюю энергию на $3000\,\text{Дж}$.
    Определите работу, совершённую газом.
}
\answer{%
    \begin{align*}
    Q &= A' + \Delta U, A' = \frac 12 Q \implies Q \cdot \cbr{1 - \frac 12} = \Delta U \implies Q = \frac{\Delta U}{1 - \frac 12} = \frac{ 3000\,\text{Дж} }{1 - \frac 12} \approx 6000\,\text{Дж}.
    \\
    A' &= \frac 12 Q
        = \frac 12 \cdot \frac{\Delta U}{1 - \frac 12}
        = \frac{\Delta U}{2 - 1}
        = \frac{ 3000\,\text{Дж} }{2 - 1} \approx 3000\,\text{Дж}.
    \end{align*}
}
\solutionspace{60pt}

\tasknumber{16}%
\task{%
    Два конденсатора ёмкостей $C_1 = 30\,\text{нФ}$ и $C_2 = 40\,\text{нФ}$ последовательно подключают
    к источнику напряжения $U = 400\,\text{В}$ (см.
    рис.).
    % Определите заряды каждого из конденсаторов.
    Определите заряд второго конденсатора.

    \begin{tikzpicture}[circuit ee IEC, semithick]
        \draw  (0, 0) to [capacitor={info={$C_1$}}] (1, 0)
                       to [capacitor={info={$C_2$}}] (2, 0)
        ;
        % \draw [-o] (0, 0) -- ++(-0.5, 0) node[left] {$-$};
        % \draw [-o] (2, 0) -- ++(0.5, 0) node[right] {$+$};
        \draw [-o] (0, 0) -- ++(-0.5, 0) node[left] {};
        \draw [-o] (2, 0) -- ++(0.5, 0) node[right] {};
    \end{tikzpicture}
}
\answer{%
    $
        Q_1
            = Q_2
            = CU
            = \frac{ U }{\frac1{C_1} + \frac1{C_2}}
            = \frac{C_1C_2U}{C_1 + C_2}
            = \frac{
                30\,\text{нФ} \cdot 40\,\text{нФ} \cdot 400\,\text{В}
            }{
                30\,\text{нФ} + 40\,\text{нФ}
            }
            = 6{,}86\,\text{мкКл}
    $
}
\solutionspace{120pt}

\tasknumber{17}%
\task{%
    В вакууме вдоль одной прямой расположены четыре отрицательных заряда так,
    что расстояние между соседними зарядами равно $l$.
    Сделайте рисунок,
    и определите силу, действующую на крайний заряд.
    Модули всех зарядов равны $Q$ ($Q > 0$).
}
\answer{%
    $F = \sum_i F_i = \ldots = \frac{49}{36} \frac{kQ^2}{l^2}.$
}
\solutionspace{80pt}

\tasknumber{18}%
\task{%
    Юлия проводит эксперименты c 2 кусками одинаковой медной проволки, причём второй кусок в два раза длиннее первого.
    В одном из экспериментов Юлия подаёт на первый кусок проволки напряжение в три раза раз больше, чем на второй.
    Определите отношения в двух проволках в этом эксперименте (второй к первой):
    \begin{itemize}
        \item отношение сил тока,
        \item отношение выделяющихся мощностей.
    \end{itemize}
}
\answer{%
    $R_2 = 2R_1, U_1 = 3U_2 \implies  \eli_2 / \eli_1 = \frac{U_2 / R_2}{U_1 / R_1} = \frac{U_2}{U_1} \cdot \frac{R_1}{R_2} = \frac16, P_2 / P_1 = \frac{U_2^2 / R_2}{U_1^2 / R_1} = \sqr{\frac{U_2}{U_1}} \cdot \frac{R_1}{R_2} = \frac1{18}.$
}

\variantsplitter

\addpersonalvariant{Виктория Легонькова}

\tasknumber{1}%
\task{%
    Женя стартует на мотоцикле и в течение $t = 3\,\text{c}$ двигается с постоянным ускорением $1{,}5\,\frac{\text{м}}{\text{с}^{2}}$.
    Определите
    \begin{itemize}
        \item какую скорость при этом удастся достичь,
        \item какой путь за это время будет пройден,
        \item среднюю скорость за всё время движения, если после начального ускорения продолжить движение равномерно ещё в течение времени $2t$
    \end{itemize}
}
\answer{%
    \begin{align*}
    v &= v_0 + a t = at = 1{,}5\,\frac{\text{м}}{\text{с}^{2}} \cdot 3\,\text{c} = 4{,}5\,\frac{\text{м}}{\text{с}}, \\
    s_x &= v_0t + \frac{a t^2}2 = \frac{a t^2}2 = \frac{1{,}5\,\frac{\text{м}}{\text{с}^{2}} \cdot \sqr{ 3\,\text{c} }}2 = 6{,}8\,\text{м}, \\
    v_\text{сред.} &= \frac{s_\text{общ}}{t_\text{общ.}} = \frac{s_x + v \cdot 2t}{t + 2t} = \frac{\frac{a t^2}2 + at \cdot 2t}{t (1 + 2)} = \\
    &= at \cdot \frac{\frac 12 + 2}{1 + 2} = 1{,}5\,\frac{\text{м}}{\text{с}^{2}} \cdot 3\,\text{c} \cdot \frac{\frac 12 + 2}{1 + 2} \approx 3{,}75\,\frac{\text{м}}{\text{c}}.
    \end{align*}
}
\solutionspace{120pt}

\tasknumber{2}%
\task{%
    Какой путь тело пройдёт за вторую секунду после начала свободного падения?
    Какую скорость в конце этой секунды оно имеет?
}
\answer{%
    \begin{align*}
    s &= -s_y = -(y_2-y_1) = y_1 - y_2 = \cbr{y_{0y} + v_{0y}t_1 - \frac{gt_1^2}2} - \cbr{y_{0y} + v_{0y}t_2 - \frac{gt_2^2}2} = \\
    &= \frac{gt_2^2}2 - \frac{gt_1^2}2 = \frac g2\cbr{t_2^2 - t_1^2} = 15{,}0\,\text{м}, \\
    v_y &= v_{0y} - gt = -gt = 10\,\frac{\text{м}}{\text{с}^{2}} \cdot 2\,\text{с} = -20\,\frac{\text{м}}{\text{с}}.
    \end{align*}
}
\solutionspace{120pt}

\tasknumber{3}%
\task{%
    Карусель радиусом $4\,\text{м}$ равномерно совершает 10 оборотов в минуту.
    Определите
    \begin{itemize}
        \item период и частоту её обращения,
        \item скорость и ускорение крайних её точек.
    \end{itemize}
}
\answer{%
    \begin{align*}
    t &= 60\,\text{с}, r = 4{,}0\,\text{м}, n = 10\units{оборотов}, \\
    T &= \frac tN = \frac{ 60\,\text{с} }{10} \approx 6{,}00\,\text{c}, \\
    \nu &= \frac 1T = \frac{10}{ 60\,\text{с} } \approx 0{,}17\,\text{Гц}, \\
    v &= \frac{2 \pi r}{T} = \frac{2 \pi r}{T} =  \frac{2 \pi r n}{t} \approx 4{,}19\,\frac{\text{м}}{\text{c}}, \\
    a &= \frac{v^2}{r} =  \frac{4 \pi^2 r n^2}{t^2} \approx 4{,}39\,\frac{\text{м}}{\text{с}^{2}}.
    \end{align*}
}
\solutionspace{80pt}

\tasknumber{4}%
\task{%
    Даша стоит на обрыве над рекой и методично и строго горизонтально кидает в неё камушки.
    За этим всем наблюдает экспериментатор Глюк, который уже выяснил, что камушки падают в реку спустя $1{,}3\,\text{с}$ после броска,
    а вот дальность полёта оценить сложнее: придётся лезть в воду.
    Выручите Глюка и определите:
    \begin{itemize}
        \item высоту обрыва (вместе с ростом Даши).
        \item дальность полёта камушков (по горизонтали) и их скорость при падении, приняв начальную скорость броска равной $v_0 = 13\,\frac{\text{м}}{\text{с}}$.
    \end{itemize}
    Сопротивлением воздуха пренебречь.
}
\answer{%
    \begin{align*}
    y &= y_0 + v_{0y}t - \frac{gt^2}2 = h - \frac{gt^2}2, \qquad y(\tau) = 0 \implies h - \frac{g\tau^2}2 = 0 \implies h = \frac{g\tau^2}2 \approx 8{,}5\,\text{м}.
    \\
    x &= x_0 + v_{0x}t = v_0t \implies L = v_0\tau \approx 16{,}9\,\text{м}.
    \\
    &v = \sqrt{v_x^2 + v_y^2} = \sqrt{v_{0x}^2 + \sqr{v_{0y} - g\tau}} = \sqrt{v_0^2 + \sqr{g\tau}} \approx 18{,}4\,\frac{\text{м}}{\text{c}}.
    \end{align*}
}
\solutionspace{120pt}

\tasknumber{5}%
\task{%
    Пять одинаковых брусков массой $3\,\text{кг}$ каждый лежат на гладком горизонтальном столе.
    Бруски пронумерованы от 1 до 5 и последовательно связаны между собой
    невесомыми нерастяжимыми нитями: 1 со 2, 2 с 3 (ну и с 1) и т.д.
    Экспериментатор Глюк прикладывает постоянную горизонтальную силу $90\,\text{Н}$ к бруску с наименьшим номером.
    С каким ускорением двигается система? Чему равна сила натяжения нити, связывающей бруски 1 и 2?
}
\answer{%
    \begin{align*}
    a &= \frac{F}{5 m} = \frac{90\,\text{Н}}{5 \cdot 3\,\text{кг}} \approx 6{,}0\,\frac{\text{м}}{\text{c}^{2}}, \\
    T &= m'a = 4m \cdot \frac{F}{5 m} = \frac{4}{5} F \approx 72{,}0\,\text{Н}.
    \end{align*}
}
\solutionspace{120pt}

\tasknumber{6}%
\task{%
    Два бруска связаны лёгкой нерастяжимой нитью и перекинуты через неподвижный блок (см.
    рис.).
    Определите силу натяжения нити и ускорения брусков.
    Силами трения пренебречь, массы брусков
    равны $m_1 = 5\,\text{кг}$ и $m_2 = 10\,\text{кг}$.
    % $g = 10\,\frac{\text{м}}{\text{с}^{2}}$.

    \begin{tikzpicture}[x=1.5cm,y=1.5cm,thick]
        \draw
            (-0.4, 0) rectangle (-0.2, 1.2)
            (0.15, 0.5) rectangle (0.45, 1)
            (0, 2) circle [radius=0.3] -- ++(up:0.5)
            (-0.3, 1.2) -- ++(up:0.8)
            (0.3, 1) -- ++(up:1)
            (-0.7, 2.5) -- (0.7, 2.5)
            ;
        \draw[pattern={Lines[angle=51,distance=3pt]},pattern color=black,draw=none] (-0.7, 2.5) rectangle (0.7, 2.75);
        \node [left] (left) at (-0.4, 0.6) { $m_1$ };
        \node [right] (right) at (0.4, 0.75) { $m_2$ };
    \end{tikzpicture}
}
\answer{%
    Предположим, что левый брусок ускоряется вверх, тогда правый ускоряется вниз (с тем же ускорением).
    Запишем 2-й закон Ньютона 2 раза (для обоих тел) в проекции на вертикальную оси, направив её вверх.
    \begin{align*}
        &\begin{cases}
            T - m_1g = m_1a, \\
            T - m_2g = -m_2a,
        \end{cases} \\
        &\begin{cases}
            m_2g - m_1g = m_1a + m_2a, \\
            T = m_1a + m_1g, \\
        \end{cases} \\
        a &= \frac{m_2 - m_1}{m_1 + m_2} \cdot g = \frac{10\,\text{кг} - 5\,\text{кг}}{5\,\text{кг} + 10\,\text{кг}} \cdot 10\,\frac{\text{м}}{\text{с}^{2}} \approx 3{,}33\,\frac{\text{м}}{\text{c}^{2}}, \\
        T &= m_1(a + g) = m_1 \cdot g \cdot \cbr{\frac{m_2 - m_1}{m_1 + m_2} + 1} = m_1 \cdot g \cdot \frac{2m_2}{m_1 + m_2} = \\
            &= \frac{2 m_2 m_1 g}{m_1 + m_2} = \frac{2 \cdot 10\,\text{кг} \cdot 5\,\text{кг} \cdot 10\,\frac{\text{м}}{\text{с}^{2}}}{5\,\text{кг} + 10\,\text{кг}} \approx 66{,}7\,\text{Н}.
    \end{align*}
    Отрицательный ответ говорит, что мы лишь не угадали с направлением ускорений.
    Сила же всегда положительна.
}
\solutionspace{80pt}

\tasknumber{7}%
\task{%
    Тело массой $1{,}4\,\text{кг}$ лежит на горизонтальной поверхности.
    Коэффициент трения между поверхностью и телом $0{,}25$.
    К телу приложена горизонтальная сила $3{,}5\,\text{Н}$.
    Определите силу трения, действующую на тело, и ускорение тела.
    % $g = 10\,\frac{\text{м}}{\text{с}^{2}}$.
}
\answer{%
    \begin{align*}
    &F_\text{ трения покоя $\max$ } = \mu N = \mu m g = 0{,}25 \cdot 1{,}4\,\text{кг} \cdot 10\,\frac{\text{м}}{\text{с}^{2}} = 3{,}50\,\text{Н}, \\
    &F_\text{ трения покоя $\max$ } \le F \implies F_\text{ трения } = 3{,}50\,\text{Н}, a = \frac{F - F_\text{ трения }}{ m } = 0\,\frac{\text{м}}{\text{c}^{2}}, \\
    &\text{при равенстве возможны оба варианта: и едет, и не едет, но на ответы это не влияет.}
    \end{align*}
}
\solutionspace{120pt}

\tasknumber{8}%
\task{%
    Определите плотность неизвестного вещества, если известно, что опускании тела из него
    в подсолнечное масло оно будет плавать и на половину выступать над поверхностью жидкости.
}
\answer{%
    $F_\text{Арх.} = F_\text{тяж.} \implies \rho_\text{ж.} g V_\text{погр.} = m g \implies\rho_\text{ж.} g \cbr{V -\frac V2} = \rho V g \implies \rho = \rho_\text{ж.}\cbr{1 -\frac 12} \approx 450\,\frac{\text{кг}}{\text{м}^{3}}$
}
\solutionspace{120pt}

\tasknumber{9}%
\task{%
    	Определите силу, действующую на правую опору однородного горизонтального стержня длиной $l = 7\,\text{м}$
    	и массой $M = 5\,\text{кг}$, к которому подвешен груз массой $m = 4\,\text{кг}$ на расстоянии $4\,\text{м}$ от правого конца (см.
    рис.).

        \begin{tikzpicture}[thick]
            \draw
                (-2, -0.1) rectangle (2, 0.1)
                (-0.5, -0.1) -- (-0.5, -1)
                (-0.7, -1) rectangle (-0.3, -1.3)
           		(-2, -0.1) -- +(0.15,-0.9) -- +(-0.15,-0.9) -- cycle
            	(2, -0.1) -- +(0.15,-0.9) -- +(-0.15,-0.9) -- cycle
            ;
            \draw[pattern={Lines[angle=51,distance=2pt]},pattern color=black,draw=none]
            	(-2.15, -1.15) rectangle +(0.3, 0.15)
            	(2.15, -1.15) rectangle +(-0.3, 0.15)
            ;
            \node [right] (m_small) at (-0.3, -1.15) { $m$ };
            \node [above] (M_big) at (0, 0.1) { $M$ };
        \end{tikzpicture}
}
\answer{%
    \begin{align*}
        &\begin{cases}
            F_1 + F_2 - mg - Mg= 0, \\
            F_1 \cdot 0 - mg \cdot a - Mg \cdot \frac l2 + F_2 \cdot l = 0,
        \end{cases} \\
        F_2 &= \frac{mga + Mg\frac l2}l = \frac al \cdot mg + \frac{Mg}2 \approx 42{,}1\,\text{Н}, \\
        F_1 &= mg + Mg - F_2 = mg + Mg - \frac al \cdot mg - \frac{Mg}2 = \frac bl \cdot mg + \frac{Mg}2 \approx 47{,}9\,\text{Н}.
    \end{align*}
}
\solutionspace{80pt}

\tasknumber{10}%
\task{%
    Тонкий однородный шест длиной $2\,\text{м}$ и массой $20\,\text{кг}$ лежит на горизонтальной поверхности.
    \begin{itemize}
        \item Какую минимальную силу надо приложить к одному из его концов, чтобы оторвать его от этой поверхности?
        \item Какую минимальную работу надо совершить, чтобы поставить его на землю в вертикальное положение?
    \end{itemize}
    % Примите $g = 10\,\frac{\text{м}}{\text{с}^{2}}$.
}
\answer{%
    $F = \frac{mg}2 \approx 200\,\text{Н}, A = mg\frac l2 = 200\,\text{Дж}$
}
\solutionspace{120pt}

\tasknumber{11}%
\task{%
    Определите работу силы, которая обеспечит спуск тела массой $2\,\text{кг}$ на высоту $5\,\text{м}$ с постоянным ускорением $2\,\frac{\text{м}}{\text{c}^{2}}$.
    % Примите $g = 10\,\frac{\text{м}}{\text{с}^{2}}$.
}
\answer{%
    \begin{align*}
    &\text{Для подъёма:} A = Fh = (mg + ma) h = m(g+a)h, \\
    &\text{Для спуска:} A = -Fh = -(mg - ma) h = -m(g-a)h, \\
    &\text{В результате получаем:} -80\,\text{Дж}.
    \end{align*}
}
\solutionspace{60pt}

\tasknumber{12}%
\task{%
    Тело бросили вертикально вверх со скоростью $10\,\frac{\text{м}}{\text{c}}$.
    На какой высоте кинетическая энергия тела составит треть от потенциальной?
}
\answer{%
    \begin{align*}
    &0 + \frac{mv_0^2}2 = E_p + E_k, E_k = \frac 13 E_p \implies \\
    &\implies \frac{mv_0^2}2 = E_p + \frac 13 E_p = E_p\cbr{1 + \frac 13} = mgh\cbr{1 + \frac 13} \implies \\
    &\implies h = \frac{\frac{mv_0^2}2}{mg\cbr{1 + \frac 13}} = \frac{v_0^2}{2g} \cdot \frac 1{1 + \frac 13} \approx 3{,}8\,\text{м}.
    \end{align*}
}
\solutionspace{100pt}

\tasknumber{13}%
\task{%
    Плотность воздуха при нормальных условиях равна $1{,}3\,\frac{\text{кг}}{\text{м}^{3}}$.
    Чему равна плотность воздуха
    при температуре $50\celsius$ и давлении $50\,\text{кПа}$?
}
\answer{%
    \begin{align*}
    &\text{В общем случае:} PV = \frac m{\mu} RT \implies \rho = \frac mV = \frac m{\frac{\frac m{\mu} RT}P} = \frac{P\mu}{RT}, \\
    &\text{У нас 2 состояния:} \rho_1 = \frac{P_1\mu}{RT_1}, \rho_2 = \frac{P_2\mu}{RT_2} \implies \frac{\rho_2}{\rho_1} = \frac{\frac{P_2\mu}{RT_2}}{\frac{P_1\mu}{RT_1}} = \frac{P_2T_1}{P_1T_2} \implies \\
    &\implies \rho_2 = \rho_1 \cdot  \frac{P_2T_1}{P_1T_2} = 1{,}3\,\frac{\text{кг}}{\text{м}^{3}} \cdot \frac{50\,\text{кПа} \cdot 273\units{К}}{100\,\text{кПа} \cdot 323\units{К}} \approx 0{,}55\,\frac{\text{кг}}{\text{м}^{3}}.
    \end{align*}
}
\solutionspace{120pt}

\tasknumber{14}%
\task{%
    Небольшую цилиндрическую пробирку с воздухом погружают на некоторую глубину в глубокое пресное озеро,
    после чего воздух занимает в ней лишь шестую часть от общего объема.
    Определите глубину, на которую погрузили пробирку.
    Температуру считать постоянной $T = 291\,\text{К}$, давлением паров воды пренебречь,
    атмосферное давление принять равным $p_{\text{aтм}} = 100\,\text{кПа}$.
}
\answer{%
    \begin{align*}
    T\text{— const} &\implies P_1V_1 = \nu RT = P_2V_2.
    \\
    V_2 = \frac 16 V_1 &\implies P_1V_1 = P_2 \cdot \frac 16V_1 \implies P_2 = 6P_1 = 6p_{\text{aтм}}.
    \\
    P_2 = p_{\text{aтм}} + \rho_{\text{в}} g h \implies h = \frac{P_2 - p_{\text{aтм}}}{\rho_{\text{в}} g} &= \frac{6p_{\text{aтм}} - p_{\text{aтм}}}{\rho_{\text{в}} g} = \frac{5 \cdot p_{\text{aтм}}}{\rho_{\text{в}} g} =  \\
     &= \frac{5 \cdot 100\,\text{кПа}}{1000\,\frac{\text{кг}}{\text{м}^{3}} \cdot  10\,\frac{\text{м}}{\text{с}^{2}}} \approx 50\,\text{м}.
    \end{align*}
}
\solutionspace{120pt}

\tasknumber{15}%
\task{%
    Газу сообщили некоторое количество теплоты,
    при этом треть его он потратил на совершение работы,
    одновременно увеличив свою внутреннюю энергию на $1500\,\text{Дж}$.
    Определите работу, совершённую газом.
}
\answer{%
    \begin{align*}
    Q &= A' + \Delta U, A' = \frac 13 Q \implies Q \cdot \cbr{1 - \frac 13} = \Delta U \implies Q = \frac{\Delta U}{1 - \frac 13} = \frac{ 1500\,\text{Дж} }{1 - \frac 13} \approx 2250\,\text{Дж}.
    \\
    A' &= \frac 13 Q
        = \frac 13 \cdot \frac{\Delta U}{1 - \frac 13}
        = \frac{\Delta U}{3 - 1}
        = \frac{ 1500\,\text{Дж} }{3 - 1} \approx 750\,\text{Дж}.
    \end{align*}
}
\solutionspace{60pt}

\tasknumber{16}%
\task{%
    Два конденсатора ёмкостей $C_1 = 20\,\text{нФ}$ и $C_2 = 40\,\text{нФ}$ последовательно подключают
    к источнику напряжения $V = 300\,\text{В}$ (см.
    рис.).
    % Определите заряды каждого из конденсаторов.
    Определите заряд первого конденсатора.

    \begin{tikzpicture}[circuit ee IEC, semithick]
        \draw  (0, 0) to [capacitor={info={$C_1$}}] (1, 0)
                       to [capacitor={info={$C_2$}}] (2, 0)
        ;
        % \draw [-o] (0, 0) -- ++(-0.5, 0) node[left] {$-$};
        % \draw [-o] (2, 0) -- ++(0.5, 0) node[right] {$+$};
        \draw [-o] (0, 0) -- ++(-0.5, 0) node[left] {};
        \draw [-o] (2, 0) -- ++(0.5, 0) node[right] {};
    \end{tikzpicture}
}
\answer{%
    $
        Q_1
            = Q_2
            = CV
            = \frac{ V }{\frac1{C_1} + \frac1{C_2}}
            = \frac{C_1C_2V}{C_1 + C_2}
            = \frac{
                20\,\text{нФ} \cdot 40\,\text{нФ} \cdot 300\,\text{В}
            }{
                20\,\text{нФ} + 40\,\text{нФ}
            }
            = 4{,}00\,\text{мкКл}
    $
}
\solutionspace{120pt}

\tasknumber{17}%
\task{%
    В вакууме вдоль одной прямой расположены три отрицательных заряда так,
    что расстояние между соседними зарядами равно $d$.
    Сделайте рисунок,
    и определите силу, действующую на крайний заряд.
    Модули всех зарядов равны $Q$ ($Q > 0$).
}
\answer{%
    $F = \sum_i F_i = \ldots = \frac54 \frac{kQ^2}{d^2}.$
}
\solutionspace{80pt}

\tasknumber{18}%
\task{%
    Юлия проводит эксперименты c 2 кусками одинаковой стальной проволки, причём второй кусок в пять раз длиннее первого.
    В одном из экспериментов Юлия подаёт на первый кусок проволки напряжение в три раза раз больше, чем на второй.
    Определите отношения в двух проволках в этом эксперименте (второй к первой):
    \begin{itemize}
        \item отношение сил тока,
        \item отношение выделяющихся мощностей.
    \end{itemize}
}
\answer{%
    $R_2 = 5R_1, U_1 = 3U_2 \implies  \eli_2 / \eli_1 = \frac{U_2 / R_2}{U_1 / R_1} = \frac{U_2}{U_1} \cdot \frac{R_1}{R_2} = \frac1{15}, P_2 / P_1 = \frac{U_2^2 / R_2}{U_1^2 / R_1} = \sqr{\frac{U_2}{U_1}} \cdot \frac{R_1}{R_2} = \frac1{45}.$
}

\variantsplitter

\addpersonalvariant{Семён Мартынов}

\tasknumber{1}%
\task{%
    Женя стартует на мотоцикле и в течение $t = 10\,\text{c}$ двигается с постоянным ускорением $1{,}5\,\frac{\text{м}}{\text{с}^{2}}$.
    Определите
    \begin{itemize}
        \item какую скорость при этом удастся достичь,
        \item какой путь за это время будет пройден,
        \item среднюю скорость за всё время движения, если после начального ускорения продолжить движение равномерно ещё в течение времени $2t$
    \end{itemize}
}
\answer{%
    \begin{align*}
    v &= v_0 + a t = at = 1{,}5\,\frac{\text{м}}{\text{с}^{2}} \cdot 10\,\text{c} = 15{,}0\,\frac{\text{м}}{\text{с}}, \\
    s_x &= v_0t + \frac{a t^2}2 = \frac{a t^2}2 = \frac{1{,}5\,\frac{\text{м}}{\text{с}^{2}} \cdot \sqr{ 10\,\text{c} }}2 = 75{,}0\,\text{м}, \\
    v_\text{сред.} &= \frac{s_\text{общ}}{t_\text{общ.}} = \frac{s_x + v \cdot 2t}{t + 2t} = \frac{\frac{a t^2}2 + at \cdot 2t}{t (1 + 2)} = \\
    &= at \cdot \frac{\frac 12 + 2}{1 + 2} = 1{,}5\,\frac{\text{м}}{\text{с}^{2}} \cdot 10\,\text{c} \cdot \frac{\frac 12 + 2}{1 + 2} \approx 12{,}50\,\frac{\text{м}}{\text{c}}.
    \end{align*}
}
\solutionspace{120pt}

\tasknumber{2}%
\task{%
    Какой путь тело пройдёт за третью секунду после начала свободного падения?
    Какую скорость в начале этой секунды оно имеет?
}
\answer{%
    \begin{align*}
    s &= -s_y = -(y_2-y_1) = y_1 - y_2 = \cbr{y_{0y} + v_{0y}t_1 - \frac{gt_1^2}2} - \cbr{y_{0y} + v_{0y}t_2 - \frac{gt_2^2}2} = \\
    &= \frac{gt_2^2}2 - \frac{gt_1^2}2 = \frac g2\cbr{t_2^2 - t_1^2} = 25{,}0\,\text{м}, \\
    v_y &= v_{0y} - gt = -gt = 10\,\frac{\text{м}}{\text{с}^{2}} \cdot 2\,\text{с} = -20\,\frac{\text{м}}{\text{с}}.
    \end{align*}
}
\solutionspace{120pt}

\tasknumber{3}%
\task{%
    Карусель диаметром $5\,\text{м}$ равномерно совершает 10 оборотов в минуту.
    Определите
    \begin{itemize}
        \item период и частоту её обращения,
        \item скорость и ускорение крайних её точек.
    \end{itemize}
}
\answer{%
    \begin{align*}
    t &= 60\,\text{с}, r = 2{,}5\,\text{м}, n = 10\units{оборотов}, \\
    T &= \frac tN = \frac{ 60\,\text{с} }{10} \approx 6{,}00\,\text{c}, \\
    \nu &= \frac 1T = \frac{10}{ 60\,\text{с} } \approx 0{,}17\,\text{Гц}, \\
    v &= \frac{2 \pi r}{T} = \frac{2 \pi r}{T} =  \frac{2 \pi r n}{t} \approx 2{,}62\,\frac{\text{м}}{\text{c}}, \\
    a &= \frac{v^2}{r} =  \frac{4 \pi^2 r n^2}{t^2} \approx 2{,}74\,\frac{\text{м}}{\text{с}^{2}}.
    \end{align*}
}
\solutionspace{80pt}

\tasknumber{4}%
\task{%
    Паша стоит на обрыве над рекой и методично и строго горизонтально кидает в неё камушки.
    За этим всем наблюдает экспериментатор Глюк, который уже выяснил, что камушки падают в реку спустя $1{,}3\,\text{с}$ после броска,
    а вот дальность полёта оценить сложнее: придётся лезть в воду.
    Выручите Глюка и определите:
    \begin{itemize}
        \item высоту обрыва (вместе с ростом Паши).
        \item дальность полёта камушков (по горизонтали) и их скорость при падении, приняв начальную скорость броска равной $v_0 = 18\,\frac{\text{м}}{\text{с}}$.
    \end{itemize}
    Сопротивлением воздуха пренебречь.
}
\answer{%
    \begin{align*}
    y &= y_0 + v_{0y}t - \frac{gt^2}2 = h - \frac{gt^2}2, \qquad y(\tau) = 0 \implies h - \frac{g\tau^2}2 = 0 \implies h = \frac{g\tau^2}2 \approx 8{,}5\,\text{м}.
    \\
    x &= x_0 + v_{0x}t = v_0t \implies L = v_0\tau \approx 23{,}4\,\text{м}.
    \\
    &v = \sqrt{v_x^2 + v_y^2} = \sqrt{v_{0x}^2 + \sqr{v_{0y} - g\tau}} = \sqrt{v_0^2 + \sqr{g\tau}} \approx 22{,}2\,\frac{\text{м}}{\text{c}}.
    \end{align*}
}
\solutionspace{120pt}

\tasknumber{5}%
\task{%
    Пять одинаковых брусков массой $2\,\text{кг}$ каждый лежат на гладком горизонтальном столе.
    Бруски пронумерованы от 1 до 5 и последовательно связаны между собой
    невесомыми нерастяжимыми нитями: 1 со 2, 2 с 3 (ну и с 1) и т.д.
    Экспериментатор Глюк прикладывает постоянную горизонтальную силу $90\,\text{Н}$ к бруску с наименьшим номером.
    С каким ускорением двигается система? Чему равна сила натяжения нити, связывающей бруски 1 и 2?
}
\answer{%
    \begin{align*}
    a &= \frac{F}{5 m} = \frac{90\,\text{Н}}{5 \cdot 2\,\text{кг}} \approx 9{,}0\,\frac{\text{м}}{\text{c}^{2}}, \\
    T &= m'a = 4m \cdot \frac{F}{5 m} = \frac{4}{5} F \approx 72{,}0\,\text{Н}.
    \end{align*}
}
\solutionspace{120pt}

\tasknumber{6}%
\task{%
    Два бруска связаны лёгкой нерастяжимой нитью и перекинуты через неподвижный блок (см.
    рис.).
    Определите силу натяжения нити и ускорения брусков.
    Силами трения пренебречь, массы брусков
    равны $m_1 = 8\,\text{кг}$ и $m_2 = 10\,\text{кг}$.
    % $g = 10\,\frac{\text{м}}{\text{с}^{2}}$.

    \begin{tikzpicture}[x=1.5cm,y=1.5cm,thick]
        \draw
            (-0.4, 0) rectangle (-0.2, 1.2)
            (0.15, 0.5) rectangle (0.45, 1)
            (0, 2) circle [radius=0.3] -- ++(up:0.5)
            (-0.3, 1.2) -- ++(up:0.8)
            (0.3, 1) -- ++(up:1)
            (-0.7, 2.5) -- (0.7, 2.5)
            ;
        \draw[pattern={Lines[angle=51,distance=3pt]},pattern color=black,draw=none] (-0.7, 2.5) rectangle (0.7, 2.75);
        \node [left] (left) at (-0.4, 0.6) { $m_1$ };
        \node [right] (right) at (0.4, 0.75) { $m_2$ };
    \end{tikzpicture}
}
\answer{%
    Предположим, что левый брусок ускоряется вверх, тогда правый ускоряется вниз (с тем же ускорением).
    Запишем 2-й закон Ньютона 2 раза (для обоих тел) в проекции на вертикальную оси, направив её вверх.
    \begin{align*}
        &\begin{cases}
            T - m_1g = m_1a, \\
            T - m_2g = -m_2a,
        \end{cases} \\
        &\begin{cases}
            m_2g - m_1g = m_1a + m_2a, \\
            T = m_1a + m_1g, \\
        \end{cases} \\
        a &= \frac{m_2 - m_1}{m_1 + m_2} \cdot g = \frac{10\,\text{кг} - 8\,\text{кг}}{8\,\text{кг} + 10\,\text{кг}} \cdot 10\,\frac{\text{м}}{\text{с}^{2}} \approx 1{,}11\,\frac{\text{м}}{\text{c}^{2}}, \\
        T &= m_1(a + g) = m_1 \cdot g \cdot \cbr{\frac{m_2 - m_1}{m_1 + m_2} + 1} = m_1 \cdot g \cdot \frac{2m_2}{m_1 + m_2} = \\
            &= \frac{2 m_2 m_1 g}{m_1 + m_2} = \frac{2 \cdot 10\,\text{кг} \cdot 8\,\text{кг} \cdot 10\,\frac{\text{м}}{\text{с}^{2}}}{8\,\text{кг} + 10\,\text{кг}} \approx 88{,}9\,\text{Н}.
    \end{align*}
    Отрицательный ответ говорит, что мы лишь не угадали с направлением ускорений.
    Сила же всегда положительна.
}
\solutionspace{80pt}

\tasknumber{7}%
\task{%
    Тело массой $1{,}4\,\text{кг}$ лежит на горизонтальной поверхности.
    Коэффициент трения между поверхностью и телом $0{,}2$.
    К телу приложена горизонтальная сила $4{,}5\,\text{Н}$.
    Определите силу трения, действующую на тело, и ускорение тела.
    % $g = 10\,\frac{\text{м}}{\text{с}^{2}}$.
}
\answer{%
    \begin{align*}
    &F_\text{ трения покоя $\max$ } = \mu N = \mu m g = 0{,}2 \cdot 1{,}4\,\text{кг} \cdot 10\,\frac{\text{м}}{\text{с}^{2}} = 2{,}80\,\text{Н}, \\
    &F_\text{ трения покоя $\max$ } \le F \implies F_\text{ трения } = 2{,}80\,\text{Н}, a = \frac{F - F_\text{ трения }}{ m } = 1{,}21\,\frac{\text{м}}{\text{c}^{2}}, \\
    &\text{при равенстве возможны оба варианта: и едет, и не едет, но на ответы это не влияет.}
    \end{align*}
}
\solutionspace{120pt}

\tasknumber{8}%
\task{%
    Определите плотность неизвестного вещества, если известно, что опускании тела из него
    в подсолнечное масло оно будет плавать и на четверть выступать над поверхностью жидкости.
}
\answer{%
    $F_\text{Арх.} = F_\text{тяж.} \implies \rho_\text{ж.} g V_\text{погр.} = m g \implies\rho_\text{ж.} g \cbr{V -\frac V4} = \rho V g \implies \rho = \rho_\text{ж.}\cbr{1 -\frac 14} \approx 675\,\frac{\text{кг}}{\text{м}^{3}}$
}
\solutionspace{120pt}

\tasknumber{9}%
\task{%
    	Определите силу, действующую на правую опору однородного горизонтального стержня длиной $l = 9\,\text{м}$
    	и массой $M = 1\,\text{кг}$, к которому подвешен груз массой $m = 4\,\text{кг}$ на расстоянии $4\,\text{м}$ от правого конца (см.
    рис.).

        \begin{tikzpicture}[thick]
            \draw
                (-2, -0.1) rectangle (2, 0.1)
                (-0.5, -0.1) -- (-0.5, -1)
                (-0.7, -1) rectangle (-0.3, -1.3)
           		(-2, -0.1) -- +(0.15,-0.9) -- +(-0.15,-0.9) -- cycle
            	(2, -0.1) -- +(0.15,-0.9) -- +(-0.15,-0.9) -- cycle
            ;
            \draw[pattern={Lines[angle=51,distance=2pt]},pattern color=black,draw=none]
            	(-2.15, -1.15) rectangle +(0.3, 0.15)
            	(2.15, -1.15) rectangle +(-0.3, 0.15)
            ;
            \node [right] (m_small) at (-0.3, -1.15) { $m$ };
            \node [above] (M_big) at (0, 0.1) { $M$ };
        \end{tikzpicture}
}
\answer{%
    \begin{align*}
        &\begin{cases}
            F_1 + F_2 - mg - Mg= 0, \\
            F_1 \cdot 0 - mg \cdot a - Mg \cdot \frac l2 + F_2 \cdot l = 0,
        \end{cases} \\
        F_2 &= \frac{mga + Mg\frac l2}l = \frac al \cdot mg + \frac{Mg}2 \approx 27{,}2\,\text{Н}, \\
        F_1 &= mg + Mg - F_2 = mg + Mg - \frac al \cdot mg - \frac{Mg}2 = \frac bl \cdot mg + \frac{Mg}2 \approx 22{,}8\,\text{Н}.
    \end{align*}
}
\solutionspace{80pt}

\tasknumber{10}%
\task{%
    Тонкий однородный лом длиной $2\,\text{м}$ и массой $10\,\text{кг}$ лежит на горизонтальной поверхности.
    \begin{itemize}
        \item Какую минимальную силу надо приложить к одному из его концов, чтобы оторвать его от этой поверхности?
        \item Какую минимальную работу надо совершить, чтобы поставить его на землю в вертикальное положение?
    \end{itemize}
    % Примите $g = 10\,\frac{\text{м}}{\text{с}^{2}}$.
}
\answer{%
    $F = \frac{mg}2 \approx 100\,\text{Н}, A = mg\frac l2 = 100\,\text{Дж}$
}
\solutionspace{120pt}

\tasknumber{11}%
\task{%
    Определите работу силы, которая обеспечит спуск тела массой $5\,\text{кг}$ на высоту $10\,\text{м}$ с постоянным ускорением $3\,\frac{\text{м}}{\text{c}^{2}}$.
    % Примите $g = 10\,\frac{\text{м}}{\text{с}^{2}}$.
}
\answer{%
    \begin{align*}
    &\text{Для подъёма:} A = Fh = (mg + ma) h = m(g+a)h, \\
    &\text{Для спуска:} A = -Fh = -(mg - ma) h = -m(g-a)h, \\
    &\text{В результате получаем:} -350\,\text{Дж}.
    \end{align*}
}
\solutionspace{60pt}

\tasknumber{12}%
\task{%
    Тело бросили вертикально вверх со скоростью $14\,\frac{\text{м}}{\text{c}}$.
    На какой высоте кинетическая энергия тела составит треть от потенциальной?
}
\answer{%
    \begin{align*}
    &0 + \frac{mv_0^2}2 = E_p + E_k, E_k = \frac 13 E_p \implies \\
    &\implies \frac{mv_0^2}2 = E_p + \frac 13 E_p = E_p\cbr{1 + \frac 13} = mgh\cbr{1 + \frac 13} \implies \\
    &\implies h = \frac{\frac{mv_0^2}2}{mg\cbr{1 + \frac 13}} = \frac{v_0^2}{2g} \cdot \frac 1{1 + \frac 13} \approx 7{,}4\,\text{м}.
    \end{align*}
}
\solutionspace{100pt}

\tasknumber{13}%
\task{%
    Плотность воздуха при нормальных условиях равна $1{,}3\,\frac{\text{кг}}{\text{м}^{3}}$.
    Чему равна плотность воздуха
    при температуре $150\celsius$ и давлении $80\,\text{кПа}$?
}
\answer{%
    \begin{align*}
    &\text{В общем случае:} PV = \frac m{\mu} RT \implies \rho = \frac mV = \frac m{\frac{\frac m{\mu} RT}P} = \frac{P\mu}{RT}, \\
    &\text{У нас 2 состояния:} \rho_1 = \frac{P_1\mu}{RT_1}, \rho_2 = \frac{P_2\mu}{RT_2} \implies \frac{\rho_2}{\rho_1} = \frac{\frac{P_2\mu}{RT_2}}{\frac{P_1\mu}{RT_1}} = \frac{P_2T_1}{P_1T_2} \implies \\
    &\implies \rho_2 = \rho_1 \cdot  \frac{P_2T_1}{P_1T_2} = 1{,}3\,\frac{\text{кг}}{\text{м}^{3}} \cdot \frac{80\,\text{кПа} \cdot 273\units{К}}{100\,\text{кПа} \cdot 423\units{К}} \approx 0{,}67\,\frac{\text{кг}}{\text{м}^{3}}.
    \end{align*}
}
\solutionspace{120pt}

\tasknumber{14}%
\task{%
    Небольшую цилиндрическую пробирку с воздухом погружают на некоторую глубину в глубокое пресное озеро,
    после чего воздух занимает в ней лишь третью часть от общего объема.
    Определите глубину, на которую погрузили пробирку.
    Температуру считать постоянной $T = 290\,\text{К}$, давлением паров воды пренебречь,
    атмосферное давление принять равным $p_{\text{aтм}} = 100\,\text{кПа}$.
}
\answer{%
    \begin{align*}
    T\text{— const} &\implies P_1V_1 = \nu RT = P_2V_2.
    \\
    V_2 = \frac 13 V_1 &\implies P_1V_1 = P_2 \cdot \frac 13V_1 \implies P_2 = 3P_1 = 3p_{\text{aтм}}.
    \\
    P_2 = p_{\text{aтм}} + \rho_{\text{в}} g h \implies h = \frac{P_2 - p_{\text{aтм}}}{\rho_{\text{в}} g} &= \frac{3p_{\text{aтм}} - p_{\text{aтм}}}{\rho_{\text{в}} g} = \frac{2 \cdot p_{\text{aтм}}}{\rho_{\text{в}} g} =  \\
     &= \frac{2 \cdot 100\,\text{кПа}}{1000\,\frac{\text{кг}}{\text{м}^{3}} \cdot  10\,\frac{\text{м}}{\text{с}^{2}}} \approx 20\,\text{м}.
    \end{align*}
}
\solutionspace{120pt}

\tasknumber{15}%
\task{%
    Газу сообщили некоторое количество теплоты,
    при этом четверть его он потратил на совершение работы,
    одновременно увеличив свою внутреннюю энергию на $3000\,\text{Дж}$.
    Определите работу, совершённую газом.
}
\answer{%
    \begin{align*}
    Q &= A' + \Delta U, A' = \frac 14 Q \implies Q \cdot \cbr{1 - \frac 14} = \Delta U \implies Q = \frac{\Delta U}{1 - \frac 14} = \frac{ 3000\,\text{Дж} }{1 - \frac 14} \approx 4000\,\text{Дж}.
    \\
    A' &= \frac 14 Q
        = \frac 14 \cdot \frac{\Delta U}{1 - \frac 14}
        = \frac{\Delta U}{4 - 1}
        = \frac{ 3000\,\text{Дж} }{4 - 1} \approx 1000\,\text{Дж}.
    \end{align*}
}
\solutionspace{60pt}

\tasknumber{16}%
\task{%
    Два конденсатора ёмкостей $C_1 = 40\,\text{нФ}$ и $C_2 = 60\,\text{нФ}$ последовательно подключают
    к источнику напряжения $V = 200\,\text{В}$ (см.
    рис.).
    % Определите заряды каждого из конденсаторов.
    Определите заряд первого конденсатора.

    \begin{tikzpicture}[circuit ee IEC, semithick]
        \draw  (0, 0) to [capacitor={info={$C_1$}}] (1, 0)
                       to [capacitor={info={$C_2$}}] (2, 0)
        ;
        % \draw [-o] (0, 0) -- ++(-0.5, 0) node[left] {$-$};
        % \draw [-o] (2, 0) -- ++(0.5, 0) node[right] {$+$};
        \draw [-o] (0, 0) -- ++(-0.5, 0) node[left] {};
        \draw [-o] (2, 0) -- ++(0.5, 0) node[right] {};
    \end{tikzpicture}
}
\answer{%
    $
        Q_1
            = Q_2
            = CV
            = \frac{ V }{\frac1{C_1} + \frac1{C_2}}
            = \frac{C_1C_2V}{C_1 + C_2}
            = \frac{
                40\,\text{нФ} \cdot 60\,\text{нФ} \cdot 200\,\text{В}
            }{
                40\,\text{нФ} + 60\,\text{нФ}
            }
            = 4{,}80\,\text{мкКл}
    $
}
\solutionspace{120pt}

\tasknumber{17}%
\task{%
    В вакууме вдоль одной прямой расположены три отрицательных заряда так,
    что расстояние между соседними зарядами равно $l$.
    Сделайте рисунок,
    и определите силу, действующую на крайний заряд.
    Модули всех зарядов равны $Q$ ($Q > 0$).
}
\answer{%
    $F = \sum_i F_i = \ldots = \frac54 \frac{kQ^2}{l^2}.$
}
\solutionspace{80pt}

\tasknumber{18}%
\task{%
    Юлия проводит эксперименты c 2 кусками одинаковой алюминиевой проволки, причём второй кусок в семь раз длиннее первого.
    В одном из экспериментов Юлия подаёт на первый кусок проволки напряжение в восемь раз раз больше, чем на второй.
    Определите отношения в двух проволках в этом эксперименте (второй к первой):
    \begin{itemize}
        \item отношение сил тока,
        \item отношение выделяющихся мощностей.
    \end{itemize}
}
\answer{%
    $R_2 = 7R_1, U_1 = 8U_2 \implies  \eli_2 / \eli_1 = \frac{U_2 / R_2}{U_1 / R_1} = \frac{U_2}{U_1} \cdot \frac{R_1}{R_2} = \frac1{56}, P_2 / P_1 = \frac{U_2^2 / R_2}{U_1^2 / R_1} = \sqr{\frac{U_2}{U_1}} \cdot \frac{R_1}{R_2} = \frac1{448}.$
}

\variantsplitter

\addpersonalvariant{Варвара Минаева}

\tasknumber{1}%
\task{%
    Валя стартует на лошади и в течение $t = 4\,\text{c}$ двигается с постоянным ускорением $0{,}5\,\frac{\text{м}}{\text{с}^{2}}$.
    Определите
    \begin{itemize}
        \item какую скорость при этом удастся достичь,
        \item какой путь за это время будет пройден,
        \item среднюю скорость за всё время движения, если после начального ускорения продолжить движение равномерно ещё в течение времени $3t$
    \end{itemize}
}
\answer{%
    \begin{align*}
    v &= v_0 + a t = at = 0{,}5\,\frac{\text{м}}{\text{с}^{2}} \cdot 4\,\text{c} = 2{,}0\,\frac{\text{м}}{\text{с}}, \\
    s_x &= v_0t + \frac{a t^2}2 = \frac{a t^2}2 = \frac{0{,}5\,\frac{\text{м}}{\text{с}^{2}} \cdot \sqr{ 4\,\text{c} }}2 = 4{,}0\,\text{м}, \\
    v_\text{сред.} &= \frac{s_\text{общ}}{t_\text{общ.}} = \frac{s_x + v \cdot 3t}{t + 3t} = \frac{\frac{a t^2}2 + at \cdot 3t}{t (1 + 3)} = \\
    &= at \cdot \frac{\frac 12 + 3}{1 + 3} = 0{,}5\,\frac{\text{м}}{\text{с}^{2}} \cdot 4\,\text{c} \cdot \frac{\frac 12 + 3}{1 + 3} \approx 1{,}75\,\frac{\text{м}}{\text{c}}.
    \end{align*}
}
\solutionspace{120pt}

\tasknumber{2}%
\task{%
    Какой путь тело пройдёт за вторую секунду после начала свободного падения?
    Какую скорость в начале этой секунды оно имеет?
}
\answer{%
    \begin{align*}
    s &= -s_y = -(y_2-y_1) = y_1 - y_2 = \cbr{y_{0y} + v_{0y}t_1 - \frac{gt_1^2}2} - \cbr{y_{0y} + v_{0y}t_2 - \frac{gt_2^2}2} = \\
    &= \frac{gt_2^2}2 - \frac{gt_1^2}2 = \frac g2\cbr{t_2^2 - t_1^2} = 15{,}0\,\text{м}, \\
    v_y &= v_{0y} - gt = -gt = 10\,\frac{\text{м}}{\text{с}^{2}} \cdot 1\,\text{с} = -10\,\frac{\text{м}}{\text{с}}.
    \end{align*}
}
\solutionspace{120pt}

\tasknumber{3}%
\task{%
    Карусель диаметром $4\,\text{м}$ равномерно совершает 10 оборотов в минуту.
    Определите
    \begin{itemize}
        \item период и частоту её обращения,
        \item скорость и ускорение крайних её точек.
    \end{itemize}
}
\answer{%
    \begin{align*}
    t &= 60\,\text{с}, r = 2{,}0\,\text{м}, n = 10\units{оборотов}, \\
    T &= \frac tN = \frac{ 60\,\text{с} }{10} \approx 6{,}00\,\text{c}, \\
    \nu &= \frac 1T = \frac{10}{ 60\,\text{с} } \approx 0{,}17\,\text{Гц}, \\
    v &= \frac{2 \pi r}{T} = \frac{2 \pi r}{T} =  \frac{2 \pi r n}{t} \approx 2{,}09\,\frac{\text{м}}{\text{c}}, \\
    a &= \frac{v^2}{r} =  \frac{4 \pi^2 r n^2}{t^2} \approx 2{,}19\,\frac{\text{м}}{\text{с}^{2}}.
    \end{align*}
}
\solutionspace{80pt}

\tasknumber{4}%
\task{%
    Миша стоит на обрыве над рекой и методично и строго горизонтально кидает в неё камушки.
    За этим всем наблюдает экспериментатор Глюк, который уже выяснил, что камушки падают в реку спустя $1{,}5\,\text{с}$ после броска,
    а вот дальность полёта оценить сложнее: придётся лезть в воду.
    Выручите Глюка и определите:
    \begin{itemize}
        \item высоту обрыва (вместе с ростом Миши).
        \item дальность полёта камушков (по горизонтали) и их скорость при падении, приняв начальную скорость броска равной $v_0 = 16\,\frac{\text{м}}{\text{с}}$.
    \end{itemize}
    Сопротивлением воздуха пренебречь.
}
\answer{%
    \begin{align*}
    y &= y_0 + v_{0y}t - \frac{gt^2}2 = h - \frac{gt^2}2, \qquad y(\tau) = 0 \implies h - \frac{g\tau^2}2 = 0 \implies h = \frac{g\tau^2}2 \approx 11{,}2\,\text{м}.
    \\
    x &= x_0 + v_{0x}t = v_0t \implies L = v_0\tau \approx 24{,}0\,\text{м}.
    \\
    &v = \sqrt{v_x^2 + v_y^2} = \sqrt{v_{0x}^2 + \sqr{v_{0y} - g\tau}} = \sqrt{v_0^2 + \sqr{g\tau}} \approx 21{,}9\,\frac{\text{м}}{\text{c}}.
    \end{align*}
}
\solutionspace{120pt}

\tasknumber{5}%
\task{%
    Четыре одинаковых брусков массой $3\,\text{кг}$ каждый лежат на гладком горизонтальном столе.
    Бруски пронумерованы от 1 до 4 и последовательно связаны между собой
    невесомыми нерастяжимыми нитями: 1 со 2, 2 с 3 (ну и с 1) и т.д.
    Экспериментатор Глюк прикладывает постоянную горизонтальную силу $60\,\text{Н}$ к бруску с наименьшим номером.
    С каким ускорением двигается система? Чему равна сила натяжения нити, связывающей бруски 3 и 4?
}
\answer{%
    \begin{align*}
    a &= \frac{F}{4 m} = \frac{60\,\text{Н}}{4 \cdot 3\,\text{кг}} \approx 5{,}0\,\frac{\text{м}}{\text{c}^{2}}, \\
    T &= m'a = 1m \cdot \frac{F}{4 m} = \frac{1}{4} F \approx 15{,}0\,\text{Н}.
    \end{align*}
}
\solutionspace{120pt}

\tasknumber{6}%
\task{%
    Два бруска связаны лёгкой нерастяжимой нитью и перекинуты через неподвижный блок (см.
    рис.).
    Определите силу натяжения нити и ускорения брусков.
    Силами трения пренебречь, массы брусков
    равны $m_1 = 8\,\text{кг}$ и $m_2 = 10\,\text{кг}$.
    % $g = 10\,\frac{\text{м}}{\text{с}^{2}}$.

    \begin{tikzpicture}[x=1.5cm,y=1.5cm,thick]
        \draw
            (-0.4, 0) rectangle (-0.2, 1.2)
            (0.15, 0.5) rectangle (0.45, 1)
            (0, 2) circle [radius=0.3] -- ++(up:0.5)
            (-0.3, 1.2) -- ++(up:0.8)
            (0.3, 1) -- ++(up:1)
            (-0.7, 2.5) -- (0.7, 2.5)
            ;
        \draw[pattern={Lines[angle=51,distance=3pt]},pattern color=black,draw=none] (-0.7, 2.5) rectangle (0.7, 2.75);
        \node [left] (left) at (-0.4, 0.6) { $m_1$ };
        \node [right] (right) at (0.4, 0.75) { $m_2$ };
    \end{tikzpicture}
}
\answer{%
    Предположим, что левый брусок ускоряется вверх, тогда правый ускоряется вниз (с тем же ускорением).
    Запишем 2-й закон Ньютона 2 раза (для обоих тел) в проекции на вертикальную оси, направив её вверх.
    \begin{align*}
        &\begin{cases}
            T - m_1g = m_1a, \\
            T - m_2g = -m_2a,
        \end{cases} \\
        &\begin{cases}
            m_2g - m_1g = m_1a + m_2a, \\
            T = m_1a + m_1g, \\
        \end{cases} \\
        a &= \frac{m_2 - m_1}{m_1 + m_2} \cdot g = \frac{10\,\text{кг} - 8\,\text{кг}}{8\,\text{кг} + 10\,\text{кг}} \cdot 10\,\frac{\text{м}}{\text{с}^{2}} \approx 1{,}11\,\frac{\text{м}}{\text{c}^{2}}, \\
        T &= m_1(a + g) = m_1 \cdot g \cdot \cbr{\frac{m_2 - m_1}{m_1 + m_2} + 1} = m_1 \cdot g \cdot \frac{2m_2}{m_1 + m_2} = \\
            &= \frac{2 m_2 m_1 g}{m_1 + m_2} = \frac{2 \cdot 10\,\text{кг} \cdot 8\,\text{кг} \cdot 10\,\frac{\text{м}}{\text{с}^{2}}}{8\,\text{кг} + 10\,\text{кг}} \approx 88{,}9\,\text{Н}.
    \end{align*}
    Отрицательный ответ говорит, что мы лишь не угадали с направлением ускорений.
    Сила же всегда положительна.
}
\solutionspace{80pt}

\tasknumber{7}%
\task{%
    Тело массой $2{,}7\,\text{кг}$ лежит на горизонтальной поверхности.
    Коэффициент трения между поверхностью и телом $0{,}15$.
    К телу приложена горизонтальная сила $3{,}5\,\text{Н}$.
    Определите силу трения, действующую на тело, и ускорение тела.
    % $g = 10\,\frac{\text{м}}{\text{с}^{2}}$.
}
\answer{%
    \begin{align*}
    &F_\text{ трения покоя $\max$ } = \mu N = \mu m g = 0{,}15 \cdot 2{,}7\,\text{кг} \cdot 10\,\frac{\text{м}}{\text{с}^{2}} = 4{,}05\,\text{Н}, \\
    &F_\text{ трения покоя $\max$ } > F \implies F_\text{ трения } = 3{,}50\,\text{Н}, a = \frac{F - F_\text{ трения }}{ m } = 0\,\frac{\text{м}}{\text{c}^{2}}, \\
    &\text{при равенстве возможны оба варианта: и едет, и не едет, но на ответы это не влияет.}
    \end{align*}
}
\solutionspace{120pt}

\tasknumber{8}%
\task{%
    Определите плотность неизвестного вещества, если известно, что опускании тела из него
    в подсолнечное масло оно будет плавать и на четверть выступать над поверхностью жидкости.
}
\answer{%
    $F_\text{Арх.} = F_\text{тяж.} \implies \rho_\text{ж.} g V_\text{погр.} = m g \implies\rho_\text{ж.} g \cbr{V -\frac V4} = \rho V g \implies \rho = \rho_\text{ж.}\cbr{1 -\frac 14} \approx 675\,\frac{\text{кг}}{\text{м}^{3}}$
}
\solutionspace{120pt}

\tasknumber{9}%
\task{%
    	Определите силу, действующую на правую опору однородного горизонтального стержня длиной $l = 5\,\text{м}$
    	и массой $M = 5\,\text{кг}$, к которому подвешен груз массой $m = 2\,\text{кг}$ на расстоянии $4\,\text{м}$ от правого конца (см.
    рис.).

        \begin{tikzpicture}[thick]
            \draw
                (-2, -0.1) rectangle (2, 0.1)
                (-0.5, -0.1) -- (-0.5, -1)
                (-0.7, -1) rectangle (-0.3, -1.3)
           		(-2, -0.1) -- +(0.15,-0.9) -- +(-0.15,-0.9) -- cycle
            	(2, -0.1) -- +(0.15,-0.9) -- +(-0.15,-0.9) -- cycle
            ;
            \draw[pattern={Lines[angle=51,distance=2pt]},pattern color=black,draw=none]
            	(-2.15, -1.15) rectangle +(0.3, 0.15)
            	(2.15, -1.15) rectangle +(-0.3, 0.15)
            ;
            \node [right] (m_small) at (-0.3, -1.15) { $m$ };
            \node [above] (M_big) at (0, 0.1) { $M$ };
        \end{tikzpicture}
}
\answer{%
    \begin{align*}
        &\begin{cases}
            F_1 + F_2 - mg - Mg= 0, \\
            F_1 \cdot 0 - mg \cdot a - Mg \cdot \frac l2 + F_2 \cdot l = 0,
        \end{cases} \\
        F_2 &= \frac{mga + Mg\frac l2}l = \frac al \cdot mg + \frac{Mg}2 \approx 29{,}0\,\text{Н}, \\
        F_1 &= mg + Mg - F_2 = mg + Mg - \frac al \cdot mg - \frac{Mg}2 = \frac bl \cdot mg + \frac{Mg}2 \approx 41{,}0\,\text{Н}.
    \end{align*}
}
\solutionspace{80pt}

\tasknumber{10}%
\task{%
    Тонкий однородный шест длиной $3\,\text{м}$ и массой $10\,\text{кг}$ лежит на горизонтальной поверхности.
    \begin{itemize}
        \item Какую минимальную силу надо приложить к одному из его концов, чтобы оторвать его от этой поверхности?
        \item Какую минимальную работу надо совершить, чтобы поставить его на землю в вертикальное положение?
    \end{itemize}
    % Примите $g = 10\,\frac{\text{м}}{\text{с}^{2}}$.
}
\answer{%
    $F = \frac{mg}2 \approx 100\,\text{Н}, A = mg\frac l2 = 150\,\text{Дж}$
}
\solutionspace{120pt}

\tasknumber{11}%
\task{%
    Определите работу силы, которая обеспечит спуск тела массой $2\,\text{кг}$ на высоту $10\,\text{м}$ с постоянным ускорением $4\,\frac{\text{м}}{\text{c}^{2}}$.
    % Примите $g = 10\,\frac{\text{м}}{\text{с}^{2}}$.
}
\answer{%
    \begin{align*}
    &\text{Для подъёма:} A = Fh = (mg + ma) h = m(g+a)h, \\
    &\text{Для спуска:} A = -Fh = -(mg - ma) h = -m(g-a)h, \\
    &\text{В результате получаем:} -120\,\text{Дж}.
    \end{align*}
}
\solutionspace{60pt}

\tasknumber{12}%
\task{%
    Тело бросили вертикально вверх со скоростью $14\,\frac{\text{м}}{\text{c}}$.
    На какой высоте кинетическая энергия тела составит треть от потенциальной?
}
\answer{%
    \begin{align*}
    &0 + \frac{mv_0^2}2 = E_p + E_k, E_k = \frac 13 E_p \implies \\
    &\implies \frac{mv_0^2}2 = E_p + \frac 13 E_p = E_p\cbr{1 + \frac 13} = mgh\cbr{1 + \frac 13} \implies \\
    &\implies h = \frac{\frac{mv_0^2}2}{mg\cbr{1 + \frac 13}} = \frac{v_0^2}{2g} \cdot \frac 1{1 + \frac 13} \approx 7{,}4\,\text{м}.
    \end{align*}
}
\solutionspace{100pt}

\tasknumber{13}%
\task{%
    Плотность воздуха при нормальных условиях равна $1{,}3\,\frac{\text{кг}}{\text{м}^{3}}$.
    Чему равна плотность воздуха
    при температуре $200\celsius$ и давлении $120\,\text{кПа}$?
}
\answer{%
    \begin{align*}
    &\text{В общем случае:} PV = \frac m{\mu} RT \implies \rho = \frac mV = \frac m{\frac{\frac m{\mu} RT}P} = \frac{P\mu}{RT}, \\
    &\text{У нас 2 состояния:} \rho_1 = \frac{P_1\mu}{RT_1}, \rho_2 = \frac{P_2\mu}{RT_2} \implies \frac{\rho_2}{\rho_1} = \frac{\frac{P_2\mu}{RT_2}}{\frac{P_1\mu}{RT_1}} = \frac{P_2T_1}{P_1T_2} \implies \\
    &\implies \rho_2 = \rho_1 \cdot  \frac{P_2T_1}{P_1T_2} = 1{,}3\,\frac{\text{кг}}{\text{м}^{3}} \cdot \frac{120\,\text{кПа} \cdot 273\units{К}}{100\,\text{кПа} \cdot 473\units{К}} \approx 0{,}90\,\frac{\text{кг}}{\text{м}^{3}}.
    \end{align*}
}
\solutionspace{120pt}

\tasknumber{14}%
\task{%
    Небольшую цилиндрическую пробирку с воздухом погружают на некоторую глубину в глубокое пресное озеро,
    после чего воздух занимает в ней лишь пятую часть от общего объема.
    Определите глубину, на которую погрузили пробирку.
    Температуру считать постоянной $T = 278\,\text{К}$, давлением паров воды пренебречь,
    атмосферное давление принять равным $p_{\text{aтм}} = 100\,\text{кПа}$.
}
\answer{%
    \begin{align*}
    T\text{— const} &\implies P_1V_1 = \nu RT = P_2V_2.
    \\
    V_2 = \frac 15 V_1 &\implies P_1V_1 = P_2 \cdot \frac 15V_1 \implies P_2 = 5P_1 = 5p_{\text{aтм}}.
    \\
    P_2 = p_{\text{aтм}} + \rho_{\text{в}} g h \implies h = \frac{P_2 - p_{\text{aтм}}}{\rho_{\text{в}} g} &= \frac{5p_{\text{aтм}} - p_{\text{aтм}}}{\rho_{\text{в}} g} = \frac{4 \cdot p_{\text{aтм}}}{\rho_{\text{в}} g} =  \\
     &= \frac{4 \cdot 100\,\text{кПа}}{1000\,\frac{\text{кг}}{\text{м}^{3}} \cdot  10\,\frac{\text{м}}{\text{с}^{2}}} \approx 40\,\text{м}.
    \end{align*}
}
\solutionspace{120pt}

\tasknumber{15}%
\task{%
    Газу сообщили некоторое количество теплоты,
    при этом половину его он потратил на совершение работы,
    одновременно увеличив свою внутреннюю энергию на $1500\,\text{Дж}$.
    Определите работу, совершённую газом.
}
\answer{%
    \begin{align*}
    Q &= A' + \Delta U, A' = \frac 12 Q \implies Q \cdot \cbr{1 - \frac 12} = \Delta U \implies Q = \frac{\Delta U}{1 - \frac 12} = \frac{ 1500\,\text{Дж} }{1 - \frac 12} \approx 3000\,\text{Дж}.
    \\
    A' &= \frac 12 Q
        = \frac 12 \cdot \frac{\Delta U}{1 - \frac 12}
        = \frac{\Delta U}{2 - 1}
        = \frac{ 1500\,\text{Дж} }{2 - 1} \approx 1500\,\text{Дж}.
    \end{align*}
}
\solutionspace{60pt}

\tasknumber{16}%
\task{%
    Два конденсатора ёмкостей $C_1 = 60\,\text{нФ}$ и $C_2 = 30\,\text{нФ}$ последовательно подключают
    к источнику напряжения $V = 400\,\text{В}$ (см.
    рис.).
    % Определите заряды каждого из конденсаторов.
    Определите заряд первого конденсатора.

    \begin{tikzpicture}[circuit ee IEC, semithick]
        \draw  (0, 0) to [capacitor={info={$C_1$}}] (1, 0)
                       to [capacitor={info={$C_2$}}] (2, 0)
        ;
        % \draw [-o] (0, 0) -- ++(-0.5, 0) node[left] {$-$};
        % \draw [-o] (2, 0) -- ++(0.5, 0) node[right] {$+$};
        \draw [-o] (0, 0) -- ++(-0.5, 0) node[left] {};
        \draw [-o] (2, 0) -- ++(0.5, 0) node[right] {};
    \end{tikzpicture}
}
\answer{%
    $
        Q_1
            = Q_2
            = CV
            = \frac{ V }{\frac1{C_1} + \frac1{C_2}}
            = \frac{C_1C_2V}{C_1 + C_2}
            = \frac{
                60\,\text{нФ} \cdot 30\,\text{нФ} \cdot 400\,\text{В}
            }{
                60\,\text{нФ} + 30\,\text{нФ}
            }
            = 8{,}00\,\text{мкКл}
    $
}
\solutionspace{120pt}

\tasknumber{17}%
\task{%
    В вакууме вдоль одной прямой расположены три положительных заряда так,
    что расстояние между соседними зарядами равно $d$.
    Сделайте рисунок,
    и определите силу, действующую на крайний заряд.
    Модули всех зарядов равны $q$ ($q > 0$).
}
\answer{%
    $F = \sum_i F_i = \ldots = \frac54 \frac{kq^2}{d^2}.$
}
\solutionspace{80pt}

\tasknumber{18}%
\task{%
    Юлия проводит эксперименты c 2 кусками одинаковой алюминиевой проволки, причём второй кусок в восемь раз длиннее первого.
    В одном из экспериментов Юлия подаёт на первый кусок проволки напряжение в четыре раза раз больше, чем на второй.
    Определите отношения в двух проволках в этом эксперименте (второй к первой):
    \begin{itemize}
        \item отношение сил тока,
        \item отношение выделяющихся мощностей.
    \end{itemize}
}
\answer{%
    $R_2 = 8R_1, U_1 = 4U_2 \implies  \eli_2 / \eli_1 = \frac{U_2 / R_2}{U_1 / R_1} = \frac{U_2}{U_1} \cdot \frac{R_1}{R_2} = \frac1{32}, P_2 / P_1 = \frac{U_2^2 / R_2}{U_1^2 / R_1} = \sqr{\frac{U_2}{U_1}} \cdot \frac{R_1}{R_2} = \frac1{128}.$
}

\variantsplitter

\addpersonalvariant{Леонид Никитин}

\tasknumber{1}%
\task{%
    Саша стартует на велосипеде и в течение $t = 4\,\text{c}$ двигается с постоянным ускорением $2{,}5\,\frac{\text{м}}{\text{с}^{2}}$.
    Определите
    \begin{itemize}
        \item какую скорость при этом удастся достичь,
        \item какой путь за это время будет пройден,
        \item среднюю скорость за всё время движения, если после начального ускорения продолжить движение равномерно ещё в течение времени $2t$
    \end{itemize}
}
\answer{%
    \begin{align*}
    v &= v_0 + a t = at = 2{,}5\,\frac{\text{м}}{\text{с}^{2}} \cdot 4\,\text{c} = 10{,}0\,\frac{\text{м}}{\text{с}}, \\
    s_x &= v_0t + \frac{a t^2}2 = \frac{a t^2}2 = \frac{2{,}5\,\frac{\text{м}}{\text{с}^{2}} \cdot \sqr{ 4\,\text{c} }}2 = 20{,}0\,\text{м}, \\
    v_\text{сред.} &= \frac{s_\text{общ}}{t_\text{общ.}} = \frac{s_x + v \cdot 2t}{t + 2t} = \frac{\frac{a t^2}2 + at \cdot 2t}{t (1 + 2)} = \\
    &= at \cdot \frac{\frac 12 + 2}{1 + 2} = 2{,}5\,\frac{\text{м}}{\text{с}^{2}} \cdot 4\,\text{c} \cdot \frac{\frac 12 + 2}{1 + 2} \approx 8{,}33\,\frac{\text{м}}{\text{c}}.
    \end{align*}
}
\solutionspace{120pt}

\tasknumber{2}%
\task{%
    Какой путь тело пройдёт за третью секунду после начала свободного падения?
    Какую скорость в конце этой секунды оно имеет?
}
\answer{%
    \begin{align*}
    s &= -s_y = -(y_2-y_1) = y_1 - y_2 = \cbr{y_{0y} + v_{0y}t_1 - \frac{gt_1^2}2} - \cbr{y_{0y} + v_{0y}t_2 - \frac{gt_2^2}2} = \\
    &= \frac{gt_2^2}2 - \frac{gt_1^2}2 = \frac g2\cbr{t_2^2 - t_1^2} = 25{,}0\,\text{м}, \\
    v_y &= v_{0y} - gt = -gt = 10\,\frac{\text{м}}{\text{с}^{2}} \cdot 3\,\text{с} = -30\,\frac{\text{м}}{\text{с}}.
    \end{align*}
}
\solutionspace{120pt}

\tasknumber{3}%
\task{%
    Карусель радиусом $5\,\text{м}$ равномерно совершает 6 оборотов в минуту.
    Определите
    \begin{itemize}
        \item период и частоту её обращения,
        \item скорость и ускорение крайних её точек.
    \end{itemize}
}
\answer{%
    \begin{align*}
    t &= 60\,\text{с}, r = 5{,}0\,\text{м}, n = 6\units{оборотов}, \\
    T &= \frac tN = \frac{ 60\,\text{с} }{6} \approx 10{,}00\,\text{c}, \\
    \nu &= \frac 1T = \frac{6}{ 60\,\text{с} } \approx 0{,}10\,\text{Гц}, \\
    v &= \frac{2 \pi r}{T} = \frac{2 \pi r}{T} =  \frac{2 \pi r n}{t} \approx 3{,}14\,\frac{\text{м}}{\text{c}}, \\
    a &= \frac{v^2}{r} =  \frac{4 \pi^2 r n^2}{t^2} \approx 1{,}97\,\frac{\text{м}}{\text{с}^{2}}.
    \end{align*}
}
\solutionspace{80pt}

\tasknumber{4}%
\task{%
    Маша стоит на обрыве над рекой и методично и строго горизонтально кидает в неё камушки.
    За этим всем наблюдает экспериментатор Глюк, который уже выяснил, что камушки падают в реку спустя $1{,}5\,\text{с}$ после броска,
    а вот дальность полёта оценить сложнее: придётся лезть в воду.
    Выручите Глюка и определите:
    \begin{itemize}
        \item высоту обрыва (вместе с ростом Маши).
        \item дальность полёта камушков (по горизонтали) и их скорость при падении, приняв начальную скорость броска равной $v_0 = 18\,\frac{\text{м}}{\text{с}}$.
    \end{itemize}
    Сопротивлением воздуха пренебречь.
}
\answer{%
    \begin{align*}
    y &= y_0 + v_{0y}t - \frac{gt^2}2 = h - \frac{gt^2}2, \qquad y(\tau) = 0 \implies h - \frac{g\tau^2}2 = 0 \implies h = \frac{g\tau^2}2 \approx 11{,}2\,\text{м}.
    \\
    x &= x_0 + v_{0x}t = v_0t \implies L = v_0\tau \approx 27{,}0\,\text{м}.
    \\
    &v = \sqrt{v_x^2 + v_y^2} = \sqrt{v_{0x}^2 + \sqr{v_{0y} - g\tau}} = \sqrt{v_0^2 + \sqr{g\tau}} \approx 23{,}4\,\frac{\text{м}}{\text{c}}.
    \end{align*}
}
\solutionspace{120pt}

\tasknumber{5}%
\task{%
    Шесть одинаковых брусков массой $2\,\text{кг}$ каждый лежат на гладком горизонтальном столе.
    Бруски пронумерованы от 1 до 6 и последовательно связаны между собой
    невесомыми нерастяжимыми нитями: 1 со 2, 2 с 3 (ну и с 1) и т.д.
    Экспериментатор Глюк прикладывает постоянную горизонтальную силу $90\,\text{Н}$ к бруску с наименьшим номером.
    С каким ускорением двигается система? Чему равна сила натяжения нити, связывающей бруски 2 и 3?
}
\answer{%
    \begin{align*}
    a &= \frac{F}{6 m} = \frac{90\,\text{Н}}{6 \cdot 2\,\text{кг}} \approx 7{,}5\,\frac{\text{м}}{\text{c}^{2}}, \\
    T &= m'a = 4m \cdot \frac{F}{6 m} = \frac{4}{6} F \approx 60{,}0\,\text{Н}.
    \end{align*}
}
\solutionspace{120pt}

\tasknumber{6}%
\task{%
    Два бруска связаны лёгкой нерастяжимой нитью и перекинуты через неподвижный блок (см.
    рис.).
    Определите силу натяжения нити и ускорения брусков.
    Силами трения пренебречь, массы брусков
    равны $m_1 = 11\,\text{кг}$ и $m_2 = 4\,\text{кг}$.
    % $g = 10\,\frac{\text{м}}{\text{с}^{2}}$.

    \begin{tikzpicture}[x=1.5cm,y=1.5cm,thick]
        \draw
            (-0.4, 0) rectangle (-0.2, 1.2)
            (0.15, 0.5) rectangle (0.45, 1)
            (0, 2) circle [radius=0.3] -- ++(up:0.5)
            (-0.3, 1.2) -- ++(up:0.8)
            (0.3, 1) -- ++(up:1)
            (-0.7, 2.5) -- (0.7, 2.5)
            ;
        \draw[pattern={Lines[angle=51,distance=3pt]},pattern color=black,draw=none] (-0.7, 2.5) rectangle (0.7, 2.75);
        \node [left] (left) at (-0.4, 0.6) { $m_1$ };
        \node [right] (right) at (0.4, 0.75) { $m_2$ };
    \end{tikzpicture}
}
\answer{%
    Предположим, что левый брусок ускоряется вверх, тогда правый ускоряется вниз (с тем же ускорением).
    Запишем 2-й закон Ньютона 2 раза (для обоих тел) в проекции на вертикальную оси, направив её вверх.
    \begin{align*}
        &\begin{cases}
            T - m_1g = m_1a, \\
            T - m_2g = -m_2a,
        \end{cases} \\
        &\begin{cases}
            m_2g - m_1g = m_1a + m_2a, \\
            T = m_1a + m_1g, \\
        \end{cases} \\
        a &= \frac{m_2 - m_1}{m_1 + m_2} \cdot g = \frac{4\,\text{кг} - 11\,\text{кг}}{11\,\text{кг} + 4\,\text{кг}} \cdot 10\,\frac{\text{м}}{\text{с}^{2}} \approx -4{,}670\,\frac{\text{м}}{\text{c}^{2}}, \\
        T &= m_1(a + g) = m_1 \cdot g \cdot \cbr{\frac{m_2 - m_1}{m_1 + m_2} + 1} = m_1 \cdot g \cdot \frac{2m_2}{m_1 + m_2} = \\
            &= \frac{2 m_2 m_1 g}{m_1 + m_2} = \frac{2 \cdot 4\,\text{кг} \cdot 11\,\text{кг} \cdot 10\,\frac{\text{м}}{\text{с}^{2}}}{11\,\text{кг} + 4\,\text{кг}} \approx 58{,}7\,\text{Н}.
    \end{align*}
    Отрицательный ответ говорит, что мы лишь не угадали с направлением ускорений.
    Сила же всегда положительна.
}
\solutionspace{80pt}

\tasknumber{7}%
\task{%
    Тело массой $1{,}4\,\text{кг}$ лежит на горизонтальной поверхности.
    Коэффициент трения между поверхностью и телом $0{,}2$.
    К телу приложена горизонтальная сила $3{,}5\,\text{Н}$.
    Определите силу трения, действующую на тело, и ускорение тела.
    % $g = 10\,\frac{\text{м}}{\text{с}^{2}}$.
}
\answer{%
    \begin{align*}
    &F_\text{ трения покоя $\max$ } = \mu N = \mu m g = 0{,}2 \cdot 1{,}4\,\text{кг} \cdot 10\,\frac{\text{м}}{\text{с}^{2}} = 2{,}80\,\text{Н}, \\
    &F_\text{ трения покоя $\max$ } \le F \implies F_\text{ трения } = 2{,}80\,\text{Н}, a = \frac{F - F_\text{ трения }}{ m } = 0{,}50\,\frac{\text{м}}{\text{c}^{2}}, \\
    &\text{при равенстве возможны оба варианта: и едет, и не едет, но на ответы это не влияет.}
    \end{align*}
}
\solutionspace{120pt}

\tasknumber{8}%
\task{%
    Определите плотность неизвестного вещества, если известно, что опускании тела из него
    в подсолнечное масло оно будет плавать и на четверть выступать над поверхностью жидкости.
}
\answer{%
    $F_\text{Арх.} = F_\text{тяж.} \implies \rho_\text{ж.} g V_\text{погр.} = m g \implies\rho_\text{ж.} g \cbr{V -\frac V4} = \rho V g \implies \rho = \rho_\text{ж.}\cbr{1 -\frac 14} \approx 675\,\frac{\text{кг}}{\text{м}^{3}}$
}
\solutionspace{120pt}

\tasknumber{9}%
\task{%
    	Определите силу, действующую на правую опору однородного горизонтального стержня длиной $l = 9\,\text{м}$
    	и массой $M = 1\,\text{кг}$, к которому подвешен груз массой $m = 3\,\text{кг}$ на расстоянии $4\,\text{м}$ от правого конца (см.
    рис.).

        \begin{tikzpicture}[thick]
            \draw
                (-2, -0.1) rectangle (2, 0.1)
                (-0.5, -0.1) -- (-0.5, -1)
                (-0.7, -1) rectangle (-0.3, -1.3)
           		(-2, -0.1) -- +(0.15,-0.9) -- +(-0.15,-0.9) -- cycle
            	(2, -0.1) -- +(0.15,-0.9) -- +(-0.15,-0.9) -- cycle
            ;
            \draw[pattern={Lines[angle=51,distance=2pt]},pattern color=black,draw=none]
            	(-2.15, -1.15) rectangle +(0.3, 0.15)
            	(2.15, -1.15) rectangle +(-0.3, 0.15)
            ;
            \node [right] (m_small) at (-0.3, -1.15) { $m$ };
            \node [above] (M_big) at (0, 0.1) { $M$ };
        \end{tikzpicture}
}
\answer{%
    \begin{align*}
        &\begin{cases}
            F_1 + F_2 - mg - Mg= 0, \\
            F_1 \cdot 0 - mg \cdot a - Mg \cdot \frac l2 + F_2 \cdot l = 0,
        \end{cases} \\
        F_2 &= \frac{mga + Mg\frac l2}l = \frac al \cdot mg + \frac{Mg}2 \approx 21{,}7\,\text{Н}, \\
        F_1 &= mg + Mg - F_2 = mg + Mg - \frac al \cdot mg - \frac{Mg}2 = \frac bl \cdot mg + \frac{Mg}2 \approx 18{,}3\,\text{Н}.
    \end{align*}
}
\solutionspace{80pt}

\tasknumber{10}%
\task{%
    Тонкий однородный лом длиной $3\,\text{м}$ и массой $20\,\text{кг}$ лежит на горизонтальной поверхности.
    \begin{itemize}
        \item Какую минимальную силу надо приложить к одному из его концов, чтобы оторвать его от этой поверхности?
        \item Какую минимальную работу надо совершить, чтобы поставить его на землю в вертикальное положение?
    \end{itemize}
    % Примите $g = 10\,\frac{\text{м}}{\text{с}^{2}}$.
}
\answer{%
    $F = \frac{mg}2 \approx 200\,\text{Н}, A = mg\frac l2 = 300\,\text{Дж}$
}
\solutionspace{120pt}

\tasknumber{11}%
\task{%
    Определите работу силы, которая обеспечит спуск тела массой $5\,\text{кг}$ на высоту $10\,\text{м}$ с постоянным ускорением $6\,\frac{\text{м}}{\text{c}^{2}}$.
    % Примите $g = 10\,\frac{\text{м}}{\text{с}^{2}}$.
}
\answer{%
    \begin{align*}
    &\text{Для подъёма:} A = Fh = (mg + ma) h = m(g+a)h, \\
    &\text{Для спуска:} A = -Fh = -(mg - ma) h = -m(g-a)h, \\
    &\text{В результате получаем:} -200\,\text{Дж}.
    \end{align*}
}
\solutionspace{60pt}

\tasknumber{12}%
\task{%
    Тело бросили вертикально вверх со скоростью $20\,\frac{\text{м}}{\text{c}}$.
    На какой высоте кинетическая энергия тела составит треть от потенциальной?
}
\answer{%
    \begin{align*}
    &0 + \frac{mv_0^2}2 = E_p + E_k, E_k = \frac 13 E_p \implies \\
    &\implies \frac{mv_0^2}2 = E_p + \frac 13 E_p = E_p\cbr{1 + \frac 13} = mgh\cbr{1 + \frac 13} \implies \\
    &\implies h = \frac{\frac{mv_0^2}2}{mg\cbr{1 + \frac 13}} = \frac{v_0^2}{2g} \cdot \frac 1{1 + \frac 13} \approx 15{,}0\,\text{м}.
    \end{align*}
}
\solutionspace{100pt}

\tasknumber{13}%
\task{%
    Плотность воздуха при нормальных условиях равна $1{,}3\,\frac{\text{кг}}{\text{м}^{3}}$.
    Чему равна плотность воздуха
    при температуре $100\celsius$ и давлении $120\,\text{кПа}$?
}
\answer{%
    \begin{align*}
    &\text{В общем случае:} PV = \frac m{\mu} RT \implies \rho = \frac mV = \frac m{\frac{\frac m{\mu} RT}P} = \frac{P\mu}{RT}, \\
    &\text{У нас 2 состояния:} \rho_1 = \frac{P_1\mu}{RT_1}, \rho_2 = \frac{P_2\mu}{RT_2} \implies \frac{\rho_2}{\rho_1} = \frac{\frac{P_2\mu}{RT_2}}{\frac{P_1\mu}{RT_1}} = \frac{P_2T_1}{P_1T_2} \implies \\
    &\implies \rho_2 = \rho_1 \cdot  \frac{P_2T_1}{P_1T_2} = 1{,}3\,\frac{\text{кг}}{\text{м}^{3}} \cdot \frac{120\,\text{кПа} \cdot 273\units{К}}{100\,\text{кПа} \cdot 373\units{К}} \approx 1{,}14\,\frac{\text{кг}}{\text{м}^{3}}.
    \end{align*}
}
\solutionspace{120pt}

\tasknumber{14}%
\task{%
    Небольшую цилиндрическую пробирку с воздухом погружают на некоторую глубину в глубокое пресное озеро,
    после чего воздух занимает в ней лишь четвертую часть от общего объема.
    Определите глубину, на которую погрузили пробирку.
    Температуру считать постоянной $T = 288\,\text{К}$, давлением паров воды пренебречь,
    атмосферное давление принять равным $p_{\text{aтм}} = 100\,\text{кПа}$.
}
\answer{%
    \begin{align*}
    T\text{— const} &\implies P_1V_1 = \nu RT = P_2V_2.
    \\
    V_2 = \frac 14 V_1 &\implies P_1V_1 = P_2 \cdot \frac 14V_1 \implies P_2 = 4P_1 = 4p_{\text{aтм}}.
    \\
    P_2 = p_{\text{aтм}} + \rho_{\text{в}} g h \implies h = \frac{P_2 - p_{\text{aтм}}}{\rho_{\text{в}} g} &= \frac{4p_{\text{aтм}} - p_{\text{aтм}}}{\rho_{\text{в}} g} = \frac{3 \cdot p_{\text{aтм}}}{\rho_{\text{в}} g} =  \\
     &= \frac{3 \cdot 100\,\text{кПа}}{1000\,\frac{\text{кг}}{\text{м}^{3}} \cdot  10\,\frac{\text{м}}{\text{с}^{2}}} \approx 30\,\text{м}.
    \end{align*}
}
\solutionspace{120pt}

\tasknumber{15}%
\task{%
    Газу сообщили некоторое количество теплоты,
    при этом половину его он потратил на совершение работы,
    одновременно увеличив свою внутреннюю энергию на $1200\,\text{Дж}$.
    Определите работу, совершённую газом.
}
\answer{%
    \begin{align*}
    Q &= A' + \Delta U, A' = \frac 12 Q \implies Q \cdot \cbr{1 - \frac 12} = \Delta U \implies Q = \frac{\Delta U}{1 - \frac 12} = \frac{ 1200\,\text{Дж} }{1 - \frac 12} \approx 2400\,\text{Дж}.
    \\
    A' &= \frac 12 Q
        = \frac 12 \cdot \frac{\Delta U}{1 - \frac 12}
        = \frac{\Delta U}{2 - 1}
        = \frac{ 1200\,\text{Дж} }{2 - 1} \approx 1200\,\text{Дж}.
    \end{align*}
}
\solutionspace{60pt}

\tasknumber{16}%
\task{%
    Два конденсатора ёмкостей $C_1 = 20\,\text{нФ}$ и $C_2 = 60\,\text{нФ}$ последовательно подключают
    к источнику напряжения $U = 200\,\text{В}$ (см.
    рис.).
    % Определите заряды каждого из конденсаторов.
    Определите заряд второго конденсатора.

    \begin{tikzpicture}[circuit ee IEC, semithick]
        \draw  (0, 0) to [capacitor={info={$C_1$}}] (1, 0)
                       to [capacitor={info={$C_2$}}] (2, 0)
        ;
        % \draw [-o] (0, 0) -- ++(-0.5, 0) node[left] {$-$};
        % \draw [-o] (2, 0) -- ++(0.5, 0) node[right] {$+$};
        \draw [-o] (0, 0) -- ++(-0.5, 0) node[left] {};
        \draw [-o] (2, 0) -- ++(0.5, 0) node[right] {};
    \end{tikzpicture}
}
\answer{%
    $
        Q_1
            = Q_2
            = CU
            = \frac{ U }{\frac1{C_1} + \frac1{C_2}}
            = \frac{C_1C_2U}{C_1 + C_2}
            = \frac{
                20\,\text{нФ} \cdot 60\,\text{нФ} \cdot 200\,\text{В}
            }{
                20\,\text{нФ} + 60\,\text{нФ}
            }
            = 3{,}00\,\text{мкКл}
    $
}
\solutionspace{120pt}

\tasknumber{17}%
\task{%
    В вакууме вдоль одной прямой расположены три положительных заряда так,
    что расстояние между соседними зарядами равно $d$.
    Сделайте рисунок,
    и определите силу, действующую на крайний заряд.
    Модули всех зарядов равны $Q$ ($Q > 0$).
}
\answer{%
    $F = \sum_i F_i = \ldots = \frac54 \frac{kQ^2}{d^2}.$
}
\solutionspace{80pt}

\tasknumber{18}%
\task{%
    Юлия проводит эксперименты c 2 кусками одинаковой медной проволки, причём второй кусок в шесть раз длиннее первого.
    В одном из экспериментов Юлия подаёт на первый кусок проволки напряжение в два раза раз больше, чем на второй.
    Определите отношения в двух проволках в этом эксперименте (второй к первой):
    \begin{itemize}
        \item отношение сил тока,
        \item отношение выделяющихся мощностей.
    \end{itemize}
}
\answer{%
    $R_2 = 6R_1, U_1 = 2U_2 \implies  \eli_2 / \eli_1 = \frac{U_2 / R_2}{U_1 / R_1} = \frac{U_2}{U_1} \cdot \frac{R_1}{R_2} = \frac1{12}, P_2 / P_1 = \frac{U_2^2 / R_2}{U_1^2 / R_1} = \sqr{\frac{U_2}{U_1}} \cdot \frac{R_1}{R_2} = \frac1{24}.$
}

\variantsplitter

\addpersonalvariant{Тимофей Полетаев}

\tasknumber{1}%
\task{%
    Саша стартует на велосипеде и в течение $t = 4\,\text{c}$ двигается с постоянным ускорением $2{,}5\,\frac{\text{м}}{\text{с}^{2}}$.
    Определите
    \begin{itemize}
        \item какую скорость при этом удастся достичь,
        \item какой путь за это время будет пройден,
        \item среднюю скорость за всё время движения, если после начального ускорения продолжить движение равномерно ещё в течение времени $3t$
    \end{itemize}
}
\answer{%
    \begin{align*}
    v &= v_0 + a t = at = 2{,}5\,\frac{\text{м}}{\text{с}^{2}} \cdot 4\,\text{c} = 10{,}0\,\frac{\text{м}}{\text{с}}, \\
    s_x &= v_0t + \frac{a t^2}2 = \frac{a t^2}2 = \frac{2{,}5\,\frac{\text{м}}{\text{с}^{2}} \cdot \sqr{ 4\,\text{c} }}2 = 20{,}0\,\text{м}, \\
    v_\text{сред.} &= \frac{s_\text{общ}}{t_\text{общ.}} = \frac{s_x + v \cdot 3t}{t + 3t} = \frac{\frac{a t^2}2 + at \cdot 3t}{t (1 + 3)} = \\
    &= at \cdot \frac{\frac 12 + 3}{1 + 3} = 2{,}5\,\frac{\text{м}}{\text{с}^{2}} \cdot 4\,\text{c} \cdot \frac{\frac 12 + 3}{1 + 3} \approx 8{,}75\,\frac{\text{м}}{\text{c}}.
    \end{align*}
}
\solutionspace{120pt}

\tasknumber{2}%
\task{%
    Какой путь тело пройдёт за вторую секунду после начала свободного падения?
    Какую скорость в конце этой секунды оно имеет?
}
\answer{%
    \begin{align*}
    s &= -s_y = -(y_2-y_1) = y_1 - y_2 = \cbr{y_{0y} + v_{0y}t_1 - \frac{gt_1^2}2} - \cbr{y_{0y} + v_{0y}t_2 - \frac{gt_2^2}2} = \\
    &= \frac{gt_2^2}2 - \frac{gt_1^2}2 = \frac g2\cbr{t_2^2 - t_1^2} = 15{,}0\,\text{м}, \\
    v_y &= v_{0y} - gt = -gt = 10\,\frac{\text{м}}{\text{с}^{2}} \cdot 2\,\text{с} = -20\,\frac{\text{м}}{\text{с}}.
    \end{align*}
}
\solutionspace{120pt}

\tasknumber{3}%
\task{%
    Карусель радиусом $3\,\text{м}$ равномерно совершает 10 оборотов в минуту.
    Определите
    \begin{itemize}
        \item период и частоту её обращения,
        \item скорость и ускорение крайних её точек.
    \end{itemize}
}
\answer{%
    \begin{align*}
    t &= 60\,\text{с}, r = 3{,}0\,\text{м}, n = 10\units{оборотов}, \\
    T &= \frac tN = \frac{ 60\,\text{с} }{10} \approx 6{,}00\,\text{c}, \\
    \nu &= \frac 1T = \frac{10}{ 60\,\text{с} } \approx 0{,}17\,\text{Гц}, \\
    v &= \frac{2 \pi r}{T} = \frac{2 \pi r}{T} =  \frac{2 \pi r n}{t} \approx 3{,}14\,\frac{\text{м}}{\text{c}}, \\
    a &= \frac{v^2}{r} =  \frac{4 \pi^2 r n^2}{t^2} \approx 3{,}29\,\frac{\text{м}}{\text{с}^{2}}.
    \end{align*}
}
\solutionspace{80pt}

\tasknumber{4}%
\task{%
    Маша стоит на обрыве над рекой и методично и строго горизонтально кидает в неё камушки.
    За этим всем наблюдает экспериментатор Глюк, который уже выяснил, что камушки падают в реку спустя $1{,}3\,\text{с}$ после броска,
    а вот дальность полёта оценить сложнее: придётся лезть в воду.
    Выручите Глюка и определите:
    \begin{itemize}
        \item высоту обрыва (вместе с ростом Маши).
        \item дальность полёта камушков (по горизонтали) и их скорость при падении, приняв начальную скорость броска равной $v_0 = 13\,\frac{\text{м}}{\text{с}}$.
    \end{itemize}
    Сопротивлением воздуха пренебречь.
}
\answer{%
    \begin{align*}
    y &= y_0 + v_{0y}t - \frac{gt^2}2 = h - \frac{gt^2}2, \qquad y(\tau) = 0 \implies h - \frac{g\tau^2}2 = 0 \implies h = \frac{g\tau^2}2 \approx 8{,}5\,\text{м}.
    \\
    x &= x_0 + v_{0x}t = v_0t \implies L = v_0\tau \approx 16{,}9\,\text{м}.
    \\
    &v = \sqrt{v_x^2 + v_y^2} = \sqrt{v_{0x}^2 + \sqr{v_{0y} - g\tau}} = \sqrt{v_0^2 + \sqr{g\tau}} \approx 18{,}4\,\frac{\text{м}}{\text{c}}.
    \end{align*}
}
\solutionspace{120pt}

\tasknumber{5}%
\task{%
    Шесть одинаковых брусков массой $2\,\text{кг}$ каждый лежат на гладком горизонтальном столе.
    Бруски пронумерованы от 1 до 6 и последовательно связаны между собой
    невесомыми нерастяжимыми нитями: 1 со 2, 2 с 3 (ну и с 1) и т.д.
    Экспериментатор Глюк прикладывает постоянную горизонтальную силу $90\,\text{Н}$ к бруску с наибольшим номером.
    С каким ускорением двигается система? Чему равна сила натяжения нити, связывающей бруски 1 и 2?
}
\answer{%
    \begin{align*}
    a &= \frac{F}{6 m} = \frac{90\,\text{Н}}{6 \cdot 2\,\text{кг}} \approx 7{,}5\,\frac{\text{м}}{\text{c}^{2}}, \\
    T &= m'a = 1m \cdot \frac{F}{6 m} = \frac{1}{6} F \approx 15{,}0\,\text{Н}.
    \end{align*}
}
\solutionspace{120pt}

\tasknumber{6}%
\task{%
    Два бруска связаны лёгкой нерастяжимой нитью и перекинуты через неподвижный блок (см.
    рис.).
    Определите силу натяжения нити и ускорения брусков.
    Силами трения пренебречь, массы брусков
    равны $m_1 = 11\,\text{кг}$ и $m_2 = 10\,\text{кг}$.
    % $g = 10\,\frac{\text{м}}{\text{с}^{2}}$.

    \begin{tikzpicture}[x=1.5cm,y=1.5cm,thick]
        \draw
            (-0.4, 0) rectangle (-0.2, 1.2)
            (0.15, 0.5) rectangle (0.45, 1)
            (0, 2) circle [radius=0.3] -- ++(up:0.5)
            (-0.3, 1.2) -- ++(up:0.8)
            (0.3, 1) -- ++(up:1)
            (-0.7, 2.5) -- (0.7, 2.5)
            ;
        \draw[pattern={Lines[angle=51,distance=3pt]},pattern color=black,draw=none] (-0.7, 2.5) rectangle (0.7, 2.75);
        \node [left] (left) at (-0.4, 0.6) { $m_1$ };
        \node [right] (right) at (0.4, 0.75) { $m_2$ };
    \end{tikzpicture}
}
\answer{%
    Предположим, что левый брусок ускоряется вверх, тогда правый ускоряется вниз (с тем же ускорением).
    Запишем 2-й закон Ньютона 2 раза (для обоих тел) в проекции на вертикальную оси, направив её вверх.
    \begin{align*}
        &\begin{cases}
            T - m_1g = m_1a, \\
            T - m_2g = -m_2a,
        \end{cases} \\
        &\begin{cases}
            m_2g - m_1g = m_1a + m_2a, \\
            T = m_1a + m_1g, \\
        \end{cases} \\
        a &= \frac{m_2 - m_1}{m_1 + m_2} \cdot g = \frac{10\,\text{кг} - 11\,\text{кг}}{11\,\text{кг} + 10\,\text{кг}} \cdot 10\,\frac{\text{м}}{\text{с}^{2}} \approx -0{,}4800\,\frac{\text{м}}{\text{c}^{2}}, \\
        T &= m_1(a + g) = m_1 \cdot g \cdot \cbr{\frac{m_2 - m_1}{m_1 + m_2} + 1} = m_1 \cdot g \cdot \frac{2m_2}{m_1 + m_2} = \\
            &= \frac{2 m_2 m_1 g}{m_1 + m_2} = \frac{2 \cdot 10\,\text{кг} \cdot 11\,\text{кг} \cdot 10\,\frac{\text{м}}{\text{с}^{2}}}{11\,\text{кг} + 10\,\text{кг}} \approx 104{,}8\,\text{Н}.
    \end{align*}
    Отрицательный ответ говорит, что мы лишь не угадали с направлением ускорений.
    Сила же всегда положительна.
}
\solutionspace{80pt}

\tasknumber{7}%
\task{%
    Тело массой $1{,}4\,\text{кг}$ лежит на горизонтальной поверхности.
    Коэффициент трения между поверхностью и телом $0{,}25$.
    К телу приложена горизонтальная сила $2{,}5\,\text{Н}$.
    Определите силу трения, действующую на тело, и ускорение тела.
    % $g = 10\,\frac{\text{м}}{\text{с}^{2}}$.
}
\answer{%
    \begin{align*}
    &F_\text{ трения покоя $\max$ } = \mu N = \mu m g = 0{,}25 \cdot 1{,}4\,\text{кг} \cdot 10\,\frac{\text{м}}{\text{с}^{2}} = 3{,}50\,\text{Н}, \\
    &F_\text{ трения покоя $\max$ } > F \implies F_\text{ трения } = 2{,}50\,\text{Н}, a = \frac{F - F_\text{ трения }}{ m } = 0\,\frac{\text{м}}{\text{c}^{2}}, \\
    &\text{при равенстве возможны оба варианта: и едет, и не едет, но на ответы это не влияет.}
    \end{align*}
}
\solutionspace{120pt}

\tasknumber{8}%
\task{%
    Определите плотность неизвестного вещества, если известно, что опускании тела из него
    в подсолнечное масло оно будет плавать и на четверть выступать над поверхностью жидкости.
}
\answer{%
    $F_\text{Арх.} = F_\text{тяж.} \implies \rho_\text{ж.} g V_\text{погр.} = m g \implies\rho_\text{ж.} g \cbr{V -\frac V4} = \rho V g \implies \rho = \rho_\text{ж.}\cbr{1 -\frac 14} \approx 675\,\frac{\text{кг}}{\text{м}^{3}}$
}
\solutionspace{120pt}

\tasknumber{9}%
\task{%
    	Определите силу, действующую на левую опору однородного горизонтального стержня длиной $l = 7\,\text{м}$
    	и массой $M = 1\,\text{кг}$, к которому подвешен груз массой $m = 3\,\text{кг}$ на расстоянии $4\,\text{м}$ от правого конца (см.
    рис.).

        \begin{tikzpicture}[thick]
            \draw
                (-2, -0.1) rectangle (2, 0.1)
                (-0.5, -0.1) -- (-0.5, -1)
                (-0.7, -1) rectangle (-0.3, -1.3)
           		(-2, -0.1) -- +(0.15,-0.9) -- +(-0.15,-0.9) -- cycle
            	(2, -0.1) -- +(0.15,-0.9) -- +(-0.15,-0.9) -- cycle
            ;
            \draw[pattern={Lines[angle=51,distance=2pt]},pattern color=black,draw=none]
            	(-2.15, -1.15) rectangle +(0.3, 0.15)
            	(2.15, -1.15) rectangle +(-0.3, 0.15)
            ;
            \node [right] (m_small) at (-0.3, -1.15) { $m$ };
            \node [above] (M_big) at (0, 0.1) { $M$ };
        \end{tikzpicture}
}
\answer{%
    \begin{align*}
        &\begin{cases}
            F_1 + F_2 - mg - Mg= 0, \\
            F_1 \cdot 0 - mg \cdot a - Mg \cdot \frac l2 + F_2 \cdot l = 0,
        \end{cases} \\
        F_2 &= \frac{mga + Mg\frac l2}l = \frac al \cdot mg + \frac{Mg}2 \approx 17{,}9\,\text{Н}, \\
        F_1 &= mg + Mg - F_2 = mg + Mg - \frac al \cdot mg - \frac{Mg}2 = \frac bl \cdot mg + \frac{Mg}2 \approx 22{,}1\,\text{Н}.
    \end{align*}
}
\solutionspace{80pt}

\tasknumber{10}%
\task{%
    Тонкий однородный шест длиной $3\,\text{м}$ и массой $30\,\text{кг}$ лежит на горизонтальной поверхности.
    \begin{itemize}
        \item Какую минимальную силу надо приложить к одному из его концов, чтобы оторвать его от этой поверхности?
        \item Какую минимальную работу надо совершить, чтобы поставить его на землю в вертикальное положение?
    \end{itemize}
    % Примите $g = 10\,\frac{\text{м}}{\text{с}^{2}}$.
}
\answer{%
    $F = \frac{mg}2 \approx 300\,\text{Н}, A = mg\frac l2 = 450\,\text{Дж}$
}
\solutionspace{120pt}

\tasknumber{11}%
\task{%
    Определите работу силы, которая обеспечит подъём тела массой $2\,\text{кг}$ на высоту $5\,\text{м}$ с постоянным ускорением $3\,\frac{\text{м}}{\text{c}^{2}}$.
    % Примите $g = 10\,\frac{\text{м}}{\text{с}^{2}}$.
}
\answer{%
    \begin{align*}
    &\text{Для подъёма:} A = Fh = (mg + ma) h = m(g+a)h, \\
    &\text{Для спуска:} A = -Fh = -(mg - ma) h = -m(g-a)h, \\
    &\text{В результате получаем:} 130\,\text{Дж}.
    \end{align*}
}
\solutionspace{60pt}

\tasknumber{12}%
\task{%
    Тело бросили вертикально вверх со скоростью $14\,\frac{\text{м}}{\text{c}}$.
    На какой высоте кинетическая энергия тела составит треть от потенциальной?
}
\answer{%
    \begin{align*}
    &0 + \frac{mv_0^2}2 = E_p + E_k, E_k = \frac 13 E_p \implies \\
    &\implies \frac{mv_0^2}2 = E_p + \frac 13 E_p = E_p\cbr{1 + \frac 13} = mgh\cbr{1 + \frac 13} \implies \\
    &\implies h = \frac{\frac{mv_0^2}2}{mg\cbr{1 + \frac 13}} = \frac{v_0^2}{2g} \cdot \frac 1{1 + \frac 13} \approx 7{,}4\,\text{м}.
    \end{align*}
}
\solutionspace{100pt}

\tasknumber{13}%
\task{%
    Плотность воздуха при нормальных условиях равна $1{,}3\,\frac{\text{кг}}{\text{м}^{3}}$.
    Чему равна плотность воздуха
    при температуре $50\celsius$ и давлении $50\,\text{кПа}$?
}
\answer{%
    \begin{align*}
    &\text{В общем случае:} PV = \frac m{\mu} RT \implies \rho = \frac mV = \frac m{\frac{\frac m{\mu} RT}P} = \frac{P\mu}{RT}, \\
    &\text{У нас 2 состояния:} \rho_1 = \frac{P_1\mu}{RT_1}, \rho_2 = \frac{P_2\mu}{RT_2} \implies \frac{\rho_2}{\rho_1} = \frac{\frac{P_2\mu}{RT_2}}{\frac{P_1\mu}{RT_1}} = \frac{P_2T_1}{P_1T_2} \implies \\
    &\implies \rho_2 = \rho_1 \cdot  \frac{P_2T_1}{P_1T_2} = 1{,}3\,\frac{\text{кг}}{\text{м}^{3}} \cdot \frac{50\,\text{кПа} \cdot 273\units{К}}{100\,\text{кПа} \cdot 323\units{К}} \approx 0{,}55\,\frac{\text{кг}}{\text{м}^{3}}.
    \end{align*}
}
\solutionspace{120pt}

\tasknumber{14}%
\task{%
    Небольшую цилиндрическую пробирку с воздухом погружают на некоторую глубину в глубокое пресное озеро,
    после чего воздух занимает в ней лишь пятую часть от общего объема.
    Определите глубину, на которую погрузили пробирку.
    Температуру считать постоянной $T = 286\,\text{К}$, давлением паров воды пренебречь,
    атмосферное давление принять равным $p_{\text{aтм}} = 100\,\text{кПа}$.
}
\answer{%
    \begin{align*}
    T\text{— const} &\implies P_1V_1 = \nu RT = P_2V_2.
    \\
    V_2 = \frac 15 V_1 &\implies P_1V_1 = P_2 \cdot \frac 15V_1 \implies P_2 = 5P_1 = 5p_{\text{aтм}}.
    \\
    P_2 = p_{\text{aтм}} + \rho_{\text{в}} g h \implies h = \frac{P_2 - p_{\text{aтм}}}{\rho_{\text{в}} g} &= \frac{5p_{\text{aтм}} - p_{\text{aтм}}}{\rho_{\text{в}} g} = \frac{4 \cdot p_{\text{aтм}}}{\rho_{\text{в}} g} =  \\
     &= \frac{4 \cdot 100\,\text{кПа}}{1000\,\frac{\text{кг}}{\text{м}^{3}} \cdot  10\,\frac{\text{м}}{\text{с}^{2}}} \approx 40\,\text{м}.
    \end{align*}
}
\solutionspace{120pt}

\tasknumber{15}%
\task{%
    Газу сообщили некоторое количество теплоты,
    при этом четверть его он потратил на совершение работы,
    одновременно увеличив свою внутреннюю энергию на $2400\,\text{Дж}$.
    Определите работу, совершённую газом.
}
\answer{%
    \begin{align*}
    Q &= A' + \Delta U, A' = \frac 14 Q \implies Q \cdot \cbr{1 - \frac 14} = \Delta U \implies Q = \frac{\Delta U}{1 - \frac 14} = \frac{ 2400\,\text{Дж} }{1 - \frac 14} \approx 3200\,\text{Дж}.
    \\
    A' &= \frac 14 Q
        = \frac 14 \cdot \frac{\Delta U}{1 - \frac 14}
        = \frac{\Delta U}{4 - 1}
        = \frac{ 2400\,\text{Дж} }{4 - 1} \approx 800\,\text{Дж}.
    \end{align*}
}
\solutionspace{60pt}

\tasknumber{16}%
\task{%
    Два конденсатора ёмкостей $C_1 = 60\,\text{нФ}$ и $C_2 = 30\,\text{нФ}$ последовательно подключают
    к источнику напряжения $U = 200\,\text{В}$ (см.
    рис.).
    % Определите заряды каждого из конденсаторов.
    Определите заряд первого конденсатора.

    \begin{tikzpicture}[circuit ee IEC, semithick]
        \draw  (0, 0) to [capacitor={info={$C_1$}}] (1, 0)
                       to [capacitor={info={$C_2$}}] (2, 0)
        ;
        % \draw [-o] (0, 0) -- ++(-0.5, 0) node[left] {$-$};
        % \draw [-o] (2, 0) -- ++(0.5, 0) node[right] {$+$};
        \draw [-o] (0, 0) -- ++(-0.5, 0) node[left] {};
        \draw [-o] (2, 0) -- ++(0.5, 0) node[right] {};
    \end{tikzpicture}
}
\answer{%
    $
        Q_1
            = Q_2
            = CU
            = \frac{ U }{\frac1{C_1} + \frac1{C_2}}
            = \frac{C_1C_2U}{C_1 + C_2}
            = \frac{
                60\,\text{нФ} \cdot 30\,\text{нФ} \cdot 200\,\text{В}
            }{
                60\,\text{нФ} + 30\,\text{нФ}
            }
            = 4{,}00\,\text{мкКл}
    $
}
\solutionspace{120pt}

\tasknumber{17}%
\task{%
    В вакууме вдоль одной прямой расположены три отрицательных заряда так,
    что расстояние между соседними зарядами равно $r$.
    Сделайте рисунок,
    и определите силу, действующую на крайний заряд.
    Модули всех зарядов равны $q$ ($q > 0$).
}
\answer{%
    $F = \sum_i F_i = \ldots = \frac54 \frac{kq^2}{r^2}.$
}
\solutionspace{80pt}

\tasknumber{18}%
\task{%
    Юлия проводит эксперименты c 2 кусками одинаковой стальной проволки, причём второй кусок в пять раз длиннее первого.
    В одном из экспериментов Юлия подаёт на первый кусок проволки напряжение в два раза раз больше, чем на второй.
    Определите отношения в двух проволках в этом эксперименте (второй к первой):
    \begin{itemize}
        \item отношение сил тока,
        \item отношение выделяющихся мощностей.
    \end{itemize}
}
\answer{%
    $R_2 = 5R_1, U_1 = 2U_2 \implies  \eli_2 / \eli_1 = \frac{U_2 / R_2}{U_1 / R_1} = \frac{U_2}{U_1} \cdot \frac{R_1}{R_2} = \frac1{10}, P_2 / P_1 = \frac{U_2^2 / R_2}{U_1^2 / R_1} = \sqr{\frac{U_2}{U_1}} \cdot \frac{R_1}{R_2} = \frac1{20}.$
}

\variantsplitter

\addpersonalvariant{Андрей Рожков}

\tasknumber{1}%
\task{%
    Валя стартует на велосипеде и в течение $t = 4\,\text{c}$ двигается с постоянным ускорением $1{,}5\,\frac{\text{м}}{\text{с}^{2}}$.
    Определите
    \begin{itemize}
        \item какую скорость при этом удастся достичь,
        \item какой путь за это время будет пройден,
        \item среднюю скорость за всё время движения, если после начального ускорения продолжить движение равномерно ещё в течение времени $3t$
    \end{itemize}
}
\answer{%
    \begin{align*}
    v &= v_0 + a t = at = 1{,}5\,\frac{\text{м}}{\text{с}^{2}} \cdot 4\,\text{c} = 6{,}0\,\frac{\text{м}}{\text{с}}, \\
    s_x &= v_0t + \frac{a t^2}2 = \frac{a t^2}2 = \frac{1{,}5\,\frac{\text{м}}{\text{с}^{2}} \cdot \sqr{ 4\,\text{c} }}2 = 12{,}0\,\text{м}, \\
    v_\text{сред.} &= \frac{s_\text{общ}}{t_\text{общ.}} = \frac{s_x + v \cdot 3t}{t + 3t} = \frac{\frac{a t^2}2 + at \cdot 3t}{t (1 + 3)} = \\
    &= at \cdot \frac{\frac 12 + 3}{1 + 3} = 1{,}5\,\frac{\text{м}}{\text{с}^{2}} \cdot 4\,\text{c} \cdot \frac{\frac 12 + 3}{1 + 3} \approx 5{,}25\,\frac{\text{м}}{\text{c}}.
    \end{align*}
}
\solutionspace{120pt}

\tasknumber{2}%
\task{%
    Какой путь тело пройдёт за третью секунду после начала свободного падения?
    Какую скорость в конце этой секунды оно имеет?
}
\answer{%
    \begin{align*}
    s &= -s_y = -(y_2-y_1) = y_1 - y_2 = \cbr{y_{0y} + v_{0y}t_1 - \frac{gt_1^2}2} - \cbr{y_{0y} + v_{0y}t_2 - \frac{gt_2^2}2} = \\
    &= \frac{gt_2^2}2 - \frac{gt_1^2}2 = \frac g2\cbr{t_2^2 - t_1^2} = 25{,}0\,\text{м}, \\
    v_y &= v_{0y} - gt = -gt = 10\,\frac{\text{м}}{\text{с}^{2}} \cdot 3\,\text{с} = -30\,\frac{\text{м}}{\text{с}}.
    \end{align*}
}
\solutionspace{120pt}

\tasknumber{3}%
\task{%
    Карусель радиусом $2\,\text{м}$ равномерно совершает 10 оборотов в минуту.
    Определите
    \begin{itemize}
        \item период и частоту её обращения,
        \item скорость и ускорение крайних её точек.
    \end{itemize}
}
\answer{%
    \begin{align*}
    t &= 60\,\text{с}, r = 2{,}0\,\text{м}, n = 10\units{оборотов}, \\
    T &= \frac tN = \frac{ 60\,\text{с} }{10} \approx 6{,}00\,\text{c}, \\
    \nu &= \frac 1T = \frac{10}{ 60\,\text{с} } \approx 0{,}17\,\text{Гц}, \\
    v &= \frac{2 \pi r}{T} = \frac{2 \pi r}{T} =  \frac{2 \pi r n}{t} \approx 2{,}09\,\frac{\text{м}}{\text{c}}, \\
    a &= \frac{v^2}{r} =  \frac{4 \pi^2 r n^2}{t^2} \approx 2{,}19\,\frac{\text{м}}{\text{с}^{2}}.
    \end{align*}
}
\solutionspace{80pt}

\tasknumber{4}%
\task{%
    Даша стоит на обрыве над рекой и методично и строго горизонтально кидает в неё камушки.
    За этим всем наблюдает экспериментатор Глюк, который уже выяснил, что камушки падают в реку спустя $1{,}5\,\text{с}$ после броска,
    а вот дальность полёта оценить сложнее: придётся лезть в воду.
    Выручите Глюка и определите:
    \begin{itemize}
        \item высоту обрыва (вместе с ростом Даши).
        \item дальность полёта камушков (по горизонтали) и их скорость при падении, приняв начальную скорость броска равной $v_0 = 18\,\frac{\text{м}}{\text{с}}$.
    \end{itemize}
    Сопротивлением воздуха пренебречь.
}
\answer{%
    \begin{align*}
    y &= y_0 + v_{0y}t - \frac{gt^2}2 = h - \frac{gt^2}2, \qquad y(\tau) = 0 \implies h - \frac{g\tau^2}2 = 0 \implies h = \frac{g\tau^2}2 \approx 11{,}2\,\text{м}.
    \\
    x &= x_0 + v_{0x}t = v_0t \implies L = v_0\tau \approx 27{,}0\,\text{м}.
    \\
    &v = \sqrt{v_x^2 + v_y^2} = \sqrt{v_{0x}^2 + \sqr{v_{0y} - g\tau}} = \sqrt{v_0^2 + \sqr{g\tau}} \approx 23{,}4\,\frac{\text{м}}{\text{c}}.
    \end{align*}
}
\solutionspace{120pt}

\tasknumber{5}%
\task{%
    Пять одинаковых брусков массой $3\,\text{кг}$ каждый лежат на гладком горизонтальном столе.
    Бруски пронумерованы от 1 до 5 и последовательно связаны между собой
    невесомыми нерастяжимыми нитями: 1 со 2, 2 с 3 (ну и с 1) и т.д.
    Экспериментатор Глюк прикладывает постоянную горизонтальную силу $90\,\text{Н}$ к бруску с наименьшим номером.
    С каким ускорением двигается система? Чему равна сила натяжения нити, связывающей бруски 3 и 4?
}
\answer{%
    \begin{align*}
    a &= \frac{F}{5 m} = \frac{90\,\text{Н}}{5 \cdot 3\,\text{кг}} \approx 6{,}0\,\frac{\text{м}}{\text{c}^{2}}, \\
    T &= m'a = 2m \cdot \frac{F}{5 m} = \frac{2}{5} F \approx 36{,}0\,\text{Н}.
    \end{align*}
}
\solutionspace{120pt}

\tasknumber{6}%
\task{%
    Два бруска связаны лёгкой нерастяжимой нитью и перекинуты через неподвижный блок (см.
    рис.).
    Определите силу натяжения нити и ускорения брусков.
    Силами трения пренебречь, массы брусков
    равны $m_1 = 8\,\text{кг}$ и $m_2 = 14\,\text{кг}$.
    % $g = 10\,\frac{\text{м}}{\text{с}^{2}}$.

    \begin{tikzpicture}[x=1.5cm,y=1.5cm,thick]
        \draw
            (-0.4, 0) rectangle (-0.2, 1.2)
            (0.15, 0.5) rectangle (0.45, 1)
            (0, 2) circle [radius=0.3] -- ++(up:0.5)
            (-0.3, 1.2) -- ++(up:0.8)
            (0.3, 1) -- ++(up:1)
            (-0.7, 2.5) -- (0.7, 2.5)
            ;
        \draw[pattern={Lines[angle=51,distance=3pt]},pattern color=black,draw=none] (-0.7, 2.5) rectangle (0.7, 2.75);
        \node [left] (left) at (-0.4, 0.6) { $m_1$ };
        \node [right] (right) at (0.4, 0.75) { $m_2$ };
    \end{tikzpicture}
}
\answer{%
    Предположим, что левый брусок ускоряется вверх, тогда правый ускоряется вниз (с тем же ускорением).
    Запишем 2-й закон Ньютона 2 раза (для обоих тел) в проекции на вертикальную оси, направив её вверх.
    \begin{align*}
        &\begin{cases}
            T - m_1g = m_1a, \\
            T - m_2g = -m_2a,
        \end{cases} \\
        &\begin{cases}
            m_2g - m_1g = m_1a + m_2a, \\
            T = m_1a + m_1g, \\
        \end{cases} \\
        a &= \frac{m_2 - m_1}{m_1 + m_2} \cdot g = \frac{14\,\text{кг} - 8\,\text{кг}}{8\,\text{кг} + 14\,\text{кг}} \cdot 10\,\frac{\text{м}}{\text{с}^{2}} \approx 2{,}73\,\frac{\text{м}}{\text{c}^{2}}, \\
        T &= m_1(a + g) = m_1 \cdot g \cdot \cbr{\frac{m_2 - m_1}{m_1 + m_2} + 1} = m_1 \cdot g \cdot \frac{2m_2}{m_1 + m_2} = \\
            &= \frac{2 m_2 m_1 g}{m_1 + m_2} = \frac{2 \cdot 14\,\text{кг} \cdot 8\,\text{кг} \cdot 10\,\frac{\text{м}}{\text{с}^{2}}}{8\,\text{кг} + 14\,\text{кг}} \approx 101{,}8\,\text{Н}.
    \end{align*}
    Отрицательный ответ говорит, что мы лишь не угадали с направлением ускорений.
    Сила же всегда положительна.
}
\solutionspace{80pt}

\tasknumber{7}%
\task{%
    Тело массой $1{,}4\,\text{кг}$ лежит на горизонтальной поверхности.
    Коэффициент трения между поверхностью и телом $0{,}2$.
    К телу приложена горизонтальная сила $4{,}5\,\text{Н}$.
    Определите силу трения, действующую на тело, и ускорение тела.
    % $g = 10\,\frac{\text{м}}{\text{с}^{2}}$.
}
\answer{%
    \begin{align*}
    &F_\text{ трения покоя $\max$ } = \mu N = \mu m g = 0{,}2 \cdot 1{,}4\,\text{кг} \cdot 10\,\frac{\text{м}}{\text{с}^{2}} = 2{,}80\,\text{Н}, \\
    &F_\text{ трения покоя $\max$ } \le F \implies F_\text{ трения } = 2{,}80\,\text{Н}, a = \frac{F - F_\text{ трения }}{ m } = 1{,}21\,\frac{\text{м}}{\text{c}^{2}}, \\
    &\text{при равенстве возможны оба варианта: и едет, и не едет, но на ответы это не влияет.}
    \end{align*}
}
\solutionspace{120pt}

\tasknumber{8}%
\task{%
    Определите плотность неизвестного вещества, если известно, что опускании тела из него
    в керосин оно будет плавать и на четверть выступать над поверхностью жидкости.
}
\answer{%
    $F_\text{Арх.} = F_\text{тяж.} \implies \rho_\text{ж.} g V_\text{погр.} = m g \implies\rho_\text{ж.} g \cbr{V -\frac V4} = \rho V g \implies \rho = \rho_\text{ж.}\cbr{1 -\frac 14} \approx 600\,\frac{\text{кг}}{\text{м}^{3}}$
}
\solutionspace{120pt}

\tasknumber{9}%
\task{%
    	Определите силу, действующую на левую опору однородного горизонтального стержня длиной $l = 5\,\text{м}$
    	и массой $M = 1\,\text{кг}$, к которому подвешен груз массой $m = 2\,\text{кг}$ на расстоянии $2\,\text{м}$ от правого конца (см.
    рис.).

        \begin{tikzpicture}[thick]
            \draw
                (-2, -0.1) rectangle (2, 0.1)
                (-0.5, -0.1) -- (-0.5, -1)
                (-0.7, -1) rectangle (-0.3, -1.3)
           		(-2, -0.1) -- +(0.15,-0.9) -- +(-0.15,-0.9) -- cycle
            	(2, -0.1) -- +(0.15,-0.9) -- +(-0.15,-0.9) -- cycle
            ;
            \draw[pattern={Lines[angle=51,distance=2pt]},pattern color=black,draw=none]
            	(-2.15, -1.15) rectangle +(0.3, 0.15)
            	(2.15, -1.15) rectangle +(-0.3, 0.15)
            ;
            \node [right] (m_small) at (-0.3, -1.15) { $m$ };
            \node [above] (M_big) at (0, 0.1) { $M$ };
        \end{tikzpicture}
}
\answer{%
    \begin{align*}
        &\begin{cases}
            F_1 + F_2 - mg - Mg= 0, \\
            F_1 \cdot 0 - mg \cdot a - Mg \cdot \frac l2 + F_2 \cdot l = 0,
        \end{cases} \\
        F_2 &= \frac{mga + Mg\frac l2}l = \frac al \cdot mg + \frac{Mg}2 \approx 17{,}0\,\text{Н}, \\
        F_1 &= mg + Mg - F_2 = mg + Mg - \frac al \cdot mg - \frac{Mg}2 = \frac bl \cdot mg + \frac{Mg}2 \approx 13{,}0\,\text{Н}.
    \end{align*}
}
\solutionspace{80pt}

\tasknumber{10}%
\task{%
    Тонкий однородный лом длиной $3\,\text{м}$ и массой $30\,\text{кг}$ лежит на горизонтальной поверхности.
    \begin{itemize}
        \item Какую минимальную силу надо приложить к одному из его концов, чтобы оторвать его от этой поверхности?
        \item Какую минимальную работу надо совершить, чтобы поставить его на землю в вертикальное положение?
    \end{itemize}
    % Примите $g = 10\,\frac{\text{м}}{\text{с}^{2}}$.
}
\answer{%
    $F = \frac{mg}2 \approx 300\,\text{Н}, A = mg\frac l2 = 450\,\text{Дж}$
}
\solutionspace{120pt}

\tasknumber{11}%
\task{%
    Определите работу силы, которая обеспечит спуск тела массой $5\,\text{кг}$ на высоту $10\,\text{м}$ с постоянным ускорением $4\,\frac{\text{м}}{\text{c}^{2}}$.
    % Примите $g = 10\,\frac{\text{м}}{\text{с}^{2}}$.
}
\answer{%
    \begin{align*}
    &\text{Для подъёма:} A = Fh = (mg + ma) h = m(g+a)h, \\
    &\text{Для спуска:} A = -Fh = -(mg - ma) h = -m(g-a)h, \\
    &\text{В результате получаем:} -300\,\text{Дж}.
    \end{align*}
}
\solutionspace{60pt}

\tasknumber{12}%
\task{%
    Тело бросили вертикально вверх со скоростью $14\,\frac{\text{м}}{\text{c}}$.
    На какой высоте кинетическая энергия тела составит треть от потенциальной?
}
\answer{%
    \begin{align*}
    &0 + \frac{mv_0^2}2 = E_p + E_k, E_k = \frac 13 E_p \implies \\
    &\implies \frac{mv_0^2}2 = E_p + \frac 13 E_p = E_p\cbr{1 + \frac 13} = mgh\cbr{1 + \frac 13} \implies \\
    &\implies h = \frac{\frac{mv_0^2}2}{mg\cbr{1 + \frac 13}} = \frac{v_0^2}{2g} \cdot \frac 1{1 + \frac 13} \approx 7{,}4\,\text{м}.
    \end{align*}
}
\solutionspace{100pt}

\tasknumber{13}%
\task{%
    Плотность воздуха при нормальных условиях равна $1{,}3\,\frac{\text{кг}}{\text{м}^{3}}$.
    Чему равна плотность воздуха
    при температуре $150\celsius$ и давлении $150\,\text{кПа}$?
}
\answer{%
    \begin{align*}
    &\text{В общем случае:} PV = \frac m{\mu} RT \implies \rho = \frac mV = \frac m{\frac{\frac m{\mu} RT}P} = \frac{P\mu}{RT}, \\
    &\text{У нас 2 состояния:} \rho_1 = \frac{P_1\mu}{RT_1}, \rho_2 = \frac{P_2\mu}{RT_2} \implies \frac{\rho_2}{\rho_1} = \frac{\frac{P_2\mu}{RT_2}}{\frac{P_1\mu}{RT_1}} = \frac{P_2T_1}{P_1T_2} \implies \\
    &\implies \rho_2 = \rho_1 \cdot  \frac{P_2T_1}{P_1T_2} = 1{,}3\,\frac{\text{кг}}{\text{м}^{3}} \cdot \frac{150\,\text{кПа} \cdot 273\units{К}}{100\,\text{кПа} \cdot 423\units{К}} \approx 1{,}26\,\frac{\text{кг}}{\text{м}^{3}}.
    \end{align*}
}
\solutionspace{120pt}

\tasknumber{14}%
\task{%
    Небольшую цилиндрическую пробирку с воздухом погружают на некоторую глубину в глубокое пресное озеро,
    после чего воздух занимает в ней лишь пятую часть от общего объема.
    Определите глубину, на которую погрузили пробирку.
    Температуру считать постоянной $T = 292\,\text{К}$, давлением паров воды пренебречь,
    атмосферное давление принять равным $p_{\text{aтм}} = 100\,\text{кПа}$.
}
\answer{%
    \begin{align*}
    T\text{— const} &\implies P_1V_1 = \nu RT = P_2V_2.
    \\
    V_2 = \frac 15 V_1 &\implies P_1V_1 = P_2 \cdot \frac 15V_1 \implies P_2 = 5P_1 = 5p_{\text{aтм}}.
    \\
    P_2 = p_{\text{aтм}} + \rho_{\text{в}} g h \implies h = \frac{P_2 - p_{\text{aтм}}}{\rho_{\text{в}} g} &= \frac{5p_{\text{aтм}} - p_{\text{aтм}}}{\rho_{\text{в}} g} = \frac{4 \cdot p_{\text{aтм}}}{\rho_{\text{в}} g} =  \\
     &= \frac{4 \cdot 100\,\text{кПа}}{1000\,\frac{\text{кг}}{\text{м}^{3}} \cdot  10\,\frac{\text{м}}{\text{с}^{2}}} \approx 40\,\text{м}.
    \end{align*}
}
\solutionspace{120pt}

\tasknumber{15}%
\task{%
    Газу сообщили некоторое количество теплоты,
    при этом половину его он потратил на совершение работы,
    одновременно увеличив свою внутреннюю энергию на $3000\,\text{Дж}$.
    Определите работу, совершённую газом.
}
\answer{%
    \begin{align*}
    Q &= A' + \Delta U, A' = \frac 12 Q \implies Q \cdot \cbr{1 - \frac 12} = \Delta U \implies Q = \frac{\Delta U}{1 - \frac 12} = \frac{ 3000\,\text{Дж} }{1 - \frac 12} \approx 6000\,\text{Дж}.
    \\
    A' &= \frac 12 Q
        = \frac 12 \cdot \frac{\Delta U}{1 - \frac 12}
        = \frac{\Delta U}{2 - 1}
        = \frac{ 3000\,\text{Дж} }{2 - 1} \approx 3000\,\text{Дж}.
    \end{align*}
}
\solutionspace{60pt}

\tasknumber{16}%
\task{%
    Два конденсатора ёмкостей $C_1 = 60\,\text{нФ}$ и $C_2 = 30\,\text{нФ}$ последовательно подключают
    к источнику напряжения $U = 300\,\text{В}$ (см.
    рис.).
    % Определите заряды каждого из конденсаторов.
    Определите заряд второго конденсатора.

    \begin{tikzpicture}[circuit ee IEC, semithick]
        \draw  (0, 0) to [capacitor={info={$C_1$}}] (1, 0)
                       to [capacitor={info={$C_2$}}] (2, 0)
        ;
        % \draw [-o] (0, 0) -- ++(-0.5, 0) node[left] {$-$};
        % \draw [-o] (2, 0) -- ++(0.5, 0) node[right] {$+$};
        \draw [-o] (0, 0) -- ++(-0.5, 0) node[left] {};
        \draw [-o] (2, 0) -- ++(0.5, 0) node[right] {};
    \end{tikzpicture}
}
\answer{%
    $
        Q_1
            = Q_2
            = CU
            = \frac{ U }{\frac1{C_1} + \frac1{C_2}}
            = \frac{C_1C_2U}{C_1 + C_2}
            = \frac{
                60\,\text{нФ} \cdot 30\,\text{нФ} \cdot 300\,\text{В}
            }{
                60\,\text{нФ} + 30\,\text{нФ}
            }
            = 6{,}00\,\text{мкКл}
    $
}
\solutionspace{120pt}

\tasknumber{17}%
\task{%
    В вакууме вдоль одной прямой расположены четыре отрицательных заряда так,
    что расстояние между соседними зарядами равно $d$.
    Сделайте рисунок,
    и определите силу, действующую на крайний заряд.
    Модули всех зарядов равны $Q$ ($Q > 0$).
}
\answer{%
    $F = \sum_i F_i = \ldots = \frac{49}{36} \frac{kQ^2}{d^2}.$
}
\solutionspace{80pt}

\tasknumber{18}%
\task{%
    Юлия проводит эксперименты c 2 кусками одинаковой медной проволки, причём второй кусок в два раза длиннее первого.
    В одном из экспериментов Юлия подаёт на первый кусок проволки напряжение в десять раз раз больше, чем на второй.
    Определите отношения в двух проволках в этом эксперименте (второй к первой):
    \begin{itemize}
        \item отношение сил тока,
        \item отношение выделяющихся мощностей.
    \end{itemize}
}
\answer{%
    $R_2 = 2R_1, U_1 = 10U_2 \implies  \eli_2 / \eli_1 = \frac{U_2 / R_2}{U_1 / R_1} = \frac{U_2}{U_1} \cdot \frac{R_1}{R_2} = \frac1{20}, P_2 / P_1 = \frac{U_2^2 / R_2}{U_1^2 / R_1} = \sqr{\frac{U_2}{U_1}} \cdot \frac{R_1}{R_2} = \frac1{200}.$
}

\variantsplitter

\addpersonalvariant{Рената Таржиманова}

\tasknumber{1}%
\task{%
    Женя стартует на мотоцикле и в течение $t = 10\,\text{c}$ двигается с постоянным ускорением $2\,\frac{\text{м}}{\text{с}^{2}}$.
    Определите
    \begin{itemize}
        \item какую скорость при этом удастся достичь,
        \item какой путь за это время будет пройден,
        \item среднюю скорость за всё время движения, если после начального ускорения продолжить движение равномерно ещё в течение времени $2t$
    \end{itemize}
}
\answer{%
    \begin{align*}
    v &= v_0 + a t = at = 2\,\frac{\text{м}}{\text{с}^{2}} \cdot 10\,\text{c} = 20{,}0\,\frac{\text{м}}{\text{с}}, \\
    s_x &= v_0t + \frac{a t^2}2 = \frac{a t^2}2 = \frac{2\,\frac{\text{м}}{\text{с}^{2}} \cdot \sqr{ 10\,\text{c} }}2 = 100{,}0\,\text{м}, \\
    v_\text{сред.} &= \frac{s_\text{общ}}{t_\text{общ.}} = \frac{s_x + v \cdot 2t}{t + 2t} = \frac{\frac{a t^2}2 + at \cdot 2t}{t (1 + 2)} = \\
    &= at \cdot \frac{\frac 12 + 2}{1 + 2} = 2\,\frac{\text{м}}{\text{с}^{2}} \cdot 10\,\text{c} \cdot \frac{\frac 12 + 2}{1 + 2} \approx 16{,}67\,\frac{\text{м}}{\text{c}}.
    \end{align*}
}
\solutionspace{120pt}

\tasknumber{2}%
\task{%
    Какой путь тело пройдёт за шестую секунду после начала свободного падения?
    Какую скорость в конце этой секунды оно имеет?
}
\answer{%
    \begin{align*}
    s &= -s_y = -(y_2-y_1) = y_1 - y_2 = \cbr{y_{0y} + v_{0y}t_1 - \frac{gt_1^2}2} - \cbr{y_{0y} + v_{0y}t_2 - \frac{gt_2^2}2} = \\
    &= \frac{gt_2^2}2 - \frac{gt_1^2}2 = \frac g2\cbr{t_2^2 - t_1^2} = 55{,}0\,\text{м}, \\
    v_y &= v_{0y} - gt = -gt = 10\,\frac{\text{м}}{\text{с}^{2}} \cdot 6\,\text{с} = -60\,\frac{\text{м}}{\text{с}}.
    \end{align*}
}
\solutionspace{120pt}

\tasknumber{3}%
\task{%
    Карусель диаметром $5\,\text{м}$ равномерно совершает 6 оборотов в минуту.
    Определите
    \begin{itemize}
        \item период и частоту её обращения,
        \item скорость и ускорение крайних её точек.
    \end{itemize}
}
\answer{%
    \begin{align*}
    t &= 60\,\text{с}, r = 2{,}5\,\text{м}, n = 6\units{оборотов}, \\
    T &= \frac tN = \frac{ 60\,\text{с} }{6} \approx 10{,}00\,\text{c}, \\
    \nu &= \frac 1T = \frac{6}{ 60\,\text{с} } \approx 0{,}10\,\text{Гц}, \\
    v &= \frac{2 \pi r}{T} = \frac{2 \pi r}{T} =  \frac{2 \pi r n}{t} \approx 1{,}57\,\frac{\text{м}}{\text{c}}, \\
    a &= \frac{v^2}{r} =  \frac{4 \pi^2 r n^2}{t^2} \approx 0{,}99\,\frac{\text{м}}{\text{с}^{2}}.
    \end{align*}
}
\solutionspace{80pt}

\tasknumber{4}%
\task{%
    Миша стоит на обрыве над рекой и методично и строго горизонтально кидает в неё камушки.
    За этим всем наблюдает экспериментатор Глюк, который уже выяснил, что камушки падают в реку спустя $1{,}7\,\text{с}$ после броска,
    а вот дальность полёта оценить сложнее: придётся лезть в воду.
    Выручите Глюка и определите:
    \begin{itemize}
        \item высоту обрыва (вместе с ростом Миши).
        \item дальность полёта камушков (по горизонтали) и их скорость при падении, приняв начальную скорость броска равной $v_0 = 12\,\frac{\text{м}}{\text{с}}$.
    \end{itemize}
    Сопротивлением воздуха пренебречь.
}
\answer{%
    \begin{align*}
    y &= y_0 + v_{0y}t - \frac{gt^2}2 = h - \frac{gt^2}2, \qquad y(\tau) = 0 \implies h - \frac{g\tau^2}2 = 0 \implies h = \frac{g\tau^2}2 \approx 14{,}4\,\text{м}.
    \\
    x &= x_0 + v_{0x}t = v_0t \implies L = v_0\tau \approx 20{,}4\,\text{м}.
    \\
    &v = \sqrt{v_x^2 + v_y^2} = \sqrt{v_{0x}^2 + \sqr{v_{0y} - g\tau}} = \sqrt{v_0^2 + \sqr{g\tau}} \approx 20{,}8\,\frac{\text{м}}{\text{c}}.
    \end{align*}
}
\solutionspace{120pt}

\tasknumber{5}%
\task{%
    Четыре одинаковых брусков массой $3\,\text{кг}$ каждый лежат на гладком горизонтальном столе.
    Бруски пронумерованы от 1 до 4 и последовательно связаны между собой
    невесомыми нерастяжимыми нитями: 1 со 2, 2 с 3 (ну и с 1) и т.д.
    Экспериментатор Глюк прикладывает постоянную горизонтальную силу $120\,\text{Н}$ к бруску с наименьшим номером.
    С каким ускорением двигается система? Чему равна сила натяжения нити, связывающей бруски 3 и 4?
}
\answer{%
    \begin{align*}
    a &= \frac{F}{4 m} = \frac{120\,\text{Н}}{4 \cdot 3\,\text{кг}} \approx 10{,}0\,\frac{\text{м}}{\text{c}^{2}}, \\
    T &= m'a = 1m \cdot \frac{F}{4 m} = \frac{1}{4} F \approx 30{,}0\,\text{Н}.
    \end{align*}
}
\solutionspace{120pt}

\tasknumber{6}%
\task{%
    Два бруска связаны лёгкой нерастяжимой нитью и перекинуты через неподвижный блок (см.
    рис.).
    Определите силу натяжения нити и ускорения брусков.
    Силами трения пренебречь, массы брусков
    равны $m_1 = 5\,\text{кг}$ и $m_2 = 14\,\text{кг}$.
    % $g = 10\,\frac{\text{м}}{\text{с}^{2}}$.

    \begin{tikzpicture}[x=1.5cm,y=1.5cm,thick]
        \draw
            (-0.4, 0) rectangle (-0.2, 1.2)
            (0.15, 0.5) rectangle (0.45, 1)
            (0, 2) circle [radius=0.3] -- ++(up:0.5)
            (-0.3, 1.2) -- ++(up:0.8)
            (0.3, 1) -- ++(up:1)
            (-0.7, 2.5) -- (0.7, 2.5)
            ;
        \draw[pattern={Lines[angle=51,distance=3pt]},pattern color=black,draw=none] (-0.7, 2.5) rectangle (0.7, 2.75);
        \node [left] (left) at (-0.4, 0.6) { $m_1$ };
        \node [right] (right) at (0.4, 0.75) { $m_2$ };
    \end{tikzpicture}
}
\answer{%
    Предположим, что левый брусок ускоряется вверх, тогда правый ускоряется вниз (с тем же ускорением).
    Запишем 2-й закон Ньютона 2 раза (для обоих тел) в проекции на вертикальную оси, направив её вверх.
    \begin{align*}
        &\begin{cases}
            T - m_1g = m_1a, \\
            T - m_2g = -m_2a,
        \end{cases} \\
        &\begin{cases}
            m_2g - m_1g = m_1a + m_2a, \\
            T = m_1a + m_1g, \\
        \end{cases} \\
        a &= \frac{m_2 - m_1}{m_1 + m_2} \cdot g = \frac{14\,\text{кг} - 5\,\text{кг}}{5\,\text{кг} + 14\,\text{кг}} \cdot 10\,\frac{\text{м}}{\text{с}^{2}} \approx 4{,}74\,\frac{\text{м}}{\text{c}^{2}}, \\
        T &= m_1(a + g) = m_1 \cdot g \cdot \cbr{\frac{m_2 - m_1}{m_1 + m_2} + 1} = m_1 \cdot g \cdot \frac{2m_2}{m_1 + m_2} = \\
            &= \frac{2 m_2 m_1 g}{m_1 + m_2} = \frac{2 \cdot 14\,\text{кг} \cdot 5\,\text{кг} \cdot 10\,\frac{\text{м}}{\text{с}^{2}}}{5\,\text{кг} + 14\,\text{кг}} \approx 73{,}7\,\text{Н}.
    \end{align*}
    Отрицательный ответ говорит, что мы лишь не угадали с направлением ускорений.
    Сила же всегда положительна.
}
\solutionspace{80pt}

\tasknumber{7}%
\task{%
    Тело массой $2{,}7\,\text{кг}$ лежит на горизонтальной поверхности.
    Коэффициент трения между поверхностью и телом $0{,}2$.
    К телу приложена горизонтальная сила $4{,}5\,\text{Н}$.
    Определите силу трения, действующую на тело, и ускорение тела.
    % $g = 10\,\frac{\text{м}}{\text{с}^{2}}$.
}
\answer{%
    \begin{align*}
    &F_\text{ трения покоя $\max$ } = \mu N = \mu m g = 0{,}2 \cdot 2{,}7\,\text{кг} \cdot 10\,\frac{\text{м}}{\text{с}^{2}} = 5{,}40\,\text{Н}, \\
    &F_\text{ трения покоя $\max$ } > F \implies F_\text{ трения } = 4{,}50\,\text{Н}, a = \frac{F - F_\text{ трения }}{ m } = 0\,\frac{\text{м}}{\text{c}^{2}}, \\
    &\text{при равенстве возможны оба варианта: и едет, и не едет, но на ответы это не влияет.}
    \end{align*}
}
\solutionspace{120pt}

\tasknumber{8}%
\task{%
    Определите плотность неизвестного вещества, если известно, что опускании тела из него
    в подсолнечное масло оно будет плавать и на половину выступать над поверхностью жидкости.
}
\answer{%
    $F_\text{Арх.} = F_\text{тяж.} \implies \rho_\text{ж.} g V_\text{погр.} = m g \implies\rho_\text{ж.} g \cbr{V -\frac V2} = \rho V g \implies \rho = \rho_\text{ж.}\cbr{1 -\frac 12} \approx 450\,\frac{\text{кг}}{\text{м}^{3}}$
}
\solutionspace{120pt}

\tasknumber{9}%
\task{%
    	Определите силу, действующую на левую опору однородного горизонтального стержня длиной $l = 5\,\text{м}$
    	и массой $M = 5\,\text{кг}$, к которому подвешен груз массой $m = 4\,\text{кг}$ на расстоянии $2\,\text{м}$ от правого конца (см.
    рис.).

        \begin{tikzpicture}[thick]
            \draw
                (-2, -0.1) rectangle (2, 0.1)
                (-0.5, -0.1) -- (-0.5, -1)
                (-0.7, -1) rectangle (-0.3, -1.3)
           		(-2, -0.1) -- +(0.15,-0.9) -- +(-0.15,-0.9) -- cycle
            	(2, -0.1) -- +(0.15,-0.9) -- +(-0.15,-0.9) -- cycle
            ;
            \draw[pattern={Lines[angle=51,distance=2pt]},pattern color=black,draw=none]
            	(-2.15, -1.15) rectangle +(0.3, 0.15)
            	(2.15, -1.15) rectangle +(-0.3, 0.15)
            ;
            \node [right] (m_small) at (-0.3, -1.15) { $m$ };
            \node [above] (M_big) at (0, 0.1) { $M$ };
        \end{tikzpicture}
}
\answer{%
    \begin{align*}
        &\begin{cases}
            F_1 + F_2 - mg - Mg= 0, \\
            F_1 \cdot 0 - mg \cdot a - Mg \cdot \frac l2 + F_2 \cdot l = 0,
        \end{cases} \\
        F_2 &= \frac{mga + Mg\frac l2}l = \frac al \cdot mg + \frac{Mg}2 \approx 49{,}0\,\text{Н}, \\
        F_1 &= mg + Mg - F_2 = mg + Mg - \frac al \cdot mg - \frac{Mg}2 = \frac bl \cdot mg + \frac{Mg}2 \approx 41{,}0\,\text{Н}.
    \end{align*}
}
\solutionspace{80pt}

\tasknumber{10}%
\task{%
    Тонкий однородный шест длиной $3\,\text{м}$ и массой $30\,\text{кг}$ лежит на горизонтальной поверхности.
    \begin{itemize}
        \item Какую минимальную силу надо приложить к одному из его концов, чтобы оторвать его от этой поверхности?
        \item Какую минимальную работу надо совершить, чтобы поставить его на землю в вертикальное положение?
    \end{itemize}
    % Примите $g = 10\,\frac{\text{м}}{\text{с}^{2}}$.
}
\answer{%
    $F = \frac{mg}2 \approx 300\,\text{Н}, A = mg\frac l2 = 450\,\text{Дж}$
}
\solutionspace{120pt}

\tasknumber{11}%
\task{%
    Определите работу силы, которая обеспечит подъём тела массой $5\,\text{кг}$ на высоту $10\,\text{м}$ с постоянным ускорением $4\,\frac{\text{м}}{\text{c}^{2}}$.
    % Примите $g = 10\,\frac{\text{м}}{\text{с}^{2}}$.
}
\answer{%
    \begin{align*}
    &\text{Для подъёма:} A = Fh = (mg + ma) h = m(g+a)h, \\
    &\text{Для спуска:} A = -Fh = -(mg - ma) h = -m(g-a)h, \\
    &\text{В результате получаем:} 700\,\text{Дж}.
    \end{align*}
}
\solutionspace{60pt}

\tasknumber{12}%
\task{%
    Тело бросили вертикально вверх со скоростью $14\,\frac{\text{м}}{\text{c}}$.
    На какой высоте кинетическая энергия тела составит треть от потенциальной?
}
\answer{%
    \begin{align*}
    &0 + \frac{mv_0^2}2 = E_p + E_k, E_k = \frac 13 E_p \implies \\
    &\implies \frac{mv_0^2}2 = E_p + \frac 13 E_p = E_p\cbr{1 + \frac 13} = mgh\cbr{1 + \frac 13} \implies \\
    &\implies h = \frac{\frac{mv_0^2}2}{mg\cbr{1 + \frac 13}} = \frac{v_0^2}{2g} \cdot \frac 1{1 + \frac 13} \approx 7{,}4\,\text{м}.
    \end{align*}
}
\solutionspace{100pt}

\tasknumber{13}%
\task{%
    Плотность воздуха при нормальных условиях равна $1{,}3\,\frac{\text{кг}}{\text{м}^{3}}$.
    Чему равна плотность воздуха
    при температуре $100\celsius$ и давлении $50\,\text{кПа}$?
}
\answer{%
    \begin{align*}
    &\text{В общем случае:} PV = \frac m{\mu} RT \implies \rho = \frac mV = \frac m{\frac{\frac m{\mu} RT}P} = \frac{P\mu}{RT}, \\
    &\text{У нас 2 состояния:} \rho_1 = \frac{P_1\mu}{RT_1}, \rho_2 = \frac{P_2\mu}{RT_2} \implies \frac{\rho_2}{\rho_1} = \frac{\frac{P_2\mu}{RT_2}}{\frac{P_1\mu}{RT_1}} = \frac{P_2T_1}{P_1T_2} \implies \\
    &\implies \rho_2 = \rho_1 \cdot  \frac{P_2T_1}{P_1T_2} = 1{,}3\,\frac{\text{кг}}{\text{м}^{3}} \cdot \frac{50\,\text{кПа} \cdot 273\units{К}}{100\,\text{кПа} \cdot 373\units{К}} \approx 0{,}48\,\frac{\text{кг}}{\text{м}^{3}}.
    \end{align*}
}
\solutionspace{120pt}

\tasknumber{14}%
\task{%
    Небольшую цилиндрическую пробирку с воздухом погружают на некоторую глубину в глубокое пресное озеро,
    после чего воздух занимает в ней лишь третью часть от общего объема.
    Определите глубину, на которую погрузили пробирку.
    Температуру считать постоянной $T = 289\,\text{К}$, давлением паров воды пренебречь,
    атмосферное давление принять равным $p_{\text{aтм}} = 100\,\text{кПа}$.
}
\answer{%
    \begin{align*}
    T\text{— const} &\implies P_1V_1 = \nu RT = P_2V_2.
    \\
    V_2 = \frac 13 V_1 &\implies P_1V_1 = P_2 \cdot \frac 13V_1 \implies P_2 = 3P_1 = 3p_{\text{aтм}}.
    \\
    P_2 = p_{\text{aтм}} + \rho_{\text{в}} g h \implies h = \frac{P_2 - p_{\text{aтм}}}{\rho_{\text{в}} g} &= \frac{3p_{\text{aтм}} - p_{\text{aтм}}}{\rho_{\text{в}} g} = \frac{2 \cdot p_{\text{aтм}}}{\rho_{\text{в}} g} =  \\
     &= \frac{2 \cdot 100\,\text{кПа}}{1000\,\frac{\text{кг}}{\text{м}^{3}} \cdot  10\,\frac{\text{м}}{\text{с}^{2}}} \approx 20\,\text{м}.
    \end{align*}
}
\solutionspace{120pt}

\tasknumber{15}%
\task{%
    Газу сообщили некоторое количество теплоты,
    при этом треть его он потратил на совершение работы,
    одновременно увеличив свою внутреннюю энергию на $2400\,\text{Дж}$.
    Определите работу, совершённую газом.
}
\answer{%
    \begin{align*}
    Q &= A' + \Delta U, A' = \frac 13 Q \implies Q \cdot \cbr{1 - \frac 13} = \Delta U \implies Q = \frac{\Delta U}{1 - \frac 13} = \frac{ 2400\,\text{Дж} }{1 - \frac 13} \approx 3600\,\text{Дж}.
    \\
    A' &= \frac 13 Q
        = \frac 13 \cdot \frac{\Delta U}{1 - \frac 13}
        = \frac{\Delta U}{3 - 1}
        = \frac{ 2400\,\text{Дж} }{3 - 1} \approx 1200\,\text{Дж}.
    \end{align*}
}
\solutionspace{60pt}

\tasknumber{16}%
\task{%
    Два конденсатора ёмкостей $C_1 = 60\,\text{нФ}$ и $C_2 = 20\,\text{нФ}$ последовательно подключают
    к источнику напряжения $U = 450\,\text{В}$ (см.
    рис.).
    % Определите заряды каждого из конденсаторов.
    Определите заряд второго конденсатора.

    \begin{tikzpicture}[circuit ee IEC, semithick]
        \draw  (0, 0) to [capacitor={info={$C_1$}}] (1, 0)
                       to [capacitor={info={$C_2$}}] (2, 0)
        ;
        % \draw [-o] (0, 0) -- ++(-0.5, 0) node[left] {$-$};
        % \draw [-o] (2, 0) -- ++(0.5, 0) node[right] {$+$};
        \draw [-o] (0, 0) -- ++(-0.5, 0) node[left] {};
        \draw [-o] (2, 0) -- ++(0.5, 0) node[right] {};
    \end{tikzpicture}
}
\answer{%
    $
        Q_1
            = Q_2
            = CU
            = \frac{ U }{\frac1{C_1} + \frac1{C_2}}
            = \frac{C_1C_2U}{C_1 + C_2}
            = \frac{
                60\,\text{нФ} \cdot 20\,\text{нФ} \cdot 450\,\text{В}
            }{
                60\,\text{нФ} + 20\,\text{нФ}
            }
            = 6{,}75\,\text{мкКл}
    $
}
\solutionspace{120pt}

\tasknumber{17}%
\task{%
    В вакууме вдоль одной прямой расположены четыре положительных заряда так,
    что расстояние между соседними зарядами равно $r$.
    Сделайте рисунок,
    и определите силу, действующую на крайний заряд.
    Модули всех зарядов равны $Q$ ($Q > 0$).
}
\answer{%
    $F = \sum_i F_i = \ldots = \frac{49}{36} \frac{kQ^2}{r^2}.$
}
\solutionspace{80pt}

\tasknumber{18}%
\task{%
    Юлия проводит эксперименты c 2 кусками одинаковой стальной проволки, причём второй кусок в четыре раза длиннее первого.
    В одном из экспериментов Юлия подаёт на первый кусок проволки напряжение в семь раз раз больше, чем на второй.
    Определите отношения в двух проволках в этом эксперименте (второй к первой):
    \begin{itemize}
        \item отношение сил тока,
        \item отношение выделяющихся мощностей.
    \end{itemize}
}
\answer{%
    $R_2 = 4R_1, U_1 = 7U_2 \implies  \eli_2 / \eli_1 = \frac{U_2 / R_2}{U_1 / R_1} = \frac{U_2}{U_1} \cdot \frac{R_1}{R_2} = \frac1{28}, P_2 / P_1 = \frac{U_2^2 / R_2}{U_1^2 / R_1} = \sqr{\frac{U_2}{U_1}} \cdot \frac{R_1}{R_2} = \frac1{196}.$
}

\variantsplitter

\addpersonalvariant{Андрей Щербаков}

\tasknumber{1}%
\task{%
    Женя стартует на мотоцикле и в течение $t = 10\,\text{c}$ двигается с постоянным ускорением $2\,\frac{\text{м}}{\text{с}^{2}}$.
    Определите
    \begin{itemize}
        \item какую скорость при этом удастся достичь,
        \item какой путь за это время будет пройден,
        \item среднюю скорость за всё время движения, если после начального ускорения продолжить движение равномерно ещё в течение времени $2t$
    \end{itemize}
}
\answer{%
    \begin{align*}
    v &= v_0 + a t = at = 2\,\frac{\text{м}}{\text{с}^{2}} \cdot 10\,\text{c} = 20{,}0\,\frac{\text{м}}{\text{с}}, \\
    s_x &= v_0t + \frac{a t^2}2 = \frac{a t^2}2 = \frac{2\,\frac{\text{м}}{\text{с}^{2}} \cdot \sqr{ 10\,\text{c} }}2 = 100{,}0\,\text{м}, \\
    v_\text{сред.} &= \frac{s_\text{общ}}{t_\text{общ.}} = \frac{s_x + v \cdot 2t}{t + 2t} = \frac{\frac{a t^2}2 + at \cdot 2t}{t (1 + 2)} = \\
    &= at \cdot \frac{\frac 12 + 2}{1 + 2} = 2\,\frac{\text{м}}{\text{с}^{2}} \cdot 10\,\text{c} \cdot \frac{\frac 12 + 2}{1 + 2} \approx 16{,}67\,\frac{\text{м}}{\text{c}}.
    \end{align*}
}
\solutionspace{120pt}

\tasknumber{2}%
\task{%
    Какой путь тело пройдёт за пятую секунду после начала свободного падения?
    Какую скорость в начале этой секунды оно имеет?
}
\answer{%
    \begin{align*}
    s &= -s_y = -(y_2-y_1) = y_1 - y_2 = \cbr{y_{0y} + v_{0y}t_1 - \frac{gt_1^2}2} - \cbr{y_{0y} + v_{0y}t_2 - \frac{gt_2^2}2} = \\
    &= \frac{gt_2^2}2 - \frac{gt_1^2}2 = \frac g2\cbr{t_2^2 - t_1^2} = 45{,}0\,\text{м}, \\
    v_y &= v_{0y} - gt = -gt = 10\,\frac{\text{м}}{\text{с}^{2}} \cdot 4\,\text{с} = -40\,\frac{\text{м}}{\text{с}}.
    \end{align*}
}
\solutionspace{120pt}

\tasknumber{3}%
\task{%
    Карусель радиусом $4\,\text{м}$ равномерно совершает 5 оборотов в минуту.
    Определите
    \begin{itemize}
        \item период и частоту её обращения,
        \item скорость и ускорение крайних её точек.
    \end{itemize}
}
\answer{%
    \begin{align*}
    t &= 60\,\text{с}, r = 4{,}0\,\text{м}, n = 5\units{оборотов}, \\
    T &= \frac tN = \frac{ 60\,\text{с} }{5} \approx 12{,}00\,\text{c}, \\
    \nu &= \frac 1T = \frac{5}{ 60\,\text{с} } \approx 0{,}08\,\text{Гц}, \\
    v &= \frac{2 \pi r}{T} = \frac{2 \pi r}{T} =  \frac{2 \pi r n}{t} \approx 2{,}09\,\frac{\text{м}}{\text{c}}, \\
    a &= \frac{v^2}{r} =  \frac{4 \pi^2 r n^2}{t^2} \approx 1{,}10\,\frac{\text{м}}{\text{с}^{2}}.
    \end{align*}
}
\solutionspace{80pt}

\tasknumber{4}%
\task{%
    Паша стоит на обрыве над рекой и методично и строго горизонтально кидает в неё камушки.
    За этим всем наблюдает экспериментатор Глюк, который уже выяснил, что камушки падают в реку спустя $1{,}2\,\text{с}$ после броска,
    а вот дальность полёта оценить сложнее: придётся лезть в воду.
    Выручите Глюка и определите:
    \begin{itemize}
        \item высоту обрыва (вместе с ростом Паши).
        \item дальность полёта камушков (по горизонтали) и их скорость при падении, приняв начальную скорость броска равной $v_0 = 15\,\frac{\text{м}}{\text{с}}$.
    \end{itemize}
    Сопротивлением воздуха пренебречь.
}
\answer{%
    \begin{align*}
    y &= y_0 + v_{0y}t - \frac{gt^2}2 = h - \frac{gt^2}2, \qquad y(\tau) = 0 \implies h - \frac{g\tau^2}2 = 0 \implies h = \frac{g\tau^2}2 \approx 7{,}2\,\text{м}.
    \\
    x &= x_0 + v_{0x}t = v_0t \implies L = v_0\tau \approx 18{,}0\,\text{м}.
    \\
    &v = \sqrt{v_x^2 + v_y^2} = \sqrt{v_{0x}^2 + \sqr{v_{0y} - g\tau}} = \sqrt{v_0^2 + \sqr{g\tau}} \approx 19{,}2\,\frac{\text{м}}{\text{c}}.
    \end{align*}
}
\solutionspace{120pt}

\tasknumber{5}%
\task{%
    Шесть одинаковых брусков массой $2\,\text{кг}$ каждый лежат на гладком горизонтальном столе.
    Бруски пронумерованы от 1 до 6 и последовательно связаны между собой
    невесомыми нерастяжимыми нитями: 1 со 2, 2 с 3 (ну и с 1) и т.д.
    Экспериментатор Глюк прикладывает постоянную горизонтальную силу $60\,\text{Н}$ к бруску с наименьшим номером.
    С каким ускорением двигается система? Чему равна сила натяжения нити, связывающей бруски 2 и 3?
}
\answer{%
    \begin{align*}
    a &= \frac{F}{6 m} = \frac{60\,\text{Н}}{6 \cdot 2\,\text{кг}} \approx 5{,}0\,\frac{\text{м}}{\text{c}^{2}}, \\
    T &= m'a = 4m \cdot \frac{F}{6 m} = \frac{4}{6} F \approx 40{,}0\,\text{Н}.
    \end{align*}
}
\solutionspace{120pt}

\tasknumber{6}%
\task{%
    Два бруска связаны лёгкой нерастяжимой нитью и перекинуты через неподвижный блок (см.
    рис.).
    Определите силу натяжения нити и ускорения брусков.
    Силами трения пренебречь, массы брусков
    равны $m_1 = 8\,\text{кг}$ и $m_2 = 10\,\text{кг}$.
    % $g = 10\,\frac{\text{м}}{\text{с}^{2}}$.

    \begin{tikzpicture}[x=1.5cm,y=1.5cm,thick]
        \draw
            (-0.4, 0) rectangle (-0.2, 1.2)
            (0.15, 0.5) rectangle (0.45, 1)
            (0, 2) circle [radius=0.3] -- ++(up:0.5)
            (-0.3, 1.2) -- ++(up:0.8)
            (0.3, 1) -- ++(up:1)
            (-0.7, 2.5) -- (0.7, 2.5)
            ;
        \draw[pattern={Lines[angle=51,distance=3pt]},pattern color=black,draw=none] (-0.7, 2.5) rectangle (0.7, 2.75);
        \node [left] (left) at (-0.4, 0.6) { $m_1$ };
        \node [right] (right) at (0.4, 0.75) { $m_2$ };
    \end{tikzpicture}
}
\answer{%
    Предположим, что левый брусок ускоряется вверх, тогда правый ускоряется вниз (с тем же ускорением).
    Запишем 2-й закон Ньютона 2 раза (для обоих тел) в проекции на вертикальную оси, направив её вверх.
    \begin{align*}
        &\begin{cases}
            T - m_1g = m_1a, \\
            T - m_2g = -m_2a,
        \end{cases} \\
        &\begin{cases}
            m_2g - m_1g = m_1a + m_2a, \\
            T = m_1a + m_1g, \\
        \end{cases} \\
        a &= \frac{m_2 - m_1}{m_1 + m_2} \cdot g = \frac{10\,\text{кг} - 8\,\text{кг}}{8\,\text{кг} + 10\,\text{кг}} \cdot 10\,\frac{\text{м}}{\text{с}^{2}} \approx 1{,}11\,\frac{\text{м}}{\text{c}^{2}}, \\
        T &= m_1(a + g) = m_1 \cdot g \cdot \cbr{\frac{m_2 - m_1}{m_1 + m_2} + 1} = m_1 \cdot g \cdot \frac{2m_2}{m_1 + m_2} = \\
            &= \frac{2 m_2 m_1 g}{m_1 + m_2} = \frac{2 \cdot 10\,\text{кг} \cdot 8\,\text{кг} \cdot 10\,\frac{\text{м}}{\text{с}^{2}}}{8\,\text{кг} + 10\,\text{кг}} \approx 88{,}9\,\text{Н}.
    \end{align*}
    Отрицательный ответ говорит, что мы лишь не угадали с направлением ускорений.
    Сила же всегда положительна.
}
\solutionspace{80pt}

\tasknumber{7}%
\task{%
    Тело массой $1{,}4\,\text{кг}$ лежит на горизонтальной поверхности.
    Коэффициент трения между поверхностью и телом $0{,}2$.
    К телу приложена горизонтальная сила $3{,}5\,\text{Н}$.
    Определите силу трения, действующую на тело, и ускорение тела.
    % $g = 10\,\frac{\text{м}}{\text{с}^{2}}$.
}
\answer{%
    \begin{align*}
    &F_\text{ трения покоя $\max$ } = \mu N = \mu m g = 0{,}2 \cdot 1{,}4\,\text{кг} \cdot 10\,\frac{\text{м}}{\text{с}^{2}} = 2{,}80\,\text{Н}, \\
    &F_\text{ трения покоя $\max$ } \le F \implies F_\text{ трения } = 2{,}80\,\text{Н}, a = \frac{F - F_\text{ трения }}{ m } = 0{,}50\,\frac{\text{м}}{\text{c}^{2}}, \\
    &\text{при равенстве возможны оба варианта: и едет, и не едет, но на ответы это не влияет.}
    \end{align*}
}
\solutionspace{120pt}

\tasknumber{8}%
\task{%
    Определите плотность неизвестного вещества, если известно, что опускании тела из него
    в керосин оно будет плавать и на треть выступать над поверхностью жидкости.
}
\answer{%
    $F_\text{Арх.} = F_\text{тяж.} \implies \rho_\text{ж.} g V_\text{погр.} = m g \implies\rho_\text{ж.} g \cbr{V -\frac V3} = \rho V g \implies \rho = \rho_\text{ж.}\cbr{1 -\frac 13} \approx 533\,\frac{\text{кг}}{\text{м}^{3}}$
}
\solutionspace{120pt}

\tasknumber{9}%
\task{%
    	Определите силу, действующую на левую опору однородного горизонтального стержня длиной $l = 9\,\text{м}$
    	и массой $M = 5\,\text{кг}$, к которому подвешен груз массой $m = 3\,\text{кг}$ на расстоянии $4\,\text{м}$ от правого конца (см.
    рис.).

        \begin{tikzpicture}[thick]
            \draw
                (-2, -0.1) rectangle (2, 0.1)
                (-0.5, -0.1) -- (-0.5, -1)
                (-0.7, -1) rectangle (-0.3, -1.3)
           		(-2, -0.1) -- +(0.15,-0.9) -- +(-0.15,-0.9) -- cycle
            	(2, -0.1) -- +(0.15,-0.9) -- +(-0.15,-0.9) -- cycle
            ;
            \draw[pattern={Lines[angle=51,distance=2pt]},pattern color=black,draw=none]
            	(-2.15, -1.15) rectangle +(0.3, 0.15)
            	(2.15, -1.15) rectangle +(-0.3, 0.15)
            ;
            \node [right] (m_small) at (-0.3, -1.15) { $m$ };
            \node [above] (M_big) at (0, 0.1) { $M$ };
        \end{tikzpicture}
}
\answer{%
    \begin{align*}
        &\begin{cases}
            F_1 + F_2 - mg - Mg= 0, \\
            F_1 \cdot 0 - mg \cdot a - Mg \cdot \frac l2 + F_2 \cdot l = 0,
        \end{cases} \\
        F_2 &= \frac{mga + Mg\frac l2}l = \frac al \cdot mg + \frac{Mg}2 \approx 41{,}7\,\text{Н}, \\
        F_1 &= mg + Mg - F_2 = mg + Mg - \frac al \cdot mg - \frac{Mg}2 = \frac bl \cdot mg + \frac{Mg}2 \approx 38{,}3\,\text{Н}.
    \end{align*}
}
\solutionspace{80pt}

\tasknumber{10}%
\task{%
    Тонкий однородный кусок арматуры длиной $1\,\text{м}$ и массой $30\,\text{кг}$ лежит на горизонтальной поверхности.
    \begin{itemize}
        \item Какую минимальную силу надо приложить к одному из его концов, чтобы оторвать его от этой поверхности?
        \item Какую минимальную работу надо совершить, чтобы поставить его на землю в вертикальное положение?
    \end{itemize}
    % Примите $g = 10\,\frac{\text{м}}{\text{с}^{2}}$.
}
\answer{%
    $F = \frac{mg}2 \approx 300\,\text{Н}, A = mg\frac l2 = 150\,\text{Дж}$
}
\solutionspace{120pt}

\tasknumber{11}%
\task{%
    Определите работу силы, которая обеспечит спуск тела массой $5\,\text{кг}$ на высоту $10\,\text{м}$ с постоянным ускорением $3\,\frac{\text{м}}{\text{c}^{2}}$.
    % Примите $g = 10\,\frac{\text{м}}{\text{с}^{2}}$.
}
\answer{%
    \begin{align*}
    &\text{Для подъёма:} A = Fh = (mg + ma) h = m(g+a)h, \\
    &\text{Для спуска:} A = -Fh = -(mg - ma) h = -m(g-a)h, \\
    &\text{В результате получаем:} -350\,\text{Дж}.
    \end{align*}
}
\solutionspace{60pt}

\tasknumber{12}%
\task{%
    Тело бросили вертикально вверх со скоростью $10\,\frac{\text{м}}{\text{c}}$.
    На какой высоте кинетическая энергия тела составит треть от потенциальной?
}
\answer{%
    \begin{align*}
    &0 + \frac{mv_0^2}2 = E_p + E_k, E_k = \frac 13 E_p \implies \\
    &\implies \frac{mv_0^2}2 = E_p + \frac 13 E_p = E_p\cbr{1 + \frac 13} = mgh\cbr{1 + \frac 13} \implies \\
    &\implies h = \frac{\frac{mv_0^2}2}{mg\cbr{1 + \frac 13}} = \frac{v_0^2}{2g} \cdot \frac 1{1 + \frac 13} \approx 3{,}8\,\text{м}.
    \end{align*}
}
\solutionspace{100pt}

\tasknumber{13}%
\task{%
    Плотность воздуха при нормальных условиях равна $1{,}3\,\frac{\text{кг}}{\text{м}^{3}}$.
    Чему равна плотность воздуха
    при температуре $200\celsius$ и давлении $50\,\text{кПа}$?
}
\answer{%
    \begin{align*}
    &\text{В общем случае:} PV = \frac m{\mu} RT \implies \rho = \frac mV = \frac m{\frac{\frac m{\mu} RT}P} = \frac{P\mu}{RT}, \\
    &\text{У нас 2 состояния:} \rho_1 = \frac{P_1\mu}{RT_1}, \rho_2 = \frac{P_2\mu}{RT_2} \implies \frac{\rho_2}{\rho_1} = \frac{\frac{P_2\mu}{RT_2}}{\frac{P_1\mu}{RT_1}} = \frac{P_2T_1}{P_1T_2} \implies \\
    &\implies \rho_2 = \rho_1 \cdot  \frac{P_2T_1}{P_1T_2} = 1{,}3\,\frac{\text{кг}}{\text{м}^{3}} \cdot \frac{50\,\text{кПа} \cdot 273\units{К}}{100\,\text{кПа} \cdot 473\units{К}} \approx 0{,}38\,\frac{\text{кг}}{\text{м}^{3}}.
    \end{align*}
}
\solutionspace{120pt}

\tasknumber{14}%
\task{%
    Небольшую цилиндрическую пробирку с воздухом погружают на некоторую глубину в глубокое пресное озеро,
    после чего воздух занимает в ней лишь третью часть от общего объема.
    Определите глубину, на которую погрузили пробирку.
    Температуру считать постоянной $T = 292\,\text{К}$, давлением паров воды пренебречь,
    атмосферное давление принять равным $p_{\text{aтм}} = 100\,\text{кПа}$.
}
\answer{%
    \begin{align*}
    T\text{— const} &\implies P_1V_1 = \nu RT = P_2V_2.
    \\
    V_2 = \frac 13 V_1 &\implies P_1V_1 = P_2 \cdot \frac 13V_1 \implies P_2 = 3P_1 = 3p_{\text{aтм}}.
    \\
    P_2 = p_{\text{aтм}} + \rho_{\text{в}} g h \implies h = \frac{P_2 - p_{\text{aтм}}}{\rho_{\text{в}} g} &= \frac{3p_{\text{aтм}} - p_{\text{aтм}}}{\rho_{\text{в}} g} = \frac{2 \cdot p_{\text{aтм}}}{\rho_{\text{в}} g} =  \\
     &= \frac{2 \cdot 100\,\text{кПа}}{1000\,\frac{\text{кг}}{\text{м}^{3}} \cdot  10\,\frac{\text{м}}{\text{с}^{2}}} \approx 20\,\text{м}.
    \end{align*}
}
\solutionspace{120pt}

\tasknumber{15}%
\task{%
    Газу сообщили некоторое количество теплоты,
    при этом четверть его он потратил на совершение работы,
    одновременно увеличив свою внутреннюю энергию на $1200\,\text{Дж}$.
    Определите количество теплоты, сообщённое газу.
}
\answer{%
    \begin{align*}
    Q &= A' + \Delta U, A' = \frac 14 Q \implies Q \cdot \cbr{1 - \frac 14} = \Delta U \implies Q = \frac{\Delta U}{1 - \frac 14} = \frac{ 1200\,\text{Дж} }{1 - \frac 14} \approx 1600\,\text{Дж}.
    \\
    A' &= \frac 14 Q
        = \frac 14 \cdot \frac{\Delta U}{1 - \frac 14}
        = \frac{\Delta U}{4 - 1}
        = \frac{ 1200\,\text{Дж} }{4 - 1} \approx 400\,\text{Дж}.
    \end{align*}
}
\solutionspace{60pt}

\tasknumber{16}%
\task{%
    Два конденсатора ёмкостей $C_1 = 30\,\text{нФ}$ и $C_2 = 20\,\text{нФ}$ последовательно подключают
    к источнику напряжения $V = 450\,\text{В}$ (см.
    рис.).
    % Определите заряды каждого из конденсаторов.
    Определите заряд первого конденсатора.

    \begin{tikzpicture}[circuit ee IEC, semithick]
        \draw  (0, 0) to [capacitor={info={$C_1$}}] (1, 0)
                       to [capacitor={info={$C_2$}}] (2, 0)
        ;
        % \draw [-o] (0, 0) -- ++(-0.5, 0) node[left] {$-$};
        % \draw [-o] (2, 0) -- ++(0.5, 0) node[right] {$+$};
        \draw [-o] (0, 0) -- ++(-0.5, 0) node[left] {};
        \draw [-o] (2, 0) -- ++(0.5, 0) node[right] {};
    \end{tikzpicture}
}
\answer{%
    $
        Q_1
            = Q_2
            = CV
            = \frac{ V }{\frac1{C_1} + \frac1{C_2}}
            = \frac{C_1C_2V}{C_1 + C_2}
            = \frac{
                30\,\text{нФ} \cdot 20\,\text{нФ} \cdot 450\,\text{В}
            }{
                30\,\text{нФ} + 20\,\text{нФ}
            }
            = 5{,}40\,\text{мкКл}
    $
}
\solutionspace{120pt}

\tasknumber{17}%
\task{%
    В вакууме вдоль одной прямой расположены четыре отрицательных заряда так,
    что расстояние между соседними зарядами равно $l$.
    Сделайте рисунок,
    и определите силу, действующую на крайний заряд.
    Модули всех зарядов равны $Q$ ($Q > 0$).
}
\answer{%
    $F = \sum_i F_i = \ldots = \frac{49}{36} \frac{kQ^2}{l^2}.$
}
\solutionspace{80pt}

\tasknumber{18}%
\task{%
    Юлия проводит эксперименты c 2 кусками одинаковой стальной проволки, причём второй кусок в пять раз длиннее первого.
    В одном из экспериментов Юлия подаёт на первый кусок проволки напряжение в восемь раз раз больше, чем на второй.
    Определите отношения в двух проволках в этом эксперименте (второй к первой):
    \begin{itemize}
        \item отношение сил тока,
        \item отношение выделяющихся мощностей.
    \end{itemize}
}
\answer{%
    $R_2 = 5R_1, U_1 = 8U_2 \implies  \eli_2 / \eli_1 = \frac{U_2 / R_2}{U_1 / R_1} = \frac{U_2}{U_1} \cdot \frac{R_1}{R_2} = \frac1{40}, P_2 / P_1 = \frac{U_2^2 / R_2}{U_1^2 / R_1} = \sqr{\frac{U_2}{U_1}} \cdot \frac{R_1}{R_2} = \frac1{320}.$
}

\variantsplitter

\addpersonalvariant{Михаил Ярошевский}

\tasknumber{1}%
\task{%
    Саша стартует на лошади и в течение $t = 10\,\text{c}$ двигается с постоянным ускорением $1{,}5\,\frac{\text{м}}{\text{с}^{2}}$.
    Определите
    \begin{itemize}
        \item какую скорость при этом удастся достичь,
        \item какой путь за это время будет пройден,
        \item среднюю скорость за всё время движения, если после начального ускорения продолжить движение равномерно ещё в течение времени $3t$
    \end{itemize}
}
\answer{%
    \begin{align*}
    v &= v_0 + a t = at = 1{,}5\,\frac{\text{м}}{\text{с}^{2}} \cdot 10\,\text{c} = 15{,}0\,\frac{\text{м}}{\text{с}}, \\
    s_x &= v_0t + \frac{a t^2}2 = \frac{a t^2}2 = \frac{1{,}5\,\frac{\text{м}}{\text{с}^{2}} \cdot \sqr{ 10\,\text{c} }}2 = 75{,}0\,\text{м}, \\
    v_\text{сред.} &= \frac{s_\text{общ}}{t_\text{общ.}} = \frac{s_x + v \cdot 3t}{t + 3t} = \frac{\frac{a t^2}2 + at \cdot 3t}{t (1 + 3)} = \\
    &= at \cdot \frac{\frac 12 + 3}{1 + 3} = 1{,}5\,\frac{\text{м}}{\text{с}^{2}} \cdot 10\,\text{c} \cdot \frac{\frac 12 + 3}{1 + 3} \approx 13{,}12\,\frac{\text{м}}{\text{c}}.
    \end{align*}
}
\solutionspace{120pt}

\tasknumber{2}%
\task{%
    Какой путь тело пройдёт за четвёртую секунду после начала свободного падения?
    Какую скорость в конце этой секунды оно имеет?
}
\answer{%
    \begin{align*}
    s &= -s_y = -(y_2-y_1) = y_1 - y_2 = \cbr{y_{0y} + v_{0y}t_1 - \frac{gt_1^2}2} - \cbr{y_{0y} + v_{0y}t_2 - \frac{gt_2^2}2} = \\
    &= \frac{gt_2^2}2 - \frac{gt_1^2}2 = \frac g2\cbr{t_2^2 - t_1^2} = 35{,}0\,\text{м}, \\
    v_y &= v_{0y} - gt = -gt = 10\,\frac{\text{м}}{\text{с}^{2}} \cdot 4\,\text{с} = -40\,\frac{\text{м}}{\text{с}}.
    \end{align*}
}
\solutionspace{120pt}

\tasknumber{3}%
\task{%
    Карусель диаметром $3\,\text{м}$ равномерно совершает 10 оборотов в минуту.
    Определите
    \begin{itemize}
        \item период и частоту её обращения,
        \item скорость и ускорение крайних её точек.
    \end{itemize}
}
\answer{%
    \begin{align*}
    t &= 60\,\text{с}, r = 1{,}5\,\text{м}, n = 10\units{оборотов}, \\
    T &= \frac tN = \frac{ 60\,\text{с} }{10} \approx 6{,}00\,\text{c}, \\
    \nu &= \frac 1T = \frac{10}{ 60\,\text{с} } \approx 0{,}17\,\text{Гц}, \\
    v &= \frac{2 \pi r}{T} = \frac{2 \pi r}{T} =  \frac{2 \pi r n}{t} \approx 1{,}57\,\frac{\text{м}}{\text{c}}, \\
    a &= \frac{v^2}{r} =  \frac{4 \pi^2 r n^2}{t^2} \approx 1{,}64\,\frac{\text{м}}{\text{с}^{2}}.
    \end{align*}
}
\solutionspace{80pt}

\tasknumber{4}%
\task{%
    Миша стоит на обрыве над рекой и методично и строго горизонтально кидает в неё камушки.
    За этим всем наблюдает экспериментатор Глюк, который уже выяснил, что камушки падают в реку спустя $1{,}6\,\text{с}$ после броска,
    а вот дальность полёта оценить сложнее: придётся лезть в воду.
    Выручите Глюка и определите:
    \begin{itemize}
        \item высоту обрыва (вместе с ростом Миши).
        \item дальность полёта камушков (по горизонтали) и их скорость при падении, приняв начальную скорость броска равной $v_0 = 14\,\frac{\text{м}}{\text{с}}$.
    \end{itemize}
    Сопротивлением воздуха пренебречь.
}
\answer{%
    \begin{align*}
    y &= y_0 + v_{0y}t - \frac{gt^2}2 = h - \frac{gt^2}2, \qquad y(\tau) = 0 \implies h - \frac{g\tau^2}2 = 0 \implies h = \frac{g\tau^2}2 \approx 12{,}8\,\text{м}.
    \\
    x &= x_0 + v_{0x}t = v_0t \implies L = v_0\tau \approx 22{,}4\,\text{м}.
    \\
    &v = \sqrt{v_x^2 + v_y^2} = \sqrt{v_{0x}^2 + \sqr{v_{0y} - g\tau}} = \sqrt{v_0^2 + \sqr{g\tau}} \approx 21{,}3\,\frac{\text{м}}{\text{c}}.
    \end{align*}
}
\solutionspace{120pt}

\tasknumber{5}%
\task{%
    Шесть одинаковых брусков массой $2\,\text{кг}$ каждый лежат на гладком горизонтальном столе.
    Бруски пронумерованы от 1 до 6 и последовательно связаны между собой
    невесомыми нерастяжимыми нитями: 1 со 2, 2 с 3 (ну и с 1) и т.д.
    Экспериментатор Глюк прикладывает постоянную горизонтальную силу $90\,\text{Н}$ к бруску с наибольшим номером.
    С каким ускорением двигается система? Чему равна сила натяжения нити, связывающей бруски 3 и 4?
}
\answer{%
    \begin{align*}
    a &= \frac{F}{6 m} = \frac{90\,\text{Н}}{6 \cdot 2\,\text{кг}} \approx 7{,}5\,\frac{\text{м}}{\text{c}^{2}}, \\
    T &= m'a = 3m \cdot \frac{F}{6 m} = \frac{3}{6} F \approx 45{,}0\,\text{Н}.
    \end{align*}
}
\solutionspace{120pt}

\tasknumber{6}%
\task{%
    Два бруска связаны лёгкой нерастяжимой нитью и перекинуты через неподвижный блок (см.
    рис.).
    Определите силу натяжения нити и ускорения брусков.
    Силами трения пренебречь, массы брусков
    равны $m_1 = 11\,\text{кг}$ и $m_2 = 10\,\text{кг}$.
    % $g = 10\,\frac{\text{м}}{\text{с}^{2}}$.

    \begin{tikzpicture}[x=1.5cm,y=1.5cm,thick]
        \draw
            (-0.4, 0) rectangle (-0.2, 1.2)
            (0.15, 0.5) rectangle (0.45, 1)
            (0, 2) circle [radius=0.3] -- ++(up:0.5)
            (-0.3, 1.2) -- ++(up:0.8)
            (0.3, 1) -- ++(up:1)
            (-0.7, 2.5) -- (0.7, 2.5)
            ;
        \draw[pattern={Lines[angle=51,distance=3pt]},pattern color=black,draw=none] (-0.7, 2.5) rectangle (0.7, 2.75);
        \node [left] (left) at (-0.4, 0.6) { $m_1$ };
        \node [right] (right) at (0.4, 0.75) { $m_2$ };
    \end{tikzpicture}
}
\answer{%
    Предположим, что левый брусок ускоряется вверх, тогда правый ускоряется вниз (с тем же ускорением).
    Запишем 2-й закон Ньютона 2 раза (для обоих тел) в проекции на вертикальную оси, направив её вверх.
    \begin{align*}
        &\begin{cases}
            T - m_1g = m_1a, \\
            T - m_2g = -m_2a,
        \end{cases} \\
        &\begin{cases}
            m_2g - m_1g = m_1a + m_2a, \\
            T = m_1a + m_1g, \\
        \end{cases} \\
        a &= \frac{m_2 - m_1}{m_1 + m_2} \cdot g = \frac{10\,\text{кг} - 11\,\text{кг}}{11\,\text{кг} + 10\,\text{кг}} \cdot 10\,\frac{\text{м}}{\text{с}^{2}} \approx -0{,}4800\,\frac{\text{м}}{\text{c}^{2}}, \\
        T &= m_1(a + g) = m_1 \cdot g \cdot \cbr{\frac{m_2 - m_1}{m_1 + m_2} + 1} = m_1 \cdot g \cdot \frac{2m_2}{m_1 + m_2} = \\
            &= \frac{2 m_2 m_1 g}{m_1 + m_2} = \frac{2 \cdot 10\,\text{кг} \cdot 11\,\text{кг} \cdot 10\,\frac{\text{м}}{\text{с}^{2}}}{11\,\text{кг} + 10\,\text{кг}} \approx 104{,}8\,\text{Н}.
    \end{align*}
    Отрицательный ответ говорит, что мы лишь не угадали с направлением ускорений.
    Сила же всегда положительна.
}
\solutionspace{80pt}

\tasknumber{7}%
\task{%
    Тело массой $2\,\text{кг}$ лежит на горизонтальной поверхности.
    Коэффициент трения между поверхностью и телом $0{,}15$.
    К телу приложена горизонтальная сила $5{,}5\,\text{Н}$.
    Определите силу трения, действующую на тело, и ускорение тела.
    % $g = 10\,\frac{\text{м}}{\text{с}^{2}}$.
}
\answer{%
    \begin{align*}
    &F_\text{ трения покоя $\max$ } = \mu N = \mu m g = 0{,}15 \cdot 2\,\text{кг} \cdot 10\,\frac{\text{м}}{\text{с}^{2}} = 3{,}00\,\text{Н}, \\
    &F_\text{ трения покоя $\max$ } \le F \implies F_\text{ трения } = 3{,}00\,\text{Н}, a = \frac{F - F_\text{ трения }}{ m } = 1{,}25\,\frac{\text{м}}{\text{c}^{2}}, \\
    &\text{при равенстве возможны оба варианта: и едет, и не едет, но на ответы это не влияет.}
    \end{align*}
}
\solutionspace{120pt}

\tasknumber{8}%
\task{%
    Определите плотность неизвестного вещества, если известно, что опускании тела из него
    в подсолнечное масло оно будет плавать и на треть выступать над поверхностью жидкости.
}
\answer{%
    $F_\text{Арх.} = F_\text{тяж.} \implies \rho_\text{ж.} g V_\text{погр.} = m g \implies\rho_\text{ж.} g \cbr{V -\frac V3} = \rho V g \implies \rho = \rho_\text{ж.}\cbr{1 -\frac 13} \approx 600\,\frac{\text{кг}}{\text{м}^{3}}$
}
\solutionspace{120pt}

\tasknumber{9}%
\task{%
    	Определите силу, действующую на левую опору однородного горизонтального стержня длиной $l = 5\,\text{м}$
    	и массой $M = 1\,\text{кг}$, к которому подвешен груз массой $m = 2\,\text{кг}$ на расстоянии $2\,\text{м}$ от правого конца (см.
    рис.).

        \begin{tikzpicture}[thick]
            \draw
                (-2, -0.1) rectangle (2, 0.1)
                (-0.5, -0.1) -- (-0.5, -1)
                (-0.7, -1) rectangle (-0.3, -1.3)
           		(-2, -0.1) -- +(0.15,-0.9) -- +(-0.15,-0.9) -- cycle
            	(2, -0.1) -- +(0.15,-0.9) -- +(-0.15,-0.9) -- cycle
            ;
            \draw[pattern={Lines[angle=51,distance=2pt]},pattern color=black,draw=none]
            	(-2.15, -1.15) rectangle +(0.3, 0.15)
            	(2.15, -1.15) rectangle +(-0.3, 0.15)
            ;
            \node [right] (m_small) at (-0.3, -1.15) { $m$ };
            \node [above] (M_big) at (0, 0.1) { $M$ };
        \end{tikzpicture}
}
\answer{%
    \begin{align*}
        &\begin{cases}
            F_1 + F_2 - mg - Mg= 0, \\
            F_1 \cdot 0 - mg \cdot a - Mg \cdot \frac l2 + F_2 \cdot l = 0,
        \end{cases} \\
        F_2 &= \frac{mga + Mg\frac l2}l = \frac al \cdot mg + \frac{Mg}2 \approx 17{,}0\,\text{Н}, \\
        F_1 &= mg + Mg - F_2 = mg + Mg - \frac al \cdot mg - \frac{Mg}2 = \frac bl \cdot mg + \frac{Mg}2 \approx 13{,}0\,\text{Н}.
    \end{align*}
}
\solutionspace{80pt}

\tasknumber{10}%
\task{%
    Тонкий однородный лом длиной $1\,\text{м}$ и массой $30\,\text{кг}$ лежит на горизонтальной поверхности.
    \begin{itemize}
        \item Какую минимальную силу надо приложить к одному из его концов, чтобы оторвать его от этой поверхности?
        \item Какую минимальную работу надо совершить, чтобы поставить его на землю в вертикальное положение?
    \end{itemize}
    % Примите $g = 10\,\frac{\text{м}}{\text{с}^{2}}$.
}
\answer{%
    $F = \frac{mg}2 \approx 300\,\text{Н}, A = mg\frac l2 = 150\,\text{Дж}$
}
\solutionspace{120pt}

\tasknumber{11}%
\task{%
    Определите работу силы, которая обеспечит спуск тела массой $3\,\text{кг}$ на высоту $10\,\text{м}$ с постоянным ускорением $6\,\frac{\text{м}}{\text{c}^{2}}$.
    % Примите $g = 10\,\frac{\text{м}}{\text{с}^{2}}$.
}
\answer{%
    \begin{align*}
    &\text{Для подъёма:} A = Fh = (mg + ma) h = m(g+a)h, \\
    &\text{Для спуска:} A = -Fh = -(mg - ma) h = -m(g-a)h, \\
    &\text{В результате получаем:} -120\,\text{Дж}.
    \end{align*}
}
\solutionspace{60pt}

\tasknumber{12}%
\task{%
    Тело бросили вертикально вверх со скоростью $14\,\frac{\text{м}}{\text{c}}$.
    На какой высоте кинетическая энергия тела составит половину от потенциальной?
}
\answer{%
    \begin{align*}
    &0 + \frac{mv_0^2}2 = E_p + E_k, E_k = \frac 12 E_p \implies \\
    &\implies \frac{mv_0^2}2 = E_p + \frac 12 E_p = E_p\cbr{1 + \frac 12} = mgh\cbr{1 + \frac 12} \implies \\
    &\implies h = \frac{\frac{mv_0^2}2}{mg\cbr{1 + \frac 12}} = \frac{v_0^2}{2g} \cdot \frac 1{1 + \frac 12} \approx 6{,}5\,\text{м}.
    \end{align*}
}
\solutionspace{100pt}

\tasknumber{13}%
\task{%
    Плотность воздуха при нормальных условиях равна $1{,}3\,\frac{\text{кг}}{\text{м}^{3}}$.
    Чему равна плотность воздуха
    при температуре $100\celsius$ и давлении $50\,\text{кПа}$?
}
\answer{%
    \begin{align*}
    &\text{В общем случае:} PV = \frac m{\mu} RT \implies \rho = \frac mV = \frac m{\frac{\frac m{\mu} RT}P} = \frac{P\mu}{RT}, \\
    &\text{У нас 2 состояния:} \rho_1 = \frac{P_1\mu}{RT_1}, \rho_2 = \frac{P_2\mu}{RT_2} \implies \frac{\rho_2}{\rho_1} = \frac{\frac{P_2\mu}{RT_2}}{\frac{P_1\mu}{RT_1}} = \frac{P_2T_1}{P_1T_2} \implies \\
    &\implies \rho_2 = \rho_1 \cdot  \frac{P_2T_1}{P_1T_2} = 1{,}3\,\frac{\text{кг}}{\text{м}^{3}} \cdot \frac{50\,\text{кПа} \cdot 273\units{К}}{100\,\text{кПа} \cdot 373\units{К}} \approx 0{,}48\,\frac{\text{кг}}{\text{м}^{3}}.
    \end{align*}
}
\solutionspace{120pt}

\tasknumber{14}%
\task{%
    Небольшую цилиндрическую пробирку с воздухом погружают на некоторую глубину в глубокое пресное озеро,
    после чего воздух занимает в ней лишь четвертую часть от общего объема.
    Определите глубину, на которую погрузили пробирку.
    Температуру считать постоянной $T = 287\,\text{К}$, давлением паров воды пренебречь,
    атмосферное давление принять равным $p_{\text{aтм}} = 100\,\text{кПа}$.
}
\answer{%
    \begin{align*}
    T\text{— const} &\implies P_1V_1 = \nu RT = P_2V_2.
    \\
    V_2 = \frac 14 V_1 &\implies P_1V_1 = P_2 \cdot \frac 14V_1 \implies P_2 = 4P_1 = 4p_{\text{aтм}}.
    \\
    P_2 = p_{\text{aтм}} + \rho_{\text{в}} g h \implies h = \frac{P_2 - p_{\text{aтм}}}{\rho_{\text{в}} g} &= \frac{4p_{\text{aтм}} - p_{\text{aтм}}}{\rho_{\text{в}} g} = \frac{3 \cdot p_{\text{aтм}}}{\rho_{\text{в}} g} =  \\
     &= \frac{3 \cdot 100\,\text{кПа}}{1000\,\frac{\text{кг}}{\text{м}^{3}} \cdot  10\,\frac{\text{м}}{\text{с}^{2}}} \approx 30\,\text{м}.
    \end{align*}
}
\solutionspace{120pt}

\tasknumber{15}%
\task{%
    Газу сообщили некоторое количество теплоты,
    при этом половину его он потратил на совершение работы,
    одновременно увеличив свою внутреннюю энергию на $1500\,\text{Дж}$.
    Определите работу, совершённую газом.
}
\answer{%
    \begin{align*}
    Q &= A' + \Delta U, A' = \frac 12 Q \implies Q \cdot \cbr{1 - \frac 12} = \Delta U \implies Q = \frac{\Delta U}{1 - \frac 12} = \frac{ 1500\,\text{Дж} }{1 - \frac 12} \approx 3000\,\text{Дж}.
    \\
    A' &= \frac 12 Q
        = \frac 12 \cdot \frac{\Delta U}{1 - \frac 12}
        = \frac{\Delta U}{2 - 1}
        = \frac{ 1500\,\text{Дж} }{2 - 1} \approx 1500\,\text{Дж}.
    \end{align*}
}
\solutionspace{60pt}

\tasknumber{16}%
\task{%
    Два конденсатора ёмкостей $C_1 = 20\,\text{нФ}$ и $C_2 = 30\,\text{нФ}$ последовательно подключают
    к источнику напряжения $V = 200\,\text{В}$ (см.
    рис.).
    % Определите заряды каждого из конденсаторов.
    Определите заряд второго конденсатора.

    \begin{tikzpicture}[circuit ee IEC, semithick]
        \draw  (0, 0) to [capacitor={info={$C_1$}}] (1, 0)
                       to [capacitor={info={$C_2$}}] (2, 0)
        ;
        % \draw [-o] (0, 0) -- ++(-0.5, 0) node[left] {$-$};
        % \draw [-o] (2, 0) -- ++(0.5, 0) node[right] {$+$};
        \draw [-o] (0, 0) -- ++(-0.5, 0) node[left] {};
        \draw [-o] (2, 0) -- ++(0.5, 0) node[right] {};
    \end{tikzpicture}
}
\answer{%
    $
        Q_1
            = Q_2
            = CV
            = \frac{ V }{\frac1{C_1} + \frac1{C_2}}
            = \frac{C_1C_2V}{C_1 + C_2}
            = \frac{
                20\,\text{нФ} \cdot 30\,\text{нФ} \cdot 200\,\text{В}
            }{
                20\,\text{нФ} + 30\,\text{нФ}
            }
            = 2{,}40\,\text{мкКл}
    $
}
\solutionspace{120pt}

\tasknumber{17}%
\task{%
    В вакууме вдоль одной прямой расположены три отрицательных заряда так,
    что расстояние между соседними зарядами равно $r$.
    Сделайте рисунок,
    и определите силу, действующую на крайний заряд.
    Модули всех зарядов равны $Q$ ($Q > 0$).
}
\answer{%
    $F = \sum_i F_i = \ldots = \frac54 \frac{kQ^2}{r^2}.$
}
\solutionspace{80pt}

\tasknumber{18}%
\task{%
    Юлия проводит эксперименты c 2 кусками одинаковой стальной проволки, причём второй кусок в четыре раза длиннее первого.
    В одном из экспериментов Юлия подаёт на первый кусок проволки напряжение в три раза раз больше, чем на второй.
    Определите отношения в двух проволках в этом эксперименте (второй к первой):
    \begin{itemize}
        \item отношение сил тока,
        \item отношение выделяющихся мощностей.
    \end{itemize}
}
\answer{%
    $R_2 = 4R_1, U_1 = 3U_2 \implies  \eli_2 / \eli_1 = \frac{U_2 / R_2}{U_1 / R_1} = \frac{U_2}{U_1} \cdot \frac{R_1}{R_2} = \frac1{12}, P_2 / P_1 = \frac{U_2^2 / R_2}{U_1^2 / R_1} = \sqr{\frac{U_2}{U_1}} \cdot \frac{R_1}{R_2} = \frac1{36}.$
}

\variantsplitter

\addpersonalvariant{Алексей Алимпиев}

\tasknumber{1}%
\task{%
    Женя стартует на мотоцикле и в течение $t = 3\,\text{c}$ двигается с постоянным ускорением $2{,}5\,\frac{\text{м}}{\text{с}^{2}}$.
    Определите
    \begin{itemize}
        \item какую скорость при этом удастся достичь,
        \item какой путь за это время будет пройден,
        \item среднюю скорость за всё время движения, если после начального ускорения продолжить движение равномерно ещё в течение времени $3t$
    \end{itemize}
}
\answer{%
    \begin{align*}
    v &= v_0 + a t = at = 2{,}5\,\frac{\text{м}}{\text{с}^{2}} \cdot 3\,\text{c} = 7{,}5\,\frac{\text{м}}{\text{с}}, \\
    s_x &= v_0t + \frac{a t^2}2 = \frac{a t^2}2 = \frac{2{,}5\,\frac{\text{м}}{\text{с}^{2}} \cdot \sqr{ 3\,\text{c} }}2 = 11{,}2\,\text{м}, \\
    v_\text{сред.} &= \frac{s_\text{общ}}{t_\text{общ.}} = \frac{s_x + v \cdot 3t}{t + 3t} = \frac{\frac{a t^2}2 + at \cdot 3t}{t (1 + 3)} = \\
    &= at \cdot \frac{\frac 12 + 3}{1 + 3} = 2{,}5\,\frac{\text{м}}{\text{с}^{2}} \cdot 3\,\text{c} \cdot \frac{\frac 12 + 3}{1 + 3} \approx 6{,}56\,\frac{\text{м}}{\text{c}}.
    \end{align*}
}
\solutionspace{120pt}

\tasknumber{2}%
\task{%
    Какой путь тело пройдёт за вторую секунду после начала свободного падения?
    Какую скорость в начале этой секунды оно имеет?
}
\answer{%
    \begin{align*}
    s &= -s_y = -(y_2-y_1) = y_1 - y_2 = \cbr{y_{0y} + v_{0y}t_1 - \frac{gt_1^2}2} - \cbr{y_{0y} + v_{0y}t_2 - \frac{gt_2^2}2} = \\
    &= \frac{gt_2^2}2 - \frac{gt_1^2}2 = \frac g2\cbr{t_2^2 - t_1^2} = 15{,}0\,\text{м}, \\
    v_y &= v_{0y} - gt = -gt = 10\,\frac{\text{м}}{\text{с}^{2}} \cdot 1\,\text{с} = -10\,\frac{\text{м}}{\text{с}}.
    \end{align*}
}
\solutionspace{120pt}

\tasknumber{3}%
\task{%
    Карусель диаметром $4\,\text{м}$ равномерно совершает 6 оборотов в минуту.
    Определите
    \begin{itemize}
        \item период и частоту её обращения,
        \item скорость и ускорение крайних её точек.
    \end{itemize}
}
\answer{%
    \begin{align*}
    t &= 60\,\text{с}, r = 2{,}0\,\text{м}, n = 6\units{оборотов}, \\
    T &= \frac tN = \frac{ 60\,\text{с} }{6} \approx 10{,}00\,\text{c}, \\
    \nu &= \frac 1T = \frac{6}{ 60\,\text{с} } \approx 0{,}10\,\text{Гц}, \\
    v &= \frac{2 \pi r}{T} = \frac{2 \pi r}{T} =  \frac{2 \pi r n}{t} \approx 1{,}26\,\frac{\text{м}}{\text{c}}, \\
    a &= \frac{v^2}{r} =  \frac{4 \pi^2 r n^2}{t^2} \approx 0{,}79\,\frac{\text{м}}{\text{с}^{2}}.
    \end{align*}
}
\solutionspace{80pt}

\tasknumber{4}%
\task{%
    Даша стоит на обрыве над рекой и методично и строго горизонтально кидает в неё камушки.
    За этим всем наблюдает экспериментатор Глюк, который уже выяснил, что камушки падают в реку спустя $1{,}3\,\text{с}$ после броска,
    а вот дальность полёта оценить сложнее: придётся лезть в воду.
    Выручите Глюка и определите:
    \begin{itemize}
        \item высоту обрыва (вместе с ростом Даши).
        \item дальность полёта камушков (по горизонтали) и их скорость при падении, приняв начальную скорость броска равной $v_0 = 13\,\frac{\text{м}}{\text{с}}$.
    \end{itemize}
    Сопротивлением воздуха пренебречь.
}
\answer{%
    \begin{align*}
    y &= y_0 + v_{0y}t - \frac{gt^2}2 = h - \frac{gt^2}2, \qquad y(\tau) = 0 \implies h - \frac{g\tau^2}2 = 0 \implies h = \frac{g\tau^2}2 \approx 8{,}5\,\text{м}.
    \\
    x &= x_0 + v_{0x}t = v_0t \implies L = v_0\tau \approx 16{,}9\,\text{м}.
    \\
    &v = \sqrt{v_x^2 + v_y^2} = \sqrt{v_{0x}^2 + \sqr{v_{0y} - g\tau}} = \sqrt{v_0^2 + \sqr{g\tau}} \approx 18{,}4\,\frac{\text{м}}{\text{c}}.
    \end{align*}
}
\solutionspace{120pt}

\tasknumber{5}%
\task{%
    Пять одинаковых брусков массой $2\,\text{кг}$ каждый лежат на гладком горизонтальном столе.
    Бруски пронумерованы от 1 до 5 и последовательно связаны между собой
    невесомыми нерастяжимыми нитями: 1 со 2, 2 с 3 (ну и с 1) и т.д.
    Экспериментатор Глюк прикладывает постоянную горизонтальную силу $90\,\text{Н}$ к бруску с наибольшим номером.
    С каким ускорением двигается система? Чему равна сила натяжения нити, связывающей бруски 3 и 4?
}
\answer{%
    \begin{align*}
    a &= \frac{F}{5 m} = \frac{90\,\text{Н}}{5 \cdot 2\,\text{кг}} \approx 9{,}0\,\frac{\text{м}}{\text{c}^{2}}, \\
    T &= m'a = 3m \cdot \frac{F}{5 m} = \frac{3}{5} F \approx 54{,}0\,\text{Н}.
    \end{align*}
}
\solutionspace{120pt}

\tasknumber{6}%
\task{%
    Два бруска связаны лёгкой нерастяжимой нитью и перекинуты через неподвижный блок (см.
    рис.).
    Определите силу натяжения нити и ускорения брусков.
    Силами трения пренебречь, массы брусков
    равны $m_1 = 8\,\text{кг}$ и $m_2 = 6\,\text{кг}$.
    % $g = 10\,\frac{\text{м}}{\text{с}^{2}}$.

    \begin{tikzpicture}[x=1.5cm,y=1.5cm,thick]
        \draw
            (-0.4, 0) rectangle (-0.2, 1.2)
            (0.15, 0.5) rectangle (0.45, 1)
            (0, 2) circle [radius=0.3] -- ++(up:0.5)
            (-0.3, 1.2) -- ++(up:0.8)
            (0.3, 1) -- ++(up:1)
            (-0.7, 2.5) -- (0.7, 2.5)
            ;
        \draw[pattern={Lines[angle=51,distance=3pt]},pattern color=black,draw=none] (-0.7, 2.5) rectangle (0.7, 2.75);
        \node [left] (left) at (-0.4, 0.6) { $m_1$ };
        \node [right] (right) at (0.4, 0.75) { $m_2$ };
    \end{tikzpicture}
}
\answer{%
    Предположим, что левый брусок ускоряется вверх, тогда правый ускоряется вниз (с тем же ускорением).
    Запишем 2-й закон Ньютона 2 раза (для обоих тел) в проекции на вертикальную оси, направив её вверх.
    \begin{align*}
        &\begin{cases}
            T - m_1g = m_1a, \\
            T - m_2g = -m_2a,
        \end{cases} \\
        &\begin{cases}
            m_2g - m_1g = m_1a + m_2a, \\
            T = m_1a + m_1g, \\
        \end{cases} \\
        a &= \frac{m_2 - m_1}{m_1 + m_2} \cdot g = \frac{6\,\text{кг} - 8\,\text{кг}}{8\,\text{кг} + 6\,\text{кг}} \cdot 10\,\frac{\text{м}}{\text{с}^{2}} \approx -1{,}4300\,\frac{\text{м}}{\text{c}^{2}}, \\
        T &= m_1(a + g) = m_1 \cdot g \cdot \cbr{\frac{m_2 - m_1}{m_1 + m_2} + 1} = m_1 \cdot g \cdot \frac{2m_2}{m_1 + m_2} = \\
            &= \frac{2 m_2 m_1 g}{m_1 + m_2} = \frac{2 \cdot 6\,\text{кг} \cdot 8\,\text{кг} \cdot 10\,\frac{\text{м}}{\text{с}^{2}}}{8\,\text{кг} + 6\,\text{кг}} \approx 68{,}6\,\text{Н}.
    \end{align*}
    Отрицательный ответ говорит, что мы лишь не угадали с направлением ускорений.
    Сила же всегда положительна.
}
\solutionspace{80pt}

\tasknumber{7}%
\task{%
    Тело массой $2{,}7\,\text{кг}$ лежит на горизонтальной поверхности.
    Коэффициент трения между поверхностью и телом $0{,}15$.
    К телу приложена горизонтальная сила $5{,}5\,\text{Н}$.
    Определите силу трения, действующую на тело, и ускорение тела.
    % $g = 10\,\frac{\text{м}}{\text{с}^{2}}$.
}
\answer{%
    \begin{align*}
    &F_\text{ трения покоя $\max$ } = \mu N = \mu m g = 0{,}15 \cdot 2{,}7\,\text{кг} \cdot 10\,\frac{\text{м}}{\text{с}^{2}} = 4{,}05\,\text{Н}, \\
    &F_\text{ трения покоя $\max$ } \le F \implies F_\text{ трения } = 4{,}05\,\text{Н}, a = \frac{F - F_\text{ трения }}{ m } = 0{,}54\,\frac{\text{м}}{\text{c}^{2}}, \\
    &\text{при равенстве возможны оба варианта: и едет, и не едет, но на ответы это не влияет.}
    \end{align*}
}
\solutionspace{120pt}

\tasknumber{8}%
\task{%
    Определите плотность неизвестного вещества, если известно, что опускании тела из него
    в подсолнечное масло оно будет плавать и на четверть выступать над поверхностью жидкости.
}
\answer{%
    $F_\text{Арх.} = F_\text{тяж.} \implies \rho_\text{ж.} g V_\text{погр.} = m g \implies\rho_\text{ж.} g \cbr{V -\frac V4} = \rho V g \implies \rho = \rho_\text{ж.}\cbr{1 -\frac 14} \approx 675\,\frac{\text{кг}}{\text{м}^{3}}$
}
\solutionspace{120pt}

\tasknumber{9}%
\task{%
    	Определите силу, действующую на левую опору однородного горизонтального стержня длиной $l = 9\,\text{м}$
    	и массой $M = 5\,\text{кг}$, к которому подвешен груз массой $m = 3\,\text{кг}$ на расстоянии $4\,\text{м}$ от правого конца (см.
    рис.).

        \begin{tikzpicture}[thick]
            \draw
                (-2, -0.1) rectangle (2, 0.1)
                (-0.5, -0.1) -- (-0.5, -1)
                (-0.7, -1) rectangle (-0.3, -1.3)
           		(-2, -0.1) -- +(0.15,-0.9) -- +(-0.15,-0.9) -- cycle
            	(2, -0.1) -- +(0.15,-0.9) -- +(-0.15,-0.9) -- cycle
            ;
            \draw[pattern={Lines[angle=51,distance=2pt]},pattern color=black,draw=none]
            	(-2.15, -1.15) rectangle +(0.3, 0.15)
            	(2.15, -1.15) rectangle +(-0.3, 0.15)
            ;
            \node [right] (m_small) at (-0.3, -1.15) { $m$ };
            \node [above] (M_big) at (0, 0.1) { $M$ };
        \end{tikzpicture}
}
\answer{%
    \begin{align*}
        &\begin{cases}
            F_1 + F_2 - mg - Mg= 0, \\
            F_1 \cdot 0 - mg \cdot a - Mg \cdot \frac l2 + F_2 \cdot l = 0,
        \end{cases} \\
        F_2 &= \frac{mga + Mg\frac l2}l = \frac al \cdot mg + \frac{Mg}2 \approx 41{,}7\,\text{Н}, \\
        F_1 &= mg + Mg - F_2 = mg + Mg - \frac al \cdot mg - \frac{Mg}2 = \frac bl \cdot mg + \frac{Mg}2 \approx 38{,}3\,\text{Н}.
    \end{align*}
}
\solutionspace{80pt}

\tasknumber{10}%
\task{%
    Тонкий однородный кусок арматуры длиной $1\,\text{м}$ и массой $10\,\text{кг}$ лежит на горизонтальной поверхности.
    \begin{itemize}
        \item Какую минимальную силу надо приложить к одному из его концов, чтобы оторвать его от этой поверхности?
        \item Какую минимальную работу надо совершить, чтобы поставить его на землю в вертикальное положение?
    \end{itemize}
    % Примите $g = 10\,\frac{\text{м}}{\text{с}^{2}}$.
}
\answer{%
    $F = \frac{mg}2 \approx 100\,\text{Н}, A = mg\frac l2 = 50\,\text{Дж}$
}
\solutionspace{120pt}

\tasknumber{11}%
\task{%
    Определите работу силы, которая обеспечит подъём тела массой $2\,\text{кг}$ на высоту $2\,\text{м}$ с постоянным ускорением $4\,\frac{\text{м}}{\text{c}^{2}}$.
    % Примите $g = 10\,\frac{\text{м}}{\text{с}^{2}}$.
}
\answer{%
    \begin{align*}
    &\text{Для подъёма:} A = Fh = (mg + ma) h = m(g+a)h, \\
    &\text{Для спуска:} A = -Fh = -(mg - ma) h = -m(g-a)h, \\
    &\text{В результате получаем:} 56\,\text{Дж}.
    \end{align*}
}
\solutionspace{60pt}

\tasknumber{12}%
\task{%
    Тело бросили вертикально вверх со скоростью $14\,\frac{\text{м}}{\text{c}}$.
    На какой высоте кинетическая энергия тела составит треть от потенциальной?
}
\answer{%
    \begin{align*}
    &0 + \frac{mv_0^2}2 = E_p + E_k, E_k = \frac 13 E_p \implies \\
    &\implies \frac{mv_0^2}2 = E_p + \frac 13 E_p = E_p\cbr{1 + \frac 13} = mgh\cbr{1 + \frac 13} \implies \\
    &\implies h = \frac{\frac{mv_0^2}2}{mg\cbr{1 + \frac 13}} = \frac{v_0^2}{2g} \cdot \frac 1{1 + \frac 13} \approx 7{,}4\,\text{м}.
    \end{align*}
}
\solutionspace{100pt}

\tasknumber{13}%
\task{%
    Плотность воздуха при нормальных условиях равна $1{,}3\,\frac{\text{кг}}{\text{м}^{3}}$.
    Чему равна плотность воздуха
    при температуре $100\celsius$ и давлении $80\,\text{кПа}$?
}
\answer{%
    \begin{align*}
    &\text{В общем случае:} PV = \frac m{\mu} RT \implies \rho = \frac mV = \frac m{\frac{\frac m{\mu} RT}P} = \frac{P\mu}{RT}, \\
    &\text{У нас 2 состояния:} \rho_1 = \frac{P_1\mu}{RT_1}, \rho_2 = \frac{P_2\mu}{RT_2} \implies \frac{\rho_2}{\rho_1} = \frac{\frac{P_2\mu}{RT_2}}{\frac{P_1\mu}{RT_1}} = \frac{P_2T_1}{P_1T_2} \implies \\
    &\implies \rho_2 = \rho_1 \cdot  \frac{P_2T_1}{P_1T_2} = 1{,}3\,\frac{\text{кг}}{\text{м}^{3}} \cdot \frac{80\,\text{кПа} \cdot 273\units{К}}{100\,\text{кПа} \cdot 373\units{К}} \approx 0{,}76\,\frac{\text{кг}}{\text{м}^{3}}.
    \end{align*}
}
\solutionspace{120pt}

\tasknumber{14}%
\task{%
    Небольшую цилиндрическую пробирку с воздухом погружают на некоторую глубину в глубокое пресное озеро,
    после чего воздух занимает в ней лишь шестую часть от общего объема.
    Определите глубину, на которую погрузили пробирку.
    Температуру считать постоянной $T = 287\,\text{К}$, давлением паров воды пренебречь,
    атмосферное давление принять равным $p_{\text{aтм}} = 100\,\text{кПа}$.
}
\answer{%
    \begin{align*}
    T\text{— const} &\implies P_1V_1 = \nu RT = P_2V_2.
    \\
    V_2 = \frac 16 V_1 &\implies P_1V_1 = P_2 \cdot \frac 16V_1 \implies P_2 = 6P_1 = 6p_{\text{aтм}}.
    \\
    P_2 = p_{\text{aтм}} + \rho_{\text{в}} g h \implies h = \frac{P_2 - p_{\text{aтм}}}{\rho_{\text{в}} g} &= \frac{6p_{\text{aтм}} - p_{\text{aтм}}}{\rho_{\text{в}} g} = \frac{5 \cdot p_{\text{aтм}}}{\rho_{\text{в}} g} =  \\
     &= \frac{5 \cdot 100\,\text{кПа}}{1000\,\frac{\text{кг}}{\text{м}^{3}} \cdot  10\,\frac{\text{м}}{\text{с}^{2}}} \approx 50\,\text{м}.
    \end{align*}
}
\solutionspace{120pt}

\tasknumber{15}%
\task{%
    Газу сообщили некоторое количество теплоты,
    при этом четверть его он потратил на совершение работы,
    одновременно увеличив свою внутреннюю энергию на $1500\,\text{Дж}$.
    Определите количество теплоты, сообщённое газу.
}
\answer{%
    \begin{align*}
    Q &= A' + \Delta U, A' = \frac 14 Q \implies Q \cdot \cbr{1 - \frac 14} = \Delta U \implies Q = \frac{\Delta U}{1 - \frac 14} = \frac{ 1500\,\text{Дж} }{1 - \frac 14} \approx 2000\,\text{Дж}.
    \\
    A' &= \frac 14 Q
        = \frac 14 \cdot \frac{\Delta U}{1 - \frac 14}
        = \frac{\Delta U}{4 - 1}
        = \frac{ 1500\,\text{Дж} }{4 - 1} \approx 500\,\text{Дж}.
    \end{align*}
}
\solutionspace{60pt}

\tasknumber{16}%
\task{%
    Два конденсатора ёмкостей $C_1 = 60\,\text{нФ}$ и $C_2 = 30\,\text{нФ}$ последовательно подключают
    к источнику напряжения $U = 200\,\text{В}$ (см.
    рис.).
    % Определите заряды каждого из конденсаторов.
    Определите заряд первого конденсатора.

    \begin{tikzpicture}[circuit ee IEC, semithick]
        \draw  (0, 0) to [capacitor={info={$C_1$}}] (1, 0)
                       to [capacitor={info={$C_2$}}] (2, 0)
        ;
        % \draw [-o] (0, 0) -- ++(-0.5, 0) node[left] {$-$};
        % \draw [-o] (2, 0) -- ++(0.5, 0) node[right] {$+$};
        \draw [-o] (0, 0) -- ++(-0.5, 0) node[left] {};
        \draw [-o] (2, 0) -- ++(0.5, 0) node[right] {};
    \end{tikzpicture}
}
\answer{%
    $
        Q_1
            = Q_2
            = CU
            = \frac{ U }{\frac1{C_1} + \frac1{C_2}}
            = \frac{C_1C_2U}{C_1 + C_2}
            = \frac{
                60\,\text{нФ} \cdot 30\,\text{нФ} \cdot 200\,\text{В}
            }{
                60\,\text{нФ} + 30\,\text{нФ}
            }
            = 4{,}00\,\text{мкКл}
    $
}
\solutionspace{120pt}

\tasknumber{17}%
\task{%
    В вакууме вдоль одной прямой расположены три отрицательных заряда так,
    что расстояние между соседними зарядами равно $a$.
    Сделайте рисунок,
    и определите силу, действующую на крайний заряд.
    Модули всех зарядов равны $q$ ($q > 0$).
}
\answer{%
    $F = \sum_i F_i = \ldots = \frac54 \frac{kq^2}{a^2}.$
}
\solutionspace{80pt}

\tasknumber{18}%
\task{%
    Юлия проводит эксперименты c 2 кусками одинаковой алюминиевой проволки, причём второй кусок в восемь раз длиннее первого.
    В одном из экспериментов Юлия подаёт на первый кусок проволки напряжение в три раза раз больше, чем на второй.
    Определите отношения в двух проволках в этом эксперименте (второй к первой):
    \begin{itemize}
        \item отношение сил тока,
        \item отношение выделяющихся мощностей.
    \end{itemize}
}
\answer{%
    $R_2 = 8R_1, U_1 = 3U_2 \implies  \eli_2 / \eli_1 = \frac{U_2 / R_2}{U_1 / R_1} = \frac{U_2}{U_1} \cdot \frac{R_1}{R_2} = \frac1{24}, P_2 / P_1 = \frac{U_2^2 / R_2}{U_1^2 / R_1} = \sqr{\frac{U_2}{U_1}} \cdot \frac{R_1}{R_2} = \frac1{72}.$
}

\variantsplitter

\addpersonalvariant{Евгений Васин}

\tasknumber{1}%
\task{%
    Женя стартует на велосипеде и в течение $t = 5\,\text{c}$ двигается с постоянным ускорением $0{,}5\,\frac{\text{м}}{\text{с}^{2}}$.
    Определите
    \begin{itemize}
        \item какую скорость при этом удастся достичь,
        \item какой путь за это время будет пройден,
        \item среднюю скорость за всё время движения, если после начального ускорения продолжить движение равномерно ещё в течение времени $3t$
    \end{itemize}
}
\answer{%
    \begin{align*}
    v &= v_0 + a t = at = 0{,}5\,\frac{\text{м}}{\text{с}^{2}} \cdot 5\,\text{c} = 2{,}5\,\frac{\text{м}}{\text{с}}, \\
    s_x &= v_0t + \frac{a t^2}2 = \frac{a t^2}2 = \frac{0{,}5\,\frac{\text{м}}{\text{с}^{2}} \cdot \sqr{ 5\,\text{c} }}2 = 6{,}2\,\text{м}, \\
    v_\text{сред.} &= \frac{s_\text{общ}}{t_\text{общ.}} = \frac{s_x + v \cdot 3t}{t + 3t} = \frac{\frac{a t^2}2 + at \cdot 3t}{t (1 + 3)} = \\
    &= at \cdot \frac{\frac 12 + 3}{1 + 3} = 0{,}5\,\frac{\text{м}}{\text{с}^{2}} \cdot 5\,\text{c} \cdot \frac{\frac 12 + 3}{1 + 3} \approx 2{,}19\,\frac{\text{м}}{\text{c}}.
    \end{align*}
}
\solutionspace{120pt}

\tasknumber{2}%
\task{%
    Какой путь тело пройдёт за шестую секунду после начала свободного падения?
    Какую скорость в начале этой секунды оно имеет?
}
\answer{%
    \begin{align*}
    s &= -s_y = -(y_2-y_1) = y_1 - y_2 = \cbr{y_{0y} + v_{0y}t_1 - \frac{gt_1^2}2} - \cbr{y_{0y} + v_{0y}t_2 - \frac{gt_2^2}2} = \\
    &= \frac{gt_2^2}2 - \frac{gt_1^2}2 = \frac g2\cbr{t_2^2 - t_1^2} = 55{,}0\,\text{м}, \\
    v_y &= v_{0y} - gt = -gt = 10\,\frac{\text{м}}{\text{с}^{2}} \cdot 5\,\text{с} = -50\,\frac{\text{м}}{\text{с}}.
    \end{align*}
}
\solutionspace{120pt}

\tasknumber{3}%
\task{%
    Карусель радиусом $5\,\text{м}$ равномерно совершает 6 оборотов в минуту.
    Определите
    \begin{itemize}
        \item период и частоту её обращения,
        \item скорость и ускорение крайних её точек.
    \end{itemize}
}
\answer{%
    \begin{align*}
    t &= 60\,\text{с}, r = 5{,}0\,\text{м}, n = 6\units{оборотов}, \\
    T &= \frac tN = \frac{ 60\,\text{с} }{6} \approx 10{,}00\,\text{c}, \\
    \nu &= \frac 1T = \frac{6}{ 60\,\text{с} } \approx 0{,}10\,\text{Гц}, \\
    v &= \frac{2 \pi r}{T} = \frac{2 \pi r}{T} =  \frac{2 \pi r n}{t} \approx 3{,}14\,\frac{\text{м}}{\text{c}}, \\
    a &= \frac{v^2}{r} =  \frac{4 \pi^2 r n^2}{t^2} \approx 1{,}97\,\frac{\text{м}}{\text{с}^{2}}.
    \end{align*}
}
\solutionspace{80pt}

\tasknumber{4}%
\task{%
    Маша стоит на обрыве над рекой и методично и строго горизонтально кидает в неё камушки.
    За этим всем наблюдает экспериментатор Глюк, который уже выяснил, что камушки падают в реку спустя $1{,}7\,\text{с}$ после броска,
    а вот дальность полёта оценить сложнее: придётся лезть в воду.
    Выручите Глюка и определите:
    \begin{itemize}
        \item высоту обрыва (вместе с ростом Маши).
        \item дальность полёта камушков (по горизонтали) и их скорость при падении, приняв начальную скорость броска равной $v_0 = 15\,\frac{\text{м}}{\text{с}}$.
    \end{itemize}
    Сопротивлением воздуха пренебречь.
}
\answer{%
    \begin{align*}
    y &= y_0 + v_{0y}t - \frac{gt^2}2 = h - \frac{gt^2}2, \qquad y(\tau) = 0 \implies h - \frac{g\tau^2}2 = 0 \implies h = \frac{g\tau^2}2 \approx 14{,}4\,\text{м}.
    \\
    x &= x_0 + v_{0x}t = v_0t \implies L = v_0\tau \approx 25{,}5\,\text{м}.
    \\
    &v = \sqrt{v_x^2 + v_y^2} = \sqrt{v_{0x}^2 + \sqr{v_{0y} - g\tau}} = \sqrt{v_0^2 + \sqr{g\tau}} \approx 22{,}7\,\frac{\text{м}}{\text{c}}.
    \end{align*}
}
\solutionspace{120pt}

\tasknumber{5}%
\task{%
    Шесть одинаковых брусков массой $3\,\text{кг}$ каждый лежат на гладком горизонтальном столе.
    Бруски пронумерованы от 1 до 6 и последовательно связаны между собой
    невесомыми нерастяжимыми нитями: 1 со 2, 2 с 3 (ну и с 1) и т.д.
    Экспериментатор Глюк прикладывает постоянную горизонтальную силу $120\,\text{Н}$ к бруску с наибольшим номером.
    С каким ускорением двигается система? Чему равна сила натяжения нити, связывающей бруски 1 и 2?
}
\answer{%
    \begin{align*}
    a &= \frac{F}{6 m} = \frac{120\,\text{Н}}{6 \cdot 3\,\text{кг}} \approx 6{,}7\,\frac{\text{м}}{\text{c}^{2}}, \\
    T &= m'a = 1m \cdot \frac{F}{6 m} = \frac{1}{6} F \approx 20{,}0\,\text{Н}.
    \end{align*}
}
\solutionspace{120pt}

\tasknumber{6}%
\task{%
    Два бруска связаны лёгкой нерастяжимой нитью и перекинуты через неподвижный блок (см.
    рис.).
    Определите силу натяжения нити и ускорения брусков.
    Силами трения пренебречь, массы брусков
    равны $m_1 = 5\,\text{кг}$ и $m_2 = 14\,\text{кг}$.
    % $g = 10\,\frac{\text{м}}{\text{с}^{2}}$.

    \begin{tikzpicture}[x=1.5cm,y=1.5cm,thick]
        \draw
            (-0.4, 0) rectangle (-0.2, 1.2)
            (0.15, 0.5) rectangle (0.45, 1)
            (0, 2) circle [radius=0.3] -- ++(up:0.5)
            (-0.3, 1.2) -- ++(up:0.8)
            (0.3, 1) -- ++(up:1)
            (-0.7, 2.5) -- (0.7, 2.5)
            ;
        \draw[pattern={Lines[angle=51,distance=3pt]},pattern color=black,draw=none] (-0.7, 2.5) rectangle (0.7, 2.75);
        \node [left] (left) at (-0.4, 0.6) { $m_1$ };
        \node [right] (right) at (0.4, 0.75) { $m_2$ };
    \end{tikzpicture}
}
\answer{%
    Предположим, что левый брусок ускоряется вверх, тогда правый ускоряется вниз (с тем же ускорением).
    Запишем 2-й закон Ньютона 2 раза (для обоих тел) в проекции на вертикальную оси, направив её вверх.
    \begin{align*}
        &\begin{cases}
            T - m_1g = m_1a, \\
            T - m_2g = -m_2a,
        \end{cases} \\
        &\begin{cases}
            m_2g - m_1g = m_1a + m_2a, \\
            T = m_1a + m_1g, \\
        \end{cases} \\
        a &= \frac{m_2 - m_1}{m_1 + m_2} \cdot g = \frac{14\,\text{кг} - 5\,\text{кг}}{5\,\text{кг} + 14\,\text{кг}} \cdot 10\,\frac{\text{м}}{\text{с}^{2}} \approx 4{,}74\,\frac{\text{м}}{\text{c}^{2}}, \\
        T &= m_1(a + g) = m_1 \cdot g \cdot \cbr{\frac{m_2 - m_1}{m_1 + m_2} + 1} = m_1 \cdot g \cdot \frac{2m_2}{m_1 + m_2} = \\
            &= \frac{2 m_2 m_1 g}{m_1 + m_2} = \frac{2 \cdot 14\,\text{кг} \cdot 5\,\text{кг} \cdot 10\,\frac{\text{м}}{\text{с}^{2}}}{5\,\text{кг} + 14\,\text{кг}} \approx 73{,}7\,\text{Н}.
    \end{align*}
    Отрицательный ответ говорит, что мы лишь не угадали с направлением ускорений.
    Сила же всегда положительна.
}
\solutionspace{80pt}

\tasknumber{7}%
\task{%
    Тело массой $1{,}4\,\text{кг}$ лежит на горизонтальной поверхности.
    Коэффициент трения между поверхностью и телом $0{,}25$.
    К телу приложена горизонтальная сила $2{,}5\,\text{Н}$.
    Определите силу трения, действующую на тело, и ускорение тела.
    % $g = 10\,\frac{\text{м}}{\text{с}^{2}}$.
}
\answer{%
    \begin{align*}
    &F_\text{ трения покоя $\max$ } = \mu N = \mu m g = 0{,}25 \cdot 1{,}4\,\text{кг} \cdot 10\,\frac{\text{м}}{\text{с}^{2}} = 3{,}50\,\text{Н}, \\
    &F_\text{ трения покоя $\max$ } > F \implies F_\text{ трения } = 2{,}50\,\text{Н}, a = \frac{F - F_\text{ трения }}{ m } = 0\,\frac{\text{м}}{\text{c}^{2}}, \\
    &\text{при равенстве возможны оба варианта: и едет, и не едет, но на ответы это не влияет.}
    \end{align*}
}
\solutionspace{120pt}

\tasknumber{8}%
\task{%
    Определите плотность неизвестного вещества, если известно, что опускании тела из него
    в керосин оно будет плавать и на половину выступать над поверхностью жидкости.
}
\answer{%
    $F_\text{Арх.} = F_\text{тяж.} \implies \rho_\text{ж.} g V_\text{погр.} = m g \implies\rho_\text{ж.} g \cbr{V -\frac V2} = \rho V g \implies \rho = \rho_\text{ж.}\cbr{1 -\frac 12} \approx 400\,\frac{\text{кг}}{\text{м}^{3}}$
}
\solutionspace{120pt}

\tasknumber{9}%
\task{%
    	Определите силу, действующую на левую опору однородного горизонтального стержня длиной $l = 5\,\text{м}$
    	и массой $M = 5\,\text{кг}$, к которому подвешен груз массой $m = 2\,\text{кг}$ на расстоянии $4\,\text{м}$ от правого конца (см.
    рис.).

        \begin{tikzpicture}[thick]
            \draw
                (-2, -0.1) rectangle (2, 0.1)
                (-0.5, -0.1) -- (-0.5, -1)
                (-0.7, -1) rectangle (-0.3, -1.3)
           		(-2, -0.1) -- +(0.15,-0.9) -- +(-0.15,-0.9) -- cycle
            	(2, -0.1) -- +(0.15,-0.9) -- +(-0.15,-0.9) -- cycle
            ;
            \draw[pattern={Lines[angle=51,distance=2pt]},pattern color=black,draw=none]
            	(-2.15, -1.15) rectangle +(0.3, 0.15)
            	(2.15, -1.15) rectangle +(-0.3, 0.15)
            ;
            \node [right] (m_small) at (-0.3, -1.15) { $m$ };
            \node [above] (M_big) at (0, 0.1) { $M$ };
        \end{tikzpicture}
}
\answer{%
    \begin{align*}
        &\begin{cases}
            F_1 + F_2 - mg - Mg= 0, \\
            F_1 \cdot 0 - mg \cdot a - Mg \cdot \frac l2 + F_2 \cdot l = 0,
        \end{cases} \\
        F_2 &= \frac{mga + Mg\frac l2}l = \frac al \cdot mg + \frac{Mg}2 \approx 29{,}0\,\text{Н}, \\
        F_1 &= mg + Mg - F_2 = mg + Mg - \frac al \cdot mg - \frac{Mg}2 = \frac bl \cdot mg + \frac{Mg}2 \approx 41{,}0\,\text{Н}.
    \end{align*}
}
\solutionspace{80pt}

\tasknumber{10}%
\task{%
    Тонкий однородный шест длиной $2\,\text{м}$ и массой $10\,\text{кг}$ лежит на горизонтальной поверхности.
    \begin{itemize}
        \item Какую минимальную силу надо приложить к одному из его концов, чтобы оторвать его от этой поверхности?
        \item Какую минимальную работу надо совершить, чтобы поставить его на землю в вертикальное положение?
    \end{itemize}
    % Примите $g = 10\,\frac{\text{м}}{\text{с}^{2}}$.
}
\answer{%
    $F = \frac{mg}2 \approx 100\,\text{Н}, A = mg\frac l2 = 100\,\text{Дж}$
}
\solutionspace{120pt}

\tasknumber{11}%
\task{%
    Определите работу силы, которая обеспечит подъём тела массой $5\,\text{кг}$ на высоту $5\,\text{м}$ с постоянным ускорением $4\,\frac{\text{м}}{\text{c}^{2}}$.
    % Примите $g = 10\,\frac{\text{м}}{\text{с}^{2}}$.
}
\answer{%
    \begin{align*}
    &\text{Для подъёма:} A = Fh = (mg + ma) h = m(g+a)h, \\
    &\text{Для спуска:} A = -Fh = -(mg - ma) h = -m(g-a)h, \\
    &\text{В результате получаем:} 350\,\text{Дж}.
    \end{align*}
}
\solutionspace{60pt}

\tasknumber{12}%
\task{%
    Тело бросили вертикально вверх со скоростью $20\,\frac{\text{м}}{\text{c}}$.
    На какой высоте кинетическая энергия тела составит половину от потенциальной?
}
\answer{%
    \begin{align*}
    &0 + \frac{mv_0^2}2 = E_p + E_k, E_k = \frac 12 E_p \implies \\
    &\implies \frac{mv_0^2}2 = E_p + \frac 12 E_p = E_p\cbr{1 + \frac 12} = mgh\cbr{1 + \frac 12} \implies \\
    &\implies h = \frac{\frac{mv_0^2}2}{mg\cbr{1 + \frac 12}} = \frac{v_0^2}{2g} \cdot \frac 1{1 + \frac 12} \approx 13{,}3\,\text{м}.
    \end{align*}
}
\solutionspace{100pt}

\tasknumber{13}%
\task{%
    Плотность воздуха при нормальных условиях равна $1{,}3\,\frac{\text{кг}}{\text{м}^{3}}$.
    Чему равна плотность воздуха
    при температуре $100\celsius$ и давлении $80\,\text{кПа}$?
}
\answer{%
    \begin{align*}
    &\text{В общем случае:} PV = \frac m{\mu} RT \implies \rho = \frac mV = \frac m{\frac{\frac m{\mu} RT}P} = \frac{P\mu}{RT}, \\
    &\text{У нас 2 состояния:} \rho_1 = \frac{P_1\mu}{RT_1}, \rho_2 = \frac{P_2\mu}{RT_2} \implies \frac{\rho_2}{\rho_1} = \frac{\frac{P_2\mu}{RT_2}}{\frac{P_1\mu}{RT_1}} = \frac{P_2T_1}{P_1T_2} \implies \\
    &\implies \rho_2 = \rho_1 \cdot  \frac{P_2T_1}{P_1T_2} = 1{,}3\,\frac{\text{кг}}{\text{м}^{3}} \cdot \frac{80\,\text{кПа} \cdot 273\units{К}}{100\,\text{кПа} \cdot 373\units{К}} \approx 0{,}76\,\frac{\text{кг}}{\text{м}^{3}}.
    \end{align*}
}
\solutionspace{120pt}

\tasknumber{14}%
\task{%
    Небольшую цилиндрическую пробирку с воздухом погружают на некоторую глубину в глубокое пресное озеро,
    после чего воздух занимает в ней лишь третью часть от общего объема.
    Определите глубину, на которую погрузили пробирку.
    Температуру считать постоянной $T = 290\,\text{К}$, давлением паров воды пренебречь,
    атмосферное давление принять равным $p_{\text{aтм}} = 100\,\text{кПа}$.
}
\answer{%
    \begin{align*}
    T\text{— const} &\implies P_1V_1 = \nu RT = P_2V_2.
    \\
    V_2 = \frac 13 V_1 &\implies P_1V_1 = P_2 \cdot \frac 13V_1 \implies P_2 = 3P_1 = 3p_{\text{aтм}}.
    \\
    P_2 = p_{\text{aтм}} + \rho_{\text{в}} g h \implies h = \frac{P_2 - p_{\text{aтм}}}{\rho_{\text{в}} g} &= \frac{3p_{\text{aтм}} - p_{\text{aтм}}}{\rho_{\text{в}} g} = \frac{2 \cdot p_{\text{aтм}}}{\rho_{\text{в}} g} =  \\
     &= \frac{2 \cdot 100\,\text{кПа}}{1000\,\frac{\text{кг}}{\text{м}^{3}} \cdot  10\,\frac{\text{м}}{\text{с}^{2}}} \approx 20\,\text{м}.
    \end{align*}
}
\solutionspace{120pt}

\tasknumber{15}%
\task{%
    Газу сообщили некоторое количество теплоты,
    при этом половину его он потратил на совершение работы,
    одновременно увеличив свою внутреннюю энергию на $1500\,\text{Дж}$.
    Определите работу, совершённую газом.
}
\answer{%
    \begin{align*}
    Q &= A' + \Delta U, A' = \frac 12 Q \implies Q \cdot \cbr{1 - \frac 12} = \Delta U \implies Q = \frac{\Delta U}{1 - \frac 12} = \frac{ 1500\,\text{Дж} }{1 - \frac 12} \approx 3000\,\text{Дж}.
    \\
    A' &= \frac 12 Q
        = \frac 12 \cdot \frac{\Delta U}{1 - \frac 12}
        = \frac{\Delta U}{2 - 1}
        = \frac{ 1500\,\text{Дж} }{2 - 1} \approx 1500\,\text{Дж}.
    \end{align*}
}
\solutionspace{60pt}

\tasknumber{16}%
\task{%
    Два конденсатора ёмкостей $C_1 = 60\,\text{нФ}$ и $C_2 = 20\,\text{нФ}$ последовательно подключают
    к источнику напряжения $U = 450\,\text{В}$ (см.
    рис.).
    % Определите заряды каждого из конденсаторов.
    Определите заряд второго конденсатора.

    \begin{tikzpicture}[circuit ee IEC, semithick]
        \draw  (0, 0) to [capacitor={info={$C_1$}}] (1, 0)
                       to [capacitor={info={$C_2$}}] (2, 0)
        ;
        % \draw [-o] (0, 0) -- ++(-0.5, 0) node[left] {$-$};
        % \draw [-o] (2, 0) -- ++(0.5, 0) node[right] {$+$};
        \draw [-o] (0, 0) -- ++(-0.5, 0) node[left] {};
        \draw [-o] (2, 0) -- ++(0.5, 0) node[right] {};
    \end{tikzpicture}
}
\answer{%
    $
        Q_1
            = Q_2
            = CU
            = \frac{ U }{\frac1{C_1} + \frac1{C_2}}
            = \frac{C_1C_2U}{C_1 + C_2}
            = \frac{
                60\,\text{нФ} \cdot 20\,\text{нФ} \cdot 450\,\text{В}
            }{
                60\,\text{нФ} + 20\,\text{нФ}
            }
            = 6{,}75\,\text{мкКл}
    $
}
\solutionspace{120pt}

\tasknumber{17}%
\task{%
    В вакууме вдоль одной прямой расположены три положительных заряда так,
    что расстояние между соседними зарядами равно $l$.
    Сделайте рисунок,
    и определите силу, действующую на крайний заряд.
    Модули всех зарядов равны $Q$ ($Q > 0$).
}
\answer{%
    $F = \sum_i F_i = \ldots = \frac54 \frac{kQ^2}{l^2}.$
}
\solutionspace{80pt}

\tasknumber{18}%
\task{%
    Юлия проводит эксперименты c 2 кусками одинаковой стальной проволки, причём второй кусок в десять раз длиннее первого.
    В одном из экспериментов Юлия подаёт на первый кусок проволки напряжение в два раза раз больше, чем на второй.
    Определите отношения в двух проволках в этом эксперименте (второй к первой):
    \begin{itemize}
        \item отношение сил тока,
        \item отношение выделяющихся мощностей.
    \end{itemize}
}
\answer{%
    $R_2 = 10R_1, U_1 = 2U_2 \implies  \eli_2 / \eli_1 = \frac{U_2 / R_2}{U_1 / R_1} = \frac{U_2}{U_1} \cdot \frac{R_1}{R_2} = \frac1{20}, P_2 / P_1 = \frac{U_2^2 / R_2}{U_1^2 / R_1} = \sqr{\frac{U_2}{U_1}} \cdot \frac{R_1}{R_2} = \frac1{40}.$
}

\variantsplitter

\addpersonalvariant{Вячеслав Волохов}

\tasknumber{1}%
\task{%
    Саша стартует на лошади и в течение $t = 2\,\text{c}$ двигается с постоянным ускорением $1{,}5\,\frac{\text{м}}{\text{с}^{2}}$.
    Определите
    \begin{itemize}
        \item какую скорость при этом удастся достичь,
        \item какой путь за это время будет пройден,
        \item среднюю скорость за всё время движения, если после начального ускорения продолжить движение равномерно ещё в течение времени $3t$
    \end{itemize}
}
\answer{%
    \begin{align*}
    v &= v_0 + a t = at = 1{,}5\,\frac{\text{м}}{\text{с}^{2}} \cdot 2\,\text{c} = 3{,}0\,\frac{\text{м}}{\text{с}}, \\
    s_x &= v_0t + \frac{a t^2}2 = \frac{a t^2}2 = \frac{1{,}5\,\frac{\text{м}}{\text{с}^{2}} \cdot \sqr{ 2\,\text{c} }}2 = 3{,}0\,\text{м}, \\
    v_\text{сред.} &= \frac{s_\text{общ}}{t_\text{общ.}} = \frac{s_x + v \cdot 3t}{t + 3t} = \frac{\frac{a t^2}2 + at \cdot 3t}{t (1 + 3)} = \\
    &= at \cdot \frac{\frac 12 + 3}{1 + 3} = 1{,}5\,\frac{\text{м}}{\text{с}^{2}} \cdot 2\,\text{c} \cdot \frac{\frac 12 + 3}{1 + 3} \approx 2{,}62\,\frac{\text{м}}{\text{c}}.
    \end{align*}
}
\solutionspace{120pt}

\tasknumber{2}%
\task{%
    Какой путь тело пройдёт за пятую секунду после начала свободного падения?
    Какую скорость в конце этой секунды оно имеет?
}
\answer{%
    \begin{align*}
    s &= -s_y = -(y_2-y_1) = y_1 - y_2 = \cbr{y_{0y} + v_{0y}t_1 - \frac{gt_1^2}2} - \cbr{y_{0y} + v_{0y}t_2 - \frac{gt_2^2}2} = \\
    &= \frac{gt_2^2}2 - \frac{gt_1^2}2 = \frac g2\cbr{t_2^2 - t_1^2} = 45{,}0\,\text{м}, \\
    v_y &= v_{0y} - gt = -gt = 10\,\frac{\text{м}}{\text{с}^{2}} \cdot 5\,\text{с} = -50\,\frac{\text{м}}{\text{с}}.
    \end{align*}
}
\solutionspace{120pt}

\tasknumber{3}%
\task{%
    Карусель радиусом $3\,\text{м}$ равномерно совершает 6 оборотов в минуту.
    Определите
    \begin{itemize}
        \item период и частоту её обращения,
        \item скорость и ускорение крайних её точек.
    \end{itemize}
}
\answer{%
    \begin{align*}
    t &= 60\,\text{с}, r = 3{,}0\,\text{м}, n = 6\units{оборотов}, \\
    T &= \frac tN = \frac{ 60\,\text{с} }{6} \approx 10{,}00\,\text{c}, \\
    \nu &= \frac 1T = \frac{6}{ 60\,\text{с} } \approx 0{,}10\,\text{Гц}, \\
    v &= \frac{2 \pi r}{T} = \frac{2 \pi r}{T} =  \frac{2 \pi r n}{t} \approx 1{,}88\,\frac{\text{м}}{\text{c}}, \\
    a &= \frac{v^2}{r} =  \frac{4 \pi^2 r n^2}{t^2} \approx 1{,}18\,\frac{\text{м}}{\text{с}^{2}}.
    \end{align*}
}
\solutionspace{80pt}

\tasknumber{4}%
\task{%
    Миша стоит на обрыве над рекой и методично и строго горизонтально кидает в неё камушки.
    За этим всем наблюдает экспериментатор Глюк, который уже выяснил, что камушки падают в реку спустя $1{,}2\,\text{с}$ после броска,
    а вот дальность полёта оценить сложнее: придётся лезть в воду.
    Выручите Глюка и определите:
    \begin{itemize}
        \item высоту обрыва (вместе с ростом Миши).
        \item дальность полёта камушков (по горизонтали) и их скорость при падении, приняв начальную скорость броска равной $v_0 = 12\,\frac{\text{м}}{\text{с}}$.
    \end{itemize}
    Сопротивлением воздуха пренебречь.
}
\answer{%
    \begin{align*}
    y &= y_0 + v_{0y}t - \frac{gt^2}2 = h - \frac{gt^2}2, \qquad y(\tau) = 0 \implies h - \frac{g\tau^2}2 = 0 \implies h = \frac{g\tau^2}2 \approx 7{,}2\,\text{м}.
    \\
    x &= x_0 + v_{0x}t = v_0t \implies L = v_0\tau \approx 14{,}4\,\text{м}.
    \\
    &v = \sqrt{v_x^2 + v_y^2} = \sqrt{v_{0x}^2 + \sqr{v_{0y} - g\tau}} = \sqrt{v_0^2 + \sqr{g\tau}} \approx 17{,}0\,\frac{\text{м}}{\text{c}}.
    \end{align*}
}
\solutionspace{120pt}

\tasknumber{5}%
\task{%
    Четыре одинаковых брусков массой $3\,\text{кг}$ каждый лежат на гладком горизонтальном столе.
    Бруски пронумерованы от 1 до 4 и последовательно связаны между собой
    невесомыми нерастяжимыми нитями: 1 со 2, 2 с 3 (ну и с 1) и т.д.
    Экспериментатор Глюк прикладывает постоянную горизонтальную силу $90\,\text{Н}$ к бруску с наименьшим номером.
    С каким ускорением двигается система? Чему равна сила натяжения нити, связывающей бруски 1 и 2?
}
\answer{%
    \begin{align*}
    a &= \frac{F}{4 m} = \frac{90\,\text{Н}}{4 \cdot 3\,\text{кг}} \approx 7{,}5\,\frac{\text{м}}{\text{c}^{2}}, \\
    T &= m'a = 3m \cdot \frac{F}{4 m} = \frac{3}{4} F \approx 67{,}5\,\text{Н}.
    \end{align*}
}
\solutionspace{120pt}

\tasknumber{6}%
\task{%
    Два бруска связаны лёгкой нерастяжимой нитью и перекинуты через неподвижный блок (см.
    рис.).
    Определите силу натяжения нити и ускорения брусков.
    Силами трения пренебречь, массы брусков
    равны $m_1 = 5\,\text{кг}$ и $m_2 = 6\,\text{кг}$.
    % $g = 10\,\frac{\text{м}}{\text{с}^{2}}$.

    \begin{tikzpicture}[x=1.5cm,y=1.5cm,thick]
        \draw
            (-0.4, 0) rectangle (-0.2, 1.2)
            (0.15, 0.5) rectangle (0.45, 1)
            (0, 2) circle [radius=0.3] -- ++(up:0.5)
            (-0.3, 1.2) -- ++(up:0.8)
            (0.3, 1) -- ++(up:1)
            (-0.7, 2.5) -- (0.7, 2.5)
            ;
        \draw[pattern={Lines[angle=51,distance=3pt]},pattern color=black,draw=none] (-0.7, 2.5) rectangle (0.7, 2.75);
        \node [left] (left) at (-0.4, 0.6) { $m_1$ };
        \node [right] (right) at (0.4, 0.75) { $m_2$ };
    \end{tikzpicture}
}
\answer{%
    Предположим, что левый брусок ускоряется вверх, тогда правый ускоряется вниз (с тем же ускорением).
    Запишем 2-й закон Ньютона 2 раза (для обоих тел) в проекции на вертикальную оси, направив её вверх.
    \begin{align*}
        &\begin{cases}
            T - m_1g = m_1a, \\
            T - m_2g = -m_2a,
        \end{cases} \\
        &\begin{cases}
            m_2g - m_1g = m_1a + m_2a, \\
            T = m_1a + m_1g, \\
        \end{cases} \\
        a &= \frac{m_2 - m_1}{m_1 + m_2} \cdot g = \frac{6\,\text{кг} - 5\,\text{кг}}{5\,\text{кг} + 6\,\text{кг}} \cdot 10\,\frac{\text{м}}{\text{с}^{2}} \approx 0{,}91\,\frac{\text{м}}{\text{c}^{2}}, \\
        T &= m_1(a + g) = m_1 \cdot g \cdot \cbr{\frac{m_2 - m_1}{m_1 + m_2} + 1} = m_1 \cdot g \cdot \frac{2m_2}{m_1 + m_2} = \\
            &= \frac{2 m_2 m_1 g}{m_1 + m_2} = \frac{2 \cdot 6\,\text{кг} \cdot 5\,\text{кг} \cdot 10\,\frac{\text{м}}{\text{с}^{2}}}{5\,\text{кг} + 6\,\text{кг}} \approx 54{,}5\,\text{Н}.
    \end{align*}
    Отрицательный ответ говорит, что мы лишь не угадали с направлением ускорений.
    Сила же всегда положительна.
}
\solutionspace{80pt}

\tasknumber{7}%
\task{%
    Тело массой $1{,}4\,\text{кг}$ лежит на горизонтальной поверхности.
    Коэффициент трения между поверхностью и телом $0{,}25$.
    К телу приложена горизонтальная сила $4{,}5\,\text{Н}$.
    Определите силу трения, действующую на тело, и ускорение тела.
    % $g = 10\,\frac{\text{м}}{\text{с}^{2}}$.
}
\answer{%
    \begin{align*}
    &F_\text{ трения покоя $\max$ } = \mu N = \mu m g = 0{,}25 \cdot 1{,}4\,\text{кг} \cdot 10\,\frac{\text{м}}{\text{с}^{2}} = 3{,}50\,\text{Н}, \\
    &F_\text{ трения покоя $\max$ } \le F \implies F_\text{ трения } = 3{,}50\,\text{Н}, a = \frac{F - F_\text{ трения }}{ m } = 0{,}71\,\frac{\text{м}}{\text{c}^{2}}, \\
    &\text{при равенстве возможны оба варианта: и едет, и не едет, но на ответы это не влияет.}
    \end{align*}
}
\solutionspace{120pt}

\tasknumber{8}%
\task{%
    Определите плотность неизвестного вещества, если известно, что опускании тела из него
    в подсолнечное масло оно будет плавать и на треть выступать над поверхностью жидкости.
}
\answer{%
    $F_\text{Арх.} = F_\text{тяж.} \implies \rho_\text{ж.} g V_\text{погр.} = m g \implies\rho_\text{ж.} g \cbr{V -\frac V3} = \rho V g \implies \rho = \rho_\text{ж.}\cbr{1 -\frac 13} \approx 600\,\frac{\text{кг}}{\text{м}^{3}}$
}
\solutionspace{120pt}

\tasknumber{9}%
\task{%
    	Определите силу, действующую на правую опору однородного горизонтального стержня длиной $l = 9\,\text{м}$
    	и массой $M = 1\,\text{кг}$, к которому подвешен груз массой $m = 3\,\text{кг}$ на расстоянии $4\,\text{м}$ от правого конца (см.
    рис.).

        \begin{tikzpicture}[thick]
            \draw
                (-2, -0.1) rectangle (2, 0.1)
                (-0.5, -0.1) -- (-0.5, -1)
                (-0.7, -1) rectangle (-0.3, -1.3)
           		(-2, -0.1) -- +(0.15,-0.9) -- +(-0.15,-0.9) -- cycle
            	(2, -0.1) -- +(0.15,-0.9) -- +(-0.15,-0.9) -- cycle
            ;
            \draw[pattern={Lines[angle=51,distance=2pt]},pattern color=black,draw=none]
            	(-2.15, -1.15) rectangle +(0.3, 0.15)
            	(2.15, -1.15) rectangle +(-0.3, 0.15)
            ;
            \node [right] (m_small) at (-0.3, -1.15) { $m$ };
            \node [above] (M_big) at (0, 0.1) { $M$ };
        \end{tikzpicture}
}
\answer{%
    \begin{align*}
        &\begin{cases}
            F_1 + F_2 - mg - Mg= 0, \\
            F_1 \cdot 0 - mg \cdot a - Mg \cdot \frac l2 + F_2 \cdot l = 0,
        \end{cases} \\
        F_2 &= \frac{mga + Mg\frac l2}l = \frac al \cdot mg + \frac{Mg}2 \approx 21{,}7\,\text{Н}, \\
        F_1 &= mg + Mg - F_2 = mg + Mg - \frac al \cdot mg - \frac{Mg}2 = \frac bl \cdot mg + \frac{Mg}2 \approx 18{,}3\,\text{Н}.
    \end{align*}
}
\solutionspace{80pt}

\tasknumber{10}%
\task{%
    Тонкий однородный лом длиной $3\,\text{м}$ и массой $10\,\text{кг}$ лежит на горизонтальной поверхности.
    \begin{itemize}
        \item Какую минимальную силу надо приложить к одному из его концов, чтобы оторвать его от этой поверхности?
        \item Какую минимальную работу надо совершить, чтобы поставить его на землю в вертикальное положение?
    \end{itemize}
    % Примите $g = 10\,\frac{\text{м}}{\text{с}^{2}}$.
}
\answer{%
    $F = \frac{mg}2 \approx 100\,\text{Н}, A = mg\frac l2 = 150\,\text{Дж}$
}
\solutionspace{120pt}

\tasknumber{11}%
\task{%
    Определите работу силы, которая обеспечит подъём тела массой $5\,\text{кг}$ на высоту $10\,\text{м}$ с постоянным ускорением $2\,\frac{\text{м}}{\text{c}^{2}}$.
    % Примите $g = 10\,\frac{\text{м}}{\text{с}^{2}}$.
}
\answer{%
    \begin{align*}
    &\text{Для подъёма:} A = Fh = (mg + ma) h = m(g+a)h, \\
    &\text{Для спуска:} A = -Fh = -(mg - ma) h = -m(g-a)h, \\
    &\text{В результате получаем:} 600\,\text{Дж}.
    \end{align*}
}
\solutionspace{60pt}

\tasknumber{12}%
\task{%
    Тело бросили вертикально вверх со скоростью $10\,\frac{\text{м}}{\text{c}}$.
    На какой высоте кинетическая энергия тела составит треть от потенциальной?
}
\answer{%
    \begin{align*}
    &0 + \frac{mv_0^2}2 = E_p + E_k, E_k = \frac 13 E_p \implies \\
    &\implies \frac{mv_0^2}2 = E_p + \frac 13 E_p = E_p\cbr{1 + \frac 13} = mgh\cbr{1 + \frac 13} \implies \\
    &\implies h = \frac{\frac{mv_0^2}2}{mg\cbr{1 + \frac 13}} = \frac{v_0^2}{2g} \cdot \frac 1{1 + \frac 13} \approx 3{,}8\,\text{м}.
    \end{align*}
}
\solutionspace{100pt}

\tasknumber{13}%
\task{%
    Плотность воздуха при нормальных условиях равна $1{,}3\,\frac{\text{кг}}{\text{м}^{3}}$.
    Чему равна плотность воздуха
    при температуре $150\celsius$ и давлении $120\,\text{кПа}$?
}
\answer{%
    \begin{align*}
    &\text{В общем случае:} PV = \frac m{\mu} RT \implies \rho = \frac mV = \frac m{\frac{\frac m{\mu} RT}P} = \frac{P\mu}{RT}, \\
    &\text{У нас 2 состояния:} \rho_1 = \frac{P_1\mu}{RT_1}, \rho_2 = \frac{P_2\mu}{RT_2} \implies \frac{\rho_2}{\rho_1} = \frac{\frac{P_2\mu}{RT_2}}{\frac{P_1\mu}{RT_1}} = \frac{P_2T_1}{P_1T_2} \implies \\
    &\implies \rho_2 = \rho_1 \cdot  \frac{P_2T_1}{P_1T_2} = 1{,}3\,\frac{\text{кг}}{\text{м}^{3}} \cdot \frac{120\,\text{кПа} \cdot 273\units{К}}{100\,\text{кПа} \cdot 423\units{К}} \approx 1{,}01\,\frac{\text{кг}}{\text{м}^{3}}.
    \end{align*}
}
\solutionspace{120pt}

\tasknumber{14}%
\task{%
    Небольшую цилиндрическую пробирку с воздухом погружают на некоторую глубину в глубокое пресное озеро,
    после чего воздух занимает в ней лишь четвертую часть от общего объема.
    Определите глубину, на которую погрузили пробирку.
    Температуру считать постоянной $T = 289\,\text{К}$, давлением паров воды пренебречь,
    атмосферное давление принять равным $p_{\text{aтм}} = 100\,\text{кПа}$.
}
\answer{%
    \begin{align*}
    T\text{— const} &\implies P_1V_1 = \nu RT = P_2V_2.
    \\
    V_2 = \frac 14 V_1 &\implies P_1V_1 = P_2 \cdot \frac 14V_1 \implies P_2 = 4P_1 = 4p_{\text{aтм}}.
    \\
    P_2 = p_{\text{aтм}} + \rho_{\text{в}} g h \implies h = \frac{P_2 - p_{\text{aтм}}}{\rho_{\text{в}} g} &= \frac{4p_{\text{aтм}} - p_{\text{aтм}}}{\rho_{\text{в}} g} = \frac{3 \cdot p_{\text{aтм}}}{\rho_{\text{в}} g} =  \\
     &= \frac{3 \cdot 100\,\text{кПа}}{1000\,\frac{\text{кг}}{\text{м}^{3}} \cdot  10\,\frac{\text{м}}{\text{с}^{2}}} \approx 30\,\text{м}.
    \end{align*}
}
\solutionspace{120pt}

\tasknumber{15}%
\task{%
    Газу сообщили некоторое количество теплоты,
    при этом четверть его он потратил на совершение работы,
    одновременно увеличив свою внутреннюю энергию на $1500\,\text{Дж}$.
    Определите количество теплоты, сообщённое газу.
}
\answer{%
    \begin{align*}
    Q &= A' + \Delta U, A' = \frac 14 Q \implies Q \cdot \cbr{1 - \frac 14} = \Delta U \implies Q = \frac{\Delta U}{1 - \frac 14} = \frac{ 1500\,\text{Дж} }{1 - \frac 14} \approx 2000\,\text{Дж}.
    \\
    A' &= \frac 14 Q
        = \frac 14 \cdot \frac{\Delta U}{1 - \frac 14}
        = \frac{\Delta U}{4 - 1}
        = \frac{ 1500\,\text{Дж} }{4 - 1} \approx 500\,\text{Дж}.
    \end{align*}
}
\solutionspace{60pt}

\tasknumber{16}%
\task{%
    Два конденсатора ёмкостей $C_1 = 20\,\text{нФ}$ и $C_2 = 30\,\text{нФ}$ последовательно подключают
    к источнику напряжения $U = 200\,\text{В}$ (см.
    рис.).
    % Определите заряды каждого из конденсаторов.
    Определите заряд второго конденсатора.

    \begin{tikzpicture}[circuit ee IEC, semithick]
        \draw  (0, 0) to [capacitor={info={$C_1$}}] (1, 0)
                       to [capacitor={info={$C_2$}}] (2, 0)
        ;
        % \draw [-o] (0, 0) -- ++(-0.5, 0) node[left] {$-$};
        % \draw [-o] (2, 0) -- ++(0.5, 0) node[right] {$+$};
        \draw [-o] (0, 0) -- ++(-0.5, 0) node[left] {};
        \draw [-o] (2, 0) -- ++(0.5, 0) node[right] {};
    \end{tikzpicture}
}
\answer{%
    $
        Q_1
            = Q_2
            = CU
            = \frac{ U }{\frac1{C_1} + \frac1{C_2}}
            = \frac{C_1C_2U}{C_1 + C_2}
            = \frac{
                20\,\text{нФ} \cdot 30\,\text{нФ} \cdot 200\,\text{В}
            }{
                20\,\text{нФ} + 30\,\text{нФ}
            }
            = 2{,}40\,\text{мкКл}
    $
}
\solutionspace{120pt}

\tasknumber{17}%
\task{%
    В вакууме вдоль одной прямой расположены четыре положительных заряда так,
    что расстояние между соседними зарядами равно $d$.
    Сделайте рисунок,
    и определите силу, действующую на крайний заряд.
    Модули всех зарядов равны $Q$ ($Q > 0$).
}
\answer{%
    $F = \sum_i F_i = \ldots = \frac{49}{36} \frac{kQ^2}{d^2}.$
}
\solutionspace{80pt}

\tasknumber{18}%
\task{%
    Юлия проводит эксперименты c 2 кусками одинаковой стальной проволки, причём второй кусок в девять раз длиннее первого.
    В одном из экспериментов Юлия подаёт на первый кусок проволки напряжение в четыре раза раз больше, чем на второй.
    Определите отношения в двух проволках в этом эксперименте (второй к первой):
    \begin{itemize}
        \item отношение сил тока,
        \item отношение выделяющихся мощностей.
    \end{itemize}
}
\answer{%
    $R_2 = 9R_1, U_1 = 4U_2 \implies  \eli_2 / \eli_1 = \frac{U_2 / R_2}{U_1 / R_1} = \frac{U_2}{U_1} \cdot \frac{R_1}{R_2} = \frac1{36}, P_2 / P_1 = \frac{U_2^2 / R_2}{U_1^2 / R_1} = \sqr{\frac{U_2}{U_1}} \cdot \frac{R_1}{R_2} = \frac1{144}.$
}

\variantsplitter

\addpersonalvariant{Герман Говоров}

\tasknumber{1}%
\task{%
    Женя стартует на лошади и в течение $t = 3\,\text{c}$ двигается с постоянным ускорением $1{,}5\,\frac{\text{м}}{\text{с}^{2}}$.
    Определите
    \begin{itemize}
        \item какую скорость при этом удастся достичь,
        \item какой путь за это время будет пройден,
        \item среднюю скорость за всё время движения, если после начального ускорения продолжить движение равномерно ещё в течение времени $2t$
    \end{itemize}
}
\answer{%
    \begin{align*}
    v &= v_0 + a t = at = 1{,}5\,\frac{\text{м}}{\text{с}^{2}} \cdot 3\,\text{c} = 4{,}5\,\frac{\text{м}}{\text{с}}, \\
    s_x &= v_0t + \frac{a t^2}2 = \frac{a t^2}2 = \frac{1{,}5\,\frac{\text{м}}{\text{с}^{2}} \cdot \sqr{ 3\,\text{c} }}2 = 6{,}8\,\text{м}, \\
    v_\text{сред.} &= \frac{s_\text{общ}}{t_\text{общ.}} = \frac{s_x + v \cdot 2t}{t + 2t} = \frac{\frac{a t^2}2 + at \cdot 2t}{t (1 + 2)} = \\
    &= at \cdot \frac{\frac 12 + 2}{1 + 2} = 1{,}5\,\frac{\text{м}}{\text{с}^{2}} \cdot 3\,\text{c} \cdot \frac{\frac 12 + 2}{1 + 2} \approx 3{,}75\,\frac{\text{м}}{\text{c}}.
    \end{align*}
}
\solutionspace{120pt}

\tasknumber{2}%
\task{%
    Какой путь тело пройдёт за шестую секунду после начала свободного падения?
    Какую скорость в начале этой секунды оно имеет?
}
\answer{%
    \begin{align*}
    s &= -s_y = -(y_2-y_1) = y_1 - y_2 = \cbr{y_{0y} + v_{0y}t_1 - \frac{gt_1^2}2} - \cbr{y_{0y} + v_{0y}t_2 - \frac{gt_2^2}2} = \\
    &= \frac{gt_2^2}2 - \frac{gt_1^2}2 = \frac g2\cbr{t_2^2 - t_1^2} = 55{,}0\,\text{м}, \\
    v_y &= v_{0y} - gt = -gt = 10\,\frac{\text{м}}{\text{с}^{2}} \cdot 5\,\text{с} = -50\,\frac{\text{м}}{\text{с}}.
    \end{align*}
}
\solutionspace{120pt}

\tasknumber{3}%
\task{%
    Карусель диаметром $3\,\text{м}$ равномерно совершает 5 оборотов в минуту.
    Определите
    \begin{itemize}
        \item период и частоту её обращения,
        \item скорость и ускорение крайних её точек.
    \end{itemize}
}
\answer{%
    \begin{align*}
    t &= 60\,\text{с}, r = 1{,}5\,\text{м}, n = 5\units{оборотов}, \\
    T &= \frac tN = \frac{ 60\,\text{с} }{5} \approx 12{,}00\,\text{c}, \\
    \nu &= \frac 1T = \frac{5}{ 60\,\text{с} } \approx 0{,}08\,\text{Гц}, \\
    v &= \frac{2 \pi r}{T} = \frac{2 \pi r}{T} =  \frac{2 \pi r n}{t} \approx 0{,}79\,\frac{\text{м}}{\text{c}}, \\
    a &= \frac{v^2}{r} =  \frac{4 \pi^2 r n^2}{t^2} \approx 0{,}41\,\frac{\text{м}}{\text{с}^{2}}.
    \end{align*}
}
\solutionspace{80pt}

\tasknumber{4}%
\task{%
    Маша стоит на обрыве над рекой и методично и строго горизонтально кидает в неё камушки.
    За этим всем наблюдает экспериментатор Глюк, который уже выяснил, что камушки падают в реку спустя $1{,}7\,\text{с}$ после броска,
    а вот дальность полёта оценить сложнее: придётся лезть в воду.
    Выручите Глюка и определите:
    \begin{itemize}
        \item высоту обрыва (вместе с ростом Маши).
        \item дальность полёта камушков (по горизонтали) и их скорость при падении, приняв начальную скорость броска равной $v_0 = 14\,\frac{\text{м}}{\text{с}}$.
    \end{itemize}
    Сопротивлением воздуха пренебречь.
}
\answer{%
    \begin{align*}
    y &= y_0 + v_{0y}t - \frac{gt^2}2 = h - \frac{gt^2}2, \qquad y(\tau) = 0 \implies h - \frac{g\tau^2}2 = 0 \implies h = \frac{g\tau^2}2 \approx 14{,}4\,\text{м}.
    \\
    x &= x_0 + v_{0x}t = v_0t \implies L = v_0\tau \approx 23{,}8\,\text{м}.
    \\
    &v = \sqrt{v_x^2 + v_y^2} = \sqrt{v_{0x}^2 + \sqr{v_{0y} - g\tau}} = \sqrt{v_0^2 + \sqr{g\tau}} \approx 22{,}0\,\frac{\text{м}}{\text{c}}.
    \end{align*}
}
\solutionspace{120pt}

\tasknumber{5}%
\task{%
    Четыре одинаковых брусков массой $2\,\text{кг}$ каждый лежат на гладком горизонтальном столе.
    Бруски пронумерованы от 1 до 4 и последовательно связаны между собой
    невесомыми нерастяжимыми нитями: 1 со 2, 2 с 3 (ну и с 1) и т.д.
    Экспериментатор Глюк прикладывает постоянную горизонтальную силу $90\,\text{Н}$ к бруску с наибольшим номером.
    С каким ускорением двигается система? Чему равна сила натяжения нити, связывающей бруски 3 и 4?
}
\answer{%
    \begin{align*}
    a &= \frac{F}{4 m} = \frac{90\,\text{Н}}{4 \cdot 2\,\text{кг}} \approx 11{,}2\,\frac{\text{м}}{\text{c}^{2}}, \\
    T &= m'a = 3m \cdot \frac{F}{4 m} = \frac{3}{4} F \approx 67{,}5\,\text{Н}.
    \end{align*}
}
\solutionspace{120pt}

\tasknumber{6}%
\task{%
    Два бруска связаны лёгкой нерастяжимой нитью и перекинуты через неподвижный блок (см.
    рис.).
    Определите силу натяжения нити и ускорения брусков.
    Силами трения пренебречь, массы брусков
    равны $m_1 = 5\,\text{кг}$ и $m_2 = 6\,\text{кг}$.
    % $g = 10\,\frac{\text{м}}{\text{с}^{2}}$.

    \begin{tikzpicture}[x=1.5cm,y=1.5cm,thick]
        \draw
            (-0.4, 0) rectangle (-0.2, 1.2)
            (0.15, 0.5) rectangle (0.45, 1)
            (0, 2) circle [radius=0.3] -- ++(up:0.5)
            (-0.3, 1.2) -- ++(up:0.8)
            (0.3, 1) -- ++(up:1)
            (-0.7, 2.5) -- (0.7, 2.5)
            ;
        \draw[pattern={Lines[angle=51,distance=3pt]},pattern color=black,draw=none] (-0.7, 2.5) rectangle (0.7, 2.75);
        \node [left] (left) at (-0.4, 0.6) { $m_1$ };
        \node [right] (right) at (0.4, 0.75) { $m_2$ };
    \end{tikzpicture}
}
\answer{%
    Предположим, что левый брусок ускоряется вверх, тогда правый ускоряется вниз (с тем же ускорением).
    Запишем 2-й закон Ньютона 2 раза (для обоих тел) в проекции на вертикальную оси, направив её вверх.
    \begin{align*}
        &\begin{cases}
            T - m_1g = m_1a, \\
            T - m_2g = -m_2a,
        \end{cases} \\
        &\begin{cases}
            m_2g - m_1g = m_1a + m_2a, \\
            T = m_1a + m_1g, \\
        \end{cases} \\
        a &= \frac{m_2 - m_1}{m_1 + m_2} \cdot g = \frac{6\,\text{кг} - 5\,\text{кг}}{5\,\text{кг} + 6\,\text{кг}} \cdot 10\,\frac{\text{м}}{\text{с}^{2}} \approx 0{,}91\,\frac{\text{м}}{\text{c}^{2}}, \\
        T &= m_1(a + g) = m_1 \cdot g \cdot \cbr{\frac{m_2 - m_1}{m_1 + m_2} + 1} = m_1 \cdot g \cdot \frac{2m_2}{m_1 + m_2} = \\
            &= \frac{2 m_2 m_1 g}{m_1 + m_2} = \frac{2 \cdot 6\,\text{кг} \cdot 5\,\text{кг} \cdot 10\,\frac{\text{м}}{\text{с}^{2}}}{5\,\text{кг} + 6\,\text{кг}} \approx 54{,}5\,\text{Н}.
    \end{align*}
    Отрицательный ответ говорит, что мы лишь не угадали с направлением ускорений.
    Сила же всегда положительна.
}
\solutionspace{80pt}

\tasknumber{7}%
\task{%
    Тело массой $2{,}7\,\text{кг}$ лежит на горизонтальной поверхности.
    Коэффициент трения между поверхностью и телом $0{,}15$.
    К телу приложена горизонтальная сила $5{,}5\,\text{Н}$.
    Определите силу трения, действующую на тело, и ускорение тела.
    % $g = 10\,\frac{\text{м}}{\text{с}^{2}}$.
}
\answer{%
    \begin{align*}
    &F_\text{ трения покоя $\max$ } = \mu N = \mu m g = 0{,}15 \cdot 2{,}7\,\text{кг} \cdot 10\,\frac{\text{м}}{\text{с}^{2}} = 4{,}05\,\text{Н}, \\
    &F_\text{ трения покоя $\max$ } \le F \implies F_\text{ трения } = 4{,}05\,\text{Н}, a = \frac{F - F_\text{ трения }}{ m } = 0{,}54\,\frac{\text{м}}{\text{c}^{2}}, \\
    &\text{при равенстве возможны оба варианта: и едет, и не едет, но на ответы это не влияет.}
    \end{align*}
}
\solutionspace{120pt}

\tasknumber{8}%
\task{%
    Определите плотность неизвестного вещества, если известно, что опускании тела из него
    в подсолнечное масло оно будет плавать и на треть выступать над поверхностью жидкости.
}
\answer{%
    $F_\text{Арх.} = F_\text{тяж.} \implies \rho_\text{ж.} g V_\text{погр.} = m g \implies\rho_\text{ж.} g \cbr{V -\frac V3} = \rho V g \implies \rho = \rho_\text{ж.}\cbr{1 -\frac 13} \approx 600\,\frac{\text{кг}}{\text{м}^{3}}$
}
\solutionspace{120pt}

\tasknumber{9}%
\task{%
    	Определите силу, действующую на правую опору однородного горизонтального стержня длиной $l = 9\,\text{м}$
    	и массой $M = 1\,\text{кг}$, к которому подвешен груз массой $m = 4\,\text{кг}$ на расстоянии $4\,\text{м}$ от правого конца (см.
    рис.).

        \begin{tikzpicture}[thick]
            \draw
                (-2, -0.1) rectangle (2, 0.1)
                (-0.5, -0.1) -- (-0.5, -1)
                (-0.7, -1) rectangle (-0.3, -1.3)
           		(-2, -0.1) -- +(0.15,-0.9) -- +(-0.15,-0.9) -- cycle
            	(2, -0.1) -- +(0.15,-0.9) -- +(-0.15,-0.9) -- cycle
            ;
            \draw[pattern={Lines[angle=51,distance=2pt]},pattern color=black,draw=none]
            	(-2.15, -1.15) rectangle +(0.3, 0.15)
            	(2.15, -1.15) rectangle +(-0.3, 0.15)
            ;
            \node [right] (m_small) at (-0.3, -1.15) { $m$ };
            \node [above] (M_big) at (0, 0.1) { $M$ };
        \end{tikzpicture}
}
\answer{%
    \begin{align*}
        &\begin{cases}
            F_1 + F_2 - mg - Mg= 0, \\
            F_1 \cdot 0 - mg \cdot a - Mg \cdot \frac l2 + F_2 \cdot l = 0,
        \end{cases} \\
        F_2 &= \frac{mga + Mg\frac l2}l = \frac al \cdot mg + \frac{Mg}2 \approx 27{,}2\,\text{Н}, \\
        F_1 &= mg + Mg - F_2 = mg + Mg - \frac al \cdot mg - \frac{Mg}2 = \frac bl \cdot mg + \frac{Mg}2 \approx 22{,}8\,\text{Н}.
    \end{align*}
}
\solutionspace{80pt}

\tasknumber{10}%
\task{%
    Тонкий однородный шест длиной $2\,\text{м}$ и массой $30\,\text{кг}$ лежит на горизонтальной поверхности.
    \begin{itemize}
        \item Какую минимальную силу надо приложить к одному из его концов, чтобы оторвать его от этой поверхности?
        \item Какую минимальную работу надо совершить, чтобы поставить его на землю в вертикальное положение?
    \end{itemize}
    % Примите $g = 10\,\frac{\text{м}}{\text{с}^{2}}$.
}
\answer{%
    $F = \frac{mg}2 \approx 300\,\text{Н}, A = mg\frac l2 = 300\,\text{Дж}$
}
\solutionspace{120pt}

\tasknumber{11}%
\task{%
    Определите работу силы, которая обеспечит подъём тела массой $5\,\text{кг}$ на высоту $10\,\text{м}$ с постоянным ускорением $4\,\frac{\text{м}}{\text{c}^{2}}$.
    % Примите $g = 10\,\frac{\text{м}}{\text{с}^{2}}$.
}
\answer{%
    \begin{align*}
    &\text{Для подъёма:} A = Fh = (mg + ma) h = m(g+a)h, \\
    &\text{Для спуска:} A = -Fh = -(mg - ma) h = -m(g-a)h, \\
    &\text{В результате получаем:} 700\,\text{Дж}.
    \end{align*}
}
\solutionspace{60pt}

\tasknumber{12}%
\task{%
    Тело бросили вертикально вверх со скоростью $20\,\frac{\text{м}}{\text{c}}$.
    На какой высоте кинетическая энергия тела составит половину от потенциальной?
}
\answer{%
    \begin{align*}
    &0 + \frac{mv_0^2}2 = E_p + E_k, E_k = \frac 12 E_p \implies \\
    &\implies \frac{mv_0^2}2 = E_p + \frac 12 E_p = E_p\cbr{1 + \frac 12} = mgh\cbr{1 + \frac 12} \implies \\
    &\implies h = \frac{\frac{mv_0^2}2}{mg\cbr{1 + \frac 12}} = \frac{v_0^2}{2g} \cdot \frac 1{1 + \frac 12} \approx 13{,}3\,\text{м}.
    \end{align*}
}
\solutionspace{100pt}

\tasknumber{13}%
\task{%
    Плотность воздуха при нормальных условиях равна $1{,}3\,\frac{\text{кг}}{\text{м}^{3}}$.
    Чему равна плотность воздуха
    при температуре $150\celsius$ и давлении $120\,\text{кПа}$?
}
\answer{%
    \begin{align*}
    &\text{В общем случае:} PV = \frac m{\mu} RT \implies \rho = \frac mV = \frac m{\frac{\frac m{\mu} RT}P} = \frac{P\mu}{RT}, \\
    &\text{У нас 2 состояния:} \rho_1 = \frac{P_1\mu}{RT_1}, \rho_2 = \frac{P_2\mu}{RT_2} \implies \frac{\rho_2}{\rho_1} = \frac{\frac{P_2\mu}{RT_2}}{\frac{P_1\mu}{RT_1}} = \frac{P_2T_1}{P_1T_2} \implies \\
    &\implies \rho_2 = \rho_1 \cdot  \frac{P_2T_1}{P_1T_2} = 1{,}3\,\frac{\text{кг}}{\text{м}^{3}} \cdot \frac{120\,\text{кПа} \cdot 273\units{К}}{100\,\text{кПа} \cdot 423\units{К}} \approx 1{,}01\,\frac{\text{кг}}{\text{м}^{3}}.
    \end{align*}
}
\solutionspace{120pt}

\tasknumber{14}%
\task{%
    Небольшую цилиндрическую пробирку с воздухом погружают на некоторую глубину в глубокое пресное озеро,
    после чего воздух занимает в ней лишь третью часть от общего объема.
    Определите глубину, на которую погрузили пробирку.
    Температуру считать постоянной $T = 280\,\text{К}$, давлением паров воды пренебречь,
    атмосферное давление принять равным $p_{\text{aтм}} = 100\,\text{кПа}$.
}
\answer{%
    \begin{align*}
    T\text{— const} &\implies P_1V_1 = \nu RT = P_2V_2.
    \\
    V_2 = \frac 13 V_1 &\implies P_1V_1 = P_2 \cdot \frac 13V_1 \implies P_2 = 3P_1 = 3p_{\text{aтм}}.
    \\
    P_2 = p_{\text{aтм}} + \rho_{\text{в}} g h \implies h = \frac{P_2 - p_{\text{aтм}}}{\rho_{\text{в}} g} &= \frac{3p_{\text{aтм}} - p_{\text{aтм}}}{\rho_{\text{в}} g} = \frac{2 \cdot p_{\text{aтм}}}{\rho_{\text{в}} g} =  \\
     &= \frac{2 \cdot 100\,\text{кПа}}{1000\,\frac{\text{кг}}{\text{м}^{3}} \cdot  10\,\frac{\text{м}}{\text{с}^{2}}} \approx 20\,\text{м}.
    \end{align*}
}
\solutionspace{120pt}

\tasknumber{15}%
\task{%
    Газу сообщили некоторое количество теплоты,
    при этом четверть его он потратил на совершение работы,
    одновременно увеличив свою внутреннюю энергию на $3000\,\text{Дж}$.
    Определите работу, совершённую газом.
}
\answer{%
    \begin{align*}
    Q &= A' + \Delta U, A' = \frac 14 Q \implies Q \cdot \cbr{1 - \frac 14} = \Delta U \implies Q = \frac{\Delta U}{1 - \frac 14} = \frac{ 3000\,\text{Дж} }{1 - \frac 14} \approx 4000\,\text{Дж}.
    \\
    A' &= \frac 14 Q
        = \frac 14 \cdot \frac{\Delta U}{1 - \frac 14}
        = \frac{\Delta U}{4 - 1}
        = \frac{ 3000\,\text{Дж} }{4 - 1} \approx 1000\,\text{Дж}.
    \end{align*}
}
\solutionspace{60pt}

\tasknumber{16}%
\task{%
    Два конденсатора ёмкостей $C_1 = 20\,\text{нФ}$ и $C_2 = 60\,\text{нФ}$ последовательно подключают
    к источнику напряжения $V = 200\,\text{В}$ (см.
    рис.).
    % Определите заряды каждого из конденсаторов.
    Определите заряд первого конденсатора.

    \begin{tikzpicture}[circuit ee IEC, semithick]
        \draw  (0, 0) to [capacitor={info={$C_1$}}] (1, 0)
                       to [capacitor={info={$C_2$}}] (2, 0)
        ;
        % \draw [-o] (0, 0) -- ++(-0.5, 0) node[left] {$-$};
        % \draw [-o] (2, 0) -- ++(0.5, 0) node[right] {$+$};
        \draw [-o] (0, 0) -- ++(-0.5, 0) node[left] {};
        \draw [-o] (2, 0) -- ++(0.5, 0) node[right] {};
    \end{tikzpicture}
}
\answer{%
    $
        Q_1
            = Q_2
            = CV
            = \frac{ V }{\frac1{C_1} + \frac1{C_2}}
            = \frac{C_1C_2V}{C_1 + C_2}
            = \frac{
                20\,\text{нФ} \cdot 60\,\text{нФ} \cdot 200\,\text{В}
            }{
                20\,\text{нФ} + 60\,\text{нФ}
            }
            = 3{,}00\,\text{мкКл}
    $
}
\solutionspace{120pt}

\tasknumber{17}%
\task{%
    В вакууме вдоль одной прямой расположены три положительных заряда так,
    что расстояние между соседними зарядами равно $l$.
    Сделайте рисунок,
    и определите силу, действующую на крайний заряд.
    Модули всех зарядов равны $q$ ($q > 0$).
}
\answer{%
    $F = \sum_i F_i = \ldots = \frac54 \frac{kq^2}{l^2}.$
}
\solutionspace{80pt}

\tasknumber{18}%
\task{%
    Юлия проводит эксперименты c 2 кусками одинаковой стальной проволки, причём второй кусок в три раза длиннее первого.
    В одном из экспериментов Юлия подаёт на первый кусок проволки напряжение в два раза раз больше, чем на второй.
    Определите отношения в двух проволках в этом эксперименте (второй к первой):
    \begin{itemize}
        \item отношение сил тока,
        \item отношение выделяющихся мощностей.
    \end{itemize}
}
\answer{%
    $R_2 = 3R_1, U_1 = 2U_2 \implies  \eli_2 / \eli_1 = \frac{U_2 / R_2}{U_1 / R_1} = \frac{U_2}{U_1} \cdot \frac{R_1}{R_2} = \frac16, P_2 / P_1 = \frac{U_2^2 / R_2}{U_1^2 / R_1} = \sqr{\frac{U_2}{U_1}} \cdot \frac{R_1}{R_2} = \frac1{12}.$
}

\variantsplitter

\addpersonalvariant{София Журавлёва}

\tasknumber{1}%
\task{%
    Саша стартует на лошади и в течение $t = 5\,\text{c}$ двигается с постоянным ускорением $2{,}5\,\frac{\text{м}}{\text{с}^{2}}$.
    Определите
    \begin{itemize}
        \item какую скорость при этом удастся достичь,
        \item какой путь за это время будет пройден,
        \item среднюю скорость за всё время движения, если после начального ускорения продолжить движение равномерно ещё в течение времени $3t$
    \end{itemize}
}
\answer{%
    \begin{align*}
    v &= v_0 + a t = at = 2{,}5\,\frac{\text{м}}{\text{с}^{2}} \cdot 5\,\text{c} = 12{,}5\,\frac{\text{м}}{\text{с}}, \\
    s_x &= v_0t + \frac{a t^2}2 = \frac{a t^2}2 = \frac{2{,}5\,\frac{\text{м}}{\text{с}^{2}} \cdot \sqr{ 5\,\text{c} }}2 = 31{,}2\,\text{м}, \\
    v_\text{сред.} &= \frac{s_\text{общ}}{t_\text{общ.}} = \frac{s_x + v \cdot 3t}{t + 3t} = \frac{\frac{a t^2}2 + at \cdot 3t}{t (1 + 3)} = \\
    &= at \cdot \frac{\frac 12 + 3}{1 + 3} = 2{,}5\,\frac{\text{м}}{\text{с}^{2}} \cdot 5\,\text{c} \cdot \frac{\frac 12 + 3}{1 + 3} \approx 10{,}94\,\frac{\text{м}}{\text{c}}.
    \end{align*}
}
\solutionspace{120pt}

\tasknumber{2}%
\task{%
    Какой путь тело пройдёт за третью секунду после начала свободного падения?
    Какую скорость в конце этой секунды оно имеет?
}
\answer{%
    \begin{align*}
    s &= -s_y = -(y_2-y_1) = y_1 - y_2 = \cbr{y_{0y} + v_{0y}t_1 - \frac{gt_1^2}2} - \cbr{y_{0y} + v_{0y}t_2 - \frac{gt_2^2}2} = \\
    &= \frac{gt_2^2}2 - \frac{gt_1^2}2 = \frac g2\cbr{t_2^2 - t_1^2} = 25{,}0\,\text{м}, \\
    v_y &= v_{0y} - gt = -gt = 10\,\frac{\text{м}}{\text{с}^{2}} \cdot 3\,\text{с} = -30\,\frac{\text{м}}{\text{с}}.
    \end{align*}
}
\solutionspace{120pt}

\tasknumber{3}%
\task{%
    Карусель диаметром $2\,\text{м}$ равномерно совершает 10 оборотов в минуту.
    Определите
    \begin{itemize}
        \item период и частоту её обращения,
        \item скорость и ускорение крайних её точек.
    \end{itemize}
}
\answer{%
    \begin{align*}
    t &= 60\,\text{с}, r = 1{,}0\,\text{м}, n = 10\units{оборотов}, \\
    T &= \frac tN = \frac{ 60\,\text{с} }{10} \approx 6{,}00\,\text{c}, \\
    \nu &= \frac 1T = \frac{10}{ 60\,\text{с} } \approx 0{,}17\,\text{Гц}, \\
    v &= \frac{2 \pi r}{T} = \frac{2 \pi r}{T} =  \frac{2 \pi r n}{t} \approx 1{,}05\,\frac{\text{м}}{\text{c}}, \\
    a &= \frac{v^2}{r} =  \frac{4 \pi^2 r n^2}{t^2} \approx 1{,}10\,\frac{\text{м}}{\text{с}^{2}}.
    \end{align*}
}
\solutionspace{80pt}

\tasknumber{4}%
\task{%
    Паша стоит на обрыве над рекой и методично и строго горизонтально кидает в неё камушки.
    За этим всем наблюдает экспериментатор Глюк, который уже выяснил, что камушки падают в реку спустя $1{,}4\,\text{с}$ после броска,
    а вот дальность полёта оценить сложнее: придётся лезть в воду.
    Выручите Глюка и определите:
    \begin{itemize}
        \item высоту обрыва (вместе с ростом Паши).
        \item дальность полёта камушков (по горизонтали) и их скорость при падении, приняв начальную скорость броска равной $v_0 = 16\,\frac{\text{м}}{\text{с}}$.
    \end{itemize}
    Сопротивлением воздуха пренебречь.
}
\answer{%
    \begin{align*}
    y &= y_0 + v_{0y}t - \frac{gt^2}2 = h - \frac{gt^2}2, \qquad y(\tau) = 0 \implies h - \frac{g\tau^2}2 = 0 \implies h = \frac{g\tau^2}2 \approx 9{,}8\,\text{м}.
    \\
    x &= x_0 + v_{0x}t = v_0t \implies L = v_0\tau \approx 22{,}4\,\text{м}.
    \\
    &v = \sqrt{v_x^2 + v_y^2} = \sqrt{v_{0x}^2 + \sqr{v_{0y} - g\tau}} = \sqrt{v_0^2 + \sqr{g\tau}} \approx 21{,}3\,\frac{\text{м}}{\text{c}}.
    \end{align*}
}
\solutionspace{120pt}

\tasknumber{5}%
\task{%
    Шесть одинаковых брусков массой $2\,\text{кг}$ каждый лежат на гладком горизонтальном столе.
    Бруски пронумерованы от 1 до 6 и последовательно связаны между собой
    невесомыми нерастяжимыми нитями: 1 со 2, 2 с 3 (ну и с 1) и т.д.
    Экспериментатор Глюк прикладывает постоянную горизонтальную силу $60\,\text{Н}$ к бруску с наименьшим номером.
    С каким ускорением двигается система? Чему равна сила натяжения нити, связывающей бруски 1 и 2?
}
\answer{%
    \begin{align*}
    a &= \frac{F}{6 m} = \frac{60\,\text{Н}}{6 \cdot 2\,\text{кг}} \approx 5{,}0\,\frac{\text{м}}{\text{c}^{2}}, \\
    T &= m'a = 5m \cdot \frac{F}{6 m} = \frac{5}{6} F \approx 50{,}0\,\text{Н}.
    \end{align*}
}
\solutionspace{120pt}

\tasknumber{6}%
\task{%
    Два бруска связаны лёгкой нерастяжимой нитью и перекинуты через неподвижный блок (см.
    рис.).
    Определите силу натяжения нити и ускорения брусков.
    Силами трения пренебречь, массы брусков
    равны $m_1 = 11\,\text{кг}$ и $m_2 = 10\,\text{кг}$.
    % $g = 10\,\frac{\text{м}}{\text{с}^{2}}$.

    \begin{tikzpicture}[x=1.5cm,y=1.5cm,thick]
        \draw
            (-0.4, 0) rectangle (-0.2, 1.2)
            (0.15, 0.5) rectangle (0.45, 1)
            (0, 2) circle [radius=0.3] -- ++(up:0.5)
            (-0.3, 1.2) -- ++(up:0.8)
            (0.3, 1) -- ++(up:1)
            (-0.7, 2.5) -- (0.7, 2.5)
            ;
        \draw[pattern={Lines[angle=51,distance=3pt]},pattern color=black,draw=none] (-0.7, 2.5) rectangle (0.7, 2.75);
        \node [left] (left) at (-0.4, 0.6) { $m_1$ };
        \node [right] (right) at (0.4, 0.75) { $m_2$ };
    \end{tikzpicture}
}
\answer{%
    Предположим, что левый брусок ускоряется вверх, тогда правый ускоряется вниз (с тем же ускорением).
    Запишем 2-й закон Ньютона 2 раза (для обоих тел) в проекции на вертикальную оси, направив её вверх.
    \begin{align*}
        &\begin{cases}
            T - m_1g = m_1a, \\
            T - m_2g = -m_2a,
        \end{cases} \\
        &\begin{cases}
            m_2g - m_1g = m_1a + m_2a, \\
            T = m_1a + m_1g, \\
        \end{cases} \\
        a &= \frac{m_2 - m_1}{m_1 + m_2} \cdot g = \frac{10\,\text{кг} - 11\,\text{кг}}{11\,\text{кг} + 10\,\text{кг}} \cdot 10\,\frac{\text{м}}{\text{с}^{2}} \approx -0{,}4800\,\frac{\text{м}}{\text{c}^{2}}, \\
        T &= m_1(a + g) = m_1 \cdot g \cdot \cbr{\frac{m_2 - m_1}{m_1 + m_2} + 1} = m_1 \cdot g \cdot \frac{2m_2}{m_1 + m_2} = \\
            &= \frac{2 m_2 m_1 g}{m_1 + m_2} = \frac{2 \cdot 10\,\text{кг} \cdot 11\,\text{кг} \cdot 10\,\frac{\text{м}}{\text{с}^{2}}}{11\,\text{кг} + 10\,\text{кг}} \approx 104{,}8\,\text{Н}.
    \end{align*}
    Отрицательный ответ говорит, что мы лишь не угадали с направлением ускорений.
    Сила же всегда положительна.
}
\solutionspace{80pt}

\tasknumber{7}%
\task{%
    Тело массой $2\,\text{кг}$ лежит на горизонтальной поверхности.
    Коэффициент трения между поверхностью и телом $0{,}2$.
    К телу приложена горизонтальная сила $3{,}5\,\text{Н}$.
    Определите силу трения, действующую на тело, и ускорение тела.
    % $g = 10\,\frac{\text{м}}{\text{с}^{2}}$.
}
\answer{%
    \begin{align*}
    &F_\text{ трения покоя $\max$ } = \mu N = \mu m g = 0{,}2 \cdot 2\,\text{кг} \cdot 10\,\frac{\text{м}}{\text{с}^{2}} = 4{,}00\,\text{Н}, \\
    &F_\text{ трения покоя $\max$ } > F \implies F_\text{ трения } = 3{,}50\,\text{Н}, a = \frac{F - F_\text{ трения }}{ m } = 0\,\frac{\text{м}}{\text{c}^{2}}, \\
    &\text{при равенстве возможны оба варианта: и едет, и не едет, но на ответы это не влияет.}
    \end{align*}
}
\solutionspace{120pt}

\tasknumber{8}%
\task{%
    Определите плотность неизвестного вещества, если известно, что опускании тела из него
    в подсолнечное масло оно будет плавать и на половину выступать над поверхностью жидкости.
}
\answer{%
    $F_\text{Арх.} = F_\text{тяж.} \implies \rho_\text{ж.} g V_\text{погр.} = m g \implies\rho_\text{ж.} g \cbr{V -\frac V2} = \rho V g \implies \rho = \rho_\text{ж.}\cbr{1 -\frac 12} \approx 450\,\frac{\text{кг}}{\text{м}^{3}}$
}
\solutionspace{120pt}

\tasknumber{9}%
\task{%
    	Определите силу, действующую на правую опору однородного горизонтального стержня длиной $l = 5\,\text{м}$
    	и массой $M = 1\,\text{кг}$, к которому подвешен груз массой $m = 4\,\text{кг}$ на расстоянии $2\,\text{м}$ от правого конца (см.
    рис.).

        \begin{tikzpicture}[thick]
            \draw
                (-2, -0.1) rectangle (2, 0.1)
                (-0.5, -0.1) -- (-0.5, -1)
                (-0.7, -1) rectangle (-0.3, -1.3)
           		(-2, -0.1) -- +(0.15,-0.9) -- +(-0.15,-0.9) -- cycle
            	(2, -0.1) -- +(0.15,-0.9) -- +(-0.15,-0.9) -- cycle
            ;
            \draw[pattern={Lines[angle=51,distance=2pt]},pattern color=black,draw=none]
            	(-2.15, -1.15) rectangle +(0.3, 0.15)
            	(2.15, -1.15) rectangle +(-0.3, 0.15)
            ;
            \node [right] (m_small) at (-0.3, -1.15) { $m$ };
            \node [above] (M_big) at (0, 0.1) { $M$ };
        \end{tikzpicture}
}
\answer{%
    \begin{align*}
        &\begin{cases}
            F_1 + F_2 - mg - Mg= 0, \\
            F_1 \cdot 0 - mg \cdot a - Mg \cdot \frac l2 + F_2 \cdot l = 0,
        \end{cases} \\
        F_2 &= \frac{mga + Mg\frac l2}l = \frac al \cdot mg + \frac{Mg}2 \approx 29{,}0\,\text{Н}, \\
        F_1 &= mg + Mg - F_2 = mg + Mg - \frac al \cdot mg - \frac{Mg}2 = \frac bl \cdot mg + \frac{Mg}2 \approx 21{,}0\,\text{Н}.
    \end{align*}
}
\solutionspace{80pt}

\tasknumber{10}%
\task{%
    Тонкий однородный кусок арматуры длиной $2\,\text{м}$ и массой $30\,\text{кг}$ лежит на горизонтальной поверхности.
    \begin{itemize}
        \item Какую минимальную силу надо приложить к одному из его концов, чтобы оторвать его от этой поверхности?
        \item Какую минимальную работу надо совершить, чтобы поставить его на землю в вертикальное положение?
    \end{itemize}
    % Примите $g = 10\,\frac{\text{м}}{\text{с}^{2}}$.
}
\answer{%
    $F = \frac{mg}2 \approx 300\,\text{Н}, A = mg\frac l2 = 300\,\text{Дж}$
}
\solutionspace{120pt}

\tasknumber{11}%
\task{%
    Определите работу силы, которая обеспечит подъём тела массой $5\,\text{кг}$ на высоту $2\,\text{м}$ с постоянным ускорением $2\,\frac{\text{м}}{\text{c}^{2}}$.
    % Примите $g = 10\,\frac{\text{м}}{\text{с}^{2}}$.
}
\answer{%
    \begin{align*}
    &\text{Для подъёма:} A = Fh = (mg + ma) h = m(g+a)h, \\
    &\text{Для спуска:} A = -Fh = -(mg - ma) h = -m(g-a)h, \\
    &\text{В результате получаем:} 120\,\text{Дж}.
    \end{align*}
}
\solutionspace{60pt}

\tasknumber{12}%
\task{%
    Тело бросили вертикально вверх со скоростью $14\,\frac{\text{м}}{\text{c}}$.
    На какой высоте кинетическая энергия тела составит треть от потенциальной?
}
\answer{%
    \begin{align*}
    &0 + \frac{mv_0^2}2 = E_p + E_k, E_k = \frac 13 E_p \implies \\
    &\implies \frac{mv_0^2}2 = E_p + \frac 13 E_p = E_p\cbr{1 + \frac 13} = mgh\cbr{1 + \frac 13} \implies \\
    &\implies h = \frac{\frac{mv_0^2}2}{mg\cbr{1 + \frac 13}} = \frac{v_0^2}{2g} \cdot \frac 1{1 + \frac 13} \approx 7{,}4\,\text{м}.
    \end{align*}
}
\solutionspace{100pt}

\tasknumber{13}%
\task{%
    Плотность воздуха при нормальных условиях равна $1{,}3\,\frac{\text{кг}}{\text{м}^{3}}$.
    Чему равна плотность воздуха
    при температуре $100\celsius$ и давлении $120\,\text{кПа}$?
}
\answer{%
    \begin{align*}
    &\text{В общем случае:} PV = \frac m{\mu} RT \implies \rho = \frac mV = \frac m{\frac{\frac m{\mu} RT}P} = \frac{P\mu}{RT}, \\
    &\text{У нас 2 состояния:} \rho_1 = \frac{P_1\mu}{RT_1}, \rho_2 = \frac{P_2\mu}{RT_2} \implies \frac{\rho_2}{\rho_1} = \frac{\frac{P_2\mu}{RT_2}}{\frac{P_1\mu}{RT_1}} = \frac{P_2T_1}{P_1T_2} \implies \\
    &\implies \rho_2 = \rho_1 \cdot  \frac{P_2T_1}{P_1T_2} = 1{,}3\,\frac{\text{кг}}{\text{м}^{3}} \cdot \frac{120\,\text{кПа} \cdot 273\units{К}}{100\,\text{кПа} \cdot 373\units{К}} \approx 1{,}14\,\frac{\text{кг}}{\text{м}^{3}}.
    \end{align*}
}
\solutionspace{120pt}

\tasknumber{14}%
\task{%
    Небольшую цилиндрическую пробирку с воздухом погружают на некоторую глубину в глубокое пресное озеро,
    после чего воздух занимает в ней лишь пятую часть от общего объема.
    Определите глубину, на которую погрузили пробирку.
    Температуру считать постоянной $T = 281\,\text{К}$, давлением паров воды пренебречь,
    атмосферное давление принять равным $p_{\text{aтм}} = 100\,\text{кПа}$.
}
\answer{%
    \begin{align*}
    T\text{— const} &\implies P_1V_1 = \nu RT = P_2V_2.
    \\
    V_2 = \frac 15 V_1 &\implies P_1V_1 = P_2 \cdot \frac 15V_1 \implies P_2 = 5P_1 = 5p_{\text{aтм}}.
    \\
    P_2 = p_{\text{aтм}} + \rho_{\text{в}} g h \implies h = \frac{P_2 - p_{\text{aтм}}}{\rho_{\text{в}} g} &= \frac{5p_{\text{aтм}} - p_{\text{aтм}}}{\rho_{\text{в}} g} = \frac{4 \cdot p_{\text{aтм}}}{\rho_{\text{в}} g} =  \\
     &= \frac{4 \cdot 100\,\text{кПа}}{1000\,\frac{\text{кг}}{\text{м}^{3}} \cdot  10\,\frac{\text{м}}{\text{с}^{2}}} \approx 40\,\text{м}.
    \end{align*}
}
\solutionspace{120pt}

\tasknumber{15}%
\task{%
    Газу сообщили некоторое количество теплоты,
    при этом четверть его он потратил на совершение работы,
    одновременно увеличив свою внутреннюю энергию на $1500\,\text{Дж}$.
    Определите количество теплоты, сообщённое газу.
}
\answer{%
    \begin{align*}
    Q &= A' + \Delta U, A' = \frac 14 Q \implies Q \cdot \cbr{1 - \frac 14} = \Delta U \implies Q = \frac{\Delta U}{1 - \frac 14} = \frac{ 1500\,\text{Дж} }{1 - \frac 14} \approx 2000\,\text{Дж}.
    \\
    A' &= \frac 14 Q
        = \frac 14 \cdot \frac{\Delta U}{1 - \frac 14}
        = \frac{\Delta U}{4 - 1}
        = \frac{ 1500\,\text{Дж} }{4 - 1} \approx 500\,\text{Дж}.
    \end{align*}
}
\solutionspace{60pt}

\tasknumber{16}%
\task{%
    Два конденсатора ёмкостей $C_1 = 20\,\text{нФ}$ и $C_2 = 30\,\text{нФ}$ последовательно подключают
    к источнику напряжения $U = 150\,\text{В}$ (см.
    рис.).
    % Определите заряды каждого из конденсаторов.
    Определите заряд второго конденсатора.

    \begin{tikzpicture}[circuit ee IEC, semithick]
        \draw  (0, 0) to [capacitor={info={$C_1$}}] (1, 0)
                       to [capacitor={info={$C_2$}}] (2, 0)
        ;
        % \draw [-o] (0, 0) -- ++(-0.5, 0) node[left] {$-$};
        % \draw [-o] (2, 0) -- ++(0.5, 0) node[right] {$+$};
        \draw [-o] (0, 0) -- ++(-0.5, 0) node[left] {};
        \draw [-o] (2, 0) -- ++(0.5, 0) node[right] {};
    \end{tikzpicture}
}
\answer{%
    $
        Q_1
            = Q_2
            = CU
            = \frac{ U }{\frac1{C_1} + \frac1{C_2}}
            = \frac{C_1C_2U}{C_1 + C_2}
            = \frac{
                20\,\text{нФ} \cdot 30\,\text{нФ} \cdot 150\,\text{В}
            }{
                20\,\text{нФ} + 30\,\text{нФ}
            }
            = 1{,}80\,\text{мкКл}
    $
}
\solutionspace{120pt}

\tasknumber{17}%
\task{%
    В вакууме вдоль одной прямой расположены три отрицательных заряда так,
    что расстояние между соседними зарядами равно $d$.
    Сделайте рисунок,
    и определите силу, действующую на крайний заряд.
    Модули всех зарядов равны $Q$ ($Q > 0$).
}
\answer{%
    $F = \sum_i F_i = \ldots = \frac54 \frac{kQ^2}{d^2}.$
}
\solutionspace{80pt}

\tasknumber{18}%
\task{%
    Юлия проводит эксперименты c 2 кусками одинаковой медной проволки, причём второй кусок в семь раз длиннее первого.
    В одном из экспериментов Юлия подаёт на первый кусок проволки напряжение в три раза раз больше, чем на второй.
    Определите отношения в двух проволках в этом эксперименте (второй к первой):
    \begin{itemize}
        \item отношение сил тока,
        \item отношение выделяющихся мощностей.
    \end{itemize}
}
\answer{%
    $R_2 = 7R_1, U_1 = 3U_2 \implies  \eli_2 / \eli_1 = \frac{U_2 / R_2}{U_1 / R_1} = \frac{U_2}{U_1} \cdot \frac{R_1}{R_2} = \frac1{21}, P_2 / P_1 = \frac{U_2^2 / R_2}{U_1^2 / R_1} = \sqr{\frac{U_2}{U_1}} \cdot \frac{R_1}{R_2} = \frac1{63}.$
}

\variantsplitter

\addpersonalvariant{Константин Козлов}

\tasknumber{1}%
\task{%
    Женя стартует на велосипеде и в течение $t = 3\,\text{c}$ двигается с постоянным ускорением $2{,}5\,\frac{\text{м}}{\text{с}^{2}}$.
    Определите
    \begin{itemize}
        \item какую скорость при этом удастся достичь,
        \item какой путь за это время будет пройден,
        \item среднюю скорость за всё время движения, если после начального ускорения продолжить движение равномерно ещё в течение времени $3t$
    \end{itemize}
}
\answer{%
    \begin{align*}
    v &= v_0 + a t = at = 2{,}5\,\frac{\text{м}}{\text{с}^{2}} \cdot 3\,\text{c} = 7{,}5\,\frac{\text{м}}{\text{с}}, \\
    s_x &= v_0t + \frac{a t^2}2 = \frac{a t^2}2 = \frac{2{,}5\,\frac{\text{м}}{\text{с}^{2}} \cdot \sqr{ 3\,\text{c} }}2 = 11{,}2\,\text{м}, \\
    v_\text{сред.} &= \frac{s_\text{общ}}{t_\text{общ.}} = \frac{s_x + v \cdot 3t}{t + 3t} = \frac{\frac{a t^2}2 + at \cdot 3t}{t (1 + 3)} = \\
    &= at \cdot \frac{\frac 12 + 3}{1 + 3} = 2{,}5\,\frac{\text{м}}{\text{с}^{2}} \cdot 3\,\text{c} \cdot \frac{\frac 12 + 3}{1 + 3} \approx 6{,}56\,\frac{\text{м}}{\text{c}}.
    \end{align*}
}
\solutionspace{120pt}

\tasknumber{2}%
\task{%
    Какой путь тело пройдёт за четвёртую секунду после начала свободного падения?
    Какую скорость в начале этой секунды оно имеет?
}
\answer{%
    \begin{align*}
    s &= -s_y = -(y_2-y_1) = y_1 - y_2 = \cbr{y_{0y} + v_{0y}t_1 - \frac{gt_1^2}2} - \cbr{y_{0y} + v_{0y}t_2 - \frac{gt_2^2}2} = \\
    &= \frac{gt_2^2}2 - \frac{gt_1^2}2 = \frac g2\cbr{t_2^2 - t_1^2} = 35{,}0\,\text{м}, \\
    v_y &= v_{0y} - gt = -gt = 10\,\frac{\text{м}}{\text{с}^{2}} \cdot 3\,\text{с} = -30\,\frac{\text{м}}{\text{с}}.
    \end{align*}
}
\solutionspace{120pt}

\tasknumber{3}%
\task{%
    Карусель диаметром $2\,\text{м}$ равномерно совершает 6 оборотов в минуту.
    Определите
    \begin{itemize}
        \item период и частоту её обращения,
        \item скорость и ускорение крайних её точек.
    \end{itemize}
}
\answer{%
    \begin{align*}
    t &= 60\,\text{с}, r = 1{,}0\,\text{м}, n = 6\units{оборотов}, \\
    T &= \frac tN = \frac{ 60\,\text{с} }{6} \approx 10{,}00\,\text{c}, \\
    \nu &= \frac 1T = \frac{6}{ 60\,\text{с} } \approx 0{,}10\,\text{Гц}, \\
    v &= \frac{2 \pi r}{T} = \frac{2 \pi r}{T} =  \frac{2 \pi r n}{t} \approx 0{,}63\,\frac{\text{м}}{\text{c}}, \\
    a &= \frac{v^2}{r} =  \frac{4 \pi^2 r n^2}{t^2} \approx 0{,}39\,\frac{\text{м}}{\text{с}^{2}}.
    \end{align*}
}
\solutionspace{80pt}

\tasknumber{4}%
\task{%
    Миша стоит на обрыве над рекой и методично и строго горизонтально кидает в неё камушки.
    За этим всем наблюдает экспериментатор Глюк, который уже выяснил, что камушки падают в реку спустя $1{,}6\,\text{с}$ после броска,
    а вот дальность полёта оценить сложнее: придётся лезть в воду.
    Выручите Глюка и определите:
    \begin{itemize}
        \item высоту обрыва (вместе с ростом Миши).
        \item дальность полёта камушков (по горизонтали) и их скорость при падении, приняв начальную скорость броска равной $v_0 = 16\,\frac{\text{м}}{\text{с}}$.
    \end{itemize}
    Сопротивлением воздуха пренебречь.
}
\answer{%
    \begin{align*}
    y &= y_0 + v_{0y}t - \frac{gt^2}2 = h - \frac{gt^2}2, \qquad y(\tau) = 0 \implies h - \frac{g\tau^2}2 = 0 \implies h = \frac{g\tau^2}2 \approx 12{,}8\,\text{м}.
    \\
    x &= x_0 + v_{0x}t = v_0t \implies L = v_0\tau \approx 25{,}6\,\text{м}.
    \\
    &v = \sqrt{v_x^2 + v_y^2} = \sqrt{v_{0x}^2 + \sqr{v_{0y} - g\tau}} = \sqrt{v_0^2 + \sqr{g\tau}} \approx 22{,}6\,\frac{\text{м}}{\text{c}}.
    \end{align*}
}
\solutionspace{120pt}

\tasknumber{5}%
\task{%
    Шесть одинаковых брусков массой $2\,\text{кг}$ каждый лежат на гладком горизонтальном столе.
    Бруски пронумерованы от 1 до 6 и последовательно связаны между собой
    невесомыми нерастяжимыми нитями: 1 со 2, 2 с 3 (ну и с 1) и т.д.
    Экспериментатор Глюк прикладывает постоянную горизонтальную силу $90\,\text{Н}$ к бруску с наименьшим номером.
    С каким ускорением двигается система? Чему равна сила натяжения нити, связывающей бруски 3 и 4?
}
\answer{%
    \begin{align*}
    a &= \frac{F}{6 m} = \frac{90\,\text{Н}}{6 \cdot 2\,\text{кг}} \approx 7{,}5\,\frac{\text{м}}{\text{c}^{2}}, \\
    T &= m'a = 3m \cdot \frac{F}{6 m} = \frac{3}{6} F \approx 45{,}0\,\text{Н}.
    \end{align*}
}
\solutionspace{120pt}

\tasknumber{6}%
\task{%
    Два бруска связаны лёгкой нерастяжимой нитью и перекинуты через неподвижный блок (см.
    рис.).
    Определите силу натяжения нити и ускорения брусков.
    Силами трения пренебречь, массы брусков
    равны $m_1 = 11\,\text{кг}$ и $m_2 = 10\,\text{кг}$.
    % $g = 10\,\frac{\text{м}}{\text{с}^{2}}$.

    \begin{tikzpicture}[x=1.5cm,y=1.5cm,thick]
        \draw
            (-0.4, 0) rectangle (-0.2, 1.2)
            (0.15, 0.5) rectangle (0.45, 1)
            (0, 2) circle [radius=0.3] -- ++(up:0.5)
            (-0.3, 1.2) -- ++(up:0.8)
            (0.3, 1) -- ++(up:1)
            (-0.7, 2.5) -- (0.7, 2.5)
            ;
        \draw[pattern={Lines[angle=51,distance=3pt]},pattern color=black,draw=none] (-0.7, 2.5) rectangle (0.7, 2.75);
        \node [left] (left) at (-0.4, 0.6) { $m_1$ };
        \node [right] (right) at (0.4, 0.75) { $m_2$ };
    \end{tikzpicture}
}
\answer{%
    Предположим, что левый брусок ускоряется вверх, тогда правый ускоряется вниз (с тем же ускорением).
    Запишем 2-й закон Ньютона 2 раза (для обоих тел) в проекции на вертикальную оси, направив её вверх.
    \begin{align*}
        &\begin{cases}
            T - m_1g = m_1a, \\
            T - m_2g = -m_2a,
        \end{cases} \\
        &\begin{cases}
            m_2g - m_1g = m_1a + m_2a, \\
            T = m_1a + m_1g, \\
        \end{cases} \\
        a &= \frac{m_2 - m_1}{m_1 + m_2} \cdot g = \frac{10\,\text{кг} - 11\,\text{кг}}{11\,\text{кг} + 10\,\text{кг}} \cdot 10\,\frac{\text{м}}{\text{с}^{2}} \approx -0{,}4800\,\frac{\text{м}}{\text{c}^{2}}, \\
        T &= m_1(a + g) = m_1 \cdot g \cdot \cbr{\frac{m_2 - m_1}{m_1 + m_2} + 1} = m_1 \cdot g \cdot \frac{2m_2}{m_1 + m_2} = \\
            &= \frac{2 m_2 m_1 g}{m_1 + m_2} = \frac{2 \cdot 10\,\text{кг} \cdot 11\,\text{кг} \cdot 10\,\frac{\text{м}}{\text{с}^{2}}}{11\,\text{кг} + 10\,\text{кг}} \approx 104{,}8\,\text{Н}.
    \end{align*}
    Отрицательный ответ говорит, что мы лишь не угадали с направлением ускорений.
    Сила же всегда положительна.
}
\solutionspace{80pt}

\tasknumber{7}%
\task{%
    Тело массой $2\,\text{кг}$ лежит на горизонтальной поверхности.
    Коэффициент трения между поверхностью и телом $0{,}25$.
    К телу приложена горизонтальная сила $5{,}5\,\text{Н}$.
    Определите силу трения, действующую на тело, и ускорение тела.
    % $g = 10\,\frac{\text{м}}{\text{с}^{2}}$.
}
\answer{%
    \begin{align*}
    &F_\text{ трения покоя $\max$ } = \mu N = \mu m g = 0{,}25 \cdot 2\,\text{кг} \cdot 10\,\frac{\text{м}}{\text{с}^{2}} = 5{,}00\,\text{Н}, \\
    &F_\text{ трения покоя $\max$ } \le F \implies F_\text{ трения } = 5{,}00\,\text{Н}, a = \frac{F - F_\text{ трения }}{ m } = 0{,}25\,\frac{\text{м}}{\text{c}^{2}}, \\
    &\text{при равенстве возможны оба варианта: и едет, и не едет, но на ответы это не влияет.}
    \end{align*}
}
\solutionspace{120pt}

\tasknumber{8}%
\task{%
    Определите плотность неизвестного вещества, если известно, что опускании тела из него
    в подсолнечное масло оно будет плавать и на половину выступать над поверхностью жидкости.
}
\answer{%
    $F_\text{Арх.} = F_\text{тяж.} \implies \rho_\text{ж.} g V_\text{погр.} = m g \implies\rho_\text{ж.} g \cbr{V -\frac V2} = \rho V g \implies \rho = \rho_\text{ж.}\cbr{1 -\frac 12} \approx 450\,\frac{\text{кг}}{\text{м}^{3}}$
}
\solutionspace{120pt}

\tasknumber{9}%
\task{%
    	Определите силу, действующую на правую опору однородного горизонтального стержня длиной $l = 9\,\text{м}$
    	и массой $M = 1\,\text{кг}$, к которому подвешен груз массой $m = 3\,\text{кг}$ на расстоянии $4\,\text{м}$ от правого конца (см.
    рис.).

        \begin{tikzpicture}[thick]
            \draw
                (-2, -0.1) rectangle (2, 0.1)
                (-0.5, -0.1) -- (-0.5, -1)
                (-0.7, -1) rectangle (-0.3, -1.3)
           		(-2, -0.1) -- +(0.15,-0.9) -- +(-0.15,-0.9) -- cycle
            	(2, -0.1) -- +(0.15,-0.9) -- +(-0.15,-0.9) -- cycle
            ;
            \draw[pattern={Lines[angle=51,distance=2pt]},pattern color=black,draw=none]
            	(-2.15, -1.15) rectangle +(0.3, 0.15)
            	(2.15, -1.15) rectangle +(-0.3, 0.15)
            ;
            \node [right] (m_small) at (-0.3, -1.15) { $m$ };
            \node [above] (M_big) at (0, 0.1) { $M$ };
        \end{tikzpicture}
}
\answer{%
    \begin{align*}
        &\begin{cases}
            F_1 + F_2 - mg - Mg= 0, \\
            F_1 \cdot 0 - mg \cdot a - Mg \cdot \frac l2 + F_2 \cdot l = 0,
        \end{cases} \\
        F_2 &= \frac{mga + Mg\frac l2}l = \frac al \cdot mg + \frac{Mg}2 \approx 21{,}7\,\text{Н}, \\
        F_1 &= mg + Mg - F_2 = mg + Mg - \frac al \cdot mg - \frac{Mg}2 = \frac bl \cdot mg + \frac{Mg}2 \approx 18{,}3\,\text{Н}.
    \end{align*}
}
\solutionspace{80pt}

\tasknumber{10}%
\task{%
    Тонкий однородный кусок арматуры длиной $3\,\text{м}$ и массой $10\,\text{кг}$ лежит на горизонтальной поверхности.
    \begin{itemize}
        \item Какую минимальную силу надо приложить к одному из его концов, чтобы оторвать его от этой поверхности?
        \item Какую минимальную работу надо совершить, чтобы поставить его на землю в вертикальное положение?
    \end{itemize}
    % Примите $g = 10\,\frac{\text{м}}{\text{с}^{2}}$.
}
\answer{%
    $F = \frac{mg}2 \approx 100\,\text{Н}, A = mg\frac l2 = 150\,\text{Дж}$
}
\solutionspace{120pt}

\tasknumber{11}%
\task{%
    Определите работу силы, которая обеспечит спуск тела массой $2\,\text{кг}$ на высоту $5\,\text{м}$ с постоянным ускорением $6\,\frac{\text{м}}{\text{c}^{2}}$.
    % Примите $g = 10\,\frac{\text{м}}{\text{с}^{2}}$.
}
\answer{%
    \begin{align*}
    &\text{Для подъёма:} A = Fh = (mg + ma) h = m(g+a)h, \\
    &\text{Для спуска:} A = -Fh = -(mg - ma) h = -m(g-a)h, \\
    &\text{В результате получаем:} -40\,\text{Дж}.
    \end{align*}
}
\solutionspace{60pt}

\tasknumber{12}%
\task{%
    Тело бросили вертикально вверх со скоростью $20\,\frac{\text{м}}{\text{c}}$.
    На какой высоте кинетическая энергия тела составит половину от потенциальной?
}
\answer{%
    \begin{align*}
    &0 + \frac{mv_0^2}2 = E_p + E_k, E_k = \frac 12 E_p \implies \\
    &\implies \frac{mv_0^2}2 = E_p + \frac 12 E_p = E_p\cbr{1 + \frac 12} = mgh\cbr{1 + \frac 12} \implies \\
    &\implies h = \frac{\frac{mv_0^2}2}{mg\cbr{1 + \frac 12}} = \frac{v_0^2}{2g} \cdot \frac 1{1 + \frac 12} \approx 13{,}3\,\text{м}.
    \end{align*}
}
\solutionspace{100pt}

\tasknumber{13}%
\task{%
    Плотность воздуха при нормальных условиях равна $1{,}3\,\frac{\text{кг}}{\text{м}^{3}}$.
    Чему равна плотность воздуха
    при температуре $200\celsius$ и давлении $150\,\text{кПа}$?
}
\answer{%
    \begin{align*}
    &\text{В общем случае:} PV = \frac m{\mu} RT \implies \rho = \frac mV = \frac m{\frac{\frac m{\mu} RT}P} = \frac{P\mu}{RT}, \\
    &\text{У нас 2 состояния:} \rho_1 = \frac{P_1\mu}{RT_1}, \rho_2 = \frac{P_2\mu}{RT_2} \implies \frac{\rho_2}{\rho_1} = \frac{\frac{P_2\mu}{RT_2}}{\frac{P_1\mu}{RT_1}} = \frac{P_2T_1}{P_1T_2} \implies \\
    &\implies \rho_2 = \rho_1 \cdot  \frac{P_2T_1}{P_1T_2} = 1{,}3\,\frac{\text{кг}}{\text{м}^{3}} \cdot \frac{150\,\text{кПа} \cdot 273\units{К}}{100\,\text{кПа} \cdot 473\units{К}} \approx 1{,}13\,\frac{\text{кг}}{\text{м}^{3}}.
    \end{align*}
}
\solutionspace{120pt}

\tasknumber{14}%
\task{%
    Небольшую цилиндрическую пробирку с воздухом погружают на некоторую глубину в глубокое пресное озеро,
    после чего воздух занимает в ней лишь пятую часть от общего объема.
    Определите глубину, на которую погрузили пробирку.
    Температуру считать постоянной $T = 292\,\text{К}$, давлением паров воды пренебречь,
    атмосферное давление принять равным $p_{\text{aтм}} = 100\,\text{кПа}$.
}
\answer{%
    \begin{align*}
    T\text{— const} &\implies P_1V_1 = \nu RT = P_2V_2.
    \\
    V_2 = \frac 15 V_1 &\implies P_1V_1 = P_2 \cdot \frac 15V_1 \implies P_2 = 5P_1 = 5p_{\text{aтм}}.
    \\
    P_2 = p_{\text{aтм}} + \rho_{\text{в}} g h \implies h = \frac{P_2 - p_{\text{aтм}}}{\rho_{\text{в}} g} &= \frac{5p_{\text{aтм}} - p_{\text{aтм}}}{\rho_{\text{в}} g} = \frac{4 \cdot p_{\text{aтм}}}{\rho_{\text{в}} g} =  \\
     &= \frac{4 \cdot 100\,\text{кПа}}{1000\,\frac{\text{кг}}{\text{м}^{3}} \cdot  10\,\frac{\text{м}}{\text{с}^{2}}} \approx 40\,\text{м}.
    \end{align*}
}
\solutionspace{120pt}

\tasknumber{15}%
\task{%
    Газу сообщили некоторое количество теплоты,
    при этом четверть его он потратил на совершение работы,
    одновременно увеличив свою внутреннюю энергию на $3000\,\text{Дж}$.
    Определите количество теплоты, сообщённое газу.
}
\answer{%
    \begin{align*}
    Q &= A' + \Delta U, A' = \frac 14 Q \implies Q \cdot \cbr{1 - \frac 14} = \Delta U \implies Q = \frac{\Delta U}{1 - \frac 14} = \frac{ 3000\,\text{Дж} }{1 - \frac 14} \approx 4000\,\text{Дж}.
    \\
    A' &= \frac 14 Q
        = \frac 14 \cdot \frac{\Delta U}{1 - \frac 14}
        = \frac{\Delta U}{4 - 1}
        = \frac{ 3000\,\text{Дж} }{4 - 1} \approx 1000\,\text{Дж}.
    \end{align*}
}
\solutionspace{60pt}

\tasknumber{16}%
\task{%
    Два конденсатора ёмкостей $C_1 = 60\,\text{нФ}$ и $C_2 = 30\,\text{нФ}$ последовательно подключают
    к источнику напряжения $U = 300\,\text{В}$ (см.
    рис.).
    % Определите заряды каждого из конденсаторов.
    Определите заряд первого конденсатора.

    \begin{tikzpicture}[circuit ee IEC, semithick]
        \draw  (0, 0) to [capacitor={info={$C_1$}}] (1, 0)
                       to [capacitor={info={$C_2$}}] (2, 0)
        ;
        % \draw [-o] (0, 0) -- ++(-0.5, 0) node[left] {$-$};
        % \draw [-o] (2, 0) -- ++(0.5, 0) node[right] {$+$};
        \draw [-o] (0, 0) -- ++(-0.5, 0) node[left] {};
        \draw [-o] (2, 0) -- ++(0.5, 0) node[right] {};
    \end{tikzpicture}
}
\answer{%
    $
        Q_1
            = Q_2
            = CU
            = \frac{ U }{\frac1{C_1} + \frac1{C_2}}
            = \frac{C_1C_2U}{C_1 + C_2}
            = \frac{
                60\,\text{нФ} \cdot 30\,\text{нФ} \cdot 300\,\text{В}
            }{
                60\,\text{нФ} + 30\,\text{нФ}
            }
            = 6{,}00\,\text{мкКл}
    $
}
\solutionspace{120pt}

\tasknumber{17}%
\task{%
    В вакууме вдоль одной прямой расположены четыре отрицательных заряда так,
    что расстояние между соседними зарядами равно $r$.
    Сделайте рисунок,
    и определите силу, действующую на крайний заряд.
    Модули всех зарядов равны $Q$ ($Q > 0$).
}
\answer{%
    $F = \sum_i F_i = \ldots = \frac{49}{36} \frac{kQ^2}{r^2}.$
}
\solutionspace{80pt}

\tasknumber{18}%
\task{%
    Юлия проводит эксперименты c 2 кусками одинаковой стальной проволки, причём второй кусок в семь раз длиннее первого.
    В одном из экспериментов Юлия подаёт на первый кусок проволки напряжение в десять раз раз больше, чем на второй.
    Определите отношения в двух проволках в этом эксперименте (второй к первой):
    \begin{itemize}
        \item отношение сил тока,
        \item отношение выделяющихся мощностей.
    \end{itemize}
}
\answer{%
    $R_2 = 7R_1, U_1 = 10U_2 \implies  \eli_2 / \eli_1 = \frac{U_2 / R_2}{U_1 / R_1} = \frac{U_2}{U_1} \cdot \frac{R_1}{R_2} = \frac1{70}, P_2 / P_1 = \frac{U_2^2 / R_2}{U_1^2 / R_1} = \sqr{\frac{U_2}{U_1}} \cdot \frac{R_1}{R_2} = \frac1{700}.$
}

\variantsplitter

\addpersonalvariant{Наталья Кравченко}

\tasknumber{1}%
\task{%
    Саша стартует на лошади и в течение $t = 3\,\text{c}$ двигается с постоянным ускорением $0{,}5\,\frac{\text{м}}{\text{с}^{2}}$.
    Определите
    \begin{itemize}
        \item какую скорость при этом удастся достичь,
        \item какой путь за это время будет пройден,
        \item среднюю скорость за всё время движения, если после начального ускорения продолжить движение равномерно ещё в течение времени $2t$
    \end{itemize}
}
\answer{%
    \begin{align*}
    v &= v_0 + a t = at = 0{,}5\,\frac{\text{м}}{\text{с}^{2}} \cdot 3\,\text{c} = 1{,}5\,\frac{\text{м}}{\text{с}}, \\
    s_x &= v_0t + \frac{a t^2}2 = \frac{a t^2}2 = \frac{0{,}5\,\frac{\text{м}}{\text{с}^{2}} \cdot \sqr{ 3\,\text{c} }}2 = 2{,}2\,\text{м}, \\
    v_\text{сред.} &= \frac{s_\text{общ}}{t_\text{общ.}} = \frac{s_x + v \cdot 2t}{t + 2t} = \frac{\frac{a t^2}2 + at \cdot 2t}{t (1 + 2)} = \\
    &= at \cdot \frac{\frac 12 + 2}{1 + 2} = 0{,}5\,\frac{\text{м}}{\text{с}^{2}} \cdot 3\,\text{c} \cdot \frac{\frac 12 + 2}{1 + 2} \approx 1{,}25\,\frac{\text{м}}{\text{c}}.
    \end{align*}
}
\solutionspace{120pt}

\tasknumber{2}%
\task{%
    Какой путь тело пройдёт за шестую секунду после начала свободного падения?
    Какую скорость в конце этой секунды оно имеет?
}
\answer{%
    \begin{align*}
    s &= -s_y = -(y_2-y_1) = y_1 - y_2 = \cbr{y_{0y} + v_{0y}t_1 - \frac{gt_1^2}2} - \cbr{y_{0y} + v_{0y}t_2 - \frac{gt_2^2}2} = \\
    &= \frac{gt_2^2}2 - \frac{gt_1^2}2 = \frac g2\cbr{t_2^2 - t_1^2} = 55{,}0\,\text{м}, \\
    v_y &= v_{0y} - gt = -gt = 10\,\frac{\text{м}}{\text{с}^{2}} \cdot 6\,\text{с} = -60\,\frac{\text{м}}{\text{с}}.
    \end{align*}
}
\solutionspace{120pt}

\tasknumber{3}%
\task{%
    Карусель диаметром $2\,\text{м}$ равномерно совершает 6 оборотов в минуту.
    Определите
    \begin{itemize}
        \item период и частоту её обращения,
        \item скорость и ускорение крайних её точек.
    \end{itemize}
}
\answer{%
    \begin{align*}
    t &= 60\,\text{с}, r = 1{,}0\,\text{м}, n = 6\units{оборотов}, \\
    T &= \frac tN = \frac{ 60\,\text{с} }{6} \approx 10{,}00\,\text{c}, \\
    \nu &= \frac 1T = \frac{6}{ 60\,\text{с} } \approx 0{,}10\,\text{Гц}, \\
    v &= \frac{2 \pi r}{T} = \frac{2 \pi r}{T} =  \frac{2 \pi r n}{t} \approx 0{,}63\,\frac{\text{м}}{\text{c}}, \\
    a &= \frac{v^2}{r} =  \frac{4 \pi^2 r n^2}{t^2} \approx 0{,}39\,\frac{\text{м}}{\text{с}^{2}}.
    \end{align*}
}
\solutionspace{80pt}

\tasknumber{4}%
\task{%
    Маша стоит на обрыве над рекой и методично и строго горизонтально кидает в неё камушки.
    За этим всем наблюдает экспериментатор Глюк, который уже выяснил, что камушки падают в реку спустя $1{,}2\,\text{с}$ после броска,
    а вот дальность полёта оценить сложнее: придётся лезть в воду.
    Выручите Глюка и определите:
    \begin{itemize}
        \item высоту обрыва (вместе с ростом Маши).
        \item дальность полёта камушков (по горизонтали) и их скорость при падении, приняв начальную скорость броска равной $v_0 = 15\,\frac{\text{м}}{\text{с}}$.
    \end{itemize}
    Сопротивлением воздуха пренебречь.
}
\answer{%
    \begin{align*}
    y &= y_0 + v_{0y}t - \frac{gt^2}2 = h - \frac{gt^2}2, \qquad y(\tau) = 0 \implies h - \frac{g\tau^2}2 = 0 \implies h = \frac{g\tau^2}2 \approx 7{,}2\,\text{м}.
    \\
    x &= x_0 + v_{0x}t = v_0t \implies L = v_0\tau \approx 18{,}0\,\text{м}.
    \\
    &v = \sqrt{v_x^2 + v_y^2} = \sqrt{v_{0x}^2 + \sqr{v_{0y} - g\tau}} = \sqrt{v_0^2 + \sqr{g\tau}} \approx 19{,}2\,\frac{\text{м}}{\text{c}}.
    \end{align*}
}
\solutionspace{120pt}

\tasknumber{5}%
\task{%
    Шесть одинаковых брусков массой $2\,\text{кг}$ каждый лежат на гладком горизонтальном столе.
    Бруски пронумерованы от 1 до 6 и последовательно связаны между собой
    невесомыми нерастяжимыми нитями: 1 со 2, 2 с 3 (ну и с 1) и т.д.
    Экспериментатор Глюк прикладывает постоянную горизонтальную силу $60\,\text{Н}$ к бруску с наибольшим номером.
    С каким ускорением двигается система? Чему равна сила натяжения нити, связывающей бруски 3 и 4?
}
\answer{%
    \begin{align*}
    a &= \frac{F}{6 m} = \frac{60\,\text{Н}}{6 \cdot 2\,\text{кг}} \approx 5{,}0\,\frac{\text{м}}{\text{c}^{2}}, \\
    T &= m'a = 3m \cdot \frac{F}{6 m} = \frac{3}{6} F \approx 30{,}0\,\text{Н}.
    \end{align*}
}
\solutionspace{120pt}

\tasknumber{6}%
\task{%
    Два бруска связаны лёгкой нерастяжимой нитью и перекинуты через неподвижный блок (см.
    рис.).
    Определите силу натяжения нити и ускорения брусков.
    Силами трения пренебречь, массы брусков
    равны $m_1 = 8\,\text{кг}$ и $m_2 = 6\,\text{кг}$.
    % $g = 10\,\frac{\text{м}}{\text{с}^{2}}$.

    \begin{tikzpicture}[x=1.5cm,y=1.5cm,thick]
        \draw
            (-0.4, 0) rectangle (-0.2, 1.2)
            (0.15, 0.5) rectangle (0.45, 1)
            (0, 2) circle [radius=0.3] -- ++(up:0.5)
            (-0.3, 1.2) -- ++(up:0.8)
            (0.3, 1) -- ++(up:1)
            (-0.7, 2.5) -- (0.7, 2.5)
            ;
        \draw[pattern={Lines[angle=51,distance=3pt]},pattern color=black,draw=none] (-0.7, 2.5) rectangle (0.7, 2.75);
        \node [left] (left) at (-0.4, 0.6) { $m_1$ };
        \node [right] (right) at (0.4, 0.75) { $m_2$ };
    \end{tikzpicture}
}
\answer{%
    Предположим, что левый брусок ускоряется вверх, тогда правый ускоряется вниз (с тем же ускорением).
    Запишем 2-й закон Ньютона 2 раза (для обоих тел) в проекции на вертикальную оси, направив её вверх.
    \begin{align*}
        &\begin{cases}
            T - m_1g = m_1a, \\
            T - m_2g = -m_2a,
        \end{cases} \\
        &\begin{cases}
            m_2g - m_1g = m_1a + m_2a, \\
            T = m_1a + m_1g, \\
        \end{cases} \\
        a &= \frac{m_2 - m_1}{m_1 + m_2} \cdot g = \frac{6\,\text{кг} - 8\,\text{кг}}{8\,\text{кг} + 6\,\text{кг}} \cdot 10\,\frac{\text{м}}{\text{с}^{2}} \approx -1{,}4300\,\frac{\text{м}}{\text{c}^{2}}, \\
        T &= m_1(a + g) = m_1 \cdot g \cdot \cbr{\frac{m_2 - m_1}{m_1 + m_2} + 1} = m_1 \cdot g \cdot \frac{2m_2}{m_1 + m_2} = \\
            &= \frac{2 m_2 m_1 g}{m_1 + m_2} = \frac{2 \cdot 6\,\text{кг} \cdot 8\,\text{кг} \cdot 10\,\frac{\text{м}}{\text{с}^{2}}}{8\,\text{кг} + 6\,\text{кг}} \approx 68{,}6\,\text{Н}.
    \end{align*}
    Отрицательный ответ говорит, что мы лишь не угадали с направлением ускорений.
    Сила же всегда положительна.
}
\solutionspace{80pt}

\tasknumber{7}%
\task{%
    Тело массой $2\,\text{кг}$ лежит на горизонтальной поверхности.
    Коэффициент трения между поверхностью и телом $0{,}2$.
    К телу приложена горизонтальная сила $3{,}5\,\text{Н}$.
    Определите силу трения, действующую на тело, и ускорение тела.
    % $g = 10\,\frac{\text{м}}{\text{с}^{2}}$.
}
\answer{%
    \begin{align*}
    &F_\text{ трения покоя $\max$ } = \mu N = \mu m g = 0{,}2 \cdot 2\,\text{кг} \cdot 10\,\frac{\text{м}}{\text{с}^{2}} = 4{,}00\,\text{Н}, \\
    &F_\text{ трения покоя $\max$ } > F \implies F_\text{ трения } = 3{,}50\,\text{Н}, a = \frac{F - F_\text{ трения }}{ m } = 0\,\frac{\text{м}}{\text{c}^{2}}, \\
    &\text{при равенстве возможны оба варианта: и едет, и не едет, но на ответы это не влияет.}
    \end{align*}
}
\solutionspace{120pt}

\tasknumber{8}%
\task{%
    Определите плотность неизвестного вещества, если известно, что опускании тела из него
    в керосин оно будет плавать и на треть выступать над поверхностью жидкости.
}
\answer{%
    $F_\text{Арх.} = F_\text{тяж.} \implies \rho_\text{ж.} g V_\text{погр.} = m g \implies\rho_\text{ж.} g \cbr{V -\frac V3} = \rho V g \implies \rho = \rho_\text{ж.}\cbr{1 -\frac 13} \approx 533\,\frac{\text{кг}}{\text{м}^{3}}$
}
\solutionspace{120pt}

\tasknumber{9}%
\task{%
    	Определите силу, действующую на левую опору однородного горизонтального стержня длиной $l = 9\,\text{м}$
    	и массой $M = 5\,\text{кг}$, к которому подвешен груз массой $m = 2\,\text{кг}$ на расстоянии $4\,\text{м}$ от правого конца (см.
    рис.).

        \begin{tikzpicture}[thick]
            \draw
                (-2, -0.1) rectangle (2, 0.1)
                (-0.5, -0.1) -- (-0.5, -1)
                (-0.7, -1) rectangle (-0.3, -1.3)
           		(-2, -0.1) -- +(0.15,-0.9) -- +(-0.15,-0.9) -- cycle
            	(2, -0.1) -- +(0.15,-0.9) -- +(-0.15,-0.9) -- cycle
            ;
            \draw[pattern={Lines[angle=51,distance=2pt]},pattern color=black,draw=none]
            	(-2.15, -1.15) rectangle +(0.3, 0.15)
            	(2.15, -1.15) rectangle +(-0.3, 0.15)
            ;
            \node [right] (m_small) at (-0.3, -1.15) { $m$ };
            \node [above] (M_big) at (0, 0.1) { $M$ };
        \end{tikzpicture}
}
\answer{%
    \begin{align*}
        &\begin{cases}
            F_1 + F_2 - mg - Mg= 0, \\
            F_1 \cdot 0 - mg \cdot a - Mg \cdot \frac l2 + F_2 \cdot l = 0,
        \end{cases} \\
        F_2 &= \frac{mga + Mg\frac l2}l = \frac al \cdot mg + \frac{Mg}2 \approx 36{,}1\,\text{Н}, \\
        F_1 &= mg + Mg - F_2 = mg + Mg - \frac al \cdot mg - \frac{Mg}2 = \frac bl \cdot mg + \frac{Mg}2 \approx 33{,}9\,\text{Н}.
    \end{align*}
}
\solutionspace{80pt}

\tasknumber{10}%
\task{%
    Тонкий однородный лом длиной $1\,\text{м}$ и массой $20\,\text{кг}$ лежит на горизонтальной поверхности.
    \begin{itemize}
        \item Какую минимальную силу надо приложить к одному из его концов, чтобы оторвать его от этой поверхности?
        \item Какую минимальную работу надо совершить, чтобы поставить его на землю в вертикальное положение?
    \end{itemize}
    % Примите $g = 10\,\frac{\text{м}}{\text{с}^{2}}$.
}
\answer{%
    $F = \frac{mg}2 \approx 200\,\text{Н}, A = mg\frac l2 = 100\,\text{Дж}$
}
\solutionspace{120pt}

\tasknumber{11}%
\task{%
    Определите работу силы, которая обеспечит спуск тела массой $2\,\text{кг}$ на высоту $2\,\text{м}$ с постоянным ускорением $4\,\frac{\text{м}}{\text{c}^{2}}$.
    % Примите $g = 10\,\frac{\text{м}}{\text{с}^{2}}$.
}
\answer{%
    \begin{align*}
    &\text{Для подъёма:} A = Fh = (mg + ma) h = m(g+a)h, \\
    &\text{Для спуска:} A = -Fh = -(mg - ma) h = -m(g-a)h, \\
    &\text{В результате получаем:} -24\,\text{Дж}.
    \end{align*}
}
\solutionspace{60pt}

\tasknumber{12}%
\task{%
    Тело бросили вертикально вверх со скоростью $10\,\frac{\text{м}}{\text{c}}$.
    На какой высоте кинетическая энергия тела составит треть от потенциальной?
}
\answer{%
    \begin{align*}
    &0 + \frac{mv_0^2}2 = E_p + E_k, E_k = \frac 13 E_p \implies \\
    &\implies \frac{mv_0^2}2 = E_p + \frac 13 E_p = E_p\cbr{1 + \frac 13} = mgh\cbr{1 + \frac 13} \implies \\
    &\implies h = \frac{\frac{mv_0^2}2}{mg\cbr{1 + \frac 13}} = \frac{v_0^2}{2g} \cdot \frac 1{1 + \frac 13} \approx 3{,}8\,\text{м}.
    \end{align*}
}
\solutionspace{100pt}

\tasknumber{13}%
\task{%
    Плотность воздуха при нормальных условиях равна $1{,}3\,\frac{\text{кг}}{\text{м}^{3}}$.
    Чему равна плотность воздуха
    при температуре $150\celsius$ и давлении $120\,\text{кПа}$?
}
\answer{%
    \begin{align*}
    &\text{В общем случае:} PV = \frac m{\mu} RT \implies \rho = \frac mV = \frac m{\frac{\frac m{\mu} RT}P} = \frac{P\mu}{RT}, \\
    &\text{У нас 2 состояния:} \rho_1 = \frac{P_1\mu}{RT_1}, \rho_2 = \frac{P_2\mu}{RT_2} \implies \frac{\rho_2}{\rho_1} = \frac{\frac{P_2\mu}{RT_2}}{\frac{P_1\mu}{RT_1}} = \frac{P_2T_1}{P_1T_2} \implies \\
    &\implies \rho_2 = \rho_1 \cdot  \frac{P_2T_1}{P_1T_2} = 1{,}3\,\frac{\text{кг}}{\text{м}^{3}} \cdot \frac{120\,\text{кПа} \cdot 273\units{К}}{100\,\text{кПа} \cdot 423\units{К}} \approx 1{,}01\,\frac{\text{кг}}{\text{м}^{3}}.
    \end{align*}
}
\solutionspace{120pt}

\tasknumber{14}%
\task{%
    Небольшую цилиндрическую пробирку с воздухом погружают на некоторую глубину в глубокое пресное озеро,
    после чего воздух занимает в ней лишь третью часть от общего объема.
    Определите глубину, на которую погрузили пробирку.
    Температуру считать постоянной $T = 291\,\text{К}$, давлением паров воды пренебречь,
    атмосферное давление принять равным $p_{\text{aтм}} = 100\,\text{кПа}$.
}
\answer{%
    \begin{align*}
    T\text{— const} &\implies P_1V_1 = \nu RT = P_2V_2.
    \\
    V_2 = \frac 13 V_1 &\implies P_1V_1 = P_2 \cdot \frac 13V_1 \implies P_2 = 3P_1 = 3p_{\text{aтм}}.
    \\
    P_2 = p_{\text{aтм}} + \rho_{\text{в}} g h \implies h = \frac{P_2 - p_{\text{aтм}}}{\rho_{\text{в}} g} &= \frac{3p_{\text{aтм}} - p_{\text{aтм}}}{\rho_{\text{в}} g} = \frac{2 \cdot p_{\text{aтм}}}{\rho_{\text{в}} g} =  \\
     &= \frac{2 \cdot 100\,\text{кПа}}{1000\,\frac{\text{кг}}{\text{м}^{3}} \cdot  10\,\frac{\text{м}}{\text{с}^{2}}} \approx 20\,\text{м}.
    \end{align*}
}
\solutionspace{120pt}

\tasknumber{15}%
\task{%
    Газу сообщили некоторое количество теплоты,
    при этом половину его он потратил на совершение работы,
    одновременно увеличив свою внутреннюю энергию на $1200\,\text{Дж}$.
    Определите количество теплоты, сообщённое газу.
}
\answer{%
    \begin{align*}
    Q &= A' + \Delta U, A' = \frac 12 Q \implies Q \cdot \cbr{1 - \frac 12} = \Delta U \implies Q = \frac{\Delta U}{1 - \frac 12} = \frac{ 1200\,\text{Дж} }{1 - \frac 12} \approx 2400\,\text{Дж}.
    \\
    A' &= \frac 12 Q
        = \frac 12 \cdot \frac{\Delta U}{1 - \frac 12}
        = \frac{\Delta U}{2 - 1}
        = \frac{ 1200\,\text{Дж} }{2 - 1} \approx 1200\,\text{Дж}.
    \end{align*}
}
\solutionspace{60pt}

\tasknumber{16}%
\task{%
    Два конденсатора ёмкостей $C_1 = 20\,\text{нФ}$ и $C_2 = 40\,\text{нФ}$ последовательно подключают
    к источнику напряжения $U = 300\,\text{В}$ (см.
    рис.).
    % Определите заряды каждого из конденсаторов.
    Определите заряд второго конденсатора.

    \begin{tikzpicture}[circuit ee IEC, semithick]
        \draw  (0, 0) to [capacitor={info={$C_1$}}] (1, 0)
                       to [capacitor={info={$C_2$}}] (2, 0)
        ;
        % \draw [-o] (0, 0) -- ++(-0.5, 0) node[left] {$-$};
        % \draw [-o] (2, 0) -- ++(0.5, 0) node[right] {$+$};
        \draw [-o] (0, 0) -- ++(-0.5, 0) node[left] {};
        \draw [-o] (2, 0) -- ++(0.5, 0) node[right] {};
    \end{tikzpicture}
}
\answer{%
    $
        Q_1
            = Q_2
            = CU
            = \frac{ U }{\frac1{C_1} + \frac1{C_2}}
            = \frac{C_1C_2U}{C_1 + C_2}
            = \frac{
                20\,\text{нФ} \cdot 40\,\text{нФ} \cdot 300\,\text{В}
            }{
                20\,\text{нФ} + 40\,\text{нФ}
            }
            = 4{,}00\,\text{мкКл}
    $
}
\solutionspace{120pt}

\tasknumber{17}%
\task{%
    В вакууме вдоль одной прямой расположены три положительных заряда так,
    что расстояние между соседними зарядами равно $r$.
    Сделайте рисунок,
    и определите силу, действующую на крайний заряд.
    Модули всех зарядов равны $q$ ($q > 0$).
}
\answer{%
    $F = \sum_i F_i = \ldots = \frac54 \frac{kq^2}{r^2}.$
}
\solutionspace{80pt}

\tasknumber{18}%
\task{%
    Юлия проводит эксперименты c 2 кусками одинаковой алюминиевой проволки, причём второй кусок в шесть раз длиннее первого.
    В одном из экспериментов Юлия подаёт на первый кусок проволки напряжение в пять раз раз больше, чем на второй.
    Определите отношения в двух проволках в этом эксперименте (второй к первой):
    \begin{itemize}
        \item отношение сил тока,
        \item отношение выделяющихся мощностей.
    \end{itemize}
}
\answer{%
    $R_2 = 6R_1, U_1 = 5U_2 \implies  \eli_2 / \eli_1 = \frac{U_2 / R_2}{U_1 / R_1} = \frac{U_2}{U_1} \cdot \frac{R_1}{R_2} = \frac1{30}, P_2 / P_1 = \frac{U_2^2 / R_2}{U_1^2 / R_1} = \sqr{\frac{U_2}{U_1}} \cdot \frac{R_1}{R_2} = \frac1{150}.$
}

\variantsplitter

\addpersonalvariant{Матвей Кузьмин}

\tasknumber{1}%
\task{%
    Саша стартует на лошади и в течение $t = 10\,\text{c}$ двигается с постоянным ускорением $1{,}5\,\frac{\text{м}}{\text{с}^{2}}$.
    Определите
    \begin{itemize}
        \item какую скорость при этом удастся достичь,
        \item какой путь за это время будет пройден,
        \item среднюю скорость за всё время движения, если после начального ускорения продолжить движение равномерно ещё в течение времени $3t$
    \end{itemize}
}
\answer{%
    \begin{align*}
    v &= v_0 + a t = at = 1{,}5\,\frac{\text{м}}{\text{с}^{2}} \cdot 10\,\text{c} = 15{,}0\,\frac{\text{м}}{\text{с}}, \\
    s_x &= v_0t + \frac{a t^2}2 = \frac{a t^2}2 = \frac{1{,}5\,\frac{\text{м}}{\text{с}^{2}} \cdot \sqr{ 10\,\text{c} }}2 = 75{,}0\,\text{м}, \\
    v_\text{сред.} &= \frac{s_\text{общ}}{t_\text{общ.}} = \frac{s_x + v \cdot 3t}{t + 3t} = \frac{\frac{a t^2}2 + at \cdot 3t}{t (1 + 3)} = \\
    &= at \cdot \frac{\frac 12 + 3}{1 + 3} = 1{,}5\,\frac{\text{м}}{\text{с}^{2}} \cdot 10\,\text{c} \cdot \frac{\frac 12 + 3}{1 + 3} \approx 13{,}12\,\frac{\text{м}}{\text{c}}.
    \end{align*}
}
\solutionspace{120pt}

\tasknumber{2}%
\task{%
    Какой путь тело пройдёт за вторую секунду после начала свободного падения?
    Какую скорость в конце этой секунды оно имеет?
}
\answer{%
    \begin{align*}
    s &= -s_y = -(y_2-y_1) = y_1 - y_2 = \cbr{y_{0y} + v_{0y}t_1 - \frac{gt_1^2}2} - \cbr{y_{0y} + v_{0y}t_2 - \frac{gt_2^2}2} = \\
    &= \frac{gt_2^2}2 - \frac{gt_1^2}2 = \frac g2\cbr{t_2^2 - t_1^2} = 15{,}0\,\text{м}, \\
    v_y &= v_{0y} - gt = -gt = 10\,\frac{\text{м}}{\text{с}^{2}} \cdot 2\,\text{с} = -20\,\frac{\text{м}}{\text{с}}.
    \end{align*}
}
\solutionspace{120pt}

\tasknumber{3}%
\task{%
    Карусель диаметром $3\,\text{м}$ равномерно совершает 10 оборотов в минуту.
    Определите
    \begin{itemize}
        \item период и частоту её обращения,
        \item скорость и ускорение крайних её точек.
    \end{itemize}
}
\answer{%
    \begin{align*}
    t &= 60\,\text{с}, r = 1{,}5\,\text{м}, n = 10\units{оборотов}, \\
    T &= \frac tN = \frac{ 60\,\text{с} }{10} \approx 6{,}00\,\text{c}, \\
    \nu &= \frac 1T = \frac{10}{ 60\,\text{с} } \approx 0{,}17\,\text{Гц}, \\
    v &= \frac{2 \pi r}{T} = \frac{2 \pi r}{T} =  \frac{2 \pi r n}{t} \approx 1{,}57\,\frac{\text{м}}{\text{c}}, \\
    a &= \frac{v^2}{r} =  \frac{4 \pi^2 r n^2}{t^2} \approx 1{,}64\,\frac{\text{м}}{\text{с}^{2}}.
    \end{align*}
}
\solutionspace{80pt}

\tasknumber{4}%
\task{%
    Даша стоит на обрыве над рекой и методично и строго горизонтально кидает в неё камушки.
    За этим всем наблюдает экспериментатор Глюк, который уже выяснил, что камушки падают в реку спустя $1{,}6\,\text{с}$ после броска,
    а вот дальность полёта оценить сложнее: придётся лезть в воду.
    Выручите Глюка и определите:
    \begin{itemize}
        \item высоту обрыва (вместе с ростом Даши).
        \item дальность полёта камушков (по горизонтали) и их скорость при падении, приняв начальную скорость броска равной $v_0 = 13\,\frac{\text{м}}{\text{с}}$.
    \end{itemize}
    Сопротивлением воздуха пренебречь.
}
\answer{%
    \begin{align*}
    y &= y_0 + v_{0y}t - \frac{gt^2}2 = h - \frac{gt^2}2, \qquad y(\tau) = 0 \implies h - \frac{g\tau^2}2 = 0 \implies h = \frac{g\tau^2}2 \approx 12{,}8\,\text{м}.
    \\
    x &= x_0 + v_{0x}t = v_0t \implies L = v_0\tau \approx 20{,}8\,\text{м}.
    \\
    &v = \sqrt{v_x^2 + v_y^2} = \sqrt{v_{0x}^2 + \sqr{v_{0y} - g\tau}} = \sqrt{v_0^2 + \sqr{g\tau}} \approx 20{,}6\,\frac{\text{м}}{\text{c}}.
    \end{align*}
}
\solutionspace{120pt}

\tasknumber{5}%
\task{%
    Пять одинаковых брусков массой $2\,\text{кг}$ каждый лежат на гладком горизонтальном столе.
    Бруски пронумерованы от 1 до 5 и последовательно связаны между собой
    невесомыми нерастяжимыми нитями: 1 со 2, 2 с 3 (ну и с 1) и т.д.
    Экспериментатор Глюк прикладывает постоянную горизонтальную силу $120\,\text{Н}$ к бруску с наибольшим номером.
    С каким ускорением двигается система? Чему равна сила натяжения нити, связывающей бруски 2 и 3?
}
\answer{%
    \begin{align*}
    a &= \frac{F}{5 m} = \frac{120\,\text{Н}}{5 \cdot 2\,\text{кг}} \approx 12{,}0\,\frac{\text{м}}{\text{c}^{2}}, \\
    T &= m'a = 2m \cdot \frac{F}{5 m} = \frac{2}{5} F \approx 48{,}0\,\text{Н}.
    \end{align*}
}
\solutionspace{120pt}

\tasknumber{6}%
\task{%
    Два бруска связаны лёгкой нерастяжимой нитью и перекинуты через неподвижный блок (см.
    рис.).
    Определите силу натяжения нити и ускорения брусков.
    Силами трения пренебречь, массы брусков
    равны $m_1 = 11\,\text{кг}$ и $m_2 = 10\,\text{кг}$.
    % $g = 10\,\frac{\text{м}}{\text{с}^{2}}$.

    \begin{tikzpicture}[x=1.5cm,y=1.5cm,thick]
        \draw
            (-0.4, 0) rectangle (-0.2, 1.2)
            (0.15, 0.5) rectangle (0.45, 1)
            (0, 2) circle [radius=0.3] -- ++(up:0.5)
            (-0.3, 1.2) -- ++(up:0.8)
            (0.3, 1) -- ++(up:1)
            (-0.7, 2.5) -- (0.7, 2.5)
            ;
        \draw[pattern={Lines[angle=51,distance=3pt]},pattern color=black,draw=none] (-0.7, 2.5) rectangle (0.7, 2.75);
        \node [left] (left) at (-0.4, 0.6) { $m_1$ };
        \node [right] (right) at (0.4, 0.75) { $m_2$ };
    \end{tikzpicture}
}
\answer{%
    Предположим, что левый брусок ускоряется вверх, тогда правый ускоряется вниз (с тем же ускорением).
    Запишем 2-й закон Ньютона 2 раза (для обоих тел) в проекции на вертикальную оси, направив её вверх.
    \begin{align*}
        &\begin{cases}
            T - m_1g = m_1a, \\
            T - m_2g = -m_2a,
        \end{cases} \\
        &\begin{cases}
            m_2g - m_1g = m_1a + m_2a, \\
            T = m_1a + m_1g, \\
        \end{cases} \\
        a &= \frac{m_2 - m_1}{m_1 + m_2} \cdot g = \frac{10\,\text{кг} - 11\,\text{кг}}{11\,\text{кг} + 10\,\text{кг}} \cdot 10\,\frac{\text{м}}{\text{с}^{2}} \approx -0{,}4800\,\frac{\text{м}}{\text{c}^{2}}, \\
        T &= m_1(a + g) = m_1 \cdot g \cdot \cbr{\frac{m_2 - m_1}{m_1 + m_2} + 1} = m_1 \cdot g \cdot \frac{2m_2}{m_1 + m_2} = \\
            &= \frac{2 m_2 m_1 g}{m_1 + m_2} = \frac{2 \cdot 10\,\text{кг} \cdot 11\,\text{кг} \cdot 10\,\frac{\text{м}}{\text{с}^{2}}}{11\,\text{кг} + 10\,\text{кг}} \approx 104{,}8\,\text{Н}.
    \end{align*}
    Отрицательный ответ говорит, что мы лишь не угадали с направлением ускорений.
    Сила же всегда положительна.
}
\solutionspace{80pt}

\tasknumber{7}%
\task{%
    Тело массой $2{,}7\,\text{кг}$ лежит на горизонтальной поверхности.
    Коэффициент трения между поверхностью и телом $0{,}25$.
    К телу приложена горизонтальная сила $4{,}5\,\text{Н}$.
    Определите силу трения, действующую на тело, и ускорение тела.
    % $g = 10\,\frac{\text{м}}{\text{с}^{2}}$.
}
\answer{%
    \begin{align*}
    &F_\text{ трения покоя $\max$ } = \mu N = \mu m g = 0{,}25 \cdot 2{,}7\,\text{кг} \cdot 10\,\frac{\text{м}}{\text{с}^{2}} = 6{,}75\,\text{Н}, \\
    &F_\text{ трения покоя $\max$ } > F \implies F_\text{ трения } = 4{,}50\,\text{Н}, a = \frac{F - F_\text{ трения }}{ m } = 0\,\frac{\text{м}}{\text{c}^{2}}, \\
    &\text{при равенстве возможны оба варианта: и едет, и не едет, но на ответы это не влияет.}
    \end{align*}
}
\solutionspace{120pt}

\tasknumber{8}%
\task{%
    Определите плотность неизвестного вещества, если известно, что опускании тела из него
    в керосин оно будет плавать и на половину выступать над поверхностью жидкости.
}
\answer{%
    $F_\text{Арх.} = F_\text{тяж.} \implies \rho_\text{ж.} g V_\text{погр.} = m g \implies\rho_\text{ж.} g \cbr{V -\frac V2} = \rho V g \implies \rho = \rho_\text{ж.}\cbr{1 -\frac 12} \approx 400\,\frac{\text{кг}}{\text{м}^{3}}$
}
\solutionspace{120pt}

\tasknumber{9}%
\task{%
    	Определите силу, действующую на правую опору однородного горизонтального стержня длиной $l = 9\,\text{м}$
    	и массой $M = 1\,\text{кг}$, к которому подвешен груз массой $m = 2\,\text{кг}$ на расстоянии $4\,\text{м}$ от правого конца (см.
    рис.).

        \begin{tikzpicture}[thick]
            \draw
                (-2, -0.1) rectangle (2, 0.1)
                (-0.5, -0.1) -- (-0.5, -1)
                (-0.7, -1) rectangle (-0.3, -1.3)
           		(-2, -0.1) -- +(0.15,-0.9) -- +(-0.15,-0.9) -- cycle
            	(2, -0.1) -- +(0.15,-0.9) -- +(-0.15,-0.9) -- cycle
            ;
            \draw[pattern={Lines[angle=51,distance=2pt]},pattern color=black,draw=none]
            	(-2.15, -1.15) rectangle +(0.3, 0.15)
            	(2.15, -1.15) rectangle +(-0.3, 0.15)
            ;
            \node [right] (m_small) at (-0.3, -1.15) { $m$ };
            \node [above] (M_big) at (0, 0.1) { $M$ };
        \end{tikzpicture}
}
\answer{%
    \begin{align*}
        &\begin{cases}
            F_1 + F_2 - mg - Mg= 0, \\
            F_1 \cdot 0 - mg \cdot a - Mg \cdot \frac l2 + F_2 \cdot l = 0,
        \end{cases} \\
        F_2 &= \frac{mga + Mg\frac l2}l = \frac al \cdot mg + \frac{Mg}2 \approx 16{,}1\,\text{Н}, \\
        F_1 &= mg + Mg - F_2 = mg + Mg - \frac al \cdot mg - \frac{Mg}2 = \frac bl \cdot mg + \frac{Mg}2 \approx 13{,}9\,\text{Н}.
    \end{align*}
}
\solutionspace{80pt}

\tasknumber{10}%
\task{%
    Тонкий однородный шест длиной $2\,\text{м}$ и массой $10\,\text{кг}$ лежит на горизонтальной поверхности.
    \begin{itemize}
        \item Какую минимальную силу надо приложить к одному из его концов, чтобы оторвать его от этой поверхности?
        \item Какую минимальную работу надо совершить, чтобы поставить его на землю в вертикальное положение?
    \end{itemize}
    % Примите $g = 10\,\frac{\text{м}}{\text{с}^{2}}$.
}
\answer{%
    $F = \frac{mg}2 \approx 100\,\text{Н}, A = mg\frac l2 = 100\,\text{Дж}$
}
\solutionspace{120pt}

\tasknumber{11}%
\task{%
    Определите работу силы, которая обеспечит спуск тела массой $5\,\text{кг}$ на высоту $5\,\text{м}$ с постоянным ускорением $4\,\frac{\text{м}}{\text{c}^{2}}$.
    % Примите $g = 10\,\frac{\text{м}}{\text{с}^{2}}$.
}
\answer{%
    \begin{align*}
    &\text{Для подъёма:} A = Fh = (mg + ma) h = m(g+a)h, \\
    &\text{Для спуска:} A = -Fh = -(mg - ma) h = -m(g-a)h, \\
    &\text{В результате получаем:} -150\,\text{Дж}.
    \end{align*}
}
\solutionspace{60pt}

\tasknumber{12}%
\task{%
    Тело бросили вертикально вверх со скоростью $10\,\frac{\text{м}}{\text{c}}$.
    На какой высоте кинетическая энергия тела составит треть от потенциальной?
}
\answer{%
    \begin{align*}
    &0 + \frac{mv_0^2}2 = E_p + E_k, E_k = \frac 13 E_p \implies \\
    &\implies \frac{mv_0^2}2 = E_p + \frac 13 E_p = E_p\cbr{1 + \frac 13} = mgh\cbr{1 + \frac 13} \implies \\
    &\implies h = \frac{\frac{mv_0^2}2}{mg\cbr{1 + \frac 13}} = \frac{v_0^2}{2g} \cdot \frac 1{1 + \frac 13} \approx 3{,}8\,\text{м}.
    \end{align*}
}
\solutionspace{100pt}

\tasknumber{13}%
\task{%
    Плотность воздуха при нормальных условиях равна $1{,}3\,\frac{\text{кг}}{\text{м}^{3}}$.
    Чему равна плотность воздуха
    при температуре $50\celsius$ и давлении $80\,\text{кПа}$?
}
\answer{%
    \begin{align*}
    &\text{В общем случае:} PV = \frac m{\mu} RT \implies \rho = \frac mV = \frac m{\frac{\frac m{\mu} RT}P} = \frac{P\mu}{RT}, \\
    &\text{У нас 2 состояния:} \rho_1 = \frac{P_1\mu}{RT_1}, \rho_2 = \frac{P_2\mu}{RT_2} \implies \frac{\rho_2}{\rho_1} = \frac{\frac{P_2\mu}{RT_2}}{\frac{P_1\mu}{RT_1}} = \frac{P_2T_1}{P_1T_2} \implies \\
    &\implies \rho_2 = \rho_1 \cdot  \frac{P_2T_1}{P_1T_2} = 1{,}3\,\frac{\text{кг}}{\text{м}^{3}} \cdot \frac{80\,\text{кПа} \cdot 273\units{К}}{100\,\text{кПа} \cdot 323\units{К}} \approx 0{,}88\,\frac{\text{кг}}{\text{м}^{3}}.
    \end{align*}
}
\solutionspace{120pt}

\tasknumber{14}%
\task{%
    Небольшую цилиндрическую пробирку с воздухом погружают на некоторую глубину в глубокое пресное озеро,
    после чего воздух занимает в ней лишь третью часть от общего объема.
    Определите глубину, на которую погрузили пробирку.
    Температуру считать постоянной $T = 287\,\text{К}$, давлением паров воды пренебречь,
    атмосферное давление принять равным $p_{\text{aтм}} = 100\,\text{кПа}$.
}
\answer{%
    \begin{align*}
    T\text{— const} &\implies P_1V_1 = \nu RT = P_2V_2.
    \\
    V_2 = \frac 13 V_1 &\implies P_1V_1 = P_2 \cdot \frac 13V_1 \implies P_2 = 3P_1 = 3p_{\text{aтм}}.
    \\
    P_2 = p_{\text{aтм}} + \rho_{\text{в}} g h \implies h = \frac{P_2 - p_{\text{aтм}}}{\rho_{\text{в}} g} &= \frac{3p_{\text{aтм}} - p_{\text{aтм}}}{\rho_{\text{в}} g} = \frac{2 \cdot p_{\text{aтм}}}{\rho_{\text{в}} g} =  \\
     &= \frac{2 \cdot 100\,\text{кПа}}{1000\,\frac{\text{кг}}{\text{м}^{3}} \cdot  10\,\frac{\text{м}}{\text{с}^{2}}} \approx 20\,\text{м}.
    \end{align*}
}
\solutionspace{120pt}

\tasknumber{15}%
\task{%
    Газу сообщили некоторое количество теплоты,
    при этом половину его он потратил на совершение работы,
    одновременно увеличив свою внутреннюю энергию на $1200\,\text{Дж}$.
    Определите количество теплоты, сообщённое газу.
}
\answer{%
    \begin{align*}
    Q &= A' + \Delta U, A' = \frac 12 Q \implies Q \cdot \cbr{1 - \frac 12} = \Delta U \implies Q = \frac{\Delta U}{1 - \frac 12} = \frac{ 1200\,\text{Дж} }{1 - \frac 12} \approx 2400\,\text{Дж}.
    \\
    A' &= \frac 12 Q
        = \frac 12 \cdot \frac{\Delta U}{1 - \frac 12}
        = \frac{\Delta U}{2 - 1}
        = \frac{ 1200\,\text{Дж} }{2 - 1} \approx 1200\,\text{Дж}.
    \end{align*}
}
\solutionspace{60pt}

\tasknumber{16}%
\task{%
    Два конденсатора ёмкостей $C_1 = 30\,\text{нФ}$ и $C_2 = 60\,\text{нФ}$ последовательно подключают
    к источнику напряжения $U = 450\,\text{В}$ (см.
    рис.).
    % Определите заряды каждого из конденсаторов.
    Определите заряд первого конденсатора.

    \begin{tikzpicture}[circuit ee IEC, semithick]
        \draw  (0, 0) to [capacitor={info={$C_1$}}] (1, 0)
                       to [capacitor={info={$C_2$}}] (2, 0)
        ;
        % \draw [-o] (0, 0) -- ++(-0.5, 0) node[left] {$-$};
        % \draw [-o] (2, 0) -- ++(0.5, 0) node[right] {$+$};
        \draw [-o] (0, 0) -- ++(-0.5, 0) node[left] {};
        \draw [-o] (2, 0) -- ++(0.5, 0) node[right] {};
    \end{tikzpicture}
}
\answer{%
    $
        Q_1
            = Q_2
            = CU
            = \frac{ U }{\frac1{C_1} + \frac1{C_2}}
            = \frac{C_1C_2U}{C_1 + C_2}
            = \frac{
                30\,\text{нФ} \cdot 60\,\text{нФ} \cdot 450\,\text{В}
            }{
                30\,\text{нФ} + 60\,\text{нФ}
            }
            = 9{,}00\,\text{мкКл}
    $
}
\solutionspace{120pt}

\tasknumber{17}%
\task{%
    В вакууме вдоль одной прямой расположены четыре положительных заряда так,
    что расстояние между соседними зарядами равно $r$.
    Сделайте рисунок,
    и определите силу, действующую на крайний заряд.
    Модули всех зарядов равны $q$ ($q > 0$).
}
\answer{%
    $F = \sum_i F_i = \ldots = \frac{49}{36} \frac{kq^2}{r^2}.$
}
\solutionspace{80pt}

\tasknumber{18}%
\task{%
    Юлия проводит эксперименты c 2 кусками одинаковой стальной проволки, причём второй кусок в семь раз длиннее первого.
    В одном из экспериментов Юлия подаёт на первый кусок проволки напряжение в два раза раз больше, чем на второй.
    Определите отношения в двух проволках в этом эксперименте (второй к первой):
    \begin{itemize}
        \item отношение сил тока,
        \item отношение выделяющихся мощностей.
    \end{itemize}
}
\answer{%
    $R_2 = 7R_1, U_1 = 2U_2 \implies  \eli_2 / \eli_1 = \frac{U_2 / R_2}{U_1 / R_1} = \frac{U_2}{U_1} \cdot \frac{R_1}{R_2} = \frac1{14}, P_2 / P_1 = \frac{U_2^2 / R_2}{U_1^2 / R_1} = \sqr{\frac{U_2}{U_1}} \cdot \frac{R_1}{R_2} = \frac1{28}.$
}

\variantsplitter

\addpersonalvariant{Сергей Малышев}

\tasknumber{1}%
\task{%
    Валя стартует на лошади и в течение $t = 4\,\text{c}$ двигается с постоянным ускорением $2{,}5\,\frac{\text{м}}{\text{с}^{2}}$.
    Определите
    \begin{itemize}
        \item какую скорость при этом удастся достичь,
        \item какой путь за это время будет пройден,
        \item среднюю скорость за всё время движения, если после начального ускорения продолжить движение равномерно ещё в течение времени $3t$
    \end{itemize}
}
\answer{%
    \begin{align*}
    v &= v_0 + a t = at = 2{,}5\,\frac{\text{м}}{\text{с}^{2}} \cdot 4\,\text{c} = 10{,}0\,\frac{\text{м}}{\text{с}}, \\
    s_x &= v_0t + \frac{a t^2}2 = \frac{a t^2}2 = \frac{2{,}5\,\frac{\text{м}}{\text{с}^{2}} \cdot \sqr{ 4\,\text{c} }}2 = 20{,}0\,\text{м}, \\
    v_\text{сред.} &= \frac{s_\text{общ}}{t_\text{общ.}} = \frac{s_x + v \cdot 3t}{t + 3t} = \frac{\frac{a t^2}2 + at \cdot 3t}{t (1 + 3)} = \\
    &= at \cdot \frac{\frac 12 + 3}{1 + 3} = 2{,}5\,\frac{\text{м}}{\text{с}^{2}} \cdot 4\,\text{c} \cdot \frac{\frac 12 + 3}{1 + 3} \approx 8{,}75\,\frac{\text{м}}{\text{c}}.
    \end{align*}
}
\solutionspace{120pt}

\tasknumber{2}%
\task{%
    Какой путь тело пройдёт за пятую секунду после начала свободного падения?
    Какую скорость в начале этой секунды оно имеет?
}
\answer{%
    \begin{align*}
    s &= -s_y = -(y_2-y_1) = y_1 - y_2 = \cbr{y_{0y} + v_{0y}t_1 - \frac{gt_1^2}2} - \cbr{y_{0y} + v_{0y}t_2 - \frac{gt_2^2}2} = \\
    &= \frac{gt_2^2}2 - \frac{gt_1^2}2 = \frac g2\cbr{t_2^2 - t_1^2} = 45{,}0\,\text{м}, \\
    v_y &= v_{0y} - gt = -gt = 10\,\frac{\text{м}}{\text{с}^{2}} \cdot 4\,\text{с} = -40\,\frac{\text{м}}{\text{с}}.
    \end{align*}
}
\solutionspace{120pt}

\tasknumber{3}%
\task{%
    Карусель диаметром $2\,\text{м}$ равномерно совершает 6 оборотов в минуту.
    Определите
    \begin{itemize}
        \item период и частоту её обращения,
        \item скорость и ускорение крайних её точек.
    \end{itemize}
}
\answer{%
    \begin{align*}
    t &= 60\,\text{с}, r = 1{,}0\,\text{м}, n = 6\units{оборотов}, \\
    T &= \frac tN = \frac{ 60\,\text{с} }{6} \approx 10{,}00\,\text{c}, \\
    \nu &= \frac 1T = \frac{6}{ 60\,\text{с} } \approx 0{,}10\,\text{Гц}, \\
    v &= \frac{2 \pi r}{T} = \frac{2 \pi r}{T} =  \frac{2 \pi r n}{t} \approx 0{,}63\,\frac{\text{м}}{\text{c}}, \\
    a &= \frac{v^2}{r} =  \frac{4 \pi^2 r n^2}{t^2} \approx 0{,}39\,\frac{\text{м}}{\text{с}^{2}}.
    \end{align*}
}
\solutionspace{80pt}

\tasknumber{4}%
\task{%
    Даша стоит на обрыве над рекой и методично и строго горизонтально кидает в неё камушки.
    За этим всем наблюдает экспериментатор Глюк, который уже выяснил, что камушки падают в реку спустя $1{,}6\,\text{с}$ после броска,
    а вот дальность полёта оценить сложнее: придётся лезть в воду.
    Выручите Глюка и определите:
    \begin{itemize}
        \item высоту обрыва (вместе с ростом Даши).
        \item дальность полёта камушков (по горизонтали) и их скорость при падении, приняв начальную скорость броска равной $v_0 = 14\,\frac{\text{м}}{\text{с}}$.
    \end{itemize}
    Сопротивлением воздуха пренебречь.
}
\answer{%
    \begin{align*}
    y &= y_0 + v_{0y}t - \frac{gt^2}2 = h - \frac{gt^2}2, \qquad y(\tau) = 0 \implies h - \frac{g\tau^2}2 = 0 \implies h = \frac{g\tau^2}2 \approx 12{,}8\,\text{м}.
    \\
    x &= x_0 + v_{0x}t = v_0t \implies L = v_0\tau \approx 22{,}4\,\text{м}.
    \\
    &v = \sqrt{v_x^2 + v_y^2} = \sqrt{v_{0x}^2 + \sqr{v_{0y} - g\tau}} = \sqrt{v_0^2 + \sqr{g\tau}} \approx 21{,}3\,\frac{\text{м}}{\text{c}}.
    \end{align*}
}
\solutionspace{120pt}

\tasknumber{5}%
\task{%
    Четыре одинаковых брусков массой $2\,\text{кг}$ каждый лежат на гладком горизонтальном столе.
    Бруски пронумерованы от 1 до 4 и последовательно связаны между собой
    невесомыми нерастяжимыми нитями: 1 со 2, 2 с 3 (ну и с 1) и т.д.
    Экспериментатор Глюк прикладывает постоянную горизонтальную силу $60\,\text{Н}$ к бруску с наибольшим номером.
    С каким ускорением двигается система? Чему равна сила натяжения нити, связывающей бруски 1 и 2?
}
\answer{%
    \begin{align*}
    a &= \frac{F}{4 m} = \frac{60\,\text{Н}}{4 \cdot 2\,\text{кг}} \approx 7{,}5\,\frac{\text{м}}{\text{c}^{2}}, \\
    T &= m'a = 1m \cdot \frac{F}{4 m} = \frac{1}{4} F \approx 15{,}0\,\text{Н}.
    \end{align*}
}
\solutionspace{120pt}

\tasknumber{6}%
\task{%
    Два бруска связаны лёгкой нерастяжимой нитью и перекинуты через неподвижный блок (см.
    рис.).
    Определите силу натяжения нити и ускорения брусков.
    Силами трения пренебречь, массы брусков
    равны $m_1 = 8\,\text{кг}$ и $m_2 = 14\,\text{кг}$.
    % $g = 10\,\frac{\text{м}}{\text{с}^{2}}$.

    \begin{tikzpicture}[x=1.5cm,y=1.5cm,thick]
        \draw
            (-0.4, 0) rectangle (-0.2, 1.2)
            (0.15, 0.5) rectangle (0.45, 1)
            (0, 2) circle [radius=0.3] -- ++(up:0.5)
            (-0.3, 1.2) -- ++(up:0.8)
            (0.3, 1) -- ++(up:1)
            (-0.7, 2.5) -- (0.7, 2.5)
            ;
        \draw[pattern={Lines[angle=51,distance=3pt]},pattern color=black,draw=none] (-0.7, 2.5) rectangle (0.7, 2.75);
        \node [left] (left) at (-0.4, 0.6) { $m_1$ };
        \node [right] (right) at (0.4, 0.75) { $m_2$ };
    \end{tikzpicture}
}
\answer{%
    Предположим, что левый брусок ускоряется вверх, тогда правый ускоряется вниз (с тем же ускорением).
    Запишем 2-й закон Ньютона 2 раза (для обоих тел) в проекции на вертикальную оси, направив её вверх.
    \begin{align*}
        &\begin{cases}
            T - m_1g = m_1a, \\
            T - m_2g = -m_2a,
        \end{cases} \\
        &\begin{cases}
            m_2g - m_1g = m_1a + m_2a, \\
            T = m_1a + m_1g, \\
        \end{cases} \\
        a &= \frac{m_2 - m_1}{m_1 + m_2} \cdot g = \frac{14\,\text{кг} - 8\,\text{кг}}{8\,\text{кг} + 14\,\text{кг}} \cdot 10\,\frac{\text{м}}{\text{с}^{2}} \approx 2{,}73\,\frac{\text{м}}{\text{c}^{2}}, \\
        T &= m_1(a + g) = m_1 \cdot g \cdot \cbr{\frac{m_2 - m_1}{m_1 + m_2} + 1} = m_1 \cdot g \cdot \frac{2m_2}{m_1 + m_2} = \\
            &= \frac{2 m_2 m_1 g}{m_1 + m_2} = \frac{2 \cdot 14\,\text{кг} \cdot 8\,\text{кг} \cdot 10\,\frac{\text{м}}{\text{с}^{2}}}{8\,\text{кг} + 14\,\text{кг}} \approx 101{,}8\,\text{Н}.
    \end{align*}
    Отрицательный ответ говорит, что мы лишь не угадали с направлением ускорений.
    Сила же всегда положительна.
}
\solutionspace{80pt}

\tasknumber{7}%
\task{%
    Тело массой $2\,\text{кг}$ лежит на горизонтальной поверхности.
    Коэффициент трения между поверхностью и телом $0{,}25$.
    К телу приложена горизонтальная сила $5{,}5\,\text{Н}$.
    Определите силу трения, действующую на тело, и ускорение тела.
    % $g = 10\,\frac{\text{м}}{\text{с}^{2}}$.
}
\answer{%
    \begin{align*}
    &F_\text{ трения покоя $\max$ } = \mu N = \mu m g = 0{,}25 \cdot 2\,\text{кг} \cdot 10\,\frac{\text{м}}{\text{с}^{2}} = 5{,}00\,\text{Н}, \\
    &F_\text{ трения покоя $\max$ } \le F \implies F_\text{ трения } = 5{,}00\,\text{Н}, a = \frac{F - F_\text{ трения }}{ m } = 0{,}25\,\frac{\text{м}}{\text{c}^{2}}, \\
    &\text{при равенстве возможны оба варианта: и едет, и не едет, но на ответы это не влияет.}
    \end{align*}
}
\solutionspace{120pt}

\tasknumber{8}%
\task{%
    Определите плотность неизвестного вещества, если известно, что опускании тела из него
    в подсолнечное масло оно будет плавать и на четверть выступать над поверхностью жидкости.
}
\answer{%
    $F_\text{Арх.} = F_\text{тяж.} \implies \rho_\text{ж.} g V_\text{погр.} = m g \implies\rho_\text{ж.} g \cbr{V -\frac V4} = \rho V g \implies \rho = \rho_\text{ж.}\cbr{1 -\frac 14} \approx 675\,\frac{\text{кг}}{\text{м}^{3}}$
}
\solutionspace{120pt}

\tasknumber{9}%
\task{%
    	Определите силу, действующую на левую опору однородного горизонтального стержня длиной $l = 5\,\text{м}$
    	и массой $M = 1\,\text{кг}$, к которому подвешен груз массой $m = 4\,\text{кг}$ на расстоянии $2\,\text{м}$ от правого конца (см.
    рис.).

        \begin{tikzpicture}[thick]
            \draw
                (-2, -0.1) rectangle (2, 0.1)
                (-0.5, -0.1) -- (-0.5, -1)
                (-0.7, -1) rectangle (-0.3, -1.3)
           		(-2, -0.1) -- +(0.15,-0.9) -- +(-0.15,-0.9) -- cycle
            	(2, -0.1) -- +(0.15,-0.9) -- +(-0.15,-0.9) -- cycle
            ;
            \draw[pattern={Lines[angle=51,distance=2pt]},pattern color=black,draw=none]
            	(-2.15, -1.15) rectangle +(0.3, 0.15)
            	(2.15, -1.15) rectangle +(-0.3, 0.15)
            ;
            \node [right] (m_small) at (-0.3, -1.15) { $m$ };
            \node [above] (M_big) at (0, 0.1) { $M$ };
        \end{tikzpicture}
}
\answer{%
    \begin{align*}
        &\begin{cases}
            F_1 + F_2 - mg - Mg= 0, \\
            F_1 \cdot 0 - mg \cdot a - Mg \cdot \frac l2 + F_2 \cdot l = 0,
        \end{cases} \\
        F_2 &= \frac{mga + Mg\frac l2}l = \frac al \cdot mg + \frac{Mg}2 \approx 29{,}0\,\text{Н}, \\
        F_1 &= mg + Mg - F_2 = mg + Mg - \frac al \cdot mg - \frac{Mg}2 = \frac bl \cdot mg + \frac{Mg}2 \approx 21{,}0\,\text{Н}.
    \end{align*}
}
\solutionspace{80pt}

\tasknumber{10}%
\task{%
    Тонкий однородный кусок арматуры длиной $2\,\text{м}$ и массой $10\,\text{кг}$ лежит на горизонтальной поверхности.
    \begin{itemize}
        \item Какую минимальную силу надо приложить к одному из его концов, чтобы оторвать его от этой поверхности?
        \item Какую минимальную работу надо совершить, чтобы поставить его на землю в вертикальное положение?
    \end{itemize}
    % Примите $g = 10\,\frac{\text{м}}{\text{с}^{2}}$.
}
\answer{%
    $F = \frac{mg}2 \approx 100\,\text{Н}, A = mg\frac l2 = 100\,\text{Дж}$
}
\solutionspace{120pt}

\tasknumber{11}%
\task{%
    Определите работу силы, которая обеспечит спуск тела массой $3\,\text{кг}$ на высоту $2\,\text{м}$ с постоянным ускорением $4\,\frac{\text{м}}{\text{c}^{2}}$.
    % Примите $g = 10\,\frac{\text{м}}{\text{с}^{2}}$.
}
\answer{%
    \begin{align*}
    &\text{Для подъёма:} A = Fh = (mg + ma) h = m(g+a)h, \\
    &\text{Для спуска:} A = -Fh = -(mg - ma) h = -m(g-a)h, \\
    &\text{В результате получаем:} -36\,\text{Дж}.
    \end{align*}
}
\solutionspace{60pt}

\tasknumber{12}%
\task{%
    Тело бросили вертикально вверх со скоростью $10\,\frac{\text{м}}{\text{c}}$.
    На какой высоте кинетическая энергия тела составит половину от потенциальной?
}
\answer{%
    \begin{align*}
    &0 + \frac{mv_0^2}2 = E_p + E_k, E_k = \frac 12 E_p \implies \\
    &\implies \frac{mv_0^2}2 = E_p + \frac 12 E_p = E_p\cbr{1 + \frac 12} = mgh\cbr{1 + \frac 12} \implies \\
    &\implies h = \frac{\frac{mv_0^2}2}{mg\cbr{1 + \frac 12}} = \frac{v_0^2}{2g} \cdot \frac 1{1 + \frac 12} \approx 3{,}3\,\text{м}.
    \end{align*}
}
\solutionspace{100pt}

\tasknumber{13}%
\task{%
    Плотность воздуха при нормальных условиях равна $1{,}3\,\frac{\text{кг}}{\text{м}^{3}}$.
    Чему равна плотность воздуха
    при температуре $200\celsius$ и давлении $80\,\text{кПа}$?
}
\answer{%
    \begin{align*}
    &\text{В общем случае:} PV = \frac m{\mu} RT \implies \rho = \frac mV = \frac m{\frac{\frac m{\mu} RT}P} = \frac{P\mu}{RT}, \\
    &\text{У нас 2 состояния:} \rho_1 = \frac{P_1\mu}{RT_1}, \rho_2 = \frac{P_2\mu}{RT_2} \implies \frac{\rho_2}{\rho_1} = \frac{\frac{P_2\mu}{RT_2}}{\frac{P_1\mu}{RT_1}} = \frac{P_2T_1}{P_1T_2} \implies \\
    &\implies \rho_2 = \rho_1 \cdot  \frac{P_2T_1}{P_1T_2} = 1{,}3\,\frac{\text{кг}}{\text{м}^{3}} \cdot \frac{80\,\text{кПа} \cdot 273\units{К}}{100\,\text{кПа} \cdot 473\units{К}} \approx 0{,}60\,\frac{\text{кг}}{\text{м}^{3}}.
    \end{align*}
}
\solutionspace{120pt}

\tasknumber{14}%
\task{%
    Небольшую цилиндрическую пробирку с воздухом погружают на некоторую глубину в глубокое пресное озеро,
    после чего воздух занимает в ней лишь пятую часть от общего объема.
    Определите глубину, на которую погрузили пробирку.
    Температуру считать постоянной $T = 291\,\text{К}$, давлением паров воды пренебречь,
    атмосферное давление принять равным $p_{\text{aтм}} = 100\,\text{кПа}$.
}
\answer{%
    \begin{align*}
    T\text{— const} &\implies P_1V_1 = \nu RT = P_2V_2.
    \\
    V_2 = \frac 15 V_1 &\implies P_1V_1 = P_2 \cdot \frac 15V_1 \implies P_2 = 5P_1 = 5p_{\text{aтм}}.
    \\
    P_2 = p_{\text{aтм}} + \rho_{\text{в}} g h \implies h = \frac{P_2 - p_{\text{aтм}}}{\rho_{\text{в}} g} &= \frac{5p_{\text{aтм}} - p_{\text{aтм}}}{\rho_{\text{в}} g} = \frac{4 \cdot p_{\text{aтм}}}{\rho_{\text{в}} g} =  \\
     &= \frac{4 \cdot 100\,\text{кПа}}{1000\,\frac{\text{кг}}{\text{м}^{3}} \cdot  10\,\frac{\text{м}}{\text{с}^{2}}} \approx 40\,\text{м}.
    \end{align*}
}
\solutionspace{120pt}

\tasknumber{15}%
\task{%
    Газу сообщили некоторое количество теплоты,
    при этом треть его он потратил на совершение работы,
    одновременно увеличив свою внутреннюю энергию на $1200\,\text{Дж}$.
    Определите количество теплоты, сообщённое газу.
}
\answer{%
    \begin{align*}
    Q &= A' + \Delta U, A' = \frac 13 Q \implies Q \cdot \cbr{1 - \frac 13} = \Delta U \implies Q = \frac{\Delta U}{1 - \frac 13} = \frac{ 1200\,\text{Дж} }{1 - \frac 13} \approx 1800\,\text{Дж}.
    \\
    A' &= \frac 13 Q
        = \frac 13 \cdot \frac{\Delta U}{1 - \frac 13}
        = \frac{\Delta U}{3 - 1}
        = \frac{ 1200\,\text{Дж} }{3 - 1} \approx 600\,\text{Дж}.
    \end{align*}
}
\solutionspace{60pt}

\tasknumber{16}%
\task{%
    Два конденсатора ёмкостей $C_1 = 60\,\text{нФ}$ и $C_2 = 40\,\text{нФ}$ последовательно подключают
    к источнику напряжения $U = 150\,\text{В}$ (см.
    рис.).
    % Определите заряды каждого из конденсаторов.
    Определите заряд первого конденсатора.

    \begin{tikzpicture}[circuit ee IEC, semithick]
        \draw  (0, 0) to [capacitor={info={$C_1$}}] (1, 0)
                       to [capacitor={info={$C_2$}}] (2, 0)
        ;
        % \draw [-o] (0, 0) -- ++(-0.5, 0) node[left] {$-$};
        % \draw [-o] (2, 0) -- ++(0.5, 0) node[right] {$+$};
        \draw [-o] (0, 0) -- ++(-0.5, 0) node[left] {};
        \draw [-o] (2, 0) -- ++(0.5, 0) node[right] {};
    \end{tikzpicture}
}
\answer{%
    $
        Q_1
            = Q_2
            = CU
            = \frac{ U }{\frac1{C_1} + \frac1{C_2}}
            = \frac{C_1C_2U}{C_1 + C_2}
            = \frac{
                60\,\text{нФ} \cdot 40\,\text{нФ} \cdot 150\,\text{В}
            }{
                60\,\text{нФ} + 40\,\text{нФ}
            }
            = 3{,}60\,\text{мкКл}
    $
}
\solutionspace{120pt}

\tasknumber{17}%
\task{%
    В вакууме вдоль одной прямой расположены четыре отрицательных заряда так,
    что расстояние между соседними зарядами равно $a$.
    Сделайте рисунок,
    и определите силу, действующую на крайний заряд.
    Модули всех зарядов равны $q$ ($q > 0$).
}
\answer{%
    $F = \sum_i F_i = \ldots = \frac{49}{36} \frac{kq^2}{a^2}.$
}
\solutionspace{80pt}

\tasknumber{18}%
\task{%
    Юлия проводит эксперименты c 2 кусками одинаковой алюминиевой проволки, причём второй кусок в семь раз длиннее первого.
    В одном из экспериментов Юлия подаёт на первый кусок проволки напряжение в три раза раз больше, чем на второй.
    Определите отношения в двух проволках в этом эксперименте (второй к первой):
    \begin{itemize}
        \item отношение сил тока,
        \item отношение выделяющихся мощностей.
    \end{itemize}
}
\answer{%
    $R_2 = 7R_1, U_1 = 3U_2 \implies  \eli_2 / \eli_1 = \frac{U_2 / R_2}{U_1 / R_1} = \frac{U_2}{U_1} \cdot \frac{R_1}{R_2} = \frac1{21}, P_2 / P_1 = \frac{U_2^2 / R_2}{U_1^2 / R_1} = \sqr{\frac{U_2}{U_1}} \cdot \frac{R_1}{R_2} = \frac1{63}.$
}

\variantsplitter

\addpersonalvariant{Алина Полканова}

\tasknumber{1}%
\task{%
    Женя стартует на лошади и в течение $t = 4\,\text{c}$ двигается с постоянным ускорением $2\,\frac{\text{м}}{\text{с}^{2}}$.
    Определите
    \begin{itemize}
        \item какую скорость при этом удастся достичь,
        \item какой путь за это время будет пройден,
        \item среднюю скорость за всё время движения, если после начального ускорения продолжить движение равномерно ещё в течение времени $2t$
    \end{itemize}
}
\answer{%
    \begin{align*}
    v &= v_0 + a t = at = 2\,\frac{\text{м}}{\text{с}^{2}} \cdot 4\,\text{c} = 8{,}0\,\frac{\text{м}}{\text{с}}, \\
    s_x &= v_0t + \frac{a t^2}2 = \frac{a t^2}2 = \frac{2\,\frac{\text{м}}{\text{с}^{2}} \cdot \sqr{ 4\,\text{c} }}2 = 16{,}0\,\text{м}, \\
    v_\text{сред.} &= \frac{s_\text{общ}}{t_\text{общ.}} = \frac{s_x + v \cdot 2t}{t + 2t} = \frac{\frac{a t^2}2 + at \cdot 2t}{t (1 + 2)} = \\
    &= at \cdot \frac{\frac 12 + 2}{1 + 2} = 2\,\frac{\text{м}}{\text{с}^{2}} \cdot 4\,\text{c} \cdot \frac{\frac 12 + 2}{1 + 2} \approx 6{,}67\,\frac{\text{м}}{\text{c}}.
    \end{align*}
}
\solutionspace{120pt}

\tasknumber{2}%
\task{%
    Какой путь тело пройдёт за третью секунду после начала свободного падения?
    Какую скорость в начале этой секунды оно имеет?
}
\answer{%
    \begin{align*}
    s &= -s_y = -(y_2-y_1) = y_1 - y_2 = \cbr{y_{0y} + v_{0y}t_1 - \frac{gt_1^2}2} - \cbr{y_{0y} + v_{0y}t_2 - \frac{gt_2^2}2} = \\
    &= \frac{gt_2^2}2 - \frac{gt_1^2}2 = \frac g2\cbr{t_2^2 - t_1^2} = 25{,}0\,\text{м}, \\
    v_y &= v_{0y} - gt = -gt = 10\,\frac{\text{м}}{\text{с}^{2}} \cdot 2\,\text{с} = -20\,\frac{\text{м}}{\text{с}}.
    \end{align*}
}
\solutionspace{120pt}

\tasknumber{3}%
\task{%
    Карусель радиусом $4\,\text{м}$ равномерно совершает 10 оборотов в минуту.
    Определите
    \begin{itemize}
        \item период и частоту её обращения,
        \item скорость и ускорение крайних её точек.
    \end{itemize}
}
\answer{%
    \begin{align*}
    t &= 60\,\text{с}, r = 4{,}0\,\text{м}, n = 10\units{оборотов}, \\
    T &= \frac tN = \frac{ 60\,\text{с} }{10} \approx 6{,}00\,\text{c}, \\
    \nu &= \frac 1T = \frac{10}{ 60\,\text{с} } \approx 0{,}17\,\text{Гц}, \\
    v &= \frac{2 \pi r}{T} = \frac{2 \pi r}{T} =  \frac{2 \pi r n}{t} \approx 4{,}19\,\frac{\text{м}}{\text{c}}, \\
    a &= \frac{v^2}{r} =  \frac{4 \pi^2 r n^2}{t^2} \approx 4{,}39\,\frac{\text{м}}{\text{с}^{2}}.
    \end{align*}
}
\solutionspace{80pt}

\tasknumber{4}%
\task{%
    Маша стоит на обрыве над рекой и методично и строго горизонтально кидает в неё камушки.
    За этим всем наблюдает экспериментатор Глюк, который уже выяснил, что камушки падают в реку спустя $1{,}6\,\text{с}$ после броска,
    а вот дальность полёта оценить сложнее: придётся лезть в воду.
    Выручите Глюка и определите:
    \begin{itemize}
        \item высоту обрыва (вместе с ростом Маши).
        \item дальность полёта камушков (по горизонтали) и их скорость при падении, приняв начальную скорость броска равной $v_0 = 12\,\frac{\text{м}}{\text{с}}$.
    \end{itemize}
    Сопротивлением воздуха пренебречь.
}
\answer{%
    \begin{align*}
    y &= y_0 + v_{0y}t - \frac{gt^2}2 = h - \frac{gt^2}2, \qquad y(\tau) = 0 \implies h - \frac{g\tau^2}2 = 0 \implies h = \frac{g\tau^2}2 \approx 12{,}8\,\text{м}.
    \\
    x &= x_0 + v_{0x}t = v_0t \implies L = v_0\tau \approx 19{,}2\,\text{м}.
    \\
    &v = \sqrt{v_x^2 + v_y^2} = \sqrt{v_{0x}^2 + \sqr{v_{0y} - g\tau}} = \sqrt{v_0^2 + \sqr{g\tau}} \approx 20{,}0\,\frac{\text{м}}{\text{c}}.
    \end{align*}
}
\solutionspace{120pt}

\tasknumber{5}%
\task{%
    Шесть одинаковых брусков массой $2\,\text{кг}$ каждый лежат на гладком горизонтальном столе.
    Бруски пронумерованы от 1 до 6 и последовательно связаны между собой
    невесомыми нерастяжимыми нитями: 1 со 2, 2 с 3 (ну и с 1) и т.д.
    Экспериментатор Глюк прикладывает постоянную горизонтальную силу $120\,\text{Н}$ к бруску с наименьшим номером.
    С каким ускорением двигается система? Чему равна сила натяжения нити, связывающей бруски 1 и 2?
}
\answer{%
    \begin{align*}
    a &= \frac{F}{6 m} = \frac{120\,\text{Н}}{6 \cdot 2\,\text{кг}} \approx 10{,}0\,\frac{\text{м}}{\text{c}^{2}}, \\
    T &= m'a = 5m \cdot \frac{F}{6 m} = \frac{5}{6} F \approx 100{,}0\,\text{Н}.
    \end{align*}
}
\solutionspace{120pt}

\tasknumber{6}%
\task{%
    Два бруска связаны лёгкой нерастяжимой нитью и перекинуты через неподвижный блок (см.
    рис.).
    Определите силу натяжения нити и ускорения брусков.
    Силами трения пренебречь, массы брусков
    равны $m_1 = 5\,\text{кг}$ и $m_2 = 14\,\text{кг}$.
    % $g = 10\,\frac{\text{м}}{\text{с}^{2}}$.

    \begin{tikzpicture}[x=1.5cm,y=1.5cm,thick]
        \draw
            (-0.4, 0) rectangle (-0.2, 1.2)
            (0.15, 0.5) rectangle (0.45, 1)
            (0, 2) circle [radius=0.3] -- ++(up:0.5)
            (-0.3, 1.2) -- ++(up:0.8)
            (0.3, 1) -- ++(up:1)
            (-0.7, 2.5) -- (0.7, 2.5)
            ;
        \draw[pattern={Lines[angle=51,distance=3pt]},pattern color=black,draw=none] (-0.7, 2.5) rectangle (0.7, 2.75);
        \node [left] (left) at (-0.4, 0.6) { $m_1$ };
        \node [right] (right) at (0.4, 0.75) { $m_2$ };
    \end{tikzpicture}
}
\answer{%
    Предположим, что левый брусок ускоряется вверх, тогда правый ускоряется вниз (с тем же ускорением).
    Запишем 2-й закон Ньютона 2 раза (для обоих тел) в проекции на вертикальную оси, направив её вверх.
    \begin{align*}
        &\begin{cases}
            T - m_1g = m_1a, \\
            T - m_2g = -m_2a,
        \end{cases} \\
        &\begin{cases}
            m_2g - m_1g = m_1a + m_2a, \\
            T = m_1a + m_1g, \\
        \end{cases} \\
        a &= \frac{m_2 - m_1}{m_1 + m_2} \cdot g = \frac{14\,\text{кг} - 5\,\text{кг}}{5\,\text{кг} + 14\,\text{кг}} \cdot 10\,\frac{\text{м}}{\text{с}^{2}} \approx 4{,}74\,\frac{\text{м}}{\text{c}^{2}}, \\
        T &= m_1(a + g) = m_1 \cdot g \cdot \cbr{\frac{m_2 - m_1}{m_1 + m_2} + 1} = m_1 \cdot g \cdot \frac{2m_2}{m_1 + m_2} = \\
            &= \frac{2 m_2 m_1 g}{m_1 + m_2} = \frac{2 \cdot 14\,\text{кг} \cdot 5\,\text{кг} \cdot 10\,\frac{\text{м}}{\text{с}^{2}}}{5\,\text{кг} + 14\,\text{кг}} \approx 73{,}7\,\text{Н}.
    \end{align*}
    Отрицательный ответ говорит, что мы лишь не угадали с направлением ускорений.
    Сила же всегда положительна.
}
\solutionspace{80pt}

\tasknumber{7}%
\task{%
    Тело массой $2{,}7\,\text{кг}$ лежит на горизонтальной поверхности.
    Коэффициент трения между поверхностью и телом $0{,}25$.
    К телу приложена горизонтальная сила $4{,}5\,\text{Н}$.
    Определите силу трения, действующую на тело, и ускорение тела.
    % $g = 10\,\frac{\text{м}}{\text{с}^{2}}$.
}
\answer{%
    \begin{align*}
    &F_\text{ трения покоя $\max$ } = \mu N = \mu m g = 0{,}25 \cdot 2{,}7\,\text{кг} \cdot 10\,\frac{\text{м}}{\text{с}^{2}} = 6{,}75\,\text{Н}, \\
    &F_\text{ трения покоя $\max$ } > F \implies F_\text{ трения } = 4{,}50\,\text{Н}, a = \frac{F - F_\text{ трения }}{ m } = 0\,\frac{\text{м}}{\text{c}^{2}}, \\
    &\text{при равенстве возможны оба варианта: и едет, и не едет, но на ответы это не влияет.}
    \end{align*}
}
\solutionspace{120pt}

\tasknumber{8}%
\task{%
    Определите плотность неизвестного вещества, если известно, что опускании тела из него
    в керосин оно будет плавать и на четверть выступать над поверхностью жидкости.
}
\answer{%
    $F_\text{Арх.} = F_\text{тяж.} \implies \rho_\text{ж.} g V_\text{погр.} = m g \implies\rho_\text{ж.} g \cbr{V -\frac V4} = \rho V g \implies \rho = \rho_\text{ж.}\cbr{1 -\frac 14} \approx 600\,\frac{\text{кг}}{\text{м}^{3}}$
}
\solutionspace{120pt}

\tasknumber{9}%
\task{%
    	Определите силу, действующую на левую опору однородного горизонтального стержня длиной $l = 9\,\text{м}$
    	и массой $M = 1\,\text{кг}$, к которому подвешен груз массой $m = 4\,\text{кг}$ на расстоянии $4\,\text{м}$ от правого конца (см.
    рис.).

        \begin{tikzpicture}[thick]
            \draw
                (-2, -0.1) rectangle (2, 0.1)
                (-0.5, -0.1) -- (-0.5, -1)
                (-0.7, -1) rectangle (-0.3, -1.3)
           		(-2, -0.1) -- +(0.15,-0.9) -- +(-0.15,-0.9) -- cycle
            	(2, -0.1) -- +(0.15,-0.9) -- +(-0.15,-0.9) -- cycle
            ;
            \draw[pattern={Lines[angle=51,distance=2pt]},pattern color=black,draw=none]
            	(-2.15, -1.15) rectangle +(0.3, 0.15)
            	(2.15, -1.15) rectangle +(-0.3, 0.15)
            ;
            \node [right] (m_small) at (-0.3, -1.15) { $m$ };
            \node [above] (M_big) at (0, 0.1) { $M$ };
        \end{tikzpicture}
}
\answer{%
    \begin{align*}
        &\begin{cases}
            F_1 + F_2 - mg - Mg= 0, \\
            F_1 \cdot 0 - mg \cdot a - Mg \cdot \frac l2 + F_2 \cdot l = 0,
        \end{cases} \\
        F_2 &= \frac{mga + Mg\frac l2}l = \frac al \cdot mg + \frac{Mg}2 \approx 27{,}2\,\text{Н}, \\
        F_1 &= mg + Mg - F_2 = mg + Mg - \frac al \cdot mg - \frac{Mg}2 = \frac bl \cdot mg + \frac{Mg}2 \approx 22{,}8\,\text{Н}.
    \end{align*}
}
\solutionspace{80pt}

\tasknumber{10}%
\task{%
    Тонкий однородный шест длиной $3\,\text{м}$ и массой $10\,\text{кг}$ лежит на горизонтальной поверхности.
    \begin{itemize}
        \item Какую минимальную силу надо приложить к одному из его концов, чтобы оторвать его от этой поверхности?
        \item Какую минимальную работу надо совершить, чтобы поставить его на землю в вертикальное положение?
    \end{itemize}
    % Примите $g = 10\,\frac{\text{м}}{\text{с}^{2}}$.
}
\answer{%
    $F = \frac{mg}2 \approx 100\,\text{Н}, A = mg\frac l2 = 150\,\text{Дж}$
}
\solutionspace{120pt}

\tasknumber{11}%
\task{%
    Определите работу силы, которая обеспечит подъём тела массой $2\,\text{кг}$ на высоту $5\,\text{м}$ с постоянным ускорением $2\,\frac{\text{м}}{\text{c}^{2}}$.
    % Примите $g = 10\,\frac{\text{м}}{\text{с}^{2}}$.
}
\answer{%
    \begin{align*}
    &\text{Для подъёма:} A = Fh = (mg + ma) h = m(g+a)h, \\
    &\text{Для спуска:} A = -Fh = -(mg - ma) h = -m(g-a)h, \\
    &\text{В результате получаем:} 120\,\text{Дж}.
    \end{align*}
}
\solutionspace{60pt}

\tasknumber{12}%
\task{%
    Тело бросили вертикально вверх со скоростью $20\,\frac{\text{м}}{\text{c}}$.
    На какой высоте кинетическая энергия тела составит половину от потенциальной?
}
\answer{%
    \begin{align*}
    &0 + \frac{mv_0^2}2 = E_p + E_k, E_k = \frac 12 E_p \implies \\
    &\implies \frac{mv_0^2}2 = E_p + \frac 12 E_p = E_p\cbr{1 + \frac 12} = mgh\cbr{1 + \frac 12} \implies \\
    &\implies h = \frac{\frac{mv_0^2}2}{mg\cbr{1 + \frac 12}} = \frac{v_0^2}{2g} \cdot \frac 1{1 + \frac 12} \approx 13{,}3\,\text{м}.
    \end{align*}
}
\solutionspace{100pt}

\tasknumber{13}%
\task{%
    Плотность воздуха при нормальных условиях равна $1{,}3\,\frac{\text{кг}}{\text{м}^{3}}$.
    Чему равна плотность воздуха
    при температуре $200\celsius$ и давлении $50\,\text{кПа}$?
}
\answer{%
    \begin{align*}
    &\text{В общем случае:} PV = \frac m{\mu} RT \implies \rho = \frac mV = \frac m{\frac{\frac m{\mu} RT}P} = \frac{P\mu}{RT}, \\
    &\text{У нас 2 состояния:} \rho_1 = \frac{P_1\mu}{RT_1}, \rho_2 = \frac{P_2\mu}{RT_2} \implies \frac{\rho_2}{\rho_1} = \frac{\frac{P_2\mu}{RT_2}}{\frac{P_1\mu}{RT_1}} = \frac{P_2T_1}{P_1T_2} \implies \\
    &\implies \rho_2 = \rho_1 \cdot  \frac{P_2T_1}{P_1T_2} = 1{,}3\,\frac{\text{кг}}{\text{м}^{3}} \cdot \frac{50\,\text{кПа} \cdot 273\units{К}}{100\,\text{кПа} \cdot 473\units{К}} \approx 0{,}38\,\frac{\text{кг}}{\text{м}^{3}}.
    \end{align*}
}
\solutionspace{120pt}

\tasknumber{14}%
\task{%
    Небольшую цилиндрическую пробирку с воздухом погружают на некоторую глубину в глубокое пресное озеро,
    после чего воздух занимает в ней лишь четвертую часть от общего объема.
    Определите глубину, на которую погрузили пробирку.
    Температуру считать постоянной $T = 288\,\text{К}$, давлением паров воды пренебречь,
    атмосферное давление принять равным $p_{\text{aтм}} = 100\,\text{кПа}$.
}
\answer{%
    \begin{align*}
    T\text{— const} &\implies P_1V_1 = \nu RT = P_2V_2.
    \\
    V_2 = \frac 14 V_1 &\implies P_1V_1 = P_2 \cdot \frac 14V_1 \implies P_2 = 4P_1 = 4p_{\text{aтм}}.
    \\
    P_2 = p_{\text{aтм}} + \rho_{\text{в}} g h \implies h = \frac{P_2 - p_{\text{aтм}}}{\rho_{\text{в}} g} &= \frac{4p_{\text{aтм}} - p_{\text{aтм}}}{\rho_{\text{в}} g} = \frac{3 \cdot p_{\text{aтм}}}{\rho_{\text{в}} g} =  \\
     &= \frac{3 \cdot 100\,\text{кПа}}{1000\,\frac{\text{кг}}{\text{м}^{3}} \cdot  10\,\frac{\text{м}}{\text{с}^{2}}} \approx 30\,\text{м}.
    \end{align*}
}
\solutionspace{120pt}

\tasknumber{15}%
\task{%
    Газу сообщили некоторое количество теплоты,
    при этом треть его он потратил на совершение работы,
    одновременно увеличив свою внутреннюю энергию на $2400\,\text{Дж}$.
    Определите количество теплоты, сообщённое газу.
}
\answer{%
    \begin{align*}
    Q &= A' + \Delta U, A' = \frac 13 Q \implies Q \cdot \cbr{1 - \frac 13} = \Delta U \implies Q = \frac{\Delta U}{1 - \frac 13} = \frac{ 2400\,\text{Дж} }{1 - \frac 13} \approx 3600\,\text{Дж}.
    \\
    A' &= \frac 13 Q
        = \frac 13 \cdot \frac{\Delta U}{1 - \frac 13}
        = \frac{\Delta U}{3 - 1}
        = \frac{ 2400\,\text{Дж} }{3 - 1} \approx 1200\,\text{Дж}.
    \end{align*}
}
\solutionspace{60pt}

\tasknumber{16}%
\task{%
    Два конденсатора ёмкостей $C_1 = 40\,\text{нФ}$ и $C_2 = 60\,\text{нФ}$ последовательно подключают
    к источнику напряжения $V = 150\,\text{В}$ (см.
    рис.).
    % Определите заряды каждого из конденсаторов.
    Определите заряд первого конденсатора.

    \begin{tikzpicture}[circuit ee IEC, semithick]
        \draw  (0, 0) to [capacitor={info={$C_1$}}] (1, 0)
                       to [capacitor={info={$C_2$}}] (2, 0)
        ;
        % \draw [-o] (0, 0) -- ++(-0.5, 0) node[left] {$-$};
        % \draw [-o] (2, 0) -- ++(0.5, 0) node[right] {$+$};
        \draw [-o] (0, 0) -- ++(-0.5, 0) node[left] {};
        \draw [-o] (2, 0) -- ++(0.5, 0) node[right] {};
    \end{tikzpicture}
}
\answer{%
    $
        Q_1
            = Q_2
            = CV
            = \frac{ V }{\frac1{C_1} + \frac1{C_2}}
            = \frac{C_1C_2V}{C_1 + C_2}
            = \frac{
                40\,\text{нФ} \cdot 60\,\text{нФ} \cdot 150\,\text{В}
            }{
                40\,\text{нФ} + 60\,\text{нФ}
            }
            = 3{,}60\,\text{мкКл}
    $
}
\solutionspace{120pt}

\tasknumber{17}%
\task{%
    В вакууме вдоль одной прямой расположены три отрицательных заряда так,
    что расстояние между соседними зарядами равно $l$.
    Сделайте рисунок,
    и определите силу, действующую на крайний заряд.
    Модули всех зарядов равны $q$ ($q > 0$).
}
\answer{%
    $F = \sum_i F_i = \ldots = \frac54 \frac{kq^2}{l^2}.$
}
\solutionspace{80pt}

\tasknumber{18}%
\task{%
    Юлия проводит эксперименты c 2 кусками одинаковой алюминиевой проволки, причём второй кусок в три раза длиннее первого.
    В одном из экспериментов Юлия подаёт на первый кусок проволки напряжение в два раза раз больше, чем на второй.
    Определите отношения в двух проволках в этом эксперименте (второй к первой):
    \begin{itemize}
        \item отношение сил тока,
        \item отношение выделяющихся мощностей.
    \end{itemize}
}
\answer{%
    $R_2 = 3R_1, U_1 = 2U_2 \implies  \eli_2 / \eli_1 = \frac{U_2 / R_2}{U_1 / R_1} = \frac{U_2}{U_1} \cdot \frac{R_1}{R_2} = \frac16, P_2 / P_1 = \frac{U_2^2 / R_2}{U_1^2 / R_1} = \sqr{\frac{U_2}{U_1}} \cdot \frac{R_1}{R_2} = \frac1{12}.$
}

\variantsplitter

\addpersonalvariant{Сергей Пономарёв}

\tasknumber{1}%
\task{%
    Валя стартует на велосипеде и в течение $t = 2\,\text{c}$ двигается с постоянным ускорением $0{,}5\,\frac{\text{м}}{\text{с}^{2}}$.
    Определите
    \begin{itemize}
        \item какую скорость при этом удастся достичь,
        \item какой путь за это время будет пройден,
        \item среднюю скорость за всё время движения, если после начального ускорения продолжить движение равномерно ещё в течение времени $3t$
    \end{itemize}
}
\answer{%
    \begin{align*}
    v &= v_0 + a t = at = 0{,}5\,\frac{\text{м}}{\text{с}^{2}} \cdot 2\,\text{c} = 1{,}0\,\frac{\text{м}}{\text{с}}, \\
    s_x &= v_0t + \frac{a t^2}2 = \frac{a t^2}2 = \frac{0{,}5\,\frac{\text{м}}{\text{с}^{2}} \cdot \sqr{ 2\,\text{c} }}2 = 1{,}0\,\text{м}, \\
    v_\text{сред.} &= \frac{s_\text{общ}}{t_\text{общ.}} = \frac{s_x + v \cdot 3t}{t + 3t} = \frac{\frac{a t^2}2 + at \cdot 3t}{t (1 + 3)} = \\
    &= at \cdot \frac{\frac 12 + 3}{1 + 3} = 0{,}5\,\frac{\text{м}}{\text{с}^{2}} \cdot 2\,\text{c} \cdot \frac{\frac 12 + 3}{1 + 3} \approx 0{,}88\,\frac{\text{м}}{\text{c}}.
    \end{align*}
}
\solutionspace{120pt}

\tasknumber{2}%
\task{%
    Какой путь тело пройдёт за вторую секунду после начала свободного падения?
    Какую скорость в начале этой секунды оно имеет?
}
\answer{%
    \begin{align*}
    s &= -s_y = -(y_2-y_1) = y_1 - y_2 = \cbr{y_{0y} + v_{0y}t_1 - \frac{gt_1^2}2} - \cbr{y_{0y} + v_{0y}t_2 - \frac{gt_2^2}2} = \\
    &= \frac{gt_2^2}2 - \frac{gt_1^2}2 = \frac g2\cbr{t_2^2 - t_1^2} = 15{,}0\,\text{м}, \\
    v_y &= v_{0y} - gt = -gt = 10\,\frac{\text{м}}{\text{с}^{2}} \cdot 1\,\text{с} = -10\,\frac{\text{м}}{\text{с}}.
    \end{align*}
}
\solutionspace{120pt}

\tasknumber{3}%
\task{%
    Карусель диаметром $5\,\text{м}$ равномерно совершает 6 оборотов в минуту.
    Определите
    \begin{itemize}
        \item период и частоту её обращения,
        \item скорость и ускорение крайних её точек.
    \end{itemize}
}
\answer{%
    \begin{align*}
    t &= 60\,\text{с}, r = 2{,}5\,\text{м}, n = 6\units{оборотов}, \\
    T &= \frac tN = \frac{ 60\,\text{с} }{6} \approx 10{,}00\,\text{c}, \\
    \nu &= \frac 1T = \frac{6}{ 60\,\text{с} } \approx 0{,}10\,\text{Гц}, \\
    v &= \frac{2 \pi r}{T} = \frac{2 \pi r}{T} =  \frac{2 \pi r n}{t} \approx 1{,}57\,\frac{\text{м}}{\text{c}}, \\
    a &= \frac{v^2}{r} =  \frac{4 \pi^2 r n^2}{t^2} \approx 0{,}99\,\frac{\text{м}}{\text{с}^{2}}.
    \end{align*}
}
\solutionspace{80pt}

\tasknumber{4}%
\task{%
    Даша стоит на обрыве над рекой и методично и строго горизонтально кидает в неё камушки.
    За этим всем наблюдает экспериментатор Глюк, который уже выяснил, что камушки падают в реку спустя $1{,}3\,\text{с}$ после броска,
    а вот дальность полёта оценить сложнее: придётся лезть в воду.
    Выручите Глюка и определите:
    \begin{itemize}
        \item высоту обрыва (вместе с ростом Даши).
        \item дальность полёта камушков (по горизонтали) и их скорость при падении, приняв начальную скорость броска равной $v_0 = 15\,\frac{\text{м}}{\text{с}}$.
    \end{itemize}
    Сопротивлением воздуха пренебречь.
}
\answer{%
    \begin{align*}
    y &= y_0 + v_{0y}t - \frac{gt^2}2 = h - \frac{gt^2}2, \qquad y(\tau) = 0 \implies h - \frac{g\tau^2}2 = 0 \implies h = \frac{g\tau^2}2 \approx 8{,}5\,\text{м}.
    \\
    x &= x_0 + v_{0x}t = v_0t \implies L = v_0\tau \approx 19{,}5\,\text{м}.
    \\
    &v = \sqrt{v_x^2 + v_y^2} = \sqrt{v_{0x}^2 + \sqr{v_{0y} - g\tau}} = \sqrt{v_0^2 + \sqr{g\tau}} \approx 19{,}8\,\frac{\text{м}}{\text{c}}.
    \end{align*}
}
\solutionspace{120pt}

\tasknumber{5}%
\task{%
    Шесть одинаковых брусков массой $2\,\text{кг}$ каждый лежат на гладком горизонтальном столе.
    Бруски пронумерованы от 1 до 6 и последовательно связаны между собой
    невесомыми нерастяжимыми нитями: 1 со 2, 2 с 3 (ну и с 1) и т.д.
    Экспериментатор Глюк прикладывает постоянную горизонтальную силу $120\,\text{Н}$ к бруску с наибольшим номером.
    С каким ускорением двигается система? Чему равна сила натяжения нити, связывающей бруски 3 и 4?
}
\answer{%
    \begin{align*}
    a &= \frac{F}{6 m} = \frac{120\,\text{Н}}{6 \cdot 2\,\text{кг}} \approx 10{,}0\,\frac{\text{м}}{\text{c}^{2}}, \\
    T &= m'a = 3m \cdot \frac{F}{6 m} = \frac{3}{6} F \approx 60{,}0\,\text{Н}.
    \end{align*}
}
\solutionspace{120pt}

\tasknumber{6}%
\task{%
    Два бруска связаны лёгкой нерастяжимой нитью и перекинуты через неподвижный блок (см.
    рис.).
    Определите силу натяжения нити и ускорения брусков.
    Силами трения пренебречь, массы брусков
    равны $m_1 = 5\,\text{кг}$ и $m_2 = 10\,\text{кг}$.
    % $g = 10\,\frac{\text{м}}{\text{с}^{2}}$.

    \begin{tikzpicture}[x=1.5cm,y=1.5cm,thick]
        \draw
            (-0.4, 0) rectangle (-0.2, 1.2)
            (0.15, 0.5) rectangle (0.45, 1)
            (0, 2) circle [radius=0.3] -- ++(up:0.5)
            (-0.3, 1.2) -- ++(up:0.8)
            (0.3, 1) -- ++(up:1)
            (-0.7, 2.5) -- (0.7, 2.5)
            ;
        \draw[pattern={Lines[angle=51,distance=3pt]},pattern color=black,draw=none] (-0.7, 2.5) rectangle (0.7, 2.75);
        \node [left] (left) at (-0.4, 0.6) { $m_1$ };
        \node [right] (right) at (0.4, 0.75) { $m_2$ };
    \end{tikzpicture}
}
\answer{%
    Предположим, что левый брусок ускоряется вверх, тогда правый ускоряется вниз (с тем же ускорением).
    Запишем 2-й закон Ньютона 2 раза (для обоих тел) в проекции на вертикальную оси, направив её вверх.
    \begin{align*}
        &\begin{cases}
            T - m_1g = m_1a, \\
            T - m_2g = -m_2a,
        \end{cases} \\
        &\begin{cases}
            m_2g - m_1g = m_1a + m_2a, \\
            T = m_1a + m_1g, \\
        \end{cases} \\
        a &= \frac{m_2 - m_1}{m_1 + m_2} \cdot g = \frac{10\,\text{кг} - 5\,\text{кг}}{5\,\text{кг} + 10\,\text{кг}} \cdot 10\,\frac{\text{м}}{\text{с}^{2}} \approx 3{,}33\,\frac{\text{м}}{\text{c}^{2}}, \\
        T &= m_1(a + g) = m_1 \cdot g \cdot \cbr{\frac{m_2 - m_1}{m_1 + m_2} + 1} = m_1 \cdot g \cdot \frac{2m_2}{m_1 + m_2} = \\
            &= \frac{2 m_2 m_1 g}{m_1 + m_2} = \frac{2 \cdot 10\,\text{кг} \cdot 5\,\text{кг} \cdot 10\,\frac{\text{м}}{\text{с}^{2}}}{5\,\text{кг} + 10\,\text{кг}} \approx 66{,}7\,\text{Н}.
    \end{align*}
    Отрицательный ответ говорит, что мы лишь не угадали с направлением ускорений.
    Сила же всегда положительна.
}
\solutionspace{80pt}

\tasknumber{7}%
\task{%
    Тело массой $2{,}7\,\text{кг}$ лежит на горизонтальной поверхности.
    Коэффициент трения между поверхностью и телом $0{,}25$.
    К телу приложена горизонтальная сила $3{,}5\,\text{Н}$.
    Определите силу трения, действующую на тело, и ускорение тела.
    % $g = 10\,\frac{\text{м}}{\text{с}^{2}}$.
}
\answer{%
    \begin{align*}
    &F_\text{ трения покоя $\max$ } = \mu N = \mu m g = 0{,}25 \cdot 2{,}7\,\text{кг} \cdot 10\,\frac{\text{м}}{\text{с}^{2}} = 6{,}75\,\text{Н}, \\
    &F_\text{ трения покоя $\max$ } > F \implies F_\text{ трения } = 3{,}50\,\text{Н}, a = \frac{F - F_\text{ трения }}{ m } = 0\,\frac{\text{м}}{\text{c}^{2}}, \\
    &\text{при равенстве возможны оба варианта: и едет, и не едет, но на ответы это не влияет.}
    \end{align*}
}
\solutionspace{120pt}

\tasknumber{8}%
\task{%
    Определите плотность неизвестного вещества, если известно, что опускании тела из него
    в керосин оно будет плавать и на четверть выступать над поверхностью жидкости.
}
\answer{%
    $F_\text{Арх.} = F_\text{тяж.} \implies \rho_\text{ж.} g V_\text{погр.} = m g \implies\rho_\text{ж.} g \cbr{V -\frac V4} = \rho V g \implies \rho = \rho_\text{ж.}\cbr{1 -\frac 14} \approx 600\,\frac{\text{кг}}{\text{м}^{3}}$
}
\solutionspace{120pt}

\tasknumber{9}%
\task{%
    	Определите силу, действующую на правую опору однородного горизонтального стержня длиной $l = 3\,\text{м}$
    	и массой $M = 1\,\text{кг}$, к которому подвешен груз массой $m = 3\,\text{кг}$ на расстоянии $2\,\text{м}$ от правого конца (см.
    рис.).

        \begin{tikzpicture}[thick]
            \draw
                (-2, -0.1) rectangle (2, 0.1)
                (-0.5, -0.1) -- (-0.5, -1)
                (-0.7, -1) rectangle (-0.3, -1.3)
           		(-2, -0.1) -- +(0.15,-0.9) -- +(-0.15,-0.9) -- cycle
            	(2, -0.1) -- +(0.15,-0.9) -- +(-0.15,-0.9) -- cycle
            ;
            \draw[pattern={Lines[angle=51,distance=2pt]},pattern color=black,draw=none]
            	(-2.15, -1.15) rectangle +(0.3, 0.15)
            	(2.15, -1.15) rectangle +(-0.3, 0.15)
            ;
            \node [right] (m_small) at (-0.3, -1.15) { $m$ };
            \node [above] (M_big) at (0, 0.1) { $M$ };
        \end{tikzpicture}
}
\answer{%
    \begin{align*}
        &\begin{cases}
            F_1 + F_2 - mg - Mg= 0, \\
            F_1 \cdot 0 - mg \cdot a - Mg \cdot \frac l2 + F_2 \cdot l = 0,
        \end{cases} \\
        F_2 &= \frac{mga + Mg\frac l2}l = \frac al \cdot mg + \frac{Mg}2 \approx 15{,}0\,\text{Н}, \\
        F_1 &= mg + Mg - F_2 = mg + Mg - \frac al \cdot mg - \frac{Mg}2 = \frac bl \cdot mg + \frac{Mg}2 \approx 25{,}0\,\text{Н}.
    \end{align*}
}
\solutionspace{80pt}

\tasknumber{10}%
\task{%
    Тонкий однородный шест длиной $3\,\text{м}$ и массой $30\,\text{кг}$ лежит на горизонтальной поверхности.
    \begin{itemize}
        \item Какую минимальную силу надо приложить к одному из его концов, чтобы оторвать его от этой поверхности?
        \item Какую минимальную работу надо совершить, чтобы поставить его на землю в вертикальное положение?
    \end{itemize}
    % Примите $g = 10\,\frac{\text{м}}{\text{с}^{2}}$.
}
\answer{%
    $F = \frac{mg}2 \approx 300\,\text{Н}, A = mg\frac l2 = 450\,\text{Дж}$
}
\solutionspace{120pt}

\tasknumber{11}%
\task{%
    Определите работу силы, которая обеспечит спуск тела массой $5\,\text{кг}$ на высоту $10\,\text{м}$ с постоянным ускорением $2\,\frac{\text{м}}{\text{c}^{2}}$.
    % Примите $g = 10\,\frac{\text{м}}{\text{с}^{2}}$.
}
\answer{%
    \begin{align*}
    &\text{Для подъёма:} A = Fh = (mg + ma) h = m(g+a)h, \\
    &\text{Для спуска:} A = -Fh = -(mg - ma) h = -m(g-a)h, \\
    &\text{В результате получаем:} -400\,\text{Дж}.
    \end{align*}
}
\solutionspace{60pt}

\tasknumber{12}%
\task{%
    Тело бросили вертикально вверх со скоростью $14\,\frac{\text{м}}{\text{c}}$.
    На какой высоте кинетическая энергия тела составит половину от потенциальной?
}
\answer{%
    \begin{align*}
    &0 + \frac{mv_0^2}2 = E_p + E_k, E_k = \frac 12 E_p \implies \\
    &\implies \frac{mv_0^2}2 = E_p + \frac 12 E_p = E_p\cbr{1 + \frac 12} = mgh\cbr{1 + \frac 12} \implies \\
    &\implies h = \frac{\frac{mv_0^2}2}{mg\cbr{1 + \frac 12}} = \frac{v_0^2}{2g} \cdot \frac 1{1 + \frac 12} \approx 6{,}5\,\text{м}.
    \end{align*}
}
\solutionspace{100pt}

\tasknumber{13}%
\task{%
    Плотность воздуха при нормальных условиях равна $1{,}3\,\frac{\text{кг}}{\text{м}^{3}}$.
    Чему равна плотность воздуха
    при температуре $150\celsius$ и давлении $120\,\text{кПа}$?
}
\answer{%
    \begin{align*}
    &\text{В общем случае:} PV = \frac m{\mu} RT \implies \rho = \frac mV = \frac m{\frac{\frac m{\mu} RT}P} = \frac{P\mu}{RT}, \\
    &\text{У нас 2 состояния:} \rho_1 = \frac{P_1\mu}{RT_1}, \rho_2 = \frac{P_2\mu}{RT_2} \implies \frac{\rho_2}{\rho_1} = \frac{\frac{P_2\mu}{RT_2}}{\frac{P_1\mu}{RT_1}} = \frac{P_2T_1}{P_1T_2} \implies \\
    &\implies \rho_2 = \rho_1 \cdot  \frac{P_2T_1}{P_1T_2} = 1{,}3\,\frac{\text{кг}}{\text{м}^{3}} \cdot \frac{120\,\text{кПа} \cdot 273\units{К}}{100\,\text{кПа} \cdot 423\units{К}} \approx 1{,}01\,\frac{\text{кг}}{\text{м}^{3}}.
    \end{align*}
}
\solutionspace{120pt}

\tasknumber{14}%
\task{%
    Небольшую цилиндрическую пробирку с воздухом погружают на некоторую глубину в глубокое пресное озеро,
    после чего воздух занимает в ней лишь четвертую часть от общего объема.
    Определите глубину, на которую погрузили пробирку.
    Температуру считать постоянной $T = 279\,\text{К}$, давлением паров воды пренебречь,
    атмосферное давление принять равным $p_{\text{aтм}} = 100\,\text{кПа}$.
}
\answer{%
    \begin{align*}
    T\text{— const} &\implies P_1V_1 = \nu RT = P_2V_2.
    \\
    V_2 = \frac 14 V_1 &\implies P_1V_1 = P_2 \cdot \frac 14V_1 \implies P_2 = 4P_1 = 4p_{\text{aтм}}.
    \\
    P_2 = p_{\text{aтм}} + \rho_{\text{в}} g h \implies h = \frac{P_2 - p_{\text{aтм}}}{\rho_{\text{в}} g} &= \frac{4p_{\text{aтм}} - p_{\text{aтм}}}{\rho_{\text{в}} g} = \frac{3 \cdot p_{\text{aтм}}}{\rho_{\text{в}} g} =  \\
     &= \frac{3 \cdot 100\,\text{кПа}}{1000\,\frac{\text{кг}}{\text{м}^{3}} \cdot  10\,\frac{\text{м}}{\text{с}^{2}}} \approx 30\,\text{м}.
    \end{align*}
}
\solutionspace{120pt}

\tasknumber{15}%
\task{%
    Газу сообщили некоторое количество теплоты,
    при этом половину его он потратил на совершение работы,
    одновременно увеличив свою внутреннюю энергию на $1500\,\text{Дж}$.
    Определите количество теплоты, сообщённое газу.
}
\answer{%
    \begin{align*}
    Q &= A' + \Delta U, A' = \frac 12 Q \implies Q \cdot \cbr{1 - \frac 12} = \Delta U \implies Q = \frac{\Delta U}{1 - \frac 12} = \frac{ 1500\,\text{Дж} }{1 - \frac 12} \approx 3000\,\text{Дж}.
    \\
    A' &= \frac 12 Q
        = \frac 12 \cdot \frac{\Delta U}{1 - \frac 12}
        = \frac{\Delta U}{2 - 1}
        = \frac{ 1500\,\text{Дж} }{2 - 1} \approx 1500\,\text{Дж}.
    \end{align*}
}
\solutionspace{60pt}

\tasknumber{16}%
\task{%
    Два конденсатора ёмкостей $C_1 = 30\,\text{нФ}$ и $C_2 = 40\,\text{нФ}$ последовательно подключают
    к источнику напряжения $V = 450\,\text{В}$ (см.
    рис.).
    % Определите заряды каждого из конденсаторов.
    Определите заряд первого конденсатора.

    \begin{tikzpicture}[circuit ee IEC, semithick]
        \draw  (0, 0) to [capacitor={info={$C_1$}}] (1, 0)
                       to [capacitor={info={$C_2$}}] (2, 0)
        ;
        % \draw [-o] (0, 0) -- ++(-0.5, 0) node[left] {$-$};
        % \draw [-o] (2, 0) -- ++(0.5, 0) node[right] {$+$};
        \draw [-o] (0, 0) -- ++(-0.5, 0) node[left] {};
        \draw [-o] (2, 0) -- ++(0.5, 0) node[right] {};
    \end{tikzpicture}
}
\answer{%
    $
        Q_1
            = Q_2
            = CV
            = \frac{ V }{\frac1{C_1} + \frac1{C_2}}
            = \frac{C_1C_2V}{C_1 + C_2}
            = \frac{
                30\,\text{нФ} \cdot 40\,\text{нФ} \cdot 450\,\text{В}
            }{
                30\,\text{нФ} + 40\,\text{нФ}
            }
            = 7{,}71\,\text{мкКл}
    $
}
\solutionspace{120pt}

\tasknumber{17}%
\task{%
    В вакууме вдоль одной прямой расположены четыре отрицательных заряда так,
    что расстояние между соседними зарядами равно $l$.
    Сделайте рисунок,
    и определите силу, действующую на крайний заряд.
    Модули всех зарядов равны $Q$ ($Q > 0$).
}
\answer{%
    $F = \sum_i F_i = \ldots = \frac{49}{36} \frac{kQ^2}{l^2}.$
}
\solutionspace{80pt}

\tasknumber{18}%
\task{%
    Юлия проводит эксперименты c 2 кусками одинаковой медной проволки, причём второй кусок в шесть раз длиннее первого.
    В одном из экспериментов Юлия подаёт на первый кусок проволки напряжение в два раза раз больше, чем на второй.
    Определите отношения в двух проволках в этом эксперименте (второй к первой):
    \begin{itemize}
        \item отношение сил тока,
        \item отношение выделяющихся мощностей.
    \end{itemize}
}
\answer{%
    $R_2 = 6R_1, U_1 = 2U_2 \implies  \eli_2 / \eli_1 = \frac{U_2 / R_2}{U_1 / R_1} = \frac{U_2}{U_1} \cdot \frac{R_1}{R_2} = \frac1{12}, P_2 / P_1 = \frac{U_2^2 / R_2}{U_1^2 / R_1} = \sqr{\frac{U_2}{U_1}} \cdot \frac{R_1}{R_2} = \frac1{24}.$
}

\variantsplitter

\addpersonalvariant{Егор Свистушкин}

\tasknumber{1}%
\task{%
    Саша стартует на лошади и в течение $t = 10\,\text{c}$ двигается с постоянным ускорением $2\,\frac{\text{м}}{\text{с}^{2}}$.
    Определите
    \begin{itemize}
        \item какую скорость при этом удастся достичь,
        \item какой путь за это время будет пройден,
        \item среднюю скорость за всё время движения, если после начального ускорения продолжить движение равномерно ещё в течение времени $3t$
    \end{itemize}
}
\answer{%
    \begin{align*}
    v &= v_0 + a t = at = 2\,\frac{\text{м}}{\text{с}^{2}} \cdot 10\,\text{c} = 20{,}0\,\frac{\text{м}}{\text{с}}, \\
    s_x &= v_0t + \frac{a t^2}2 = \frac{a t^2}2 = \frac{2\,\frac{\text{м}}{\text{с}^{2}} \cdot \sqr{ 10\,\text{c} }}2 = 100{,}0\,\text{м}, \\
    v_\text{сред.} &= \frac{s_\text{общ}}{t_\text{общ.}} = \frac{s_x + v \cdot 3t}{t + 3t} = \frac{\frac{a t^2}2 + at \cdot 3t}{t (1 + 3)} = \\
    &= at \cdot \frac{\frac 12 + 3}{1 + 3} = 2\,\frac{\text{м}}{\text{с}^{2}} \cdot 10\,\text{c} \cdot \frac{\frac 12 + 3}{1 + 3} \approx 17{,}50\,\frac{\text{м}}{\text{c}}.
    \end{align*}
}
\solutionspace{120pt}

\tasknumber{2}%
\task{%
    Какой путь тело пройдёт за пятую секунду после начала свободного падения?
    Какую скорость в конце этой секунды оно имеет?
}
\answer{%
    \begin{align*}
    s &= -s_y = -(y_2-y_1) = y_1 - y_2 = \cbr{y_{0y} + v_{0y}t_1 - \frac{gt_1^2}2} - \cbr{y_{0y} + v_{0y}t_2 - \frac{gt_2^2}2} = \\
    &= \frac{gt_2^2}2 - \frac{gt_1^2}2 = \frac g2\cbr{t_2^2 - t_1^2} = 45{,}0\,\text{м}, \\
    v_y &= v_{0y} - gt = -gt = 10\,\frac{\text{м}}{\text{с}^{2}} \cdot 5\,\text{с} = -50\,\frac{\text{м}}{\text{с}}.
    \end{align*}
}
\solutionspace{120pt}

\tasknumber{3}%
\task{%
    Карусель радиусом $3\,\text{м}$ равномерно совершает 6 оборотов в минуту.
    Определите
    \begin{itemize}
        \item период и частоту её обращения,
        \item скорость и ускорение крайних её точек.
    \end{itemize}
}
\answer{%
    \begin{align*}
    t &= 60\,\text{с}, r = 3{,}0\,\text{м}, n = 6\units{оборотов}, \\
    T &= \frac tN = \frac{ 60\,\text{с} }{6} \approx 10{,}00\,\text{c}, \\
    \nu &= \frac 1T = \frac{6}{ 60\,\text{с} } \approx 0{,}10\,\text{Гц}, \\
    v &= \frac{2 \pi r}{T} = \frac{2 \pi r}{T} =  \frac{2 \pi r n}{t} \approx 1{,}88\,\frac{\text{м}}{\text{c}}, \\
    a &= \frac{v^2}{r} =  \frac{4 \pi^2 r n^2}{t^2} \approx 1{,}18\,\frac{\text{м}}{\text{с}^{2}}.
    \end{align*}
}
\solutionspace{80pt}

\tasknumber{4}%
\task{%
    Миша стоит на обрыве над рекой и методично и строго горизонтально кидает в неё камушки.
    За этим всем наблюдает экспериментатор Глюк, который уже выяснил, что камушки падают в реку спустя $1{,}7\,\text{с}$ после броска,
    а вот дальность полёта оценить сложнее: придётся лезть в воду.
    Выручите Глюка и определите:
    \begin{itemize}
        \item высоту обрыва (вместе с ростом Миши).
        \item дальность полёта камушков (по горизонтали) и их скорость при падении, приняв начальную скорость броска равной $v_0 = 17\,\frac{\text{м}}{\text{с}}$.
    \end{itemize}
    Сопротивлением воздуха пренебречь.
}
\answer{%
    \begin{align*}
    y &= y_0 + v_{0y}t - \frac{gt^2}2 = h - \frac{gt^2}2, \qquad y(\tau) = 0 \implies h - \frac{g\tau^2}2 = 0 \implies h = \frac{g\tau^2}2 \approx 14{,}4\,\text{м}.
    \\
    x &= x_0 + v_{0x}t = v_0t \implies L = v_0\tau \approx 28{,}9\,\text{м}.
    \\
    &v = \sqrt{v_x^2 + v_y^2} = \sqrt{v_{0x}^2 + \sqr{v_{0y} - g\tau}} = \sqrt{v_0^2 + \sqr{g\tau}} \approx 24{,}0\,\frac{\text{м}}{\text{c}}.
    \end{align*}
}
\solutionspace{120pt}

\tasknumber{5}%
\task{%
    Пять одинаковых брусков массой $2\,\text{кг}$ каждый лежат на гладком горизонтальном столе.
    Бруски пронумерованы от 1 до 5 и последовательно связаны между собой
    невесомыми нерастяжимыми нитями: 1 со 2, 2 с 3 (ну и с 1) и т.д.
    Экспериментатор Глюк прикладывает постоянную горизонтальную силу $60\,\text{Н}$ к бруску с наибольшим номером.
    С каким ускорением двигается система? Чему равна сила натяжения нити, связывающей бруски 2 и 3?
}
\answer{%
    \begin{align*}
    a &= \frac{F}{5 m} = \frac{60\,\text{Н}}{5 \cdot 2\,\text{кг}} \approx 6{,}0\,\frac{\text{м}}{\text{c}^{2}}, \\
    T &= m'a = 2m \cdot \frac{F}{5 m} = \frac{2}{5} F \approx 24{,}0\,\text{Н}.
    \end{align*}
}
\solutionspace{120pt}

\tasknumber{6}%
\task{%
    Два бруска связаны лёгкой нерастяжимой нитью и перекинуты через неподвижный блок (см.
    рис.).
    Определите силу натяжения нити и ускорения брусков.
    Силами трения пренебречь, массы брусков
    равны $m_1 = 11\,\text{кг}$ и $m_2 = 10\,\text{кг}$.
    % $g = 10\,\frac{\text{м}}{\text{с}^{2}}$.

    \begin{tikzpicture}[x=1.5cm,y=1.5cm,thick]
        \draw
            (-0.4, 0) rectangle (-0.2, 1.2)
            (0.15, 0.5) rectangle (0.45, 1)
            (0, 2) circle [radius=0.3] -- ++(up:0.5)
            (-0.3, 1.2) -- ++(up:0.8)
            (0.3, 1) -- ++(up:1)
            (-0.7, 2.5) -- (0.7, 2.5)
            ;
        \draw[pattern={Lines[angle=51,distance=3pt]},pattern color=black,draw=none] (-0.7, 2.5) rectangle (0.7, 2.75);
        \node [left] (left) at (-0.4, 0.6) { $m_1$ };
        \node [right] (right) at (0.4, 0.75) { $m_2$ };
    \end{tikzpicture}
}
\answer{%
    Предположим, что левый брусок ускоряется вверх, тогда правый ускоряется вниз (с тем же ускорением).
    Запишем 2-й закон Ньютона 2 раза (для обоих тел) в проекции на вертикальную оси, направив её вверх.
    \begin{align*}
        &\begin{cases}
            T - m_1g = m_1a, \\
            T - m_2g = -m_2a,
        \end{cases} \\
        &\begin{cases}
            m_2g - m_1g = m_1a + m_2a, \\
            T = m_1a + m_1g, \\
        \end{cases} \\
        a &= \frac{m_2 - m_1}{m_1 + m_2} \cdot g = \frac{10\,\text{кг} - 11\,\text{кг}}{11\,\text{кг} + 10\,\text{кг}} \cdot 10\,\frac{\text{м}}{\text{с}^{2}} \approx -0{,}4800\,\frac{\text{м}}{\text{c}^{2}}, \\
        T &= m_1(a + g) = m_1 \cdot g \cdot \cbr{\frac{m_2 - m_1}{m_1 + m_2} + 1} = m_1 \cdot g \cdot \frac{2m_2}{m_1 + m_2} = \\
            &= \frac{2 m_2 m_1 g}{m_1 + m_2} = \frac{2 \cdot 10\,\text{кг} \cdot 11\,\text{кг} \cdot 10\,\frac{\text{м}}{\text{с}^{2}}}{11\,\text{кг} + 10\,\text{кг}} \approx 104{,}8\,\text{Н}.
    \end{align*}
    Отрицательный ответ говорит, что мы лишь не угадали с направлением ускорений.
    Сила же всегда положительна.
}
\solutionspace{80pt}

\tasknumber{7}%
\task{%
    Тело массой $1{,}4\,\text{кг}$ лежит на горизонтальной поверхности.
    Коэффициент трения между поверхностью и телом $0{,}15$.
    К телу приложена горизонтальная сила $3{,}5\,\text{Н}$.
    Определите силу трения, действующую на тело, и ускорение тела.
    % $g = 10\,\frac{\text{м}}{\text{с}^{2}}$.
}
\answer{%
    \begin{align*}
    &F_\text{ трения покоя $\max$ } = \mu N = \mu m g = 0{,}15 \cdot 1{,}4\,\text{кг} \cdot 10\,\frac{\text{м}}{\text{с}^{2}} = 2{,}10\,\text{Н}, \\
    &F_\text{ трения покоя $\max$ } \le F \implies F_\text{ трения } = 2{,}10\,\text{Н}, a = \frac{F - F_\text{ трения }}{ m } = 1{,}00\,\frac{\text{м}}{\text{c}^{2}}, \\
    &\text{при равенстве возможны оба варианта: и едет, и не едет, но на ответы это не влияет.}
    \end{align*}
}
\solutionspace{120pt}

\tasknumber{8}%
\task{%
    Определите плотность неизвестного вещества, если известно, что опускании тела из него
    в керосин оно будет плавать и на половину выступать над поверхностью жидкости.
}
\answer{%
    $F_\text{Арх.} = F_\text{тяж.} \implies \rho_\text{ж.} g V_\text{погр.} = m g \implies\rho_\text{ж.} g \cbr{V -\frac V2} = \rho V g \implies \rho = \rho_\text{ж.}\cbr{1 -\frac 12} \approx 400\,\frac{\text{кг}}{\text{м}^{3}}$
}
\solutionspace{120pt}

\tasknumber{9}%
\task{%
    	Определите силу, действующую на правую опору однородного горизонтального стержня длиной $l = 5\,\text{м}$
    	и массой $M = 1\,\text{кг}$, к которому подвешен груз массой $m = 2\,\text{кг}$ на расстоянии $2\,\text{м}$ от правого конца (см.
    рис.).

        \begin{tikzpicture}[thick]
            \draw
                (-2, -0.1) rectangle (2, 0.1)
                (-0.5, -0.1) -- (-0.5, -1)
                (-0.7, -1) rectangle (-0.3, -1.3)
           		(-2, -0.1) -- +(0.15,-0.9) -- +(-0.15,-0.9) -- cycle
            	(2, -0.1) -- +(0.15,-0.9) -- +(-0.15,-0.9) -- cycle
            ;
            \draw[pattern={Lines[angle=51,distance=2pt]},pattern color=black,draw=none]
            	(-2.15, -1.15) rectangle +(0.3, 0.15)
            	(2.15, -1.15) rectangle +(-0.3, 0.15)
            ;
            \node [right] (m_small) at (-0.3, -1.15) { $m$ };
            \node [above] (M_big) at (0, 0.1) { $M$ };
        \end{tikzpicture}
}
\answer{%
    \begin{align*}
        &\begin{cases}
            F_1 + F_2 - mg - Mg= 0, \\
            F_1 \cdot 0 - mg \cdot a - Mg \cdot \frac l2 + F_2 \cdot l = 0,
        \end{cases} \\
        F_2 &= \frac{mga + Mg\frac l2}l = \frac al \cdot mg + \frac{Mg}2 \approx 17{,}0\,\text{Н}, \\
        F_1 &= mg + Mg - F_2 = mg + Mg - \frac al \cdot mg - \frac{Mg}2 = \frac bl \cdot mg + \frac{Mg}2 \approx 13{,}0\,\text{Н}.
    \end{align*}
}
\solutionspace{80pt}

\tasknumber{10}%
\task{%
    Тонкий однородный шест длиной $1\,\text{м}$ и массой $10\,\text{кг}$ лежит на горизонтальной поверхности.
    \begin{itemize}
        \item Какую минимальную силу надо приложить к одному из его концов, чтобы оторвать его от этой поверхности?
        \item Какую минимальную работу надо совершить, чтобы поставить его на землю в вертикальное положение?
    \end{itemize}
    % Примите $g = 10\,\frac{\text{м}}{\text{с}^{2}}$.
}
\answer{%
    $F = \frac{mg}2 \approx 100\,\text{Н}, A = mg\frac l2 = 50\,\text{Дж}$
}
\solutionspace{120pt}

\tasknumber{11}%
\task{%
    Определите работу силы, которая обеспечит спуск тела массой $5\,\text{кг}$ на высоту $10\,\text{м}$ с постоянным ускорением $6\,\frac{\text{м}}{\text{c}^{2}}$.
    % Примите $g = 10\,\frac{\text{м}}{\text{с}^{2}}$.
}
\answer{%
    \begin{align*}
    &\text{Для подъёма:} A = Fh = (mg + ma) h = m(g+a)h, \\
    &\text{Для спуска:} A = -Fh = -(mg - ma) h = -m(g-a)h, \\
    &\text{В результате получаем:} -200\,\text{Дж}.
    \end{align*}
}
\solutionspace{60pt}

\tasknumber{12}%
\task{%
    Тело бросили вертикально вверх со скоростью $10\,\frac{\text{м}}{\text{c}}$.
    На какой высоте кинетическая энергия тела составит треть от потенциальной?
}
\answer{%
    \begin{align*}
    &0 + \frac{mv_0^2}2 = E_p + E_k, E_k = \frac 13 E_p \implies \\
    &\implies \frac{mv_0^2}2 = E_p + \frac 13 E_p = E_p\cbr{1 + \frac 13} = mgh\cbr{1 + \frac 13} \implies \\
    &\implies h = \frac{\frac{mv_0^2}2}{mg\cbr{1 + \frac 13}} = \frac{v_0^2}{2g} \cdot \frac 1{1 + \frac 13} \approx 3{,}8\,\text{м}.
    \end{align*}
}
\solutionspace{100pt}

\tasknumber{13}%
\task{%
    Плотность воздуха при нормальных условиях равна $1{,}3\,\frac{\text{кг}}{\text{м}^{3}}$.
    Чему равна плотность воздуха
    при температуре $200\celsius$ и давлении $150\,\text{кПа}$?
}
\answer{%
    \begin{align*}
    &\text{В общем случае:} PV = \frac m{\mu} RT \implies \rho = \frac mV = \frac m{\frac{\frac m{\mu} RT}P} = \frac{P\mu}{RT}, \\
    &\text{У нас 2 состояния:} \rho_1 = \frac{P_1\mu}{RT_1}, \rho_2 = \frac{P_2\mu}{RT_2} \implies \frac{\rho_2}{\rho_1} = \frac{\frac{P_2\mu}{RT_2}}{\frac{P_1\mu}{RT_1}} = \frac{P_2T_1}{P_1T_2} \implies \\
    &\implies \rho_2 = \rho_1 \cdot  \frac{P_2T_1}{P_1T_2} = 1{,}3\,\frac{\text{кг}}{\text{м}^{3}} \cdot \frac{150\,\text{кПа} \cdot 273\units{К}}{100\,\text{кПа} \cdot 473\units{К}} \approx 1{,}13\,\frac{\text{кг}}{\text{м}^{3}}.
    \end{align*}
}
\solutionspace{120pt}

\tasknumber{14}%
\task{%
    Небольшую цилиндрическую пробирку с воздухом погружают на некоторую глубину в глубокое пресное озеро,
    после чего воздух занимает в ней лишь третью часть от общего объема.
    Определите глубину, на которую погрузили пробирку.
    Температуру считать постоянной $T = 282\,\text{К}$, давлением паров воды пренебречь,
    атмосферное давление принять равным $p_{\text{aтм}} = 100\,\text{кПа}$.
}
\answer{%
    \begin{align*}
    T\text{— const} &\implies P_1V_1 = \nu RT = P_2V_2.
    \\
    V_2 = \frac 13 V_1 &\implies P_1V_1 = P_2 \cdot \frac 13V_1 \implies P_2 = 3P_1 = 3p_{\text{aтм}}.
    \\
    P_2 = p_{\text{aтм}} + \rho_{\text{в}} g h \implies h = \frac{P_2 - p_{\text{aтм}}}{\rho_{\text{в}} g} &= \frac{3p_{\text{aтм}} - p_{\text{aтм}}}{\rho_{\text{в}} g} = \frac{2 \cdot p_{\text{aтм}}}{\rho_{\text{в}} g} =  \\
     &= \frac{2 \cdot 100\,\text{кПа}}{1000\,\frac{\text{кг}}{\text{м}^{3}} \cdot  10\,\frac{\text{м}}{\text{с}^{2}}} \approx 20\,\text{м}.
    \end{align*}
}
\solutionspace{120pt}

\tasknumber{15}%
\task{%
    Газу сообщили некоторое количество теплоты,
    при этом половину его он потратил на совершение работы,
    одновременно увеличив свою внутреннюю энергию на $1200\,\text{Дж}$.
    Определите количество теплоты, сообщённое газу.
}
\answer{%
    \begin{align*}
    Q &= A' + \Delta U, A' = \frac 12 Q \implies Q \cdot \cbr{1 - \frac 12} = \Delta U \implies Q = \frac{\Delta U}{1 - \frac 12} = \frac{ 1200\,\text{Дж} }{1 - \frac 12} \approx 2400\,\text{Дж}.
    \\
    A' &= \frac 12 Q
        = \frac 12 \cdot \frac{\Delta U}{1 - \frac 12}
        = \frac{\Delta U}{2 - 1}
        = \frac{ 1200\,\text{Дж} }{2 - 1} \approx 1200\,\text{Дж}.
    \end{align*}
}
\solutionspace{60pt}

\tasknumber{16}%
\task{%
    Два конденсатора ёмкостей $C_1 = 30\,\text{нФ}$ и $C_2 = 20\,\text{нФ}$ последовательно подключают
    к источнику напряжения $U = 200\,\text{В}$ (см.
    рис.).
    % Определите заряды каждого из конденсаторов.
    Определите заряд второго конденсатора.

    \begin{tikzpicture}[circuit ee IEC, semithick]
        \draw  (0, 0) to [capacitor={info={$C_1$}}] (1, 0)
                       to [capacitor={info={$C_2$}}] (2, 0)
        ;
        % \draw [-o] (0, 0) -- ++(-0.5, 0) node[left] {$-$};
        % \draw [-o] (2, 0) -- ++(0.5, 0) node[right] {$+$};
        \draw [-o] (0, 0) -- ++(-0.5, 0) node[left] {};
        \draw [-o] (2, 0) -- ++(0.5, 0) node[right] {};
    \end{tikzpicture}
}
\answer{%
    $
        Q_1
            = Q_2
            = CU
            = \frac{ U }{\frac1{C_1} + \frac1{C_2}}
            = \frac{C_1C_2U}{C_1 + C_2}
            = \frac{
                30\,\text{нФ} \cdot 20\,\text{нФ} \cdot 200\,\text{В}
            }{
                30\,\text{нФ} + 20\,\text{нФ}
            }
            = 2{,}40\,\text{мкКл}
    $
}
\solutionspace{120pt}

\tasknumber{17}%
\task{%
    В вакууме вдоль одной прямой расположены три отрицательных заряда так,
    что расстояние между соседними зарядами равно $l$.
    Сделайте рисунок,
    и определите силу, действующую на крайний заряд.
    Модули всех зарядов равны $Q$ ($Q > 0$).
}
\answer{%
    $F = \sum_i F_i = \ldots = \frac54 \frac{kQ^2}{l^2}.$
}
\solutionspace{80pt}

\tasknumber{18}%
\task{%
    Юлия проводит эксперименты c 2 кусками одинаковой медной проволки, причём второй кусок в четыре раза длиннее первого.
    В одном из экспериментов Юлия подаёт на первый кусок проволки напряжение в три раза раз больше, чем на второй.
    Определите отношения в двух проволках в этом эксперименте (второй к первой):
    \begin{itemize}
        \item отношение сил тока,
        \item отношение выделяющихся мощностей.
    \end{itemize}
}
\answer{%
    $R_2 = 4R_1, U_1 = 3U_2 \implies  \eli_2 / \eli_1 = \frac{U_2 / R_2}{U_1 / R_1} = \frac{U_2}{U_1} \cdot \frac{R_1}{R_2} = \frac1{12}, P_2 / P_1 = \frac{U_2^2 / R_2}{U_1^2 / R_1} = \sqr{\frac{U_2}{U_1}} \cdot \frac{R_1}{R_2} = \frac1{36}.$
}

\variantsplitter

\addpersonalvariant{Дмитрий Соколов}

\tasknumber{1}%
\task{%
    Валя стартует на лошади и в течение $t = 5\,\text{c}$ двигается с постоянным ускорением $1{,}5\,\frac{\text{м}}{\text{с}^{2}}$.
    Определите
    \begin{itemize}
        \item какую скорость при этом удастся достичь,
        \item какой путь за это время будет пройден,
        \item среднюю скорость за всё время движения, если после начального ускорения продолжить движение равномерно ещё в течение времени $3t$
    \end{itemize}
}
\answer{%
    \begin{align*}
    v &= v_0 + a t = at = 1{,}5\,\frac{\text{м}}{\text{с}^{2}} \cdot 5\,\text{c} = 7{,}5\,\frac{\text{м}}{\text{с}}, \\
    s_x &= v_0t + \frac{a t^2}2 = \frac{a t^2}2 = \frac{1{,}5\,\frac{\text{м}}{\text{с}^{2}} \cdot \sqr{ 5\,\text{c} }}2 = 18{,}8\,\text{м}, \\
    v_\text{сред.} &= \frac{s_\text{общ}}{t_\text{общ.}} = \frac{s_x + v \cdot 3t}{t + 3t} = \frac{\frac{a t^2}2 + at \cdot 3t}{t (1 + 3)} = \\
    &= at \cdot \frac{\frac 12 + 3}{1 + 3} = 1{,}5\,\frac{\text{м}}{\text{с}^{2}} \cdot 5\,\text{c} \cdot \frac{\frac 12 + 3}{1 + 3} \approx 6{,}56\,\frac{\text{м}}{\text{c}}.
    \end{align*}
}
\solutionspace{120pt}

\tasknumber{2}%
\task{%
    Какой путь тело пройдёт за третью секунду после начала свободного падения?
    Какую скорость в конце этой секунды оно имеет?
}
\answer{%
    \begin{align*}
    s &= -s_y = -(y_2-y_1) = y_1 - y_2 = \cbr{y_{0y} + v_{0y}t_1 - \frac{gt_1^2}2} - \cbr{y_{0y} + v_{0y}t_2 - \frac{gt_2^2}2} = \\
    &= \frac{gt_2^2}2 - \frac{gt_1^2}2 = \frac g2\cbr{t_2^2 - t_1^2} = 25{,}0\,\text{м}, \\
    v_y &= v_{0y} - gt = -gt = 10\,\frac{\text{м}}{\text{с}^{2}} \cdot 3\,\text{с} = -30\,\frac{\text{м}}{\text{с}}.
    \end{align*}
}
\solutionspace{120pt}

\tasknumber{3}%
\task{%
    Карусель диаметром $2\,\text{м}$ равномерно совершает 6 оборотов в минуту.
    Определите
    \begin{itemize}
        \item период и частоту её обращения,
        \item скорость и ускорение крайних её точек.
    \end{itemize}
}
\answer{%
    \begin{align*}
    t &= 60\,\text{с}, r = 1{,}0\,\text{м}, n = 6\units{оборотов}, \\
    T &= \frac tN = \frac{ 60\,\text{с} }{6} \approx 10{,}00\,\text{c}, \\
    \nu &= \frac 1T = \frac{6}{ 60\,\text{с} } \approx 0{,}10\,\text{Гц}, \\
    v &= \frac{2 \pi r}{T} = \frac{2 \pi r}{T} =  \frac{2 \pi r n}{t} \approx 0{,}63\,\frac{\text{м}}{\text{c}}, \\
    a &= \frac{v^2}{r} =  \frac{4 \pi^2 r n^2}{t^2} \approx 0{,}39\,\frac{\text{м}}{\text{с}^{2}}.
    \end{align*}
}
\solutionspace{80pt}

\tasknumber{4}%
\task{%
    Паша стоит на обрыве над рекой и методично и строго горизонтально кидает в неё камушки.
    За этим всем наблюдает экспериментатор Глюк, который уже выяснил, что камушки падают в реку спустя $1{,}7\,\text{с}$ после броска,
    а вот дальность полёта оценить сложнее: придётся лезть в воду.
    Выручите Глюка и определите:
    \begin{itemize}
        \item высоту обрыва (вместе с ростом Паши).
        \item дальность полёта камушков (по горизонтали) и их скорость при падении, приняв начальную скорость броска равной $v_0 = 17\,\frac{\text{м}}{\text{с}}$.
    \end{itemize}
    Сопротивлением воздуха пренебречь.
}
\answer{%
    \begin{align*}
    y &= y_0 + v_{0y}t - \frac{gt^2}2 = h - \frac{gt^2}2, \qquad y(\tau) = 0 \implies h - \frac{g\tau^2}2 = 0 \implies h = \frac{g\tau^2}2 \approx 14{,}4\,\text{м}.
    \\
    x &= x_0 + v_{0x}t = v_0t \implies L = v_0\tau \approx 28{,}9\,\text{м}.
    \\
    &v = \sqrt{v_x^2 + v_y^2} = \sqrt{v_{0x}^2 + \sqr{v_{0y} - g\tau}} = \sqrt{v_0^2 + \sqr{g\tau}} \approx 24{,}0\,\frac{\text{м}}{\text{c}}.
    \end{align*}
}
\solutionspace{120pt}

\tasknumber{5}%
\task{%
    Шесть одинаковых брусков массой $2\,\text{кг}$ каждый лежат на гладком горизонтальном столе.
    Бруски пронумерованы от 1 до 6 и последовательно связаны между собой
    невесомыми нерастяжимыми нитями: 1 со 2, 2 с 3 (ну и с 1) и т.д.
    Экспериментатор Глюк прикладывает постоянную горизонтальную силу $90\,\text{Н}$ к бруску с наибольшим номером.
    С каким ускорением двигается система? Чему равна сила натяжения нити, связывающей бруски 2 и 3?
}
\answer{%
    \begin{align*}
    a &= \frac{F}{6 m} = \frac{90\,\text{Н}}{6 \cdot 2\,\text{кг}} \approx 7{,}5\,\frac{\text{м}}{\text{c}^{2}}, \\
    T &= m'a = 2m \cdot \frac{F}{6 m} = \frac{2}{6} F \approx 30{,}0\,\text{Н}.
    \end{align*}
}
\solutionspace{120pt}

\tasknumber{6}%
\task{%
    Два бруска связаны лёгкой нерастяжимой нитью и перекинуты через неподвижный блок (см.
    рис.).
    Определите силу натяжения нити и ускорения брусков.
    Силами трения пренебречь, массы брусков
    равны $m_1 = 11\,\text{кг}$ и $m_2 = 14\,\text{кг}$.
    % $g = 10\,\frac{\text{м}}{\text{с}^{2}}$.

    \begin{tikzpicture}[x=1.5cm,y=1.5cm,thick]
        \draw
            (-0.4, 0) rectangle (-0.2, 1.2)
            (0.15, 0.5) rectangle (0.45, 1)
            (0, 2) circle [radius=0.3] -- ++(up:0.5)
            (-0.3, 1.2) -- ++(up:0.8)
            (0.3, 1) -- ++(up:1)
            (-0.7, 2.5) -- (0.7, 2.5)
            ;
        \draw[pattern={Lines[angle=51,distance=3pt]},pattern color=black,draw=none] (-0.7, 2.5) rectangle (0.7, 2.75);
        \node [left] (left) at (-0.4, 0.6) { $m_1$ };
        \node [right] (right) at (0.4, 0.75) { $m_2$ };
    \end{tikzpicture}
}
\answer{%
    Предположим, что левый брусок ускоряется вверх, тогда правый ускоряется вниз (с тем же ускорением).
    Запишем 2-й закон Ньютона 2 раза (для обоих тел) в проекции на вертикальную оси, направив её вверх.
    \begin{align*}
        &\begin{cases}
            T - m_1g = m_1a, \\
            T - m_2g = -m_2a,
        \end{cases} \\
        &\begin{cases}
            m_2g - m_1g = m_1a + m_2a, \\
            T = m_1a + m_1g, \\
        \end{cases} \\
        a &= \frac{m_2 - m_1}{m_1 + m_2} \cdot g = \frac{14\,\text{кг} - 11\,\text{кг}}{11\,\text{кг} + 14\,\text{кг}} \cdot 10\,\frac{\text{м}}{\text{с}^{2}} \approx 1{,}20\,\frac{\text{м}}{\text{c}^{2}}, \\
        T &= m_1(a + g) = m_1 \cdot g \cdot \cbr{\frac{m_2 - m_1}{m_1 + m_2} + 1} = m_1 \cdot g \cdot \frac{2m_2}{m_1 + m_2} = \\
            &= \frac{2 m_2 m_1 g}{m_1 + m_2} = \frac{2 \cdot 14\,\text{кг} \cdot 11\,\text{кг} \cdot 10\,\frac{\text{м}}{\text{с}^{2}}}{11\,\text{кг} + 14\,\text{кг}} \approx 123{,}2\,\text{Н}.
    \end{align*}
    Отрицательный ответ говорит, что мы лишь не угадали с направлением ускорений.
    Сила же всегда положительна.
}
\solutionspace{80pt}

\tasknumber{7}%
\task{%
    Тело массой $1{,}4\,\text{кг}$ лежит на горизонтальной поверхности.
    Коэффициент трения между поверхностью и телом $0{,}25$.
    К телу приложена горизонтальная сила $3{,}5\,\text{Н}$.
    Определите силу трения, действующую на тело, и ускорение тела.
    % $g = 10\,\frac{\text{м}}{\text{с}^{2}}$.
}
\answer{%
    \begin{align*}
    &F_\text{ трения покоя $\max$ } = \mu N = \mu m g = 0{,}25 \cdot 1{,}4\,\text{кг} \cdot 10\,\frac{\text{м}}{\text{с}^{2}} = 3{,}50\,\text{Н}, \\
    &F_\text{ трения покоя $\max$ } \le F \implies F_\text{ трения } = 3{,}50\,\text{Н}, a = \frac{F - F_\text{ трения }}{ m } = 0\,\frac{\text{м}}{\text{c}^{2}}, \\
    &\text{при равенстве возможны оба варианта: и едет, и не едет, но на ответы это не влияет.}
    \end{align*}
}
\solutionspace{120pt}

\tasknumber{8}%
\task{%
    Определите плотность неизвестного вещества, если известно, что опускании тела из него
    в подсолнечное масло оно будет плавать и на треть выступать над поверхностью жидкости.
}
\answer{%
    $F_\text{Арх.} = F_\text{тяж.} \implies \rho_\text{ж.} g V_\text{погр.} = m g \implies\rho_\text{ж.} g \cbr{V -\frac V3} = \rho V g \implies \rho = \rho_\text{ж.}\cbr{1 -\frac 13} \approx 600\,\frac{\text{кг}}{\text{м}^{3}}$
}
\solutionspace{120pt}

\tasknumber{9}%
\task{%
    	Определите силу, действующую на левую опору однородного горизонтального стержня длиной $l = 5\,\text{м}$
    	и массой $M = 5\,\text{кг}$, к которому подвешен груз массой $m = 2\,\text{кг}$ на расстоянии $4\,\text{м}$ от правого конца (см.
    рис.).

        \begin{tikzpicture}[thick]
            \draw
                (-2, -0.1) rectangle (2, 0.1)
                (-0.5, -0.1) -- (-0.5, -1)
                (-0.7, -1) rectangle (-0.3, -1.3)
           		(-2, -0.1) -- +(0.15,-0.9) -- +(-0.15,-0.9) -- cycle
            	(2, -0.1) -- +(0.15,-0.9) -- +(-0.15,-0.9) -- cycle
            ;
            \draw[pattern={Lines[angle=51,distance=2pt]},pattern color=black,draw=none]
            	(-2.15, -1.15) rectangle +(0.3, 0.15)
            	(2.15, -1.15) rectangle +(-0.3, 0.15)
            ;
            \node [right] (m_small) at (-0.3, -1.15) { $m$ };
            \node [above] (M_big) at (0, 0.1) { $M$ };
        \end{tikzpicture}
}
\answer{%
    \begin{align*}
        &\begin{cases}
            F_1 + F_2 - mg - Mg= 0, \\
            F_1 \cdot 0 - mg \cdot a - Mg \cdot \frac l2 + F_2 \cdot l = 0,
        \end{cases} \\
        F_2 &= \frac{mga + Mg\frac l2}l = \frac al \cdot mg + \frac{Mg}2 \approx 29{,}0\,\text{Н}, \\
        F_1 &= mg + Mg - F_2 = mg + Mg - \frac al \cdot mg - \frac{Mg}2 = \frac bl \cdot mg + \frac{Mg}2 \approx 41{,}0\,\text{Н}.
    \end{align*}
}
\solutionspace{80pt}

\tasknumber{10}%
\task{%
    Тонкий однородный лом длиной $3\,\text{м}$ и массой $10\,\text{кг}$ лежит на горизонтальной поверхности.
    \begin{itemize}
        \item Какую минимальную силу надо приложить к одному из его концов, чтобы оторвать его от этой поверхности?
        \item Какую минимальную работу надо совершить, чтобы поставить его на землю в вертикальное положение?
    \end{itemize}
    % Примите $g = 10\,\frac{\text{м}}{\text{с}^{2}}$.
}
\answer{%
    $F = \frac{mg}2 \approx 100\,\text{Н}, A = mg\frac l2 = 150\,\text{Дж}$
}
\solutionspace{120pt}

\tasknumber{11}%
\task{%
    Определите работу силы, которая обеспечит подъём тела массой $5\,\text{кг}$ на высоту $5\,\text{м}$ с постоянным ускорением $3\,\frac{\text{м}}{\text{c}^{2}}$.
    % Примите $g = 10\,\frac{\text{м}}{\text{с}^{2}}$.
}
\answer{%
    \begin{align*}
    &\text{Для подъёма:} A = Fh = (mg + ma) h = m(g+a)h, \\
    &\text{Для спуска:} A = -Fh = -(mg - ma) h = -m(g-a)h, \\
    &\text{В результате получаем:} 325\,\text{Дж}.
    \end{align*}
}
\solutionspace{60pt}

\tasknumber{12}%
\task{%
    Тело бросили вертикально вверх со скоростью $14\,\frac{\text{м}}{\text{c}}$.
    На какой высоте кинетическая энергия тела составит треть от потенциальной?
}
\answer{%
    \begin{align*}
    &0 + \frac{mv_0^2}2 = E_p + E_k, E_k = \frac 13 E_p \implies \\
    &\implies \frac{mv_0^2}2 = E_p + \frac 13 E_p = E_p\cbr{1 + \frac 13} = mgh\cbr{1 + \frac 13} \implies \\
    &\implies h = \frac{\frac{mv_0^2}2}{mg\cbr{1 + \frac 13}} = \frac{v_0^2}{2g} \cdot \frac 1{1 + \frac 13} \approx 7{,}4\,\text{м}.
    \end{align*}
}
\solutionspace{100pt}

\tasknumber{13}%
\task{%
    Плотность воздуха при нормальных условиях равна $1{,}3\,\frac{\text{кг}}{\text{м}^{3}}$.
    Чему равна плотность воздуха
    при температуре $150\celsius$ и давлении $150\,\text{кПа}$?
}
\answer{%
    \begin{align*}
    &\text{В общем случае:} PV = \frac m{\mu} RT \implies \rho = \frac mV = \frac m{\frac{\frac m{\mu} RT}P} = \frac{P\mu}{RT}, \\
    &\text{У нас 2 состояния:} \rho_1 = \frac{P_1\mu}{RT_1}, \rho_2 = \frac{P_2\mu}{RT_2} \implies \frac{\rho_2}{\rho_1} = \frac{\frac{P_2\mu}{RT_2}}{\frac{P_1\mu}{RT_1}} = \frac{P_2T_1}{P_1T_2} \implies \\
    &\implies \rho_2 = \rho_1 \cdot  \frac{P_2T_1}{P_1T_2} = 1{,}3\,\frac{\text{кг}}{\text{м}^{3}} \cdot \frac{150\,\text{кПа} \cdot 273\units{К}}{100\,\text{кПа} \cdot 423\units{К}} \approx 1{,}26\,\frac{\text{кг}}{\text{м}^{3}}.
    \end{align*}
}
\solutionspace{120pt}

\tasknumber{14}%
\task{%
    Небольшую цилиндрическую пробирку с воздухом погружают на некоторую глубину в глубокое пресное озеро,
    после чего воздух занимает в ней лишь шестую часть от общего объема.
    Определите глубину, на которую погрузили пробирку.
    Температуру считать постоянной $T = 279\,\text{К}$, давлением паров воды пренебречь,
    атмосферное давление принять равным $p_{\text{aтм}} = 100\,\text{кПа}$.
}
\answer{%
    \begin{align*}
    T\text{— const} &\implies P_1V_1 = \nu RT = P_2V_2.
    \\
    V_2 = \frac 16 V_1 &\implies P_1V_1 = P_2 \cdot \frac 16V_1 \implies P_2 = 6P_1 = 6p_{\text{aтм}}.
    \\
    P_2 = p_{\text{aтм}} + \rho_{\text{в}} g h \implies h = \frac{P_2 - p_{\text{aтм}}}{\rho_{\text{в}} g} &= \frac{6p_{\text{aтм}} - p_{\text{aтм}}}{\rho_{\text{в}} g} = \frac{5 \cdot p_{\text{aтм}}}{\rho_{\text{в}} g} =  \\
     &= \frac{5 \cdot 100\,\text{кПа}}{1000\,\frac{\text{кг}}{\text{м}^{3}} \cdot  10\,\frac{\text{м}}{\text{с}^{2}}} \approx 50\,\text{м}.
    \end{align*}
}
\solutionspace{120pt}

\tasknumber{15}%
\task{%
    Газу сообщили некоторое количество теплоты,
    при этом треть его он потратил на совершение работы,
    одновременно увеличив свою внутреннюю энергию на $1500\,\text{Дж}$.
    Определите работу, совершённую газом.
}
\answer{%
    \begin{align*}
    Q &= A' + \Delta U, A' = \frac 13 Q \implies Q \cdot \cbr{1 - \frac 13} = \Delta U \implies Q = \frac{\Delta U}{1 - \frac 13} = \frac{ 1500\,\text{Дж} }{1 - \frac 13} \approx 2250\,\text{Дж}.
    \\
    A' &= \frac 13 Q
        = \frac 13 \cdot \frac{\Delta U}{1 - \frac 13}
        = \frac{\Delta U}{3 - 1}
        = \frac{ 1500\,\text{Дж} }{3 - 1} \approx 750\,\text{Дж}.
    \end{align*}
}
\solutionspace{60pt}

\tasknumber{16}%
\task{%
    Два конденсатора ёмкостей $C_1 = 60\,\text{нФ}$ и $C_2 = 40\,\text{нФ}$ последовательно подключают
    к источнику напряжения $U = 200\,\text{В}$ (см.
    рис.).
    % Определите заряды каждого из конденсаторов.
    Определите заряд второго конденсатора.

    \begin{tikzpicture}[circuit ee IEC, semithick]
        \draw  (0, 0) to [capacitor={info={$C_1$}}] (1, 0)
                       to [capacitor={info={$C_2$}}] (2, 0)
        ;
        % \draw [-o] (0, 0) -- ++(-0.5, 0) node[left] {$-$};
        % \draw [-o] (2, 0) -- ++(0.5, 0) node[right] {$+$};
        \draw [-o] (0, 0) -- ++(-0.5, 0) node[left] {};
        \draw [-o] (2, 0) -- ++(0.5, 0) node[right] {};
    \end{tikzpicture}
}
\answer{%
    $
        Q_1
            = Q_2
            = CU
            = \frac{ U }{\frac1{C_1} + \frac1{C_2}}
            = \frac{C_1C_2U}{C_1 + C_2}
            = \frac{
                60\,\text{нФ} \cdot 40\,\text{нФ} \cdot 200\,\text{В}
            }{
                60\,\text{нФ} + 40\,\text{нФ}
            }
            = 4{,}80\,\text{мкКл}
    $
}
\solutionspace{120pt}

\tasknumber{17}%
\task{%
    В вакууме вдоль одной прямой расположены три положительных заряда так,
    что расстояние между соседними зарядами равно $d$.
    Сделайте рисунок,
    и определите силу, действующую на крайний заряд.
    Модули всех зарядов равны $q$ ($q > 0$).
}
\answer{%
    $F = \sum_i F_i = \ldots = \frac54 \frac{kq^2}{d^2}.$
}
\solutionspace{80pt}

\tasknumber{18}%
\task{%
    Юлия проводит эксперименты c 2 кусками одинаковой медной проволки, причём второй кусок в два раза длиннее первого.
    В одном из экспериментов Юлия подаёт на первый кусок проволки напряжение в семь раз раз больше, чем на второй.
    Определите отношения в двух проволках в этом эксперименте (второй к первой):
    \begin{itemize}
        \item отношение сил тока,
        \item отношение выделяющихся мощностей.
    \end{itemize}
}
\answer{%
    $R_2 = 2R_1, U_1 = 7U_2 \implies  \eli_2 / \eli_1 = \frac{U_2 / R_2}{U_1 / R_1} = \frac{U_2}{U_1} \cdot \frac{R_1}{R_2} = \frac1{14}, P_2 / P_1 = \frac{U_2^2 / R_2}{U_1^2 / R_1} = \sqr{\frac{U_2}{U_1}} \cdot \frac{R_1}{R_2} = \frac1{98}.$
}

\variantsplitter

\addpersonalvariant{Арсений Трофимов}

\tasknumber{1}%
\task{%
    Валя стартует на велосипеде и в течение $t = 3\,\text{c}$ двигается с постоянным ускорением $2{,}5\,\frac{\text{м}}{\text{с}^{2}}$.
    Определите
    \begin{itemize}
        \item какую скорость при этом удастся достичь,
        \item какой путь за это время будет пройден,
        \item среднюю скорость за всё время движения, если после начального ускорения продолжить движение равномерно ещё в течение времени $3t$
    \end{itemize}
}
\answer{%
    \begin{align*}
    v &= v_0 + a t = at = 2{,}5\,\frac{\text{м}}{\text{с}^{2}} \cdot 3\,\text{c} = 7{,}5\,\frac{\text{м}}{\text{с}}, \\
    s_x &= v_0t + \frac{a t^2}2 = \frac{a t^2}2 = \frac{2{,}5\,\frac{\text{м}}{\text{с}^{2}} \cdot \sqr{ 3\,\text{c} }}2 = 11{,}2\,\text{м}, \\
    v_\text{сред.} &= \frac{s_\text{общ}}{t_\text{общ.}} = \frac{s_x + v \cdot 3t}{t + 3t} = \frac{\frac{a t^2}2 + at \cdot 3t}{t (1 + 3)} = \\
    &= at \cdot \frac{\frac 12 + 3}{1 + 3} = 2{,}5\,\frac{\text{м}}{\text{с}^{2}} \cdot 3\,\text{c} \cdot \frac{\frac 12 + 3}{1 + 3} \approx 6{,}56\,\frac{\text{м}}{\text{c}}.
    \end{align*}
}
\solutionspace{120pt}

\tasknumber{2}%
\task{%
    Какой путь тело пройдёт за четвёртую секунду после начала свободного падения?
    Какую скорость в начале этой секунды оно имеет?
}
\answer{%
    \begin{align*}
    s &= -s_y = -(y_2-y_1) = y_1 - y_2 = \cbr{y_{0y} + v_{0y}t_1 - \frac{gt_1^2}2} - \cbr{y_{0y} + v_{0y}t_2 - \frac{gt_2^2}2} = \\
    &= \frac{gt_2^2}2 - \frac{gt_1^2}2 = \frac g2\cbr{t_2^2 - t_1^2} = 35{,}0\,\text{м}, \\
    v_y &= v_{0y} - gt = -gt = 10\,\frac{\text{м}}{\text{с}^{2}} \cdot 3\,\text{с} = -30\,\frac{\text{м}}{\text{с}}.
    \end{align*}
}
\solutionspace{120pt}

\tasknumber{3}%
\task{%
    Карусель радиусом $4\,\text{м}$ равномерно совершает 5 оборотов в минуту.
    Определите
    \begin{itemize}
        \item период и частоту её обращения,
        \item скорость и ускорение крайних её точек.
    \end{itemize}
}
\answer{%
    \begin{align*}
    t &= 60\,\text{с}, r = 4{,}0\,\text{м}, n = 5\units{оборотов}, \\
    T &= \frac tN = \frac{ 60\,\text{с} }{5} \approx 12{,}00\,\text{c}, \\
    \nu &= \frac 1T = \frac{5}{ 60\,\text{с} } \approx 0{,}08\,\text{Гц}, \\
    v &= \frac{2 \pi r}{T} = \frac{2 \pi r}{T} =  \frac{2 \pi r n}{t} \approx 2{,}09\,\frac{\text{м}}{\text{c}}, \\
    a &= \frac{v^2}{r} =  \frac{4 \pi^2 r n^2}{t^2} \approx 1{,}10\,\frac{\text{м}}{\text{с}^{2}}.
    \end{align*}
}
\solutionspace{80pt}

\tasknumber{4}%
\task{%
    Миша стоит на обрыве над рекой и методично и строго горизонтально кидает в неё камушки.
    За этим всем наблюдает экспериментатор Глюк, который уже выяснил, что камушки падают в реку спустя $1{,}3\,\text{с}$ после броска,
    а вот дальность полёта оценить сложнее: придётся лезть в воду.
    Выручите Глюка и определите:
    \begin{itemize}
        \item высоту обрыва (вместе с ростом Миши).
        \item дальность полёта камушков (по горизонтали) и их скорость при падении, приняв начальную скорость броска равной $v_0 = 15\,\frac{\text{м}}{\text{с}}$.
    \end{itemize}
    Сопротивлением воздуха пренебречь.
}
\answer{%
    \begin{align*}
    y &= y_0 + v_{0y}t - \frac{gt^2}2 = h - \frac{gt^2}2, \qquad y(\tau) = 0 \implies h - \frac{g\tau^2}2 = 0 \implies h = \frac{g\tau^2}2 \approx 8{,}5\,\text{м}.
    \\
    x &= x_0 + v_{0x}t = v_0t \implies L = v_0\tau \approx 19{,}5\,\text{м}.
    \\
    &v = \sqrt{v_x^2 + v_y^2} = \sqrt{v_{0x}^2 + \sqr{v_{0y} - g\tau}} = \sqrt{v_0^2 + \sqr{g\tau}} \approx 19{,}8\,\frac{\text{м}}{\text{c}}.
    \end{align*}
}
\solutionspace{120pt}

\tasknumber{5}%
\task{%
    Четыре одинаковых брусков массой $3\,\text{кг}$ каждый лежат на гладком горизонтальном столе.
    Бруски пронумерованы от 1 до 4 и последовательно связаны между собой
    невесомыми нерастяжимыми нитями: 1 со 2, 2 с 3 (ну и с 1) и т.д.
    Экспериментатор Глюк прикладывает постоянную горизонтальную силу $90\,\text{Н}$ к бруску с наибольшим номером.
    С каким ускорением двигается система? Чему равна сила натяжения нити, связывающей бруски 2 и 3?
}
\answer{%
    \begin{align*}
    a &= \frac{F}{4 m} = \frac{90\,\text{Н}}{4 \cdot 3\,\text{кг}} \approx 7{,}5\,\frac{\text{м}}{\text{c}^{2}}, \\
    T &= m'a = 2m \cdot \frac{F}{4 m} = \frac{2}{4} F \approx 45{,}0\,\text{Н}.
    \end{align*}
}
\solutionspace{120pt}

\tasknumber{6}%
\task{%
    Два бруска связаны лёгкой нерастяжимой нитью и перекинуты через неподвижный блок (см.
    рис.).
    Определите силу натяжения нити и ускорения брусков.
    Силами трения пренебречь, массы брусков
    равны $m_1 = 5\,\text{кг}$ и $m_2 = 4\,\text{кг}$.
    % $g = 10\,\frac{\text{м}}{\text{с}^{2}}$.

    \begin{tikzpicture}[x=1.5cm,y=1.5cm,thick]
        \draw
            (-0.4, 0) rectangle (-0.2, 1.2)
            (0.15, 0.5) rectangle (0.45, 1)
            (0, 2) circle [radius=0.3] -- ++(up:0.5)
            (-0.3, 1.2) -- ++(up:0.8)
            (0.3, 1) -- ++(up:1)
            (-0.7, 2.5) -- (0.7, 2.5)
            ;
        \draw[pattern={Lines[angle=51,distance=3pt]},pattern color=black,draw=none] (-0.7, 2.5) rectangle (0.7, 2.75);
        \node [left] (left) at (-0.4, 0.6) { $m_1$ };
        \node [right] (right) at (0.4, 0.75) { $m_2$ };
    \end{tikzpicture}
}
\answer{%
    Предположим, что левый брусок ускоряется вверх, тогда правый ускоряется вниз (с тем же ускорением).
    Запишем 2-й закон Ньютона 2 раза (для обоих тел) в проекции на вертикальную оси, направив её вверх.
    \begin{align*}
        &\begin{cases}
            T - m_1g = m_1a, \\
            T - m_2g = -m_2a,
        \end{cases} \\
        &\begin{cases}
            m_2g - m_1g = m_1a + m_2a, \\
            T = m_1a + m_1g, \\
        \end{cases} \\
        a &= \frac{m_2 - m_1}{m_1 + m_2} \cdot g = \frac{4\,\text{кг} - 5\,\text{кг}}{5\,\text{кг} + 4\,\text{кг}} \cdot 10\,\frac{\text{м}}{\text{с}^{2}} \approx -1{,}1100\,\frac{\text{м}}{\text{c}^{2}}, \\
        T &= m_1(a + g) = m_1 \cdot g \cdot \cbr{\frac{m_2 - m_1}{m_1 + m_2} + 1} = m_1 \cdot g \cdot \frac{2m_2}{m_1 + m_2} = \\
            &= \frac{2 m_2 m_1 g}{m_1 + m_2} = \frac{2 \cdot 4\,\text{кг} \cdot 5\,\text{кг} \cdot 10\,\frac{\text{м}}{\text{с}^{2}}}{5\,\text{кг} + 4\,\text{кг}} \approx 44{,}4\,\text{Н}.
    \end{align*}
    Отрицательный ответ говорит, что мы лишь не угадали с направлением ускорений.
    Сила же всегда положительна.
}
\solutionspace{80pt}

\tasknumber{7}%
\task{%
    Тело массой $2{,}7\,\text{кг}$ лежит на горизонтальной поверхности.
    Коэффициент трения между поверхностью и телом $0{,}15$.
    К телу приложена горизонтальная сила $4{,}5\,\text{Н}$.
    Определите силу трения, действующую на тело, и ускорение тела.
    % $g = 10\,\frac{\text{м}}{\text{с}^{2}}$.
}
\answer{%
    \begin{align*}
    &F_\text{ трения покоя $\max$ } = \mu N = \mu m g = 0{,}15 \cdot 2{,}7\,\text{кг} \cdot 10\,\frac{\text{м}}{\text{с}^{2}} = 4{,}05\,\text{Н}, \\
    &F_\text{ трения покоя $\max$ } \le F \implies F_\text{ трения } = 4{,}05\,\text{Н}, a = \frac{F - F_\text{ трения }}{ m } = 0{,}17\,\frac{\text{м}}{\text{c}^{2}}, \\
    &\text{при равенстве возможны оба варианта: и едет, и не едет, но на ответы это не влияет.}
    \end{align*}
}
\solutionspace{120pt}

\tasknumber{8}%
\task{%
    Определите плотность неизвестного вещества, если известно, что опускании тела из него
    в керосин оно будет плавать и на треть выступать над поверхностью жидкости.
}
\answer{%
    $F_\text{Арх.} = F_\text{тяж.} \implies \rho_\text{ж.} g V_\text{погр.} = m g \implies\rho_\text{ж.} g \cbr{V -\frac V3} = \rho V g \implies \rho = \rho_\text{ж.}\cbr{1 -\frac 13} \approx 533\,\frac{\text{кг}}{\text{м}^{3}}$
}
\solutionspace{120pt}

\tasknumber{9}%
\task{%
    	Определите силу, действующую на левую опору однородного горизонтального стержня длиной $l = 5\,\text{м}$
    	и массой $M = 5\,\text{кг}$, к которому подвешен груз массой $m = 4\,\text{кг}$ на расстоянии $2\,\text{м}$ от правого конца (см.
    рис.).

        \begin{tikzpicture}[thick]
            \draw
                (-2, -0.1) rectangle (2, 0.1)
                (-0.5, -0.1) -- (-0.5, -1)
                (-0.7, -1) rectangle (-0.3, -1.3)
           		(-2, -0.1) -- +(0.15,-0.9) -- +(-0.15,-0.9) -- cycle
            	(2, -0.1) -- +(0.15,-0.9) -- +(-0.15,-0.9) -- cycle
            ;
            \draw[pattern={Lines[angle=51,distance=2pt]},pattern color=black,draw=none]
            	(-2.15, -1.15) rectangle +(0.3, 0.15)
            	(2.15, -1.15) rectangle +(-0.3, 0.15)
            ;
            \node [right] (m_small) at (-0.3, -1.15) { $m$ };
            \node [above] (M_big) at (0, 0.1) { $M$ };
        \end{tikzpicture}
}
\answer{%
    \begin{align*}
        &\begin{cases}
            F_1 + F_2 - mg - Mg= 0, \\
            F_1 \cdot 0 - mg \cdot a - Mg \cdot \frac l2 + F_2 \cdot l = 0,
        \end{cases} \\
        F_2 &= \frac{mga + Mg\frac l2}l = \frac al \cdot mg + \frac{Mg}2 \approx 49{,}0\,\text{Н}, \\
        F_1 &= mg + Mg - F_2 = mg + Mg - \frac al \cdot mg - \frac{Mg}2 = \frac bl \cdot mg + \frac{Mg}2 \approx 41{,}0\,\text{Н}.
    \end{align*}
}
\solutionspace{80pt}

\tasknumber{10}%
\task{%
    Тонкий однородный шест длиной $2\,\text{м}$ и массой $20\,\text{кг}$ лежит на горизонтальной поверхности.
    \begin{itemize}
        \item Какую минимальную силу надо приложить к одному из его концов, чтобы оторвать его от этой поверхности?
        \item Какую минимальную работу надо совершить, чтобы поставить его на землю в вертикальное положение?
    \end{itemize}
    % Примите $g = 10\,\frac{\text{м}}{\text{с}^{2}}$.
}
\answer{%
    $F = \frac{mg}2 \approx 200\,\text{Н}, A = mg\frac l2 = 200\,\text{Дж}$
}
\solutionspace{120pt}

\tasknumber{11}%
\task{%
    Определите работу силы, которая обеспечит спуск тела массой $3\,\text{кг}$ на высоту $2\,\text{м}$ с постоянным ускорением $6\,\frac{\text{м}}{\text{c}^{2}}$.
    % Примите $g = 10\,\frac{\text{м}}{\text{с}^{2}}$.
}
\answer{%
    \begin{align*}
    &\text{Для подъёма:} A = Fh = (mg + ma) h = m(g+a)h, \\
    &\text{Для спуска:} A = -Fh = -(mg - ma) h = -m(g-a)h, \\
    &\text{В результате получаем:} -24\,\text{Дж}.
    \end{align*}
}
\solutionspace{60pt}

\tasknumber{12}%
\task{%
    Тело бросили вертикально вверх со скоростью $14\,\frac{\text{м}}{\text{c}}$.
    На какой высоте кинетическая энергия тела составит треть от потенциальной?
}
\answer{%
    \begin{align*}
    &0 + \frac{mv_0^2}2 = E_p + E_k, E_k = \frac 13 E_p \implies \\
    &\implies \frac{mv_0^2}2 = E_p + \frac 13 E_p = E_p\cbr{1 + \frac 13} = mgh\cbr{1 + \frac 13} \implies \\
    &\implies h = \frac{\frac{mv_0^2}2}{mg\cbr{1 + \frac 13}} = \frac{v_0^2}{2g} \cdot \frac 1{1 + \frac 13} \approx 7{,}4\,\text{м}.
    \end{align*}
}
\solutionspace{100pt}

\tasknumber{13}%
\task{%
    Плотность воздуха при нормальных условиях равна $1{,}3\,\frac{\text{кг}}{\text{м}^{3}}$.
    Чему равна плотность воздуха
    при температуре $100\celsius$ и давлении $80\,\text{кПа}$?
}
\answer{%
    \begin{align*}
    &\text{В общем случае:} PV = \frac m{\mu} RT \implies \rho = \frac mV = \frac m{\frac{\frac m{\mu} RT}P} = \frac{P\mu}{RT}, \\
    &\text{У нас 2 состояния:} \rho_1 = \frac{P_1\mu}{RT_1}, \rho_2 = \frac{P_2\mu}{RT_2} \implies \frac{\rho_2}{\rho_1} = \frac{\frac{P_2\mu}{RT_2}}{\frac{P_1\mu}{RT_1}} = \frac{P_2T_1}{P_1T_2} \implies \\
    &\implies \rho_2 = \rho_1 \cdot  \frac{P_2T_1}{P_1T_2} = 1{,}3\,\frac{\text{кг}}{\text{м}^{3}} \cdot \frac{80\,\text{кПа} \cdot 273\units{К}}{100\,\text{кПа} \cdot 373\units{К}} \approx 0{,}76\,\frac{\text{кг}}{\text{м}^{3}}.
    \end{align*}
}
\solutionspace{120pt}

\tasknumber{14}%
\task{%
    Небольшую цилиндрическую пробирку с воздухом погружают на некоторую глубину в глубокое пресное озеро,
    после чего воздух занимает в ней лишь третью часть от общего объема.
    Определите глубину, на которую погрузили пробирку.
    Температуру считать постоянной $T = 288\,\text{К}$, давлением паров воды пренебречь,
    атмосферное давление принять равным $p_{\text{aтм}} = 100\,\text{кПа}$.
}
\answer{%
    \begin{align*}
    T\text{— const} &\implies P_1V_1 = \nu RT = P_2V_2.
    \\
    V_2 = \frac 13 V_1 &\implies P_1V_1 = P_2 \cdot \frac 13V_1 \implies P_2 = 3P_1 = 3p_{\text{aтм}}.
    \\
    P_2 = p_{\text{aтм}} + \rho_{\text{в}} g h \implies h = \frac{P_2 - p_{\text{aтм}}}{\rho_{\text{в}} g} &= \frac{3p_{\text{aтм}} - p_{\text{aтм}}}{\rho_{\text{в}} g} = \frac{2 \cdot p_{\text{aтм}}}{\rho_{\text{в}} g} =  \\
     &= \frac{2 \cdot 100\,\text{кПа}}{1000\,\frac{\text{кг}}{\text{м}^{3}} \cdot  10\,\frac{\text{м}}{\text{с}^{2}}} \approx 20\,\text{м}.
    \end{align*}
}
\solutionspace{120pt}

\tasknumber{15}%
\task{%
    Газу сообщили некоторое количество теплоты,
    при этом четверть его он потратил на совершение работы,
    одновременно увеличив свою внутреннюю энергию на $1500\,\text{Дж}$.
    Определите работу, совершённую газом.
}
\answer{%
    \begin{align*}
    Q &= A' + \Delta U, A' = \frac 14 Q \implies Q \cdot \cbr{1 - \frac 14} = \Delta U \implies Q = \frac{\Delta U}{1 - \frac 14} = \frac{ 1500\,\text{Дж} }{1 - \frac 14} \approx 2000\,\text{Дж}.
    \\
    A' &= \frac 14 Q
        = \frac 14 \cdot \frac{\Delta U}{1 - \frac 14}
        = \frac{\Delta U}{4 - 1}
        = \frac{ 1500\,\text{Дж} }{4 - 1} \approx 500\,\text{Дж}.
    \end{align*}
}
\solutionspace{60pt}

\tasknumber{16}%
\task{%
    Два конденсатора ёмкостей $C_1 = 40\,\text{нФ}$ и $C_2 = 60\,\text{нФ}$ последовательно подключают
    к источнику напряжения $U = 450\,\text{В}$ (см.
    рис.).
    % Определите заряды каждого из конденсаторов.
    Определите заряд первого конденсатора.

    \begin{tikzpicture}[circuit ee IEC, semithick]
        \draw  (0, 0) to [capacitor={info={$C_1$}}] (1, 0)
                       to [capacitor={info={$C_2$}}] (2, 0)
        ;
        % \draw [-o] (0, 0) -- ++(-0.5, 0) node[left] {$-$};
        % \draw [-o] (2, 0) -- ++(0.5, 0) node[right] {$+$};
        \draw [-o] (0, 0) -- ++(-0.5, 0) node[left] {};
        \draw [-o] (2, 0) -- ++(0.5, 0) node[right] {};
    \end{tikzpicture}
}
\answer{%
    $
        Q_1
            = Q_2
            = CU
            = \frac{ U }{\frac1{C_1} + \frac1{C_2}}
            = \frac{C_1C_2U}{C_1 + C_2}
            = \frac{
                40\,\text{нФ} \cdot 60\,\text{нФ} \cdot 450\,\text{В}
            }{
                40\,\text{нФ} + 60\,\text{нФ}
            }
            = 10{,}80\,\text{мкКл}
    $
}
\solutionspace{120pt}

\tasknumber{17}%
\task{%
    В вакууме вдоль одной прямой расположены три отрицательных заряда так,
    что расстояние между соседними зарядами равно $l$.
    Сделайте рисунок,
    и определите силу, действующую на крайний заряд.
    Модули всех зарядов равны $q$ ($q > 0$).
}
\answer{%
    $F = \sum_i F_i = \ldots = \frac54 \frac{kq^2}{l^2}.$
}
\solutionspace{80pt}

\tasknumber{18}%
\task{%
    Юлия проводит эксперименты c 2 кусками одинаковой стальной проволки, причём второй кусок в девять раз длиннее первого.
    В одном из экспериментов Юлия подаёт на первый кусок проволки напряжение в девять раз раз больше, чем на второй.
    Определите отношения в двух проволках в этом эксперименте (второй к первой):
    \begin{itemize}
        \item отношение сил тока,
        \item отношение выделяющихся мощностей.
    \end{itemize}
}
\answer{%
    $R_2 = 9R_1, U_1 = 9U_2 \implies  \eli_2 / \eli_1 = \frac{U_2 / R_2}{U_1 / R_1} = \frac{U_2}{U_1} \cdot \frac{R_1}{R_2} = \frac1{81}, P_2 / P_1 = \frac{U_2^2 / R_2}{U_1^2 / R_1} = \sqr{\frac{U_2}{U_1}} \cdot \frac{R_1}{R_2} = \frac1{729}.$
}
% autogenerated
