\setdate{30~апреля~2021}
\setclass{9«М»}

\addpersonalvariant{Михаил Бурмистров}

\tasknumber{1}%
\task{%
    В ядре электрически нейтрального атома 121 частиц.
    Вокруг ядра обращается 51 электронов.
    Сколько в ядре этого атома протонов и нейтронов?
    Назовите этот элемент.
}
\answer{%
    $Z = 51$ протонов и $A - Z = 70$ нейтронов, так что это \text{сурьма-121}: $\ce{^{121}_{51}{Sb}}$
}
\solutionspace{120pt}

\tasknumber{2}%
\task{%
    Запишите реакцию $\beta$-распада $\ce{^{22}_{11}{Na}}$.
}
\answer{%
    $\ce{^{22}_{11}{Na}} \to \ce{^{22}_{12}{Mg}} + e^- + \tilde\nu_e$
}
\solutionspace{80pt}

\tasknumber{3}%
\task{%
    Энергия связи ядра трития \ce{^{3}_{1}H} (T) равна $8{,}48\,\text{МэВ}$.
    Найти дефект массы этого ядра.
    Ответ выразите в а.е.м.
    и кг.
    Скорость света $c = 2{,}998 \cdot 10^{8}\,\frac{\text{м}}{\text{с}}$, элементарный заряд $e = 1{,}6 \cdot 10^{-19}\,\text{Кл}$.
}
\answer{%
    \begin{align*}
    E_\text{св.} &= \Delta m c^2 \implies \\
    \implies
            \Delta m &= \frac {E_\text{св.}}{c^2} = \frac{ 8{,}48\,\text{МэВ} }{ \sqr{ 2{,}998 \cdot 10^{8}\,\frac{\text{м}}{\text{с}} } }
            = \frac{8{,}48 \cdot 10^6 \cdot 1{,}6 \cdot 10^{-19}\,\text{Дж}}{ \sqr{ 2{,}998 \cdot 10^{8}\,\frac{\text{м}}{\text{с}} } }
            \approx 1{,}510 \cdot 10^{-29}\,\text{кг} \approx 0{,}00909\,\text{а.е.м.}
    \end{align*}
}
\solutionspace{100pt}

\tasknumber{4}%
\task{%
    Определите деффект массы (в а.е.м.) и энергию связи (в МэВ) ядра атома \ce{^{6}_{2}{He}},
    если его масса составляет $6{,}0189\,\text{а.е.м.}$.
    Считать $m_{p} = 1{,}00728\,\text{а.е.м.}$, $m_{n} = 1{,}00867\,\text{а.е.м.}$.
}
\answer{%
    \begin{align*}
    \Delta m &= (A - Z)m_{n} + Zm_{p} - m = 4 \cdot 1{,}00867\,\text{а.е.м.} + 2 \cdot 1{,}00728\,\text{а.е.м.} - 6{,}0189\,\text{а.е.м.} \approx 0{,}03034\,\text{а.е.м.} \\
    E_\text{ св.
    } &= \Delta m c^2 \approx 0{,}0303 \cdot 931{,}5\,\text{МэВ} \approx 28{,}26\,\text{МэВ}
    \end{align*}
}
\solutionspace{150pt}

\tasknumber{5}%
\task{%
    Сделайте схематичный рисунок энергетических уровней атома водорода
    и отметьте на нём первый (основной) уровень и последующие.
    Сколько различных длин волн может испустить атом водорода,
    находящийся в 4-м возбуждённом состоянии?
    Отметьте все соответствующие переходы на рисунке и укажите,
    при каком переходе (среди отмеченных) длина волны излучённого фотона максимальна.
}
\answer{%
    $N = 6{,}0, \text{самая короткая линия}$
}

\variantsplitter

\addpersonalvariant{Артём Глембо}

\tasknumber{1}%
\task{%
    В ядре электрически нейтрального атома 123 частиц.
    Вокруг ядра обращается 51 электронов.
    Сколько в ядре этого атома протонов и нейтронов?
    Назовите этот элемент.
}
\answer{%
    $Z = 51$ протонов и $A - Z = 72$ нейтронов, так что это \text{сурьма-123}: $\ce{^{123}_{51}{Sb}}$
}
\solutionspace{120pt}

\tasknumber{2}%
\task{%
    Запишите реакцию $\beta$-распада $\ce{^{22}_{11}{Na}}$.
}
\answer{%
    $\ce{^{22}_{11}{Na}} \to \ce{^{22}_{12}{Mg}} + e^- + \tilde\nu_e$
}
\solutionspace{80pt}

\tasknumber{3}%
\task{%
    Энергия связи ядра дейтерия \ce{^{2}_{1}H} (D) равна $2{,}22\,\text{МэВ}$.
    Найти дефект массы этого ядра.
    Ответ выразите в а.е.м.
    и кг.
    Скорость света $c = 2{,}998 \cdot 10^{8}\,\frac{\text{м}}{\text{с}}$, элементарный заряд $e = 1{,}6 \cdot 10^{-19}\,\text{Кл}$.
}
\answer{%
    \begin{align*}
    E_\text{св.} &= \Delta m c^2 \implies \\
    \implies
            \Delta m &= \frac {E_\text{св.}}{c^2} = \frac{ 2{,}22\,\text{МэВ} }{ \sqr{ 2{,}998 \cdot 10^{8}\,\frac{\text{м}}{\text{с}} } }
            = \frac{2{,}22 \cdot 10^6 \cdot 1{,}6 \cdot 10^{-19}\,\text{Дж}}{ \sqr{ 2{,}998 \cdot 10^{8}\,\frac{\text{м}}{\text{с}} } }
            \approx 0{,}395 \cdot 10^{-29}\,\text{кг} \approx 0{,}00238\,\text{а.е.м.}
    \end{align*}
}
\solutionspace{100pt}

\tasknumber{4}%
\task{%
    Определите деффект массы (в а.е.м.) и энергию связи (в МэВ) ядра атома \ce{^{3}_{2}{He}},
    если его масса составляет $3{,}01603\,\text{а.е.м.}$.
    Считать $m_{p} = 1{,}00728\,\text{а.е.м.}$, $m_{n} = 1{,}00867\,\text{а.е.м.}$.
}
\answer{%
    \begin{align*}
    \Delta m &= (A - Z)m_{n} + Zm_{p} - m = 1 \cdot 1{,}00867\,\text{а.е.м.} + 2 \cdot 1{,}00728\,\text{а.е.м.} - 3{,}01603\,\text{а.е.м.} \approx 0{,}00720\,\text{а.е.м.} \\
    E_\text{ св.
    } &= \Delta m c^2 \approx 0{,}0072 \cdot 931{,}5\,\text{МэВ} \approx 6{,}71\,\text{МэВ}
    \end{align*}
}
\solutionspace{150pt}

\tasknumber{5}%
\task{%
    Сделайте схематичный рисунок энергетических уровней атома водорода
    и отметьте на нём первый (основной) уровень и последующие.
    Сколько различных длин волн может испустить атом водорода,
    находящийся в 3-м возбуждённом состоянии?
    Отметьте все соответствующие переходы на рисунке и укажите,
    при каком переходе (среди отмеченных) частота излучённого фотона минимальна.
}
\answer{%
    $N = 3{,}0, \text{самая короткая линия}$
}

\variantsplitter

\addpersonalvariant{Наталья Гончарова}

\tasknumber{1}%
\task{%
    В ядре электрически нейтрального атома 123 частиц.
    Вокруг ядра обращается 51 электронов.
    Сколько в ядре этого атома протонов и нейтронов?
    Назовите этот элемент.
}
\answer{%
    $Z = 51$ протонов и $A - Z = 72$ нейтронов, так что это \text{сурьма-123}: $\ce{^{123}_{51}{Sb}}$
}
\solutionspace{120pt}

\tasknumber{2}%
\task{%
    Запишите реакцию $\beta$-распада $\ce{^{22}_{11}{Na}}$.
}
\answer{%
    $\ce{^{22}_{11}{Na}} \to \ce{^{22}_{12}{Mg}} + e^- + \tilde\nu_e$
}
\solutionspace{80pt}

\tasknumber{3}%
\task{%
    Энергия связи ядра лития \ce{^{6}_{3}Li} равна $31{,}99\,\text{МэВ}$.
    Найти дефект массы этого ядра.
    Ответ выразите в а.е.м.
    и кг.
    Скорость света $c = 2{,}998 \cdot 10^{8}\,\frac{\text{м}}{\text{с}}$, элементарный заряд $e = 1{,}6 \cdot 10^{-19}\,\text{Кл}$.
}
\answer{%
    \begin{align*}
    E_\text{св.} &= \Delta m c^2 \implies \\
    \implies
            \Delta m &= \frac {E_\text{св.}}{c^2} = \frac{ 31{,}99\,\text{МэВ} }{ \sqr{ 2{,}998 \cdot 10^{8}\,\frac{\text{м}}{\text{с}} } }
            = \frac{31{,}99 \cdot 10^6 \cdot 1{,}6 \cdot 10^{-19}\,\text{Дж}}{ \sqr{ 2{,}998 \cdot 10^{8}\,\frac{\text{м}}{\text{с}} } }
            \approx 5{,}69 \cdot 10^{-29}\,\text{кг} \approx 0{,}0343\,\text{а.е.м.}
    \end{align*}
}
\solutionspace{100pt}

\tasknumber{4}%
\task{%
    Определите деффект массы (в а.е.м.) и энергию связи (в МэВ) ядра атома \ce{^{4}_{2}{He}},
    если его масса составляет $4{,}0026\,\text{а.е.м.}$.
    Считать $m_{p} = 1{,}00728\,\text{а.е.м.}$, $m_{n} = 1{,}00867\,\text{а.е.м.}$.
}
\answer{%
    \begin{align*}
    \Delta m &= (A - Z)m_{n} + Zm_{p} - m = 2 \cdot 1{,}00867\,\text{а.е.м.} + 2 \cdot 1{,}00728\,\text{а.е.м.} - 4{,}0026\,\text{а.е.м.} \approx 0{,}02930\,\text{а.е.м.} \\
    E_\text{ св.
    } &= \Delta m c^2 \approx 0{,}0293 \cdot 931{,}5\,\text{МэВ} \approx 27{,}29\,\text{МэВ}
    \end{align*}
}
\solutionspace{150pt}

\tasknumber{5}%
\task{%
    Сделайте схематичный рисунок энергетических уровней атома водорода
    и отметьте на нём первый (основной) уровень и последующие.
    Сколько различных длин волн может испустить атом водорода,
    находящийся в 5-м возбуждённом состоянии?
    Отметьте все соответствующие переходы на рисунке и укажите,
    при каком переходе (среди отмеченных) энергия излучённого фотона максимальна.
}
\answer{%
    $N = 10{,}0, \text{самая длинная линия}$
}

\variantsplitter

\addpersonalvariant{Файёзбек Касымов}

\tasknumber{1}%
\task{%
    В ядре электрически нейтрального атома 123 частиц.
    Вокруг ядра обращается 51 электронов.
    Сколько в ядре этого атома протонов и нейтронов?
    Назовите этот элемент.
}
\answer{%
    $Z = 51$ протонов и $A - Z = 72$ нейтронов, так что это \text{сурьма-123}: $\ce{^{123}_{51}{Sb}}$
}
\solutionspace{120pt}

\tasknumber{2}%
\task{%
    Запишите реакцию $\beta$-распада $\ce{^{137}_{55}{Cs}}$.
}
\answer{%
    $\ce{^{137}_{55}{Cs}} \to \ce{^{137}_{56}{Ba}} + e^- + \tilde\nu_e$
}
\solutionspace{80pt}

\tasknumber{3}%
\task{%
    Энергия связи ядра азота \ce{^{14}_{7}N} равна $115{,}5\,\text{МэВ}$.
    Найти дефект массы этого ядра.
    Ответ выразите в а.е.м.
    и кг.
    Скорость света $c = 2{,}998 \cdot 10^{8}\,\frac{\text{м}}{\text{с}}$, элементарный заряд $e = 1{,}6 \cdot 10^{-19}\,\text{Кл}$.
}
\answer{%
    \begin{align*}
    E_\text{св.} &= \Delta m c^2 \implies \\
    \implies
            \Delta m &= \frac {E_\text{св.}}{c^2} = \frac{ 115{,}5\,\text{МэВ} }{ \sqr{ 2{,}998 \cdot 10^{8}\,\frac{\text{м}}{\text{с}} } }
            = \frac{115{,}5 \cdot 10^6 \cdot 1{,}6 \cdot 10^{-19}\,\text{Дж}}{ \sqr{ 2{,}998 \cdot 10^{8}\,\frac{\text{м}}{\text{с}} } }
            \approx 20{,}6 \cdot 10^{-29}\,\text{кг} \approx 0{,}1238\,\text{а.е.м.}
    \end{align*}
}
\solutionspace{100pt}

\tasknumber{4}%
\task{%
    Определите деффект массы (в а.е.м.) и энергию связи (в МэВ) ядра атома \ce{^{6}_{2}{He}},
    если его масса составляет $6{,}0189\,\text{а.е.м.}$.
    Считать $m_{p} = 1{,}00728\,\text{а.е.м.}$, $m_{n} = 1{,}00867\,\text{а.е.м.}$.
}
\answer{%
    \begin{align*}
    \Delta m &= (A - Z)m_{n} + Zm_{p} - m = 4 \cdot 1{,}00867\,\text{а.е.м.} + 2 \cdot 1{,}00728\,\text{а.е.м.} - 6{,}0189\,\text{а.е.м.} \approx 0{,}03034\,\text{а.е.м.} \\
    E_\text{ св.
    } &= \Delta m c^2 \approx 0{,}0303 \cdot 931{,}5\,\text{МэВ} \approx 28{,}26\,\text{МэВ}
    \end{align*}
}
\solutionspace{150pt}

\tasknumber{5}%
\task{%
    Сделайте схематичный рисунок энергетических уровней атома водорода
    и отметьте на нём первый (основной) уровень и последующие.
    Сколько различных длин волн может испустить атом водорода,
    находящийся в 4-м возбуждённом состоянии?
    Отметьте все соответствующие переходы на рисунке и укажите,
    при каком переходе (среди отмеченных) длина волны излучённого фотона максимальна.
}
\answer{%
    $N = 6{,}0, \text{самая короткая линия}$
}

\variantsplitter

\addpersonalvariant{Александр Козинец}

\tasknumber{1}%
\task{%
    В ядре электрически нейтрального атома 190 частиц.
    Вокруг ядра обращается 78 электронов.
    Сколько в ядре этого атома протонов и нейтронов?
    Назовите этот элемент.
}
\answer{%
    $Z = 78$ протонов и $A - Z = 112$ нейтронов, так что это \text{платина-190}: $\ce{^{190}_{78}{Pt}}$
}
\solutionspace{120pt}

\tasknumber{2}%
\task{%
    Запишите реакцию $\alpha$-распада $\ce{^{153}_{63}{Eu}}$.
}
\answer{%
    $\ce{^{153}_{63}{Eu}} \to \ce{^{149}_{61}{Pm}} + \ce{ ^4_2 He }$
}
\solutionspace{80pt}

\tasknumber{3}%
\task{%
    Энергия связи ядра углерода \ce{^{13}_{6}C} равна $97{,}1\,\text{МэВ}$.
    Найти дефект массы этого ядра.
    Ответ выразите в а.е.м.
    и кг.
    Скорость света $c = 2{,}998 \cdot 10^{8}\,\frac{\text{м}}{\text{с}}$, элементарный заряд $e = 1{,}6 \cdot 10^{-19}\,\text{Кл}$.
}
\answer{%
    \begin{align*}
    E_\text{св.} &= \Delta m c^2 \implies \\
    \implies
            \Delta m &= \frac {E_\text{св.}}{c^2} = \frac{ 97{,}1\,\text{МэВ} }{ \sqr{ 2{,}998 \cdot 10^{8}\,\frac{\text{м}}{\text{с}} } }
            = \frac{97{,}1 \cdot 10^6 \cdot 1{,}6 \cdot 10^{-19}\,\text{Дж}}{ \sqr{ 2{,}998 \cdot 10^{8}\,\frac{\text{м}}{\text{с}} } }
            \approx 17{,}29 \cdot 10^{-29}\,\text{кг} \approx 0{,}1041\,\text{а.е.м.}
    \end{align*}
}
\solutionspace{100pt}

\tasknumber{4}%
\task{%
    Определите деффект массы (в а.е.м.) и энергию связи (в МэВ) ядра атома \ce{^{8}_{2}{He}},
    если его масса составляет $8{,}0225\,\text{а.е.м.}$.
    Считать $m_{p} = 1{,}00728\,\text{а.е.м.}$, $m_{n} = 1{,}00867\,\text{а.е.м.}$.
}
\answer{%
    \begin{align*}
    \Delta m &= (A - Z)m_{n} + Zm_{p} - m = 6 \cdot 1{,}00867\,\text{а.е.м.} + 2 \cdot 1{,}00728\,\text{а.е.м.} - 8{,}0225\,\text{а.е.м.} \approx 0{,}04408\,\text{а.е.м.} \\
    E_\text{ св.
    } &= \Delta m c^2 \approx 0{,}0441 \cdot 931{,}5\,\text{МэВ} \approx 41{,}06\,\text{МэВ}
    \end{align*}
}
\solutionspace{150pt}

\tasknumber{5}%
\task{%
    Сделайте схематичный рисунок энергетических уровней атома водорода
    и отметьте на нём первый (основной) уровень и последующие.
    Сколько различных длин волн может испустить атом водорода,
    находящийся в 4-м возбуждённом состоянии?
    Отметьте все соответствующие переходы на рисунке и укажите,
    при каком переходе (среди отмеченных) энергия излучённого фотона минимальна.
}
\answer{%
    $N = 6{,}0, \text{самая короткая линия}$
}

\variantsplitter

\addpersonalvariant{Андрей Куликовский}

\tasknumber{1}%
\task{%
    В ядре электрически нейтрального атома 63 частиц.
    Вокруг ядра обращается 29 электронов.
    Сколько в ядре этого атома протонов и нейтронов?
    Назовите этот элемент.
}
\answer{%
    $Z = 29$ протонов и $A - Z = 34$ нейтронов, так что это \text{медь-63}: $\ce{^{63}_{29}{Cu}}$
}
\solutionspace{120pt}

\tasknumber{2}%
\task{%
    Запишите реакцию $\alpha$-распада $\ce{^{147}_{62}{Sm}}$.
}
\answer{%
    $\ce{^{147}_{62}{Sm}} \to \ce{^{143}_{60}{Nd}} + \ce{ ^4_2 He }$
}
\solutionspace{80pt}

\tasknumber{3}%
\task{%
    Энергия связи ядра лития \ce{^{6}_{3}Li} равна $31{,}99\,\text{МэВ}$.
    Найти дефект массы этого ядра.
    Ответ выразите в а.е.м.
    и кг.
    Скорость света $c = 2{,}998 \cdot 10^{8}\,\frac{\text{м}}{\text{с}}$, элементарный заряд $e = 1{,}6 \cdot 10^{-19}\,\text{Кл}$.
}
\answer{%
    \begin{align*}
    E_\text{св.} &= \Delta m c^2 \implies \\
    \implies
            \Delta m &= \frac {E_\text{св.}}{c^2} = \frac{ 31{,}99\,\text{МэВ} }{ \sqr{ 2{,}998 \cdot 10^{8}\,\frac{\text{м}}{\text{с}} } }
            = \frac{31{,}99 \cdot 10^6 \cdot 1{,}6 \cdot 10^{-19}\,\text{Дж}}{ \sqr{ 2{,}998 \cdot 10^{8}\,\frac{\text{м}}{\text{с}} } }
            \approx 5{,}69 \cdot 10^{-29}\,\text{кг} \approx 0{,}0343\,\text{а.е.м.}
    \end{align*}
}
\solutionspace{100pt}

\tasknumber{4}%
\task{%
    Определите деффект массы (в а.е.м.) и энергию связи (в МэВ) ядра атома \ce{^{6}_{2}{He}},
    если его масса составляет $6{,}0189\,\text{а.е.м.}$.
    Считать $m_{p} = 1{,}00728\,\text{а.е.м.}$, $m_{n} = 1{,}00867\,\text{а.е.м.}$.
}
\answer{%
    \begin{align*}
    \Delta m &= (A - Z)m_{n} + Zm_{p} - m = 4 \cdot 1{,}00867\,\text{а.е.м.} + 2 \cdot 1{,}00728\,\text{а.е.м.} - 6{,}0189\,\text{а.е.м.} \approx 0{,}03034\,\text{а.е.м.} \\
    E_\text{ св.
    } &= \Delta m c^2 \approx 0{,}0303 \cdot 931{,}5\,\text{МэВ} \approx 28{,}26\,\text{МэВ}
    \end{align*}
}
\solutionspace{150pt}

\tasknumber{5}%
\task{%
    Сделайте схематичный рисунок энергетических уровней атома водорода
    и отметьте на нём первый (основной) уровень и последующие.
    Сколько различных длин волн может испустить атом водорода,
    находящийся в 4-м возбуждённом состоянии?
    Отметьте все соответствующие переходы на рисунке и укажите,
    при каком переходе (среди отмеченных) частота излучённого фотона минимальна.
}
\answer{%
    $N = 6{,}0, \text{самая короткая линия}$
}

\variantsplitter

\addpersonalvariant{Полина Лоткова}

\tasknumber{1}%
\task{%
    В ядре электрически нейтрального атома 121 частиц.
    Вокруг ядра обращается 51 электронов.
    Сколько в ядре этого атома протонов и нейтронов?
    Назовите этот элемент.
}
\answer{%
    $Z = 51$ протонов и $A - Z = 70$ нейтронов, так что это \text{сурьма-121}: $\ce{^{121}_{51}{Sb}}$
}
\solutionspace{120pt}

\tasknumber{2}%
\task{%
    Запишите реакцию $\alpha$-распада $\ce{^{147}_{62}{Sm}}$.
}
\answer{%
    $\ce{^{147}_{62}{Sm}} \to \ce{^{143}_{60}{Nd}} + \ce{ ^4_2 He }$
}
\solutionspace{80pt}

\tasknumber{3}%
\task{%
    Энергия связи ядра бора \ce{^{11}_{5}B} равна $76{,}2\,\text{МэВ}$.
    Найти дефект массы этого ядра.
    Ответ выразите в а.е.м.
    и кг.
    Скорость света $c = 2{,}998 \cdot 10^{8}\,\frac{\text{м}}{\text{с}}$, элементарный заряд $e = 1{,}6 \cdot 10^{-19}\,\text{Кл}$.
}
\answer{%
    \begin{align*}
    E_\text{св.} &= \Delta m c^2 \implies \\
    \implies
            \Delta m &= \frac {E_\text{св.}}{c^2} = \frac{ 76{,}2\,\text{МэВ} }{ \sqr{ 2{,}998 \cdot 10^{8}\,\frac{\text{м}}{\text{с}} } }
            = \frac{76{,}2 \cdot 10^6 \cdot 1{,}6 \cdot 10^{-19}\,\text{Дж}}{ \sqr{ 2{,}998 \cdot 10^{8}\,\frac{\text{м}}{\text{с}} } }
            \approx 13{,}56 \cdot 10^{-29}\,\text{кг} \approx 0{,}0817\,\text{а.е.м.}
    \end{align*}
}
\solutionspace{100pt}

\tasknumber{4}%
\task{%
    Определите деффект массы (в а.е.м.) и энергию связи (в МэВ) ядра атома \ce{^{3}_{1}{T}},
    если его масса составляет $3{,}01605\,\text{а.е.м.}$.
    Считать $m_{p} = 1{,}00728\,\text{а.е.м.}$, $m_{n} = 1{,}00867\,\text{а.е.м.}$.
}
\answer{%
    \begin{align*}
    \Delta m &= (A - Z)m_{n} + Zm_{p} - m = 2 \cdot 1{,}00867\,\text{а.е.м.} + 1 \cdot 1{,}00728\,\text{а.е.м.} - 3{,}01605\,\text{а.е.м.} \approx 0{,}00857\,\text{а.е.м.} \\
    E_\text{ св.
    } &= \Delta m c^2 \approx 0{,}0086 \cdot 931{,}5\,\text{МэВ} \approx 7{,}98\,\text{МэВ}
    \end{align*}
}
\solutionspace{150pt}

\tasknumber{5}%
\task{%
    Сделайте схематичный рисунок энергетических уровней атома водорода
    и отметьте на нём первый (основной) уровень и последующие.
    Сколько различных длин волн может испустить атом водорода,
    находящийся в 5-м возбуждённом состоянии?
    Отметьте все соответствующие переходы на рисунке и укажите,
    при каком переходе (среди отмеченных) частота излучённого фотона минимальна.
}
\answer{%
    $N = 10{,}0, \text{самая короткая линия}$
}

\variantsplitter

\addpersonalvariant{Екатерина Медведева}

\tasknumber{1}%
\task{%
    В ядре электрически нейтрального атома 121 частиц.
    Вокруг ядра обращается 51 электронов.
    Сколько в ядре этого атома протонов и нейтронов?
    Назовите этот элемент.
}
\answer{%
    $Z = 51$ протонов и $A - Z = 70$ нейтронов, так что это \text{сурьма-121}: $\ce{^{121}_{51}{Sb}}$
}
\solutionspace{120pt}

\tasknumber{2}%
\task{%
    Запишите реакцию $\alpha$-распада $\ce{^{147}_{62}{Sm}}$.
}
\answer{%
    $\ce{^{147}_{62}{Sm}} \to \ce{^{143}_{60}{Nd}} + \ce{ ^4_2 He }$
}
\solutionspace{80pt}

\tasknumber{3}%
\task{%
    Энергия связи ядра углерода \ce{^{13}_{6}C} равна $97{,}1\,\text{МэВ}$.
    Найти дефект массы этого ядра.
    Ответ выразите в а.е.м.
    и кг.
    Скорость света $c = 2{,}998 \cdot 10^{8}\,\frac{\text{м}}{\text{с}}$, элементарный заряд $e = 1{,}6 \cdot 10^{-19}\,\text{Кл}$.
}
\answer{%
    \begin{align*}
    E_\text{св.} &= \Delta m c^2 \implies \\
    \implies
            \Delta m &= \frac {E_\text{св.}}{c^2} = \frac{ 97{,}1\,\text{МэВ} }{ \sqr{ 2{,}998 \cdot 10^{8}\,\frac{\text{м}}{\text{с}} } }
            = \frac{97{,}1 \cdot 10^6 \cdot 1{,}6 \cdot 10^{-19}\,\text{Дж}}{ \sqr{ 2{,}998 \cdot 10^{8}\,\frac{\text{м}}{\text{с}} } }
            \approx 17{,}29 \cdot 10^{-29}\,\text{кг} \approx 0{,}1041\,\text{а.е.м.}
    \end{align*}
}
\solutionspace{100pt}

\tasknumber{4}%
\task{%
    Определите деффект массы (в а.е.м.) и энергию связи (в МэВ) ядра атома \ce{^{2}_{1}{D}},
    если его масса составляет $2{,}0141\,\text{а.е.м.}$.
    Считать $m_{p} = 1{,}00728\,\text{а.е.м.}$, $m_{n} = 1{,}00867\,\text{а.е.м.}$.
}
\answer{%
    \begin{align*}
    \Delta m &= (A - Z)m_{n} + Zm_{p} - m = 1 \cdot 1{,}00867\,\text{а.е.м.} + 1 \cdot 1{,}00728\,\text{а.е.м.} - 2{,}0141\,\text{а.е.м.} \approx 0{,}00185\,\text{а.е.м.} \\
    E_\text{ св.
    } &= \Delta m c^2 \approx 0{,}0019 \cdot 931{,}5\,\text{МэВ} \approx 1{,}72\,\text{МэВ}
    \end{align*}
}
\solutionspace{150pt}

\tasknumber{5}%
\task{%
    Сделайте схематичный рисунок энергетических уровней атома водорода
    и отметьте на нём первый (основной) уровень и последующие.
    Сколько различных длин волн может испустить атом водорода,
    находящийся в 4-м возбуждённом состоянии?
    Отметьте все соответствующие переходы на рисунке и укажите,
    при каком переходе (среди отмеченных) энергия излучённого фотона максимальна.
}
\answer{%
    $N = 6{,}0, \text{самая длинная линия}$
}

\variantsplitter

\addpersonalvariant{Константин Мельник}

\tasknumber{1}%
\task{%
    В ядре электрически нейтрального атома 190 частиц.
    Вокруг ядра обращается 78 электронов.
    Сколько в ядре этого атома протонов и нейтронов?
    Назовите этот элемент.
}
\answer{%
    $Z = 78$ протонов и $A - Z = 112$ нейтронов, так что это \text{платина-190}: $\ce{^{190}_{78}{Pt}}$
}
\solutionspace{120pt}

\tasknumber{2}%
\task{%
    Запишите реакцию $\beta$-распада $\ce{^{137}_{55}{Cs}}$.
}
\answer{%
    $\ce{^{137}_{55}{Cs}} \to \ce{^{137}_{56}{Ba}} + e^- + \tilde\nu_e$
}
\solutionspace{80pt}

\tasknumber{3}%
\task{%
    Энергия связи ядра лития \ce{^{6}_{3}Li} равна $31{,}99\,\text{МэВ}$.
    Найти дефект массы этого ядра.
    Ответ выразите в а.е.м.
    и кг.
    Скорость света $c = 2{,}998 \cdot 10^{8}\,\frac{\text{м}}{\text{с}}$, элементарный заряд $e = 1{,}6 \cdot 10^{-19}\,\text{Кл}$.
}
\answer{%
    \begin{align*}
    E_\text{св.} &= \Delta m c^2 \implies \\
    \implies
            \Delta m &= \frac {E_\text{св.}}{c^2} = \frac{ 31{,}99\,\text{МэВ} }{ \sqr{ 2{,}998 \cdot 10^{8}\,\frac{\text{м}}{\text{с}} } }
            = \frac{31{,}99 \cdot 10^6 \cdot 1{,}6 \cdot 10^{-19}\,\text{Дж}}{ \sqr{ 2{,}998 \cdot 10^{8}\,\frac{\text{м}}{\text{с}} } }
            \approx 5{,}69 \cdot 10^{-29}\,\text{кг} \approx 0{,}0343\,\text{а.е.м.}
    \end{align*}
}
\solutionspace{100pt}

\tasknumber{4}%
\task{%
    Определите деффект массы (в а.е.м.) и энергию связи (в МэВ) ядра атома \ce{^{3}_{2}{He}},
    если его масса составляет $3{,}01603\,\text{а.е.м.}$.
    Считать $m_{p} = 1{,}00728\,\text{а.е.м.}$, $m_{n} = 1{,}00867\,\text{а.е.м.}$.
}
\answer{%
    \begin{align*}
    \Delta m &= (A - Z)m_{n} + Zm_{p} - m = 1 \cdot 1{,}00867\,\text{а.е.м.} + 2 \cdot 1{,}00728\,\text{а.е.м.} - 3{,}01603\,\text{а.е.м.} \approx 0{,}00720\,\text{а.е.м.} \\
    E_\text{ св.
    } &= \Delta m c^2 \approx 0{,}0072 \cdot 931{,}5\,\text{МэВ} \approx 6{,}71\,\text{МэВ}
    \end{align*}
}
\solutionspace{150pt}

\tasknumber{5}%
\task{%
    Сделайте схематичный рисунок энергетических уровней атома водорода
    и отметьте на нём первый (основной) уровень и последующие.
    Сколько различных длин волн может испустить атом водорода,
    находящийся в 4-м возбуждённом состоянии?
    Отметьте все соответствующие переходы на рисунке и укажите,
    при каком переходе (среди отмеченных) частота излучённого фотона максимальна.
}
\answer{%
    $N = 6{,}0, \text{самая длинная линия}$
}

\variantsplitter

\addpersonalvariant{Степан Небоваренков}

\tasknumber{1}%
\task{%
    В ядре электрически нейтрального атома 108 частиц.
    Вокруг ядра обращается 47 электронов.
    Сколько в ядре этого атома протонов и нейтронов?
    Назовите этот элемент.
}
\answer{%
    $Z = 47$ протонов и $A - Z = 61$ нейтронов, так что это \text{серебро-108}: $\ce{^{108}_{47}{Ag}}$
}
\solutionspace{120pt}

\tasknumber{2}%
\task{%
    Запишите реакцию $\alpha$-распада $\ce{^{180}_{74}{W}}$.
}
\answer{%
    $\ce{^{180}_{74}{W}} \to \ce{^{176}_{72}{Hf}} + \ce{ ^4_2 He }$
}
\solutionspace{80pt}

\tasknumber{3}%
\task{%
    Энергия связи ядра бора \ce{^{10}_{5}B} равна $64{,}7\,\text{МэВ}$.
    Найти дефект массы этого ядра.
    Ответ выразите в а.е.м.
    и кг.
    Скорость света $c = 2{,}998 \cdot 10^{8}\,\frac{\text{м}}{\text{с}}$, элементарный заряд $e = 1{,}6 \cdot 10^{-19}\,\text{Кл}$.
}
\answer{%
    \begin{align*}
    E_\text{св.} &= \Delta m c^2 \implies \\
    \implies
            \Delta m &= \frac {E_\text{св.}}{c^2} = \frac{ 64{,}7\,\text{МэВ} }{ \sqr{ 2{,}998 \cdot 10^{8}\,\frac{\text{м}}{\text{с}} } }
            = \frac{64{,}7 \cdot 10^6 \cdot 1{,}6 \cdot 10^{-19}\,\text{Дж}}{ \sqr{ 2{,}998 \cdot 10^{8}\,\frac{\text{м}}{\text{с}} } }
            \approx 11{,}52 \cdot 10^{-29}\,\text{кг} \approx 0{,}0694\,\text{а.е.м.}
    \end{align*}
}
\solutionspace{100pt}

\tasknumber{4}%
\task{%
    Определите деффект массы (в а.е.м.) и энергию связи (в МэВ) ядра атома \ce{^{2}_{1}{D}},
    если его масса составляет $2{,}0141\,\text{а.е.м.}$.
    Считать $m_{p} = 1{,}00728\,\text{а.е.м.}$, $m_{n} = 1{,}00867\,\text{а.е.м.}$.
}
\answer{%
    \begin{align*}
    \Delta m &= (A - Z)m_{n} + Zm_{p} - m = 1 \cdot 1{,}00867\,\text{а.е.м.} + 1 \cdot 1{,}00728\,\text{а.е.м.} - 2{,}0141\,\text{а.е.м.} \approx 0{,}00185\,\text{а.е.м.} \\
    E_\text{ св.
    } &= \Delta m c^2 \approx 0{,}0019 \cdot 931{,}5\,\text{МэВ} \approx 1{,}72\,\text{МэВ}
    \end{align*}
}
\solutionspace{150pt}

\tasknumber{5}%
\task{%
    Сделайте схематичный рисунок энергетических уровней атома водорода
    и отметьте на нём первый (основной) уровень и последующие.
    Сколько различных длин волн может испустить атом водорода,
    находящийся в 5-м возбуждённом состоянии?
    Отметьте все соответствующие переходы на рисунке и укажите,
    при каком переходе (среди отмеченных) длина волны излучённого фотона минимальна.
}
\answer{%
    $N = 10{,}0, \text{самая длинная линия}$
}

\variantsplitter

\addpersonalvariant{Матвей Неретин}

\tasknumber{1}%
\task{%
    В ядре электрически нейтрального атома 190 частиц.
    Вокруг ядра обращается 78 электронов.
    Сколько в ядре этого атома протонов и нейтронов?
    Назовите этот элемент.
}
\answer{%
    $Z = 78$ протонов и $A - Z = 112$ нейтронов, так что это \text{платина-190}: $\ce{^{190}_{78}{Pt}}$
}
\solutionspace{120pt}

\tasknumber{2}%
\task{%
    Запишите реакцию $\beta$-распада $\ce{^{22}_{11}{Na}}$.
}
\answer{%
    $\ce{^{22}_{11}{Na}} \to \ce{^{22}_{12}{Mg}} + e^- + \tilde\nu_e$
}
\solutionspace{80pt}

\tasknumber{3}%
\task{%
    Энергия связи ядра лития \ce{^{6}_{3}Li} равна $31{,}99\,\text{МэВ}$.
    Найти дефект массы этого ядра.
    Ответ выразите в а.е.м.
    и кг.
    Скорость света $c = 2{,}998 \cdot 10^{8}\,\frac{\text{м}}{\text{с}}$, элементарный заряд $e = 1{,}6 \cdot 10^{-19}\,\text{Кл}$.
}
\answer{%
    \begin{align*}
    E_\text{св.} &= \Delta m c^2 \implies \\
    \implies
            \Delta m &= \frac {E_\text{св.}}{c^2} = \frac{ 31{,}99\,\text{МэВ} }{ \sqr{ 2{,}998 \cdot 10^{8}\,\frac{\text{м}}{\text{с}} } }
            = \frac{31{,}99 \cdot 10^6 \cdot 1{,}6 \cdot 10^{-19}\,\text{Дж}}{ \sqr{ 2{,}998 \cdot 10^{8}\,\frac{\text{м}}{\text{с}} } }
            \approx 5{,}69 \cdot 10^{-29}\,\text{кг} \approx 0{,}0343\,\text{а.е.м.}
    \end{align*}
}
\solutionspace{100pt}

\tasknumber{4}%
\task{%
    Определите деффект массы (в а.е.м.) и энергию связи (в МэВ) ядра атома \ce{^{6}_{2}{He}},
    если его масса составляет $6{,}0189\,\text{а.е.м.}$.
    Считать $m_{p} = 1{,}00728\,\text{а.е.м.}$, $m_{n} = 1{,}00867\,\text{а.е.м.}$.
}
\answer{%
    \begin{align*}
    \Delta m &= (A - Z)m_{n} + Zm_{p} - m = 4 \cdot 1{,}00867\,\text{а.е.м.} + 2 \cdot 1{,}00728\,\text{а.е.м.} - 6{,}0189\,\text{а.е.м.} \approx 0{,}03034\,\text{а.е.м.} \\
    E_\text{ св.
    } &= \Delta m c^2 \approx 0{,}0303 \cdot 931{,}5\,\text{МэВ} \approx 28{,}26\,\text{МэВ}
    \end{align*}
}
\solutionspace{150pt}

\tasknumber{5}%
\task{%
    Сделайте схематичный рисунок энергетических уровней атома водорода
    и отметьте на нём первый (основной) уровень и последующие.
    Сколько различных длин волн может испустить атом водорода,
    находящийся в 5-м возбуждённом состоянии?
    Отметьте все соответствующие переходы на рисунке и укажите,
    при каком переходе (среди отмеченных) частота излучённого фотона максимальна.
}
\answer{%
    $N = 10{,}0, \text{самая длинная линия}$
}

\variantsplitter

\addpersonalvariant{Мария Никонова}

\tasknumber{1}%
\task{%
    В ядре электрически нейтрального атома 63 частиц.
    Вокруг ядра обращается 29 электронов.
    Сколько в ядре этого атома протонов и нейтронов?
    Назовите этот элемент.
}
\answer{%
    $Z = 29$ протонов и $A - Z = 34$ нейтронов, так что это \text{медь-63}: $\ce{^{63}_{29}{Cu}}$
}
\solutionspace{120pt}

\tasknumber{2}%
\task{%
    Запишите реакцию $\beta$-распада $\ce{^{137}_{55}{Cs}}$.
}
\answer{%
    $\ce{^{137}_{55}{Cs}} \to \ce{^{137}_{56}{Ba}} + e^- + \tilde\nu_e$
}
\solutionspace{80pt}

\tasknumber{3}%
\task{%
    Энергия связи ядра азота \ce{^{14}_{7}N} равна $115{,}5\,\text{МэВ}$.
    Найти дефект массы этого ядра.
    Ответ выразите в а.е.м.
    и кг.
    Скорость света $c = 2{,}998 \cdot 10^{8}\,\frac{\text{м}}{\text{с}}$, элементарный заряд $e = 1{,}6 \cdot 10^{-19}\,\text{Кл}$.
}
\answer{%
    \begin{align*}
    E_\text{св.} &= \Delta m c^2 \implies \\
    \implies
            \Delta m &= \frac {E_\text{св.}}{c^2} = \frac{ 115{,}5\,\text{МэВ} }{ \sqr{ 2{,}998 \cdot 10^{8}\,\frac{\text{м}}{\text{с}} } }
            = \frac{115{,}5 \cdot 10^6 \cdot 1{,}6 \cdot 10^{-19}\,\text{Дж}}{ \sqr{ 2{,}998 \cdot 10^{8}\,\frac{\text{м}}{\text{с}} } }
            \approx 20{,}6 \cdot 10^{-29}\,\text{кг} \approx 0{,}1238\,\text{а.е.м.}
    \end{align*}
}
\solutionspace{100pt}

\tasknumber{4}%
\task{%
    Определите деффект массы (в а.е.м.) и энергию связи (в МэВ) ядра атома \ce{^{3}_{1}{T}},
    если его масса составляет $3{,}01605\,\text{а.е.м.}$.
    Считать $m_{p} = 1{,}00728\,\text{а.е.м.}$, $m_{n} = 1{,}00867\,\text{а.е.м.}$.
}
\answer{%
    \begin{align*}
    \Delta m &= (A - Z)m_{n} + Zm_{p} - m = 2 \cdot 1{,}00867\,\text{а.е.м.} + 1 \cdot 1{,}00728\,\text{а.е.м.} - 3{,}01605\,\text{а.е.м.} \approx 0{,}00857\,\text{а.е.м.} \\
    E_\text{ св.
    } &= \Delta m c^2 \approx 0{,}0086 \cdot 931{,}5\,\text{МэВ} \approx 7{,}98\,\text{МэВ}
    \end{align*}
}
\solutionspace{150pt}

\tasknumber{5}%
\task{%
    Сделайте схематичный рисунок энергетических уровней атома водорода
    и отметьте на нём первый (основной) уровень и последующие.
    Сколько различных длин волн может испустить атом водорода,
    находящийся в 3-м возбуждённом состоянии?
    Отметьте все соответствующие переходы на рисунке и укажите,
    при каком переходе (среди отмеченных) частота излучённого фотона минимальна.
}
\answer{%
    $N = 3{,}0, \text{самая короткая линия}$
}

\variantsplitter

\addpersonalvariant{Даниил Палаткин}

\tasknumber{1}%
\task{%
    В ядре электрически нейтрального атома 121 частиц.
    Вокруг ядра обращается 51 электронов.
    Сколько в ядре этого атома протонов и нейтронов?
    Назовите этот элемент.
}
\answer{%
    $Z = 51$ протонов и $A - Z = 70$ нейтронов, так что это \text{сурьма-121}: $\ce{^{121}_{51}{Sb}}$
}
\solutionspace{120pt}

\tasknumber{2}%
\task{%
    Запишите реакцию $\beta$-распада $\ce{^{137}_{55}{Cs}}$.
}
\answer{%
    $\ce{^{137}_{55}{Cs}} \to \ce{^{137}_{56}{Ba}} + e^- + \tilde\nu_e$
}
\solutionspace{80pt}

\tasknumber{3}%
\task{%
    Энергия связи ядра бора \ce{^{11}_{5}B} равна $76{,}2\,\text{МэВ}$.
    Найти дефект массы этого ядра.
    Ответ выразите в а.е.м.
    и кг.
    Скорость света $c = 2{,}998 \cdot 10^{8}\,\frac{\text{м}}{\text{с}}$, элементарный заряд $e = 1{,}6 \cdot 10^{-19}\,\text{Кл}$.
}
\answer{%
    \begin{align*}
    E_\text{св.} &= \Delta m c^2 \implies \\
    \implies
            \Delta m &= \frac {E_\text{св.}}{c^2} = \frac{ 76{,}2\,\text{МэВ} }{ \sqr{ 2{,}998 \cdot 10^{8}\,\frac{\text{м}}{\text{с}} } }
            = \frac{76{,}2 \cdot 10^6 \cdot 1{,}6 \cdot 10^{-19}\,\text{Дж}}{ \sqr{ 2{,}998 \cdot 10^{8}\,\frac{\text{м}}{\text{с}} } }
            \approx 13{,}56 \cdot 10^{-29}\,\text{кг} \approx 0{,}0817\,\text{а.е.м.}
    \end{align*}
}
\solutionspace{100pt}

\tasknumber{4}%
\task{%
    Определите деффект массы (в а.е.м.) и энергию связи (в МэВ) ядра атома \ce{^{6}_{2}{He}},
    если его масса составляет $6{,}0189\,\text{а.е.м.}$.
    Считать $m_{p} = 1{,}00728\,\text{а.е.м.}$, $m_{n} = 1{,}00867\,\text{а.е.м.}$.
}
\answer{%
    \begin{align*}
    \Delta m &= (A - Z)m_{n} + Zm_{p} - m = 4 \cdot 1{,}00867\,\text{а.е.м.} + 2 \cdot 1{,}00728\,\text{а.е.м.} - 6{,}0189\,\text{а.е.м.} \approx 0{,}03034\,\text{а.е.м.} \\
    E_\text{ св.
    } &= \Delta m c^2 \approx 0{,}0303 \cdot 931{,}5\,\text{МэВ} \approx 28{,}26\,\text{МэВ}
    \end{align*}
}
\solutionspace{150pt}

\tasknumber{5}%
\task{%
    Сделайте схематичный рисунок энергетических уровней атома водорода
    и отметьте на нём первый (основной) уровень и последующие.
    Сколько различных длин волн может испустить атом водорода,
    находящийся в 4-м возбуждённом состоянии?
    Отметьте все соответствующие переходы на рисунке и укажите,
    при каком переходе (среди отмеченных) энергия излучённого фотона минимальна.
}
\answer{%
    $N = 6{,}0, \text{самая короткая линия}$
}

\variantsplitter

\addpersonalvariant{Станислав Пикун}

\tasknumber{1}%
\task{%
    В ядре электрически нейтрального атома 63 частиц.
    Вокруг ядра обращается 29 электронов.
    Сколько в ядре этого атома протонов и нейтронов?
    Назовите этот элемент.
}
\answer{%
    $Z = 29$ протонов и $A - Z = 34$ нейтронов, так что это \text{медь-63}: $\ce{^{63}_{29}{Cu}}$
}
\solutionspace{120pt}

\tasknumber{2}%
\task{%
    Запишите реакцию $\beta$-распада $\ce{^{137}_{55}{Cs}}$.
}
\answer{%
    $\ce{^{137}_{55}{Cs}} \to \ce{^{137}_{56}{Ba}} + e^- + \tilde\nu_e$
}
\solutionspace{80pt}

\tasknumber{3}%
\task{%
    Энергия связи ядра гелия \ce{^{3}_{2}He} равна $7{,}72\,\text{МэВ}$.
    Найти дефект массы этого ядра.
    Ответ выразите в а.е.м.
    и кг.
    Скорость света $c = 2{,}998 \cdot 10^{8}\,\frac{\text{м}}{\text{с}}$, элементарный заряд $e = 1{,}6 \cdot 10^{-19}\,\text{Кл}$.
}
\answer{%
    \begin{align*}
    E_\text{св.} &= \Delta m c^2 \implies \\
    \implies
            \Delta m &= \frac {E_\text{св.}}{c^2} = \frac{ 7{,}72\,\text{МэВ} }{ \sqr{ 2{,}998 \cdot 10^{8}\,\frac{\text{м}}{\text{с}} } }
            = \frac{7{,}72 \cdot 10^6 \cdot 1{,}6 \cdot 10^{-19}\,\text{Дж}}{ \sqr{ 2{,}998 \cdot 10^{8}\,\frac{\text{м}}{\text{с}} } }
            \approx 1{,}374 \cdot 10^{-29}\,\text{кг} \approx 0{,}00828\,\text{а.е.м.}
    \end{align*}
}
\solutionspace{100pt}

\tasknumber{4}%
\task{%
    Определите деффект массы (в а.е.м.) и энергию связи (в МэВ) ядра атома \ce{^{3}_{2}{He}},
    если его масса составляет $3{,}01603\,\text{а.е.м.}$.
    Считать $m_{p} = 1{,}00728\,\text{а.е.м.}$, $m_{n} = 1{,}00867\,\text{а.е.м.}$.
}
\answer{%
    \begin{align*}
    \Delta m &= (A - Z)m_{n} + Zm_{p} - m = 1 \cdot 1{,}00867\,\text{а.е.м.} + 2 \cdot 1{,}00728\,\text{а.е.м.} - 3{,}01603\,\text{а.е.м.} \approx 0{,}00720\,\text{а.е.м.} \\
    E_\text{ св.
    } &= \Delta m c^2 \approx 0{,}0072 \cdot 931{,}5\,\text{МэВ} \approx 6{,}71\,\text{МэВ}
    \end{align*}
}
\solutionspace{150pt}

\tasknumber{5}%
\task{%
    Сделайте схематичный рисунок энергетических уровней атома водорода
    и отметьте на нём первый (основной) уровень и последующие.
    Сколько различных длин волн может испустить атом водорода,
    находящийся в 4-м возбуждённом состоянии?
    Отметьте все соответствующие переходы на рисунке и укажите,
    при каком переходе (среди отмеченных) частота излучённого фотона минимальна.
}
\answer{%
    $N = 6{,}0, \text{самая короткая линия}$
}

\variantsplitter

\addpersonalvariant{Илья Пичугин}

\tasknumber{1}%
\task{%
    В ядре электрически нейтрального атома 121 частиц.
    Вокруг ядра обращается 51 электронов.
    Сколько в ядре этого атома протонов и нейтронов?
    Назовите этот элемент.
}
\answer{%
    $Z = 51$ протонов и $A - Z = 70$ нейтронов, так что это \text{сурьма-121}: $\ce{^{121}_{51}{Sb}}$
}
\solutionspace{120pt}

\tasknumber{2}%
\task{%
    Запишите реакцию $\alpha$-распада $\ce{^{180}_{74}{W}}$.
}
\answer{%
    $\ce{^{180}_{74}{W}} \to \ce{^{176}_{72}{Hf}} + \ce{ ^4_2 He }$
}
\solutionspace{80pt}

\tasknumber{3}%
\task{%
    Энергия связи ядра бора \ce{^{11}_{5}B} равна $76{,}2\,\text{МэВ}$.
    Найти дефект массы этого ядра.
    Ответ выразите в а.е.м.
    и кг.
    Скорость света $c = 2{,}998 \cdot 10^{8}\,\frac{\text{м}}{\text{с}}$, элементарный заряд $e = 1{,}6 \cdot 10^{-19}\,\text{Кл}$.
}
\answer{%
    \begin{align*}
    E_\text{св.} &= \Delta m c^2 \implies \\
    \implies
            \Delta m &= \frac {E_\text{св.}}{c^2} = \frac{ 76{,}2\,\text{МэВ} }{ \sqr{ 2{,}998 \cdot 10^{8}\,\frac{\text{м}}{\text{с}} } }
            = \frac{76{,}2 \cdot 10^6 \cdot 1{,}6 \cdot 10^{-19}\,\text{Дж}}{ \sqr{ 2{,}998 \cdot 10^{8}\,\frac{\text{м}}{\text{с}} } }
            \approx 13{,}56 \cdot 10^{-29}\,\text{кг} \approx 0{,}0817\,\text{а.е.м.}
    \end{align*}
}
\solutionspace{100pt}

\tasknumber{4}%
\task{%
    Определите деффект массы (в а.е.м.) и энергию связи (в МэВ) ядра атома \ce{^{8}_{2}{He}},
    если его масса составляет $8{,}0225\,\text{а.е.м.}$.
    Считать $m_{p} = 1{,}00728\,\text{а.е.м.}$, $m_{n} = 1{,}00867\,\text{а.е.м.}$.
}
\answer{%
    \begin{align*}
    \Delta m &= (A - Z)m_{n} + Zm_{p} - m = 6 \cdot 1{,}00867\,\text{а.е.м.} + 2 \cdot 1{,}00728\,\text{а.е.м.} - 8{,}0225\,\text{а.е.м.} \approx 0{,}04408\,\text{а.е.м.} \\
    E_\text{ св.
    } &= \Delta m c^2 \approx 0{,}0441 \cdot 931{,}5\,\text{МэВ} \approx 41{,}06\,\text{МэВ}
    \end{align*}
}
\solutionspace{150pt}

\tasknumber{5}%
\task{%
    Сделайте схематичный рисунок энергетических уровней атома водорода
    и отметьте на нём первый (основной) уровень и последующие.
    Сколько различных длин волн может испустить атом водорода,
    находящийся в 4-м возбуждённом состоянии?
    Отметьте все соответствующие переходы на рисунке и укажите,
    при каком переходе (среди отмеченных) длина волны излучённого фотона максимальна.
}
\answer{%
    $N = 6{,}0, \text{самая короткая линия}$
}

\variantsplitter

\addpersonalvariant{Кирилл Севрюгин}

\tasknumber{1}%
\task{%
    В ядре электрически нейтрального атома 108 частиц.
    Вокруг ядра обращается 47 электронов.
    Сколько в ядре этого атома протонов и нейтронов?
    Назовите этот элемент.
}
\answer{%
    $Z = 47$ протонов и $A - Z = 61$ нейтронов, так что это \text{серебро-108}: $\ce{^{108}_{47}{Ag}}$
}
\solutionspace{120pt}

\tasknumber{2}%
\task{%
    Запишите реакцию $\alpha$-распада $\ce{^{144}_{60}{Nd}}$.
}
\answer{%
    $\ce{^{144}_{60}{Nd}} \to \ce{^{140}_{58}{Ce}} + \ce{ ^4_2 He }$
}
\solutionspace{80pt}

\tasknumber{3}%
\task{%
    Энергия связи ядра бора \ce{^{11}_{5}B} равна $76{,}2\,\text{МэВ}$.
    Найти дефект массы этого ядра.
    Ответ выразите в а.е.м.
    и кг.
    Скорость света $c = 2{,}998 \cdot 10^{8}\,\frac{\text{м}}{\text{с}}$, элементарный заряд $e = 1{,}6 \cdot 10^{-19}\,\text{Кл}$.
}
\answer{%
    \begin{align*}
    E_\text{св.} &= \Delta m c^2 \implies \\
    \implies
            \Delta m &= \frac {E_\text{св.}}{c^2} = \frac{ 76{,}2\,\text{МэВ} }{ \sqr{ 2{,}998 \cdot 10^{8}\,\frac{\text{м}}{\text{с}} } }
            = \frac{76{,}2 \cdot 10^6 \cdot 1{,}6 \cdot 10^{-19}\,\text{Дж}}{ \sqr{ 2{,}998 \cdot 10^{8}\,\frac{\text{м}}{\text{с}} } }
            \approx 13{,}56 \cdot 10^{-29}\,\text{кг} \approx 0{,}0817\,\text{а.е.м.}
    \end{align*}
}
\solutionspace{100pt}

\tasknumber{4}%
\task{%
    Определите деффект массы (в а.е.м.) и энергию связи (в МэВ) ядра атома \ce{^{3}_{1}{T}},
    если его масса составляет $3{,}01605\,\text{а.е.м.}$.
    Считать $m_{p} = 1{,}00728\,\text{а.е.м.}$, $m_{n} = 1{,}00867\,\text{а.е.м.}$.
}
\answer{%
    \begin{align*}
    \Delta m &= (A - Z)m_{n} + Zm_{p} - m = 2 \cdot 1{,}00867\,\text{а.е.м.} + 1 \cdot 1{,}00728\,\text{а.е.м.} - 3{,}01605\,\text{а.е.м.} \approx 0{,}00857\,\text{а.е.м.} \\
    E_\text{ св.
    } &= \Delta m c^2 \approx 0{,}0086 \cdot 931{,}5\,\text{МэВ} \approx 7{,}98\,\text{МэВ}
    \end{align*}
}
\solutionspace{150pt}

\tasknumber{5}%
\task{%
    Сделайте схематичный рисунок энергетических уровней атома водорода
    и отметьте на нём первый (основной) уровень и последующие.
    Сколько различных длин волн может испустить атом водорода,
    находящийся в 4-м возбуждённом состоянии?
    Отметьте все соответствующие переходы на рисунке и укажите,
    при каком переходе (среди отмеченных) длина волны излучённого фотона максимальна.
}
\answer{%
    $N = 6{,}0, \text{самая короткая линия}$
}

\variantsplitter

\addpersonalvariant{Илья Стратонников}

\tasknumber{1}%
\task{%
    В ядре электрически нейтрального атома 108 частиц.
    Вокруг ядра обращается 47 электронов.
    Сколько в ядре этого атома протонов и нейтронов?
    Назовите этот элемент.
}
\answer{%
    $Z = 47$ протонов и $A - Z = 61$ нейтронов, так что это \text{серебро-108}: $\ce{^{108}_{47}{Ag}}$
}
\solutionspace{120pt}

\tasknumber{2}%
\task{%
    Запишите реакцию $\beta$-распада $\ce{^{22}_{11}{Na}}$.
}
\answer{%
    $\ce{^{22}_{11}{Na}} \to \ce{^{22}_{12}{Mg}} + e^- + \tilde\nu_e$
}
\solutionspace{80pt}

\tasknumber{3}%
\task{%
    Энергия связи ядра кислорода \ce{^{16}_{8}O} равна $127{,}6\,\text{МэВ}$.
    Найти дефект массы этого ядра.
    Ответ выразите в а.е.м.
    и кг.
    Скорость света $c = 2{,}998 \cdot 10^{8}\,\frac{\text{м}}{\text{с}}$, элементарный заряд $e = 1{,}6 \cdot 10^{-19}\,\text{Кл}$.
}
\answer{%
    \begin{align*}
    E_\text{св.} &= \Delta m c^2 \implies \\
    \implies
            \Delta m &= \frac {E_\text{св.}}{c^2} = \frac{ 127{,}6\,\text{МэВ} }{ \sqr{ 2{,}998 \cdot 10^{8}\,\frac{\text{м}}{\text{с}} } }
            = \frac{127{,}6 \cdot 10^6 \cdot 1{,}6 \cdot 10^{-19}\,\text{Дж}}{ \sqr{ 2{,}998 \cdot 10^{8}\,\frac{\text{м}}{\text{с}} } }
            \approx 22{,}7 \cdot 10^{-29}\,\text{кг} \approx 0{,}1368\,\text{а.е.м.}
    \end{align*}
}
\solutionspace{100pt}

\tasknumber{4}%
\task{%
    Определите деффект массы (в а.е.м.) и энергию связи (в МэВ) ядра атома \ce{^{2}_{1}{D}},
    если его масса составляет $2{,}0141\,\text{а.е.м.}$.
    Считать $m_{p} = 1{,}00728\,\text{а.е.м.}$, $m_{n} = 1{,}00867\,\text{а.е.м.}$.
}
\answer{%
    \begin{align*}
    \Delta m &= (A - Z)m_{n} + Zm_{p} - m = 1 \cdot 1{,}00867\,\text{а.е.м.} + 1 \cdot 1{,}00728\,\text{а.е.м.} - 2{,}0141\,\text{а.е.м.} \approx 0{,}00185\,\text{а.е.м.} \\
    E_\text{ св.
    } &= \Delta m c^2 \approx 0{,}0019 \cdot 931{,}5\,\text{МэВ} \approx 1{,}72\,\text{МэВ}
    \end{align*}
}
\solutionspace{150pt}

\tasknumber{5}%
\task{%
    Сделайте схематичный рисунок энергетических уровней атома водорода
    и отметьте на нём первый (основной) уровень и последующие.
    Сколько различных длин волн может испустить атом водорода,
    находящийся в 3-м возбуждённом состоянии?
    Отметьте все соответствующие переходы на рисунке и укажите,
    при каком переходе (среди отмеченных) частота излучённого фотона минимальна.
}
\answer{%
    $N = 3{,}0, \text{самая короткая линия}$
}

\variantsplitter

\addpersonalvariant{Иван Шустов}

\tasknumber{1}%
\task{%
    В ядре электрически нейтрального атома 121 частиц.
    Вокруг ядра обращается 51 электронов.
    Сколько в ядре этого атома протонов и нейтронов?
    Назовите этот элемент.
}
\answer{%
    $Z = 51$ протонов и $A - Z = 70$ нейтронов, так что это \text{сурьма-121}: $\ce{^{121}_{51}{Sb}}$
}
\solutionspace{120pt}

\tasknumber{2}%
\task{%
    Запишите реакцию $\beta$-распада $\ce{^{22}_{11}{Na}}$.
}
\answer{%
    $\ce{^{22}_{11}{Na}} \to \ce{^{22}_{12}{Mg}} + e^- + \tilde\nu_e$
}
\solutionspace{80pt}

\tasknumber{3}%
\task{%
    Энергия связи ядра азота \ce{^{14}_{7}N} равна $104{,}7\,\text{МэВ}$.
    Найти дефект массы этого ядра.
    Ответ выразите в а.е.м.
    и кг.
    Скорость света $c = 2{,}998 \cdot 10^{8}\,\frac{\text{м}}{\text{с}}$, элементарный заряд $e = 1{,}6 \cdot 10^{-19}\,\text{Кл}$.
}
\answer{%
    \begin{align*}
    E_\text{св.} &= \Delta m c^2 \implies \\
    \implies
            \Delta m &= \frac {E_\text{св.}}{c^2} = \frac{ 104{,}7\,\text{МэВ} }{ \sqr{ 2{,}998 \cdot 10^{8}\,\frac{\text{м}}{\text{с}} } }
            = \frac{104{,}7 \cdot 10^6 \cdot 1{,}6 \cdot 10^{-19}\,\text{Дж}}{ \sqr{ 2{,}998 \cdot 10^{8}\,\frac{\text{м}}{\text{с}} } }
            \approx 18{,}64 \cdot 10^{-29}\,\text{кг} \approx 0{,}1122\,\text{а.е.м.}
    \end{align*}
}
\solutionspace{100pt}

\tasknumber{4}%
\task{%
    Определите деффект массы (в а.е.м.) и энергию связи (в МэВ) ядра атома \ce{^{3}_{2}{He}},
    если его масса составляет $3{,}01603\,\text{а.е.м.}$.
    Считать $m_{p} = 1{,}00728\,\text{а.е.м.}$, $m_{n} = 1{,}00867\,\text{а.е.м.}$.
}
\answer{%
    \begin{align*}
    \Delta m &= (A - Z)m_{n} + Zm_{p} - m = 1 \cdot 1{,}00867\,\text{а.е.м.} + 2 \cdot 1{,}00728\,\text{а.е.м.} - 3{,}01603\,\text{а.е.м.} \approx 0{,}00720\,\text{а.е.м.} \\
    E_\text{ св.
    } &= \Delta m c^2 \approx 0{,}0072 \cdot 931{,}5\,\text{МэВ} \approx 6{,}71\,\text{МэВ}
    \end{align*}
}
\solutionspace{150pt}

\tasknumber{5}%
\task{%
    Сделайте схематичный рисунок энергетических уровней атома водорода
    и отметьте на нём первый (основной) уровень и последующие.
    Сколько различных длин волн может испустить атом водорода,
    находящийся в 5-м возбуждённом состоянии?
    Отметьте все соответствующие переходы на рисунке и укажите,
    при каком переходе (среди отмеченных) длина волны излучённого фотона минимальна.
}
\answer{%
    $N = 10{,}0, \text{самая длинная линия}$
}
% autogenerated
