\setdate{30~апреля~2021}
\setclass{9«М»}

\addpersonalvariant{Михаил Бурмистров}

\tasknumber{1}%
\task{%
    В ядре электрически нейтрального атома 121 частиц.
    Вокруг ядра обращается 51 электронов.
    Сколько в ядре этого атома протонов и нейтронов?
    Назовите этот элемент.
}
\answer{%
    $Z = 51$ протонов и $A - Z = 70$ нейтронов
}
\solutionspace{120pt}

\tasknumber{2}%
\task{%
    Запишите реакцию $\beta$-распада $\ce{^{22}_{11}{Na}}$.
}
\solutionspace{80pt}

\tasknumber{3}%
\task{%
    Энергия связи ядра трития \ce{^{3}_{1}H} (T) равна $8{,}48\,\text{МэВ}$.
    Найти дефект массы этого ядра.
    Ответ выразите в а.е.м.
    и кг.
    Скорость света $c = 2{,}998 \cdot 10^{8}\,\frac{\text{м}}{\text{с}}$, элементарный заряд $e = 1{,}6 \cdot 10^{-19}\,\text{Кл}$.
}
\answer{%
    \begin{align*}
    E_\text{св.} &= \Delta m c^2 \implies \\
    \implies
            \Delta m &= \frac {E_\text{св.}}{c^2} = \frac{ 8{,}48\,\text{МэВ} }{ \sqr{ 3 \cdot 10^{8}\,\frac{\text{м}}{\text{с}} } }
            = \frac{8{,}48 \cdot 10^6 \cdot 1{,}6 \cdot 10^{-19}\,\text{Дж}}{ \sqr{ 3 \cdot 10^{8}\,\frac{\text{м}}{\text{с}} } }
            \approx 1{,}508 \cdot 10^{-29}\,\text{кг} \approx 0{,}00908\,\text{а.е.м.}
    \end{align*}
}
\solutionspace{100pt}

\tasknumber{4}%
\task{%
    Определите деффект массы (в а.е.м.) и энергию связи (в МэВ) ядра атома \ce{^{6}_{2}{He}},
    если его масса составляет $6{,}0189\,\text{а.е.м.}$.
    Считать $m_{p} = 1{,}00728\,\text{а.е.м.}$, $m_{n} = 1{,}00867\,\text{а.е.м.}$.
}
\solutionspace{150pt}

\tasknumber{5}%
\task{%
    Сделайте схематичный рисунок энергетических уровней атома водорода
    и отметьте на нём первый (основной) уровень и последующие.
    Сколько различных длин волн может испустить атом водорода,
    находящийся в 4-м возбуждённом состоянии?
    Отметьте все соответствующие переходы на рисунке и укажите,
    при каком переходе (среди отмеченных) длина волны излучённого фотона максимальна.
}
\answer{%
    $N = 6{,}0, \text{самая короткая линия}$
}

\variantsplitter

\addpersonalvariant{Артём Глембо}

\tasknumber{1}%
\task{%
    В ядре электрически нейтрального атома 123 частиц.
    Вокруг ядра обращается 51 электронов.
    Сколько в ядре этого атома протонов и нейтронов?
    Назовите этот элемент.
}
\answer{%
    $Z = 51$ протонов и $A - Z = 72$ нейтронов
}
\solutionspace{120pt}

\tasknumber{2}%
\task{%
    Запишите реакцию $\beta$-распада $\ce{^{22}_{11}{Na}}$.
}
\solutionspace{80pt}

\tasknumber{3}%
\task{%
    Энергия связи ядра дейтерия \ce{^{2}_{1}H} (D) равна $2{,}22\,\text{МэВ}$.
    Найти дефект массы этого ядра.
    Ответ выразите в а.е.м.
    и кг.
    Скорость света $c = 2{,}998 \cdot 10^{8}\,\frac{\text{м}}{\text{с}}$, элементарный заряд $e = 1{,}6 \cdot 10^{-19}\,\text{Кл}$.
}
\answer{%
    \begin{align*}
    E_\text{св.} &= \Delta m c^2 \implies \\
    \implies
            \Delta m &= \frac {E_\text{св.}}{c^2} = \frac{ 2{,}22\,\text{МэВ} }{ \sqr{ 3 \cdot 10^{8}\,\frac{\text{м}}{\text{с}} } }
            = \frac{2{,}22 \cdot 10^6 \cdot 1{,}6 \cdot 10^{-19}\,\text{Дж}}{ \sqr{ 3 \cdot 10^{8}\,\frac{\text{м}}{\text{с}} } }
            \approx 0{,}395 \cdot 10^{-29}\,\text{кг} \approx 0{,}00238\,\text{а.е.м.}
    \end{align*}
}
\solutionspace{100pt}

\tasknumber{4}%
\task{%
    Определите деффект массы (в а.е.м.) и энергию связи (в МэВ) ядра атома \ce{^{3}_{2}{He}},
    если его масса составляет $3{,}01603\,\text{а.е.м.}$.
    Считать $m_{p} = 1{,}00728\,\text{а.е.м.}$, $m_{n} = 1{,}00867\,\text{а.е.м.}$.
}
\solutionspace{150pt}

\tasknumber{5}%
\task{%
    Сделайте схематичный рисунок энергетических уровней атома водорода
    и отметьте на нём первый (основной) уровень и последующие.
    Сколько различных длин волн может испустить атом водорода,
    находящийся в 3-м возбуждённом состоянии?
    Отметьте все соответствующие переходы на рисунке и укажите,
    при каком переходе (среди отмеченных) частота излучённого фотона минимальна.
}
\answer{%
    $N = 3{,}0, \text{самая короткая линия}$
}

\variantsplitter

\addpersonalvariant{Наталья Гончарова}

\tasknumber{1}%
\task{%
    В ядре электрически нейтрального атома 123 частиц.
    Вокруг ядра обращается 51 электронов.
    Сколько в ядре этого атома протонов и нейтронов?
    Назовите этот элемент.
}
\answer{%
    $Z = 51$ протонов и $A - Z = 72$ нейтронов
}
\solutionspace{120pt}

\tasknumber{2}%
\task{%
    Запишите реакцию $\beta$-распада $\ce{^{22}_{11}{Na}}$.
}
\solutionspace{80pt}

\tasknumber{3}%
\task{%
    Энергия связи ядра лития \ce{^{6}_{3}Li} равна $31{,}99\,\text{МэВ}$.
    Найти дефект массы этого ядра.
    Ответ выразите в а.е.м.
    и кг.
    Скорость света $c = 2{,}998 \cdot 10^{8}\,\frac{\text{м}}{\text{с}}$, элементарный заряд $e = 1{,}6 \cdot 10^{-19}\,\text{Кл}$.
}
\answer{%
    \begin{align*}
    E_\text{св.} &= \Delta m c^2 \implies \\
    \implies
            \Delta m &= \frac {E_\text{св.}}{c^2} = \frac{ 31{,}99\,\text{МэВ} }{ \sqr{ 3 \cdot 10^{8}\,\frac{\text{м}}{\text{с}} } }
            = \frac{31{,}99 \cdot 10^6 \cdot 1{,}6 \cdot 10^{-19}\,\text{Дж}}{ \sqr{ 3 \cdot 10^{8}\,\frac{\text{м}}{\text{с}} } }
            \approx 5{,}69 \cdot 10^{-29}\,\text{кг} \approx 0{,}0342\,\text{а.е.м.}
    \end{align*}
}
\solutionspace{100pt}

\tasknumber{4}%
\task{%
    Определите деффект массы (в а.е.м.) и энергию связи (в МэВ) ядра атома \ce{^{4}_{2}{He}},
    если его масса составляет $4{,}0026\,\text{а.е.м.}$.
    Считать $m_{p} = 1{,}00728\,\text{а.е.м.}$, $m_{n} = 1{,}00867\,\text{а.е.м.}$.
}
\solutionspace{150pt}

\tasknumber{5}%
\task{%
    Сделайте схематичный рисунок энергетических уровней атома водорода
    и отметьте на нём первый (основной) уровень и последующие.
    Сколько различных длин волн может испустить атом водорода,
    находящийся в 5-м возбуждённом состоянии?
    Отметьте все соответствующие переходы на рисунке и укажите,
    при каком переходе (среди отмеченных) энергия излучённого фотона максимальна.
}
\answer{%
    $N = 10{,}0, \text{самая длинная линия}$
}

\variantsplitter

\addpersonalvariant{Файёзбек Касымов}

\tasknumber{1}%
\task{%
    В ядре электрически нейтрального атома 123 частиц.
    Вокруг ядра обращается 51 электронов.
    Сколько в ядре этого атома протонов и нейтронов?
    Назовите этот элемент.
}
\answer{%
    $Z = 51$ протонов и $A - Z = 72$ нейтронов
}
\solutionspace{120pt}

\tasknumber{2}%
\task{%
    Запишите реакцию $\beta$-распада $\ce{^{137}_{55}{Cs}}$.
}
\solutionspace{80pt}

\tasknumber{3}%
\task{%
    Энергия связи ядра азота \ce{^{14}_{7}N} равна $115{,}5\,\text{МэВ}$.
    Найти дефект массы этого ядра.
    Ответ выразите в а.е.м.
    и кг.
    Скорость света $c = 2{,}998 \cdot 10^{8}\,\frac{\text{м}}{\text{с}}$, элементарный заряд $e = 1{,}6 \cdot 10^{-19}\,\text{Кл}$.
}
\answer{%
    \begin{align*}
    E_\text{св.} &= \Delta m c^2 \implies \\
    \implies
            \Delta m &= \frac {E_\text{св.}}{c^2} = \frac{ 115{,}5\,\text{МэВ} }{ \sqr{ 3 \cdot 10^{8}\,\frac{\text{м}}{\text{с}} } }
            = \frac{115{,}5 \cdot 10^6 \cdot 1{,}6 \cdot 10^{-19}\,\text{Дж}}{ \sqr{ 3 \cdot 10^{8}\,\frac{\text{м}}{\text{с}} } }
            \approx 20{,}5 \cdot 10^{-29}\,\text{кг} \approx 0{,}1237\,\text{а.е.м.}
    \end{align*}
}
\solutionspace{100pt}

\tasknumber{4}%
\task{%
    Определите деффект массы (в а.е.м.) и энергию связи (в МэВ) ядра атома \ce{^{6}_{2}{He}},
    если его масса составляет $6{,}0189\,\text{а.е.м.}$.
    Считать $m_{p} = 1{,}00728\,\text{а.е.м.}$, $m_{n} = 1{,}00867\,\text{а.е.м.}$.
}
\solutionspace{150pt}

\tasknumber{5}%
\task{%
    Сделайте схематичный рисунок энергетических уровней атома водорода
    и отметьте на нём первый (основной) уровень и последующие.
    Сколько различных длин волн может испустить атом водорода,
    находящийся в 4-м возбуждённом состоянии?
    Отметьте все соответствующие переходы на рисунке и укажите,
    при каком переходе (среди отмеченных) длина волны излучённого фотона максимальна.
}
\answer{%
    $N = 6{,}0, \text{самая короткая линия}$
}

\variantsplitter

\addpersonalvariant{Александр Козинец}

\tasknumber{1}%
\task{%
    В ядре электрически нейтрального атома 190 частиц.
    Вокруг ядра обращается 78 электронов.
    Сколько в ядре этого атома протонов и нейтронов?
    Назовите этот элемент.
}
\answer{%
    $Z = 78$ протонов и $A - Z = 112$ нейтронов
}
\solutionspace{120pt}

\tasknumber{2}%
\task{%
    Запишите реакцию $\alpha$-распада $\ce{^{153}_{63}{Eu}}$.
}
\solutionspace{80pt}

\tasknumber{3}%
\task{%
    Энергия связи ядра углерода \ce{^{13}_{6}C} равна $97{,}1\,\text{МэВ}$.
    Найти дефект массы этого ядра.
    Ответ выразите в а.е.м.
    и кг.
    Скорость света $c = 2{,}998 \cdot 10^{8}\,\frac{\text{м}}{\text{с}}$, элементарный заряд $e = 1{,}6 \cdot 10^{-19}\,\text{Кл}$.
}
\answer{%
    \begin{align*}
    E_\text{св.} &= \Delta m c^2 \implies \\
    \implies
            \Delta m &= \frac {E_\text{св.}}{c^2} = \frac{ 97{,}1\,\text{МэВ} }{ \sqr{ 3 \cdot 10^{8}\,\frac{\text{м}}{\text{с}} } }
            = \frac{97{,}1 \cdot 10^6 \cdot 1{,}6 \cdot 10^{-19}\,\text{Дж}}{ \sqr{ 3 \cdot 10^{8}\,\frac{\text{м}}{\text{с}} } }
            \approx 17{,}26 \cdot 10^{-29}\,\text{кг} \approx 0{,}1040\,\text{а.е.м.}
    \end{align*}
}
\solutionspace{100pt}

\tasknumber{4}%
\task{%
    Определите деффект массы (в а.е.м.) и энергию связи (в МэВ) ядра атома \ce{^{8}_{2}{He}},
    если его масса составляет $8{,}0225\,\text{а.е.м.}$.
    Считать $m_{p} = 1{,}00728\,\text{а.е.м.}$, $m_{n} = 1{,}00867\,\text{а.е.м.}$.
}
\solutionspace{150pt}

\tasknumber{5}%
\task{%
    Сделайте схематичный рисунок энергетических уровней атома водорода
    и отметьте на нём первый (основной) уровень и последующие.
    Сколько различных длин волн может испустить атом водорода,
    находящийся в 4-м возбуждённом состоянии?
    Отметьте все соответствующие переходы на рисунке и укажите,
    при каком переходе (среди отмеченных) энергия излучённого фотона минимальна.
}
\answer{%
    $N = 6{,}0, \text{самая короткая линия}$
}

\variantsplitter

\addpersonalvariant{Андрей Куликовский}

\tasknumber{1}%
\task{%
    В ядре электрически нейтрального атома 63 частиц.
    Вокруг ядра обращается 29 электронов.
    Сколько в ядре этого атома протонов и нейтронов?
    Назовите этот элемент.
}
\answer{%
    $Z = 29$ протонов и $A - Z = 34$ нейтронов
}
\solutionspace{120pt}

\tasknumber{2}%
\task{%
    Запишите реакцию $\alpha$-распада $\ce{^{147}_{62}{Sm}}$.
}
\solutionspace{80pt}

\tasknumber{3}%
\task{%
    Энергия связи ядра лития \ce{^{6}_{3}Li} равна $31{,}99\,\text{МэВ}$.
    Найти дефект массы этого ядра.
    Ответ выразите в а.е.м.
    и кг.
    Скорость света $c = 2{,}998 \cdot 10^{8}\,\frac{\text{м}}{\text{с}}$, элементарный заряд $e = 1{,}6 \cdot 10^{-19}\,\text{Кл}$.
}
\answer{%
    \begin{align*}
    E_\text{св.} &= \Delta m c^2 \implies \\
    \implies
            \Delta m &= \frac {E_\text{св.}}{c^2} = \frac{ 31{,}99\,\text{МэВ} }{ \sqr{ 3 \cdot 10^{8}\,\frac{\text{м}}{\text{с}} } }
            = \frac{31{,}99 \cdot 10^6 \cdot 1{,}6 \cdot 10^{-19}\,\text{Дж}}{ \sqr{ 3 \cdot 10^{8}\,\frac{\text{м}}{\text{с}} } }
            \approx 5{,}69 \cdot 10^{-29}\,\text{кг} \approx 0{,}0342\,\text{а.е.м.}
    \end{align*}
}
\solutionspace{100pt}

\tasknumber{4}%
\task{%
    Определите деффект массы (в а.е.м.) и энергию связи (в МэВ) ядра атома \ce{^{6}_{2}{He}},
    если его масса составляет $6{,}0189\,\text{а.е.м.}$.
    Считать $m_{p} = 1{,}00728\,\text{а.е.м.}$, $m_{n} = 1{,}00867\,\text{а.е.м.}$.
}
\solutionspace{150pt}

\tasknumber{5}%
\task{%
    Сделайте схематичный рисунок энергетических уровней атома водорода
    и отметьте на нём первый (основной) уровень и последующие.
    Сколько различных длин волн может испустить атом водорода,
    находящийся в 4-м возбуждённом состоянии?
    Отметьте все соответствующие переходы на рисунке и укажите,
    при каком переходе (среди отмеченных) частота излучённого фотона минимальна.
}
\answer{%
    $N = 6{,}0, \text{самая короткая линия}$
}

\variantsplitter

\addpersonalvariant{Полина Лоткова}

\tasknumber{1}%
\task{%
    В ядре электрически нейтрального атома 121 частиц.
    Вокруг ядра обращается 51 электронов.
    Сколько в ядре этого атома протонов и нейтронов?
    Назовите этот элемент.
}
\answer{%
    $Z = 51$ протонов и $A - Z = 70$ нейтронов
}
\solutionspace{120pt}

\tasknumber{2}%
\task{%
    Запишите реакцию $\alpha$-распада $\ce{^{147}_{62}{Sm}}$.
}
\solutionspace{80pt}

\tasknumber{3}%
\task{%
    Энергия связи ядра бора \ce{^{11}_{5}B} равна $76{,}2\,\text{МэВ}$.
    Найти дефект массы этого ядра.
    Ответ выразите в а.е.м.
    и кг.
    Скорость света $c = 2{,}998 \cdot 10^{8}\,\frac{\text{м}}{\text{с}}$, элементарный заряд $e = 1{,}6 \cdot 10^{-19}\,\text{Кл}$.
}
\answer{%
    \begin{align*}
    E_\text{св.} &= \Delta m c^2 \implies \\
    \implies
            \Delta m &= \frac {E_\text{св.}}{c^2} = \frac{ 76{,}2\,\text{МэВ} }{ \sqr{ 3 \cdot 10^{8}\,\frac{\text{м}}{\text{с}} } }
            = \frac{76{,}2 \cdot 10^6 \cdot 1{,}6 \cdot 10^{-19}\,\text{Дж}}{ \sqr{ 3 \cdot 10^{8}\,\frac{\text{м}}{\text{с}} } }
            \approx 13{,}55 \cdot 10^{-29}\,\text{кг} \approx 0{,}0816\,\text{а.е.м.}
    \end{align*}
}
\solutionspace{100pt}

\tasknumber{4}%
\task{%
    Определите деффект массы (в а.е.м.) и энергию связи (в МэВ) ядра атома \ce{^{3}_{1}{T}},
    если его масса составляет $3{,}01605\,\text{а.е.м.}$.
    Считать $m_{p} = 1{,}00728\,\text{а.е.м.}$, $m_{n} = 1{,}00867\,\text{а.е.м.}$.
}
\solutionspace{150pt}

\tasknumber{5}%
\task{%
    Сделайте схематичный рисунок энергетических уровней атома водорода
    и отметьте на нём первый (основной) уровень и последующие.
    Сколько различных длин волн может испустить атом водорода,
    находящийся в 5-м возбуждённом состоянии?
    Отметьте все соответствующие переходы на рисунке и укажите,
    при каком переходе (среди отмеченных) частота излучённого фотона минимальна.
}
\answer{%
    $N = 10{,}0, \text{самая короткая линия}$
}

\variantsplitter

\addpersonalvariant{Екатерина Медведева}

\tasknumber{1}%
\task{%
    В ядре электрически нейтрального атома 121 частиц.
    Вокруг ядра обращается 51 электронов.
    Сколько в ядре этого атома протонов и нейтронов?
    Назовите этот элемент.
}
\answer{%
    $Z = 51$ протонов и $A - Z = 70$ нейтронов
}
\solutionspace{120pt}

\tasknumber{2}%
\task{%
    Запишите реакцию $\alpha$-распада $\ce{^{147}_{62}{Sm}}$.
}
\solutionspace{80pt}

\tasknumber{3}%
\task{%
    Энергия связи ядра углерода \ce{^{13}_{6}C} равна $97{,}1\,\text{МэВ}$.
    Найти дефект массы этого ядра.
    Ответ выразите в а.е.м.
    и кг.
    Скорость света $c = 2{,}998 \cdot 10^{8}\,\frac{\text{м}}{\text{с}}$, элементарный заряд $e = 1{,}6 \cdot 10^{-19}\,\text{Кл}$.
}
\answer{%
    \begin{align*}
    E_\text{св.} &= \Delta m c^2 \implies \\
    \implies
            \Delta m &= \frac {E_\text{св.}}{c^2} = \frac{ 97{,}1\,\text{МэВ} }{ \sqr{ 3 \cdot 10^{8}\,\frac{\text{м}}{\text{с}} } }
            = \frac{97{,}1 \cdot 10^6 \cdot 1{,}6 \cdot 10^{-19}\,\text{Дж}}{ \sqr{ 3 \cdot 10^{8}\,\frac{\text{м}}{\text{с}} } }
            \approx 17{,}26 \cdot 10^{-29}\,\text{кг} \approx 0{,}1040\,\text{а.е.м.}
    \end{align*}
}
\solutionspace{100pt}

\tasknumber{4}%
\task{%
    Определите деффект массы (в а.е.м.) и энергию связи (в МэВ) ядра атома \ce{^{2}_{1}{D}},
    если его масса составляет $2{,}0141\,\text{а.е.м.}$.
    Считать $m_{p} = 1{,}00728\,\text{а.е.м.}$, $m_{n} = 1{,}00867\,\text{а.е.м.}$.
}
\solutionspace{150pt}

\tasknumber{5}%
\task{%
    Сделайте схематичный рисунок энергетических уровней атома водорода
    и отметьте на нём первый (основной) уровень и последующие.
    Сколько различных длин волн может испустить атом водорода,
    находящийся в 4-м возбуждённом состоянии?
    Отметьте все соответствующие переходы на рисунке и укажите,
    при каком переходе (среди отмеченных) энергия излучённого фотона максимальна.
}
\answer{%
    $N = 6{,}0, \text{самая длинная линия}$
}

\variantsplitter

\addpersonalvariant{Константин Мельник}

\tasknumber{1}%
\task{%
    В ядре электрически нейтрального атома 190 частиц.
    Вокруг ядра обращается 78 электронов.
    Сколько в ядре этого атома протонов и нейтронов?
    Назовите этот элемент.
}
\answer{%
    $Z = 78$ протонов и $A - Z = 112$ нейтронов
}
\solutionspace{120pt}

\tasknumber{2}%
\task{%
    Запишите реакцию $\beta$-распада $\ce{^{137}_{55}{Cs}}$.
}
\solutionspace{80pt}

\tasknumber{3}%
\task{%
    Энергия связи ядра лития \ce{^{6}_{3}Li} равна $31{,}99\,\text{МэВ}$.
    Найти дефект массы этого ядра.
    Ответ выразите в а.е.м.
    и кг.
    Скорость света $c = 2{,}998 \cdot 10^{8}\,\frac{\text{м}}{\text{с}}$, элементарный заряд $e = 1{,}6 \cdot 10^{-19}\,\text{Кл}$.
}
\answer{%
    \begin{align*}
    E_\text{св.} &= \Delta m c^2 \implies \\
    \implies
            \Delta m &= \frac {E_\text{св.}}{c^2} = \frac{ 31{,}99\,\text{МэВ} }{ \sqr{ 3 \cdot 10^{8}\,\frac{\text{м}}{\text{с}} } }
            = \frac{31{,}99 \cdot 10^6 \cdot 1{,}6 \cdot 10^{-19}\,\text{Дж}}{ \sqr{ 3 \cdot 10^{8}\,\frac{\text{м}}{\text{с}} } }
            \approx 5{,}69 \cdot 10^{-29}\,\text{кг} \approx 0{,}0342\,\text{а.е.м.}
    \end{align*}
}
\solutionspace{100pt}

\tasknumber{4}%
\task{%
    Определите деффект массы (в а.е.м.) и энергию связи (в МэВ) ядра атома \ce{^{3}_{2}{He}},
    если его масса составляет $3{,}01603\,\text{а.е.м.}$.
    Считать $m_{p} = 1{,}00728\,\text{а.е.м.}$, $m_{n} = 1{,}00867\,\text{а.е.м.}$.
}
\solutionspace{150pt}

\tasknumber{5}%
\task{%
    Сделайте схематичный рисунок энергетических уровней атома водорода
    и отметьте на нём первый (основной) уровень и последующие.
    Сколько различных длин волн может испустить атом водорода,
    находящийся в 4-м возбуждённом состоянии?
    Отметьте все соответствующие переходы на рисунке и укажите,
    при каком переходе (среди отмеченных) частота излучённого фотона максимальна.
}
\answer{%
    $N = 6{,}0, \text{самая длинная линия}$
}

\variantsplitter

\addpersonalvariant{Степан Небоваренков}

\tasknumber{1}%
\task{%
    В ядре электрически нейтрального атома 108 частиц.
    Вокруг ядра обращается 47 электронов.
    Сколько в ядре этого атома протонов и нейтронов?
    Назовите этот элемент.
}
\answer{%
    $Z = 47$ протонов и $A - Z = 61$ нейтронов
}
\solutionspace{120pt}

\tasknumber{2}%
\task{%
    Запишите реакцию $\alpha$-распада $\ce{^{180}_{74}{W}}$.
}
\solutionspace{80pt}

\tasknumber{3}%
\task{%
    Энергия связи ядра бора \ce{^{10}_{5}B} равна $64{,}7\,\text{МэВ}$.
    Найти дефект массы этого ядра.
    Ответ выразите в а.е.м.
    и кг.
    Скорость света $c = 2{,}998 \cdot 10^{8}\,\frac{\text{м}}{\text{с}}$, элементарный заряд $e = 1{,}6 \cdot 10^{-19}\,\text{Кл}$.
}
\answer{%
    \begin{align*}
    E_\text{св.} &= \Delta m c^2 \implies \\
    \implies
            \Delta m &= \frac {E_\text{св.}}{c^2} = \frac{ 64{,}7\,\text{МэВ} }{ \sqr{ 3 \cdot 10^{8}\,\frac{\text{м}}{\text{с}} } }
            = \frac{64{,}7 \cdot 10^6 \cdot 1{,}6 \cdot 10^{-19}\,\text{Дж}}{ \sqr{ 3 \cdot 10^{8}\,\frac{\text{м}}{\text{с}} } }
            \approx 11{,}50 \cdot 10^{-29}\,\text{кг} \approx 0{,}0693\,\text{а.е.м.}
    \end{align*}
}
\solutionspace{100pt}

\tasknumber{4}%
\task{%
    Определите деффект массы (в а.е.м.) и энергию связи (в МэВ) ядра атома \ce{^{2}_{1}{D}},
    если его масса составляет $2{,}0141\,\text{а.е.м.}$.
    Считать $m_{p} = 1{,}00728\,\text{а.е.м.}$, $m_{n} = 1{,}00867\,\text{а.е.м.}$.
}
\solutionspace{150pt}

\tasknumber{5}%
\task{%
    Сделайте схематичный рисунок энергетических уровней атома водорода
    и отметьте на нём первый (основной) уровень и последующие.
    Сколько различных длин волн может испустить атом водорода,
    находящийся в 5-м возбуждённом состоянии?
    Отметьте все соответствующие переходы на рисунке и укажите,
    при каком переходе (среди отмеченных) длина волны излучённого фотона минимальна.
}
\answer{%
    $N = 10{,}0, \text{самая длинная линия}$
}

\variantsplitter

\addpersonalvariant{Матвей Неретин}

\tasknumber{1}%
\task{%
    В ядре электрически нейтрального атома 190 частиц.
    Вокруг ядра обращается 78 электронов.
    Сколько в ядре этого атома протонов и нейтронов?
    Назовите этот элемент.
}
\answer{%
    $Z = 78$ протонов и $A - Z = 112$ нейтронов
}
\solutionspace{120pt}

\tasknumber{2}%
\task{%
    Запишите реакцию $\beta$-распада $\ce{^{22}_{11}{Na}}$.
}
\solutionspace{80pt}

\tasknumber{3}%
\task{%
    Энергия связи ядра лития \ce{^{6}_{3}Li} равна $31{,}99\,\text{МэВ}$.
    Найти дефект массы этого ядра.
    Ответ выразите в а.е.м.
    и кг.
    Скорость света $c = 2{,}998 \cdot 10^{8}\,\frac{\text{м}}{\text{с}}$, элементарный заряд $e = 1{,}6 \cdot 10^{-19}\,\text{Кл}$.
}
\answer{%
    \begin{align*}
    E_\text{св.} &= \Delta m c^2 \implies \\
    \implies
            \Delta m &= \frac {E_\text{св.}}{c^2} = \frac{ 31{,}99\,\text{МэВ} }{ \sqr{ 3 \cdot 10^{8}\,\frac{\text{м}}{\text{с}} } }
            = \frac{31{,}99 \cdot 10^6 \cdot 1{,}6 \cdot 10^{-19}\,\text{Дж}}{ \sqr{ 3 \cdot 10^{8}\,\frac{\text{м}}{\text{с}} } }
            \approx 5{,}69 \cdot 10^{-29}\,\text{кг} \approx 0{,}0342\,\text{а.е.м.}
    \end{align*}
}
\solutionspace{100pt}

\tasknumber{4}%
\task{%
    Определите деффект массы (в а.е.м.) и энергию связи (в МэВ) ядра атома \ce{^{6}_{2}{He}},
    если его масса составляет $6{,}0189\,\text{а.е.м.}$.
    Считать $m_{p} = 1{,}00728\,\text{а.е.м.}$, $m_{n} = 1{,}00867\,\text{а.е.м.}$.
}
\solutionspace{150pt}

\tasknumber{5}%
\task{%
    Сделайте схематичный рисунок энергетических уровней атома водорода
    и отметьте на нём первый (основной) уровень и последующие.
    Сколько различных длин волн может испустить атом водорода,
    находящийся в 5-м возбуждённом состоянии?
    Отметьте все соответствующие переходы на рисунке и укажите,
    при каком переходе (среди отмеченных) частота излучённого фотона максимальна.
}
\answer{%
    $N = 10{,}0, \text{самая длинная линия}$
}

\variantsplitter

\addpersonalvariant{Мария Никонова}

\tasknumber{1}%
\task{%
    В ядре электрически нейтрального атома 63 частиц.
    Вокруг ядра обращается 29 электронов.
    Сколько в ядре этого атома протонов и нейтронов?
    Назовите этот элемент.
}
\answer{%
    $Z = 29$ протонов и $A - Z = 34$ нейтронов
}
\solutionspace{120pt}

\tasknumber{2}%
\task{%
    Запишите реакцию $\beta$-распада $\ce{^{137}_{55}{Cs}}$.
}
\solutionspace{80pt}

\tasknumber{3}%
\task{%
    Энергия связи ядра азота \ce{^{14}_{7}N} равна $115{,}5\,\text{МэВ}$.
    Найти дефект массы этого ядра.
    Ответ выразите в а.е.м.
    и кг.
    Скорость света $c = 2{,}998 \cdot 10^{8}\,\frac{\text{м}}{\text{с}}$, элементарный заряд $e = 1{,}6 \cdot 10^{-19}\,\text{Кл}$.
}
\answer{%
    \begin{align*}
    E_\text{св.} &= \Delta m c^2 \implies \\
    \implies
            \Delta m &= \frac {E_\text{св.}}{c^2} = \frac{ 115{,}5\,\text{МэВ} }{ \sqr{ 3 \cdot 10^{8}\,\frac{\text{м}}{\text{с}} } }
            = \frac{115{,}5 \cdot 10^6 \cdot 1{,}6 \cdot 10^{-19}\,\text{Дж}}{ \sqr{ 3 \cdot 10^{8}\,\frac{\text{м}}{\text{с}} } }
            \approx 20{,}5 \cdot 10^{-29}\,\text{кг} \approx 0{,}1237\,\text{а.е.м.}
    \end{align*}
}
\solutionspace{100pt}

\tasknumber{4}%
\task{%
    Определите деффект массы (в а.е.м.) и энергию связи (в МэВ) ядра атома \ce{^{3}_{1}{T}},
    если его масса составляет $3{,}01605\,\text{а.е.м.}$.
    Считать $m_{p} = 1{,}00728\,\text{а.е.м.}$, $m_{n} = 1{,}00867\,\text{а.е.м.}$.
}
\solutionspace{150pt}

\tasknumber{5}%
\task{%
    Сделайте схематичный рисунок энергетических уровней атома водорода
    и отметьте на нём первый (основной) уровень и последующие.
    Сколько различных длин волн может испустить атом водорода,
    находящийся в 3-м возбуждённом состоянии?
    Отметьте все соответствующие переходы на рисунке и укажите,
    при каком переходе (среди отмеченных) частота излучённого фотона минимальна.
}
\answer{%
    $N = 3{,}0, \text{самая короткая линия}$
}

\variantsplitter

\addpersonalvariant{Даниил Палаткин}

\tasknumber{1}%
\task{%
    В ядре электрически нейтрального атома 121 частиц.
    Вокруг ядра обращается 51 электронов.
    Сколько в ядре этого атома протонов и нейтронов?
    Назовите этот элемент.
}
\answer{%
    $Z = 51$ протонов и $A - Z = 70$ нейтронов
}
\solutionspace{120pt}

\tasknumber{2}%
\task{%
    Запишите реакцию $\beta$-распада $\ce{^{137}_{55}{Cs}}$.
}
\solutionspace{80pt}

\tasknumber{3}%
\task{%
    Энергия связи ядра бора \ce{^{11}_{5}B} равна $76{,}2\,\text{МэВ}$.
    Найти дефект массы этого ядра.
    Ответ выразите в а.е.м.
    и кг.
    Скорость света $c = 2{,}998 \cdot 10^{8}\,\frac{\text{м}}{\text{с}}$, элементарный заряд $e = 1{,}6 \cdot 10^{-19}\,\text{Кл}$.
}
\answer{%
    \begin{align*}
    E_\text{св.} &= \Delta m c^2 \implies \\
    \implies
            \Delta m &= \frac {E_\text{св.}}{c^2} = \frac{ 76{,}2\,\text{МэВ} }{ \sqr{ 3 \cdot 10^{8}\,\frac{\text{м}}{\text{с}} } }
            = \frac{76{,}2 \cdot 10^6 \cdot 1{,}6 \cdot 10^{-19}\,\text{Дж}}{ \sqr{ 3 \cdot 10^{8}\,\frac{\text{м}}{\text{с}} } }
            \approx 13{,}55 \cdot 10^{-29}\,\text{кг} \approx 0{,}0816\,\text{а.е.м.}
    \end{align*}
}
\solutionspace{100pt}

\tasknumber{4}%
\task{%
    Определите деффект массы (в а.е.м.) и энергию связи (в МэВ) ядра атома \ce{^{6}_{2}{He}},
    если его масса составляет $6{,}0189\,\text{а.е.м.}$.
    Считать $m_{p} = 1{,}00728\,\text{а.е.м.}$, $m_{n} = 1{,}00867\,\text{а.е.м.}$.
}
\solutionspace{150pt}

\tasknumber{5}%
\task{%
    Сделайте схематичный рисунок энергетических уровней атома водорода
    и отметьте на нём первый (основной) уровень и последующие.
    Сколько различных длин волн может испустить атом водорода,
    находящийся в 4-м возбуждённом состоянии?
    Отметьте все соответствующие переходы на рисунке и укажите,
    при каком переходе (среди отмеченных) энергия излучённого фотона минимальна.
}
\answer{%
    $N = 6{,}0, \text{самая короткая линия}$
}

\variantsplitter

\addpersonalvariant{Станислав Пикун}

\tasknumber{1}%
\task{%
    В ядре электрически нейтрального атома 63 частиц.
    Вокруг ядра обращается 29 электронов.
    Сколько в ядре этого атома протонов и нейтронов?
    Назовите этот элемент.
}
\answer{%
    $Z = 29$ протонов и $A - Z = 34$ нейтронов
}
\solutionspace{120pt}

\tasknumber{2}%
\task{%
    Запишите реакцию $\beta$-распада $\ce{^{137}_{55}{Cs}}$.
}
\solutionspace{80pt}

\tasknumber{3}%
\task{%
    Энергия связи ядра гелия \ce{^{3}_{2}He} равна $7{,}72\,\text{МэВ}$.
    Найти дефект массы этого ядра.
    Ответ выразите в а.е.м.
    и кг.
    Скорость света $c = 2{,}998 \cdot 10^{8}\,\frac{\text{м}}{\text{с}}$, элементарный заряд $e = 1{,}6 \cdot 10^{-19}\,\text{Кл}$.
}
\answer{%
    \begin{align*}
    E_\text{св.} &= \Delta m c^2 \implies \\
    \implies
            \Delta m &= \frac {E_\text{св.}}{c^2} = \frac{ 7{,}72\,\text{МэВ} }{ \sqr{ 3 \cdot 10^{8}\,\frac{\text{м}}{\text{с}} } }
            = \frac{7{,}72 \cdot 10^6 \cdot 1{,}6 \cdot 10^{-19}\,\text{Дж}}{ \sqr{ 3 \cdot 10^{8}\,\frac{\text{м}}{\text{с}} } }
            \approx 1{,}372 \cdot 10^{-29}\,\text{кг} \approx 0{,}00827\,\text{а.е.м.}
    \end{align*}
}
\solutionspace{100pt}

\tasknumber{4}%
\task{%
    Определите деффект массы (в а.е.м.) и энергию связи (в МэВ) ядра атома \ce{^{3}_{2}{He}},
    если его масса составляет $3{,}01603\,\text{а.е.м.}$.
    Считать $m_{p} = 1{,}00728\,\text{а.е.м.}$, $m_{n} = 1{,}00867\,\text{а.е.м.}$.
}
\solutionspace{150pt}

\tasknumber{5}%
\task{%
    Сделайте схематичный рисунок энергетических уровней атома водорода
    и отметьте на нём первый (основной) уровень и последующие.
    Сколько различных длин волн может испустить атом водорода,
    находящийся в 4-м возбуждённом состоянии?
    Отметьте все соответствующие переходы на рисунке и укажите,
    при каком переходе (среди отмеченных) частота излучённого фотона минимальна.
}
\answer{%
    $N = 6{,}0, \text{самая короткая линия}$
}

\variantsplitter

\addpersonalvariant{Илья Пичугин}

\tasknumber{1}%
\task{%
    В ядре электрически нейтрального атома 121 частиц.
    Вокруг ядра обращается 51 электронов.
    Сколько в ядре этого атома протонов и нейтронов?
    Назовите этот элемент.
}
\answer{%
    $Z = 51$ протонов и $A - Z = 70$ нейтронов
}
\solutionspace{120pt}

\tasknumber{2}%
\task{%
    Запишите реакцию $\alpha$-распада $\ce{^{180}_{74}{W}}$.
}
\solutionspace{80pt}

\tasknumber{3}%
\task{%
    Энергия связи ядра бора \ce{^{11}_{5}B} равна $76{,}2\,\text{МэВ}$.
    Найти дефект массы этого ядра.
    Ответ выразите в а.е.м.
    и кг.
    Скорость света $c = 2{,}998 \cdot 10^{8}\,\frac{\text{м}}{\text{с}}$, элементарный заряд $e = 1{,}6 \cdot 10^{-19}\,\text{Кл}$.
}
\answer{%
    \begin{align*}
    E_\text{св.} &= \Delta m c^2 \implies \\
    \implies
            \Delta m &= \frac {E_\text{св.}}{c^2} = \frac{ 76{,}2\,\text{МэВ} }{ \sqr{ 3 \cdot 10^{8}\,\frac{\text{м}}{\text{с}} } }
            = \frac{76{,}2 \cdot 10^6 \cdot 1{,}6 \cdot 10^{-19}\,\text{Дж}}{ \sqr{ 3 \cdot 10^{8}\,\frac{\text{м}}{\text{с}} } }
            \approx 13{,}55 \cdot 10^{-29}\,\text{кг} \approx 0{,}0816\,\text{а.е.м.}
    \end{align*}
}
\solutionspace{100pt}

\tasknumber{4}%
\task{%
    Определите деффект массы (в а.е.м.) и энергию связи (в МэВ) ядра атома \ce{^{8}_{2}{He}},
    если его масса составляет $8{,}0225\,\text{а.е.м.}$.
    Считать $m_{p} = 1{,}00728\,\text{а.е.м.}$, $m_{n} = 1{,}00867\,\text{а.е.м.}$.
}
\solutionspace{150pt}

\tasknumber{5}%
\task{%
    Сделайте схематичный рисунок энергетических уровней атома водорода
    и отметьте на нём первый (основной) уровень и последующие.
    Сколько различных длин волн может испустить атом водорода,
    находящийся в 4-м возбуждённом состоянии?
    Отметьте все соответствующие переходы на рисунке и укажите,
    при каком переходе (среди отмеченных) длина волны излучённого фотона максимальна.
}
\answer{%
    $N = 6{,}0, \text{самая короткая линия}$
}

\variantsplitter

\addpersonalvariant{Кирилл Севрюгин}

\tasknumber{1}%
\task{%
    В ядре электрически нейтрального атома 108 частиц.
    Вокруг ядра обращается 47 электронов.
    Сколько в ядре этого атома протонов и нейтронов?
    Назовите этот элемент.
}
\answer{%
    $Z = 47$ протонов и $A - Z = 61$ нейтронов
}
\solutionspace{120pt}

\tasknumber{2}%
\task{%
    Запишите реакцию $\alpha$-распада $\ce{^{144}_{60}{Nd}}$.
}
\solutionspace{80pt}

\tasknumber{3}%
\task{%
    Энергия связи ядра бора \ce{^{11}_{5}B} равна $76{,}2\,\text{МэВ}$.
    Найти дефект массы этого ядра.
    Ответ выразите в а.е.м.
    и кг.
    Скорость света $c = 2{,}998 \cdot 10^{8}\,\frac{\text{м}}{\text{с}}$, элементарный заряд $e = 1{,}6 \cdot 10^{-19}\,\text{Кл}$.
}
\answer{%
    \begin{align*}
    E_\text{св.} &= \Delta m c^2 \implies \\
    \implies
            \Delta m &= \frac {E_\text{св.}}{c^2} = \frac{ 76{,}2\,\text{МэВ} }{ \sqr{ 3 \cdot 10^{8}\,\frac{\text{м}}{\text{с}} } }
            = \frac{76{,}2 \cdot 10^6 \cdot 1{,}6 \cdot 10^{-19}\,\text{Дж}}{ \sqr{ 3 \cdot 10^{8}\,\frac{\text{м}}{\text{с}} } }
            \approx 13{,}55 \cdot 10^{-29}\,\text{кг} \approx 0{,}0816\,\text{а.е.м.}
    \end{align*}
}
\solutionspace{100pt}

\tasknumber{4}%
\task{%
    Определите деффект массы (в а.е.м.) и энергию связи (в МэВ) ядра атома \ce{^{3}_{1}{T}},
    если его масса составляет $3{,}01605\,\text{а.е.м.}$.
    Считать $m_{p} = 1{,}00728\,\text{а.е.м.}$, $m_{n} = 1{,}00867\,\text{а.е.м.}$.
}
\solutionspace{150pt}

\tasknumber{5}%
\task{%
    Сделайте схематичный рисунок энергетических уровней атома водорода
    и отметьте на нём первый (основной) уровень и последующие.
    Сколько различных длин волн может испустить атом водорода,
    находящийся в 4-м возбуждённом состоянии?
    Отметьте все соответствующие переходы на рисунке и укажите,
    при каком переходе (среди отмеченных) длина волны излучённого фотона максимальна.
}
\answer{%
    $N = 6{,}0, \text{самая короткая линия}$
}

\variantsplitter

\addpersonalvariant{Илья Стратонников}

\tasknumber{1}%
\task{%
    В ядре электрически нейтрального атома 108 частиц.
    Вокруг ядра обращается 47 электронов.
    Сколько в ядре этого атома протонов и нейтронов?
    Назовите этот элемент.
}
\answer{%
    $Z = 47$ протонов и $A - Z = 61$ нейтронов
}
\solutionspace{120pt}

\tasknumber{2}%
\task{%
    Запишите реакцию $\beta$-распада $\ce{^{22}_{11}{Na}}$.
}
\solutionspace{80pt}

\tasknumber{3}%
\task{%
    Энергия связи ядра кислорода \ce{^{16}_{8}O} равна $127{,}6\,\text{МэВ}$.
    Найти дефект массы этого ядра.
    Ответ выразите в а.е.м.
    и кг.
    Скорость света $c = 2{,}998 \cdot 10^{8}\,\frac{\text{м}}{\text{с}}$, элементарный заряд $e = 1{,}6 \cdot 10^{-19}\,\text{Кл}$.
}
\answer{%
    \begin{align*}
    E_\text{св.} &= \Delta m c^2 \implies \\
    \implies
            \Delta m &= \frac {E_\text{св.}}{c^2} = \frac{ 127{,}6\,\text{МэВ} }{ \sqr{ 3 \cdot 10^{8}\,\frac{\text{м}}{\text{с}} } }
            = \frac{127{,}6 \cdot 10^6 \cdot 1{,}6 \cdot 10^{-19}\,\text{Дж}}{ \sqr{ 3 \cdot 10^{8}\,\frac{\text{м}}{\text{с}} } }
            \approx 22{,}7 \cdot 10^{-29}\,\text{кг} \approx 0{,}1366\,\text{а.е.м.}
    \end{align*}
}
\solutionspace{100pt}

\tasknumber{4}%
\task{%
    Определите деффект массы (в а.е.м.) и энергию связи (в МэВ) ядра атома \ce{^{2}_{1}{D}},
    если его масса составляет $2{,}0141\,\text{а.е.м.}$.
    Считать $m_{p} = 1{,}00728\,\text{а.е.м.}$, $m_{n} = 1{,}00867\,\text{а.е.м.}$.
}
\solutionspace{150pt}

\tasknumber{5}%
\task{%
    Сделайте схематичный рисунок энергетических уровней атома водорода
    и отметьте на нём первый (основной) уровень и последующие.
    Сколько различных длин волн может испустить атом водорода,
    находящийся в 3-м возбуждённом состоянии?
    Отметьте все соответствующие переходы на рисунке и укажите,
    при каком переходе (среди отмеченных) частота излучённого фотона минимальна.
}
\answer{%
    $N = 3{,}0, \text{самая короткая линия}$
}

\variantsplitter

\addpersonalvariant{Иван Шустов}

\tasknumber{1}%
\task{%
    В ядре электрически нейтрального атома 121 частиц.
    Вокруг ядра обращается 51 электронов.
    Сколько в ядре этого атома протонов и нейтронов?
    Назовите этот элемент.
}
\answer{%
    $Z = 51$ протонов и $A - Z = 70$ нейтронов
}
\solutionspace{120pt}

\tasknumber{2}%
\task{%
    Запишите реакцию $\beta$-распада $\ce{^{22}_{11}{Na}}$.
}
\solutionspace{80pt}

\tasknumber{3}%
\task{%
    Энергия связи ядра азота \ce{^{14}_{7}N} равна $104{,}7\,\text{МэВ}$.
    Найти дефект массы этого ядра.
    Ответ выразите в а.е.м.
    и кг.
    Скорость света $c = 2{,}998 \cdot 10^{8}\,\frac{\text{м}}{\text{с}}$, элементарный заряд $e = 1{,}6 \cdot 10^{-19}\,\text{Кл}$.
}
\answer{%
    \begin{align*}
    E_\text{св.} &= \Delta m c^2 \implies \\
    \implies
            \Delta m &= \frac {E_\text{св.}}{c^2} = \frac{ 104{,}7\,\text{МэВ} }{ \sqr{ 3 \cdot 10^{8}\,\frac{\text{м}}{\text{с}} } }
            = \frac{104{,}7 \cdot 10^6 \cdot 1{,}6 \cdot 10^{-19}\,\text{Дж}}{ \sqr{ 3 \cdot 10^{8}\,\frac{\text{м}}{\text{с}} } }
            \approx 18{,}61 \cdot 10^{-29}\,\text{кг} \approx 0{,}1121\,\text{а.е.м.}
    \end{align*}
}
\solutionspace{100pt}

\tasknumber{4}%
\task{%
    Определите деффект массы (в а.е.м.) и энергию связи (в МэВ) ядра атома \ce{^{3}_{2}{He}},
    если его масса составляет $3{,}01603\,\text{а.е.м.}$.
    Считать $m_{p} = 1{,}00728\,\text{а.е.м.}$, $m_{n} = 1{,}00867\,\text{а.е.м.}$.
}
\solutionspace{150pt}

\tasknumber{5}%
\task{%
    Сделайте схематичный рисунок энергетических уровней атома водорода
    и отметьте на нём первый (основной) уровень и последующие.
    Сколько различных длин волн может испустить атом водорода,
    находящийся в 5-м возбуждённом состоянии?
    Отметьте все соответствующие переходы на рисунке и укажите,
    при каком переходе (среди отмеченных) длина волны излучённого фотона минимальна.
}
\answer{%
    $N = 10{,}0, \text{самая длинная линия}$
}
% autogenerated
