\setdate{26~ноября~2020}
\setclass{10«АБ»}

\addpersonalvariant{Михаил Бурмистров}

\tasknumber{1}%
\task{%
    Шарики массами $1\,\text{кг}$ и $4\,\text{кг}$ движутся параллельно друг другу в одном направлении
    со скоростями $5\,\frac{\text{м}}{\text{с}}$ и $6\,\frac{\text{м}}{\text{с}}$ соответственно.
    Определите общий импульс шариков.
}
\answer{%
    \begin{align*}
    p_1 &= m_1v_1 = 1\,\text{кг} \cdot 5\,\frac{\text{м}}{\text{с}} = 5\,\frac{\text{кг}\cdot\text{м}}{\text{с}}, \\
    p_2 &= m_2v_2 = 4\,\text{кг} \cdot 6\,\frac{\text{м}}{\text{с}} = 24\,\frac{\text{кг}\cdot\text{м}}{\text{с}}, \\
    p &= p_1 + p_2 = m_1v_1 + m_2v_2 = 29\,\frac{\text{кг}\cdot\text{м}}{\text{с}}.
    \end{align*}
}

\tasknumber{2}%
\task{%
    Два шарика, масса каждого из которых составляет $5\,\text{кг}$, движутся навстречу друг другу.
    Скорость одного из них $2\,\frac{\text{м}}{\text{с}}$, а другого~--- $6\,\frac{\text{м}}{\text{с}}$.
    Определите общий импульс шариков.
}
\answer{%
    \begin{align*}
    p_1 &= mv_1 = 5\,\text{кг} \cdot 2\,\frac{\text{м}}{\text{с}} = 10\,\frac{\text{кг}\cdot\text{м}}{\text{с}}, \\
    p_2 &= mv_2 = 5\,\text{кг} \cdot 6\,\frac{\text{м}}{\text{с}} = 30\,\frac{\text{кг}\cdot\text{м}}{\text{с}}, \\
    p &= \abs{p_1 - p_2} = \abs{m(v_1 - v_2)}= 20\,\frac{\text{кг}\cdot\text{м}}{\text{с}}.
    \end{align*}
}

\tasknumber{3}%
\task{%
    Два одинаковых шарика массами по $2\,\text{кг}$ движутся во взаимно перпендикулярных направлениях.
    Скорости шариков составляют $3\,\frac{\text{м}}{\text{с}}$ и $4\,\frac{\text{м}}{\text{с}}$.
    Определите полный импульс системы.
}
\answer{%
    \begin{align*}
    p_1 &= mv_1 = 2\,\text{кг} \cdot 3\,\frac{\text{м}}{\text{с}} = 6\,\frac{\text{кг}\cdot\text{м}}{\text{с}}, \\
    p_2 &= mv_2 = 2\,\text{кг} \cdot 4\,\frac{\text{м}}{\text{с}} = 8\,\frac{\text{кг}\cdot\text{м}}{\text{с}}, \\
    p &= \sqrt{p_1^2 + p_2^2} = m\sqrt{v_1^2 + v_2^2} = 10\,\frac{\text{кг}\cdot\text{м}}{\text{с}}.
    \end{align*}
}

\tasknumber{4}%
\task{%
    Шарик массой $1\,\text{кг}$ свободно упал на горизонтальную площадку, имея в момент падения скорость $15\,\frac{\text{м}}{\text{c}}$.
    Считая удар абсолютно упругим, определите изменение импульса шарика.
    В ответе укажите модуль полученной величины.
}
\answer{%
    \begin{align*}
    \Delta p &= 2 \cdot mv = 2 \cdot 1\,\text{кг} \cdot 15\,\frac{\text{м}}{\text{c}} = 30\,\frac{\text{кг}\cdot\text{м}}{\text{с}}.
    \end{align*}
}

\tasknumber{5}%
\task{%
    Два тела двигаются навстречу друг другу.
    Скорость каждого из них составляет $7\,\frac{\text{м}}{\text{с}}$.
    После соударения тела слиплись и продолжили движение уже со скоростью $5\,\frac{\text{м}}{\text{с}}$.
    Определите отношение масс тел (большей к меньшей).
}
\answer{%
    \begin{align*}
    &\text{ЗСИ в проекции на ось, соединяющую центры тел:} m_1 v_1 - m_2 v_1 = (m_1 + m_2) v_2 \implies \\
    &\implies \frac{m_1}{m_2} v_1 - v_1 = \cbr{\frac{m_1}{m_2} + 1} v_2 \implies
        \frac{m_1}{m_2} (v_1 - v_2) = v_2 + v_1 \implies \frac{m_1}{m_2} = \frac{v_2 + v_1}{v_1 - v_2} = 6
    \end{align*}
}

\tasknumber{6}%
\task{%
    Шар движется с некоторой скоростью и абсолютно неупруго соударяется с телом, масса которого в 7 раз больше.
    Определите во сколько раз уменьшилась скорость шара после столкновения.
}
\answer{%
    \begin{align*}
    &\text{ЗСИ в проекции на ось, соединяющую центры тел:}  \\
    &mv + 7m \cdot 0 = (m + 7m) v' \implies \\
    &v' = v\frac{m}{7m + m} = \frac{v}{7 + 1} \implies \frac{v}{v'} = 8
    \end{align*}
}

\tasknumber{7}%
\task{%
    Определите работу силы, которая обеспечит подъём тела массой $5\,\text{кг}$ на высоту $10\,\text{м}$ с постоянным ускорением $2\,\frac{\text{м}}{\text{c}^{2}}$.
    % Примите $g = 10\,\frac{\text{м}}{\text{с}^{2}}$.
}
\answer{%
    \begin{align*}
    &\text{Для подъёма:} A = Fh = (mg + ma) h = m(g+a)h, \\
    &\text{Для спуска:} A = -Fh = -(mg - ma) h = -m(g-a)h, \\
    &\text{В результате получаем:} 600\,\text{Дж}.
    \end{align*}
}

\tasknumber{8}%
\task{%
    Тело массой 3\,\text{кг} бросили с обрыва под углом $45\degrees$ к горизонту с начальной скоростью $2\,\frac{\text{м}}{\text{c}}$.
    Через некоторое время скорость тела составила $12\,\frac{\text{м}}{\text{c}}$.
    Пренебрегая сопротивлением воздуха и считая падение тела свободным, определите работу силы тяжести в течение наблюдаемого промежутка времени.
}
\answer{%
    \begin{align*}
    &\text{Изменение кинетической энергии равно работе внешних сил:} \\
    &\Delta E_k = E_k' - E_k = A_\text{тяж} \implies A_\text{тяж} = \frac{mv'^2}2 - \frac{mv_0^2}2 = 210\,\text{Дж}.
    \end{align*}
}

\tasknumber{9}%
\task{%
    Тонкий однородный шест длиной $3\,\text{м}$ и массой $20\,\text{кг}$ лежит на горизонтальной поверхности.
    \begin{itemize}
        \item Какую минимальную силу надо приложить к одному из его концов, чтобы оторвать его от этой поверхности?
        \item Какую минимальную работу надо совершить, чтобы поставить его на землю в вертикальное положение?
    \end{itemize}
    % Примите $g = 10\,\frac{\text{м}}{\text{с}^{2}}$.
}
\answer{%
    $F = \frac{mg}2 \approx 200\,\text{Н}, A = mg\frac l2 = 300\,\text{Дж}$
}

\tasknumber{10}%
\task{%
    Для того, чтобы разогать тело из состояния покоя до скорости $v$ с постоянным ускорением,
    требуется совершить работу $20\,\text{Дж}$.
    Какую работу нужно совершить, чтобы увеличить скорость этого тела от $v$ до $2v$?
}
\answer{%
    \begin{align*}
    &\text{Изменение кинетической энергии равно работе внешних сил:} \\
    &A_1 = \frac{mv^2}2 - \frac{m \cdot 0^2}2 = \frac{mv^2}2, A_2 = \frac{m\sqr{2v}}2 - \frac{mv^2}2 \implies  \\
    &\implies A_2 = \frac{mv^2}2 \cbr{2^2 - 1} = A_1 \cdot \cbr{2^2 - 1} = 60\,\text{Дж}.
    \end{align*}
}

\variantsplitter

\addpersonalvariant{Ирина Ан}

\tasknumber{1}%
\task{%
    Шарики массами $1\,\text{кг}$ и $4\,\text{кг}$ движутся параллельно друг другу в одном направлении
    со скоростями $10\,\frac{\text{м}}{\text{с}}$ и $3\,\frac{\text{м}}{\text{с}}$ соответственно.
    Определите общий импульс шариков.
}
\answer{%
    \begin{align*}
    p_1 &= m_1v_1 = 1\,\text{кг} \cdot 10\,\frac{\text{м}}{\text{с}} = 10\,\frac{\text{кг}\cdot\text{м}}{\text{с}}, \\
    p_2 &= m_2v_2 = 4\,\text{кг} \cdot 3\,\frac{\text{м}}{\text{с}} = 12\,\frac{\text{кг}\cdot\text{м}}{\text{с}}, \\
    p &= p_1 + p_2 = m_1v_1 + m_2v_2 = 22\,\frac{\text{кг}\cdot\text{м}}{\text{с}}.
    \end{align*}
}

\tasknumber{2}%
\task{%
    Два шарика, масса каждого из которых составляет $2\,\text{кг}$, движутся навстречу друг другу.
    Скорость одного из них $10\,\frac{\text{м}}{\text{с}}$, а другого~--- $8\,\frac{\text{м}}{\text{с}}$.
    Определите общий импульс шариков.
}
\answer{%
    \begin{align*}
    p_1 &= mv_1 = 2\,\text{кг} \cdot 10\,\frac{\text{м}}{\text{с}} = 20\,\frac{\text{кг}\cdot\text{м}}{\text{с}}, \\
    p_2 &= mv_2 = 2\,\text{кг} \cdot 8\,\frac{\text{м}}{\text{с}} = 16\,\frac{\text{кг}\cdot\text{м}}{\text{с}}, \\
    p &= \abs{p_1 - p_2} = \abs{m(v_1 - v_2)}= 4\,\frac{\text{кг}\cdot\text{м}}{\text{с}}.
    \end{align*}
}

\tasknumber{3}%
\task{%
    Два одинаковых шарика массами по $10\,\text{кг}$ движутся во взаимно перпендикулярных направлениях.
    Скорости шариков составляют $5\,\frac{\text{м}}{\text{с}}$ и $12\,\frac{\text{м}}{\text{с}}$.
    Определите полный импульс системы.
}
\answer{%
    \begin{align*}
    p_1 &= mv_1 = 10\,\text{кг} \cdot 5\,\frac{\text{м}}{\text{с}} = 50\,\frac{\text{кг}\cdot\text{м}}{\text{с}}, \\
    p_2 &= mv_2 = 10\,\text{кг} \cdot 12\,\frac{\text{м}}{\text{с}} = 120\,\frac{\text{кг}\cdot\text{м}}{\text{с}}, \\
    p &= \sqrt{p_1^2 + p_2^2} = m\sqrt{v_1^2 + v_2^2} = 130\,\frac{\text{кг}\cdot\text{м}}{\text{с}}.
    \end{align*}
}

\tasknumber{4}%
\task{%
    Шарик массой $2\,\text{кг}$ свободно упал на горизонтальную площадку, имея в момент падения скорость $10\,\frac{\text{м}}{\text{c}}$.
    Считая удар абсолютно неупругим, определите изменение импульса шарика.
    В ответе укажите модуль полученной величины.
}
\answer{%
    \begin{align*}
    \Delta p &= 1 \cdot mv = 1 \cdot 2\,\text{кг} \cdot 10\,\frac{\text{м}}{\text{c}} = 20\,\frac{\text{кг}\cdot\text{м}}{\text{с}}.
    \end{align*}
}

\tasknumber{5}%
\task{%
    Два тела двигаются навстречу друг другу.
    Скорость каждого из них составляет $2\,\frac{\text{м}}{\text{с}}$.
    После соударения тела слиплись и продолжили движение уже со скоростью $1\,\frac{\text{м}}{\text{с}}$.
    Определите отношение масс тел (большей к меньшей).
}
\answer{%
    \begin{align*}
    &\text{ЗСИ в проекции на ось, соединяющую центры тел:} m_1 v_1 - m_2 v_1 = (m_1 + m_2) v_2 \implies \\
    &\implies \frac{m_1}{m_2} v_1 - v_1 = \cbr{\frac{m_1}{m_2} + 1} v_2 \implies
        \frac{m_1}{m_2} (v_1 - v_2) = v_2 + v_1 \implies \frac{m_1}{m_2} = \frac{v_2 + v_1}{v_1 - v_2} = 3
    \end{align*}
}

\tasknumber{6}%
\task{%
    Шар движется с некоторой скоростью и абсолютно неупруго соударяется с телом, масса которого в 14 раз больше.
    Определите во сколько раз уменьшилась скорость шара после столкновения.
}
\answer{%
    \begin{align*}
    &\text{ЗСИ в проекции на ось, соединяющую центры тел:}  \\
    &mv + 14m \cdot 0 = (m + 14m) v' \implies \\
    &v' = v\frac{m}{14m + m} = \frac{v}{14 + 1} \implies \frac{v}{v'} = 15
    \end{align*}
}

\tasknumber{7}%
\task{%
    Определите работу силы, которая обеспечит спуск тела массой $2\,\text{кг}$ на высоту $10\,\text{м}$ с постоянным ускорением $4\,\frac{\text{м}}{\text{c}^{2}}$.
    % Примите $g = 10\,\frac{\text{м}}{\text{с}^{2}}$.
}
\answer{%
    \begin{align*}
    &\text{Для подъёма:} A = Fh = (mg + ma) h = m(g+a)h, \\
    &\text{Для спуска:} A = -Fh = -(mg - ma) h = -m(g-a)h, \\
    &\text{В результате получаем:} -120\,\text{Дж}.
    \end{align*}
}

\tasknumber{8}%
\task{%
    Тело массой 1\,\text{кг} бросили с обрыва горизонтально с начальной скоростью $4\,\frac{\text{м}}{\text{c}}$.
    Через некоторое время скорость тела составила $8\,\frac{\text{м}}{\text{c}}$.
    Пренебрегая сопротивлением воздуха и считая падение тела свободным, определите работу силы тяжести в течение наблюдаемого промежутка времени.
}
\answer{%
    \begin{align*}
    &\text{Изменение кинетической энергии равно работе внешних сил:} \\
    &\Delta E_k = E_k' - E_k = A_\text{тяж} \implies A_\text{тяж} = \frac{mv'^2}2 - \frac{mv_0^2}2 = 24\,\text{Дж}.
    \end{align*}
}

\tasknumber{9}%
\task{%
    Тонкий однородный шест длиной $3\,\text{м}$ и массой $30\,\text{кг}$ лежит на горизонтальной поверхности.
    \begin{itemize}
        \item Какую минимальную силу надо приложить к одному из его концов, чтобы оторвать его от этой поверхности?
        \item Какую минимальную работу надо совершить, чтобы поставить его на землю в вертикальное положение?
    \end{itemize}
    % Примите $g = 10\,\frac{\text{м}}{\text{с}^{2}}$.
}
\answer{%
    $F = \frac{mg}2 \approx 300\,\text{Н}, A = mg\frac l2 = 450\,\text{Дж}$
}

\tasknumber{10}%
\task{%
    Для того, чтобы разогать тело из состояния покоя до скорости $v$ с постоянным ускорением,
    требуется совершить работу $40\,\text{Дж}$.
    Какую работу нужно совершить, чтобы увеличить скорость этого тела от $v$ до $4v$?
}
\answer{%
    \begin{align*}
    &\text{Изменение кинетической энергии равно работе внешних сил:} \\
    &A_1 = \frac{mv^2}2 - \frac{m \cdot 0^2}2 = \frac{mv^2}2, A_2 = \frac{m\sqr{4v}}2 - \frac{mv^2}2 \implies  \\
    &\implies A_2 = \frac{mv^2}2 \cbr{4^2 - 1} = A_1 \cdot \cbr{4^2 - 1} = 600\,\text{Дж}.
    \end{align*}
}

\variantsplitter

\addpersonalvariant{Софья Андрианова}

\tasknumber{1}%
\task{%
    Шарики массами $2\,\text{кг}$ и $3\,\text{кг}$ движутся параллельно друг другу в одном направлении
    со скоростями $2\,\frac{\text{м}}{\text{с}}$ и $6\,\frac{\text{м}}{\text{с}}$ соответственно.
    Определите общий импульс шариков.
}
\answer{%
    \begin{align*}
    p_1 &= m_1v_1 = 2\,\text{кг} \cdot 2\,\frac{\text{м}}{\text{с}} = 4\,\frac{\text{кг}\cdot\text{м}}{\text{с}}, \\
    p_2 &= m_2v_2 = 3\,\text{кг} \cdot 6\,\frac{\text{м}}{\text{с}} = 18\,\frac{\text{кг}\cdot\text{м}}{\text{с}}, \\
    p &= p_1 + p_2 = m_1v_1 + m_2v_2 = 22\,\frac{\text{кг}\cdot\text{м}}{\text{с}}.
    \end{align*}
}

\tasknumber{2}%
\task{%
    Два шарика, масса каждого из которых составляет $2\,\text{кг}$, движутся навстречу друг другу.
    Скорость одного из них $2\,\frac{\text{м}}{\text{с}}$, а другого~--- $8\,\frac{\text{м}}{\text{с}}$.
    Определите общий импульс шариков.
}
\answer{%
    \begin{align*}
    p_1 &= mv_1 = 2\,\text{кг} \cdot 2\,\frac{\text{м}}{\text{с}} = 4\,\frac{\text{кг}\cdot\text{м}}{\text{с}}, \\
    p_2 &= mv_2 = 2\,\text{кг} \cdot 8\,\frac{\text{м}}{\text{с}} = 16\,\frac{\text{кг}\cdot\text{м}}{\text{с}}, \\
    p &= \abs{p_1 - p_2} = \abs{m(v_1 - v_2)}= 12\,\frac{\text{кг}\cdot\text{м}}{\text{с}}.
    \end{align*}
}

\tasknumber{3}%
\task{%
    Два одинаковых шарика массами по $2\,\text{кг}$ движутся во взаимно перпендикулярных направлениях.
    Скорости шариков составляют $5\,\frac{\text{м}}{\text{с}}$ и $12\,\frac{\text{м}}{\text{с}}$.
    Определите полный импульс системы.
}
\answer{%
    \begin{align*}
    p_1 &= mv_1 = 2\,\text{кг} \cdot 5\,\frac{\text{м}}{\text{с}} = 10\,\frac{\text{кг}\cdot\text{м}}{\text{с}}, \\
    p_2 &= mv_2 = 2\,\text{кг} \cdot 12\,\frac{\text{м}}{\text{с}} = 24\,\frac{\text{кг}\cdot\text{м}}{\text{с}}, \\
    p &= \sqrt{p_1^2 + p_2^2} = m\sqrt{v_1^2 + v_2^2} = 26\,\frac{\text{кг}\cdot\text{м}}{\text{с}}.
    \end{align*}
}

\tasknumber{4}%
\task{%
    Шарик массой $4\,\text{кг}$ свободно упал на горизонтальную площадку, имея в момент падения скорость $10\,\frac{\text{м}}{\text{c}}$.
    Считая удар абсолютно неупругим, определите изменение импульса шарика.
    В ответе укажите модуль полученной величины.
}
\answer{%
    \begin{align*}
    \Delta p &= 1 \cdot mv = 1 \cdot 4\,\text{кг} \cdot 10\,\frac{\text{м}}{\text{c}} = 40\,\frac{\text{кг}\cdot\text{м}}{\text{с}}.
    \end{align*}
}

\tasknumber{5}%
\task{%
    Два тела двигаются навстречу друг другу.
    Скорость каждого из них составляет $4\,\frac{\text{м}}{\text{с}}$.
    После соударения тела слиплись и продолжили движение уже со скоростью $3\,\frac{\text{м}}{\text{с}}$.
    Определите отношение масс тел (большей к меньшей).
}
\answer{%
    \begin{align*}
    &\text{ЗСИ в проекции на ось, соединяющую центры тел:} m_1 v_1 - m_2 v_1 = (m_1 + m_2) v_2 \implies \\
    &\implies \frac{m_1}{m_2} v_1 - v_1 = \cbr{\frac{m_1}{m_2} + 1} v_2 \implies
        \frac{m_1}{m_2} (v_1 - v_2) = v_2 + v_1 \implies \frac{m_1}{m_2} = \frac{v_2 + v_1}{v_1 - v_2} = 7
    \end{align*}
}

\tasknumber{6}%
\task{%
    Шар движется с некоторой скоростью и абсолютно неупруго соударяется с телом, масса которого в 10 раз больше.
    Определите во сколько раз уменьшилась скорость шара после столкновения.
}
\answer{%
    \begin{align*}
    &\text{ЗСИ в проекции на ось, соединяющую центры тел:}  \\
    &mv + 10m \cdot 0 = (m + 10m) v' \implies \\
    &v' = v\frac{m}{10m + m} = \frac{v}{10 + 1} \implies \frac{v}{v'} = 11
    \end{align*}
}

\tasknumber{7}%
\task{%
    Определите работу силы, которая обеспечит спуск тела массой $3\,\text{кг}$ на высоту $5\,\text{м}$ с постоянным ускорением $2\,\frac{\text{м}}{\text{c}^{2}}$.
    % Примите $g = 10\,\frac{\text{м}}{\text{с}^{2}}$.
}
\answer{%
    \begin{align*}
    &\text{Для подъёма:} A = Fh = (mg + ma) h = m(g+a)h, \\
    &\text{Для спуска:} A = -Fh = -(mg - ma) h = -m(g-a)h, \\
    &\text{В результате получаем:} -120\,\text{Дж}.
    \end{align*}
}

\tasknumber{8}%
\task{%
    Тело массой 2\,\text{кг} бросили с обрыва горизонтально с начальной скоростью $6\,\frac{\text{м}}{\text{c}}$.
    Через некоторое время скорость тела составила $8\,\frac{\text{м}}{\text{c}}$.
    Пренебрегая сопротивлением воздуха и считая падение тела свободным, определите работу силы тяжести в течение наблюдаемого промежутка времени.
}
\answer{%
    \begin{align*}
    &\text{Изменение кинетической энергии равно работе внешних сил:} \\
    &\Delta E_k = E_k' - E_k = A_\text{тяж} \implies A_\text{тяж} = \frac{mv'^2}2 - \frac{mv_0^2}2 = 28\,\text{Дж}.
    \end{align*}
}

\tasknumber{9}%
\task{%
    Тонкий однородный кусок арматуры длиной $1\,\text{м}$ и массой $20\,\text{кг}$ лежит на горизонтальной поверхности.
    \begin{itemize}
        \item Какую минимальную силу надо приложить к одному из его концов, чтобы оторвать его от этой поверхности?
        \item Какую минимальную работу надо совершить, чтобы поставить его на землю в вертикальное положение?
    \end{itemize}
    % Примите $g = 10\,\frac{\text{м}}{\text{с}^{2}}$.
}
\answer{%
    $F = \frac{mg}2 \approx 200\,\text{Н}, A = mg\frac l2 = 100\,\text{Дж}$
}

\tasknumber{10}%
\task{%
    Для того, чтобы разогать тело из состояния покоя до скорости $v$ с постоянным ускорением,
    требуется совершить работу $200\,\text{Дж}$.
    Какую работу нужно совершить, чтобы увеличить скорость этого тела от $v$ до $3v$?
}
\answer{%
    \begin{align*}
    &\text{Изменение кинетической энергии равно работе внешних сил:} \\
    &A_1 = \frac{mv^2}2 - \frac{m \cdot 0^2}2 = \frac{mv^2}2, A_2 = \frac{m\sqr{3v}}2 - \frac{mv^2}2 \implies  \\
    &\implies A_2 = \frac{mv^2}2 \cbr{3^2 - 1} = A_1 \cdot \cbr{3^2 - 1} = 1600\,\text{Дж}.
    \end{align*}
}

\variantsplitter

\addpersonalvariant{Владимир Артемчук}

\tasknumber{1}%
\task{%
    Шарики массами $1\,\text{кг}$ и $3\,\text{кг}$ движутся параллельно друг другу в одном направлении
    со скоростями $2\,\frac{\text{м}}{\text{с}}$ и $6\,\frac{\text{м}}{\text{с}}$ соответственно.
    Определите общий импульс шариков.
}
\answer{%
    \begin{align*}
    p_1 &= m_1v_1 = 1\,\text{кг} \cdot 2\,\frac{\text{м}}{\text{с}} = 2\,\frac{\text{кг}\cdot\text{м}}{\text{с}}, \\
    p_2 &= m_2v_2 = 3\,\text{кг} \cdot 6\,\frac{\text{м}}{\text{с}} = 18\,\frac{\text{кг}\cdot\text{м}}{\text{с}}, \\
    p &= p_1 + p_2 = m_1v_1 + m_2v_2 = 20\,\frac{\text{кг}\cdot\text{м}}{\text{с}}.
    \end{align*}
}

\tasknumber{2}%
\task{%
    Два шарика, масса каждого из которых составляет $5\,\text{кг}$, движутся навстречу друг другу.
    Скорость одного из них $10\,\frac{\text{м}}{\text{с}}$, а другого~--- $8\,\frac{\text{м}}{\text{с}}$.
    Определите общий импульс шариков.
}
\answer{%
    \begin{align*}
    p_1 &= mv_1 = 5\,\text{кг} \cdot 10\,\frac{\text{м}}{\text{с}} = 50\,\frac{\text{кг}\cdot\text{м}}{\text{с}}, \\
    p_2 &= mv_2 = 5\,\text{кг} \cdot 8\,\frac{\text{м}}{\text{с}} = 40\,\frac{\text{кг}\cdot\text{м}}{\text{с}}, \\
    p &= \abs{p_1 - p_2} = \abs{m(v_1 - v_2)}= 10\,\frac{\text{кг}\cdot\text{м}}{\text{с}}.
    \end{align*}
}

\tasknumber{3}%
\task{%
    Два одинаковых шарика массами по $2\,\text{кг}$ движутся во взаимно перпендикулярных направлениях.
    Скорости шариков составляют $7\,\frac{\text{м}}{\text{с}}$ и $24\,\frac{\text{м}}{\text{с}}$.
    Определите полный импульс системы.
}
\answer{%
    \begin{align*}
    p_1 &= mv_1 = 2\,\text{кг} \cdot 7\,\frac{\text{м}}{\text{с}} = 14\,\frac{\text{кг}\cdot\text{м}}{\text{с}}, \\
    p_2 &= mv_2 = 2\,\text{кг} \cdot 24\,\frac{\text{м}}{\text{с}} = 48\,\frac{\text{кг}\cdot\text{м}}{\text{с}}, \\
    p &= \sqrt{p_1^2 + p_2^2} = m\sqrt{v_1^2 + v_2^2} = 50\,\frac{\text{кг}\cdot\text{м}}{\text{с}}.
    \end{align*}
}

\tasknumber{4}%
\task{%
    Шарик массой $2\,\text{кг}$ свободно упал на горизонтальную площадку, имея в момент падения скорость $15\,\frac{\text{м}}{\text{c}}$.
    Считая удар абсолютно неупругим, определите изменение импульса шарика.
    В ответе укажите модуль полученной величины.
}
\answer{%
    \begin{align*}
    \Delta p &= 1 \cdot mv = 1 \cdot 2\,\text{кг} \cdot 15\,\frac{\text{м}}{\text{c}} = 30\,\frac{\text{кг}\cdot\text{м}}{\text{с}}.
    \end{align*}
}

\tasknumber{5}%
\task{%
    Два тела двигаются навстречу друг другу.
    Скорость каждого из них составляет $9\,\frac{\text{м}}{\text{с}}$.
    После соударения тела слиплись и продолжили движение уже со скоростью $7\,\frac{\text{м}}{\text{с}}$.
    Определите отношение масс тел (большей к меньшей).
}
\answer{%
    \begin{align*}
    &\text{ЗСИ в проекции на ось, соединяющую центры тел:} m_1 v_1 - m_2 v_1 = (m_1 + m_2) v_2 \implies \\
    &\implies \frac{m_1}{m_2} v_1 - v_1 = \cbr{\frac{m_1}{m_2} + 1} v_2 \implies
        \frac{m_1}{m_2} (v_1 - v_2) = v_2 + v_1 \implies \frac{m_1}{m_2} = \frac{v_2 + v_1}{v_1 - v_2} = 8
    \end{align*}
}

\tasknumber{6}%
\task{%
    Шар движется с некоторой скоростью и абсолютно неупруго соударяется с телом, масса которого в 12 раз больше.
    Определите во сколько раз уменьшилась скорость шара после столкновения.
}
\answer{%
    \begin{align*}
    &\text{ЗСИ в проекции на ось, соединяющую центры тел:}  \\
    &mv + 12m \cdot 0 = (m + 12m) v' \implies \\
    &v' = v\frac{m}{12m + m} = \frac{v}{12 + 1} \implies \frac{v}{v'} = 13
    \end{align*}
}

\tasknumber{7}%
\task{%
    Определите работу силы, которая обеспечит подъём тела массой $3\,\text{кг}$ на высоту $2\,\text{м}$ с постоянным ускорением $4\,\frac{\text{м}}{\text{c}^{2}}$.
    % Примите $g = 10\,\frac{\text{м}}{\text{с}^{2}}$.
}
\answer{%
    \begin{align*}
    &\text{Для подъёма:} A = Fh = (mg + ma) h = m(g+a)h, \\
    &\text{Для спуска:} A = -Fh = -(mg - ma) h = -m(g-a)h, \\
    &\text{В результате получаем:} 84\,\text{Дж}.
    \end{align*}
}

\tasknumber{8}%
\task{%
    Тело массой 2\,\text{кг} бросили с обрыва горизонтально с начальной скоростью $2\,\frac{\text{м}}{\text{c}}$.
    Через некоторое время скорость тела составила $10\,\frac{\text{м}}{\text{c}}$.
    Пренебрегая сопротивлением воздуха и считая падение тела свободным, определите работу силы тяжести в течение наблюдаемого промежутка времени.
}
\answer{%
    \begin{align*}
    &\text{Изменение кинетической энергии равно работе внешних сил:} \\
    &\Delta E_k = E_k' - E_k = A_\text{тяж} \implies A_\text{тяж} = \frac{mv'^2}2 - \frac{mv_0^2}2 = 96\,\text{Дж}.
    \end{align*}
}

\tasknumber{9}%
\task{%
    Тонкий однородный кусок арматуры длиной $1\,\text{м}$ и массой $10\,\text{кг}$ лежит на горизонтальной поверхности.
    \begin{itemize}
        \item Какую минимальную силу надо приложить к одному из его концов, чтобы оторвать его от этой поверхности?
        \item Какую минимальную работу надо совершить, чтобы поставить его на землю в вертикальное положение?
    \end{itemize}
    % Примите $g = 10\,\frac{\text{м}}{\text{с}^{2}}$.
}
\answer{%
    $F = \frac{mg}2 \approx 100\,\text{Н}, A = mg\frac l2 = 50\,\text{Дж}$
}

\tasknumber{10}%
\task{%
    Для того, чтобы разогать тело из состояния покоя до скорости $v$ с постоянным ускорением,
    требуется совершить работу $20\,\text{Дж}$.
    Какую работу нужно совершить, чтобы увеличить скорость этого тела от $v$ до $5v$?
}
\answer{%
    \begin{align*}
    &\text{Изменение кинетической энергии равно работе внешних сил:} \\
    &A_1 = \frac{mv^2}2 - \frac{m \cdot 0^2}2 = \frac{mv^2}2, A_2 = \frac{m\sqr{5v}}2 - \frac{mv^2}2 \implies  \\
    &\implies A_2 = \frac{mv^2}2 \cbr{5^2 - 1} = A_1 \cdot \cbr{5^2 - 1} = 480\,\text{Дж}.
    \end{align*}
}

\variantsplitter

\addpersonalvariant{Софья Белянкина}

\tasknumber{1}%
\task{%
    Шарики массами $4\,\text{кг}$ и $2\,\text{кг}$ движутся параллельно друг другу в одном направлении
    со скоростями $5\,\frac{\text{м}}{\text{с}}$ и $3\,\frac{\text{м}}{\text{с}}$ соответственно.
    Определите общий импульс шариков.
}
\answer{%
    \begin{align*}
    p_1 &= m_1v_1 = 4\,\text{кг} \cdot 5\,\frac{\text{м}}{\text{с}} = 20\,\frac{\text{кг}\cdot\text{м}}{\text{с}}, \\
    p_2 &= m_2v_2 = 2\,\text{кг} \cdot 3\,\frac{\text{м}}{\text{с}} = 6\,\frac{\text{кг}\cdot\text{м}}{\text{с}}, \\
    p &= p_1 + p_2 = m_1v_1 + m_2v_2 = 26\,\frac{\text{кг}\cdot\text{м}}{\text{с}}.
    \end{align*}
}

\tasknumber{2}%
\task{%
    Два шарика, масса каждого из которых составляет $2\,\text{кг}$, движутся навстречу друг другу.
    Скорость одного из них $2\,\frac{\text{м}}{\text{с}}$, а другого~--- $3\,\frac{\text{м}}{\text{с}}$.
    Определите общий импульс шариков.
}
\answer{%
    \begin{align*}
    p_1 &= mv_1 = 2\,\text{кг} \cdot 2\,\frac{\text{м}}{\text{с}} = 4\,\frac{\text{кг}\cdot\text{м}}{\text{с}}, \\
    p_2 &= mv_2 = 2\,\text{кг} \cdot 3\,\frac{\text{м}}{\text{с}} = 6\,\frac{\text{кг}\cdot\text{м}}{\text{с}}, \\
    p &= \abs{p_1 - p_2} = \abs{m(v_1 - v_2)}= 2\,\frac{\text{кг}\cdot\text{м}}{\text{с}}.
    \end{align*}
}

\tasknumber{3}%
\task{%
    Два одинаковых шарика массами по $2\,\text{кг}$ движутся во взаимно перпендикулярных направлениях.
    Скорости шариков составляют $5\,\frac{\text{м}}{\text{с}}$ и $12\,\frac{\text{м}}{\text{с}}$.
    Определите полный импульс системы.
}
\answer{%
    \begin{align*}
    p_1 &= mv_1 = 2\,\text{кг} \cdot 5\,\frac{\text{м}}{\text{с}} = 10\,\frac{\text{кг}\cdot\text{м}}{\text{с}}, \\
    p_2 &= mv_2 = 2\,\text{кг} \cdot 12\,\frac{\text{м}}{\text{с}} = 24\,\frac{\text{кг}\cdot\text{м}}{\text{с}}, \\
    p &= \sqrt{p_1^2 + p_2^2} = m\sqrt{v_1^2 + v_2^2} = 26\,\frac{\text{кг}\cdot\text{м}}{\text{с}}.
    \end{align*}
}

\tasknumber{4}%
\task{%
    Шарик массой $1\,\text{кг}$ свободно упал на горизонтальную площадку, имея в момент падения скорость $10\,\frac{\text{м}}{\text{c}}$.
    Считая удар абсолютно неупругим, определите изменение импульса шарика.
    В ответе укажите модуль полученной величины.
}
\answer{%
    \begin{align*}
    \Delta p &= 1 \cdot mv = 1 \cdot 1\,\text{кг} \cdot 10\,\frac{\text{м}}{\text{c}} = 10\,\frac{\text{кг}\cdot\text{м}}{\text{с}}.
    \end{align*}
}

\tasknumber{5}%
\task{%
    Два тела двигаются навстречу друг другу.
    Скорость каждого из них составляет $2\,\frac{\text{м}}{\text{с}}$.
    После соударения тела слиплись и продолжили движение уже со скоростью $1\,\frac{\text{м}}{\text{с}}$.
    Определите отношение масс тел (большей к меньшей).
}
\answer{%
    \begin{align*}
    &\text{ЗСИ в проекции на ось, соединяющую центры тел:} m_1 v_1 - m_2 v_1 = (m_1 + m_2) v_2 \implies \\
    &\implies \frac{m_1}{m_2} v_1 - v_1 = \cbr{\frac{m_1}{m_2} + 1} v_2 \implies
        \frac{m_1}{m_2} (v_1 - v_2) = v_2 + v_1 \implies \frac{m_1}{m_2} = \frac{v_2 + v_1}{v_1 - v_2} = 3
    \end{align*}
}

\tasknumber{6}%
\task{%
    Шар движется с некоторой скоростью и абсолютно неупруго соударяется с телом, масса которого в 9 раз больше.
    Определите во сколько раз уменьшилась скорость шара после столкновения.
}
\answer{%
    \begin{align*}
    &\text{ЗСИ в проекции на ось, соединяющую центры тел:}  \\
    &mv + 9m \cdot 0 = (m + 9m) v' \implies \\
    &v' = v\frac{m}{9m + m} = \frac{v}{9 + 1} \implies \frac{v}{v'} = 10
    \end{align*}
}

\tasknumber{7}%
\task{%
    Определите работу силы, которая обеспечит подъём тела массой $2\,\text{кг}$ на высоту $10\,\text{м}$ с постоянным ускорением $2\,\frac{\text{м}}{\text{c}^{2}}$.
    % Примите $g = 10\,\frac{\text{м}}{\text{с}^{2}}$.
}
\answer{%
    \begin{align*}
    &\text{Для подъёма:} A = Fh = (mg + ma) h = m(g+a)h, \\
    &\text{Для спуска:} A = -Fh = -(mg - ma) h = -m(g-a)h, \\
    &\text{В результате получаем:} 240\,\text{Дж}.
    \end{align*}
}

\tasknumber{8}%
\task{%
    Тело массой 2\,\text{кг} бросили с обрыва вертикально вверх с начальной скоростью $4\,\frac{\text{м}}{\text{c}}$.
    Через некоторое время скорость тела составила $10\,\frac{\text{м}}{\text{c}}$.
    Пренебрегая сопротивлением воздуха и считая падение тела свободным, определите работу силы тяжести в течение наблюдаемого промежутка времени.
}
\answer{%
    \begin{align*}
    &\text{Изменение кинетической энергии равно работе внешних сил:} \\
    &\Delta E_k = E_k' - E_k = A_\text{тяж} \implies A_\text{тяж} = \frac{mv'^2}2 - \frac{mv_0^2}2 = 84\,\text{Дж}.
    \end{align*}
}

\tasknumber{9}%
\task{%
    Тонкий однородный лом длиной $2\,\text{м}$ и массой $10\,\text{кг}$ лежит на горизонтальной поверхности.
    \begin{itemize}
        \item Какую минимальную силу надо приложить к одному из его концов, чтобы оторвать его от этой поверхности?
        \item Какую минимальную работу надо совершить, чтобы поставить его на землю в вертикальное положение?
    \end{itemize}
    % Примите $g = 10\,\frac{\text{м}}{\text{с}^{2}}$.
}
\answer{%
    $F = \frac{mg}2 \approx 100\,\text{Н}, A = mg\frac l2 = 100\,\text{Дж}$
}

\tasknumber{10}%
\task{%
    Для того, чтобы разогать тело из состояния покоя до скорости $v$ с постоянным ускорением,
    требуется совершить работу $10\,\text{Дж}$.
    Какую работу нужно совершить, чтобы увеличить скорость этого тела от $v$ до $4v$?
}
\answer{%
    \begin{align*}
    &\text{Изменение кинетической энергии равно работе внешних сил:} \\
    &A_1 = \frac{mv^2}2 - \frac{m \cdot 0^2}2 = \frac{mv^2}2, A_2 = \frac{m\sqr{4v}}2 - \frac{mv^2}2 \implies  \\
    &\implies A_2 = \frac{mv^2}2 \cbr{4^2 - 1} = A_1 \cdot \cbr{4^2 - 1} = 150\,\text{Дж}.
    \end{align*}
}

\variantsplitter

\addpersonalvariant{Варвара Егиазарян}

\tasknumber{1}%
\task{%
    Шарики массами $3\,\text{кг}$ и $2\,\text{кг}$ движутся параллельно друг другу в одном направлении
    со скоростями $2\,\frac{\text{м}}{\text{с}}$ и $3\,\frac{\text{м}}{\text{с}}$ соответственно.
    Определите общий импульс шариков.
}
\answer{%
    \begin{align*}
    p_1 &= m_1v_1 = 3\,\text{кг} \cdot 2\,\frac{\text{м}}{\text{с}} = 6\,\frac{\text{кг}\cdot\text{м}}{\text{с}}, \\
    p_2 &= m_2v_2 = 2\,\text{кг} \cdot 3\,\frac{\text{м}}{\text{с}} = 6\,\frac{\text{кг}\cdot\text{м}}{\text{с}}, \\
    p &= p_1 + p_2 = m_1v_1 + m_2v_2 = 12\,\frac{\text{кг}\cdot\text{м}}{\text{с}}.
    \end{align*}
}

\tasknumber{2}%
\task{%
    Два шарика, масса каждого из которых составляет $5\,\text{кг}$, движутся навстречу друг другу.
    Скорость одного из них $1\,\frac{\text{м}}{\text{с}}$, а другого~--- $3\,\frac{\text{м}}{\text{с}}$.
    Определите общий импульс шариков.
}
\answer{%
    \begin{align*}
    p_1 &= mv_1 = 5\,\text{кг} \cdot 1\,\frac{\text{м}}{\text{с}} = 5\,\frac{\text{кг}\cdot\text{м}}{\text{с}}, \\
    p_2 &= mv_2 = 5\,\text{кг} \cdot 3\,\frac{\text{м}}{\text{с}} = 15\,\frac{\text{кг}\cdot\text{м}}{\text{с}}, \\
    p &= \abs{p_1 - p_2} = \abs{m(v_1 - v_2)}= 10\,\frac{\text{кг}\cdot\text{м}}{\text{с}}.
    \end{align*}
}

\tasknumber{3}%
\task{%
    Два одинаковых шарика массами по $2\,\text{кг}$ движутся во взаимно перпендикулярных направлениях.
    Скорости шариков составляют $3\,\frac{\text{м}}{\text{с}}$ и $4\,\frac{\text{м}}{\text{с}}$.
    Определите полный импульс системы.
}
\answer{%
    \begin{align*}
    p_1 &= mv_1 = 2\,\text{кг} \cdot 3\,\frac{\text{м}}{\text{с}} = 6\,\frac{\text{кг}\cdot\text{м}}{\text{с}}, \\
    p_2 &= mv_2 = 2\,\text{кг} \cdot 4\,\frac{\text{м}}{\text{с}} = 8\,\frac{\text{кг}\cdot\text{м}}{\text{с}}, \\
    p &= \sqrt{p_1^2 + p_2^2} = m\sqrt{v_1^2 + v_2^2} = 10\,\frac{\text{кг}\cdot\text{м}}{\text{с}}.
    \end{align*}
}

\tasknumber{4}%
\task{%
    Шарик массой $1\,\text{кг}$ свободно упал на горизонтальную площадку, имея в момент падения скорость $15\,\frac{\text{м}}{\text{c}}$.
    Считая удар абсолютно неупругим, определите изменение импульса шарика.
    В ответе укажите модуль полученной величины.
}
\answer{%
    \begin{align*}
    \Delta p &= 1 \cdot mv = 1 \cdot 1\,\text{кг} \cdot 15\,\frac{\text{м}}{\text{c}} = 15\,\frac{\text{кг}\cdot\text{м}}{\text{с}}.
    \end{align*}
}

\tasknumber{5}%
\task{%
    Два тела двигаются навстречу друг другу.
    Скорость каждого из них составляет $5\,\frac{\text{м}}{\text{с}}$.
    После соударения тела слиплись и продолжили движение уже со скоростью $4\,\frac{\text{м}}{\text{с}}$.
    Определите отношение масс тел (большей к меньшей).
}
\answer{%
    \begin{align*}
    &\text{ЗСИ в проекции на ось, соединяющую центры тел:} m_1 v_1 - m_2 v_1 = (m_1 + m_2) v_2 \implies \\
    &\implies \frac{m_1}{m_2} v_1 - v_1 = \cbr{\frac{m_1}{m_2} + 1} v_2 \implies
        \frac{m_1}{m_2} (v_1 - v_2) = v_2 + v_1 \implies \frac{m_1}{m_2} = \frac{v_2 + v_1}{v_1 - v_2} = 9
    \end{align*}
}

\tasknumber{6}%
\task{%
    Шар движется с некоторой скоростью и абсолютно неупруго соударяется с телом, масса которого в 10 раз больше.
    Определите во сколько раз уменьшилась скорость шара после столкновения.
}
\answer{%
    \begin{align*}
    &\text{ЗСИ в проекции на ось, соединяющую центры тел:}  \\
    &mv + 10m \cdot 0 = (m + 10m) v' \implies \\
    &v' = v\frac{m}{10m + m} = \frac{v}{10 + 1} \implies \frac{v}{v'} = 11
    \end{align*}
}

\tasknumber{7}%
\task{%
    Определите работу силы, которая обеспечит подъём тела массой $3\,\text{кг}$ на высоту $5\,\text{м}$ с постоянным ускорением $2\,\frac{\text{м}}{\text{c}^{2}}$.
    % Примите $g = 10\,\frac{\text{м}}{\text{с}^{2}}$.
}
\answer{%
    \begin{align*}
    &\text{Для подъёма:} A = Fh = (mg + ma) h = m(g+a)h, \\
    &\text{Для спуска:} A = -Fh = -(mg - ma) h = -m(g-a)h, \\
    &\text{В результате получаем:} 180\,\text{Дж}.
    \end{align*}
}

\tasknumber{8}%
\task{%
    Тело массой 3\,\text{кг} бросили с обрыва вертикально вверх с начальной скоростью $2\,\frac{\text{м}}{\text{c}}$.
    Через некоторое время скорость тела составила $10\,\frac{\text{м}}{\text{c}}$.
    Пренебрегая сопротивлением воздуха и считая падение тела свободным, определите работу силы тяжести в течение наблюдаемого промежутка времени.
}
\answer{%
    \begin{align*}
    &\text{Изменение кинетической энергии равно работе внешних сил:} \\
    &\Delta E_k = E_k' - E_k = A_\text{тяж} \implies A_\text{тяж} = \frac{mv'^2}2 - \frac{mv_0^2}2 = 144\,\text{Дж}.
    \end{align*}
}

\tasknumber{9}%
\task{%
    Тонкий однородный шест длиной $3\,\text{м}$ и массой $10\,\text{кг}$ лежит на горизонтальной поверхности.
    \begin{itemize}
        \item Какую минимальную силу надо приложить к одному из его концов, чтобы оторвать его от этой поверхности?
        \item Какую минимальную работу надо совершить, чтобы поставить его на землю в вертикальное положение?
    \end{itemize}
    % Примите $g = 10\,\frac{\text{м}}{\text{с}^{2}}$.
}
\answer{%
    $F = \frac{mg}2 \approx 100\,\text{Н}, A = mg\frac l2 = 150\,\text{Дж}$
}

\tasknumber{10}%
\task{%
    Для того, чтобы разогать тело из состояния покоя до скорости $v$ с постоянным ускорением,
    требуется совершить работу $10\,\text{Дж}$.
    Какую работу нужно совершить, чтобы увеличить скорость этого тела от $v$ до $2v$?
}
\answer{%
    \begin{align*}
    &\text{Изменение кинетической энергии равно работе внешних сил:} \\
    &A_1 = \frac{mv^2}2 - \frac{m \cdot 0^2}2 = \frac{mv^2}2, A_2 = \frac{m\sqr{2v}}2 - \frac{mv^2}2 \implies  \\
    &\implies A_2 = \frac{mv^2}2 \cbr{2^2 - 1} = A_1 \cdot \cbr{2^2 - 1} = 30\,\text{Дж}.
    \end{align*}
}

\variantsplitter

\addpersonalvariant{Владислав Емелин}

\tasknumber{1}%
\task{%
    Шарики массами $4\,\text{кг}$ и $3\,\text{кг}$ движутся параллельно друг другу в одном направлении
    со скоростями $5\,\frac{\text{м}}{\text{с}}$ и $8\,\frac{\text{м}}{\text{с}}$ соответственно.
    Определите общий импульс шариков.
}
\answer{%
    \begin{align*}
    p_1 &= m_1v_1 = 4\,\text{кг} \cdot 5\,\frac{\text{м}}{\text{с}} = 20\,\frac{\text{кг}\cdot\text{м}}{\text{с}}, \\
    p_2 &= m_2v_2 = 3\,\text{кг} \cdot 8\,\frac{\text{м}}{\text{с}} = 24\,\frac{\text{кг}\cdot\text{м}}{\text{с}}, \\
    p &= p_1 + p_2 = m_1v_1 + m_2v_2 = 44\,\frac{\text{кг}\cdot\text{м}}{\text{с}}.
    \end{align*}
}

\tasknumber{2}%
\task{%
    Два шарика, масса каждого из которых составляет $2\,\text{кг}$, движутся навстречу друг другу.
    Скорость одного из них $5\,\frac{\text{м}}{\text{с}}$, а другого~--- $8\,\frac{\text{м}}{\text{с}}$.
    Определите общий импульс шариков.
}
\answer{%
    \begin{align*}
    p_1 &= mv_1 = 2\,\text{кг} \cdot 5\,\frac{\text{м}}{\text{с}} = 10\,\frac{\text{кг}\cdot\text{м}}{\text{с}}, \\
    p_2 &= mv_2 = 2\,\text{кг} \cdot 8\,\frac{\text{м}}{\text{с}} = 16\,\frac{\text{кг}\cdot\text{м}}{\text{с}}, \\
    p &= \abs{p_1 - p_2} = \abs{m(v_1 - v_2)}= 6\,\frac{\text{кг}\cdot\text{м}}{\text{с}}.
    \end{align*}
}

\tasknumber{3}%
\task{%
    Два одинаковых шарика массами по $2\,\text{кг}$ движутся во взаимно перпендикулярных направлениях.
    Скорости шариков составляют $7\,\frac{\text{м}}{\text{с}}$ и $24\,\frac{\text{м}}{\text{с}}$.
    Определите полный импульс системы.
}
\answer{%
    \begin{align*}
    p_1 &= mv_1 = 2\,\text{кг} \cdot 7\,\frac{\text{м}}{\text{с}} = 14\,\frac{\text{кг}\cdot\text{м}}{\text{с}}, \\
    p_2 &= mv_2 = 2\,\text{кг} \cdot 24\,\frac{\text{м}}{\text{с}} = 48\,\frac{\text{кг}\cdot\text{м}}{\text{с}}, \\
    p &= \sqrt{p_1^2 + p_2^2} = m\sqrt{v_1^2 + v_2^2} = 50\,\frac{\text{кг}\cdot\text{м}}{\text{с}}.
    \end{align*}
}

\tasknumber{4}%
\task{%
    Шарик массой $4\,\text{кг}$ свободно упал на горизонтальную площадку, имея в момент падения скорость $25\,\frac{\text{м}}{\text{c}}$.
    Считая удар абсолютно упругим, определите изменение импульса шарика.
    В ответе укажите модуль полученной величины.
}
\answer{%
    \begin{align*}
    \Delta p &= 2 \cdot mv = 2 \cdot 4\,\text{кг} \cdot 25\,\frac{\text{м}}{\text{c}} = 200\,\frac{\text{кг}\cdot\text{м}}{\text{с}}.
    \end{align*}
}

\tasknumber{5}%
\task{%
    Два тела двигаются навстречу друг другу.
    Скорость каждого из них составляет $5\,\frac{\text{м}}{\text{с}}$.
    После соударения тела слиплись и продолжили движение уже со скоростью $4\,\frac{\text{м}}{\text{с}}$.
    Определите отношение масс тел (большей к меньшей).
}
\answer{%
    \begin{align*}
    &\text{ЗСИ в проекции на ось, соединяющую центры тел:} m_1 v_1 - m_2 v_1 = (m_1 + m_2) v_2 \implies \\
    &\implies \frac{m_1}{m_2} v_1 - v_1 = \cbr{\frac{m_1}{m_2} + 1} v_2 \implies
        \frac{m_1}{m_2} (v_1 - v_2) = v_2 + v_1 \implies \frac{m_1}{m_2} = \frac{v_2 + v_1}{v_1 - v_2} = 9
    \end{align*}
}

\tasknumber{6}%
\task{%
    Шар движется с некоторой скоростью и абсолютно неупруго соударяется с телом, масса которого в 7 раз больше.
    Определите во сколько раз уменьшилась скорость шара после столкновения.
}
\answer{%
    \begin{align*}
    &\text{ЗСИ в проекции на ось, соединяющую центры тел:}  \\
    &mv + 7m \cdot 0 = (m + 7m) v' \implies \\
    &v' = v\frac{m}{7m + m} = \frac{v}{7 + 1} \implies \frac{v}{v'} = 8
    \end{align*}
}

\tasknumber{7}%
\task{%
    Определите работу силы, которая обеспечит спуск тела массой $5\,\text{кг}$ на высоту $10\,\text{м}$ с постоянным ускорением $6\,\frac{\text{м}}{\text{c}^{2}}$.
    % Примите $g = 10\,\frac{\text{м}}{\text{с}^{2}}$.
}
\answer{%
    \begin{align*}
    &\text{Для подъёма:} A = Fh = (mg + ma) h = m(g+a)h, \\
    &\text{Для спуска:} A = -Fh = -(mg - ma) h = -m(g-a)h, \\
    &\text{В результате получаем:} -200\,\text{Дж}.
    \end{align*}
}

\tasknumber{8}%
\task{%
    Тело массой 2\,\text{кг} бросили с обрыва вертикально вверх с начальной скоростью $6\,\frac{\text{м}}{\text{c}}$.
    Через некоторое время скорость тела составила $10\,\frac{\text{м}}{\text{c}}$.
    Пренебрегая сопротивлением воздуха и считая падение тела свободным, определите работу силы тяжести в течение наблюдаемого промежутка времени.
}
\answer{%
    \begin{align*}
    &\text{Изменение кинетической энергии равно работе внешних сил:} \\
    &\Delta E_k = E_k' - E_k = A_\text{тяж} \implies A_\text{тяж} = \frac{mv'^2}2 - \frac{mv_0^2}2 = 64\,\text{Дж}.
    \end{align*}
}

\tasknumber{9}%
\task{%
    Тонкий однородный шест длиной $2\,\text{м}$ и массой $30\,\text{кг}$ лежит на горизонтальной поверхности.
    \begin{itemize}
        \item Какую минимальную силу надо приложить к одному из его концов, чтобы оторвать его от этой поверхности?
        \item Какую минимальную работу надо совершить, чтобы поставить его на землю в вертикальное положение?
    \end{itemize}
    % Примите $g = 10\,\frac{\text{м}}{\text{с}^{2}}$.
}
\answer{%
    $F = \frac{mg}2 \approx 300\,\text{Н}, A = mg\frac l2 = 300\,\text{Дж}$
}

\tasknumber{10}%
\task{%
    Для того, чтобы разогать тело из состояния покоя до скорости $v$ с постоянным ускорением,
    требуется совершить работу $200\,\text{Дж}$.
    Какую работу нужно совершить, чтобы увеличить скорость этого тела от $v$ до $4v$?
}
\answer{%
    \begin{align*}
    &\text{Изменение кинетической энергии равно работе внешних сил:} \\
    &A_1 = \frac{mv^2}2 - \frac{m \cdot 0^2}2 = \frac{mv^2}2, A_2 = \frac{m\sqr{4v}}2 - \frac{mv^2}2 \implies  \\
    &\implies A_2 = \frac{mv^2}2 \cbr{4^2 - 1} = A_1 \cdot \cbr{4^2 - 1} = 3000\,\text{Дж}.
    \end{align*}
}

\variantsplitter

\addpersonalvariant{Артём Жичин}

\tasknumber{1}%
\task{%
    Шарики массами $3\,\text{кг}$ и $2\,\text{кг}$ движутся параллельно друг другу в одном направлении
    со скоростями $10\,\frac{\text{м}}{\text{с}}$ и $8\,\frac{\text{м}}{\text{с}}$ соответственно.
    Определите общий импульс шариков.
}
\answer{%
    \begin{align*}
    p_1 &= m_1v_1 = 3\,\text{кг} \cdot 10\,\frac{\text{м}}{\text{с}} = 30\,\frac{\text{кг}\cdot\text{м}}{\text{с}}, \\
    p_2 &= m_2v_2 = 2\,\text{кг} \cdot 8\,\frac{\text{м}}{\text{с}} = 16\,\frac{\text{кг}\cdot\text{м}}{\text{с}}, \\
    p &= p_1 + p_2 = m_1v_1 + m_2v_2 = 46\,\frac{\text{кг}\cdot\text{м}}{\text{с}}.
    \end{align*}
}

\tasknumber{2}%
\task{%
    Два шарика, масса каждого из которых составляет $10\,\text{кг}$, движутся навстречу друг другу.
    Скорость одного из них $1\,\frac{\text{м}}{\text{с}}$, а другого~--- $8\,\frac{\text{м}}{\text{с}}$.
    Определите общий импульс шариков.
}
\answer{%
    \begin{align*}
    p_1 &= mv_1 = 10\,\text{кг} \cdot 1\,\frac{\text{м}}{\text{с}} = 10\,\frac{\text{кг}\cdot\text{м}}{\text{с}}, \\
    p_2 &= mv_2 = 10\,\text{кг} \cdot 8\,\frac{\text{м}}{\text{с}} = 80\,\frac{\text{кг}\cdot\text{м}}{\text{с}}, \\
    p &= \abs{p_1 - p_2} = \abs{m(v_1 - v_2)}= 70\,\frac{\text{кг}\cdot\text{м}}{\text{с}}.
    \end{align*}
}

\tasknumber{3}%
\task{%
    Два одинаковых шарика массами по $5\,\text{кг}$ движутся во взаимно перпендикулярных направлениях.
    Скорости шариков составляют $7\,\frac{\text{м}}{\text{с}}$ и $24\,\frac{\text{м}}{\text{с}}$.
    Определите полный импульс системы.
}
\answer{%
    \begin{align*}
    p_1 &= mv_1 = 5\,\text{кг} \cdot 7\,\frac{\text{м}}{\text{с}} = 35\,\frac{\text{кг}\cdot\text{м}}{\text{с}}, \\
    p_2 &= mv_2 = 5\,\text{кг} \cdot 24\,\frac{\text{м}}{\text{с}} = 120\,\frac{\text{кг}\cdot\text{м}}{\text{с}}, \\
    p &= \sqrt{p_1^2 + p_2^2} = m\sqrt{v_1^2 + v_2^2} = 125\,\frac{\text{кг}\cdot\text{м}}{\text{с}}.
    \end{align*}
}

\tasknumber{4}%
\task{%
    Шарик массой $2\,\text{кг}$ свободно упал на горизонтальную площадку, имея в момент падения скорость $10\,\frac{\text{м}}{\text{c}}$.
    Считая удар абсолютно упругим, определите изменение импульса шарика.
    В ответе укажите модуль полученной величины.
}
\answer{%
    \begin{align*}
    \Delta p &= 2 \cdot mv = 2 \cdot 2\,\text{кг} \cdot 10\,\frac{\text{м}}{\text{c}} = 40\,\frac{\text{кг}\cdot\text{м}}{\text{с}}.
    \end{align*}
}

\tasknumber{5}%
\task{%
    Два тела двигаются навстречу друг другу.
    Скорость каждого из них составляет $6\,\frac{\text{м}}{\text{с}}$.
    После соударения тела слиплись и продолжили движение уже со скоростью $4\,\frac{\text{м}}{\text{с}}$.
    Определите отношение масс тел (большей к меньшей).
}
\answer{%
    \begin{align*}
    &\text{ЗСИ в проекции на ось, соединяющую центры тел:} m_1 v_1 - m_2 v_1 = (m_1 + m_2) v_2 \implies \\
    &\implies \frac{m_1}{m_2} v_1 - v_1 = \cbr{\frac{m_1}{m_2} + 1} v_2 \implies
        \frac{m_1}{m_2} (v_1 - v_2) = v_2 + v_1 \implies \frac{m_1}{m_2} = \frac{v_2 + v_1}{v_1 - v_2} = 5
    \end{align*}
}

\tasknumber{6}%
\task{%
    Шар движется с некоторой скоростью и абсолютно неупруго соударяется с телом, масса которого в 10 раз больше.
    Определите во сколько раз уменьшилась скорость шара после столкновения.
}
\answer{%
    \begin{align*}
    &\text{ЗСИ в проекции на ось, соединяющую центры тел:}  \\
    &mv + 10m \cdot 0 = (m + 10m) v' \implies \\
    &v' = v\frac{m}{10m + m} = \frac{v}{10 + 1} \implies \frac{v}{v'} = 11
    \end{align*}
}

\tasknumber{7}%
\task{%
    Определите работу силы, которая обеспечит спуск тела массой $3\,\text{кг}$ на высоту $5\,\text{м}$ с постоянным ускорением $3\,\frac{\text{м}}{\text{c}^{2}}$.
    % Примите $g = 10\,\frac{\text{м}}{\text{с}^{2}}$.
}
\answer{%
    \begin{align*}
    &\text{Для подъёма:} A = Fh = (mg + ma) h = m(g+a)h, \\
    &\text{Для спуска:} A = -Fh = -(mg - ma) h = -m(g-a)h, \\
    &\text{В результате получаем:} -105\,\text{Дж}.
    \end{align*}
}

\tasknumber{8}%
\task{%
    Тело массой 3\,\text{кг} бросили с обрыва вертикально вверх с начальной скоростью $4\,\frac{\text{м}}{\text{c}}$.
    Через некоторое время скорость тела составила $12\,\frac{\text{м}}{\text{c}}$.
    Пренебрегая сопротивлением воздуха и считая падение тела свободным, определите работу силы тяжести в течение наблюдаемого промежутка времени.
}
\answer{%
    \begin{align*}
    &\text{Изменение кинетической энергии равно работе внешних сил:} \\
    &\Delta E_k = E_k' - E_k = A_\text{тяж} \implies A_\text{тяж} = \frac{mv'^2}2 - \frac{mv_0^2}2 = 192\,\text{Дж}.
    \end{align*}
}

\tasknumber{9}%
\task{%
    Тонкий однородный лом длиной $1\,\text{м}$ и массой $30\,\text{кг}$ лежит на горизонтальной поверхности.
    \begin{itemize}
        \item Какую минимальную силу надо приложить к одному из его концов, чтобы оторвать его от этой поверхности?
        \item Какую минимальную работу надо совершить, чтобы поставить его на землю в вертикальное положение?
    \end{itemize}
    % Примите $g = 10\,\frac{\text{м}}{\text{с}^{2}}$.
}
\answer{%
    $F = \frac{mg}2 \approx 300\,\text{Н}, A = mg\frac l2 = 150\,\text{Дж}$
}

\tasknumber{10}%
\task{%
    Для того, чтобы разогать тело из состояния покоя до скорости $v$ с постоянным ускорением,
    требуется совершить работу $200\,\text{Дж}$.
    Какую работу нужно совершить, чтобы увеличить скорость этого тела от $v$ до $4v$?
}
\answer{%
    \begin{align*}
    &\text{Изменение кинетической энергии равно работе внешних сил:} \\
    &A_1 = \frac{mv^2}2 - \frac{m \cdot 0^2}2 = \frac{mv^2}2, A_2 = \frac{m\sqr{4v}}2 - \frac{mv^2}2 \implies  \\
    &\implies A_2 = \frac{mv^2}2 \cbr{4^2 - 1} = A_1 \cdot \cbr{4^2 - 1} = 3000\,\text{Дж}.
    \end{align*}
}

\variantsplitter

\addpersonalvariant{Дарья Кошман}

\tasknumber{1}%
\task{%
    Шарики массами $3\,\text{кг}$ и $1\,\text{кг}$ движутся параллельно друг другу в одном направлении
    со скоростями $5\,\frac{\text{м}}{\text{с}}$ и $8\,\frac{\text{м}}{\text{с}}$ соответственно.
    Определите общий импульс шариков.
}
\answer{%
    \begin{align*}
    p_1 &= m_1v_1 = 3\,\text{кг} \cdot 5\,\frac{\text{м}}{\text{с}} = 15\,\frac{\text{кг}\cdot\text{м}}{\text{с}}, \\
    p_2 &= m_2v_2 = 1\,\text{кг} \cdot 8\,\frac{\text{м}}{\text{с}} = 8\,\frac{\text{кг}\cdot\text{м}}{\text{с}}, \\
    p &= p_1 + p_2 = m_1v_1 + m_2v_2 = 23\,\frac{\text{кг}\cdot\text{м}}{\text{с}}.
    \end{align*}
}

\tasknumber{2}%
\task{%
    Два шарика, масса каждого из которых составляет $5\,\text{кг}$, движутся навстречу друг другу.
    Скорость одного из них $2\,\frac{\text{м}}{\text{с}}$, а другого~--- $6\,\frac{\text{м}}{\text{с}}$.
    Определите общий импульс шариков.
}
\answer{%
    \begin{align*}
    p_1 &= mv_1 = 5\,\text{кг} \cdot 2\,\frac{\text{м}}{\text{с}} = 10\,\frac{\text{кг}\cdot\text{м}}{\text{с}}, \\
    p_2 &= mv_2 = 5\,\text{кг} \cdot 6\,\frac{\text{м}}{\text{с}} = 30\,\frac{\text{кг}\cdot\text{м}}{\text{с}}, \\
    p &= \abs{p_1 - p_2} = \abs{m(v_1 - v_2)}= 20\,\frac{\text{кг}\cdot\text{м}}{\text{с}}.
    \end{align*}
}

\tasknumber{3}%
\task{%
    Два одинаковых шарика массами по $2\,\text{кг}$ движутся во взаимно перпендикулярных направлениях.
    Скорости шариков составляют $5\,\frac{\text{м}}{\text{с}}$ и $12\,\frac{\text{м}}{\text{с}}$.
    Определите полный импульс системы.
}
\answer{%
    \begin{align*}
    p_1 &= mv_1 = 2\,\text{кг} \cdot 5\,\frac{\text{м}}{\text{с}} = 10\,\frac{\text{кг}\cdot\text{м}}{\text{с}}, \\
    p_2 &= mv_2 = 2\,\text{кг} \cdot 12\,\frac{\text{м}}{\text{с}} = 24\,\frac{\text{кг}\cdot\text{м}}{\text{с}}, \\
    p &= \sqrt{p_1^2 + p_2^2} = m\sqrt{v_1^2 + v_2^2} = 26\,\frac{\text{кг}\cdot\text{м}}{\text{с}}.
    \end{align*}
}

\tasknumber{4}%
\task{%
    Шарик массой $1\,\text{кг}$ свободно упал на горизонтальную площадку, имея в момент падения скорость $25\,\frac{\text{м}}{\text{c}}$.
    Считая удар абсолютно упругим, определите изменение импульса шарика.
    В ответе укажите модуль полученной величины.
}
\answer{%
    \begin{align*}
    \Delta p &= 2 \cdot mv = 2 \cdot 1\,\text{кг} \cdot 25\,\frac{\text{м}}{\text{c}} = 50\,\frac{\text{кг}\cdot\text{м}}{\text{с}}.
    \end{align*}
}

\tasknumber{5}%
\task{%
    Два тела двигаются навстречу друг другу.
    Скорость каждого из них составляет $3\,\frac{\text{м}}{\text{с}}$.
    После соударения тела слиплись и продолжили движение уже со скоростью $1\,\frac{\text{м}}{\text{с}}$.
    Определите отношение масс тел (большей к меньшей).
}
\answer{%
    \begin{align*}
    &\text{ЗСИ в проекции на ось, соединяющую центры тел:} m_1 v_1 - m_2 v_1 = (m_1 + m_2) v_2 \implies \\
    &\implies \frac{m_1}{m_2} v_1 - v_1 = \cbr{\frac{m_1}{m_2} + 1} v_2 \implies
        \frac{m_1}{m_2} (v_1 - v_2) = v_2 + v_1 \implies \frac{m_1}{m_2} = \frac{v_2 + v_1}{v_1 - v_2} = 2
    \end{align*}
}

\tasknumber{6}%
\task{%
    Шар движется с некоторой скоростью и абсолютно неупруго соударяется с телом, масса которого в 10 раз больше.
    Определите во сколько раз уменьшилась скорость шара после столкновения.
}
\answer{%
    \begin{align*}
    &\text{ЗСИ в проекции на ось, соединяющую центры тел:}  \\
    &mv + 10m \cdot 0 = (m + 10m) v' \implies \\
    &v' = v\frac{m}{10m + m} = \frac{v}{10 + 1} \implies \frac{v}{v'} = 11
    \end{align*}
}

\tasknumber{7}%
\task{%
    Определите работу силы, которая обеспечит подъём тела массой $3\,\text{кг}$ на высоту $5\,\text{м}$ с постоянным ускорением $3\,\frac{\text{м}}{\text{c}^{2}}$.
    % Примите $g = 10\,\frac{\text{м}}{\text{с}^{2}}$.
}
\answer{%
    \begin{align*}
    &\text{Для подъёма:} A = Fh = (mg + ma) h = m(g+a)h, \\
    &\text{Для спуска:} A = -Fh = -(mg - ma) h = -m(g-a)h, \\
    &\text{В результате получаем:} 195\,\text{Дж}.
    \end{align*}
}

\tasknumber{8}%
\task{%
    Тело массой 2\,\text{кг} бросили с обрыва горизонтально с начальной скоростью $6\,\frac{\text{м}}{\text{c}}$.
    Через некоторое время скорость тела составила $12\,\frac{\text{м}}{\text{c}}$.
    Пренебрегая сопротивлением воздуха и считая падение тела свободным, определите работу силы тяжести в течение наблюдаемого промежутка времени.
}
\answer{%
    \begin{align*}
    &\text{Изменение кинетической энергии равно работе внешних сил:} \\
    &\Delta E_k = E_k' - E_k = A_\text{тяж} \implies A_\text{тяж} = \frac{mv'^2}2 - \frac{mv_0^2}2 = 108\,\text{Дж}.
    \end{align*}
}

\tasknumber{9}%
\task{%
    Тонкий однородный лом длиной $1\,\text{м}$ и массой $20\,\text{кг}$ лежит на горизонтальной поверхности.
    \begin{itemize}
        \item Какую минимальную силу надо приложить к одному из его концов, чтобы оторвать его от этой поверхности?
        \item Какую минимальную работу надо совершить, чтобы поставить его на землю в вертикальное положение?
    \end{itemize}
    % Примите $g = 10\,\frac{\text{м}}{\text{с}^{2}}$.
}
\answer{%
    $F = \frac{mg}2 \approx 200\,\text{Н}, A = mg\frac l2 = 100\,\text{Дж}$
}

\tasknumber{10}%
\task{%
    Для того, чтобы разогать тело из состояния покоя до скорости $v$ с постоянным ускорением,
    требуется совершить работу $10\,\text{Дж}$.
    Какую работу нужно совершить, чтобы увеличить скорость этого тела от $v$ до $3v$?
}
\answer{%
    \begin{align*}
    &\text{Изменение кинетической энергии равно работе внешних сил:} \\
    &A_1 = \frac{mv^2}2 - \frac{m \cdot 0^2}2 = \frac{mv^2}2, A_2 = \frac{m\sqr{3v}}2 - \frac{mv^2}2 \implies  \\
    &\implies A_2 = \frac{mv^2}2 \cbr{3^2 - 1} = A_1 \cdot \cbr{3^2 - 1} = 80\,\text{Дж}.
    \end{align*}
}

\variantsplitter

\addpersonalvariant{Анна Кузьмичёва}

\tasknumber{1}%
\task{%
    Шарики массами $1\,\text{кг}$ и $2\,\text{кг}$ движутся параллельно друг другу в одном направлении
    со скоростями $2\,\frac{\text{м}}{\text{с}}$ и $3\,\frac{\text{м}}{\text{с}}$ соответственно.
    Определите общий импульс шариков.
}
\answer{%
    \begin{align*}
    p_1 &= m_1v_1 = 1\,\text{кг} \cdot 2\,\frac{\text{м}}{\text{с}} = 2\,\frac{\text{кг}\cdot\text{м}}{\text{с}}, \\
    p_2 &= m_2v_2 = 2\,\text{кг} \cdot 3\,\frac{\text{м}}{\text{с}} = 6\,\frac{\text{кг}\cdot\text{м}}{\text{с}}, \\
    p &= p_1 + p_2 = m_1v_1 + m_2v_2 = 8\,\frac{\text{кг}\cdot\text{м}}{\text{с}}.
    \end{align*}
}

\tasknumber{2}%
\task{%
    Два шарика, масса каждого из которых составляет $2\,\text{кг}$, движутся навстречу друг другу.
    Скорость одного из них $1\,\frac{\text{м}}{\text{с}}$, а другого~--- $8\,\frac{\text{м}}{\text{с}}$.
    Определите общий импульс шариков.
}
\answer{%
    \begin{align*}
    p_1 &= mv_1 = 2\,\text{кг} \cdot 1\,\frac{\text{м}}{\text{с}} = 2\,\frac{\text{кг}\cdot\text{м}}{\text{с}}, \\
    p_2 &= mv_2 = 2\,\text{кг} \cdot 8\,\frac{\text{м}}{\text{с}} = 16\,\frac{\text{кг}\cdot\text{м}}{\text{с}}, \\
    p &= \abs{p_1 - p_2} = \abs{m(v_1 - v_2)}= 14\,\frac{\text{кг}\cdot\text{м}}{\text{с}}.
    \end{align*}
}

\tasknumber{3}%
\task{%
    Два одинаковых шарика массами по $5\,\text{кг}$ движутся во взаимно перпендикулярных направлениях.
    Скорости шариков составляют $3\,\frac{\text{м}}{\text{с}}$ и $4\,\frac{\text{м}}{\text{с}}$.
    Определите полный импульс системы.
}
\answer{%
    \begin{align*}
    p_1 &= mv_1 = 5\,\text{кг} \cdot 3\,\frac{\text{м}}{\text{с}} = 15\,\frac{\text{кг}\cdot\text{м}}{\text{с}}, \\
    p_2 &= mv_2 = 5\,\text{кг} \cdot 4\,\frac{\text{м}}{\text{с}} = 20\,\frac{\text{кг}\cdot\text{м}}{\text{с}}, \\
    p &= \sqrt{p_1^2 + p_2^2} = m\sqrt{v_1^2 + v_2^2} = 25\,\frac{\text{кг}\cdot\text{м}}{\text{с}}.
    \end{align*}
}

\tasknumber{4}%
\task{%
    Шарик массой $1\,\text{кг}$ свободно упал на горизонтальную площадку, имея в момент падения скорость $15\,\frac{\text{м}}{\text{c}}$.
    Считая удар абсолютно упругим, определите изменение импульса шарика.
    В ответе укажите модуль полученной величины.
}
\answer{%
    \begin{align*}
    \Delta p &= 2 \cdot mv = 2 \cdot 1\,\text{кг} \cdot 15\,\frac{\text{м}}{\text{c}} = 30\,\frac{\text{кг}\cdot\text{м}}{\text{с}}.
    \end{align*}
}

\tasknumber{5}%
\task{%
    Два тела двигаются навстречу друг другу.
    Скорость каждого из них составляет $4\,\frac{\text{м}}{\text{с}}$.
    После соударения тела слиплись и продолжили движение уже со скоростью $3\,\frac{\text{м}}{\text{с}}$.
    Определите отношение масс тел (большей к меньшей).
}
\answer{%
    \begin{align*}
    &\text{ЗСИ в проекции на ось, соединяющую центры тел:} m_1 v_1 - m_2 v_1 = (m_1 + m_2) v_2 \implies \\
    &\implies \frac{m_1}{m_2} v_1 - v_1 = \cbr{\frac{m_1}{m_2} + 1} v_2 \implies
        \frac{m_1}{m_2} (v_1 - v_2) = v_2 + v_1 \implies \frac{m_1}{m_2} = \frac{v_2 + v_1}{v_1 - v_2} = 7
    \end{align*}
}

\tasknumber{6}%
\task{%
    Шар движется с некоторой скоростью и абсолютно неупруго соударяется с телом, масса которого в 12 раз больше.
    Определите во сколько раз уменьшилась скорость шара после столкновения.
}
\answer{%
    \begin{align*}
    &\text{ЗСИ в проекции на ось, соединяющую центры тел:}  \\
    &mv + 12m \cdot 0 = (m + 12m) v' \implies \\
    &v' = v\frac{m}{12m + m} = \frac{v}{12 + 1} \implies \frac{v}{v'} = 13
    \end{align*}
}

\tasknumber{7}%
\task{%
    Определите работу силы, которая обеспечит спуск тела массой $5\,\text{кг}$ на высоту $10\,\text{м}$ с постоянным ускорением $3\,\frac{\text{м}}{\text{c}^{2}}$.
    % Примите $g = 10\,\frac{\text{м}}{\text{с}^{2}}$.
}
\answer{%
    \begin{align*}
    &\text{Для подъёма:} A = Fh = (mg + ma) h = m(g+a)h, \\
    &\text{Для спуска:} A = -Fh = -(mg - ma) h = -m(g-a)h, \\
    &\text{В результате получаем:} -350\,\text{Дж}.
    \end{align*}
}

\tasknumber{8}%
\task{%
    Тело массой 2\,\text{кг} бросили с обрыва горизонтально с начальной скоростью $4\,\frac{\text{м}}{\text{c}}$.
    Через некоторое время скорость тела составила $8\,\frac{\text{м}}{\text{c}}$.
    Пренебрегая сопротивлением воздуха и считая падение тела свободным, определите работу силы тяжести в течение наблюдаемого промежутка времени.
}
\answer{%
    \begin{align*}
    &\text{Изменение кинетической энергии равно работе внешних сил:} \\
    &\Delta E_k = E_k' - E_k = A_\text{тяж} \implies A_\text{тяж} = \frac{mv'^2}2 - \frac{mv_0^2}2 = 48\,\text{Дж}.
    \end{align*}
}

\tasknumber{9}%
\task{%
    Тонкий однородный кусок арматуры длиной $1\,\text{м}$ и массой $30\,\text{кг}$ лежит на горизонтальной поверхности.
    \begin{itemize}
        \item Какую минимальную силу надо приложить к одному из его концов, чтобы оторвать его от этой поверхности?
        \item Какую минимальную работу надо совершить, чтобы поставить его на землю в вертикальное положение?
    \end{itemize}
    % Примите $g = 10\,\frac{\text{м}}{\text{с}^{2}}$.
}
\answer{%
    $F = \frac{mg}2 \approx 300\,\text{Н}, A = mg\frac l2 = 150\,\text{Дж}$
}

\tasknumber{10}%
\task{%
    Для того, чтобы разогать тело из состояния покоя до скорости $v$ с постоянным ускорением,
    требуется совершить работу $200\,\text{Дж}$.
    Какую работу нужно совершить, чтобы увеличить скорость этого тела от $v$ до $3v$?
}
\answer{%
    \begin{align*}
    &\text{Изменение кинетической энергии равно работе внешних сил:} \\
    &A_1 = \frac{mv^2}2 - \frac{m \cdot 0^2}2 = \frac{mv^2}2, A_2 = \frac{m\sqr{3v}}2 - \frac{mv^2}2 \implies  \\
    &\implies A_2 = \frac{mv^2}2 \cbr{3^2 - 1} = A_1 \cdot \cbr{3^2 - 1} = 1600\,\text{Дж}.
    \end{align*}
}

\variantsplitter

\addpersonalvariant{Алёна Куприянова}

\tasknumber{1}%
\task{%
    Шарики массами $3\,\text{кг}$ и $2\,\text{кг}$ движутся параллельно друг другу в одном направлении
    со скоростями $2\,\frac{\text{м}}{\text{с}}$ и $8\,\frac{\text{м}}{\text{с}}$ соответственно.
    Определите общий импульс шариков.
}
\answer{%
    \begin{align*}
    p_1 &= m_1v_1 = 3\,\text{кг} \cdot 2\,\frac{\text{м}}{\text{с}} = 6\,\frac{\text{кг}\cdot\text{м}}{\text{с}}, \\
    p_2 &= m_2v_2 = 2\,\text{кг} \cdot 8\,\frac{\text{м}}{\text{с}} = 16\,\frac{\text{кг}\cdot\text{м}}{\text{с}}, \\
    p &= p_1 + p_2 = m_1v_1 + m_2v_2 = 22\,\frac{\text{кг}\cdot\text{м}}{\text{с}}.
    \end{align*}
}

\tasknumber{2}%
\task{%
    Два шарика, масса каждого из которых составляет $5\,\text{кг}$, движутся навстречу друг другу.
    Скорость одного из них $10\,\frac{\text{м}}{\text{с}}$, а другого~--- $8\,\frac{\text{м}}{\text{с}}$.
    Определите общий импульс шариков.
}
\answer{%
    \begin{align*}
    p_1 &= mv_1 = 5\,\text{кг} \cdot 10\,\frac{\text{м}}{\text{с}} = 50\,\frac{\text{кг}\cdot\text{м}}{\text{с}}, \\
    p_2 &= mv_2 = 5\,\text{кг} \cdot 8\,\frac{\text{м}}{\text{с}} = 40\,\frac{\text{кг}\cdot\text{м}}{\text{с}}, \\
    p &= \abs{p_1 - p_2} = \abs{m(v_1 - v_2)}= 10\,\frac{\text{кг}\cdot\text{м}}{\text{с}}.
    \end{align*}
}

\tasknumber{3}%
\task{%
    Два одинаковых шарика массами по $10\,\text{кг}$ движутся во взаимно перпендикулярных направлениях.
    Скорости шариков составляют $5\,\frac{\text{м}}{\text{с}}$ и $12\,\frac{\text{м}}{\text{с}}$.
    Определите полный импульс системы.
}
\answer{%
    \begin{align*}
    p_1 &= mv_1 = 10\,\text{кг} \cdot 5\,\frac{\text{м}}{\text{с}} = 50\,\frac{\text{кг}\cdot\text{м}}{\text{с}}, \\
    p_2 &= mv_2 = 10\,\text{кг} \cdot 12\,\frac{\text{м}}{\text{с}} = 120\,\frac{\text{кг}\cdot\text{м}}{\text{с}}, \\
    p &= \sqrt{p_1^2 + p_2^2} = m\sqrt{v_1^2 + v_2^2} = 130\,\frac{\text{кг}\cdot\text{м}}{\text{с}}.
    \end{align*}
}

\tasknumber{4}%
\task{%
    Шарик массой $1\,\text{кг}$ свободно упал на горизонтальную площадку, имея в момент падения скорость $25\,\frac{\text{м}}{\text{c}}$.
    Считая удар абсолютно неупругим, определите изменение импульса шарика.
    В ответе укажите модуль полученной величины.
}
\answer{%
    \begin{align*}
    \Delta p &= 1 \cdot mv = 1 \cdot 1\,\text{кг} \cdot 25\,\frac{\text{м}}{\text{c}} = 25\,\frac{\text{кг}\cdot\text{м}}{\text{с}}.
    \end{align*}
}

\tasknumber{5}%
\task{%
    Два тела двигаются навстречу друг другу.
    Скорость каждого из них составляет $6\,\frac{\text{м}}{\text{с}}$.
    После соударения тела слиплись и продолжили движение уже со скоростью $4\,\frac{\text{м}}{\text{с}}$.
    Определите отношение масс тел (большей к меньшей).
}
\answer{%
    \begin{align*}
    &\text{ЗСИ в проекции на ось, соединяющую центры тел:} m_1 v_1 - m_2 v_1 = (m_1 + m_2) v_2 \implies \\
    &\implies \frac{m_1}{m_2} v_1 - v_1 = \cbr{\frac{m_1}{m_2} + 1} v_2 \implies
        \frac{m_1}{m_2} (v_1 - v_2) = v_2 + v_1 \implies \frac{m_1}{m_2} = \frac{v_2 + v_1}{v_1 - v_2} = 5
    \end{align*}
}

\tasknumber{6}%
\task{%
    Шар движется с некоторой скоростью и абсолютно неупруго соударяется с телом, масса которого в 9 раз больше.
    Определите во сколько раз уменьшилась скорость шара после столкновения.
}
\answer{%
    \begin{align*}
    &\text{ЗСИ в проекции на ось, соединяющую центры тел:}  \\
    &mv + 9m \cdot 0 = (m + 9m) v' \implies \\
    &v' = v\frac{m}{9m + m} = \frac{v}{9 + 1} \implies \frac{v}{v'} = 10
    \end{align*}
}

\tasknumber{7}%
\task{%
    Определите работу силы, которая обеспечит подъём тела массой $5\,\text{кг}$ на высоту $2\,\text{м}$ с постоянным ускорением $4\,\frac{\text{м}}{\text{c}^{2}}$.
    % Примите $g = 10\,\frac{\text{м}}{\text{с}^{2}}$.
}
\answer{%
    \begin{align*}
    &\text{Для подъёма:} A = Fh = (mg + ma) h = m(g+a)h, \\
    &\text{Для спуска:} A = -Fh = -(mg - ma) h = -m(g-a)h, \\
    &\text{В результате получаем:} 140\,\text{Дж}.
    \end{align*}
}

\tasknumber{8}%
\task{%
    Тело массой 1\,\text{кг} бросили с обрыва вертикально вверх с начальной скоростью $6\,\frac{\text{м}}{\text{c}}$.
    Через некоторое время скорость тела составила $12\,\frac{\text{м}}{\text{c}}$.
    Пренебрегая сопротивлением воздуха и считая падение тела свободным, определите работу силы тяжести в течение наблюдаемого промежутка времени.
}
\answer{%
    \begin{align*}
    &\text{Изменение кинетической энергии равно работе внешних сил:} \\
    &\Delta E_k = E_k' - E_k = A_\text{тяж} \implies A_\text{тяж} = \frac{mv'^2}2 - \frac{mv_0^2}2 = 54\,\text{Дж}.
    \end{align*}
}

\tasknumber{9}%
\task{%
    Тонкий однородный кусок арматуры длиной $3\,\text{м}$ и массой $30\,\text{кг}$ лежит на горизонтальной поверхности.
    \begin{itemize}
        \item Какую минимальную силу надо приложить к одному из его концов, чтобы оторвать его от этой поверхности?
        \item Какую минимальную работу надо совершить, чтобы поставить его на землю в вертикальное положение?
    \end{itemize}
    % Примите $g = 10\,\frac{\text{м}}{\text{с}^{2}}$.
}
\answer{%
    $F = \frac{mg}2 \approx 300\,\text{Н}, A = mg\frac l2 = 450\,\text{Дж}$
}

\tasknumber{10}%
\task{%
    Для того, чтобы разогать тело из состояния покоя до скорости $v$ с постоянным ускорением,
    требуется совершить работу $200\,\text{Дж}$.
    Какую работу нужно совершить, чтобы увеличить скорость этого тела от $v$ до $4v$?
}
\answer{%
    \begin{align*}
    &\text{Изменение кинетической энергии равно работе внешних сил:} \\
    &A_1 = \frac{mv^2}2 - \frac{m \cdot 0^2}2 = \frac{mv^2}2, A_2 = \frac{m\sqr{4v}}2 - \frac{mv^2}2 \implies  \\
    &\implies A_2 = \frac{mv^2}2 \cbr{4^2 - 1} = A_1 \cdot \cbr{4^2 - 1} = 3000\,\text{Дж}.
    \end{align*}
}

\variantsplitter

\addpersonalvariant{Ярослав Лавровский}

\tasknumber{1}%
\task{%
    Шарики массами $3\,\text{кг}$ и $4\,\text{кг}$ движутся параллельно друг другу в одном направлении
    со скоростями $10\,\frac{\text{м}}{\text{с}}$ и $6\,\frac{\text{м}}{\text{с}}$ соответственно.
    Определите общий импульс шариков.
}
\answer{%
    \begin{align*}
    p_1 &= m_1v_1 = 3\,\text{кг} \cdot 10\,\frac{\text{м}}{\text{с}} = 30\,\frac{\text{кг}\cdot\text{м}}{\text{с}}, \\
    p_2 &= m_2v_2 = 4\,\text{кг} \cdot 6\,\frac{\text{м}}{\text{с}} = 24\,\frac{\text{кг}\cdot\text{м}}{\text{с}}, \\
    p &= p_1 + p_2 = m_1v_1 + m_2v_2 = 54\,\frac{\text{кг}\cdot\text{м}}{\text{с}}.
    \end{align*}
}

\tasknumber{2}%
\task{%
    Два шарика, масса каждого из которых составляет $2\,\text{кг}$, движутся навстречу друг другу.
    Скорость одного из них $1\,\frac{\text{м}}{\text{с}}$, а другого~--- $6\,\frac{\text{м}}{\text{с}}$.
    Определите общий импульс шариков.
}
\answer{%
    \begin{align*}
    p_1 &= mv_1 = 2\,\text{кг} \cdot 1\,\frac{\text{м}}{\text{с}} = 2\,\frac{\text{кг}\cdot\text{м}}{\text{с}}, \\
    p_2 &= mv_2 = 2\,\text{кг} \cdot 6\,\frac{\text{м}}{\text{с}} = 12\,\frac{\text{кг}\cdot\text{м}}{\text{с}}, \\
    p &= \abs{p_1 - p_2} = \abs{m(v_1 - v_2)}= 10\,\frac{\text{кг}\cdot\text{м}}{\text{с}}.
    \end{align*}
}

\tasknumber{3}%
\task{%
    Два одинаковых шарика массами по $2\,\text{кг}$ движутся во взаимно перпендикулярных направлениях.
    Скорости шариков составляют $5\,\frac{\text{м}}{\text{с}}$ и $12\,\frac{\text{м}}{\text{с}}$.
    Определите полный импульс системы.
}
\answer{%
    \begin{align*}
    p_1 &= mv_1 = 2\,\text{кг} \cdot 5\,\frac{\text{м}}{\text{с}} = 10\,\frac{\text{кг}\cdot\text{м}}{\text{с}}, \\
    p_2 &= mv_2 = 2\,\text{кг} \cdot 12\,\frac{\text{м}}{\text{с}} = 24\,\frac{\text{кг}\cdot\text{м}}{\text{с}}, \\
    p &= \sqrt{p_1^2 + p_2^2} = m\sqrt{v_1^2 + v_2^2} = 26\,\frac{\text{кг}\cdot\text{м}}{\text{с}}.
    \end{align*}
}

\tasknumber{4}%
\task{%
    Шарик массой $1\,\text{кг}$ свободно упал на горизонтальную площадку, имея в момент падения скорость $10\,\frac{\text{м}}{\text{c}}$.
    Считая удар абсолютно упругим, определите изменение импульса шарика.
    В ответе укажите модуль полученной величины.
}
\answer{%
    \begin{align*}
    \Delta p &= 2 \cdot mv = 2 \cdot 1\,\text{кг} \cdot 10\,\frac{\text{м}}{\text{c}} = 20\,\frac{\text{кг}\cdot\text{м}}{\text{с}}.
    \end{align*}
}

\tasknumber{5}%
\task{%
    Два тела двигаются навстречу друг другу.
    Скорость каждого из них составляет $2\,\frac{\text{м}}{\text{с}}$.
    После соударения тела слиплись и продолжили движение уже со скоростью $1\,\frac{\text{м}}{\text{с}}$.
    Определите отношение масс тел (большей к меньшей).
}
\answer{%
    \begin{align*}
    &\text{ЗСИ в проекции на ось, соединяющую центры тел:} m_1 v_1 - m_2 v_1 = (m_1 + m_2) v_2 \implies \\
    &\implies \frac{m_1}{m_2} v_1 - v_1 = \cbr{\frac{m_1}{m_2} + 1} v_2 \implies
        \frac{m_1}{m_2} (v_1 - v_2) = v_2 + v_1 \implies \frac{m_1}{m_2} = \frac{v_2 + v_1}{v_1 - v_2} = 3
    \end{align*}
}

\tasknumber{6}%
\task{%
    Шар движется с некоторой скоростью и абсолютно неупруго соударяется с телом, масса которого в 12 раз больше.
    Определите во сколько раз уменьшилась скорость шара после столкновения.
}
\answer{%
    \begin{align*}
    &\text{ЗСИ в проекции на ось, соединяющую центры тел:}  \\
    &mv + 12m \cdot 0 = (m + 12m) v' \implies \\
    &v' = v\frac{m}{12m + m} = \frac{v}{12 + 1} \implies \frac{v}{v'} = 13
    \end{align*}
}

\tasknumber{7}%
\task{%
    Определите работу силы, которая обеспечит спуск тела массой $3\,\text{кг}$ на высоту $2\,\text{м}$ с постоянным ускорением $3\,\frac{\text{м}}{\text{c}^{2}}$.
    % Примите $g = 10\,\frac{\text{м}}{\text{с}^{2}}$.
}
\answer{%
    \begin{align*}
    &\text{Для подъёма:} A = Fh = (mg + ma) h = m(g+a)h, \\
    &\text{Для спуска:} A = -Fh = -(mg - ma) h = -m(g-a)h, \\
    &\text{В результате получаем:} -42\,\text{Дж}.
    \end{align*}
}

\tasknumber{8}%
\task{%
    Тело массой 2\,\text{кг} бросили с обрыва под углом $45\degrees$ к горизонту с начальной скоростью $4\,\frac{\text{м}}{\text{c}}$.
    Через некоторое время скорость тела составила $8\,\frac{\text{м}}{\text{c}}$.
    Пренебрегая сопротивлением воздуха и считая падение тела свободным, определите работу силы тяжести в течение наблюдаемого промежутка времени.
}
\answer{%
    \begin{align*}
    &\text{Изменение кинетической энергии равно работе внешних сил:} \\
    &\Delta E_k = E_k' - E_k = A_\text{тяж} \implies A_\text{тяж} = \frac{mv'^2}2 - \frac{mv_0^2}2 = 48\,\text{Дж}.
    \end{align*}
}

\tasknumber{9}%
\task{%
    Тонкий однородный кусок арматуры длиной $3\,\text{м}$ и массой $20\,\text{кг}$ лежит на горизонтальной поверхности.
    \begin{itemize}
        \item Какую минимальную силу надо приложить к одному из его концов, чтобы оторвать его от этой поверхности?
        \item Какую минимальную работу надо совершить, чтобы поставить его на землю в вертикальное положение?
    \end{itemize}
    % Примите $g = 10\,\frac{\text{м}}{\text{с}^{2}}$.
}
\answer{%
    $F = \frac{mg}2 \approx 200\,\text{Н}, A = mg\frac l2 = 300\,\text{Дж}$
}

\tasknumber{10}%
\task{%
    Для того, чтобы разогать тело из состояния покоя до скорости $v$ с постоянным ускорением,
    требуется совершить работу $100\,\text{Дж}$.
    Какую работу нужно совершить, чтобы увеличить скорость этого тела от $v$ до $4v$?
}
\answer{%
    \begin{align*}
    &\text{Изменение кинетической энергии равно работе внешних сил:} \\
    &A_1 = \frac{mv^2}2 - \frac{m \cdot 0^2}2 = \frac{mv^2}2, A_2 = \frac{m\sqr{4v}}2 - \frac{mv^2}2 \implies  \\
    &\implies A_2 = \frac{mv^2}2 \cbr{4^2 - 1} = A_1 \cdot \cbr{4^2 - 1} = 1500\,\text{Дж}.
    \end{align*}
}

\variantsplitter

\addpersonalvariant{Анастасия Ламанова}

\tasknumber{1}%
\task{%
    Шарики массами $3\,\text{кг}$ и $1\,\text{кг}$ движутся параллельно друг другу в одном направлении
    со скоростями $5\,\frac{\text{м}}{\text{с}}$ и $3\,\frac{\text{м}}{\text{с}}$ соответственно.
    Определите общий импульс шариков.
}
\answer{%
    \begin{align*}
    p_1 &= m_1v_1 = 3\,\text{кг} \cdot 5\,\frac{\text{м}}{\text{с}} = 15\,\frac{\text{кг}\cdot\text{м}}{\text{с}}, \\
    p_2 &= m_2v_2 = 1\,\text{кг} \cdot 3\,\frac{\text{м}}{\text{с}} = 3\,\frac{\text{кг}\cdot\text{м}}{\text{с}}, \\
    p &= p_1 + p_2 = m_1v_1 + m_2v_2 = 18\,\frac{\text{кг}\cdot\text{м}}{\text{с}}.
    \end{align*}
}

\tasknumber{2}%
\task{%
    Два шарика, масса каждого из которых составляет $10\,\text{кг}$, движутся навстречу друг другу.
    Скорость одного из них $10\,\frac{\text{м}}{\text{с}}$, а другого~--- $3\,\frac{\text{м}}{\text{с}}$.
    Определите общий импульс шариков.
}
\answer{%
    \begin{align*}
    p_1 &= mv_1 = 10\,\text{кг} \cdot 10\,\frac{\text{м}}{\text{с}} = 100\,\frac{\text{кг}\cdot\text{м}}{\text{с}}, \\
    p_2 &= mv_2 = 10\,\text{кг} \cdot 3\,\frac{\text{м}}{\text{с}} = 30\,\frac{\text{кг}\cdot\text{м}}{\text{с}}, \\
    p &= \abs{p_1 - p_2} = \abs{m(v_1 - v_2)}= 70\,\frac{\text{кг}\cdot\text{м}}{\text{с}}.
    \end{align*}
}

\tasknumber{3}%
\task{%
    Два одинаковых шарика массами по $10\,\text{кг}$ движутся во взаимно перпендикулярных направлениях.
    Скорости шариков составляют $5\,\frac{\text{м}}{\text{с}}$ и $12\,\frac{\text{м}}{\text{с}}$.
    Определите полный импульс системы.
}
\answer{%
    \begin{align*}
    p_1 &= mv_1 = 10\,\text{кг} \cdot 5\,\frac{\text{м}}{\text{с}} = 50\,\frac{\text{кг}\cdot\text{м}}{\text{с}}, \\
    p_2 &= mv_2 = 10\,\text{кг} \cdot 12\,\frac{\text{м}}{\text{с}} = 120\,\frac{\text{кг}\cdot\text{м}}{\text{с}}, \\
    p &= \sqrt{p_1^2 + p_2^2} = m\sqrt{v_1^2 + v_2^2} = 130\,\frac{\text{кг}\cdot\text{м}}{\text{с}}.
    \end{align*}
}

\tasknumber{4}%
\task{%
    Шарик массой $1\,\text{кг}$ свободно упал на горизонтальную площадку, имея в момент падения скорость $25\,\frac{\text{м}}{\text{c}}$.
    Считая удар абсолютно упругим, определите изменение импульса шарика.
    В ответе укажите модуль полученной величины.
}
\answer{%
    \begin{align*}
    \Delta p &= 2 \cdot mv = 2 \cdot 1\,\text{кг} \cdot 25\,\frac{\text{м}}{\text{c}} = 50\,\frac{\text{кг}\cdot\text{м}}{\text{с}}.
    \end{align*}
}

\tasknumber{5}%
\task{%
    Два тела двигаются навстречу друг другу.
    Скорость каждого из них составляет $5\,\frac{\text{м}}{\text{с}}$.
    После соударения тела слиплись и продолжили движение уже со скоростью $4\,\frac{\text{м}}{\text{с}}$.
    Определите отношение масс тел (большей к меньшей).
}
\answer{%
    \begin{align*}
    &\text{ЗСИ в проекции на ось, соединяющую центры тел:} m_1 v_1 - m_2 v_1 = (m_1 + m_2) v_2 \implies \\
    &\implies \frac{m_1}{m_2} v_1 - v_1 = \cbr{\frac{m_1}{m_2} + 1} v_2 \implies
        \frac{m_1}{m_2} (v_1 - v_2) = v_2 + v_1 \implies \frac{m_1}{m_2} = \frac{v_2 + v_1}{v_1 - v_2} = 9
    \end{align*}
}

\tasknumber{6}%
\task{%
    Шар движется с некоторой скоростью и абсолютно неупруго соударяется с телом, масса которого в 12 раз больше.
    Определите во сколько раз уменьшилась скорость шара после столкновения.
}
\answer{%
    \begin{align*}
    &\text{ЗСИ в проекции на ось, соединяющую центры тел:}  \\
    &mv + 12m \cdot 0 = (m + 12m) v' \implies \\
    &v' = v\frac{m}{12m + m} = \frac{v}{12 + 1} \implies \frac{v}{v'} = 13
    \end{align*}
}

\tasknumber{7}%
\task{%
    Определите работу силы, которая обеспечит спуск тела массой $3\,\text{кг}$ на высоту $2\,\text{м}$ с постоянным ускорением $3\,\frac{\text{м}}{\text{c}^{2}}$.
    % Примите $g = 10\,\frac{\text{м}}{\text{с}^{2}}$.
}
\answer{%
    \begin{align*}
    &\text{Для подъёма:} A = Fh = (mg + ma) h = m(g+a)h, \\
    &\text{Для спуска:} A = -Fh = -(mg - ma) h = -m(g-a)h, \\
    &\text{В результате получаем:} -42\,\text{Дж}.
    \end{align*}
}

\tasknumber{8}%
\task{%
    Тело массой 1\,\text{кг} бросили с обрыва вертикально вверх с начальной скоростью $4\,\frac{\text{м}}{\text{c}}$.
    Через некоторое время скорость тела составила $10\,\frac{\text{м}}{\text{c}}$.
    Пренебрегая сопротивлением воздуха и считая падение тела свободным, определите работу силы тяжести в течение наблюдаемого промежутка времени.
}
\answer{%
    \begin{align*}
    &\text{Изменение кинетической энергии равно работе внешних сил:} \\
    &\Delta E_k = E_k' - E_k = A_\text{тяж} \implies A_\text{тяж} = \frac{mv'^2}2 - \frac{mv_0^2}2 = 42\,\text{Дж}.
    \end{align*}
}

\tasknumber{9}%
\task{%
    Тонкий однородный лом длиной $2\,\text{м}$ и массой $20\,\text{кг}$ лежит на горизонтальной поверхности.
    \begin{itemize}
        \item Какую минимальную силу надо приложить к одному из его концов, чтобы оторвать его от этой поверхности?
        \item Какую минимальную работу надо совершить, чтобы поставить его на землю в вертикальное положение?
    \end{itemize}
    % Примите $g = 10\,\frac{\text{м}}{\text{с}^{2}}$.
}
\answer{%
    $F = \frac{mg}2 \approx 200\,\text{Н}, A = mg\frac l2 = 200\,\text{Дж}$
}

\tasknumber{10}%
\task{%
    Для того, чтобы разогать тело из состояния покоя до скорости $v$ с постоянным ускорением,
    требуется совершить работу $20\,\text{Дж}$.
    Какую работу нужно совершить, чтобы увеличить скорость этого тела от $v$ до $4v$?
}
\answer{%
    \begin{align*}
    &\text{Изменение кинетической энергии равно работе внешних сил:} \\
    &A_1 = \frac{mv^2}2 - \frac{m \cdot 0^2}2 = \frac{mv^2}2, A_2 = \frac{m\sqr{4v}}2 - \frac{mv^2}2 \implies  \\
    &\implies A_2 = \frac{mv^2}2 \cbr{4^2 - 1} = A_1 \cdot \cbr{4^2 - 1} = 300\,\text{Дж}.
    \end{align*}
}

\variantsplitter

\addpersonalvariant{Виктория Легонькова}

\tasknumber{1}%
\task{%
    Шарики массами $3\,\text{кг}$ и $2\,\text{кг}$ движутся параллельно друг другу в одном направлении
    со скоростями $10\,\frac{\text{м}}{\text{с}}$ и $8\,\frac{\text{м}}{\text{с}}$ соответственно.
    Определите общий импульс шариков.
}
\answer{%
    \begin{align*}
    p_1 &= m_1v_1 = 3\,\text{кг} \cdot 10\,\frac{\text{м}}{\text{с}} = 30\,\frac{\text{кг}\cdot\text{м}}{\text{с}}, \\
    p_2 &= m_2v_2 = 2\,\text{кг} \cdot 8\,\frac{\text{м}}{\text{с}} = 16\,\frac{\text{кг}\cdot\text{м}}{\text{с}}, \\
    p &= p_1 + p_2 = m_1v_1 + m_2v_2 = 46\,\frac{\text{кг}\cdot\text{м}}{\text{с}}.
    \end{align*}
}

\tasknumber{2}%
\task{%
    Два шарика, масса каждого из которых составляет $10\,\text{кг}$, движутся навстречу друг другу.
    Скорость одного из них $10\,\frac{\text{м}}{\text{с}}$, а другого~--- $8\,\frac{\text{м}}{\text{с}}$.
    Определите общий импульс шариков.
}
\answer{%
    \begin{align*}
    p_1 &= mv_1 = 10\,\text{кг} \cdot 10\,\frac{\text{м}}{\text{с}} = 100\,\frac{\text{кг}\cdot\text{м}}{\text{с}}, \\
    p_2 &= mv_2 = 10\,\text{кг} \cdot 8\,\frac{\text{м}}{\text{с}} = 80\,\frac{\text{кг}\cdot\text{м}}{\text{с}}, \\
    p &= \abs{p_1 - p_2} = \abs{m(v_1 - v_2)}= 20\,\frac{\text{кг}\cdot\text{м}}{\text{с}}.
    \end{align*}
}

\tasknumber{3}%
\task{%
    Два одинаковых шарика массами по $5\,\text{кг}$ движутся во взаимно перпендикулярных направлениях.
    Скорости шариков составляют $7\,\frac{\text{м}}{\text{с}}$ и $24\,\frac{\text{м}}{\text{с}}$.
    Определите полный импульс системы.
}
\answer{%
    \begin{align*}
    p_1 &= mv_1 = 5\,\text{кг} \cdot 7\,\frac{\text{м}}{\text{с}} = 35\,\frac{\text{кг}\cdot\text{м}}{\text{с}}, \\
    p_2 &= mv_2 = 5\,\text{кг} \cdot 24\,\frac{\text{м}}{\text{с}} = 120\,\frac{\text{кг}\cdot\text{м}}{\text{с}}, \\
    p &= \sqrt{p_1^2 + p_2^2} = m\sqrt{v_1^2 + v_2^2} = 125\,\frac{\text{кг}\cdot\text{м}}{\text{с}}.
    \end{align*}
}

\tasknumber{4}%
\task{%
    Шарик массой $4\,\text{кг}$ свободно упал на горизонтальную площадку, имея в момент падения скорость $10\,\frac{\text{м}}{\text{c}}$.
    Считая удар абсолютно упругим, определите изменение импульса шарика.
    В ответе укажите модуль полученной величины.
}
\answer{%
    \begin{align*}
    \Delta p &= 2 \cdot mv = 2 \cdot 4\,\text{кг} \cdot 10\,\frac{\text{м}}{\text{c}} = 80\,\frac{\text{кг}\cdot\text{м}}{\text{с}}.
    \end{align*}
}

\tasknumber{5}%
\task{%
    Два тела двигаются навстречу друг другу.
    Скорость каждого из них составляет $5\,\frac{\text{м}}{\text{с}}$.
    После соударения тела слиплись и продолжили движение уже со скоростью $4\,\frac{\text{м}}{\text{с}}$.
    Определите отношение масс тел (большей к меньшей).
}
\answer{%
    \begin{align*}
    &\text{ЗСИ в проекции на ось, соединяющую центры тел:} m_1 v_1 - m_2 v_1 = (m_1 + m_2) v_2 \implies \\
    &\implies \frac{m_1}{m_2} v_1 - v_1 = \cbr{\frac{m_1}{m_2} + 1} v_2 \implies
        \frac{m_1}{m_2} (v_1 - v_2) = v_2 + v_1 \implies \frac{m_1}{m_2} = \frac{v_2 + v_1}{v_1 - v_2} = 9
    \end{align*}
}

\tasknumber{6}%
\task{%
    Шар движется с некоторой скоростью и абсолютно неупруго соударяется с телом, масса которого в 5 раз больше.
    Определите во сколько раз уменьшилась скорость шара после столкновения.
}
\answer{%
    \begin{align*}
    &\text{ЗСИ в проекции на ось, соединяющую центры тел:}  \\
    &mv + 5m \cdot 0 = (m + 5m) v' \implies \\
    &v' = v\frac{m}{5m + m} = \frac{v}{5 + 1} \implies \frac{v}{v'} = 6
    \end{align*}
}

\tasknumber{7}%
\task{%
    Определите работу силы, которая обеспечит подъём тела массой $2\,\text{кг}$ на высоту $5\,\text{м}$ с постоянным ускорением $3\,\frac{\text{м}}{\text{c}^{2}}$.
    % Примите $g = 10\,\frac{\text{м}}{\text{с}^{2}}$.
}
\answer{%
    \begin{align*}
    &\text{Для подъёма:} A = Fh = (mg + ma) h = m(g+a)h, \\
    &\text{Для спуска:} A = -Fh = -(mg - ma) h = -m(g-a)h, \\
    &\text{В результате получаем:} 130\,\text{Дж}.
    \end{align*}
}

\tasknumber{8}%
\task{%
    Тело массой 3\,\text{кг} бросили с обрыва вертикально вверх с начальной скоростью $4\,\frac{\text{м}}{\text{c}}$.
    Через некоторое время скорость тела составила $12\,\frac{\text{м}}{\text{c}}$.
    Пренебрегая сопротивлением воздуха и считая падение тела свободным, определите работу силы тяжести в течение наблюдаемого промежутка времени.
}
\answer{%
    \begin{align*}
    &\text{Изменение кинетической энергии равно работе внешних сил:} \\
    &\Delta E_k = E_k' - E_k = A_\text{тяж} \implies A_\text{тяж} = \frac{mv'^2}2 - \frac{mv_0^2}2 = 192\,\text{Дж}.
    \end{align*}
}

\tasknumber{9}%
\task{%
    Тонкий однородный лом длиной $3\,\text{м}$ и массой $10\,\text{кг}$ лежит на горизонтальной поверхности.
    \begin{itemize}
        \item Какую минимальную силу надо приложить к одному из его концов, чтобы оторвать его от этой поверхности?
        \item Какую минимальную работу надо совершить, чтобы поставить его на землю в вертикальное положение?
    \end{itemize}
    % Примите $g = 10\,\frac{\text{м}}{\text{с}^{2}}$.
}
\answer{%
    $F = \frac{mg}2 \approx 100\,\text{Н}, A = mg\frac l2 = 150\,\text{Дж}$
}

\tasknumber{10}%
\task{%
    Для того, чтобы разогать тело из состояния покоя до скорости $v$ с постоянным ускорением,
    требуется совершить работу $10\,\text{Дж}$.
    Какую работу нужно совершить, чтобы увеличить скорость этого тела от $v$ до $3v$?
}
\answer{%
    \begin{align*}
    &\text{Изменение кинетической энергии равно работе внешних сил:} \\
    &A_1 = \frac{mv^2}2 - \frac{m \cdot 0^2}2 = \frac{mv^2}2, A_2 = \frac{m\sqr{3v}}2 - \frac{mv^2}2 \implies  \\
    &\implies A_2 = \frac{mv^2}2 \cbr{3^2 - 1} = A_1 \cdot \cbr{3^2 - 1} = 80\,\text{Дж}.
    \end{align*}
}

\variantsplitter

\addpersonalvariant{Семён Мартынов}

\tasknumber{1}%
\task{%
    Шарики массами $2\,\text{кг}$ и $3\,\text{кг}$ движутся параллельно друг другу в одном направлении
    со скоростями $5\,\frac{\text{м}}{\text{с}}$ и $6\,\frac{\text{м}}{\text{с}}$ соответственно.
    Определите общий импульс шариков.
}
\answer{%
    \begin{align*}
    p_1 &= m_1v_1 = 2\,\text{кг} \cdot 5\,\frac{\text{м}}{\text{с}} = 10\,\frac{\text{кг}\cdot\text{м}}{\text{с}}, \\
    p_2 &= m_2v_2 = 3\,\text{кг} \cdot 6\,\frac{\text{м}}{\text{с}} = 18\,\frac{\text{кг}\cdot\text{м}}{\text{с}}, \\
    p &= p_1 + p_2 = m_1v_1 + m_2v_2 = 28\,\frac{\text{кг}\cdot\text{м}}{\text{с}}.
    \end{align*}
}

\tasknumber{2}%
\task{%
    Два шарика, масса каждого из которых составляет $2\,\text{кг}$, движутся навстречу друг другу.
    Скорость одного из них $2\,\frac{\text{м}}{\text{с}}$, а другого~--- $6\,\frac{\text{м}}{\text{с}}$.
    Определите общий импульс шариков.
}
\answer{%
    \begin{align*}
    p_1 &= mv_1 = 2\,\text{кг} \cdot 2\,\frac{\text{м}}{\text{с}} = 4\,\frac{\text{кг}\cdot\text{м}}{\text{с}}, \\
    p_2 &= mv_2 = 2\,\text{кг} \cdot 6\,\frac{\text{м}}{\text{с}} = 12\,\frac{\text{кг}\cdot\text{м}}{\text{с}}, \\
    p &= \abs{p_1 - p_2} = \abs{m(v_1 - v_2)}= 8\,\frac{\text{кг}\cdot\text{м}}{\text{с}}.
    \end{align*}
}

\tasknumber{3}%
\task{%
    Два одинаковых шарика массами по $2\,\text{кг}$ движутся во взаимно перпендикулярных направлениях.
    Скорости шариков составляют $5\,\frac{\text{м}}{\text{с}}$ и $12\,\frac{\text{м}}{\text{с}}$.
    Определите полный импульс системы.
}
\answer{%
    \begin{align*}
    p_1 &= mv_1 = 2\,\text{кг} \cdot 5\,\frac{\text{м}}{\text{с}} = 10\,\frac{\text{кг}\cdot\text{м}}{\text{с}}, \\
    p_2 &= mv_2 = 2\,\text{кг} \cdot 12\,\frac{\text{м}}{\text{с}} = 24\,\frac{\text{кг}\cdot\text{м}}{\text{с}}, \\
    p &= \sqrt{p_1^2 + p_2^2} = m\sqrt{v_1^2 + v_2^2} = 26\,\frac{\text{кг}\cdot\text{м}}{\text{с}}.
    \end{align*}
}

\tasknumber{4}%
\task{%
    Шарик массой $2\,\text{кг}$ свободно упал на горизонтальную площадку, имея в момент падения скорость $20\,\frac{\text{м}}{\text{c}}$.
    Считая удар абсолютно неупругим, определите изменение импульса шарика.
    В ответе укажите модуль полученной величины.
}
\answer{%
    \begin{align*}
    \Delta p &= 1 \cdot mv = 1 \cdot 2\,\text{кг} \cdot 20\,\frac{\text{м}}{\text{c}} = 40\,\frac{\text{кг}\cdot\text{м}}{\text{с}}.
    \end{align*}
}

\tasknumber{5}%
\task{%
    Два тела двигаются навстречу друг другу.
    Скорость каждого из них составляет $2\,\frac{\text{м}}{\text{с}}$.
    После соударения тела слиплись и продолжили движение уже со скоростью $1\,\frac{\text{м}}{\text{с}}$.
    Определите отношение масс тел (большей к меньшей).
}
\answer{%
    \begin{align*}
    &\text{ЗСИ в проекции на ось, соединяющую центры тел:} m_1 v_1 - m_2 v_1 = (m_1 + m_2) v_2 \implies \\
    &\implies \frac{m_1}{m_2} v_1 - v_1 = \cbr{\frac{m_1}{m_2} + 1} v_2 \implies
        \frac{m_1}{m_2} (v_1 - v_2) = v_2 + v_1 \implies \frac{m_1}{m_2} = \frac{v_2 + v_1}{v_1 - v_2} = 3
    \end{align*}
}

\tasknumber{6}%
\task{%
    Шар движется с некоторой скоростью и абсолютно неупруго соударяется с телом, масса которого в 6 раз больше.
    Определите во сколько раз уменьшилась скорость шара после столкновения.
}
\answer{%
    \begin{align*}
    &\text{ЗСИ в проекции на ось, соединяющую центры тел:}  \\
    &mv + 6m \cdot 0 = (m + 6m) v' \implies \\
    &v' = v\frac{m}{6m + m} = \frac{v}{6 + 1} \implies \frac{v}{v'} = 7
    \end{align*}
}

\tasknumber{7}%
\task{%
    Определите работу силы, которая обеспечит спуск тела массой $3\,\text{кг}$ на высоту $2\,\text{м}$ с постоянным ускорением $2\,\frac{\text{м}}{\text{c}^{2}}$.
    % Примите $g = 10\,\frac{\text{м}}{\text{с}^{2}}$.
}
\answer{%
    \begin{align*}
    &\text{Для подъёма:} A = Fh = (mg + ma) h = m(g+a)h, \\
    &\text{Для спуска:} A = -Fh = -(mg - ma) h = -m(g-a)h, \\
    &\text{В результате получаем:} -48\,\text{Дж}.
    \end{align*}
}

\tasknumber{8}%
\task{%
    Тело массой 3\,\text{кг} бросили с обрыва горизонтально с начальной скоростью $4\,\frac{\text{м}}{\text{c}}$.
    Через некоторое время скорость тела составила $8\,\frac{\text{м}}{\text{c}}$.
    Пренебрегая сопротивлением воздуха и считая падение тела свободным, определите работу силы тяжести в течение наблюдаемого промежутка времени.
}
\answer{%
    \begin{align*}
    &\text{Изменение кинетической энергии равно работе внешних сил:} \\
    &\Delta E_k = E_k' - E_k = A_\text{тяж} \implies A_\text{тяж} = \frac{mv'^2}2 - \frac{mv_0^2}2 = 72\,\text{Дж}.
    \end{align*}
}

\tasknumber{9}%
\task{%
    Тонкий однородный кусок арматуры длиной $1\,\text{м}$ и массой $10\,\text{кг}$ лежит на горизонтальной поверхности.
    \begin{itemize}
        \item Какую минимальную силу надо приложить к одному из его концов, чтобы оторвать его от этой поверхности?
        \item Какую минимальную работу надо совершить, чтобы поставить его на землю в вертикальное положение?
    \end{itemize}
    % Примите $g = 10\,\frac{\text{м}}{\text{с}^{2}}$.
}
\answer{%
    $F = \frac{mg}2 \approx 100\,\text{Н}, A = mg\frac l2 = 50\,\text{Дж}$
}

\tasknumber{10}%
\task{%
    Для того, чтобы разогать тело из состояния покоя до скорости $v$ с постоянным ускорением,
    требуется совершить работу $40\,\text{Дж}$.
    Какую работу нужно совершить, чтобы увеличить скорость этого тела от $v$ до $4v$?
}
\answer{%
    \begin{align*}
    &\text{Изменение кинетической энергии равно работе внешних сил:} \\
    &A_1 = \frac{mv^2}2 - \frac{m \cdot 0^2}2 = \frac{mv^2}2, A_2 = \frac{m\sqr{4v}}2 - \frac{mv^2}2 \implies  \\
    &\implies A_2 = \frac{mv^2}2 \cbr{4^2 - 1} = A_1 \cdot \cbr{4^2 - 1} = 600\,\text{Дж}.
    \end{align*}
}

\variantsplitter

\addpersonalvariant{Варвара Минаева}

\tasknumber{1}%
\task{%
    Шарики массами $2\,\text{кг}$ и $3\,\text{кг}$ движутся параллельно друг другу в одном направлении
    со скоростями $4\,\frac{\text{м}}{\text{с}}$ и $6\,\frac{\text{м}}{\text{с}}$ соответственно.
    Определите общий импульс шариков.
}
\answer{%
    \begin{align*}
    p_1 &= m_1v_1 = 2\,\text{кг} \cdot 4\,\frac{\text{м}}{\text{с}} = 8\,\frac{\text{кг}\cdot\text{м}}{\text{с}}, \\
    p_2 &= m_2v_2 = 3\,\text{кг} \cdot 6\,\frac{\text{м}}{\text{с}} = 18\,\frac{\text{кг}\cdot\text{м}}{\text{с}}, \\
    p &= p_1 + p_2 = m_1v_1 + m_2v_2 = 26\,\frac{\text{кг}\cdot\text{м}}{\text{с}}.
    \end{align*}
}

\tasknumber{2}%
\task{%
    Два шарика, масса каждого из которых составляет $2\,\text{кг}$, движутся навстречу друг другу.
    Скорость одного из них $1\,\frac{\text{м}}{\text{с}}$, а другого~--- $6\,\frac{\text{м}}{\text{с}}$.
    Определите общий импульс шариков.
}
\answer{%
    \begin{align*}
    p_1 &= mv_1 = 2\,\text{кг} \cdot 1\,\frac{\text{м}}{\text{с}} = 2\,\frac{\text{кг}\cdot\text{м}}{\text{с}}, \\
    p_2 &= mv_2 = 2\,\text{кг} \cdot 6\,\frac{\text{м}}{\text{с}} = 12\,\frac{\text{кг}\cdot\text{м}}{\text{с}}, \\
    p &= \abs{p_1 - p_2} = \abs{m(v_1 - v_2)}= 10\,\frac{\text{кг}\cdot\text{м}}{\text{с}}.
    \end{align*}
}

\tasknumber{3}%
\task{%
    Два одинаковых шарика массами по $5\,\text{кг}$ движутся во взаимно перпендикулярных направлениях.
    Скорости шариков составляют $7\,\frac{\text{м}}{\text{с}}$ и $24\,\frac{\text{м}}{\text{с}}$.
    Определите полный импульс системы.
}
\answer{%
    \begin{align*}
    p_1 &= mv_1 = 5\,\text{кг} \cdot 7\,\frac{\text{м}}{\text{с}} = 35\,\frac{\text{кг}\cdot\text{м}}{\text{с}}, \\
    p_2 &= mv_2 = 5\,\text{кг} \cdot 24\,\frac{\text{м}}{\text{с}} = 120\,\frac{\text{кг}\cdot\text{м}}{\text{с}}, \\
    p &= \sqrt{p_1^2 + p_2^2} = m\sqrt{v_1^2 + v_2^2} = 125\,\frac{\text{кг}\cdot\text{м}}{\text{с}}.
    \end{align*}
}

\tasknumber{4}%
\task{%
    Шарик массой $2\,\text{кг}$ свободно упал на горизонтальную площадку, имея в момент падения скорость $15\,\frac{\text{м}}{\text{c}}$.
    Считая удар абсолютно неупругим, определите изменение импульса шарика.
    В ответе укажите модуль полученной величины.
}
\answer{%
    \begin{align*}
    \Delta p &= 1 \cdot mv = 1 \cdot 2\,\text{кг} \cdot 15\,\frac{\text{м}}{\text{c}} = 30\,\frac{\text{кг}\cdot\text{м}}{\text{с}}.
    \end{align*}
}

\tasknumber{5}%
\task{%
    Два тела двигаются навстречу друг другу.
    Скорость каждого из них составляет $9\,\frac{\text{м}}{\text{с}}$.
    После соударения тела слиплись и продолжили движение уже со скоростью $7\,\frac{\text{м}}{\text{с}}$.
    Определите отношение масс тел (большей к меньшей).
}
\answer{%
    \begin{align*}
    &\text{ЗСИ в проекции на ось, соединяющую центры тел:} m_1 v_1 - m_2 v_1 = (m_1 + m_2) v_2 \implies \\
    &\implies \frac{m_1}{m_2} v_1 - v_1 = \cbr{\frac{m_1}{m_2} + 1} v_2 \implies
        \frac{m_1}{m_2} (v_1 - v_2) = v_2 + v_1 \implies \frac{m_1}{m_2} = \frac{v_2 + v_1}{v_1 - v_2} = 8
    \end{align*}
}

\tasknumber{6}%
\task{%
    Шар движется с некоторой скоростью и абсолютно неупруго соударяется с телом, масса которого в 8 раз больше.
    Определите во сколько раз уменьшилась скорость шара после столкновения.
}
\answer{%
    \begin{align*}
    &\text{ЗСИ в проекции на ось, соединяющую центры тел:}  \\
    &mv + 8m \cdot 0 = (m + 8m) v' \implies \\
    &v' = v\frac{m}{8m + m} = \frac{v}{8 + 1} \implies \frac{v}{v'} = 9
    \end{align*}
}

\tasknumber{7}%
\task{%
    Определите работу силы, которая обеспечит спуск тела массой $3\,\text{кг}$ на высоту $10\,\text{м}$ с постоянным ускорением $2\,\frac{\text{м}}{\text{c}^{2}}$.
    % Примите $g = 10\,\frac{\text{м}}{\text{с}^{2}}$.
}
\answer{%
    \begin{align*}
    &\text{Для подъёма:} A = Fh = (mg + ma) h = m(g+a)h, \\
    &\text{Для спуска:} A = -Fh = -(mg - ma) h = -m(g-a)h, \\
    &\text{В результате получаем:} -240\,\text{Дж}.
    \end{align*}
}

\tasknumber{8}%
\task{%
    Тело массой 1\,\text{кг} бросили с обрыва горизонтально с начальной скоростью $6\,\frac{\text{м}}{\text{c}}$.
    Через некоторое время скорость тела составила $8\,\frac{\text{м}}{\text{c}}$.
    Пренебрегая сопротивлением воздуха и считая падение тела свободным, определите работу силы тяжести в течение наблюдаемого промежутка времени.
}
\answer{%
    \begin{align*}
    &\text{Изменение кинетической энергии равно работе внешних сил:} \\
    &\Delta E_k = E_k' - E_k = A_\text{тяж} \implies A_\text{тяж} = \frac{mv'^2}2 - \frac{mv_0^2}2 = 14\,\text{Дж}.
    \end{align*}
}

\tasknumber{9}%
\task{%
    Тонкий однородный кусок арматуры длиной $2\,\text{м}$ и массой $30\,\text{кг}$ лежит на горизонтальной поверхности.
    \begin{itemize}
        \item Какую минимальную силу надо приложить к одному из его концов, чтобы оторвать его от этой поверхности?
        \item Какую минимальную работу надо совершить, чтобы поставить его на землю в вертикальное положение?
    \end{itemize}
    % Примите $g = 10\,\frac{\text{м}}{\text{с}^{2}}$.
}
\answer{%
    $F = \frac{mg}2 \approx 300\,\text{Н}, A = mg\frac l2 = 300\,\text{Дж}$
}

\tasknumber{10}%
\task{%
    Для того, чтобы разогать тело из состояния покоя до скорости $v$ с постоянным ускорением,
    требуется совершить работу $10\,\text{Дж}$.
    Какую работу нужно совершить, чтобы увеличить скорость этого тела от $v$ до $3v$?
}
\answer{%
    \begin{align*}
    &\text{Изменение кинетической энергии равно работе внешних сил:} \\
    &A_1 = \frac{mv^2}2 - \frac{m \cdot 0^2}2 = \frac{mv^2}2, A_2 = \frac{m\sqr{3v}}2 - \frac{mv^2}2 \implies  \\
    &\implies A_2 = \frac{mv^2}2 \cbr{3^2 - 1} = A_1 \cdot \cbr{3^2 - 1} = 80\,\text{Дж}.
    \end{align*}
}

\variantsplitter

\addpersonalvariant{Леонид Никитин}

\tasknumber{1}%
\task{%
    Шарики массами $2\,\text{кг}$ и $3\,\text{кг}$ движутся параллельно друг другу в одном направлении
    со скоростями $10\,\frac{\text{м}}{\text{с}}$ и $6\,\frac{\text{м}}{\text{с}}$ соответственно.
    Определите общий импульс шариков.
}
\answer{%
    \begin{align*}
    p_1 &= m_1v_1 = 2\,\text{кг} \cdot 10\,\frac{\text{м}}{\text{с}} = 20\,\frac{\text{кг}\cdot\text{м}}{\text{с}}, \\
    p_2 &= m_2v_2 = 3\,\text{кг} \cdot 6\,\frac{\text{м}}{\text{с}} = 18\,\frac{\text{кг}\cdot\text{м}}{\text{с}}, \\
    p &= p_1 + p_2 = m_1v_1 + m_2v_2 = 38\,\frac{\text{кг}\cdot\text{м}}{\text{с}}.
    \end{align*}
}

\tasknumber{2}%
\task{%
    Два шарика, масса каждого из которых составляет $10\,\text{кг}$, движутся навстречу друг другу.
    Скорость одного из них $2\,\frac{\text{м}}{\text{с}}$, а другого~--- $3\,\frac{\text{м}}{\text{с}}$.
    Определите общий импульс шариков.
}
\answer{%
    \begin{align*}
    p_1 &= mv_1 = 10\,\text{кг} \cdot 2\,\frac{\text{м}}{\text{с}} = 20\,\frac{\text{кг}\cdot\text{м}}{\text{с}}, \\
    p_2 &= mv_2 = 10\,\text{кг} \cdot 3\,\frac{\text{м}}{\text{с}} = 30\,\frac{\text{кг}\cdot\text{м}}{\text{с}}, \\
    p &= \abs{p_1 - p_2} = \abs{m(v_1 - v_2)}= 10\,\frac{\text{кг}\cdot\text{м}}{\text{с}}.
    \end{align*}
}

\tasknumber{3}%
\task{%
    Два одинаковых шарика массами по $5\,\text{кг}$ движутся во взаимно перпендикулярных направлениях.
    Скорости шариков составляют $3\,\frac{\text{м}}{\text{с}}$ и $4\,\frac{\text{м}}{\text{с}}$.
    Определите полный импульс системы.
}
\answer{%
    \begin{align*}
    p_1 &= mv_1 = 5\,\text{кг} \cdot 3\,\frac{\text{м}}{\text{с}} = 15\,\frac{\text{кг}\cdot\text{м}}{\text{с}}, \\
    p_2 &= mv_2 = 5\,\text{кг} \cdot 4\,\frac{\text{м}}{\text{с}} = 20\,\frac{\text{кг}\cdot\text{м}}{\text{с}}, \\
    p &= \sqrt{p_1^2 + p_2^2} = m\sqrt{v_1^2 + v_2^2} = 25\,\frac{\text{кг}\cdot\text{м}}{\text{с}}.
    \end{align*}
}

\tasknumber{4}%
\task{%
    Шарик массой $2\,\text{кг}$ свободно упал на горизонтальную площадку, имея в момент падения скорость $10\,\frac{\text{м}}{\text{c}}$.
    Считая удар абсолютно неупругим, определите изменение импульса шарика.
    В ответе укажите модуль полученной величины.
}
\answer{%
    \begin{align*}
    \Delta p &= 1 \cdot mv = 1 \cdot 2\,\text{кг} \cdot 10\,\frac{\text{м}}{\text{c}} = 20\,\frac{\text{кг}\cdot\text{м}}{\text{с}}.
    \end{align*}
}

\tasknumber{5}%
\task{%
    Два тела двигаются навстречу друг другу.
    Скорость каждого из них составляет $2\,\frac{\text{м}}{\text{с}}$.
    После соударения тела слиплись и продолжили движение уже со скоростью $1\,\frac{\text{м}}{\text{с}}$.
    Определите отношение масс тел (большей к меньшей).
}
\answer{%
    \begin{align*}
    &\text{ЗСИ в проекции на ось, соединяющую центры тел:} m_1 v_1 - m_2 v_1 = (m_1 + m_2) v_2 \implies \\
    &\implies \frac{m_1}{m_2} v_1 - v_1 = \cbr{\frac{m_1}{m_2} + 1} v_2 \implies
        \frac{m_1}{m_2} (v_1 - v_2) = v_2 + v_1 \implies \frac{m_1}{m_2} = \frac{v_2 + v_1}{v_1 - v_2} = 3
    \end{align*}
}

\tasknumber{6}%
\task{%
    Шар движется с некоторой скоростью и абсолютно неупруго соударяется с телом, масса которого в 13 раз больше.
    Определите во сколько раз уменьшилась скорость шара после столкновения.
}
\answer{%
    \begin{align*}
    &\text{ЗСИ в проекции на ось, соединяющую центры тел:}  \\
    &mv + 13m \cdot 0 = (m + 13m) v' \implies \\
    &v' = v\frac{m}{13m + m} = \frac{v}{13 + 1} \implies \frac{v}{v'} = 14
    \end{align*}
}

\tasknumber{7}%
\task{%
    Определите работу силы, которая обеспечит подъём тела массой $5\,\text{кг}$ на высоту $2\,\text{м}$ с постоянным ускорением $6\,\frac{\text{м}}{\text{c}^{2}}$.
    % Примите $g = 10\,\frac{\text{м}}{\text{с}^{2}}$.
}
\answer{%
    \begin{align*}
    &\text{Для подъёма:} A = Fh = (mg + ma) h = m(g+a)h, \\
    &\text{Для спуска:} A = -Fh = -(mg - ma) h = -m(g-a)h, \\
    &\text{В результате получаем:} 160\,\text{Дж}.
    \end{align*}
}

\tasknumber{8}%
\task{%
    Тело массой 1\,\text{кг} бросили с обрыва горизонтально с начальной скоростью $6\,\frac{\text{м}}{\text{c}}$.
    Через некоторое время скорость тела составила $10\,\frac{\text{м}}{\text{c}}$.
    Пренебрегая сопротивлением воздуха и считая падение тела свободным, определите работу силы тяжести в течение наблюдаемого промежутка времени.
}
\answer{%
    \begin{align*}
    &\text{Изменение кинетической энергии равно работе внешних сил:} \\
    &\Delta E_k = E_k' - E_k = A_\text{тяж} \implies A_\text{тяж} = \frac{mv'^2}2 - \frac{mv_0^2}2 = 32\,\text{Дж}.
    \end{align*}
}

\tasknumber{9}%
\task{%
    Тонкий однородный лом длиной $3\,\text{м}$ и массой $30\,\text{кг}$ лежит на горизонтальной поверхности.
    \begin{itemize}
        \item Какую минимальную силу надо приложить к одному из его концов, чтобы оторвать его от этой поверхности?
        \item Какую минимальную работу надо совершить, чтобы поставить его на землю в вертикальное положение?
    \end{itemize}
    % Примите $g = 10\,\frac{\text{м}}{\text{с}^{2}}$.
}
\answer{%
    $F = \frac{mg}2 \approx 300\,\text{Н}, A = mg\frac l2 = 450\,\text{Дж}$
}

\tasknumber{10}%
\task{%
    Для того, чтобы разогать тело из состояния покоя до скорости $v$ с постоянным ускорением,
    требуется совершить работу $10\,\text{Дж}$.
    Какую работу нужно совершить, чтобы увеличить скорость этого тела от $v$ до $3v$?
}
\answer{%
    \begin{align*}
    &\text{Изменение кинетической энергии равно работе внешних сил:} \\
    &A_1 = \frac{mv^2}2 - \frac{m \cdot 0^2}2 = \frac{mv^2}2, A_2 = \frac{m\sqr{3v}}2 - \frac{mv^2}2 \implies  \\
    &\implies A_2 = \frac{mv^2}2 \cbr{3^2 - 1} = A_1 \cdot \cbr{3^2 - 1} = 80\,\text{Дж}.
    \end{align*}
}

\variantsplitter

\addpersonalvariant{Тимофей Полетаев}

\tasknumber{1}%
\task{%
    Шарики массами $4\,\text{кг}$ и $2\,\text{кг}$ движутся параллельно друг другу в одном направлении
    со скоростями $5\,\frac{\text{м}}{\text{с}}$ и $6\,\frac{\text{м}}{\text{с}}$ соответственно.
    Определите общий импульс шариков.
}
\answer{%
    \begin{align*}
    p_1 &= m_1v_1 = 4\,\text{кг} \cdot 5\,\frac{\text{м}}{\text{с}} = 20\,\frac{\text{кг}\cdot\text{м}}{\text{с}}, \\
    p_2 &= m_2v_2 = 2\,\text{кг} \cdot 6\,\frac{\text{м}}{\text{с}} = 12\,\frac{\text{кг}\cdot\text{м}}{\text{с}}, \\
    p &= p_1 + p_2 = m_1v_1 + m_2v_2 = 32\,\frac{\text{кг}\cdot\text{м}}{\text{с}}.
    \end{align*}
}

\tasknumber{2}%
\task{%
    Два шарика, масса каждого из которых составляет $10\,\text{кг}$, движутся навстречу друг другу.
    Скорость одного из них $1\,\frac{\text{м}}{\text{с}}$, а другого~--- $3\,\frac{\text{м}}{\text{с}}$.
    Определите общий импульс шариков.
}
\answer{%
    \begin{align*}
    p_1 &= mv_1 = 10\,\text{кг} \cdot 1\,\frac{\text{м}}{\text{с}} = 10\,\frac{\text{кг}\cdot\text{м}}{\text{с}}, \\
    p_2 &= mv_2 = 10\,\text{кг} \cdot 3\,\frac{\text{м}}{\text{с}} = 30\,\frac{\text{кг}\cdot\text{м}}{\text{с}}, \\
    p &= \abs{p_1 - p_2} = \abs{m(v_1 - v_2)}= 20\,\frac{\text{кг}\cdot\text{м}}{\text{с}}.
    \end{align*}
}

\tasknumber{3}%
\task{%
    Два одинаковых шарика массами по $2\,\text{кг}$ движутся во взаимно перпендикулярных направлениях.
    Скорости шариков составляют $7\,\frac{\text{м}}{\text{с}}$ и $24\,\frac{\text{м}}{\text{с}}$.
    Определите полный импульс системы.
}
\answer{%
    \begin{align*}
    p_1 &= mv_1 = 2\,\text{кг} \cdot 7\,\frac{\text{м}}{\text{с}} = 14\,\frac{\text{кг}\cdot\text{м}}{\text{с}}, \\
    p_2 &= mv_2 = 2\,\text{кг} \cdot 24\,\frac{\text{м}}{\text{с}} = 48\,\frac{\text{кг}\cdot\text{м}}{\text{с}}, \\
    p &= \sqrt{p_1^2 + p_2^2} = m\sqrt{v_1^2 + v_2^2} = 50\,\frac{\text{кг}\cdot\text{м}}{\text{с}}.
    \end{align*}
}

\tasknumber{4}%
\task{%
    Шарик массой $1\,\text{кг}$ свободно упал на горизонтальную площадку, имея в момент падения скорость $15\,\frac{\text{м}}{\text{c}}$.
    Считая удар абсолютно упругим, определите изменение импульса шарика.
    В ответе укажите модуль полученной величины.
}
\answer{%
    \begin{align*}
    \Delta p &= 2 \cdot mv = 2 \cdot 1\,\text{кг} \cdot 15\,\frac{\text{м}}{\text{c}} = 30\,\frac{\text{кг}\cdot\text{м}}{\text{с}}.
    \end{align*}
}

\tasknumber{5}%
\task{%
    Два тела двигаются навстречу друг другу.
    Скорость каждого из них составляет $3\,\frac{\text{м}}{\text{с}}$.
    После соударения тела слиплись и продолжили движение уже со скоростью $1\,\frac{\text{м}}{\text{с}}$.
    Определите отношение масс тел (большей к меньшей).
}
\answer{%
    \begin{align*}
    &\text{ЗСИ в проекции на ось, соединяющую центры тел:} m_1 v_1 - m_2 v_1 = (m_1 + m_2) v_2 \implies \\
    &\implies \frac{m_1}{m_2} v_1 - v_1 = \cbr{\frac{m_1}{m_2} + 1} v_2 \implies
        \frac{m_1}{m_2} (v_1 - v_2) = v_2 + v_1 \implies \frac{m_1}{m_2} = \frac{v_2 + v_1}{v_1 - v_2} = 2
    \end{align*}
}

\tasknumber{6}%
\task{%
    Шар движется с некоторой скоростью и абсолютно неупруго соударяется с телом, масса которого в 14 раз больше.
    Определите во сколько раз уменьшилась скорость шара после столкновения.
}
\answer{%
    \begin{align*}
    &\text{ЗСИ в проекции на ось, соединяющую центры тел:}  \\
    &mv + 14m \cdot 0 = (m + 14m) v' \implies \\
    &v' = v\frac{m}{14m + m} = \frac{v}{14 + 1} \implies \frac{v}{v'} = 15
    \end{align*}
}

\tasknumber{7}%
\task{%
    Определите работу силы, которая обеспечит спуск тела массой $2\,\text{кг}$ на высоту $2\,\text{м}$ с постоянным ускорением $4\,\frac{\text{м}}{\text{c}^{2}}$.
    % Примите $g = 10\,\frac{\text{м}}{\text{с}^{2}}$.
}
\answer{%
    \begin{align*}
    &\text{Для подъёма:} A = Fh = (mg + ma) h = m(g+a)h, \\
    &\text{Для спуска:} A = -Fh = -(mg - ma) h = -m(g-a)h, \\
    &\text{В результате получаем:} -24\,\text{Дж}.
    \end{align*}
}

\tasknumber{8}%
\task{%
    Тело массой 1\,\text{кг} бросили с обрыва вертикально вверх с начальной скоростью $2\,\frac{\text{м}}{\text{c}}$.
    Через некоторое время скорость тела составила $10\,\frac{\text{м}}{\text{c}}$.
    Пренебрегая сопротивлением воздуха и считая падение тела свободным, определите работу силы тяжести в течение наблюдаемого промежутка времени.
}
\answer{%
    \begin{align*}
    &\text{Изменение кинетической энергии равно работе внешних сил:} \\
    &\Delta E_k = E_k' - E_k = A_\text{тяж} \implies A_\text{тяж} = \frac{mv'^2}2 - \frac{mv_0^2}2 = 48\,\text{Дж}.
    \end{align*}
}

\tasknumber{9}%
\task{%
    Тонкий однородный лом длиной $1\,\text{м}$ и массой $20\,\text{кг}$ лежит на горизонтальной поверхности.
    \begin{itemize}
        \item Какую минимальную силу надо приложить к одному из его концов, чтобы оторвать его от этой поверхности?
        \item Какую минимальную работу надо совершить, чтобы поставить его на землю в вертикальное положение?
    \end{itemize}
    % Примите $g = 10\,\frac{\text{м}}{\text{с}^{2}}$.
}
\answer{%
    $F = \frac{mg}2 \approx 200\,\text{Н}, A = mg\frac l2 = 100\,\text{Дж}$
}

\tasknumber{10}%
\task{%
    Для того, чтобы разогать тело из состояния покоя до скорости $v$ с постоянным ускорением,
    требуется совершить работу $100\,\text{Дж}$.
    Какую работу нужно совершить, чтобы увеличить скорость этого тела от $v$ до $2v$?
}
\answer{%
    \begin{align*}
    &\text{Изменение кинетической энергии равно работе внешних сил:} \\
    &A_1 = \frac{mv^2}2 - \frac{m \cdot 0^2}2 = \frac{mv^2}2, A_2 = \frac{m\sqr{2v}}2 - \frac{mv^2}2 \implies  \\
    &\implies A_2 = \frac{mv^2}2 \cbr{2^2 - 1} = A_1 \cdot \cbr{2^2 - 1} = 300\,\text{Дж}.
    \end{align*}
}

\variantsplitter

\addpersonalvariant{Андрей Рожков}

\tasknumber{1}%
\task{%
    Шарики массами $2\,\text{кг}$ и $4\,\text{кг}$ движутся параллельно друг другу в одном направлении
    со скоростями $4\,\frac{\text{м}}{\text{с}}$ и $8\,\frac{\text{м}}{\text{с}}$ соответственно.
    Определите общий импульс шариков.
}
\answer{%
    \begin{align*}
    p_1 &= m_1v_1 = 2\,\text{кг} \cdot 4\,\frac{\text{м}}{\text{с}} = 8\,\frac{\text{кг}\cdot\text{м}}{\text{с}}, \\
    p_2 &= m_2v_2 = 4\,\text{кг} \cdot 8\,\frac{\text{м}}{\text{с}} = 32\,\frac{\text{кг}\cdot\text{м}}{\text{с}}, \\
    p &= p_1 + p_2 = m_1v_1 + m_2v_2 = 40\,\frac{\text{кг}\cdot\text{м}}{\text{с}}.
    \end{align*}
}

\tasknumber{2}%
\task{%
    Два шарика, масса каждого из которых составляет $2\,\text{кг}$, движутся навстречу друг другу.
    Скорость одного из них $1\,\frac{\text{м}}{\text{с}}$, а другого~--- $6\,\frac{\text{м}}{\text{с}}$.
    Определите общий импульс шариков.
}
\answer{%
    \begin{align*}
    p_1 &= mv_1 = 2\,\text{кг} \cdot 1\,\frac{\text{м}}{\text{с}} = 2\,\frac{\text{кг}\cdot\text{м}}{\text{с}}, \\
    p_2 &= mv_2 = 2\,\text{кг} \cdot 6\,\frac{\text{м}}{\text{с}} = 12\,\frac{\text{кг}\cdot\text{м}}{\text{с}}, \\
    p &= \abs{p_1 - p_2} = \abs{m(v_1 - v_2)}= 10\,\frac{\text{кг}\cdot\text{м}}{\text{с}}.
    \end{align*}
}

\tasknumber{3}%
\task{%
    Два одинаковых шарика массами по $10\,\text{кг}$ движутся во взаимно перпендикулярных направлениях.
    Скорости шариков составляют $7\,\frac{\text{м}}{\text{с}}$ и $24\,\frac{\text{м}}{\text{с}}$.
    Определите полный импульс системы.
}
\answer{%
    \begin{align*}
    p_1 &= mv_1 = 10\,\text{кг} \cdot 7\,\frac{\text{м}}{\text{с}} = 70\,\frac{\text{кг}\cdot\text{м}}{\text{с}}, \\
    p_2 &= mv_2 = 10\,\text{кг} \cdot 24\,\frac{\text{м}}{\text{с}} = 240\,\frac{\text{кг}\cdot\text{м}}{\text{с}}, \\
    p &= \sqrt{p_1^2 + p_2^2} = m\sqrt{v_1^2 + v_2^2} = 250\,\frac{\text{кг}\cdot\text{м}}{\text{с}}.
    \end{align*}
}

\tasknumber{4}%
\task{%
    Шарик массой $4\,\text{кг}$ свободно упал на горизонтальную площадку, имея в момент падения скорость $15\,\frac{\text{м}}{\text{c}}$.
    Считая удар абсолютно упругим, определите изменение импульса шарика.
    В ответе укажите модуль полученной величины.
}
\answer{%
    \begin{align*}
    \Delta p &= 2 \cdot mv = 2 \cdot 4\,\text{кг} \cdot 15\,\frac{\text{м}}{\text{c}} = 120\,\frac{\text{кг}\cdot\text{м}}{\text{с}}.
    \end{align*}
}

\tasknumber{5}%
\task{%
    Два тела двигаются навстречу друг другу.
    Скорость каждого из них составляет $6\,\frac{\text{м}}{\text{с}}$.
    После соударения тела слиплись и продолжили движение уже со скоростью $4\,\frac{\text{м}}{\text{с}}$.
    Определите отношение масс тел (большей к меньшей).
}
\answer{%
    \begin{align*}
    &\text{ЗСИ в проекции на ось, соединяющую центры тел:} m_1 v_1 - m_2 v_1 = (m_1 + m_2) v_2 \implies \\
    &\implies \frac{m_1}{m_2} v_1 - v_1 = \cbr{\frac{m_1}{m_2} + 1} v_2 \implies
        \frac{m_1}{m_2} (v_1 - v_2) = v_2 + v_1 \implies \frac{m_1}{m_2} = \frac{v_2 + v_1}{v_1 - v_2} = 5
    \end{align*}
}

\tasknumber{6}%
\task{%
    Шар движется с некоторой скоростью и абсолютно неупруго соударяется с телом, масса которого в 11 раз больше.
    Определите во сколько раз уменьшилась скорость шара после столкновения.
}
\answer{%
    \begin{align*}
    &\text{ЗСИ в проекции на ось, соединяющую центры тел:}  \\
    &mv + 11m \cdot 0 = (m + 11m) v' \implies \\
    &v' = v\frac{m}{11m + m} = \frac{v}{11 + 1} \implies \frac{v}{v'} = 12
    \end{align*}
}

\tasknumber{7}%
\task{%
    Определите работу силы, которая обеспечит спуск тела массой $3\,\text{кг}$ на высоту $10\,\text{м}$ с постоянным ускорением $2\,\frac{\text{м}}{\text{c}^{2}}$.
    % Примите $g = 10\,\frac{\text{м}}{\text{с}^{2}}$.
}
\answer{%
    \begin{align*}
    &\text{Для подъёма:} A = Fh = (mg + ma) h = m(g+a)h, \\
    &\text{Для спуска:} A = -Fh = -(mg - ma) h = -m(g-a)h, \\
    &\text{В результате получаем:} -240\,\text{Дж}.
    \end{align*}
}

\tasknumber{8}%
\task{%
    Тело массой 1\,\text{кг} бросили с обрыва вертикально вверх с начальной скоростью $6\,\frac{\text{м}}{\text{c}}$.
    Через некоторое время скорость тела составила $8\,\frac{\text{м}}{\text{c}}$.
    Пренебрегая сопротивлением воздуха и считая падение тела свободным, определите работу силы тяжести в течение наблюдаемого промежутка времени.
}
\answer{%
    \begin{align*}
    &\text{Изменение кинетической энергии равно работе внешних сил:} \\
    &\Delta E_k = E_k' - E_k = A_\text{тяж} \implies A_\text{тяж} = \frac{mv'^2}2 - \frac{mv_0^2}2 = 14\,\text{Дж}.
    \end{align*}
}

\tasknumber{9}%
\task{%
    Тонкий однородный кусок арматуры длиной $2\,\text{м}$ и массой $10\,\text{кг}$ лежит на горизонтальной поверхности.
    \begin{itemize}
        \item Какую минимальную силу надо приложить к одному из его концов, чтобы оторвать его от этой поверхности?
        \item Какую минимальную работу надо совершить, чтобы поставить его на землю в вертикальное положение?
    \end{itemize}
    % Примите $g = 10\,\frac{\text{м}}{\text{с}^{2}}$.
}
\answer{%
    $F = \frac{mg}2 \approx 100\,\text{Н}, A = mg\frac l2 = 100\,\text{Дж}$
}

\tasknumber{10}%
\task{%
    Для того, чтобы разогать тело из состояния покоя до скорости $v$ с постоянным ускорением,
    требуется совершить работу $200\,\text{Дж}$.
    Какую работу нужно совершить, чтобы увеличить скорость этого тела от $v$ до $4v$?
}
\answer{%
    \begin{align*}
    &\text{Изменение кинетической энергии равно работе внешних сил:} \\
    &A_1 = \frac{mv^2}2 - \frac{m \cdot 0^2}2 = \frac{mv^2}2, A_2 = \frac{m\sqr{4v}}2 - \frac{mv^2}2 \implies  \\
    &\implies A_2 = \frac{mv^2}2 \cbr{4^2 - 1} = A_1 \cdot \cbr{4^2 - 1} = 3000\,\text{Дж}.
    \end{align*}
}

\variantsplitter

\addpersonalvariant{Рената Таржиманова}

\tasknumber{1}%
\task{%
    Шарики массами $1\,\text{кг}$ и $4\,\text{кг}$ движутся параллельно друг другу в одном направлении
    со скоростями $5\,\frac{\text{м}}{\text{с}}$ и $8\,\frac{\text{м}}{\text{с}}$ соответственно.
    Определите общий импульс шариков.
}
\answer{%
    \begin{align*}
    p_1 &= m_1v_1 = 1\,\text{кг} \cdot 5\,\frac{\text{м}}{\text{с}} = 5\,\frac{\text{кг}\cdot\text{м}}{\text{с}}, \\
    p_2 &= m_2v_2 = 4\,\text{кг} \cdot 8\,\frac{\text{м}}{\text{с}} = 32\,\frac{\text{кг}\cdot\text{м}}{\text{с}}, \\
    p &= p_1 + p_2 = m_1v_1 + m_2v_2 = 37\,\frac{\text{кг}\cdot\text{м}}{\text{с}}.
    \end{align*}
}

\tasknumber{2}%
\task{%
    Два шарика, масса каждого из которых составляет $5\,\text{кг}$, движутся навстречу друг другу.
    Скорость одного из них $2\,\frac{\text{м}}{\text{с}}$, а другого~--- $8\,\frac{\text{м}}{\text{с}}$.
    Определите общий импульс шариков.
}
\answer{%
    \begin{align*}
    p_1 &= mv_1 = 5\,\text{кг} \cdot 2\,\frac{\text{м}}{\text{с}} = 10\,\frac{\text{кг}\cdot\text{м}}{\text{с}}, \\
    p_2 &= mv_2 = 5\,\text{кг} \cdot 8\,\frac{\text{м}}{\text{с}} = 40\,\frac{\text{кг}\cdot\text{м}}{\text{с}}, \\
    p &= \abs{p_1 - p_2} = \abs{m(v_1 - v_2)}= 30\,\frac{\text{кг}\cdot\text{м}}{\text{с}}.
    \end{align*}
}

\tasknumber{3}%
\task{%
    Два одинаковых шарика массами по $10\,\text{кг}$ движутся во взаимно перпендикулярных направлениях.
    Скорости шариков составляют $5\,\frac{\text{м}}{\text{с}}$ и $12\,\frac{\text{м}}{\text{с}}$.
    Определите полный импульс системы.
}
\answer{%
    \begin{align*}
    p_1 &= mv_1 = 10\,\text{кг} \cdot 5\,\frac{\text{м}}{\text{с}} = 50\,\frac{\text{кг}\cdot\text{м}}{\text{с}}, \\
    p_2 &= mv_2 = 10\,\text{кг} \cdot 12\,\frac{\text{м}}{\text{с}} = 120\,\frac{\text{кг}\cdot\text{м}}{\text{с}}, \\
    p &= \sqrt{p_1^2 + p_2^2} = m\sqrt{v_1^2 + v_2^2} = 130\,\frac{\text{кг}\cdot\text{м}}{\text{с}}.
    \end{align*}
}

\tasknumber{4}%
\task{%
    Шарик массой $4\,\text{кг}$ свободно упал на горизонтальную площадку, имея в момент падения скорость $20\,\frac{\text{м}}{\text{c}}$.
    Считая удар абсолютно упругим, определите изменение импульса шарика.
    В ответе укажите модуль полученной величины.
}
\answer{%
    \begin{align*}
    \Delta p &= 2 \cdot mv = 2 \cdot 4\,\text{кг} \cdot 20\,\frac{\text{м}}{\text{c}} = 160\,\frac{\text{кг}\cdot\text{м}}{\text{с}}.
    \end{align*}
}

\tasknumber{5}%
\task{%
    Два тела двигаются навстречу друг другу.
    Скорость каждого из них составляет $3\,\frac{\text{м}}{\text{с}}$.
    После соударения тела слиплись и продолжили движение уже со скоростью $1\,\frac{\text{м}}{\text{с}}$.
    Определите отношение масс тел (большей к меньшей).
}
\answer{%
    \begin{align*}
    &\text{ЗСИ в проекции на ось, соединяющую центры тел:} m_1 v_1 - m_2 v_1 = (m_1 + m_2) v_2 \implies \\
    &\implies \frac{m_1}{m_2} v_1 - v_1 = \cbr{\frac{m_1}{m_2} + 1} v_2 \implies
        \frac{m_1}{m_2} (v_1 - v_2) = v_2 + v_1 \implies \frac{m_1}{m_2} = \frac{v_2 + v_1}{v_1 - v_2} = 2
    \end{align*}
}

\tasknumber{6}%
\task{%
    Шар движется с некоторой скоростью и абсолютно неупруго соударяется с телом, масса которого в 9 раз больше.
    Определите во сколько раз уменьшилась скорость шара после столкновения.
}
\answer{%
    \begin{align*}
    &\text{ЗСИ в проекции на ось, соединяющую центры тел:}  \\
    &mv + 9m \cdot 0 = (m + 9m) v' \implies \\
    &v' = v\frac{m}{9m + m} = \frac{v}{9 + 1} \implies \frac{v}{v'} = 10
    \end{align*}
}

\tasknumber{7}%
\task{%
    Определите работу силы, которая обеспечит подъём тела массой $3\,\text{кг}$ на высоту $2\,\text{м}$ с постоянным ускорением $6\,\frac{\text{м}}{\text{c}^{2}}$.
    % Примите $g = 10\,\frac{\text{м}}{\text{с}^{2}}$.
}
\answer{%
    \begin{align*}
    &\text{Для подъёма:} A = Fh = (mg + ma) h = m(g+a)h, \\
    &\text{Для спуска:} A = -Fh = -(mg - ma) h = -m(g-a)h, \\
    &\text{В результате получаем:} 96\,\text{Дж}.
    \end{align*}
}

\tasknumber{8}%
\task{%
    Тело массой 2\,\text{кг} бросили с обрыва под углом $45\degrees$ к горизонту с начальной скоростью $4\,\frac{\text{м}}{\text{c}}$.
    Через некоторое время скорость тела составила $8\,\frac{\text{м}}{\text{c}}$.
    Пренебрегая сопротивлением воздуха и считая падение тела свободным, определите работу силы тяжести в течение наблюдаемого промежутка времени.
}
\answer{%
    \begin{align*}
    &\text{Изменение кинетической энергии равно работе внешних сил:} \\
    &\Delta E_k = E_k' - E_k = A_\text{тяж} \implies A_\text{тяж} = \frac{mv'^2}2 - \frac{mv_0^2}2 = 48\,\text{Дж}.
    \end{align*}
}

\tasknumber{9}%
\task{%
    Тонкий однородный лом длиной $1\,\text{м}$ и массой $10\,\text{кг}$ лежит на горизонтальной поверхности.
    \begin{itemize}
        \item Какую минимальную силу надо приложить к одному из его концов, чтобы оторвать его от этой поверхности?
        \item Какую минимальную работу надо совершить, чтобы поставить его на землю в вертикальное положение?
    \end{itemize}
    % Примите $g = 10\,\frac{\text{м}}{\text{с}^{2}}$.
}
\answer{%
    $F = \frac{mg}2 \approx 100\,\text{Н}, A = mg\frac l2 = 50\,\text{Дж}$
}

\tasknumber{10}%
\task{%
    Для того, чтобы разогать тело из состояния покоя до скорости $v$ с постоянным ускорением,
    требуется совершить работу $200\,\text{Дж}$.
    Какую работу нужно совершить, чтобы увеличить скорость этого тела от $v$ до $3v$?
}
\answer{%
    \begin{align*}
    &\text{Изменение кинетической энергии равно работе внешних сил:} \\
    &A_1 = \frac{mv^2}2 - \frac{m \cdot 0^2}2 = \frac{mv^2}2, A_2 = \frac{m\sqr{3v}}2 - \frac{mv^2}2 \implies  \\
    &\implies A_2 = \frac{mv^2}2 \cbr{3^2 - 1} = A_1 \cdot \cbr{3^2 - 1} = 1600\,\text{Дж}.
    \end{align*}
}

\variantsplitter

\addpersonalvariant{Андрей Щербаков}

\tasknumber{1}%
\task{%
    Шарики массами $3\,\text{кг}$ и $2\,\text{кг}$ движутся параллельно друг другу в одном направлении
    со скоростями $4\,\frac{\text{м}}{\text{с}}$ и $3\,\frac{\text{м}}{\text{с}}$ соответственно.
    Определите общий импульс шариков.
}
\answer{%
    \begin{align*}
    p_1 &= m_1v_1 = 3\,\text{кг} \cdot 4\,\frac{\text{м}}{\text{с}} = 12\,\frac{\text{кг}\cdot\text{м}}{\text{с}}, \\
    p_2 &= m_2v_2 = 2\,\text{кг} \cdot 3\,\frac{\text{м}}{\text{с}} = 6\,\frac{\text{кг}\cdot\text{м}}{\text{с}}, \\
    p &= p_1 + p_2 = m_1v_1 + m_2v_2 = 18\,\frac{\text{кг}\cdot\text{м}}{\text{с}}.
    \end{align*}
}

\tasknumber{2}%
\task{%
    Два шарика, масса каждого из которых составляет $10\,\text{кг}$, движутся навстречу друг другу.
    Скорость одного из них $1\,\frac{\text{м}}{\text{с}}$, а другого~--- $3\,\frac{\text{м}}{\text{с}}$.
    Определите общий импульс шариков.
}
\answer{%
    \begin{align*}
    p_1 &= mv_1 = 10\,\text{кг} \cdot 1\,\frac{\text{м}}{\text{с}} = 10\,\frac{\text{кг}\cdot\text{м}}{\text{с}}, \\
    p_2 &= mv_2 = 10\,\text{кг} \cdot 3\,\frac{\text{м}}{\text{с}} = 30\,\frac{\text{кг}\cdot\text{м}}{\text{с}}, \\
    p &= \abs{p_1 - p_2} = \abs{m(v_1 - v_2)}= 20\,\frac{\text{кг}\cdot\text{м}}{\text{с}}.
    \end{align*}
}

\tasknumber{3}%
\task{%
    Два одинаковых шарика массами по $5\,\text{кг}$ движутся во взаимно перпендикулярных направлениях.
    Скорости шариков составляют $7\,\frac{\text{м}}{\text{с}}$ и $24\,\frac{\text{м}}{\text{с}}$.
    Определите полный импульс системы.
}
\answer{%
    \begin{align*}
    p_1 &= mv_1 = 5\,\text{кг} \cdot 7\,\frac{\text{м}}{\text{с}} = 35\,\frac{\text{кг}\cdot\text{м}}{\text{с}}, \\
    p_2 &= mv_2 = 5\,\text{кг} \cdot 24\,\frac{\text{м}}{\text{с}} = 120\,\frac{\text{кг}\cdot\text{м}}{\text{с}}, \\
    p &= \sqrt{p_1^2 + p_2^2} = m\sqrt{v_1^2 + v_2^2} = 125\,\frac{\text{кг}\cdot\text{м}}{\text{с}}.
    \end{align*}
}

\tasknumber{4}%
\task{%
    Шарик массой $4\,\text{кг}$ свободно упал на горизонтальную площадку, имея в момент падения скорость $25\,\frac{\text{м}}{\text{c}}$.
    Считая удар абсолютно упругим, определите изменение импульса шарика.
    В ответе укажите модуль полученной величины.
}
\answer{%
    \begin{align*}
    \Delta p &= 2 \cdot mv = 2 \cdot 4\,\text{кг} \cdot 25\,\frac{\text{м}}{\text{c}} = 200\,\frac{\text{кг}\cdot\text{м}}{\text{с}}.
    \end{align*}
}

\tasknumber{5}%
\task{%
    Два тела двигаются навстречу друг другу.
    Скорость каждого из них составляет $4\,\frac{\text{м}}{\text{с}}$.
    После соударения тела слиплись и продолжили движение уже со скоростью $3\,\frac{\text{м}}{\text{с}}$.
    Определите отношение масс тел (большей к меньшей).
}
\answer{%
    \begin{align*}
    &\text{ЗСИ в проекции на ось, соединяющую центры тел:} m_1 v_1 - m_2 v_1 = (m_1 + m_2) v_2 \implies \\
    &\implies \frac{m_1}{m_2} v_1 - v_1 = \cbr{\frac{m_1}{m_2} + 1} v_2 \implies
        \frac{m_1}{m_2} (v_1 - v_2) = v_2 + v_1 \implies \frac{m_1}{m_2} = \frac{v_2 + v_1}{v_1 - v_2} = 7
    \end{align*}
}

\tasknumber{6}%
\task{%
    Шар движется с некоторой скоростью и абсолютно неупруго соударяется с телом, масса которого в 8 раз больше.
    Определите во сколько раз уменьшилась скорость шара после столкновения.
}
\answer{%
    \begin{align*}
    &\text{ЗСИ в проекции на ось, соединяющую центры тел:}  \\
    &mv + 8m \cdot 0 = (m + 8m) v' \implies \\
    &v' = v\frac{m}{8m + m} = \frac{v}{8 + 1} \implies \frac{v}{v'} = 9
    \end{align*}
}

\tasknumber{7}%
\task{%
    Определите работу силы, которая обеспечит спуск тела массой $3\,\text{кг}$ на высоту $5\,\text{м}$ с постоянным ускорением $3\,\frac{\text{м}}{\text{c}^{2}}$.
    % Примите $g = 10\,\frac{\text{м}}{\text{с}^{2}}$.
}
\answer{%
    \begin{align*}
    &\text{Для подъёма:} A = Fh = (mg + ma) h = m(g+a)h, \\
    &\text{Для спуска:} A = -Fh = -(mg - ma) h = -m(g-a)h, \\
    &\text{В результате получаем:} -105\,\text{Дж}.
    \end{align*}
}

\tasknumber{8}%
\task{%
    Тело массой 1\,\text{кг} бросили с обрыва горизонтально с начальной скоростью $6\,\frac{\text{м}}{\text{c}}$.
    Через некоторое время скорость тела составила $8\,\frac{\text{м}}{\text{c}}$.
    Пренебрегая сопротивлением воздуха и считая падение тела свободным, определите работу силы тяжести в течение наблюдаемого промежутка времени.
}
\answer{%
    \begin{align*}
    &\text{Изменение кинетической энергии равно работе внешних сил:} \\
    &\Delta E_k = E_k' - E_k = A_\text{тяж} \implies A_\text{тяж} = \frac{mv'^2}2 - \frac{mv_0^2}2 = 14\,\text{Дж}.
    \end{align*}
}

\tasknumber{9}%
\task{%
    Тонкий однородный шест длиной $3\,\text{м}$ и массой $10\,\text{кг}$ лежит на горизонтальной поверхности.
    \begin{itemize}
        \item Какую минимальную силу надо приложить к одному из его концов, чтобы оторвать его от этой поверхности?
        \item Какую минимальную работу надо совершить, чтобы поставить его на землю в вертикальное положение?
    \end{itemize}
    % Примите $g = 10\,\frac{\text{м}}{\text{с}^{2}}$.
}
\answer{%
    $F = \frac{mg}2 \approx 100\,\text{Н}, A = mg\frac l2 = 150\,\text{Дж}$
}

\tasknumber{10}%
\task{%
    Для того, чтобы разогать тело из состояния покоя до скорости $v$ с постоянным ускорением,
    требуется совершить работу $40\,\text{Дж}$.
    Какую работу нужно совершить, чтобы увеличить скорость этого тела от $v$ до $5v$?
}
\answer{%
    \begin{align*}
    &\text{Изменение кинетической энергии равно работе внешних сил:} \\
    &A_1 = \frac{mv^2}2 - \frac{m \cdot 0^2}2 = \frac{mv^2}2, A_2 = \frac{m\sqr{5v}}2 - \frac{mv^2}2 \implies  \\
    &\implies A_2 = \frac{mv^2}2 \cbr{5^2 - 1} = A_1 \cdot \cbr{5^2 - 1} = 960\,\text{Дж}.
    \end{align*}
}

\variantsplitter

\addpersonalvariant{Михаил Ярошевский}

\tasknumber{1}%
\task{%
    Шарики массами $3\,\text{кг}$ и $4\,\text{кг}$ движутся параллельно друг другу в одном направлении
    со скоростями $4\,\frac{\text{м}}{\text{с}}$ и $6\,\frac{\text{м}}{\text{с}}$ соответственно.
    Определите общий импульс шариков.
}
\answer{%
    \begin{align*}
    p_1 &= m_1v_1 = 3\,\text{кг} \cdot 4\,\frac{\text{м}}{\text{с}} = 12\,\frac{\text{кг}\cdot\text{м}}{\text{с}}, \\
    p_2 &= m_2v_2 = 4\,\text{кг} \cdot 6\,\frac{\text{м}}{\text{с}} = 24\,\frac{\text{кг}\cdot\text{м}}{\text{с}}, \\
    p &= p_1 + p_2 = m_1v_1 + m_2v_2 = 36\,\frac{\text{кг}\cdot\text{м}}{\text{с}}.
    \end{align*}
}

\tasknumber{2}%
\task{%
    Два шарика, масса каждого из которых составляет $5\,\text{кг}$, движутся навстречу друг другу.
    Скорость одного из них $1\,\frac{\text{м}}{\text{с}}$, а другого~--- $6\,\frac{\text{м}}{\text{с}}$.
    Определите общий импульс шариков.
}
\answer{%
    \begin{align*}
    p_1 &= mv_1 = 5\,\text{кг} \cdot 1\,\frac{\text{м}}{\text{с}} = 5\,\frac{\text{кг}\cdot\text{м}}{\text{с}}, \\
    p_2 &= mv_2 = 5\,\text{кг} \cdot 6\,\frac{\text{м}}{\text{с}} = 30\,\frac{\text{кг}\cdot\text{м}}{\text{с}}, \\
    p &= \abs{p_1 - p_2} = \abs{m(v_1 - v_2)}= 25\,\frac{\text{кг}\cdot\text{м}}{\text{с}}.
    \end{align*}
}

\tasknumber{3}%
\task{%
    Два одинаковых шарика массами по $2\,\text{кг}$ движутся во взаимно перпендикулярных направлениях.
    Скорости шариков составляют $5\,\frac{\text{м}}{\text{с}}$ и $12\,\frac{\text{м}}{\text{с}}$.
    Определите полный импульс системы.
}
\answer{%
    \begin{align*}
    p_1 &= mv_1 = 2\,\text{кг} \cdot 5\,\frac{\text{м}}{\text{с}} = 10\,\frac{\text{кг}\cdot\text{м}}{\text{с}}, \\
    p_2 &= mv_2 = 2\,\text{кг} \cdot 12\,\frac{\text{м}}{\text{с}} = 24\,\frac{\text{кг}\cdot\text{м}}{\text{с}}, \\
    p &= \sqrt{p_1^2 + p_2^2} = m\sqrt{v_1^2 + v_2^2} = 26\,\frac{\text{кг}\cdot\text{м}}{\text{с}}.
    \end{align*}
}

\tasknumber{4}%
\task{%
    Шарик массой $2\,\text{кг}$ свободно упал на горизонтальную площадку, имея в момент падения скорость $20\,\frac{\text{м}}{\text{c}}$.
    Считая удар абсолютно упругим, определите изменение импульса шарика.
    В ответе укажите модуль полученной величины.
}
\answer{%
    \begin{align*}
    \Delta p &= 2 \cdot mv = 2 \cdot 2\,\text{кг} \cdot 20\,\frac{\text{м}}{\text{c}} = 80\,\frac{\text{кг}\cdot\text{м}}{\text{с}}.
    \end{align*}
}

\tasknumber{5}%
\task{%
    Два тела двигаются навстречу друг другу.
    Скорость каждого из них составляет $6\,\frac{\text{м}}{\text{с}}$.
    После соударения тела слиплись и продолжили движение уже со скоростью $3\,\frac{\text{м}}{\text{с}}$.
    Определите отношение масс тел (большей к меньшей).
}
\answer{%
    \begin{align*}
    &\text{ЗСИ в проекции на ось, соединяющую центры тел:} m_1 v_1 - m_2 v_1 = (m_1 + m_2) v_2 \implies \\
    &\implies \frac{m_1}{m_2} v_1 - v_1 = \cbr{\frac{m_1}{m_2} + 1} v_2 \implies
        \frac{m_1}{m_2} (v_1 - v_2) = v_2 + v_1 \implies \frac{m_1}{m_2} = \frac{v_2 + v_1}{v_1 - v_2} = 3
    \end{align*}
}

\tasknumber{6}%
\task{%
    Шар движется с некоторой скоростью и абсолютно неупруго соударяется с телом, масса которого в 12 раз больше.
    Определите во сколько раз уменьшилась скорость шара после столкновения.
}
\answer{%
    \begin{align*}
    &\text{ЗСИ в проекции на ось, соединяющую центры тел:}  \\
    &mv + 12m \cdot 0 = (m + 12m) v' \implies \\
    &v' = v\frac{m}{12m + m} = \frac{v}{12 + 1} \implies \frac{v}{v'} = 13
    \end{align*}
}

\tasknumber{7}%
\task{%
    Определите работу силы, которая обеспечит подъём тела массой $3\,\text{кг}$ на высоту $5\,\text{м}$ с постоянным ускорением $6\,\frac{\text{м}}{\text{c}^{2}}$.
    % Примите $g = 10\,\frac{\text{м}}{\text{с}^{2}}$.
}
\answer{%
    \begin{align*}
    &\text{Для подъёма:} A = Fh = (mg + ma) h = m(g+a)h, \\
    &\text{Для спуска:} A = -Fh = -(mg - ma) h = -m(g-a)h, \\
    &\text{В результате получаем:} 240\,\text{Дж}.
    \end{align*}
}

\tasknumber{8}%
\task{%
    Тело массой 1\,\text{кг} бросили с обрыва вертикально вверх с начальной скоростью $6\,\frac{\text{м}}{\text{c}}$.
    Через некоторое время скорость тела составила $10\,\frac{\text{м}}{\text{c}}$.
    Пренебрегая сопротивлением воздуха и считая падение тела свободным, определите работу силы тяжести в течение наблюдаемого промежутка времени.
}
\answer{%
    \begin{align*}
    &\text{Изменение кинетической энергии равно работе внешних сил:} \\
    &\Delta E_k = E_k' - E_k = A_\text{тяж} \implies A_\text{тяж} = \frac{mv'^2}2 - \frac{mv_0^2}2 = 32\,\text{Дж}.
    \end{align*}
}

\tasknumber{9}%
\task{%
    Тонкий однородный шест длиной $2\,\text{м}$ и массой $30\,\text{кг}$ лежит на горизонтальной поверхности.
    \begin{itemize}
        \item Какую минимальную силу надо приложить к одному из его концов, чтобы оторвать его от этой поверхности?
        \item Какую минимальную работу надо совершить, чтобы поставить его на землю в вертикальное положение?
    \end{itemize}
    % Примите $g = 10\,\frac{\text{м}}{\text{с}^{2}}$.
}
\answer{%
    $F = \frac{mg}2 \approx 300\,\text{Н}, A = mg\frac l2 = 300\,\text{Дж}$
}

\tasknumber{10}%
\task{%
    Для того, чтобы разогать тело из состояния покоя до скорости $v$ с постоянным ускорением,
    требуется совершить работу $20\,\text{Дж}$.
    Какую работу нужно совершить, чтобы увеличить скорость этого тела от $v$ до $5v$?
}
\answer{%
    \begin{align*}
    &\text{Изменение кинетической энергии равно работе внешних сил:} \\
    &A_1 = \frac{mv^2}2 - \frac{m \cdot 0^2}2 = \frac{mv^2}2, A_2 = \frac{m\sqr{5v}}2 - \frac{mv^2}2 \implies  \\
    &\implies A_2 = \frac{mv^2}2 \cbr{5^2 - 1} = A_1 \cdot \cbr{5^2 - 1} = 480\,\text{Дж}.
    \end{align*}
}

\variantsplitter

\addpersonalvariant{Алексей Алимпиев}

\tasknumber{1}%
\task{%
    Шарики массами $3\,\text{кг}$ и $2\,\text{кг}$ движутся параллельно друг другу в одном направлении
    со скоростями $2\,\frac{\text{м}}{\text{с}}$ и $3\,\frac{\text{м}}{\text{с}}$ соответственно.
    Определите общий импульс шариков.
}
\answer{%
    \begin{align*}
    p_1 &= m_1v_1 = 3\,\text{кг} \cdot 2\,\frac{\text{м}}{\text{с}} = 6\,\frac{\text{кг}\cdot\text{м}}{\text{с}}, \\
    p_2 &= m_2v_2 = 2\,\text{кг} \cdot 3\,\frac{\text{м}}{\text{с}} = 6\,\frac{\text{кг}\cdot\text{м}}{\text{с}}, \\
    p &= p_1 + p_2 = m_1v_1 + m_2v_2 = 12\,\frac{\text{кг}\cdot\text{м}}{\text{с}}.
    \end{align*}
}

\tasknumber{2}%
\task{%
    Два шарика, масса каждого из которых составляет $2\,\text{кг}$, движутся навстречу друг другу.
    Скорость одного из них $10\,\frac{\text{м}}{\text{с}}$, а другого~--- $6\,\frac{\text{м}}{\text{с}}$.
    Определите общий импульс шариков.
}
\answer{%
    \begin{align*}
    p_1 &= mv_1 = 2\,\text{кг} \cdot 10\,\frac{\text{м}}{\text{с}} = 20\,\frac{\text{кг}\cdot\text{м}}{\text{с}}, \\
    p_2 &= mv_2 = 2\,\text{кг} \cdot 6\,\frac{\text{м}}{\text{с}} = 12\,\frac{\text{кг}\cdot\text{м}}{\text{с}}, \\
    p &= \abs{p_1 - p_2} = \abs{m(v_1 - v_2)}= 8\,\frac{\text{кг}\cdot\text{м}}{\text{с}}.
    \end{align*}
}

\tasknumber{3}%
\task{%
    Два одинаковых шарика массами по $10\,\text{кг}$ движутся во взаимно перпендикулярных направлениях.
    Скорости шариков составляют $3\,\frac{\text{м}}{\text{с}}$ и $4\,\frac{\text{м}}{\text{с}}$.
    Определите полный импульс системы.
}
\answer{%
    \begin{align*}
    p_1 &= mv_1 = 10\,\text{кг} \cdot 3\,\frac{\text{м}}{\text{с}} = 30\,\frac{\text{кг}\cdot\text{м}}{\text{с}}, \\
    p_2 &= mv_2 = 10\,\text{кг} \cdot 4\,\frac{\text{м}}{\text{с}} = 40\,\frac{\text{кг}\cdot\text{м}}{\text{с}}, \\
    p &= \sqrt{p_1^2 + p_2^2} = m\sqrt{v_1^2 + v_2^2} = 50\,\frac{\text{кг}\cdot\text{м}}{\text{с}}.
    \end{align*}
}

\tasknumber{4}%
\task{%
    Шарик массой $2\,\text{кг}$ свободно упал на горизонтальную площадку, имея в момент падения скорость $25\,\frac{\text{м}}{\text{c}}$.
    Считая удар абсолютно неупругим, определите изменение импульса шарика.
    В ответе укажите модуль полученной величины.
}
\answer{%
    \begin{align*}
    \Delta p &= 1 \cdot mv = 1 \cdot 2\,\text{кг} \cdot 25\,\frac{\text{м}}{\text{c}} = 50\,\frac{\text{кг}\cdot\text{м}}{\text{с}}.
    \end{align*}
}

\tasknumber{5}%
\task{%
    Два тела двигаются навстречу друг другу.
    Скорость каждого из них составляет $5\,\frac{\text{м}}{\text{с}}$.
    После соударения тела слиплись и продолжили движение уже со скоростью $3\,\frac{\text{м}}{\text{с}}$.
    Определите отношение масс тел (большей к меньшей).
}
\answer{%
    \begin{align*}
    &\text{ЗСИ в проекции на ось, соединяющую центры тел:} m_1 v_1 - m_2 v_1 = (m_1 + m_2) v_2 \implies \\
    &\implies \frac{m_1}{m_2} v_1 - v_1 = \cbr{\frac{m_1}{m_2} + 1} v_2 \implies
        \frac{m_1}{m_2} (v_1 - v_2) = v_2 + v_1 \implies \frac{m_1}{m_2} = \frac{v_2 + v_1}{v_1 - v_2} = 4
    \end{align*}
}

\tasknumber{6}%
\task{%
    Шар движется с некоторой скоростью и абсолютно неупруго соударяется с телом, масса которого в 6 раз больше.
    Определите во сколько раз уменьшилась скорость шара после столкновения.
}
\answer{%
    \begin{align*}
    &\text{ЗСИ в проекции на ось, соединяющую центры тел:}  \\
    &mv + 6m \cdot 0 = (m + 6m) v' \implies \\
    &v' = v\frac{m}{6m + m} = \frac{v}{6 + 1} \implies \frac{v}{v'} = 7
    \end{align*}
}

\tasknumber{7}%
\task{%
    Определите работу силы, которая обеспечит спуск тела массой $3\,\text{кг}$ на высоту $2\,\text{м}$ с постоянным ускорением $3\,\frac{\text{м}}{\text{c}^{2}}$.
    % Примите $g = 10\,\frac{\text{м}}{\text{с}^{2}}$.
}
\answer{%
    \begin{align*}
    &\text{Для подъёма:} A = Fh = (mg + ma) h = m(g+a)h, \\
    &\text{Для спуска:} A = -Fh = -(mg - ma) h = -m(g-a)h, \\
    &\text{В результате получаем:} -42\,\text{Дж}.
    \end{align*}
}

\tasknumber{8}%
\task{%
    Тело массой 2\,\text{кг} бросили с обрыва вертикально вверх с начальной скоростью $6\,\frac{\text{м}}{\text{c}}$.
    Через некоторое время скорость тела составила $8\,\frac{\text{м}}{\text{c}}$.
    Пренебрегая сопротивлением воздуха и считая падение тела свободным, определите работу силы тяжести в течение наблюдаемого промежутка времени.
}
\answer{%
    \begin{align*}
    &\text{Изменение кинетической энергии равно работе внешних сил:} \\
    &\Delta E_k = E_k' - E_k = A_\text{тяж} \implies A_\text{тяж} = \frac{mv'^2}2 - \frac{mv_0^2}2 = 28\,\text{Дж}.
    \end{align*}
}

\tasknumber{9}%
\task{%
    Тонкий однородный кусок арматуры длиной $2\,\text{м}$ и массой $30\,\text{кг}$ лежит на горизонтальной поверхности.
    \begin{itemize}
        \item Какую минимальную силу надо приложить к одному из его концов, чтобы оторвать его от этой поверхности?
        \item Какую минимальную работу надо совершить, чтобы поставить его на землю в вертикальное положение?
    \end{itemize}
    % Примите $g = 10\,\frac{\text{м}}{\text{с}^{2}}$.
}
\answer{%
    $F = \frac{mg}2 \approx 300\,\text{Н}, A = mg\frac l2 = 300\,\text{Дж}$
}

\tasknumber{10}%
\task{%
    Для того, чтобы разогать тело из состояния покоя до скорости $v$ с постоянным ускорением,
    требуется совершить работу $10\,\text{Дж}$.
    Какую работу нужно совершить, чтобы увеличить скорость этого тела от $v$ до $4v$?
}
\answer{%
    \begin{align*}
    &\text{Изменение кинетической энергии равно работе внешних сил:} \\
    &A_1 = \frac{mv^2}2 - \frac{m \cdot 0^2}2 = \frac{mv^2}2, A_2 = \frac{m\sqr{4v}}2 - \frac{mv^2}2 \implies  \\
    &\implies A_2 = \frac{mv^2}2 \cbr{4^2 - 1} = A_1 \cdot \cbr{4^2 - 1} = 150\,\text{Дж}.
    \end{align*}
}

\variantsplitter

\addpersonalvariant{Евгений Васин}

\tasknumber{1}%
\task{%
    Шарики массами $1\,\text{кг}$ и $2\,\text{кг}$ движутся параллельно друг другу в одном направлении
    со скоростями $4\,\frac{\text{м}}{\text{с}}$ и $3\,\frac{\text{м}}{\text{с}}$ соответственно.
    Определите общий импульс шариков.
}
\answer{%
    \begin{align*}
    p_1 &= m_1v_1 = 1\,\text{кг} \cdot 4\,\frac{\text{м}}{\text{с}} = 4\,\frac{\text{кг}\cdot\text{м}}{\text{с}}, \\
    p_2 &= m_2v_2 = 2\,\text{кг} \cdot 3\,\frac{\text{м}}{\text{с}} = 6\,\frac{\text{кг}\cdot\text{м}}{\text{с}}, \\
    p &= p_1 + p_2 = m_1v_1 + m_2v_2 = 10\,\frac{\text{кг}\cdot\text{м}}{\text{с}}.
    \end{align*}
}

\tasknumber{2}%
\task{%
    Два шарика, масса каждого из которых составляет $5\,\text{кг}$, движутся навстречу друг другу.
    Скорость одного из них $10\,\frac{\text{м}}{\text{с}}$, а другого~--- $8\,\frac{\text{м}}{\text{с}}$.
    Определите общий импульс шариков.
}
\answer{%
    \begin{align*}
    p_1 &= mv_1 = 5\,\text{кг} \cdot 10\,\frac{\text{м}}{\text{с}} = 50\,\frac{\text{кг}\cdot\text{м}}{\text{с}}, \\
    p_2 &= mv_2 = 5\,\text{кг} \cdot 8\,\frac{\text{м}}{\text{с}} = 40\,\frac{\text{кг}\cdot\text{м}}{\text{с}}, \\
    p &= \abs{p_1 - p_2} = \abs{m(v_1 - v_2)}= 10\,\frac{\text{кг}\cdot\text{м}}{\text{с}}.
    \end{align*}
}

\tasknumber{3}%
\task{%
    Два одинаковых шарика массами по $2\,\text{кг}$ движутся во взаимно перпендикулярных направлениях.
    Скорости шариков составляют $7\,\frac{\text{м}}{\text{с}}$ и $24\,\frac{\text{м}}{\text{с}}$.
    Определите полный импульс системы.
}
\answer{%
    \begin{align*}
    p_1 &= mv_1 = 2\,\text{кг} \cdot 7\,\frac{\text{м}}{\text{с}} = 14\,\frac{\text{кг}\cdot\text{м}}{\text{с}}, \\
    p_2 &= mv_2 = 2\,\text{кг} \cdot 24\,\frac{\text{м}}{\text{с}} = 48\,\frac{\text{кг}\cdot\text{м}}{\text{с}}, \\
    p &= \sqrt{p_1^2 + p_2^2} = m\sqrt{v_1^2 + v_2^2} = 50\,\frac{\text{кг}\cdot\text{м}}{\text{с}}.
    \end{align*}
}

\tasknumber{4}%
\task{%
    Шарик массой $4\,\text{кг}$ свободно упал на горизонтальную площадку, имея в момент падения скорость $25\,\frac{\text{м}}{\text{c}}$.
    Считая удар абсолютно неупругим, определите изменение импульса шарика.
    В ответе укажите модуль полученной величины.
}
\answer{%
    \begin{align*}
    \Delta p &= 1 \cdot mv = 1 \cdot 4\,\text{кг} \cdot 25\,\frac{\text{м}}{\text{c}} = 100\,\frac{\text{кг}\cdot\text{м}}{\text{с}}.
    \end{align*}
}

\tasknumber{5}%
\task{%
    Два тела двигаются навстречу друг другу.
    Скорость каждого из них составляет $3\,\frac{\text{м}}{\text{с}}$.
    После соударения тела слиплись и продолжили движение уже со скоростью $1\,\frac{\text{м}}{\text{с}}$.
    Определите отношение масс тел (большей к меньшей).
}
\answer{%
    \begin{align*}
    &\text{ЗСИ в проекции на ось, соединяющую центры тел:} m_1 v_1 - m_2 v_1 = (m_1 + m_2) v_2 \implies \\
    &\implies \frac{m_1}{m_2} v_1 - v_1 = \cbr{\frac{m_1}{m_2} + 1} v_2 \implies
        \frac{m_1}{m_2} (v_1 - v_2) = v_2 + v_1 \implies \frac{m_1}{m_2} = \frac{v_2 + v_1}{v_1 - v_2} = 2
    \end{align*}
}

\tasknumber{6}%
\task{%
    Шар движется с некоторой скоростью и абсолютно неупруго соударяется с телом, масса которого в 12 раз больше.
    Определите во сколько раз уменьшилась скорость шара после столкновения.
}
\answer{%
    \begin{align*}
    &\text{ЗСИ в проекции на ось, соединяющую центры тел:}  \\
    &mv + 12m \cdot 0 = (m + 12m) v' \implies \\
    &v' = v\frac{m}{12m + m} = \frac{v}{12 + 1} \implies \frac{v}{v'} = 13
    \end{align*}
}

\tasknumber{7}%
\task{%
    Определите работу силы, которая обеспечит подъём тела массой $2\,\text{кг}$ на высоту $10\,\text{м}$ с постоянным ускорением $6\,\frac{\text{м}}{\text{c}^{2}}$.
    % Примите $g = 10\,\frac{\text{м}}{\text{с}^{2}}$.
}
\answer{%
    \begin{align*}
    &\text{Для подъёма:} A = Fh = (mg + ma) h = m(g+a)h, \\
    &\text{Для спуска:} A = -Fh = -(mg - ma) h = -m(g-a)h, \\
    &\text{В результате получаем:} 320\,\text{Дж}.
    \end{align*}
}

\tasknumber{8}%
\task{%
    Тело массой 2\,\text{кг} бросили с обрыва под углом $45\degrees$ к горизонту с начальной скоростью $2\,\frac{\text{м}}{\text{c}}$.
    Через некоторое время скорость тела составила $8\,\frac{\text{м}}{\text{c}}$.
    Пренебрегая сопротивлением воздуха и считая падение тела свободным, определите работу силы тяжести в течение наблюдаемого промежутка времени.
}
\answer{%
    \begin{align*}
    &\text{Изменение кинетической энергии равно работе внешних сил:} \\
    &\Delta E_k = E_k' - E_k = A_\text{тяж} \implies A_\text{тяж} = \frac{mv'^2}2 - \frac{mv_0^2}2 = 60\,\text{Дж}.
    \end{align*}
}

\tasknumber{9}%
\task{%
    Тонкий однородный лом длиной $3\,\text{м}$ и массой $20\,\text{кг}$ лежит на горизонтальной поверхности.
    \begin{itemize}
        \item Какую минимальную силу надо приложить к одному из его концов, чтобы оторвать его от этой поверхности?
        \item Какую минимальную работу надо совершить, чтобы поставить его на землю в вертикальное положение?
    \end{itemize}
    % Примите $g = 10\,\frac{\text{м}}{\text{с}^{2}}$.
}
\answer{%
    $F = \frac{mg}2 \approx 200\,\text{Н}, A = mg\frac l2 = 300\,\text{Дж}$
}

\tasknumber{10}%
\task{%
    Для того, чтобы разогать тело из состояния покоя до скорости $v$ с постоянным ускорением,
    требуется совершить работу $20\,\text{Дж}$.
    Какую работу нужно совершить, чтобы увеличить скорость этого тела от $v$ до $5v$?
}
\answer{%
    \begin{align*}
    &\text{Изменение кинетической энергии равно работе внешних сил:} \\
    &A_1 = \frac{mv^2}2 - \frac{m \cdot 0^2}2 = \frac{mv^2}2, A_2 = \frac{m\sqr{5v}}2 - \frac{mv^2}2 \implies  \\
    &\implies A_2 = \frac{mv^2}2 \cbr{5^2 - 1} = A_1 \cdot \cbr{5^2 - 1} = 480\,\text{Дж}.
    \end{align*}
}

\variantsplitter

\addpersonalvariant{Вячеслав Волохов}

\tasknumber{1}%
\task{%
    Шарики массами $2\,\text{кг}$ и $1\,\text{кг}$ движутся параллельно друг другу в одном направлении
    со скоростями $5\,\frac{\text{м}}{\text{с}}$ и $6\,\frac{\text{м}}{\text{с}}$ соответственно.
    Определите общий импульс шариков.
}
\answer{%
    \begin{align*}
    p_1 &= m_1v_1 = 2\,\text{кг} \cdot 5\,\frac{\text{м}}{\text{с}} = 10\,\frac{\text{кг}\cdot\text{м}}{\text{с}}, \\
    p_2 &= m_2v_2 = 1\,\text{кг} \cdot 6\,\frac{\text{м}}{\text{с}} = 6\,\frac{\text{кг}\cdot\text{м}}{\text{с}}, \\
    p &= p_1 + p_2 = m_1v_1 + m_2v_2 = 16\,\frac{\text{кг}\cdot\text{м}}{\text{с}}.
    \end{align*}
}

\tasknumber{2}%
\task{%
    Два шарика, масса каждого из которых составляет $2\,\text{кг}$, движутся навстречу друг другу.
    Скорость одного из них $1\,\frac{\text{м}}{\text{с}}$, а другого~--- $6\,\frac{\text{м}}{\text{с}}$.
    Определите общий импульс шариков.
}
\answer{%
    \begin{align*}
    p_1 &= mv_1 = 2\,\text{кг} \cdot 1\,\frac{\text{м}}{\text{с}} = 2\,\frac{\text{кг}\cdot\text{м}}{\text{с}}, \\
    p_2 &= mv_2 = 2\,\text{кг} \cdot 6\,\frac{\text{м}}{\text{с}} = 12\,\frac{\text{кг}\cdot\text{м}}{\text{с}}, \\
    p &= \abs{p_1 - p_2} = \abs{m(v_1 - v_2)}= 10\,\frac{\text{кг}\cdot\text{м}}{\text{с}}.
    \end{align*}
}

\tasknumber{3}%
\task{%
    Два одинаковых шарика массами по $10\,\text{кг}$ движутся во взаимно перпендикулярных направлениях.
    Скорости шариков составляют $5\,\frac{\text{м}}{\text{с}}$ и $12\,\frac{\text{м}}{\text{с}}$.
    Определите полный импульс системы.
}
\answer{%
    \begin{align*}
    p_1 &= mv_1 = 10\,\text{кг} \cdot 5\,\frac{\text{м}}{\text{с}} = 50\,\frac{\text{кг}\cdot\text{м}}{\text{с}}, \\
    p_2 &= mv_2 = 10\,\text{кг} \cdot 12\,\frac{\text{м}}{\text{с}} = 120\,\frac{\text{кг}\cdot\text{м}}{\text{с}}, \\
    p &= \sqrt{p_1^2 + p_2^2} = m\sqrt{v_1^2 + v_2^2} = 130\,\frac{\text{кг}\cdot\text{м}}{\text{с}}.
    \end{align*}
}

\tasknumber{4}%
\task{%
    Шарик массой $1\,\text{кг}$ свободно упал на горизонтальную площадку, имея в момент падения скорость $10\,\frac{\text{м}}{\text{c}}$.
    Считая удар абсолютно неупругим, определите изменение импульса шарика.
    В ответе укажите модуль полученной величины.
}
\answer{%
    \begin{align*}
    \Delta p &= 1 \cdot mv = 1 \cdot 1\,\text{кг} \cdot 10\,\frac{\text{м}}{\text{c}} = 10\,\frac{\text{кг}\cdot\text{м}}{\text{с}}.
    \end{align*}
}

\tasknumber{5}%
\task{%
    Два тела двигаются навстречу друг другу.
    Скорость каждого из них составляет $5\,\frac{\text{м}}{\text{с}}$.
    После соударения тела слиплись и продолжили движение уже со скоростью $4\,\frac{\text{м}}{\text{с}}$.
    Определите отношение масс тел (большей к меньшей).
}
\answer{%
    \begin{align*}
    &\text{ЗСИ в проекции на ось, соединяющую центры тел:} m_1 v_1 - m_2 v_1 = (m_1 + m_2) v_2 \implies \\
    &\implies \frac{m_1}{m_2} v_1 - v_1 = \cbr{\frac{m_1}{m_2} + 1} v_2 \implies
        \frac{m_1}{m_2} (v_1 - v_2) = v_2 + v_1 \implies \frac{m_1}{m_2} = \frac{v_2 + v_1}{v_1 - v_2} = 9
    \end{align*}
}

\tasknumber{6}%
\task{%
    Шар движется с некоторой скоростью и абсолютно неупруго соударяется с телом, масса которого в 14 раз больше.
    Определите во сколько раз уменьшилась скорость шара после столкновения.
}
\answer{%
    \begin{align*}
    &\text{ЗСИ в проекции на ось, соединяющую центры тел:}  \\
    &mv + 14m \cdot 0 = (m + 14m) v' \implies \\
    &v' = v\frac{m}{14m + m} = \frac{v}{14 + 1} \implies \frac{v}{v'} = 15
    \end{align*}
}

\tasknumber{7}%
\task{%
    Определите работу силы, которая обеспечит подъём тела массой $3\,\text{кг}$ на высоту $5\,\text{м}$ с постоянным ускорением $2\,\frac{\text{м}}{\text{c}^{2}}$.
    % Примите $g = 10\,\frac{\text{м}}{\text{с}^{2}}$.
}
\answer{%
    \begin{align*}
    &\text{Для подъёма:} A = Fh = (mg + ma) h = m(g+a)h, \\
    &\text{Для спуска:} A = -Fh = -(mg - ma) h = -m(g-a)h, \\
    &\text{В результате получаем:} 180\,\text{Дж}.
    \end{align*}
}

\tasknumber{8}%
\task{%
    Тело массой 2\,\text{кг} бросили с обрыва горизонтально с начальной скоростью $4\,\frac{\text{м}}{\text{c}}$.
    Через некоторое время скорость тела составила $10\,\frac{\text{м}}{\text{c}}$.
    Пренебрегая сопротивлением воздуха и считая падение тела свободным, определите работу силы тяжести в течение наблюдаемого промежутка времени.
}
\answer{%
    \begin{align*}
    &\text{Изменение кинетической энергии равно работе внешних сил:} \\
    &\Delta E_k = E_k' - E_k = A_\text{тяж} \implies A_\text{тяж} = \frac{mv'^2}2 - \frac{mv_0^2}2 = 84\,\text{Дж}.
    \end{align*}
}

\tasknumber{9}%
\task{%
    Тонкий однородный лом длиной $1\,\text{м}$ и массой $10\,\text{кг}$ лежит на горизонтальной поверхности.
    \begin{itemize}
        \item Какую минимальную силу надо приложить к одному из его концов, чтобы оторвать его от этой поверхности?
        \item Какую минимальную работу надо совершить, чтобы поставить его на землю в вертикальное положение?
    \end{itemize}
    % Примите $g = 10\,\frac{\text{м}}{\text{с}^{2}}$.
}
\answer{%
    $F = \frac{mg}2 \approx 100\,\text{Н}, A = mg\frac l2 = 50\,\text{Дж}$
}

\tasknumber{10}%
\task{%
    Для того, чтобы разогать тело из состояния покоя до скорости $v$ с постоянным ускорением,
    требуется совершить работу $100\,\text{Дж}$.
    Какую работу нужно совершить, чтобы увеличить скорость этого тела от $v$ до $3v$?
}
\answer{%
    \begin{align*}
    &\text{Изменение кинетической энергии равно работе внешних сил:} \\
    &A_1 = \frac{mv^2}2 - \frac{m \cdot 0^2}2 = \frac{mv^2}2, A_2 = \frac{m\sqr{3v}}2 - \frac{mv^2}2 \implies  \\
    &\implies A_2 = \frac{mv^2}2 \cbr{3^2 - 1} = A_1 \cdot \cbr{3^2 - 1} = 800\,\text{Дж}.
    \end{align*}
}

\variantsplitter

\addpersonalvariant{Герман Говоров}

\tasknumber{1}%
\task{%
    Шарики массами $2\,\text{кг}$ и $1\,\text{кг}$ движутся параллельно друг другу в одном направлении
    со скоростями $10\,\frac{\text{м}}{\text{с}}$ и $8\,\frac{\text{м}}{\text{с}}$ соответственно.
    Определите общий импульс шариков.
}
\answer{%
    \begin{align*}
    p_1 &= m_1v_1 = 2\,\text{кг} \cdot 10\,\frac{\text{м}}{\text{с}} = 20\,\frac{\text{кг}\cdot\text{м}}{\text{с}}, \\
    p_2 &= m_2v_2 = 1\,\text{кг} \cdot 8\,\frac{\text{м}}{\text{с}} = 8\,\frac{\text{кг}\cdot\text{м}}{\text{с}}, \\
    p &= p_1 + p_2 = m_1v_1 + m_2v_2 = 28\,\frac{\text{кг}\cdot\text{м}}{\text{с}}.
    \end{align*}
}

\tasknumber{2}%
\task{%
    Два шарика, масса каждого из которых составляет $2\,\text{кг}$, движутся навстречу друг другу.
    Скорость одного из них $5\,\frac{\text{м}}{\text{с}}$, а другого~--- $8\,\frac{\text{м}}{\text{с}}$.
    Определите общий импульс шариков.
}
\answer{%
    \begin{align*}
    p_1 &= mv_1 = 2\,\text{кг} \cdot 5\,\frac{\text{м}}{\text{с}} = 10\,\frac{\text{кг}\cdot\text{м}}{\text{с}}, \\
    p_2 &= mv_2 = 2\,\text{кг} \cdot 8\,\frac{\text{м}}{\text{с}} = 16\,\frac{\text{кг}\cdot\text{м}}{\text{с}}, \\
    p &= \abs{p_1 - p_2} = \abs{m(v_1 - v_2)}= 6\,\frac{\text{кг}\cdot\text{м}}{\text{с}}.
    \end{align*}
}

\tasknumber{3}%
\task{%
    Два одинаковых шарика массами по $5\,\text{кг}$ движутся во взаимно перпендикулярных направлениях.
    Скорости шариков составляют $5\,\frac{\text{м}}{\text{с}}$ и $12\,\frac{\text{м}}{\text{с}}$.
    Определите полный импульс системы.
}
\answer{%
    \begin{align*}
    p_1 &= mv_1 = 5\,\text{кг} \cdot 5\,\frac{\text{м}}{\text{с}} = 25\,\frac{\text{кг}\cdot\text{м}}{\text{с}}, \\
    p_2 &= mv_2 = 5\,\text{кг} \cdot 12\,\frac{\text{м}}{\text{с}} = 60\,\frac{\text{кг}\cdot\text{м}}{\text{с}}, \\
    p &= \sqrt{p_1^2 + p_2^2} = m\sqrt{v_1^2 + v_2^2} = 65\,\frac{\text{кг}\cdot\text{м}}{\text{с}}.
    \end{align*}
}

\tasknumber{4}%
\task{%
    Шарик массой $2\,\text{кг}$ свободно упал на горизонтальную площадку, имея в момент падения скорость $20\,\frac{\text{м}}{\text{c}}$.
    Считая удар абсолютно неупругим, определите изменение импульса шарика.
    В ответе укажите модуль полученной величины.
}
\answer{%
    \begin{align*}
    \Delta p &= 1 \cdot mv = 1 \cdot 2\,\text{кг} \cdot 20\,\frac{\text{м}}{\text{c}} = 40\,\frac{\text{кг}\cdot\text{м}}{\text{с}}.
    \end{align*}
}

\tasknumber{5}%
\task{%
    Два тела двигаются навстречу друг другу.
    Скорость каждого из них составляет $5\,\frac{\text{м}}{\text{с}}$.
    После соударения тела слиплись и продолжили движение уже со скоростью $4\,\frac{\text{м}}{\text{с}}$.
    Определите отношение масс тел (большей к меньшей).
}
\answer{%
    \begin{align*}
    &\text{ЗСИ в проекции на ось, соединяющую центры тел:} m_1 v_1 - m_2 v_1 = (m_1 + m_2) v_2 \implies \\
    &\implies \frac{m_1}{m_2} v_1 - v_1 = \cbr{\frac{m_1}{m_2} + 1} v_2 \implies
        \frac{m_1}{m_2} (v_1 - v_2) = v_2 + v_1 \implies \frac{m_1}{m_2} = \frac{v_2 + v_1}{v_1 - v_2} = 9
    \end{align*}
}

\tasknumber{6}%
\task{%
    Шар движется с некоторой скоростью и абсолютно неупруго соударяется с телом, масса которого в 14 раз больше.
    Определите во сколько раз уменьшилась скорость шара после столкновения.
}
\answer{%
    \begin{align*}
    &\text{ЗСИ в проекции на ось, соединяющую центры тел:}  \\
    &mv + 14m \cdot 0 = (m + 14m) v' \implies \\
    &v' = v\frac{m}{14m + m} = \frac{v}{14 + 1} \implies \frac{v}{v'} = 15
    \end{align*}
}

\tasknumber{7}%
\task{%
    Определите работу силы, которая обеспечит спуск тела массой $2\,\text{кг}$ на высоту $2\,\text{м}$ с постоянным ускорением $6\,\frac{\text{м}}{\text{c}^{2}}$.
    % Примите $g = 10\,\frac{\text{м}}{\text{с}^{2}}$.
}
\answer{%
    \begin{align*}
    &\text{Для подъёма:} A = Fh = (mg + ma) h = m(g+a)h, \\
    &\text{Для спуска:} A = -Fh = -(mg - ma) h = -m(g-a)h, \\
    &\text{В результате получаем:} -16\,\text{Дж}.
    \end{align*}
}

\tasknumber{8}%
\task{%
    Тело массой 2\,\text{кг} бросили с обрыва горизонтально с начальной скоростью $6\,\frac{\text{м}}{\text{c}}$.
    Через некоторое время скорость тела составила $10\,\frac{\text{м}}{\text{c}}$.
    Пренебрегая сопротивлением воздуха и считая падение тела свободным, определите работу силы тяжести в течение наблюдаемого промежутка времени.
}
\answer{%
    \begin{align*}
    &\text{Изменение кинетической энергии равно работе внешних сил:} \\
    &\Delta E_k = E_k' - E_k = A_\text{тяж} \implies A_\text{тяж} = \frac{mv'^2}2 - \frac{mv_0^2}2 = 64\,\text{Дж}.
    \end{align*}
}

\tasknumber{9}%
\task{%
    Тонкий однородный кусок арматуры длиной $2\,\text{м}$ и массой $10\,\text{кг}$ лежит на горизонтальной поверхности.
    \begin{itemize}
        \item Какую минимальную силу надо приложить к одному из его концов, чтобы оторвать его от этой поверхности?
        \item Какую минимальную работу надо совершить, чтобы поставить его на землю в вертикальное положение?
    \end{itemize}
    % Примите $g = 10\,\frac{\text{м}}{\text{с}^{2}}$.
}
\answer{%
    $F = \frac{mg}2 \approx 100\,\text{Н}, A = mg\frac l2 = 100\,\text{Дж}$
}

\tasknumber{10}%
\task{%
    Для того, чтобы разогать тело из состояния покоя до скорости $v$ с постоянным ускорением,
    требуется совершить работу $10\,\text{Дж}$.
    Какую работу нужно совершить, чтобы увеличить скорость этого тела от $v$ до $3v$?
}
\answer{%
    \begin{align*}
    &\text{Изменение кинетической энергии равно работе внешних сил:} \\
    &A_1 = \frac{mv^2}2 - \frac{m \cdot 0^2}2 = \frac{mv^2}2, A_2 = \frac{m\sqr{3v}}2 - \frac{mv^2}2 \implies  \\
    &\implies A_2 = \frac{mv^2}2 \cbr{3^2 - 1} = A_1 \cdot \cbr{3^2 - 1} = 80\,\text{Дж}.
    \end{align*}
}

\variantsplitter

\addpersonalvariant{София Журавлёва}

\tasknumber{1}%
\task{%
    Шарики массами $1\,\text{кг}$ и $3\,\text{кг}$ движутся параллельно друг другу в одном направлении
    со скоростями $4\,\frac{\text{м}}{\text{с}}$ и $6\,\frac{\text{м}}{\text{с}}$ соответственно.
    Определите общий импульс шариков.
}
\answer{%
    \begin{align*}
    p_1 &= m_1v_1 = 1\,\text{кг} \cdot 4\,\frac{\text{м}}{\text{с}} = 4\,\frac{\text{кг}\cdot\text{м}}{\text{с}}, \\
    p_2 &= m_2v_2 = 3\,\text{кг} \cdot 6\,\frac{\text{м}}{\text{с}} = 18\,\frac{\text{кг}\cdot\text{м}}{\text{с}}, \\
    p &= p_1 + p_2 = m_1v_1 + m_2v_2 = 22\,\frac{\text{кг}\cdot\text{м}}{\text{с}}.
    \end{align*}
}

\tasknumber{2}%
\task{%
    Два шарика, масса каждого из которых составляет $5\,\text{кг}$, движутся навстречу друг другу.
    Скорость одного из них $2\,\frac{\text{м}}{\text{с}}$, а другого~--- $8\,\frac{\text{м}}{\text{с}}$.
    Определите общий импульс шариков.
}
\answer{%
    \begin{align*}
    p_1 &= mv_1 = 5\,\text{кг} \cdot 2\,\frac{\text{м}}{\text{с}} = 10\,\frac{\text{кг}\cdot\text{м}}{\text{с}}, \\
    p_2 &= mv_2 = 5\,\text{кг} \cdot 8\,\frac{\text{м}}{\text{с}} = 40\,\frac{\text{кг}\cdot\text{м}}{\text{с}}, \\
    p &= \abs{p_1 - p_2} = \abs{m(v_1 - v_2)}= 30\,\frac{\text{кг}\cdot\text{м}}{\text{с}}.
    \end{align*}
}

\tasknumber{3}%
\task{%
    Два одинаковых шарика массами по $2\,\text{кг}$ движутся во взаимно перпендикулярных направлениях.
    Скорости шариков составляют $7\,\frac{\text{м}}{\text{с}}$ и $24\,\frac{\text{м}}{\text{с}}$.
    Определите полный импульс системы.
}
\answer{%
    \begin{align*}
    p_1 &= mv_1 = 2\,\text{кг} \cdot 7\,\frac{\text{м}}{\text{с}} = 14\,\frac{\text{кг}\cdot\text{м}}{\text{с}}, \\
    p_2 &= mv_2 = 2\,\text{кг} \cdot 24\,\frac{\text{м}}{\text{с}} = 48\,\frac{\text{кг}\cdot\text{м}}{\text{с}}, \\
    p &= \sqrt{p_1^2 + p_2^2} = m\sqrt{v_1^2 + v_2^2} = 50\,\frac{\text{кг}\cdot\text{м}}{\text{с}}.
    \end{align*}
}

\tasknumber{4}%
\task{%
    Шарик массой $2\,\text{кг}$ свободно упал на горизонтальную площадку, имея в момент падения скорость $15\,\frac{\text{м}}{\text{c}}$.
    Считая удар абсолютно упругим, определите изменение импульса шарика.
    В ответе укажите модуль полученной величины.
}
\answer{%
    \begin{align*}
    \Delta p &= 2 \cdot mv = 2 \cdot 2\,\text{кг} \cdot 15\,\frac{\text{м}}{\text{c}} = 60\,\frac{\text{кг}\cdot\text{м}}{\text{с}}.
    \end{align*}
}

\tasknumber{5}%
\task{%
    Два тела двигаются навстречу друг другу.
    Скорость каждого из них составляет $3\,\frac{\text{м}}{\text{с}}$.
    После соударения тела слиплись и продолжили движение уже со скоростью $1\,\frac{\text{м}}{\text{с}}$.
    Определите отношение масс тел (большей к меньшей).
}
\answer{%
    \begin{align*}
    &\text{ЗСИ в проекции на ось, соединяющую центры тел:} m_1 v_1 - m_2 v_1 = (m_1 + m_2) v_2 \implies \\
    &\implies \frac{m_1}{m_2} v_1 - v_1 = \cbr{\frac{m_1}{m_2} + 1} v_2 \implies
        \frac{m_1}{m_2} (v_1 - v_2) = v_2 + v_1 \implies \frac{m_1}{m_2} = \frac{v_2 + v_1}{v_1 - v_2} = 2
    \end{align*}
}

\tasknumber{6}%
\task{%
    Шар движется с некоторой скоростью и абсолютно неупруго соударяется с телом, масса которого в 5 раз больше.
    Определите во сколько раз уменьшилась скорость шара после столкновения.
}
\answer{%
    \begin{align*}
    &\text{ЗСИ в проекции на ось, соединяющую центры тел:}  \\
    &mv + 5m \cdot 0 = (m + 5m) v' \implies \\
    &v' = v\frac{m}{5m + m} = \frac{v}{5 + 1} \implies \frac{v}{v'} = 6
    \end{align*}
}

\tasknumber{7}%
\task{%
    Определите работу силы, которая обеспечит подъём тела массой $2\,\text{кг}$ на высоту $10\,\text{м}$ с постоянным ускорением $3\,\frac{\text{м}}{\text{c}^{2}}$.
    % Примите $g = 10\,\frac{\text{м}}{\text{с}^{2}}$.
}
\answer{%
    \begin{align*}
    &\text{Для подъёма:} A = Fh = (mg + ma) h = m(g+a)h, \\
    &\text{Для спуска:} A = -Fh = -(mg - ma) h = -m(g-a)h, \\
    &\text{В результате получаем:} 260\,\text{Дж}.
    \end{align*}
}

\tasknumber{8}%
\task{%
    Тело массой 1\,\text{кг} бросили с обрыва под углом $45\degrees$ к горизонту с начальной скоростью $4\,\frac{\text{м}}{\text{c}}$.
    Через некоторое время скорость тела составила $8\,\frac{\text{м}}{\text{c}}$.
    Пренебрегая сопротивлением воздуха и считая падение тела свободным, определите работу силы тяжести в течение наблюдаемого промежутка времени.
}
\answer{%
    \begin{align*}
    &\text{Изменение кинетической энергии равно работе внешних сил:} \\
    &\Delta E_k = E_k' - E_k = A_\text{тяж} \implies A_\text{тяж} = \frac{mv'^2}2 - \frac{mv_0^2}2 = 24\,\text{Дж}.
    \end{align*}
}

\tasknumber{9}%
\task{%
    Тонкий однородный кусок арматуры длиной $3\,\text{м}$ и массой $30\,\text{кг}$ лежит на горизонтальной поверхности.
    \begin{itemize}
        \item Какую минимальную силу надо приложить к одному из его концов, чтобы оторвать его от этой поверхности?
        \item Какую минимальную работу надо совершить, чтобы поставить его на землю в вертикальное положение?
    \end{itemize}
    % Примите $g = 10\,\frac{\text{м}}{\text{с}^{2}}$.
}
\answer{%
    $F = \frac{mg}2 \approx 300\,\text{Н}, A = mg\frac l2 = 450\,\text{Дж}$
}

\tasknumber{10}%
\task{%
    Для того, чтобы разогать тело из состояния покоя до скорости $v$ с постоянным ускорением,
    требуется совершить работу $10\,\text{Дж}$.
    Какую работу нужно совершить, чтобы увеличить скорость этого тела от $v$ до $5v$?
}
\answer{%
    \begin{align*}
    &\text{Изменение кинетической энергии равно работе внешних сил:} \\
    &A_1 = \frac{mv^2}2 - \frac{m \cdot 0^2}2 = \frac{mv^2}2, A_2 = \frac{m\sqr{5v}}2 - \frac{mv^2}2 \implies  \\
    &\implies A_2 = \frac{mv^2}2 \cbr{5^2 - 1} = A_1 \cdot \cbr{5^2 - 1} = 240\,\text{Дж}.
    \end{align*}
}

\variantsplitter

\addpersonalvariant{Константин Козлов}

\tasknumber{1}%
\task{%
    Шарики массами $3\,\text{кг}$ и $1\,\text{кг}$ движутся параллельно друг другу в одном направлении
    со скоростями $4\,\frac{\text{м}}{\text{с}}$ и $6\,\frac{\text{м}}{\text{с}}$ соответственно.
    Определите общий импульс шариков.
}
\answer{%
    \begin{align*}
    p_1 &= m_1v_1 = 3\,\text{кг} \cdot 4\,\frac{\text{м}}{\text{с}} = 12\,\frac{\text{кг}\cdot\text{м}}{\text{с}}, \\
    p_2 &= m_2v_2 = 1\,\text{кг} \cdot 6\,\frac{\text{м}}{\text{с}} = 6\,\frac{\text{кг}\cdot\text{м}}{\text{с}}, \\
    p &= p_1 + p_2 = m_1v_1 + m_2v_2 = 18\,\frac{\text{кг}\cdot\text{м}}{\text{с}}.
    \end{align*}
}

\tasknumber{2}%
\task{%
    Два шарика, масса каждого из которых составляет $2\,\text{кг}$, движутся навстречу друг другу.
    Скорость одного из них $10\,\frac{\text{м}}{\text{с}}$, а другого~--- $3\,\frac{\text{м}}{\text{с}}$.
    Определите общий импульс шариков.
}
\answer{%
    \begin{align*}
    p_1 &= mv_1 = 2\,\text{кг} \cdot 10\,\frac{\text{м}}{\text{с}} = 20\,\frac{\text{кг}\cdot\text{м}}{\text{с}}, \\
    p_2 &= mv_2 = 2\,\text{кг} \cdot 3\,\frac{\text{м}}{\text{с}} = 6\,\frac{\text{кг}\cdot\text{м}}{\text{с}}, \\
    p &= \abs{p_1 - p_2} = \abs{m(v_1 - v_2)}= 14\,\frac{\text{кг}\cdot\text{м}}{\text{с}}.
    \end{align*}
}

\tasknumber{3}%
\task{%
    Два одинаковых шарика массами по $5\,\text{кг}$ движутся во взаимно перпендикулярных направлениях.
    Скорости шариков составляют $3\,\frac{\text{м}}{\text{с}}$ и $4\,\frac{\text{м}}{\text{с}}$.
    Определите полный импульс системы.
}
\answer{%
    \begin{align*}
    p_1 &= mv_1 = 5\,\text{кг} \cdot 3\,\frac{\text{м}}{\text{с}} = 15\,\frac{\text{кг}\cdot\text{м}}{\text{с}}, \\
    p_2 &= mv_2 = 5\,\text{кг} \cdot 4\,\frac{\text{м}}{\text{с}} = 20\,\frac{\text{кг}\cdot\text{м}}{\text{с}}, \\
    p &= \sqrt{p_1^2 + p_2^2} = m\sqrt{v_1^2 + v_2^2} = 25\,\frac{\text{кг}\cdot\text{м}}{\text{с}}.
    \end{align*}
}

\tasknumber{4}%
\task{%
    Шарик массой $4\,\text{кг}$ свободно упал на горизонтальную площадку, имея в момент падения скорость $25\,\frac{\text{м}}{\text{c}}$.
    Считая удар абсолютно неупругим, определите изменение импульса шарика.
    В ответе укажите модуль полученной величины.
}
\answer{%
    \begin{align*}
    \Delta p &= 1 \cdot mv = 1 \cdot 4\,\text{кг} \cdot 25\,\frac{\text{м}}{\text{c}} = 100\,\frac{\text{кг}\cdot\text{м}}{\text{с}}.
    \end{align*}
}

\tasknumber{5}%
\task{%
    Два тела двигаются навстречу друг другу.
    Скорость каждого из них составляет $6\,\frac{\text{м}}{\text{с}}$.
    После соударения тела слиплись и продолжили движение уже со скоростью $3\,\frac{\text{м}}{\text{с}}$.
    Определите отношение масс тел (большей к меньшей).
}
\answer{%
    \begin{align*}
    &\text{ЗСИ в проекции на ось, соединяющую центры тел:} m_1 v_1 - m_2 v_1 = (m_1 + m_2) v_2 \implies \\
    &\implies \frac{m_1}{m_2} v_1 - v_1 = \cbr{\frac{m_1}{m_2} + 1} v_2 \implies
        \frac{m_1}{m_2} (v_1 - v_2) = v_2 + v_1 \implies \frac{m_1}{m_2} = \frac{v_2 + v_1}{v_1 - v_2} = 3
    \end{align*}
}

\tasknumber{6}%
\task{%
    Шар движется с некоторой скоростью и абсолютно неупруго соударяется с телом, масса которого в 6 раз больше.
    Определите во сколько раз уменьшилась скорость шара после столкновения.
}
\answer{%
    \begin{align*}
    &\text{ЗСИ в проекции на ось, соединяющую центры тел:}  \\
    &mv + 6m \cdot 0 = (m + 6m) v' \implies \\
    &v' = v\frac{m}{6m + m} = \frac{v}{6 + 1} \implies \frac{v}{v'} = 7
    \end{align*}
}

\tasknumber{7}%
\task{%
    Определите работу силы, которая обеспечит подъём тела массой $5\,\text{кг}$ на высоту $2\,\text{м}$ с постоянным ускорением $6\,\frac{\text{м}}{\text{c}^{2}}$.
    % Примите $g = 10\,\frac{\text{м}}{\text{с}^{2}}$.
}
\answer{%
    \begin{align*}
    &\text{Для подъёма:} A = Fh = (mg + ma) h = m(g+a)h, \\
    &\text{Для спуска:} A = -Fh = -(mg - ma) h = -m(g-a)h, \\
    &\text{В результате получаем:} 160\,\text{Дж}.
    \end{align*}
}

\tasknumber{8}%
\task{%
    Тело массой 2\,\text{кг} бросили с обрыва горизонтально с начальной скоростью $4\,\frac{\text{м}}{\text{c}}$.
    Через некоторое время скорость тела составила $8\,\frac{\text{м}}{\text{c}}$.
    Пренебрегая сопротивлением воздуха и считая падение тела свободным, определите работу силы тяжести в течение наблюдаемого промежутка времени.
}
\answer{%
    \begin{align*}
    &\text{Изменение кинетической энергии равно работе внешних сил:} \\
    &\Delta E_k = E_k' - E_k = A_\text{тяж} \implies A_\text{тяж} = \frac{mv'^2}2 - \frac{mv_0^2}2 = 48\,\text{Дж}.
    \end{align*}
}

\tasknumber{9}%
\task{%
    Тонкий однородный шест длиной $2\,\text{м}$ и массой $20\,\text{кг}$ лежит на горизонтальной поверхности.
    \begin{itemize}
        \item Какую минимальную силу надо приложить к одному из его концов, чтобы оторвать его от этой поверхности?
        \item Какую минимальную работу надо совершить, чтобы поставить его на землю в вертикальное положение?
    \end{itemize}
    % Примите $g = 10\,\frac{\text{м}}{\text{с}^{2}}$.
}
\answer{%
    $F = \frac{mg}2 \approx 200\,\text{Н}, A = mg\frac l2 = 200\,\text{Дж}$
}

\tasknumber{10}%
\task{%
    Для того, чтобы разогать тело из состояния покоя до скорости $v$ с постоянным ускорением,
    требуется совершить работу $10\,\text{Дж}$.
    Какую работу нужно совершить, чтобы увеличить скорость этого тела от $v$ до $4v$?
}
\answer{%
    \begin{align*}
    &\text{Изменение кинетической энергии равно работе внешних сил:} \\
    &A_1 = \frac{mv^2}2 - \frac{m \cdot 0^2}2 = \frac{mv^2}2, A_2 = \frac{m\sqr{4v}}2 - \frac{mv^2}2 \implies  \\
    &\implies A_2 = \frac{mv^2}2 \cbr{4^2 - 1} = A_1 \cdot \cbr{4^2 - 1} = 150\,\text{Дж}.
    \end{align*}
}

\variantsplitter

\addpersonalvariant{Наталья Кравченко}

\tasknumber{1}%
\task{%
    Шарики массами $3\,\text{кг}$ и $1\,\text{кг}$ движутся параллельно друг другу в одном направлении
    со скоростями $4\,\frac{\text{м}}{\text{с}}$ и $6\,\frac{\text{м}}{\text{с}}$ соответственно.
    Определите общий импульс шариков.
}
\answer{%
    \begin{align*}
    p_1 &= m_1v_1 = 3\,\text{кг} \cdot 4\,\frac{\text{м}}{\text{с}} = 12\,\frac{\text{кг}\cdot\text{м}}{\text{с}}, \\
    p_2 &= m_2v_2 = 1\,\text{кг} \cdot 6\,\frac{\text{м}}{\text{с}} = 6\,\frac{\text{кг}\cdot\text{м}}{\text{с}}, \\
    p &= p_1 + p_2 = m_1v_1 + m_2v_2 = 18\,\frac{\text{кг}\cdot\text{м}}{\text{с}}.
    \end{align*}
}

\tasknumber{2}%
\task{%
    Два шарика, масса каждого из которых составляет $10\,\text{кг}$, движутся навстречу друг другу.
    Скорость одного из них $1\,\frac{\text{м}}{\text{с}}$, а другого~--- $6\,\frac{\text{м}}{\text{с}}$.
    Определите общий импульс шариков.
}
\answer{%
    \begin{align*}
    p_1 &= mv_1 = 10\,\text{кг} \cdot 1\,\frac{\text{м}}{\text{с}} = 10\,\frac{\text{кг}\cdot\text{м}}{\text{с}}, \\
    p_2 &= mv_2 = 10\,\text{кг} \cdot 6\,\frac{\text{м}}{\text{с}} = 60\,\frac{\text{кг}\cdot\text{м}}{\text{с}}, \\
    p &= \abs{p_1 - p_2} = \abs{m(v_1 - v_2)}= 50\,\frac{\text{кг}\cdot\text{м}}{\text{с}}.
    \end{align*}
}

\tasknumber{3}%
\task{%
    Два одинаковых шарика массами по $5\,\text{кг}$ движутся во взаимно перпендикулярных направлениях.
    Скорости шариков составляют $5\,\frac{\text{м}}{\text{с}}$ и $12\,\frac{\text{м}}{\text{с}}$.
    Определите полный импульс системы.
}
\answer{%
    \begin{align*}
    p_1 &= mv_1 = 5\,\text{кг} \cdot 5\,\frac{\text{м}}{\text{с}} = 25\,\frac{\text{кг}\cdot\text{м}}{\text{с}}, \\
    p_2 &= mv_2 = 5\,\text{кг} \cdot 12\,\frac{\text{м}}{\text{с}} = 60\,\frac{\text{кг}\cdot\text{м}}{\text{с}}, \\
    p &= \sqrt{p_1^2 + p_2^2} = m\sqrt{v_1^2 + v_2^2} = 65\,\frac{\text{кг}\cdot\text{м}}{\text{с}}.
    \end{align*}
}

\tasknumber{4}%
\task{%
    Шарик массой $4\,\text{кг}$ свободно упал на горизонтальную площадку, имея в момент падения скорость $25\,\frac{\text{м}}{\text{c}}$.
    Считая удар абсолютно упругим, определите изменение импульса шарика.
    В ответе укажите модуль полученной величины.
}
\answer{%
    \begin{align*}
    \Delta p &= 2 \cdot mv = 2 \cdot 4\,\text{кг} \cdot 25\,\frac{\text{м}}{\text{c}} = 200\,\frac{\text{кг}\cdot\text{м}}{\text{с}}.
    \end{align*}
}

\tasknumber{5}%
\task{%
    Два тела двигаются навстречу друг другу.
    Скорость каждого из них составляет $5\,\frac{\text{м}}{\text{с}}$.
    После соударения тела слиплись и продолжили движение уже со скоростью $4\,\frac{\text{м}}{\text{с}}$.
    Определите отношение масс тел (большей к меньшей).
}
\answer{%
    \begin{align*}
    &\text{ЗСИ в проекции на ось, соединяющую центры тел:} m_1 v_1 - m_2 v_1 = (m_1 + m_2) v_2 \implies \\
    &\implies \frac{m_1}{m_2} v_1 - v_1 = \cbr{\frac{m_1}{m_2} + 1} v_2 \implies
        \frac{m_1}{m_2} (v_1 - v_2) = v_2 + v_1 \implies \frac{m_1}{m_2} = \frac{v_2 + v_1}{v_1 - v_2} = 9
    \end{align*}
}

\tasknumber{6}%
\task{%
    Шар движется с некоторой скоростью и абсолютно неупруго соударяется с телом, масса которого в 7 раз больше.
    Определите во сколько раз уменьшилась скорость шара после столкновения.
}
\answer{%
    \begin{align*}
    &\text{ЗСИ в проекции на ось, соединяющую центры тел:}  \\
    &mv + 7m \cdot 0 = (m + 7m) v' \implies \\
    &v' = v\frac{m}{7m + m} = \frac{v}{7 + 1} \implies \frac{v}{v'} = 8
    \end{align*}
}

\tasknumber{7}%
\task{%
    Определите работу силы, которая обеспечит подъём тела массой $3\,\text{кг}$ на высоту $2\,\text{м}$ с постоянным ускорением $6\,\frac{\text{м}}{\text{c}^{2}}$.
    % Примите $g = 10\,\frac{\text{м}}{\text{с}^{2}}$.
}
\answer{%
    \begin{align*}
    &\text{Для подъёма:} A = Fh = (mg + ma) h = m(g+a)h, \\
    &\text{Для спуска:} A = -Fh = -(mg - ma) h = -m(g-a)h, \\
    &\text{В результате получаем:} 96\,\text{Дж}.
    \end{align*}
}

\tasknumber{8}%
\task{%
    Тело массой 2\,\text{кг} бросили с обрыва под углом $45\degrees$ к горизонту с начальной скоростью $4\,\frac{\text{м}}{\text{c}}$.
    Через некоторое время скорость тела составила $8\,\frac{\text{м}}{\text{c}}$.
    Пренебрегая сопротивлением воздуха и считая падение тела свободным, определите работу силы тяжести в течение наблюдаемого промежутка времени.
}
\answer{%
    \begin{align*}
    &\text{Изменение кинетической энергии равно работе внешних сил:} \\
    &\Delta E_k = E_k' - E_k = A_\text{тяж} \implies A_\text{тяж} = \frac{mv'^2}2 - \frac{mv_0^2}2 = 48\,\text{Дж}.
    \end{align*}
}

\tasknumber{9}%
\task{%
    Тонкий однородный кусок арматуры длиной $2\,\text{м}$ и массой $20\,\text{кг}$ лежит на горизонтальной поверхности.
    \begin{itemize}
        \item Какую минимальную силу надо приложить к одному из его концов, чтобы оторвать его от этой поверхности?
        \item Какую минимальную работу надо совершить, чтобы поставить его на землю в вертикальное положение?
    \end{itemize}
    % Примите $g = 10\,\frac{\text{м}}{\text{с}^{2}}$.
}
\answer{%
    $F = \frac{mg}2 \approx 200\,\text{Н}, A = mg\frac l2 = 200\,\text{Дж}$
}

\tasknumber{10}%
\task{%
    Для того, чтобы разогать тело из состояния покоя до скорости $v$ с постоянным ускорением,
    требуется совершить работу $40\,\text{Дж}$.
    Какую работу нужно совершить, чтобы увеличить скорость этого тела от $v$ до $3v$?
}
\answer{%
    \begin{align*}
    &\text{Изменение кинетической энергии равно работе внешних сил:} \\
    &A_1 = \frac{mv^2}2 - \frac{m \cdot 0^2}2 = \frac{mv^2}2, A_2 = \frac{m\sqr{3v}}2 - \frac{mv^2}2 \implies  \\
    &\implies A_2 = \frac{mv^2}2 \cbr{3^2 - 1} = A_1 \cdot \cbr{3^2 - 1} = 320\,\text{Дж}.
    \end{align*}
}

\variantsplitter

\addpersonalvariant{Матвей Кузьмин}

\tasknumber{1}%
\task{%
    Шарики массами $4\,\text{кг}$ и $1\,\text{кг}$ движутся параллельно друг другу в одном направлении
    со скоростями $10\,\frac{\text{м}}{\text{с}}$ и $6\,\frac{\text{м}}{\text{с}}$ соответственно.
    Определите общий импульс шариков.
}
\answer{%
    \begin{align*}
    p_1 &= m_1v_1 = 4\,\text{кг} \cdot 10\,\frac{\text{м}}{\text{с}} = 40\,\frac{\text{кг}\cdot\text{м}}{\text{с}}, \\
    p_2 &= m_2v_2 = 1\,\text{кг} \cdot 6\,\frac{\text{м}}{\text{с}} = 6\,\frac{\text{кг}\cdot\text{м}}{\text{с}}, \\
    p &= p_1 + p_2 = m_1v_1 + m_2v_2 = 46\,\frac{\text{кг}\cdot\text{м}}{\text{с}}.
    \end{align*}
}

\tasknumber{2}%
\task{%
    Два шарика, масса каждого из которых составляет $5\,\text{кг}$, движутся навстречу друг другу.
    Скорость одного из них $2\,\frac{\text{м}}{\text{с}}$, а другого~--- $6\,\frac{\text{м}}{\text{с}}$.
    Определите общий импульс шариков.
}
\answer{%
    \begin{align*}
    p_1 &= mv_1 = 5\,\text{кг} \cdot 2\,\frac{\text{м}}{\text{с}} = 10\,\frac{\text{кг}\cdot\text{м}}{\text{с}}, \\
    p_2 &= mv_2 = 5\,\text{кг} \cdot 6\,\frac{\text{м}}{\text{с}} = 30\,\frac{\text{кг}\cdot\text{м}}{\text{с}}, \\
    p &= \abs{p_1 - p_2} = \abs{m(v_1 - v_2)}= 20\,\frac{\text{кг}\cdot\text{м}}{\text{с}}.
    \end{align*}
}

\tasknumber{3}%
\task{%
    Два одинаковых шарика массами по $5\,\text{кг}$ движутся во взаимно перпендикулярных направлениях.
    Скорости шариков составляют $3\,\frac{\text{м}}{\text{с}}$ и $4\,\frac{\text{м}}{\text{с}}$.
    Определите полный импульс системы.
}
\answer{%
    \begin{align*}
    p_1 &= mv_1 = 5\,\text{кг} \cdot 3\,\frac{\text{м}}{\text{с}} = 15\,\frac{\text{кг}\cdot\text{м}}{\text{с}}, \\
    p_2 &= mv_2 = 5\,\text{кг} \cdot 4\,\frac{\text{м}}{\text{с}} = 20\,\frac{\text{кг}\cdot\text{м}}{\text{с}}, \\
    p &= \sqrt{p_1^2 + p_2^2} = m\sqrt{v_1^2 + v_2^2} = 25\,\frac{\text{кг}\cdot\text{м}}{\text{с}}.
    \end{align*}
}

\tasknumber{4}%
\task{%
    Шарик массой $4\,\text{кг}$ свободно упал на горизонтальную площадку, имея в момент падения скорость $20\,\frac{\text{м}}{\text{c}}$.
    Считая удар абсолютно упругим, определите изменение импульса шарика.
    В ответе укажите модуль полученной величины.
}
\answer{%
    \begin{align*}
    \Delta p &= 2 \cdot mv = 2 \cdot 4\,\text{кг} \cdot 20\,\frac{\text{м}}{\text{c}} = 160\,\frac{\text{кг}\cdot\text{м}}{\text{с}}.
    \end{align*}
}

\tasknumber{5}%
\task{%
    Два тела двигаются навстречу друг другу.
    Скорость каждого из них составляет $5\,\frac{\text{м}}{\text{с}}$.
    После соударения тела слиплись и продолжили движение уже со скоростью $4\,\frac{\text{м}}{\text{с}}$.
    Определите отношение масс тел (большей к меньшей).
}
\answer{%
    \begin{align*}
    &\text{ЗСИ в проекции на ось, соединяющую центры тел:} m_1 v_1 - m_2 v_1 = (m_1 + m_2) v_2 \implies \\
    &\implies \frac{m_1}{m_2} v_1 - v_1 = \cbr{\frac{m_1}{m_2} + 1} v_2 \implies
        \frac{m_1}{m_2} (v_1 - v_2) = v_2 + v_1 \implies \frac{m_1}{m_2} = \frac{v_2 + v_1}{v_1 - v_2} = 9
    \end{align*}
}

\tasknumber{6}%
\task{%
    Шар движется с некоторой скоростью и абсолютно неупруго соударяется с телом, масса которого в 10 раз больше.
    Определите во сколько раз уменьшилась скорость шара после столкновения.
}
\answer{%
    \begin{align*}
    &\text{ЗСИ в проекции на ось, соединяющую центры тел:}  \\
    &mv + 10m \cdot 0 = (m + 10m) v' \implies \\
    &v' = v\frac{m}{10m + m} = \frac{v}{10 + 1} \implies \frac{v}{v'} = 11
    \end{align*}
}

\tasknumber{7}%
\task{%
    Определите работу силы, которая обеспечит спуск тела массой $5\,\text{кг}$ на высоту $10\,\text{м}$ с постоянным ускорением $2\,\frac{\text{м}}{\text{c}^{2}}$.
    % Примите $g = 10\,\frac{\text{м}}{\text{с}^{2}}$.
}
\answer{%
    \begin{align*}
    &\text{Для подъёма:} A = Fh = (mg + ma) h = m(g+a)h, \\
    &\text{Для спуска:} A = -Fh = -(mg - ma) h = -m(g-a)h, \\
    &\text{В результате получаем:} -400\,\text{Дж}.
    \end{align*}
}

\tasknumber{8}%
\task{%
    Тело массой 3\,\text{кг} бросили с обрыва горизонтально с начальной скоростью $4\,\frac{\text{м}}{\text{c}}$.
    Через некоторое время скорость тела составила $10\,\frac{\text{м}}{\text{c}}$.
    Пренебрегая сопротивлением воздуха и считая падение тела свободным, определите работу силы тяжести в течение наблюдаемого промежутка времени.
}
\answer{%
    \begin{align*}
    &\text{Изменение кинетической энергии равно работе внешних сил:} \\
    &\Delta E_k = E_k' - E_k = A_\text{тяж} \implies A_\text{тяж} = \frac{mv'^2}2 - \frac{mv_0^2}2 = 126\,\text{Дж}.
    \end{align*}
}

\tasknumber{9}%
\task{%
    Тонкий однородный кусок арматуры длиной $1\,\text{м}$ и массой $20\,\text{кг}$ лежит на горизонтальной поверхности.
    \begin{itemize}
        \item Какую минимальную силу надо приложить к одному из его концов, чтобы оторвать его от этой поверхности?
        \item Какую минимальную работу надо совершить, чтобы поставить его на землю в вертикальное положение?
    \end{itemize}
    % Примите $g = 10\,\frac{\text{м}}{\text{с}^{2}}$.
}
\answer{%
    $F = \frac{mg}2 \approx 200\,\text{Н}, A = mg\frac l2 = 100\,\text{Дж}$
}

\tasknumber{10}%
\task{%
    Для того, чтобы разогать тело из состояния покоя до скорости $v$ с постоянным ускорением,
    требуется совершить работу $40\,\text{Дж}$.
    Какую работу нужно совершить, чтобы увеличить скорость этого тела от $v$ до $2v$?
}
\answer{%
    \begin{align*}
    &\text{Изменение кинетической энергии равно работе внешних сил:} \\
    &A_1 = \frac{mv^2}2 - \frac{m \cdot 0^2}2 = \frac{mv^2}2, A_2 = \frac{m\sqr{2v}}2 - \frac{mv^2}2 \implies  \\
    &\implies A_2 = \frac{mv^2}2 \cbr{2^2 - 1} = A_1 \cdot \cbr{2^2 - 1} = 120\,\text{Дж}.
    \end{align*}
}

\variantsplitter

\addpersonalvariant{Сергей Малышев}

\tasknumber{1}%
\task{%
    Шарики массами $1\,\text{кг}$ и $3\,\text{кг}$ движутся параллельно друг другу в одном направлении
    со скоростями $10\,\frac{\text{м}}{\text{с}}$ и $3\,\frac{\text{м}}{\text{с}}$ соответственно.
    Определите общий импульс шариков.
}
\answer{%
    \begin{align*}
    p_1 &= m_1v_1 = 1\,\text{кг} \cdot 10\,\frac{\text{м}}{\text{с}} = 10\,\frac{\text{кг}\cdot\text{м}}{\text{с}}, \\
    p_2 &= m_2v_2 = 3\,\text{кг} \cdot 3\,\frac{\text{м}}{\text{с}} = 9\,\frac{\text{кг}\cdot\text{м}}{\text{с}}, \\
    p &= p_1 + p_2 = m_1v_1 + m_2v_2 = 19\,\frac{\text{кг}\cdot\text{м}}{\text{с}}.
    \end{align*}
}

\tasknumber{2}%
\task{%
    Два шарика, масса каждого из которых составляет $10\,\text{кг}$, движутся навстречу друг другу.
    Скорость одного из них $2\,\frac{\text{м}}{\text{с}}$, а другого~--- $6\,\frac{\text{м}}{\text{с}}$.
    Определите общий импульс шариков.
}
\answer{%
    \begin{align*}
    p_1 &= mv_1 = 10\,\text{кг} \cdot 2\,\frac{\text{м}}{\text{с}} = 20\,\frac{\text{кг}\cdot\text{м}}{\text{с}}, \\
    p_2 &= mv_2 = 10\,\text{кг} \cdot 6\,\frac{\text{м}}{\text{с}} = 60\,\frac{\text{кг}\cdot\text{м}}{\text{с}}, \\
    p &= \abs{p_1 - p_2} = \abs{m(v_1 - v_2)}= 40\,\frac{\text{кг}\cdot\text{м}}{\text{с}}.
    \end{align*}
}

\tasknumber{3}%
\task{%
    Два одинаковых шарика массами по $10\,\text{кг}$ движутся во взаимно перпендикулярных направлениях.
    Скорости шариков составляют $7\,\frac{\text{м}}{\text{с}}$ и $24\,\frac{\text{м}}{\text{с}}$.
    Определите полный импульс системы.
}
\answer{%
    \begin{align*}
    p_1 &= mv_1 = 10\,\text{кг} \cdot 7\,\frac{\text{м}}{\text{с}} = 70\,\frac{\text{кг}\cdot\text{м}}{\text{с}}, \\
    p_2 &= mv_2 = 10\,\text{кг} \cdot 24\,\frac{\text{м}}{\text{с}} = 240\,\frac{\text{кг}\cdot\text{м}}{\text{с}}, \\
    p &= \sqrt{p_1^2 + p_2^2} = m\sqrt{v_1^2 + v_2^2} = 250\,\frac{\text{кг}\cdot\text{м}}{\text{с}}.
    \end{align*}
}

\tasknumber{4}%
\task{%
    Шарик массой $1\,\text{кг}$ свободно упал на горизонтальную площадку, имея в момент падения скорость $10\,\frac{\text{м}}{\text{c}}$.
    Считая удар абсолютно неупругим, определите изменение импульса шарика.
    В ответе укажите модуль полученной величины.
}
\answer{%
    \begin{align*}
    \Delta p &= 1 \cdot mv = 1 \cdot 1\,\text{кг} \cdot 10\,\frac{\text{м}}{\text{c}} = 10\,\frac{\text{кг}\cdot\text{м}}{\text{с}}.
    \end{align*}
}

\tasknumber{5}%
\task{%
    Два тела двигаются навстречу друг другу.
    Скорость каждого из них составляет $6\,\frac{\text{м}}{\text{с}}$.
    После соударения тела слиплись и продолжили движение уже со скоростью $3\,\frac{\text{м}}{\text{с}}$.
    Определите отношение масс тел (большей к меньшей).
}
\answer{%
    \begin{align*}
    &\text{ЗСИ в проекции на ось, соединяющую центры тел:} m_1 v_1 - m_2 v_1 = (m_1 + m_2) v_2 \implies \\
    &\implies \frac{m_1}{m_2} v_1 - v_1 = \cbr{\frac{m_1}{m_2} + 1} v_2 \implies
        \frac{m_1}{m_2} (v_1 - v_2) = v_2 + v_1 \implies \frac{m_1}{m_2} = \frac{v_2 + v_1}{v_1 - v_2} = 3
    \end{align*}
}

\tasknumber{6}%
\task{%
    Шар движется с некоторой скоростью и абсолютно неупруго соударяется с телом, масса которого в 12 раз больше.
    Определите во сколько раз уменьшилась скорость шара после столкновения.
}
\answer{%
    \begin{align*}
    &\text{ЗСИ в проекции на ось, соединяющую центры тел:}  \\
    &mv + 12m \cdot 0 = (m + 12m) v' \implies \\
    &v' = v\frac{m}{12m + m} = \frac{v}{12 + 1} \implies \frac{v}{v'} = 13
    \end{align*}
}

\tasknumber{7}%
\task{%
    Определите работу силы, которая обеспечит подъём тела массой $5\,\text{кг}$ на высоту $5\,\text{м}$ с постоянным ускорением $4\,\frac{\text{м}}{\text{c}^{2}}$.
    % Примите $g = 10\,\frac{\text{м}}{\text{с}^{2}}$.
}
\answer{%
    \begin{align*}
    &\text{Для подъёма:} A = Fh = (mg + ma) h = m(g+a)h, \\
    &\text{Для спуска:} A = -Fh = -(mg - ma) h = -m(g-a)h, \\
    &\text{В результате получаем:} 350\,\text{Дж}.
    \end{align*}
}

\tasknumber{8}%
\task{%
    Тело массой 3\,\text{кг} бросили с обрыва под углом $45\degrees$ к горизонту с начальной скоростью $6\,\frac{\text{м}}{\text{c}}$.
    Через некоторое время скорость тела составила $8\,\frac{\text{м}}{\text{c}}$.
    Пренебрегая сопротивлением воздуха и считая падение тела свободным, определите работу силы тяжести в течение наблюдаемого промежутка времени.
}
\answer{%
    \begin{align*}
    &\text{Изменение кинетической энергии равно работе внешних сил:} \\
    &\Delta E_k = E_k' - E_k = A_\text{тяж} \implies A_\text{тяж} = \frac{mv'^2}2 - \frac{mv_0^2}2 = 42\,\text{Дж}.
    \end{align*}
}

\tasknumber{9}%
\task{%
    Тонкий однородный лом длиной $2\,\text{м}$ и массой $10\,\text{кг}$ лежит на горизонтальной поверхности.
    \begin{itemize}
        \item Какую минимальную силу надо приложить к одному из его концов, чтобы оторвать его от этой поверхности?
        \item Какую минимальную работу надо совершить, чтобы поставить его на землю в вертикальное положение?
    \end{itemize}
    % Примите $g = 10\,\frac{\text{м}}{\text{с}^{2}}$.
}
\answer{%
    $F = \frac{mg}2 \approx 100\,\text{Н}, A = mg\frac l2 = 100\,\text{Дж}$
}

\tasknumber{10}%
\task{%
    Для того, чтобы разогать тело из состояния покоя до скорости $v$ с постоянным ускорением,
    требуется совершить работу $20\,\text{Дж}$.
    Какую работу нужно совершить, чтобы увеличить скорость этого тела от $v$ до $5v$?
}
\answer{%
    \begin{align*}
    &\text{Изменение кинетической энергии равно работе внешних сил:} \\
    &A_1 = \frac{mv^2}2 - \frac{m \cdot 0^2}2 = \frac{mv^2}2, A_2 = \frac{m\sqr{5v}}2 - \frac{mv^2}2 \implies  \\
    &\implies A_2 = \frac{mv^2}2 \cbr{5^2 - 1} = A_1 \cdot \cbr{5^2 - 1} = 480\,\text{Дж}.
    \end{align*}
}

\variantsplitter

\addpersonalvariant{Алина Полканова}

\tasknumber{1}%
\task{%
    Шарики массами $1\,\text{кг}$ и $4\,\text{кг}$ движутся параллельно друг другу в одном направлении
    со скоростями $5\,\frac{\text{м}}{\text{с}}$ и $3\,\frac{\text{м}}{\text{с}}$ соответственно.
    Определите общий импульс шариков.
}
\answer{%
    \begin{align*}
    p_1 &= m_1v_1 = 1\,\text{кг} \cdot 5\,\frac{\text{м}}{\text{с}} = 5\,\frac{\text{кг}\cdot\text{м}}{\text{с}}, \\
    p_2 &= m_2v_2 = 4\,\text{кг} \cdot 3\,\frac{\text{м}}{\text{с}} = 12\,\frac{\text{кг}\cdot\text{м}}{\text{с}}, \\
    p &= p_1 + p_2 = m_1v_1 + m_2v_2 = 17\,\frac{\text{кг}\cdot\text{м}}{\text{с}}.
    \end{align*}
}

\tasknumber{2}%
\task{%
    Два шарика, масса каждого из которых составляет $2\,\text{кг}$, движутся навстречу друг другу.
    Скорость одного из них $1\,\frac{\text{м}}{\text{с}}$, а другого~--- $8\,\frac{\text{м}}{\text{с}}$.
    Определите общий импульс шариков.
}
\answer{%
    \begin{align*}
    p_1 &= mv_1 = 2\,\text{кг} \cdot 1\,\frac{\text{м}}{\text{с}} = 2\,\frac{\text{кг}\cdot\text{м}}{\text{с}}, \\
    p_2 &= mv_2 = 2\,\text{кг} \cdot 8\,\frac{\text{м}}{\text{с}} = 16\,\frac{\text{кг}\cdot\text{м}}{\text{с}}, \\
    p &= \abs{p_1 - p_2} = \abs{m(v_1 - v_2)}= 14\,\frac{\text{кг}\cdot\text{м}}{\text{с}}.
    \end{align*}
}

\tasknumber{3}%
\task{%
    Два одинаковых шарика массами по $5\,\text{кг}$ движутся во взаимно перпендикулярных направлениях.
    Скорости шариков составляют $3\,\frac{\text{м}}{\text{с}}$ и $4\,\frac{\text{м}}{\text{с}}$.
    Определите полный импульс системы.
}
\answer{%
    \begin{align*}
    p_1 &= mv_1 = 5\,\text{кг} \cdot 3\,\frac{\text{м}}{\text{с}} = 15\,\frac{\text{кг}\cdot\text{м}}{\text{с}}, \\
    p_2 &= mv_2 = 5\,\text{кг} \cdot 4\,\frac{\text{м}}{\text{с}} = 20\,\frac{\text{кг}\cdot\text{м}}{\text{с}}, \\
    p &= \sqrt{p_1^2 + p_2^2} = m\sqrt{v_1^2 + v_2^2} = 25\,\frac{\text{кг}\cdot\text{м}}{\text{с}}.
    \end{align*}
}

\tasknumber{4}%
\task{%
    Шарик массой $2\,\text{кг}$ свободно упал на горизонтальную площадку, имея в момент падения скорость $10\,\frac{\text{м}}{\text{c}}$.
    Считая удар абсолютно неупругим, определите изменение импульса шарика.
    В ответе укажите модуль полученной величины.
}
\answer{%
    \begin{align*}
    \Delta p &= 1 \cdot mv = 1 \cdot 2\,\text{кг} \cdot 10\,\frac{\text{м}}{\text{c}} = 20\,\frac{\text{кг}\cdot\text{м}}{\text{с}}.
    \end{align*}
}

\tasknumber{5}%
\task{%
    Два тела двигаются навстречу друг другу.
    Скорость каждого из них составляет $6\,\frac{\text{м}}{\text{с}}$.
    После соударения тела слиплись и продолжили движение уже со скоростью $4\,\frac{\text{м}}{\text{с}}$.
    Определите отношение масс тел (большей к меньшей).
}
\answer{%
    \begin{align*}
    &\text{ЗСИ в проекции на ось, соединяющую центры тел:} m_1 v_1 - m_2 v_1 = (m_1 + m_2) v_2 \implies \\
    &\implies \frac{m_1}{m_2} v_1 - v_1 = \cbr{\frac{m_1}{m_2} + 1} v_2 \implies
        \frac{m_1}{m_2} (v_1 - v_2) = v_2 + v_1 \implies \frac{m_1}{m_2} = \frac{v_2 + v_1}{v_1 - v_2} = 5
    \end{align*}
}

\tasknumber{6}%
\task{%
    Шар движется с некоторой скоростью и абсолютно неупруго соударяется с телом, масса которого в 5 раз больше.
    Определите во сколько раз уменьшилась скорость шара после столкновения.
}
\answer{%
    \begin{align*}
    &\text{ЗСИ в проекции на ось, соединяющую центры тел:}  \\
    &mv + 5m \cdot 0 = (m + 5m) v' \implies \\
    &v' = v\frac{m}{5m + m} = \frac{v}{5 + 1} \implies \frac{v}{v'} = 6
    \end{align*}
}

\tasknumber{7}%
\task{%
    Определите работу силы, которая обеспечит подъём тела массой $3\,\text{кг}$ на высоту $2\,\text{м}$ с постоянным ускорением $2\,\frac{\text{м}}{\text{c}^{2}}$.
    % Примите $g = 10\,\frac{\text{м}}{\text{с}^{2}}$.
}
\answer{%
    \begin{align*}
    &\text{Для подъёма:} A = Fh = (mg + ma) h = m(g+a)h, \\
    &\text{Для спуска:} A = -Fh = -(mg - ma) h = -m(g-a)h, \\
    &\text{В результате получаем:} 72\,\text{Дж}.
    \end{align*}
}

\tasknumber{8}%
\task{%
    Тело массой 2\,\text{кг} бросили с обрыва вертикально вверх с начальной скоростью $6\,\frac{\text{м}}{\text{c}}$.
    Через некоторое время скорость тела составила $12\,\frac{\text{м}}{\text{c}}$.
    Пренебрегая сопротивлением воздуха и считая падение тела свободным, определите работу силы тяжести в течение наблюдаемого промежутка времени.
}
\answer{%
    \begin{align*}
    &\text{Изменение кинетической энергии равно работе внешних сил:} \\
    &\Delta E_k = E_k' - E_k = A_\text{тяж} \implies A_\text{тяж} = \frac{mv'^2}2 - \frac{mv_0^2}2 = 108\,\text{Дж}.
    \end{align*}
}

\tasknumber{9}%
\task{%
    Тонкий однородный лом длиной $2\,\text{м}$ и массой $10\,\text{кг}$ лежит на горизонтальной поверхности.
    \begin{itemize}
        \item Какую минимальную силу надо приложить к одному из его концов, чтобы оторвать его от этой поверхности?
        \item Какую минимальную работу надо совершить, чтобы поставить его на землю в вертикальное положение?
    \end{itemize}
    % Примите $g = 10\,\frac{\text{м}}{\text{с}^{2}}$.
}
\answer{%
    $F = \frac{mg}2 \approx 100\,\text{Н}, A = mg\frac l2 = 100\,\text{Дж}$
}

\tasknumber{10}%
\task{%
    Для того, чтобы разогать тело из состояния покоя до скорости $v$ с постоянным ускорением,
    требуется совершить работу $10\,\text{Дж}$.
    Какую работу нужно совершить, чтобы увеличить скорость этого тела от $v$ до $5v$?
}
\answer{%
    \begin{align*}
    &\text{Изменение кинетической энергии равно работе внешних сил:} \\
    &A_1 = \frac{mv^2}2 - \frac{m \cdot 0^2}2 = \frac{mv^2}2, A_2 = \frac{m\sqr{5v}}2 - \frac{mv^2}2 \implies  \\
    &\implies A_2 = \frac{mv^2}2 \cbr{5^2 - 1} = A_1 \cdot \cbr{5^2 - 1} = 240\,\text{Дж}.
    \end{align*}
}

\variantsplitter

\addpersonalvariant{Сергей Пономарёв}

\tasknumber{1}%
\task{%
    Шарики массами $2\,\text{кг}$ и $1\,\text{кг}$ движутся параллельно друг другу в одном направлении
    со скоростями $4\,\frac{\text{м}}{\text{с}}$ и $6\,\frac{\text{м}}{\text{с}}$ соответственно.
    Определите общий импульс шариков.
}
\answer{%
    \begin{align*}
    p_1 &= m_1v_1 = 2\,\text{кг} \cdot 4\,\frac{\text{м}}{\text{с}} = 8\,\frac{\text{кг}\cdot\text{м}}{\text{с}}, \\
    p_2 &= m_2v_2 = 1\,\text{кг} \cdot 6\,\frac{\text{м}}{\text{с}} = 6\,\frac{\text{кг}\cdot\text{м}}{\text{с}}, \\
    p &= p_1 + p_2 = m_1v_1 + m_2v_2 = 14\,\frac{\text{кг}\cdot\text{м}}{\text{с}}.
    \end{align*}
}

\tasknumber{2}%
\task{%
    Два шарика, масса каждого из которых составляет $2\,\text{кг}$, движутся навстречу друг другу.
    Скорость одного из них $10\,\frac{\text{м}}{\text{с}}$, а другого~--- $3\,\frac{\text{м}}{\text{с}}$.
    Определите общий импульс шариков.
}
\answer{%
    \begin{align*}
    p_1 &= mv_1 = 2\,\text{кг} \cdot 10\,\frac{\text{м}}{\text{с}} = 20\,\frac{\text{кг}\cdot\text{м}}{\text{с}}, \\
    p_2 &= mv_2 = 2\,\text{кг} \cdot 3\,\frac{\text{м}}{\text{с}} = 6\,\frac{\text{кг}\cdot\text{м}}{\text{с}}, \\
    p &= \abs{p_1 - p_2} = \abs{m(v_1 - v_2)}= 14\,\frac{\text{кг}\cdot\text{м}}{\text{с}}.
    \end{align*}
}

\tasknumber{3}%
\task{%
    Два одинаковых шарика массами по $5\,\text{кг}$ движутся во взаимно перпендикулярных направлениях.
    Скорости шариков составляют $5\,\frac{\text{м}}{\text{с}}$ и $12\,\frac{\text{м}}{\text{с}}$.
    Определите полный импульс системы.
}
\answer{%
    \begin{align*}
    p_1 &= mv_1 = 5\,\text{кг} \cdot 5\,\frac{\text{м}}{\text{с}} = 25\,\frac{\text{кг}\cdot\text{м}}{\text{с}}, \\
    p_2 &= mv_2 = 5\,\text{кг} \cdot 12\,\frac{\text{м}}{\text{с}} = 60\,\frac{\text{кг}\cdot\text{м}}{\text{с}}, \\
    p &= \sqrt{p_1^2 + p_2^2} = m\sqrt{v_1^2 + v_2^2} = 65\,\frac{\text{кг}\cdot\text{м}}{\text{с}}.
    \end{align*}
}

\tasknumber{4}%
\task{%
    Шарик массой $4\,\text{кг}$ свободно упал на горизонтальную площадку, имея в момент падения скорость $15\,\frac{\text{м}}{\text{c}}$.
    Считая удар абсолютно неупругим, определите изменение импульса шарика.
    В ответе укажите модуль полученной величины.
}
\answer{%
    \begin{align*}
    \Delta p &= 1 \cdot mv = 1 \cdot 4\,\text{кг} \cdot 15\,\frac{\text{м}}{\text{c}} = 60\,\frac{\text{кг}\cdot\text{м}}{\text{с}}.
    \end{align*}
}

\tasknumber{5}%
\task{%
    Два тела двигаются навстречу друг другу.
    Скорость каждого из них составляет $4\,\frac{\text{м}}{\text{с}}$.
    После соударения тела слиплись и продолжили движение уже со скоростью $3\,\frac{\text{м}}{\text{с}}$.
    Определите отношение масс тел (большей к меньшей).
}
\answer{%
    \begin{align*}
    &\text{ЗСИ в проекции на ось, соединяющую центры тел:} m_1 v_1 - m_2 v_1 = (m_1 + m_2) v_2 \implies \\
    &\implies \frac{m_1}{m_2} v_1 - v_1 = \cbr{\frac{m_1}{m_2} + 1} v_2 \implies
        \frac{m_1}{m_2} (v_1 - v_2) = v_2 + v_1 \implies \frac{m_1}{m_2} = \frac{v_2 + v_1}{v_1 - v_2} = 7
    \end{align*}
}

\tasknumber{6}%
\task{%
    Шар движется с некоторой скоростью и абсолютно неупруго соударяется с телом, масса которого в 12 раз больше.
    Определите во сколько раз уменьшилась скорость шара после столкновения.
}
\answer{%
    \begin{align*}
    &\text{ЗСИ в проекции на ось, соединяющую центры тел:}  \\
    &mv + 12m \cdot 0 = (m + 12m) v' \implies \\
    &v' = v\frac{m}{12m + m} = \frac{v}{12 + 1} \implies \frac{v}{v'} = 13
    \end{align*}
}

\tasknumber{7}%
\task{%
    Определите работу силы, которая обеспечит подъём тела массой $5\,\text{кг}$ на высоту $10\,\text{м}$ с постоянным ускорением $2\,\frac{\text{м}}{\text{c}^{2}}$.
    % Примите $g = 10\,\frac{\text{м}}{\text{с}^{2}}$.
}
\answer{%
    \begin{align*}
    &\text{Для подъёма:} A = Fh = (mg + ma) h = m(g+a)h, \\
    &\text{Для спуска:} A = -Fh = -(mg - ma) h = -m(g-a)h, \\
    &\text{В результате получаем:} 600\,\text{Дж}.
    \end{align*}
}

\tasknumber{8}%
\task{%
    Тело массой 3\,\text{кг} бросили с обрыва под углом $45\degrees$ к горизонту с начальной скоростью $2\,\frac{\text{м}}{\text{c}}$.
    Через некоторое время скорость тела составила $12\,\frac{\text{м}}{\text{c}}$.
    Пренебрегая сопротивлением воздуха и считая падение тела свободным, определите работу силы тяжести в течение наблюдаемого промежутка времени.
}
\answer{%
    \begin{align*}
    &\text{Изменение кинетической энергии равно работе внешних сил:} \\
    &\Delta E_k = E_k' - E_k = A_\text{тяж} \implies A_\text{тяж} = \frac{mv'^2}2 - \frac{mv_0^2}2 = 210\,\text{Дж}.
    \end{align*}
}

\tasknumber{9}%
\task{%
    Тонкий однородный шест длиной $2\,\text{м}$ и массой $20\,\text{кг}$ лежит на горизонтальной поверхности.
    \begin{itemize}
        \item Какую минимальную силу надо приложить к одному из его концов, чтобы оторвать его от этой поверхности?
        \item Какую минимальную работу надо совершить, чтобы поставить его на землю в вертикальное положение?
    \end{itemize}
    % Примите $g = 10\,\frac{\text{м}}{\text{с}^{2}}$.
}
\answer{%
    $F = \frac{mg}2 \approx 200\,\text{Н}, A = mg\frac l2 = 200\,\text{Дж}$
}

\tasknumber{10}%
\task{%
    Для того, чтобы разогать тело из состояния покоя до скорости $v$ с постоянным ускорением,
    требуется совершить работу $100\,\text{Дж}$.
    Какую работу нужно совершить, чтобы увеличить скорость этого тела от $v$ до $4v$?
}
\answer{%
    \begin{align*}
    &\text{Изменение кинетической энергии равно работе внешних сил:} \\
    &A_1 = \frac{mv^2}2 - \frac{m \cdot 0^2}2 = \frac{mv^2}2, A_2 = \frac{m\sqr{4v}}2 - \frac{mv^2}2 \implies  \\
    &\implies A_2 = \frac{mv^2}2 \cbr{4^2 - 1} = A_1 \cdot \cbr{4^2 - 1} = 1500\,\text{Дж}.
    \end{align*}
}

\variantsplitter

\addpersonalvariant{Егор Свистушкин}

\tasknumber{1}%
\task{%
    Шарики массами $4\,\text{кг}$ и $2\,\text{кг}$ движутся параллельно друг другу в одном направлении
    со скоростями $10\,\frac{\text{м}}{\text{с}}$ и $8\,\frac{\text{м}}{\text{с}}$ соответственно.
    Определите общий импульс шариков.
}
\answer{%
    \begin{align*}
    p_1 &= m_1v_1 = 4\,\text{кг} \cdot 10\,\frac{\text{м}}{\text{с}} = 40\,\frac{\text{кг}\cdot\text{м}}{\text{с}}, \\
    p_2 &= m_2v_2 = 2\,\text{кг} \cdot 8\,\frac{\text{м}}{\text{с}} = 16\,\frac{\text{кг}\cdot\text{м}}{\text{с}}, \\
    p &= p_1 + p_2 = m_1v_1 + m_2v_2 = 56\,\frac{\text{кг}\cdot\text{м}}{\text{с}}.
    \end{align*}
}

\tasknumber{2}%
\task{%
    Два шарика, масса каждого из которых составляет $5\,\text{кг}$, движутся навстречу друг другу.
    Скорость одного из них $2\,\frac{\text{м}}{\text{с}}$, а другого~--- $8\,\frac{\text{м}}{\text{с}}$.
    Определите общий импульс шариков.
}
\answer{%
    \begin{align*}
    p_1 &= mv_1 = 5\,\text{кг} \cdot 2\,\frac{\text{м}}{\text{с}} = 10\,\frac{\text{кг}\cdot\text{м}}{\text{с}}, \\
    p_2 &= mv_2 = 5\,\text{кг} \cdot 8\,\frac{\text{м}}{\text{с}} = 40\,\frac{\text{кг}\cdot\text{м}}{\text{с}}, \\
    p &= \abs{p_1 - p_2} = \abs{m(v_1 - v_2)}= 30\,\frac{\text{кг}\cdot\text{м}}{\text{с}}.
    \end{align*}
}

\tasknumber{3}%
\task{%
    Два одинаковых шарика массами по $10\,\text{кг}$ движутся во взаимно перпендикулярных направлениях.
    Скорости шариков составляют $5\,\frac{\text{м}}{\text{с}}$ и $12\,\frac{\text{м}}{\text{с}}$.
    Определите полный импульс системы.
}
\answer{%
    \begin{align*}
    p_1 &= mv_1 = 10\,\text{кг} \cdot 5\,\frac{\text{м}}{\text{с}} = 50\,\frac{\text{кг}\cdot\text{м}}{\text{с}}, \\
    p_2 &= mv_2 = 10\,\text{кг} \cdot 12\,\frac{\text{м}}{\text{с}} = 120\,\frac{\text{кг}\cdot\text{м}}{\text{с}}, \\
    p &= \sqrt{p_1^2 + p_2^2} = m\sqrt{v_1^2 + v_2^2} = 130\,\frac{\text{кг}\cdot\text{м}}{\text{с}}.
    \end{align*}
}

\tasknumber{4}%
\task{%
    Шарик массой $1\,\text{кг}$ свободно упал на горизонтальную площадку, имея в момент падения скорость $15\,\frac{\text{м}}{\text{c}}$.
    Считая удар абсолютно упругим, определите изменение импульса шарика.
    В ответе укажите модуль полученной величины.
}
\answer{%
    \begin{align*}
    \Delta p &= 2 \cdot mv = 2 \cdot 1\,\text{кг} \cdot 15\,\frac{\text{м}}{\text{c}} = 30\,\frac{\text{кг}\cdot\text{м}}{\text{с}}.
    \end{align*}
}

\tasknumber{5}%
\task{%
    Два тела двигаются навстречу друг другу.
    Скорость каждого из них составляет $4\,\frac{\text{м}}{\text{с}}$.
    После соударения тела слиплись и продолжили движение уже со скоростью $3\,\frac{\text{м}}{\text{с}}$.
    Определите отношение масс тел (большей к меньшей).
}
\answer{%
    \begin{align*}
    &\text{ЗСИ в проекции на ось, соединяющую центры тел:} m_1 v_1 - m_2 v_1 = (m_1 + m_2) v_2 \implies \\
    &\implies \frac{m_1}{m_2} v_1 - v_1 = \cbr{\frac{m_1}{m_2} + 1} v_2 \implies
        \frac{m_1}{m_2} (v_1 - v_2) = v_2 + v_1 \implies \frac{m_1}{m_2} = \frac{v_2 + v_1}{v_1 - v_2} = 7
    \end{align*}
}

\tasknumber{6}%
\task{%
    Шар движется с некоторой скоростью и абсолютно неупруго соударяется с телом, масса которого в 5 раз больше.
    Определите во сколько раз уменьшилась скорость шара после столкновения.
}
\answer{%
    \begin{align*}
    &\text{ЗСИ в проекции на ось, соединяющую центры тел:}  \\
    &mv + 5m \cdot 0 = (m + 5m) v' \implies \\
    &v' = v\frac{m}{5m + m} = \frac{v}{5 + 1} \implies \frac{v}{v'} = 6
    \end{align*}
}

\tasknumber{7}%
\task{%
    Определите работу силы, которая обеспечит спуск тела массой $2\,\text{кг}$ на высоту $2\,\text{м}$ с постоянным ускорением $4\,\frac{\text{м}}{\text{c}^{2}}$.
    % Примите $g = 10\,\frac{\text{м}}{\text{с}^{2}}$.
}
\answer{%
    \begin{align*}
    &\text{Для подъёма:} A = Fh = (mg + ma) h = m(g+a)h, \\
    &\text{Для спуска:} A = -Fh = -(mg - ma) h = -m(g-a)h, \\
    &\text{В результате получаем:} -24\,\text{Дж}.
    \end{align*}
}

\tasknumber{8}%
\task{%
    Тело массой 2\,\text{кг} бросили с обрыва под углом $45\degrees$ к горизонту с начальной скоростью $2\,\frac{\text{м}}{\text{c}}$.
    Через некоторое время скорость тела составила $12\,\frac{\text{м}}{\text{c}}$.
    Пренебрегая сопротивлением воздуха и считая падение тела свободным, определите работу силы тяжести в течение наблюдаемого промежутка времени.
}
\answer{%
    \begin{align*}
    &\text{Изменение кинетической энергии равно работе внешних сил:} \\
    &\Delta E_k = E_k' - E_k = A_\text{тяж} \implies A_\text{тяж} = \frac{mv'^2}2 - \frac{mv_0^2}2 = 140\,\text{Дж}.
    \end{align*}
}

\tasknumber{9}%
\task{%
    Тонкий однородный шест длиной $2\,\text{м}$ и массой $20\,\text{кг}$ лежит на горизонтальной поверхности.
    \begin{itemize}
        \item Какую минимальную силу надо приложить к одному из его концов, чтобы оторвать его от этой поверхности?
        \item Какую минимальную работу надо совершить, чтобы поставить его на землю в вертикальное положение?
    \end{itemize}
    % Примите $g = 10\,\frac{\text{м}}{\text{с}^{2}}$.
}
\answer{%
    $F = \frac{mg}2 \approx 200\,\text{Н}, A = mg\frac l2 = 200\,\text{Дж}$
}

\tasknumber{10}%
\task{%
    Для того, чтобы разогать тело из состояния покоя до скорости $v$ с постоянным ускорением,
    требуется совершить работу $20\,\text{Дж}$.
    Какую работу нужно совершить, чтобы увеличить скорость этого тела от $v$ до $5v$?
}
\answer{%
    \begin{align*}
    &\text{Изменение кинетической энергии равно работе внешних сил:} \\
    &A_1 = \frac{mv^2}2 - \frac{m \cdot 0^2}2 = \frac{mv^2}2, A_2 = \frac{m\sqr{5v}}2 - \frac{mv^2}2 \implies  \\
    &\implies A_2 = \frac{mv^2}2 \cbr{5^2 - 1} = A_1 \cdot \cbr{5^2 - 1} = 480\,\text{Дж}.
    \end{align*}
}

\variantsplitter

\addpersonalvariant{Дмитрий Соколов}

\tasknumber{1}%
\task{%
    Шарики массами $1\,\text{кг}$ и $2\,\text{кг}$ движутся параллельно друг другу в одном направлении
    со скоростями $5\,\frac{\text{м}}{\text{с}}$ и $3\,\frac{\text{м}}{\text{с}}$ соответственно.
    Определите общий импульс шариков.
}
\answer{%
    \begin{align*}
    p_1 &= m_1v_1 = 1\,\text{кг} \cdot 5\,\frac{\text{м}}{\text{с}} = 5\,\frac{\text{кг}\cdot\text{м}}{\text{с}}, \\
    p_2 &= m_2v_2 = 2\,\text{кг} \cdot 3\,\frac{\text{м}}{\text{с}} = 6\,\frac{\text{кг}\cdot\text{м}}{\text{с}}, \\
    p &= p_1 + p_2 = m_1v_1 + m_2v_2 = 11\,\frac{\text{кг}\cdot\text{м}}{\text{с}}.
    \end{align*}
}

\tasknumber{2}%
\task{%
    Два шарика, масса каждого из которых составляет $10\,\text{кг}$, движутся навстречу друг другу.
    Скорость одного из них $2\,\frac{\text{м}}{\text{с}}$, а другого~--- $8\,\frac{\text{м}}{\text{с}}$.
    Определите общий импульс шариков.
}
\answer{%
    \begin{align*}
    p_1 &= mv_1 = 10\,\text{кг} \cdot 2\,\frac{\text{м}}{\text{с}} = 20\,\frac{\text{кг}\cdot\text{м}}{\text{с}}, \\
    p_2 &= mv_2 = 10\,\text{кг} \cdot 8\,\frac{\text{м}}{\text{с}} = 80\,\frac{\text{кг}\cdot\text{м}}{\text{с}}, \\
    p &= \abs{p_1 - p_2} = \abs{m(v_1 - v_2)}= 60\,\frac{\text{кг}\cdot\text{м}}{\text{с}}.
    \end{align*}
}

\tasknumber{3}%
\task{%
    Два одинаковых шарика массами по $5\,\text{кг}$ движутся во взаимно перпендикулярных направлениях.
    Скорости шариков составляют $7\,\frac{\text{м}}{\text{с}}$ и $24\,\frac{\text{м}}{\text{с}}$.
    Определите полный импульс системы.
}
\answer{%
    \begin{align*}
    p_1 &= mv_1 = 5\,\text{кг} \cdot 7\,\frac{\text{м}}{\text{с}} = 35\,\frac{\text{кг}\cdot\text{м}}{\text{с}}, \\
    p_2 &= mv_2 = 5\,\text{кг} \cdot 24\,\frac{\text{м}}{\text{с}} = 120\,\frac{\text{кг}\cdot\text{м}}{\text{с}}, \\
    p &= \sqrt{p_1^2 + p_2^2} = m\sqrt{v_1^2 + v_2^2} = 125\,\frac{\text{кг}\cdot\text{м}}{\text{с}}.
    \end{align*}
}

\tasknumber{4}%
\task{%
    Шарик массой $4\,\text{кг}$ свободно упал на горизонтальную площадку, имея в момент падения скорость $25\,\frac{\text{м}}{\text{c}}$.
    Считая удар абсолютно упругим, определите изменение импульса шарика.
    В ответе укажите модуль полученной величины.
}
\answer{%
    \begin{align*}
    \Delta p &= 2 \cdot mv = 2 \cdot 4\,\text{кг} \cdot 25\,\frac{\text{м}}{\text{c}} = 200\,\frac{\text{кг}\cdot\text{м}}{\text{с}}.
    \end{align*}
}

\tasknumber{5}%
\task{%
    Два тела двигаются навстречу друг другу.
    Скорость каждого из них составляет $6\,\frac{\text{м}}{\text{с}}$.
    После соударения тела слиплись и продолжили движение уже со скоростью $4\,\frac{\text{м}}{\text{с}}$.
    Определите отношение масс тел (большей к меньшей).
}
\answer{%
    \begin{align*}
    &\text{ЗСИ в проекции на ось, соединяющую центры тел:} m_1 v_1 - m_2 v_1 = (m_1 + m_2) v_2 \implies \\
    &\implies \frac{m_1}{m_2} v_1 - v_1 = \cbr{\frac{m_1}{m_2} + 1} v_2 \implies
        \frac{m_1}{m_2} (v_1 - v_2) = v_2 + v_1 \implies \frac{m_1}{m_2} = \frac{v_2 + v_1}{v_1 - v_2} = 5
    \end{align*}
}

\tasknumber{6}%
\task{%
    Шар движется с некоторой скоростью и абсолютно неупруго соударяется с телом, масса которого в 11 раз больше.
    Определите во сколько раз уменьшилась скорость шара после столкновения.
}
\answer{%
    \begin{align*}
    &\text{ЗСИ в проекции на ось, соединяющую центры тел:}  \\
    &mv + 11m \cdot 0 = (m + 11m) v' \implies \\
    &v' = v\frac{m}{11m + m} = \frac{v}{11 + 1} \implies \frac{v}{v'} = 12
    \end{align*}
}

\tasknumber{7}%
\task{%
    Определите работу силы, которая обеспечит подъём тела массой $5\,\text{кг}$ на высоту $5\,\text{м}$ с постоянным ускорением $6\,\frac{\text{м}}{\text{c}^{2}}$.
    % Примите $g = 10\,\frac{\text{м}}{\text{с}^{2}}$.
}
\answer{%
    \begin{align*}
    &\text{Для подъёма:} A = Fh = (mg + ma) h = m(g+a)h, \\
    &\text{Для спуска:} A = -Fh = -(mg - ma) h = -m(g-a)h, \\
    &\text{В результате получаем:} 400\,\text{Дж}.
    \end{align*}
}

\tasknumber{8}%
\task{%
    Тело массой 2\,\text{кг} бросили с обрыва под углом $45\degrees$ к горизонту с начальной скоростью $6\,\frac{\text{м}}{\text{c}}$.
    Через некоторое время скорость тела составила $12\,\frac{\text{м}}{\text{c}}$.
    Пренебрегая сопротивлением воздуха и считая падение тела свободным, определите работу силы тяжести в течение наблюдаемого промежутка времени.
}
\answer{%
    \begin{align*}
    &\text{Изменение кинетической энергии равно работе внешних сил:} \\
    &\Delta E_k = E_k' - E_k = A_\text{тяж} \implies A_\text{тяж} = \frac{mv'^2}2 - \frac{mv_0^2}2 = 108\,\text{Дж}.
    \end{align*}
}

\tasknumber{9}%
\task{%
    Тонкий однородный лом длиной $3\,\text{м}$ и массой $10\,\text{кг}$ лежит на горизонтальной поверхности.
    \begin{itemize}
        \item Какую минимальную силу надо приложить к одному из его концов, чтобы оторвать его от этой поверхности?
        \item Какую минимальную работу надо совершить, чтобы поставить его на землю в вертикальное положение?
    \end{itemize}
    % Примите $g = 10\,\frac{\text{м}}{\text{с}^{2}}$.
}
\answer{%
    $F = \frac{mg}2 \approx 100\,\text{Н}, A = mg\frac l2 = 150\,\text{Дж}$
}

\tasknumber{10}%
\task{%
    Для того, чтобы разогать тело из состояния покоя до скорости $v$ с постоянным ускорением,
    требуется совершить работу $10\,\text{Дж}$.
    Какую работу нужно совершить, чтобы увеличить скорость этого тела от $v$ до $2v$?
}
\answer{%
    \begin{align*}
    &\text{Изменение кинетической энергии равно работе внешних сил:} \\
    &A_1 = \frac{mv^2}2 - \frac{m \cdot 0^2}2 = \frac{mv^2}2, A_2 = \frac{m\sqr{2v}}2 - \frac{mv^2}2 \implies  \\
    &\implies A_2 = \frac{mv^2}2 \cbr{2^2 - 1} = A_1 \cdot \cbr{2^2 - 1} = 30\,\text{Дж}.
    \end{align*}
}

\variantsplitter

\addpersonalvariant{Арсений Трофимов}

\tasknumber{1}%
\task{%
    Шарики массами $2\,\text{кг}$ и $1\,\text{кг}$ движутся параллельно друг другу в одном направлении
    со скоростями $2\,\frac{\text{м}}{\text{с}}$ и $3\,\frac{\text{м}}{\text{с}}$ соответственно.
    Определите общий импульс шариков.
}
\answer{%
    \begin{align*}
    p_1 &= m_1v_1 = 2\,\text{кг} \cdot 2\,\frac{\text{м}}{\text{с}} = 4\,\frac{\text{кг}\cdot\text{м}}{\text{с}}, \\
    p_2 &= m_2v_2 = 1\,\text{кг} \cdot 3\,\frac{\text{м}}{\text{с}} = 3\,\frac{\text{кг}\cdot\text{м}}{\text{с}}, \\
    p &= p_1 + p_2 = m_1v_1 + m_2v_2 = 7\,\frac{\text{кг}\cdot\text{м}}{\text{с}}.
    \end{align*}
}

\tasknumber{2}%
\task{%
    Два шарика, масса каждого из которых составляет $2\,\text{кг}$, движутся навстречу друг другу.
    Скорость одного из них $10\,\frac{\text{м}}{\text{с}}$, а другого~--- $8\,\frac{\text{м}}{\text{с}}$.
    Определите общий импульс шариков.
}
\answer{%
    \begin{align*}
    p_1 &= mv_1 = 2\,\text{кг} \cdot 10\,\frac{\text{м}}{\text{с}} = 20\,\frac{\text{кг}\cdot\text{м}}{\text{с}}, \\
    p_2 &= mv_2 = 2\,\text{кг} \cdot 8\,\frac{\text{м}}{\text{с}} = 16\,\frac{\text{кг}\cdot\text{м}}{\text{с}}, \\
    p &= \abs{p_1 - p_2} = \abs{m(v_1 - v_2)}= 4\,\frac{\text{кг}\cdot\text{м}}{\text{с}}.
    \end{align*}
}

\tasknumber{3}%
\task{%
    Два одинаковых шарика массами по $10\,\text{кг}$ движутся во взаимно перпендикулярных направлениях.
    Скорости шариков составляют $5\,\frac{\text{м}}{\text{с}}$ и $12\,\frac{\text{м}}{\text{с}}$.
    Определите полный импульс системы.
}
\answer{%
    \begin{align*}
    p_1 &= mv_1 = 10\,\text{кг} \cdot 5\,\frac{\text{м}}{\text{с}} = 50\,\frac{\text{кг}\cdot\text{м}}{\text{с}}, \\
    p_2 &= mv_2 = 10\,\text{кг} \cdot 12\,\frac{\text{м}}{\text{с}} = 120\,\frac{\text{кг}\cdot\text{м}}{\text{с}}, \\
    p &= \sqrt{p_1^2 + p_2^2} = m\sqrt{v_1^2 + v_2^2} = 130\,\frac{\text{кг}\cdot\text{м}}{\text{с}}.
    \end{align*}
}

\tasknumber{4}%
\task{%
    Шарик массой $4\,\text{кг}$ свободно упал на горизонтальную площадку, имея в момент падения скорость $15\,\frac{\text{м}}{\text{c}}$.
    Считая удар абсолютно упругим, определите изменение импульса шарика.
    В ответе укажите модуль полученной величины.
}
\answer{%
    \begin{align*}
    \Delta p &= 2 \cdot mv = 2 \cdot 4\,\text{кг} \cdot 15\,\frac{\text{м}}{\text{c}} = 120\,\frac{\text{кг}\cdot\text{м}}{\text{с}}.
    \end{align*}
}

\tasknumber{5}%
\task{%
    Два тела двигаются навстречу друг другу.
    Скорость каждого из них составляет $6\,\frac{\text{м}}{\text{с}}$.
    После соударения тела слиплись и продолжили движение уже со скоростью $3\,\frac{\text{м}}{\text{с}}$.
    Определите отношение масс тел (большей к меньшей).
}
\answer{%
    \begin{align*}
    &\text{ЗСИ в проекции на ось, соединяющую центры тел:} m_1 v_1 - m_2 v_1 = (m_1 + m_2) v_2 \implies \\
    &\implies \frac{m_1}{m_2} v_1 - v_1 = \cbr{\frac{m_1}{m_2} + 1} v_2 \implies
        \frac{m_1}{m_2} (v_1 - v_2) = v_2 + v_1 \implies \frac{m_1}{m_2} = \frac{v_2 + v_1}{v_1 - v_2} = 3
    \end{align*}
}

\tasknumber{6}%
\task{%
    Шар движется с некоторой скоростью и абсолютно неупруго соударяется с телом, масса которого в 13 раз больше.
    Определите во сколько раз уменьшилась скорость шара после столкновения.
}
\answer{%
    \begin{align*}
    &\text{ЗСИ в проекции на ось, соединяющую центры тел:}  \\
    &mv + 13m \cdot 0 = (m + 13m) v' \implies \\
    &v' = v\frac{m}{13m + m} = \frac{v}{13 + 1} \implies \frac{v}{v'} = 14
    \end{align*}
}

\tasknumber{7}%
\task{%
    Определите работу силы, которая обеспечит спуск тела массой $5\,\text{кг}$ на высоту $2\,\text{м}$ с постоянным ускорением $6\,\frac{\text{м}}{\text{c}^{2}}$.
    % Примите $g = 10\,\frac{\text{м}}{\text{с}^{2}}$.
}
\answer{%
    \begin{align*}
    &\text{Для подъёма:} A = Fh = (mg + ma) h = m(g+a)h, \\
    &\text{Для спуска:} A = -Fh = -(mg - ma) h = -m(g-a)h, \\
    &\text{В результате получаем:} -40\,\text{Дж}.
    \end{align*}
}

\tasknumber{8}%
\task{%
    Тело массой 3\,\text{кг} бросили с обрыва под углом $45\degrees$ к горизонту с начальной скоростью $4\,\frac{\text{м}}{\text{c}}$.
    Через некоторое время скорость тела составила $10\,\frac{\text{м}}{\text{c}}$.
    Пренебрегая сопротивлением воздуха и считая падение тела свободным, определите работу силы тяжести в течение наблюдаемого промежутка времени.
}
\answer{%
    \begin{align*}
    &\text{Изменение кинетической энергии равно работе внешних сил:} \\
    &\Delta E_k = E_k' - E_k = A_\text{тяж} \implies A_\text{тяж} = \frac{mv'^2}2 - \frac{mv_0^2}2 = 126\,\text{Дж}.
    \end{align*}
}

\tasknumber{9}%
\task{%
    Тонкий однородный лом длиной $1\,\text{м}$ и массой $10\,\text{кг}$ лежит на горизонтальной поверхности.
    \begin{itemize}
        \item Какую минимальную силу надо приложить к одному из его концов, чтобы оторвать его от этой поверхности?
        \item Какую минимальную работу надо совершить, чтобы поставить его на землю в вертикальное положение?
    \end{itemize}
    % Примите $g = 10\,\frac{\text{м}}{\text{с}^{2}}$.
}
\answer{%
    $F = \frac{mg}2 \approx 100\,\text{Н}, A = mg\frac l2 = 50\,\text{Дж}$
}

\tasknumber{10}%
\task{%
    Для того, чтобы разогать тело из состояния покоя до скорости $v$ с постоянным ускорением,
    требуется совершить работу $10\,\text{Дж}$.
    Какую работу нужно совершить, чтобы увеличить скорость этого тела от $v$ до $3v$?
}
\answer{%
    \begin{align*}
    &\text{Изменение кинетической энергии равно работе внешних сил:} \\
    &A_1 = \frac{mv^2}2 - \frac{m \cdot 0^2}2 = \frac{mv^2}2, A_2 = \frac{m\sqr{3v}}2 - \frac{mv^2}2 \implies  \\
    &\implies A_2 = \frac{mv^2}2 \cbr{3^2 - 1} = A_1 \cdot \cbr{3^2 - 1} = 80\,\text{Дж}.
    \end{align*}
}
% autogenerated
