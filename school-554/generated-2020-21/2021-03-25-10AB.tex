\setdate{25~марта~2021}
\setclass{10«АБ»}

\addpersonalvariant{Михаил Бурмистров}

\tasknumber{1}%
\task{%
    С какой силой взаимодействуют 2 точечных заряда $q_1 = 3\,\text{нКл}$ и $q_2 = 4\,\text{нКл}$,
    находящиеся на расстоянии $r = 6\,\text{см}$?
}
\answer{%
    $
        F
            = k\frac{q_1q_2}{r^2}
            = 9 \cdot 10^{9}\,\frac{\text{Н}\cdot\text{м}^{2}}{\text{Кл}^{2}} \cdot \frac{3\,\text{нКл} \cdot 4\,\text{нКл}}{\sqr{ 6\,\text{см} }}
            = 3 \cdot 10^{31}\units{Н}
              \approx {3{,}00} \cdot 10^{31}\units{Н}
    $
}
\solutionspace{120pt}

\tasknumber{2}%
\task{%
    Два одинаковых маленьких проводящих заряженных шарика находятся на расстоянии~$l$ друг от друга.
    Заряд первого равен~$+3Q$, второго~--- $+8Q$.
    Шарики приводят в соприкосновение, а после опять разводят на расстояние~$2l$.
    \begin{itemize}
        \item Каким стал заряд каждого из шариков?
        \item Определите характер (притяжение или отталкивание) и силу взаимодействия шариков до и после соприкосновения.
        \item Как изменилась сила взаимодействия шариков после соприкосновения?
    \end{itemize}
}
\answer{%
    \begin{align*}
    F &= k\frac{\abs{q_1}\abs {q_2}}{\sqr{2 l}}   = k\frac{\abs{+3Q} \cdot \abs{+8Q}}{2^2 \cdot l^2}, \text{отталкивание}; \\
        q'_1 &= q'_2, q_1 + q_2 = q'_1 + q'_2 \implies  q'_1 = q'_2 = \frac{q_1 + q_2}2 = \frac{+3Q + +8Q}2 = \frac{11}2Q \implies \\
        \implies F'  &= k\frac{\abs{q'_1}\abs{q'_2}}{\sqr{2 l}}
            = k\frac{\sqr{\frac{11}2Q}}{2^2 \cdot l^2},
        \text{отталкивание}, \\
    \frac{F'}{F} &= \frac{\sqr{\frac{11}2Q}}{2^2 \cdot \abs{+3Q} \cdot \abs{+8Q}} = \frac{121}{384}.
    \end{align*}
}
\solutionspace{120pt}

\tasknumber{3}%
\task{%
    На координатной плоскости в точках $(-d; 0)$ и $(d; 0)$
    находятся заряды, соответственно, $+Q$ и $+Q$.
    Сделайте рисунок, определите величину напряжённости электрического поля
    и укажите её направление в двух точках: $(0; -d)$ и $(2d; 0)$.
}
\solutionspace{120pt}

\tasknumber{4}%
\task{%
    Заряд $q_1$ создает в точке $A$ электрическое поле
    по величине равное~$E_1=300\funits{В}{м}$,
    а $q_2$~--- $E_2=400\funits{В}{м}$.
    Угол между векторами $\vect{E_1}$ и $\vect{E_2}$ равен $\varphi$.
    Определите величину суммарного электрического поля в точке $A$,
    создаваемого обоими зарядами $q_1$ и $q_2$.
    Сделайте рисунки и вычислите значение для двух значений угла $\varphi$:
    $\varphi_1=0^\circ$ и $\varphi_2=90^\circ$.
}

\variantsplitter

\addpersonalvariant{Ирина Ан}

\tasknumber{1}%
\task{%
    С какой силой взаимодействуют 2 точечных заряда $q_1 = 4\,\text{нКл}$ и $q_2 = 2\,\text{нКл}$,
    находящиеся на расстоянии $l = 6\,\text{см}$?
}
\answer{%
    $
        F
            = k\frac{q_1q_2}{l^2}
            = 9 \cdot 10^{9}\,\frac{\text{Н}\cdot\text{м}^{2}}{\text{Кл}^{2}} \cdot \frac{4\,\text{нКл} \cdot 2\,\text{нКл}}{\sqr{ 6\,\text{см} }}
            = 2 \cdot 10^{31}\units{Н}
              \approx {2{,}00} \cdot 10^{31}\units{Н}
    $
}
\solutionspace{120pt}

\tasknumber{2}%
\task{%
    Два одинаковых маленьких проводящих заряженных шарика находятся на расстоянии~$r$ друг от друга.
    Заряд первого равен~$-7q$, второго~--- $-2q$.
    Шарики приводят в соприкосновение, а после опять разводят на расстояние~$2r$.
    \begin{itemize}
        \item Каким стал заряд каждого из шариков?
        \item Определите характер (притяжение или отталкивание) и силу взаимодействия шариков до и после соприкосновения.
        \item Как изменилась сила взаимодействия шариков после соприкосновения?
    \end{itemize}
}
\answer{%
    \begin{align*}
    F &= k\frac{\abs{q_1}\abs {q_2}}{\sqr{2 r}}   = k\frac{\abs{-7q} \cdot \abs{-2q}}{2^2 \cdot r^2}, \text{отталкивание}; \\
        q'_1 &= q'_2, q_1 + q_2 = q'_1 + q'_2 \implies  q'_1 = q'_2 = \frac{q_1 + q_2}2 = \frac{-7q -2q}2 = -\frac92q \implies \\
        \implies F'  &= k\frac{\abs{q'_1}\abs{q'_2}}{\sqr{2 r}}
            = k\frac{\sqr{-\frac92q}}{2^2 \cdot r^2},
        \text{отталкивание}, \\
    \frac{F'}{F} &= \frac{\sqr{-\frac92q}}{2^2 \cdot \abs{-7q} \cdot \abs{-2q}} = \frac{81}{224}.
    \end{align*}
}
\solutionspace{120pt}

\tasknumber{3}%
\task{%
    На координатной плоскости в точках $(-d; 0)$ и $(d; 0)$
    находятся заряды, соответственно, $+Q$ и $+Q$.
    Сделайте рисунок, определите величину напряжённости электрического поля
    и укажите её направление в двух точках: $(0; d)$ и $(2d; 0)$.
}
\solutionspace{120pt}

\tasknumber{4}%
\task{%
    Заряд $q_1$ создает в точке $A$ электрическое поле
    по величине равное~$E_1=300\funits{В}{м}$,
    а $q_2$~--- $E_2=400\funits{В}{м}$.
    Угол между векторами $\vect{E_1}$ и $\vect{E_2}$ равен $\varphi$.
    Определите величину суммарного электрического поля в точке $A$,
    создаваемого обоими зарядами $q_1$ и $q_2$.
    Сделайте рисунки и вычислите значение для двух значений угла $\varphi$:
    $\varphi_1=90^\circ$ и $\varphi_2=180^\circ$.
}

\variantsplitter

\addpersonalvariant{Софья Андрианова}

\tasknumber{1}%
\task{%
    С какой силой взаимодействуют 2 точечных заряда $q_1 = 3\,\text{нКл}$ и $q_2 = 4\,\text{нКл}$,
    находящиеся на расстоянии $l = 5\,\text{см}$?
}
\answer{%
    $
        F
            = k\frac{q_1q_2}{l^2}
            = 9 \cdot 10^{9}\,\frac{\text{Н}\cdot\text{м}^{2}}{\text{Кл}^{2}} \cdot \frac{3\,\text{нКл} \cdot 4\,\text{нКл}}{\sqr{ 5\,\text{см} }}
            = \frac{108}{25} \cdot 10^{31}\units{Н}
              \approx {4{,}32} \cdot 10^{31}\units{Н}
    $
}
\solutionspace{120pt}

\tasknumber{2}%
\task{%
    Два одинаковых маленьких проводящих заряженных шарика находятся на расстоянии~$l$ друг от друга.
    Заряд первого равен~$-5q$, второго~--- $+6q$.
    Шарики приводят в соприкосновение, а после опять разводят на расстояние~$4l$.
    \begin{itemize}
        \item Каким стал заряд каждого из шариков?
        \item Определите характер (притяжение или отталкивание) и силу взаимодействия шариков до и после соприкосновения.
        \item Как изменилась сила взаимодействия шариков после соприкосновения?
    \end{itemize}
}
\answer{%
    \begin{align*}
    F &= k\frac{\abs{q_1}\abs {q_2}}{\sqr{4 l}}   = k\frac{\abs{-5q} \cdot \abs{+6q}}{4^2 \cdot l^2}, \text{притяжение}; \\
        q'_1 &= q'_2, q_1 + q_2 = q'_1 + q'_2 \implies  q'_1 = q'_2 = \frac{q_1 + q_2}2 = \frac{-5q + +6q}2 = \frac12q \implies \\
        \implies F'  &= k\frac{\abs{q'_1}\abs{q'_2}}{\sqr{4 l}}
            = k\frac{\sqr{\frac12q}}{4^2 \cdot l^2},
        \text{отталкивание}, \\
    \frac{F'}{F} &= \frac{\sqr{\frac12q}}{4^2 \cdot \abs{-5q} \cdot \abs{+6q}} = \frac1{1920}.
    \end{align*}
}
\solutionspace{120pt}

\tasknumber{3}%
\task{%
    На координатной плоскости в точках $(-a; 0)$ и $(a; 0)$
    находятся заряды, соответственно, $-q$ и $-q$.
    Сделайте рисунок, определите величину напряжённости электрического поля
    и укажите её направление в двух точках: $(0; -a)$ и $(-2a; 0)$.
}
\solutionspace{120pt}

\tasknumber{4}%
\task{%
    Заряд $q_1$ создает в точке $A$ электрическое поле
    по величине равное~$E_1=200\funits{В}{м}$,
    а $q_2$~--- $E_2=200\funits{В}{м}$.
    Угол между векторами $\vect{E_1}$ и $\vect{E_2}$ равен $\alpha$.
    Определите величину суммарного электрического поля в точке $A$,
    создаваемого обоими зарядами $q_1$ и $q_2$.
    Сделайте рисунки и вычислите значение для двух значений угла $\alpha$:
    $\alpha_1=0^\circ$ и $\alpha_2=60^\circ$.
}

\variantsplitter

\addpersonalvariant{Владимир Артемчук}

\tasknumber{1}%
\task{%
    С какой силой взаимодействуют 2 точечных заряда $q_1 = 2\,\text{нКл}$ и $q_2 = 3\,\text{нКл}$,
    находящиеся на расстоянии $r = 2\,\text{см}$?
}
\answer{%
    $
        F
            = k\frac{q_1q_2}{r^2}
            = 9 \cdot 10^{9}\,\frac{\text{Н}\cdot\text{м}^{2}}{\text{Кл}^{2}} \cdot \frac{2\,\text{нКл} \cdot 3\,\text{нКл}}{\sqr{ 2\,\text{см} }}
            = \frac{27}2 \cdot 10^{31}\units{Н}
              \approx {13{,}50} \cdot 10^{31}\units{Н}
    $
}
\solutionspace{120pt}

\tasknumber{2}%
\task{%
    Два одинаковых маленьких проводящих заряженных шарика находятся на расстоянии~$d$ друг от друга.
    Заряд первого равен~$+7Q$, второго~--- $+6Q$.
    Шарики приводят в соприкосновение, а после опять разводят на расстояние~$4d$.
    \begin{itemize}
        \item Каким стал заряд каждого из шариков?
        \item Определите характер (притяжение или отталкивание) и силу взаимодействия шариков до и после соприкосновения.
        \item Как изменилась сила взаимодействия шариков после соприкосновения?
    \end{itemize}
}
\answer{%
    \begin{align*}
    F &= k\frac{\abs{q_1}\abs {q_2}}{\sqr{4 d}}   = k\frac{\abs{+7Q} \cdot \abs{+6Q}}{4^2 \cdot d^2}, \text{отталкивание}; \\
        q'_1 &= q'_2, q_1 + q_2 = q'_1 + q'_2 \implies  q'_1 = q'_2 = \frac{q_1 + q_2}2 = \frac{+7Q + +6Q}2 = \frac{13}2Q \implies \\
        \implies F'  &= k\frac{\abs{q'_1}\abs{q'_2}}{\sqr{4 d}}
            = k\frac{\sqr{\frac{13}2Q}}{4^2 \cdot d^2},
        \text{отталкивание}, \\
    \frac{F'}{F} &= \frac{\sqr{\frac{13}2Q}}{4^2 \cdot \abs{+7Q} \cdot \abs{+6Q}} = \frac{169}{2688}.
    \end{align*}
}
\solutionspace{120pt}

\tasknumber{3}%
\task{%
    На координатной плоскости в точках $(-a; 0)$ и $(a; 0)$
    находятся заряды, соответственно, $+Q$ и $+Q$.
    Сделайте рисунок, определите величину напряжённости электрического поля
    и укажите её направление в двух точках: $(0; a)$ и $(-2a; 0)$.
}
\solutionspace{120pt}

\tasknumber{4}%
\task{%
    Заряд $q_1$ создает в точке $A$ электрическое поле
    по величине равное~$E_1=200\funits{В}{м}$,
    а $q_2$~--- $E_2=200\funits{В}{м}$.
    Угол между векторами $\vect{E_1}$ и $\vect{E_2}$ равен $\varphi$.
    Определите величину суммарного электрического поля в точке $A$,
    создаваемого обоими зарядами $q_1$ и $q_2$.
    Сделайте рисунки и вычислите значение для двух значений угла $\varphi$:
    $\varphi_1=0^\circ$ и $\varphi_2=60^\circ$.
}

\variantsplitter

\addpersonalvariant{Софья Белянкина}

\tasknumber{1}%
\task{%
    С какой силой взаимодействуют 2 точечных заряда $q_1 = 3\,\text{нКл}$ и $q_2 = 4\,\text{нКл}$,
    находящиеся на расстоянии $l = 5\,\text{см}$?
}
\answer{%
    $
        F
            = k\frac{q_1q_2}{l^2}
            = 9 \cdot 10^{9}\,\frac{\text{Н}\cdot\text{м}^{2}}{\text{Кл}^{2}} \cdot \frac{3\,\text{нКл} \cdot 4\,\text{нКл}}{\sqr{ 5\,\text{см} }}
            = \frac{108}{25} \cdot 10^{31}\units{Н}
              \approx {4{,}32} \cdot 10^{31}\units{Н}
    $
}
\solutionspace{120pt}

\tasknumber{2}%
\task{%
    Два одинаковых маленьких проводящих заряженных шарика находятся на расстоянии~$d$ друг от друга.
    Заряд первого равен~$+7Q$, второго~--- $+8Q$.
    Шарики приводят в соприкосновение, а после опять разводят на расстояние~$2d$.
    \begin{itemize}
        \item Каким стал заряд каждого из шариков?
        \item Определите характер (притяжение или отталкивание) и силу взаимодействия шариков до и после соприкосновения.
        \item Как изменилась сила взаимодействия шариков после соприкосновения?
    \end{itemize}
}
\answer{%
    \begin{align*}
    F &= k\frac{\abs{q_1}\abs {q_2}}{\sqr{2 d}}   = k\frac{\abs{+7Q} \cdot \abs{+8Q}}{2^2 \cdot d^2}, \text{отталкивание}; \\
        q'_1 &= q'_2, q_1 + q_2 = q'_1 + q'_2 \implies  q'_1 = q'_2 = \frac{q_1 + q_2}2 = \frac{+7Q + +8Q}2 = \frac{15}2Q \implies \\
        \implies F'  &= k\frac{\abs{q'_1}\abs{q'_2}}{\sqr{2 d}}
            = k\frac{\sqr{\frac{15}2Q}}{2^2 \cdot d^2},
        \text{отталкивание}, \\
    \frac{F'}{F} &= \frac{\sqr{\frac{15}2Q}}{2^2 \cdot \abs{+7Q} \cdot \abs{+8Q}} = \frac{225}{896}.
    \end{align*}
}
\solutionspace{120pt}

\tasknumber{3}%
\task{%
    На координатной плоскости в точках $(-l; 0)$ и $(l; 0)$
    находятся заряды, соответственно, $+Q$ и $+Q$.
    Сделайте рисунок, определите величину напряжённости электрического поля
    и укажите её направление в двух точках: $(0; l)$ и $(2l; 0)$.
}
\solutionspace{120pt}

\tasknumber{4}%
\task{%
    Заряд $q_1$ создает в точке $A$ электрическое поле
    по величине равное~$E_1=250\funits{В}{м}$,
    а $q_2$~--- $E_2=250\funits{В}{м}$.
    Угол между векторами $\vect{E_1}$ и $\vect{E_2}$ равен $\varphi$.
    Определите величину суммарного электрического поля в точке $A$,
    создаваемого обоими зарядами $q_1$ и $q_2$.
    Сделайте рисунки и вычислите значение для двух значений угла $\varphi$:
    $\varphi_1=0^\circ$ и $\varphi_2=60^\circ$.
}

\variantsplitter

\addpersonalvariant{Варвара Егиазарян}

\tasknumber{1}%
\task{%
    С какой силой взаимодействуют 2 точечных заряда $q_1 = 3\,\text{нКл}$ и $q_2 = 4\,\text{нКл}$,
    находящиеся на расстоянии $l = 3\,\text{см}$?
}
\answer{%
    $
        F
            = k\frac{q_1q_2}{l^2}
            = 9 \cdot 10^{9}\,\frac{\text{Н}\cdot\text{м}^{2}}{\text{Кл}^{2}} \cdot \frac{3\,\text{нКл} \cdot 4\,\text{нКл}}{\sqr{ 3\,\text{см} }}
            = 12 \cdot 10^{31}\units{Н}
              \approx {12{,}00} \cdot 10^{31}\units{Н}
    $
}
\solutionspace{120pt}

\tasknumber{2}%
\task{%
    Два одинаковых маленьких проводящих заряженных шарика находятся на расстоянии~$l$ друг от друга.
    Заряд первого равен~$-5Q$, второго~--- $+8Q$.
    Шарики приводят в соприкосновение, а после опять разводят на расстояние~$3l$.
    \begin{itemize}
        \item Каким стал заряд каждого из шариков?
        \item Определите характер (притяжение или отталкивание) и силу взаимодействия шариков до и после соприкосновения.
        \item Как изменилась сила взаимодействия шариков после соприкосновения?
    \end{itemize}
}
\answer{%
    \begin{align*}
    F &= k\frac{\abs{q_1}\abs {q_2}}{\sqr{3 l}}   = k\frac{\abs{-5Q} \cdot \abs{+8Q}}{3^2 \cdot l^2}, \text{притяжение}; \\
        q'_1 &= q'_2, q_1 + q_2 = q'_1 + q'_2 \implies  q'_1 = q'_2 = \frac{q_1 + q_2}2 = \frac{-5Q + +8Q}2 = \frac32Q \implies \\
        \implies F'  &= k\frac{\abs{q'_1}\abs{q'_2}}{\sqr{3 l}}
            = k\frac{\sqr{\frac32Q}}{3^2 \cdot l^2},
        \text{отталкивание}, \\
    \frac{F'}{F} &= \frac{\sqr{\frac32Q}}{3^2 \cdot \abs{-5Q} \cdot \abs{+8Q}} = \frac1{160}.
    \end{align*}
}
\solutionspace{120pt}

\tasknumber{3}%
\task{%
    На координатной плоскости в точках $(-a; 0)$ и $(a; 0)$
    находятся заряды, соответственно, $+Q$ и $+Q$.
    Сделайте рисунок, определите величину напряжённости электрического поля
    и укажите её направление в двух точках: $(0; -a)$ и $(2a; 0)$.
}
\solutionspace{120pt}

\tasknumber{4}%
\task{%
    Заряд $q_1$ создает в точке $A$ электрическое поле
    по величине равное~$E_1=300\funits{В}{м}$,
    а $q_2$~--- $E_2=400\funits{В}{м}$.
    Угол между векторами $\vect{E_1}$ и $\vect{E_2}$ равен $\alpha$.
    Определите величину суммарного электрического поля в точке $A$,
    создаваемого обоими зарядами $q_1$ и $q_2$.
    Сделайте рисунки и вычислите значение для двух значений угла $\alpha$:
    $\alpha_1=90^\circ$ и $\alpha_2=180^\circ$.
}

\variantsplitter

\addpersonalvariant{Владислав Емелин}

\tasknumber{1}%
\task{%
    С какой силой взаимодействуют 2 точечных заряда $q_1 = 4\,\text{нКл}$ и $q_2 = 3\,\text{нКл}$,
    находящиеся на расстоянии $r = 2\,\text{см}$?
}
\answer{%
    $
        F
            = k\frac{q_1q_2}{r^2}
            = 9 \cdot 10^{9}\,\frac{\text{Н}\cdot\text{м}^{2}}{\text{Кл}^{2}} \cdot \frac{4\,\text{нКл} \cdot 3\,\text{нКл}}{\sqr{ 2\,\text{см} }}
            = 27 \cdot 10^{31}\units{Н}
              \approx {27{,}00} \cdot 10^{31}\units{Н}
    $
}
\solutionspace{120pt}

\tasknumber{2}%
\task{%
    Два одинаковых маленьких проводящих заряженных шарика находятся на расстоянии~$d$ друг от друга.
    Заряд первого равен~$-7Q$, второго~--- $-2Q$.
    Шарики приводят в соприкосновение, а после опять разводят на расстояние~$3d$.
    \begin{itemize}
        \item Каким стал заряд каждого из шариков?
        \item Определите характер (притяжение или отталкивание) и силу взаимодействия шариков до и после соприкосновения.
        \item Как изменилась сила взаимодействия шариков после соприкосновения?
    \end{itemize}
}
\answer{%
    \begin{align*}
    F &= k\frac{\abs{q_1}\abs {q_2}}{\sqr{3 d}}   = k\frac{\abs{-7Q} \cdot \abs{-2Q}}{3^2 \cdot d^2}, \text{отталкивание}; \\
        q'_1 &= q'_2, q_1 + q_2 = q'_1 + q'_2 \implies  q'_1 = q'_2 = \frac{q_1 + q_2}2 = \frac{-7Q -2Q}2 = -\frac92Q \implies \\
        \implies F'  &= k\frac{\abs{q'_1}\abs{q'_2}}{\sqr{3 d}}
            = k\frac{\sqr{-\frac92Q}}{3^2 \cdot d^2},
        \text{отталкивание}, \\
    \frac{F'}{F} &= \frac{\sqr{-\frac92Q}}{3^2 \cdot \abs{-7Q} \cdot \abs{-2Q}} = \frac9{56}.
    \end{align*}
}
\solutionspace{120pt}

\tasknumber{3}%
\task{%
    На координатной плоскости в точках $(-r; 0)$ и $(r; 0)$
    находятся заряды, соответственно, $-Q$ и $+Q$.
    Сделайте рисунок, определите величину напряжённости электрического поля
    и укажите её направление в двух точках: $(0; r)$ и $(-2r; 0)$.
}
\solutionspace{120pt}

\tasknumber{4}%
\task{%
    Заряд $q_1$ создает в точке $A$ электрическое поле
    по величине равное~$E_1=120\funits{В}{м}$,
    а $q_2$~--- $E_2=50\funits{В}{м}$.
    Угол между векторами $\vect{E_1}$ и $\vect{E_2}$ равен $\varphi$.
    Определите величину суммарного электрического поля в точке $A$,
    создаваемого обоими зарядами $q_1$ и $q_2$.
    Сделайте рисунки и вычислите значение для двух значений угла $\varphi$:
    $\varphi_1=90^\circ$ и $\varphi_2=180^\circ$.
}

\variantsplitter

\addpersonalvariant{Артём Жичин}

\tasknumber{1}%
\task{%
    С какой силой взаимодействуют 2 точечных заряда $q_1 = 3\,\text{нКл}$ и $q_2 = 2\,\text{нКл}$,
    находящиеся на расстоянии $r = 6\,\text{см}$?
}
\answer{%
    $
        F
            = k\frac{q_1q_2}{r^2}
            = 9 \cdot 10^{9}\,\frac{\text{Н}\cdot\text{м}^{2}}{\text{Кл}^{2}} \cdot \frac{3\,\text{нКл} \cdot 2\,\text{нКл}}{\sqr{ 6\,\text{см} }}
            = \frac32 \cdot 10^{31}\units{Н}
              \approx {1{,}50} \cdot 10^{31}\units{Н}
    $
}
\solutionspace{120pt}

\tasknumber{2}%
\task{%
    Два одинаковых маленьких проводящих заряженных шарика находятся на расстоянии~$d$ друг от друга.
    Заряд первого равен~$-7q$, второго~--- $-4q$.
    Шарики приводят в соприкосновение, а после опять разводят на расстояние~$4d$.
    \begin{itemize}
        \item Каким стал заряд каждого из шариков?
        \item Определите характер (притяжение или отталкивание) и силу взаимодействия шариков до и после соприкосновения.
        \item Как изменилась сила взаимодействия шариков после соприкосновения?
    \end{itemize}
}
\answer{%
    \begin{align*}
    F &= k\frac{\abs{q_1}\abs {q_2}}{\sqr{4 d}}   = k\frac{\abs{-7q} \cdot \abs{-4q}}{4^2 \cdot d^2}, \text{отталкивание}; \\
        q'_1 &= q'_2, q_1 + q_2 = q'_1 + q'_2 \implies  q'_1 = q'_2 = \frac{q_1 + q_2}2 = \frac{-7q -4q}2 = -\frac{11}2q \implies \\
        \implies F'  &= k\frac{\abs{q'_1}\abs{q'_2}}{\sqr{4 d}}
            = k\frac{\sqr{-\frac{11}2q}}{4^2 \cdot d^2},
        \text{отталкивание}, \\
    \frac{F'}{F} &= \frac{\sqr{-\frac{11}2q}}{4^2 \cdot \abs{-7q} \cdot \abs{-4q}} = \frac{121}{1792}.
    \end{align*}
}
\solutionspace{120pt}

\tasknumber{3}%
\task{%
    На координатной плоскости в точках $(-d; 0)$ и $(d; 0)$
    находятся заряды, соответственно, $+Q$ и $+Q$.
    Сделайте рисунок, определите величину напряжённости электрического поля
    и укажите её направление в двух точках: $(0; -d)$ и $(2d; 0)$.
}
\solutionspace{120pt}

\tasknumber{4}%
\task{%
    Заряд $q_1$ создает в точке $A$ электрическое поле
    по величине равное~$E_1=7\funits{В}{м}$,
    а $q_2$~--- $E_2=24\funits{В}{м}$.
    Угол между векторами $\vect{E_1}$ и $\vect{E_2}$ равен $\varphi$.
    Определите величину суммарного электрического поля в точке $A$,
    создаваемого обоими зарядами $q_1$ и $q_2$.
    Сделайте рисунки и вычислите значение для двух значений угла $\varphi$:
    $\varphi_1=0^\circ$ и $\varphi_2=90^\circ$.
}

\variantsplitter

\addpersonalvariant{Дарья Кошман}

\tasknumber{1}%
\task{%
    С какой силой взаимодействуют 2 точечных заряда $q_1 = 4\,\text{нКл}$ и $q_2 = 3\,\text{нКл}$,
    находящиеся на расстоянии $l = 6\,\text{см}$?
}
\answer{%
    $
        F
            = k\frac{q_1q_2}{l^2}
            = 9 \cdot 10^{9}\,\frac{\text{Н}\cdot\text{м}^{2}}{\text{Кл}^{2}} \cdot \frac{4\,\text{нКл} \cdot 3\,\text{нКл}}{\sqr{ 6\,\text{см} }}
            = 3 \cdot 10^{31}\units{Н}
              \approx {3{,}00} \cdot 10^{31}\units{Н}
    $
}
\solutionspace{120pt}

\tasknumber{2}%
\task{%
    Два одинаковых маленьких проводящих заряженных шарика находятся на расстоянии~$l$ друг от друга.
    Заряд первого равен~$-5q$, второго~--- $-2q$.
    Шарики приводят в соприкосновение, а после опять разводят на расстояние~$2l$.
    \begin{itemize}
        \item Каким стал заряд каждого из шариков?
        \item Определите характер (притяжение или отталкивание) и силу взаимодействия шариков до и после соприкосновения.
        \item Как изменилась сила взаимодействия шариков после соприкосновения?
    \end{itemize}
}
\answer{%
    \begin{align*}
    F &= k\frac{\abs{q_1}\abs {q_2}}{\sqr{2 l}}   = k\frac{\abs{-5q} \cdot \abs{-2q}}{2^2 \cdot l^2}, \text{отталкивание}; \\
        q'_1 &= q'_2, q_1 + q_2 = q'_1 + q'_2 \implies  q'_1 = q'_2 = \frac{q_1 + q_2}2 = \frac{-5q -2q}2 = -\frac72q \implies \\
        \implies F'  &= k\frac{\abs{q'_1}\abs{q'_2}}{\sqr{2 l}}
            = k\frac{\sqr{-\frac72q}}{2^2 \cdot l^2},
        \text{отталкивание}, \\
    \frac{F'}{F} &= \frac{\sqr{-\frac72q}}{2^2 \cdot \abs{-5q} \cdot \abs{-2q}} = \frac{49}{160}.
    \end{align*}
}
\solutionspace{120pt}

\tasknumber{3}%
\task{%
    На координатной плоскости в точках $(-l; 0)$ и $(l; 0)$
    находятся заряды, соответственно, $-q$ и $-q$.
    Сделайте рисунок, определите величину напряжённости электрического поля
    и укажите её направление в двух точках: $(0; l)$ и $(2l; 0)$.
}
\solutionspace{120pt}

\tasknumber{4}%
\task{%
    Заряд $q_1$ создает в точке $A$ электрическое поле
    по величине равное~$E_1=24\funits{В}{м}$,
    а $q_2$~--- $E_2=7\funits{В}{м}$.
    Угол между векторами $\vect{E_1}$ и $\vect{E_2}$ равен $\alpha$.
    Определите величину суммарного электрического поля в точке $A$,
    создаваемого обоими зарядами $q_1$ и $q_2$.
    Сделайте рисунки и вычислите значение для двух значений угла $\alpha$:
    $\alpha_1=90^\circ$ и $\alpha_2=180^\circ$.
}

\variantsplitter

\addpersonalvariant{Анна Кузьмичёва}

\tasknumber{1}%
\task{%
    С какой силой взаимодействуют 2 точечных заряда $q_1 = 2\,\text{нКл}$ и $q_2 = 3\,\text{нКл}$,
    находящиеся на расстоянии $d = 3\,\text{см}$?
}
\answer{%
    $
        F
            = k\frac{q_1q_2}{d^2}
            = 9 \cdot 10^{9}\,\frac{\text{Н}\cdot\text{м}^{2}}{\text{Кл}^{2}} \cdot \frac{2\,\text{нКл} \cdot 3\,\text{нКл}}{\sqr{ 3\,\text{см} }}
            = 6 \cdot 10^{31}\units{Н}
              \approx {6{,}00} \cdot 10^{31}\units{Н}
    $
}
\solutionspace{120pt}

\tasknumber{2}%
\task{%
    Два одинаковых маленьких проводящих заряженных шарика находятся на расстоянии~$d$ друг от друга.
    Заряд первого равен~$-7q$, второго~--- $-2q$.
    Шарики приводят в соприкосновение, а после опять разводят на расстояние~$2d$.
    \begin{itemize}
        \item Каким стал заряд каждого из шариков?
        \item Определите характер (притяжение или отталкивание) и силу взаимодействия шариков до и после соприкосновения.
        \item Как изменилась сила взаимодействия шариков после соприкосновения?
    \end{itemize}
}
\answer{%
    \begin{align*}
    F &= k\frac{\abs{q_1}\abs {q_2}}{\sqr{2 d}}   = k\frac{\abs{-7q} \cdot \abs{-2q}}{2^2 \cdot d^2}, \text{отталкивание}; \\
        q'_1 &= q'_2, q_1 + q_2 = q'_1 + q'_2 \implies  q'_1 = q'_2 = \frac{q_1 + q_2}2 = \frac{-7q -2q}2 = -\frac92q \implies \\
        \implies F'  &= k\frac{\abs{q'_1}\abs{q'_2}}{\sqr{2 d}}
            = k\frac{\sqr{-\frac92q}}{2^2 \cdot d^2},
        \text{отталкивание}, \\
    \frac{F'}{F} &= \frac{\sqr{-\frac92q}}{2^2 \cdot \abs{-7q} \cdot \abs{-2q}} = \frac{81}{224}.
    \end{align*}
}
\solutionspace{120pt}

\tasknumber{3}%
\task{%
    На координатной плоскости в точках $(-r; 0)$ и $(r; 0)$
    находятся заряды, соответственно, $-q$ и $-q$.
    Сделайте рисунок, определите величину напряжённости электрического поля
    и укажите её направление в двух точках: $(0; r)$ и $(-2r; 0)$.
}
\solutionspace{120pt}

\tasknumber{4}%
\task{%
    Заряд $q_1$ создает в точке $A$ электрическое поле
    по величине равное~$E_1=250\funits{В}{м}$,
    а $q_2$~--- $E_2=250\funits{В}{м}$.
    Угол между векторами $\vect{E_1}$ и $\vect{E_2}$ равен $\varphi$.
    Определите величину суммарного электрического поля в точке $A$,
    создаваемого обоими зарядами $q_1$ и $q_2$.
    Сделайте рисунки и вычислите значение для двух значений угла $\varphi$:
    $\varphi_1=0^\circ$ и $\varphi_2=60^\circ$.
}

\variantsplitter

\addpersonalvariant{Алёна Куприянова}

\tasknumber{1}%
\task{%
    С какой силой взаимодействуют 2 точечных заряда $q_1 = 2\,\text{нКл}$ и $q_2 = 3\,\text{нКл}$,
    находящиеся на расстоянии $r = 5\,\text{см}$?
}
\answer{%
    $
        F
            = k\frac{q_1q_2}{r^2}
            = 9 \cdot 10^{9}\,\frac{\text{Н}\cdot\text{м}^{2}}{\text{Кл}^{2}} \cdot \frac{2\,\text{нКл} \cdot 3\,\text{нКл}}{\sqr{ 5\,\text{см} }}
            = \frac{54}{25} \cdot 10^{31}\units{Н}
              \approx {2{,}16} \cdot 10^{31}\units{Н}
    $
}
\solutionspace{120pt}

\tasknumber{2}%
\task{%
    Два одинаковых маленьких проводящих заряженных шарика находятся на расстоянии~$r$ друг от друга.
    Заряд первого равен~$-3q$, второго~--- $+4q$.
    Шарики приводят в соприкосновение, а после опять разводят на расстояние~$4r$.
    \begin{itemize}
        \item Каким стал заряд каждого из шариков?
        \item Определите характер (притяжение или отталкивание) и силу взаимодействия шариков до и после соприкосновения.
        \item Как изменилась сила взаимодействия шариков после соприкосновения?
    \end{itemize}
}
\answer{%
    \begin{align*}
    F &= k\frac{\abs{q_1}\abs {q_2}}{\sqr{4 r}}   = k\frac{\abs{-3q} \cdot \abs{+4q}}{4^2 \cdot r^2}, \text{притяжение}; \\
        q'_1 &= q'_2, q_1 + q_2 = q'_1 + q'_2 \implies  q'_1 = q'_2 = \frac{q_1 + q_2}2 = \frac{-3q + +4q}2 = \frac12q \implies \\
        \implies F'  &= k\frac{\abs{q'_1}\abs{q'_2}}{\sqr{4 r}}
            = k\frac{\sqr{\frac12q}}{4^2 \cdot r^2},
        \text{отталкивание}, \\
    \frac{F'}{F} &= \frac{\sqr{\frac12q}}{4^2 \cdot \abs{-3q} \cdot \abs{+4q}} = \frac1{768}.
    \end{align*}
}
\solutionspace{120pt}

\tasknumber{3}%
\task{%
    На координатной плоскости в точках $(-d; 0)$ и $(d; 0)$
    находятся заряды, соответственно, $-Q$ и $+Q$.
    Сделайте рисунок, определите величину напряжённости электрического поля
    и укажите её направление в двух точках: $(0; -d)$ и $(2d; 0)$.
}
\solutionspace{120pt}

\tasknumber{4}%
\task{%
    Заряд $q_1$ создает в точке $A$ электрическое поле
    по величине равное~$E_1=120\funits{В}{м}$,
    а $q_2$~--- $E_2=50\funits{В}{м}$.
    Угол между векторами $\vect{E_1}$ и $\vect{E_2}$ равен $\alpha$.
    Определите величину суммарного электрического поля в точке $A$,
    создаваемого обоими зарядами $q_1$ и $q_2$.
    Сделайте рисунки и вычислите значение для двух значений угла $\alpha$:
    $\alpha_1=90^\circ$ и $\alpha_2=180^\circ$.
}

\variantsplitter

\addpersonalvariant{Ярослав Лавровский}

\tasknumber{1}%
\task{%
    С какой силой взаимодействуют 2 точечных заряда $q_1 = 3\,\text{нКл}$ и $q_2 = 2\,\text{нКл}$,
    находящиеся на расстоянии $r = 5\,\text{см}$?
}
\answer{%
    $
        F
            = k\frac{q_1q_2}{r^2}
            = 9 \cdot 10^{9}\,\frac{\text{Н}\cdot\text{м}^{2}}{\text{Кл}^{2}} \cdot \frac{3\,\text{нКл} \cdot 2\,\text{нКл}}{\sqr{ 5\,\text{см} }}
            = \frac{54}{25} \cdot 10^{31}\units{Н}
              \approx {2{,}16} \cdot 10^{31}\units{Н}
    $
}
\solutionspace{120pt}

\tasknumber{2}%
\task{%
    Два одинаковых маленьких проводящих заряженных шарика находятся на расстоянии~$d$ друг от друга.
    Заряд первого равен~$-5Q$, второго~--- $-8Q$.
    Шарики приводят в соприкосновение, а после опять разводят на расстояние~$4d$.
    \begin{itemize}
        \item Каким стал заряд каждого из шариков?
        \item Определите характер (притяжение или отталкивание) и силу взаимодействия шариков до и после соприкосновения.
        \item Как изменилась сила взаимодействия шариков после соприкосновения?
    \end{itemize}
}
\answer{%
    \begin{align*}
    F &= k\frac{\abs{q_1}\abs {q_2}}{\sqr{4 d}}   = k\frac{\abs{-5Q} \cdot \abs{-8Q}}{4^2 \cdot d^2}, \text{отталкивание}; \\
        q'_1 &= q'_2, q_1 + q_2 = q'_1 + q'_2 \implies  q'_1 = q'_2 = \frac{q_1 + q_2}2 = \frac{-5Q -8Q}2 = -\frac{13}2Q \implies \\
        \implies F'  &= k\frac{\abs{q'_1}\abs{q'_2}}{\sqr{4 d}}
            = k\frac{\sqr{-\frac{13}2Q}}{4^2 \cdot d^2},
        \text{отталкивание}, \\
    \frac{F'}{F} &= \frac{\sqr{-\frac{13}2Q}}{4^2 \cdot \abs{-5Q} \cdot \abs{-8Q}} = \frac{169}{2560}.
    \end{align*}
}
\solutionspace{120pt}

\tasknumber{3}%
\task{%
    На координатной плоскости в точках $(-r; 0)$ и $(r; 0)$
    находятся заряды, соответственно, $-q$ и $-q$.
    Сделайте рисунок, определите величину напряжённости электрического поля
    и укажите её направление в двух точках: $(0; -r)$ и $(2r; 0)$.
}
\solutionspace{120pt}

\tasknumber{4}%
\task{%
    Заряд $q_1$ создает в точке $A$ электрическое поле
    по величине равное~$E_1=7\funits{В}{м}$,
    а $q_2$~--- $E_2=24\funits{В}{м}$.
    Угол между векторами $\vect{E_1}$ и $\vect{E_2}$ равен $\varphi$.
    Определите величину суммарного электрического поля в точке $A$,
    создаваемого обоими зарядами $q_1$ и $q_2$.
    Сделайте рисунки и вычислите значение для двух значений угла $\varphi$:
    $\varphi_1=0^\circ$ и $\varphi_2=90^\circ$.
}

\variantsplitter

\addpersonalvariant{Анастасия Ламанова}

\tasknumber{1}%
\task{%
    С какой силой взаимодействуют 2 точечных заряда $q_1 = 3\,\text{нКл}$ и $q_2 = 2\,\text{нКл}$,
    находящиеся на расстоянии $d = 5\,\text{см}$?
}
\answer{%
    $
        F
            = k\frac{q_1q_2}{d^2}
            = 9 \cdot 10^{9}\,\frac{\text{Н}\cdot\text{м}^{2}}{\text{Кл}^{2}} \cdot \frac{3\,\text{нКл} \cdot 2\,\text{нКл}}{\sqr{ 5\,\text{см} }}
            = \frac{54}{25} \cdot 10^{31}\units{Н}
              \approx {2{,}16} \cdot 10^{31}\units{Н}
    $
}
\solutionspace{120pt}

\tasknumber{2}%
\task{%
    Два одинаковых маленьких проводящих заряженных шарика находятся на расстоянии~$l$ друг от друга.
    Заряд первого равен~$+7Q$, второго~--- $-8Q$.
    Шарики приводят в соприкосновение, а после опять разводят на расстояние~$4l$.
    \begin{itemize}
        \item Каким стал заряд каждого из шариков?
        \item Определите характер (притяжение или отталкивание) и силу взаимодействия шариков до и после соприкосновения.
        \item Как изменилась сила взаимодействия шариков после соприкосновения?
    \end{itemize}
}
\answer{%
    \begin{align*}
    F &= k\frac{\abs{q_1}\abs {q_2}}{\sqr{4 l}}   = k\frac{\abs{+7Q} \cdot \abs{-8Q}}{4^2 \cdot l^2}, \text{притяжение}; \\
        q'_1 &= q'_2, q_1 + q_2 = q'_1 + q'_2 \implies  q'_1 = q'_2 = \frac{q_1 + q_2}2 = \frac{+7Q -8Q}2 = -\frac12Q \implies \\
        \implies F'  &= k\frac{\abs{q'_1}\abs{q'_2}}{\sqr{4 l}}
            = k\frac{\sqr{-\frac12Q}}{4^2 \cdot l^2},
        \text{отталкивание}, \\
    \frac{F'}{F} &= \frac{\sqr{-\frac12Q}}{4^2 \cdot \abs{+7Q} \cdot \abs{-8Q}} = \frac1{3584}.
    \end{align*}
}
\solutionspace{120pt}

\tasknumber{3}%
\task{%
    На координатной плоскости в точках $(-r; 0)$ и $(r; 0)$
    находятся заряды, соответственно, $-Q$ и $+Q$.
    Сделайте рисунок, определите величину напряжённости электрического поля
    и укажите её направление в двух точках: $(0; r)$ и $(-2r; 0)$.
}
\solutionspace{120pt}

\tasknumber{4}%
\task{%
    Заряд $q_1$ создает в точке $A$ электрическое поле
    по величине равное~$E_1=72\funits{В}{м}$,
    а $q_2$~--- $E_2=72\funits{В}{м}$.
    Угол между векторами $\vect{E_1}$ и $\vect{E_2}$ равен $\alpha$.
    Определите величину суммарного электрического поля в точке $A$,
    создаваемого обоими зарядами $q_1$ и $q_2$.
    Сделайте рисунки и вычислите значение для двух значений угла $\alpha$:
    $\alpha_1=0^\circ$ и $\alpha_2=120^\circ$.
}

\variantsplitter

\addpersonalvariant{Виктория Легонькова}

\tasknumber{1}%
\task{%
    С какой силой взаимодействуют 2 точечных заряда $q_1 = 4\,\text{нКл}$ и $q_2 = 3\,\text{нКл}$,
    находящиеся на расстоянии $l = 3\,\text{см}$?
}
\answer{%
    $
        F
            = k\frac{q_1q_2}{l^2}
            = 9 \cdot 10^{9}\,\frac{\text{Н}\cdot\text{м}^{2}}{\text{Кл}^{2}} \cdot \frac{4\,\text{нКл} \cdot 3\,\text{нКл}}{\sqr{ 3\,\text{см} }}
            = 12 \cdot 10^{31}\units{Н}
              \approx {12{,}00} \cdot 10^{31}\units{Н}
    $
}
\solutionspace{120pt}

\tasknumber{2}%
\task{%
    Два одинаковых маленьких проводящих заряженных шарика находятся на расстоянии~$d$ друг от друга.
    Заряд первого равен~$-3q$, второго~--- $-4q$.
    Шарики приводят в соприкосновение, а после опять разводят на расстояние~$2d$.
    \begin{itemize}
        \item Каким стал заряд каждого из шариков?
        \item Определите характер (притяжение или отталкивание) и силу взаимодействия шариков до и после соприкосновения.
        \item Как изменилась сила взаимодействия шариков после соприкосновения?
    \end{itemize}
}
\answer{%
    \begin{align*}
    F &= k\frac{\abs{q_1}\abs {q_2}}{\sqr{2 d}}   = k\frac{\abs{-3q} \cdot \abs{-4q}}{2^2 \cdot d^2}, \text{отталкивание}; \\
        q'_1 &= q'_2, q_1 + q_2 = q'_1 + q'_2 \implies  q'_1 = q'_2 = \frac{q_1 + q_2}2 = \frac{-3q -4q}2 = -\frac72q \implies \\
        \implies F'  &= k\frac{\abs{q'_1}\abs{q'_2}}{\sqr{2 d}}
            = k\frac{\sqr{-\frac72q}}{2^2 \cdot d^2},
        \text{отталкивание}, \\
    \frac{F'}{F} &= \frac{\sqr{-\frac72q}}{2^2 \cdot \abs{-3q} \cdot \abs{-4q}} = \frac{49}{192}.
    \end{align*}
}
\solutionspace{120pt}

\tasknumber{3}%
\task{%
    На координатной плоскости в точках $(-r; 0)$ и $(r; 0)$
    находятся заряды, соответственно, $-q$ и $-q$.
    Сделайте рисунок, определите величину напряжённости электрического поля
    и укажите её направление в двух точках: $(0; -r)$ и $(2r; 0)$.
}
\solutionspace{120pt}

\tasknumber{4}%
\task{%
    Заряд $q_1$ создает в точке $A$ электрическое поле
    по величине равное~$E_1=300\funits{В}{м}$,
    а $q_2$~--- $E_2=400\funits{В}{м}$.
    Угол между векторами $\vect{E_1}$ и $\vect{E_2}$ равен $\varphi$.
    Определите величину суммарного электрического поля в точке $A$,
    создаваемого обоими зарядами $q_1$ и $q_2$.
    Сделайте рисунки и вычислите значение для двух значений угла $\varphi$:
    $\varphi_1=90^\circ$ и $\varphi_2=180^\circ$.
}

\variantsplitter

\addpersonalvariant{Семён Мартынов}

\tasknumber{1}%
\task{%
    С какой силой взаимодействуют 2 точечных заряда $q_1 = 2\,\text{нКл}$ и $q_2 = 4\,\text{нКл}$,
    находящиеся на расстоянии $r = 5\,\text{см}$?
}
\answer{%
    $
        F
            = k\frac{q_1q_2}{r^2}
            = 9 \cdot 10^{9}\,\frac{\text{Н}\cdot\text{м}^{2}}{\text{Кл}^{2}} \cdot \frac{2\,\text{нКл} \cdot 4\,\text{нКл}}{\sqr{ 5\,\text{см} }}
            = \frac{72}{25} \cdot 10^{31}\units{Н}
              \approx {2{,}88} \cdot 10^{31}\units{Н}
    $
}
\solutionspace{120pt}

\tasknumber{2}%
\task{%
    Два одинаковых маленьких проводящих заряженных шарика находятся на расстоянии~$l$ друг от друга.
    Заряд первого равен~$-5q$, второго~--- $+2q$.
    Шарики приводят в соприкосновение, а после опять разводят на расстояние~$2l$.
    \begin{itemize}
        \item Каким стал заряд каждого из шариков?
        \item Определите характер (притяжение или отталкивание) и силу взаимодействия шариков до и после соприкосновения.
        \item Как изменилась сила взаимодействия шариков после соприкосновения?
    \end{itemize}
}
\answer{%
    \begin{align*}
    F &= k\frac{\abs{q_1}\abs {q_2}}{\sqr{2 l}}   = k\frac{\abs{-5q} \cdot \abs{+2q}}{2^2 \cdot l^2}, \text{притяжение}; \\
        q'_1 &= q'_2, q_1 + q_2 = q'_1 + q'_2 \implies  q'_1 = q'_2 = \frac{q_1 + q_2}2 = \frac{-5q + +2q}2 = -\frac32q \implies \\
        \implies F'  &= k\frac{\abs{q'_1}\abs{q'_2}}{\sqr{2 l}}
            = k\frac{\sqr{-\frac32q}}{2^2 \cdot l^2},
        \text{отталкивание}, \\
    \frac{F'}{F} &= \frac{\sqr{-\frac32q}}{2^2 \cdot \abs{-5q} \cdot \abs{+2q}} = \frac9{160}.
    \end{align*}
}
\solutionspace{120pt}

\tasknumber{3}%
\task{%
    На координатной плоскости в точках $(-r; 0)$ и $(r; 0)$
    находятся заряды, соответственно, $+Q$ и $+Q$.
    Сделайте рисунок, определите величину напряжённости электрического поля
    и укажите её направление в двух точках: $(0; -r)$ и $(2r; 0)$.
}
\solutionspace{120pt}

\tasknumber{4}%
\task{%
    Заряд $q_1$ создает в точке $A$ электрическое поле
    по величине равное~$E_1=7\funits{В}{м}$,
    а $q_2$~--- $E_2=24\funits{В}{м}$.
    Угол между векторами $\vect{E_1}$ и $\vect{E_2}$ равен $\alpha$.
    Определите величину суммарного электрического поля в точке $A$,
    создаваемого обоими зарядами $q_1$ и $q_2$.
    Сделайте рисунки и вычислите значение для двух значений угла $\alpha$:
    $\alpha_1=0^\circ$ и $\alpha_2=90^\circ$.
}

\variantsplitter

\addpersonalvariant{Варвара Минаева}

\tasknumber{1}%
\task{%
    С какой силой взаимодействуют 2 точечных заряда $q_1 = 4\,\text{нКл}$ и $q_2 = 3\,\text{нКл}$,
    находящиеся на расстоянии $d = 5\,\text{см}$?
}
\answer{%
    $
        F
            = k\frac{q_1q_2}{d^2}
            = 9 \cdot 10^{9}\,\frac{\text{Н}\cdot\text{м}^{2}}{\text{Кл}^{2}} \cdot \frac{4\,\text{нКл} \cdot 3\,\text{нКл}}{\sqr{ 5\,\text{см} }}
            = \frac{108}{25} \cdot 10^{31}\units{Н}
              \approx {4{,}32} \cdot 10^{31}\units{Н}
    $
}
\solutionspace{120pt}

\tasknumber{2}%
\task{%
    Два одинаковых маленьких проводящих заряженных шарика находятся на расстоянии~$l$ друг от друга.
    Заряд первого равен~$+7q$, второго~--- $+2q$.
    Шарики приводят в соприкосновение, а после опять разводят на расстояние~$3l$.
    \begin{itemize}
        \item Каким стал заряд каждого из шариков?
        \item Определите характер (притяжение или отталкивание) и силу взаимодействия шариков до и после соприкосновения.
        \item Как изменилась сила взаимодействия шариков после соприкосновения?
    \end{itemize}
}
\answer{%
    \begin{align*}
    F &= k\frac{\abs{q_1}\abs {q_2}}{\sqr{3 l}}   = k\frac{\abs{+7q} \cdot \abs{+2q}}{3^2 \cdot l^2}, \text{отталкивание}; \\
        q'_1 &= q'_2, q_1 + q_2 = q'_1 + q'_2 \implies  q'_1 = q'_2 = \frac{q_1 + q_2}2 = \frac{+7q + +2q}2 = \frac92q \implies \\
        \implies F'  &= k\frac{\abs{q'_1}\abs{q'_2}}{\sqr{3 l}}
            = k\frac{\sqr{\frac92q}}{3^2 \cdot l^2},
        \text{отталкивание}, \\
    \frac{F'}{F} &= \frac{\sqr{\frac92q}}{3^2 \cdot \abs{+7q} \cdot \abs{+2q}} = \frac9{56}.
    \end{align*}
}
\solutionspace{120pt}

\tasknumber{3}%
\task{%
    На координатной плоскости в точках $(-l; 0)$ и $(l; 0)$
    находятся заряды, соответственно, $-q$ и $-q$.
    Сделайте рисунок, определите величину напряжённости электрического поля
    и укажите её направление в двух точках: $(0; l)$ и $(-2l; 0)$.
}
\solutionspace{120pt}

\tasknumber{4}%
\task{%
    Заряд $q_1$ создает в точке $A$ электрическое поле
    по величине равное~$E_1=300\funits{В}{м}$,
    а $q_2$~--- $E_2=400\funits{В}{м}$.
    Угол между векторами $\vect{E_1}$ и $\vect{E_2}$ равен $\alpha$.
    Определите величину суммарного электрического поля в точке $A$,
    создаваемого обоими зарядами $q_1$ и $q_2$.
    Сделайте рисунки и вычислите значение для двух значений угла $\alpha$:
    $\alpha_1=90^\circ$ и $\alpha_2=180^\circ$.
}

\variantsplitter

\addpersonalvariant{Леонид Никитин}

\tasknumber{1}%
\task{%
    С какой силой взаимодействуют 2 точечных заряда $q_1 = 3\,\text{нКл}$ и $q_2 = 4\,\text{нКл}$,
    находящиеся на расстоянии $d = 2\,\text{см}$?
}
\answer{%
    $
        F
            = k\frac{q_1q_2}{d^2}
            = 9 \cdot 10^{9}\,\frac{\text{Н}\cdot\text{м}^{2}}{\text{Кл}^{2}} \cdot \frac{3\,\text{нКл} \cdot 4\,\text{нКл}}{\sqr{ 2\,\text{см} }}
            = 27 \cdot 10^{31}\units{Н}
              \approx {27{,}00} \cdot 10^{31}\units{Н}
    $
}
\solutionspace{120pt}

\tasknumber{2}%
\task{%
    Два одинаковых маленьких проводящих заряженных шарика находятся на расстоянии~$d$ друг от друга.
    Заряд первого равен~$-3q$, второго~--- $+8q$.
    Шарики приводят в соприкосновение, а после опять разводят на расстояние~$2d$.
    \begin{itemize}
        \item Каким стал заряд каждого из шариков?
        \item Определите характер (притяжение или отталкивание) и силу взаимодействия шариков до и после соприкосновения.
        \item Как изменилась сила взаимодействия шариков после соприкосновения?
    \end{itemize}
}
\answer{%
    \begin{align*}
    F &= k\frac{\abs{q_1}\abs {q_2}}{\sqr{2 d}}   = k\frac{\abs{-3q} \cdot \abs{+8q}}{2^2 \cdot d^2}, \text{притяжение}; \\
        q'_1 &= q'_2, q_1 + q_2 = q'_1 + q'_2 \implies  q'_1 = q'_2 = \frac{q_1 + q_2}2 = \frac{-3q + +8q}2 = \frac52q \implies \\
        \implies F'  &= k\frac{\abs{q'_1}\abs{q'_2}}{\sqr{2 d}}
            = k\frac{\sqr{\frac52q}}{2^2 \cdot d^2},
        \text{отталкивание}, \\
    \frac{F'}{F} &= \frac{\sqr{\frac52q}}{2^2 \cdot \abs{-3q} \cdot \abs{+8q}} = \frac{25}{384}.
    \end{align*}
}
\solutionspace{120pt}

\tasknumber{3}%
\task{%
    На координатной плоскости в точках $(-d; 0)$ и $(d; 0)$
    находятся заряды, соответственно, $-q$ и $-q$.
    Сделайте рисунок, определите величину напряжённости электрического поля
    и укажите её направление в двух точках: $(0; -d)$ и $(2d; 0)$.
}
\solutionspace{120pt}

\tasknumber{4}%
\task{%
    Заряд $q_1$ создает в точке $A$ электрическое поле
    по величине равное~$E_1=250\funits{В}{м}$,
    а $q_2$~--- $E_2=250\funits{В}{м}$.
    Угол между векторами $\vect{E_1}$ и $\vect{E_2}$ равен $\varphi$.
    Определите величину суммарного электрического поля в точке $A$,
    создаваемого обоими зарядами $q_1$ и $q_2$.
    Сделайте рисунки и вычислите значение для двух значений угла $\varphi$:
    $\varphi_1=0^\circ$ и $\varphi_2=60^\circ$.
}

\variantsplitter

\addpersonalvariant{Тимофей Полетаев}

\tasknumber{1}%
\task{%
    С какой силой взаимодействуют 2 точечных заряда $q_1 = 3\,\text{нКл}$ и $q_2 = 4\,\text{нКл}$,
    находящиеся на расстоянии $d = 3\,\text{см}$?
}
\answer{%
    $
        F
            = k\frac{q_1q_2}{d^2}
            = 9 \cdot 10^{9}\,\frac{\text{Н}\cdot\text{м}^{2}}{\text{Кл}^{2}} \cdot \frac{3\,\text{нКл} \cdot 4\,\text{нКл}}{\sqr{ 3\,\text{см} }}
            = 12 \cdot 10^{31}\units{Н}
              \approx {12{,}00} \cdot 10^{31}\units{Н}
    $
}
\solutionspace{120pt}

\tasknumber{2}%
\task{%
    Два одинаковых маленьких проводящих заряженных шарика находятся на расстоянии~$r$ друг от друга.
    Заряд первого равен~$-7Q$, второго~--- $+2Q$.
    Шарики приводят в соприкосновение, а после опять разводят на расстояние~$2r$.
    \begin{itemize}
        \item Каким стал заряд каждого из шариков?
        \item Определите характер (притяжение или отталкивание) и силу взаимодействия шариков до и после соприкосновения.
        \item Как изменилась сила взаимодействия шариков после соприкосновения?
    \end{itemize}
}
\answer{%
    \begin{align*}
    F &= k\frac{\abs{q_1}\abs {q_2}}{\sqr{2 r}}   = k\frac{\abs{-7Q} \cdot \abs{+2Q}}{2^2 \cdot r^2}, \text{притяжение}; \\
        q'_1 &= q'_2, q_1 + q_2 = q'_1 + q'_2 \implies  q'_1 = q'_2 = \frac{q_1 + q_2}2 = \frac{-7Q + +2Q}2 = -\frac52Q \implies \\
        \implies F'  &= k\frac{\abs{q'_1}\abs{q'_2}}{\sqr{2 r}}
            = k\frac{\sqr{-\frac52Q}}{2^2 \cdot r^2},
        \text{отталкивание}, \\
    \frac{F'}{F} &= \frac{\sqr{-\frac52Q}}{2^2 \cdot \abs{-7Q} \cdot \abs{+2Q}} = \frac{25}{224}.
    \end{align*}
}
\solutionspace{120pt}

\tasknumber{3}%
\task{%
    На координатной плоскости в точках $(-d; 0)$ и $(d; 0)$
    находятся заряды, соответственно, $-q$ и $-q$.
    Сделайте рисунок, определите величину напряжённости электрического поля
    и укажите её направление в двух точках: $(0; d)$ и $(2d; 0)$.
}
\solutionspace{120pt}

\tasknumber{4}%
\task{%
    Заряд $q_1$ создает в точке $A$ электрическое поле
    по величине равное~$E_1=50\funits{В}{м}$,
    а $q_2$~--- $E_2=120\funits{В}{м}$.
    Угол между векторами $\vect{E_1}$ и $\vect{E_2}$ равен $\alpha$.
    Определите величину суммарного электрического поля в точке $A$,
    создаваемого обоими зарядами $q_1$ и $q_2$.
    Сделайте рисунки и вычислите значение для двух значений угла $\alpha$:
    $\alpha_1=0^\circ$ и $\alpha_2=90^\circ$.
}

\variantsplitter

\addpersonalvariant{Андрей Рожков}

\tasknumber{1}%
\task{%
    С какой силой взаимодействуют 2 точечных заряда $q_1 = 2\,\text{нКл}$ и $q_2 = 4\,\text{нКл}$,
    находящиеся на расстоянии $l = 2\,\text{см}$?
}
\answer{%
    $
        F
            = k\frac{q_1q_2}{l^2}
            = 9 \cdot 10^{9}\,\frac{\text{Н}\cdot\text{м}^{2}}{\text{Кл}^{2}} \cdot \frac{2\,\text{нКл} \cdot 4\,\text{нКл}}{\sqr{ 2\,\text{см} }}
            = 18 \cdot 10^{31}\units{Н}
              \approx {18{,}00} \cdot 10^{31}\units{Н}
    $
}
\solutionspace{120pt}

\tasknumber{2}%
\task{%
    Два одинаковых маленьких проводящих заряженных шарика находятся на расстоянии~$r$ друг от друга.
    Заряд первого равен~$+7q$, второго~--- $+8q$.
    Шарики приводят в соприкосновение, а после опять разводят на расстояние~$2r$.
    \begin{itemize}
        \item Каким стал заряд каждого из шариков?
        \item Определите характер (притяжение или отталкивание) и силу взаимодействия шариков до и после соприкосновения.
        \item Как изменилась сила взаимодействия шариков после соприкосновения?
    \end{itemize}
}
\answer{%
    \begin{align*}
    F &= k\frac{\abs{q_1}\abs {q_2}}{\sqr{2 r}}   = k\frac{\abs{+7q} \cdot \abs{+8q}}{2^2 \cdot r^2}, \text{отталкивание}; \\
        q'_1 &= q'_2, q_1 + q_2 = q'_1 + q'_2 \implies  q'_1 = q'_2 = \frac{q_1 + q_2}2 = \frac{+7q + +8q}2 = \frac{15}2q \implies \\
        \implies F'  &= k\frac{\abs{q'_1}\abs{q'_2}}{\sqr{2 r}}
            = k\frac{\sqr{\frac{15}2q}}{2^2 \cdot r^2},
        \text{отталкивание}, \\
    \frac{F'}{F} &= \frac{\sqr{\frac{15}2q}}{2^2 \cdot \abs{+7q} \cdot \abs{+8q}} = \frac{225}{896}.
    \end{align*}
}
\solutionspace{120pt}

\tasknumber{3}%
\task{%
    На координатной плоскости в точках $(-r; 0)$ и $(r; 0)$
    находятся заряды, соответственно, $-Q$ и $+Q$.
    Сделайте рисунок, определите величину напряжённости электрического поля
    и укажите её направление в двух точках: $(0; -r)$ и $(-2r; 0)$.
}
\solutionspace{120pt}

\tasknumber{4}%
\task{%
    Заряд $q_1$ создает в точке $A$ электрическое поле
    по величине равное~$E_1=50\funits{В}{м}$,
    а $q_2$~--- $E_2=120\funits{В}{м}$.
    Угол между векторами $\vect{E_1}$ и $\vect{E_2}$ равен $\alpha$.
    Определите величину суммарного электрического поля в точке $A$,
    создаваемого обоими зарядами $q_1$ и $q_2$.
    Сделайте рисунки и вычислите значение для двух значений угла $\alpha$:
    $\alpha_1=0^\circ$ и $\alpha_2=90^\circ$.
}

\variantsplitter

\addpersonalvariant{Рената Таржиманова}

\tasknumber{1}%
\task{%
    С какой силой взаимодействуют 2 точечных заряда $q_1 = 2\,\text{нКл}$ и $q_2 = 4\,\text{нКл}$,
    находящиеся на расстоянии $l = 2\,\text{см}$?
}
\answer{%
    $
        F
            = k\frac{q_1q_2}{l^2}
            = 9 \cdot 10^{9}\,\frac{\text{Н}\cdot\text{м}^{2}}{\text{Кл}^{2}} \cdot \frac{2\,\text{нКл} \cdot 4\,\text{нКл}}{\sqr{ 2\,\text{см} }}
            = 18 \cdot 10^{31}\units{Н}
              \approx {18{,}00} \cdot 10^{31}\units{Н}
    $
}
\solutionspace{120pt}

\tasknumber{2}%
\task{%
    Два одинаковых маленьких проводящих заряженных шарика находятся на расстоянии~$r$ друг от друга.
    Заряд первого равен~$-3q$, второго~--- $+2q$.
    Шарики приводят в соприкосновение, а после опять разводят на расстояние~$3r$.
    \begin{itemize}
        \item Каким стал заряд каждого из шариков?
        \item Определите характер (притяжение или отталкивание) и силу взаимодействия шариков до и после соприкосновения.
        \item Как изменилась сила взаимодействия шариков после соприкосновения?
    \end{itemize}
}
\answer{%
    \begin{align*}
    F &= k\frac{\abs{q_1}\abs {q_2}}{\sqr{3 r}}   = k\frac{\abs{-3q} \cdot \abs{+2q}}{3^2 \cdot r^2}, \text{притяжение}; \\
        q'_1 &= q'_2, q_1 + q_2 = q'_1 + q'_2 \implies  q'_1 = q'_2 = \frac{q_1 + q_2}2 = \frac{-3q + +2q}2 = -\frac12q \implies \\
        \implies F'  &= k\frac{\abs{q'_1}\abs{q'_2}}{\sqr{3 r}}
            = k\frac{\sqr{-\frac12q}}{3^2 \cdot r^2},
        \text{отталкивание}, \\
    \frac{F'}{F} &= \frac{\sqr{-\frac12q}}{3^2 \cdot \abs{-3q} \cdot \abs{+2q}} = \frac1{216}.
    \end{align*}
}
\solutionspace{120pt}

\tasknumber{3}%
\task{%
    На координатной плоскости в точках $(-l; 0)$ и $(l; 0)$
    находятся заряды, соответственно, $+q$ и $-q$.
    Сделайте рисунок, определите величину напряжённости электрического поля
    и укажите её направление в двух точках: $(0; l)$ и $(-2l; 0)$.
}
\solutionspace{120pt}

\tasknumber{4}%
\task{%
    Заряд $q_1$ создает в точке $A$ электрическое поле
    по величине равное~$E_1=300\funits{В}{м}$,
    а $q_2$~--- $E_2=400\funits{В}{м}$.
    Угол между векторами $\vect{E_1}$ и $\vect{E_2}$ равен $\alpha$.
    Определите величину суммарного электрического поля в точке $A$,
    создаваемого обоими зарядами $q_1$ и $q_2$.
    Сделайте рисунки и вычислите значение для двух значений угла $\alpha$:
    $\alpha_1=0^\circ$ и $\alpha_2=90^\circ$.
}

\variantsplitter

\addpersonalvariant{Андрей Щербаков}

\tasknumber{1}%
\task{%
    С какой силой взаимодействуют 2 точечных заряда $q_1 = 4\,\text{нКл}$ и $q_2 = 2\,\text{нКл}$,
    находящиеся на расстоянии $l = 3\,\text{см}$?
}
\answer{%
    $
        F
            = k\frac{q_1q_2}{l^2}
            = 9 \cdot 10^{9}\,\frac{\text{Н}\cdot\text{м}^{2}}{\text{Кл}^{2}} \cdot \frac{4\,\text{нКл} \cdot 2\,\text{нКл}}{\sqr{ 3\,\text{см} }}
            = 8 \cdot 10^{31}\units{Н}
              \approx {8{,}00} \cdot 10^{31}\units{Н}
    $
}
\solutionspace{120pt}

\tasknumber{2}%
\task{%
    Два одинаковых маленьких проводящих заряженных шарика находятся на расстоянии~$r$ друг от друга.
    Заряд первого равен~$+3q$, второго~--- $+6q$.
    Шарики приводят в соприкосновение, а после опять разводят на расстояние~$2r$.
    \begin{itemize}
        \item Каким стал заряд каждого из шариков?
        \item Определите характер (притяжение или отталкивание) и силу взаимодействия шариков до и после соприкосновения.
        \item Как изменилась сила взаимодействия шариков после соприкосновения?
    \end{itemize}
}
\answer{%
    \begin{align*}
    F &= k\frac{\abs{q_1}\abs {q_2}}{\sqr{2 r}}   = k\frac{\abs{+3q} \cdot \abs{+6q}}{2^2 \cdot r^2}, \text{отталкивание}; \\
        q'_1 &= q'_2, q_1 + q_2 = q'_1 + q'_2 \implies  q'_1 = q'_2 = \frac{q_1 + q_2}2 = \frac{+3q + +6q}2 = \frac92q \implies \\
        \implies F'  &= k\frac{\abs{q'_1}\abs{q'_2}}{\sqr{2 r}}
            = k\frac{\sqr{\frac92q}}{2^2 \cdot r^2},
        \text{отталкивание}, \\
    \frac{F'}{F} &= \frac{\sqr{\frac92q}}{2^2 \cdot \abs{+3q} \cdot \abs{+6q}} = \frac9{32}.
    \end{align*}
}
\solutionspace{120pt}

\tasknumber{3}%
\task{%
    На координатной плоскости в точках $(-r; 0)$ и $(r; 0)$
    находятся заряды, соответственно, $-q$ и $-q$.
    Сделайте рисунок, определите величину напряжённости электрического поля
    и укажите её направление в двух точках: $(0; -r)$ и $(-2r; 0)$.
}
\solutionspace{120pt}

\tasknumber{4}%
\task{%
    Заряд $q_1$ создает в точке $A$ электрическое поле
    по величине равное~$E_1=7\funits{В}{м}$,
    а $q_2$~--- $E_2=24\funits{В}{м}$.
    Угол между векторами $\vect{E_1}$ и $\vect{E_2}$ равен $\varphi$.
    Определите величину суммарного электрического поля в точке $A$,
    создаваемого обоими зарядами $q_1$ и $q_2$.
    Сделайте рисунки и вычислите значение для двух значений угла $\varphi$:
    $\varphi_1=0^\circ$ и $\varphi_2=90^\circ$.
}

\variantsplitter

\addpersonalvariant{Михаил Ярошевский}

\tasknumber{1}%
\task{%
    С какой силой взаимодействуют 2 точечных заряда $q_1 = 2\,\text{нКл}$ и $q_2 = 3\,\text{нКл}$,
    находящиеся на расстоянии $d = 5\,\text{см}$?
}
\answer{%
    $
        F
            = k\frac{q_1q_2}{d^2}
            = 9 \cdot 10^{9}\,\frac{\text{Н}\cdot\text{м}^{2}}{\text{Кл}^{2}} \cdot \frac{2\,\text{нКл} \cdot 3\,\text{нКл}}{\sqr{ 5\,\text{см} }}
            = \frac{54}{25} \cdot 10^{31}\units{Н}
              \approx {2{,}16} \cdot 10^{31}\units{Н}
    $
}
\solutionspace{120pt}

\tasknumber{2}%
\task{%
    Два одинаковых маленьких проводящих заряженных шарика находятся на расстоянии~$r$ друг от друга.
    Заряд первого равен~$+3q$, второго~--- $+6q$.
    Шарики приводят в соприкосновение, а после опять разводят на расстояние~$2r$.
    \begin{itemize}
        \item Каким стал заряд каждого из шариков?
        \item Определите характер (притяжение или отталкивание) и силу взаимодействия шариков до и после соприкосновения.
        \item Как изменилась сила взаимодействия шариков после соприкосновения?
    \end{itemize}
}
\answer{%
    \begin{align*}
    F &= k\frac{\abs{q_1}\abs {q_2}}{\sqr{2 r}}   = k\frac{\abs{+3q} \cdot \abs{+6q}}{2^2 \cdot r^2}, \text{отталкивание}; \\
        q'_1 &= q'_2, q_1 + q_2 = q'_1 + q'_2 \implies  q'_1 = q'_2 = \frac{q_1 + q_2}2 = \frac{+3q + +6q}2 = \frac92q \implies \\
        \implies F'  &= k\frac{\abs{q'_1}\abs{q'_2}}{\sqr{2 r}}
            = k\frac{\sqr{\frac92q}}{2^2 \cdot r^2},
        \text{отталкивание}, \\
    \frac{F'}{F} &= \frac{\sqr{\frac92q}}{2^2 \cdot \abs{+3q} \cdot \abs{+6q}} = \frac9{32}.
    \end{align*}
}
\solutionspace{120pt}

\tasknumber{3}%
\task{%
    На координатной плоскости в точках $(-r; 0)$ и $(r; 0)$
    находятся заряды, соответственно, $-Q$ и $+Q$.
    Сделайте рисунок, определите величину напряжённости электрического поля
    и укажите её направление в двух точках: $(0; -r)$ и $(2r; 0)$.
}
\solutionspace{120pt}

\tasknumber{4}%
\task{%
    Заряд $q_1$ создает в точке $A$ электрическое поле
    по величине равное~$E_1=120\funits{В}{м}$,
    а $q_2$~--- $E_2=50\funits{В}{м}$.
    Угол между векторами $\vect{E_1}$ и $\vect{E_2}$ равен $\varphi$.
    Определите величину суммарного электрического поля в точке $A$,
    создаваемого обоими зарядами $q_1$ и $q_2$.
    Сделайте рисунки и вычислите значение для двух значений угла $\varphi$:
    $\varphi_1=90^\circ$ и $\varphi_2=180^\circ$.
}

\variantsplitter

\addpersonalvariant{Алексей Алимпиев}

\tasknumber{1}%
\task{%
    С какой силой взаимодействуют 2 точечных заряда $q_1 = 4\,\text{нКл}$ и $q_2 = 2\,\text{нКл}$,
    находящиеся на расстоянии $r = 2\,\text{см}$?
}
\answer{%
    $
        F
            = k\frac{q_1q_2}{r^2}
            = 9 \cdot 10^{9}\,\frac{\text{Н}\cdot\text{м}^{2}}{\text{Кл}^{2}} \cdot \frac{4\,\text{нКл} \cdot 2\,\text{нКл}}{\sqr{ 2\,\text{см} }}
            = 18 \cdot 10^{31}\units{Н}
              \approx {18{,}00} \cdot 10^{31}\units{Н}
    $
}
\solutionspace{120pt}

\tasknumber{2}%
\task{%
    Два одинаковых маленьких проводящих заряженных шарика находятся на расстоянии~$d$ друг от друга.
    Заряд первого равен~$+3Q$, второго~--- $+6Q$.
    Шарики приводят в соприкосновение, а после опять разводят на расстояние~$3d$.
    \begin{itemize}
        \item Каким стал заряд каждого из шариков?
        \item Определите характер (притяжение или отталкивание) и силу взаимодействия шариков до и после соприкосновения.
        \item Как изменилась сила взаимодействия шариков после соприкосновения?
    \end{itemize}
}
\answer{%
    \begin{align*}
    F &= k\frac{\abs{q_1}\abs {q_2}}{\sqr{3 d}}   = k\frac{\abs{+3Q} \cdot \abs{+6Q}}{3^2 \cdot d^2}, \text{отталкивание}; \\
        q'_1 &= q'_2, q_1 + q_2 = q'_1 + q'_2 \implies  q'_1 = q'_2 = \frac{q_1 + q_2}2 = \frac{+3Q + +6Q}2 = \frac92Q \implies \\
        \implies F'  &= k\frac{\abs{q'_1}\abs{q'_2}}{\sqr{3 d}}
            = k\frac{\sqr{\frac92Q}}{3^2 \cdot d^2},
        \text{отталкивание}, \\
    \frac{F'}{F} &= \frac{\sqr{\frac92Q}}{3^2 \cdot \abs{+3Q} \cdot \abs{+6Q}} = \frac18.
    \end{align*}
}
\solutionspace{120pt}

\tasknumber{3}%
\task{%
    На координатной плоскости в точках $(-r; 0)$ и $(r; 0)$
    находятся заряды, соответственно, $+Q$ и $+Q$.
    Сделайте рисунок, определите величину напряжённости электрического поля
    и укажите её направление в двух точках: $(0; r)$ и $(-2r; 0)$.
}
\solutionspace{120pt}

\tasknumber{4}%
\task{%
    Заряд $q_1$ создает в точке $A$ электрическое поле
    по величине равное~$E_1=7\funits{В}{м}$,
    а $q_2$~--- $E_2=24\funits{В}{м}$.
    Угол между векторами $\vect{E_1}$ и $\vect{E_2}$ равен $\alpha$.
    Определите величину суммарного электрического поля в точке $A$,
    создаваемого обоими зарядами $q_1$ и $q_2$.
    Сделайте рисунки и вычислите значение для двух значений угла $\alpha$:
    $\alpha_1=0^\circ$ и $\alpha_2=90^\circ$.
}

\variantsplitter

\addpersonalvariant{Евгений Васин}

\tasknumber{1}%
\task{%
    С какой силой взаимодействуют 2 точечных заряда $q_1 = 4\,\text{нКл}$ и $q_2 = 2\,\text{нКл}$,
    находящиеся на расстоянии $d = 2\,\text{см}$?
}
\answer{%
    $
        F
            = k\frac{q_1q_2}{d^2}
            = 9 \cdot 10^{9}\,\frac{\text{Н}\cdot\text{м}^{2}}{\text{Кл}^{2}} \cdot \frac{4\,\text{нКл} \cdot 2\,\text{нКл}}{\sqr{ 2\,\text{см} }}
            = 18 \cdot 10^{31}\units{Н}
              \approx {18{,}00} \cdot 10^{31}\units{Н}
    $
}
\solutionspace{120pt}

\tasknumber{2}%
\task{%
    Два одинаковых маленьких проводящих заряженных шарика находятся на расстоянии~$l$ друг от друга.
    Заряд первого равен~$+7Q$, второго~--- $-8Q$.
    Шарики приводят в соприкосновение, а после опять разводят на расстояние~$2l$.
    \begin{itemize}
        \item Каким стал заряд каждого из шариков?
        \item Определите характер (притяжение или отталкивание) и силу взаимодействия шариков до и после соприкосновения.
        \item Как изменилась сила взаимодействия шариков после соприкосновения?
    \end{itemize}
}
\answer{%
    \begin{align*}
    F &= k\frac{\abs{q_1}\abs {q_2}}{\sqr{2 l}}   = k\frac{\abs{+7Q} \cdot \abs{-8Q}}{2^2 \cdot l^2}, \text{притяжение}; \\
        q'_1 &= q'_2, q_1 + q_2 = q'_1 + q'_2 \implies  q'_1 = q'_2 = \frac{q_1 + q_2}2 = \frac{+7Q -8Q}2 = -\frac12Q \implies \\
        \implies F'  &= k\frac{\abs{q'_1}\abs{q'_2}}{\sqr{2 l}}
            = k\frac{\sqr{-\frac12Q}}{2^2 \cdot l^2},
        \text{отталкивание}, \\
    \frac{F'}{F} &= \frac{\sqr{-\frac12Q}}{2^2 \cdot \abs{+7Q} \cdot \abs{-8Q}} = \frac1{896}.
    \end{align*}
}
\solutionspace{120pt}

\tasknumber{3}%
\task{%
    На координатной плоскости в точках $(-l; 0)$ и $(l; 0)$
    находятся заряды, соответственно, $-q$ и $-q$.
    Сделайте рисунок, определите величину напряжённости электрического поля
    и укажите её направление в двух точках: $(0; l)$ и $(-2l; 0)$.
}
\solutionspace{120pt}

\tasknumber{4}%
\task{%
    Заряд $q_1$ создает в точке $A$ электрическое поле
    по величине равное~$E_1=7\funits{В}{м}$,
    а $q_2$~--- $E_2=24\funits{В}{м}$.
    Угол между векторами $\vect{E_1}$ и $\vect{E_2}$ равен $\alpha$.
    Определите величину суммарного электрического поля в точке $A$,
    создаваемого обоими зарядами $q_1$ и $q_2$.
    Сделайте рисунки и вычислите значение для двух значений угла $\alpha$:
    $\alpha_1=0^\circ$ и $\alpha_2=90^\circ$.
}

\variantsplitter

\addpersonalvariant{Вячеслав Волохов}

\tasknumber{1}%
\task{%
    С какой силой взаимодействуют 2 точечных заряда $q_1 = 2\,\text{нКл}$ и $q_2 = 3\,\text{нКл}$,
    находящиеся на расстоянии $r = 5\,\text{см}$?
}
\answer{%
    $
        F
            = k\frac{q_1q_2}{r^2}
            = 9 \cdot 10^{9}\,\frac{\text{Н}\cdot\text{м}^{2}}{\text{Кл}^{2}} \cdot \frac{2\,\text{нКл} \cdot 3\,\text{нКл}}{\sqr{ 5\,\text{см} }}
            = \frac{54}{25} \cdot 10^{31}\units{Н}
              \approx {2{,}16} \cdot 10^{31}\units{Н}
    $
}
\solutionspace{120pt}

\tasknumber{2}%
\task{%
    Два одинаковых маленьких проводящих заряженных шарика находятся на расстоянии~$r$ друг от друга.
    Заряд первого равен~$+3Q$, второго~--- $-4Q$.
    Шарики приводят в соприкосновение, а после опять разводят на расстояние~$4r$.
    \begin{itemize}
        \item Каким стал заряд каждого из шариков?
        \item Определите характер (притяжение или отталкивание) и силу взаимодействия шариков до и после соприкосновения.
        \item Как изменилась сила взаимодействия шариков после соприкосновения?
    \end{itemize}
}
\answer{%
    \begin{align*}
    F &= k\frac{\abs{q_1}\abs {q_2}}{\sqr{4 r}}   = k\frac{\abs{+3Q} \cdot \abs{-4Q}}{4^2 \cdot r^2}, \text{притяжение}; \\
        q'_1 &= q'_2, q_1 + q_2 = q'_1 + q'_2 \implies  q'_1 = q'_2 = \frac{q_1 + q_2}2 = \frac{+3Q -4Q}2 = -\frac12Q \implies \\
        \implies F'  &= k\frac{\abs{q'_1}\abs{q'_2}}{\sqr{4 r}}
            = k\frac{\sqr{-\frac12Q}}{4^2 \cdot r^2},
        \text{отталкивание}, \\
    \frac{F'}{F} &= \frac{\sqr{-\frac12Q}}{4^2 \cdot \abs{+3Q} \cdot \abs{-4Q}} = \frac1{768}.
    \end{align*}
}
\solutionspace{120pt}

\tasknumber{3}%
\task{%
    На координатной плоскости в точках $(-d; 0)$ и $(d; 0)$
    находятся заряды, соответственно, $-Q$ и $+Q$.
    Сделайте рисунок, определите величину напряжённости электрического поля
    и укажите её направление в двух точках: $(0; -d)$ и $(-2d; 0)$.
}
\solutionspace{120pt}

\tasknumber{4}%
\task{%
    Заряд $q_1$ создает в точке $A$ электрическое поле
    по величине равное~$E_1=120\funits{В}{м}$,
    а $q_2$~--- $E_2=50\funits{В}{м}$.
    Угол между векторами $\vect{E_1}$ и $\vect{E_2}$ равен $\varphi$.
    Определите величину суммарного электрического поля в точке $A$,
    создаваемого обоими зарядами $q_1$ и $q_2$.
    Сделайте рисунки и вычислите значение для двух значений угла $\varphi$:
    $\varphi_1=90^\circ$ и $\varphi_2=180^\circ$.
}

\variantsplitter

\addpersonalvariant{Герман Говоров}

\tasknumber{1}%
\task{%
    С какой силой взаимодействуют 2 точечных заряда $q_1 = 4\,\text{нКл}$ и $q_2 = 3\,\text{нКл}$,
    находящиеся на расстоянии $l = 3\,\text{см}$?
}
\answer{%
    $
        F
            = k\frac{q_1q_2}{l^2}
            = 9 \cdot 10^{9}\,\frac{\text{Н}\cdot\text{м}^{2}}{\text{Кл}^{2}} \cdot \frac{4\,\text{нКл} \cdot 3\,\text{нКл}}{\sqr{ 3\,\text{см} }}
            = 12 \cdot 10^{31}\units{Н}
              \approx {12{,}00} \cdot 10^{31}\units{Н}
    $
}
\solutionspace{120pt}

\tasknumber{2}%
\task{%
    Два одинаковых маленьких проводящих заряженных шарика находятся на расстоянии~$r$ друг от друга.
    Заряд первого равен~$-7q$, второго~--- $-6q$.
    Шарики приводят в соприкосновение, а после опять разводят на расстояние~$2r$.
    \begin{itemize}
        \item Каким стал заряд каждого из шариков?
        \item Определите характер (притяжение или отталкивание) и силу взаимодействия шариков до и после соприкосновения.
        \item Как изменилась сила взаимодействия шариков после соприкосновения?
    \end{itemize}
}
\answer{%
    \begin{align*}
    F &= k\frac{\abs{q_1}\abs {q_2}}{\sqr{2 r}}   = k\frac{\abs{-7q} \cdot \abs{-6q}}{2^2 \cdot r^2}, \text{отталкивание}; \\
        q'_1 &= q'_2, q_1 + q_2 = q'_1 + q'_2 \implies  q'_1 = q'_2 = \frac{q_1 + q_2}2 = \frac{-7q -6q}2 = -\frac{13}2q \implies \\
        \implies F'  &= k\frac{\abs{q'_1}\abs{q'_2}}{\sqr{2 r}}
            = k\frac{\sqr{-\frac{13}2q}}{2^2 \cdot r^2},
        \text{отталкивание}, \\
    \frac{F'}{F} &= \frac{\sqr{-\frac{13}2q}}{2^2 \cdot \abs{-7q} \cdot \abs{-6q}} = \frac{169}{672}.
    \end{align*}
}
\solutionspace{120pt}

\tasknumber{3}%
\task{%
    На координатной плоскости в точках $(-a; 0)$ и $(a; 0)$
    находятся заряды, соответственно, $-Q$ и $+Q$.
    Сделайте рисунок, определите величину напряжённости электрического поля
    и укажите её направление в двух точках: $(0; a)$ и $(-2a; 0)$.
}
\solutionspace{120pt}

\tasknumber{4}%
\task{%
    Заряд $q_1$ создает в точке $A$ электрическое поле
    по величине равное~$E_1=50\funits{В}{м}$,
    а $q_2$~--- $E_2=120\funits{В}{м}$.
    Угол между векторами $\vect{E_1}$ и $\vect{E_2}$ равен $\alpha$.
    Определите величину суммарного электрического поля в точке $A$,
    создаваемого обоими зарядами $q_1$ и $q_2$.
    Сделайте рисунки и вычислите значение для двух значений угла $\alpha$:
    $\alpha_1=0^\circ$ и $\alpha_2=90^\circ$.
}

\variantsplitter

\addpersonalvariant{София Журавлёва}

\tasknumber{1}%
\task{%
    С какой силой взаимодействуют 2 точечных заряда $q_1 = 4\,\text{нКл}$ и $q_2 = 3\,\text{нКл}$,
    находящиеся на расстоянии $l = 3\,\text{см}$?
}
\answer{%
    $
        F
            = k\frac{q_1q_2}{l^2}
            = 9 \cdot 10^{9}\,\frac{\text{Н}\cdot\text{м}^{2}}{\text{Кл}^{2}} \cdot \frac{4\,\text{нКл} \cdot 3\,\text{нКл}}{\sqr{ 3\,\text{см} }}
            = 12 \cdot 10^{31}\units{Н}
              \approx {12{,}00} \cdot 10^{31}\units{Н}
    $
}
\solutionspace{120pt}

\tasknumber{2}%
\task{%
    Два одинаковых маленьких проводящих заряженных шарика находятся на расстоянии~$d$ друг от друга.
    Заряд первого равен~$+5Q$, второго~--- $-4Q$.
    Шарики приводят в соприкосновение, а после опять разводят на расстояние~$3d$.
    \begin{itemize}
        \item Каким стал заряд каждого из шариков?
        \item Определите характер (притяжение или отталкивание) и силу взаимодействия шариков до и после соприкосновения.
        \item Как изменилась сила взаимодействия шариков после соприкосновения?
    \end{itemize}
}
\answer{%
    \begin{align*}
    F &= k\frac{\abs{q_1}\abs {q_2}}{\sqr{3 d}}   = k\frac{\abs{+5Q} \cdot \abs{-4Q}}{3^2 \cdot d^2}, \text{притяжение}; \\
        q'_1 &= q'_2, q_1 + q_2 = q'_1 + q'_2 \implies  q'_1 = q'_2 = \frac{q_1 + q_2}2 = \frac{+5Q -4Q}2 = \frac12Q \implies \\
        \implies F'  &= k\frac{\abs{q'_1}\abs{q'_2}}{\sqr{3 d}}
            = k\frac{\sqr{\frac12Q}}{3^2 \cdot d^2},
        \text{отталкивание}, \\
    \frac{F'}{F} &= \frac{\sqr{\frac12Q}}{3^2 \cdot \abs{+5Q} \cdot \abs{-4Q}} = \frac1{720}.
    \end{align*}
}
\solutionspace{120pt}

\tasknumber{3}%
\task{%
    На координатной плоскости в точках $(-d; 0)$ и $(d; 0)$
    находятся заряды, соответственно, $-q$ и $-q$.
    Сделайте рисунок, определите величину напряжённости электрического поля
    и укажите её направление в двух точках: $(0; -d)$ и $(-2d; 0)$.
}
\solutionspace{120pt}

\tasknumber{4}%
\task{%
    Заряд $q_1$ создает в точке $A$ электрическое поле
    по величине равное~$E_1=72\funits{В}{м}$,
    а $q_2$~--- $E_2=72\funits{В}{м}$.
    Угол между векторами $\vect{E_1}$ и $\vect{E_2}$ равен $\varphi$.
    Определите величину суммарного электрического поля в точке $A$,
    создаваемого обоими зарядами $q_1$ и $q_2$.
    Сделайте рисунки и вычислите значение для двух значений угла $\varphi$:
    $\varphi_1=0^\circ$ и $\varphi_2=120^\circ$.
}

\variantsplitter

\addpersonalvariant{Константин Козлов}

\tasknumber{1}%
\task{%
    С какой силой взаимодействуют 2 точечных заряда $q_1 = 4\,\text{нКл}$ и $q_2 = 2\,\text{нКл}$,
    находящиеся на расстоянии $l = 6\,\text{см}$?
}
\answer{%
    $
        F
            = k\frac{q_1q_2}{l^2}
            = 9 \cdot 10^{9}\,\frac{\text{Н}\cdot\text{м}^{2}}{\text{Кл}^{2}} \cdot \frac{4\,\text{нКл} \cdot 2\,\text{нКл}}{\sqr{ 6\,\text{см} }}
            = 2 \cdot 10^{31}\units{Н}
              \approx {2{,}00} \cdot 10^{31}\units{Н}
    $
}
\solutionspace{120pt}

\tasknumber{2}%
\task{%
    Два одинаковых маленьких проводящих заряженных шарика находятся на расстоянии~$l$ друг от друга.
    Заряд первого равен~$-3q$, второго~--- $-4q$.
    Шарики приводят в соприкосновение, а после опять разводят на расстояние~$2l$.
    \begin{itemize}
        \item Каким стал заряд каждого из шариков?
        \item Определите характер (притяжение или отталкивание) и силу взаимодействия шариков до и после соприкосновения.
        \item Как изменилась сила взаимодействия шариков после соприкосновения?
    \end{itemize}
}
\answer{%
    \begin{align*}
    F &= k\frac{\abs{q_1}\abs {q_2}}{\sqr{2 l}}   = k\frac{\abs{-3q} \cdot \abs{-4q}}{2^2 \cdot l^2}, \text{отталкивание}; \\
        q'_1 &= q'_2, q_1 + q_2 = q'_1 + q'_2 \implies  q'_1 = q'_2 = \frac{q_1 + q_2}2 = \frac{-3q -4q}2 = -\frac72q \implies \\
        \implies F'  &= k\frac{\abs{q'_1}\abs{q'_2}}{\sqr{2 l}}
            = k\frac{\sqr{-\frac72q}}{2^2 \cdot l^2},
        \text{отталкивание}, \\
    \frac{F'}{F} &= \frac{\sqr{-\frac72q}}{2^2 \cdot \abs{-3q} \cdot \abs{-4q}} = \frac{49}{192}.
    \end{align*}
}
\solutionspace{120pt}

\tasknumber{3}%
\task{%
    На координатной плоскости в точках $(-a; 0)$ и $(a; 0)$
    находятся заряды, соответственно, $+q$ и $-q$.
    Сделайте рисунок, определите величину напряжённости электрического поля
    и укажите её направление в двух точках: $(0; a)$ и $(-2a; 0)$.
}
\solutionspace{120pt}

\tasknumber{4}%
\task{%
    Заряд $q_1$ создает в точке $A$ электрическое поле
    по величине равное~$E_1=300\funits{В}{м}$,
    а $q_2$~--- $E_2=400\funits{В}{м}$.
    Угол между векторами $\vect{E_1}$ и $\vect{E_2}$ равен $\alpha$.
    Определите величину суммарного электрического поля в точке $A$,
    создаваемого обоими зарядами $q_1$ и $q_2$.
    Сделайте рисунки и вычислите значение для двух значений угла $\alpha$:
    $\alpha_1=0^\circ$ и $\alpha_2=90^\circ$.
}

\variantsplitter

\addpersonalvariant{Наталья Кравченко}

\tasknumber{1}%
\task{%
    С какой силой взаимодействуют 2 точечных заряда $q_1 = 4\,\text{нКл}$ и $q_2 = 2\,\text{нКл}$,
    находящиеся на расстоянии $r = 3\,\text{см}$?
}
\answer{%
    $
        F
            = k\frac{q_1q_2}{r^2}
            = 9 \cdot 10^{9}\,\frac{\text{Н}\cdot\text{м}^{2}}{\text{Кл}^{2}} \cdot \frac{4\,\text{нКл} \cdot 2\,\text{нКл}}{\sqr{ 3\,\text{см} }}
            = 8 \cdot 10^{31}\units{Н}
              \approx {8{,}00} \cdot 10^{31}\units{Н}
    $
}
\solutionspace{120pt}

\tasknumber{2}%
\task{%
    Два одинаковых маленьких проводящих заряженных шарика находятся на расстоянии~$r$ друг от друга.
    Заряд первого равен~$-5Q$, второго~--- $-4Q$.
    Шарики приводят в соприкосновение, а после опять разводят на расстояние~$2r$.
    \begin{itemize}
        \item Каким стал заряд каждого из шариков?
        \item Определите характер (притяжение или отталкивание) и силу взаимодействия шариков до и после соприкосновения.
        \item Как изменилась сила взаимодействия шариков после соприкосновения?
    \end{itemize}
}
\answer{%
    \begin{align*}
    F &= k\frac{\abs{q_1}\abs {q_2}}{\sqr{2 r}}   = k\frac{\abs{-5Q} \cdot \abs{-4Q}}{2^2 \cdot r^2}, \text{отталкивание}; \\
        q'_1 &= q'_2, q_1 + q_2 = q'_1 + q'_2 \implies  q'_1 = q'_2 = \frac{q_1 + q_2}2 = \frac{-5Q -4Q}2 = -\frac92Q \implies \\
        \implies F'  &= k\frac{\abs{q'_1}\abs{q'_2}}{\sqr{2 r}}
            = k\frac{\sqr{-\frac92Q}}{2^2 \cdot r^2},
        \text{отталкивание}, \\
    \frac{F'}{F} &= \frac{\sqr{-\frac92Q}}{2^2 \cdot \abs{-5Q} \cdot \abs{-4Q}} = \frac{81}{320}.
    \end{align*}
}
\solutionspace{120pt}

\tasknumber{3}%
\task{%
    На координатной плоскости в точках $(-d; 0)$ и $(d; 0)$
    находятся заряды, соответственно, $+q$ и $-q$.
    Сделайте рисунок, определите величину напряжённости электрического поля
    и укажите её направление в двух точках: $(0; d)$ и $(2d; 0)$.
}
\solutionspace{120pt}

\tasknumber{4}%
\task{%
    Заряд $q_1$ создает в точке $A$ электрическое поле
    по величине равное~$E_1=50\funits{В}{м}$,
    а $q_2$~--- $E_2=120\funits{В}{м}$.
    Угол между векторами $\vect{E_1}$ и $\vect{E_2}$ равен $\varphi$.
    Определите величину суммарного электрического поля в точке $A$,
    создаваемого обоими зарядами $q_1$ и $q_2$.
    Сделайте рисунки и вычислите значение для двух значений угла $\varphi$:
    $\varphi_1=0^\circ$ и $\varphi_2=90^\circ$.
}

\variantsplitter

\addpersonalvariant{Матвей Кузьмин}

\tasknumber{1}%
\task{%
    С какой силой взаимодействуют 2 точечных заряда $q_1 = 2\,\text{нКл}$ и $q_2 = 3\,\text{нКл}$,
    находящиеся на расстоянии $r = 2\,\text{см}$?
}
\answer{%
    $
        F
            = k\frac{q_1q_2}{r^2}
            = 9 \cdot 10^{9}\,\frac{\text{Н}\cdot\text{м}^{2}}{\text{Кл}^{2}} \cdot \frac{2\,\text{нКл} \cdot 3\,\text{нКл}}{\sqr{ 2\,\text{см} }}
            = \frac{27}2 \cdot 10^{31}\units{Н}
              \approx {13{,}50} \cdot 10^{31}\units{Н}
    $
}
\solutionspace{120pt}

\tasknumber{2}%
\task{%
    Два одинаковых маленьких проводящих заряженных шарика находятся на расстоянии~$r$ друг от друга.
    Заряд первого равен~$+7q$, второго~--- $-6q$.
    Шарики приводят в соприкосновение, а после опять разводят на расстояние~$4r$.
    \begin{itemize}
        \item Каким стал заряд каждого из шариков?
        \item Определите характер (притяжение или отталкивание) и силу взаимодействия шариков до и после соприкосновения.
        \item Как изменилась сила взаимодействия шариков после соприкосновения?
    \end{itemize}
}
\answer{%
    \begin{align*}
    F &= k\frac{\abs{q_1}\abs {q_2}}{\sqr{4 r}}   = k\frac{\abs{+7q} \cdot \abs{-6q}}{4^2 \cdot r^2}, \text{притяжение}; \\
        q'_1 &= q'_2, q_1 + q_2 = q'_1 + q'_2 \implies  q'_1 = q'_2 = \frac{q_1 + q_2}2 = \frac{+7q -6q}2 = \frac12q \implies \\
        \implies F'  &= k\frac{\abs{q'_1}\abs{q'_2}}{\sqr{4 r}}
            = k\frac{\sqr{\frac12q}}{4^2 \cdot r^2},
        \text{отталкивание}, \\
    \frac{F'}{F} &= \frac{\sqr{\frac12q}}{4^2 \cdot \abs{+7q} \cdot \abs{-6q}} = \frac1{2688}.
    \end{align*}
}
\solutionspace{120pt}

\tasknumber{3}%
\task{%
    На координатной плоскости в точках $(-r; 0)$ и $(r; 0)$
    находятся заряды, соответственно, $+q$ и $-q$.
    Сделайте рисунок, определите величину напряжённости электрического поля
    и укажите её направление в двух точках: $(0; r)$ и $(2r; 0)$.
}
\solutionspace{120pt}

\tasknumber{4}%
\task{%
    Заряд $q_1$ создает в точке $A$ электрическое поле
    по величине равное~$E_1=500\funits{В}{м}$,
    а $q_2$~--- $E_2=500\funits{В}{м}$.
    Угол между векторами $\vect{E_1}$ и $\vect{E_2}$ равен $\varphi$.
    Определите величину суммарного электрического поля в точке $A$,
    создаваемого обоими зарядами $q_1$ и $q_2$.
    Сделайте рисунки и вычислите значение для двух значений угла $\varphi$:
    $\varphi_1=0^\circ$ и $\varphi_2=120^\circ$.
}

\variantsplitter

\addpersonalvariant{Сергей Малышев}

\tasknumber{1}%
\task{%
    С какой силой взаимодействуют 2 точечных заряда $q_1 = 4\,\text{нКл}$ и $q_2 = 3\,\text{нКл}$,
    находящиеся на расстоянии $d = 5\,\text{см}$?
}
\answer{%
    $
        F
            = k\frac{q_1q_2}{d^2}
            = 9 \cdot 10^{9}\,\frac{\text{Н}\cdot\text{м}^{2}}{\text{Кл}^{2}} \cdot \frac{4\,\text{нКл} \cdot 3\,\text{нКл}}{\sqr{ 5\,\text{см} }}
            = \frac{108}{25} \cdot 10^{31}\units{Н}
              \approx {4{,}32} \cdot 10^{31}\units{Н}
    $
}
\solutionspace{120pt}

\tasknumber{2}%
\task{%
    Два одинаковых маленьких проводящих заряженных шарика находятся на расстоянии~$r$ друг от друга.
    Заряд первого равен~$-5q$, второго~--- $+8q$.
    Шарики приводят в соприкосновение, а после опять разводят на расстояние~$3r$.
    \begin{itemize}
        \item Каким стал заряд каждого из шариков?
        \item Определите характер (притяжение или отталкивание) и силу взаимодействия шариков до и после соприкосновения.
        \item Как изменилась сила взаимодействия шариков после соприкосновения?
    \end{itemize}
}
\answer{%
    \begin{align*}
    F &= k\frac{\abs{q_1}\abs {q_2}}{\sqr{3 r}}   = k\frac{\abs{-5q} \cdot \abs{+8q}}{3^2 \cdot r^2}, \text{притяжение}; \\
        q'_1 &= q'_2, q_1 + q_2 = q'_1 + q'_2 \implies  q'_1 = q'_2 = \frac{q_1 + q_2}2 = \frac{-5q + +8q}2 = \frac32q \implies \\
        \implies F'  &= k\frac{\abs{q'_1}\abs{q'_2}}{\sqr{3 r}}
            = k\frac{\sqr{\frac32q}}{3^2 \cdot r^2},
        \text{отталкивание}, \\
    \frac{F'}{F} &= \frac{\sqr{\frac32q}}{3^2 \cdot \abs{-5q} \cdot \abs{+8q}} = \frac1{160}.
    \end{align*}
}
\solutionspace{120pt}

\tasknumber{3}%
\task{%
    На координатной плоскости в точках $(-l; 0)$ и $(l; 0)$
    находятся заряды, соответственно, $-q$ и $-q$.
    Сделайте рисунок, определите величину напряжённости электрического поля
    и укажите её направление в двух точках: $(0; l)$ и $(-2l; 0)$.
}
\solutionspace{120pt}

\tasknumber{4}%
\task{%
    Заряд $q_1$ создает в точке $A$ электрическое поле
    по величине равное~$E_1=50\funits{В}{м}$,
    а $q_2$~--- $E_2=120\funits{В}{м}$.
    Угол между векторами $\vect{E_1}$ и $\vect{E_2}$ равен $\varphi$.
    Определите величину суммарного электрического поля в точке $A$,
    создаваемого обоими зарядами $q_1$ и $q_2$.
    Сделайте рисунки и вычислите значение для двух значений угла $\varphi$:
    $\varphi_1=0^\circ$ и $\varphi_2=90^\circ$.
}

\variantsplitter

\addpersonalvariant{Алина Полканова}

\tasknumber{1}%
\task{%
    С какой силой взаимодействуют 2 точечных заряда $q_1 = 4\,\text{нКл}$ и $q_2 = 3\,\text{нКл}$,
    находящиеся на расстоянии $l = 3\,\text{см}$?
}
\answer{%
    $
        F
            = k\frac{q_1q_2}{l^2}
            = 9 \cdot 10^{9}\,\frac{\text{Н}\cdot\text{м}^{2}}{\text{Кл}^{2}} \cdot \frac{4\,\text{нКл} \cdot 3\,\text{нКл}}{\sqr{ 3\,\text{см} }}
            = 12 \cdot 10^{31}\units{Н}
              \approx {12{,}00} \cdot 10^{31}\units{Н}
    $
}
\solutionspace{120pt}

\tasknumber{2}%
\task{%
    Два одинаковых маленьких проводящих заряженных шарика находятся на расстоянии~$r$ друг от друга.
    Заряд первого равен~$-3q$, второго~--- $-6q$.
    Шарики приводят в соприкосновение, а после опять разводят на расстояние~$2r$.
    \begin{itemize}
        \item Каким стал заряд каждого из шариков?
        \item Определите характер (притяжение или отталкивание) и силу взаимодействия шариков до и после соприкосновения.
        \item Как изменилась сила взаимодействия шариков после соприкосновения?
    \end{itemize}
}
\answer{%
    \begin{align*}
    F &= k\frac{\abs{q_1}\abs {q_2}}{\sqr{2 r}}   = k\frac{\abs{-3q} \cdot \abs{-6q}}{2^2 \cdot r^2}, \text{отталкивание}; \\
        q'_1 &= q'_2, q_1 + q_2 = q'_1 + q'_2 \implies  q'_1 = q'_2 = \frac{q_1 + q_2}2 = \frac{-3q -6q}2 = -\frac92q \implies \\
        \implies F'  &= k\frac{\abs{q'_1}\abs{q'_2}}{\sqr{2 r}}
            = k\frac{\sqr{-\frac92q}}{2^2 \cdot r^2},
        \text{отталкивание}, \\
    \frac{F'}{F} &= \frac{\sqr{-\frac92q}}{2^2 \cdot \abs{-3q} \cdot \abs{-6q}} = \frac9{32}.
    \end{align*}
}
\solutionspace{120pt}

\tasknumber{3}%
\task{%
    На координатной плоскости в точках $(-a; 0)$ и $(a; 0)$
    находятся заряды, соответственно, $+q$ и $-q$.
    Сделайте рисунок, определите величину напряжённости электрического поля
    и укажите её направление в двух точках: $(0; -a)$ и $(2a; 0)$.
}
\solutionspace{120pt}

\tasknumber{4}%
\task{%
    Заряд $q_1$ создает в точке $A$ электрическое поле
    по величине равное~$E_1=200\funits{В}{м}$,
    а $q_2$~--- $E_2=200\funits{В}{м}$.
    Угол между векторами $\vect{E_1}$ и $\vect{E_2}$ равен $\alpha$.
    Определите величину суммарного электрического поля в точке $A$,
    создаваемого обоими зарядами $q_1$ и $q_2$.
    Сделайте рисунки и вычислите значение для двух значений угла $\alpha$:
    $\alpha_1=0^\circ$ и $\alpha_2=60^\circ$.
}

\variantsplitter

\addpersonalvariant{Сергей Пономарёв}

\tasknumber{1}%
\task{%
    С какой силой взаимодействуют 2 точечных заряда $q_1 = 2\,\text{нКл}$ и $q_2 = 4\,\text{нКл}$,
    находящиеся на расстоянии $d = 2\,\text{см}$?
}
\answer{%
    $
        F
            = k\frac{q_1q_2}{d^2}
            = 9 \cdot 10^{9}\,\frac{\text{Н}\cdot\text{м}^{2}}{\text{Кл}^{2}} \cdot \frac{2\,\text{нКл} \cdot 4\,\text{нКл}}{\sqr{ 2\,\text{см} }}
            = 18 \cdot 10^{31}\units{Н}
              \approx {18{,}00} \cdot 10^{31}\units{Н}
    $
}
\solutionspace{120pt}

\tasknumber{2}%
\task{%
    Два одинаковых маленьких проводящих заряженных шарика находятся на расстоянии~$r$ друг от друга.
    Заряд первого равен~$+7Q$, второго~--- $-8Q$.
    Шарики приводят в соприкосновение, а после опять разводят на расстояние~$4r$.
    \begin{itemize}
        \item Каким стал заряд каждого из шариков?
        \item Определите характер (притяжение или отталкивание) и силу взаимодействия шариков до и после соприкосновения.
        \item Как изменилась сила взаимодействия шариков после соприкосновения?
    \end{itemize}
}
\answer{%
    \begin{align*}
    F &= k\frac{\abs{q_1}\abs {q_2}}{\sqr{4 r}}   = k\frac{\abs{+7Q} \cdot \abs{-8Q}}{4^2 \cdot r^2}, \text{притяжение}; \\
        q'_1 &= q'_2, q_1 + q_2 = q'_1 + q'_2 \implies  q'_1 = q'_2 = \frac{q_1 + q_2}2 = \frac{+7Q -8Q}2 = -\frac12Q \implies \\
        \implies F'  &= k\frac{\abs{q'_1}\abs{q'_2}}{\sqr{4 r}}
            = k\frac{\sqr{-\frac12Q}}{4^2 \cdot r^2},
        \text{отталкивание}, \\
    \frac{F'}{F} &= \frac{\sqr{-\frac12Q}}{4^2 \cdot \abs{+7Q} \cdot \abs{-8Q}} = \frac1{3584}.
    \end{align*}
}
\solutionspace{120pt}

\tasknumber{3}%
\task{%
    На координатной плоскости в точках $(-l; 0)$ и $(l; 0)$
    находятся заряды, соответственно, $-q$ и $-q$.
    Сделайте рисунок, определите величину напряжённости электрического поля
    и укажите её направление в двух точках: $(0; -l)$ и $(2l; 0)$.
}
\solutionspace{120pt}

\tasknumber{4}%
\task{%
    Заряд $q_1$ создает в точке $A$ электрическое поле
    по величине равное~$E_1=24\funits{В}{м}$,
    а $q_2$~--- $E_2=7\funits{В}{м}$.
    Угол между векторами $\vect{E_1}$ и $\vect{E_2}$ равен $\alpha$.
    Определите величину суммарного электрического поля в точке $A$,
    создаваемого обоими зарядами $q_1$ и $q_2$.
    Сделайте рисунки и вычислите значение для двух значений угла $\alpha$:
    $\alpha_1=90^\circ$ и $\alpha_2=180^\circ$.
}

\variantsplitter

\addpersonalvariant{Егор Свистушкин}

\tasknumber{1}%
\task{%
    С какой силой взаимодействуют 2 точечных заряда $q_1 = 2\,\text{нКл}$ и $q_2 = 4\,\text{нКл}$,
    находящиеся на расстоянии $l = 6\,\text{см}$?
}
\answer{%
    $
        F
            = k\frac{q_1q_2}{l^2}
            = 9 \cdot 10^{9}\,\frac{\text{Н}\cdot\text{м}^{2}}{\text{Кл}^{2}} \cdot \frac{2\,\text{нКл} \cdot 4\,\text{нКл}}{\sqr{ 6\,\text{см} }}
            = 2 \cdot 10^{31}\units{Н}
              \approx {2{,}00} \cdot 10^{31}\units{Н}
    $
}
\solutionspace{120pt}

\tasknumber{2}%
\task{%
    Два одинаковых маленьких проводящих заряженных шарика находятся на расстоянии~$d$ друг от друга.
    Заряд первого равен~$-3Q$, второго~--- $-4Q$.
    Шарики приводят в соприкосновение, а после опять разводят на расстояние~$3d$.
    \begin{itemize}
        \item Каким стал заряд каждого из шариков?
        \item Определите характер (притяжение или отталкивание) и силу взаимодействия шариков до и после соприкосновения.
        \item Как изменилась сила взаимодействия шариков после соприкосновения?
    \end{itemize}
}
\answer{%
    \begin{align*}
    F &= k\frac{\abs{q_1}\abs {q_2}}{\sqr{3 d}}   = k\frac{\abs{-3Q} \cdot \abs{-4Q}}{3^2 \cdot d^2}, \text{отталкивание}; \\
        q'_1 &= q'_2, q_1 + q_2 = q'_1 + q'_2 \implies  q'_1 = q'_2 = \frac{q_1 + q_2}2 = \frac{-3Q -4Q}2 = -\frac72Q \implies \\
        \implies F'  &= k\frac{\abs{q'_1}\abs{q'_2}}{\sqr{3 d}}
            = k\frac{\sqr{-\frac72Q}}{3^2 \cdot d^2},
        \text{отталкивание}, \\
    \frac{F'}{F} &= \frac{\sqr{-\frac72Q}}{3^2 \cdot \abs{-3Q} \cdot \abs{-4Q}} = \frac{49}{432}.
    \end{align*}
}
\solutionspace{120pt}

\tasknumber{3}%
\task{%
    На координатной плоскости в точках $(-a; 0)$ и $(a; 0)$
    находятся заряды, соответственно, $-Q$ и $+Q$.
    Сделайте рисунок, определите величину напряжённости электрического поля
    и укажите её направление в двух точках: $(0; a)$ и $(-2a; 0)$.
}
\solutionspace{120pt}

\tasknumber{4}%
\task{%
    Заряд $q_1$ создает в точке $A$ электрическое поле
    по величине равное~$E_1=300\funits{В}{м}$,
    а $q_2$~--- $E_2=400\funits{В}{м}$.
    Угол между векторами $\vect{E_1}$ и $\vect{E_2}$ равен $\alpha$.
    Определите величину суммарного электрического поля в точке $A$,
    создаваемого обоими зарядами $q_1$ и $q_2$.
    Сделайте рисунки и вычислите значение для двух значений угла $\alpha$:
    $\alpha_1=90^\circ$ и $\alpha_2=180^\circ$.
}

\variantsplitter

\addpersonalvariant{Дмитрий Соколов}

\tasknumber{1}%
\task{%
    С какой силой взаимодействуют 2 точечных заряда $q_1 = 2\,\text{нКл}$ и $q_2 = 4\,\text{нКл}$,
    находящиеся на расстоянии $l = 3\,\text{см}$?
}
\answer{%
    $
        F
            = k\frac{q_1q_2}{l^2}
            = 9 \cdot 10^{9}\,\frac{\text{Н}\cdot\text{м}^{2}}{\text{Кл}^{2}} \cdot \frac{2\,\text{нКл} \cdot 4\,\text{нКл}}{\sqr{ 3\,\text{см} }}
            = 8 \cdot 10^{31}\units{Н}
              \approx {8{,}00} \cdot 10^{31}\units{Н}
    $
}
\solutionspace{120pt}

\tasknumber{2}%
\task{%
    Два одинаковых маленьких проводящих заряженных шарика находятся на расстоянии~$d$ друг от друга.
    Заряд первого равен~$-7q$, второго~--- $-4q$.
    Шарики приводят в соприкосновение, а после опять разводят на расстояние~$2d$.
    \begin{itemize}
        \item Каким стал заряд каждого из шариков?
        \item Определите характер (притяжение или отталкивание) и силу взаимодействия шариков до и после соприкосновения.
        \item Как изменилась сила взаимодействия шариков после соприкосновения?
    \end{itemize}
}
\answer{%
    \begin{align*}
    F &= k\frac{\abs{q_1}\abs {q_2}}{\sqr{2 d}}   = k\frac{\abs{-7q} \cdot \abs{-4q}}{2^2 \cdot d^2}, \text{отталкивание}; \\
        q'_1 &= q'_2, q_1 + q_2 = q'_1 + q'_2 \implies  q'_1 = q'_2 = \frac{q_1 + q_2}2 = \frac{-7q -4q}2 = -\frac{11}2q \implies \\
        \implies F'  &= k\frac{\abs{q'_1}\abs{q'_2}}{\sqr{2 d}}
            = k\frac{\sqr{-\frac{11}2q}}{2^2 \cdot d^2},
        \text{отталкивание}, \\
    \frac{F'}{F} &= \frac{\sqr{-\frac{11}2q}}{2^2 \cdot \abs{-7q} \cdot \abs{-4q}} = \frac{121}{448}.
    \end{align*}
}
\solutionspace{120pt}

\tasknumber{3}%
\task{%
    На координатной плоскости в точках $(-r; 0)$ и $(r; 0)$
    находятся заряды, соответственно, $+q$ и $-q$.
    Сделайте рисунок, определите величину напряжённости электрического поля
    и укажите её направление в двух точках: $(0; -r)$ и $(-2r; 0)$.
}
\solutionspace{120pt}

\tasknumber{4}%
\task{%
    Заряд $q_1$ создает в точке $A$ электрическое поле
    по величине равное~$E_1=50\funits{В}{м}$,
    а $q_2$~--- $E_2=120\funits{В}{м}$.
    Угол между векторами $\vect{E_1}$ и $\vect{E_2}$ равен $\varphi$.
    Определите величину суммарного электрического поля в точке $A$,
    создаваемого обоими зарядами $q_1$ и $q_2$.
    Сделайте рисунки и вычислите значение для двух значений угла $\varphi$:
    $\varphi_1=0^\circ$ и $\varphi_2=90^\circ$.
}

\variantsplitter

\addpersonalvariant{Арсений Трофимов}

\tasknumber{1}%
\task{%
    С какой силой взаимодействуют 2 точечных заряда $q_1 = 4\,\text{нКл}$ и $q_2 = 3\,\text{нКл}$,
    находящиеся на расстоянии $d = 6\,\text{см}$?
}
\answer{%
    $
        F
            = k\frac{q_1q_2}{d^2}
            = 9 \cdot 10^{9}\,\frac{\text{Н}\cdot\text{м}^{2}}{\text{Кл}^{2}} \cdot \frac{4\,\text{нКл} \cdot 3\,\text{нКл}}{\sqr{ 6\,\text{см} }}
            = 3 \cdot 10^{31}\units{Н}
              \approx {3{,}00} \cdot 10^{31}\units{Н}
    $
}
\solutionspace{120pt}

\tasknumber{2}%
\task{%
    Два одинаковых маленьких проводящих заряженных шарика находятся на расстоянии~$r$ друг от друга.
    Заряд первого равен~$-3q$, второго~--- $+4q$.
    Шарики приводят в соприкосновение, а после опять разводят на расстояние~$3r$.
    \begin{itemize}
        \item Каким стал заряд каждого из шариков?
        \item Определите характер (притяжение или отталкивание) и силу взаимодействия шариков до и после соприкосновения.
        \item Как изменилась сила взаимодействия шариков после соприкосновения?
    \end{itemize}
}
\answer{%
    \begin{align*}
    F &= k\frac{\abs{q_1}\abs {q_2}}{\sqr{3 r}}   = k\frac{\abs{-3q} \cdot \abs{+4q}}{3^2 \cdot r^2}, \text{притяжение}; \\
        q'_1 &= q'_2, q_1 + q_2 = q'_1 + q'_2 \implies  q'_1 = q'_2 = \frac{q_1 + q_2}2 = \frac{-3q + +4q}2 = \frac12q \implies \\
        \implies F'  &= k\frac{\abs{q'_1}\abs{q'_2}}{\sqr{3 r}}
            = k\frac{\sqr{\frac12q}}{3^2 \cdot r^2},
        \text{отталкивание}, \\
    \frac{F'}{F} &= \frac{\sqr{\frac12q}}{3^2 \cdot \abs{-3q} \cdot \abs{+4q}} = \frac1{432}.
    \end{align*}
}
\solutionspace{120pt}

\tasknumber{3}%
\task{%
    На координатной плоскости в точках $(-d; 0)$ и $(d; 0)$
    находятся заряды, соответственно, $+q$ и $-q$.
    Сделайте рисунок, определите величину напряжённости электрического поля
    и укажите её направление в двух точках: $(0; -d)$ и $(2d; 0)$.
}
\solutionspace{120pt}

\tasknumber{4}%
\task{%
    Заряд $q_1$ создает в точке $A$ электрическое поле
    по величине равное~$E_1=120\funits{В}{м}$,
    а $q_2$~--- $E_2=50\funits{В}{м}$.
    Угол между векторами $\vect{E_1}$ и $\vect{E_2}$ равен $\alpha$.
    Определите величину суммарного электрического поля в точке $A$,
    создаваемого обоими зарядами $q_1$ и $q_2$.
    Сделайте рисунки и вычислите значение для двух значений угла $\alpha$:
    $\alpha_1=90^\circ$ и $\alpha_2=180^\circ$.
}
% autogenerated
