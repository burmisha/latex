\setdate{23~марта~2021}
\setclass{9«М»}

\addpersonalvariant{Михаил Бурмистров}

\tasknumber{1}%
\task{%
    Сколько фотонов испускает за 10 минут лазер,
    если мощность его излучения $75\,\text{мВт}$?
    Длина волны излучения $750\,\text{нм}$.
    $h = 6{,}626 \cdot 10^{-34}\,\text{Дж}\cdot\text{с}$.
}
\answer{%
    $
        N
            = \frac{E_{\text{общая}}}{E_{\text{одного фотона}}}
            = \frac{Pt}{h\nu} = \frac{Pt}{h \frac c\lambda}
            = \frac{Pt\lambda}{hc}
            = \frac{
                75\,\text{мВт} \cdot 10 \cdot 60 \units{с} \cdot 750\,\text{нм}
            }{
                6{,}626 \cdot 10^{-34}\,\text{Дж}\cdot\text{с} \cdot 3 \cdot 10^{8}\,\frac{\text{м}}{\text{с}}
            }
            \approx 1{,}70 \cdot 10^{20}\units{фотонов}
    $
}
\solutionspace{120pt}

\tasknumber{2}%
\task{%
    Определите название цвета по длине волны в вакууме
    и частоту колебаний электромагнитного поля в ней:
    \begin{enumerate}
        \item $660\,\text{нм}$,
        \item $390\,\text{нм}$,
        \item $610\,\text{нм}$,
        \item $420\,\text{нм}$.
    \end{enumerate}
}
\answer{%
    \begin{enumerate}
        \item $660\,\text{нм} \to$ красный, $\nu_1 = \frac c{\lambda_1} \approx 4{,}55 \cdot 10^{14}\,\text{Гц}$,
        \item $390\,\text{нм} \to$ фиолетовый, $\nu_2 = \frac c{\lambda_2} \approx 7{,}69 \cdot 10^{14}\,\text{Гц}$,
        \item $610\,\text{нм} \to$ оранжевый, $\nu_3 = \frac c{\lambda_3} \approx 4{,}92 \cdot 10^{14}\,\text{Гц}$,
        \item $420\,\text{нм} \to$ фиолетовый, $\nu_4 = \frac c{\lambda_4} \approx 7{,}14 \cdot 10^{14}\,\text{Гц}$.
    \end{enumerate}

    $\nu = \frac 1 T = \frac c{\lambda} = \frac {2{,}998 \cdot 10^{8}\,\frac{\text{м}}{\text{с}}}{l \cdot 1\,\text{мкм}} \approx \frac{2{,}998 \cdot 10^{14}\,\text{Гц}}l$,
    где $l$~--- численное значение длины волны в мкм.
}
\solutionspace{80pt}

\tasknumber{3}%
\task{%
    Определите энергию фотона излучения частотой $6 \cdot 10^{16}\,\text{Гц}$.
    Ответ получите в джоулях и в электронвольтах.
}
\answer{%
    $E = h \nu = 6{,}626 \cdot 10^{-34}\,\text{Дж}\cdot\text{с} \cdot 6 \cdot 10^{16}\,\text{Гц} \approx 40 \cdot 10^{-18}\,\text{Дж} \approx 250\,\text{эВ}$
}
\solutionspace{80pt}

\tasknumber{4}%
\task{%
    Определите энергию фотона с длиной волны $150\,\text{нм}$.
    Ответ выразите в электронвольтах.
    Способен ли человеческий глаз увидеть один такой квант? А импульс таких квантов?'
}
\answer{%
    $E = h\nu = \frac{hc}{\lambda} = \frac{6{,}626 \cdot 10^{-34}\,\text{Дж}\cdot\text{с} \cdot 3 \cdot 10^{8}\,\frac{\text{м}}{\text{с}}}{150\,\text{нм}} \approx 0{,}133 \cdot 10^{-17}\,\text{Дж} \approx 8{,}3\,\text{эВ}$
}
\solutionspace{80pt}

\tasknumber{5}%
\task{%
    Определите частоту колебаний вектора напряженности индукции магнитного поля
    в электромагнитной волне в вакууме, длина который составляет $5\,\text{см}$.
}
\answer{%
    \begin{align*}
    \lambda &= c T \implies T = \frac{\lambda}c = \frac{5\,\text{см}}{3 \cdot 10^{8}\,\frac{\text{м}}{\text{с}}} = 1{,}7 \cdot 10^{-10}\,\text{с}, \\
    \lambda &= c T = c \cdot \frac 1\nu \implies \nu = \frac c{\lambda} = \frac{3 \cdot 10^{8}\,\frac{\text{м}}{\text{с}}}{5\,\text{см}} = 0{,}6 \cdot 10^{10}\,\text{Гц}.
    \end{align*}
}
\solutionspace{80pt}

\tasknumber{6}%
\task{%
    Из формулы Планка выразите (нужен вывод, не только ответ)...
    \begin{enumerate}
        \item длину соответствующей электромагнитной волны,
        \item период колебаний индукции магнитного поля в соответствующей электромагнитной волне.
    \end{enumerate}
}

\variantsplitter

\addpersonalvariant{Артём Глембо}

\tasknumber{1}%
\task{%
    Сколько фотонов испускает за 5 минут лазер,
    если мощность его излучения $200\,\text{мВт}$?
    Длина волны излучения $500\,\text{нм}$.
    $h = 6{,}626 \cdot 10^{-34}\,\text{Дж}\cdot\text{с}$.
}
\answer{%
    $
        N
            = \frac{E_{\text{общая}}}{E_{\text{одного фотона}}}
            = \frac{Pt}{h\nu} = \frac{Pt}{h \frac c\lambda}
            = \frac{Pt\lambda}{hc}
            = \frac{
                200\,\text{мВт} \cdot 5 \cdot 60 \units{с} \cdot 500\,\text{нм}
            }{
                6{,}626 \cdot 10^{-34}\,\text{Дж}\cdot\text{с} \cdot 3 \cdot 10^{8}\,\frac{\text{м}}{\text{с}}
            }
            \approx 1{,}51 \cdot 10^{20}\units{фотонов}
    $
}
\solutionspace{120pt}

\tasknumber{2}%
\task{%
    Определите название цвета по длине волны в вакууме
    и частоту колебаний электромагнитного поля в ней:
    \begin{enumerate}
        \item $450\,\text{нм}$,
        \item $390\,\text{нм}$,
        \item $610\,\text{нм}$,
        \item $420\,\text{нм}$.
    \end{enumerate}
}
\answer{%
    \begin{enumerate}
        \item $450\,\text{нм} \to$ синий, $\nu_1 = \frac c{\lambda_1} \approx 6{,}67 \cdot 10^{14}\,\text{Гц}$,
        \item $390\,\text{нм} \to$ фиолетовый, $\nu_2 = \frac c{\lambda_2} \approx 7{,}69 \cdot 10^{14}\,\text{Гц}$,
        \item $610\,\text{нм} \to$ оранжевый, $\nu_3 = \frac c{\lambda_3} \approx 4{,}92 \cdot 10^{14}\,\text{Гц}$,
        \item $420\,\text{нм} \to$ фиолетовый, $\nu_4 = \frac c{\lambda_4} \approx 7{,}14 \cdot 10^{14}\,\text{Гц}$.
    \end{enumerate}

    $\nu = \frac 1 T = \frac c{\lambda} = \frac {2{,}998 \cdot 10^{8}\,\frac{\text{м}}{\text{с}}}{l \cdot 1\,\text{мкм}} \approx \frac{2{,}998 \cdot 10^{14}\,\text{Гц}}l$,
    где $l$~--- численное значение длины волны в мкм.
}
\solutionspace{80pt}

\tasknumber{3}%
\task{%
    Определите энергию фотона излучения частотой $7 \cdot 10^{16}\,\text{Гц}$.
    Ответ получите в джоулях и в электронвольтах.
}
\answer{%
    $E = h \nu = 6{,}626 \cdot 10^{-34}\,\text{Дж}\cdot\text{с} \cdot 7 \cdot 10^{16}\,\text{Гц} \approx 46 \cdot 10^{-18}\,\text{Дж} \approx 290\,\text{эВ}$
}
\solutionspace{80pt}

\tasknumber{4}%
\task{%
    Определите энергию фотона с длиной волны $400\,\text{нм}$.
    Ответ выразите в джоулях.
    Способен ли человеческий глаз увидеть один такой квант? А импульс таких квантов?'
}
\answer{%
    $E = h\nu = \frac{hc}{\lambda} = \frac{6{,}626 \cdot 10^{-34}\,\text{Дж}\cdot\text{с} \cdot 3 \cdot 10^{8}\,\frac{\text{м}}{\text{с}}}{400\,\text{нм}} \approx 0{,}0497 \cdot 10^{-17}\,\text{Дж} \approx 3{,}1\,\text{эВ}$
}
\solutionspace{80pt}

\tasknumber{5}%
\task{%
    Определите частоту колебаний вектора напряженности электрического поля
    в электромагнитной волне в вакууме, длина который составляет $2\,\text{м}$.
}
\answer{%
    \begin{align*}
    \lambda &= c T \implies T = \frac{\lambda}c = \frac{2\,\text{м}}{3 \cdot 10^{8}\,\frac{\text{м}}{\text{с}}} = 0{,}7 \cdot 10^{-8}\,\text{с}, \\
    \lambda &= c T = c \cdot \frac 1\nu \implies \nu = \frac c{\lambda} = \frac{3 \cdot 10^{8}\,\frac{\text{м}}{\text{с}}}{2\,\text{м}} = 1{,}5 \cdot 10^{8}\,\text{Гц}.
    \end{align*}
}
\solutionspace{80pt}

\tasknumber{6}%
\task{%
    Из формулы Планка выразите (нужен вывод, не только ответ)...
    \begin{enumerate}
        \item длину соответствующей электромагнитной волны,
        \item период колебаний индукции магнитного поля в соответствующей электромагнитной волне.
    \end{enumerate}
}

\variantsplitter

\addpersonalvariant{Наталья Гончарова}

\tasknumber{1}%
\task{%
    Сколько фотонов испускает за 30 минут лазер,
    если мощность его излучения $75\,\text{мВт}$?
    Длина волны излучения $600\,\text{нм}$.
    $h = 6{,}626 \cdot 10^{-34}\,\text{Дж}\cdot\text{с}$.
}
\answer{%
    $
        N
            = \frac{E_{\text{общая}}}{E_{\text{одного фотона}}}
            = \frac{Pt}{h\nu} = \frac{Pt}{h \frac c\lambda}
            = \frac{Pt\lambda}{hc}
            = \frac{
                75\,\text{мВт} \cdot 30 \cdot 60 \units{с} \cdot 600\,\text{нм}
            }{
                6{,}626 \cdot 10^{-34}\,\text{Дж}\cdot\text{с} \cdot 3 \cdot 10^{8}\,\frac{\text{м}}{\text{с}}
            }
            \approx 4{,}07 \cdot 10^{20}\units{фотонов}
    $
}
\solutionspace{120pt}

\tasknumber{2}%
\task{%
    Определите название цвета по длине волны в вакууме
    и частоту колебаний электромагнитного поля в ней:
    \begin{enumerate}
        \item $660\,\text{нм}$,
        \item $595\,\text{нм}$,
        \item $610\,\text{нм}$,
        \item $420\,\text{нм}$.
    \end{enumerate}
}
\answer{%
    \begin{enumerate}
        \item $660\,\text{нм} \to$ красный, $\nu_1 = \frac c{\lambda_1} \approx 4{,}55 \cdot 10^{14}\,\text{Гц}$,
        \item $595\,\text{нм} \to$ оранжевый, $\nu_2 = \frac c{\lambda_2} \approx 5{,}04 \cdot 10^{14}\,\text{Гц}$,
        \item $610\,\text{нм} \to$ оранжевый, $\nu_3 = \frac c{\lambda_3} \approx 4{,}92 \cdot 10^{14}\,\text{Гц}$,
        \item $420\,\text{нм} \to$ фиолетовый, $\nu_4 = \frac c{\lambda_4} \approx 7{,}14 \cdot 10^{14}\,\text{Гц}$.
    \end{enumerate}

    $\nu = \frac 1 T = \frac c{\lambda} = \frac {2{,}998 \cdot 10^{8}\,\frac{\text{м}}{\text{с}}}{l \cdot 1\,\text{мкм}} \approx \frac{2{,}998 \cdot 10^{14}\,\text{Гц}}l$,
    где $l$~--- численное значение длины волны в мкм.
}
\solutionspace{80pt}

\tasknumber{3}%
\task{%
    Определите энергию фотона излучения частотой $8 \cdot 10^{16}\,\text{Гц}$.
    Ответ получите в джоулях и в электронвольтах.
}
\answer{%
    $E = h \nu = 6{,}626 \cdot 10^{-34}\,\text{Дж}\cdot\text{с} \cdot 8 \cdot 10^{16}\,\text{Гц} \approx 53 \cdot 10^{-18}\,\text{Дж} \approx 330\,\text{эВ}$
}
\solutionspace{80pt}

\tasknumber{4}%
\task{%
    Определите энергию кванта света с длиной волны $200\,\text{нм}$.
    Ответ выразите в электронвольтах.
    Способен ли человеческий глаз увидеть один такой квант? А импульс таких квантов?'
}
\answer{%
    $E = h\nu = \frac{hc}{\lambda} = \frac{6{,}626 \cdot 10^{-34}\,\text{Дж}\cdot\text{с} \cdot 3 \cdot 10^{8}\,\frac{\text{м}}{\text{с}}}{200\,\text{нм}} \approx 0{,}0994 \cdot 10^{-17}\,\text{Дж} \approx 6{,}2\,\text{эВ}$
}
\solutionspace{80pt}

\tasknumber{5}%
\task{%
    Определите период колебаний вектора напряженности электрического поля
    в электромагнитной волне в вакууме, длина который составляет $3\,\text{см}$.
}
\answer{%
    \begin{align*}
    \lambda &= c T \implies T = \frac{\lambda}c = \frac{3\,\text{см}}{3 \cdot 10^{8}\,\frac{\text{м}}{\text{с}}} = 1{,}0 \cdot 10^{-10}\,\text{с}, \\
    \lambda &= c T = c \cdot \frac 1\nu \implies \nu = \frac c{\lambda} = \frac{3 \cdot 10^{8}\,\frac{\text{м}}{\text{с}}}{3\,\text{см}} = 1{,}0 \cdot 10^{10}\,\text{Гц}.
    \end{align*}
}
\solutionspace{80pt}

\tasknumber{6}%
\task{%
    Из формулы Планка выразите (нужен вывод, не только ответ)...
    \begin{enumerate}
        \item длину соответствующей электромагнитной волны,
        \item период колебаний индукции магнитного поля в соответствующей электромагнитной волне.
    \end{enumerate}
}

\variantsplitter

\addpersonalvariant{Файёзбек Касымов}

\tasknumber{1}%
\task{%
    Сколько фотонов испускает за 40 минут лазер,
    если мощность его излучения $200\,\text{мВт}$?
    Длина волны излучения $600\,\text{нм}$.
    $h = 6{,}626 \cdot 10^{-34}\,\text{Дж}\cdot\text{с}$.
}
\answer{%
    $
        N
            = \frac{E_{\text{общая}}}{E_{\text{одного фотона}}}
            = \frac{Pt}{h\nu} = \frac{Pt}{h \frac c\lambda}
            = \frac{Pt\lambda}{hc}
            = \frac{
                200\,\text{мВт} \cdot 40 \cdot 60 \units{с} \cdot 600\,\text{нм}
            }{
                6{,}626 \cdot 10^{-34}\,\text{Дж}\cdot\text{с} \cdot 3 \cdot 10^{8}\,\frac{\text{м}}{\text{с}}
            }
            \approx 14{,}49 \cdot 10^{20}\units{фотонов}
    $
}
\solutionspace{120pt}

\tasknumber{2}%
\task{%
    Определите название цвета по длине волны в вакууме
    и частоту колебаний электромагнитного поля в ней:
    \begin{enumerate}
        \item $660\,\text{нм}$,
        \item $470\,\text{нм}$,
        \item $530\,\text{нм}$,
        \item $580\,\text{нм}$.
    \end{enumerate}
}
\answer{%
    \begin{enumerate}
        \item $660\,\text{нм} \to$ красный, $\nu_1 = \frac c{\lambda_1} \approx 4{,}55 \cdot 10^{14}\,\text{Гц}$,
        \item $470\,\text{нм} \to$ синий, $\nu_2 = \frac c{\lambda_2} \approx 6{,}38 \cdot 10^{14}\,\text{Гц}$,
        \item $530\,\text{нм} \to$ зелёный, $\nu_3 = \frac c{\lambda_3} \approx 5{,}66 \cdot 10^{14}\,\text{Гц}$,
        \item $580\,\text{нм} \to$ зелёный, $\nu_4 = \frac c{\lambda_4} \approx 5{,}17 \cdot 10^{14}\,\text{Гц}$.
    \end{enumerate}

    $\nu = \frac 1 T = \frac c{\lambda} = \frac {2{,}998 \cdot 10^{8}\,\frac{\text{м}}{\text{с}}}{l \cdot 1\,\text{мкм}} \approx \frac{2{,}998 \cdot 10^{14}\,\text{Гц}}l$,
    где $l$~--- численное значение длины волны в мкм.
}
\solutionspace{80pt}

\tasknumber{3}%
\task{%
    Определите энергию фотона излучения частотой $4 \cdot 10^{16}\,\text{Гц}$.
    Ответ получите в джоулях и в электронвольтах.
}
\answer{%
    $E = h \nu = 6{,}626 \cdot 10^{-34}\,\text{Дж}\cdot\text{с} \cdot 4 \cdot 10^{16}\,\text{Гц} \approx 27 \cdot 10^{-18}\,\text{Дж} \approx 166\,\text{эВ}$
}
\solutionspace{80pt}

\tasknumber{4}%
\task{%
    Определите энергию фотона с длиной волны $500\,\text{нм}$.
    Ответ выразите в джоулях.
    Способен ли человеческий глаз увидеть один такой квант? А импульс таких квантов?'
}
\answer{%
    $E = h\nu = \frac{hc}{\lambda} = \frac{6{,}626 \cdot 10^{-34}\,\text{Дж}\cdot\text{с} \cdot 3 \cdot 10^{8}\,\frac{\text{м}}{\text{с}}}{500\,\text{нм}} \approx 0{,}0398 \cdot 10^{-17}\,\text{Дж} \approx 2{,}5\,\text{эВ}$
}
\solutionspace{80pt}

\tasknumber{5}%
\task{%
    Определите период колебаний вектора напряженности индукции магнитного поля
    в электромагнитной волне в вакууме, длина который составляет $3\,\text{м}$.
}
\answer{%
    \begin{align*}
    \lambda &= c T \implies T = \frac{\lambda}c = \frac{3\,\text{м}}{3 \cdot 10^{8}\,\frac{\text{м}}{\text{с}}} = 1{,}0 \cdot 10^{-8}\,\text{с}, \\
    \lambda &= c T = c \cdot \frac 1\nu \implies \nu = \frac c{\lambda} = \frac{3 \cdot 10^{8}\,\frac{\text{м}}{\text{с}}}{3\,\text{м}} = 1{,}0 \cdot 10^{8}\,\text{Гц}.
    \end{align*}
}
\solutionspace{80pt}

\tasknumber{6}%
\task{%
    Из формулы Планка выразите (нужен вывод, не только ответ)...
    \begin{enumerate}
        \item длину соответствующей электромагнитной волны,
        \item период колебаний индукции магнитного поля в соответствующей электромагнитной волне.
    \end{enumerate}
}

\variantsplitter

\addpersonalvariant{Александр Козинец}

\tasknumber{1}%
\task{%
    Сколько фотонов испускает за 30 минут лазер,
    если мощность его излучения $75\,\text{мВт}$?
    Длина волны излучения $600\,\text{нм}$.
    $h = 6{,}626 \cdot 10^{-34}\,\text{Дж}\cdot\text{с}$.
}
\answer{%
    $
        N
            = \frac{E_{\text{общая}}}{E_{\text{одного фотона}}}
            = \frac{Pt}{h\nu} = \frac{Pt}{h \frac c\lambda}
            = \frac{Pt\lambda}{hc}
            = \frac{
                75\,\text{мВт} \cdot 30 \cdot 60 \units{с} \cdot 600\,\text{нм}
            }{
                6{,}626 \cdot 10^{-34}\,\text{Дж}\cdot\text{с} \cdot 3 \cdot 10^{8}\,\frac{\text{м}}{\text{с}}
            }
            \approx 4{,}07 \cdot 10^{20}\units{фотонов}
    $
}
\solutionspace{120pt}

\tasknumber{2}%
\task{%
    Определите название цвета по длине волны в вакууме
    и частоту колебаний электромагнитного поля в ней:
    \begin{enumerate}
        \item $660\,\text{нм}$,
        \item $470\,\text{нм}$,
        \item $610\,\text{нм}$,
        \item $420\,\text{нм}$.
    \end{enumerate}
}
\answer{%
    \begin{enumerate}
        \item $660\,\text{нм} \to$ красный, $\nu_1 = \frac c{\lambda_1} \approx 4{,}55 \cdot 10^{14}\,\text{Гц}$,
        \item $470\,\text{нм} \to$ синий, $\nu_2 = \frac c{\lambda_2} \approx 6{,}38 \cdot 10^{14}\,\text{Гц}$,
        \item $610\,\text{нм} \to$ оранжевый, $\nu_3 = \frac c{\lambda_3} \approx 4{,}92 \cdot 10^{14}\,\text{Гц}$,
        \item $420\,\text{нм} \to$ фиолетовый, $\nu_4 = \frac c{\lambda_4} \approx 7{,}14 \cdot 10^{14}\,\text{Гц}$.
    \end{enumerate}

    $\nu = \frac 1 T = \frac c{\lambda} = \frac {2{,}998 \cdot 10^{8}\,\frac{\text{м}}{\text{с}}}{l \cdot 1\,\text{мкм}} \approx \frac{2{,}998 \cdot 10^{14}\,\text{Гц}}l$,
    где $l$~--- численное значение длины волны в мкм.
}
\solutionspace{80pt}

\tasknumber{3}%
\task{%
    Определите энергию фотона излучения частотой $5 \cdot 10^{16}\,\text{Гц}$.
    Ответ получите в джоулях и в электронвольтах.
}
\answer{%
    $E = h \nu = 6{,}626 \cdot 10^{-34}\,\text{Дж}\cdot\text{с} \cdot 5 \cdot 10^{16}\,\text{Гц} \approx 33 \cdot 10^{-18}\,\text{Дж} \approx 210\,\text{эВ}$
}
\solutionspace{80pt}

\tasknumber{4}%
\task{%
    Определите энергию кванта света с длиной волны $200\,\text{нм}$.
    Ответ выразите в электронвольтах.
    Способен ли человеческий глаз увидеть один такой квант? А импульс таких квантов?'
}
\answer{%
    $E = h\nu = \frac{hc}{\lambda} = \frac{6{,}626 \cdot 10^{-34}\,\text{Дж}\cdot\text{с} \cdot 3 \cdot 10^{8}\,\frac{\text{м}}{\text{с}}}{200\,\text{нм}} \approx 0{,}0994 \cdot 10^{-17}\,\text{Дж} \approx 6{,}2\,\text{эВ}$
}
\solutionspace{80pt}

\tasknumber{5}%
\task{%
    Определите период колебаний вектора напряженности индукции магнитного поля
    в электромагнитной волне в вакууме, длина который составляет $3\,\text{см}$.
}
\answer{%
    \begin{align*}
    \lambda &= c T \implies T = \frac{\lambda}c = \frac{3\,\text{см}}{3 \cdot 10^{8}\,\frac{\text{м}}{\text{с}}} = 1{,}0 \cdot 10^{-10}\,\text{с}, \\
    \lambda &= c T = c \cdot \frac 1\nu \implies \nu = \frac c{\lambda} = \frac{3 \cdot 10^{8}\,\frac{\text{м}}{\text{с}}}{3\,\text{см}} = 1{,}0 \cdot 10^{10}\,\text{Гц}.
    \end{align*}
}
\solutionspace{80pt}

\tasknumber{6}%
\task{%
    Из формулы Планка выразите (нужен вывод, не только ответ)...
    \begin{enumerate}
        \item длину соответствующей электромагнитной волны,
        \item период колебаний индукции магнитного поля в соответствующей электромагнитной волне.
    \end{enumerate}
}

\variantsplitter

\addpersonalvariant{Андрей Куликовский}

\tasknumber{1}%
\task{%
    Сколько фотонов испускает за 60 минут лазер,
    если мощность его излучения $40\,\text{мВт}$?
    Длина волны излучения $600\,\text{нм}$.
    $h = 6{,}626 \cdot 10^{-34}\,\text{Дж}\cdot\text{с}$.
}
\answer{%
    $
        N
            = \frac{E_{\text{общая}}}{E_{\text{одного фотона}}}
            = \frac{Pt}{h\nu} = \frac{Pt}{h \frac c\lambda}
            = \frac{Pt\lambda}{hc}
            = \frac{
                40\,\text{мВт} \cdot 60 \cdot 60 \units{с} \cdot 600\,\text{нм}
            }{
                6{,}626 \cdot 10^{-34}\,\text{Дж}\cdot\text{с} \cdot 3 \cdot 10^{8}\,\frac{\text{м}}{\text{с}}
            }
            \approx 4{,}35 \cdot 10^{20}\units{фотонов}
    $
}
\solutionspace{120pt}

\tasknumber{2}%
\task{%
    Определите название цвета по длине волны в вакууме
    и частоту колебаний электромагнитного поля в ней:
    \begin{enumerate}
        \item $660\,\text{нм}$,
        \item $390\,\text{нм}$,
        \item $610\,\text{нм}$,
        \item $580\,\text{нм}$.
    \end{enumerate}
}
\answer{%
    \begin{enumerate}
        \item $660\,\text{нм} \to$ красный, $\nu_1 = \frac c{\lambda_1} \approx 4{,}55 \cdot 10^{14}\,\text{Гц}$,
        \item $390\,\text{нм} \to$ фиолетовый, $\nu_2 = \frac c{\lambda_2} \approx 7{,}69 \cdot 10^{14}\,\text{Гц}$,
        \item $610\,\text{нм} \to$ оранжевый, $\nu_3 = \frac c{\lambda_3} \approx 4{,}92 \cdot 10^{14}\,\text{Гц}$,
        \item $580\,\text{нм} \to$ зелёный, $\nu_4 = \frac c{\lambda_4} \approx 5{,}17 \cdot 10^{14}\,\text{Гц}$.
    \end{enumerate}

    $\nu = \frac 1 T = \frac c{\lambda} = \frac {2{,}998 \cdot 10^{8}\,\frac{\text{м}}{\text{с}}}{l \cdot 1\,\text{мкм}} \approx \frac{2{,}998 \cdot 10^{14}\,\text{Гц}}l$,
    где $l$~--- численное значение длины волны в мкм.
}
\solutionspace{80pt}

\tasknumber{3}%
\task{%
    Определите энергию фотона излучения частотой $6 \cdot 10^{16}\,\text{Гц}$.
    Ответ получите в джоулях и в электронвольтах.
}
\answer{%
    $E = h \nu = 6{,}626 \cdot 10^{-34}\,\text{Дж}\cdot\text{с} \cdot 6 \cdot 10^{16}\,\text{Гц} \approx 40 \cdot 10^{-18}\,\text{Дж} \approx 250\,\text{эВ}$
}
\solutionspace{80pt}

\tasknumber{4}%
\task{%
    Определите энергию фотона с длиной волны $400\,\text{нм}$.
    Ответ выразите в джоулях.
    Способен ли человеческий глаз увидеть один такой квант? А импульс таких квантов?'
}
\answer{%
    $E = h\nu = \frac{hc}{\lambda} = \frac{6{,}626 \cdot 10^{-34}\,\text{Дж}\cdot\text{с} \cdot 3 \cdot 10^{8}\,\frac{\text{м}}{\text{с}}}{400\,\text{нм}} \approx 0{,}0497 \cdot 10^{-17}\,\text{Дж} \approx 3{,}1\,\text{эВ}$
}
\solutionspace{80pt}

\tasknumber{5}%
\task{%
    Определите период колебаний вектора напряженности электрического поля
    в электромагнитной волне в вакууме, длина который составляет $3\,\text{см}$.
}
\answer{%
    \begin{align*}
    \lambda &= c T \implies T = \frac{\lambda}c = \frac{3\,\text{см}}{3 \cdot 10^{8}\,\frac{\text{м}}{\text{с}}} = 1{,}0 \cdot 10^{-10}\,\text{с}, \\
    \lambda &= c T = c \cdot \frac 1\nu \implies \nu = \frac c{\lambda} = \frac{3 \cdot 10^{8}\,\frac{\text{м}}{\text{с}}}{3\,\text{см}} = 1{,}0 \cdot 10^{10}\,\text{Гц}.
    \end{align*}
}
\solutionspace{80pt}

\tasknumber{6}%
\task{%
    Из формулы Планка выразите (нужен вывод, не только ответ)...
    \begin{enumerate}
        \item длину соответствующей электромагнитной волны,
        \item период колебаний индукции магнитного поля в соответствующей электромагнитной волне.
    \end{enumerate}
}

\variantsplitter

\addpersonalvariant{Полина Лоткова}

\tasknumber{1}%
\task{%
    Сколько фотонов испускает за 30 минут лазер,
    если мощность его излучения $40\,\text{мВт}$?
    Длина волны излучения $750\,\text{нм}$.
    $h = 6{,}626 \cdot 10^{-34}\,\text{Дж}\cdot\text{с}$.
}
\answer{%
    $
        N
            = \frac{E_{\text{общая}}}{E_{\text{одного фотона}}}
            = \frac{Pt}{h\nu} = \frac{Pt}{h \frac c\lambda}
            = \frac{Pt\lambda}{hc}
            = \frac{
                40\,\text{мВт} \cdot 30 \cdot 60 \units{с} \cdot 750\,\text{нм}
            }{
                6{,}626 \cdot 10^{-34}\,\text{Дж}\cdot\text{с} \cdot 3 \cdot 10^{8}\,\frac{\text{м}}{\text{с}}
            }
            \approx 2{,}72 \cdot 10^{20}\units{фотонов}
    $
}
\solutionspace{120pt}

\tasknumber{2}%
\task{%
    Определите название цвета по длине волны в вакууме
    и частоту колебаний электромагнитного поля в ней:
    \begin{enumerate}
        \item $450\,\text{нм}$,
        \item $470\,\text{нм}$,
        \item $530\,\text{нм}$,
        \item $420\,\text{нм}$.
    \end{enumerate}
}
\answer{%
    \begin{enumerate}
        \item $450\,\text{нм} \to$ синий, $\nu_1 = \frac c{\lambda_1} \approx 6{,}67 \cdot 10^{14}\,\text{Гц}$,
        \item $470\,\text{нм} \to$ синий, $\nu_2 = \frac c{\lambda_2} \approx 6{,}38 \cdot 10^{14}\,\text{Гц}$,
        \item $530\,\text{нм} \to$ зелёный, $\nu_3 = \frac c{\lambda_3} \approx 5{,}66 \cdot 10^{14}\,\text{Гц}$,
        \item $420\,\text{нм} \to$ фиолетовый, $\nu_4 = \frac c{\lambda_4} \approx 7{,}14 \cdot 10^{14}\,\text{Гц}$.
    \end{enumerate}

    $\nu = \frac 1 T = \frac c{\lambda} = \frac {2{,}998 \cdot 10^{8}\,\frac{\text{м}}{\text{с}}}{l \cdot 1\,\text{мкм}} \approx \frac{2{,}998 \cdot 10^{14}\,\text{Гц}}l$,
    где $l$~--- численное значение длины волны в мкм.
}
\solutionspace{80pt}

\tasknumber{3}%
\task{%
    Определите энергию фотона излучения частотой $5 \cdot 10^{16}\,\text{Гц}$.
    Ответ получите в джоулях и в электронвольтах.
}
\answer{%
    $E = h \nu = 6{,}626 \cdot 10^{-34}\,\text{Дж}\cdot\text{с} \cdot 5 \cdot 10^{16}\,\text{Гц} \approx 33 \cdot 10^{-18}\,\text{Дж} \approx 210\,\text{эВ}$
}
\solutionspace{80pt}

\tasknumber{4}%
\task{%
    Определите энергию кванта света с длиной волны $700\,\text{нм}$.
    Ответ выразите в джоулях.
    Способен ли человеческий глаз увидеть один такой квант? А импульс таких квантов?'
}
\answer{%
    $E = h\nu = \frac{hc}{\lambda} = \frac{6{,}626 \cdot 10^{-34}\,\text{Дж}\cdot\text{с} \cdot 3 \cdot 10^{8}\,\frac{\text{м}}{\text{с}}}{700\,\text{нм}} \approx 0{,}0284 \cdot 10^{-17}\,\text{Дж} \approx 1{,}77\,\text{эВ}$
}
\solutionspace{80pt}

\tasknumber{5}%
\task{%
    Определите частоту колебаний вектора напряженности индукции магнитного поля
    в электромагнитной волне в вакууме, длина который составляет $2\,\text{м}$.
}
\answer{%
    \begin{align*}
    \lambda &= c T \implies T = \frac{\lambda}c = \frac{2\,\text{м}}{3 \cdot 10^{8}\,\frac{\text{м}}{\text{с}}} = 0{,}7 \cdot 10^{-8}\,\text{с}, \\
    \lambda &= c T = c \cdot \frac 1\nu \implies \nu = \frac c{\lambda} = \frac{3 \cdot 10^{8}\,\frac{\text{м}}{\text{с}}}{2\,\text{м}} = 1{,}5 \cdot 10^{8}\,\text{Гц}.
    \end{align*}
}
\solutionspace{80pt}

\tasknumber{6}%
\task{%
    Из формулы Планка выразите (нужен вывод, не только ответ)...
    \begin{enumerate}
        \item длину соответствующей электромагнитной волны,
        \item период колебаний электрического поля в соответствующей электромагнитной волне.
    \end{enumerate}
}

\variantsplitter

\addpersonalvariant{Екатерина Медведева}

\tasknumber{1}%
\task{%
    Сколько фотонов испускает за 5 минут лазер,
    если мощность его излучения $40\,\text{мВт}$?
    Длина волны излучения $750\,\text{нм}$.
    $h = 6{,}626 \cdot 10^{-34}\,\text{Дж}\cdot\text{с}$.
}
\answer{%
    $
        N
            = \frac{E_{\text{общая}}}{E_{\text{одного фотона}}}
            = \frac{Pt}{h\nu} = \frac{Pt}{h \frac c\lambda}
            = \frac{Pt\lambda}{hc}
            = \frac{
                40\,\text{мВт} \cdot 5 \cdot 60 \units{с} \cdot 750\,\text{нм}
            }{
                6{,}626 \cdot 10^{-34}\,\text{Дж}\cdot\text{с} \cdot 3 \cdot 10^{8}\,\frac{\text{м}}{\text{с}}
            }
            \approx 0{,}45 \cdot 10^{20}\units{фотонов}
    $
}
\solutionspace{120pt}

\tasknumber{2}%
\task{%
    Определите название цвета по длине волны в вакууме
    и частоту колебаний электромагнитного поля в ней:
    \begin{enumerate}
        \item $450\,\text{нм}$,
        \item $470\,\text{нм}$,
        \item $610\,\text{нм}$,
        \item $490\,\text{нм}$.
    \end{enumerate}
}
\answer{%
    \begin{enumerate}
        \item $450\,\text{нм} \to$ синий, $\nu_1 = \frac c{\lambda_1} \approx 6{,}67 \cdot 10^{14}\,\text{Гц}$,
        \item $470\,\text{нм} \to$ синий, $\nu_2 = \frac c{\lambda_2} \approx 6{,}38 \cdot 10^{14}\,\text{Гц}$,
        \item $610\,\text{нм} \to$ оранжевый, $\nu_3 = \frac c{\lambda_3} \approx 4{,}92 \cdot 10^{14}\,\text{Гц}$,
        \item $490\,\text{нм} \to$ голубой, $\nu_4 = \frac c{\lambda_4} \approx 6{,}12 \cdot 10^{14}\,\text{Гц}$.
    \end{enumerate}

    $\nu = \frac 1 T = \frac c{\lambda} = \frac {2{,}998 \cdot 10^{8}\,\frac{\text{м}}{\text{с}}}{l \cdot 1\,\text{мкм}} \approx \frac{2{,}998 \cdot 10^{14}\,\text{Гц}}l$,
    где $l$~--- численное значение длины волны в мкм.
}
\solutionspace{80pt}

\tasknumber{3}%
\task{%
    Определите энергию фотона излучения частотой $7 \cdot 10^{16}\,\text{Гц}$.
    Ответ получите в джоулях и в электронвольтах.
}
\answer{%
    $E = h \nu = 6{,}626 \cdot 10^{-34}\,\text{Дж}\cdot\text{с} \cdot 7 \cdot 10^{16}\,\text{Гц} \approx 46 \cdot 10^{-18}\,\text{Дж} \approx 290\,\text{эВ}$
}
\solutionspace{80pt}

\tasknumber{4}%
\task{%
    Определите энергию фотона с длиной волны $200\,\text{нм}$.
    Ответ выразите в джоулях.
    Способен ли человеческий глаз увидеть один такой квант? А импульс таких квантов?'
}
\answer{%
    $E = h\nu = \frac{hc}{\lambda} = \frac{6{,}626 \cdot 10^{-34}\,\text{Дж}\cdot\text{с} \cdot 3 \cdot 10^{8}\,\frac{\text{м}}{\text{с}}}{200\,\text{нм}} \approx 0{,}0994 \cdot 10^{-17}\,\text{Дж} \approx 6{,}2\,\text{эВ}$
}
\solutionspace{80pt}

\tasknumber{5}%
\task{%
    Определите период колебаний вектора напряженности индукции магнитного поля
    в электромагнитной волне в вакууме, длина который составляет $3\,\text{см}$.
}
\answer{%
    \begin{align*}
    \lambda &= c T \implies T = \frac{\lambda}c = \frac{3\,\text{см}}{3 \cdot 10^{8}\,\frac{\text{м}}{\text{с}}} = 1{,}0 \cdot 10^{-10}\,\text{с}, \\
    \lambda &= c T = c \cdot \frac 1\nu \implies \nu = \frac c{\lambda} = \frac{3 \cdot 10^{8}\,\frac{\text{м}}{\text{с}}}{3\,\text{см}} = 1{,}0 \cdot 10^{10}\,\text{Гц}.
    \end{align*}
}
\solutionspace{80pt}

\tasknumber{6}%
\task{%
    Из формулы Планка выразите (нужен вывод, не только ответ)...
    \begin{enumerate}
        \item длину соответствующей электромагнитной волны,
        \item период колебаний электрического поля в соответствующей электромагнитной волне.
    \end{enumerate}
}

\variantsplitter

\addpersonalvariant{Константин Мельник}

\tasknumber{1}%
\task{%
    Сколько фотонов испускает за 30 минут лазер,
    если мощность его излучения $200\,\text{мВт}$?
    Длина волны излучения $600\,\text{нм}$.
    $h = 6{,}626 \cdot 10^{-34}\,\text{Дж}\cdot\text{с}$.
}
\answer{%
    $
        N
            = \frac{E_{\text{общая}}}{E_{\text{одного фотона}}}
            = \frac{Pt}{h\nu} = \frac{Pt}{h \frac c\lambda}
            = \frac{Pt\lambda}{hc}
            = \frac{
                200\,\text{мВт} \cdot 30 \cdot 60 \units{с} \cdot 600\,\text{нм}
            }{
                6{,}626 \cdot 10^{-34}\,\text{Дж}\cdot\text{с} \cdot 3 \cdot 10^{8}\,\frac{\text{м}}{\text{с}}
            }
            \approx 10{,}87 \cdot 10^{20}\units{фотонов}
    $
}
\solutionspace{120pt}

\tasknumber{2}%
\task{%
    Определите название цвета по длине волны в вакууме
    и частоту колебаний электромагнитного поля в ней:
    \begin{enumerate}
        \item $660\,\text{нм}$,
        \item $595\,\text{нм}$,
        \item $720\,\text{нм}$,
        \item $490\,\text{нм}$.
    \end{enumerate}
}
\answer{%
    \begin{enumerate}
        \item $660\,\text{нм} \to$ красный, $\nu_1 = \frac c{\lambda_1} \approx 4{,}55 \cdot 10^{14}\,\text{Гц}$,
        \item $595\,\text{нм} \to$ оранжевый, $\nu_2 = \frac c{\lambda_2} \approx 5{,}04 \cdot 10^{14}\,\text{Гц}$,
        \item $720\,\text{нм} \to$ красный, $\nu_3 = \frac c{\lambda_3} \approx 4{,}17 \cdot 10^{14}\,\text{Гц}$,
        \item $490\,\text{нм} \to$ голубой, $\nu_4 = \frac c{\lambda_4} \approx 6{,}12 \cdot 10^{14}\,\text{Гц}$.
    \end{enumerate}

    $\nu = \frac 1 T = \frac c{\lambda} = \frac {2{,}998 \cdot 10^{8}\,\frac{\text{м}}{\text{с}}}{l \cdot 1\,\text{мкм}} \approx \frac{2{,}998 \cdot 10^{14}\,\text{Гц}}l$,
    где $l$~--- численное значение длины волны в мкм.
}
\solutionspace{80pt}

\tasknumber{3}%
\task{%
    Определите энергию фотона излучения частотой $8 \cdot 10^{16}\,\text{Гц}$.
    Ответ получите в джоулях и в электронвольтах.
}
\answer{%
    $E = h \nu = 6{,}626 \cdot 10^{-34}\,\text{Дж}\cdot\text{с} \cdot 8 \cdot 10^{16}\,\text{Гц} \approx 53 \cdot 10^{-18}\,\text{Дж} \approx 330\,\text{эВ}$
}
\solutionspace{80pt}

\tasknumber{4}%
\task{%
    Определите энергию кванта света с длиной волны $600\,\text{нм}$.
    Ответ выразите в электронвольтах.
    Способен ли человеческий глаз увидеть один такой квант? А импульс таких квантов?'
}
\answer{%
    $E = h\nu = \frac{hc}{\lambda} = \frac{6{,}626 \cdot 10^{-34}\,\text{Дж}\cdot\text{с} \cdot 3 \cdot 10^{8}\,\frac{\text{м}}{\text{с}}}{600\,\text{нм}} \approx 0{,}0331 \cdot 10^{-17}\,\text{Дж} \approx 2{,}1\,\text{эВ}$
}
\solutionspace{80pt}

\tasknumber{5}%
\task{%
    Определите частоту колебаний вектора напряженности электрического поля
    в электромагнитной волне в вакууме, длина который составляет $3\,\text{м}$.
}
\answer{%
    \begin{align*}
    \lambda &= c T \implies T = \frac{\lambda}c = \frac{3\,\text{м}}{3 \cdot 10^{8}\,\frac{\text{м}}{\text{с}}} = 1{,}0 \cdot 10^{-8}\,\text{с}, \\
    \lambda &= c T = c \cdot \frac 1\nu \implies \nu = \frac c{\lambda} = \frac{3 \cdot 10^{8}\,\frac{\text{м}}{\text{с}}}{3\,\text{м}} = 1{,}0 \cdot 10^{8}\,\text{Гц}.
    \end{align*}
}
\solutionspace{80pt}

\tasknumber{6}%
\task{%
    Из формулы Планка выразите (нужен вывод, не только ответ)...
    \begin{enumerate}
        \item длину соответствующей электромагнитной волны,
        \item период колебаний электрического поля в соответствующей электромагнитной волне.
    \end{enumerate}
}

\variantsplitter

\addpersonalvariant{Степан Небоваренков}

\tasknumber{1}%
\task{%
    Сколько фотонов испускает за 120 минут лазер,
    если мощность его излучения $15\,\text{мВт}$?
    Длина волны излучения $600\,\text{нм}$.
    $h = 6{,}626 \cdot 10^{-34}\,\text{Дж}\cdot\text{с}$.
}
\answer{%
    $
        N
            = \frac{E_{\text{общая}}}{E_{\text{одного фотона}}}
            = \frac{Pt}{h\nu} = \frac{Pt}{h \frac c\lambda}
            = \frac{Pt\lambda}{hc}
            = \frac{
                15\,\text{мВт} \cdot 120 \cdot 60 \units{с} \cdot 600\,\text{нм}
            }{
                6{,}626 \cdot 10^{-34}\,\text{Дж}\cdot\text{с} \cdot 3 \cdot 10^{8}\,\frac{\text{м}}{\text{с}}
            }
            \approx 3{,}26 \cdot 10^{20}\units{фотонов}
    $
}
\solutionspace{120pt}

\tasknumber{2}%
\task{%
    Определите название цвета по длине волны в вакууме
    и частоту колебаний электромагнитного поля в ней:
    \begin{enumerate}
        \item $450\,\text{нм}$,
        \item $470\,\text{нм}$,
        \item $530\,\text{нм}$,
        \item $490\,\text{нм}$.
    \end{enumerate}
}
\answer{%
    \begin{enumerate}
        \item $450\,\text{нм} \to$ синий, $\nu_1 = \frac c{\lambda_1} \approx 6{,}67 \cdot 10^{14}\,\text{Гц}$,
        \item $470\,\text{нм} \to$ синий, $\nu_2 = \frac c{\lambda_2} \approx 6{,}38 \cdot 10^{14}\,\text{Гц}$,
        \item $530\,\text{нм} \to$ зелёный, $\nu_3 = \frac c{\lambda_3} \approx 5{,}66 \cdot 10^{14}\,\text{Гц}$,
        \item $490\,\text{нм} \to$ голубой, $\nu_4 = \frac c{\lambda_4} \approx 6{,}12 \cdot 10^{14}\,\text{Гц}$.
    \end{enumerate}

    $\nu = \frac 1 T = \frac c{\lambda} = \frac {2{,}998 \cdot 10^{8}\,\frac{\text{м}}{\text{с}}}{l \cdot 1\,\text{мкм}} \approx \frac{2{,}998 \cdot 10^{14}\,\text{Гц}}l$,
    где $l$~--- численное значение длины волны в мкм.
}
\solutionspace{80pt}

\tasknumber{3}%
\task{%
    Определите энергию фотона излучения частотой $6 \cdot 10^{16}\,\text{Гц}$.
    Ответ получите в джоулях и в электронвольтах.
}
\answer{%
    $E = h \nu = 6{,}626 \cdot 10^{-34}\,\text{Дж}\cdot\text{с} \cdot 6 \cdot 10^{16}\,\text{Гц} \approx 40 \cdot 10^{-18}\,\text{Дж} \approx 250\,\text{эВ}$
}
\solutionspace{80pt}

\tasknumber{4}%
\task{%
    Определите энергию кванта света с длиной волны $200\,\text{нм}$.
    Ответ выразите в электронвольтах.
    Способен ли человеческий глаз увидеть один такой квант? А импульс таких квантов?'
}
\answer{%
    $E = h\nu = \frac{hc}{\lambda} = \frac{6{,}626 \cdot 10^{-34}\,\text{Дж}\cdot\text{с} \cdot 3 \cdot 10^{8}\,\frac{\text{м}}{\text{с}}}{200\,\text{нм}} \approx 0{,}0994 \cdot 10^{-17}\,\text{Дж} \approx 6{,}2\,\text{эВ}$
}
\solutionspace{80pt}

\tasknumber{5}%
\task{%
    Определите период колебаний вектора напряженности индукции магнитного поля
    в электромагнитной волне в вакууме, длина который составляет $3\,\text{м}$.
}
\answer{%
    \begin{align*}
    \lambda &= c T \implies T = \frac{\lambda}c = \frac{3\,\text{м}}{3 \cdot 10^{8}\,\frac{\text{м}}{\text{с}}} = 1{,}0 \cdot 10^{-8}\,\text{с}, \\
    \lambda &= c T = c \cdot \frac 1\nu \implies \nu = \frac c{\lambda} = \frac{3 \cdot 10^{8}\,\frac{\text{м}}{\text{с}}}{3\,\text{м}} = 1{,}0 \cdot 10^{8}\,\text{Гц}.
    \end{align*}
}
\solutionspace{80pt}

\tasknumber{6}%
\task{%
    Из формулы Планка выразите (нужен вывод, не только ответ)...
    \begin{enumerate}
        \item длину соответствующей электромагнитной волны,
        \item период колебаний электрического поля в соответствующей электромагнитной волне.
    \end{enumerate}
}

\variantsplitter

\addpersonalvariant{Матвей Неретин}

\tasknumber{1}%
\task{%
    Сколько фотонов испускает за 120 минут лазер,
    если мощность его излучения $40\,\text{мВт}$?
    Длина волны излучения $500\,\text{нм}$.
    $h = 6{,}626 \cdot 10^{-34}\,\text{Дж}\cdot\text{с}$.
}
\answer{%
    $
        N
            = \frac{E_{\text{общая}}}{E_{\text{одного фотона}}}
            = \frac{Pt}{h\nu} = \frac{Pt}{h \frac c\lambda}
            = \frac{Pt\lambda}{hc}
            = \frac{
                40\,\text{мВт} \cdot 120 \cdot 60 \units{с} \cdot 500\,\text{нм}
            }{
                6{,}626 \cdot 10^{-34}\,\text{Дж}\cdot\text{с} \cdot 3 \cdot 10^{8}\,\frac{\text{м}}{\text{с}}
            }
            \approx 7{,}24 \cdot 10^{20}\units{фотонов}
    $
}
\solutionspace{120pt}

\tasknumber{2}%
\task{%
    Определите название цвета по длине волны в вакууме
    и частоту колебаний электромагнитного поля в ней:
    \begin{enumerate}
        \item $580\,\text{нм}$,
        \item $390\,\text{нм}$,
        \item $720\,\text{нм}$,
        \item $580\,\text{нм}$.
    \end{enumerate}
}
\answer{%
    \begin{enumerate}
        \item $580\,\text{нм} \to$ жёлтый , $\nu_1 = \frac c{\lambda_1} \approx 5{,}17 \cdot 10^{14}\,\text{Гц}$,
        \item $390\,\text{нм} \to$ фиолетовый, $\nu_2 = \frac c{\lambda_2} \approx 7{,}69 \cdot 10^{14}\,\text{Гц}$,
        \item $720\,\text{нм} \to$ красный, $\nu_3 = \frac c{\lambda_3} \approx 4{,}17 \cdot 10^{14}\,\text{Гц}$,
        \item $580\,\text{нм} \to$ зелёный, $\nu_4 = \frac c{\lambda_4} \approx 5{,}17 \cdot 10^{14}\,\text{Гц}$.
    \end{enumerate}

    $\nu = \frac 1 T = \frac c{\lambda} = \frac {2{,}998 \cdot 10^{8}\,\frac{\text{м}}{\text{с}}}{l \cdot 1\,\text{мкм}} \approx \frac{2{,}998 \cdot 10^{14}\,\text{Гц}}l$,
    где $l$~--- численное значение длины волны в мкм.
}
\solutionspace{80pt}

\tasknumber{3}%
\task{%
    Определите энергию фотона излучения частотой $8 \cdot 10^{16}\,\text{Гц}$.
    Ответ получите в джоулях и в электронвольтах.
}
\answer{%
    $E = h \nu = 6{,}626 \cdot 10^{-34}\,\text{Дж}\cdot\text{с} \cdot 8 \cdot 10^{16}\,\text{Гц} \approx 53 \cdot 10^{-18}\,\text{Дж} \approx 330\,\text{эВ}$
}
\solutionspace{80pt}

\tasknumber{4}%
\task{%
    Определите энергию фотона с длиной волны $850\,\text{нм}$.
    Ответ выразите в электронвольтах.
    Способен ли человеческий глаз увидеть один такой квант? А импульс таких квантов?'
}
\answer{%
    $E = h\nu = \frac{hc}{\lambda} = \frac{6{,}626 \cdot 10^{-34}\,\text{Дж}\cdot\text{с} \cdot 3 \cdot 10^{8}\,\frac{\text{м}}{\text{с}}}{850\,\text{нм}} \approx 0{,}0234 \cdot 10^{-17}\,\text{Дж} \approx 1{,}46\,\text{эВ}$
}
\solutionspace{80pt}

\tasknumber{5}%
\task{%
    Определите период колебаний вектора напряженности индукции магнитного поля
    в электромагнитной волне в вакууме, длина который составляет $3\,\text{м}$.
}
\answer{%
    \begin{align*}
    \lambda &= c T \implies T = \frac{\lambda}c = \frac{3\,\text{м}}{3 \cdot 10^{8}\,\frac{\text{м}}{\text{с}}} = 1{,}0 \cdot 10^{-8}\,\text{с}, \\
    \lambda &= c T = c \cdot \frac 1\nu \implies \nu = \frac c{\lambda} = \frac{3 \cdot 10^{8}\,\frac{\text{м}}{\text{с}}}{3\,\text{м}} = 1{,}0 \cdot 10^{8}\,\text{Гц}.
    \end{align*}
}
\solutionspace{80pt}

\tasknumber{6}%
\task{%
    Из формулы Планка выразите (нужен вывод, не только ответ)...
    \begin{enumerate}
        \item длину соответствующей электромагнитной волны,
        \item период колебаний индукции магнитного поля в соответствующей электромагнитной волне.
    \end{enumerate}
}

\variantsplitter

\addpersonalvariant{Мария Никонова}

\tasknumber{1}%
\task{%
    Сколько фотонов испускает за 30 минут лазер,
    если мощность его излучения $200\,\text{мВт}$?
    Длина волны излучения $600\,\text{нм}$.
    $h = 6{,}626 \cdot 10^{-34}\,\text{Дж}\cdot\text{с}$.
}
\answer{%
    $
        N
            = \frac{E_{\text{общая}}}{E_{\text{одного фотона}}}
            = \frac{Pt}{h\nu} = \frac{Pt}{h \frac c\lambda}
            = \frac{Pt\lambda}{hc}
            = \frac{
                200\,\text{мВт} \cdot 30 \cdot 60 \units{с} \cdot 600\,\text{нм}
            }{
                6{,}626 \cdot 10^{-34}\,\text{Дж}\cdot\text{с} \cdot 3 \cdot 10^{8}\,\frac{\text{м}}{\text{с}}
            }
            \approx 10{,}87 \cdot 10^{20}\units{фотонов}
    $
}
\solutionspace{120pt}

\tasknumber{2}%
\task{%
    Определите название цвета по длине волны в вакууме
    и частоту колебаний электромагнитного поля в ней:
    \begin{enumerate}
        \item $450\,\text{нм}$,
        \item $595\,\text{нм}$,
        \item $720\,\text{нм}$,
        \item $580\,\text{нм}$.
    \end{enumerate}
}
\answer{%
    \begin{enumerate}
        \item $450\,\text{нм} \to$ синий, $\nu_1 = \frac c{\lambda_1} \approx 6{,}67 \cdot 10^{14}\,\text{Гц}$,
        \item $595\,\text{нм} \to$ оранжевый, $\nu_2 = \frac c{\lambda_2} \approx 5{,}04 \cdot 10^{14}\,\text{Гц}$,
        \item $720\,\text{нм} \to$ красный, $\nu_3 = \frac c{\lambda_3} \approx 4{,}17 \cdot 10^{14}\,\text{Гц}$,
        \item $580\,\text{нм} \to$ зелёный, $\nu_4 = \frac c{\lambda_4} \approx 5{,}17 \cdot 10^{14}\,\text{Гц}$.
    \end{enumerate}

    $\nu = \frac 1 T = \frac c{\lambda} = \frac {2{,}998 \cdot 10^{8}\,\frac{\text{м}}{\text{с}}}{l \cdot 1\,\text{мкм}} \approx \frac{2{,}998 \cdot 10^{14}\,\text{Гц}}l$,
    где $l$~--- численное значение длины волны в мкм.
}
\solutionspace{80pt}

\tasknumber{3}%
\task{%
    Определите энергию фотона излучения частотой $6 \cdot 10^{16}\,\text{Гц}$.
    Ответ получите в джоулях и в электронвольтах.
}
\answer{%
    $E = h \nu = 6{,}626 \cdot 10^{-34}\,\text{Дж}\cdot\text{с} \cdot 6 \cdot 10^{16}\,\text{Гц} \approx 40 \cdot 10^{-18}\,\text{Дж} \approx 250\,\text{эВ}$
}
\solutionspace{80pt}

\tasknumber{4}%
\task{%
    Определите энергию фотона с длиной волны $150\,\text{нм}$.
    Ответ выразите в электронвольтах.
    Способен ли человеческий глаз увидеть один такой квант? А импульс таких квантов?'
}
\answer{%
    $E = h\nu = \frac{hc}{\lambda} = \frac{6{,}626 \cdot 10^{-34}\,\text{Дж}\cdot\text{с} \cdot 3 \cdot 10^{8}\,\frac{\text{м}}{\text{с}}}{150\,\text{нм}} \approx 0{,}133 \cdot 10^{-17}\,\text{Дж} \approx 8{,}3\,\text{эВ}$
}
\solutionspace{80pt}

\tasknumber{5}%
\task{%
    Определите частоту колебаний вектора напряженности индукции магнитного поля
    в электромагнитной волне в вакууме, длина который составляет $2\,\text{см}$.
}
\answer{%
    \begin{align*}
    \lambda &= c T \implies T = \frac{\lambda}c = \frac{2\,\text{см}}{3 \cdot 10^{8}\,\frac{\text{м}}{\text{с}}} = 0{,}7 \cdot 10^{-10}\,\text{с}, \\
    \lambda &= c T = c \cdot \frac 1\nu \implies \nu = \frac c{\lambda} = \frac{3 \cdot 10^{8}\,\frac{\text{м}}{\text{с}}}{2\,\text{см}} = 1{,}5 \cdot 10^{10}\,\text{Гц}.
    \end{align*}
}
\solutionspace{80pt}

\tasknumber{6}%
\task{%
    Из формулы Планка выразите (нужен вывод, не только ответ)...
    \begin{enumerate}
        \item длину соответствующей электромагнитной волны,
        \item период колебаний индукции магнитного поля в соответствующей электромагнитной волне.
    \end{enumerate}
}

\variantsplitter

\addpersonalvariant{Даниил Палаткин}

\tasknumber{1}%
\task{%
    Сколько фотонов испускает за 5 минут лазер,
    если мощность его излучения $200\,\text{мВт}$?
    Длина волны излучения $600\,\text{нм}$.
    $h = 6{,}626 \cdot 10^{-34}\,\text{Дж}\cdot\text{с}$.
}
\answer{%
    $
        N
            = \frac{E_{\text{общая}}}{E_{\text{одного фотона}}}
            = \frac{Pt}{h\nu} = \frac{Pt}{h \frac c\lambda}
            = \frac{Pt\lambda}{hc}
            = \frac{
                200\,\text{мВт} \cdot 5 \cdot 60 \units{с} \cdot 600\,\text{нм}
            }{
                6{,}626 \cdot 10^{-34}\,\text{Дж}\cdot\text{с} \cdot 3 \cdot 10^{8}\,\frac{\text{м}}{\text{с}}
            }
            \approx 1{,}81 \cdot 10^{20}\units{фотонов}
    $
}
\solutionspace{120pt}

\tasknumber{2}%
\task{%
    Определите название цвета по длине волны в вакууме
    и частоту колебаний электромагнитного поля в ней:
    \begin{enumerate}
        \item $580\,\text{нм}$,
        \item $390\,\text{нм}$,
        \item $530\,\text{нм}$,
        \item $490\,\text{нм}$.
    \end{enumerate}
}
\answer{%
    \begin{enumerate}
        \item $580\,\text{нм} \to$ жёлтый , $\nu_1 = \frac c{\lambda_1} \approx 5{,}17 \cdot 10^{14}\,\text{Гц}$,
        \item $390\,\text{нм} \to$ фиолетовый, $\nu_2 = \frac c{\lambda_2} \approx 7{,}69 \cdot 10^{14}\,\text{Гц}$,
        \item $530\,\text{нм} \to$ зелёный, $\nu_3 = \frac c{\lambda_3} \approx 5{,}66 \cdot 10^{14}\,\text{Гц}$,
        \item $490\,\text{нм} \to$ голубой, $\nu_4 = \frac c{\lambda_4} \approx 6{,}12 \cdot 10^{14}\,\text{Гц}$.
    \end{enumerate}

    $\nu = \frac 1 T = \frac c{\lambda} = \frac {2{,}998 \cdot 10^{8}\,\frac{\text{м}}{\text{с}}}{l \cdot 1\,\text{мкм}} \approx \frac{2{,}998 \cdot 10^{14}\,\text{Гц}}l$,
    где $l$~--- численное значение длины волны в мкм.
}
\solutionspace{80pt}

\tasknumber{3}%
\task{%
    Определите энергию фотона излучения частотой $7 \cdot 10^{16}\,\text{Гц}$.
    Ответ получите в джоулях и в электронвольтах.
}
\answer{%
    $E = h \nu = 6{,}626 \cdot 10^{-34}\,\text{Дж}\cdot\text{с} \cdot 7 \cdot 10^{16}\,\text{Гц} \approx 46 \cdot 10^{-18}\,\text{Дж} \approx 290\,\text{эВ}$
}
\solutionspace{80pt}

\tasknumber{4}%
\task{%
    Определите энергию фотона с длиной волны $200\,\text{нм}$.
    Ответ выразите в джоулях.
    Способен ли человеческий глаз увидеть один такой квант? А импульс таких квантов?'
}
\answer{%
    $E = h\nu = \frac{hc}{\lambda} = \frac{6{,}626 \cdot 10^{-34}\,\text{Дж}\cdot\text{с} \cdot 3 \cdot 10^{8}\,\frac{\text{м}}{\text{с}}}{200\,\text{нм}} \approx 0{,}0994 \cdot 10^{-17}\,\text{Дж} \approx 6{,}2\,\text{эВ}$
}
\solutionspace{80pt}

\tasknumber{5}%
\task{%
    Определите период колебаний вектора напряженности электрического поля
    в электромагнитной волне в вакууме, длина который составляет $5\,\text{м}$.
}
\answer{%
    \begin{align*}
    \lambda &= c T \implies T = \frac{\lambda}c = \frac{5\,\text{м}}{3 \cdot 10^{8}\,\frac{\text{м}}{\text{с}}} = 1{,}7 \cdot 10^{-8}\,\text{с}, \\
    \lambda &= c T = c \cdot \frac 1\nu \implies \nu = \frac c{\lambda} = \frac{3 \cdot 10^{8}\,\frac{\text{м}}{\text{с}}}{5\,\text{м}} = 0{,}6 \cdot 10^{8}\,\text{Гц}.
    \end{align*}
}
\solutionspace{80pt}

\tasknumber{6}%
\task{%
    Из формулы Планка выразите (нужен вывод, не только ответ)...
    \begin{enumerate}
        \item длину соответствующей электромагнитной волны,
        \item период колебаний электрического поля в соответствующей электромагнитной волне.
    \end{enumerate}
}

\variantsplitter

\addpersonalvariant{Станислав Пикун}

\tasknumber{1}%
\task{%
    Сколько фотонов испускает за 60 минут лазер,
    если мощность его излучения $15\,\text{мВт}$?
    Длина волны излучения $600\,\text{нм}$.
    $h = 6{,}626 \cdot 10^{-34}\,\text{Дж}\cdot\text{с}$.
}
\answer{%
    $
        N
            = \frac{E_{\text{общая}}}{E_{\text{одного фотона}}}
            = \frac{Pt}{h\nu} = \frac{Pt}{h \frac c\lambda}
            = \frac{Pt\lambda}{hc}
            = \frac{
                15\,\text{мВт} \cdot 60 \cdot 60 \units{с} \cdot 600\,\text{нм}
            }{
                6{,}626 \cdot 10^{-34}\,\text{Дж}\cdot\text{с} \cdot 3 \cdot 10^{8}\,\frac{\text{м}}{\text{с}}
            }
            \approx 1{,}63 \cdot 10^{20}\units{фотонов}
    $
}
\solutionspace{120pt}

\tasknumber{2}%
\task{%
    Определите название цвета по длине волны в вакууме
    и частоту колебаний электромагнитного поля в ней:
    \begin{enumerate}
        \item $580\,\text{нм}$,
        \item $595\,\text{нм}$,
        \item $720\,\text{нм}$,
        \item $420\,\text{нм}$.
    \end{enumerate}
}
\answer{%
    \begin{enumerate}
        \item $580\,\text{нм} \to$ жёлтый , $\nu_1 = \frac c{\lambda_1} \approx 5{,}17 \cdot 10^{14}\,\text{Гц}$,
        \item $595\,\text{нм} \to$ оранжевый, $\nu_2 = \frac c{\lambda_2} \approx 5{,}04 \cdot 10^{14}\,\text{Гц}$,
        \item $720\,\text{нм} \to$ красный, $\nu_3 = \frac c{\lambda_3} \approx 4{,}17 \cdot 10^{14}\,\text{Гц}$,
        \item $420\,\text{нм} \to$ фиолетовый, $\nu_4 = \frac c{\lambda_4} \approx 7{,}14 \cdot 10^{14}\,\text{Гц}$.
    \end{enumerate}

    $\nu = \frac 1 T = \frac c{\lambda} = \frac {2{,}998 \cdot 10^{8}\,\frac{\text{м}}{\text{с}}}{l \cdot 1\,\text{мкм}} \approx \frac{2{,}998 \cdot 10^{14}\,\text{Гц}}l$,
    где $l$~--- численное значение длины волны в мкм.
}
\solutionspace{80pt}

\tasknumber{3}%
\task{%
    Определите энергию фотона излучения частотой $5 \cdot 10^{16}\,\text{Гц}$.
    Ответ получите в джоулях и в электронвольтах.
}
\answer{%
    $E = h \nu = 6{,}626 \cdot 10^{-34}\,\text{Дж}\cdot\text{с} \cdot 5 \cdot 10^{16}\,\text{Гц} \approx 33 \cdot 10^{-18}\,\text{Дж} \approx 210\,\text{эВ}$
}
\solutionspace{80pt}

\tasknumber{4}%
\task{%
    Определите энергию фотона с длиной волны $900\,\text{нм}$.
    Ответ выразите в джоулях.
    Способен ли человеческий глаз увидеть один такой квант? А импульс таких квантов?'
}
\answer{%
    $E = h\nu = \frac{hc}{\lambda} = \frac{6{,}626 \cdot 10^{-34}\,\text{Дж}\cdot\text{с} \cdot 3 \cdot 10^{8}\,\frac{\text{м}}{\text{с}}}{900\,\text{нм}} \approx 0{,}0221 \cdot 10^{-17}\,\text{Дж} \approx 1{,}38\,\text{эВ}$
}
\solutionspace{80pt}

\tasknumber{5}%
\task{%
    Определите частоту колебаний вектора напряженности индукции магнитного поля
    в электромагнитной волне в вакууме, длина который составляет $2\,\text{см}$.
}
\answer{%
    \begin{align*}
    \lambda &= c T \implies T = \frac{\lambda}c = \frac{2\,\text{см}}{3 \cdot 10^{8}\,\frac{\text{м}}{\text{с}}} = 0{,}7 \cdot 10^{-10}\,\text{с}, \\
    \lambda &= c T = c \cdot \frac 1\nu \implies \nu = \frac c{\lambda} = \frac{3 \cdot 10^{8}\,\frac{\text{м}}{\text{с}}}{2\,\text{см}} = 1{,}5 \cdot 10^{10}\,\text{Гц}.
    \end{align*}
}
\solutionspace{80pt}

\tasknumber{6}%
\task{%
    Из формулы Планка выразите (нужен вывод, не только ответ)...
    \begin{enumerate}
        \item длину соответствующей электромагнитной волны,
        \item период колебаний электрического поля в соответствующей электромагнитной волне.
    \end{enumerate}
}

\variantsplitter

\addpersonalvariant{Илья Пичугин}

\tasknumber{1}%
\task{%
    Сколько фотонов испускает за 40 минут лазер,
    если мощность его излучения $200\,\text{мВт}$?
    Длина волны излучения $500\,\text{нм}$.
    $h = 6{,}626 \cdot 10^{-34}\,\text{Дж}\cdot\text{с}$.
}
\answer{%
    $
        N
            = \frac{E_{\text{общая}}}{E_{\text{одного фотона}}}
            = \frac{Pt}{h\nu} = \frac{Pt}{h \frac c\lambda}
            = \frac{Pt\lambda}{hc}
            = \frac{
                200\,\text{мВт} \cdot 40 \cdot 60 \units{с} \cdot 500\,\text{нм}
            }{
                6{,}626 \cdot 10^{-34}\,\text{Дж}\cdot\text{с} \cdot 3 \cdot 10^{8}\,\frac{\text{м}}{\text{с}}
            }
            \approx 12{,}07 \cdot 10^{20}\units{фотонов}
    $
}
\solutionspace{120pt}

\tasknumber{2}%
\task{%
    Определите название цвета по длине волны в вакууме
    и частоту колебаний электромагнитного поля в ней:
    \begin{enumerate}
        \item $580\,\text{нм}$,
        \item $470\,\text{нм}$,
        \item $720\,\text{нм}$,
        \item $580\,\text{нм}$.
    \end{enumerate}
}
\answer{%
    \begin{enumerate}
        \item $580\,\text{нм} \to$ жёлтый , $\nu_1 = \frac c{\lambda_1} \approx 5{,}17 \cdot 10^{14}\,\text{Гц}$,
        \item $470\,\text{нм} \to$ синий, $\nu_2 = \frac c{\lambda_2} \approx 6{,}38 \cdot 10^{14}\,\text{Гц}$,
        \item $720\,\text{нм} \to$ красный, $\nu_3 = \frac c{\lambda_3} \approx 4{,}17 \cdot 10^{14}\,\text{Гц}$,
        \item $580\,\text{нм} \to$ зелёный, $\nu_4 = \frac c{\lambda_4} \approx 5{,}17 \cdot 10^{14}\,\text{Гц}$.
    \end{enumerate}

    $\nu = \frac 1 T = \frac c{\lambda} = \frac {2{,}998 \cdot 10^{8}\,\frac{\text{м}}{\text{с}}}{l \cdot 1\,\text{мкм}} \approx \frac{2{,}998 \cdot 10^{14}\,\text{Гц}}l$,
    где $l$~--- численное значение длины волны в мкм.
}
\solutionspace{80pt}

\tasknumber{3}%
\task{%
    Определите энергию фотона излучения частотой $9 \cdot 10^{16}\,\text{Гц}$.
    Ответ получите в джоулях и в электронвольтах.
}
\answer{%
    $E = h \nu = 6{,}626 \cdot 10^{-34}\,\text{Дж}\cdot\text{с} \cdot 9 \cdot 10^{16}\,\text{Гц} \approx 60 \cdot 10^{-18}\,\text{Дж} \approx 370\,\text{эВ}$
}
\solutionspace{80pt}

\tasknumber{4}%
\task{%
    Определите энергию кванта света с длиной волны $400\,\text{нм}$.
    Ответ выразите в электронвольтах.
    Способен ли человеческий глаз увидеть один такой квант? А импульс таких квантов?'
}
\answer{%
    $E = h\nu = \frac{hc}{\lambda} = \frac{6{,}626 \cdot 10^{-34}\,\text{Дж}\cdot\text{с} \cdot 3 \cdot 10^{8}\,\frac{\text{м}}{\text{с}}}{400\,\text{нм}} \approx 0{,}0497 \cdot 10^{-17}\,\text{Дж} \approx 3{,}1\,\text{эВ}$
}
\solutionspace{80pt}

\tasknumber{5}%
\task{%
    Определите период колебаний вектора напряженности электрического поля
    в электромагнитной волне в вакууме, длина который составляет $3\,\text{м}$.
}
\answer{%
    \begin{align*}
    \lambda &= c T \implies T = \frac{\lambda}c = \frac{3\,\text{м}}{3 \cdot 10^{8}\,\frac{\text{м}}{\text{с}}} = 1{,}0 \cdot 10^{-8}\,\text{с}, \\
    \lambda &= c T = c \cdot \frac 1\nu \implies \nu = \frac c{\lambda} = \frac{3 \cdot 10^{8}\,\frac{\text{м}}{\text{с}}}{3\,\text{м}} = 1{,}0 \cdot 10^{8}\,\text{Гц}.
    \end{align*}
}
\solutionspace{80pt}

\tasknumber{6}%
\task{%
    Из формулы Планка выразите (нужен вывод, не только ответ)...
    \begin{enumerate}
        \item длину соответствующей электромагнитной волны,
        \item период колебаний электрического поля в соответствующей электромагнитной волне.
    \end{enumerate}
}

\variantsplitter

\addpersonalvariant{Кирилл Севрюгин}

\tasknumber{1}%
\task{%
    Сколько фотонов испускает за 10 минут лазер,
    если мощность его излучения $15\,\text{мВт}$?
    Длина волны излучения $600\,\text{нм}$.
    $h = 6{,}626 \cdot 10^{-34}\,\text{Дж}\cdot\text{с}$.
}
\answer{%
    $
        N
            = \frac{E_{\text{общая}}}{E_{\text{одного фотона}}}
            = \frac{Pt}{h\nu} = \frac{Pt}{h \frac c\lambda}
            = \frac{Pt\lambda}{hc}
            = \frac{
                15\,\text{мВт} \cdot 10 \cdot 60 \units{с} \cdot 600\,\text{нм}
            }{
                6{,}626 \cdot 10^{-34}\,\text{Дж}\cdot\text{с} \cdot 3 \cdot 10^{8}\,\frac{\text{м}}{\text{с}}
            }
            \approx 0{,}27 \cdot 10^{20}\units{фотонов}
    $
}
\solutionspace{120pt}

\tasknumber{2}%
\task{%
    Определите название цвета по длине волны в вакууме
    и частоту колебаний электромагнитного поля в ней:
    \begin{enumerate}
        \item $660\,\text{нм}$,
        \item $595\,\text{нм}$,
        \item $610\,\text{нм}$,
        \item $420\,\text{нм}$.
    \end{enumerate}
}
\answer{%
    \begin{enumerate}
        \item $660\,\text{нм} \to$ красный, $\nu_1 = \frac c{\lambda_1} \approx 4{,}55 \cdot 10^{14}\,\text{Гц}$,
        \item $595\,\text{нм} \to$ оранжевый, $\nu_2 = \frac c{\lambda_2} \approx 5{,}04 \cdot 10^{14}\,\text{Гц}$,
        \item $610\,\text{нм} \to$ оранжевый, $\nu_3 = \frac c{\lambda_3} \approx 4{,}92 \cdot 10^{14}\,\text{Гц}$,
        \item $420\,\text{нм} \to$ фиолетовый, $\nu_4 = \frac c{\lambda_4} \approx 7{,}14 \cdot 10^{14}\,\text{Гц}$.
    \end{enumerate}

    $\nu = \frac 1 T = \frac c{\lambda} = \frac {2{,}998 \cdot 10^{8}\,\frac{\text{м}}{\text{с}}}{l \cdot 1\,\text{мкм}} \approx \frac{2{,}998 \cdot 10^{14}\,\text{Гц}}l$,
    где $l$~--- численное значение длины волны в мкм.
}
\solutionspace{80pt}

\tasknumber{3}%
\task{%
    Определите энергию фотона излучения частотой $5 \cdot 10^{16}\,\text{Гц}$.
    Ответ получите в джоулях и в электронвольтах.
}
\answer{%
    $E = h \nu = 6{,}626 \cdot 10^{-34}\,\text{Дж}\cdot\text{с} \cdot 5 \cdot 10^{16}\,\text{Гц} \approx 33 \cdot 10^{-18}\,\text{Дж} \approx 210\,\text{эВ}$
}
\solutionspace{80pt}

\tasknumber{4}%
\task{%
    Определите энергию кванта света с длиной волны $900\,\text{нм}$.
    Ответ выразите в джоулях.
    Способен ли человеческий глаз увидеть один такой квант? А импульс таких квантов?'
}
\answer{%
    $E = h\nu = \frac{hc}{\lambda} = \frac{6{,}626 \cdot 10^{-34}\,\text{Дж}\cdot\text{с} \cdot 3 \cdot 10^{8}\,\frac{\text{м}}{\text{с}}}{900\,\text{нм}} \approx 0{,}0221 \cdot 10^{-17}\,\text{Дж} \approx 1{,}38\,\text{эВ}$
}
\solutionspace{80pt}

\tasknumber{5}%
\task{%
    Определите частоту колебаний вектора напряженности электрического поля
    в электромагнитной волне в вакууме, длина который составляет $3\,\text{см}$.
}
\answer{%
    \begin{align*}
    \lambda &= c T \implies T = \frac{\lambda}c = \frac{3\,\text{см}}{3 \cdot 10^{8}\,\frac{\text{м}}{\text{с}}} = 1{,}0 \cdot 10^{-10}\,\text{с}, \\
    \lambda &= c T = c \cdot \frac 1\nu \implies \nu = \frac c{\lambda} = \frac{3 \cdot 10^{8}\,\frac{\text{м}}{\text{с}}}{3\,\text{см}} = 1{,}0 \cdot 10^{10}\,\text{Гц}.
    \end{align*}
}
\solutionspace{80pt}

\tasknumber{6}%
\task{%
    Из формулы Планка выразите (нужен вывод, не только ответ)...
    \begin{enumerate}
        \item длину соответствующей электромагнитной волны,
        \item период колебаний индукции магнитного поля в соответствующей электромагнитной волне.
    \end{enumerate}
}

\variantsplitter

\addpersonalvariant{Илья Стратонников}

\tasknumber{1}%
\task{%
    Сколько фотонов испускает за 30 минут лазер,
    если мощность его излучения $75\,\text{мВт}$?
    Длина волны излучения $500\,\text{нм}$.
    $h = 6{,}626 \cdot 10^{-34}\,\text{Дж}\cdot\text{с}$.
}
\answer{%
    $
        N
            = \frac{E_{\text{общая}}}{E_{\text{одного фотона}}}
            = \frac{Pt}{h\nu} = \frac{Pt}{h \frac c\lambda}
            = \frac{Pt\lambda}{hc}
            = \frac{
                75\,\text{мВт} \cdot 30 \cdot 60 \units{с} \cdot 500\,\text{нм}
            }{
                6{,}626 \cdot 10^{-34}\,\text{Дж}\cdot\text{с} \cdot 3 \cdot 10^{8}\,\frac{\text{м}}{\text{с}}
            }
            \approx 3{,}40 \cdot 10^{20}\units{фотонов}
    $
}
\solutionspace{120pt}

\tasknumber{2}%
\task{%
    Определите название цвета по длине волны в вакууме
    и частоту колебаний электромагнитного поля в ней:
    \begin{enumerate}
        \item $450\,\text{нм}$,
        \item $595\,\text{нм}$,
        \item $720\,\text{нм}$,
        \item $580\,\text{нм}$.
    \end{enumerate}
}
\answer{%
    \begin{enumerate}
        \item $450\,\text{нм} \to$ синий, $\nu_1 = \frac c{\lambda_1} \approx 6{,}67 \cdot 10^{14}\,\text{Гц}$,
        \item $595\,\text{нм} \to$ оранжевый, $\nu_2 = \frac c{\lambda_2} \approx 5{,}04 \cdot 10^{14}\,\text{Гц}$,
        \item $720\,\text{нм} \to$ красный, $\nu_3 = \frac c{\lambda_3} \approx 4{,}17 \cdot 10^{14}\,\text{Гц}$,
        \item $580\,\text{нм} \to$ зелёный, $\nu_4 = \frac c{\lambda_4} \approx 5{,}17 \cdot 10^{14}\,\text{Гц}$.
    \end{enumerate}

    $\nu = \frac 1 T = \frac c{\lambda} = \frac {2{,}998 \cdot 10^{8}\,\frac{\text{м}}{\text{с}}}{l \cdot 1\,\text{мкм}} \approx \frac{2{,}998 \cdot 10^{14}\,\text{Гц}}l$,
    где $l$~--- численное значение длины волны в мкм.
}
\solutionspace{80pt}

\tasknumber{3}%
\task{%
    Определите энергию фотона излучения частотой $6 \cdot 10^{16}\,\text{Гц}$.
    Ответ получите в джоулях и в электронвольтах.
}
\answer{%
    $E = h \nu = 6{,}626 \cdot 10^{-34}\,\text{Дж}\cdot\text{с} \cdot 6 \cdot 10^{16}\,\text{Гц} \approx 40 \cdot 10^{-18}\,\text{Дж} \approx 250\,\text{эВ}$
}
\solutionspace{80pt}

\tasknumber{4}%
\task{%
    Определите энергию кванта света с длиной волны $900\,\text{нм}$.
    Ответ выразите в электронвольтах.
    Способен ли человеческий глаз увидеть один такой квант? А импульс таких квантов?'
}
\answer{%
    $E = h\nu = \frac{hc}{\lambda} = \frac{6{,}626 \cdot 10^{-34}\,\text{Дж}\cdot\text{с} \cdot 3 \cdot 10^{8}\,\frac{\text{м}}{\text{с}}}{900\,\text{нм}} \approx 0{,}0221 \cdot 10^{-17}\,\text{Дж} \approx 1{,}38\,\text{эВ}$
}
\solutionspace{80pt}

\tasknumber{5}%
\task{%
    Определите период колебаний вектора напряженности электрического поля
    в электромагнитной волне в вакууме, длина который составляет $3\,\text{м}$.
}
\answer{%
    \begin{align*}
    \lambda &= c T \implies T = \frac{\lambda}c = \frac{3\,\text{м}}{3 \cdot 10^{8}\,\frac{\text{м}}{\text{с}}} = 1{,}0 \cdot 10^{-8}\,\text{с}, \\
    \lambda &= c T = c \cdot \frac 1\nu \implies \nu = \frac c{\lambda} = \frac{3 \cdot 10^{8}\,\frac{\text{м}}{\text{с}}}{3\,\text{м}} = 1{,}0 \cdot 10^{8}\,\text{Гц}.
    \end{align*}
}
\solutionspace{80pt}

\tasknumber{6}%
\task{%
    Из формулы Планка выразите (нужен вывод, не только ответ)...
    \begin{enumerate}
        \item длину соответствующей электромагнитной волны,
        \item период колебаний индукции магнитного поля в соответствующей электромагнитной волне.
    \end{enumerate}
}

\variantsplitter

\addpersonalvariant{Иван Шустов}

\tasknumber{1}%
\task{%
    Сколько фотонов испускает за 40 минут лазер,
    если мощность его излучения $75\,\text{мВт}$?
    Длина волны излучения $750\,\text{нм}$.
    $h = 6{,}626 \cdot 10^{-34}\,\text{Дж}\cdot\text{с}$.
}
\answer{%
    $
        N
            = \frac{E_{\text{общая}}}{E_{\text{одного фотона}}}
            = \frac{Pt}{h\nu} = \frac{Pt}{h \frac c\lambda}
            = \frac{Pt\lambda}{hc}
            = \frac{
                75\,\text{мВт} \cdot 40 \cdot 60 \units{с} \cdot 750\,\text{нм}
            }{
                6{,}626 \cdot 10^{-34}\,\text{Дж}\cdot\text{с} \cdot 3 \cdot 10^{8}\,\frac{\text{м}}{\text{с}}
            }
            \approx 6{,}79 \cdot 10^{20}\units{фотонов}
    $
}
\solutionspace{120pt}

\tasknumber{2}%
\task{%
    Определите название цвета по длине волны в вакууме
    и частоту колебаний электромагнитного поля в ней:
    \begin{enumerate}
        \item $450\,\text{нм}$,
        \item $595\,\text{нм}$,
        \item $530\,\text{нм}$,
        \item $420\,\text{нм}$.
    \end{enumerate}
}
\answer{%
    \begin{enumerate}
        \item $450\,\text{нм} \to$ синий, $\nu_1 = \frac c{\lambda_1} \approx 6{,}67 \cdot 10^{14}\,\text{Гц}$,
        \item $595\,\text{нм} \to$ оранжевый, $\nu_2 = \frac c{\lambda_2} \approx 5{,}04 \cdot 10^{14}\,\text{Гц}$,
        \item $530\,\text{нм} \to$ зелёный, $\nu_3 = \frac c{\lambda_3} \approx 5{,}66 \cdot 10^{14}\,\text{Гц}$,
        \item $420\,\text{нм} \to$ фиолетовый, $\nu_4 = \frac c{\lambda_4} \approx 7{,}14 \cdot 10^{14}\,\text{Гц}$.
    \end{enumerate}

    $\nu = \frac 1 T = \frac c{\lambda} = \frac {2{,}998 \cdot 10^{8}\,\frac{\text{м}}{\text{с}}}{l \cdot 1\,\text{мкм}} \approx \frac{2{,}998 \cdot 10^{14}\,\text{Гц}}l$,
    где $l$~--- численное значение длины волны в мкм.
}
\solutionspace{80pt}

\tasknumber{3}%
\task{%
    Определите энергию фотона излучения частотой $5 \cdot 10^{16}\,\text{Гц}$.
    Ответ получите в джоулях и в электронвольтах.
}
\answer{%
    $E = h \nu = 6{,}626 \cdot 10^{-34}\,\text{Дж}\cdot\text{с} \cdot 5 \cdot 10^{16}\,\text{Гц} \approx 33 \cdot 10^{-18}\,\text{Дж} \approx 210\,\text{эВ}$
}
\solutionspace{80pt}

\tasknumber{4}%
\task{%
    Определите энергию фотона с длиной волны $150\,\text{нм}$.
    Ответ выразите в джоулях.
    Способен ли человеческий глаз увидеть один такой квант? А импульс таких квантов?'
}
\answer{%
    $E = h\nu = \frac{hc}{\lambda} = \frac{6{,}626 \cdot 10^{-34}\,\text{Дж}\cdot\text{с} \cdot 3 \cdot 10^{8}\,\frac{\text{м}}{\text{с}}}{150\,\text{нм}} \approx 0{,}133 \cdot 10^{-17}\,\text{Дж} \approx 8{,}3\,\text{эВ}$
}
\solutionspace{80pt}

\tasknumber{5}%
\task{%
    Определите частоту колебаний вектора напряженности индукции магнитного поля
    в электромагнитной волне в вакууме, длина который составляет $5\,\text{см}$.
}
\answer{%
    \begin{align*}
    \lambda &= c T \implies T = \frac{\lambda}c = \frac{5\,\text{см}}{3 \cdot 10^{8}\,\frac{\text{м}}{\text{с}}} = 1{,}7 \cdot 10^{-10}\,\text{с}, \\
    \lambda &= c T = c \cdot \frac 1\nu \implies \nu = \frac c{\lambda} = \frac{3 \cdot 10^{8}\,\frac{\text{м}}{\text{с}}}{5\,\text{см}} = 0{,}6 \cdot 10^{10}\,\text{Гц}.
    \end{align*}
}
\solutionspace{80pt}

\tasknumber{6}%
\task{%
    Из формулы Планка выразите (нужен вывод, не только ответ)...
    \begin{enumerate}
        \item длину соответствующей электромагнитной волны,
        \item период колебаний индукции магнитного поля в соответствующей электромагнитной волне.
    \end{enumerate}
}
% autogenerated
