\setdate{23~марта~2021}
\setclass{9«М»}

\addpersonalvariant{Михаил Бурмистров}

\tasknumber{1}%
\task{%
    Сколько фотонов испускает за 10 минут лазер,
    если мощность его излучения $75\,\text{мВт}$?
    Длина волны излучения $750\,\text{нм}$.
    $h = 6{,}626 \cdot 10^{-34}\,\text{Дж}\cdot\text{с}$.
}
\answer{%
    $
        N = \frac{Pt\lambda}{hc}
           = \frac{
                75\,\text{мВт} \cdot 10 \cdot 60 \units{с} \cdot 750\,\text{нм}
            }{
                6{,}626 \cdot 10^{-34}\,\text{Дж}\cdot\text{с} \cdot 3 \cdot 10^{8}\,\frac{\text{м}}{\text{с}}
           }
           \approx 1{,}70 \cdot 10^{20}\units{фотонов}
    $
}
\solutionspace{120pt}

\tasknumber{2}%
\task{%
    Определите название цвета по длине волны в вакууме
    и частоту колебаний электромагнитного поля в ней:
    \begin{enumerate}
        \item $660\,\text{нм}$,
        \item $390\,\text{нм}$,
        \item $610\,\text{нм}$,
        \item $420\,\text{нм}$.
    \end{enumerate}
}
\solutionspace{80pt}

\tasknumber{3}%
\task{%
    Определите энергию фотона излучения частотой $6 \cdot 10^{16}\,\text{Гц}$.
    Ответ получите в джоулях и в электронвольтах.
}
\solutionspace{80pt}

\tasknumber{4}%
\task{%
    Определите энергию фотона с длиной волны $150\,\text{нм}$.
    Ответ выразите в электронвольтах.
    Способен ли человеческий глаз увидеть один такой квант? А импульс таких квантов?'
}
\solutionspace{80pt}

\tasknumber{5}%
\task{%
    Определите частоту колебаний вектора напряженности индукции магнитного поля
    в электромагнитной волне в вакууме, длина который составляет $5\,\text{см}$.
}
\solutionspace{80pt}

\tasknumber{6}%
\task{%
    Из формулы Планка выразите (нужен вывод, не только ответ)...
    \begin{enumerate}
        \item длину соответствующей электромагнитной волны,
        \item период колебаний индукции магнитного поля в соответствующей электромагнитной волне.
    \end{enumerate}
}

\variantsplitter

\addpersonalvariant{Артём Глембо}

\tasknumber{1}%
\task{%
    Сколько фотонов испускает за 5 минут лазер,
    если мощность его излучения $200\,\text{мВт}$?
    Длина волны излучения $500\,\text{нм}$.
    $h = 6{,}626 \cdot 10^{-34}\,\text{Дж}\cdot\text{с}$.
}
\answer{%
    $
        N = \frac{Pt\lambda}{hc}
           = \frac{
                200\,\text{мВт} \cdot 5 \cdot 60 \units{с} \cdot 500\,\text{нм}
            }{
                6{,}626 \cdot 10^{-34}\,\text{Дж}\cdot\text{с} \cdot 3 \cdot 10^{8}\,\frac{\text{м}}{\text{с}}
           }
           \approx 1{,}51 \cdot 10^{20}\units{фотонов}
    $
}
\solutionspace{120pt}

\tasknumber{2}%
\task{%
    Определите название цвета по длине волны в вакууме
    и частоту колебаний электромагнитного поля в ней:
    \begin{enumerate}
        \item $450\,\text{нм}$,
        \item $390\,\text{нм}$,
        \item $610\,\text{нм}$,
        \item $420\,\text{нм}$.
    \end{enumerate}
}
\solutionspace{80pt}

\tasknumber{3}%
\task{%
    Определите энергию фотона излучения частотой $7 \cdot 10^{16}\,\text{Гц}$.
    Ответ получите в джоулях и в электронвольтах.
}
\solutionspace{80pt}

\tasknumber{4}%
\task{%
    Определите энергию фотона с длиной волны $400\,\text{нм}$.
    Ответ выразите в джоулях.
    Способен ли человеческий глаз увидеть один такой квант? А импульс таких квантов?'
}
\solutionspace{80pt}

\tasknumber{5}%
\task{%
    Определите частоту колебаний вектора напряженности электрического поля
    в электромагнитной волне в вакууме, длина который составляет $2\,\text{м}$.
}
\solutionspace{80pt}

\tasknumber{6}%
\task{%
    Из формулы Планка выразите (нужен вывод, не только ответ)...
    \begin{enumerate}
        \item длину соответствующей электромагнитной волны,
        \item период колебаний индукции магнитного поля в соответствующей электромагнитной волне.
    \end{enumerate}
}

\variantsplitter

\addpersonalvariant{Наталья Гончарова}

\tasknumber{1}%
\task{%
    Сколько фотонов испускает за 30 минут лазер,
    если мощность его излучения $75\,\text{мВт}$?
    Длина волны излучения $600\,\text{нм}$.
    $h = 6{,}626 \cdot 10^{-34}\,\text{Дж}\cdot\text{с}$.
}
\answer{%
    $
        N = \frac{Pt\lambda}{hc}
           = \frac{
                75\,\text{мВт} \cdot 30 \cdot 60 \units{с} \cdot 600\,\text{нм}
            }{
                6{,}626 \cdot 10^{-34}\,\text{Дж}\cdot\text{с} \cdot 3 \cdot 10^{8}\,\frac{\text{м}}{\text{с}}
           }
           \approx 4{,}07 \cdot 10^{20}\units{фотонов}
    $
}
\solutionspace{120pt}

\tasknumber{2}%
\task{%
    Определите название цвета по длине волны в вакууме
    и частоту колебаний электромагнитного поля в ней:
    \begin{enumerate}
        \item $660\,\text{нм}$,
        \item $595\,\text{нм}$,
        \item $610\,\text{нм}$,
        \item $420\,\text{нм}$.
    \end{enumerate}
}
\solutionspace{80pt}

\tasknumber{3}%
\task{%
    Определите энергию фотона излучения частотой $8 \cdot 10^{16}\,\text{Гц}$.
    Ответ получите в джоулях и в электронвольтах.
}
\solutionspace{80pt}

\tasknumber{4}%
\task{%
    Определите энергию кванта света с длиной волны $200\,\text{нм}$.
    Ответ выразите в электронвольтах.
    Способен ли человеческий глаз увидеть один такой квант? А импульс таких квантов?'
}
\solutionspace{80pt}

\tasknumber{5}%
\task{%
    Определите период колебаний вектора напряженности электрического поля
    в электромагнитной волне в вакууме, длина который составляет $3\,\text{см}$.
}
\solutionspace{80pt}

\tasknumber{6}%
\task{%
    Из формулы Планка выразите (нужен вывод, не только ответ)...
    \begin{enumerate}
        \item длину соответствующей электромагнитной волны,
        \item период колебаний индукции магнитного поля в соответствующей электромагнитной волне.
    \end{enumerate}
}

\variantsplitter

\addpersonalvariant{Файёзбек Касымов}

\tasknumber{1}%
\task{%
    Сколько фотонов испускает за 40 минут лазер,
    если мощность его излучения $200\,\text{мВт}$?
    Длина волны излучения $600\,\text{нм}$.
    $h = 6{,}626 \cdot 10^{-34}\,\text{Дж}\cdot\text{с}$.
}
\answer{%
    $
        N = \frac{Pt\lambda}{hc}
           = \frac{
                200\,\text{мВт} \cdot 40 \cdot 60 \units{с} \cdot 600\,\text{нм}
            }{
                6{,}626 \cdot 10^{-34}\,\text{Дж}\cdot\text{с} \cdot 3 \cdot 10^{8}\,\frac{\text{м}}{\text{с}}
           }
           \approx 14{,}49 \cdot 10^{20}\units{фотонов}
    $
}
\solutionspace{120pt}

\tasknumber{2}%
\task{%
    Определите название цвета по длине волны в вакууме
    и частоту колебаний электромагнитного поля в ней:
    \begin{enumerate}
        \item $660\,\text{нм}$,
        \item $470\,\text{нм}$,
        \item $530\,\text{нм}$,
        \item $580\,\text{нм}$.
    \end{enumerate}
}
\solutionspace{80pt}

\tasknumber{3}%
\task{%
    Определите энергию фотона излучения частотой $4 \cdot 10^{16}\,\text{Гц}$.
    Ответ получите в джоулях и в электронвольтах.
}
\solutionspace{80pt}

\tasknumber{4}%
\task{%
    Определите энергию фотона с длиной волны $500\,\text{нм}$.
    Ответ выразите в джоулях.
    Способен ли человеческий глаз увидеть один такой квант? А импульс таких квантов?'
}
\solutionspace{80pt}

\tasknumber{5}%
\task{%
    Определите период колебаний вектора напряженности индукции магнитного поля
    в электромагнитной волне в вакууме, длина который составляет $3\,\text{м}$.
}
\solutionspace{80pt}

\tasknumber{6}%
\task{%
    Из формулы Планка выразите (нужен вывод, не только ответ)...
    \begin{enumerate}
        \item длину соответствующей электромагнитной волны,
        \item период колебаний индукции магнитного поля в соответствующей электромагнитной волне.
    \end{enumerate}
}

\variantsplitter

\addpersonalvariant{Александр Козинец}

\tasknumber{1}%
\task{%
    Сколько фотонов испускает за 30 минут лазер,
    если мощность его излучения $75\,\text{мВт}$?
    Длина волны излучения $600\,\text{нм}$.
    $h = 6{,}626 \cdot 10^{-34}\,\text{Дж}\cdot\text{с}$.
}
\answer{%
    $
        N = \frac{Pt\lambda}{hc}
           = \frac{
                75\,\text{мВт} \cdot 30 \cdot 60 \units{с} \cdot 600\,\text{нм}
            }{
                6{,}626 \cdot 10^{-34}\,\text{Дж}\cdot\text{с} \cdot 3 \cdot 10^{8}\,\frac{\text{м}}{\text{с}}
           }
           \approx 4{,}07 \cdot 10^{20}\units{фотонов}
    $
}
\solutionspace{120pt}

\tasknumber{2}%
\task{%
    Определите название цвета по длине волны в вакууме
    и частоту колебаний электромагнитного поля в ней:
    \begin{enumerate}
        \item $660\,\text{нм}$,
        \item $470\,\text{нм}$,
        \item $610\,\text{нм}$,
        \item $420\,\text{нм}$.
    \end{enumerate}
}
\solutionspace{80pt}

\tasknumber{3}%
\task{%
    Определите энергию фотона излучения частотой $5 \cdot 10^{16}\,\text{Гц}$.
    Ответ получите в джоулях и в электронвольтах.
}
\solutionspace{80pt}

\tasknumber{4}%
\task{%
    Определите энергию кванта света с длиной волны $200\,\text{нм}$.
    Ответ выразите в электронвольтах.
    Способен ли человеческий глаз увидеть один такой квант? А импульс таких квантов?'
}
\solutionspace{80pt}

\tasknumber{5}%
\task{%
    Определите период колебаний вектора напряженности индукции магнитного поля
    в электромагнитной волне в вакууме, длина который составляет $3\,\text{см}$.
}
\solutionspace{80pt}

\tasknumber{6}%
\task{%
    Из формулы Планка выразите (нужен вывод, не только ответ)...
    \begin{enumerate}
        \item длину соответствующей электромагнитной волны,
        \item период колебаний индукции магнитного поля в соответствующей электромагнитной волне.
    \end{enumerate}
}

\variantsplitter

\addpersonalvariant{Андрей Куликовский}

\tasknumber{1}%
\task{%
    Сколько фотонов испускает за 60 минут лазер,
    если мощность его излучения $40\,\text{мВт}$?
    Длина волны излучения $600\,\text{нм}$.
    $h = 6{,}626 \cdot 10^{-34}\,\text{Дж}\cdot\text{с}$.
}
\answer{%
    $
        N = \frac{Pt\lambda}{hc}
           = \frac{
                40\,\text{мВт} \cdot 60 \cdot 60 \units{с} \cdot 600\,\text{нм}
            }{
                6{,}626 \cdot 10^{-34}\,\text{Дж}\cdot\text{с} \cdot 3 \cdot 10^{8}\,\frac{\text{м}}{\text{с}}
           }
           \approx 4{,}35 \cdot 10^{20}\units{фотонов}
    $
}
\solutionspace{120pt}

\tasknumber{2}%
\task{%
    Определите название цвета по длине волны в вакууме
    и частоту колебаний электромагнитного поля в ней:
    \begin{enumerate}
        \item $660\,\text{нм}$,
        \item $390\,\text{нм}$,
        \item $610\,\text{нм}$,
        \item $580\,\text{нм}$.
    \end{enumerate}
}
\solutionspace{80pt}

\tasknumber{3}%
\task{%
    Определите энергию фотона излучения частотой $6 \cdot 10^{16}\,\text{Гц}$.
    Ответ получите в джоулях и в электронвольтах.
}
\solutionspace{80pt}

\tasknumber{4}%
\task{%
    Определите энергию фотона с длиной волны $400\,\text{нм}$.
    Ответ выразите в джоулях.
    Способен ли человеческий глаз увидеть один такой квант? А импульс таких квантов?'
}
\solutionspace{80pt}

\tasknumber{5}%
\task{%
    Определите период колебаний вектора напряженности электрического поля
    в электромагнитной волне в вакууме, длина который составляет $3\,\text{см}$.
}
\solutionspace{80pt}

\tasknumber{6}%
\task{%
    Из формулы Планка выразите (нужен вывод, не только ответ)...
    \begin{enumerate}
        \item длину соответствующей электромагнитной волны,
        \item период колебаний индукции магнитного поля в соответствующей электромагнитной волне.
    \end{enumerate}
}

\variantsplitter

\addpersonalvariant{Полина Лоткова}

\tasknumber{1}%
\task{%
    Сколько фотонов испускает за 30 минут лазер,
    если мощность его излучения $40\,\text{мВт}$?
    Длина волны излучения $750\,\text{нм}$.
    $h = 6{,}626 \cdot 10^{-34}\,\text{Дж}\cdot\text{с}$.
}
\answer{%
    $
        N = \frac{Pt\lambda}{hc}
           = \frac{
                40\,\text{мВт} \cdot 30 \cdot 60 \units{с} \cdot 750\,\text{нм}
            }{
                6{,}626 \cdot 10^{-34}\,\text{Дж}\cdot\text{с} \cdot 3 \cdot 10^{8}\,\frac{\text{м}}{\text{с}}
           }
           \approx 2{,}72 \cdot 10^{20}\units{фотонов}
    $
}
\solutionspace{120pt}

\tasknumber{2}%
\task{%
    Определите название цвета по длине волны в вакууме
    и частоту колебаний электромагнитного поля в ней:
    \begin{enumerate}
        \item $450\,\text{нм}$,
        \item $470\,\text{нм}$,
        \item $530\,\text{нм}$,
        \item $420\,\text{нм}$.
    \end{enumerate}
}
\solutionspace{80pt}

\tasknumber{3}%
\task{%
    Определите энергию фотона излучения частотой $5 \cdot 10^{16}\,\text{Гц}$.
    Ответ получите в джоулях и в электронвольтах.
}
\solutionspace{80pt}

\tasknumber{4}%
\task{%
    Определите энергию кванта света с длиной волны $700\,\text{нм}$.
    Ответ выразите в джоулях.
    Способен ли человеческий глаз увидеть один такой квант? А импульс таких квантов?'
}
\solutionspace{80pt}

\tasknumber{5}%
\task{%
    Определите частоту колебаний вектора напряженности индукции магнитного поля
    в электромагнитной волне в вакууме, длина который составляет $2\,\text{м}$.
}
\solutionspace{80pt}

\tasknumber{6}%
\task{%
    Из формулы Планка выразите (нужен вывод, не только ответ)...
    \begin{enumerate}
        \item длину соответствующей электромагнитной волны,
        \item период колебаний электрического поля в соответствующей электромагнитной волне.
    \end{enumerate}
}

\variantsplitter

\addpersonalvariant{Екатерина Медведева}

\tasknumber{1}%
\task{%
    Сколько фотонов испускает за 5 минут лазер,
    если мощность его излучения $40\,\text{мВт}$?
    Длина волны излучения $750\,\text{нм}$.
    $h = 6{,}626 \cdot 10^{-34}\,\text{Дж}\cdot\text{с}$.
}
\answer{%
    $
        N = \frac{Pt\lambda}{hc}
           = \frac{
                40\,\text{мВт} \cdot 5 \cdot 60 \units{с} \cdot 750\,\text{нм}
            }{
                6{,}626 \cdot 10^{-34}\,\text{Дж}\cdot\text{с} \cdot 3 \cdot 10^{8}\,\frac{\text{м}}{\text{с}}
           }
           \approx 0{,}45 \cdot 10^{20}\units{фотонов}
    $
}
\solutionspace{120pt}

\tasknumber{2}%
\task{%
    Определите название цвета по длине волны в вакууме
    и частоту колебаний электромагнитного поля в ней:
    \begin{enumerate}
        \item $450\,\text{нм}$,
        \item $470\,\text{нм}$,
        \item $610\,\text{нм}$,
        \item $490\,\text{нм}$.
    \end{enumerate}
}
\solutionspace{80pt}

\tasknumber{3}%
\task{%
    Определите энергию фотона излучения частотой $7 \cdot 10^{16}\,\text{Гц}$.
    Ответ получите в джоулях и в электронвольтах.
}
\solutionspace{80pt}

\tasknumber{4}%
\task{%
    Определите энергию фотона с длиной волны $200\,\text{нм}$.
    Ответ выразите в джоулях.
    Способен ли человеческий глаз увидеть один такой квант? А импульс таких квантов?'
}
\solutionspace{80pt}

\tasknumber{5}%
\task{%
    Определите период колебаний вектора напряженности индукции магнитного поля
    в электромагнитной волне в вакууме, длина который составляет $3\,\text{см}$.
}
\solutionspace{80pt}

\tasknumber{6}%
\task{%
    Из формулы Планка выразите (нужен вывод, не только ответ)...
    \begin{enumerate}
        \item длину соответствующей электромагнитной волны,
        \item период колебаний электрического поля в соответствующей электромагнитной волне.
    \end{enumerate}
}

\variantsplitter

\addpersonalvariant{Константин Мельник}

\tasknumber{1}%
\task{%
    Сколько фотонов испускает за 30 минут лазер,
    если мощность его излучения $200\,\text{мВт}$?
    Длина волны излучения $600\,\text{нм}$.
    $h = 6{,}626 \cdot 10^{-34}\,\text{Дж}\cdot\text{с}$.
}
\answer{%
    $
        N = \frac{Pt\lambda}{hc}
           = \frac{
                200\,\text{мВт} \cdot 30 \cdot 60 \units{с} \cdot 600\,\text{нм}
            }{
                6{,}626 \cdot 10^{-34}\,\text{Дж}\cdot\text{с} \cdot 3 \cdot 10^{8}\,\frac{\text{м}}{\text{с}}
           }
           \approx 10{,}87 \cdot 10^{20}\units{фотонов}
    $
}
\solutionspace{120pt}

\tasknumber{2}%
\task{%
    Определите название цвета по длине волны в вакууме
    и частоту колебаний электромагнитного поля в ней:
    \begin{enumerate}
        \item $660\,\text{нм}$,
        \item $595\,\text{нм}$,
        \item $720\,\text{нм}$,
        \item $490\,\text{нм}$.
    \end{enumerate}
}
\solutionspace{80pt}

\tasknumber{3}%
\task{%
    Определите энергию фотона излучения частотой $8 \cdot 10^{16}\,\text{Гц}$.
    Ответ получите в джоулях и в электронвольтах.
}
\solutionspace{80pt}

\tasknumber{4}%
\task{%
    Определите энергию кванта света с длиной волны $600\,\text{нм}$.
    Ответ выразите в электронвольтах.
    Способен ли человеческий глаз увидеть один такой квант? А импульс таких квантов?'
}
\solutionspace{80pt}

\tasknumber{5}%
\task{%
    Определите частоту колебаний вектора напряженности электрического поля
    в электромагнитной волне в вакууме, длина который составляет $3\,\text{м}$.
}
\solutionspace{80pt}

\tasknumber{6}%
\task{%
    Из формулы Планка выразите (нужен вывод, не только ответ)...
    \begin{enumerate}
        \item длину соответствующей электромагнитной волны,
        \item период колебаний электрического поля в соответствующей электромагнитной волне.
    \end{enumerate}
}

\variantsplitter

\addpersonalvariant{Степан Небоваренков}

\tasknumber{1}%
\task{%
    Сколько фотонов испускает за 120 минут лазер,
    если мощность его излучения $15\,\text{мВт}$?
    Длина волны излучения $600\,\text{нм}$.
    $h = 6{,}626 \cdot 10^{-34}\,\text{Дж}\cdot\text{с}$.
}
\answer{%
    $
        N = \frac{Pt\lambda}{hc}
           = \frac{
                15\,\text{мВт} \cdot 120 \cdot 60 \units{с} \cdot 600\,\text{нм}
            }{
                6{,}626 \cdot 10^{-34}\,\text{Дж}\cdot\text{с} \cdot 3 \cdot 10^{8}\,\frac{\text{м}}{\text{с}}
           }
           \approx 3{,}26 \cdot 10^{20}\units{фотонов}
    $
}
\solutionspace{120pt}

\tasknumber{2}%
\task{%
    Определите название цвета по длине волны в вакууме
    и частоту колебаний электромагнитного поля в ней:
    \begin{enumerate}
        \item $450\,\text{нм}$,
        \item $470\,\text{нм}$,
        \item $530\,\text{нм}$,
        \item $490\,\text{нм}$.
    \end{enumerate}
}
\solutionspace{80pt}

\tasknumber{3}%
\task{%
    Определите энергию фотона излучения частотой $6 \cdot 10^{16}\,\text{Гц}$.
    Ответ получите в джоулях и в электронвольтах.
}
\solutionspace{80pt}

\tasknumber{4}%
\task{%
    Определите энергию кванта света с длиной волны $200\,\text{нм}$.
    Ответ выразите в электронвольтах.
    Способен ли человеческий глаз увидеть один такой квант? А импульс таких квантов?'
}
\solutionspace{80pt}

\tasknumber{5}%
\task{%
    Определите период колебаний вектора напряженности индукции магнитного поля
    в электромагнитной волне в вакууме, длина который составляет $3\,\text{м}$.
}
\solutionspace{80pt}

\tasknumber{6}%
\task{%
    Из формулы Планка выразите (нужен вывод, не только ответ)...
    \begin{enumerate}
        \item длину соответствующей электромагнитной волны,
        \item период колебаний электрического поля в соответствующей электромагнитной волне.
    \end{enumerate}
}

\variantsplitter

\addpersonalvariant{Матвей Неретин}

\tasknumber{1}%
\task{%
    Сколько фотонов испускает за 120 минут лазер,
    если мощность его излучения $40\,\text{мВт}$?
    Длина волны излучения $500\,\text{нм}$.
    $h = 6{,}626 \cdot 10^{-34}\,\text{Дж}\cdot\text{с}$.
}
\answer{%
    $
        N = \frac{Pt\lambda}{hc}
           = \frac{
                40\,\text{мВт} \cdot 120 \cdot 60 \units{с} \cdot 500\,\text{нм}
            }{
                6{,}626 \cdot 10^{-34}\,\text{Дж}\cdot\text{с} \cdot 3 \cdot 10^{8}\,\frac{\text{м}}{\text{с}}
           }
           \approx 7{,}24 \cdot 10^{20}\units{фотонов}
    $
}
\solutionspace{120pt}

\tasknumber{2}%
\task{%
    Определите название цвета по длине волны в вакууме
    и частоту колебаний электромагнитного поля в ней:
    \begin{enumerate}
        \item $580\,\text{нм}$,
        \item $390\,\text{нм}$,
        \item $720\,\text{нм}$,
        \item $580\,\text{нм}$.
    \end{enumerate}
}
\solutionspace{80pt}

\tasknumber{3}%
\task{%
    Определите энергию фотона излучения частотой $8 \cdot 10^{16}\,\text{Гц}$.
    Ответ получите в джоулях и в электронвольтах.
}
\solutionspace{80pt}

\tasknumber{4}%
\task{%
    Определите энергию фотона с длиной волны $850\,\text{нм}$.
    Ответ выразите в электронвольтах.
    Способен ли человеческий глаз увидеть один такой квант? А импульс таких квантов?'
}
\solutionspace{80pt}

\tasknumber{5}%
\task{%
    Определите период колебаний вектора напряженности индукции магнитного поля
    в электромагнитной волне в вакууме, длина который составляет $3\,\text{м}$.
}
\solutionspace{80pt}

\tasknumber{6}%
\task{%
    Из формулы Планка выразите (нужен вывод, не только ответ)...
    \begin{enumerate}
        \item длину соответствующей электромагнитной волны,
        \item период колебаний индукции магнитного поля в соответствующей электромагнитной волне.
    \end{enumerate}
}

\variantsplitter

\addpersonalvariant{Мария Никонова}

\tasknumber{1}%
\task{%
    Сколько фотонов испускает за 30 минут лазер,
    если мощность его излучения $200\,\text{мВт}$?
    Длина волны излучения $600\,\text{нм}$.
    $h = 6{,}626 \cdot 10^{-34}\,\text{Дж}\cdot\text{с}$.
}
\answer{%
    $
        N = \frac{Pt\lambda}{hc}
           = \frac{
                200\,\text{мВт} \cdot 30 \cdot 60 \units{с} \cdot 600\,\text{нм}
            }{
                6{,}626 \cdot 10^{-34}\,\text{Дж}\cdot\text{с} \cdot 3 \cdot 10^{8}\,\frac{\text{м}}{\text{с}}
           }
           \approx 10{,}87 \cdot 10^{20}\units{фотонов}
    $
}
\solutionspace{120pt}

\tasknumber{2}%
\task{%
    Определите название цвета по длине волны в вакууме
    и частоту колебаний электромагнитного поля в ней:
    \begin{enumerate}
        \item $450\,\text{нм}$,
        \item $595\,\text{нм}$,
        \item $720\,\text{нм}$,
        \item $580\,\text{нм}$.
    \end{enumerate}
}
\solutionspace{80pt}

\tasknumber{3}%
\task{%
    Определите энергию фотона излучения частотой $6 \cdot 10^{16}\,\text{Гц}$.
    Ответ получите в джоулях и в электронвольтах.
}
\solutionspace{80pt}

\tasknumber{4}%
\task{%
    Определите энергию фотона с длиной волны $150\,\text{нм}$.
    Ответ выразите в электронвольтах.
    Способен ли человеческий глаз увидеть один такой квант? А импульс таких квантов?'
}
\solutionspace{80pt}

\tasknumber{5}%
\task{%
    Определите частоту колебаний вектора напряженности индукции магнитного поля
    в электромагнитной волне в вакууме, длина который составляет $2\,\text{см}$.
}
\solutionspace{80pt}

\tasknumber{6}%
\task{%
    Из формулы Планка выразите (нужен вывод, не только ответ)...
    \begin{enumerate}
        \item длину соответствующей электромагнитной волны,
        \item период колебаний индукции магнитного поля в соответствующей электромагнитной волне.
    \end{enumerate}
}

\variantsplitter

\addpersonalvariant{Даниил Палаткин}

\tasknumber{1}%
\task{%
    Сколько фотонов испускает за 5 минут лазер,
    если мощность его излучения $200\,\text{мВт}$?
    Длина волны излучения $600\,\text{нм}$.
    $h = 6{,}626 \cdot 10^{-34}\,\text{Дж}\cdot\text{с}$.
}
\answer{%
    $
        N = \frac{Pt\lambda}{hc}
           = \frac{
                200\,\text{мВт} \cdot 5 \cdot 60 \units{с} \cdot 600\,\text{нм}
            }{
                6{,}626 \cdot 10^{-34}\,\text{Дж}\cdot\text{с} \cdot 3 \cdot 10^{8}\,\frac{\text{м}}{\text{с}}
           }
           \approx 1{,}81 \cdot 10^{20}\units{фотонов}
    $
}
\solutionspace{120pt}

\tasknumber{2}%
\task{%
    Определите название цвета по длине волны в вакууме
    и частоту колебаний электромагнитного поля в ней:
    \begin{enumerate}
        \item $580\,\text{нм}$,
        \item $390\,\text{нм}$,
        \item $530\,\text{нм}$,
        \item $490\,\text{нм}$.
    \end{enumerate}
}
\solutionspace{80pt}

\tasknumber{3}%
\task{%
    Определите энергию фотона излучения частотой $7 \cdot 10^{16}\,\text{Гц}$.
    Ответ получите в джоулях и в электронвольтах.
}
\solutionspace{80pt}

\tasknumber{4}%
\task{%
    Определите энергию фотона с длиной волны $200\,\text{нм}$.
    Ответ выразите в джоулях.
    Способен ли человеческий глаз увидеть один такой квант? А импульс таких квантов?'
}
\solutionspace{80pt}

\tasknumber{5}%
\task{%
    Определите период колебаний вектора напряженности электрического поля
    в электромагнитной волне в вакууме, длина который составляет $5\,\text{м}$.
}
\solutionspace{80pt}

\tasknumber{6}%
\task{%
    Из формулы Планка выразите (нужен вывод, не только ответ)...
    \begin{enumerate}
        \item длину соответствующей электромагнитной волны,
        \item период колебаний электрического поля в соответствующей электромагнитной волне.
    \end{enumerate}
}

\variantsplitter

\addpersonalvariant{Станислав Пикун}

\tasknumber{1}%
\task{%
    Сколько фотонов испускает за 60 минут лазер,
    если мощность его излучения $15\,\text{мВт}$?
    Длина волны излучения $600\,\text{нм}$.
    $h = 6{,}626 \cdot 10^{-34}\,\text{Дж}\cdot\text{с}$.
}
\answer{%
    $
        N = \frac{Pt\lambda}{hc}
           = \frac{
                15\,\text{мВт} \cdot 60 \cdot 60 \units{с} \cdot 600\,\text{нм}
            }{
                6{,}626 \cdot 10^{-34}\,\text{Дж}\cdot\text{с} \cdot 3 \cdot 10^{8}\,\frac{\text{м}}{\text{с}}
           }
           \approx 1{,}63 \cdot 10^{20}\units{фотонов}
    $
}
\solutionspace{120pt}

\tasknumber{2}%
\task{%
    Определите название цвета по длине волны в вакууме
    и частоту колебаний электромагнитного поля в ней:
    \begin{enumerate}
        \item $580\,\text{нм}$,
        \item $595\,\text{нм}$,
        \item $720\,\text{нм}$,
        \item $420\,\text{нм}$.
    \end{enumerate}
}
\solutionspace{80pt}

\tasknumber{3}%
\task{%
    Определите энергию фотона излучения частотой $5 \cdot 10^{16}\,\text{Гц}$.
    Ответ получите в джоулях и в электронвольтах.
}
\solutionspace{80pt}

\tasknumber{4}%
\task{%
    Определите энергию фотона с длиной волны $900\,\text{нм}$.
    Ответ выразите в джоулях.
    Способен ли человеческий глаз увидеть один такой квант? А импульс таких квантов?'
}
\solutionspace{80pt}

\tasknumber{5}%
\task{%
    Определите частоту колебаний вектора напряженности индукции магнитного поля
    в электромагнитной волне в вакууме, длина который составляет $2\,\text{см}$.
}
\solutionspace{80pt}

\tasknumber{6}%
\task{%
    Из формулы Планка выразите (нужен вывод, не только ответ)...
    \begin{enumerate}
        \item длину соответствующей электромагнитной волны,
        \item период колебаний электрического поля в соответствующей электромагнитной волне.
    \end{enumerate}
}

\variantsplitter

\addpersonalvariant{Илья Пичугин}

\tasknumber{1}%
\task{%
    Сколько фотонов испускает за 40 минут лазер,
    если мощность его излучения $200\,\text{мВт}$?
    Длина волны излучения $500\,\text{нм}$.
    $h = 6{,}626 \cdot 10^{-34}\,\text{Дж}\cdot\text{с}$.
}
\answer{%
    $
        N = \frac{Pt\lambda}{hc}
           = \frac{
                200\,\text{мВт} \cdot 40 \cdot 60 \units{с} \cdot 500\,\text{нм}
            }{
                6{,}626 \cdot 10^{-34}\,\text{Дж}\cdot\text{с} \cdot 3 \cdot 10^{8}\,\frac{\text{м}}{\text{с}}
           }
           \approx 12{,}07 \cdot 10^{20}\units{фотонов}
    $
}
\solutionspace{120pt}

\tasknumber{2}%
\task{%
    Определите название цвета по длине волны в вакууме
    и частоту колебаний электромагнитного поля в ней:
    \begin{enumerate}
        \item $580\,\text{нм}$,
        \item $470\,\text{нм}$,
        \item $720\,\text{нм}$,
        \item $580\,\text{нм}$.
    \end{enumerate}
}
\solutionspace{80pt}

\tasknumber{3}%
\task{%
    Определите энергию фотона излучения частотой $9 \cdot 10^{16}\,\text{Гц}$.
    Ответ получите в джоулях и в электронвольтах.
}
\solutionspace{80pt}

\tasknumber{4}%
\task{%
    Определите энергию кванта света с длиной волны $400\,\text{нм}$.
    Ответ выразите в электронвольтах.
    Способен ли человеческий глаз увидеть один такой квант? А импульс таких квантов?'
}
\solutionspace{80pt}

\tasknumber{5}%
\task{%
    Определите период колебаний вектора напряженности электрического поля
    в электромагнитной волне в вакууме, длина который составляет $3\,\text{м}$.
}
\solutionspace{80pt}

\tasknumber{6}%
\task{%
    Из формулы Планка выразите (нужен вывод, не только ответ)...
    \begin{enumerate}
        \item длину соответствующей электромагнитной волны,
        \item период колебаний электрического поля в соответствующей электромагнитной волне.
    \end{enumerate}
}

\variantsplitter

\addpersonalvariant{Кирилл Севрюгин}

\tasknumber{1}%
\task{%
    Сколько фотонов испускает за 10 минут лазер,
    если мощность его излучения $15\,\text{мВт}$?
    Длина волны излучения $600\,\text{нм}$.
    $h = 6{,}626 \cdot 10^{-34}\,\text{Дж}\cdot\text{с}$.
}
\answer{%
    $
        N = \frac{Pt\lambda}{hc}
           = \frac{
                15\,\text{мВт} \cdot 10 \cdot 60 \units{с} \cdot 600\,\text{нм}
            }{
                6{,}626 \cdot 10^{-34}\,\text{Дж}\cdot\text{с} \cdot 3 \cdot 10^{8}\,\frac{\text{м}}{\text{с}}
           }
           \approx 0{,}27 \cdot 10^{20}\units{фотонов}
    $
}
\solutionspace{120pt}

\tasknumber{2}%
\task{%
    Определите название цвета по длине волны в вакууме
    и частоту колебаний электромагнитного поля в ней:
    \begin{enumerate}
        \item $660\,\text{нм}$,
        \item $595\,\text{нм}$,
        \item $610\,\text{нм}$,
        \item $420\,\text{нм}$.
    \end{enumerate}
}
\solutionspace{80pt}

\tasknumber{3}%
\task{%
    Определите энергию фотона излучения частотой $5 \cdot 10^{16}\,\text{Гц}$.
    Ответ получите в джоулях и в электронвольтах.
}
\solutionspace{80pt}

\tasknumber{4}%
\task{%
    Определите энергию кванта света с длиной волны $900\,\text{нм}$.
    Ответ выразите в джоулях.
    Способен ли человеческий глаз увидеть один такой квант? А импульс таких квантов?'
}
\solutionspace{80pt}

\tasknumber{5}%
\task{%
    Определите частоту колебаний вектора напряженности электрического поля
    в электромагнитной волне в вакууме, длина который составляет $3\,\text{см}$.
}
\solutionspace{80pt}

\tasknumber{6}%
\task{%
    Из формулы Планка выразите (нужен вывод, не только ответ)...
    \begin{enumerate}
        \item длину соответствующей электромагнитной волны,
        \item период колебаний индукции магнитного поля в соответствующей электромагнитной волне.
    \end{enumerate}
}

\variantsplitter

\addpersonalvariant{Илья Стратонников}

\tasknumber{1}%
\task{%
    Сколько фотонов испускает за 30 минут лазер,
    если мощность его излучения $75\,\text{мВт}$?
    Длина волны излучения $500\,\text{нм}$.
    $h = 6{,}626 \cdot 10^{-34}\,\text{Дж}\cdot\text{с}$.
}
\answer{%
    $
        N = \frac{Pt\lambda}{hc}
           = \frac{
                75\,\text{мВт} \cdot 30 \cdot 60 \units{с} \cdot 500\,\text{нм}
            }{
                6{,}626 \cdot 10^{-34}\,\text{Дж}\cdot\text{с} \cdot 3 \cdot 10^{8}\,\frac{\text{м}}{\text{с}}
           }
           \approx 3{,}40 \cdot 10^{20}\units{фотонов}
    $
}
\solutionspace{120pt}

\tasknumber{2}%
\task{%
    Определите название цвета по длине волны в вакууме
    и частоту колебаний электромагнитного поля в ней:
    \begin{enumerate}
        \item $450\,\text{нм}$,
        \item $595\,\text{нм}$,
        \item $720\,\text{нм}$,
        \item $580\,\text{нм}$.
    \end{enumerate}
}
\solutionspace{80pt}

\tasknumber{3}%
\task{%
    Определите энергию фотона излучения частотой $6 \cdot 10^{16}\,\text{Гц}$.
    Ответ получите в джоулях и в электронвольтах.
}
\solutionspace{80pt}

\tasknumber{4}%
\task{%
    Определите энергию кванта света с длиной волны $900\,\text{нм}$.
    Ответ выразите в электронвольтах.
    Способен ли человеческий глаз увидеть один такой квант? А импульс таких квантов?'
}
\solutionspace{80pt}

\tasknumber{5}%
\task{%
    Определите период колебаний вектора напряженности электрического поля
    в электромагнитной волне в вакууме, длина который составляет $3\,\text{м}$.
}
\solutionspace{80pt}

\tasknumber{6}%
\task{%
    Из формулы Планка выразите (нужен вывод, не только ответ)...
    \begin{enumerate}
        \item длину соответствующей электромагнитной волны,
        \item период колебаний индукции магнитного поля в соответствующей электромагнитной волне.
    \end{enumerate}
}

\variantsplitter

\addpersonalvariant{Иван Шустов}

\tasknumber{1}%
\task{%
    Сколько фотонов испускает за 40 минут лазер,
    если мощность его излучения $75\,\text{мВт}$?
    Длина волны излучения $750\,\text{нм}$.
    $h = 6{,}626 \cdot 10^{-34}\,\text{Дж}\cdot\text{с}$.
}
\answer{%
    $
        N = \frac{Pt\lambda}{hc}
           = \frac{
                75\,\text{мВт} \cdot 40 \cdot 60 \units{с} \cdot 750\,\text{нм}
            }{
                6{,}626 \cdot 10^{-34}\,\text{Дж}\cdot\text{с} \cdot 3 \cdot 10^{8}\,\frac{\text{м}}{\text{с}}
           }
           \approx 6{,}79 \cdot 10^{20}\units{фотонов}
    $
}
\solutionspace{120pt}

\tasknumber{2}%
\task{%
    Определите название цвета по длине волны в вакууме
    и частоту колебаний электромагнитного поля в ней:
    \begin{enumerate}
        \item $450\,\text{нм}$,
        \item $595\,\text{нм}$,
        \item $530\,\text{нм}$,
        \item $420\,\text{нм}$.
    \end{enumerate}
}
\solutionspace{80pt}

\tasknumber{3}%
\task{%
    Определите энергию фотона излучения частотой $5 \cdot 10^{16}\,\text{Гц}$.
    Ответ получите в джоулях и в электронвольтах.
}
\solutionspace{80pt}

\tasknumber{4}%
\task{%
    Определите энергию фотона с длиной волны $150\,\text{нм}$.
    Ответ выразите в джоулях.
    Способен ли человеческий глаз увидеть один такой квант? А импульс таких квантов?'
}
\solutionspace{80pt}

\tasknumber{5}%
\task{%
    Определите частоту колебаний вектора напряженности индукции магнитного поля
    в электромагнитной волне в вакууме, длина который составляет $5\,\text{см}$.
}
\solutionspace{80pt}

\tasknumber{6}%
\task{%
    Из формулы Планка выразите (нужен вывод, не только ответ)...
    \begin{enumerate}
        \item длину соответствующей электромагнитной волны,
        \item период колебаний индукции магнитного поля в соответствующей электромагнитной волне.
    \end{enumerate}
}
% autogenerated
