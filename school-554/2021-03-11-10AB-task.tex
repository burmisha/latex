\documentclass[12pt,a4paper]{amsart}%DVI-mode.
\usepackage{graphics,graphicx,epsfig}%DVI-mode.
% \documentclass[pdftex,12pt]{amsart} %PDF-mode.
% \usepackage[pdftex]{graphicx}       %PDF-mode.
% \usepackage[babel=true]{microtype}
% \usepackage[T1]{fontenc}
% \usepackage{lmodern}

\usepackage{cmap}
%\usepackage{a4wide}                 % Fit the text to A4 page tightly.
% \usepackage[utf8]{inputenc}
\usepackage[T2A]{fontenc}
\usepackage[english,russian]{babel} % Download Russian fonts.
\usepackage{amsmath,amsfonts,amssymb,amsthm,amscd,mathrsfs} % Use AMS symbols.
\usepackage{tikz}
\usetikzlibrary{circuits.ee.IEC}
\usetikzlibrary{shapes.geometric}
\usetikzlibrary{decorations.markings}
%\usetikzlibrary{dashs}
%\usetikzlibrary{info}


\textheight=28cm % высота текста
\textwidth=18cm % ширина текста
\topmargin=-2.5cm % отступ от верхнего края
\parskip=2pt % интервал между абзацами
\oddsidemargin=-1.5cm
\evensidemargin=-1.5cm 

\parindent=0pt % абзацный отступ
\tolerance=500 % терпимость к "жидким" строкам
\binoppenalty=10000 % штраф за перенос формул - 10000 - абсолютный запрет
\relpenalty=10000
\flushbottom % выравнивание высоты страниц
\pagenumbering{gobble}

\newcommand\bivec[2]{\begin{pmatrix} #1 \\ #2 \end{pmatrix}}

\newcommand\ol[1]{\overline{#1}}

\newcommand\p[1]{\Prob\!\left(#1\right)}
\newcommand\e[1]{\mathsf{E}\!\left(#1\right)}
\newcommand\disp[1]{\mathsf{D}\!\left(#1\right)}
%\newcommand\norm[2]{\mathcal{N}\!\cbr{#1,#2}}
\newcommand\sign{\text{ sign }}

\newcommand\al[1]{\begin{align*} #1 \end{align*}}
\newcommand\begcas[1]{\begin{cases}#1\end{cases}}
\newcommand\tab[2]{	\vspace{-#1pt}
						\begin{tabbing} 
						#2
						\end{tabbing}
					\vspace{-#1pt}
					}

\newcommand\maintext[1]{{\bfseries\sffamily{#1}}}
\newcommand\skipped[1]{ {\ensuremath{\text{\small{\sffamily{Пропущено:} #1} } } } }
\newcommand\simpletitle[1]{\begin{center} \maintext{#1} \end{center}}

\def\le{\leqslant}
\def\ge{\geqslant}
\def\Ell{\mathcal{L}}
\def\eps{{\varepsilon}}
\def\Rn{\mathbb{R}^n}
\def\RSS{\mathsf{RSS}}

\newcommand\foral[1]{\forall\,#1\:}
\newcommand\exist[1]{\exists\,#1\:\colon}

\newcommand\cbr[1]{\left(#1\right)} %circled brackets
\newcommand\fbr[1]{\left\{#1\right\}} %figure brackets
\newcommand\sbr[1]{\left[#1\right]} %square brackets
\newcommand\modul[1]{\left|#1\right|}

\newcommand\sqr[1]{\cbr{#1}^2}
\newcommand\inv[1]{\cbr{#1}^{-1}}

\newcommand\cdf[2]{\cdot\frac{#1}{#2}}
\newcommand\dd[2]{\frac{\partial#1}{\partial#2}}

\newcommand\integr[2]{\int\limits_{#1}^{#2}}
\newcommand\suml[2]{\sum\limits_{#1}^{#2}}
\newcommand\isum[2]{\sum\limits_{#1=#2}^{+\infty}}
\newcommand\idots[3]{#1_{#2},\ldots,#1_{#3}}
\newcommand\fdots[5]{#4{#1_{#2}}#5\ldots#5#4{#1_{#3}}}

\newcommand\obol[1]{O\!\cbr{#1}}
\newcommand\omal[1]{o\!\cbr{#1}}

\newcommand\addeps[2]{
	\begin{figure} [!ht] %lrp
		\centering
		\includegraphics[height=320px]{#1.eps}
		\vspace{-10pt}
		\caption{#2}
		\label{eps:#1}
	\end{figure}
}

\newcommand\addepssize[3]{
	\begin{figure} [!ht] %lrp hp
		\centering
		\includegraphics[height=#3px]{#1.eps}
		\vspace{-10pt}
		\caption{#2}
		\label{eps:#1}
	\end{figure}
}


\newcommand\norm[1]{\ensuremath{\left\|{#1}\right\|}}
\newcommand\ort{\bot}
\newcommand\theorem[1]{{\sffamily Теорема #1\ }}
\newcommand\lemma[1]{{\sffamily Лемма #1\ }}
\newcommand\difflim[2]{\frac{#1\cbr{#2 + \Delta#2} - #1\cbr{#2}}{\Delta #2}}
\renewcommand\proof[1]{\par\noindent$\square$ #1 \hfill$\blacksquare$\par}
\newcommand\defenition[1]{{\sffamilyОпределение #1\ }}

% \begin{document}
% %\raggedright
% \addclassdate{7}{20 апреля 2018}

\task 1
Площадь большого поршня гидравлического домкрата $S_1 = 20\units{см}^2$, а малого $S_2 = 0{,}5\units{см}^2.$ Груз какой максимальной массы можно поднять этим домкратом, если на малый поршень давить с силой не более $F=200\units{Н}?$ Силой трения от стенки цилиндров пренебречь.

\task 2
В сосуд налита вода. Расстояние от поверхности воды до дна $H = 0{,}5\units{м},$ площадь дна $S = 0{,}1\units{м}^2.$ Найти гидростатическое давление $P_1$ и полное давление $P_2$ вблизи дна. Найти силу давления воды на дно. Плотность воды \rhowater

\task 3
На лёгкий поршень площадью $S=900\units{см}^2,$ касающийся поверхности воды, поставили гирю массы $m=3\units{кг}$. Высота слоя воды в сосуде с вертикальными стенками $H = 20\units{см}$. Определить давление жидкости вблизи дна, если плотность воды \rhowater

\task 4
Давление газов в конце сгорания в цилиндре дизельного двигателя трактора $P = 9\units{МПа}.$ Диаметр цилиндра $d = 130\units{мм}.$ С какой силой газы давят на поршень в цилиндре? Площадь круга диаметром $D$ равна $S = \cfrac{\pi D^2}4.$

\task 5
Площадь малого поршня гидравлического подъёмника $S_1 = 0{,}8\units{см}^2$, а большого $S_2 = 40\units{см}^2.$ Какую силу $F$ надо приложить к малому поршню, чтобы поднять груз весом $P = 8\units{кН}?$

\task 6
Герметичный сосуд полностью заполнен водой и стоит на столе. На небольшой поршень площадью $S$ давят рукой с силой $F$. Поршень находится ниже крышки сосуда на $H_1$, выше дна на $H_2$ и может свободно перемещаться. Плотность воды $\rho$, атмосферное давление $P_A$. Найти давления $P_1$ и $P_2$ в воде вблизи крышки и дна сосуда.
\\ \\
\addclassdate{7}{20 апреля 2018}

\task 1
Площадь большого поршня гидравлического домкрата $S_1 = 20\units{см}^2$, а малого $S_2 = 0{,}5\units{см}^2.$ Груз какой максимальной массы можно поднять этим домкратом, если на малый поршень давить с силой не более $F=200\units{Н}?$ Силой трения от стенки цилиндров пренебречь.

\task 2
В сосуд налита вода. Расстояние от поверхности воды до дна $H = 0{,}5\units{м},$ площадь дна $S = 0{,}1\units{м}^2.$ Найти гидростатическое давление $P_1$ и полное давление $P_2$ вблизи дна. Найти силу давления воды на дно. Плотность воды \rhowater

\task 3
На лёгкий поршень площадью $S=900\units{см}^2,$ касающийся поверхности воды, поставили гирю массы $m=3\units{кг}$. Высота слоя воды в сосуде с вертикальными стенками $H = 20\units{см}$. Определить давление жидкости вблизи дна, если плотность воды \rhowater

\task 4
Давление газов в конце сгорания в цилиндре дизельного двигателя трактора $P = 9\units{МПа}.$ Диаметр цилиндра $d = 130\units{мм}.$ С какой силой газы давят на поршень в цилиндре? Площадь круга диаметром $D$ равна $S = \cfrac{\pi D^2}4.$

\task 5
Площадь малого поршня гидравлического подъёмника $S_1 = 0{,}8\units{см}^2$, а большого $S_2 = 40\units{см}^2.$ Какую силу $F$ надо приложить к малому поршню, чтобы поднять груз весом $P = 8\units{кН}?$

\task 6
Герметичный сосуд полностью заполнен водой и стоит на столе. На небольшой поршень площадью $S$ давят рукой с силой $F$. Поршень находится ниже крышки сосуда на $H_1$, выше дна на $H_2$ и может свободно перемещаться. Плотность воды $\rho$, атмосферное давление $P_A$. Найти давления $P_1$ и $P_2$ в воде вблизи крышки и дна сосуда.

\newpage

\adddate{8 класс. 20 апреля 2018}

\task 1
Между точками $A$ и $B$ электрической цепи подключены последовательно резисторы $R_1 = 10\units{Ом}$ и $R_2 = 20\units{Ом}$ и параллельно им $R_3 = 30\units{Ом}.$ Найдите эквивалентное сопротивление $R_{AB}$ этого участка цепи.

\task 2
Электрическая цепь состоит из последовательности $N$ одинаковых звеньев, в которых каждый резистор имеет сопротивление $r$. Последнее звено замкнуто резистором сопротивлением $R$. При каком соотношении $\cfrac{R}{r}$ сопротивление цепи не зависит от числа звеньев?

\task 3
Для измерения сопротивления $R$ проводника собрана электрическая цепь. Вольтметр $V$ показывает напряжение $U_V = 5\units{В},$ показание амперметра $A$ равно $I_A = 25\units{мА}.$ Найдите величину $R$ сопротивления проводника. Внутреннее сопротивление вольтметра $R_V = 1{,}0\units{кОм},$ внутреннее сопротивление амперметра $R_A = 2{,}0\units{Ом}.$

\task 4
Шкала гальванометра имеет $N=100$ делений, цена деления $\delta = 1\units{мкА}$. Внутреннее сопротивление гальванометра $R_G = 1{,}0\units{кОм}.$ Как из этого прибора сделать вольтметр для измерения напряжений до $U = 100\units{В}$ или амперметр для измерения токов силой до $I = 1\units{А}?$

\\ \\ \\ \\ \\ \\ \\ \\
\adddate{8 класс. 20 апреля 2018}

\task 1
Между точками $A$ и $B$ электрической цепи подключены последовательно резисторы $R_1 = 10\units{Ом}$ и $R_2 = 20\units{Ом}$ и параллельно им $R_3 = 30\units{Ом}.$ Найдите эквивалентное сопротивление $R_{AB}$ этого участка цепи.

\task 2
Электрическая цепь состоит из последовательности $N$ одинаковых звеньев, в которых каждый резистор имеет сопротивление $r$. Последнее звено замкнуто резистором сопротивлением $R$. При каком соотношении $\cfrac{R}{r}$ сопротивление цепи не зависит от числа звеньев?

\task 3
Для измерения сопротивления $R$ проводника собрана электрическая цепь. Вольтметр $V$ показывает напряжение $U_V = 5\units{В},$ показание амперметра $A$ равно $I_A = 25\units{мА}.$ Найдите величину $R$ сопротивления проводника. Внутреннее сопротивление вольтметра $R_V = 1{,}0\units{кОм},$ внутреннее сопротивление амперметра $R_A = 2{,}0\units{Ом}.$

\task 4
Шкала гальванометра имеет $N=100$ делений, цена деления $\delta = 1\units{мкА}$. Внутреннее сопротивление гальванометра $R_G = 1{,}0\units{кОм}.$ Как из этого прибора сделать вольтметр для измерения напряжений до $U = 100\units{В}$ или амперметр для измерения токов силой до $I = 1\units{А}?$


% % \begin{flushright}
\textsc{ГБОУ школа №554, 20 ноября 2018\,г.}
\end{flushright}

\begin{center}
\LARGE \textsc{Математический бой, 8 класс}
\end{center}

\problem{1} Есть тридцать карточек, на каждой написано по одному числу: на десяти карточках~–~$a$,  на десяти других~–~$b$ и на десяти оставшихся~–~$c$ (числа  различны). Известно, что к любым пяти карточкам можно подобрать ещё пять так, что сумма чисел на этих десяти карточках будет равна нулю. Докажите, что~одно из~чисел~$a, b, c$ равно нулю.

\problem{2} Вокруг стола стола пустили пакет с орешками. Первый взял один орешек, второй — 2, третий — 3 и так далее: каждый следующий брал на 1 орешек больше. Известно, что на втором круге было взято в сумме на 100 орешков больше, чем на первом. Сколько человек сидело за столом?

% \problem{2} Натуральное число разрешено увеличить на любое целое число процентов от 1 до 100, если при этом получаем натуральное число. Найдите наименьшее натуральное число, которое нельзя при помощи таких операций получить из~числа 1.

% \problem{3} Найти сумму $1^2 - 2^2 + 3^2 - 4^2 + 5^2 + \ldots - 2018^2$.

\problem{3} В кружке рукоделия, где занимается Валя, более 93\% участников~—~девочки. Какое наименьшее число участников может быть в таком кружке?

\problem{4} Произведение 2018 целых чисел равно 1. Может ли их сумма оказаться равной~0?

% \problem{4} Можно ли все натуральные числа от~1 до~9 записать в~клетки таблицы~$3\times3$ так, чтобы сумма в~любых двух соседних (по~вертикали или горизонтали) клетках равнялось простому числу?

\problem{5} На доске написано 2018 нулей и 2019 единиц. Женя стирает 2 числа и, если они были одинаковы, дописывает к оставшимся один ноль, а~если разные — единицу. Потом Женя повторяет эту операцию снова, потом ещё и~так далее. В~результате на~доске останется только одно число. Что это за~число?

\problem{6} Докажите, что в~любой компании людей найдутся 2~человека, имеющие равное число знакомых в этой компании (если $A$~знаком с~$B$, то~и $B$~знаком с~$A$).

\problem{7} Три колокола начинают бить одновременно. Интервалы между ударами колоколов соответственно составляют $\cfrac43$~секунды, $\cfrac53$~секунды и $2$~секунды. Совпавшие по времени удары воспринимаются за~один. Сколько ударов будет услышано за 1~минуту, включая первый и последний удары?

\problem{8} Восемь одинаковых момент расположены по кругу. Известно, что три из~них~— фальшивые, и они расположены рядом друг с~другом. Вес фальшивой монеты отличается от~веса настоящей. Все фальшивые монеты весят одинаково, но неизвестно, тяжелее или легче фальшивая монета настоящей. Покажите, что за~3~взвешивания на~чашечных весах без~гирь можно определить все фальшивые монеты.

% \end{document}

\begin{document}
\noanswers

\setdate{11~марта~2021}
\setclass{10«АБ»}

\addpersonalvariant{Михаил Бурмистров}

\tasknumber{1}%
\task{%
    Напротив физической величины укажите её обозначение и единицы измерения в СИ или запишите физический закон или формулу (в пункте «г)»):
    \begin{enumerate}
        \item количество теплоты,
        \item работа внешних сил,
        \item удельная теплоёмкость,
        \item первое начало термодинамики.
    \end{enumerate}
}
\solutionspace{20pt}

\tasknumber{2}%
\task{%
    Определите объём идеального одноатомного газа,
    если его внутренняя энергия при давлении $2\,\text{атм}$ составляет $500\,\text{кДж}$.
    $p_{\text{aтм}} = 100\,\text{кПа}$.
}
\answer{%
    $U = \frac 32 \nu R T = \frac 32 PV \implies V = \frac 23 \cdot \frac UP= \frac 23 \cdot \frac{ 500\,\text{кДж} }{ 2\,\text{атм} } \approx 1{,}67\,\text{м}^{3}.$
}
\solutionspace{40pt}

\tasknumber{3}%
\task{%
    Газ расширился от $250\,\text{л}$ до $650\,\text{л}$.
    Давление газа при этом оставалось постоянным и равным $1{,}2\,\text{атм}$.
    Определите работу газа, ответ выразите в килоджоулях.
    $p_{\text{aтм}} = 100\,\text{кПа}$.
}
\answer{%
    $A = P\Delta V = P(V_2 - V_1) = 1{,}2\,\text{атм} \cdot\cbr{650\,\text{л} - 250\,\text{л}} = 4{,}80\,\text{кДж}.$
}
\solutionspace{40pt}

\tasknumber{4}%
\task{%
    $40\,\text{моль}$ идеального одноатомного в результате адиабатического процесса нагрелся на $80\,\text{К}$.
    Определите работу газа.
    Кто совершил положительную работу: газ или внешние силы?
    Универсальная газовая постоянная $R = 8{,}31\,\frac{\text{Дж}}{\text{моль}\cdot\text{К}}$.
}
\answer{%
    \begin{align*}
    Q &= 0, Q = \Delta U + A_\text{газа} \implies \\
    \implies A_\text{газа} &= - \Delta U = - \frac 32 \nu R \Delta T = - \frac 32 \cdot 40\,\text{моль} \cdot8{,}31\,\frac{\text{Дж}}{\text{моль}\cdot\text{К}} \cdot80\,\text{К}= -39{,}90\,\text{кДж}, \text{внешние силы.}
    \end{align*}
}
\solutionspace{40pt}

\tasknumber{5}%
\task{%
    Как изменилась внутренняя энергия одноатомного идеального газа при переходе из состояния 1 в состояние 2?
    $P_1 = 2\,\text{МПа}$, $V_1 = 3\,\text{л}$, $P_2 = 1{,}5\,\text{МПа}$, $V_2 = 8\,\text{л}$.
    Как изменилась при этом температура газа?
}
\answer{%
    \begin{align*}
    P_1V_1 &= \nu R T_1, P_2V_2 = \nu R T_2, \\
    \Delta U &= U_2-U_1 = \frac 32 \nu R T_2- \frac 32 \nu R T_1 = \frac 32 P_2 V_2 - \frac 32 P_1 V_1= \frac 32 \cdot \cbr{1{,}5\,\text{МПа} \cdot8\,\text{л} - 2\,\text{МПа} \cdot3\,\text{л}} = 9000\,\text{Дж}.
    \\
    \frac{T_2}{T_1} &= \frac{\frac{P_2V_2}{\nu R}}{\frac{P_1V_1}{\nu R}} = \frac{P_2V_2}{P_1V_1}= \frac{1{,}5\,\text{МПа} \cdot8\,\text{л}}{2\,\text{МПа} \cdot3\,\text{л}} \approx 2{,}00.
    \end{align*}
}
\solutionspace{80pt}

\tasknumber{6}%
\task{%
    $3\,\text{моль}$ идеального одноатомного газа нагрели на $20\,\text{К}$.
    Определите изменение внутренней энергии газа.
    Увеличилась она или уменьшилась?
    Универсальная газовая постоянная $R = 8{,}31\,\frac{\text{Дж}}{\text{моль}\cdot\text{К}}$.
}
\answer{%
    $
        \Delta U = \frac 32 \nu R \Delta T
            =  \frac 32 \cdot 3\,\text{моль} \cdot8{,}31\,\frac{\text{Дж}}{\text{моль}\cdot\text{К}} \cdot20\,\text{К}
            = 747\,\text{Дж}.
            \text{Увеличилась.}
    $
}
\solutionspace{40pt}

\tasknumber{7}%
\task{%
    Газу сообщили некоторое количество теплоты,
    при этом половину его он потратил на совершение работы,
    одновременно увеличив свою внутреннюю энергию на $1200\,\text{Дж}$.
    Определите количество теплоты, сообщённое газу.
}
\answer{%
    \begin{align*}
    Q &= A' + \Delta U, A' = \frac 12 Q \implies Q\cdot\cbr{1 - \frac 12} = \Delta U \implies Q = \frac{\Delta U}{1 - \frac 12} = \frac{1200\,\text{Дж}}{1 - \frac 12} \approx 2400\,\text{Дж}.
    \\
    A' &= \frac 12 Q
        = \frac 12 \cdot \frac{\Delta U}{1 - \frac 12}
        = \frac{\Delta U}{2 - 1}
        = \frac{1200\,\text{Дж}}{2 - 1} \approx 1200\,\text{Дж}.
    \end{align*}
}
\solutionspace{60pt}

\tasknumber{8}%
\task{%
    В некотором процессе внешние силы совершили над газом работу $300\,\text{Дж}$,
    при этом его внутренняя энергия увеличилась на $250\,\text{Дж}$.
    Определите количество тепла, переданное при этом процессе газу.
    Явно пропишите, подводили газу тепло или же отводили.
}
\answer{%
    $
        Q = A_\text{газа} + \Delta U, A_\text{газа} = -A_\text{внешняя}
        \implies Q = A_\text{газа} + \Delta U = - 300\,\text{Дж} +  250\,\text{Дж} = -50\,\text{Дж}.
        \text{ Отводили.}
    $
}

\variantsplitter

\addpersonalvariant{Ирина Ан}

\tasknumber{1}%
\task{%
    Напротив физической величины укажите её обозначение и единицы измерения в СИ или запишите физический закон или формулу (в пункте «г)»):
    \begin{enumerate}
        \item изменение внутренней энергии,
        \item работа газа,
        \item молярная теплоёмкость,
        \item внутренняя энергия идеального одноатомного газа.
    \end{enumerate}
}
\solutionspace{20pt}

\tasknumber{2}%
\task{%
    Определите объём идеального одноатомного газа,
    если его внутренняя энергия при давлении $2\,\text{атм}$ составляет $500\,\text{кДж}$.
    $p_{\text{aтм}} = 100\,\text{кПа}$.
}
\answer{%
    $U = \frac 32 \nu R T = \frac 32 PV \implies V = \frac 23 \cdot \frac UP= \frac 23 \cdot \frac{ 500\,\text{кДж} }{ 2\,\text{атм} } \approx 1{,}67\,\text{м}^{3}.$
}
\solutionspace{40pt}

\tasknumber{3}%
\task{%
    Газ расширился от $150\,\text{л}$ до $650\,\text{л}$.
    Давление газа при этом оставалось постоянным и равным $1{,}5\,\text{атм}$.
    Определите работу газа, ответ выразите в килоджоулях.
    $p_{\text{aтм}} = 100\,\text{кПа}$.
}
\answer{%
    $A = P\Delta V = P(V_2 - V_1) = 1{,}5\,\text{атм} \cdot\cbr{650\,\text{л} - 150\,\text{л}} = 7{,}50\,\text{кДж}.$
}
\solutionspace{40pt}

\tasknumber{4}%
\task{%
    $50\,\text{моль}$ идеального одноатомного в результате адиабатического процесса нагрелся на $80\,\text{К}$.
    Определите работу газа.
    Кто совершил положительную работу: газ или внешние силы?
    Универсальная газовая постоянная $R = 8{,}31\,\frac{\text{Дж}}{\text{моль}\cdot\text{К}}$.
}
\answer{%
    \begin{align*}
    Q &= 0, Q = \Delta U + A_\text{газа} \implies \\
    \implies A_\text{газа} &= - \Delta U = - \frac 32 \nu R \Delta T = - \frac 32 \cdot 50\,\text{моль} \cdot8{,}31\,\frac{\text{Дж}}{\text{моль}\cdot\text{К}} \cdot80\,\text{К}= -49{,}90\,\text{кДж}, \text{внешние силы.}
    \end{align*}
}
\solutionspace{40pt}

\tasknumber{5}%
\task{%
    Как изменилась внутренняя энергия одноатомного идеального газа при переходе из состояния 1 в состояние 2?
    $P_1 = 4\,\text{МПа}$, $V_1 = 3\,\text{л}$, $P_2 = 1{,}5\,\text{МПа}$, $V_2 = 8\,\text{л}$.
    Как изменилась при этом температура газа?
}
\answer{%
    \begin{align*}
    P_1V_1 &= \nu R T_1, P_2V_2 = \nu R T_2, \\
    \Delta U &= U_2-U_1 = \frac 32 \nu R T_2- \frac 32 \nu R T_1 = \frac 32 P_2 V_2 - \frac 32 P_1 V_1= \frac 32 \cdot \cbr{1{,}5\,\text{МПа} \cdot8\,\text{л} - 4\,\text{МПа} \cdot3\,\text{л}} = 0\,\text{Дж}.
    \\
    \frac{T_2}{T_1} &= \frac{\frac{P_2V_2}{\nu R}}{\frac{P_1V_1}{\nu R}} = \frac{P_2V_2}{P_1V_1}= \frac{1{,}5\,\text{МПа} \cdot8\,\text{л}}{4\,\text{МПа} \cdot3\,\text{л}} \approx 1{,}00.
    \end{align*}
}
\solutionspace{80pt}

\tasknumber{6}%
\task{%
    $4\,\text{моль}$ идеального одноатомного газа нагрели на $20\,\text{К}$.
    Определите изменение внутренней энергии газа.
    Увеличилась она или уменьшилась?
    Универсальная газовая постоянная $R = 8{,}31\,\frac{\text{Дж}}{\text{моль}\cdot\text{К}}$.
}
\answer{%
    $
        \Delta U = \frac 32 \nu R \Delta T
            =  \frac 32 \cdot 4\,\text{моль} \cdot8{,}31\,\frac{\text{Дж}}{\text{моль}\cdot\text{К}} \cdot20\,\text{К}
            = 997\,\text{Дж}.
            \text{Увеличилась.}
    $
}
\solutionspace{40pt}

\tasknumber{7}%
\task{%
    Газу сообщили некоторое количество теплоты,
    при этом треть его он потратил на совершение работы,
    одновременно увеличив свою внутреннюю энергию на $3000\,\text{Дж}$.
    Определите количество теплоты, сообщённое газу.
}
\answer{%
    \begin{align*}
    Q &= A' + \Delta U, A' = \frac 13 Q \implies Q\cdot\cbr{1 - \frac 13} = \Delta U \implies Q = \frac{\Delta U}{1 - \frac 13} = \frac{3000\,\text{Дж}}{1 - \frac 13} \approx 4500\,\text{Дж}.
    \\
    A' &= \frac 13 Q
        = \frac 13 \cdot \frac{\Delta U}{1 - \frac 13}
        = \frac{\Delta U}{3 - 1}
        = \frac{3000\,\text{Дж}}{3 - 1} \approx 1500\,\text{Дж}.
    \end{align*}
}
\solutionspace{60pt}

\tasknumber{8}%
\task{%
    В некотором процессе внешние силы совершили над газом работу $100\,\text{Дж}$,
    при этом его внутренняя энергия увеличилась на $450\,\text{Дж}$.
    Определите количество тепла, переданное при этом процессе газу.
    Явно пропишите, подводили газу тепло или же отводили.
}
\answer{%
    $
        Q = A_\text{газа} + \Delta U, A_\text{газа} = -A_\text{внешняя}
        \implies Q = A_\text{газа} + \Delta U = - 100\,\text{Дж} +  450\,\text{Дж} = 350\,\text{Дж}.
        \text{ Подводили.}
    $
}

\variantsplitter

\addpersonalvariant{Софья Андрианова}

\tasknumber{1}%
\task{%
    Напротив физической величины укажите её обозначение и единицы измерения в СИ или запишите физический закон или формулу (в пункте «г)»):
    \begin{enumerate}
        \item изменение внутренней энергии,
        \item работа газа,
        \item удельная теплоёмкость,
        \item первое начало термодинамики.
    \end{enumerate}
}
\solutionspace{20pt}

\tasknumber{2}%
\task{%
    Определите объём идеального одноатомного газа,
    если его внутренняя энергия при давлении $3\,\text{атм}$ составляет $250\,\text{кДж}$.
    $p_{\text{aтм}} = 100\,\text{кПа}$.
}
\answer{%
    $U = \frac 32 \nu R T = \frac 32 PV \implies V = \frac 23 \cdot \frac UP= \frac 23 \cdot \frac{ 250\,\text{кДж} }{ 3\,\text{атм} } \approx 0{,}56\,\text{м}^{3}.$
}
\solutionspace{40pt}

\tasknumber{3}%
\task{%
    Газ расширился от $250\,\text{л}$ до $650\,\text{л}$.
    Давление газа при этом оставалось постоянным и равным $1{,}5\,\text{атм}$.
    Определите работу газа, ответ выразите в килоджоулях.
    $p_{\text{aтм}} = 100\,\text{кПа}$.
}
\answer{%
    $A = P\Delta V = P(V_2 - V_1) = 1{,}5\,\text{атм} \cdot\cbr{650\,\text{л} - 250\,\text{л}} = 6{,}00\,\text{кДж}.$
}
\solutionspace{40pt}

\tasknumber{4}%
\task{%
    $40\,\text{моль}$ идеального одноатомного в результате адиабатического процесса нагрелся на $45\,\text{К}$.
    Определите работу газа.
    Кто совершил положительную работу: газ или внешние силы?
    Универсальная газовая постоянная $R = 8{,}31\,\frac{\text{Дж}}{\text{моль}\cdot\text{К}}$.
}
\answer{%
    \begin{align*}
    Q &= 0, Q = \Delta U + A_\text{газа} \implies \\
    \implies A_\text{газа} &= - \Delta U = - \frac 32 \nu R \Delta T = - \frac 32 \cdot 40\,\text{моль} \cdot8{,}31\,\frac{\text{Дж}}{\text{моль}\cdot\text{К}} \cdot45\,\text{К}= -22{,}40\,\text{кДж}, \text{внешние силы.}
    \end{align*}
}
\solutionspace{40pt}

\tasknumber{5}%
\task{%
    Как изменилась внутренняя энергия одноатомного идеального газа при переходе из состояния 1 в состояние 2?
    $P_1 = 3\,\text{МПа}$, $V_1 = 7\,\text{л}$, $P_2 = 2{,}5\,\text{МПа}$, $V_2 = 8\,\text{л}$.
    Как изменилась при этом температура газа?
}
\answer{%
    \begin{align*}
    P_1V_1 &= \nu R T_1, P_2V_2 = \nu R T_2, \\
    \Delta U &= U_2-U_1 = \frac 32 \nu R T_2- \frac 32 \nu R T_1 = \frac 32 P_2 V_2 - \frac 32 P_1 V_1= \frac 32 \cdot \cbr{2{,}5\,\text{МПа} \cdot8\,\text{л} - 3\,\text{МПа} \cdot7\,\text{л}} = -1500\,\text{Дж}.
    \\
    \frac{T_2}{T_1} &= \frac{\frac{P_2V_2}{\nu R}}{\frac{P_1V_1}{\nu R}} = \frac{P_2V_2}{P_1V_1}= \frac{2{,}5\,\text{МПа} \cdot8\,\text{л}}{3\,\text{МПа} \cdot7\,\text{л}} \approx 0{,}95.
    \end{align*}
}
\solutionspace{80pt}

\tasknumber{6}%
\task{%
    $3\,\text{моль}$ идеального одноатомного газа нагрели на $30\,\text{К}$.
    Определите изменение внутренней энергии газа.
    Увеличилась она или уменьшилась?
    Универсальная газовая постоянная $R = 8{,}31\,\frac{\text{Дж}}{\text{моль}\cdot\text{К}}$.
}
\answer{%
    $
        \Delta U = \frac 32 \nu R \Delta T
            =  \frac 32 \cdot 3\,\text{моль} \cdot8{,}31\,\frac{\text{Дж}}{\text{моль}\cdot\text{К}} \cdot30\,\text{К}
            = 1121\,\text{Дж}.
            \text{Увеличилась.}
    $
}
\solutionspace{40pt}

\tasknumber{7}%
\task{%
    Газу сообщили некоторое количество теплоты,
    при этом треть его он потратил на совершение работы,
    одновременно увеличив свою внутреннюю энергию на $3000\,\text{Дж}$.
    Определите количество теплоты, сообщённое газу.
}
\answer{%
    \begin{align*}
    Q &= A' + \Delta U, A' = \frac 13 Q \implies Q\cdot\cbr{1 - \frac 13} = \Delta U \implies Q = \frac{\Delta U}{1 - \frac 13} = \frac{3000\,\text{Дж}}{1 - \frac 13} \approx 4500\,\text{Дж}.
    \\
    A' &= \frac 13 Q
        = \frac 13 \cdot \frac{\Delta U}{1 - \frac 13}
        = \frac{\Delta U}{3 - 1}
        = \frac{3000\,\text{Дж}}{3 - 1} \approx 1500\,\text{Дж}.
    \end{align*}
}
\solutionspace{60pt}

\tasknumber{8}%
\task{%
    В некотором процессе внешние силы совершили над газом работу $300\,\text{Дж}$,
    при этом его внутренняя энергия увеличилась на $350\,\text{Дж}$.
    Определите количество тепла, переданное при этом процессе газу.
    Явно пропишите, подводили газу тепло или же отводили.
}
\answer{%
    $
        Q = A_\text{газа} + \Delta U, A_\text{газа} = -A_\text{внешняя}
        \implies Q = A_\text{газа} + \Delta U = - 300\,\text{Дж} +  350\,\text{Дж} = 50\,\text{Дж}.
        \text{ Подводили.}
    $
}

\variantsplitter

\addpersonalvariant{Владимир Артемчук}

\tasknumber{1}%
\task{%
    Напротив физической величины укажите её обозначение и единицы измерения в СИ или запишите физический закон или формулу (в пункте «г)»):
    \begin{enumerate}
        \item количество теплоты,
        \item работа газа,
        \item молярная теплоёмкость,
        \item первое начало термодинамики.
    \end{enumerate}
}
\solutionspace{20pt}

\tasknumber{2}%
\task{%
    Определите объём идеального одноатомного газа,
    если его внутренняя энергия при давлении $5\,\text{атм}$ составляет $250\,\text{кДж}$.
    $p_{\text{aтм}} = 100\,\text{кПа}$.
}
\answer{%
    $U = \frac 32 \nu R T = \frac 32 PV \implies V = \frac 23 \cdot \frac UP= \frac 23 \cdot \frac{ 250\,\text{кДж} }{ 5\,\text{атм} } \approx 0{,}33\,\text{м}^{3}.$
}
\solutionspace{40pt}

\tasknumber{3}%
\task{%
    Газ расширился от $150\,\text{л}$ до $550\,\text{л}$.
    Давление газа при этом оставалось постоянным и равным $2{,}5\,\text{атм}$.
    Определите работу газа, ответ выразите в килоджоулях.
    $p_{\text{aтм}} = 100\,\text{кПа}$.
}
\answer{%
    $A = P\Delta V = P(V_2 - V_1) = 2{,}5\,\text{атм} \cdot\cbr{550\,\text{л} - 150\,\text{л}} = 10{,}00\,\text{кДж}.$
}
\solutionspace{40pt}

\tasknumber{4}%
\task{%
    $40\,\text{моль}$ идеального одноатомного в результате адиабатического процесса нагрелся на $25\,\text{К}$.
    Определите работу газа.
    Кто совершил положительную работу: газ или внешние силы?
    Универсальная газовая постоянная $R = 8{,}31\,\frac{\text{Дж}}{\text{моль}\cdot\text{К}}$.
}
\answer{%
    \begin{align*}
    Q &= 0, Q = \Delta U + A_\text{газа} \implies \\
    \implies A_\text{газа} &= - \Delta U = - \frac 32 \nu R \Delta T = - \frac 32 \cdot 40\,\text{моль} \cdot8{,}31\,\frac{\text{Дж}}{\text{моль}\cdot\text{К}} \cdot25\,\text{К}= -12{,}500\,\text{кДж}, \text{внешние силы.}
    \end{align*}
}
\solutionspace{40pt}

\tasknumber{5}%
\task{%
    Как изменилась внутренняя энергия одноатомного идеального газа при переходе из состояния 1 в состояние 2?
    $P_1 = 4\,\text{МПа}$, $V_1 = 7\,\text{л}$, $P_2 = 3{,}5\,\text{МПа}$, $V_2 = 4\,\text{л}$.
    Как изменилась при этом температура газа?
}
\answer{%
    \begin{align*}
    P_1V_1 &= \nu R T_1, P_2V_2 = \nu R T_2, \\
    \Delta U &= U_2-U_1 = \frac 32 \nu R T_2- \frac 32 \nu R T_1 = \frac 32 P_2 V_2 - \frac 32 P_1 V_1= \frac 32 \cdot \cbr{3{,}5\,\text{МПа} \cdot4\,\text{л} - 4\,\text{МПа} \cdot7\,\text{л}} = -21000\,\text{Дж}.
    \\
    \frac{T_2}{T_1} &= \frac{\frac{P_2V_2}{\nu R}}{\frac{P_1V_1}{\nu R}} = \frac{P_2V_2}{P_1V_1}= \frac{3{,}5\,\text{МПа} \cdot4\,\text{л}}{4\,\text{МПа} \cdot7\,\text{л}} \approx 0{,}50.
    \end{align*}
}
\solutionspace{80pt}

\tasknumber{6}%
\task{%
    $3\,\text{моль}$ идеального одноатомного газа нагрели на $20\,\text{К}$.
    Определите изменение внутренней энергии газа.
    Увеличилась она или уменьшилась?
    Универсальная газовая постоянная $R = 8{,}31\,\frac{\text{Дж}}{\text{моль}\cdot\text{К}}$.
}
\answer{%
    $
        \Delta U = \frac 32 \nu R \Delta T
            =  \frac 32 \cdot 3\,\text{моль} \cdot8{,}31\,\frac{\text{Дж}}{\text{моль}\cdot\text{К}} \cdot20\,\text{К}
            = 747\,\text{Дж}.
            \text{Увеличилась.}
    $
}
\solutionspace{40pt}

\tasknumber{7}%
\task{%
    Газу сообщили некоторое количество теплоты,
    при этом половину его он потратил на совершение работы,
    одновременно увеличив свою внутреннюю энергию на $1200\,\text{Дж}$.
    Определите работу, совершённую газом.
}
\answer{%
    \begin{align*}
    Q &= A' + \Delta U, A' = \frac 12 Q \implies Q\cdot\cbr{1 - \frac 12} = \Delta U \implies Q = \frac{\Delta U}{1 - \frac 12} = \frac{1200\,\text{Дж}}{1 - \frac 12} \approx 2400\,\text{Дж}.
    \\
    A' &= \frac 12 Q
        = \frac 12 \cdot \frac{\Delta U}{1 - \frac 12}
        = \frac{\Delta U}{2 - 1}
        = \frac{1200\,\text{Дж}}{2 - 1} \approx 1200\,\text{Дж}.
    \end{align*}
}
\solutionspace{60pt}

\tasknumber{8}%
\task{%
    В некотором процессе газ совершил работу $200\,\text{Дж}$,
    при этом его внутренняя энергия увеличилась на $350\,\text{Дж}$.
    Определите количество тепла, переданное при этом процессе газу.
    Явно пропишите, подводили газу тепло или же отводили.
}
\answer{%
    $
        Q = A_\text{газа} + \Delta U, A_\text{газа} = -A_\text{внешняя}
        \implies Q = A_\text{газа} + \Delta U =  200\,\text{Дж} +  350\,\text{Дж} = 550\,\text{Дж}.
        \text{ Подводили.}
    $
}

\variantsplitter

\addpersonalvariant{Софья Белянкина}

\tasknumber{1}%
\task{%
    Напротив физической величины укажите её обозначение и единицы измерения в СИ или запишите физический закон или формулу (в пункте «г)»):
    \begin{enumerate}
        \item количество теплоты,
        \item работа внешних сил,
        \item удельная теплоёмкость,
        \item первое начало термодинамики.
    \end{enumerate}
}
\solutionspace{20pt}

\tasknumber{2}%
\task{%
    Определите объём идеального одноатомного газа,
    если его внутренняя энергия при давлении $3\,\text{атм}$ составляет $250\,\text{кДж}$.
    $p_{\text{aтм}} = 100\,\text{кПа}$.
}
\answer{%
    $U = \frac 32 \nu R T = \frac 32 PV \implies V = \frac 23 \cdot \frac UP= \frac 23 \cdot \frac{ 250\,\text{кДж} }{ 3\,\text{атм} } \approx 0{,}56\,\text{м}^{3}.$
}
\solutionspace{40pt}

\tasknumber{3}%
\task{%
    Газ расширился от $250\,\text{л}$ до $650\,\text{л}$.
    Давление газа при этом оставалось постоянным и равным $2{,}5\,\text{атм}$.
    Определите работу газа, ответ выразите в килоджоулях.
    $p_{\text{aтм}} = 100\,\text{кПа}$.
}
\answer{%
    $A = P\Delta V = P(V_2 - V_1) = 2{,}5\,\text{атм} \cdot\cbr{650\,\text{л} - 250\,\text{л}} = 10{,}00\,\text{кДж}.$
}
\solutionspace{40pt}

\tasknumber{4}%
\task{%
    $40\,\text{моль}$ идеального одноатомного в результате адиабатического процесса остыл на $60\,\text{К}$.
    Определите работу газа.
    Кто совершил положительную работу: газ или внешние силы?
    Универсальная газовая постоянная $R = 8{,}31\,\frac{\text{Дж}}{\text{моль}\cdot\text{К}}$.
}
\answer{%
    \begin{align*}
    Q &= 0, Q = \Delta U + A_\text{газа} \implies \\
    \implies A_\text{газа} &= - \Delta U = - \frac 32 \nu R \Delta T =  \frac 32 \cdot 40\,\text{моль} \cdot8{,}31\,\frac{\text{Дж}}{\text{моль}\cdot\text{К}} \cdot60\,\text{К}= 29{,}9\,\text{кДж}, \text{газ.}
    \end{align*}
}
\solutionspace{40pt}

\tasknumber{5}%
\task{%
    Как изменилась внутренняя энергия одноатомного идеального газа при переходе из состояния 1 в состояние 2?
    $P_1 = 4\,\text{МПа}$, $V_1 = 7\,\text{л}$, $P_2 = 2{,}5\,\text{МПа}$, $V_2 = 6\,\text{л}$.
    Как изменилась при этом температура газа?
}
\answer{%
    \begin{align*}
    P_1V_1 &= \nu R T_1, P_2V_2 = \nu R T_2, \\
    \Delta U &= U_2-U_1 = \frac 32 \nu R T_2- \frac 32 \nu R T_1 = \frac 32 P_2 V_2 - \frac 32 P_1 V_1= \frac 32 \cdot \cbr{2{,}5\,\text{МПа} \cdot6\,\text{л} - 4\,\text{МПа} \cdot7\,\text{л}} = -19500\,\text{Дж}.
    \\
    \frac{T_2}{T_1} &= \frac{\frac{P_2V_2}{\nu R}}{\frac{P_1V_1}{\nu R}} = \frac{P_2V_2}{P_1V_1}= \frac{2{,}5\,\text{МПа} \cdot6\,\text{л}}{4\,\text{МПа} \cdot7\,\text{л}} \approx 0{,}54.
    \end{align*}
}
\solutionspace{80pt}

\tasknumber{6}%
\task{%
    $3\,\text{моль}$ идеального одноатомного газа охладили на $10\,\text{К}$.
    Определите изменение внутренней энергии газа.
    Увеличилась она или уменьшилась?
    Универсальная газовая постоянная $R = 8{,}31\,\frac{\text{Дж}}{\text{моль}\cdot\text{К}}$.
}
\answer{%
    $
        \Delta U = \frac 32 \nu R \Delta T
            = - \frac 32 \cdot 3\,\text{моль} \cdot8{,}31\,\frac{\text{Дж}}{\text{моль}\cdot\text{К}} \cdot10\,\text{К}
            = -373\,\text{Дж}.
            \text{Уменьшилась.}
    $
}
\solutionspace{40pt}

\tasknumber{7}%
\task{%
    Газу сообщили некоторое количество теплоты,
    при этом четверть его он потратил на совершение работы,
    одновременно увеличив свою внутреннюю энергию на $2400\,\text{Дж}$.
    Определите работу, совершённую газом.
}
\answer{%
    \begin{align*}
    Q &= A' + \Delta U, A' = \frac 14 Q \implies Q\cdot\cbr{1 - \frac 14} = \Delta U \implies Q = \frac{\Delta U}{1 - \frac 14} = \frac{2400\,\text{Дж}}{1 - \frac 14} \approx 3200\,\text{Дж}.
    \\
    A' &= \frac 14 Q
        = \frac 14 \cdot \frac{\Delta U}{1 - \frac 14}
        = \frac{\Delta U}{4 - 1}
        = \frac{2400\,\text{Дж}}{4 - 1} \approx 800\,\text{Дж}.
    \end{align*}
}
\solutionspace{60pt}

\tasknumber{8}%
\task{%
    В некотором процессе газ совершил работу $100\,\text{Дж}$,
    при этом его внутренняя энергия увеличилась на $450\,\text{Дж}$.
    Определите количество тепла, переданное при этом процессе газу.
    Явно пропишите, подводили газу тепло или же отводили.
}
\answer{%
    $
        Q = A_\text{газа} + \Delta U, A_\text{газа} = -A_\text{внешняя}
        \implies Q = A_\text{газа} + \Delta U =  100\,\text{Дж} +  450\,\text{Дж} = 550\,\text{Дж}.
        \text{ Подводили.}
    $
}

\variantsplitter

\addpersonalvariant{Варвара Егиазарян}

\tasknumber{1}%
\task{%
    Напротив физической величины укажите её обозначение и единицы измерения в СИ или запишите физический закон или формулу (в пункте «г)»):
    \begin{enumerate}
        \item количество теплоты,
        \item работа газа,
        \item молярная теплоёмкость,
        \item первое начало термодинамики.
    \end{enumerate}
}
\solutionspace{20pt}

\tasknumber{2}%
\task{%
    Определите объём идеального одноатомного газа,
    если его внутренняя энергия при давлении $2\,\text{атм}$ составляет $250\,\text{кДж}$.
    $p_{\text{aтм}} = 100\,\text{кПа}$.
}
\answer{%
    $U = \frac 32 \nu R T = \frac 32 PV \implies V = \frac 23 \cdot \frac UP= \frac 23 \cdot \frac{ 250\,\text{кДж} }{ 2\,\text{атм} } \approx 0{,}83\,\text{м}^{3}.$
}
\solutionspace{40pt}

\tasknumber{3}%
\task{%
    Газ расширился от $200\,\text{л}$ до $550\,\text{л}$.
    Давление газа при этом оставалось постоянным и равным $1{,}5\,\text{атм}$.
    Определите работу газа, ответ выразите в килоджоулях.
    $p_{\text{aтм}} = 100\,\text{кПа}$.
}
\answer{%
    $A = P\Delta V = P(V_2 - V_1) = 1{,}5\,\text{атм} \cdot\cbr{550\,\text{л} - 200\,\text{л}} = 5{,}25\,\text{кДж}.$
}
\solutionspace{40pt}

\tasknumber{4}%
\task{%
    $60\,\text{моль}$ идеального одноатомного в результате адиабатического процесса нагрелся на $25\,\text{К}$.
    Определите работу газа.
    Кто совершил положительную работу: газ или внешние силы?
    Универсальная газовая постоянная $R = 8{,}31\,\frac{\text{Дж}}{\text{моль}\cdot\text{К}}$.
}
\answer{%
    \begin{align*}
    Q &= 0, Q = \Delta U + A_\text{газа} \implies \\
    \implies A_\text{газа} &= - \Delta U = - \frac 32 \nu R \Delta T = - \frac 32 \cdot 60\,\text{моль} \cdot8{,}31\,\frac{\text{Дж}}{\text{моль}\cdot\text{К}} \cdot25\,\text{К}= -18{,}700\,\text{кДж}, \text{внешние силы.}
    \end{align*}
}
\solutionspace{40pt}

\tasknumber{5}%
\task{%
    Как изменилась внутренняя энергия одноатомного идеального газа при переходе из состояния 1 в состояние 2?
    $P_1 = 2\,\text{МПа}$, $V_1 = 5\,\text{л}$, $P_2 = 1{,}5\,\text{МПа}$, $V_2 = 2\,\text{л}$.
    Как изменилась при этом температура газа?
}
\answer{%
    \begin{align*}
    P_1V_1 &= \nu R T_1, P_2V_2 = \nu R T_2, \\
    \Delta U &= U_2-U_1 = \frac 32 \nu R T_2- \frac 32 \nu R T_1 = \frac 32 P_2 V_2 - \frac 32 P_1 V_1= \frac 32 \cdot \cbr{1{,}5\,\text{МПа} \cdot2\,\text{л} - 2\,\text{МПа} \cdot5\,\text{л}} = -10500\,\text{Дж}.
    \\
    \frac{T_2}{T_1} &= \frac{\frac{P_2V_2}{\nu R}}{\frac{P_1V_1}{\nu R}} = \frac{P_2V_2}{P_1V_1}= \frac{1{,}5\,\text{МПа} \cdot2\,\text{л}}{2\,\text{МПа} \cdot5\,\text{л}} \approx 0{,}30.
    \end{align*}
}
\solutionspace{80pt}

\tasknumber{6}%
\task{%
    $5\,\text{моль}$ идеального одноатомного газа нагрели на $20\,\text{К}$.
    Определите изменение внутренней энергии газа.
    Увеличилась она или уменьшилась?
    Универсальная газовая постоянная $R = 8{,}31\,\frac{\text{Дж}}{\text{моль}\cdot\text{К}}$.
}
\answer{%
    $
        \Delta U = \frac 32 \nu R \Delta T
            =  \frac 32 \cdot 5\,\text{моль} \cdot8{,}31\,\frac{\text{Дж}}{\text{моль}\cdot\text{К}} \cdot20\,\text{К}
            = 1246\,\text{Дж}.
            \text{Увеличилась.}
    $
}
\solutionspace{40pt}

\tasknumber{7}%
\task{%
    Газу сообщили некоторое количество теплоты,
    при этом половину его он потратил на совершение работы,
    одновременно увеличив свою внутреннюю энергию на $2400\,\text{Дж}$.
    Определите работу, совершённую газом.
}
\answer{%
    \begin{align*}
    Q &= A' + \Delta U, A' = \frac 12 Q \implies Q\cdot\cbr{1 - \frac 12} = \Delta U \implies Q = \frac{\Delta U}{1 - \frac 12} = \frac{2400\,\text{Дж}}{1 - \frac 12} \approx 4800\,\text{Дж}.
    \\
    A' &= \frac 12 Q
        = \frac 12 \cdot \frac{\Delta U}{1 - \frac 12}
        = \frac{\Delta U}{2 - 1}
        = \frac{2400\,\text{Дж}}{2 - 1} \approx 2400\,\text{Дж}.
    \end{align*}
}
\solutionspace{60pt}

\tasknumber{8}%
\task{%
    В некотором процессе газ совершил работу $100\,\text{Дж}$,
    при этом его внутренняя энергия уменьшилась на $250\,\text{Дж}$.
    Определите количество тепла, переданное при этом процессе газу.
    Явно пропишите, подводили газу тепло или же отводили.
}
\answer{%
    $
        Q = A_\text{газа} + \Delta U, A_\text{газа} = -A_\text{внешняя}
        \implies Q = A_\text{газа} + \Delta U =  100\,\text{Дж} - 250\,\text{Дж} = -150\,\text{Дж}.
        \text{ Отводили.}
    $
}

\variantsplitter

\addpersonalvariant{Владислав Емелин}

\tasknumber{1}%
\task{%
    Напротив физической величины укажите её обозначение и единицы измерения в СИ или запишите физический закон или формулу (в пункте «г)»):
    \begin{enumerate}
        \item количество теплоты,
        \item работа внешних сил,
        \item удельная теплоёмкость,
        \item первое начало термодинамики.
    \end{enumerate}
}
\solutionspace{20pt}

\tasknumber{2}%
\task{%
    Определите объём идеального одноатомного газа,
    если его внутренняя энергия при давлении $5\,\text{атм}$ составляет $400\,\text{кДж}$.
    $p_{\text{aтм}} = 100\,\text{кПа}$.
}
\answer{%
    $U = \frac 32 \nu R T = \frac 32 PV \implies V = \frac 23 \cdot \frac UP= \frac 23 \cdot \frac{ 400\,\text{кДж} }{ 5\,\text{атм} } \approx 0{,}53\,\text{м}^{3}.$
}
\solutionspace{40pt}

\tasknumber{3}%
\task{%
    Газ расширился от $200\,\text{л}$ до $450\,\text{л}$.
    Давление газа при этом оставалось постоянным и равным $1{,}5\,\text{атм}$.
    Определите работу газа, ответ выразите в килоджоулях.
    $p_{\text{aтм}} = 100\,\text{кПа}$.
}
\answer{%
    $A = P\Delta V = P(V_2 - V_1) = 1{,}5\,\text{атм} \cdot\cbr{450\,\text{л} - 200\,\text{л}} = 3{,}75\,\text{кДж}.$
}
\solutionspace{40pt}

\tasknumber{4}%
\task{%
    $60\,\text{моль}$ идеального одноатомного в результате адиабатического процесса нагрелся на $120\,\text{К}$.
    Определите работу газа.
    Кто совершил положительную работу: газ или внешние силы?
    Универсальная газовая постоянная $R = 8{,}31\,\frac{\text{Дж}}{\text{моль}\cdot\text{К}}$.
}
\answer{%
    \begin{align*}
    Q &= 0, Q = \Delta U + A_\text{газа} \implies \\
    \implies A_\text{газа} &= - \Delta U = - \frac 32 \nu R \Delta T = - \frac 32 \cdot 60\,\text{моль} \cdot8{,}31\,\frac{\text{Дж}}{\text{моль}\cdot\text{К}} \cdot120\,\text{К}= -89{,}70\,\text{кДж}, \text{внешние силы.}
    \end{align*}
}
\solutionspace{40pt}

\tasknumber{5}%
\task{%
    Как изменилась внутренняя энергия одноатомного идеального газа при переходе из состояния 1 в состояние 2?
    $P_1 = 4\,\text{МПа}$, $V_1 = 5\,\text{л}$, $P_2 = 3{,}5\,\text{МПа}$, $V_2 = 6\,\text{л}$.
    Как изменилась при этом температура газа?
}
\answer{%
    \begin{align*}
    P_1V_1 &= \nu R T_1, P_2V_2 = \nu R T_2, \\
    \Delta U &= U_2-U_1 = \frac 32 \nu R T_2- \frac 32 \nu R T_1 = \frac 32 P_2 V_2 - \frac 32 P_1 V_1= \frac 32 \cdot \cbr{3{,}5\,\text{МПа} \cdot6\,\text{л} - 4\,\text{МПа} \cdot5\,\text{л}} = 1500\,\text{Дж}.
    \\
    \frac{T_2}{T_1} &= \frac{\frac{P_2V_2}{\nu R}}{\frac{P_1V_1}{\nu R}} = \frac{P_2V_2}{P_1V_1}= \frac{3{,}5\,\text{МПа} \cdot6\,\text{л}}{4\,\text{МПа} \cdot5\,\text{л}} \approx 1{,}05.
    \end{align*}
}
\solutionspace{80pt}

\tasknumber{6}%
\task{%
    $5\,\text{моль}$ идеального одноатомного газа нагрели на $30\,\text{К}$.
    Определите изменение внутренней энергии газа.
    Увеличилась она или уменьшилась?
    Универсальная газовая постоянная $R = 8{,}31\,\frac{\text{Дж}}{\text{моль}\cdot\text{К}}$.
}
\answer{%
    $
        \Delta U = \frac 32 \nu R \Delta T
            =  \frac 32 \cdot 5\,\text{моль} \cdot8{,}31\,\frac{\text{Дж}}{\text{моль}\cdot\text{К}} \cdot30\,\text{К}
            = 1869\,\text{Дж}.
            \text{Увеличилась.}
    $
}
\solutionspace{40pt}

\tasknumber{7}%
\task{%
    Газу сообщили некоторое количество теплоты,
    при этом четверть его он потратил на совершение работы,
    одновременно увеличив свою внутреннюю энергию на $2400\,\text{Дж}$.
    Определите количество теплоты, сообщённое газу.
}
\answer{%
    \begin{align*}
    Q &= A' + \Delta U, A' = \frac 14 Q \implies Q\cdot\cbr{1 - \frac 14} = \Delta U \implies Q = \frac{\Delta U}{1 - \frac 14} = \frac{2400\,\text{Дж}}{1 - \frac 14} \approx 3200\,\text{Дж}.
    \\
    A' &= \frac 14 Q
        = \frac 14 \cdot \frac{\Delta U}{1 - \frac 14}
        = \frac{\Delta U}{4 - 1}
        = \frac{2400\,\text{Дж}}{4 - 1} \approx 800\,\text{Дж}.
    \end{align*}
}
\solutionspace{60pt}

\tasknumber{8}%
\task{%
    В некотором процессе газ совершил работу $300\,\text{Дж}$,
    при этом его внутренняя энергия уменьшилась на $350\,\text{Дж}$.
    Определите количество тепла, переданное при этом процессе газу.
    Явно пропишите, подводили газу тепло или же отводили.
}
\answer{%
    $
        Q = A_\text{газа} + \Delta U, A_\text{газа} = -A_\text{внешняя}
        \implies Q = A_\text{газа} + \Delta U =  300\,\text{Дж} - 350\,\text{Дж} = -50\,\text{Дж}.
        \text{ Отводили.}
    $
}

\variantsplitter

\addpersonalvariant{Артём Жичин}

\tasknumber{1}%
\task{%
    Напротив физической величины укажите её обозначение и единицы измерения в СИ или запишите физический закон или формулу (в пункте «г)»):
    \begin{enumerate}
        \item изменение внутренней энергии,
        \item работа газа,
        \item молярная теплоёмкость,
        \item внутренняя энергия идеального одноатомного газа.
    \end{enumerate}
}
\solutionspace{20pt}

\tasknumber{2}%
\task{%
    Определите объём идеального одноатомного газа,
    если его внутренняя энергия при давлении $4\,\text{атм}$ составляет $300\,\text{кДж}$.
    $p_{\text{aтм}} = 100\,\text{кПа}$.
}
\answer{%
    $U = \frac 32 \nu R T = \frac 32 PV \implies V = \frac 23 \cdot \frac UP= \frac 23 \cdot \frac{ 300\,\text{кДж} }{ 4\,\text{атм} } \approx 0{,}50\,\text{м}^{3}.$
}
\solutionspace{40pt}

\tasknumber{3}%
\task{%
    Газ расширился от $350\,\text{л}$ до $450\,\text{л}$.
    Давление газа при этом оставалось постоянным и равным $1{,}8\,\text{атм}$.
    Определите работу газа, ответ выразите в килоджоулях.
    $p_{\text{aтм}} = 100\,\text{кПа}$.
}
\answer{%
    $A = P\Delta V = P(V_2 - V_1) = 1{,}8\,\text{атм} \cdot\cbr{450\,\text{л} - 350\,\text{л}} = 1{,}80\,\text{кДж}.$
}
\solutionspace{40pt}

\tasknumber{4}%
\task{%
    $60\,\text{моль}$ идеального одноатомного в результате адиабатического процесса остыл на $120\,\text{К}$.
    Определите работу газа.
    Кто совершил положительную работу: газ или внешние силы?
    Универсальная газовая постоянная $R = 8{,}31\,\frac{\text{Дж}}{\text{моль}\cdot\text{К}}$.
}
\answer{%
    \begin{align*}
    Q &= 0, Q = \Delta U + A_\text{газа} \implies \\
    \implies A_\text{газа} &= - \Delta U = - \frac 32 \nu R \Delta T =  \frac 32 \cdot 60\,\text{моль} \cdot8{,}31\,\frac{\text{Дж}}{\text{моль}\cdot\text{К}} \cdot120\,\text{К}= 89{,}7\,\text{кДж}, \text{газ.}
    \end{align*}
}
\solutionspace{40pt}

\tasknumber{5}%
\task{%
    Как изменилась внутренняя энергия одноатомного идеального газа при переходе из состояния 1 в состояние 2?
    $P_1 = 2\,\text{МПа}$, $V_1 = 3\,\text{л}$, $P_2 = 3{,}5\,\text{МПа}$, $V_2 = 4\,\text{л}$.
    Как изменилась при этом температура газа?
}
\answer{%
    \begin{align*}
    P_1V_1 &= \nu R T_1, P_2V_2 = \nu R T_2, \\
    \Delta U &= U_2-U_1 = \frac 32 \nu R T_2- \frac 32 \nu R T_1 = \frac 32 P_2 V_2 - \frac 32 P_1 V_1= \frac 32 \cdot \cbr{3{,}5\,\text{МПа} \cdot4\,\text{л} - 2\,\text{МПа} \cdot3\,\text{л}} = 12000\,\text{Дж}.
    \\
    \frac{T_2}{T_1} &= \frac{\frac{P_2V_2}{\nu R}}{\frac{P_1V_1}{\nu R}} = \frac{P_2V_2}{P_1V_1}= \frac{3{,}5\,\text{МПа} \cdot4\,\text{л}}{2\,\text{МПа} \cdot3\,\text{л}} \approx 2{,}33.
    \end{align*}
}
\solutionspace{80pt}

\tasknumber{6}%
\task{%
    $5\,\text{моль}$ идеального одноатомного газа охладили на $30\,\text{К}$.
    Определите изменение внутренней энергии газа.
    Увеличилась она или уменьшилась?
    Универсальная газовая постоянная $R = 8{,}31\,\frac{\text{Дж}}{\text{моль}\cdot\text{К}}$.
}
\answer{%
    $
        \Delta U = \frac 32 \nu R \Delta T
            = - \frac 32 \cdot 5\,\text{моль} \cdot8{,}31\,\frac{\text{Дж}}{\text{моль}\cdot\text{К}} \cdot30\,\text{К}
            = -1869\,\text{Дж}.
            \text{Уменьшилась.}
    $
}
\solutionspace{40pt}

\tasknumber{7}%
\task{%
    Газу сообщили некоторое количество теплоты,
    при этом треть его он потратил на совершение работы,
    одновременно увеличив свою внутреннюю энергию на $1200\,\text{Дж}$.
    Определите количество теплоты, сообщённое газу.
}
\answer{%
    \begin{align*}
    Q &= A' + \Delta U, A' = \frac 13 Q \implies Q\cdot\cbr{1 - \frac 13} = \Delta U \implies Q = \frac{\Delta U}{1 - \frac 13} = \frac{1200\,\text{Дж}}{1 - \frac 13} \approx 1800\,\text{Дж}.
    \\
    A' &= \frac 13 Q
        = \frac 13 \cdot \frac{\Delta U}{1 - \frac 13}
        = \frac{\Delta U}{3 - 1}
        = \frac{1200\,\text{Дж}}{3 - 1} \approx 600\,\text{Дж}.
    \end{align*}
}
\solutionspace{60pt}

\tasknumber{8}%
\task{%
    В некотором процессе газ совершил работу $200\,\text{Дж}$,
    при этом его внутренняя энергия уменьшилась на $250\,\text{Дж}$.
    Определите количество тепла, переданное при этом процессе газу.
    Явно пропишите, подводили газу тепло или же отводили.
}
\answer{%
    $
        Q = A_\text{газа} + \Delta U, A_\text{газа} = -A_\text{внешняя}
        \implies Q = A_\text{газа} + \Delta U =  200\,\text{Дж} - 250\,\text{Дж} = -50\,\text{Дж}.
        \text{ Отводили.}
    $
}

\variantsplitter

\addpersonalvariant{Дарья Кошман}

\tasknumber{1}%
\task{%
    Напротив физической величины укажите её обозначение и единицы измерения в СИ или запишите физический закон или формулу (в пункте «г)»):
    \begin{enumerate}
        \item изменение внутренней энергии,
        \item работа внешних сил,
        \item молярная теплоёмкость,
        \item внутренняя энергия идеального одноатомного газа.
    \end{enumerate}
}
\solutionspace{20pt}

\tasknumber{2}%
\task{%
    Определите объём идеального одноатомного газа,
    если его внутренняя энергия при давлении $4\,\text{атм}$ составляет $300\,\text{кДж}$.
    $p_{\text{aтм}} = 100\,\text{кПа}$.
}
\answer{%
    $U = \frac 32 \nu R T = \frac 32 PV \implies V = \frac 23 \cdot \frac UP= \frac 23 \cdot \frac{ 300\,\text{кДж} }{ 4\,\text{атм} } \approx 0{,}50\,\text{м}^{3}.$
}
\solutionspace{40pt}

\tasknumber{3}%
\task{%
    Газ расширился от $200\,\text{л}$ до $550\,\text{л}$.
    Давление газа при этом оставалось постоянным и равным $1{,}2\,\text{атм}$.
    Определите работу газа, ответ выразите в килоджоулях.
    $p_{\text{aтм}} = 100\,\text{кПа}$.
}
\answer{%
    $A = P\Delta V = P(V_2 - V_1) = 1{,}2\,\text{атм} \cdot\cbr{550\,\text{л} - 200\,\text{л}} = 4{,}20\,\text{кДж}.$
}
\solutionspace{40pt}

\tasknumber{4}%
\task{%
    $50\,\text{моль}$ идеального одноатомного в результате адиабатического процесса нагрелся на $25\,\text{К}$.
    Определите работу газа.
    Кто совершил положительную работу: газ или внешние силы?
    Универсальная газовая постоянная $R = 8{,}31\,\frac{\text{Дж}}{\text{моль}\cdot\text{К}}$.
}
\answer{%
    \begin{align*}
    Q &= 0, Q = \Delta U + A_\text{газа} \implies \\
    \implies A_\text{газа} &= - \Delta U = - \frac 32 \nu R \Delta T = - \frac 32 \cdot 50\,\text{моль} \cdot8{,}31\,\frac{\text{Дж}}{\text{моль}\cdot\text{К}} \cdot25\,\text{К}= -15{,}600\,\text{кДж}, \text{внешние силы.}
    \end{align*}
}
\solutionspace{40pt}

\tasknumber{5}%
\task{%
    Как изменилась внутренняя энергия одноатомного идеального газа при переходе из состояния 1 в состояние 2?
    $P_1 = 4\,\text{МПа}$, $V_1 = 7\,\text{л}$, $P_2 = 2{,}5\,\text{МПа}$, $V_2 = 4\,\text{л}$.
    Как изменилась при этом температура газа?
}
\answer{%
    \begin{align*}
    P_1V_1 &= \nu R T_1, P_2V_2 = \nu R T_2, \\
    \Delta U &= U_2-U_1 = \frac 32 \nu R T_2- \frac 32 \nu R T_1 = \frac 32 P_2 V_2 - \frac 32 P_1 V_1= \frac 32 \cdot \cbr{2{,}5\,\text{МПа} \cdot4\,\text{л} - 4\,\text{МПа} \cdot7\,\text{л}} = -27000\,\text{Дж}.
    \\
    \frac{T_2}{T_1} &= \frac{\frac{P_2V_2}{\nu R}}{\frac{P_1V_1}{\nu R}} = \frac{P_2V_2}{P_1V_1}= \frac{2{,}5\,\text{МПа} \cdot4\,\text{л}}{4\,\text{МПа} \cdot7\,\text{л}} \approx 0{,}36.
    \end{align*}
}
\solutionspace{80pt}

\tasknumber{6}%
\task{%
    $4\,\text{моль}$ идеального одноатомного газа нагрели на $20\,\text{К}$.
    Определите изменение внутренней энергии газа.
    Увеличилась она или уменьшилась?
    Универсальная газовая постоянная $R = 8{,}31\,\frac{\text{Дж}}{\text{моль}\cdot\text{К}}$.
}
\answer{%
    $
        \Delta U = \frac 32 \nu R \Delta T
            =  \frac 32 \cdot 4\,\text{моль} \cdot8{,}31\,\frac{\text{Дж}}{\text{моль}\cdot\text{К}} \cdot20\,\text{К}
            = 997\,\text{Дж}.
            \text{Увеличилась.}
    $
}
\solutionspace{40pt}

\tasknumber{7}%
\task{%
    Газу сообщили некоторое количество теплоты,
    при этом половину его он потратил на совершение работы,
    одновременно увеличив свою внутреннюю энергию на $1500\,\text{Дж}$.
    Определите количество теплоты, сообщённое газу.
}
\answer{%
    \begin{align*}
    Q &= A' + \Delta U, A' = \frac 12 Q \implies Q\cdot\cbr{1 - \frac 12} = \Delta U \implies Q = \frac{\Delta U}{1 - \frac 12} = \frac{1500\,\text{Дж}}{1 - \frac 12} \approx 3000\,\text{Дж}.
    \\
    A' &= \frac 12 Q
        = \frac 12 \cdot \frac{\Delta U}{1 - \frac 12}
        = \frac{\Delta U}{2 - 1}
        = \frac{1500\,\text{Дж}}{2 - 1} \approx 1500\,\text{Дж}.
    \end{align*}
}
\solutionspace{60pt}

\tasknumber{8}%
\task{%
    В некотором процессе газ совершил работу $100\,\text{Дж}$,
    при этом его внутренняя энергия уменьшилась на $450\,\text{Дж}$.
    Определите количество тепла, переданное при этом процессе газу.
    Явно пропишите, подводили газу тепло или же отводили.
}
\answer{%
    $
        Q = A_\text{газа} + \Delta U, A_\text{газа} = -A_\text{внешняя}
        \implies Q = A_\text{газа} + \Delta U =  100\,\text{Дж} - 450\,\text{Дж} = -350\,\text{Дж}.
        \text{ Отводили.}
    $
}

\variantsplitter

\addpersonalvariant{Анна Кузьмичёва}

\tasknumber{1}%
\task{%
    Напротив физической величины укажите её обозначение и единицы измерения в СИ или запишите физический закон или формулу (в пункте «г)»):
    \begin{enumerate}
        \item изменение внутренней энергии,
        \item работа внешних сил,
        \item молярная теплоёмкость,
        \item внутренняя энергия идеального одноатомного газа.
    \end{enumerate}
}
\solutionspace{20pt}

\tasknumber{2}%
\task{%
    Определите объём идеального одноатомного газа,
    если его внутренняя энергия при давлении $6\,\text{атм}$ составляет $300\,\text{кДж}$.
    $p_{\text{aтм}} = 100\,\text{кПа}$.
}
\answer{%
    $U = \frac 32 \nu R T = \frac 32 PV \implies V = \frac 23 \cdot \frac UP= \frac 23 \cdot \frac{ 300\,\text{кДж} }{ 6\,\text{атм} } \approx 0{,}33\,\text{м}^{3}.$
}
\solutionspace{40pt}

\tasknumber{3}%
\task{%
    Газ расширился от $150\,\text{л}$ до $650\,\text{л}$.
    Давление газа при этом оставалось постоянным и равным $1{,}2\,\text{атм}$.
    Определите работу газа, ответ выразите в килоджоулях.
    $p_{\text{aтм}} = 100\,\text{кПа}$.
}
\answer{%
    $A = P\Delta V = P(V_2 - V_1) = 1{,}2\,\text{атм} \cdot\cbr{650\,\text{л} - 150\,\text{л}} = 6{,}00\,\text{кДж}.$
}
\solutionspace{40pt}

\tasknumber{4}%
\task{%
    $60\,\text{моль}$ идеального одноатомного в результате адиабатического процесса остыл на $25\,\text{К}$.
    Определите работу газа.
    Кто совершил положительную работу: газ или внешние силы?
    Универсальная газовая постоянная $R = 8{,}31\,\frac{\text{Дж}}{\text{моль}\cdot\text{К}}$.
}
\answer{%
    \begin{align*}
    Q &= 0, Q = \Delta U + A_\text{газа} \implies \\
    \implies A_\text{газа} &= - \Delta U = - \frac 32 \nu R \Delta T =  \frac 32 \cdot 60\,\text{моль} \cdot8{,}31\,\frac{\text{Дж}}{\text{моль}\cdot\text{К}} \cdot25\,\text{К}= 18{,}7\,\text{кДж}, \text{газ.}
    \end{align*}
}
\solutionspace{40pt}

\tasknumber{5}%
\task{%
    Как изменилась внутренняя энергия одноатомного идеального газа при переходе из состояния 1 в состояние 2?
    $P_1 = 4\,\text{МПа}$, $V_1 = 7\,\text{л}$, $P_2 = 4{,}5\,\text{МПа}$, $V_2 = 2\,\text{л}$.
    Как изменилась при этом температура газа?
}
\answer{%
    \begin{align*}
    P_1V_1 &= \nu R T_1, P_2V_2 = \nu R T_2, \\
    \Delta U &= U_2-U_1 = \frac 32 \nu R T_2- \frac 32 \nu R T_1 = \frac 32 P_2 V_2 - \frac 32 P_1 V_1= \frac 32 \cdot \cbr{4{,}5\,\text{МПа} \cdot2\,\text{л} - 4\,\text{МПа} \cdot7\,\text{л}} = -28500\,\text{Дж}.
    \\
    \frac{T_2}{T_1} &= \frac{\frac{P_2V_2}{\nu R}}{\frac{P_1V_1}{\nu R}} = \frac{P_2V_2}{P_1V_1}= \frac{4{,}5\,\text{МПа} \cdot2\,\text{л}}{4\,\text{МПа} \cdot7\,\text{л}} \approx 0{,}32.
    \end{align*}
}
\solutionspace{80pt}

\tasknumber{6}%
\task{%
    $5\,\text{моль}$ идеального одноатомного газа охладили на $20\,\text{К}$.
    Определите изменение внутренней энергии газа.
    Увеличилась она или уменьшилась?
    Универсальная газовая постоянная $R = 8{,}31\,\frac{\text{Дж}}{\text{моль}\cdot\text{К}}$.
}
\answer{%
    $
        \Delta U = \frac 32 \nu R \Delta T
            = - \frac 32 \cdot 5\,\text{моль} \cdot8{,}31\,\frac{\text{Дж}}{\text{моль}\cdot\text{К}} \cdot20\,\text{К}
            = -1246\,\text{Дж}.
            \text{Уменьшилась.}
    $
}
\solutionspace{40pt}

\tasknumber{7}%
\task{%
    Газу сообщили некоторое количество теплоты,
    при этом треть его он потратил на совершение работы,
    одновременно увеличив свою внутреннюю энергию на $2400\,\text{Дж}$.
    Определите количество теплоты, сообщённое газу.
}
\answer{%
    \begin{align*}
    Q &= A' + \Delta U, A' = \frac 13 Q \implies Q\cdot\cbr{1 - \frac 13} = \Delta U \implies Q = \frac{\Delta U}{1 - \frac 13} = \frac{2400\,\text{Дж}}{1 - \frac 13} \approx 3600\,\text{Дж}.
    \\
    A' &= \frac 13 Q
        = \frac 13 \cdot \frac{\Delta U}{1 - \frac 13}
        = \frac{\Delta U}{3 - 1}
        = \frac{2400\,\text{Дж}}{3 - 1} \approx 1200\,\text{Дж}.
    \end{align*}
}
\solutionspace{60pt}

\tasknumber{8}%
\task{%
    В некотором процессе внешние силы совершили над газом работу $200\,\text{Дж}$,
    при этом его внутренняя энергия увеличилась на $150\,\text{Дж}$.
    Определите количество тепла, переданное при этом процессе газу.
    Явно пропишите, подводили газу тепло или же отводили.
}
\answer{%
    $
        Q = A_\text{газа} + \Delta U, A_\text{газа} = -A_\text{внешняя}
        \implies Q = A_\text{газа} + \Delta U = - 200\,\text{Дж} +  150\,\text{Дж} = -50\,\text{Дж}.
        \text{ Отводили.}
    $
}

\variantsplitter

\addpersonalvariant{Алёна Куприянова}

\tasknumber{1}%
\task{%
    Напротив физической величины укажите её обозначение и единицы измерения в СИ или запишите физический закон или формулу (в пункте «г)»):
    \begin{enumerate}
        \item изменение внутренней энергии,
        \item работа внешних сил,
        \item удельная теплоёмкость,
        \item внутренняя энергия идеального одноатомного газа.
    \end{enumerate}
}
\solutionspace{20pt}

\tasknumber{2}%
\task{%
    Определите объём идеального одноатомного газа,
    если его внутренняя энергия при давлении $3\,\text{атм}$ составляет $250\,\text{кДж}$.
    $p_{\text{aтм}} = 100\,\text{кПа}$.
}
\answer{%
    $U = \frac 32 \nu R T = \frac 32 PV \implies V = \frac 23 \cdot \frac UP= \frac 23 \cdot \frac{ 250\,\text{кДж} }{ 3\,\text{атм} } \approx 0{,}56\,\text{м}^{3}.$
}
\solutionspace{40pt}

\tasknumber{3}%
\task{%
    Газ расширился от $200\,\text{л}$ до $650\,\text{л}$.
    Давление газа при этом оставалось постоянным и равным $1{,}5\,\text{атм}$.
    Определите работу газа, ответ выразите в килоджоулях.
    $p_{\text{aтм}} = 100\,\text{кПа}$.
}
\answer{%
    $A = P\Delta V = P(V_2 - V_1) = 1{,}5\,\text{атм} \cdot\cbr{650\,\text{л} - 200\,\text{л}} = 6{,}75\,\text{кДж}.$
}
\solutionspace{40pt}

\tasknumber{4}%
\task{%
    $40\,\text{моль}$ идеального одноатомного в результате адиабатического процесса остыл на $120\,\text{К}$.
    Определите работу газа.
    Кто совершил положительную работу: газ или внешние силы?
    Универсальная газовая постоянная $R = 8{,}31\,\frac{\text{Дж}}{\text{моль}\cdot\text{К}}$.
}
\answer{%
    \begin{align*}
    Q &= 0, Q = \Delta U + A_\text{газа} \implies \\
    \implies A_\text{газа} &= - \Delta U = - \frac 32 \nu R \Delta T =  \frac 32 \cdot 40\,\text{моль} \cdot8{,}31\,\frac{\text{Дж}}{\text{моль}\cdot\text{К}} \cdot120\,\text{К}= 59{,}8\,\text{кДж}, \text{газ.}
    \end{align*}
}
\solutionspace{40pt}

\tasknumber{5}%
\task{%
    Как изменилась внутренняя энергия одноатомного идеального газа при переходе из состояния 1 в состояние 2?
    $P_1 = 2\,\text{МПа}$, $V_1 = 5\,\text{л}$, $P_2 = 4{,}5\,\text{МПа}$, $V_2 = 6\,\text{л}$.
    Как изменилась при этом температура газа?
}
\answer{%
    \begin{align*}
    P_1V_1 &= \nu R T_1, P_2V_2 = \nu R T_2, \\
    \Delta U &= U_2-U_1 = \frac 32 \nu R T_2- \frac 32 \nu R T_1 = \frac 32 P_2 V_2 - \frac 32 P_1 V_1= \frac 32 \cdot \cbr{4{,}5\,\text{МПа} \cdot6\,\text{л} - 2\,\text{МПа} \cdot5\,\text{л}} = 25500\,\text{Дж}.
    \\
    \frac{T_2}{T_1} &= \frac{\frac{P_2V_2}{\nu R}}{\frac{P_1V_1}{\nu R}} = \frac{P_2V_2}{P_1V_1}= \frac{4{,}5\,\text{МПа} \cdot6\,\text{л}}{2\,\text{МПа} \cdot5\,\text{л}} \approx 2{,}70.
    \end{align*}
}
\solutionspace{80pt}

\tasknumber{6}%
\task{%
    $3\,\text{моль}$ идеального одноатомного газа охладили на $30\,\text{К}$.
    Определите изменение внутренней энергии газа.
    Увеличилась она или уменьшилась?
    Универсальная газовая постоянная $R = 8{,}31\,\frac{\text{Дж}}{\text{моль}\cdot\text{К}}$.
}
\answer{%
    $
        \Delta U = \frac 32 \nu R \Delta T
            = - \frac 32 \cdot 3\,\text{моль} \cdot8{,}31\,\frac{\text{Дж}}{\text{моль}\cdot\text{К}} \cdot30\,\text{К}
            = -1121\,\text{Дж}.
            \text{Уменьшилась.}
    $
}
\solutionspace{40pt}

\tasknumber{7}%
\task{%
    Газу сообщили некоторое количество теплоты,
    при этом четверть его он потратил на совершение работы,
    одновременно увеличив свою внутреннюю энергию на $1500\,\text{Дж}$.
    Определите количество теплоты, сообщённое газу.
}
\answer{%
    \begin{align*}
    Q &= A' + \Delta U, A' = \frac 14 Q \implies Q\cdot\cbr{1 - \frac 14} = \Delta U \implies Q = \frac{\Delta U}{1 - \frac 14} = \frac{1500\,\text{Дж}}{1 - \frac 14} \approx 2000\,\text{Дж}.
    \\
    A' &= \frac 14 Q
        = \frac 14 \cdot \frac{\Delta U}{1 - \frac 14}
        = \frac{\Delta U}{4 - 1}
        = \frac{1500\,\text{Дж}}{4 - 1} \approx 500\,\text{Дж}.
    \end{align*}
}
\solutionspace{60pt}

\tasknumber{8}%
\task{%
    В некотором процессе внешние силы совершили над газом работу $100\,\text{Дж}$,
    при этом его внутренняя энергия увеличилась на $150\,\text{Дж}$.
    Определите количество тепла, переданное при этом процессе газу.
    Явно пропишите, подводили газу тепло или же отводили.
}
\answer{%
    $
        Q = A_\text{газа} + \Delta U, A_\text{газа} = -A_\text{внешняя}
        \implies Q = A_\text{газа} + \Delta U = - 100\,\text{Дж} +  150\,\text{Дж} = 50\,\text{Дж}.
        \text{ Подводили.}
    $
}

\variantsplitter

\addpersonalvariant{Ярослав Лавровский}

\tasknumber{1}%
\task{%
    Напротив физической величины укажите её обозначение и единицы измерения в СИ или запишите физический закон или формулу (в пункте «г)»):
    \begin{enumerate}
        \item количество теплоты,
        \item работа внешних сил,
        \item молярная теплоёмкость,
        \item внутренняя энергия идеального одноатомного газа.
    \end{enumerate}
}
\solutionspace{20pt}

\tasknumber{2}%
\task{%
    Определите объём идеального одноатомного газа,
    если его внутренняя энергия при давлении $2\,\text{атм}$ составляет $250\,\text{кДж}$.
    $p_{\text{aтм}} = 100\,\text{кПа}$.
}
\answer{%
    $U = \frac 32 \nu R T = \frac 32 PV \implies V = \frac 23 \cdot \frac UP= \frac 23 \cdot \frac{ 250\,\text{кДж} }{ 2\,\text{атм} } \approx 0{,}83\,\text{м}^{3}.$
}
\solutionspace{40pt}

\tasknumber{3}%
\task{%
    Газ расширился от $150\,\text{л}$ до $550\,\text{л}$.
    Давление газа при этом оставалось постоянным и равным $1{,}5\,\text{атм}$.
    Определите работу газа, ответ выразите в килоджоулях.
    $p_{\text{aтм}} = 100\,\text{кПа}$.
}
\answer{%
    $A = P\Delta V = P(V_2 - V_1) = 1{,}5\,\text{атм} \cdot\cbr{550\,\text{л} - 150\,\text{л}} = 6{,}00\,\text{кДж}.$
}
\solutionspace{40pt}

\tasknumber{4}%
\task{%
    $40\,\text{моль}$ идеального одноатомного в результате адиабатического процесса нагрелся на $25\,\text{К}$.
    Определите работу газа.
    Кто совершил положительную работу: газ или внешние силы?
    Универсальная газовая постоянная $R = 8{,}31\,\frac{\text{Дж}}{\text{моль}\cdot\text{К}}$.
}
\answer{%
    \begin{align*}
    Q &= 0, Q = \Delta U + A_\text{газа} \implies \\
    \implies A_\text{газа} &= - \Delta U = - \frac 32 \nu R \Delta T = - \frac 32 \cdot 40\,\text{моль} \cdot8{,}31\,\frac{\text{Дж}}{\text{моль}\cdot\text{К}} \cdot25\,\text{К}= -12{,}500\,\text{кДж}, \text{внешние силы.}
    \end{align*}
}
\solutionspace{40pt}

\tasknumber{5}%
\task{%
    Как изменилась внутренняя энергия одноатомного идеального газа при переходе из состояния 1 в состояние 2?
    $P_1 = 2\,\text{МПа}$, $V_1 = 3\,\text{л}$, $P_2 = 4{,}5\,\text{МПа}$, $V_2 = 4\,\text{л}$.
    Как изменилась при этом температура газа?
}
\answer{%
    \begin{align*}
    P_1V_1 &= \nu R T_1, P_2V_2 = \nu R T_2, \\
    \Delta U &= U_2-U_1 = \frac 32 \nu R T_2- \frac 32 \nu R T_1 = \frac 32 P_2 V_2 - \frac 32 P_1 V_1= \frac 32 \cdot \cbr{4{,}5\,\text{МПа} \cdot4\,\text{л} - 2\,\text{МПа} \cdot3\,\text{л}} = 18000\,\text{Дж}.
    \\
    \frac{T_2}{T_1} &= \frac{\frac{P_2V_2}{\nu R}}{\frac{P_1V_1}{\nu R}} = \frac{P_2V_2}{P_1V_1}= \frac{4{,}5\,\text{МПа} \cdot4\,\text{л}}{2\,\text{МПа} \cdot3\,\text{л}} \approx 3{,}00.
    \end{align*}
}
\solutionspace{80pt}

\tasknumber{6}%
\task{%
    $3\,\text{моль}$ идеального одноатомного газа нагрели на $20\,\text{К}$.
    Определите изменение внутренней энергии газа.
    Увеличилась она или уменьшилась?
    Универсальная газовая постоянная $R = 8{,}31\,\frac{\text{Дж}}{\text{моль}\cdot\text{К}}$.
}
\answer{%
    $
        \Delta U = \frac 32 \nu R \Delta T
            =  \frac 32 \cdot 3\,\text{моль} \cdot8{,}31\,\frac{\text{Дж}}{\text{моль}\cdot\text{К}} \cdot20\,\text{К}
            = 747\,\text{Дж}.
            \text{Увеличилась.}
    $
}
\solutionspace{40pt}

\tasknumber{7}%
\task{%
    Газу сообщили некоторое количество теплоты,
    при этом половину его он потратил на совершение работы,
    одновременно увеличив свою внутреннюю энергию на $1200\,\text{Дж}$.
    Определите работу, совершённую газом.
}
\answer{%
    \begin{align*}
    Q &= A' + \Delta U, A' = \frac 12 Q \implies Q\cdot\cbr{1 - \frac 12} = \Delta U \implies Q = \frac{\Delta U}{1 - \frac 12} = \frac{1200\,\text{Дж}}{1 - \frac 12} \approx 2400\,\text{Дж}.
    \\
    A' &= \frac 12 Q
        = \frac 12 \cdot \frac{\Delta U}{1 - \frac 12}
        = \frac{\Delta U}{2 - 1}
        = \frac{1200\,\text{Дж}}{2 - 1} \approx 1200\,\text{Дж}.
    \end{align*}
}
\solutionspace{60pt}

\tasknumber{8}%
\task{%
    В некотором процессе газ совершил работу $200\,\text{Дж}$,
    при этом его внутренняя энергия уменьшилась на $350\,\text{Дж}$.
    Определите количество тепла, переданное при этом процессе газу.
    Явно пропишите, подводили газу тепло или же отводили.
}
\answer{%
    $
        Q = A_\text{газа} + \Delta U, A_\text{газа} = -A_\text{внешняя}
        \implies Q = A_\text{газа} + \Delta U =  200\,\text{Дж} - 350\,\text{Дж} = -150\,\text{Дж}.
        \text{ Отводили.}
    $
}

\variantsplitter

\addpersonalvariant{Анастасия Ламанова}

\tasknumber{1}%
\task{%
    Напротив физической величины укажите её обозначение и единицы измерения в СИ или запишите физический закон или формулу (в пункте «г)»):
    \begin{enumerate}
        \item количество теплоты,
        \item работа газа,
        \item молярная теплоёмкость,
        \item первое начало термодинамики.
    \end{enumerate}
}
\solutionspace{20pt}

\tasknumber{2}%
\task{%
    Определите объём идеального одноатомного газа,
    если его внутренняя энергия при давлении $2\,\text{атм}$ составляет $400\,\text{кДж}$.
    $p_{\text{aтм}} = 100\,\text{кПа}$.
}
\answer{%
    $U = \frac 32 \nu R T = \frac 32 PV \implies V = \frac 23 \cdot \frac UP= \frac 23 \cdot \frac{ 400\,\text{кДж} }{ 2\,\text{атм} } \approx 1{,}33\,\text{м}^{3}.$
}
\solutionspace{40pt}

\tasknumber{3}%
\task{%
    Газ расширился от $250\,\text{л}$ до $650\,\text{л}$.
    Давление газа при этом оставалось постоянным и равным $1{,}5\,\text{атм}$.
    Определите работу газа, ответ выразите в килоджоулях.
    $p_{\text{aтм}} = 100\,\text{кПа}$.
}
\answer{%
    $A = P\Delta V = P(V_2 - V_1) = 1{,}5\,\text{атм} \cdot\cbr{650\,\text{л} - 250\,\text{л}} = 6{,}00\,\text{кДж}.$
}
\solutionspace{40pt}

\tasknumber{4}%
\task{%
    $40\,\text{моль}$ идеального одноатомного в результате адиабатического процесса нагрелся на $80\,\text{К}$.
    Определите работу газа.
    Кто совершил положительную работу: газ или внешние силы?
    Универсальная газовая постоянная $R = 8{,}31\,\frac{\text{Дж}}{\text{моль}\cdot\text{К}}$.
}
\answer{%
    \begin{align*}
    Q &= 0, Q = \Delta U + A_\text{газа} \implies \\
    \implies A_\text{газа} &= - \Delta U = - \frac 32 \nu R \Delta T = - \frac 32 \cdot 40\,\text{моль} \cdot8{,}31\,\frac{\text{Дж}}{\text{моль}\cdot\text{К}} \cdot80\,\text{К}= -39{,}90\,\text{кДж}, \text{внешние силы.}
    \end{align*}
}
\solutionspace{40pt}

\tasknumber{5}%
\task{%
    Как изменилась внутренняя энергия одноатомного идеального газа при переходе из состояния 1 в состояние 2?
    $P_1 = 2\,\text{МПа}$, $V_1 = 5\,\text{л}$, $P_2 = 1{,}5\,\text{МПа}$, $V_2 = 2\,\text{л}$.
    Как изменилась при этом температура газа?
}
\answer{%
    \begin{align*}
    P_1V_1 &= \nu R T_1, P_2V_2 = \nu R T_2, \\
    \Delta U &= U_2-U_1 = \frac 32 \nu R T_2- \frac 32 \nu R T_1 = \frac 32 P_2 V_2 - \frac 32 P_1 V_1= \frac 32 \cdot \cbr{1{,}5\,\text{МПа} \cdot2\,\text{л} - 2\,\text{МПа} \cdot5\,\text{л}} = -10500\,\text{Дж}.
    \\
    \frac{T_2}{T_1} &= \frac{\frac{P_2V_2}{\nu R}}{\frac{P_1V_1}{\nu R}} = \frac{P_2V_2}{P_1V_1}= \frac{1{,}5\,\text{МПа} \cdot2\,\text{л}}{2\,\text{МПа} \cdot5\,\text{л}} \approx 0{,}30.
    \end{align*}
}
\solutionspace{80pt}

\tasknumber{6}%
\task{%
    $3\,\text{моль}$ идеального одноатомного газа нагрели на $20\,\text{К}$.
    Определите изменение внутренней энергии газа.
    Увеличилась она или уменьшилась?
    Универсальная газовая постоянная $R = 8{,}31\,\frac{\text{Дж}}{\text{моль}\cdot\text{К}}$.
}
\answer{%
    $
        \Delta U = \frac 32 \nu R \Delta T
            =  \frac 32 \cdot 3\,\text{моль} \cdot8{,}31\,\frac{\text{Дж}}{\text{моль}\cdot\text{К}} \cdot20\,\text{К}
            = 747\,\text{Дж}.
            \text{Увеличилась.}
    $
}
\solutionspace{40pt}

\tasknumber{7}%
\task{%
    Газу сообщили некоторое количество теплоты,
    при этом треть его он потратил на совершение работы,
    одновременно увеличив свою внутреннюю энергию на $2400\,\text{Дж}$.
    Определите количество теплоты, сообщённое газу.
}
\answer{%
    \begin{align*}
    Q &= A' + \Delta U, A' = \frac 13 Q \implies Q\cdot\cbr{1 - \frac 13} = \Delta U \implies Q = \frac{\Delta U}{1 - \frac 13} = \frac{2400\,\text{Дж}}{1 - \frac 13} \approx 3600\,\text{Дж}.
    \\
    A' &= \frac 13 Q
        = \frac 13 \cdot \frac{\Delta U}{1 - \frac 13}
        = \frac{\Delta U}{3 - 1}
        = \frac{2400\,\text{Дж}}{3 - 1} \approx 1200\,\text{Дж}.
    \end{align*}
}
\solutionspace{60pt}

\tasknumber{8}%
\task{%
    В некотором процессе внешние силы совершили над газом работу $100\,\text{Дж}$,
    при этом его внутренняя энергия уменьшилась на $350\,\text{Дж}$.
    Определите количество тепла, переданное при этом процессе газу.
    Явно пропишите, подводили газу тепло или же отводили.
}
\answer{%
    $
        Q = A_\text{газа} + \Delta U, A_\text{газа} = -A_\text{внешняя}
        \implies Q = A_\text{газа} + \Delta U = - 100\,\text{Дж} - 350\,\text{Дж} = -450\,\text{Дж}.
        \text{ Отводили.}
    $
}

\variantsplitter

\addpersonalvariant{Виктория Легонькова}

\tasknumber{1}%
\task{%
    Напротив физической величины укажите её обозначение и единицы измерения в СИ или запишите физический закон или формулу (в пункте «г)»):
    \begin{enumerate}
        \item изменение внутренней энергии,
        \item работа газа,
        \item удельная теплоёмкость,
        \item внутренняя энергия идеального одноатомного газа.
    \end{enumerate}
}
\solutionspace{20pt}

\tasknumber{2}%
\task{%
    Определите объём идеального одноатомного газа,
    если его внутренняя энергия при давлении $3\,\text{атм}$ составляет $300\,\text{кДж}$.
    $p_{\text{aтм}} = 100\,\text{кПа}$.
}
\answer{%
    $U = \frac 32 \nu R T = \frac 32 PV \implies V = \frac 23 \cdot \frac UP= \frac 23 \cdot \frac{ 300\,\text{кДж} }{ 3\,\text{атм} } \approx 0{,}67\,\text{м}^{3}.$
}
\solutionspace{40pt}

\tasknumber{3}%
\task{%
    Газ расширился от $200\,\text{л}$ до $450\,\text{л}$.
    Давление газа при этом оставалось постоянным и равным $1{,}5\,\text{атм}$.
    Определите работу газа, ответ выразите в килоджоулях.
    $p_{\text{aтм}} = 100\,\text{кПа}$.
}
\answer{%
    $A = P\Delta V = P(V_2 - V_1) = 1{,}5\,\text{атм} \cdot\cbr{450\,\text{л} - 200\,\text{л}} = 3{,}75\,\text{кДж}.$
}
\solutionspace{40pt}

\tasknumber{4}%
\task{%
    $40\,\text{моль}$ идеального одноатомного в результате адиабатического процесса нагрелся на $45\,\text{К}$.
    Определите работу газа.
    Кто совершил положительную работу: газ или внешние силы?
    Универсальная газовая постоянная $R = 8{,}31\,\frac{\text{Дж}}{\text{моль}\cdot\text{К}}$.
}
\answer{%
    \begin{align*}
    Q &= 0, Q = \Delta U + A_\text{газа} \implies \\
    \implies A_\text{газа} &= - \Delta U = - \frac 32 \nu R \Delta T = - \frac 32 \cdot 40\,\text{моль} \cdot8{,}31\,\frac{\text{Дж}}{\text{моль}\cdot\text{К}} \cdot45\,\text{К}= -22{,}40\,\text{кДж}, \text{внешние силы.}
    \end{align*}
}
\solutionspace{40pt}

\tasknumber{5}%
\task{%
    Как изменилась внутренняя энергия одноатомного идеального газа при переходе из состояния 1 в состояние 2?
    $P_1 = 4\,\text{МПа}$, $V_1 = 3\,\text{л}$, $P_2 = 4{,}5\,\text{МПа}$, $V_2 = 8\,\text{л}$.
    Как изменилась при этом температура газа?
}
\answer{%
    \begin{align*}
    P_1V_1 &= \nu R T_1, P_2V_2 = \nu R T_2, \\
    \Delta U &= U_2-U_1 = \frac 32 \nu R T_2- \frac 32 \nu R T_1 = \frac 32 P_2 V_2 - \frac 32 P_1 V_1= \frac 32 \cdot \cbr{4{,}5\,\text{МПа} \cdot8\,\text{л} - 4\,\text{МПа} \cdot3\,\text{л}} = 36000\,\text{Дж}.
    \\
    \frac{T_2}{T_1} &= \frac{\frac{P_2V_2}{\nu R}}{\frac{P_1V_1}{\nu R}} = \frac{P_2V_2}{P_1V_1}= \frac{4{,}5\,\text{МПа} \cdot8\,\text{л}}{4\,\text{МПа} \cdot3\,\text{л}} \approx 3{,}00.
    \end{align*}
}
\solutionspace{80pt}

\tasknumber{6}%
\task{%
    $3\,\text{моль}$ идеального одноатомного газа нагрели на $30\,\text{К}$.
    Определите изменение внутренней энергии газа.
    Увеличилась она или уменьшилась?
    Универсальная газовая постоянная $R = 8{,}31\,\frac{\text{Дж}}{\text{моль}\cdot\text{К}}$.
}
\answer{%
    $
        \Delta U = \frac 32 \nu R \Delta T
            =  \frac 32 \cdot 3\,\text{моль} \cdot8{,}31\,\frac{\text{Дж}}{\text{моль}\cdot\text{К}} \cdot30\,\text{К}
            = 1121\,\text{Дж}.
            \text{Увеличилась.}
    $
}
\solutionspace{40pt}

\tasknumber{7}%
\task{%
    Газу сообщили некоторое количество теплоты,
    при этом половину его он потратил на совершение работы,
    одновременно увеличив свою внутреннюю энергию на $3000\,\text{Дж}$.
    Определите работу, совершённую газом.
}
\answer{%
    \begin{align*}
    Q &= A' + \Delta U, A' = \frac 12 Q \implies Q\cdot\cbr{1 - \frac 12} = \Delta U \implies Q = \frac{\Delta U}{1 - \frac 12} = \frac{3000\,\text{Дж}}{1 - \frac 12} \approx 6000\,\text{Дж}.
    \\
    A' &= \frac 12 Q
        = \frac 12 \cdot \frac{\Delta U}{1 - \frac 12}
        = \frac{\Delta U}{2 - 1}
        = \frac{3000\,\text{Дж}}{2 - 1} \approx 3000\,\text{Дж}.
    \end{align*}
}
\solutionspace{60pt}

\tasknumber{8}%
\task{%
    В некотором процессе газ совершил работу $200\,\text{Дж}$,
    при этом его внутренняя энергия увеличилась на $350\,\text{Дж}$.
    Определите количество тепла, переданное при этом процессе газу.
    Явно пропишите, подводили газу тепло или же отводили.
}
\answer{%
    $
        Q = A_\text{газа} + \Delta U, A_\text{газа} = -A_\text{внешняя}
        \implies Q = A_\text{газа} + \Delta U =  200\,\text{Дж} +  350\,\text{Дж} = 550\,\text{Дж}.
        \text{ Подводили.}
    $
}

\variantsplitter

\addpersonalvariant{Семён Мартынов}

\tasknumber{1}%
\task{%
    Напротив физической величины укажите её обозначение и единицы измерения в СИ или запишите физический закон или формулу (в пункте «г)»):
    \begin{enumerate}
        \item изменение внутренней энергии,
        \item работа внешних сил,
        \item молярная теплоёмкость,
        \item первое начало термодинамики.
    \end{enumerate}
}
\solutionspace{20pt}

\tasknumber{2}%
\task{%
    Определите объём идеального одноатомного газа,
    если его внутренняя энергия при давлении $5\,\text{атм}$ составляет $300\,\text{кДж}$.
    $p_{\text{aтм}} = 100\,\text{кПа}$.
}
\answer{%
    $U = \frac 32 \nu R T = \frac 32 PV \implies V = \frac 23 \cdot \frac UP= \frac 23 \cdot \frac{ 300\,\text{кДж} }{ 5\,\text{атм} } \approx 0{,}40\,\text{м}^{3}.$
}
\solutionspace{40pt}

\tasknumber{3}%
\task{%
    Газ расширился от $200\,\text{л}$ до $650\,\text{л}$.
    Давление газа при этом оставалось постоянным и равным $1{,}2\,\text{атм}$.
    Определите работу газа, ответ выразите в килоджоулях.
    $p_{\text{aтм}} = 100\,\text{кПа}$.
}
\answer{%
    $A = P\Delta V = P(V_2 - V_1) = 1{,}2\,\text{атм} \cdot\cbr{650\,\text{л} - 200\,\text{л}} = 5{,}40\,\text{кДж}.$
}
\solutionspace{40pt}

\tasknumber{4}%
\task{%
    $50\,\text{моль}$ идеального одноатомного в результате адиабатического процесса нагрелся на $60\,\text{К}$.
    Определите работу газа.
    Кто совершил положительную работу: газ или внешние силы?
    Универсальная газовая постоянная $R = 8{,}31\,\frac{\text{Дж}}{\text{моль}\cdot\text{К}}$.
}
\answer{%
    \begin{align*}
    Q &= 0, Q = \Delta U + A_\text{газа} \implies \\
    \implies A_\text{газа} &= - \Delta U = - \frac 32 \nu R \Delta T = - \frac 32 \cdot 50\,\text{моль} \cdot8{,}31\,\frac{\text{Дж}}{\text{моль}\cdot\text{К}} \cdot60\,\text{К}= -37{,}40\,\text{кДж}, \text{внешние силы.}
    \end{align*}
}
\solutionspace{40pt}

\tasknumber{5}%
\task{%
    Как изменилась внутренняя энергия одноатомного идеального газа при переходе из состояния 1 в состояние 2?
    $P_1 = 2\,\text{МПа}$, $V_1 = 7\,\text{л}$, $P_2 = 4{,}5\,\text{МПа}$, $V_2 = 6\,\text{л}$.
    Как изменилась при этом температура газа?
}
\answer{%
    \begin{align*}
    P_1V_1 &= \nu R T_1, P_2V_2 = \nu R T_2, \\
    \Delta U &= U_2-U_1 = \frac 32 \nu R T_2- \frac 32 \nu R T_1 = \frac 32 P_2 V_2 - \frac 32 P_1 V_1= \frac 32 \cdot \cbr{4{,}5\,\text{МПа} \cdot6\,\text{л} - 2\,\text{МПа} \cdot7\,\text{л}} = 19500\,\text{Дж}.
    \\
    \frac{T_2}{T_1} &= \frac{\frac{P_2V_2}{\nu R}}{\frac{P_1V_1}{\nu R}} = \frac{P_2V_2}{P_1V_1}= \frac{4{,}5\,\text{МПа} \cdot6\,\text{л}}{2\,\text{МПа} \cdot7\,\text{л}} \approx 1{,}93.
    \end{align*}
}
\solutionspace{80pt}

\tasknumber{6}%
\task{%
    $4\,\text{моль}$ идеального одноатомного газа нагрели на $10\,\text{К}$.
    Определите изменение внутренней энергии газа.
    Увеличилась она или уменьшилась?
    Универсальная газовая постоянная $R = 8{,}31\,\frac{\text{Дж}}{\text{моль}\cdot\text{К}}$.
}
\answer{%
    $
        \Delta U = \frac 32 \nu R \Delta T
            =  \frac 32 \cdot 4\,\text{моль} \cdot8{,}31\,\frac{\text{Дж}}{\text{моль}\cdot\text{К}} \cdot10\,\text{К}
            = 498\,\text{Дж}.
            \text{Увеличилась.}
    $
}
\solutionspace{40pt}

\tasknumber{7}%
\task{%
    Газу сообщили некоторое количество теплоты,
    при этом четверть его он потратил на совершение работы,
    одновременно увеличив свою внутреннюю энергию на $1200\,\text{Дж}$.
    Определите работу, совершённую газом.
}
\answer{%
    \begin{align*}
    Q &= A' + \Delta U, A' = \frac 14 Q \implies Q\cdot\cbr{1 - \frac 14} = \Delta U \implies Q = \frac{\Delta U}{1 - \frac 14} = \frac{1200\,\text{Дж}}{1 - \frac 14} \approx 1600\,\text{Дж}.
    \\
    A' &= \frac 14 Q
        = \frac 14 \cdot \frac{\Delta U}{1 - \frac 14}
        = \frac{\Delta U}{4 - 1}
        = \frac{1200\,\text{Дж}}{4 - 1} \approx 400\,\text{Дж}.
    \end{align*}
}
\solutionspace{60pt}

\tasknumber{8}%
\task{%
    В некотором процессе внешние силы совершили над газом работу $200\,\text{Дж}$,
    при этом его внутренняя энергия уменьшилась на $250\,\text{Дж}$.
    Определите количество тепла, переданное при этом процессе газу.
    Явно пропишите, подводили газу тепло или же отводили.
}
\answer{%
    $
        Q = A_\text{газа} + \Delta U, A_\text{газа} = -A_\text{внешняя}
        \implies Q = A_\text{газа} + \Delta U = - 200\,\text{Дж} - 250\,\text{Дж} = -450\,\text{Дж}.
        \text{ Отводили.}
    $
}

\variantsplitter

\addpersonalvariant{Варвара Минаева}

\tasknumber{1}%
\task{%
    Напротив физической величины укажите её обозначение и единицы измерения в СИ или запишите физический закон или формулу (в пункте «г)»):
    \begin{enumerate}
        \item количество теплоты,
        \item работа внешних сил,
        \item молярная теплоёмкость,
        \item первое начало термодинамики.
    \end{enumerate}
}
\solutionspace{20pt}

\tasknumber{2}%
\task{%
    Определите объём идеального одноатомного газа,
    если его внутренняя энергия при давлении $3\,\text{атм}$ составляет $500\,\text{кДж}$.
    $p_{\text{aтм}} = 100\,\text{кПа}$.
}
\answer{%
    $U = \frac 32 \nu R T = \frac 32 PV \implies V = \frac 23 \cdot \frac UP= \frac 23 \cdot \frac{ 500\,\text{кДж} }{ 3\,\text{атм} } \approx 1{,}11\,\text{м}^{3}.$
}
\solutionspace{40pt}

\tasknumber{3}%
\task{%
    Газ расширился от $200\,\text{л}$ до $550\,\text{л}$.
    Давление газа при этом оставалось постоянным и равным $2{,}5\,\text{атм}$.
    Определите работу газа, ответ выразите в килоджоулях.
    $p_{\text{aтм}} = 100\,\text{кПа}$.
}
\answer{%
    $A = P\Delta V = P(V_2 - V_1) = 2{,}5\,\text{атм} \cdot\cbr{550\,\text{л} - 200\,\text{л}} = 8{,}75\,\text{кДж}.$
}
\solutionspace{40pt}

\tasknumber{4}%
\task{%
    $50\,\text{моль}$ идеального одноатомного в результате адиабатического процесса остыл на $80\,\text{К}$.
    Определите работу газа.
    Кто совершил положительную работу: газ или внешние силы?
    Универсальная газовая постоянная $R = 8{,}31\,\frac{\text{Дж}}{\text{моль}\cdot\text{К}}$.
}
\answer{%
    \begin{align*}
    Q &= 0, Q = \Delta U + A_\text{газа} \implies \\
    \implies A_\text{газа} &= - \Delta U = - \frac 32 \nu R \Delta T =  \frac 32 \cdot 50\,\text{моль} \cdot8{,}31\,\frac{\text{Дж}}{\text{моль}\cdot\text{К}} \cdot80\,\text{К}= 49{,}9\,\text{кДж}, \text{газ.}
    \end{align*}
}
\solutionspace{40pt}

\tasknumber{5}%
\task{%
    Как изменилась внутренняя энергия одноатомного идеального газа при переходе из состояния 1 в состояние 2?
    $P_1 = 4\,\text{МПа}$, $V_1 = 7\,\text{л}$, $P_2 = 3{,}5\,\text{МПа}$, $V_2 = 4\,\text{л}$.
    Как изменилась при этом температура газа?
}
\answer{%
    \begin{align*}
    P_1V_1 &= \nu R T_1, P_2V_2 = \nu R T_2, \\
    \Delta U &= U_2-U_1 = \frac 32 \nu R T_2- \frac 32 \nu R T_1 = \frac 32 P_2 V_2 - \frac 32 P_1 V_1= \frac 32 \cdot \cbr{3{,}5\,\text{МПа} \cdot4\,\text{л} - 4\,\text{МПа} \cdot7\,\text{л}} = -21000\,\text{Дж}.
    \\
    \frac{T_2}{T_1} &= \frac{\frac{P_2V_2}{\nu R}}{\frac{P_1V_1}{\nu R}} = \frac{P_2V_2}{P_1V_1}= \frac{3{,}5\,\text{МПа} \cdot4\,\text{л}}{4\,\text{МПа} \cdot7\,\text{л}} \approx 0{,}50.
    \end{align*}
}
\solutionspace{80pt}

\tasknumber{6}%
\task{%
    $4\,\text{моль}$ идеального одноатомного газа охладили на $20\,\text{К}$.
    Определите изменение внутренней энергии газа.
    Увеличилась она или уменьшилась?
    Универсальная газовая постоянная $R = 8{,}31\,\frac{\text{Дж}}{\text{моль}\cdot\text{К}}$.
}
\answer{%
    $
        \Delta U = \frac 32 \nu R \Delta T
            = - \frac 32 \cdot 4\,\text{моль} \cdot8{,}31\,\frac{\text{Дж}}{\text{моль}\cdot\text{К}} \cdot20\,\text{К}
            = -997\,\text{Дж}.
            \text{Уменьшилась.}
    $
}
\solutionspace{40pt}

\tasknumber{7}%
\task{%
    Газу сообщили некоторое количество теплоты,
    при этом половину его он потратил на совершение работы,
    одновременно увеличив свою внутреннюю энергию на $2400\,\text{Дж}$.
    Определите количество теплоты, сообщённое газу.
}
\answer{%
    \begin{align*}
    Q &= A' + \Delta U, A' = \frac 12 Q \implies Q\cdot\cbr{1 - \frac 12} = \Delta U \implies Q = \frac{\Delta U}{1 - \frac 12} = \frac{2400\,\text{Дж}}{1 - \frac 12} \approx 4800\,\text{Дж}.
    \\
    A' &= \frac 12 Q
        = \frac 12 \cdot \frac{\Delta U}{1 - \frac 12}
        = \frac{\Delta U}{2 - 1}
        = \frac{2400\,\text{Дж}}{2 - 1} \approx 2400\,\text{Дж}.
    \end{align*}
}
\solutionspace{60pt}

\tasknumber{8}%
\task{%
    В некотором процессе внешние силы совершили над газом работу $200\,\text{Дж}$,
    при этом его внутренняя энергия уменьшилась на $150\,\text{Дж}$.
    Определите количество тепла, переданное при этом процессе газу.
    Явно пропишите, подводили газу тепло или же отводили.
}
\answer{%
    $
        Q = A_\text{газа} + \Delta U, A_\text{газа} = -A_\text{внешняя}
        \implies Q = A_\text{газа} + \Delta U = - 200\,\text{Дж} - 150\,\text{Дж} = -350\,\text{Дж}.
        \text{ Отводили.}
    $
}

\variantsplitter

\addpersonalvariant{Леонид Никитин}

\tasknumber{1}%
\task{%
    Напротив физической величины укажите её обозначение и единицы измерения в СИ или запишите физический закон или формулу (в пункте «г)»):
    \begin{enumerate}
        \item количество теплоты,
        \item работа внешних сил,
        \item удельная теплоёмкость,
        \item внутренняя энергия идеального одноатомного газа.
    \end{enumerate}
}
\solutionspace{20pt}

\tasknumber{2}%
\task{%
    Определите объём идеального одноатомного газа,
    если его внутренняя энергия при давлении $3\,\text{атм}$ составляет $250\,\text{кДж}$.
    $p_{\text{aтм}} = 100\,\text{кПа}$.
}
\answer{%
    $U = \frac 32 \nu R T = \frac 32 PV \implies V = \frac 23 \cdot \frac UP= \frac 23 \cdot \frac{ 250\,\text{кДж} }{ 3\,\text{атм} } \approx 0{,}56\,\text{м}^{3}.$
}
\solutionspace{40pt}

\tasknumber{3}%
\task{%
    Газ расширился от $350\,\text{л}$ до $550\,\text{л}$.
    Давление газа при этом оставалось постоянным и равным $3{,}5\,\text{атм}$.
    Определите работу газа, ответ выразите в килоджоулях.
    $p_{\text{aтм}} = 100\,\text{кПа}$.
}
\answer{%
    $A = P\Delta V = P(V_2 - V_1) = 3{,}5\,\text{атм} \cdot\cbr{550\,\text{л} - 350\,\text{л}} = 7{,}00\,\text{кДж}.$
}
\solutionspace{40pt}

\tasknumber{4}%
\task{%
    $60\,\text{моль}$ идеального одноатомного в результате адиабатического процесса остыл на $120\,\text{К}$.
    Определите работу газа.
    Кто совершил положительную работу: газ или внешние силы?
    Универсальная газовая постоянная $R = 8{,}31\,\frac{\text{Дж}}{\text{моль}\cdot\text{К}}$.
}
\answer{%
    \begin{align*}
    Q &= 0, Q = \Delta U + A_\text{газа} \implies \\
    \implies A_\text{газа} &= - \Delta U = - \frac 32 \nu R \Delta T =  \frac 32 \cdot 60\,\text{моль} \cdot8{,}31\,\frac{\text{Дж}}{\text{моль}\cdot\text{К}} \cdot120\,\text{К}= 89{,}7\,\text{кДж}, \text{газ.}
    \end{align*}
}
\solutionspace{40pt}

\tasknumber{5}%
\task{%
    Как изменилась внутренняя энергия одноатомного идеального газа при переходе из состояния 1 в состояние 2?
    $P_1 = 4\,\text{МПа}$, $V_1 = 3\,\text{л}$, $P_2 = 4{,}5\,\text{МПа}$, $V_2 = 2\,\text{л}$.
    Как изменилась при этом температура газа?
}
\answer{%
    \begin{align*}
    P_1V_1 &= \nu R T_1, P_2V_2 = \nu R T_2, \\
    \Delta U &= U_2-U_1 = \frac 32 \nu R T_2- \frac 32 \nu R T_1 = \frac 32 P_2 V_2 - \frac 32 P_1 V_1= \frac 32 \cdot \cbr{4{,}5\,\text{МПа} \cdot2\,\text{л} - 4\,\text{МПа} \cdot3\,\text{л}} = -4500\,\text{Дж}.
    \\
    \frac{T_2}{T_1} &= \frac{\frac{P_2V_2}{\nu R}}{\frac{P_1V_1}{\nu R}} = \frac{P_2V_2}{P_1V_1}= \frac{4{,}5\,\text{МПа} \cdot2\,\text{л}}{4\,\text{МПа} \cdot3\,\text{л}} \approx 0{,}75.
    \end{align*}
}
\solutionspace{80pt}

\tasknumber{6}%
\task{%
    $5\,\text{моль}$ идеального одноатомного газа охладили на $30\,\text{К}$.
    Определите изменение внутренней энергии газа.
    Увеличилась она или уменьшилась?
    Универсальная газовая постоянная $R = 8{,}31\,\frac{\text{Дж}}{\text{моль}\cdot\text{К}}$.
}
\answer{%
    $
        \Delta U = \frac 32 \nu R \Delta T
            = - \frac 32 \cdot 5\,\text{моль} \cdot8{,}31\,\frac{\text{Дж}}{\text{моль}\cdot\text{К}} \cdot30\,\text{К}
            = -1869\,\text{Дж}.
            \text{Уменьшилась.}
    $
}
\solutionspace{40pt}

\tasknumber{7}%
\task{%
    Газу сообщили некоторое количество теплоты,
    при этом четверть его он потратил на совершение работы,
    одновременно увеличив свою внутреннюю энергию на $1200\,\text{Дж}$.
    Определите работу, совершённую газом.
}
\answer{%
    \begin{align*}
    Q &= A' + \Delta U, A' = \frac 14 Q \implies Q\cdot\cbr{1 - \frac 14} = \Delta U \implies Q = \frac{\Delta U}{1 - \frac 14} = \frac{1200\,\text{Дж}}{1 - \frac 14} \approx 1600\,\text{Дж}.
    \\
    A' &= \frac 14 Q
        = \frac 14 \cdot \frac{\Delta U}{1 - \frac 14}
        = \frac{\Delta U}{4 - 1}
        = \frac{1200\,\text{Дж}}{4 - 1} \approx 400\,\text{Дж}.
    \end{align*}
}
\solutionspace{60pt}

\tasknumber{8}%
\task{%
    В некотором процессе газ совершил работу $100\,\text{Дж}$,
    при этом его внутренняя энергия увеличилась на $450\,\text{Дж}$.
    Определите количество тепла, переданное при этом процессе газу.
    Явно пропишите, подводили газу тепло или же отводили.
}
\answer{%
    $
        Q = A_\text{газа} + \Delta U, A_\text{газа} = -A_\text{внешняя}
        \implies Q = A_\text{газа} + \Delta U =  100\,\text{Дж} +  450\,\text{Дж} = 550\,\text{Дж}.
        \text{ Подводили.}
    $
}

\variantsplitter

\addpersonalvariant{Тимофей Полетаев}

\tasknumber{1}%
\task{%
    Напротив физической величины укажите её обозначение и единицы измерения в СИ или запишите физический закон или формулу (в пункте «г)»):
    \begin{enumerate}
        \item количество теплоты,
        \item работа газа,
        \item удельная теплоёмкость,
        \item внутренняя энергия идеального одноатомного газа.
    \end{enumerate}
}
\solutionspace{20pt}

\tasknumber{2}%
\task{%
    Определите объём идеального одноатомного газа,
    если его внутренняя энергия при давлении $6\,\text{атм}$ составляет $300\,\text{кДж}$.
    $p_{\text{aтм}} = 100\,\text{кПа}$.
}
\answer{%
    $U = \frac 32 \nu R T = \frac 32 PV \implies V = \frac 23 \cdot \frac UP= \frac 23 \cdot \frac{ 300\,\text{кДж} }{ 6\,\text{атм} } \approx 0{,}33\,\text{м}^{3}.$
}
\solutionspace{40pt}

\tasknumber{3}%
\task{%
    Газ расширился от $350\,\text{л}$ до $550\,\text{л}$.
    Давление газа при этом оставалось постоянным и равным $1{,}5\,\text{атм}$.
    Определите работу газа, ответ выразите в килоджоулях.
    $p_{\text{aтм}} = 100\,\text{кПа}$.
}
\answer{%
    $A = P\Delta V = P(V_2 - V_1) = 1{,}5\,\text{атм} \cdot\cbr{550\,\text{л} - 350\,\text{л}} = 3{,}00\,\text{кДж}.$
}
\solutionspace{40pt}

\tasknumber{4}%
\task{%
    $30\,\text{моль}$ идеального одноатомного в результате адиабатического процесса нагрелся на $80\,\text{К}$.
    Определите работу газа.
    Кто совершил положительную работу: газ или внешние силы?
    Универсальная газовая постоянная $R = 8{,}31\,\frac{\text{Дж}}{\text{моль}\cdot\text{К}}$.
}
\answer{%
    \begin{align*}
    Q &= 0, Q = \Delta U + A_\text{газа} \implies \\
    \implies A_\text{газа} &= - \Delta U = - \frac 32 \nu R \Delta T = - \frac 32 \cdot 30\,\text{моль} \cdot8{,}31\,\frac{\text{Дж}}{\text{моль}\cdot\text{К}} \cdot80\,\text{К}= -29{,}90\,\text{кДж}, \text{внешние силы.}
    \end{align*}
}
\solutionspace{40pt}

\tasknumber{5}%
\task{%
    Как изменилась внутренняя энергия одноатомного идеального газа при переходе из состояния 1 в состояние 2?
    $P_1 = 3\,\text{МПа}$, $V_1 = 3\,\text{л}$, $P_2 = 2{,}5\,\text{МПа}$, $V_2 = 6\,\text{л}$.
    Как изменилась при этом температура газа?
}
\answer{%
    \begin{align*}
    P_1V_1 &= \nu R T_1, P_2V_2 = \nu R T_2, \\
    \Delta U &= U_2-U_1 = \frac 32 \nu R T_2- \frac 32 \nu R T_1 = \frac 32 P_2 V_2 - \frac 32 P_1 V_1= \frac 32 \cdot \cbr{2{,}5\,\text{МПа} \cdot6\,\text{л} - 3\,\text{МПа} \cdot3\,\text{л}} = 9000\,\text{Дж}.
    \\
    \frac{T_2}{T_1} &= \frac{\frac{P_2V_2}{\nu R}}{\frac{P_1V_1}{\nu R}} = \frac{P_2V_2}{P_1V_1}= \frac{2{,}5\,\text{МПа} \cdot6\,\text{л}}{3\,\text{МПа} \cdot3\,\text{л}} \approx 1{,}67.
    \end{align*}
}
\solutionspace{80pt}

\tasknumber{6}%
\task{%
    $2\,\text{моль}$ идеального одноатомного газа нагрели на $20\,\text{К}$.
    Определите изменение внутренней энергии газа.
    Увеличилась она или уменьшилась?
    Универсальная газовая постоянная $R = 8{,}31\,\frac{\text{Дж}}{\text{моль}\cdot\text{К}}$.
}
\answer{%
    $
        \Delta U = \frac 32 \nu R \Delta T
            =  \frac 32 \cdot 2\,\text{моль} \cdot8{,}31\,\frac{\text{Дж}}{\text{моль}\cdot\text{К}} \cdot20\,\text{К}
            = 498\,\text{Дж}.
            \text{Увеличилась.}
    $
}
\solutionspace{40pt}

\tasknumber{7}%
\task{%
    Газу сообщили некоторое количество теплоты,
    при этом треть его он потратил на совершение работы,
    одновременно увеличив свою внутреннюю энергию на $2400\,\text{Дж}$.
    Определите работу, совершённую газом.
}
\answer{%
    \begin{align*}
    Q &= A' + \Delta U, A' = \frac 13 Q \implies Q\cdot\cbr{1 - \frac 13} = \Delta U \implies Q = \frac{\Delta U}{1 - \frac 13} = \frac{2400\,\text{Дж}}{1 - \frac 13} \approx 3600\,\text{Дж}.
    \\
    A' &= \frac 13 Q
        = \frac 13 \cdot \frac{\Delta U}{1 - \frac 13}
        = \frac{\Delta U}{3 - 1}
        = \frac{2400\,\text{Дж}}{3 - 1} \approx 1200\,\text{Дж}.
    \end{align*}
}
\solutionspace{60pt}

\tasknumber{8}%
\task{%
    В некотором процессе внешние силы совершили над газом работу $200\,\text{Дж}$,
    при этом его внутренняя энергия уменьшилась на $350\,\text{Дж}$.
    Определите количество тепла, переданное при этом процессе газу.
    Явно пропишите, подводили газу тепло или же отводили.
}
\answer{%
    $
        Q = A_\text{газа} + \Delta U, A_\text{газа} = -A_\text{внешняя}
        \implies Q = A_\text{газа} + \Delta U = - 200\,\text{Дж} - 350\,\text{Дж} = -550\,\text{Дж}.
        \text{ Отводили.}
    $
}

\variantsplitter

\addpersonalvariant{Андрей Рожков}

\tasknumber{1}%
\task{%
    Напротив физической величины укажите её обозначение и единицы измерения в СИ или запишите физический закон или формулу (в пункте «г)»):
    \begin{enumerate}
        \item изменение внутренней энергии,
        \item работа газа,
        \item удельная теплоёмкость,
        \item первое начало термодинамики.
    \end{enumerate}
}
\solutionspace{20pt}

\tasknumber{2}%
\task{%
    Определите объём идеального одноатомного газа,
    если его внутренняя энергия при давлении $2\,\text{атм}$ составляет $300\,\text{кДж}$.
    $p_{\text{aтм}} = 100\,\text{кПа}$.
}
\answer{%
    $U = \frac 32 \nu R T = \frac 32 PV \implies V = \frac 23 \cdot \frac UP= \frac 23 \cdot \frac{ 300\,\text{кДж} }{ 2\,\text{атм} } \approx 1{,}00\,\text{м}^{3}.$
}
\solutionspace{40pt}

\tasknumber{3}%
\task{%
    Газ расширился от $200\,\text{л}$ до $650\,\text{л}$.
    Давление газа при этом оставалось постоянным и равным $1{,}5\,\text{атм}$.
    Определите работу газа, ответ выразите в килоджоулях.
    $p_{\text{aтм}} = 100\,\text{кПа}$.
}
\answer{%
    $A = P\Delta V = P(V_2 - V_1) = 1{,}5\,\text{атм} \cdot\cbr{650\,\text{л} - 200\,\text{л}} = 6{,}75\,\text{кДж}.$
}
\solutionspace{40pt}

\tasknumber{4}%
\task{%
    $30\,\text{моль}$ идеального одноатомного в результате адиабатического процесса остыл на $60\,\text{К}$.
    Определите работу газа.
    Кто совершил положительную работу: газ или внешние силы?
    Универсальная газовая постоянная $R = 8{,}31\,\frac{\text{Дж}}{\text{моль}\cdot\text{К}}$.
}
\answer{%
    \begin{align*}
    Q &= 0, Q = \Delta U + A_\text{газа} \implies \\
    \implies A_\text{газа} &= - \Delta U = - \frac 32 \nu R \Delta T =  \frac 32 \cdot 30\,\text{моль} \cdot8{,}31\,\frac{\text{Дж}}{\text{моль}\cdot\text{К}} \cdot60\,\text{К}= 22{,}4\,\text{кДж}, \text{газ.}
    \end{align*}
}
\solutionspace{40pt}

\tasknumber{5}%
\task{%
    Как изменилась внутренняя энергия одноатомного идеального газа при переходе из состояния 1 в состояние 2?
    $P_1 = 2\,\text{МПа}$, $V_1 = 5\,\text{л}$, $P_2 = 4{,}5\,\text{МПа}$, $V_2 = 6\,\text{л}$.
    Как изменилась при этом температура газа?
}
\answer{%
    \begin{align*}
    P_1V_1 &= \nu R T_1, P_2V_2 = \nu R T_2, \\
    \Delta U &= U_2-U_1 = \frac 32 \nu R T_2- \frac 32 \nu R T_1 = \frac 32 P_2 V_2 - \frac 32 P_1 V_1= \frac 32 \cdot \cbr{4{,}5\,\text{МПа} \cdot6\,\text{л} - 2\,\text{МПа} \cdot5\,\text{л}} = 25500\,\text{Дж}.
    \\
    \frac{T_2}{T_1} &= \frac{\frac{P_2V_2}{\nu R}}{\frac{P_1V_1}{\nu R}} = \frac{P_2V_2}{P_1V_1}= \frac{4{,}5\,\text{МПа} \cdot6\,\text{л}}{2\,\text{МПа} \cdot5\,\text{л}} \approx 2{,}70.
    \end{align*}
}
\solutionspace{80pt}

\tasknumber{6}%
\task{%
    $2\,\text{моль}$ идеального одноатомного газа охладили на $10\,\text{К}$.
    Определите изменение внутренней энергии газа.
    Увеличилась она или уменьшилась?
    Универсальная газовая постоянная $R = 8{,}31\,\frac{\text{Дж}}{\text{моль}\cdot\text{К}}$.
}
\answer{%
    $
        \Delta U = \frac 32 \nu R \Delta T
            = - \frac 32 \cdot 2\,\text{моль} \cdot8{,}31\,\frac{\text{Дж}}{\text{моль}\cdot\text{К}} \cdot10\,\text{К}
            = -249\,\text{Дж}.
            \text{Уменьшилась.}
    $
}
\solutionspace{40pt}

\tasknumber{7}%
\task{%
    Газу сообщили некоторое количество теплоты,
    при этом треть его он потратил на совершение работы,
    одновременно увеличив свою внутреннюю энергию на $3000\,\text{Дж}$.
    Определите количество теплоты, сообщённое газу.
}
\answer{%
    \begin{align*}
    Q &= A' + \Delta U, A' = \frac 13 Q \implies Q\cdot\cbr{1 - \frac 13} = \Delta U \implies Q = \frac{\Delta U}{1 - \frac 13} = \frac{3000\,\text{Дж}}{1 - \frac 13} \approx 4500\,\text{Дж}.
    \\
    A' &= \frac 13 Q
        = \frac 13 \cdot \frac{\Delta U}{1 - \frac 13}
        = \frac{\Delta U}{3 - 1}
        = \frac{3000\,\text{Дж}}{3 - 1} \approx 1500\,\text{Дж}.
    \end{align*}
}
\solutionspace{60pt}

\tasknumber{8}%
\task{%
    В некотором процессе газ совершил работу $200\,\text{Дж}$,
    при этом его внутренняя энергия увеличилась на $250\,\text{Дж}$.
    Определите количество тепла, переданное при этом процессе газу.
    Явно пропишите, подводили газу тепло или же отводили.
}
\answer{%
    $
        Q = A_\text{газа} + \Delta U, A_\text{газа} = -A_\text{внешняя}
        \implies Q = A_\text{газа} + \Delta U =  200\,\text{Дж} +  250\,\text{Дж} = 450\,\text{Дж}.
        \text{ Подводили.}
    $
}

\variantsplitter

\addpersonalvariant{Рената Таржиманова}

\tasknumber{1}%
\task{%
    Напротив физической величины укажите её обозначение и единицы измерения в СИ или запишите физический закон или формулу (в пункте «г)»):
    \begin{enumerate}
        \item количество теплоты,
        \item работа газа,
        \item удельная теплоёмкость,
        \item первое начало термодинамики.
    \end{enumerate}
}
\solutionspace{20pt}

\tasknumber{2}%
\task{%
    Определите объём идеального одноатомного газа,
    если его внутренняя энергия при давлении $5\,\text{атм}$ составляет $300\,\text{кДж}$.
    $p_{\text{aтм}} = 100\,\text{кПа}$.
}
\answer{%
    $U = \frac 32 \nu R T = \frac 32 PV \implies V = \frac 23 \cdot \frac UP= \frac 23 \cdot \frac{ 300\,\text{кДж} }{ 5\,\text{атм} } \approx 0{,}40\,\text{м}^{3}.$
}
\solutionspace{40pt}

\tasknumber{3}%
\task{%
    Газ расширился от $350\,\text{л}$ до $650\,\text{л}$.
    Давление газа при этом оставалось постоянным и равным $1{,}5\,\text{атм}$.
    Определите работу газа, ответ выразите в килоджоулях.
    $p_{\text{aтм}} = 100\,\text{кПа}$.
}
\answer{%
    $A = P\Delta V = P(V_2 - V_1) = 1{,}5\,\text{атм} \cdot\cbr{650\,\text{л} - 350\,\text{л}} = 4{,}50\,\text{кДж}.$
}
\solutionspace{40pt}

\tasknumber{4}%
\task{%
    $40\,\text{моль}$ идеального одноатомного в результате адиабатического процесса остыл на $80\,\text{К}$.
    Определите работу газа.
    Кто совершил положительную работу: газ или внешние силы?
    Универсальная газовая постоянная $R = 8{,}31\,\frac{\text{Дж}}{\text{моль}\cdot\text{К}}$.
}
\answer{%
    \begin{align*}
    Q &= 0, Q = \Delta U + A_\text{газа} \implies \\
    \implies A_\text{газа} &= - \Delta U = - \frac 32 \nu R \Delta T =  \frac 32 \cdot 40\,\text{моль} \cdot8{,}31\,\frac{\text{Дж}}{\text{моль}\cdot\text{К}} \cdot80\,\text{К}= 39{,}9\,\text{кДж}, \text{газ.}
    \end{align*}
}
\solutionspace{40pt}

\tasknumber{5}%
\task{%
    Как изменилась внутренняя энергия одноатомного идеального газа при переходе из состояния 1 в состояние 2?
    $P_1 = 4\,\text{МПа}$, $V_1 = 7\,\text{л}$, $P_2 = 3{,}5\,\text{МПа}$, $V_2 = 6\,\text{л}$.
    Как изменилась при этом температура газа?
}
\answer{%
    \begin{align*}
    P_1V_1 &= \nu R T_1, P_2V_2 = \nu R T_2, \\
    \Delta U &= U_2-U_1 = \frac 32 \nu R T_2- \frac 32 \nu R T_1 = \frac 32 P_2 V_2 - \frac 32 P_1 V_1= \frac 32 \cdot \cbr{3{,}5\,\text{МПа} \cdot6\,\text{л} - 4\,\text{МПа} \cdot7\,\text{л}} = -10500\,\text{Дж}.
    \\
    \frac{T_2}{T_1} &= \frac{\frac{P_2V_2}{\nu R}}{\frac{P_1V_1}{\nu R}} = \frac{P_2V_2}{P_1V_1}= \frac{3{,}5\,\text{МПа} \cdot6\,\text{л}}{4\,\text{МПа} \cdot7\,\text{л}} \approx 0{,}75.
    \end{align*}
}
\solutionspace{80pt}

\tasknumber{6}%
\task{%
    $3\,\text{моль}$ идеального одноатомного газа охладили на $20\,\text{К}$.
    Определите изменение внутренней энергии газа.
    Увеличилась она или уменьшилась?
    Универсальная газовая постоянная $R = 8{,}31\,\frac{\text{Дж}}{\text{моль}\cdot\text{К}}$.
}
\answer{%
    $
        \Delta U = \frac 32 \nu R \Delta T
            = - \frac 32 \cdot 3\,\text{моль} \cdot8{,}31\,\frac{\text{Дж}}{\text{моль}\cdot\text{К}} \cdot20\,\text{К}
            = -747\,\text{Дж}.
            \text{Уменьшилась.}
    $
}
\solutionspace{40pt}

\tasknumber{7}%
\task{%
    Газу сообщили некоторое количество теплоты,
    при этом треть его он потратил на совершение работы,
    одновременно увеличив свою внутреннюю энергию на $1200\,\text{Дж}$.
    Определите количество теплоты, сообщённое газу.
}
\answer{%
    \begin{align*}
    Q &= A' + \Delta U, A' = \frac 13 Q \implies Q\cdot\cbr{1 - \frac 13} = \Delta U \implies Q = \frac{\Delta U}{1 - \frac 13} = \frac{1200\,\text{Дж}}{1 - \frac 13} \approx 1800\,\text{Дж}.
    \\
    A' &= \frac 13 Q
        = \frac 13 \cdot \frac{\Delta U}{1 - \frac 13}
        = \frac{\Delta U}{3 - 1}
        = \frac{1200\,\text{Дж}}{3 - 1} \approx 600\,\text{Дж}.
    \end{align*}
}
\solutionspace{60pt}

\tasknumber{8}%
\task{%
    В некотором процессе внешние силы совершили над газом работу $100\,\text{Дж}$,
    при этом его внутренняя энергия уменьшилась на $250\,\text{Дж}$.
    Определите количество тепла, переданное при этом процессе газу.
    Явно пропишите, подводили газу тепло или же отводили.
}
\answer{%
    $
        Q = A_\text{газа} + \Delta U, A_\text{газа} = -A_\text{внешняя}
        \implies Q = A_\text{газа} + \Delta U = - 100\,\text{Дж} - 250\,\text{Дж} = -350\,\text{Дж}.
        \text{ Отводили.}
    $
}

\variantsplitter

\addpersonalvariant{Арсений Трофимов}

\tasknumber{1}%
\task{%
    Напротив физической величины укажите её обозначение и единицы измерения в СИ или запишите физический закон или формулу (в пункте «г)»):
    \begin{enumerate}
        \item изменение внутренней энергии,
        \item работа внешних сил,
        \item молярная теплоёмкость,
        \item внутренняя энергия идеального одноатомного газа.
    \end{enumerate}
}
\solutionspace{20pt}

\tasknumber{2}%
\task{%
    Определите объём идеального одноатомного газа,
    если его внутренняя энергия при давлении $6\,\text{атм}$ составляет $400\,\text{кДж}$.
    $p_{\text{aтм}} = 100\,\text{кПа}$.
}
\answer{%
    $U = \frac 32 \nu R T = \frac 32 PV \implies V = \frac 23 \cdot \frac UP= \frac 23 \cdot \frac{ 400\,\text{кДж} }{ 6\,\text{атм} } \approx 0{,}44\,\text{м}^{3}.$
}
\solutionspace{40pt}

\tasknumber{3}%
\task{%
    Газ расширился от $200\,\text{л}$ до $450\,\text{л}$.
    Давление газа при этом оставалось постоянным и равным $3{,}5\,\text{атм}$.
    Определите работу газа, ответ выразите в килоджоулях.
    $p_{\text{aтм}} = 100\,\text{кПа}$.
}
\answer{%
    $A = P\Delta V = P(V_2 - V_1) = 3{,}5\,\text{атм} \cdot\cbr{450\,\text{л} - 200\,\text{л}} = 8{,}75\,\text{кДж}.$
}
\solutionspace{40pt}

\tasknumber{4}%
\task{%
    $50\,\text{моль}$ идеального одноатомного в результате адиабатического процесса остыл на $80\,\text{К}$.
    Определите работу газа.
    Кто совершил положительную работу: газ или внешние силы?
    Универсальная газовая постоянная $R = 8{,}31\,\frac{\text{Дж}}{\text{моль}\cdot\text{К}}$.
}
\answer{%
    \begin{align*}
    Q &= 0, Q = \Delta U + A_\text{газа} \implies \\
    \implies A_\text{газа} &= - \Delta U = - \frac 32 \nu R \Delta T =  \frac 32 \cdot 50\,\text{моль} \cdot8{,}31\,\frac{\text{Дж}}{\text{моль}\cdot\text{К}} \cdot80\,\text{К}= 49{,}9\,\text{кДж}, \text{газ.}
    \end{align*}
}
\solutionspace{40pt}

\tasknumber{5}%
\task{%
    Как изменилась внутренняя энергия одноатомного идеального газа при переходе из состояния 1 в состояние 2?
    $P_1 = 2\,\text{МПа}$, $V_1 = 3\,\text{л}$, $P_2 = 3{,}5\,\text{МПа}$, $V_2 = 6\,\text{л}$.
    Как изменилась при этом температура газа?
}
\answer{%
    \begin{align*}
    P_1V_1 &= \nu R T_1, P_2V_2 = \nu R T_2, \\
    \Delta U &= U_2-U_1 = \frac 32 \nu R T_2- \frac 32 \nu R T_1 = \frac 32 P_2 V_2 - \frac 32 P_1 V_1= \frac 32 \cdot \cbr{3{,}5\,\text{МПа} \cdot6\,\text{л} - 2\,\text{МПа} \cdot3\,\text{л}} = 22500\,\text{Дж}.
    \\
    \frac{T_2}{T_1} &= \frac{\frac{P_2V_2}{\nu R}}{\frac{P_1V_1}{\nu R}} = \frac{P_2V_2}{P_1V_1}= \frac{3{,}5\,\text{МПа} \cdot6\,\text{л}}{2\,\text{МПа} \cdot3\,\text{л}} \approx 3{,}50.
    \end{align*}
}
\solutionspace{80pt}

\tasknumber{6}%
\task{%
    $4\,\text{моль}$ идеального одноатомного газа охладили на $20\,\text{К}$.
    Определите изменение внутренней энергии газа.
    Увеличилась она или уменьшилась?
    Универсальная газовая постоянная $R = 8{,}31\,\frac{\text{Дж}}{\text{моль}\cdot\text{К}}$.
}
\answer{%
    $
        \Delta U = \frac 32 \nu R \Delta T
            = - \frac 32 \cdot 4\,\text{моль} \cdot8{,}31\,\frac{\text{Дж}}{\text{моль}\cdot\text{К}} \cdot20\,\text{К}
            = -997\,\text{Дж}.
            \text{Уменьшилась.}
    $
}
\solutionspace{40pt}

\tasknumber{7}%
\task{%
    Газу сообщили некоторое количество теплоты,
    при этом половину его он потратил на совершение работы,
    одновременно увеличив свою внутреннюю энергию на $1200\,\text{Дж}$.
    Определите работу, совершённую газом.
}
\answer{%
    \begin{align*}
    Q &= A' + \Delta U, A' = \frac 12 Q \implies Q\cdot\cbr{1 - \frac 12} = \Delta U \implies Q = \frac{\Delta U}{1 - \frac 12} = \frac{1200\,\text{Дж}}{1 - \frac 12} \approx 2400\,\text{Дж}.
    \\
    A' &= \frac 12 Q
        = \frac 12 \cdot \frac{\Delta U}{1 - \frac 12}
        = \frac{\Delta U}{2 - 1}
        = \frac{1200\,\text{Дж}}{2 - 1} \approx 1200\,\text{Дж}.
    \end{align*}
}
\solutionspace{60pt}

\tasknumber{8}%
\task{%
    В некотором процессе газ совершил работу $300\,\text{Дж}$,
    при этом его внутренняя энергия уменьшилась на $150\,\text{Дж}$.
    Определите количество тепла, переданное при этом процессе газу.
    Явно пропишите, подводили газу тепло или же отводили.
}
\answer{%
    $
        Q = A_\text{газа} + \Delta U, A_\text{газа} = -A_\text{внешняя}
        \implies Q = A_\text{газа} + \Delta U =  300\,\text{Дж} - 150\,\text{Дж} = 150\,\text{Дж}.
        \text{ Подводили.}
    $
}

\variantsplitter

\addpersonalvariant{Андрей Щербаков}

\tasknumber{1}%
\task{%
    Напротив физической величины укажите её обозначение и единицы измерения в СИ или запишите физический закон или формулу (в пункте «г)»):
    \begin{enumerate}
        \item изменение внутренней энергии,
        \item работа внешних сил,
        \item удельная теплоёмкость,
        \item первое начало термодинамики.
    \end{enumerate}
}
\solutionspace{20pt}

\tasknumber{2}%
\task{%
    Определите объём идеального одноатомного газа,
    если его внутренняя энергия при давлении $5\,\text{атм}$ составляет $300\,\text{кДж}$.
    $p_{\text{aтм}} = 100\,\text{кПа}$.
}
\answer{%
    $U = \frac 32 \nu R T = \frac 32 PV \implies V = \frac 23 \cdot \frac UP= \frac 23 \cdot \frac{ 300\,\text{кДж} }{ 5\,\text{атм} } \approx 0{,}40\,\text{м}^{3}.$
}
\solutionspace{40pt}

\tasknumber{3}%
\task{%
    Газ расширился от $250\,\text{л}$ до $450\,\text{л}$.
    Давление газа при этом оставалось постоянным и равным $1{,}5\,\text{атм}$.
    Определите работу газа, ответ выразите в килоджоулях.
    $p_{\text{aтм}} = 100\,\text{кПа}$.
}
\answer{%
    $A = P\Delta V = P(V_2 - V_1) = 1{,}5\,\text{атм} \cdot\cbr{450\,\text{л} - 250\,\text{л}} = 3{,}00\,\text{кДж}.$
}
\solutionspace{40pt}

\tasknumber{4}%
\task{%
    $60\,\text{моль}$ идеального одноатомного в результате адиабатического процесса нагрелся на $80\,\text{К}$.
    Определите работу газа.
    Кто совершил положительную работу: газ или внешние силы?
    Универсальная газовая постоянная $R = 8{,}31\,\frac{\text{Дж}}{\text{моль}\cdot\text{К}}$.
}
\answer{%
    \begin{align*}
    Q &= 0, Q = \Delta U + A_\text{газа} \implies \\
    \implies A_\text{газа} &= - \Delta U = - \frac 32 \nu R \Delta T = - \frac 32 \cdot 60\,\text{моль} \cdot8{,}31\,\frac{\text{Дж}}{\text{моль}\cdot\text{К}} \cdot80\,\text{К}= -59{,}80\,\text{кДж}, \text{внешние силы.}
    \end{align*}
}
\solutionspace{40pt}

\tasknumber{5}%
\task{%
    Как изменилась внутренняя энергия одноатомного идеального газа при переходе из состояния 1 в состояние 2?
    $P_1 = 4\,\text{МПа}$, $V_1 = 3\,\text{л}$, $P_2 = 2{,}5\,\text{МПа}$, $V_2 = 6\,\text{л}$.
    Как изменилась при этом температура газа?
}
\answer{%
    \begin{align*}
    P_1V_1 &= \nu R T_1, P_2V_2 = \nu R T_2, \\
    \Delta U &= U_2-U_1 = \frac 32 \nu R T_2- \frac 32 \nu R T_1 = \frac 32 P_2 V_2 - \frac 32 P_1 V_1= \frac 32 \cdot \cbr{2{,}5\,\text{МПа} \cdot6\,\text{л} - 4\,\text{МПа} \cdot3\,\text{л}} = 4500\,\text{Дж}.
    \\
    \frac{T_2}{T_1} &= \frac{\frac{P_2V_2}{\nu R}}{\frac{P_1V_1}{\nu R}} = \frac{P_2V_2}{P_1V_1}= \frac{2{,}5\,\text{МПа} \cdot6\,\text{л}}{4\,\text{МПа} \cdot3\,\text{л}} \approx 1{,}25.
    \end{align*}
}
\solutionspace{80pt}

\tasknumber{6}%
\task{%
    $5\,\text{моль}$ идеального одноатомного газа нагрели на $20\,\text{К}$.
    Определите изменение внутренней энергии газа.
    Увеличилась она или уменьшилась?
    Универсальная газовая постоянная $R = 8{,}31\,\frac{\text{Дж}}{\text{моль}\cdot\text{К}}$.
}
\answer{%
    $
        \Delta U = \frac 32 \nu R \Delta T
            =  \frac 32 \cdot 5\,\text{моль} \cdot8{,}31\,\frac{\text{Дж}}{\text{моль}\cdot\text{К}} \cdot20\,\text{К}
            = 1246\,\text{Дж}.
            \text{Увеличилась.}
    $
}
\solutionspace{40pt}

\tasknumber{7}%
\task{%
    Газу сообщили некоторое количество теплоты,
    при этом четверть его он потратил на совершение работы,
    одновременно увеличив свою внутреннюю энергию на $3000\,\text{Дж}$.
    Определите работу, совершённую газом.
}
\answer{%
    \begin{align*}
    Q &= A' + \Delta U, A' = \frac 14 Q \implies Q\cdot\cbr{1 - \frac 14} = \Delta U \implies Q = \frac{\Delta U}{1 - \frac 14} = \frac{3000\,\text{Дж}}{1 - \frac 14} \approx 4000\,\text{Дж}.
    \\
    A' &= \frac 14 Q
        = \frac 14 \cdot \frac{\Delta U}{1 - \frac 14}
        = \frac{\Delta U}{4 - 1}
        = \frac{3000\,\text{Дж}}{4 - 1} \approx 1000\,\text{Дж}.
    \end{align*}
}
\solutionspace{60pt}

\tasknumber{8}%
\task{%
    В некотором процессе газ совершил работу $100\,\text{Дж}$,
    при этом его внутренняя энергия уменьшилась на $250\,\text{Дж}$.
    Определите количество тепла, переданное при этом процессе газу.
    Явно пропишите, подводили газу тепло или же отводили.
}
\answer{%
    $
        Q = A_\text{газа} + \Delta U, A_\text{газа} = -A_\text{внешняя}
        \implies Q = A_\text{газа} + \Delta U =  100\,\text{Дж} - 250\,\text{Дж} = -150\,\text{Дж}.
        \text{ Отводили.}
    $
}

\variantsplitter

\addpersonalvariant{Михаил Ярошевский}

\tasknumber{1}%
\task{%
    Напротив физической величины укажите её обозначение и единицы измерения в СИ или запишите физический закон или формулу (в пункте «г)»):
    \begin{enumerate}
        \item количество теплоты,
        \item работа газа,
        \item молярная теплоёмкость,
        \item первое начало термодинамики.
    \end{enumerate}
}
\solutionspace{20pt}

\tasknumber{2}%
\task{%
    Определите объём идеального одноатомного газа,
    если его внутренняя энергия при давлении $2\,\text{атм}$ составляет $400\,\text{кДж}$.
    $p_{\text{aтм}} = 100\,\text{кПа}$.
}
\answer{%
    $U = \frac 32 \nu R T = \frac 32 PV \implies V = \frac 23 \cdot \frac UP= \frac 23 \cdot \frac{ 400\,\text{кДж} }{ 2\,\text{атм} } \approx 1{,}33\,\text{м}^{3}.$
}
\solutionspace{40pt}

\tasknumber{3}%
\task{%
    Газ расширился от $150\,\text{л}$ до $550\,\text{л}$.
    Давление газа при этом оставалось постоянным и равным $3{,}5\,\text{атм}$.
    Определите работу газа, ответ выразите в килоджоулях.
    $p_{\text{aтм}} = 100\,\text{кПа}$.
}
\answer{%
    $A = P\Delta V = P(V_2 - V_1) = 3{,}5\,\text{атм} \cdot\cbr{550\,\text{л} - 150\,\text{л}} = 14{,}00\,\text{кДж}.$
}
\solutionspace{40pt}

\tasknumber{4}%
\task{%
    $40\,\text{моль}$ идеального одноатомного в результате адиабатического процесса нагрелся на $80\,\text{К}$.
    Определите работу газа.
    Кто совершил положительную работу: газ или внешние силы?
    Универсальная газовая постоянная $R = 8{,}31\,\frac{\text{Дж}}{\text{моль}\cdot\text{К}}$.
}
\answer{%
    \begin{align*}
    Q &= 0, Q = \Delta U + A_\text{газа} \implies \\
    \implies A_\text{газа} &= - \Delta U = - \frac 32 \nu R \Delta T = - \frac 32 \cdot 40\,\text{моль} \cdot8{,}31\,\frac{\text{Дж}}{\text{моль}\cdot\text{К}} \cdot80\,\text{К}= -39{,}90\,\text{кДж}, \text{внешние силы.}
    \end{align*}
}
\solutionspace{40pt}

\tasknumber{5}%
\task{%
    Как изменилась внутренняя энергия одноатомного идеального газа при переходе из состояния 1 в состояние 2?
    $P_1 = 3\,\text{МПа}$, $V_1 = 3\,\text{л}$, $P_2 = 3{,}5\,\text{МПа}$, $V_2 = 8\,\text{л}$.
    Как изменилась при этом температура газа?
}
\answer{%
    \begin{align*}
    P_1V_1 &= \nu R T_1, P_2V_2 = \nu R T_2, \\
    \Delta U &= U_2-U_1 = \frac 32 \nu R T_2- \frac 32 \nu R T_1 = \frac 32 P_2 V_2 - \frac 32 P_1 V_1= \frac 32 \cdot \cbr{3{,}5\,\text{МПа} \cdot8\,\text{л} - 3\,\text{МПа} \cdot3\,\text{л}} = 28500\,\text{Дж}.
    \\
    \frac{T_2}{T_1} &= \frac{\frac{P_2V_2}{\nu R}}{\frac{P_1V_1}{\nu R}} = \frac{P_2V_2}{P_1V_1}= \frac{3{,}5\,\text{МПа} \cdot8\,\text{л}}{3\,\text{МПа} \cdot3\,\text{л}} \approx 3{,}11.
    \end{align*}
}
\solutionspace{80pt}

\tasknumber{6}%
\task{%
    $3\,\text{моль}$ идеального одноатомного газа нагрели на $20\,\text{К}$.
    Определите изменение внутренней энергии газа.
    Увеличилась она или уменьшилась?
    Универсальная газовая постоянная $R = 8{,}31\,\frac{\text{Дж}}{\text{моль}\cdot\text{К}}$.
}
\answer{%
    $
        \Delta U = \frac 32 \nu R \Delta T
            =  \frac 32 \cdot 3\,\text{моль} \cdot8{,}31\,\frac{\text{Дж}}{\text{моль}\cdot\text{К}} \cdot20\,\text{К}
            = 747\,\text{Дж}.
            \text{Увеличилась.}
    $
}
\solutionspace{40pt}

\tasknumber{7}%
\task{%
    Газу сообщили некоторое количество теплоты,
    при этом половину его он потратил на совершение работы,
    одновременно увеличив свою внутреннюю энергию на $1500\,\text{Дж}$.
    Определите работу, совершённую газом.
}
\answer{%
    \begin{align*}
    Q &= A' + \Delta U, A' = \frac 12 Q \implies Q\cdot\cbr{1 - \frac 12} = \Delta U \implies Q = \frac{\Delta U}{1 - \frac 12} = \frac{1500\,\text{Дж}}{1 - \frac 12} \approx 3000\,\text{Дж}.
    \\
    A' &= \frac 12 Q
        = \frac 12 \cdot \frac{\Delta U}{1 - \frac 12}
        = \frac{\Delta U}{2 - 1}
        = \frac{1500\,\text{Дж}}{2 - 1} \approx 1500\,\text{Дж}.
    \end{align*}
}
\solutionspace{60pt}

\tasknumber{8}%
\task{%
    В некотором процессе газ совершил работу $200\,\text{Дж}$,
    при этом его внутренняя энергия увеличилась на $350\,\text{Дж}$.
    Определите количество тепла, переданное при этом процессе газу.
    Явно пропишите, подводили газу тепло или же отводили.
}
\answer{%
    $
        Q = A_\text{газа} + \Delta U, A_\text{газа} = -A_\text{внешняя}
        \implies Q = A_\text{газа} + \Delta U =  200\,\text{Дж} +  350\,\text{Дж} = 550\,\text{Дж}.
        \text{ Подводили.}
    $
}

\variantsplitter

\addpersonalvariant{Алексей Алимпиев}

\tasknumber{1}%
\task{%
    Напротив физической величины укажите её обозначение и единицы измерения в СИ или запишите физический закон или формулу (в пункте «г)»):
    \begin{enumerate}
        \item количество теплоты,
        \item работа внешних сил,
        \item молярная теплоёмкость,
        \item первое начало термодинамики.
    \end{enumerate}
}
\solutionspace{20pt}

\tasknumber{2}%
\task{%
    Определите объём идеального одноатомного газа,
    если его внутренняя энергия при давлении $5\,\text{атм}$ составляет $400\,\text{кДж}$.
    $p_{\text{aтм}} = 100\,\text{кПа}$.
}
\answer{%
    $U = \frac 32 \nu R T = \frac 32 PV \implies V = \frac 23 \cdot \frac UP= \frac 23 \cdot \frac{ 400\,\text{кДж} }{ 5\,\text{атм} } \approx 0{,}53\,\text{м}^{3}.$
}
\solutionspace{40pt}

\tasknumber{3}%
\task{%
    Газ расширился от $200\,\text{л}$ до $550\,\text{л}$.
    Давление газа при этом оставалось постоянным и равным $1{,}2\,\text{атм}$.
    Определите работу газа, ответ выразите в килоджоулях.
    $p_{\text{aтм}} = 100\,\text{кПа}$.
}
\answer{%
    $A = P\Delta V = P(V_2 - V_1) = 1{,}2\,\text{атм} \cdot\cbr{550\,\text{л} - 200\,\text{л}} = 4{,}20\,\text{кДж}.$
}
\solutionspace{40pt}

\tasknumber{4}%
\task{%
    $50\,\text{моль}$ идеального одноатомного в результате адиабатического процесса нагрелся на $15\,\text{К}$.
    Определите работу газа.
    Кто совершил положительную работу: газ или внешние силы?
    Универсальная газовая постоянная $R = 8{,}31\,\frac{\text{Дж}}{\text{моль}\cdot\text{К}}$.
}
\answer{%
    \begin{align*}
    Q &= 0, Q = \Delta U + A_\text{газа} \implies \\
    \implies A_\text{газа} &= - \Delta U = - \frac 32 \nu R \Delta T = - \frac 32 \cdot 50\,\text{моль} \cdot8{,}31\,\frac{\text{Дж}}{\text{моль}\cdot\text{К}} \cdot15\,\text{К}= -9{,}30\,\text{кДж}, \text{внешние силы.}
    \end{align*}
}
\solutionspace{40pt}

\tasknumber{5}%
\task{%
    Как изменилась внутренняя энергия одноатомного идеального газа при переходе из состояния 1 в состояние 2?
    $P_1 = 2\,\text{МПа}$, $V_1 = 7\,\text{л}$, $P_2 = 1{,}5\,\text{МПа}$, $V_2 = 2\,\text{л}$.
    Как изменилась при этом температура газа?
}
\answer{%
    \begin{align*}
    P_1V_1 &= \nu R T_1, P_2V_2 = \nu R T_2, \\
    \Delta U &= U_2-U_1 = \frac 32 \nu R T_2- \frac 32 \nu R T_1 = \frac 32 P_2 V_2 - \frac 32 P_1 V_1= \frac 32 \cdot \cbr{1{,}5\,\text{МПа} \cdot2\,\text{л} - 2\,\text{МПа} \cdot7\,\text{л}} = -16500\,\text{Дж}.
    \\
    \frac{T_2}{T_1} &= \frac{\frac{P_2V_2}{\nu R}}{\frac{P_1V_1}{\nu R}} = \frac{P_2V_2}{P_1V_1}= \frac{1{,}5\,\text{МПа} \cdot2\,\text{л}}{2\,\text{МПа} \cdot7\,\text{л}} \approx 0{,}21.
    \end{align*}
}
\solutionspace{80pt}

\tasknumber{6}%
\task{%
    $4\,\text{моль}$ идеального одноатомного газа нагрели на $10\,\text{К}$.
    Определите изменение внутренней энергии газа.
    Увеличилась она или уменьшилась?
    Универсальная газовая постоянная $R = 8{,}31\,\frac{\text{Дж}}{\text{моль}\cdot\text{К}}$.
}
\answer{%
    $
        \Delta U = \frac 32 \nu R \Delta T
            =  \frac 32 \cdot 4\,\text{моль} \cdot8{,}31\,\frac{\text{Дж}}{\text{моль}\cdot\text{К}} \cdot10\,\text{К}
            = 498\,\text{Дж}.
            \text{Увеличилась.}
    $
}
\solutionspace{40pt}

\tasknumber{7}%
\task{%
    Газу сообщили некоторое количество теплоты,
    при этом четверть его он потратил на совершение работы,
    одновременно увеличив свою внутреннюю энергию на $2400\,\text{Дж}$.
    Определите количество теплоты, сообщённое газу.
}
\answer{%
    \begin{align*}
    Q &= A' + \Delta U, A' = \frac 14 Q \implies Q\cdot\cbr{1 - \frac 14} = \Delta U \implies Q = \frac{\Delta U}{1 - \frac 14} = \frac{2400\,\text{Дж}}{1 - \frac 14} \approx 3200\,\text{Дж}.
    \\
    A' &= \frac 14 Q
        = \frac 14 \cdot \frac{\Delta U}{1 - \frac 14}
        = \frac{\Delta U}{4 - 1}
        = \frac{2400\,\text{Дж}}{4 - 1} \approx 800\,\text{Дж}.
    \end{align*}
}
\solutionspace{60pt}

\tasknumber{8}%
\task{%
    В некотором процессе внешние силы совершили над газом работу $200\,\text{Дж}$,
    при этом его внутренняя энергия увеличилась на $450\,\text{Дж}$.
    Определите количество тепла, переданное при этом процессе газу.
    Явно пропишите, подводили газу тепло или же отводили.
}
\answer{%
    $
        Q = A_\text{газа} + \Delta U, A_\text{газа} = -A_\text{внешняя}
        \implies Q = A_\text{газа} + \Delta U = - 200\,\text{Дж} +  450\,\text{Дж} = 250\,\text{Дж}.
        \text{ Подводили.}
    $
}

\variantsplitter

\addpersonalvariant{Евгений Васин}

\tasknumber{1}%
\task{%
    Напротив физической величины укажите её обозначение и единицы измерения в СИ или запишите физический закон или формулу (в пункте «г)»):
    \begin{enumerate}
        \item изменение внутренней энергии,
        \item работа газа,
        \item удельная теплоёмкость,
        \item внутренняя энергия идеального одноатомного газа.
    \end{enumerate}
}
\solutionspace{20pt}

\tasknumber{2}%
\task{%
    Определите объём идеального одноатомного газа,
    если его внутренняя энергия при давлении $2\,\text{атм}$ составляет $300\,\text{кДж}$.
    $p_{\text{aтм}} = 100\,\text{кПа}$.
}
\answer{%
    $U = \frac 32 \nu R T = \frac 32 PV \implies V = \frac 23 \cdot \frac UP= \frac 23 \cdot \frac{ 300\,\text{кДж} }{ 2\,\text{атм} } \approx 1{,}00\,\text{м}^{3}.$
}
\solutionspace{40pt}

\tasknumber{3}%
\task{%
    Газ расширился от $350\,\text{л}$ до $550\,\text{л}$.
    Давление газа при этом оставалось постоянным и равным $3{,}5\,\text{атм}$.
    Определите работу газа, ответ выразите в килоджоулях.
    $p_{\text{aтм}} = 100\,\text{кПа}$.
}
\answer{%
    $A = P\Delta V = P(V_2 - V_1) = 3{,}5\,\text{атм} \cdot\cbr{550\,\text{л} - 350\,\text{л}} = 7{,}00\,\text{кДж}.$
}
\solutionspace{40pt}

\tasknumber{4}%
\task{%
    $40\,\text{моль}$ идеального одноатомного в результате адиабатического процесса нагрелся на $25\,\text{К}$.
    Определите работу газа.
    Кто совершил положительную работу: газ или внешние силы?
    Универсальная газовая постоянная $R = 8{,}31\,\frac{\text{Дж}}{\text{моль}\cdot\text{К}}$.
}
\answer{%
    \begin{align*}
    Q &= 0, Q = \Delta U + A_\text{газа} \implies \\
    \implies A_\text{газа} &= - \Delta U = - \frac 32 \nu R \Delta T = - \frac 32 \cdot 40\,\text{моль} \cdot8{,}31\,\frac{\text{Дж}}{\text{моль}\cdot\text{К}} \cdot25\,\text{К}= -12{,}500\,\text{кДж}, \text{внешние силы.}
    \end{align*}
}
\solutionspace{40pt}

\tasknumber{5}%
\task{%
    Как изменилась внутренняя энергия одноатомного идеального газа при переходе из состояния 1 в состояние 2?
    $P_1 = 2\,\text{МПа}$, $V_1 = 7\,\text{л}$, $P_2 = 3{,}5\,\text{МПа}$, $V_2 = 2\,\text{л}$.
    Как изменилась при этом температура газа?
}
\answer{%
    \begin{align*}
    P_1V_1 &= \nu R T_1, P_2V_2 = \nu R T_2, \\
    \Delta U &= U_2-U_1 = \frac 32 \nu R T_2- \frac 32 \nu R T_1 = \frac 32 P_2 V_2 - \frac 32 P_1 V_1= \frac 32 \cdot \cbr{3{,}5\,\text{МПа} \cdot2\,\text{л} - 2\,\text{МПа} \cdot7\,\text{л}} = -10500\,\text{Дж}.
    \\
    \frac{T_2}{T_1} &= \frac{\frac{P_2V_2}{\nu R}}{\frac{P_1V_1}{\nu R}} = \frac{P_2V_2}{P_1V_1}= \frac{3{,}5\,\text{МПа} \cdot2\,\text{л}}{2\,\text{МПа} \cdot7\,\text{л}} \approx 0{,}50.
    \end{align*}
}
\solutionspace{80pt}

\tasknumber{6}%
\task{%
    $3\,\text{моль}$ идеального одноатомного газа нагрели на $20\,\text{К}$.
    Определите изменение внутренней энергии газа.
    Увеличилась она или уменьшилась?
    Универсальная газовая постоянная $R = 8{,}31\,\frac{\text{Дж}}{\text{моль}\cdot\text{К}}$.
}
\answer{%
    $
        \Delta U = \frac 32 \nu R \Delta T
            =  \frac 32 \cdot 3\,\text{моль} \cdot8{,}31\,\frac{\text{Дж}}{\text{моль}\cdot\text{К}} \cdot20\,\text{К}
            = 747\,\text{Дж}.
            \text{Увеличилась.}
    $
}
\solutionspace{40pt}

\tasknumber{7}%
\task{%
    Газу сообщили некоторое количество теплоты,
    при этом треть его он потратил на совершение работы,
    одновременно увеличив свою внутреннюю энергию на $1200\,\text{Дж}$.
    Определите работу, совершённую газом.
}
\answer{%
    \begin{align*}
    Q &= A' + \Delta U, A' = \frac 13 Q \implies Q\cdot\cbr{1 - \frac 13} = \Delta U \implies Q = \frac{\Delta U}{1 - \frac 13} = \frac{1200\,\text{Дж}}{1 - \frac 13} \approx 1800\,\text{Дж}.
    \\
    A' &= \frac 13 Q
        = \frac 13 \cdot \frac{\Delta U}{1 - \frac 13}
        = \frac{\Delta U}{3 - 1}
        = \frac{1200\,\text{Дж}}{3 - 1} \approx 600\,\text{Дж}.
    \end{align*}
}
\solutionspace{60pt}

\tasknumber{8}%
\task{%
    В некотором процессе внешние силы совершили над газом работу $200\,\text{Дж}$,
    при этом его внутренняя энергия уменьшилась на $250\,\text{Дж}$.
    Определите количество тепла, переданное при этом процессе газу.
    Явно пропишите, подводили газу тепло или же отводили.
}
\answer{%
    $
        Q = A_\text{газа} + \Delta U, A_\text{газа} = -A_\text{внешняя}
        \implies Q = A_\text{газа} + \Delta U = - 200\,\text{Дж} - 250\,\text{Дж} = -450\,\text{Дж}.
        \text{ Отводили.}
    $
}

\variantsplitter

\addpersonalvariant{Волохов Вячеслав}

\tasknumber{1}%
\task{%
    Напротив физической величины укажите её обозначение и единицы измерения в СИ или запишите физический закон или формулу (в пункте «г)»):
    \begin{enumerate}
        \item количество теплоты,
        \item работа внешних сил,
        \item молярная теплоёмкость,
        \item внутренняя энергия идеального одноатомного газа.
    \end{enumerate}
}
\solutionspace{20pt}

\tasknumber{2}%
\task{%
    Определите объём идеального одноатомного газа,
    если его внутренняя энергия при давлении $6\,\text{атм}$ составляет $250\,\text{кДж}$.
    $p_{\text{aтм}} = 100\,\text{кПа}$.
}
\answer{%
    $U = \frac 32 \nu R T = \frac 32 PV \implies V = \frac 23 \cdot \frac UP= \frac 23 \cdot \frac{ 250\,\text{кДж} }{ 6\,\text{атм} } \approx 0{,}28\,\text{м}^{3}.$
}
\solutionspace{40pt}

\tasknumber{3}%
\task{%
    Газ расширился от $200\,\text{л}$ до $650\,\text{л}$.
    Давление газа при этом оставалось постоянным и равным $1{,}5\,\text{атм}$.
    Определите работу газа, ответ выразите в килоджоулях.
    $p_{\text{aтм}} = 100\,\text{кПа}$.
}
\answer{%
    $A = P\Delta V = P(V_2 - V_1) = 1{,}5\,\text{атм} \cdot\cbr{650\,\text{л} - 200\,\text{л}} = 6{,}75\,\text{кДж}.$
}
\solutionspace{40pt}

\tasknumber{4}%
\task{%
    $40\,\text{моль}$ идеального одноатомного в результате адиабатического процесса остыл на $15\,\text{К}$.
    Определите работу газа.
    Кто совершил положительную работу: газ или внешние силы?
    Универсальная газовая постоянная $R = 8{,}31\,\frac{\text{Дж}}{\text{моль}\cdot\text{К}}$.
}
\answer{%
    \begin{align*}
    Q &= 0, Q = \Delta U + A_\text{газа} \implies \\
    \implies A_\text{газа} &= - \Delta U = - \frac 32 \nu R \Delta T =  \frac 32 \cdot 40\,\text{моль} \cdot8{,}31\,\frac{\text{Дж}}{\text{моль}\cdot\text{К}} \cdot15\,\text{К}= 7{,}5\,\text{кДж}, \text{газ.}
    \end{align*}
}
\solutionspace{40pt}

\tasknumber{5}%
\task{%
    Как изменилась внутренняя энергия одноатомного идеального газа при переходе из состояния 1 в состояние 2?
    $P_1 = 4\,\text{МПа}$, $V_1 = 5\,\text{л}$, $P_2 = 3{,}5\,\text{МПа}$, $V_2 = 6\,\text{л}$.
    Как изменилась при этом температура газа?
}
\answer{%
    \begin{align*}
    P_1V_1 &= \nu R T_1, P_2V_2 = \nu R T_2, \\
    \Delta U &= U_2-U_1 = \frac 32 \nu R T_2- \frac 32 \nu R T_1 = \frac 32 P_2 V_2 - \frac 32 P_1 V_1= \frac 32 \cdot \cbr{3{,}5\,\text{МПа} \cdot6\,\text{л} - 4\,\text{МПа} \cdot5\,\text{л}} = 1500\,\text{Дж}.
    \\
    \frac{T_2}{T_1} &= \frac{\frac{P_2V_2}{\nu R}}{\frac{P_1V_1}{\nu R}} = \frac{P_2V_2}{P_1V_1}= \frac{3{,}5\,\text{МПа} \cdot6\,\text{л}}{4\,\text{МПа} \cdot5\,\text{л}} \approx 1{,}05.
    \end{align*}
}
\solutionspace{80pt}

\tasknumber{6}%
\task{%
    $3\,\text{моль}$ идеального одноатомного газа охладили на $10\,\text{К}$.
    Определите изменение внутренней энергии газа.
    Увеличилась она или уменьшилась?
    Универсальная газовая постоянная $R = 8{,}31\,\frac{\text{Дж}}{\text{моль}\cdot\text{К}}$.
}
\answer{%
    $
        \Delta U = \frac 32 \nu R \Delta T
            = - \frac 32 \cdot 3\,\text{моль} \cdot8{,}31\,\frac{\text{Дж}}{\text{моль}\cdot\text{К}} \cdot10\,\text{К}
            = -373\,\text{Дж}.
            \text{Уменьшилась.}
    $
}
\solutionspace{40pt}

\tasknumber{7}%
\task{%
    Газу сообщили некоторое количество теплоты,
    при этом четверть его он потратил на совершение работы,
    одновременно увеличив свою внутреннюю энергию на $2400\,\text{Дж}$.
    Определите работу, совершённую газом.
}
\answer{%
    \begin{align*}
    Q &= A' + \Delta U, A' = \frac 14 Q \implies Q\cdot\cbr{1 - \frac 14} = \Delta U \implies Q = \frac{\Delta U}{1 - \frac 14} = \frac{2400\,\text{Дж}}{1 - \frac 14} \approx 3200\,\text{Дж}.
    \\
    A' &= \frac 14 Q
        = \frac 14 \cdot \frac{\Delta U}{1 - \frac 14}
        = \frac{\Delta U}{4 - 1}
        = \frac{2400\,\text{Дж}}{4 - 1} \approx 800\,\text{Дж}.
    \end{align*}
}
\solutionspace{60pt}

\tasknumber{8}%
\task{%
    В некотором процессе внешние силы совершили над газом работу $100\,\text{Дж}$,
    при этом его внутренняя энергия уменьшилась на $250\,\text{Дж}$.
    Определите количество тепла, переданное при этом процессе газу.
    Явно пропишите, подводили газу тепло или же отводили.
}
\answer{%
    $
        Q = A_\text{газа} + \Delta U, A_\text{газа} = -A_\text{внешняя}
        \implies Q = A_\text{газа} + \Delta U = - 100\,\text{Дж} - 250\,\text{Дж} = -350\,\text{Дж}.
        \text{ Отводили.}
    $
}

\variantsplitter

\addpersonalvariant{Герман Говоров}

\tasknumber{1}%
\task{%
    Напротив физической величины укажите её обозначение и единицы измерения в СИ или запишите физический закон или формулу (в пункте «г)»):
    \begin{enumerate}
        \item изменение внутренней энергии,
        \item работа внешних сил,
        \item удельная теплоёмкость,
        \item первое начало термодинамики.
    \end{enumerate}
}
\solutionspace{20pt}

\tasknumber{2}%
\task{%
    Определите объём идеального одноатомного газа,
    если его внутренняя энергия при давлении $4\,\text{атм}$ составляет $500\,\text{кДж}$.
    $p_{\text{aтм}} = 100\,\text{кПа}$.
}
\answer{%
    $U = \frac 32 \nu R T = \frac 32 PV \implies V = \frac 23 \cdot \frac UP= \frac 23 \cdot \frac{ 500\,\text{кДж} }{ 4\,\text{атм} } \approx 0{,}83\,\text{м}^{3}.$
}
\solutionspace{40pt}

\tasknumber{3}%
\task{%
    Газ расширился от $150\,\text{л}$ до $550\,\text{л}$.
    Давление газа при этом оставалось постоянным и равным $1{,}2\,\text{атм}$.
    Определите работу газа, ответ выразите в килоджоулях.
    $p_{\text{aтм}} = 100\,\text{кПа}$.
}
\answer{%
    $A = P\Delta V = P(V_2 - V_1) = 1{,}2\,\text{атм} \cdot\cbr{550\,\text{л} - 150\,\text{л}} = 4{,}80\,\text{кДж}.$
}
\solutionspace{40pt}

\tasknumber{4}%
\task{%
    $60\,\text{моль}$ идеального одноатомного в результате адиабатического процесса нагрелся на $45\,\text{К}$.
    Определите работу газа.
    Кто совершил положительную работу: газ или внешние силы?
    Универсальная газовая постоянная $R = 8{,}31\,\frac{\text{Дж}}{\text{моль}\cdot\text{К}}$.
}
\answer{%
    \begin{align*}
    Q &= 0, Q = \Delta U + A_\text{газа} \implies \\
    \implies A_\text{газа} &= - \Delta U = - \frac 32 \nu R \Delta T = - \frac 32 \cdot 60\,\text{моль} \cdot8{,}31\,\frac{\text{Дж}}{\text{моль}\cdot\text{К}} \cdot45\,\text{К}= -33{,}70\,\text{кДж}, \text{внешние силы.}
    \end{align*}
}
\solutionspace{40pt}

\tasknumber{5}%
\task{%
    Как изменилась внутренняя энергия одноатомного идеального газа при переходе из состояния 1 в состояние 2?
    $P_1 = 3\,\text{МПа}$, $V_1 = 3\,\text{л}$, $P_2 = 2{,}5\,\text{МПа}$, $V_2 = 4\,\text{л}$.
    Как изменилась при этом температура газа?
}
\answer{%
    \begin{align*}
    P_1V_1 &= \nu R T_1, P_2V_2 = \nu R T_2, \\
    \Delta U &= U_2-U_1 = \frac 32 \nu R T_2- \frac 32 \nu R T_1 = \frac 32 P_2 V_2 - \frac 32 P_1 V_1= \frac 32 \cdot \cbr{2{,}5\,\text{МПа} \cdot4\,\text{л} - 3\,\text{МПа} \cdot3\,\text{л}} = 1500\,\text{Дж}.
    \\
    \frac{T_2}{T_1} &= \frac{\frac{P_2V_2}{\nu R}}{\frac{P_1V_1}{\nu R}} = \frac{P_2V_2}{P_1V_1}= \frac{2{,}5\,\text{МПа} \cdot4\,\text{л}}{3\,\text{МПа} \cdot3\,\text{л}} \approx 1{,}11.
    \end{align*}
}
\solutionspace{80pt}

\tasknumber{6}%
\task{%
    $5\,\text{моль}$ идеального одноатомного газа нагрели на $30\,\text{К}$.
    Определите изменение внутренней энергии газа.
    Увеличилась она или уменьшилась?
    Универсальная газовая постоянная $R = 8{,}31\,\frac{\text{Дж}}{\text{моль}\cdot\text{К}}$.
}
\answer{%
    $
        \Delta U = \frac 32 \nu R \Delta T
            =  \frac 32 \cdot 5\,\text{моль} \cdot8{,}31\,\frac{\text{Дж}}{\text{моль}\cdot\text{К}} \cdot30\,\text{К}
            = 1869\,\text{Дж}.
            \text{Увеличилась.}
    $
}
\solutionspace{40pt}

\tasknumber{7}%
\task{%
    Газу сообщили некоторое количество теплоты,
    при этом четверть его он потратил на совершение работы,
    одновременно увеличив свою внутреннюю энергию на $1500\,\text{Дж}$.
    Определите количество теплоты, сообщённое газу.
}
\answer{%
    \begin{align*}
    Q &= A' + \Delta U, A' = \frac 14 Q \implies Q\cdot\cbr{1 - \frac 14} = \Delta U \implies Q = \frac{\Delta U}{1 - \frac 14} = \frac{1500\,\text{Дж}}{1 - \frac 14} \approx 2000\,\text{Дж}.
    \\
    A' &= \frac 14 Q
        = \frac 14 \cdot \frac{\Delta U}{1 - \frac 14}
        = \frac{\Delta U}{4 - 1}
        = \frac{1500\,\text{Дж}}{4 - 1} \approx 500\,\text{Дж}.
    \end{align*}
}
\solutionspace{60pt}

\tasknumber{8}%
\task{%
    В некотором процессе внешние силы совершили над газом работу $100\,\text{Дж}$,
    при этом его внутренняя энергия уменьшилась на $250\,\text{Дж}$.
    Определите количество тепла, переданное при этом процессе газу.
    Явно пропишите, подводили газу тепло или же отводили.
}
\answer{%
    $
        Q = A_\text{газа} + \Delta U, A_\text{газа} = -A_\text{внешняя}
        \implies Q = A_\text{газа} + \Delta U = - 100\,\text{Дж} - 250\,\text{Дж} = -350\,\text{Дж}.
        \text{ Отводили.}
    $
}

\variantsplitter

\addpersonalvariant{София Журавлева}

\tasknumber{1}%
\task{%
    Напротив физической величины укажите её обозначение и единицы измерения в СИ или запишите физический закон или формулу (в пункте «г)»):
    \begin{enumerate}
        \item количество теплоты,
        \item работа газа,
        \item молярная теплоёмкость,
        \item внутренняя энергия идеального одноатомного газа.
    \end{enumerate}
}
\solutionspace{20pt}

\tasknumber{2}%
\task{%
    Определите объём идеального одноатомного газа,
    если его внутренняя энергия при давлении $3\,\text{атм}$ составляет $500\,\text{кДж}$.
    $p_{\text{aтм}} = 100\,\text{кПа}$.
}
\answer{%
    $U = \frac 32 \nu R T = \frac 32 PV \implies V = \frac 23 \cdot \frac UP= \frac 23 \cdot \frac{ 500\,\text{кДж} }{ 3\,\text{атм} } \approx 1{,}11\,\text{м}^{3}.$
}
\solutionspace{40pt}

\tasknumber{3}%
\task{%
    Газ расширился от $200\,\text{л}$ до $650\,\text{л}$.
    Давление газа при этом оставалось постоянным и равным $3{,}5\,\text{атм}$.
    Определите работу газа, ответ выразите в килоджоулях.
    $p_{\text{aтм}} = 100\,\text{кПа}$.
}
\answer{%
    $A = P\Delta V = P(V_2 - V_1) = 3{,}5\,\text{атм} \cdot\cbr{650\,\text{л} - 200\,\text{л}} = 15{,}75\,\text{кДж}.$
}
\solutionspace{40pt}

\tasknumber{4}%
\task{%
    $30\,\text{моль}$ идеального одноатомного в результате адиабатического процесса остыл на $80\,\text{К}$.
    Определите работу газа.
    Кто совершил положительную работу: газ или внешние силы?
    Универсальная газовая постоянная $R = 8{,}31\,\frac{\text{Дж}}{\text{моль}\cdot\text{К}}$.
}
\answer{%
    \begin{align*}
    Q &= 0, Q = \Delta U + A_\text{газа} \implies \\
    \implies A_\text{газа} &= - \Delta U = - \frac 32 \nu R \Delta T =  \frac 32 \cdot 30\,\text{моль} \cdot8{,}31\,\frac{\text{Дж}}{\text{моль}\cdot\text{К}} \cdot80\,\text{К}= 29{,}9\,\text{кДж}, \text{газ.}
    \end{align*}
}
\solutionspace{40pt}

\tasknumber{5}%
\task{%
    Как изменилась внутренняя энергия одноатомного идеального газа при переходе из состояния 1 в состояние 2?
    $P_1 = 3\,\text{МПа}$, $V_1 = 7\,\text{л}$, $P_2 = 1{,}5\,\text{МПа}$, $V_2 = 8\,\text{л}$.
    Как изменилась при этом температура газа?
}
\answer{%
    \begin{align*}
    P_1V_1 &= \nu R T_1, P_2V_2 = \nu R T_2, \\
    \Delta U &= U_2-U_1 = \frac 32 \nu R T_2- \frac 32 \nu R T_1 = \frac 32 P_2 V_2 - \frac 32 P_1 V_1= \frac 32 \cdot \cbr{1{,}5\,\text{МПа} \cdot8\,\text{л} - 3\,\text{МПа} \cdot7\,\text{л}} = -13500\,\text{Дж}.
    \\
    \frac{T_2}{T_1} &= \frac{\frac{P_2V_2}{\nu R}}{\frac{P_1V_1}{\nu R}} = \frac{P_2V_2}{P_1V_1}= \frac{1{,}5\,\text{МПа} \cdot8\,\text{л}}{3\,\text{МПа} \cdot7\,\text{л}} \approx 0{,}57.
    \end{align*}
}
\solutionspace{80pt}

\tasknumber{6}%
\task{%
    $2\,\text{моль}$ идеального одноатомного газа охладили на $20\,\text{К}$.
    Определите изменение внутренней энергии газа.
    Увеличилась она или уменьшилась?
    Универсальная газовая постоянная $R = 8{,}31\,\frac{\text{Дж}}{\text{моль}\cdot\text{К}}$.
}
\answer{%
    $
        \Delta U = \frac 32 \nu R \Delta T
            = - \frac 32 \cdot 2\,\text{моль} \cdot8{,}31\,\frac{\text{Дж}}{\text{моль}\cdot\text{К}} \cdot20\,\text{К}
            = -498\,\text{Дж}.
            \text{Уменьшилась.}
    $
}
\solutionspace{40pt}

\tasknumber{7}%
\task{%
    Газу сообщили некоторое количество теплоты,
    при этом половину его он потратил на совершение работы,
    одновременно увеличив свою внутреннюю энергию на $2400\,\text{Дж}$.
    Определите работу, совершённую газом.
}
\answer{%
    \begin{align*}
    Q &= A' + \Delta U, A' = \frac 12 Q \implies Q\cdot\cbr{1 - \frac 12} = \Delta U \implies Q = \frac{\Delta U}{1 - \frac 12} = \frac{2400\,\text{Дж}}{1 - \frac 12} \approx 4800\,\text{Дж}.
    \\
    A' &= \frac 12 Q
        = \frac 12 \cdot \frac{\Delta U}{1 - \frac 12}
        = \frac{\Delta U}{2 - 1}
        = \frac{2400\,\text{Дж}}{2 - 1} \approx 2400\,\text{Дж}.
    \end{align*}
}
\solutionspace{60pt}

\tasknumber{8}%
\task{%
    В некотором процессе внешние силы совершили над газом работу $200\,\text{Дж}$,
    при этом его внутренняя энергия уменьшилась на $150\,\text{Дж}$.
    Определите количество тепла, переданное при этом процессе газу.
    Явно пропишите, подводили газу тепло или же отводили.
}
\answer{%
    $
        Q = A_\text{газа} + \Delta U, A_\text{газа} = -A_\text{внешняя}
        \implies Q = A_\text{газа} + \Delta U = - 200\,\text{Дж} - 150\,\text{Дж} = -350\,\text{Дж}.
        \text{ Отводили.}
    $
}

\variantsplitter

\addpersonalvariant{Константин Козлов}

\tasknumber{1}%
\task{%
    Напротив физической величины укажите её обозначение и единицы измерения в СИ или запишите физический закон или формулу (в пункте «г)»):
    \begin{enumerate}
        \item изменение внутренней энергии,
        \item работа газа,
        \item молярная теплоёмкость,
        \item внутренняя энергия идеального одноатомного газа.
    \end{enumerate}
}
\solutionspace{20pt}

\tasknumber{2}%
\task{%
    Определите объём идеального одноатомного газа,
    если его внутренняя энергия при давлении $5\,\text{атм}$ составляет $500\,\text{кДж}$.
    $p_{\text{aтм}} = 100\,\text{кПа}$.
}
\answer{%
    $U = \frac 32 \nu R T = \frac 32 PV \implies V = \frac 23 \cdot \frac UP= \frac 23 \cdot \frac{ 500\,\text{кДж} }{ 5\,\text{атм} } \approx 0{,}67\,\text{м}^{3}.$
}
\solutionspace{40pt}

\tasknumber{3}%
\task{%
    Газ расширился от $350\,\text{л}$ до $450\,\text{л}$.
    Давление газа при этом оставалось постоянным и равным $1{,}8\,\text{атм}$.
    Определите работу газа, ответ выразите в килоджоулях.
    $p_{\text{aтм}} = 100\,\text{кПа}$.
}
\answer{%
    $A = P\Delta V = P(V_2 - V_1) = 1{,}8\,\text{атм} \cdot\cbr{450\,\text{л} - 350\,\text{л}} = 1{,}80\,\text{кДж}.$
}
\solutionspace{40pt}

\tasknumber{4}%
\task{%
    $30\,\text{моль}$ идеального одноатомного в результате адиабатического процесса нагрелся на $15\,\text{К}$.
    Определите работу газа.
    Кто совершил положительную работу: газ или внешние силы?
    Универсальная газовая постоянная $R = 8{,}31\,\frac{\text{Дж}}{\text{моль}\cdot\text{К}}$.
}
\answer{%
    \begin{align*}
    Q &= 0, Q = \Delta U + A_\text{газа} \implies \\
    \implies A_\text{газа} &= - \Delta U = - \frac 32 \nu R \Delta T = - \frac 32 \cdot 30\,\text{моль} \cdot8{,}31\,\frac{\text{Дж}}{\text{моль}\cdot\text{К}} \cdot15\,\text{К}= -5{,}60\,\text{кДж}, \text{внешние силы.}
    \end{align*}
}
\solutionspace{40pt}

\tasknumber{5}%
\task{%
    Как изменилась внутренняя энергия одноатомного идеального газа при переходе из состояния 1 в состояние 2?
    $P_1 = 4\,\text{МПа}$, $V_1 = 3\,\text{л}$, $P_2 = 1{,}5\,\text{МПа}$, $V_2 = 8\,\text{л}$.
    Как изменилась при этом температура газа?
}
\answer{%
    \begin{align*}
    P_1V_1 &= \nu R T_1, P_2V_2 = \nu R T_2, \\
    \Delta U &= U_2-U_1 = \frac 32 \nu R T_2- \frac 32 \nu R T_1 = \frac 32 P_2 V_2 - \frac 32 P_1 V_1= \frac 32 \cdot \cbr{1{,}5\,\text{МПа} \cdot8\,\text{л} - 4\,\text{МПа} \cdot3\,\text{л}} = 0\,\text{Дж}.
    \\
    \frac{T_2}{T_1} &= \frac{\frac{P_2V_2}{\nu R}}{\frac{P_1V_1}{\nu R}} = \frac{P_2V_2}{P_1V_1}= \frac{1{,}5\,\text{МПа} \cdot8\,\text{л}}{4\,\text{МПа} \cdot3\,\text{л}} \approx 1{,}00.
    \end{align*}
}
\solutionspace{80pt}

\tasknumber{6}%
\task{%
    $2\,\text{моль}$ идеального одноатомного газа нагрели на $10\,\text{К}$.
    Определите изменение внутренней энергии газа.
    Увеличилась она или уменьшилась?
    Универсальная газовая постоянная $R = 8{,}31\,\frac{\text{Дж}}{\text{моль}\cdot\text{К}}$.
}
\answer{%
    $
        \Delta U = \frac 32 \nu R \Delta T
            =  \frac 32 \cdot 2\,\text{моль} \cdot8{,}31\,\frac{\text{Дж}}{\text{моль}\cdot\text{К}} \cdot10\,\text{К}
            = 249\,\text{Дж}.
            \text{Увеличилась.}
    $
}
\solutionspace{40pt}

\tasknumber{7}%
\task{%
    Газу сообщили некоторое количество теплоты,
    при этом половину его он потратил на совершение работы,
    одновременно увеличив свою внутреннюю энергию на $1500\,\text{Дж}$.
    Определите количество теплоты, сообщённое газу.
}
\answer{%
    \begin{align*}
    Q &= A' + \Delta U, A' = \frac 12 Q \implies Q\cdot\cbr{1 - \frac 12} = \Delta U \implies Q = \frac{\Delta U}{1 - \frac 12} = \frac{1500\,\text{Дж}}{1 - \frac 12} \approx 3000\,\text{Дж}.
    \\
    A' &= \frac 12 Q
        = \frac 12 \cdot \frac{\Delta U}{1 - \frac 12}
        = \frac{\Delta U}{2 - 1}
        = \frac{1500\,\text{Дж}}{2 - 1} \approx 1500\,\text{Дж}.
    \end{align*}
}
\solutionspace{60pt}

\tasknumber{8}%
\task{%
    В некотором процессе внешние силы совершили над газом работу $200\,\text{Дж}$,
    при этом его внутренняя энергия увеличилась на $250\,\text{Дж}$.
    Определите количество тепла, переданное при этом процессе газу.
    Явно пропишите, подводили газу тепло или же отводили.
}
\answer{%
    $
        Q = A_\text{газа} + \Delta U, A_\text{газа} = -A_\text{внешняя}
        \implies Q = A_\text{газа} + \Delta U = - 200\,\text{Дж} +  250\,\text{Дж} = 50\,\text{Дж}.
        \text{ Подводили.}
    $
}

\variantsplitter

\addpersonalvariant{Наталья Кравченко}

\tasknumber{1}%
\task{%
    Напротив физической величины укажите её обозначение и единицы измерения в СИ или запишите физический закон или формулу (в пункте «г)»):
    \begin{enumerate}
        \item количество теплоты,
        \item работа внешних сил,
        \item удельная теплоёмкость,
        \item внутренняя энергия идеального одноатомного газа.
    \end{enumerate}
}
\solutionspace{20pt}

\tasknumber{2}%
\task{%
    Определите объём идеального одноатомного газа,
    если его внутренняя энергия при давлении $3\,\text{атм}$ составляет $400\,\text{кДж}$.
    $p_{\text{aтм}} = 100\,\text{кПа}$.
}
\answer{%
    $U = \frac 32 \nu R T = \frac 32 PV \implies V = \frac 23 \cdot \frac UP= \frac 23 \cdot \frac{ 400\,\text{кДж} }{ 3\,\text{атм} } \approx 0{,}89\,\text{м}^{3}.$
}
\solutionspace{40pt}

\tasknumber{3}%
\task{%
    Газ расширился от $350\,\text{л}$ до $450\,\text{л}$.
    Давление газа при этом оставалось постоянным и равным $3{,}5\,\text{атм}$.
    Определите работу газа, ответ выразите в килоджоулях.
    $p_{\text{aтм}} = 100\,\text{кПа}$.
}
\answer{%
    $A = P\Delta V = P(V_2 - V_1) = 3{,}5\,\text{атм} \cdot\cbr{450\,\text{л} - 350\,\text{л}} = 3{,}50\,\text{кДж}.$
}
\solutionspace{40pt}

\tasknumber{4}%
\task{%
    $30\,\text{моль}$ идеального одноатомного в результате адиабатического процесса нагрелся на $25\,\text{К}$.
    Определите работу газа.
    Кто совершил положительную работу: газ или внешние силы?
    Универсальная газовая постоянная $R = 8{,}31\,\frac{\text{Дж}}{\text{моль}\cdot\text{К}}$.
}
\answer{%
    \begin{align*}
    Q &= 0, Q = \Delta U + A_\text{газа} \implies \\
    \implies A_\text{газа} &= - \Delta U = - \frac 32 \nu R \Delta T = - \frac 32 \cdot 30\,\text{моль} \cdot8{,}31\,\frac{\text{Дж}}{\text{моль}\cdot\text{К}} \cdot25\,\text{К}= -9{,}30\,\text{кДж}, \text{внешние силы.}
    \end{align*}
}
\solutionspace{40pt}

\tasknumber{5}%
\task{%
    Как изменилась внутренняя энергия одноатомного идеального газа при переходе из состояния 1 в состояние 2?
    $P_1 = 4\,\text{МПа}$, $V_1 = 7\,\text{л}$, $P_2 = 2{,}5\,\text{МПа}$, $V_2 = 4\,\text{л}$.
    Как изменилась при этом температура газа?
}
\answer{%
    \begin{align*}
    P_1V_1 &= \nu R T_1, P_2V_2 = \nu R T_2, \\
    \Delta U &= U_2-U_1 = \frac 32 \nu R T_2- \frac 32 \nu R T_1 = \frac 32 P_2 V_2 - \frac 32 P_1 V_1= \frac 32 \cdot \cbr{2{,}5\,\text{МПа} \cdot4\,\text{л} - 4\,\text{МПа} \cdot7\,\text{л}} = -27000\,\text{Дж}.
    \\
    \frac{T_2}{T_1} &= \frac{\frac{P_2V_2}{\nu R}}{\frac{P_1V_1}{\nu R}} = \frac{P_2V_2}{P_1V_1}= \frac{2{,}5\,\text{МПа} \cdot4\,\text{л}}{4\,\text{МПа} \cdot7\,\text{л}} \approx 0{,}36.
    \end{align*}
}
\solutionspace{80pt}

\tasknumber{6}%
\task{%
    $2\,\text{моль}$ идеального одноатомного газа нагрели на $20\,\text{К}$.
    Определите изменение внутренней энергии газа.
    Увеличилась она или уменьшилась?
    Универсальная газовая постоянная $R = 8{,}31\,\frac{\text{Дж}}{\text{моль}\cdot\text{К}}$.
}
\answer{%
    $
        \Delta U = \frac 32 \nu R \Delta T
            =  \frac 32 \cdot 2\,\text{моль} \cdot8{,}31\,\frac{\text{Дж}}{\text{моль}\cdot\text{К}} \cdot20\,\text{К}
            = 498\,\text{Дж}.
            \text{Увеличилась.}
    $
}
\solutionspace{40pt}

\tasknumber{7}%
\task{%
    Газу сообщили некоторое количество теплоты,
    при этом четверть его он потратил на совершение работы,
    одновременно увеличив свою внутреннюю энергию на $2400\,\text{Дж}$.
    Определите количество теплоты, сообщённое газу.
}
\answer{%
    \begin{align*}
    Q &= A' + \Delta U, A' = \frac 14 Q \implies Q\cdot\cbr{1 - \frac 14} = \Delta U \implies Q = \frac{\Delta U}{1 - \frac 14} = \frac{2400\,\text{Дж}}{1 - \frac 14} \approx 3200\,\text{Дж}.
    \\
    A' &= \frac 14 Q
        = \frac 14 \cdot \frac{\Delta U}{1 - \frac 14}
        = \frac{\Delta U}{4 - 1}
        = \frac{2400\,\text{Дж}}{4 - 1} \approx 800\,\text{Дж}.
    \end{align*}
}
\solutionspace{60pt}

\tasknumber{8}%
\task{%
    В некотором процессе внешние силы совершили над газом работу $300\,\text{Дж}$,
    при этом его внутренняя энергия уменьшилась на $450\,\text{Дж}$.
    Определите количество тепла, переданное при этом процессе газу.
    Явно пропишите, подводили газу тепло или же отводили.
}
\answer{%
    $
        Q = A_\text{газа} + \Delta U, A_\text{газа} = -A_\text{внешняя}
        \implies Q = A_\text{газа} + \Delta U = - 300\,\text{Дж} - 450\,\text{Дж} = -750\,\text{Дж}.
        \text{ Отводили.}
    $
}

\variantsplitter

\addpersonalvariant{Матвей Кузьмин}

\tasknumber{1}%
\task{%
    Напротив физической величины укажите её обозначение и единицы измерения в СИ или запишите физический закон или формулу (в пункте «г)»):
    \begin{enumerate}
        \item изменение внутренней энергии,
        \item работа внешних сил,
        \item молярная теплоёмкость,
        \item внутренняя энергия идеального одноатомного газа.
    \end{enumerate}
}
\solutionspace{20pt}

\tasknumber{2}%
\task{%
    Определите объём идеального одноатомного газа,
    если его внутренняя энергия при давлении $6\,\text{атм}$ составляет $300\,\text{кДж}$.
    $p_{\text{aтм}} = 100\,\text{кПа}$.
}
\answer{%
    $U = \frac 32 \nu R T = \frac 32 PV \implies V = \frac 23 \cdot \frac UP= \frac 23 \cdot \frac{ 300\,\text{кДж} }{ 6\,\text{атм} } \approx 0{,}33\,\text{м}^{3}.$
}
\solutionspace{40pt}

\tasknumber{3}%
\task{%
    Газ расширился от $200\,\text{л}$ до $450\,\text{л}$.
    Давление газа при этом оставалось постоянным и равным $2{,}5\,\text{атм}$.
    Определите работу газа, ответ выразите в килоджоулях.
    $p_{\text{aтм}} = 100\,\text{кПа}$.
}
\answer{%
    $A = P\Delta V = P(V_2 - V_1) = 2{,}5\,\text{атм} \cdot\cbr{450\,\text{л} - 200\,\text{л}} = 6{,}25\,\text{кДж}.$
}
\solutionspace{40pt}

\tasknumber{4}%
\task{%
    $30\,\text{моль}$ идеального одноатомного в результате адиабатического процесса остыл на $80\,\text{К}$.
    Определите работу газа.
    Кто совершил положительную работу: газ или внешние силы?
    Универсальная газовая постоянная $R = 8{,}31\,\frac{\text{Дж}}{\text{моль}\cdot\text{К}}$.
}
\answer{%
    \begin{align*}
    Q &= 0, Q = \Delta U + A_\text{газа} \implies \\
    \implies A_\text{газа} &= - \Delta U = - \frac 32 \nu R \Delta T =  \frac 32 \cdot 30\,\text{моль} \cdot8{,}31\,\frac{\text{Дж}}{\text{моль}\cdot\text{К}} \cdot80\,\text{К}= 29{,}9\,\text{кДж}, \text{газ.}
    \end{align*}
}
\solutionspace{40pt}

\tasknumber{5}%
\task{%
    Как изменилась внутренняя энергия одноатомного идеального газа при переходе из состояния 1 в состояние 2?
    $P_1 = 4\,\text{МПа}$, $V_1 = 3\,\text{л}$, $P_2 = 3{,}5\,\text{МПа}$, $V_2 = 2\,\text{л}$.
    Как изменилась при этом температура газа?
}
\answer{%
    \begin{align*}
    P_1V_1 &= \nu R T_1, P_2V_2 = \nu R T_2, \\
    \Delta U &= U_2-U_1 = \frac 32 \nu R T_2- \frac 32 \nu R T_1 = \frac 32 P_2 V_2 - \frac 32 P_1 V_1= \frac 32 \cdot \cbr{3{,}5\,\text{МПа} \cdot2\,\text{л} - 4\,\text{МПа} \cdot3\,\text{л}} = -7500\,\text{Дж}.
    \\
    \frac{T_2}{T_1} &= \frac{\frac{P_2V_2}{\nu R}}{\frac{P_1V_1}{\nu R}} = \frac{P_2V_2}{P_1V_1}= \frac{3{,}5\,\text{МПа} \cdot2\,\text{л}}{4\,\text{МПа} \cdot3\,\text{л}} \approx 0{,}58.
    \end{align*}
}
\solutionspace{80pt}

\tasknumber{6}%
\task{%
    $2\,\text{моль}$ идеального одноатомного газа охладили на $20\,\text{К}$.
    Определите изменение внутренней энергии газа.
    Увеличилась она или уменьшилась?
    Универсальная газовая постоянная $R = 8{,}31\,\frac{\text{Дж}}{\text{моль}\cdot\text{К}}$.
}
\answer{%
    $
        \Delta U = \frac 32 \nu R \Delta T
            = - \frac 32 \cdot 2\,\text{моль} \cdot8{,}31\,\frac{\text{Дж}}{\text{моль}\cdot\text{К}} \cdot20\,\text{К}
            = -498\,\text{Дж}.
            \text{Уменьшилась.}
    $
}
\solutionspace{40pt}

\tasknumber{7}%
\task{%
    Газу сообщили некоторое количество теплоты,
    при этом треть его он потратил на совершение работы,
    одновременно увеличив свою внутреннюю энергию на $1200\,\text{Дж}$.
    Определите количество теплоты, сообщённое газу.
}
\answer{%
    \begin{align*}
    Q &= A' + \Delta U, A' = \frac 13 Q \implies Q\cdot\cbr{1 - \frac 13} = \Delta U \implies Q = \frac{\Delta U}{1 - \frac 13} = \frac{1200\,\text{Дж}}{1 - \frac 13} \approx 1800\,\text{Дж}.
    \\
    A' &= \frac 13 Q
        = \frac 13 \cdot \frac{\Delta U}{1 - \frac 13}
        = \frac{\Delta U}{3 - 1}
        = \frac{1200\,\text{Дж}}{3 - 1} \approx 600\,\text{Дж}.
    \end{align*}
}
\solutionspace{60pt}

\tasknumber{8}%
\task{%
    В некотором процессе газ совершил работу $200\,\text{Дж}$,
    при этом его внутренняя энергия увеличилась на $450\,\text{Дж}$.
    Определите количество тепла, переданное при этом процессе газу.
    Явно пропишите, подводили газу тепло или же отводили.
}
\answer{%
    $
        Q = A_\text{газа} + \Delta U, A_\text{газа} = -A_\text{внешняя}
        \implies Q = A_\text{газа} + \Delta U =  200\,\text{Дж} +  450\,\text{Дж} = 650\,\text{Дж}.
        \text{ Подводили.}
    $
}

\variantsplitter

\addpersonalvariant{Сергей Малышев}

\tasknumber{1}%
\task{%
    Напротив физической величины укажите её обозначение и единицы измерения в СИ или запишите физический закон или формулу (в пункте «г)»):
    \begin{enumerate}
        \item изменение внутренней энергии,
        \item работа внешних сил,
        \item удельная теплоёмкость,
        \item первое начало термодинамики.
    \end{enumerate}
}
\solutionspace{20pt}

\tasknumber{2}%
\task{%
    Определите объём идеального одноатомного газа,
    если его внутренняя энергия при давлении $2\,\text{атм}$ составляет $250\,\text{кДж}$.
    $p_{\text{aтм}} = 100\,\text{кПа}$.
}
\answer{%
    $U = \frac 32 \nu R T = \frac 32 PV \implies V = \frac 23 \cdot \frac UP= \frac 23 \cdot \frac{ 250\,\text{кДж} }{ 2\,\text{атм} } \approx 0{,}83\,\text{м}^{3}.$
}
\solutionspace{40pt}

\tasknumber{3}%
\task{%
    Газ расширился от $350\,\text{л}$ до $550\,\text{л}$.
    Давление газа при этом оставалось постоянным и равным $1{,}8\,\text{атм}$.
    Определите работу газа, ответ выразите в килоджоулях.
    $p_{\text{aтм}} = 100\,\text{кПа}$.
}
\answer{%
    $A = P\Delta V = P(V_2 - V_1) = 1{,}8\,\text{атм} \cdot\cbr{550\,\text{л} - 350\,\text{л}} = 3{,}60\,\text{кДж}.$
}
\solutionspace{40pt}

\tasknumber{4}%
\task{%
    $40\,\text{моль}$ идеального одноатомного в результате адиабатического процесса нагрелся на $15\,\text{К}$.
    Определите работу газа.
    Кто совершил положительную работу: газ или внешние силы?
    Универсальная газовая постоянная $R = 8{,}31\,\frac{\text{Дж}}{\text{моль}\cdot\text{К}}$.
}
\answer{%
    \begin{align*}
    Q &= 0, Q = \Delta U + A_\text{газа} \implies \\
    \implies A_\text{газа} &= - \Delta U = - \frac 32 \nu R \Delta T = - \frac 32 \cdot 40\,\text{моль} \cdot8{,}31\,\frac{\text{Дж}}{\text{моль}\cdot\text{К}} \cdot15\,\text{К}= -7{,}50\,\text{кДж}, \text{внешние силы.}
    \end{align*}
}
\solutionspace{40pt}

\tasknumber{5}%
\task{%
    Как изменилась внутренняя энергия одноатомного идеального газа при переходе из состояния 1 в состояние 2?
    $P_1 = 4\,\text{МПа}$, $V_1 = 3\,\text{л}$, $P_2 = 2{,}5\,\text{МПа}$, $V_2 = 2\,\text{л}$.
    Как изменилась при этом температура газа?
}
\answer{%
    \begin{align*}
    P_1V_1 &= \nu R T_1, P_2V_2 = \nu R T_2, \\
    \Delta U &= U_2-U_1 = \frac 32 \nu R T_2- \frac 32 \nu R T_1 = \frac 32 P_2 V_2 - \frac 32 P_1 V_1= \frac 32 \cdot \cbr{2{,}5\,\text{МПа} \cdot2\,\text{л} - 4\,\text{МПа} \cdot3\,\text{л}} = -10500\,\text{Дж}.
    \\
    \frac{T_2}{T_1} &= \frac{\frac{P_2V_2}{\nu R}}{\frac{P_1V_1}{\nu R}} = \frac{P_2V_2}{P_1V_1}= \frac{2{,}5\,\text{МПа} \cdot2\,\text{л}}{4\,\text{МПа} \cdot3\,\text{л}} \approx 0{,}42.
    \end{align*}
}
\solutionspace{80pt}

\tasknumber{6}%
\task{%
    $3\,\text{моль}$ идеального одноатомного газа нагрели на $10\,\text{К}$.
    Определите изменение внутренней энергии газа.
    Увеличилась она или уменьшилась?
    Универсальная газовая постоянная $R = 8{,}31\,\frac{\text{Дж}}{\text{моль}\cdot\text{К}}$.
}
\answer{%
    $
        \Delta U = \frac 32 \nu R \Delta T
            =  \frac 32 \cdot 3\,\text{моль} \cdot8{,}31\,\frac{\text{Дж}}{\text{моль}\cdot\text{К}} \cdot10\,\text{К}
            = 373\,\text{Дж}.
            \text{Увеличилась.}
    $
}
\solutionspace{40pt}

\tasknumber{7}%
\task{%
    Газу сообщили некоторое количество теплоты,
    при этом четверть его он потратил на совершение работы,
    одновременно увеличив свою внутреннюю энергию на $3000\,\text{Дж}$.
    Определите количество теплоты, сообщённое газу.
}
\answer{%
    \begin{align*}
    Q &= A' + \Delta U, A' = \frac 14 Q \implies Q\cdot\cbr{1 - \frac 14} = \Delta U \implies Q = \frac{\Delta U}{1 - \frac 14} = \frac{3000\,\text{Дж}}{1 - \frac 14} \approx 4000\,\text{Дж}.
    \\
    A' &= \frac 14 Q
        = \frac 14 \cdot \frac{\Delta U}{1 - \frac 14}
        = \frac{\Delta U}{4 - 1}
        = \frac{3000\,\text{Дж}}{4 - 1} \approx 1000\,\text{Дж}.
    \end{align*}
}
\solutionspace{60pt}

\tasknumber{8}%
\task{%
    В некотором процессе газ совершил работу $100\,\text{Дж}$,
    при этом его внутренняя энергия увеличилась на $350\,\text{Дж}$.
    Определите количество тепла, переданное при этом процессе газу.
    Явно пропишите, подводили газу тепло или же отводили.
}
\answer{%
    $
        Q = A_\text{газа} + \Delta U, A_\text{газа} = -A_\text{внешняя}
        \implies Q = A_\text{газа} + \Delta U =  100\,\text{Дж} +  350\,\text{Дж} = 450\,\text{Дж}.
        \text{ Подводили.}
    $
}

\variantsplitter

\addpersonalvariant{Алина Полканова}

\tasknumber{1}%
\task{%
    Напротив физической величины укажите её обозначение и единицы измерения в СИ или запишите физический закон или формулу (в пункте «г)»):
    \begin{enumerate}
        \item количество теплоты,
        \item работа внешних сил,
        \item удельная теплоёмкость,
        \item внутренняя энергия идеального одноатомного газа.
    \end{enumerate}
}
\solutionspace{20pt}

\tasknumber{2}%
\task{%
    Определите объём идеального одноатомного газа,
    если его внутренняя энергия при давлении $4\,\text{атм}$ составляет $300\,\text{кДж}$.
    $p_{\text{aтм}} = 100\,\text{кПа}$.
}
\answer{%
    $U = \frac 32 \nu R T = \frac 32 PV \implies V = \frac 23 \cdot \frac UP= \frac 23 \cdot \frac{ 300\,\text{кДж} }{ 4\,\text{атм} } \approx 0{,}50\,\text{м}^{3}.$
}
\solutionspace{40pt}

\tasknumber{3}%
\task{%
    Газ расширился от $200\,\text{л}$ до $550\,\text{л}$.
    Давление газа при этом оставалось постоянным и равным $1{,}2\,\text{атм}$.
    Определите работу газа, ответ выразите в килоджоулях.
    $p_{\text{aтм}} = 100\,\text{кПа}$.
}
\answer{%
    $A = P\Delta V = P(V_2 - V_1) = 1{,}2\,\text{атм} \cdot\cbr{550\,\text{л} - 200\,\text{л}} = 4{,}20\,\text{кДж}.$
}
\solutionspace{40pt}

\tasknumber{4}%
\task{%
    $50\,\text{моль}$ идеального одноатомного в результате адиабатического процесса нагрелся на $25\,\text{К}$.
    Определите работу газа.
    Кто совершил положительную работу: газ или внешние силы?
    Универсальная газовая постоянная $R = 8{,}31\,\frac{\text{Дж}}{\text{моль}\cdot\text{К}}$.
}
\answer{%
    \begin{align*}
    Q &= 0, Q = \Delta U + A_\text{газа} \implies \\
    \implies A_\text{газа} &= - \Delta U = - \frac 32 \nu R \Delta T = - \frac 32 \cdot 50\,\text{моль} \cdot8{,}31\,\frac{\text{Дж}}{\text{моль}\cdot\text{К}} \cdot25\,\text{К}= -15{,}600\,\text{кДж}, \text{внешние силы.}
    \end{align*}
}
\solutionspace{40pt}

\tasknumber{5}%
\task{%
    Как изменилась внутренняя энергия одноатомного идеального газа при переходе из состояния 1 в состояние 2?
    $P_1 = 2\,\text{МПа}$, $V_1 = 3\,\text{л}$, $P_2 = 4{,}5\,\text{МПа}$, $V_2 = 6\,\text{л}$.
    Как изменилась при этом температура газа?
}
\answer{%
    \begin{align*}
    P_1V_1 &= \nu R T_1, P_2V_2 = \nu R T_2, \\
    \Delta U &= U_2-U_1 = \frac 32 \nu R T_2- \frac 32 \nu R T_1 = \frac 32 P_2 V_2 - \frac 32 P_1 V_1= \frac 32 \cdot \cbr{4{,}5\,\text{МПа} \cdot6\,\text{л} - 2\,\text{МПа} \cdot3\,\text{л}} = 31500\,\text{Дж}.
    \\
    \frac{T_2}{T_1} &= \frac{\frac{P_2V_2}{\nu R}}{\frac{P_1V_1}{\nu R}} = \frac{P_2V_2}{P_1V_1}= \frac{4{,}5\,\text{МПа} \cdot6\,\text{л}}{2\,\text{МПа} \cdot3\,\text{л}} \approx 4{,}50.
    \end{align*}
}
\solutionspace{80pt}

\tasknumber{6}%
\task{%
    $4\,\text{моль}$ идеального одноатомного газа нагрели на $20\,\text{К}$.
    Определите изменение внутренней энергии газа.
    Увеличилась она или уменьшилась?
    Универсальная газовая постоянная $R = 8{,}31\,\frac{\text{Дж}}{\text{моль}\cdot\text{К}}$.
}
\answer{%
    $
        \Delta U = \frac 32 \nu R \Delta T
            =  \frac 32 \cdot 4\,\text{моль} \cdot8{,}31\,\frac{\text{Дж}}{\text{моль}\cdot\text{К}} \cdot20\,\text{К}
            = 997\,\text{Дж}.
            \text{Увеличилась.}
    $
}
\solutionspace{40pt}

\tasknumber{7}%
\task{%
    Газу сообщили некоторое количество теплоты,
    при этом половину его он потратил на совершение работы,
    одновременно увеличив свою внутреннюю энергию на $1500\,\text{Дж}$.
    Определите работу, совершённую газом.
}
\answer{%
    \begin{align*}
    Q &= A' + \Delta U, A' = \frac 12 Q \implies Q\cdot\cbr{1 - \frac 12} = \Delta U \implies Q = \frac{\Delta U}{1 - \frac 12} = \frac{1500\,\text{Дж}}{1 - \frac 12} \approx 3000\,\text{Дж}.
    \\
    A' &= \frac 12 Q
        = \frac 12 \cdot \frac{\Delta U}{1 - \frac 12}
        = \frac{\Delta U}{2 - 1}
        = \frac{1500\,\text{Дж}}{2 - 1} \approx 1500\,\text{Дж}.
    \end{align*}
}
\solutionspace{60pt}

\tasknumber{8}%
\task{%
    В некотором процессе внешние силы совершили над газом работу $300\,\text{Дж}$,
    при этом его внутренняя энергия увеличилась на $350\,\text{Дж}$.
    Определите количество тепла, переданное при этом процессе газу.
    Явно пропишите, подводили газу тепло или же отводили.
}
\answer{%
    $
        Q = A_\text{газа} + \Delta U, A_\text{газа} = -A_\text{внешняя}
        \implies Q = A_\text{газа} + \Delta U = - 300\,\text{Дж} +  350\,\text{Дж} = 50\,\text{Дж}.
        \text{ Подводили.}
    $
}

\variantsplitter

\addpersonalvariant{Сергей Пономарёв}

\tasknumber{1}%
\task{%
    Напротив физической величины укажите её обозначение и единицы измерения в СИ или запишите физический закон или формулу (в пункте «г)»):
    \begin{enumerate}
        \item количество теплоты,
        \item работа внешних сил,
        \item удельная теплоёмкость,
        \item первое начало термодинамики.
    \end{enumerate}
}
\solutionspace{20pt}

\tasknumber{2}%
\task{%
    Определите объём идеального одноатомного газа,
    если его внутренняя энергия при давлении $4\,\text{атм}$ составляет $400\,\text{кДж}$.
    $p_{\text{aтм}} = 100\,\text{кПа}$.
}
\answer{%
    $U = \frac 32 \nu R T = \frac 32 PV \implies V = \frac 23 \cdot \frac UP= \frac 23 \cdot \frac{ 400\,\text{кДж} }{ 4\,\text{атм} } \approx 0{,}67\,\text{м}^{3}.$
}
\solutionspace{40pt}

\tasknumber{3}%
\task{%
    Газ расширился от $250\,\text{л}$ до $550\,\text{л}$.
    Давление газа при этом оставалось постоянным и равным $1{,}8\,\text{атм}$.
    Определите работу газа, ответ выразите в килоджоулях.
    $p_{\text{aтм}} = 100\,\text{кПа}$.
}
\answer{%
    $A = P\Delta V = P(V_2 - V_1) = 1{,}8\,\text{атм} \cdot\cbr{550\,\text{л} - 250\,\text{л}} = 5{,}40\,\text{кДж}.$
}
\solutionspace{40pt}

\tasknumber{4}%
\task{%
    $60\,\text{моль}$ идеального одноатомного в результате адиабатического процесса остыл на $15\,\text{К}$.
    Определите работу газа.
    Кто совершил положительную работу: газ или внешние силы?
    Универсальная газовая постоянная $R = 8{,}31\,\frac{\text{Дж}}{\text{моль}\cdot\text{К}}$.
}
\answer{%
    \begin{align*}
    Q &= 0, Q = \Delta U + A_\text{газа} \implies \\
    \implies A_\text{газа} &= - \Delta U = - \frac 32 \nu R \Delta T =  \frac 32 \cdot 60\,\text{моль} \cdot8{,}31\,\frac{\text{Дж}}{\text{моль}\cdot\text{К}} \cdot15\,\text{К}= 11{,}2\,\text{кДж}, \text{газ.}
    \end{align*}
}
\solutionspace{40pt}

\tasknumber{5}%
\task{%
    Как изменилась внутренняя энергия одноатомного идеального газа при переходе из состояния 1 в состояние 2?
    $P_1 = 2\,\text{МПа}$, $V_1 = 5\,\text{л}$, $P_2 = 1{,}5\,\text{МПа}$, $V_2 = 2\,\text{л}$.
    Как изменилась при этом температура газа?
}
\answer{%
    \begin{align*}
    P_1V_1 &= \nu R T_1, P_2V_2 = \nu R T_2, \\
    \Delta U &= U_2-U_1 = \frac 32 \nu R T_2- \frac 32 \nu R T_1 = \frac 32 P_2 V_2 - \frac 32 P_1 V_1= \frac 32 \cdot \cbr{1{,}5\,\text{МПа} \cdot2\,\text{л} - 2\,\text{МПа} \cdot5\,\text{л}} = -10500\,\text{Дж}.
    \\
    \frac{T_2}{T_1} &= \frac{\frac{P_2V_2}{\nu R}}{\frac{P_1V_1}{\nu R}} = \frac{P_2V_2}{P_1V_1}= \frac{1{,}5\,\text{МПа} \cdot2\,\text{л}}{2\,\text{МПа} \cdot5\,\text{л}} \approx 0{,}30.
    \end{align*}
}
\solutionspace{80pt}

\tasknumber{6}%
\task{%
    $5\,\text{моль}$ идеального одноатомного газа охладили на $10\,\text{К}$.
    Определите изменение внутренней энергии газа.
    Увеличилась она или уменьшилась?
    Универсальная газовая постоянная $R = 8{,}31\,\frac{\text{Дж}}{\text{моль}\cdot\text{К}}$.
}
\answer{%
    $
        \Delta U = \frac 32 \nu R \Delta T
            = - \frac 32 \cdot 5\,\text{моль} \cdot8{,}31\,\frac{\text{Дж}}{\text{моль}\cdot\text{К}} \cdot10\,\text{К}
            = -623\,\text{Дж}.
            \text{Уменьшилась.}
    $
}
\solutionspace{40pt}

\tasknumber{7}%
\task{%
    Газу сообщили некоторое количество теплоты,
    при этом треть его он потратил на совершение работы,
    одновременно увеличив свою внутреннюю энергию на $1200\,\text{Дж}$.
    Определите количество теплоты, сообщённое газу.
}
\answer{%
    \begin{align*}
    Q &= A' + \Delta U, A' = \frac 13 Q \implies Q\cdot\cbr{1 - \frac 13} = \Delta U \implies Q = \frac{\Delta U}{1 - \frac 13} = \frac{1200\,\text{Дж}}{1 - \frac 13} \approx 1800\,\text{Дж}.
    \\
    A' &= \frac 13 Q
        = \frac 13 \cdot \frac{\Delta U}{1 - \frac 13}
        = \frac{\Delta U}{3 - 1}
        = \frac{1200\,\text{Дж}}{3 - 1} \approx 600\,\text{Дж}.
    \end{align*}
}
\solutionspace{60pt}

\tasknumber{8}%
\task{%
    В некотором процессе газ совершил работу $100\,\text{Дж}$,
    при этом его внутренняя энергия увеличилась на $250\,\text{Дж}$.
    Определите количество тепла, переданное при этом процессе газу.
    Явно пропишите, подводили газу тепло или же отводили.
}
\answer{%
    $
        Q = A_\text{газа} + \Delta U, A_\text{газа} = -A_\text{внешняя}
        \implies Q = A_\text{газа} + \Delta U =  100\,\text{Дж} +  250\,\text{Дж} = 350\,\text{Дж}.
        \text{ Подводили.}
    $
}

\variantsplitter

\addpersonalvariant{Егор Свистушкин}

\tasknumber{1}%
\task{%
    Напротив физической величины укажите её обозначение и единицы измерения в СИ или запишите физический закон или формулу (в пункте «г)»):
    \begin{enumerate}
        \item количество теплоты,
        \item работа газа,
        \item удельная теплоёмкость,
        \item внутренняя энергия идеального одноатомного газа.
    \end{enumerate}
}
\solutionspace{20pt}

\tasknumber{2}%
\task{%
    Определите объём идеального одноатомного газа,
    если его внутренняя энергия при давлении $6\,\text{атм}$ составляет $300\,\text{кДж}$.
    $p_{\text{aтм}} = 100\,\text{кПа}$.
}
\answer{%
    $U = \frac 32 \nu R T = \frac 32 PV \implies V = \frac 23 \cdot \frac UP= \frac 23 \cdot \frac{ 300\,\text{кДж} }{ 6\,\text{атм} } \approx 0{,}33\,\text{м}^{3}.$
}
\solutionspace{40pt}

\tasknumber{3}%
\task{%
    Газ расширился от $350\,\text{л}$ до $650\,\text{л}$.
    Давление газа при этом оставалось постоянным и равным $1{,}2\,\text{атм}$.
    Определите работу газа, ответ выразите в килоджоулях.
    $p_{\text{aтм}} = 100\,\text{кПа}$.
}
\answer{%
    $A = P\Delta V = P(V_2 - V_1) = 1{,}2\,\text{атм} \cdot\cbr{650\,\text{л} - 350\,\text{л}} = 3{,}60\,\text{кДж}.$
}
\solutionspace{40pt}

\tasknumber{4}%
\task{%
    $30\,\text{моль}$ идеального одноатомного в результате адиабатического процесса остыл на $120\,\text{К}$.
    Определите работу газа.
    Кто совершил положительную работу: газ или внешние силы?
    Универсальная газовая постоянная $R = 8{,}31\,\frac{\text{Дж}}{\text{моль}\cdot\text{К}}$.
}
\answer{%
    \begin{align*}
    Q &= 0, Q = \Delta U + A_\text{газа} \implies \\
    \implies A_\text{газа} &= - \Delta U = - \frac 32 \nu R \Delta T =  \frac 32 \cdot 30\,\text{моль} \cdot8{,}31\,\frac{\text{Дж}}{\text{моль}\cdot\text{К}} \cdot120\,\text{К}= 44{,}9\,\text{кДж}, \text{газ.}
    \end{align*}
}
\solutionspace{40pt}

\tasknumber{5}%
\task{%
    Как изменилась внутренняя энергия одноатомного идеального газа при переходе из состояния 1 в состояние 2?
    $P_1 = 2\,\text{МПа}$, $V_1 = 5\,\text{л}$, $P_2 = 2{,}5\,\text{МПа}$, $V_2 = 8\,\text{л}$.
    Как изменилась при этом температура газа?
}
\answer{%
    \begin{align*}
    P_1V_1 &= \nu R T_1, P_2V_2 = \nu R T_2, \\
    \Delta U &= U_2-U_1 = \frac 32 \nu R T_2- \frac 32 \nu R T_1 = \frac 32 P_2 V_2 - \frac 32 P_1 V_1= \frac 32 \cdot \cbr{2{,}5\,\text{МПа} \cdot8\,\text{л} - 2\,\text{МПа} \cdot5\,\text{л}} = 15000\,\text{Дж}.
    \\
    \frac{T_2}{T_1} &= \frac{\frac{P_2V_2}{\nu R}}{\frac{P_1V_1}{\nu R}} = \frac{P_2V_2}{P_1V_1}= \frac{2{,}5\,\text{МПа} \cdot8\,\text{л}}{2\,\text{МПа} \cdot5\,\text{л}} \approx 2{,}00.
    \end{align*}
}
\solutionspace{80pt}

\tasknumber{6}%
\task{%
    $2\,\text{моль}$ идеального одноатомного газа охладили на $30\,\text{К}$.
    Определите изменение внутренней энергии газа.
    Увеличилась она или уменьшилась?
    Универсальная газовая постоянная $R = 8{,}31\,\frac{\text{Дж}}{\text{моль}\cdot\text{К}}$.
}
\answer{%
    $
        \Delta U = \frac 32 \nu R \Delta T
            = - \frac 32 \cdot 2\,\text{моль} \cdot8{,}31\,\frac{\text{Дж}}{\text{моль}\cdot\text{К}} \cdot30\,\text{К}
            = -747\,\text{Дж}.
            \text{Уменьшилась.}
    $
}
\solutionspace{40pt}

\tasknumber{7}%
\task{%
    Газу сообщили некоторое количество теплоты,
    при этом половину его он потратил на совершение работы,
    одновременно увеличив свою внутреннюю энергию на $1500\,\text{Дж}$.
    Определите количество теплоты, сообщённое газу.
}
\answer{%
    \begin{align*}
    Q &= A' + \Delta U, A' = \frac 12 Q \implies Q\cdot\cbr{1 - \frac 12} = \Delta U \implies Q = \frac{\Delta U}{1 - \frac 12} = \frac{1500\,\text{Дж}}{1 - \frac 12} \approx 3000\,\text{Дж}.
    \\
    A' &= \frac 12 Q
        = \frac 12 \cdot \frac{\Delta U}{1 - \frac 12}
        = \frac{\Delta U}{2 - 1}
        = \frac{1500\,\text{Дж}}{2 - 1} \approx 1500\,\text{Дж}.
    \end{align*}
}
\solutionspace{60pt}

\tasknumber{8}%
\task{%
    В некотором процессе внешние силы совершили над газом работу $100\,\text{Дж}$,
    при этом его внутренняя энергия увеличилась на $150\,\text{Дж}$.
    Определите количество тепла, переданное при этом процессе газу.
    Явно пропишите, подводили газу тепло или же отводили.
}
\answer{%
    $
        Q = A_\text{газа} + \Delta U, A_\text{газа} = -A_\text{внешняя}
        \implies Q = A_\text{газа} + \Delta U = - 100\,\text{Дж} +  150\,\text{Дж} = 50\,\text{Дж}.
        \text{ Подводили.}
    $
}

\variantsplitter

\addpersonalvariant{Дмитрий Соколов}

\tasknumber{1}%
\task{%
    Напротив физической величины укажите её обозначение и единицы измерения в СИ или запишите физический закон или формулу (в пункте «г)»):
    \begin{enumerate}
        \item изменение внутренней энергии,
        \item работа газа,
        \item удельная теплоёмкость,
        \item внутренняя энергия идеального одноатомного газа.
    \end{enumerate}
}
\solutionspace{20pt}

\tasknumber{2}%
\task{%
    Определите объём идеального одноатомного газа,
    если его внутренняя энергия при давлении $4\,\text{атм}$ составляет $300\,\text{кДж}$.
    $p_{\text{aтм}} = 100\,\text{кПа}$.
}
\answer{%
    $U = \frac 32 \nu R T = \frac 32 PV \implies V = \frac 23 \cdot \frac UP= \frac 23 \cdot \frac{ 300\,\text{кДж} }{ 4\,\text{атм} } \approx 0{,}50\,\text{м}^{3}.$
}
\solutionspace{40pt}

\tasknumber{3}%
\task{%
    Газ расширился от $250\,\text{л}$ до $450\,\text{л}$.
    Давление газа при этом оставалось постоянным и равным $2{,}5\,\text{атм}$.
    Определите работу газа, ответ выразите в килоджоулях.
    $p_{\text{aтм}} = 100\,\text{кПа}$.
}
\answer{%
    $A = P\Delta V = P(V_2 - V_1) = 2{,}5\,\text{атм} \cdot\cbr{450\,\text{л} - 250\,\text{л}} = 5{,}00\,\text{кДж}.$
}
\solutionspace{40pt}

\tasknumber{4}%
\task{%
    $40\,\text{моль}$ идеального одноатомного в результате адиабатического процесса нагрелся на $15\,\text{К}$.
    Определите работу газа.
    Кто совершил положительную работу: газ или внешние силы?
    Универсальная газовая постоянная $R = 8{,}31\,\frac{\text{Дж}}{\text{моль}\cdot\text{К}}$.
}
\answer{%
    \begin{align*}
    Q &= 0, Q = \Delta U + A_\text{газа} \implies \\
    \implies A_\text{газа} &= - \Delta U = - \frac 32 \nu R \Delta T = - \frac 32 \cdot 40\,\text{моль} \cdot8{,}31\,\frac{\text{Дж}}{\text{моль}\cdot\text{К}} \cdot15\,\text{К}= -7{,}50\,\text{кДж}, \text{внешние силы.}
    \end{align*}
}
\solutionspace{40pt}

\tasknumber{5}%
\task{%
    Как изменилась внутренняя энергия одноатомного идеального газа при переходе из состояния 1 в состояние 2?
    $P_1 = 2\,\text{МПа}$, $V_1 = 3\,\text{л}$, $P_2 = 2{,}5\,\text{МПа}$, $V_2 = 2\,\text{л}$.
    Как изменилась при этом температура газа?
}
\answer{%
    \begin{align*}
    P_1V_1 &= \nu R T_1, P_2V_2 = \nu R T_2, \\
    \Delta U &= U_2-U_1 = \frac 32 \nu R T_2- \frac 32 \nu R T_1 = \frac 32 P_2 V_2 - \frac 32 P_1 V_1= \frac 32 \cdot \cbr{2{,}5\,\text{МПа} \cdot2\,\text{л} - 2\,\text{МПа} \cdot3\,\text{л}} = -1500\,\text{Дж}.
    \\
    \frac{T_2}{T_1} &= \frac{\frac{P_2V_2}{\nu R}}{\frac{P_1V_1}{\nu R}} = \frac{P_2V_2}{P_1V_1}= \frac{2{,}5\,\text{МПа} \cdot2\,\text{л}}{2\,\text{МПа} \cdot3\,\text{л}} \approx 0{,}83.
    \end{align*}
}
\solutionspace{80pt}

\tasknumber{6}%
\task{%
    $3\,\text{моль}$ идеального одноатомного газа нагрели на $10\,\text{К}$.
    Определите изменение внутренней энергии газа.
    Увеличилась она или уменьшилась?
    Универсальная газовая постоянная $R = 8{,}31\,\frac{\text{Дж}}{\text{моль}\cdot\text{К}}$.
}
\answer{%
    $
        \Delta U = \frac 32 \nu R \Delta T
            =  \frac 32 \cdot 3\,\text{моль} \cdot8{,}31\,\frac{\text{Дж}}{\text{моль}\cdot\text{К}} \cdot10\,\text{К}
            = 373\,\text{Дж}.
            \text{Увеличилась.}
    $
}
\solutionspace{40pt}

\tasknumber{7}%
\task{%
    Газу сообщили некоторое количество теплоты,
    при этом четверть его он потратил на совершение работы,
    одновременно увеличив свою внутреннюю энергию на $2400\,\text{Дж}$.
    Определите работу, совершённую газом.
}
\answer{%
    \begin{align*}
    Q &= A' + \Delta U, A' = \frac 14 Q \implies Q\cdot\cbr{1 - \frac 14} = \Delta U \implies Q = \frac{\Delta U}{1 - \frac 14} = \frac{2400\,\text{Дж}}{1 - \frac 14} \approx 3200\,\text{Дж}.
    \\
    A' &= \frac 14 Q
        = \frac 14 \cdot \frac{\Delta U}{1 - \frac 14}
        = \frac{\Delta U}{4 - 1}
        = \frac{2400\,\text{Дж}}{4 - 1} \approx 800\,\text{Дж}.
    \end{align*}
}
\solutionspace{60pt}

\tasknumber{8}%
\task{%
    В некотором процессе внешние силы совершили над газом работу $100\,\text{Дж}$,
    при этом его внутренняя энергия увеличилась на $350\,\text{Дж}$.
    Определите количество тепла, переданное при этом процессе газу.
    Явно пропишите, подводили газу тепло или же отводили.
}
\answer{%
    $
        Q = A_\text{газа} + \Delta U, A_\text{газа} = -A_\text{внешняя}
        \implies Q = A_\text{газа} + \Delta U = - 100\,\text{Дж} +  350\,\text{Дж} = 250\,\text{Дж}.
        \text{ Подводили.}
    $
}

\end{document}
% autogenerated
