\adddate{8 класс. 8 ноября 2018}
Значения табличных величин посмотреть в учебнике.

\task 1
Сравните внутреннюю энергию 1 кг веществ: вода при $0\celsius$, лёд при $0\celsius$, лёд при $-20\celsius$, лёд при $-40\celsius$, водяной пар при $115\celsius$, вода при $60\celsius$, вода при $100\celsius$, водяной пар при $100\celsius$, смесь водяного пара и воды, смесь воды и льда. Запишите вещества в столбик в порядке возрастания внутренней энергии, рядом укажите их температуру.

\task 2
Сколько энергии израсходовано на нагревание воды массой $m_\text{в} = 0{,}75\units{кг}$ от $T_1 = 20\celsius$ до $T_2 = 100\celsius$ и последующее образование пара массой $m_\text{п} = 250\units{г}$? Каково при этом изменение внутренней энергии вещества?

\task 3
Какое количество теплоты выделится при конденсации водяного пара массой $m_\text{в} = 100\units{г}$ при температуре $T_{100} = 100\celsius$ и охлаждении образовавшейся воды до $T_2 = 20\celsius$? Каково при этом изменение внутренней энергии вещества?

\task 4
Профессор Глюк в железной коробке массой $m_\units{жел} = 300\units{г}$ расплавил $m_\units{ол}=100\units{г}$ олова. Начальная температура коробки и олова $T_1 = 32\celsius$. Какое количество теплоты было передано при этом коробке и олову?

\task 5
Нагретое до $T_\text{т1} = 110\celsius$ тело опустили в сосуд с водой. В результате температура воды повысилась от $T_\text{в1} = 20\celsius$ до $T_\text{в2} = 30\celsius$. Какой стала бы температура воды $T$, если бы одновременно с первым телом в воду опустили ещё одно такое же тело, нагретое до $T_\text{т2} = 120\celsius$?