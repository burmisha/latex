\documentclass[12pt,a4paper]{amsart}%DVI-mode.
\usepackage{graphics,graphicx,epsfig}%DVI-mode.
% \documentclass[pdftex,12pt]{amsart} %PDF-mode.
% \usepackage[pdftex]{graphicx}       %PDF-mode.
% \usepackage[babel=true]{microtype}
% \usepackage[T1]{fontenc}
% \usepackage{lmodern}

\usepackage{cmap}
%\usepackage{a4wide}                 % Fit the text to A4 page tightly.
% \usepackage[utf8]{inputenc}
\usepackage[T2A]{fontenc}
\usepackage[english,russian]{babel} % Download Russian fonts.
\usepackage{amsmath,amsfonts,amssymb,amsthm,amscd,mathrsfs} % Use AMS symbols.
\usepackage{tikz}
\usetikzlibrary{circuits.ee.IEC}
\usetikzlibrary{shapes.geometric}
\usetikzlibrary{decorations.markings}
%\usetikzlibrary{dashs}
%\usetikzlibrary{info}


\textheight=28cm % высота текста
\textwidth=18cm % ширина текста
\topmargin=-2.5cm % отступ от верхнего края
\parskip=2pt % интервал между абзацами
\oddsidemargin=-1.5cm
\evensidemargin=-1.5cm 

\parindent=0pt % абзацный отступ
\tolerance=500 % терпимость к "жидким" строкам
\binoppenalty=10000 % штраф за перенос формул - 10000 - абсолютный запрет
\relpenalty=10000
\flushbottom % выравнивание высоты страниц
\pagenumbering{gobble}

\newcommand\bivec[2]{\begin{pmatrix} #1 \\ #2 \end{pmatrix}}

\newcommand\ol[1]{\overline{#1}}

\newcommand\p[1]{\Prob\!\left(#1\right)}
\newcommand\e[1]{\mathsf{E}\!\left(#1\right)}
\newcommand\disp[1]{\mathsf{D}\!\left(#1\right)}
%\newcommand\norm[2]{\mathcal{N}\!\cbr{#1,#2}}
\newcommand\sign{\text{ sign }}

\newcommand\al[1]{\begin{align*} #1 \end{align*}}
\newcommand\begcas[1]{\begin{cases}#1\end{cases}}
\newcommand\tab[2]{	\vspace{-#1pt}
						\begin{tabbing} 
						#2
						\end{tabbing}
					\vspace{-#1pt}
					}

\newcommand\maintext[1]{{\bfseries\sffamily{#1}}}
\newcommand\skipped[1]{ {\ensuremath{\text{\small{\sffamily{Пропущено:} #1} } } } }
\newcommand\simpletitle[1]{\begin{center} \maintext{#1} \end{center}}

\def\le{\leqslant}
\def\ge{\geqslant}
\def\Ell{\mathcal{L}}
\def\eps{{\varepsilon}}
\def\Rn{\mathbb{R}^n}
\def\RSS{\mathsf{RSS}}

\newcommand\foral[1]{\forall\,#1\:}
\newcommand\exist[1]{\exists\,#1\:\colon}

\newcommand\cbr[1]{\left(#1\right)} %circled brackets
\newcommand\fbr[1]{\left\{#1\right\}} %figure brackets
\newcommand\sbr[1]{\left[#1\right]} %square brackets
\newcommand\modul[1]{\left|#1\right|}

\newcommand\sqr[1]{\cbr{#1}^2}
\newcommand\inv[1]{\cbr{#1}^{-1}}

\newcommand\cdf[2]{\cdot\frac{#1}{#2}}
\newcommand\dd[2]{\frac{\partial#1}{\partial#2}}

\newcommand\integr[2]{\int\limits_{#1}^{#2}}
\newcommand\suml[2]{\sum\limits_{#1}^{#2}}
\newcommand\isum[2]{\sum\limits_{#1=#2}^{+\infty}}
\newcommand\idots[3]{#1_{#2},\ldots,#1_{#3}}
\newcommand\fdots[5]{#4{#1_{#2}}#5\ldots#5#4{#1_{#3}}}

\newcommand\obol[1]{O\!\cbr{#1}}
\newcommand\omal[1]{o\!\cbr{#1}}

\newcommand\addeps[2]{
	\begin{figure} [!ht] %lrp
		\centering
		\includegraphics[height=320px]{#1.eps}
		\vspace{-10pt}
		\caption{#2}
		\label{eps:#1}
	\end{figure}
}

\newcommand\addepssize[3]{
	\begin{figure} [!ht] %lrp hp
		\centering
		\includegraphics[height=#3px]{#1.eps}
		\vspace{-10pt}
		\caption{#2}
		\label{eps:#1}
	\end{figure}
}


\newcommand\norm[1]{\ensuremath{\left\|{#1}\right\|}}
\newcommand\ort{\bot}
\newcommand\theorem[1]{{\sffamily Теорема #1\ }}
\newcommand\lemma[1]{{\sffamily Лемма #1\ }}
\newcommand\difflim[2]{\frac{#1\cbr{#2 + \Delta#2} - #1\cbr{#2}}{\Delta #2}}
\renewcommand\proof[1]{\par\noindent$\square$ #1 \hfill$\blacksquare$\par}
\newcommand\defenition[1]{{\sffamilyОпределение #1\ }}

% \begin{document}
% %\raggedright
% \addclassdate{7}{20 апреля 2018}

\task 1
Площадь большого поршня гидравлического домкрата $S_1 = 20\units{см}^2$, а малого $S_2 = 0{,}5\units{см}^2.$ Груз какой максимальной массы можно поднять этим домкратом, если на малый поршень давить с силой не более $F=200\units{Н}?$ Силой трения от стенки цилиндров пренебречь.

\task 2
В сосуд налита вода. Расстояние от поверхности воды до дна $H = 0{,}5\units{м},$ площадь дна $S = 0{,}1\units{м}^2.$ Найти гидростатическое давление $P_1$ и полное давление $P_2$ вблизи дна. Найти силу давления воды на дно. Плотность воды \rhowater

\task 3
На лёгкий поршень площадью $S=900\units{см}^2,$ касающийся поверхности воды, поставили гирю массы $m=3\units{кг}$. Высота слоя воды в сосуде с вертикальными стенками $H = 20\units{см}$. Определить давление жидкости вблизи дна, если плотность воды \rhowater

\task 4
Давление газов в конце сгорания в цилиндре дизельного двигателя трактора $P = 9\units{МПа}.$ Диаметр цилиндра $d = 130\units{мм}.$ С какой силой газы давят на поршень в цилиндре? Площадь круга диаметром $D$ равна $S = \cfrac{\pi D^2}4.$

\task 5
Площадь малого поршня гидравлического подъёмника $S_1 = 0{,}8\units{см}^2$, а большого $S_2 = 40\units{см}^2.$ Какую силу $F$ надо приложить к малому поршню, чтобы поднять груз весом $P = 8\units{кН}?$

\task 6
Герметичный сосуд полностью заполнен водой и стоит на столе. На небольшой поршень площадью $S$ давят рукой с силой $F$. Поршень находится ниже крышки сосуда на $H_1$, выше дна на $H_2$ и может свободно перемещаться. Плотность воды $\rho$, атмосферное давление $P_A$. Найти давления $P_1$ и $P_2$ в воде вблизи крышки и дна сосуда.
\\ \\
\addclassdate{7}{20 апреля 2018}

\task 1
Площадь большого поршня гидравлического домкрата $S_1 = 20\units{см}^2$, а малого $S_2 = 0{,}5\units{см}^2.$ Груз какой максимальной массы можно поднять этим домкратом, если на малый поршень давить с силой не более $F=200\units{Н}?$ Силой трения от стенки цилиндров пренебречь.

\task 2
В сосуд налита вода. Расстояние от поверхности воды до дна $H = 0{,}5\units{м},$ площадь дна $S = 0{,}1\units{м}^2.$ Найти гидростатическое давление $P_1$ и полное давление $P_2$ вблизи дна. Найти силу давления воды на дно. Плотность воды \rhowater

\task 3
На лёгкий поршень площадью $S=900\units{см}^2,$ касающийся поверхности воды, поставили гирю массы $m=3\units{кг}$. Высота слоя воды в сосуде с вертикальными стенками $H = 20\units{см}$. Определить давление жидкости вблизи дна, если плотность воды \rhowater

\task 4
Давление газов в конце сгорания в цилиндре дизельного двигателя трактора $P = 9\units{МПа}.$ Диаметр цилиндра $d = 130\units{мм}.$ С какой силой газы давят на поршень в цилиндре? Площадь круга диаметром $D$ равна $S = \cfrac{\pi D^2}4.$

\task 5
Площадь малого поршня гидравлического подъёмника $S_1 = 0{,}8\units{см}^2$, а большого $S_2 = 40\units{см}^2.$ Какую силу $F$ надо приложить к малому поршню, чтобы поднять груз весом $P = 8\units{кН}?$

\task 6
Герметичный сосуд полностью заполнен водой и стоит на столе. На небольшой поршень площадью $S$ давят рукой с силой $F$. Поршень находится ниже крышки сосуда на $H_1$, выше дна на $H_2$ и может свободно перемещаться. Плотность воды $\rho$, атмосферное давление $P_A$. Найти давления $P_1$ и $P_2$ в воде вблизи крышки и дна сосуда.

\newpage

\adddate{8 класс. 20 апреля 2018}

\task 1
Между точками $A$ и $B$ электрической цепи подключены последовательно резисторы $R_1 = 10\units{Ом}$ и $R_2 = 20\units{Ом}$ и параллельно им $R_3 = 30\units{Ом}.$ Найдите эквивалентное сопротивление $R_{AB}$ этого участка цепи.

\task 2
Электрическая цепь состоит из последовательности $N$ одинаковых звеньев, в которых каждый резистор имеет сопротивление $r$. Последнее звено замкнуто резистором сопротивлением $R$. При каком соотношении $\cfrac{R}{r}$ сопротивление цепи не зависит от числа звеньев?

\task 3
Для измерения сопротивления $R$ проводника собрана электрическая цепь. Вольтметр $V$ показывает напряжение $U_V = 5\units{В},$ показание амперметра $A$ равно $I_A = 25\units{мА}.$ Найдите величину $R$ сопротивления проводника. Внутреннее сопротивление вольтметра $R_V = 1{,}0\units{кОм},$ внутреннее сопротивление амперметра $R_A = 2{,}0\units{Ом}.$

\task 4
Шкала гальванометра имеет $N=100$ делений, цена деления $\delta = 1\units{мкА}$. Внутреннее сопротивление гальванометра $R_G = 1{,}0\units{кОм}.$ Как из этого прибора сделать вольтметр для измерения напряжений до $U = 100\units{В}$ или амперметр для измерения токов силой до $I = 1\units{А}?$

\\ \\ \\ \\ \\ \\ \\ \\
\adddate{8 класс. 20 апреля 2018}

\task 1
Между точками $A$ и $B$ электрической цепи подключены последовательно резисторы $R_1 = 10\units{Ом}$ и $R_2 = 20\units{Ом}$ и параллельно им $R_3 = 30\units{Ом}.$ Найдите эквивалентное сопротивление $R_{AB}$ этого участка цепи.

\task 2
Электрическая цепь состоит из последовательности $N$ одинаковых звеньев, в которых каждый резистор имеет сопротивление $r$. Последнее звено замкнуто резистором сопротивлением $R$. При каком соотношении $\cfrac{R}{r}$ сопротивление цепи не зависит от числа звеньев?

\task 3
Для измерения сопротивления $R$ проводника собрана электрическая цепь. Вольтметр $V$ показывает напряжение $U_V = 5\units{В},$ показание амперметра $A$ равно $I_A = 25\units{мА}.$ Найдите величину $R$ сопротивления проводника. Внутреннее сопротивление вольтметра $R_V = 1{,}0\units{кОм},$ внутреннее сопротивление амперметра $R_A = 2{,}0\units{Ом}.$

\task 4
Шкала гальванометра имеет $N=100$ делений, цена деления $\delta = 1\units{мкА}$. Внутреннее сопротивление гальванометра $R_G = 1{,}0\units{кОм}.$ Как из этого прибора сделать вольтметр для измерения напряжений до $U = 100\units{В}$ или амперметр для измерения токов силой до $I = 1\units{А}?$


% % \begin{flushright}
\textsc{ГБОУ школа №554, 20 ноября 2018\,г.}
\end{flushright}

\begin{center}
\LARGE \textsc{Математический бой, 8 класс}
\end{center}

\problem{1} Есть тридцать карточек, на каждой написано по одному числу: на десяти карточках~–~$a$,  на десяти других~–~$b$ и на десяти оставшихся~–~$c$ (числа  различны). Известно, что к любым пяти карточкам можно подобрать ещё пять так, что сумма чисел на этих десяти карточках будет равна нулю. Докажите, что~одно из~чисел~$a, b, c$ равно нулю.

\problem{2} Вокруг стола стола пустили пакет с орешками. Первый взял один орешек, второй — 2, третий — 3 и так далее: каждый следующий брал на 1 орешек больше. Известно, что на втором круге было взято в сумме на 100 орешков больше, чем на первом. Сколько человек сидело за столом?

% \problem{2} Натуральное число разрешено увеличить на любое целое число процентов от 1 до 100, если при этом получаем натуральное число. Найдите наименьшее натуральное число, которое нельзя при помощи таких операций получить из~числа 1.

% \problem{3} Найти сумму $1^2 - 2^2 + 3^2 - 4^2 + 5^2 + \ldots - 2018^2$.

\problem{3} В кружке рукоделия, где занимается Валя, более 93\% участников~—~девочки. Какое наименьшее число участников может быть в таком кружке?

\problem{4} Произведение 2018 целых чисел равно 1. Может ли их сумма оказаться равной~0?

% \problem{4} Можно ли все натуральные числа от~1 до~9 записать в~клетки таблицы~$3\times3$ так, чтобы сумма в~любых двух соседних (по~вертикали или горизонтали) клетках равнялось простому числу?

\problem{5} На доске написано 2018 нулей и 2019 единиц. Женя стирает 2 числа и, если они были одинаковы, дописывает к оставшимся один ноль, а~если разные — единицу. Потом Женя повторяет эту операцию снова, потом ещё и~так далее. В~результате на~доске останется только одно число. Что это за~число?

\problem{6} Докажите, что в~любой компании людей найдутся 2~человека, имеющие равное число знакомых в этой компании (если $A$~знаком с~$B$, то~и $B$~знаком с~$A$).

\problem{7} Три колокола начинают бить одновременно. Интервалы между ударами колоколов соответственно составляют $\cfrac43$~секунды, $\cfrac53$~секунды и $2$~секунды. Совпавшие по времени удары воспринимаются за~один. Сколько ударов будет услышано за 1~минуту, включая первый и последний удары?

\problem{8} Восемь одинаковых момент расположены по кругу. Известно, что три из~них~— фальшивые, и они расположены рядом друг с~другом. Вес фальшивой монеты отличается от~веса настоящей. Все фальшивые монеты весят одинаково, но неизвестно, тяжелее или легче фальшивая монета настоящей. Покажите, что за~3~взвешивания на~чашечных весах без~гирь можно определить все фальшивые монеты.

% \end{document}

\begin{document}


\setdate{27~ноября~2019}
\setclass{8}

\addpersonalvariant{Михаил Бурмистров}
\tasknumber{1}\task{
    Сколько льда при температуре $0\celsius$ можно расплавить,
    сообщив ему энергию $4\,\text{МДж}$?
    Здесь (и во всех следующих задачах) используйте табличные значения из учебника.
}
\answer{
    $
                Q = \lambda m \implies m
                    = \frac Q{\lambda}
                    = \frac { 4\,\text{МДж} }{ 340\,\frac{\text{кДж}}{\text{кг}} }
                    \approx 11{,}8\,\text{кг}
            $
}

\tasknumber{2}\task{
    Какое количество теплоты выделится при затвердевании $50\,\text{кг}$ расплавленного алюминия при температуре плавления?
}
\answer{
    $
                Q
                    = - \lambda m
                    = - 390\,\frac{\text{кДж}}{\text{кг}} \cdot 50\,\text{кг}
                    = - 19{,}5\,\text{МДж} \implies \abs{Q} = 19{,}5\,\text{МДж}
            $
}

\tasknumber{3}\task{
    Какое количество теплоты необходимо для превращения воды массой $15\,\text{кг}$ при $t = 70\celsius$
    в пар при температуре $t_{100} = 100\celsius$?
}
\answer{
    $
                Q
                    = cm\Delta t + Lm
                    = m\cbr{c(t_{100} - t) + L}
                    = 15\,\text{кг} \cdot \cbr{4200\,\frac{\text{Дж}}{\text{кг}\cdot\text{К}}\cbr{100\celsius - 70\celsius} + 2{,}3\,\frac{\text{МДж}}{\text{кг}}}
                    = 36{,}39\,\text{МДж}
            $
}

\tasknumber{4}\task{
    Воду температурой $t = 10\celsius$ нагрели и превратили в пар при температуре $t_{100} = 100\celsius$,
    потратив $5000\,\text{кДж}$.
    Определите массу воды.
}
\answer{
    $
                Q
                    = cm\Delta t + Lm
                    = m\cbr{c(t_{100} - t) + L}
                \implies
                m = \frac{Q}{c(t_{100} - t) + L}
                    = \frac { 5000\,\text{кДж} }{4200\,\frac{\text{Дж}}{\text{кг}\cdot\text{К}}\cbr{100\celsius - 10\celsius} + 2{,}3\,\frac{\text{МДж}}{\text{кг}}}
                    \approx 1{,}87\,\text{кг}
            $
}

\tasknumber{5}\task{
    Цинковое тело температурой $T = 100\celsius$ опустили
    в воду температурой $t = 10\celsius$, масса которой равна массе тела.
    Определите, какая температура установится в сосуде.
}
\answer{
    \begin{align*}
        Q_1 + Q_2 &= 0, \\
        Q_1 &= c_1 m_1 \Delta t_1 = c_1 m (\theta - t_1), \\
        Q_2 &= c_2 m_2 \Delta t_2 = c_2 m (\theta - t_2), \\
        c_1 m (\theta - t_1) + c_2 m (\theta - t_2) &= 0, \\
        c_1 (\theta - t_1) + c_2 (\theta - t_2) &= 0, \\
        c_1 \theta - c_1 t_1 + c_2 \theta - c_2 t_2 &= 0, \\
        (c_1 + c_2)\theta &= c_1 t_1 + c_2 t_2, \\
        \theta &= \frac{c_1 t_1 + c_2 t_2}{c_1 + c_2}
            = \frac{4200\,\frac{\text{Дж}}{\text{кг}\cdot\text{К}} \cdot 10\celsius + 400\,\frac{\text{Дж}}{\text{кг}\cdot\text{К}} \cdot 100\celsius}{4200\,\frac{\text{Дж}}{\text{кг}\cdot\text{К}} + 400\,\frac{\text{Дж}}{\text{кг}\cdot\text{К}}}
            \approx 17.8 \celsius.
    \end{align*}
}

\addpersonalvariant{Аксенов Максим}
\tasknumber{1}\task{
    Сколько льда при температуре $0\celsius$ можно расплавить,
    сообщив ему энергию $7\,\text{МДж}$?
    Здесь (и во всех следующих задачах) используйте табличные значения из учебника.
}
\answer{
    $
                Q = \lambda m \implies m
                    = \frac Q{\lambda}
                    = \frac { 7\,\text{МДж} }{ 340\,\frac{\text{кДж}}{\text{кг}} }
                    \approx 20{,}6\,\text{кг}
            $
}

\tasknumber{2}\task{
    Какое количество теплоты выделится при затвердевании $20\,\text{кг}$ расплавленного меди при температуре плавления?
}
\answer{
    $
                Q
                    = - \lambda m
                    = - 210\,\frac{\text{кДж}}{\text{кг}} \cdot 20\,\text{кг}
                    = - 4{,}2\,\text{МДж} \implies \abs{Q} = 4{,}2\,\text{МДж}
            $
}

\tasknumber{3}\task{
    Какое количество теплоты необходимо для превращения воды массой $2\,\text{кг}$ при $t = 70\celsius$
    в пар при температуре $t_{100} = 100\celsius$?
}
\answer{
    $
                Q
                    = cm\Delta t + Lm
                    = m\cbr{c(t_{100} - t) + L}
                    = 2\,\text{кг} \cdot \cbr{4200\,\frac{\text{Дж}}{\text{кг}\cdot\text{К}}\cbr{100\celsius - 70\celsius} + 2{,}3\,\frac{\text{МДж}}{\text{кг}}}
                    = 4{,}85\,\text{МДж}
            $
}

\tasknumber{4}\task{
    Воду температурой $t = 70\celsius$ нагрели и превратили в пар при температуре $t_{100} = 100\celsius$,
    потратив $2000\,\text{кДж}$.
    Определите массу воды.
}
\answer{
    $
                Q
                    = cm\Delta t + Lm
                    = m\cbr{c(t_{100} - t) + L}
                \implies
                m = \frac{Q}{c(t_{100} - t) + L}
                    = \frac { 2000\,\text{кДж} }{4200\,\frac{\text{Дж}}{\text{кг}\cdot\text{К}}\cbr{100\celsius - 70\celsius} + 2{,}3\,\frac{\text{МДж}}{\text{кг}}}
                    \approx 0{,}82\,\text{кг}
            $
}

\tasknumber{5}\task{
    Стальное тело температурой $T = 100\celsius$ опустили
    в воду температурой $t = 10\celsius$, масса которой равна массе тела.
    Определите, какая температура установится в сосуде.
}
\answer{
    \begin{align*}
        Q_1 + Q_2 &= 0, \\
        Q_1 &= c_1 m_1 \Delta t_1 = c_1 m (\theta - t_1), \\
        Q_2 &= c_2 m_2 \Delta t_2 = c_2 m (\theta - t_2), \\
        c_1 m (\theta - t_1) + c_2 m (\theta - t_2) &= 0, \\
        c_1 (\theta - t_1) + c_2 (\theta - t_2) &= 0, \\
        c_1 \theta - c_1 t_1 + c_2 \theta - c_2 t_2 &= 0, \\
        (c_1 + c_2)\theta &= c_1 t_1 + c_2 t_2, \\
        \theta &= \frac{c_1 t_1 + c_2 t_2}{c_1 + c_2}
            = \frac{4200\,\frac{\text{Дж}}{\text{кг}\cdot\text{К}} \cdot 10\celsius + 500\,\frac{\text{Дж}}{\text{кг}\cdot\text{К}} \cdot 100\celsius}{4200\,\frac{\text{Дж}}{\text{кг}\cdot\text{К}} + 500\,\frac{\text{Дж}}{\text{кг}\cdot\text{К}}}
            \approx 19.6 \celsius.
    \end{align*}
}

\addpersonalvariant{Ахметова Маргарита}
\tasknumber{1}\task{
    Сколько льда при температуре $0\celsius$ можно расплавить,
    сообщив ему энергию $5\,\text{МДж}$?
    Здесь (и во всех следующих задачах) используйте табличные значения из учебника.
}
\answer{
    $
                Q = \lambda m \implies m
                    = \frac Q{\lambda}
                    = \frac { 5\,\text{МДж} }{ 340\,\frac{\text{кДж}}{\text{кг}} }
                    \approx 14{,}7\,\text{кг}
            $
}

\tasknumber{2}\task{
    Какое количество теплоты выделится при затвердевании $15\,\text{кг}$ расплавленного стали при температуре плавления?
}
\answer{
    $
                Q
                    = - \lambda m
                    = - 84\,\frac{\text{кДж}}{\text{кг}} \cdot 15\,\text{кг}
                    = - 1{,}3\,\text{МДж} \implies \abs{Q} = 1{,}3\,\text{МДж}
            $
}

\tasknumber{3}\task{
    Какое количество теплоты необходимо для превращения воды массой $2\,\text{кг}$ при $t = 60\celsius$
    в пар при температуре $t_{100} = 100\celsius$?
}
\answer{
    $
                Q
                    = cm\Delta t + Lm
                    = m\cbr{c(t_{100} - t) + L}
                    = 2\,\text{кг} \cdot \cbr{4200\,\frac{\text{Дж}}{\text{кг}\cdot\text{К}}\cbr{100\celsius - 60\celsius} + 2{,}3\,\frac{\text{МДж}}{\text{кг}}}
                    = 4{,}94\,\text{МДж}
            $
}

\tasknumber{4}\task{
    Воду температурой $t = 50\celsius$ нагрели и превратили в пар при температуре $t_{100} = 100\celsius$,
    потратив $2000\,\text{кДж}$.
    Определите массу воды.
}
\answer{
    $
                Q
                    = cm\Delta t + Lm
                    = m\cbr{c(t_{100} - t) + L}
                \implies
                m = \frac{Q}{c(t_{100} - t) + L}
                    = \frac { 2000\,\text{кДж} }{4200\,\frac{\text{Дж}}{\text{кг}\cdot\text{К}}\cbr{100\celsius - 50\celsius} + 2{,}3\,\frac{\text{МДж}}{\text{кг}}}
                    \approx 0{,}8\,\text{кг}
            $
}

\tasknumber{5}\task{
    Стальное тело температурой $T = 70\celsius$ опустили
    в воду температурой $t = 10\celsius$, масса которой равна массе тела.
    Определите, какая температура установится в сосуде.
}
\answer{
    \begin{align*}
        Q_1 + Q_2 &= 0, \\
        Q_1 &= c_1 m_1 \Delta t_1 = c_1 m (\theta - t_1), \\
        Q_2 &= c_2 m_2 \Delta t_2 = c_2 m (\theta - t_2), \\
        c_1 m (\theta - t_1) + c_2 m (\theta - t_2) &= 0, \\
        c_1 (\theta - t_1) + c_2 (\theta - t_2) &= 0, \\
        c_1 \theta - c_1 t_1 + c_2 \theta - c_2 t_2 &= 0, \\
        (c_1 + c_2)\theta &= c_1 t_1 + c_2 t_2, \\
        \theta &= \frac{c_1 t_1 + c_2 t_2}{c_1 + c_2}
            = \frac{4200\,\frac{\text{Дж}}{\text{кг}\cdot\text{К}} \cdot 10\celsius + 500\,\frac{\text{Дж}}{\text{кг}\cdot\text{К}} \cdot 70\celsius}{4200\,\frac{\text{Дж}}{\text{кг}\cdot\text{К}} + 500\,\frac{\text{Дж}}{\text{кг}\cdot\text{К}}}
            \approx 16.4 \celsius.
    \end{align*}
}

\addpersonalvariant{Глембо Артём}
\tasknumber{1}\task{
    Сколько льда при температуре $0\celsius$ можно расплавить,
    сообщив ему энергию $9\,\text{МДж}$?
    Здесь (и во всех следующих задачах) используйте табличные значения из учебника.
}
\answer{
    $
                Q = \lambda m \implies m
                    = \frac Q{\lambda}
                    = \frac { 9\,\text{МДж} }{ 340\,\frac{\text{кДж}}{\text{кг}} }
                    \approx 26{,}5\,\text{кг}
            $
}

\tasknumber{2}\task{
    Какое количество теплоты выделится при затвердевании $30\,\text{кг}$ расплавленного стали при температуре плавления?
}
\answer{
    $
                Q
                    = - \lambda m
                    = - 84\,\frac{\text{кДж}}{\text{кг}} \cdot 30\,\text{кг}
                    = - 2{,}5\,\text{МДж} \implies \abs{Q} = 2{,}5\,\text{МДж}
            $
}

\tasknumber{3}\task{
    Какое количество теплоты необходимо для превращения воды массой $15\,\text{кг}$ при $t = 20\celsius$
    в пар при температуре $t_{100} = 100\celsius$?
}
\answer{
    $
                Q
                    = cm\Delta t + Lm
                    = m\cbr{c(t_{100} - t) + L}
                    = 15\,\text{кг} \cdot \cbr{4200\,\frac{\text{Дж}}{\text{кг}\cdot\text{К}}\cbr{100\celsius - 20\celsius} + 2{,}3\,\frac{\text{МДж}}{\text{кг}}}
                    = 39{,}54\,\text{МДж}
            $
}

\tasknumber{4}\task{
    Воду температурой $t = 50\celsius$ нагрели и превратили в пар при температуре $t_{100} = 100\celsius$,
    потратив $4000\,\text{кДж}$.
    Определите массу воды.
}
\answer{
    $
                Q
                    = cm\Delta t + Lm
                    = m\cbr{c(t_{100} - t) + L}
                \implies
                m = \frac{Q}{c(t_{100} - t) + L}
                    = \frac { 4000\,\text{кДж} }{4200\,\frac{\text{Дж}}{\text{кг}\cdot\text{К}}\cbr{100\celsius - 50\celsius} + 2{,}3\,\frac{\text{МДж}}{\text{кг}}}
                    \approx 1{,}59\,\text{кг}
            $
}

\tasknumber{5}\task{
    Стальное тело температурой $T = 90\celsius$ опустили
    в воду температурой $t = 20\celsius$, масса которой равна массе тела.
    Определите, какая температура установится в сосуде.
}
\answer{
    \begin{align*}
        Q_1 + Q_2 &= 0, \\
        Q_1 &= c_1 m_1 \Delta t_1 = c_1 m (\theta - t_1), \\
        Q_2 &= c_2 m_2 \Delta t_2 = c_2 m (\theta - t_2), \\
        c_1 m (\theta - t_1) + c_2 m (\theta - t_2) &= 0, \\
        c_1 (\theta - t_1) + c_2 (\theta - t_2) &= 0, \\
        c_1 \theta - c_1 t_1 + c_2 \theta - c_2 t_2 &= 0, \\
        (c_1 + c_2)\theta &= c_1 t_1 + c_2 t_2, \\
        \theta &= \frac{c_1 t_1 + c_2 t_2}{c_1 + c_2}
            = \frac{4200\,\frac{\text{Дж}}{\text{кг}\cdot\text{К}} \cdot 20\celsius + 500\,\frac{\text{Дж}}{\text{кг}\cdot\text{К}} \cdot 90\celsius}{4200\,\frac{\text{Дж}}{\text{кг}\cdot\text{К}} + 500\,\frac{\text{Дж}}{\text{кг}\cdot\text{К}}}
            \approx 27.4 \celsius.
    \end{align*}
}

\addpersonalvariant{Гончарова Наталья}
\tasknumber{1}\task{
    Сколько льда при температуре $0\celsius$ можно расплавить,
    сообщив ему энергию $3\,\text{МДж}$?
    Здесь (и во всех следующих задачах) используйте табличные значения из учебника.
}
\answer{
    $
                Q = \lambda m \implies m
                    = \frac Q{\lambda}
                    = \frac { 3\,\text{МДж} }{ 340\,\frac{\text{кДж}}{\text{кг}} }
                    \approx 8{,}8\,\text{кг}
            $
}

\tasknumber{2}\task{
    Какое количество теплоты выделится при затвердевании $25\,\text{кг}$ расплавленного меди при температуре плавления?
}
\answer{
    $
                Q
                    = - \lambda m
                    = - 210\,\frac{\text{кДж}}{\text{кг}} \cdot 25\,\text{кг}
                    = - 5{,}2\,\text{МДж} \implies \abs{Q} = 5{,}2\,\text{МДж}
            $
}

\tasknumber{3}\task{
    Какое количество теплоты необходимо для превращения воды массой $5\,\text{кг}$ при $t = 60\celsius$
    в пар при температуре $t_{100} = 100\celsius$?
}
\answer{
    $
                Q
                    = cm\Delta t + Lm
                    = m\cbr{c(t_{100} - t) + L}
                    = 5\,\text{кг} \cdot \cbr{4200\,\frac{\text{Дж}}{\text{кг}\cdot\text{К}}\cbr{100\celsius - 60\celsius} + 2{,}3\,\frac{\text{МДж}}{\text{кг}}}
                    = 12{,}34\,\text{МДж}
            $
}

\tasknumber{4}\task{
    Воду температурой $t = 50\celsius$ нагрели и превратили в пар при температуре $t_{100} = 100\celsius$,
    потратив $2500\,\text{кДж}$.
    Определите массу воды.
}
\answer{
    $
                Q
                    = cm\Delta t + Lm
                    = m\cbr{c(t_{100} - t) + L}
                \implies
                m = \frac{Q}{c(t_{100} - t) + L}
                    = \frac { 2500\,\text{кДж} }{4200\,\frac{\text{Дж}}{\text{кг}\cdot\text{К}}\cbr{100\celsius - 50\celsius} + 2{,}3\,\frac{\text{МДж}}{\text{кг}}}
                    \approx 1{,}0\,\text{кг}
            $
}

\tasknumber{5}\task{
    Стальное тело температурой $T = 80\celsius$ опустили
    в воду температурой $t = 30\celsius$, масса которой равна массе тела.
    Определите, какая температура установится в сосуде.
}
\answer{
    \begin{align*}
        Q_1 + Q_2 &= 0, \\
        Q_1 &= c_1 m_1 \Delta t_1 = c_1 m (\theta - t_1), \\
        Q_2 &= c_2 m_2 \Delta t_2 = c_2 m (\theta - t_2), \\
        c_1 m (\theta - t_1) + c_2 m (\theta - t_2) &= 0, \\
        c_1 (\theta - t_1) + c_2 (\theta - t_2) &= 0, \\
        c_1 \theta - c_1 t_1 + c_2 \theta - c_2 t_2 &= 0, \\
        (c_1 + c_2)\theta &= c_1 t_1 + c_2 t_2, \\
        \theta &= \frac{c_1 t_1 + c_2 t_2}{c_1 + c_2}
            = \frac{4200\,\frac{\text{Дж}}{\text{кг}\cdot\text{К}} \cdot 30\celsius + 500\,\frac{\text{Дж}}{\text{кг}\cdot\text{К}} \cdot 80\celsius}{4200\,\frac{\text{Дж}}{\text{кг}\cdot\text{К}} + 500\,\frac{\text{Дж}}{\text{кг}\cdot\text{К}}}
            \approx 35.3 \celsius.
    \end{align*}
}

\addpersonalvariant{Касымов Файёзбек}
\tasknumber{1}\task{
    Сколько льда при температуре $0\celsius$ можно расплавить,
    сообщив ему энергию $5\,\text{МДж}$?
    Здесь (и во всех следующих задачах) используйте табличные значения из учебника.
}
\answer{
    $
                Q = \lambda m \implies m
                    = \frac Q{\lambda}
                    = \frac { 5\,\text{МДж} }{ 340\,\frac{\text{кДж}}{\text{кг}} }
                    \approx 14{,}7\,\text{кг}
            $
}

\tasknumber{2}\task{
    Какое количество теплоты выделится при затвердевании $50\,\text{кг}$ расплавленного стали при температуре плавления?
}
\answer{
    $
                Q
                    = - \lambda m
                    = - 84\,\frac{\text{кДж}}{\text{кг}} \cdot 50\,\text{кг}
                    = - 4{,}2\,\text{МДж} \implies \abs{Q} = 4{,}2\,\text{МДж}
            $
}

\tasknumber{3}\task{
    Какое количество теплоты необходимо для превращения воды массой $3\,\text{кг}$ при $t = 40\celsius$
    в пар при температуре $t_{100} = 100\celsius$?
}
\answer{
    $
                Q
                    = cm\Delta t + Lm
                    = m\cbr{c(t_{100} - t) + L}
                    = 3\,\text{кг} \cdot \cbr{4200\,\frac{\text{Дж}}{\text{кг}\cdot\text{К}}\cbr{100\celsius - 40\celsius} + 2{,}3\,\frac{\text{МДж}}{\text{кг}}}
                    = 7{,}66\,\text{МДж}
            $
}

\tasknumber{4}\task{
    Воду температурой $t = 30\celsius$ нагрели и превратили в пар при температуре $t_{100} = 100\celsius$,
    потратив $5000\,\text{кДж}$.
    Определите массу воды.
}
\answer{
    $
                Q
                    = cm\Delta t + Lm
                    = m\cbr{c(t_{100} - t) + L}
                \implies
                m = \frac{Q}{c(t_{100} - t) + L}
                    = \frac { 5000\,\text{кДж} }{4200\,\frac{\text{Дж}}{\text{кг}\cdot\text{К}}\cbr{100\celsius - 30\celsius} + 2{,}3\,\frac{\text{МДж}}{\text{кг}}}
                    \approx 1{,}93\,\text{кг}
            $
}

\tasknumber{5}\task{
    Стальное тело температурой $T = 100\celsius$ опустили
    в воду температурой $t = 20\celsius$, масса которой равна массе тела.
    Определите, какая температура установится в сосуде.
}
\answer{
    \begin{align*}
        Q_1 + Q_2 &= 0, \\
        Q_1 &= c_1 m_1 \Delta t_1 = c_1 m (\theta - t_1), \\
        Q_2 &= c_2 m_2 \Delta t_2 = c_2 m (\theta - t_2), \\
        c_1 m (\theta - t_1) + c_2 m (\theta - t_2) &= 0, \\
        c_1 (\theta - t_1) + c_2 (\theta - t_2) &= 0, \\
        c_1 \theta - c_1 t_1 + c_2 \theta - c_2 t_2 &= 0, \\
        (c_1 + c_2)\theta &= c_1 t_1 + c_2 t_2, \\
        \theta &= \frac{c_1 t_1 + c_2 t_2}{c_1 + c_2}
            = \frac{4200\,\frac{\text{Дж}}{\text{кг}\cdot\text{К}} \cdot 20\celsius + 500\,\frac{\text{Дж}}{\text{кг}\cdot\text{К}} \cdot 100\celsius}{4200\,\frac{\text{Дж}}{\text{кг}\cdot\text{К}} + 500\,\frac{\text{Дж}}{\text{кг}\cdot\text{К}}}
            \approx 28.5 \celsius.
    \end{align*}
}

\addpersonalvariant{Козинец Александр}
\tasknumber{1}\task{
    Сколько льда при температуре $0\celsius$ можно расплавить,
    сообщив ему энергию $8\,\text{МДж}$?
    Здесь (и во всех следующих задачах) используйте табличные значения из учебника.
}
\answer{
    $
                Q = \lambda m \implies m
                    = \frac Q{\lambda}
                    = \frac { 8\,\text{МДж} }{ 340\,\frac{\text{кДж}}{\text{кг}} }
                    \approx 23{,}5\,\text{кг}
            $
}

\tasknumber{2}\task{
    Какое количество теплоты выделится при затвердевании $75\,\text{кг}$ расплавленного алюминия при температуре плавления?
}
\answer{
    $
                Q
                    = - \lambda m
                    = - 390\,\frac{\text{кДж}}{\text{кг}} \cdot 75\,\text{кг}
                    = - 29{,}2\,\text{МДж} \implies \abs{Q} = 29{,}2\,\text{МДж}
            $
}

\tasknumber{3}\task{
    Какое количество теплоты необходимо для превращения воды массой $15\,\text{кг}$ при $t = 50\celsius$
    в пар при температуре $t_{100} = 100\celsius$?
}
\answer{
    $
                Q
                    = cm\Delta t + Lm
                    = m\cbr{c(t_{100} - t) + L}
                    = 15\,\text{кг} \cdot \cbr{4200\,\frac{\text{Дж}}{\text{кг}\cdot\text{К}}\cbr{100\celsius - 50\celsius} + 2{,}3\,\frac{\text{МДж}}{\text{кг}}}
                    = 37{,}65\,\text{МДж}
            $
}

\tasknumber{4}\task{
    Воду температурой $t = 60\celsius$ нагрели и превратили в пар при температуре $t_{100} = 100\celsius$,
    потратив $5000\,\text{кДж}$.
    Определите массу воды.
}
\answer{
    $
                Q
                    = cm\Delta t + Lm
                    = m\cbr{c(t_{100} - t) + L}
                \implies
                m = \frac{Q}{c(t_{100} - t) + L}
                    = \frac { 5000\,\text{кДж} }{4200\,\frac{\text{Дж}}{\text{кг}\cdot\text{К}}\cbr{100\celsius - 60\celsius} + 2{,}3\,\frac{\text{МДж}}{\text{кг}}}
                    \approx 2{,}03\,\text{кг}
            $
}

\tasknumber{5}\task{
    Стальное тело температурой $T = 90\celsius$ опустили
    в воду температурой $t = 10\celsius$, масса которой равна массе тела.
    Определите, какая температура установится в сосуде.
}
\answer{
    \begin{align*}
        Q_1 + Q_2 &= 0, \\
        Q_1 &= c_1 m_1 \Delta t_1 = c_1 m (\theta - t_1), \\
        Q_2 &= c_2 m_2 \Delta t_2 = c_2 m (\theta - t_2), \\
        c_1 m (\theta - t_1) + c_2 m (\theta - t_2) &= 0, \\
        c_1 (\theta - t_1) + c_2 (\theta - t_2) &= 0, \\
        c_1 \theta - c_1 t_1 + c_2 \theta - c_2 t_2 &= 0, \\
        (c_1 + c_2)\theta &= c_1 t_1 + c_2 t_2, \\
        \theta &= \frac{c_1 t_1 + c_2 t_2}{c_1 + c_2}
            = \frac{4200\,\frac{\text{Дж}}{\text{кг}\cdot\text{К}} \cdot 10\celsius + 500\,\frac{\text{Дж}}{\text{кг}\cdot\text{К}} \cdot 90\celsius}{4200\,\frac{\text{Дж}}{\text{кг}\cdot\text{К}} + 500\,\frac{\text{Дж}}{\text{кг}\cdot\text{К}}}
            \approx 18.5 \celsius.
    \end{align*}
}

\addpersonalvariant{Медведева Екатерина}
\tasknumber{1}\task{
    Сколько льда при температуре $0\celsius$ можно расплавить,
    сообщив ему энергию $9\,\text{МДж}$?
    Здесь (и во всех следующих задачах) используйте табличные значения из учебника.
}
\answer{
    $
                Q = \lambda m \implies m
                    = \frac Q{\lambda}
                    = \frac { 9\,\text{МДж} }{ 340\,\frac{\text{кДж}}{\text{кг}} }
                    \approx 26{,}5\,\text{кг}
            $
}

\tasknumber{2}\task{
    Какое количество теплоты выделится при затвердевании $75\,\text{кг}$ расплавленного стали при температуре плавления?
}
\answer{
    $
                Q
                    = - \lambda m
                    = - 84\,\frac{\text{кДж}}{\text{кг}} \cdot 75\,\text{кг}
                    = - 6{,}3\,\text{МДж} \implies \abs{Q} = 6{,}3\,\text{МДж}
            $
}

\tasknumber{3}\task{
    Какое количество теплоты необходимо для превращения воды массой $3\,\text{кг}$ при $t = 30\celsius$
    в пар при температуре $t_{100} = 100\celsius$?
}
\answer{
    $
                Q
                    = cm\Delta t + Lm
                    = m\cbr{c(t_{100} - t) + L}
                    = 3\,\text{кг} \cdot \cbr{4200\,\frac{\text{Дж}}{\text{кг}\cdot\text{К}}\cbr{100\celsius - 30\celsius} + 2{,}3\,\frac{\text{МДж}}{\text{кг}}}
                    = 7{,}78\,\text{МДж}
            $
}

\tasknumber{4}\task{
    Воду температурой $t = 70\celsius$ нагрели и превратили в пар при температуре $t_{100} = 100\celsius$,
    потратив $5000\,\text{кДж}$.
    Определите массу воды.
}
\answer{
    $
                Q
                    = cm\Delta t + Lm
                    = m\cbr{c(t_{100} - t) + L}
                \implies
                m = \frac{Q}{c(t_{100} - t) + L}
                    = \frac { 5000\,\text{кДж} }{4200\,\frac{\text{Дж}}{\text{кг}\cdot\text{К}}\cbr{100\celsius - 70\celsius} + 2{,}3\,\frac{\text{МДж}}{\text{кг}}}
                    \approx 2{,}06\,\text{кг}
            $
}

\tasknumber{5}\task{
    Алюминиевое тело температурой $T = 80\celsius$ опустили
    в воду температурой $t = 20\celsius$, масса которой равна массе тела.
    Определите, какая температура установится в сосуде.
}
\answer{
    \begin{align*}
        Q_1 + Q_2 &= 0, \\
        Q_1 &= c_1 m_1 \Delta t_1 = c_1 m (\theta - t_1), \\
        Q_2 &= c_2 m_2 \Delta t_2 = c_2 m (\theta - t_2), \\
        c_1 m (\theta - t_1) + c_2 m (\theta - t_2) &= 0, \\
        c_1 (\theta - t_1) + c_2 (\theta - t_2) &= 0, \\
        c_1 \theta - c_1 t_1 + c_2 \theta - c_2 t_2 &= 0, \\
        (c_1 + c_2)\theta &= c_1 t_1 + c_2 t_2, \\
        \theta &= \frac{c_1 t_1 + c_2 t_2}{c_1 + c_2}
            = \frac{4200\,\frac{\text{Дж}}{\text{кг}\cdot\text{К}} \cdot 20\celsius + 920\,\frac{\text{Дж}}{\text{кг}\cdot\text{К}} \cdot 80\celsius}{4200\,\frac{\text{Дж}}{\text{кг}\cdot\text{К}} + 920\,\frac{\text{Дж}}{\text{кг}\cdot\text{К}}}
            \approx 30.8 \celsius.
    \end{align*}
}

\addpersonalvariant{Мельник Константин}
\tasknumber{1}\task{
    Сколько льда при температуре $0\celsius$ можно расплавить,
    сообщив ему энергию $2\,\text{МДж}$?
    Здесь (и во всех следующих задачах) используйте табличные значения из учебника.
}
\answer{
    $
                Q = \lambda m \implies m
                    = \frac Q{\lambda}
                    = \frac { 2\,\text{МДж} }{ 340\,\frac{\text{кДж}}{\text{кг}} }
                    \approx 5{,}9\,\text{кг}
            $
}

\tasknumber{2}\task{
    Какое количество теплоты выделится при затвердевании $15\,\text{кг}$ расплавленного свинца при температуре плавления?
}
\answer{
    $
                Q
                    = - \lambda m
                    = - 25\,\frac{\text{кДж}}{\text{кг}} \cdot 15\,\text{кг}
                    = - 0{,}4\,\text{МДж} \implies \abs{Q} = 0{,}4\,\text{МДж}
            $
}

\tasknumber{3}\task{
    Какое количество теплоты необходимо для превращения воды массой $4\,\text{кг}$ при $t = 30\celsius$
    в пар при температуре $t_{100} = 100\celsius$?
}
\answer{
    $
                Q
                    = cm\Delta t + Lm
                    = m\cbr{c(t_{100} - t) + L}
                    = 4\,\text{кг} \cdot \cbr{4200\,\frac{\text{Дж}}{\text{кг}\cdot\text{К}}\cbr{100\celsius - 30\celsius} + 2{,}3\,\frac{\text{МДж}}{\text{кг}}}
                    = 10{,}38\,\text{МДж}
            $
}

\tasknumber{4}\task{
    Воду температурой $t = 10\celsius$ нагрели и превратили в пар при температуре $t_{100} = 100\celsius$,
    потратив $2000\,\text{кДж}$.
    Определите массу воды.
}
\answer{
    $
                Q
                    = cm\Delta t + Lm
                    = m\cbr{c(t_{100} - t) + L}
                \implies
                m = \frac{Q}{c(t_{100} - t) + L}
                    = \frac { 2000\,\text{кДж} }{4200\,\frac{\text{Дж}}{\text{кг}\cdot\text{К}}\cbr{100\celsius - 10\celsius} + 2{,}3\,\frac{\text{МДж}}{\text{кг}}}
                    \approx 0{,}75\,\text{кг}
            $
}

\tasknumber{5}\task{
    Стальное тело температурой $T = 80\celsius$ опустили
    в воду температурой $t = 20\celsius$, масса которой равна массе тела.
    Определите, какая температура установится в сосуде.
}
\answer{
    \begin{align*}
        Q_1 + Q_2 &= 0, \\
        Q_1 &= c_1 m_1 \Delta t_1 = c_1 m (\theta - t_1), \\
        Q_2 &= c_2 m_2 \Delta t_2 = c_2 m (\theta - t_2), \\
        c_1 m (\theta - t_1) + c_2 m (\theta - t_2) &= 0, \\
        c_1 (\theta - t_1) + c_2 (\theta - t_2) &= 0, \\
        c_1 \theta - c_1 t_1 + c_2 \theta - c_2 t_2 &= 0, \\
        (c_1 + c_2)\theta &= c_1 t_1 + c_2 t_2, \\
        \theta &= \frac{c_1 t_1 + c_2 t_2}{c_1 + c_2}
            = \frac{4200\,\frac{\text{Дж}}{\text{кг}\cdot\text{К}} \cdot 20\celsius + 500\,\frac{\text{Дж}}{\text{кг}\cdot\text{К}} \cdot 80\celsius}{4200\,\frac{\text{Дж}}{\text{кг}\cdot\text{К}} + 500\,\frac{\text{Дж}}{\text{кг}\cdot\text{К}}}
            \approx 26.4 \celsius.
    \end{align*}
}

\addpersonalvariant{Небоваренков Степан}
\tasknumber{1}\task{
    Сколько льда при температуре $0\celsius$ можно расплавить,
    сообщив ему энергию $7\,\text{МДж}$?
    Здесь (и во всех следующих задачах) используйте табличные значения из учебника.
}
\answer{
    $
                Q = \lambda m \implies m
                    = \frac Q{\lambda}
                    = \frac { 7\,\text{МДж} }{ 340\,\frac{\text{кДж}}{\text{кг}} }
                    \approx 20{,}6\,\text{кг}
            $
}

\tasknumber{2}\task{
    Какое количество теплоты выделится при затвердевании $75\,\text{кг}$ расплавленного меди при температуре плавления?
}
\answer{
    $
                Q
                    = - \lambda m
                    = - 210\,\frac{\text{кДж}}{\text{кг}} \cdot 75\,\text{кг}
                    = - 15{,}8\,\text{МДж} \implies \abs{Q} = 15{,}8\,\text{МДж}
            $
}

\tasknumber{3}\task{
    Какое количество теплоты необходимо для превращения воды массой $5\,\text{кг}$ при $t = 70\celsius$
    в пар при температуре $t_{100} = 100\celsius$?
}
\answer{
    $
                Q
                    = cm\Delta t + Lm
                    = m\cbr{c(t_{100} - t) + L}
                    = 5\,\text{кг} \cdot \cbr{4200\,\frac{\text{Дж}}{\text{кг}\cdot\text{К}}\cbr{100\celsius - 70\celsius} + 2{,}3\,\frac{\text{МДж}}{\text{кг}}}
                    = 12{,}13\,\text{МДж}
            $
}

\tasknumber{4}\task{
    Воду температурой $t = 50\celsius$ нагрели и превратили в пар при температуре $t_{100} = 100\celsius$,
    потратив $5000\,\text{кДж}$.
    Определите массу воды.
}
\answer{
    $
                Q
                    = cm\Delta t + Lm
                    = m\cbr{c(t_{100} - t) + L}
                \implies
                m = \frac{Q}{c(t_{100} - t) + L}
                    = \frac { 5000\,\text{кДж} }{4200\,\frac{\text{Дж}}{\text{кг}\cdot\text{К}}\cbr{100\celsius - 50\celsius} + 2{,}3\,\frac{\text{МДж}}{\text{кг}}}
                    \approx 1{,}99\,\text{кг}
            $
}

\tasknumber{5}\task{
    Алюминиевое тело температурой $T = 70\celsius$ опустили
    в воду температурой $t = 10\celsius$, масса которой равна массе тела.
    Определите, какая температура установится в сосуде.
}
\answer{
    \begin{align*}
        Q_1 + Q_2 &= 0, \\
        Q_1 &= c_1 m_1 \Delta t_1 = c_1 m (\theta - t_1), \\
        Q_2 &= c_2 m_2 \Delta t_2 = c_2 m (\theta - t_2), \\
        c_1 m (\theta - t_1) + c_2 m (\theta - t_2) &= 0, \\
        c_1 (\theta - t_1) + c_2 (\theta - t_2) &= 0, \\
        c_1 \theta - c_1 t_1 + c_2 \theta - c_2 t_2 &= 0, \\
        (c_1 + c_2)\theta &= c_1 t_1 + c_2 t_2, \\
        \theta &= \frac{c_1 t_1 + c_2 t_2}{c_1 + c_2}
            = \frac{4200\,\frac{\text{Дж}}{\text{кг}\cdot\text{К}} \cdot 10\celsius + 920\,\frac{\text{Дж}}{\text{кг}\cdot\text{К}} \cdot 70\celsius}{4200\,\frac{\text{Дж}}{\text{кг}\cdot\text{К}} + 920\,\frac{\text{Дж}}{\text{кг}\cdot\text{К}}}
            \approx 20.8 \celsius.
    \end{align*}
}

\addpersonalvariant{Неретин Матвей}
\tasknumber{1}\task{
    Сколько льда при температуре $0\celsius$ можно расплавить,
    сообщив ему энергию $6\,\text{МДж}$?
    Здесь (и во всех следующих задачах) используйте табличные значения из учебника.
}
\answer{
    $
                Q = \lambda m \implies m
                    = \frac Q{\lambda}
                    = \frac { 6\,\text{МДж} }{ 340\,\frac{\text{кДж}}{\text{кг}} }
                    \approx 17{,}6\,\text{кг}
            $
}

\tasknumber{2}\task{
    Какое количество теплоты выделится при затвердевании $75\,\text{кг}$ расплавленного свинца при температуре плавления?
}
\answer{
    $
                Q
                    = - \lambda m
                    = - 25\,\frac{\text{кДж}}{\text{кг}} \cdot 75\,\text{кг}
                    = - 1{,}9\,\text{МДж} \implies \abs{Q} = 1{,}9\,\text{МДж}
            $
}

\tasknumber{3}\task{
    Какое количество теплоты необходимо для превращения воды массой $4\,\text{кг}$ при $t = 40\celsius$
    в пар при температуре $t_{100} = 100\celsius$?
}
\answer{
    $
                Q
                    = cm\Delta t + Lm
                    = m\cbr{c(t_{100} - t) + L}
                    = 4\,\text{кг} \cdot \cbr{4200\,\frac{\text{Дж}}{\text{кг}\cdot\text{К}}\cbr{100\celsius - 40\celsius} + 2{,}3\,\frac{\text{МДж}}{\text{кг}}}
                    = 10{,}21\,\text{МДж}
            $
}

\tasknumber{4}\task{
    Воду температурой $t = 40\celsius$ нагрели и превратили в пар при температуре $t_{100} = 100\celsius$,
    потратив $5000\,\text{кДж}$.
    Определите массу воды.
}
\answer{
    $
                Q
                    = cm\Delta t + Lm
                    = m\cbr{c(t_{100} - t) + L}
                \implies
                m = \frac{Q}{c(t_{100} - t) + L}
                    = \frac { 5000\,\text{кДж} }{4200\,\frac{\text{Дж}}{\text{кг}\cdot\text{К}}\cbr{100\celsius - 40\celsius} + 2{,}3\,\frac{\text{МДж}}{\text{кг}}}
                    \approx 1{,}96\,\text{кг}
            $
}

\tasknumber{5}\task{
    Цинковое тело температурой $T = 80\celsius$ опустили
    в воду температурой $t = 10\celsius$, масса которой равна массе тела.
    Определите, какая температура установится в сосуде.
}
\answer{
    \begin{align*}
        Q_1 + Q_2 &= 0, \\
        Q_1 &= c_1 m_1 \Delta t_1 = c_1 m (\theta - t_1), \\
        Q_2 &= c_2 m_2 \Delta t_2 = c_2 m (\theta - t_2), \\
        c_1 m (\theta - t_1) + c_2 m (\theta - t_2) &= 0, \\
        c_1 (\theta - t_1) + c_2 (\theta - t_2) &= 0, \\
        c_1 \theta - c_1 t_1 + c_2 \theta - c_2 t_2 &= 0, \\
        (c_1 + c_2)\theta &= c_1 t_1 + c_2 t_2, \\
        \theta &= \frac{c_1 t_1 + c_2 t_2}{c_1 + c_2}
            = \frac{4200\,\frac{\text{Дж}}{\text{кг}\cdot\text{К}} \cdot 10\celsius + 400\,\frac{\text{Дж}}{\text{кг}\cdot\text{К}} \cdot 80\celsius}{4200\,\frac{\text{Дж}}{\text{кг}\cdot\text{К}} + 400\,\frac{\text{Дж}}{\text{кг}\cdot\text{К}}}
            \approx 16.1 \celsius.
    \end{align*}
}

\addpersonalvariant{Никонова Мария}
\tasknumber{1}\task{
    Сколько льда при температуре $0\celsius$ можно расплавить,
    сообщив ему энергию $4\,\text{МДж}$?
    Здесь (и во всех следующих задачах) используйте табличные значения из учебника.
}
\answer{
    $
                Q = \lambda m \implies m
                    = \frac Q{\lambda}
                    = \frac { 4\,\text{МДж} }{ 340\,\frac{\text{кДж}}{\text{кг}} }
                    \approx 11{,}8\,\text{кг}
            $
}

\tasknumber{2}\task{
    Какое количество теплоты выделится при затвердевании $25\,\text{кг}$ расплавленного алюминия при температуре плавления?
}
\answer{
    $
                Q
                    = - \lambda m
                    = - 390\,\frac{\text{кДж}}{\text{кг}} \cdot 25\,\text{кг}
                    = - 9{,}8\,\text{МДж} \implies \abs{Q} = 9{,}8\,\text{МДж}
            $
}

\tasknumber{3}\task{
    Какое количество теплоты необходимо для превращения воды массой $15\,\text{кг}$ при $t = 30\celsius$
    в пар при температуре $t_{100} = 100\celsius$?
}
\answer{
    $
                Q
                    = cm\Delta t + Lm
                    = m\cbr{c(t_{100} - t) + L}
                    = 15\,\text{кг} \cdot \cbr{4200\,\frac{\text{Дж}}{\text{кг}\cdot\text{К}}\cbr{100\celsius - 30\celsius} + 2{,}3\,\frac{\text{МДж}}{\text{кг}}}
                    = 38{,}91\,\text{МДж}
            $
}

\tasknumber{4}\task{
    Воду температурой $t = 60\celsius$ нагрели и превратили в пар при температуре $t_{100} = 100\celsius$,
    потратив $2500\,\text{кДж}$.
    Определите массу воды.
}
\answer{
    $
                Q
                    = cm\Delta t + Lm
                    = m\cbr{c(t_{100} - t) + L}
                \implies
                m = \frac{Q}{c(t_{100} - t) + L}
                    = \frac { 2500\,\text{кДж} }{4200\,\frac{\text{Дж}}{\text{кг}\cdot\text{К}}\cbr{100\celsius - 60\celsius} + 2{,}3\,\frac{\text{МДж}}{\text{кг}}}
                    \approx 1{,}01\,\text{кг}
            $
}

\tasknumber{5}\task{
    Алюминиевое тело температурой $T = 70\celsius$ опустили
    в воду температурой $t = 30\celsius$, масса которой равна массе тела.
    Определите, какая температура установится в сосуде.
}
\answer{
    \begin{align*}
        Q_1 + Q_2 &= 0, \\
        Q_1 &= c_1 m_1 \Delta t_1 = c_1 m (\theta - t_1), \\
        Q_2 &= c_2 m_2 \Delta t_2 = c_2 m (\theta - t_2), \\
        c_1 m (\theta - t_1) + c_2 m (\theta - t_2) &= 0, \\
        c_1 (\theta - t_1) + c_2 (\theta - t_2) &= 0, \\
        c_1 \theta - c_1 t_1 + c_2 \theta - c_2 t_2 &= 0, \\
        (c_1 + c_2)\theta &= c_1 t_1 + c_2 t_2, \\
        \theta &= \frac{c_1 t_1 + c_2 t_2}{c_1 + c_2}
            = \frac{4200\,\frac{\text{Дж}}{\text{кг}\cdot\text{К}} \cdot 30\celsius + 920\,\frac{\text{Дж}}{\text{кг}\cdot\text{К}} \cdot 70\celsius}{4200\,\frac{\text{Дж}}{\text{кг}\cdot\text{К}} + 920\,\frac{\text{Дж}}{\text{кг}\cdot\text{К}}}
            \approx 37.2 \celsius.
    \end{align*}
}

\addpersonalvariant{Палаткин Даниил}
\tasknumber{1}\task{
    Сколько льда при температуре $0\celsius$ можно расплавить,
    сообщив ему энергию $8\,\text{МДж}$?
    Здесь (и во всех следующих задачах) используйте табличные значения из учебника.
}
\answer{
    $
                Q = \lambda m \implies m
                    = \frac Q{\lambda}
                    = \frac { 8\,\text{МДж} }{ 340\,\frac{\text{кДж}}{\text{кг}} }
                    \approx 23{,}5\,\text{кг}
            $
}

\tasknumber{2}\task{
    Какое количество теплоты выделится при затвердевании $20\,\text{кг}$ расплавленного алюминия при температуре плавления?
}
\answer{
    $
                Q
                    = - \lambda m
                    = - 390\,\frac{\text{кДж}}{\text{кг}} \cdot 20\,\text{кг}
                    = - 7{,}8\,\text{МДж} \implies \abs{Q} = 7{,}8\,\text{МДж}
            $
}

\tasknumber{3}\task{
    Какое количество теплоты необходимо для превращения воды массой $5\,\text{кг}$ при $t = 40\celsius$
    в пар при температуре $t_{100} = 100\celsius$?
}
\answer{
    $
                Q
                    = cm\Delta t + Lm
                    = m\cbr{c(t_{100} - t) + L}
                    = 5\,\text{кг} \cdot \cbr{4200\,\frac{\text{Дж}}{\text{кг}\cdot\text{К}}\cbr{100\celsius - 40\celsius} + 2{,}3\,\frac{\text{МДж}}{\text{кг}}}
                    = 12{,}76\,\text{МДж}
            $
}

\tasknumber{4}\task{
    Воду температурой $t = 10\celsius$ нагрели и превратили в пар при температуре $t_{100} = 100\celsius$,
    потратив $2500\,\text{кДж}$.
    Определите массу воды.
}
\answer{
    $
                Q
                    = cm\Delta t + Lm
                    = m\cbr{c(t_{100} - t) + L}
                \implies
                m = \frac{Q}{c(t_{100} - t) + L}
                    = \frac { 2500\,\text{кДж} }{4200\,\frac{\text{Дж}}{\text{кг}\cdot\text{К}}\cbr{100\celsius - 10\celsius} + 2{,}3\,\frac{\text{МДж}}{\text{кг}}}
                    \approx 0{,}93\,\text{кг}
            $
}

\tasknumber{5}\task{
    Цинковое тело температурой $T = 90\celsius$ опустили
    в воду температурой $t = 10\celsius$, масса которой равна массе тела.
    Определите, какая температура установится в сосуде.
}
\answer{
    \begin{align*}
        Q_1 + Q_2 &= 0, \\
        Q_1 &= c_1 m_1 \Delta t_1 = c_1 m (\theta - t_1), \\
        Q_2 &= c_2 m_2 \Delta t_2 = c_2 m (\theta - t_2), \\
        c_1 m (\theta - t_1) + c_2 m (\theta - t_2) &= 0, \\
        c_1 (\theta - t_1) + c_2 (\theta - t_2) &= 0, \\
        c_1 \theta - c_1 t_1 + c_2 \theta - c_2 t_2 &= 0, \\
        (c_1 + c_2)\theta &= c_1 t_1 + c_2 t_2, \\
        \theta &= \frac{c_1 t_1 + c_2 t_2}{c_1 + c_2}
            = \frac{4200\,\frac{\text{Дж}}{\text{кг}\cdot\text{К}} \cdot 10\celsius + 400\,\frac{\text{Дж}}{\text{кг}\cdot\text{К}} \cdot 90\celsius}{4200\,\frac{\text{Дж}}{\text{кг}\cdot\text{К}} + 400\,\frac{\text{Дж}}{\text{кг}\cdot\text{К}}}
            \approx 17.0 \celsius.
    \end{align*}
}

\addpersonalvariant{Пичугин Илья}
\tasknumber{1}\task{
    Сколько льда при температуре $0\celsius$ можно расплавить,
    сообщив ему энергию $3\,\text{МДж}$?
    Здесь (и во всех следующих задачах) используйте табличные значения из учебника.
}
\answer{
    $
                Q = \lambda m \implies m
                    = \frac Q{\lambda}
                    = \frac { 3\,\text{МДж} }{ 340\,\frac{\text{кДж}}{\text{кг}} }
                    \approx 8{,}8\,\text{кг}
            $
}

\tasknumber{2}\task{
    Какое количество теплоты выделится при затвердевании $50\,\text{кг}$ расплавленного меди при температуре плавления?
}
\answer{
    $
                Q
                    = - \lambda m
                    = - 210\,\frac{\text{кДж}}{\text{кг}} \cdot 50\,\text{кг}
                    = - 10{,}5\,\text{МДж} \implies \abs{Q} = 10{,}5\,\text{МДж}
            $
}

\tasknumber{3}\task{
    Какое количество теплоты необходимо для превращения воды массой $2\,\text{кг}$ при $t = 20\celsius$
    в пар при температуре $t_{100} = 100\celsius$?
}
\answer{
    $
                Q
                    = cm\Delta t + Lm
                    = m\cbr{c(t_{100} - t) + L}
                    = 2\,\text{кг} \cdot \cbr{4200\,\frac{\text{Дж}}{\text{кг}\cdot\text{К}}\cbr{100\celsius - 20\celsius} + 2{,}3\,\frac{\text{МДж}}{\text{кг}}}
                    = 5{,}27\,\text{МДж}
            $
}

\tasknumber{4}\task{
    Воду температурой $t = 70\celsius$ нагрели и превратили в пар при температуре $t_{100} = 100\celsius$,
    потратив $4000\,\text{кДж}$.
    Определите массу воды.
}
\answer{
    $
                Q
                    = cm\Delta t + Lm
                    = m\cbr{c(t_{100} - t) + L}
                \implies
                m = \frac{Q}{c(t_{100} - t) + L}
                    = \frac { 4000\,\text{кДж} }{4200\,\frac{\text{Дж}}{\text{кг}\cdot\text{К}}\cbr{100\celsius - 70\celsius} + 2{,}3\,\frac{\text{МДж}}{\text{кг}}}
                    \approx 1{,}65\,\text{кг}
            $
}

\tasknumber{5}\task{
    Цинковое тело температурой $T = 70\celsius$ опустили
    в воду температурой $t = 20\celsius$, масса которой равна массе тела.
    Определите, какая температура установится в сосуде.
}
\answer{
    \begin{align*}
        Q_1 + Q_2 &= 0, \\
        Q_1 &= c_1 m_1 \Delta t_1 = c_1 m (\theta - t_1), \\
        Q_2 &= c_2 m_2 \Delta t_2 = c_2 m (\theta - t_2), \\
        c_1 m (\theta - t_1) + c_2 m (\theta - t_2) &= 0, \\
        c_1 (\theta - t_1) + c_2 (\theta - t_2) &= 0, \\
        c_1 \theta - c_1 t_1 + c_2 \theta - c_2 t_2 &= 0, \\
        (c_1 + c_2)\theta &= c_1 t_1 + c_2 t_2, \\
        \theta &= \frac{c_1 t_1 + c_2 t_2}{c_1 + c_2}
            = \frac{4200\,\frac{\text{Дж}}{\text{кг}\cdot\text{К}} \cdot 20\celsius + 400\,\frac{\text{Дж}}{\text{кг}\cdot\text{К}} \cdot 70\celsius}{4200\,\frac{\text{Дж}}{\text{кг}\cdot\text{К}} + 400\,\frac{\text{Дж}}{\text{кг}\cdot\text{К}}}
            \approx 24.3 \celsius.
    \end{align*}
}

\addpersonalvariant{Стратонников Илья}
\tasknumber{1}\task{
    Сколько льда при температуре $0\celsius$ можно расплавить,
    сообщив ему энергию $5\,\text{МДж}$?
    Здесь (и во всех следующих задачах) используйте табличные значения из учебника.
}
\answer{
    $
                Q = \lambda m \implies m
                    = \frac Q{\lambda}
                    = \frac { 5\,\text{МДж} }{ 340\,\frac{\text{кДж}}{\text{кг}} }
                    \approx 14{,}7\,\text{кг}
            $
}

\tasknumber{2}\task{
    Какое количество теплоты выделится при затвердевании $25\,\text{кг}$ расплавленного стали при температуре плавления?
}
\answer{
    $
                Q
                    = - \lambda m
                    = - 84\,\frac{\text{кДж}}{\text{кг}} \cdot 25\,\text{кг}
                    = - 2{,}1\,\text{МДж} \implies \abs{Q} = 2{,}1\,\text{МДж}
            $
}

\tasknumber{3}\task{
    Какое количество теплоты необходимо для превращения воды массой $2\,\text{кг}$ при $t = 30\celsius$
    в пар при температуре $t_{100} = 100\celsius$?
}
\answer{
    $
                Q
                    = cm\Delta t + Lm
                    = m\cbr{c(t_{100} - t) + L}
                    = 2\,\text{кг} \cdot \cbr{4200\,\frac{\text{Дж}}{\text{кг}\cdot\text{К}}\cbr{100\celsius - 30\celsius} + 2{,}3\,\frac{\text{МДж}}{\text{кг}}}
                    = 5{,}19\,\text{МДж}
            $
}

\tasknumber{4}\task{
    Воду температурой $t = 70\celsius$ нагрели и превратили в пар при температуре $t_{100} = 100\celsius$,
    потратив $2500\,\text{кДж}$.
    Определите массу воды.
}
\answer{
    $
                Q
                    = cm\Delta t + Lm
                    = m\cbr{c(t_{100} - t) + L}
                \implies
                m = \frac{Q}{c(t_{100} - t) + L}
                    = \frac { 2500\,\text{кДж} }{4200\,\frac{\text{Дж}}{\text{кг}\cdot\text{К}}\cbr{100\celsius - 70\celsius} + 2{,}3\,\frac{\text{МДж}}{\text{кг}}}
                    \approx 1{,}03\,\text{кг}
            $
}

\tasknumber{5}\task{
    Цинковое тело температурой $T = 100\celsius$ опустили
    в воду температурой $t = 20\celsius$, масса которой равна массе тела.
    Определите, какая температура установится в сосуде.
}
\answer{
    \begin{align*}
        Q_1 + Q_2 &= 0, \\
        Q_1 &= c_1 m_1 \Delta t_1 = c_1 m (\theta - t_1), \\
        Q_2 &= c_2 m_2 \Delta t_2 = c_2 m (\theta - t_2), \\
        c_1 m (\theta - t_1) + c_2 m (\theta - t_2) &= 0, \\
        c_1 (\theta - t_1) + c_2 (\theta - t_2) &= 0, \\
        c_1 \theta - c_1 t_1 + c_2 \theta - c_2 t_2 &= 0, \\
        (c_1 + c_2)\theta &= c_1 t_1 + c_2 t_2, \\
        \theta &= \frac{c_1 t_1 + c_2 t_2}{c_1 + c_2}
            = \frac{4200\,\frac{\text{Дж}}{\text{кг}\cdot\text{К}} \cdot 20\celsius + 400\,\frac{\text{Дж}}{\text{кг}\cdot\text{К}} \cdot 100\celsius}{4200\,\frac{\text{Дж}}{\text{кг}\cdot\text{К}} + 400\,\frac{\text{Дж}}{\text{кг}\cdot\text{К}}}
            \approx 27.0 \celsius.
    \end{align*}
}

\addpersonalvariant{Федотова Дарья}
\tasknumber{1}\task{
    Сколько льда при температуре $0\celsius$ можно расплавить,
    сообщив ему энергию $6\,\text{МДж}$?
    Здесь (и во всех следующих задачах) используйте табличные значения из учебника.
}
\answer{
    $
                Q = \lambda m \implies m
                    = \frac Q{\lambda}
                    = \frac { 6\,\text{МДж} }{ 340\,\frac{\text{кДж}}{\text{кг}} }
                    \approx 17{,}6\,\text{кг}
            $
}

\tasknumber{2}\task{
    Какое количество теплоты выделится при затвердевании $20\,\text{кг}$ расплавленного свинца при температуре плавления?
}
\answer{
    $
                Q
                    = - \lambda m
                    = - 25\,\frac{\text{кДж}}{\text{кг}} \cdot 20\,\text{кг}
                    = - 0{,}5\,\text{МДж} \implies \abs{Q} = 0{,}5\,\text{МДж}
            $
}

\tasknumber{3}\task{
    Какое количество теплоты необходимо для превращения воды массой $15\,\text{кг}$ при $t = 40\celsius$
    в пар при температуре $t_{100} = 100\celsius$?
}
\answer{
    $
                Q
                    = cm\Delta t + Lm
                    = m\cbr{c(t_{100} - t) + L}
                    = 15\,\text{кг} \cdot \cbr{4200\,\frac{\text{Дж}}{\text{кг}\cdot\text{К}}\cbr{100\celsius - 40\celsius} + 2{,}3\,\frac{\text{МДж}}{\text{кг}}}
                    = 38{,}28\,\text{МДж}
            $
}

\tasknumber{4}\task{
    Воду температурой $t = 60\celsius$ нагрели и превратили в пар при температуре $t_{100} = 100\celsius$,
    потратив $2000\,\text{кДж}$.
    Определите массу воды.
}
\answer{
    $
                Q
                    = cm\Delta t + Lm
                    = m\cbr{c(t_{100} - t) + L}
                \implies
                m = \frac{Q}{c(t_{100} - t) + L}
                    = \frac { 2000\,\text{кДж} }{4200\,\frac{\text{Дж}}{\text{кг}\cdot\text{К}}\cbr{100\celsius - 60\celsius} + 2{,}3\,\frac{\text{МДж}}{\text{кг}}}
                    \approx 0{,}81\,\text{кг}
            $
}

\tasknumber{5}\task{
    Алюминиевое тело температурой $T = 100\celsius$ опустили
    в воду температурой $t = 20\celsius$, масса которой равна массе тела.
    Определите, какая температура установится в сосуде.
}
\answer{
    \begin{align*}
        Q_1 + Q_2 &= 0, \\
        Q_1 &= c_1 m_1 \Delta t_1 = c_1 m (\theta - t_1), \\
        Q_2 &= c_2 m_2 \Delta t_2 = c_2 m (\theta - t_2), \\
        c_1 m (\theta - t_1) + c_2 m (\theta - t_2) &= 0, \\
        c_1 (\theta - t_1) + c_2 (\theta - t_2) &= 0, \\
        c_1 \theta - c_1 t_1 + c_2 \theta - c_2 t_2 &= 0, \\
        (c_1 + c_2)\theta &= c_1 t_1 + c_2 t_2, \\
        \theta &= \frac{c_1 t_1 + c_2 t_2}{c_1 + c_2}
            = \frac{4200\,\frac{\text{Дж}}{\text{кг}\cdot\text{К}} \cdot 20\celsius + 920\,\frac{\text{Дж}}{\text{кг}\cdot\text{К}} \cdot 100\celsius}{4200\,\frac{\text{Дж}}{\text{кг}\cdot\text{К}} + 920\,\frac{\text{Дж}}{\text{кг}\cdot\text{К}}}
            \approx 34.4 \celsius.
    \end{align*}
}

\addpersonalvariant{Шустов Иван}
\tasknumber{1}\task{
    Сколько льда при температуре $0\celsius$ можно расплавить,
    сообщив ему энергию $4\,\text{МДж}$?
    Здесь (и во всех следующих задачах) используйте табличные значения из учебника.
}
\answer{
    $
                Q = \lambda m \implies m
                    = \frac Q{\lambda}
                    = \frac { 4\,\text{МДж} }{ 340\,\frac{\text{кДж}}{\text{кг}} }
                    \approx 11{,}8\,\text{кг}
            $
}

\tasknumber{2}\task{
    Какое количество теплоты выделится при затвердевании $15\,\text{кг}$ расплавленного алюминия при температуре плавления?
}
\answer{
    $
                Q
                    = - \lambda m
                    = - 390\,\frac{\text{кДж}}{\text{кг}} \cdot 15\,\text{кг}
                    = - 5{,}8\,\text{МДж} \implies \abs{Q} = 5{,}8\,\text{МДж}
            $
}

\tasknumber{3}\task{
    Какое количество теплоты необходимо для превращения воды массой $4\,\text{кг}$ при $t = 20\celsius$
    в пар при температуре $t_{100} = 100\celsius$?
}
\answer{
    $
                Q
                    = cm\Delta t + Lm
                    = m\cbr{c(t_{100} - t) + L}
                    = 4\,\text{кг} \cdot \cbr{4200\,\frac{\text{Дж}}{\text{кг}\cdot\text{К}}\cbr{100\celsius - 20\celsius} + 2{,}3\,\frac{\text{МДж}}{\text{кг}}}
                    = 10{,}54\,\text{МДж}
            $
}

\tasknumber{4}\task{
    Воду температурой $t = 40\celsius$ нагрели и превратили в пар при температуре $t_{100} = 100\celsius$,
    потратив $2000\,\text{кДж}$.
    Определите массу воды.
}
\answer{
    $
                Q
                    = cm\Delta t + Lm
                    = m\cbr{c(t_{100} - t) + L}
                \implies
                m = \frac{Q}{c(t_{100} - t) + L}
                    = \frac { 2000\,\text{кДж} }{4200\,\frac{\text{Дж}}{\text{кг}\cdot\text{К}}\cbr{100\celsius - 40\celsius} + 2{,}3\,\frac{\text{МДж}}{\text{кг}}}
                    \approx 0{,}78\,\text{кг}
            $
}

\tasknumber{5}\task{
    Цинковое тело температурой $T = 80\celsius$ опустили
    в воду температурой $t = 30\celsius$, масса которой равна массе тела.
    Определите, какая температура установится в сосуде.
}
\answer{
    \begin{align*}
        Q_1 + Q_2 &= 0, \\
        Q_1 &= c_1 m_1 \Delta t_1 = c_1 m (\theta - t_1), \\
        Q_2 &= c_2 m_2 \Delta t_2 = c_2 m (\theta - t_2), \\
        c_1 m (\theta - t_1) + c_2 m (\theta - t_2) &= 0, \\
        c_1 (\theta - t_1) + c_2 (\theta - t_2) &= 0, \\
        c_1 \theta - c_1 t_1 + c_2 \theta - c_2 t_2 &= 0, \\
        (c_1 + c_2)\theta &= c_1 t_1 + c_2 t_2, \\
        \theta &= \frac{c_1 t_1 + c_2 t_2}{c_1 + c_2}
            = \frac{4200\,\frac{\text{Дж}}{\text{кг}\cdot\text{К}} \cdot 30\celsius + 400\,\frac{\text{Дж}}{\text{кг}\cdot\text{К}} \cdot 80\celsius}{4200\,\frac{\text{Дж}}{\text{кг}\cdot\text{К}} + 400\,\frac{\text{Дж}}{\text{кг}\cdot\text{К}}}
            \approx 34.3 \celsius.
    \end{align*}
}

\end{document}
% autogenerated
