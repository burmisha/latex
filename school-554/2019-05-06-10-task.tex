\documentclass[12pt,a4paper]{amsart}%DVI-mode.
\usepackage{graphics,graphicx,epsfig}%DVI-mode.
% \documentclass[pdftex,12pt]{amsart} %PDF-mode.
% \usepackage[pdftex]{graphicx}       %PDF-mode.
% \usepackage[babel=true]{microtype}
% \usepackage[T1]{fontenc}
% \usepackage{lmodern}

\usepackage{cmap}
%\usepackage{a4wide}                 % Fit the text to A4 page tightly.
% \usepackage[utf8]{inputenc}
\usepackage[T2A]{fontenc}
\usepackage[english,russian]{babel} % Download Russian fonts.
\usepackage{amsmath,amsfonts,amssymb,amsthm,amscd,mathrsfs} % Use AMS symbols.
\usepackage{tikz}
\usetikzlibrary{circuits.ee.IEC}
\usetikzlibrary{shapes.geometric}
\usetikzlibrary{decorations.markings}
%\usetikzlibrary{dashs}
%\usetikzlibrary{info}


\textheight=28cm % высота текста
\textwidth=18cm % ширина текста
\topmargin=-2.5cm % отступ от верхнего края
\parskip=2pt % интервал между абзацами
\oddsidemargin=-1.5cm
\evensidemargin=-1.5cm 

\parindent=0pt % абзацный отступ
\tolerance=500 % терпимость к "жидким" строкам
\binoppenalty=10000 % штраф за перенос формул - 10000 - абсолютный запрет
\relpenalty=10000
\flushbottom % выравнивание высоты страниц
\pagenumbering{gobble}

\newcommand\bivec[2]{\begin{pmatrix} #1 \\ #2 \end{pmatrix}}

\newcommand\ol[1]{\overline{#1}}

\newcommand\p[1]{\Prob\!\left(#1\right)}
\newcommand\e[1]{\mathsf{E}\!\left(#1\right)}
\newcommand\disp[1]{\mathsf{D}\!\left(#1\right)}
%\newcommand\norm[2]{\mathcal{N}\!\cbr{#1,#2}}
\newcommand\sign{\text{ sign }}

\newcommand\al[1]{\begin{align*} #1 \end{align*}}
\newcommand\begcas[1]{\begin{cases}#1\end{cases}}
\newcommand\tab[2]{	\vspace{-#1pt}
						\begin{tabbing} 
						#2
						\end{tabbing}
					\vspace{-#1pt}
					}

\newcommand\maintext[1]{{\bfseries\sffamily{#1}}}
\newcommand\skipped[1]{ {\ensuremath{\text{\small{\sffamily{Пропущено:} #1} } } } }
\newcommand\simpletitle[1]{\begin{center} \maintext{#1} \end{center}}

\def\le{\leqslant}
\def\ge{\geqslant}
\def\Ell{\mathcal{L}}
\def\eps{{\varepsilon}}
\def\Rn{\mathbb{R}^n}
\def\RSS{\mathsf{RSS}}

\newcommand\foral[1]{\forall\,#1\:}
\newcommand\exist[1]{\exists\,#1\:\colon}

\newcommand\cbr[1]{\left(#1\right)} %circled brackets
\newcommand\fbr[1]{\left\{#1\right\}} %figure brackets
\newcommand\sbr[1]{\left[#1\right]} %square brackets
\newcommand\modul[1]{\left|#1\right|}

\newcommand\sqr[1]{\cbr{#1}^2}
\newcommand\inv[1]{\cbr{#1}^{-1}}

\newcommand\cdf[2]{\cdot\frac{#1}{#2}}
\newcommand\dd[2]{\frac{\partial#1}{\partial#2}}

\newcommand\integr[2]{\int\limits_{#1}^{#2}}
\newcommand\suml[2]{\sum\limits_{#1}^{#2}}
\newcommand\isum[2]{\sum\limits_{#1=#2}^{+\infty}}
\newcommand\idots[3]{#1_{#2},\ldots,#1_{#3}}
\newcommand\fdots[5]{#4{#1_{#2}}#5\ldots#5#4{#1_{#3}}}

\newcommand\obol[1]{O\!\cbr{#1}}
\newcommand\omal[1]{o\!\cbr{#1}}

\newcommand\addeps[2]{
	\begin{figure} [!ht] %lrp
		\centering
		\includegraphics[height=320px]{#1.eps}
		\vspace{-10pt}
		\caption{#2}
		\label{eps:#1}
	\end{figure}
}

\newcommand\addepssize[3]{
	\begin{figure} [!ht] %lrp hp
		\centering
		\includegraphics[height=#3px]{#1.eps}
		\vspace{-10pt}
		\caption{#2}
		\label{eps:#1}
	\end{figure}
}


\newcommand\norm[1]{\ensuremath{\left\|{#1}\right\|}}
\newcommand\ort{\bot}
\newcommand\theorem[1]{{\sffamily Теорема #1\ }}
\newcommand\lemma[1]{{\sffamily Лемма #1\ }}
\newcommand\difflim[2]{\frac{#1\cbr{#2 + \Delta#2} - #1\cbr{#2}}{\Delta #2}}
\renewcommand\proof[1]{\par\noindent$\square$ #1 \hfill$\blacksquare$\par}
\newcommand\defenition[1]{{\sffamilyОпределение #1\ }}

% \begin{document}
% %\raggedright
% \addclassdate{7}{20 апреля 2018}

\task 1
Площадь большого поршня гидравлического домкрата $S_1 = 20\units{см}^2$, а малого $S_2 = 0{,}5\units{см}^2.$ Груз какой максимальной массы можно поднять этим домкратом, если на малый поршень давить с силой не более $F=200\units{Н}?$ Силой трения от стенки цилиндров пренебречь.

\task 2
В сосуд налита вода. Расстояние от поверхности воды до дна $H = 0{,}5\units{м},$ площадь дна $S = 0{,}1\units{м}^2.$ Найти гидростатическое давление $P_1$ и полное давление $P_2$ вблизи дна. Найти силу давления воды на дно. Плотность воды \rhowater

\task 3
На лёгкий поршень площадью $S=900\units{см}^2,$ касающийся поверхности воды, поставили гирю массы $m=3\units{кг}$. Высота слоя воды в сосуде с вертикальными стенками $H = 20\units{см}$. Определить давление жидкости вблизи дна, если плотность воды \rhowater

\task 4
Давление газов в конце сгорания в цилиндре дизельного двигателя трактора $P = 9\units{МПа}.$ Диаметр цилиндра $d = 130\units{мм}.$ С какой силой газы давят на поршень в цилиндре? Площадь круга диаметром $D$ равна $S = \cfrac{\pi D^2}4.$

\task 5
Площадь малого поршня гидравлического подъёмника $S_1 = 0{,}8\units{см}^2$, а большого $S_2 = 40\units{см}^2.$ Какую силу $F$ надо приложить к малому поршню, чтобы поднять груз весом $P = 8\units{кН}?$

\task 6
Герметичный сосуд полностью заполнен водой и стоит на столе. На небольшой поршень площадью $S$ давят рукой с силой $F$. Поршень находится ниже крышки сосуда на $H_1$, выше дна на $H_2$ и может свободно перемещаться. Плотность воды $\rho$, атмосферное давление $P_A$. Найти давления $P_1$ и $P_2$ в воде вблизи крышки и дна сосуда.
\\ \\
\addclassdate{7}{20 апреля 2018}

\task 1
Площадь большого поршня гидравлического домкрата $S_1 = 20\units{см}^2$, а малого $S_2 = 0{,}5\units{см}^2.$ Груз какой максимальной массы можно поднять этим домкратом, если на малый поршень давить с силой не более $F=200\units{Н}?$ Силой трения от стенки цилиндров пренебречь.

\task 2
В сосуд налита вода. Расстояние от поверхности воды до дна $H = 0{,}5\units{м},$ площадь дна $S = 0{,}1\units{м}^2.$ Найти гидростатическое давление $P_1$ и полное давление $P_2$ вблизи дна. Найти силу давления воды на дно. Плотность воды \rhowater

\task 3
На лёгкий поршень площадью $S=900\units{см}^2,$ касающийся поверхности воды, поставили гирю массы $m=3\units{кг}$. Высота слоя воды в сосуде с вертикальными стенками $H = 20\units{см}$. Определить давление жидкости вблизи дна, если плотность воды \rhowater

\task 4
Давление газов в конце сгорания в цилиндре дизельного двигателя трактора $P = 9\units{МПа}.$ Диаметр цилиндра $d = 130\units{мм}.$ С какой силой газы давят на поршень в цилиндре? Площадь круга диаметром $D$ равна $S = \cfrac{\pi D^2}4.$

\task 5
Площадь малого поршня гидравлического подъёмника $S_1 = 0{,}8\units{см}^2$, а большого $S_2 = 40\units{см}^2.$ Какую силу $F$ надо приложить к малому поршню, чтобы поднять груз весом $P = 8\units{кН}?$

\task 6
Герметичный сосуд полностью заполнен водой и стоит на столе. На небольшой поршень площадью $S$ давят рукой с силой $F$. Поршень находится ниже крышки сосуда на $H_1$, выше дна на $H_2$ и может свободно перемещаться. Плотность воды $\rho$, атмосферное давление $P_A$. Найти давления $P_1$ и $P_2$ в воде вблизи крышки и дна сосуда.

\newpage

\adddate{8 класс. 20 апреля 2018}

\task 1
Между точками $A$ и $B$ электрической цепи подключены последовательно резисторы $R_1 = 10\units{Ом}$ и $R_2 = 20\units{Ом}$ и параллельно им $R_3 = 30\units{Ом}.$ Найдите эквивалентное сопротивление $R_{AB}$ этого участка цепи.

\task 2
Электрическая цепь состоит из последовательности $N$ одинаковых звеньев, в которых каждый резистор имеет сопротивление $r$. Последнее звено замкнуто резистором сопротивлением $R$. При каком соотношении $\cfrac{R}{r}$ сопротивление цепи не зависит от числа звеньев?

\task 3
Для измерения сопротивления $R$ проводника собрана электрическая цепь. Вольтметр $V$ показывает напряжение $U_V = 5\units{В},$ показание амперметра $A$ равно $I_A = 25\units{мА}.$ Найдите величину $R$ сопротивления проводника. Внутреннее сопротивление вольтметра $R_V = 1{,}0\units{кОм},$ внутреннее сопротивление амперметра $R_A = 2{,}0\units{Ом}.$

\task 4
Шкала гальванометра имеет $N=100$ делений, цена деления $\delta = 1\units{мкА}$. Внутреннее сопротивление гальванометра $R_G = 1{,}0\units{кОм}.$ Как из этого прибора сделать вольтметр для измерения напряжений до $U = 100\units{В}$ или амперметр для измерения токов силой до $I = 1\units{А}?$

\\ \\ \\ \\ \\ \\ \\ \\
\adddate{8 класс. 20 апреля 2018}

\task 1
Между точками $A$ и $B$ электрической цепи подключены последовательно резисторы $R_1 = 10\units{Ом}$ и $R_2 = 20\units{Ом}$ и параллельно им $R_3 = 30\units{Ом}.$ Найдите эквивалентное сопротивление $R_{AB}$ этого участка цепи.

\task 2
Электрическая цепь состоит из последовательности $N$ одинаковых звеньев, в которых каждый резистор имеет сопротивление $r$. Последнее звено замкнуто резистором сопротивлением $R$. При каком соотношении $\cfrac{R}{r}$ сопротивление цепи не зависит от числа звеньев?

\task 3
Для измерения сопротивления $R$ проводника собрана электрическая цепь. Вольтметр $V$ показывает напряжение $U_V = 5\units{В},$ показание амперметра $A$ равно $I_A = 25\units{мА}.$ Найдите величину $R$ сопротивления проводника. Внутреннее сопротивление вольтметра $R_V = 1{,}0\units{кОм},$ внутреннее сопротивление амперметра $R_A = 2{,}0\units{Ом}.$

\task 4
Шкала гальванометра имеет $N=100$ делений, цена деления $\delta = 1\units{мкА}$. Внутреннее сопротивление гальванометра $R_G = 1{,}0\units{кОм}.$ Как из этого прибора сделать вольтметр для измерения напряжений до $U = 100\units{В}$ или амперметр для измерения токов силой до $I = 1\units{А}?$


% % \begin{flushright}
\textsc{ГБОУ школа №554, 20 ноября 2018\,г.}
\end{flushright}

\begin{center}
\LARGE \textsc{Математический бой, 8 класс}
\end{center}

\problem{1} Есть тридцать карточек, на каждой написано по одному числу: на десяти карточках~–~$a$,  на десяти других~–~$b$ и на десяти оставшихся~–~$c$ (числа  различны). Известно, что к любым пяти карточкам можно подобрать ещё пять так, что сумма чисел на этих десяти карточках будет равна нулю. Докажите, что~одно из~чисел~$a, b, c$ равно нулю.

\problem{2} Вокруг стола стола пустили пакет с орешками. Первый взял один орешек, второй — 2, третий — 3 и так далее: каждый следующий брал на 1 орешек больше. Известно, что на втором круге было взято в сумме на 100 орешков больше, чем на первом. Сколько человек сидело за столом?

% \problem{2} Натуральное число разрешено увеличить на любое целое число процентов от 1 до 100, если при этом получаем натуральное число. Найдите наименьшее натуральное число, которое нельзя при помощи таких операций получить из~числа 1.

% \problem{3} Найти сумму $1^2 - 2^2 + 3^2 - 4^2 + 5^2 + \ldots - 2018^2$.

\problem{3} В кружке рукоделия, где занимается Валя, более 93\% участников~—~девочки. Какое наименьшее число участников может быть в таком кружке?

\problem{4} Произведение 2018 целых чисел равно 1. Может ли их сумма оказаться равной~0?

% \problem{4} Можно ли все натуральные числа от~1 до~9 записать в~клетки таблицы~$3\times3$ так, чтобы сумма в~любых двух соседних (по~вертикали или горизонтали) клетках равнялось простому числу?

\problem{5} На доске написано 2018 нулей и 2019 единиц. Женя стирает 2 числа и, если они были одинаковы, дописывает к оставшимся один ноль, а~если разные — единицу. Потом Женя повторяет эту операцию снова, потом ещё и~так далее. В~результате на~доске останется только одно число. Что это за~число?

\problem{6} Докажите, что в~любой компании людей найдутся 2~человека, имеющие равное число знакомых в этой компании (если $A$~знаком с~$B$, то~и $B$~знаком с~$A$).

\problem{7} Три колокола начинают бить одновременно. Интервалы между ударами колоколов соответственно составляют $\cfrac43$~секунды, $\cfrac53$~секунды и $2$~секунды. Совпавшие по времени удары воспринимаются за~один. Сколько ударов будет услышано за 1~минуту, включая первый и последний удары?

\problem{8} Восемь одинаковых момент расположены по кругу. Известно, что три из~них~— фальшивые, и они расположены рядом друг с~другом. Вес фальшивой монеты отличается от~веса настоящей. Все фальшивые монеты весят одинаково, но неизвестно, тяжелее или легче фальшивая монета настоящей. Покажите, что за~3~взвешивания на~чашечных весах без~гирь можно определить все фальшивые монеты.

% \end{document}

\begin{document}
\noanswers

\setdate{6~мая~2019}
\setclass{10}

\addpersonalvariant{Михаил Бурмистров}
\tasknumber{1}\task{
    Определите ёмкость конденсатора, если при его зарядке до напряжения
    $V=20\units{кВ}$ он приобретает заряд $q=25\units{нКл}$.
    Чему при этом равны заряды обкладок конденсатора (сделайте рисунок)?
}
\answer{
    $q = CV \implies
    C = \frac{ q }{ V } = \frac{ 25 \cdot 10^{-9} \units{Кл} }{ 20 \cdot 10^{3} \units{В} } = 1{,}25 \cdot 10^{-12} \units{Ф} = 1{,}25\units{пФ}.
    \text{ Заряды обкладок: $q$ и $-q$}$
}
\vspace{120pt}

\tasknumber{2}\task{
    На конденсаторе указано: $C=120\units{пФ}$, $V=450\units{кВ}$.
    Удастся ли его использовать для накопления заряда $q=50\units{нКл}$?
}
\answer{
    $q' = CV = 120 \cdot 10^{-12} \units{Ф}\cdot450 \cdot 10^{3} \units{В} = 54000 \cdot 10^{-9} \units{Кл} = 54000\units{нКл}
    \implies q' \ge q \implies \text{удастся}$
}
\vspace{120pt}

\tasknumber{3}\task{
    Как и во сколько раз изменится ёмкость плоского конденсатора при уменьшении площади пластин в 7 раз
    и уменьшении расстояния между ними в 5 раз?
}
\answer{
    $\frac{C'}{C}
        = \frac{\eps_0\eps \frac S7}{\frac d5} \Big/ \frac{\eps_0\eps S}{d}
        = \frac{ 5 }{ 7 } < 1 \implies \text{уменьшится в $\frac75$ раз}
    $
}
\vspace{120pt}

\tasknumber{4}\task{
    Электрическая ёмкость конденсатора равна $C=600\units{пФ}$,
    при этом ему сообщён заряд $Q=900\units{нКл}$.
    Какова энергия заряженного конденсатора?
}
\answer{
    $W
        = \frac{ Q^2 }{ 2C }
        = \frac{ \sqr{ 900 \cdot 10^{-9} \units{Кл} } }{ 2 \cdot 600 \cdot 10^{-12} \units{Ф} }
        = 675{,}00 \cdot 10^{-6} \units{Дж} = 675{,}00\units{мкДж}
    $
}
\vspace{120pt}

\tasknumber{5}\task{
    Два конденсатора ёмкостей $C_1=30\units{нФ}$ и $C_2=40\units{нФ}$ последовательно подключают
    к источнику напряжения $V=150\units{В}$ (см.
    рис.).
    Определите заряды каждого из конденсаторов.
}
\answer{
    $Q_1
        = Q_2
        = CV
        = \frac{V}{\frac1{C_1} + \frac1{C_2}}
        = \frac{C_1C_2V}{C_1 + C_2}
        = \frac{
            30 \cdot 10^{-9} \units{Ф} \cdot 40 \cdot 10^{-9} \units{Ф} \cdot 150 \units{В}
        }{
            30 \cdot 10^{-9} \units{Ф} + 40 \cdot 10^{-9} \units{Ф}
        }
        = 2571{,}43 \cdot 10^{-9} \units{Кл}
        = 2571{,}43\units{нКл}
    $
}
\newpage

\addpersonalvariant{Гагик Аракелян}
\tasknumber{1}\task{
    Определите ёмкость конденсатора, если при его зарядке до напряжения
    $U=3\units{кВ}$ он приобретает заряд $Q=6\units{нКл}$.
    Чему при этом равны заряды обкладок конденсатора (сделайте рисунок)?
}
\answer{
    $Q = CU \implies
    C = \frac{ Q }{ U } = \frac{ 6 \cdot 10^{-9} \units{Кл} }{ 3 \cdot 10^{3} \units{В} } = 2{,}00 \cdot 10^{-12} \units{Ф} = 2{,}00\units{пФ}.
    \text{ Заряды обкладок: $Q$ и $-Q$}$
}
\vspace{120pt}

\tasknumber{2}\task{
    На конденсаторе указано: $C=120\units{пФ}$, $U=450\units{кВ}$.
    Удастся ли его использовать для накопления заряда $Q=60\units{нКл}$?
}
\answer{
    $Q' = CU = 120 \cdot 10^{-12} \units{Ф}\cdot450 \cdot 10^{3} \units{В} = 54000 \cdot 10^{-9} \units{Кл} = 54000\units{нКл}
    \implies Q' \ge Q \implies \text{удастся}$
}
\vspace{120pt}

\tasknumber{3}\task{
    Как и во сколько раз изменится ёмкость плоского конденсатора при уменьшении площади пластин в 3 раз
    и уменьшении расстояния между ними в 8 раз?
}
\answer{
    $\frac{C'}{C}
        = \frac{\eps_0\eps \frac S3}{\frac d8} \Big/ \frac{\eps_0\eps S}{d}
        = \frac{ 8 }{ 3 } > 1 \implies \text{увеличится в $\frac83$ раз}
    $
}
\vspace{120pt}

\tasknumber{4}\task{
    Электрическая ёмкость конденсатора равна $C=200\units{пФ}$,
    при этом ему сообщён заряд $Q=300\units{нКл}$.
    Какова энергия заряженного конденсатора?
}
\answer{
    $W
        = \frac{ Q^2 }{ 2C }
        = \frac{ \sqr{ 300 \cdot 10^{-9} \units{Кл} } }{ 2 \cdot 200 \cdot 10^{-12} \units{Ф} }
        = 225{,}00 \cdot 10^{-6} \units{Дж} = 225{,}00\units{мкДж}
    $
}
\vspace{120pt}

\tasknumber{5}\task{
    Два конденсатора ёмкостей $C_1=40\units{нФ}$ и $C_2=60\units{нФ}$ последовательно подключают
    к источнику напряжения $U=300\units{В}$ (см.
    рис.).
    Определите заряды каждого из конденсаторов.
}
\answer{
    $Q_1
        = Q_2
        = CU
        = \frac{U}{\frac1{C_1} + \frac1{C_2}}
        = \frac{C_1C_2U}{C_1 + C_2}
        = \frac{
            40 \cdot 10^{-9} \units{Ф} \cdot 60 \cdot 10^{-9} \units{Ф} \cdot 300 \units{В}
        }{
            40 \cdot 10^{-9} \units{Ф} + 60 \cdot 10^{-9} \units{Ф}
        }
        = 7200{,}00 \cdot 10^{-9} \units{Кл}
        = 7200{,}00\units{нКл}
    $
}
\newpage

\addpersonalvariant{Ирен Аракелян}
\tasknumber{1}\task{
    Определите ёмкость конденсатора, если при его зарядке до напряжения
    $V=5\units{кВ}$ он приобретает заряд $Q=15\units{нКл}$.
    Чему при этом равны заряды обкладок конденсатора (сделайте рисунок)?
}
\answer{
    $Q = CV \implies
    C = \frac{ Q }{ V } = \frac{ 15 \cdot 10^{-9} \units{Кл} }{ 5 \cdot 10^{3} \units{В} } = 3{,}00 \cdot 10^{-12} \units{Ф} = 3{,}00\units{пФ}.
    \text{ Заряды обкладок: $Q$ и $-Q$}$
}
\vspace{120pt}

\tasknumber{2}\task{
    На конденсаторе указано: $C=100\units{пФ}$, $V=200\units{кВ}$.
    Удастся ли его использовать для накопления заряда $Q=30\units{нКл}$?
}
\answer{
    $Q' = CV = 100 \cdot 10^{-12} \units{Ф}\cdot200 \cdot 10^{3} \units{В} = 20000 \cdot 10^{-9} \units{Кл} = 20000\units{нКл}
    \implies Q' \ge Q \implies \text{удастся}$
}
\vspace{120pt}

\tasknumber{3}\task{
    Как и во сколько раз изменится ёмкость плоского конденсатора при уменьшении площади пластин в 5 раз
    и уменьшении расстояния между ними в 4 раз?
}
\answer{
    $\frac{C'}{C}
        = \frac{\eps_0\eps \frac S5}{\frac d4} \Big/ \frac{\eps_0\eps S}{d}
        = \frac{ 4 }{ 5 } < 1 \implies \text{уменьшится в $\frac54$ раз}
    $
}
\vspace{120pt}

\tasknumber{4}\task{
    Электрическая ёмкость конденсатора равна $C=750\units{пФ}$,
    при этом ему сообщён заряд $Q=900\units{нКл}$.
    Какова энергия заряженного конденсатора?
}
\answer{
    $W
        = \frac{ Q^2 }{ 2C }
        = \frac{ \sqr{ 900 \cdot 10^{-9} \units{Кл} } }{ 2 \cdot 750 \cdot 10^{-12} \units{Ф} }
        = 540{,}00 \cdot 10^{-6} \units{Дж} = 540{,}00\units{мкДж}
    $
}
\vspace{120pt}

\tasknumber{5}\task{
    Два конденсатора ёмкостей $C_1=30\units{нФ}$ и $C_2=60\units{нФ}$ последовательно подключают
    к источнику напряжения $U=200\units{В}$ (см.
    рис.).
    Определите заряды каждого из конденсаторов.
}
\answer{
    $Q_1
        = Q_2
        = CU
        = \frac{U}{\frac1{C_1} + \frac1{C_2}}
        = \frac{C_1C_2U}{C_1 + C_2}
        = \frac{
            30 \cdot 10^{-9} \units{Ф} \cdot 60 \cdot 10^{-9} \units{Ф} \cdot 200 \units{В}
        }{
            30 \cdot 10^{-9} \units{Ф} + 60 \cdot 10^{-9} \units{Ф}
        }
        = 4000{,}00 \cdot 10^{-9} \units{Кл}
        = 4000{,}00\units{нКл}
    $
}
\newpage

\addpersonalvariant{Сабина Асадуллаева}
\tasknumber{1}\task{
    Определите ёмкость конденсатора, если при его зарядке до напряжения
    $U=20\units{кВ}$ он приобретает заряд $q=6\units{нКл}$.
    Чему при этом равны заряды обкладок конденсатора (сделайте рисунок)?
}
\answer{
    $q = CU \implies
    C = \frac{ q }{ U } = \frac{ 6 \cdot 10^{-9} \units{Кл} }{ 20 \cdot 10^{3} \units{В} } = 0{,}30 \cdot 10^{-12} \units{Ф} = 0{,}30\units{пФ}.
    \text{ Заряды обкладок: $q$ и $-q$}$
}
\vspace{120pt}

\tasknumber{2}\task{
    На конденсаторе указано: $C=150\units{пФ}$, $V=450\units{кВ}$.
    Удастся ли его использовать для накопления заряда $Q=30\units{нКл}$?
}
\answer{
    $Q' = CV = 150 \cdot 10^{-12} \units{Ф}\cdot450 \cdot 10^{3} \units{В} = 67500 \cdot 10^{-9} \units{Кл} = 67500\units{нКл}
    \implies Q' \ge Q \implies \text{удастся}$
}
\vspace{120pt}

\tasknumber{3}\task{
    Как и во сколько раз изменится ёмкость плоского конденсатора при уменьшении площади пластин в 3 раз
    и уменьшении расстояния между ними в 5 раз?
}
\answer{
    $\frac{C'}{C}
        = \frac{\eps_0\eps \frac S3}{\frac d5} \Big/ \frac{\eps_0\eps S}{d}
        = \frac{ 5 }{ 3 } > 1 \implies \text{увеличится в $\frac53$ раз}
    $
}
\vspace{120pt}

\tasknumber{4}\task{
    Электрическая ёмкость конденсатора равна $C=400\units{пФ}$,
    при этом ему сообщён заряд $q=900\units{нКл}$.
    Какова энергия заряженного конденсатора?
}
\answer{
    $W
        = \frac{ q^2 }{ 2C }
        = \frac{ \sqr{ 900 \cdot 10^{-9} \units{Кл} } }{ 2 \cdot 400 \cdot 10^{-12} \units{Ф} }
        = 1012{,}50 \cdot 10^{-6} \units{Дж} = 1012{,}50\units{мкДж}
    $
}
\vspace{120pt}

\tasknumber{5}\task{
    Два конденсатора ёмкостей $C_1=30\units{нФ}$ и $C_2=40\units{нФ}$ последовательно подключают
    к источнику напряжения $V=300\units{В}$ (см.
    рис.).
    Определите заряды каждого из конденсаторов.
}
\answer{
    $Q_1
        = Q_2
        = CV
        = \frac{V}{\frac1{C_1} + \frac1{C_2}}
        = \frac{C_1C_2V}{C_1 + C_2}
        = \frac{
            30 \cdot 10^{-9} \units{Ф} \cdot 40 \cdot 10^{-9} \units{Ф} \cdot 300 \units{В}
        }{
            30 \cdot 10^{-9} \units{Ф} + 40 \cdot 10^{-9} \units{Ф}
        }
        = 5142{,}86 \cdot 10^{-9} \units{Кл}
        = 5142{,}86\units{нКл}
    $
}
\newpage

\addpersonalvariant{Вероника Битерякова}
\tasknumber{1}\task{
    Определите ёмкость конденсатора, если при его зарядке до напряжения
    $U=5\units{кВ}$ он приобретает заряд $q=6\units{нКл}$.
    Чему при этом равны заряды обкладок конденсатора (сделайте рисунок)?
}
\answer{
    $q = CU \implies
    C = \frac{ q }{ U } = \frac{ 6 \cdot 10^{-9} \units{Кл} }{ 5 \cdot 10^{3} \units{В} } = 1{,}20 \cdot 10^{-12} \units{Ф} = 1{,}20\units{пФ}.
    \text{ Заряды обкладок: $q$ и $-q$}$
}
\vspace{120pt}

\tasknumber{2}\task{
    На конденсаторе указано: $C=80\units{пФ}$, $V=450\units{кВ}$.
    Удастся ли его использовать для накопления заряда $Q=50\units{нКл}$?
}
\answer{
    $Q' = CV = 80 \cdot 10^{-12} \units{Ф}\cdot450 \cdot 10^{3} \units{В} = 36000 \cdot 10^{-9} \units{Кл} = 36000\units{нКл}
    \implies Q' \ge Q \implies \text{удастся}$
}
\vspace{120pt}

\tasknumber{3}\task{
    Как и во сколько раз изменится ёмкость плоского конденсатора при уменьшении площади пластин в 4 раз
    и уменьшении расстояния между ними в 4 раз?
}
\answer{
    $\frac{C'}{C}
        = \frac{\eps_0\eps \frac S4}{\frac d4} \Big/ \frac{\eps_0\eps S}{d}
        = \frac{ 4 }{ 4 } = 1 \implies \text{не изменится}
    $
}
\vspace{120pt}

\tasknumber{4}\task{
    Электрическая ёмкость конденсатора равна $C=200\units{пФ}$,
    при этом ему сообщён заряд $q=800\units{нКл}$.
    Какова энергия заряженного конденсатора?
}
\answer{
    $W
        = \frac{ q^2 }{ 2C }
        = \frac{ \sqr{ 800 \cdot 10^{-9} \units{Кл} } }{ 2 \cdot 200 \cdot 10^{-12} \units{Ф} }
        = 1600{,}00 \cdot 10^{-6} \units{Дж} = 1600{,}00\units{мкДж}
    $
}
\vspace{120pt}

\tasknumber{5}\task{
    Два конденсатора ёмкостей $C_1=20\units{нФ}$ и $C_2=30\units{нФ}$ последовательно подключают
    к источнику напряжения $U=300\units{В}$ (см.
    рис.).
    Определите заряды каждого из конденсаторов.
}
\answer{
    $Q_1
        = Q_2
        = CU
        = \frac{U}{\frac1{C_1} + \frac1{C_2}}
        = \frac{C_1C_2U}{C_1 + C_2}
        = \frac{
            20 \cdot 10^{-9} \units{Ф} \cdot 30 \cdot 10^{-9} \units{Ф} \cdot 300 \units{В}
        }{
            20 \cdot 10^{-9} \units{Ф} + 30 \cdot 10^{-9} \units{Ф}
        }
        = 3600{,}00 \cdot 10^{-9} \units{Кл}
        = 3600{,}00\units{нКл}
    $
}
\newpage

\addpersonalvariant{Юлия Буянова}
\tasknumber{1}\task{
    Определите ёмкость конденсатора, если при его зарядке до напряжения
    $U=12\units{кВ}$ он приобретает заряд $Q=24\units{нКл}$.
    Чему при этом равны заряды обкладок конденсатора (сделайте рисунок)?
}
\answer{
    $Q = CU \implies
    C = \frac{ Q }{ U } = \frac{ 24 \cdot 10^{-9} \units{Кл} }{ 12 \cdot 10^{3} \units{В} } = 2{,}00 \cdot 10^{-12} \units{Ф} = 2{,}00\units{пФ}.
    \text{ Заряды обкладок: $Q$ и $-Q$}$
}
\vspace{120pt}

\tasknumber{2}\task{
    На конденсаторе указано: $C=150\units{пФ}$, $V=300\units{кВ}$.
    Удастся ли его использовать для накопления заряда $Q=30\units{нКл}$?
}
\answer{
    $Q' = CV = 150 \cdot 10^{-12} \units{Ф}\cdot300 \cdot 10^{3} \units{В} = 45000 \cdot 10^{-9} \units{Кл} = 45000\units{нКл}
    \implies Q' \ge Q \implies \text{удастся}$
}
\vspace{120pt}

\tasknumber{3}\task{
    Как и во сколько раз изменится ёмкость плоского конденсатора при уменьшении площади пластин в 6 раз
    и уменьшении расстояния между ними в 6 раз?
}
\answer{
    $\frac{C'}{C}
        = \frac{\eps_0\eps \frac S6}{\frac d6} \Big/ \frac{\eps_0\eps S}{d}
        = \frac{ 6 }{ 6 } = 1 \implies \text{не изменится}
    $
}
\vspace{120pt}

\tasknumber{4}\task{
    Электрическая ёмкость конденсатора равна $C=600\units{пФ}$,
    при этом ему сообщён заряд $q=300\units{нКл}$.
    Какова энергия заряженного конденсатора?
}
\answer{
    $W
        = \frac{ q^2 }{ 2C }
        = \frac{ \sqr{ 300 \cdot 10^{-9} \units{Кл} } }{ 2 \cdot 600 \cdot 10^{-12} \units{Ф} }
        = 75{,}00 \cdot 10^{-6} \units{Дж} = 75{,}00\units{мкДж}
    $
}
\vspace{120pt}

\tasknumber{5}\task{
    Два конденсатора ёмкостей $C_1=60\units{нФ}$ и $C_2=30\units{нФ}$ последовательно подключают
    к источнику напряжения $V=450\units{В}$ (см.
    рис.).
    Определите заряды каждого из конденсаторов.
}
\answer{
    $Q_1
        = Q_2
        = CV
        = \frac{V}{\frac1{C_1} + \frac1{C_2}}
        = \frac{C_1C_2V}{C_1 + C_2}
        = \frac{
            60 \cdot 10^{-9} \units{Ф} \cdot 30 \cdot 10^{-9} \units{Ф} \cdot 450 \units{В}
        }{
            60 \cdot 10^{-9} \units{Ф} + 30 \cdot 10^{-9} \units{Ф}
        }
        = 9000{,}00 \cdot 10^{-9} \units{Кл}
        = 9000{,}00\units{нКл}
    $
}
\newpage

\addpersonalvariant{Пелагея Вдовина}
\tasknumber{1}\task{
    Определите ёмкость конденсатора, если при его зарядке до напряжения
    $V=12\units{кВ}$ он приобретает заряд $q=4\units{нКл}$.
    Чему при этом равны заряды обкладок конденсатора (сделайте рисунок)?
}
\answer{
    $q = CV \implies
    C = \frac{ q }{ V } = \frac{ 4 \cdot 10^{-9} \units{Кл} }{ 12 \cdot 10^{3} \units{В} } = 0{,}33 \cdot 10^{-12} \units{Ф} = 0{,}33\units{пФ}.
    \text{ Заряды обкладок: $q$ и $-q$}$
}
\vspace{120pt}

\tasknumber{2}\task{
    На конденсаторе указано: $C=120\units{пФ}$, $U=200\units{кВ}$.
    Удастся ли его использовать для накопления заряда $Q=50\units{нКл}$?
}
\answer{
    $Q' = CU = 120 \cdot 10^{-12} \units{Ф}\cdot200 \cdot 10^{3} \units{В} = 24000 \cdot 10^{-9} \units{Кл} = 24000\units{нКл}
    \implies Q' \ge Q \implies \text{удастся}$
}
\vspace{120pt}

\tasknumber{3}\task{
    Как и во сколько раз изменится ёмкость плоского конденсатора при уменьшении площади пластин в 8 раз
    и уменьшении расстояния между ними в 8 раз?
}
\answer{
    $\frac{C'}{C}
        = \frac{\eps_0\eps \frac S8}{\frac d8} \Big/ \frac{\eps_0\eps S}{d}
        = \frac{ 8 }{ 8 } = 1 \implies \text{не изменится}
    $
}
\vspace{120pt}

\tasknumber{4}\task{
    Электрическая ёмкость конденсатора равна $C=400\units{пФ}$,
    при этом ему сообщён заряд $Q=800\units{нКл}$.
    Какова энергия заряженного конденсатора?
}
\answer{
    $W
        = \frac{ Q^2 }{ 2C }
        = \frac{ \sqr{ 800 \cdot 10^{-9} \units{Кл} } }{ 2 \cdot 400 \cdot 10^{-12} \units{Ф} }
        = 800{,}00 \cdot 10^{-6} \units{Дж} = 800{,}00\units{мкДж}
    $
}
\vspace{120pt}

\tasknumber{5}\task{
    Два конденсатора ёмкостей $C_1=40\units{нФ}$ и $C_2=30\units{нФ}$ последовательно подключают
    к источнику напряжения $U=300\units{В}$ (см.
    рис.).
    Определите заряды каждого из конденсаторов.
}
\answer{
    $Q_1
        = Q_2
        = CU
        = \frac{U}{\frac1{C_1} + \frac1{C_2}}
        = \frac{C_1C_2U}{C_1 + C_2}
        = \frac{
            40 \cdot 10^{-9} \units{Ф} \cdot 30 \cdot 10^{-9} \units{Ф} \cdot 300 \units{В}
        }{
            40 \cdot 10^{-9} \units{Ф} + 30 \cdot 10^{-9} \units{Ф}
        }
        = 5142{,}86 \cdot 10^{-9} \units{Кл}
        = 5142{,}86\units{нКл}
    $
}
\newpage

\addpersonalvariant{Леонид Викторов}
\tasknumber{1}\task{
    Определите ёмкость конденсатора, если при его зарядке до напряжения
    $U=20\units{кВ}$ он приобретает заряд $Q=18\units{нКл}$.
    Чему при этом равны заряды обкладок конденсатора (сделайте рисунок)?
}
\answer{
    $Q = CU \implies
    C = \frac{ Q }{ U } = \frac{ 18 \cdot 10^{-9} \units{Кл} }{ 20 \cdot 10^{3} \units{В} } = 0{,}90 \cdot 10^{-12} \units{Ф} = 0{,}90\units{пФ}.
    \text{ Заряды обкладок: $Q$ и $-Q$}$
}
\vspace{120pt}

\tasknumber{2}\task{
    На конденсаторе указано: $C=50\units{пФ}$, $V=400\units{кВ}$.
    Удастся ли его использовать для накопления заряда $Q=60\units{нКл}$?
}
\answer{
    $Q' = CV = 50 \cdot 10^{-12} \units{Ф}\cdot400 \cdot 10^{3} \units{В} = 20000 \cdot 10^{-9} \units{Кл} = 20000\units{нКл}
    \implies Q' \ge Q \implies \text{удастся}$
}
\vspace{120pt}

\tasknumber{3}\task{
    Как и во сколько раз изменится ёмкость плоского конденсатора при уменьшении площади пластин в 5 раз
    и уменьшении расстояния между ними в 3 раз?
}
\answer{
    $\frac{C'}{C}
        = \frac{\eps_0\eps \frac S5}{\frac d3} \Big/ \frac{\eps_0\eps S}{d}
        = \frac{ 3 }{ 5 } < 1 \implies \text{уменьшится в $\frac53$ раз}
    $
}
\vspace{120pt}

\tasknumber{4}\task{
    Электрическая ёмкость конденсатора равна $C=600\units{пФ}$,
    при этом ему сообщён заряд $Q=900\units{нКл}$.
    Какова энергия заряженного конденсатора?
}
\answer{
    $W
        = \frac{ Q^2 }{ 2C }
        = \frac{ \sqr{ 900 \cdot 10^{-9} \units{Кл} } }{ 2 \cdot 600 \cdot 10^{-12} \units{Ф} }
        = 675{,}00 \cdot 10^{-6} \units{Дж} = 675{,}00\units{мкДж}
    $
}
\vspace{120pt}

\tasknumber{5}\task{
    Два конденсатора ёмкостей $C_1=20\units{нФ}$ и $C_2=40\units{нФ}$ последовательно подключают
    к источнику напряжения $V=200\units{В}$ (см.
    рис.).
    Определите заряды каждого из конденсаторов.
}
\answer{
    $Q_1
        = Q_2
        = CV
        = \frac{V}{\frac1{C_1} + \frac1{C_2}}
        = \frac{C_1C_2V}{C_1 + C_2}
        = \frac{
            20 \cdot 10^{-9} \units{Ф} \cdot 40 \cdot 10^{-9} \units{Ф} \cdot 200 \units{В}
        }{
            20 \cdot 10^{-9} \units{Ф} + 40 \cdot 10^{-9} \units{Ф}
        }
        = 2666{,}67 \cdot 10^{-9} \units{Кл}
        = 2666{,}67\units{нКл}
    $
}
\newpage

\addpersonalvariant{Фёдор Гнутов}
\tasknumber{1}\task{
    Определите ёмкость конденсатора, если при его зарядке до напряжения
    $U=2\units{кВ}$ он приобретает заряд $Q=25\units{нКл}$.
    Чему при этом равны заряды обкладок конденсатора (сделайте рисунок)?
}
\answer{
    $Q = CU \implies
    C = \frac{ Q }{ U } = \frac{ 25 \cdot 10^{-9} \units{Кл} }{ 2 \cdot 10^{3} \units{В} } = 12{,}50 \cdot 10^{-12} \units{Ф} = 12{,}50\units{пФ}.
    \text{ Заряды обкладок: $Q$ и $-Q$}$
}
\vspace{120pt}

\tasknumber{2}\task{
    На конденсаторе указано: $C=120\units{пФ}$, $U=400\units{кВ}$.
    Удастся ли его использовать для накопления заряда $Q=30\units{нКл}$?
}
\answer{
    $Q' = CU = 120 \cdot 10^{-12} \units{Ф}\cdot400 \cdot 10^{3} \units{В} = 48000 \cdot 10^{-9} \units{Кл} = 48000\units{нКл}
    \implies Q' \ge Q \implies \text{удастся}$
}
\vspace{120pt}

\tasknumber{3}\task{
    Как и во сколько раз изменится ёмкость плоского конденсатора при уменьшении площади пластин в 3 раз
    и уменьшении расстояния между ними в 4 раз?
}
\answer{
    $\frac{C'}{C}
        = \frac{\eps_0\eps \frac S3}{\frac d4} \Big/ \frac{\eps_0\eps S}{d}
        = \frac{ 4 }{ 3 } > 1 \implies \text{увеличится в $\frac43$ раз}
    $
}
\vspace{120pt}

\tasknumber{4}\task{
    Электрическая ёмкость конденсатора равна $C=750\units{пФ}$,
    при этом ему сообщён заряд $q=300\units{нКл}$.
    Какова энергия заряженного конденсатора?
}
\answer{
    $W
        = \frac{ q^2 }{ 2C }
        = \frac{ \sqr{ 300 \cdot 10^{-9} \units{Кл} } }{ 2 \cdot 750 \cdot 10^{-12} \units{Ф} }
        = 60{,}00 \cdot 10^{-6} \units{Дж} = 60{,}00\units{мкДж}
    $
}
\vspace{120pt}

\tasknumber{5}\task{
    Два конденсатора ёмкостей $C_1=30\units{нФ}$ и $C_2=20\units{нФ}$ последовательно подключают
    к источнику напряжения $U=200\units{В}$ (см.
    рис.).
    Определите заряды каждого из конденсаторов.
}
\answer{
    $Q_1
        = Q_2
        = CU
        = \frac{U}{\frac1{C_1} + \frac1{C_2}}
        = \frac{C_1C_2U}{C_1 + C_2}
        = \frac{
            30 \cdot 10^{-9} \units{Ф} \cdot 20 \cdot 10^{-9} \units{Ф} \cdot 200 \units{В}
        }{
            30 \cdot 10^{-9} \units{Ф} + 20 \cdot 10^{-9} \units{Ф}
        }
        = 2400{,}00 \cdot 10^{-9} \units{Кл}
        = 2400{,}00\units{нКл}
    $
}
\newpage

\addpersonalvariant{Илья Гримберг}
\tasknumber{1}\task{
    Определите ёмкость конденсатора, если при его зарядке до напряжения
    $U=6\units{кВ}$ он приобретает заряд $Q=4\units{нКл}$.
    Чему при этом равны заряды обкладок конденсатора (сделайте рисунок)?
}
\answer{
    $Q = CU \implies
    C = \frac{ Q }{ U } = \frac{ 4 \cdot 10^{-9} \units{Кл} }{ 6 \cdot 10^{3} \units{В} } = 0{,}67 \cdot 10^{-12} \units{Ф} = 0{,}67\units{пФ}.
    \text{ Заряды обкладок: $Q$ и $-Q$}$
}
\vspace{120pt}

\tasknumber{2}\task{
    На конденсаторе указано: $C=50\units{пФ}$, $U=400\units{кВ}$.
    Удастся ли его использовать для накопления заряда $Q=30\units{нКл}$?
}
\answer{
    $Q' = CU = 50 \cdot 10^{-12} \units{Ф}\cdot400 \cdot 10^{3} \units{В} = 20000 \cdot 10^{-9} \units{Кл} = 20000\units{нКл}
    \implies Q' \ge Q \implies \text{удастся}$
}
\vspace{120pt}

\tasknumber{3}\task{
    Как и во сколько раз изменится ёмкость плоского конденсатора при уменьшении площади пластин в 7 раз
    и уменьшении расстояния между ними в 7 раз?
}
\answer{
    $\frac{C'}{C}
        = \frac{\eps_0\eps \frac S7}{\frac d7} \Big/ \frac{\eps_0\eps S}{d}
        = \frac{ 7 }{ 7 } = 1 \implies \text{не изменится}
    $
}
\vspace{120pt}

\tasknumber{4}\task{
    Электрическая ёмкость конденсатора равна $C=750\units{пФ}$,
    при этом ему сообщён заряд $Q=900\units{нКл}$.
    Какова энергия заряженного конденсатора?
}
\answer{
    $W
        = \frac{ Q^2 }{ 2C }
        = \frac{ \sqr{ 900 \cdot 10^{-9} \units{Кл} } }{ 2 \cdot 750 \cdot 10^{-12} \units{Ф} }
        = 540{,}00 \cdot 10^{-6} \units{Дж} = 540{,}00\units{мкДж}
    $
}
\vspace{120pt}

\tasknumber{5}\task{
    Два конденсатора ёмкостей $C_1=60\units{нФ}$ и $C_2=40\units{нФ}$ последовательно подключают
    к источнику напряжения $U=150\units{В}$ (см.
    рис.).
    Определите заряды каждого из конденсаторов.
}
\answer{
    $Q_1
        = Q_2
        = CU
        = \frac{U}{\frac1{C_1} + \frac1{C_2}}
        = \frac{C_1C_2U}{C_1 + C_2}
        = \frac{
            60 \cdot 10^{-9} \units{Ф} \cdot 40 \cdot 10^{-9} \units{Ф} \cdot 150 \units{В}
        }{
            60 \cdot 10^{-9} \units{Ф} + 40 \cdot 10^{-9} \units{Ф}
        }
        = 3600{,}00 \cdot 10^{-9} \units{Кл}
        = 3600{,}00\units{нКл}
    $
}
\newpage

\addpersonalvariant{Иван Гурьянов}
\tasknumber{1}\task{
    Определите ёмкость конденсатора, если при его зарядке до напряжения
    $V=5\units{кВ}$ он приобретает заряд $Q=6\units{нКл}$.
    Чему при этом равны заряды обкладок конденсатора (сделайте рисунок)?
}
\answer{
    $Q = CV \implies
    C = \frac{ Q }{ V } = \frac{ 6 \cdot 10^{-9} \units{Кл} }{ 5 \cdot 10^{3} \units{В} } = 1{,}20 \cdot 10^{-12} \units{Ф} = 1{,}20\units{пФ}.
    \text{ Заряды обкладок: $Q$ и $-Q$}$
}
\vspace{120pt}

\tasknumber{2}\task{
    На конденсаторе указано: $C=150\units{пФ}$, $U=450\units{кВ}$.
    Удастся ли его использовать для накопления заряда $Q=30\units{нКл}$?
}
\answer{
    $Q' = CU = 150 \cdot 10^{-12} \units{Ф}\cdot450 \cdot 10^{3} \units{В} = 67500 \cdot 10^{-9} \units{Кл} = 67500\units{нКл}
    \implies Q' \ge Q \implies \text{удастся}$
}
\vspace{120pt}

\tasknumber{3}\task{
    Как и во сколько раз изменится ёмкость плоского конденсатора при уменьшении площади пластин в 7 раз
    и уменьшении расстояния между ними в 2 раз?
}
\answer{
    $\frac{C'}{C}
        = \frac{\eps_0\eps \frac S7}{\frac d2} \Big/ \frac{\eps_0\eps S}{d}
        = \frac{ 2 }{ 7 } < 1 \implies \text{уменьшится в $\frac72$ раз}
    $
}
\vspace{120pt}

\tasknumber{4}\task{
    Электрическая ёмкость конденсатора равна $C=200\units{пФ}$,
    при этом ему сообщён заряд $q=300\units{нКл}$.
    Какова энергия заряженного конденсатора?
}
\answer{
    $W
        = \frac{ q^2 }{ 2C }
        = \frac{ \sqr{ 300 \cdot 10^{-9} \units{Кл} } }{ 2 \cdot 200 \cdot 10^{-12} \units{Ф} }
        = 225{,}00 \cdot 10^{-6} \units{Дж} = 225{,}00\units{мкДж}
    $
}
\vspace{120pt}

\tasknumber{5}\task{
    Два конденсатора ёмкостей $C_1=30\units{нФ}$ и $C_2=60\units{нФ}$ последовательно подключают
    к источнику напряжения $U=400\units{В}$ (см.
    рис.).
    Определите заряды каждого из конденсаторов.
}
\answer{
    $Q_1
        = Q_2
        = CU
        = \frac{U}{\frac1{C_1} + \frac1{C_2}}
        = \frac{C_1C_2U}{C_1 + C_2}
        = \frac{
            30 \cdot 10^{-9} \units{Ф} \cdot 60 \cdot 10^{-9} \units{Ф} \cdot 400 \units{В}
        }{
            30 \cdot 10^{-9} \units{Ф} + 60 \cdot 10^{-9} \units{Ф}
        }
        = 8000{,}00 \cdot 10^{-9} \units{Кл}
        = 8000{,}00\units{нКл}
    $
}
\newpage

\addpersonalvariant{Артём Денежкин}
\tasknumber{1}\task{
    Определите ёмкость конденсатора, если при его зарядке до напряжения
    $V=12\units{кВ}$ он приобретает заряд $q=24\units{нКл}$.
    Чему при этом равны заряды обкладок конденсатора (сделайте рисунок)?
}
\answer{
    $q = CV \implies
    C = \frac{ q }{ V } = \frac{ 24 \cdot 10^{-9} \units{Кл} }{ 12 \cdot 10^{3} \units{В} } = 2{,}00 \cdot 10^{-12} \units{Ф} = 2{,}00\units{пФ}.
    \text{ Заряды обкладок: $q$ и $-q$}$
}
\vspace{120pt}

\tasknumber{2}\task{
    На конденсаторе указано: $C=100\units{пФ}$, $U=300\units{кВ}$.
    Удастся ли его использовать для накопления заряда $Q=30\units{нКл}$?
}
\answer{
    $Q' = CU = 100 \cdot 10^{-12} \units{Ф}\cdot300 \cdot 10^{3} \units{В} = 30000 \cdot 10^{-9} \units{Кл} = 30000\units{нКл}
    \implies Q' \ge Q \implies \text{удастся}$
}
\vspace{120pt}

\tasknumber{3}\task{
    Как и во сколько раз изменится ёмкость плоского конденсатора при уменьшении площади пластин в 6 раз
    и уменьшении расстояния между ними в 3 раз?
}
\answer{
    $\frac{C'}{C}
        = \frac{\eps_0\eps \frac S6}{\frac d3} \Big/ \frac{\eps_0\eps S}{d}
        = \frac{ 3 }{ 6 } < 1 \implies \text{уменьшится в $\frac21$ раз}
    $
}
\vspace{120pt}

\tasknumber{4}\task{
    Электрическая ёмкость конденсатора равна $C=600\units{пФ}$,
    при этом ему сообщён заряд $q=800\units{нКл}$.
    Какова энергия заряженного конденсатора?
}
\answer{
    $W
        = \frac{ q^2 }{ 2C }
        = \frac{ \sqr{ 800 \cdot 10^{-9} \units{Кл} } }{ 2 \cdot 600 \cdot 10^{-12} \units{Ф} }
        = 533{,}33 \cdot 10^{-6} \units{Дж} = 533{,}33\units{мкДж}
    $
}
\vspace{120pt}

\tasknumber{5}\task{
    Два конденсатора ёмкостей $C_1=20\units{нФ}$ и $C_2=60\units{нФ}$ последовательно подключают
    к источнику напряжения $U=150\units{В}$ (см.
    рис.).
    Определите заряды каждого из конденсаторов.
}
\answer{
    $Q_1
        = Q_2
        = CU
        = \frac{U}{\frac1{C_1} + \frac1{C_2}}
        = \frac{C_1C_2U}{C_1 + C_2}
        = \frac{
            20 \cdot 10^{-9} \units{Ф} \cdot 60 \cdot 10^{-9} \units{Ф} \cdot 150 \units{В}
        }{
            20 \cdot 10^{-9} \units{Ф} + 60 \cdot 10^{-9} \units{Ф}
        }
        = 2250{,}00 \cdot 10^{-9} \units{Кл}
        = 2250{,}00\units{нКл}
    $
}
\newpage

\addpersonalvariant{Виктор Жилин}
\tasknumber{1}\task{
    Определите ёмкость конденсатора, если при его зарядке до напряжения
    $V=20\units{кВ}$ он приобретает заряд $Q=4\units{нКл}$.
    Чему при этом равны заряды обкладок конденсатора (сделайте рисунок)?
}
\answer{
    $Q = CV \implies
    C = \frac{ Q }{ V } = \frac{ 4 \cdot 10^{-9} \units{Кл} }{ 20 \cdot 10^{3} \units{В} } = 0{,}20 \cdot 10^{-12} \units{Ф} = 0{,}20\units{пФ}.
    \text{ Заряды обкладок: $Q$ и $-Q$}$
}
\vspace{120pt}

\tasknumber{2}\task{
    На конденсаторе указано: $C=150\units{пФ}$, $U=200\units{кВ}$.
    Удастся ли его использовать для накопления заряда $q=30\units{нКл}$?
}
\answer{
    $q' = CU = 150 \cdot 10^{-12} \units{Ф}\cdot200 \cdot 10^{3} \units{В} = 30000 \cdot 10^{-9} \units{Кл} = 30000\units{нКл}
    \implies q' \ge q \implies \text{удастся}$
}
\vspace{120pt}

\tasknumber{3}\task{
    Как и во сколько раз изменится ёмкость плоского конденсатора при уменьшении площади пластин в 5 раз
    и уменьшении расстояния между ними в 5 раз?
}
\answer{
    $\frac{C'}{C}
        = \frac{\eps_0\eps \frac S5}{\frac d5} \Big/ \frac{\eps_0\eps S}{d}
        = \frac{ 5 }{ 5 } = 1 \implies \text{не изменится}
    $
}
\vspace{120pt}

\tasknumber{4}\task{
    Электрическая ёмкость конденсатора равна $C=200\units{пФ}$,
    при этом ему сообщён заряд $Q=900\units{нКл}$.
    Какова энергия заряженного конденсатора?
}
\answer{
    $W
        = \frac{ Q^2 }{ 2C }
        = \frac{ \sqr{ 900 \cdot 10^{-9} \units{Кл} } }{ 2 \cdot 200 \cdot 10^{-12} \units{Ф} }
        = 2025{,}00 \cdot 10^{-6} \units{Дж} = 2025{,}00\units{мкДж}
    $
}
\vspace{120pt}

\tasknumber{5}\task{
    Два конденсатора ёмкостей $C_1=20\units{нФ}$ и $C_2=60\units{нФ}$ последовательно подключают
    к источнику напряжения $U=400\units{В}$ (см.
    рис.).
    Определите заряды каждого из конденсаторов.
}
\answer{
    $Q_1
        = Q_2
        = CU
        = \frac{U}{\frac1{C_1} + \frac1{C_2}}
        = \frac{C_1C_2U}{C_1 + C_2}
        = \frac{
            20 \cdot 10^{-9} \units{Ф} \cdot 60 \cdot 10^{-9} \units{Ф} \cdot 400 \units{В}
        }{
            20 \cdot 10^{-9} \units{Ф} + 60 \cdot 10^{-9} \units{Ф}
        }
        = 6000{,}00 \cdot 10^{-9} \units{Кл}
        = 6000{,}00\units{нКл}
    $
}
\newpage

\addpersonalvariant{Дмитрий Иванов}
\tasknumber{1}\task{
    Определите ёмкость конденсатора, если при его зарядке до напряжения
    $U=15\units{кВ}$ он приобретает заряд $Q=6\units{нКл}$.
    Чему при этом равны заряды обкладок конденсатора (сделайте рисунок)?
}
\answer{
    $Q = CU \implies
    C = \frac{ Q }{ U } = \frac{ 6 \cdot 10^{-9} \units{Кл} }{ 15 \cdot 10^{3} \units{В} } = 0{,}40 \cdot 10^{-12} \units{Ф} = 0{,}40\units{пФ}.
    \text{ Заряды обкладок: $Q$ и $-Q$}$
}
\vspace{120pt}

\tasknumber{2}\task{
    На конденсаторе указано: $C=120\units{пФ}$, $U=300\units{кВ}$.
    Удастся ли его использовать для накопления заряда $q=50\units{нКл}$?
}
\answer{
    $q' = CU = 120 \cdot 10^{-12} \units{Ф}\cdot300 \cdot 10^{3} \units{В} = 36000 \cdot 10^{-9} \units{Кл} = 36000\units{нКл}
    \implies q' \ge q \implies \text{удастся}$
}
\vspace{120pt}

\tasknumber{3}\task{
    Как и во сколько раз изменится ёмкость плоского конденсатора при уменьшении площади пластин в 3 раз
    и уменьшении расстояния между ними в 3 раз?
}
\answer{
    $\frac{C'}{C}
        = \frac{\eps_0\eps \frac S3}{\frac d3} \Big/ \frac{\eps_0\eps S}{d}
        = \frac{ 3 }{ 3 } = 1 \implies \text{не изменится}
    $
}
\vspace{120pt}

\tasknumber{4}\task{
    Электрическая ёмкость конденсатора равна $C=400\units{пФ}$,
    при этом ему сообщён заряд $q=900\units{нКл}$.
    Какова энергия заряженного конденсатора?
}
\answer{
    $W
        = \frac{ q^2 }{ 2C }
        = \frac{ \sqr{ 900 \cdot 10^{-9} \units{Кл} } }{ 2 \cdot 400 \cdot 10^{-12} \units{Ф} }
        = 1012{,}50 \cdot 10^{-6} \units{Дж} = 1012{,}50\units{мкДж}
    $
}
\vspace{120pt}

\tasknumber{5}\task{
    Два конденсатора ёмкостей $C_1=40\units{нФ}$ и $C_2=30\units{нФ}$ последовательно подключают
    к источнику напряжения $U=150\units{В}$ (см.
    рис.).
    Определите заряды каждого из конденсаторов.
}
\answer{
    $Q_1
        = Q_2
        = CU
        = \frac{U}{\frac1{C_1} + \frac1{C_2}}
        = \frac{C_1C_2U}{C_1 + C_2}
        = \frac{
            40 \cdot 10^{-9} \units{Ф} \cdot 30 \cdot 10^{-9} \units{Ф} \cdot 150 \units{В}
        }{
            40 \cdot 10^{-9} \units{Ф} + 30 \cdot 10^{-9} \units{Ф}
        }
        = 2571{,}43 \cdot 10^{-9} \units{Кл}
        = 2571{,}43\units{нКл}
    $
}
\newpage

\addpersonalvariant{Олег Климов}
\tasknumber{1}\task{
    Определите ёмкость конденсатора, если при его зарядке до напряжения
    $U=2\units{кВ}$ он приобретает заряд $q=25\units{нКл}$.
    Чему при этом равны заряды обкладок конденсатора (сделайте рисунок)?
}
\answer{
    $q = CU \implies
    C = \frac{ q }{ U } = \frac{ 25 \cdot 10^{-9} \units{Кл} }{ 2 \cdot 10^{3} \units{В} } = 12{,}50 \cdot 10^{-12} \units{Ф} = 12{,}50\units{пФ}.
    \text{ Заряды обкладок: $q$ и $-q$}$
}
\vspace{120pt}

\tasknumber{2}\task{
    На конденсаторе указано: $C=150\units{пФ}$, $V=400\units{кВ}$.
    Удастся ли его использовать для накопления заряда $q=50\units{нКл}$?
}
\answer{
    $q' = CV = 150 \cdot 10^{-12} \units{Ф}\cdot400 \cdot 10^{3} \units{В} = 60000 \cdot 10^{-9} \units{Кл} = 60000\units{нКл}
    \implies q' \ge q \implies \text{удастся}$
}
\vspace{120pt}

\tasknumber{3}\task{
    Как и во сколько раз изменится ёмкость плоского конденсатора при уменьшении площади пластин в 4 раз
    и уменьшении расстояния между ними в 5 раз?
}
\answer{
    $\frac{C'}{C}
        = \frac{\eps_0\eps \frac S4}{\frac d5} \Big/ \frac{\eps_0\eps S}{d}
        = \frac{ 5 }{ 4 } > 1 \implies \text{увеличится в $\frac54$ раз}
    $
}
\vspace{120pt}

\tasknumber{4}\task{
    Электрическая ёмкость конденсатора равна $C=600\units{пФ}$,
    при этом ему сообщён заряд $q=900\units{нКл}$.
    Какова энергия заряженного конденсатора?
}
\answer{
    $W
        = \frac{ q^2 }{ 2C }
        = \frac{ \sqr{ 900 \cdot 10^{-9} \units{Кл} } }{ 2 \cdot 600 \cdot 10^{-12} \units{Ф} }
        = 675{,}00 \cdot 10^{-6} \units{Дж} = 675{,}00\units{мкДж}
    $
}
\vspace{120pt}

\tasknumber{5}\task{
    Два конденсатора ёмкостей $C_1=40\units{нФ}$ и $C_2=30\units{нФ}$ последовательно подключают
    к источнику напряжения $U=400\units{В}$ (см.
    рис.).
    Определите заряды каждого из конденсаторов.
}
\answer{
    $Q_1
        = Q_2
        = CU
        = \frac{U}{\frac1{C_1} + \frac1{C_2}}
        = \frac{C_1C_2U}{C_1 + C_2}
        = \frac{
            40 \cdot 10^{-9} \units{Ф} \cdot 30 \cdot 10^{-9} \units{Ф} \cdot 400 \units{В}
        }{
            40 \cdot 10^{-9} \units{Ф} + 30 \cdot 10^{-9} \units{Ф}
        }
        = 6857{,}14 \cdot 10^{-9} \units{Кл}
        = 6857{,}14\units{нКл}
    $
}
\newpage

\addpersonalvariant{Анна Ковалева}
\tasknumber{1}\task{
    Определите ёмкость конденсатора, если при его зарядке до напряжения
    $V=20\units{кВ}$ он приобретает заряд $Q=15\units{нКл}$.
    Чему при этом равны заряды обкладок конденсатора (сделайте рисунок)?
}
\answer{
    $Q = CV \implies
    C = \frac{ Q }{ V } = \frac{ 15 \cdot 10^{-9} \units{Кл} }{ 20 \cdot 10^{3} \units{В} } = 0{,}75 \cdot 10^{-12} \units{Ф} = 0{,}75\units{пФ}.
    \text{ Заряды обкладок: $Q$ и $-Q$}$
}
\vspace{120pt}

\tasknumber{2}\task{
    На конденсаторе указано: $C=100\units{пФ}$, $U=200\units{кВ}$.
    Удастся ли его использовать для накопления заряда $Q=30\units{нКл}$?
}
\answer{
    $Q' = CU = 100 \cdot 10^{-12} \units{Ф}\cdot200 \cdot 10^{3} \units{В} = 20000 \cdot 10^{-9} \units{Кл} = 20000\units{нКл}
    \implies Q' \ge Q \implies \text{удастся}$
}
\vspace{120pt}

\tasknumber{3}\task{
    Как и во сколько раз изменится ёмкость плоского конденсатора при уменьшении площади пластин в 2 раз
    и уменьшении расстояния между ними в 2 раз?
}
\answer{
    $\frac{C'}{C}
        = \frac{\eps_0\eps \frac S2}{\frac d2} \Big/ \frac{\eps_0\eps S}{d}
        = \frac{ 2 }{ 2 } = 1 \implies \text{не изменится}
    $
}
\vspace{120pt}

\tasknumber{4}\task{
    Электрическая ёмкость конденсатора равна $C=750\units{пФ}$,
    при этом ему сообщён заряд $q=500\units{нКл}$.
    Какова энергия заряженного конденсатора?
}
\answer{
    $W
        = \frac{ q^2 }{ 2C }
        = \frac{ \sqr{ 500 \cdot 10^{-9} \units{Кл} } }{ 2 \cdot 750 \cdot 10^{-12} \units{Ф} }
        = 166{,}67 \cdot 10^{-6} \units{Дж} = 166{,}67\units{мкДж}
    $
}
\vspace{120pt}

\tasknumber{5}\task{
    Два конденсатора ёмкостей $C_1=40\units{нФ}$ и $C_2=60\units{нФ}$ последовательно подключают
    к источнику напряжения $U=150\units{В}$ (см.
    рис.).
    Определите заряды каждого из конденсаторов.
}
\answer{
    $Q_1
        = Q_2
        = CU
        = \frac{U}{\frac1{C_1} + \frac1{C_2}}
        = \frac{C_1C_2U}{C_1 + C_2}
        = \frac{
            40 \cdot 10^{-9} \units{Ф} \cdot 60 \cdot 10^{-9} \units{Ф} \cdot 150 \units{В}
        }{
            40 \cdot 10^{-9} \units{Ф} + 60 \cdot 10^{-9} \units{Ф}
        }
        = 3600{,}00 \cdot 10^{-9} \units{Кл}
        = 3600{,}00\units{нКл}
    $
}
\newpage

\addpersonalvariant{Глеб Ковылин}
\tasknumber{1}\task{
    Определите ёмкость конденсатора, если при его зарядке до напряжения
    $U=3\units{кВ}$ он приобретает заряд $Q=15\units{нКл}$.
    Чему при этом равны заряды обкладок конденсатора (сделайте рисунок)?
}
\answer{
    $Q = CU \implies
    C = \frac{ Q }{ U } = \frac{ 15 \cdot 10^{-9} \units{Кл} }{ 3 \cdot 10^{3} \units{В} } = 5{,}00 \cdot 10^{-12} \units{Ф} = 5{,}00\units{пФ}.
    \text{ Заряды обкладок: $Q$ и $-Q$}$
}
\vspace{120pt}

\tasknumber{2}\task{
    На конденсаторе указано: $C=50\units{пФ}$, $U=450\units{кВ}$.
    Удастся ли его использовать для накопления заряда $q=60\units{нКл}$?
}
\answer{
    $q' = CU = 50 \cdot 10^{-12} \units{Ф}\cdot450 \cdot 10^{3} \units{В} = 22500 \cdot 10^{-9} \units{Кл} = 22500\units{нКл}
    \implies q' \ge q \implies \text{удастся}$
}
\vspace{120pt}

\tasknumber{3}\task{
    Как и во сколько раз изменится ёмкость плоского конденсатора при уменьшении площади пластин в 5 раз
    и уменьшении расстояния между ними в 2 раз?
}
\answer{
    $\frac{C'}{C}
        = \frac{\eps_0\eps \frac S5}{\frac d2} \Big/ \frac{\eps_0\eps S}{d}
        = \frac{ 2 }{ 5 } < 1 \implies \text{уменьшится в $\frac52$ раз}
    $
}
\vspace{120pt}

\tasknumber{4}\task{
    Электрическая ёмкость конденсатора равна $C=400\units{пФ}$,
    при этом ему сообщён заряд $Q=900\units{нКл}$.
    Какова энергия заряженного конденсатора?
}
\answer{
    $W
        = \frac{ Q^2 }{ 2C }
        = \frac{ \sqr{ 900 \cdot 10^{-9} \units{Кл} } }{ 2 \cdot 400 \cdot 10^{-12} \units{Ф} }
        = 1012{,}50 \cdot 10^{-6} \units{Дж} = 1012{,}50\units{мкДж}
    $
}
\vspace{120pt}

\tasknumber{5}\task{
    Два конденсатора ёмкостей $C_1=60\units{нФ}$ и $C_2=40\units{нФ}$ последовательно подключают
    к источнику напряжения $V=450\units{В}$ (см.
    рис.).
    Определите заряды каждого из конденсаторов.
}
\answer{
    $Q_1
        = Q_2
        = CV
        = \frac{V}{\frac1{C_1} + \frac1{C_2}}
        = \frac{C_1C_2V}{C_1 + C_2}
        = \frac{
            60 \cdot 10^{-9} \units{Ф} \cdot 40 \cdot 10^{-9} \units{Ф} \cdot 450 \units{В}
        }{
            60 \cdot 10^{-9} \units{Ф} + 40 \cdot 10^{-9} \units{Ф}
        }
        = 10800{,}00 \cdot 10^{-9} \units{Кл}
        = 10800{,}00\units{нКл}
    $
}
\newpage

\addpersonalvariant{Даниил Космынин}
\tasknumber{1}\task{
    Определите ёмкость конденсатора, если при его зарядке до напряжения
    $U=15\units{кВ}$ он приобретает заряд $Q=4\units{нКл}$.
    Чему при этом равны заряды обкладок конденсатора (сделайте рисунок)?
}
\answer{
    $Q = CU \implies
    C = \frac{ Q }{ U } = \frac{ 4 \cdot 10^{-9} \units{Кл} }{ 15 \cdot 10^{3} \units{В} } = 0{,}27 \cdot 10^{-12} \units{Ф} = 0{,}27\units{пФ}.
    \text{ Заряды обкладок: $Q$ и $-Q$}$
}
\vspace{120pt}

\tasknumber{2}\task{
    На конденсаторе указано: $C=50\units{пФ}$, $V=400\units{кВ}$.
    Удастся ли его использовать для накопления заряда $Q=50\units{нКл}$?
}
\answer{
    $Q' = CV = 50 \cdot 10^{-12} \units{Ф}\cdot400 \cdot 10^{3} \units{В} = 20000 \cdot 10^{-9} \units{Кл} = 20000\units{нКл}
    \implies Q' \ge Q \implies \text{удастся}$
}
\vspace{120pt}

\tasknumber{3}\task{
    Как и во сколько раз изменится ёмкость плоского конденсатора при уменьшении площади пластин в 8 раз
    и уменьшении расстояния между ними в 6 раз?
}
\answer{
    $\frac{C'}{C}
        = \frac{\eps_0\eps \frac S8}{\frac d6} \Big/ \frac{\eps_0\eps S}{d}
        = \frac{ 6 }{ 8 } < 1 \implies \text{уменьшится в $\frac43$ раз}
    $
}
\vspace{120pt}

\tasknumber{4}\task{
    Электрическая ёмкость конденсатора равна $C=200\units{пФ}$,
    при этом ему сообщён заряд $q=900\units{нКл}$.
    Какова энергия заряженного конденсатора?
}
\answer{
    $W
        = \frac{ q^2 }{ 2C }
        = \frac{ \sqr{ 900 \cdot 10^{-9} \units{Кл} } }{ 2 \cdot 200 \cdot 10^{-12} \units{Ф} }
        = 2025{,}00 \cdot 10^{-6} \units{Дж} = 2025{,}00\units{мкДж}
    $
}
\vspace{120pt}

\tasknumber{5}\task{
    Два конденсатора ёмкостей $C_1=40\units{нФ}$ и $C_2=20\units{нФ}$ последовательно подключают
    к источнику напряжения $V=400\units{В}$ (см.
    рис.).
    Определите заряды каждого из конденсаторов.
}
\answer{
    $Q_1
        = Q_2
        = CV
        = \frac{V}{\frac1{C_1} + \frac1{C_2}}
        = \frac{C_1C_2V}{C_1 + C_2}
        = \frac{
            40 \cdot 10^{-9} \units{Ф} \cdot 20 \cdot 10^{-9} \units{Ф} \cdot 400 \units{В}
        }{
            40 \cdot 10^{-9} \units{Ф} + 20 \cdot 10^{-9} \units{Ф}
        }
        = 5333{,}33 \cdot 10^{-9} \units{Кл}
        = 5333{,}33\units{нКл}
    $
}
\newpage

\addpersonalvariant{Алина Леоничева}
\tasknumber{1}\task{
    Определите ёмкость конденсатора, если при его зарядке до напряжения
    $V=2\units{кВ}$ он приобретает заряд $Q=24\units{нКл}$.
    Чему при этом равны заряды обкладок конденсатора (сделайте рисунок)?
}
\answer{
    $Q = CV \implies
    C = \frac{ Q }{ V } = \frac{ 24 \cdot 10^{-9} \units{Кл} }{ 2 \cdot 10^{3} \units{В} } = 12{,}00 \cdot 10^{-12} \units{Ф} = 12{,}00\units{пФ}.
    \text{ Заряды обкладок: $Q$ и $-Q$}$
}
\vspace{120pt}

\tasknumber{2}\task{
    На конденсаторе указано: $C=100\units{пФ}$, $U=450\units{кВ}$.
    Удастся ли его использовать для накопления заряда $q=50\units{нКл}$?
}
\answer{
    $q' = CU = 100 \cdot 10^{-12} \units{Ф}\cdot450 \cdot 10^{3} \units{В} = 45000 \cdot 10^{-9} \units{Кл} = 45000\units{нКл}
    \implies q' \ge q \implies \text{удастся}$
}
\vspace{120pt}

\tasknumber{3}\task{
    Как и во сколько раз изменится ёмкость плоского конденсатора при уменьшении площади пластин в 4 раз
    и уменьшении расстояния между ними в 2 раз?
}
\answer{
    $\frac{C'}{C}
        = \frac{\eps_0\eps \frac S4}{\frac d2} \Big/ \frac{\eps_0\eps S}{d}
        = \frac{ 2 }{ 4 } < 1 \implies \text{уменьшится в $\frac21$ раз}
    $
}
\vspace{120pt}

\tasknumber{4}\task{
    Электрическая ёмкость конденсатора равна $C=750\units{пФ}$,
    при этом ему сообщён заряд $Q=500\units{нКл}$.
    Какова энергия заряженного конденсатора?
}
\answer{
    $W
        = \frac{ Q^2 }{ 2C }
        = \frac{ \sqr{ 500 \cdot 10^{-9} \units{Кл} } }{ 2 \cdot 750 \cdot 10^{-12} \units{Ф} }
        = 166{,}67 \cdot 10^{-6} \units{Дж} = 166{,}67\units{мкДж}
    $
}
\vspace{120pt}

\tasknumber{5}\task{
    Два конденсатора ёмкостей $C_1=40\units{нФ}$ и $C_2=20\units{нФ}$ последовательно подключают
    к источнику напряжения $V=300\units{В}$ (см.
    рис.).
    Определите заряды каждого из конденсаторов.
}
\answer{
    $Q_1
        = Q_2
        = CV
        = \frac{V}{\frac1{C_1} + \frac1{C_2}}
        = \frac{C_1C_2V}{C_1 + C_2}
        = \frac{
            40 \cdot 10^{-9} \units{Ф} \cdot 20 \cdot 10^{-9} \units{Ф} \cdot 300 \units{В}
        }{
            40 \cdot 10^{-9} \units{Ф} + 20 \cdot 10^{-9} \units{Ф}
        }
        = 4000{,}00 \cdot 10^{-9} \units{Кл}
        = 4000{,}00\units{нКл}
    $
}
\newpage

\addpersonalvariant{Ирина Лин}
\tasknumber{1}\task{
    Определите ёмкость конденсатора, если при его зарядке до напряжения
    $V=3\units{кВ}$ он приобретает заряд $Q=6\units{нКл}$.
    Чему при этом равны заряды обкладок конденсатора (сделайте рисунок)?
}
\answer{
    $Q = CV \implies
    C = \frac{ Q }{ V } = \frac{ 6 \cdot 10^{-9} \units{Кл} }{ 3 \cdot 10^{3} \units{В} } = 2{,}00 \cdot 10^{-12} \units{Ф} = 2{,}00\units{пФ}.
    \text{ Заряды обкладок: $Q$ и $-Q$}$
}
\vspace{120pt}

\tasknumber{2}\task{
    На конденсаторе указано: $C=100\units{пФ}$, $V=200\units{кВ}$.
    Удастся ли его использовать для накопления заряда $q=60\units{нКл}$?
}
\answer{
    $q' = CV = 100 \cdot 10^{-12} \units{Ф}\cdot200 \cdot 10^{3} \units{В} = 20000 \cdot 10^{-9} \units{Кл} = 20000\units{нКл}
    \implies q' \ge q \implies \text{удастся}$
}
\vspace{120pt}

\tasknumber{3}\task{
    Как и во сколько раз изменится ёмкость плоского конденсатора при уменьшении площади пластин в 4 раз
    и уменьшении расстояния между ними в 7 раз?
}
\answer{
    $\frac{C'}{C}
        = \frac{\eps_0\eps \frac S4}{\frac d7} \Big/ \frac{\eps_0\eps S}{d}
        = \frac{ 7 }{ 4 } > 1 \implies \text{увеличится в $\frac74$ раз}
    $
}
\vspace{120pt}

\tasknumber{4}\task{
    Электрическая ёмкость конденсатора равна $C=750\units{пФ}$,
    при этом ему сообщён заряд $q=900\units{нКл}$.
    Какова энергия заряженного конденсатора?
}
\answer{
    $W
        = \frac{ q^2 }{ 2C }
        = \frac{ \sqr{ 900 \cdot 10^{-9} \units{Кл} } }{ 2 \cdot 750 \cdot 10^{-12} \units{Ф} }
        = 540{,}00 \cdot 10^{-6} \units{Дж} = 540{,}00\units{мкДж}
    $
}
\vspace{120pt}

\tasknumber{5}\task{
    Два конденсатора ёмкостей $C_1=30\units{нФ}$ и $C_2=40\units{нФ}$ последовательно подключают
    к источнику напряжения $U=450\units{В}$ (см.
    рис.).
    Определите заряды каждого из конденсаторов.
}
\answer{
    $Q_1
        = Q_2
        = CU
        = \frac{U}{\frac1{C_1} + \frac1{C_2}}
        = \frac{C_1C_2U}{C_1 + C_2}
        = \frac{
            30 \cdot 10^{-9} \units{Ф} \cdot 40 \cdot 10^{-9} \units{Ф} \cdot 450 \units{В}
        }{
            30 \cdot 10^{-9} \units{Ф} + 40 \cdot 10^{-9} \units{Ф}
        }
        = 7714{,}29 \cdot 10^{-9} \units{Кл}
        = 7714{,}29\units{нКл}
    $
}
\newpage

\addpersonalvariant{Олег Мальцев}
\tasknumber{1}\task{
    Определите ёмкость конденсатора, если при его зарядке до напряжения
    $U=20\units{кВ}$ он приобретает заряд $Q=15\units{нКл}$.
    Чему при этом равны заряды обкладок конденсатора (сделайте рисунок)?
}
\answer{
    $Q = CU \implies
    C = \frac{ Q }{ U } = \frac{ 15 \cdot 10^{-9} \units{Кл} }{ 20 \cdot 10^{3} \units{В} } = 0{,}75 \cdot 10^{-12} \units{Ф} = 0{,}75\units{пФ}.
    \text{ Заряды обкладок: $Q$ и $-Q$}$
}
\vspace{120pt}

\tasknumber{2}\task{
    На конденсаторе указано: $C=100\units{пФ}$, $U=300\units{кВ}$.
    Удастся ли его использовать для накопления заряда $q=50\units{нКл}$?
}
\answer{
    $q' = CU = 100 \cdot 10^{-12} \units{Ф}\cdot300 \cdot 10^{3} \units{В} = 30000 \cdot 10^{-9} \units{Кл} = 30000\units{нКл}
    \implies q' \ge q \implies \text{удастся}$
}
\vspace{120pt}

\tasknumber{3}\task{
    Как и во сколько раз изменится ёмкость плоского конденсатора при уменьшении площади пластин в 2 раз
    и уменьшении расстояния между ними в 8 раз?
}
\answer{
    $\frac{C'}{C}
        = \frac{\eps_0\eps \frac S2}{\frac d8} \Big/ \frac{\eps_0\eps S}{d}
        = \frac{ 8 }{ 2 } > 1 \implies \text{увеличится в $\frac41$ раз}
    $
}
\vspace{120pt}

\tasknumber{4}\task{
    Электрическая ёмкость конденсатора равна $C=750\units{пФ}$,
    при этом ему сообщён заряд $q=800\units{нКл}$.
    Какова энергия заряженного конденсатора?
}
\answer{
    $W
        = \frac{ q^2 }{ 2C }
        = \frac{ \sqr{ 800 \cdot 10^{-9} \units{Кл} } }{ 2 \cdot 750 \cdot 10^{-12} \units{Ф} }
        = 426{,}67 \cdot 10^{-6} \units{Дж} = 426{,}67\units{мкДж}
    $
}
\vspace{120pt}

\tasknumber{5}\task{
    Два конденсатора ёмкостей $C_1=40\units{нФ}$ и $C_2=60\units{нФ}$ последовательно подключают
    к источнику напряжения $U=400\units{В}$ (см.
    рис.).
    Определите заряды каждого из конденсаторов.
}
\answer{
    $Q_1
        = Q_2
        = CU
        = \frac{U}{\frac1{C_1} + \frac1{C_2}}
        = \frac{C_1C_2U}{C_1 + C_2}
        = \frac{
            40 \cdot 10^{-9} \units{Ф} \cdot 60 \cdot 10^{-9} \units{Ф} \cdot 400 \units{В}
        }{
            40 \cdot 10^{-9} \units{Ф} + 60 \cdot 10^{-9} \units{Ф}
        }
        = 9600{,}00 \cdot 10^{-9} \units{Кл}
        = 9600{,}00\units{нКл}
    $
}
\newpage

\addpersonalvariant{Ислам Мунаев}
\tasknumber{1}\task{
    Определите ёмкость конденсатора, если при его зарядке до напряжения
    $V=5\units{кВ}$ он приобретает заряд $Q=4\units{нКл}$.
    Чему при этом равны заряды обкладок конденсатора (сделайте рисунок)?
}
\answer{
    $Q = CV \implies
    C = \frac{ Q }{ V } = \frac{ 4 \cdot 10^{-9} \units{Кл} }{ 5 \cdot 10^{3} \units{В} } = 0{,}80 \cdot 10^{-12} \units{Ф} = 0{,}80\units{пФ}.
    \text{ Заряды обкладок: $Q$ и $-Q$}$
}
\vspace{120pt}

\tasknumber{2}\task{
    На конденсаторе указано: $C=150\units{пФ}$, $V=400\units{кВ}$.
    Удастся ли его использовать для накопления заряда $q=60\units{нКл}$?
}
\answer{
    $q' = CV = 150 \cdot 10^{-12} \units{Ф}\cdot400 \cdot 10^{3} \units{В} = 60000 \cdot 10^{-9} \units{Кл} = 60000\units{нКл}
    \implies q' \ge q \implies \text{удастся}$
}
\vspace{120pt}

\tasknumber{3}\task{
    Как и во сколько раз изменится ёмкость плоского конденсатора при уменьшении площади пластин в 3 раз
    и уменьшении расстояния между ними в 6 раз?
}
\answer{
    $\frac{C'}{C}
        = \frac{\eps_0\eps \frac S3}{\frac d6} \Big/ \frac{\eps_0\eps S}{d}
        = \frac{ 6 }{ 3 } > 1 \implies \text{увеличится в $\frac21$ раз}
    $
}
\vspace{120pt}

\tasknumber{4}\task{
    Электрическая ёмкость конденсатора равна $C=750\units{пФ}$,
    при этом ему сообщён заряд $q=500\units{нКл}$.
    Какова энергия заряженного конденсатора?
}
\answer{
    $W
        = \frac{ q^2 }{ 2C }
        = \frac{ \sqr{ 500 \cdot 10^{-9} \units{Кл} } }{ 2 \cdot 750 \cdot 10^{-12} \units{Ф} }
        = 166{,}67 \cdot 10^{-6} \units{Дж} = 166{,}67\units{мкДж}
    $
}
\vspace{120pt}

\tasknumber{5}\task{
    Два конденсатора ёмкостей $C_1=60\units{нФ}$ и $C_2=20\units{нФ}$ последовательно подключают
    к источнику напряжения $U=200\units{В}$ (см.
    рис.).
    Определите заряды каждого из конденсаторов.
}
\answer{
    $Q_1
        = Q_2
        = CU
        = \frac{U}{\frac1{C_1} + \frac1{C_2}}
        = \frac{C_1C_2U}{C_1 + C_2}
        = \frac{
            60 \cdot 10^{-9} \units{Ф} \cdot 20 \cdot 10^{-9} \units{Ф} \cdot 200 \units{В}
        }{
            60 \cdot 10^{-9} \units{Ф} + 20 \cdot 10^{-9} \units{Ф}
        }
        = 3000{,}00 \cdot 10^{-9} \units{Кл}
        = 3000{,}00\units{нКл}
    $
}
\newpage

\addpersonalvariant{Александр Наумов}
\tasknumber{1}\task{
    Определите ёмкость конденсатора, если при его зарядке до напряжения
    $V=15\units{кВ}$ он приобретает заряд $q=15\units{нКл}$.
    Чему при этом равны заряды обкладок конденсатора (сделайте рисунок)?
}
\answer{
    $q = CV \implies
    C = \frac{ q }{ V } = \frac{ 15 \cdot 10^{-9} \units{Кл} }{ 15 \cdot 10^{3} \units{В} } = 1{,}00 \cdot 10^{-12} \units{Ф} = 1{,}00\units{пФ}.
    \text{ Заряды обкладок: $q$ и $-q$}$
}
\vspace{120pt}

\tasknumber{2}\task{
    На конденсаторе указано: $C=150\units{пФ}$, $V=450\units{кВ}$.
    Удастся ли его использовать для накопления заряда $q=30\units{нКл}$?
}
\answer{
    $q' = CV = 150 \cdot 10^{-12} \units{Ф}\cdot450 \cdot 10^{3} \units{В} = 67500 \cdot 10^{-9} \units{Кл} = 67500\units{нКл}
    \implies q' \ge q \implies \text{удастся}$
}
\vspace{120pt}

\tasknumber{3}\task{
    Как и во сколько раз изменится ёмкость плоского конденсатора при уменьшении площади пластин в 6 раз
    и уменьшении расстояния между ними в 8 раз?
}
\answer{
    $\frac{C'}{C}
        = \frac{\eps_0\eps \frac S6}{\frac d8} \Big/ \frac{\eps_0\eps S}{d}
        = \frac{ 8 }{ 6 } > 1 \implies \text{увеличится в $\frac43$ раз}
    $
}
\vspace{120pt}

\tasknumber{4}\task{
    Электрическая ёмкость конденсатора равна $C=600\units{пФ}$,
    при этом ему сообщён заряд $q=500\units{нКл}$.
    Какова энергия заряженного конденсатора?
}
\answer{
    $W
        = \frac{ q^2 }{ 2C }
        = \frac{ \sqr{ 500 \cdot 10^{-9} \units{Кл} } }{ 2 \cdot 600 \cdot 10^{-12} \units{Ф} }
        = 208{,}33 \cdot 10^{-6} \units{Дж} = 208{,}33\units{мкДж}
    $
}
\vspace{120pt}

\tasknumber{5}\task{
    Два конденсатора ёмкостей $C_1=20\units{нФ}$ и $C_2=30\units{нФ}$ последовательно подключают
    к источнику напряжения $V=200\units{В}$ (см.
    рис.).
    Определите заряды каждого из конденсаторов.
}
\answer{
    $Q_1
        = Q_2
        = CV
        = \frac{V}{\frac1{C_1} + \frac1{C_2}}
        = \frac{C_1C_2V}{C_1 + C_2}
        = \frac{
            20 \cdot 10^{-9} \units{Ф} \cdot 30 \cdot 10^{-9} \units{Ф} \cdot 200 \units{В}
        }{
            20 \cdot 10^{-9} \units{Ф} + 30 \cdot 10^{-9} \units{Ф}
        }
        = 2400{,}00 \cdot 10^{-9} \units{Кл}
        = 2400{,}00\units{нКл}
    $
}
\newpage

\addpersonalvariant{Георгий Новиков}
\tasknumber{1}\task{
    Определите ёмкость конденсатора, если при его зарядке до напряжения
    $U=3\units{кВ}$ он приобретает заряд $Q=24\units{нКл}$.
    Чему при этом равны заряды обкладок конденсатора (сделайте рисунок)?
}
\answer{
    $Q = CU \implies
    C = \frac{ Q }{ U } = \frac{ 24 \cdot 10^{-9} \units{Кл} }{ 3 \cdot 10^{3} \units{В} } = 8{,}00 \cdot 10^{-12} \units{Ф} = 8{,}00\units{пФ}.
    \text{ Заряды обкладок: $Q$ и $-Q$}$
}
\vspace{120pt}

\tasknumber{2}\task{
    На конденсаторе указано: $C=50\units{пФ}$, $V=400\units{кВ}$.
    Удастся ли его использовать для накопления заряда $Q=30\units{нКл}$?
}
\answer{
    $Q' = CV = 50 \cdot 10^{-12} \units{Ф}\cdot400 \cdot 10^{3} \units{В} = 20000 \cdot 10^{-9} \units{Кл} = 20000\units{нКл}
    \implies Q' \ge Q \implies \text{удастся}$
}
\vspace{120pt}

\tasknumber{3}\task{
    Как и во сколько раз изменится ёмкость плоского конденсатора при уменьшении площади пластин в 8 раз
    и уменьшении расстояния между ними в 5 раз?
}
\answer{
    $\frac{C'}{C}
        = \frac{\eps_0\eps \frac S8}{\frac d5} \Big/ \frac{\eps_0\eps S}{d}
        = \frac{ 5 }{ 8 } < 1 \implies \text{уменьшится в $\frac85$ раз}
    $
}
\vspace{120pt}

\tasknumber{4}\task{
    Электрическая ёмкость конденсатора равна $C=600\units{пФ}$,
    при этом ему сообщён заряд $Q=300\units{нКл}$.
    Какова энергия заряженного конденсатора?
}
\answer{
    $W
        = \frac{ Q^2 }{ 2C }
        = \frac{ \sqr{ 300 \cdot 10^{-9} \units{Кл} } }{ 2 \cdot 600 \cdot 10^{-12} \units{Ф} }
        = 75{,}00 \cdot 10^{-6} \units{Дж} = 75{,}00\units{мкДж}
    $
}
\vspace{120pt}

\tasknumber{5}\task{
    Два конденсатора ёмкостей $C_1=20\units{нФ}$ и $C_2=60\units{нФ}$ последовательно подключают
    к источнику напряжения $U=450\units{В}$ (см.
    рис.).
    Определите заряды каждого из конденсаторов.
}
\answer{
    $Q_1
        = Q_2
        = CU
        = \frac{U}{\frac1{C_1} + \frac1{C_2}}
        = \frac{C_1C_2U}{C_1 + C_2}
        = \frac{
            20 \cdot 10^{-9} \units{Ф} \cdot 60 \cdot 10^{-9} \units{Ф} \cdot 450 \units{В}
        }{
            20 \cdot 10^{-9} \units{Ф} + 60 \cdot 10^{-9} \units{Ф}
        }
        = 6750{,}00 \cdot 10^{-9} \units{Кл}
        = 6750{,}00\units{нКл}
    $
}
\newpage

\addpersonalvariant{Егор Осипов}
\tasknumber{1}\task{
    Определите ёмкость конденсатора, если при его зарядке до напряжения
    $V=6\units{кВ}$ он приобретает заряд $q=15\units{нКл}$.
    Чему при этом равны заряды обкладок конденсатора (сделайте рисунок)?
}
\answer{
    $q = CV \implies
    C = \frac{ q }{ V } = \frac{ 15 \cdot 10^{-9} \units{Кл} }{ 6 \cdot 10^{3} \units{В} } = 2{,}50 \cdot 10^{-12} \units{Ф} = 2{,}50\units{пФ}.
    \text{ Заряды обкладок: $q$ и $-q$}$
}
\vspace{120pt}

\tasknumber{2}\task{
    На конденсаторе указано: $C=100\units{пФ}$, $V=200\units{кВ}$.
    Удастся ли его использовать для накопления заряда $Q=60\units{нКл}$?
}
\answer{
    $Q' = CV = 100 \cdot 10^{-12} \units{Ф}\cdot200 \cdot 10^{3} \units{В} = 20000 \cdot 10^{-9} \units{Кл} = 20000\units{нКл}
    \implies Q' \ge Q \implies \text{удастся}$
}
\vspace{120pt}

\tasknumber{3}\task{
    Как и во сколько раз изменится ёмкость плоского конденсатора при уменьшении площади пластин в 8 раз
    и уменьшении расстояния между ними в 2 раз?
}
\answer{
    $\frac{C'}{C}
        = \frac{\eps_0\eps \frac S8}{\frac d2} \Big/ \frac{\eps_0\eps S}{d}
        = \frac{ 2 }{ 8 } < 1 \implies \text{уменьшится в $\frac41$ раз}
    $
}
\vspace{120pt}

\tasknumber{4}\task{
    Электрическая ёмкость конденсатора равна $C=600\units{пФ}$,
    при этом ему сообщён заряд $q=500\units{нКл}$.
    Какова энергия заряженного конденсатора?
}
\answer{
    $W
        = \frac{ q^2 }{ 2C }
        = \frac{ \sqr{ 500 \cdot 10^{-9} \units{Кл} } }{ 2 \cdot 600 \cdot 10^{-12} \units{Ф} }
        = 208{,}33 \cdot 10^{-6} \units{Дж} = 208{,}33\units{мкДж}
    $
}
\vspace{120pt}

\tasknumber{5}\task{
    Два конденсатора ёмкостей $C_1=30\units{нФ}$ и $C_2=20\units{нФ}$ последовательно подключают
    к источнику напряжения $V=400\units{В}$ (см.
    рис.).
    Определите заряды каждого из конденсаторов.
}
\answer{
    $Q_1
        = Q_2
        = CV
        = \frac{V}{\frac1{C_1} + \frac1{C_2}}
        = \frac{C_1C_2V}{C_1 + C_2}
        = \frac{
            30 \cdot 10^{-9} \units{Ф} \cdot 20 \cdot 10^{-9} \units{Ф} \cdot 400 \units{В}
        }{
            30 \cdot 10^{-9} \units{Ф} + 20 \cdot 10^{-9} \units{Ф}
        }
        = 4800{,}00 \cdot 10^{-9} \units{Кл}
        = 4800{,}00\units{нКл}
    $
}
\newpage

\addpersonalvariant{Руслан Перепелица}
\tasknumber{1}\task{
    Определите ёмкость конденсатора, если при его зарядке до напряжения
    $V=2\units{кВ}$ он приобретает заряд $q=24\units{нКл}$.
    Чему при этом равны заряды обкладок конденсатора (сделайте рисунок)?
}
\answer{
    $q = CV \implies
    C = \frac{ q }{ V } = \frac{ 24 \cdot 10^{-9} \units{Кл} }{ 2 \cdot 10^{3} \units{В} } = 12{,}00 \cdot 10^{-12} \units{Ф} = 12{,}00\units{пФ}.
    \text{ Заряды обкладок: $q$ и $-q$}$
}
\vspace{120pt}

\tasknumber{2}\task{
    На конденсаторе указано: $C=80\units{пФ}$, $U=300\units{кВ}$.
    Удастся ли его использовать для накопления заряда $Q=30\units{нКл}$?
}
\answer{
    $Q' = CU = 80 \cdot 10^{-12} \units{Ф}\cdot300 \cdot 10^{3} \units{В} = 24000 \cdot 10^{-9} \units{Кл} = 24000\units{нКл}
    \implies Q' \ge Q \implies \text{удастся}$
}
\vspace{120pt}

\tasknumber{3}\task{
    Как и во сколько раз изменится ёмкость плоского конденсатора при уменьшении площади пластин в 3 раз
    и уменьшении расстояния между ними в 2 раз?
}
\answer{
    $\frac{C'}{C}
        = \frac{\eps_0\eps \frac S3}{\frac d2} \Big/ \frac{\eps_0\eps S}{d}
        = \frac{ 2 }{ 3 } < 1 \implies \text{уменьшится в $\frac32$ раз}
    $
}
\vspace{120pt}

\tasknumber{4}\task{
    Электрическая ёмкость конденсатора равна $C=600\units{пФ}$,
    при этом ему сообщён заряд $Q=800\units{нКл}$.
    Какова энергия заряженного конденсатора?
}
\answer{
    $W
        = \frac{ Q^2 }{ 2C }
        = \frac{ \sqr{ 800 \cdot 10^{-9} \units{Кл} } }{ 2 \cdot 600 \cdot 10^{-12} \units{Ф} }
        = 533{,}33 \cdot 10^{-6} \units{Дж} = 533{,}33\units{мкДж}
    $
}
\vspace{120pt}

\tasknumber{5}\task{
    Два конденсатора ёмкостей $C_1=60\units{нФ}$ и $C_2=30\units{нФ}$ последовательно подключают
    к источнику напряжения $V=200\units{В}$ (см.
    рис.).
    Определите заряды каждого из конденсаторов.
}
\answer{
    $Q_1
        = Q_2
        = CV
        = \frac{V}{\frac1{C_1} + \frac1{C_2}}
        = \frac{C_1C_2V}{C_1 + C_2}
        = \frac{
            60 \cdot 10^{-9} \units{Ф} \cdot 30 \cdot 10^{-9} \units{Ф} \cdot 200 \units{В}
        }{
            60 \cdot 10^{-9} \units{Ф} + 30 \cdot 10^{-9} \units{Ф}
        }
        = 4000{,}00 \cdot 10^{-9} \units{Кл}
        = 4000{,}00\units{нКл}
    $
}
\newpage

\addpersonalvariant{Михаил Перин}
\tasknumber{1}\task{
    Определите ёмкость конденсатора, если при его зарядке до напряжения
    $V=2\units{кВ}$ он приобретает заряд $q=6\units{нКл}$.
    Чему при этом равны заряды обкладок конденсатора (сделайте рисунок)?
}
\answer{
    $q = CV \implies
    C = \frac{ q }{ V } = \frac{ 6 \cdot 10^{-9} \units{Кл} }{ 2 \cdot 10^{3} \units{В} } = 3{,}00 \cdot 10^{-12} \units{Ф} = 3{,}00\units{пФ}.
    \text{ Заряды обкладок: $q$ и $-q$}$
}
\vspace{120pt}

\tasknumber{2}\task{
    На конденсаторе указано: $C=100\units{пФ}$, $V=300\units{кВ}$.
    Удастся ли его использовать для накопления заряда $Q=30\units{нКл}$?
}
\answer{
    $Q' = CV = 100 \cdot 10^{-12} \units{Ф}\cdot300 \cdot 10^{3} \units{В} = 30000 \cdot 10^{-9} \units{Кл} = 30000\units{нКл}
    \implies Q' \ge Q \implies \text{удастся}$
}
\vspace{120pt}

\tasknumber{3}\task{
    Как и во сколько раз изменится ёмкость плоского конденсатора при уменьшении площади пластин в 8 раз
    и уменьшении расстояния между ними в 7 раз?
}
\answer{
    $\frac{C'}{C}
        = \frac{\eps_0\eps \frac S8}{\frac d7} \Big/ \frac{\eps_0\eps S}{d}
        = \frac{ 7 }{ 8 } < 1 \implies \text{уменьшится в $\frac87$ раз}
    $
}
\vspace{120pt}

\tasknumber{4}\task{
    Электрическая ёмкость конденсатора равна $C=600\units{пФ}$,
    при этом ему сообщён заряд $Q=800\units{нКл}$.
    Какова энергия заряженного конденсатора?
}
\answer{
    $W
        = \frac{ Q^2 }{ 2C }
        = \frac{ \sqr{ 800 \cdot 10^{-9} \units{Кл} } }{ 2 \cdot 600 \cdot 10^{-12} \units{Ф} }
        = 533{,}33 \cdot 10^{-6} \units{Дж} = 533{,}33\units{мкДж}
    $
}
\vspace{120pt}

\tasknumber{5}\task{
    Два конденсатора ёмкостей $C_1=60\units{нФ}$ и $C_2=30\units{нФ}$ последовательно подключают
    к источнику напряжения $V=400\units{В}$ (см.
    рис.).
    Определите заряды каждого из конденсаторов.
}
\answer{
    $Q_1
        = Q_2
        = CV
        = \frac{V}{\frac1{C_1} + \frac1{C_2}}
        = \frac{C_1C_2V}{C_1 + C_2}
        = \frac{
            60 \cdot 10^{-9} \units{Ф} \cdot 30 \cdot 10^{-9} \units{Ф} \cdot 400 \units{В}
        }{
            60 \cdot 10^{-9} \units{Ф} + 30 \cdot 10^{-9} \units{Ф}
        }
        = 8000{,}00 \cdot 10^{-9} \units{Кл}
        = 8000{,}00\units{нКл}
    $
}
\newpage

\addpersonalvariant{Егор Подуровский}
\tasknumber{1}\task{
    Определите ёмкость конденсатора, если при его зарядке до напряжения
    $U=15\units{кВ}$ он приобретает заряд $q=4\units{нКл}$.
    Чему при этом равны заряды обкладок конденсатора (сделайте рисунок)?
}
\answer{
    $q = CU \implies
    C = \frac{ q }{ U } = \frac{ 4 \cdot 10^{-9} \units{Кл} }{ 15 \cdot 10^{3} \units{В} } = 0{,}27 \cdot 10^{-12} \units{Ф} = 0{,}27\units{пФ}.
    \text{ Заряды обкладок: $q$ и $-q$}$
}
\vspace{120pt}

\tasknumber{2}\task{
    На конденсаторе указано: $C=150\units{пФ}$, $U=300\units{кВ}$.
    Удастся ли его использовать для накопления заряда $Q=60\units{нКл}$?
}
\answer{
    $Q' = CU = 150 \cdot 10^{-12} \units{Ф}\cdot300 \cdot 10^{3} \units{В} = 45000 \cdot 10^{-9} \units{Кл} = 45000\units{нКл}
    \implies Q' \ge Q \implies \text{удастся}$
}
\vspace{120pt}

\tasknumber{3}\task{
    Как и во сколько раз изменится ёмкость плоского конденсатора при уменьшении площади пластин в 2 раз
    и уменьшении расстояния между ними в 3 раз?
}
\answer{
    $\frac{C'}{C}
        = \frac{\eps_0\eps \frac S2}{\frac d3} \Big/ \frac{\eps_0\eps S}{d}
        = \frac{ 3 }{ 2 } > 1 \implies \text{увеличится в $\frac32$ раз}
    $
}
\vspace{120pt}

\tasknumber{4}\task{
    Электрическая ёмкость конденсатора равна $C=200\units{пФ}$,
    при этом ему сообщён заряд $q=800\units{нКл}$.
    Какова энергия заряженного конденсатора?
}
\answer{
    $W
        = \frac{ q^2 }{ 2C }
        = \frac{ \sqr{ 800 \cdot 10^{-9} \units{Кл} } }{ 2 \cdot 200 \cdot 10^{-12} \units{Ф} }
        = 1600{,}00 \cdot 10^{-6} \units{Дж} = 1600{,}00\units{мкДж}
    $
}
\vspace{120pt}

\tasknumber{5}\task{
    Два конденсатора ёмкостей $C_1=40\units{нФ}$ и $C_2=60\units{нФ}$ последовательно подключают
    к источнику напряжения $V=400\units{В}$ (см.
    рис.).
    Определите заряды каждого из конденсаторов.
}
\answer{
    $Q_1
        = Q_2
        = CV
        = \frac{V}{\frac1{C_1} + \frac1{C_2}}
        = \frac{C_1C_2V}{C_1 + C_2}
        = \frac{
            40 \cdot 10^{-9} \units{Ф} \cdot 60 \cdot 10^{-9} \units{Ф} \cdot 400 \units{В}
        }{
            40 \cdot 10^{-9} \units{Ф} + 60 \cdot 10^{-9} \units{Ф}
        }
        = 9600{,}00 \cdot 10^{-9} \units{Кл}
        = 9600{,}00\units{нКл}
    $
}
\newpage

\addpersonalvariant{Роман Прибылов}
\tasknumber{1}\task{
    Определите ёмкость конденсатора, если при его зарядке до напряжения
    $U=20\units{кВ}$ он приобретает заряд $q=4\units{нКл}$.
    Чему при этом равны заряды обкладок конденсатора (сделайте рисунок)?
}
\answer{
    $q = CU \implies
    C = \frac{ q }{ U } = \frac{ 4 \cdot 10^{-9} \units{Кл} }{ 20 \cdot 10^{3} \units{В} } = 0{,}20 \cdot 10^{-12} \units{Ф} = 0{,}20\units{пФ}.
    \text{ Заряды обкладок: $q$ и $-q$}$
}
\vspace{120pt}

\tasknumber{2}\task{
    На конденсаторе указано: $C=100\units{пФ}$, $U=300\units{кВ}$.
    Удастся ли его использовать для накопления заряда $q=60\units{нКл}$?
}
\answer{
    $q' = CU = 100 \cdot 10^{-12} \units{Ф}\cdot300 \cdot 10^{3} \units{В} = 30000 \cdot 10^{-9} \units{Кл} = 30000\units{нКл}
    \implies q' \ge q \implies \text{удастся}$
}
\vspace{120pt}

\tasknumber{3}\task{
    Как и во сколько раз изменится ёмкость плоского конденсатора при уменьшении площади пластин в 7 раз
    и уменьшении расстояния между ними в 3 раз?
}
\answer{
    $\frac{C'}{C}
        = \frac{\eps_0\eps \frac S7}{\frac d3} \Big/ \frac{\eps_0\eps S}{d}
        = \frac{ 3 }{ 7 } < 1 \implies \text{уменьшится в $\frac73$ раз}
    $
}
\vspace{120pt}

\tasknumber{4}\task{
    Электрическая ёмкость конденсатора равна $C=400\units{пФ}$,
    при этом ему сообщён заряд $q=300\units{нКл}$.
    Какова энергия заряженного конденсатора?
}
\answer{
    $W
        = \frac{ q^2 }{ 2C }
        = \frac{ \sqr{ 300 \cdot 10^{-9} \units{Кл} } }{ 2 \cdot 400 \cdot 10^{-12} \units{Ф} }
        = 112{,}50 \cdot 10^{-6} \units{Дж} = 112{,}50\units{мкДж}
    $
}
\vspace{120pt}

\tasknumber{5}\task{
    Два конденсатора ёмкостей $C_1=20\units{нФ}$ и $C_2=40\units{нФ}$ последовательно подключают
    к источнику напряжения $V=300\units{В}$ (см.
    рис.).
    Определите заряды каждого из конденсаторов.
}
\answer{
    $Q_1
        = Q_2
        = CV
        = \frac{V}{\frac1{C_1} + \frac1{C_2}}
        = \frac{C_1C_2V}{C_1 + C_2}
        = \frac{
            20 \cdot 10^{-9} \units{Ф} \cdot 40 \cdot 10^{-9} \units{Ф} \cdot 300 \units{В}
        }{
            20 \cdot 10^{-9} \units{Ф} + 40 \cdot 10^{-9} \units{Ф}
        }
        = 4000{,}00 \cdot 10^{-9} \units{Кл}
        = 4000{,}00\units{нКл}
    $
}
\newpage

\addpersonalvariant{Александр Селехметьев}
\tasknumber{1}\task{
    Определите ёмкость конденсатора, если при его зарядке до напряжения
    $U=15\units{кВ}$ он приобретает заряд $Q=18\units{нКл}$.
    Чему при этом равны заряды обкладок конденсатора (сделайте рисунок)?
}
\answer{
    $Q = CU \implies
    C = \frac{ Q }{ U } = \frac{ 18 \cdot 10^{-9} \units{Кл} }{ 15 \cdot 10^{3} \units{В} } = 1{,}20 \cdot 10^{-12} \units{Ф} = 1{,}20\units{пФ}.
    \text{ Заряды обкладок: $Q$ и $-Q$}$
}
\vspace{120pt}

\tasknumber{2}\task{
    На конденсаторе указано: $C=100\units{пФ}$, $V=400\units{кВ}$.
    Удастся ли его использовать для накопления заряда $q=60\units{нКл}$?
}
\answer{
    $q' = CV = 100 \cdot 10^{-12} \units{Ф}\cdot400 \cdot 10^{3} \units{В} = 40000 \cdot 10^{-9} \units{Кл} = 40000\units{нКл}
    \implies q' \ge q \implies \text{удастся}$
}
\vspace{120pt}

\tasknumber{3}\task{
    Как и во сколько раз изменится ёмкость плоского конденсатора при уменьшении площади пластин в 2 раз
    и уменьшении расстояния между ними в 6 раз?
}
\answer{
    $\frac{C'}{C}
        = \frac{\eps_0\eps \frac S2}{\frac d6} \Big/ \frac{\eps_0\eps S}{d}
        = \frac{ 6 }{ 2 } > 1 \implies \text{увеличится в $\frac31$ раз}
    $
}
\vspace{120pt}

\tasknumber{4}\task{
    Электрическая ёмкость конденсатора равна $C=400\units{пФ}$,
    при этом ему сообщён заряд $Q=300\units{нКл}$.
    Какова энергия заряженного конденсатора?
}
\answer{
    $W
        = \frac{ Q^2 }{ 2C }
        = \frac{ \sqr{ 300 \cdot 10^{-9} \units{Кл} } }{ 2 \cdot 400 \cdot 10^{-12} \units{Ф} }
        = 112{,}50 \cdot 10^{-6} \units{Дж} = 112{,}50\units{мкДж}
    $
}
\vspace{120pt}

\tasknumber{5}\task{
    Два конденсатора ёмкостей $C_1=40\units{нФ}$ и $C_2=20\units{нФ}$ последовательно подключают
    к источнику напряжения $U=400\units{В}$ (см.
    рис.).
    Определите заряды каждого из конденсаторов.
}
\answer{
    $Q_1
        = Q_2
        = CU
        = \frac{U}{\frac1{C_1} + \frac1{C_2}}
        = \frac{C_1C_2U}{C_1 + C_2}
        = \frac{
            40 \cdot 10^{-9} \units{Ф} \cdot 20 \cdot 10^{-9} \units{Ф} \cdot 400 \units{В}
        }{
            40 \cdot 10^{-9} \units{Ф} + 20 \cdot 10^{-9} \units{Ф}
        }
        = 5333{,}33 \cdot 10^{-9} \units{Кл}
        = 5333{,}33\units{нКл}
    $
}
\newpage

\addpersonalvariant{Алексей Тихонов}
\tasknumber{1}\task{
    Определите ёмкость конденсатора, если при его зарядке до напряжения
    $V=15\units{кВ}$ он приобретает заряд $q=4\units{нКл}$.
    Чему при этом равны заряды обкладок конденсатора (сделайте рисунок)?
}
\answer{
    $q = CV \implies
    C = \frac{ q }{ V } = \frac{ 4 \cdot 10^{-9} \units{Кл} }{ 15 \cdot 10^{3} \units{В} } = 0{,}27 \cdot 10^{-12} \units{Ф} = 0{,}27\units{пФ}.
    \text{ Заряды обкладок: $q$ и $-q$}$
}
\vspace{120pt}

\tasknumber{2}\task{
    На конденсаторе указано: $C=80\units{пФ}$, $V=200\units{кВ}$.
    Удастся ли его использовать для накопления заряда $q=50\units{нКл}$?
}
\answer{
    $q' = CV = 80 \cdot 10^{-12} \units{Ф}\cdot200 \cdot 10^{3} \units{В} = 16000 \cdot 10^{-9} \units{Кл} = 16000\units{нКл}
    \implies q' \ge q \implies \text{удастся}$
}
\vspace{120pt}

\tasknumber{3}\task{
    Как и во сколько раз изменится ёмкость плоского конденсатора при уменьшении площади пластин в 7 раз
    и уменьшении расстояния между ними в 8 раз?
}
\answer{
    $\frac{C'}{C}
        = \frac{\eps_0\eps \frac S7}{\frac d8} \Big/ \frac{\eps_0\eps S}{d}
        = \frac{ 8 }{ 7 } > 1 \implies \text{увеличится в $\frac87$ раз}
    $
}
\vspace{120pt}

\tasknumber{4}\task{
    Электрическая ёмкость конденсатора равна $C=200\units{пФ}$,
    при этом ему сообщён заряд $Q=300\units{нКл}$.
    Какова энергия заряженного конденсатора?
}
\answer{
    $W
        = \frac{ Q^2 }{ 2C }
        = \frac{ \sqr{ 300 \cdot 10^{-9} \units{Кл} } }{ 2 \cdot 200 \cdot 10^{-12} \units{Ф} }
        = 225{,}00 \cdot 10^{-6} \units{Дж} = 225{,}00\units{мкДж}
    $
}
\vspace{120pt}

\tasknumber{5}\task{
    Два конденсатора ёмкостей $C_1=40\units{нФ}$ и $C_2=20\units{нФ}$ последовательно подключают
    к источнику напряжения $U=200\units{В}$ (см.
    рис.).
    Определите заряды каждого из конденсаторов.
}
\answer{
    $Q_1
        = Q_2
        = CU
        = \frac{U}{\frac1{C_1} + \frac1{C_2}}
        = \frac{C_1C_2U}{C_1 + C_2}
        = \frac{
            40 \cdot 10^{-9} \units{Ф} \cdot 20 \cdot 10^{-9} \units{Ф} \cdot 200 \units{В}
        }{
            40 \cdot 10^{-9} \units{Ф} + 20 \cdot 10^{-9} \units{Ф}
        }
        = 2666{,}67 \cdot 10^{-9} \units{Кл}
        = 2666{,}67\units{нКл}
    $
}
\newpage

\addpersonalvariant{Алина Филиппова}
\tasknumber{1}\task{
    Определите ёмкость конденсатора, если при его зарядке до напряжения
    $U=2\units{кВ}$ он приобретает заряд $Q=6\units{нКл}$.
    Чему при этом равны заряды обкладок конденсатора (сделайте рисунок)?
}
\answer{
    $Q = CU \implies
    C = \frac{ Q }{ U } = \frac{ 6 \cdot 10^{-9} \units{Кл} }{ 2 \cdot 10^{3} \units{В} } = 3{,}00 \cdot 10^{-12} \units{Ф} = 3{,}00\units{пФ}.
    \text{ Заряды обкладок: $Q$ и $-Q$}$
}
\vspace{120pt}

\tasknumber{2}\task{
    На конденсаторе указано: $C=120\units{пФ}$, $V=200\units{кВ}$.
    Удастся ли его использовать для накопления заряда $q=30\units{нКл}$?
}
\answer{
    $q' = CV = 120 \cdot 10^{-12} \units{Ф}\cdot200 \cdot 10^{3} \units{В} = 24000 \cdot 10^{-9} \units{Кл} = 24000\units{нКл}
    \implies q' \ge q \implies \text{удастся}$
}
\vspace{120pt}

\tasknumber{3}\task{
    Как и во сколько раз изменится ёмкость плоского конденсатора при уменьшении площади пластин в 6 раз
    и уменьшении расстояния между ними в 4 раз?
}
\answer{
    $\frac{C'}{C}
        = \frac{\eps_0\eps \frac S6}{\frac d4} \Big/ \frac{\eps_0\eps S}{d}
        = \frac{ 4 }{ 6 } < 1 \implies \text{уменьшится в $\frac32$ раз}
    $
}
\vspace{120pt}

\tasknumber{4}\task{
    Электрическая ёмкость конденсатора равна $C=400\units{пФ}$,
    при этом ему сообщён заряд $q=500\units{нКл}$.
    Какова энергия заряженного конденсатора?
}
\answer{
    $W
        = \frac{ q^2 }{ 2C }
        = \frac{ \sqr{ 500 \cdot 10^{-9} \units{Кл} } }{ 2 \cdot 400 \cdot 10^{-12} \units{Ф} }
        = 312{,}50 \cdot 10^{-6} \units{Дж} = 312{,}50\units{мкДж}
    $
}
\vspace{120pt}

\tasknumber{5}\task{
    Два конденсатора ёмкостей $C_1=30\units{нФ}$ и $C_2=20\units{нФ}$ последовательно подключают
    к источнику напряжения $U=150\units{В}$ (см.
    рис.).
    Определите заряды каждого из конденсаторов.
}
\answer{
    $Q_1
        = Q_2
        = CU
        = \frac{U}{\frac1{C_1} + \frac1{C_2}}
        = \frac{C_1C_2U}{C_1 + C_2}
        = \frac{
            30 \cdot 10^{-9} \units{Ф} \cdot 20 \cdot 10^{-9} \units{Ф} \cdot 150 \units{В}
        }{
            30 \cdot 10^{-9} \units{Ф} + 20 \cdot 10^{-9} \units{Ф}
        }
        = 1800{,}00 \cdot 10^{-9} \units{Кл}
        = 1800{,}00\units{нКл}
    $
}
\newpage

\addpersonalvariant{Алина Яшина}
\tasknumber{1}\task{
    Определите ёмкость конденсатора, если при его зарядке до напряжения
    $U=12\units{кВ}$ он приобретает заряд $Q=6\units{нКл}$.
    Чему при этом равны заряды обкладок конденсатора (сделайте рисунок)?
}
\answer{
    $Q = CU \implies
    C = \frac{ Q }{ U } = \frac{ 6 \cdot 10^{-9} \units{Кл} }{ 12 \cdot 10^{3} \units{В} } = 0{,}50 \cdot 10^{-12} \units{Ф} = 0{,}50\units{пФ}.
    \text{ Заряды обкладок: $Q$ и $-Q$}$
}
\vspace{120pt}

\tasknumber{2}\task{
    На конденсаторе указано: $C=150\units{пФ}$, $U=300\units{кВ}$.
    Удастся ли его использовать для накопления заряда $Q=30\units{нКл}$?
}
\answer{
    $Q' = CU = 150 \cdot 10^{-12} \units{Ф}\cdot300 \cdot 10^{3} \units{В} = 45000 \cdot 10^{-9} \units{Кл} = 45000\units{нКл}
    \implies Q' \ge Q \implies \text{удастся}$
}
\vspace{120pt}

\tasknumber{3}\task{
    Как и во сколько раз изменится ёмкость плоского конденсатора при уменьшении площади пластин в 2 раз
    и уменьшении расстояния между ними в 4 раз?
}
\answer{
    $\frac{C'}{C}
        = \frac{\eps_0\eps \frac S2}{\frac d4} \Big/ \frac{\eps_0\eps S}{d}
        = \frac{ 4 }{ 2 } > 1 \implies \text{увеличится в $\frac21$ раз}
    $
}
\vspace{120pt}

\tasknumber{4}\task{
    Электрическая ёмкость конденсатора равна $C=200\units{пФ}$,
    при этом ему сообщён заряд $q=500\units{нКл}$.
    Какова энергия заряженного конденсатора?
}
\answer{
    $W
        = \frac{ q^2 }{ 2C }
        = \frac{ \sqr{ 500 \cdot 10^{-9} \units{Кл} } }{ 2 \cdot 200 \cdot 10^{-12} \units{Ф} }
        = 625{,}00 \cdot 10^{-6} \units{Дж} = 625{,}00\units{мкДж}
    $
}
\vspace{120pt}

\tasknumber{5}\task{
    Два конденсатора ёмкостей $C_1=40\units{нФ}$ и $C_2=60\units{нФ}$ последовательно подключают
    к источнику напряжения $U=200\units{В}$ (см.
    рис.).
    Определите заряды каждого из конденсаторов.
}
\answer{
    $Q_1
        = Q_2
        = CU
        = \frac{U}{\frac1{C_1} + \frac1{C_2}}
        = \frac{C_1C_2U}{C_1 + C_2}
        = \frac{
            40 \cdot 10^{-9} \units{Ф} \cdot 60 \cdot 10^{-9} \units{Ф} \cdot 200 \units{В}
        }{
            40 \cdot 10^{-9} \units{Ф} + 60 \cdot 10^{-9} \units{Ф}
        }
        = 4800{,}00 \cdot 10^{-9} \units{Кл}
        = 4800{,}00\units{нКл}
    $
}

\end{document}
% autogenerated
