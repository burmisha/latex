\documentclass[12pt,a4paper]{amsart}%DVI-mode.
\usepackage{graphics,graphicx,epsfig}%DVI-mode.
%\documentclass[pdftex,12pt]{amsart} %PDF-mode.
%\usepackage[pdftex]{graphicx}       %PDF-mode.

%\usepackage{a4wide}                 % Fit the text to A4 page tightly.
\usepackage[utf8]{inputenc}
\usepackage[T2A]{fontenc}
\usepackage[english,russian]{babel} % Download Russian fonts.
\usepackage{amsmath,amsfonts,amssymb,amsthm,amscd,mathrsfs} % Use AMS symbols.
\usepackage{tikz}
\usetikzlibrary{circuits.ee.IEC}
\usetikzlibrary{shapes.geometric}
\usetikzlibrary{decorations.markings}
%\usetikzlibrary{dashs}
%\usetikzlibrary{info}


\textheight=29cm % высота текста
\textwidth=18cm % ширина текста
\topmargin=-2.5cm % отступ от верхнего края
\parskip=6pt % интервал между абзацами
\oddsidemargin=-1.5cm
\evensidemargin=-1.5cm 

% wide docs
% \oddsidemargin=0cm
% \evensidemargin=0cm 
% \textheight=29cm % высота текста
% \textwidth=15cm % ширина текста
% \topmargin=-1.5cm % отступ от верхнего края
% \parskip=18pt % интервал между абзацами


\parindent=0pt % абзацный отступ
\tolerance=500 % терпимость к "жидким" строкам
\binoppenalty=10000 % штраф за перенос формул - 10000 - абсолютный запрет
\relpenalty=10000
\flushbottom % выравнивание высоты страниц
\def\baselinestretch{1.00}
\pagenumbering{gobble}

\begin{document}
\newcommand\bivec[2]{\begin{pmatrix} #1 \\ #2 \end{pmatrix}}

\newcommand\ol[1]{\overline{#1}}

\newcommand\p[1]{\ensuremath{\Prob\!\left(#1\right)}}
\def\cond{\,|\,}
\newcommand\e[1]{\mathsf{E}\!\left(#1\right)}
\newcommand\disp[1]{\mathsf{D}\!\left(#1\right)}
%\newcommand\norm[2]{\mathcal{N}\!\cbr{#1,#2}}
\newcommand\sign{\text{ sign }}

\newcommand\al[1]{\begin{align*} #1 \end{align*}}
\newcommand\begcas[1]{\begin{cases}#1\end{cases}}
\newcommand\tab[2]{	\vspace{-#1pt}
						\begin{tabbing} 
						#2
						\end{tabbing}
					\vspace{-#1pt}
					}


\newcommand\maintext[1]{{\bfseries\sffamily{#1}}}
\newcommand\simpletitle[1]{\begin{center} \maintext{#1} \end{center}}

\def\le{\leqslant}
\def\ge{\geqslant}
\def\Ell{\mathcal{L}}
\def\eps{\varepsilon}
\def\x{\ensuremath{\textbf{x}}}
\def\y{\ensuremath{\textbf{y}}}
\def\Rn{\ensuremath{\mathbb{R}^n}}
\def\RSS{\mathsf{RSS}}

\newcommand\mb[1]{\ensuremath{\boldsymbol{\mathbf{#1}}}}
\newcommand\argmax[1]{\arg\underset{#1}\max\,} % \operatornamewithlimits
%\newcommand{\prodl}{\mathop{\textstyle\prod}\limits}
\newcommand{\prodl}{\prod\limits}
\newcommand{\suml}{\sum\limits}
\newcommand\foral[1]{\forall\,#1\:}
\newcommand\exist[1]{\exists\,#1\:\colon}

\newcommand\cbr[1]{\left(#1\right)} %circled brackets
\newcommand\fbr[1]{\left\{#1\right\}} %figure brackets
\newcommand\sbr[1]{\left[#1\right]} %square brackets
\newcommand\modul[1]{\left|#1\right|}
\newcommand\cdf[2]{\cdot\frac{#1}{#2}}
\newcommand\integr[3]{\int\limits_{#1}^{#2}{#3}}
\newcommand\obol[1]{O\!\cbr{#1}}
\newcommand\norm[1]{\ensuremath{\left\|{#1}\right\|}}

\newcommand\dd[2]{\frac{\partial#1}{\partial#2}}

\newcommand\addeps[2]{
	\begin{figure} [!ht] %lrp
		\centering
		\includegraphics[height=240px]{#1.eps}
		\vspace{-10pt}
		\caption{#2}
		\label{eps:#1}
	\end{figure}
}

\newcommand\addtikz[4]{
	\begin{figure} [!ht] %lrp
		\centering
		\begin{tikzpicture}[x=#2cm,y=#2cm,#3]
			\input{#1.tikz}
		\end{tikzpicture}
		\vspace{-10pt}
		\caption{#4}
		\label{tikz:#1}	
	\end{figure}
}



\newcommand\addepssize[3]{
	\begin{figure} [!ht] %lrp hp
		\centering
		\includegraphics[height=#3px]{#1.eps}
		\vspace{-10pt}
		\caption{#2}
		\label{eps:#1}
	\end{figure}
}

\def\algorithmicrequire{\textbf{Вход:}}
\def\algorithmicensure{\textbf{Выход:}}
\def\algorithmicif{\textbf{если}}
\def\algorithmicthen{\textbf{то}}
\def\algorithmicelse{\textbf{иначе}}
\def\algorithmicelsif{\textbf{иначе если}}
\def\algorithmicfor{\textbf{для}}
\def\algorithmicforall{\textbf{для всех}}
\def\algorithmicdo{}
\def\algorithmicwhile{\textbf{пока}}
\def\algorithmicrepeat{\textbf{повторять}}
\def\algorithmicuntil{\textbf{пока}}
\def\algorithmicloop{\textbf{цикл}}
% переопределение стиля комментариев
\def\algorithmiccomment#1{\quad// {\sl #1}}
%\raggedright
\classdate{7}{20 апреля 2018}

\task 1
Площадь большого поршня гидравлического домкрата $S_1 = 20\units{см}^2$, а малого $S_2 = 0{,}5\units{см}^2.$ Груз какой максимальной массы можно поднять этим домкратом, если на малый поршень давить с силой не более $F=200\units{Н}?$ Силой трения от стенки цилиндров пренебречь.

\task 2
В сосуд налита вода. Расстояние от поверхности воды до дна $H = 0{,}5\units{м},$ площадь дна $S = 0{,}1\units{м}^2.$ Найти гидростатическое давление $P_1$ и полное давление $P_2$ вблизи дна. Найти силу давления воды на дно. Плотность воды \rhowater

\task 3
На лёгкий поршень площадью $S=900\units{см}^2,$ касающийся поверхности воды, поставили гирю массы $m=3\units{кг}$. Высота слоя воды в сосуде с вертикальными стенками $H = 20\units{см}$. Определить давление жидкости вблизи дна, если плотность воды \rhowater

\task 4
Давление газов в конце сгорания в цилиндре дизельного двигателя трактора $P = 9\units{МПа}.$ Диаметр цилиндра $d = 130\units{мм}.$ С какой силой газы давят на поршень в цилиндре? Площадь круга диаметром $D$ равна $S = \cfrac{\pi D^2}4.$

\task 5
Площадь малого поршня гидравлического подъёмника $S_1 = 0{,}8\units{см}^2$, а большого $S_2 = 40\units{см}^2.$ Какую силу $F$ надо приложить к малому поршню, чтобы поднять груз весом $P = 8\units{кН}?$

\task 6
Герметичный сосуд полностью заполнен водой и стоит на столе. На небольшой поршень площадью $S$ давят рукой с силой $F$. Поршень находится ниже крышки сосуда на $H_1$, выше дна на $H_2$ и может свободно перемещаться. Плотность воды $\rho$, атмосферное давление $P_A$. Найти давления $P_1$ и $P_2$ в воде вблизи крышки и дна сосуда.
\\ \\
\classdate{7}{20 апреля 2018}

\task 1
Площадь большого поршня гидравлического домкрата $S_1 = 20\units{см}^2$, а малого $S_2 = 0{,}5\units{см}^2.$ Груз какой максимальной массы можно поднять этим домкратом, если на малый поршень давить с силой не более $F=200\units{Н}?$ Силой трения от стенки цилиндров пренебречь.

\task 2
В сосуд налита вода. Расстояние от поверхности воды до дна $H = 0{,}5\units{м},$ площадь дна $S = 0{,}1\units{м}^2.$ Найти гидростатическое давление $P_1$ и полное давление $P_2$ вблизи дна. Найти силу давления воды на дно. Плотность воды \rhowater

\task 3
На лёгкий поршень площадью $S=900\units{см}^2,$ касающийся поверхности воды, поставили гирю массы $m=3\units{кг}$. Высота слоя воды в сосуде с вертикальными стенками $H = 20\units{см}$. Определить давление жидкости вблизи дна, если плотность воды \rhowater

\task 4
Давление газов в конце сгорания в цилиндре дизельного двигателя трактора $P = 9\units{МПа}.$ Диаметр цилиндра $d = 130\units{мм}.$ С какой силой газы давят на поршень в цилиндре? Площадь круга диаметром $D$ равна $S = \cfrac{\pi D^2}4.$

\task 5
Площадь малого поршня гидравлического подъёмника $S_1 = 0{,}8\units{см}^2$, а большого $S_2 = 40\units{см}^2.$ Какую силу $F$ надо приложить к малому поршню, чтобы поднять груз весом $P = 8\units{кН}?$

\task 6
Герметичный сосуд полностью заполнен водой и стоит на столе. На небольшой поршень площадью $S$ давят рукой с силой $F$. Поршень находится ниже крышки сосуда на $H_1$, выше дна на $H_2$ и может свободно перемещаться. Плотность воды $\rho$, атмосферное давление $P_A$. Найти давления $P_1$ и $P_2$ в воде вблизи крышки и дна сосуда.

\newpage

\adddate{8 класс. 20 апреля 2018}

\task 1
Между точками $A$ и $B$ электрической цепи подключены последовательно резисторы $R_1 = 10\units{Ом}$ и $R_2 = 20\units{Ом}$ и параллельно им $R_3 = 30\units{Ом}.$ Найдите эквивалентное сопротивление $R_{AB}$ этого участка цепи.

\task 2
Электрическая цепь состоит из последовательности $N$ одинаковых звеньев, в которых каждый резистор имеет сопротивление $r$. Последнее звено замкнуто резистором сопротивлением $R$. При каком соотношении $\cfrac{R}{r}$ сопротивление цепи не зависит от числа звеньев?

\task 3
Для измерения сопротивления $R$ проводника собрана электрическая цепь. Вольтметр $V$ показывает напряжение $U_V = 5\units{В},$ показание амперметра $A$ равно $I_A = 25\units{мА}.$ Найдите величину $R$ сопротивления проводника. Внутреннее сопротивление вольтметра $R_V = 1{,}0\units{кОм},$ внутреннее сопротивление амперметра $R_A = 2{,}0\units{Ом}.$

\task 4
Шкала гальванометра имеет $N=100$ делений, цена деления $\delta = 1\units{мкА}$. Внутреннее сопротивление гальванометра $R_G = 1{,}0\units{кОм}.$ Как из этого прибора сделать вольтметр для измерения напряжений до $U = 100\units{В}$ или амперметр для измерения токов силой до $I = 1\units{А}?$

\\ \\ \\ \\ \\ \\ \\ \\
\adddate{8 класс. 20 апреля 2018}

\task 1
Между точками $A$ и $B$ электрической цепи подключены последовательно резисторы $R_1 = 10\units{Ом}$ и $R_2 = 20\units{Ом}$ и параллельно им $R_3 = 30\units{Ом}.$ Найдите эквивалентное сопротивление $R_{AB}$ этого участка цепи.

\task 2
Электрическая цепь состоит из последовательности $N$ одинаковых звеньев, в которых каждый резистор имеет сопротивление $r$. Последнее звено замкнуто резистором сопротивлением $R$. При каком соотношении $\cfrac{R}{r}$ сопротивление цепи не зависит от числа звеньев?

\task 3
Для измерения сопротивления $R$ проводника собрана электрическая цепь. Вольтметр $V$ показывает напряжение $U_V = 5\units{В},$ показание амперметра $A$ равно $I_A = 25\units{мА}.$ Найдите величину $R$ сопротивления проводника. Внутреннее сопротивление вольтметра $R_V = 1{,}0\units{кОм},$ внутреннее сопротивление амперметра $R_A = 2{,}0\units{Ом}.$

\task 4
Шкала гальванометра имеет $N=100$ делений, цена деления $\delta = 1\units{мкА}$. Внутреннее сопротивление гальванометра $R_G = 1{,}0\units{кОм}.$ Как из этого прибора сделать вольтметр для измерения напряжений до $U = 100\units{В}$ или амперметр для измерения токов силой до $I = 1\units{А}?$


% \begin{flushright}
\textsc{ГБОУ школа №554, 20 ноября 2018\,г.}
\end{flushright}

\begin{center}
\LARGE \textsc{Математический бой, 8 класс}
\end{center}

\problem{1} Есть тридцать карточек, на каждой написано по одному числу: на десяти карточках~–~$a$,  на десяти других~–~$b$ и на десяти оставшихся~–~$c$ (числа  различны). Известно, что к любым пяти карточкам можно подобрать ещё пять так, что сумма чисел на этих десяти карточках будет равна нулю. Докажите, что~одно из~чисел~$a, b, c$ равно нулю.

\problem{2} Вокруг стола стола пустили пакет с орешками. Первый взял один орешек, второй — 2, третий — 3 и так далее: каждый следующий брал на 1 орешек больше. Известно, что на втором круге было взято в сумме на 100 орешков больше, чем на первом. Сколько человек сидело за столом?

% \problem{2} Натуральное число разрешено увеличить на любое целое число процентов от 1 до 100, если при этом получаем натуральное число. Найдите наименьшее натуральное число, которое нельзя при помощи таких операций получить из~числа 1.

% \problem{3} Найти сумму $1^2 - 2^2 + 3^2 - 4^2 + 5^2 + \ldots - 2018^2$.

\problem{3} В кружке рукоделия, где занимается Валя, более 93\% участников~—~девочки. Какое наименьшее число участников может быть в таком кружке?

\problem{4} Произведение 2018 целых чисел равно 1. Может ли их сумма оказаться равной~0?

% \problem{4} Можно ли все натуральные числа от~1 до~9 записать в~клетки таблицы~$3\times3$ так, чтобы сумма в~любых двух соседних (по~вертикали или горизонтали) клетках равнялось простому числу?

\problem{5} На доске написано 2018 нулей и 2019 единиц. Женя стирает 2 числа и, если они были одинаковы, дописывает к оставшимся один ноль, а~если разные — единицу. Потом Женя повторяет эту операцию снова, потом ещё и~так далее. В~результате на~доске останется только одно число. Что это за~число?

\problem{6} Докажите, что в~любой компании людей найдутся 2~человека, имеющие равное число знакомых в этой компании (если $A$~знаком с~$B$, то~и $B$~знаком с~$A$).

\problem{7} Три колокола начинают бить одновременно. Интервалы между ударами колоколов соответственно составляют $\cfrac43$~секунды, $\cfrac53$~секунды и $2$~секунды. Совпавшие по времени удары воспринимаются за~один. Сколько ударов будет услышано за 1~минуту, включая первый и последний удары?

\problem{8} Восемь одинаковых момент расположены по кругу. Известно, что три из~них~— фальшивые, и они расположены рядом друг с~другом. Вес фальшивой монеты отличается от~веса настоящей. Все фальшивые монеты весят одинаково, но неизвестно, тяжелее или легче фальшивая монета настоящей. Покажите, что за~3~взвешивания на~чашечных весах без~гирь можно определить все фальшивые монеты.

\end{document}

\begin{document}
\noanswers

\setdate{6~мая~2019}
\setclass{10}

\addpersonalvariant{Михаил Бурмистров}


\tasknumber{1}\task{
    Определите ёмкость конденсатора, если при его зарядке до напряжения
    $U = 3\,\text{кВ}$ он приобретает заряд $Q = 24\,\text{нКл}$.
    Чему при этом равны заряды обкладок конденсатора (сделайте рисунок)?
}
\answer{
    $Q = CU \implies
    C = \frac{ Q }{ U } = \frac{ 24\,\text{нКл} }{ 3\,\text{кВ} } = 8{,}0\,\text{пФ}.
    \text{ Заряды обкладок: $Q$ и $-Q$}$
}
\vspace{120pt}

\tasknumber{2}\task{
    На конденсаторе указано: $C = 120\,\text{пФ}$, $U = 400\,\text{кВ}$.
    Удастся ли его использовать для накопления заряда $q = 60\,\text{нКл}$?
}
\answer{
    $q' = CU = 120\,\text{пФ} \cdot 400\,\text{кВ} = 48000\,\text{нКл}
    \implies q' \ge q \implies \text{удастся}$
}
\vspace{120pt}

\tasknumber{3}\task{
    Как и во сколько раз изменится ёмкость плоского конденсатора при уменьшении площади пластин в 5 раз
    и уменьшении расстояния между ними в 4 раз?
}
\answer{
    $\frac{C'}{C}
        = \frac{\eps_0\eps \frac S5}{\frac d4} \Big/ \frac{\eps_0\eps S}{d}
        = \frac{ 4 }{ 5 } < 1 \implies \text{уменьшится в $\frac54$ раз}
    $
}
\vspace{120pt}

\tasknumber{4}\task{
    Электрическая ёмкость конденсатора равна $С = 200\,\text{пФ}$,
    при этом ему сообщён заряд $Q = 800\,\text{нКл}$.
    Какова энергия заряженного конденсатора?
}
\answer{
    $W
        = \frac{ Q^2 }{ 2С }
        = \frac{ \sqr{ 800\,\text{нКл} } }{ 2 \cdot 200\,\text{пФ} }
        = 1600{,}0\,\text{мкДж}
    $
}
\vspace{120pt}

\tasknumber{5}\task{
    Два конденсатора ёмкостей $С_1 = 30\,\text{нФ}$ и $С_2 = 40\,\text{нФ}$ последовательно подключают
    к источнику напряжения $V = 300\,\text{В}$ (см.
    рис.).
    Определите заряды каждого из конденсаторов.
}
\answer{
    $Q_1
        = Q_2
        = CV
        = \frac{V}{\frac1{C_1} + \frac1{C_2}}
        = \frac{C_1C_2V}{C_1 + C_2}
        = \frac{
            30\,\text{нФ} \cdot 40\,\text{нФ} \cdot 300\,\text{В}
        }{
            30\,\text{нФ} + 40\,\text{нФ}
        }
        = 5142{,}86\,\text{нКл}
    $
}
\newpage

\addpersonalvariant{Гагик Аракелян}


\tasknumber{1}\task{
    Определите ёмкость конденсатора, если при его зарядке до напряжения
    $V = 15\,\text{кВ}$ он приобретает заряд $q = 18\,\text{нКл}$.
    Чему при этом равны заряды обкладок конденсатора (сделайте рисунок)?
}
\answer{
    $q = CV \implies
    C = \frac{ q }{ V } = \frac{ 18\,\text{нКл} }{ 15\,\text{кВ} } = 1{,}2\,\text{пФ}.
    \text{ Заряды обкладок: $q$ и $-q$}$
}
\vspace{120pt}

\tasknumber{2}\task{
    На конденсаторе указано: $C = 80\,\text{пФ}$, $V = 400\,\text{кВ}$.
    Удастся ли его использовать для накопления заряда $q = 50\,\text{нКл}$?
}
\answer{
    $q' = CV = 80\,\text{пФ} \cdot 400\,\text{кВ} = 32000\,\text{нКл}
    \implies q' \ge q \implies \text{удастся}$
}
\vspace{120pt}

\tasknumber{3}\task{
    Как и во сколько раз изменится ёмкость плоского конденсатора при уменьшении площади пластин в 2 раз
    и уменьшении расстояния между ними в 8 раз?
}
\answer{
    $\frac{C'}{C}
        = \frac{\eps_0\eps \frac S2}{\frac d8} \Big/ \frac{\eps_0\eps S}{d}
        = \frac{ 8 }{ 2 } > 1 \implies \text{увеличится в $\frac41$ раз}
    $
}
\vspace{120pt}

\tasknumber{4}\task{
    Электрическая ёмкость конденсатора равна $С = 400\,\text{пФ}$,
    при этом ему сообщён заряд $q = 800\,\text{нКл}$.
    Какова энергия заряженного конденсатора?
}
\answer{
    $W
        = \frac{ q^2 }{ 2С }
        = \frac{ \sqr{ 800\,\text{нКл} } }{ 2 \cdot 400\,\text{пФ} }
        = 800{,}0\,\text{мкДж}
    $
}
\vspace{120pt}

\tasknumber{5}\task{
    Два конденсатора ёмкостей $С_1 = 60\,\text{нФ}$ и $С_2 = 20\,\text{нФ}$ последовательно подключают
    к источнику напряжения $V = 200\,\text{В}$ (см.
    рис.).
    Определите заряды каждого из конденсаторов.
}
\answer{
    $Q_1
        = Q_2
        = CV
        = \frac{V}{\frac1{C_1} + \frac1{C_2}}
        = \frac{C_1C_2V}{C_1 + C_2}
        = \frac{
            60\,\text{нФ} \cdot 20\,\text{нФ} \cdot 200\,\text{В}
        }{
            60\,\text{нФ} + 20\,\text{нФ}
        }
        = 3000{,}0\,\text{нКл}
    $
}
\newpage

\addpersonalvariant{Ирен Аракелян}


\tasknumber{1}\task{
    Определите ёмкость конденсатора, если при его зарядке до напряжения
    $V = 12\,\text{кВ}$ он приобретает заряд $q = 18\,\text{нКл}$.
    Чему при этом равны заряды обкладок конденсатора (сделайте рисунок)?
}
\answer{
    $q = CV \implies
    C = \frac{ q }{ V } = \frac{ 18\,\text{нКл} }{ 12\,\text{кВ} } = 1{,}5\,\text{пФ}.
    \text{ Заряды обкладок: $q$ и $-q$}$
}
\vspace{120pt}

\tasknumber{2}\task{
    На конденсаторе указано: $C = 50\,\text{пФ}$, $V = 300\,\text{кВ}$.
    Удастся ли его использовать для накопления заряда $q = 30\,\text{нКл}$?
}
\answer{
    $q' = CV = 50\,\text{пФ} \cdot 300\,\text{кВ} = 15000\,\text{нКл}
    \implies q' \ge q \implies \text{удастся}$
}
\vspace{120pt}

\tasknumber{3}\task{
    Как и во сколько раз изменится ёмкость плоского конденсатора при уменьшении площади пластин в 3 раз
    и уменьшении расстояния между ними в 3 раз?
}
\answer{
    $\frac{C'}{C}
        = \frac{\eps_0\eps \frac S3}{\frac d3} \Big/ \frac{\eps_0\eps S}{d}
        = \frac{ 3 }{ 3 } = 1 \implies \text{не изменится}
    $
}
\vspace{120pt}

\tasknumber{4}\task{
    Электрическая ёмкость конденсатора равна $С = 400\,\text{пФ}$,
    при этом ему сообщён заряд $Q = 500\,\text{нКл}$.
    Какова энергия заряженного конденсатора?
}
\answer{
    $W
        = \frac{ Q^2 }{ 2С }
        = \frac{ \sqr{ 500\,\text{нКл} } }{ 2 \cdot 400\,\text{пФ} }
        = 312{,}5\,\text{мкДж}
    $
}
\vspace{120pt}

\tasknumber{5}\task{
    Два конденсатора ёмкостей $С_1 = 60\,\text{нФ}$ и $С_2 = 20\,\text{нФ}$ последовательно подключают
    к источнику напряжения $U = 400\,\text{В}$ (см.
    рис.).
    Определите заряды каждого из конденсаторов.
}
\answer{
    $Q_1
        = Q_2
        = CU
        = \frac{U}{\frac1{C_1} + \frac1{C_2}}
        = \frac{C_1C_2U}{C_1 + C_2}
        = \frac{
            60\,\text{нФ} \cdot 20\,\text{нФ} \cdot 400\,\text{В}
        }{
            60\,\text{нФ} + 20\,\text{нФ}
        }
        = 6000{,}0\,\text{нКл}
    $
}
\newpage

\addpersonalvariant{Сабина Асадуллаева}


\tasknumber{1}\task{
    Определите ёмкость конденсатора, если при его зарядке до напряжения
    $V = 6\,\text{кВ}$ он приобретает заряд $Q = 18\,\text{нКл}$.
    Чему при этом равны заряды обкладок конденсатора (сделайте рисунок)?
}
\answer{
    $Q = CV \implies
    C = \frac{ Q }{ V } = \frac{ 18\,\text{нКл} }{ 6\,\text{кВ} } = 3{,}0\,\text{пФ}.
    \text{ Заряды обкладок: $Q$ и $-Q$}$
}
\vspace{120pt}

\tasknumber{2}\task{
    На конденсаторе указано: $C = 100\,\text{пФ}$, $V = 200\,\text{кВ}$.
    Удастся ли его использовать для накопления заряда $q = 60\,\text{нКл}$?
}
\answer{
    $q' = CV = 100\,\text{пФ} \cdot 200\,\text{кВ} = 20000\,\text{нКл}
    \implies q' \ge q \implies \text{удастся}$
}
\vspace{120pt}

\tasknumber{3}\task{
    Как и во сколько раз изменится ёмкость плоского конденсатора при уменьшении площади пластин в 6 раз
    и уменьшении расстояния между ними в 6 раз?
}
\answer{
    $\frac{C'}{C}
        = \frac{\eps_0\eps \frac S6}{\frac d6} \Big/ \frac{\eps_0\eps S}{d}
        = \frac{ 6 }{ 6 } = 1 \implies \text{не изменится}
    $
}
\vspace{120pt}

\tasknumber{4}\task{
    Электрическая ёмкость конденсатора равна $С = 750\,\text{пФ}$,
    при этом ему сообщён заряд $q = 300\,\text{нКл}$.
    Какова энергия заряженного конденсатора?
}
\answer{
    $W
        = \frac{ q^2 }{ 2С }
        = \frac{ \sqr{ 300\,\text{нКл} } }{ 2 \cdot 750\,\text{пФ} }
        = 60{,}0\,\text{мкДж}
    $
}
\vspace{120pt}

\tasknumber{5}\task{
    Два конденсатора ёмкостей $С_1 = 30\,\text{нФ}$ и $С_2 = 20\,\text{нФ}$ последовательно подключают
    к источнику напряжения $U = 300\,\text{В}$ (см.
    рис.).
    Определите заряды каждого из конденсаторов.
}
\answer{
    $Q_1
        = Q_2
        = CU
        = \frac{U}{\frac1{C_1} + \frac1{C_2}}
        = \frac{C_1C_2U}{C_1 + C_2}
        = \frac{
            30\,\text{нФ} \cdot 20\,\text{нФ} \cdot 300\,\text{В}
        }{
            30\,\text{нФ} + 20\,\text{нФ}
        }
        = 3600{,}0\,\text{нКл}
    $
}
\newpage

\addpersonalvariant{Вероника Битерякова}


\tasknumber{1}\task{
    Определите ёмкость конденсатора, если при его зарядке до напряжения
    $U = 5\,\text{кВ}$ он приобретает заряд $q = 15\,\text{нКл}$.
    Чему при этом равны заряды обкладок конденсатора (сделайте рисунок)?
}
\answer{
    $q = CU \implies
    C = \frac{ q }{ U } = \frac{ 15\,\text{нКл} }{ 5\,\text{кВ} } = 3{,}0\,\text{пФ}.
    \text{ Заряды обкладок: $q$ и $-q$}$
}
\vspace{120pt}

\tasknumber{2}\task{
    На конденсаторе указано: $C = 150\,\text{пФ}$, $V = 450\,\text{кВ}$.
    Удастся ли его использовать для накопления заряда $Q = 60\,\text{нКл}$?
}
\answer{
    $Q' = CV = 150\,\text{пФ} \cdot 450\,\text{кВ} = 67500\,\text{нКл}
    \implies Q' \ge Q \implies \text{удастся}$
}
\vspace{120pt}

\tasknumber{3}\task{
    Как и во сколько раз изменится ёмкость плоского конденсатора при уменьшении площади пластин в 7 раз
    и уменьшении расстояния между ними в 6 раз?
}
\answer{
    $\frac{C'}{C}
        = \frac{\eps_0\eps \frac S7}{\frac d6} \Big/ \frac{\eps_0\eps S}{d}
        = \frac{ 6 }{ 7 } < 1 \implies \text{уменьшится в $\frac76$ раз}
    $
}
\vspace{120pt}

\tasknumber{4}\task{
    Электрическая ёмкость конденсатора равна $С = 200\,\text{пФ}$,
    при этом ему сообщён заряд $q = 300\,\text{нКл}$.
    Какова энергия заряженного конденсатора?
}
\answer{
    $W
        = \frac{ q^2 }{ 2С }
        = \frac{ \sqr{ 300\,\text{нКл} } }{ 2 \cdot 200\,\text{пФ} }
        = 225{,}0\,\text{мкДж}
    $
}
\vspace{120pt}

\tasknumber{5}\task{
    Два конденсатора ёмкостей $С_1 = 40\,\text{нФ}$ и $С_2 = 60\,\text{нФ}$ последовательно подключают
    к источнику напряжения $V = 400\,\text{В}$ (см.
    рис.).
    Определите заряды каждого из конденсаторов.
}
\answer{
    $Q_1
        = Q_2
        = CV
        = \frac{V}{\frac1{C_1} + \frac1{C_2}}
        = \frac{C_1C_2V}{C_1 + C_2}
        = \frac{
            40\,\text{нФ} \cdot 60\,\text{нФ} \cdot 400\,\text{В}
        }{
            40\,\text{нФ} + 60\,\text{нФ}
        }
        = 9600{,}0\,\text{нКл}
    $
}
\newpage

\addpersonalvariant{Юлия Буянова}


\tasknumber{1}\task{
    Определите ёмкость конденсатора, если при его зарядке до напряжения
    $V = 2\,\text{кВ}$ он приобретает заряд $Q = 18\,\text{нКл}$.
    Чему при этом равны заряды обкладок конденсатора (сделайте рисунок)?
}
\answer{
    $Q = CV \implies
    C = \frac{ Q }{ V } = \frac{ 18\,\text{нКл} }{ 2\,\text{кВ} } = 9{,}0\,\text{пФ}.
    \text{ Заряды обкладок: $Q$ и $-Q$}$
}
\vspace{120pt}

\tasknumber{2}\task{
    На конденсаторе указано: $C = 150\,\text{пФ}$, $V = 300\,\text{кВ}$.
    Удастся ли его использовать для накопления заряда $Q = 50\,\text{нКл}$?
}
\answer{
    $Q' = CV = 150\,\text{пФ} \cdot 300\,\text{кВ} = 45000\,\text{нКл}
    \implies Q' \ge Q \implies \text{удастся}$
}
\vspace{120pt}

\tasknumber{3}\task{
    Как и во сколько раз изменится ёмкость плоского конденсатора при уменьшении площади пластин в 7 раз
    и уменьшении расстояния между ними в 5 раз?
}
\answer{
    $\frac{C'}{C}
        = \frac{\eps_0\eps \frac S7}{\frac d5} \Big/ \frac{\eps_0\eps S}{d}
        = \frac{ 5 }{ 7 } < 1 \implies \text{уменьшится в $\frac75$ раз}
    $
}
\vspace{120pt}

\tasknumber{4}\task{
    Электрическая ёмкость конденсатора равна $С = 750\,\text{пФ}$,
    при этом ему сообщён заряд $q = 900\,\text{нКл}$.
    Какова энергия заряженного конденсатора?
}
\answer{
    $W
        = \frac{ q^2 }{ 2С }
        = \frac{ \sqr{ 900\,\text{нКл} } }{ 2 \cdot 750\,\text{пФ} }
        = 540{,}0\,\text{мкДж}
    $
}
\vspace{120pt}

\tasknumber{5}\task{
    Два конденсатора ёмкостей $С_1 = 30\,\text{нФ}$ и $С_2 = 20\,\text{нФ}$ последовательно подключают
    к источнику напряжения $U = 400\,\text{В}$ (см.
    рис.).
    Определите заряды каждого из конденсаторов.
}
\answer{
    $Q_1
        = Q_2
        = CU
        = \frac{U}{\frac1{C_1} + \frac1{C_2}}
        = \frac{C_1C_2U}{C_1 + C_2}
        = \frac{
            30\,\text{нФ} \cdot 20\,\text{нФ} \cdot 400\,\text{В}
        }{
            30\,\text{нФ} + 20\,\text{нФ}
        }
        = 4800{,}0\,\text{нКл}
    $
}
\newpage

\addpersonalvariant{Пелагея Вдовина}


\tasknumber{1}\task{
    Определите ёмкость конденсатора, если при его зарядке до напряжения
    $V = 20\,\text{кВ}$ он приобретает заряд $q = 6\,\text{нКл}$.
    Чему при этом равны заряды обкладок конденсатора (сделайте рисунок)?
}
\answer{
    $q = CV \implies
    C = \frac{ q }{ V } = \frac{ 6\,\text{нКл} }{ 20\,\text{кВ} } = 0{,}3\,\text{пФ}.
    \text{ Заряды обкладок: $q$ и $-q$}$
}
\vspace{120pt}

\tasknumber{2}\task{
    На конденсаторе указано: $C = 150\,\text{пФ}$, $V = 200\,\text{кВ}$.
    Удастся ли его использовать для накопления заряда $q = 30\,\text{нКл}$?
}
\answer{
    $q' = CV = 150\,\text{пФ} \cdot 200\,\text{кВ} = 30000\,\text{нКл}
    \implies q' \ge q \implies \text{удастся}$
}
\vspace{120pt}

\tasknumber{3}\task{
    Как и во сколько раз изменится ёмкость плоского конденсатора при уменьшении площади пластин в 8 раз
    и уменьшении расстояния между ними в 4 раз?
}
\answer{
    $\frac{C'}{C}
        = \frac{\eps_0\eps \frac S8}{\frac d4} \Big/ \frac{\eps_0\eps S}{d}
        = \frac{ 4 }{ 8 } < 1 \implies \text{уменьшится в $\frac21$ раз}
    $
}
\vspace{120pt}

\tasknumber{4}\task{
    Электрическая ёмкость конденсатора равна $С = 750\,\text{пФ}$,
    при этом ему сообщён заряд $Q = 800\,\text{нКл}$.
    Какова энергия заряженного конденсатора?
}
\answer{
    $W
        = \frac{ Q^2 }{ 2С }
        = \frac{ \sqr{ 800\,\text{нКл} } }{ 2 \cdot 750\,\text{пФ} }
        = 426{,}67\,\text{мкДж}
    $
}
\vspace{120pt}

\tasknumber{5}\task{
    Два конденсатора ёмкостей $С_1 = 40\,\text{нФ}$ и $С_2 = 30\,\text{нФ}$ последовательно подключают
    к источнику напряжения $U = 200\,\text{В}$ (см.
    рис.).
    Определите заряды каждого из конденсаторов.
}
\answer{
    $Q_1
        = Q_2
        = CU
        = \frac{U}{\frac1{C_1} + \frac1{C_2}}
        = \frac{C_1C_2U}{C_1 + C_2}
        = \frac{
            40\,\text{нФ} \cdot 30\,\text{нФ} \cdot 200\,\text{В}
        }{
            40\,\text{нФ} + 30\,\text{нФ}
        }
        = 3428{,}57\,\text{нКл}
    $
}
\newpage

\addpersonalvariant{Леонид Викторов}


\tasknumber{1}\task{
    Определите ёмкость конденсатора, если при его зарядке до напряжения
    $U = 2\,\text{кВ}$ он приобретает заряд $Q = 6\,\text{нКл}$.
    Чему при этом равны заряды обкладок конденсатора (сделайте рисунок)?
}
\answer{
    $Q = CU \implies
    C = \frac{ Q }{ U } = \frac{ 6\,\text{нКл} }{ 2\,\text{кВ} } = 3{,}0\,\text{пФ}.
    \text{ Заряды обкладок: $Q$ и $-Q$}$
}
\vspace{120pt}

\tasknumber{2}\task{
    На конденсаторе указано: $C = 100\,\text{пФ}$, $V = 450\,\text{кВ}$.
    Удастся ли его использовать для накопления заряда $Q = 50\,\text{нКл}$?
}
\answer{
    $Q' = CV = 100\,\text{пФ} \cdot 450\,\text{кВ} = 45000\,\text{нКл}
    \implies Q' \ge Q \implies \text{удастся}$
}
\vspace{120pt}

\tasknumber{3}\task{
    Как и во сколько раз изменится ёмкость плоского конденсатора при уменьшении площади пластин в 7 раз
    и уменьшении расстояния между ними в 3 раз?
}
\answer{
    $\frac{C'}{C}
        = \frac{\eps_0\eps \frac S7}{\frac d3} \Big/ \frac{\eps_0\eps S}{d}
        = \frac{ 3 }{ 7 } < 1 \implies \text{уменьшится в $\frac73$ раз}
    $
}
\vspace{120pt}

\tasknumber{4}\task{
    Электрическая ёмкость конденсатора равна $С = 400\,\text{пФ}$,
    при этом ему сообщён заряд $Q = 800\,\text{нКл}$.
    Какова энергия заряженного конденсатора?
}
\answer{
    $W
        = \frac{ Q^2 }{ 2С }
        = \frac{ \sqr{ 800\,\text{нКл} } }{ 2 \cdot 400\,\text{пФ} }
        = 800{,}0\,\text{мкДж}
    $
}
\vspace{120pt}

\tasknumber{5}\task{
    Два конденсатора ёмкостей $С_1 = 20\,\text{нФ}$ и $С_2 = 40\,\text{нФ}$ последовательно подключают
    к источнику напряжения $V = 400\,\text{В}$ (см.
    рис.).
    Определите заряды каждого из конденсаторов.
}
\answer{
    $Q_1
        = Q_2
        = CV
        = \frac{V}{\frac1{C_1} + \frac1{C_2}}
        = \frac{C_1C_2V}{C_1 + C_2}
        = \frac{
            20\,\text{нФ} \cdot 40\,\text{нФ} \cdot 400\,\text{В}
        }{
            20\,\text{нФ} + 40\,\text{нФ}
        }
        = 5333{,}33\,\text{нКл}
    $
}
\newpage

\addpersonalvariant{Фёдор Гнутов}


\tasknumber{1}\task{
    Определите ёмкость конденсатора, если при его зарядке до напряжения
    $U = 15\,\text{кВ}$ он приобретает заряд $q = 18\,\text{нКл}$.
    Чему при этом равны заряды обкладок конденсатора (сделайте рисунок)?
}
\answer{
    $q = CU \implies
    C = \frac{ q }{ U } = \frac{ 18\,\text{нКл} }{ 15\,\text{кВ} } = 1{,}2\,\text{пФ}.
    \text{ Заряды обкладок: $q$ и $-q$}$
}
\vspace{120pt}

\tasknumber{2}\task{
    На конденсаторе указано: $C = 120\,\text{пФ}$, $U = 450\,\text{кВ}$.
    Удастся ли его использовать для накопления заряда $Q = 30\,\text{нКл}$?
}
\answer{
    $Q' = CU = 120\,\text{пФ} \cdot 450\,\text{кВ} = 54000\,\text{нКл}
    \implies Q' \ge Q \implies \text{удастся}$
}
\vspace{120pt}

\tasknumber{3}\task{
    Как и во сколько раз изменится ёмкость плоского конденсатора при уменьшении площади пластин в 3 раз
    и уменьшении расстояния между ними в 8 раз?
}
\answer{
    $\frac{C'}{C}
        = \frac{\eps_0\eps \frac S3}{\frac d8} \Big/ \frac{\eps_0\eps S}{d}
        = \frac{ 8 }{ 3 } > 1 \implies \text{увеличится в $\frac83$ раз}
    $
}
\vspace{120pt}

\tasknumber{4}\task{
    Электрическая ёмкость конденсатора равна $С = 400\,\text{пФ}$,
    при этом ему сообщён заряд $Q = 900\,\text{нКл}$.
    Какова энергия заряженного конденсатора?
}
\answer{
    $W
        = \frac{ Q^2 }{ 2С }
        = \frac{ \sqr{ 900\,\text{нКл} } }{ 2 \cdot 400\,\text{пФ} }
        = 1012{,}5\,\text{мкДж}
    $
}
\vspace{120pt}

\tasknumber{5}\task{
    Два конденсатора ёмкостей $С_1 = 60\,\text{нФ}$ и $С_2 = 20\,\text{нФ}$ последовательно подключают
    к источнику напряжения $V = 300\,\text{В}$ (см.
    рис.).
    Определите заряды каждого из конденсаторов.
}
\answer{
    $Q_1
        = Q_2
        = CV
        = \frac{V}{\frac1{C_1} + \frac1{C_2}}
        = \frac{C_1C_2V}{C_1 + C_2}
        = \frac{
            60\,\text{нФ} \cdot 20\,\text{нФ} \cdot 300\,\text{В}
        }{
            60\,\text{нФ} + 20\,\text{нФ}
        }
        = 4500{,}0\,\text{нКл}
    $
}
\newpage

\addpersonalvariant{Илья Гримберг}


\tasknumber{1}\task{
    Определите ёмкость конденсатора, если при его зарядке до напряжения
    $U = 15\,\text{кВ}$ он приобретает заряд $q = 6\,\text{нКл}$.
    Чему при этом равны заряды обкладок конденсатора (сделайте рисунок)?
}
\answer{
    $q = CU \implies
    C = \frac{ q }{ U } = \frac{ 6\,\text{нКл} }{ 15\,\text{кВ} } = 0{,}4\,\text{пФ}.
    \text{ Заряды обкладок: $q$ и $-q$}$
}
\vspace{120pt}

\tasknumber{2}\task{
    На конденсаторе указано: $C = 50\,\text{пФ}$, $V = 450\,\text{кВ}$.
    Удастся ли его использовать для накопления заряда $q = 60\,\text{нКл}$?
}
\answer{
    $q' = CV = 50\,\text{пФ} \cdot 450\,\text{кВ} = 22500\,\text{нКл}
    \implies q' \ge q \implies \text{удастся}$
}
\vspace{120pt}

\tasknumber{3}\task{
    Как и во сколько раз изменится ёмкость плоского конденсатора при уменьшении площади пластин в 7 раз
    и уменьшении расстояния между ними в 6 раз?
}
\answer{
    $\frac{C'}{C}
        = \frac{\eps_0\eps \frac S7}{\frac d6} \Big/ \frac{\eps_0\eps S}{d}
        = \frac{ 6 }{ 7 } < 1 \implies \text{уменьшится в $\frac76$ раз}
    $
}
\vspace{120pt}

\tasknumber{4}\task{
    Электрическая ёмкость конденсатора равна $С = 750\,\text{пФ}$,
    при этом ему сообщён заряд $Q = 800\,\text{нКл}$.
    Какова энергия заряженного конденсатора?
}
\answer{
    $W
        = \frac{ Q^2 }{ 2С }
        = \frac{ \sqr{ 800\,\text{нКл} } }{ 2 \cdot 750\,\text{пФ} }
        = 426{,}67\,\text{мкДж}
    $
}
\vspace{120pt}

\tasknumber{5}\task{
    Два конденсатора ёмкостей $С_1 = 40\,\text{нФ}$ и $С_2 = 30\,\text{нФ}$ последовательно подключают
    к источнику напряжения $V = 450\,\text{В}$ (см.
    рис.).
    Определите заряды каждого из конденсаторов.
}
\answer{
    $Q_1
        = Q_2
        = CV
        = \frac{V}{\frac1{C_1} + \frac1{C_2}}
        = \frac{C_1C_2V}{C_1 + C_2}
        = \frac{
            40\,\text{нФ} \cdot 30\,\text{нФ} \cdot 450\,\text{В}
        }{
            40\,\text{нФ} + 30\,\text{нФ}
        }
        = 7714{,}29\,\text{нКл}
    $
}
\newpage

\addpersonalvariant{Иван Гурьянов}


\tasknumber{1}\task{
    Определите ёмкость конденсатора, если при его зарядке до напряжения
    $U = 6\,\text{кВ}$ он приобретает заряд $q = 15\,\text{нКл}$.
    Чему при этом равны заряды обкладок конденсатора (сделайте рисунок)?
}
\answer{
    $q = CU \implies
    C = \frac{ q }{ U } = \frac{ 15\,\text{нКл} }{ 6\,\text{кВ} } = 2{,}5\,\text{пФ}.
    \text{ Заряды обкладок: $q$ и $-q$}$
}
\vspace{120pt}

\tasknumber{2}\task{
    На конденсаторе указано: $C = 100\,\text{пФ}$, $V = 450\,\text{кВ}$.
    Удастся ли его использовать для накопления заряда $Q = 30\,\text{нКл}$?
}
\answer{
    $Q' = CV = 100\,\text{пФ} \cdot 450\,\text{кВ} = 45000\,\text{нКл}
    \implies Q' \ge Q \implies \text{удастся}$
}
\vspace{120pt}

\tasknumber{3}\task{
    Как и во сколько раз изменится ёмкость плоского конденсатора при уменьшении площади пластин в 8 раз
    и уменьшении расстояния между ними в 4 раз?
}
\answer{
    $\frac{C'}{C}
        = \frac{\eps_0\eps \frac S8}{\frac d4} \Big/ \frac{\eps_0\eps S}{d}
        = \frac{ 4 }{ 8 } < 1 \implies \text{уменьшится в $\frac21$ раз}
    $
}
\vspace{120pt}

\tasknumber{4}\task{
    Электрическая ёмкость конденсатора равна $С = 200\,\text{пФ}$,
    при этом ему сообщён заряд $Q = 500\,\text{нКл}$.
    Какова энергия заряженного конденсатора?
}
\answer{
    $W
        = \frac{ Q^2 }{ 2С }
        = \frac{ \sqr{ 500\,\text{нКл} } }{ 2 \cdot 200\,\text{пФ} }
        = 625{,}0\,\text{мкДж}
    $
}
\vspace{120pt}

\tasknumber{5}\task{
    Два конденсатора ёмкостей $С_1 = 40\,\text{нФ}$ и $С_2 = 60\,\text{нФ}$ последовательно подключают
    к источнику напряжения $V = 300\,\text{В}$ (см.
    рис.).
    Определите заряды каждого из конденсаторов.
}
\answer{
    $Q_1
        = Q_2
        = CV
        = \frac{V}{\frac1{C_1} + \frac1{C_2}}
        = \frac{C_1C_2V}{C_1 + C_2}
        = \frac{
            40\,\text{нФ} \cdot 60\,\text{нФ} \cdot 300\,\text{В}
        }{
            40\,\text{нФ} + 60\,\text{нФ}
        }
        = 7200{,}0\,\text{нКл}
    $
}
\newpage

\addpersonalvariant{Артём Денежкин}


\tasknumber{1}\task{
    Определите ёмкость конденсатора, если при его зарядке до напряжения
    $U = 5\,\text{кВ}$ он приобретает заряд $q = 6\,\text{нКл}$.
    Чему при этом равны заряды обкладок конденсатора (сделайте рисунок)?
}
\answer{
    $q = CU \implies
    C = \frac{ q }{ U } = \frac{ 6\,\text{нКл} }{ 5\,\text{кВ} } = 1{,}2\,\text{пФ}.
    \text{ Заряды обкладок: $q$ и $-q$}$
}
\vspace{120pt}

\tasknumber{2}\task{
    На конденсаторе указано: $C = 120\,\text{пФ}$, $V = 450\,\text{кВ}$.
    Удастся ли его использовать для накопления заряда $Q = 60\,\text{нКл}$?
}
\answer{
    $Q' = CV = 120\,\text{пФ} \cdot 450\,\text{кВ} = 54000\,\text{нКл}
    \implies Q' \ge Q \implies \text{удастся}$
}
\vspace{120pt}

\tasknumber{3}\task{
    Как и во сколько раз изменится ёмкость плоского конденсатора при уменьшении площади пластин в 5 раз
    и уменьшении расстояния между ними в 5 раз?
}
\answer{
    $\frac{C'}{C}
        = \frac{\eps_0\eps \frac S5}{\frac d5} \Big/ \frac{\eps_0\eps S}{d}
        = \frac{ 5 }{ 5 } = 1 \implies \text{не изменится}
    $
}
\vspace{120pt}

\tasknumber{4}\task{
    Электрическая ёмкость конденсатора равна $С = 750\,\text{пФ}$,
    при этом ему сообщён заряд $q = 900\,\text{нКл}$.
    Какова энергия заряженного конденсатора?
}
\answer{
    $W
        = \frac{ q^2 }{ 2С }
        = \frac{ \sqr{ 900\,\text{нКл} } }{ 2 \cdot 750\,\text{пФ} }
        = 540{,}0\,\text{мкДж}
    $
}
\vspace{120pt}

\tasknumber{5}\task{
    Два конденсатора ёмкостей $С_1 = 40\,\text{нФ}$ и $С_2 = 30\,\text{нФ}$ последовательно подключают
    к источнику напряжения $V = 400\,\text{В}$ (см.
    рис.).
    Определите заряды каждого из конденсаторов.
}
\answer{
    $Q_1
        = Q_2
        = CV
        = \frac{V}{\frac1{C_1} + \frac1{C_2}}
        = \frac{C_1C_2V}{C_1 + C_2}
        = \frac{
            40\,\text{нФ} \cdot 30\,\text{нФ} \cdot 400\,\text{В}
        }{
            40\,\text{нФ} + 30\,\text{нФ}
        }
        = 6857{,}14\,\text{нКл}
    $
}
\newpage

\addpersonalvariant{Виктор Жилин}


\tasknumber{1}\task{
    Определите ёмкость конденсатора, если при его зарядке до напряжения
    $V = 15\,\text{кВ}$ он приобретает заряд $q = 24\,\text{нКл}$.
    Чему при этом равны заряды обкладок конденсатора (сделайте рисунок)?
}
\answer{
    $q = CV \implies
    C = \frac{ q }{ V } = \frac{ 24\,\text{нКл} }{ 15\,\text{кВ} } = 1{,}6\,\text{пФ}.
    \text{ Заряды обкладок: $q$ и $-q$}$
}
\vspace{120pt}

\tasknumber{2}\task{
    На конденсаторе указано: $C = 50\,\text{пФ}$, $V = 200\,\text{кВ}$.
    Удастся ли его использовать для накопления заряда $Q = 60\,\text{нКл}$?
}
\answer{
    $Q' = CV = 50\,\text{пФ} \cdot 200\,\text{кВ} = 10000\,\text{нКл}
    \implies Q' \ge Q \implies \text{удастся}$
}
\vspace{120pt}

\tasknumber{3}\task{
    Как и во сколько раз изменится ёмкость плоского конденсатора при уменьшении площади пластин в 8 раз
    и уменьшении расстояния между ними в 2 раз?
}
\answer{
    $\frac{C'}{C}
        = \frac{\eps_0\eps \frac S8}{\frac d2} \Big/ \frac{\eps_0\eps S}{d}
        = \frac{ 2 }{ 8 } < 1 \implies \text{уменьшится в $\frac41$ раз}
    $
}
\vspace{120pt}

\tasknumber{4}\task{
    Электрическая ёмкость конденсатора равна $С = 600\,\text{пФ}$,
    при этом ему сообщён заряд $Q = 300\,\text{нКл}$.
    Какова энергия заряженного конденсатора?
}
\answer{
    $W
        = \frac{ Q^2 }{ 2С }
        = \frac{ \sqr{ 300\,\text{нКл} } }{ 2 \cdot 600\,\text{пФ} }
        = 75{,}0\,\text{мкДж}
    $
}
\vspace{120pt}

\tasknumber{5}\task{
    Два конденсатора ёмкостей $С_1 = 60\,\text{нФ}$ и $С_2 = 30\,\text{нФ}$ последовательно подключают
    к источнику напряжения $U = 150\,\text{В}$ (см.
    рис.).
    Определите заряды каждого из конденсаторов.
}
\answer{
    $Q_1
        = Q_2
        = CU
        = \frac{U}{\frac1{C_1} + \frac1{C_2}}
        = \frac{C_1C_2U}{C_1 + C_2}
        = \frac{
            60\,\text{нФ} \cdot 30\,\text{нФ} \cdot 150\,\text{В}
        }{
            60\,\text{нФ} + 30\,\text{нФ}
        }
        = 3000{,}0\,\text{нКл}
    $
}
\newpage

\addpersonalvariant{Дмитрий Иванов}


\tasknumber{1}\task{
    Определите ёмкость конденсатора, если при его зарядке до напряжения
    $U = 6\,\text{кВ}$ он приобретает заряд $q = 6\,\text{нКл}$.
    Чему при этом равны заряды обкладок конденсатора (сделайте рисунок)?
}
\answer{
    $q = CU \implies
    C = \frac{ q }{ U } = \frac{ 6\,\text{нКл} }{ 6\,\text{кВ} } = 1{,}0\,\text{пФ}.
    \text{ Заряды обкладок: $q$ и $-q$}$
}
\vspace{120pt}

\tasknumber{2}\task{
    На конденсаторе указано: $C = 100\,\text{пФ}$, $V = 400\,\text{кВ}$.
    Удастся ли его использовать для накопления заряда $Q = 50\,\text{нКл}$?
}
\answer{
    $Q' = CV = 100\,\text{пФ} \cdot 400\,\text{кВ} = 40000\,\text{нКл}
    \implies Q' \ge Q \implies \text{удастся}$
}
\vspace{120pt}

\tasknumber{3}\task{
    Как и во сколько раз изменится ёмкость плоского конденсатора при уменьшении площади пластин в 2 раз
    и уменьшении расстояния между ними в 3 раз?
}
\answer{
    $\frac{C'}{C}
        = \frac{\eps_0\eps \frac S2}{\frac d3} \Big/ \frac{\eps_0\eps S}{d}
        = \frac{ 3 }{ 2 } > 1 \implies \text{увеличится в $\frac32$ раз}
    $
}
\vspace{120pt}

\tasknumber{4}\task{
    Электрическая ёмкость конденсатора равна $С = 750\,\text{пФ}$,
    при этом ему сообщён заряд $q = 800\,\text{нКл}$.
    Какова энергия заряженного конденсатора?
}
\answer{
    $W
        = \frac{ q^2 }{ 2С }
        = \frac{ \sqr{ 800\,\text{нКл} } }{ 2 \cdot 750\,\text{пФ} }
        = 426{,}67\,\text{мкДж}
    $
}
\vspace{120pt}

\tasknumber{5}\task{
    Два конденсатора ёмкостей $С_1 = 40\,\text{нФ}$ и $С_2 = 30\,\text{нФ}$ последовательно подключают
    к источнику напряжения $V = 150\,\text{В}$ (см.
    рис.).
    Определите заряды каждого из конденсаторов.
}
\answer{
    $Q_1
        = Q_2
        = CV
        = \frac{V}{\frac1{C_1} + \frac1{C_2}}
        = \frac{C_1C_2V}{C_1 + C_2}
        = \frac{
            40\,\text{нФ} \cdot 30\,\text{нФ} \cdot 150\,\text{В}
        }{
            40\,\text{нФ} + 30\,\text{нФ}
        }
        = 2571{,}43\,\text{нКл}
    $
}
\newpage

\addpersonalvariant{Олег Климов}


\tasknumber{1}\task{
    Определите ёмкость конденсатора, если при его зарядке до напряжения
    $V = 15\,\text{кВ}$ он приобретает заряд $Q = 18\,\text{нКл}$.
    Чему при этом равны заряды обкладок конденсатора (сделайте рисунок)?
}
\answer{
    $Q = CV \implies
    C = \frac{ Q }{ V } = \frac{ 18\,\text{нКл} }{ 15\,\text{кВ} } = 1{,}2\,\text{пФ}.
    \text{ Заряды обкладок: $Q$ и $-Q$}$
}
\vspace{120pt}

\tasknumber{2}\task{
    На конденсаторе указано: $C = 80\,\text{пФ}$, $V = 300\,\text{кВ}$.
    Удастся ли его использовать для накопления заряда $q = 50\,\text{нКл}$?
}
\answer{
    $q' = CV = 80\,\text{пФ} \cdot 300\,\text{кВ} = 24000\,\text{нКл}
    \implies q' \ge q \implies \text{удастся}$
}
\vspace{120pt}

\tasknumber{3}\task{
    Как и во сколько раз изменится ёмкость плоского конденсатора при уменьшении площади пластин в 4 раз
    и уменьшении расстояния между ними в 2 раз?
}
\answer{
    $\frac{C'}{C}
        = \frac{\eps_0\eps \frac S4}{\frac d2} \Big/ \frac{\eps_0\eps S}{d}
        = \frac{ 2 }{ 4 } < 1 \implies \text{уменьшится в $\frac21$ раз}
    $
}
\vspace{120pt}

\tasknumber{4}\task{
    Электрическая ёмкость конденсатора равна $С = 750\,\text{пФ}$,
    при этом ему сообщён заряд $q = 800\,\text{нКл}$.
    Какова энергия заряженного конденсатора?
}
\answer{
    $W
        = \frac{ q^2 }{ 2С }
        = \frac{ \sqr{ 800\,\text{нКл} } }{ 2 \cdot 750\,\text{пФ} }
        = 426{,}67\,\text{мкДж}
    $
}
\vspace{120pt}

\tasknumber{5}\task{
    Два конденсатора ёмкостей $С_1 = 30\,\text{нФ}$ и $С_2 = 20\,\text{нФ}$ последовательно подключают
    к источнику напряжения $U = 450\,\text{В}$ (см.
    рис.).
    Определите заряды каждого из конденсаторов.
}
\answer{
    $Q_1
        = Q_2
        = CU
        = \frac{U}{\frac1{C_1} + \frac1{C_2}}
        = \frac{C_1C_2U}{C_1 + C_2}
        = \frac{
            30\,\text{нФ} \cdot 20\,\text{нФ} \cdot 450\,\text{В}
        }{
            30\,\text{нФ} + 20\,\text{нФ}
        }
        = 5400{,}0\,\text{нКл}
    $
}
\newpage

\addpersonalvariant{Анна Ковалева}


\tasknumber{1}\task{
    Определите ёмкость конденсатора, если при его зарядке до напряжения
    $V = 15\,\text{кВ}$ он приобретает заряд $Q = 25\,\text{нКл}$.
    Чему при этом равны заряды обкладок конденсатора (сделайте рисунок)?
}
\answer{
    $Q = CV \implies
    C = \frac{ Q }{ V } = \frac{ 25\,\text{нКл} }{ 15\,\text{кВ} } = 1{,}67\,\text{пФ}.
    \text{ Заряды обкладок: $Q$ и $-Q$}$
}
\vspace{120pt}

\tasknumber{2}\task{
    На конденсаторе указано: $C = 50\,\text{пФ}$, $U = 200\,\text{кВ}$.
    Удастся ли его использовать для накопления заряда $Q = 50\,\text{нКл}$?
}
\answer{
    $Q' = CU = 50\,\text{пФ} \cdot 200\,\text{кВ} = 10000\,\text{нКл}
    \implies Q' \ge Q \implies \text{удастся}$
}
\vspace{120pt}

\tasknumber{3}\task{
    Как и во сколько раз изменится ёмкость плоского конденсатора при уменьшении площади пластин в 3 раз
    и уменьшении расстояния между ними в 4 раз?
}
\answer{
    $\frac{C'}{C}
        = \frac{\eps_0\eps \frac S3}{\frac d4} \Big/ \frac{\eps_0\eps S}{d}
        = \frac{ 4 }{ 3 } > 1 \implies \text{увеличится в $\frac43$ раз}
    $
}
\vspace{120pt}

\tasknumber{4}\task{
    Электрическая ёмкость конденсатора равна $С = 400\,\text{пФ}$,
    при этом ему сообщён заряд $Q = 500\,\text{нКл}$.
    Какова энергия заряженного конденсатора?
}
\answer{
    $W
        = \frac{ Q^2 }{ 2С }
        = \frac{ \sqr{ 500\,\text{нКл} } }{ 2 \cdot 400\,\text{пФ} }
        = 312{,}5\,\text{мкДж}
    $
}
\vspace{120pt}

\tasknumber{5}\task{
    Два конденсатора ёмкостей $С_1 = 30\,\text{нФ}$ и $С_2 = 60\,\text{нФ}$ последовательно подключают
    к источнику напряжения $U = 150\,\text{В}$ (см.
    рис.).
    Определите заряды каждого из конденсаторов.
}
\answer{
    $Q_1
        = Q_2
        = CU
        = \frac{U}{\frac1{C_1} + \frac1{C_2}}
        = \frac{C_1C_2U}{C_1 + C_2}
        = \frac{
            30\,\text{нФ} \cdot 60\,\text{нФ} \cdot 150\,\text{В}
        }{
            30\,\text{нФ} + 60\,\text{нФ}
        }
        = 3000{,}0\,\text{нКл}
    $
}
\newpage

\addpersonalvariant{Глеб Ковылин}


\tasknumber{1}\task{
    Определите ёмкость конденсатора, если при его зарядке до напряжения
    $V = 6\,\text{кВ}$ он приобретает заряд $q = 6\,\text{нКл}$.
    Чему при этом равны заряды обкладок конденсатора (сделайте рисунок)?
}
\answer{
    $q = CV \implies
    C = \frac{ q }{ V } = \frac{ 6\,\text{нКл} }{ 6\,\text{кВ} } = 1{,}0\,\text{пФ}.
    \text{ Заряды обкладок: $q$ и $-q$}$
}
\vspace{120pt}

\tasknumber{2}\task{
    На конденсаторе указано: $C = 80\,\text{пФ}$, $V = 300\,\text{кВ}$.
    Удастся ли его использовать для накопления заряда $Q = 30\,\text{нКл}$?
}
\answer{
    $Q' = CV = 80\,\text{пФ} \cdot 300\,\text{кВ} = 24000\,\text{нКл}
    \implies Q' \ge Q \implies \text{удастся}$
}
\vspace{120pt}

\tasknumber{3}\task{
    Как и во сколько раз изменится ёмкость плоского конденсатора при уменьшении площади пластин в 7 раз
    и уменьшении расстояния между ними в 3 раз?
}
\answer{
    $\frac{C'}{C}
        = \frac{\eps_0\eps \frac S7}{\frac d3} \Big/ \frac{\eps_0\eps S}{d}
        = \frac{ 3 }{ 7 } < 1 \implies \text{уменьшится в $\frac73$ раз}
    $
}
\vspace{120pt}

\tasknumber{4}\task{
    Электрическая ёмкость конденсатора равна $С = 750\,\text{пФ}$,
    при этом ему сообщён заряд $q = 500\,\text{нКл}$.
    Какова энергия заряженного конденсатора?
}
\answer{
    $W
        = \frac{ q^2 }{ 2С }
        = \frac{ \sqr{ 500\,\text{нКл} } }{ 2 \cdot 750\,\text{пФ} }
        = 166{,}67\,\text{мкДж}
    $
}
\vspace{120pt}

\tasknumber{5}\task{
    Два конденсатора ёмкостей $С_1 = 40\,\text{нФ}$ и $С_2 = 30\,\text{нФ}$ последовательно подключают
    к источнику напряжения $U = 300\,\text{В}$ (см.
    рис.).
    Определите заряды каждого из конденсаторов.
}
\answer{
    $Q_1
        = Q_2
        = CU
        = \frac{U}{\frac1{C_1} + \frac1{C_2}}
        = \frac{C_1C_2U}{C_1 + C_2}
        = \frac{
            40\,\text{нФ} \cdot 30\,\text{нФ} \cdot 300\,\text{В}
        }{
            40\,\text{нФ} + 30\,\text{нФ}
        }
        = 5142{,}86\,\text{нКл}
    $
}
\newpage

\addpersonalvariant{Даниил Космынин}


\tasknumber{1}\task{
    Определите ёмкость конденсатора, если при его зарядке до напряжения
    $U = 6\,\text{кВ}$ он приобретает заряд $Q = 25\,\text{нКл}$.
    Чему при этом равны заряды обкладок конденсатора (сделайте рисунок)?
}
\answer{
    $Q = CU \implies
    C = \frac{ Q }{ U } = \frac{ 25\,\text{нКл} }{ 6\,\text{кВ} } = 4{,}17\,\text{пФ}.
    \text{ Заряды обкладок: $Q$ и $-Q$}$
}
\vspace{120pt}

\tasknumber{2}\task{
    На конденсаторе указано: $C = 50\,\text{пФ}$, $V = 400\,\text{кВ}$.
    Удастся ли его использовать для накопления заряда $Q = 50\,\text{нКл}$?
}
\answer{
    $Q' = CV = 50\,\text{пФ} \cdot 400\,\text{кВ} = 20000\,\text{нКл}
    \implies Q' \ge Q \implies \text{удастся}$
}
\vspace{120pt}

\tasknumber{3}\task{
    Как и во сколько раз изменится ёмкость плоского конденсатора при уменьшении площади пластин в 2 раз
    и уменьшении расстояния между ними в 8 раз?
}
\answer{
    $\frac{C'}{C}
        = \frac{\eps_0\eps \frac S2}{\frac d8} \Big/ \frac{\eps_0\eps S}{d}
        = \frac{ 8 }{ 2 } > 1 \implies \text{увеличится в $\frac41$ раз}
    $
}
\vspace{120pt}

\tasknumber{4}\task{
    Электрическая ёмкость конденсатора равна $С = 400\,\text{пФ}$,
    при этом ему сообщён заряд $Q = 800\,\text{нКл}$.
    Какова энергия заряженного конденсатора?
}
\answer{
    $W
        = \frac{ Q^2 }{ 2С }
        = \frac{ \sqr{ 800\,\text{нКл} } }{ 2 \cdot 400\,\text{пФ} }
        = 800{,}0\,\text{мкДж}
    $
}
\vspace{120pt}

\tasknumber{5}\task{
    Два конденсатора ёмкостей $С_1 = 30\,\text{нФ}$ и $С_2 = 60\,\text{нФ}$ последовательно подключают
    к источнику напряжения $V = 150\,\text{В}$ (см.
    рис.).
    Определите заряды каждого из конденсаторов.
}
\answer{
    $Q_1
        = Q_2
        = CV
        = \frac{V}{\frac1{C_1} + \frac1{C_2}}
        = \frac{C_1C_2V}{C_1 + C_2}
        = \frac{
            30\,\text{нФ} \cdot 60\,\text{нФ} \cdot 150\,\text{В}
        }{
            30\,\text{нФ} + 60\,\text{нФ}
        }
        = 3000{,}0\,\text{нКл}
    $
}
\newpage

\addpersonalvariant{Алина Леоничева}


\tasknumber{1}\task{
    Определите ёмкость конденсатора, если при его зарядке до напряжения
    $U = 12\,\text{кВ}$ он приобретает заряд $Q = 15\,\text{нКл}$.
    Чему при этом равны заряды обкладок конденсатора (сделайте рисунок)?
}
\answer{
    $Q = CU \implies
    C = \frac{ Q }{ U } = \frac{ 15\,\text{нКл} }{ 12\,\text{кВ} } = 1{,}25\,\text{пФ}.
    \text{ Заряды обкладок: $Q$ и $-Q$}$
}
\vspace{120pt}

\tasknumber{2}\task{
    На конденсаторе указано: $C = 80\,\text{пФ}$, $V = 200\,\text{кВ}$.
    Удастся ли его использовать для накопления заряда $q = 60\,\text{нКл}$?
}
\answer{
    $q' = CV = 80\,\text{пФ} \cdot 200\,\text{кВ} = 16000\,\text{нКл}
    \implies q' \ge q \implies \text{удастся}$
}
\vspace{120pt}

\tasknumber{3}\task{
    Как и во сколько раз изменится ёмкость плоского конденсатора при уменьшении площади пластин в 2 раз
    и уменьшении расстояния между ними в 3 раз?
}
\answer{
    $\frac{C'}{C}
        = \frac{\eps_0\eps \frac S2}{\frac d3} \Big/ \frac{\eps_0\eps S}{d}
        = \frac{ 3 }{ 2 } > 1 \implies \text{увеличится в $\frac32$ раз}
    $
}
\vspace{120pt}

\tasknumber{4}\task{
    Электрическая ёмкость конденсатора равна $С = 600\,\text{пФ}$,
    при этом ему сообщён заряд $q = 300\,\text{нКл}$.
    Какова энергия заряженного конденсатора?
}
\answer{
    $W
        = \frac{ q^2 }{ 2С }
        = \frac{ \sqr{ 300\,\text{нКл} } }{ 2 \cdot 600\,\text{пФ} }
        = 75{,}0\,\text{мкДж}
    $
}
\vspace{120pt}

\tasknumber{5}\task{
    Два конденсатора ёмкостей $С_1 = 20\,\text{нФ}$ и $С_2 = 60\,\text{нФ}$ последовательно подключают
    к источнику напряжения $U = 300\,\text{В}$ (см.
    рис.).
    Определите заряды каждого из конденсаторов.
}
\answer{
    $Q_1
        = Q_2
        = CU
        = \frac{U}{\frac1{C_1} + \frac1{C_2}}
        = \frac{C_1C_2U}{C_1 + C_2}
        = \frac{
            20\,\text{нФ} \cdot 60\,\text{нФ} \cdot 300\,\text{В}
        }{
            20\,\text{нФ} + 60\,\text{нФ}
        }
        = 4500{,}0\,\text{нКл}
    $
}
\newpage

\addpersonalvariant{Ирина Лин}


\tasknumber{1}\task{
    Определите ёмкость конденсатора, если при его зарядке до напряжения
    $V = 20\,\text{кВ}$ он приобретает заряд $q = 25\,\text{нКл}$.
    Чему при этом равны заряды обкладок конденсатора (сделайте рисунок)?
}
\answer{
    $q = CV \implies
    C = \frac{ q }{ V } = \frac{ 25\,\text{нКл} }{ 20\,\text{кВ} } = 1{,}25\,\text{пФ}.
    \text{ Заряды обкладок: $q$ и $-q$}$
}
\vspace{120pt}

\tasknumber{2}\task{
    На конденсаторе указано: $C = 120\,\text{пФ}$, $V = 200\,\text{кВ}$.
    Удастся ли его использовать для накопления заряда $q = 50\,\text{нКл}$?
}
\answer{
    $q' = CV = 120\,\text{пФ} \cdot 200\,\text{кВ} = 24000\,\text{нКл}
    \implies q' \ge q \implies \text{удастся}$
}
\vspace{120pt}

\tasknumber{3}\task{
    Как и во сколько раз изменится ёмкость плоского конденсатора при уменьшении площади пластин в 3 раз
    и уменьшении расстояния между ними в 8 раз?
}
\answer{
    $\frac{C'}{C}
        = \frac{\eps_0\eps \frac S3}{\frac d8} \Big/ \frac{\eps_0\eps S}{d}
        = \frac{ 8 }{ 3 } > 1 \implies \text{увеличится в $\frac83$ раз}
    $
}
\vspace{120pt}

\tasknumber{4}\task{
    Электрическая ёмкость конденсатора равна $С = 750\,\text{пФ}$,
    при этом ему сообщён заряд $q = 900\,\text{нКл}$.
    Какова энергия заряженного конденсатора?
}
\answer{
    $W
        = \frac{ q^2 }{ 2С }
        = \frac{ \sqr{ 900\,\text{нКл} } }{ 2 \cdot 750\,\text{пФ} }
        = 540{,}0\,\text{мкДж}
    $
}
\vspace{120pt}

\tasknumber{5}\task{
    Два конденсатора ёмкостей $С_1 = 60\,\text{нФ}$ и $С_2 = 40\,\text{нФ}$ последовательно подключают
    к источнику напряжения $U = 200\,\text{В}$ (см.
    рис.).
    Определите заряды каждого из конденсаторов.
}
\answer{
    $Q_1
        = Q_2
        = CU
        = \frac{U}{\frac1{C_1} + \frac1{C_2}}
        = \frac{C_1C_2U}{C_1 + C_2}
        = \frac{
            60\,\text{нФ} \cdot 40\,\text{нФ} \cdot 200\,\text{В}
        }{
            60\,\text{нФ} + 40\,\text{нФ}
        }
        = 4800{,}0\,\text{нКл}
    $
}
\newpage

\addpersonalvariant{Олег Мальцев}


\tasknumber{1}\task{
    Определите ёмкость конденсатора, если при его зарядке до напряжения
    $U = 12\,\text{кВ}$ он приобретает заряд $q = 24\,\text{нКл}$.
    Чему при этом равны заряды обкладок конденсатора (сделайте рисунок)?
}
\answer{
    $q = CU \implies
    C = \frac{ q }{ U } = \frac{ 24\,\text{нКл} }{ 12\,\text{кВ} } = 2{,}0\,\text{пФ}.
    \text{ Заряды обкладок: $q$ и $-q$}$
}
\vspace{120pt}

\tasknumber{2}\task{
    На конденсаторе указано: $C = 80\,\text{пФ}$, $V = 450\,\text{кВ}$.
    Удастся ли его использовать для накопления заряда $q = 30\,\text{нКл}$?
}
\answer{
    $q' = CV = 80\,\text{пФ} \cdot 450\,\text{кВ} = 36000\,\text{нКл}
    \implies q' \ge q \implies \text{удастся}$
}
\vspace{120pt}

\tasknumber{3}\task{
    Как и во сколько раз изменится ёмкость плоского конденсатора при уменьшении площади пластин в 2 раз
    и уменьшении расстояния между ними в 4 раз?
}
\answer{
    $\frac{C'}{C}
        = \frac{\eps_0\eps \frac S2}{\frac d4} \Big/ \frac{\eps_0\eps S}{d}
        = \frac{ 4 }{ 2 } > 1 \implies \text{увеличится в $\frac21$ раз}
    $
}
\vspace{120pt}

\tasknumber{4}\task{
    Электрическая ёмкость конденсатора равна $С = 600\,\text{пФ}$,
    при этом ему сообщён заряд $Q = 300\,\text{нКл}$.
    Какова энергия заряженного конденсатора?
}
\answer{
    $W
        = \frac{ Q^2 }{ 2С }
        = \frac{ \sqr{ 300\,\text{нКл} } }{ 2 \cdot 600\,\text{пФ} }
        = 75{,}0\,\text{мкДж}
    $
}
\vspace{120pt}

\tasknumber{5}\task{
    Два конденсатора ёмкостей $С_1 = 60\,\text{нФ}$ и $С_2 = 30\,\text{нФ}$ последовательно подключают
    к источнику напряжения $V = 400\,\text{В}$ (см.
    рис.).
    Определите заряды каждого из конденсаторов.
}
\answer{
    $Q_1
        = Q_2
        = CV
        = \frac{V}{\frac1{C_1} + \frac1{C_2}}
        = \frac{C_1C_2V}{C_1 + C_2}
        = \frac{
            60\,\text{нФ} \cdot 30\,\text{нФ} \cdot 400\,\text{В}
        }{
            60\,\text{нФ} + 30\,\text{нФ}
        }
        = 8000{,}0\,\text{нКл}
    $
}
\newpage

\addpersonalvariant{Ислам Мунаев}


\tasknumber{1}\task{
    Определите ёмкость конденсатора, если при его зарядке до напряжения
    $V = 5\,\text{кВ}$ он приобретает заряд $Q = 15\,\text{нКл}$.
    Чему при этом равны заряды обкладок конденсатора (сделайте рисунок)?
}
\answer{
    $Q = CV \implies
    C = \frac{ Q }{ V } = \frac{ 15\,\text{нКл} }{ 5\,\text{кВ} } = 3{,}0\,\text{пФ}.
    \text{ Заряды обкладок: $Q$ и $-Q$}$
}
\vspace{120pt}

\tasknumber{2}\task{
    На конденсаторе указано: $C = 100\,\text{пФ}$, $U = 400\,\text{кВ}$.
    Удастся ли его использовать для накопления заряда $Q = 30\,\text{нКл}$?
}
\answer{
    $Q' = CU = 100\,\text{пФ} \cdot 400\,\text{кВ} = 40000\,\text{нКл}
    \implies Q' \ge Q \implies \text{удастся}$
}
\vspace{120pt}

\tasknumber{3}\task{
    Как и во сколько раз изменится ёмкость плоского конденсатора при уменьшении площади пластин в 4 раз
    и уменьшении расстояния между ними в 7 раз?
}
\answer{
    $\frac{C'}{C}
        = \frac{\eps_0\eps \frac S4}{\frac d7} \Big/ \frac{\eps_0\eps S}{d}
        = \frac{ 7 }{ 4 } > 1 \implies \text{увеличится в $\frac74$ раз}
    $
}
\vspace{120pt}

\tasknumber{4}\task{
    Электрическая ёмкость конденсатора равна $С = 600\,\text{пФ}$,
    при этом ему сообщён заряд $Q = 900\,\text{нКл}$.
    Какова энергия заряженного конденсатора?
}
\answer{
    $W
        = \frac{ Q^2 }{ 2С }
        = \frac{ \sqr{ 900\,\text{нКл} } }{ 2 \cdot 600\,\text{пФ} }
        = 675{,}0\,\text{мкДж}
    $
}
\vspace{120pt}

\tasknumber{5}\task{
    Два конденсатора ёмкостей $С_1 = 20\,\text{нФ}$ и $С_2 = 60\,\text{нФ}$ последовательно подключают
    к источнику напряжения $V = 150\,\text{В}$ (см.
    рис.).
    Определите заряды каждого из конденсаторов.
}
\answer{
    $Q_1
        = Q_2
        = CV
        = \frac{V}{\frac1{C_1} + \frac1{C_2}}
        = \frac{C_1C_2V}{C_1 + C_2}
        = \frac{
            20\,\text{нФ} \cdot 60\,\text{нФ} \cdot 150\,\text{В}
        }{
            20\,\text{нФ} + 60\,\text{нФ}
        }
        = 2250{,}0\,\text{нКл}
    $
}
\newpage

\addpersonalvariant{Александр Наумов}


\tasknumber{1}\task{
    Определите ёмкость конденсатора, если при его зарядке до напряжения
    $U = 6\,\text{кВ}$ он приобретает заряд $Q = 6\,\text{нКл}$.
    Чему при этом равны заряды обкладок конденсатора (сделайте рисунок)?
}
\answer{
    $Q = CU \implies
    C = \frac{ Q }{ U } = \frac{ 6\,\text{нКл} }{ 6\,\text{кВ} } = 1{,}0\,\text{пФ}.
    \text{ Заряды обкладок: $Q$ и $-Q$}$
}
\vspace{120pt}

\tasknumber{2}\task{
    На конденсаторе указано: $C = 150\,\text{пФ}$, $U = 450\,\text{кВ}$.
    Удастся ли его использовать для накопления заряда $q = 30\,\text{нКл}$?
}
\answer{
    $q' = CU = 150\,\text{пФ} \cdot 450\,\text{кВ} = 67500\,\text{нКл}
    \implies q' \ge q \implies \text{удастся}$
}
\vspace{120pt}

\tasknumber{3}\task{
    Как и во сколько раз изменится ёмкость плоского конденсатора при уменьшении площади пластин в 6 раз
    и уменьшении расстояния между ними в 4 раз?
}
\answer{
    $\frac{C'}{C}
        = \frac{\eps_0\eps \frac S6}{\frac d4} \Big/ \frac{\eps_0\eps S}{d}
        = \frac{ 4 }{ 6 } < 1 \implies \text{уменьшится в $\frac32$ раз}
    $
}
\vspace{120pt}

\tasknumber{4}\task{
    Электрическая ёмкость конденсатора равна $С = 400\,\text{пФ}$,
    при этом ему сообщён заряд $q = 900\,\text{нКл}$.
    Какова энергия заряженного конденсатора?
}
\answer{
    $W
        = \frac{ q^2 }{ 2С }
        = \frac{ \sqr{ 900\,\text{нКл} } }{ 2 \cdot 400\,\text{пФ} }
        = 1012{,}5\,\text{мкДж}
    $
}
\vspace{120pt}

\tasknumber{5}\task{
    Два конденсатора ёмкостей $С_1 = 20\,\text{нФ}$ и $С_2 = 40\,\text{нФ}$ последовательно подключают
    к источнику напряжения $V = 450\,\text{В}$ (см.
    рис.).
    Определите заряды каждого из конденсаторов.
}
\answer{
    $Q_1
        = Q_2
        = CV
        = \frac{V}{\frac1{C_1} + \frac1{C_2}}
        = \frac{C_1C_2V}{C_1 + C_2}
        = \frac{
            20\,\text{нФ} \cdot 40\,\text{нФ} \cdot 450\,\text{В}
        }{
            20\,\text{нФ} + 40\,\text{нФ}
        }
        = 6000{,}0\,\text{нКл}
    $
}
\newpage

\addpersonalvariant{Георгий Новиков}


\tasknumber{1}\task{
    Определите ёмкость конденсатора, если при его зарядке до напряжения
    $V = 2\,\text{кВ}$ он приобретает заряд $q = 6\,\text{нКл}$.
    Чему при этом равны заряды обкладок конденсатора (сделайте рисунок)?
}
\answer{
    $q = CV \implies
    C = \frac{ q }{ V } = \frac{ 6\,\text{нКл} }{ 2\,\text{кВ} } = 3{,}0\,\text{пФ}.
    \text{ Заряды обкладок: $q$ и $-q$}$
}
\vspace{120pt}

\tasknumber{2}\task{
    На конденсаторе указано: $C = 80\,\text{пФ}$, $V = 450\,\text{кВ}$.
    Удастся ли его использовать для накопления заряда $Q = 30\,\text{нКл}$?
}
\answer{
    $Q' = CV = 80\,\text{пФ} \cdot 450\,\text{кВ} = 36000\,\text{нКл}
    \implies Q' \ge Q \implies \text{удастся}$
}
\vspace{120pt}

\tasknumber{3}\task{
    Как и во сколько раз изменится ёмкость плоского конденсатора при уменьшении площади пластин в 8 раз
    и уменьшении расстояния между ними в 2 раз?
}
\answer{
    $\frac{C'}{C}
        = \frac{\eps_0\eps \frac S8}{\frac d2} \Big/ \frac{\eps_0\eps S}{d}
        = \frac{ 2 }{ 8 } < 1 \implies \text{уменьшится в $\frac41$ раз}
    $
}
\vspace{120pt}

\tasknumber{4}\task{
    Электрическая ёмкость конденсатора равна $С = 750\,\text{пФ}$,
    при этом ему сообщён заряд $q = 300\,\text{нКл}$.
    Какова энергия заряженного конденсатора?
}
\answer{
    $W
        = \frac{ q^2 }{ 2С }
        = \frac{ \sqr{ 300\,\text{нКл} } }{ 2 \cdot 750\,\text{пФ} }
        = 60{,}0\,\text{мкДж}
    $
}
\vspace{120pt}

\tasknumber{5}\task{
    Два конденсатора ёмкостей $С_1 = 40\,\text{нФ}$ и $С_2 = 30\,\text{нФ}$ последовательно подключают
    к источнику напряжения $V = 300\,\text{В}$ (см.
    рис.).
    Определите заряды каждого из конденсаторов.
}
\answer{
    $Q_1
        = Q_2
        = CV
        = \frac{V}{\frac1{C_1} + \frac1{C_2}}
        = \frac{C_1C_2V}{C_1 + C_2}
        = \frac{
            40\,\text{нФ} \cdot 30\,\text{нФ} \cdot 300\,\text{В}
        }{
            40\,\text{нФ} + 30\,\text{нФ}
        }
        = 5142{,}86\,\text{нКл}
    $
}
\newpage

\addpersonalvariant{Егор Осипов}


\tasknumber{1}\task{
    Определите ёмкость конденсатора, если при его зарядке до напряжения
    $U = 20\,\text{кВ}$ он приобретает заряд $q = 15\,\text{нКл}$.
    Чему при этом равны заряды обкладок конденсатора (сделайте рисунок)?
}
\answer{
    $q = CU \implies
    C = \frac{ q }{ U } = \frac{ 15\,\text{нКл} }{ 20\,\text{кВ} } = 0{,}75\,\text{пФ}.
    \text{ Заряды обкладок: $q$ и $-q$}$
}
\vspace{120pt}

\tasknumber{2}\task{
    На конденсаторе указано: $C = 120\,\text{пФ}$, $U = 200\,\text{кВ}$.
    Удастся ли его использовать для накопления заряда $Q = 50\,\text{нКл}$?
}
\answer{
    $Q' = CU = 120\,\text{пФ} \cdot 200\,\text{кВ} = 24000\,\text{нКл}
    \implies Q' \ge Q \implies \text{удастся}$
}
\vspace{120pt}

\tasknumber{3}\task{
    Как и во сколько раз изменится ёмкость плоского конденсатора при уменьшении площади пластин в 5 раз
    и уменьшении расстояния между ними в 5 раз?
}
\answer{
    $\frac{C'}{C}
        = \frac{\eps_0\eps \frac S5}{\frac d5} \Big/ \frac{\eps_0\eps S}{d}
        = \frac{ 5 }{ 5 } = 1 \implies \text{не изменится}
    $
}
\vspace{120pt}

\tasknumber{4}\task{
    Электрическая ёмкость конденсатора равна $С = 200\,\text{пФ}$,
    при этом ему сообщён заряд $q = 800\,\text{нКл}$.
    Какова энергия заряженного конденсатора?
}
\answer{
    $W
        = \frac{ q^2 }{ 2С }
        = \frac{ \sqr{ 800\,\text{нКл} } }{ 2 \cdot 200\,\text{пФ} }
        = 1600{,}0\,\text{мкДж}
    $
}
\vspace{120pt}

\tasknumber{5}\task{
    Два конденсатора ёмкостей $С_1 = 40\,\text{нФ}$ и $С_2 = 60\,\text{нФ}$ последовательно подключают
    к источнику напряжения $U = 150\,\text{В}$ (см.
    рис.).
    Определите заряды каждого из конденсаторов.
}
\answer{
    $Q_1
        = Q_2
        = CU
        = \frac{U}{\frac1{C_1} + \frac1{C_2}}
        = \frac{C_1C_2U}{C_1 + C_2}
        = \frac{
            40\,\text{нФ} \cdot 60\,\text{нФ} \cdot 150\,\text{В}
        }{
            40\,\text{нФ} + 60\,\text{нФ}
        }
        = 3600{,}0\,\text{нКл}
    $
}
\newpage

\addpersonalvariant{Руслан Перепелица}


\tasknumber{1}\task{
    Определите ёмкость конденсатора, если при его зарядке до напряжения
    $V = 3\,\text{кВ}$ он приобретает заряд $Q = 4\,\text{нКл}$.
    Чему при этом равны заряды обкладок конденсатора (сделайте рисунок)?
}
\answer{
    $Q = CV \implies
    C = \frac{ Q }{ V } = \frac{ 4\,\text{нКл} }{ 3\,\text{кВ} } = 1{,}33\,\text{пФ}.
    \text{ Заряды обкладок: $Q$ и $-Q$}$
}
\vspace{120pt}

\tasknumber{2}\task{
    На конденсаторе указано: $C = 100\,\text{пФ}$, $V = 400\,\text{кВ}$.
    Удастся ли его использовать для накопления заряда $Q = 60\,\text{нКл}$?
}
\answer{
    $Q' = CV = 100\,\text{пФ} \cdot 400\,\text{кВ} = 40000\,\text{нКл}
    \implies Q' \ge Q \implies \text{удастся}$
}
\vspace{120pt}

\tasknumber{3}\task{
    Как и во сколько раз изменится ёмкость плоского конденсатора при уменьшении площади пластин в 4 раз
    и уменьшении расстояния между ними в 7 раз?
}
\answer{
    $\frac{C'}{C}
        = \frac{\eps_0\eps \frac S4}{\frac d7} \Big/ \frac{\eps_0\eps S}{d}
        = \frac{ 7 }{ 4 } > 1 \implies \text{увеличится в $\frac74$ раз}
    $
}
\vspace{120pt}

\tasknumber{4}\task{
    Электрическая ёмкость конденсатора равна $С = 200\,\text{пФ}$,
    при этом ему сообщён заряд $Q = 300\,\text{нКл}$.
    Какова энергия заряженного конденсатора?
}
\answer{
    $W
        = \frac{ Q^2 }{ 2С }
        = \frac{ \sqr{ 300\,\text{нКл} } }{ 2 \cdot 200\,\text{пФ} }
        = 225{,}0\,\text{мкДж}
    $
}
\vspace{120pt}

\tasknumber{5}\task{
    Два конденсатора ёмкостей $С_1 = 20\,\text{нФ}$ и $С_2 = 30\,\text{нФ}$ последовательно подключают
    к источнику напряжения $V = 200\,\text{В}$ (см.
    рис.).
    Определите заряды каждого из конденсаторов.
}
\answer{
    $Q_1
        = Q_2
        = CV
        = \frac{V}{\frac1{C_1} + \frac1{C_2}}
        = \frac{C_1C_2V}{C_1 + C_2}
        = \frac{
            20\,\text{нФ} \cdot 30\,\text{нФ} \cdot 200\,\text{В}
        }{
            20\,\text{нФ} + 30\,\text{нФ}
        }
        = 2400{,}0\,\text{нКл}
    $
}
\newpage

\addpersonalvariant{Михаил Перин}


\tasknumber{1}\task{
    Определите ёмкость конденсатора, если при его зарядке до напряжения
    $U = 3\,\text{кВ}$ он приобретает заряд $q = 4\,\text{нКл}$.
    Чему при этом равны заряды обкладок конденсатора (сделайте рисунок)?
}
\answer{
    $q = CU \implies
    C = \frac{ q }{ U } = \frac{ 4\,\text{нКл} }{ 3\,\text{кВ} } = 1{,}33\,\text{пФ}.
    \text{ Заряды обкладок: $q$ и $-q$}$
}
\vspace{120pt}

\tasknumber{2}\task{
    На конденсаторе указано: $C = 120\,\text{пФ}$, $U = 200\,\text{кВ}$.
    Удастся ли его использовать для накопления заряда $q = 30\,\text{нКл}$?
}
\answer{
    $q' = CU = 120\,\text{пФ} \cdot 200\,\text{кВ} = 24000\,\text{нКл}
    \implies q' \ge q \implies \text{удастся}$
}
\vspace{120pt}

\tasknumber{3}\task{
    Как и во сколько раз изменится ёмкость плоского конденсатора при уменьшении площади пластин в 6 раз
    и уменьшении расстояния между ними в 6 раз?
}
\answer{
    $\frac{C'}{C}
        = \frac{\eps_0\eps \frac S6}{\frac d6} \Big/ \frac{\eps_0\eps S}{d}
        = \frac{ 6 }{ 6 } = 1 \implies \text{не изменится}
    $
}
\vspace{120pt}

\tasknumber{4}\task{
    Электрическая ёмкость конденсатора равна $С = 600\,\text{пФ}$,
    при этом ему сообщён заряд $q = 300\,\text{нКл}$.
    Какова энергия заряженного конденсатора?
}
\answer{
    $W
        = \frac{ q^2 }{ 2С }
        = \frac{ \sqr{ 300\,\text{нКл} } }{ 2 \cdot 600\,\text{пФ} }
        = 75{,}0\,\text{мкДж}
    $
}
\vspace{120pt}

\tasknumber{5}\task{
    Два конденсатора ёмкостей $С_1 = 40\,\text{нФ}$ и $С_2 = 20\,\text{нФ}$ последовательно подключают
    к источнику напряжения $U = 200\,\text{В}$ (см.
    рис.).
    Определите заряды каждого из конденсаторов.
}
\answer{
    $Q_1
        = Q_2
        = CU
        = \frac{U}{\frac1{C_1} + \frac1{C_2}}
        = \frac{C_1C_2U}{C_1 + C_2}
        = \frac{
            40\,\text{нФ} \cdot 20\,\text{нФ} \cdot 200\,\text{В}
        }{
            40\,\text{нФ} + 20\,\text{нФ}
        }
        = 2666{,}67\,\text{нКл}
    $
}
\newpage

\addpersonalvariant{Егор Подуровский}


\tasknumber{1}\task{
    Определите ёмкость конденсатора, если при его зарядке до напряжения
    $U = 2\,\text{кВ}$ он приобретает заряд $Q = 15\,\text{нКл}$.
    Чему при этом равны заряды обкладок конденсатора (сделайте рисунок)?
}
\answer{
    $Q = CU \implies
    C = \frac{ Q }{ U } = \frac{ 15\,\text{нКл} }{ 2\,\text{кВ} } = 7{,}5\,\text{пФ}.
    \text{ Заряды обкладок: $Q$ и $-Q$}$
}
\vspace{120pt}

\tasknumber{2}\task{
    На конденсаторе указано: $C = 50\,\text{пФ}$, $V = 300\,\text{кВ}$.
    Удастся ли его использовать для накопления заряда $q = 50\,\text{нКл}$?
}
\answer{
    $q' = CV = 50\,\text{пФ} \cdot 300\,\text{кВ} = 15000\,\text{нКл}
    \implies q' \ge q \implies \text{удастся}$
}
\vspace{120pt}

\tasknumber{3}\task{
    Как и во сколько раз изменится ёмкость плоского конденсатора при уменьшении площади пластин в 4 раз
    и уменьшении расстояния между ними в 4 раз?
}
\answer{
    $\frac{C'}{C}
        = \frac{\eps_0\eps \frac S4}{\frac d4} \Big/ \frac{\eps_0\eps S}{d}
        = \frac{ 4 }{ 4 } = 1 \implies \text{не изменится}
    $
}
\vspace{120pt}

\tasknumber{4}\task{
    Электрическая ёмкость конденсатора равна $С = 600\,\text{пФ}$,
    при этом ему сообщён заряд $Q = 800\,\text{нКл}$.
    Какова энергия заряженного конденсатора?
}
\answer{
    $W
        = \frac{ Q^2 }{ 2С }
        = \frac{ \sqr{ 800\,\text{нКл} } }{ 2 \cdot 600\,\text{пФ} }
        = 533{,}33\,\text{мкДж}
    $
}
\vspace{120pt}

\tasknumber{5}\task{
    Два конденсатора ёмкостей $С_1 = 20\,\text{нФ}$ и $С_2 = 60\,\text{нФ}$ последовательно подключают
    к источнику напряжения $U = 450\,\text{В}$ (см.
    рис.).
    Определите заряды каждого из конденсаторов.
}
\answer{
    $Q_1
        = Q_2
        = CU
        = \frac{U}{\frac1{C_1} + \frac1{C_2}}
        = \frac{C_1C_2U}{C_1 + C_2}
        = \frac{
            20\,\text{нФ} \cdot 60\,\text{нФ} \cdot 450\,\text{В}
        }{
            20\,\text{нФ} + 60\,\text{нФ}
        }
        = 6750{,}0\,\text{нКл}
    $
}
\newpage

\addpersonalvariant{Роман Прибылов}


\tasknumber{1}\task{
    Определите ёмкость конденсатора, если при его зарядке до напряжения
    $U = 15\,\text{кВ}$ он приобретает заряд $q = 4\,\text{нКл}$.
    Чему при этом равны заряды обкладок конденсатора (сделайте рисунок)?
}
\answer{
    $q = CU \implies
    C = \frac{ q }{ U } = \frac{ 4\,\text{нКл} }{ 15\,\text{кВ} } = 0{,}27\,\text{пФ}.
    \text{ Заряды обкладок: $q$ и $-q$}$
}
\vspace{120pt}

\tasknumber{2}\task{
    На конденсаторе указано: $C = 50\,\text{пФ}$, $U = 450\,\text{кВ}$.
    Удастся ли его использовать для накопления заряда $Q = 30\,\text{нКл}$?
}
\answer{
    $Q' = CU = 50\,\text{пФ} \cdot 450\,\text{кВ} = 22500\,\text{нКл}
    \implies Q' \ge Q \implies \text{удастся}$
}
\vspace{120pt}

\tasknumber{3}\task{
    Как и во сколько раз изменится ёмкость плоского конденсатора при уменьшении площади пластин в 8 раз
    и уменьшении расстояния между ними в 5 раз?
}
\answer{
    $\frac{C'}{C}
        = \frac{\eps_0\eps \frac S8}{\frac d5} \Big/ \frac{\eps_0\eps S}{d}
        = \frac{ 5 }{ 8 } < 1 \implies \text{уменьшится в $\frac85$ раз}
    $
}
\vspace{120pt}

\tasknumber{4}\task{
    Электрическая ёмкость конденсатора равна $С = 600\,\text{пФ}$,
    при этом ему сообщён заряд $q = 800\,\text{нКл}$.
    Какова энергия заряженного конденсатора?
}
\answer{
    $W
        = \frac{ q^2 }{ 2С }
        = \frac{ \sqr{ 800\,\text{нКл} } }{ 2 \cdot 600\,\text{пФ} }
        = 533{,}33\,\text{мкДж}
    $
}
\vspace{120pt}

\tasknumber{5}\task{
    Два конденсатора ёмкостей $С_1 = 40\,\text{нФ}$ и $С_2 = 20\,\text{нФ}$ последовательно подключают
    к источнику напряжения $V = 300\,\text{В}$ (см.
    рис.).
    Определите заряды каждого из конденсаторов.
}
\answer{
    $Q_1
        = Q_2
        = CV
        = \frac{V}{\frac1{C_1} + \frac1{C_2}}
        = \frac{C_1C_2V}{C_1 + C_2}
        = \frac{
            40\,\text{нФ} \cdot 20\,\text{нФ} \cdot 300\,\text{В}
        }{
            40\,\text{нФ} + 20\,\text{нФ}
        }
        = 4000{,}0\,\text{нКл}
    $
}
\newpage

\addpersonalvariant{Александр Селехметьев}


\tasknumber{1}\task{
    Определите ёмкость конденсатора, если при его зарядке до напряжения
    $V = 3\,\text{кВ}$ он приобретает заряд $q = 25\,\text{нКл}$.
    Чему при этом равны заряды обкладок конденсатора (сделайте рисунок)?
}
\answer{
    $q = CV \implies
    C = \frac{ q }{ V } = \frac{ 25\,\text{нКл} }{ 3\,\text{кВ} } = 8{,}33\,\text{пФ}.
    \text{ Заряды обкладок: $q$ и $-q$}$
}
\vspace{120pt}

\tasknumber{2}\task{
    На конденсаторе указано: $C = 50\,\text{пФ}$, $V = 300\,\text{кВ}$.
    Удастся ли его использовать для накопления заряда $Q = 50\,\text{нКл}$?
}
\answer{
    $Q' = CV = 50\,\text{пФ} \cdot 300\,\text{кВ} = 15000\,\text{нКл}
    \implies Q' \ge Q \implies \text{удастся}$
}
\vspace{120pt}

\tasknumber{3}\task{
    Как и во сколько раз изменится ёмкость плоского конденсатора при уменьшении площади пластин в 3 раз
    и уменьшении расстояния между ними в 4 раз?
}
\answer{
    $\frac{C'}{C}
        = \frac{\eps_0\eps \frac S3}{\frac d4} \Big/ \frac{\eps_0\eps S}{d}
        = \frac{ 4 }{ 3 } > 1 \implies \text{увеличится в $\frac43$ раз}
    $
}
\vspace{120pt}

\tasknumber{4}\task{
    Электрическая ёмкость конденсатора равна $С = 750\,\text{пФ}$,
    при этом ему сообщён заряд $Q = 800\,\text{нКл}$.
    Какова энергия заряженного конденсатора?
}
\answer{
    $W
        = \frac{ Q^2 }{ 2С }
        = \frac{ \sqr{ 800\,\text{нКл} } }{ 2 \cdot 750\,\text{пФ} }
        = 426{,}67\,\text{мкДж}
    $
}
\vspace{120pt}

\tasknumber{5}\task{
    Два конденсатора ёмкостей $С_1 = 60\,\text{нФ}$ и $С_2 = 40\,\text{нФ}$ последовательно подключают
    к источнику напряжения $U = 300\,\text{В}$ (см.
    рис.).
    Определите заряды каждого из конденсаторов.
}
\answer{
    $Q_1
        = Q_2
        = CU
        = \frac{U}{\frac1{C_1} + \frac1{C_2}}
        = \frac{C_1C_2U}{C_1 + C_2}
        = \frac{
            60\,\text{нФ} \cdot 40\,\text{нФ} \cdot 300\,\text{В}
        }{
            60\,\text{нФ} + 40\,\text{нФ}
        }
        = 7200{,}0\,\text{нКл}
    $
}
\newpage

\addpersonalvariant{Алексей Тихонов}


\tasknumber{1}\task{
    Определите ёмкость конденсатора, если при его зарядке до напряжения
    $V = 15\,\text{кВ}$ он приобретает заряд $Q = 4\,\text{нКл}$.
    Чему при этом равны заряды обкладок конденсатора (сделайте рисунок)?
}
\answer{
    $Q = CV \implies
    C = \frac{ Q }{ V } = \frac{ 4\,\text{нКл} }{ 15\,\text{кВ} } = 0{,}27\,\text{пФ}.
    \text{ Заряды обкладок: $Q$ и $-Q$}$
}
\vspace{120pt}

\tasknumber{2}\task{
    На конденсаторе указано: $C = 50\,\text{пФ}$, $V = 200\,\text{кВ}$.
    Удастся ли его использовать для накопления заряда $Q = 30\,\text{нКл}$?
}
\answer{
    $Q' = CV = 50\,\text{пФ} \cdot 200\,\text{кВ} = 10000\,\text{нКл}
    \implies Q' \ge Q \implies \text{удастся}$
}
\vspace{120pt}

\tasknumber{3}\task{
    Как и во сколько раз изменится ёмкость плоского конденсатора при уменьшении площади пластин в 2 раз
    и уменьшении расстояния между ними в 6 раз?
}
\answer{
    $\frac{C'}{C}
        = \frac{\eps_0\eps \frac S2}{\frac d6} \Big/ \frac{\eps_0\eps S}{d}
        = \frac{ 6 }{ 2 } > 1 \implies \text{увеличится в $\frac31$ раз}
    $
}
\vspace{120pt}

\tasknumber{4}\task{
    Электрическая ёмкость конденсатора равна $С = 200\,\text{пФ}$,
    при этом ему сообщён заряд $q = 800\,\text{нКл}$.
    Какова энергия заряженного конденсатора?
}
\answer{
    $W
        = \frac{ q^2 }{ 2С }
        = \frac{ \sqr{ 800\,\text{нКл} } }{ 2 \cdot 200\,\text{пФ} }
        = 1600{,}0\,\text{мкДж}
    $
}
\vspace{120pt}

\tasknumber{5}\task{
    Два конденсатора ёмкостей $С_1 = 20\,\text{нФ}$ и $С_2 = 30\,\text{нФ}$ последовательно подключают
    к источнику напряжения $U = 150\,\text{В}$ (см.
    рис.).
    Определите заряды каждого из конденсаторов.
}
\answer{
    $Q_1
        = Q_2
        = CU
        = \frac{U}{\frac1{C_1} + \frac1{C_2}}
        = \frac{C_1C_2U}{C_1 + C_2}
        = \frac{
            20\,\text{нФ} \cdot 30\,\text{нФ} \cdot 150\,\text{В}
        }{
            20\,\text{нФ} + 30\,\text{нФ}
        }
        = 1800{,}0\,\text{нКл}
    $
}
\newpage

\addpersonalvariant{Алина Филиппова}


\tasknumber{1}\task{
    Определите ёмкость конденсатора, если при его зарядке до напряжения
    $U = 5\,\text{кВ}$ он приобретает заряд $Q = 25\,\text{нКл}$.
    Чему при этом равны заряды обкладок конденсатора (сделайте рисунок)?
}
\answer{
    $Q = CU \implies
    C = \frac{ Q }{ U } = \frac{ 25\,\text{нКл} }{ 5\,\text{кВ} } = 5{,}0\,\text{пФ}.
    \text{ Заряды обкладок: $Q$ и $-Q$}$
}
\vspace{120pt}

\tasknumber{2}\task{
    На конденсаторе указано: $C = 100\,\text{пФ}$, $V = 400\,\text{кВ}$.
    Удастся ли его использовать для накопления заряда $q = 30\,\text{нКл}$?
}
\answer{
    $q' = CV = 100\,\text{пФ} \cdot 400\,\text{кВ} = 40000\,\text{нКл}
    \implies q' \ge q \implies \text{удастся}$
}
\vspace{120pt}

\tasknumber{3}\task{
    Как и во сколько раз изменится ёмкость плоского конденсатора при уменьшении площади пластин в 4 раз
    и уменьшении расстояния между ними в 3 раз?
}
\answer{
    $\frac{C'}{C}
        = \frac{\eps_0\eps \frac S4}{\frac d3} \Big/ \frac{\eps_0\eps S}{d}
        = \frac{ 3 }{ 4 } < 1 \implies \text{уменьшится в $\frac43$ раз}
    $
}
\vspace{120pt}

\tasknumber{4}\task{
    Электрическая ёмкость конденсатора равна $С = 400\,\text{пФ}$,
    при этом ему сообщён заряд $q = 500\,\text{нКл}$.
    Какова энергия заряженного конденсатора?
}
\answer{
    $W
        = \frac{ q^2 }{ 2С }
        = \frac{ \sqr{ 500\,\text{нКл} } }{ 2 \cdot 400\,\text{пФ} }
        = 312{,}5\,\text{мкДж}
    $
}
\vspace{120pt}

\tasknumber{5}\task{
    Два конденсатора ёмкостей $С_1 = 30\,\text{нФ}$ и $С_2 = 60\,\text{нФ}$ последовательно подключают
    к источнику напряжения $V = 200\,\text{В}$ (см.
    рис.).
    Определите заряды каждого из конденсаторов.
}
\answer{
    $Q_1
        = Q_2
        = CV
        = \frac{V}{\frac1{C_1} + \frac1{C_2}}
        = \frac{C_1C_2V}{C_1 + C_2}
        = \frac{
            30\,\text{нФ} \cdot 60\,\text{нФ} \cdot 200\,\text{В}
        }{
            30\,\text{нФ} + 60\,\text{нФ}
        }
        = 4000{,}0\,\text{нКл}
    $
}
\newpage

\addpersonalvariant{Алина Яшина}


\tasknumber{1}\task{
    Определите ёмкость конденсатора, если при его зарядке до напряжения
    $V = 5\,\text{кВ}$ он приобретает заряд $q = 4\,\text{нКл}$.
    Чему при этом равны заряды обкладок конденсатора (сделайте рисунок)?
}
\answer{
    $q = CV \implies
    C = \frac{ q }{ V } = \frac{ 4\,\text{нКл} }{ 5\,\text{кВ} } = 0{,}8\,\text{пФ}.
    \text{ Заряды обкладок: $q$ и $-q$}$
}
\vspace{120pt}

\tasknumber{2}\task{
    На конденсаторе указано: $C = 80\,\text{пФ}$, $U = 400\,\text{кВ}$.
    Удастся ли его использовать для накопления заряда $Q = 50\,\text{нКл}$?
}
\answer{
    $Q' = CU = 80\,\text{пФ} \cdot 400\,\text{кВ} = 32000\,\text{нКл}
    \implies Q' \ge Q \implies \text{удастся}$
}
\vspace{120pt}

\tasknumber{3}\task{
    Как и во сколько раз изменится ёмкость плоского конденсатора при уменьшении площади пластин в 8 раз
    и уменьшении расстояния между ними в 4 раз?
}
\answer{
    $\frac{C'}{C}
        = \frac{\eps_0\eps \frac S8}{\frac d4} \Big/ \frac{\eps_0\eps S}{d}
        = \frac{ 4 }{ 8 } < 1 \implies \text{уменьшится в $\frac21$ раз}
    $
}
\vspace{120pt}

\tasknumber{4}\task{
    Электрическая ёмкость конденсатора равна $С = 600\,\text{пФ}$,
    при этом ему сообщён заряд $q = 300\,\text{нКл}$.
    Какова энергия заряженного конденсатора?
}
\answer{
    $W
        = \frac{ q^2 }{ 2С }
        = \frac{ \sqr{ 300\,\text{нКл} } }{ 2 \cdot 600\,\text{пФ} }
        = 75{,}0\,\text{мкДж}
    $
}
\vspace{120pt}

\tasknumber{5}\task{
    Два конденсатора ёмкостей $С_1 = 40\,\text{нФ}$ и $С_2 = 20\,\text{нФ}$ последовательно подключают
    к источнику напряжения $V = 150\,\text{В}$ (см.
    рис.).
    Определите заряды каждого из конденсаторов.
}
\answer{
    $Q_1
        = Q_2
        = CV
        = \frac{V}{\frac1{C_1} + \frac1{C_2}}
        = \frac{C_1C_2V}{C_1 + C_2}
        = \frac{
            40\,\text{нФ} \cdot 20\,\text{нФ} \cdot 150\,\text{В}
        }{
            40\,\text{нФ} + 20\,\text{нФ}
        }
        = 2000{,}0\,\text{нКл}
    $
}

\end{document}
% autogenerated
