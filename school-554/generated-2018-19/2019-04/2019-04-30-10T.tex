\setdate{30~апреля~2019}
\setclass{10«Т»}

\addpersonalvariant{Михаил Бурмистров}

\tasknumber{1}%
\task{%
    В однородном электрическом поле напряжённостью $E = 4\,\frac{\text{кВ}}{\text{м}}$
    переместили заряд $q = 25\,\text{нКл}$ в направлении силовой линии
    на $d = 4\,\text{см}$.
    Определите
    \begin{itemize}
        \item работу поля,
        \item изменение потенциальной энергии заряда.
        % \item напряжение между начальной и конечной точками перемещения.
    \end{itemize}
}
\answer{%
    \begin{align*}
    A &= F \cdot d \cdot \cos \alpha = Eq \cdot d \cdot 1 = Eqd = 4\,\frac{\text{кВ}}{\text{м}} \cdot 25\,\text{нКл} \cdot 4\,\text{см} = 4\,\text{мкДж}, \\
    \Delta E_\text{пот.} &= -A = -4\,\text{мкДж}
    \end{align*}
}
\solutionspace{80pt}

\tasknumber{2}%
\task{%
    Напряжение между двумя точками, лежащими на одной линии напряжённости
    однородного электрического поля, равно $U = 3\,\text{кВ}$.
    Расстояние между точками $l = 30\,\text{см}$.
    Какова напряжённость этого поля?
}
\answer{%
    $
        E_x = -\frac{\Delta \varphi}{\Delta x} \implies
        E = \frac{U}{l} = \frac{3\,\text{кВ}}{30\,\text{см}} = 10\,\frac{\text{кВ}}{\text{м}}.
    $
}
\solutionspace{40pt}

\tasknumber{3}%
\task{%
    Найти напряжение между точками $A$ и $B$ в однородном электрическом поле
    (см.
    рис.
    на доске), если $AB=l = 8\,\text{см}$, $\alpha=45^\circ$,
    $E = 60\,\frac{\text{В}}{\text{м}}$.
    Потенциал какой из точек $A$ и $B$ больше?
}
\solutionspace{120pt}

\tasknumber{4}%
\task{%
    При какой скорости нейтрона его кинетическая энергия равна $E_\text{к} = 8\,\text{эВ}$?
}
\solutionspace{40pt}

\tasknumber{5}%
\task{%
    Электрон $e^-$ вылетает из точки, потенциал которой $\varphi = 400\,\text{В}$,
    со скоростью $v = 6000000\,\frac{\text{м}}{\text{с}}$ параллельно линиям напряжённости однородного электрического поля.
    % Будет поле его ускорять или тормозить?
    В некоторой точке частица остановилась.
    Каков потенциал этой точки?
    Вдоль и против поля влетела изначально частица?
}
\answer{%
    \begin{align*}
    A_\text{внешних сил} &= \Delta E_\text{кин.} \implies A_\text{эл.
    поля} = 0 - \frac{mv^2}2.
    \\
    A_\text{эл.
    поля} &= q(\varphi_1 - \varphi_2) \implies\varphi_2 = \varphi_1 - \frac{A_\text{эл.
    поля}}q = \varphi_1 - \frac{- \frac{mv^2}2}q = \varphi_1 + \frac{mv^2}{2q} =  \\
    &= 400\,\text{В} + \frac{9{,}1 \cdot 10^{-31}\,\text{кг} \cdot \sqr{6000000\,\frac{\text{м}}{\text{с}}}}{2  * (-1)  \cdot 1{,}6 \cdot 10^{-19}\,\text{Кл}} \approx 297{,}6\,\text{В}.
    \end{align*}
}

\variantsplitter

\addpersonalvariant{Гагик Аракелян}

\tasknumber{1}%
\task{%
    В однородном электрическом поле напряжённостью $E = 20\,\frac{\text{кВ}}{\text{м}}$
    переместили заряд $Q = -40\,\text{нКл}$ в направлении силовой линии
    на $l = 10\,\text{см}$.
    Определите
    \begin{itemize}
        \item работу поля,
        \item изменение потенциальной энергии заряда.
        % \item напряжение между начальной и конечной точками перемещения.
    \end{itemize}
}
\answer{%
    \begin{align*}
    A &= F \cdot l \cdot \cos \alpha = EQ \cdot l \cdot 1 = EQl = 20\,\frac{\text{кВ}}{\text{м}} \cdot -40\,\text{нКл} \cdot 10\,\text{см} = -80\,\text{мкДж}, \\
    \Delta E_\text{пот.} &= -A = 80\,\text{мкДж}
    \end{align*}
}
\solutionspace{80pt}

\tasknumber{2}%
\task{%
    Напряжение между двумя точками, лежащими на одной линии напряжённости
    однородного электрического поля, равно $V = 4\,\text{кВ}$.
    Расстояние между точками $d = 40\,\text{см}$.
    Какова напряжённость этого поля?
}
\answer{%
    $
        E_x = -\frac{\Delta \varphi}{\Delta x} \implies
        E = \frac{V}{d} = \frac{4\,\text{кВ}}{40\,\text{см}} = 10\,\frac{\text{кВ}}{\text{м}}.
    $
}
\solutionspace{40pt}

\tasknumber{3}%
\task{%
    Найти напряжение между точками $A$ и $B$ в однородном электрическом поле
    (см.
    рис.
    на доске), если $AB=d = 4\,\text{см}$, $\alpha=30^\circ$,
    $E = 50\,\frac{\text{В}}{\text{м}}$.
    Потенциал какой из точек $A$ и $B$ больше?
}
\solutionspace{120pt}

\tasknumber{4}%
\task{%
    При какой скорости нейтрона его кинетическая энергия равна $E_\text{к} = 20\,\text{эВ}$?
}
\solutionspace{40pt}

\tasknumber{5}%
\task{%
    Позитрон $e^+$ вылетает из точки, потенциал которой $\varphi = 800\,\text{В}$,
    со скоростью $v = 10000000\,\frac{\text{м}}{\text{с}}$ параллельно линиям напряжённости однородного электрического поля.
    % Будет поле его ускорять или тормозить?
    В некоторой точке частица остановилась.
    Каков потенциал этой точки?
    Вдоль и против поля влетела изначально частица?
}
\answer{%
    \begin{align*}
    A_\text{внешних сил} &= \Delta E_\text{кин.} \implies A_\text{эл.
    поля} = 0 - \frac{mv^2}2.
    \\
    A_\text{эл.
    поля} &= q(\varphi_1 - \varphi_2) \implies\varphi_2 = \varphi_1 - \frac{A_\text{эл.
    поля}}q = \varphi_1 - \frac{- \frac{mv^2}2}q = \varphi_1 + \frac{mv^2}{2q} =  \\
    &= 800\,\text{В} + \frac{9{,}1 \cdot 10^{-31}\,\text{кг} \cdot \sqr{10000000\,\frac{\text{м}}{\text{с}}}}{2  \cdot 1{,}6 \cdot 10^{-19}\,\text{Кл}} \approx 1084{,}4\,\text{В}.
    \end{align*}
}

\variantsplitter

\addpersonalvariant{Ирен Аракелян}

\tasknumber{1}%
\task{%
    В однородном электрическом поле напряжённостью $E = 2\,\frac{\text{кВ}}{\text{м}}$
    переместили заряд $Q = 25\,\text{нКл}$ в направлении силовой линии
    на $d = 4\,\text{см}$.
    Определите
    \begin{itemize}
        \item работу поля,
        \item изменение потенциальной энергии заряда.
        % \item напряжение между начальной и конечной точками перемещения.
    \end{itemize}
}
\answer{%
    \begin{align*}
    A &= F \cdot d \cdot \cos \alpha = EQ \cdot d \cdot 1 = EQd = 2\,\frac{\text{кВ}}{\text{м}} \cdot 25\,\text{нКл} \cdot 4\,\text{см} = 2\,\text{мкДж}, \\
    \Delta E_\text{пот.} &= -A = -2\,\text{мкДж}
    \end{align*}
}
\solutionspace{80pt}

\tasknumber{2}%
\task{%
    Напряжение между двумя точками, лежащими на одной линии напряжённости
    однородного электрического поля, равно $U = 3\,\text{кВ}$.
    Расстояние между точками $r = 40\,\text{см}$.
    Какова напряжённость этого поля?
}
\answer{%
    $
        E_x = -\frac{\Delta \varphi}{\Delta x} \implies
        E = \frac{U}{r} = \frac{3\,\text{кВ}}{40\,\text{см}} = 7{,}5\,\frac{\text{кВ}}{\text{м}}.
    $
}
\solutionspace{40pt}

\tasknumber{3}%
\task{%
    Найти напряжение между точками $A$ и $B$ в однородном электрическом поле
    (см.
    рис.
    на доске), если $AB=r = 8\,\text{см}$, $\alpha=30^\circ$,
    $E = 100\,\frac{\text{В}}{\text{м}}$.
    Потенциал какой из точек $A$ и $B$ больше?
}
\solutionspace{120pt}

\tasknumber{4}%
\task{%
    При какой скорости протона его кинетическая энергия равна $E_\text{к} = 200\,\text{эВ}$?
}
\solutionspace{40pt}

\tasknumber{5}%
\task{%
    Электрон $e^-$ вылетает из точки, потенциал которой $\varphi = 400\,\text{В}$,
    со скоростью $v = 6000000\,\frac{\text{м}}{\text{с}}$ параллельно линиям напряжённости однородного электрического поля.
    % Будет поле его ускорять или тормозить?
    В некоторой точке частица остановилась.
    Каков потенциал этой точки?
    Вдоль и против поля влетела изначально частица?
}
\answer{%
    \begin{align*}
    A_\text{внешних сил} &= \Delta E_\text{кин.} \implies A_\text{эл.
    поля} = 0 - \frac{mv^2}2.
    \\
    A_\text{эл.
    поля} &= q(\varphi_1 - \varphi_2) \implies\varphi_2 = \varphi_1 - \frac{A_\text{эл.
    поля}}q = \varphi_1 - \frac{- \frac{mv^2}2}q = \varphi_1 + \frac{mv^2}{2q} =  \\
    &= 400\,\text{В} + \frac{9{,}1 \cdot 10^{-31}\,\text{кг} \cdot \sqr{6000000\,\frac{\text{м}}{\text{с}}}}{2  * (-1)  \cdot 1{,}6 \cdot 10^{-19}\,\text{Кл}} \approx 297{,}6\,\text{В}.
    \end{align*}
}

\variantsplitter

\addpersonalvariant{Сабина Асадуллаева}

\tasknumber{1}%
\task{%
    В однородном электрическом поле напряжённостью $E = 2\,\frac{\text{кВ}}{\text{м}}$
    переместили заряд $Q = -10\,\text{нКл}$ в направлении силовой линии
    на $r = 2\,\text{см}$.
    Определите
    \begin{itemize}
        \item работу поля,
        \item изменение потенциальной энергии заряда.
        % \item напряжение между начальной и конечной точками перемещения.
    \end{itemize}
}
\answer{%
    \begin{align*}
    A &= F \cdot r \cdot \cos \alpha = EQ \cdot r \cdot 1 = EQr = 2\,\frac{\text{кВ}}{\text{м}} \cdot -10\,\text{нКл} \cdot 2\,\text{см} = -0{,}4\,\text{мкДж}, \\
    \Delta E_\text{пот.} &= -A = 0{,}4\,\text{мкДж}
    \end{align*}
}
\solutionspace{80pt}

\tasknumber{2}%
\task{%
    Напряжение между двумя точками, лежащими на одной линии напряжённости
    однородного электрического поля, равно $U = 6\,\text{кВ}$.
    Расстояние между точками $d = 10\,\text{см}$.
    Какова напряжённость этого поля?
}
\answer{%
    $
        E_x = -\frac{\Delta \varphi}{\Delta x} \implies
        E = \frac{U}{d} = \frac{6\,\text{кВ}}{10\,\text{см}} = 60\,\frac{\text{кВ}}{\text{м}}.
    $
}
\solutionspace{40pt}

\tasknumber{3}%
\task{%
    Найти напряжение между точками $A$ и $B$ в однородном электрическом поле
    (см.
    рис.
    на доске), если $AB=r = 4\,\text{см}$, $\alpha=30^\circ$,
    $E = 60\,\frac{\text{В}}{\text{м}}$.
    Потенциал какой из точек $A$ и $B$ больше?
}
\solutionspace{120pt}

\tasknumber{4}%
\task{%
    При какой скорости электрона его кинетическая энергия равна $E_\text{к} = 600\,\text{эВ}$?
}
\solutionspace{40pt}

\tasknumber{5}%
\task{%
    Позитрон $e^+$ вылетает из точки, потенциал которой $\varphi = 400\,\text{В}$,
    со скоростью $v = 12000000\,\frac{\text{м}}{\text{с}}$ параллельно линиям напряжённости однородного электрического поля.
    % Будет поле его ускорять или тормозить?
    В некоторой точке частица остановилась.
    Каков потенциал этой точки?
    Вдоль и против поля влетела изначально частица?
}
\answer{%
    \begin{align*}
    A_\text{внешних сил} &= \Delta E_\text{кин.} \implies A_\text{эл.
    поля} = 0 - \frac{mv^2}2.
    \\
    A_\text{эл.
    поля} &= q(\varphi_1 - \varphi_2) \implies\varphi_2 = \varphi_1 - \frac{A_\text{эл.
    поля}}q = \varphi_1 - \frac{- \frac{mv^2}2}q = \varphi_1 + \frac{mv^2}{2q} =  \\
    &= 400\,\text{В} + \frac{9{,}1 \cdot 10^{-31}\,\text{кг} \cdot \sqr{12000000\,\frac{\text{м}}{\text{с}}}}{2  \cdot 1{,}6 \cdot 10^{-19}\,\text{Кл}} \approx 809{,}5\,\text{В}.
    \end{align*}
}

\variantsplitter

\addpersonalvariant{Вероника Битерякова}

\tasknumber{1}%
\task{%
    В однородном электрическом поле напряжённостью $E = 20\,\frac{\text{кВ}}{\text{м}}$
    переместили заряд $q = 40\,\text{нКл}$ в направлении силовой линии
    на $r = 4\,\text{см}$.
    Определите
    \begin{itemize}
        \item работу поля,
        \item изменение потенциальной энергии заряда.
        % \item напряжение между начальной и конечной точками перемещения.
    \end{itemize}
}
\answer{%
    \begin{align*}
    A &= F \cdot r \cdot \cos \alpha = Eq \cdot r \cdot 1 = Eqr = 20\,\frac{\text{кВ}}{\text{м}} \cdot 40\,\text{нКл} \cdot 4\,\text{см} = 32\,\text{мкДж}, \\
    \Delta E_\text{пот.} &= -A = -32\,\text{мкДж}
    \end{align*}
}
\solutionspace{80pt}

\tasknumber{2}%
\task{%
    Напряжение между двумя точками, лежащими на одной линии напряжённости
    однородного электрического поля, равно $U = 2\,\text{кВ}$.
    Расстояние между точками $r = 20\,\text{см}$.
    Какова напряжённость этого поля?
}
\answer{%
    $
        E_x = -\frac{\Delta \varphi}{\Delta x} \implies
        E = \frac{U}{r} = \frac{2\,\text{кВ}}{20\,\text{см}} = 10\,\frac{\text{кВ}}{\text{м}}.
    $
}
\solutionspace{40pt}

\tasknumber{3}%
\task{%
    Найти напряжение между точками $A$ и $B$ в однородном электрическом поле
    (см.
    рис.
    на доске), если $AB=l = 6\,\text{см}$, $\varphi=30^\circ$,
    $E = 50\,\frac{\text{В}}{\text{м}}$.
    Потенциал какой из точек $A$ и $B$ больше?
}
\solutionspace{120pt}

\tasknumber{4}%
\task{%
    При какой скорости электрона его кинетическая энергия равна $E_\text{к} = 200\,\text{эВ}$?
}
\solutionspace{40pt}

\tasknumber{5}%
\task{%
    Электрон $e^-$ вылетает из точки, потенциал которой $\varphi = 600\,\text{В}$,
    со скоростью $v = 4000000\,\frac{\text{м}}{\text{с}}$ параллельно линиям напряжённости однородного электрического поля.
    % Будет поле его ускорять или тормозить?
    В некоторой точке частица остановилась.
    Каков потенциал этой точки?
    Вдоль и против поля влетела изначально частица?
}
\answer{%
    \begin{align*}
    A_\text{внешних сил} &= \Delta E_\text{кин.} \implies A_\text{эл.
    поля} = 0 - \frac{mv^2}2.
    \\
    A_\text{эл.
    поля} &= q(\varphi_1 - \varphi_2) \implies\varphi_2 = \varphi_1 - \frac{A_\text{эл.
    поля}}q = \varphi_1 - \frac{- \frac{mv^2}2}q = \varphi_1 + \frac{mv^2}{2q} =  \\
    &= 600\,\text{В} + \frac{9{,}1 \cdot 10^{-31}\,\text{кг} \cdot \sqr{4000000\,\frac{\text{м}}{\text{с}}}}{2  * (-1)  \cdot 1{,}6 \cdot 10^{-19}\,\text{Кл}} \approx 554{,}5\,\text{В}.
    \end{align*}
}

\variantsplitter

\addpersonalvariant{Юлия Буянова}

\tasknumber{1}%
\task{%
    В однородном электрическом поле напряжённостью $E = 2\,\frac{\text{кВ}}{\text{м}}$
    переместили заряд $q = -40\,\text{нКл}$ в направлении силовой линии
    на $r = 5\,\text{см}$.
    Определите
    \begin{itemize}
        \item работу поля,
        \item изменение потенциальной энергии заряда.
        % \item напряжение между начальной и конечной точками перемещения.
    \end{itemize}
}
\answer{%
    \begin{align*}
    A &= F \cdot r \cdot \cos \alpha = Eq \cdot r \cdot 1 = Eqr = 2\,\frac{\text{кВ}}{\text{м}} \cdot -40\,\text{нКл} \cdot 5\,\text{см} = -4\,\text{мкДж}, \\
    \Delta E_\text{пот.} &= -A = 4\,\text{мкДж}
    \end{align*}
}
\solutionspace{80pt}

\tasknumber{2}%
\task{%
    Напряжение между двумя точками, лежащими на одной линии напряжённости
    однородного электрического поля, равно $U = 2\,\text{кВ}$.
    Расстояние между точками $d = 20\,\text{см}$.
    Какова напряжённость этого поля?
}
\answer{%
    $
        E_x = -\frac{\Delta \varphi}{\Delta x} \implies
        E = \frac{U}{d} = \frac{2\,\text{кВ}}{20\,\text{см}} = 10\,\frac{\text{кВ}}{\text{м}}.
    $
}
\solutionspace{40pt}

\tasknumber{3}%
\task{%
    Найти напряжение между точками $A$ и $B$ в однородном электрическом поле
    (см.
    рис.
    на доске), если $AB=d = 4\,\text{см}$, $\alpha=45^\circ$,
    $E = 30\,\frac{\text{В}}{\text{м}}$.
    Потенциал какой из точек $A$ и $B$ больше?
}
\solutionspace{120pt}

\tasknumber{4}%
\task{%
    При какой скорости протона его кинетическая энергия равна $E_\text{к} = 200\,\text{эВ}$?
}
\solutionspace{40pt}

\tasknumber{5}%
\task{%
    Позитрон $e^+$ вылетает из точки, потенциал которой $\varphi = 800\,\text{В}$,
    со скоростью $v = 10000000\,\frac{\text{м}}{\text{с}}$ параллельно линиям напряжённости однородного электрического поля.
    % Будет поле его ускорять или тормозить?
    В некоторой точке частица остановилась.
    Каков потенциал этой точки?
    Вдоль и против поля влетела изначально частица?
}
\answer{%
    \begin{align*}
    A_\text{внешних сил} &= \Delta E_\text{кин.} \implies A_\text{эл.
    поля} = 0 - \frac{mv^2}2.
    \\
    A_\text{эл.
    поля} &= q(\varphi_1 - \varphi_2) \implies\varphi_2 = \varphi_1 - \frac{A_\text{эл.
    поля}}q = \varphi_1 - \frac{- \frac{mv^2}2}q = \varphi_1 + \frac{mv^2}{2q} =  \\
    &= 800\,\text{В} + \frac{9{,}1 \cdot 10^{-31}\,\text{кг} \cdot \sqr{10000000\,\frac{\text{м}}{\text{с}}}}{2  \cdot 1{,}6 \cdot 10^{-19}\,\text{Кл}} \approx 1084{,}4\,\text{В}.
    \end{align*}
}

\variantsplitter

\addpersonalvariant{Пелагея Вдовина}

\tasknumber{1}%
\task{%
    В однородном электрическом поле напряжённостью $E = 2\,\frac{\text{кВ}}{\text{м}}$
    переместили заряд $q = -25\,\text{нКл}$ в направлении силовой линии
    на $d = 2\,\text{см}$.
    Определите
    \begin{itemize}
        \item работу поля,
        \item изменение потенциальной энергии заряда.
        % \item напряжение между начальной и конечной точками перемещения.
    \end{itemize}
}
\answer{%
    \begin{align*}
    A &= F \cdot d \cdot \cos \alpha = Eq \cdot d \cdot 1 = Eqd = 2\,\frac{\text{кВ}}{\text{м}} \cdot -25\,\text{нКл} \cdot 2\,\text{см} = -1\,\text{мкДж}, \\
    \Delta E_\text{пот.} &= -A = 1\,\text{мкДж}
    \end{align*}
}
\solutionspace{80pt}

\tasknumber{2}%
\task{%
    Напряжение между двумя точками, лежащими на одной линии напряжённости
    однородного электрического поля, равно $U = 4\,\text{кВ}$.
    Расстояние между точками $d = 30\,\text{см}$.
    Какова напряжённость этого поля?
}
\answer{%
    $
        E_x = -\frac{\Delta \varphi}{\Delta x} \implies
        E = \frac{U}{d} = \frac{4\,\text{кВ}}{30\,\text{см}} = 13{,}3\,\frac{\text{кВ}}{\text{м}}.
    $
}
\solutionspace{40pt}

\tasknumber{3}%
\task{%
    Найти напряжение между точками $A$ и $B$ в однородном электрическом поле
    (см.
    рис.
    на доске), если $AB=r = 4\,\text{см}$, $\varphi=30^\circ$,
    $E = 100\,\frac{\text{В}}{\text{м}}$.
    Потенциал какой из точек $A$ и $B$ больше?
}
\solutionspace{120pt}

\tasknumber{4}%
\task{%
    При какой скорости протона его кинетическая энергия равна $E_\text{к} = 600\,\text{эВ}$?
}
\solutionspace{40pt}

\tasknumber{5}%
\task{%
    Позитрон $e^+$ вылетает из точки, потенциал которой $\varphi = 600\,\text{В}$,
    со скоростью $v = 12000000\,\frac{\text{м}}{\text{с}}$ параллельно линиям напряжённости однородного электрического поля.
    % Будет поле его ускорять или тормозить?
    В некоторой точке частица остановилась.
    Каков потенциал этой точки?
    Вдоль и против поля влетела изначально частица?
}
\answer{%
    \begin{align*}
    A_\text{внешних сил} &= \Delta E_\text{кин.} \implies A_\text{эл.
    поля} = 0 - \frac{mv^2}2.
    \\
    A_\text{эл.
    поля} &= q(\varphi_1 - \varphi_2) \implies\varphi_2 = \varphi_1 - \frac{A_\text{эл.
    поля}}q = \varphi_1 - \frac{- \frac{mv^2}2}q = \varphi_1 + \frac{mv^2}{2q} =  \\
    &= 600\,\text{В} + \frac{9{,}1 \cdot 10^{-31}\,\text{кг} \cdot \sqr{12000000\,\frac{\text{м}}{\text{с}}}}{2  \cdot 1{,}6 \cdot 10^{-19}\,\text{Кл}} \approx 1009{,}5\,\text{В}.
    \end{align*}
}

\variantsplitter

\addpersonalvariant{Леонид Викторов}

\tasknumber{1}%
\task{%
    В однородном электрическом поле напряжённостью $E = 2\,\frac{\text{кВ}}{\text{м}}$
    переместили заряд $Q = 25\,\text{нКл}$ в направлении силовой линии
    на $l = 2\,\text{см}$.
    Определите
    \begin{itemize}
        \item работу поля,
        \item изменение потенциальной энергии заряда.
        % \item напряжение между начальной и конечной точками перемещения.
    \end{itemize}
}
\answer{%
    \begin{align*}
    A &= F \cdot l \cdot \cos \alpha = EQ \cdot l \cdot 1 = EQl = 2\,\frac{\text{кВ}}{\text{м}} \cdot 25\,\text{нКл} \cdot 2\,\text{см} = 1\,\text{мкДж}, \\
    \Delta E_\text{пот.} &= -A = -1\,\text{мкДж}
    \end{align*}
}
\solutionspace{80pt}

\tasknumber{2}%
\task{%
    Напряжение между двумя точками, лежащими на одной линии напряжённости
    однородного электрического поля, равно $U = 4\,\text{кВ}$.
    Расстояние между точками $r = 20\,\text{см}$.
    Какова напряжённость этого поля?
}
\answer{%
    $
        E_x = -\frac{\Delta \varphi}{\Delta x} \implies
        E = \frac{U}{r} = \frac{4\,\text{кВ}}{20\,\text{см}} = 20\,\frac{\text{кВ}}{\text{м}}.
    $
}
\solutionspace{40pt}

\tasknumber{3}%
\task{%
    Найти напряжение между точками $A$ и $B$ в однородном электрическом поле
    (см.
    рис.
    на доске), если $AB=r = 6\,\text{см}$, $\alpha=45^\circ$,
    $E = 100\,\frac{\text{В}}{\text{м}}$.
    Потенциал какой из точек $A$ и $B$ больше?
}
\solutionspace{120pt}

\tasknumber{4}%
\task{%
    При какой скорости протона его кинетическая энергия равна $E_\text{к} = 50\,\text{эВ}$?
}
\solutionspace{40pt}

\tasknumber{5}%
\task{%
    Позитрон $e^+$ вылетает из точки, потенциал которой $\varphi = 600\,\text{В}$,
    со скоростью $v = 10000000\,\frac{\text{м}}{\text{с}}$ параллельно линиям напряжённости однородного электрического поля.
    % Будет поле его ускорять или тормозить?
    В некоторой точке частица остановилась.
    Каков потенциал этой точки?
    Вдоль и против поля влетела изначально частица?
}
\answer{%
    \begin{align*}
    A_\text{внешних сил} &= \Delta E_\text{кин.} \implies A_\text{эл.
    поля} = 0 - \frac{mv^2}2.
    \\
    A_\text{эл.
    поля} &= q(\varphi_1 - \varphi_2) \implies\varphi_2 = \varphi_1 - \frac{A_\text{эл.
    поля}}q = \varphi_1 - \frac{- \frac{mv^2}2}q = \varphi_1 + \frac{mv^2}{2q} =  \\
    &= 600\,\text{В} + \frac{9{,}1 \cdot 10^{-31}\,\text{кг} \cdot \sqr{10000000\,\frac{\text{м}}{\text{с}}}}{2  \cdot 1{,}6 \cdot 10^{-19}\,\text{Кл}} \approx 884{,}4\,\text{В}.
    \end{align*}
}

\variantsplitter

\addpersonalvariant{Фёдор Гнутов}

\tasknumber{1}%
\task{%
    В однородном электрическом поле напряжённостью $E = 20\,\frac{\text{кВ}}{\text{м}}$
    переместили заряд $q = 10\,\text{нКл}$ в направлении силовой линии
    на $d = 2\,\text{см}$.
    Определите
    \begin{itemize}
        \item работу поля,
        \item изменение потенциальной энергии заряда.
        % \item напряжение между начальной и конечной точками перемещения.
    \end{itemize}
}
\answer{%
    \begin{align*}
    A &= F \cdot d \cdot \cos \alpha = Eq \cdot d \cdot 1 = Eqd = 20\,\frac{\text{кВ}}{\text{м}} \cdot 10\,\text{нКл} \cdot 2\,\text{см} = 4\,\text{мкДж}, \\
    \Delta E_\text{пот.} &= -A = -4\,\text{мкДж}
    \end{align*}
}
\solutionspace{80pt}

\tasknumber{2}%
\task{%
    Напряжение между двумя точками, лежащими на одной линии напряжённости
    однородного электрического поля, равно $U = 4\,\text{кВ}$.
    Расстояние между точками $l = 30\,\text{см}$.
    Какова напряжённость этого поля?
}
\answer{%
    $
        E_x = -\frac{\Delta \varphi}{\Delta x} \implies
        E = \frac{U}{l} = \frac{4\,\text{кВ}}{30\,\text{см}} = 13{,}3\,\frac{\text{кВ}}{\text{м}}.
    $
}
\solutionspace{40pt}

\tasknumber{3}%
\task{%
    Найти напряжение между точками $A$ и $B$ в однородном электрическом поле
    (см.
    рис.
    на доске), если $AB=l = 10\,\text{см}$, $\alpha=45^\circ$,
    $E = 50\,\frac{\text{В}}{\text{м}}$.
    Потенциал какой из точек $A$ и $B$ больше?
}
\solutionspace{120pt}

\tasknumber{4}%
\task{%
    При какой скорости нейтрона его кинетическая энергия равна $E_\text{к} = 200\,\text{эВ}$?
}
\solutionspace{40pt}

\tasknumber{5}%
\task{%
    Электрон $e^-$ вылетает из точки, потенциал которой $\varphi = 800\,\text{В}$,
    со скоростью $v = 12000000\,\frac{\text{м}}{\text{с}}$ параллельно линиям напряжённости однородного электрического поля.
    % Будет поле его ускорять или тормозить?
    В некоторой точке частица остановилась.
    Каков потенциал этой точки?
    Вдоль и против поля влетела изначально частица?
}
\answer{%
    \begin{align*}
    A_\text{внешних сил} &= \Delta E_\text{кин.} \implies A_\text{эл.
    поля} = 0 - \frac{mv^2}2.
    \\
    A_\text{эл.
    поля} &= q(\varphi_1 - \varphi_2) \implies\varphi_2 = \varphi_1 - \frac{A_\text{эл.
    поля}}q = \varphi_1 - \frac{- \frac{mv^2}2}q = \varphi_1 + \frac{mv^2}{2q} =  \\
    &= 800\,\text{В} + \frac{9{,}1 \cdot 10^{-31}\,\text{кг} \cdot \sqr{12000000\,\frac{\text{м}}{\text{с}}}}{2  * (-1)  \cdot 1{,}6 \cdot 10^{-19}\,\text{Кл}} \approx 390{,}5\,\text{В}.
    \end{align*}
}

\variantsplitter

\addpersonalvariant{Илья Гримберг}

\tasknumber{1}%
\task{%
    В однородном электрическом поле напряжённостью $E = 4\,\frac{\text{кВ}}{\text{м}}$
    переместили заряд $q = -10\,\text{нКл}$ в направлении силовой линии
    на $r = 5\,\text{см}$.
    Определите
    \begin{itemize}
        \item работу поля,
        \item изменение потенциальной энергии заряда.
        % \item напряжение между начальной и конечной точками перемещения.
    \end{itemize}
}
\answer{%
    \begin{align*}
    A &= F \cdot r \cdot \cos \alpha = Eq \cdot r \cdot 1 = Eqr = 4\,\frac{\text{кВ}}{\text{м}} \cdot -10\,\text{нКл} \cdot 5\,\text{см} = -2\,\text{мкДж}, \\
    \Delta E_\text{пот.} &= -A = 2\,\text{мкДж}
    \end{align*}
}
\solutionspace{80pt}

\tasknumber{2}%
\task{%
    Напряжение между двумя точками, лежащими на одной линии напряжённости
    однородного электрического поля, равно $V = 2\,\text{кВ}$.
    Расстояние между точками $r = 40\,\text{см}$.
    Какова напряжённость этого поля?
}
\answer{%
    $
        E_x = -\frac{\Delta \varphi}{\Delta x} \implies
        E = \frac{V}{r} = \frac{2\,\text{кВ}}{40\,\text{см}} = 5\,\frac{\text{кВ}}{\text{м}}.
    $
}
\solutionspace{40pt}

\tasknumber{3}%
\task{%
    Найти напряжение между точками $A$ и $B$ в однородном электрическом поле
    (см.
    рис.
    на доске), если $AB=l = 8\,\text{см}$, $\alpha=60^\circ$,
    $E = 30\,\frac{\text{В}}{\text{м}}$.
    Потенциал какой из точек $A$ и $B$ больше?
}
\solutionspace{120pt}

\tasknumber{4}%
\task{%
    При какой скорости протона его кинетическая энергия равна $E_\text{к} = 400\,\text{эВ}$?
}
\solutionspace{40pt}

\tasknumber{5}%
\task{%
    Позитрон $e^+$ вылетает из точки, потенциал которой $\varphi = 200\,\text{В}$,
    со скоростью $v = 10000000\,\frac{\text{м}}{\text{с}}$ параллельно линиям напряжённости однородного электрического поля.
    % Будет поле его ускорять или тормозить?
    В некоторой точке частица остановилась.
    Каков потенциал этой точки?
    Вдоль и против поля влетела изначально частица?
}
\answer{%
    \begin{align*}
    A_\text{внешних сил} &= \Delta E_\text{кин.} \implies A_\text{эл.
    поля} = 0 - \frac{mv^2}2.
    \\
    A_\text{эл.
    поля} &= q(\varphi_1 - \varphi_2) \implies\varphi_2 = \varphi_1 - \frac{A_\text{эл.
    поля}}q = \varphi_1 - \frac{- \frac{mv^2}2}q = \varphi_1 + \frac{mv^2}{2q} =  \\
    &= 200\,\text{В} + \frac{9{,}1 \cdot 10^{-31}\,\text{кг} \cdot \sqr{10000000\,\frac{\text{м}}{\text{с}}}}{2  \cdot 1{,}6 \cdot 10^{-19}\,\text{Кл}} \approx 484{,}4\,\text{В}.
    \end{align*}
}

\variantsplitter

\addpersonalvariant{Иван Гурьянов}

\tasknumber{1}%
\task{%
    В однородном электрическом поле напряжённостью $E = 4\,\frac{\text{кВ}}{\text{м}}$
    переместили заряд $q = -25\,\text{нКл}$ в направлении силовой линии
    на $r = 10\,\text{см}$.
    Определите
    \begin{itemize}
        \item работу поля,
        \item изменение потенциальной энергии заряда.
        % \item напряжение между начальной и конечной точками перемещения.
    \end{itemize}
}
\answer{%
    \begin{align*}
    A &= F \cdot r \cdot \cos \alpha = Eq \cdot r \cdot 1 = Eqr = 4\,\frac{\text{кВ}}{\text{м}} \cdot -25\,\text{нКл} \cdot 10\,\text{см} = -10\,\text{мкДж}, \\
    \Delta E_\text{пот.} &= -A = 10\,\text{мкДж}
    \end{align*}
}
\solutionspace{80pt}

\tasknumber{2}%
\task{%
    Напряжение между двумя точками, лежащими на одной линии напряжённости
    однородного электрического поля, равно $U = 3\,\text{кВ}$.
    Расстояние между точками $d = 30\,\text{см}$.
    Какова напряжённость этого поля?
}
\answer{%
    $
        E_x = -\frac{\Delta \varphi}{\Delta x} \implies
        E = \frac{U}{d} = \frac{3\,\text{кВ}}{30\,\text{см}} = 10\,\frac{\text{кВ}}{\text{м}}.
    $
}
\solutionspace{40pt}

\tasknumber{3}%
\task{%
    Найти напряжение между точками $A$ и $B$ в однородном электрическом поле
    (см.
    рис.
    на доске), если $AB=d = 8\,\text{см}$, $\alpha=60^\circ$,
    $E = 60\,\frac{\text{В}}{\text{м}}$.
    Потенциал какой из точек $A$ и $B$ больше?
}
\solutionspace{120pt}

\tasknumber{4}%
\task{%
    При какой скорости позитрона его кинетическая энергия равна $E_\text{к} = 20\,\text{эВ}$?
}
\solutionspace{40pt}

\tasknumber{5}%
\task{%
    Позитрон $e^+$ вылетает из точки, потенциал которой $\varphi = 800\,\text{В}$,
    со скоростью $v = 4000000\,\frac{\text{м}}{\text{с}}$ параллельно линиям напряжённости однородного электрического поля.
    % Будет поле его ускорять или тормозить?
    В некоторой точке частица остановилась.
    Каков потенциал этой точки?
    Вдоль и против поля влетела изначально частица?
}
\answer{%
    \begin{align*}
    A_\text{внешних сил} &= \Delta E_\text{кин.} \implies A_\text{эл.
    поля} = 0 - \frac{mv^2}2.
    \\
    A_\text{эл.
    поля} &= q(\varphi_1 - \varphi_2) \implies\varphi_2 = \varphi_1 - \frac{A_\text{эл.
    поля}}q = \varphi_1 - \frac{- \frac{mv^2}2}q = \varphi_1 + \frac{mv^2}{2q} =  \\
    &= 800\,\text{В} + \frac{9{,}1 \cdot 10^{-31}\,\text{кг} \cdot \sqr{4000000\,\frac{\text{м}}{\text{с}}}}{2  \cdot 1{,}6 \cdot 10^{-19}\,\text{Кл}} \approx 845{,}5\,\text{В}.
    \end{align*}
}

\variantsplitter

\addpersonalvariant{Артём Денежкин}

\tasknumber{1}%
\task{%
    В однородном электрическом поле напряжённостью $E = 4\,\frac{\text{кВ}}{\text{м}}$
    переместили заряд $q = 40\,\text{нКл}$ в направлении силовой линии
    на $r = 4\,\text{см}$.
    Определите
    \begin{itemize}
        \item работу поля,
        \item изменение потенциальной энергии заряда.
        % \item напряжение между начальной и конечной точками перемещения.
    \end{itemize}
}
\answer{%
    \begin{align*}
    A &= F \cdot r \cdot \cos \alpha = Eq \cdot r \cdot 1 = Eqr = 4\,\frac{\text{кВ}}{\text{м}} \cdot 40\,\text{нКл} \cdot 4\,\text{см} = 6{,}4\,\text{мкДж}, \\
    \Delta E_\text{пот.} &= -A = -6{,}4\,\text{мкДж}
    \end{align*}
}
\solutionspace{80pt}

\tasknumber{2}%
\task{%
    Напряжение между двумя точками, лежащими на одной линии напряжённости
    однородного электрического поля, равно $U = 2\,\text{кВ}$.
    Расстояние между точками $l = 10\,\text{см}$.
    Какова напряжённость этого поля?
}
\answer{%
    $
        E_x = -\frac{\Delta \varphi}{\Delta x} \implies
        E = \frac{U}{l} = \frac{2\,\text{кВ}}{10\,\text{см}} = 20\,\frac{\text{кВ}}{\text{м}}.
    $
}
\solutionspace{40pt}

\tasknumber{3}%
\task{%
    Найти напряжение между точками $A$ и $B$ в однородном электрическом поле
    (см.
    рис.
    на доске), если $AB=l = 6\,\text{см}$, $\varphi=45^\circ$,
    $E = 100\,\frac{\text{В}}{\text{м}}$.
    Потенциал какой из точек $A$ и $B$ больше?
}
\solutionspace{120pt}

\tasknumber{4}%
\task{%
    При какой скорости нейтрона его кинетическая энергия равна $E_\text{к} = 600\,\text{эВ}$?
}
\solutionspace{40pt}

\tasknumber{5}%
\task{%
    Электрон $e^-$ вылетает из точки, потенциал которой $\varphi = 1000\,\text{В}$,
    со скоростью $v = 4000000\,\frac{\text{м}}{\text{с}}$ параллельно линиям напряжённости однородного электрического поля.
    % Будет поле его ускорять или тормозить?
    В некоторой точке частица остановилась.
    Каков потенциал этой точки?
    Вдоль и против поля влетела изначально частица?
}
\answer{%
    \begin{align*}
    A_\text{внешних сил} &= \Delta E_\text{кин.} \implies A_\text{эл.
    поля} = 0 - \frac{mv^2}2.
    \\
    A_\text{эл.
    поля} &= q(\varphi_1 - \varphi_2) \implies\varphi_2 = \varphi_1 - \frac{A_\text{эл.
    поля}}q = \varphi_1 - \frac{- \frac{mv^2}2}q = \varphi_1 + \frac{mv^2}{2q} =  \\
    &= 1000\,\text{В} + \frac{9{,}1 \cdot 10^{-31}\,\text{кг} \cdot \sqr{4000000\,\frac{\text{м}}{\text{с}}}}{2  * (-1)  \cdot 1{,}6 \cdot 10^{-19}\,\text{Кл}} \approx 954{,}5\,\text{В}.
    \end{align*}
}

\variantsplitter

\addpersonalvariant{Виктор Жилин}

\tasknumber{1}%
\task{%
    В однородном электрическом поле напряжённостью $E = 4\,\frac{\text{кВ}}{\text{м}}$
    переместили заряд $q = 40\,\text{нКл}$ в направлении силовой линии
    на $d = 10\,\text{см}$.
    Определите
    \begin{itemize}
        \item работу поля,
        \item изменение потенциальной энергии заряда.
        % \item напряжение между начальной и конечной точками перемещения.
    \end{itemize}
}
\answer{%
    \begin{align*}
    A &= F \cdot d \cdot \cos \alpha = Eq \cdot d \cdot 1 = Eqd = 4\,\frac{\text{кВ}}{\text{м}} \cdot 40\,\text{нКл} \cdot 10\,\text{см} = 16\,\text{мкДж}, \\
    \Delta E_\text{пот.} &= -A = -16\,\text{мкДж}
    \end{align*}
}
\solutionspace{80pt}

\tasknumber{2}%
\task{%
    Напряжение между двумя точками, лежащими на одной линии напряжённости
    однородного электрического поля, равно $U = 2\,\text{кВ}$.
    Расстояние между точками $l = 30\,\text{см}$.
    Какова напряжённость этого поля?
}
\answer{%
    $
        E_x = -\frac{\Delta \varphi}{\Delta x} \implies
        E = \frac{U}{l} = \frac{2\,\text{кВ}}{30\,\text{см}} = 6{,}7\,\frac{\text{кВ}}{\text{м}}.
    $
}
\solutionspace{40pt}

\tasknumber{3}%
\task{%
    Найти напряжение между точками $A$ и $B$ в однородном электрическом поле
    (см.
    рис.
    на доске), если $AB=r = 6\,\text{см}$, $\alpha=45^\circ$,
    $E = 120\,\frac{\text{В}}{\text{м}}$.
    Потенциал какой из точек $A$ и $B$ больше?
}
\solutionspace{120pt}

\tasknumber{4}%
\task{%
    При какой скорости протона его кинетическая энергия равна $E_\text{к} = 600\,\text{эВ}$?
}
\solutionspace{40pt}

\tasknumber{5}%
\task{%
    Позитрон $e^+$ вылетает из точки, потенциал которой $\varphi = 400\,\text{В}$,
    со скоростью $v = 10000000\,\frac{\text{м}}{\text{с}}$ параллельно линиям напряжённости однородного электрического поля.
    % Будет поле его ускорять или тормозить?
    В некоторой точке частица остановилась.
    Каков потенциал этой точки?
    Вдоль и против поля влетела изначально частица?
}
\answer{%
    \begin{align*}
    A_\text{внешних сил} &= \Delta E_\text{кин.} \implies A_\text{эл.
    поля} = 0 - \frac{mv^2}2.
    \\
    A_\text{эл.
    поля} &= q(\varphi_1 - \varphi_2) \implies\varphi_2 = \varphi_1 - \frac{A_\text{эл.
    поля}}q = \varphi_1 - \frac{- \frac{mv^2}2}q = \varphi_1 + \frac{mv^2}{2q} =  \\
    &= 400\,\text{В} + \frac{9{,}1 \cdot 10^{-31}\,\text{кг} \cdot \sqr{10000000\,\frac{\text{м}}{\text{с}}}}{2  \cdot 1{,}6 \cdot 10^{-19}\,\text{Кл}} \approx 684{,}4\,\text{В}.
    \end{align*}
}

\variantsplitter

\addpersonalvariant{Дмитрий Иванов}

\tasknumber{1}%
\task{%
    В однородном электрическом поле напряжённостью $E = 4\,\frac{\text{кВ}}{\text{м}}$
    переместили заряд $Q = 40\,\text{нКл}$ в направлении силовой линии
    на $r = 10\,\text{см}$.
    Определите
    \begin{itemize}
        \item работу поля,
        \item изменение потенциальной энергии заряда.
        % \item напряжение между начальной и конечной точками перемещения.
    \end{itemize}
}
\answer{%
    \begin{align*}
    A &= F \cdot r \cdot \cos \alpha = EQ \cdot r \cdot 1 = EQr = 4\,\frac{\text{кВ}}{\text{м}} \cdot 40\,\text{нКл} \cdot 10\,\text{см} = 16\,\text{мкДж}, \\
    \Delta E_\text{пот.} &= -A = -16\,\text{мкДж}
    \end{align*}
}
\solutionspace{80pt}

\tasknumber{2}%
\task{%
    Напряжение между двумя точками, лежащими на одной линии напряжённости
    однородного электрического поля, равно $U = 6\,\text{кВ}$.
    Расстояние между точками $d = 20\,\text{см}$.
    Какова напряжённость этого поля?
}
\answer{%
    $
        E_x = -\frac{\Delta \varphi}{\Delta x} \implies
        E = \frac{U}{d} = \frac{6\,\text{кВ}}{20\,\text{см}} = 30\,\frac{\text{кВ}}{\text{м}}.
    $
}
\solutionspace{40pt}

\tasknumber{3}%
\task{%
    Найти напряжение между точками $A$ и $B$ в однородном электрическом поле
    (см.
    рис.
    на доске), если $AB=r = 8\,\text{см}$, $\alpha=30^\circ$,
    $E = 60\,\frac{\text{В}}{\text{м}}$.
    Потенциал какой из точек $A$ и $B$ больше?
}
\solutionspace{120pt}

\tasknumber{4}%
\task{%
    При какой скорости позитрона его кинетическая энергия равна $E_\text{к} = 30\,\text{эВ}$?
}
\solutionspace{40pt}

\tasknumber{5}%
\task{%
    Позитрон $e^+$ вылетает из точки, потенциал которой $\varphi = 1000\,\text{В}$,
    со скоростью $v = 10000000\,\frac{\text{м}}{\text{с}}$ параллельно линиям напряжённости однородного электрического поля.
    % Будет поле его ускорять или тормозить?
    В некоторой точке частица остановилась.
    Каков потенциал этой точки?
    Вдоль и против поля влетела изначально частица?
}
\answer{%
    \begin{align*}
    A_\text{внешних сил} &= \Delta E_\text{кин.} \implies A_\text{эл.
    поля} = 0 - \frac{mv^2}2.
    \\
    A_\text{эл.
    поля} &= q(\varphi_1 - \varphi_2) \implies\varphi_2 = \varphi_1 - \frac{A_\text{эл.
    поля}}q = \varphi_1 - \frac{- \frac{mv^2}2}q = \varphi_1 + \frac{mv^2}{2q} =  \\
    &= 1000\,\text{В} + \frac{9{,}1 \cdot 10^{-31}\,\text{кг} \cdot \sqr{10000000\,\frac{\text{м}}{\text{с}}}}{2  \cdot 1{,}6 \cdot 10^{-19}\,\text{Кл}} \approx 1284{,}4\,\text{В}.
    \end{align*}
}

\variantsplitter

\addpersonalvariant{Олег Климов}

\tasknumber{1}%
\task{%
    В однородном электрическом поле напряжённостью $E = 2\,\frac{\text{кВ}}{\text{м}}$
    переместили заряд $Q = 10\,\text{нКл}$ в направлении силовой линии
    на $d = 10\,\text{см}$.
    Определите
    \begin{itemize}
        \item работу поля,
        \item изменение потенциальной энергии заряда.
        % \item напряжение между начальной и конечной точками перемещения.
    \end{itemize}
}
\answer{%
    \begin{align*}
    A &= F \cdot d \cdot \cos \alpha = EQ \cdot d \cdot 1 = EQd = 2\,\frac{\text{кВ}}{\text{м}} \cdot 10\,\text{нКл} \cdot 10\,\text{см} = 2\,\text{мкДж}, \\
    \Delta E_\text{пот.} &= -A = -2\,\text{мкДж}
    \end{align*}
}
\solutionspace{80pt}

\tasknumber{2}%
\task{%
    Напряжение между двумя точками, лежащими на одной линии напряжённости
    однородного электрического поля, равно $V = 4\,\text{кВ}$.
    Расстояние между точками $d = 10\,\text{см}$.
    Какова напряжённость этого поля?
}
\answer{%
    $
        E_x = -\frac{\Delta \varphi}{\Delta x} \implies
        E = \frac{V}{d} = \frac{4\,\text{кВ}}{10\,\text{см}} = 40\,\frac{\text{кВ}}{\text{м}}.
    $
}
\solutionspace{40pt}

\tasknumber{3}%
\task{%
    Найти напряжение между точками $A$ и $B$ в однородном электрическом поле
    (см.
    рис.
    на доске), если $AB=l = 4\,\text{см}$, $\alpha=45^\circ$,
    $E = 60\,\frac{\text{В}}{\text{м}}$.
    Потенциал какой из точек $A$ и $B$ больше?
}
\solutionspace{120pt}

\tasknumber{4}%
\task{%
    При какой скорости позитрона его кинетическая энергия равна $E_\text{к} = 200\,\text{эВ}$?
}
\solutionspace{40pt}

\tasknumber{5}%
\task{%
    Позитрон $e^+$ вылетает из точки, потенциал которой $\varphi = 600\,\text{В}$,
    со скоростью $v = 4000000\,\frac{\text{м}}{\text{с}}$ параллельно линиям напряжённости однородного электрического поля.
    % Будет поле его ускорять или тормозить?
    В некоторой точке частица остановилась.
    Каков потенциал этой точки?
    Вдоль и против поля влетела изначально частица?
}
\answer{%
    \begin{align*}
    A_\text{внешних сил} &= \Delta E_\text{кин.} \implies A_\text{эл.
    поля} = 0 - \frac{mv^2}2.
    \\
    A_\text{эл.
    поля} &= q(\varphi_1 - \varphi_2) \implies\varphi_2 = \varphi_1 - \frac{A_\text{эл.
    поля}}q = \varphi_1 - \frac{- \frac{mv^2}2}q = \varphi_1 + \frac{mv^2}{2q} =  \\
    &= 600\,\text{В} + \frac{9{,}1 \cdot 10^{-31}\,\text{кг} \cdot \sqr{4000000\,\frac{\text{м}}{\text{с}}}}{2  \cdot 1{,}6 \cdot 10^{-19}\,\text{Кл}} \approx 645{,}5\,\text{В}.
    \end{align*}
}

\variantsplitter

\addpersonalvariant{Анна Ковалева}

\tasknumber{1}%
\task{%
    В однородном электрическом поле напряжённостью $E = 2\,\frac{\text{кВ}}{\text{м}}$
    переместили заряд $Q = 10\,\text{нКл}$ в направлении силовой линии
    на $d = 4\,\text{см}$.
    Определите
    \begin{itemize}
        \item работу поля,
        \item изменение потенциальной энергии заряда.
        % \item напряжение между начальной и конечной точками перемещения.
    \end{itemize}
}
\answer{%
    \begin{align*}
    A &= F \cdot d \cdot \cos \alpha = EQ \cdot d \cdot 1 = EQd = 2\,\frac{\text{кВ}}{\text{м}} \cdot 10\,\text{нКл} \cdot 4\,\text{см} = 0{,}8\,\text{мкДж}, \\
    \Delta E_\text{пот.} &= -A = -0{,}8\,\text{мкДж}
    \end{align*}
}
\solutionspace{80pt}

\tasknumber{2}%
\task{%
    Напряжение между двумя точками, лежащими на одной линии напряжённости
    однородного электрического поля, равно $U = 3\,\text{кВ}$.
    Расстояние между точками $r = 40\,\text{см}$.
    Какова напряжённость этого поля?
}
\answer{%
    $
        E_x = -\frac{\Delta \varphi}{\Delta x} \implies
        E = \frac{U}{r} = \frac{3\,\text{кВ}}{40\,\text{см}} = 7{,}5\,\frac{\text{кВ}}{\text{м}}.
    $
}
\solutionspace{40pt}

\tasknumber{3}%
\task{%
    Найти напряжение между точками $A$ и $B$ в однородном электрическом поле
    (см.
    рис.
    на доске), если $AB=r = 8\,\text{см}$, $\varphi=30^\circ$,
    $E = 120\,\frac{\text{В}}{\text{м}}$.
    Потенциал какой из точек $A$ и $B$ больше?
}
\solutionspace{120pt}

\tasknumber{4}%
\task{%
    При какой скорости позитрона его кинетическая энергия равна $E_\text{к} = 4\,\text{эВ}$?
}
\solutionspace{40pt}

\tasknumber{5}%
\task{%
    Электрон $e^-$ вылетает из точки, потенциал которой $\varphi = 600\,\text{В}$,
    со скоростью $v = 6000000\,\frac{\text{м}}{\text{с}}$ параллельно линиям напряжённости однородного электрического поля.
    % Будет поле его ускорять или тормозить?
    В некоторой точке частица остановилась.
    Каков потенциал этой точки?
    Вдоль и против поля влетела изначально частица?
}
\answer{%
    \begin{align*}
    A_\text{внешних сил} &= \Delta E_\text{кин.} \implies A_\text{эл.
    поля} = 0 - \frac{mv^2}2.
    \\
    A_\text{эл.
    поля} &= q(\varphi_1 - \varphi_2) \implies\varphi_2 = \varphi_1 - \frac{A_\text{эл.
    поля}}q = \varphi_1 - \frac{- \frac{mv^2}2}q = \varphi_1 + \frac{mv^2}{2q} =  \\
    &= 600\,\text{В} + \frac{9{,}1 \cdot 10^{-31}\,\text{кг} \cdot \sqr{6000000\,\frac{\text{м}}{\text{с}}}}{2  * (-1)  \cdot 1{,}6 \cdot 10^{-19}\,\text{Кл}} \approx 497{,}6\,\text{В}.
    \end{align*}
}

\variantsplitter

\addpersonalvariant{Глеб Ковылин}

\tasknumber{1}%
\task{%
    В однородном электрическом поле напряжённостью $E = 2\,\frac{\text{кВ}}{\text{м}}$
    переместили заряд $Q = 10\,\text{нКл}$ в направлении силовой линии
    на $d = 10\,\text{см}$.
    Определите
    \begin{itemize}
        \item работу поля,
        \item изменение потенциальной энергии заряда.
        % \item напряжение между начальной и конечной точками перемещения.
    \end{itemize}
}
\answer{%
    \begin{align*}
    A &= F \cdot d \cdot \cos \alpha = EQ \cdot d \cdot 1 = EQd = 2\,\frac{\text{кВ}}{\text{м}} \cdot 10\,\text{нКл} \cdot 10\,\text{см} = 2\,\text{мкДж}, \\
    \Delta E_\text{пот.} &= -A = -2\,\text{мкДж}
    \end{align*}
}
\solutionspace{80pt}

\tasknumber{2}%
\task{%
    Напряжение между двумя точками, лежащими на одной линии напряжённости
    однородного электрического поля, равно $U = 3\,\text{кВ}$.
    Расстояние между точками $d = 40\,\text{см}$.
    Какова напряжённость этого поля?
}
\answer{%
    $
        E_x = -\frac{\Delta \varphi}{\Delta x} \implies
        E = \frac{U}{d} = \frac{3\,\text{кВ}}{40\,\text{см}} = 7{,}5\,\frac{\text{кВ}}{\text{м}}.
    $
}
\solutionspace{40pt}

\tasknumber{3}%
\task{%
    Найти напряжение между точками $A$ и $B$ в однородном электрическом поле
    (см.
    рис.
    на доске), если $AB=l = 4\,\text{см}$, $\varphi=30^\circ$,
    $E = 30\,\frac{\text{В}}{\text{м}}$.
    Потенциал какой из точек $A$ и $B$ больше?
}
\solutionspace{120pt}

\tasknumber{4}%
\task{%
    При какой скорости электрона его кинетическая энергия равна $E_\text{к} = 4\,\text{эВ}$?
}
\solutionspace{40pt}

\tasknumber{5}%
\task{%
    Электрон $e^-$ вылетает из точки, потенциал которой $\varphi = 800\,\text{В}$,
    со скоростью $v = 12000000\,\frac{\text{м}}{\text{с}}$ параллельно линиям напряжённости однородного электрического поля.
    % Будет поле его ускорять или тормозить?
    В некоторой точке частица остановилась.
    Каков потенциал этой точки?
    Вдоль и против поля влетела изначально частица?
}
\answer{%
    \begin{align*}
    A_\text{внешних сил} &= \Delta E_\text{кин.} \implies A_\text{эл.
    поля} = 0 - \frac{mv^2}2.
    \\
    A_\text{эл.
    поля} &= q(\varphi_1 - \varphi_2) \implies\varphi_2 = \varphi_1 - \frac{A_\text{эл.
    поля}}q = \varphi_1 - \frac{- \frac{mv^2}2}q = \varphi_1 + \frac{mv^2}{2q} =  \\
    &= 800\,\text{В} + \frac{9{,}1 \cdot 10^{-31}\,\text{кг} \cdot \sqr{12000000\,\frac{\text{м}}{\text{с}}}}{2  * (-1)  \cdot 1{,}6 \cdot 10^{-19}\,\text{Кл}} \approx 390{,}5\,\text{В}.
    \end{align*}
}

\variantsplitter

\addpersonalvariant{Даниил Космынин}

\tasknumber{1}%
\task{%
    В однородном электрическом поле напряжённостью $E = 4\,\frac{\text{кВ}}{\text{м}}$
    переместили заряд $Q = 40\,\text{нКл}$ в направлении силовой линии
    на $d = 4\,\text{см}$.
    Определите
    \begin{itemize}
        \item работу поля,
        \item изменение потенциальной энергии заряда.
        % \item напряжение между начальной и конечной точками перемещения.
    \end{itemize}
}
\answer{%
    \begin{align*}
    A &= F \cdot d \cdot \cos \alpha = EQ \cdot d \cdot 1 = EQd = 4\,\frac{\text{кВ}}{\text{м}} \cdot 40\,\text{нКл} \cdot 4\,\text{см} = 6{,}4\,\text{мкДж}, \\
    \Delta E_\text{пот.} &= -A = -6{,}4\,\text{мкДж}
    \end{align*}
}
\solutionspace{80pt}

\tasknumber{2}%
\task{%
    Напряжение между двумя точками, лежащими на одной линии напряжённости
    однородного электрического поля, равно $U = 3\,\text{кВ}$.
    Расстояние между точками $l = 40\,\text{см}$.
    Какова напряжённость этого поля?
}
\answer{%
    $
        E_x = -\frac{\Delta \varphi}{\Delta x} \implies
        E = \frac{U}{l} = \frac{3\,\text{кВ}}{40\,\text{см}} = 7{,}5\,\frac{\text{кВ}}{\text{м}}.
    $
}
\solutionspace{40pt}

\tasknumber{3}%
\task{%
    Найти напряжение между точками $A$ и $B$ в однородном электрическом поле
    (см.
    рис.
    на доске), если $AB=d = 10\,\text{см}$, $\alpha=60^\circ$,
    $E = 120\,\frac{\text{В}}{\text{м}}$.
    Потенциал какой из точек $A$ и $B$ больше?
}
\solutionspace{120pt}

\tasknumber{4}%
\task{%
    При какой скорости позитрона его кинетическая энергия равна $E_\text{к} = 20\,\text{эВ}$?
}
\solutionspace{40pt}

\tasknumber{5}%
\task{%
    Электрон $e^-$ вылетает из точки, потенциал которой $\varphi = 1000\,\text{В}$,
    со скоростью $v = 10000000\,\frac{\text{м}}{\text{с}}$ параллельно линиям напряжённости однородного электрического поля.
    % Будет поле его ускорять или тормозить?
    В некоторой точке частица остановилась.
    Каков потенциал этой точки?
    Вдоль и против поля влетела изначально частица?
}
\answer{%
    \begin{align*}
    A_\text{внешних сил} &= \Delta E_\text{кин.} \implies A_\text{эл.
    поля} = 0 - \frac{mv^2}2.
    \\
    A_\text{эл.
    поля} &= q(\varphi_1 - \varphi_2) \implies\varphi_2 = \varphi_1 - \frac{A_\text{эл.
    поля}}q = \varphi_1 - \frac{- \frac{mv^2}2}q = \varphi_1 + \frac{mv^2}{2q} =  \\
    &= 1000\,\text{В} + \frac{9{,}1 \cdot 10^{-31}\,\text{кг} \cdot \sqr{10000000\,\frac{\text{м}}{\text{с}}}}{2  * (-1)  \cdot 1{,}6 \cdot 10^{-19}\,\text{Кл}} \approx 715{,}6\,\text{В}.
    \end{align*}
}

\variantsplitter

\addpersonalvariant{Алина Леоничева}

\tasknumber{1}%
\task{%
    В однородном электрическом поле напряжённостью $E = 20\,\frac{\text{кВ}}{\text{м}}$
    переместили заряд $q = -25\,\text{нКл}$ в направлении силовой линии
    на $l = 4\,\text{см}$.
    Определите
    \begin{itemize}
        \item работу поля,
        \item изменение потенциальной энергии заряда.
        % \item напряжение между начальной и конечной точками перемещения.
    \end{itemize}
}
\answer{%
    \begin{align*}
    A &= F \cdot l \cdot \cos \alpha = Eq \cdot l \cdot 1 = Eql = 20\,\frac{\text{кВ}}{\text{м}} \cdot -25\,\text{нКл} \cdot 4\,\text{см} = -20\,\text{мкДж}, \\
    \Delta E_\text{пот.} &= -A = 20\,\text{мкДж}
    \end{align*}
}
\solutionspace{80pt}

\tasknumber{2}%
\task{%
    Напряжение между двумя точками, лежащими на одной линии напряжённости
    однородного электрического поля, равно $V = 4\,\text{кВ}$.
    Расстояние между точками $d = 10\,\text{см}$.
    Какова напряжённость этого поля?
}
\answer{%
    $
        E_x = -\frac{\Delta \varphi}{\Delta x} \implies
        E = \frac{V}{d} = \frac{4\,\text{кВ}}{10\,\text{см}} = 40\,\frac{\text{кВ}}{\text{м}}.
    $
}
\solutionspace{40pt}

\tasknumber{3}%
\task{%
    Найти напряжение между точками $A$ и $B$ в однородном электрическом поле
    (см.
    рис.
    на доске), если $AB=r = 10\,\text{см}$, $\alpha=30^\circ$,
    $E = 100\,\frac{\text{В}}{\text{м}}$.
    Потенциал какой из точек $A$ и $B$ больше?
}
\solutionspace{120pt}

\tasknumber{4}%
\task{%
    При какой скорости позитрона его кинетическая энергия равна $E_\text{к} = 400\,\text{эВ}$?
}
\solutionspace{40pt}

\tasknumber{5}%
\task{%
    Позитрон $e^+$ вылетает из точки, потенциал которой $\varphi = 600\,\text{В}$,
    со скоростью $v = 3000000\,\frac{\text{м}}{\text{с}}$ параллельно линиям напряжённости однородного электрического поля.
    % Будет поле его ускорять или тормозить?
    В некоторой точке частица остановилась.
    Каков потенциал этой точки?
    Вдоль и против поля влетела изначально частица?
}
\answer{%
    \begin{align*}
    A_\text{внешних сил} &= \Delta E_\text{кин.} \implies A_\text{эл.
    поля} = 0 - \frac{mv^2}2.
    \\
    A_\text{эл.
    поля} &= q(\varphi_1 - \varphi_2) \implies\varphi_2 = \varphi_1 - \frac{A_\text{эл.
    поля}}q = \varphi_1 - \frac{- \frac{mv^2}2}q = \varphi_1 + \frac{mv^2}{2q} =  \\
    &= 600\,\text{В} + \frac{9{,}1 \cdot 10^{-31}\,\text{кг} \cdot \sqr{3000000\,\frac{\text{м}}{\text{с}}}}{2  \cdot 1{,}6 \cdot 10^{-19}\,\text{Кл}} \approx 625{,}6\,\text{В}.
    \end{align*}
}

\variantsplitter

\addpersonalvariant{Ирина Лин}

\tasknumber{1}%
\task{%
    В однородном электрическом поле напряжённостью $E = 20\,\frac{\text{кВ}}{\text{м}}$
    переместили заряд $q = -25\,\text{нКл}$ в направлении силовой линии
    на $l = 10\,\text{см}$.
    Определите
    \begin{itemize}
        \item работу поля,
        \item изменение потенциальной энергии заряда.
        % \item напряжение между начальной и конечной точками перемещения.
    \end{itemize}
}
\answer{%
    \begin{align*}
    A &= F \cdot l \cdot \cos \alpha = Eq \cdot l \cdot 1 = Eql = 20\,\frac{\text{кВ}}{\text{м}} \cdot -25\,\text{нКл} \cdot 10\,\text{см} = -50\,\text{мкДж}, \\
    \Delta E_\text{пот.} &= -A = 50\,\text{мкДж}
    \end{align*}
}
\solutionspace{80pt}

\tasknumber{2}%
\task{%
    Напряжение между двумя точками, лежащими на одной линии напряжённости
    однородного электрического поля, равно $U = 2\,\text{кВ}$.
    Расстояние между точками $d = 30\,\text{см}$.
    Какова напряжённость этого поля?
}
\answer{%
    $
        E_x = -\frac{\Delta \varphi}{\Delta x} \implies
        E = \frac{U}{d} = \frac{2\,\text{кВ}}{30\,\text{см}} = 6{,}7\,\frac{\text{кВ}}{\text{м}}.
    $
}
\solutionspace{40pt}

\tasknumber{3}%
\task{%
    Найти напряжение между точками $A$ и $B$ в однородном электрическом поле
    (см.
    рис.
    на доске), если $AB=l = 4\,\text{см}$, $\alpha=30^\circ$,
    $E = 100\,\frac{\text{В}}{\text{м}}$.
    Потенциал какой из точек $A$ и $B$ больше?
}
\solutionspace{120pt}

\tasknumber{4}%
\task{%
    При какой скорости протона его кинетическая энергия равна $E_\text{к} = 400\,\text{эВ}$?
}
\solutionspace{40pt}

\tasknumber{5}%
\task{%
    Электрон $e^-$ вылетает из точки, потенциал которой $\varphi = 200\,\text{В}$,
    со скоростью $v = 3000000\,\frac{\text{м}}{\text{с}}$ параллельно линиям напряжённости однородного электрического поля.
    % Будет поле его ускорять или тормозить?
    В некоторой точке частица остановилась.
    Каков потенциал этой точки?
    Вдоль и против поля влетела изначально частица?
}
\answer{%
    \begin{align*}
    A_\text{внешних сил} &= \Delta E_\text{кин.} \implies A_\text{эл.
    поля} = 0 - \frac{mv^2}2.
    \\
    A_\text{эл.
    поля} &= q(\varphi_1 - \varphi_2) \implies\varphi_2 = \varphi_1 - \frac{A_\text{эл.
    поля}}q = \varphi_1 - \frac{- \frac{mv^2}2}q = \varphi_1 + \frac{mv^2}{2q} =  \\
    &= 200\,\text{В} + \frac{9{,}1 \cdot 10^{-31}\,\text{кг} \cdot \sqr{3000000\,\frac{\text{м}}{\text{с}}}}{2  * (-1)  \cdot 1{,}6 \cdot 10^{-19}\,\text{Кл}} \approx 174{,}4\,\text{В}.
    \end{align*}
}

\variantsplitter

\addpersonalvariant{Олег Мальцев}

\tasknumber{1}%
\task{%
    В однородном электрическом поле напряжённостью $E = 4\,\frac{\text{кВ}}{\text{м}}$
    переместили заряд $q = 10\,\text{нКл}$ в направлении силовой линии
    на $d = 10\,\text{см}$.
    Определите
    \begin{itemize}
        \item работу поля,
        \item изменение потенциальной энергии заряда.
        % \item напряжение между начальной и конечной точками перемещения.
    \end{itemize}
}
\answer{%
    \begin{align*}
    A &= F \cdot d \cdot \cos \alpha = Eq \cdot d \cdot 1 = Eqd = 4\,\frac{\text{кВ}}{\text{м}} \cdot 10\,\text{нКл} \cdot 10\,\text{см} = 4\,\text{мкДж}, \\
    \Delta E_\text{пот.} &= -A = -4\,\text{мкДж}
    \end{align*}
}
\solutionspace{80pt}

\tasknumber{2}%
\task{%
    Напряжение между двумя точками, лежащими на одной линии напряжённости
    однородного электрического поля, равно $U = 3\,\text{кВ}$.
    Расстояние между точками $d = 40\,\text{см}$.
    Какова напряжённость этого поля?
}
\answer{%
    $
        E_x = -\frac{\Delta \varphi}{\Delta x} \implies
        E = \frac{U}{d} = \frac{3\,\text{кВ}}{40\,\text{см}} = 7{,}5\,\frac{\text{кВ}}{\text{м}}.
    $
}
\solutionspace{40pt}

\tasknumber{3}%
\task{%
    Найти напряжение между точками $A$ и $B$ в однородном электрическом поле
    (см.
    рис.
    на доске), если $AB=l = 4\,\text{см}$, $\alpha=60^\circ$,
    $E = 60\,\frac{\text{В}}{\text{м}}$.
    Потенциал какой из точек $A$ и $B$ больше?
}
\solutionspace{120pt}

\tasknumber{4}%
\task{%
    При какой скорости позитрона его кинетическая энергия равна $E_\text{к} = 30\,\text{эВ}$?
}
\solutionspace{40pt}

\tasknumber{5}%
\task{%
    Позитрон $e^+$ вылетает из точки, потенциал которой $\varphi = 200\,\text{В}$,
    со скоростью $v = 6000000\,\frac{\text{м}}{\text{с}}$ параллельно линиям напряжённости однородного электрического поля.
    % Будет поле его ускорять или тормозить?
    В некоторой точке частица остановилась.
    Каков потенциал этой точки?
    Вдоль и против поля влетела изначально частица?
}
\answer{%
    \begin{align*}
    A_\text{внешних сил} &= \Delta E_\text{кин.} \implies A_\text{эл.
    поля} = 0 - \frac{mv^2}2.
    \\
    A_\text{эл.
    поля} &= q(\varphi_1 - \varphi_2) \implies\varphi_2 = \varphi_1 - \frac{A_\text{эл.
    поля}}q = \varphi_1 - \frac{- \frac{mv^2}2}q = \varphi_1 + \frac{mv^2}{2q} =  \\
    &= 200\,\text{В} + \frac{9{,}1 \cdot 10^{-31}\,\text{кг} \cdot \sqr{6000000\,\frac{\text{м}}{\text{с}}}}{2  \cdot 1{,}6 \cdot 10^{-19}\,\text{Кл}} \approx 302{,}4\,\text{В}.
    \end{align*}
}

\variantsplitter

\addpersonalvariant{Ислам Мунаев}

\tasknumber{1}%
\task{%
    В однородном электрическом поле напряжённостью $E = 2\,\frac{\text{кВ}}{\text{м}}$
    переместили заряд $Q = -40\,\text{нКл}$ в направлении силовой линии
    на $r = 2\,\text{см}$.
    Определите
    \begin{itemize}
        \item работу поля,
        \item изменение потенциальной энергии заряда.
        % \item напряжение между начальной и конечной точками перемещения.
    \end{itemize}
}
\answer{%
    \begin{align*}
    A &= F \cdot r \cdot \cos \alpha = EQ \cdot r \cdot 1 = EQr = 2\,\frac{\text{кВ}}{\text{м}} \cdot -40\,\text{нКл} \cdot 2\,\text{см} = -1{,}6\,\text{мкДж}, \\
    \Delta E_\text{пот.} &= -A = 1{,}6\,\text{мкДж}
    \end{align*}
}
\solutionspace{80pt}

\tasknumber{2}%
\task{%
    Напряжение между двумя точками, лежащими на одной линии напряжённости
    однородного электрического поля, равно $V = 4\,\text{кВ}$.
    Расстояние между точками $l = 10\,\text{см}$.
    Какова напряжённость этого поля?
}
\answer{%
    $
        E_x = -\frac{\Delta \varphi}{\Delta x} \implies
        E = \frac{V}{l} = \frac{4\,\text{кВ}}{10\,\text{см}} = 40\,\frac{\text{кВ}}{\text{м}}.
    $
}
\solutionspace{40pt}

\tasknumber{3}%
\task{%
    Найти напряжение между точками $A$ и $B$ в однородном электрическом поле
    (см.
    рис.
    на доске), если $AB=r = 6\,\text{см}$, $\alpha=30^\circ$,
    $E = 100\,\frac{\text{В}}{\text{м}}$.
    Потенциал какой из точек $A$ и $B$ больше?
}
\solutionspace{120pt}

\tasknumber{4}%
\task{%
    При какой скорости протона его кинетическая энергия равна $E_\text{к} = 1000\,\text{эВ}$?
}
\solutionspace{40pt}

\tasknumber{5}%
\task{%
    Электрон $e^-$ вылетает из точки, потенциал которой $\varphi = 400\,\text{В}$,
    со скоростью $v = 12000000\,\frac{\text{м}}{\text{с}}$ параллельно линиям напряжённости однородного электрического поля.
    % Будет поле его ускорять или тормозить?
    В некоторой точке частица остановилась.
    Каков потенциал этой точки?
    Вдоль и против поля влетела изначально частица?
}
\answer{%
    \begin{align*}
    A_\text{внешних сил} &= \Delta E_\text{кин.} \implies A_\text{эл.
    поля} = 0 - \frac{mv^2}2.
    \\
    A_\text{эл.
    поля} &= q(\varphi_1 - \varphi_2) \implies\varphi_2 = \varphi_1 - \frac{A_\text{эл.
    поля}}q = \varphi_1 - \frac{- \frac{mv^2}2}q = \varphi_1 + \frac{mv^2}{2q} =  \\
    &= 400\,\text{В} + \frac{9{,}1 \cdot 10^{-31}\,\text{кг} \cdot \sqr{12000000\,\frac{\text{м}}{\text{с}}}}{2  * (-1)  \cdot 1{,}6 \cdot 10^{-19}\,\text{Кл}} \approx -9{,}5\,\text{В}.
    \end{align*}
}

\variantsplitter

\addpersonalvariant{Александр Наумов}

\tasknumber{1}%
\task{%
    В однородном электрическом поле напряжённостью $E = 2\,\frac{\text{кВ}}{\text{м}}$
    переместили заряд $Q = -40\,\text{нКл}$ в направлении силовой линии
    на $d = 2\,\text{см}$.
    Определите
    \begin{itemize}
        \item работу поля,
        \item изменение потенциальной энергии заряда.
        % \item напряжение между начальной и конечной точками перемещения.
    \end{itemize}
}
\answer{%
    \begin{align*}
    A &= F \cdot d \cdot \cos \alpha = EQ \cdot d \cdot 1 = EQd = 2\,\frac{\text{кВ}}{\text{м}} \cdot -40\,\text{нКл} \cdot 2\,\text{см} = -1{,}6\,\text{мкДж}, \\
    \Delta E_\text{пот.} &= -A = 1{,}6\,\text{мкДж}
    \end{align*}
}
\solutionspace{80pt}

\tasknumber{2}%
\task{%
    Напряжение между двумя точками, лежащими на одной линии напряжённости
    однородного электрического поля, равно $U = 2\,\text{кВ}$.
    Расстояние между точками $r = 20\,\text{см}$.
    Какова напряжённость этого поля?
}
\answer{%
    $
        E_x = -\frac{\Delta \varphi}{\Delta x} \implies
        E = \frac{U}{r} = \frac{2\,\text{кВ}}{20\,\text{см}} = 10\,\frac{\text{кВ}}{\text{м}}.
    $
}
\solutionspace{40pt}

\tasknumber{3}%
\task{%
    Найти напряжение между точками $A$ и $B$ в однородном электрическом поле
    (см.
    рис.
    на доске), если $AB=r = 10\,\text{см}$, $\alpha=60^\circ$,
    $E = 60\,\frac{\text{В}}{\text{м}}$.
    Потенциал какой из точек $A$ и $B$ больше?
}
\solutionspace{120pt}

\tasknumber{4}%
\task{%
    При какой скорости электрона его кинетическая энергия равна $E_\text{к} = 1000\,\text{эВ}$?
}
\solutionspace{40pt}

\tasknumber{5}%
\task{%
    Позитрон $e^+$ вылетает из точки, потенциал которой $\varphi = 600\,\text{В}$,
    со скоростью $v = 4000000\,\frac{\text{м}}{\text{с}}$ параллельно линиям напряжённости однородного электрического поля.
    % Будет поле его ускорять или тормозить?
    В некоторой точке частица остановилась.
    Каков потенциал этой точки?
    Вдоль и против поля влетела изначально частица?
}
\answer{%
    \begin{align*}
    A_\text{внешних сил} &= \Delta E_\text{кин.} \implies A_\text{эл.
    поля} = 0 - \frac{mv^2}2.
    \\
    A_\text{эл.
    поля} &= q(\varphi_1 - \varphi_2) \implies\varphi_2 = \varphi_1 - \frac{A_\text{эл.
    поля}}q = \varphi_1 - \frac{- \frac{mv^2}2}q = \varphi_1 + \frac{mv^2}{2q} =  \\
    &= 600\,\text{В} + \frac{9{,}1 \cdot 10^{-31}\,\text{кг} \cdot \sqr{4000000\,\frac{\text{м}}{\text{с}}}}{2  \cdot 1{,}6 \cdot 10^{-19}\,\text{Кл}} \approx 645{,}5\,\text{В}.
    \end{align*}
}

\variantsplitter

\addpersonalvariant{Георгий Новиков}

\tasknumber{1}%
\task{%
    В однородном электрическом поле напряжённостью $E = 20\,\frac{\text{кВ}}{\text{м}}$
    переместили заряд $Q = 10\,\text{нКл}$ в направлении силовой линии
    на $l = 4\,\text{см}$.
    Определите
    \begin{itemize}
        \item работу поля,
        \item изменение потенциальной энергии заряда.
        % \item напряжение между начальной и конечной точками перемещения.
    \end{itemize}
}
\answer{%
    \begin{align*}
    A &= F \cdot l \cdot \cos \alpha = EQ \cdot l \cdot 1 = EQl = 20\,\frac{\text{кВ}}{\text{м}} \cdot 10\,\text{нКл} \cdot 4\,\text{см} = 8\,\text{мкДж}, \\
    \Delta E_\text{пот.} &= -A = -8\,\text{мкДж}
    \end{align*}
}
\solutionspace{80pt}

\tasknumber{2}%
\task{%
    Напряжение между двумя точками, лежащими на одной линии напряжённости
    однородного электрического поля, равно $V = 6\,\text{кВ}$.
    Расстояние между точками $l = 20\,\text{см}$.
    Какова напряжённость этого поля?
}
\answer{%
    $
        E_x = -\frac{\Delta \varphi}{\Delta x} \implies
        E = \frac{V}{l} = \frac{6\,\text{кВ}}{20\,\text{см}} = 30\,\frac{\text{кВ}}{\text{м}}.
    $
}
\solutionspace{40pt}

\tasknumber{3}%
\task{%
    Найти напряжение между точками $A$ и $B$ в однородном электрическом поле
    (см.
    рис.
    на доске), если $AB=r = 4\,\text{см}$, $\varphi=30^\circ$,
    $E = 50\,\frac{\text{В}}{\text{м}}$.
    Потенциал какой из точек $A$ и $B$ больше?
}
\solutionspace{120pt}

\tasknumber{4}%
\task{%
    При какой скорости протона его кинетическая энергия равна $E_\text{к} = 30\,\text{эВ}$?
}
\solutionspace{40pt}

\tasknumber{5}%
\task{%
    Позитрон $e^+$ вылетает из точки, потенциал которой $\varphi = 1000\,\text{В}$,
    со скоростью $v = 6000000\,\frac{\text{м}}{\text{с}}$ параллельно линиям напряжённости однородного электрического поля.
    % Будет поле его ускорять или тормозить?
    В некоторой точке частица остановилась.
    Каков потенциал этой точки?
    Вдоль и против поля влетела изначально частица?
}
\answer{%
    \begin{align*}
    A_\text{внешних сил} &= \Delta E_\text{кин.} \implies A_\text{эл.
    поля} = 0 - \frac{mv^2}2.
    \\
    A_\text{эл.
    поля} &= q(\varphi_1 - \varphi_2) \implies\varphi_2 = \varphi_1 - \frac{A_\text{эл.
    поля}}q = \varphi_1 - \frac{- \frac{mv^2}2}q = \varphi_1 + \frac{mv^2}{2q} =  \\
    &= 1000\,\text{В} + \frac{9{,}1 \cdot 10^{-31}\,\text{кг} \cdot \sqr{6000000\,\frac{\text{м}}{\text{с}}}}{2  \cdot 1{,}6 \cdot 10^{-19}\,\text{Кл}} \approx 1102{,}4\,\text{В}.
    \end{align*}
}

\variantsplitter

\addpersonalvariant{Егор Осипов}

\tasknumber{1}%
\task{%
    В однородном электрическом поле напряжённостью $E = 2\,\frac{\text{кВ}}{\text{м}}$
    переместили заряд $q = -25\,\text{нКл}$ в направлении силовой линии
    на $r = 2\,\text{см}$.
    Определите
    \begin{itemize}
        \item работу поля,
        \item изменение потенциальной энергии заряда.
        % \item напряжение между начальной и конечной точками перемещения.
    \end{itemize}
}
\answer{%
    \begin{align*}
    A &= F \cdot r \cdot \cos \alpha = Eq \cdot r \cdot 1 = Eqr = 2\,\frac{\text{кВ}}{\text{м}} \cdot -25\,\text{нКл} \cdot 2\,\text{см} = -1\,\text{мкДж}, \\
    \Delta E_\text{пот.} &= -A = 1\,\text{мкДж}
    \end{align*}
}
\solutionspace{80pt}

\tasknumber{2}%
\task{%
    Напряжение между двумя точками, лежащими на одной линии напряжённости
    однородного электрического поля, равно $U = 4\,\text{кВ}$.
    Расстояние между точками $r = 10\,\text{см}$.
    Какова напряжённость этого поля?
}
\answer{%
    $
        E_x = -\frac{\Delta \varphi}{\Delta x} \implies
        E = \frac{U}{r} = \frac{4\,\text{кВ}}{10\,\text{см}} = 40\,\frac{\text{кВ}}{\text{м}}.
    $
}
\solutionspace{40pt}

\tasknumber{3}%
\task{%
    Найти напряжение между точками $A$ и $B$ в однородном электрическом поле
    (см.
    рис.
    на доске), если $AB=d = 6\,\text{см}$, $\alpha=45^\circ$,
    $E = 60\,\frac{\text{В}}{\text{м}}$.
    Потенциал какой из точек $A$ и $B$ больше?
}
\solutionspace{120pt}

\tasknumber{4}%
\task{%
    При какой скорости протона его кинетическая энергия равна $E_\text{к} = 4\,\text{эВ}$?
}
\solutionspace{40pt}

\tasknumber{5}%
\task{%
    Электрон $e^-$ вылетает из точки, потенциал которой $\varphi = 400\,\text{В}$,
    со скоростью $v = 3000000\,\frac{\text{м}}{\text{с}}$ параллельно линиям напряжённости однородного электрического поля.
    % Будет поле его ускорять или тормозить?
    В некоторой точке частица остановилась.
    Каков потенциал этой точки?
    Вдоль и против поля влетела изначально частица?
}
\answer{%
    \begin{align*}
    A_\text{внешних сил} &= \Delta E_\text{кин.} \implies A_\text{эл.
    поля} = 0 - \frac{mv^2}2.
    \\
    A_\text{эл.
    поля} &= q(\varphi_1 - \varphi_2) \implies\varphi_2 = \varphi_1 - \frac{A_\text{эл.
    поля}}q = \varphi_1 - \frac{- \frac{mv^2}2}q = \varphi_1 + \frac{mv^2}{2q} =  \\
    &= 400\,\text{В} + \frac{9{,}1 \cdot 10^{-31}\,\text{кг} \cdot \sqr{3000000\,\frac{\text{м}}{\text{с}}}}{2  * (-1)  \cdot 1{,}6 \cdot 10^{-19}\,\text{Кл}} \approx 374{,}4\,\text{В}.
    \end{align*}
}

\variantsplitter

\addpersonalvariant{Руслан Перепелица}

\tasknumber{1}%
\task{%
    В однородном электрическом поле напряжённостью $E = 20\,\frac{\text{кВ}}{\text{м}}$
    переместили заряд $Q = 25\,\text{нКл}$ в направлении силовой линии
    на $r = 10\,\text{см}$.
    Определите
    \begin{itemize}
        \item работу поля,
        \item изменение потенциальной энергии заряда.
        % \item напряжение между начальной и конечной точками перемещения.
    \end{itemize}
}
\answer{%
    \begin{align*}
    A &= F \cdot r \cdot \cos \alpha = EQ \cdot r \cdot 1 = EQr = 20\,\frac{\text{кВ}}{\text{м}} \cdot 25\,\text{нКл} \cdot 10\,\text{см} = 50\,\text{мкДж}, \\
    \Delta E_\text{пот.} &= -A = -50\,\text{мкДж}
    \end{align*}
}
\solutionspace{80pt}

\tasknumber{2}%
\task{%
    Напряжение между двумя точками, лежащими на одной линии напряжённости
    однородного электрического поля, равно $V = 3\,\text{кВ}$.
    Расстояние между точками $l = 40\,\text{см}$.
    Какова напряжённость этого поля?
}
\answer{%
    $
        E_x = -\frac{\Delta \varphi}{\Delta x} \implies
        E = \frac{V}{l} = \frac{3\,\text{кВ}}{40\,\text{см}} = 7{,}5\,\frac{\text{кВ}}{\text{м}}.
    $
}
\solutionspace{40pt}

\tasknumber{3}%
\task{%
    Найти напряжение между точками $A$ и $B$ в однородном электрическом поле
    (см.
    рис.
    на доске), если $AB=l = 12\,\text{см}$, $\varphi=45^\circ$,
    $E = 100\,\frac{\text{В}}{\text{м}}$.
    Потенциал какой из точек $A$ и $B$ больше?
}
\solutionspace{120pt}

\tasknumber{4}%
\task{%
    При какой скорости электрона его кинетическая энергия равна $E_\text{к} = 20\,\text{эВ}$?
}
\solutionspace{40pt}

\tasknumber{5}%
\task{%
    Электрон $e^-$ вылетает из точки, потенциал которой $\varphi = 200\,\text{В}$,
    со скоростью $v = 6000000\,\frac{\text{м}}{\text{с}}$ параллельно линиям напряжённости однородного электрического поля.
    % Будет поле его ускорять или тормозить?
    В некоторой точке частица остановилась.
    Каков потенциал этой точки?
    Вдоль и против поля влетела изначально частица?
}
\answer{%
    \begin{align*}
    A_\text{внешних сил} &= \Delta E_\text{кин.} \implies A_\text{эл.
    поля} = 0 - \frac{mv^2}2.
    \\
    A_\text{эл.
    поля} &= q(\varphi_1 - \varphi_2) \implies\varphi_2 = \varphi_1 - \frac{A_\text{эл.
    поля}}q = \varphi_1 - \frac{- \frac{mv^2}2}q = \varphi_1 + \frac{mv^2}{2q} =  \\
    &= 200\,\text{В} + \frac{9{,}1 \cdot 10^{-31}\,\text{кг} \cdot \sqr{6000000\,\frac{\text{м}}{\text{с}}}}{2  * (-1)  \cdot 1{,}6 \cdot 10^{-19}\,\text{Кл}} \approx 97{,}6\,\text{В}.
    \end{align*}
}

\variantsplitter

\addpersonalvariant{Михаил Перин}

\tasknumber{1}%
\task{%
    В однородном электрическом поле напряжённостью $E = 20\,\frac{\text{кВ}}{\text{м}}$
    переместили заряд $q = 40\,\text{нКл}$ в направлении силовой линии
    на $r = 2\,\text{см}$.
    Определите
    \begin{itemize}
        \item работу поля,
        \item изменение потенциальной энергии заряда.
        % \item напряжение между начальной и конечной точками перемещения.
    \end{itemize}
}
\answer{%
    \begin{align*}
    A &= F \cdot r \cdot \cos \alpha = Eq \cdot r \cdot 1 = Eqr = 20\,\frac{\text{кВ}}{\text{м}} \cdot 40\,\text{нКл} \cdot 2\,\text{см} = 16\,\text{мкДж}, \\
    \Delta E_\text{пот.} &= -A = -16\,\text{мкДж}
    \end{align*}
}
\solutionspace{80pt}

\tasknumber{2}%
\task{%
    Напряжение между двумя точками, лежащими на одной линии напряжённости
    однородного электрического поля, равно $V = 4\,\text{кВ}$.
    Расстояние между точками $r = 40\,\text{см}$.
    Какова напряжённость этого поля?
}
\answer{%
    $
        E_x = -\frac{\Delta \varphi}{\Delta x} \implies
        E = \frac{V}{r} = \frac{4\,\text{кВ}}{40\,\text{см}} = 10\,\frac{\text{кВ}}{\text{м}}.
    $
}
\solutionspace{40pt}

\tasknumber{3}%
\task{%
    Найти напряжение между точками $A$ и $B$ в однородном электрическом поле
    (см.
    рис.
    на доске), если $AB=d = 4\,\text{см}$, $\alpha=45^\circ$,
    $E = 50\,\frac{\text{В}}{\text{м}}$.
    Потенциал какой из точек $A$ и $B$ больше?
}
\solutionspace{120pt}

\tasknumber{4}%
\task{%
    При какой скорости позитрона его кинетическая энергия равна $E_\text{к} = 400\,\text{эВ}$?
}
\solutionspace{40pt}

\tasknumber{5}%
\task{%
    Электрон $e^-$ вылетает из точки, потенциал которой $\varphi = 800\,\text{В}$,
    со скоростью $v = 12000000\,\frac{\text{м}}{\text{с}}$ параллельно линиям напряжённости однородного электрического поля.
    % Будет поле его ускорять или тормозить?
    В некоторой точке частица остановилась.
    Каков потенциал этой точки?
    Вдоль и против поля влетела изначально частица?
}
\answer{%
    \begin{align*}
    A_\text{внешних сил} &= \Delta E_\text{кин.} \implies A_\text{эл.
    поля} = 0 - \frac{mv^2}2.
    \\
    A_\text{эл.
    поля} &= q(\varphi_1 - \varphi_2) \implies\varphi_2 = \varphi_1 - \frac{A_\text{эл.
    поля}}q = \varphi_1 - \frac{- \frac{mv^2}2}q = \varphi_1 + \frac{mv^2}{2q} =  \\
    &= 800\,\text{В} + \frac{9{,}1 \cdot 10^{-31}\,\text{кг} \cdot \sqr{12000000\,\frac{\text{м}}{\text{с}}}}{2  * (-1)  \cdot 1{,}6 \cdot 10^{-19}\,\text{Кл}} \approx 390{,}5\,\text{В}.
    \end{align*}
}

\variantsplitter

\addpersonalvariant{Егор Подуровский}

\tasknumber{1}%
\task{%
    В однородном электрическом поле напряжённостью $E = 2\,\frac{\text{кВ}}{\text{м}}$
    переместили заряд $Q = -40\,\text{нКл}$ в направлении силовой линии
    на $r = 2\,\text{см}$.
    Определите
    \begin{itemize}
        \item работу поля,
        \item изменение потенциальной энергии заряда.
        % \item напряжение между начальной и конечной точками перемещения.
    \end{itemize}
}
\answer{%
    \begin{align*}
    A &= F \cdot r \cdot \cos \alpha = EQ \cdot r \cdot 1 = EQr = 2\,\frac{\text{кВ}}{\text{м}} \cdot -40\,\text{нКл} \cdot 2\,\text{см} = -1{,}6\,\text{мкДж}, \\
    \Delta E_\text{пот.} &= -A = 1{,}6\,\text{мкДж}
    \end{align*}
}
\solutionspace{80pt}

\tasknumber{2}%
\task{%
    Напряжение между двумя точками, лежащими на одной линии напряжённости
    однородного электрического поля, равно $U = 3\,\text{кВ}$.
    Расстояние между точками $d = 20\,\text{см}$.
    Какова напряжённость этого поля?
}
\answer{%
    $
        E_x = -\frac{\Delta \varphi}{\Delta x} \implies
        E = \frac{U}{d} = \frac{3\,\text{кВ}}{20\,\text{см}} = 15\,\frac{\text{кВ}}{\text{м}}.
    $
}
\solutionspace{40pt}

\tasknumber{3}%
\task{%
    Найти напряжение между точками $A$ и $B$ в однородном электрическом поле
    (см.
    рис.
    на доске), если $AB=d = 4\,\text{см}$, $\varphi=45^\circ$,
    $E = 100\,\frac{\text{В}}{\text{м}}$.
    Потенциал какой из точек $A$ и $B$ больше?
}
\solutionspace{120pt}

\tasknumber{4}%
\task{%
    При какой скорости протона его кинетическая энергия равна $E_\text{к} = 20\,\text{эВ}$?
}
\solutionspace{40pt}

\tasknumber{5}%
\task{%
    Позитрон $e^+$ вылетает из точки, потенциал которой $\varphi = 600\,\text{В}$,
    со скоростью $v = 6000000\,\frac{\text{м}}{\text{с}}$ параллельно линиям напряжённости однородного электрического поля.
    % Будет поле его ускорять или тормозить?
    В некоторой точке частица остановилась.
    Каков потенциал этой точки?
    Вдоль и против поля влетела изначально частица?
}
\answer{%
    \begin{align*}
    A_\text{внешних сил} &= \Delta E_\text{кин.} \implies A_\text{эл.
    поля} = 0 - \frac{mv^2}2.
    \\
    A_\text{эл.
    поля} &= q(\varphi_1 - \varphi_2) \implies\varphi_2 = \varphi_1 - \frac{A_\text{эл.
    поля}}q = \varphi_1 - \frac{- \frac{mv^2}2}q = \varphi_1 + \frac{mv^2}{2q} =  \\
    &= 600\,\text{В} + \frac{9{,}1 \cdot 10^{-31}\,\text{кг} \cdot \sqr{6000000\,\frac{\text{м}}{\text{с}}}}{2  \cdot 1{,}6 \cdot 10^{-19}\,\text{Кл}} \approx 702{,}4\,\text{В}.
    \end{align*}
}

\variantsplitter

\addpersonalvariant{Роман Прибылов}

\tasknumber{1}%
\task{%
    В однородном электрическом поле напряжённостью $E = 2\,\frac{\text{кВ}}{\text{м}}$
    переместили заряд $Q = -40\,\text{нКл}$ в направлении силовой линии
    на $l = 10\,\text{см}$.
    Определите
    \begin{itemize}
        \item работу поля,
        \item изменение потенциальной энергии заряда.
        % \item напряжение между начальной и конечной точками перемещения.
    \end{itemize}
}
\answer{%
    \begin{align*}
    A &= F \cdot l \cdot \cos \alpha = EQ \cdot l \cdot 1 = EQl = 2\,\frac{\text{кВ}}{\text{м}} \cdot -40\,\text{нКл} \cdot 10\,\text{см} = -8\,\text{мкДж}, \\
    \Delta E_\text{пот.} &= -A = 8\,\text{мкДж}
    \end{align*}
}
\solutionspace{80pt}

\tasknumber{2}%
\task{%
    Напряжение между двумя точками, лежащими на одной линии напряжённости
    однородного электрического поля, равно $U = 6\,\text{кВ}$.
    Расстояние между точками $l = 20\,\text{см}$.
    Какова напряжённость этого поля?
}
\answer{%
    $
        E_x = -\frac{\Delta \varphi}{\Delta x} \implies
        E = \frac{U}{l} = \frac{6\,\text{кВ}}{20\,\text{см}} = 30\,\frac{\text{кВ}}{\text{м}}.
    $
}
\solutionspace{40pt}

\tasknumber{3}%
\task{%
    Найти напряжение между точками $A$ и $B$ в однородном электрическом поле
    (см.
    рис.
    на доске), если $AB=r = 10\,\text{см}$, $\alpha=30^\circ$,
    $E = 50\,\frac{\text{В}}{\text{м}}$.
    Потенциал какой из точек $A$ и $B$ больше?
}
\solutionspace{120pt}

\tasknumber{4}%
\task{%
    При какой скорости нейтрона его кинетическая энергия равна $E_\text{к} = 4\,\text{эВ}$?
}
\solutionspace{40pt}

\tasknumber{5}%
\task{%
    Позитрон $e^+$ вылетает из точки, потенциал которой $\varphi = 800\,\text{В}$,
    со скоростью $v = 3000000\,\frac{\text{м}}{\text{с}}$ параллельно линиям напряжённости однородного электрического поля.
    % Будет поле его ускорять или тормозить?
    В некоторой точке частица остановилась.
    Каков потенциал этой точки?
    Вдоль и против поля влетела изначально частица?
}
\answer{%
    \begin{align*}
    A_\text{внешних сил} &= \Delta E_\text{кин.} \implies A_\text{эл.
    поля} = 0 - \frac{mv^2}2.
    \\
    A_\text{эл.
    поля} &= q(\varphi_1 - \varphi_2) \implies\varphi_2 = \varphi_1 - \frac{A_\text{эл.
    поля}}q = \varphi_1 - \frac{- \frac{mv^2}2}q = \varphi_1 + \frac{mv^2}{2q} =  \\
    &= 800\,\text{В} + \frac{9{,}1 \cdot 10^{-31}\,\text{кг} \cdot \sqr{3000000\,\frac{\text{м}}{\text{с}}}}{2  \cdot 1{,}6 \cdot 10^{-19}\,\text{Кл}} \approx 825{,}6\,\text{В}.
    \end{align*}
}

\variantsplitter

\addpersonalvariant{Александр Селехметьев}

\tasknumber{1}%
\task{%
    В однородном электрическом поле напряжённостью $E = 20\,\frac{\text{кВ}}{\text{м}}$
    переместили заряд $q = -10\,\text{нКл}$ в направлении силовой линии
    на $l = 5\,\text{см}$.
    Определите
    \begin{itemize}
        \item работу поля,
        \item изменение потенциальной энергии заряда.
        % \item напряжение между начальной и конечной точками перемещения.
    \end{itemize}
}
\answer{%
    \begin{align*}
    A &= F \cdot l \cdot \cos \alpha = Eq \cdot l \cdot 1 = Eql = 20\,\frac{\text{кВ}}{\text{м}} \cdot -10\,\text{нКл} \cdot 5\,\text{см} = -10\,\text{мкДж}, \\
    \Delta E_\text{пот.} &= -A = 10\,\text{мкДж}
    \end{align*}
}
\solutionspace{80pt}

\tasknumber{2}%
\task{%
    Напряжение между двумя точками, лежащими на одной линии напряжённости
    однородного электрического поля, равно $V = 3\,\text{кВ}$.
    Расстояние между точками $l = 40\,\text{см}$.
    Какова напряжённость этого поля?
}
\answer{%
    $
        E_x = -\frac{\Delta \varphi}{\Delta x} \implies
        E = \frac{V}{l} = \frac{3\,\text{кВ}}{40\,\text{см}} = 7{,}5\,\frac{\text{кВ}}{\text{м}}.
    $
}
\solutionspace{40pt}

\tasknumber{3}%
\task{%
    Найти напряжение между точками $A$ и $B$ в однородном электрическом поле
    (см.
    рис.
    на доске), если $AB=r = 6\,\text{см}$, $\varphi=45^\circ$,
    $E = 50\,\frac{\text{В}}{\text{м}}$.
    Потенциал какой из точек $A$ и $B$ больше?
}
\solutionspace{120pt}

\tasknumber{4}%
\task{%
    При какой скорости электрона его кинетическая энергия равна $E_\text{к} = 1000\,\text{эВ}$?
}
\solutionspace{40pt}

\tasknumber{5}%
\task{%
    Электрон $e^-$ вылетает из точки, потенциал которой $\varphi = 800\,\text{В}$,
    со скоростью $v = 4000000\,\frac{\text{м}}{\text{с}}$ параллельно линиям напряжённости однородного электрического поля.
    % Будет поле его ускорять или тормозить?
    В некоторой точке частица остановилась.
    Каков потенциал этой точки?
    Вдоль и против поля влетела изначально частица?
}
\answer{%
    \begin{align*}
    A_\text{внешних сил} &= \Delta E_\text{кин.} \implies A_\text{эл.
    поля} = 0 - \frac{mv^2}2.
    \\
    A_\text{эл.
    поля} &= q(\varphi_1 - \varphi_2) \implies\varphi_2 = \varphi_1 - \frac{A_\text{эл.
    поля}}q = \varphi_1 - \frac{- \frac{mv^2}2}q = \varphi_1 + \frac{mv^2}{2q} =  \\
    &= 800\,\text{В} + \frac{9{,}1 \cdot 10^{-31}\,\text{кг} \cdot \sqr{4000000\,\frac{\text{м}}{\text{с}}}}{2  * (-1)  \cdot 1{,}6 \cdot 10^{-19}\,\text{Кл}} \approx 754{,}5\,\text{В}.
    \end{align*}
}

\variantsplitter

\addpersonalvariant{Алексей Тихонов}

\tasknumber{1}%
\task{%
    В однородном электрическом поле напряжённостью $E = 2\,\frac{\text{кВ}}{\text{м}}$
    переместили заряд $q = 10\,\text{нКл}$ в направлении силовой линии
    на $l = 2\,\text{см}$.
    Определите
    \begin{itemize}
        \item работу поля,
        \item изменение потенциальной энергии заряда.
        % \item напряжение между начальной и конечной точками перемещения.
    \end{itemize}
}
\answer{%
    \begin{align*}
    A &= F \cdot l \cdot \cos \alpha = Eq \cdot l \cdot 1 = Eql = 2\,\frac{\text{кВ}}{\text{м}} \cdot 10\,\text{нКл} \cdot 2\,\text{см} = 0{,}4\,\text{мкДж}, \\
    \Delta E_\text{пот.} &= -A = -0{,}4\,\text{мкДж}
    \end{align*}
}
\solutionspace{80pt}

\tasknumber{2}%
\task{%
    Напряжение между двумя точками, лежащими на одной линии напряжённости
    однородного электрического поля, равно $U = 6\,\text{кВ}$.
    Расстояние между точками $d = 20\,\text{см}$.
    Какова напряжённость этого поля?
}
\answer{%
    $
        E_x = -\frac{\Delta \varphi}{\Delta x} \implies
        E = \frac{U}{d} = \frac{6\,\text{кВ}}{20\,\text{см}} = 30\,\frac{\text{кВ}}{\text{м}}.
    $
}
\solutionspace{40pt}

\tasknumber{3}%
\task{%
    Найти напряжение между точками $A$ и $B$ в однородном электрическом поле
    (см.
    рис.
    на доске), если $AB=r = 12\,\text{см}$, $\alpha=30^\circ$,
    $E = 50\,\frac{\text{В}}{\text{м}}$.
    Потенциал какой из точек $A$ и $B$ больше?
}
\solutionspace{120pt}

\tasknumber{4}%
\task{%
    При какой скорости электрона его кинетическая энергия равна $E_\text{к} = 4\,\text{эВ}$?
}
\solutionspace{40pt}

\tasknumber{5}%
\task{%
    Электрон $e^-$ вылетает из точки, потенциал которой $\varphi = 800\,\text{В}$,
    со скоростью $v = 12000000\,\frac{\text{м}}{\text{с}}$ параллельно линиям напряжённости однородного электрического поля.
    % Будет поле его ускорять или тормозить?
    В некоторой точке частица остановилась.
    Каков потенциал этой точки?
    Вдоль и против поля влетела изначально частица?
}
\answer{%
    \begin{align*}
    A_\text{внешних сил} &= \Delta E_\text{кин.} \implies A_\text{эл.
    поля} = 0 - \frac{mv^2}2.
    \\
    A_\text{эл.
    поля} &= q(\varphi_1 - \varphi_2) \implies\varphi_2 = \varphi_1 - \frac{A_\text{эл.
    поля}}q = \varphi_1 - \frac{- \frac{mv^2}2}q = \varphi_1 + \frac{mv^2}{2q} =  \\
    &= 800\,\text{В} + \frac{9{,}1 \cdot 10^{-31}\,\text{кг} \cdot \sqr{12000000\,\frac{\text{м}}{\text{с}}}}{2  * (-1)  \cdot 1{,}6 \cdot 10^{-19}\,\text{Кл}} \approx 390{,}5\,\text{В}.
    \end{align*}
}

\variantsplitter

\addpersonalvariant{Алина Филиппова}

\tasknumber{1}%
\task{%
    В однородном электрическом поле напряжённостью $E = 2\,\frac{\text{кВ}}{\text{м}}$
    переместили заряд $q = 40\,\text{нКл}$ в направлении силовой линии
    на $l = 5\,\text{см}$.
    Определите
    \begin{itemize}
        \item работу поля,
        \item изменение потенциальной энергии заряда.
        % \item напряжение между начальной и конечной точками перемещения.
    \end{itemize}
}
\answer{%
    \begin{align*}
    A &= F \cdot l \cdot \cos \alpha = Eq \cdot l \cdot 1 = Eql = 2\,\frac{\text{кВ}}{\text{м}} \cdot 40\,\text{нКл} \cdot 5\,\text{см} = 4\,\text{мкДж}, \\
    \Delta E_\text{пот.} &= -A = -4\,\text{мкДж}
    \end{align*}
}
\solutionspace{80pt}

\tasknumber{2}%
\task{%
    Напряжение между двумя точками, лежащими на одной линии напряжённости
    однородного электрического поля, равно $U = 2\,\text{кВ}$.
    Расстояние между точками $d = 30\,\text{см}$.
    Какова напряжённость этого поля?
}
\answer{%
    $
        E_x = -\frac{\Delta \varphi}{\Delta x} \implies
        E = \frac{U}{d} = \frac{2\,\text{кВ}}{30\,\text{см}} = 6{,}7\,\frac{\text{кВ}}{\text{м}}.
    $
}
\solutionspace{40pt}

\tasknumber{3}%
\task{%
    Найти напряжение между точками $A$ и $B$ в однородном электрическом поле
    (см.
    рис.
    на доске), если $AB=d = 12\,\text{см}$, $\varphi=30^\circ$,
    $E = 50\,\frac{\text{В}}{\text{м}}$.
    Потенциал какой из точек $A$ и $B$ больше?
}
\solutionspace{120pt}

\tasknumber{4}%
\task{%
    При какой скорости нейтрона его кинетическая энергия равна $E_\text{к} = 50\,\text{эВ}$?
}
\solutionspace{40pt}

\tasknumber{5}%
\task{%
    Позитрон $e^+$ вылетает из точки, потенциал которой $\varphi = 800\,\text{В}$,
    со скоростью $v = 4000000\,\frac{\text{м}}{\text{с}}$ параллельно линиям напряжённости однородного электрического поля.
    % Будет поле его ускорять или тормозить?
    В некоторой точке частица остановилась.
    Каков потенциал этой точки?
    Вдоль и против поля влетела изначально частица?
}
\answer{%
    \begin{align*}
    A_\text{внешних сил} &= \Delta E_\text{кин.} \implies A_\text{эл.
    поля} = 0 - \frac{mv^2}2.
    \\
    A_\text{эл.
    поля} &= q(\varphi_1 - \varphi_2) \implies\varphi_2 = \varphi_1 - \frac{A_\text{эл.
    поля}}q = \varphi_1 - \frac{- \frac{mv^2}2}q = \varphi_1 + \frac{mv^2}{2q} =  \\
    &= 800\,\text{В} + \frac{9{,}1 \cdot 10^{-31}\,\text{кг} \cdot \sqr{4000000\,\frac{\text{м}}{\text{с}}}}{2  \cdot 1{,}6 \cdot 10^{-19}\,\text{Кл}} \approx 845{,}5\,\text{В}.
    \end{align*}
}

\variantsplitter

\addpersonalvariant{Алина Яшина}

\tasknumber{1}%
\task{%
    В однородном электрическом поле напряжённостью $E = 2\,\frac{\text{кВ}}{\text{м}}$
    переместили заряд $q = 10\,\text{нКл}$ в направлении силовой линии
    на $d = 2\,\text{см}$.
    Определите
    \begin{itemize}
        \item работу поля,
        \item изменение потенциальной энергии заряда.
        % \item напряжение между начальной и конечной точками перемещения.
    \end{itemize}
}
\answer{%
    \begin{align*}
    A &= F \cdot d \cdot \cos \alpha = Eq \cdot d \cdot 1 = Eqd = 2\,\frac{\text{кВ}}{\text{м}} \cdot 10\,\text{нКл} \cdot 2\,\text{см} = 0{,}4\,\text{мкДж}, \\
    \Delta E_\text{пот.} &= -A = -0{,}4\,\text{мкДж}
    \end{align*}
}
\solutionspace{80pt}

\tasknumber{2}%
\task{%
    Напряжение между двумя точками, лежащими на одной линии напряжённости
    однородного электрического поля, равно $V = 4\,\text{кВ}$.
    Расстояние между точками $d = 20\,\text{см}$.
    Какова напряжённость этого поля?
}
\answer{%
    $
        E_x = -\frac{\Delta \varphi}{\Delta x} \implies
        E = \frac{V}{d} = \frac{4\,\text{кВ}}{20\,\text{см}} = 20\,\frac{\text{кВ}}{\text{м}}.
    $
}
\solutionspace{40pt}

\tasknumber{3}%
\task{%
    Найти напряжение между точками $A$ и $B$ в однородном электрическом поле
    (см.
    рис.
    на доске), если $AB=d = 6\,\text{см}$, $\varphi=30^\circ$,
    $E = 120\,\frac{\text{В}}{\text{м}}$.
    Потенциал какой из точек $A$ и $B$ больше?
}
\solutionspace{120pt}

\tasknumber{4}%
\task{%
    При какой скорости нейтрона его кинетическая энергия равна $E_\text{к} = 20\,\text{эВ}$?
}
\solutionspace{40pt}

\tasknumber{5}%
\task{%
    Электрон $e^-$ вылетает из точки, потенциал которой $\varphi = 800\,\text{В}$,
    со скоростью $v = 3000000\,\frac{\text{м}}{\text{с}}$ параллельно линиям напряжённости однородного электрического поля.
    % Будет поле его ускорять или тормозить?
    В некоторой точке частица остановилась.
    Каков потенциал этой точки?
    Вдоль и против поля влетела изначально частица?
}
\answer{%
    \begin{align*}
    A_\text{внешних сил} &= \Delta E_\text{кин.} \implies A_\text{эл.
    поля} = 0 - \frac{mv^2}2.
    \\
    A_\text{эл.
    поля} &= q(\varphi_1 - \varphi_2) \implies\varphi_2 = \varphi_1 - \frac{A_\text{эл.
    поля}}q = \varphi_1 - \frac{- \frac{mv^2}2}q = \varphi_1 + \frac{mv^2}{2q} =  \\
    &= 800\,\text{В} + \frac{9{,}1 \cdot 10^{-31}\,\text{кг} \cdot \sqr{3000000\,\frac{\text{м}}{\text{с}}}}{2  * (-1)  \cdot 1{,}6 \cdot 10^{-19}\,\text{Кл}} \approx 774{,}4\,\text{В}.
    \end{align*}
}
% autogenerated
