\setdate{30~апреля~2019}
\setclass{11«А»}

\addpersonalvariant{Михаил Бурмистров}

\tasknumber{1}%
\task{%
    Определите длину волны лучей, фотоны которых имеют энергию
    равную кинетической энергии электрона, ускоренного напряжением $233\,\text{В}$.
}
\answer{%
    $E = h\frac c\lambda = e U \implies \lambda = \frac{hc}{eU} = \frac{6{,}626 \cdot 10^{-34}\,\text{Дж}\cdot\text{с} \cdot 3 \cdot 10^{8}\,\frac{\text{м}}{\text{с}}}{1{,}6 \cdot 10^{-19}\,\text{Кл} \cdot 233\,\text{В}} \approx 5{,}33\,\text{нм}.$
}
\solutionspace{120pt}

\tasknumber{2}%
\task{%
    Лучше всего нейтронное излучение ослабляет вода: в 4 раза лучше бетона и в 3 раза лучше свинца.
    Толщина слоя половинного ослабления $\gamma$-излучения для воды равна $3\,\text{см}$.
    Во сколько раз ослабит нейтронное излучение слой воды толщиной $h = 60\,\text{см}$?
}

\variantsplitter

\addpersonalvariant{Никита Бекасов}

\tasknumber{1}%
\task{%
    Определите длину волны лучей, фотоны которых имеют энергию
    равную кинетической энергии электрона, ускоренного напряжением $8\,\text{В}$.
}
\answer{%
    $E = h\frac c\lambda = e U \implies \lambda = \frac{hc}{eU} = \frac{6{,}626 \cdot 10^{-34}\,\text{Дж}\cdot\text{с} \cdot 3 \cdot 10^{8}\,\frac{\text{м}}{\text{с}}}{1{,}6 \cdot 10^{-19}\,\text{Кл} \cdot 8\,\text{В}} \approx 155\,\text{нм}.$
}
\solutionspace{120pt}

\tasknumber{2}%
\task{%
    Лучше всего нейтронное излучение ослабляет вода: в 4 раза лучше бетона и в 3 раза лучше свинца.
    Толщина слоя половинного ослабления $\gamma$-излучения для воды равна $3\,\text{см}$.
    Во сколько раз ослабит нейтронное излучение слой воды толщиной $h = 120\,\text{см}$?
}

\variantsplitter

\addpersonalvariant{Ольга Борисова}

\tasknumber{1}%
\task{%
    Определите длину волны лучей, фотоны которых имеют энергию
    равную кинетической энергии электрона, ускоренного напряжением $144\,\text{В}$.
}
\answer{%
    $E = h\frac c\lambda = e U \implies \lambda = \frac{hc}{eU} = \frac{6{,}626 \cdot 10^{-34}\,\text{Дж}\cdot\text{с} \cdot 3 \cdot 10^{8}\,\frac{\text{м}}{\text{с}}}{1{,}6 \cdot 10^{-19}\,\text{Кл} \cdot 144\,\text{В}} \approx 8{,}63\,\text{нм}.$
}
\solutionspace{120pt}

\tasknumber{2}%
\task{%
    Лучше всего нейтронное излучение ослабляет вода: в 4 раза лучше бетона и в 3 раза лучше свинца.
    Толщина слоя половинного ослабления $\gamma$-излучения для воды равна $3\,\text{см}$.
    Во сколько раз ослабит нейтронное излучение слой воды толщиной $h = 30\,\text{см}$?
}

\variantsplitter

\addpersonalvariant{Александр Воробьев}

\tasknumber{1}%
\task{%
    Определите длину волны лучей, фотоны которых имеют энергию
    равную кинетической энергии электрона, ускоренного напряжением $55\,\text{В}$.
}
\answer{%
    $E = h\frac c\lambda = e U \implies \lambda = \frac{hc}{eU} = \frac{6{,}626 \cdot 10^{-34}\,\text{Дж}\cdot\text{с} \cdot 3 \cdot 10^{8}\,\frac{\text{м}}{\text{с}}}{1{,}6 \cdot 10^{-19}\,\text{Кл} \cdot 55\,\text{В}} \approx 22{,}6\,\text{нм}.$
}
\solutionspace{120pt}

\tasknumber{2}%
\task{%
    Лучше всего нейтронное излучение ослабляет вода: в 4 раза лучше бетона и в 3 раза лучше свинца.
    Толщина слоя половинного ослабления $\gamma$-излучения для воды равна $3\,\text{см}$.
    Во сколько раз ослабит нейтронное излучение слой воды толщиной $d = 120\,\text{см}$?
}

\variantsplitter

\addpersonalvariant{Юлия Глотова}

\tasknumber{1}%
\task{%
    Определите длину волны лучей, фотоны которых имеют энергию
    равную кинетической энергии электрона, ускоренного напряжением $377\,\text{В}$.
}
\answer{%
    $E = h\frac c\lambda = e U \implies \lambda = \frac{hc}{eU} = \frac{6{,}626 \cdot 10^{-34}\,\text{Дж}\cdot\text{с} \cdot 3 \cdot 10^{8}\,\frac{\text{м}}{\text{с}}}{1{,}6 \cdot 10^{-19}\,\text{Кл} \cdot 377\,\text{В}} \approx 3{,}30\,\text{нм}.$
}
\solutionspace{120pt}

\tasknumber{2}%
\task{%
    Лучше всего нейтронное излучение ослабляет вода: в 4 раза лучше бетона и в 3 раза лучше свинца.
    Толщина слоя половинного ослабления $\gamma$-излучения для воды равна $3\,\text{см}$.
    Во сколько раз ослабит нейтронное излучение слой воды толщиной $h = 30\,\text{см}$?
}

\variantsplitter

\addpersonalvariant{Александр Гришков}

\tasknumber{1}%
\task{%
    Определите длину волны лучей, фотоны которых имеют энергию
    равную кинетической энергии электрона, ускоренного напряжением $21\,\text{В}$.
}
\answer{%
    $E = h\frac c\lambda = e U \implies \lambda = \frac{hc}{eU} = \frac{6{,}626 \cdot 10^{-34}\,\text{Дж}\cdot\text{с} \cdot 3 \cdot 10^{8}\,\frac{\text{м}}{\text{с}}}{1{,}6 \cdot 10^{-19}\,\text{Кл} \cdot 21\,\text{В}} \approx 59{,}2\,\text{нм}.$
}
\solutionspace{120pt}

\tasknumber{2}%
\task{%
    Лучше всего нейтронное излучение ослабляет вода: в 4 раза лучше бетона и в 3 раза лучше свинца.
    Толщина слоя половинного ослабления $\gamma$-излучения для воды равна $3\,\text{см}$.
    Во сколько раз ослабит нейтронное излучение слой воды толщиной $d = 60\,\text{см}$?
}

\variantsplitter

\addpersonalvariant{Валерия Жмурина}

\tasknumber{1}%
\task{%
    Определите длину волны лучей, фотоны которых имеют энергию
    равную кинетической энергии электрона, ускоренного напряжением $233\,\text{В}$.
}
\answer{%
    $E = h\frac c\lambda = e U \implies \lambda = \frac{hc}{eU} = \frac{6{,}626 \cdot 10^{-34}\,\text{Дж}\cdot\text{с} \cdot 3 \cdot 10^{8}\,\frac{\text{м}}{\text{с}}}{1{,}6 \cdot 10^{-19}\,\text{Кл} \cdot 233\,\text{В}} \approx 5{,}33\,\text{нм}.$
}
\solutionspace{120pt}

\tasknumber{2}%
\task{%
    Лучше всего нейтронное излучение ослабляет вода: в 4 раза лучше бетона и в 3 раза лучше свинца.
    Толщина слоя половинного ослабления $\gamma$-излучения для воды равна $3\,\text{см}$.
    Во сколько раз ослабит нейтронное излучение слой воды толщиной $d = 15\,\text{см}$?
}

\variantsplitter

\addpersonalvariant{Камиля Измайлова}

\tasknumber{1}%
\task{%
    Определите длину волны лучей, фотоны которых имеют энергию
    равную кинетической энергии электрона, ускоренного напряжением $5\,\text{В}$.
}
\answer{%
    $E = h\frac c\lambda = e U \implies \lambda = \frac{hc}{eU} = \frac{6{,}626 \cdot 10^{-34}\,\text{Дж}\cdot\text{с} \cdot 3 \cdot 10^{8}\,\frac{\text{м}}{\text{с}}}{1{,}6 \cdot 10^{-19}\,\text{Кл} \cdot 5\,\text{В}} \approx 250\,\text{нм}.$
}
\solutionspace{120pt}

\tasknumber{2}%
\task{%
    Лучше всего нейтронное излучение ослабляет вода: в 4 раза лучше бетона и в 3 раза лучше свинца.
    Толщина слоя половинного ослабления $\gamma$-излучения для воды равна $3\,\text{см}$.
    Во сколько раз ослабит нейтронное излучение слой воды толщиной $d = 120\,\text{см}$?
}

\variantsplitter

\addpersonalvariant{Константин Ичанский}

\tasknumber{1}%
\task{%
    Определите длину волны лучей, фотоны которых имеют энергию
    равную кинетической энергии электрона, ускоренного напряжением $1\,\text{В}$.
}
\answer{%
    $E = h\frac c\lambda = e U \implies \lambda = \frac{hc}{eU} = \frac{6{,}626 \cdot 10^{-34}\,\text{Дж}\cdot\text{с} \cdot 3 \cdot 10^{8}\,\frac{\text{м}}{\text{с}}}{1{,}6 \cdot 10^{-19}\,\text{Кл} \cdot 1\,\text{В}} \approx 1240\,\text{нм}.$
}
\solutionspace{120pt}

\tasknumber{2}%
\task{%
    Лучше всего нейтронное излучение ослабляет вода: в 4 раза лучше бетона и в 3 раза лучше свинца.
    Толщина слоя половинного ослабления $\gamma$-излучения для воды равна $3\,\text{см}$.
    Во сколько раз ослабит нейтронное излучение слой воды толщиной $d = 60\,\text{см}$?
}

\variantsplitter

\addpersonalvariant{Алексей Карчава}

\tasknumber{1}%
\task{%
    Определите длину волны лучей, фотоны которых имеют энергию
    равную кинетической энергии электрона, ускоренного напряжением $144\,\text{В}$.
}
\answer{%
    $E = h\frac c\lambda = e U \implies \lambda = \frac{hc}{eU} = \frac{6{,}626 \cdot 10^{-34}\,\text{Дж}\cdot\text{с} \cdot 3 \cdot 10^{8}\,\frac{\text{м}}{\text{с}}}{1{,}6 \cdot 10^{-19}\,\text{Кл} \cdot 144\,\text{В}} \approx 8{,}63\,\text{нм}.$
}
\solutionspace{120pt}

\tasknumber{2}%
\task{%
    Лучше всего нейтронное излучение ослабляет вода: в 4 раза лучше бетона и в 3 раза лучше свинца.
    Толщина слоя половинного ослабления $\gamma$-излучения для воды равна $3\,\text{см}$.
    Во сколько раз ослабит нейтронное излучение слой воды толщиной $d = 120\,\text{см}$?
}

\variantsplitter

\addpersonalvariant{Данил Колобашкин}

\tasknumber{1}%
\task{%
    Определите длину волны лучей, фотоны которых имеют энергию
    равную кинетической энергии электрона, ускоренного напряжением $34\,\text{В}$.
}
\answer{%
    $E = h\frac c\lambda = e U \implies \lambda = \frac{hc}{eU} = \frac{6{,}626 \cdot 10^{-34}\,\text{Дж}\cdot\text{с} \cdot 3 \cdot 10^{8}\,\frac{\text{м}}{\text{с}}}{1{,}6 \cdot 10^{-19}\,\text{Кл} \cdot 34\,\text{В}} \approx 36{,}5\,\text{нм}.$
}
\solutionspace{120pt}

\tasknumber{2}%
\task{%
    Лучше всего нейтронное излучение ослабляет вода: в 4 раза лучше бетона и в 3 раза лучше свинца.
    Толщина слоя половинного ослабления $\gamma$-излучения для воды равна $3\,\text{см}$.
    Во сколько раз ослабит нейтронное излучение слой воды толщиной $d = 60\,\text{см}$?
}

\variantsplitter

\addpersonalvariant{Анастасия Межова}

\tasknumber{1}%
\task{%
    Определите длину волны лучей, фотоны которых имеют энергию
    равную кинетической энергии электрона, ускоренного напряжением $144\,\text{В}$.
}
\answer{%
    $E = h\frac c\lambda = e U \implies \lambda = \frac{hc}{eU} = \frac{6{,}626 \cdot 10^{-34}\,\text{Дж}\cdot\text{с} \cdot 3 \cdot 10^{8}\,\frac{\text{м}}{\text{с}}}{1{,}6 \cdot 10^{-19}\,\text{Кл} \cdot 144\,\text{В}} \approx 8{,}63\,\text{нм}.$
}
\solutionspace{120pt}

\tasknumber{2}%
\task{%
    Лучше всего нейтронное излучение ослабляет вода: в 4 раза лучше бетона и в 3 раза лучше свинца.
    Толщина слоя половинного ослабления $\gamma$-излучения для воды равна $3\,\text{см}$.
    Во сколько раз ослабит нейтронное излучение слой воды толщиной $d = 15\,\text{см}$?
}

\variantsplitter

\addpersonalvariant{Роман Мигдисов}

\tasknumber{1}%
\task{%
    Определите длину волны лучей, фотоны которых имеют энергию
    равную кинетической энергии электрона, ускоренного напряжением $144\,\text{В}$.
}
\answer{%
    $E = h\frac c\lambda = e U \implies \lambda = \frac{hc}{eU} = \frac{6{,}626 \cdot 10^{-34}\,\text{Дж}\cdot\text{с} \cdot 3 \cdot 10^{8}\,\frac{\text{м}}{\text{с}}}{1{,}6 \cdot 10^{-19}\,\text{Кл} \cdot 144\,\text{В}} \approx 8{,}63\,\text{нм}.$
}
\solutionspace{120pt}

\tasknumber{2}%
\task{%
    Лучше всего нейтронное излучение ослабляет вода: в 4 раза лучше бетона и в 3 раза лучше свинца.
    Толщина слоя половинного ослабления $\gamma$-излучения для воды равна $3\,\text{см}$.
    Во сколько раз ослабит нейтронное излучение слой воды толщиной $d = 120\,\text{см}$?
}

\variantsplitter

\addpersonalvariant{Валерия Никулина}

\tasknumber{1}%
\task{%
    Определите длину волны лучей, фотоны которых имеют энергию
    равную кинетической энергии электрона, ускоренного напряжением $3\,\text{В}$.
}
\answer{%
    $E = h\frac c\lambda = e U \implies \lambda = \frac{hc}{eU} = \frac{6{,}626 \cdot 10^{-34}\,\text{Дж}\cdot\text{с} \cdot 3 \cdot 10^{8}\,\frac{\text{м}}{\text{с}}}{1{,}6 \cdot 10^{-19}\,\text{Кл} \cdot 3\,\text{В}} \approx 410\,\text{нм}.$
}
\solutionspace{120pt}

\tasknumber{2}%
\task{%
    Лучше всего нейтронное излучение ослабляет вода: в 4 раза лучше бетона и в 3 раза лучше свинца.
    Толщина слоя половинного ослабления $\gamma$-излучения для воды равна $3\,\text{см}$.
    Во сколько раз ослабит нейтронное излучение слой воды толщиной $h = 15\,\text{см}$?
}

\variantsplitter

\addpersonalvariant{Даниил Пахомов}

\tasknumber{1}%
\task{%
    Определите длину волны лучей, фотоны которых имеют энергию
    равную кинетической энергии электрона, ускоренного напряжением $8\,\text{В}$.
}
\answer{%
    $E = h\frac c\lambda = e U \implies \lambda = \frac{hc}{eU} = \frac{6{,}626 \cdot 10^{-34}\,\text{Дж}\cdot\text{с} \cdot 3 \cdot 10^{8}\,\frac{\text{м}}{\text{с}}}{1{,}6 \cdot 10^{-19}\,\text{Кл} \cdot 8\,\text{В}} \approx 155\,\text{нм}.$
}
\solutionspace{120pt}

\tasknumber{2}%
\task{%
    Лучше всего нейтронное излучение ослабляет вода: в 4 раза лучше бетона и в 3 раза лучше свинца.
    Толщина слоя половинного ослабления $\gamma$-излучения для воды равна $3\,\text{см}$.
    Во сколько раз ослабит нейтронное излучение слой воды толщиной $h = 30\,\text{см}$?
}

\variantsplitter

\addpersonalvariant{Дарья Рогова}

\tasknumber{1}%
\task{%
    Определите длину волны лучей, фотоны которых имеют энергию
    равную кинетической энергии электрона, ускоренного напряжением $13\,\text{В}$.
}
\answer{%
    $E = h\frac c\lambda = e U \implies \lambda = \frac{hc}{eU} = \frac{6{,}626 \cdot 10^{-34}\,\text{Дж}\cdot\text{с} \cdot 3 \cdot 10^{8}\,\frac{\text{м}}{\text{с}}}{1{,}6 \cdot 10^{-19}\,\text{Кл} \cdot 13\,\text{В}} \approx 96\,\text{нм}.$
}
\solutionspace{120pt}

\tasknumber{2}%
\task{%
    Лучше всего нейтронное излучение ослабляет вода: в 4 раза лучше бетона и в 3 раза лучше свинца.
    Толщина слоя половинного ослабления $\gamma$-излучения для воды равна $3\,\text{см}$.
    Во сколько раз ослабит нейтронное излучение слой воды толщиной $h = 15\,\text{см}$?
}

\variantsplitter

\addpersonalvariant{Валерия Румянцева}

\tasknumber{1}%
\task{%
    Определите длину волны лучей, фотоны которых имеют энергию
    равную кинетической энергии электрона, ускоренного напряжением $2\,\text{В}$.
}
\answer{%
    $E = h\frac c\lambda = e U \implies \lambda = \frac{hc}{eU} = \frac{6{,}626 \cdot 10^{-34}\,\text{Дж}\cdot\text{с} \cdot 3 \cdot 10^{8}\,\frac{\text{м}}{\text{с}}}{1{,}6 \cdot 10^{-19}\,\text{Кл} \cdot 2\,\text{В}} \approx 620\,\text{нм}.$
}
\solutionspace{120pt}

\tasknumber{2}%
\task{%
    Лучше всего нейтронное излучение ослабляет вода: в 4 раза лучше бетона и в 3 раза лучше свинца.
    Толщина слоя половинного ослабления $\gamma$-излучения для воды равна $3\,\text{см}$.
    Во сколько раз ослабит нейтронное излучение слой воды толщиной $l = 30\,\text{см}$?
}

\variantsplitter

\addpersonalvariant{Светлана Румянцева}

\tasknumber{1}%
\task{%
    Определите длину волны лучей, фотоны которых имеют энергию
    равную кинетической энергии электрона, ускоренного напряжением $34\,\text{В}$.
}
\answer{%
    $E = h\frac c\lambda = e U \implies \lambda = \frac{hc}{eU} = \frac{6{,}626 \cdot 10^{-34}\,\text{Дж}\cdot\text{с} \cdot 3 \cdot 10^{8}\,\frac{\text{м}}{\text{с}}}{1{,}6 \cdot 10^{-19}\,\text{Кл} \cdot 34\,\text{В}} \approx 36{,}5\,\text{нм}.$
}
\solutionspace{120pt}

\tasknumber{2}%
\task{%
    Лучше всего нейтронное излучение ослабляет вода: в 4 раза лучше бетона и в 3 раза лучше свинца.
    Толщина слоя половинного ослабления $\gamma$-излучения для воды равна $3\,\text{см}$.
    Во сколько раз ослабит нейтронное излучение слой воды толщиной $l = 15\,\text{см}$?
}

\variantsplitter

\addpersonalvariant{Назар Сабинов}

\tasknumber{1}%
\task{%
    Определите длину волны лучей, фотоны которых имеют энергию
    равную кинетической энергии электрона, ускоренного напряжением $5\,\text{В}$.
}
\answer{%
    $E = h\frac c\lambda = e U \implies \lambda = \frac{hc}{eU} = \frac{6{,}626 \cdot 10^{-34}\,\text{Дж}\cdot\text{с} \cdot 3 \cdot 10^{8}\,\frac{\text{м}}{\text{с}}}{1{,}6 \cdot 10^{-19}\,\text{Кл} \cdot 5\,\text{В}} \approx 250\,\text{нм}.$
}
\solutionspace{120pt}

\tasknumber{2}%
\task{%
    Лучше всего нейтронное излучение ослабляет вода: в 4 раза лучше бетона и в 3 раза лучше свинца.
    Толщина слоя половинного ослабления $\gamma$-излучения для воды равна $3\,\text{см}$.
    Во сколько раз ослабит нейтронное излучение слой воды толщиной $l = 15\,\text{см}$?
}

\variantsplitter

\addpersonalvariant{Михаил Тетерин}

\tasknumber{1}%
\task{%
    Определите длину волны лучей, фотоны которых имеют энергию
    равную кинетической энергии электрона, ускоренного напряжением $1\,\text{В}$.
}
\answer{%
    $E = h\frac c\lambda = e U \implies \lambda = \frac{hc}{eU} = \frac{6{,}626 \cdot 10^{-34}\,\text{Дж}\cdot\text{с} \cdot 3 \cdot 10^{8}\,\frac{\text{м}}{\text{с}}}{1{,}6 \cdot 10^{-19}\,\text{Кл} \cdot 1\,\text{В}} \approx 1240\,\text{нм}.$
}
\solutionspace{120pt}

\tasknumber{2}%
\task{%
    Лучше всего нейтронное излучение ослабляет вода: в 4 раза лучше бетона и в 3 раза лучше свинца.
    Толщина слоя половинного ослабления $\gamma$-излучения для воды равна $3\,\text{см}$.
    Во сколько раз ослабит нейтронное излучение слой воды толщиной $d = 60\,\text{см}$?
}

\variantsplitter

\addpersonalvariant{Арсланхан Уматалиев}

\tasknumber{1}%
\task{%
    Определите длину волны лучей, фотоны которых имеют энергию
    равную кинетической энергии электрона, ускоренного напряжением $13\,\text{В}$.
}
\answer{%
    $E = h\frac c\lambda = e U \implies \lambda = \frac{hc}{eU} = \frac{6{,}626 \cdot 10^{-34}\,\text{Дж}\cdot\text{с} \cdot 3 \cdot 10^{8}\,\frac{\text{м}}{\text{с}}}{1{,}6 \cdot 10^{-19}\,\text{Кл} \cdot 13\,\text{В}} \approx 96\,\text{нм}.$
}
\solutionspace{120pt}

\tasknumber{2}%
\task{%
    Лучше всего нейтронное излучение ослабляет вода: в 4 раза лучше бетона и в 3 раза лучше свинца.
    Толщина слоя половинного ослабления $\gamma$-излучения для воды равна $3\,\text{см}$.
    Во сколько раз ослабит нейтронное излучение слой воды толщиной $h = 120\,\text{см}$?
}

\variantsplitter

\addpersonalvariant{Дарья Холодная}

\tasknumber{1}%
\task{%
    Определите длину волны лучей, фотоны которых имеют энергию
    равную кинетической энергии электрона, ускоренного напряжением $144\,\text{В}$.
}
\answer{%
    $E = h\frac c\lambda = e U \implies \lambda = \frac{hc}{eU} = \frac{6{,}626 \cdot 10^{-34}\,\text{Дж}\cdot\text{с} \cdot 3 \cdot 10^{8}\,\frac{\text{м}}{\text{с}}}{1{,}6 \cdot 10^{-19}\,\text{Кл} \cdot 144\,\text{В}} \approx 8{,}63\,\text{нм}.$
}
\solutionspace{120pt}

\tasknumber{2}%
\task{%
    Лучше всего нейтронное излучение ослабляет вода: в 4 раза лучше бетона и в 3 раза лучше свинца.
    Толщина слоя половинного ослабления $\gamma$-излучения для воды равна $3\,\text{см}$.
    Во сколько раз ослабит нейтронное излучение слой воды толщиной $l = 30\,\text{см}$?
}
% autogenerated
