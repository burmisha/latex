\setdate{30~апреля~2019}
\setclass{11«А»}

\addpersonalvariant{Михаил Бурмистров}

\tasknumber{1}%
\task{%
    Определите длину волны лучей, фотоны которых имеют энергию
    равную кинетической энергии электрона, ускоренного напряжением $233\,\text{В}$.
}
\solutionspace{120pt}

\tasknumber{2}%
\task{%
    Лучше всего нейтронное излучение ослабляет вода: в 4 раза лучше бетона и в 3 раза лучше свинца.
    Толщина слоя половинного ослабления $\gamma$-излучения для воды равна $3\,\text{см}$.
    Во сколько раз ослабит нейтронное излучение слой воды толщиной $h = 60\,\text{см}$?
}

\variantsplitter

\addpersonalvariant{Никита Бекасов}

\tasknumber{1}%
\task{%
    Определите длину волны лучей, фотоны которых имеют энергию
    равную кинетической энергии электрона, ускоренного напряжением $8\,\text{В}$.
}
\solutionspace{120pt}

\tasknumber{2}%
\task{%
    Лучше всего нейтронное излучение ослабляет вода: в 4 раза лучше бетона и в 3 раза лучше свинца.
    Толщина слоя половинного ослабления $\gamma$-излучения для воды равна $3\,\text{см}$.
    Во сколько раз ослабит нейтронное излучение слой воды толщиной $h = 120\,\text{см}$?
}

\variantsplitter

\addpersonalvariant{Ольга Борисова}

\tasknumber{1}%
\task{%
    Определите длину волны лучей, фотоны которых имеют энергию
    равную кинетической энергии электрона, ускоренного напряжением $144\,\text{В}$.
}
\solutionspace{120pt}

\tasknumber{2}%
\task{%
    Лучше всего нейтронное излучение ослабляет вода: в 4 раза лучше бетона и в 3 раза лучше свинца.
    Толщина слоя половинного ослабления $\gamma$-излучения для воды равна $3\,\text{см}$.
    Во сколько раз ослабит нейтронное излучение слой воды толщиной $h = 30\,\text{см}$?
}

\variantsplitter

\addpersonalvariant{Александр Воробьев}

\tasknumber{1}%
\task{%
    Определите длину волны лучей, фотоны которых имеют энергию
    равную кинетической энергии электрона, ускоренного напряжением $55\,\text{В}$.
}
\solutionspace{120pt}

\tasknumber{2}%
\task{%
    Лучше всего нейтронное излучение ослабляет вода: в 4 раза лучше бетона и в 3 раза лучше свинца.
    Толщина слоя половинного ослабления $\gamma$-излучения для воды равна $3\,\text{см}$.
    Во сколько раз ослабит нейтронное излучение слой воды толщиной $d = 120\,\text{см}$?
}

\variantsplitter

\addpersonalvariant{Юлия Глотова}

\tasknumber{1}%
\task{%
    Определите длину волны лучей, фотоны которых имеют энергию
    равную кинетической энергии электрона, ускоренного напряжением $377\,\text{В}$.
}
\solutionspace{120pt}

\tasknumber{2}%
\task{%
    Лучше всего нейтронное излучение ослабляет вода: в 4 раза лучше бетона и в 3 раза лучше свинца.
    Толщина слоя половинного ослабления $\gamma$-излучения для воды равна $3\,\text{см}$.
    Во сколько раз ослабит нейтронное излучение слой воды толщиной $h = 30\,\text{см}$?
}

\variantsplitter

\addpersonalvariant{Александр Гришков}

\tasknumber{1}%
\task{%
    Определите длину волны лучей, фотоны которых имеют энергию
    равную кинетической энергии электрона, ускоренного напряжением $21\,\text{В}$.
}
\solutionspace{120pt}

\tasknumber{2}%
\task{%
    Лучше всего нейтронное излучение ослабляет вода: в 4 раза лучше бетона и в 3 раза лучше свинца.
    Толщина слоя половинного ослабления $\gamma$-излучения для воды равна $3\,\text{см}$.
    Во сколько раз ослабит нейтронное излучение слой воды толщиной $d = 60\,\text{см}$?
}

\variantsplitter

\addpersonalvariant{Валерия Жмурина}

\tasknumber{1}%
\task{%
    Определите длину волны лучей, фотоны которых имеют энергию
    равную кинетической энергии электрона, ускоренного напряжением $233\,\text{В}$.
}
\solutionspace{120pt}

\tasknumber{2}%
\task{%
    Лучше всего нейтронное излучение ослабляет вода: в 4 раза лучше бетона и в 3 раза лучше свинца.
    Толщина слоя половинного ослабления $\gamma$-излучения для воды равна $3\,\text{см}$.
    Во сколько раз ослабит нейтронное излучение слой воды толщиной $d = 15\,\text{см}$?
}

\variantsplitter

\addpersonalvariant{Камиля Измайлова}

\tasknumber{1}%
\task{%
    Определите длину волны лучей, фотоны которых имеют энергию
    равную кинетической энергии электрона, ускоренного напряжением $5\,\text{В}$.
}
\solutionspace{120pt}

\tasknumber{2}%
\task{%
    Лучше всего нейтронное излучение ослабляет вода: в 4 раза лучше бетона и в 3 раза лучше свинца.
    Толщина слоя половинного ослабления $\gamma$-излучения для воды равна $3\,\text{см}$.
    Во сколько раз ослабит нейтронное излучение слой воды толщиной $d = 120\,\text{см}$?
}

\variantsplitter

\addpersonalvariant{Константин Ичанский}

\tasknumber{1}%
\task{%
    Определите длину волны лучей, фотоны которых имеют энергию
    равную кинетической энергии электрона, ускоренного напряжением $1\,\text{В}$.
}
\solutionspace{120pt}

\tasknumber{2}%
\task{%
    Лучше всего нейтронное излучение ослабляет вода: в 4 раза лучше бетона и в 3 раза лучше свинца.
    Толщина слоя половинного ослабления $\gamma$-излучения для воды равна $3\,\text{см}$.
    Во сколько раз ослабит нейтронное излучение слой воды толщиной $d = 60\,\text{см}$?
}

\variantsplitter

\addpersonalvariant{Алексей Карчава}

\tasknumber{1}%
\task{%
    Определите длину волны лучей, фотоны которых имеют энергию
    равную кинетической энергии электрона, ускоренного напряжением $144\,\text{В}$.
}
\solutionspace{120pt}

\tasknumber{2}%
\task{%
    Лучше всего нейтронное излучение ослабляет вода: в 4 раза лучше бетона и в 3 раза лучше свинца.
    Толщина слоя половинного ослабления $\gamma$-излучения для воды равна $3\,\text{см}$.
    Во сколько раз ослабит нейтронное излучение слой воды толщиной $d = 120\,\text{см}$?
}

\variantsplitter

\addpersonalvariant{Данил Колобашкин}

\tasknumber{1}%
\task{%
    Определите длину волны лучей, фотоны которых имеют энергию
    равную кинетической энергии электрона, ускоренного напряжением $34\,\text{В}$.
}
\solutionspace{120pt}

\tasknumber{2}%
\task{%
    Лучше всего нейтронное излучение ослабляет вода: в 4 раза лучше бетона и в 3 раза лучше свинца.
    Толщина слоя половинного ослабления $\gamma$-излучения для воды равна $3\,\text{см}$.
    Во сколько раз ослабит нейтронное излучение слой воды толщиной $d = 60\,\text{см}$?
}

\variantsplitter

\addpersonalvariant{Анастасия Межова}

\tasknumber{1}%
\task{%
    Определите длину волны лучей, фотоны которых имеют энергию
    равную кинетической энергии электрона, ускоренного напряжением $144\,\text{В}$.
}
\solutionspace{120pt}

\tasknumber{2}%
\task{%
    Лучше всего нейтронное излучение ослабляет вода: в 4 раза лучше бетона и в 3 раза лучше свинца.
    Толщина слоя половинного ослабления $\gamma$-излучения для воды равна $3\,\text{см}$.
    Во сколько раз ослабит нейтронное излучение слой воды толщиной $d = 15\,\text{см}$?
}

\variantsplitter

\addpersonalvariant{Роман Мигдисов}

\tasknumber{1}%
\task{%
    Определите длину волны лучей, фотоны которых имеют энергию
    равную кинетической энергии электрона, ускоренного напряжением $144\,\text{В}$.
}
\solutionspace{120pt}

\tasknumber{2}%
\task{%
    Лучше всего нейтронное излучение ослабляет вода: в 4 раза лучше бетона и в 3 раза лучше свинца.
    Толщина слоя половинного ослабления $\gamma$-излучения для воды равна $3\,\text{см}$.
    Во сколько раз ослабит нейтронное излучение слой воды толщиной $d = 120\,\text{см}$?
}

\variantsplitter

\addpersonalvariant{Валерия Никулина}

\tasknumber{1}%
\task{%
    Определите длину волны лучей, фотоны которых имеют энергию
    равную кинетической энергии электрона, ускоренного напряжением $3\,\text{В}$.
}
\solutionspace{120pt}

\tasknumber{2}%
\task{%
    Лучше всего нейтронное излучение ослабляет вода: в 4 раза лучше бетона и в 3 раза лучше свинца.
    Толщина слоя половинного ослабления $\gamma$-излучения для воды равна $3\,\text{см}$.
    Во сколько раз ослабит нейтронное излучение слой воды толщиной $h = 15\,\text{см}$?
}

\variantsplitter

\addpersonalvariant{Даниил Пахомов}

\tasknumber{1}%
\task{%
    Определите длину волны лучей, фотоны которых имеют энергию
    равную кинетической энергии электрона, ускоренного напряжением $8\,\text{В}$.
}
\solutionspace{120pt}

\tasknumber{2}%
\task{%
    Лучше всего нейтронное излучение ослабляет вода: в 4 раза лучше бетона и в 3 раза лучше свинца.
    Толщина слоя половинного ослабления $\gamma$-излучения для воды равна $3\,\text{см}$.
    Во сколько раз ослабит нейтронное излучение слой воды толщиной $h = 30\,\text{см}$?
}

\variantsplitter

\addpersonalvariant{Дарья Рогова}

\tasknumber{1}%
\task{%
    Определите длину волны лучей, фотоны которых имеют энергию
    равную кинетической энергии электрона, ускоренного напряжением $13\,\text{В}$.
}
\solutionspace{120pt}

\tasknumber{2}%
\task{%
    Лучше всего нейтронное излучение ослабляет вода: в 4 раза лучше бетона и в 3 раза лучше свинца.
    Толщина слоя половинного ослабления $\gamma$-излучения для воды равна $3\,\text{см}$.
    Во сколько раз ослабит нейтронное излучение слой воды толщиной $h = 15\,\text{см}$?
}

\variantsplitter

\addpersonalvariant{Валерия Румянцева}

\tasknumber{1}%
\task{%
    Определите длину волны лучей, фотоны которых имеют энергию
    равную кинетической энергии электрона, ускоренного напряжением $2\,\text{В}$.
}
\solutionspace{120pt}

\tasknumber{2}%
\task{%
    Лучше всего нейтронное излучение ослабляет вода: в 4 раза лучше бетона и в 3 раза лучше свинца.
    Толщина слоя половинного ослабления $\gamma$-излучения для воды равна $3\,\text{см}$.
    Во сколько раз ослабит нейтронное излучение слой воды толщиной $l = 30\,\text{см}$?
}

\variantsplitter

\addpersonalvariant{Светлана Румянцева}

\tasknumber{1}%
\task{%
    Определите длину волны лучей, фотоны которых имеют энергию
    равную кинетической энергии электрона, ускоренного напряжением $34\,\text{В}$.
}
\solutionspace{120pt}

\tasknumber{2}%
\task{%
    Лучше всего нейтронное излучение ослабляет вода: в 4 раза лучше бетона и в 3 раза лучше свинца.
    Толщина слоя половинного ослабления $\gamma$-излучения для воды равна $3\,\text{см}$.
    Во сколько раз ослабит нейтронное излучение слой воды толщиной $l = 15\,\text{см}$?
}

\variantsplitter

\addpersonalvariant{Назар Сабинов}

\tasknumber{1}%
\task{%
    Определите длину волны лучей, фотоны которых имеют энергию
    равную кинетической энергии электрона, ускоренного напряжением $5\,\text{В}$.
}
\solutionspace{120pt}

\tasknumber{2}%
\task{%
    Лучше всего нейтронное излучение ослабляет вода: в 4 раза лучше бетона и в 3 раза лучше свинца.
    Толщина слоя половинного ослабления $\gamma$-излучения для воды равна $3\,\text{см}$.
    Во сколько раз ослабит нейтронное излучение слой воды толщиной $l = 15\,\text{см}$?
}

\variantsplitter

\addpersonalvariant{Михаил Тетерин}

\tasknumber{1}%
\task{%
    Определите длину волны лучей, фотоны которых имеют энергию
    равную кинетической энергии электрона, ускоренного напряжением $1\,\text{В}$.
}
\solutionspace{120pt}

\tasknumber{2}%
\task{%
    Лучше всего нейтронное излучение ослабляет вода: в 4 раза лучше бетона и в 3 раза лучше свинца.
    Толщина слоя половинного ослабления $\gamma$-излучения для воды равна $3\,\text{см}$.
    Во сколько раз ослабит нейтронное излучение слой воды толщиной $d = 60\,\text{см}$?
}

\variantsplitter

\addpersonalvariant{Арсланхан Уматалиев}

\tasknumber{1}%
\task{%
    Определите длину волны лучей, фотоны которых имеют энергию
    равную кинетической энергии электрона, ускоренного напряжением $13\,\text{В}$.
}
\solutionspace{120pt}

\tasknumber{2}%
\task{%
    Лучше всего нейтронное излучение ослабляет вода: в 4 раза лучше бетона и в 3 раза лучше свинца.
    Толщина слоя половинного ослабления $\gamma$-излучения для воды равна $3\,\text{см}$.
    Во сколько раз ослабит нейтронное излучение слой воды толщиной $h = 120\,\text{см}$?
}

\variantsplitter

\addpersonalvariant{Дарья Холодная}

\tasknumber{1}%
\task{%
    Определите длину волны лучей, фотоны которых имеют энергию
    равную кинетической энергии электрона, ускоренного напряжением $144\,\text{В}$.
}
\solutionspace{120pt}

\tasknumber{2}%
\task{%
    Лучше всего нейтронное излучение ослабляет вода: в 4 раза лучше бетона и в 3 раза лучше свинца.
    Толщина слоя половинного ослабления $\gamma$-излучения для воды равна $3\,\text{см}$.
    Во сколько раз ослабит нейтронное излучение слой воды толщиной $l = 30\,\text{см}$?
}
% autogenerated
