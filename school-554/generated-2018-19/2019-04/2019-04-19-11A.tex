\setdate{19~апреля~2019}
\setclass{11«А»}

\addpersonalvariant{Михаил Бурмистров}

\tasknumber{1}%
\task{%
    Сколько фотонов испускает за $5\,\text{мин}$ лазер,
    если мощность его излучения $40\,\text{мВт}$?
    Длина волны излучения $750\,\text{нм}$.
    $h = 6{,}626 \cdot 10^{-34}\,\text{Дж}\cdot\text{с}$.
}
\answer{%
    $
        N
            = \frac{E_{\text{общая}}}{E_{\text{одного фотона}}}
            = \frac{Pt}{h\nu} = \frac{Pt}{h \frac c\lambda}
            = \frac{Pt\lambda}{hc}
            = \frac{40\,\text{мВт} \cdot 5\,\text{мин} \cdot 750\,\text{нм}}{6{,}626 \cdot 10^{-34}\,\text{Дж}\cdot\text{с} \cdot 3 \cdot 10^{8}\,\frac{\text{м}}{\text{с}}}
            \approx 45{,}3 \cdot 10^{18}\units{фотонов}
    $
}
\solutionspace{120pt}

\tasknumber{2}%
\task{%
    В ядре электрически нейтрального атома 63 частиц.
    Вокруг ядра обращается 29 электронов.
    Сколько в ядре этого атома протонов и нейтронов?
    Назовите этот элемент.
}
\answer{%
    $Z = 29$ протонов и $A - Z = 34$ нейтронов, так что это \text{медь-63}: $\ce{^{63}_{29}{Cu}}$
}
\solutionspace{40pt}

\tasknumber{3}%
\task{%
    Запишите реакцию $\beta$-распада $\ce{^{22}_{11}{Na}}$.
}
\answer{%
    $\ce{^{22}_{11}{Na}} \to \ce{^{22}_{12}{Mg}} + e^- + \tilde\nu_e$
}
\solutionspace{80pt}

\tasknumber{4}%
\task{%
    Каков период полураспада радиоактивного изотопа,
    если за 12 ч в среднем распадается 7500 атомов из 8000?
}
\answer{%
    $
        N(t) = N_0 \cdot 2^{-\frac t{\tau_{\frac12}}}
        \implies \log_2\frac N{N_0} = - \frac t{\tau_\frac 12}
        \implies \tau_\frac 12 = - \frac t{\log_2\frac N{N_0}}
                                  =   \frac t{\log_2\frac{N_0}N}
        = \frac{
            12 \units{ч}
        }
        {
            \log_2\frac{8000}{8000 - 7500}
        }
        \approx 3\,\text{ч}.
    $
}

\variantsplitter

\addpersonalvariant{Никита Бекасов}

\tasknumber{1}%
\task{%
    Сколько фотонов испускает за $120\,\text{мин}$ лазер,
    если мощность его излучения $15\,\text{мВт}$?
    Длина волны излучения $600\,\text{нм}$.
    $h = 6{,}626 \cdot 10^{-34}\,\text{Дж}\cdot\text{с}$.
}
\answer{%
    $
        N
            = \frac{E_{\text{общая}}}{E_{\text{одного фотона}}}
            = \frac{Pt}{h\nu} = \frac{Pt}{h \frac c\lambda}
            = \frac{Pt\lambda}{hc}
            = \frac{15\,\text{мВт} \cdot 120\,\text{мин} \cdot 600\,\text{нм}}{6{,}626 \cdot 10^{-34}\,\text{Дж}\cdot\text{с} \cdot 3 \cdot 10^{8}\,\frac{\text{м}}{\text{с}}}
            \approx 326 \cdot 10^{18}\units{фотонов}
    $
}
\solutionspace{120pt}

\tasknumber{2}%
\task{%
    В ядре электрически нейтрального атома 123 частиц.
    Вокруг ядра обращается 51 электронов.
    Сколько в ядре этого атома протонов и нейтронов?
    Назовите этот элемент.
}
\answer{%
    $Z = 51$ протонов и $A - Z = 72$ нейтронов, так что это \text{сурьма-123}: $\ce{^{123}_{51}{Sb}}$
}
\solutionspace{40pt}

\tasknumber{3}%
\task{%
    Запишите реакцию $\alpha$-распада $\ce{^{153}_{63}{Eu}}$.
}
\answer{%
    $\ce{^{153}_{63}{Eu}} \to \ce{^{149}_{61}{Pm}} + \ce{^4_2{He}}$
}
\solutionspace{80pt}

\tasknumber{4}%
\task{%
    Каков период полураспада радиоактивного изотопа,
    если за 12 ч в среднем распадается 7500 атомов из 8000?
}
\answer{%
    $
        N(t) = N_0 \cdot 2^{-\frac t{\tau_{\frac12}}}
        \implies \log_2\frac N{N_0} = - \frac t{\tau_\frac 12}
        \implies \tau_\frac 12 = - \frac t{\log_2\frac N{N_0}}
                                  =   \frac t{\log_2\frac{N_0}N}
        = \frac{
            12 \units{ч}
        }
        {
            \log_2\frac{8000}{8000 - 7500}
        }
        \approx 3\,\text{ч}.
    $
}

\variantsplitter

\addpersonalvariant{Ольга Борисова}

\tasknumber{1}%
\task{%
    Сколько фотонов испускает за $10\,\text{мин}$ лазер,
    если мощность его излучения $200\,\text{мВт}$?
    Длина волны излучения $600\,\text{нм}$.
    $h = 6{,}626 \cdot 10^{-34}\,\text{Дж}\cdot\text{с}$.
}
\answer{%
    $
        N
            = \frac{E_{\text{общая}}}{E_{\text{одного фотона}}}
            = \frac{Pt}{h\nu} = \frac{Pt}{h \frac c\lambda}
            = \frac{Pt\lambda}{hc}
            = \frac{200\,\text{мВт} \cdot 10\,\text{мин} \cdot 600\,\text{нм}}{6{,}626 \cdot 10^{-34}\,\text{Дж}\cdot\text{с} \cdot 3 \cdot 10^{8}\,\frac{\text{м}}{\text{с}}}
            \approx 362 \cdot 10^{18}\units{фотонов}
    $
}
\solutionspace{120pt}

\tasknumber{2}%
\task{%
    В ядре электрически нейтрального атома 108 частиц.
    Вокруг ядра обращается 47 электронов.
    Сколько в ядре этого атома протонов и нейтронов?
    Назовите этот элемент.
}
\answer{%
    $Z = 47$ протонов и $A - Z = 61$ нейтронов, так что это \text{серебро-108}: $\ce{^{108}_{47}{Ag}}$
}
\solutionspace{40pt}

\tasknumber{3}%
\task{%
    Запишите реакцию $\alpha$-распада $\ce{^{144}_{60}{Nd}}$.
}
\answer{%
    $\ce{^{144}_{60}{Nd}} \to \ce{^{140}_{58}{Ce}} + \ce{^4_2{He}}$
}
\solutionspace{80pt}

\tasknumber{4}%
\task{%
    Каков период полураспада радиоактивного изотопа,
    если за 8 ч в среднем распадается 37500 атомов из 40000?
}
\answer{%
    $
        N(t) = N_0 \cdot 2^{-\frac t{\tau_{\frac12}}}
        \implies \log_2\frac N{N_0} = - \frac t{\tau_\frac 12}
        \implies \tau_\frac 12 = - \frac t{\log_2\frac N{N_0}}
                                  =   \frac t{\log_2\frac{N_0}N}
        = \frac{
            8 \units{ч}
        }
        {
            \log_2\frac{40000}{40000 - 37500}
        }
        \approx 2\,\text{ч}.
    $
}

\variantsplitter

\addpersonalvariant{Александр Воробьев}

\tasknumber{1}%
\task{%
    Сколько фотонов испускает за $5\,\text{мин}$ лазер,
    если мощность его излучения $15\,\text{мВт}$?
    Длина волны излучения $600\,\text{нм}$.
    $h = 6{,}626 \cdot 10^{-34}\,\text{Дж}\cdot\text{с}$.
}
\answer{%
    $
        N
            = \frac{E_{\text{общая}}}{E_{\text{одного фотона}}}
            = \frac{Pt}{h\nu} = \frac{Pt}{h \frac c\lambda}
            = \frac{Pt\lambda}{hc}
            = \frac{15\,\text{мВт} \cdot 5\,\text{мин} \cdot 600\,\text{нм}}{6{,}626 \cdot 10^{-34}\,\text{Дж}\cdot\text{с} \cdot 3 \cdot 10^{8}\,\frac{\text{м}}{\text{с}}}
            \approx 13{,}58 \cdot 10^{18}\units{фотонов}
    $
}
\solutionspace{120pt}

\tasknumber{2}%
\task{%
    В ядре электрически нейтрального атома 65 частиц.
    Вокруг ядра обращается 29 электронов.
    Сколько в ядре этого атома протонов и нейтронов?
    Назовите этот элемент.
}
\answer{%
    $Z = 29$ протонов и $A - Z = 36$ нейтронов, так что это \text{медь-65}: $\ce{^{65}_{29}{Cu}}$
}
\solutionspace{40pt}

\tasknumber{3}%
\task{%
    Запишите реакцию $\alpha$-распада $\ce{^{147}_{62}{Sm}}$.
}
\answer{%
    $\ce{^{147}_{62}{Sm}} \to \ce{^{143}_{60}{Nd}} + \ce{^4_2{He}}$
}
\solutionspace{80pt}

\tasknumber{4}%
\task{%
    Каков период полураспада радиоактивного изотопа,
    если за 6 ч в среднем распадается 3500 атомов из 4000?
}
\answer{%
    $
        N(t) = N_0 \cdot 2^{-\frac t{\tau_{\frac12}}}
        \implies \log_2\frac N{N_0} = - \frac t{\tau_\frac 12}
        \implies \tau_\frac 12 = - \frac t{\log_2\frac N{N_0}}
                                  =   \frac t{\log_2\frac{N_0}N}
        = \frac{
            6 \units{ч}
        }
        {
            \log_2\frac{4000}{4000 - 3500}
        }
        \approx 2\,\text{ч}.
    $
}

\variantsplitter

\addpersonalvariant{Юлия Глотова}

\tasknumber{1}%
\task{%
    Сколько фотонов испускает за $120\,\text{мин}$ лазер,
    если мощность его излучения $15\,\text{мВт}$?
    Длина волны излучения $500\,\text{нм}$.
    $h = 6{,}626 \cdot 10^{-34}\,\text{Дж}\cdot\text{с}$.
}
\answer{%
    $
        N
            = \frac{E_{\text{общая}}}{E_{\text{одного фотона}}}
            = \frac{Pt}{h\nu} = \frac{Pt}{h \frac c\lambda}
            = \frac{Pt\lambda}{hc}
            = \frac{15\,\text{мВт} \cdot 120\,\text{мин} \cdot 500\,\text{нм}}{6{,}626 \cdot 10^{-34}\,\text{Дж}\cdot\text{с} \cdot 3 \cdot 10^{8}\,\frac{\text{м}}{\text{с}}}
            \approx 272 \cdot 10^{18}\units{фотонов}
    $
}
\solutionspace{120pt}

\tasknumber{2}%
\task{%
    В ядре электрически нейтрального атома 63 частиц.
    Вокруг ядра обращается 29 электронов.
    Сколько в ядре этого атома протонов и нейтронов?
    Назовите этот элемент.
}
\answer{%
    $Z = 29$ протонов и $A - Z = 34$ нейтронов, так что это \text{медь-63}: $\ce{^{63}_{29}{Cu}}$
}
\solutionspace{40pt}

\tasknumber{3}%
\task{%
    Запишите реакцию $\alpha$-распада $\ce{^{147}_{62}{Sm}}$.
}
\answer{%
    $\ce{^{147}_{62}{Sm}} \to \ce{^{143}_{60}{Nd}} + \ce{^4_2{He}}$
}
\solutionspace{80pt}

\tasknumber{4}%
\task{%
    Каков период полураспада радиоактивного изотопа,
    если за 24 ч в среднем распадается 75000 атомов из 80000?
}
\answer{%
    $
        N(t) = N_0 \cdot 2^{-\frac t{\tau_{\frac12}}}
        \implies \log_2\frac N{N_0} = - \frac t{\tau_\frac 12}
        \implies \tau_\frac 12 = - \frac t{\log_2\frac N{N_0}}
                                  =   \frac t{\log_2\frac{N_0}N}
        = \frac{
            24 \units{ч}
        }
        {
            \log_2\frac{80000}{80000 - 75000}
        }
        \approx 6\,\text{ч}.
    $
}

\variantsplitter

\addpersonalvariant{Александр Гришков}

\tasknumber{1}%
\task{%
    Сколько фотонов испускает за $5\,\text{мин}$ лазер,
    если мощность его излучения $200\,\text{мВт}$?
    Длина волны излучения $500\,\text{нм}$.
    $h = 6{,}626 \cdot 10^{-34}\,\text{Дж}\cdot\text{с}$.
}
\answer{%
    $
        N
            = \frac{E_{\text{общая}}}{E_{\text{одного фотона}}}
            = \frac{Pt}{h\nu} = \frac{Pt}{h \frac c\lambda}
            = \frac{Pt\lambda}{hc}
            = \frac{200\,\text{мВт} \cdot 5\,\text{мин} \cdot 500\,\text{нм}}{6{,}626 \cdot 10^{-34}\,\text{Дж}\cdot\text{с} \cdot 3 \cdot 10^{8}\,\frac{\text{м}}{\text{с}}}
            \approx 150{,}9 \cdot 10^{18}\units{фотонов}
    $
}
\solutionspace{120pt}

\tasknumber{2}%
\task{%
    В ядре электрически нейтрального атома 65 частиц.
    Вокруг ядра обращается 29 электронов.
    Сколько в ядре этого атома протонов и нейтронов?
    Назовите этот элемент.
}
\answer{%
    $Z = 29$ протонов и $A - Z = 36$ нейтронов, так что это \text{медь-65}: $\ce{^{65}_{29}{Cu}}$
}
\solutionspace{40pt}

\tasknumber{3}%
\task{%
    Запишите реакцию $\alpha$-распада $\ce{^{144}_{60}{Nd}}$.
}
\answer{%
    $\ce{^{144}_{60}{Nd}} \to \ce{^{140}_{58}{Ce}} + \ce{^4_2{He}}$
}
\solutionspace{80pt}

\tasknumber{4}%
\task{%
    Каков период полураспада радиоактивного изотопа,
    если за 24 ч в среднем распадается 75000 атомов из 80000?
}
\answer{%
    $
        N(t) = N_0 \cdot 2^{-\frac t{\tau_{\frac12}}}
        \implies \log_2\frac N{N_0} = - \frac t{\tau_\frac 12}
        \implies \tau_\frac 12 = - \frac t{\log_2\frac N{N_0}}
                                  =   \frac t{\log_2\frac{N_0}N}
        = \frac{
            24 \units{ч}
        }
        {
            \log_2\frac{80000}{80000 - 75000}
        }
        \approx 6\,\text{ч}.
    $
}

\variantsplitter

\addpersonalvariant{Валерия Жмурина}

\tasknumber{1}%
\task{%
    Сколько фотонов испускает за $120\,\text{мин}$ лазер,
    если мощность его излучения $200\,\text{мВт}$?
    Длина волны излучения $500\,\text{нм}$.
    $h = 6{,}626 \cdot 10^{-34}\,\text{Дж}\cdot\text{с}$.
}
\answer{%
    $
        N
            = \frac{E_{\text{общая}}}{E_{\text{одного фотона}}}
            = \frac{Pt}{h\nu} = \frac{Pt}{h \frac c\lambda}
            = \frac{Pt\lambda}{hc}
            = \frac{200\,\text{мВт} \cdot 120\,\text{мин} \cdot 500\,\text{нм}}{6{,}626 \cdot 10^{-34}\,\text{Дж}\cdot\text{с} \cdot 3 \cdot 10^{8}\,\frac{\text{м}}{\text{с}}}
            \approx 3{,}622 \cdot 10^{21}\units{фотонов}
    $
}
\solutionspace{120pt}

\tasknumber{2}%
\task{%
    В ядре электрически нейтрального атома 65 частиц.
    Вокруг ядра обращается 29 электронов.
    Сколько в ядре этого атома протонов и нейтронов?
    Назовите этот элемент.
}
\answer{%
    $Z = 29$ протонов и $A - Z = 36$ нейтронов, так что это \text{медь-65}: $\ce{^{65}_{29}{Cu}}$
}
\solutionspace{40pt}

\tasknumber{3}%
\task{%
    Запишите реакцию $\alpha$-распада $\ce{^{238}_{92}{U}}$.
}
\answer{%
    $\ce{^{238}_{92}{U}} \to \ce{^{234}_{90}{Th}} + \ce{^4_2{He}}$
}
\solutionspace{80pt}

\tasknumber{4}%
\task{%
    Каков период полураспада радиоактивного изотопа,
    если за 8 ч в среднем распадается 300 атомов из 400?
}
\answer{%
    $
        N(t) = N_0 \cdot 2^{-\frac t{\tau_{\frac12}}}
        \implies \log_2\frac N{N_0} = - \frac t{\tau_\frac 12}
        \implies \tau_\frac 12 = - \frac t{\log_2\frac N{N_0}}
                                  =   \frac t{\log_2\frac{N_0}N}
        = \frac{
            8 \units{ч}
        }
        {
            \log_2\frac{400}{400 - 300}
        }
        \approx 4\,\text{ч}.
    $
}

\variantsplitter

\addpersonalvariant{Камиля Измайлова}

\tasknumber{1}%
\task{%
    Сколько фотонов испускает за $120\,\text{мин}$ лазер,
    если мощность его излучения $15\,\text{мВт}$?
    Длина волны излучения $600\,\text{нм}$.
    $h = 6{,}626 \cdot 10^{-34}\,\text{Дж}\cdot\text{с}$.
}
\answer{%
    $
        N
            = \frac{E_{\text{общая}}}{E_{\text{одного фотона}}}
            = \frac{Pt}{h\nu} = \frac{Pt}{h \frac c\lambda}
            = \frac{Pt\lambda}{hc}
            = \frac{15\,\text{мВт} \cdot 120\,\text{мин} \cdot 600\,\text{нм}}{6{,}626 \cdot 10^{-34}\,\text{Дж}\cdot\text{с} \cdot 3 \cdot 10^{8}\,\frac{\text{м}}{\text{с}}}
            \approx 326 \cdot 10^{18}\units{фотонов}
    $
}
\solutionspace{120pt}

\tasknumber{2}%
\task{%
    В ядре электрически нейтрального атома 63 частиц.
    Вокруг ядра обращается 29 электронов.
    Сколько в ядре этого атома протонов и нейтронов?
    Назовите этот элемент.
}
\answer{%
    $Z = 29$ протонов и $A - Z = 34$ нейтронов, так что это \text{медь-63}: $\ce{^{63}_{29}{Cu}}$
}
\solutionspace{40pt}

\tasknumber{3}%
\task{%
    Запишите реакцию $\beta$-распада $\ce{^{137}_{55}{Cs}}$.
}
\answer{%
    $\ce{^{137}_{55}{Cs}} \to \ce{^{137}_{56}{Ba}} + e^- + \tilde\nu_e$
}
\solutionspace{80pt}

\tasknumber{4}%
\task{%
    Каков период полураспада радиоактивного изотопа,
    если за 24 ч в среднем распадается 75000 атомов из 80000?
}
\answer{%
    $
        N(t) = N_0 \cdot 2^{-\frac t{\tau_{\frac12}}}
        \implies \log_2\frac N{N_0} = - \frac t{\tau_\frac 12}
        \implies \tau_\frac 12 = - \frac t{\log_2\frac N{N_0}}
                                  =   \frac t{\log_2\frac{N_0}N}
        = \frac{
            24 \units{ч}
        }
        {
            \log_2\frac{80000}{80000 - 75000}
        }
        \approx 6\,\text{ч}.
    $
}

\variantsplitter

\addpersonalvariant{Константин Ичанский}

\tasknumber{1}%
\task{%
    Сколько фотонов испускает за $40\,\text{мин}$ лазер,
    если мощность его излучения $15\,\text{мВт}$?
    Длина волны излучения $500\,\text{нм}$.
    $h = 6{,}626 \cdot 10^{-34}\,\text{Дж}\cdot\text{с}$.
}
\answer{%
    $
        N
            = \frac{E_{\text{общая}}}{E_{\text{одного фотона}}}
            = \frac{Pt}{h\nu} = \frac{Pt}{h \frac c\lambda}
            = \frac{Pt\lambda}{hc}
            = \frac{15\,\text{мВт} \cdot 40\,\text{мин} \cdot 500\,\text{нм}}{6{,}626 \cdot 10^{-34}\,\text{Дж}\cdot\text{с} \cdot 3 \cdot 10^{8}\,\frac{\text{м}}{\text{с}}}
            \approx 90{,}6 \cdot 10^{18}\units{фотонов}
    $
}
\solutionspace{120pt}

\tasknumber{2}%
\task{%
    В ядре электрически нейтрального атома 65 частиц.
    Вокруг ядра обращается 29 электронов.
    Сколько в ядре этого атома протонов и нейтронов?
    Назовите этот элемент.
}
\answer{%
    $Z = 29$ протонов и $A - Z = 36$ нейтронов, так что это \text{медь-65}: $\ce{^{65}_{29}{Cu}}$
}
\solutionspace{40pt}

\tasknumber{3}%
\task{%
    Запишите реакцию $\alpha$-распада $\ce{^{147}_{62}{Sm}}$.
}
\answer{%
    $\ce{^{147}_{62}{Sm}} \to \ce{^{143}_{60}{Nd}} + \ce{^4_2{He}}$
}
\solutionspace{80pt}

\tasknumber{4}%
\task{%
    Каков период полураспада радиоактивного изотопа,
    если за 8 ч в среднем распадается 300 атомов из 400?
}
\answer{%
    $
        N(t) = N_0 \cdot 2^{-\frac t{\tau_{\frac12}}}
        \implies \log_2\frac N{N_0} = - \frac t{\tau_\frac 12}
        \implies \tau_\frac 12 = - \frac t{\log_2\frac N{N_0}}
                                  =   \frac t{\log_2\frac{N_0}N}
        = \frac{
            8 \units{ч}
        }
        {
            \log_2\frac{400}{400 - 300}
        }
        \approx 4\,\text{ч}.
    $
}

\variantsplitter

\addpersonalvariant{Алексей Карчава}

\tasknumber{1}%
\task{%
    Сколько фотонов испускает за $60\,\text{мин}$ лазер,
    если мощность его излучения $75\,\text{мВт}$?
    Длина волны излучения $500\,\text{нм}$.
    $h = 6{,}626 \cdot 10^{-34}\,\text{Дж}\cdot\text{с}$.
}
\answer{%
    $
        N
            = \frac{E_{\text{общая}}}{E_{\text{одного фотона}}}
            = \frac{Pt}{h\nu} = \frac{Pt}{h \frac c\lambda}
            = \frac{Pt\lambda}{hc}
            = \frac{75\,\text{мВт} \cdot 60\,\text{мин} \cdot 500\,\text{нм}}{6{,}626 \cdot 10^{-34}\,\text{Дж}\cdot\text{с} \cdot 3 \cdot 10^{8}\,\frac{\text{м}}{\text{с}}}
            \approx 679{,}1 \cdot 10^{18}\units{фотонов}
    $
}
\solutionspace{120pt}

\tasknumber{2}%
\task{%
    В ядре электрически нейтрального атома 190 частиц.
    Вокруг ядра обращается 78 электронов.
    Сколько в ядре этого атома протонов и нейтронов?
    Назовите этот элемент.
}
\answer{%
    $Z = 78$ протонов и $A - Z = 112$ нейтронов, так что это \text{платина-190}: $\ce{^{190}_{78}{Pt}}$
}
\solutionspace{40pt}

\tasknumber{3}%
\task{%
    Запишите реакцию $\alpha$-распада $\ce{^{180}_{74}{W}}$.
}
\answer{%
    $\ce{^{180}_{74}{W}} \to \ce{^{176}_{72}{Hf}} + \ce{^4_2{He}}$
}
\solutionspace{80pt}

\tasknumber{4}%
\task{%
    Каков период полураспада радиоактивного изотопа,
    если за 12 ч в среднем распадается 7500 атомов из 8000?
}
\answer{%
    $
        N(t) = N_0 \cdot 2^{-\frac t{\tau_{\frac12}}}
        \implies \log_2\frac N{N_0} = - \frac t{\tau_\frac 12}
        \implies \tau_\frac 12 = - \frac t{\log_2\frac N{N_0}}
                                  =   \frac t{\log_2\frac{N_0}N}
        = \frac{
            12 \units{ч}
        }
        {
            \log_2\frac{8000}{8000 - 7500}
        }
        \approx 3\,\text{ч}.
    $
}

\variantsplitter

\addpersonalvariant{Данил Колобашкин}

\tasknumber{1}%
\task{%
    Сколько фотонов испускает за $20\,\text{мин}$ лазер,
    если мощность его излучения $15\,\text{мВт}$?
    Длина волны излучения $500\,\text{нм}$.
    $h = 6{,}626 \cdot 10^{-34}\,\text{Дж}\cdot\text{с}$.
}
\answer{%
    $
        N
            = \frac{E_{\text{общая}}}{E_{\text{одного фотона}}}
            = \frac{Pt}{h\nu} = \frac{Pt}{h \frac c\lambda}
            = \frac{Pt\lambda}{hc}
            = \frac{15\,\text{мВт} \cdot 20\,\text{мин} \cdot 500\,\text{нм}}{6{,}626 \cdot 10^{-34}\,\text{Дж}\cdot\text{с} \cdot 3 \cdot 10^{8}\,\frac{\text{м}}{\text{с}}}
            \approx 45{,}3 \cdot 10^{18}\units{фотонов}
    $
}
\solutionspace{120pt}

\tasknumber{2}%
\task{%
    В ядре электрически нейтрального атома 121 частиц.
    Вокруг ядра обращается 51 электронов.
    Сколько в ядре этого атома протонов и нейтронов?
    Назовите этот элемент.
}
\answer{%
    $Z = 51$ протонов и $A - Z = 70$ нейтронов, так что это \text{сурьма-121}: $\ce{^{121}_{51}{Sb}}$
}
\solutionspace{40pt}

\tasknumber{3}%
\task{%
    Запишите реакцию $\beta$-распада $\ce{^{22}_{11}{Na}}$.
}
\answer{%
    $\ce{^{22}_{11}{Na}} \to \ce{^{22}_{12}{Mg}} + e^- + \tilde\nu_e$
}
\solutionspace{80pt}

\tasknumber{4}%
\task{%
    Каков период полураспада радиоактивного изотопа,
    если за 6 ч в среднем распадается 3500 атомов из 4000?
}
\answer{%
    $
        N(t) = N_0 \cdot 2^{-\frac t{\tau_{\frac12}}}
        \implies \log_2\frac N{N_0} = - \frac t{\tau_\frac 12}
        \implies \tau_\frac 12 = - \frac t{\log_2\frac N{N_0}}
                                  =   \frac t{\log_2\frac{N_0}N}
        = \frac{
            6 \units{ч}
        }
        {
            \log_2\frac{4000}{4000 - 3500}
        }
        \approx 2\,\text{ч}.
    $
}

\variantsplitter

\addpersonalvariant{Анастасия Межова}

\tasknumber{1}%
\task{%
    Сколько фотонов испускает за $20\,\text{мин}$ лазер,
    если мощность его излучения $40\,\text{мВт}$?
    Длина волны излучения $750\,\text{нм}$.
    $h = 6{,}626 \cdot 10^{-34}\,\text{Дж}\cdot\text{с}$.
}
\answer{%
    $
        N
            = \frac{E_{\text{общая}}}{E_{\text{одного фотона}}}
            = \frac{Pt}{h\nu} = \frac{Pt}{h \frac c\lambda}
            = \frac{Pt\lambda}{hc}
            = \frac{40\,\text{мВт} \cdot 20\,\text{мин} \cdot 750\,\text{нм}}{6{,}626 \cdot 10^{-34}\,\text{Дж}\cdot\text{с} \cdot 3 \cdot 10^{8}\,\frac{\text{м}}{\text{с}}}
            \approx 181{,}10 \cdot 10^{18}\units{фотонов}
    $
}
\solutionspace{120pt}

\tasknumber{2}%
\task{%
    В ядре электрически нейтрального атома 63 частиц.
    Вокруг ядра обращается 29 электронов.
    Сколько в ядре этого атома протонов и нейтронов?
    Назовите этот элемент.
}
\answer{%
    $Z = 29$ протонов и $A - Z = 34$ нейтронов, так что это \text{медь-63}: $\ce{^{63}_{29}{Cu}}$
}
\solutionspace{40pt}

\tasknumber{3}%
\task{%
    Запишите реакцию $\alpha$-распада $\ce{^{238}_{92}{U}}$.
}
\answer{%
    $\ce{^{238}_{92}{U}} \to \ce{^{234}_{90}{Th}} + \ce{^4_2{He}}$
}
\solutionspace{80pt}

\tasknumber{4}%
\task{%
    Каков период полураспада радиоактивного изотопа,
    если за 8 ч в среднем распадается 300 атомов из 400?
}
\answer{%
    $
        N(t) = N_0 \cdot 2^{-\frac t{\tau_{\frac12}}}
        \implies \log_2\frac N{N_0} = - \frac t{\tau_\frac 12}
        \implies \tau_\frac 12 = - \frac t{\log_2\frac N{N_0}}
                                  =   \frac t{\log_2\frac{N_0}N}
        = \frac{
            8 \units{ч}
        }
        {
            \log_2\frac{400}{400 - 300}
        }
        \approx 4\,\text{ч}.
    $
}

\variantsplitter

\addpersonalvariant{Роман Мигдисов}

\tasknumber{1}%
\task{%
    Сколько фотонов испускает за $20\,\text{мин}$ лазер,
    если мощность его излучения $75\,\text{мВт}$?
    Длина волны излучения $750\,\text{нм}$.
    $h = 6{,}626 \cdot 10^{-34}\,\text{Дж}\cdot\text{с}$.
}
\answer{%
    $
        N
            = \frac{E_{\text{общая}}}{E_{\text{одного фотона}}}
            = \frac{Pt}{h\nu} = \frac{Pt}{h \frac c\lambda}
            = \frac{Pt\lambda}{hc}
            = \frac{75\,\text{мВт} \cdot 20\,\text{мин} \cdot 750\,\text{нм}}{6{,}626 \cdot 10^{-34}\,\text{Дж}\cdot\text{с} \cdot 3 \cdot 10^{8}\,\frac{\text{м}}{\text{с}}}
            \approx 339{,}6 \cdot 10^{18}\units{фотонов}
    $
}
\solutionspace{120pt}

\tasknumber{2}%
\task{%
    В ядре электрически нейтрального атома 63 частиц.
    Вокруг ядра обращается 29 электронов.
    Сколько в ядре этого атома протонов и нейтронов?
    Назовите этот элемент.
}
\answer{%
    $Z = 29$ протонов и $A - Z = 34$ нейтронов, так что это \text{медь-63}: $\ce{^{63}_{29}{Cu}}$
}
\solutionspace{40pt}

\tasknumber{3}%
\task{%
    Запишите реакцию $\alpha$-распада $\ce{^{148}_{62}{Sm}}$.
}
\answer{%
    $\ce{^{148}_{62}{Sm}} \to \ce{^{144}_{60}{Nd}} + \ce{^4_2{He}}$
}
\solutionspace{80pt}

\tasknumber{4}%
\task{%
    Каков период полураспада радиоактивного изотопа,
    если за 12 ч в среднем распадается 7500 атомов из 8000?
}
\answer{%
    $
        N(t) = N_0 \cdot 2^{-\frac t{\tau_{\frac12}}}
        \implies \log_2\frac N{N_0} = - \frac t{\tau_\frac 12}
        \implies \tau_\frac 12 = - \frac t{\log_2\frac N{N_0}}
                                  =   \frac t{\log_2\frac{N_0}N}
        = \frac{
            12 \units{ч}
        }
        {
            \log_2\frac{8000}{8000 - 7500}
        }
        \approx 3\,\text{ч}.
    $
}

\variantsplitter

\addpersonalvariant{Валерия Никулина}

\tasknumber{1}%
\task{%
    Сколько фотонов испускает за $30\,\text{мин}$ лазер,
    если мощность его излучения $15\,\text{мВт}$?
    Длина волны излучения $500\,\text{нм}$.
    $h = 6{,}626 \cdot 10^{-34}\,\text{Дж}\cdot\text{с}$.
}
\answer{%
    $
        N
            = \frac{E_{\text{общая}}}{E_{\text{одного фотона}}}
            = \frac{Pt}{h\nu} = \frac{Pt}{h \frac c\lambda}
            = \frac{Pt\lambda}{hc}
            = \frac{15\,\text{мВт} \cdot 30\,\text{мин} \cdot 500\,\text{нм}}{6{,}626 \cdot 10^{-34}\,\text{Дж}\cdot\text{с} \cdot 3 \cdot 10^{8}\,\frac{\text{м}}{\text{с}}}
            \approx 67{,}9 \cdot 10^{18}\units{фотонов}
    $
}
\solutionspace{120pt}

\tasknumber{2}%
\task{%
    В ядре электрически нейтрального атома 65 частиц.
    Вокруг ядра обращается 29 электронов.
    Сколько в ядре этого атома протонов и нейтронов?
    Назовите этот элемент.
}
\answer{%
    $Z = 29$ протонов и $A - Z = 36$ нейтронов, так что это \text{медь-65}: $\ce{^{65}_{29}{Cu}}$
}
\solutionspace{40pt}

\tasknumber{3}%
\task{%
    Запишите реакцию $\alpha$-распада $\ce{^{238}_{92}{U}}$.
}
\answer{%
    $\ce{^{238}_{92}{U}} \to \ce{^{234}_{90}{Th}} + \ce{^4_2{He}}$
}
\solutionspace{80pt}

\tasknumber{4}%
\task{%
    Каков период полураспада радиоактивного изотопа,
    если за 24 ч в среднем распадается 75000 атомов из 80000?
}
\answer{%
    $
        N(t) = N_0 \cdot 2^{-\frac t{\tau_{\frac12}}}
        \implies \log_2\frac N{N_0} = - \frac t{\tau_\frac 12}
        \implies \tau_\frac 12 = - \frac t{\log_2\frac N{N_0}}
                                  =   \frac t{\log_2\frac{N_0}N}
        = \frac{
            24 \units{ч}
        }
        {
            \log_2\frac{80000}{80000 - 75000}
        }
        \approx 6\,\text{ч}.
    $
}

\variantsplitter

\addpersonalvariant{Даниил Пахомов}

\tasknumber{1}%
\task{%
    Сколько фотонов испускает за $30\,\text{мин}$ лазер,
    если мощность его излучения $75\,\text{мВт}$?
    Длина волны излучения $600\,\text{нм}$.
    $h = 6{,}626 \cdot 10^{-34}\,\text{Дж}\cdot\text{с}$.
}
\answer{%
    $
        N
            = \frac{E_{\text{общая}}}{E_{\text{одного фотона}}}
            = \frac{Pt}{h\nu} = \frac{Pt}{h \frac c\lambda}
            = \frac{Pt\lambda}{hc}
            = \frac{75\,\text{мВт} \cdot 30\,\text{мин} \cdot 600\,\text{нм}}{6{,}626 \cdot 10^{-34}\,\text{Дж}\cdot\text{с} \cdot 3 \cdot 10^{8}\,\frac{\text{м}}{\text{с}}}
            \approx 407{,}5 \cdot 10^{18}\units{фотонов}
    $
}
\solutionspace{120pt}

\tasknumber{2}%
\task{%
    В ядре электрически нейтрального атома 63 частиц.
    Вокруг ядра обращается 29 электронов.
    Сколько в ядре этого атома протонов и нейтронов?
    Назовите этот элемент.
}
\answer{%
    $Z = 29$ протонов и $A - Z = 34$ нейтронов, так что это \text{медь-63}: $\ce{^{63}_{29}{Cu}}$
}
\solutionspace{40pt}

\tasknumber{3}%
\task{%
    Запишите реакцию $\alpha$-распада $\ce{^{148}_{62}{Sm}}$.
}
\answer{%
    $\ce{^{148}_{62}{Sm}} \to \ce{^{144}_{60}{Nd}} + \ce{^4_2{He}}$
}
\solutionspace{80pt}

\tasknumber{4}%
\task{%
    Каков период полураспада радиоактивного изотопа,
    если за 12 ч в среднем распадается 7500 атомов из 8000?
}
\answer{%
    $
        N(t) = N_0 \cdot 2^{-\frac t{\tau_{\frac12}}}
        \implies \log_2\frac N{N_0} = - \frac t{\tau_\frac 12}
        \implies \tau_\frac 12 = - \frac t{\log_2\frac N{N_0}}
                                  =   \frac t{\log_2\frac{N_0}N}
        = \frac{
            12 \units{ч}
        }
        {
            \log_2\frac{8000}{8000 - 7500}
        }
        \approx 3\,\text{ч}.
    $
}

\variantsplitter

\addpersonalvariant{Дарья Рогова}

\tasknumber{1}%
\task{%
    Сколько фотонов испускает за $30\,\text{мин}$ лазер,
    если мощность его излучения $15\,\text{мВт}$?
    Длина волны излучения $500\,\text{нм}$.
    $h = 6{,}626 \cdot 10^{-34}\,\text{Дж}\cdot\text{с}$.
}
\answer{%
    $
        N
            = \frac{E_{\text{общая}}}{E_{\text{одного фотона}}}
            = \frac{Pt}{h\nu} = \frac{Pt}{h \frac c\lambda}
            = \frac{Pt\lambda}{hc}
            = \frac{15\,\text{мВт} \cdot 30\,\text{мин} \cdot 500\,\text{нм}}{6{,}626 \cdot 10^{-34}\,\text{Дж}\cdot\text{с} \cdot 3 \cdot 10^{8}\,\frac{\text{м}}{\text{с}}}
            \approx 67{,}9 \cdot 10^{18}\units{фотонов}
    $
}
\solutionspace{120pt}

\tasknumber{2}%
\task{%
    В ядре электрически нейтрального атома 190 частиц.
    Вокруг ядра обращается 78 электронов.
    Сколько в ядре этого атома протонов и нейтронов?
    Назовите этот элемент.
}
\answer{%
    $Z = 78$ протонов и $A - Z = 112$ нейтронов, так что это \text{платина-190}: $\ce{^{190}_{78}{Pt}}$
}
\solutionspace{40pt}

\tasknumber{3}%
\task{%
    Запишите реакцию $\alpha$-распада $\ce{^{147}_{62}{Sm}}$.
}
\answer{%
    $\ce{^{147}_{62}{Sm}} \to \ce{^{143}_{60}{Nd}} + \ce{^4_2{He}}$
}
\solutionspace{80pt}

\tasknumber{4}%
\task{%
    Каков период полураспада радиоактивного изотопа,
    если за 8 ч в среднем распадается 37500 атомов из 40000?
}
\answer{%
    $
        N(t) = N_0 \cdot 2^{-\frac t{\tau_{\frac12}}}
        \implies \log_2\frac N{N_0} = - \frac t{\tau_\frac 12}
        \implies \tau_\frac 12 = - \frac t{\log_2\frac N{N_0}}
                                  =   \frac t{\log_2\frac{N_0}N}
        = \frac{
            8 \units{ч}
        }
        {
            \log_2\frac{40000}{40000 - 37500}
        }
        \approx 2\,\text{ч}.
    $
}

\variantsplitter

\addpersonalvariant{Валерия Румянцева}

\tasknumber{1}%
\task{%
    Сколько фотонов испускает за $120\,\text{мин}$ лазер,
    если мощность его излучения $75\,\text{мВт}$?
    Длина волны излучения $500\,\text{нм}$.
    $h = 6{,}626 \cdot 10^{-34}\,\text{Дж}\cdot\text{с}$.
}
\answer{%
    $
        N
            = \frac{E_{\text{общая}}}{E_{\text{одного фотона}}}
            = \frac{Pt}{h\nu} = \frac{Pt}{h \frac c\lambda}
            = \frac{Pt\lambda}{hc}
            = \frac{75\,\text{мВт} \cdot 120\,\text{мин} \cdot 500\,\text{нм}}{6{,}626 \cdot 10^{-34}\,\text{Дж}\cdot\text{с} \cdot 3 \cdot 10^{8}\,\frac{\text{м}}{\text{с}}}
            \approx 1{,}3583 \cdot 10^{21}\units{фотонов}
    $
}
\solutionspace{120pt}

\tasknumber{2}%
\task{%
    В ядре электрически нейтрального атома 190 частиц.
    Вокруг ядра обращается 78 электронов.
    Сколько в ядре этого атома протонов и нейтронов?
    Назовите этот элемент.
}
\answer{%
    $Z = 78$ протонов и $A - Z = 112$ нейтронов, так что это \text{платина-190}: $\ce{^{190}_{78}{Pt}}$
}
\solutionspace{40pt}

\tasknumber{3}%
\task{%
    Запишите реакцию $\beta$-распада $\ce{^{22}_{11}{Na}}$.
}
\answer{%
    $\ce{^{22}_{11}{Na}} \to \ce{^{22}_{12}{Mg}} + e^- + \tilde\nu_e$
}
\solutionspace{80pt}

\tasknumber{4}%
\task{%
    Каков период полураспада радиоактивного изотопа,
    если за 12 ч в среднем распадается 7500 атомов из 8000?
}
\answer{%
    $
        N(t) = N_0 \cdot 2^{-\frac t{\tau_{\frac12}}}
        \implies \log_2\frac N{N_0} = - \frac t{\tau_\frac 12}
        \implies \tau_\frac 12 = - \frac t{\log_2\frac N{N_0}}
                                  =   \frac t{\log_2\frac{N_0}N}
        = \frac{
            12 \units{ч}
        }
        {
            \log_2\frac{8000}{8000 - 7500}
        }
        \approx 3\,\text{ч}.
    $
}

\variantsplitter

\addpersonalvariant{Светлана Румянцева}

\tasknumber{1}%
\task{%
    Сколько фотонов испускает за $120\,\text{мин}$ лазер,
    если мощность его излучения $75\,\text{мВт}$?
    Длина волны излучения $600\,\text{нм}$.
    $h = 6{,}626 \cdot 10^{-34}\,\text{Дж}\cdot\text{с}$.
}
\answer{%
    $
        N
            = \frac{E_{\text{общая}}}{E_{\text{одного фотона}}}
            = \frac{Pt}{h\nu} = \frac{Pt}{h \frac c\lambda}
            = \frac{Pt\lambda}{hc}
            = \frac{75\,\text{мВт} \cdot 120\,\text{мин} \cdot 600\,\text{нм}}{6{,}626 \cdot 10^{-34}\,\text{Дж}\cdot\text{с} \cdot 3 \cdot 10^{8}\,\frac{\text{м}}{\text{с}}}
            \approx 1{,}6299 \cdot 10^{21}\units{фотонов}
    $
}
\solutionspace{120pt}

\tasknumber{2}%
\task{%
    В ядре электрически нейтрального атома 108 частиц.
    Вокруг ядра обращается 47 электронов.
    Сколько в ядре этого атома протонов и нейтронов?
    Назовите этот элемент.
}
\answer{%
    $Z = 47$ протонов и $A - Z = 61$ нейтронов, так что это \text{серебро-108}: $\ce{^{108}_{47}{Ag}}$
}
\solutionspace{40pt}

\tasknumber{3}%
\task{%
    Запишите реакцию $\alpha$-распада $\ce{^{147}_{62}{Sm}}$.
}
\answer{%
    $\ce{^{147}_{62}{Sm}} \to \ce{^{143}_{60}{Nd}} + \ce{^4_2{He}}$
}
\solutionspace{80pt}

\tasknumber{4}%
\task{%
    Каков период полураспада радиоактивного изотопа,
    если за 8 ч в среднем распадается 37500 атомов из 40000?
}
\answer{%
    $
        N(t) = N_0 \cdot 2^{-\frac t{\tau_{\frac12}}}
        \implies \log_2\frac N{N_0} = - \frac t{\tau_\frac 12}
        \implies \tau_\frac 12 = - \frac t{\log_2\frac N{N_0}}
                                  =   \frac t{\log_2\frac{N_0}N}
        = \frac{
            8 \units{ч}
        }
        {
            \log_2\frac{40000}{40000 - 37500}
        }
        \approx 2\,\text{ч}.
    $
}

\variantsplitter

\addpersonalvariant{Назар Сабинов}

\tasknumber{1}%
\task{%
    Сколько фотонов испускает за $30\,\text{мин}$ лазер,
    если мощность его излучения $75\,\text{мВт}$?
    Длина волны излучения $500\,\text{нм}$.
    $h = 6{,}626 \cdot 10^{-34}\,\text{Дж}\cdot\text{с}$.
}
\answer{%
    $
        N
            = \frac{E_{\text{общая}}}{E_{\text{одного фотона}}}
            = \frac{Pt}{h\nu} = \frac{Pt}{h \frac c\lambda}
            = \frac{Pt\lambda}{hc}
            = \frac{75\,\text{мВт} \cdot 30\,\text{мин} \cdot 500\,\text{нм}}{6{,}626 \cdot 10^{-34}\,\text{Дж}\cdot\text{с} \cdot 3 \cdot 10^{8}\,\frac{\text{м}}{\text{с}}}
            \approx 339{,}6 \cdot 10^{18}\units{фотонов}
    $
}
\solutionspace{120pt}

\tasknumber{2}%
\task{%
    В ядре электрически нейтрального атома 121 частиц.
    Вокруг ядра обращается 51 электронов.
    Сколько в ядре этого атома протонов и нейтронов?
    Назовите этот элемент.
}
\answer{%
    $Z = 51$ протонов и $A - Z = 70$ нейтронов, так что это \text{сурьма-121}: $\ce{^{121}_{51}{Sb}}$
}
\solutionspace{40pt}

\tasknumber{3}%
\task{%
    Запишите реакцию $\alpha$-распада $\ce{^{238}_{92}{U}}$.
}
\answer{%
    $\ce{^{238}_{92}{U}} \to \ce{^{234}_{90}{Th}} + \ce{^4_2{He}}$
}
\solutionspace{80pt}

\tasknumber{4}%
\task{%
    Каков период полураспада радиоактивного изотопа,
    если за 24 ч в среднем распадается 75000 атомов из 80000?
}
\answer{%
    $
        N(t) = N_0 \cdot 2^{-\frac t{\tau_{\frac12}}}
        \implies \log_2\frac N{N_0} = - \frac t{\tau_\frac 12}
        \implies \tau_\frac 12 = - \frac t{\log_2\frac N{N_0}}
                                  =   \frac t{\log_2\frac{N_0}N}
        = \frac{
            24 \units{ч}
        }
        {
            \log_2\frac{80000}{80000 - 75000}
        }
        \approx 6\,\text{ч}.
    $
}

\variantsplitter

\addpersonalvariant{Михаил Тетерин}

\tasknumber{1}%
\task{%
    Сколько фотонов испускает за $120\,\text{мин}$ лазер,
    если мощность его излучения $75\,\text{мВт}$?
    Длина волны излучения $600\,\text{нм}$.
    $h = 6{,}626 \cdot 10^{-34}\,\text{Дж}\cdot\text{с}$.
}
\answer{%
    $
        N
            = \frac{E_{\text{общая}}}{E_{\text{одного фотона}}}
            = \frac{Pt}{h\nu} = \frac{Pt}{h \frac c\lambda}
            = \frac{Pt\lambda}{hc}
            = \frac{75\,\text{мВт} \cdot 120\,\text{мин} \cdot 600\,\text{нм}}{6{,}626 \cdot 10^{-34}\,\text{Дж}\cdot\text{с} \cdot 3 \cdot 10^{8}\,\frac{\text{м}}{\text{с}}}
            \approx 1{,}6299 \cdot 10^{21}\units{фотонов}
    $
}
\solutionspace{120pt}

\tasknumber{2}%
\task{%
    В ядре электрически нейтрального атома 63 частиц.
    Вокруг ядра обращается 29 электронов.
    Сколько в ядре этого атома протонов и нейтронов?
    Назовите этот элемент.
}
\answer{%
    $Z = 29$ протонов и $A - Z = 34$ нейтронов, так что это \text{медь-63}: $\ce{^{63}_{29}{Cu}}$
}
\solutionspace{40pt}

\tasknumber{3}%
\task{%
    Запишите реакцию $\alpha$-распада $\ce{^{147}_{62}{Sm}}$.
}
\answer{%
    $\ce{^{147}_{62}{Sm}} \to \ce{^{143}_{60}{Nd}} + \ce{^4_2{He}}$
}
\solutionspace{80pt}

\tasknumber{4}%
\task{%
    Каков период полураспада радиоактивного изотопа,
    если за 6 ч в среднем распадается 3500 атомов из 4000?
}
\answer{%
    $
        N(t) = N_0 \cdot 2^{-\frac t{\tau_{\frac12}}}
        \implies \log_2\frac N{N_0} = - \frac t{\tau_\frac 12}
        \implies \tau_\frac 12 = - \frac t{\log_2\frac N{N_0}}
                                  =   \frac t{\log_2\frac{N_0}N}
        = \frac{
            6 \units{ч}
        }
        {
            \log_2\frac{4000}{4000 - 3500}
        }
        \approx 2\,\text{ч}.
    $
}

\variantsplitter

\addpersonalvariant{Арсланхан Уматалиев}

\tasknumber{1}%
\task{%
    Сколько фотонов испускает за $10\,\text{мин}$ лазер,
    если мощность его излучения $40\,\text{мВт}$?
    Длина волны излучения $600\,\text{нм}$.
    $h = 6{,}626 \cdot 10^{-34}\,\text{Дж}\cdot\text{с}$.
}
\answer{%
    $
        N
            = \frac{E_{\text{общая}}}{E_{\text{одного фотона}}}
            = \frac{Pt}{h\nu} = \frac{Pt}{h \frac c\lambda}
            = \frac{Pt\lambda}{hc}
            = \frac{40\,\text{мВт} \cdot 10\,\text{мин} \cdot 600\,\text{нм}}{6{,}626 \cdot 10^{-34}\,\text{Дж}\cdot\text{с} \cdot 3 \cdot 10^{8}\,\frac{\text{м}}{\text{с}}}
            \approx 72{,}4 \cdot 10^{18}\units{фотонов}
    $
}
\solutionspace{120pt}

\tasknumber{2}%
\task{%
    В ядре электрически нейтрального атома 190 частиц.
    Вокруг ядра обращается 78 электронов.
    Сколько в ядре этого атома протонов и нейтронов?
    Назовите этот элемент.
}
\answer{%
    $Z = 78$ протонов и $A - Z = 112$ нейтронов, так что это \text{платина-190}: $\ce{^{190}_{78}{Pt}}$
}
\solutionspace{40pt}

\tasknumber{3}%
\task{%
    Запишите реакцию $\alpha$-распада $\ce{^{147}_{62}{Sm}}$.
}
\answer{%
    $\ce{^{147}_{62}{Sm}} \to \ce{^{143}_{60}{Nd}} + \ce{^4_2{He}}$
}
\solutionspace{80pt}

\tasknumber{4}%
\task{%
    Каков период полураспада радиоактивного изотопа,
    если за 8 ч в среднем распадается 37500 атомов из 40000?
}
\answer{%
    $
        N(t) = N_0 \cdot 2^{-\frac t{\tau_{\frac12}}}
        \implies \log_2\frac N{N_0} = - \frac t{\tau_\frac 12}
        \implies \tau_\frac 12 = - \frac t{\log_2\frac N{N_0}}
                                  =   \frac t{\log_2\frac{N_0}N}
        = \frac{
            8 \units{ч}
        }
        {
            \log_2\frac{40000}{40000 - 37500}
        }
        \approx 2\,\text{ч}.
    $
}

\variantsplitter

\addpersonalvariant{Дарья Холодная}

\tasknumber{1}%
\task{%
    Сколько фотонов испускает за $120\,\text{мин}$ лазер,
    если мощность его излучения $40\,\text{мВт}$?
    Длина волны излучения $750\,\text{нм}$.
    $h = 6{,}626 \cdot 10^{-34}\,\text{Дж}\cdot\text{с}$.
}
\answer{%
    $
        N
            = \frac{E_{\text{общая}}}{E_{\text{одного фотона}}}
            = \frac{Pt}{h\nu} = \frac{Pt}{h \frac c\lambda}
            = \frac{Pt\lambda}{hc}
            = \frac{40\,\text{мВт} \cdot 120\,\text{мин} \cdot 750\,\text{нм}}{6{,}626 \cdot 10^{-34}\,\text{Дж}\cdot\text{с} \cdot 3 \cdot 10^{8}\,\frac{\text{м}}{\text{с}}}
            \approx 1{,}0866 \cdot 10^{21}\units{фотонов}
    $
}
\solutionspace{120pt}

\tasknumber{2}%
\task{%
    В ядре электрически нейтрального атома 123 частиц.
    Вокруг ядра обращается 51 электронов.
    Сколько в ядре этого атома протонов и нейтронов?
    Назовите этот элемент.
}
\answer{%
    $Z = 51$ протонов и $A - Z = 72$ нейтронов, так что это \text{сурьма-123}: $\ce{^{123}_{51}{Sb}}$
}
\solutionspace{40pt}

\tasknumber{3}%
\task{%
    Запишите реакцию $\alpha$-распада $\ce{^{147}_{62}{Sm}}$.
}
\answer{%
    $\ce{^{147}_{62}{Sm}} \to \ce{^{143}_{60}{Nd}} + \ce{^4_2{He}}$
}
\solutionspace{80pt}

\tasknumber{4}%
\task{%
    Каков период полураспада радиоактивного изотопа,
    если за 24 ч в среднем распадается 75000 атомов из 80000?
}
\answer{%
    $
        N(t) = N_0 \cdot 2^{-\frac t{\tau_{\frac12}}}
        \implies \log_2\frac N{N_0} = - \frac t{\tau_\frac 12}
        \implies \tau_\frac 12 = - \frac t{\log_2\frac N{N_0}}
                                  =   \frac t{\log_2\frac{N_0}N}
        = \frac{
            24 \units{ч}
        }
        {
            \log_2\frac{80000}{80000 - 75000}
        }
        \approx 6\,\text{ч}.
    $
}
% autogenerated
