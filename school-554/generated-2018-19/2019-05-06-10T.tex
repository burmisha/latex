\setdate{6~мая~2019}
\setclass{10«Т»}

\addpersonalvariant{Михаил Бурмистров}

\tasknumber{1}%
\task{%
    Определите ёмкость конденсатора, если при его зарядке до напряжения
    $U = 40\,\text{кВ}$ он приобретает заряд $Q = 24\,\text{мКл}$.
    % Чему при этом равны заряды обкладок конденсатора (сделайте рисунок и укажите их)?
    Ответ выразите в нанофарадах.
}
\answer{%
    $
        Q = CU \implies
        C = \frac{ Q }{ U } = \frac{ 24\,\text{мКл} }{ 40\,\text{кВ} } = 600\,\text{нФ}.
        \text{Заряды обкладок: $Q$ и $-Q$}
    $
}
\solutionspace{120pt}

\tasknumber{2}%
\task{%
    На конденсаторе указано: $C = 150\,\text{пФ}$, $U = 450\,\text{В}$.
    Удастся ли его использовать для накопления заряда $q = 60\,\text{нКл}$?
    (в ответе укажите «да» или «нет»)
}
\answer{%
    $
        q_{ \text{ max } } = CU = 150\,\text{пФ} \cdot 450\,\text{В} = 67\,\text{нКл}
        \implies q_{ \text{ max } } \ge q \implies \text{удастся}
    $
}
\solutionspace{80pt}

\tasknumber{3}%
\task{%
    Как и во сколько раз изменится ёмкость плоского конденсатора
    при уменьшении площади пластин в 2 раз
    и уменьшении расстояния между ними в 3 раз?
    В ответе укажите простую дробь или число — отношение новой ёмкости к старой.
}
\answer{%
    $
        \frac{C'}{C}
            = \frac{\eps_0\eps \frac S2}{\frac d3} \Big/ \frac{\eps_0\eps S}{d}
            = \frac{3}{2} = > 1 \implies \text{увеличится в $\frac32$ раз}
    $
}
\solutionspace{80pt}

\tasknumber{4}%
\task{%
    Электрическая ёмкость конденсатора равна $C = 750\,\text{пФ}$,
    при этом ему сообщён заряд $Q = 800\,\text{нКл}$.
    Какова энергия заряженного конденсатора?
    Ответ выразите в микроджоулях и округлите до целого.
}
\answer{%
    $
        W
        = \frac{Q^2}{2C}
        = \frac{\sqr{ 800\,\text{нКл} }}{2 \cdot 750\,\text{пФ}}
        = 426{,}67\,\text{мкДж}
    $
}
\solutionspace{80pt}

\tasknumber{5}%
\task{%
    \begin{tikzpicture}[circuit ee IEC, x=1cm, y=1cm, semithick]
        \draw  (0, 0) to [capacitor={info={$C_1$}}] (1, 0)
                       to [capacitor={info={$C_2$}}] (2, 0)
        ;
        \draw [-o] (0, 0) -- ++(-0.5, 0) node[left] {$-$};
        \draw [-o] (2, 0) -- ++(0.5, 0) node[right] {$+$};

        \node [right,text width = 14cm, align=justify] at (3.5,0) {
        Два конденсатора ёмкостей $C_1 = 30\,\text{нФ}$ и $C_2 = 60\,\text{нФ}$ последовательно подключают
        к источнику напряжения $U = 300\,\text{В}$ (см.
        рис.).
        Определите заряды каждого из конденсаторов.
       };
    \end{tikzpicture}
}
\answer{%
    $
        Q_1
            = Q_2
            = CU
            = \frac{U}{\frac1{C_1} + \frac1{C_2}}
            = \frac{C_1C_2U}{C_1 + C_2}
            = \frac{
                30\,\text{нФ} \cdot 60\,\text{нФ} \cdot 300\,\text{В}
            }{
                30\,\text{нФ} + 60\,\text{нФ}
            }
            = 6000{,}00\,\text{нКл}
    $
}

\variantsplitter

\addpersonalvariant{Гагик Аракелян}

\tasknumber{1}%
\task{%
    Определите ёмкость конденсатора, если при его зарядке до напряжения
    $U = 4\,\text{кВ}$ он приобретает заряд $q = 4\,\text{мКл}$.
    % Чему при этом равны заряды обкладок конденсатора (сделайте рисунок и укажите их)?
    Ответ выразите в нанофарадах.
}
\answer{%
    $
        q = CU \implies
        C = \frac{ q }{ U } = \frac{ 4\,\text{мКл} }{ 4\,\text{кВ} } = 1000\,\text{нФ}.
        \text{Заряды обкладок: $q$ и $-q$}
    $
}
\solutionspace{120pt}

\tasknumber{2}%
\task{%
    На конденсаторе указано: $C = 50\,\text{пФ}$, $V = 200\,\text{В}$.
    Удастся ли его использовать для накопления заряда $Q = 60\,\text{нКл}$?
    (в ответе укажите «да» или «нет»)
}
\answer{%
    $
        Q_{ \text{ max } } = CV = 50\,\text{пФ} \cdot 200\,\text{В} = 10\,\text{нКл}
        \implies Q_{ \text{ max } }  <  Q \implies \text{не удастся}
    $
}
\solutionspace{80pt}

\tasknumber{3}%
\task{%
    Как и во сколько раз изменится ёмкость плоского конденсатора
    при уменьшении площади пластин в 5 раз
    и уменьшении расстояния между ними в 3 раз?
    В ответе укажите простую дробь или число — отношение новой ёмкости к старой.
}
\answer{%
    $
        \frac{C'}{C}
            = \frac{\eps_0\eps \frac S5}{\frac d3} \Big/ \frac{\eps_0\eps S}{d}
            = \frac{3}{5} = < 1 \implies \text{уменьшится в $\frac53$ раз}
    $
}
\solutionspace{80pt}

\tasknumber{4}%
\task{%
    Электрическая ёмкость конденсатора равна $C = 200\,\text{пФ}$,
    при этом ему сообщён заряд $Q = 500\,\text{нКл}$.
    Какова энергия заряженного конденсатора?
    Ответ выразите в микроджоулях и округлите до целого.
}
\answer{%
    $
        W
        = \frac{Q^2}{2C}
        = \frac{\sqr{ 500\,\text{нКл} }}{2 \cdot 200\,\text{пФ}}
        = 625{,}00\,\text{мкДж}
    $
}
\solutionspace{80pt}

\tasknumber{5}%
\task{%
    \begin{tikzpicture}[circuit ee IEC, x=1cm, y=1cm, semithick]
        \draw  (0, 0) to [capacitor={info={$C_1$}}] (1, 0)
                       to [capacitor={info={$C_2$}}] (2, 0)
        ;
        \draw [-o] (0, 0) -- ++(-0.5, 0) node[left] {$-$};
        \draw [-o] (2, 0) -- ++(0.5, 0) node[right] {$+$};

        \node [right,text width = 14cm, align=justify] at (3.5,0) {
        Два конденсатора ёмкостей $C_1 = 30\,\text{нФ}$ и $C_2 = 40\,\text{нФ}$ последовательно подключают
        к источнику напряжения $U = 200\,\text{В}$ (см.
        рис.).
        Определите заряды каждого из конденсаторов.
       };
    \end{tikzpicture}
}
\answer{%
    $
        Q_1
            = Q_2
            = CU
            = \frac{U}{\frac1{C_1} + \frac1{C_2}}
            = \frac{C_1C_2U}{C_1 + C_2}
            = \frac{
                30\,\text{нФ} \cdot 40\,\text{нФ} \cdot 200\,\text{В}
            }{
                30\,\text{нФ} + 40\,\text{нФ}
            }
            = 3428{,}57\,\text{нКл}
    $
}

\variantsplitter

\addpersonalvariant{Ирен Аракелян}

\tasknumber{1}%
\task{%
    Определите ёмкость конденсатора, если при его зарядке до напряжения
    $V = 20\,\text{кВ}$ он приобретает заряд $Q = 24\,\text{мКл}$.
    % Чему при этом равны заряды обкладок конденсатора (сделайте рисунок и укажите их)?
    Ответ выразите в нанофарадах.
}
\answer{%
    $
        Q = CV \implies
        C = \frac{ Q }{ V } = \frac{ 24\,\text{мКл} }{ 20\,\text{кВ} } = 1200\,\text{нФ}.
        \text{Заряды обкладок: $Q$ и $-Q$}
    $
}
\solutionspace{120pt}

\tasknumber{2}%
\task{%
    На конденсаторе указано: $C = 50\,\text{пФ}$, $U = 200\,\text{В}$.
    Удастся ли его использовать для накопления заряда $q = 60\,\text{нКл}$?
    (в ответе укажите «да» или «нет»)
}
\answer{%
    $
        q_{ \text{ max } } = CU = 50\,\text{пФ} \cdot 200\,\text{В} = 10\,\text{нКл}
        \implies q_{ \text{ max } }  <  q \implies \text{не удастся}
    $
}
\solutionspace{80pt}

\tasknumber{3}%
\task{%
    Как и во сколько раз изменится ёмкость плоского конденсатора
    при уменьшении площади пластин в 5 раз
    и уменьшении расстояния между ними в 4 раз?
    В ответе укажите простую дробь или число — отношение новой ёмкости к старой.
}
\answer{%
    $
        \frac{C'}{C}
            = \frac{\eps_0\eps \frac S5}{\frac d4} \Big/ \frac{\eps_0\eps S}{d}
            = \frac{4}{5} = < 1 \implies \text{уменьшится в $\frac54$ раз}
    $
}
\solutionspace{80pt}

\tasknumber{4}%
\task{%
    Электрическая ёмкость конденсатора равна $C = 600\,\text{пФ}$,
    при этом ему сообщён заряд $q = 800\,\text{нКл}$.
    Какова энергия заряженного конденсатора?
    Ответ выразите в микроджоулях и округлите до целого.
}
\answer{%
    $
        W
        = \frac{q^2}{2C}
        = \frac{\sqr{ 800\,\text{нКл} }}{2 \cdot 600\,\text{пФ}}
        = 533{,}33\,\text{мкДж}
    $
}
\solutionspace{80pt}

\tasknumber{5}%
\task{%
    \begin{tikzpicture}[circuit ee IEC, x=1cm, y=1cm, semithick]
        \draw  (0, 0) to [capacitor={info={$C_1$}}] (1, 0)
                       to [capacitor={info={$C_2$}}] (2, 0)
        ;
        \draw [-o] (0, 0) -- ++(-0.5, 0) node[left] {$-$};
        \draw [-o] (2, 0) -- ++(0.5, 0) node[right] {$+$};

        \node [right,text width = 14cm, align=justify] at (3.5,0) {
        Два конденсатора ёмкостей $C_1 = 30\,\text{нФ}$ и $C_2 = 40\,\text{нФ}$ последовательно подключают
        к источнику напряжения $V = 200\,\text{В}$ (см.
        рис.).
        Определите заряды каждого из конденсаторов.
       };
    \end{tikzpicture}
}
\answer{%
    $
        Q_1
            = Q_2
            = CV
            = \frac{V}{\frac1{C_1} + \frac1{C_2}}
            = \frac{C_1C_2V}{C_1 + C_2}
            = \frac{
                30\,\text{нФ} \cdot 40\,\text{нФ} \cdot 200\,\text{В}
            }{
                30\,\text{нФ} + 40\,\text{нФ}
            }
            = 3428{,}57\,\text{нКл}
    $
}

\variantsplitter

\addpersonalvariant{Сабина Асадуллаева}

\tasknumber{1}%
\task{%
    Определите ёмкость конденсатора, если при его зарядке до напряжения
    $V = 20\,\text{кВ}$ он приобретает заряд $Q = 6\,\text{мКл}$.
    % Чему при этом равны заряды обкладок конденсатора (сделайте рисунок и укажите их)?
    Ответ выразите в нанофарадах.
}
\answer{%
    $
        Q = CV \implies
        C = \frac{ Q }{ V } = \frac{ 6\,\text{мКл} }{ 20\,\text{кВ} } = 300\,\text{нФ}.
        \text{Заряды обкладок: $Q$ и $-Q$}
    $
}
\solutionspace{120pt}

\tasknumber{2}%
\task{%
    На конденсаторе указано: $C = 80\,\text{пФ}$, $U = 400\,\text{В}$.
    Удастся ли его использовать для накопления заряда $Q = 30\,\text{нКл}$?
    (в ответе укажите «да» или «нет»)
}
\answer{%
    $
        Q_{ \text{ max } } = CU = 80\,\text{пФ} \cdot 400\,\text{В} = 32\,\text{нКл}
        \implies Q_{ \text{ max } } \ge Q \implies \text{удастся}
    $
}
\solutionspace{80pt}

\tasknumber{3}%
\task{%
    Как и во сколько раз изменится ёмкость плоского конденсатора
    при уменьшении площади пластин в 5 раз
    и уменьшении расстояния между ними в 2 раз?
    В ответе укажите простую дробь или число — отношение новой ёмкости к старой.
}
\answer{%
    $
        \frac{C'}{C}
            = \frac{\eps_0\eps \frac S5}{\frac d2} \Big/ \frac{\eps_0\eps S}{d}
            = \frac{2}{5} = < 1 \implies \text{уменьшится в $\frac52$ раз}
    $
}
\solutionspace{80pt}

\tasknumber{4}%
\task{%
    Электрическая ёмкость конденсатора равна $C = 600\,\text{пФ}$,
    при этом ему сообщён заряд $Q = 900\,\text{нКл}$.
    Какова энергия заряженного конденсатора?
    Ответ выразите в микроджоулях и округлите до целого.
}
\answer{%
    $
        W
        = \frac{Q^2}{2C}
        = \frac{\sqr{ 900\,\text{нКл} }}{2 \cdot 600\,\text{пФ}}
        = 675{,}00\,\text{мкДж}
    $
}
\solutionspace{80pt}

\tasknumber{5}%
\task{%
    \begin{tikzpicture}[circuit ee IEC, x=1cm, y=1cm, semithick]
        \draw  (0, 0) to [capacitor={info={$C_1$}}] (1, 0)
                       to [capacitor={info={$C_2$}}] (2, 0)
        ;
        \draw [-o] (0, 0) -- ++(-0.5, 0) node[left] {$-$};
        \draw [-o] (2, 0) -- ++(0.5, 0) node[right] {$+$};

        \node [right,text width = 14cm, align=justify] at (3.5,0) {
        Два конденсатора ёмкостей $C_1 = 20\,\text{нФ}$ и $C_2 = 30\,\text{нФ}$ последовательно подключают
        к источнику напряжения $U = 200\,\text{В}$ (см.
        рис.).
        Определите заряды каждого из конденсаторов.
       };
    \end{tikzpicture}
}
\answer{%
    $
        Q_1
            = Q_2
            = CU
            = \frac{U}{\frac1{C_1} + \frac1{C_2}}
            = \frac{C_1C_2U}{C_1 + C_2}
            = \frac{
                20\,\text{нФ} \cdot 30\,\text{нФ} \cdot 200\,\text{В}
            }{
                20\,\text{нФ} + 30\,\text{нФ}
            }
            = 2400{,}00\,\text{нКл}
    $
}

\variantsplitter

\addpersonalvariant{Вероника Битерякова}

\tasknumber{1}%
\task{%
    Определите ёмкость конденсатора, если при его зарядке до напряжения
    $U = 40\,\text{кВ}$ он приобретает заряд $q = 24\,\text{мКл}$.
    % Чему при этом равны заряды обкладок конденсатора (сделайте рисунок и укажите их)?
    Ответ выразите в нанофарадах.
}
\answer{%
    $
        q = CU \implies
        C = \frac{ q }{ U } = \frac{ 24\,\text{мКл} }{ 40\,\text{кВ} } = 600\,\text{нФ}.
        \text{Заряды обкладок: $q$ и $-q$}
    $
}
\solutionspace{120pt}

\tasknumber{2}%
\task{%
    На конденсаторе указано: $C = 120\,\text{пФ}$, $V = 400\,\text{В}$.
    Удастся ли его использовать для накопления заряда $q = 60\,\text{нКл}$?
    (в ответе укажите «да» или «нет»)
}
\answer{%
    $
        q_{ \text{ max } } = CV = 120\,\text{пФ} \cdot 400\,\text{В} = 48\,\text{нКл}
        \implies q_{ \text{ max } }  <  q \implies \text{не удастся}
    $
}
\solutionspace{80pt}

\tasknumber{3}%
\task{%
    Как и во сколько раз изменится ёмкость плоского конденсатора
    при уменьшении площади пластин в 2 раз
    и уменьшении расстояния между ними в 4 раз?
    В ответе укажите простую дробь или число — отношение новой ёмкости к старой.
}
\answer{%
    $
        \frac{C'}{C}
            = \frac{\eps_0\eps \frac S2}{\frac d4} \Big/ \frac{\eps_0\eps S}{d}
            = \frac{4}{2} = > 1 \implies \text{увеличится в $2$ раз}
    $
}
\solutionspace{80pt}

\tasknumber{4}%
\task{%
    Электрическая ёмкость конденсатора равна $C = 600\,\text{пФ}$,
    при этом ему сообщён заряд $q = 900\,\text{нКл}$.
    Какова энергия заряженного конденсатора?
    Ответ выразите в микроджоулях и округлите до целого.
}
\answer{%
    $
        W
        = \frac{q^2}{2C}
        = \frac{\sqr{ 900\,\text{нКл} }}{2 \cdot 600\,\text{пФ}}
        = 675{,}00\,\text{мкДж}
    $
}
\solutionspace{80pt}

\tasknumber{5}%
\task{%
    \begin{tikzpicture}[circuit ee IEC, x=1cm, y=1cm, semithick]
        \draw  (0, 0) to [capacitor={info={$C_1$}}] (1, 0)
                       to [capacitor={info={$C_2$}}] (2, 0)
        ;
        \draw [-o] (0, 0) -- ++(-0.5, 0) node[left] {$-$};
        \draw [-o] (2, 0) -- ++(0.5, 0) node[right] {$+$};

        \node [right,text width = 14cm, align=justify] at (3.5,0) {
        Два конденсатора ёмкостей $C_1 = 60\,\text{нФ}$ и $C_2 = 30\,\text{нФ}$ последовательно подключают
        к источнику напряжения $V = 200\,\text{В}$ (см.
        рис.).
        Определите заряды каждого из конденсаторов.
       };
    \end{tikzpicture}
}
\answer{%
    $
        Q_1
            = Q_2
            = CV
            = \frac{V}{\frac1{C_1} + \frac1{C_2}}
            = \frac{C_1C_2V}{C_1 + C_2}
            = \frac{
                60\,\text{нФ} \cdot 30\,\text{нФ} \cdot 200\,\text{В}
            }{
                60\,\text{нФ} + 30\,\text{нФ}
            }
            = 4000{,}00\,\text{нКл}
    $
}

\variantsplitter

\addpersonalvariant{Юлия Буянова}

\tasknumber{1}%
\task{%
    Определите ёмкость конденсатора, если при его зарядке до напряжения
    $V = 4\,\text{кВ}$ он приобретает заряд $q = 15\,\text{мКл}$.
    % Чему при этом равны заряды обкладок конденсатора (сделайте рисунок и укажите их)?
    Ответ выразите в нанофарадах.
}
\answer{%
    $
        q = CV \implies
        C = \frac{ q }{ V } = \frac{ 15\,\text{мКл} }{ 4\,\text{кВ} } = 3750\,\text{нФ}.
        \text{Заряды обкладок: $q$ и $-q$}
    $
}
\solutionspace{120pt}

\tasknumber{2}%
\task{%
    На конденсаторе указано: $C = 150\,\text{пФ}$, $V = 400\,\text{В}$.
    Удастся ли его использовать для накопления заряда $q = 50\,\text{нКл}$?
    (в ответе укажите «да» или «нет»)
}
\answer{%
    $
        q_{ \text{ max } } = CV = 150\,\text{пФ} \cdot 400\,\text{В} = 60\,\text{нКл}
        \implies q_{ \text{ max } } \ge q \implies \text{удастся}
    $
}
\solutionspace{80pt}

\tasknumber{3}%
\task{%
    Как и во сколько раз изменится ёмкость плоского конденсатора
    при уменьшении площади пластин в 4 раз
    и уменьшении расстояния между ними в 8 раз?
    В ответе укажите простую дробь или число — отношение новой ёмкости к старой.
}
\answer{%
    $
        \frac{C'}{C}
            = \frac{\eps_0\eps \frac S4}{\frac d8} \Big/ \frac{\eps_0\eps S}{d}
            = \frac{8}{4} = > 1 \implies \text{увеличится в $2$ раз}
    $
}
\solutionspace{80pt}

\tasknumber{4}%
\task{%
    Электрическая ёмкость конденсатора равна $C = 750\,\text{пФ}$,
    при этом ему сообщён заряд $Q = 800\,\text{нКл}$.
    Какова энергия заряженного конденсатора?
    Ответ выразите в микроджоулях и округлите до целого.
}
\answer{%
    $
        W
        = \frac{Q^2}{2C}
        = \frac{\sqr{ 800\,\text{нКл} }}{2 \cdot 750\,\text{пФ}}
        = 426{,}67\,\text{мкДж}
    $
}
\solutionspace{80pt}

\tasknumber{5}%
\task{%
    \begin{tikzpicture}[circuit ee IEC, x=1cm, y=1cm, semithick]
        \draw  (0, 0) to [capacitor={info={$C_1$}}] (1, 0)
                       to [capacitor={info={$C_2$}}] (2, 0)
        ;
        \draw [-o] (0, 0) -- ++(-0.5, 0) node[left] {$-$};
        \draw [-o] (2, 0) -- ++(0.5, 0) node[right] {$+$};

        \node [right,text width = 14cm, align=justify] at (3.5,0) {
        Два конденсатора ёмкостей $C_1 = 20\,\text{нФ}$ и $C_2 = 40\,\text{нФ}$ последовательно подключают
        к источнику напряжения $V = 400\,\text{В}$ (см.
        рис.).
        Определите заряды каждого из конденсаторов.
       };
    \end{tikzpicture}
}
\answer{%
    $
        Q_1
            = Q_2
            = CV
            = \frac{V}{\frac1{C_1} + \frac1{C_2}}
            = \frac{C_1C_2V}{C_1 + C_2}
            = \frac{
                20\,\text{нФ} \cdot 40\,\text{нФ} \cdot 400\,\text{В}
            }{
                20\,\text{нФ} + 40\,\text{нФ}
            }
            = 5333{,}33\,\text{нКл}
    $
}

\variantsplitter

\addpersonalvariant{Пелагея Вдовина}

\tasknumber{1}%
\task{%
    Определите ёмкость конденсатора, если при его зарядке до напряжения
    $U = 4\,\text{кВ}$ он приобретает заряд $Q = 18\,\text{мКл}$.
    % Чему при этом равны заряды обкладок конденсатора (сделайте рисунок и укажите их)?
    Ответ выразите в нанофарадах.
}
\answer{%
    $
        Q = CU \implies
        C = \frac{ Q }{ U } = \frac{ 18\,\text{мКл} }{ 4\,\text{кВ} } = 4500\,\text{нФ}.
        \text{Заряды обкладок: $Q$ и $-Q$}
    $
}
\solutionspace{120pt}

\tasknumber{2}%
\task{%
    На конденсаторе указано: $C = 50\,\text{пФ}$, $U = 300\,\text{В}$.
    Удастся ли его использовать для накопления заряда $q = 50\,\text{нКл}$?
    (в ответе укажите «да» или «нет»)
}
\answer{%
    $
        q_{ \text{ max } } = CU = 50\,\text{пФ} \cdot 300\,\text{В} = 15\,\text{нКл}
        \implies q_{ \text{ max } }  <  q \implies \text{не удастся}
    $
}
\solutionspace{80pt}

\tasknumber{3}%
\task{%
    Как и во сколько раз изменится ёмкость плоского конденсатора
    при уменьшении площади пластин в 3 раз
    и уменьшении расстояния между ними в 7 раз?
    В ответе укажите простую дробь или число — отношение новой ёмкости к старой.
}
\answer{%
    $
        \frac{C'}{C}
            = \frac{\eps_0\eps \frac S3}{\frac d7} \Big/ \frac{\eps_0\eps S}{d}
            = \frac{7}{3} = > 1 \implies \text{увеличится в $\frac73$ раз}
    $
}
\solutionspace{80pt}

\tasknumber{4}%
\task{%
    Электрическая ёмкость конденсатора равна $C = 600\,\text{пФ}$,
    при этом ему сообщён заряд $q = 800\,\text{нКл}$.
    Какова энергия заряженного конденсатора?
    Ответ выразите в микроджоулях и округлите до целого.
}
\answer{%
    $
        W
        = \frac{q^2}{2C}
        = \frac{\sqr{ 800\,\text{нКл} }}{2 \cdot 600\,\text{пФ}}
        = 533{,}33\,\text{мкДж}
    $
}
\solutionspace{80pt}

\tasknumber{5}%
\task{%
    \begin{tikzpicture}[circuit ee IEC, x=1cm, y=1cm, semithick]
        \draw  (0, 0) to [capacitor={info={$C_1$}}] (1, 0)
                       to [capacitor={info={$C_2$}}] (2, 0)
        ;
        \draw [-o] (0, 0) -- ++(-0.5, 0) node[left] {$-$};
        \draw [-o] (2, 0) -- ++(0.5, 0) node[right] {$+$};

        \node [right,text width = 14cm, align=justify] at (3.5,0) {
        Два конденсатора ёмкостей $C_1 = 30\,\text{нФ}$ и $C_2 = 20\,\text{нФ}$ последовательно подключают
        к источнику напряжения $V = 300\,\text{В}$ (см.
        рис.).
        Определите заряды каждого из конденсаторов.
       };
    \end{tikzpicture}
}
\answer{%
    $
        Q_1
            = Q_2
            = CV
            = \frac{V}{\frac1{C_1} + \frac1{C_2}}
            = \frac{C_1C_2V}{C_1 + C_2}
            = \frac{
                30\,\text{нФ} \cdot 20\,\text{нФ} \cdot 300\,\text{В}
            }{
                30\,\text{нФ} + 20\,\text{нФ}
            }
            = 3600{,}00\,\text{нКл}
    $
}

\variantsplitter

\addpersonalvariant{Леонид Викторов}

\tasknumber{1}%
\task{%
    Определите ёмкость конденсатора, если при его зарядке до напряжения
    $U = 50\,\text{кВ}$ он приобретает заряд $q = 25\,\text{мКл}$.
    % Чему при этом равны заряды обкладок конденсатора (сделайте рисунок и укажите их)?
    Ответ выразите в нанофарадах.
}
\answer{%
    $
        q = CU \implies
        C = \frac{ q }{ U } = \frac{ 25\,\text{мКл} }{ 50\,\text{кВ} } = 500\,\text{нФ}.
        \text{Заряды обкладок: $q$ и $-q$}
    $
}
\solutionspace{120pt}

\tasknumber{2}%
\task{%
    На конденсаторе указано: $C = 120\,\text{пФ}$, $U = 400\,\text{В}$.
    Удастся ли его использовать для накопления заряда $Q = 50\,\text{нКл}$?
    (в ответе укажите «да» или «нет»)
}
\answer{%
    $
        Q_{ \text{ max } } = CU = 120\,\text{пФ} \cdot 400\,\text{В} = 48\,\text{нКл}
        \implies Q_{ \text{ max } }  <  Q \implies \text{не удастся}
    $
}
\solutionspace{80pt}

\tasknumber{3}%
\task{%
    Как и во сколько раз изменится ёмкость плоского конденсатора
    при уменьшении площади пластин в 8 раз
    и уменьшении расстояния между ними в 3 раз?
    В ответе укажите простую дробь или число — отношение новой ёмкости к старой.
}
\answer{%
    $
        \frac{C'}{C}
            = \frac{\eps_0\eps \frac S8}{\frac d3} \Big/ \frac{\eps_0\eps S}{d}
            = \frac{3}{8} = < 1 \implies \text{уменьшится в $\frac83$ раз}
    $
}
\solutionspace{80pt}

\tasknumber{4}%
\task{%
    Электрическая ёмкость конденсатора равна $C = 750\,\text{пФ}$,
    при этом ему сообщён заряд $q = 800\,\text{нКл}$.
    Какова энергия заряженного конденсатора?
    Ответ выразите в микроджоулях и округлите до целого.
}
\answer{%
    $
        W
        = \frac{q^2}{2C}
        = \frac{\sqr{ 800\,\text{нКл} }}{2 \cdot 750\,\text{пФ}}
        = 426{,}67\,\text{мкДж}
    $
}
\solutionspace{80pt}

\tasknumber{5}%
\task{%
    \begin{tikzpicture}[circuit ee IEC, x=1cm, y=1cm, semithick]
        \draw  (0, 0) to [capacitor={info={$C_1$}}] (1, 0)
                       to [capacitor={info={$C_2$}}] (2, 0)
        ;
        \draw [-o] (0, 0) -- ++(-0.5, 0) node[left] {$-$};
        \draw [-o] (2, 0) -- ++(0.5, 0) node[right] {$+$};

        \node [right,text width = 14cm, align=justify] at (3.5,0) {
        Два конденсатора ёмкостей $C_1 = 40\,\text{нФ}$ и $C_2 = 60\,\text{нФ}$ последовательно подключают
        к источнику напряжения $V = 450\,\text{В}$ (см.
        рис.).
        Определите заряды каждого из конденсаторов.
       };
    \end{tikzpicture}
}
\answer{%
    $
        Q_1
            = Q_2
            = CV
            = \frac{V}{\frac1{C_1} + \frac1{C_2}}
            = \frac{C_1C_2V}{C_1 + C_2}
            = \frac{
                40\,\text{нФ} \cdot 60\,\text{нФ} \cdot 450\,\text{В}
            }{
                40\,\text{нФ} + 60\,\text{нФ}
            }
            = 10800{,}00\,\text{нКл}
    $
}

\variantsplitter

\addpersonalvariant{Фёдор Гнутов}

\tasknumber{1}%
\task{%
    Определите ёмкость конденсатора, если при его зарядке до напряжения
    $U = 50\,\text{кВ}$ он приобретает заряд $Q = 6\,\text{мКл}$.
    % Чему при этом равны заряды обкладок конденсатора (сделайте рисунок и укажите их)?
    Ответ выразите в нанофарадах.
}
\answer{%
    $
        Q = CU \implies
        C = \frac{ Q }{ U } = \frac{ 6\,\text{мКл} }{ 50\,\text{кВ} } = 120\,\text{нФ}.
        \text{Заряды обкладок: $Q$ и $-Q$}
    $
}
\solutionspace{120pt}

\tasknumber{2}%
\task{%
    На конденсаторе указано: $C = 50\,\text{пФ}$, $U = 400\,\text{В}$.
    Удастся ли его использовать для накопления заряда $q = 50\,\text{нКл}$?
    (в ответе укажите «да» или «нет»)
}
\answer{%
    $
        q_{ \text{ max } } = CU = 50\,\text{пФ} \cdot 400\,\text{В} = 20\,\text{нКл}
        \implies q_{ \text{ max } }  <  q \implies \text{не удастся}
    $
}
\solutionspace{80pt}

\tasknumber{3}%
\task{%
    Как и во сколько раз изменится ёмкость плоского конденсатора
    при уменьшении площади пластин в 6 раз
    и уменьшении расстояния между ними в 6 раз?
    В ответе укажите простую дробь или число — отношение новой ёмкости к старой.
}
\answer{%
    $
        \frac{C'}{C}
            = \frac{\eps_0\eps \frac S6}{\frac d6} \Big/ \frac{\eps_0\eps S}{d}
            = \frac{6}{6} = = 1 \implies \text{не изменится}
    $
}
\solutionspace{80pt}

\tasknumber{4}%
\task{%
    Электрическая ёмкость конденсатора равна $C = 400\,\text{пФ}$,
    при этом ему сообщён заряд $q = 800\,\text{нКл}$.
    Какова энергия заряженного конденсатора?
    Ответ выразите в микроджоулях и округлите до целого.
}
\answer{%
    $
        W
        = \frac{q^2}{2C}
        = \frac{\sqr{ 800\,\text{нКл} }}{2 \cdot 400\,\text{пФ}}
        = 800{,}00\,\text{мкДж}
    $
}
\solutionspace{80pt}

\tasknumber{5}%
\task{%
    \begin{tikzpicture}[circuit ee IEC, x=1cm, y=1cm, semithick]
        \draw  (0, 0) to [capacitor={info={$C_1$}}] (1, 0)
                       to [capacitor={info={$C_2$}}] (2, 0)
        ;
        \draw [-o] (0, 0) -- ++(-0.5, 0) node[left] {$-$};
        \draw [-o] (2, 0) -- ++(0.5, 0) node[right] {$+$};

        \node [right,text width = 14cm, align=justify] at (3.5,0) {
        Два конденсатора ёмкостей $C_1 = 40\,\text{нФ}$ и $C_2 = 60\,\text{нФ}$ последовательно подключают
        к источнику напряжения $U = 150\,\text{В}$ (см.
        рис.).
        Определите заряды каждого из конденсаторов.
       };
    \end{tikzpicture}
}
\answer{%
    $
        Q_1
            = Q_2
            = CU
            = \frac{U}{\frac1{C_1} + \frac1{C_2}}
            = \frac{C_1C_2U}{C_1 + C_2}
            = \frac{
                40\,\text{нФ} \cdot 60\,\text{нФ} \cdot 150\,\text{В}
            }{
                40\,\text{нФ} + 60\,\text{нФ}
            }
            = 3600{,}00\,\text{нКл}
    $
}

\variantsplitter

\addpersonalvariant{Илья Гримберг}

\tasknumber{1}%
\task{%
    Определите ёмкость конденсатора, если при его зарядке до напряжения
    $U = 2\,\text{кВ}$ он приобретает заряд $Q = 18\,\text{мКл}$.
    % Чему при этом равны заряды обкладок конденсатора (сделайте рисунок и укажите их)?
    Ответ выразите в нанофарадах.
}
\answer{%
    $
        Q = CU \implies
        C = \frac{ Q }{ U } = \frac{ 18\,\text{мКл} }{ 2\,\text{кВ} } = 9000\,\text{нФ}.
        \text{Заряды обкладок: $Q$ и $-Q$}
    $
}
\solutionspace{120pt}

\tasknumber{2}%
\task{%
    На конденсаторе указано: $C = 150\,\text{пФ}$, $U = 450\,\text{В}$.
    Удастся ли его использовать для накопления заряда $Q = 60\,\text{нКл}$?
    (в ответе укажите «да» или «нет»)
}
\answer{%
    $
        Q_{ \text{ max } } = CU = 150\,\text{пФ} \cdot 450\,\text{В} = 67\,\text{нКл}
        \implies Q_{ \text{ max } } \ge Q \implies \text{удастся}
    $
}
\solutionspace{80pt}

\tasknumber{3}%
\task{%
    Как и во сколько раз изменится ёмкость плоского конденсатора
    при уменьшении площади пластин в 3 раз
    и уменьшении расстояния между ними в 4 раз?
    В ответе укажите простую дробь или число — отношение новой ёмкости к старой.
}
\answer{%
    $
        \frac{C'}{C}
            = \frac{\eps_0\eps \frac S3}{\frac d4} \Big/ \frac{\eps_0\eps S}{d}
            = \frac{4}{3} = > 1 \implies \text{увеличится в $\frac43$ раз}
    $
}
\solutionspace{80pt}

\tasknumber{4}%
\task{%
    Электрическая ёмкость конденсатора равна $C = 200\,\text{пФ}$,
    при этом ему сообщён заряд $Q = 500\,\text{нКл}$.
    Какова энергия заряженного конденсатора?
    Ответ выразите в микроджоулях и округлите до целого.
}
\answer{%
    $
        W
        = \frac{Q^2}{2C}
        = \frac{\sqr{ 500\,\text{нКл} }}{2 \cdot 200\,\text{пФ}}
        = 625{,}00\,\text{мкДж}
    $
}
\solutionspace{80pt}

\tasknumber{5}%
\task{%
    \begin{tikzpicture}[circuit ee IEC, x=1cm, y=1cm, semithick]
        \draw  (0, 0) to [capacitor={info={$C_1$}}] (1, 0)
                       to [capacitor={info={$C_2$}}] (2, 0)
        ;
        \draw [-o] (0, 0) -- ++(-0.5, 0) node[left] {$-$};
        \draw [-o] (2, 0) -- ++(0.5, 0) node[right] {$+$};

        \node [right,text width = 14cm, align=justify] at (3.5,0) {
        Два конденсатора ёмкостей $C_1 = 60\,\text{нФ}$ и $C_2 = 30\,\text{нФ}$ последовательно подключают
        к источнику напряжения $U = 450\,\text{В}$ (см.
        рис.).
        Определите заряды каждого из конденсаторов.
       };
    \end{tikzpicture}
}
\answer{%
    $
        Q_1
            = Q_2
            = CU
            = \frac{U}{\frac1{C_1} + \frac1{C_2}}
            = \frac{C_1C_2U}{C_1 + C_2}
            = \frac{
                60\,\text{нФ} \cdot 30\,\text{нФ} \cdot 450\,\text{В}
            }{
                60\,\text{нФ} + 30\,\text{нФ}
            }
            = 9000{,}00\,\text{нКл}
    $
}

\variantsplitter

\addpersonalvariant{Иван Гурьянов}

\tasknumber{1}%
\task{%
    Определите ёмкость конденсатора, если при его зарядке до напряжения
    $V = 20\,\text{кВ}$ он приобретает заряд $Q = 24\,\text{мКл}$.
    % Чему при этом равны заряды обкладок конденсатора (сделайте рисунок и укажите их)?
    Ответ выразите в нанофарадах.
}
\answer{%
    $
        Q = CV \implies
        C = \frac{ Q }{ V } = \frac{ 24\,\text{мКл} }{ 20\,\text{кВ} } = 1200\,\text{нФ}.
        \text{Заряды обкладок: $Q$ и $-Q$}
    $
}
\solutionspace{120pt}

\tasknumber{2}%
\task{%
    На конденсаторе указано: $C = 100\,\text{пФ}$, $U = 450\,\text{В}$.
    Удастся ли его использовать для накопления заряда $q = 30\,\text{нКл}$?
    (в ответе укажите «да» или «нет»)
}
\answer{%
    $
        q_{ \text{ max } } = CU = 100\,\text{пФ} \cdot 450\,\text{В} = 45\,\text{нКл}
        \implies q_{ \text{ max } } \ge q \implies \text{удастся}
    $
}
\solutionspace{80pt}

\tasknumber{3}%
\task{%
    Как и во сколько раз изменится ёмкость плоского конденсатора
    при уменьшении площади пластин в 3 раз
    и уменьшении расстояния между ними в 6 раз?
    В ответе укажите простую дробь или число — отношение новой ёмкости к старой.
}
\answer{%
    $
        \frac{C'}{C}
            = \frac{\eps_0\eps \frac S3}{\frac d6} \Big/ \frac{\eps_0\eps S}{d}
            = \frac{6}{3} = > 1 \implies \text{увеличится в $2$ раз}
    $
}
\solutionspace{80pt}

\tasknumber{4}%
\task{%
    Электрическая ёмкость конденсатора равна $C = 600\,\text{пФ}$,
    при этом ему сообщён заряд $Q = 500\,\text{нКл}$.
    Какова энергия заряженного конденсатора?
    Ответ выразите в микроджоулях и округлите до целого.
}
\answer{%
    $
        W
        = \frac{Q^2}{2C}
        = \frac{\sqr{ 500\,\text{нКл} }}{2 \cdot 600\,\text{пФ}}
        = 208{,}33\,\text{мкДж}
    $
}
\solutionspace{80pt}

\tasknumber{5}%
\task{%
    \begin{tikzpicture}[circuit ee IEC, x=1cm, y=1cm, semithick]
        \draw  (0, 0) to [capacitor={info={$C_1$}}] (1, 0)
                       to [capacitor={info={$C_2$}}] (2, 0)
        ;
        \draw [-o] (0, 0) -- ++(-0.5, 0) node[left] {$-$};
        \draw [-o] (2, 0) -- ++(0.5, 0) node[right] {$+$};

        \node [right,text width = 14cm, align=justify] at (3.5,0) {
        Два конденсатора ёмкостей $C_1 = 40\,\text{нФ}$ и $C_2 = 60\,\text{нФ}$ последовательно подключают
        к источнику напряжения $V = 300\,\text{В}$ (см.
        рис.).
        Определите заряды каждого из конденсаторов.
       };
    \end{tikzpicture}
}
\answer{%
    $
        Q_1
            = Q_2
            = CV
            = \frac{V}{\frac1{C_1} + \frac1{C_2}}
            = \frac{C_1C_2V}{C_1 + C_2}
            = \frac{
                40\,\text{нФ} \cdot 60\,\text{нФ} \cdot 300\,\text{В}
            }{
                40\,\text{нФ} + 60\,\text{нФ}
            }
            = 7200{,}00\,\text{нКл}
    $
}

\variantsplitter

\addpersonalvariant{Артём Денежкин}

\tasknumber{1}%
\task{%
    Определите ёмкость конденсатора, если при его зарядке до напряжения
    $U = 2\,\text{кВ}$ он приобретает заряд $Q = 4\,\text{мКл}$.
    % Чему при этом равны заряды обкладок конденсатора (сделайте рисунок и укажите их)?
    Ответ выразите в нанофарадах.
}
\answer{%
    $
        Q = CU \implies
        C = \frac{ Q }{ U } = \frac{ 4\,\text{мКл} }{ 2\,\text{кВ} } = 2000\,\text{нФ}.
        \text{Заряды обкладок: $Q$ и $-Q$}
    $
}
\solutionspace{120pt}

\tasknumber{2}%
\task{%
    На конденсаторе указано: $C = 50\,\text{пФ}$, $V = 300\,\text{В}$.
    Удастся ли его использовать для накопления заряда $q = 30\,\text{нКл}$?
    (в ответе укажите «да» или «нет»)
}
\answer{%
    $
        q_{ \text{ max } } = CV = 50\,\text{пФ} \cdot 300\,\text{В} = 15\,\text{нКл}
        \implies q_{ \text{ max } }  <  q \implies \text{не удастся}
    $
}
\solutionspace{80pt}

\tasknumber{3}%
\task{%
    Как и во сколько раз изменится ёмкость плоского конденсатора
    при уменьшении площади пластин в 8 раз
    и уменьшении расстояния между ними в 5 раз?
    В ответе укажите простую дробь или число — отношение новой ёмкости к старой.
}
\answer{%
    $
        \frac{C'}{C}
            = \frac{\eps_0\eps \frac S8}{\frac d5} \Big/ \frac{\eps_0\eps S}{d}
            = \frac{5}{8} = < 1 \implies \text{уменьшится в $\frac85$ раз}
    $
}
\solutionspace{80pt}

\tasknumber{4}%
\task{%
    Электрическая ёмкость конденсатора равна $C = 400\,\text{пФ}$,
    при этом ему сообщён заряд $Q = 900\,\text{нКл}$.
    Какова энергия заряженного конденсатора?
    Ответ выразите в микроджоулях и округлите до целого.
}
\answer{%
    $
        W
        = \frac{Q^2}{2C}
        = \frac{\sqr{ 900\,\text{нКл} }}{2 \cdot 400\,\text{пФ}}
        = 1012{,}50\,\text{мкДж}
    $
}
\solutionspace{80pt}

\tasknumber{5}%
\task{%
    \begin{tikzpicture}[circuit ee IEC, x=1cm, y=1cm, semithick]
        \draw  (0, 0) to [capacitor={info={$C_1$}}] (1, 0)
                       to [capacitor={info={$C_2$}}] (2, 0)
        ;
        \draw [-o] (0, 0) -- ++(-0.5, 0) node[left] {$-$};
        \draw [-o] (2, 0) -- ++(0.5, 0) node[right] {$+$};

        \node [right,text width = 14cm, align=justify] at (3.5,0) {
        Два конденсатора ёмкостей $C_1 = 60\,\text{нФ}$ и $C_2 = 30\,\text{нФ}$ последовательно подключают
        к источнику напряжения $U = 200\,\text{В}$ (см.
        рис.).
        Определите заряды каждого из конденсаторов.
       };
    \end{tikzpicture}
}
\answer{%
    $
        Q_1
            = Q_2
            = CU
            = \frac{U}{\frac1{C_1} + \frac1{C_2}}
            = \frac{C_1C_2U}{C_1 + C_2}
            = \frac{
                60\,\text{нФ} \cdot 30\,\text{нФ} \cdot 200\,\text{В}
            }{
                60\,\text{нФ} + 30\,\text{нФ}
            }
            = 4000{,}00\,\text{нКл}
    $
}

\variantsplitter

\addpersonalvariant{Виктор Жилин}

\tasknumber{1}%
\task{%
    Определите ёмкость конденсатора, если при его зарядке до напряжения
    $U = 50\,\text{кВ}$ он приобретает заряд $Q = 25\,\text{мКл}$.
    % Чему при этом равны заряды обкладок конденсатора (сделайте рисунок и укажите их)?
    Ответ выразите в нанофарадах.
}
\answer{%
    $
        Q = CU \implies
        C = \frac{ Q }{ U } = \frac{ 25\,\text{мКл} }{ 50\,\text{кВ} } = 500\,\text{нФ}.
        \text{Заряды обкладок: $Q$ и $-Q$}
    $
}
\solutionspace{120pt}

\tasknumber{2}%
\task{%
    На конденсаторе указано: $C = 120\,\text{пФ}$, $V = 450\,\text{В}$.
    Удастся ли его использовать для накопления заряда $Q = 60\,\text{нКл}$?
    (в ответе укажите «да» или «нет»)
}
\answer{%
    $
        Q_{ \text{ max } } = CV = 120\,\text{пФ} \cdot 450\,\text{В} = 54\,\text{нКл}
        \implies Q_{ \text{ max } }  <  Q \implies \text{не удастся}
    $
}
\solutionspace{80pt}

\tasknumber{3}%
\task{%
    Как и во сколько раз изменится ёмкость плоского конденсатора
    при уменьшении площади пластин в 5 раз
    и уменьшении расстояния между ними в 5 раз?
    В ответе укажите простую дробь или число — отношение новой ёмкости к старой.
}
\answer{%
    $
        \frac{C'}{C}
            = \frac{\eps_0\eps \frac S5}{\frac d5} \Big/ \frac{\eps_0\eps S}{d}
            = \frac{5}{5} = = 1 \implies \text{не изменится}
    $
}
\solutionspace{80pt}

\tasknumber{4}%
\task{%
    Электрическая ёмкость конденсатора равна $C = 400\,\text{пФ}$,
    при этом ему сообщён заряд $q = 300\,\text{нКл}$.
    Какова энергия заряженного конденсатора?
    Ответ выразите в микроджоулях и округлите до целого.
}
\answer{%
    $
        W
        = \frac{q^2}{2C}
        = \frac{\sqr{ 300\,\text{нКл} }}{2 \cdot 400\,\text{пФ}}
        = 112{,}50\,\text{мкДж}
    $
}
\solutionspace{80pt}

\tasknumber{5}%
\task{%
    \begin{tikzpicture}[circuit ee IEC, x=1cm, y=1cm, semithick]
        \draw  (0, 0) to [capacitor={info={$C_1$}}] (1, 0)
                       to [capacitor={info={$C_2$}}] (2, 0)
        ;
        \draw [-o] (0, 0) -- ++(-0.5, 0) node[left] {$-$};
        \draw [-o] (2, 0) -- ++(0.5, 0) node[right] {$+$};

        \node [right,text width = 14cm, align=justify] at (3.5,0) {
        Два конденсатора ёмкостей $C_1 = 20\,\text{нФ}$ и $C_2 = 60\,\text{нФ}$ последовательно подключают
        к источнику напряжения $V = 400\,\text{В}$ (см.
        рис.).
        Определите заряды каждого из конденсаторов.
       };
    \end{tikzpicture}
}
\answer{%
    $
        Q_1
            = Q_2
            = CV
            = \frac{V}{\frac1{C_1} + \frac1{C_2}}
            = \frac{C_1C_2V}{C_1 + C_2}
            = \frac{
                20\,\text{нФ} \cdot 60\,\text{нФ} \cdot 400\,\text{В}
            }{
                20\,\text{нФ} + 60\,\text{нФ}
            }
            = 6000{,}00\,\text{нКл}
    $
}

\variantsplitter

\addpersonalvariant{Дмитрий Иванов}

\tasknumber{1}%
\task{%
    Определите ёмкость конденсатора, если при его зарядке до напряжения
    $V = 40\,\text{кВ}$ он приобретает заряд $q = 25\,\text{мКл}$.
    % Чему при этом равны заряды обкладок конденсатора (сделайте рисунок и укажите их)?
    Ответ выразите в нанофарадах.
}
\answer{%
    $
        q = CV \implies
        C = \frac{ q }{ V } = \frac{ 25\,\text{мКл} }{ 40\,\text{кВ} } = 625\,\text{нФ}.
        \text{Заряды обкладок: $q$ и $-q$}
    $
}
\solutionspace{120pt}

\tasknumber{2}%
\task{%
    На конденсаторе указано: $C = 80\,\text{пФ}$, $V = 400\,\text{В}$.
    Удастся ли его использовать для накопления заряда $q = 60\,\text{нКл}$?
    (в ответе укажите «да» или «нет»)
}
\answer{%
    $
        q_{ \text{ max } } = CV = 80\,\text{пФ} \cdot 400\,\text{В} = 32\,\text{нКл}
        \implies q_{ \text{ max } }  <  q \implies \text{не удастся}
    $
}
\solutionspace{80pt}

\tasknumber{3}%
\task{%
    Как и во сколько раз изменится ёмкость плоского конденсатора
    при уменьшении площади пластин в 5 раз
    и уменьшении расстояния между ними в 3 раз?
    В ответе укажите простую дробь или число — отношение новой ёмкости к старой.
}
\answer{%
    $
        \frac{C'}{C}
            = \frac{\eps_0\eps \frac S5}{\frac d3} \Big/ \frac{\eps_0\eps S}{d}
            = \frac{3}{5} = < 1 \implies \text{уменьшится в $\frac53$ раз}
    $
}
\solutionspace{80pt}

\tasknumber{4}%
\task{%
    Электрическая ёмкость конденсатора равна $C = 750\,\text{пФ}$,
    при этом ему сообщён заряд $q = 300\,\text{нКл}$.
    Какова энергия заряженного конденсатора?
    Ответ выразите в микроджоулях и округлите до целого.
}
\answer{%
    $
        W
        = \frac{q^2}{2C}
        = \frac{\sqr{ 300\,\text{нКл} }}{2 \cdot 750\,\text{пФ}}
        = 60{,}00\,\text{мкДж}
    $
}
\solutionspace{80pt}

\tasknumber{5}%
\task{%
    \begin{tikzpicture}[circuit ee IEC, x=1cm, y=1cm, semithick]
        \draw  (0, 0) to [capacitor={info={$C_1$}}] (1, 0)
                       to [capacitor={info={$C_2$}}] (2, 0)
        ;
        \draw [-o] (0, 0) -- ++(-0.5, 0) node[left] {$-$};
        \draw [-o] (2, 0) -- ++(0.5, 0) node[right] {$+$};

        \node [right,text width = 14cm, align=justify] at (3.5,0) {
        Два конденсатора ёмкостей $C_1 = 30\,\text{нФ}$ и $C_2 = 60\,\text{нФ}$ последовательно подключают
        к источнику напряжения $U = 400\,\text{В}$ (см.
        рис.).
        Определите заряды каждого из конденсаторов.
       };
    \end{tikzpicture}
}
\answer{%
    $
        Q_1
            = Q_2
            = CU
            = \frac{U}{\frac1{C_1} + \frac1{C_2}}
            = \frac{C_1C_2U}{C_1 + C_2}
            = \frac{
                30\,\text{нФ} \cdot 60\,\text{нФ} \cdot 400\,\text{В}
            }{
                30\,\text{нФ} + 60\,\text{нФ}
            }
            = 8000{,}00\,\text{нКл}
    $
}

\variantsplitter

\addpersonalvariant{Олег Климов}

\tasknumber{1}%
\task{%
    Определите ёмкость конденсатора, если при его зарядке до напряжения
    $V = 50\,\text{кВ}$ он приобретает заряд $Q = 15\,\text{мКл}$.
    % Чему при этом равны заряды обкладок конденсатора (сделайте рисунок и укажите их)?
    Ответ выразите в нанофарадах.
}
\answer{%
    $
        Q = CV \implies
        C = \frac{ Q }{ V } = \frac{ 15\,\text{мКл} }{ 50\,\text{кВ} } = 300\,\text{нФ}.
        \text{Заряды обкладок: $Q$ и $-Q$}
    $
}
\solutionspace{120pt}

\tasknumber{2}%
\task{%
    На конденсаторе указано: $C = 150\,\text{пФ}$, $V = 450\,\text{В}$.
    Удастся ли его использовать для накопления заряда $q = 30\,\text{нКл}$?
    (в ответе укажите «да» или «нет»)
}
\answer{%
    $
        q_{ \text{ max } } = CV = 150\,\text{пФ} \cdot 450\,\text{В} = 67\,\text{нКл}
        \implies q_{ \text{ max } } \ge q \implies \text{удастся}
    $
}
\solutionspace{80pt}

\tasknumber{3}%
\task{%
    Как и во сколько раз изменится ёмкость плоского конденсатора
    при уменьшении площади пластин в 6 раз
    и уменьшении расстояния между ними в 3 раз?
    В ответе укажите простую дробь или число — отношение новой ёмкости к старой.
}
\answer{%
    $
        \frac{C'}{C}
            = \frac{\eps_0\eps \frac S6}{\frac d3} \Big/ \frac{\eps_0\eps S}{d}
            = \frac{3}{6} = < 1 \implies \text{уменьшится в $2$ раз}
    $
}
\solutionspace{80pt}

\tasknumber{4}%
\task{%
    Электрическая ёмкость конденсатора равна $C = 200\,\text{пФ}$,
    при этом ему сообщён заряд $Q = 900\,\text{нКл}$.
    Какова энергия заряженного конденсатора?
    Ответ выразите в микроджоулях и округлите до целого.
}
\answer{%
    $
        W
        = \frac{Q^2}{2C}
        = \frac{\sqr{ 900\,\text{нКл} }}{2 \cdot 200\,\text{пФ}}
        = 2025{,}00\,\text{мкДж}
    $
}
\solutionspace{80pt}

\tasknumber{5}%
\task{%
    \begin{tikzpicture}[circuit ee IEC, x=1cm, y=1cm, semithick]
        \draw  (0, 0) to [capacitor={info={$C_1$}}] (1, 0)
                       to [capacitor={info={$C_2$}}] (2, 0)
        ;
        \draw [-o] (0, 0) -- ++(-0.5, 0) node[left] {$-$};
        \draw [-o] (2, 0) -- ++(0.5, 0) node[right] {$+$};

        \node [right,text width = 14cm, align=justify] at (3.5,0) {
        Два конденсатора ёмкостей $C_1 = 40\,\text{нФ}$ и $C_2 = 30\,\text{нФ}$ последовательно подключают
        к источнику напряжения $V = 300\,\text{В}$ (см.
        рис.).
        Определите заряды каждого из конденсаторов.
       };
    \end{tikzpicture}
}
\answer{%
    $
        Q_1
            = Q_2
            = CV
            = \frac{V}{\frac1{C_1} + \frac1{C_2}}
            = \frac{C_1C_2V}{C_1 + C_2}
            = \frac{
                40\,\text{нФ} \cdot 30\,\text{нФ} \cdot 300\,\text{В}
            }{
                40\,\text{нФ} + 30\,\text{нФ}
            }
            = 5142{,}86\,\text{нКл}
    $
}

\variantsplitter

\addpersonalvariant{Анна Ковалева}

\tasknumber{1}%
\task{%
    Определите ёмкость конденсатора, если при его зарядке до напряжения
    $U = 4\,\text{кВ}$ он приобретает заряд $Q = 18\,\text{мКл}$.
    % Чему при этом равны заряды обкладок конденсатора (сделайте рисунок и укажите их)?
    Ответ выразите в нанофарадах.
}
\answer{%
    $
        Q = CU \implies
        C = \frac{ Q }{ U } = \frac{ 18\,\text{мКл} }{ 4\,\text{кВ} } = 4500\,\text{нФ}.
        \text{Заряды обкладок: $Q$ и $-Q$}
    $
}
\solutionspace{120pt}

\tasknumber{2}%
\task{%
    На конденсаторе указано: $C = 50\,\text{пФ}$, $U = 200\,\text{В}$.
    Удастся ли его использовать для накопления заряда $q = 60\,\text{нКл}$?
    (в ответе укажите «да» или «нет»)
}
\answer{%
    $
        q_{ \text{ max } } = CU = 50\,\text{пФ} \cdot 200\,\text{В} = 10\,\text{нКл}
        \implies q_{ \text{ max } }  <  q \implies \text{не удастся}
    $
}
\solutionspace{80pt}

\tasknumber{3}%
\task{%
    Как и во сколько раз изменится ёмкость плоского конденсатора
    при уменьшении площади пластин в 2 раз
    и уменьшении расстояния между ними в 6 раз?
    В ответе укажите простую дробь или число — отношение новой ёмкости к старой.
}
\answer{%
    $
        \frac{C'}{C}
            = \frac{\eps_0\eps \frac S2}{\frac d6} \Big/ \frac{\eps_0\eps S}{d}
            = \frac{6}{2} = > 1 \implies \text{увеличится в $3$ раз}
    $
}
\solutionspace{80pt}

\tasknumber{4}%
\task{%
    Электрическая ёмкость конденсатора равна $C = 600\,\text{пФ}$,
    при этом ему сообщён заряд $q = 300\,\text{нКл}$.
    Какова энергия заряженного конденсатора?
    Ответ выразите в микроджоулях и округлите до целого.
}
\answer{%
    $
        W
        = \frac{q^2}{2C}
        = \frac{\sqr{ 300\,\text{нКл} }}{2 \cdot 600\,\text{пФ}}
        = 75{,}00\,\text{мкДж}
    $
}
\solutionspace{80pt}

\tasknumber{5}%
\task{%
    \begin{tikzpicture}[circuit ee IEC, x=1cm, y=1cm, semithick]
        \draw  (0, 0) to [capacitor={info={$C_1$}}] (1, 0)
                       to [capacitor={info={$C_2$}}] (2, 0)
        ;
        \draw [-o] (0, 0) -- ++(-0.5, 0) node[left] {$-$};
        \draw [-o] (2, 0) -- ++(0.5, 0) node[right] {$+$};

        \node [right,text width = 14cm, align=justify] at (3.5,0) {
        Два конденсатора ёмкостей $C_1 = 60\,\text{нФ}$ и $C_2 = 30\,\text{нФ}$ последовательно подключают
        к источнику напряжения $V = 200\,\text{В}$ (см.
        рис.).
        Определите заряды каждого из конденсаторов.
       };
    \end{tikzpicture}
}
\answer{%
    $
        Q_1
            = Q_2
            = CV
            = \frac{V}{\frac1{C_1} + \frac1{C_2}}
            = \frac{C_1C_2V}{C_1 + C_2}
            = \frac{
                60\,\text{нФ} \cdot 30\,\text{нФ} \cdot 200\,\text{В}
            }{
                60\,\text{нФ} + 30\,\text{нФ}
            }
            = 4000{,}00\,\text{нКл}
    $
}

\variantsplitter

\addpersonalvariant{Глеб Ковылин}

\tasknumber{1}%
\task{%
    Определите ёмкость конденсатора, если при его зарядке до напряжения
    $V = 20\,\text{кВ}$ он приобретает заряд $Q = 18\,\text{мКл}$.
    % Чему при этом равны заряды обкладок конденсатора (сделайте рисунок и укажите их)?
    Ответ выразите в нанофарадах.
}
\answer{%
    $
        Q = CV \implies
        C = \frac{ Q }{ V } = \frac{ 18\,\text{мКл} }{ 20\,\text{кВ} } = 900\,\text{нФ}.
        \text{Заряды обкладок: $Q$ и $-Q$}
    $
}
\solutionspace{120pt}

\tasknumber{2}%
\task{%
    На конденсаторе указано: $C = 120\,\text{пФ}$, $V = 300\,\text{В}$.
    Удастся ли его использовать для накопления заряда $Q = 50\,\text{нКл}$?
    (в ответе укажите «да» или «нет»)
}
\answer{%
    $
        Q_{ \text{ max } } = CV = 120\,\text{пФ} \cdot 300\,\text{В} = 36\,\text{нКл}
        \implies Q_{ \text{ max } }  <  Q \implies \text{не удастся}
    $
}
\solutionspace{80pt}

\tasknumber{3}%
\task{%
    Как и во сколько раз изменится ёмкость плоского конденсатора
    при уменьшении площади пластин в 2 раз
    и уменьшении расстояния между ними в 8 раз?
    В ответе укажите простую дробь или число — отношение новой ёмкости к старой.
}
\answer{%
    $
        \frac{C'}{C}
            = \frac{\eps_0\eps \frac S2}{\frac d8} \Big/ \frac{\eps_0\eps S}{d}
            = \frac{8}{2} = > 1 \implies \text{увеличится в $4$ раз}
    $
}
\solutionspace{80pt}

\tasknumber{4}%
\task{%
    Электрическая ёмкость конденсатора равна $C = 200\,\text{пФ}$,
    при этом ему сообщён заряд $q = 300\,\text{нКл}$.
    Какова энергия заряженного конденсатора?
    Ответ выразите в микроджоулях и округлите до целого.
}
\answer{%
    $
        W
        = \frac{q^2}{2C}
        = \frac{\sqr{ 300\,\text{нКл} }}{2 \cdot 200\,\text{пФ}}
        = 225{,}00\,\text{мкДж}
    $
}
\solutionspace{80pt}

\tasknumber{5}%
\task{%
    \begin{tikzpicture}[circuit ee IEC, x=1cm, y=1cm, semithick]
        \draw  (0, 0) to [capacitor={info={$C_1$}}] (1, 0)
                       to [capacitor={info={$C_2$}}] (2, 0)
        ;
        \draw [-o] (0, 0) -- ++(-0.5, 0) node[left] {$-$};
        \draw [-o] (2, 0) -- ++(0.5, 0) node[right] {$+$};

        \node [right,text width = 14cm, align=justify] at (3.5,0) {
        Два конденсатора ёмкостей $C_1 = 60\,\text{нФ}$ и $C_2 = 20\,\text{нФ}$ последовательно подключают
        к источнику напряжения $V = 300\,\text{В}$ (см.
        рис.).
        Определите заряды каждого из конденсаторов.
       };
    \end{tikzpicture}
}
\answer{%
    $
        Q_1
            = Q_2
            = CV
            = \frac{V}{\frac1{C_1} + \frac1{C_2}}
            = \frac{C_1C_2V}{C_1 + C_2}
            = \frac{
                60\,\text{нФ} \cdot 20\,\text{нФ} \cdot 300\,\text{В}
            }{
                60\,\text{нФ} + 20\,\text{нФ}
            }
            = 4500{,}00\,\text{нКл}
    $
}

\variantsplitter

\addpersonalvariant{Даниил Космынин}

\tasknumber{1}%
\task{%
    Определите ёмкость конденсатора, если при его зарядке до напряжения
    $U = 4\,\text{кВ}$ он приобретает заряд $q = 6\,\text{мКл}$.
    % Чему при этом равны заряды обкладок конденсатора (сделайте рисунок и укажите их)?
    Ответ выразите в нанофарадах.
}
\answer{%
    $
        q = CU \implies
        C = \frac{ q }{ U } = \frac{ 6\,\text{мКл} }{ 4\,\text{кВ} } = 1500\,\text{нФ}.
        \text{Заряды обкладок: $q$ и $-q$}
    $
}
\solutionspace{120pt}

\tasknumber{2}%
\task{%
    На конденсаторе указано: $C = 100\,\text{пФ}$, $V = 300\,\text{В}$.
    Удастся ли его использовать для накопления заряда $q = 60\,\text{нКл}$?
    (в ответе укажите «да» или «нет»)
}
\answer{%
    $
        q_{ \text{ max } } = CV = 100\,\text{пФ} \cdot 300\,\text{В} = 30\,\text{нКл}
        \implies q_{ \text{ max } }  <  q \implies \text{не удастся}
    $
}
\solutionspace{80pt}

\tasknumber{3}%
\task{%
    Как и во сколько раз изменится ёмкость плоского конденсатора
    при уменьшении площади пластин в 8 раз
    и уменьшении расстояния между ними в 2 раз?
    В ответе укажите простую дробь или число — отношение новой ёмкости к старой.
}
\answer{%
    $
        \frac{C'}{C}
            = \frac{\eps_0\eps \frac S8}{\frac d2} \Big/ \frac{\eps_0\eps S}{d}
            = \frac{2}{8} = < 1 \implies \text{уменьшится в $4$ раз}
    $
}
\solutionspace{80pt}

\tasknumber{4}%
\task{%
    Электрическая ёмкость конденсатора равна $C = 600\,\text{пФ}$,
    при этом ему сообщён заряд $q = 500\,\text{нКл}$.
    Какова энергия заряженного конденсатора?
    Ответ выразите в микроджоулях и округлите до целого.
}
\answer{%
    $
        W
        = \frac{q^2}{2C}
        = \frac{\sqr{ 500\,\text{нКл} }}{2 \cdot 600\,\text{пФ}}
        = 208{,}33\,\text{мкДж}
    $
}
\solutionspace{80pt}

\tasknumber{5}%
\task{%
    \begin{tikzpicture}[circuit ee IEC, x=1cm, y=1cm, semithick]
        \draw  (0, 0) to [capacitor={info={$C_1$}}] (1, 0)
                       to [capacitor={info={$C_2$}}] (2, 0)
        ;
        \draw [-o] (0, 0) -- ++(-0.5, 0) node[left] {$-$};
        \draw [-o] (2, 0) -- ++(0.5, 0) node[right] {$+$};

        \node [right,text width = 14cm, align=justify] at (3.5,0) {
        Два конденсатора ёмкостей $C_1 = 20\,\text{нФ}$ и $C_2 = 40\,\text{нФ}$ последовательно подключают
        к источнику напряжения $U = 200\,\text{В}$ (см.
        рис.).
        Определите заряды каждого из конденсаторов.
       };
    \end{tikzpicture}
}
\answer{%
    $
        Q_1
            = Q_2
            = CU
            = \frac{U}{\frac1{C_1} + \frac1{C_2}}
            = \frac{C_1C_2U}{C_1 + C_2}
            = \frac{
                20\,\text{нФ} \cdot 40\,\text{нФ} \cdot 200\,\text{В}
            }{
                20\,\text{нФ} + 40\,\text{нФ}
            }
            = 2666{,}67\,\text{нКл}
    $
}

\variantsplitter

\addpersonalvariant{Алина Леоничева}

\tasknumber{1}%
\task{%
    Определите ёмкость конденсатора, если при его зарядке до напряжения
    $U = 2\,\text{кВ}$ он приобретает заряд $q = 25\,\text{мКл}$.
    % Чему при этом равны заряды обкладок конденсатора (сделайте рисунок и укажите их)?
    Ответ выразите в нанофарадах.
}
\answer{%
    $
        q = CU \implies
        C = \frac{ q }{ U } = \frac{ 25\,\text{мКл} }{ 2\,\text{кВ} } = 12500\,\text{нФ}.
        \text{Заряды обкладок: $q$ и $-q$}
    $
}
\solutionspace{120pt}

\tasknumber{2}%
\task{%
    На конденсаторе указано: $C = 100\,\text{пФ}$, $V = 200\,\text{В}$.
    Удастся ли его использовать для накопления заряда $Q = 30\,\text{нКл}$?
    (в ответе укажите «да» или «нет»)
}
\answer{%
    $
        Q_{ \text{ max } } = CV = 100\,\text{пФ} \cdot 200\,\text{В} = 20\,\text{нКл}
        \implies Q_{ \text{ max } }  <  Q \implies \text{не удастся}
    $
}
\solutionspace{80pt}

\tasknumber{3}%
\task{%
    Как и во сколько раз изменится ёмкость плоского конденсатора
    при уменьшении площади пластин в 5 раз
    и уменьшении расстояния между ними в 7 раз?
    В ответе укажите простую дробь или число — отношение новой ёмкости к старой.
}
\answer{%
    $
        \frac{C'}{C}
            = \frac{\eps_0\eps \frac S5}{\frac d7} \Big/ \frac{\eps_0\eps S}{d}
            = \frac{7}{5} = > 1 \implies \text{увеличится в $\frac75$ раз}
    $
}
\solutionspace{80pt}

\tasknumber{4}%
\task{%
    Электрическая ёмкость конденсатора равна $C = 600\,\text{пФ}$,
    при этом ему сообщён заряд $Q = 900\,\text{нКл}$.
    Какова энергия заряженного конденсатора?
    Ответ выразите в микроджоулях и округлите до целого.
}
\answer{%
    $
        W
        = \frac{Q^2}{2C}
        = \frac{\sqr{ 900\,\text{нКл} }}{2 \cdot 600\,\text{пФ}}
        = 675{,}00\,\text{мкДж}
    $
}
\solutionspace{80pt}

\tasknumber{5}%
\task{%
    \begin{tikzpicture}[circuit ee IEC, x=1cm, y=1cm, semithick]
        \draw  (0, 0) to [capacitor={info={$C_1$}}] (1, 0)
                       to [capacitor={info={$C_2$}}] (2, 0)
        ;
        \draw [-o] (0, 0) -- ++(-0.5, 0) node[left] {$-$};
        \draw [-o] (2, 0) -- ++(0.5, 0) node[right] {$+$};

        \node [right,text width = 14cm, align=justify] at (3.5,0) {
        Два конденсатора ёмкостей $C_1 = 20\,\text{нФ}$ и $C_2 = 60\,\text{нФ}$ последовательно подключают
        к источнику напряжения $U = 200\,\text{В}$ (см.
        рис.).
        Определите заряды каждого из конденсаторов.
       };
    \end{tikzpicture}
}
\answer{%
    $
        Q_1
            = Q_2
            = CU
            = \frac{U}{\frac1{C_1} + \frac1{C_2}}
            = \frac{C_1C_2U}{C_1 + C_2}
            = \frac{
                20\,\text{нФ} \cdot 60\,\text{нФ} \cdot 200\,\text{В}
            }{
                20\,\text{нФ} + 60\,\text{нФ}
            }
            = 3000{,}00\,\text{нКл}
    $
}

\variantsplitter

\addpersonalvariant{Ирина Лин}

\tasknumber{1}%
\task{%
    Определите ёмкость конденсатора, если при его зарядке до напряжения
    $V = 2\,\text{кВ}$ он приобретает заряд $Q = 25\,\text{мКл}$.
    % Чему при этом равны заряды обкладок конденсатора (сделайте рисунок и укажите их)?
    Ответ выразите в нанофарадах.
}
\answer{%
    $
        Q = CV \implies
        C = \frac{ Q }{ V } = \frac{ 25\,\text{мКл} }{ 2\,\text{кВ} } = 12500\,\text{нФ}.
        \text{Заряды обкладок: $Q$ и $-Q$}
    $
}
\solutionspace{120pt}

\tasknumber{2}%
\task{%
    На конденсаторе указано: $C = 100\,\text{пФ}$, $U = 200\,\text{В}$.
    Удастся ли его использовать для накопления заряда $q = 60\,\text{нКл}$?
    (в ответе укажите «да» или «нет»)
}
\answer{%
    $
        q_{ \text{ max } } = CU = 100\,\text{пФ} \cdot 200\,\text{В} = 20\,\text{нКл}
        \implies q_{ \text{ max } }  <  q \implies \text{не удастся}
    $
}
\solutionspace{80pt}

\tasknumber{3}%
\task{%
    Как и во сколько раз изменится ёмкость плоского конденсатора
    при уменьшении площади пластин в 8 раз
    и уменьшении расстояния между ними в 6 раз?
    В ответе укажите простую дробь или число — отношение новой ёмкости к старой.
}
\answer{%
    $
        \frac{C'}{C}
            = \frac{\eps_0\eps \frac S8}{\frac d6} \Big/ \frac{\eps_0\eps S}{d}
            = \frac{6}{8} = < 1 \implies \text{уменьшится в $\frac43$ раз}
    $
}
\solutionspace{80pt}

\tasknumber{4}%
\task{%
    Электрическая ёмкость конденсатора равна $C = 200\,\text{пФ}$,
    при этом ему сообщён заряд $q = 900\,\text{нКл}$.
    Какова энергия заряженного конденсатора?
    Ответ выразите в микроджоулях и округлите до целого.
}
\answer{%
    $
        W
        = \frac{q^2}{2C}
        = \frac{\sqr{ 900\,\text{нКл} }}{2 \cdot 200\,\text{пФ}}
        = 2025{,}00\,\text{мкДж}
    $
}
\solutionspace{80pt}

\tasknumber{5}%
\task{%
    \begin{tikzpicture}[circuit ee IEC, x=1cm, y=1cm, semithick]
        \draw  (0, 0) to [capacitor={info={$C_1$}}] (1, 0)
                       to [capacitor={info={$C_2$}}] (2, 0)
        ;
        \draw [-o] (0, 0) -- ++(-0.5, 0) node[left] {$-$};
        \draw [-o] (2, 0) -- ++(0.5, 0) node[right] {$+$};

        \node [right,text width = 14cm, align=justify] at (3.5,0) {
        Два конденсатора ёмкостей $C_1 = 40\,\text{нФ}$ и $C_2 = 60\,\text{нФ}$ последовательно подключают
        к источнику напряжения $U = 450\,\text{В}$ (см.
        рис.).
        Определите заряды каждого из конденсаторов.
       };
    \end{tikzpicture}
}
\answer{%
    $
        Q_1
            = Q_2
            = CU
            = \frac{U}{\frac1{C_1} + \frac1{C_2}}
            = \frac{C_1C_2U}{C_1 + C_2}
            = \frac{
                40\,\text{нФ} \cdot 60\,\text{нФ} \cdot 450\,\text{В}
            }{
                40\,\text{нФ} + 60\,\text{нФ}
            }
            = 10800{,}00\,\text{нКл}
    $
}

\variantsplitter

\addpersonalvariant{Олег Мальцев}

\tasknumber{1}%
\task{%
    Определите ёмкость конденсатора, если при его зарядке до напряжения
    $U = 4\,\text{кВ}$ он приобретает заряд $Q = 25\,\text{мКл}$.
    % Чему при этом равны заряды обкладок конденсатора (сделайте рисунок и укажите их)?
    Ответ выразите в нанофарадах.
}
\answer{%
    $
        Q = CU \implies
        C = \frac{ Q }{ U } = \frac{ 25\,\text{мКл} }{ 4\,\text{кВ} } = 6250\,\text{нФ}.
        \text{Заряды обкладок: $Q$ и $-Q$}
    $
}
\solutionspace{120pt}

\tasknumber{2}%
\task{%
    На конденсаторе указано: $C = 80\,\text{пФ}$, $V = 400\,\text{В}$.
    Удастся ли его использовать для накопления заряда $Q = 50\,\text{нКл}$?
    (в ответе укажите «да» или «нет»)
}
\answer{%
    $
        Q_{ \text{ max } } = CV = 80\,\text{пФ} \cdot 400\,\text{В} = 32\,\text{нКл}
        \implies Q_{ \text{ max } }  <  Q \implies \text{не удастся}
    $
}
\solutionspace{80pt}

\tasknumber{3}%
\task{%
    Как и во сколько раз изменится ёмкость плоского конденсатора
    при уменьшении площади пластин в 4 раз
    и уменьшении расстояния между ними в 2 раз?
    В ответе укажите простую дробь или число — отношение новой ёмкости к старой.
}
\answer{%
    $
        \frac{C'}{C}
            = \frac{\eps_0\eps \frac S4}{\frac d2} \Big/ \frac{\eps_0\eps S}{d}
            = \frac{2}{4} = < 1 \implies \text{уменьшится в $2$ раз}
    $
}
\solutionspace{80pt}

\tasknumber{4}%
\task{%
    Электрическая ёмкость конденсатора равна $C = 200\,\text{пФ}$,
    при этом ему сообщён заряд $q = 500\,\text{нКл}$.
    Какова энергия заряженного конденсатора?
    Ответ выразите в микроджоулях и округлите до целого.
}
\answer{%
    $
        W
        = \frac{q^2}{2C}
        = \frac{\sqr{ 500\,\text{нКл} }}{2 \cdot 200\,\text{пФ}}
        = 625{,}00\,\text{мкДж}
    $
}
\solutionspace{80pt}

\tasknumber{5}%
\task{%
    \begin{tikzpicture}[circuit ee IEC, x=1cm, y=1cm, semithick]
        \draw  (0, 0) to [capacitor={info={$C_1$}}] (1, 0)
                       to [capacitor={info={$C_2$}}] (2, 0)
        ;
        \draw [-o] (0, 0) -- ++(-0.5, 0) node[left] {$-$};
        \draw [-o] (2, 0) -- ++(0.5, 0) node[right] {$+$};

        \node [right,text width = 14cm, align=justify] at (3.5,0) {
        Два конденсатора ёмкостей $C_1 = 40\,\text{нФ}$ и $C_2 = 30\,\text{нФ}$ последовательно подключают
        к источнику напряжения $V = 400\,\text{В}$ (см.
        рис.).
        Определите заряды каждого из конденсаторов.
       };
    \end{tikzpicture}
}
\answer{%
    $
        Q_1
            = Q_2
            = CV
            = \frac{V}{\frac1{C_1} + \frac1{C_2}}
            = \frac{C_1C_2V}{C_1 + C_2}
            = \frac{
                40\,\text{нФ} \cdot 30\,\text{нФ} \cdot 400\,\text{В}
            }{
                40\,\text{нФ} + 30\,\text{нФ}
            }
            = 6857{,}14\,\text{нКл}
    $
}

\variantsplitter

\addpersonalvariant{Ислам Мунаев}

\tasknumber{1}%
\task{%
    Определите ёмкость конденсатора, если при его зарядке до напряжения
    $U = 4\,\text{кВ}$ он приобретает заряд $q = 25\,\text{мКл}$.
    % Чему при этом равны заряды обкладок конденсатора (сделайте рисунок и укажите их)?
    Ответ выразите в нанофарадах.
}
\answer{%
    $
        q = CU \implies
        C = \frac{ q }{ U } = \frac{ 25\,\text{мКл} }{ 4\,\text{кВ} } = 6250\,\text{нФ}.
        \text{Заряды обкладок: $q$ и $-q$}
    $
}
\solutionspace{120pt}

\tasknumber{2}%
\task{%
    На конденсаторе указано: $C = 50\,\text{пФ}$, $V = 200\,\text{В}$.
    Удастся ли его использовать для накопления заряда $q = 60\,\text{нКл}$?
    (в ответе укажите «да» или «нет»)
}
\answer{%
    $
        q_{ \text{ max } } = CV = 50\,\text{пФ} \cdot 200\,\text{В} = 10\,\text{нКл}
        \implies q_{ \text{ max } }  <  q \implies \text{не удастся}
    $
}
\solutionspace{80pt}

\tasknumber{3}%
\task{%
    Как и во сколько раз изменится ёмкость плоского конденсатора
    при уменьшении площади пластин в 5 раз
    и уменьшении расстояния между ними в 2 раз?
    В ответе укажите простую дробь или число — отношение новой ёмкости к старой.
}
\answer{%
    $
        \frac{C'}{C}
            = \frac{\eps_0\eps \frac S5}{\frac d2} \Big/ \frac{\eps_0\eps S}{d}
            = \frac{2}{5} = < 1 \implies \text{уменьшится в $\frac52$ раз}
    $
}
\solutionspace{80pt}

\tasknumber{4}%
\task{%
    Электрическая ёмкость конденсатора равна $C = 750\,\text{пФ}$,
    при этом ему сообщён заряд $Q = 300\,\text{нКл}$.
    Какова энергия заряженного конденсатора?
    Ответ выразите в микроджоулях и округлите до целого.
}
\answer{%
    $
        W
        = \frac{Q^2}{2C}
        = \frac{\sqr{ 300\,\text{нКл} }}{2 \cdot 750\,\text{пФ}}
        = 60{,}00\,\text{мкДж}
    $
}
\solutionspace{80pt}

\tasknumber{5}%
\task{%
    \begin{tikzpicture}[circuit ee IEC, x=1cm, y=1cm, semithick]
        \draw  (0, 0) to [capacitor={info={$C_1$}}] (1, 0)
                       to [capacitor={info={$C_2$}}] (2, 0)
        ;
        \draw [-o] (0, 0) -- ++(-0.5, 0) node[left] {$-$};
        \draw [-o] (2, 0) -- ++(0.5, 0) node[right] {$+$};

        \node [right,text width = 14cm, align=justify] at (3.5,0) {
        Два конденсатора ёмкостей $C_1 = 30\,\text{нФ}$ и $C_2 = 40\,\text{нФ}$ последовательно подключают
        к источнику напряжения $U = 200\,\text{В}$ (см.
        рис.).
        Определите заряды каждого из конденсаторов.
       };
    \end{tikzpicture}
}
\answer{%
    $
        Q_1
            = Q_2
            = CU
            = \frac{U}{\frac1{C_1} + \frac1{C_2}}
            = \frac{C_1C_2U}{C_1 + C_2}
            = \frac{
                30\,\text{нФ} \cdot 40\,\text{нФ} \cdot 200\,\text{В}
            }{
                30\,\text{нФ} + 40\,\text{нФ}
            }
            = 3428{,}57\,\text{нКл}
    $
}

\variantsplitter

\addpersonalvariant{Александр Наумов}

\tasknumber{1}%
\task{%
    Определите ёмкость конденсатора, если при его зарядке до напряжения
    $U = 4\,\text{кВ}$ он приобретает заряд $Q = 18\,\text{мКл}$.
    % Чему при этом равны заряды обкладок конденсатора (сделайте рисунок и укажите их)?
    Ответ выразите в нанофарадах.
}
\answer{%
    $
        Q = CU \implies
        C = \frac{ Q }{ U } = \frac{ 18\,\text{мКл} }{ 4\,\text{кВ} } = 4500\,\text{нФ}.
        \text{Заряды обкладок: $Q$ и $-Q$}
    $
}
\solutionspace{120pt}

\tasknumber{2}%
\task{%
    На конденсаторе указано: $C = 120\,\text{пФ}$, $U = 450\,\text{В}$.
    Удастся ли его использовать для накопления заряда $q = 50\,\text{нКл}$?
    (в ответе укажите «да» или «нет»)
}
\answer{%
    $
        q_{ \text{ max } } = CU = 120\,\text{пФ} \cdot 450\,\text{В} = 54\,\text{нКл}
        \implies q_{ \text{ max } } \ge q \implies \text{удастся}
    $
}
\solutionspace{80pt}

\tasknumber{3}%
\task{%
    Как и во сколько раз изменится ёмкость плоского конденсатора
    при уменьшении площади пластин в 4 раз
    и уменьшении расстояния между ними в 3 раз?
    В ответе укажите простую дробь или число — отношение новой ёмкости к старой.
}
\answer{%
    $
        \frac{C'}{C}
            = \frac{\eps_0\eps \frac S4}{\frac d3} \Big/ \frac{\eps_0\eps S}{d}
            = \frac{3}{4} = < 1 \implies \text{уменьшится в $\frac43$ раз}
    $
}
\solutionspace{80pt}

\tasknumber{4}%
\task{%
    Электрическая ёмкость конденсатора равна $C = 200\,\text{пФ}$,
    при этом ему сообщён заряд $Q = 800\,\text{нКл}$.
    Какова энергия заряженного конденсатора?
    Ответ выразите в микроджоулях и округлите до целого.
}
\answer{%
    $
        W
        = \frac{Q^2}{2C}
        = \frac{\sqr{ 800\,\text{нКл} }}{2 \cdot 200\,\text{пФ}}
        = 1600{,}00\,\text{мкДж}
    $
}
\solutionspace{80pt}

\tasknumber{5}%
\task{%
    \begin{tikzpicture}[circuit ee IEC, x=1cm, y=1cm, semithick]
        \draw  (0, 0) to [capacitor={info={$C_1$}}] (1, 0)
                       to [capacitor={info={$C_2$}}] (2, 0)
        ;
        \draw [-o] (0, 0) -- ++(-0.5, 0) node[left] {$-$};
        \draw [-o] (2, 0) -- ++(0.5, 0) node[right] {$+$};

        \node [right,text width = 14cm, align=justify] at (3.5,0) {
        Два конденсатора ёмкостей $C_1 = 30\,\text{нФ}$ и $C_2 = 60\,\text{нФ}$ последовательно подключают
        к источнику напряжения $V = 450\,\text{В}$ (см.
        рис.).
        Определите заряды каждого из конденсаторов.
       };
    \end{tikzpicture}
}
\answer{%
    $
        Q_1
            = Q_2
            = CV
            = \frac{V}{\frac1{C_1} + \frac1{C_2}}
            = \frac{C_1C_2V}{C_1 + C_2}
            = \frac{
                30\,\text{нФ} \cdot 60\,\text{нФ} \cdot 450\,\text{В}
            }{
                30\,\text{нФ} + 60\,\text{нФ}
            }
            = 9000{,}00\,\text{нКл}
    $
}

\variantsplitter

\addpersonalvariant{Георгий Новиков}

\tasknumber{1}%
\task{%
    Определите ёмкость конденсатора, если при его зарядке до напряжения
    $V = 20\,\text{кВ}$ он приобретает заряд $Q = 18\,\text{мКл}$.
    % Чему при этом равны заряды обкладок конденсатора (сделайте рисунок и укажите их)?
    Ответ выразите в нанофарадах.
}
\answer{%
    $
        Q = CV \implies
        C = \frac{ Q }{ V } = \frac{ 18\,\text{мКл} }{ 20\,\text{кВ} } = 900\,\text{нФ}.
        \text{Заряды обкладок: $Q$ и $-Q$}
    $
}
\solutionspace{120pt}

\tasknumber{2}%
\task{%
    На конденсаторе указано: $C = 150\,\text{пФ}$, $V = 450\,\text{В}$.
    Удастся ли его использовать для накопления заряда $q = 50\,\text{нКл}$?
    (в ответе укажите «да» или «нет»)
}
\answer{%
    $
        q_{ \text{ max } } = CV = 150\,\text{пФ} \cdot 450\,\text{В} = 67\,\text{нКл}
        \implies q_{ \text{ max } } \ge q \implies \text{удастся}
    $
}
\solutionspace{80pt}

\tasknumber{3}%
\task{%
    Как и во сколько раз изменится ёмкость плоского конденсатора
    при уменьшении площади пластин в 4 раз
    и уменьшении расстояния между ними в 8 раз?
    В ответе укажите простую дробь или число — отношение новой ёмкости к старой.
}
\answer{%
    $
        \frac{C'}{C}
            = \frac{\eps_0\eps \frac S4}{\frac d8} \Big/ \frac{\eps_0\eps S}{d}
            = \frac{8}{4} = > 1 \implies \text{увеличится в $2$ раз}
    $
}
\solutionspace{80pt}

\tasknumber{4}%
\task{%
    Электрическая ёмкость конденсатора равна $C = 200\,\text{пФ}$,
    при этом ему сообщён заряд $q = 900\,\text{нКл}$.
    Какова энергия заряженного конденсатора?
    Ответ выразите в микроджоулях и округлите до целого.
}
\answer{%
    $
        W
        = \frac{q^2}{2C}
        = \frac{\sqr{ 900\,\text{нКл} }}{2 \cdot 200\,\text{пФ}}
        = 2025{,}00\,\text{мкДж}
    $
}
\solutionspace{80pt}

\tasknumber{5}%
\task{%
    \begin{tikzpicture}[circuit ee IEC, x=1cm, y=1cm, semithick]
        \draw  (0, 0) to [capacitor={info={$C_1$}}] (1, 0)
                       to [capacitor={info={$C_2$}}] (2, 0)
        ;
        \draw [-o] (0, 0) -- ++(-0.5, 0) node[left] {$-$};
        \draw [-o] (2, 0) -- ++(0.5, 0) node[right] {$+$};

        \node [right,text width = 14cm, align=justify] at (3.5,0) {
        Два конденсатора ёмкостей $C_1 = 60\,\text{нФ}$ и $C_2 = 20\,\text{нФ}$ последовательно подключают
        к источнику напряжения $V = 200\,\text{В}$ (см.
        рис.).
        Определите заряды каждого из конденсаторов.
       };
    \end{tikzpicture}
}
\answer{%
    $
        Q_1
            = Q_2
            = CV
            = \frac{V}{\frac1{C_1} + \frac1{C_2}}
            = \frac{C_1C_2V}{C_1 + C_2}
            = \frac{
                60\,\text{нФ} \cdot 20\,\text{нФ} \cdot 200\,\text{В}
            }{
                60\,\text{нФ} + 20\,\text{нФ}
            }
            = 3000{,}00\,\text{нКл}
    $
}

\variantsplitter

\addpersonalvariant{Егор Осипов}

\tasknumber{1}%
\task{%
    Определите ёмкость конденсатора, если при его зарядке до напряжения
    $V = 20\,\text{кВ}$ он приобретает заряд $Q = 6\,\text{мКл}$.
    % Чему при этом равны заряды обкладок конденсатора (сделайте рисунок и укажите их)?
    Ответ выразите в нанофарадах.
}
\answer{%
    $
        Q = CV \implies
        C = \frac{ Q }{ V } = \frac{ 6\,\text{мКл} }{ 20\,\text{кВ} } = 300\,\text{нФ}.
        \text{Заряды обкладок: $Q$ и $-Q$}
    $
}
\solutionspace{120pt}

\tasknumber{2}%
\task{%
    На конденсаторе указано: $C = 100\,\text{пФ}$, $V = 450\,\text{В}$.
    Удастся ли его использовать для накопления заряда $q = 50\,\text{нКл}$?
    (в ответе укажите «да» или «нет»)
}
\answer{%
    $
        q_{ \text{ max } } = CV = 100\,\text{пФ} \cdot 450\,\text{В} = 45\,\text{нКл}
        \implies q_{ \text{ max } }  <  q \implies \text{не удастся}
    $
}
\solutionspace{80pt}

\tasknumber{3}%
\task{%
    Как и во сколько раз изменится ёмкость плоского конденсатора
    при уменьшении площади пластин в 2 раз
    и уменьшении расстояния между ними в 2 раз?
    В ответе укажите простую дробь или число — отношение новой ёмкости к старой.
}
\answer{%
    $
        \frac{C'}{C}
            = \frac{\eps_0\eps \frac S2}{\frac d2} \Big/ \frac{\eps_0\eps S}{d}
            = \frac{2}{2} = = 1 \implies \text{не изменится}
    $
}
\solutionspace{80pt}

\tasknumber{4}%
\task{%
    Электрическая ёмкость конденсатора равна $C = 400\,\text{пФ}$,
    при этом ему сообщён заряд $Q = 300\,\text{нКл}$.
    Какова энергия заряженного конденсатора?
    Ответ выразите в микроджоулях и округлите до целого.
}
\answer{%
    $
        W
        = \frac{Q^2}{2C}
        = \frac{\sqr{ 300\,\text{нКл} }}{2 \cdot 400\,\text{пФ}}
        = 112{,}50\,\text{мкДж}
    $
}
\solutionspace{80pt}

\tasknumber{5}%
\task{%
    \begin{tikzpicture}[circuit ee IEC, x=1cm, y=1cm, semithick]
        \draw  (0, 0) to [capacitor={info={$C_1$}}] (1, 0)
                       to [capacitor={info={$C_2$}}] (2, 0)
        ;
        \draw [-o] (0, 0) -- ++(-0.5, 0) node[left] {$-$};
        \draw [-o] (2, 0) -- ++(0.5, 0) node[right] {$+$};

        \node [right,text width = 14cm, align=justify] at (3.5,0) {
        Два конденсатора ёмкостей $C_1 = 30\,\text{нФ}$ и $C_2 = 60\,\text{нФ}$ последовательно подключают
        к источнику напряжения $V = 450\,\text{В}$ (см.
        рис.).
        Определите заряды каждого из конденсаторов.
       };
    \end{tikzpicture}
}
\answer{%
    $
        Q_1
            = Q_2
            = CV
            = \frac{V}{\frac1{C_1} + \frac1{C_2}}
            = \frac{C_1C_2V}{C_1 + C_2}
            = \frac{
                30\,\text{нФ} \cdot 60\,\text{нФ} \cdot 450\,\text{В}
            }{
                30\,\text{нФ} + 60\,\text{нФ}
            }
            = 9000{,}00\,\text{нКл}
    $
}

\variantsplitter

\addpersonalvariant{Руслан Перепелица}

\tasknumber{1}%
\task{%
    Определите ёмкость конденсатора, если при его зарядке до напряжения
    $V = 4\,\text{кВ}$ он приобретает заряд $Q = 15\,\text{мКл}$.
    % Чему при этом равны заряды обкладок конденсатора (сделайте рисунок и укажите их)?
    Ответ выразите в нанофарадах.
}
\answer{%
    $
        Q = CV \implies
        C = \frac{ Q }{ V } = \frac{ 15\,\text{мКл} }{ 4\,\text{кВ} } = 3750\,\text{нФ}.
        \text{Заряды обкладок: $Q$ и $-Q$}
    $
}
\solutionspace{120pt}

\tasknumber{2}%
\task{%
    На конденсаторе указано: $C = 100\,\text{пФ}$, $U = 200\,\text{В}$.
    Удастся ли его использовать для накопления заряда $q = 50\,\text{нКл}$?
    (в ответе укажите «да» или «нет»)
}
\answer{%
    $
        q_{ \text{ max } } = CU = 100\,\text{пФ} \cdot 200\,\text{В} = 20\,\text{нКл}
        \implies q_{ \text{ max } }  <  q \implies \text{не удастся}
    $
}
\solutionspace{80pt}

\tasknumber{3}%
\task{%
    Как и во сколько раз изменится ёмкость плоского конденсатора
    при уменьшении площади пластин в 4 раз
    и уменьшении расстояния между ними в 4 раз?
    В ответе укажите простую дробь или число — отношение новой ёмкости к старой.
}
\answer{%
    $
        \frac{C'}{C}
            = \frac{\eps_0\eps \frac S4}{\frac d4} \Big/ \frac{\eps_0\eps S}{d}
            = \frac{4}{4} = = 1 \implies \text{не изменится}
    $
}
\solutionspace{80pt}

\tasknumber{4}%
\task{%
    Электрическая ёмкость конденсатора равна $C = 600\,\text{пФ}$,
    при этом ему сообщён заряд $q = 900\,\text{нКл}$.
    Какова энергия заряженного конденсатора?
    Ответ выразите в микроджоулях и округлите до целого.
}
\answer{%
    $
        W
        = \frac{q^2}{2C}
        = \frac{\sqr{ 900\,\text{нКл} }}{2 \cdot 600\,\text{пФ}}
        = 675{,}00\,\text{мкДж}
    $
}
\solutionspace{80pt}

\tasknumber{5}%
\task{%
    \begin{tikzpicture}[circuit ee IEC, x=1cm, y=1cm, semithick]
        \draw  (0, 0) to [capacitor={info={$C_1$}}] (1, 0)
                       to [capacitor={info={$C_2$}}] (2, 0)
        ;
        \draw [-o] (0, 0) -- ++(-0.5, 0) node[left] {$-$};
        \draw [-o] (2, 0) -- ++(0.5, 0) node[right] {$+$};

        \node [right,text width = 14cm, align=justify] at (3.5,0) {
        Два конденсатора ёмкостей $C_1 = 20\,\text{нФ}$ и $C_2 = 60\,\text{нФ}$ последовательно подключают
        к источнику напряжения $V = 200\,\text{В}$ (см.
        рис.).
        Определите заряды каждого из конденсаторов.
       };
    \end{tikzpicture}
}
\answer{%
    $
        Q_1
            = Q_2
            = CV
            = \frac{V}{\frac1{C_1} + \frac1{C_2}}
            = \frac{C_1C_2V}{C_1 + C_2}
            = \frac{
                20\,\text{нФ} \cdot 60\,\text{нФ} \cdot 200\,\text{В}
            }{
                20\,\text{нФ} + 60\,\text{нФ}
            }
            = 3000{,}00\,\text{нКл}
    $
}

\variantsplitter

\addpersonalvariant{Михаил Перин}

\tasknumber{1}%
\task{%
    Определите ёмкость конденсатора, если при его зарядке до напряжения
    $V = 4\,\text{кВ}$ он приобретает заряд $q = 24\,\text{мКл}$.
    % Чему при этом равны заряды обкладок конденсатора (сделайте рисунок и укажите их)?
    Ответ выразите в нанофарадах.
}
\answer{%
    $
        q = CV \implies
        C = \frac{ q }{ V } = \frac{ 24\,\text{мКл} }{ 4\,\text{кВ} } = 6000\,\text{нФ}.
        \text{Заряды обкладок: $q$ и $-q$}
    $
}
\solutionspace{120pt}

\tasknumber{2}%
\task{%
    На конденсаторе указано: $C = 50\,\text{пФ}$, $U = 450\,\text{В}$.
    Удастся ли его использовать для накопления заряда $q = 30\,\text{нКл}$?
    (в ответе укажите «да» или «нет»)
}
\answer{%
    $
        q_{ \text{ max } } = CU = 50\,\text{пФ} \cdot 450\,\text{В} = 22\,\text{нКл}
        \implies q_{ \text{ max } }  <  q \implies \text{не удастся}
    $
}
\solutionspace{80pt}

\tasknumber{3}%
\task{%
    Как и во сколько раз изменится ёмкость плоского конденсатора
    при уменьшении площади пластин в 7 раз
    и уменьшении расстояния между ними в 4 раз?
    В ответе укажите простую дробь или число — отношение новой ёмкости к старой.
}
\answer{%
    $
        \frac{C'}{C}
            = \frac{\eps_0\eps \frac S7}{\frac d4} \Big/ \frac{\eps_0\eps S}{d}
            = \frac{4}{7} = < 1 \implies \text{уменьшится в $\frac74$ раз}
    $
}
\solutionspace{80pt}

\tasknumber{4}%
\task{%
    Электрическая ёмкость конденсатора равна $C = 400\,\text{пФ}$,
    при этом ему сообщён заряд $q = 500\,\text{нКл}$.
    Какова энергия заряженного конденсатора?
    Ответ выразите в микроджоулях и округлите до целого.
}
\answer{%
    $
        W
        = \frac{q^2}{2C}
        = \frac{\sqr{ 500\,\text{нКл} }}{2 \cdot 400\,\text{пФ}}
        = 312{,}50\,\text{мкДж}
    $
}
\solutionspace{80pt}

\tasknumber{5}%
\task{%
    \begin{tikzpicture}[circuit ee IEC, x=1cm, y=1cm, semithick]
        \draw  (0, 0) to [capacitor={info={$C_1$}}] (1, 0)
                       to [capacitor={info={$C_2$}}] (2, 0)
        ;
        \draw [-o] (0, 0) -- ++(-0.5, 0) node[left] {$-$};
        \draw [-o] (2, 0) -- ++(0.5, 0) node[right] {$+$};

        \node [right,text width = 14cm, align=justify] at (3.5,0) {
        Два конденсатора ёмкостей $C_1 = 40\,\text{нФ}$ и $C_2 = 60\,\text{нФ}$ последовательно подключают
        к источнику напряжения $U = 450\,\text{В}$ (см.
        рис.).
        Определите заряды каждого из конденсаторов.
       };
    \end{tikzpicture}
}
\answer{%
    $
        Q_1
            = Q_2
            = CU
            = \frac{U}{\frac1{C_1} + \frac1{C_2}}
            = \frac{C_1C_2U}{C_1 + C_2}
            = \frac{
                40\,\text{нФ} \cdot 60\,\text{нФ} \cdot 450\,\text{В}
            }{
                40\,\text{нФ} + 60\,\text{нФ}
            }
            = 10800{,}00\,\text{нКл}
    $
}

\variantsplitter

\addpersonalvariant{Егор Подуровский}

\tasknumber{1}%
\task{%
    Определите ёмкость конденсатора, если при его зарядке до напряжения
    $V = 20\,\text{кВ}$ он приобретает заряд $Q = 6\,\text{мКл}$.
    % Чему при этом равны заряды обкладок конденсатора (сделайте рисунок и укажите их)?
    Ответ выразите в нанофарадах.
}
\answer{%
    $
        Q = CV \implies
        C = \frac{ Q }{ V } = \frac{ 6\,\text{мКл} }{ 20\,\text{кВ} } = 300\,\text{нФ}.
        \text{Заряды обкладок: $Q$ и $-Q$}
    $
}
\solutionspace{120pt}

\tasknumber{2}%
\task{%
    На конденсаторе указано: $C = 150\,\text{пФ}$, $U = 450\,\text{В}$.
    Удастся ли его использовать для накопления заряда $q = 30\,\text{нКл}$?
    (в ответе укажите «да» или «нет»)
}
\answer{%
    $
        q_{ \text{ max } } = CU = 150\,\text{пФ} \cdot 450\,\text{В} = 67\,\text{нКл}
        \implies q_{ \text{ max } } \ge q \implies \text{удастся}
    $
}
\solutionspace{80pt}

\tasknumber{3}%
\task{%
    Как и во сколько раз изменится ёмкость плоского конденсатора
    при уменьшении площади пластин в 8 раз
    и уменьшении расстояния между ними в 5 раз?
    В ответе укажите простую дробь или число — отношение новой ёмкости к старой.
}
\answer{%
    $
        \frac{C'}{C}
            = \frac{\eps_0\eps \frac S8}{\frac d5} \Big/ \frac{\eps_0\eps S}{d}
            = \frac{5}{8} = < 1 \implies \text{уменьшится в $\frac85$ раз}
    $
}
\solutionspace{80pt}

\tasknumber{4}%
\task{%
    Электрическая ёмкость конденсатора равна $C = 600\,\text{пФ}$,
    при этом ему сообщён заряд $q = 800\,\text{нКл}$.
    Какова энергия заряженного конденсатора?
    Ответ выразите в микроджоулях и округлите до целого.
}
\answer{%
    $
        W
        = \frac{q^2}{2C}
        = \frac{\sqr{ 800\,\text{нКл} }}{2 \cdot 600\,\text{пФ}}
        = 533{,}33\,\text{мкДж}
    $
}
\solutionspace{80pt}

\tasknumber{5}%
\task{%
    \begin{tikzpicture}[circuit ee IEC, x=1cm, y=1cm, semithick]
        \draw  (0, 0) to [capacitor={info={$C_1$}}] (1, 0)
                       to [capacitor={info={$C_2$}}] (2, 0)
        ;
        \draw [-o] (0, 0) -- ++(-0.5, 0) node[left] {$-$};
        \draw [-o] (2, 0) -- ++(0.5, 0) node[right] {$+$};

        \node [right,text width = 14cm, align=justify] at (3.5,0) {
        Два конденсатора ёмкостей $C_1 = 40\,\text{нФ}$ и $C_2 = 60\,\text{нФ}$ последовательно подключают
        к источнику напряжения $U = 300\,\text{В}$ (см.
        рис.).
        Определите заряды каждого из конденсаторов.
       };
    \end{tikzpicture}
}
\answer{%
    $
        Q_1
            = Q_2
            = CU
            = \frac{U}{\frac1{C_1} + \frac1{C_2}}
            = \frac{C_1C_2U}{C_1 + C_2}
            = \frac{
                40\,\text{нФ} \cdot 60\,\text{нФ} \cdot 300\,\text{В}
            }{
                40\,\text{нФ} + 60\,\text{нФ}
            }
            = 7200{,}00\,\text{нКл}
    $
}

\variantsplitter

\addpersonalvariant{Роман Прибылов}

\tasknumber{1}%
\task{%
    Определите ёмкость конденсатора, если при его зарядке до напряжения
    $U = 20\,\text{кВ}$ он приобретает заряд $Q = 6\,\text{мКл}$.
    % Чему при этом равны заряды обкладок конденсатора (сделайте рисунок и укажите их)?
    Ответ выразите в нанофарадах.
}
\answer{%
    $
        Q = CU \implies
        C = \frac{ Q }{ U } = \frac{ 6\,\text{мКл} }{ 20\,\text{кВ} } = 300\,\text{нФ}.
        \text{Заряды обкладок: $Q$ и $-Q$}
    $
}
\solutionspace{120pt}

\tasknumber{2}%
\task{%
    На конденсаторе указано: $C = 120\,\text{пФ}$, $V = 300\,\text{В}$.
    Удастся ли его использовать для накопления заряда $q = 50\,\text{нКл}$?
    (в ответе укажите «да» или «нет»)
}
\answer{%
    $
        q_{ \text{ max } } = CV = 120\,\text{пФ} \cdot 300\,\text{В} = 36\,\text{нКл}
        \implies q_{ \text{ max } }  <  q \implies \text{не удастся}
    $
}
\solutionspace{80pt}

\tasknumber{3}%
\task{%
    Как и во сколько раз изменится ёмкость плоского конденсатора
    при уменьшении площади пластин в 8 раз
    и уменьшении расстояния между ними в 3 раз?
    В ответе укажите простую дробь или число — отношение новой ёмкости к старой.
}
\answer{%
    $
        \frac{C'}{C}
            = \frac{\eps_0\eps \frac S8}{\frac d3} \Big/ \frac{\eps_0\eps S}{d}
            = \frac{3}{8} = < 1 \implies \text{уменьшится в $\frac83$ раз}
    $
}
\solutionspace{80pt}

\tasknumber{4}%
\task{%
    Электрическая ёмкость конденсатора равна $C = 600\,\text{пФ}$,
    при этом ему сообщён заряд $Q = 500\,\text{нКл}$.
    Какова энергия заряженного конденсатора?
    Ответ выразите в микроджоулях и округлите до целого.
}
\answer{%
    $
        W
        = \frac{Q^2}{2C}
        = \frac{\sqr{ 500\,\text{нКл} }}{2 \cdot 600\,\text{пФ}}
        = 208{,}33\,\text{мкДж}
    $
}
\solutionspace{80pt}

\tasknumber{5}%
\task{%
    \begin{tikzpicture}[circuit ee IEC, x=1cm, y=1cm, semithick]
        \draw  (0, 0) to [capacitor={info={$C_1$}}] (1, 0)
                       to [capacitor={info={$C_2$}}] (2, 0)
        ;
        \draw [-o] (0, 0) -- ++(-0.5, 0) node[left] {$-$};
        \draw [-o] (2, 0) -- ++(0.5, 0) node[right] {$+$};

        \node [right,text width = 14cm, align=justify] at (3.5,0) {
        Два конденсатора ёмкостей $C_1 = 60\,\text{нФ}$ и $C_2 = 30\,\text{нФ}$ последовательно подключают
        к источнику напряжения $U = 200\,\text{В}$ (см.
        рис.).
        Определите заряды каждого из конденсаторов.
       };
    \end{tikzpicture}
}
\answer{%
    $
        Q_1
            = Q_2
            = CU
            = \frac{U}{\frac1{C_1} + \frac1{C_2}}
            = \frac{C_1C_2U}{C_1 + C_2}
            = \frac{
                60\,\text{нФ} \cdot 30\,\text{нФ} \cdot 200\,\text{В}
            }{
                60\,\text{нФ} + 30\,\text{нФ}
            }
            = 4000{,}00\,\text{нКл}
    $
}

\variantsplitter

\addpersonalvariant{Александр Селехметьев}

\tasknumber{1}%
\task{%
    Определите ёмкость конденсатора, если при его зарядке до напряжения
    $U = 20\,\text{кВ}$ он приобретает заряд $q = 25\,\text{мКл}$.
    % Чему при этом равны заряды обкладок конденсатора (сделайте рисунок и укажите их)?
    Ответ выразите в нанофарадах.
}
\answer{%
    $
        q = CU \implies
        C = \frac{ q }{ U } = \frac{ 25\,\text{мКл} }{ 20\,\text{кВ} } = 1250\,\text{нФ}.
        \text{Заряды обкладок: $q$ и $-q$}
    $
}
\solutionspace{120pt}

\tasknumber{2}%
\task{%
    На конденсаторе указано: $C = 100\,\text{пФ}$, $V = 400\,\text{В}$.
    Удастся ли его использовать для накопления заряда $q = 60\,\text{нКл}$?
    (в ответе укажите «да» или «нет»)
}
\answer{%
    $
        q_{ \text{ max } } = CV = 100\,\text{пФ} \cdot 400\,\text{В} = 40\,\text{нКл}
        \implies q_{ \text{ max } }  <  q \implies \text{не удастся}
    $
}
\solutionspace{80pt}

\tasknumber{3}%
\task{%
    Как и во сколько раз изменится ёмкость плоского конденсатора
    при уменьшении площади пластин в 5 раз
    и уменьшении расстояния между ними в 4 раз?
    В ответе укажите простую дробь или число — отношение новой ёмкости к старой.
}
\answer{%
    $
        \frac{C'}{C}
            = \frac{\eps_0\eps \frac S5}{\frac d4} \Big/ \frac{\eps_0\eps S}{d}
            = \frac{4}{5} = < 1 \implies \text{уменьшится в $\frac54$ раз}
    $
}
\solutionspace{80pt}

\tasknumber{4}%
\task{%
    Электрическая ёмкость конденсатора равна $C = 400\,\text{пФ}$,
    при этом ему сообщён заряд $q = 900\,\text{нКл}$.
    Какова энергия заряженного конденсатора?
    Ответ выразите в микроджоулях и округлите до целого.
}
\answer{%
    $
        W
        = \frac{q^2}{2C}
        = \frac{\sqr{ 900\,\text{нКл} }}{2 \cdot 400\,\text{пФ}}
        = 1012{,}50\,\text{мкДж}
    $
}
\solutionspace{80pt}

\tasknumber{5}%
\task{%
    \begin{tikzpicture}[circuit ee IEC, x=1cm, y=1cm, semithick]
        \draw  (0, 0) to [capacitor={info={$C_1$}}] (1, 0)
                       to [capacitor={info={$C_2$}}] (2, 0)
        ;
        \draw [-o] (0, 0) -- ++(-0.5, 0) node[left] {$-$};
        \draw [-o] (2, 0) -- ++(0.5, 0) node[right] {$+$};

        \node [right,text width = 14cm, align=justify] at (3.5,0) {
        Два конденсатора ёмкостей $C_1 = 60\,\text{нФ}$ и $C_2 = 40\,\text{нФ}$ последовательно подключают
        к источнику напряжения $V = 150\,\text{В}$ (см.
        рис.).
        Определите заряды каждого из конденсаторов.
       };
    \end{tikzpicture}
}
\answer{%
    $
        Q_1
            = Q_2
            = CV
            = \frac{V}{\frac1{C_1} + \frac1{C_2}}
            = \frac{C_1C_2V}{C_1 + C_2}
            = \frac{
                60\,\text{нФ} \cdot 40\,\text{нФ} \cdot 150\,\text{В}
            }{
                60\,\text{нФ} + 40\,\text{нФ}
            }
            = 3600{,}00\,\text{нКл}
    $
}

\variantsplitter

\addpersonalvariant{Алексей Тихонов}

\tasknumber{1}%
\task{%
    Определите ёмкость конденсатора, если при его зарядке до напряжения
    $U = 40\,\text{кВ}$ он приобретает заряд $q = 18\,\text{мКл}$.
    % Чему при этом равны заряды обкладок конденсатора (сделайте рисунок и укажите их)?
    Ответ выразите в нанофарадах.
}
\answer{%
    $
        q = CU \implies
        C = \frac{ q }{ U } = \frac{ 18\,\text{мКл} }{ 40\,\text{кВ} } = 450\,\text{нФ}.
        \text{Заряды обкладок: $q$ и $-q$}
    $
}
\solutionspace{120pt}

\tasknumber{2}%
\task{%
    На конденсаторе указано: $C = 100\,\text{пФ}$, $V = 200\,\text{В}$.
    Удастся ли его использовать для накопления заряда $q = 30\,\text{нКл}$?
    (в ответе укажите «да» или «нет»)
}
\answer{%
    $
        q_{ \text{ max } } = CV = 100\,\text{пФ} \cdot 200\,\text{В} = 20\,\text{нКл}
        \implies q_{ \text{ max } }  <  q \implies \text{не удастся}
    $
}
\solutionspace{80pt}

\tasknumber{3}%
\task{%
    Как и во сколько раз изменится ёмкость плоского конденсатора
    при уменьшении площади пластин в 6 раз
    и уменьшении расстояния между ними в 7 раз?
    В ответе укажите простую дробь или число — отношение новой ёмкости к старой.
}
\answer{%
    $
        \frac{C'}{C}
            = \frac{\eps_0\eps \frac S6}{\frac d7} \Big/ \frac{\eps_0\eps S}{d}
            = \frac{7}{6} = > 1 \implies \text{увеличится в $\frac76$ раз}
    $
}
\solutionspace{80pt}

\tasknumber{4}%
\task{%
    Электрическая ёмкость конденсатора равна $C = 750\,\text{пФ}$,
    при этом ему сообщён заряд $q = 500\,\text{нКл}$.
    Какова энергия заряженного конденсатора?
    Ответ выразите в микроджоулях и округлите до целого.
}
\answer{%
    $
        W
        = \frac{q^2}{2C}
        = \frac{\sqr{ 500\,\text{нКл} }}{2 \cdot 750\,\text{пФ}}
        = 166{,}67\,\text{мкДж}
    $
}
\solutionspace{80pt}

\tasknumber{5}%
\task{%
    \begin{tikzpicture}[circuit ee IEC, x=1cm, y=1cm, semithick]
        \draw  (0, 0) to [capacitor={info={$C_1$}}] (1, 0)
                       to [capacitor={info={$C_2$}}] (2, 0)
        ;
        \draw [-o] (0, 0) -- ++(-0.5, 0) node[left] {$-$};
        \draw [-o] (2, 0) -- ++(0.5, 0) node[right] {$+$};

        \node [right,text width = 14cm, align=justify] at (3.5,0) {
        Два конденсатора ёмкостей $C_1 = 20\,\text{нФ}$ и $C_2 = 30\,\text{нФ}$ последовательно подключают
        к источнику напряжения $U = 400\,\text{В}$ (см.
        рис.).
        Определите заряды каждого из конденсаторов.
       };
    \end{tikzpicture}
}
\answer{%
    $
        Q_1
            = Q_2
            = CU
            = \frac{U}{\frac1{C_1} + \frac1{C_2}}
            = \frac{C_1C_2U}{C_1 + C_2}
            = \frac{
                20\,\text{нФ} \cdot 30\,\text{нФ} \cdot 400\,\text{В}
            }{
                20\,\text{нФ} + 30\,\text{нФ}
            }
            = 4800{,}00\,\text{нКл}
    $
}

\variantsplitter

\addpersonalvariant{Алина Филиппова}

\tasknumber{1}%
\task{%
    Определите ёмкость конденсатора, если при его зарядке до напряжения
    $V = 40\,\text{кВ}$ он приобретает заряд $q = 18\,\text{мКл}$.
    % Чему при этом равны заряды обкладок конденсатора (сделайте рисунок и укажите их)?
    Ответ выразите в нанофарадах.
}
\answer{%
    $
        q = CV \implies
        C = \frac{ q }{ V } = \frac{ 18\,\text{мКл} }{ 40\,\text{кВ} } = 450\,\text{нФ}.
        \text{Заряды обкладок: $q$ и $-q$}
    $
}
\solutionspace{120pt}

\tasknumber{2}%
\task{%
    На конденсаторе указано: $C = 120\,\text{пФ}$, $V = 450\,\text{В}$.
    Удастся ли его использовать для накопления заряда $Q = 50\,\text{нКл}$?
    (в ответе укажите «да» или «нет»)
}
\answer{%
    $
        Q_{ \text{ max } } = CV = 120\,\text{пФ} \cdot 450\,\text{В} = 54\,\text{нКл}
        \implies Q_{ \text{ max } } \ge Q \implies \text{удастся}
    $
}
\solutionspace{80pt}

\tasknumber{3}%
\task{%
    Как и во сколько раз изменится ёмкость плоского конденсатора
    при уменьшении площади пластин в 7 раз
    и уменьшении расстояния между ними в 5 раз?
    В ответе укажите простую дробь или число — отношение новой ёмкости к старой.
}
\answer{%
    $
        \frac{C'}{C}
            = \frac{\eps_0\eps \frac S7}{\frac d5} \Big/ \frac{\eps_0\eps S}{d}
            = \frac{5}{7} = < 1 \implies \text{уменьшится в $\frac75$ раз}
    $
}
\solutionspace{80pt}

\tasknumber{4}%
\task{%
    Электрическая ёмкость конденсатора равна $C = 600\,\text{пФ}$,
    при этом ему сообщён заряд $q = 800\,\text{нКл}$.
    Какова энергия заряженного конденсатора?
    Ответ выразите в микроджоулях и округлите до целого.
}
\answer{%
    $
        W
        = \frac{q^2}{2C}
        = \frac{\sqr{ 800\,\text{нКл} }}{2 \cdot 600\,\text{пФ}}
        = 533{,}33\,\text{мкДж}
    $
}
\solutionspace{80pt}

\tasknumber{5}%
\task{%
    \begin{tikzpicture}[circuit ee IEC, x=1cm, y=1cm, semithick]
        \draw  (0, 0) to [capacitor={info={$C_1$}}] (1, 0)
                       to [capacitor={info={$C_2$}}] (2, 0)
        ;
        \draw [-o] (0, 0) -- ++(-0.5, 0) node[left] {$-$};
        \draw [-o] (2, 0) -- ++(0.5, 0) node[right] {$+$};

        \node [right,text width = 14cm, align=justify] at (3.5,0) {
        Два конденсатора ёмкостей $C_1 = 30\,\text{нФ}$ и $C_2 = 40\,\text{нФ}$ последовательно подключают
        к источнику напряжения $V = 200\,\text{В}$ (см.
        рис.).
        Определите заряды каждого из конденсаторов.
       };
    \end{tikzpicture}
}
\answer{%
    $
        Q_1
            = Q_2
            = CV
            = \frac{V}{\frac1{C_1} + \frac1{C_2}}
            = \frac{C_1C_2V}{C_1 + C_2}
            = \frac{
                30\,\text{нФ} \cdot 40\,\text{нФ} \cdot 200\,\text{В}
            }{
                30\,\text{нФ} + 40\,\text{нФ}
            }
            = 3428{,}57\,\text{нКл}
    $
}

\variantsplitter

\addpersonalvariant{Алина Яшина}

\tasknumber{1}%
\task{%
    Определите ёмкость конденсатора, если при его зарядке до напряжения
    $U = 20\,\text{кВ}$ он приобретает заряд $q = 25\,\text{мКл}$.
    % Чему при этом равны заряды обкладок конденсатора (сделайте рисунок и укажите их)?
    Ответ выразите в нанофарадах.
}
\answer{%
    $
        q = CU \implies
        C = \frac{ q }{ U } = \frac{ 25\,\text{мКл} }{ 20\,\text{кВ} } = 1250\,\text{нФ}.
        \text{Заряды обкладок: $q$ и $-q$}
    $
}
\solutionspace{120pt}

\tasknumber{2}%
\task{%
    На конденсаторе указано: $C = 100\,\text{пФ}$, $U = 450\,\text{В}$.
    Удастся ли его использовать для накопления заряда $Q = 30\,\text{нКл}$?
    (в ответе укажите «да» или «нет»)
}
\answer{%
    $
        Q_{ \text{ max } } = CU = 100\,\text{пФ} \cdot 450\,\text{В} = 45\,\text{нКл}
        \implies Q_{ \text{ max } } \ge Q \implies \text{удастся}
    $
}
\solutionspace{80pt}

\tasknumber{3}%
\task{%
    Как и во сколько раз изменится ёмкость плоского конденсатора
    при уменьшении площади пластин в 6 раз
    и уменьшении расстояния между ними в 6 раз?
    В ответе укажите простую дробь или число — отношение новой ёмкости к старой.
}
\answer{%
    $
        \frac{C'}{C}
            = \frac{\eps_0\eps \frac S6}{\frac d6} \Big/ \frac{\eps_0\eps S}{d}
            = \frac{6}{6} = = 1 \implies \text{не изменится}
    $
}
\solutionspace{80pt}

\tasknumber{4}%
\task{%
    Электрическая ёмкость конденсатора равна $C = 200\,\text{пФ}$,
    при этом ему сообщён заряд $Q = 800\,\text{нКл}$.
    Какова энергия заряженного конденсатора?
    Ответ выразите в микроджоулях и округлите до целого.
}
\answer{%
    $
        W
        = \frac{Q^2}{2C}
        = \frac{\sqr{ 800\,\text{нКл} }}{2 \cdot 200\,\text{пФ}}
        = 1600{,}00\,\text{мкДж}
    $
}
\solutionspace{80pt}

\tasknumber{5}%
\task{%
    \begin{tikzpicture}[circuit ee IEC, x=1cm, y=1cm, semithick]
        \draw  (0, 0) to [capacitor={info={$C_1$}}] (1, 0)
                       to [capacitor={info={$C_2$}}] (2, 0)
        ;
        \draw [-o] (0, 0) -- ++(-0.5, 0) node[left] {$-$};
        \draw [-o] (2, 0) -- ++(0.5, 0) node[right] {$+$};

        \node [right,text width = 14cm, align=justify] at (3.5,0) {
        Два конденсатора ёмкостей $C_1 = 30\,\text{нФ}$ и $C_2 = 20\,\text{нФ}$ последовательно подключают
        к источнику напряжения $V = 150\,\text{В}$ (см.
        рис.).
        Определите заряды каждого из конденсаторов.
       };
    \end{tikzpicture}
}
\answer{%
    $
        Q_1
            = Q_2
            = CV
            = \frac{V}{\frac1{C_1} + \frac1{C_2}}
            = \frac{C_1C_2V}{C_1 + C_2}
            = \frac{
                30\,\text{нФ} \cdot 20\,\text{нФ} \cdot 150\,\text{В}
            }{
                30\,\text{нФ} + 20\,\text{нФ}
            }
            = 1800{,}00\,\text{нКл}
    $
}
% autogenerated
