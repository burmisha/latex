\newcommand\rootpath{../../..}
\documentclass[12pt,a4paper]{amsart}%DVI-mode.
\usepackage{graphics,graphicx,epsfig}%DVI-mode.
% \documentclass[pdftex,12pt]{amsart} %PDF-mode.
% \usepackage[pdftex]{graphicx}       %PDF-mode.
% \usepackage[babel=true]{microtype}
% \usepackage[T1]{fontenc}
% \usepackage{lmodern}

\usepackage{cmap}
%\usepackage{a4wide}                 % Fit the text to A4 page tightly.
% \usepackage[utf8]{inputenc}
\usepackage[T2A]{fontenc}
\usepackage[english,russian]{babel} % Download Russian fonts.
\usepackage{amsmath,amsfonts,amssymb,amsthm,amscd,mathrsfs} % Use AMS symbols.
\usepackage{tikz}
\usetikzlibrary{circuits.ee.IEC}
\usetikzlibrary{shapes.geometric}
\usetikzlibrary{decorations.markings}
%\usetikzlibrary{dashs}
%\usetikzlibrary{info}


\textheight=28cm % высота текста
\textwidth=18cm % ширина текста
\topmargin=-2.5cm % отступ от верхнего края
\parskip=2pt % интервал между абзацами
\oddsidemargin=-1.5cm
\evensidemargin=-1.5cm 

\parindent=0pt % абзацный отступ
\tolerance=500 % терпимость к "жидким" строкам
\binoppenalty=10000 % штраф за перенос формул - 10000 - абсолютный запрет
\relpenalty=10000
\flushbottom % выравнивание высоты страниц
\pagenumbering{gobble}

\newcommand\bivec[2]{\begin{pmatrix} #1 \\ #2 \end{pmatrix}}

\newcommand\ol[1]{\overline{#1}}

\newcommand\p[1]{\Prob\!\left(#1\right)}
\newcommand\e[1]{\mathsf{E}\!\left(#1\right)}
\newcommand\disp[1]{\mathsf{D}\!\left(#1\right)}
%\newcommand\norm[2]{\mathcal{N}\!\cbr{#1,#2}}
\newcommand\sign{\text{ sign }}

\newcommand\al[1]{\begin{align*} #1 \end{align*}}
\newcommand\begcas[1]{\begin{cases}#1\end{cases}}
\newcommand\tab[2]{	\vspace{-#1pt}
						\begin{tabbing} 
						#2
						\end{tabbing}
					\vspace{-#1pt}
					}

\newcommand\maintext[1]{{\bfseries\sffamily{#1}}}
\newcommand\skipped[1]{ {\ensuremath{\text{\small{\sffamily{Пропущено:} #1} } } } }
\newcommand\simpletitle[1]{\begin{center} \maintext{#1} \end{center}}

\def\le{\leqslant}
\def\ge{\geqslant}
\def\Ell{\mathcal{L}}
\def\eps{{\varepsilon}}
\def\Rn{\mathbb{R}^n}
\def\RSS{\mathsf{RSS}}

\newcommand\foral[1]{\forall\,#1\:}
\newcommand\exist[1]{\exists\,#1\:\colon}

\newcommand\cbr[1]{\left(#1\right)} %circled brackets
\newcommand\fbr[1]{\left\{#1\right\}} %figure brackets
\newcommand\sbr[1]{\left[#1\right]} %square brackets
\newcommand\modul[1]{\left|#1\right|}

\newcommand\sqr[1]{\cbr{#1}^2}
\newcommand\inv[1]{\cbr{#1}^{-1}}

\newcommand\cdf[2]{\cdot\frac{#1}{#2}}
\newcommand\dd[2]{\frac{\partial#1}{\partial#2}}

\newcommand\integr[2]{\int\limits_{#1}^{#2}}
\newcommand\suml[2]{\sum\limits_{#1}^{#2}}
\newcommand\isum[2]{\sum\limits_{#1=#2}^{+\infty}}
\newcommand\idots[3]{#1_{#2},\ldots,#1_{#3}}
\newcommand\fdots[5]{#4{#1_{#2}}#5\ldots#5#4{#1_{#3}}}

\newcommand\obol[1]{O\!\cbr{#1}}
\newcommand\omal[1]{o\!\cbr{#1}}

\newcommand\addeps[2]{
	\begin{figure} [!ht] %lrp
		\centering
		\includegraphics[height=320px]{#1.eps}
		\vspace{-10pt}
		\caption{#2}
		\label{eps:#1}
	\end{figure}
}

\newcommand\addepssize[3]{
	\begin{figure} [!ht] %lrp hp
		\centering
		\includegraphics[height=#3px]{#1.eps}
		\vspace{-10pt}
		\caption{#2}
		\label{eps:#1}
	\end{figure}
}


\newcommand\norm[1]{\ensuremath{\left\|{#1}\right\|}}
\newcommand\ort{\bot}
\newcommand\theorem[1]{{\sffamily Теорема #1\ }}
\newcommand\lemma[1]{{\sffamily Лемма #1\ }}
\newcommand\difflim[2]{\frac{#1\cbr{#2 + \Delta#2} - #1\cbr{#2}}{\Delta #2}}
\renewcommand\proof[1]{\par\noindent$\square$ #1 \hfill$\blacksquare$\par}
\newcommand\defenition[1]{{\sffamilyОпределение #1\ }}

% \begin{document}
% %\raggedright
% \addclassdate{7}{20 апреля 2018}

\task 1
Площадь большого поршня гидравлического домкрата $S_1 = 20\units{см}^2$, а малого $S_2 = 0{,}5\units{см}^2.$ Груз какой максимальной массы можно поднять этим домкратом, если на малый поршень давить с силой не более $F=200\units{Н}?$ Силой трения от стенки цилиндров пренебречь.

\task 2
В сосуд налита вода. Расстояние от поверхности воды до дна $H = 0{,}5\units{м},$ площадь дна $S = 0{,}1\units{м}^2.$ Найти гидростатическое давление $P_1$ и полное давление $P_2$ вблизи дна. Найти силу давления воды на дно. Плотность воды \rhowater

\task 3
На лёгкий поршень площадью $S=900\units{см}^2,$ касающийся поверхности воды, поставили гирю массы $m=3\units{кг}$. Высота слоя воды в сосуде с вертикальными стенками $H = 20\units{см}$. Определить давление жидкости вблизи дна, если плотность воды \rhowater

\task 4
Давление газов в конце сгорания в цилиндре дизельного двигателя трактора $P = 9\units{МПа}.$ Диаметр цилиндра $d = 130\units{мм}.$ С какой силой газы давят на поршень в цилиндре? Площадь круга диаметром $D$ равна $S = \cfrac{\pi D^2}4.$

\task 5
Площадь малого поршня гидравлического подъёмника $S_1 = 0{,}8\units{см}^2$, а большого $S_2 = 40\units{см}^2.$ Какую силу $F$ надо приложить к малому поршню, чтобы поднять груз весом $P = 8\units{кН}?$

\task 6
Герметичный сосуд полностью заполнен водой и стоит на столе. На небольшой поршень площадью $S$ давят рукой с силой $F$. Поршень находится ниже крышки сосуда на $H_1$, выше дна на $H_2$ и может свободно перемещаться. Плотность воды $\rho$, атмосферное давление $P_A$. Найти давления $P_1$ и $P_2$ в воде вблизи крышки и дна сосуда.
\\ \\
\addclassdate{7}{20 апреля 2018}

\task 1
Площадь большого поршня гидравлического домкрата $S_1 = 20\units{см}^2$, а малого $S_2 = 0{,}5\units{см}^2.$ Груз какой максимальной массы можно поднять этим домкратом, если на малый поршень давить с силой не более $F=200\units{Н}?$ Силой трения от стенки цилиндров пренебречь.

\task 2
В сосуд налита вода. Расстояние от поверхности воды до дна $H = 0{,}5\units{м},$ площадь дна $S = 0{,}1\units{м}^2.$ Найти гидростатическое давление $P_1$ и полное давление $P_2$ вблизи дна. Найти силу давления воды на дно. Плотность воды \rhowater

\task 3
На лёгкий поршень площадью $S=900\units{см}^2,$ касающийся поверхности воды, поставили гирю массы $m=3\units{кг}$. Высота слоя воды в сосуде с вертикальными стенками $H = 20\units{см}$. Определить давление жидкости вблизи дна, если плотность воды \rhowater

\task 4
Давление газов в конце сгорания в цилиндре дизельного двигателя трактора $P = 9\units{МПа}.$ Диаметр цилиндра $d = 130\units{мм}.$ С какой силой газы давят на поршень в цилиндре? Площадь круга диаметром $D$ равна $S = \cfrac{\pi D^2}4.$

\task 5
Площадь малого поршня гидравлического подъёмника $S_1 = 0{,}8\units{см}^2$, а большого $S_2 = 40\units{см}^2.$ Какую силу $F$ надо приложить к малому поршню, чтобы поднять груз весом $P = 8\units{кН}?$

\task 6
Герметичный сосуд полностью заполнен водой и стоит на столе. На небольшой поршень площадью $S$ давят рукой с силой $F$. Поршень находится ниже крышки сосуда на $H_1$, выше дна на $H_2$ и может свободно перемещаться. Плотность воды $\rho$, атмосферное давление $P_A$. Найти давления $P_1$ и $P_2$ в воде вблизи крышки и дна сосуда.

\newpage

\adddate{8 класс. 20 апреля 2018}

\task 1
Между точками $A$ и $B$ электрической цепи подключены последовательно резисторы $R_1 = 10\units{Ом}$ и $R_2 = 20\units{Ом}$ и параллельно им $R_3 = 30\units{Ом}.$ Найдите эквивалентное сопротивление $R_{AB}$ этого участка цепи.

\task 2
Электрическая цепь состоит из последовательности $N$ одинаковых звеньев, в которых каждый резистор имеет сопротивление $r$. Последнее звено замкнуто резистором сопротивлением $R$. При каком соотношении $\cfrac{R}{r}$ сопротивление цепи не зависит от числа звеньев?

\task 3
Для измерения сопротивления $R$ проводника собрана электрическая цепь. Вольтметр $V$ показывает напряжение $U_V = 5\units{В},$ показание амперметра $A$ равно $I_A = 25\units{мА}.$ Найдите величину $R$ сопротивления проводника. Внутреннее сопротивление вольтметра $R_V = 1{,}0\units{кОм},$ внутреннее сопротивление амперметра $R_A = 2{,}0\units{Ом}.$

\task 4
Шкала гальванометра имеет $N=100$ делений, цена деления $\delta = 1\units{мкА}$. Внутреннее сопротивление гальванометра $R_G = 1{,}0\units{кОм}.$ Как из этого прибора сделать вольтметр для измерения напряжений до $U = 100\units{В}$ или амперметр для измерения токов силой до $I = 1\units{А}?$

\\ \\ \\ \\ \\ \\ \\ \\
\adddate{8 класс. 20 апреля 2018}

\task 1
Между точками $A$ и $B$ электрической цепи подключены последовательно резисторы $R_1 = 10\units{Ом}$ и $R_2 = 20\units{Ом}$ и параллельно им $R_3 = 30\units{Ом}.$ Найдите эквивалентное сопротивление $R_{AB}$ этого участка цепи.

\task 2
Электрическая цепь состоит из последовательности $N$ одинаковых звеньев, в которых каждый резистор имеет сопротивление $r$. Последнее звено замкнуто резистором сопротивлением $R$. При каком соотношении $\cfrac{R}{r}$ сопротивление цепи не зависит от числа звеньев?

\task 3
Для измерения сопротивления $R$ проводника собрана электрическая цепь. Вольтметр $V$ показывает напряжение $U_V = 5\units{В},$ показание амперметра $A$ равно $I_A = 25\units{мА}.$ Найдите величину $R$ сопротивления проводника. Внутреннее сопротивление вольтметра $R_V = 1{,}0\units{кОм},$ внутреннее сопротивление амперметра $R_A = 2{,}0\units{Ом}.$

\task 4
Шкала гальванометра имеет $N=100$ делений, цена деления $\delta = 1\units{мкА}$. Внутреннее сопротивление гальванометра $R_G = 1{,}0\units{кОм}.$ Как из этого прибора сделать вольтметр для измерения напряжений до $U = 100\units{В}$ или амперметр для измерения токов силой до $I = 1\units{А}?$


% % \begin{flushright}
\textsc{ГБОУ школа №554, 20 ноября 2018\,г.}
\end{flushright}

\begin{center}
\LARGE \textsc{Математический бой, 8 класс}
\end{center}

\problem{1} Есть тридцать карточек, на каждой написано по одному числу: на десяти карточках~–~$a$,  на десяти других~–~$b$ и на десяти оставшихся~–~$c$ (числа  различны). Известно, что к любым пяти карточкам можно подобрать ещё пять так, что сумма чисел на этих десяти карточках будет равна нулю. Докажите, что~одно из~чисел~$a, b, c$ равно нулю.

\problem{2} Вокруг стола стола пустили пакет с орешками. Первый взял один орешек, второй — 2, третий — 3 и так далее: каждый следующий брал на 1 орешек больше. Известно, что на втором круге было взято в сумме на 100 орешков больше, чем на первом. Сколько человек сидело за столом?

% \problem{2} Натуральное число разрешено увеличить на любое целое число процентов от 1 до 100, если при этом получаем натуральное число. Найдите наименьшее натуральное число, которое нельзя при помощи таких операций получить из~числа 1.

% \problem{3} Найти сумму $1^2 - 2^2 + 3^2 - 4^2 + 5^2 + \ldots - 2018^2$.

\problem{3} В кружке рукоделия, где занимается Валя, более 93\% участников~—~девочки. Какое наименьшее число участников может быть в таком кружке?

\problem{4} Произведение 2018 целых чисел равно 1. Может ли их сумма оказаться равной~0?

% \problem{4} Можно ли все натуральные числа от~1 до~9 записать в~клетки таблицы~$3\times3$ так, чтобы сумма в~любых двух соседних (по~вертикали или горизонтали) клетках равнялось простому числу?

\problem{5} На доске написано 2018 нулей и 2019 единиц. Женя стирает 2 числа и, если они были одинаковы, дописывает к оставшимся один ноль, а~если разные — единицу. Потом Женя повторяет эту операцию снова, потом ещё и~так далее. В~результате на~доске останется только одно число. Что это за~число?

\problem{6} Докажите, что в~любой компании людей найдутся 2~человека, имеющие равное число знакомых в этой компании (если $A$~знаком с~$B$, то~и $B$~знаком с~$A$).

\problem{7} Три колокола начинают бить одновременно. Интервалы между ударами колоколов соответственно составляют $\cfrac43$~секунды, $\cfrac53$~секунды и $2$~секунды. Совпавшие по времени удары воспринимаются за~один. Сколько ударов будет услышано за 1~минуту, включая первый и последний удары?

\problem{8} Восемь одинаковых момент расположены по кругу. Известно, что три из~них~— фальшивые, и они расположены рядом друг с~другом. Вес фальшивой монеты отличается от~веса настоящей. Все фальшивые монеты весят одинаково, но неизвестно, тяжелее или легче фальшивая монета настоящей. Покажите, что за~3~взвешивания на~чашечных весах без~гирь можно определить все фальшивые монеты.

% \end{document}

\begin{document}

\setdate{14~мая~2019}
\setclass{10«Т»}

\addpersonalvariant{Михаил Бурмистров}

\tasknumber{1}%
\task{%
    На резистор сопротивлением $r = 18\,\text{Ом}$ подали напряжение $V = 120\,\text{В}$.
    Определите ток, который потечёт через резистор, и мощность, выделяющуюся на нём.
}
\answer{%
    \begin{align*}
    \eli &= \frac{V}{r} = \frac{120\,\text{В}}{18\,\text{Ом}} = 6{,}67\,\text{А},  \\
    P &= \frac{V^2}{r} = \frac{\sqr{120\,\text{В}}}{18\,\text{Ом}} = 800\,\text{Вт}
    \end{align*}
}
\solutionspace{60pt}

\tasknumber{2}%
\task{%
    Через резистор сопротивлением $r = 12\,\text{Ом}$ протекает электрический ток $\eli = 10\,\text{А}$.
    Определите, чему равны напряжение на резисторе и мощность, выделяющаяся на нём.
}
\answer{%
    \begin{align*}
    U &= \eli r = 10\,\text{А} \cdot 12\,\text{Ом} = 120\,\text{В},  \\
    P &= \eli^2r = \sqr{10\,\text{А}} \cdot 12\,\text{Ом} = 1200\,\text{Вт}
    \end{align*}
}
\solutionspace{60pt}

\tasknumber{3}%
\task{%
    Замкнутая электрическая цепь состоит из ЭДС $\ele = 1\,\text{В}$ и сопротивлением $r$
    и резистора $R = 30\,\text{Ом}$.
    Определите ток, протекающий в цепи.
    Какая тепловая энергия выделится на резисторе за время
    $\tau = 10\,\text{с}$? Какая работа будет совершена ЭДС за это время? Каков знак этой работы? Чему равен КПД цепи?
    Вычислите значения для 2 случаев: $r=0$ и $r = 10\,\text{Ом}$.
}
\answer{%
    \begin{align*}
    \eli_1 &= \frac{\ele}{R} = \frac{1\,\text{В}}{30\,\text{Ом}} = \frac1{30}\units{А} \approx 0{,}03\,\text{А},  \\
    \eli_2 &= \frac{\ele}{R + r} = \frac{1\,\text{В}}{30\,\text{Ом} + 10\,\text{Ом}} = \frac1{40}\units{А} \approx 0{,}03\,\text{А},  \\
    Q_1 &= \eli_1^2R\tau = \sqr{\frac{\ele}{R}} R \tau
            = \sqr{\frac{1\,\text{В}}{30\,\text{Ом}}} \cdot 30\,\text{Ом} \cdot 10\,\text{с} = \frac13\units{Дж} \approx 0{,}333\,\text{Дж},  \\
    Q_2 &= \eli_2^2R\tau = \sqr{\frac{\ele}{R + r}} R \tau
            = \sqr{\frac{1\,\text{В}}{30\,\text{Ом} + 10\,\text{Ом}}} \cdot 30\,\text{Ом} \cdot 10\,\text{с} = \frac3{16}\units{Дж} \approx 0{,}188\,\text{Дж},  \\
    A_1 &= q_1\ele = \eli_1\tau\ele = \frac{\ele}{R} \tau \ele
            = \frac{\ele^2 \tau}{R} = \frac{\sqr{1\,\text{В}} \cdot 10\,\text{с}}{30\,\text{Ом}}
            = \frac13\units{Дж} \approx 0{,}333\,\text{Дж}, \text{положительна},  \\
    A_2 &= q_2\ele = \eli_2\tau\ele = \frac{\ele}{R + r} \tau \ele
            = \frac{\ele^2 \tau}{R + r} = \frac{\sqr{1\,\text{В}} \cdot 10\,\text{с}}{30\,\text{Ом} + 10\,\text{Ом}}
            = \frac14\units{Дж} \approx 0{,}250\,\text{Дж}, \text{положительна},  \\
    \eta_1 &= \frac{Q_1}{A_1} = \ldots = \frac{R}{R} = 1,  \\
    \eta_2 &= \frac{Q_2}{A_2} = \ldots = \frac{R}{R + r} = \frac34 \approx 0{,}75.
    \end{align*}
}
\solutionspace{180pt}

\tasknumber{4}%
\task{%
    Лампочки, сопротивления которых $R_1 = 3\,\text{Ом}$ и $R_2 = 48\,\text{Ом}$, поочерёдно подключённные к некоторому источнику тока,
    потребляют одинаковую мощность.
    Найти внутреннее сопротивление источника и КПД цепи в каждом случае.
}
\answer{%
    \begin{align*}
        P_1 &= \sqr{\frac{\ele}{R_1 + r}}R_1,
        P_2  = \sqr{\frac{\ele}{R_2 + r}}R_2,
        P_1 = P_2 \implies  \\
        &\implies R_1 \sqr{R_2 + r} = R_2 \sqr{R_1 + r} \implies  \\
        &\implies R_1 R_2^2 + 2 R_1 R_2 r + R_1 r^2 =
                    R_2 R_1^2 + 2 R_2 R_1 r + R_2 r^2  \implies  \\
    &\implies r^2 (R_2 - R_1) = R_2^2 R_2 - R_1^2 R_2 \implies  \\
    &\implies r
            = \sqrt{R_1 R_2 \frac{R_2 - R_1}{R_2 - R_1}}
            = \sqrt{R_1 R_2}
            = \sqrt{3\,\text{Ом} \cdot 48\,\text{Ом}}
            = 12\,\text{Ом}.
            \\
    \eta_1
            &= \frac{R_1}{R_1 + r}
            = \frac{\sqrt{R_1}}{\sqrt{R_1} + \sqrt{R_2}}
            = 0{,}200,  \\
    \eta_2
            &= \frac{R_2}{R_2 + r}
            = \frac{\sqrt{R_2}}{\sqrt{R_2} + \sqrt{R_1}}
            = 0{,}800
    \end{align*}
}
\solutionspace{120pt}

\tasknumber{5}%
\task{%
    Определите ток, протекающий через резистор $R = 10\,\text{Ом}$ и разность потенциалов на нём (см.
    рис.
    на доске),
    если $r_1 = 2\,\text{Ом}$, $r_2 = 1\,\text{Ом}$, $\ele_1 = 30\,\text{В}$, $\ele_2 = 20\,\text{В}$.
}
\answer{%
    Обозначим на рисунке все токи: направление произвольно, но его надо зафиксировать.
    Всего на рисунке 3 контура и 2 узла.
    Поэтому можно записать $3 - 1 = 2$ уравнения законов Кирхгофа для замкнутого контура и $2 - 1 = 1$ — для узлов
    (остальные уравнения тоже можно записать, но они не дадут полезной информации, а будут лишь следствиями уже записанных).

    Отметим на рисунке 2 контура (и не забуем указать направление) и 1 узел (точка «1»ы, выделена жирным).
    Выбор контуров и узлов не критичен: получившаяся система может быть чуть проще или сложнее, но не слишком.

    \begin{tikzpicture}[circuit ee IEC, thick]
        \draw  (0, 0) to [current direction={near end, info=$\eli_1$}] (0, 3)
                to [battery={rotate=-180,info={$\ele_1, r_1$}}]
                (3, 3)
                to [battery={info'={$\ele_2, r_2$}}]
                (6, 3) to [current direction'={near start, info=$\eli_2$}] (6, 0) -- (0, 0)
                (3, 0) to [current direction={near start, info=$\eli$}, resistor={near end, info=$R$}] (3, 3);
        \draw [-{Latex},color=red] (1.2, 1.7) arc [start angle = 135, end angle = -160, radius = 0.6];
        \draw [-{Latex},color=blue] (4.2, 1.7) arc [start angle = 135, end angle = -160, radius = 0.6];
        \node [contact,color=green!71!black] (bottomc) at (3, 0) {};
        \node [below] (bottom) at (3, 0) {$2$};
        \node [above] (top) at (3, 3) {$1$};
    \end{tikzpicture}

    \begin{align*}
        &\begin{cases}
            {\color{red} \ele_1 = \eli_1 r_1 - \eli R}, \\
            {\color{blue} -\ele_2 = -\eli_2 r_2 + \eli R}, \\
            {\color{green!71!black} - \eli - \eli_1 - \eli_2 = 0 };
        \end{cases}
        \qquad \implies \qquad
        \begin{cases}
            \eli_1 = \frac{\ele_1 + \eli R}{r_1}, \\
            \eli_2 = \frac{\ele_2 + \eli R}{r_2}, \\
            \eli + \eli_1 + \eli_2 = 0;
        \end{cases} \implies \\
        &\implies
         \eli + \frac{\ele_1 + \eli R}{r_1:L} + \frac{\ele_2 + \eli R}{r_2:L} = 0, \\
        &\eli\cbr{ 1 + \frac R{r_1:L} + \frac R{r_2:L}} + \frac{\ele_1 }{r_1:L} + \frac{\ele_2 }{r_2:L} = 0, \\
        &\eli
            = - \frac{\frac{\ele_1 }{r_1:L} + \frac{\ele_2 }{r_2:L}}{ 1 + \frac R{r_1:L} + \frac R{r_2:L}}
            = - \frac{\frac{30\,\text{В}}{2\,\text{Ом}} + \frac{20\,\text{В}}{1\,\text{Ом}}}{ 1 + \frac{10\,\text{Ом}}{2\,\text{Ом}} + \frac{10\,\text{Ом}}{1\,\text{Ом}}}
            = - \frac{35}{16}\units{А}
            \approx -2{,}2\,\text{А}, \\
        &U  = \varphi_2 - \varphi_1 = \eli R
            = - \frac{\frac{\ele_1 }{r_1:L} + \frac{\ele_2 }{r_2:L}}{ 1 + \frac R{r_1:L} + \frac R{r_2:L}} R
            \approx -21{,}9\,\text{В}.
    \end{align*}
    Оба ответа отрицательны, потому что мы изначально «не угадали» с направлением тока.
    Расчёт же показал,
    что ток через резистор $R$ течёт в противоположную сторону: вниз на рисунке, а потенциал точки 1 больше потенциала точки 2,
    а электрический ток ожидаемо течёт из точки с большим потенциалов в точку с меньшим.

    Кстати, если продолжить расчёт и вычислить значения ещё двух токов (формулы для $\eli_1$ и $\eli_2$, куда подставлять, выписаны выше),
    то по их знакам можно будет понять: угадали ли мы с их направлением или нет.
}

\end{document}
% autogenerated
