\setdate{14~мая~2019}
\setclass{10«Т»}

\addpersonalvariant{Михаил Бурмистров}

\tasknumber{1}%
\task{%
    На резистор сопротивлением $r = 18\,\text{Ом}$ подали напряжение $V = 120\,\text{В}$.
    Определите ток, который потечёт через резистор, и мощность, выделяющуюся на нём.
}
\answer{%
    \begin{align*}
    \eli &= \frac{V}{r} = \frac{120\,\text{В}}{18\,\text{Ом}} = 6{,}67\,\text{А},  \\
    P &= \frac{V^2}{r} = \frac{\sqr{120\,\text{В}}}{18\,\text{Ом}} = 800{,}00\,\text{Вт}
    \end{align*}
}
\solutionspace{60pt}

\tasknumber{2}%
\task{%
    Через резистор сопротивлением $r = 12\,\text{Ом}$ протекает электрический ток $\eli = 10{,}00\,\text{А}$.
    Определите, чему равны напряжение на резисторе и мощность, выделяющаяся на нём.
}
\answer{%
    \begin{align*}
    U &= \eli r = 10{,}00\,\text{А} \cdot 12\,\text{Ом} = 120\,\text{В},  \\
    P &= \eli^2r = \sqr{10{,}00\,\text{А}} \cdot 12\,\text{Ом} = 1200\,\text{Вт}
    \end{align*}
}
\solutionspace{60pt}

\tasknumber{3}%
\task{%
    Замкнутая электрическая цепь состоит из ЭДС $\ele = 1\,\text{В}$ и сопротивлением $r$
    и резистора $R = 30\,\text{Ом}$.
    Определите ток, протекающий в цепи.
    Какая тепловая энергия выделится на резисторе за время
    $\tau = 10\,\text{с}$? Какая работа будет совершена ЭДС за это время? Каков знак этой работы? Чему равен КПД цепи?
    Вычислите значения для 2 случаев: $r=0$ и $r = 10\,\text{Ом}$.
}
\answer{%
    \begin{align*}
    \eli_1 &= \frac{\ele}{R} = \frac{1\,\text{В}}{30\,\text{Ом}} = \frac1{30}\units{А} \approx 0{,}03\,\text{А},  \\
    \eli_2 &= \frac{\ele}{R + r} = \frac{1\,\text{В}}{30\,\text{Ом} + 10\,\text{Ом}} = \frac1{40}\units{А} \approx 0{,}03\,\text{А},  \\
    Q_1 &= \eli_1^2R\tau = \sqr{\frac{\ele}{R}} R \tau
            = \sqr{\frac{1\,\text{В}}{30\,\text{Ом}}} \cdot 30\,\text{Ом} \cdot 10\,\text{с} = \frac13\units{Дж} \approx 0{,}333\,\text{Дж},  \\
    Q_2 &= \eli_2^2R\tau = \sqr{\frac{\ele}{R + r}} R \tau
            = \sqr{\frac{1\,\text{В}}{30\,\text{Ом} + 10\,\text{Ом}}} \cdot 30\,\text{Ом} \cdot 10\,\text{с} = \frac3{16}\units{Дж} \approx 0{,}188\,\text{Дж},  \\
    A_1 &= q_1\ele = \eli_1\tau\ele = \frac{\ele}{R} \tau \ele
            = \frac{\ele^2 \tau}{R} = \frac{\sqr{1\,\text{В}} \cdot 10\,\text{с}}{30\,\text{Ом}}
            = \frac13\units{Дж} \approx 0{,}333\,\text{Дж}, \text{положительна},  \\
    A_2 &= q_2\ele = \eli_2\tau\ele = \frac{\ele}{R + r} \tau \ele
            = \frac{\ele^2 \tau}{R + r} = \frac{\sqr{1\,\text{В}} \cdot 10\,\text{с}}{30\,\text{Ом} + 10\,\text{Ом}}
            = \frac14\units{Дж} \approx 0{,}250\,\text{Дж}, \text{положительна},  \\
    \eta_1 &= \frac{Q_1}{A_1} = \ldots = \frac{R}{R} = 1,  \\
    \eta_2 &= \frac{Q_2}{A_2} = \ldots = \frac{R}{R + r} = \frac34 \approx 0{,}75.
    \end{align*}
}
\solutionspace{180pt}

\tasknumber{4}%
\task{%
    Лампочки, сопротивления которых $R_1 = 3{,}00\,\text{Ом}$ и $R_2 = 48{,}00\,\text{Ом}$, поочерёдно подключённные к некоторому источнику тока,
    потребляют одинаковую мощность.
    Найти внутреннее сопротивление источника и КПД цепи в каждом случае.
}
\answer{%
    \begin{align*}
        P_1 &= \sqr{\frac{\ele}{R_1 + r}}R_1,
        P_2  = \sqr{\frac{\ele}{R_2 + r}}R_2,
        P_1 = P_2 \implies  \\
        &\implies R_1 \sqr{R_2 + r} = R_2 \sqr{R_1 + r} \implies  \\
        &\implies R_1 R_2^2 + 2 R_1 R_2 r + R_1 r^2 =
                    R_2 R_1^2 + 2 R_2 R_1 r + R_2 r^2  \implies  \\
    &\implies r^2 (R_2 - R_1) = R_2^2 R_2 - R_1^2 R_2 \implies  \\
    &\implies r
            = \sqrt{R_1 R_2 \frac{R_2 - R_1}{R_2 - R_1}}
            = \sqrt{R_1 R_2}
            = \sqrt{3{,}00\,\text{Ом} \cdot 48{,}00\,\text{Ом}}
            = 12{,}0\,\text{Ом}.
            \\
    \eta_1
            &= \frac{R_1}{R_1 + r}
            = \frac{\sqrt{R_1}}{\sqrt{R_1} + \sqrt{R_2}}
            = 0{,}200,  \\
    \eta_2
            &= \frac{R_2}{R_2 + r}
            = \frac{\sqrt{R_2}}{\sqrt{R_2} + \sqrt{R_1}}
            = 0{,}800
    \end{align*}
}
\solutionspace{120pt}

\tasknumber{5}%
\task{%
    Определите ток, протекающий через резистор $R = 10\,\text{Ом}$ и разность потенциалов на нём (см.
    рис.
    на доске),
    если $r_1 = 2\,\text{Ом}$, $r_2 = 1\,\text{Ом}$, $\ele_1 = 30\,\text{В}$, $\ele_2 = 20\,\text{В}$.
}
\answer{%
    Обозначим на рисунке все токи: направление произвольно, но его надо зафиксировать.
    Всего на рисунке 3 контура и 2 узла.
    Поэтому можно записать $3 - 1 = 2$ уравнения законов Кирхгофа для замкнутого контура и $2 - 1 = 1$ — для узлов
    (остальные уравнения тоже можно записать, но они не дадут полезной информации, а будут лишь следствиями уже записанных).

    Отметим на рисунке 2 контура (и не забуем указать направление) и 1 узел (точка «1»ы, выделена жирным).
    Выбор контуров и узлов не критичен: получившаяся система может быть чуть проще или сложнее, но не слишком.

    \begin{tikzpicture}[circuit ee IEC, thick]
        \draw  (0, 0) to [current direction={near end, info=$\eli_1$}] (0, 3)
                to [battery={rotate=-180,info={$\ele_1, r_1$}}]
                (3, 3)
                to [battery={info'={$\ele_2, r_2$}}]
                (6, 3) to [current direction'={near start, info=$\eli_2$}] (6, 0) -- (0, 0)
                (3, 0) to [current direction={near start, info=$\eli$}, resistor={near end, info=$R$}] (3, 3);
        \draw [-{Latex},color=red] (1.2, 1.7) arc [start angle = 135, end angle = -160, radius = 0.6];
        \draw [-{Latex},color=blue] (4.2, 1.7) arc [start angle = 135, end angle = -160, radius = 0.6];
        \node [contact,color=green!71!black] (bottomc) at (3, 0) {};
        \node [below] (bottom) at (3, 0) {$2$};
        \node [above] (top) at (3, 3) {$1$};
    \end{tikzpicture}

    \begin{align*}
        &\begin{cases}
            {\color{red} \ele_1 = \eli_1 r_1 - \eli R}, \\
            {\color{blue} -\ele_2 = -\eli_2 r_2 + \eli R}, \\
            {\color{green!71!black} - \eli - \eli_1 - \eli_2 = 0 };
        \end{cases}
        \qquad \implies \qquad
        \begin{cases}
            \eli_1 = \frac{\ele_1 + \eli R}{r_1}, \\
            \eli_2 = \frac{\ele_2 + \eli R}{r_2}, \\
            \eli + \eli_1 + \eli_2 = 0;
        \end{cases} \implies \\
        &\implies
         \eli + \frac{\ele_1 + \eli R}{r_1:L} + \frac{\ele_2 + \eli R}{r_2:L} = 0, \\
        &\eli\cbr{ 1 + \frac R{r_1:L} + \frac R{r_2:L}} + \frac{\ele_1 }{r_1:L} + \frac{\ele_2 }{r_2:L} = 0, \\
        &\eli
            = - \frac{\frac{\ele_1 }{r_1:L} + \frac{\ele_2 }{r_2:L}}{ 1 + \frac R{r_1:L} + \frac R{r_2:L}}
            = - \frac{\frac{30\,\text{В}}{2\,\text{Ом}} + \frac{20\,\text{В}}{1\,\text{Ом}}}{ 1 + \frac{10\,\text{Ом}}{2\,\text{Ом}} + \frac{10\,\text{Ом}}{1\,\text{Ом}}}
            = - \frac{35}{16}\units{А}
            \approx -2{,}20\,\text{А}, \\
        &U  = \varphi_2 - \varphi_1 = \eli R
            = - \frac{\frac{\ele_1 }{r_1:L} + \frac{\ele_2 }{r_2:L}}{ 1 + \frac R{r_1:L} + \frac R{r_2:L}} R
            \approx -21{,}90\,\text{В}.
    \end{align*}
    Оба ответа отрицательны, потому что мы изначально «не угадали» с направлением тока.
    Расчёт же показал,
    что ток через резистор $R$ течёт в противоположную сторону: вниз на рисунке, а потенциал точки 1 больше потенциала точки 2,
    а электрический ток ожидаемо течёт из точки с большим потенциалов в точку с меньшим.

    Кстати, если продолжить расчёт и вычислить значения ещё двух токов (формулы для $\eli_1$ и $\eli_2$, куда подставлять, выписаны выше),
    то по их знакам можно будет понять: угадали ли мы с их направлением или нет.
}

\variantsplitter

\addpersonalvariant{Гагик Аракелян}

\tasknumber{1}%
\task{%
    На резистор сопротивлением $R = 30\,\text{Ом}$ подали напряжение $V = 240\,\text{В}$.
    Определите ток, который потечёт через резистор, и мощность, выделяющуюся на нём.
}
\answer{%
    \begin{align*}
    \eli &= \frac{V}{R} = \frac{240\,\text{В}}{30\,\text{Ом}} = 8{,}00\,\text{А},  \\
    P &= \frac{V^2}{R} = \frac{\sqr{240\,\text{В}}}{30\,\text{Ом}} = 1920{,}00\,\text{Вт}
    \end{align*}
}
\solutionspace{60pt}

\tasknumber{2}%
\task{%
    Через резистор сопротивлением $r = 18\,\text{Ом}$ протекает электрический ток $\eli = 5{,}00\,\text{А}$.
    Определите, чему равны напряжение на резисторе и мощность, выделяющаяся на нём.
}
\answer{%
    \begin{align*}
    U &= \eli r = 5{,}00\,\text{А} \cdot 18\,\text{Ом} = 90\,\text{В},  \\
    P &= \eli^2r = \sqr{5{,}00\,\text{А}} \cdot 18\,\text{Ом} = 450\,\text{Вт}
    \end{align*}
}
\solutionspace{60pt}

\tasknumber{3}%
\task{%
    Замкнутая электрическая цепь состоит из ЭДС $\ele = 2\,\text{В}$ и сопротивлением $r$
    и резистора $R = 15\,\text{Ом}$.
    Определите ток, протекающий в цепи.
    Какая тепловая энергия выделится на резисторе за время
    $\tau = 5\,\text{с}$? Какая работа будет совершена ЭДС за это время? Каков знак этой работы? Чему равен КПД цепи?
    Вычислите значения для 2 случаев: $r=0$ и $r = 20\,\text{Ом}$.
}
\answer{%
    \begin{align*}
    \eli_1 &= \frac{\ele}{R} = \frac{2\,\text{В}}{15\,\text{Ом}} = \frac2{15}\units{А} \approx 0{,}13\,\text{А},  \\
    \eli_2 &= \frac{\ele}{R + r} = \frac{2\,\text{В}}{15\,\text{Ом} + 20\,\text{Ом}} = \frac2{35}\units{А} \approx 0{,}06\,\text{А},  \\
    Q_1 &= \eli_1^2R\tau = \sqr{\frac{\ele}{R}} R \tau
            = \sqr{\frac{2\,\text{В}}{15\,\text{Ом}}} \cdot 15\,\text{Ом} \cdot 5\,\text{с} = \frac43\units{Дж} \approx 1{,}333\,\text{Дж},  \\
    Q_2 &= \eli_2^2R\tau = \sqr{\frac{\ele}{R + r}} R \tau
            = \sqr{\frac{2\,\text{В}}{15\,\text{Ом} + 20\,\text{Ом}}} \cdot 15\,\text{Ом} \cdot 5\,\text{с} = \frac{12}{49}\units{Дж} \approx 0{,}245\,\text{Дж},  \\
    A_1 &= q_1\ele = \eli_1\tau\ele = \frac{\ele}{R} \tau \ele
            = \frac{\ele^2 \tau}{R} = \frac{\sqr{2\,\text{В}} \cdot 5\,\text{с}}{15\,\text{Ом}}
            = \frac43\units{Дж} \approx 1{,}333\,\text{Дж}, \text{положительна},  \\
    A_2 &= q_2\ele = \eli_2\tau\ele = \frac{\ele}{R + r} \tau \ele
            = \frac{\ele^2 \tau}{R + r} = \frac{\sqr{2\,\text{В}} \cdot 5\,\text{с}}{15\,\text{Ом} + 20\,\text{Ом}}
            = \frac47\units{Дж} \approx 0{,}571\,\text{Дж}, \text{положительна},  \\
    \eta_1 &= \frac{Q_1}{A_1} = \ldots = \frac{R}{R} = 1,  \\
    \eta_2 &= \frac{Q_2}{A_2} = \ldots = \frac{R}{R + r} = \frac37 \approx 0{,}43.
    \end{align*}
}
\solutionspace{180pt}

\tasknumber{4}%
\task{%
    Лампочки, сопротивления которых $R_1 = 1{,}00\,\text{Ом}$ и $R_2 = 4{,}00\,\text{Ом}$, поочерёдно подключённные к некоторому источнику тока,
    потребляют одинаковую мощность.
    Найти внутреннее сопротивление источника и КПД цепи в каждом случае.
}
\answer{%
    \begin{align*}
        P_1 &= \sqr{\frac{\ele}{R_1 + r}}R_1,
        P_2  = \sqr{\frac{\ele}{R_2 + r}}R_2,
        P_1 = P_2 \implies  \\
        &\implies R_1 \sqr{R_2 + r} = R_2 \sqr{R_1 + r} \implies  \\
        &\implies R_1 R_2^2 + 2 R_1 R_2 r + R_1 r^2 =
                    R_2 R_1^2 + 2 R_2 R_1 r + R_2 r^2  \implies  \\
    &\implies r^2 (R_2 - R_1) = R_2^2 R_2 - R_1^2 R_2 \implies  \\
    &\implies r
            = \sqrt{R_1 R_2 \frac{R_2 - R_1}{R_2 - R_1}}
            = \sqrt{R_1 R_2}
            = \sqrt{1{,}00\,\text{Ом} \cdot 4{,}00\,\text{Ом}}
            = 2{,}0\,\text{Ом}.
            \\
    \eta_1
            &= \frac{R_1}{R_1 + r}
            = \frac{\sqrt{R_1}}{\sqrt{R_1} + \sqrt{R_2}}
            = 0{,}333,  \\
    \eta_2
            &= \frac{R_2}{R_2 + r}
            = \frac{\sqrt{R_2}}{\sqrt{R_2} + \sqrt{R_1}}
            = 0{,}667
    \end{align*}
}
\solutionspace{120pt}

\tasknumber{5}%
\task{%
    Определите ток, протекающий через резистор $R = 15\,\text{Ом}$ и разность потенциалов на нём (см.
    рис.
    на доске),
    если $r_1 = 1\,\text{Ом}$, $r_2 = 3\,\text{Ом}$, $\ele_1 = 40\,\text{В}$, $\ele_2 = 60\,\text{В}$.
}
\answer{%
    Обозначим на рисунке все токи: направление произвольно, но его надо зафиксировать.
    Всего на рисунке 3 контура и 2 узла.
    Поэтому можно записать $3 - 1 = 2$ уравнения законов Кирхгофа для замкнутого контура и $2 - 1 = 1$ — для узлов
    (остальные уравнения тоже можно записать, но они не дадут полезной информации, а будут лишь следствиями уже записанных).

    Отметим на рисунке 2 контура (и не забуем указать направление) и 1 узел (точка «1»ы, выделена жирным).
    Выбор контуров и узлов не критичен: получившаяся система может быть чуть проще или сложнее, но не слишком.

    \begin{tikzpicture}[circuit ee IEC, thick]
        \draw  (0, 0) to [current direction={near end, info=$\eli_1$}] (0, 3)
                to [battery={rotate=-180,info={$\ele_1, r_1$}}]
                (3, 3)
                to [battery={info'={$\ele_2, r_2$}}]
                (6, 3) to [current direction'={near start, info=$\eli_2$}] (6, 0) -- (0, 0)
                (3, 0) to [current direction={near start, info=$\eli$}, resistor={near end, info=$R$}] (3, 3);
        \draw [-{Latex},color=red] (1.2, 1.7) arc [start angle = 135, end angle = -160, radius = 0.6];
        \draw [-{Latex},color=blue] (4.2, 1.7) arc [start angle = 135, end angle = -160, radius = 0.6];
        \node [contact,color=green!71!black] (bottomc) at (3, 0) {};
        \node [below] (bottom) at (3, 0) {$2$};
        \node [above] (top) at (3, 3) {$1$};
    \end{tikzpicture}

    \begin{align*}
        &\begin{cases}
            {\color{red} \ele_1 = \eli_1 r_1 - \eli R}, \\
            {\color{blue} -\ele_2 = -\eli_2 r_2 + \eli R}, \\
            {\color{green!71!black} - \eli - \eli_1 - \eli_2 = 0 };
        \end{cases}
        \qquad \implies \qquad
        \begin{cases}
            \eli_1 = \frac{\ele_1 + \eli R}{r_1}, \\
            \eli_2 = \frac{\ele_2 + \eli R}{r_2}, \\
            \eli + \eli_1 + \eli_2 = 0;
        \end{cases} \implies \\
        &\implies
         \eli + \frac{\ele_1 + \eli R}{r_1:L} + \frac{\ele_2 + \eli R}{r_2:L} = 0, \\
        &\eli\cbr{ 1 + \frac R{r_1:L} + \frac R{r_2:L}} + \frac{\ele_1 }{r_1:L} + \frac{\ele_2 }{r_2:L} = 0, \\
        &\eli
            = - \frac{\frac{\ele_1 }{r_1:L} + \frac{\ele_2 }{r_2:L}}{ 1 + \frac R{r_1:L} + \frac R{r_2:L}}
            = - \frac{\frac{40\,\text{В}}{1\,\text{Ом}} + \frac{60\,\text{В}}{3\,\text{Ом}}}{ 1 + \frac{15\,\text{Ом}}{1\,\text{Ом}} + \frac{15\,\text{Ом}}{3\,\text{Ом}}}
            = - \frac{20}7\units{А}
            \approx -2{,}90\,\text{А}, \\
        &U  = \varphi_2 - \varphi_1 = \eli R
            = - \frac{\frac{\ele_1 }{r_1:L} + \frac{\ele_2 }{r_2:L}}{ 1 + \frac R{r_1:L} + \frac R{r_2:L}} R
            \approx -42{,}90\,\text{В}.
    \end{align*}
    Оба ответа отрицательны, потому что мы изначально «не угадали» с направлением тока.
    Расчёт же показал,
    что ток через резистор $R$ течёт в противоположную сторону: вниз на рисунке, а потенциал точки 1 больше потенциала точки 2,
    а электрический ток ожидаемо течёт из точки с большим потенциалов в точку с меньшим.

    Кстати, если продолжить расчёт и вычислить значения ещё двух токов (формулы для $\eli_1$ и $\eli_2$, куда подставлять, выписаны выше),
    то по их знакам можно будет понять: угадали ли мы с их направлением или нет.
}

\variantsplitter

\addpersonalvariant{Ирен Аракелян}

\tasknumber{1}%
\task{%
    На резистор сопротивлением $r = 30\,\text{Ом}$ подали напряжение $V = 240\,\text{В}$.
    Определите ток, который потечёт через резистор, и мощность, выделяющуюся на нём.
}
\answer{%
    \begin{align*}
    \eli &= \frac{V}{r} = \frac{240\,\text{В}}{30\,\text{Ом}} = 8{,}00\,\text{А},  \\
    P &= \frac{V^2}{r} = \frac{\sqr{240\,\text{В}}}{30\,\text{Ом}} = 1920{,}00\,\text{Вт}
    \end{align*}
}
\solutionspace{60pt}

\tasknumber{2}%
\task{%
    Через резистор сопротивлением $r = 12\,\text{Ом}$ протекает электрический ток $\eli = 8{,}00\,\text{А}$.
    Определите, чему равны напряжение на резисторе и мощность, выделяющаяся на нём.
}
\answer{%
    \begin{align*}
    U &= \eli r = 8{,}00\,\text{А} \cdot 12\,\text{Ом} = 96\,\text{В},  \\
    P &= \eli^2r = \sqr{8{,}00\,\text{А}} \cdot 12\,\text{Ом} = 768\,\text{Вт}
    \end{align*}
}
\solutionspace{60pt}

\tasknumber{3}%
\task{%
    Замкнутая электрическая цепь состоит из ЭДС $\ele = 2\,\text{В}$ и сопротивлением $r$
    и резистора $R = 15\,\text{Ом}$.
    Определите ток, протекающий в цепи.
    Какая тепловая энергия выделится на резисторе за время
    $\tau = 2\,\text{с}$? Какая работа будет совершена ЭДС за это время? Каков знак этой работы? Чему равен КПД цепи?
    Вычислите значения для 2 случаев: $r=0$ и $r = 30\,\text{Ом}$.
}
\answer{%
    \begin{align*}
    \eli_1 &= \frac{\ele}{R} = \frac{2\,\text{В}}{15\,\text{Ом}} = \frac2{15}\units{А} \approx 0{,}13\,\text{А},  \\
    \eli_2 &= \frac{\ele}{R + r} = \frac{2\,\text{В}}{15\,\text{Ом} + 30\,\text{Ом}} = \frac2{45}\units{А} \approx 0{,}04\,\text{А},  \\
    Q_1 &= \eli_1^2R\tau = \sqr{\frac{\ele}{R}} R \tau
            = \sqr{\frac{2\,\text{В}}{15\,\text{Ом}}} \cdot 15\,\text{Ом} \cdot 2\,\text{с} = \frac8{15}\units{Дж} \approx 0{,}533\,\text{Дж},  \\
    Q_2 &= \eli_2^2R\tau = \sqr{\frac{\ele}{R + r}} R \tau
            = \sqr{\frac{2\,\text{В}}{15\,\text{Ом} + 30\,\text{Ом}}} \cdot 15\,\text{Ом} \cdot 2\,\text{с} = \frac8{135}\units{Дж} \approx 0{,}059\,\text{Дж},  \\
    A_1 &= q_1\ele = \eli_1\tau\ele = \frac{\ele}{R} \tau \ele
            = \frac{\ele^2 \tau}{R} = \frac{\sqr{2\,\text{В}} \cdot 2\,\text{с}}{15\,\text{Ом}}
            = \frac8{15}\units{Дж} \approx 0{,}533\,\text{Дж}, \text{положительна},  \\
    A_2 &= q_2\ele = \eli_2\tau\ele = \frac{\ele}{R + r} \tau \ele
            = \frac{\ele^2 \tau}{R + r} = \frac{\sqr{2\,\text{В}} \cdot 2\,\text{с}}{15\,\text{Ом} + 30\,\text{Ом}}
            = \frac8{45}\units{Дж} \approx 0{,}178\,\text{Дж}, \text{положительна},  \\
    \eta_1 &= \frac{Q_1}{A_1} = \ldots = \frac{R}{R} = 1,  \\
    \eta_2 &= \frac{Q_2}{A_2} = \ldots = \frac{R}{R + r} = \frac13 \approx 0{,}33.
    \end{align*}
}
\solutionspace{180pt}

\tasknumber{4}%
\task{%
    Лампочки, сопротивления которых $R_1 = 4{,}00\,\text{Ом}$ и $R_2 = 36{,}00\,\text{Ом}$, поочерёдно подключённные к некоторому источнику тока,
    потребляют одинаковую мощность.
    Найти внутреннее сопротивление источника и КПД цепи в каждом случае.
}
\answer{%
    \begin{align*}
        P_1 &= \sqr{\frac{\ele}{R_1 + r}}R_1,
        P_2  = \sqr{\frac{\ele}{R_2 + r}}R_2,
        P_1 = P_2 \implies  \\
        &\implies R_1 \sqr{R_2 + r} = R_2 \sqr{R_1 + r} \implies  \\
        &\implies R_1 R_2^2 + 2 R_1 R_2 r + R_1 r^2 =
                    R_2 R_1^2 + 2 R_2 R_1 r + R_2 r^2  \implies  \\
    &\implies r^2 (R_2 - R_1) = R_2^2 R_2 - R_1^2 R_2 \implies  \\
    &\implies r
            = \sqrt{R_1 R_2 \frac{R_2 - R_1}{R_2 - R_1}}
            = \sqrt{R_1 R_2}
            = \sqrt{4{,}00\,\text{Ом} \cdot 36{,}00\,\text{Ом}}
            = 12{,}0\,\text{Ом}.
            \\
    \eta_1
            &= \frac{R_1}{R_1 + r}
            = \frac{\sqrt{R_1}}{\sqrt{R_1} + \sqrt{R_2}}
            = 0{,}250,  \\
    \eta_2
            &= \frac{R_2}{R_2 + r}
            = \frac{\sqrt{R_2}}{\sqrt{R_2} + \sqrt{R_1}}
            = 0{,}750
    \end{align*}
}
\solutionspace{120pt}

\tasknumber{5}%
\task{%
    Определите ток, протекающий через резистор $R = 10\,\text{Ом}$ и разность потенциалов на нём (см.
    рис.
    на доске),
    если $r_1 = 2\,\text{Ом}$, $r_2 = 2\,\text{Ом}$, $\ele_1 = 20\,\text{В}$, $\ele_2 = 20\,\text{В}$.
}
\answer{%
    Обозначим на рисунке все токи: направление произвольно, но его надо зафиксировать.
    Всего на рисунке 3 контура и 2 узла.
    Поэтому можно записать $3 - 1 = 2$ уравнения законов Кирхгофа для замкнутого контура и $2 - 1 = 1$ — для узлов
    (остальные уравнения тоже можно записать, но они не дадут полезной информации, а будут лишь следствиями уже записанных).

    Отметим на рисунке 2 контура (и не забуем указать направление) и 1 узел (точка «1»ы, выделена жирным).
    Выбор контуров и узлов не критичен: получившаяся система может быть чуть проще или сложнее, но не слишком.

    \begin{tikzpicture}[circuit ee IEC, thick]
        \draw  (0, 0) to [current direction={near end, info=$\eli_1$}] (0, 3)
                to [battery={rotate=-180,info={$\ele_1, r_1$}}]
                (3, 3)
                to [battery={info'={$\ele_2, r_2$}}]
                (6, 3) to [current direction'={near start, info=$\eli_2$}] (6, 0) -- (0, 0)
                (3, 0) to [current direction={near start, info=$\eli$}, resistor={near end, info=$R$}] (3, 3);
        \draw [-{Latex},color=red] (1.2, 1.7) arc [start angle = 135, end angle = -160, radius = 0.6];
        \draw [-{Latex},color=blue] (4.2, 1.7) arc [start angle = 135, end angle = -160, radius = 0.6];
        \node [contact,color=green!71!black] (bottomc) at (3, 0) {};
        \node [below] (bottom) at (3, 0) {$2$};
        \node [above] (top) at (3, 3) {$1$};
    \end{tikzpicture}

    \begin{align*}
        &\begin{cases}
            {\color{red} \ele_1 = \eli_1 r_1 - \eli R}, \\
            {\color{blue} -\ele_2 = -\eli_2 r_2 + \eli R}, \\
            {\color{green!71!black} - \eli - \eli_1 - \eli_2 = 0 };
        \end{cases}
        \qquad \implies \qquad
        \begin{cases}
            \eli_1 = \frac{\ele_1 + \eli R}{r_1}, \\
            \eli_2 = \frac{\ele_2 + \eli R}{r_2}, \\
            \eli + \eli_1 + \eli_2 = 0;
        \end{cases} \implies \\
        &\implies
         \eli + \frac{\ele_1 + \eli R}{r_1:L} + \frac{\ele_2 + \eli R}{r_2:L} = 0, \\
        &\eli\cbr{ 1 + \frac R{r_1:L} + \frac R{r_2:L}} + \frac{\ele_1 }{r_1:L} + \frac{\ele_2 }{r_2:L} = 0, \\
        &\eli
            = - \frac{\frac{\ele_1 }{r_1:L} + \frac{\ele_2 }{r_2:L}}{ 1 + \frac R{r_1:L} + \frac R{r_2:L}}
            = - \frac{\frac{20\,\text{В}}{2\,\text{Ом}} + \frac{20\,\text{В}}{2\,\text{Ом}}}{ 1 + \frac{10\,\text{Ом}}{2\,\text{Ом}} + \frac{10\,\text{Ом}}{2\,\text{Ом}}}
            = - \frac{20}{11}\units{А}
            \approx -1{,}800\,\text{А}, \\
        &U  = \varphi_2 - \varphi_1 = \eli R
            = - \frac{\frac{\ele_1 }{r_1:L} + \frac{\ele_2 }{r_2:L}}{ 1 + \frac R{r_1:L} + \frac R{r_2:L}} R
            \approx -18{,}200\,\text{В}.
    \end{align*}
    Оба ответа отрицательны, потому что мы изначально «не угадали» с направлением тока.
    Расчёт же показал,
    что ток через резистор $R$ течёт в противоположную сторону: вниз на рисунке, а потенциал точки 1 больше потенциала точки 2,
    а электрический ток ожидаемо течёт из точки с большим потенциалов в точку с меньшим.

    Кстати, если продолжить расчёт и вычислить значения ещё двух токов (формулы для $\eli_1$ и $\eli_2$, куда подставлять, выписаны выше),
    то по их знакам можно будет понять: угадали ли мы с их направлением или нет.
}

\variantsplitter

\addpersonalvariant{Сабина Асадуллаева}

\tasknumber{1}%
\task{%
    На резистор сопротивлением $r = 18\,\text{Ом}$ подали напряжение $V = 150\,\text{В}$.
    Определите ток, который потечёт через резистор, и мощность, выделяющуюся на нём.
}
\answer{%
    \begin{align*}
    \eli &= \frac{V}{r} = \frac{150\,\text{В}}{18\,\text{Ом}} = 8{,}33\,\text{А},  \\
    P &= \frac{V^2}{r} = \frac{\sqr{150\,\text{В}}}{18\,\text{Ом}} = 1250{,}00\,\text{Вт}
    \end{align*}
}
\solutionspace{60pt}

\tasknumber{2}%
\task{%
    Через резистор сопротивлением $r = 30\,\text{Ом}$ протекает электрический ток $\eli = 15{,}00\,\text{А}$.
    Определите, чему равны напряжение на резисторе и мощность, выделяющаяся на нём.
}
\answer{%
    \begin{align*}
    U &= \eli r = 15{,}00\,\text{А} \cdot 30\,\text{Ом} = 450\,\text{В},  \\
    P &= \eli^2r = \sqr{15{,}00\,\text{А}} \cdot 30\,\text{Ом} = 6750\,\text{Вт}
    \end{align*}
}
\solutionspace{60pt}

\tasknumber{3}%
\task{%
    Замкнутая электрическая цепь состоит из ЭДС $\ele = 1\,\text{В}$ и сопротивлением $r$
    и резистора $R = 15\,\text{Ом}$.
    Определите ток, протекающий в цепи.
    Какая тепловая энергия выделится на резисторе за время
    $\tau = 5\,\text{с}$? Какая работа будет совершена ЭДС за это время? Каков знак этой работы? Чему равен КПД цепи?
    Вычислите значения для 2 случаев: $r=0$ и $r = 10\,\text{Ом}$.
}
\answer{%
    \begin{align*}
    \eli_1 &= \frac{\ele}{R} = \frac{1\,\text{В}}{15\,\text{Ом}} = \frac1{15}\units{А} \approx 0{,}07\,\text{А},  \\
    \eli_2 &= \frac{\ele}{R + r} = \frac{1\,\text{В}}{15\,\text{Ом} + 10\,\text{Ом}} = \frac1{25}\units{А} \approx 0{,}04\,\text{А},  \\
    Q_1 &= \eli_1^2R\tau = \sqr{\frac{\ele}{R}} R \tau
            = \sqr{\frac{1\,\text{В}}{15\,\text{Ом}}} \cdot 15\,\text{Ом} \cdot 5\,\text{с} = \frac13\units{Дж} \approx 0{,}333\,\text{Дж},  \\
    Q_2 &= \eli_2^2R\tau = \sqr{\frac{\ele}{R + r}} R \tau
            = \sqr{\frac{1\,\text{В}}{15\,\text{Ом} + 10\,\text{Ом}}} \cdot 15\,\text{Ом} \cdot 5\,\text{с} = \frac3{25}\units{Дж} \approx 0{,}120\,\text{Дж},  \\
    A_1 &= q_1\ele = \eli_1\tau\ele = \frac{\ele}{R} \tau \ele
            = \frac{\ele^2 \tau}{R} = \frac{\sqr{1\,\text{В}} \cdot 5\,\text{с}}{15\,\text{Ом}}
            = \frac13\units{Дж} \approx 0{,}333\,\text{Дж}, \text{положительна},  \\
    A_2 &= q_2\ele = \eli_2\tau\ele = \frac{\ele}{R + r} \tau \ele
            = \frac{\ele^2 \tau}{R + r} = \frac{\sqr{1\,\text{В}} \cdot 5\,\text{с}}{15\,\text{Ом} + 10\,\text{Ом}}
            = \frac15\units{Дж} \approx 0{,}200\,\text{Дж}, \text{положительна},  \\
    \eta_1 &= \frac{Q_1}{A_1} = \ldots = \frac{R}{R} = 1,  \\
    \eta_2 &= \frac{Q_2}{A_2} = \ldots = \frac{R}{R + r} = \frac35 \approx 0{,}60.
    \end{align*}
}
\solutionspace{180pt}

\tasknumber{4}%
\task{%
    Лампочки, сопротивления которых $R_1 = 5{,}00\,\text{Ом}$ и $R_2 = 80{,}00\,\text{Ом}$, поочерёдно подключённные к некоторому источнику тока,
    потребляют одинаковую мощность.
    Найти внутреннее сопротивление источника и КПД цепи в каждом случае.
}
\answer{%
    \begin{align*}
        P_1 &= \sqr{\frac{\ele}{R_1 + r}}R_1,
        P_2  = \sqr{\frac{\ele}{R_2 + r}}R_2,
        P_1 = P_2 \implies  \\
        &\implies R_1 \sqr{R_2 + r} = R_2 \sqr{R_1 + r} \implies  \\
        &\implies R_1 R_2^2 + 2 R_1 R_2 r + R_1 r^2 =
                    R_2 R_1^2 + 2 R_2 R_1 r + R_2 r^2  \implies  \\
    &\implies r^2 (R_2 - R_1) = R_2^2 R_2 - R_1^2 R_2 \implies  \\
    &\implies r
            = \sqrt{R_1 R_2 \frac{R_2 - R_1}{R_2 - R_1}}
            = \sqrt{R_1 R_2}
            = \sqrt{5{,}00\,\text{Ом} \cdot 80{,}00\,\text{Ом}}
            = 20{,}0\,\text{Ом}.
            \\
    \eta_1
            &= \frac{R_1}{R_1 + r}
            = \frac{\sqrt{R_1}}{\sqrt{R_1} + \sqrt{R_2}}
            = 0{,}200,  \\
    \eta_2
            &= \frac{R_2}{R_2 + r}
            = \frac{\sqrt{R_2}}{\sqrt{R_2} + \sqrt{R_1}}
            = 0{,}800
    \end{align*}
}
\solutionspace{120pt}

\tasknumber{5}%
\task{%
    Определите ток, протекающий через резистор $R = 10\,\text{Ом}$ и разность потенциалов на нём (см.
    рис.
    на доске),
    если $r_1 = 3\,\text{Ом}$, $r_2 = 3\,\text{Ом}$, $\ele_1 = 20\,\text{В}$, $\ele_2 = 40\,\text{В}$.
}
\answer{%
    Обозначим на рисунке все токи: направление произвольно, но его надо зафиксировать.
    Всего на рисунке 3 контура и 2 узла.
    Поэтому можно записать $3 - 1 = 2$ уравнения законов Кирхгофа для замкнутого контура и $2 - 1 = 1$ — для узлов
    (остальные уравнения тоже можно записать, но они не дадут полезной информации, а будут лишь следствиями уже записанных).

    Отметим на рисунке 2 контура (и не забуем указать направление) и 1 узел (точка «1»ы, выделена жирным).
    Выбор контуров и узлов не критичен: получившаяся система может быть чуть проще или сложнее, но не слишком.

    \begin{tikzpicture}[circuit ee IEC, thick]
        \draw  (0, 0) to [current direction={near end, info=$\eli_1$}] (0, 3)
                to [battery={rotate=-180,info={$\ele_1, r_1$}}]
                (3, 3)
                to [battery={info'={$\ele_2, r_2$}}]
                (6, 3) to [current direction'={near start, info=$\eli_2$}] (6, 0) -- (0, 0)
                (3, 0) to [current direction={near start, info=$\eli$}, resistor={near end, info=$R$}] (3, 3);
        \draw [-{Latex},color=red] (1.2, 1.7) arc [start angle = 135, end angle = -160, radius = 0.6];
        \draw [-{Latex},color=blue] (4.2, 1.7) arc [start angle = 135, end angle = -160, radius = 0.6];
        \node [contact,color=green!71!black] (bottomc) at (3, 0) {};
        \node [below] (bottom) at (3, 0) {$2$};
        \node [above] (top) at (3, 3) {$1$};
    \end{tikzpicture}

    \begin{align*}
        &\begin{cases}
            {\color{red} \ele_1 = \eli_1 r_1 - \eli R}, \\
            {\color{blue} -\ele_2 = -\eli_2 r_2 + \eli R}, \\
            {\color{green!71!black} - \eli - \eli_1 - \eli_2 = 0 };
        \end{cases}
        \qquad \implies \qquad
        \begin{cases}
            \eli_1 = \frac{\ele_1 + \eli R}{r_1}, \\
            \eli_2 = \frac{\ele_2 + \eli R}{r_2}, \\
            \eli + \eli_1 + \eli_2 = 0;
        \end{cases} \implies \\
        &\implies
         \eli + \frac{\ele_1 + \eli R}{r_1:L} + \frac{\ele_2 + \eli R}{r_2:L} = 0, \\
        &\eli\cbr{ 1 + \frac R{r_1:L} + \frac R{r_2:L}} + \frac{\ele_1 }{r_1:L} + \frac{\ele_2 }{r_2:L} = 0, \\
        &\eli
            = - \frac{\frac{\ele_1 }{r_1:L} + \frac{\ele_2 }{r_2:L}}{ 1 + \frac R{r_1:L} + \frac R{r_2:L}}
            = - \frac{\frac{20\,\text{В}}{3\,\text{Ом}} + \frac{40\,\text{В}}{3\,\text{Ом}}}{ 1 + \frac{10\,\text{Ом}}{3\,\text{Ом}} + \frac{10\,\text{Ом}}{3\,\text{Ом}}}
            = - \frac{60}{23}\units{А}
            \approx -2{,}60\,\text{А}, \\
        &U  = \varphi_2 - \varphi_1 = \eli R
            = - \frac{\frac{\ele_1 }{r_1:L} + \frac{\ele_2 }{r_2:L}}{ 1 + \frac R{r_1:L} + \frac R{r_2:L}} R
            \approx -26{,}10\,\text{В}.
    \end{align*}
    Оба ответа отрицательны, потому что мы изначально «не угадали» с направлением тока.
    Расчёт же показал,
    что ток через резистор $R$ течёт в противоположную сторону: вниз на рисунке, а потенциал точки 1 больше потенциала точки 2,
    а электрический ток ожидаемо течёт из точки с большим потенциалов в точку с меньшим.

    Кстати, если продолжить расчёт и вычислить значения ещё двух токов (формулы для $\eli_1$ и $\eli_2$, куда подставлять, выписаны выше),
    то по их знакам можно будет понять: угадали ли мы с их направлением или нет.
}

\variantsplitter

\addpersonalvariant{Вероника Битерякова}

\tasknumber{1}%
\task{%
    На резистор сопротивлением $r = 12\,\text{Ом}$ подали напряжение $V = 180\,\text{В}$.
    Определите ток, который потечёт через резистор, и мощность, выделяющуюся на нём.
}
\answer{%
    \begin{align*}
    \eli &= \frac{V}{r} = \frac{180\,\text{В}}{12\,\text{Ом}} = 15{,}00\,\text{А},  \\
    P &= \frac{V^2}{r} = \frac{\sqr{180\,\text{В}}}{12\,\text{Ом}} = 2700{,}00\,\text{Вт}
    \end{align*}
}
\solutionspace{60pt}

\tasknumber{2}%
\task{%
    Через резистор сопротивлением $R = 5\,\text{Ом}$ протекает электрический ток $\eli = 6{,}00\,\text{А}$.
    Определите, чему равны напряжение на резисторе и мощность, выделяющаяся на нём.
}
\answer{%
    \begin{align*}
    U &= \eli R = 6{,}00\,\text{А} \cdot 5\,\text{Ом} = 30\,\text{В},  \\
    P &= \eli^2R = \sqr{6{,}00\,\text{А}} \cdot 5\,\text{Ом} = 180\,\text{Вт}
    \end{align*}
}
\solutionspace{60pt}

\tasknumber{3}%
\task{%
    Замкнутая электрическая цепь состоит из ЭДС $\ele = 3\,\text{В}$ и сопротивлением $r$
    и резистора $R = 24\,\text{Ом}$.
    Определите ток, протекающий в цепи.
    Какая тепловая энергия выделится на резисторе за время
    $\tau = 5\,\text{с}$? Какая работа будет совершена ЭДС за это время? Каков знак этой работы? Чему равен КПД цепи?
    Вычислите значения для 2 случаев: $r=0$ и $r = 10\,\text{Ом}$.
}
\answer{%
    \begin{align*}
    \eli_1 &= \frac{\ele}{R} = \frac{3\,\text{В}}{24\,\text{Ом}} = \frac18\units{А} \approx 0{,}12\,\text{А},  \\
    \eli_2 &= \frac{\ele}{R + r} = \frac{3\,\text{В}}{24\,\text{Ом} + 10\,\text{Ом}} = \frac3{34}\units{А} \approx 0{,}09\,\text{А},  \\
    Q_1 &= \eli_1^2R\tau = \sqr{\frac{\ele}{R}} R \tau
            = \sqr{\frac{3\,\text{В}}{24\,\text{Ом}}} \cdot 24\,\text{Ом} \cdot 5\,\text{с} = \frac{15}8\units{Дж} \approx 1{,}875\,\text{Дж},  \\
    Q_2 &= \eli_2^2R\tau = \sqr{\frac{\ele}{R + r}} R \tau
            = \sqr{\frac{3\,\text{В}}{24\,\text{Ом} + 10\,\text{Ом}}} \cdot 24\,\text{Ом} \cdot 5\,\text{с} = \frac{270}{289}\units{Дж} \approx 0{,}934\,\text{Дж},  \\
    A_1 &= q_1\ele = \eli_1\tau\ele = \frac{\ele}{R} \tau \ele
            = \frac{\ele^2 \tau}{R} = \frac{\sqr{3\,\text{В}} \cdot 5\,\text{с}}{24\,\text{Ом}}
            = \frac{15}8\units{Дж} \approx 1{,}875\,\text{Дж}, \text{положительна},  \\
    A_2 &= q_2\ele = \eli_2\tau\ele = \frac{\ele}{R + r} \tau \ele
            = \frac{\ele^2 \tau}{R + r} = \frac{\sqr{3\,\text{В}} \cdot 5\,\text{с}}{24\,\text{Ом} + 10\,\text{Ом}}
            = \frac{45}{34}\units{Дж} \approx 1{,}324\,\text{Дж}, \text{положительна},  \\
    \eta_1 &= \frac{Q_1}{A_1} = \ldots = \frac{R}{R} = 1,  \\
    \eta_2 &= \frac{Q_2}{A_2} = \ldots = \frac{R}{R + r} = \frac{12}{17} \approx 0{,}71.
    \end{align*}
}
\solutionspace{180pt}

\tasknumber{4}%
\task{%
    Лампочки, сопротивления которых $R_1 = 6{,}00\,\text{Ом}$ и $R_2 = 54{,}00\,\text{Ом}$, поочерёдно подключённные к некоторому источнику тока,
    потребляют одинаковую мощность.
    Найти внутреннее сопротивление источника и КПД цепи в каждом случае.
}
\answer{%
    \begin{align*}
        P_1 &= \sqr{\frac{\ele}{R_1 + r}}R_1,
        P_2  = \sqr{\frac{\ele}{R_2 + r}}R_2,
        P_1 = P_2 \implies  \\
        &\implies R_1 \sqr{R_2 + r} = R_2 \sqr{R_1 + r} \implies  \\
        &\implies R_1 R_2^2 + 2 R_1 R_2 r + R_1 r^2 =
                    R_2 R_1^2 + 2 R_2 R_1 r + R_2 r^2  \implies  \\
    &\implies r^2 (R_2 - R_1) = R_2^2 R_2 - R_1^2 R_2 \implies  \\
    &\implies r
            = \sqrt{R_1 R_2 \frac{R_2 - R_1}{R_2 - R_1}}
            = \sqrt{R_1 R_2}
            = \sqrt{6{,}00\,\text{Ом} \cdot 54{,}00\,\text{Ом}}
            = 18{,}0\,\text{Ом}.
            \\
    \eta_1
            &= \frac{R_1}{R_1 + r}
            = \frac{\sqrt{R_1}}{\sqrt{R_1} + \sqrt{R_2}}
            = 0{,}250,  \\
    \eta_2
            &= \frac{R_2}{R_2 + r}
            = \frac{\sqrt{R_2}}{\sqrt{R_2} + \sqrt{R_1}}
            = 0{,}750
    \end{align*}
}
\solutionspace{120pt}

\tasknumber{5}%
\task{%
    Определите ток, протекающий через резистор $R = 10\,\text{Ом}$ и разность потенциалов на нём (см.
    рис.
    на доске),
    если $r_1 = 3\,\text{Ом}$, $r_2 = 3\,\text{Ом}$, $\ele_1 = 20\,\text{В}$, $\ele_2 = 30\,\text{В}$.
}
\answer{%
    Обозначим на рисунке все токи: направление произвольно, но его надо зафиксировать.
    Всего на рисунке 3 контура и 2 узла.
    Поэтому можно записать $3 - 1 = 2$ уравнения законов Кирхгофа для замкнутого контура и $2 - 1 = 1$ — для узлов
    (остальные уравнения тоже можно записать, но они не дадут полезной информации, а будут лишь следствиями уже записанных).

    Отметим на рисунке 2 контура (и не забуем указать направление) и 1 узел (точка «1»ы, выделена жирным).
    Выбор контуров и узлов не критичен: получившаяся система может быть чуть проще или сложнее, но не слишком.

    \begin{tikzpicture}[circuit ee IEC, thick]
        \draw  (0, 0) to [current direction={near end, info=$\eli_1$}] (0, 3)
                to [battery={rotate=-180,info={$\ele_1, r_1$}}]
                (3, 3)
                to [battery={info'={$\ele_2, r_2$}}]
                (6, 3) to [current direction'={near start, info=$\eli_2$}] (6, 0) -- (0, 0)
                (3, 0) to [current direction={near start, info=$\eli$}, resistor={near end, info=$R$}] (3, 3);
        \draw [-{Latex},color=red] (1.2, 1.7) arc [start angle = 135, end angle = -160, radius = 0.6];
        \draw [-{Latex},color=blue] (4.2, 1.7) arc [start angle = 135, end angle = -160, radius = 0.6];
        \node [contact,color=green!71!black] (bottomc) at (3, 0) {};
        \node [below] (bottom) at (3, 0) {$2$};
        \node [above] (top) at (3, 3) {$1$};
    \end{tikzpicture}

    \begin{align*}
        &\begin{cases}
            {\color{red} \ele_1 = \eli_1 r_1 - \eli R}, \\
            {\color{blue} -\ele_2 = -\eli_2 r_2 + \eli R}, \\
            {\color{green!71!black} - \eli - \eli_1 - \eli_2 = 0 };
        \end{cases}
        \qquad \implies \qquad
        \begin{cases}
            \eli_1 = \frac{\ele_1 + \eli R}{r_1}, \\
            \eli_2 = \frac{\ele_2 + \eli R}{r_2}, \\
            \eli + \eli_1 + \eli_2 = 0;
        \end{cases} \implies \\
        &\implies
         \eli + \frac{\ele_1 + \eli R}{r_1:L} + \frac{\ele_2 + \eli R}{r_2:L} = 0, \\
        &\eli\cbr{ 1 + \frac R{r_1:L} + \frac R{r_2:L}} + \frac{\ele_1 }{r_1:L} + \frac{\ele_2 }{r_2:L} = 0, \\
        &\eli
            = - \frac{\frac{\ele_1 }{r_1:L} + \frac{\ele_2 }{r_2:L}}{ 1 + \frac R{r_1:L} + \frac R{r_2:L}}
            = - \frac{\frac{20\,\text{В}}{3\,\text{Ом}} + \frac{30\,\text{В}}{3\,\text{Ом}}}{ 1 + \frac{10\,\text{Ом}}{3\,\text{Ом}} + \frac{10\,\text{Ом}}{3\,\text{Ом}}}
            = - \frac{50}{23}\units{А}
            \approx -2{,}20\,\text{А}, \\
        &U  = \varphi_2 - \varphi_1 = \eli R
            = - \frac{\frac{\ele_1 }{r_1:L} + \frac{\ele_2 }{r_2:L}}{ 1 + \frac R{r_1:L} + \frac R{r_2:L}} R
            \approx -21{,}70\,\text{В}.
    \end{align*}
    Оба ответа отрицательны, потому что мы изначально «не угадали» с направлением тока.
    Расчёт же показал,
    что ток через резистор $R$ течёт в противоположную сторону: вниз на рисунке, а потенциал точки 1 больше потенциала точки 2,
    а электрический ток ожидаемо течёт из точки с большим потенциалов в точку с меньшим.

    Кстати, если продолжить расчёт и вычислить значения ещё двух токов (формулы для $\eli_1$ и $\eli_2$, куда подставлять, выписаны выше),
    то по их знакам можно будет понять: угадали ли мы с их направлением или нет.
}

\variantsplitter

\addpersonalvariant{Юлия Буянова}

\tasknumber{1}%
\task{%
    На резистор сопротивлением $r = 30\,\text{Ом}$ подали напряжение $V = 180\,\text{В}$.
    Определите ток, который потечёт через резистор, и мощность, выделяющуюся на нём.
}
\answer{%
    \begin{align*}
    \eli &= \frac{V}{r} = \frac{180\,\text{В}}{30\,\text{Ом}} = 6{,}00\,\text{А},  \\
    P &= \frac{V^2}{r} = \frac{\sqr{180\,\text{В}}}{30\,\text{Ом}} = 1080{,}00\,\text{Вт}
    \end{align*}
}
\solutionspace{60pt}

\tasknumber{2}%
\task{%
    Через резистор сопротивлением $r = 18\,\text{Ом}$ протекает электрический ток $\eli = 15{,}00\,\text{А}$.
    Определите, чему равны напряжение на резисторе и мощность, выделяющаяся на нём.
}
\answer{%
    \begin{align*}
    U &= \eli r = 15{,}00\,\text{А} \cdot 18\,\text{Ом} = 270\,\text{В},  \\
    P &= \eli^2r = \sqr{15{,}00\,\text{А}} \cdot 18\,\text{Ом} = 4050\,\text{Вт}
    \end{align*}
}
\solutionspace{60pt}

\tasknumber{3}%
\task{%
    Замкнутая электрическая цепь состоит из ЭДС $\ele = 3\,\text{В}$ и сопротивлением $r$
    и резистора $R = 30\,\text{Ом}$.
    Определите ток, протекающий в цепи.
    Какая тепловая энергия выделится на резисторе за время
    $\tau = 10\,\text{с}$? Какая работа будет совершена ЭДС за это время? Каков знак этой работы? Чему равен КПД цепи?
    Вычислите значения для 2 случаев: $r=0$ и $r = 30\,\text{Ом}$.
}
\answer{%
    \begin{align*}
    \eli_1 &= \frac{\ele}{R} = \frac{3\,\text{В}}{30\,\text{Ом}} = \frac1{10}\units{А} \approx 0{,}10\,\text{А},  \\
    \eli_2 &= \frac{\ele}{R + r} = \frac{3\,\text{В}}{30\,\text{Ом} + 30\,\text{Ом}} = \frac1{20}\units{А} \approx 0{,}05\,\text{А},  \\
    Q_1 &= \eli_1^2R\tau = \sqr{\frac{\ele}{R}} R \tau
            = \sqr{\frac{3\,\text{В}}{30\,\text{Ом}}} \cdot 30\,\text{Ом} \cdot 10\,\text{с} = 3\units{Дж} \approx 3{,}000\,\text{Дж},  \\
    Q_2 &= \eli_2^2R\tau = \sqr{\frac{\ele}{R + r}} R \tau
            = \sqr{\frac{3\,\text{В}}{30\,\text{Ом} + 30\,\text{Ом}}} \cdot 30\,\text{Ом} \cdot 10\,\text{с} = \frac34\units{Дж} \approx 0{,}750\,\text{Дж},  \\
    A_1 &= q_1\ele = \eli_1\tau\ele = \frac{\ele}{R} \tau \ele
            = \frac{\ele^2 \tau}{R} = \frac{\sqr{3\,\text{В}} \cdot 10\,\text{с}}{30\,\text{Ом}}
            = 3\units{Дж} \approx 3{,}000\,\text{Дж}, \text{положительна},  \\
    A_2 &= q_2\ele = \eli_2\tau\ele = \frac{\ele}{R + r} \tau \ele
            = \frac{\ele^2 \tau}{R + r} = \frac{\sqr{3\,\text{В}} \cdot 10\,\text{с}}{30\,\text{Ом} + 30\,\text{Ом}}
            = \frac32\units{Дж} \approx 1{,}500\,\text{Дж}, \text{положительна},  \\
    \eta_1 &= \frac{Q_1}{A_1} = \ldots = \frac{R}{R} = 1,  \\
    \eta_2 &= \frac{Q_2}{A_2} = \ldots = \frac{R}{R + r} = \frac12 \approx 0{,}50.
    \end{align*}
}
\solutionspace{180pt}

\tasknumber{4}%
\task{%
    Лампочки, сопротивления которых $R_1 = 0{,}50\,\text{Ом}$ и $R_2 = 18{,}00\,\text{Ом}$, поочерёдно подключённные к некоторому источнику тока,
    потребляют одинаковую мощность.
    Найти внутреннее сопротивление источника и КПД цепи в каждом случае.
}
\answer{%
    \begin{align*}
        P_1 &= \sqr{\frac{\ele}{R_1 + r}}R_1,
        P_2  = \sqr{\frac{\ele}{R_2 + r}}R_2,
        P_1 = P_2 \implies  \\
        &\implies R_1 \sqr{R_2 + r} = R_2 \sqr{R_1 + r} \implies  \\
        &\implies R_1 R_2^2 + 2 R_1 R_2 r + R_1 r^2 =
                    R_2 R_1^2 + 2 R_2 R_1 r + R_2 r^2  \implies  \\
    &\implies r^2 (R_2 - R_1) = R_2^2 R_2 - R_1^2 R_2 \implies  \\
    &\implies r
            = \sqrt{R_1 R_2 \frac{R_2 - R_1}{R_2 - R_1}}
            = \sqrt{R_1 R_2}
            = \sqrt{0{,}50\,\text{Ом} \cdot 18{,}00\,\text{Ом}}
            = 3{,}0\,\text{Ом}.
            \\
    \eta_1
            &= \frac{R_1}{R_1 + r}
            = \frac{\sqrt{R_1}}{\sqrt{R_1} + \sqrt{R_2}}
            = 0{,}143,  \\
    \eta_2
            &= \frac{R_2}{R_2 + r}
            = \frac{\sqrt{R_2}}{\sqrt{R_2} + \sqrt{R_1}}
            = 0{,}857
    \end{align*}
}
\solutionspace{120pt}

\tasknumber{5}%
\task{%
    Определите ток, протекающий через резистор $R = 10\,\text{Ом}$ и разность потенциалов на нём (см.
    рис.
    на доске),
    если $r_1 = 1\,\text{Ом}$, $r_2 = 1\,\text{Ом}$, $\ele_1 = 20\,\text{В}$, $\ele_2 = 20\,\text{В}$.
}
\answer{%
    Обозначим на рисунке все токи: направление произвольно, но его надо зафиксировать.
    Всего на рисунке 3 контура и 2 узла.
    Поэтому можно записать $3 - 1 = 2$ уравнения законов Кирхгофа для замкнутого контура и $2 - 1 = 1$ — для узлов
    (остальные уравнения тоже можно записать, но они не дадут полезной информации, а будут лишь следствиями уже записанных).

    Отметим на рисунке 2 контура (и не забуем указать направление) и 1 узел (точка «1»ы, выделена жирным).
    Выбор контуров и узлов не критичен: получившаяся система может быть чуть проще или сложнее, но не слишком.

    \begin{tikzpicture}[circuit ee IEC, thick]
        \draw  (0, 0) to [current direction={near end, info=$\eli_1$}] (0, 3)
                to [battery={rotate=-180,info={$\ele_1, r_1$}}]
                (3, 3)
                to [battery={info'={$\ele_2, r_2$}}]
                (6, 3) to [current direction'={near start, info=$\eli_2$}] (6, 0) -- (0, 0)
                (3, 0) to [current direction={near start, info=$\eli$}, resistor={near end, info=$R$}] (3, 3);
        \draw [-{Latex},color=red] (1.2, 1.7) arc [start angle = 135, end angle = -160, radius = 0.6];
        \draw [-{Latex},color=blue] (4.2, 1.7) arc [start angle = 135, end angle = -160, radius = 0.6];
        \node [contact,color=green!71!black] (bottomc) at (3, 0) {};
        \node [below] (bottom) at (3, 0) {$2$};
        \node [above] (top) at (3, 3) {$1$};
    \end{tikzpicture}

    \begin{align*}
        &\begin{cases}
            {\color{red} \ele_1 = \eli_1 r_1 - \eli R}, \\
            {\color{blue} -\ele_2 = -\eli_2 r_2 + \eli R}, \\
            {\color{green!71!black} - \eli - \eli_1 - \eli_2 = 0 };
        \end{cases}
        \qquad \implies \qquad
        \begin{cases}
            \eli_1 = \frac{\ele_1 + \eli R}{r_1}, \\
            \eli_2 = \frac{\ele_2 + \eli R}{r_2}, \\
            \eli + \eli_1 + \eli_2 = 0;
        \end{cases} \implies \\
        &\implies
         \eli + \frac{\ele_1 + \eli R}{r_1:L} + \frac{\ele_2 + \eli R}{r_2:L} = 0, \\
        &\eli\cbr{ 1 + \frac R{r_1:L} + \frac R{r_2:L}} + \frac{\ele_1 }{r_1:L} + \frac{\ele_2 }{r_2:L} = 0, \\
        &\eli
            = - \frac{\frac{\ele_1 }{r_1:L} + \frac{\ele_2 }{r_2:L}}{ 1 + \frac R{r_1:L} + \frac R{r_2:L}}
            = - \frac{\frac{20\,\text{В}}{1\,\text{Ом}} + \frac{20\,\text{В}}{1\,\text{Ом}}}{ 1 + \frac{10\,\text{Ом}}{1\,\text{Ом}} + \frac{10\,\text{Ом}}{1\,\text{Ом}}}
            = - \frac{40}{21}\units{А}
            \approx -1{,}900\,\text{А}, \\
        &U  = \varphi_2 - \varphi_1 = \eli R
            = - \frac{\frac{\ele_1 }{r_1:L} + \frac{\ele_2 }{r_2:L}}{ 1 + \frac R{r_1:L} + \frac R{r_2:L}} R
            \approx -19{,}000\,\text{В}.
    \end{align*}
    Оба ответа отрицательны, потому что мы изначально «не угадали» с направлением тока.
    Расчёт же показал,
    что ток через резистор $R$ течёт в противоположную сторону: вниз на рисунке, а потенциал точки 1 больше потенциала точки 2,
    а электрический ток ожидаемо течёт из точки с большим потенциалов в точку с меньшим.

    Кстати, если продолжить расчёт и вычислить значения ещё двух токов (формулы для $\eli_1$ и $\eli_2$, куда подставлять, выписаны выше),
    то по их знакам можно будет понять: угадали ли мы с их направлением или нет.
}

\variantsplitter

\addpersonalvariant{Пелагея Вдовина}

\tasknumber{1}%
\task{%
    На резистор сопротивлением $r = 12\,\text{Ом}$ подали напряжение $V = 120\,\text{В}$.
    Определите ток, который потечёт через резистор, и мощность, выделяющуюся на нём.
}
\answer{%
    \begin{align*}
    \eli &= \frac{V}{r} = \frac{120\,\text{В}}{12\,\text{Ом}} = 10{,}00\,\text{А},  \\
    P &= \frac{V^2}{r} = \frac{\sqr{120\,\text{В}}}{12\,\text{Ом}} = 1200{,}00\,\text{Вт}
    \end{align*}
}
\solutionspace{60pt}

\tasknumber{2}%
\task{%
    Через резистор сопротивлением $r = 5\,\text{Ом}$ протекает электрический ток $\eli = 3{,}00\,\text{А}$.
    Определите, чему равны напряжение на резисторе и мощность, выделяющаяся на нём.
}
\answer{%
    \begin{align*}
    U &= \eli r = 3{,}00\,\text{А} \cdot 5\,\text{Ом} = 15\,\text{В},  \\
    P &= \eli^2r = \sqr{3{,}00\,\text{А}} \cdot 5\,\text{Ом} = 45\,\text{Вт}
    \end{align*}
}
\solutionspace{60pt}

\tasknumber{3}%
\task{%
    Замкнутая электрическая цепь состоит из ЭДС $\ele = 1\,\text{В}$ и сопротивлением $r$
    и резистора $R = 24\,\text{Ом}$.
    Определите ток, протекающий в цепи.
    Какая тепловая энергия выделится на резисторе за время
    $\tau = 10\,\text{с}$? Какая работа будет совершена ЭДС за это время? Каков знак этой работы? Чему равен КПД цепи?
    Вычислите значения для 2 случаев: $r=0$ и $r = 30\,\text{Ом}$.
}
\answer{%
    \begin{align*}
    \eli_1 &= \frac{\ele}{R} = \frac{1\,\text{В}}{24\,\text{Ом}} = \frac1{24}\units{А} \approx 0{,}04\,\text{А},  \\
    \eli_2 &= \frac{\ele}{R + r} = \frac{1\,\text{В}}{24\,\text{Ом} + 30\,\text{Ом}} = \frac1{54}\units{А} \approx 0{,}02\,\text{А},  \\
    Q_1 &= \eli_1^2R\tau = \sqr{\frac{\ele}{R}} R \tau
            = \sqr{\frac{1\,\text{В}}{24\,\text{Ом}}} \cdot 24\,\text{Ом} \cdot 10\,\text{с} = \frac5{12}\units{Дж} \approx 0{,}417\,\text{Дж},  \\
    Q_2 &= \eli_2^2R\tau = \sqr{\frac{\ele}{R + r}} R \tau
            = \sqr{\frac{1\,\text{В}}{24\,\text{Ом} + 30\,\text{Ом}}} \cdot 24\,\text{Ом} \cdot 10\,\text{с} = \frac{20}{243}\units{Дж} \approx 0{,}082\,\text{Дж},  \\
    A_1 &= q_1\ele = \eli_1\tau\ele = \frac{\ele}{R} \tau \ele
            = \frac{\ele^2 \tau}{R} = \frac{\sqr{1\,\text{В}} \cdot 10\,\text{с}}{24\,\text{Ом}}
            = \frac5{12}\units{Дж} \approx 0{,}417\,\text{Дж}, \text{положительна},  \\
    A_2 &= q_2\ele = \eli_2\tau\ele = \frac{\ele}{R + r} \tau \ele
            = \frac{\ele^2 \tau}{R + r} = \frac{\sqr{1\,\text{В}} \cdot 10\,\text{с}}{24\,\text{Ом} + 30\,\text{Ом}}
            = \frac5{27}\units{Дж} \approx 0{,}185\,\text{Дж}, \text{положительна},  \\
    \eta_1 &= \frac{Q_1}{A_1} = \ldots = \frac{R}{R} = 1,  \\
    \eta_2 &= \frac{Q_2}{A_2} = \ldots = \frac{R}{R + r} = \frac49 \approx 0{,}44.
    \end{align*}
}
\solutionspace{180pt}

\tasknumber{4}%
\task{%
    Лампочки, сопротивления которых $R_1 = 0{,}50\,\text{Ом}$ и $R_2 = 18{,}00\,\text{Ом}$, поочерёдно подключённные к некоторому источнику тока,
    потребляют одинаковую мощность.
    Найти внутреннее сопротивление источника и КПД цепи в каждом случае.
}
\answer{%
    \begin{align*}
        P_1 &= \sqr{\frac{\ele}{R_1 + r}}R_1,
        P_2  = \sqr{\frac{\ele}{R_2 + r}}R_2,
        P_1 = P_2 \implies  \\
        &\implies R_1 \sqr{R_2 + r} = R_2 \sqr{R_1 + r} \implies  \\
        &\implies R_1 R_2^2 + 2 R_1 R_2 r + R_1 r^2 =
                    R_2 R_1^2 + 2 R_2 R_1 r + R_2 r^2  \implies  \\
    &\implies r^2 (R_2 - R_1) = R_2^2 R_2 - R_1^2 R_2 \implies  \\
    &\implies r
            = \sqrt{R_1 R_2 \frac{R_2 - R_1}{R_2 - R_1}}
            = \sqrt{R_1 R_2}
            = \sqrt{0{,}50\,\text{Ом} \cdot 18{,}00\,\text{Ом}}
            = 3{,}0\,\text{Ом}.
            \\
    \eta_1
            &= \frac{R_1}{R_1 + r}
            = \frac{\sqrt{R_1}}{\sqrt{R_1} + \sqrt{R_2}}
            = 0{,}143,  \\
    \eta_2
            &= \frac{R_2}{R_2 + r}
            = \frac{\sqrt{R_2}}{\sqrt{R_2} + \sqrt{R_1}}
            = 0{,}857
    \end{align*}
}
\solutionspace{120pt}

\tasknumber{5}%
\task{%
    Определите ток, протекающий через резистор $R = 20\,\text{Ом}$ и разность потенциалов на нём (см.
    рис.
    на доске),
    если $r_1 = 3\,\text{Ом}$, $r_2 = 1\,\text{Ом}$, $\ele_1 = 60\,\text{В}$, $\ele_2 = 30\,\text{В}$.
}
\answer{%
    Обозначим на рисунке все токи: направление произвольно, но его надо зафиксировать.
    Всего на рисунке 3 контура и 2 узла.
    Поэтому можно записать $3 - 1 = 2$ уравнения законов Кирхгофа для замкнутого контура и $2 - 1 = 1$ — для узлов
    (остальные уравнения тоже можно записать, но они не дадут полезной информации, а будут лишь следствиями уже записанных).

    Отметим на рисунке 2 контура (и не забуем указать направление) и 1 узел (точка «1»ы, выделена жирным).
    Выбор контуров и узлов не критичен: получившаяся система может быть чуть проще или сложнее, но не слишком.

    \begin{tikzpicture}[circuit ee IEC, thick]
        \draw  (0, 0) to [current direction={near end, info=$\eli_1$}] (0, 3)
                to [battery={rotate=-180,info={$\ele_1, r_1$}}]
                (3, 3)
                to [battery={info'={$\ele_2, r_2$}}]
                (6, 3) to [current direction'={near start, info=$\eli_2$}] (6, 0) -- (0, 0)
                (3, 0) to [current direction={near start, info=$\eli$}, resistor={near end, info=$R$}] (3, 3);
        \draw [-{Latex},color=red] (1.2, 1.7) arc [start angle = 135, end angle = -160, radius = 0.6];
        \draw [-{Latex},color=blue] (4.2, 1.7) arc [start angle = 135, end angle = -160, radius = 0.6];
        \node [contact,color=green!71!black] (bottomc) at (3, 0) {};
        \node [below] (bottom) at (3, 0) {$2$};
        \node [above] (top) at (3, 3) {$1$};
    \end{tikzpicture}

    \begin{align*}
        &\begin{cases}
            {\color{red} \ele_1 = \eli_1 r_1 - \eli R}, \\
            {\color{blue} -\ele_2 = -\eli_2 r_2 + \eli R}, \\
            {\color{green!71!black} - \eli - \eli_1 - \eli_2 = 0 };
        \end{cases}
        \qquad \implies \qquad
        \begin{cases}
            \eli_1 = \frac{\ele_1 + \eli R}{r_1}, \\
            \eli_2 = \frac{\ele_2 + \eli R}{r_2}, \\
            \eli + \eli_1 + \eli_2 = 0;
        \end{cases} \implies \\
        &\implies
         \eli + \frac{\ele_1 + \eli R}{r_1:L} + \frac{\ele_2 + \eli R}{r_2:L} = 0, \\
        &\eli\cbr{ 1 + \frac R{r_1:L} + \frac R{r_2:L}} + \frac{\ele_1 }{r_1:L} + \frac{\ele_2 }{r_2:L} = 0, \\
        &\eli
            = - \frac{\frac{\ele_1 }{r_1:L} + \frac{\ele_2 }{r_2:L}}{ 1 + \frac R{r_1:L} + \frac R{r_2:L}}
            = - \frac{\frac{60\,\text{В}}{3\,\text{Ом}} + \frac{30\,\text{В}}{1\,\text{Ом}}}{ 1 + \frac{20\,\text{Ом}}{3\,\text{Ом}} + \frac{20\,\text{Ом}}{1\,\text{Ом}}}
            = - \frac{150}{83}\units{А}
            \approx -1{,}800\,\text{А}, \\
        &U  = \varphi_2 - \varphi_1 = \eli R
            = - \frac{\frac{\ele_1 }{r_1:L} + \frac{\ele_2 }{r_2:L}}{ 1 + \frac R{r_1:L} + \frac R{r_2:L}} R
            \approx -36{,}10\,\text{В}.
    \end{align*}
    Оба ответа отрицательны, потому что мы изначально «не угадали» с направлением тока.
    Расчёт же показал,
    что ток через резистор $R$ течёт в противоположную сторону: вниз на рисунке, а потенциал точки 1 больше потенциала точки 2,
    а электрический ток ожидаемо течёт из точки с большим потенциалов в точку с меньшим.

    Кстати, если продолжить расчёт и вычислить значения ещё двух токов (формулы для $\eli_1$ и $\eli_2$, куда подставлять, выписаны выше),
    то по их знакам можно будет понять: угадали ли мы с их направлением или нет.
}

\variantsplitter

\addpersonalvariant{Леонид Викторов}

\tasknumber{1}%
\task{%
    На резистор сопротивлением $r = 30\,\text{Ом}$ подали напряжение $U = 240\,\text{В}$.
    Определите ток, который потечёт через резистор, и мощность, выделяющуюся на нём.
}
\answer{%
    \begin{align*}
    \eli &= \frac{U}{r} = \frac{240\,\text{В}}{30\,\text{Ом}} = 8{,}00\,\text{А},  \\
    P &= \frac{U^2}{r} = \frac{\sqr{240\,\text{В}}}{30\,\text{Ом}} = 1920{,}00\,\text{Вт}
    \end{align*}
}
\solutionspace{60pt}

\tasknumber{2}%
\task{%
    Через резистор сопротивлением $R = 5\,\text{Ом}$ протекает электрический ток $\eli = 8{,}00\,\text{А}$.
    Определите, чему равны напряжение на резисторе и мощность, выделяющаяся на нём.
}
\answer{%
    \begin{align*}
    U &= \eli R = 8{,}00\,\text{А} \cdot 5\,\text{Ом} = 40\,\text{В},  \\
    P &= \eli^2R = \sqr{8{,}00\,\text{А}} \cdot 5\,\text{Ом} = 320\,\text{Вт}
    \end{align*}
}
\solutionspace{60pt}

\tasknumber{3}%
\task{%
    Замкнутая электрическая цепь состоит из ЭДС $\ele = 4\,\text{В}$ и сопротивлением $r$
    и резистора $R = 10\,\text{Ом}$.
    Определите ток, протекающий в цепи.
    Какая тепловая энергия выделится на резисторе за время
    $\tau = 5\,\text{с}$? Какая работа будет совершена ЭДС за это время? Каков знак этой работы? Чему равен КПД цепи?
    Вычислите значения для 2 случаев: $r=0$ и $r = 60\,\text{Ом}$.
}
\answer{%
    \begin{align*}
    \eli_1 &= \frac{\ele}{R} = \frac{4\,\text{В}}{10\,\text{Ом}} = \frac25\units{А} \approx 0{,}40\,\text{А},  \\
    \eli_2 &= \frac{\ele}{R + r} = \frac{4\,\text{В}}{10\,\text{Ом} + 60\,\text{Ом}} = \frac2{35}\units{А} \approx 0{,}06\,\text{А},  \\
    Q_1 &= \eli_1^2R\tau = \sqr{\frac{\ele}{R}} R \tau
            = \sqr{\frac{4\,\text{В}}{10\,\text{Ом}}} \cdot 10\,\text{Ом} \cdot 5\,\text{с} = 8\units{Дж} \approx 8{,}000\,\text{Дж},  \\
    Q_2 &= \eli_2^2R\tau = \sqr{\frac{\ele}{R + r}} R \tau
            = \sqr{\frac{4\,\text{В}}{10\,\text{Ом} + 60\,\text{Ом}}} \cdot 10\,\text{Ом} \cdot 5\,\text{с} = \frac8{49}\units{Дж} \approx 0{,}163\,\text{Дж},  \\
    A_1 &= q_1\ele = \eli_1\tau\ele = \frac{\ele}{R} \tau \ele
            = \frac{\ele^2 \tau}{R} = \frac{\sqr{4\,\text{В}} \cdot 5\,\text{с}}{10\,\text{Ом}}
            = 8\units{Дж} \approx 8{,}000\,\text{Дж}, \text{положительна},  \\
    A_2 &= q_2\ele = \eli_2\tau\ele = \frac{\ele}{R + r} \tau \ele
            = \frac{\ele^2 \tau}{R + r} = \frac{\sqr{4\,\text{В}} \cdot 5\,\text{с}}{10\,\text{Ом} + 60\,\text{Ом}}
            = \frac87\units{Дж} \approx 1{,}143\,\text{Дж}, \text{положительна},  \\
    \eta_1 &= \frac{Q_1}{A_1} = \ldots = \frac{R}{R} = 1,  \\
    \eta_2 &= \frac{Q_2}{A_2} = \ldots = \frac{R}{R + r} = \frac17 \approx 0{,}14.
    \end{align*}
}
\solutionspace{180pt}

\tasknumber{4}%
\task{%
    Лампочки, сопротивления которых $R_1 = 1{,}00\,\text{Ом}$ и $R_2 = 9{,}00\,\text{Ом}$, поочерёдно подключённные к некоторому источнику тока,
    потребляют одинаковую мощность.
    Найти внутреннее сопротивление источника и КПД цепи в каждом случае.
}
\answer{%
    \begin{align*}
        P_1 &= \sqr{\frac{\ele}{R_1 + r}}R_1,
        P_2  = \sqr{\frac{\ele}{R_2 + r}}R_2,
        P_1 = P_2 \implies  \\
        &\implies R_1 \sqr{R_2 + r} = R_2 \sqr{R_1 + r} \implies  \\
        &\implies R_1 R_2^2 + 2 R_1 R_2 r + R_1 r^2 =
                    R_2 R_1^2 + 2 R_2 R_1 r + R_2 r^2  \implies  \\
    &\implies r^2 (R_2 - R_1) = R_2^2 R_2 - R_1^2 R_2 \implies  \\
    &\implies r
            = \sqrt{R_1 R_2 \frac{R_2 - R_1}{R_2 - R_1}}
            = \sqrt{R_1 R_2}
            = \sqrt{1{,}00\,\text{Ом} \cdot 9{,}00\,\text{Ом}}
            = 3{,}0\,\text{Ом}.
            \\
    \eta_1
            &= \frac{R_1}{R_1 + r}
            = \frac{\sqrt{R_1}}{\sqrt{R_1} + \sqrt{R_2}}
            = 0{,}250,  \\
    \eta_2
            &= \frac{R_2}{R_2 + r}
            = \frac{\sqrt{R_2}}{\sqrt{R_2} + \sqrt{R_1}}
            = 0{,}750
    \end{align*}
}
\solutionspace{120pt}

\tasknumber{5}%
\task{%
    Определите ток, протекающий через резистор $R = 20\,\text{Ом}$ и разность потенциалов на нём (см.
    рис.
    на доске),
    если $r_1 = 3\,\text{Ом}$, $r_2 = 2\,\text{Ом}$, $\ele_1 = 60\,\text{В}$, $\ele_2 = 20\,\text{В}$.
}
\answer{%
    Обозначим на рисунке все токи: направление произвольно, но его надо зафиксировать.
    Всего на рисунке 3 контура и 2 узла.
    Поэтому можно записать $3 - 1 = 2$ уравнения законов Кирхгофа для замкнутого контура и $2 - 1 = 1$ — для узлов
    (остальные уравнения тоже можно записать, но они не дадут полезной информации, а будут лишь следствиями уже записанных).

    Отметим на рисунке 2 контура (и не забуем указать направление) и 1 узел (точка «1»ы, выделена жирным).
    Выбор контуров и узлов не критичен: получившаяся система может быть чуть проще или сложнее, но не слишком.

    \begin{tikzpicture}[circuit ee IEC, thick]
        \draw  (0, 0) to [current direction={near end, info=$\eli_1$}] (0, 3)
                to [battery={rotate=-180,info={$\ele_1, r_1$}}]
                (3, 3)
                to [battery={info'={$\ele_2, r_2$}}]
                (6, 3) to [current direction'={near start, info=$\eli_2$}] (6, 0) -- (0, 0)
                (3, 0) to [current direction={near start, info=$\eli$}, resistor={near end, info=$R$}] (3, 3);
        \draw [-{Latex},color=red] (1.2, 1.7) arc [start angle = 135, end angle = -160, radius = 0.6];
        \draw [-{Latex},color=blue] (4.2, 1.7) arc [start angle = 135, end angle = -160, radius = 0.6];
        \node [contact,color=green!71!black] (bottomc) at (3, 0) {};
        \node [below] (bottom) at (3, 0) {$2$};
        \node [above] (top) at (3, 3) {$1$};
    \end{tikzpicture}

    \begin{align*}
        &\begin{cases}
            {\color{red} \ele_1 = \eli_1 r_1 - \eli R}, \\
            {\color{blue} -\ele_2 = -\eli_2 r_2 + \eli R}, \\
            {\color{green!71!black} - \eli - \eli_1 - \eli_2 = 0 };
        \end{cases}
        \qquad \implies \qquad
        \begin{cases}
            \eli_1 = \frac{\ele_1 + \eli R}{r_1}, \\
            \eli_2 = \frac{\ele_2 + \eli R}{r_2}, \\
            \eli + \eli_1 + \eli_2 = 0;
        \end{cases} \implies \\
        &\implies
         \eli + \frac{\ele_1 + \eli R}{r_1:L} + \frac{\ele_2 + \eli R}{r_2:L} = 0, \\
        &\eli\cbr{ 1 + \frac R{r_1:L} + \frac R{r_2:L}} + \frac{\ele_1 }{r_1:L} + \frac{\ele_2 }{r_2:L} = 0, \\
        &\eli
            = - \frac{\frac{\ele_1 }{r_1:L} + \frac{\ele_2 }{r_2:L}}{ 1 + \frac R{r_1:L} + \frac R{r_2:L}}
            = - \frac{\frac{60\,\text{В}}{3\,\text{Ом}} + \frac{20\,\text{В}}{2\,\text{Ом}}}{ 1 + \frac{20\,\text{Ом}}{3\,\text{Ом}} + \frac{20\,\text{Ом}}{2\,\text{Ом}}}
            = - \frac{90}{53}\units{А}
            \approx -1{,}700\,\text{А}, \\
        &U  = \varphi_2 - \varphi_1 = \eli R
            = - \frac{\frac{\ele_1 }{r_1:L} + \frac{\ele_2 }{r_2:L}}{ 1 + \frac R{r_1:L} + \frac R{r_2:L}} R
            \approx -34{,}00\,\text{В}.
    \end{align*}
    Оба ответа отрицательны, потому что мы изначально «не угадали» с направлением тока.
    Расчёт же показал,
    что ток через резистор $R$ течёт в противоположную сторону: вниз на рисунке, а потенциал точки 1 больше потенциала точки 2,
    а электрический ток ожидаемо течёт из точки с большим потенциалов в точку с меньшим.

    Кстати, если продолжить расчёт и вычислить значения ещё двух токов (формулы для $\eli_1$ и $\eli_2$, куда подставлять, выписаны выше),
    то по их знакам можно будет понять: угадали ли мы с их направлением или нет.
}

\variantsplitter

\addpersonalvariant{Фёдор Гнутов}

\tasknumber{1}%
\task{%
    На резистор сопротивлением $R = 30\,\text{Ом}$ подали напряжение $V = 150\,\text{В}$.
    Определите ток, который потечёт через резистор, и мощность, выделяющуюся на нём.
}
\answer{%
    \begin{align*}
    \eli &= \frac{V}{R} = \frac{150\,\text{В}}{30\,\text{Ом}} = 5{,}00\,\text{А},  \\
    P &= \frac{V^2}{R} = \frac{\sqr{150\,\text{В}}}{30\,\text{Ом}} = 750{,}00\,\text{Вт}
    \end{align*}
}
\solutionspace{60pt}

\tasknumber{2}%
\task{%
    Через резистор сопротивлением $r = 18\,\text{Ом}$ протекает электрический ток $\eli = 10{,}00\,\text{А}$.
    Определите, чему равны напряжение на резисторе и мощность, выделяющаяся на нём.
}
\answer{%
    \begin{align*}
    U &= \eli r = 10{,}00\,\text{А} \cdot 18\,\text{Ом} = 180\,\text{В},  \\
    P &= \eli^2r = \sqr{10{,}00\,\text{А}} \cdot 18\,\text{Ом} = 1800\,\text{Вт}
    \end{align*}
}
\solutionspace{60pt}

\tasknumber{3}%
\task{%
    Замкнутая электрическая цепь состоит из ЭДС $\ele = 1\,\text{В}$ и сопротивлением $r$
    и резистора $R = 15\,\text{Ом}$.
    Определите ток, протекающий в цепи.
    Какая тепловая энергия выделится на резисторе за время
    $\tau = 5\,\text{с}$? Какая работа будет совершена ЭДС за это время? Каков знак этой работы? Чему равен КПД цепи?
    Вычислите значения для 2 случаев: $r=0$ и $r = 20\,\text{Ом}$.
}
\answer{%
    \begin{align*}
    \eli_1 &= \frac{\ele}{R} = \frac{1\,\text{В}}{15\,\text{Ом}} = \frac1{15}\units{А} \approx 0{,}07\,\text{А},  \\
    \eli_2 &= \frac{\ele}{R + r} = \frac{1\,\text{В}}{15\,\text{Ом} + 20\,\text{Ом}} = \frac1{35}\units{А} \approx 0{,}03\,\text{А},  \\
    Q_1 &= \eli_1^2R\tau = \sqr{\frac{\ele}{R}} R \tau
            = \sqr{\frac{1\,\text{В}}{15\,\text{Ом}}} \cdot 15\,\text{Ом} \cdot 5\,\text{с} = \frac13\units{Дж} \approx 0{,}333\,\text{Дж},  \\
    Q_2 &= \eli_2^2R\tau = \sqr{\frac{\ele}{R + r}} R \tau
            = \sqr{\frac{1\,\text{В}}{15\,\text{Ом} + 20\,\text{Ом}}} \cdot 15\,\text{Ом} \cdot 5\,\text{с} = \frac3{49}\units{Дж} \approx 0{,}061\,\text{Дж},  \\
    A_1 &= q_1\ele = \eli_1\tau\ele = \frac{\ele}{R} \tau \ele
            = \frac{\ele^2 \tau}{R} = \frac{\sqr{1\,\text{В}} \cdot 5\,\text{с}}{15\,\text{Ом}}
            = \frac13\units{Дж} \approx 0{,}333\,\text{Дж}, \text{положительна},  \\
    A_2 &= q_2\ele = \eli_2\tau\ele = \frac{\ele}{R + r} \tau \ele
            = \frac{\ele^2 \tau}{R + r} = \frac{\sqr{1\,\text{В}} \cdot 5\,\text{с}}{15\,\text{Ом} + 20\,\text{Ом}}
            = \frac17\units{Дж} \approx 0{,}143\,\text{Дж}, \text{положительна},  \\
    \eta_1 &= \frac{Q_1}{A_1} = \ldots = \frac{R}{R} = 1,  \\
    \eta_2 &= \frac{Q_2}{A_2} = \ldots = \frac{R}{R + r} = \frac37 \approx 0{,}43.
    \end{align*}
}
\solutionspace{180pt}

\tasknumber{4}%
\task{%
    Лампочки, сопротивления которых $R_1 = 5{,}00\,\text{Ом}$ и $R_2 = 80{,}00\,\text{Ом}$, поочерёдно подключённные к некоторому источнику тока,
    потребляют одинаковую мощность.
    Найти внутреннее сопротивление источника и КПД цепи в каждом случае.
}
\answer{%
    \begin{align*}
        P_1 &= \sqr{\frac{\ele}{R_1 + r}}R_1,
        P_2  = \sqr{\frac{\ele}{R_2 + r}}R_2,
        P_1 = P_2 \implies  \\
        &\implies R_1 \sqr{R_2 + r} = R_2 \sqr{R_1 + r} \implies  \\
        &\implies R_1 R_2^2 + 2 R_1 R_2 r + R_1 r^2 =
                    R_2 R_1^2 + 2 R_2 R_1 r + R_2 r^2  \implies  \\
    &\implies r^2 (R_2 - R_1) = R_2^2 R_2 - R_1^2 R_2 \implies  \\
    &\implies r
            = \sqrt{R_1 R_2 \frac{R_2 - R_1}{R_2 - R_1}}
            = \sqrt{R_1 R_2}
            = \sqrt{5{,}00\,\text{Ом} \cdot 80{,}00\,\text{Ом}}
            = 20{,}0\,\text{Ом}.
            \\
    \eta_1
            &= \frac{R_1}{R_1 + r}
            = \frac{\sqrt{R_1}}{\sqrt{R_1} + \sqrt{R_2}}
            = 0{,}200,  \\
    \eta_2
            &= \frac{R_2}{R_2 + r}
            = \frac{\sqrt{R_2}}{\sqrt{R_2} + \sqrt{R_1}}
            = 0{,}800
    \end{align*}
}
\solutionspace{120pt}

\tasknumber{5}%
\task{%
    Определите ток, протекающий через резистор $R = 15\,\text{Ом}$ и разность потенциалов на нём (см.
    рис.
    на доске),
    если $r_1 = 2\,\text{Ом}$, $r_2 = 2\,\text{Ом}$, $\ele_1 = 40\,\text{В}$, $\ele_2 = 20\,\text{В}$.
}
\answer{%
    Обозначим на рисунке все токи: направление произвольно, но его надо зафиксировать.
    Всего на рисунке 3 контура и 2 узла.
    Поэтому можно записать $3 - 1 = 2$ уравнения законов Кирхгофа для замкнутого контура и $2 - 1 = 1$ — для узлов
    (остальные уравнения тоже можно записать, но они не дадут полезной информации, а будут лишь следствиями уже записанных).

    Отметим на рисунке 2 контура (и не забуем указать направление) и 1 узел (точка «1»ы, выделена жирным).
    Выбор контуров и узлов не критичен: получившаяся система может быть чуть проще или сложнее, но не слишком.

    \begin{tikzpicture}[circuit ee IEC, thick]
        \draw  (0, 0) to [current direction={near end, info=$\eli_1$}] (0, 3)
                to [battery={rotate=-180,info={$\ele_1, r_1$}}]
                (3, 3)
                to [battery={info'={$\ele_2, r_2$}}]
                (6, 3) to [current direction'={near start, info=$\eli_2$}] (6, 0) -- (0, 0)
                (3, 0) to [current direction={near start, info=$\eli$}, resistor={near end, info=$R$}] (3, 3);
        \draw [-{Latex},color=red] (1.2, 1.7) arc [start angle = 135, end angle = -160, radius = 0.6];
        \draw [-{Latex},color=blue] (4.2, 1.7) arc [start angle = 135, end angle = -160, radius = 0.6];
        \node [contact,color=green!71!black] (bottomc) at (3, 0) {};
        \node [below] (bottom) at (3, 0) {$2$};
        \node [above] (top) at (3, 3) {$1$};
    \end{tikzpicture}

    \begin{align*}
        &\begin{cases}
            {\color{red} \ele_1 = \eli_1 r_1 - \eli R}, \\
            {\color{blue} -\ele_2 = -\eli_2 r_2 + \eli R}, \\
            {\color{green!71!black} - \eli - \eli_1 - \eli_2 = 0 };
        \end{cases}
        \qquad \implies \qquad
        \begin{cases}
            \eli_1 = \frac{\ele_1 + \eli R}{r_1}, \\
            \eli_2 = \frac{\ele_2 + \eli R}{r_2}, \\
            \eli + \eli_1 + \eli_2 = 0;
        \end{cases} \implies \\
        &\implies
         \eli + \frac{\ele_1 + \eli R}{r_1:L} + \frac{\ele_2 + \eli R}{r_2:L} = 0, \\
        &\eli\cbr{ 1 + \frac R{r_1:L} + \frac R{r_2:L}} + \frac{\ele_1 }{r_1:L} + \frac{\ele_2 }{r_2:L} = 0, \\
        &\eli
            = - \frac{\frac{\ele_1 }{r_1:L} + \frac{\ele_2 }{r_2:L}}{ 1 + \frac R{r_1:L} + \frac R{r_2:L}}
            = - \frac{\frac{40\,\text{В}}{2\,\text{Ом}} + \frac{20\,\text{В}}{2\,\text{Ом}}}{ 1 + \frac{15\,\text{Ом}}{2\,\text{Ом}} + \frac{15\,\text{Ом}}{2\,\text{Ом}}}
            = - \frac{15}8\units{А}
            \approx -1{,}900\,\text{А}, \\
        &U  = \varphi_2 - \varphi_1 = \eli R
            = - \frac{\frac{\ele_1 }{r_1:L} + \frac{\ele_2 }{r_2:L}}{ 1 + \frac R{r_1:L} + \frac R{r_2:L}} R
            \approx -28{,}10\,\text{В}.
    \end{align*}
    Оба ответа отрицательны, потому что мы изначально «не угадали» с направлением тока.
    Расчёт же показал,
    что ток через резистор $R$ течёт в противоположную сторону: вниз на рисунке, а потенциал точки 1 больше потенциала точки 2,
    а электрический ток ожидаемо течёт из точки с большим потенциалов в точку с меньшим.

    Кстати, если продолжить расчёт и вычислить значения ещё двух токов (формулы для $\eli_1$ и $\eli_2$, куда подставлять, выписаны выше),
    то по их знакам можно будет понять: угадали ли мы с их направлением или нет.
}

\variantsplitter

\addpersonalvariant{Илья Гримберг}

\tasknumber{1}%
\task{%
    На резистор сопротивлением $r = 5\,\text{Ом}$ подали напряжение $U = 180\,\text{В}$.
    Определите ток, который потечёт через резистор, и мощность, выделяющуюся на нём.
}
\answer{%
    \begin{align*}
    \eli &= \frac{U}{r} = \frac{180\,\text{В}}{5\,\text{Ом}} = 36{,}00\,\text{А},  \\
    P &= \frac{U^2}{r} = \frac{\sqr{180\,\text{В}}}{5\,\text{Ом}} = 6480{,}00\,\text{Вт}
    \end{align*}
}
\solutionspace{60pt}

\tasknumber{2}%
\task{%
    Через резистор сопротивлением $r = 5\,\text{Ом}$ протекает электрический ток $\eli = 4{,}00\,\text{А}$.
    Определите, чему равны напряжение на резисторе и мощность, выделяющаяся на нём.
}
\answer{%
    \begin{align*}
    U &= \eli r = 4{,}00\,\text{А} \cdot 5\,\text{Ом} = 20\,\text{В},  \\
    P &= \eli^2r = \sqr{4{,}00\,\text{А}} \cdot 5\,\text{Ом} = 80\,\text{Вт}
    \end{align*}
}
\solutionspace{60pt}

\tasknumber{3}%
\task{%
    Замкнутая электрическая цепь состоит из ЭДС $\ele = 2\,\text{В}$ и сопротивлением $r$
    и резистора $R = 10\,\text{Ом}$.
    Определите ток, протекающий в цепи.
    Какая тепловая энергия выделится на резисторе за время
    $\tau = 5\,\text{с}$? Какая работа будет совершена ЭДС за это время? Каков знак этой работы? Чему равен КПД цепи?
    Вычислите значения для 2 случаев: $r=0$ и $r = 20\,\text{Ом}$.
}
\answer{%
    \begin{align*}
    \eli_1 &= \frac{\ele}{R} = \frac{2\,\text{В}}{10\,\text{Ом}} = \frac15\units{А} \approx 0{,}20\,\text{А},  \\
    \eli_2 &= \frac{\ele}{R + r} = \frac{2\,\text{В}}{10\,\text{Ом} + 20\,\text{Ом}} = \frac1{15}\units{А} \approx 0{,}07\,\text{А},  \\
    Q_1 &= \eli_1^2R\tau = \sqr{\frac{\ele}{R}} R \tau
            = \sqr{\frac{2\,\text{В}}{10\,\text{Ом}}} \cdot 10\,\text{Ом} \cdot 5\,\text{с} = 2\units{Дж} \approx 2{,}000\,\text{Дж},  \\
    Q_2 &= \eli_2^2R\tau = \sqr{\frac{\ele}{R + r}} R \tau
            = \sqr{\frac{2\,\text{В}}{10\,\text{Ом} + 20\,\text{Ом}}} \cdot 10\,\text{Ом} \cdot 5\,\text{с} = \frac29\units{Дж} \approx 0{,}222\,\text{Дж},  \\
    A_1 &= q_1\ele = \eli_1\tau\ele = \frac{\ele}{R} \tau \ele
            = \frac{\ele^2 \tau}{R} = \frac{\sqr{2\,\text{В}} \cdot 5\,\text{с}}{10\,\text{Ом}}
            = 2\units{Дж} \approx 2{,}000\,\text{Дж}, \text{положительна},  \\
    A_2 &= q_2\ele = \eli_2\tau\ele = \frac{\ele}{R + r} \tau \ele
            = \frac{\ele^2 \tau}{R + r} = \frac{\sqr{2\,\text{В}} \cdot 5\,\text{с}}{10\,\text{Ом} + 20\,\text{Ом}}
            = \frac23\units{Дж} \approx 0{,}667\,\text{Дж}, \text{положительна},  \\
    \eta_1 &= \frac{Q_1}{A_1} = \ldots = \frac{R}{R} = 1,  \\
    \eta_2 &= \frac{Q_2}{A_2} = \ldots = \frac{R}{R + r} = \frac13 \approx 0{,}33.
    \end{align*}
}
\solutionspace{180pt}

\tasknumber{4}%
\task{%
    Лампочки, сопротивления которых $R_1 = 0{,}50\,\text{Ом}$ и $R_2 = 18{,}00\,\text{Ом}$, поочерёдно подключённные к некоторому источнику тока,
    потребляют одинаковую мощность.
    Найти внутреннее сопротивление источника и КПД цепи в каждом случае.
}
\answer{%
    \begin{align*}
        P_1 &= \sqr{\frac{\ele}{R_1 + r}}R_1,
        P_2  = \sqr{\frac{\ele}{R_2 + r}}R_2,
        P_1 = P_2 \implies  \\
        &\implies R_1 \sqr{R_2 + r} = R_2 \sqr{R_1 + r} \implies  \\
        &\implies R_1 R_2^2 + 2 R_1 R_2 r + R_1 r^2 =
                    R_2 R_1^2 + 2 R_2 R_1 r + R_2 r^2  \implies  \\
    &\implies r^2 (R_2 - R_1) = R_2^2 R_2 - R_1^2 R_2 \implies  \\
    &\implies r
            = \sqrt{R_1 R_2 \frac{R_2 - R_1}{R_2 - R_1}}
            = \sqrt{R_1 R_2}
            = \sqrt{0{,}50\,\text{Ом} \cdot 18{,}00\,\text{Ом}}
            = 3{,}0\,\text{Ом}.
            \\
    \eta_1
            &= \frac{R_1}{R_1 + r}
            = \frac{\sqrt{R_1}}{\sqrt{R_1} + \sqrt{R_2}}
            = 0{,}143,  \\
    \eta_2
            &= \frac{R_2}{R_2 + r}
            = \frac{\sqrt{R_2}}{\sqrt{R_2} + \sqrt{R_1}}
            = 0{,}857
    \end{align*}
}
\solutionspace{120pt}

\tasknumber{5}%
\task{%
    Определите ток, протекающий через резистор $R = 10\,\text{Ом}$ и разность потенциалов на нём (см.
    рис.
    на доске),
    если $r_1 = 3\,\text{Ом}$, $r_2 = 3\,\text{Ом}$, $\ele_1 = 60\,\text{В}$, $\ele_2 = 30\,\text{В}$.
}
\answer{%
    Обозначим на рисунке все токи: направление произвольно, но его надо зафиксировать.
    Всего на рисунке 3 контура и 2 узла.
    Поэтому можно записать $3 - 1 = 2$ уравнения законов Кирхгофа для замкнутого контура и $2 - 1 = 1$ — для узлов
    (остальные уравнения тоже можно записать, но они не дадут полезной информации, а будут лишь следствиями уже записанных).

    Отметим на рисунке 2 контура (и не забуем указать направление) и 1 узел (точка «1»ы, выделена жирным).
    Выбор контуров и узлов не критичен: получившаяся система может быть чуть проще или сложнее, но не слишком.

    \begin{tikzpicture}[circuit ee IEC, thick]
        \draw  (0, 0) to [current direction={near end, info=$\eli_1$}] (0, 3)
                to [battery={rotate=-180,info={$\ele_1, r_1$}}]
                (3, 3)
                to [battery={info'={$\ele_2, r_2$}}]
                (6, 3) to [current direction'={near start, info=$\eli_2$}] (6, 0) -- (0, 0)
                (3, 0) to [current direction={near start, info=$\eli$}, resistor={near end, info=$R$}] (3, 3);
        \draw [-{Latex},color=red] (1.2, 1.7) arc [start angle = 135, end angle = -160, radius = 0.6];
        \draw [-{Latex},color=blue] (4.2, 1.7) arc [start angle = 135, end angle = -160, radius = 0.6];
        \node [contact,color=green!71!black] (bottomc) at (3, 0) {};
        \node [below] (bottom) at (3, 0) {$2$};
        \node [above] (top) at (3, 3) {$1$};
    \end{tikzpicture}

    \begin{align*}
        &\begin{cases}
            {\color{red} \ele_1 = \eli_1 r_1 - \eli R}, \\
            {\color{blue} -\ele_2 = -\eli_2 r_2 + \eli R}, \\
            {\color{green!71!black} - \eli - \eli_1 - \eli_2 = 0 };
        \end{cases}
        \qquad \implies \qquad
        \begin{cases}
            \eli_1 = \frac{\ele_1 + \eli R}{r_1}, \\
            \eli_2 = \frac{\ele_2 + \eli R}{r_2}, \\
            \eli + \eli_1 + \eli_2 = 0;
        \end{cases} \implies \\
        &\implies
         \eli + \frac{\ele_1 + \eli R}{r_1:L} + \frac{\ele_2 + \eli R}{r_2:L} = 0, \\
        &\eli\cbr{ 1 + \frac R{r_1:L} + \frac R{r_2:L}} + \frac{\ele_1 }{r_1:L} + \frac{\ele_2 }{r_2:L} = 0, \\
        &\eli
            = - \frac{\frac{\ele_1 }{r_1:L} + \frac{\ele_2 }{r_2:L}}{ 1 + \frac R{r_1:L} + \frac R{r_2:L}}
            = - \frac{\frac{60\,\text{В}}{3\,\text{Ом}} + \frac{30\,\text{В}}{3\,\text{Ом}}}{ 1 + \frac{10\,\text{Ом}}{3\,\text{Ом}} + \frac{10\,\text{Ом}}{3\,\text{Ом}}}
            = - \frac{90}{23}\units{А}
            \approx -3{,}90\,\text{А}, \\
        &U  = \varphi_2 - \varphi_1 = \eli R
            = - \frac{\frac{\ele_1 }{r_1:L} + \frac{\ele_2 }{r_2:L}}{ 1 + \frac R{r_1:L} + \frac R{r_2:L}} R
            \approx -39{,}10\,\text{В}.
    \end{align*}
    Оба ответа отрицательны, потому что мы изначально «не угадали» с направлением тока.
    Расчёт же показал,
    что ток через резистор $R$ течёт в противоположную сторону: вниз на рисунке, а потенциал точки 1 больше потенциала точки 2,
    а электрический ток ожидаемо течёт из точки с большим потенциалов в точку с меньшим.

    Кстати, если продолжить расчёт и вычислить значения ещё двух токов (формулы для $\eli_1$ и $\eli_2$, куда подставлять, выписаны выше),
    то по их знакам можно будет понять: угадали ли мы с их направлением или нет.
}

\variantsplitter

\addpersonalvariant{Иван Гурьянов}

\tasknumber{1}%
\task{%
    На резистор сопротивлением $r = 30\,\text{Ом}$ подали напряжение $U = 240\,\text{В}$.
    Определите ток, который потечёт через резистор, и мощность, выделяющуюся на нём.
}
\answer{%
    \begin{align*}
    \eli &= \frac{U}{r} = \frac{240\,\text{В}}{30\,\text{Ом}} = 8{,}00\,\text{А},  \\
    P &= \frac{U^2}{r} = \frac{\sqr{240\,\text{В}}}{30\,\text{Ом}} = 1920{,}00\,\text{Вт}
    \end{align*}
}
\solutionspace{60pt}

\tasknumber{2}%
\task{%
    Через резистор сопротивлением $R = 18\,\text{Ом}$ протекает электрический ток $\eli = 15{,}00\,\text{А}$.
    Определите, чему равны напряжение на резисторе и мощность, выделяющаяся на нём.
}
\answer{%
    \begin{align*}
    U &= \eli R = 15{,}00\,\text{А} \cdot 18\,\text{Ом} = 270\,\text{В},  \\
    P &= \eli^2R = \sqr{15{,}00\,\text{А}} \cdot 18\,\text{Ом} = 4050\,\text{Вт}
    \end{align*}
}
\solutionspace{60pt}

\tasknumber{3}%
\task{%
    Замкнутая электрическая цепь состоит из ЭДС $\ele = 3\,\text{В}$ и сопротивлением $r$
    и резистора $R = 15\,\text{Ом}$.
    Определите ток, протекающий в цепи.
    Какая тепловая энергия выделится на резисторе за время
    $\tau = 10\,\text{с}$? Какая работа будет совершена ЭДС за это время? Каков знак этой работы? Чему равен КПД цепи?
    Вычислите значения для 2 случаев: $r=0$ и $r = 60\,\text{Ом}$.
}
\answer{%
    \begin{align*}
    \eli_1 &= \frac{\ele}{R} = \frac{3\,\text{В}}{15\,\text{Ом}} = \frac15\units{А} \approx 0{,}20\,\text{А},  \\
    \eli_2 &= \frac{\ele}{R + r} = \frac{3\,\text{В}}{15\,\text{Ом} + 60\,\text{Ом}} = \frac1{25}\units{А} \approx 0{,}04\,\text{А},  \\
    Q_1 &= \eli_1^2R\tau = \sqr{\frac{\ele}{R}} R \tau
            = \sqr{\frac{3\,\text{В}}{15\,\text{Ом}}} \cdot 15\,\text{Ом} \cdot 10\,\text{с} = 6\units{Дж} \approx 6{,}000\,\text{Дж},  \\
    Q_2 &= \eli_2^2R\tau = \sqr{\frac{\ele}{R + r}} R \tau
            = \sqr{\frac{3\,\text{В}}{15\,\text{Ом} + 60\,\text{Ом}}} \cdot 15\,\text{Ом} \cdot 10\,\text{с} = \frac6{25}\units{Дж} \approx 0{,}240\,\text{Дж},  \\
    A_1 &= q_1\ele = \eli_1\tau\ele = \frac{\ele}{R} \tau \ele
            = \frac{\ele^2 \tau}{R} = \frac{\sqr{3\,\text{В}} \cdot 10\,\text{с}}{15\,\text{Ом}}
            = 6\units{Дж} \approx 6{,}000\,\text{Дж}, \text{положительна},  \\
    A_2 &= q_2\ele = \eli_2\tau\ele = \frac{\ele}{R + r} \tau \ele
            = \frac{\ele^2 \tau}{R + r} = \frac{\sqr{3\,\text{В}} \cdot 10\,\text{с}}{15\,\text{Ом} + 60\,\text{Ом}}
            = \frac65\units{Дж} \approx 1{,}200\,\text{Дж}, \text{положительна},  \\
    \eta_1 &= \frac{Q_1}{A_1} = \ldots = \frac{R}{R} = 1,  \\
    \eta_2 &= \frac{Q_2}{A_2} = \ldots = \frac{R}{R + r} = \frac15 \approx 0{,}20.
    \end{align*}
}
\solutionspace{180pt}

\tasknumber{4}%
\task{%
    Лампочки, сопротивления которых $R_1 = 0{,}50\,\text{Ом}$ и $R_2 = 2{,}00\,\text{Ом}$, поочерёдно подключённные к некоторому источнику тока,
    потребляют одинаковую мощность.
    Найти внутреннее сопротивление источника и КПД цепи в каждом случае.
}
\answer{%
    \begin{align*}
        P_1 &= \sqr{\frac{\ele}{R_1 + r}}R_1,
        P_2  = \sqr{\frac{\ele}{R_2 + r}}R_2,
        P_1 = P_2 \implies  \\
        &\implies R_1 \sqr{R_2 + r} = R_2 \sqr{R_1 + r} \implies  \\
        &\implies R_1 R_2^2 + 2 R_1 R_2 r + R_1 r^2 =
                    R_2 R_1^2 + 2 R_2 R_1 r + R_2 r^2  \implies  \\
    &\implies r^2 (R_2 - R_1) = R_2^2 R_2 - R_1^2 R_2 \implies  \\
    &\implies r
            = \sqrt{R_1 R_2 \frac{R_2 - R_1}{R_2 - R_1}}
            = \sqrt{R_1 R_2}
            = \sqrt{0{,}50\,\text{Ом} \cdot 2{,}00\,\text{Ом}}
            = 1{,}0\,\text{Ом}.
            \\
    \eta_1
            &= \frac{R_1}{R_1 + r}
            = \frac{\sqrt{R_1}}{\sqrt{R_1} + \sqrt{R_2}}
            = 0{,}333,  \\
    \eta_2
            &= \frac{R_2}{R_2 + r}
            = \frac{\sqrt{R_2}}{\sqrt{R_2} + \sqrt{R_1}}
            = 0{,}667
    \end{align*}
}
\solutionspace{120pt}

\tasknumber{5}%
\task{%
    Определите ток, протекающий через резистор $R = 15\,\text{Ом}$ и разность потенциалов на нём (см.
    рис.
    на доске),
    если $r_1 = 1\,\text{Ом}$, $r_2 = 2\,\text{Ом}$, $\ele_1 = 20\,\text{В}$, $\ele_2 = 60\,\text{В}$.
}
\answer{%
    Обозначим на рисунке все токи: направление произвольно, но его надо зафиксировать.
    Всего на рисунке 3 контура и 2 узла.
    Поэтому можно записать $3 - 1 = 2$ уравнения законов Кирхгофа для замкнутого контура и $2 - 1 = 1$ — для узлов
    (остальные уравнения тоже можно записать, но они не дадут полезной информации, а будут лишь следствиями уже записанных).

    Отметим на рисунке 2 контура (и не забуем указать направление) и 1 узел (точка «1»ы, выделена жирным).
    Выбор контуров и узлов не критичен: получившаяся система может быть чуть проще или сложнее, но не слишком.

    \begin{tikzpicture}[circuit ee IEC, thick]
        \draw  (0, 0) to [current direction={near end, info=$\eli_1$}] (0, 3)
                to [battery={rotate=-180,info={$\ele_1, r_1$}}]
                (3, 3)
                to [battery={info'={$\ele_2, r_2$}}]
                (6, 3) to [current direction'={near start, info=$\eli_2$}] (6, 0) -- (0, 0)
                (3, 0) to [current direction={near start, info=$\eli$}, resistor={near end, info=$R$}] (3, 3);
        \draw [-{Latex},color=red] (1.2, 1.7) arc [start angle = 135, end angle = -160, radius = 0.6];
        \draw [-{Latex},color=blue] (4.2, 1.7) arc [start angle = 135, end angle = -160, radius = 0.6];
        \node [contact,color=green!71!black] (bottomc) at (3, 0) {};
        \node [below] (bottom) at (3, 0) {$2$};
        \node [above] (top) at (3, 3) {$1$};
    \end{tikzpicture}

    \begin{align*}
        &\begin{cases}
            {\color{red} \ele_1 = \eli_1 r_1 - \eli R}, \\
            {\color{blue} -\ele_2 = -\eli_2 r_2 + \eli R}, \\
            {\color{green!71!black} - \eli - \eli_1 - \eli_2 = 0 };
        \end{cases}
        \qquad \implies \qquad
        \begin{cases}
            \eli_1 = \frac{\ele_1 + \eli R}{r_1}, \\
            \eli_2 = \frac{\ele_2 + \eli R}{r_2}, \\
            \eli + \eli_1 + \eli_2 = 0;
        \end{cases} \implies \\
        &\implies
         \eli + \frac{\ele_1 + \eli R}{r_1:L} + \frac{\ele_2 + \eli R}{r_2:L} = 0, \\
        &\eli\cbr{ 1 + \frac R{r_1:L} + \frac R{r_2:L}} + \frac{\ele_1 }{r_1:L} + \frac{\ele_2 }{r_2:L} = 0, \\
        &\eli
            = - \frac{\frac{\ele_1 }{r_1:L} + \frac{\ele_2 }{r_2:L}}{ 1 + \frac R{r_1:L} + \frac R{r_2:L}}
            = - \frac{\frac{20\,\text{В}}{1\,\text{Ом}} + \frac{60\,\text{В}}{2\,\text{Ом}}}{ 1 + \frac{15\,\text{Ом}}{1\,\text{Ом}} + \frac{15\,\text{Ом}}{2\,\text{Ом}}}
            = - \frac{100}{47}\units{А}
            \approx -2{,}10\,\text{А}, \\
        &U  = \varphi_2 - \varphi_1 = \eli R
            = - \frac{\frac{\ele_1 }{r_1:L} + \frac{\ele_2 }{r_2:L}}{ 1 + \frac R{r_1:L} + \frac R{r_2:L}} R
            \approx -31{,}90\,\text{В}.
    \end{align*}
    Оба ответа отрицательны, потому что мы изначально «не угадали» с направлением тока.
    Расчёт же показал,
    что ток через резистор $R$ течёт в противоположную сторону: вниз на рисунке, а потенциал точки 1 больше потенциала точки 2,
    а электрический ток ожидаемо течёт из точки с большим потенциалов в точку с меньшим.

    Кстати, если продолжить расчёт и вычислить значения ещё двух токов (формулы для $\eli_1$ и $\eli_2$, куда подставлять, выписаны выше),
    то по их знакам можно будет понять: угадали ли мы с их направлением или нет.
}

\variantsplitter

\addpersonalvariant{Артём Денежкин}

\tasknumber{1}%
\task{%
    На резистор сопротивлением $R = 5\,\text{Ом}$ подали напряжение $U = 150\,\text{В}$.
    Определите ток, который потечёт через резистор, и мощность, выделяющуюся на нём.
}
\answer{%
    \begin{align*}
    \eli &= \frac{U}{R} = \frac{150\,\text{В}}{5\,\text{Ом}} = 30{,}00\,\text{А},  \\
    P &= \frac{U^2}{R} = \frac{\sqr{150\,\text{В}}}{5\,\text{Ом}} = 4500{,}00\,\text{Вт}
    \end{align*}
}
\solutionspace{60pt}

\tasknumber{2}%
\task{%
    Через резистор сопротивлением $r = 30\,\text{Ом}$ протекает электрический ток $\eli = 3{,}00\,\text{А}$.
    Определите, чему равны напряжение на резисторе и мощность, выделяющаяся на нём.
}
\answer{%
    \begin{align*}
    U &= \eli r = 3{,}00\,\text{А} \cdot 30\,\text{Ом} = 90\,\text{В},  \\
    P &= \eli^2r = \sqr{3{,}00\,\text{А}} \cdot 30\,\text{Ом} = 270\,\text{Вт}
    \end{align*}
}
\solutionspace{60pt}

\tasknumber{3}%
\task{%
    Замкнутая электрическая цепь состоит из ЭДС $\ele = 4\,\text{В}$ и сопротивлением $r$
    и резистора $R = 24\,\text{Ом}$.
    Определите ток, протекающий в цепи.
    Какая тепловая энергия выделится на резисторе за время
    $\tau = 10\,\text{с}$? Какая работа будет совершена ЭДС за это время? Каков знак этой работы? Чему равен КПД цепи?
    Вычислите значения для 2 случаев: $r=0$ и $r = 10\,\text{Ом}$.
}
\answer{%
    \begin{align*}
    \eli_1 &= \frac{\ele}{R} = \frac{4\,\text{В}}{24\,\text{Ом}} = \frac16\units{А} \approx 0{,}17\,\text{А},  \\
    \eli_2 &= \frac{\ele}{R + r} = \frac{4\,\text{В}}{24\,\text{Ом} + 10\,\text{Ом}} = \frac2{17}\units{А} \approx 0{,}12\,\text{А},  \\
    Q_1 &= \eli_1^2R\tau = \sqr{\frac{\ele}{R}} R \tau
            = \sqr{\frac{4\,\text{В}}{24\,\text{Ом}}} \cdot 24\,\text{Ом} \cdot 10\,\text{с} = \frac{20}3\units{Дж} \approx 6{,}667\,\text{Дж},  \\
    Q_2 &= \eli_2^2R\tau = \sqr{\frac{\ele}{R + r}} R \tau
            = \sqr{\frac{4\,\text{В}}{24\,\text{Ом} + 10\,\text{Ом}}} \cdot 24\,\text{Ом} \cdot 10\,\text{с} = \frac{960}{289}\units{Дж} \approx 3{,}322\,\text{Дж},  \\
    A_1 &= q_1\ele = \eli_1\tau\ele = \frac{\ele}{R} \tau \ele
            = \frac{\ele^2 \tau}{R} = \frac{\sqr{4\,\text{В}} \cdot 10\,\text{с}}{24\,\text{Ом}}
            = \frac{20}3\units{Дж} \approx 6{,}667\,\text{Дж}, \text{положительна},  \\
    A_2 &= q_2\ele = \eli_2\tau\ele = \frac{\ele}{R + r} \tau \ele
            = \frac{\ele^2 \tau}{R + r} = \frac{\sqr{4\,\text{В}} \cdot 10\,\text{с}}{24\,\text{Ом} + 10\,\text{Ом}}
            = \frac{80}{17}\units{Дж} \approx 4{,}706\,\text{Дж}, \text{положительна},  \\
    \eta_1 &= \frac{Q_1}{A_1} = \ldots = \frac{R}{R} = 1,  \\
    \eta_2 &= \frac{Q_2}{A_2} = \ldots = \frac{R}{R + r} = \frac{12}{17} \approx 0{,}71.
    \end{align*}
}
\solutionspace{180pt}

\tasknumber{4}%
\task{%
    Лампочки, сопротивления которых $R_1 = 0{,}50\,\text{Ом}$ и $R_2 = 2{,}00\,\text{Ом}$, поочерёдно подключённные к некоторому источнику тока,
    потребляют одинаковую мощность.
    Найти внутреннее сопротивление источника и КПД цепи в каждом случае.
}
\answer{%
    \begin{align*}
        P_1 &= \sqr{\frac{\ele}{R_1 + r}}R_1,
        P_2  = \sqr{\frac{\ele}{R_2 + r}}R_2,
        P_1 = P_2 \implies  \\
        &\implies R_1 \sqr{R_2 + r} = R_2 \sqr{R_1 + r} \implies  \\
        &\implies R_1 R_2^2 + 2 R_1 R_2 r + R_1 r^2 =
                    R_2 R_1^2 + 2 R_2 R_1 r + R_2 r^2  \implies  \\
    &\implies r^2 (R_2 - R_1) = R_2^2 R_2 - R_1^2 R_2 \implies  \\
    &\implies r
            = \sqrt{R_1 R_2 \frac{R_2 - R_1}{R_2 - R_1}}
            = \sqrt{R_1 R_2}
            = \sqrt{0{,}50\,\text{Ом} \cdot 2{,}00\,\text{Ом}}
            = 1{,}0\,\text{Ом}.
            \\
    \eta_1
            &= \frac{R_1}{R_1 + r}
            = \frac{\sqrt{R_1}}{\sqrt{R_1} + \sqrt{R_2}}
            = 0{,}333,  \\
    \eta_2
            &= \frac{R_2}{R_2 + r}
            = \frac{\sqrt{R_2}}{\sqrt{R_2} + \sqrt{R_1}}
            = 0{,}667
    \end{align*}
}
\solutionspace{120pt}

\tasknumber{5}%
\task{%
    Определите ток, протекающий через резистор $R = 18\,\text{Ом}$ и разность потенциалов на нём (см.
    рис.
    на доске),
    если $r_1 = 2\,\text{Ом}$, $r_2 = 2\,\text{Ом}$, $\ele_1 = 60\,\text{В}$, $\ele_2 = 20\,\text{В}$.
}
\answer{%
    Обозначим на рисунке все токи: направление произвольно, но его надо зафиксировать.
    Всего на рисунке 3 контура и 2 узла.
    Поэтому можно записать $3 - 1 = 2$ уравнения законов Кирхгофа для замкнутого контура и $2 - 1 = 1$ — для узлов
    (остальные уравнения тоже можно записать, но они не дадут полезной информации, а будут лишь следствиями уже записанных).

    Отметим на рисунке 2 контура (и не забуем указать направление) и 1 узел (точка «1»ы, выделена жирным).
    Выбор контуров и узлов не критичен: получившаяся система может быть чуть проще или сложнее, но не слишком.

    \begin{tikzpicture}[circuit ee IEC, thick]
        \draw  (0, 0) to [current direction={near end, info=$\eli_1$}] (0, 3)
                to [battery={rotate=-180,info={$\ele_1, r_1$}}]
                (3, 3)
                to [battery={info'={$\ele_2, r_2$}}]
                (6, 3) to [current direction'={near start, info=$\eli_2$}] (6, 0) -- (0, 0)
                (3, 0) to [current direction={near start, info=$\eli$}, resistor={near end, info=$R$}] (3, 3);
        \draw [-{Latex},color=red] (1.2, 1.7) arc [start angle = 135, end angle = -160, radius = 0.6];
        \draw [-{Latex},color=blue] (4.2, 1.7) arc [start angle = 135, end angle = -160, radius = 0.6];
        \node [contact,color=green!71!black] (bottomc) at (3, 0) {};
        \node [below] (bottom) at (3, 0) {$2$};
        \node [above] (top) at (3, 3) {$1$};
    \end{tikzpicture}

    \begin{align*}
        &\begin{cases}
            {\color{red} \ele_1 = \eli_1 r_1 - \eli R}, \\
            {\color{blue} -\ele_2 = -\eli_2 r_2 + \eli R}, \\
            {\color{green!71!black} - \eli - \eli_1 - \eli_2 = 0 };
        \end{cases}
        \qquad \implies \qquad
        \begin{cases}
            \eli_1 = \frac{\ele_1 + \eli R}{r_1}, \\
            \eli_2 = \frac{\ele_2 + \eli R}{r_2}, \\
            \eli + \eli_1 + \eli_2 = 0;
        \end{cases} \implies \\
        &\implies
         \eli + \frac{\ele_1 + \eli R}{r_1:L} + \frac{\ele_2 + \eli R}{r_2:L} = 0, \\
        &\eli\cbr{ 1 + \frac R{r_1:L} + \frac R{r_2:L}} + \frac{\ele_1 }{r_1:L} + \frac{\ele_2 }{r_2:L} = 0, \\
        &\eli
            = - \frac{\frac{\ele_1 }{r_1:L} + \frac{\ele_2 }{r_2:L}}{ 1 + \frac R{r_1:L} + \frac R{r_2:L}}
            = - \frac{\frac{60\,\text{В}}{2\,\text{Ом}} + \frac{20\,\text{В}}{2\,\text{Ом}}}{ 1 + \frac{18\,\text{Ом}}{2\,\text{Ом}} + \frac{18\,\text{Ом}}{2\,\text{Ом}}}
            = - \frac{40}{19}\units{А}
            \approx -2{,}10\,\text{А}, \\
        &U  = \varphi_2 - \varphi_1 = \eli R
            = - \frac{\frac{\ele_1 }{r_1:L} + \frac{\ele_2 }{r_2:L}}{ 1 + \frac R{r_1:L} + \frac R{r_2:L}} R
            \approx -37{,}90\,\text{В}.
    \end{align*}
    Оба ответа отрицательны, потому что мы изначально «не угадали» с направлением тока.
    Расчёт же показал,
    что ток через резистор $R$ течёт в противоположную сторону: вниз на рисунке, а потенциал точки 1 больше потенциала точки 2,
    а электрический ток ожидаемо течёт из точки с большим потенциалов в точку с меньшим.

    Кстати, если продолжить расчёт и вычислить значения ещё двух токов (формулы для $\eli_1$ и $\eli_2$, куда подставлять, выписаны выше),
    то по их знакам можно будет понять: угадали ли мы с их направлением или нет.
}

\variantsplitter

\addpersonalvariant{Виктор Жилин}

\tasknumber{1}%
\task{%
    На резистор сопротивлением $R = 12\,\text{Ом}$ подали напряжение $U = 240\,\text{В}$.
    Определите ток, который потечёт через резистор, и мощность, выделяющуюся на нём.
}
\answer{%
    \begin{align*}
    \eli &= \frac{U}{R} = \frac{240\,\text{В}}{12\,\text{Ом}} = 20{,}00\,\text{А},  \\
    P &= \frac{U^2}{R} = \frac{\sqr{240\,\text{В}}}{12\,\text{Ом}} = 4800{,}00\,\text{Вт}
    \end{align*}
}
\solutionspace{60pt}

\tasknumber{2}%
\task{%
    Через резистор сопротивлением $R = 18\,\text{Ом}$ протекает электрический ток $\eli = 2{,}00\,\text{А}$.
    Определите, чему равны напряжение на резисторе и мощность, выделяющаяся на нём.
}
\answer{%
    \begin{align*}
    U &= \eli R = 2{,}00\,\text{А} \cdot 18\,\text{Ом} = 36\,\text{В},  \\
    P &= \eli^2R = \sqr{2{,}00\,\text{А}} \cdot 18\,\text{Ом} = 72\,\text{Вт}
    \end{align*}
}
\solutionspace{60pt}

\tasknumber{3}%
\task{%
    Замкнутая электрическая цепь состоит из ЭДС $\ele = 4\,\text{В}$ и сопротивлением $r$
    и резистора $R = 24\,\text{Ом}$.
    Определите ток, протекающий в цепи.
    Какая тепловая энергия выделится на резисторе за время
    $\tau = 2\,\text{с}$? Какая работа будет совершена ЭДС за это время? Каков знак этой работы? Чему равен КПД цепи?
    Вычислите значения для 2 случаев: $r=0$ и $r = 60\,\text{Ом}$.
}
\answer{%
    \begin{align*}
    \eli_1 &= \frac{\ele}{R} = \frac{4\,\text{В}}{24\,\text{Ом}} = \frac16\units{А} \approx 0{,}17\,\text{А},  \\
    \eli_2 &= \frac{\ele}{R + r} = \frac{4\,\text{В}}{24\,\text{Ом} + 60\,\text{Ом}} = \frac1{21}\units{А} \approx 0{,}05\,\text{А},  \\
    Q_1 &= \eli_1^2R\tau = \sqr{\frac{\ele}{R}} R \tau
            = \sqr{\frac{4\,\text{В}}{24\,\text{Ом}}} \cdot 24\,\text{Ом} \cdot 2\,\text{с} = \frac43\units{Дж} \approx 1{,}333\,\text{Дж},  \\
    Q_2 &= \eli_2^2R\tau = \sqr{\frac{\ele}{R + r}} R \tau
            = \sqr{\frac{4\,\text{В}}{24\,\text{Ом} + 60\,\text{Ом}}} \cdot 24\,\text{Ом} \cdot 2\,\text{с} = \frac{16}{147}\units{Дж} \approx 0{,}109\,\text{Дж},  \\
    A_1 &= q_1\ele = \eli_1\tau\ele = \frac{\ele}{R} \tau \ele
            = \frac{\ele^2 \tau}{R} = \frac{\sqr{4\,\text{В}} \cdot 2\,\text{с}}{24\,\text{Ом}}
            = \frac43\units{Дж} \approx 1{,}333\,\text{Дж}, \text{положительна},  \\
    A_2 &= q_2\ele = \eli_2\tau\ele = \frac{\ele}{R + r} \tau \ele
            = \frac{\ele^2 \tau}{R + r} = \frac{\sqr{4\,\text{В}} \cdot 2\,\text{с}}{24\,\text{Ом} + 60\,\text{Ом}}
            = \frac8{21}\units{Дж} \approx 0{,}381\,\text{Дж}, \text{положительна},  \\
    \eta_1 &= \frac{Q_1}{A_1} = \ldots = \frac{R}{R} = 1,  \\
    \eta_2 &= \frac{Q_2}{A_2} = \ldots = \frac{R}{R + r} = \frac27 \approx 0{,}29.
    \end{align*}
}
\solutionspace{180pt}

\tasknumber{4}%
\task{%
    Лампочки, сопротивления которых $R_1 = 0{,}50\,\text{Ом}$ и $R_2 = 18{,}00\,\text{Ом}$, поочерёдно подключённные к некоторому источнику тока,
    потребляют одинаковую мощность.
    Найти внутреннее сопротивление источника и КПД цепи в каждом случае.
}
\answer{%
    \begin{align*}
        P_1 &= \sqr{\frac{\ele}{R_1 + r}}R_1,
        P_2  = \sqr{\frac{\ele}{R_2 + r}}R_2,
        P_1 = P_2 \implies  \\
        &\implies R_1 \sqr{R_2 + r} = R_2 \sqr{R_1 + r} \implies  \\
        &\implies R_1 R_2^2 + 2 R_1 R_2 r + R_1 r^2 =
                    R_2 R_1^2 + 2 R_2 R_1 r + R_2 r^2  \implies  \\
    &\implies r^2 (R_2 - R_1) = R_2^2 R_2 - R_1^2 R_2 \implies  \\
    &\implies r
            = \sqrt{R_1 R_2 \frac{R_2 - R_1}{R_2 - R_1}}
            = \sqrt{R_1 R_2}
            = \sqrt{0{,}50\,\text{Ом} \cdot 18{,}00\,\text{Ом}}
            = 3{,}0\,\text{Ом}.
            \\
    \eta_1
            &= \frac{R_1}{R_1 + r}
            = \frac{\sqrt{R_1}}{\sqrt{R_1} + \sqrt{R_2}}
            = 0{,}143,  \\
    \eta_2
            &= \frac{R_2}{R_2 + r}
            = \frac{\sqrt{R_2}}{\sqrt{R_2} + \sqrt{R_1}}
            = 0{,}857
    \end{align*}
}
\solutionspace{120pt}

\tasknumber{5}%
\task{%
    Определите ток, протекающий через резистор $R = 20\,\text{Ом}$ и разность потенциалов на нём (см.
    рис.
    на доске),
    если $r_1 = 3\,\text{Ом}$, $r_2 = 2\,\text{Ом}$, $\ele_1 = 20\,\text{В}$, $\ele_2 = 40\,\text{В}$.
}
\answer{%
    Обозначим на рисунке все токи: направление произвольно, но его надо зафиксировать.
    Всего на рисунке 3 контура и 2 узла.
    Поэтому можно записать $3 - 1 = 2$ уравнения законов Кирхгофа для замкнутого контура и $2 - 1 = 1$ — для узлов
    (остальные уравнения тоже можно записать, но они не дадут полезной информации, а будут лишь следствиями уже записанных).

    Отметим на рисунке 2 контура (и не забуем указать направление) и 1 узел (точка «1»ы, выделена жирным).
    Выбор контуров и узлов не критичен: получившаяся система может быть чуть проще или сложнее, но не слишком.

    \begin{tikzpicture}[circuit ee IEC, thick]
        \draw  (0, 0) to [current direction={near end, info=$\eli_1$}] (0, 3)
                to [battery={rotate=-180,info={$\ele_1, r_1$}}]
                (3, 3)
                to [battery={info'={$\ele_2, r_2$}}]
                (6, 3) to [current direction'={near start, info=$\eli_2$}] (6, 0) -- (0, 0)
                (3, 0) to [current direction={near start, info=$\eli$}, resistor={near end, info=$R$}] (3, 3);
        \draw [-{Latex},color=red] (1.2, 1.7) arc [start angle = 135, end angle = -160, radius = 0.6];
        \draw [-{Latex},color=blue] (4.2, 1.7) arc [start angle = 135, end angle = -160, radius = 0.6];
        \node [contact,color=green!71!black] (bottomc) at (3, 0) {};
        \node [below] (bottom) at (3, 0) {$2$};
        \node [above] (top) at (3, 3) {$1$};
    \end{tikzpicture}

    \begin{align*}
        &\begin{cases}
            {\color{red} \ele_1 = \eli_1 r_1 - \eli R}, \\
            {\color{blue} -\ele_2 = -\eli_2 r_2 + \eli R}, \\
            {\color{green!71!black} - \eli - \eli_1 - \eli_2 = 0 };
        \end{cases}
        \qquad \implies \qquad
        \begin{cases}
            \eli_1 = \frac{\ele_1 + \eli R}{r_1}, \\
            \eli_2 = \frac{\ele_2 + \eli R}{r_2}, \\
            \eli + \eli_1 + \eli_2 = 0;
        \end{cases} \implies \\
        &\implies
         \eli + \frac{\ele_1 + \eli R}{r_1:L} + \frac{\ele_2 + \eli R}{r_2:L} = 0, \\
        &\eli\cbr{ 1 + \frac R{r_1:L} + \frac R{r_2:L}} + \frac{\ele_1 }{r_1:L} + \frac{\ele_2 }{r_2:L} = 0, \\
        &\eli
            = - \frac{\frac{\ele_1 }{r_1:L} + \frac{\ele_2 }{r_2:L}}{ 1 + \frac R{r_1:L} + \frac R{r_2:L}}
            = - \frac{\frac{20\,\text{В}}{3\,\text{Ом}} + \frac{40\,\text{В}}{2\,\text{Ом}}}{ 1 + \frac{20\,\text{Ом}}{3\,\text{Ом}} + \frac{20\,\text{Ом}}{2\,\text{Ом}}}
            = - \frac{80}{53}\units{А}
            \approx -1{,}500\,\text{А}, \\
        &U  = \varphi_2 - \varphi_1 = \eli R
            = - \frac{\frac{\ele_1 }{r_1:L} + \frac{\ele_2 }{r_2:L}}{ 1 + \frac R{r_1:L} + \frac R{r_2:L}} R
            \approx -30{,}20\,\text{В}.
    \end{align*}
    Оба ответа отрицательны, потому что мы изначально «не угадали» с направлением тока.
    Расчёт же показал,
    что ток через резистор $R$ течёт в противоположную сторону: вниз на рисунке, а потенциал точки 1 больше потенциала точки 2,
    а электрический ток ожидаемо течёт из точки с большим потенциалов в точку с меньшим.

    Кстати, если продолжить расчёт и вычислить значения ещё двух токов (формулы для $\eli_1$ и $\eli_2$, куда подставлять, выписаны выше),
    то по их знакам можно будет понять: угадали ли мы с их направлением или нет.
}

\variantsplitter

\addpersonalvariant{Дмитрий Иванов}

\tasknumber{1}%
\task{%
    На резистор сопротивлением $r = 5\,\text{Ом}$ подали напряжение $V = 180\,\text{В}$.
    Определите ток, который потечёт через резистор, и мощность, выделяющуюся на нём.
}
\answer{%
    \begin{align*}
    \eli &= \frac{V}{r} = \frac{180\,\text{В}}{5\,\text{Ом}} = 36{,}00\,\text{А},  \\
    P &= \frac{V^2}{r} = \frac{\sqr{180\,\text{В}}}{5\,\text{Ом}} = 6480{,}00\,\text{Вт}
    \end{align*}
}
\solutionspace{60pt}

\tasknumber{2}%
\task{%
    Через резистор сопротивлением $r = 12\,\text{Ом}$ протекает электрический ток $\eli = 10{,}00\,\text{А}$.
    Определите, чему равны напряжение на резисторе и мощность, выделяющаяся на нём.
}
\answer{%
    \begin{align*}
    U &= \eli r = 10{,}00\,\text{А} \cdot 12\,\text{Ом} = 120\,\text{В},  \\
    P &= \eli^2r = \sqr{10{,}00\,\text{А}} \cdot 12\,\text{Ом} = 1200\,\text{Вт}
    \end{align*}
}
\solutionspace{60pt}

\tasknumber{3}%
\task{%
    Замкнутая электрическая цепь состоит из ЭДС $\ele = 4\,\text{В}$ и сопротивлением $r$
    и резистора $R = 15\,\text{Ом}$.
    Определите ток, протекающий в цепи.
    Какая тепловая энергия выделится на резисторе за время
    $\tau = 10\,\text{с}$? Какая работа будет совершена ЭДС за это время? Каков знак этой работы? Чему равен КПД цепи?
    Вычислите значения для 2 случаев: $r=0$ и $r = 20\,\text{Ом}$.
}
\answer{%
    \begin{align*}
    \eli_1 &= \frac{\ele}{R} = \frac{4\,\text{В}}{15\,\text{Ом}} = \frac4{15}\units{А} \approx 0{,}27\,\text{А},  \\
    \eli_2 &= \frac{\ele}{R + r} = \frac{4\,\text{В}}{15\,\text{Ом} + 20\,\text{Ом}} = \frac4{35}\units{А} \approx 0{,}11\,\text{А},  \\
    Q_1 &= \eli_1^2R\tau = \sqr{\frac{\ele}{R}} R \tau
            = \sqr{\frac{4\,\text{В}}{15\,\text{Ом}}} \cdot 15\,\text{Ом} \cdot 10\,\text{с} = \frac{32}3\units{Дж} \approx 10{,}667\,\text{Дж},  \\
    Q_2 &= \eli_2^2R\tau = \sqr{\frac{\ele}{R + r}} R \tau
            = \sqr{\frac{4\,\text{В}}{15\,\text{Ом} + 20\,\text{Ом}}} \cdot 15\,\text{Ом} \cdot 10\,\text{с} = \frac{96}{49}\units{Дж} \approx 1{,}959\,\text{Дж},  \\
    A_1 &= q_1\ele = \eli_1\tau\ele = \frac{\ele}{R} \tau \ele
            = \frac{\ele^2 \tau}{R} = \frac{\sqr{4\,\text{В}} \cdot 10\,\text{с}}{15\,\text{Ом}}
            = \frac{32}3\units{Дж} \approx 10{,}667\,\text{Дж}, \text{положительна},  \\
    A_2 &= q_2\ele = \eli_2\tau\ele = \frac{\ele}{R + r} \tau \ele
            = \frac{\ele^2 \tau}{R + r} = \frac{\sqr{4\,\text{В}} \cdot 10\,\text{с}}{15\,\text{Ом} + 20\,\text{Ом}}
            = \frac{32}7\units{Дж} \approx 4{,}571\,\text{Дж}, \text{положительна},  \\
    \eta_1 &= \frac{Q_1}{A_1} = \ldots = \frac{R}{R} = 1,  \\
    \eta_2 &= \frac{Q_2}{A_2} = \ldots = \frac{R}{R + r} = \frac37 \approx 0{,}43.
    \end{align*}
}
\solutionspace{180pt}

\tasknumber{4}%
\task{%
    Лампочки, сопротивления которых $R_1 = 5{,}00\,\text{Ом}$ и $R_2 = 80{,}00\,\text{Ом}$, поочерёдно подключённные к некоторому источнику тока,
    потребляют одинаковую мощность.
    Найти внутреннее сопротивление источника и КПД цепи в каждом случае.
}
\answer{%
    \begin{align*}
        P_1 &= \sqr{\frac{\ele}{R_1 + r}}R_1,
        P_2  = \sqr{\frac{\ele}{R_2 + r}}R_2,
        P_1 = P_2 \implies  \\
        &\implies R_1 \sqr{R_2 + r} = R_2 \sqr{R_1 + r} \implies  \\
        &\implies R_1 R_2^2 + 2 R_1 R_2 r + R_1 r^2 =
                    R_2 R_1^2 + 2 R_2 R_1 r + R_2 r^2  \implies  \\
    &\implies r^2 (R_2 - R_1) = R_2^2 R_2 - R_1^2 R_2 \implies  \\
    &\implies r
            = \sqrt{R_1 R_2 \frac{R_2 - R_1}{R_2 - R_1}}
            = \sqrt{R_1 R_2}
            = \sqrt{5{,}00\,\text{Ом} \cdot 80{,}00\,\text{Ом}}
            = 20{,}0\,\text{Ом}.
            \\
    \eta_1
            &= \frac{R_1}{R_1 + r}
            = \frac{\sqrt{R_1}}{\sqrt{R_1} + \sqrt{R_2}}
            = 0{,}200,  \\
    \eta_2
            &= \frac{R_2}{R_2 + r}
            = \frac{\sqrt{R_2}}{\sqrt{R_2} + \sqrt{R_1}}
            = 0{,}800
    \end{align*}
}
\solutionspace{120pt}

\tasknumber{5}%
\task{%
    Определите ток, протекающий через резистор $R = 10\,\text{Ом}$ и разность потенциалов на нём (см.
    рис.
    на доске),
    если $r_1 = 2\,\text{Ом}$, $r_2 = 1\,\text{Ом}$, $\ele_1 = 60\,\text{В}$, $\ele_2 = 20\,\text{В}$.
}
\answer{%
    Обозначим на рисунке все токи: направление произвольно, но его надо зафиксировать.
    Всего на рисунке 3 контура и 2 узла.
    Поэтому можно записать $3 - 1 = 2$ уравнения законов Кирхгофа для замкнутого контура и $2 - 1 = 1$ — для узлов
    (остальные уравнения тоже можно записать, но они не дадут полезной информации, а будут лишь следствиями уже записанных).

    Отметим на рисунке 2 контура (и не забуем указать направление) и 1 узел (точка «1»ы, выделена жирным).
    Выбор контуров и узлов не критичен: получившаяся система может быть чуть проще или сложнее, но не слишком.

    \begin{tikzpicture}[circuit ee IEC, thick]
        \draw  (0, 0) to [current direction={near end, info=$\eli_1$}] (0, 3)
                to [battery={rotate=-180,info={$\ele_1, r_1$}}]
                (3, 3)
                to [battery={info'={$\ele_2, r_2$}}]
                (6, 3) to [current direction'={near start, info=$\eli_2$}] (6, 0) -- (0, 0)
                (3, 0) to [current direction={near start, info=$\eli$}, resistor={near end, info=$R$}] (3, 3);
        \draw [-{Latex},color=red] (1.2, 1.7) arc [start angle = 135, end angle = -160, radius = 0.6];
        \draw [-{Latex},color=blue] (4.2, 1.7) arc [start angle = 135, end angle = -160, radius = 0.6];
        \node [contact,color=green!71!black] (bottomc) at (3, 0) {};
        \node [below] (bottom) at (3, 0) {$2$};
        \node [above] (top) at (3, 3) {$1$};
    \end{tikzpicture}

    \begin{align*}
        &\begin{cases}
            {\color{red} \ele_1 = \eli_1 r_1 - \eli R}, \\
            {\color{blue} -\ele_2 = -\eli_2 r_2 + \eli R}, \\
            {\color{green!71!black} - \eli - \eli_1 - \eli_2 = 0 };
        \end{cases}
        \qquad \implies \qquad
        \begin{cases}
            \eli_1 = \frac{\ele_1 + \eli R}{r_1}, \\
            \eli_2 = \frac{\ele_2 + \eli R}{r_2}, \\
            \eli + \eli_1 + \eli_2 = 0;
        \end{cases} \implies \\
        &\implies
         \eli + \frac{\ele_1 + \eli R}{r_1:L} + \frac{\ele_2 + \eli R}{r_2:L} = 0, \\
        &\eli\cbr{ 1 + \frac R{r_1:L} + \frac R{r_2:L}} + \frac{\ele_1 }{r_1:L} + \frac{\ele_2 }{r_2:L} = 0, \\
        &\eli
            = - \frac{\frac{\ele_1 }{r_1:L} + \frac{\ele_2 }{r_2:L}}{ 1 + \frac R{r_1:L} + \frac R{r_2:L}}
            = - \frac{\frac{60\,\text{В}}{2\,\text{Ом}} + \frac{20\,\text{В}}{1\,\text{Ом}}}{ 1 + \frac{10\,\text{Ом}}{2\,\text{Ом}} + \frac{10\,\text{Ом}}{1\,\text{Ом}}}
            = - \frac{25}8\units{А}
            \approx -3{,}10\,\text{А}, \\
        &U  = \varphi_2 - \varphi_1 = \eli R
            = - \frac{\frac{\ele_1 }{r_1:L} + \frac{\ele_2 }{r_2:L}}{ 1 + \frac R{r_1:L} + \frac R{r_2:L}} R
            \approx -31{,}20\,\text{В}.
    \end{align*}
    Оба ответа отрицательны, потому что мы изначально «не угадали» с направлением тока.
    Расчёт же показал,
    что ток через резистор $R$ течёт в противоположную сторону: вниз на рисунке, а потенциал точки 1 больше потенциала точки 2,
    а электрический ток ожидаемо течёт из точки с большим потенциалов в точку с меньшим.

    Кстати, если продолжить расчёт и вычислить значения ещё двух токов (формулы для $\eli_1$ и $\eli_2$, куда подставлять, выписаны выше),
    то по их знакам можно будет понять: угадали ли мы с их направлением или нет.
}

\variantsplitter

\addpersonalvariant{Олег Климов}

\tasknumber{1}%
\task{%
    На резистор сопротивлением $r = 30\,\text{Ом}$ подали напряжение $V = 240\,\text{В}$.
    Определите ток, который потечёт через резистор, и мощность, выделяющуюся на нём.
}
\answer{%
    \begin{align*}
    \eli &= \frac{V}{r} = \frac{240\,\text{В}}{30\,\text{Ом}} = 8{,}00\,\text{А},  \\
    P &= \frac{V^2}{r} = \frac{\sqr{240\,\text{В}}}{30\,\text{Ом}} = 1920{,}00\,\text{Вт}
    \end{align*}
}
\solutionspace{60pt}

\tasknumber{2}%
\task{%
    Через резистор сопротивлением $R = 18\,\text{Ом}$ протекает электрический ток $\eli = 15{,}00\,\text{А}$.
    Определите, чему равны напряжение на резисторе и мощность, выделяющаяся на нём.
}
\answer{%
    \begin{align*}
    U &= \eli R = 15{,}00\,\text{А} \cdot 18\,\text{Ом} = 270\,\text{В},  \\
    P &= \eli^2R = \sqr{15{,}00\,\text{А}} \cdot 18\,\text{Ом} = 4050\,\text{Вт}
    \end{align*}
}
\solutionspace{60pt}

\tasknumber{3}%
\task{%
    Замкнутая электрическая цепь состоит из ЭДС $\ele = 4\,\text{В}$ и сопротивлением $r$
    и резистора $R = 15\,\text{Ом}$.
    Определите ток, протекающий в цепи.
    Какая тепловая энергия выделится на резисторе за время
    $\tau = 10\,\text{с}$? Какая работа будет совершена ЭДС за это время? Каков знак этой работы? Чему равен КПД цепи?
    Вычислите значения для 2 случаев: $r=0$ и $r = 20\,\text{Ом}$.
}
\answer{%
    \begin{align*}
    \eli_1 &= \frac{\ele}{R} = \frac{4\,\text{В}}{15\,\text{Ом}} = \frac4{15}\units{А} \approx 0{,}27\,\text{А},  \\
    \eli_2 &= \frac{\ele}{R + r} = \frac{4\,\text{В}}{15\,\text{Ом} + 20\,\text{Ом}} = \frac4{35}\units{А} \approx 0{,}11\,\text{А},  \\
    Q_1 &= \eli_1^2R\tau = \sqr{\frac{\ele}{R}} R \tau
            = \sqr{\frac{4\,\text{В}}{15\,\text{Ом}}} \cdot 15\,\text{Ом} \cdot 10\,\text{с} = \frac{32}3\units{Дж} \approx 10{,}667\,\text{Дж},  \\
    Q_2 &= \eli_2^2R\tau = \sqr{\frac{\ele}{R + r}} R \tau
            = \sqr{\frac{4\,\text{В}}{15\,\text{Ом} + 20\,\text{Ом}}} \cdot 15\,\text{Ом} \cdot 10\,\text{с} = \frac{96}{49}\units{Дж} \approx 1{,}959\,\text{Дж},  \\
    A_1 &= q_1\ele = \eli_1\tau\ele = \frac{\ele}{R} \tau \ele
            = \frac{\ele^2 \tau}{R} = \frac{\sqr{4\,\text{В}} \cdot 10\,\text{с}}{15\,\text{Ом}}
            = \frac{32}3\units{Дж} \approx 10{,}667\,\text{Дж}, \text{положительна},  \\
    A_2 &= q_2\ele = \eli_2\tau\ele = \frac{\ele}{R + r} \tau \ele
            = \frac{\ele^2 \tau}{R + r} = \frac{\sqr{4\,\text{В}} \cdot 10\,\text{с}}{15\,\text{Ом} + 20\,\text{Ом}}
            = \frac{32}7\units{Дж} \approx 4{,}571\,\text{Дж}, \text{положительна},  \\
    \eta_1 &= \frac{Q_1}{A_1} = \ldots = \frac{R}{R} = 1,  \\
    \eta_2 &= \frac{Q_2}{A_2} = \ldots = \frac{R}{R + r} = \frac37 \approx 0{,}43.
    \end{align*}
}
\solutionspace{180pt}

\tasknumber{4}%
\task{%
    Лампочки, сопротивления которых $R_1 = 4{,}00\,\text{Ом}$ и $R_2 = 36{,}00\,\text{Ом}$, поочерёдно подключённные к некоторому источнику тока,
    потребляют одинаковую мощность.
    Найти внутреннее сопротивление источника и КПД цепи в каждом случае.
}
\answer{%
    \begin{align*}
        P_1 &= \sqr{\frac{\ele}{R_1 + r}}R_1,
        P_2  = \sqr{\frac{\ele}{R_2 + r}}R_2,
        P_1 = P_2 \implies  \\
        &\implies R_1 \sqr{R_2 + r} = R_2 \sqr{R_1 + r} \implies  \\
        &\implies R_1 R_2^2 + 2 R_1 R_2 r + R_1 r^2 =
                    R_2 R_1^2 + 2 R_2 R_1 r + R_2 r^2  \implies  \\
    &\implies r^2 (R_2 - R_1) = R_2^2 R_2 - R_1^2 R_2 \implies  \\
    &\implies r
            = \sqrt{R_1 R_2 \frac{R_2 - R_1}{R_2 - R_1}}
            = \sqrt{R_1 R_2}
            = \sqrt{4{,}00\,\text{Ом} \cdot 36{,}00\,\text{Ом}}
            = 12{,}0\,\text{Ом}.
            \\
    \eta_1
            &= \frac{R_1}{R_1 + r}
            = \frac{\sqrt{R_1}}{\sqrt{R_1} + \sqrt{R_2}}
            = 0{,}250,  \\
    \eta_2
            &= \frac{R_2}{R_2 + r}
            = \frac{\sqrt{R_2}}{\sqrt{R_2} + \sqrt{R_1}}
            = 0{,}750
    \end{align*}
}
\solutionspace{120pt}

\tasknumber{5}%
\task{%
    Определите ток, протекающий через резистор $R = 15\,\text{Ом}$ и разность потенциалов на нём (см.
    рис.
    на доске),
    если $r_1 = 2\,\text{Ом}$, $r_2 = 2\,\text{Ом}$, $\ele_1 = 60\,\text{В}$, $\ele_2 = 40\,\text{В}$.
}
\answer{%
    Обозначим на рисунке все токи: направление произвольно, но его надо зафиксировать.
    Всего на рисунке 3 контура и 2 узла.
    Поэтому можно записать $3 - 1 = 2$ уравнения законов Кирхгофа для замкнутого контура и $2 - 1 = 1$ — для узлов
    (остальные уравнения тоже можно записать, но они не дадут полезной информации, а будут лишь следствиями уже записанных).

    Отметим на рисунке 2 контура (и не забуем указать направление) и 1 узел (точка «1»ы, выделена жирным).
    Выбор контуров и узлов не критичен: получившаяся система может быть чуть проще или сложнее, но не слишком.

    \begin{tikzpicture}[circuit ee IEC, thick]
        \draw  (0, 0) to [current direction={near end, info=$\eli_1$}] (0, 3)
                to [battery={rotate=-180,info={$\ele_1, r_1$}}]
                (3, 3)
                to [battery={info'={$\ele_2, r_2$}}]
                (6, 3) to [current direction'={near start, info=$\eli_2$}] (6, 0) -- (0, 0)
                (3, 0) to [current direction={near start, info=$\eli$}, resistor={near end, info=$R$}] (3, 3);
        \draw [-{Latex},color=red] (1.2, 1.7) arc [start angle = 135, end angle = -160, radius = 0.6];
        \draw [-{Latex},color=blue] (4.2, 1.7) arc [start angle = 135, end angle = -160, radius = 0.6];
        \node [contact,color=green!71!black] (bottomc) at (3, 0) {};
        \node [below] (bottom) at (3, 0) {$2$};
        \node [above] (top) at (3, 3) {$1$};
    \end{tikzpicture}

    \begin{align*}
        &\begin{cases}
            {\color{red} \ele_1 = \eli_1 r_1 - \eli R}, \\
            {\color{blue} -\ele_2 = -\eli_2 r_2 + \eli R}, \\
            {\color{green!71!black} - \eli - \eli_1 - \eli_2 = 0 };
        \end{cases}
        \qquad \implies \qquad
        \begin{cases}
            \eli_1 = \frac{\ele_1 + \eli R}{r_1}, \\
            \eli_2 = \frac{\ele_2 + \eli R}{r_2}, \\
            \eli + \eli_1 + \eli_2 = 0;
        \end{cases} \implies \\
        &\implies
         \eli + \frac{\ele_1 + \eli R}{r_1:L} + \frac{\ele_2 + \eli R}{r_2:L} = 0, \\
        &\eli\cbr{ 1 + \frac R{r_1:L} + \frac R{r_2:L}} + \frac{\ele_1 }{r_1:L} + \frac{\ele_2 }{r_2:L} = 0, \\
        &\eli
            = - \frac{\frac{\ele_1 }{r_1:L} + \frac{\ele_2 }{r_2:L}}{ 1 + \frac R{r_1:L} + \frac R{r_2:L}}
            = - \frac{\frac{60\,\text{В}}{2\,\text{Ом}} + \frac{40\,\text{В}}{2\,\text{Ом}}}{ 1 + \frac{15\,\text{Ом}}{2\,\text{Ом}} + \frac{15\,\text{Ом}}{2\,\text{Ом}}}
            = - \frac{25}8\units{А}
            \approx -3{,}10\,\text{А}, \\
        &U  = \varphi_2 - \varphi_1 = \eli R
            = - \frac{\frac{\ele_1 }{r_1:L} + \frac{\ele_2 }{r_2:L}}{ 1 + \frac R{r_1:L} + \frac R{r_2:L}} R
            \approx -46{,}90\,\text{В}.
    \end{align*}
    Оба ответа отрицательны, потому что мы изначально «не угадали» с направлением тока.
    Расчёт же показал,
    что ток через резистор $R$ течёт в противоположную сторону: вниз на рисунке, а потенциал точки 1 больше потенциала точки 2,
    а электрический ток ожидаемо течёт из точки с большим потенциалов в точку с меньшим.

    Кстати, если продолжить расчёт и вычислить значения ещё двух токов (формулы для $\eli_1$ и $\eli_2$, куда подставлять, выписаны выше),
    то по их знакам можно будет понять: угадали ли мы с их направлением или нет.
}

\variantsplitter

\addpersonalvariant{Анна Ковалева}

\tasknumber{1}%
\task{%
    На резистор сопротивлением $R = 12\,\text{Ом}$ подали напряжение $U = 150\,\text{В}$.
    Определите ток, который потечёт через резистор, и мощность, выделяющуюся на нём.
}
\answer{%
    \begin{align*}
    \eli &= \frac{U}{R} = \frac{150\,\text{В}}{12\,\text{Ом}} = 12{,}50\,\text{А},  \\
    P &= \frac{U^2}{R} = \frac{\sqr{150\,\text{В}}}{12\,\text{Ом}} = 1875{,}00\,\text{Вт}
    \end{align*}
}
\solutionspace{60pt}

\tasknumber{2}%
\task{%
    Через резистор сопротивлением $R = 18\,\text{Ом}$ протекает электрический ток $\eli = 6{,}00\,\text{А}$.
    Определите, чему равны напряжение на резисторе и мощность, выделяющаяся на нём.
}
\answer{%
    \begin{align*}
    U &= \eli R = 6{,}00\,\text{А} \cdot 18\,\text{Ом} = 108\,\text{В},  \\
    P &= \eli^2R = \sqr{6{,}00\,\text{А}} \cdot 18\,\text{Ом} = 648\,\text{Вт}
    \end{align*}
}
\solutionspace{60pt}

\tasknumber{3}%
\task{%
    Замкнутая электрическая цепь состоит из ЭДС $\ele = 1\,\text{В}$ и сопротивлением $r$
    и резистора $R = 30\,\text{Ом}$.
    Определите ток, протекающий в цепи.
    Какая тепловая энергия выделится на резисторе за время
    $\tau = 10\,\text{с}$? Какая работа будет совершена ЭДС за это время? Каков знак этой работы? Чему равен КПД цепи?
    Вычислите значения для 2 случаев: $r=0$ и $r = 10\,\text{Ом}$.
}
\answer{%
    \begin{align*}
    \eli_1 &= \frac{\ele}{R} = \frac{1\,\text{В}}{30\,\text{Ом}} = \frac1{30}\units{А} \approx 0{,}03\,\text{А},  \\
    \eli_2 &= \frac{\ele}{R + r} = \frac{1\,\text{В}}{30\,\text{Ом} + 10\,\text{Ом}} = \frac1{40}\units{А} \approx 0{,}03\,\text{А},  \\
    Q_1 &= \eli_1^2R\tau = \sqr{\frac{\ele}{R}} R \tau
            = \sqr{\frac{1\,\text{В}}{30\,\text{Ом}}} \cdot 30\,\text{Ом} \cdot 10\,\text{с} = \frac13\units{Дж} \approx 0{,}333\,\text{Дж},  \\
    Q_2 &= \eli_2^2R\tau = \sqr{\frac{\ele}{R + r}} R \tau
            = \sqr{\frac{1\,\text{В}}{30\,\text{Ом} + 10\,\text{Ом}}} \cdot 30\,\text{Ом} \cdot 10\,\text{с} = \frac3{16}\units{Дж} \approx 0{,}188\,\text{Дж},  \\
    A_1 &= q_1\ele = \eli_1\tau\ele = \frac{\ele}{R} \tau \ele
            = \frac{\ele^2 \tau}{R} = \frac{\sqr{1\,\text{В}} \cdot 10\,\text{с}}{30\,\text{Ом}}
            = \frac13\units{Дж} \approx 0{,}333\,\text{Дж}, \text{положительна},  \\
    A_2 &= q_2\ele = \eli_2\tau\ele = \frac{\ele}{R + r} \tau \ele
            = \frac{\ele^2 \tau}{R + r} = \frac{\sqr{1\,\text{В}} \cdot 10\,\text{с}}{30\,\text{Ом} + 10\,\text{Ом}}
            = \frac14\units{Дж} \approx 0{,}250\,\text{Дж}, \text{положительна},  \\
    \eta_1 &= \frac{Q_1}{A_1} = \ldots = \frac{R}{R} = 1,  \\
    \eta_2 &= \frac{Q_2}{A_2} = \ldots = \frac{R}{R + r} = \frac34 \approx 0{,}75.
    \end{align*}
}
\solutionspace{180pt}

\tasknumber{4}%
\task{%
    Лампочки, сопротивления которых $R_1 = 1{,}00\,\text{Ом}$ и $R_2 = 49{,}00\,\text{Ом}$, поочерёдно подключённные к некоторому источнику тока,
    потребляют одинаковую мощность.
    Найти внутреннее сопротивление источника и КПД цепи в каждом случае.
}
\answer{%
    \begin{align*}
        P_1 &= \sqr{\frac{\ele}{R_1 + r}}R_1,
        P_2  = \sqr{\frac{\ele}{R_2 + r}}R_2,
        P_1 = P_2 \implies  \\
        &\implies R_1 \sqr{R_2 + r} = R_2 \sqr{R_1 + r} \implies  \\
        &\implies R_1 R_2^2 + 2 R_1 R_2 r + R_1 r^2 =
                    R_2 R_1^2 + 2 R_2 R_1 r + R_2 r^2  \implies  \\
    &\implies r^2 (R_2 - R_1) = R_2^2 R_2 - R_1^2 R_2 \implies  \\
    &\implies r
            = \sqrt{R_1 R_2 \frac{R_2 - R_1}{R_2 - R_1}}
            = \sqrt{R_1 R_2}
            = \sqrt{1{,}00\,\text{Ом} \cdot 49{,}00\,\text{Ом}}
            = 7{,}0\,\text{Ом}.
            \\
    \eta_1
            &= \frac{R_1}{R_1 + r}
            = \frac{\sqrt{R_1}}{\sqrt{R_1} + \sqrt{R_2}}
            = 0{,}125,  \\
    \eta_2
            &= \frac{R_2}{R_2 + r}
            = \frac{\sqrt{R_2}}{\sqrt{R_2} + \sqrt{R_1}}
            = 0{,}875
    \end{align*}
}
\solutionspace{120pt}

\tasknumber{5}%
\task{%
    Определите ток, протекающий через резистор $R = 10\,\text{Ом}$ и разность потенциалов на нём (см.
    рис.
    на доске),
    если $r_1 = 1\,\text{Ом}$, $r_2 = 3\,\text{Ом}$, $\ele_1 = 40\,\text{В}$, $\ele_2 = 20\,\text{В}$.
}
\answer{%
    Обозначим на рисунке все токи: направление произвольно, но его надо зафиксировать.
    Всего на рисунке 3 контура и 2 узла.
    Поэтому можно записать $3 - 1 = 2$ уравнения законов Кирхгофа для замкнутого контура и $2 - 1 = 1$ — для узлов
    (остальные уравнения тоже можно записать, но они не дадут полезной информации, а будут лишь следствиями уже записанных).

    Отметим на рисунке 2 контура (и не забуем указать направление) и 1 узел (точка «1»ы, выделена жирным).
    Выбор контуров и узлов не критичен: получившаяся система может быть чуть проще или сложнее, но не слишком.

    \begin{tikzpicture}[circuit ee IEC, thick]
        \draw  (0, 0) to [current direction={near end, info=$\eli_1$}] (0, 3)
                to [battery={rotate=-180,info={$\ele_1, r_1$}}]
                (3, 3)
                to [battery={info'={$\ele_2, r_2$}}]
                (6, 3) to [current direction'={near start, info=$\eli_2$}] (6, 0) -- (0, 0)
                (3, 0) to [current direction={near start, info=$\eli$}, resistor={near end, info=$R$}] (3, 3);
        \draw [-{Latex},color=red] (1.2, 1.7) arc [start angle = 135, end angle = -160, radius = 0.6];
        \draw [-{Latex},color=blue] (4.2, 1.7) arc [start angle = 135, end angle = -160, radius = 0.6];
        \node [contact,color=green!71!black] (bottomc) at (3, 0) {};
        \node [below] (bottom) at (3, 0) {$2$};
        \node [above] (top) at (3, 3) {$1$};
    \end{tikzpicture}

    \begin{align*}
        &\begin{cases}
            {\color{red} \ele_1 = \eli_1 r_1 - \eli R}, \\
            {\color{blue} -\ele_2 = -\eli_2 r_2 + \eli R}, \\
            {\color{green!71!black} - \eli - \eli_1 - \eli_2 = 0 };
        \end{cases}
        \qquad \implies \qquad
        \begin{cases}
            \eli_1 = \frac{\ele_1 + \eli R}{r_1}, \\
            \eli_2 = \frac{\ele_2 + \eli R}{r_2}, \\
            \eli + \eli_1 + \eli_2 = 0;
        \end{cases} \implies \\
        &\implies
         \eli + \frac{\ele_1 + \eli R}{r_1:L} + \frac{\ele_2 + \eli R}{r_2:L} = 0, \\
        &\eli\cbr{ 1 + \frac R{r_1:L} + \frac R{r_2:L}} + \frac{\ele_1 }{r_1:L} + \frac{\ele_2 }{r_2:L} = 0, \\
        &\eli
            = - \frac{\frac{\ele_1 }{r_1:L} + \frac{\ele_2 }{r_2:L}}{ 1 + \frac R{r_1:L} + \frac R{r_2:L}}
            = - \frac{\frac{40\,\text{В}}{1\,\text{Ом}} + \frac{20\,\text{В}}{3\,\text{Ом}}}{ 1 + \frac{10\,\text{Ом}}{1\,\text{Ом}} + \frac{10\,\text{Ом}}{3\,\text{Ом}}}
            = - \frac{140}{43}\units{А}
            \approx -3{,}30\,\text{А}, \\
        &U  = \varphi_2 - \varphi_1 = \eli R
            = - \frac{\frac{\ele_1 }{r_1:L} + \frac{\ele_2 }{r_2:L}}{ 1 + \frac R{r_1:L} + \frac R{r_2:L}} R
            \approx -32{,}60\,\text{В}.
    \end{align*}
    Оба ответа отрицательны, потому что мы изначально «не угадали» с направлением тока.
    Расчёт же показал,
    что ток через резистор $R$ течёт в противоположную сторону: вниз на рисунке, а потенциал точки 1 больше потенциала точки 2,
    а электрический ток ожидаемо течёт из точки с большим потенциалов в точку с меньшим.

    Кстати, если продолжить расчёт и вычислить значения ещё двух токов (формулы для $\eli_1$ и $\eli_2$, куда подставлять, выписаны выше),
    то по их знакам можно будет понять: угадали ли мы с их направлением или нет.
}

\variantsplitter

\addpersonalvariant{Глеб Ковылин}

\tasknumber{1}%
\task{%
    На резистор сопротивлением $r = 30\,\text{Ом}$ подали напряжение $U = 240\,\text{В}$.
    Определите ток, который потечёт через резистор, и мощность, выделяющуюся на нём.
}
\answer{%
    \begin{align*}
    \eli &= \frac{U}{r} = \frac{240\,\text{В}}{30\,\text{Ом}} = 8{,}00\,\text{А},  \\
    P &= \frac{U^2}{r} = \frac{\sqr{240\,\text{В}}}{30\,\text{Ом}} = 1920{,}00\,\text{Вт}
    \end{align*}
}
\solutionspace{60pt}

\tasknumber{2}%
\task{%
    Через резистор сопротивлением $r = 5\,\text{Ом}$ протекает электрический ток $\eli = 8{,}00\,\text{А}$.
    Определите, чему равны напряжение на резисторе и мощность, выделяющаяся на нём.
}
\answer{%
    \begin{align*}
    U &= \eli r = 8{,}00\,\text{А} \cdot 5\,\text{Ом} = 40\,\text{В},  \\
    P &= \eli^2r = \sqr{8{,}00\,\text{А}} \cdot 5\,\text{Ом} = 320\,\text{Вт}
    \end{align*}
}
\solutionspace{60pt}

\tasknumber{3}%
\task{%
    Замкнутая электрическая цепь состоит из ЭДС $\ele = 4\,\text{В}$ и сопротивлением $r$
    и резистора $R = 10\,\text{Ом}$.
    Определите ток, протекающий в цепи.
    Какая тепловая энергия выделится на резисторе за время
    $\tau = 10\,\text{с}$? Какая работа будет совершена ЭДС за это время? Каков знак этой работы? Чему равен КПД цепи?
    Вычислите значения для 2 случаев: $r=0$ и $r = 60\,\text{Ом}$.
}
\answer{%
    \begin{align*}
    \eli_1 &= \frac{\ele}{R} = \frac{4\,\text{В}}{10\,\text{Ом}} = \frac25\units{А} \approx 0{,}40\,\text{А},  \\
    \eli_2 &= \frac{\ele}{R + r} = \frac{4\,\text{В}}{10\,\text{Ом} + 60\,\text{Ом}} = \frac2{35}\units{А} \approx 0{,}06\,\text{А},  \\
    Q_1 &= \eli_1^2R\tau = \sqr{\frac{\ele}{R}} R \tau
            = \sqr{\frac{4\,\text{В}}{10\,\text{Ом}}} \cdot 10\,\text{Ом} \cdot 10\,\text{с} = 16\units{Дж} \approx 16{,}000\,\text{Дж},  \\
    Q_2 &= \eli_2^2R\tau = \sqr{\frac{\ele}{R + r}} R \tau
            = \sqr{\frac{4\,\text{В}}{10\,\text{Ом} + 60\,\text{Ом}}} \cdot 10\,\text{Ом} \cdot 10\,\text{с} = \frac{16}{49}\units{Дж} \approx 0{,}327\,\text{Дж},  \\
    A_1 &= q_1\ele = \eli_1\tau\ele = \frac{\ele}{R} \tau \ele
            = \frac{\ele^2 \tau}{R} = \frac{\sqr{4\,\text{В}} \cdot 10\,\text{с}}{10\,\text{Ом}}
            = 16\units{Дж} \approx 16{,}000\,\text{Дж}, \text{положительна},  \\
    A_2 &= q_2\ele = \eli_2\tau\ele = \frac{\ele}{R + r} \tau \ele
            = \frac{\ele^2 \tau}{R + r} = \frac{\sqr{4\,\text{В}} \cdot 10\,\text{с}}{10\,\text{Ом} + 60\,\text{Ом}}
            = \frac{16}7\units{Дж} \approx 2{,}286\,\text{Дж}, \text{положительна},  \\
    \eta_1 &= \frac{Q_1}{A_1} = \ldots = \frac{R}{R} = 1,  \\
    \eta_2 &= \frac{Q_2}{A_2} = \ldots = \frac{R}{R + r} = \frac17 \approx 0{,}14.
    \end{align*}
}
\solutionspace{180pt}

\tasknumber{4}%
\task{%
    Лампочки, сопротивления которых $R_1 = 0{,}25\,\text{Ом}$ и $R_2 = 64{,}00\,\text{Ом}$, поочерёдно подключённные к некоторому источнику тока,
    потребляют одинаковую мощность.
    Найти внутреннее сопротивление источника и КПД цепи в каждом случае.
}
\answer{%
    \begin{align*}
        P_1 &= \sqr{\frac{\ele}{R_1 + r}}R_1,
        P_2  = \sqr{\frac{\ele}{R_2 + r}}R_2,
        P_1 = P_2 \implies  \\
        &\implies R_1 \sqr{R_2 + r} = R_2 \sqr{R_1 + r} \implies  \\
        &\implies R_1 R_2^2 + 2 R_1 R_2 r + R_1 r^2 =
                    R_2 R_1^2 + 2 R_2 R_1 r + R_2 r^2  \implies  \\
    &\implies r^2 (R_2 - R_1) = R_2^2 R_2 - R_1^2 R_2 \implies  \\
    &\implies r
            = \sqrt{R_1 R_2 \frac{R_2 - R_1}{R_2 - R_1}}
            = \sqrt{R_1 R_2}
            = \sqrt{0{,}25\,\text{Ом} \cdot 64{,}00\,\text{Ом}}
            = 4{,}0\,\text{Ом}.
            \\
    \eta_1
            &= \frac{R_1}{R_1 + r}
            = \frac{\sqrt{R_1}}{\sqrt{R_1} + \sqrt{R_2}}
            = 0{,}059,  \\
    \eta_2
            &= \frac{R_2}{R_2 + r}
            = \frac{\sqrt{R_2}}{\sqrt{R_2} + \sqrt{R_1}}
            = 0{,}941
    \end{align*}
}
\solutionspace{120pt}

\tasknumber{5}%
\task{%
    Определите ток, протекающий через резистор $R = 10\,\text{Ом}$ и разность потенциалов на нём (см.
    рис.
    на доске),
    если $r_1 = 2\,\text{Ом}$, $r_2 = 2\,\text{Ом}$, $\ele_1 = 60\,\text{В}$, $\ele_2 = 30\,\text{В}$.
}
\answer{%
    Обозначим на рисунке все токи: направление произвольно, но его надо зафиксировать.
    Всего на рисунке 3 контура и 2 узла.
    Поэтому можно записать $3 - 1 = 2$ уравнения законов Кирхгофа для замкнутого контура и $2 - 1 = 1$ — для узлов
    (остальные уравнения тоже можно записать, но они не дадут полезной информации, а будут лишь следствиями уже записанных).

    Отметим на рисунке 2 контура (и не забуем указать направление) и 1 узел (точка «1»ы, выделена жирным).
    Выбор контуров и узлов не критичен: получившаяся система может быть чуть проще или сложнее, но не слишком.

    \begin{tikzpicture}[circuit ee IEC, thick]
        \draw  (0, 0) to [current direction={near end, info=$\eli_1$}] (0, 3)
                to [battery={rotate=-180,info={$\ele_1, r_1$}}]
                (3, 3)
                to [battery={info'={$\ele_2, r_2$}}]
                (6, 3) to [current direction'={near start, info=$\eli_2$}] (6, 0) -- (0, 0)
                (3, 0) to [current direction={near start, info=$\eli$}, resistor={near end, info=$R$}] (3, 3);
        \draw [-{Latex},color=red] (1.2, 1.7) arc [start angle = 135, end angle = -160, radius = 0.6];
        \draw [-{Latex},color=blue] (4.2, 1.7) arc [start angle = 135, end angle = -160, radius = 0.6];
        \node [contact,color=green!71!black] (bottomc) at (3, 0) {};
        \node [below] (bottom) at (3, 0) {$2$};
        \node [above] (top) at (3, 3) {$1$};
    \end{tikzpicture}

    \begin{align*}
        &\begin{cases}
            {\color{red} \ele_1 = \eli_1 r_1 - \eli R}, \\
            {\color{blue} -\ele_2 = -\eli_2 r_2 + \eli R}, \\
            {\color{green!71!black} - \eli - \eli_1 - \eli_2 = 0 };
        \end{cases}
        \qquad \implies \qquad
        \begin{cases}
            \eli_1 = \frac{\ele_1 + \eli R}{r_1}, \\
            \eli_2 = \frac{\ele_2 + \eli R}{r_2}, \\
            \eli + \eli_1 + \eli_2 = 0;
        \end{cases} \implies \\
        &\implies
         \eli + \frac{\ele_1 + \eli R}{r_1:L} + \frac{\ele_2 + \eli R}{r_2:L} = 0, \\
        &\eli\cbr{ 1 + \frac R{r_1:L} + \frac R{r_2:L}} + \frac{\ele_1 }{r_1:L} + \frac{\ele_2 }{r_2:L} = 0, \\
        &\eli
            = - \frac{\frac{\ele_1 }{r_1:L} + \frac{\ele_2 }{r_2:L}}{ 1 + \frac R{r_1:L} + \frac R{r_2:L}}
            = - \frac{\frac{60\,\text{В}}{2\,\text{Ом}} + \frac{30\,\text{В}}{2\,\text{Ом}}}{ 1 + \frac{10\,\text{Ом}}{2\,\text{Ом}} + \frac{10\,\text{Ом}}{2\,\text{Ом}}}
            = - \frac{45}{11}\units{А}
            \approx -4{,}10\,\text{А}, \\
        &U  = \varphi_2 - \varphi_1 = \eli R
            = - \frac{\frac{\ele_1 }{r_1:L} + \frac{\ele_2 }{r_2:L}}{ 1 + \frac R{r_1:L} + \frac R{r_2:L}} R
            \approx -40{,}90\,\text{В}.
    \end{align*}
    Оба ответа отрицательны, потому что мы изначально «не угадали» с направлением тока.
    Расчёт же показал,
    что ток через резистор $R$ течёт в противоположную сторону: вниз на рисунке, а потенциал точки 1 больше потенциала точки 2,
    а электрический ток ожидаемо течёт из точки с большим потенциалов в точку с меньшим.

    Кстати, если продолжить расчёт и вычислить значения ещё двух токов (формулы для $\eli_1$ и $\eli_2$, куда подставлять, выписаны выше),
    то по их знакам можно будет понять: угадали ли мы с их направлением или нет.
}

\variantsplitter

\addpersonalvariant{Даниил Космынин}

\tasknumber{1}%
\task{%
    На резистор сопротивлением $r = 5\,\text{Ом}$ подали напряжение $U = 150\,\text{В}$.
    Определите ток, который потечёт через резистор, и мощность, выделяющуюся на нём.
}
\answer{%
    \begin{align*}
    \eli &= \frac{U}{r} = \frac{150\,\text{В}}{5\,\text{Ом}} = 30{,}00\,\text{А},  \\
    P &= \frac{U^2}{r} = \frac{\sqr{150\,\text{В}}}{5\,\text{Ом}} = 4500{,}00\,\text{Вт}
    \end{align*}
}
\solutionspace{60pt}

\tasknumber{2}%
\task{%
    Через резистор сопротивлением $R = 5\,\text{Ом}$ протекает электрический ток $\eli = 15{,}00\,\text{А}$.
    Определите, чему равны напряжение на резисторе и мощность, выделяющаяся на нём.
}
\answer{%
    \begin{align*}
    U &= \eli R = 15{,}00\,\text{А} \cdot 5\,\text{Ом} = 75\,\text{В},  \\
    P &= \eli^2R = \sqr{15{,}00\,\text{А}} \cdot 5\,\text{Ом} = 1125\,\text{Вт}
    \end{align*}
}
\solutionspace{60pt}

\tasknumber{3}%
\task{%
    Замкнутая электрическая цепь состоит из ЭДС $\ele = 3\,\text{В}$ и сопротивлением $r$
    и резистора $R = 10\,\text{Ом}$.
    Определите ток, протекающий в цепи.
    Какая тепловая энергия выделится на резисторе за время
    $\tau = 2\,\text{с}$? Какая работа будет совершена ЭДС за это время? Каков знак этой работы? Чему равен КПД цепи?
    Вычислите значения для 2 случаев: $r=0$ и $r = 10\,\text{Ом}$.
}
\answer{%
    \begin{align*}
    \eli_1 &= \frac{\ele}{R} = \frac{3\,\text{В}}{10\,\text{Ом}} = \frac3{10}\units{А} \approx 0{,}30\,\text{А},  \\
    \eli_2 &= \frac{\ele}{R + r} = \frac{3\,\text{В}}{10\,\text{Ом} + 10\,\text{Ом}} = \frac3{20}\units{А} \approx 0{,}15\,\text{А},  \\
    Q_1 &= \eli_1^2R\tau = \sqr{\frac{\ele}{R}} R \tau
            = \sqr{\frac{3\,\text{В}}{10\,\text{Ом}}} \cdot 10\,\text{Ом} \cdot 2\,\text{с} = \frac95\units{Дж} \approx 1{,}800\,\text{Дж},  \\
    Q_2 &= \eli_2^2R\tau = \sqr{\frac{\ele}{R + r}} R \tau
            = \sqr{\frac{3\,\text{В}}{10\,\text{Ом} + 10\,\text{Ом}}} \cdot 10\,\text{Ом} \cdot 2\,\text{с} = \frac9{20}\units{Дж} \approx 0{,}450\,\text{Дж},  \\
    A_1 &= q_1\ele = \eli_1\tau\ele = \frac{\ele}{R} \tau \ele
            = \frac{\ele^2 \tau}{R} = \frac{\sqr{3\,\text{В}} \cdot 2\,\text{с}}{10\,\text{Ом}}
            = \frac95\units{Дж} \approx 1{,}800\,\text{Дж}, \text{положительна},  \\
    A_2 &= q_2\ele = \eli_2\tau\ele = \frac{\ele}{R + r} \tau \ele
            = \frac{\ele^2 \tau}{R + r} = \frac{\sqr{3\,\text{В}} \cdot 2\,\text{с}}{10\,\text{Ом} + 10\,\text{Ом}}
            = \frac9{10}\units{Дж} \approx 0{,}900\,\text{Дж}, \text{положительна},  \\
    \eta_1 &= \frac{Q_1}{A_1} = \ldots = \frac{R}{R} = 1,  \\
    \eta_2 &= \frac{Q_2}{A_2} = \ldots = \frac{R}{R + r} = \frac12 \approx 0{,}50.
    \end{align*}
}
\solutionspace{180pt}

\tasknumber{4}%
\task{%
    Лампочки, сопротивления которых $R_1 = 0{,}50\,\text{Ом}$ и $R_2 = 18{,}00\,\text{Ом}$, поочерёдно подключённные к некоторому источнику тока,
    потребляют одинаковую мощность.
    Найти внутреннее сопротивление источника и КПД цепи в каждом случае.
}
\answer{%
    \begin{align*}
        P_1 &= \sqr{\frac{\ele}{R_1 + r}}R_1,
        P_2  = \sqr{\frac{\ele}{R_2 + r}}R_2,
        P_1 = P_2 \implies  \\
        &\implies R_1 \sqr{R_2 + r} = R_2 \sqr{R_1 + r} \implies  \\
        &\implies R_1 R_2^2 + 2 R_1 R_2 r + R_1 r^2 =
                    R_2 R_1^2 + 2 R_2 R_1 r + R_2 r^2  \implies  \\
    &\implies r^2 (R_2 - R_1) = R_2^2 R_2 - R_1^2 R_2 \implies  \\
    &\implies r
            = \sqrt{R_1 R_2 \frac{R_2 - R_1}{R_2 - R_1}}
            = \sqrt{R_1 R_2}
            = \sqrt{0{,}50\,\text{Ом} \cdot 18{,}00\,\text{Ом}}
            = 3{,}0\,\text{Ом}.
            \\
    \eta_1
            &= \frac{R_1}{R_1 + r}
            = \frac{\sqrt{R_1}}{\sqrt{R_1} + \sqrt{R_2}}
            = 0{,}143,  \\
    \eta_2
            &= \frac{R_2}{R_2 + r}
            = \frac{\sqrt{R_2}}{\sqrt{R_2} + \sqrt{R_1}}
            = 0{,}857
    \end{align*}
}
\solutionspace{120pt}

\tasknumber{5}%
\task{%
    Определите ток, протекающий через резистор $R = 20\,\text{Ом}$ и разность потенциалов на нём (см.
    рис.
    на доске),
    если $r_1 = 2\,\text{Ом}$, $r_2 = 2\,\text{Ом}$, $\ele_1 = 20\,\text{В}$, $\ele_2 = 40\,\text{В}$.
}
\answer{%
    Обозначим на рисунке все токи: направление произвольно, но его надо зафиксировать.
    Всего на рисунке 3 контура и 2 узла.
    Поэтому можно записать $3 - 1 = 2$ уравнения законов Кирхгофа для замкнутого контура и $2 - 1 = 1$ — для узлов
    (остальные уравнения тоже можно записать, но они не дадут полезной информации, а будут лишь следствиями уже записанных).

    Отметим на рисунке 2 контура (и не забуем указать направление) и 1 узел (точка «1»ы, выделена жирным).
    Выбор контуров и узлов не критичен: получившаяся система может быть чуть проще или сложнее, но не слишком.

    \begin{tikzpicture}[circuit ee IEC, thick]
        \draw  (0, 0) to [current direction={near end, info=$\eli_1$}] (0, 3)
                to [battery={rotate=-180,info={$\ele_1, r_1$}}]
                (3, 3)
                to [battery={info'={$\ele_2, r_2$}}]
                (6, 3) to [current direction'={near start, info=$\eli_2$}] (6, 0) -- (0, 0)
                (3, 0) to [current direction={near start, info=$\eli$}, resistor={near end, info=$R$}] (3, 3);
        \draw [-{Latex},color=red] (1.2, 1.7) arc [start angle = 135, end angle = -160, radius = 0.6];
        \draw [-{Latex},color=blue] (4.2, 1.7) arc [start angle = 135, end angle = -160, radius = 0.6];
        \node [contact,color=green!71!black] (bottomc) at (3, 0) {};
        \node [below] (bottom) at (3, 0) {$2$};
        \node [above] (top) at (3, 3) {$1$};
    \end{tikzpicture}

    \begin{align*}
        &\begin{cases}
            {\color{red} \ele_1 = \eli_1 r_1 - \eli R}, \\
            {\color{blue} -\ele_2 = -\eli_2 r_2 + \eli R}, \\
            {\color{green!71!black} - \eli - \eli_1 - \eli_2 = 0 };
        \end{cases}
        \qquad \implies \qquad
        \begin{cases}
            \eli_1 = \frac{\ele_1 + \eli R}{r_1}, \\
            \eli_2 = \frac{\ele_2 + \eli R}{r_2}, \\
            \eli + \eli_1 + \eli_2 = 0;
        \end{cases} \implies \\
        &\implies
         \eli + \frac{\ele_1 + \eli R}{r_1:L} + \frac{\ele_2 + \eli R}{r_2:L} = 0, \\
        &\eli\cbr{ 1 + \frac R{r_1:L} + \frac R{r_2:L}} + \frac{\ele_1 }{r_1:L} + \frac{\ele_2 }{r_2:L} = 0, \\
        &\eli
            = - \frac{\frac{\ele_1 }{r_1:L} + \frac{\ele_2 }{r_2:L}}{ 1 + \frac R{r_1:L} + \frac R{r_2:L}}
            = - \frac{\frac{20\,\text{В}}{2\,\text{Ом}} + \frac{40\,\text{В}}{2\,\text{Ом}}}{ 1 + \frac{20\,\text{Ом}}{2\,\text{Ом}} + \frac{20\,\text{Ом}}{2\,\text{Ом}}}
            = - \frac{10}7\units{А}
            \approx -1{,}400\,\text{А}, \\
        &U  = \varphi_2 - \varphi_1 = \eli R
            = - \frac{\frac{\ele_1 }{r_1:L} + \frac{\ele_2 }{r_2:L}}{ 1 + \frac R{r_1:L} + \frac R{r_2:L}} R
            \approx -28{,}60\,\text{В}.
    \end{align*}
    Оба ответа отрицательны, потому что мы изначально «не угадали» с направлением тока.
    Расчёт же показал,
    что ток через резистор $R$ течёт в противоположную сторону: вниз на рисунке, а потенциал точки 1 больше потенциала точки 2,
    а электрический ток ожидаемо течёт из точки с большим потенциалов в точку с меньшим.

    Кстати, если продолжить расчёт и вычислить значения ещё двух токов (формулы для $\eli_1$ и $\eli_2$, куда подставлять, выписаны выше),
    то по их знакам можно будет понять: угадали ли мы с их направлением или нет.
}

\variantsplitter

\addpersonalvariant{Алина Леоничева}

\tasknumber{1}%
\task{%
    На резистор сопротивлением $r = 12\,\text{Ом}$ подали напряжение $V = 180\,\text{В}$.
    Определите ток, который потечёт через резистор, и мощность, выделяющуюся на нём.
}
\answer{%
    \begin{align*}
    \eli &= \frac{V}{r} = \frac{180\,\text{В}}{12\,\text{Ом}} = 15{,}00\,\text{А},  \\
    P &= \frac{V^2}{r} = \frac{\sqr{180\,\text{В}}}{12\,\text{Ом}} = 2700{,}00\,\text{Вт}
    \end{align*}
}
\solutionspace{60pt}

\tasknumber{2}%
\task{%
    Через резистор сопротивлением $R = 5\,\text{Ом}$ протекает электрический ток $\eli = 8{,}00\,\text{А}$.
    Определите, чему равны напряжение на резисторе и мощность, выделяющаяся на нём.
}
\answer{%
    \begin{align*}
    U &= \eli R = 8{,}00\,\text{А} \cdot 5\,\text{Ом} = 40\,\text{В},  \\
    P &= \eli^2R = \sqr{8{,}00\,\text{А}} \cdot 5\,\text{Ом} = 320\,\text{Вт}
    \end{align*}
}
\solutionspace{60pt}

\tasknumber{3}%
\task{%
    Замкнутая электрическая цепь состоит из ЭДС $\ele = 4\,\text{В}$ и сопротивлением $r$
    и резистора $R = 10\,\text{Ом}$.
    Определите ток, протекающий в цепи.
    Какая тепловая энергия выделится на резисторе за время
    $\tau = 2\,\text{с}$? Какая работа будет совершена ЭДС за это время? Каков знак этой работы? Чему равен КПД цепи?
    Вычислите значения для 2 случаев: $r=0$ и $r = 60\,\text{Ом}$.
}
\answer{%
    \begin{align*}
    \eli_1 &= \frac{\ele}{R} = \frac{4\,\text{В}}{10\,\text{Ом}} = \frac25\units{А} \approx 0{,}40\,\text{А},  \\
    \eli_2 &= \frac{\ele}{R + r} = \frac{4\,\text{В}}{10\,\text{Ом} + 60\,\text{Ом}} = \frac2{35}\units{А} \approx 0{,}06\,\text{А},  \\
    Q_1 &= \eli_1^2R\tau = \sqr{\frac{\ele}{R}} R \tau
            = \sqr{\frac{4\,\text{В}}{10\,\text{Ом}}} \cdot 10\,\text{Ом} \cdot 2\,\text{с} = \frac{16}5\units{Дж} \approx 3{,}200\,\text{Дж},  \\
    Q_2 &= \eli_2^2R\tau = \sqr{\frac{\ele}{R + r}} R \tau
            = \sqr{\frac{4\,\text{В}}{10\,\text{Ом} + 60\,\text{Ом}}} \cdot 10\,\text{Ом} \cdot 2\,\text{с} = \frac{16}{245}\units{Дж} \approx 0{,}065\,\text{Дж},  \\
    A_1 &= q_1\ele = \eli_1\tau\ele = \frac{\ele}{R} \tau \ele
            = \frac{\ele^2 \tau}{R} = \frac{\sqr{4\,\text{В}} \cdot 2\,\text{с}}{10\,\text{Ом}}
            = \frac{16}5\units{Дж} \approx 3{,}200\,\text{Дж}, \text{положительна},  \\
    A_2 &= q_2\ele = \eli_2\tau\ele = \frac{\ele}{R + r} \tau \ele
            = \frac{\ele^2 \tau}{R + r} = \frac{\sqr{4\,\text{В}} \cdot 2\,\text{с}}{10\,\text{Ом} + 60\,\text{Ом}}
            = \frac{16}{35}\units{Дж} \approx 0{,}457\,\text{Дж}, \text{положительна},  \\
    \eta_1 &= \frac{Q_1}{A_1} = \ldots = \frac{R}{R} = 1,  \\
    \eta_2 &= \frac{Q_2}{A_2} = \ldots = \frac{R}{R + r} = \frac17 \approx 0{,}14.
    \end{align*}
}
\solutionspace{180pt}

\tasknumber{4}%
\task{%
    Лампочки, сопротивления которых $R_1 = 0{,}25\,\text{Ом}$ и $R_2 = 16{,}00\,\text{Ом}$, поочерёдно подключённные к некоторому источнику тока,
    потребляют одинаковую мощность.
    Найти внутреннее сопротивление источника и КПД цепи в каждом случае.
}
\answer{%
    \begin{align*}
        P_1 &= \sqr{\frac{\ele}{R_1 + r}}R_1,
        P_2  = \sqr{\frac{\ele}{R_2 + r}}R_2,
        P_1 = P_2 \implies  \\
        &\implies R_1 \sqr{R_2 + r} = R_2 \sqr{R_1 + r} \implies  \\
        &\implies R_1 R_2^2 + 2 R_1 R_2 r + R_1 r^2 =
                    R_2 R_1^2 + 2 R_2 R_1 r + R_2 r^2  \implies  \\
    &\implies r^2 (R_2 - R_1) = R_2^2 R_2 - R_1^2 R_2 \implies  \\
    &\implies r
            = \sqrt{R_1 R_2 \frac{R_2 - R_1}{R_2 - R_1}}
            = \sqrt{R_1 R_2}
            = \sqrt{0{,}25\,\text{Ом} \cdot 16{,}00\,\text{Ом}}
            = 2{,}0\,\text{Ом}.
            \\
    \eta_1
            &= \frac{R_1}{R_1 + r}
            = \frac{\sqrt{R_1}}{\sqrt{R_1} + \sqrt{R_2}}
            = 0{,}111,  \\
    \eta_2
            &= \frac{R_2}{R_2 + r}
            = \frac{\sqrt{R_2}}{\sqrt{R_2} + \sqrt{R_1}}
            = 0{,}889
    \end{align*}
}
\solutionspace{120pt}

\tasknumber{5}%
\task{%
    Определите ток, протекающий через резистор $R = 12\,\text{Ом}$ и разность потенциалов на нём (см.
    рис.
    на доске),
    если $r_1 = 3\,\text{Ом}$, $r_2 = 3\,\text{Ом}$, $\ele_1 = 60\,\text{В}$, $\ele_2 = 20\,\text{В}$.
}
\answer{%
    Обозначим на рисунке все токи: направление произвольно, но его надо зафиксировать.
    Всего на рисунке 3 контура и 2 узла.
    Поэтому можно записать $3 - 1 = 2$ уравнения законов Кирхгофа для замкнутого контура и $2 - 1 = 1$ — для узлов
    (остальные уравнения тоже можно записать, но они не дадут полезной информации, а будут лишь следствиями уже записанных).

    Отметим на рисунке 2 контура (и не забуем указать направление) и 1 узел (точка «1»ы, выделена жирным).
    Выбор контуров и узлов не критичен: получившаяся система может быть чуть проще или сложнее, но не слишком.

    \begin{tikzpicture}[circuit ee IEC, thick]
        \draw  (0, 0) to [current direction={near end, info=$\eli_1$}] (0, 3)
                to [battery={rotate=-180,info={$\ele_1, r_1$}}]
                (3, 3)
                to [battery={info'={$\ele_2, r_2$}}]
                (6, 3) to [current direction'={near start, info=$\eli_2$}] (6, 0) -- (0, 0)
                (3, 0) to [current direction={near start, info=$\eli$}, resistor={near end, info=$R$}] (3, 3);
        \draw [-{Latex},color=red] (1.2, 1.7) arc [start angle = 135, end angle = -160, radius = 0.6];
        \draw [-{Latex},color=blue] (4.2, 1.7) arc [start angle = 135, end angle = -160, radius = 0.6];
        \node [contact,color=green!71!black] (bottomc) at (3, 0) {};
        \node [below] (bottom) at (3, 0) {$2$};
        \node [above] (top) at (3, 3) {$1$};
    \end{tikzpicture}

    \begin{align*}
        &\begin{cases}
            {\color{red} \ele_1 = \eli_1 r_1 - \eli R}, \\
            {\color{blue} -\ele_2 = -\eli_2 r_2 + \eli R}, \\
            {\color{green!71!black} - \eli - \eli_1 - \eli_2 = 0 };
        \end{cases}
        \qquad \implies \qquad
        \begin{cases}
            \eli_1 = \frac{\ele_1 + \eli R}{r_1}, \\
            \eli_2 = \frac{\ele_2 + \eli R}{r_2}, \\
            \eli + \eli_1 + \eli_2 = 0;
        \end{cases} \implies \\
        &\implies
         \eli + \frac{\ele_1 + \eli R}{r_1:L} + \frac{\ele_2 + \eli R}{r_2:L} = 0, \\
        &\eli\cbr{ 1 + \frac R{r_1:L} + \frac R{r_2:L}} + \frac{\ele_1 }{r_1:L} + \frac{\ele_2 }{r_2:L} = 0, \\
        &\eli
            = - \frac{\frac{\ele_1 }{r_1:L} + \frac{\ele_2 }{r_2:L}}{ 1 + \frac R{r_1:L} + \frac R{r_2:L}}
            = - \frac{\frac{60\,\text{В}}{3\,\text{Ом}} + \frac{20\,\text{В}}{3\,\text{Ом}}}{ 1 + \frac{12\,\text{Ом}}{3\,\text{Ом}} + \frac{12\,\text{Ом}}{3\,\text{Ом}}}
            = - \frac{80}{27}\units{А}
            \approx -3{,}00\,\text{А}, \\
        &U  = \varphi_2 - \varphi_1 = \eli R
            = - \frac{\frac{\ele_1 }{r_1:L} + \frac{\ele_2 }{r_2:L}}{ 1 + \frac R{r_1:L} + \frac R{r_2:L}} R
            \approx -35{,}60\,\text{В}.
    \end{align*}
    Оба ответа отрицательны, потому что мы изначально «не угадали» с направлением тока.
    Расчёт же показал,
    что ток через резистор $R$ течёт в противоположную сторону: вниз на рисунке, а потенциал точки 1 больше потенциала точки 2,
    а электрический ток ожидаемо течёт из точки с большим потенциалов в точку с меньшим.

    Кстати, если продолжить расчёт и вычислить значения ещё двух токов (формулы для $\eli_1$ и $\eli_2$, куда подставлять, выписаны выше),
    то по их знакам можно будет понять: угадали ли мы с их направлением или нет.
}

\variantsplitter

\addpersonalvariant{Ирина Лин}

\tasknumber{1}%
\task{%
    На резистор сопротивлением $r = 30\,\text{Ом}$ подали напряжение $V = 120\,\text{В}$.
    Определите ток, который потечёт через резистор, и мощность, выделяющуюся на нём.
}
\answer{%
    \begin{align*}
    \eli &= \frac{V}{r} = \frac{120\,\text{В}}{30\,\text{Ом}} = 4{,}00\,\text{А},  \\
    P &= \frac{V^2}{r} = \frac{\sqr{120\,\text{В}}}{30\,\text{Ом}} = 480{,}00\,\text{Вт}
    \end{align*}
}
\solutionspace{60pt}

\tasknumber{2}%
\task{%
    Через резистор сопротивлением $R = 18\,\text{Ом}$ протекает электрический ток $\eli = 5{,}00\,\text{А}$.
    Определите, чему равны напряжение на резисторе и мощность, выделяющаяся на нём.
}
\answer{%
    \begin{align*}
    U &= \eli R = 5{,}00\,\text{А} \cdot 18\,\text{Ом} = 90\,\text{В},  \\
    P &= \eli^2R = \sqr{5{,}00\,\text{А}} \cdot 18\,\text{Ом} = 450\,\text{Вт}
    \end{align*}
}
\solutionspace{60pt}

\tasknumber{3}%
\task{%
    Замкнутая электрическая цепь состоит из ЭДС $\ele = 4\,\text{В}$ и сопротивлением $r$
    и резистора $R = 15\,\text{Ом}$.
    Определите ток, протекающий в цепи.
    Какая тепловая энергия выделится на резисторе за время
    $\tau = 5\,\text{с}$? Какая работа будет совершена ЭДС за это время? Каков знак этой работы? Чему равен КПД цепи?
    Вычислите значения для 2 случаев: $r=0$ и $r = 30\,\text{Ом}$.
}
\answer{%
    \begin{align*}
    \eli_1 &= \frac{\ele}{R} = \frac{4\,\text{В}}{15\,\text{Ом}} = \frac4{15}\units{А} \approx 0{,}27\,\text{А},  \\
    \eli_2 &= \frac{\ele}{R + r} = \frac{4\,\text{В}}{15\,\text{Ом} + 30\,\text{Ом}} = \frac4{45}\units{А} \approx 0{,}09\,\text{А},  \\
    Q_1 &= \eli_1^2R\tau = \sqr{\frac{\ele}{R}} R \tau
            = \sqr{\frac{4\,\text{В}}{15\,\text{Ом}}} \cdot 15\,\text{Ом} \cdot 5\,\text{с} = \frac{16}3\units{Дж} \approx 5{,}333\,\text{Дж},  \\
    Q_2 &= \eli_2^2R\tau = \sqr{\frac{\ele}{R + r}} R \tau
            = \sqr{\frac{4\,\text{В}}{15\,\text{Ом} + 30\,\text{Ом}}} \cdot 15\,\text{Ом} \cdot 5\,\text{с} = \frac{16}{27}\units{Дж} \approx 0{,}593\,\text{Дж},  \\
    A_1 &= q_1\ele = \eli_1\tau\ele = \frac{\ele}{R} \tau \ele
            = \frac{\ele^2 \tau}{R} = \frac{\sqr{4\,\text{В}} \cdot 5\,\text{с}}{15\,\text{Ом}}
            = \frac{16}3\units{Дж} \approx 5{,}333\,\text{Дж}, \text{положительна},  \\
    A_2 &= q_2\ele = \eli_2\tau\ele = \frac{\ele}{R + r} \tau \ele
            = \frac{\ele^2 \tau}{R + r} = \frac{\sqr{4\,\text{В}} \cdot 5\,\text{с}}{15\,\text{Ом} + 30\,\text{Ом}}
            = \frac{16}9\units{Дж} \approx 1{,}778\,\text{Дж}, \text{положительна},  \\
    \eta_1 &= \frac{Q_1}{A_1} = \ldots = \frac{R}{R} = 1,  \\
    \eta_2 &= \frac{Q_2}{A_2} = \ldots = \frac{R}{R + r} = \frac13 \approx 0{,}33.
    \end{align*}
}
\solutionspace{180pt}

\tasknumber{4}%
\task{%
    Лампочки, сопротивления которых $R_1 = 0{,}50\,\text{Ом}$ и $R_2 = 4{,}50\,\text{Ом}$, поочерёдно подключённные к некоторому источнику тока,
    потребляют одинаковую мощность.
    Найти внутреннее сопротивление источника и КПД цепи в каждом случае.
}
\answer{%
    \begin{align*}
        P_1 &= \sqr{\frac{\ele}{R_1 + r}}R_1,
        P_2  = \sqr{\frac{\ele}{R_2 + r}}R_2,
        P_1 = P_2 \implies  \\
        &\implies R_1 \sqr{R_2 + r} = R_2 \sqr{R_1 + r} \implies  \\
        &\implies R_1 R_2^2 + 2 R_1 R_2 r + R_1 r^2 =
                    R_2 R_1^2 + 2 R_2 R_1 r + R_2 r^2  \implies  \\
    &\implies r^2 (R_2 - R_1) = R_2^2 R_2 - R_1^2 R_2 \implies  \\
    &\implies r
            = \sqrt{R_1 R_2 \frac{R_2 - R_1}{R_2 - R_1}}
            = \sqrt{R_1 R_2}
            = \sqrt{0{,}50\,\text{Ом} \cdot 4{,}50\,\text{Ом}}
            = 1{,}5\,\text{Ом}.
            \\
    \eta_1
            &= \frac{R_1}{R_1 + r}
            = \frac{\sqrt{R_1}}{\sqrt{R_1} + \sqrt{R_2}}
            = 0{,}250,  \\
    \eta_2
            &= \frac{R_2}{R_2 + r}
            = \frac{\sqrt{R_2}}{\sqrt{R_2} + \sqrt{R_1}}
            = 0{,}750
    \end{align*}
}
\solutionspace{120pt}

\tasknumber{5}%
\task{%
    Определите ток, протекающий через резистор $R = 18\,\text{Ом}$ и разность потенциалов на нём (см.
    рис.
    на доске),
    если $r_1 = 2\,\text{Ом}$, $r_2 = 3\,\text{Ом}$, $\ele_1 = 30\,\text{В}$, $\ele_2 = 30\,\text{В}$.
}
\answer{%
    Обозначим на рисунке все токи: направление произвольно, но его надо зафиксировать.
    Всего на рисунке 3 контура и 2 узла.
    Поэтому можно записать $3 - 1 = 2$ уравнения законов Кирхгофа для замкнутого контура и $2 - 1 = 1$ — для узлов
    (остальные уравнения тоже можно записать, но они не дадут полезной информации, а будут лишь следствиями уже записанных).

    Отметим на рисунке 2 контура (и не забуем указать направление) и 1 узел (точка «1»ы, выделена жирным).
    Выбор контуров и узлов не критичен: получившаяся система может быть чуть проще или сложнее, но не слишком.

    \begin{tikzpicture}[circuit ee IEC, thick]
        \draw  (0, 0) to [current direction={near end, info=$\eli_1$}] (0, 3)
                to [battery={rotate=-180,info={$\ele_1, r_1$}}]
                (3, 3)
                to [battery={info'={$\ele_2, r_2$}}]
                (6, 3) to [current direction'={near start, info=$\eli_2$}] (6, 0) -- (0, 0)
                (3, 0) to [current direction={near start, info=$\eli$}, resistor={near end, info=$R$}] (3, 3);
        \draw [-{Latex},color=red] (1.2, 1.7) arc [start angle = 135, end angle = -160, radius = 0.6];
        \draw [-{Latex},color=blue] (4.2, 1.7) arc [start angle = 135, end angle = -160, radius = 0.6];
        \node [contact,color=green!71!black] (bottomc) at (3, 0) {};
        \node [below] (bottom) at (3, 0) {$2$};
        \node [above] (top) at (3, 3) {$1$};
    \end{tikzpicture}

    \begin{align*}
        &\begin{cases}
            {\color{red} \ele_1 = \eli_1 r_1 - \eli R}, \\
            {\color{blue} -\ele_2 = -\eli_2 r_2 + \eli R}, \\
            {\color{green!71!black} - \eli - \eli_1 - \eli_2 = 0 };
        \end{cases}
        \qquad \implies \qquad
        \begin{cases}
            \eli_1 = \frac{\ele_1 + \eli R}{r_1}, \\
            \eli_2 = \frac{\ele_2 + \eli R}{r_2}, \\
            \eli + \eli_1 + \eli_2 = 0;
        \end{cases} \implies \\
        &\implies
         \eli + \frac{\ele_1 + \eli R}{r_1:L} + \frac{\ele_2 + \eli R}{r_2:L} = 0, \\
        &\eli\cbr{ 1 + \frac R{r_1:L} + \frac R{r_2:L}} + \frac{\ele_1 }{r_1:L} + \frac{\ele_2 }{r_2:L} = 0, \\
        &\eli
            = - \frac{\frac{\ele_1 }{r_1:L} + \frac{\ele_2 }{r_2:L}}{ 1 + \frac R{r_1:L} + \frac R{r_2:L}}
            = - \frac{\frac{30\,\text{В}}{2\,\text{Ом}} + \frac{30\,\text{В}}{3\,\text{Ом}}}{ 1 + \frac{18\,\text{Ом}}{2\,\text{Ом}} + \frac{18\,\text{Ом}}{3\,\text{Ом}}}
            = - \frac{25}{16}\units{А}
            \approx -1{,}600\,\text{А}, \\
        &U  = \varphi_2 - \varphi_1 = \eli R
            = - \frac{\frac{\ele_1 }{r_1:L} + \frac{\ele_2 }{r_2:L}}{ 1 + \frac R{r_1:L} + \frac R{r_2:L}} R
            \approx -28{,}10\,\text{В}.
    \end{align*}
    Оба ответа отрицательны, потому что мы изначально «не угадали» с направлением тока.
    Расчёт же показал,
    что ток через резистор $R$ течёт в противоположную сторону: вниз на рисунке, а потенциал точки 1 больше потенциала точки 2,
    а электрический ток ожидаемо течёт из точки с большим потенциалов в точку с меньшим.

    Кстати, если продолжить расчёт и вычислить значения ещё двух токов (формулы для $\eli_1$ и $\eli_2$, куда подставлять, выписаны выше),
    то по их знакам можно будет понять: угадали ли мы с их направлением или нет.
}

\variantsplitter

\addpersonalvariant{Олег Мальцев}

\tasknumber{1}%
\task{%
    На резистор сопротивлением $R = 5\,\text{Ом}$ подали напряжение $U = 180\,\text{В}$.
    Определите ток, который потечёт через резистор, и мощность, выделяющуюся на нём.
}
\answer{%
    \begin{align*}
    \eli &= \frac{U}{R} = \frac{180\,\text{В}}{5\,\text{Ом}} = 36{,}00\,\text{А},  \\
    P &= \frac{U^2}{R} = \frac{\sqr{180\,\text{В}}}{5\,\text{Ом}} = 6480{,}00\,\text{Вт}
    \end{align*}
}
\solutionspace{60pt}

\tasknumber{2}%
\task{%
    Через резистор сопротивлением $r = 5\,\text{Ом}$ протекает электрический ток $\eli = 3{,}00\,\text{А}$.
    Определите, чему равны напряжение на резисторе и мощность, выделяющаяся на нём.
}
\answer{%
    \begin{align*}
    U &= \eli r = 3{,}00\,\text{А} \cdot 5\,\text{Ом} = 15\,\text{В},  \\
    P &= \eli^2r = \sqr{3{,}00\,\text{А}} \cdot 5\,\text{Ом} = 45\,\text{Вт}
    \end{align*}
}
\solutionspace{60pt}

\tasknumber{3}%
\task{%
    Замкнутая электрическая цепь состоит из ЭДС $\ele = 3\,\text{В}$ и сопротивлением $r$
    и резистора $R = 30\,\text{Ом}$.
    Определите ток, протекающий в цепи.
    Какая тепловая энергия выделится на резисторе за время
    $\tau = 10\,\text{с}$? Какая работа будет совершена ЭДС за это время? Каков знак этой работы? Чему равен КПД цепи?
    Вычислите значения для 2 случаев: $r=0$ и $r = 60\,\text{Ом}$.
}
\answer{%
    \begin{align*}
    \eli_1 &= \frac{\ele}{R} = \frac{3\,\text{В}}{30\,\text{Ом}} = \frac1{10}\units{А} \approx 0{,}10\,\text{А},  \\
    \eli_2 &= \frac{\ele}{R + r} = \frac{3\,\text{В}}{30\,\text{Ом} + 60\,\text{Ом}} = \frac1{30}\units{А} \approx 0{,}03\,\text{А},  \\
    Q_1 &= \eli_1^2R\tau = \sqr{\frac{\ele}{R}} R \tau
            = \sqr{\frac{3\,\text{В}}{30\,\text{Ом}}} \cdot 30\,\text{Ом} \cdot 10\,\text{с} = 3\units{Дж} \approx 3{,}000\,\text{Дж},  \\
    Q_2 &= \eli_2^2R\tau = \sqr{\frac{\ele}{R + r}} R \tau
            = \sqr{\frac{3\,\text{В}}{30\,\text{Ом} + 60\,\text{Ом}}} \cdot 30\,\text{Ом} \cdot 10\,\text{с} = \frac13\units{Дж} \approx 0{,}333\,\text{Дж},  \\
    A_1 &= q_1\ele = \eli_1\tau\ele = \frac{\ele}{R} \tau \ele
            = \frac{\ele^2 \tau}{R} = \frac{\sqr{3\,\text{В}} \cdot 10\,\text{с}}{30\,\text{Ом}}
            = 3\units{Дж} \approx 3{,}000\,\text{Дж}, \text{положительна},  \\
    A_2 &= q_2\ele = \eli_2\tau\ele = \frac{\ele}{R + r} \tau \ele
            = \frac{\ele^2 \tau}{R + r} = \frac{\sqr{3\,\text{В}} \cdot 10\,\text{с}}{30\,\text{Ом} + 60\,\text{Ом}}
            = 1\units{Дж} \approx 1{,}000\,\text{Дж}, \text{положительна},  \\
    \eta_1 &= \frac{Q_1}{A_1} = \ldots = \frac{R}{R} = 1,  \\
    \eta_2 &= \frac{Q_2}{A_2} = \ldots = \frac{R}{R + r} = \frac13 \approx 0{,}33.
    \end{align*}
}
\solutionspace{180pt}

\tasknumber{4}%
\task{%
    Лампочки, сопротивления которых $R_1 = 0{,}50\,\text{Ом}$ и $R_2 = 2{,}00\,\text{Ом}$, поочерёдно подключённные к некоторому источнику тока,
    потребляют одинаковую мощность.
    Найти внутреннее сопротивление источника и КПД цепи в каждом случае.
}
\answer{%
    \begin{align*}
        P_1 &= \sqr{\frac{\ele}{R_1 + r}}R_1,
        P_2  = \sqr{\frac{\ele}{R_2 + r}}R_2,
        P_1 = P_2 \implies  \\
        &\implies R_1 \sqr{R_2 + r} = R_2 \sqr{R_1 + r} \implies  \\
        &\implies R_1 R_2^2 + 2 R_1 R_2 r + R_1 r^2 =
                    R_2 R_1^2 + 2 R_2 R_1 r + R_2 r^2  \implies  \\
    &\implies r^2 (R_2 - R_1) = R_2^2 R_2 - R_1^2 R_2 \implies  \\
    &\implies r
            = \sqrt{R_1 R_2 \frac{R_2 - R_1}{R_2 - R_1}}
            = \sqrt{R_1 R_2}
            = \sqrt{0{,}50\,\text{Ом} \cdot 2{,}00\,\text{Ом}}
            = 1{,}0\,\text{Ом}.
            \\
    \eta_1
            &= \frac{R_1}{R_1 + r}
            = \frac{\sqrt{R_1}}{\sqrt{R_1} + \sqrt{R_2}}
            = 0{,}333,  \\
    \eta_2
            &= \frac{R_2}{R_2 + r}
            = \frac{\sqrt{R_2}}{\sqrt{R_2} + \sqrt{R_1}}
            = 0{,}667
    \end{align*}
}
\solutionspace{120pt}

\tasknumber{5}%
\task{%
    Определите ток, протекающий через резистор $R = 12\,\text{Ом}$ и разность потенциалов на нём (см.
    рис.
    на доске),
    если $r_1 = 3\,\text{Ом}$, $r_2 = 3\,\text{Ом}$, $\ele_1 = 60\,\text{В}$, $\ele_2 = 30\,\text{В}$.
}
\answer{%
    Обозначим на рисунке все токи: направление произвольно, но его надо зафиксировать.
    Всего на рисунке 3 контура и 2 узла.
    Поэтому можно записать $3 - 1 = 2$ уравнения законов Кирхгофа для замкнутого контура и $2 - 1 = 1$ — для узлов
    (остальные уравнения тоже можно записать, но они не дадут полезной информации, а будут лишь следствиями уже записанных).

    Отметим на рисунке 2 контура (и не забуем указать направление) и 1 узел (точка «1»ы, выделена жирным).
    Выбор контуров и узлов не критичен: получившаяся система может быть чуть проще или сложнее, но не слишком.

    \begin{tikzpicture}[circuit ee IEC, thick]
        \draw  (0, 0) to [current direction={near end, info=$\eli_1$}] (0, 3)
                to [battery={rotate=-180,info={$\ele_1, r_1$}}]
                (3, 3)
                to [battery={info'={$\ele_2, r_2$}}]
                (6, 3) to [current direction'={near start, info=$\eli_2$}] (6, 0) -- (0, 0)
                (3, 0) to [current direction={near start, info=$\eli$}, resistor={near end, info=$R$}] (3, 3);
        \draw [-{Latex},color=red] (1.2, 1.7) arc [start angle = 135, end angle = -160, radius = 0.6];
        \draw [-{Latex},color=blue] (4.2, 1.7) arc [start angle = 135, end angle = -160, radius = 0.6];
        \node [contact,color=green!71!black] (bottomc) at (3, 0) {};
        \node [below] (bottom) at (3, 0) {$2$};
        \node [above] (top) at (3, 3) {$1$};
    \end{tikzpicture}

    \begin{align*}
        &\begin{cases}
            {\color{red} \ele_1 = \eli_1 r_1 - \eli R}, \\
            {\color{blue} -\ele_2 = -\eli_2 r_2 + \eli R}, \\
            {\color{green!71!black} - \eli - \eli_1 - \eli_2 = 0 };
        \end{cases}
        \qquad \implies \qquad
        \begin{cases}
            \eli_1 = \frac{\ele_1 + \eli R}{r_1}, \\
            \eli_2 = \frac{\ele_2 + \eli R}{r_2}, \\
            \eli + \eli_1 + \eli_2 = 0;
        \end{cases} \implies \\
        &\implies
         \eli + \frac{\ele_1 + \eli R}{r_1:L} + \frac{\ele_2 + \eli R}{r_2:L} = 0, \\
        &\eli\cbr{ 1 + \frac R{r_1:L} + \frac R{r_2:L}} + \frac{\ele_1 }{r_1:L} + \frac{\ele_2 }{r_2:L} = 0, \\
        &\eli
            = - \frac{\frac{\ele_1 }{r_1:L} + \frac{\ele_2 }{r_2:L}}{ 1 + \frac R{r_1:L} + \frac R{r_2:L}}
            = - \frac{\frac{60\,\text{В}}{3\,\text{Ом}} + \frac{30\,\text{В}}{3\,\text{Ом}}}{ 1 + \frac{12\,\text{Ом}}{3\,\text{Ом}} + \frac{12\,\text{Ом}}{3\,\text{Ом}}}
            = - \frac{10}3\units{А}
            \approx -3{,}30\,\text{А}, \\
        &U  = \varphi_2 - \varphi_1 = \eli R
            = - \frac{\frac{\ele_1 }{r_1:L} + \frac{\ele_2 }{r_2:L}}{ 1 + \frac R{r_1:L} + \frac R{r_2:L}} R
            \approx -40{,}00\,\text{В}.
    \end{align*}
    Оба ответа отрицательны, потому что мы изначально «не угадали» с направлением тока.
    Расчёт же показал,
    что ток через резистор $R$ течёт в противоположную сторону: вниз на рисунке, а потенциал точки 1 больше потенциала точки 2,
    а электрический ток ожидаемо течёт из точки с большим потенциалов в точку с меньшим.

    Кстати, если продолжить расчёт и вычислить значения ещё двух токов (формулы для $\eli_1$ и $\eli_2$, куда подставлять, выписаны выше),
    то по их знакам можно будет понять: угадали ли мы с их направлением или нет.
}

\variantsplitter

\addpersonalvariant{Ислам Мунаев}

\tasknumber{1}%
\task{%
    На резистор сопротивлением $R = 30\,\text{Ом}$ подали напряжение $U = 240\,\text{В}$.
    Определите ток, который потечёт через резистор, и мощность, выделяющуюся на нём.
}
\answer{%
    \begin{align*}
    \eli &= \frac{U}{R} = \frac{240\,\text{В}}{30\,\text{Ом}} = 8{,}00\,\text{А},  \\
    P &= \frac{U^2}{R} = \frac{\sqr{240\,\text{В}}}{30\,\text{Ом}} = 1920{,}00\,\text{Вт}
    \end{align*}
}
\solutionspace{60pt}

\tasknumber{2}%
\task{%
    Через резистор сопротивлением $r = 5\,\text{Ом}$ протекает электрический ток $\eli = 10{,}00\,\text{А}$.
    Определите, чему равны напряжение на резисторе и мощность, выделяющаяся на нём.
}
\answer{%
    \begin{align*}
    U &= \eli r = 10{,}00\,\text{А} \cdot 5\,\text{Ом} = 50\,\text{В},  \\
    P &= \eli^2r = \sqr{10{,}00\,\text{А}} \cdot 5\,\text{Ом} = 500\,\text{Вт}
    \end{align*}
}
\solutionspace{60pt}

\tasknumber{3}%
\task{%
    Замкнутая электрическая цепь состоит из ЭДС $\ele = 3\,\text{В}$ и сопротивлением $r$
    и резистора $R = 10\,\text{Ом}$.
    Определите ток, протекающий в цепи.
    Какая тепловая энергия выделится на резисторе за время
    $\tau = 10\,\text{с}$? Какая работа будет совершена ЭДС за это время? Каков знак этой работы? Чему равен КПД цепи?
    Вычислите значения для 2 случаев: $r=0$ и $r = 60\,\text{Ом}$.
}
\answer{%
    \begin{align*}
    \eli_1 &= \frac{\ele}{R} = \frac{3\,\text{В}}{10\,\text{Ом}} = \frac3{10}\units{А} \approx 0{,}30\,\text{А},  \\
    \eli_2 &= \frac{\ele}{R + r} = \frac{3\,\text{В}}{10\,\text{Ом} + 60\,\text{Ом}} = \frac3{70}\units{А} \approx 0{,}04\,\text{А},  \\
    Q_1 &= \eli_1^2R\tau = \sqr{\frac{\ele}{R}} R \tau
            = \sqr{\frac{3\,\text{В}}{10\,\text{Ом}}} \cdot 10\,\text{Ом} \cdot 10\,\text{с} = 9\units{Дж} \approx 9{,}000\,\text{Дж},  \\
    Q_2 &= \eli_2^2R\tau = \sqr{\frac{\ele}{R + r}} R \tau
            = \sqr{\frac{3\,\text{В}}{10\,\text{Ом} + 60\,\text{Ом}}} \cdot 10\,\text{Ом} \cdot 10\,\text{с} = \frac9{49}\units{Дж} \approx 0{,}184\,\text{Дж},  \\
    A_1 &= q_1\ele = \eli_1\tau\ele = \frac{\ele}{R} \tau \ele
            = \frac{\ele^2 \tau}{R} = \frac{\sqr{3\,\text{В}} \cdot 10\,\text{с}}{10\,\text{Ом}}
            = 9\units{Дж} \approx 9{,}000\,\text{Дж}, \text{положительна},  \\
    A_2 &= q_2\ele = \eli_2\tau\ele = \frac{\ele}{R + r} \tau \ele
            = \frac{\ele^2 \tau}{R + r} = \frac{\sqr{3\,\text{В}} \cdot 10\,\text{с}}{10\,\text{Ом} + 60\,\text{Ом}}
            = \frac97\units{Дж} \approx 1{,}286\,\text{Дж}, \text{положительна},  \\
    \eta_1 &= \frac{Q_1}{A_1} = \ldots = \frac{R}{R} = 1,  \\
    \eta_2 &= \frac{Q_2}{A_2} = \ldots = \frac{R}{R + r} = \frac17 \approx 0{,}14.
    \end{align*}
}
\solutionspace{180pt}

\tasknumber{4}%
\task{%
    Лампочки, сопротивления которых $R_1 = 0{,}50\,\text{Ом}$ и $R_2 = 18{,}00\,\text{Ом}$, поочерёдно подключённные к некоторому источнику тока,
    потребляют одинаковую мощность.
    Найти внутреннее сопротивление источника и КПД цепи в каждом случае.
}
\answer{%
    \begin{align*}
        P_1 &= \sqr{\frac{\ele}{R_1 + r}}R_1,
        P_2  = \sqr{\frac{\ele}{R_2 + r}}R_2,
        P_1 = P_2 \implies  \\
        &\implies R_1 \sqr{R_2 + r} = R_2 \sqr{R_1 + r} \implies  \\
        &\implies R_1 R_2^2 + 2 R_1 R_2 r + R_1 r^2 =
                    R_2 R_1^2 + 2 R_2 R_1 r + R_2 r^2  \implies  \\
    &\implies r^2 (R_2 - R_1) = R_2^2 R_2 - R_1^2 R_2 \implies  \\
    &\implies r
            = \sqrt{R_1 R_2 \frac{R_2 - R_1}{R_2 - R_1}}
            = \sqrt{R_1 R_2}
            = \sqrt{0{,}50\,\text{Ом} \cdot 18{,}00\,\text{Ом}}
            = 3{,}0\,\text{Ом}.
            \\
    \eta_1
            &= \frac{R_1}{R_1 + r}
            = \frac{\sqrt{R_1}}{\sqrt{R_1} + \sqrt{R_2}}
            = 0{,}143,  \\
    \eta_2
            &= \frac{R_2}{R_2 + r}
            = \frac{\sqrt{R_2}}{\sqrt{R_2} + \sqrt{R_1}}
            = 0{,}857
    \end{align*}
}
\solutionspace{120pt}

\tasknumber{5}%
\task{%
    Определите ток, протекающий через резистор $R = 15\,\text{Ом}$ и разность потенциалов на нём (см.
    рис.
    на доске),
    если $r_1 = 1\,\text{Ом}$, $r_2 = 3\,\text{Ом}$, $\ele_1 = 20\,\text{В}$, $\ele_2 = 60\,\text{В}$.
}
\answer{%
    Обозначим на рисунке все токи: направление произвольно, но его надо зафиксировать.
    Всего на рисунке 3 контура и 2 узла.
    Поэтому можно записать $3 - 1 = 2$ уравнения законов Кирхгофа для замкнутого контура и $2 - 1 = 1$ — для узлов
    (остальные уравнения тоже можно записать, но они не дадут полезной информации, а будут лишь следствиями уже записанных).

    Отметим на рисунке 2 контура (и не забуем указать направление) и 1 узел (точка «1»ы, выделена жирным).
    Выбор контуров и узлов не критичен: получившаяся система может быть чуть проще или сложнее, но не слишком.

    \begin{tikzpicture}[circuit ee IEC, thick]
        \draw  (0, 0) to [current direction={near end, info=$\eli_1$}] (0, 3)
                to [battery={rotate=-180,info={$\ele_1, r_1$}}]
                (3, 3)
                to [battery={info'={$\ele_2, r_2$}}]
                (6, 3) to [current direction'={near start, info=$\eli_2$}] (6, 0) -- (0, 0)
                (3, 0) to [current direction={near start, info=$\eli$}, resistor={near end, info=$R$}] (3, 3);
        \draw [-{Latex},color=red] (1.2, 1.7) arc [start angle = 135, end angle = -160, radius = 0.6];
        \draw [-{Latex},color=blue] (4.2, 1.7) arc [start angle = 135, end angle = -160, radius = 0.6];
        \node [contact,color=green!71!black] (bottomc) at (3, 0) {};
        \node [below] (bottom) at (3, 0) {$2$};
        \node [above] (top) at (3, 3) {$1$};
    \end{tikzpicture}

    \begin{align*}
        &\begin{cases}
            {\color{red} \ele_1 = \eli_1 r_1 - \eli R}, \\
            {\color{blue} -\ele_2 = -\eli_2 r_2 + \eli R}, \\
            {\color{green!71!black} - \eli - \eli_1 - \eli_2 = 0 };
        \end{cases}
        \qquad \implies \qquad
        \begin{cases}
            \eli_1 = \frac{\ele_1 + \eli R}{r_1}, \\
            \eli_2 = \frac{\ele_2 + \eli R}{r_2}, \\
            \eli + \eli_1 + \eli_2 = 0;
        \end{cases} \implies \\
        &\implies
         \eli + \frac{\ele_1 + \eli R}{r_1:L} + \frac{\ele_2 + \eli R}{r_2:L} = 0, \\
        &\eli\cbr{ 1 + \frac R{r_1:L} + \frac R{r_2:L}} + \frac{\ele_1 }{r_1:L} + \frac{\ele_2 }{r_2:L} = 0, \\
        &\eli
            = - \frac{\frac{\ele_1 }{r_1:L} + \frac{\ele_2 }{r_2:L}}{ 1 + \frac R{r_1:L} + \frac R{r_2:L}}
            = - \frac{\frac{20\,\text{В}}{1\,\text{Ом}} + \frac{60\,\text{В}}{3\,\text{Ом}}}{ 1 + \frac{15\,\text{Ом}}{1\,\text{Ом}} + \frac{15\,\text{Ом}}{3\,\text{Ом}}}
            = - \frac{40}{21}\units{А}
            \approx -1{,}900\,\text{А}, \\
        &U  = \varphi_2 - \varphi_1 = \eli R
            = - \frac{\frac{\ele_1 }{r_1:L} + \frac{\ele_2 }{r_2:L}}{ 1 + \frac R{r_1:L} + \frac R{r_2:L}} R
            \approx -28{,}60\,\text{В}.
    \end{align*}
    Оба ответа отрицательны, потому что мы изначально «не угадали» с направлением тока.
    Расчёт же показал,
    что ток через резистор $R$ течёт в противоположную сторону: вниз на рисунке, а потенциал точки 1 больше потенциала точки 2,
    а электрический ток ожидаемо течёт из точки с большим потенциалов в точку с меньшим.

    Кстати, если продолжить расчёт и вычислить значения ещё двух токов (формулы для $\eli_1$ и $\eli_2$, куда подставлять, выписаны выше),
    то по их знакам можно будет понять: угадали ли мы с их направлением или нет.
}

\variantsplitter

\addpersonalvariant{Александр Наумов}

\tasknumber{1}%
\task{%
    На резистор сопротивлением $R = 5\,\text{Ом}$ подали напряжение $U = 150\,\text{В}$.
    Определите ток, который потечёт через резистор, и мощность, выделяющуюся на нём.
}
\answer{%
    \begin{align*}
    \eli &= \frac{U}{R} = \frac{150\,\text{В}}{5\,\text{Ом}} = 30{,}00\,\text{А},  \\
    P &= \frac{U^2}{R} = \frac{\sqr{150\,\text{В}}}{5\,\text{Ом}} = 4500{,}00\,\text{Вт}
    \end{align*}
}
\solutionspace{60pt}

\tasknumber{2}%
\task{%
    Через резистор сопротивлением $r = 12\,\text{Ом}$ протекает электрический ток $\eli = 15{,}00\,\text{А}$.
    Определите, чему равны напряжение на резисторе и мощность, выделяющаяся на нём.
}
\answer{%
    \begin{align*}
    U &= \eli r = 15{,}00\,\text{А} \cdot 12\,\text{Ом} = 180\,\text{В},  \\
    P &= \eli^2r = \sqr{15{,}00\,\text{А}} \cdot 12\,\text{Ом} = 2700\,\text{Вт}
    \end{align*}
}
\solutionspace{60pt}

\tasknumber{3}%
\task{%
    Замкнутая электрическая цепь состоит из ЭДС $\ele = 2\,\text{В}$ и сопротивлением $r$
    и резистора $R = 24\,\text{Ом}$.
    Определите ток, протекающий в цепи.
    Какая тепловая энергия выделится на резисторе за время
    $\tau = 2\,\text{с}$? Какая работа будет совершена ЭДС за это время? Каков знак этой работы? Чему равен КПД цепи?
    Вычислите значения для 2 случаев: $r=0$ и $r = 30\,\text{Ом}$.
}
\answer{%
    \begin{align*}
    \eli_1 &= \frac{\ele}{R} = \frac{2\,\text{В}}{24\,\text{Ом}} = \frac1{12}\units{А} \approx 0{,}08\,\text{А},  \\
    \eli_2 &= \frac{\ele}{R + r} = \frac{2\,\text{В}}{24\,\text{Ом} + 30\,\text{Ом}} = \frac1{27}\units{А} \approx 0{,}04\,\text{А},  \\
    Q_1 &= \eli_1^2R\tau = \sqr{\frac{\ele}{R}} R \tau
            = \sqr{\frac{2\,\text{В}}{24\,\text{Ом}}} \cdot 24\,\text{Ом} \cdot 2\,\text{с} = \frac13\units{Дж} \approx 0{,}333\,\text{Дж},  \\
    Q_2 &= \eli_2^2R\tau = \sqr{\frac{\ele}{R + r}} R \tau
            = \sqr{\frac{2\,\text{В}}{24\,\text{Ом} + 30\,\text{Ом}}} \cdot 24\,\text{Ом} \cdot 2\,\text{с} = \frac{16}{243}\units{Дж} \approx 0{,}066\,\text{Дж},  \\
    A_1 &= q_1\ele = \eli_1\tau\ele = \frac{\ele}{R} \tau \ele
            = \frac{\ele^2 \tau}{R} = \frac{\sqr{2\,\text{В}} \cdot 2\,\text{с}}{24\,\text{Ом}}
            = \frac13\units{Дж} \approx 0{,}333\,\text{Дж}, \text{положительна},  \\
    A_2 &= q_2\ele = \eli_2\tau\ele = \frac{\ele}{R + r} \tau \ele
            = \frac{\ele^2 \tau}{R + r} = \frac{\sqr{2\,\text{В}} \cdot 2\,\text{с}}{24\,\text{Ом} + 30\,\text{Ом}}
            = \frac4{27}\units{Дж} \approx 0{,}148\,\text{Дж}, \text{положительна},  \\
    \eta_1 &= \frac{Q_1}{A_1} = \ldots = \frac{R}{R} = 1,  \\
    \eta_2 &= \frac{Q_2}{A_2} = \ldots = \frac{R}{R + r} = \frac49 \approx 0{,}44.
    \end{align*}
}
\solutionspace{180pt}

\tasknumber{4}%
\task{%
    Лампочки, сопротивления которых $R_1 = 3{,}00\,\text{Ом}$ и $R_2 = 12{,}00\,\text{Ом}$, поочерёдно подключённные к некоторому источнику тока,
    потребляют одинаковую мощность.
    Найти внутреннее сопротивление источника и КПД цепи в каждом случае.
}
\answer{%
    \begin{align*}
        P_1 &= \sqr{\frac{\ele}{R_1 + r}}R_1,
        P_2  = \sqr{\frac{\ele}{R_2 + r}}R_2,
        P_1 = P_2 \implies  \\
        &\implies R_1 \sqr{R_2 + r} = R_2 \sqr{R_1 + r} \implies  \\
        &\implies R_1 R_2^2 + 2 R_1 R_2 r + R_1 r^2 =
                    R_2 R_1^2 + 2 R_2 R_1 r + R_2 r^2  \implies  \\
    &\implies r^2 (R_2 - R_1) = R_2^2 R_2 - R_1^2 R_2 \implies  \\
    &\implies r
            = \sqrt{R_1 R_2 \frac{R_2 - R_1}{R_2 - R_1}}
            = \sqrt{R_1 R_2}
            = \sqrt{3{,}00\,\text{Ом} \cdot 12{,}00\,\text{Ом}}
            = 6{,}0\,\text{Ом}.
            \\
    \eta_1
            &= \frac{R_1}{R_1 + r}
            = \frac{\sqrt{R_1}}{\sqrt{R_1} + \sqrt{R_2}}
            = 0{,}333,  \\
    \eta_2
            &= \frac{R_2}{R_2 + r}
            = \frac{\sqrt{R_2}}{\sqrt{R_2} + \sqrt{R_1}}
            = 0{,}667
    \end{align*}
}
\solutionspace{120pt}

\tasknumber{5}%
\task{%
    Определите ток, протекающий через резистор $R = 20\,\text{Ом}$ и разность потенциалов на нём (см.
    рис.
    на доске),
    если $r_1 = 2\,\text{Ом}$, $r_2 = 1\,\text{Ом}$, $\ele_1 = 60\,\text{В}$, $\ele_2 = 60\,\text{В}$.
}
\answer{%
    Обозначим на рисунке все токи: направление произвольно, но его надо зафиксировать.
    Всего на рисунке 3 контура и 2 узла.
    Поэтому можно записать $3 - 1 = 2$ уравнения законов Кирхгофа для замкнутого контура и $2 - 1 = 1$ — для узлов
    (остальные уравнения тоже можно записать, но они не дадут полезной информации, а будут лишь следствиями уже записанных).

    Отметим на рисунке 2 контура (и не забуем указать направление) и 1 узел (точка «1»ы, выделена жирным).
    Выбор контуров и узлов не критичен: получившаяся система может быть чуть проще или сложнее, но не слишком.

    \begin{tikzpicture}[circuit ee IEC, thick]
        \draw  (0, 0) to [current direction={near end, info=$\eli_1$}] (0, 3)
                to [battery={rotate=-180,info={$\ele_1, r_1$}}]
                (3, 3)
                to [battery={info'={$\ele_2, r_2$}}]
                (6, 3) to [current direction'={near start, info=$\eli_2$}] (6, 0) -- (0, 0)
                (3, 0) to [current direction={near start, info=$\eli$}, resistor={near end, info=$R$}] (3, 3);
        \draw [-{Latex},color=red] (1.2, 1.7) arc [start angle = 135, end angle = -160, radius = 0.6];
        \draw [-{Latex},color=blue] (4.2, 1.7) arc [start angle = 135, end angle = -160, radius = 0.6];
        \node [contact,color=green!71!black] (bottomc) at (3, 0) {};
        \node [below] (bottom) at (3, 0) {$2$};
        \node [above] (top) at (3, 3) {$1$};
    \end{tikzpicture}

    \begin{align*}
        &\begin{cases}
            {\color{red} \ele_1 = \eli_1 r_1 - \eli R}, \\
            {\color{blue} -\ele_2 = -\eli_2 r_2 + \eli R}, \\
            {\color{green!71!black} - \eli - \eli_1 - \eli_2 = 0 };
        \end{cases}
        \qquad \implies \qquad
        \begin{cases}
            \eli_1 = \frac{\ele_1 + \eli R}{r_1}, \\
            \eli_2 = \frac{\ele_2 + \eli R}{r_2}, \\
            \eli + \eli_1 + \eli_2 = 0;
        \end{cases} \implies \\
        &\implies
         \eli + \frac{\ele_1 + \eli R}{r_1:L} + \frac{\ele_2 + \eli R}{r_2:L} = 0, \\
        &\eli\cbr{ 1 + \frac R{r_1:L} + \frac R{r_2:L}} + \frac{\ele_1 }{r_1:L} + \frac{\ele_2 }{r_2:L} = 0, \\
        &\eli
            = - \frac{\frac{\ele_1 }{r_1:L} + \frac{\ele_2 }{r_2:L}}{ 1 + \frac R{r_1:L} + \frac R{r_2:L}}
            = - \frac{\frac{60\,\text{В}}{2\,\text{Ом}} + \frac{60\,\text{В}}{1\,\text{Ом}}}{ 1 + \frac{20\,\text{Ом}}{2\,\text{Ом}} + \frac{20\,\text{Ом}}{1\,\text{Ом}}}
            = - \frac{90}{31}\units{А}
            \approx -2{,}90\,\text{А}, \\
        &U  = \varphi_2 - \varphi_1 = \eli R
            = - \frac{\frac{\ele_1 }{r_1:L} + \frac{\ele_2 }{r_2:L}}{ 1 + \frac R{r_1:L} + \frac R{r_2:L}} R
            \approx -58{,}10\,\text{В}.
    \end{align*}
    Оба ответа отрицательны, потому что мы изначально «не угадали» с направлением тока.
    Расчёт же показал,
    что ток через резистор $R$ течёт в противоположную сторону: вниз на рисунке, а потенциал точки 1 больше потенциала точки 2,
    а электрический ток ожидаемо течёт из точки с большим потенциалов в точку с меньшим.

    Кстати, если продолжить расчёт и вычислить значения ещё двух токов (формулы для $\eli_1$ и $\eli_2$, куда подставлять, выписаны выше),
    то по их знакам можно будет понять: угадали ли мы с их направлением или нет.
}

\variantsplitter

\addpersonalvariant{Георгий Новиков}

\tasknumber{1}%
\task{%
    На резистор сопротивлением $r = 5\,\text{Ом}$ подали напряжение $V = 180\,\text{В}$.
    Определите ток, который потечёт через резистор, и мощность, выделяющуюся на нём.
}
\answer{%
    \begin{align*}
    \eli &= \frac{V}{r} = \frac{180\,\text{В}}{5\,\text{Ом}} = 36{,}00\,\text{А},  \\
    P &= \frac{V^2}{r} = \frac{\sqr{180\,\text{В}}}{5\,\text{Ом}} = 6480{,}00\,\text{Вт}
    \end{align*}
}
\solutionspace{60pt}

\tasknumber{2}%
\task{%
    Через резистор сопротивлением $R = 5\,\text{Ом}$ протекает электрический ток $\eli = 15{,}00\,\text{А}$.
    Определите, чему равны напряжение на резисторе и мощность, выделяющаяся на нём.
}
\answer{%
    \begin{align*}
    U &= \eli R = 15{,}00\,\text{А} \cdot 5\,\text{Ом} = 75\,\text{В},  \\
    P &= \eli^2R = \sqr{15{,}00\,\text{А}} \cdot 5\,\text{Ом} = 1125\,\text{Вт}
    \end{align*}
}
\solutionspace{60pt}

\tasknumber{3}%
\task{%
    Замкнутая электрическая цепь состоит из ЭДС $\ele = 4\,\text{В}$ и сопротивлением $r$
    и резистора $R = 30\,\text{Ом}$.
    Определите ток, протекающий в цепи.
    Какая тепловая энергия выделится на резисторе за время
    $\tau = 10\,\text{с}$? Какая работа будет совершена ЭДС за это время? Каков знак этой работы? Чему равен КПД цепи?
    Вычислите значения для 2 случаев: $r=0$ и $r = 20\,\text{Ом}$.
}
\answer{%
    \begin{align*}
    \eli_1 &= \frac{\ele}{R} = \frac{4\,\text{В}}{30\,\text{Ом}} = \frac2{15}\units{А} \approx 0{,}13\,\text{А},  \\
    \eli_2 &= \frac{\ele}{R + r} = \frac{4\,\text{В}}{30\,\text{Ом} + 20\,\text{Ом}} = \frac2{25}\units{А} \approx 0{,}08\,\text{А},  \\
    Q_1 &= \eli_1^2R\tau = \sqr{\frac{\ele}{R}} R \tau
            = \sqr{\frac{4\,\text{В}}{30\,\text{Ом}}} \cdot 30\,\text{Ом} \cdot 10\,\text{с} = \frac{16}3\units{Дж} \approx 5{,}333\,\text{Дж},  \\
    Q_2 &= \eli_2^2R\tau = \sqr{\frac{\ele}{R + r}} R \tau
            = \sqr{\frac{4\,\text{В}}{30\,\text{Ом} + 20\,\text{Ом}}} \cdot 30\,\text{Ом} \cdot 10\,\text{с} = \frac{48}{25}\units{Дж} \approx 1{,}920\,\text{Дж},  \\
    A_1 &= q_1\ele = \eli_1\tau\ele = \frac{\ele}{R} \tau \ele
            = \frac{\ele^2 \tau}{R} = \frac{\sqr{4\,\text{В}} \cdot 10\,\text{с}}{30\,\text{Ом}}
            = \frac{16}3\units{Дж} \approx 5{,}333\,\text{Дж}, \text{положительна},  \\
    A_2 &= q_2\ele = \eli_2\tau\ele = \frac{\ele}{R + r} \tau \ele
            = \frac{\ele^2 \tau}{R + r} = \frac{\sqr{4\,\text{В}} \cdot 10\,\text{с}}{30\,\text{Ом} + 20\,\text{Ом}}
            = \frac{16}5\units{Дж} \approx 3{,}200\,\text{Дж}, \text{положительна},  \\
    \eta_1 &= \frac{Q_1}{A_1} = \ldots = \frac{R}{R} = 1,  \\
    \eta_2 &= \frac{Q_2}{A_2} = \ldots = \frac{R}{R + r} = \frac35 \approx 0{,}60.
    \end{align*}
}
\solutionspace{180pt}

\tasknumber{4}%
\task{%
    Лампочки, сопротивления которых $R_1 = 1{,}00\,\text{Ом}$ и $R_2 = 9{,}00\,\text{Ом}$, поочерёдно подключённные к некоторому источнику тока,
    потребляют одинаковую мощность.
    Найти внутреннее сопротивление источника и КПД цепи в каждом случае.
}
\answer{%
    \begin{align*}
        P_1 &= \sqr{\frac{\ele}{R_1 + r}}R_1,
        P_2  = \sqr{\frac{\ele}{R_2 + r}}R_2,
        P_1 = P_2 \implies  \\
        &\implies R_1 \sqr{R_2 + r} = R_2 \sqr{R_1 + r} \implies  \\
        &\implies R_1 R_2^2 + 2 R_1 R_2 r + R_1 r^2 =
                    R_2 R_1^2 + 2 R_2 R_1 r + R_2 r^2  \implies  \\
    &\implies r^2 (R_2 - R_1) = R_2^2 R_2 - R_1^2 R_2 \implies  \\
    &\implies r
            = \sqrt{R_1 R_2 \frac{R_2 - R_1}{R_2 - R_1}}
            = \sqrt{R_1 R_2}
            = \sqrt{1{,}00\,\text{Ом} \cdot 9{,}00\,\text{Ом}}
            = 3{,}0\,\text{Ом}.
            \\
    \eta_1
            &= \frac{R_1}{R_1 + r}
            = \frac{\sqrt{R_1}}{\sqrt{R_1} + \sqrt{R_2}}
            = 0{,}250,  \\
    \eta_2
            &= \frac{R_2}{R_2 + r}
            = \frac{\sqrt{R_2}}{\sqrt{R_2} + \sqrt{R_1}}
            = 0{,}750
    \end{align*}
}
\solutionspace{120pt}

\tasknumber{5}%
\task{%
    Определите ток, протекающий через резистор $R = 12\,\text{Ом}$ и разность потенциалов на нём (см.
    рис.
    на доске),
    если $r_1 = 1\,\text{Ом}$, $r_2 = 3\,\text{Ом}$, $\ele_1 = 30\,\text{В}$, $\ele_2 = 30\,\text{В}$.
}
\answer{%
    Обозначим на рисунке все токи: направление произвольно, но его надо зафиксировать.
    Всего на рисунке 3 контура и 2 узла.
    Поэтому можно записать $3 - 1 = 2$ уравнения законов Кирхгофа для замкнутого контура и $2 - 1 = 1$ — для узлов
    (остальные уравнения тоже можно записать, но они не дадут полезной информации, а будут лишь следствиями уже записанных).

    Отметим на рисунке 2 контура (и не забуем указать направление) и 1 узел (точка «1»ы, выделена жирным).
    Выбор контуров и узлов не критичен: получившаяся система может быть чуть проще или сложнее, но не слишком.

    \begin{tikzpicture}[circuit ee IEC, thick]
        \draw  (0, 0) to [current direction={near end, info=$\eli_1$}] (0, 3)
                to [battery={rotate=-180,info={$\ele_1, r_1$}}]
                (3, 3)
                to [battery={info'={$\ele_2, r_2$}}]
                (6, 3) to [current direction'={near start, info=$\eli_2$}] (6, 0) -- (0, 0)
                (3, 0) to [current direction={near start, info=$\eli$}, resistor={near end, info=$R$}] (3, 3);
        \draw [-{Latex},color=red] (1.2, 1.7) arc [start angle = 135, end angle = -160, radius = 0.6];
        \draw [-{Latex},color=blue] (4.2, 1.7) arc [start angle = 135, end angle = -160, radius = 0.6];
        \node [contact,color=green!71!black] (bottomc) at (3, 0) {};
        \node [below] (bottom) at (3, 0) {$2$};
        \node [above] (top) at (3, 3) {$1$};
    \end{tikzpicture}

    \begin{align*}
        &\begin{cases}
            {\color{red} \ele_1 = \eli_1 r_1 - \eli R}, \\
            {\color{blue} -\ele_2 = -\eli_2 r_2 + \eli R}, \\
            {\color{green!71!black} - \eli - \eli_1 - \eli_2 = 0 };
        \end{cases}
        \qquad \implies \qquad
        \begin{cases}
            \eli_1 = \frac{\ele_1 + \eli R}{r_1}, \\
            \eli_2 = \frac{\ele_2 + \eli R}{r_2}, \\
            \eli + \eli_1 + \eli_2 = 0;
        \end{cases} \implies \\
        &\implies
         \eli + \frac{\ele_1 + \eli R}{r_1:L} + \frac{\ele_2 + \eli R}{r_2:L} = 0, \\
        &\eli\cbr{ 1 + \frac R{r_1:L} + \frac R{r_2:L}} + \frac{\ele_1 }{r_1:L} + \frac{\ele_2 }{r_2:L} = 0, \\
        &\eli
            = - \frac{\frac{\ele_1 }{r_1:L} + \frac{\ele_2 }{r_2:L}}{ 1 + \frac R{r_1:L} + \frac R{r_2:L}}
            = - \frac{\frac{30\,\text{В}}{1\,\text{Ом}} + \frac{30\,\text{В}}{3\,\text{Ом}}}{ 1 + \frac{12\,\text{Ом}}{1\,\text{Ом}} + \frac{12\,\text{Ом}}{3\,\text{Ом}}}
            = - \frac{40}{17}\units{А}
            \approx -2{,}40\,\text{А}, \\
        &U  = \varphi_2 - \varphi_1 = \eli R
            = - \frac{\frac{\ele_1 }{r_1:L} + \frac{\ele_2 }{r_2:L}}{ 1 + \frac R{r_1:L} + \frac R{r_2:L}} R
            \approx -28{,}20\,\text{В}.
    \end{align*}
    Оба ответа отрицательны, потому что мы изначально «не угадали» с направлением тока.
    Расчёт же показал,
    что ток через резистор $R$ течёт в противоположную сторону: вниз на рисунке, а потенциал точки 1 больше потенциала точки 2,
    а электрический ток ожидаемо течёт из точки с большим потенциалов в точку с меньшим.

    Кстати, если продолжить расчёт и вычислить значения ещё двух токов (формулы для $\eli_1$ и $\eli_2$, куда подставлять, выписаны выше),
    то по их знакам можно будет понять: угадали ли мы с их направлением или нет.
}

\variantsplitter

\addpersonalvariant{Егор Осипов}

\tasknumber{1}%
\task{%
    На резистор сопротивлением $R = 18\,\text{Ом}$ подали напряжение $U = 120\,\text{В}$.
    Определите ток, который потечёт через резистор, и мощность, выделяющуюся на нём.
}
\answer{%
    \begin{align*}
    \eli &= \frac{U}{R} = \frac{120\,\text{В}}{18\,\text{Ом}} = 6{,}67\,\text{А},  \\
    P &= \frac{U^2}{R} = \frac{\sqr{120\,\text{В}}}{18\,\text{Ом}} = 800{,}00\,\text{Вт}
    \end{align*}
}
\solutionspace{60pt}

\tasknumber{2}%
\task{%
    Через резистор сопротивлением $r = 18\,\text{Ом}$ протекает электрический ток $\eli = 6{,}00\,\text{А}$.
    Определите, чему равны напряжение на резисторе и мощность, выделяющаяся на нём.
}
\answer{%
    \begin{align*}
    U &= \eli r = 6{,}00\,\text{А} \cdot 18\,\text{Ом} = 108\,\text{В},  \\
    P &= \eli^2r = \sqr{6{,}00\,\text{А}} \cdot 18\,\text{Ом} = 648\,\text{Вт}
    \end{align*}
}
\solutionspace{60pt}

\tasknumber{3}%
\task{%
    Замкнутая электрическая цепь состоит из ЭДС $\ele = 1\,\text{В}$ и сопротивлением $r$
    и резистора $R = 30\,\text{Ом}$.
    Определите ток, протекающий в цепи.
    Какая тепловая энергия выделится на резисторе за время
    $\tau = 10\,\text{с}$? Какая работа будет совершена ЭДС за это время? Каков знак этой работы? Чему равен КПД цепи?
    Вычислите значения для 2 случаев: $r=0$ и $r = 60\,\text{Ом}$.
}
\answer{%
    \begin{align*}
    \eli_1 &= \frac{\ele}{R} = \frac{1\,\text{В}}{30\,\text{Ом}} = \frac1{30}\units{А} \approx 0{,}03\,\text{А},  \\
    \eli_2 &= \frac{\ele}{R + r} = \frac{1\,\text{В}}{30\,\text{Ом} + 60\,\text{Ом}} = \frac1{90}\units{А} \approx 0{,}010\,\text{А},  \\
    Q_1 &= \eli_1^2R\tau = \sqr{\frac{\ele}{R}} R \tau
            = \sqr{\frac{1\,\text{В}}{30\,\text{Ом}}} \cdot 30\,\text{Ом} \cdot 10\,\text{с} = \frac13\units{Дж} \approx 0{,}333\,\text{Дж},  \\
    Q_2 &= \eli_2^2R\tau = \sqr{\frac{\ele}{R + r}} R \tau
            = \sqr{\frac{1\,\text{В}}{30\,\text{Ом} + 60\,\text{Ом}}} \cdot 30\,\text{Ом} \cdot 10\,\text{с} = \frac1{27}\units{Дж} \approx 0{,}037\,\text{Дж},  \\
    A_1 &= q_1\ele = \eli_1\tau\ele = \frac{\ele}{R} \tau \ele
            = \frac{\ele^2 \tau}{R} = \frac{\sqr{1\,\text{В}} \cdot 10\,\text{с}}{30\,\text{Ом}}
            = \frac13\units{Дж} \approx 0{,}333\,\text{Дж}, \text{положительна},  \\
    A_2 &= q_2\ele = \eli_2\tau\ele = \frac{\ele}{R + r} \tau \ele
            = \frac{\ele^2 \tau}{R + r} = \frac{\sqr{1\,\text{В}} \cdot 10\,\text{с}}{30\,\text{Ом} + 60\,\text{Ом}}
            = \frac19\units{Дж} \approx 0{,}111\,\text{Дж}, \text{положительна},  \\
    \eta_1 &= \frac{Q_1}{A_1} = \ldots = \frac{R}{R} = 1,  \\
    \eta_2 &= \frac{Q_2}{A_2} = \ldots = \frac{R}{R + r} = \frac13 \approx 0{,}33.
    \end{align*}
}
\solutionspace{180pt}

\tasknumber{4}%
\task{%
    Лампочки, сопротивления которых $R_1 = 6{,}00\,\text{Ом}$ и $R_2 = 24{,}00\,\text{Ом}$, поочерёдно подключённные к некоторому источнику тока,
    потребляют одинаковую мощность.
    Найти внутреннее сопротивление источника и КПД цепи в каждом случае.
}
\answer{%
    \begin{align*}
        P_1 &= \sqr{\frac{\ele}{R_1 + r}}R_1,
        P_2  = \sqr{\frac{\ele}{R_2 + r}}R_2,
        P_1 = P_2 \implies  \\
        &\implies R_1 \sqr{R_2 + r} = R_2 \sqr{R_1 + r} \implies  \\
        &\implies R_1 R_2^2 + 2 R_1 R_2 r + R_1 r^2 =
                    R_2 R_1^2 + 2 R_2 R_1 r + R_2 r^2  \implies  \\
    &\implies r^2 (R_2 - R_1) = R_2^2 R_2 - R_1^2 R_2 \implies  \\
    &\implies r
            = \sqrt{R_1 R_2 \frac{R_2 - R_1}{R_2 - R_1}}
            = \sqrt{R_1 R_2}
            = \sqrt{6{,}00\,\text{Ом} \cdot 24{,}00\,\text{Ом}}
            = 12{,}0\,\text{Ом}.
            \\
    \eta_1
            &= \frac{R_1}{R_1 + r}
            = \frac{\sqrt{R_1}}{\sqrt{R_1} + \sqrt{R_2}}
            = 0{,}333,  \\
    \eta_2
            &= \frac{R_2}{R_2 + r}
            = \frac{\sqrt{R_2}}{\sqrt{R_2} + \sqrt{R_1}}
            = 0{,}667
    \end{align*}
}
\solutionspace{120pt}

\tasknumber{5}%
\task{%
    Определите ток, протекающий через резистор $R = 10\,\text{Ом}$ и разность потенциалов на нём (см.
    рис.
    на доске),
    если $r_1 = 1\,\text{Ом}$, $r_2 = 1\,\text{Ом}$, $\ele_1 = 60\,\text{В}$, $\ele_2 = 30\,\text{В}$.
}
\answer{%
    Обозначим на рисунке все токи: направление произвольно, но его надо зафиксировать.
    Всего на рисунке 3 контура и 2 узла.
    Поэтому можно записать $3 - 1 = 2$ уравнения законов Кирхгофа для замкнутого контура и $2 - 1 = 1$ — для узлов
    (остальные уравнения тоже можно записать, но они не дадут полезной информации, а будут лишь следствиями уже записанных).

    Отметим на рисунке 2 контура (и не забуем указать направление) и 1 узел (точка «1»ы, выделена жирным).
    Выбор контуров и узлов не критичен: получившаяся система может быть чуть проще или сложнее, но не слишком.

    \begin{tikzpicture}[circuit ee IEC, thick]
        \draw  (0, 0) to [current direction={near end, info=$\eli_1$}] (0, 3)
                to [battery={rotate=-180,info={$\ele_1, r_1$}}]
                (3, 3)
                to [battery={info'={$\ele_2, r_2$}}]
                (6, 3) to [current direction'={near start, info=$\eli_2$}] (6, 0) -- (0, 0)
                (3, 0) to [current direction={near start, info=$\eli$}, resistor={near end, info=$R$}] (3, 3);
        \draw [-{Latex},color=red] (1.2, 1.7) arc [start angle = 135, end angle = -160, radius = 0.6];
        \draw [-{Latex},color=blue] (4.2, 1.7) arc [start angle = 135, end angle = -160, radius = 0.6];
        \node [contact,color=green!71!black] (bottomc) at (3, 0) {};
        \node [below] (bottom) at (3, 0) {$2$};
        \node [above] (top) at (3, 3) {$1$};
    \end{tikzpicture}

    \begin{align*}
        &\begin{cases}
            {\color{red} \ele_1 = \eli_1 r_1 - \eli R}, \\
            {\color{blue} -\ele_2 = -\eli_2 r_2 + \eli R}, \\
            {\color{green!71!black} - \eli - \eli_1 - \eli_2 = 0 };
        \end{cases}
        \qquad \implies \qquad
        \begin{cases}
            \eli_1 = \frac{\ele_1 + \eli R}{r_1}, \\
            \eli_2 = \frac{\ele_2 + \eli R}{r_2}, \\
            \eli + \eli_1 + \eli_2 = 0;
        \end{cases} \implies \\
        &\implies
         \eli + \frac{\ele_1 + \eli R}{r_1:L} + \frac{\ele_2 + \eli R}{r_2:L} = 0, \\
        &\eli\cbr{ 1 + \frac R{r_1:L} + \frac R{r_2:L}} + \frac{\ele_1 }{r_1:L} + \frac{\ele_2 }{r_2:L} = 0, \\
        &\eli
            = - \frac{\frac{\ele_1 }{r_1:L} + \frac{\ele_2 }{r_2:L}}{ 1 + \frac R{r_1:L} + \frac R{r_2:L}}
            = - \frac{\frac{60\,\text{В}}{1\,\text{Ом}} + \frac{30\,\text{В}}{1\,\text{Ом}}}{ 1 + \frac{10\,\text{Ом}}{1\,\text{Ом}} + \frac{10\,\text{Ом}}{1\,\text{Ом}}}
            = - \frac{30}7\units{А}
            \approx -4{,}30\,\text{А}, \\
        &U  = \varphi_2 - \varphi_1 = \eli R
            = - \frac{\frac{\ele_1 }{r_1:L} + \frac{\ele_2 }{r_2:L}}{ 1 + \frac R{r_1:L} + \frac R{r_2:L}} R
            \approx -42{,}90\,\text{В}.
    \end{align*}
    Оба ответа отрицательны, потому что мы изначально «не угадали» с направлением тока.
    Расчёт же показал,
    что ток через резистор $R$ течёт в противоположную сторону: вниз на рисунке, а потенциал точки 1 больше потенциала точки 2,
    а электрический ток ожидаемо течёт из точки с большим потенциалов в точку с меньшим.

    Кстати, если продолжить расчёт и вычислить значения ещё двух токов (формулы для $\eli_1$ и $\eli_2$, куда подставлять, выписаны выше),
    то по их знакам можно будет понять: угадали ли мы с их направлением или нет.
}

\variantsplitter

\addpersonalvariant{Руслан Перепелица}

\tasknumber{1}%
\task{%
    На резистор сопротивлением $r = 5\,\text{Ом}$ подали напряжение $U = 240\,\text{В}$.
    Определите ток, который потечёт через резистор, и мощность, выделяющуюся на нём.
}
\answer{%
    \begin{align*}
    \eli &= \frac{U}{r} = \frac{240\,\text{В}}{5\,\text{Ом}} = 48{,}00\,\text{А},  \\
    P &= \frac{U^2}{r} = \frac{\sqr{240\,\text{В}}}{5\,\text{Ом}} = 11520{,}00\,\text{Вт}
    \end{align*}
}
\solutionspace{60pt}

\tasknumber{2}%
\task{%
    Через резистор сопротивлением $r = 5\,\text{Ом}$ протекает электрический ток $\eli = 4{,}00\,\text{А}$.
    Определите, чему равны напряжение на резисторе и мощность, выделяющаяся на нём.
}
\answer{%
    \begin{align*}
    U &= \eli r = 4{,}00\,\text{А} \cdot 5\,\text{Ом} = 20\,\text{В},  \\
    P &= \eli^2r = \sqr{4{,}00\,\text{А}} \cdot 5\,\text{Ом} = 80\,\text{Вт}
    \end{align*}
}
\solutionspace{60pt}

\tasknumber{3}%
\task{%
    Замкнутая электрическая цепь состоит из ЭДС $\ele = 1\,\text{В}$ и сопротивлением $r$
    и резистора $R = 30\,\text{Ом}$.
    Определите ток, протекающий в цепи.
    Какая тепловая энергия выделится на резисторе за время
    $\tau = 2\,\text{с}$? Какая работа будет совершена ЭДС за это время? Каков знак этой работы? Чему равен КПД цепи?
    Вычислите значения для 2 случаев: $r=0$ и $r = 30\,\text{Ом}$.
}
\answer{%
    \begin{align*}
    \eli_1 &= \frac{\ele}{R} = \frac{1\,\text{В}}{30\,\text{Ом}} = \frac1{30}\units{А} \approx 0{,}03\,\text{А},  \\
    \eli_2 &= \frac{\ele}{R + r} = \frac{1\,\text{В}}{30\,\text{Ом} + 30\,\text{Ом}} = \frac1{60}\units{А} \approx 0{,}02\,\text{А},  \\
    Q_1 &= \eli_1^2R\tau = \sqr{\frac{\ele}{R}} R \tau
            = \sqr{\frac{1\,\text{В}}{30\,\text{Ом}}} \cdot 30\,\text{Ом} \cdot 2\,\text{с} = \frac1{15}\units{Дж} \approx 0{,}067\,\text{Дж},  \\
    Q_2 &= \eli_2^2R\tau = \sqr{\frac{\ele}{R + r}} R \tau
            = \sqr{\frac{1\,\text{В}}{30\,\text{Ом} + 30\,\text{Ом}}} \cdot 30\,\text{Ом} \cdot 2\,\text{с} = \frac1{60}\units{Дж} \approx 0{,}017\,\text{Дж},  \\
    A_1 &= q_1\ele = \eli_1\tau\ele = \frac{\ele}{R} \tau \ele
            = \frac{\ele^2 \tau}{R} = \frac{\sqr{1\,\text{В}} \cdot 2\,\text{с}}{30\,\text{Ом}}
            = \frac1{15}\units{Дж} \approx 0{,}067\,\text{Дж}, \text{положительна},  \\
    A_2 &= q_2\ele = \eli_2\tau\ele = \frac{\ele}{R + r} \tau \ele
            = \frac{\ele^2 \tau}{R + r} = \frac{\sqr{1\,\text{В}} \cdot 2\,\text{с}}{30\,\text{Ом} + 30\,\text{Ом}}
            = \frac1{30}\units{Дж} \approx 0{,}033\,\text{Дж}, \text{положительна},  \\
    \eta_1 &= \frac{Q_1}{A_1} = \ldots = \frac{R}{R} = 1,  \\
    \eta_2 &= \frac{Q_2}{A_2} = \ldots = \frac{R}{R + r} = \frac12 \approx 0{,}50.
    \end{align*}
}
\solutionspace{180pt}

\tasknumber{4}%
\task{%
    Лампочки, сопротивления которых $R_1 = 0{,}25\,\text{Ом}$ и $R_2 = 4{,}00\,\text{Ом}$, поочерёдно подключённные к некоторому источнику тока,
    потребляют одинаковую мощность.
    Найти внутреннее сопротивление источника и КПД цепи в каждом случае.
}
\answer{%
    \begin{align*}
        P_1 &= \sqr{\frac{\ele}{R_1 + r}}R_1,
        P_2  = \sqr{\frac{\ele}{R_2 + r}}R_2,
        P_1 = P_2 \implies  \\
        &\implies R_1 \sqr{R_2 + r} = R_2 \sqr{R_1 + r} \implies  \\
        &\implies R_1 R_2^2 + 2 R_1 R_2 r + R_1 r^2 =
                    R_2 R_1^2 + 2 R_2 R_1 r + R_2 r^2  \implies  \\
    &\implies r^2 (R_2 - R_1) = R_2^2 R_2 - R_1^2 R_2 \implies  \\
    &\implies r
            = \sqrt{R_1 R_2 \frac{R_2 - R_1}{R_2 - R_1}}
            = \sqrt{R_1 R_2}
            = \sqrt{0{,}25\,\text{Ом} \cdot 4{,}00\,\text{Ом}}
            = 1{,}0\,\text{Ом}.
            \\
    \eta_1
            &= \frac{R_1}{R_1 + r}
            = \frac{\sqrt{R_1}}{\sqrt{R_1} + \sqrt{R_2}}
            = 0{,}200,  \\
    \eta_2
            &= \frac{R_2}{R_2 + r}
            = \frac{\sqrt{R_2}}{\sqrt{R_2} + \sqrt{R_1}}
            = 0{,}800
    \end{align*}
}
\solutionspace{120pt}

\tasknumber{5}%
\task{%
    Определите ток, протекающий через резистор $R = 18\,\text{Ом}$ и разность потенциалов на нём (см.
    рис.
    на доске),
    если $r_1 = 2\,\text{Ом}$, $r_2 = 2\,\text{Ом}$, $\ele_1 = 30\,\text{В}$, $\ele_2 = 30\,\text{В}$.
}
\answer{%
    Обозначим на рисунке все токи: направление произвольно, но его надо зафиксировать.
    Всего на рисунке 3 контура и 2 узла.
    Поэтому можно записать $3 - 1 = 2$ уравнения законов Кирхгофа для замкнутого контура и $2 - 1 = 1$ — для узлов
    (остальные уравнения тоже можно записать, но они не дадут полезной информации, а будут лишь следствиями уже записанных).

    Отметим на рисунке 2 контура (и не забуем указать направление) и 1 узел (точка «1»ы, выделена жирным).
    Выбор контуров и узлов не критичен: получившаяся система может быть чуть проще или сложнее, но не слишком.

    \begin{tikzpicture}[circuit ee IEC, thick]
        \draw  (0, 0) to [current direction={near end, info=$\eli_1$}] (0, 3)
                to [battery={rotate=-180,info={$\ele_1, r_1$}}]
                (3, 3)
                to [battery={info'={$\ele_2, r_2$}}]
                (6, 3) to [current direction'={near start, info=$\eli_2$}] (6, 0) -- (0, 0)
                (3, 0) to [current direction={near start, info=$\eli$}, resistor={near end, info=$R$}] (3, 3);
        \draw [-{Latex},color=red] (1.2, 1.7) arc [start angle = 135, end angle = -160, radius = 0.6];
        \draw [-{Latex},color=blue] (4.2, 1.7) arc [start angle = 135, end angle = -160, radius = 0.6];
        \node [contact,color=green!71!black] (bottomc) at (3, 0) {};
        \node [below] (bottom) at (3, 0) {$2$};
        \node [above] (top) at (3, 3) {$1$};
    \end{tikzpicture}

    \begin{align*}
        &\begin{cases}
            {\color{red} \ele_1 = \eli_1 r_1 - \eli R}, \\
            {\color{blue} -\ele_2 = -\eli_2 r_2 + \eli R}, \\
            {\color{green!71!black} - \eli - \eli_1 - \eli_2 = 0 };
        \end{cases}
        \qquad \implies \qquad
        \begin{cases}
            \eli_1 = \frac{\ele_1 + \eli R}{r_1}, \\
            \eli_2 = \frac{\ele_2 + \eli R}{r_2}, \\
            \eli + \eli_1 + \eli_2 = 0;
        \end{cases} \implies \\
        &\implies
         \eli + \frac{\ele_1 + \eli R}{r_1:L} + \frac{\ele_2 + \eli R}{r_2:L} = 0, \\
        &\eli\cbr{ 1 + \frac R{r_1:L} + \frac R{r_2:L}} + \frac{\ele_1 }{r_1:L} + \frac{\ele_2 }{r_2:L} = 0, \\
        &\eli
            = - \frac{\frac{\ele_1 }{r_1:L} + \frac{\ele_2 }{r_2:L}}{ 1 + \frac R{r_1:L} + \frac R{r_2:L}}
            = - \frac{\frac{30\,\text{В}}{2\,\text{Ом}} + \frac{30\,\text{В}}{2\,\text{Ом}}}{ 1 + \frac{18\,\text{Ом}}{2\,\text{Ом}} + \frac{18\,\text{Ом}}{2\,\text{Ом}}}
            = - \frac{30}{19}\units{А}
            \approx -1{,}600\,\text{А}, \\
        &U  = \varphi_2 - \varphi_1 = \eli R
            = - \frac{\frac{\ele_1 }{r_1:L} + \frac{\ele_2 }{r_2:L}}{ 1 + \frac R{r_1:L} + \frac R{r_2:L}} R
            \approx -28{,}40\,\text{В}.
    \end{align*}
    Оба ответа отрицательны, потому что мы изначально «не угадали» с направлением тока.
    Расчёт же показал,
    что ток через резистор $R$ течёт в противоположную сторону: вниз на рисунке, а потенциал точки 1 больше потенциала точки 2,
    а электрический ток ожидаемо течёт из точки с большим потенциалов в точку с меньшим.

    Кстати, если продолжить расчёт и вычислить значения ещё двух токов (формулы для $\eli_1$ и $\eli_2$, куда подставлять, выписаны выше),
    то по их знакам можно будет понять: угадали ли мы с их направлением или нет.
}

\variantsplitter

\addpersonalvariant{Михаил Перин}

\tasknumber{1}%
\task{%
    На резистор сопротивлением $R = 5\,\text{Ом}$ подали напряжение $U = 120\,\text{В}$.
    Определите ток, который потечёт через резистор, и мощность, выделяющуюся на нём.
}
\answer{%
    \begin{align*}
    \eli &= \frac{U}{R} = \frac{120\,\text{В}}{5\,\text{Ом}} = 24{,}00\,\text{А},  \\
    P &= \frac{U^2}{R} = \frac{\sqr{120\,\text{В}}}{5\,\text{Ом}} = 2880{,}00\,\text{Вт}
    \end{align*}
}
\solutionspace{60pt}

\tasknumber{2}%
\task{%
    Через резистор сопротивлением $R = 5\,\text{Ом}$ протекает электрический ток $\eli = 6{,}00\,\text{А}$.
    Определите, чему равны напряжение на резисторе и мощность, выделяющаяся на нём.
}
\answer{%
    \begin{align*}
    U &= \eli R = 6{,}00\,\text{А} \cdot 5\,\text{Ом} = 30\,\text{В},  \\
    P &= \eli^2R = \sqr{6{,}00\,\text{А}} \cdot 5\,\text{Ом} = 180\,\text{Вт}
    \end{align*}
}
\solutionspace{60pt}

\tasknumber{3}%
\task{%
    Замкнутая электрическая цепь состоит из ЭДС $\ele = 4\,\text{В}$ и сопротивлением $r$
    и резистора $R = 15\,\text{Ом}$.
    Определите ток, протекающий в цепи.
    Какая тепловая энергия выделится на резисторе за время
    $\tau = 2\,\text{с}$? Какая работа будет совершена ЭДС за это время? Каков знак этой работы? Чему равен КПД цепи?
    Вычислите значения для 2 случаев: $r=0$ и $r = 60\,\text{Ом}$.
}
\answer{%
    \begin{align*}
    \eli_1 &= \frac{\ele}{R} = \frac{4\,\text{В}}{15\,\text{Ом}} = \frac4{15}\units{А} \approx 0{,}27\,\text{А},  \\
    \eli_2 &= \frac{\ele}{R + r} = \frac{4\,\text{В}}{15\,\text{Ом} + 60\,\text{Ом}} = \frac4{75}\units{А} \approx 0{,}05\,\text{А},  \\
    Q_1 &= \eli_1^2R\tau = \sqr{\frac{\ele}{R}} R \tau
            = \sqr{\frac{4\,\text{В}}{15\,\text{Ом}}} \cdot 15\,\text{Ом} \cdot 2\,\text{с} = \frac{32}{15}\units{Дж} \approx 2{,}133\,\text{Дж},  \\
    Q_2 &= \eli_2^2R\tau = \sqr{\frac{\ele}{R + r}} R \tau
            = \sqr{\frac{4\,\text{В}}{15\,\text{Ом} + 60\,\text{Ом}}} \cdot 15\,\text{Ом} \cdot 2\,\text{с} = \frac{32}{375}\units{Дж} \approx 0{,}085\,\text{Дж},  \\
    A_1 &= q_1\ele = \eli_1\tau\ele = \frac{\ele}{R} \tau \ele
            = \frac{\ele^2 \tau}{R} = \frac{\sqr{4\,\text{В}} \cdot 2\,\text{с}}{15\,\text{Ом}}
            = \frac{32}{15}\units{Дж} \approx 2{,}133\,\text{Дж}, \text{положительна},  \\
    A_2 &= q_2\ele = \eli_2\tau\ele = \frac{\ele}{R + r} \tau \ele
            = \frac{\ele^2 \tau}{R + r} = \frac{\sqr{4\,\text{В}} \cdot 2\,\text{с}}{15\,\text{Ом} + 60\,\text{Ом}}
            = \frac{32}{75}\units{Дж} \approx 0{,}427\,\text{Дж}, \text{положительна},  \\
    \eta_1 &= \frac{Q_1}{A_1} = \ldots = \frac{R}{R} = 1,  \\
    \eta_2 &= \frac{Q_2}{A_2} = \ldots = \frac{R}{R + r} = \frac15 \approx 0{,}20.
    \end{align*}
}
\solutionspace{180pt}

\tasknumber{4}%
\task{%
    Лампочки, сопротивления которых $R_1 = 0{,}25\,\text{Ом}$ и $R_2 = 64{,}00\,\text{Ом}$, поочерёдно подключённные к некоторому источнику тока,
    потребляют одинаковую мощность.
    Найти внутреннее сопротивление источника и КПД цепи в каждом случае.
}
\answer{%
    \begin{align*}
        P_1 &= \sqr{\frac{\ele}{R_1 + r}}R_1,
        P_2  = \sqr{\frac{\ele}{R_2 + r}}R_2,
        P_1 = P_2 \implies  \\
        &\implies R_1 \sqr{R_2 + r} = R_2 \sqr{R_1 + r} \implies  \\
        &\implies R_1 R_2^2 + 2 R_1 R_2 r + R_1 r^2 =
                    R_2 R_1^2 + 2 R_2 R_1 r + R_2 r^2  \implies  \\
    &\implies r^2 (R_2 - R_1) = R_2^2 R_2 - R_1^2 R_2 \implies  \\
    &\implies r
            = \sqrt{R_1 R_2 \frac{R_2 - R_1}{R_2 - R_1}}
            = \sqrt{R_1 R_2}
            = \sqrt{0{,}25\,\text{Ом} \cdot 64{,}00\,\text{Ом}}
            = 4{,}0\,\text{Ом}.
            \\
    \eta_1
            &= \frac{R_1}{R_1 + r}
            = \frac{\sqrt{R_1}}{\sqrt{R_1} + \sqrt{R_2}}
            = 0{,}059,  \\
    \eta_2
            &= \frac{R_2}{R_2 + r}
            = \frac{\sqrt{R_2}}{\sqrt{R_2} + \sqrt{R_1}}
            = 0{,}941
    \end{align*}
}
\solutionspace{120pt}

\tasknumber{5}%
\task{%
    Определите ток, протекающий через резистор $R = 12\,\text{Ом}$ и разность потенциалов на нём (см.
    рис.
    на доске),
    если $r_1 = 1\,\text{Ом}$, $r_2 = 2\,\text{Ом}$, $\ele_1 = 30\,\text{В}$, $\ele_2 = 40\,\text{В}$.
}
\answer{%
    Обозначим на рисунке все токи: направление произвольно, но его надо зафиксировать.
    Всего на рисунке 3 контура и 2 узла.
    Поэтому можно записать $3 - 1 = 2$ уравнения законов Кирхгофа для замкнутого контура и $2 - 1 = 1$ — для узлов
    (остальные уравнения тоже можно записать, но они не дадут полезной информации, а будут лишь следствиями уже записанных).

    Отметим на рисунке 2 контура (и не забуем указать направление) и 1 узел (точка «1»ы, выделена жирным).
    Выбор контуров и узлов не критичен: получившаяся система может быть чуть проще или сложнее, но не слишком.

    \begin{tikzpicture}[circuit ee IEC, thick]
        \draw  (0, 0) to [current direction={near end, info=$\eli_1$}] (0, 3)
                to [battery={rotate=-180,info={$\ele_1, r_1$}}]
                (3, 3)
                to [battery={info'={$\ele_2, r_2$}}]
                (6, 3) to [current direction'={near start, info=$\eli_2$}] (6, 0) -- (0, 0)
                (3, 0) to [current direction={near start, info=$\eli$}, resistor={near end, info=$R$}] (3, 3);
        \draw [-{Latex},color=red] (1.2, 1.7) arc [start angle = 135, end angle = -160, radius = 0.6];
        \draw [-{Latex},color=blue] (4.2, 1.7) arc [start angle = 135, end angle = -160, radius = 0.6];
        \node [contact,color=green!71!black] (bottomc) at (3, 0) {};
        \node [below] (bottom) at (3, 0) {$2$};
        \node [above] (top) at (3, 3) {$1$};
    \end{tikzpicture}

    \begin{align*}
        &\begin{cases}
            {\color{red} \ele_1 = \eli_1 r_1 - \eli R}, \\
            {\color{blue} -\ele_2 = -\eli_2 r_2 + \eli R}, \\
            {\color{green!71!black} - \eli - \eli_1 - \eli_2 = 0 };
        \end{cases}
        \qquad \implies \qquad
        \begin{cases}
            \eli_1 = \frac{\ele_1 + \eli R}{r_1}, \\
            \eli_2 = \frac{\ele_2 + \eli R}{r_2}, \\
            \eli + \eli_1 + \eli_2 = 0;
        \end{cases} \implies \\
        &\implies
         \eli + \frac{\ele_1 + \eli R}{r_1:L} + \frac{\ele_2 + \eli R}{r_2:L} = 0, \\
        &\eli\cbr{ 1 + \frac R{r_1:L} + \frac R{r_2:L}} + \frac{\ele_1 }{r_1:L} + \frac{\ele_2 }{r_2:L} = 0, \\
        &\eli
            = - \frac{\frac{\ele_1 }{r_1:L} + \frac{\ele_2 }{r_2:L}}{ 1 + \frac R{r_1:L} + \frac R{r_2:L}}
            = - \frac{\frac{30\,\text{В}}{1\,\text{Ом}} + \frac{40\,\text{В}}{2\,\text{Ом}}}{ 1 + \frac{12\,\text{Ом}}{1\,\text{Ом}} + \frac{12\,\text{Ом}}{2\,\text{Ом}}}
            = - \frac{50}{19}\units{А}
            \approx -2{,}60\,\text{А}, \\
        &U  = \varphi_2 - \varphi_1 = \eli R
            = - \frac{\frac{\ele_1 }{r_1:L} + \frac{\ele_2 }{r_2:L}}{ 1 + \frac R{r_1:L} + \frac R{r_2:L}} R
            \approx -31{,}60\,\text{В}.
    \end{align*}
    Оба ответа отрицательны, потому что мы изначально «не угадали» с направлением тока.
    Расчёт же показал,
    что ток через резистор $R$ течёт в противоположную сторону: вниз на рисунке, а потенциал точки 1 больше потенциала точки 2,
    а электрический ток ожидаемо течёт из точки с большим потенциалов в точку с меньшим.

    Кстати, если продолжить расчёт и вычислить значения ещё двух токов (формулы для $\eli_1$ и $\eli_2$, куда подставлять, выписаны выше),
    то по их знакам можно будет понять: угадали ли мы с их направлением или нет.
}

\variantsplitter

\addpersonalvariant{Егор Подуровский}

\tasknumber{1}%
\task{%
    На резистор сопротивлением $R = 12\,\text{Ом}$ подали напряжение $V = 180\,\text{В}$.
    Определите ток, который потечёт через резистор, и мощность, выделяющуюся на нём.
}
\answer{%
    \begin{align*}
    \eli &= \frac{V}{R} = \frac{180\,\text{В}}{12\,\text{Ом}} = 15{,}00\,\text{А},  \\
    P &= \frac{V^2}{R} = \frac{\sqr{180\,\text{В}}}{12\,\text{Ом}} = 2700{,}00\,\text{Вт}
    \end{align*}
}
\solutionspace{60pt}

\tasknumber{2}%
\task{%
    Через резистор сопротивлением $r = 12\,\text{Ом}$ протекает электрический ток $\eli = 3{,}00\,\text{А}$.
    Определите, чему равны напряжение на резисторе и мощность, выделяющаяся на нём.
}
\answer{%
    \begin{align*}
    U &= \eli r = 3{,}00\,\text{А} \cdot 12\,\text{Ом} = 36\,\text{В},  \\
    P &= \eli^2r = \sqr{3{,}00\,\text{А}} \cdot 12\,\text{Ом} = 108\,\text{Вт}
    \end{align*}
}
\solutionspace{60pt}

\tasknumber{3}%
\task{%
    Замкнутая электрическая цепь состоит из ЭДС $\ele = 3\,\text{В}$ и сопротивлением $r$
    и резистора $R = 30\,\text{Ом}$.
    Определите ток, протекающий в цепи.
    Какая тепловая энергия выделится на резисторе за время
    $\tau = 2\,\text{с}$? Какая работа будет совершена ЭДС за это время? Каков знак этой работы? Чему равен КПД цепи?
    Вычислите значения для 2 случаев: $r=0$ и $r = 60\,\text{Ом}$.
}
\answer{%
    \begin{align*}
    \eli_1 &= \frac{\ele}{R} = \frac{3\,\text{В}}{30\,\text{Ом}} = \frac1{10}\units{А} \approx 0{,}10\,\text{А},  \\
    \eli_2 &= \frac{\ele}{R + r} = \frac{3\,\text{В}}{30\,\text{Ом} + 60\,\text{Ом}} = \frac1{30}\units{А} \approx 0{,}03\,\text{А},  \\
    Q_1 &= \eli_1^2R\tau = \sqr{\frac{\ele}{R}} R \tau
            = \sqr{\frac{3\,\text{В}}{30\,\text{Ом}}} \cdot 30\,\text{Ом} \cdot 2\,\text{с} = \frac35\units{Дж} \approx 0{,}600\,\text{Дж},  \\
    Q_2 &= \eli_2^2R\tau = \sqr{\frac{\ele}{R + r}} R \tau
            = \sqr{\frac{3\,\text{В}}{30\,\text{Ом} + 60\,\text{Ом}}} \cdot 30\,\text{Ом} \cdot 2\,\text{с} = \frac1{15}\units{Дж} \approx 0{,}067\,\text{Дж},  \\
    A_1 &= q_1\ele = \eli_1\tau\ele = \frac{\ele}{R} \tau \ele
            = \frac{\ele^2 \tau}{R} = \frac{\sqr{3\,\text{В}} \cdot 2\,\text{с}}{30\,\text{Ом}}
            = \frac35\units{Дж} \approx 0{,}600\,\text{Дж}, \text{положительна},  \\
    A_2 &= q_2\ele = \eli_2\tau\ele = \frac{\ele}{R + r} \tau \ele
            = \frac{\ele^2 \tau}{R + r} = \frac{\sqr{3\,\text{В}} \cdot 2\,\text{с}}{30\,\text{Ом} + 60\,\text{Ом}}
            = \frac15\units{Дж} \approx 0{,}200\,\text{Дж}, \text{положительна},  \\
    \eta_1 &= \frac{Q_1}{A_1} = \ldots = \frac{R}{R} = 1,  \\
    \eta_2 &= \frac{Q_2}{A_2} = \ldots = \frac{R}{R + r} = \frac13 \approx 0{,}33.
    \end{align*}
}
\solutionspace{180pt}

\tasknumber{4}%
\task{%
    Лампочки, сопротивления которых $R_1 = 0{,}25\,\text{Ом}$ и $R_2 = 16{,}00\,\text{Ом}$, поочерёдно подключённные к некоторому источнику тока,
    потребляют одинаковую мощность.
    Найти внутреннее сопротивление источника и КПД цепи в каждом случае.
}
\answer{%
    \begin{align*}
        P_1 &= \sqr{\frac{\ele}{R_1 + r}}R_1,
        P_2  = \sqr{\frac{\ele}{R_2 + r}}R_2,
        P_1 = P_2 \implies  \\
        &\implies R_1 \sqr{R_2 + r} = R_2 \sqr{R_1 + r} \implies  \\
        &\implies R_1 R_2^2 + 2 R_1 R_2 r + R_1 r^2 =
                    R_2 R_1^2 + 2 R_2 R_1 r + R_2 r^2  \implies  \\
    &\implies r^2 (R_2 - R_1) = R_2^2 R_2 - R_1^2 R_2 \implies  \\
    &\implies r
            = \sqrt{R_1 R_2 \frac{R_2 - R_1}{R_2 - R_1}}
            = \sqrt{R_1 R_2}
            = \sqrt{0{,}25\,\text{Ом} \cdot 16{,}00\,\text{Ом}}
            = 2{,}0\,\text{Ом}.
            \\
    \eta_1
            &= \frac{R_1}{R_1 + r}
            = \frac{\sqrt{R_1}}{\sqrt{R_1} + \sqrt{R_2}}
            = 0{,}111,  \\
    \eta_2
            &= \frac{R_2}{R_2 + r}
            = \frac{\sqrt{R_2}}{\sqrt{R_2} + \sqrt{R_1}}
            = 0{,}889
    \end{align*}
}
\solutionspace{120pt}

\tasknumber{5}%
\task{%
    Определите ток, протекающий через резистор $R = 12\,\text{Ом}$ и разность потенциалов на нём (см.
    рис.
    на доске),
    если $r_1 = 1\,\text{Ом}$, $r_2 = 2\,\text{Ом}$, $\ele_1 = 60\,\text{В}$, $\ele_2 = 60\,\text{В}$.
}
\answer{%
    Обозначим на рисунке все токи: направление произвольно, но его надо зафиксировать.
    Всего на рисунке 3 контура и 2 узла.
    Поэтому можно записать $3 - 1 = 2$ уравнения законов Кирхгофа для замкнутого контура и $2 - 1 = 1$ — для узлов
    (остальные уравнения тоже можно записать, но они не дадут полезной информации, а будут лишь следствиями уже записанных).

    Отметим на рисунке 2 контура (и не забуем указать направление) и 1 узел (точка «1»ы, выделена жирным).
    Выбор контуров и узлов не критичен: получившаяся система может быть чуть проще или сложнее, но не слишком.

    \begin{tikzpicture}[circuit ee IEC, thick]
        \draw  (0, 0) to [current direction={near end, info=$\eli_1$}] (0, 3)
                to [battery={rotate=-180,info={$\ele_1, r_1$}}]
                (3, 3)
                to [battery={info'={$\ele_2, r_2$}}]
                (6, 3) to [current direction'={near start, info=$\eli_2$}] (6, 0) -- (0, 0)
                (3, 0) to [current direction={near start, info=$\eli$}, resistor={near end, info=$R$}] (3, 3);
        \draw [-{Latex},color=red] (1.2, 1.7) arc [start angle = 135, end angle = -160, radius = 0.6];
        \draw [-{Latex},color=blue] (4.2, 1.7) arc [start angle = 135, end angle = -160, radius = 0.6];
        \node [contact,color=green!71!black] (bottomc) at (3, 0) {};
        \node [below] (bottom) at (3, 0) {$2$};
        \node [above] (top) at (3, 3) {$1$};
    \end{tikzpicture}

    \begin{align*}
        &\begin{cases}
            {\color{red} \ele_1 = \eli_1 r_1 - \eli R}, \\
            {\color{blue} -\ele_2 = -\eli_2 r_2 + \eli R}, \\
            {\color{green!71!black} - \eli - \eli_1 - \eli_2 = 0 };
        \end{cases}
        \qquad \implies \qquad
        \begin{cases}
            \eli_1 = \frac{\ele_1 + \eli R}{r_1}, \\
            \eli_2 = \frac{\ele_2 + \eli R}{r_2}, \\
            \eli + \eli_1 + \eli_2 = 0;
        \end{cases} \implies \\
        &\implies
         \eli + \frac{\ele_1 + \eli R}{r_1:L} + \frac{\ele_2 + \eli R}{r_2:L} = 0, \\
        &\eli\cbr{ 1 + \frac R{r_1:L} + \frac R{r_2:L}} + \frac{\ele_1 }{r_1:L} + \frac{\ele_2 }{r_2:L} = 0, \\
        &\eli
            = - \frac{\frac{\ele_1 }{r_1:L} + \frac{\ele_2 }{r_2:L}}{ 1 + \frac R{r_1:L} + \frac R{r_2:L}}
            = - \frac{\frac{60\,\text{В}}{1\,\text{Ом}} + \frac{60\,\text{В}}{2\,\text{Ом}}}{ 1 + \frac{12\,\text{Ом}}{1\,\text{Ом}} + \frac{12\,\text{Ом}}{2\,\text{Ом}}}
            = - \frac{90}{19}\units{А}
            \approx -4{,}70\,\text{А}, \\
        &U  = \varphi_2 - \varphi_1 = \eli R
            = - \frac{\frac{\ele_1 }{r_1:L} + \frac{\ele_2 }{r_2:L}}{ 1 + \frac R{r_1:L} + \frac R{r_2:L}} R
            \approx -56{,}80\,\text{В}.
    \end{align*}
    Оба ответа отрицательны, потому что мы изначально «не угадали» с направлением тока.
    Расчёт же показал,
    что ток через резистор $R$ течёт в противоположную сторону: вниз на рисунке, а потенциал точки 1 больше потенциала точки 2,
    а электрический ток ожидаемо течёт из точки с большим потенциалов в точку с меньшим.

    Кстати, если продолжить расчёт и вычислить значения ещё двух токов (формулы для $\eli_1$ и $\eli_2$, куда подставлять, выписаны выше),
    то по их знакам можно будет понять: угадали ли мы с их направлением или нет.
}

\variantsplitter

\addpersonalvariant{Роман Прибылов}

\tasknumber{1}%
\task{%
    На резистор сопротивлением $r = 18\,\text{Ом}$ подали напряжение $U = 120\,\text{В}$.
    Определите ток, который потечёт через резистор, и мощность, выделяющуюся на нём.
}
\answer{%
    \begin{align*}
    \eli &= \frac{U}{r} = \frac{120\,\text{В}}{18\,\text{Ом}} = 6{,}67\,\text{А},  \\
    P &= \frac{U^2}{r} = \frac{\sqr{120\,\text{В}}}{18\,\text{Ом}} = 800{,}00\,\text{Вт}
    \end{align*}
}
\solutionspace{60pt}

\tasknumber{2}%
\task{%
    Через резистор сопротивлением $r = 18\,\text{Ом}$ протекает электрический ток $\eli = 4{,}00\,\text{А}$.
    Определите, чему равны напряжение на резисторе и мощность, выделяющаяся на нём.
}
\answer{%
    \begin{align*}
    U &= \eli r = 4{,}00\,\text{А} \cdot 18\,\text{Ом} = 72\,\text{В},  \\
    P &= \eli^2r = \sqr{4{,}00\,\text{А}} \cdot 18\,\text{Ом} = 288\,\text{Вт}
    \end{align*}
}
\solutionspace{60pt}

\tasknumber{3}%
\task{%
    Замкнутая электрическая цепь состоит из ЭДС $\ele = 1\,\text{В}$ и сопротивлением $r$
    и резистора $R = 24\,\text{Ом}$.
    Определите ток, протекающий в цепи.
    Какая тепловая энергия выделится на резисторе за время
    $\tau = 5\,\text{с}$? Какая работа будет совершена ЭДС за это время? Каков знак этой работы? Чему равен КПД цепи?
    Вычислите значения для 2 случаев: $r=0$ и $r = 20\,\text{Ом}$.
}
\answer{%
    \begin{align*}
    \eli_1 &= \frac{\ele}{R} = \frac{1\,\text{В}}{24\,\text{Ом}} = \frac1{24}\units{А} \approx 0{,}04\,\text{А},  \\
    \eli_2 &= \frac{\ele}{R + r} = \frac{1\,\text{В}}{24\,\text{Ом} + 20\,\text{Ом}} = \frac1{44}\units{А} \approx 0{,}02\,\text{А},  \\
    Q_1 &= \eli_1^2R\tau = \sqr{\frac{\ele}{R}} R \tau
            = \sqr{\frac{1\,\text{В}}{24\,\text{Ом}}} \cdot 24\,\text{Ом} \cdot 5\,\text{с} = \frac5{24}\units{Дж} \approx 0{,}208\,\text{Дж},  \\
    Q_2 &= \eli_2^2R\tau = \sqr{\frac{\ele}{R + r}} R \tau
            = \sqr{\frac{1\,\text{В}}{24\,\text{Ом} + 20\,\text{Ом}}} \cdot 24\,\text{Ом} \cdot 5\,\text{с} = \frac{15}{242}\units{Дж} \approx 0{,}062\,\text{Дж},  \\
    A_1 &= q_1\ele = \eli_1\tau\ele = \frac{\ele}{R} \tau \ele
            = \frac{\ele^2 \tau}{R} = \frac{\sqr{1\,\text{В}} \cdot 5\,\text{с}}{24\,\text{Ом}}
            = \frac5{24}\units{Дж} \approx 0{,}208\,\text{Дж}, \text{положительна},  \\
    A_2 &= q_2\ele = \eli_2\tau\ele = \frac{\ele}{R + r} \tau \ele
            = \frac{\ele^2 \tau}{R + r} = \frac{\sqr{1\,\text{В}} \cdot 5\,\text{с}}{24\,\text{Ом} + 20\,\text{Ом}}
            = \frac5{44}\units{Дж} \approx 0{,}114\,\text{Дж}, \text{положительна},  \\
    \eta_1 &= \frac{Q_1}{A_1} = \ldots = \frac{R}{R} = 1,  \\
    \eta_2 &= \frac{Q_2}{A_2} = \ldots = \frac{R}{R + r} = \frac6{11} \approx 0{,}55.
    \end{align*}
}
\solutionspace{180pt}

\tasknumber{4}%
\task{%
    Лампочки, сопротивления которых $R_1 = 1{,}00\,\text{Ом}$ и $R_2 = 4{,}00\,\text{Ом}$, поочерёдно подключённные к некоторому источнику тока,
    потребляют одинаковую мощность.
    Найти внутреннее сопротивление источника и КПД цепи в каждом случае.
}
\answer{%
    \begin{align*}
        P_1 &= \sqr{\frac{\ele}{R_1 + r}}R_1,
        P_2  = \sqr{\frac{\ele}{R_2 + r}}R_2,
        P_1 = P_2 \implies  \\
        &\implies R_1 \sqr{R_2 + r} = R_2 \sqr{R_1 + r} \implies  \\
        &\implies R_1 R_2^2 + 2 R_1 R_2 r + R_1 r^2 =
                    R_2 R_1^2 + 2 R_2 R_1 r + R_2 r^2  \implies  \\
    &\implies r^2 (R_2 - R_1) = R_2^2 R_2 - R_1^2 R_2 \implies  \\
    &\implies r
            = \sqrt{R_1 R_2 \frac{R_2 - R_1}{R_2 - R_1}}
            = \sqrt{R_1 R_2}
            = \sqrt{1{,}00\,\text{Ом} \cdot 4{,}00\,\text{Ом}}
            = 2{,}0\,\text{Ом}.
            \\
    \eta_1
            &= \frac{R_1}{R_1 + r}
            = \frac{\sqrt{R_1}}{\sqrt{R_1} + \sqrt{R_2}}
            = 0{,}333,  \\
    \eta_2
            &= \frac{R_2}{R_2 + r}
            = \frac{\sqrt{R_2}}{\sqrt{R_2} + \sqrt{R_1}}
            = 0{,}667
    \end{align*}
}
\solutionspace{120pt}

\tasknumber{5}%
\task{%
    Определите ток, протекающий через резистор $R = 10\,\text{Ом}$ и разность потенциалов на нём (см.
    рис.
    на доске),
    если $r_1 = 1\,\text{Ом}$, $r_2 = 2\,\text{Ом}$, $\ele_1 = 20\,\text{В}$, $\ele_2 = 60\,\text{В}$.
}
\answer{%
    Обозначим на рисунке все токи: направление произвольно, но его надо зафиксировать.
    Всего на рисунке 3 контура и 2 узла.
    Поэтому можно записать $3 - 1 = 2$ уравнения законов Кирхгофа для замкнутого контура и $2 - 1 = 1$ — для узлов
    (остальные уравнения тоже можно записать, но они не дадут полезной информации, а будут лишь следствиями уже записанных).

    Отметим на рисунке 2 контура (и не забуем указать направление) и 1 узел (точка «1»ы, выделена жирным).
    Выбор контуров и узлов не критичен: получившаяся система может быть чуть проще или сложнее, но не слишком.

    \begin{tikzpicture}[circuit ee IEC, thick]
        \draw  (0, 0) to [current direction={near end, info=$\eli_1$}] (0, 3)
                to [battery={rotate=-180,info={$\ele_1, r_1$}}]
                (3, 3)
                to [battery={info'={$\ele_2, r_2$}}]
                (6, 3) to [current direction'={near start, info=$\eli_2$}] (6, 0) -- (0, 0)
                (3, 0) to [current direction={near start, info=$\eli$}, resistor={near end, info=$R$}] (3, 3);
        \draw [-{Latex},color=red] (1.2, 1.7) arc [start angle = 135, end angle = -160, radius = 0.6];
        \draw [-{Latex},color=blue] (4.2, 1.7) arc [start angle = 135, end angle = -160, radius = 0.6];
        \node [contact,color=green!71!black] (bottomc) at (3, 0) {};
        \node [below] (bottom) at (3, 0) {$2$};
        \node [above] (top) at (3, 3) {$1$};
    \end{tikzpicture}

    \begin{align*}
        &\begin{cases}
            {\color{red} \ele_1 = \eli_1 r_1 - \eli R}, \\
            {\color{blue} -\ele_2 = -\eli_2 r_2 + \eli R}, \\
            {\color{green!71!black} - \eli - \eli_1 - \eli_2 = 0 };
        \end{cases}
        \qquad \implies \qquad
        \begin{cases}
            \eli_1 = \frac{\ele_1 + \eli R}{r_1}, \\
            \eli_2 = \frac{\ele_2 + \eli R}{r_2}, \\
            \eli + \eli_1 + \eli_2 = 0;
        \end{cases} \implies \\
        &\implies
         \eli + \frac{\ele_1 + \eli R}{r_1:L} + \frac{\ele_2 + \eli R}{r_2:L} = 0, \\
        &\eli\cbr{ 1 + \frac R{r_1:L} + \frac R{r_2:L}} + \frac{\ele_1 }{r_1:L} + \frac{\ele_2 }{r_2:L} = 0, \\
        &\eli
            = - \frac{\frac{\ele_1 }{r_1:L} + \frac{\ele_2 }{r_2:L}}{ 1 + \frac R{r_1:L} + \frac R{r_2:L}}
            = - \frac{\frac{20\,\text{В}}{1\,\text{Ом}} + \frac{60\,\text{В}}{2\,\text{Ом}}}{ 1 + \frac{10\,\text{Ом}}{1\,\text{Ом}} + \frac{10\,\text{Ом}}{2\,\text{Ом}}}
            = - \frac{25}8\units{А}
            \approx -3{,}10\,\text{А}, \\
        &U  = \varphi_2 - \varphi_1 = \eli R
            = - \frac{\frac{\ele_1 }{r_1:L} + \frac{\ele_2 }{r_2:L}}{ 1 + \frac R{r_1:L} + \frac R{r_2:L}} R
            \approx -31{,}20\,\text{В}.
    \end{align*}
    Оба ответа отрицательны, потому что мы изначально «не угадали» с направлением тока.
    Расчёт же показал,
    что ток через резистор $R$ течёт в противоположную сторону: вниз на рисунке, а потенциал точки 1 больше потенциала точки 2,
    а электрический ток ожидаемо течёт из точки с большим потенциалов в точку с меньшим.

    Кстати, если продолжить расчёт и вычислить значения ещё двух токов (формулы для $\eli_1$ и $\eli_2$, куда подставлять, выписаны выше),
    то по их знакам можно будет понять: угадали ли мы с их направлением или нет.
}

\variantsplitter

\addpersonalvariant{Александр Селехметьев}

\tasknumber{1}%
\task{%
    На резистор сопротивлением $r = 12\,\text{Ом}$ подали напряжение $U = 150\,\text{В}$.
    Определите ток, который потечёт через резистор, и мощность, выделяющуюся на нём.
}
\answer{%
    \begin{align*}
    \eli &= \frac{U}{r} = \frac{150\,\text{В}}{12\,\text{Ом}} = 12{,}50\,\text{А},  \\
    P &= \frac{U^2}{r} = \frac{\sqr{150\,\text{В}}}{12\,\text{Ом}} = 1875{,}00\,\text{Вт}
    \end{align*}
}
\solutionspace{60pt}

\tasknumber{2}%
\task{%
    Через резистор сопротивлением $r = 18\,\text{Ом}$ протекает электрический ток $\eli = 2{,}00\,\text{А}$.
    Определите, чему равны напряжение на резисторе и мощность, выделяющаяся на нём.
}
\answer{%
    \begin{align*}
    U &= \eli r = 2{,}00\,\text{А} \cdot 18\,\text{Ом} = 36\,\text{В},  \\
    P &= \eli^2r = \sqr{2{,}00\,\text{А}} \cdot 18\,\text{Ом} = 72\,\text{Вт}
    \end{align*}
}
\solutionspace{60pt}

\tasknumber{3}%
\task{%
    Замкнутая электрическая цепь состоит из ЭДС $\ele = 3\,\text{В}$ и сопротивлением $r$
    и резистора $R = 10\,\text{Ом}$.
    Определите ток, протекающий в цепи.
    Какая тепловая энергия выделится на резисторе за время
    $\tau = 5\,\text{с}$? Какая работа будет совершена ЭДС за это время? Каков знак этой работы? Чему равен КПД цепи?
    Вычислите значения для 2 случаев: $r=0$ и $r = 30\,\text{Ом}$.
}
\answer{%
    \begin{align*}
    \eli_1 &= \frac{\ele}{R} = \frac{3\,\text{В}}{10\,\text{Ом}} = \frac3{10}\units{А} \approx 0{,}30\,\text{А},  \\
    \eli_2 &= \frac{\ele}{R + r} = \frac{3\,\text{В}}{10\,\text{Ом} + 30\,\text{Ом}} = \frac3{40}\units{А} \approx 0{,}07\,\text{А},  \\
    Q_1 &= \eli_1^2R\tau = \sqr{\frac{\ele}{R}} R \tau
            = \sqr{\frac{3\,\text{В}}{10\,\text{Ом}}} \cdot 10\,\text{Ом} \cdot 5\,\text{с} = \frac92\units{Дж} \approx 4{,}500\,\text{Дж},  \\
    Q_2 &= \eli_2^2R\tau = \sqr{\frac{\ele}{R + r}} R \tau
            = \sqr{\frac{3\,\text{В}}{10\,\text{Ом} + 30\,\text{Ом}}} \cdot 10\,\text{Ом} \cdot 5\,\text{с} = \frac9{32}\units{Дж} \approx 0{,}281\,\text{Дж},  \\
    A_1 &= q_1\ele = \eli_1\tau\ele = \frac{\ele}{R} \tau \ele
            = \frac{\ele^2 \tau}{R} = \frac{\sqr{3\,\text{В}} \cdot 5\,\text{с}}{10\,\text{Ом}}
            = \frac92\units{Дж} \approx 4{,}500\,\text{Дж}, \text{положительна},  \\
    A_2 &= q_2\ele = \eli_2\tau\ele = \frac{\ele}{R + r} \tau \ele
            = \frac{\ele^2 \tau}{R + r} = \frac{\sqr{3\,\text{В}} \cdot 5\,\text{с}}{10\,\text{Ом} + 30\,\text{Ом}}
            = \frac98\units{Дж} \approx 1{,}125\,\text{Дж}, \text{положительна},  \\
    \eta_1 &= \frac{Q_1}{A_1} = \ldots = \frac{R}{R} = 1,  \\
    \eta_2 &= \frac{Q_2}{A_2} = \ldots = \frac{R}{R + r} = \frac14 \approx 0{,}25.
    \end{align*}
}
\solutionspace{180pt}

\tasknumber{4}%
\task{%
    Лампочки, сопротивления которых $R_1 = 4{,}00\,\text{Ом}$ и $R_2 = 100{,}00\,\text{Ом}$, поочерёдно подключённные к некоторому источнику тока,
    потребляют одинаковую мощность.
    Найти внутреннее сопротивление источника и КПД цепи в каждом случае.
}
\answer{%
    \begin{align*}
        P_1 &= \sqr{\frac{\ele}{R_1 + r}}R_1,
        P_2  = \sqr{\frac{\ele}{R_2 + r}}R_2,
        P_1 = P_2 \implies  \\
        &\implies R_1 \sqr{R_2 + r} = R_2 \sqr{R_1 + r} \implies  \\
        &\implies R_1 R_2^2 + 2 R_1 R_2 r + R_1 r^2 =
                    R_2 R_1^2 + 2 R_2 R_1 r + R_2 r^2  \implies  \\
    &\implies r^2 (R_2 - R_1) = R_2^2 R_2 - R_1^2 R_2 \implies  \\
    &\implies r
            = \sqrt{R_1 R_2 \frac{R_2 - R_1}{R_2 - R_1}}
            = \sqrt{R_1 R_2}
            = \sqrt{4{,}00\,\text{Ом} \cdot 100{,}00\,\text{Ом}}
            = 20{,}0\,\text{Ом}.
            \\
    \eta_1
            &= \frac{R_1}{R_1 + r}
            = \frac{\sqrt{R_1}}{\sqrt{R_1} + \sqrt{R_2}}
            = 0{,}167,  \\
    \eta_2
            &= \frac{R_2}{R_2 + r}
            = \frac{\sqrt{R_2}}{\sqrt{R_2} + \sqrt{R_1}}
            = 0{,}833
    \end{align*}
}
\solutionspace{120pt}

\tasknumber{5}%
\task{%
    Определите ток, протекающий через резистор $R = 12\,\text{Ом}$ и разность потенциалов на нём (см.
    рис.
    на доске),
    если $r_1 = 3\,\text{Ом}$, $r_2 = 1\,\text{Ом}$, $\ele_1 = 40\,\text{В}$, $\ele_2 = 20\,\text{В}$.
}
\answer{%
    Обозначим на рисунке все токи: направление произвольно, но его надо зафиксировать.
    Всего на рисунке 3 контура и 2 узла.
    Поэтому можно записать $3 - 1 = 2$ уравнения законов Кирхгофа для замкнутого контура и $2 - 1 = 1$ — для узлов
    (остальные уравнения тоже можно записать, но они не дадут полезной информации, а будут лишь следствиями уже записанных).

    Отметим на рисунке 2 контура (и не забуем указать направление) и 1 узел (точка «1»ы, выделена жирным).
    Выбор контуров и узлов не критичен: получившаяся система может быть чуть проще или сложнее, но не слишком.

    \begin{tikzpicture}[circuit ee IEC, thick]
        \draw  (0, 0) to [current direction={near end, info=$\eli_1$}] (0, 3)
                to [battery={rotate=-180,info={$\ele_1, r_1$}}]
                (3, 3)
                to [battery={info'={$\ele_2, r_2$}}]
                (6, 3) to [current direction'={near start, info=$\eli_2$}] (6, 0) -- (0, 0)
                (3, 0) to [current direction={near start, info=$\eli$}, resistor={near end, info=$R$}] (3, 3);
        \draw [-{Latex},color=red] (1.2, 1.7) arc [start angle = 135, end angle = -160, radius = 0.6];
        \draw [-{Latex},color=blue] (4.2, 1.7) arc [start angle = 135, end angle = -160, radius = 0.6];
        \node [contact,color=green!71!black] (bottomc) at (3, 0) {};
        \node [below] (bottom) at (3, 0) {$2$};
        \node [above] (top) at (3, 3) {$1$};
    \end{tikzpicture}

    \begin{align*}
        &\begin{cases}
            {\color{red} \ele_1 = \eli_1 r_1 - \eli R}, \\
            {\color{blue} -\ele_2 = -\eli_2 r_2 + \eli R}, \\
            {\color{green!71!black} - \eli - \eli_1 - \eli_2 = 0 };
        \end{cases}
        \qquad \implies \qquad
        \begin{cases}
            \eli_1 = \frac{\ele_1 + \eli R}{r_1}, \\
            \eli_2 = \frac{\ele_2 + \eli R}{r_2}, \\
            \eli + \eli_1 + \eli_2 = 0;
        \end{cases} \implies \\
        &\implies
         \eli + \frac{\ele_1 + \eli R}{r_1:L} + \frac{\ele_2 + \eli R}{r_2:L} = 0, \\
        &\eli\cbr{ 1 + \frac R{r_1:L} + \frac R{r_2:L}} + \frac{\ele_1 }{r_1:L} + \frac{\ele_2 }{r_2:L} = 0, \\
        &\eli
            = - \frac{\frac{\ele_1 }{r_1:L} + \frac{\ele_2 }{r_2:L}}{ 1 + \frac R{r_1:L} + \frac R{r_2:L}}
            = - \frac{\frac{40\,\text{В}}{3\,\text{Ом}} + \frac{20\,\text{В}}{1\,\text{Ом}}}{ 1 + \frac{12\,\text{Ом}}{3\,\text{Ом}} + \frac{12\,\text{Ом}}{1\,\text{Ом}}}
            = - \frac{100}{51}\units{А}
            \approx -2{,}00\,\text{А}, \\
        &U  = \varphi_2 - \varphi_1 = \eli R
            = - \frac{\frac{\ele_1 }{r_1:L} + \frac{\ele_2 }{r_2:L}}{ 1 + \frac R{r_1:L} + \frac R{r_2:L}} R
            \approx -23{,}50\,\text{В}.
    \end{align*}
    Оба ответа отрицательны, потому что мы изначально «не угадали» с направлением тока.
    Расчёт же показал,
    что ток через резистор $R$ течёт в противоположную сторону: вниз на рисунке, а потенциал точки 1 больше потенциала точки 2,
    а электрический ток ожидаемо течёт из точки с большим потенциалов в точку с меньшим.

    Кстати, если продолжить расчёт и вычислить значения ещё двух токов (формулы для $\eli_1$ и $\eli_2$, куда подставлять, выписаны выше),
    то по их знакам можно будет понять: угадали ли мы с их направлением или нет.
}

\variantsplitter

\addpersonalvariant{Алексей Тихонов}

\tasknumber{1}%
\task{%
    На резистор сопротивлением $R = 18\,\text{Ом}$ подали напряжение $U = 180\,\text{В}$.
    Определите ток, который потечёт через резистор, и мощность, выделяющуюся на нём.
}
\answer{%
    \begin{align*}
    \eli &= \frac{U}{R} = \frac{180\,\text{В}}{18\,\text{Ом}} = 10{,}00\,\text{А},  \\
    P &= \frac{U^2}{R} = \frac{\sqr{180\,\text{В}}}{18\,\text{Ом}} = 1800{,}00\,\text{Вт}
    \end{align*}
}
\solutionspace{60pt}

\tasknumber{2}%
\task{%
    Через резистор сопротивлением $r = 30\,\text{Ом}$ протекает электрический ток $\eli = 4{,}00\,\text{А}$.
    Определите, чему равны напряжение на резисторе и мощность, выделяющаяся на нём.
}
\answer{%
    \begin{align*}
    U &= \eli r = 4{,}00\,\text{А} \cdot 30\,\text{Ом} = 120\,\text{В},  \\
    P &= \eli^2r = \sqr{4{,}00\,\text{А}} \cdot 30\,\text{Ом} = 480\,\text{Вт}
    \end{align*}
}
\solutionspace{60pt}

\tasknumber{3}%
\task{%
    Замкнутая электрическая цепь состоит из ЭДС $\ele = 4\,\text{В}$ и сопротивлением $r$
    и резистора $R = 15\,\text{Ом}$.
    Определите ток, протекающий в цепи.
    Какая тепловая энергия выделится на резисторе за время
    $\tau = 10\,\text{с}$? Какая работа будет совершена ЭДС за это время? Каков знак этой работы? Чему равен КПД цепи?
    Вычислите значения для 2 случаев: $r=0$ и $r = 30\,\text{Ом}$.
}
\answer{%
    \begin{align*}
    \eli_1 &= \frac{\ele}{R} = \frac{4\,\text{В}}{15\,\text{Ом}} = \frac4{15}\units{А} \approx 0{,}27\,\text{А},  \\
    \eli_2 &= \frac{\ele}{R + r} = \frac{4\,\text{В}}{15\,\text{Ом} + 30\,\text{Ом}} = \frac4{45}\units{А} \approx 0{,}09\,\text{А},  \\
    Q_1 &= \eli_1^2R\tau = \sqr{\frac{\ele}{R}} R \tau
            = \sqr{\frac{4\,\text{В}}{15\,\text{Ом}}} \cdot 15\,\text{Ом} \cdot 10\,\text{с} = \frac{32}3\units{Дж} \approx 10{,}667\,\text{Дж},  \\
    Q_2 &= \eli_2^2R\tau = \sqr{\frac{\ele}{R + r}} R \tau
            = \sqr{\frac{4\,\text{В}}{15\,\text{Ом} + 30\,\text{Ом}}} \cdot 15\,\text{Ом} \cdot 10\,\text{с} = \frac{32}{27}\units{Дж} \approx 1{,}185\,\text{Дж},  \\
    A_1 &= q_1\ele = \eli_1\tau\ele = \frac{\ele}{R} \tau \ele
            = \frac{\ele^2 \tau}{R} = \frac{\sqr{4\,\text{В}} \cdot 10\,\text{с}}{15\,\text{Ом}}
            = \frac{32}3\units{Дж} \approx 10{,}667\,\text{Дж}, \text{положительна},  \\
    A_2 &= q_2\ele = \eli_2\tau\ele = \frac{\ele}{R + r} \tau \ele
            = \frac{\ele^2 \tau}{R + r} = \frac{\sqr{4\,\text{В}} \cdot 10\,\text{с}}{15\,\text{Ом} + 30\,\text{Ом}}
            = \frac{32}9\units{Дж} \approx 3{,}556\,\text{Дж}, \text{положительна},  \\
    \eta_1 &= \frac{Q_1}{A_1} = \ldots = \frac{R}{R} = 1,  \\
    \eta_2 &= \frac{Q_2}{A_2} = \ldots = \frac{R}{R + r} = \frac13 \approx 0{,}33.
    \end{align*}
}
\solutionspace{180pt}

\tasknumber{4}%
\task{%
    Лампочки, сопротивления которых $R_1 = 6{,}00\,\text{Ом}$ и $R_2 = 54{,}00\,\text{Ом}$, поочерёдно подключённные к некоторому источнику тока,
    потребляют одинаковую мощность.
    Найти внутреннее сопротивление источника и КПД цепи в каждом случае.
}
\answer{%
    \begin{align*}
        P_1 &= \sqr{\frac{\ele}{R_1 + r}}R_1,
        P_2  = \sqr{\frac{\ele}{R_2 + r}}R_2,
        P_1 = P_2 \implies  \\
        &\implies R_1 \sqr{R_2 + r} = R_2 \sqr{R_1 + r} \implies  \\
        &\implies R_1 R_2^2 + 2 R_1 R_2 r + R_1 r^2 =
                    R_2 R_1^2 + 2 R_2 R_1 r + R_2 r^2  \implies  \\
    &\implies r^2 (R_2 - R_1) = R_2^2 R_2 - R_1^2 R_2 \implies  \\
    &\implies r
            = \sqrt{R_1 R_2 \frac{R_2 - R_1}{R_2 - R_1}}
            = \sqrt{R_1 R_2}
            = \sqrt{6{,}00\,\text{Ом} \cdot 54{,}00\,\text{Ом}}
            = 18{,}0\,\text{Ом}.
            \\
    \eta_1
            &= \frac{R_1}{R_1 + r}
            = \frac{\sqrt{R_1}}{\sqrt{R_1} + \sqrt{R_2}}
            = 0{,}250,  \\
    \eta_2
            &= \frac{R_2}{R_2 + r}
            = \frac{\sqrt{R_2}}{\sqrt{R_2} + \sqrt{R_1}}
            = 0{,}750
    \end{align*}
}
\solutionspace{120pt}

\tasknumber{5}%
\task{%
    Определите ток, протекающий через резистор $R = 15\,\text{Ом}$ и разность потенциалов на нём (см.
    рис.
    на доске),
    если $r_1 = 2\,\text{Ом}$, $r_2 = 1\,\text{Ом}$, $\ele_1 = 20\,\text{В}$, $\ele_2 = 60\,\text{В}$.
}
\answer{%
    Обозначим на рисунке все токи: направление произвольно, но его надо зафиксировать.
    Всего на рисунке 3 контура и 2 узла.
    Поэтому можно записать $3 - 1 = 2$ уравнения законов Кирхгофа для замкнутого контура и $2 - 1 = 1$ — для узлов
    (остальные уравнения тоже можно записать, но они не дадут полезной информации, а будут лишь следствиями уже записанных).

    Отметим на рисунке 2 контура (и не забуем указать направление) и 1 узел (точка «1»ы, выделена жирным).
    Выбор контуров и узлов не критичен: получившаяся система может быть чуть проще или сложнее, но не слишком.

    \begin{tikzpicture}[circuit ee IEC, thick]
        \draw  (0, 0) to [current direction={near end, info=$\eli_1$}] (0, 3)
                to [battery={rotate=-180,info={$\ele_1, r_1$}}]
                (3, 3)
                to [battery={info'={$\ele_2, r_2$}}]
                (6, 3) to [current direction'={near start, info=$\eli_2$}] (6, 0) -- (0, 0)
                (3, 0) to [current direction={near start, info=$\eli$}, resistor={near end, info=$R$}] (3, 3);
        \draw [-{Latex},color=red] (1.2, 1.7) arc [start angle = 135, end angle = -160, radius = 0.6];
        \draw [-{Latex},color=blue] (4.2, 1.7) arc [start angle = 135, end angle = -160, radius = 0.6];
        \node [contact,color=green!71!black] (bottomc) at (3, 0) {};
        \node [below] (bottom) at (3, 0) {$2$};
        \node [above] (top) at (3, 3) {$1$};
    \end{tikzpicture}

    \begin{align*}
        &\begin{cases}
            {\color{red} \ele_1 = \eli_1 r_1 - \eli R}, \\
            {\color{blue} -\ele_2 = -\eli_2 r_2 + \eli R}, \\
            {\color{green!71!black} - \eli - \eli_1 - \eli_2 = 0 };
        \end{cases}
        \qquad \implies \qquad
        \begin{cases}
            \eli_1 = \frac{\ele_1 + \eli R}{r_1}, \\
            \eli_2 = \frac{\ele_2 + \eli R}{r_2}, \\
            \eli + \eli_1 + \eli_2 = 0;
        \end{cases} \implies \\
        &\implies
         \eli + \frac{\ele_1 + \eli R}{r_1:L} + \frac{\ele_2 + \eli R}{r_2:L} = 0, \\
        &\eli\cbr{ 1 + \frac R{r_1:L} + \frac R{r_2:L}} + \frac{\ele_1 }{r_1:L} + \frac{\ele_2 }{r_2:L} = 0, \\
        &\eli
            = - \frac{\frac{\ele_1 }{r_1:L} + \frac{\ele_2 }{r_2:L}}{ 1 + \frac R{r_1:L} + \frac R{r_2:L}}
            = - \frac{\frac{20\,\text{В}}{2\,\text{Ом}} + \frac{60\,\text{В}}{1\,\text{Ом}}}{ 1 + \frac{15\,\text{Ом}}{2\,\text{Ом}} + \frac{15\,\text{Ом}}{1\,\text{Ом}}}
            = - \frac{140}{47}\units{А}
            \approx -3{,}00\,\text{А}, \\
        &U  = \varphi_2 - \varphi_1 = \eli R
            = - \frac{\frac{\ele_1 }{r_1:L} + \frac{\ele_2 }{r_2:L}}{ 1 + \frac R{r_1:L} + \frac R{r_2:L}} R
            \approx -44{,}70\,\text{В}.
    \end{align*}
    Оба ответа отрицательны, потому что мы изначально «не угадали» с направлением тока.
    Расчёт же показал,
    что ток через резистор $R$ течёт в противоположную сторону: вниз на рисунке, а потенциал точки 1 больше потенциала точки 2,
    а электрический ток ожидаемо течёт из точки с большим потенциалов в точку с меньшим.

    Кстати, если продолжить расчёт и вычислить значения ещё двух токов (формулы для $\eli_1$ и $\eli_2$, куда подставлять, выписаны выше),
    то по их знакам можно будет понять: угадали ли мы с их направлением или нет.
}

\variantsplitter

\addpersonalvariant{Алина Филиппова}

\tasknumber{1}%
\task{%
    На резистор сопротивлением $r = 5\,\text{Ом}$ подали напряжение $U = 180\,\text{В}$.
    Определите ток, который потечёт через резистор, и мощность, выделяющуюся на нём.
}
\answer{%
    \begin{align*}
    \eli &= \frac{U}{r} = \frac{180\,\text{В}}{5\,\text{Ом}} = 36{,}00\,\text{А},  \\
    P &= \frac{U^2}{r} = \frac{\sqr{180\,\text{В}}}{5\,\text{Ом}} = 6480{,}00\,\text{Вт}
    \end{align*}
}
\solutionspace{60pt}

\tasknumber{2}%
\task{%
    Через резистор сопротивлением $R = 5\,\text{Ом}$ протекает электрический ток $\eli = 10{,}00\,\text{А}$.
    Определите, чему равны напряжение на резисторе и мощность, выделяющаяся на нём.
}
\answer{%
    \begin{align*}
    U &= \eli R = 10{,}00\,\text{А} \cdot 5\,\text{Ом} = 50\,\text{В},  \\
    P &= \eli^2R = \sqr{10{,}00\,\text{А}} \cdot 5\,\text{Ом} = 500\,\text{Вт}
    \end{align*}
}
\solutionspace{60pt}

\tasknumber{3}%
\task{%
    Замкнутая электрическая цепь состоит из ЭДС $\ele = 4\,\text{В}$ и сопротивлением $r$
    и резистора $R = 10\,\text{Ом}$.
    Определите ток, протекающий в цепи.
    Какая тепловая энергия выделится на резисторе за время
    $\tau = 10\,\text{с}$? Какая работа будет совершена ЭДС за это время? Каков знак этой работы? Чему равен КПД цепи?
    Вычислите значения для 2 случаев: $r=0$ и $r = 10\,\text{Ом}$.
}
\answer{%
    \begin{align*}
    \eli_1 &= \frac{\ele}{R} = \frac{4\,\text{В}}{10\,\text{Ом}} = \frac25\units{А} \approx 0{,}40\,\text{А},  \\
    \eli_2 &= \frac{\ele}{R + r} = \frac{4\,\text{В}}{10\,\text{Ом} + 10\,\text{Ом}} = \frac15\units{А} \approx 0{,}20\,\text{А},  \\
    Q_1 &= \eli_1^2R\tau = \sqr{\frac{\ele}{R}} R \tau
            = \sqr{\frac{4\,\text{В}}{10\,\text{Ом}}} \cdot 10\,\text{Ом} \cdot 10\,\text{с} = 16\units{Дж} \approx 16{,}000\,\text{Дж},  \\
    Q_2 &= \eli_2^2R\tau = \sqr{\frac{\ele}{R + r}} R \tau
            = \sqr{\frac{4\,\text{В}}{10\,\text{Ом} + 10\,\text{Ом}}} \cdot 10\,\text{Ом} \cdot 10\,\text{с} = 4\units{Дж} \approx 4{,}000\,\text{Дж},  \\
    A_1 &= q_1\ele = \eli_1\tau\ele = \frac{\ele}{R} \tau \ele
            = \frac{\ele^2 \tau}{R} = \frac{\sqr{4\,\text{В}} \cdot 10\,\text{с}}{10\,\text{Ом}}
            = 16\units{Дж} \approx 16{,}000\,\text{Дж}, \text{положительна},  \\
    A_2 &= q_2\ele = \eli_2\tau\ele = \frac{\ele}{R + r} \tau \ele
            = \frac{\ele^2 \tau}{R + r} = \frac{\sqr{4\,\text{В}} \cdot 10\,\text{с}}{10\,\text{Ом} + 10\,\text{Ом}}
            = 8\units{Дж} \approx 8{,}000\,\text{Дж}, \text{положительна},  \\
    \eta_1 &= \frac{Q_1}{A_1} = \ldots = \frac{R}{R} = 1,  \\
    \eta_2 &= \frac{Q_2}{A_2} = \ldots = \frac{R}{R + r} = \frac12 \approx 0{,}50.
    \end{align*}
}
\solutionspace{180pt}

\tasknumber{4}%
\task{%
    Лампочки, сопротивления которых $R_1 = 3{,}00\,\text{Ом}$ и $R_2 = 12{,}00\,\text{Ом}$, поочерёдно подключённные к некоторому источнику тока,
    потребляют одинаковую мощность.
    Найти внутреннее сопротивление источника и КПД цепи в каждом случае.
}
\answer{%
    \begin{align*}
        P_1 &= \sqr{\frac{\ele}{R_1 + r}}R_1,
        P_2  = \sqr{\frac{\ele}{R_2 + r}}R_2,
        P_1 = P_2 \implies  \\
        &\implies R_1 \sqr{R_2 + r} = R_2 \sqr{R_1 + r} \implies  \\
        &\implies R_1 R_2^2 + 2 R_1 R_2 r + R_1 r^2 =
                    R_2 R_1^2 + 2 R_2 R_1 r + R_2 r^2  \implies  \\
    &\implies r^2 (R_2 - R_1) = R_2^2 R_2 - R_1^2 R_2 \implies  \\
    &\implies r
            = \sqrt{R_1 R_2 \frac{R_2 - R_1}{R_2 - R_1}}
            = \sqrt{R_1 R_2}
            = \sqrt{3{,}00\,\text{Ом} \cdot 12{,}00\,\text{Ом}}
            = 6{,}0\,\text{Ом}.
            \\
    \eta_1
            &= \frac{R_1}{R_1 + r}
            = \frac{\sqrt{R_1}}{\sqrt{R_1} + \sqrt{R_2}}
            = 0{,}333,  \\
    \eta_2
            &= \frac{R_2}{R_2 + r}
            = \frac{\sqrt{R_2}}{\sqrt{R_2} + \sqrt{R_1}}
            = 0{,}667
    \end{align*}
}
\solutionspace{120pt}

\tasknumber{5}%
\task{%
    Определите ток, протекающий через резистор $R = 15\,\text{Ом}$ и разность потенциалов на нём (см.
    рис.
    на доске),
    если $r_1 = 2\,\text{Ом}$, $r_2 = 1\,\text{Ом}$, $\ele_1 = 20\,\text{В}$, $\ele_2 = 20\,\text{В}$.
}
\answer{%
    Обозначим на рисунке все токи: направление произвольно, но его надо зафиксировать.
    Всего на рисунке 3 контура и 2 узла.
    Поэтому можно записать $3 - 1 = 2$ уравнения законов Кирхгофа для замкнутого контура и $2 - 1 = 1$ — для узлов
    (остальные уравнения тоже можно записать, но они не дадут полезной информации, а будут лишь следствиями уже записанных).

    Отметим на рисунке 2 контура (и не забуем указать направление) и 1 узел (точка «1»ы, выделена жирным).
    Выбор контуров и узлов не критичен: получившаяся система может быть чуть проще или сложнее, но не слишком.

    \begin{tikzpicture}[circuit ee IEC, thick]
        \draw  (0, 0) to [current direction={near end, info=$\eli_1$}] (0, 3)
                to [battery={rotate=-180,info={$\ele_1, r_1$}}]
                (3, 3)
                to [battery={info'={$\ele_2, r_2$}}]
                (6, 3) to [current direction'={near start, info=$\eli_2$}] (6, 0) -- (0, 0)
                (3, 0) to [current direction={near start, info=$\eli$}, resistor={near end, info=$R$}] (3, 3);
        \draw [-{Latex},color=red] (1.2, 1.7) arc [start angle = 135, end angle = -160, radius = 0.6];
        \draw [-{Latex},color=blue] (4.2, 1.7) arc [start angle = 135, end angle = -160, radius = 0.6];
        \node [contact,color=green!71!black] (bottomc) at (3, 0) {};
        \node [below] (bottom) at (3, 0) {$2$};
        \node [above] (top) at (3, 3) {$1$};
    \end{tikzpicture}

    \begin{align*}
        &\begin{cases}
            {\color{red} \ele_1 = \eli_1 r_1 - \eli R}, \\
            {\color{blue} -\ele_2 = -\eli_2 r_2 + \eli R}, \\
            {\color{green!71!black} - \eli - \eli_1 - \eli_2 = 0 };
        \end{cases}
        \qquad \implies \qquad
        \begin{cases}
            \eli_1 = \frac{\ele_1 + \eli R}{r_1}, \\
            \eli_2 = \frac{\ele_2 + \eli R}{r_2}, \\
            \eli + \eli_1 + \eli_2 = 0;
        \end{cases} \implies \\
        &\implies
         \eli + \frac{\ele_1 + \eli R}{r_1:L} + \frac{\ele_2 + \eli R}{r_2:L} = 0, \\
        &\eli\cbr{ 1 + \frac R{r_1:L} + \frac R{r_2:L}} + \frac{\ele_1 }{r_1:L} + \frac{\ele_2 }{r_2:L} = 0, \\
        &\eli
            = - \frac{\frac{\ele_1 }{r_1:L} + \frac{\ele_2 }{r_2:L}}{ 1 + \frac R{r_1:L} + \frac R{r_2:L}}
            = - \frac{\frac{20\,\text{В}}{2\,\text{Ом}} + \frac{20\,\text{В}}{1\,\text{Ом}}}{ 1 + \frac{15\,\text{Ом}}{2\,\text{Ом}} + \frac{15\,\text{Ом}}{1\,\text{Ом}}}
            = - \frac{60}{47}\units{А}
            \approx -1{,}300\,\text{А}, \\
        &U  = \varphi_2 - \varphi_1 = \eli R
            = - \frac{\frac{\ele_1 }{r_1:L} + \frac{\ele_2 }{r_2:L}}{ 1 + \frac R{r_1:L} + \frac R{r_2:L}} R
            \approx -19{,}100\,\text{В}.
    \end{align*}
    Оба ответа отрицательны, потому что мы изначально «не угадали» с направлением тока.
    Расчёт же показал,
    что ток через резистор $R$ течёт в противоположную сторону: вниз на рисунке, а потенциал точки 1 больше потенциала точки 2,
    а электрический ток ожидаемо течёт из точки с большим потенциалов в точку с меньшим.

    Кстати, если продолжить расчёт и вычислить значения ещё двух токов (формулы для $\eli_1$ и $\eli_2$, куда подставлять, выписаны выше),
    то по их знакам можно будет понять: угадали ли мы с их направлением или нет.
}

\variantsplitter

\addpersonalvariant{Алина Яшина}

\tasknumber{1}%
\task{%
    На резистор сопротивлением $R = 12\,\text{Ом}$ подали напряжение $U = 150\,\text{В}$.
    Определите ток, который потечёт через резистор, и мощность, выделяющуюся на нём.
}
\answer{%
    \begin{align*}
    \eli &= \frac{U}{R} = \frac{150\,\text{В}}{12\,\text{Ом}} = 12{,}50\,\text{А},  \\
    P &= \frac{U^2}{R} = \frac{\sqr{150\,\text{В}}}{12\,\text{Ом}} = 1875{,}00\,\text{Вт}
    \end{align*}
}
\solutionspace{60pt}

\tasknumber{2}%
\task{%
    Через резистор сопротивлением $r = 5\,\text{Ом}$ протекает электрический ток $\eli = 10{,}00\,\text{А}$.
    Определите, чему равны напряжение на резисторе и мощность, выделяющаяся на нём.
}
\answer{%
    \begin{align*}
    U &= \eli r = 10{,}00\,\text{А} \cdot 5\,\text{Ом} = 50\,\text{В},  \\
    P &= \eli^2r = \sqr{10{,}00\,\text{А}} \cdot 5\,\text{Ом} = 500\,\text{Вт}
    \end{align*}
}
\solutionspace{60pt}

\tasknumber{3}%
\task{%
    Замкнутая электрическая цепь состоит из ЭДС $\ele = 2\,\text{В}$ и сопротивлением $r$
    и резистора $R = 10\,\text{Ом}$.
    Определите ток, протекающий в цепи.
    Какая тепловая энергия выделится на резисторе за время
    $\tau = 2\,\text{с}$? Какая работа будет совершена ЭДС за это время? Каков знак этой работы? Чему равен КПД цепи?
    Вычислите значения для 2 случаев: $r=0$ и $r = 20\,\text{Ом}$.
}
\answer{%
    \begin{align*}
    \eli_1 &= \frac{\ele}{R} = \frac{2\,\text{В}}{10\,\text{Ом}} = \frac15\units{А} \approx 0{,}20\,\text{А},  \\
    \eli_2 &= \frac{\ele}{R + r} = \frac{2\,\text{В}}{10\,\text{Ом} + 20\,\text{Ом}} = \frac1{15}\units{А} \approx 0{,}07\,\text{А},  \\
    Q_1 &= \eli_1^2R\tau = \sqr{\frac{\ele}{R}} R \tau
            = \sqr{\frac{2\,\text{В}}{10\,\text{Ом}}} \cdot 10\,\text{Ом} \cdot 2\,\text{с} = \frac45\units{Дж} \approx 0{,}800\,\text{Дж},  \\
    Q_2 &= \eli_2^2R\tau = \sqr{\frac{\ele}{R + r}} R \tau
            = \sqr{\frac{2\,\text{В}}{10\,\text{Ом} + 20\,\text{Ом}}} \cdot 10\,\text{Ом} \cdot 2\,\text{с} = \frac4{45}\units{Дж} \approx 0{,}089\,\text{Дж},  \\
    A_1 &= q_1\ele = \eli_1\tau\ele = \frac{\ele}{R} \tau \ele
            = \frac{\ele^2 \tau}{R} = \frac{\sqr{2\,\text{В}} \cdot 2\,\text{с}}{10\,\text{Ом}}
            = \frac45\units{Дж} \approx 0{,}800\,\text{Дж}, \text{положительна},  \\
    A_2 &= q_2\ele = \eli_2\tau\ele = \frac{\ele}{R + r} \tau \ele
            = \frac{\ele^2 \tau}{R + r} = \frac{\sqr{2\,\text{В}} \cdot 2\,\text{с}}{10\,\text{Ом} + 20\,\text{Ом}}
            = \frac4{15}\units{Дж} \approx 0{,}267\,\text{Дж}, \text{положительна},  \\
    \eta_1 &= \frac{Q_1}{A_1} = \ldots = \frac{R}{R} = 1,  \\
    \eta_2 &= \frac{Q_2}{A_2} = \ldots = \frac{R}{R + r} = \frac13 \approx 0{,}33.
    \end{align*}
}
\solutionspace{180pt}

\tasknumber{4}%
\task{%
    Лампочки, сопротивления которых $R_1 = 4{,}00\,\text{Ом}$ и $R_2 = 36{,}00\,\text{Ом}$, поочерёдно подключённные к некоторому источнику тока,
    потребляют одинаковую мощность.
    Найти внутреннее сопротивление источника и КПД цепи в каждом случае.
}
\answer{%
    \begin{align*}
        P_1 &= \sqr{\frac{\ele}{R_1 + r}}R_1,
        P_2  = \sqr{\frac{\ele}{R_2 + r}}R_2,
        P_1 = P_2 \implies  \\
        &\implies R_1 \sqr{R_2 + r} = R_2 \sqr{R_1 + r} \implies  \\
        &\implies R_1 R_2^2 + 2 R_1 R_2 r + R_1 r^2 =
                    R_2 R_1^2 + 2 R_2 R_1 r + R_2 r^2  \implies  \\
    &\implies r^2 (R_2 - R_1) = R_2^2 R_2 - R_1^2 R_2 \implies  \\
    &\implies r
            = \sqrt{R_1 R_2 \frac{R_2 - R_1}{R_2 - R_1}}
            = \sqrt{R_1 R_2}
            = \sqrt{4{,}00\,\text{Ом} \cdot 36{,}00\,\text{Ом}}
            = 12{,}0\,\text{Ом}.
            \\
    \eta_1
            &= \frac{R_1}{R_1 + r}
            = \frac{\sqrt{R_1}}{\sqrt{R_1} + \sqrt{R_2}}
            = 0{,}250,  \\
    \eta_2
            &= \frac{R_2}{R_2 + r}
            = \frac{\sqrt{R_2}}{\sqrt{R_2} + \sqrt{R_1}}
            = 0{,}750
    \end{align*}
}
\solutionspace{120pt}

\tasknumber{5}%
\task{%
    Определите ток, протекающий через резистор $R = 10\,\text{Ом}$ и разность потенциалов на нём (см.
    рис.
    на доске),
    если $r_1 = 3\,\text{Ом}$, $r_2 = 3\,\text{Ом}$, $\ele_1 = 40\,\text{В}$, $\ele_2 = 60\,\text{В}$.
}
\answer{%
    Обозначим на рисунке все токи: направление произвольно, но его надо зафиксировать.
    Всего на рисунке 3 контура и 2 узла.
    Поэтому можно записать $3 - 1 = 2$ уравнения законов Кирхгофа для замкнутого контура и $2 - 1 = 1$ — для узлов
    (остальные уравнения тоже можно записать, но они не дадут полезной информации, а будут лишь следствиями уже записанных).

    Отметим на рисунке 2 контура (и не забуем указать направление) и 1 узел (точка «1»ы, выделена жирным).
    Выбор контуров и узлов не критичен: получившаяся система может быть чуть проще или сложнее, но не слишком.

    \begin{tikzpicture}[circuit ee IEC, thick]
        \draw  (0, 0) to [current direction={near end, info=$\eli_1$}] (0, 3)
                to [battery={rotate=-180,info={$\ele_1, r_1$}}]
                (3, 3)
                to [battery={info'={$\ele_2, r_2$}}]
                (6, 3) to [current direction'={near start, info=$\eli_2$}] (6, 0) -- (0, 0)
                (3, 0) to [current direction={near start, info=$\eli$}, resistor={near end, info=$R$}] (3, 3);
        \draw [-{Latex},color=red] (1.2, 1.7) arc [start angle = 135, end angle = -160, radius = 0.6];
        \draw [-{Latex},color=blue] (4.2, 1.7) arc [start angle = 135, end angle = -160, radius = 0.6];
        \node [contact,color=green!71!black] (bottomc) at (3, 0) {};
        \node [below] (bottom) at (3, 0) {$2$};
        \node [above] (top) at (3, 3) {$1$};
    \end{tikzpicture}

    \begin{align*}
        &\begin{cases}
            {\color{red} \ele_1 = \eli_1 r_1 - \eli R}, \\
            {\color{blue} -\ele_2 = -\eli_2 r_2 + \eli R}, \\
            {\color{green!71!black} - \eli - \eli_1 - \eli_2 = 0 };
        \end{cases}
        \qquad \implies \qquad
        \begin{cases}
            \eli_1 = \frac{\ele_1 + \eli R}{r_1}, \\
            \eli_2 = \frac{\ele_2 + \eli R}{r_2}, \\
            \eli + \eli_1 + \eli_2 = 0;
        \end{cases} \implies \\
        &\implies
         \eli + \frac{\ele_1 + \eli R}{r_1:L} + \frac{\ele_2 + \eli R}{r_2:L} = 0, \\
        &\eli\cbr{ 1 + \frac R{r_1:L} + \frac R{r_2:L}} + \frac{\ele_1 }{r_1:L} + \frac{\ele_2 }{r_2:L} = 0, \\
        &\eli
            = - \frac{\frac{\ele_1 }{r_1:L} + \frac{\ele_2 }{r_2:L}}{ 1 + \frac R{r_1:L} + \frac R{r_2:L}}
            = - \frac{\frac{40\,\text{В}}{3\,\text{Ом}} + \frac{60\,\text{В}}{3\,\text{Ом}}}{ 1 + \frac{10\,\text{Ом}}{3\,\text{Ом}} + \frac{10\,\text{Ом}}{3\,\text{Ом}}}
            = - \frac{100}{23}\units{А}
            \approx -4{,}30\,\text{А}, \\
        &U  = \varphi_2 - \varphi_1 = \eli R
            = - \frac{\frac{\ele_1 }{r_1:L} + \frac{\ele_2 }{r_2:L}}{ 1 + \frac R{r_1:L} + \frac R{r_2:L}} R
            \approx -43{,}50\,\text{В}.
    \end{align*}
    Оба ответа отрицательны, потому что мы изначально «не угадали» с направлением тока.
    Расчёт же показал,
    что ток через резистор $R$ течёт в противоположную сторону: вниз на рисунке, а потенциал точки 1 больше потенциала точки 2,
    а электрический ток ожидаемо течёт из точки с большим потенциалов в точку с меньшим.

    Кстати, если продолжить расчёт и вычислить значения ещё двух токов (формулы для $\eli_1$ и $\eli_2$, куда подставлять, выписаны выше),
    то по их знакам можно будет понять: угадали ли мы с их направлением или нет.
}
% autogenerated
