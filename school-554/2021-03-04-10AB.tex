\setdate{4~марта~2021}
\setclass{10«АБ»}

\addpersonalvariant{Михаил Бурмистров}

\tasknumber{1}%
\task{%
    Укажите, верны ли утверждения («да» или «нет» слева от каждого утверждения):
    \begin{enumerate}
        \item При изобарном расширении идеальный газ совершает ровно столько работы, сколько внутренней энергии теряет.
        % \item В силу третьего закона Ньютона, совершённая газом работа и работа, совершённая над ним, всегда равны по модулю и противоположны по знаку.
        \item Работу газа в некотором процессе можно вычислять как площадь под графиком в системе координат $PV$, главное лишь правильно расположить оси.
        % \item Дважды два четыре.
        \item При изохорном процессе внутренняя энергия идеального одноатомного газа не изменяется, даже если ему подводят тепло.
        \item Газ может совершить ненулевую работу в изобарном процессе.
        % \item Адиабатический процесс лишь по воле случая не имеет приставки «изо»: в нём изменяются давление, температура и объём, но это не все макропараметры идеального газа.
        \item Полученное выражение для внутренней энергии идеального газа ($\frac 32 \nu RT$) применимо к трёхатомному газу, при этом, например, уравнение состояния идеального газа применимо независимо от числа атомов в молекулах газа.
    \end{enumerate}
}
\answer{%
    $\text{нет, да, нет, да, нет}$
}

\tasknumber{2}%
\task{%
    Определите давление одноатомного идеального газа, занимающего объём $4\,\text{л}$,
    если его внутренняя энергия составляет $500\,\text{Дж}$.
}
\answer{%
    $U = \frac 32 \nu R T = \frac 32 PV \implies P = \frac 23 \cdot \frac UV= \frac 23 \cdot \frac{ 500\,\text{Дж} }{ 4\,\text{л} } \approx 83\,\text{кПа}.$
}
\solutionspace{40pt}

\tasknumber{3}%
\task{%
    Газ расширился от $150\,\text{л}$ до $650\,\text{л}$.
    Давление газа при этом оставалось постоянным и равным $1{,}5\,\text{атм}$.
    Определите работу газа, ответ выразите в килоджоулях.
    $p_{\text{aтм}} = 100\,\text{кПа}$.
}
\answer{%
    $A = P\Delta V = P(V_2 - V_1) = 1{,}5\,\text{атм} \cdot \cbr{650\,\text{л} - 150\,\text{л}} = 75{,}0\,\text{кДж}.$
}
\solutionspace{40pt}

\tasknumber{4}%
\task{%
    Как изменилась внутренняя энергия одноатомного идеального газа при переходе из состояния 1 в состояние 2?
    $P_1 = 4\,\text{МПа}$, $V_1 = 7\,\text{л}$, $P_2 = 2{,}5\,\text{МПа}$, $V_2 = 8\,\text{л}$.
    Как изменилась при этом температура газа?
}
\answer{%
    \begin{align*}
    P_1V_1 &= \nu R T_1, P_2V_2 = \nu R T_2, \\
    \Delta U &= U_2-U_1 = \frac 32 \nu R T_2- \frac 32 \nu R T_1 = \frac 32 P_2 V_2 - \frac 32 P_1 V_1= \frac 32 \cdot \cbr{2{,}5\,\text{МПа} \cdot 8\,\text{л} - 4\,\text{МПа} \cdot 7\,\text{л}} = -12000\,\text{Дж}.
    \\
    \frac{T_2}{T_1} &= \frac{\frac{P_2V_2}{\nu R}}{\frac{P_1V_1}{\nu R}} = \frac{P_2V_2}{P_1V_1}= \frac{2{,}5\,\text{МПа} \cdot 8\,\text{л}}{4\,\text{МПа} \cdot 7\,\text{л}} \approx 0{,}71.
    \end{align*}
}
\solutionspace{80pt}

\tasknumber{5}%
\task{%
    $4\,\text{моль}$ идеального одноатомного газа охладили на $30\,\text{К}$.
    Определите изменение внутренней энергии газа.
    Увеличилась она или уменьшилась?
    Универсальная газовая постоянная $R = 8{,}31\,\frac{\text{Дж}}{\text{моль}\cdot\text{К}}$.
}
\answer{%
    $
        \Delta U = \frac 32 \nu R \Delta T
            = - \frac 32 \cdot 4\,\text{моль} \cdot 8{,}31\,\frac{\text{Дж}}{\text{моль}\cdot\text{К}} \cdot 30\,\text{К}
            = -1495\,\text{Дж}.
            \text{Уменьшилась.}
    $
}
\solutionspace{40pt}

\tasknumber{6}%
\task{%
    Газу сообщили некоторое количество теплоты,
    при этом треть его он потратил на совершение работы,
    одновременно увеличив свою внутреннюю энергию на $2400\,\text{Дж}$.
    Определите работу, совершённую газом.
}
\answer{%
    \begin{align*}
    Q &= A' + \Delta U, A' = \frac 13 Q \implies Q \cdot \cbr{1 - \frac 13} = \Delta U \implies Q = \frac{\Delta U}{1 - \frac 13} = \frac{2400\,\text{Дж}}{1 - \frac 13} \approx 3600\,\text{Дж}.
    \\
    A' &= \frac 13 Q
        = \frac 13 \cdot \frac{\Delta U}{1 - \frac 13}
        = \frac{\Delta U}{3 - 1}
        = \frac{2400\,\text{Дж}}{3 - 1} \approx 1200\,\text{Дж}.
    \end{align*}
}
\solutionspace{60pt}

\tasknumber{7}%
\task{%
    В некотором процессе газ совершил работу $100\,\text{Дж}$,
    при этом его внутренняя энергия увеличилась на $450\,\text{Дж}$.
    Определите количество тепла, переданное при этом процессе газу.
    Явно пропишите, подводили газу тепло или же отводили.
}
\answer{%
    $
        Q = A_\text{газа} + \Delta U, A_\text{газа} = -A_\text{внешняя}
        \implies Q = A_\text{газа} + \Delta U =  100\,\text{Дж} +  450\,\text{Дж} = 550\,\text{Дж}.
        \text{ Подводили.}
    $
}

\variantsplitter

\addpersonalvariant{Ирина Ан}

\tasknumber{1}%
\task{%
    Укажите, верны ли утверждения («да» или «нет» слева от каждого утверждения):
    \begin{enumerate}
        \item При изобарном расширении идеальный газ совершает ровно столько работы, сколько внутренней энергии теряет.
        % \item В силу третьего закона Ньютона, совершённая газом работа и работа, совершённая над ним, всегда равны по модулю и противоположны по знаку.
        \item Работу газа в некотором процессе можно вычислять как площадь под графиком в системе координат $PV$, главное лишь правильно расположить оси.
        % \item Дважды два пять.
        \item При изохорном процессе внутренняя энергия идеального одноатомного газа не изменяется, даже если ему подводят тепло.
        \item Газ может совершить ненулевую работу в изотермическом процессе.
        % \item Адиабатический процесс лишь по воле случая не имеет приставки «изо»: в нём изменяются давление, температура и объём, но это не все макропараметры идеального газа.
        \item Полученное выражение для внутренней энергии идеального газа ($\frac 32 \nu RT$) применимо к двухоатомному газу, при этом, например, уравнение состояния идеального газа применимо независимо от числа атомов в молекулах газа.
    \end{enumerate}
}
\answer{%
    $\text{нет, да, нет, да, нет}$
}

\tasknumber{2}%
\task{%
    Определите давление одноатомного идеального газа, занимающего объём $5\,\text{л}$,
    если его внутренняя энергия составляет $300\,\text{Дж}$.
}
\answer{%
    $U = \frac 32 \nu R T = \frac 32 PV \implies P = \frac 23 \cdot \frac UV= \frac 23 \cdot \frac{ 300\,\text{Дж} }{ 5\,\text{л} } \approx 40\,\text{кПа}.$
}
\solutionspace{40pt}

\tasknumber{3}%
\task{%
    Газ расширился от $150\,\text{л}$ до $650\,\text{л}$.
    Давление газа при этом оставалось постоянным и равным $2{,}5\,\text{атм}$.
    Определите работу газа, ответ выразите в килоджоулях.
    $p_{\text{aтм}} = 100\,\text{кПа}$.
}
\answer{%
    $A = P\Delta V = P(V_2 - V_1) = 2{,}5\,\text{атм} \cdot \cbr{650\,\text{л} - 150\,\text{л}} = 125{,}0\,\text{кДж}.$
}
\solutionspace{40pt}

\tasknumber{4}%
\task{%
    Как изменилась внутренняя энергия одноатомного идеального газа при переходе из состояния 1 в состояние 2?
    $P_1 = 4\,\text{МПа}$, $V_1 = 5\,\text{л}$, $P_2 = 3{,}5\,\text{МПа}$, $V_2 = 4\,\text{л}$.
    Как изменилась при этом температура газа?
}
\answer{%
    \begin{align*}
    P_1V_1 &= \nu R T_1, P_2V_2 = \nu R T_2, \\
    \Delta U &= U_2-U_1 = \frac 32 \nu R T_2- \frac 32 \nu R T_1 = \frac 32 P_2 V_2 - \frac 32 P_1 V_1= \frac 32 \cdot \cbr{3{,}5\,\text{МПа} \cdot 4\,\text{л} - 4\,\text{МПа} \cdot 5\,\text{л}} = -9000\,\text{Дж}.
    \\
    \frac{T_2}{T_1} &= \frac{\frac{P_2V_2}{\nu R}}{\frac{P_1V_1}{\nu R}} = \frac{P_2V_2}{P_1V_1}= \frac{3{,}5\,\text{МПа} \cdot 4\,\text{л}}{4\,\text{МПа} \cdot 5\,\text{л}} \approx 0{,}70.
    \end{align*}
}
\solutionspace{80pt}

\tasknumber{5}%
\task{%
    $4\,\text{моль}$ идеального одноатомного газа охладили на $30\,\text{К}$.
    Определите изменение внутренней энергии газа.
    Увеличилась она или уменьшилась?
    Универсальная газовая постоянная $R = 8{,}31\,\frac{\text{Дж}}{\text{моль}\cdot\text{К}}$.
}
\answer{%
    $
        \Delta U = \frac 32 \nu R \Delta T
            = - \frac 32 \cdot 4\,\text{моль} \cdot 8{,}31\,\frac{\text{Дж}}{\text{моль}\cdot\text{К}} \cdot 30\,\text{К}
            = -1495\,\text{Дж}.
            \text{Уменьшилась.}
    $
}
\solutionspace{40pt}

\tasknumber{6}%
\task{%
    Газу сообщили некоторое количество теплоты,
    при этом половину его он потратил на совершение работы,
    одновременно увеличив свою внутреннюю энергию на $2400\,\text{Дж}$.
    Определите количество теплоты, сообщённое газу.
}
\answer{%
    \begin{align*}
    Q &= A' + \Delta U, A' = \frac 12 Q \implies Q \cdot \cbr{1 - \frac 12} = \Delta U \implies Q = \frac{\Delta U}{1 - \frac 12} = \frac{2400\,\text{Дж}}{1 - \frac 12} \approx 4800\,\text{Дж}.
    \\
    A' &= \frac 12 Q
        = \frac 12 \cdot \frac{\Delta U}{1 - \frac 12}
        = \frac{\Delta U}{2 - 1}
        = \frac{2400\,\text{Дж}}{2 - 1} \approx 2400\,\text{Дж}.
    \end{align*}
}
\solutionspace{60pt}

\tasknumber{7}%
\task{%
    В некотором процессе газ совершил работу $100\,\text{Дж}$,
    при этом его внутренняя энергия уменьшилась на $150\,\text{Дж}$.
    Определите количество тепла, переданное при этом процессе газу.
    Явно пропишите, подводили газу тепло или же отводили.
}
\answer{%
    $
        Q = A_\text{газа} + \Delta U, A_\text{газа} = -A_\text{внешняя}
        \implies Q = A_\text{газа} + \Delta U =  100\,\text{Дж} - 150\,\text{Дж} = -50\,\text{Дж}.
        \text{ Отводили.}
    $
}

\variantsplitter

\addpersonalvariant{Софья Андрианова}

\tasknumber{1}%
\task{%
    Укажите, верны ли утверждения («да» или «нет» слева от каждого утверждения):
    \begin{enumerate}
        \item При адиабатическом расширении идеальный газ совершает ровно столько работы, сколько внутренней энергии теряет.
        % \item В силу третьего закона Ньютона, совершённая газом работа и работа, совершённая над ним, всегда равны по модулю и противоположны по знаку.
        \item Работу газа в некотором процессе можно вычислять как площадь под графиком в системе координат $PT$, главное лишь правильно расположить оси.
        % \item Дважды два три.
        \item При изохорном процессе внутренняя энергия идеального одноатомного газа не изменяется, даже если ему подводят тепло.
        \item Газ может совершить ненулевую работу в изотермическом процессе.
        % \item Адиабатический процесс лишь по воле случая не имеет приставки «изо»: в нём изменяются давление, температура и объём, но это не все макропараметры идеального газа.
        \item Полученное выражение для внутренней энергии идеального газа ($\frac 32 \nu RT$) применимо к одноатомному газу, при этом, например, уравнение состояния идеального газа применимо независимо от числа атомов в молекулах газа.
    \end{enumerate}
}
\answer{%
    $\text{да, нет, нет, да, да}$
}

\tasknumber{2}%
\task{%
    Определите давление одноатомного идеального газа, занимающего объём $6\,\text{л}$,
    если его внутренняя энергия составляет $250\,\text{Дж}$.
}
\answer{%
    $U = \frac 32 \nu R T = \frac 32 PV \implies P = \frac 23 \cdot \frac UV= \frac 23 \cdot \frac{ 250\,\text{Дж} }{ 6\,\text{л} } \approx 27\,\text{кПа}.$
}
\solutionspace{40pt}

\tasknumber{3}%
\task{%
    Газ расширился от $250\,\text{л}$ до $450\,\text{л}$.
    Давление газа при этом оставалось постоянным и равным $3{,}5\,\text{атм}$.
    Определите работу газа, ответ выразите в килоджоулях.
    $p_{\text{aтм}} = 100\,\text{кПа}$.
}
\answer{%
    $A = P\Delta V = P(V_2 - V_1) = 3{,}5\,\text{атм} \cdot \cbr{450\,\text{л} - 250\,\text{л}} = 70{,}0\,\text{кДж}.$
}
\solutionspace{40pt}

\tasknumber{4}%
\task{%
    Как изменилась внутренняя энергия одноатомного идеального газа при переходе из состояния 1 в состояние 2?
    $P_1 = 2\,\text{МПа}$, $V_1 = 5\,\text{л}$, $P_2 = 2{,}5\,\text{МПа}$, $V_2 = 2\,\text{л}$.
    Как изменилась при этом температура газа?
}
\answer{%
    \begin{align*}
    P_1V_1 &= \nu R T_1, P_2V_2 = \nu R T_2, \\
    \Delta U &= U_2-U_1 = \frac 32 \nu R T_2- \frac 32 \nu R T_1 = \frac 32 P_2 V_2 - \frac 32 P_1 V_1= \frac 32 \cdot \cbr{2{,}5\,\text{МПа} \cdot 2\,\text{л} - 2\,\text{МПа} \cdot 5\,\text{л}} = -7500\,\text{Дж}.
    \\
    \frac{T_2}{T_1} &= \frac{\frac{P_2V_2}{\nu R}}{\frac{P_1V_1}{\nu R}} = \frac{P_2V_2}{P_1V_1}= \frac{2{,}5\,\text{МПа} \cdot 2\,\text{л}}{2\,\text{МПа} \cdot 5\,\text{л}} \approx 0{,}50.
    \end{align*}
}
\solutionspace{80pt}

\tasknumber{5}%
\task{%
    $4\,\text{моль}$ идеального одноатомного газа нагрели на $10\,\text{К}$.
    Определите изменение внутренней энергии газа.
    Увеличилась она или уменьшилась?
    Универсальная газовая постоянная $R = 8{,}31\,\frac{\text{Дж}}{\text{моль}\cdot\text{К}}$.
}
\answer{%
    $
        \Delta U = \frac 32 \nu R \Delta T
            =  \frac 32 \cdot 4\,\text{моль} \cdot 8{,}31\,\frac{\text{Дж}}{\text{моль}\cdot\text{К}} \cdot 10\,\text{К}
            = 498\,\text{Дж}.
            \text{Увеличилась.}
    $
}
\solutionspace{40pt}

\tasknumber{6}%
\task{%
    Газу сообщили некоторое количество теплоты,
    при этом треть его он потратил на совершение работы,
    одновременно увеличив свою внутреннюю энергию на $2400\,\text{Дж}$.
    Определите количество теплоты, сообщённое газу.
}
\answer{%
    \begin{align*}
    Q &= A' + \Delta U, A' = \frac 13 Q \implies Q \cdot \cbr{1 - \frac 13} = \Delta U \implies Q = \frac{\Delta U}{1 - \frac 13} = \frac{2400\,\text{Дж}}{1 - \frac 13} \approx 3600\,\text{Дж}.
    \\
    A' &= \frac 13 Q
        = \frac 13 \cdot \frac{\Delta U}{1 - \frac 13}
        = \frac{\Delta U}{3 - 1}
        = \frac{2400\,\text{Дж}}{3 - 1} \approx 1200\,\text{Дж}.
    \end{align*}
}
\solutionspace{60pt}

\tasknumber{7}%
\task{%
    В некотором процессе внешние силы совершили над газом работу $200\,\text{Дж}$,
    при этом его внутренняя энергия уменьшилась на $250\,\text{Дж}$.
    Определите количество тепла, переданное при этом процессе газу.
    Явно пропишите, подводили газу тепло или же отводили.
}
\answer{%
    $
        Q = A_\text{газа} + \Delta U, A_\text{газа} = -A_\text{внешняя}
        \implies Q = A_\text{газа} + \Delta U = - 200\,\text{Дж} - 250\,\text{Дж} = -450\,\text{Дж}.
        \text{ Отводили.}
    $
}

\variantsplitter

\addpersonalvariant{Владимир Артемчук}

\tasknumber{1}%
\task{%
    Укажите, верны ли утверждения («да» или «нет» слева от каждого утверждения):
    \begin{enumerate}
        \item При адиабатическом расширении идеальный газ совершает ровно столько работы, сколько внутренней энергии теряет.
        % \item В силу третьего закона Ньютона, совершённая газом работа и работа, совершённая над ним, всегда равны по модулю и противоположны по знаку.
        \item Работу газа в некотором процессе можно вычислять как площадь под графиком в системе координат $VT$, главное лишь правильно расположить оси.
        % \item Дважды два пять.
        \item При изотермическом процессе внутренняя энергия идеального одноатомного газа не изменяется, даже если ему подводят тепло.
        \item Газ может совершить ненулевую работу в изотермическом процессе.
        % \item Адиабатический процесс лишь по воле случая не имеет приставки «изо»: в нём изменяются давление, температура и объём, но это не все макропараметры идеального газа.
        \item Полученное выражение для внутренней энергии идеального газа ($\frac 32 \nu RT$) применимо к одноатомному газу, при этом, например, уравнение состояния идеального газа применимо независимо от числа атомов в молекулах газа.
    \end{enumerate}
}
\answer{%
    $\text{да, нет, да, да, да}$
}

\tasknumber{2}%
\task{%
    Определите давление одноатомного идеального газа, занимающего объём $5\,\text{л}$,
    если его внутренняя энергия составляет $400\,\text{Дж}$.
}
\answer{%
    $U = \frac 32 \nu R T = \frac 32 PV \implies P = \frac 23 \cdot \frac UV= \frac 23 \cdot \frac{ 400\,\text{Дж} }{ 5\,\text{л} } \approx 53\,\text{кПа}.$
}
\solutionspace{40pt}

\tasknumber{3}%
\task{%
    Газ расширился от $200\,\text{л}$ до $650\,\text{л}$.
    Давление газа при этом оставалось постоянным и равным $1{,}5\,\text{атм}$.
    Определите работу газа, ответ выразите в килоджоулях.
    $p_{\text{aтм}} = 100\,\text{кПа}$.
}
\answer{%
    $A = P\Delta V = P(V_2 - V_1) = 1{,}5\,\text{атм} \cdot \cbr{650\,\text{л} - 200\,\text{л}} = 67{,}5\,\text{кДж}.$
}
\solutionspace{40pt}

\tasknumber{4}%
\task{%
    Как изменилась внутренняя энергия одноатомного идеального газа при переходе из состояния 1 в состояние 2?
    $P_1 = 2\,\text{МПа}$, $V_1 = 7\,\text{л}$, $P_2 = 1{,}5\,\text{МПа}$, $V_2 = 8\,\text{л}$.
    Как изменилась при этом температура газа?
}
\answer{%
    \begin{align*}
    P_1V_1 &= \nu R T_1, P_2V_2 = \nu R T_2, \\
    \Delta U &= U_2-U_1 = \frac 32 \nu R T_2- \frac 32 \nu R T_1 = \frac 32 P_2 V_2 - \frac 32 P_1 V_1= \frac 32 \cdot \cbr{1{,}5\,\text{МПа} \cdot 8\,\text{л} - 2\,\text{МПа} \cdot 7\,\text{л}} = -3000\,\text{Дж}.
    \\
    \frac{T_2}{T_1} &= \frac{\frac{P_2V_2}{\nu R}}{\frac{P_1V_1}{\nu R}} = \frac{P_2V_2}{P_1V_1}= \frac{1{,}5\,\text{МПа} \cdot 8\,\text{л}}{2\,\text{МПа} \cdot 7\,\text{л}} \approx 0{,}86.
    \end{align*}
}
\solutionspace{80pt}

\tasknumber{5}%
\task{%
    $5\,\text{моль}$ идеального одноатомного газа нагрели на $20\,\text{К}$.
    Определите изменение внутренней энергии газа.
    Увеличилась она или уменьшилась?
    Универсальная газовая постоянная $R = 8{,}31\,\frac{\text{Дж}}{\text{моль}\cdot\text{К}}$.
}
\answer{%
    $
        \Delta U = \frac 32 \nu R \Delta T
            =  \frac 32 \cdot 5\,\text{моль} \cdot 8{,}31\,\frac{\text{Дж}}{\text{моль}\cdot\text{К}} \cdot 20\,\text{К}
            = 1246\,\text{Дж}.
            \text{Увеличилась.}
    $
}
\solutionspace{40pt}

\tasknumber{6}%
\task{%
    Газу сообщили некоторое количество теплоты,
    при этом четверть его он потратил на совершение работы,
    одновременно увеличив свою внутреннюю энергию на $1500\,\text{Дж}$.
    Определите работу, совершённую газом.
}
\answer{%
    \begin{align*}
    Q &= A' + \Delta U, A' = \frac 14 Q \implies Q \cdot \cbr{1 - \frac 14} = \Delta U \implies Q = \frac{\Delta U}{1 - \frac 14} = \frac{1500\,\text{Дж}}{1 - \frac 14} \approx 2000\,\text{Дж}.
    \\
    A' &= \frac 14 Q
        = \frac 14 \cdot \frac{\Delta U}{1 - \frac 14}
        = \frac{\Delta U}{4 - 1}
        = \frac{1500\,\text{Дж}}{4 - 1} \approx 500\,\text{Дж}.
    \end{align*}
}
\solutionspace{60pt}

\tasknumber{7}%
\task{%
    В некотором процессе газ совершил работу $100\,\text{Дж}$,
    при этом его внутренняя энергия увеличилась на $150\,\text{Дж}$.
    Определите количество тепла, переданное при этом процессе газу.
    Явно пропишите, подводили газу тепло или же отводили.
}
\answer{%
    $
        Q = A_\text{газа} + \Delta U, A_\text{газа} = -A_\text{внешняя}
        \implies Q = A_\text{газа} + \Delta U =  100\,\text{Дж} +  150\,\text{Дж} = 250\,\text{Дж}.
        \text{ Подводили.}
    $
}

\variantsplitter

\addpersonalvariant{Софья Белянкина}

\tasknumber{1}%
\task{%
    Укажите, верны ли утверждения («да» или «нет» слева от каждого утверждения):
    \begin{enumerate}
        \item При адиабатическом расширении идеальный газ совершает ровно столько работы, сколько внутренней энергии теряет.
        % \item В силу третьего закона Ньютона, совершённая газом работа и работа, совершённая над ним, всегда равны по модулю и противоположны по знаку.
        \item Работу газа в некотором процессе можно вычислять как площадь под графиком в системе координат $VT$, главное лишь правильно расположить оси.
        % \item Дважды два пять.
        \item При изобарном процессе внутренняя энергия идеального одноатомного газа не изменяется, даже если ему подводят тепло.
        \item Газ может совершить ненулевую работу в изохорном процессе.
        % \item Адиабатический процесс лишь по воле случая не имеет приставки «изо»: в нём изменяются давление, температура и объём, но это не все макропараметры идеального газа.
        \item Полученное выражение для внутренней энергии идеального газа ($\frac 32 \nu RT$) применимо к трёхатомному газу, при этом, например, уравнение состояния идеального газа применимо независимо от числа атомов в молекулах газа.
    \end{enumerate}
}
\answer{%
    $\text{да, нет, нет, нет, нет}$
}

\tasknumber{2}%
\task{%
    Определите давление одноатомного идеального газа, занимающего объём $4\,\text{л}$,
    если его внутренняя энергия составляет $300\,\text{Дж}$.
}
\answer{%
    $U = \frac 32 \nu R T = \frac 32 PV \implies P = \frac 23 \cdot \frac UV= \frac 23 \cdot \frac{ 300\,\text{Дж} }{ 4\,\text{л} } \approx 50\,\text{кПа}.$
}
\solutionspace{40pt}

\tasknumber{3}%
\task{%
    Газ расширился от $250\,\text{л}$ до $450\,\text{л}$.
    Давление газа при этом оставалось постоянным и равным $1{,}2\,\text{атм}$.
    Определите работу газа, ответ выразите в килоджоулях.
    $p_{\text{aтм}} = 100\,\text{кПа}$.
}
\answer{%
    $A = P\Delta V = P(V_2 - V_1) = 1{,}2\,\text{атм} \cdot \cbr{450\,\text{л} - 250\,\text{л}} = 24{,}0\,\text{кДж}.$
}
\solutionspace{40pt}

\tasknumber{4}%
\task{%
    Как изменилась внутренняя энергия одноатомного идеального газа при переходе из состояния 1 в состояние 2?
    $P_1 = 2\,\text{МПа}$, $V_1 = 3\,\text{л}$, $P_2 = 4{,}5\,\text{МПа}$, $V_2 = 8\,\text{л}$.
    Как изменилась при этом температура газа?
}
\answer{%
    \begin{align*}
    P_1V_1 &= \nu R T_1, P_2V_2 = \nu R T_2, \\
    \Delta U &= U_2-U_1 = \frac 32 \nu R T_2- \frac 32 \nu R T_1 = \frac 32 P_2 V_2 - \frac 32 P_1 V_1= \frac 32 \cdot \cbr{4{,}5\,\text{МПа} \cdot 8\,\text{л} - 2\,\text{МПа} \cdot 3\,\text{л}} = 45000\,\text{Дж}.
    \\
    \frac{T_2}{T_1} &= \frac{\frac{P_2V_2}{\nu R}}{\frac{P_1V_1}{\nu R}} = \frac{P_2V_2}{P_1V_1}= \frac{4{,}5\,\text{МПа} \cdot 8\,\text{л}}{2\,\text{МПа} \cdot 3\,\text{л}} \approx 6{,}00.
    \end{align*}
}
\solutionspace{80pt}

\tasknumber{5}%
\task{%
    $4\,\text{моль}$ идеального одноатомного газа охладили на $10\,\text{К}$.
    Определите изменение внутренней энергии газа.
    Увеличилась она или уменьшилась?
    Универсальная газовая постоянная $R = 8{,}31\,\frac{\text{Дж}}{\text{моль}\cdot\text{К}}$.
}
\answer{%
    $
        \Delta U = \frac 32 \nu R \Delta T
            = - \frac 32 \cdot 4\,\text{моль} \cdot 8{,}31\,\frac{\text{Дж}}{\text{моль}\cdot\text{К}} \cdot 10\,\text{К}
            = -498\,\text{Дж}.
            \text{Уменьшилась.}
    $
}
\solutionspace{40pt}

\tasknumber{6}%
\task{%
    Газу сообщили некоторое количество теплоты,
    при этом треть его он потратил на совершение работы,
    одновременно увеличив свою внутреннюю энергию на $1500\,\text{Дж}$.
    Определите работу, совершённую газом.
}
\answer{%
    \begin{align*}
    Q &= A' + \Delta U, A' = \frac 13 Q \implies Q \cdot \cbr{1 - \frac 13} = \Delta U \implies Q = \frac{\Delta U}{1 - \frac 13} = \frac{1500\,\text{Дж}}{1 - \frac 13} \approx 2250\,\text{Дж}.
    \\
    A' &= \frac 13 Q
        = \frac 13 \cdot \frac{\Delta U}{1 - \frac 13}
        = \frac{\Delta U}{3 - 1}
        = \frac{1500\,\text{Дж}}{3 - 1} \approx 750\,\text{Дж}.
    \end{align*}
}
\solutionspace{60pt}

\tasknumber{7}%
\task{%
    В некотором процессе газ совершил работу $300\,\text{Дж}$,
    при этом его внутренняя энергия уменьшилась на $250\,\text{Дж}$.
    Определите количество тепла, переданное при этом процессе газу.
    Явно пропишите, подводили газу тепло или же отводили.
}
\answer{%
    $
        Q = A_\text{газа} + \Delta U, A_\text{газа} = -A_\text{внешняя}
        \implies Q = A_\text{газа} + \Delta U =  300\,\text{Дж} - 250\,\text{Дж} = 50\,\text{Дж}.
        \text{ Подводили.}
    $
}

\variantsplitter

\addpersonalvariant{Варвара Егиазарян}

\tasknumber{1}%
\task{%
    Укажите, верны ли утверждения («да» или «нет» слева от каждого утверждения):
    \begin{enumerate}
        \item При изобарном расширении идеальный газ совершает ровно столько работы, сколько внутренней энергии теряет.
        % \item В силу третьего закона Ньютона, совершённая газом работа и работа, совершённая над ним, всегда равны по модулю и противоположны по знаку.
        \item Работу газа в некотором процессе можно вычислять как площадь под графиком в системе координат $PT$, главное лишь правильно расположить оси.
        % \item Дважды два три.
        \item При изохорном процессе внутренняя энергия идеального одноатомного газа не изменяется, даже если ему подводят тепло.
        \item Газ может совершить ненулевую работу в изотермическом процессе.
        % \item Адиабатический процесс лишь по воле случая не имеет приставки «изо»: в нём изменяются давление, температура и объём, но это не все макропараметры идеального газа.
        \item Полученное выражение для внутренней энергии идеального газа ($\frac 32 \nu RT$) применимо к двухоатомному газу, при этом, например, уравнение состояния идеального газа применимо независимо от числа атомов в молекулах газа.
    \end{enumerate}
}
\answer{%
    $\text{нет, нет, нет, да, нет}$
}

\tasknumber{2}%
\task{%
    Определите давление одноатомного идеального газа, занимающего объём $2\,\text{л}$,
    если его внутренняя энергия составляет $300\,\text{Дж}$.
}
\answer{%
    $U = \frac 32 \nu R T = \frac 32 PV \implies P = \frac 23 \cdot \frac UV= \frac 23 \cdot \frac{ 300\,\text{Дж} }{ 2\,\text{л} } \approx 100\,\text{кПа}.$
}
\solutionspace{40pt}

\tasknumber{3}%
\task{%
    Газ расширился от $200\,\text{л}$ до $650\,\text{л}$.
    Давление газа при этом оставалось постоянным и равным $1{,}2\,\text{атм}$.
    Определите работу газа, ответ выразите в килоджоулях.
    $p_{\text{aтм}} = 100\,\text{кПа}$.
}
\answer{%
    $A = P\Delta V = P(V_2 - V_1) = 1{,}2\,\text{атм} \cdot \cbr{650\,\text{л} - 200\,\text{л}} = 54{,}0\,\text{кДж}.$
}
\solutionspace{40pt}

\tasknumber{4}%
\task{%
    Как изменилась внутренняя энергия одноатомного идеального газа при переходе из состояния 1 в состояние 2?
    $P_1 = 4\,\text{МПа}$, $V_1 = 5\,\text{л}$, $P_2 = 3{,}5\,\text{МПа}$, $V_2 = 2\,\text{л}$.
    Как изменилась при этом температура газа?
}
\answer{%
    \begin{align*}
    P_1V_1 &= \nu R T_1, P_2V_2 = \nu R T_2, \\
    \Delta U &= U_2-U_1 = \frac 32 \nu R T_2- \frac 32 \nu R T_1 = \frac 32 P_2 V_2 - \frac 32 P_1 V_1= \frac 32 \cdot \cbr{3{,}5\,\text{МПа} \cdot 2\,\text{л} - 4\,\text{МПа} \cdot 5\,\text{л}} = -19500\,\text{Дж}.
    \\
    \frac{T_2}{T_1} &= \frac{\frac{P_2V_2}{\nu R}}{\frac{P_1V_1}{\nu R}} = \frac{P_2V_2}{P_1V_1}= \frac{3{,}5\,\text{МПа} \cdot 2\,\text{л}}{4\,\text{МПа} \cdot 5\,\text{л}} \approx 0{,}35.
    \end{align*}
}
\solutionspace{80pt}

\tasknumber{5}%
\task{%
    $3\,\text{моль}$ идеального одноатомного газа охладили на $30\,\text{К}$.
    Определите изменение внутренней энергии газа.
    Увеличилась она или уменьшилась?
    Универсальная газовая постоянная $R = 8{,}31\,\frac{\text{Дж}}{\text{моль}\cdot\text{К}}$.
}
\answer{%
    $
        \Delta U = \frac 32 \nu R \Delta T
            = - \frac 32 \cdot 3\,\text{моль} \cdot 8{,}31\,\frac{\text{Дж}}{\text{моль}\cdot\text{К}} \cdot 30\,\text{К}
            = -1121\,\text{Дж}.
            \text{Уменьшилась.}
    $
}
\solutionspace{40pt}

\tasknumber{6}%
\task{%
    Газу сообщили некоторое количество теплоты,
    при этом половину его он потратил на совершение работы,
    одновременно увеличив свою внутреннюю энергию на $1500\,\text{Дж}$.
    Определите работу, совершённую газом.
}
\answer{%
    \begin{align*}
    Q &= A' + \Delta U, A' = \frac 12 Q \implies Q \cdot \cbr{1 - \frac 12} = \Delta U \implies Q = \frac{\Delta U}{1 - \frac 12} = \frac{1500\,\text{Дж}}{1 - \frac 12} \approx 3000\,\text{Дж}.
    \\
    A' &= \frac 12 Q
        = \frac 12 \cdot \frac{\Delta U}{1 - \frac 12}
        = \frac{\Delta U}{2 - 1}
        = \frac{1500\,\text{Дж}}{2 - 1} \approx 1500\,\text{Дж}.
    \end{align*}
}
\solutionspace{60pt}

\tasknumber{7}%
\task{%
    В некотором процессе внешние силы совершили над газом работу $200\,\text{Дж}$,
    при этом его внутренняя энергия увеличилась на $250\,\text{Дж}$.
    Определите количество тепла, переданное при этом процессе газу.
    Явно пропишите, подводили газу тепло или же отводили.
}
\answer{%
    $
        Q = A_\text{газа} + \Delta U, A_\text{газа} = -A_\text{внешняя}
        \implies Q = A_\text{газа} + \Delta U = - 200\,\text{Дж} +  250\,\text{Дж} = 50\,\text{Дж}.
        \text{ Подводили.}
    $
}

\variantsplitter

\addpersonalvariant{Владислав Емелин}

\tasknumber{1}%
\task{%
    Укажите, верны ли утверждения («да» или «нет» слева от каждого утверждения):
    \begin{enumerate}
        \item При изобарном расширении идеальный газ совершает ровно столько работы, сколько внутренней энергии теряет.
        % \item В силу третьего закона Ньютона, совершённая газом работа и работа, совершённая над ним, всегда равны по модулю и противоположны по знаку.
        \item Работу газа в некотором процессе можно вычислять как площадь под графиком в системе координат $PV$, главное лишь правильно расположить оси.
        % \item Дважды два четыре.
        \item При изохорном процессе внутренняя энергия идеального одноатомного газа не изменяется, даже если ему подводят тепло.
        \item Газ может совершить ненулевую работу в изотермическом процессе.
        % \item Адиабатический процесс лишь по воле случая не имеет приставки «изо»: в нём изменяются давление, температура и объём, но это не все макропараметры идеального газа.
        \item Полученное выражение для внутренней энергии идеального газа ($\frac 32 \nu RT$) применимо к трёхатомному газу, при этом, например, уравнение состояния идеального газа применимо независимо от числа атомов в молекулах газа.
    \end{enumerate}
}
\answer{%
    $\text{нет, да, нет, да, нет}$
}

\tasknumber{2}%
\task{%
    Определите давление одноатомного идеального газа, занимающего объём $6\,\text{л}$,
    если его внутренняя энергия составляет $500\,\text{Дж}$.
}
\answer{%
    $U = \frac 32 \nu R T = \frac 32 PV \implies P = \frac 23 \cdot \frac UV= \frac 23 \cdot \frac{ 500\,\text{Дж} }{ 6\,\text{л} } \approx 55\,\text{кПа}.$
}
\solutionspace{40pt}

\tasknumber{3}%
\task{%
    Газ расширился от $200\,\text{л}$ до $450\,\text{л}$.
    Давление газа при этом оставалось постоянным и равным $1{,}5\,\text{атм}$.
    Определите работу газа, ответ выразите в килоджоулях.
    $p_{\text{aтм}} = 100\,\text{кПа}$.
}
\answer{%
    $A = P\Delta V = P(V_2 - V_1) = 1{,}5\,\text{атм} \cdot \cbr{450\,\text{л} - 200\,\text{л}} = 37{,}5\,\text{кДж}.$
}
\solutionspace{40pt}

\tasknumber{4}%
\task{%
    Как изменилась внутренняя энергия одноатомного идеального газа при переходе из состояния 1 в состояние 2?
    $P_1 = 3\,\text{МПа}$, $V_1 = 3\,\text{л}$, $P_2 = 3{,}5\,\text{МПа}$, $V_2 = 6\,\text{л}$.
    Как изменилась при этом температура газа?
}
\answer{%
    \begin{align*}
    P_1V_1 &= \nu R T_1, P_2V_2 = \nu R T_2, \\
    \Delta U &= U_2-U_1 = \frac 32 \nu R T_2- \frac 32 \nu R T_1 = \frac 32 P_2 V_2 - \frac 32 P_1 V_1= \frac 32 \cdot \cbr{3{,}5\,\text{МПа} \cdot 6\,\text{л} - 3\,\text{МПа} \cdot 3\,\text{л}} = 18000\,\text{Дж}.
    \\
    \frac{T_2}{T_1} &= \frac{\frac{P_2V_2}{\nu R}}{\frac{P_1V_1}{\nu R}} = \frac{P_2V_2}{P_1V_1}= \frac{3{,}5\,\text{МПа} \cdot 6\,\text{л}}{3\,\text{МПа} \cdot 3\,\text{л}} \approx 2{,}33.
    \end{align*}
}
\solutionspace{80pt}

\tasknumber{5}%
\task{%
    $5\,\text{моль}$ идеального одноатомного газа нагрели на $10\,\text{К}$.
    Определите изменение внутренней энергии газа.
    Увеличилась она или уменьшилась?
    Универсальная газовая постоянная $R = 8{,}31\,\frac{\text{Дж}}{\text{моль}\cdot\text{К}}$.
}
\answer{%
    $
        \Delta U = \frac 32 \nu R \Delta T
            =  \frac 32 \cdot 5\,\text{моль} \cdot 8{,}31\,\frac{\text{Дж}}{\text{моль}\cdot\text{К}} \cdot 10\,\text{К}
            = 623\,\text{Дж}.
            \text{Увеличилась.}
    $
}
\solutionspace{40pt}

\tasknumber{6}%
\task{%
    Газу сообщили некоторое количество теплоты,
    при этом четверть его он потратил на совершение работы,
    одновременно увеличив свою внутреннюю энергию на $1200\,\text{Дж}$.
    Определите количество теплоты, сообщённое газу.
}
\answer{%
    \begin{align*}
    Q &= A' + \Delta U, A' = \frac 14 Q \implies Q \cdot \cbr{1 - \frac 14} = \Delta U \implies Q = \frac{\Delta U}{1 - \frac 14} = \frac{1200\,\text{Дж}}{1 - \frac 14} \approx 1600\,\text{Дж}.
    \\
    A' &= \frac 14 Q
        = \frac 14 \cdot \frac{\Delta U}{1 - \frac 14}
        = \frac{\Delta U}{4 - 1}
        = \frac{1200\,\text{Дж}}{4 - 1} \approx 400\,\text{Дж}.
    \end{align*}
}
\solutionspace{60pt}

\tasknumber{7}%
\task{%
    В некотором процессе газ совершил работу $200\,\text{Дж}$,
    при этом его внутренняя энергия уменьшилась на $450\,\text{Дж}$.
    Определите количество тепла, переданное при этом процессе газу.
    Явно пропишите, подводили газу тепло или же отводили.
}
\answer{%
    $
        Q = A_\text{газа} + \Delta U, A_\text{газа} = -A_\text{внешняя}
        \implies Q = A_\text{газа} + \Delta U =  200\,\text{Дж} - 450\,\text{Дж} = -250\,\text{Дж}.
        \text{ Отводили.}
    $
}

\variantsplitter

\addpersonalvariant{Артём Жичин}

\tasknumber{1}%
\task{%
    Укажите, верны ли утверждения («да» или «нет» слева от каждого утверждения):
    \begin{enumerate}
        \item При изобарном расширении идеальный газ совершает ровно столько работы, сколько внутренней энергии теряет.
        % \item В силу третьего закона Ньютона, совершённая газом работа и работа, совершённая над ним, всегда равны по модулю и противоположны по знаку.
        \item Работу газа в некотором процессе можно вычислять как площадь под графиком в системе координат $VT$, главное лишь правильно расположить оси.
        % \item Дважды два три.
        \item При изохорном процессе внутренняя энергия идеального одноатомного газа не изменяется, даже если ему подводят тепло.
        \item Газ может совершить ненулевую работу в изобарном процессе.
        % \item Адиабатический процесс лишь по воле случая не имеет приставки «изо»: в нём изменяются давление, температура и объём, но это не все макропараметры идеального газа.
        \item Полученное выражение для внутренней энергии идеального газа ($\frac 32 \nu RT$) применимо к трёхатомному газу, при этом, например, уравнение состояния идеального газа применимо независимо от числа атомов в молекулах газа.
    \end{enumerate}
}
\answer{%
    $\text{нет, нет, нет, да, нет}$
}

\tasknumber{2}%
\task{%
    Определите давление одноатомного идеального газа, занимающего объём $5\,\text{л}$,
    если его внутренняя энергия составляет $300\,\text{Дж}$.
}
\answer{%
    $U = \frac 32 \nu R T = \frac 32 PV \implies P = \frac 23 \cdot \frac UV= \frac 23 \cdot \frac{ 300\,\text{Дж} }{ 5\,\text{л} } \approx 40\,\text{кПа}.$
}
\solutionspace{40pt}

\tasknumber{3}%
\task{%
    Газ расширился от $250\,\text{л}$ до $650\,\text{л}$.
    Давление газа при этом оставалось постоянным и равным $2{,}5\,\text{атм}$.
    Определите работу газа, ответ выразите в килоджоулях.
    $p_{\text{aтм}} = 100\,\text{кПа}$.
}
\answer{%
    $A = P\Delta V = P(V_2 - V_1) = 2{,}5\,\text{атм} \cdot \cbr{650\,\text{л} - 250\,\text{л}} = 100{,}0\,\text{кДж}.$
}
\solutionspace{40pt}

\tasknumber{4}%
\task{%
    Как изменилась внутренняя энергия одноатомного идеального газа при переходе из состояния 1 в состояние 2?
    $P_1 = 3\,\text{МПа}$, $V_1 = 3\,\text{л}$, $P_2 = 3{,}5\,\text{МПа}$, $V_2 = 8\,\text{л}$.
    Как изменилась при этом температура газа?
}
\answer{%
    \begin{align*}
    P_1V_1 &= \nu R T_1, P_2V_2 = \nu R T_2, \\
    \Delta U &= U_2-U_1 = \frac 32 \nu R T_2- \frac 32 \nu R T_1 = \frac 32 P_2 V_2 - \frac 32 P_1 V_1= \frac 32 \cdot \cbr{3{,}5\,\text{МПа} \cdot 8\,\text{л} - 3\,\text{МПа} \cdot 3\,\text{л}} = 28500\,\text{Дж}.
    \\
    \frac{T_2}{T_1} &= \frac{\frac{P_2V_2}{\nu R}}{\frac{P_1V_1}{\nu R}} = \frac{P_2V_2}{P_1V_1}= \frac{3{,}5\,\text{МПа} \cdot 8\,\text{л}}{3\,\text{МПа} \cdot 3\,\text{л}} \approx 3{,}11.
    \end{align*}
}
\solutionspace{80pt}

\tasknumber{5}%
\task{%
    $3\,\text{моль}$ идеального одноатомного газа нагрели на $30\,\text{К}$.
    Определите изменение внутренней энергии газа.
    Увеличилась она или уменьшилась?
    Универсальная газовая постоянная $R = 8{,}31\,\frac{\text{Дж}}{\text{моль}\cdot\text{К}}$.
}
\answer{%
    $
        \Delta U = \frac 32 \nu R \Delta T
            =  \frac 32 \cdot 3\,\text{моль} \cdot 8{,}31\,\frac{\text{Дж}}{\text{моль}\cdot\text{К}} \cdot 30\,\text{К}
            = 1121\,\text{Дж}.
            \text{Увеличилась.}
    $
}
\solutionspace{40pt}

\tasknumber{6}%
\task{%
    Газу сообщили некоторое количество теплоты,
    при этом половину его он потратил на совершение работы,
    одновременно увеличив свою внутреннюю энергию на $3000\,\text{Дж}$.
    Определите количество теплоты, сообщённое газу.
}
\answer{%
    \begin{align*}
    Q &= A' + \Delta U, A' = \frac 12 Q \implies Q \cdot \cbr{1 - \frac 12} = \Delta U \implies Q = \frac{\Delta U}{1 - \frac 12} = \frac{3000\,\text{Дж}}{1 - \frac 12} \approx 6000\,\text{Дж}.
    \\
    A' &= \frac 12 Q
        = \frac 12 \cdot \frac{\Delta U}{1 - \frac 12}
        = \frac{\Delta U}{2 - 1}
        = \frac{3000\,\text{Дж}}{2 - 1} \approx 3000\,\text{Дж}.
    \end{align*}
}
\solutionspace{60pt}

\tasknumber{7}%
\task{%
    В некотором процессе газ совершил работу $100\,\text{Дж}$,
    при этом его внутренняя энергия уменьшилась на $350\,\text{Дж}$.
    Определите количество тепла, переданное при этом процессе газу.
    Явно пропишите, подводили газу тепло или же отводили.
}
\answer{%
    $
        Q = A_\text{газа} + \Delta U, A_\text{газа} = -A_\text{внешняя}
        \implies Q = A_\text{газа} + \Delta U =  100\,\text{Дж} - 350\,\text{Дж} = -250\,\text{Дж}.
        \text{ Отводили.}
    $
}

\variantsplitter

\addpersonalvariant{Дарья Кошман}

\tasknumber{1}%
\task{%
    Укажите, верны ли утверждения («да» или «нет» слева от каждого утверждения):
    \begin{enumerate}
        \item При изобарном расширении идеальный газ совершает ровно столько работы, сколько внутренней энергии теряет.
        % \item В силу третьего закона Ньютона, совершённая газом работа и работа, совершённая над ним, всегда равны по модулю и противоположны по знаку.
        \item Работу газа в некотором процессе можно вычислять как площадь под графиком в системе координат $PT$, главное лишь правильно расположить оси.
        % \item Дважды два три.
        \item При изотермическом процессе внутренняя энергия идеального одноатомного газа не изменяется, даже если ему подводят тепло.
        \item Газ может совершить ненулевую работу в изохорном процессе.
        % \item Адиабатический процесс лишь по воле случая не имеет приставки «изо»: в нём изменяются давление, температура и объём, но это не все макропараметры идеального газа.
        \item Полученное выражение для внутренней энергии идеального газа ($\frac 32 \nu RT$) применимо к двухоатомному газу, при этом, например, уравнение состояния идеального газа применимо независимо от числа атомов в молекулах газа.
    \end{enumerate}
}
\answer{%
    $\text{нет, нет, да, нет, нет}$
}

\tasknumber{2}%
\task{%
    Определите давление одноатомного идеального газа, занимающего объём $2\,\text{л}$,
    если его внутренняя энергия составляет $300\,\text{Дж}$.
}
\answer{%
    $U = \frac 32 \nu R T = \frac 32 PV \implies P = \frac 23 \cdot \frac UV= \frac 23 \cdot \frac{ 300\,\text{Дж} }{ 2\,\text{л} } \approx 100\,\text{кПа}.$
}
\solutionspace{40pt}

\tasknumber{3}%
\task{%
    Газ расширился от $350\,\text{л}$ до $650\,\text{л}$.
    Давление газа при этом оставалось постоянным и равным $2{,}5\,\text{атм}$.
    Определите работу газа, ответ выразите в килоджоулях.
    $p_{\text{aтм}} = 100\,\text{кПа}$.
}
\answer{%
    $A = P\Delta V = P(V_2 - V_1) = 2{,}5\,\text{атм} \cdot \cbr{650\,\text{л} - 350\,\text{л}} = 75{,}0\,\text{кДж}.$
}
\solutionspace{40pt}

\tasknumber{4}%
\task{%
    Как изменилась внутренняя энергия одноатомного идеального газа при переходе из состояния 1 в состояние 2?
    $P_1 = 4\,\text{МПа}$, $V_1 = 3\,\text{л}$, $P_2 = 4{,}5\,\text{МПа}$, $V_2 = 2\,\text{л}$.
    Как изменилась при этом температура газа?
}
\answer{%
    \begin{align*}
    P_1V_1 &= \nu R T_1, P_2V_2 = \nu R T_2, \\
    \Delta U &= U_2-U_1 = \frac 32 \nu R T_2- \frac 32 \nu R T_1 = \frac 32 P_2 V_2 - \frac 32 P_1 V_1= \frac 32 \cdot \cbr{4{,}5\,\text{МПа} \cdot 2\,\text{л} - 4\,\text{МПа} \cdot 3\,\text{л}} = -4500\,\text{Дж}.
    \\
    \frac{T_2}{T_1} &= \frac{\frac{P_2V_2}{\nu R}}{\frac{P_1V_1}{\nu R}} = \frac{P_2V_2}{P_1V_1}= \frac{4{,}5\,\text{МПа} \cdot 2\,\text{л}}{4\,\text{МПа} \cdot 3\,\text{л}} \approx 0{,}75.
    \end{align*}
}
\solutionspace{80pt}

\tasknumber{5}%
\task{%
    $2\,\text{моль}$ идеального одноатомного газа охладили на $20\,\text{К}$.
    Определите изменение внутренней энергии газа.
    Увеличилась она или уменьшилась?
    Универсальная газовая постоянная $R = 8{,}31\,\frac{\text{Дж}}{\text{моль}\cdot\text{К}}$.
}
\answer{%
    $
        \Delta U = \frac 32 \nu R \Delta T
            = - \frac 32 \cdot 2\,\text{моль} \cdot 8{,}31\,\frac{\text{Дж}}{\text{моль}\cdot\text{К}} \cdot 20\,\text{К}
            = -498\,\text{Дж}.
            \text{Уменьшилась.}
    $
}
\solutionspace{40pt}

\tasknumber{6}%
\task{%
    Газу сообщили некоторое количество теплоты,
    при этом половину его он потратил на совершение работы,
    одновременно увеличив свою внутреннюю энергию на $3000\,\text{Дж}$.
    Определите количество теплоты, сообщённое газу.
}
\answer{%
    \begin{align*}
    Q &= A' + \Delta U, A' = \frac 12 Q \implies Q \cdot \cbr{1 - \frac 12} = \Delta U \implies Q = \frac{\Delta U}{1 - \frac 12} = \frac{3000\,\text{Дж}}{1 - \frac 12} \approx 6000\,\text{Дж}.
    \\
    A' &= \frac 12 Q
        = \frac 12 \cdot \frac{\Delta U}{1 - \frac 12}
        = \frac{\Delta U}{2 - 1}
        = \frac{3000\,\text{Дж}}{2 - 1} \approx 3000\,\text{Дж}.
    \end{align*}
}
\solutionspace{60pt}

\tasknumber{7}%
\task{%
    В некотором процессе внешние силы совершили над газом работу $100\,\text{Дж}$,
    при этом его внутренняя энергия уменьшилась на $150\,\text{Дж}$.
    Определите количество тепла, переданное при этом процессе газу.
    Явно пропишите, подводили газу тепло или же отводили.
}
\answer{%
    $
        Q = A_\text{газа} + \Delta U, A_\text{газа} = -A_\text{внешняя}
        \implies Q = A_\text{газа} + \Delta U = - 100\,\text{Дж} - 150\,\text{Дж} = -250\,\text{Дж}.
        \text{ Отводили.}
    $
}

\variantsplitter

\addpersonalvariant{Анна Кузьмичёва}

\tasknumber{1}%
\task{%
    Укажите, верны ли утверждения («да» или «нет» слева от каждого утверждения):
    \begin{enumerate}
        \item При адиабатическом расширении идеальный газ совершает ровно столько работы, сколько внутренней энергии теряет.
        % \item В силу третьего закона Ньютона, совершённая газом работа и работа, совершённая над ним, всегда равны по модулю и противоположны по знаку.
        \item Работу газа в некотором процессе можно вычислять как площадь под графиком в системе координат $PT$, главное лишь правильно расположить оси.
        % \item Дважды два четыре.
        \item При изохорном процессе внутренняя энергия идеального одноатомного газа не изменяется, даже если ему подводят тепло.
        \item Газ может совершить ненулевую работу в изотермическом процессе.
        % \item Адиабатический процесс лишь по воле случая не имеет приставки «изо»: в нём изменяются давление, температура и объём, но это не все макропараметры идеального газа.
        \item Полученное выражение для внутренней энергии идеального газа ($\frac 32 \nu RT$) применимо к одноатомному газу, при этом, например, уравнение состояния идеального газа применимо независимо от числа атомов в молекулах газа.
    \end{enumerate}
}
\answer{%
    $\text{да, нет, нет, да, да}$
}

\tasknumber{2}%
\task{%
    Определите давление одноатомного идеального газа, занимающего объём $3\,\text{л}$,
    если его внутренняя энергия составляет $250\,\text{Дж}$.
}
\answer{%
    $U = \frac 32 \nu R T = \frac 32 PV \implies P = \frac 23 \cdot \frac UV= \frac 23 \cdot \frac{ 250\,\text{Дж} }{ 3\,\text{л} } \approx 55\,\text{кПа}.$
}
\solutionspace{40pt}

\tasknumber{3}%
\task{%
    Газ расширился от $350\,\text{л}$ до $550\,\text{л}$.
    Давление газа при этом оставалось постоянным и равным $1{,}8\,\text{атм}$.
    Определите работу газа, ответ выразите в килоджоулях.
    $p_{\text{aтм}} = 100\,\text{кПа}$.
}
\answer{%
    $A = P\Delta V = P(V_2 - V_1) = 1{,}8\,\text{атм} \cdot \cbr{550\,\text{л} - 350\,\text{л}} = 36{,}0\,\text{кДж}.$
}
\solutionspace{40pt}

\tasknumber{4}%
\task{%
    Как изменилась внутренняя энергия одноатомного идеального газа при переходе из состояния 1 в состояние 2?
    $P_1 = 3\,\text{МПа}$, $V_1 = 7\,\text{л}$, $P_2 = 3{,}5\,\text{МПа}$, $V_2 = 2\,\text{л}$.
    Как изменилась при этом температура газа?
}
\answer{%
    \begin{align*}
    P_1V_1 &= \nu R T_1, P_2V_2 = \nu R T_2, \\
    \Delta U &= U_2-U_1 = \frac 32 \nu R T_2- \frac 32 \nu R T_1 = \frac 32 P_2 V_2 - \frac 32 P_1 V_1= \frac 32 \cdot \cbr{3{,}5\,\text{МПа} \cdot 2\,\text{л} - 3\,\text{МПа} \cdot 7\,\text{л}} = -21000\,\text{Дж}.
    \\
    \frac{T_2}{T_1} &= \frac{\frac{P_2V_2}{\nu R}}{\frac{P_1V_1}{\nu R}} = \frac{P_2V_2}{P_1V_1}= \frac{3{,}5\,\text{МПа} \cdot 2\,\text{л}}{3\,\text{МПа} \cdot 7\,\text{л}} \approx 0{,}33.
    \end{align*}
}
\solutionspace{80pt}

\tasknumber{5}%
\task{%
    $2\,\text{моль}$ идеального одноатомного газа нагрели на $30\,\text{К}$.
    Определите изменение внутренней энергии газа.
    Увеличилась она или уменьшилась?
    Универсальная газовая постоянная $R = 8{,}31\,\frac{\text{Дж}}{\text{моль}\cdot\text{К}}$.
}
\answer{%
    $
        \Delta U = \frac 32 \nu R \Delta T
            =  \frac 32 \cdot 2\,\text{моль} \cdot 8{,}31\,\frac{\text{Дж}}{\text{моль}\cdot\text{К}} \cdot 30\,\text{К}
            = 747\,\text{Дж}.
            \text{Увеличилась.}
    $
}
\solutionspace{40pt}

\tasknumber{6}%
\task{%
    Газу сообщили некоторое количество теплоты,
    при этом четверть его он потратил на совершение работы,
    одновременно увеличив свою внутреннюю энергию на $2400\,\text{Дж}$.
    Определите количество теплоты, сообщённое газу.
}
\answer{%
    \begin{align*}
    Q &= A' + \Delta U, A' = \frac 14 Q \implies Q \cdot \cbr{1 - \frac 14} = \Delta U \implies Q = \frac{\Delta U}{1 - \frac 14} = \frac{2400\,\text{Дж}}{1 - \frac 14} \approx 3200\,\text{Дж}.
    \\
    A' &= \frac 14 Q
        = \frac 14 \cdot \frac{\Delta U}{1 - \frac 14}
        = \frac{\Delta U}{4 - 1}
        = \frac{2400\,\text{Дж}}{4 - 1} \approx 800\,\text{Дж}.
    \end{align*}
}
\solutionspace{60pt}

\tasknumber{7}%
\task{%
    В некотором процессе внешние силы совершили над газом работу $300\,\text{Дж}$,
    при этом его внутренняя энергия уменьшилась на $350\,\text{Дж}$.
    Определите количество тепла, переданное при этом процессе газу.
    Явно пропишите, подводили газу тепло или же отводили.
}
\answer{%
    $
        Q = A_\text{газа} + \Delta U, A_\text{газа} = -A_\text{внешняя}
        \implies Q = A_\text{газа} + \Delta U = - 300\,\text{Дж} - 350\,\text{Дж} = -650\,\text{Дж}.
        \text{ Отводили.}
    $
}

\variantsplitter

\addpersonalvariant{Алёна Куприянова}

\tasknumber{1}%
\task{%
    Укажите, верны ли утверждения («да» или «нет» слева от каждого утверждения):
    \begin{enumerate}
        \item При адиабатическом расширении идеальный газ совершает ровно столько работы, сколько внутренней энергии теряет.
        % \item В силу третьего закона Ньютона, совершённая газом работа и работа, совершённая над ним, всегда равны по модулю и противоположны по знаку.
        \item Работу газа в некотором процессе можно вычислять как площадь под графиком в системе координат $VT$, главное лишь правильно расположить оси.
        % \item Дважды два три.
        \item При изобарном процессе внутренняя энергия идеального одноатомного газа не изменяется, даже если ему подводят тепло.
        \item Газ может совершить ненулевую работу в изохорном процессе.
        % \item Адиабатический процесс лишь по воле случая не имеет приставки «изо»: в нём изменяются давление, температура и объём, но это не все макропараметры идеального газа.
        \item Полученное выражение для внутренней энергии идеального газа ($\frac 32 \nu RT$) применимо к одноатомному газу, при этом, например, уравнение состояния идеального газа применимо независимо от числа атомов в молекулах газа.
    \end{enumerate}
}
\answer{%
    $\text{да, нет, нет, нет, да}$
}

\tasknumber{2}%
\task{%
    Определите давление одноатомного идеального газа, занимающего объём $3\,\text{л}$,
    если его внутренняя энергия составляет $400\,\text{Дж}$.
}
\answer{%
    $U = \frac 32 \nu R T = \frac 32 PV \implies P = \frac 23 \cdot \frac UV= \frac 23 \cdot \frac{ 400\,\text{Дж} }{ 3\,\text{л} } \approx 88\,\text{кПа}.$
}
\solutionspace{40pt}

\tasknumber{3}%
\task{%
    Газ расширился от $200\,\text{л}$ до $450\,\text{л}$.
    Давление газа при этом оставалось постоянным и равным $3{,}5\,\text{атм}$.
    Определите работу газа, ответ выразите в килоджоулях.
    $p_{\text{aтм}} = 100\,\text{кПа}$.
}
\answer{%
    $A = P\Delta V = P(V_2 - V_1) = 3{,}5\,\text{атм} \cdot \cbr{450\,\text{л} - 200\,\text{л}} = 87{,}5\,\text{кДж}.$
}
\solutionspace{40pt}

\tasknumber{4}%
\task{%
    Как изменилась внутренняя энергия одноатомного идеального газа при переходе из состояния 1 в состояние 2?
    $P_1 = 3\,\text{МПа}$, $V_1 = 7\,\text{л}$, $P_2 = 4{,}5\,\text{МПа}$, $V_2 = 8\,\text{л}$.
    Как изменилась при этом температура газа?
}
\answer{%
    \begin{align*}
    P_1V_1 &= \nu R T_1, P_2V_2 = \nu R T_2, \\
    \Delta U &= U_2-U_1 = \frac 32 \nu R T_2- \frac 32 \nu R T_1 = \frac 32 P_2 V_2 - \frac 32 P_1 V_1= \frac 32 \cdot \cbr{4{,}5\,\text{МПа} \cdot 8\,\text{л} - 3\,\text{МПа} \cdot 7\,\text{л}} = 22500\,\text{Дж}.
    \\
    \frac{T_2}{T_1} &= \frac{\frac{P_2V_2}{\nu R}}{\frac{P_1V_1}{\nu R}} = \frac{P_2V_2}{P_1V_1}= \frac{4{,}5\,\text{МПа} \cdot 8\,\text{л}}{3\,\text{МПа} \cdot 7\,\text{л}} \approx 1{,}71.
    \end{align*}
}
\solutionspace{80pt}

\tasknumber{5}%
\task{%
    $3\,\text{моль}$ идеального одноатомного газа охладили на $30\,\text{К}$.
    Определите изменение внутренней энергии газа.
    Увеличилась она или уменьшилась?
    Универсальная газовая постоянная $R = 8{,}31\,\frac{\text{Дж}}{\text{моль}\cdot\text{К}}$.
}
\answer{%
    $
        \Delta U = \frac 32 \nu R \Delta T
            = - \frac 32 \cdot 3\,\text{моль} \cdot 8{,}31\,\frac{\text{Дж}}{\text{моль}\cdot\text{К}} \cdot 30\,\text{К}
            = -1121\,\text{Дж}.
            \text{Уменьшилась.}
    $
}
\solutionspace{40pt}

\tasknumber{6}%
\task{%
    Газу сообщили некоторое количество теплоты,
    при этом половину его он потратил на совершение работы,
    одновременно увеличив свою внутреннюю энергию на $3000\,\text{Дж}$.
    Определите работу, совершённую газом.
}
\answer{%
    \begin{align*}
    Q &= A' + \Delta U, A' = \frac 12 Q \implies Q \cdot \cbr{1 - \frac 12} = \Delta U \implies Q = \frac{\Delta U}{1 - \frac 12} = \frac{3000\,\text{Дж}}{1 - \frac 12} \approx 6000\,\text{Дж}.
    \\
    A' &= \frac 12 Q
        = \frac 12 \cdot \frac{\Delta U}{1 - \frac 12}
        = \frac{\Delta U}{2 - 1}
        = \frac{3000\,\text{Дж}}{2 - 1} \approx 3000\,\text{Дж}.
    \end{align*}
}
\solutionspace{60pt}

\tasknumber{7}%
\task{%
    В некотором процессе внешние силы совершили над газом работу $300\,\text{Дж}$,
    при этом его внутренняя энергия увеличилась на $150\,\text{Дж}$.
    Определите количество тепла, переданное при этом процессе газу.
    Явно пропишите, подводили газу тепло или же отводили.
}
\answer{%
    $
        Q = A_\text{газа} + \Delta U, A_\text{газа} = -A_\text{внешняя}
        \implies Q = A_\text{газа} + \Delta U = - 300\,\text{Дж} +  150\,\text{Дж} = -150\,\text{Дж}.
        \text{ Отводили.}
    $
}

\variantsplitter

\addpersonalvariant{Ярослав Лавровский}

\tasknumber{1}%
\task{%
    Укажите, верны ли утверждения («да» или «нет» слева от каждого утверждения):
    \begin{enumerate}
        \item При изобарном расширении идеальный газ совершает ровно столько работы, сколько внутренней энергии теряет.
        % \item В силу третьего закона Ньютона, совершённая газом работа и работа, совершённая над ним, всегда равны по модулю и противоположны по знаку.
        \item Работу газа в некотором процессе можно вычислять как площадь под графиком в системе координат $PT$, главное лишь правильно расположить оси.
        % \item Дважды два пять.
        \item При изотермическом процессе внутренняя энергия идеального одноатомного газа не изменяется, даже если ему подводят тепло.
        \item Газ может совершить ненулевую работу в изобарном процессе.
        % \item Адиабатический процесс лишь по воле случая не имеет приставки «изо»: в нём изменяются давление, температура и объём, но это не все макропараметры идеального газа.
        \item Полученное выражение для внутренней энергии идеального газа ($\frac 32 \nu RT$) применимо к двухоатомному газу, при этом, например, уравнение состояния идеального газа применимо независимо от числа атомов в молекулах газа.
    \end{enumerate}
}
\answer{%
    $\text{нет, нет, да, да, нет}$
}

\tasknumber{2}%
\task{%
    Определите давление одноатомного идеального газа, занимающего объём $6\,\text{л}$,
    если его внутренняя энергия составляет $500\,\text{Дж}$.
}
\answer{%
    $U = \frac 32 \nu R T = \frac 32 PV \implies P = \frac 23 \cdot \frac UV= \frac 23 \cdot \frac{ 500\,\text{Дж} }{ 6\,\text{л} } \approx 55\,\text{кПа}.$
}
\solutionspace{40pt}

\tasknumber{3}%
\task{%
    Газ расширился от $350\,\text{л}$ до $450\,\text{л}$.
    Давление газа при этом оставалось постоянным и равным $1{,}2\,\text{атм}$.
    Определите работу газа, ответ выразите в килоджоулях.
    $p_{\text{aтм}} = 100\,\text{кПа}$.
}
\answer{%
    $A = P\Delta V = P(V_2 - V_1) = 1{,}2\,\text{атм} \cdot \cbr{450\,\text{л} - 350\,\text{л}} = 12{,}0\,\text{кДж}.$
}
\solutionspace{40pt}

\tasknumber{4}%
\task{%
    Как изменилась внутренняя энергия одноатомного идеального газа при переходе из состояния 1 в состояние 2?
    $P_1 = 4\,\text{МПа}$, $V_1 = 5\,\text{л}$, $P_2 = 3{,}5\,\text{МПа}$, $V_2 = 4\,\text{л}$.
    Как изменилась при этом температура газа?
}
\answer{%
    \begin{align*}
    P_1V_1 &= \nu R T_1, P_2V_2 = \nu R T_2, \\
    \Delta U &= U_2-U_1 = \frac 32 \nu R T_2- \frac 32 \nu R T_1 = \frac 32 P_2 V_2 - \frac 32 P_1 V_1= \frac 32 \cdot \cbr{3{,}5\,\text{МПа} \cdot 4\,\text{л} - 4\,\text{МПа} \cdot 5\,\text{л}} = -9000\,\text{Дж}.
    \\
    \frac{T_2}{T_1} &= \frac{\frac{P_2V_2}{\nu R}}{\frac{P_1V_1}{\nu R}} = \frac{P_2V_2}{P_1V_1}= \frac{3{,}5\,\text{МПа} \cdot 4\,\text{л}}{4\,\text{МПа} \cdot 5\,\text{л}} \approx 0{,}70.
    \end{align*}
}
\solutionspace{80pt}

\tasknumber{5}%
\task{%
    $2\,\text{моль}$ идеального одноатомного газа охладили на $20\,\text{К}$.
    Определите изменение внутренней энергии газа.
    Увеличилась она или уменьшилась?
    Универсальная газовая постоянная $R = 8{,}31\,\frac{\text{Дж}}{\text{моль}\cdot\text{К}}$.
}
\answer{%
    $
        \Delta U = \frac 32 \nu R \Delta T
            = - \frac 32 \cdot 2\,\text{моль} \cdot 8{,}31\,\frac{\text{Дж}}{\text{моль}\cdot\text{К}} \cdot 20\,\text{К}
            = -498\,\text{Дж}.
            \text{Уменьшилась.}
    $
}
\solutionspace{40pt}

\tasknumber{6}%
\task{%
    Газу сообщили некоторое количество теплоты,
    при этом половину его он потратил на совершение работы,
    одновременно увеличив свою внутреннюю энергию на $2400\,\text{Дж}$.
    Определите количество теплоты, сообщённое газу.
}
\answer{%
    \begin{align*}
    Q &= A' + \Delta U, A' = \frac 12 Q \implies Q \cdot \cbr{1 - \frac 12} = \Delta U \implies Q = \frac{\Delta U}{1 - \frac 12} = \frac{2400\,\text{Дж}}{1 - \frac 12} \approx 4800\,\text{Дж}.
    \\
    A' &= \frac 12 Q
        = \frac 12 \cdot \frac{\Delta U}{1 - \frac 12}
        = \frac{\Delta U}{2 - 1}
        = \frac{2400\,\text{Дж}}{2 - 1} \approx 2400\,\text{Дж}.
    \end{align*}
}
\solutionspace{60pt}

\tasknumber{7}%
\task{%
    В некотором процессе внешние силы совершили над газом работу $100\,\text{Дж}$,
    при этом его внутренняя энергия увеличилась на $150\,\text{Дж}$.
    Определите количество тепла, переданное при этом процессе газу.
    Явно пропишите, подводили газу тепло или же отводили.
}
\answer{%
    $
        Q = A_\text{газа} + \Delta U, A_\text{газа} = -A_\text{внешняя}
        \implies Q = A_\text{газа} + \Delta U = - 100\,\text{Дж} +  150\,\text{Дж} = 50\,\text{Дж}.
        \text{ Подводили.}
    $
}

\variantsplitter

\addpersonalvariant{Анастасия Ламанова}

\tasknumber{1}%
\task{%
    Укажите, верны ли утверждения («да» или «нет» слева от каждого утверждения):
    \begin{enumerate}
        \item При адиабатическом расширении идеальный газ совершает ровно столько работы, сколько внутренней энергии теряет.
        % \item В силу третьего закона Ньютона, совершённая газом работа и работа, совершённая над ним, всегда равны по модулю и противоположны по знаку.
        \item Работу газа в некотором процессе можно вычислять как площадь под графиком в системе координат $VT$, главное лишь правильно расположить оси.
        % \item Дважды два три.
        \item При изотермическом процессе внутренняя энергия идеального одноатомного газа не изменяется, даже если ему подводят тепло.
        \item Газ может совершить ненулевую работу в изобарном процессе.
        % \item Адиабатический процесс лишь по воле случая не имеет приставки «изо»: в нём изменяются давление, температура и объём, но это не все макропараметры идеального газа.
        \item Полученное выражение для внутренней энергии идеального газа ($\frac 32 \nu RT$) применимо к трёхатомному газу, при этом, например, уравнение состояния идеального газа применимо независимо от числа атомов в молекулах газа.
    \end{enumerate}
}
\answer{%
    $\text{да, нет, да, да, нет}$
}

\tasknumber{2}%
\task{%
    Определите давление одноатомного идеального газа, занимающего объём $2\,\text{л}$,
    если его внутренняя энергия составляет $500\,\text{Дж}$.
}
\answer{%
    $U = \frac 32 \nu R T = \frac 32 PV \implies P = \frac 23 \cdot \frac UV= \frac 23 \cdot \frac{ 500\,\text{Дж} }{ 2\,\text{л} } \approx 166\,\text{кПа}.$
}
\solutionspace{40pt}

\tasknumber{3}%
\task{%
    Газ расширился от $350\,\text{л}$ до $650\,\text{л}$.
    Давление газа при этом оставалось постоянным и равным $1{,}2\,\text{атм}$.
    Определите работу газа, ответ выразите в килоджоулях.
    $p_{\text{aтм}} = 100\,\text{кПа}$.
}
\answer{%
    $A = P\Delta V = P(V_2 - V_1) = 1{,}2\,\text{атм} \cdot \cbr{650\,\text{л} - 350\,\text{л}} = 36{,}0\,\text{кДж}.$
}
\solutionspace{40pt}

\tasknumber{4}%
\task{%
    Как изменилась внутренняя энергия одноатомного идеального газа при переходе из состояния 1 в состояние 2?
    $P_1 = 2\,\text{МПа}$, $V_1 = 3\,\text{л}$, $P_2 = 4{,}5\,\text{МПа}$, $V_2 = 8\,\text{л}$.
    Как изменилась при этом температура газа?
}
\answer{%
    \begin{align*}
    P_1V_1 &= \nu R T_1, P_2V_2 = \nu R T_2, \\
    \Delta U &= U_2-U_1 = \frac 32 \nu R T_2- \frac 32 \nu R T_1 = \frac 32 P_2 V_2 - \frac 32 P_1 V_1= \frac 32 \cdot \cbr{4{,}5\,\text{МПа} \cdot 8\,\text{л} - 2\,\text{МПа} \cdot 3\,\text{л}} = 45000\,\text{Дж}.
    \\
    \frac{T_2}{T_1} &= \frac{\frac{P_2V_2}{\nu R}}{\frac{P_1V_1}{\nu R}} = \frac{P_2V_2}{P_1V_1}= \frac{4{,}5\,\text{МПа} \cdot 8\,\text{л}}{2\,\text{МПа} \cdot 3\,\text{л}} \approx 6{,}00.
    \end{align*}
}
\solutionspace{80pt}

\tasknumber{5}%
\task{%
    $2\,\text{моль}$ идеального одноатомного газа нагрели на $30\,\text{К}$.
    Определите изменение внутренней энергии газа.
    Увеличилась она или уменьшилась?
    Универсальная газовая постоянная $R = 8{,}31\,\frac{\text{Дж}}{\text{моль}\cdot\text{К}}$.
}
\answer{%
    $
        \Delta U = \frac 32 \nu R \Delta T
            =  \frac 32 \cdot 2\,\text{моль} \cdot 8{,}31\,\frac{\text{Дж}}{\text{моль}\cdot\text{К}} \cdot 30\,\text{К}
            = 747\,\text{Дж}.
            \text{Увеличилась.}
    $
}
\solutionspace{40pt}

\tasknumber{6}%
\task{%
    Газу сообщили некоторое количество теплоты,
    при этом четверть его он потратил на совершение работы,
    одновременно увеличив свою внутреннюю энергию на $1500\,\text{Дж}$.
    Определите количество теплоты, сообщённое газу.
}
\answer{%
    \begin{align*}
    Q &= A' + \Delta U, A' = \frac 14 Q \implies Q \cdot \cbr{1 - \frac 14} = \Delta U \implies Q = \frac{\Delta U}{1 - \frac 14} = \frac{1500\,\text{Дж}}{1 - \frac 14} \approx 2000\,\text{Дж}.
    \\
    A' &= \frac 14 Q
        = \frac 14 \cdot \frac{\Delta U}{1 - \frac 14}
        = \frac{\Delta U}{4 - 1}
        = \frac{1500\,\text{Дж}}{4 - 1} \approx 500\,\text{Дж}.
    \end{align*}
}
\solutionspace{60pt}

\tasknumber{7}%
\task{%
    В некотором процессе внешние силы совершили над газом работу $300\,\text{Дж}$,
    при этом его внутренняя энергия увеличилась на $350\,\text{Дж}$.
    Определите количество тепла, переданное при этом процессе газу.
    Явно пропишите, подводили газу тепло или же отводили.
}
\answer{%
    $
        Q = A_\text{газа} + \Delta U, A_\text{газа} = -A_\text{внешняя}
        \implies Q = A_\text{газа} + \Delta U = - 300\,\text{Дж} +  350\,\text{Дж} = 50\,\text{Дж}.
        \text{ Подводили.}
    $
}

\variantsplitter

\addpersonalvariant{Виктория Легонькова}

\tasknumber{1}%
\task{%
    Укажите, верны ли утверждения («да» или «нет» слева от каждого утверждения):
    \begin{enumerate}
        \item При адиабатическом расширении идеальный газ совершает ровно столько работы, сколько внутренней энергии теряет.
        % \item В силу третьего закона Ньютона, совершённая газом работа и работа, совершённая над ним, всегда равны по модулю и противоположны по знаку.
        \item Работу газа в некотором процессе можно вычислять как площадь под графиком в системе координат $VT$, главное лишь правильно расположить оси.
        % \item Дважды два пять.
        \item При изобарном процессе внутренняя энергия идеального одноатомного газа не изменяется, даже если ему подводят тепло.
        \item Газ может совершить ненулевую работу в изотермическом процессе.
        % \item Адиабатический процесс лишь по воле случая не имеет приставки «изо»: в нём изменяются давление, температура и объём, но это не все макропараметры идеального газа.
        \item Полученное выражение для внутренней энергии идеального газа ($\frac 32 \nu RT$) применимо к одноатомному газу, при этом, например, уравнение состояния идеального газа применимо независимо от числа атомов в молекулах газа.
    \end{enumerate}
}
\answer{%
    $\text{да, нет, нет, да, да}$
}

\tasknumber{2}%
\task{%
    Определите давление одноатомного идеального газа, занимающего объём $3\,\text{л}$,
    если его внутренняя энергия составляет $300\,\text{Дж}$.
}
\answer{%
    $U = \frac 32 \nu R T = \frac 32 PV \implies P = \frac 23 \cdot \frac UV= \frac 23 \cdot \frac{ 300\,\text{Дж} }{ 3\,\text{л} } \approx 66\,\text{кПа}.$
}
\solutionspace{40pt}

\tasknumber{3}%
\task{%
    Газ расширился от $250\,\text{л}$ до $550\,\text{л}$.
    Давление газа при этом оставалось постоянным и равным $3{,}5\,\text{атм}$.
    Определите работу газа, ответ выразите в килоджоулях.
    $p_{\text{aтм}} = 100\,\text{кПа}$.
}
\answer{%
    $A = P\Delta V = P(V_2 - V_1) = 3{,}5\,\text{атм} \cdot \cbr{550\,\text{л} - 250\,\text{л}} = 105{,}0\,\text{кДж}.$
}
\solutionspace{40pt}

\tasknumber{4}%
\task{%
    Как изменилась внутренняя энергия одноатомного идеального газа при переходе из состояния 1 в состояние 2?
    $P_1 = 4\,\text{МПа}$, $V_1 = 7\,\text{л}$, $P_2 = 4{,}5\,\text{МПа}$, $V_2 = 4\,\text{л}$.
    Как изменилась при этом температура газа?
}
\answer{%
    \begin{align*}
    P_1V_1 &= \nu R T_1, P_2V_2 = \nu R T_2, \\
    \Delta U &= U_2-U_1 = \frac 32 \nu R T_2- \frac 32 \nu R T_1 = \frac 32 P_2 V_2 - \frac 32 P_1 V_1= \frac 32 \cdot \cbr{4{,}5\,\text{МПа} \cdot 4\,\text{л} - 4\,\text{МПа} \cdot 7\,\text{л}} = -15000\,\text{Дж}.
    \\
    \frac{T_2}{T_1} &= \frac{\frac{P_2V_2}{\nu R}}{\frac{P_1V_1}{\nu R}} = \frac{P_2V_2}{P_1V_1}= \frac{4{,}5\,\text{МПа} \cdot 4\,\text{л}}{4\,\text{МПа} \cdot 7\,\text{л}} \approx 0{,}64.
    \end{align*}
}
\solutionspace{80pt}

\tasknumber{5}%
\task{%
    $3\,\text{моль}$ идеального одноатомного газа охладили на $10\,\text{К}$.
    Определите изменение внутренней энергии газа.
    Увеличилась она или уменьшилась?
    Универсальная газовая постоянная $R = 8{,}31\,\frac{\text{Дж}}{\text{моль}\cdot\text{К}}$.
}
\answer{%
    $
        \Delta U = \frac 32 \nu R \Delta T
            = - \frac 32 \cdot 3\,\text{моль} \cdot 8{,}31\,\frac{\text{Дж}}{\text{моль}\cdot\text{К}} \cdot 10\,\text{К}
            = -373\,\text{Дж}.
            \text{Уменьшилась.}
    $
}
\solutionspace{40pt}

\tasknumber{6}%
\task{%
    Газу сообщили некоторое количество теплоты,
    при этом половину его он потратил на совершение работы,
    одновременно увеличив свою внутреннюю энергию на $3000\,\text{Дж}$.
    Определите количество теплоты, сообщённое газу.
}
\answer{%
    \begin{align*}
    Q &= A' + \Delta U, A' = \frac 12 Q \implies Q \cdot \cbr{1 - \frac 12} = \Delta U \implies Q = \frac{\Delta U}{1 - \frac 12} = \frac{3000\,\text{Дж}}{1 - \frac 12} \approx 6000\,\text{Дж}.
    \\
    A' &= \frac 12 Q
        = \frac 12 \cdot \frac{\Delta U}{1 - \frac 12}
        = \frac{\Delta U}{2 - 1}
        = \frac{3000\,\text{Дж}}{2 - 1} \approx 3000\,\text{Дж}.
    \end{align*}
}
\solutionspace{60pt}

\tasknumber{7}%
\task{%
    В некотором процессе газ совершил работу $300\,\text{Дж}$,
    при этом его внутренняя энергия уменьшилась на $250\,\text{Дж}$.
    Определите количество тепла, переданное при этом процессе газу.
    Явно пропишите, подводили газу тепло или же отводили.
}
\answer{%
    $
        Q = A_\text{газа} + \Delta U, A_\text{газа} = -A_\text{внешняя}
        \implies Q = A_\text{газа} + \Delta U =  300\,\text{Дж} - 250\,\text{Дж} = 50\,\text{Дж}.
        \text{ Подводили.}
    $
}

\variantsplitter

\addpersonalvariant{Семён Мартынов}

\tasknumber{1}%
\task{%
    Укажите, верны ли утверждения («да» или «нет» слева от каждого утверждения):
    \begin{enumerate}
        \item При изобарном расширении идеальный газ совершает ровно столько работы, сколько внутренней энергии теряет.
        % \item В силу третьего закона Ньютона, совершённая газом работа и работа, совершённая над ним, всегда равны по модулю и противоположны по знаку.
        \item Работу газа в некотором процессе можно вычислять как площадь под графиком в системе координат $PV$, главное лишь правильно расположить оси.
        % \item Дважды два пять.
        \item При изотермическом процессе внутренняя энергия идеального одноатомного газа не изменяется, даже если ему подводят тепло.
        \item Газ может совершить ненулевую работу в изохорном процессе.
        % \item Адиабатический процесс лишь по воле случая не имеет приставки «изо»: в нём изменяются давление, температура и объём, но это не все макропараметры идеального газа.
        \item Полученное выражение для внутренней энергии идеального газа ($\frac 32 \nu RT$) применимо к трёхатомному газу, при этом, например, уравнение состояния идеального газа применимо независимо от числа атомов в молекулах газа.
    \end{enumerate}
}
\answer{%
    $\text{нет, да, да, нет, нет}$
}

\tasknumber{2}%
\task{%
    Определите давление одноатомного идеального газа, занимающего объём $4\,\text{л}$,
    если его внутренняя энергия составляет $250\,\text{Дж}$.
}
\answer{%
    $U = \frac 32 \nu R T = \frac 32 PV \implies P = \frac 23 \cdot \frac UV= \frac 23 \cdot \frac{ 250\,\text{Дж} }{ 4\,\text{л} } \approx 41\,\text{кПа}.$
}
\solutionspace{40pt}

\tasknumber{3}%
\task{%
    Газ расширился от $350\,\text{л}$ до $550\,\text{л}$.
    Давление газа при этом оставалось постоянным и равным $3{,}5\,\text{атм}$.
    Определите работу газа, ответ выразите в килоджоулях.
    $p_{\text{aтм}} = 100\,\text{кПа}$.
}
\answer{%
    $A = P\Delta V = P(V_2 - V_1) = 3{,}5\,\text{атм} \cdot \cbr{550\,\text{л} - 350\,\text{л}} = 70{,}0\,\text{кДж}.$
}
\solutionspace{40pt}

\tasknumber{4}%
\task{%
    Как изменилась внутренняя энергия одноатомного идеального газа при переходе из состояния 1 в состояние 2?
    $P_1 = 4\,\text{МПа}$, $V_1 = 7\,\text{л}$, $P_2 = 3{,}5\,\text{МПа}$, $V_2 = 8\,\text{л}$.
    Как изменилась при этом температура газа?
}
\answer{%
    \begin{align*}
    P_1V_1 &= \nu R T_1, P_2V_2 = \nu R T_2, \\
    \Delta U &= U_2-U_1 = \frac 32 \nu R T_2- \frac 32 \nu R T_1 = \frac 32 P_2 V_2 - \frac 32 P_1 V_1= \frac 32 \cdot \cbr{3{,}5\,\text{МПа} \cdot 8\,\text{л} - 4\,\text{МПа} \cdot 7\,\text{л}} = 0\,\text{Дж}.
    \\
    \frac{T_2}{T_1} &= \frac{\frac{P_2V_2}{\nu R}}{\frac{P_1V_1}{\nu R}} = \frac{P_2V_2}{P_1V_1}= \frac{3{,}5\,\text{МПа} \cdot 8\,\text{л}}{4\,\text{МПа} \cdot 7\,\text{л}} \approx 1{,}00.
    \end{align*}
}
\solutionspace{80pt}

\tasknumber{5}%
\task{%
    $4\,\text{моль}$ идеального одноатомного газа нагрели на $20\,\text{К}$.
    Определите изменение внутренней энергии газа.
    Увеличилась она или уменьшилась?
    Универсальная газовая постоянная $R = 8{,}31\,\frac{\text{Дж}}{\text{моль}\cdot\text{К}}$.
}
\answer{%
    $
        \Delta U = \frac 32 \nu R \Delta T
            =  \frac 32 \cdot 4\,\text{моль} \cdot 8{,}31\,\frac{\text{Дж}}{\text{моль}\cdot\text{К}} \cdot 20\,\text{К}
            = 997\,\text{Дж}.
            \text{Увеличилась.}
    $
}
\solutionspace{40pt}

\tasknumber{6}%
\task{%
    Газу сообщили некоторое количество теплоты,
    при этом половину его он потратил на совершение работы,
    одновременно увеличив свою внутреннюю энергию на $1500\,\text{Дж}$.
    Определите количество теплоты, сообщённое газу.
}
\answer{%
    \begin{align*}
    Q &= A' + \Delta U, A' = \frac 12 Q \implies Q \cdot \cbr{1 - \frac 12} = \Delta U \implies Q = \frac{\Delta U}{1 - \frac 12} = \frac{1500\,\text{Дж}}{1 - \frac 12} \approx 3000\,\text{Дж}.
    \\
    A' &= \frac 12 Q
        = \frac 12 \cdot \frac{\Delta U}{1 - \frac 12}
        = \frac{\Delta U}{2 - 1}
        = \frac{1500\,\text{Дж}}{2 - 1} \approx 1500\,\text{Дж}.
    \end{align*}
}
\solutionspace{60pt}

\tasknumber{7}%
\task{%
    В некотором процессе внешние силы совершили над газом работу $100\,\text{Дж}$,
    при этом его внутренняя энергия уменьшилась на $450\,\text{Дж}$.
    Определите количество тепла, переданное при этом процессе газу.
    Явно пропишите, подводили газу тепло или же отводили.
}
\answer{%
    $
        Q = A_\text{газа} + \Delta U, A_\text{газа} = -A_\text{внешняя}
        \implies Q = A_\text{газа} + \Delta U = - 100\,\text{Дж} - 450\,\text{Дж} = -550\,\text{Дж}.
        \text{ Отводили.}
    $
}

\variantsplitter

\addpersonalvariant{Варвара Минаева}

\tasknumber{1}%
\task{%
    Укажите, верны ли утверждения («да» или «нет» слева от каждого утверждения):
    \begin{enumerate}
        \item При изобарном расширении идеальный газ совершает ровно столько работы, сколько внутренней энергии теряет.
        % \item В силу третьего закона Ньютона, совершённая газом работа и работа, совершённая над ним, всегда равны по модулю и противоположны по знаку.
        \item Работу газа в некотором процессе можно вычислять как площадь под графиком в системе координат $PT$, главное лишь правильно расположить оси.
        % \item Дважды два пять.
        \item При изохорном процессе внутренняя энергия идеального одноатомного газа не изменяется, даже если ему подводят тепло.
        \item Газ может совершить ненулевую работу в изохорном процессе.
        % \item Адиабатический процесс лишь по воле случая не имеет приставки «изо»: в нём изменяются давление, температура и объём, но это не все макропараметры идеального газа.
        \item Полученное выражение для внутренней энергии идеального газа ($\frac 32 \nu RT$) применимо к трёхатомному газу, при этом, например, уравнение состояния идеального газа применимо независимо от числа атомов в молекулах газа.
    \end{enumerate}
}
\answer{%
    $\text{нет, нет, нет, нет, нет}$
}

\tasknumber{2}%
\task{%
    Определите давление одноатомного идеального газа, занимающего объём $5\,\text{л}$,
    если его внутренняя энергия составляет $400\,\text{Дж}$.
}
\answer{%
    $U = \frac 32 \nu R T = \frac 32 PV \implies P = \frac 23 \cdot \frac UV= \frac 23 \cdot \frac{ 400\,\text{Дж} }{ 5\,\text{л} } \approx 53\,\text{кПа}.$
}
\solutionspace{40pt}

\tasknumber{3}%
\task{%
    Газ расширился от $350\,\text{л}$ до $550\,\text{л}$.
    Давление газа при этом оставалось постоянным и равным $1{,}5\,\text{атм}$.
    Определите работу газа, ответ выразите в килоджоулях.
    $p_{\text{aтм}} = 100\,\text{кПа}$.
}
\answer{%
    $A = P\Delta V = P(V_2 - V_1) = 1{,}5\,\text{атм} \cdot \cbr{550\,\text{л} - 350\,\text{л}} = 30{,}0\,\text{кДж}.$
}
\solutionspace{40pt}

\tasknumber{4}%
\task{%
    Как изменилась внутренняя энергия одноатомного идеального газа при переходе из состояния 1 в состояние 2?
    $P_1 = 4\,\text{МПа}$, $V_1 = 7\,\text{л}$, $P_2 = 3{,}5\,\text{МПа}$, $V_2 = 6\,\text{л}$.
    Как изменилась при этом температура газа?
}
\answer{%
    \begin{align*}
    P_1V_1 &= \nu R T_1, P_2V_2 = \nu R T_2, \\
    \Delta U &= U_2-U_1 = \frac 32 \nu R T_2- \frac 32 \nu R T_1 = \frac 32 P_2 V_2 - \frac 32 P_1 V_1= \frac 32 \cdot \cbr{3{,}5\,\text{МПа} \cdot 6\,\text{л} - 4\,\text{МПа} \cdot 7\,\text{л}} = -10500\,\text{Дж}.
    \\
    \frac{T_2}{T_1} &= \frac{\frac{P_2V_2}{\nu R}}{\frac{P_1V_1}{\nu R}} = \frac{P_2V_2}{P_1V_1}= \frac{3{,}5\,\text{МПа} \cdot 6\,\text{л}}{4\,\text{МПа} \cdot 7\,\text{л}} \approx 0{,}75.
    \end{align*}
}
\solutionspace{80pt}

\tasknumber{5}%
\task{%
    $3\,\text{моль}$ идеального одноатомного газа охладили на $30\,\text{К}$.
    Определите изменение внутренней энергии газа.
    Увеличилась она или уменьшилась?
    Универсальная газовая постоянная $R = 8{,}31\,\frac{\text{Дж}}{\text{моль}\cdot\text{К}}$.
}
\answer{%
    $
        \Delta U = \frac 32 \nu R \Delta T
            = - \frac 32 \cdot 3\,\text{моль} \cdot 8{,}31\,\frac{\text{Дж}}{\text{моль}\cdot\text{К}} \cdot 30\,\text{К}
            = -1121\,\text{Дж}.
            \text{Уменьшилась.}
    $
}
\solutionspace{40pt}

\tasknumber{6}%
\task{%
    Газу сообщили некоторое количество теплоты,
    при этом треть его он потратил на совершение работы,
    одновременно увеличив свою внутреннюю энергию на $1500\,\text{Дж}$.
    Определите количество теплоты, сообщённое газу.
}
\answer{%
    \begin{align*}
    Q &= A' + \Delta U, A' = \frac 13 Q \implies Q \cdot \cbr{1 - \frac 13} = \Delta U \implies Q = \frac{\Delta U}{1 - \frac 13} = \frac{1500\,\text{Дж}}{1 - \frac 13} \approx 2250\,\text{Дж}.
    \\
    A' &= \frac 13 Q
        = \frac 13 \cdot \frac{\Delta U}{1 - \frac 13}
        = \frac{\Delta U}{3 - 1}
        = \frac{1500\,\text{Дж}}{3 - 1} \approx 750\,\text{Дж}.
    \end{align*}
}
\solutionspace{60pt}

\tasknumber{7}%
\task{%
    В некотором процессе внешние силы совершили над газом работу $300\,\text{Дж}$,
    при этом его внутренняя энергия уменьшилась на $150\,\text{Дж}$.
    Определите количество тепла, переданное при этом процессе газу.
    Явно пропишите, подводили газу тепло или же отводили.
}
\answer{%
    $
        Q = A_\text{газа} + \Delta U, A_\text{газа} = -A_\text{внешняя}
        \implies Q = A_\text{газа} + \Delta U = - 300\,\text{Дж} - 150\,\text{Дж} = -450\,\text{Дж}.
        \text{ Отводили.}
    $
}

\variantsplitter

\addpersonalvariant{Леонид Никитин}

\tasknumber{1}%
\task{%
    Укажите, верны ли утверждения («да» или «нет» слева от каждого утверждения):
    \begin{enumerate}
        \item При изобарном расширении идеальный газ совершает ровно столько работы, сколько внутренней энергии теряет.
        % \item В силу третьего закона Ньютона, совершённая газом работа и работа, совершённая над ним, всегда равны по модулю и противоположны по знаку.
        \item Работу газа в некотором процессе можно вычислять как площадь под графиком в системе координат $PT$, главное лишь правильно расположить оси.
        % \item Дважды два пять.
        \item При изохорном процессе внутренняя энергия идеального одноатомного газа не изменяется, даже если ему подводят тепло.
        \item Газ может совершить ненулевую работу в изохорном процессе.
        % \item Адиабатический процесс лишь по воле случая не имеет приставки «изо»: в нём изменяются давление, температура и объём, но это не все макропараметры идеального газа.
        \item Полученное выражение для внутренней энергии идеального газа ($\frac 32 \nu RT$) применимо к трёхатомному газу, при этом, например, уравнение состояния идеального газа применимо независимо от числа атомов в молекулах газа.
    \end{enumerate}
}
\answer{%
    $\text{нет, нет, нет, нет, нет}$
}

\tasknumber{2}%
\task{%
    Определите давление одноатомного идеального газа, занимающего объём $2\,\text{л}$,
    если его внутренняя энергия составляет $300\,\text{Дж}$.
}
\answer{%
    $U = \frac 32 \nu R T = \frac 32 PV \implies P = \frac 23 \cdot \frac UV= \frac 23 \cdot \frac{ 300\,\text{Дж} }{ 2\,\text{л} } \approx 100\,\text{кПа}.$
}
\solutionspace{40pt}

\tasknumber{3}%
\task{%
    Газ расширился от $350\,\text{л}$ до $450\,\text{л}$.
    Давление газа при этом оставалось постоянным и равным $1{,}8\,\text{атм}$.
    Определите работу газа, ответ выразите в килоджоулях.
    $p_{\text{aтм}} = 100\,\text{кПа}$.
}
\answer{%
    $A = P\Delta V = P(V_2 - V_1) = 1{,}8\,\text{атм} \cdot \cbr{450\,\text{л} - 350\,\text{л}} = 18{,}0\,\text{кДж}.$
}
\solutionspace{40pt}

\tasknumber{4}%
\task{%
    Как изменилась внутренняя энергия одноатомного идеального газа при переходе из состояния 1 в состояние 2?
    $P_1 = 3\,\text{МПа}$, $V_1 = 5\,\text{л}$, $P_2 = 3{,}5\,\text{МПа}$, $V_2 = 4\,\text{л}$.
    Как изменилась при этом температура газа?
}
\answer{%
    \begin{align*}
    P_1V_1 &= \nu R T_1, P_2V_2 = \nu R T_2, \\
    \Delta U &= U_2-U_1 = \frac 32 \nu R T_2- \frac 32 \nu R T_1 = \frac 32 P_2 V_2 - \frac 32 P_1 V_1= \frac 32 \cdot \cbr{3{,}5\,\text{МПа} \cdot 4\,\text{л} - 3\,\text{МПа} \cdot 5\,\text{л}} = -1500\,\text{Дж}.
    \\
    \frac{T_2}{T_1} &= \frac{\frac{P_2V_2}{\nu R}}{\frac{P_1V_1}{\nu R}} = \frac{P_2V_2}{P_1V_1}= \frac{3{,}5\,\text{МПа} \cdot 4\,\text{л}}{3\,\text{МПа} \cdot 5\,\text{л}} \approx 0{,}93.
    \end{align*}
}
\solutionspace{80pt}

\tasknumber{5}%
\task{%
    $2\,\text{моль}$ идеального одноатомного газа нагрели на $30\,\text{К}$.
    Определите изменение внутренней энергии газа.
    Увеличилась она или уменьшилась?
    Универсальная газовая постоянная $R = 8{,}31\,\frac{\text{Дж}}{\text{моль}\cdot\text{К}}$.
}
\answer{%
    $
        \Delta U = \frac 32 \nu R \Delta T
            =  \frac 32 \cdot 2\,\text{моль} \cdot 8{,}31\,\frac{\text{Дж}}{\text{моль}\cdot\text{К}} \cdot 30\,\text{К}
            = 747\,\text{Дж}.
            \text{Увеличилась.}
    $
}
\solutionspace{40pt}

\tasknumber{6}%
\task{%
    Газу сообщили некоторое количество теплоты,
    при этом четверть его он потратил на совершение работы,
    одновременно увеличив свою внутреннюю энергию на $1500\,\text{Дж}$.
    Определите количество теплоты, сообщённое газу.
}
\answer{%
    \begin{align*}
    Q &= A' + \Delta U, A' = \frac 14 Q \implies Q \cdot \cbr{1 - \frac 14} = \Delta U \implies Q = \frac{\Delta U}{1 - \frac 14} = \frac{1500\,\text{Дж}}{1 - \frac 14} \approx 2000\,\text{Дж}.
    \\
    A' &= \frac 14 Q
        = \frac 14 \cdot \frac{\Delta U}{1 - \frac 14}
        = \frac{\Delta U}{4 - 1}
        = \frac{1500\,\text{Дж}}{4 - 1} \approx 500\,\text{Дж}.
    \end{align*}
}
\solutionspace{60pt}

\tasknumber{7}%
\task{%
    В некотором процессе газ совершил работу $100\,\text{Дж}$,
    при этом его внутренняя энергия увеличилась на $150\,\text{Дж}$.
    Определите количество тепла, переданное при этом процессе газу.
    Явно пропишите, подводили газу тепло или же отводили.
}
\answer{%
    $
        Q = A_\text{газа} + \Delta U, A_\text{газа} = -A_\text{внешняя}
        \implies Q = A_\text{газа} + \Delta U =  100\,\text{Дж} +  150\,\text{Дж} = 250\,\text{Дж}.
        \text{ Подводили.}
    $
}

\variantsplitter

\addpersonalvariant{Тимофей Полетаев}

\tasknumber{1}%
\task{%
    Укажите, верны ли утверждения («да» или «нет» слева от каждого утверждения):
    \begin{enumerate}
        \item При изобарном расширении идеальный газ совершает ровно столько работы, сколько внутренней энергии теряет.
        % \item В силу третьего закона Ньютона, совершённая газом работа и работа, совершённая над ним, всегда равны по модулю и противоположны по знаку.
        \item Работу газа в некотором процессе можно вычислять как площадь под графиком в системе координат $VT$, главное лишь правильно расположить оси.
        % \item Дважды два три.
        \item При изотермическом процессе внутренняя энергия идеального одноатомного газа не изменяется, даже если ему подводят тепло.
        \item Газ может совершить ненулевую работу в изотермическом процессе.
        % \item Адиабатический процесс лишь по воле случая не имеет приставки «изо»: в нём изменяются давление, температура и объём, но это не все макропараметры идеального газа.
        \item Полученное выражение для внутренней энергии идеального газа ($\frac 32 \nu RT$) применимо к трёхатомному газу, при этом, например, уравнение состояния идеального газа применимо независимо от числа атомов в молекулах газа.
    \end{enumerate}
}
\answer{%
    $\text{нет, нет, да, да, нет}$
}

\tasknumber{2}%
\task{%
    Определите давление одноатомного идеального газа, занимающего объём $2\,\text{л}$,
    если его внутренняя энергия составляет $250\,\text{Дж}$.
}
\answer{%
    $U = \frac 32 \nu R T = \frac 32 PV \implies P = \frac 23 \cdot \frac UV= \frac 23 \cdot \frac{ 250\,\text{Дж} }{ 2\,\text{л} } \approx 83\,\text{кПа}.$
}
\solutionspace{40pt}

\tasknumber{3}%
\task{%
    Газ расширился от $150\,\text{л}$ до $450\,\text{л}$.
    Давление газа при этом оставалось постоянным и равным $2{,}5\,\text{атм}$.
    Определите работу газа, ответ выразите в килоджоулях.
    $p_{\text{aтм}} = 100\,\text{кПа}$.
}
\answer{%
    $A = P\Delta V = P(V_2 - V_1) = 2{,}5\,\text{атм} \cdot \cbr{450\,\text{л} - 150\,\text{л}} = 75{,}0\,\text{кДж}.$
}
\solutionspace{40pt}

\tasknumber{4}%
\task{%
    Как изменилась внутренняя энергия одноатомного идеального газа при переходе из состояния 1 в состояние 2?
    $P_1 = 2\,\text{МПа}$, $V_1 = 5\,\text{л}$, $P_2 = 1{,}5\,\text{МПа}$, $V_2 = 8\,\text{л}$.
    Как изменилась при этом температура газа?
}
\answer{%
    \begin{align*}
    P_1V_1 &= \nu R T_1, P_2V_2 = \nu R T_2, \\
    \Delta U &= U_2-U_1 = \frac 32 \nu R T_2- \frac 32 \nu R T_1 = \frac 32 P_2 V_2 - \frac 32 P_1 V_1= \frac 32 \cdot \cbr{1{,}5\,\text{МПа} \cdot 8\,\text{л} - 2\,\text{МПа} \cdot 5\,\text{л}} = 3000\,\text{Дж}.
    \\
    \frac{T_2}{T_1} &= \frac{\frac{P_2V_2}{\nu R}}{\frac{P_1V_1}{\nu R}} = \frac{P_2V_2}{P_1V_1}= \frac{1{,}5\,\text{МПа} \cdot 8\,\text{л}}{2\,\text{МПа} \cdot 5\,\text{л}} \approx 1{,}20.
    \end{align*}
}
\solutionspace{80pt}

\tasknumber{5}%
\task{%
    $4\,\text{моль}$ идеального одноатомного газа охладили на $20\,\text{К}$.
    Определите изменение внутренней энергии газа.
    Увеличилась она или уменьшилась?
    Универсальная газовая постоянная $R = 8{,}31\,\frac{\text{Дж}}{\text{моль}\cdot\text{К}}$.
}
\answer{%
    $
        \Delta U = \frac 32 \nu R \Delta T
            = - \frac 32 \cdot 4\,\text{моль} \cdot 8{,}31\,\frac{\text{Дж}}{\text{моль}\cdot\text{К}} \cdot 20\,\text{К}
            = -997\,\text{Дж}.
            \text{Уменьшилась.}
    $
}
\solutionspace{40pt}

\tasknumber{6}%
\task{%
    Газу сообщили некоторое количество теплоты,
    при этом половину его он потратил на совершение работы,
    одновременно увеличив свою внутреннюю энергию на $1500\,\text{Дж}$.
    Определите количество теплоты, сообщённое газу.
}
\answer{%
    \begin{align*}
    Q &= A' + \Delta U, A' = \frac 12 Q \implies Q \cdot \cbr{1 - \frac 12} = \Delta U \implies Q = \frac{\Delta U}{1 - \frac 12} = \frac{1500\,\text{Дж}}{1 - \frac 12} \approx 3000\,\text{Дж}.
    \\
    A' &= \frac 12 Q
        = \frac 12 \cdot \frac{\Delta U}{1 - \frac 12}
        = \frac{\Delta U}{2 - 1}
        = \frac{1500\,\text{Дж}}{2 - 1} \approx 1500\,\text{Дж}.
    \end{align*}
}
\solutionspace{60pt}

\tasknumber{7}%
\task{%
    В некотором процессе внешние силы совершили над газом работу $200\,\text{Дж}$,
    при этом его внутренняя энергия уменьшилась на $150\,\text{Дж}$.
    Определите количество тепла, переданное при этом процессе газу.
    Явно пропишите, подводили газу тепло или же отводили.
}
\answer{%
    $
        Q = A_\text{газа} + \Delta U, A_\text{газа} = -A_\text{внешняя}
        \implies Q = A_\text{газа} + \Delta U = - 200\,\text{Дж} - 150\,\text{Дж} = -350\,\text{Дж}.
        \text{ Отводили.}
    $
}

\variantsplitter

\addpersonalvariant{Андрей Рожков}

\tasknumber{1}%
\task{%
    Укажите, верны ли утверждения («да» или «нет» слева от каждого утверждения):
    \begin{enumerate}
        \item При адиабатическом расширении идеальный газ совершает ровно столько работы, сколько внутренней энергии теряет.
        % \item В силу третьего закона Ньютона, совершённая газом работа и работа, совершённая над ним, всегда равны по модулю и противоположны по знаку.
        \item Работу газа в некотором процессе можно вычислять как площадь под графиком в системе координат $PV$, главное лишь правильно расположить оси.
        % \item Дважды два три.
        \item При изобарном процессе внутренняя энергия идеального одноатомного газа не изменяется, даже если ему подводят тепло.
        \item Газ может совершить ненулевую работу в изохорном процессе.
        % \item Адиабатический процесс лишь по воле случая не имеет приставки «изо»: в нём изменяются давление, температура и объём, но это не все макропараметры идеального газа.
        \item Полученное выражение для внутренней энергии идеального газа ($\frac 32 \nu RT$) применимо к двухоатомному газу, при этом, например, уравнение состояния идеального газа применимо независимо от числа атомов в молекулах газа.
    \end{enumerate}
}
\answer{%
    $\text{да, да, нет, нет, нет}$
}

\tasknumber{2}%
\task{%
    Определите давление одноатомного идеального газа, занимающего объём $3\,\text{л}$,
    если его внутренняя энергия составляет $500\,\text{Дж}$.
}
\answer{%
    $U = \frac 32 \nu R T = \frac 32 PV \implies P = \frac 23 \cdot \frac UV= \frac 23 \cdot \frac{ 500\,\text{Дж} }{ 3\,\text{л} } \approx 111\,\text{кПа}.$
}
\solutionspace{40pt}

\tasknumber{3}%
\task{%
    Газ расширился от $250\,\text{л}$ до $650\,\text{л}$.
    Давление газа при этом оставалось постоянным и равным $3{,}5\,\text{атм}$.
    Определите работу газа, ответ выразите в килоджоулях.
    $p_{\text{aтм}} = 100\,\text{кПа}$.
}
\answer{%
    $A = P\Delta V = P(V_2 - V_1) = 3{,}5\,\text{атм} \cdot \cbr{650\,\text{л} - 250\,\text{л}} = 140{,}0\,\text{кДж}.$
}
\solutionspace{40pt}

\tasknumber{4}%
\task{%
    Как изменилась внутренняя энергия одноатомного идеального газа при переходе из состояния 1 в состояние 2?
    $P_1 = 3\,\text{МПа}$, $V_1 = 3\,\text{л}$, $P_2 = 3{,}5\,\text{МПа}$, $V_2 = 4\,\text{л}$.
    Как изменилась при этом температура газа?
}
\answer{%
    \begin{align*}
    P_1V_1 &= \nu R T_1, P_2V_2 = \nu R T_2, \\
    \Delta U &= U_2-U_1 = \frac 32 \nu R T_2- \frac 32 \nu R T_1 = \frac 32 P_2 V_2 - \frac 32 P_1 V_1= \frac 32 \cdot \cbr{3{,}5\,\text{МПа} \cdot 4\,\text{л} - 3\,\text{МПа} \cdot 3\,\text{л}} = 7500\,\text{Дж}.
    \\
    \frac{T_2}{T_1} &= \frac{\frac{P_2V_2}{\nu R}}{\frac{P_1V_1}{\nu R}} = \frac{P_2V_2}{P_1V_1}= \frac{3{,}5\,\text{МПа} \cdot 4\,\text{л}}{3\,\text{МПа} \cdot 3\,\text{л}} \approx 1{,}56.
    \end{align*}
}
\solutionspace{80pt}

\tasknumber{5}%
\task{%
    $2\,\text{моль}$ идеального одноатомного газа нагрели на $20\,\text{К}$.
    Определите изменение внутренней энергии газа.
    Увеличилась она или уменьшилась?
    Универсальная газовая постоянная $R = 8{,}31\,\frac{\text{Дж}}{\text{моль}\cdot\text{К}}$.
}
\answer{%
    $
        \Delta U = \frac 32 \nu R \Delta T
            =  \frac 32 \cdot 2\,\text{моль} \cdot 8{,}31\,\frac{\text{Дж}}{\text{моль}\cdot\text{К}} \cdot 20\,\text{К}
            = 498\,\text{Дж}.
            \text{Увеличилась.}
    $
}
\solutionspace{40pt}

\tasknumber{6}%
\task{%
    Газу сообщили некоторое количество теплоты,
    при этом треть его он потратил на совершение работы,
    одновременно увеличив свою внутреннюю энергию на $3000\,\text{Дж}$.
    Определите количество теплоты, сообщённое газу.
}
\answer{%
    \begin{align*}
    Q &= A' + \Delta U, A' = \frac 13 Q \implies Q \cdot \cbr{1 - \frac 13} = \Delta U \implies Q = \frac{\Delta U}{1 - \frac 13} = \frac{3000\,\text{Дж}}{1 - \frac 13} \approx 4500\,\text{Дж}.
    \\
    A' &= \frac 13 Q
        = \frac 13 \cdot \frac{\Delta U}{1 - \frac 13}
        = \frac{\Delta U}{3 - 1}
        = \frac{3000\,\text{Дж}}{3 - 1} \approx 1500\,\text{Дж}.
    \end{align*}
}
\solutionspace{60pt}

\tasknumber{7}%
\task{%
    В некотором процессе внешние силы совершили над газом работу $200\,\text{Дж}$,
    при этом его внутренняя энергия уменьшилась на $250\,\text{Дж}$.
    Определите количество тепла, переданное при этом процессе газу.
    Явно пропишите, подводили газу тепло или же отводили.
}
\answer{%
    $
        Q = A_\text{газа} + \Delta U, A_\text{газа} = -A_\text{внешняя}
        \implies Q = A_\text{газа} + \Delta U = - 200\,\text{Дж} - 250\,\text{Дж} = -450\,\text{Дж}.
        \text{ Отводили.}
    $
}

\variantsplitter

\addpersonalvariant{Рената Таржиманова}

\tasknumber{1}%
\task{%
    Укажите, верны ли утверждения («да» или «нет» слева от каждого утверждения):
    \begin{enumerate}
        \item При адиабатическом расширении идеальный газ совершает ровно столько работы, сколько внутренней энергии теряет.
        % \item В силу третьего закона Ньютона, совершённая газом работа и работа, совершённая над ним, всегда равны по модулю и противоположны по знаку.
        \item Работу газа в некотором процессе можно вычислять как площадь под графиком в системе координат $VT$, главное лишь правильно расположить оси.
        % \item Дважды два пять.
        \item При изобарном процессе внутренняя энергия идеального одноатомного газа не изменяется, даже если ему подводят тепло.
        \item Газ может совершить ненулевую работу в изотермическом процессе.
        % \item Адиабатический процесс лишь по воле случая не имеет приставки «изо»: в нём изменяются давление, температура и объём, но это не все макропараметры идеального газа.
        \item Полученное выражение для внутренней энергии идеального газа ($\frac 32 \nu RT$) применимо к двухоатомному газу, при этом, например, уравнение состояния идеального газа применимо независимо от числа атомов в молекулах газа.
    \end{enumerate}
}
\answer{%
    $\text{да, нет, нет, да, нет}$
}

\tasknumber{2}%
\task{%
    Определите давление одноатомного идеального газа, занимающего объём $3\,\text{л}$,
    если его внутренняя энергия составляет $250\,\text{Дж}$.
}
\answer{%
    $U = \frac 32 \nu R T = \frac 32 PV \implies P = \frac 23 \cdot \frac UV= \frac 23 \cdot \frac{ 250\,\text{Дж} }{ 3\,\text{л} } \approx 55\,\text{кПа}.$
}
\solutionspace{40pt}

\tasknumber{3}%
\task{%
    Газ расширился от $150\,\text{л}$ до $650\,\text{л}$.
    Давление газа при этом оставалось постоянным и равным $1{,}2\,\text{атм}$.
    Определите работу газа, ответ выразите в килоджоулях.
    $p_{\text{aтм}} = 100\,\text{кПа}$.
}
\answer{%
    $A = P\Delta V = P(V_2 - V_1) = 1{,}2\,\text{атм} \cdot \cbr{650\,\text{л} - 150\,\text{л}} = 60{,}0\,\text{кДж}.$
}
\solutionspace{40pt}

\tasknumber{4}%
\task{%
    Как изменилась внутренняя энергия одноатомного идеального газа при переходе из состояния 1 в состояние 2?
    $P_1 = 2\,\text{МПа}$, $V_1 = 5\,\text{л}$, $P_2 = 1{,}5\,\text{МПа}$, $V_2 = 6\,\text{л}$.
    Как изменилась при этом температура газа?
}
\answer{%
    \begin{align*}
    P_1V_1 &= \nu R T_1, P_2V_2 = \nu R T_2, \\
    \Delta U &= U_2-U_1 = \frac 32 \nu R T_2- \frac 32 \nu R T_1 = \frac 32 P_2 V_2 - \frac 32 P_1 V_1= \frac 32 \cdot \cbr{1{,}5\,\text{МПа} \cdot 6\,\text{л} - 2\,\text{МПа} \cdot 5\,\text{л}} = -1500\,\text{Дж}.
    \\
    \frac{T_2}{T_1} &= \frac{\frac{P_2V_2}{\nu R}}{\frac{P_1V_1}{\nu R}} = \frac{P_2V_2}{P_1V_1}= \frac{1{,}5\,\text{МПа} \cdot 6\,\text{л}}{2\,\text{МПа} \cdot 5\,\text{л}} \approx 0{,}90.
    \end{align*}
}
\solutionspace{80pt}

\tasknumber{5}%
\task{%
    $4\,\text{моль}$ идеального одноатомного газа нагрели на $10\,\text{К}$.
    Определите изменение внутренней энергии газа.
    Увеличилась она или уменьшилась?
    Универсальная газовая постоянная $R = 8{,}31\,\frac{\text{Дж}}{\text{моль}\cdot\text{К}}$.
}
\answer{%
    $
        \Delta U = \frac 32 \nu R \Delta T
            =  \frac 32 \cdot 4\,\text{моль} \cdot 8{,}31\,\frac{\text{Дж}}{\text{моль}\cdot\text{К}} \cdot 10\,\text{К}
            = 498\,\text{Дж}.
            \text{Увеличилась.}
    $
}
\solutionspace{40pt}

\tasknumber{6}%
\task{%
    Газу сообщили некоторое количество теплоты,
    при этом треть его он потратил на совершение работы,
    одновременно увеличив свою внутреннюю энергию на $1500\,\text{Дж}$.
    Определите количество теплоты, сообщённое газу.
}
\answer{%
    \begin{align*}
    Q &= A' + \Delta U, A' = \frac 13 Q \implies Q \cdot \cbr{1 - \frac 13} = \Delta U \implies Q = \frac{\Delta U}{1 - \frac 13} = \frac{1500\,\text{Дж}}{1 - \frac 13} \approx 2250\,\text{Дж}.
    \\
    A' &= \frac 13 Q
        = \frac 13 \cdot \frac{\Delta U}{1 - \frac 13}
        = \frac{\Delta U}{3 - 1}
        = \frac{1500\,\text{Дж}}{3 - 1} \approx 750\,\text{Дж}.
    \end{align*}
}
\solutionspace{60pt}

\tasknumber{7}%
\task{%
    В некотором процессе газ совершил работу $100\,\text{Дж}$,
    при этом его внутренняя энергия увеличилась на $150\,\text{Дж}$.
    Определите количество тепла, переданное при этом процессе газу.
    Явно пропишите, подводили газу тепло или же отводили.
}
\answer{%
    $
        Q = A_\text{газа} + \Delta U, A_\text{газа} = -A_\text{внешняя}
        \implies Q = A_\text{газа} + \Delta U =  100\,\text{Дж} +  150\,\text{Дж} = 250\,\text{Дж}.
        \text{ Подводили.}
    $
}

\variantsplitter

\addpersonalvariant{Андрей Щербаков}

\tasknumber{1}%
\task{%
    Укажите, верны ли утверждения («да» или «нет» слева от каждого утверждения):
    \begin{enumerate}
        \item При адиабатическом расширении идеальный газ совершает ровно столько работы, сколько внутренней энергии теряет.
        % \item В силу третьего закона Ньютона, совершённая газом работа и работа, совершённая над ним, всегда равны по модулю и противоположны по знаку.
        \item Работу газа в некотором процессе можно вычислять как площадь под графиком в системе координат $VT$, главное лишь правильно расположить оси.
        % \item Дважды два три.
        \item При изотермическом процессе внутренняя энергия идеального одноатомного газа не изменяется, даже если ему подводят тепло.
        \item Газ может совершить ненулевую работу в изобарном процессе.
        % \item Адиабатический процесс лишь по воле случая не имеет приставки «изо»: в нём изменяются давление, температура и объём, но это не все макропараметры идеального газа.
        \item Полученное выражение для внутренней энергии идеального газа ($\frac 32 \nu RT$) применимо к двухоатомному газу, при этом, например, уравнение состояния идеального газа применимо независимо от числа атомов в молекулах газа.
    \end{enumerate}
}
\answer{%
    $\text{да, нет, да, да, нет}$
}

\tasknumber{2}%
\task{%
    Определите давление одноатомного идеального газа, занимающего объём $5\,\text{л}$,
    если его внутренняя энергия составляет $250\,\text{Дж}$.
}
\answer{%
    $U = \frac 32 \nu R T = \frac 32 PV \implies P = \frac 23 \cdot \frac UV= \frac 23 \cdot \frac{ 250\,\text{Дж} }{ 5\,\text{л} } \approx 33\,\text{кПа}.$
}
\solutionspace{40pt}

\tasknumber{3}%
\task{%
    Газ расширился от $150\,\text{л}$ до $550\,\text{л}$.
    Давление газа при этом оставалось постоянным и равным $2{,}5\,\text{атм}$.
    Определите работу газа, ответ выразите в килоджоулях.
    $p_{\text{aтм}} = 100\,\text{кПа}$.
}
\answer{%
    $A = P\Delta V = P(V_2 - V_1) = 2{,}5\,\text{атм} \cdot \cbr{550\,\text{л} - 150\,\text{л}} = 100{,}0\,\text{кДж}.$
}
\solutionspace{40pt}

\tasknumber{4}%
\task{%
    Как изменилась внутренняя энергия одноатомного идеального газа при переходе из состояния 1 в состояние 2?
    $P_1 = 4\,\text{МПа}$, $V_1 = 7\,\text{л}$, $P_2 = 2{,}5\,\text{МПа}$, $V_2 = 4\,\text{л}$.
    Как изменилась при этом температура газа?
}
\answer{%
    \begin{align*}
    P_1V_1 &= \nu R T_1, P_2V_2 = \nu R T_2, \\
    \Delta U &= U_2-U_1 = \frac 32 \nu R T_2- \frac 32 \nu R T_1 = \frac 32 P_2 V_2 - \frac 32 P_1 V_1= \frac 32 \cdot \cbr{2{,}5\,\text{МПа} \cdot 4\,\text{л} - 4\,\text{МПа} \cdot 7\,\text{л}} = -27000\,\text{Дж}.
    \\
    \frac{T_2}{T_1} &= \frac{\frac{P_2V_2}{\nu R}}{\frac{P_1V_1}{\nu R}} = \frac{P_2V_2}{P_1V_1}= \frac{2{,}5\,\text{МПа} \cdot 4\,\text{л}}{4\,\text{МПа} \cdot 7\,\text{л}} \approx 0{,}36.
    \end{align*}
}
\solutionspace{80pt}

\tasknumber{5}%
\task{%
    $3\,\text{моль}$ идеального одноатомного газа охладили на $20\,\text{К}$.
    Определите изменение внутренней энергии газа.
    Увеличилась она или уменьшилась?
    Универсальная газовая постоянная $R = 8{,}31\,\frac{\text{Дж}}{\text{моль}\cdot\text{К}}$.
}
\answer{%
    $
        \Delta U = \frac 32 \nu R \Delta T
            = - \frac 32 \cdot 3\,\text{моль} \cdot 8{,}31\,\frac{\text{Дж}}{\text{моль}\cdot\text{К}} \cdot 20\,\text{К}
            = -747\,\text{Дж}.
            \text{Уменьшилась.}
    $
}
\solutionspace{40pt}

\tasknumber{6}%
\task{%
    Газу сообщили некоторое количество теплоты,
    при этом половину его он потратил на совершение работы,
    одновременно увеличив свою внутреннюю энергию на $1200\,\text{Дж}$.
    Определите количество теплоты, сообщённое газу.
}
\answer{%
    \begin{align*}
    Q &= A' + \Delta U, A' = \frac 12 Q \implies Q \cdot \cbr{1 - \frac 12} = \Delta U \implies Q = \frac{\Delta U}{1 - \frac 12} = \frac{1200\,\text{Дж}}{1 - \frac 12} \approx 2400\,\text{Дж}.
    \\
    A' &= \frac 12 Q
        = \frac 12 \cdot \frac{\Delta U}{1 - \frac 12}
        = \frac{\Delta U}{2 - 1}
        = \frac{1200\,\text{Дж}}{2 - 1} \approx 1200\,\text{Дж}.
    \end{align*}
}
\solutionspace{60pt}

\tasknumber{7}%
\task{%
    В некотором процессе внешние силы совершили над газом работу $200\,\text{Дж}$,
    при этом его внутренняя энергия увеличилась на $250\,\text{Дж}$.
    Определите количество тепла, переданное при этом процессе газу.
    Явно пропишите, подводили газу тепло или же отводили.
}
\answer{%
    $
        Q = A_\text{газа} + \Delta U, A_\text{газа} = -A_\text{внешняя}
        \implies Q = A_\text{газа} + \Delta U = - 200\,\text{Дж} +  250\,\text{Дж} = 50\,\text{Дж}.
        \text{ Подводили.}
    $
}

\variantsplitter

\addpersonalvariant{Михаил Ярошевский}

\tasknumber{1}%
\task{%
    Укажите, верны ли утверждения («да» или «нет» слева от каждого утверждения):
    \begin{enumerate}
        \item При изобарном расширении идеальный газ совершает ровно столько работы, сколько внутренней энергии теряет.
        % \item В силу третьего закона Ньютона, совершённая газом работа и работа, совершённая над ним, всегда равны по модулю и противоположны по знаку.
        \item Работу газа в некотором процессе можно вычислять как площадь под графиком в системе координат $VT$, главное лишь правильно расположить оси.
        % \item Дважды два четыре.
        \item При изохорном процессе внутренняя энергия идеального одноатомного газа не изменяется, даже если ему подводят тепло.
        \item Газ может совершить ненулевую работу в изохорном процессе.
        % \item Адиабатический процесс лишь по воле случая не имеет приставки «изо»: в нём изменяются давление, температура и объём, но это не все макропараметры идеального газа.
        \item Полученное выражение для внутренней энергии идеального газа ($\frac 32 \nu RT$) применимо к трёхатомному газу, при этом, например, уравнение состояния идеального газа применимо независимо от числа атомов в молекулах газа.
    \end{enumerate}
}
\answer{%
    $\text{нет, нет, нет, нет, нет}$
}

\tasknumber{2}%
\task{%
    Определите давление одноатомного идеального газа, занимающего объём $4\,\text{л}$,
    если его внутренняя энергия составляет $400\,\text{Дж}$.
}
\answer{%
    $U = \frac 32 \nu R T = \frac 32 PV \implies P = \frac 23 \cdot \frac UV= \frac 23 \cdot \frac{ 400\,\text{Дж} }{ 4\,\text{л} } \approx 66\,\text{кПа}.$
}
\solutionspace{40pt}

\tasknumber{3}%
\task{%
    Газ расширился от $350\,\text{л}$ до $650\,\text{л}$.
    Давление газа при этом оставалось постоянным и равным $1{,}5\,\text{атм}$.
    Определите работу газа, ответ выразите в килоджоулях.
    $p_{\text{aтм}} = 100\,\text{кПа}$.
}
\answer{%
    $A = P\Delta V = P(V_2 - V_1) = 1{,}5\,\text{атм} \cdot \cbr{650\,\text{л} - 350\,\text{л}} = 45{,}0\,\text{кДж}.$
}
\solutionspace{40pt}

\tasknumber{4}%
\task{%
    Как изменилась внутренняя энергия одноатомного идеального газа при переходе из состояния 1 в состояние 2?
    $P_1 = 4\,\text{МПа}$, $V_1 = 7\,\text{л}$, $P_2 = 1{,}5\,\text{МПа}$, $V_2 = 2\,\text{л}$.
    Как изменилась при этом температура газа?
}
\answer{%
    \begin{align*}
    P_1V_1 &= \nu R T_1, P_2V_2 = \nu R T_2, \\
    \Delta U &= U_2-U_1 = \frac 32 \nu R T_2- \frac 32 \nu R T_1 = \frac 32 P_2 V_2 - \frac 32 P_1 V_1= \frac 32 \cdot \cbr{1{,}5\,\text{МПа} \cdot 2\,\text{л} - 4\,\text{МПа} \cdot 7\,\text{л}} = -37500\,\text{Дж}.
    \\
    \frac{T_2}{T_1} &= \frac{\frac{P_2V_2}{\nu R}}{\frac{P_1V_1}{\nu R}} = \frac{P_2V_2}{P_1V_1}= \frac{1{,}5\,\text{МПа} \cdot 2\,\text{л}}{4\,\text{МПа} \cdot 7\,\text{л}} \approx 0{,}11.
    \end{align*}
}
\solutionspace{80pt}

\tasknumber{5}%
\task{%
    $5\,\text{моль}$ идеального одноатомного газа нагрели на $20\,\text{К}$.
    Определите изменение внутренней энергии газа.
    Увеличилась она или уменьшилась?
    Универсальная газовая постоянная $R = 8{,}31\,\frac{\text{Дж}}{\text{моль}\cdot\text{К}}$.
}
\answer{%
    $
        \Delta U = \frac 32 \nu R \Delta T
            =  \frac 32 \cdot 5\,\text{моль} \cdot 8{,}31\,\frac{\text{Дж}}{\text{моль}\cdot\text{К}} \cdot 20\,\text{К}
            = 1246\,\text{Дж}.
            \text{Увеличилась.}
    $
}
\solutionspace{40pt}

\tasknumber{6}%
\task{%
    Газу сообщили некоторое количество теплоты,
    при этом половину его он потратил на совершение работы,
    одновременно увеличив свою внутреннюю энергию на $3000\,\text{Дж}$.
    Определите работу, совершённую газом.
}
\answer{%
    \begin{align*}
    Q &= A' + \Delta U, A' = \frac 12 Q \implies Q \cdot \cbr{1 - \frac 12} = \Delta U \implies Q = \frac{\Delta U}{1 - \frac 12} = \frac{3000\,\text{Дж}}{1 - \frac 12} \approx 6000\,\text{Дж}.
    \\
    A' &= \frac 12 Q
        = \frac 12 \cdot \frac{\Delta U}{1 - \frac 12}
        = \frac{\Delta U}{2 - 1}
        = \frac{3000\,\text{Дж}}{2 - 1} \approx 3000\,\text{Дж}.
    \end{align*}
}
\solutionspace{60pt}

\tasknumber{7}%
\task{%
    В некотором процессе внешние силы совершили над газом работу $300\,\text{Дж}$,
    при этом его внутренняя энергия уменьшилась на $350\,\text{Дж}$.
    Определите количество тепла, переданное при этом процессе газу.
    Явно пропишите, подводили газу тепло или же отводили.
}
\answer{%
    $
        Q = A_\text{газа} + \Delta U, A_\text{газа} = -A_\text{внешняя}
        \implies Q = A_\text{газа} + \Delta U = - 300\,\text{Дж} - 350\,\text{Дж} = -650\,\text{Дж}.
        \text{ Отводили.}
    $
}

\variantsplitter

\addpersonalvariant{Алексей Алимпиев}

\tasknumber{1}%
\task{%
    Укажите, верны ли утверждения («да» или «нет» слева от каждого утверждения):
    \begin{enumerate}
        \item При изобарном расширении идеальный газ совершает ровно столько работы, сколько внутренней энергии теряет.
        % \item В силу третьего закона Ньютона, совершённая газом работа и работа, совершённая над ним, всегда равны по модулю и противоположны по знаку.
        \item Работу газа в некотором процессе можно вычислять как площадь под графиком в системе координат $PV$, главное лишь правильно расположить оси.
        % \item Дважды два три.
        \item При изотермическом процессе внутренняя энергия идеального одноатомного газа не изменяется, даже если ему подводят тепло.
        \item Газ может совершить ненулевую работу в изобарном процессе.
        % \item Адиабатический процесс лишь по воле случая не имеет приставки «изо»: в нём изменяются давление, температура и объём, но это не все макропараметры идеального газа.
        \item Полученное выражение для внутренней энергии идеального газа ($\frac 32 \nu RT$) применимо к двухоатомному газу, при этом, например, уравнение состояния идеального газа применимо независимо от числа атомов в молекулах газа.
    \end{enumerate}
}
\answer{%
    $\text{нет, да, да, да, нет}$
}

\tasknumber{2}%
\task{%
    Определите давление одноатомного идеального газа, занимающего объём $6\,\text{л}$,
    если его внутренняя энергия составляет $300\,\text{Дж}$.
}
\answer{%
    $U = \frac 32 \nu R T = \frac 32 PV \implies P = \frac 23 \cdot \frac UV= \frac 23 \cdot \frac{ 300\,\text{Дж} }{ 6\,\text{л} } \approx 33\,\text{кПа}.$
}
\solutionspace{40pt}

\tasknumber{3}%
\task{%
    Газ расширился от $150\,\text{л}$ до $550\,\text{л}$.
    Давление газа при этом оставалось постоянным и равным $3{,}5\,\text{атм}$.
    Определите работу газа, ответ выразите в килоджоулях.
    $p_{\text{aтм}} = 100\,\text{кПа}$.
}
\answer{%
    $A = P\Delta V = P(V_2 - V_1) = 3{,}5\,\text{атм} \cdot \cbr{550\,\text{л} - 150\,\text{л}} = 140{,}0\,\text{кДж}.$
}
\solutionspace{40pt}

\tasknumber{4}%
\task{%
    Как изменилась внутренняя энергия одноатомного идеального газа при переходе из состояния 1 в состояние 2?
    $P_1 = 3\,\text{МПа}$, $V_1 = 7\,\text{л}$, $P_2 = 4{,}5\,\text{МПа}$, $V_2 = 2\,\text{л}$.
    Как изменилась при этом температура газа?
}
\answer{%
    \begin{align*}
    P_1V_1 &= \nu R T_1, P_2V_2 = \nu R T_2, \\
    \Delta U &= U_2-U_1 = \frac 32 \nu R T_2- \frac 32 \nu R T_1 = \frac 32 P_2 V_2 - \frac 32 P_1 V_1= \frac 32 \cdot \cbr{4{,}5\,\text{МПа} \cdot 2\,\text{л} - 3\,\text{МПа} \cdot 7\,\text{л}} = -18000\,\text{Дж}.
    \\
    \frac{T_2}{T_1} &= \frac{\frac{P_2V_2}{\nu R}}{\frac{P_1V_1}{\nu R}} = \frac{P_2V_2}{P_1V_1}= \frac{4{,}5\,\text{МПа} \cdot 2\,\text{л}}{3\,\text{МПа} \cdot 7\,\text{л}} \approx 0{,}43.
    \end{align*}
}
\solutionspace{80pt}

\tasknumber{5}%
\task{%
    $2\,\text{моль}$ идеального одноатомного газа охладили на $20\,\text{К}$.
    Определите изменение внутренней энергии газа.
    Увеличилась она или уменьшилась?
    Универсальная газовая постоянная $R = 8{,}31\,\frac{\text{Дж}}{\text{моль}\cdot\text{К}}$.
}
\answer{%
    $
        \Delta U = \frac 32 \nu R \Delta T
            = - \frac 32 \cdot 2\,\text{моль} \cdot 8{,}31\,\frac{\text{Дж}}{\text{моль}\cdot\text{К}} \cdot 20\,\text{К}
            = -498\,\text{Дж}.
            \text{Уменьшилась.}
    $
}
\solutionspace{40pt}

\tasknumber{6}%
\task{%
    Газу сообщили некоторое количество теплоты,
    при этом половину его он потратил на совершение работы,
    одновременно увеличив свою внутреннюю энергию на $1200\,\text{Дж}$.
    Определите количество теплоты, сообщённое газу.
}
\answer{%
    \begin{align*}
    Q &= A' + \Delta U, A' = \frac 12 Q \implies Q \cdot \cbr{1 - \frac 12} = \Delta U \implies Q = \frac{\Delta U}{1 - \frac 12} = \frac{1200\,\text{Дж}}{1 - \frac 12} \approx 2400\,\text{Дж}.
    \\
    A' &= \frac 12 Q
        = \frac 12 \cdot \frac{\Delta U}{1 - \frac 12}
        = \frac{\Delta U}{2 - 1}
        = \frac{1200\,\text{Дж}}{2 - 1} \approx 1200\,\text{Дж}.
    \end{align*}
}
\solutionspace{60pt}

\tasknumber{7}%
\task{%
    В некотором процессе газ совершил работу $200\,\text{Дж}$,
    при этом его внутренняя энергия увеличилась на $150\,\text{Дж}$.
    Определите количество тепла, переданное при этом процессе газу.
    Явно пропишите, подводили газу тепло или же отводили.
}
\answer{%
    $
        Q = A_\text{газа} + \Delta U, A_\text{газа} = -A_\text{внешняя}
        \implies Q = A_\text{газа} + \Delta U =  200\,\text{Дж} +  150\,\text{Дж} = 350\,\text{Дж}.
        \text{ Подводили.}
    $
}

\variantsplitter

\addpersonalvariant{Евгений Васин}

\tasknumber{1}%
\task{%
    Укажите, верны ли утверждения («да» или «нет» слева от каждого утверждения):
    \begin{enumerate}
        \item При изобарном расширении идеальный газ совершает ровно столько работы, сколько внутренней энергии теряет.
        % \item В силу третьего закона Ньютона, совершённая газом работа и работа, совершённая над ним, всегда равны по модулю и противоположны по знаку.
        \item Работу газа в некотором процессе можно вычислять как площадь под графиком в системе координат $PV$, главное лишь правильно расположить оси.
        % \item Дважды два три.
        \item При изотермическом процессе внутренняя энергия идеального одноатомного газа не изменяется, даже если ему подводят тепло.
        \item Газ может совершить ненулевую работу в изотермическом процессе.
        % \item Адиабатический процесс лишь по воле случая не имеет приставки «изо»: в нём изменяются давление, температура и объём, но это не все макропараметры идеального газа.
        \item Полученное выражение для внутренней энергии идеального газа ($\frac 32 \nu RT$) применимо к двухоатомному газу, при этом, например, уравнение состояния идеального газа применимо независимо от числа атомов в молекулах газа.
    \end{enumerate}
}
\answer{%
    $\text{нет, да, да, да, нет}$
}

\tasknumber{2}%
\task{%
    Определите давление одноатомного идеального газа, занимающего объём $6\,\text{л}$,
    если его внутренняя энергия составляет $400\,\text{Дж}$.
}
\answer{%
    $U = \frac 32 \nu R T = \frac 32 PV \implies P = \frac 23 \cdot \frac UV= \frac 23 \cdot \frac{ 400\,\text{Дж} }{ 6\,\text{л} } \approx 44\,\text{кПа}.$
}
\solutionspace{40pt}

\tasknumber{3}%
\task{%
    Газ расширился от $150\,\text{л}$ до $450\,\text{л}$.
    Давление газа при этом оставалось постоянным и равным $1{,}2\,\text{атм}$.
    Определите работу газа, ответ выразите в килоджоулях.
    $p_{\text{aтм}} = 100\,\text{кПа}$.
}
\answer{%
    $A = P\Delta V = P(V_2 - V_1) = 1{,}2\,\text{атм} \cdot \cbr{450\,\text{л} - 150\,\text{л}} = 36{,}0\,\text{кДж}.$
}
\solutionspace{40pt}

\tasknumber{4}%
\task{%
    Как изменилась внутренняя энергия одноатомного идеального газа при переходе из состояния 1 в состояние 2?
    $P_1 = 2\,\text{МПа}$, $V_1 = 3\,\text{л}$, $P_2 = 4{,}5\,\text{МПа}$, $V_2 = 6\,\text{л}$.
    Как изменилась при этом температура газа?
}
\answer{%
    \begin{align*}
    P_1V_1 &= \nu R T_1, P_2V_2 = \nu R T_2, \\
    \Delta U &= U_2-U_1 = \frac 32 \nu R T_2- \frac 32 \nu R T_1 = \frac 32 P_2 V_2 - \frac 32 P_1 V_1= \frac 32 \cdot \cbr{4{,}5\,\text{МПа} \cdot 6\,\text{л} - 2\,\text{МПа} \cdot 3\,\text{л}} = 31500\,\text{Дж}.
    \\
    \frac{T_2}{T_1} &= \frac{\frac{P_2V_2}{\nu R}}{\frac{P_1V_1}{\nu R}} = \frac{P_2V_2}{P_1V_1}= \frac{4{,}5\,\text{МПа} \cdot 6\,\text{л}}{2\,\text{МПа} \cdot 3\,\text{л}} \approx 4{,}50.
    \end{align*}
}
\solutionspace{80pt}

\tasknumber{5}%
\task{%
    $3\,\text{моль}$ идеального одноатомного газа охладили на $20\,\text{К}$.
    Определите изменение внутренней энергии газа.
    Увеличилась она или уменьшилась?
    Универсальная газовая постоянная $R = 8{,}31\,\frac{\text{Дж}}{\text{моль}\cdot\text{К}}$.
}
\answer{%
    $
        \Delta U = \frac 32 \nu R \Delta T
            = - \frac 32 \cdot 3\,\text{моль} \cdot 8{,}31\,\frac{\text{Дж}}{\text{моль}\cdot\text{К}} \cdot 20\,\text{К}
            = -747\,\text{Дж}.
            \text{Уменьшилась.}
    $
}
\solutionspace{40pt}

\tasknumber{6}%
\task{%
    Газу сообщили некоторое количество теплоты,
    при этом четверть его он потратил на совершение работы,
    одновременно увеличив свою внутреннюю энергию на $1500\,\text{Дж}$.
    Определите работу, совершённую газом.
}
\answer{%
    \begin{align*}
    Q &= A' + \Delta U, A' = \frac 14 Q \implies Q \cdot \cbr{1 - \frac 14} = \Delta U \implies Q = \frac{\Delta U}{1 - \frac 14} = \frac{1500\,\text{Дж}}{1 - \frac 14} \approx 2000\,\text{Дж}.
    \\
    A' &= \frac 14 Q
        = \frac 14 \cdot \frac{\Delta U}{1 - \frac 14}
        = \frac{\Delta U}{4 - 1}
        = \frac{1500\,\text{Дж}}{4 - 1} \approx 500\,\text{Дж}.
    \end{align*}
}
\solutionspace{60pt}

\tasknumber{7}%
\task{%
    В некотором процессе газ совершил работу $300\,\text{Дж}$,
    при этом его внутренняя энергия уменьшилась на $150\,\text{Дж}$.
    Определите количество тепла, переданное при этом процессе газу.
    Явно пропишите, подводили газу тепло или же отводили.
}
\answer{%
    $
        Q = A_\text{газа} + \Delta U, A_\text{газа} = -A_\text{внешняя}
        \implies Q = A_\text{газа} + \Delta U =  300\,\text{Дж} - 150\,\text{Дж} = 150\,\text{Дж}.
        \text{ Подводили.}
    $
}

\variantsplitter

\addpersonalvariant{Вячеслав Волохов}

\tasknumber{1}%
\task{%
    Укажите, верны ли утверждения («да» или «нет» слева от каждого утверждения):
    \begin{enumerate}
        \item При адиабатическом расширении идеальный газ совершает ровно столько работы, сколько внутренней энергии теряет.
        % \item В силу третьего закона Ньютона, совершённая газом работа и работа, совершённая над ним, всегда равны по модулю и противоположны по знаку.
        \item Работу газа в некотором процессе можно вычислять как площадь под графиком в системе координат $VT$, главное лишь правильно расположить оси.
        % \item Дважды два четыре.
        \item При изобарном процессе внутренняя энергия идеального одноатомного газа не изменяется, даже если ему подводят тепло.
        \item Газ может совершить ненулевую работу в изобарном процессе.
        % \item Адиабатический процесс лишь по воле случая не имеет приставки «изо»: в нём изменяются давление, температура и объём, но это не все макропараметры идеального газа.
        \item Полученное выражение для внутренней энергии идеального газа ($\frac 32 \nu RT$) применимо к двухоатомному газу, при этом, например, уравнение состояния идеального газа применимо независимо от числа атомов в молекулах газа.
    \end{enumerate}
}
\answer{%
    $\text{да, нет, нет, да, нет}$
}

\tasknumber{2}%
\task{%
    Определите давление одноатомного идеального газа, занимающего объём $2\,\text{л}$,
    если его внутренняя энергия составляет $300\,\text{Дж}$.
}
\answer{%
    $U = \frac 32 \nu R T = \frac 32 PV \implies P = \frac 23 \cdot \frac UV= \frac 23 \cdot \frac{ 300\,\text{Дж} }{ 2\,\text{л} } \approx 100\,\text{кПа}.$
}
\solutionspace{40pt}

\tasknumber{3}%
\task{%
    Газ расширился от $250\,\text{л}$ до $450\,\text{л}$.
    Давление газа при этом оставалось постоянным и равным $1{,}2\,\text{атм}$.
    Определите работу газа, ответ выразите в килоджоулях.
    $p_{\text{aтм}} = 100\,\text{кПа}$.
}
\answer{%
    $A = P\Delta V = P(V_2 - V_1) = 1{,}2\,\text{атм} \cdot \cbr{450\,\text{л} - 250\,\text{л}} = 24{,}0\,\text{кДж}.$
}
\solutionspace{40pt}

\tasknumber{4}%
\task{%
    Как изменилась внутренняя энергия одноатомного идеального газа при переходе из состояния 1 в состояние 2?
    $P_1 = 2\,\text{МПа}$, $V_1 = 7\,\text{л}$, $P_2 = 3{,}5\,\text{МПа}$, $V_2 = 6\,\text{л}$.
    Как изменилась при этом температура газа?
}
\answer{%
    \begin{align*}
    P_1V_1 &= \nu R T_1, P_2V_2 = \nu R T_2, \\
    \Delta U &= U_2-U_1 = \frac 32 \nu R T_2- \frac 32 \nu R T_1 = \frac 32 P_2 V_2 - \frac 32 P_1 V_1= \frac 32 \cdot \cbr{3{,}5\,\text{МПа} \cdot 6\,\text{л} - 2\,\text{МПа} \cdot 7\,\text{л}} = 10500\,\text{Дж}.
    \\
    \frac{T_2}{T_1} &= \frac{\frac{P_2V_2}{\nu R}}{\frac{P_1V_1}{\nu R}} = \frac{P_2V_2}{P_1V_1}= \frac{3{,}5\,\text{МПа} \cdot 6\,\text{л}}{2\,\text{МПа} \cdot 7\,\text{л}} \approx 1{,}50.
    \end{align*}
}
\solutionspace{80pt}

\tasknumber{5}%
\task{%
    $5\,\text{моль}$ идеального одноатомного газа охладили на $10\,\text{К}$.
    Определите изменение внутренней энергии газа.
    Увеличилась она или уменьшилась?
    Универсальная газовая постоянная $R = 8{,}31\,\frac{\text{Дж}}{\text{моль}\cdot\text{К}}$.
}
\answer{%
    $
        \Delta U = \frac 32 \nu R \Delta T
            = - \frac 32 \cdot 5\,\text{моль} \cdot 8{,}31\,\frac{\text{Дж}}{\text{моль}\cdot\text{К}} \cdot 10\,\text{К}
            = -623\,\text{Дж}.
            \text{Уменьшилась.}
    $
}
\solutionspace{40pt}

\tasknumber{6}%
\task{%
    Газу сообщили некоторое количество теплоты,
    при этом треть его он потратил на совершение работы,
    одновременно увеличив свою внутреннюю энергию на $1200\,\text{Дж}$.
    Определите работу, совершённую газом.
}
\answer{%
    \begin{align*}
    Q &= A' + \Delta U, A' = \frac 13 Q \implies Q \cdot \cbr{1 - \frac 13} = \Delta U \implies Q = \frac{\Delta U}{1 - \frac 13} = \frac{1200\,\text{Дж}}{1 - \frac 13} \approx 1800\,\text{Дж}.
    \\
    A' &= \frac 13 Q
        = \frac 13 \cdot \frac{\Delta U}{1 - \frac 13}
        = \frac{\Delta U}{3 - 1}
        = \frac{1200\,\text{Дж}}{3 - 1} \approx 600\,\text{Дж}.
    \end{align*}
}
\solutionspace{60pt}

\tasknumber{7}%
\task{%
    В некотором процессе внешние силы совершили над газом работу $300\,\text{Дж}$,
    при этом его внутренняя энергия увеличилась на $350\,\text{Дж}$.
    Определите количество тепла, переданное при этом процессе газу.
    Явно пропишите, подводили газу тепло или же отводили.
}
\answer{%
    $
        Q = A_\text{газа} + \Delta U, A_\text{газа} = -A_\text{внешняя}
        \implies Q = A_\text{газа} + \Delta U = - 300\,\text{Дж} +  350\,\text{Дж} = 50\,\text{Дж}.
        \text{ Подводили.}
    $
}

\variantsplitter

\addpersonalvariant{Герман Говоров}

\tasknumber{1}%
\task{%
    Укажите, верны ли утверждения («да» или «нет» слева от каждого утверждения):
    \begin{enumerate}
        \item При изобарном расширении идеальный газ совершает ровно столько работы, сколько внутренней энергии теряет.
        % \item В силу третьего закона Ньютона, совершённая газом работа и работа, совершённая над ним, всегда равны по модулю и противоположны по знаку.
        \item Работу газа в некотором процессе можно вычислять как площадь под графиком в системе координат $VT$, главное лишь правильно расположить оси.
        % \item Дважды два пять.
        \item При изохорном процессе внутренняя энергия идеального одноатомного газа не изменяется, даже если ему подводят тепло.
        \item Газ может совершить ненулевую работу в изобарном процессе.
        % \item Адиабатический процесс лишь по воле случая не имеет приставки «изо»: в нём изменяются давление, температура и объём, но это не все макропараметры идеального газа.
        \item Полученное выражение для внутренней энергии идеального газа ($\frac 32 \nu RT$) применимо к одноатомному газу, при этом, например, уравнение состояния идеального газа применимо независимо от числа атомов в молекулах газа.
    \end{enumerate}
}
\answer{%
    $\text{нет, нет, нет, да, да}$
}

\tasknumber{2}%
\task{%
    Определите давление одноатомного идеального газа, занимающего объём $6\,\text{л}$,
    если его внутренняя энергия составляет $300\,\text{Дж}$.
}
\answer{%
    $U = \frac 32 \nu R T = \frac 32 PV \implies P = \frac 23 \cdot \frac UV= \frac 23 \cdot \frac{ 300\,\text{Дж} }{ 6\,\text{л} } \approx 33\,\text{кПа}.$
}
\solutionspace{40pt}

\tasknumber{3}%
\task{%
    Газ расширился от $350\,\text{л}$ до $550\,\text{л}$.
    Давление газа при этом оставалось постоянным и равным $2{,}5\,\text{атм}$.
    Определите работу газа, ответ выразите в килоджоулях.
    $p_{\text{aтм}} = 100\,\text{кПа}$.
}
\answer{%
    $A = P\Delta V = P(V_2 - V_1) = 2{,}5\,\text{атм} \cdot \cbr{550\,\text{л} - 350\,\text{л}} = 50{,}0\,\text{кДж}.$
}
\solutionspace{40pt}

\tasknumber{4}%
\task{%
    Как изменилась внутренняя энергия одноатомного идеального газа при переходе из состояния 1 в состояние 2?
    $P_1 = 4\,\text{МПа}$, $V_1 = 5\,\text{л}$, $P_2 = 3{,}5\,\text{МПа}$, $V_2 = 8\,\text{л}$.
    Как изменилась при этом температура газа?
}
\answer{%
    \begin{align*}
    P_1V_1 &= \nu R T_1, P_2V_2 = \nu R T_2, \\
    \Delta U &= U_2-U_1 = \frac 32 \nu R T_2- \frac 32 \nu R T_1 = \frac 32 P_2 V_2 - \frac 32 P_1 V_1= \frac 32 \cdot \cbr{3{,}5\,\text{МПа} \cdot 8\,\text{л} - 4\,\text{МПа} \cdot 5\,\text{л}} = 12000\,\text{Дж}.
    \\
    \frac{T_2}{T_1} &= \frac{\frac{P_2V_2}{\nu R}}{\frac{P_1V_1}{\nu R}} = \frac{P_2V_2}{P_1V_1}= \frac{3{,}5\,\text{МПа} \cdot 8\,\text{л}}{4\,\text{МПа} \cdot 5\,\text{л}} \approx 1{,}40.
    \end{align*}
}
\solutionspace{80pt}

\tasknumber{5}%
\task{%
    $4\,\text{моль}$ идеального одноатомного газа нагрели на $10\,\text{К}$.
    Определите изменение внутренней энергии газа.
    Увеличилась она или уменьшилась?
    Универсальная газовая постоянная $R = 8{,}31\,\frac{\text{Дж}}{\text{моль}\cdot\text{К}}$.
}
\answer{%
    $
        \Delta U = \frac 32 \nu R \Delta T
            =  \frac 32 \cdot 4\,\text{моль} \cdot 8{,}31\,\frac{\text{Дж}}{\text{моль}\cdot\text{К}} \cdot 10\,\text{К}
            = 498\,\text{Дж}.
            \text{Увеличилась.}
    $
}
\solutionspace{40pt}

\tasknumber{6}%
\task{%
    Газу сообщили некоторое количество теплоты,
    при этом половину его он потратил на совершение работы,
    одновременно увеличив свою внутреннюю энергию на $2400\,\text{Дж}$.
    Определите количество теплоты, сообщённое газу.
}
\answer{%
    \begin{align*}
    Q &= A' + \Delta U, A' = \frac 12 Q \implies Q \cdot \cbr{1 - \frac 12} = \Delta U \implies Q = \frac{\Delta U}{1 - \frac 12} = \frac{2400\,\text{Дж}}{1 - \frac 12} \approx 4800\,\text{Дж}.
    \\
    A' &= \frac 12 Q
        = \frac 12 \cdot \frac{\Delta U}{1 - \frac 12}
        = \frac{\Delta U}{2 - 1}
        = \frac{2400\,\text{Дж}}{2 - 1} \approx 2400\,\text{Дж}.
    \end{align*}
}
\solutionspace{60pt}

\tasknumber{7}%
\task{%
    В некотором процессе газ совершил работу $200\,\text{Дж}$,
    при этом его внутренняя энергия уменьшилась на $450\,\text{Дж}$.
    Определите количество тепла, переданное при этом процессе газу.
    Явно пропишите, подводили газу тепло или же отводили.
}
\answer{%
    $
        Q = A_\text{газа} + \Delta U, A_\text{газа} = -A_\text{внешняя}
        \implies Q = A_\text{газа} + \Delta U =  200\,\text{Дж} - 450\,\text{Дж} = -250\,\text{Дж}.
        \text{ Отводили.}
    $
}

\variantsplitter

\addpersonalvariant{София Журавлёва}

\tasknumber{1}%
\task{%
    Укажите, верны ли утверждения («да» или «нет» слева от каждого утверждения):
    \begin{enumerate}
        \item При адиабатическом расширении идеальный газ совершает ровно столько работы, сколько внутренней энергии теряет.
        % \item В силу третьего закона Ньютона, совершённая газом работа и работа, совершённая над ним, всегда равны по модулю и противоположны по знаку.
        \item Работу газа в некотором процессе можно вычислять как площадь под графиком в системе координат $VT$, главное лишь правильно расположить оси.
        % \item Дважды два четыре.
        \item При изотермическом процессе внутренняя энергия идеального одноатомного газа не изменяется, даже если ему подводят тепло.
        \item Газ может совершить ненулевую работу в изотермическом процессе.
        % \item Адиабатический процесс лишь по воле случая не имеет приставки «изо»: в нём изменяются давление, температура и объём, но это не все макропараметры идеального газа.
        \item Полученное выражение для внутренней энергии идеального газа ($\frac 32 \nu RT$) применимо к одноатомному газу, при этом, например, уравнение состояния идеального газа применимо независимо от числа атомов в молекулах газа.
    \end{enumerate}
}
\answer{%
    $\text{да, нет, да, да, да}$
}

\tasknumber{2}%
\task{%
    Определите давление одноатомного идеального газа, занимающего объём $6\,\text{л}$,
    если его внутренняя энергия составляет $400\,\text{Дж}$.
}
\answer{%
    $U = \frac 32 \nu R T = \frac 32 PV \implies P = \frac 23 \cdot \frac UV= \frac 23 \cdot \frac{ 400\,\text{Дж} }{ 6\,\text{л} } \approx 44\,\text{кПа}.$
}
\solutionspace{40pt}

\tasknumber{3}%
\task{%
    Газ расширился от $250\,\text{л}$ до $650\,\text{л}$.
    Давление газа при этом оставалось постоянным и равным $1{,}8\,\text{атм}$.
    Определите работу газа, ответ выразите в килоджоулях.
    $p_{\text{aтм}} = 100\,\text{кПа}$.
}
\answer{%
    $A = P\Delta V = P(V_2 - V_1) = 1{,}8\,\text{атм} \cdot \cbr{650\,\text{л} - 250\,\text{л}} = 72{,}0\,\text{кДж}.$
}
\solutionspace{40pt}

\tasknumber{4}%
\task{%
    Как изменилась внутренняя энергия одноатомного идеального газа при переходе из состояния 1 в состояние 2?
    $P_1 = 4\,\text{МПа}$, $V_1 = 5\,\text{л}$, $P_2 = 3{,}5\,\text{МПа}$, $V_2 = 8\,\text{л}$.
    Как изменилась при этом температура газа?
}
\answer{%
    \begin{align*}
    P_1V_1 &= \nu R T_1, P_2V_2 = \nu R T_2, \\
    \Delta U &= U_2-U_1 = \frac 32 \nu R T_2- \frac 32 \nu R T_1 = \frac 32 P_2 V_2 - \frac 32 P_1 V_1= \frac 32 \cdot \cbr{3{,}5\,\text{МПа} \cdot 8\,\text{л} - 4\,\text{МПа} \cdot 5\,\text{л}} = 12000\,\text{Дж}.
    \\
    \frac{T_2}{T_1} &= \frac{\frac{P_2V_2}{\nu R}}{\frac{P_1V_1}{\nu R}} = \frac{P_2V_2}{P_1V_1}= \frac{3{,}5\,\text{МПа} \cdot 8\,\text{л}}{4\,\text{МПа} \cdot 5\,\text{л}} \approx 1{,}40.
    \end{align*}
}
\solutionspace{80pt}

\tasknumber{5}%
\task{%
    $5\,\text{моль}$ идеального одноатомного газа охладили на $30\,\text{К}$.
    Определите изменение внутренней энергии газа.
    Увеличилась она или уменьшилась?
    Универсальная газовая постоянная $R = 8{,}31\,\frac{\text{Дж}}{\text{моль}\cdot\text{К}}$.
}
\answer{%
    $
        \Delta U = \frac 32 \nu R \Delta T
            = - \frac 32 \cdot 5\,\text{моль} \cdot 8{,}31\,\frac{\text{Дж}}{\text{моль}\cdot\text{К}} \cdot 30\,\text{К}
            = -1869\,\text{Дж}.
            \text{Уменьшилась.}
    $
}
\solutionspace{40pt}

\tasknumber{6}%
\task{%
    Газу сообщили некоторое количество теплоты,
    при этом четверть его он потратил на совершение работы,
    одновременно увеличив свою внутреннюю энергию на $3000\,\text{Дж}$.
    Определите количество теплоты, сообщённое газу.
}
\answer{%
    \begin{align*}
    Q &= A' + \Delta U, A' = \frac 14 Q \implies Q \cdot \cbr{1 - \frac 14} = \Delta U \implies Q = \frac{\Delta U}{1 - \frac 14} = \frac{3000\,\text{Дж}}{1 - \frac 14} \approx 4000\,\text{Дж}.
    \\
    A' &= \frac 14 Q
        = \frac 14 \cdot \frac{\Delta U}{1 - \frac 14}
        = \frac{\Delta U}{4 - 1}
        = \frac{3000\,\text{Дж}}{4 - 1} \approx 1000\,\text{Дж}.
    \end{align*}
}
\solutionspace{60pt}

\tasknumber{7}%
\task{%
    В некотором процессе внешние силы совершили над газом работу $200\,\text{Дж}$,
    при этом его внутренняя энергия уменьшилась на $150\,\text{Дж}$.
    Определите количество тепла, переданное при этом процессе газу.
    Явно пропишите, подводили газу тепло или же отводили.
}
\answer{%
    $
        Q = A_\text{газа} + \Delta U, A_\text{газа} = -A_\text{внешняя}
        \implies Q = A_\text{газа} + \Delta U = - 200\,\text{Дж} - 150\,\text{Дж} = -350\,\text{Дж}.
        \text{ Отводили.}
    $
}

\variantsplitter

\addpersonalvariant{Константин Козлов}

\tasknumber{1}%
\task{%
    Укажите, верны ли утверждения («да» или «нет» слева от каждого утверждения):
    \begin{enumerate}
        \item При адиабатическом расширении идеальный газ совершает ровно столько работы, сколько внутренней энергии теряет.
        % \item В силу третьего закона Ньютона, совершённая газом работа и работа, совершённая над ним, всегда равны по модулю и противоположны по знаку.
        \item Работу газа в некотором процессе можно вычислять как площадь под графиком в системе координат $PV$, главное лишь правильно расположить оси.
        % \item Дважды два пять.
        \item При изотермическом процессе внутренняя энергия идеального одноатомного газа не изменяется, даже если ему подводят тепло.
        \item Газ может совершить ненулевую работу в изобарном процессе.
        % \item Адиабатический процесс лишь по воле случая не имеет приставки «изо»: в нём изменяются давление, температура и объём, но это не все макропараметры идеального газа.
        \item Полученное выражение для внутренней энергии идеального газа ($\frac 32 \nu RT$) применимо к трёхатомному газу, при этом, например, уравнение состояния идеального газа применимо независимо от числа атомов в молекулах газа.
    \end{enumerate}
}
\answer{%
    $\text{да, да, да, да, нет}$
}

\tasknumber{2}%
\task{%
    Определите давление одноатомного идеального газа, занимающего объём $4\,\text{л}$,
    если его внутренняя энергия составляет $500\,\text{Дж}$.
}
\answer{%
    $U = \frac 32 \nu R T = \frac 32 PV \implies P = \frac 23 \cdot \frac UV= \frac 23 \cdot \frac{ 500\,\text{Дж} }{ 4\,\text{л} } \approx 83\,\text{кПа}.$
}
\solutionspace{40pt}

\tasknumber{3}%
\task{%
    Газ расширился от $150\,\text{л}$ до $550\,\text{л}$.
    Давление газа при этом оставалось постоянным и равным $1{,}8\,\text{атм}$.
    Определите работу газа, ответ выразите в килоджоулях.
    $p_{\text{aтм}} = 100\,\text{кПа}$.
}
\answer{%
    $A = P\Delta V = P(V_2 - V_1) = 1{,}8\,\text{атм} \cdot \cbr{550\,\text{л} - 150\,\text{л}} = 72{,}0\,\text{кДж}.$
}
\solutionspace{40pt}

\tasknumber{4}%
\task{%
    Как изменилась внутренняя энергия одноатомного идеального газа при переходе из состояния 1 в состояние 2?
    $P_1 = 2\,\text{МПа}$, $V_1 = 7\,\text{л}$, $P_2 = 4{,}5\,\text{МПа}$, $V_2 = 6\,\text{л}$.
    Как изменилась при этом температура газа?
}
\answer{%
    \begin{align*}
    P_1V_1 &= \nu R T_1, P_2V_2 = \nu R T_2, \\
    \Delta U &= U_2-U_1 = \frac 32 \nu R T_2- \frac 32 \nu R T_1 = \frac 32 P_2 V_2 - \frac 32 P_1 V_1= \frac 32 \cdot \cbr{4{,}5\,\text{МПа} \cdot 6\,\text{л} - 2\,\text{МПа} \cdot 7\,\text{л}} = 19500\,\text{Дж}.
    \\
    \frac{T_2}{T_1} &= \frac{\frac{P_2V_2}{\nu R}}{\frac{P_1V_1}{\nu R}} = \frac{P_2V_2}{P_1V_1}= \frac{4{,}5\,\text{МПа} \cdot 6\,\text{л}}{2\,\text{МПа} \cdot 7\,\text{л}} \approx 1{,}93.
    \end{align*}
}
\solutionspace{80pt}

\tasknumber{5}%
\task{%
    $5\,\text{моль}$ идеального одноатомного газа нагрели на $20\,\text{К}$.
    Определите изменение внутренней энергии газа.
    Увеличилась она или уменьшилась?
    Универсальная газовая постоянная $R = 8{,}31\,\frac{\text{Дж}}{\text{моль}\cdot\text{К}}$.
}
\answer{%
    $
        \Delta U = \frac 32 \nu R \Delta T
            =  \frac 32 \cdot 5\,\text{моль} \cdot 8{,}31\,\frac{\text{Дж}}{\text{моль}\cdot\text{К}} \cdot 20\,\text{К}
            = 1246\,\text{Дж}.
            \text{Увеличилась.}
    $
}
\solutionspace{40pt}

\tasknumber{6}%
\task{%
    Газу сообщили некоторое количество теплоты,
    при этом половину его он потратил на совершение работы,
    одновременно увеличив свою внутреннюю энергию на $3000\,\text{Дж}$.
    Определите количество теплоты, сообщённое газу.
}
\answer{%
    \begin{align*}
    Q &= A' + \Delta U, A' = \frac 12 Q \implies Q \cdot \cbr{1 - \frac 12} = \Delta U \implies Q = \frac{\Delta U}{1 - \frac 12} = \frac{3000\,\text{Дж}}{1 - \frac 12} \approx 6000\,\text{Дж}.
    \\
    A' &= \frac 12 Q
        = \frac 12 \cdot \frac{\Delta U}{1 - \frac 12}
        = \frac{\Delta U}{2 - 1}
        = \frac{3000\,\text{Дж}}{2 - 1} \approx 3000\,\text{Дж}.
    \end{align*}
}
\solutionspace{60pt}

\tasknumber{7}%
\task{%
    В некотором процессе газ совершил работу $200\,\text{Дж}$,
    при этом его внутренняя энергия уменьшилась на $350\,\text{Дж}$.
    Определите количество тепла, переданное при этом процессе газу.
    Явно пропишите, подводили газу тепло или же отводили.
}
\answer{%
    $
        Q = A_\text{газа} + \Delta U, A_\text{газа} = -A_\text{внешняя}
        \implies Q = A_\text{газа} + \Delta U =  200\,\text{Дж} - 350\,\text{Дж} = -150\,\text{Дж}.
        \text{ Отводили.}
    $
}

\variantsplitter

\addpersonalvariant{Наталья Кравченко}

\tasknumber{1}%
\task{%
    Укажите, верны ли утверждения («да» или «нет» слева от каждого утверждения):
    \begin{enumerate}
        \item При изобарном расширении идеальный газ совершает ровно столько работы, сколько внутренней энергии теряет.
        % \item В силу третьего закона Ньютона, совершённая газом работа и работа, совершённая над ним, всегда равны по модулю и противоположны по знаку.
        \item Работу газа в некотором процессе можно вычислять как площадь под графиком в системе координат $PV$, главное лишь правильно расположить оси.
        % \item Дважды два четыре.
        \item При изотермическом процессе внутренняя энергия идеального одноатомного газа не изменяется, даже если ему подводят тепло.
        \item Газ может совершить ненулевую работу в изобарном процессе.
        % \item Адиабатический процесс лишь по воле случая не имеет приставки «изо»: в нём изменяются давление, температура и объём, но это не все макропараметры идеального газа.
        \item Полученное выражение для внутренней энергии идеального газа ($\frac 32 \nu RT$) применимо к трёхатомному газу, при этом, например, уравнение состояния идеального газа применимо независимо от числа атомов в молекулах газа.
    \end{enumerate}
}
\answer{%
    $\text{нет, да, да, да, нет}$
}

\tasknumber{2}%
\task{%
    Определите давление одноатомного идеального газа, занимающего объём $4\,\text{л}$,
    если его внутренняя энергия составляет $400\,\text{Дж}$.
}
\answer{%
    $U = \frac 32 \nu R T = \frac 32 PV \implies P = \frac 23 \cdot \frac UV= \frac 23 \cdot \frac{ 400\,\text{Дж} }{ 4\,\text{л} } \approx 66\,\text{кПа}.$
}
\solutionspace{40pt}

\tasknumber{3}%
\task{%
    Газ расширился от $250\,\text{л}$ до $650\,\text{л}$.
    Давление газа при этом оставалось постоянным и равным $3{,}5\,\text{атм}$.
    Определите работу газа, ответ выразите в килоджоулях.
    $p_{\text{aтм}} = 100\,\text{кПа}$.
}
\answer{%
    $A = P\Delta V = P(V_2 - V_1) = 3{,}5\,\text{атм} \cdot \cbr{650\,\text{л} - 250\,\text{л}} = 140{,}0\,\text{кДж}.$
}
\solutionspace{40pt}

\tasknumber{4}%
\task{%
    Как изменилась внутренняя энергия одноатомного идеального газа при переходе из состояния 1 в состояние 2?
    $P_1 = 2\,\text{МПа}$, $V_1 = 3\,\text{л}$, $P_2 = 3{,}5\,\text{МПа}$, $V_2 = 8\,\text{л}$.
    Как изменилась при этом температура газа?
}
\answer{%
    \begin{align*}
    P_1V_1 &= \nu R T_1, P_2V_2 = \nu R T_2, \\
    \Delta U &= U_2-U_1 = \frac 32 \nu R T_2- \frac 32 \nu R T_1 = \frac 32 P_2 V_2 - \frac 32 P_1 V_1= \frac 32 \cdot \cbr{3{,}5\,\text{МПа} \cdot 8\,\text{л} - 2\,\text{МПа} \cdot 3\,\text{л}} = 33000\,\text{Дж}.
    \\
    \frac{T_2}{T_1} &= \frac{\frac{P_2V_2}{\nu R}}{\frac{P_1V_1}{\nu R}} = \frac{P_2V_2}{P_1V_1}= \frac{3{,}5\,\text{МПа} \cdot 8\,\text{л}}{2\,\text{МПа} \cdot 3\,\text{л}} \approx 4{,}67.
    \end{align*}
}
\solutionspace{80pt}

\tasknumber{5}%
\task{%
    $3\,\text{моль}$ идеального одноатомного газа нагрели на $10\,\text{К}$.
    Определите изменение внутренней энергии газа.
    Увеличилась она или уменьшилась?
    Универсальная газовая постоянная $R = 8{,}31\,\frac{\text{Дж}}{\text{моль}\cdot\text{К}}$.
}
\answer{%
    $
        \Delta U = \frac 32 \nu R \Delta T
            =  \frac 32 \cdot 3\,\text{моль} \cdot 8{,}31\,\frac{\text{Дж}}{\text{моль}\cdot\text{К}} \cdot 10\,\text{К}
            = 373\,\text{Дж}.
            \text{Увеличилась.}
    $
}
\solutionspace{40pt}

\tasknumber{6}%
\task{%
    Газу сообщили некоторое количество теплоты,
    при этом четверть его он потратил на совершение работы,
    одновременно увеличив свою внутреннюю энергию на $1500\,\text{Дж}$.
    Определите количество теплоты, сообщённое газу.
}
\answer{%
    \begin{align*}
    Q &= A' + \Delta U, A' = \frac 14 Q \implies Q \cdot \cbr{1 - \frac 14} = \Delta U \implies Q = \frac{\Delta U}{1 - \frac 14} = \frac{1500\,\text{Дж}}{1 - \frac 14} \approx 2000\,\text{Дж}.
    \\
    A' &= \frac 14 Q
        = \frac 14 \cdot \frac{\Delta U}{1 - \frac 14}
        = \frac{\Delta U}{4 - 1}
        = \frac{1500\,\text{Дж}}{4 - 1} \approx 500\,\text{Дж}.
    \end{align*}
}
\solutionspace{60pt}

\tasknumber{7}%
\task{%
    В некотором процессе внешние силы совершили над газом работу $300\,\text{Дж}$,
    при этом его внутренняя энергия уменьшилась на $350\,\text{Дж}$.
    Определите количество тепла, переданное при этом процессе газу.
    Явно пропишите, подводили газу тепло или же отводили.
}
\answer{%
    $
        Q = A_\text{газа} + \Delta U, A_\text{газа} = -A_\text{внешняя}
        \implies Q = A_\text{газа} + \Delta U = - 300\,\text{Дж} - 350\,\text{Дж} = -650\,\text{Дж}.
        \text{ Отводили.}
    $
}

\variantsplitter

\addpersonalvariant{Матвей Кузьмин}

\tasknumber{1}%
\task{%
    Укажите, верны ли утверждения («да» или «нет» слева от каждого утверждения):
    \begin{enumerate}
        \item При изобарном расширении идеальный газ совершает ровно столько работы, сколько внутренней энергии теряет.
        % \item В силу третьего закона Ньютона, совершённая газом работа и работа, совершённая над ним, всегда равны по модулю и противоположны по знаку.
        \item Работу газа в некотором процессе можно вычислять как площадь под графиком в системе координат $PV$, главное лишь правильно расположить оси.
        % \item Дважды два три.
        \item При изобарном процессе внутренняя энергия идеального одноатомного газа не изменяется, даже если ему подводят тепло.
        \item Газ может совершить ненулевую работу в изобарном процессе.
        % \item Адиабатический процесс лишь по воле случая не имеет приставки «изо»: в нём изменяются давление, температура и объём, но это не все макропараметры идеального газа.
        \item Полученное выражение для внутренней энергии идеального газа ($\frac 32 \nu RT$) применимо к одноатомному газу, при этом, например, уравнение состояния идеального газа применимо независимо от числа атомов в молекулах газа.
    \end{enumerate}
}
\answer{%
    $\text{нет, да, нет, да, да}$
}

\tasknumber{2}%
\task{%
    Определите давление одноатомного идеального газа, занимающего объём $5\,\text{л}$,
    если его внутренняя энергия составляет $400\,\text{Дж}$.
}
\answer{%
    $U = \frac 32 \nu R T = \frac 32 PV \implies P = \frac 23 \cdot \frac UV= \frac 23 \cdot \frac{ 400\,\text{Дж} }{ 5\,\text{л} } \approx 53\,\text{кПа}.$
}
\solutionspace{40pt}

\tasknumber{3}%
\task{%
    Газ расширился от $150\,\text{л}$ до $650\,\text{л}$.
    Давление газа при этом оставалось постоянным и равным $1{,}2\,\text{атм}$.
    Определите работу газа, ответ выразите в килоджоулях.
    $p_{\text{aтм}} = 100\,\text{кПа}$.
}
\answer{%
    $A = P\Delta V = P(V_2 - V_1) = 1{,}2\,\text{атм} \cdot \cbr{650\,\text{л} - 150\,\text{л}} = 60{,}0\,\text{кДж}.$
}
\solutionspace{40pt}

\tasknumber{4}%
\task{%
    Как изменилась внутренняя энергия одноатомного идеального газа при переходе из состояния 1 в состояние 2?
    $P_1 = 2\,\text{МПа}$, $V_1 = 7\,\text{л}$, $P_2 = 3{,}5\,\text{МПа}$, $V_2 = 2\,\text{л}$.
    Как изменилась при этом температура газа?
}
\answer{%
    \begin{align*}
    P_1V_1 &= \nu R T_1, P_2V_2 = \nu R T_2, \\
    \Delta U &= U_2-U_1 = \frac 32 \nu R T_2- \frac 32 \nu R T_1 = \frac 32 P_2 V_2 - \frac 32 P_1 V_1= \frac 32 \cdot \cbr{3{,}5\,\text{МПа} \cdot 2\,\text{л} - 2\,\text{МПа} \cdot 7\,\text{л}} = -10500\,\text{Дж}.
    \\
    \frac{T_2}{T_1} &= \frac{\frac{P_2V_2}{\nu R}}{\frac{P_1V_1}{\nu R}} = \frac{P_2V_2}{P_1V_1}= \frac{3{,}5\,\text{МПа} \cdot 2\,\text{л}}{2\,\text{МПа} \cdot 7\,\text{л}} \approx 0{,}50.
    \end{align*}
}
\solutionspace{80pt}

\tasknumber{5}%
\task{%
    $2\,\text{моль}$ идеального одноатомного газа нагрели на $20\,\text{К}$.
    Определите изменение внутренней энергии газа.
    Увеличилась она или уменьшилась?
    Универсальная газовая постоянная $R = 8{,}31\,\frac{\text{Дж}}{\text{моль}\cdot\text{К}}$.
}
\answer{%
    $
        \Delta U = \frac 32 \nu R \Delta T
            =  \frac 32 \cdot 2\,\text{моль} \cdot 8{,}31\,\frac{\text{Дж}}{\text{моль}\cdot\text{К}} \cdot 20\,\text{К}
            = 498\,\text{Дж}.
            \text{Увеличилась.}
    $
}
\solutionspace{40pt}

\tasknumber{6}%
\task{%
    Газу сообщили некоторое количество теплоты,
    при этом половину его он потратил на совершение работы,
    одновременно увеличив свою внутреннюю энергию на $3000\,\text{Дж}$.
    Определите количество теплоты, сообщённое газу.
}
\answer{%
    \begin{align*}
    Q &= A' + \Delta U, A' = \frac 12 Q \implies Q \cdot \cbr{1 - \frac 12} = \Delta U \implies Q = \frac{\Delta U}{1 - \frac 12} = \frac{3000\,\text{Дж}}{1 - \frac 12} \approx 6000\,\text{Дж}.
    \\
    A' &= \frac 12 Q
        = \frac 12 \cdot \frac{\Delta U}{1 - \frac 12}
        = \frac{\Delta U}{2 - 1}
        = \frac{3000\,\text{Дж}}{2 - 1} \approx 3000\,\text{Дж}.
    \end{align*}
}
\solutionspace{60pt}

\tasknumber{7}%
\task{%
    В некотором процессе внешние силы совершили над газом работу $100\,\text{Дж}$,
    при этом его внутренняя энергия уменьшилась на $150\,\text{Дж}$.
    Определите количество тепла, переданное при этом процессе газу.
    Явно пропишите, подводили газу тепло или же отводили.
}
\answer{%
    $
        Q = A_\text{газа} + \Delta U, A_\text{газа} = -A_\text{внешняя}
        \implies Q = A_\text{газа} + \Delta U = - 100\,\text{Дж} - 150\,\text{Дж} = -250\,\text{Дж}.
        \text{ Отводили.}
    $
}

\variantsplitter

\addpersonalvariant{Сергей Малышев}

\tasknumber{1}%
\task{%
    Укажите, верны ли утверждения («да» или «нет» слева от каждого утверждения):
    \begin{enumerate}
        \item При адиабатическом расширении идеальный газ совершает ровно столько работы, сколько внутренней энергии теряет.
        % \item В силу третьего закона Ньютона, совершённая газом работа и работа, совершённая над ним, всегда равны по модулю и противоположны по знаку.
        \item Работу газа в некотором процессе можно вычислять как площадь под графиком в системе координат $VT$, главное лишь правильно расположить оси.
        % \item Дважды два три.
        \item При изотермическом процессе внутренняя энергия идеального одноатомного газа не изменяется, даже если ему подводят тепло.
        \item Газ может совершить ненулевую работу в изобарном процессе.
        % \item Адиабатический процесс лишь по воле случая не имеет приставки «изо»: в нём изменяются давление, температура и объём, но это не все макропараметры идеального газа.
        \item Полученное выражение для внутренней энергии идеального газа ($\frac 32 \nu RT$) применимо к двухоатомному газу, при этом, например, уравнение состояния идеального газа применимо независимо от числа атомов в молекулах газа.
    \end{enumerate}
}
\answer{%
    $\text{да, нет, да, да, нет}$
}

\tasknumber{2}%
\task{%
    Определите давление одноатомного идеального газа, занимающего объём $3\,\text{л}$,
    если его внутренняя энергия составляет $300\,\text{Дж}$.
}
\answer{%
    $U = \frac 32 \nu R T = \frac 32 PV \implies P = \frac 23 \cdot \frac UV= \frac 23 \cdot \frac{ 300\,\text{Дж} }{ 3\,\text{л} } \approx 66\,\text{кПа}.$
}
\solutionspace{40pt}

\tasknumber{3}%
\task{%
    Газ расширился от $200\,\text{л}$ до $550\,\text{л}$.
    Давление газа при этом оставалось постоянным и равным $1{,}8\,\text{атм}$.
    Определите работу газа, ответ выразите в килоджоулях.
    $p_{\text{aтм}} = 100\,\text{кПа}$.
}
\answer{%
    $A = P\Delta V = P(V_2 - V_1) = 1{,}8\,\text{атм} \cdot \cbr{550\,\text{л} - 200\,\text{л}} = 63{,}0\,\text{кДж}.$
}
\solutionspace{40pt}

\tasknumber{4}%
\task{%
    Как изменилась внутренняя энергия одноатомного идеального газа при переходе из состояния 1 в состояние 2?
    $P_1 = 3\,\text{МПа}$, $V_1 = 5\,\text{л}$, $P_2 = 3{,}5\,\text{МПа}$, $V_2 = 2\,\text{л}$.
    Как изменилась при этом температура газа?
}
\answer{%
    \begin{align*}
    P_1V_1 &= \nu R T_1, P_2V_2 = \nu R T_2, \\
    \Delta U &= U_2-U_1 = \frac 32 \nu R T_2- \frac 32 \nu R T_1 = \frac 32 P_2 V_2 - \frac 32 P_1 V_1= \frac 32 \cdot \cbr{3{,}5\,\text{МПа} \cdot 2\,\text{л} - 3\,\text{МПа} \cdot 5\,\text{л}} = -12000\,\text{Дж}.
    \\
    \frac{T_2}{T_1} &= \frac{\frac{P_2V_2}{\nu R}}{\frac{P_1V_1}{\nu R}} = \frac{P_2V_2}{P_1V_1}= \frac{3{,}5\,\text{МПа} \cdot 2\,\text{л}}{3\,\text{МПа} \cdot 5\,\text{л}} \approx 0{,}47.
    \end{align*}
}
\solutionspace{80pt}

\tasknumber{5}%
\task{%
    $2\,\text{моль}$ идеального одноатомного газа нагрели на $30\,\text{К}$.
    Определите изменение внутренней энергии газа.
    Увеличилась она или уменьшилась?
    Универсальная газовая постоянная $R = 8{,}31\,\frac{\text{Дж}}{\text{моль}\cdot\text{К}}$.
}
\answer{%
    $
        \Delta U = \frac 32 \nu R \Delta T
            =  \frac 32 \cdot 2\,\text{моль} \cdot 8{,}31\,\frac{\text{Дж}}{\text{моль}\cdot\text{К}} \cdot 30\,\text{К}
            = 747\,\text{Дж}.
            \text{Увеличилась.}
    $
}
\solutionspace{40pt}

\tasknumber{6}%
\task{%
    Газу сообщили некоторое количество теплоты,
    при этом треть его он потратил на совершение работы,
    одновременно увеличив свою внутреннюю энергию на $3000\,\text{Дж}$.
    Определите работу, совершённую газом.
}
\answer{%
    \begin{align*}
    Q &= A' + \Delta U, A' = \frac 13 Q \implies Q \cdot \cbr{1 - \frac 13} = \Delta U \implies Q = \frac{\Delta U}{1 - \frac 13} = \frac{3000\,\text{Дж}}{1 - \frac 13} \approx 4500\,\text{Дж}.
    \\
    A' &= \frac 13 Q
        = \frac 13 \cdot \frac{\Delta U}{1 - \frac 13}
        = \frac{\Delta U}{3 - 1}
        = \frac{3000\,\text{Дж}}{3 - 1} \approx 1500\,\text{Дж}.
    \end{align*}
}
\solutionspace{60pt}

\tasknumber{7}%
\task{%
    В некотором процессе газ совершил работу $100\,\text{Дж}$,
    при этом его внутренняя энергия уменьшилась на $150\,\text{Дж}$.
    Определите количество тепла, переданное при этом процессе газу.
    Явно пропишите, подводили газу тепло или же отводили.
}
\answer{%
    $
        Q = A_\text{газа} + \Delta U, A_\text{газа} = -A_\text{внешняя}
        \implies Q = A_\text{газа} + \Delta U =  100\,\text{Дж} - 150\,\text{Дж} = -50\,\text{Дж}.
        \text{ Отводили.}
    $
}

\variantsplitter

\addpersonalvariant{Алина Полканова}

\tasknumber{1}%
\task{%
    Укажите, верны ли утверждения («да» или «нет» слева от каждого утверждения):
    \begin{enumerate}
        \item При адиабатическом расширении идеальный газ совершает ровно столько работы, сколько внутренней энергии теряет.
        % \item В силу третьего закона Ньютона, совершённая газом работа и работа, совершённая над ним, всегда равны по модулю и противоположны по знаку.
        \item Работу газа в некотором процессе можно вычислять как площадь под графиком в системе координат $PT$, главное лишь правильно расположить оси.
        % \item Дважды два четыре.
        \item При изобарном процессе внутренняя энергия идеального одноатомного газа не изменяется, даже если ему подводят тепло.
        \item Газ может совершить ненулевую работу в изобарном процессе.
        % \item Адиабатический процесс лишь по воле случая не имеет приставки «изо»: в нём изменяются давление, температура и объём, но это не все макропараметры идеального газа.
        \item Полученное выражение для внутренней энергии идеального газа ($\frac 32 \nu RT$) применимо к трёхатомному газу, при этом, например, уравнение состояния идеального газа применимо независимо от числа атомов в молекулах газа.
    \end{enumerate}
}
\answer{%
    $\text{да, нет, нет, да, нет}$
}

\tasknumber{2}%
\task{%
    Определите давление одноатомного идеального газа, занимающего объём $5\,\text{л}$,
    если его внутренняя энергия составляет $500\,\text{Дж}$.
}
\answer{%
    $U = \frac 32 \nu R T = \frac 32 PV \implies P = \frac 23 \cdot \frac UV= \frac 23 \cdot \frac{ 500\,\text{Дж} }{ 5\,\text{л} } \approx 66\,\text{кПа}.$
}
\solutionspace{40pt}

\tasknumber{3}%
\task{%
    Газ расширился от $250\,\text{л}$ до $650\,\text{л}$.
    Давление газа при этом оставалось постоянным и равным $3{,}5\,\text{атм}$.
    Определите работу газа, ответ выразите в килоджоулях.
    $p_{\text{aтм}} = 100\,\text{кПа}$.
}
\answer{%
    $A = P\Delta V = P(V_2 - V_1) = 3{,}5\,\text{атм} \cdot \cbr{650\,\text{л} - 250\,\text{л}} = 140{,}0\,\text{кДж}.$
}
\solutionspace{40pt}

\tasknumber{4}%
\task{%
    Как изменилась внутренняя энергия одноатомного идеального газа при переходе из состояния 1 в состояние 2?
    $P_1 = 3\,\text{МПа}$, $V_1 = 3\,\text{л}$, $P_2 = 3{,}5\,\text{МПа}$, $V_2 = 4\,\text{л}$.
    Как изменилась при этом температура газа?
}
\answer{%
    \begin{align*}
    P_1V_1 &= \nu R T_1, P_2V_2 = \nu R T_2, \\
    \Delta U &= U_2-U_1 = \frac 32 \nu R T_2- \frac 32 \nu R T_1 = \frac 32 P_2 V_2 - \frac 32 P_1 V_1= \frac 32 \cdot \cbr{3{,}5\,\text{МПа} \cdot 4\,\text{л} - 3\,\text{МПа} \cdot 3\,\text{л}} = 7500\,\text{Дж}.
    \\
    \frac{T_2}{T_1} &= \frac{\frac{P_2V_2}{\nu R}}{\frac{P_1V_1}{\nu R}} = \frac{P_2V_2}{P_1V_1}= \frac{3{,}5\,\text{МПа} \cdot 4\,\text{л}}{3\,\text{МПа} \cdot 3\,\text{л}} \approx 1{,}56.
    \end{align*}
}
\solutionspace{80pt}

\tasknumber{5}%
\task{%
    $5\,\text{моль}$ идеального одноатомного газа охладили на $10\,\text{К}$.
    Определите изменение внутренней энергии газа.
    Увеличилась она или уменьшилась?
    Универсальная газовая постоянная $R = 8{,}31\,\frac{\text{Дж}}{\text{моль}\cdot\text{К}}$.
}
\answer{%
    $
        \Delta U = \frac 32 \nu R \Delta T
            = - \frac 32 \cdot 5\,\text{моль} \cdot 8{,}31\,\frac{\text{Дж}}{\text{моль}\cdot\text{К}} \cdot 10\,\text{К}
            = -623\,\text{Дж}.
            \text{Уменьшилась.}
    $
}
\solutionspace{40pt}

\tasknumber{6}%
\task{%
    Газу сообщили некоторое количество теплоты,
    при этом четверть его он потратил на совершение работы,
    одновременно увеличив свою внутреннюю энергию на $3000\,\text{Дж}$.
    Определите количество теплоты, сообщённое газу.
}
\answer{%
    \begin{align*}
    Q &= A' + \Delta U, A' = \frac 14 Q \implies Q \cdot \cbr{1 - \frac 14} = \Delta U \implies Q = \frac{\Delta U}{1 - \frac 14} = \frac{3000\,\text{Дж}}{1 - \frac 14} \approx 4000\,\text{Дж}.
    \\
    A' &= \frac 14 Q
        = \frac 14 \cdot \frac{\Delta U}{1 - \frac 14}
        = \frac{\Delta U}{4 - 1}
        = \frac{3000\,\text{Дж}}{4 - 1} \approx 1000\,\text{Дж}.
    \end{align*}
}
\solutionspace{60pt}

\tasknumber{7}%
\task{%
    В некотором процессе внешние силы совершили над газом работу $300\,\text{Дж}$,
    при этом его внутренняя энергия увеличилась на $350\,\text{Дж}$.
    Определите количество тепла, переданное при этом процессе газу.
    Явно пропишите, подводили газу тепло или же отводили.
}
\answer{%
    $
        Q = A_\text{газа} + \Delta U, A_\text{газа} = -A_\text{внешняя}
        \implies Q = A_\text{газа} + \Delta U = - 300\,\text{Дж} +  350\,\text{Дж} = 50\,\text{Дж}.
        \text{ Подводили.}
    $
}

\variantsplitter

\addpersonalvariant{Сергей Пономарёв}

\tasknumber{1}%
\task{%
    Укажите, верны ли утверждения («да» или «нет» слева от каждого утверждения):
    \begin{enumerate}
        \item При изобарном расширении идеальный газ совершает ровно столько работы, сколько внутренней энергии теряет.
        % \item В силу третьего закона Ньютона, совершённая газом работа и работа, совершённая над ним, всегда равны по модулю и противоположны по знаку.
        \item Работу газа в некотором процессе можно вычислять как площадь под графиком в системе координат $VT$, главное лишь правильно расположить оси.
        % \item Дважды два три.
        \item При изобарном процессе внутренняя энергия идеального одноатомного газа не изменяется, даже если ему подводят тепло.
        \item Газ может совершить ненулевую работу в изобарном процессе.
        % \item Адиабатический процесс лишь по воле случая не имеет приставки «изо»: в нём изменяются давление, температура и объём, но это не все макропараметры идеального газа.
        \item Полученное выражение для внутренней энергии идеального газа ($\frac 32 \nu RT$) применимо к одноатомному газу, при этом, например, уравнение состояния идеального газа применимо независимо от числа атомов в молекулах газа.
    \end{enumerate}
}
\answer{%
    $\text{нет, нет, нет, да, да}$
}

\tasknumber{2}%
\task{%
    Определите давление одноатомного идеального газа, занимающего объём $4\,\text{л}$,
    если его внутренняя энергия составляет $250\,\text{Дж}$.
}
\answer{%
    $U = \frac 32 \nu R T = \frac 32 PV \implies P = \frac 23 \cdot \frac UV= \frac 23 \cdot \frac{ 250\,\text{Дж} }{ 4\,\text{л} } \approx 41\,\text{кПа}.$
}
\solutionspace{40pt}

\tasknumber{3}%
\task{%
    Газ расширился от $250\,\text{л}$ до $550\,\text{л}$.
    Давление газа при этом оставалось постоянным и равным $1{,}8\,\text{атм}$.
    Определите работу газа, ответ выразите в килоджоулях.
    $p_{\text{aтм}} = 100\,\text{кПа}$.
}
\answer{%
    $A = P\Delta V = P(V_2 - V_1) = 1{,}8\,\text{атм} \cdot \cbr{550\,\text{л} - 250\,\text{л}} = 54{,}0\,\text{кДж}.$
}
\solutionspace{40pt}

\tasknumber{4}%
\task{%
    Как изменилась внутренняя энергия одноатомного идеального газа при переходе из состояния 1 в состояние 2?
    $P_1 = 3\,\text{МПа}$, $V_1 = 5\,\text{л}$, $P_2 = 2{,}5\,\text{МПа}$, $V_2 = 4\,\text{л}$.
    Как изменилась при этом температура газа?
}
\answer{%
    \begin{align*}
    P_1V_1 &= \nu R T_1, P_2V_2 = \nu R T_2, \\
    \Delta U &= U_2-U_1 = \frac 32 \nu R T_2- \frac 32 \nu R T_1 = \frac 32 P_2 V_2 - \frac 32 P_1 V_1= \frac 32 \cdot \cbr{2{,}5\,\text{МПа} \cdot 4\,\text{л} - 3\,\text{МПа} \cdot 5\,\text{л}} = -7500\,\text{Дж}.
    \\
    \frac{T_2}{T_1} &= \frac{\frac{P_2V_2}{\nu R}}{\frac{P_1V_1}{\nu R}} = \frac{P_2V_2}{P_1V_1}= \frac{2{,}5\,\text{МПа} \cdot 4\,\text{л}}{3\,\text{МПа} \cdot 5\,\text{л}} \approx 0{,}67.
    \end{align*}
}
\solutionspace{80pt}

\tasknumber{5}%
\task{%
    $3\,\text{моль}$ идеального одноатомного газа охладили на $20\,\text{К}$.
    Определите изменение внутренней энергии газа.
    Увеличилась она или уменьшилась?
    Универсальная газовая постоянная $R = 8{,}31\,\frac{\text{Дж}}{\text{моль}\cdot\text{К}}$.
}
\answer{%
    $
        \Delta U = \frac 32 \nu R \Delta T
            = - \frac 32 \cdot 3\,\text{моль} \cdot 8{,}31\,\frac{\text{Дж}}{\text{моль}\cdot\text{К}} \cdot 20\,\text{К}
            = -747\,\text{Дж}.
            \text{Уменьшилась.}
    $
}
\solutionspace{40pt}

\tasknumber{6}%
\task{%
    Газу сообщили некоторое количество теплоты,
    при этом четверть его он потратил на совершение работы,
    одновременно увеличив свою внутреннюю энергию на $1200\,\text{Дж}$.
    Определите количество теплоты, сообщённое газу.
}
\answer{%
    \begin{align*}
    Q &= A' + \Delta U, A' = \frac 14 Q \implies Q \cdot \cbr{1 - \frac 14} = \Delta U \implies Q = \frac{\Delta U}{1 - \frac 14} = \frac{1200\,\text{Дж}}{1 - \frac 14} \approx 1600\,\text{Дж}.
    \\
    A' &= \frac 14 Q
        = \frac 14 \cdot \frac{\Delta U}{1 - \frac 14}
        = \frac{\Delta U}{4 - 1}
        = \frac{1200\,\text{Дж}}{4 - 1} \approx 400\,\text{Дж}.
    \end{align*}
}
\solutionspace{60pt}

\tasknumber{7}%
\task{%
    В некотором процессе внешние силы совершили над газом работу $200\,\text{Дж}$,
    при этом его внутренняя энергия уменьшилась на $450\,\text{Дж}$.
    Определите количество тепла, переданное при этом процессе газу.
    Явно пропишите, подводили газу тепло или же отводили.
}
\answer{%
    $
        Q = A_\text{газа} + \Delta U, A_\text{газа} = -A_\text{внешняя}
        \implies Q = A_\text{газа} + \Delta U = - 200\,\text{Дж} - 450\,\text{Дж} = -650\,\text{Дж}.
        \text{ Отводили.}
    $
}

\variantsplitter

\addpersonalvariant{Егор Свистушкин}

\tasknumber{1}%
\task{%
    Укажите, верны ли утверждения («да» или «нет» слева от каждого утверждения):
    \begin{enumerate}
        \item При изобарном расширении идеальный газ совершает ровно столько работы, сколько внутренней энергии теряет.
        % \item В силу третьего закона Ньютона, совершённая газом работа и работа, совершённая над ним, всегда равны по модулю и противоположны по знаку.
        \item Работу газа в некотором процессе можно вычислять как площадь под графиком в системе координат $PV$, главное лишь правильно расположить оси.
        % \item Дважды два пять.
        \item При изобарном процессе внутренняя энергия идеального одноатомного газа не изменяется, даже если ему подводят тепло.
        \item Газ может совершить ненулевую работу в изобарном процессе.
        % \item Адиабатический процесс лишь по воле случая не имеет приставки «изо»: в нём изменяются давление, температура и объём, но это не все макропараметры идеального газа.
        \item Полученное выражение для внутренней энергии идеального газа ($\frac 32 \nu RT$) применимо к одноатомному газу, при этом, например, уравнение состояния идеального газа применимо независимо от числа атомов в молекулах газа.
    \end{enumerate}
}
\answer{%
    $\text{нет, да, нет, да, да}$
}

\tasknumber{2}%
\task{%
    Определите давление одноатомного идеального газа, занимающего объём $5\,\text{л}$,
    если его внутренняя энергия составляет $250\,\text{Дж}$.
}
\answer{%
    $U = \frac 32 \nu R T = \frac 32 PV \implies P = \frac 23 \cdot \frac UV= \frac 23 \cdot \frac{ 250\,\text{Дж} }{ 5\,\text{л} } \approx 33\,\text{кПа}.$
}
\solutionspace{40pt}

\tasknumber{3}%
\task{%
    Газ расширился от $250\,\text{л}$ до $550\,\text{л}$.
    Давление газа при этом оставалось постоянным и равным $3{,}5\,\text{атм}$.
    Определите работу газа, ответ выразите в килоджоулях.
    $p_{\text{aтм}} = 100\,\text{кПа}$.
}
\answer{%
    $A = P\Delta V = P(V_2 - V_1) = 3{,}5\,\text{атм} \cdot \cbr{550\,\text{л} - 250\,\text{л}} = 105{,}0\,\text{кДж}.$
}
\solutionspace{40pt}

\tasknumber{4}%
\task{%
    Как изменилась внутренняя энергия одноатомного идеального газа при переходе из состояния 1 в состояние 2?
    $P_1 = 2\,\text{МПа}$, $V_1 = 7\,\text{л}$, $P_2 = 2{,}5\,\text{МПа}$, $V_2 = 2\,\text{л}$.
    Как изменилась при этом температура газа?
}
\answer{%
    \begin{align*}
    P_1V_1 &= \nu R T_1, P_2V_2 = \nu R T_2, \\
    \Delta U &= U_2-U_1 = \frac 32 \nu R T_2- \frac 32 \nu R T_1 = \frac 32 P_2 V_2 - \frac 32 P_1 V_1= \frac 32 \cdot \cbr{2{,}5\,\text{МПа} \cdot 2\,\text{л} - 2\,\text{МПа} \cdot 7\,\text{л}} = -13500\,\text{Дж}.
    \\
    \frac{T_2}{T_1} &= \frac{\frac{P_2V_2}{\nu R}}{\frac{P_1V_1}{\nu R}} = \frac{P_2V_2}{P_1V_1}= \frac{2{,}5\,\text{МПа} \cdot 2\,\text{л}}{2\,\text{МПа} \cdot 7\,\text{л}} \approx 0{,}36.
    \end{align*}
}
\solutionspace{80pt}

\tasknumber{5}%
\task{%
    $5\,\text{моль}$ идеального одноатомного газа охладили на $10\,\text{К}$.
    Определите изменение внутренней энергии газа.
    Увеличилась она или уменьшилась?
    Универсальная газовая постоянная $R = 8{,}31\,\frac{\text{Дж}}{\text{моль}\cdot\text{К}}$.
}
\answer{%
    $
        \Delta U = \frac 32 \nu R \Delta T
            = - \frac 32 \cdot 5\,\text{моль} \cdot 8{,}31\,\frac{\text{Дж}}{\text{моль}\cdot\text{К}} \cdot 10\,\text{К}
            = -623\,\text{Дж}.
            \text{Уменьшилась.}
    $
}
\solutionspace{40pt}

\tasknumber{6}%
\task{%
    Газу сообщили некоторое количество теплоты,
    при этом треть его он потратил на совершение работы,
    одновременно увеличив свою внутреннюю энергию на $1200\,\text{Дж}$.
    Определите количество теплоты, сообщённое газу.
}
\answer{%
    \begin{align*}
    Q &= A' + \Delta U, A' = \frac 13 Q \implies Q \cdot \cbr{1 - \frac 13} = \Delta U \implies Q = \frac{\Delta U}{1 - \frac 13} = \frac{1200\,\text{Дж}}{1 - \frac 13} \approx 1800\,\text{Дж}.
    \\
    A' &= \frac 13 Q
        = \frac 13 \cdot \frac{\Delta U}{1 - \frac 13}
        = \frac{\Delta U}{3 - 1}
        = \frac{1200\,\text{Дж}}{3 - 1} \approx 600\,\text{Дж}.
    \end{align*}
}
\solutionspace{60pt}

\tasknumber{7}%
\task{%
    В некотором процессе внешние силы совершили над газом работу $200\,\text{Дж}$,
    при этом его внутренняя энергия уменьшилась на $450\,\text{Дж}$.
    Определите количество тепла, переданное при этом процессе газу.
    Явно пропишите, подводили газу тепло или же отводили.
}
\answer{%
    $
        Q = A_\text{газа} + \Delta U, A_\text{газа} = -A_\text{внешняя}
        \implies Q = A_\text{газа} + \Delta U = - 200\,\text{Дж} - 450\,\text{Дж} = -650\,\text{Дж}.
        \text{ Отводили.}
    $
}

\variantsplitter

\addpersonalvariant{Дмитрий Соколов}

\tasknumber{1}%
\task{%
    Укажите, верны ли утверждения («да» или «нет» слева от каждого утверждения):
    \begin{enumerate}
        \item При изобарном расширении идеальный газ совершает ровно столько работы, сколько внутренней энергии теряет.
        % \item В силу третьего закона Ньютона, совершённая газом работа и работа, совершённая над ним, всегда равны по модулю и противоположны по знаку.
        \item Работу газа в некотором процессе можно вычислять как площадь под графиком в системе координат $VT$, главное лишь правильно расположить оси.
        % \item Дважды два пять.
        \item При изохорном процессе внутренняя энергия идеального одноатомного газа не изменяется, даже если ему подводят тепло.
        \item Газ может совершить ненулевую работу в изохорном процессе.
        % \item Адиабатический процесс лишь по воле случая не имеет приставки «изо»: в нём изменяются давление, температура и объём, но это не все макропараметры идеального газа.
        \item Полученное выражение для внутренней энергии идеального газа ($\frac 32 \nu RT$) применимо к одноатомному газу, при этом, например, уравнение состояния идеального газа применимо независимо от числа атомов в молекулах газа.
    \end{enumerate}
}
\answer{%
    $\text{нет, нет, нет, нет, да}$
}

\tasknumber{2}%
\task{%
    Определите давление одноатомного идеального газа, занимающего объём $3\,\text{л}$,
    если его внутренняя энергия составляет $250\,\text{Дж}$.
}
\answer{%
    $U = \frac 32 \nu R T = \frac 32 PV \implies P = \frac 23 \cdot \frac UV= \frac 23 \cdot \frac{ 250\,\text{Дж} }{ 3\,\text{л} } \approx 55\,\text{кПа}.$
}
\solutionspace{40pt}

\tasknumber{3}%
\task{%
    Газ расширился от $150\,\text{л}$ до $450\,\text{л}$.
    Давление газа при этом оставалось постоянным и равным $3{,}5\,\text{атм}$.
    Определите работу газа, ответ выразите в килоджоулях.
    $p_{\text{aтм}} = 100\,\text{кПа}$.
}
\answer{%
    $A = P\Delta V = P(V_2 - V_1) = 3{,}5\,\text{атм} \cdot \cbr{450\,\text{л} - 150\,\text{л}} = 105{,}0\,\text{кДж}.$
}
\solutionspace{40pt}

\tasknumber{4}%
\task{%
    Как изменилась внутренняя энергия одноатомного идеального газа при переходе из состояния 1 в состояние 2?
    $P_1 = 4\,\text{МПа}$, $V_1 = 7\,\text{л}$, $P_2 = 4{,}5\,\text{МПа}$, $V_2 = 4\,\text{л}$.
    Как изменилась при этом температура газа?
}
\answer{%
    \begin{align*}
    P_1V_1 &= \nu R T_1, P_2V_2 = \nu R T_2, \\
    \Delta U &= U_2-U_1 = \frac 32 \nu R T_2- \frac 32 \nu R T_1 = \frac 32 P_2 V_2 - \frac 32 P_1 V_1= \frac 32 \cdot \cbr{4{,}5\,\text{МПа} \cdot 4\,\text{л} - 4\,\text{МПа} \cdot 7\,\text{л}} = -15000\,\text{Дж}.
    \\
    \frac{T_2}{T_1} &= \frac{\frac{P_2V_2}{\nu R}}{\frac{P_1V_1}{\nu R}} = \frac{P_2V_2}{P_1V_1}= \frac{4{,}5\,\text{МПа} \cdot 4\,\text{л}}{4\,\text{МПа} \cdot 7\,\text{л}} \approx 0{,}64.
    \end{align*}
}
\solutionspace{80pt}

\tasknumber{5}%
\task{%
    $5\,\text{моль}$ идеального одноатомного газа нагрели на $20\,\text{К}$.
    Определите изменение внутренней энергии газа.
    Увеличилась она или уменьшилась?
    Универсальная газовая постоянная $R = 8{,}31\,\frac{\text{Дж}}{\text{моль}\cdot\text{К}}$.
}
\answer{%
    $
        \Delta U = \frac 32 \nu R \Delta T
            =  \frac 32 \cdot 5\,\text{моль} \cdot 8{,}31\,\frac{\text{Дж}}{\text{моль}\cdot\text{К}} \cdot 20\,\text{К}
            = 1246\,\text{Дж}.
            \text{Увеличилась.}
    $
}
\solutionspace{40pt}

\tasknumber{6}%
\task{%
    Газу сообщили некоторое количество теплоты,
    при этом четверть его он потратил на совершение работы,
    одновременно увеличив свою внутреннюю энергию на $1200\,\text{Дж}$.
    Определите работу, совершённую газом.
}
\answer{%
    \begin{align*}
    Q &= A' + \Delta U, A' = \frac 14 Q \implies Q \cdot \cbr{1 - \frac 14} = \Delta U \implies Q = \frac{\Delta U}{1 - \frac 14} = \frac{1200\,\text{Дж}}{1 - \frac 14} \approx 1600\,\text{Дж}.
    \\
    A' &= \frac 14 Q
        = \frac 14 \cdot \frac{\Delta U}{1 - \frac 14}
        = \frac{\Delta U}{4 - 1}
        = \frac{1200\,\text{Дж}}{4 - 1} \approx 400\,\text{Дж}.
    \end{align*}
}
\solutionspace{60pt}

\tasknumber{7}%
\task{%
    В некотором процессе газ совершил работу $300\,\text{Дж}$,
    при этом его внутренняя энергия увеличилась на $350\,\text{Дж}$.
    Определите количество тепла, переданное при этом процессе газу.
    Явно пропишите, подводили газу тепло или же отводили.
}
\answer{%
    $
        Q = A_\text{газа} + \Delta U, A_\text{газа} = -A_\text{внешняя}
        \implies Q = A_\text{газа} + \Delta U =  300\,\text{Дж} +  350\,\text{Дж} = 650\,\text{Дж}.
        \text{ Подводили.}
    $
}

\variantsplitter

\addpersonalvariant{Арсений Трофимов}

\tasknumber{1}%
\task{%
    Укажите, верны ли утверждения («да» или «нет» слева от каждого утверждения):
    \begin{enumerate}
        \item При адиабатическом расширении идеальный газ совершает ровно столько работы, сколько внутренней энергии теряет.
        % \item В силу третьего закона Ньютона, совершённая газом работа и работа, совершённая над ним, всегда равны по модулю и противоположны по знаку.
        \item Работу газа в некотором процессе можно вычислять как площадь под графиком в системе координат $PV$, главное лишь правильно расположить оси.
        % \item Дважды два четыре.
        \item При изотермическом процессе внутренняя энергия идеального одноатомного газа не изменяется, даже если ему подводят тепло.
        \item Газ может совершить ненулевую работу в изохорном процессе.
        % \item Адиабатический процесс лишь по воле случая не имеет приставки «изо»: в нём изменяются давление, температура и объём, но это не все макропараметры идеального газа.
        \item Полученное выражение для внутренней энергии идеального газа ($\frac 32 \nu RT$) применимо к двухоатомному газу, при этом, например, уравнение состояния идеального газа применимо независимо от числа атомов в молекулах газа.
    \end{enumerate}
}
\answer{%
    $\text{да, да, да, нет, нет}$
}

\tasknumber{2}%
\task{%
    Определите давление одноатомного идеального газа, занимающего объём $5\,\text{л}$,
    если его внутренняя энергия составляет $300\,\text{Дж}$.
}
\answer{%
    $U = \frac 32 \nu R T = \frac 32 PV \implies P = \frac 23 \cdot \frac UV= \frac 23 \cdot \frac{ 300\,\text{Дж} }{ 5\,\text{л} } \approx 40\,\text{кПа}.$
}
\solutionspace{40pt}

\tasknumber{3}%
\task{%
    Газ расширился от $350\,\text{л}$ до $450\,\text{л}$.
    Давление газа при этом оставалось постоянным и равным $1{,}5\,\text{атм}$.
    Определите работу газа, ответ выразите в килоджоулях.
    $p_{\text{aтм}} = 100\,\text{кПа}$.
}
\answer{%
    $A = P\Delta V = P(V_2 - V_1) = 1{,}5\,\text{атм} \cdot \cbr{450\,\text{л} - 350\,\text{л}} = 15{,}0\,\text{кДж}.$
}
\solutionspace{40pt}

\tasknumber{4}%
\task{%
    Как изменилась внутренняя энергия одноатомного идеального газа при переходе из состояния 1 в состояние 2?
    $P_1 = 4\,\text{МПа}$, $V_1 = 5\,\text{л}$, $P_2 = 4{,}5\,\text{МПа}$, $V_2 = 6\,\text{л}$.
    Как изменилась при этом температура газа?
}
\answer{%
    \begin{align*}
    P_1V_1 &= \nu R T_1, P_2V_2 = \nu R T_2, \\
    \Delta U &= U_2-U_1 = \frac 32 \nu R T_2- \frac 32 \nu R T_1 = \frac 32 P_2 V_2 - \frac 32 P_1 V_1= \frac 32 \cdot \cbr{4{,}5\,\text{МПа} \cdot 6\,\text{л} - 4\,\text{МПа} \cdot 5\,\text{л}} = 10500\,\text{Дж}.
    \\
    \frac{T_2}{T_1} &= \frac{\frac{P_2V_2}{\nu R}}{\frac{P_1V_1}{\nu R}} = \frac{P_2V_2}{P_1V_1}= \frac{4{,}5\,\text{МПа} \cdot 6\,\text{л}}{4\,\text{МПа} \cdot 5\,\text{л}} \approx 1{,}35.
    \end{align*}
}
\solutionspace{80pt}

\tasknumber{5}%
\task{%
    $2\,\text{моль}$ идеального одноатомного газа нагрели на $10\,\text{К}$.
    Определите изменение внутренней энергии газа.
    Увеличилась она или уменьшилась?
    Универсальная газовая постоянная $R = 8{,}31\,\frac{\text{Дж}}{\text{моль}\cdot\text{К}}$.
}
\answer{%
    $
        \Delta U = \frac 32 \nu R \Delta T
            =  \frac 32 \cdot 2\,\text{моль} \cdot 8{,}31\,\frac{\text{Дж}}{\text{моль}\cdot\text{К}} \cdot 10\,\text{К}
            = 249\,\text{Дж}.
            \text{Увеличилась.}
    $
}
\solutionspace{40pt}

\tasknumber{6}%
\task{%
    Газу сообщили некоторое количество теплоты,
    при этом четверть его он потратил на совершение работы,
    одновременно увеличив свою внутреннюю энергию на $3000\,\text{Дж}$.
    Определите количество теплоты, сообщённое газу.
}
\answer{%
    \begin{align*}
    Q &= A' + \Delta U, A' = \frac 14 Q \implies Q \cdot \cbr{1 - \frac 14} = \Delta U \implies Q = \frac{\Delta U}{1 - \frac 14} = \frac{3000\,\text{Дж}}{1 - \frac 14} \approx 4000\,\text{Дж}.
    \\
    A' &= \frac 14 Q
        = \frac 14 \cdot \frac{\Delta U}{1 - \frac 14}
        = \frac{\Delta U}{4 - 1}
        = \frac{3000\,\text{Дж}}{4 - 1} \approx 1000\,\text{Дж}.
    \end{align*}
}
\solutionspace{60pt}

\tasknumber{7}%
\task{%
    В некотором процессе внешние силы совершили над газом работу $300\,\text{Дж}$,
    при этом его внутренняя энергия уменьшилась на $350\,\text{Дж}$.
    Определите количество тепла, переданное при этом процессе газу.
    Явно пропишите, подводили газу тепло или же отводили.
}
\answer{%
    $
        Q = A_\text{газа} + \Delta U, A_\text{газа} = -A_\text{внешняя}
        \implies Q = A_\text{газа} + \Delta U = - 300\,\text{Дж} - 350\,\text{Дж} = -650\,\text{Дж}.
        \text{ Отводили.}
    $
}
% autogenerated
