\documentclass[12pt,a4paper]{amsart}%DVI-mode.
\usepackage{graphics,graphicx,epsfig}%DVI-mode.
%\documentclass[pdftex,12pt]{amsart} %PDF-mode.
%\usepackage[pdftex]{graphicx}       %PDF-mode.

%\usepackage{a4wide}                 % Fit the text to A4 page tightly.
\usepackage[utf8]{inputenc}
\usepackage[T2A]{fontenc}
\usepackage[english,russian]{babel} % Download Russian fonts.
\usepackage{amsmath,amsfonts,amssymb,amsthm,amscd,mathrsfs} % Use AMS symbols.
\usepackage{tikz}
\usetikzlibrary{circuits.ee.IEC}
\usetikzlibrary{shapes.geometric}
\usetikzlibrary{decorations.markings}
%\usetikzlibrary{dashs}
%\usetikzlibrary{info}


\textheight=29cm % высота текста
\textwidth=18cm % ширина текста
\topmargin=-2.5cm % отступ от верхнего края
\parskip=6pt % интервал между абзацами
\oddsidemargin=-1.5cm
\evensidemargin=-1.5cm 

% wide docs
% \oddsidemargin=0cm
% \evensidemargin=0cm 
% \textheight=29cm % высота текста
% \textwidth=15cm % ширина текста
% \topmargin=-1.5cm % отступ от верхнего края
% \parskip=18pt % интервал между абзацами


\parindent=0pt % абзацный отступ
\tolerance=500 % терпимость к "жидким" строкам
\binoppenalty=10000 % штраф за перенос формул - 10000 - абсолютный запрет
\relpenalty=10000
\flushbottom % выравнивание высоты страниц
\def\baselinestretch{1.00}
\pagenumbering{gobble}

\begin{document}
\newcommand\bivec[2]{\begin{pmatrix} #1 \\ #2 \end{pmatrix}}

\newcommand\ol[1]{\overline{#1}}

\newcommand\p[1]{\ensuremath{\Prob\!\left(#1\right)}}
\def\cond{\,|\,}
\newcommand\e[1]{\mathsf{E}\!\left(#1\right)}
\newcommand\disp[1]{\mathsf{D}\!\left(#1\right)}
%\newcommand\norm[2]{\mathcal{N}\!\cbr{#1,#2}}
\newcommand\sign{\text{ sign }}

\newcommand\al[1]{\begin{align*} #1 \end{align*}}
\newcommand\begcas[1]{\begin{cases}#1\end{cases}}
\newcommand\tab[2]{	\vspace{-#1pt}
						\begin{tabbing} 
						#2
						\end{tabbing}
					\vspace{-#1pt}
					}


\newcommand\maintext[1]{{\bfseries\sffamily{#1}}}
\newcommand\simpletitle[1]{\begin{center} \maintext{#1} \end{center}}

\def\le{\leqslant}
\def\ge{\geqslant}
\def\Ell{\mathcal{L}}
\def\eps{\varepsilon}
\def\x{\ensuremath{\textbf{x}}}
\def\y{\ensuremath{\textbf{y}}}
\def\Rn{\ensuremath{\mathbb{R}^n}}
\def\RSS{\mathsf{RSS}}

\newcommand\mb[1]{\ensuremath{\boldsymbol{\mathbf{#1}}}}
\newcommand\argmax[1]{\arg\underset{#1}\max\,} % \operatornamewithlimits
%\newcommand{\prodl}{\mathop{\textstyle\prod}\limits}
\newcommand{\prodl}{\prod\limits}
\newcommand{\suml}{\sum\limits}
\newcommand\foral[1]{\forall\,#1\:}
\newcommand\exist[1]{\exists\,#1\:\colon}

\newcommand\cbr[1]{\left(#1\right)} %circled brackets
\newcommand\fbr[1]{\left\{#1\right\}} %figure brackets
\newcommand\sbr[1]{\left[#1\right]} %square brackets
\newcommand\modul[1]{\left|#1\right|}
\newcommand\cdf[2]{\cdot\frac{#1}{#2}}
\newcommand\integr[3]{\int\limits_{#1}^{#2}{#3}}
\newcommand\obol[1]{O\!\cbr{#1}}
\newcommand\norm[1]{\ensuremath{\left\|{#1}\right\|}}

\newcommand\dd[2]{\frac{\partial#1}{\partial#2}}

\newcommand\addeps[2]{
	\begin{figure} [!ht] %lrp
		\centering
		\includegraphics[height=240px]{#1.eps}
		\vspace{-10pt}
		\caption{#2}
		\label{eps:#1}
	\end{figure}
}

\newcommand\addtikz[4]{
	\begin{figure} [!ht] %lrp
		\centering
		\begin{tikzpicture}[x=#2cm,y=#2cm,#3]
			\input{#1.tikz}
		\end{tikzpicture}
		\vspace{-10pt}
		\caption{#4}
		\label{tikz:#1}	
	\end{figure}
}



\newcommand\addepssize[3]{
	\begin{figure} [!ht] %lrp hp
		\centering
		\includegraphics[height=#3px]{#1.eps}
		\vspace{-10pt}
		\caption{#2}
		\label{eps:#1}
	\end{figure}
}

\def\algorithmicrequire{\textbf{Вход:}}
\def\algorithmicensure{\textbf{Выход:}}
\def\algorithmicif{\textbf{если}}
\def\algorithmicthen{\textbf{то}}
\def\algorithmicelse{\textbf{иначе}}
\def\algorithmicelsif{\textbf{иначе если}}
\def\algorithmicfor{\textbf{для}}
\def\algorithmicforall{\textbf{для всех}}
\def\algorithmicdo{}
\def\algorithmicwhile{\textbf{пока}}
\def\algorithmicrepeat{\textbf{повторять}}
\def\algorithmicuntil{\textbf{пока}}
\def\algorithmicloop{\textbf{цикл}}
% переопределение стиля комментариев
\def\algorithmiccomment#1{\quad// {\sl #1}}
%\raggedright
\classdate{7}{20 апреля 2018}

\task 1
Площадь большого поршня гидравлического домкрата $S_1 = 20\units{см}^2$, а малого $S_2 = 0{,}5\units{см}^2.$ Груз какой максимальной массы можно поднять этим домкратом, если на малый поршень давить с силой не более $F=200\units{Н}?$ Силой трения от стенки цилиндров пренебречь.

\task 2
В сосуд налита вода. Расстояние от поверхности воды до дна $H = 0{,}5\units{м},$ площадь дна $S = 0{,}1\units{м}^2.$ Найти гидростатическое давление $P_1$ и полное давление $P_2$ вблизи дна. Найти силу давления воды на дно. Плотность воды \rhowater

\task 3
На лёгкий поршень площадью $S=900\units{см}^2,$ касающийся поверхности воды, поставили гирю массы $m=3\units{кг}$. Высота слоя воды в сосуде с вертикальными стенками $H = 20\units{см}$. Определить давление жидкости вблизи дна, если плотность воды \rhowater

\task 4
Давление газов в конце сгорания в цилиндре дизельного двигателя трактора $P = 9\units{МПа}.$ Диаметр цилиндра $d = 130\units{мм}.$ С какой силой газы давят на поршень в цилиндре? Площадь круга диаметром $D$ равна $S = \cfrac{\pi D^2}4.$

\task 5
Площадь малого поршня гидравлического подъёмника $S_1 = 0{,}8\units{см}^2$, а большого $S_2 = 40\units{см}^2.$ Какую силу $F$ надо приложить к малому поршню, чтобы поднять груз весом $P = 8\units{кН}?$

\task 6
Герметичный сосуд полностью заполнен водой и стоит на столе. На небольшой поршень площадью $S$ давят рукой с силой $F$. Поршень находится ниже крышки сосуда на $H_1$, выше дна на $H_2$ и может свободно перемещаться. Плотность воды $\rho$, атмосферное давление $P_A$. Найти давления $P_1$ и $P_2$ в воде вблизи крышки и дна сосуда.
\\ \\
\classdate{7}{20 апреля 2018}

\task 1
Площадь большого поршня гидравлического домкрата $S_1 = 20\units{см}^2$, а малого $S_2 = 0{,}5\units{см}^2.$ Груз какой максимальной массы можно поднять этим домкратом, если на малый поршень давить с силой не более $F=200\units{Н}?$ Силой трения от стенки цилиндров пренебречь.

\task 2
В сосуд налита вода. Расстояние от поверхности воды до дна $H = 0{,}5\units{м},$ площадь дна $S = 0{,}1\units{м}^2.$ Найти гидростатическое давление $P_1$ и полное давление $P_2$ вблизи дна. Найти силу давления воды на дно. Плотность воды \rhowater

\task 3
На лёгкий поршень площадью $S=900\units{см}^2,$ касающийся поверхности воды, поставили гирю массы $m=3\units{кг}$. Высота слоя воды в сосуде с вертикальными стенками $H = 20\units{см}$. Определить давление жидкости вблизи дна, если плотность воды \rhowater

\task 4
Давление газов в конце сгорания в цилиндре дизельного двигателя трактора $P = 9\units{МПа}.$ Диаметр цилиндра $d = 130\units{мм}.$ С какой силой газы давят на поршень в цилиндре? Площадь круга диаметром $D$ равна $S = \cfrac{\pi D^2}4.$

\task 5
Площадь малого поршня гидравлического подъёмника $S_1 = 0{,}8\units{см}^2$, а большого $S_2 = 40\units{см}^2.$ Какую силу $F$ надо приложить к малому поршню, чтобы поднять груз весом $P = 8\units{кН}?$

\task 6
Герметичный сосуд полностью заполнен водой и стоит на столе. На небольшой поршень площадью $S$ давят рукой с силой $F$. Поршень находится ниже крышки сосуда на $H_1$, выше дна на $H_2$ и может свободно перемещаться. Плотность воды $\rho$, атмосферное давление $P_A$. Найти давления $P_1$ и $P_2$ в воде вблизи крышки и дна сосуда.

\newpage

\adddate{8 класс. 20 апреля 2018}

\task 1
Между точками $A$ и $B$ электрической цепи подключены последовательно резисторы $R_1 = 10\units{Ом}$ и $R_2 = 20\units{Ом}$ и параллельно им $R_3 = 30\units{Ом}.$ Найдите эквивалентное сопротивление $R_{AB}$ этого участка цепи.

\task 2
Электрическая цепь состоит из последовательности $N$ одинаковых звеньев, в которых каждый резистор имеет сопротивление $r$. Последнее звено замкнуто резистором сопротивлением $R$. При каком соотношении $\cfrac{R}{r}$ сопротивление цепи не зависит от числа звеньев?

\task 3
Для измерения сопротивления $R$ проводника собрана электрическая цепь. Вольтметр $V$ показывает напряжение $U_V = 5\units{В},$ показание амперметра $A$ равно $I_A = 25\units{мА}.$ Найдите величину $R$ сопротивления проводника. Внутреннее сопротивление вольтметра $R_V = 1{,}0\units{кОм},$ внутреннее сопротивление амперметра $R_A = 2{,}0\units{Ом}.$

\task 4
Шкала гальванометра имеет $N=100$ делений, цена деления $\delta = 1\units{мкА}$. Внутреннее сопротивление гальванометра $R_G = 1{,}0\units{кОм}.$ Как из этого прибора сделать вольтметр для измерения напряжений до $U = 100\units{В}$ или амперметр для измерения токов силой до $I = 1\units{А}?$

\\ \\ \\ \\ \\ \\ \\ \\
\adddate{8 класс. 20 апреля 2018}

\task 1
Между точками $A$ и $B$ электрической цепи подключены последовательно резисторы $R_1 = 10\units{Ом}$ и $R_2 = 20\units{Ом}$ и параллельно им $R_3 = 30\units{Ом}.$ Найдите эквивалентное сопротивление $R_{AB}$ этого участка цепи.

\task 2
Электрическая цепь состоит из последовательности $N$ одинаковых звеньев, в которых каждый резистор имеет сопротивление $r$. Последнее звено замкнуто резистором сопротивлением $R$. При каком соотношении $\cfrac{R}{r}$ сопротивление цепи не зависит от числа звеньев?

\task 3
Для измерения сопротивления $R$ проводника собрана электрическая цепь. Вольтметр $V$ показывает напряжение $U_V = 5\units{В},$ показание амперметра $A$ равно $I_A = 25\units{мА}.$ Найдите величину $R$ сопротивления проводника. Внутреннее сопротивление вольтметра $R_V = 1{,}0\units{кОм},$ внутреннее сопротивление амперметра $R_A = 2{,}0\units{Ом}.$

\task 4
Шкала гальванометра имеет $N=100$ делений, цена деления $\delta = 1\units{мкА}$. Внутреннее сопротивление гальванометра $R_G = 1{,}0\units{кОм}.$ Как из этого прибора сделать вольтметр для измерения напряжений до $U = 100\units{В}$ или амперметр для измерения токов силой до $I = 1\units{А}?$


% \begin{flushright}
\textsc{ГБОУ школа №554, 20 ноября 2018\,г.}
\end{flushright}

\begin{center}
\LARGE \textsc{Математический бой, 8 класс}
\end{center}

\problem{1} Есть тридцать карточек, на каждой написано по одному числу: на десяти карточках~–~$a$,  на десяти других~–~$b$ и на десяти оставшихся~–~$c$ (числа  различны). Известно, что к любым пяти карточкам можно подобрать ещё пять так, что сумма чисел на этих десяти карточках будет равна нулю. Докажите, что~одно из~чисел~$a, b, c$ равно нулю.

\problem{2} Вокруг стола стола пустили пакет с орешками. Первый взял один орешек, второй — 2, третий — 3 и так далее: каждый следующий брал на 1 орешек больше. Известно, что на втором круге было взято в сумме на 100 орешков больше, чем на первом. Сколько человек сидело за столом?

% \problem{2} Натуральное число разрешено увеличить на любое целое число процентов от 1 до 100, если при этом получаем натуральное число. Найдите наименьшее натуральное число, которое нельзя при помощи таких операций получить из~числа 1.

% \problem{3} Найти сумму $1^2 - 2^2 + 3^2 - 4^2 + 5^2 + \ldots - 2018^2$.

\problem{3} В кружке рукоделия, где занимается Валя, более 93\% участников~—~девочки. Какое наименьшее число участников может быть в таком кружке?

\problem{4} Произведение 2018 целых чисел равно 1. Может ли их сумма оказаться равной~0?

% \problem{4} Можно ли все натуральные числа от~1 до~9 записать в~клетки таблицы~$3\times3$ так, чтобы сумма в~любых двух соседних (по~вертикали или горизонтали) клетках равнялось простому числу?

\problem{5} На доске написано 2018 нулей и 2019 единиц. Женя стирает 2 числа и, если они были одинаковы, дописывает к оставшимся один ноль, а~если разные — единицу. Потом Женя повторяет эту операцию снова, потом ещё и~так далее. В~результате на~доске останется только одно число. Что это за~число?

\problem{6} Докажите, что в~любой компании людей найдутся 2~человека, имеющие равное число знакомых в этой компании (если $A$~знаком с~$B$, то~и $B$~знаком с~$A$).

\problem{7} Три колокола начинают бить одновременно. Интервалы между ударами колоколов соответственно составляют $\cfrac43$~секунды, $\cfrac53$~секунды и $2$~секунды. Совпавшие по времени удары воспринимаются за~один. Сколько ударов будет услышано за 1~минуту, включая первый и последний удары?

\problem{8} Восемь одинаковых момент расположены по кругу. Известно, что три из~них~— фальшивые, и они расположены рядом друг с~другом. Вес фальшивой монеты отличается от~веса настоящей. Все фальшивые монеты весят одинаково, но неизвестно, тяжелее или легче фальшивая монета настоящей. Покажите, что за~3~взвешивания на~чашечных весах без~гирь можно определить все фальшивые монеты.

\end{document}

\narrow

\begin{document}
\begin{flushright}
\textsc{Летняя научная школа \\ ГБОУ школа №554 \\ 12~июня 2020\,г.}
\end{flushright}

\begin{center}
\LARGE \textsc{Математический марафон}
\end{center}

Правила участия: \href{https://www.notion.so/Math-Marathon-8acf3ff3b2874cefabbfa78d2db4f07e}{https://notion.so/Math-Marathon-8acf3ff3b2874cefabbfa78d2db4f07e}.

% Добрынин
\problem{1.1} Найдите все значения параметра~$a$, при~котором уравнение 
$2ax=3$ имеет единственное решение.
\answer{$a\ne0$}

\problem{1.2} Найдите все значения параметра~$a$, при~котором уравнение 
$-ax=4a+2$ имеет единственное решение.
\answer{$a\ne0$}

\problem{1.3} Проанализируйте, в~зависимости от~параметра~$a$, 
число корней уравнения~$\cbr{a+1}x=a-1.$
\answer{при $a\ne-1$ единственный корень, при $a = -1$ корней нет.}

\problem{1.4} Проанализируйте, в~зависимости от~параметра~$a$, 
число корней уравнения~$\cbr{2a-3}\cbr{4a+4}x=a+1.$
\answer{при $a\ne1{,}5$ единственный корень, при $a = 1{,}5$ корней нет, при $a=-1$ бесконечно много корней}

\problem{1.5} Найдите все значения параметра~$a$, при~котором уравнение
$(a+2)(a+3)x=(a+1)(a+2)$ имеет решения.
\answer{$a\ne-3$}

\problem{1.6} Найдите все значения параметра~$a$, при~котором уравнение
$2ax+3=3x-a$ имеет единственное решение.
\answer{$a\ne1,5$}

\problem{1.7} Найдите все значения параметра~$a$, при~котором уравнение
$-10a+4x=5ax-14$ имеет корень~$x=0$.
\answer{$a = 1{,}4$}

\problem{1.8} Найдите все значения параметра~$a$, при~котором уравнение
$\cbr{x+2}\cbr{a+1}=5x-3a+4$ имеет корень~$x=-3.$
\answer{$a = -5$}

\problem{1.9} Найдите все значения параметра~$a$, при~котором уравнение
$\cbr{x+1}\cbr{a-1}=2a+3$ не~имеет корней.
\answer{$a = 1$}

\problem{1.10} Найдите все значения параметра~$a$, при~котором уравнение
$2x+3a-5=4ax-a(x+1)$ не~имеет своим корнем число 5.
\answer{$a\ne\frac{5}{11}$}

\problem{1.11} Найдите все значения параметра~$a$, при~котором уравнения
$x=2a+1$ и~$x=3a-5$ имеют общий корень.
\answer{$a = 6$}

\problem{1.12} Найдите все значения параметра~$a$, при~котором уравнения
$3x=a-9$ и~$-5x=2a-9$ имеют общий корень.
\answer{$a = \frac{72}{11}$}

\problem{1.13} Найдите все значения параметра~$a$, при~котором уравнение
$y-x=a$ имеет решение~$(1;-1)$.
\answer{$a = -2$}

\problem{1.14} Найдите все значения параметра~$a$, при~котором уравнение
$7ay+ax=2$ не~имеет решений.
\answer{$a = 0$}

\problem{1.15} Найдите все значения параметра~$a$, при~котором уравнение
$\cbr{a^2-4}y+\cbr{a-2}x=-1$ не~имеет решений.
\answer{$a = 2$}

\problem{1.16} Найдите все значения параметра~$a$, при~котором уравнение
$-4ay+\cbr{2a+1}x=2$ имеет решение.
\answer{$a$~--- любое число}

\problem{1.17} Найдите все значения параметра~$a$, при~котором уравнения
$y=3ax$ и~$y=2ax$ имеют единственное общее решение.
\answer{$a\ne0$}

\problem{1.18} Найдите все значения параметра~$a$, при~котором уравнения
$y=(3a-2)x$ и~$y=(2a+3)x$ имеют единственное общее решение.
\answer{$a$~---любое число}

\problem{1.19} Найдите все значения параметра~$a$, при~котором уравнения
$y=x$ и~$y=2x+a$ имеют общее решение.
\answer{$a$~---любое число}

\problem{1.20} Найдите все значения параметра~$a$, при~котором уравнения
$y=3x+a$ и~$y=ax-3$ имеют единственное общее решение.
\answer{$a\ne3$}

% Мария Шишкина
\problem{2.1}
Произведение нескольких различный простых чисел делится на~каждое из~этих чисел, уменьшенное на~1. Чему может быть равно это произведение?
Если в~ответе несколько вариантов, перечислите их~в~порядке возрастания через «;».
\answer{6;42;1806}

\problem{2.2}
Найдите~$(5160, 16920)$ и~$[5160, 16920]$ через разложение на~множители.
Ответ укажите в~порядке возрастания через «;».
\answer{120;727560}

\problem{2.3}
Найдите все пары натуральных чисел, наименьшее общее кратное которых равно 78, а~наибольший общий делитель равен 13.
Пары укажите в~скобках через «,».
\answer{(13,78), (26,39) или (26,39), (13, 78)}

\problem{2.4}
Найдите наибольший общий делитель всех чисел вида~$p^2-1$, где~$p$~--- простое число, большее 3, но~меньшее 2010.
\answer{24}

\problem{2.5}
Найти число различных делителей 1440, включая единицу и~само число.
\answer{36}

\problem{2.6}
Дано число $a=2^{2002}+3^{2002}$. Найдите последнюю цифру числа~$a$ и~остаток от~деления числа~$a$ на~11
Ответ записать в~порядке возрастания через «;».
\answer{2;3}

\problem{2.7}
Представить наибольший общий делитель чисел 27 и~96 в~виде~$d=27x+96y$, где~$x$ и~$y$~--- целые числа. В~ответе указать значения~$x$ и~$y$ через «;».
\answer{-7;2}

\problem{2.8}
Натуральные числа~$3n+2$ и~$8n+3$ делятся на~натуральное число~$p \ne 1$. Найдите~$p$.
\answer{7}

\problem{2.9}
Можно ли~вычеркнуть несколько цифр из~числа 123456789 так, чтобы получилось число, кратное 72?
Ответ запишите с~большой буквы.
\answer{Да}

\problem{2.10}
Можно ли~вычеркнуть несколько цифр из~числа 846927531 так, чтобы получилось число, кратное 72?
Ответ запишите с~большой буквы.
\answer{Нет}

\problem{2.11}
Какое наибольшее количество цифр можно вычеркнуть из~числа 124875963 так, чтобы получилось число, кратное 72?
\answer{5}

\problem{2.12}
Делится ли~число~$a = 4 \cdot 16^{12} - 2^{40}$ на~33?
Ответ запишите с~большой буквы.
\answer{Да}

\problem{2.13}
Сократима ли~дробь~$\frac{2a+b}{5a+b}$, если известно, что дробь~$\frac{a}{b}$ несократима ($a$ и~$b$~--- натуральные числа)?
Ответ запишите с~большой буквы.
\answer{Нет}

\problem{2.14}
Какое наибольшее количество натуральных чисел можно записать в~строку так, чтобы сумма любых трех соседних чисел была четной, а~сумма любых соседних чисел была нечётной?
\answer{3}  % was 5

\problem{2.15}
Пятизначное число делится на~72, причём три его цифры~--- единицы. Найти все такие числа.
Ответ запишите в~порядке возрастания через «;».
\answer{11016;11160;14112;41112}

% Анна Сергеевна
\problem{3.1}
Катер проходит по~реке расстояние между двумя пунктами (в обе стороны) за~14~часов. Чему равно это расстояние, если скорость катера в~стоячей воде~$35\fracunits{км}{}{ч}{}$, а~скорость течения реки~---~$5\fracunits{км}{}{ч}{}$? Ответ дайте в~метрах.
\answer{240}

\problem{3.2}
Автомобиль ехал 2~минуты со~скоростью~$10\fracunits{м}{}{с}{}$, а~затем проехал ещё 500~метров за~30~секунд. Определить среднюю скорость движения в~$\fracunits{м}{}{с}{}$.
\answer{11}

\problem{3.3}
Какую массу имеет куб с~площадью поверхности~$150\dunits{см}2$, если плотность вещества, из~которого он~изготовлен, равна~$2700\fracunits{кг}{}{м}{3}$? Ответ дайте в~граммах.
\answer{337,5}

\problem{3.4}
В~сообщающиеся сосуды одинакового сечения налита вода. 
В~один из~сосудов поверх воды долили масло высотой 40~см. 
На~сколько сантиметров изменится уровень воды в~другом сосуде? 
Плотность масла~$800\fracunits{кг}{}{м}{3}$.
\answer{16}

\problem{3.5}
Тело весит в~воздухе~$3\units{Н}$, в~воде~$1{,}8\units{Н}$ и~в жидкости неизвестной плотности 
$2{,}04\units{Н}$. 
Какова плотность этой неизвестной жидкости? 
Ответ дайте в~$\fracunits{кг}{}{м}{3}$.
\answer{800}

\problem{3.6}
Найти в~мм удлинение пружины, возникающее под действием подвешенного к~ней груза массой~$200\units{г}$, если жесткость пружины равна~$1000\units{Н}$/м.
\answer{2}

\problem{3.7}
Плечи рычага, находящегося в~равновесии, равны~$40\units{см}$ и~$10\units{см}$. Большая сила, действующая на~рычаг, равна~$20\units{Н}$. Определите меньшую силу в~ньютонах.
\answer{5}

\problem{3.8}
При выстреле из~пушки снаряд массой~$10\units{кг}$
вылетает со~скоростью~$800\fracunits{м}{}{с}{}$. Чему равна скорость отката 
ствола орудия, если его масса~$1000\units{кг}$? Начальная скорость орудия равна нулю.
\answer{8}

\problem{3.9}
Какую силу надо приложить, чтобы удержать под водой бетонную плиту, 
масса которой~$720\units{кг}$? 
Ответ дайте в~ньютонах.
\answer{4200}

\problem{3.10}
Какую мощность развивает подъемник, поднимающий груз весом~$24\units{кН}$ на~высоту~$20\units{м}$ за~2~мин? 
Ответ дайте в~кВт.
\answer{4}

\problem{3.11}
Девочки сделали снеговика, а~мальчики соорудили точную его копию, но~в~два раза большей высоты. Какова масса копии, если масса оригинала равна~$50\units{кг}$? (Плотность снега в~обоих снеговиках одинаковая.) 
\answer{400}

\problem{3.12}
В~доску толщиной~$5\units{см}$ забили гвоздь длиной~$10\units{см}$ так, что половина гвоздя прошла навылет. 
Чтобы вытащить его из~доски, необходимо приложить силу~$1{,}8\units{кН}$. Гвоздь вытащили из~доски. Какую при~этом совершили механическую работу? 
Ответ дайте в~джоулях.
\answer{135}

\problem{3.13}
Вес тела в~воде в~2~раза меньше, чем в~воздухе. Какова плотность вещества тела? 
Ответ дайте в~$\fracunits{г}{}{см}{3}$.
\answer{2}

\problem{3.14}
Льдина плавает в~воде. Объем части над водой~--- $20\dunits{м}3$. Какой объем подводной части? 
Ответ дайте в~$\dunits{м}3$.
\answer{180}

\problem{3.15}
Найдите силу, с~которой воздух давит на~поверхность стола длиной~$1{,}2\units{м}$ и~шириной~$60\units{см}$. (Атмосферной давление равно~$100000\units{Па}$). 
Ответ дайте в~кН.
\answer{72}

\problem{3.16}
Аквариум наполнен водой доверху. С~какой силой давит вода на~стенку аквариума длиной~$50\units{см}$ и~высотой~$30\units{см}$?
Ответ дайте в~ньютона)
\answer{225}

% Елена Анатольевна
\problem{4.1} Четыре рубашки дешевле куртки на~8\%. На~сколько процентов пять рубашек дороже куртки?
\answer{15}

\problem{4.2} Под строительную площадку отвели участок прямоугольной
формы. При утверждении плана застройки ширину участка уменьшили на~20\%, 
а длину уменьшили на~10\%. На~сколько процентов уменьшилась площадь участка после утверждения плана застройки?
\answer{28}

\problem{4.3} В~понедельник акции компании подорожали на~некоторое число
процентов, а~во вторник подешевели на~то же~самое число процентов.
В~результате они стали стоить на~49\% дешевле, чем при~открытии
торгов в~понедельник. На~сколько процентов подорожали акции компании в~понедельник?
\answer{70}

\problem{4.4} Налог на~доходы составляет 13\% от~заработной платы. Сколько рублей составляет заработная плата Павла Сергеевича до~вычета налогов, если после вычета у~него остаётся 13~050~рублей?
\answer{15000}

\problem{4.5} В~субботу и~воскресенье Маша и~Вася ходили за~грибами. В~субботу Маша собрала на~5\% меньше грибов, чем Вася, а~в~воскресенье — на~50\% больше, чем Вася. За~два дня Маша собрала на~20\% больше грибов, чем Вася. Какое наименьшее число грибов они могли собрать вместе?
\answer{242}

\problem{4.6} В~кружке по~лингвистике занимаются и~мальчики, и~девочки, но~мальчики составляют больше 66\% участников. Какое наименьшее число участников может быть в~этом кружке? 
\answer{3}

\problem{4.7} Брюки дороже рубашки на~25\% и~дешевле пиджака на~20\%.
На~сколько процентов рубашка дешевле пиджака?
\answer{36}

\problem{4.8} Митя, Антон, Гоша и~Борис учредили компанию с~уставным
капиталом 200~000~рублей. Митя внёс 14\% уставного капитала, Антон~--- 42~000~рублей, Гоша~---~$0{,}12$ уставного капитала, а~оставшуюся
часть уставного капитала внёс Борис. Учредители договорились делить ежегодную прибыль пропорционально внесённому в~уставной капитал вкладу. Сколько рублей от~прибыли в~1 000~000~рублей причитается Борису?
\answer{530000}

\problem{4.9} В~магазине два отдела: бакалеи и~гастрономии. Если бы~дневная выручка отдела гастрономии сократилась вдвое, дневная выручка магазина уменьшилась бы~на 34\%. На~сколько процентов увеличилась бы~дневная выручка магазина, если бы~дневная выручка отдела бакалеи выросла втрое?
\answer{64}

\problem{4.10} Процент числа учеников восьмого класса, принявших участие в~олимпиаде, заключён в~пределах от~96,8\% до~97,2\%. Найдите наименьшее возможное число учеников этого класса.
\answer{32}

\problem{4.11} Проходная комната имеет форму прямоугольного параллелепипеда. Её длина и~ширина равны 7 м~и~4 м~соответственно, высота потолка равна 3 м. 
Ширина каждого из~двух дверных проёмов комнаты равна 1 м, высота 2 м. В~комнате два одинаковых квадратных окна шириной~$1{,}5$ м~каждое. 
Для оклейки стен комнаты обоями необходимо купить их~с запасом в~10\% от~оклеиваемой площади.
Сколько рулонов обоев нужно купить, если площадь одного рулона равна 6~кв.~м и~рулоны продаются только целиком?
\answer{11}

\problem{4.12} Молодая семья состоит из~двух человек: мужа и~жены. Доход семьи складывается из~их зарплат. Если бы~зарплата мужа увеличилась втрое, доход семьи вырос бы~на 108\%. Сколько процентов дохода семьи составляет зарплата жены?
\answer{46}

\problem{4.13} 31~декабря 2014 года Алексей взял в~банке 6~902~000~рублей в~кредит под $12{,}5\%$ годовых. Схема выплат кредита следующая: 
31~декабря каждого следующего года банк начисляет проценты на~оставшуюся сумму долга (т. е. увеличивает долг на~12,5\%), 
затем Алексей переводит в~банк~$x$ рублей. Какой должна быть сумма~$x$, 
чтобы Алексей выплатил долг четырьмя равными платежами (т. е. за~четыре года)?
\answer{2296350}

\problem{4.14} 31~декабря 2014 года Валерий взял в~банке 1~млн рублей в~кредит. Схема выплаты кредита следующая: 
31~декабря каждого следующего года банк начисляет проценты на~оставшуюся сумму долга (т. е. увеличивает долг на~определённое количество процентов), 
затем Валерий переводит очередной транш. Валерий выплатил кредит за~два транша, 
переводя в~первый раз 660~тыс. рублей, во~второй~--- 484~тыс. рублей. 
Под какой процент банк выдал кредит Валерию?
\answer{10}

\problem{4.15} 31~декабря 2014 года Тимофей взял в~банке 7~007~000~рублей в~кредит под 20\% годовых. Схема выплаты кредита следующая: 31~декабря каждого следующего года банк начисляет проценты на~оставшуюся сумму долга (т. е. увеличивает долг на~20\%), затем Тимофей переводит в~банк платёж. Весь долг Тимофей выплатил за~3~равных платежа. На~сколько рублей меньше он~бы отдал банку, если бы~смог выплатить долг за~2~равных платежа?
\answer{806400}

\problem{4.16} Магазин закупил некоторое количество товара и~начал его реализацию по~цене на~25\% выше цены, назначенной производителем, чтобы чтобы покрыть затраты, связанные с~его транспортировкой, и~другие дополнительные расходы. Оставшуюся после реализации часть товара магазин уценил на~16\% с~тем, чтобы покрыть только затраты на~покупку этой части товара у~производителя и~его транспортировку. Сколько процентов от~цены, назначенной производителем, составляла цена транспортировки товара?
\answer{5}

% Тимбухтин
\problem{5.1}
На~тарелке 12~пирожков: 5 с~мясом, 4 с~капустой и~3 с~вишней. Наташа наугад выбирает один пирожок. Найдите вероятность того, что он~окажется с~вишней.
\answer{0,25}

\problem{5.2}
В~магазин поступило 20~холодильников, пять из~которых имеют заводской дефект. Случайным образом выбирают один холодильник. Какова вероятность того, что он~будет без дефекта?
\answer{0,75}

\problem{5.3}
Для экзамена подготовили билеты с~номерами от~1~до~50. Какова вероятность того, что наугад взятый учеником билет имеет однозначный номер?
\answer{0,18}

\problem{5.4}
Бросают две монеты. Найти вероятность того, что появится хотя бы~одна «решка».
\answer{0,75 }

\problem{5.5}
Бросают две игральные игральные кости. Найти вероятность того, что на~верхних гранях сумма числа очков не~превосходит 5. 
\answer{10/36 (0.278)}

\problem{5.6}
Сколькими способами можно расставить на~полке в~ряд 5~книг?
\answer{120}

\problem{5.7}
Сколькими способами можно с~помощью букв~$K$,~$L$,~$M$,~$H$ обозначить вершины четырехугольника?
\answer{24}

\problem{5.8}
В~пассажирском поезде 9~вагонов. Сколькими способами можно рассадить в~поезде 4~человека, при~условии, что все они должны ехать в~различных вагонах?
\answer{3024}

\problem{5.9}
В~почтовом отделении продаются открытки 8~типов. Сколькими способами можно купить 6~разных открыток? 
\answer{28}

\problem{5.10}
Ученику необходимо выучить 4~темы за~7~дней. Сколькими способами он~может составить своё расписание, если за~один день он~может выучить только одну тему? 
\answer{840}

\problem{5.11}
В~урне содержится 5~черных и~6~белых шаров. Случайным образом вынимают 4~шара. Найдите вероятность того, что среди них имеется 2~белых шара. 
\answer{150/330}

\problem{5.12}
В~урне 3~белых и~4 чёрных шара. Вынимают 3~шара. Найти вероятность того, что вынули 2 чёрных и~один белый шар. 
\answer{18/35}

\problem{6.1}
В~треугольнике~$\triangle KLM$ $KP$~--- медиана, причём~$KL = 4 KP$, 
а~$\angle PKL = 60\degrees$. 
Найдите угол~$\angle MKL$.
\answer{150}

\problem{6.2}
Треугольник~$ABC$~--- равнобедренный с~основанием~$BC$ и~углом~$A$, равным~$30\degrees$. 
$AB = AC = 10$. $BD$~--- высота этого треугольника, проведённая к~стороне~$AC$, 
$DK$~--- высота треугольника~$\triangle ABD$, проведённая к~стороне~$AB$. Найдите~$AK$.
\answer{$7{,}5$}

\begin{wrapfigure}[0]{r}{30mm}
\vspace{-6mm}
\begin{tikzpicture}[rotate=-10]
    \tkzDefPoints{0/0/A,3/2/B,6/0/C,2/0/D}
    \tkzDrawPoints(A,B,C,D)
    \tkzMarkSegments[pos=.5,mark=||](D,C B,C A,B)
    \tkzMarkSegments[pos=.5,mark=|](A,D D,B)
    \tkzDrawPolygon[thick](A,B,C)
    \tkzDrawSegment(D,B);
\end{tikzpicture}
\end{wrapfigure}

\problem{6.3}
Равнобедренный треугольник таков, что его можно разрезать на~два меньших равнобедренных треугольника. На~рисунке показано, как это можно сделать. 
Найдите угол при~основании исходного треугольника.
\parshape=5 0em 20em 0em 20em 0em 20em 0em 20em 0em 20em
\answer{36}

\problem{6.4}
В~треугольнике ABC проведены медиана~$CK$, высота~$BH$ и~биссектриса~$AL$. Отрезки~$AL$ и~$KH$ пересекаются в~точке~$O$, причём~$HO$ =~$OK$,~$AO$ =~$OL$. Найдите угол~$\angle ABC$.
\answer{60}

\begin{wrapfigure}[0]{r}{30mm}
\vspace{-10mm}
\begin{tikzpicture}
    \tkzDefPoints{0/1/A,0.5/2.5/B,4.5/3/C,4/1.5/D,0/0/O,4.6/0/X}
    \tkzDefPointsBy[projection=onto O--X](A,B,C,D){A1,B1,C1,D1}

    \tkzDrawPolygon[thick](A,B,C,D)
    \tkzDrawPoints(A,B,C,D)
    \tkzDrawLine(O,X)
    \tkzDrawSegments[dashed](A,A1 B,B1 C,C1 D,D1);
    \tkzLabelLine[left](A,A1){$4$}
    \tkzLabelLine[pos=0.3,right](B,B1){$x$}
    \tkzLabelLine[right](C,C1){$9$}
    \tkzLabelLine[left](D,D1){$5$}
    \tkzMarkRightAngles[size=0.15](A,A1,X B,B1,X C,C1,X D,D1,X)
\end{tikzpicture}
\end{wrapfigure}

\problem{6.5}
Прямая не~пересекает сторон параллелограмма. Расстояния от~трёх его вершин до~этой прямой равны последовательно 4, 5 и~9. Найдите расстояние до~прямой от~четвёртой его вершины.
\parshape=5 0em 20em 0em 20em 0em 20em 0em 20em 0em 20em
\answer{8}

\begin{wrapfigure}[0]{r}{30mm}
\vspace{-10mm}
\begin{tikzpicture}
    \tkzDefPoints{0/0/A,1/3/B,3.5/3/C,2.5/0/D}
    \tkzDefMidPoint(C,D) \tkzGetPoint{K}
    \tkzDefPointsBy[projection=onto A--K](B){H}

    \tkzLabelPoints[left](A,B)
    \tkzLabelPoints[right](C,D,K)
    \tkzLabelPoints[below](H)
    \tkzDrawPoints(A,B,C,D,K,H)
    \tkzDrawPolygon[thick](A,B,C,D)
    \tkzDrawSegments(A,K B,H)
    \tkzLabelAngle[pos=1.45](D,A,K){$30\degrees$}
    \tkzMarkAngle[arc=l,mark=none](D,A,K)
    \tkzMarkRightAngle[size=0.15](B,H,K)
    \tkzMarkSegments[pos=.5,mark=||](D,K K,C)
\end{tikzpicture}
\end{wrapfigure}

\problem{6.6}
Вершину параллелограмма~$ABCD$, стороны~$AB$ и~$BC$ которого равны 12 и~8, соединили с~серединой~$K$ его противоположной стороны~$CD$. Угол~$\angle KAD$ равен~$30\degrees$.  Найдите~$BH$.
\parshape=5 0em 20em 0em 20em 0em 20em 0em 20em 0em 20em
\answer{8}

\begin{wrapfigure}[0]{r}{30mm}
\vspace{-5mm}
\begin{tikzpicture}[scale=1.3]
    \tkzDefPoints{0/0/A,4/3/B,4/0/C}
    \tkzDrawPolygon[thick](A,B,C)
    \tkzDefCircle[in](A,B,C) \tkzGetPoint{O}\tkzGetLength{rIN}
    \tkzDrawCircle[R](O,\rIN pt)
    \tkzDefPointsBy[projection=onto A--B](O){C1}
    \tkzDefPointsBy[projection=onto A--C](O){B1}
    \tkzDefPointsBy[projection=onto B--C](O){A1}
    \tkzDefPointsBy[projection=onto C1--B1](A1){H}
    \tkzInterLL(A1,H)(A,C) \tkzGetPoint{X}
    \tkzDrawPoints(A,B,C,A1,B1,C1,H,X)
    \tkzDrawSegments(A1,X B1,C1)
    \tkzMarkRightAngle[size=0.15](C1,H,A1)
    \tkzLabelSegment[below](A,X){$x$}
\end{tikzpicture}
\end{wrapfigure}

\problem{6.7}
В~прямоугольный треугольник со~сторонами 3, 4 и~5~вписана окружность. Через точку её касания с~коротким катетом проведена прямая, перпендикулярная хорде, соединяющей две другие точки касания. 
Эта прямая разбивает второй катет на~два отрезка. Найдите длину меньшего из~них (отрезок $x$ на~рисунке).
\parshape=5 0em 20em 0em 20em 0em 20em 0em 20em 0em 20em
\answer{1}

\begin{wrapfigure}[0]{r}{30mm}
\vspace{-15mm}
\begin{tikzpicture}[scale=0.7]
    \tkzDefPoints{0/0/A,3/4/B,4/0/C}
    \tkzDrawPolygon[thick](A,B,C)
    \tkzDefCircle[in](A,B,C) \tkzGetPoint{oIN} \tkzGetLength{rIN}
    \tkzDefCircle[ex](B,A,C) \tkzGetPoint{oOUT} \tkzGetSecondPoint{M} \tkzGetLength{rOUT}
    \tkzDrawCircle[R](oIN,\rIN pt)
    \tkzDrawCircle[R](oOUT,\rOUT pt)
    \tkzDefPointsBy[projection=onto B--C](oIN,oOUT){K,M}
    \tkzDrawPoints(A,B,C,K,M,oIN,oOUT)
    \tkzLabelLine[right](K,M){$x$}
    \tkzDrawLines[add=0 and 0.7](A,B)
    \tkzDrawLines[add=0 and 1.4](A,C)
    \tkzLabelPoints[left](K)
    \tkzLabelPoints[right](M)
\end{tikzpicture}
\end{wrapfigure}

\problem{6.8}
\noindent
Одна из~сторон треугольника равна 8. В~противоположный от~неё угол вписаны две окружности, которые касаются данной стороны в~двух точках. Найдите расстояние между этими точками (отрезок~$MK$ на~рисунке), если две другие стороны треугольника равны 10 и~12.
\parshape=5 0em 20em 0em 20em 0em 20em 0em 20em 0em 20em
\answer{2}

\problem{6.9}
Две стороны треугольника равны 10 и~14. В~угол, образованный этими сторонами, вписаны две окружности, которые касаются третьей стороны треугольника в~двух точках. Найдите его третью сторону, если указанные точки делят её на~три равные части.
\answer{12}

\begin{wrapfigure}[0]{r}{30mm}
\vspace{-8mm}
\begin{tikzpicture}[scale=0.8]
    \tkzDefPoints{0/0/A,0/4/B,7/4/C,7/0/D,1/3/X}
    \tkzDrawPolygon[thick](A,B,C,D)
    \tkzDrawSegments(A,X B,X C,X D,X)
    \tkzDrawPoints(A,B,C,D,X)
    \tkzLabelLine[right](A,X){$x$}
    \tkzLabelLine[below](B,X){$1$}
    \tkzLabelLine[below](C,X){$7$}
    \tkzLabelLine[above](D,X){$8$}
\end{tikzpicture}
\end{wrapfigure}

\problem{6.10}
Внутри прямоугольника взяли точку. Оказалось, что расстояния от~неё до~трех его вершин равны последовательно 1, 7 и~8. Найдите расстояние от~данной точки до~четвертой вершины.
\parshape=4 0em 20em 0em 20em 0em 20em 0em 20em
\answer{4}

\problem{7.1}
При каких~$a$ оба корня уравнения $x^2 - 6ax + 2 -2a + 9a^2 = 0$ больше~3?
\answer{$a\in\cbr{\frac{11}9;+\infty}$}

\problem{7.2}
Найти все значения параметра $c$, при~которых обе корня квадратного уравнения $x^2 + 4cx + 1 -2c + 4c^2 = 0$ меньше, чем $(-1)$.
\answer{$c>1$}

\problem{7.3}
Найдите все значения $k$, при~которых один корень уравнения 
$x^2 - (k+1)x + k^2 + k - 8 = 0$ больше~2, а другой корень меньше~2.
\answer{$(-2;3)$}

\problem{7.4}
Найдите все значения $k$, при~которых один корень уравнения 
$x^2 - (k+1)x + k^2 + k - 8 = 0$ меньше~1, а другой корень больше~2.
\answer{$(5;24)$}

\problem{7.5}
При каких~$a$ корни уравнения $(2-a)x^2 - 3ax + 2a = 0$ больше~$\frac 12$?
\answer{$a\in\left[\frac{16}{17};2\right)$}

\problem{7.6}
При каких~$a$ один из~корней уравнения 
$(a^2 + a + 1)x^2 + (2a-3)x + a - 5 = 0$ больше~1,
а другой меньше 1?
\answer{$a\in\cbr{-2-\sqrt{11};-2+\sqrt{11}}$}

\problem{7.7}
При каких значениях параметра $a$ один из~корней уравнения 
$(a - 5)x^2 - 2ax + a - 4 = 0$ меньше~1,
а другой больше 2?
\answer{$a\in(5;24)$}

\problem{7.8}
При каких значениях параметра $a$ корни уравнения 
$(a + 1)x^2 - 3ax + 4a = 0$ принадлежат интервалу $(2;5)$?
\answer{$a\in\left[-\frac{16}7;-2\right)$}

\problem{7.9}
При каких~$k$ корни уравнения $kx^2 - (k+1)x + 2 = 0$ будут по модулю меньше 1?
\answer{$k\ge3+\sqrt{8}$}

\problem{7.10}
При каких~$m$ один из~корней уравнения 
$x^2 - (2m+1)x + m^2 + m -2 = 0$
находится между числами 1 и 3, а второй между числами 4 и 6?
\answer{$m\in(2;4)$}

\problem{7.11}
При всех $a$ решить уравнение $(a^2 - 4)x^2 + (6a+12)x + 3a + 6 = 0$.
\answer{Если $a=-2$, то $x\in\mathbb{R}$; если $a=2$, то $x=0{,}5$; если $a=5$, то $x=-1$, если $a>5$, то $x_{1,2}=\frac{-6\pm\sqrt{60-12a}}{2(a-2)}$; если $a<5$ и $a\ne\pm2$, то решений нет}

\problem{7.12}
При каких~$m$ корни уравнения $mx^2 - 2(m-1)x + 3m - 2 = 0$ отрицательны?
\answer{$m\in\left(\frac23;\frac{\sqrt{2}}2\right]$}

\problem{7.13}
При каких~$a$ уравнение $(2-x)(x+1) = a$ имеет положительные корни?
\answer{$a\in\cbr{2;\frac94}$}

\problem{7.14}
При каких~$m$ корни уравнения $x^2 + 2x + m = 0$ больше $m$?
\answer{$m\in\cbr{-\infty;-3}$}

\problem{7.15}
При каких~$m$ один из~корней уравнения
$(m^2 - 1)x^2 + (6m-1)x - m^4 = 0$
меньше $m$, а другой больше $m$?
\answer{$m\in\cbr{-1;0} \cup \cbr{\frac12;1}$}

\problem{7.16}
Найти все значения $a$, при~которых \underline{хотя бы один корень} уравнения 
$x^2 - 2(a+1)x + 9a - 5 = 0$
больше~1.
\answer{$a\in\cbr{-\infty; 1}\cup\cbr{6;+\infty}$}

\end{document}