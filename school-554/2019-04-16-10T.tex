\setdate{16~апреля~2019}
\setclass{10«Т»}

\addpersonalvariant{Михаил Бурмистров}

\tasknumber{1}%
\task{%
    С какой силой взаимодействуют 2 точечных заряда $q_1 = 2\,\text{нКл}$ и $q_2 = 4\,\text{нКл}$,
    находящиеся на расстоянии $r = 2\,\text{см}$?
}
\answer{%
    $
        F
            = k\frac{q_1q_2}{r^2}
            = 9 \cdot 10^{9}\,\frac{\text{Н}\cdot\text{м}^{2}}{\text{Кл}^{2}} \cdot \frac{2\,\text{нКл} *4\,\text{нКл}}{\sqr{ 2\,\text{см} }}
            = 18 \cdot 10^{31}\units{Н}
              \approx {18{,}00} \cdot 10^{31}\units{Н}
    $
}
\solutionspace{120pt}

\tasknumber{2}%
\task{%
    Два одинаковых маленьких проводящих заряженных шарика находятся
    на расстоянии~$r$ друг от друга.
    Заряд первого равен~$-3Q$, второго~---$+5Q$.
    Шарики приводят в соприкосновение, а после опять разводят на то же самое расстояние~$r$.
    \begin{itemize}
        \item Каким стал заряд каждого из шариков?
        \item Определите характер (притяжение или отталкивание) и силу взаимодействия шариков до и после соприкосновения.
        \item Как изменилась сила взаимодействия шариков после соприкосновения?
    \end{itemize}
}
\answer{%
    \begin{align*}
    F &= k\frac{q_1 q_2}{r^2} = k\frac{(-3Q) \cdot (+5Q)}{r^2}, \text{отталкивание}; \\
        q'_1 = q'_2 = \frac{q_1 + q_2}2 = \frac{-3Q + +5Q}2 \implies
        F'  &= k\frac{q'_1 q'_2}{r^2}
            = k\frac{\sqr{\frac{(-3Q) + (+5Q)}2}}{r^2},
        \text{отталкивание}.
    \end{align*}
}
\solutionspace{120pt}

\tasknumber{3}%
\task{%
    На координатной плоскости в точках $(-l; 0)$ и $(l; 0)$
    находятся заряды, соответственно, $+Q$ и $+Q$.
    Сделайте рисунок, определите величину напряжённости электрического поля
    в точках $(0; -l)$ и $(2l; 0)$ и укажите её направление.
}
\solutionspace{120pt}

\tasknumber{4}%
\task{%
    Заряд $q_1$ создает в точке $A$ электрическое поле
    по величине равное~$E_1=300\funits{В}{м}$,
    а $q_2$~--- $E_2=400\funits{В}{м}$.
    Угол между векторами $\vect{E_1}$ и $\vect{E_2}$ равен $\varphi$.
    Определите величину суммарного электрического поля в точке $A$,
    создаваемого обоими зарядами $q_1$ и $q_2$.
    Сделайте рисунки и вычислите значение для двух значений угла $\varphi$:
    $\varphi_1=0^\circ$ и $\varphi_2=90^\circ$.
}

\variantsplitter

\addpersonalvariant{Гагик Аракелян}

\tasknumber{1}%
\task{%
    С какой силой взаимодействуют 2 точечных заряда $q_1 = 3\,\text{нКл}$ и $q_2 = 2\,\text{нКл}$,
    находящиеся на расстоянии $l = 3\,\text{см}$?
}
\answer{%
    $
        F
            = k\frac{q_1q_2}{l^2}
            = 9 \cdot 10^{9}\,\frac{\text{Н}\cdot\text{м}^{2}}{\text{Кл}^{2}} \cdot \frac{3\,\text{нКл} *2\,\text{нКл}}{\sqr{ 3\,\text{см} }}
            = 6 \cdot 10^{31}\units{Н}
              \approx {6{,}00} \cdot 10^{31}\units{Н}
    $
}
\solutionspace{120pt}

\tasknumber{2}%
\task{%
    Два одинаковых маленьких проводящих заряженных шарика находятся
    на расстоянии~$d$ друг от друга.
    Заряд первого равен~$-2q$, второго~---$-3q$.
    Шарики приводят в соприкосновение, а после опять разводят на то же самое расстояние~$d$.
    \begin{itemize}
        \item Каким стал заряд каждого из шариков?
        \item Определите характер (притяжение или отталкивание) и силу взаимодействия шариков до и после соприкосновения.
        \item Как изменилась сила взаимодействия шариков после соприкосновения?
    \end{itemize}
}
\answer{%
    \begin{align*}
    F &= k\frac{q_1 q_2}{d^2} = k\frac{(-2q) \cdot (-3q)}{d^2}, \text{отталкивание}; \\
        q'_1 = q'_2 = \frac{q_1 + q_2}2 = \frac{-2q -3q}2 \implies
        F'  &= k\frac{q'_1 q'_2}{d^2}
            = k\frac{\sqr{\frac{(-2q) + (-3q)}2}}{d^2},
        \text{отталкивание}.
    \end{align*}
}
\solutionspace{120pt}

\tasknumber{3}%
\task{%
    На координатной плоскости в точках $(-r; 0)$ и $(r; 0)$
    находятся заряды, соответственно, $+q$ и $-q$.
    Сделайте рисунок, определите величину напряжённости электрического поля
    в точках $(0; r)$ и $(-2r; 0)$ и укажите её направление.
}
\solutionspace{120pt}

\tasknumber{4}%
\task{%
    Заряд $q_1$ создает в точке $A$ электрическое поле
    по величине равное~$E_1=250\funits{В}{м}$,
    а $q_2$~--- $E_2=250\funits{В}{м}$.
    Угол между векторами $\vect{E_1}$ и $\vect{E_2}$ равен $\varphi$.
    Определите величину суммарного электрического поля в точке $A$,
    создаваемого обоими зарядами $q_1$ и $q_2$.
    Сделайте рисунки и вычислите значение для двух значений угла $\varphi$:
    $\varphi_1=0^\circ$ и $\varphi_2=60^\circ$.
}

\variantsplitter

\addpersonalvariant{Ирен Аракелян}

\tasknumber{1}%
\task{%
    С какой силой взаимодействуют 2 точечных заряда $q_1 = 4\,\text{нКл}$ и $q_2 = 2\,\text{нКл}$,
    находящиеся на расстоянии $r = 5\,\text{см}$?
}
\answer{%
    $
        F
            = k\frac{q_1q_2}{r^2}
            = 9 \cdot 10^{9}\,\frac{\text{Н}\cdot\text{м}^{2}}{\text{Кл}^{2}} \cdot \frac{4\,\text{нКл} *2\,\text{нКл}}{\sqr{ 5\,\text{см} }}
            = \frac{72}{25} \cdot 10^{31}\units{Н}
              \approx {2{,}88} \cdot 10^{31}\units{Н}
    $
}
\solutionspace{120pt}

\tasknumber{2}%
\task{%
    Два одинаковых маленьких проводящих заряженных шарика находятся
    на расстоянии~$d$ друг от друга.
    Заряд первого равен~$-4Q$, второго~---$-5Q$.
    Шарики приводят в соприкосновение, а после опять разводят на то же самое расстояние~$d$.
    \begin{itemize}
        \item Каким стал заряд каждого из шариков?
        \item Определите характер (притяжение или отталкивание) и силу взаимодействия шариков до и после соприкосновения.
        \item Как изменилась сила взаимодействия шариков после соприкосновения?
    \end{itemize}
}
\answer{%
    \begin{align*}
    F &= k\frac{q_1 q_2}{d^2} = k\frac{(-4Q) \cdot (-5Q)}{d^2}, \text{отталкивание}; \\
        q'_1 = q'_2 = \frac{q_1 + q_2}2 = \frac{-4Q -5Q}2 \implies
        F'  &= k\frac{q'_1 q'_2}{d^2}
            = k\frac{\sqr{\frac{(-4Q) + (-5Q)}2}}{d^2},
        \text{отталкивание}.
    \end{align*}
}
\solutionspace{120pt}

\tasknumber{3}%
\task{%
    На координатной плоскости в точках $(-r; 0)$ и $(r; 0)$
    находятся заряды, соответственно, $-Q$ и $+Q$.
    Сделайте рисунок, определите величину напряжённости электрического поля
    в точках $(0; -r)$ и $(2r; 0)$ и укажите её направление.
}
\solutionspace{120pt}

\tasknumber{4}%
\task{%
    Заряд $q_1$ создает в точке $A$ электрическое поле
    по величине равное~$E_1=50\funits{В}{м}$,
    а $q_2$~--- $E_2=120\funits{В}{м}$.
    Угол между векторами $\vect{E_1}$ и $\vect{E_2}$ равен $\alpha$.
    Определите величину суммарного электрического поля в точке $A$,
    создаваемого обоими зарядами $q_1$ и $q_2$.
    Сделайте рисунки и вычислите значение для двух значений угла $\alpha$:
    $\alpha_1=0^\circ$ и $\alpha_2=90^\circ$.
}

\variantsplitter

\addpersonalvariant{Сабина Асадуллаева}

\tasknumber{1}%
\task{%
    С какой силой взаимодействуют 2 точечных заряда $q_1 = 4\,\text{нКл}$ и $q_2 = 3\,\text{нКл}$,
    находящиеся на расстоянии $d = 6\,\text{см}$?
}
\answer{%
    $
        F
            = k\frac{q_1q_2}{d^2}
            = 9 \cdot 10^{9}\,\frac{\text{Н}\cdot\text{м}^{2}}{\text{Кл}^{2}} \cdot \frac{4\,\text{нКл} *3\,\text{нКл}}{\sqr{ 6\,\text{см} }}
            = 3 \cdot 10^{31}\units{Н}
              \approx {3{,}00} \cdot 10^{31}\units{Н}
    $
}
\solutionspace{120pt}

\tasknumber{2}%
\task{%
    Два одинаковых маленьких проводящих заряженных шарика находятся
    на расстоянии~$r$ друг от друга.
    Заряд первого равен~$+2q$, второго~---$-3q$.
    Шарики приводят в соприкосновение, а после опять разводят на то же самое расстояние~$r$.
    \begin{itemize}
        \item Каким стал заряд каждого из шариков?
        \item Определите характер (притяжение или отталкивание) и силу взаимодействия шариков до и после соприкосновения.
        \item Как изменилась сила взаимодействия шариков после соприкосновения?
    \end{itemize}
}
\answer{%
    \begin{align*}
    F &= k\frac{q_1 q_2}{r^2} = k\frac{(+2q) \cdot (-3q)}{r^2}, \text{отталкивание}; \\
        q'_1 = q'_2 = \frac{q_1 + q_2}2 = \frac{+2q -3q}2 \implies
        F'  &= k\frac{q'_1 q'_2}{r^2}
            = k\frac{\sqr{\frac{(+2q) + (-3q)}2}}{r^2},
        \text{отталкивание}.
    \end{align*}
}
\solutionspace{120pt}

\tasknumber{3}%
\task{%
    На координатной плоскости в точках $(-d; 0)$ и $(d; 0)$
    находятся заряды, соответственно, $-q$ и $-q$.
    Сделайте рисунок, определите величину напряжённости электрического поля
    в точках $(0; -d)$ и $(2d; 0)$ и укажите её направление.
}
\solutionspace{120pt}

\tasknumber{4}%
\task{%
    Заряд $q_1$ создает в точке $A$ электрическое поле
    по величине равное~$E_1=300\funits{В}{м}$,
    а $q_2$~--- $E_2=400\funits{В}{м}$.
    Угол между векторами $\vect{E_1}$ и $\vect{E_2}$ равен $\alpha$.
    Определите величину суммарного электрического поля в точке $A$,
    создаваемого обоими зарядами $q_1$ и $q_2$.
    Сделайте рисунки и вычислите значение для двух значений угла $\alpha$:
    $\alpha_1=90^\circ$ и $\alpha_2=180^\circ$.
}

\variantsplitter

\addpersonalvariant{Вероника Битерякова}

\tasknumber{1}%
\task{%
    С какой силой взаимодействуют 2 точечных заряда $q_1 = 2\,\text{нКл}$ и $q_2 = 4\,\text{нКл}$,
    находящиеся на расстоянии $l = 5\,\text{см}$?
}
\answer{%
    $
        F
            = k\frac{q_1q_2}{l^2}
            = 9 \cdot 10^{9}\,\frac{\text{Н}\cdot\text{м}^{2}}{\text{Кл}^{2}} \cdot \frac{2\,\text{нКл} *4\,\text{нКл}}{\sqr{ 5\,\text{см} }}
            = \frac{72}{25} \cdot 10^{31}\units{Н}
              \approx {2{,}88} \cdot 10^{31}\units{Н}
    $
}
\solutionspace{120pt}

\tasknumber{2}%
\task{%
    Два одинаковых маленьких проводящих заряженных шарика находятся
    на расстоянии~$l$ друг от друга.
    Заряд первого равен~$+3Q$, второго~---$-4Q$.
    Шарики приводят в соприкосновение, а после опять разводят на то же самое расстояние~$l$.
    \begin{itemize}
        \item Каким стал заряд каждого из шариков?
        \item Определите характер (притяжение или отталкивание) и силу взаимодействия шариков до и после соприкосновения.
        \item Как изменилась сила взаимодействия шариков после соприкосновения?
    \end{itemize}
}
\answer{%
    \begin{align*}
    F &= k\frac{q_1 q_2}{l^2} = k\frac{(+3Q) \cdot (-4Q)}{l^2}, \text{отталкивание}; \\
        q'_1 = q'_2 = \frac{q_1 + q_2}2 = \frac{+3Q -4Q}2 \implies
        F'  &= k\frac{q'_1 q'_2}{l^2}
            = k\frac{\sqr{\frac{(+3Q) + (-4Q)}2}}{l^2},
        \text{отталкивание}.
    \end{align*}
}
\solutionspace{120pt}

\tasknumber{3}%
\task{%
    На координатной плоскости в точках $(-a; 0)$ и $(a; 0)$
    находятся заряды, соответственно, $-Q$ и $+Q$.
    Сделайте рисунок, определите величину напряжённости электрического поля
    в точках $(0; -a)$ и $(-2a; 0)$ и укажите её направление.
}
\solutionspace{120pt}

\tasknumber{4}%
\task{%
    Заряд $q_1$ создает в точке $A$ электрическое поле
    по величине равное~$E_1=300\funits{В}{м}$,
    а $q_2$~--- $E_2=400\funits{В}{м}$.
    Угол между векторами $\vect{E_1}$ и $\vect{E_2}$ равен $\varphi$.
    Определите величину суммарного электрического поля в точке $A$,
    создаваемого обоими зарядами $q_1$ и $q_2$.
    Сделайте рисунки и вычислите значение для двух значений угла $\varphi$:
    $\varphi_1=90^\circ$ и $\varphi_2=180^\circ$.
}

\variantsplitter

\addpersonalvariant{Юлия Буянова}

\tasknumber{1}%
\task{%
    С какой силой взаимодействуют 2 точечных заряда $q_1 = 3\,\text{нКл}$ и $q_2 = 4\,\text{нКл}$,
    находящиеся на расстоянии $d = 6\,\text{см}$?
}
\answer{%
    $
        F
            = k\frac{q_1q_2}{d^2}
            = 9 \cdot 10^{9}\,\frac{\text{Н}\cdot\text{м}^{2}}{\text{Кл}^{2}} \cdot \frac{3\,\text{нКл} *4\,\text{нКл}}{\sqr{ 6\,\text{см} }}
            = 3 \cdot 10^{31}\units{Н}
              \approx {3{,}00} \cdot 10^{31}\units{Н}
    $
}
\solutionspace{120pt}

\tasknumber{2}%
\task{%
    Два одинаковых маленьких проводящих заряженных шарика находятся
    на расстоянии~$r$ друг от друга.
    Заряд первого равен~$+3q$, второго~---$-2q$.
    Шарики приводят в соприкосновение, а после опять разводят на то же самое расстояние~$r$.
    \begin{itemize}
        \item Каким стал заряд каждого из шариков?
        \item Определите характер (притяжение или отталкивание) и силу взаимодействия шариков до и после соприкосновения.
        \item Как изменилась сила взаимодействия шариков после соприкосновения?
    \end{itemize}
}
\answer{%
    \begin{align*}
    F &= k\frac{q_1 q_2}{r^2} = k\frac{(+3q) \cdot (-2q)}{r^2}, \text{отталкивание}; \\
        q'_1 = q'_2 = \frac{q_1 + q_2}2 = \frac{+3q -2q}2 \implies
        F'  &= k\frac{q'_1 q'_2}{r^2}
            = k\frac{\sqr{\frac{(+3q) + (-2q)}2}}{r^2},
        \text{отталкивание}.
    \end{align*}
}
\solutionspace{120pt}

\tasknumber{3}%
\task{%
    На координатной плоскости в точках $(-d; 0)$ и $(d; 0)$
    находятся заряды, соответственно, $+q$ и $-q$.
    Сделайте рисунок, определите величину напряжённости электрического поля
    в точках $(0; -d)$ и $(2d; 0)$ и укажите её направление.
}
\solutionspace{120pt}

\tasknumber{4}%
\task{%
    Заряд $q_1$ создает в точке $A$ электрическое поле
    по величине равное~$E_1=200\funits{В}{м}$,
    а $q_2$~--- $E_2=200\funits{В}{м}$.
    Угол между векторами $\vect{E_1}$ и $\vect{E_2}$ равен $\alpha$.
    Определите величину суммарного электрического поля в точке $A$,
    создаваемого обоими зарядами $q_1$ и $q_2$.
    Сделайте рисунки и вычислите значение для двух значений угла $\alpha$:
    $\alpha_1=0^\circ$ и $\alpha_2=60^\circ$.
}

\variantsplitter

\addpersonalvariant{Пелагея Вдовина}

\tasknumber{1}%
\task{%
    С какой силой взаимодействуют 2 точечных заряда $q_1 = 2\,\text{нКл}$ и $q_2 = 4\,\text{нКл}$,
    находящиеся на расстоянии $l = 2\,\text{см}$?
}
\answer{%
    $
        F
            = k\frac{q_1q_2}{l^2}
            = 9 \cdot 10^{9}\,\frac{\text{Н}\cdot\text{м}^{2}}{\text{Кл}^{2}} \cdot \frac{2\,\text{нКл} *4\,\text{нКл}}{\sqr{ 2\,\text{см} }}
            = 18 \cdot 10^{31}\units{Н}
              \approx {18{,}00} \cdot 10^{31}\units{Н}
    $
}
\solutionspace{120pt}

\tasknumber{2}%
\task{%
    Два одинаковых маленьких проводящих заряженных шарика находятся
    на расстоянии~$l$ друг от друга.
    Заряд первого равен~$-2Q$, второго~---$+4Q$.
    Шарики приводят в соприкосновение, а после опять разводят на то же самое расстояние~$l$.
    \begin{itemize}
        \item Каким стал заряд каждого из шариков?
        \item Определите характер (притяжение или отталкивание) и силу взаимодействия шариков до и после соприкосновения.
        \item Как изменилась сила взаимодействия шариков после соприкосновения?
    \end{itemize}
}
\answer{%
    \begin{align*}
    F &= k\frac{q_1 q_2}{l^2} = k\frac{(-2Q) \cdot (+4Q)}{l^2}, \text{отталкивание}; \\
        q'_1 = q'_2 = \frac{q_1 + q_2}2 = \frac{-2Q + +4Q}2 \implies
        F'  &= k\frac{q'_1 q'_2}{l^2}
            = k\frac{\sqr{\frac{(-2Q) + (+4Q)}2}}{l^2},
        \text{отталкивание}.
    \end{align*}
}
\solutionspace{120pt}

\tasknumber{3}%
\task{%
    На координатной плоскости в точках $(-d; 0)$ и $(d; 0)$
    находятся заряды, соответственно, $-Q$ и $+Q$.
    Сделайте рисунок, определите величину напряжённости электрического поля
    в точках $(0; -d)$ и $(2d; 0)$ и укажите её направление.
}
\solutionspace{120pt}

\tasknumber{4}%
\task{%
    Заряд $q_1$ создает в точке $A$ электрическое поле
    по величине равное~$E_1=250\funits{В}{м}$,
    а $q_2$~--- $E_2=250\funits{В}{м}$.
    Угол между векторами $\vect{E_1}$ и $\vect{E_2}$ равен $\alpha$.
    Определите величину суммарного электрического поля в точке $A$,
    создаваемого обоими зарядами $q_1$ и $q_2$.
    Сделайте рисунки и вычислите значение для двух значений угла $\alpha$:
    $\alpha_1=0^\circ$ и $\alpha_2=60^\circ$.
}

\variantsplitter

\addpersonalvariant{Леонид Викторов}

\tasknumber{1}%
\task{%
    С какой силой взаимодействуют 2 точечных заряда $q_1 = 2\,\text{нКл}$ и $q_2 = 4\,\text{нКл}$,
    находящиеся на расстоянии $d = 5\,\text{см}$?
}
\answer{%
    $
        F
            = k\frac{q_1q_2}{d^2}
            = 9 \cdot 10^{9}\,\frac{\text{Н}\cdot\text{м}^{2}}{\text{Кл}^{2}} \cdot \frac{2\,\text{нКл} *4\,\text{нКл}}{\sqr{ 5\,\text{см} }}
            = \frac{72}{25} \cdot 10^{31}\units{Н}
              \approx {2{,}88} \cdot 10^{31}\units{Н}
    $
}
\solutionspace{120pt}

\tasknumber{2}%
\task{%
    Два одинаковых маленьких проводящих заряженных шарика находятся
    на расстоянии~$r$ друг от друга.
    Заряд первого равен~$-3q$, второго~---$-4q$.
    Шарики приводят в соприкосновение, а после опять разводят на то же самое расстояние~$r$.
    \begin{itemize}
        \item Каким стал заряд каждого из шариков?
        \item Определите характер (притяжение или отталкивание) и силу взаимодействия шариков до и после соприкосновения.
        \item Как изменилась сила взаимодействия шариков после соприкосновения?
    \end{itemize}
}
\answer{%
    \begin{align*}
    F &= k\frac{q_1 q_2}{r^2} = k\frac{(-3q) \cdot (-4q)}{r^2}, \text{отталкивание}; \\
        q'_1 = q'_2 = \frac{q_1 + q_2}2 = \frac{-3q -4q}2 \implies
        F'  &= k\frac{q'_1 q'_2}{r^2}
            = k\frac{\sqr{\frac{(-3q) + (-4q)}2}}{r^2},
        \text{отталкивание}.
    \end{align*}
}
\solutionspace{120pt}

\tasknumber{3}%
\task{%
    На координатной плоскости в точках $(-d; 0)$ и $(d; 0)$
    находятся заряды, соответственно, $+Q$ и $+Q$.
    Сделайте рисунок, определите величину напряжённости электрического поля
    в точках $(0; -d)$ и $(-2d; 0)$ и укажите её направление.
}
\solutionspace{120pt}

\tasknumber{4}%
\task{%
    Заряд $q_1$ создает в точке $A$ электрическое поле
    по величине равное~$E_1=24\funits{В}{м}$,
    а $q_2$~--- $E_2=7\funits{В}{м}$.
    Угол между векторами $\vect{E_1}$ и $\vect{E_2}$ равен $\varphi$.
    Определите величину суммарного электрического поля в точке $A$,
    создаваемого обоими зарядами $q_1$ и $q_2$.
    Сделайте рисунки и вычислите значение для двух значений угла $\varphi$:
    $\varphi_1=90^\circ$ и $\varphi_2=180^\circ$.
}

\variantsplitter

\addpersonalvariant{Фёдор Гнутов}

\tasknumber{1}%
\task{%
    С какой силой взаимодействуют 2 точечных заряда $q_1 = 4\,\text{нКл}$ и $q_2 = 2\,\text{нКл}$,
    находящиеся на расстоянии $r = 3\,\text{см}$?
}
\answer{%
    $
        F
            = k\frac{q_1q_2}{r^2}
            = 9 \cdot 10^{9}\,\frac{\text{Н}\cdot\text{м}^{2}}{\text{Кл}^{2}} \cdot \frac{4\,\text{нКл} *2\,\text{нКл}}{\sqr{ 3\,\text{см} }}
            = 8 \cdot 10^{31}\units{Н}
              \approx {8{,}00} \cdot 10^{31}\units{Н}
    $
}
\solutionspace{120pt}

\tasknumber{2}%
\task{%
    Два одинаковых маленьких проводящих заряженных шарика находятся
    на расстоянии~$r$ друг от друга.
    Заряд первого равен~$+5q$, второго~---$-3q$.
    Шарики приводят в соприкосновение, а после опять разводят на то же самое расстояние~$r$.
    \begin{itemize}
        \item Каким стал заряд каждого из шариков?
        \item Определите характер (притяжение или отталкивание) и силу взаимодействия шариков до и после соприкосновения.
        \item Как изменилась сила взаимодействия шариков после соприкосновения?
    \end{itemize}
}
\answer{%
    \begin{align*}
    F &= k\frac{q_1 q_2}{r^2} = k\frac{(+5q) \cdot (-3q)}{r^2}, \text{отталкивание}; \\
        q'_1 = q'_2 = \frac{q_1 + q_2}2 = \frac{+5q -3q}2 \implies
        F'  &= k\frac{q'_1 q'_2}{r^2}
            = k\frac{\sqr{\frac{(+5q) + (-3q)}2}}{r^2},
        \text{отталкивание}.
    \end{align*}
}
\solutionspace{120pt}

\tasknumber{3}%
\task{%
    На координатной плоскости в точках $(-d; 0)$ и $(d; 0)$
    находятся заряды, соответственно, $-q$ и $-q$.
    Сделайте рисунок, определите величину напряжённости электрического поля
    в точках $(0; -d)$ и $(2d; 0)$ и укажите её направление.
}
\solutionspace{120pt}

\tasknumber{4}%
\task{%
    Заряд $q_1$ создает в точке $A$ электрическое поле
    по величине равное~$E_1=120\funits{В}{м}$,
    а $q_2$~--- $E_2=50\funits{В}{м}$.
    Угол между векторами $\vect{E_1}$ и $\vect{E_2}$ равен $\alpha$.
    Определите величину суммарного электрического поля в точке $A$,
    создаваемого обоими зарядами $q_1$ и $q_2$.
    Сделайте рисунки и вычислите значение для двух значений угла $\alpha$:
    $\alpha_1=90^\circ$ и $\alpha_2=180^\circ$.
}

\variantsplitter

\addpersonalvariant{Илья Гримберг}

\tasknumber{1}%
\task{%
    С какой силой взаимодействуют 2 точечных заряда $q_1 = 4\,\text{нКл}$ и $q_2 = 2\,\text{нКл}$,
    находящиеся на расстоянии $d = 6\,\text{см}$?
}
\answer{%
    $
        F
            = k\frac{q_1q_2}{d^2}
            = 9 \cdot 10^{9}\,\frac{\text{Н}\cdot\text{м}^{2}}{\text{Кл}^{2}} \cdot \frac{4\,\text{нКл} *2\,\text{нКл}}{\sqr{ 6\,\text{см} }}
            = 2 \cdot 10^{31}\units{Н}
              \approx {2{,}00} \cdot 10^{31}\units{Н}
    $
}
\solutionspace{120pt}

\tasknumber{2}%
\task{%
    Два одинаковых маленьких проводящих заряженных шарика находятся
    на расстоянии~$d$ друг от друга.
    Заряд первого равен~$-4q$, второго~---$-5q$.
    Шарики приводят в соприкосновение, а после опять разводят на то же самое расстояние~$d$.
    \begin{itemize}
        \item Каким стал заряд каждого из шариков?
        \item Определите характер (притяжение или отталкивание) и силу взаимодействия шариков до и после соприкосновения.
        \item Как изменилась сила взаимодействия шариков после соприкосновения?
    \end{itemize}
}
\answer{%
    \begin{align*}
    F &= k\frac{q_1 q_2}{d^2} = k\frac{(-4q) \cdot (-5q)}{d^2}, \text{отталкивание}; \\
        q'_1 = q'_2 = \frac{q_1 + q_2}2 = \frac{-4q -5q}2 \implies
        F'  &= k\frac{q'_1 q'_2}{d^2}
            = k\frac{\sqr{\frac{(-4q) + (-5q)}2}}{d^2},
        \text{отталкивание}.
    \end{align*}
}
\solutionspace{120pt}

\tasknumber{3}%
\task{%
    На координатной плоскости в точках $(-d; 0)$ и $(d; 0)$
    находятся заряды, соответственно, $+q$ и $-q$.
    Сделайте рисунок, определите величину напряжённости электрического поля
    в точках $(0; -d)$ и $(-2d; 0)$ и укажите её направление.
}
\solutionspace{120pt}

\tasknumber{4}%
\task{%
    Заряд $q_1$ создает в точке $A$ электрическое поле
    по величине равное~$E_1=24\funits{В}{м}$,
    а $q_2$~--- $E_2=7\funits{В}{м}$.
    Угол между векторами $\vect{E_1}$ и $\vect{E_2}$ равен $\varphi$.
    Определите величину суммарного электрического поля в точке $A$,
    создаваемого обоими зарядами $q_1$ и $q_2$.
    Сделайте рисунки и вычислите значение для двух значений угла $\varphi$:
    $\varphi_1=90^\circ$ и $\varphi_2=180^\circ$.
}

\variantsplitter

\addpersonalvariant{Иван Гурьянов}

\tasknumber{1}%
\task{%
    С какой силой взаимодействуют 2 точечных заряда $q_1 = 4\,\text{нКл}$ и $q_2 = 2\,\text{нКл}$,
    находящиеся на расстоянии $d = 2\,\text{см}$?
}
\answer{%
    $
        F
            = k\frac{q_1q_2}{d^2}
            = 9 \cdot 10^{9}\,\frac{\text{Н}\cdot\text{м}^{2}}{\text{Кл}^{2}} \cdot \frac{4\,\text{нКл} *2\,\text{нКл}}{\sqr{ 2\,\text{см} }}
            = 18 \cdot 10^{31}\units{Н}
              \approx {18{,}00} \cdot 10^{31}\units{Н}
    $
}
\solutionspace{120pt}

\tasknumber{2}%
\task{%
    Два одинаковых маленьких проводящих заряженных шарика находятся
    на расстоянии~$l$ друг от друга.
    Заряд первого равен~$+3Q$, второго~---$-2Q$.
    Шарики приводят в соприкосновение, а после опять разводят на то же самое расстояние~$l$.
    \begin{itemize}
        \item Каким стал заряд каждого из шариков?
        \item Определите характер (притяжение или отталкивание) и силу взаимодействия шариков до и после соприкосновения.
        \item Как изменилась сила взаимодействия шариков после соприкосновения?
    \end{itemize}
}
\answer{%
    \begin{align*}
    F &= k\frac{q_1 q_2}{l^2} = k\frac{(+3Q) \cdot (-2Q)}{l^2}, \text{отталкивание}; \\
        q'_1 = q'_2 = \frac{q_1 + q_2}2 = \frac{+3Q -2Q}2 \implies
        F'  &= k\frac{q'_1 q'_2}{l^2}
            = k\frac{\sqr{\frac{(+3Q) + (-2Q)}2}}{l^2},
        \text{отталкивание}.
    \end{align*}
}
\solutionspace{120pt}

\tasknumber{3}%
\task{%
    На координатной плоскости в точках $(-r; 0)$ и $(r; 0)$
    находятся заряды, соответственно, $-Q$ и $+Q$.
    Сделайте рисунок, определите величину напряжённости электрического поля
    в точках $(0; r)$ и $(2r; 0)$ и укажите её направление.
}
\solutionspace{120pt}

\tasknumber{4}%
\task{%
    Заряд $q_1$ создает в точке $A$ электрическое поле
    по величине равное~$E_1=200\funits{В}{м}$,
    а $q_2$~--- $E_2=200\funits{В}{м}$.
    Угол между векторами $\vect{E_1}$ и $\vect{E_2}$ равен $\varphi$.
    Определите величину суммарного электрического поля в точке $A$,
    создаваемого обоими зарядами $q_1$ и $q_2$.
    Сделайте рисунки и вычислите значение для двух значений угла $\varphi$:
    $\varphi_1=0^\circ$ и $\varphi_2=60^\circ$.
}

\variantsplitter

\addpersonalvariant{Артём Денежкин}

\tasknumber{1}%
\task{%
    С какой силой взаимодействуют 2 точечных заряда $q_1 = 4\,\text{нКл}$ и $q_2 = 2\,\text{нКл}$,
    находящиеся на расстоянии $d = 5\,\text{см}$?
}
\answer{%
    $
        F
            = k\frac{q_1q_2}{d^2}
            = 9 \cdot 10^{9}\,\frac{\text{Н}\cdot\text{м}^{2}}{\text{Кл}^{2}} \cdot \frac{4\,\text{нКл} *2\,\text{нКл}}{\sqr{ 5\,\text{см} }}
            = \frac{72}{25} \cdot 10^{31}\units{Н}
              \approx {2{,}88} \cdot 10^{31}\units{Н}
    $
}
\solutionspace{120pt}

\tasknumber{2}%
\task{%
    Два одинаковых маленьких проводящих заряженных шарика находятся
    на расстоянии~$r$ друг от друга.
    Заряд первого равен~$+3Q$, второго~---$+5Q$.
    Шарики приводят в соприкосновение, а после опять разводят на то же самое расстояние~$r$.
    \begin{itemize}
        \item Каким стал заряд каждого из шариков?
        \item Определите характер (притяжение или отталкивание) и силу взаимодействия шариков до и после соприкосновения.
        \item Как изменилась сила взаимодействия шариков после соприкосновения?
    \end{itemize}
}
\answer{%
    \begin{align*}
    F &= k\frac{q_1 q_2}{r^2} = k\frac{(+3Q) \cdot (+5Q)}{r^2}, \text{отталкивание}; \\
        q'_1 = q'_2 = \frac{q_1 + q_2}2 = \frac{+3Q + +5Q}2 \implies
        F'  &= k\frac{q'_1 q'_2}{r^2}
            = k\frac{\sqr{\frac{(+3Q) + (+5Q)}2}}{r^2},
        \text{отталкивание}.
    \end{align*}
}
\solutionspace{120pt}

\tasknumber{3}%
\task{%
    На координатной плоскости в точках $(-a; 0)$ и $(a; 0)$
    находятся заряды, соответственно, $-q$ и $-q$.
    Сделайте рисунок, определите величину напряжённости электрического поля
    в точках $(0; -a)$ и $(-2a; 0)$ и укажите её направление.
}
\solutionspace{120pt}

\tasknumber{4}%
\task{%
    Заряд $q_1$ создает в точке $A$ электрическое поле
    по величине равное~$E_1=300\funits{В}{м}$,
    а $q_2$~--- $E_2=400\funits{В}{м}$.
    Угол между векторами $\vect{E_1}$ и $\vect{E_2}$ равен $\alpha$.
    Определите величину суммарного электрического поля в точке $A$,
    создаваемого обоими зарядами $q_1$ и $q_2$.
    Сделайте рисунки и вычислите значение для двух значений угла $\alpha$:
    $\alpha_1=90^\circ$ и $\alpha_2=180^\circ$.
}

\variantsplitter

\addpersonalvariant{Виктор Жилин}

\tasknumber{1}%
\task{%
    С какой силой взаимодействуют 2 точечных заряда $q_1 = 2\,\text{нКл}$ и $q_2 = 3\,\text{нКл}$,
    находящиеся на расстоянии $l = 6\,\text{см}$?
}
\answer{%
    $
        F
            = k\frac{q_1q_2}{l^2}
            = 9 \cdot 10^{9}\,\frac{\text{Н}\cdot\text{м}^{2}}{\text{Кл}^{2}} \cdot \frac{2\,\text{нКл} *3\,\text{нКл}}{\sqr{ 6\,\text{см} }}
            = \frac32 \cdot 10^{31}\units{Н}
              \approx {1{,}50} \cdot 10^{31}\units{Н}
    $
}
\solutionspace{120pt}

\tasknumber{2}%
\task{%
    Два одинаковых маленьких проводящих заряженных шарика находятся
    на расстоянии~$r$ друг от друга.
    Заряд первого равен~$+5Q$, второго~---$+2Q$.
    Шарики приводят в соприкосновение, а после опять разводят на то же самое расстояние~$r$.
    \begin{itemize}
        \item Каким стал заряд каждого из шариков?
        \item Определите характер (притяжение или отталкивание) и силу взаимодействия шариков до и после соприкосновения.
        \item Как изменилась сила взаимодействия шариков после соприкосновения?
    \end{itemize}
}
\answer{%
    \begin{align*}
    F &= k\frac{q_1 q_2}{r^2} = k\frac{(+5Q) \cdot (+2Q)}{r^2}, \text{отталкивание}; \\
        q'_1 = q'_2 = \frac{q_1 + q_2}2 = \frac{+5Q + +2Q}2 \implies
        F'  &= k\frac{q'_1 q'_2}{r^2}
            = k\frac{\sqr{\frac{(+5Q) + (+2Q)}2}}{r^2},
        \text{отталкивание}.
    \end{align*}
}
\solutionspace{120pt}

\tasknumber{3}%
\task{%
    На координатной плоскости в точках $(-r; 0)$ и $(r; 0)$
    находятся заряды, соответственно, $+Q$ и $+Q$.
    Сделайте рисунок, определите величину напряжённости электрического поля
    в точках $(0; r)$ и $(2r; 0)$ и укажите её направление.
}
\solutionspace{120pt}

\tasknumber{4}%
\task{%
    Заряд $q_1$ создает в точке $A$ электрическое поле
    по величине равное~$E_1=7\funits{В}{м}$,
    а $q_2$~--- $E_2=24\funits{В}{м}$.
    Угол между векторами $\vect{E_1}$ и $\vect{E_2}$ равен $\alpha$.
    Определите величину суммарного электрического поля в точке $A$,
    создаваемого обоими зарядами $q_1$ и $q_2$.
    Сделайте рисунки и вычислите значение для двух значений угла $\alpha$:
    $\alpha_1=0^\circ$ и $\alpha_2=90^\circ$.
}

\variantsplitter

\addpersonalvariant{Дмитрий Иванов}

\tasknumber{1}%
\task{%
    С какой силой взаимодействуют 2 точечных заряда $q_1 = 3\,\text{нКл}$ и $q_2 = 2\,\text{нКл}$,
    находящиеся на расстоянии $r = 2\,\text{см}$?
}
\answer{%
    $
        F
            = k\frac{q_1q_2}{r^2}
            = 9 \cdot 10^{9}\,\frac{\text{Н}\cdot\text{м}^{2}}{\text{Кл}^{2}} \cdot \frac{3\,\text{нКл} *2\,\text{нКл}}{\sqr{ 2\,\text{см} }}
            = \frac{27}2 \cdot 10^{31}\units{Н}
              \approx {13{,}50} \cdot 10^{31}\units{Н}
    $
}
\solutionspace{120pt}

\tasknumber{2}%
\task{%
    Два одинаковых маленьких проводящих заряженных шарика находятся
    на расстоянии~$d$ друг от друга.
    Заряд первого равен~$-5q$, второго~---$+2q$.
    Шарики приводят в соприкосновение, а после опять разводят на то же самое расстояние~$d$.
    \begin{itemize}
        \item Каким стал заряд каждого из шариков?
        \item Определите характер (притяжение или отталкивание) и силу взаимодействия шариков до и после соприкосновения.
        \item Как изменилась сила взаимодействия шариков после соприкосновения?
    \end{itemize}
}
\answer{%
    \begin{align*}
    F &= k\frac{q_1 q_2}{d^2} = k\frac{(-5q) \cdot (+2q)}{d^2}, \text{отталкивание}; \\
        q'_1 = q'_2 = \frac{q_1 + q_2}2 = \frac{-5q + +2q}2 \implies
        F'  &= k\frac{q'_1 q'_2}{d^2}
            = k\frac{\sqr{\frac{(-5q) + (+2q)}2}}{d^2},
        \text{отталкивание}.
    \end{align*}
}
\solutionspace{120pt}

\tasknumber{3}%
\task{%
    На координатной плоскости в точках $(-r; 0)$ и $(r; 0)$
    находятся заряды, соответственно, $-Q$ и $+Q$.
    Сделайте рисунок, определите величину напряжённости электрического поля
    в точках $(0; -r)$ и $(-2r; 0)$ и укажите её направление.
}
\solutionspace{120pt}

\tasknumber{4}%
\task{%
    Заряд $q_1$ создает в точке $A$ электрическое поле
    по величине равное~$E_1=200\funits{В}{м}$,
    а $q_2$~--- $E_2=200\funits{В}{м}$.
    Угол между векторами $\vect{E_1}$ и $\vect{E_2}$ равен $\varphi$.
    Определите величину суммарного электрического поля в точке $A$,
    создаваемого обоими зарядами $q_1$ и $q_2$.
    Сделайте рисунки и вычислите значение для двух значений угла $\varphi$:
    $\varphi_1=0^\circ$ и $\varphi_2=60^\circ$.
}

\variantsplitter

\addpersonalvariant{Олег Климов}

\tasknumber{1}%
\task{%
    С какой силой взаимодействуют 2 точечных заряда $q_1 = 3\,\text{нКл}$ и $q_2 = 2\,\text{нКл}$,
    находящиеся на расстоянии $l = 6\,\text{см}$?
}
\answer{%
    $
        F
            = k\frac{q_1q_2}{l^2}
            = 9 \cdot 10^{9}\,\frac{\text{Н}\cdot\text{м}^{2}}{\text{Кл}^{2}} \cdot \frac{3\,\text{нКл} *2\,\text{нКл}}{\sqr{ 6\,\text{см} }}
            = \frac32 \cdot 10^{31}\units{Н}
              \approx {1{,}50} \cdot 10^{31}\units{Н}
    $
}
\solutionspace{120pt}

\tasknumber{2}%
\task{%
    Два одинаковых маленьких проводящих заряженных шарика находятся
    на расстоянии~$d$ друг от друга.
    Заряд первого равен~$+3q$, второго~---$+2q$.
    Шарики приводят в соприкосновение, а после опять разводят на то же самое расстояние~$d$.
    \begin{itemize}
        \item Каким стал заряд каждого из шариков?
        \item Определите характер (притяжение или отталкивание) и силу взаимодействия шариков до и после соприкосновения.
        \item Как изменилась сила взаимодействия шариков после соприкосновения?
    \end{itemize}
}
\answer{%
    \begin{align*}
    F &= k\frac{q_1 q_2}{d^2} = k\frac{(+3q) \cdot (+2q)}{d^2}, \text{отталкивание}; \\
        q'_1 = q'_2 = \frac{q_1 + q_2}2 = \frac{+3q + +2q}2 \implies
        F'  &= k\frac{q'_1 q'_2}{d^2}
            = k\frac{\sqr{\frac{(+3q) + (+2q)}2}}{d^2},
        \text{отталкивание}.
    \end{align*}
}
\solutionspace{120pt}

\tasknumber{3}%
\task{%
    На координатной плоскости в точках $(-r; 0)$ и $(r; 0)$
    находятся заряды, соответственно, $-q$ и $-q$.
    Сделайте рисунок, определите величину напряжённости электрического поля
    в точках $(0; -r)$ и $(-2r; 0)$ и укажите её направление.
}
\solutionspace{120pt}

\tasknumber{4}%
\task{%
    Заряд $q_1$ создает в точке $A$ электрическое поле
    по величине равное~$E_1=500\funits{В}{м}$,
    а $q_2$~--- $E_2=500\funits{В}{м}$.
    Угол между векторами $\vect{E_1}$ и $\vect{E_2}$ равен $\varphi$.
    Определите величину суммарного электрического поля в точке $A$,
    создаваемого обоими зарядами $q_1$ и $q_2$.
    Сделайте рисунки и вычислите значение для двух значений угла $\varphi$:
    $\varphi_1=0^\circ$ и $\varphi_2=120^\circ$.
}

\variantsplitter

\addpersonalvariant{Анна Ковалева}

\tasknumber{1}%
\task{%
    С какой силой взаимодействуют 2 точечных заряда $q_1 = 4\,\text{нКл}$ и $q_2 = 2\,\text{нКл}$,
    находящиеся на расстоянии $l = 5\,\text{см}$?
}
\answer{%
    $
        F
            = k\frac{q_1q_2}{l^2}
            = 9 \cdot 10^{9}\,\frac{\text{Н}\cdot\text{м}^{2}}{\text{Кл}^{2}} \cdot \frac{4\,\text{нКл} *2\,\text{нКл}}{\sqr{ 5\,\text{см} }}
            = \frac{72}{25} \cdot 10^{31}\units{Н}
              \approx {2{,}88} \cdot 10^{31}\units{Н}
    $
}
\solutionspace{120pt}

\tasknumber{2}%
\task{%
    Два одинаковых маленьких проводящих заряженных шарика находятся
    на расстоянии~$l$ друг от друга.
    Заряд первого равен~$-5q$, второго~---$+4q$.
    Шарики приводят в соприкосновение, а после опять разводят на то же самое расстояние~$l$.
    \begin{itemize}
        \item Каким стал заряд каждого из шариков?
        \item Определите характер (притяжение или отталкивание) и силу взаимодействия шариков до и после соприкосновения.
        \item Как изменилась сила взаимодействия шариков после соприкосновения?
    \end{itemize}
}
\answer{%
    \begin{align*}
    F &= k\frac{q_1 q_2}{l^2} = k\frac{(-5q) \cdot (+4q)}{l^2}, \text{отталкивание}; \\
        q'_1 = q'_2 = \frac{q_1 + q_2}2 = \frac{-5q + +4q}2 \implies
        F'  &= k\frac{q'_1 q'_2}{l^2}
            = k\frac{\sqr{\frac{(-5q) + (+4q)}2}}{l^2},
        \text{отталкивание}.
    \end{align*}
}
\solutionspace{120pt}

\tasknumber{3}%
\task{%
    На координатной плоскости в точках $(-r; 0)$ и $(r; 0)$
    находятся заряды, соответственно, $+q$ и $-q$.
    Сделайте рисунок, определите величину напряжённости электрического поля
    в точках $(0; r)$ и $(2r; 0)$ и укажите её направление.
}
\solutionspace{120pt}

\tasknumber{4}%
\task{%
    Заряд $q_1$ создает в точке $A$ электрическое поле
    по величине равное~$E_1=300\funits{В}{м}$,
    а $q_2$~--- $E_2=400\funits{В}{м}$.
    Угол между векторами $\vect{E_1}$ и $\vect{E_2}$ равен $\alpha$.
    Определите величину суммарного электрического поля в точке $A$,
    создаваемого обоими зарядами $q_1$ и $q_2$.
    Сделайте рисунки и вычислите значение для двух значений угла $\alpha$:
    $\alpha_1=0^\circ$ и $\alpha_2=90^\circ$.
}

\variantsplitter

\addpersonalvariant{Глеб Ковылин}

\tasknumber{1}%
\task{%
    С какой силой взаимодействуют 2 точечных заряда $q_1 = 3\,\text{нКл}$ и $q_2 = 2\,\text{нКл}$,
    находящиеся на расстоянии $r = 3\,\text{см}$?
}
\answer{%
    $
        F
            = k\frac{q_1q_2}{r^2}
            = 9 \cdot 10^{9}\,\frac{\text{Н}\cdot\text{м}^{2}}{\text{Кл}^{2}} \cdot \frac{3\,\text{нКл} *2\,\text{нКл}}{\sqr{ 3\,\text{см} }}
            = 6 \cdot 10^{31}\units{Н}
              \approx {6{,}00} \cdot 10^{31}\units{Н}
    $
}
\solutionspace{120pt}

\tasknumber{2}%
\task{%
    Два одинаковых маленьких проводящих заряженных шарика находятся
    на расстоянии~$d$ друг от друга.
    Заряд первого равен~$-4q$, второго~---$+2q$.
    Шарики приводят в соприкосновение, а после опять разводят на то же самое расстояние~$d$.
    \begin{itemize}
        \item Каким стал заряд каждого из шариков?
        \item Определите характер (притяжение или отталкивание) и силу взаимодействия шариков до и после соприкосновения.
        \item Как изменилась сила взаимодействия шариков после соприкосновения?
    \end{itemize}
}
\answer{%
    \begin{align*}
    F &= k\frac{q_1 q_2}{d^2} = k\frac{(-4q) \cdot (+2q)}{d^2}, \text{отталкивание}; \\
        q'_1 = q'_2 = \frac{q_1 + q_2}2 = \frac{-4q + +2q}2 \implies
        F'  &= k\frac{q'_1 q'_2}{d^2}
            = k\frac{\sqr{\frac{(-4q) + (+2q)}2}}{d^2},
        \text{отталкивание}.
    \end{align*}
}
\solutionspace{120pt}

\tasknumber{3}%
\task{%
    На координатной плоскости в точках $(-r; 0)$ и $(r; 0)$
    находятся заряды, соответственно, $+q$ и $-q$.
    Сделайте рисунок, определите величину напряжённости электрического поля
    в точках $(0; r)$ и $(-2r; 0)$ и укажите её направление.
}
\solutionspace{120pt}

\tasknumber{4}%
\task{%
    Заряд $q_1$ создает в точке $A$ электрическое поле
    по величине равное~$E_1=72\funits{В}{м}$,
    а $q_2$~--- $E_2=72\funits{В}{м}$.
    Угол между векторами $\vect{E_1}$ и $\vect{E_2}$ равен $\varphi$.
    Определите величину суммарного электрического поля в точке $A$,
    создаваемого обоими зарядами $q_1$ и $q_2$.
    Сделайте рисунки и вычислите значение для двух значений угла $\varphi$:
    $\varphi_1=0^\circ$ и $\varphi_2=120^\circ$.
}

\variantsplitter

\addpersonalvariant{Даниил Космынин}

\tasknumber{1}%
\task{%
    С какой силой взаимодействуют 2 точечных заряда $q_1 = 2\,\text{нКл}$ и $q_2 = 4\,\text{нКл}$,
    находящиеся на расстоянии $d = 5\,\text{см}$?
}
\answer{%
    $
        F
            = k\frac{q_1q_2}{d^2}
            = 9 \cdot 10^{9}\,\frac{\text{Н}\cdot\text{м}^{2}}{\text{Кл}^{2}} \cdot \frac{2\,\text{нКл} *4\,\text{нКл}}{\sqr{ 5\,\text{см} }}
            = \frac{72}{25} \cdot 10^{31}\units{Н}
              \approx {2{,}88} \cdot 10^{31}\units{Н}
    $
}
\solutionspace{120pt}

\tasknumber{2}%
\task{%
    Два одинаковых маленьких проводящих заряженных шарика находятся
    на расстоянии~$l$ друг от друга.
    Заряд первого равен~$+5q$, второго~---$-3q$.
    Шарики приводят в соприкосновение, а после опять разводят на то же самое расстояние~$l$.
    \begin{itemize}
        \item Каким стал заряд каждого из шариков?
        \item Определите характер (притяжение или отталкивание) и силу взаимодействия шариков до и после соприкосновения.
        \item Как изменилась сила взаимодействия шариков после соприкосновения?
    \end{itemize}
}
\answer{%
    \begin{align*}
    F &= k\frac{q_1 q_2}{l^2} = k\frac{(+5q) \cdot (-3q)}{l^2}, \text{отталкивание}; \\
        q'_1 = q'_2 = \frac{q_1 + q_2}2 = \frac{+5q -3q}2 \implies
        F'  &= k\frac{q'_1 q'_2}{l^2}
            = k\frac{\sqr{\frac{(+5q) + (-3q)}2}}{l^2},
        \text{отталкивание}.
    \end{align*}
}
\solutionspace{120pt}

\tasknumber{3}%
\task{%
    На координатной плоскости в точках $(-a; 0)$ и $(a; 0)$
    находятся заряды, соответственно, $+Q$ и $+Q$.
    Сделайте рисунок, определите величину напряжённости электрического поля
    в точках $(0; -a)$ и $(2a; 0)$ и укажите её направление.
}
\solutionspace{120pt}

\tasknumber{4}%
\task{%
    Заряд $q_1$ создает в точке $A$ электрическое поле
    по величине равное~$E_1=24\funits{В}{м}$,
    а $q_2$~--- $E_2=7\funits{В}{м}$.
    Угол между векторами $\vect{E_1}$ и $\vect{E_2}$ равен $\varphi$.
    Определите величину суммарного электрического поля в точке $A$,
    создаваемого обоими зарядами $q_1$ и $q_2$.
    Сделайте рисунки и вычислите значение для двух значений угла $\varphi$:
    $\varphi_1=90^\circ$ и $\varphi_2=180^\circ$.
}

\variantsplitter

\addpersonalvariant{Алина Леоничева}

\tasknumber{1}%
\task{%
    С какой силой взаимодействуют 2 точечных заряда $q_1 = 3\,\text{нКл}$ и $q_2 = 2\,\text{нКл}$,
    находящиеся на расстоянии $l = 2\,\text{см}$?
}
\answer{%
    $
        F
            = k\frac{q_1q_2}{l^2}
            = 9 \cdot 10^{9}\,\frac{\text{Н}\cdot\text{м}^{2}}{\text{Кл}^{2}} \cdot \frac{3\,\text{нКл} *2\,\text{нКл}}{\sqr{ 2\,\text{см} }}
            = \frac{27}2 \cdot 10^{31}\units{Н}
              \approx {13{,}50} \cdot 10^{31}\units{Н}
    $
}
\solutionspace{120pt}

\tasknumber{2}%
\task{%
    Два одинаковых маленьких проводящих заряженных шарика находятся
    на расстоянии~$l$ друг от друга.
    Заряд первого равен~$+4Q$, второго~---$+3Q$.
    Шарики приводят в соприкосновение, а после опять разводят на то же самое расстояние~$l$.
    \begin{itemize}
        \item Каким стал заряд каждого из шариков?
        \item Определите характер (притяжение или отталкивание) и силу взаимодействия шариков до и после соприкосновения.
        \item Как изменилась сила взаимодействия шариков после соприкосновения?
    \end{itemize}
}
\answer{%
    \begin{align*}
    F &= k\frac{q_1 q_2}{l^2} = k\frac{(+4Q) \cdot (+3Q)}{l^2}, \text{отталкивание}; \\
        q'_1 = q'_2 = \frac{q_1 + q_2}2 = \frac{+4Q + +3Q}2 \implies
        F'  &= k\frac{q'_1 q'_2}{l^2}
            = k\frac{\sqr{\frac{(+4Q) + (+3Q)}2}}{l^2},
        \text{отталкивание}.
    \end{align*}
}
\solutionspace{120pt}

\tasknumber{3}%
\task{%
    На координатной плоскости в точках $(-d; 0)$ и $(d; 0)$
    находятся заряды, соответственно, $+q$ и $-q$.
    Сделайте рисунок, определите величину напряжённости электрического поля
    в точках $(0; -d)$ и $(2d; 0)$ и укажите её направление.
}
\solutionspace{120pt}

\tasknumber{4}%
\task{%
    Заряд $q_1$ создает в точке $A$ электрическое поле
    по величине равное~$E_1=300\funits{В}{м}$,
    а $q_2$~--- $E_2=400\funits{В}{м}$.
    Угол между векторами $\vect{E_1}$ и $\vect{E_2}$ равен $\alpha$.
    Определите величину суммарного электрического поля в точке $A$,
    создаваемого обоими зарядами $q_1$ и $q_2$.
    Сделайте рисунки и вычислите значение для двух значений угла $\alpha$:
    $\alpha_1=0^\circ$ и $\alpha_2=90^\circ$.
}

\variantsplitter

\addpersonalvariant{Ирина Лин}

\tasknumber{1}%
\task{%
    С какой силой взаимодействуют 2 точечных заряда $q_1 = 4\,\text{нКл}$ и $q_2 = 3\,\text{нКл}$,
    находящиеся на расстоянии $d = 3\,\text{см}$?
}
\answer{%
    $
        F
            = k\frac{q_1q_2}{d^2}
            = 9 \cdot 10^{9}\,\frac{\text{Н}\cdot\text{м}^{2}}{\text{Кл}^{2}} \cdot \frac{4\,\text{нКл} *3\,\text{нКл}}{\sqr{ 3\,\text{см} }}
            = 12 \cdot 10^{31}\units{Н}
              \approx {12{,}00} \cdot 10^{31}\units{Н}
    $
}
\solutionspace{120pt}

\tasknumber{2}%
\task{%
    Два одинаковых маленьких проводящих заряженных шарика находятся
    на расстоянии~$d$ друг от друга.
    Заряд первого равен~$-3q$, второго~---$-5q$.
    Шарики приводят в соприкосновение, а после опять разводят на то же самое расстояние~$d$.
    \begin{itemize}
        \item Каким стал заряд каждого из шариков?
        \item Определите характер (притяжение или отталкивание) и силу взаимодействия шариков до и после соприкосновения.
        \item Как изменилась сила взаимодействия шариков после соприкосновения?
    \end{itemize}
}
\answer{%
    \begin{align*}
    F &= k\frac{q_1 q_2}{d^2} = k\frac{(-3q) \cdot (-5q)}{d^2}, \text{отталкивание}; \\
        q'_1 = q'_2 = \frac{q_1 + q_2}2 = \frac{-3q -5q}2 \implies
        F'  &= k\frac{q'_1 q'_2}{d^2}
            = k\frac{\sqr{\frac{(-3q) + (-5q)}2}}{d^2},
        \text{отталкивание}.
    \end{align*}
}
\solutionspace{120pt}

\tasknumber{3}%
\task{%
    На координатной плоскости в точках $(-a; 0)$ и $(a; 0)$
    находятся заряды, соответственно, $-Q$ и $+Q$.
    Сделайте рисунок, определите величину напряжённости электрического поля
    в точках $(0; a)$ и $(2a; 0)$ и укажите её направление.
}
\solutionspace{120pt}

\tasknumber{4}%
\task{%
    Заряд $q_1$ создает в точке $A$ электрическое поле
    по величине равное~$E_1=300\funits{В}{м}$,
    а $q_2$~--- $E_2=400\funits{В}{м}$.
    Угол между векторами $\vect{E_1}$ и $\vect{E_2}$ равен $\alpha$.
    Определите величину суммарного электрического поля в точке $A$,
    создаваемого обоими зарядами $q_1$ и $q_2$.
    Сделайте рисунки и вычислите значение для двух значений угла $\alpha$:
    $\alpha_1=90^\circ$ и $\alpha_2=180^\circ$.
}

\variantsplitter

\addpersonalvariant{Олег Мальцев}

\tasknumber{1}%
\task{%
    С какой силой взаимодействуют 2 точечных заряда $q_1 = 4\,\text{нКл}$ и $q_2 = 2\,\text{нКл}$,
    находящиеся на расстоянии $r = 3\,\text{см}$?
}
\answer{%
    $
        F
            = k\frac{q_1q_2}{r^2}
            = 9 \cdot 10^{9}\,\frac{\text{Н}\cdot\text{м}^{2}}{\text{Кл}^{2}} \cdot \frac{4\,\text{нКл} *2\,\text{нКл}}{\sqr{ 3\,\text{см} }}
            = 8 \cdot 10^{31}\units{Н}
              \approx {8{,}00} \cdot 10^{31}\units{Н}
    $
}
\solutionspace{120pt}

\tasknumber{2}%
\task{%
    Два одинаковых маленьких проводящих заряженных шарика находятся
    на расстоянии~$r$ друг от друга.
    Заряд первого равен~$-4q$, второго~---$+3q$.
    Шарики приводят в соприкосновение, а после опять разводят на то же самое расстояние~$r$.
    \begin{itemize}
        \item Каким стал заряд каждого из шариков?
        \item Определите характер (притяжение или отталкивание) и силу взаимодействия шариков до и после соприкосновения.
        \item Как изменилась сила взаимодействия шариков после соприкосновения?
    \end{itemize}
}
\answer{%
    \begin{align*}
    F &= k\frac{q_1 q_2}{r^2} = k\frac{(-4q) \cdot (+3q)}{r^2}, \text{отталкивание}; \\
        q'_1 = q'_2 = \frac{q_1 + q_2}2 = \frac{-4q + +3q}2 \implies
        F'  &= k\frac{q'_1 q'_2}{r^2}
            = k\frac{\sqr{\frac{(-4q) + (+3q)}2}}{r^2},
        \text{отталкивание}.
    \end{align*}
}
\solutionspace{120pt}

\tasknumber{3}%
\task{%
    На координатной плоскости в точках $(-a; 0)$ и $(a; 0)$
    находятся заряды, соответственно, $-Q$ и $+Q$.
    Сделайте рисунок, определите величину напряжённости электрического поля
    в точках $(0; a)$ и $(2a; 0)$ и укажите её направление.
}
\solutionspace{120pt}

\tasknumber{4}%
\task{%
    Заряд $q_1$ создает в точке $A$ электрическое поле
    по величине равное~$E_1=120\funits{В}{м}$,
    а $q_2$~--- $E_2=50\funits{В}{м}$.
    Угол между векторами $\vect{E_1}$ и $\vect{E_2}$ равен $\varphi$.
    Определите величину суммарного электрического поля в точке $A$,
    создаваемого обоими зарядами $q_1$ и $q_2$.
    Сделайте рисунки и вычислите значение для двух значений угла $\varphi$:
    $\varphi_1=90^\circ$ и $\varphi_2=180^\circ$.
}

\variantsplitter

\addpersonalvariant{Ислам Мунаев}

\tasknumber{1}%
\task{%
    С какой силой взаимодействуют 2 точечных заряда $q_1 = 3\,\text{нКл}$ и $q_2 = 2\,\text{нКл}$,
    находящиеся на расстоянии $d = 5\,\text{см}$?
}
\answer{%
    $
        F
            = k\frac{q_1q_2}{d^2}
            = 9 \cdot 10^{9}\,\frac{\text{Н}\cdot\text{м}^{2}}{\text{Кл}^{2}} \cdot \frac{3\,\text{нКл} *2\,\text{нКл}}{\sqr{ 5\,\text{см} }}
            = \frac{54}{25} \cdot 10^{31}\units{Н}
              \approx {2{,}16} \cdot 10^{31}\units{Н}
    $
}
\solutionspace{120pt}

\tasknumber{2}%
\task{%
    Два одинаковых маленьких проводящих заряженных шарика находятся
    на расстоянии~$d$ друг от друга.
    Заряд первого равен~$-4Q$, второго~---$+5Q$.
    Шарики приводят в соприкосновение, а после опять разводят на то же самое расстояние~$d$.
    \begin{itemize}
        \item Каким стал заряд каждого из шариков?
        \item Определите характер (притяжение или отталкивание) и силу взаимодействия шариков до и после соприкосновения.
        \item Как изменилась сила взаимодействия шариков после соприкосновения?
    \end{itemize}
}
\answer{%
    \begin{align*}
    F &= k\frac{q_1 q_2}{d^2} = k\frac{(-4Q) \cdot (+5Q)}{d^2}, \text{отталкивание}; \\
        q'_1 = q'_2 = \frac{q_1 + q_2}2 = \frac{-4Q + +5Q}2 \implies
        F'  &= k\frac{q'_1 q'_2}{d^2}
            = k\frac{\sqr{\frac{(-4Q) + (+5Q)}2}}{d^2},
        \text{отталкивание}.
    \end{align*}
}
\solutionspace{120pt}

\tasknumber{3}%
\task{%
    На координатной плоскости в точках $(-a; 0)$ и $(a; 0)$
    находятся заряды, соответственно, $+Q$ и $+Q$.
    Сделайте рисунок, определите величину напряжённости электрического поля
    в точках $(0; a)$ и $(2a; 0)$ и укажите её направление.
}
\solutionspace{120pt}

\tasknumber{4}%
\task{%
    Заряд $q_1$ создает в точке $A$ электрическое поле
    по величине равное~$E_1=500\funits{В}{м}$,
    а $q_2$~--- $E_2=500\funits{В}{м}$.
    Угол между векторами $\vect{E_1}$ и $\vect{E_2}$ равен $\varphi$.
    Определите величину суммарного электрического поля в точке $A$,
    создаваемого обоими зарядами $q_1$ и $q_2$.
    Сделайте рисунки и вычислите значение для двух значений угла $\varphi$:
    $\varphi_1=0^\circ$ и $\varphi_2=120^\circ$.
}

\variantsplitter

\addpersonalvariant{Александр Наумов}

\tasknumber{1}%
\task{%
    С какой силой взаимодействуют 2 точечных заряда $q_1 = 3\,\text{нКл}$ и $q_2 = 2\,\text{нКл}$,
    находящиеся на расстоянии $r = 5\,\text{см}$?
}
\answer{%
    $
        F
            = k\frac{q_1q_2}{r^2}
            = 9 \cdot 10^{9}\,\frac{\text{Н}\cdot\text{м}^{2}}{\text{Кл}^{2}} \cdot \frac{3\,\text{нКл} *2\,\text{нКл}}{\sqr{ 5\,\text{см} }}
            = \frac{54}{25} \cdot 10^{31}\units{Н}
              \approx {2{,}16} \cdot 10^{31}\units{Н}
    $
}
\solutionspace{120pt}

\tasknumber{2}%
\task{%
    Два одинаковых маленьких проводящих заряженных шарика находятся
    на расстоянии~$l$ друг от друга.
    Заряд первого равен~$+5Q$, второго~---$-2Q$.
    Шарики приводят в соприкосновение, а после опять разводят на то же самое расстояние~$l$.
    \begin{itemize}
        \item Каким стал заряд каждого из шариков?
        \item Определите характер (притяжение или отталкивание) и силу взаимодействия шариков до и после соприкосновения.
        \item Как изменилась сила взаимодействия шариков после соприкосновения?
    \end{itemize}
}
\answer{%
    \begin{align*}
    F &= k\frac{q_1 q_2}{l^2} = k\frac{(+5Q) \cdot (-2Q)}{l^2}, \text{отталкивание}; \\
        q'_1 = q'_2 = \frac{q_1 + q_2}2 = \frac{+5Q -2Q}2 \implies
        F'  &= k\frac{q'_1 q'_2}{l^2}
            = k\frac{\sqr{\frac{(+5Q) + (-2Q)}2}}{l^2},
        \text{отталкивание}.
    \end{align*}
}
\solutionspace{120pt}

\tasknumber{3}%
\task{%
    На координатной плоскости в точках $(-d; 0)$ и $(d; 0)$
    находятся заряды, соответственно, $-q$ и $-q$.
    Сделайте рисунок, определите величину напряжённости электрического поля
    в точках $(0; d)$ и $(2d; 0)$ и укажите её направление.
}
\solutionspace{120pt}

\tasknumber{4}%
\task{%
    Заряд $q_1$ создает в точке $A$ электрическое поле
    по величине равное~$E_1=300\funits{В}{м}$,
    а $q_2$~--- $E_2=400\funits{В}{м}$.
    Угол между векторами $\vect{E_1}$ и $\vect{E_2}$ равен $\varphi$.
    Определите величину суммарного электрического поля в точке $A$,
    создаваемого обоими зарядами $q_1$ и $q_2$.
    Сделайте рисунки и вычислите значение для двух значений угла $\varphi$:
    $\varphi_1=90^\circ$ и $\varphi_2=180^\circ$.
}

\variantsplitter

\addpersonalvariant{Георгий Новиков}

\tasknumber{1}%
\task{%
    С какой силой взаимодействуют 2 точечных заряда $q_1 = 3\,\text{нКл}$ и $q_2 = 4\,\text{нКл}$,
    находящиеся на расстоянии $d = 5\,\text{см}$?
}
\answer{%
    $
        F
            = k\frac{q_1q_2}{d^2}
            = 9 \cdot 10^{9}\,\frac{\text{Н}\cdot\text{м}^{2}}{\text{Кл}^{2}} \cdot \frac{3\,\text{нКл} *4\,\text{нКл}}{\sqr{ 5\,\text{см} }}
            = \frac{108}{25} \cdot 10^{31}\units{Н}
              \approx {4{,}32} \cdot 10^{31}\units{Н}
    $
}
\solutionspace{120pt}

\tasknumber{2}%
\task{%
    Два одинаковых маленьких проводящих заряженных шарика находятся
    на расстоянии~$r$ друг от друга.
    Заряд первого равен~$+3Q$, второго~---$+5Q$.
    Шарики приводят в соприкосновение, а после опять разводят на то же самое расстояние~$r$.
    \begin{itemize}
        \item Каким стал заряд каждого из шариков?
        \item Определите характер (притяжение или отталкивание) и силу взаимодействия шариков до и после соприкосновения.
        \item Как изменилась сила взаимодействия шариков после соприкосновения?
    \end{itemize}
}
\answer{%
    \begin{align*}
    F &= k\frac{q_1 q_2}{r^2} = k\frac{(+3Q) \cdot (+5Q)}{r^2}, \text{отталкивание}; \\
        q'_1 = q'_2 = \frac{q_1 + q_2}2 = \frac{+3Q + +5Q}2 \implies
        F'  &= k\frac{q'_1 q'_2}{r^2}
            = k\frac{\sqr{\frac{(+3Q) + (+5Q)}2}}{r^2},
        \text{отталкивание}.
    \end{align*}
}
\solutionspace{120pt}

\tasknumber{3}%
\task{%
    На координатной плоскости в точках $(-r; 0)$ и $(r; 0)$
    находятся заряды, соответственно, $-q$ и $-q$.
    Сделайте рисунок, определите величину напряжённости электрического поля
    в точках $(0; r)$ и $(-2r; 0)$ и укажите её направление.
}
\solutionspace{120pt}

\tasknumber{4}%
\task{%
    Заряд $q_1$ создает в точке $A$ электрическое поле
    по величине равное~$E_1=250\funits{В}{м}$,
    а $q_2$~--- $E_2=250\funits{В}{м}$.
    Угол между векторами $\vect{E_1}$ и $\vect{E_2}$ равен $\alpha$.
    Определите величину суммарного электрического поля в точке $A$,
    создаваемого обоими зарядами $q_1$ и $q_2$.
    Сделайте рисунки и вычислите значение для двух значений угла $\alpha$:
    $\alpha_1=0^\circ$ и $\alpha_2=60^\circ$.
}

\variantsplitter

\addpersonalvariant{Егор Осипов}

\tasknumber{1}%
\task{%
    С какой силой взаимодействуют 2 точечных заряда $q_1 = 3\,\text{нКл}$ и $q_2 = 2\,\text{нКл}$,
    находящиеся на расстоянии $d = 6\,\text{см}$?
}
\answer{%
    $
        F
            = k\frac{q_1q_2}{d^2}
            = 9 \cdot 10^{9}\,\frac{\text{Н}\cdot\text{м}^{2}}{\text{Кл}^{2}} \cdot \frac{3\,\text{нКл} *2\,\text{нКл}}{\sqr{ 6\,\text{см} }}
            = \frac32 \cdot 10^{31}\units{Н}
              \approx {1{,}50} \cdot 10^{31}\units{Н}
    $
}
\solutionspace{120pt}

\tasknumber{2}%
\task{%
    Два одинаковых маленьких проводящих заряженных шарика находятся
    на расстоянии~$l$ друг от друга.
    Заряд первого равен~$-3q$, второго~---$+5q$.
    Шарики приводят в соприкосновение, а после опять разводят на то же самое расстояние~$l$.
    \begin{itemize}
        \item Каким стал заряд каждого из шариков?
        \item Определите характер (притяжение или отталкивание) и силу взаимодействия шариков до и после соприкосновения.
        \item Как изменилась сила взаимодействия шариков после соприкосновения?
    \end{itemize}
}
\answer{%
    \begin{align*}
    F &= k\frac{q_1 q_2}{l^2} = k\frac{(-3q) \cdot (+5q)}{l^2}, \text{отталкивание}; \\
        q'_1 = q'_2 = \frac{q_1 + q_2}2 = \frac{-3q + +5q}2 \implies
        F'  &= k\frac{q'_1 q'_2}{l^2}
            = k\frac{\sqr{\frac{(-3q) + (+5q)}2}}{l^2},
        \text{отталкивание}.
    \end{align*}
}
\solutionspace{120pt}

\tasknumber{3}%
\task{%
    На координатной плоскости в точках $(-d; 0)$ и $(d; 0)$
    находятся заряды, соответственно, $-q$ и $-q$.
    Сделайте рисунок, определите величину напряжённости электрического поля
    в точках $(0; d)$ и $(-2d; 0)$ и укажите её направление.
}
\solutionspace{120pt}

\tasknumber{4}%
\task{%
    Заряд $q_1$ создает в точке $A$ электрическое поле
    по величине равное~$E_1=50\funits{В}{м}$,
    а $q_2$~--- $E_2=120\funits{В}{м}$.
    Угол между векторами $\vect{E_1}$ и $\vect{E_2}$ равен $\alpha$.
    Определите величину суммарного электрического поля в точке $A$,
    создаваемого обоими зарядами $q_1$ и $q_2$.
    Сделайте рисунки и вычислите значение для двух значений угла $\alpha$:
    $\alpha_1=0^\circ$ и $\alpha_2=90^\circ$.
}

\variantsplitter

\addpersonalvariant{Руслан Перепелица}

\tasknumber{1}%
\task{%
    С какой силой взаимодействуют 2 точечных заряда $q_1 = 2\,\text{нКл}$ и $q_2 = 4\,\text{нКл}$,
    находящиеся на расстоянии $d = 3\,\text{см}$?
}
\answer{%
    $
        F
            = k\frac{q_1q_2}{d^2}
            = 9 \cdot 10^{9}\,\frac{\text{Н}\cdot\text{м}^{2}}{\text{Кл}^{2}} \cdot \frac{2\,\text{нКл} *4\,\text{нКл}}{\sqr{ 3\,\text{см} }}
            = 8 \cdot 10^{31}\units{Н}
              \approx {8{,}00} \cdot 10^{31}\units{Н}
    $
}
\solutionspace{120pt}

\tasknumber{2}%
\task{%
    Два одинаковых маленьких проводящих заряженных шарика находятся
    на расстоянии~$r$ друг от друга.
    Заряд первого равен~$-2q$, второго~---$-5q$.
    Шарики приводят в соприкосновение, а после опять разводят на то же самое расстояние~$r$.
    \begin{itemize}
        \item Каким стал заряд каждого из шариков?
        \item Определите характер (притяжение или отталкивание) и силу взаимодействия шариков до и после соприкосновения.
        \item Как изменилась сила взаимодействия шариков после соприкосновения?
    \end{itemize}
}
\answer{%
    \begin{align*}
    F &= k\frac{q_1 q_2}{r^2} = k\frac{(-2q) \cdot (-5q)}{r^2}, \text{отталкивание}; \\
        q'_1 = q'_2 = \frac{q_1 + q_2}2 = \frac{-2q -5q}2 \implies
        F'  &= k\frac{q'_1 q'_2}{r^2}
            = k\frac{\sqr{\frac{(-2q) + (-5q)}2}}{r^2},
        \text{отталкивание}.
    \end{align*}
}
\solutionspace{120pt}

\tasknumber{3}%
\task{%
    На координатной плоскости в точках $(-a; 0)$ и $(a; 0)$
    находятся заряды, соответственно, $-q$ и $-q$.
    Сделайте рисунок, определите величину напряжённости электрического поля
    в точках $(0; a)$ и $(2a; 0)$ и укажите её направление.
}
\solutionspace{120pt}

\tasknumber{4}%
\task{%
    Заряд $q_1$ создает в точке $A$ электрическое поле
    по величине равное~$E_1=72\funits{В}{м}$,
    а $q_2$~--- $E_2=72\funits{В}{м}$.
    Угол между векторами $\vect{E_1}$ и $\vect{E_2}$ равен $\varphi$.
    Определите величину суммарного электрического поля в точке $A$,
    создаваемого обоими зарядами $q_1$ и $q_2$.
    Сделайте рисунки и вычислите значение для двух значений угла $\varphi$:
    $\varphi_1=0^\circ$ и $\varphi_2=120^\circ$.
}

\variantsplitter

\addpersonalvariant{Михаил Перин}

\tasknumber{1}%
\task{%
    С какой силой взаимодействуют 2 точечных заряда $q_1 = 2\,\text{нКл}$ и $q_2 = 4\,\text{нКл}$,
    находящиеся на расстоянии $r = 2\,\text{см}$?
}
\answer{%
    $
        F
            = k\frac{q_1q_2}{r^2}
            = 9 \cdot 10^{9}\,\frac{\text{Н}\cdot\text{м}^{2}}{\text{Кл}^{2}} \cdot \frac{2\,\text{нКл} *4\,\text{нКл}}{\sqr{ 2\,\text{см} }}
            = 18 \cdot 10^{31}\units{Н}
              \approx {18{,}00} \cdot 10^{31}\units{Н}
    $
}
\solutionspace{120pt}

\tasknumber{2}%
\task{%
    Два одинаковых маленьких проводящих заряженных шарика находятся
    на расстоянии~$l$ друг от друга.
    Заряд первого равен~$-5Q$, второго~---$+3Q$.
    Шарики приводят в соприкосновение, а после опять разводят на то же самое расстояние~$l$.
    \begin{itemize}
        \item Каким стал заряд каждого из шариков?
        \item Определите характер (притяжение или отталкивание) и силу взаимодействия шариков до и после соприкосновения.
        \item Как изменилась сила взаимодействия шариков после соприкосновения?
    \end{itemize}
}
\answer{%
    \begin{align*}
    F &= k\frac{q_1 q_2}{l^2} = k\frac{(-5Q) \cdot (+3Q)}{l^2}, \text{отталкивание}; \\
        q'_1 = q'_2 = \frac{q_1 + q_2}2 = \frac{-5Q + +3Q}2 \implies
        F'  &= k\frac{q'_1 q'_2}{l^2}
            = k\frac{\sqr{\frac{(-5Q) + (+3Q)}2}}{l^2},
        \text{отталкивание}.
    \end{align*}
}
\solutionspace{120pt}

\tasknumber{3}%
\task{%
    На координатной плоскости в точках $(-r; 0)$ и $(r; 0)$
    находятся заряды, соответственно, $+Q$ и $+Q$.
    Сделайте рисунок, определите величину напряжённости электрического поля
    в точках $(0; r)$ и $(2r; 0)$ и укажите её направление.
}
\solutionspace{120pt}

\tasknumber{4}%
\task{%
    Заряд $q_1$ создает в точке $A$ электрическое поле
    по величине равное~$E_1=7\funits{В}{м}$,
    а $q_2$~--- $E_2=24\funits{В}{м}$.
    Угол между векторами $\vect{E_1}$ и $\vect{E_2}$ равен $\alpha$.
    Определите величину суммарного электрического поля в точке $A$,
    создаваемого обоими зарядами $q_1$ и $q_2$.
    Сделайте рисунки и вычислите значение для двух значений угла $\alpha$:
    $\alpha_1=0^\circ$ и $\alpha_2=90^\circ$.
}

\variantsplitter

\addpersonalvariant{Егор Подуровский}

\tasknumber{1}%
\task{%
    С какой силой взаимодействуют 2 точечных заряда $q_1 = 2\,\text{нКл}$ и $q_2 = 4\,\text{нКл}$,
    находящиеся на расстоянии $r = 3\,\text{см}$?
}
\answer{%
    $
        F
            = k\frac{q_1q_2}{r^2}
            = 9 \cdot 10^{9}\,\frac{\text{Н}\cdot\text{м}^{2}}{\text{Кл}^{2}} \cdot \frac{2\,\text{нКл} *4\,\text{нКл}}{\sqr{ 3\,\text{см} }}
            = 8 \cdot 10^{31}\units{Н}
              \approx {8{,}00} \cdot 10^{31}\units{Н}
    $
}
\solutionspace{120pt}

\tasknumber{2}%
\task{%
    Два одинаковых маленьких проводящих заряженных шарика находятся
    на расстоянии~$r$ друг от друга.
    Заряд первого равен~$+4q$, второго~---$-3q$.
    Шарики приводят в соприкосновение, а после опять разводят на то же самое расстояние~$r$.
    \begin{itemize}
        \item Каким стал заряд каждого из шариков?
        \item Определите характер (притяжение или отталкивание) и силу взаимодействия шариков до и после соприкосновения.
        \item Как изменилась сила взаимодействия шариков после соприкосновения?
    \end{itemize}
}
\answer{%
    \begin{align*}
    F &= k\frac{q_1 q_2}{r^2} = k\frac{(+4q) \cdot (-3q)}{r^2}, \text{отталкивание}; \\
        q'_1 = q'_2 = \frac{q_1 + q_2}2 = \frac{+4q -3q}2 \implies
        F'  &= k\frac{q'_1 q'_2}{r^2}
            = k\frac{\sqr{\frac{(+4q) + (-3q)}2}}{r^2},
        \text{отталкивание}.
    \end{align*}
}
\solutionspace{120pt}

\tasknumber{3}%
\task{%
    На координатной плоскости в точках $(-a; 0)$ и $(a; 0)$
    находятся заряды, соответственно, $-q$ и $-q$.
    Сделайте рисунок, определите величину напряжённости электрического поля
    в точках $(0; a)$ и $(2a; 0)$ и укажите её направление.
}
\solutionspace{120pt}

\tasknumber{4}%
\task{%
    Заряд $q_1$ создает в точке $A$ электрическое поле
    по величине равное~$E_1=7\funits{В}{м}$,
    а $q_2$~--- $E_2=24\funits{В}{м}$.
    Угол между векторами $\vect{E_1}$ и $\vect{E_2}$ равен $\varphi$.
    Определите величину суммарного электрического поля в точке $A$,
    создаваемого обоими зарядами $q_1$ и $q_2$.
    Сделайте рисунки и вычислите значение для двух значений угла $\varphi$:
    $\varphi_1=0^\circ$ и $\varphi_2=90^\circ$.
}

\variantsplitter

\addpersonalvariant{Роман Прибылов}

\tasknumber{1}%
\task{%
    С какой силой взаимодействуют 2 точечных заряда $q_1 = 4\,\text{нКл}$ и $q_2 = 3\,\text{нКл}$,
    находящиеся на расстоянии $d = 3\,\text{см}$?
}
\answer{%
    $
        F
            = k\frac{q_1q_2}{d^2}
            = 9 \cdot 10^{9}\,\frac{\text{Н}\cdot\text{м}^{2}}{\text{Кл}^{2}} \cdot \frac{4\,\text{нКл} *3\,\text{нКл}}{\sqr{ 3\,\text{см} }}
            = 12 \cdot 10^{31}\units{Н}
              \approx {12{,}00} \cdot 10^{31}\units{Н}
    $
}
\solutionspace{120pt}

\tasknumber{2}%
\task{%
    Два одинаковых маленьких проводящих заряженных шарика находятся
    на расстоянии~$d$ друг от друга.
    Заряд первого равен~$-5q$, второго~---$+2q$.
    Шарики приводят в соприкосновение, а после опять разводят на то же самое расстояние~$d$.
    \begin{itemize}
        \item Каким стал заряд каждого из шариков?
        \item Определите характер (притяжение или отталкивание) и силу взаимодействия шариков до и после соприкосновения.
        \item Как изменилась сила взаимодействия шариков после соприкосновения?
    \end{itemize}
}
\answer{%
    \begin{align*}
    F &= k\frac{q_1 q_2}{d^2} = k\frac{(-5q) \cdot (+2q)}{d^2}, \text{отталкивание}; \\
        q'_1 = q'_2 = \frac{q_1 + q_2}2 = \frac{-5q + +2q}2 \implies
        F'  &= k\frac{q'_1 q'_2}{d^2}
            = k\frac{\sqr{\frac{(-5q) + (+2q)}2}}{d^2},
        \text{отталкивание}.
    \end{align*}
}
\solutionspace{120pt}

\tasknumber{3}%
\task{%
    На координатной плоскости в точках $(-d; 0)$ и $(d; 0)$
    находятся заряды, соответственно, $+q$ и $-q$.
    Сделайте рисунок, определите величину напряжённости электрического поля
    в точках $(0; d)$ и $(-2d; 0)$ и укажите её направление.
}
\solutionspace{120pt}

\tasknumber{4}%
\task{%
    Заряд $q_1$ создает в точке $A$ электрическое поле
    по величине равное~$E_1=72\funits{В}{м}$,
    а $q_2$~--- $E_2=72\funits{В}{м}$.
    Угол между векторами $\vect{E_1}$ и $\vect{E_2}$ равен $\alpha$.
    Определите величину суммарного электрического поля в точке $A$,
    создаваемого обоими зарядами $q_1$ и $q_2$.
    Сделайте рисунки и вычислите значение для двух значений угла $\alpha$:
    $\alpha_1=0^\circ$ и $\alpha_2=120^\circ$.
}

\variantsplitter

\addpersonalvariant{Александр Селехметьев}

\tasknumber{1}%
\task{%
    С какой силой взаимодействуют 2 точечных заряда $q_1 = 3\,\text{нКл}$ и $q_2 = 2\,\text{нКл}$,
    находящиеся на расстоянии $l = 2\,\text{см}$?
}
\answer{%
    $
        F
            = k\frac{q_1q_2}{l^2}
            = 9 \cdot 10^{9}\,\frac{\text{Н}\cdot\text{м}^{2}}{\text{Кл}^{2}} \cdot \frac{3\,\text{нКл} *2\,\text{нКл}}{\sqr{ 2\,\text{см} }}
            = \frac{27}2 \cdot 10^{31}\units{Н}
              \approx {13{,}50} \cdot 10^{31}\units{Н}
    $
}
\solutionspace{120pt}

\tasknumber{2}%
\task{%
    Два одинаковых маленьких проводящих заряженных шарика находятся
    на расстоянии~$d$ друг от друга.
    Заряд первого равен~$+2Q$, второго~---$-5Q$.
    Шарики приводят в соприкосновение, а после опять разводят на то же самое расстояние~$d$.
    \begin{itemize}
        \item Каким стал заряд каждого из шариков?
        \item Определите характер (притяжение или отталкивание) и силу взаимодействия шариков до и после соприкосновения.
        \item Как изменилась сила взаимодействия шариков после соприкосновения?
    \end{itemize}
}
\answer{%
    \begin{align*}
    F &= k\frac{q_1 q_2}{d^2} = k\frac{(+2Q) \cdot (-5Q)}{d^2}, \text{отталкивание}; \\
        q'_1 = q'_2 = \frac{q_1 + q_2}2 = \frac{+2Q -5Q}2 \implies
        F'  &= k\frac{q'_1 q'_2}{d^2}
            = k\frac{\sqr{\frac{(+2Q) + (-5Q)}2}}{d^2},
        \text{отталкивание}.
    \end{align*}
}
\solutionspace{120pt}

\tasknumber{3}%
\task{%
    На координатной плоскости в точках $(-a; 0)$ и $(a; 0)$
    находятся заряды, соответственно, $-q$ и $-q$.
    Сделайте рисунок, определите величину напряжённости электрического поля
    в точках $(0; -a)$ и $(2a; 0)$ и укажите её направление.
}
\solutionspace{120pt}

\tasknumber{4}%
\task{%
    Заряд $q_1$ создает в точке $A$ электрическое поле
    по величине равное~$E_1=7\funits{В}{м}$,
    а $q_2$~--- $E_2=24\funits{В}{м}$.
    Угол между векторами $\vect{E_1}$ и $\vect{E_2}$ равен $\varphi$.
    Определите величину суммарного электрического поля в точке $A$,
    создаваемого обоими зарядами $q_1$ и $q_2$.
    Сделайте рисунки и вычислите значение для двух значений угла $\varphi$:
    $\varphi_1=0^\circ$ и $\varphi_2=90^\circ$.
}

\variantsplitter

\addpersonalvariant{Алексей Тихонов}

\tasknumber{1}%
\task{%
    С какой силой взаимодействуют 2 точечных заряда $q_1 = 3\,\text{нКл}$ и $q_2 = 2\,\text{нКл}$,
    находящиеся на расстоянии $l = 5\,\text{см}$?
}
\answer{%
    $
        F
            = k\frac{q_1q_2}{l^2}
            = 9 \cdot 10^{9}\,\frac{\text{Н}\cdot\text{м}^{2}}{\text{Кл}^{2}} \cdot \frac{3\,\text{нКл} *2\,\text{нКл}}{\sqr{ 5\,\text{см} }}
            = \frac{54}{25} \cdot 10^{31}\units{Н}
              \approx {2{,}16} \cdot 10^{31}\units{Н}
    $
}
\solutionspace{120pt}

\tasknumber{2}%
\task{%
    Два одинаковых маленьких проводящих заряженных шарика находятся
    на расстоянии~$r$ друг от друга.
    Заряд первого равен~$+4Q$, второго~---$+3Q$.
    Шарики приводят в соприкосновение, а после опять разводят на то же самое расстояние~$r$.
    \begin{itemize}
        \item Каким стал заряд каждого из шариков?
        \item Определите характер (притяжение или отталкивание) и силу взаимодействия шариков до и после соприкосновения.
        \item Как изменилась сила взаимодействия шариков после соприкосновения?
    \end{itemize}
}
\answer{%
    \begin{align*}
    F &= k\frac{q_1 q_2}{r^2} = k\frac{(+4Q) \cdot (+3Q)}{r^2}, \text{отталкивание}; \\
        q'_1 = q'_2 = \frac{q_1 + q_2}2 = \frac{+4Q + +3Q}2 \implies
        F'  &= k\frac{q'_1 q'_2}{r^2}
            = k\frac{\sqr{\frac{(+4Q) + (+3Q)}2}}{r^2},
        \text{отталкивание}.
    \end{align*}
}
\solutionspace{120pt}

\tasknumber{3}%
\task{%
    На координатной плоскости в точках $(-r; 0)$ и $(r; 0)$
    находятся заряды, соответственно, $+q$ и $-q$.
    Сделайте рисунок, определите величину напряжённости электрического поля
    в точках $(0; -r)$ и $(2r; 0)$ и укажите её направление.
}
\solutionspace{120pt}

\tasknumber{4}%
\task{%
    Заряд $q_1$ создает в точке $A$ электрическое поле
    по величине равное~$E_1=50\funits{В}{м}$,
    а $q_2$~--- $E_2=120\funits{В}{м}$.
    Угол между векторами $\vect{E_1}$ и $\vect{E_2}$ равен $\alpha$.
    Определите величину суммарного электрического поля в точке $A$,
    создаваемого обоими зарядами $q_1$ и $q_2$.
    Сделайте рисунки и вычислите значение для двух значений угла $\alpha$:
    $\alpha_1=0^\circ$ и $\alpha_2=90^\circ$.
}

\variantsplitter

\addpersonalvariant{Алина Филиппова}

\tasknumber{1}%
\task{%
    С какой силой взаимодействуют 2 точечных заряда $q_1 = 4\,\text{нКл}$ и $q_2 = 2\,\text{нКл}$,
    находящиеся на расстоянии $d = 3\,\text{см}$?
}
\answer{%
    $
        F
            = k\frac{q_1q_2}{d^2}
            = 9 \cdot 10^{9}\,\frac{\text{Н}\cdot\text{м}^{2}}{\text{Кл}^{2}} \cdot \frac{4\,\text{нКл} *2\,\text{нКл}}{\sqr{ 3\,\text{см} }}
            = 8 \cdot 10^{31}\units{Н}
              \approx {8{,}00} \cdot 10^{31}\units{Н}
    $
}
\solutionspace{120pt}

\tasknumber{2}%
\task{%
    Два одинаковых маленьких проводящих заряженных шарика находятся
    на расстоянии~$r$ друг от друга.
    Заряд первого равен~$+2Q$, второго~---$-4Q$.
    Шарики приводят в соприкосновение, а после опять разводят на то же самое расстояние~$r$.
    \begin{itemize}
        \item Каким стал заряд каждого из шариков?
        \item Определите характер (притяжение или отталкивание) и силу взаимодействия шариков до и после соприкосновения.
        \item Как изменилась сила взаимодействия шариков после соприкосновения?
    \end{itemize}
}
\answer{%
    \begin{align*}
    F &= k\frac{q_1 q_2}{r^2} = k\frac{(+2Q) \cdot (-4Q)}{r^2}, \text{отталкивание}; \\
        q'_1 = q'_2 = \frac{q_1 + q_2}2 = \frac{+2Q -4Q}2 \implies
        F'  &= k\frac{q'_1 q'_2}{r^2}
            = k\frac{\sqr{\frac{(+2Q) + (-4Q)}2}}{r^2},
        \text{отталкивание}.
    \end{align*}
}
\solutionspace{120pt}

\tasknumber{3}%
\task{%
    На координатной плоскости в точках $(-a; 0)$ и $(a; 0)$
    находятся заряды, соответственно, $-Q$ и $+Q$.
    Сделайте рисунок, определите величину напряжённости электрического поля
    в точках $(0; -a)$ и $(2a; 0)$ и укажите её направление.
}
\solutionspace{120pt}

\tasknumber{4}%
\task{%
    Заряд $q_1$ создает в точке $A$ электрическое поле
    по величине равное~$E_1=200\funits{В}{м}$,
    а $q_2$~--- $E_2=200\funits{В}{м}$.
    Угол между векторами $\vect{E_1}$ и $\vect{E_2}$ равен $\alpha$.
    Определите величину суммарного электрического поля в точке $A$,
    создаваемого обоими зарядами $q_1$ и $q_2$.
    Сделайте рисунки и вычислите значение для двух значений угла $\alpha$:
    $\alpha_1=0^\circ$ и $\alpha_2=60^\circ$.
}

\variantsplitter

\addpersonalvariant{Алина Яшина}

\tasknumber{1}%
\task{%
    С какой силой взаимодействуют 2 точечных заряда $q_1 = 2\,\text{нКл}$ и $q_2 = 3\,\text{нКл}$,
    находящиеся на расстоянии $l = 3\,\text{см}$?
}
\answer{%
    $
        F
            = k\frac{q_1q_2}{l^2}
            = 9 \cdot 10^{9}\,\frac{\text{Н}\cdot\text{м}^{2}}{\text{Кл}^{2}} \cdot \frac{2\,\text{нКл} *3\,\text{нКл}}{\sqr{ 3\,\text{см} }}
            = 6 \cdot 10^{31}\units{Н}
              \approx {6{,}00} \cdot 10^{31}\units{Н}
    $
}
\solutionspace{120pt}

\tasknumber{2}%
\task{%
    Два одинаковых маленьких проводящих заряженных шарика находятся
    на расстоянии~$r$ друг от друга.
    Заряд первого равен~$-3Q$, второго~---$-2Q$.
    Шарики приводят в соприкосновение, а после опять разводят на то же самое расстояние~$r$.
    \begin{itemize}
        \item Каким стал заряд каждого из шариков?
        \item Определите характер (притяжение или отталкивание) и силу взаимодействия шариков до и после соприкосновения.
        \item Как изменилась сила взаимодействия шариков после соприкосновения?
    \end{itemize}
}
\answer{%
    \begin{align*}
    F &= k\frac{q_1 q_2}{r^2} = k\frac{(-3Q) \cdot (-2Q)}{r^2}, \text{отталкивание}; \\
        q'_1 = q'_2 = \frac{q_1 + q_2}2 = \frac{-3Q -2Q}2 \implies
        F'  &= k\frac{q'_1 q'_2}{r^2}
            = k\frac{\sqr{\frac{(-3Q) + (-2Q)}2}}{r^2},
        \text{отталкивание}.
    \end{align*}
}
\solutionspace{120pt}

\tasknumber{3}%
\task{%
    На координатной плоскости в точках $(-d; 0)$ и $(d; 0)$
    находятся заряды, соответственно, $+Q$ и $+Q$.
    Сделайте рисунок, определите величину напряжённости электрического поля
    в точках $(0; -d)$ и $(2d; 0)$ и укажите её направление.
}
\solutionspace{120pt}

\tasknumber{4}%
\task{%
    Заряд $q_1$ создает в точке $A$ электрическое поле
    по величине равное~$E_1=120\funits{В}{м}$,
    а $q_2$~--- $E_2=50\funits{В}{м}$.
    Угол между векторами $\vect{E_1}$ и $\vect{E_2}$ равен $\alpha$.
    Определите величину суммарного электрического поля в точке $A$,
    создаваемого обоими зарядами $q_1$ и $q_2$.
    Сделайте рисунки и вычислите значение для двух значений угла $\alpha$:
    $\alpha_1=90^\circ$ и $\alpha_2=180^\circ$.
}
% autogenerated
