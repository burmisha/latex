\setdate{28~апреля~2020}
\setclass{9«Л»}

\addpersonalvariant{Михаил Бурмистров}

\tasknumber{1}%
\task{%
    Длина волны света в~вакууме $\lambda = 400\,\text{нм}$.
    Какова частота этой световой волны?
    Какова длина этой волны в среде с показателем преломления $n = 1{,}4$?
    Может ли человек увидеть такую волну света, и если да, то какой именно цвет соответствует этим волнам в вакууме и в этой среде?
}
\answer{%
    \begin{align*}
    \nu &= \frac 1T = \frac 1{\lambda/c} = \frac c\lambda = \frac{ 3 \cdot 10^{8}\,\frac{\text{м}}{\text{с}} }{ 400\,\text{нм} } \approx 7{,}50 \cdot 10^{14}\,\text{Гц}, \\
    \nu' = \nu &\cbr{\text{или } T' = T} \implies \lambda' = v'T' = \frac vn T = \frac{vt}n = \frac \lambda n = \frac{ 400\,\text{нм} }{ 1{,}4 } \approx 286 \cdot 10^{-9}\,\text{м}.
    \\
    &\text{380 нм---фиол---440---син---485---гол---500---зел---565---жёл---590---оранж---625---крас---780 нм}
    \end{align*}
}
\solutionspace{180pt}

\tasknumber{2}%
\task{%
    При переходе электрона в атоме с одной стационарной орбиты на другую
    излучается фотон с энергией $2{,}02 \cdot 10^{-19}\,\text{Дж}$.
    Какова длина волны этой линии спектра?
    Постоянная Планка $h = 6{,}626 \cdot 10^{-34}\,\text{Дж}\cdot\text{с}$, скорость света $c = 3 \cdot 10^{8}\,\frac{\text{м}}{\text{с}}$.
}
\answer{%
    $
        E = h\nu = h \frac c\lambda
        \implies \lambda = \frac{hc}{E}
            = \frac{6{,}626 \cdot 10^{-34}\,\text{Дж}\cdot\text{с} \cdot { 3 \cdot 10^{8}\,\frac{\text{м}}{\text{с}} }}{ 2{,}02 \cdot 10^{-19}\,\text{Дж} }
            = 9{,}84 \cdot 10^{-7}\,\text{м}.
    $
}
\solutionspace{150pt}

\tasknumber{3}%
\task{%
    Излучение какой длины волны поглотил атом водорода, если полная энергия в атоме увеличилась на $2 \cdot 10^{-19}\,\text{Дж}$?
    Постоянная Планка $h = 6{,}626 \cdot 10^{-34}\,\text{Дж}\cdot\text{с}$, скорость света $c = 3 \cdot 10^{8}\,\frac{\text{м}}{\text{с}}$.
}
\answer{%
    $
        E = h\nu = h \frac c\lambda
        \implies \lambda = \frac{hc}{E}
            = \frac{6{,}626 \cdot 10^{-34}\,\text{Дж}\cdot\text{с} \cdot { 3 \cdot 10^{8}\,\frac{\text{м}}{\text{с}} }}{ 2 \cdot 10^{-19}\,\text{Дж} }
            = 9{,}94 \cdot 10^{-7}\,\text{м}.
    $
}
\solutionspace{150pt}

\tasknumber{4}%
\task{%
    Сделайте схематичный рисунок энергетических уровней атома водорода
    и отметьте на нём первый (основной) уровень и последующие.
    Сколько различных длин волн может испустить атом водорода,
    находящийся в 3-м возбуждённом состоянии?
    Отметьте все соответствующие переходы на рисунке и укажите,
    при каком переходе (среди отмеченных) частота излучённого фотона максимальна.
}
\answer{%
    $N = 3{,}0, \text{самая длинная линия}$
}

\variantsplitter

\addpersonalvariant{Милана Абраамян}

\tasknumber{1}%
\task{%
    Длина волны света в~вакууме $\lambda = 400\,\text{нм}$.
    Какова частота этой световой волны?
    Какова длина этой волны в среде с показателем преломления $n = 1{,}3$?
    Может ли человек увидеть такую волну света, и если да, то какой именно цвет соответствует этим волнам в вакууме и в этой среде?
}
\answer{%
    \begin{align*}
    \nu &= \frac 1T = \frac 1{\lambda/c} = \frac c\lambda = \frac{ 3 \cdot 10^{8}\,\frac{\text{м}}{\text{с}} }{ 400\,\text{нм} } \approx 7{,}50 \cdot 10^{14}\,\text{Гц}, \\
    \nu' = \nu &\cbr{\text{или } T' = T} \implies \lambda' = v'T' = \frac vn T = \frac{vt}n = \frac \lambda n = \frac{ 400\,\text{нм} }{ 1{,}3 } \approx 308 \cdot 10^{-9}\,\text{м}.
    \\
    &\text{380 нм---фиол---440---син---485---гол---500---зел---565---жёл---590---оранж---625---крас---780 нм}
    \end{align*}
}
\solutionspace{180pt}

\tasknumber{2}%
\task{%
    При переходе электрона в атоме с одной стационарной орбиты на другую
    излучается фотон с энергией $5{,}05 \cdot 10^{-19}\,\text{Дж}$.
    Какова длина волны этой линии спектра?
    Постоянная Планка $h = 6{,}626 \cdot 10^{-34}\,\text{Дж}\cdot\text{с}$, скорость света $c = 3 \cdot 10^{8}\,\frac{\text{м}}{\text{с}}$.
}
\answer{%
    $
        E = h\nu = h \frac c\lambda
        \implies \lambda = \frac{hc}{E}
            = \frac{6{,}626 \cdot 10^{-34}\,\text{Дж}\cdot\text{с} \cdot { 3 \cdot 10^{8}\,\frac{\text{м}}{\text{с}} }}{ 5{,}05 \cdot 10^{-19}\,\text{Дж} }
            = 3{,}94 \cdot 10^{-7}\,\text{м}.
    $
}
\solutionspace{150pt}

\tasknumber{3}%
\task{%
    Излучение какой длины волны поглотил атом водорода, если полная энергия в атоме увеличилась на $3 \cdot 10^{-19}\,\text{Дж}$?
    Постоянная Планка $h = 6{,}626 \cdot 10^{-34}\,\text{Дж}\cdot\text{с}$, скорость света $c = 3 \cdot 10^{8}\,\frac{\text{м}}{\text{с}}$.
}
\answer{%
    $
        E = h\nu = h \frac c\lambda
        \implies \lambda = \frac{hc}{E}
            = \frac{6{,}626 \cdot 10^{-34}\,\text{Дж}\cdot\text{с} \cdot { 3 \cdot 10^{8}\,\frac{\text{м}}{\text{с}} }}{ 3 \cdot 10^{-19}\,\text{Дж} }
            = 6{,}63 \cdot 10^{-7}\,\text{м}.
    $
}
\solutionspace{150pt}

\tasknumber{4}%
\task{%
    Сделайте схематичный рисунок энергетических уровней атома водорода
    и отметьте на нём первый (основной) уровень и последующие.
    Сколько различных длин волн может испустить атом водорода,
    находящийся в 4-м возбуждённом состоянии?
    Отметьте все соответствующие переходы на рисунке и укажите,
    при каком переходе (среди отмеченных) частота излучённого фотона максимальна.
}
\answer{%
    $N = 6{,}0, \text{самая длинная линия}$
}

\variantsplitter

\addpersonalvariant{Тимур Азимов}

\tasknumber{1}%
\task{%
    Длина волны света в~вакууме $\lambda = 600\,\text{нм}$.
    Какова частота этой световой волны?
    Какова длина этой волны в среде с показателем преломления $n = 1{,}7$?
    Может ли человек увидеть такую волну света, и если да, то какой именно цвет соответствует этим волнам в вакууме и в этой среде?
}
\answer{%
    \begin{align*}
    \nu &= \frac 1T = \frac 1{\lambda/c} = \frac c\lambda = \frac{ 3 \cdot 10^{8}\,\frac{\text{м}}{\text{с}} }{ 600\,\text{нм} } \approx 5{,}00 \cdot 10^{14}\,\text{Гц}, \\
    \nu' = \nu &\cbr{\text{или } T' = T} \implies \lambda' = v'T' = \frac vn T = \frac{vt}n = \frac \lambda n = \frac{ 600\,\text{нм} }{ 1{,}7 } \approx 353 \cdot 10^{-9}\,\text{м}.
    \\
    &\text{380 нм---фиол---440---син---485---гол---500---зел---565---жёл---590---оранж---625---крас---780 нм}
    \end{align*}
}
\solutionspace{180pt}

\tasknumber{2}%
\task{%
    При переходе электрона в атоме с одной стационарной орбиты на другую
    излучается фотон с энергией $1{,}01 \cdot 10^{-19}\,\text{Дж}$.
    Какова длина волны этой линии спектра?
    Постоянная Планка $h = 6{,}626 \cdot 10^{-34}\,\text{Дж}\cdot\text{с}$, скорость света $c = 3 \cdot 10^{8}\,\frac{\text{м}}{\text{с}}$.
}
\answer{%
    $
        E = h\nu = h \frac c\lambda
        \implies \lambda = \frac{hc}{E}
            = \frac{6{,}626 \cdot 10^{-34}\,\text{Дж}\cdot\text{с} \cdot { 3 \cdot 10^{8}\,\frac{\text{м}}{\text{с}} }}{ 1{,}01 \cdot 10^{-19}\,\text{Дж} }
            = 19{,}68 \cdot 10^{-7}\,\text{м}.
    $
}
\solutionspace{150pt}

\tasknumber{3}%
\task{%
    Излучение какой длины волны поглотил атом водорода, если полная энергия в атоме увеличилась на $2 \cdot 10^{-19}\,\text{Дж}$?
    Постоянная Планка $h = 6{,}626 \cdot 10^{-34}\,\text{Дж}\cdot\text{с}$, скорость света $c = 3 \cdot 10^{8}\,\frac{\text{м}}{\text{с}}$.
}
\answer{%
    $
        E = h\nu = h \frac c\lambda
        \implies \lambda = \frac{hc}{E}
            = \frac{6{,}626 \cdot 10^{-34}\,\text{Дж}\cdot\text{с} \cdot { 3 \cdot 10^{8}\,\frac{\text{м}}{\text{с}} }}{ 2 \cdot 10^{-19}\,\text{Дж} }
            = 9{,}94 \cdot 10^{-7}\,\text{м}.
    $
}
\solutionspace{150pt}

\tasknumber{4}%
\task{%
    Сделайте схематичный рисунок энергетических уровней атома водорода
    и отметьте на нём первый (основной) уровень и последующие.
    Сколько различных длин волн может испустить атом водорода,
    находящийся в 4-м возбуждённом состоянии?
    Отметьте все соответствующие переходы на рисунке и укажите,
    при каком переходе (среди отмеченных) частота излучённого фотона минимальна.
}
\answer{%
    $N = 6{,}0, \text{самая короткая линия}$
}

\variantsplitter

\addpersonalvariant{Полина Алташина}

\tasknumber{1}%
\task{%
    Длина волны света в~вакууме $\lambda = 500\,\text{нм}$.
    Какова частота этой световой волны?
    Какова длина этой волны в среде с показателем преломления $n = 1{,}4$?
    Может ли человек увидеть такую волну света, и если да, то какой именно цвет соответствует этим волнам в вакууме и в этой среде?
}
\answer{%
    \begin{align*}
    \nu &= \frac 1T = \frac 1{\lambda/c} = \frac c\lambda = \frac{ 3 \cdot 10^{8}\,\frac{\text{м}}{\text{с}} }{ 500\,\text{нм} } \approx 6{,}00 \cdot 10^{14}\,\text{Гц}, \\
    \nu' = \nu &\cbr{\text{или } T' = T} \implies \lambda' = v'T' = \frac vn T = \frac{vt}n = \frac \lambda n = \frac{ 500\,\text{нм} }{ 1{,}4 } \approx 357 \cdot 10^{-9}\,\text{м}.
    \\
    &\text{380 нм---фиол---440---син---485---гол---500---зел---565---жёл---590---оранж---625---крас---780 нм}
    \end{align*}
}
\solutionspace{180pt}

\tasknumber{2}%
\task{%
    При переходе электрона в атоме с одной стационарной орбиты на другую
    излучается фотон с энергией $1{,}01 \cdot 10^{-19}\,\text{Дж}$.
    Какова длина волны этой линии спектра?
    Постоянная Планка $h = 6{,}626 \cdot 10^{-34}\,\text{Дж}\cdot\text{с}$, скорость света $c = 3 \cdot 10^{8}\,\frac{\text{м}}{\text{с}}$.
}
\answer{%
    $
        E = h\nu = h \frac c\lambda
        \implies \lambda = \frac{hc}{E}
            = \frac{6{,}626 \cdot 10^{-34}\,\text{Дж}\cdot\text{с} \cdot { 3 \cdot 10^{8}\,\frac{\text{м}}{\text{с}} }}{ 1{,}01 \cdot 10^{-19}\,\text{Дж} }
            = 19{,}68 \cdot 10^{-7}\,\text{м}.
    $
}
\solutionspace{150pt}

\tasknumber{3}%
\task{%
    Излучение какой длины волны поглотил атом водорода, если полная энергия в атоме увеличилась на $4 \cdot 10^{-19}\,\text{Дж}$?
    Постоянная Планка $h = 6{,}626 \cdot 10^{-34}\,\text{Дж}\cdot\text{с}$, скорость света $c = 3 \cdot 10^{8}\,\frac{\text{м}}{\text{с}}$.
}
\answer{%
    $
        E = h\nu = h \frac c\lambda
        \implies \lambda = \frac{hc}{E}
            = \frac{6{,}626 \cdot 10^{-34}\,\text{Дж}\cdot\text{с} \cdot { 3 \cdot 10^{8}\,\frac{\text{м}}{\text{с}} }}{ 4 \cdot 10^{-19}\,\text{Дж} }
            = 4{,}97 \cdot 10^{-7}\,\text{м}.
    $
}
\solutionspace{150pt}

\tasknumber{4}%
\task{%
    Сделайте схематичный рисунок энергетических уровней атома водорода
    и отметьте на нём первый (основной) уровень и последующие.
    Сколько различных длин волн может испустить атом водорода,
    находящийся в 3-м возбуждённом состоянии?
    Отметьте все соответствующие переходы на рисунке и укажите,
    при каком переходе (среди отмеченных) длина волны излучённого фотона минимальна.
}
\answer{%
    $N = 3{,}0, \text{самая длинная линия}$
}

\variantsplitter

\addpersonalvariant{Аркадий Артемов}

\tasknumber{1}%
\task{%
    Длина волны света в~вакууме $\lambda = 600\,\text{нм}$.
    Какова частота этой световой волны?
    Какова длина этой волны в среде с показателем преломления $n = 1{,}5$?
    Может ли человек увидеть такую волну света, и если да, то какой именно цвет соответствует этим волнам в вакууме и в этой среде?
}
\answer{%
    \begin{align*}
    \nu &= \frac 1T = \frac 1{\lambda/c} = \frac c\lambda = \frac{ 3 \cdot 10^{8}\,\frac{\text{м}}{\text{с}} }{ 600\,\text{нм} } \approx 5{,}00 \cdot 10^{14}\,\text{Гц}, \\
    \nu' = \nu &\cbr{\text{или } T' = T} \implies \lambda' = v'T' = \frac vn T = \frac{vt}n = \frac \lambda n = \frac{ 600\,\text{нм} }{ 1{,}5 } \approx 400 \cdot 10^{-9}\,\text{м}.
    \\
    &\text{380 нм---фиол---440---син---485---гол---500---зел---565---жёл---590---оранж---625---крас---780 нм}
    \end{align*}
}
\solutionspace{180pt}

\tasknumber{2}%
\task{%
    При переходе электрона в атоме с одной стационарной орбиты на другую
    излучается фотон с энергией $5{,}05 \cdot 10^{-19}\,\text{Дж}$.
    Какова длина волны этой линии спектра?
    Постоянная Планка $h = 6{,}626 \cdot 10^{-34}\,\text{Дж}\cdot\text{с}$, скорость света $c = 3 \cdot 10^{8}\,\frac{\text{м}}{\text{с}}$.
}
\answer{%
    $
        E = h\nu = h \frac c\lambda
        \implies \lambda = \frac{hc}{E}
            = \frac{6{,}626 \cdot 10^{-34}\,\text{Дж}\cdot\text{с} \cdot { 3 \cdot 10^{8}\,\frac{\text{м}}{\text{с}} }}{ 5{,}05 \cdot 10^{-19}\,\text{Дж} }
            = 3{,}94 \cdot 10^{-7}\,\text{м}.
    $
}
\solutionspace{150pt}

\tasknumber{3}%
\task{%
    Излучение какой длины волны поглотил атом водорода, если полная энергия в атоме увеличилась на $3 \cdot 10^{-19}\,\text{Дж}$?
    Постоянная Планка $h = 6{,}626 \cdot 10^{-34}\,\text{Дж}\cdot\text{с}$, скорость света $c = 3 \cdot 10^{8}\,\frac{\text{м}}{\text{с}}$.
}
\answer{%
    $
        E = h\nu = h \frac c\lambda
        \implies \lambda = \frac{hc}{E}
            = \frac{6{,}626 \cdot 10^{-34}\,\text{Дж}\cdot\text{с} \cdot { 3 \cdot 10^{8}\,\frac{\text{м}}{\text{с}} }}{ 3 \cdot 10^{-19}\,\text{Дж} }
            = 6{,}63 \cdot 10^{-7}\,\text{м}.
    $
}
\solutionspace{150pt}

\tasknumber{4}%
\task{%
    Сделайте схематичный рисунок энергетических уровней атома водорода
    и отметьте на нём первый (основной) уровень и последующие.
    Сколько различных длин волн может испустить атом водорода,
    находящийся в 5-м возбуждённом состоянии?
    Отметьте все соответствующие переходы на рисунке и укажите,
    при каком переходе (среди отмеченных) энергия излучённого фотона минимальна.
}
\answer{%
    $N = 10{,}0, \text{самая короткая линия}$
}

\variantsplitter

\addpersonalvariant{Анастасия Базарова}

\tasknumber{1}%
\task{%
    Длина волны света в~вакууме $\lambda = 600\,\text{нм}$.
    Какова частота этой световой волны?
    Какова длина этой волны в среде с показателем преломления $n = 1{,}6$?
    Может ли человек увидеть такую волну света, и если да, то какой именно цвет соответствует этим волнам в вакууме и в этой среде?
}
\answer{%
    \begin{align*}
    \nu &= \frac 1T = \frac 1{\lambda/c} = \frac c\lambda = \frac{ 3 \cdot 10^{8}\,\frac{\text{м}}{\text{с}} }{ 600\,\text{нм} } \approx 5{,}00 \cdot 10^{14}\,\text{Гц}, \\
    \nu' = \nu &\cbr{\text{или } T' = T} \implies \lambda' = v'T' = \frac vn T = \frac{vt}n = \frac \lambda n = \frac{ 600\,\text{нм} }{ 1{,}6 } \approx 375 \cdot 10^{-9}\,\text{м}.
    \\
    &\text{380 нм---фиол---440---син---485---гол---500---зел---565---жёл---590---оранж---625---крас---780 нм}
    \end{align*}
}
\solutionspace{180pt}

\tasknumber{2}%
\task{%
    При переходе электрона в атоме с одной стационарной орбиты на другую
    излучается фотон с энергией $1{,}01 \cdot 10^{-19}\,\text{Дж}$.
    Какова длина волны этой линии спектра?
    Постоянная Планка $h = 6{,}626 \cdot 10^{-34}\,\text{Дж}\cdot\text{с}$, скорость света $c = 3 \cdot 10^{8}\,\frac{\text{м}}{\text{с}}$.
}
\answer{%
    $
        E = h\nu = h \frac c\lambda
        \implies \lambda = \frac{hc}{E}
            = \frac{6{,}626 \cdot 10^{-34}\,\text{Дж}\cdot\text{с} \cdot { 3 \cdot 10^{8}\,\frac{\text{м}}{\text{с}} }}{ 1{,}01 \cdot 10^{-19}\,\text{Дж} }
            = 19{,}68 \cdot 10^{-7}\,\text{м}.
    $
}
\solutionspace{150pt}

\tasknumber{3}%
\task{%
    Излучение какой длины волны поглотил атом водорода, если полная энергия в атоме увеличилась на $2 \cdot 10^{-19}\,\text{Дж}$?
    Постоянная Планка $h = 6{,}626 \cdot 10^{-34}\,\text{Дж}\cdot\text{с}$, скорость света $c = 3 \cdot 10^{8}\,\frac{\text{м}}{\text{с}}$.
}
\answer{%
    $
        E = h\nu = h \frac c\lambda
        \implies \lambda = \frac{hc}{E}
            = \frac{6{,}626 \cdot 10^{-34}\,\text{Дж}\cdot\text{с} \cdot { 3 \cdot 10^{8}\,\frac{\text{м}}{\text{с}} }}{ 2 \cdot 10^{-19}\,\text{Дж} }
            = 9{,}94 \cdot 10^{-7}\,\text{м}.
    $
}
\solutionspace{150pt}

\tasknumber{4}%
\task{%
    Сделайте схематичный рисунок энергетических уровней атома водорода
    и отметьте на нём первый (основной) уровень и последующие.
    Сколько различных длин волн может испустить атом водорода,
    находящийся в 5-м возбуждённом состоянии?
    Отметьте все соответствующие переходы на рисунке и укажите,
    при каком переходе (среди отмеченных) частота излучённого фотона максимальна.
}
\answer{%
    $N = 10{,}0, \text{самая длинная линия}$
}

\variantsplitter

\addpersonalvariant{Артём Вартанов}

\tasknumber{1}%
\task{%
    Длина волны света в~вакууме $\lambda = 500\,\text{нм}$.
    Какова частота этой световой волны?
    Какова длина этой волны в среде с показателем преломления $n = 1{,}6$?
    Может ли человек увидеть такую волну света, и если да, то какой именно цвет соответствует этим волнам в вакууме и в этой среде?
}
\answer{%
    \begin{align*}
    \nu &= \frac 1T = \frac 1{\lambda/c} = \frac c\lambda = \frac{ 3 \cdot 10^{8}\,\frac{\text{м}}{\text{с}} }{ 500\,\text{нм} } \approx 6{,}00 \cdot 10^{14}\,\text{Гц}, \\
    \nu' = \nu &\cbr{\text{или } T' = T} \implies \lambda' = v'T' = \frac vn T = \frac{vt}n = \frac \lambda n = \frac{ 500\,\text{нм} }{ 1{,}6 } \approx 313 \cdot 10^{-9}\,\text{м}.
    \\
    &\text{380 нм---фиол---440---син---485---гол---500---зел---565---жёл---590---оранж---625---крас---780 нм}
    \end{align*}
}
\solutionspace{180pt}

\tasknumber{2}%
\task{%
    При переходе электрона в атоме с одной стационарной орбиты на другую
    излучается фотон с энергией $4{,}04 \cdot 10^{-19}\,\text{Дж}$.
    Какова длина волны этой линии спектра?
    Постоянная Планка $h = 6{,}626 \cdot 10^{-34}\,\text{Дж}\cdot\text{с}$, скорость света $c = 3 \cdot 10^{8}\,\frac{\text{м}}{\text{с}}$.
}
\answer{%
    $
        E = h\nu = h \frac c\lambda
        \implies \lambda = \frac{hc}{E}
            = \frac{6{,}626 \cdot 10^{-34}\,\text{Дж}\cdot\text{с} \cdot { 3 \cdot 10^{8}\,\frac{\text{м}}{\text{с}} }}{ 4{,}04 \cdot 10^{-19}\,\text{Дж} }
            = 4{,}92 \cdot 10^{-7}\,\text{м}.
    $
}
\solutionspace{150pt}

\tasknumber{3}%
\task{%
    Излучение какой длины волны поглотил атом водорода, если полная энергия в атоме увеличилась на $2 \cdot 10^{-19}\,\text{Дж}$?
    Постоянная Планка $h = 6{,}626 \cdot 10^{-34}\,\text{Дж}\cdot\text{с}$, скорость света $c = 3 \cdot 10^{8}\,\frac{\text{м}}{\text{с}}$.
}
\answer{%
    $
        E = h\nu = h \frac c\lambda
        \implies \lambda = \frac{hc}{E}
            = \frac{6{,}626 \cdot 10^{-34}\,\text{Дж}\cdot\text{с} \cdot { 3 \cdot 10^{8}\,\frac{\text{м}}{\text{с}} }}{ 2 \cdot 10^{-19}\,\text{Дж} }
            = 9{,}94 \cdot 10^{-7}\,\text{м}.
    $
}
\solutionspace{150pt}

\tasknumber{4}%
\task{%
    Сделайте схематичный рисунок энергетических уровней атома водорода
    и отметьте на нём первый (основной) уровень и последующие.
    Сколько различных длин волн может испустить атом водорода,
    находящийся в 3-м возбуждённом состоянии?
    Отметьте все соответствующие переходы на рисунке и укажите,
    при каком переходе (среди отмеченных) энергия излучённого фотона минимальна.
}
\answer{%
    $N = 3{,}0, \text{самая короткая линия}$
}

\variantsplitter

\addpersonalvariant{Арина Гайдукевич}

\tasknumber{1}%
\task{%
    Длина волны света в~вакууме $\lambda = 700\,\text{нм}$.
    Какова частота этой световой волны?
    Какова длина этой волны в среде с показателем преломления $n = 1{,}3$?
    Может ли человек увидеть такую волну света, и если да, то какой именно цвет соответствует этим волнам в вакууме и в этой среде?
}
\answer{%
    \begin{align*}
    \nu &= \frac 1T = \frac 1{\lambda/c} = \frac c\lambda = \frac{ 3 \cdot 10^{8}\,\frac{\text{м}}{\text{с}} }{ 700\,\text{нм} } \approx 4{,}29 \cdot 10^{14}\,\text{Гц}, \\
    \nu' = \nu &\cbr{\text{или } T' = T} \implies \lambda' = v'T' = \frac vn T = \frac{vt}n = \frac \lambda n = \frac{ 700\,\text{нм} }{ 1{,}3 } \approx 538 \cdot 10^{-9}\,\text{м}.
    \\
    &\text{380 нм---фиол---440---син---485---гол---500---зел---565---жёл---590---оранж---625---крас---780 нм}
    \end{align*}
}
\solutionspace{180pt}

\tasknumber{2}%
\task{%
    При переходе электрона в атоме с одной стационарной орбиты на другую
    излучается фотон с энергией $4{,}04 \cdot 10^{-19}\,\text{Дж}$.
    Какова длина волны этой линии спектра?
    Постоянная Планка $h = 6{,}626 \cdot 10^{-34}\,\text{Дж}\cdot\text{с}$, скорость света $c = 3 \cdot 10^{8}\,\frac{\text{м}}{\text{с}}$.
}
\answer{%
    $
        E = h\nu = h \frac c\lambda
        \implies \lambda = \frac{hc}{E}
            = \frac{6{,}626 \cdot 10^{-34}\,\text{Дж}\cdot\text{с} \cdot { 3 \cdot 10^{8}\,\frac{\text{м}}{\text{с}} }}{ 4{,}04 \cdot 10^{-19}\,\text{Дж} }
            = 4{,}92 \cdot 10^{-7}\,\text{м}.
    $
}
\solutionspace{150pt}

\tasknumber{3}%
\task{%
    Излучение какой длины волны поглотил атом водорода, если полная энергия в атоме увеличилась на $2 \cdot 10^{-19}\,\text{Дж}$?
    Постоянная Планка $h = 6{,}626 \cdot 10^{-34}\,\text{Дж}\cdot\text{с}$, скорость света $c = 3 \cdot 10^{8}\,\frac{\text{м}}{\text{с}}$.
}
\answer{%
    $
        E = h\nu = h \frac c\lambda
        \implies \lambda = \frac{hc}{E}
            = \frac{6{,}626 \cdot 10^{-34}\,\text{Дж}\cdot\text{с} \cdot { 3 \cdot 10^{8}\,\frac{\text{м}}{\text{с}} }}{ 2 \cdot 10^{-19}\,\text{Дж} }
            = 9{,}94 \cdot 10^{-7}\,\text{м}.
    $
}
\solutionspace{150pt}

\tasknumber{4}%
\task{%
    Сделайте схематичный рисунок энергетических уровней атома водорода
    и отметьте на нём первый (основной) уровень и последующие.
    Сколько различных длин волн может испустить атом водорода,
    находящийся в 5-м возбуждённом состоянии?
    Отметьте все соответствующие переходы на рисунке и укажите,
    при каком переходе (среди отмеченных) длина волны излучённого фотона минимальна.
}
\answer{%
    $N = 10{,}0, \text{самая длинная линия}$
}

\variantsplitter

\addpersonalvariant{Виктор Галан}

\tasknumber{1}%
\task{%
    Длина волны света в~вакууме $\lambda = 600\,\text{нм}$.
    Какова частота этой световой волны?
    Какова длина этой волны в среде с показателем преломления $n = 1{,}4$?
    Может ли человек увидеть такую волну света, и если да, то какой именно цвет соответствует этим волнам в вакууме и в этой среде?
}
\answer{%
    \begin{align*}
    \nu &= \frac 1T = \frac 1{\lambda/c} = \frac c\lambda = \frac{ 3 \cdot 10^{8}\,\frac{\text{м}}{\text{с}} }{ 600\,\text{нм} } \approx 5{,}00 \cdot 10^{14}\,\text{Гц}, \\
    \nu' = \nu &\cbr{\text{или } T' = T} \implies \lambda' = v'T' = \frac vn T = \frac{vt}n = \frac \lambda n = \frac{ 600\,\text{нм} }{ 1{,}4 } \approx 429 \cdot 10^{-9}\,\text{м}.
    \\
    &\text{380 нм---фиол---440---син---485---гол---500---зел---565---жёл---590---оранж---625---крас---780 нм}
    \end{align*}
}
\solutionspace{180pt}

\tasknumber{2}%
\task{%
    При переходе электрона в атоме с одной стационарной орбиты на другую
    излучается фотон с энергией $7{,}07 \cdot 10^{-19}\,\text{Дж}$.
    Какова длина волны этой линии спектра?
    Постоянная Планка $h = 6{,}626 \cdot 10^{-34}\,\text{Дж}\cdot\text{с}$, скорость света $c = 3 \cdot 10^{8}\,\frac{\text{м}}{\text{с}}$.
}
\answer{%
    $
        E = h\nu = h \frac c\lambda
        \implies \lambda = \frac{hc}{E}
            = \frac{6{,}626 \cdot 10^{-34}\,\text{Дж}\cdot\text{с} \cdot { 3 \cdot 10^{8}\,\frac{\text{м}}{\text{с}} }}{ 7{,}07 \cdot 10^{-19}\,\text{Дж} }
            = 2{,}81 \cdot 10^{-7}\,\text{м}.
    $
}
\solutionspace{150pt}

\tasknumber{3}%
\task{%
    Излучение какой длины волны поглотил атом водорода, если полная энергия в атоме увеличилась на $6 \cdot 10^{-19}\,\text{Дж}$?
    Постоянная Планка $h = 6{,}626 \cdot 10^{-34}\,\text{Дж}\cdot\text{с}$, скорость света $c = 3 \cdot 10^{8}\,\frac{\text{м}}{\text{с}}$.
}
\answer{%
    $
        E = h\nu = h \frac c\lambda
        \implies \lambda = \frac{hc}{E}
            = \frac{6{,}626 \cdot 10^{-34}\,\text{Дж}\cdot\text{с} \cdot { 3 \cdot 10^{8}\,\frac{\text{м}}{\text{с}} }}{ 6 \cdot 10^{-19}\,\text{Дж} }
            = 3{,}31 \cdot 10^{-7}\,\text{м}.
    $
}
\solutionspace{150pt}

\tasknumber{4}%
\task{%
    Сделайте схематичный рисунок энергетических уровней атома водорода
    и отметьте на нём первый (основной) уровень и последующие.
    Сколько различных длин волн может испустить атом водорода,
    находящийся в 4-м возбуждённом состоянии?
    Отметьте все соответствующие переходы на рисунке и укажите,
    при каком переходе (среди отмеченных) длина волны излучённого фотона минимальна.
}
\answer{%
    $N = 6{,}0, \text{самая длинная линия}$
}

\variantsplitter

\addpersonalvariant{Дарья Дербышева}

\tasknumber{1}%
\task{%
    Длина волны света в~вакууме $\lambda = 700\,\text{нм}$.
    Какова частота этой световой волны?
    Какова длина этой волны в среде с показателем преломления $n = 1{,}5$?
    Может ли человек увидеть такую волну света, и если да, то какой именно цвет соответствует этим волнам в вакууме и в этой среде?
}
\answer{%
    \begin{align*}
    \nu &= \frac 1T = \frac 1{\lambda/c} = \frac c\lambda = \frac{ 3 \cdot 10^{8}\,\frac{\text{м}}{\text{с}} }{ 700\,\text{нм} } \approx 4{,}29 \cdot 10^{14}\,\text{Гц}, \\
    \nu' = \nu &\cbr{\text{или } T' = T} \implies \lambda' = v'T' = \frac vn T = \frac{vt}n = \frac \lambda n = \frac{ 700\,\text{нм} }{ 1{,}5 } \approx 467 \cdot 10^{-9}\,\text{м}.
    \\
    &\text{380 нм---фиол---440---син---485---гол---500---зел---565---жёл---590---оранж---625---крас---780 нм}
    \end{align*}
}
\solutionspace{180pt}

\tasknumber{2}%
\task{%
    При переходе электрона в атоме с одной стационарной орбиты на другую
    излучается фотон с энергией $4{,}04 \cdot 10^{-19}\,\text{Дж}$.
    Какова длина волны этой линии спектра?
    Постоянная Планка $h = 6{,}626 \cdot 10^{-34}\,\text{Дж}\cdot\text{с}$, скорость света $c = 3 \cdot 10^{8}\,\frac{\text{м}}{\text{с}}$.
}
\answer{%
    $
        E = h\nu = h \frac c\lambda
        \implies \lambda = \frac{hc}{E}
            = \frac{6{,}626 \cdot 10^{-34}\,\text{Дж}\cdot\text{с} \cdot { 3 \cdot 10^{8}\,\frac{\text{м}}{\text{с}} }}{ 4{,}04 \cdot 10^{-19}\,\text{Дж} }
            = 4{,}92 \cdot 10^{-7}\,\text{м}.
    $
}
\solutionspace{150pt}

\tasknumber{3}%
\task{%
    Излучение какой длины волны поглотил атом водорода, если полная энергия в атоме увеличилась на $2 \cdot 10^{-19}\,\text{Дж}$?
    Постоянная Планка $h = 6{,}626 \cdot 10^{-34}\,\text{Дж}\cdot\text{с}$, скорость света $c = 3 \cdot 10^{8}\,\frac{\text{м}}{\text{с}}$.
}
\answer{%
    $
        E = h\nu = h \frac c\lambda
        \implies \lambda = \frac{hc}{E}
            = \frac{6{,}626 \cdot 10^{-34}\,\text{Дж}\cdot\text{с} \cdot { 3 \cdot 10^{8}\,\frac{\text{м}}{\text{с}} }}{ 2 \cdot 10^{-19}\,\text{Дж} }
            = 9{,}94 \cdot 10^{-7}\,\text{м}.
    $
}
\solutionspace{150pt}

\tasknumber{4}%
\task{%
    Сделайте схематичный рисунок энергетических уровней атома водорода
    и отметьте на нём первый (основной) уровень и последующие.
    Сколько различных длин волн может испустить атом водорода,
    находящийся в 5-м возбуждённом состоянии?
    Отметьте все соответствующие переходы на рисунке и укажите,
    при каком переходе (среди отмеченных) энергия излучённого фотона максимальна.
}
\answer{%
    $N = 10{,}0, \text{самая длинная линия}$
}

\variantsplitter

\addpersonalvariant{Данил Долматов}

\tasknumber{1}%
\task{%
    Длина волны света в~вакууме $\lambda = 700\,\text{нм}$.
    Какова частота этой световой волны?
    Какова длина этой волны в среде с показателем преломления $n = 1{,}6$?
    Может ли человек увидеть такую волну света, и если да, то какой именно цвет соответствует этим волнам в вакууме и в этой среде?
}
\answer{%
    \begin{align*}
    \nu &= \frac 1T = \frac 1{\lambda/c} = \frac c\lambda = \frac{ 3 \cdot 10^{8}\,\frac{\text{м}}{\text{с}} }{ 700\,\text{нм} } \approx 4{,}29 \cdot 10^{14}\,\text{Гц}, \\
    \nu' = \nu &\cbr{\text{или } T' = T} \implies \lambda' = v'T' = \frac vn T = \frac{vt}n = \frac \lambda n = \frac{ 700\,\text{нм} }{ 1{,}6 } \approx 438 \cdot 10^{-9}\,\text{м}.
    \\
    &\text{380 нм---фиол---440---син---485---гол---500---зел---565---жёл---590---оранж---625---крас---780 нм}
    \end{align*}
}
\solutionspace{180pt}

\tasknumber{2}%
\task{%
    При переходе электрона в атоме с одной стационарной орбиты на другую
    излучается фотон с энергией $2{,}02 \cdot 10^{-19}\,\text{Дж}$.
    Какова длина волны этой линии спектра?
    Постоянная Планка $h = 6{,}626 \cdot 10^{-34}\,\text{Дж}\cdot\text{с}$, скорость света $c = 3 \cdot 10^{8}\,\frac{\text{м}}{\text{с}}$.
}
\answer{%
    $
        E = h\nu = h \frac c\lambda
        \implies \lambda = \frac{hc}{E}
            = \frac{6{,}626 \cdot 10^{-34}\,\text{Дж}\cdot\text{с} \cdot { 3 \cdot 10^{8}\,\frac{\text{м}}{\text{с}} }}{ 2{,}02 \cdot 10^{-19}\,\text{Дж} }
            = 9{,}84 \cdot 10^{-7}\,\text{м}.
    $
}
\solutionspace{150pt}

\tasknumber{3}%
\task{%
    Излучение какой длины волны поглотил атом водорода, если полная энергия в атоме увеличилась на $2 \cdot 10^{-19}\,\text{Дж}$?
    Постоянная Планка $h = 6{,}626 \cdot 10^{-34}\,\text{Дж}\cdot\text{с}$, скорость света $c = 3 \cdot 10^{8}\,\frac{\text{м}}{\text{с}}$.
}
\answer{%
    $
        E = h\nu = h \frac c\lambda
        \implies \lambda = \frac{hc}{E}
            = \frac{6{,}626 \cdot 10^{-34}\,\text{Дж}\cdot\text{с} \cdot { 3 \cdot 10^{8}\,\frac{\text{м}}{\text{с}} }}{ 2 \cdot 10^{-19}\,\text{Дж} }
            = 9{,}94 \cdot 10^{-7}\,\text{м}.
    $
}
\solutionspace{150pt}

\tasknumber{4}%
\task{%
    Сделайте схематичный рисунок энергетических уровней атома водорода
    и отметьте на нём первый (основной) уровень и последующие.
    Сколько различных длин волн может испустить атом водорода,
    находящийся в 4-м возбуждённом состоянии?
    Отметьте все соответствующие переходы на рисунке и укажите,
    при каком переходе (среди отмеченных) частота излучённого фотона максимальна.
}
\answer{%
    $N = 6{,}0, \text{самая длинная линия}$
}

\variantsplitter

\addpersonalvariant{Зинаида Евдокимова}

\tasknumber{1}%
\task{%
    Длина волны света в~вакууме $\lambda = 700\,\text{нм}$.
    Какова частота этой световой волны?
    Какова длина этой волны в среде с показателем преломления $n = 1{,}4$?
    Может ли человек увидеть такую волну света, и если да, то какой именно цвет соответствует этим волнам в вакууме и в этой среде?
}
\answer{%
    \begin{align*}
    \nu &= \frac 1T = \frac 1{\lambda/c} = \frac c\lambda = \frac{ 3 \cdot 10^{8}\,\frac{\text{м}}{\text{с}} }{ 700\,\text{нм} } \approx 4{,}29 \cdot 10^{14}\,\text{Гц}, \\
    \nu' = \nu &\cbr{\text{или } T' = T} \implies \lambda' = v'T' = \frac vn T = \frac{vt}n = \frac \lambda n = \frac{ 700\,\text{нм} }{ 1{,}4 } \approx 500 \cdot 10^{-9}\,\text{м}.
    \\
    &\text{380 нм---фиол---440---син---485---гол---500---зел---565---жёл---590---оранж---625---крас---780 нм}
    \end{align*}
}
\solutionspace{180pt}

\tasknumber{2}%
\task{%
    При переходе электрона в атоме с одной стационарной орбиты на другую
    излучается фотон с энергией $7{,}07 \cdot 10^{-19}\,\text{Дж}$.
    Какова длина волны этой линии спектра?
    Постоянная Планка $h = 6{,}626 \cdot 10^{-34}\,\text{Дж}\cdot\text{с}$, скорость света $c = 3 \cdot 10^{8}\,\frac{\text{м}}{\text{с}}$.
}
\answer{%
    $
        E = h\nu = h \frac c\lambda
        \implies \lambda = \frac{hc}{E}
            = \frac{6{,}626 \cdot 10^{-34}\,\text{Дж}\cdot\text{с} \cdot { 3 \cdot 10^{8}\,\frac{\text{м}}{\text{с}} }}{ 7{,}07 \cdot 10^{-19}\,\text{Дж} }
            = 2{,}81 \cdot 10^{-7}\,\text{м}.
    $
}
\solutionspace{150pt}

\tasknumber{3}%
\task{%
    Излучение какой длины волны поглотил атом водорода, если полная энергия в атоме увеличилась на $6 \cdot 10^{-19}\,\text{Дж}$?
    Постоянная Планка $h = 6{,}626 \cdot 10^{-34}\,\text{Дж}\cdot\text{с}$, скорость света $c = 3 \cdot 10^{8}\,\frac{\text{м}}{\text{с}}$.
}
\answer{%
    $
        E = h\nu = h \frac c\lambda
        \implies \lambda = \frac{hc}{E}
            = \frac{6{,}626 \cdot 10^{-34}\,\text{Дж}\cdot\text{с} \cdot { 3 \cdot 10^{8}\,\frac{\text{м}}{\text{с}} }}{ 6 \cdot 10^{-19}\,\text{Дж} }
            = 3{,}31 \cdot 10^{-7}\,\text{м}.
    $
}
\solutionspace{150pt}

\tasknumber{4}%
\task{%
    Сделайте схематичный рисунок энергетических уровней атома водорода
    и отметьте на нём первый (основной) уровень и последующие.
    Сколько различных длин волн может испустить атом водорода,
    находящийся в 3-м возбуждённом состоянии?
    Отметьте все соответствующие переходы на рисунке и укажите,
    при каком переходе (среди отмеченных) длина волны излучённого фотона максимальна.
}
\answer{%
    $N = 3{,}0, \text{самая короткая линия}$
}

\variantsplitter

\addpersonalvariant{Софья Евсикова}

\tasknumber{1}%
\task{%
    Длина волны света в~вакууме $\lambda = 600\,\text{нм}$.
    Какова частота этой световой волны?
    Какова длина этой волны в среде с показателем преломления $n = 1{,}5$?
    Может ли человек увидеть такую волну света, и если да, то какой именно цвет соответствует этим волнам в вакууме и в этой среде?
}
\answer{%
    \begin{align*}
    \nu &= \frac 1T = \frac 1{\lambda/c} = \frac c\lambda = \frac{ 3 \cdot 10^{8}\,\frac{\text{м}}{\text{с}} }{ 600\,\text{нм} } \approx 5{,}00 \cdot 10^{14}\,\text{Гц}, \\
    \nu' = \nu &\cbr{\text{или } T' = T} \implies \lambda' = v'T' = \frac vn T = \frac{vt}n = \frac \lambda n = \frac{ 600\,\text{нм} }{ 1{,}5 } \approx 400 \cdot 10^{-9}\,\text{м}.
    \\
    &\text{380 нм---фиол---440---син---485---гол---500---зел---565---жёл---590---оранж---625---крас---780 нм}
    \end{align*}
}
\solutionspace{180pt}

\tasknumber{2}%
\task{%
    При переходе электрона в атоме с одной стационарной орбиты на другую
    излучается фотон с энергией $4{,}04 \cdot 10^{-19}\,\text{Дж}$.
    Какова длина волны этой линии спектра?
    Постоянная Планка $h = 6{,}626 \cdot 10^{-34}\,\text{Дж}\cdot\text{с}$, скорость света $c = 3 \cdot 10^{8}\,\frac{\text{м}}{\text{с}}$.
}
\answer{%
    $
        E = h\nu = h \frac c\lambda
        \implies \lambda = \frac{hc}{E}
            = \frac{6{,}626 \cdot 10^{-34}\,\text{Дж}\cdot\text{с} \cdot { 3 \cdot 10^{8}\,\frac{\text{м}}{\text{с}} }}{ 4{,}04 \cdot 10^{-19}\,\text{Дж} }
            = 4{,}92 \cdot 10^{-7}\,\text{м}.
    $
}
\solutionspace{150pt}

\tasknumber{3}%
\task{%
    Излучение какой длины волны поглотил атом водорода, если полная энергия в атоме увеличилась на $2 \cdot 10^{-19}\,\text{Дж}$?
    Постоянная Планка $h = 6{,}626 \cdot 10^{-34}\,\text{Дж}\cdot\text{с}$, скорость света $c = 3 \cdot 10^{8}\,\frac{\text{м}}{\text{с}}$.
}
\answer{%
    $
        E = h\nu = h \frac c\lambda
        \implies \lambda = \frac{hc}{E}
            = \frac{6{,}626 \cdot 10^{-34}\,\text{Дж}\cdot\text{с} \cdot { 3 \cdot 10^{8}\,\frac{\text{м}}{\text{с}} }}{ 2 \cdot 10^{-19}\,\text{Дж} }
            = 9{,}94 \cdot 10^{-7}\,\text{м}.
    $
}
\solutionspace{150pt}

\tasknumber{4}%
\task{%
    Сделайте схематичный рисунок энергетических уровней атома водорода
    и отметьте на нём первый (основной) уровень и последующие.
    Сколько различных длин волн может испустить атом водорода,
    находящийся в 4-м возбуждённом состоянии?
    Отметьте все соответствующие переходы на рисунке и укажите,
    при каком переходе (среди отмеченных) частота излучённого фотона минимальна.
}
\answer{%
    $N = 6{,}0, \text{самая короткая линия}$
}

\variantsplitter

\addpersonalvariant{София Заика}

\tasknumber{1}%
\task{%
    Длина волны света в~вакууме $\lambda = 700\,\text{нм}$.
    Какова частота этой световой волны?
    Какова длина этой волны в среде с показателем преломления $n = 1{,}6$?
    Может ли человек увидеть такую волну света, и если да, то какой именно цвет соответствует этим волнам в вакууме и в этой среде?
}
\answer{%
    \begin{align*}
    \nu &= \frac 1T = \frac 1{\lambda/c} = \frac c\lambda = \frac{ 3 \cdot 10^{8}\,\frac{\text{м}}{\text{с}} }{ 700\,\text{нм} } \approx 4{,}29 \cdot 10^{14}\,\text{Гц}, \\
    \nu' = \nu &\cbr{\text{или } T' = T} \implies \lambda' = v'T' = \frac vn T = \frac{vt}n = \frac \lambda n = \frac{ 700\,\text{нм} }{ 1{,}6 } \approx 438 \cdot 10^{-9}\,\text{м}.
    \\
    &\text{380 нм---фиол---440---син---485---гол---500---зел---565---жёл---590---оранж---625---крас---780 нм}
    \end{align*}
}
\solutionspace{180pt}

\tasknumber{2}%
\task{%
    При переходе электрона в атоме с одной стационарной орбиты на другую
    излучается фотон с энергией $0{,}55 \cdot 10^{-19}\,\text{Дж}$.
    Какова длина волны этой линии спектра?
    Постоянная Планка $h = 6{,}626 \cdot 10^{-34}\,\text{Дж}\cdot\text{с}$, скорость света $c = 3 \cdot 10^{8}\,\frac{\text{м}}{\text{с}}$.
}
\answer{%
    $
        E = h\nu = h \frac c\lambda
        \implies \lambda = \frac{hc}{E}
            = \frac{6{,}626 \cdot 10^{-34}\,\text{Дж}\cdot\text{с} \cdot { 3 \cdot 10^{8}\,\frac{\text{м}}{\text{с}} }}{ 0{,}55 \cdot 10^{-19}\,\text{Дж} }
            = 36{,}14 \cdot 10^{-7}\,\text{м}.
    $
}
\solutionspace{150pt}

\tasknumber{3}%
\task{%
    Излучение какой длины волны поглотил атом водорода, если полная энергия в атоме увеличилась на $6 \cdot 10^{-19}\,\text{Дж}$?
    Постоянная Планка $h = 6{,}626 \cdot 10^{-34}\,\text{Дж}\cdot\text{с}$, скорость света $c = 3 \cdot 10^{8}\,\frac{\text{м}}{\text{с}}$.
}
\answer{%
    $
        E = h\nu = h \frac c\lambda
        \implies \lambda = \frac{hc}{E}
            = \frac{6{,}626 \cdot 10^{-34}\,\text{Дж}\cdot\text{с} \cdot { 3 \cdot 10^{8}\,\frac{\text{м}}{\text{с}} }}{ 6 \cdot 10^{-19}\,\text{Дж} }
            = 3{,}31 \cdot 10^{-7}\,\text{м}.
    $
}
\solutionspace{150pt}

\tasknumber{4}%
\task{%
    Сделайте схематичный рисунок энергетических уровней атома водорода
    и отметьте на нём первый (основной) уровень и последующие.
    Сколько различных длин волн может испустить атом водорода,
    находящийся в 5-м возбуждённом состоянии?
    Отметьте все соответствующие переходы на рисунке и укажите,
    при каком переходе (среди отмеченных) энергия излучённого фотона минимальна.
}
\answer{%
    $N = 10{,}0, \text{самая короткая линия}$
}

\variantsplitter

\addpersonalvariant{Софья Иосилевская}

\tasknumber{1}%
\task{%
    Длина волны света в~вакууме $\lambda = 500\,\text{нм}$.
    Какова частота этой световой волны?
    Какова длина этой волны в среде с показателем преломления $n = 1{,}7$?
    Может ли человек увидеть такую волну света, и если да, то какой именно цвет соответствует этим волнам в вакууме и в этой среде?
}
\answer{%
    \begin{align*}
    \nu &= \frac 1T = \frac 1{\lambda/c} = \frac c\lambda = \frac{ 3 \cdot 10^{8}\,\frac{\text{м}}{\text{с}} }{ 500\,\text{нм} } \approx 6{,}00 \cdot 10^{14}\,\text{Гц}, \\
    \nu' = \nu &\cbr{\text{или } T' = T} \implies \lambda' = v'T' = \frac vn T = \frac{vt}n = \frac \lambda n = \frac{ 500\,\text{нм} }{ 1{,}7 } \approx 294 \cdot 10^{-9}\,\text{м}.
    \\
    &\text{380 нм---фиол---440---син---485---гол---500---зел---565---жёл---590---оранж---625---крас---780 нм}
    \end{align*}
}
\solutionspace{180pt}

\tasknumber{2}%
\task{%
    При переходе электрона в атоме с одной стационарной орбиты на другую
    излучается фотон с энергией $5{,}05 \cdot 10^{-19}\,\text{Дж}$.
    Какова длина волны этой линии спектра?
    Постоянная Планка $h = 6{,}626 \cdot 10^{-34}\,\text{Дж}\cdot\text{с}$, скорость света $c = 3 \cdot 10^{8}\,\frac{\text{м}}{\text{с}}$.
}
\answer{%
    $
        E = h\nu = h \frac c\lambda
        \implies \lambda = \frac{hc}{E}
            = \frac{6{,}626 \cdot 10^{-34}\,\text{Дж}\cdot\text{с} \cdot { 3 \cdot 10^{8}\,\frac{\text{м}}{\text{с}} }}{ 5{,}05 \cdot 10^{-19}\,\text{Дж} }
            = 3{,}94 \cdot 10^{-7}\,\text{м}.
    $
}
\solutionspace{150pt}

\tasknumber{3}%
\task{%
    Излучение какой длины волны поглотил атом водорода, если полная энергия в атоме увеличилась на $3 \cdot 10^{-19}\,\text{Дж}$?
    Постоянная Планка $h = 6{,}626 \cdot 10^{-34}\,\text{Дж}\cdot\text{с}$, скорость света $c = 3 \cdot 10^{8}\,\frac{\text{м}}{\text{с}}$.
}
\answer{%
    $
        E = h\nu = h \frac c\lambda
        \implies \lambda = \frac{hc}{E}
            = \frac{6{,}626 \cdot 10^{-34}\,\text{Дж}\cdot\text{с} \cdot { 3 \cdot 10^{8}\,\frac{\text{м}}{\text{с}} }}{ 3 \cdot 10^{-19}\,\text{Дж} }
            = 6{,}63 \cdot 10^{-7}\,\text{м}.
    $
}
\solutionspace{150pt}

\tasknumber{4}%
\task{%
    Сделайте схематичный рисунок энергетических уровней атома водорода
    и отметьте на нём первый (основной) уровень и последующие.
    Сколько различных длин волн может испустить атом водорода,
    находящийся в 4-м возбуждённом состоянии?
    Отметьте все соответствующие переходы на рисунке и укажите,
    при каком переходе (среди отмеченных) энергия излучённого фотона максимальна.
}
\answer{%
    $N = 6{,}0, \text{самая длинная линия}$
}

\variantsplitter

\addpersonalvariant{Маргарита Карманова}

\tasknumber{1}%
\task{%
    Длина волны света в~вакууме $\lambda = 500\,\text{нм}$.
    Какова частота этой световой волны?
    Какова длина этой волны в среде с показателем преломления $n = 1{,}4$?
    Может ли человек увидеть такую волну света, и если да, то какой именно цвет соответствует этим волнам в вакууме и в этой среде?
}
\answer{%
    \begin{align*}
    \nu &= \frac 1T = \frac 1{\lambda/c} = \frac c\lambda = \frac{ 3 \cdot 10^{8}\,\frac{\text{м}}{\text{с}} }{ 500\,\text{нм} } \approx 6{,}00 \cdot 10^{14}\,\text{Гц}, \\
    \nu' = \nu &\cbr{\text{или } T' = T} \implies \lambda' = v'T' = \frac vn T = \frac{vt}n = \frac \lambda n = \frac{ 500\,\text{нм} }{ 1{,}4 } \approx 357 \cdot 10^{-9}\,\text{м}.
    \\
    &\text{380 нм---фиол---440---син---485---гол---500---зел---565---жёл---590---оранж---625---крас---780 нм}
    \end{align*}
}
\solutionspace{180pt}

\tasknumber{2}%
\task{%
    При переходе электрона в атоме с одной стационарной орбиты на другую
    излучается фотон с энергией $7{,}07 \cdot 10^{-19}\,\text{Дж}$.
    Какова длина волны этой линии спектра?
    Постоянная Планка $h = 6{,}626 \cdot 10^{-34}\,\text{Дж}\cdot\text{с}$, скорость света $c = 3 \cdot 10^{8}\,\frac{\text{м}}{\text{с}}$.
}
\answer{%
    $
        E = h\nu = h \frac c\lambda
        \implies \lambda = \frac{hc}{E}
            = \frac{6{,}626 \cdot 10^{-34}\,\text{Дж}\cdot\text{с} \cdot { 3 \cdot 10^{8}\,\frac{\text{м}}{\text{с}} }}{ 7{,}07 \cdot 10^{-19}\,\text{Дж} }
            = 2{,}81 \cdot 10^{-7}\,\text{м}.
    $
}
\solutionspace{150pt}

\tasknumber{3}%
\task{%
    Излучение какой длины волны поглотил атом водорода, если полная энергия в атоме увеличилась на $3 \cdot 10^{-19}\,\text{Дж}$?
    Постоянная Планка $h = 6{,}626 \cdot 10^{-34}\,\text{Дж}\cdot\text{с}$, скорость света $c = 3 \cdot 10^{8}\,\frac{\text{м}}{\text{с}}$.
}
\answer{%
    $
        E = h\nu = h \frac c\lambda
        \implies \lambda = \frac{hc}{E}
            = \frac{6{,}626 \cdot 10^{-34}\,\text{Дж}\cdot\text{с} \cdot { 3 \cdot 10^{8}\,\frac{\text{м}}{\text{с}} }}{ 3 \cdot 10^{-19}\,\text{Дж} }
            = 6{,}63 \cdot 10^{-7}\,\text{м}.
    $
}
\solutionspace{150pt}

\tasknumber{4}%
\task{%
    Сделайте схематичный рисунок энергетических уровней атома водорода
    и отметьте на нём первый (основной) уровень и последующие.
    Сколько различных длин волн может испустить атом водорода,
    находящийся в 3-м возбуждённом состоянии?
    Отметьте все соответствующие переходы на рисунке и укажите,
    при каком переходе (среди отмеченных) длина волны излучённого фотона минимальна.
}
\answer{%
    $N = 3{,}0, \text{самая длинная линия}$
}

\variantsplitter

\addpersonalvariant{Варвара Карпенко}

\tasknumber{1}%
\task{%
    Длина волны света в~вакууме $\lambda = 700\,\text{нм}$.
    Какова частота этой световой волны?
    Какова длина этой волны в среде с показателем преломления $n = 1{,}3$?
    Может ли человек увидеть такую волну света, и если да, то какой именно цвет соответствует этим волнам в вакууме и в этой среде?
}
\answer{%
    \begin{align*}
    \nu &= \frac 1T = \frac 1{\lambda/c} = \frac c\lambda = \frac{ 3 \cdot 10^{8}\,\frac{\text{м}}{\text{с}} }{ 700\,\text{нм} } \approx 4{,}29 \cdot 10^{14}\,\text{Гц}, \\
    \nu' = \nu &\cbr{\text{или } T' = T} \implies \lambda' = v'T' = \frac vn T = \frac{vt}n = \frac \lambda n = \frac{ 700\,\text{нм} }{ 1{,}3 } \approx 538 \cdot 10^{-9}\,\text{м}.
    \\
    &\text{380 нм---фиол---440---син---485---гол---500---зел---565---жёл---590---оранж---625---крас---780 нм}
    \end{align*}
}
\solutionspace{180pt}

\tasknumber{2}%
\task{%
    При переходе электрона в атоме с одной стационарной орбиты на другую
    излучается фотон с энергией $5{,}05 \cdot 10^{-19}\,\text{Дж}$.
    Какова длина волны этой линии спектра?
    Постоянная Планка $h = 6{,}626 \cdot 10^{-34}\,\text{Дж}\cdot\text{с}$, скорость света $c = 3 \cdot 10^{8}\,\frac{\text{м}}{\text{с}}$.
}
\answer{%
    $
        E = h\nu = h \frac c\lambda
        \implies \lambda = \frac{hc}{E}
            = \frac{6{,}626 \cdot 10^{-34}\,\text{Дж}\cdot\text{с} \cdot { 3 \cdot 10^{8}\,\frac{\text{м}}{\text{с}} }}{ 5{,}05 \cdot 10^{-19}\,\text{Дж} }
            = 3{,}94 \cdot 10^{-7}\,\text{м}.
    $
}
\solutionspace{150pt}

\tasknumber{3}%
\task{%
    Излучение какой длины волны поглотил атом водорода, если полная энергия в атоме увеличилась на $3 \cdot 10^{-19}\,\text{Дж}$?
    Постоянная Планка $h = 6{,}626 \cdot 10^{-34}\,\text{Дж}\cdot\text{с}$, скорость света $c = 3 \cdot 10^{8}\,\frac{\text{м}}{\text{с}}$.
}
\answer{%
    $
        E = h\nu = h \frac c\lambda
        \implies \lambda = \frac{hc}{E}
            = \frac{6{,}626 \cdot 10^{-34}\,\text{Дж}\cdot\text{с} \cdot { 3 \cdot 10^{8}\,\frac{\text{м}}{\text{с}} }}{ 3 \cdot 10^{-19}\,\text{Дж} }
            = 6{,}63 \cdot 10^{-7}\,\text{м}.
    $
}
\solutionspace{150pt}

\tasknumber{4}%
\task{%
    Сделайте схематичный рисунок энергетических уровней атома водорода
    и отметьте на нём первый (основной) уровень и последующие.
    Сколько различных длин волн может испустить атом водорода,
    находящийся в 3-м возбуждённом состоянии?
    Отметьте все соответствующие переходы на рисунке и укажите,
    при каком переходе (среди отмеченных) энергия излучённого фотона максимальна.
}
\answer{%
    $N = 3{,}0, \text{самая длинная линия}$
}

\variantsplitter

\addpersonalvariant{Виктория Кемайкина}

\tasknumber{1}%
\task{%
    Длина волны света в~вакууме $\lambda = 600\,\text{нм}$.
    Какова частота этой световой волны?
    Какова длина этой волны в среде с показателем преломления $n = 1{,}7$?
    Может ли человек увидеть такую волну света, и если да, то какой именно цвет соответствует этим волнам в вакууме и в этой среде?
}
\answer{%
    \begin{align*}
    \nu &= \frac 1T = \frac 1{\lambda/c} = \frac c\lambda = \frac{ 3 \cdot 10^{8}\,\frac{\text{м}}{\text{с}} }{ 600\,\text{нм} } \approx 5{,}00 \cdot 10^{14}\,\text{Гц}, \\
    \nu' = \nu &\cbr{\text{или } T' = T} \implies \lambda' = v'T' = \frac vn T = \frac{vt}n = \frac \lambda n = \frac{ 600\,\text{нм} }{ 1{,}7 } \approx 353 \cdot 10^{-9}\,\text{м}.
    \\
    &\text{380 нм---фиол---440---син---485---гол---500---зел---565---жёл---590---оранж---625---крас---780 нм}
    \end{align*}
}
\solutionspace{180pt}

\tasknumber{2}%
\task{%
    При переходе электрона в атоме с одной стационарной орбиты на другую
    излучается фотон с энергией $2{,}02 \cdot 10^{-19}\,\text{Дж}$.
    Какова длина волны этой линии спектра?
    Постоянная Планка $h = 6{,}626 \cdot 10^{-34}\,\text{Дж}\cdot\text{с}$, скорость света $c = 3 \cdot 10^{8}\,\frac{\text{м}}{\text{с}}$.
}
\answer{%
    $
        E = h\nu = h \frac c\lambda
        \implies \lambda = \frac{hc}{E}
            = \frac{6{,}626 \cdot 10^{-34}\,\text{Дж}\cdot\text{с} \cdot { 3 \cdot 10^{8}\,\frac{\text{м}}{\text{с}} }}{ 2{,}02 \cdot 10^{-19}\,\text{Дж} }
            = 9{,}84 \cdot 10^{-7}\,\text{м}.
    $
}
\solutionspace{150pt}

\tasknumber{3}%
\task{%
    Излучение какой длины волны поглотил атом водорода, если полная энергия в атоме увеличилась на $2 \cdot 10^{-19}\,\text{Дж}$?
    Постоянная Планка $h = 6{,}626 \cdot 10^{-34}\,\text{Дж}\cdot\text{с}$, скорость света $c = 3 \cdot 10^{8}\,\frac{\text{м}}{\text{с}}$.
}
\answer{%
    $
        E = h\nu = h \frac c\lambda
        \implies \lambda = \frac{hc}{E}
            = \frac{6{,}626 \cdot 10^{-34}\,\text{Дж}\cdot\text{с} \cdot { 3 \cdot 10^{8}\,\frac{\text{м}}{\text{с}} }}{ 2 \cdot 10^{-19}\,\text{Дж} }
            = 9{,}94 \cdot 10^{-7}\,\text{м}.
    $
}
\solutionspace{150pt}

\tasknumber{4}%
\task{%
    Сделайте схематичный рисунок энергетических уровней атома водорода
    и отметьте на нём первый (основной) уровень и последующие.
    Сколько различных длин волн может испустить атом водорода,
    находящийся в 4-м возбуждённом состоянии?
    Отметьте все соответствующие переходы на рисунке и укажите,
    при каком переходе (среди отмеченных) энергия излучённого фотона минимальна.
}
\answer{%
    $N = 6{,}0, \text{самая короткая линия}$
}

\variantsplitter

\addpersonalvariant{Софья Корянова}

\tasknumber{1}%
\task{%
    Длина волны света в~вакууме $\lambda = 600\,\text{нм}$.
    Какова частота этой световой волны?
    Какова длина этой волны в среде с показателем преломления $n = 1{,}6$?
    Может ли человек увидеть такую волну света, и если да, то какой именно цвет соответствует этим волнам в вакууме и в этой среде?
}
\answer{%
    \begin{align*}
    \nu &= \frac 1T = \frac 1{\lambda/c} = \frac c\lambda = \frac{ 3 \cdot 10^{8}\,\frac{\text{м}}{\text{с}} }{ 600\,\text{нм} } \approx 5{,}00 \cdot 10^{14}\,\text{Гц}, \\
    \nu' = \nu &\cbr{\text{или } T' = T} \implies \lambda' = v'T' = \frac vn T = \frac{vt}n = \frac \lambda n = \frac{ 600\,\text{нм} }{ 1{,}6 } \approx 375 \cdot 10^{-9}\,\text{м}.
    \\
    &\text{380 нм---фиол---440---син---485---гол---500---зел---565---жёл---590---оранж---625---крас---780 нм}
    \end{align*}
}
\solutionspace{180pt}

\tasknumber{2}%
\task{%
    При переходе электрона в атоме с одной стационарной орбиты на другую
    излучается фотон с энергией $4{,}04 \cdot 10^{-19}\,\text{Дж}$.
    Какова длина волны этой линии спектра?
    Постоянная Планка $h = 6{,}626 \cdot 10^{-34}\,\text{Дж}\cdot\text{с}$, скорость света $c = 3 \cdot 10^{8}\,\frac{\text{м}}{\text{с}}$.
}
\answer{%
    $
        E = h\nu = h \frac c\lambda
        \implies \lambda = \frac{hc}{E}
            = \frac{6{,}626 \cdot 10^{-34}\,\text{Дж}\cdot\text{с} \cdot { 3 \cdot 10^{8}\,\frac{\text{м}}{\text{с}} }}{ 4{,}04 \cdot 10^{-19}\,\text{Дж} }
            = 4{,}92 \cdot 10^{-7}\,\text{м}.
    $
}
\solutionspace{150pt}

\tasknumber{3}%
\task{%
    Излучение какой длины волны поглотил атом водорода, если полная энергия в атоме увеличилась на $4 \cdot 10^{-19}\,\text{Дж}$?
    Постоянная Планка $h = 6{,}626 \cdot 10^{-34}\,\text{Дж}\cdot\text{с}$, скорость света $c = 3 \cdot 10^{8}\,\frac{\text{м}}{\text{с}}$.
}
\answer{%
    $
        E = h\nu = h \frac c\lambda
        \implies \lambda = \frac{hc}{E}
            = \frac{6{,}626 \cdot 10^{-34}\,\text{Дж}\cdot\text{с} \cdot { 3 \cdot 10^{8}\,\frac{\text{м}}{\text{с}} }}{ 4 \cdot 10^{-19}\,\text{Дж} }
            = 4{,}97 \cdot 10^{-7}\,\text{м}.
    $
}
\solutionspace{150pt}

\tasknumber{4}%
\task{%
    Сделайте схематичный рисунок энергетических уровней атома водорода
    и отметьте на нём первый (основной) уровень и последующие.
    Сколько различных длин волн может испустить атом водорода,
    находящийся в 4-м возбуждённом состоянии?
    Отметьте все соответствующие переходы на рисунке и укажите,
    при каком переходе (среди отмеченных) длина волны излучённого фотона минимальна.
}
\answer{%
    $N = 6{,}0, \text{самая длинная линия}$
}

\variantsplitter

\addpersonalvariant{Николай Кузьмин}

\tasknumber{1}%
\task{%
    Длина волны света в~вакууме $\lambda = 500\,\text{нм}$.
    Какова частота этой световой волны?
    Какова длина этой волны в среде с показателем преломления $n = 1{,}5$?
    Может ли человек увидеть такую волну света, и если да, то какой именно цвет соответствует этим волнам в вакууме и в этой среде?
}
\answer{%
    \begin{align*}
    \nu &= \frac 1T = \frac 1{\lambda/c} = \frac c\lambda = \frac{ 3 \cdot 10^{8}\,\frac{\text{м}}{\text{с}} }{ 500\,\text{нм} } \approx 6{,}00 \cdot 10^{14}\,\text{Гц}, \\
    \nu' = \nu &\cbr{\text{или } T' = T} \implies \lambda' = v'T' = \frac vn T = \frac{vt}n = \frac \lambda n = \frac{ 500\,\text{нм} }{ 1{,}5 } \approx 333 \cdot 10^{-9}\,\text{м}.
    \\
    &\text{380 нм---фиол---440---син---485---гол---500---зел---565---жёл---590---оранж---625---крас---780 нм}
    \end{align*}
}
\solutionspace{180pt}

\tasknumber{2}%
\task{%
    При переходе электрона в атоме с одной стационарной орбиты на другую
    излучается фотон с энергией $4{,}04 \cdot 10^{-19}\,\text{Дж}$.
    Какова длина волны этой линии спектра?
    Постоянная Планка $h = 6{,}626 \cdot 10^{-34}\,\text{Дж}\cdot\text{с}$, скорость света $c = 3 \cdot 10^{8}\,\frac{\text{м}}{\text{с}}$.
}
\answer{%
    $
        E = h\nu = h \frac c\lambda
        \implies \lambda = \frac{hc}{E}
            = \frac{6{,}626 \cdot 10^{-34}\,\text{Дж}\cdot\text{с} \cdot { 3 \cdot 10^{8}\,\frac{\text{м}}{\text{с}} }}{ 4{,}04 \cdot 10^{-19}\,\text{Дж} }
            = 4{,}92 \cdot 10^{-7}\,\text{м}.
    $
}
\solutionspace{150pt}

\tasknumber{3}%
\task{%
    Излучение какой длины волны поглотил атом водорода, если полная энергия в атоме увеличилась на $4 \cdot 10^{-19}\,\text{Дж}$?
    Постоянная Планка $h = 6{,}626 \cdot 10^{-34}\,\text{Дж}\cdot\text{с}$, скорость света $c = 3 \cdot 10^{8}\,\frac{\text{м}}{\text{с}}$.
}
\answer{%
    $
        E = h\nu = h \frac c\lambda
        \implies \lambda = \frac{hc}{E}
            = \frac{6{,}626 \cdot 10^{-34}\,\text{Дж}\cdot\text{с} \cdot { 3 \cdot 10^{8}\,\frac{\text{м}}{\text{с}} }}{ 4 \cdot 10^{-19}\,\text{Дж} }
            = 4{,}97 \cdot 10^{-7}\,\text{м}.
    $
}
\solutionspace{150pt}

\tasknumber{4}%
\task{%
    Сделайте схематичный рисунок энергетических уровней атома водорода
    и отметьте на нём первый (основной) уровень и последующие.
    Сколько различных длин волн может испустить атом водорода,
    находящийся в 3-м возбуждённом состоянии?
    Отметьте все соответствующие переходы на рисунке и укажите,
    при каком переходе (среди отмеченных) длина волны излучённого фотона максимальна.
}
\answer{%
    $N = 3{,}0, \text{самая короткая линия}$
}

\variantsplitter

\addpersonalvariant{Джейн Либерман}

\tasknumber{1}%
\task{%
    Длина волны света в~вакууме $\lambda = 500\,\text{нм}$.
    Какова частота этой световой волны?
    Какова длина этой волны в среде с показателем преломления $n = 1{,}7$?
    Может ли человек увидеть такую волну света, и если да, то какой именно цвет соответствует этим волнам в вакууме и в этой среде?
}
\answer{%
    \begin{align*}
    \nu &= \frac 1T = \frac 1{\lambda/c} = \frac c\lambda = \frac{ 3 \cdot 10^{8}\,\frac{\text{м}}{\text{с}} }{ 500\,\text{нм} } \approx 6{,}00 \cdot 10^{14}\,\text{Гц}, \\
    \nu' = \nu &\cbr{\text{или } T' = T} \implies \lambda' = v'T' = \frac vn T = \frac{vt}n = \frac \lambda n = \frac{ 500\,\text{нм} }{ 1{,}7 } \approx 294 \cdot 10^{-9}\,\text{м}.
    \\
    &\text{380 нм---фиол---440---син---485---гол---500---зел---565---жёл---590---оранж---625---крас---780 нм}
    \end{align*}
}
\solutionspace{180pt}

\tasknumber{2}%
\task{%
    При переходе электрона в атоме с одной стационарной орбиты на другую
    излучается фотон с энергией $7{,}07 \cdot 10^{-19}\,\text{Дж}$.
    Какова длина волны этой линии спектра?
    Постоянная Планка $h = 6{,}626 \cdot 10^{-34}\,\text{Дж}\cdot\text{с}$, скорость света $c = 3 \cdot 10^{8}\,\frac{\text{м}}{\text{с}}$.
}
\answer{%
    $
        E = h\nu = h \frac c\lambda
        \implies \lambda = \frac{hc}{E}
            = \frac{6{,}626 \cdot 10^{-34}\,\text{Дж}\cdot\text{с} \cdot { 3 \cdot 10^{8}\,\frac{\text{м}}{\text{с}} }}{ 7{,}07 \cdot 10^{-19}\,\text{Дж} }
            = 2{,}81 \cdot 10^{-7}\,\text{м}.
    $
}
\solutionspace{150pt}

\tasknumber{3}%
\task{%
    Излучение какой длины волны поглотил атом водорода, если полная энергия в атоме увеличилась на $3 \cdot 10^{-19}\,\text{Дж}$?
    Постоянная Планка $h = 6{,}626 \cdot 10^{-34}\,\text{Дж}\cdot\text{с}$, скорость света $c = 3 \cdot 10^{8}\,\frac{\text{м}}{\text{с}}$.
}
\answer{%
    $
        E = h\nu = h \frac c\lambda
        \implies \lambda = \frac{hc}{E}
            = \frac{6{,}626 \cdot 10^{-34}\,\text{Дж}\cdot\text{с} \cdot { 3 \cdot 10^{8}\,\frac{\text{м}}{\text{с}} }}{ 3 \cdot 10^{-19}\,\text{Дж} }
            = 6{,}63 \cdot 10^{-7}\,\text{м}.
    $
}
\solutionspace{150pt}

\tasknumber{4}%
\task{%
    Сделайте схематичный рисунок энергетических уровней атома водорода
    и отметьте на нём первый (основной) уровень и последующие.
    Сколько различных длин волн может испустить атом водорода,
    находящийся в 4-м возбуждённом состоянии?
    Отметьте все соответствующие переходы на рисунке и укажите,
    при каком переходе (среди отмеченных) частота излучённого фотона минимальна.
}
\answer{%
    $N = 6{,}0, \text{самая короткая линия}$
}

\variantsplitter

\addpersonalvariant{Мария Лукина}

\tasknumber{1}%
\task{%
    Длина волны света в~вакууме $\lambda = 600\,\text{нм}$.
    Какова частота этой световой волны?
    Какова длина этой волны в среде с показателем преломления $n = 1{,}5$?
    Может ли человек увидеть такую волну света, и если да, то какой именно цвет соответствует этим волнам в вакууме и в этой среде?
}
\answer{%
    \begin{align*}
    \nu &= \frac 1T = \frac 1{\lambda/c} = \frac c\lambda = \frac{ 3 \cdot 10^{8}\,\frac{\text{м}}{\text{с}} }{ 600\,\text{нм} } \approx 5{,}00 \cdot 10^{14}\,\text{Гц}, \\
    \nu' = \nu &\cbr{\text{или } T' = T} \implies \lambda' = v'T' = \frac vn T = \frac{vt}n = \frac \lambda n = \frac{ 600\,\text{нм} }{ 1{,}5 } \approx 400 \cdot 10^{-9}\,\text{м}.
    \\
    &\text{380 нм---фиол---440---син---485---гол---500---зел---565---жёл---590---оранж---625---крас---780 нм}
    \end{align*}
}
\solutionspace{180pt}

\tasknumber{2}%
\task{%
    При переходе электрона в атоме с одной стационарной орбиты на другую
    излучается фотон с энергией $2{,}02 \cdot 10^{-19}\,\text{Дж}$.
    Какова длина волны этой линии спектра?
    Постоянная Планка $h = 6{,}626 \cdot 10^{-34}\,\text{Дж}\cdot\text{с}$, скорость света $c = 3 \cdot 10^{8}\,\frac{\text{м}}{\text{с}}$.
}
\answer{%
    $
        E = h\nu = h \frac c\lambda
        \implies \lambda = \frac{hc}{E}
            = \frac{6{,}626 \cdot 10^{-34}\,\text{Дж}\cdot\text{с} \cdot { 3 \cdot 10^{8}\,\frac{\text{м}}{\text{с}} }}{ 2{,}02 \cdot 10^{-19}\,\text{Дж} }
            = 9{,}84 \cdot 10^{-7}\,\text{м}.
    $
}
\solutionspace{150pt}

\tasknumber{3}%
\task{%
    Излучение какой длины волны поглотил атом водорода, если полная энергия в атоме увеличилась на $2 \cdot 10^{-19}\,\text{Дж}$?
    Постоянная Планка $h = 6{,}626 \cdot 10^{-34}\,\text{Дж}\cdot\text{с}$, скорость света $c = 3 \cdot 10^{8}\,\frac{\text{м}}{\text{с}}$.
}
\answer{%
    $
        E = h\nu = h \frac c\lambda
        \implies \lambda = \frac{hc}{E}
            = \frac{6{,}626 \cdot 10^{-34}\,\text{Дж}\cdot\text{с} \cdot { 3 \cdot 10^{8}\,\frac{\text{м}}{\text{с}} }}{ 2 \cdot 10^{-19}\,\text{Дж} }
            = 9{,}94 \cdot 10^{-7}\,\text{м}.
    $
}
\solutionspace{150pt}

\tasknumber{4}%
\task{%
    Сделайте схематичный рисунок энергетических уровней атома водорода
    и отметьте на нём первый (основной) уровень и последующие.
    Сколько различных длин волн может испустить атом водорода,
    находящийся в 4-м возбуждённом состоянии?
    Отметьте все соответствующие переходы на рисунке и укажите,
    при каком переходе (среди отмеченных) длина волны излучённого фотона максимальна.
}
\answer{%
    $N = 6{,}0, \text{самая короткая линия}$
}

\variantsplitter

\addpersonalvariant{Елизавета Майорова}

\tasknumber{1}%
\task{%
    Длина волны света в~вакууме $\lambda = 600\,\text{нм}$.
    Какова частота этой световой волны?
    Какова длина этой волны в среде с показателем преломления $n = 1{,}5$?
    Может ли человек увидеть такую волну света, и если да, то какой именно цвет соответствует этим волнам в вакууме и в этой среде?
}
\answer{%
    \begin{align*}
    \nu &= \frac 1T = \frac 1{\lambda/c} = \frac c\lambda = \frac{ 3 \cdot 10^{8}\,\frac{\text{м}}{\text{с}} }{ 600\,\text{нм} } \approx 5{,}00 \cdot 10^{14}\,\text{Гц}, \\
    \nu' = \nu &\cbr{\text{или } T' = T} \implies \lambda' = v'T' = \frac vn T = \frac{vt}n = \frac \lambda n = \frac{ 600\,\text{нм} }{ 1{,}5 } \approx 400 \cdot 10^{-9}\,\text{м}.
    \\
    &\text{380 нм---фиол---440---син---485---гол---500---зел---565---жёл---590---оранж---625---крас---780 нм}
    \end{align*}
}
\solutionspace{180pt}

\tasknumber{2}%
\task{%
    При переходе электрона в атоме с одной стационарной орбиты на другую
    излучается фотон с энергией $2{,}02 \cdot 10^{-19}\,\text{Дж}$.
    Какова длина волны этой линии спектра?
    Постоянная Планка $h = 6{,}626 \cdot 10^{-34}\,\text{Дж}\cdot\text{с}$, скорость света $c = 3 \cdot 10^{8}\,\frac{\text{м}}{\text{с}}$.
}
\answer{%
    $
        E = h\nu = h \frac c\lambda
        \implies \lambda = \frac{hc}{E}
            = \frac{6{,}626 \cdot 10^{-34}\,\text{Дж}\cdot\text{с} \cdot { 3 \cdot 10^{8}\,\frac{\text{м}}{\text{с}} }}{ 2{,}02 \cdot 10^{-19}\,\text{Дж} }
            = 9{,}84 \cdot 10^{-7}\,\text{м}.
    $
}
\solutionspace{150pt}

\tasknumber{3}%
\task{%
    Излучение какой длины волны поглотил атом водорода, если полная энергия в атоме увеличилась на $4 \cdot 10^{-19}\,\text{Дж}$?
    Постоянная Планка $h = 6{,}626 \cdot 10^{-34}\,\text{Дж}\cdot\text{с}$, скорость света $c = 3 \cdot 10^{8}\,\frac{\text{м}}{\text{с}}$.
}
\answer{%
    $
        E = h\nu = h \frac c\lambda
        \implies \lambda = \frac{hc}{E}
            = \frac{6{,}626 \cdot 10^{-34}\,\text{Дж}\cdot\text{с} \cdot { 3 \cdot 10^{8}\,\frac{\text{м}}{\text{с}} }}{ 4 \cdot 10^{-19}\,\text{Дж} }
            = 4{,}97 \cdot 10^{-7}\,\text{м}.
    $
}
\solutionspace{150pt}

\tasknumber{4}%
\task{%
    Сделайте схематичный рисунок энергетических уровней атома водорода
    и отметьте на нём первый (основной) уровень и последующие.
    Сколько различных длин волн может испустить атом водорода,
    находящийся в 3-м возбуждённом состоянии?
    Отметьте все соответствующие переходы на рисунке и укажите,
    при каком переходе (среди отмеченных) энергия излучённого фотона максимальна.
}
\answer{%
    $N = 3{,}0, \text{самая длинная линия}$
}

\variantsplitter

\addpersonalvariant{Марьям Марова}

\tasknumber{1}%
\task{%
    Длина волны света в~вакууме $\lambda = 700\,\text{нм}$.
    Какова частота этой световой волны?
    Какова длина этой волны в среде с показателем преломления $n = 1{,}7$?
    Может ли человек увидеть такую волну света, и если да, то какой именно цвет соответствует этим волнам в вакууме и в этой среде?
}
\answer{%
    \begin{align*}
    \nu &= \frac 1T = \frac 1{\lambda/c} = \frac c\lambda = \frac{ 3 \cdot 10^{8}\,\frac{\text{м}}{\text{с}} }{ 700\,\text{нм} } \approx 4{,}29 \cdot 10^{14}\,\text{Гц}, \\
    \nu' = \nu &\cbr{\text{или } T' = T} \implies \lambda' = v'T' = \frac vn T = \frac{vt}n = \frac \lambda n = \frac{ 700\,\text{нм} }{ 1{,}7 } \approx 412 \cdot 10^{-9}\,\text{м}.
    \\
    &\text{380 нм---фиол---440---син---485---гол---500---зел---565---жёл---590---оранж---625---крас---780 нм}
    \end{align*}
}
\solutionspace{180pt}

\tasknumber{2}%
\task{%
    При переходе электрона в атоме с одной стационарной орбиты на другую
    излучается фотон с энергией $2{,}02 \cdot 10^{-19}\,\text{Дж}$.
    Какова длина волны этой линии спектра?
    Постоянная Планка $h = 6{,}626 \cdot 10^{-34}\,\text{Дж}\cdot\text{с}$, скорость света $c = 3 \cdot 10^{8}\,\frac{\text{м}}{\text{с}}$.
}
\answer{%
    $
        E = h\nu = h \frac c\lambda
        \implies \lambda = \frac{hc}{E}
            = \frac{6{,}626 \cdot 10^{-34}\,\text{Дж}\cdot\text{с} \cdot { 3 \cdot 10^{8}\,\frac{\text{м}}{\text{с}} }}{ 2{,}02 \cdot 10^{-19}\,\text{Дж} }
            = 9{,}84 \cdot 10^{-7}\,\text{м}.
    $
}
\solutionspace{150pt}

\tasknumber{3}%
\task{%
    Излучение какой длины волны поглотил атом водорода, если полная энергия в атоме увеличилась на $2 \cdot 10^{-19}\,\text{Дж}$?
    Постоянная Планка $h = 6{,}626 \cdot 10^{-34}\,\text{Дж}\cdot\text{с}$, скорость света $c = 3 \cdot 10^{8}\,\frac{\text{м}}{\text{с}}$.
}
\answer{%
    $
        E = h\nu = h \frac c\lambda
        \implies \lambda = \frac{hc}{E}
            = \frac{6{,}626 \cdot 10^{-34}\,\text{Дж}\cdot\text{с} \cdot { 3 \cdot 10^{8}\,\frac{\text{м}}{\text{с}} }}{ 2 \cdot 10^{-19}\,\text{Дж} }
            = 9{,}94 \cdot 10^{-7}\,\text{м}.
    $
}
\solutionspace{150pt}

\tasknumber{4}%
\task{%
    Сделайте схематичный рисунок энергетических уровней атома водорода
    и отметьте на нём первый (основной) уровень и последующие.
    Сколько различных длин волн может испустить атом водорода,
    находящийся в 5-м возбуждённом состоянии?
    Отметьте все соответствующие переходы на рисунке и укажите,
    при каком переходе (среди отмеченных) энергия излучённого фотона максимальна.
}
\answer{%
    $N = 10{,}0, \text{самая длинная линия}$
}

\variantsplitter

\addpersonalvariant{Гульнара Сафина}

\tasknumber{1}%
\task{%
    Длина волны света в~вакууме $\lambda = 400\,\text{нм}$.
    Какова частота этой световой волны?
    Какова длина этой волны в среде с показателем преломления $n = 1{,}5$?
    Может ли человек увидеть такую волну света, и если да, то какой именно цвет соответствует этим волнам в вакууме и в этой среде?
}
\answer{%
    \begin{align*}
    \nu &= \frac 1T = \frac 1{\lambda/c} = \frac c\lambda = \frac{ 3 \cdot 10^{8}\,\frac{\text{м}}{\text{с}} }{ 400\,\text{нм} } \approx 7{,}50 \cdot 10^{14}\,\text{Гц}, \\
    \nu' = \nu &\cbr{\text{или } T' = T} \implies \lambda' = v'T' = \frac vn T = \frac{vt}n = \frac \lambda n = \frac{ 400\,\text{нм} }{ 1{,}5 } \approx 267 \cdot 10^{-9}\,\text{м}.
    \\
    &\text{380 нм---фиол---440---син---485---гол---500---зел---565---жёл---590---оранж---625---крас---780 нм}
    \end{align*}
}
\solutionspace{180pt}

\tasknumber{2}%
\task{%
    При переходе электрона в атоме с одной стационарной орбиты на другую
    излучается фотон с энергией $1{,}01 \cdot 10^{-19}\,\text{Дж}$.
    Какова длина волны этой линии спектра?
    Постоянная Планка $h = 6{,}626 \cdot 10^{-34}\,\text{Дж}\cdot\text{с}$, скорость света $c = 3 \cdot 10^{8}\,\frac{\text{м}}{\text{с}}$.
}
\answer{%
    $
        E = h\nu = h \frac c\lambda
        \implies \lambda = \frac{hc}{E}
            = \frac{6{,}626 \cdot 10^{-34}\,\text{Дж}\cdot\text{с} \cdot { 3 \cdot 10^{8}\,\frac{\text{м}}{\text{с}} }}{ 1{,}01 \cdot 10^{-19}\,\text{Дж} }
            = 19{,}68 \cdot 10^{-7}\,\text{м}.
    $
}
\solutionspace{150pt}

\tasknumber{3}%
\task{%
    Излучение какой длины волны поглотил атом водорода, если полная энергия в атоме увеличилась на $2 \cdot 10^{-19}\,\text{Дж}$?
    Постоянная Планка $h = 6{,}626 \cdot 10^{-34}\,\text{Дж}\cdot\text{с}$, скорость света $c = 3 \cdot 10^{8}\,\frac{\text{м}}{\text{с}}$.
}
\answer{%
    $
        E = h\nu = h \frac c\lambda
        \implies \lambda = \frac{hc}{E}
            = \frac{6{,}626 \cdot 10^{-34}\,\text{Дж}\cdot\text{с} \cdot { 3 \cdot 10^{8}\,\frac{\text{м}}{\text{с}} }}{ 2 \cdot 10^{-19}\,\text{Дж} }
            = 9{,}94 \cdot 10^{-7}\,\text{м}.
    $
}
\solutionspace{150pt}

\tasknumber{4}%
\task{%
    Сделайте схематичный рисунок энергетических уровней атома водорода
    и отметьте на нём первый (основной) уровень и последующие.
    Сколько различных длин волн может испустить атом водорода,
    находящийся в 3-м возбуждённом состоянии?
    Отметьте все соответствующие переходы на рисунке и укажите,
    при каком переходе (среди отмеченных) энергия излучённого фотона минимальна.
}
\answer{%
    $N = 3{,}0, \text{самая короткая линия}$
}

\variantsplitter

\addpersonalvariant{Анастасия Свиридова}

\tasknumber{1}%
\task{%
    Длина волны света в~вакууме $\lambda = 700\,\text{нм}$.
    Какова частота этой световой волны?
    Какова длина этой волны в среде с показателем преломления $n = 1{,}3$?
    Может ли человек увидеть такую волну света, и если да, то какой именно цвет соответствует этим волнам в вакууме и в этой среде?
}
\answer{%
    \begin{align*}
    \nu &= \frac 1T = \frac 1{\lambda/c} = \frac c\lambda = \frac{ 3 \cdot 10^{8}\,\frac{\text{м}}{\text{с}} }{ 700\,\text{нм} } \approx 4{,}29 \cdot 10^{14}\,\text{Гц}, \\
    \nu' = \nu &\cbr{\text{или } T' = T} \implies \lambda' = v'T' = \frac vn T = \frac{vt}n = \frac \lambda n = \frac{ 700\,\text{нм} }{ 1{,}3 } \approx 538 \cdot 10^{-9}\,\text{м}.
    \\
    &\text{380 нм---фиол---440---син---485---гол---500---зел---565---жёл---590---оранж---625---крас---780 нм}
    \end{align*}
}
\solutionspace{180pt}

\tasknumber{2}%
\task{%
    При переходе электрона в атоме с одной стационарной орбиты на другую
    излучается фотон с энергией $7{,}07 \cdot 10^{-19}\,\text{Дж}$.
    Какова длина волны этой линии спектра?
    Постоянная Планка $h = 6{,}626 \cdot 10^{-34}\,\text{Дж}\cdot\text{с}$, скорость света $c = 3 \cdot 10^{8}\,\frac{\text{м}}{\text{с}}$.
}
\answer{%
    $
        E = h\nu = h \frac c\lambda
        \implies \lambda = \frac{hc}{E}
            = \frac{6{,}626 \cdot 10^{-34}\,\text{Дж}\cdot\text{с} \cdot { 3 \cdot 10^{8}\,\frac{\text{м}}{\text{с}} }}{ 7{,}07 \cdot 10^{-19}\,\text{Дж} }
            = 2{,}81 \cdot 10^{-7}\,\text{м}.
    $
}
\solutionspace{150pt}

\tasknumber{3}%
\task{%
    Излучение какой длины волны поглотил атом водорода, если полная энергия в атоме увеличилась на $3 \cdot 10^{-19}\,\text{Дж}$?
    Постоянная Планка $h = 6{,}626 \cdot 10^{-34}\,\text{Дж}\cdot\text{с}$, скорость света $c = 3 \cdot 10^{8}\,\frac{\text{м}}{\text{с}}$.
}
\answer{%
    $
        E = h\nu = h \frac c\lambda
        \implies \lambda = \frac{hc}{E}
            = \frac{6{,}626 \cdot 10^{-34}\,\text{Дж}\cdot\text{с} \cdot { 3 \cdot 10^{8}\,\frac{\text{м}}{\text{с}} }}{ 3 \cdot 10^{-19}\,\text{Дж} }
            = 6{,}63 \cdot 10^{-7}\,\text{м}.
    $
}
\solutionspace{150pt}

\tasknumber{4}%
\task{%
    Сделайте схематичный рисунок энергетических уровней атома водорода
    и отметьте на нём первый (основной) уровень и последующие.
    Сколько различных длин волн может испустить атом водорода,
    находящийся в 5-м возбуждённом состоянии?
    Отметьте все соответствующие переходы на рисунке и укажите,
    при каком переходе (среди отмеченных) энергия излучённого фотона максимальна.
}
\answer{%
    $N = 10{,}0, \text{самая длинная линия}$
}

\variantsplitter

\addpersonalvariant{Евгения Сивачева}

\tasknumber{1}%
\task{%
    Длина волны света в~вакууме $\lambda = 400\,\text{нм}$.
    Какова частота этой световой волны?
    Какова длина этой волны в среде с показателем преломления $n = 1{,}3$?
    Может ли человек увидеть такую волну света, и если да, то какой именно цвет соответствует этим волнам в вакууме и в этой среде?
}
\answer{%
    \begin{align*}
    \nu &= \frac 1T = \frac 1{\lambda/c} = \frac c\lambda = \frac{ 3 \cdot 10^{8}\,\frac{\text{м}}{\text{с}} }{ 400\,\text{нм} } \approx 7{,}50 \cdot 10^{14}\,\text{Гц}, \\
    \nu' = \nu &\cbr{\text{или } T' = T} \implies \lambda' = v'T' = \frac vn T = \frac{vt}n = \frac \lambda n = \frac{ 400\,\text{нм} }{ 1{,}3 } \approx 308 \cdot 10^{-9}\,\text{м}.
    \\
    &\text{380 нм---фиол---440---син---485---гол---500---зел---565---жёл---590---оранж---625---крас---780 нм}
    \end{align*}
}
\solutionspace{180pt}

\tasknumber{2}%
\task{%
    При переходе электрона в атоме с одной стационарной орбиты на другую
    излучается фотон с энергией $7{,}07 \cdot 10^{-19}\,\text{Дж}$.
    Какова длина волны этой линии спектра?
    Постоянная Планка $h = 6{,}626 \cdot 10^{-34}\,\text{Дж}\cdot\text{с}$, скорость света $c = 3 \cdot 10^{8}\,\frac{\text{м}}{\text{с}}$.
}
\answer{%
    $
        E = h\nu = h \frac c\lambda
        \implies \lambda = \frac{hc}{E}
            = \frac{6{,}626 \cdot 10^{-34}\,\text{Дж}\cdot\text{с} \cdot { 3 \cdot 10^{8}\,\frac{\text{м}}{\text{с}} }}{ 7{,}07 \cdot 10^{-19}\,\text{Дж} }
            = 2{,}81 \cdot 10^{-7}\,\text{м}.
    $
}
\solutionspace{150pt}

\tasknumber{3}%
\task{%
    Излучение какой длины волны поглотил атом водорода, если полная энергия в атоме увеличилась на $6 \cdot 10^{-19}\,\text{Дж}$?
    Постоянная Планка $h = 6{,}626 \cdot 10^{-34}\,\text{Дж}\cdot\text{с}$, скорость света $c = 3 \cdot 10^{8}\,\frac{\text{м}}{\text{с}}$.
}
\answer{%
    $
        E = h\nu = h \frac c\lambda
        \implies \lambda = \frac{hc}{E}
            = \frac{6{,}626 \cdot 10^{-34}\,\text{Дж}\cdot\text{с} \cdot { 3 \cdot 10^{8}\,\frac{\text{м}}{\text{с}} }}{ 6 \cdot 10^{-19}\,\text{Дж} }
            = 3{,}31 \cdot 10^{-7}\,\text{м}.
    $
}
\solutionspace{150pt}

\tasknumber{4}%
\task{%
    Сделайте схематичный рисунок энергетических уровней атома водорода
    и отметьте на нём первый (основной) уровень и последующие.
    Сколько различных длин волн может испустить атом водорода,
    находящийся в 4-м возбуждённом состоянии?
    Отметьте все соответствующие переходы на рисунке и укажите,
    при каком переходе (среди отмеченных) длина волны излучённого фотона максимальна.
}
\answer{%
    $N = 6{,}0, \text{самая короткая линия}$
}

\variantsplitter

\addpersonalvariant{Илья Скаков}

\tasknumber{1}%
\task{%
    Длина волны света в~вакууме $\lambda = 400\,\text{нм}$.
    Какова частота этой световой волны?
    Какова длина этой волны в среде с показателем преломления $n = 1{,}7$?
    Может ли человек увидеть такую волну света, и если да, то какой именно цвет соответствует этим волнам в вакууме и в этой среде?
}
\answer{%
    \begin{align*}
    \nu &= \frac 1T = \frac 1{\lambda/c} = \frac c\lambda = \frac{ 3 \cdot 10^{8}\,\frac{\text{м}}{\text{с}} }{ 400\,\text{нм} } \approx 7{,}50 \cdot 10^{14}\,\text{Гц}, \\
    \nu' = \nu &\cbr{\text{или } T' = T} \implies \lambda' = v'T' = \frac vn T = \frac{vt}n = \frac \lambda n = \frac{ 400\,\text{нм} }{ 1{,}7 } \approx 235 \cdot 10^{-9}\,\text{м}.
    \\
    &\text{380 нм---фиол---440---син---485---гол---500---зел---565---жёл---590---оранж---625---крас---780 нм}
    \end{align*}
}
\solutionspace{180pt}

\tasknumber{2}%
\task{%
    При переходе электрона в атоме с одной стационарной орбиты на другую
    излучается фотон с энергией $5{,}05 \cdot 10^{-19}\,\text{Дж}$.
    Какова длина волны этой линии спектра?
    Постоянная Планка $h = 6{,}626 \cdot 10^{-34}\,\text{Дж}\cdot\text{с}$, скорость света $c = 3 \cdot 10^{8}\,\frac{\text{м}}{\text{с}}$.
}
\answer{%
    $
        E = h\nu = h \frac c\lambda
        \implies \lambda = \frac{hc}{E}
            = \frac{6{,}626 \cdot 10^{-34}\,\text{Дж}\cdot\text{с} \cdot { 3 \cdot 10^{8}\,\frac{\text{м}}{\text{с}} }}{ 5{,}05 \cdot 10^{-19}\,\text{Дж} }
            = 3{,}94 \cdot 10^{-7}\,\text{м}.
    $
}
\solutionspace{150pt}

\tasknumber{3}%
\task{%
    Излучение какой длины волны поглотил атом водорода, если полная энергия в атоме увеличилась на $3 \cdot 10^{-19}\,\text{Дж}$?
    Постоянная Планка $h = 6{,}626 \cdot 10^{-34}\,\text{Дж}\cdot\text{с}$, скорость света $c = 3 \cdot 10^{8}\,\frac{\text{м}}{\text{с}}$.
}
\answer{%
    $
        E = h\nu = h \frac c\lambda
        \implies \lambda = \frac{hc}{E}
            = \frac{6{,}626 \cdot 10^{-34}\,\text{Дж}\cdot\text{с} \cdot { 3 \cdot 10^{8}\,\frac{\text{м}}{\text{с}} }}{ 3 \cdot 10^{-19}\,\text{Дж} }
            = 6{,}63 \cdot 10^{-7}\,\text{м}.
    $
}
\solutionspace{150pt}

\tasknumber{4}%
\task{%
    Сделайте схематичный рисунок энергетических уровней атома водорода
    и отметьте на нём первый (основной) уровень и последующие.
    Сколько различных длин волн может испустить атом водорода,
    находящийся в 5-м возбуждённом состоянии?
    Отметьте все соответствующие переходы на рисунке и укажите,
    при каком переходе (среди отмеченных) длина волны излучённого фотона максимальна.
}
\answer{%
    $N = 10{,}0, \text{самая короткая линия}$
}

\variantsplitter

\addpersonalvariant{Валерия Тарасова}

\tasknumber{1}%
\task{%
    Длина волны света в~вакууме $\lambda = 500\,\text{нм}$.
    Какова частота этой световой волны?
    Какова длина этой волны в среде с показателем преломления $n = 1{,}7$?
    Может ли человек увидеть такую волну света, и если да, то какой именно цвет соответствует этим волнам в вакууме и в этой среде?
}
\answer{%
    \begin{align*}
    \nu &= \frac 1T = \frac 1{\lambda/c} = \frac c\lambda = \frac{ 3 \cdot 10^{8}\,\frac{\text{м}}{\text{с}} }{ 500\,\text{нм} } \approx 6{,}00 \cdot 10^{14}\,\text{Гц}, \\
    \nu' = \nu &\cbr{\text{или } T' = T} \implies \lambda' = v'T' = \frac vn T = \frac{vt}n = \frac \lambda n = \frac{ 500\,\text{нм} }{ 1{,}7 } \approx 294 \cdot 10^{-9}\,\text{м}.
    \\
    &\text{380 нм---фиол---440---син---485---гол---500---зел---565---жёл---590---оранж---625---крас---780 нм}
    \end{align*}
}
\solutionspace{180pt}

\tasknumber{2}%
\task{%
    При переходе электрона в атоме с одной стационарной орбиты на другую
    излучается фотон с энергией $0{,}55 \cdot 10^{-19}\,\text{Дж}$.
    Какова длина волны этой линии спектра?
    Постоянная Планка $h = 6{,}626 \cdot 10^{-34}\,\text{Дж}\cdot\text{с}$, скорость света $c = 3 \cdot 10^{8}\,\frac{\text{м}}{\text{с}}$.
}
\answer{%
    $
        E = h\nu = h \frac c\lambda
        \implies \lambda = \frac{hc}{E}
            = \frac{6{,}626 \cdot 10^{-34}\,\text{Дж}\cdot\text{с} \cdot { 3 \cdot 10^{8}\,\frac{\text{м}}{\text{с}} }}{ 0{,}55 \cdot 10^{-19}\,\text{Дж} }
            = 36{,}14 \cdot 10^{-7}\,\text{м}.
    $
}
\solutionspace{150pt}

\tasknumber{3}%
\task{%
    Излучение какой длины волны поглотил атом водорода, если полная энергия в атоме увеличилась на $3 \cdot 10^{-19}\,\text{Дж}$?
    Постоянная Планка $h = 6{,}626 \cdot 10^{-34}\,\text{Дж}\cdot\text{с}$, скорость света $c = 3 \cdot 10^{8}\,\frac{\text{м}}{\text{с}}$.
}
\answer{%
    $
        E = h\nu = h \frac c\lambda
        \implies \lambda = \frac{hc}{E}
            = \frac{6{,}626 \cdot 10^{-34}\,\text{Дж}\cdot\text{с} \cdot { 3 \cdot 10^{8}\,\frac{\text{м}}{\text{с}} }}{ 3 \cdot 10^{-19}\,\text{Дж} }
            = 6{,}63 \cdot 10^{-7}\,\text{м}.
    $
}
\solutionspace{150pt}

\tasknumber{4}%
\task{%
    Сделайте схематичный рисунок энергетических уровней атома водорода
    и отметьте на нём первый (основной) уровень и последующие.
    Сколько различных длин волн может испустить атом водорода,
    находящийся в 4-м возбуждённом состоянии?
    Отметьте все соответствующие переходы на рисунке и укажите,
    при каком переходе (среди отмеченных) энергия излучённого фотона минимальна.
}
\answer{%
    $N = 6{,}0, \text{самая короткая линия}$
}

\variantsplitter

\addpersonalvariant{Алёна Шальнева}

\tasknumber{1}%
\task{%
    Длина волны света в~вакууме $\lambda = 500\,\text{нм}$.
    Какова частота этой световой волны?
    Какова длина этой волны в среде с показателем преломления $n = 1{,}6$?
    Может ли человек увидеть такую волну света, и если да, то какой именно цвет соответствует этим волнам в вакууме и в этой среде?
}
\answer{%
    \begin{align*}
    \nu &= \frac 1T = \frac 1{\lambda/c} = \frac c\lambda = \frac{ 3 \cdot 10^{8}\,\frac{\text{м}}{\text{с}} }{ 500\,\text{нм} } \approx 6{,}00 \cdot 10^{14}\,\text{Гц}, \\
    \nu' = \nu &\cbr{\text{или } T' = T} \implies \lambda' = v'T' = \frac vn T = \frac{vt}n = \frac \lambda n = \frac{ 500\,\text{нм} }{ 1{,}6 } \approx 313 \cdot 10^{-9}\,\text{м}.
    \\
    &\text{380 нм---фиол---440---син---485---гол---500---зел---565---жёл---590---оранж---625---крас---780 нм}
    \end{align*}
}
\solutionspace{180pt}

\tasknumber{2}%
\task{%
    При переходе электрона в атоме с одной стационарной орбиты на другую
    излучается фотон с энергией $7{,}07 \cdot 10^{-19}\,\text{Дж}$.
    Какова длина волны этой линии спектра?
    Постоянная Планка $h = 6{,}626 \cdot 10^{-34}\,\text{Дж}\cdot\text{с}$, скорость света $c = 3 \cdot 10^{8}\,\frac{\text{м}}{\text{с}}$.
}
\answer{%
    $
        E = h\nu = h \frac c\lambda
        \implies \lambda = \frac{hc}{E}
            = \frac{6{,}626 \cdot 10^{-34}\,\text{Дж}\cdot\text{с} \cdot { 3 \cdot 10^{8}\,\frac{\text{м}}{\text{с}} }}{ 7{,}07 \cdot 10^{-19}\,\text{Дж} }
            = 2{,}81 \cdot 10^{-7}\,\text{м}.
    $
}
\solutionspace{150pt}

\tasknumber{3}%
\task{%
    Излучение какой длины волны поглотил атом водорода, если полная энергия в атоме увеличилась на $3 \cdot 10^{-19}\,\text{Дж}$?
    Постоянная Планка $h = 6{,}626 \cdot 10^{-34}\,\text{Дж}\cdot\text{с}$, скорость света $c = 3 \cdot 10^{8}\,\frac{\text{м}}{\text{с}}$.
}
\answer{%
    $
        E = h\nu = h \frac c\lambda
        \implies \lambda = \frac{hc}{E}
            = \frac{6{,}626 \cdot 10^{-34}\,\text{Дж}\cdot\text{с} \cdot { 3 \cdot 10^{8}\,\frac{\text{м}}{\text{с}} }}{ 3 \cdot 10^{-19}\,\text{Дж} }
            = 6{,}63 \cdot 10^{-7}\,\text{м}.
    $
}
\solutionspace{150pt}

\tasknumber{4}%
\task{%
    Сделайте схематичный рисунок энергетических уровней атома водорода
    и отметьте на нём первый (основной) уровень и последующие.
    Сколько различных длин волн может испустить атом водорода,
    находящийся в 3-м возбуждённом состоянии?
    Отметьте все соответствующие переходы на рисунке и укажите,
    при каком переходе (среди отмеченных) энергия излучённого фотона минимальна.
}
\answer{%
    $N = 3{,}0, \text{самая короткая линия}$
}
% autogenerated
