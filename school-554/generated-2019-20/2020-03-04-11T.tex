\setdate{4~марта~2020}
\setclass{11«Т»}

\addpersonalvariant{Михаил Бурмистров}

\tasknumber{1}%
\task{%
    Длина волны света в~вакууме $\lambda = 500\,\text{нм}$.
    Какова частота этой световой волны?
    Какова длина этой волны в среде с показателем преломления $n = 1{,}4$?
    Может ли человек увидеть такую волну света, и если да, то какой именно цвет соответствует этим волнам в вакууме и в этой среде?
}
\answer{%
    \begin{align*}
    \nu &= \frac 1T = \frac 1{\lambda/c} = \frac c\lambda = \frac{3 \cdot 10^{8}\,\frac{\text{м}}{\text{с}}}{500\,\text{нм}} \approx 6{,}00 \cdot 10^{14}\,\text{Гц}, \\
    \nu' = \nu &\cbr{\text{или } T' = T} \implies \lambda' = v'T' = \frac vn T = \frac{ vt }n = \frac \lambda n = \frac{500\,\text{нм}}{1{,}4} \approx 357 \cdot 10^{-9}\,\text{м}.
    \\
    &\text{380 нм---фиол---440---син---485---гол---500---зел---565---жёл---590---оранж---625---крас---780 нм}
    \end{align*}
}
\solutionspace{180pt}

\tasknumber{2}%
\task{%
    Установка для наблюдения интерференции состоит
    из двух когерентных источников света и экрана.
    Расстояние между источниками $l = 500\,\text{мкм}$,
    а от каждого источника до экрана — $L = 6\,\text{м}$.
    Сделайте рисунок и укажите положение нулевого максимума освещенности,
    а также определите расстояние между первым максимумом и нулевым максимумом.
    Длина волны падающего света составляет $\lambda = 500\,\text{нм}$.
}
\solutionspace{180pt}

\tasknumber{3}%
\task{%
    На стеклянную пластинку ($\hat n = 1{,}5$) нанесена прозрачная пленка ($n = 1{,}7$).
    На плёнку нормально к поверхности падает монохроматический свет с длиной волны $\lambda = 500\,\text{нм}$.
    Какова должна быть минимальная толщина пленки, если в результате интерференции свет имеет наибольшую интенсивность?
}

\variantsplitter

\addpersonalvariant{Гагик Аракелян}

\tasknumber{1}%
\task{%
    Длина волны света в~вакууме $\lambda = 400\,\text{нм}$.
    Какова частота этой световой волны?
    Какова длина этой волны в среде с показателем преломления $n = 1{,}3$?
    Может ли человек увидеть такую волну света, и если да, то какой именно цвет соответствует этим волнам в вакууме и в этой среде?
}
\answer{%
    \begin{align*}
    \nu &= \frac 1T = \frac 1{\lambda/c} = \frac c\lambda = \frac{3 \cdot 10^{8}\,\frac{\text{м}}{\text{с}}}{400\,\text{нм}} \approx 7{,}50 \cdot 10^{14}\,\text{Гц}, \\
    \nu' = \nu &\cbr{\text{или } T' = T} \implies \lambda' = v'T' = \frac vn T = \frac{ vt }n = \frac \lambda n = \frac{400\,\text{нм}}{1{,}3} \approx 308 \cdot 10^{-9}\,\text{м}.
    \\
    &\text{380 нм---фиол---440---син---485---гол---500---зел---565---жёл---590---оранж---625---крас---780 нм}
    \end{align*}
}
\solutionspace{180pt}

\tasknumber{2}%
\task{%
    Установка для наблюдения интерференции состоит
    из двух когерентных источников света и экрана.
    Расстояние между источниками $l = 500\,\text{мкм}$,
    а от каждого источника до экрана — $L = 2\,\text{м}$.
    Сделайте рисунок и укажите положение нулевого максимума освещенности,
    а также определите расстояние между первым максимумом и нулевым максимумом.
    Длина волны падающего света составляет $\lambda = 600\,\text{нм}$.
}
\solutionspace{180pt}

\tasknumber{3}%
\task{%
    На стеклянную пластинку ($\hat n = 1{,}5$) нанесена прозрачная пленка ($n = 1{,}7$).
    На плёнку нормально к поверхности падает монохроматический свет с длиной волны $\lambda = 600\,\text{нм}$.
    Какова должна быть минимальная толщина пленки, если в результате интерференции свет имеет наибольшую интенсивность?
}

\variantsplitter

\addpersonalvariant{Ирен Аракелян}

\tasknumber{1}%
\task{%
    Длина волны света в~вакууме $\lambda = 400\,\text{нм}$.
    Какова частота этой световой волны?
    Какова длина этой волны в среде с показателем преломления $n = 1{,}6$?
    Может ли человек увидеть такую волну света, и если да, то какой именно цвет соответствует этим волнам в вакууме и в этой среде?
}
\answer{%
    \begin{align*}
    \nu &= \frac 1T = \frac 1{\lambda/c} = \frac c\lambda = \frac{3 \cdot 10^{8}\,\frac{\text{м}}{\text{с}}}{400\,\text{нм}} \approx 7{,}50 \cdot 10^{14}\,\text{Гц}, \\
    \nu' = \nu &\cbr{\text{или } T' = T} \implies \lambda' = v'T' = \frac vn T = \frac{ vt }n = \frac \lambda n = \frac{400\,\text{нм}}{1{,}6} \approx 250 \cdot 10^{-9}\,\text{м}.
    \\
    &\text{380 нм---фиол---440---син---485---гол---500---зел---565---жёл---590---оранж---625---крас---780 нм}
    \end{align*}
}
\solutionspace{180pt}

\tasknumber{2}%
\task{%
    Установка для наблюдения интерференции состоит
    из двух когерентных источников света и экрана.
    Расстояние между источниками $l = 500\,\text{мкм}$,
    а от каждого источника до экрана — $L = 6\,\text{м}$.
    Сделайте рисунок и укажите положение нулевого максимума освещенности,
    а также определите расстояние между первым максимумом и нулевым максимумом.
    Длина волны падающего света составляет $\lambda = 400\,\text{нм}$.
}
\solutionspace{180pt}

\tasknumber{3}%
\task{%
    На стеклянную пластинку ($\hat n = 1{,}6$) нанесена прозрачная пленка ($n = 1{,}8$).
    На плёнку нормально к поверхности падает монохроматический свет с длиной волны $\lambda = 400\,\text{нм}$.
    Какова должна быть минимальная толщина пленки, если в результате интерференции свет имеет наименьшую интенсивность?
}

\variantsplitter

\addpersonalvariant{Сабина Асадуллаева}

\tasknumber{1}%
\task{%
    Длина волны света в~вакууме $\lambda = 600\,\text{нм}$.
    Какова частота этой световой волны?
    Какова длина этой волны в среде с показателем преломления $n = 1{,}7$?
    Может ли человек увидеть такую волну света, и если да, то какой именно цвет соответствует этим волнам в вакууме и в этой среде?
}
\answer{%
    \begin{align*}
    \nu &= \frac 1T = \frac 1{\lambda/c} = \frac c\lambda = \frac{3 \cdot 10^{8}\,\frac{\text{м}}{\text{с}}}{600\,\text{нм}} \approx 5{,}00 \cdot 10^{14}\,\text{Гц}, \\
    \nu' = \nu &\cbr{\text{или } T' = T} \implies \lambda' = v'T' = \frac vn T = \frac{ vt }n = \frac \lambda n = \frac{600\,\text{нм}}{1{,}7} \approx 353 \cdot 10^{-9}\,\text{м}.
    \\
    &\text{380 нм---фиол---440---син---485---гол---500---зел---565---жёл---590---оранж---625---крас---780 нм}
    \end{align*}
}
\solutionspace{180pt}

\tasknumber{2}%
\task{%
    Установка для наблюдения интерференции состоит
    из двух когерентных источников света и экрана.
    Расстояние между источниками $l = 600\,\text{мкм}$,
    а от каждого источника до экрана — $L = 6\,\text{м}$.
    Сделайте рисунок и укажите положение нулевого максимума освещенности,
    а также определите расстояние между первым максимумом и нулевым максимумом.
    Длина волны падающего света составляет $\lambda = 600\,\text{нм}$.
}
\solutionspace{180pt}

\tasknumber{3}%
\task{%
    На стеклянную пластинку ($\hat n = 1{,}6$) нанесена прозрачная пленка ($n = 1{,}8$).
    На плёнку нормально к поверхности падает монохроматический свет с длиной волны $\lambda = 500\,\text{нм}$.
    Какова должна быть минимальная толщина пленки, если в результате интерференции свет имеет наибольшую интенсивность?
}

\variantsplitter

\addpersonalvariant{Вероника Битерякова}

\tasknumber{1}%
\task{%
    Длина волны света в~вакууме $\lambda = 500\,\text{нм}$.
    Какова частота этой световой волны?
    Какова длина этой волны в среде с показателем преломления $n = 1{,}7$?
    Может ли человек увидеть такую волну света, и если да, то какой именно цвет соответствует этим волнам в вакууме и в этой среде?
}
\answer{%
    \begin{align*}
    \nu &= \frac 1T = \frac 1{\lambda/c} = \frac c\lambda = \frac{3 \cdot 10^{8}\,\frac{\text{м}}{\text{с}}}{500\,\text{нм}} \approx 6{,}00 \cdot 10^{14}\,\text{Гц}, \\
    \nu' = \nu &\cbr{\text{или } T' = T} \implies \lambda' = v'T' = \frac vn T = \frac{ vt }n = \frac \lambda n = \frac{500\,\text{нм}}{1{,}7} \approx 294 \cdot 10^{-9}\,\text{м}.
    \\
    &\text{380 нм---фиол---440---син---485---гол---500---зел---565---жёл---590---оранж---625---крас---780 нм}
    \end{align*}
}
\solutionspace{180pt}

\tasknumber{2}%
\task{%
    Установка для наблюдения интерференции состоит
    из двух когерентных источников света и экрана.
    Расстояние между источниками $l = 600\,\text{мкм}$,
    а от каждого источника до экрана — $L = 4\,\text{м}$.
    Сделайте рисунок и укажите положение нулевого максимума освещенности,
    а также определите расстояние между первым максимумом и нулевым максимумом.
    Длина волны падающего света составляет $\lambda = 600\,\text{нм}$.
}
\solutionspace{180pt}

\tasknumber{3}%
\task{%
    На стеклянную пластинку ($\hat n = 1{,}6$) нанесена прозрачная пленка ($n = 1{,}2$).
    На плёнку нормально к поверхности падает монохроматический свет с длиной волны $\lambda = 500\,\text{нм}$.
    Какова должна быть минимальная толщина пленки, если в результате интерференции свет имеет наибольшую интенсивность?
}

\variantsplitter

\addpersonalvariant{Юлия Буянова}

\tasknumber{1}%
\task{%
    Длина волны света в~вакууме $\lambda = 400\,\text{нм}$.
    Какова частота этой световой волны?
    Какова длина этой волны в среде с показателем преломления $n = 1{,}5$?
    Может ли человек увидеть такую волну света, и если да, то какой именно цвет соответствует этим волнам в вакууме и в этой среде?
}
\answer{%
    \begin{align*}
    \nu &= \frac 1T = \frac 1{\lambda/c} = \frac c\lambda = \frac{3 \cdot 10^{8}\,\frac{\text{м}}{\text{с}}}{400\,\text{нм}} \approx 7{,}50 \cdot 10^{14}\,\text{Гц}, \\
    \nu' = \nu &\cbr{\text{или } T' = T} \implies \lambda' = v'T' = \frac vn T = \frac{ vt }n = \frac \lambda n = \frac{400\,\text{нм}}{1{,}5} \approx 267 \cdot 10^{-9}\,\text{м}.
    \\
    &\text{380 нм---фиол---440---син---485---гол---500---зел---565---жёл---590---оранж---625---крас---780 нм}
    \end{align*}
}
\solutionspace{180pt}

\tasknumber{2}%
\task{%
    Установка для наблюдения интерференции состоит
    из двух когерентных источников света и экрана.
    Расстояние между источниками $l = 500\,\text{мкм}$,
    а от каждого источника до экрана — $L = 5\,\text{м}$.
    Сделайте рисунок и укажите положение нулевого максимума освещенности,
    а также определите расстояние между первым минимумом и нулевым максимумом.
    Длина волны падающего света составляет $\lambda = 500\,\text{нм}$.
}
\solutionspace{180pt}

\tasknumber{3}%
\task{%
    На стеклянную пластинку ($\hat n = 1{,}6$) нанесена прозрачная пленка ($n = 1{,}8$).
    На плёнку нормально к поверхности падает монохроматический свет с длиной волны $\lambda = 400\,\text{нм}$.
    Какова должна быть минимальная толщина пленки, если в результате интерференции свет имеет наименьшую интенсивность?
}

\variantsplitter

\addpersonalvariant{Пелагея Вдовина}

\tasknumber{1}%
\task{%
    Длина волны света в~вакууме $\lambda = 600\,\text{нм}$.
    Какова частота этой световой волны?
    Какова длина этой волны в среде с показателем преломления $n = 1{,}4$?
    Может ли человек увидеть такую волну света, и если да, то какой именно цвет соответствует этим волнам в вакууме и в этой среде?
}
\answer{%
    \begin{align*}
    \nu &= \frac 1T = \frac 1{\lambda/c} = \frac c\lambda = \frac{3 \cdot 10^{8}\,\frac{\text{м}}{\text{с}}}{600\,\text{нм}} \approx 5{,}00 \cdot 10^{14}\,\text{Гц}, \\
    \nu' = \nu &\cbr{\text{или } T' = T} \implies \lambda' = v'T' = \frac vn T = \frac{ vt }n = \frac \lambda n = \frac{600\,\text{нм}}{1{,}4} \approx 429 \cdot 10^{-9}\,\text{м}.
    \\
    &\text{380 нм---фиол---440---син---485---гол---500---зел---565---жёл---590---оранж---625---крас---780 нм}
    \end{align*}
}
\solutionspace{180pt}

\tasknumber{2}%
\task{%
    Установка для наблюдения интерференции состоит
    из двух когерентных источников света и экрана.
    Расстояние между источниками $l = 600\,\text{мкм}$,
    а от каждого источника до экрана — $L = 5\,\text{м}$.
    Сделайте рисунок и укажите положение нулевого максимума освещенности,
    а также определите расстояние между первым минимумом и нулевым максимумом.
    Длина волны падающего света составляет $\lambda = 600\,\text{нм}$.
}
\solutionspace{180pt}

\tasknumber{3}%
\task{%
    На стеклянную пластинку ($\hat n = 1{,}5$) нанесена прозрачная пленка ($n = 1{,}7$).
    На плёнку нормально к поверхности падает монохроматический свет с длиной волны $\lambda = 500\,\text{нм}$.
    Какова должна быть минимальная толщина пленки, если в результате интерференции свет имеет наибольшую интенсивность?
}

\variantsplitter

\addpersonalvariant{Леонид Викторов}

\tasknumber{1}%
\task{%
    Длина волны света в~вакууме $\lambda = 500\,\text{нм}$.
    Какова частота этой световой волны?
    Какова длина этой волны в среде с показателем преломления $n = 1{,}4$?
    Может ли человек увидеть такую волну света, и если да, то какой именно цвет соответствует этим волнам в вакууме и в этой среде?
}
\answer{%
    \begin{align*}
    \nu &= \frac 1T = \frac 1{\lambda/c} = \frac c\lambda = \frac{3 \cdot 10^{8}\,\frac{\text{м}}{\text{с}}}{500\,\text{нм}} \approx 6{,}00 \cdot 10^{14}\,\text{Гц}, \\
    \nu' = \nu &\cbr{\text{или } T' = T} \implies \lambda' = v'T' = \frac vn T = \frac{ vt }n = \frac \lambda n = \frac{500\,\text{нм}}{1{,}4} \approx 357 \cdot 10^{-9}\,\text{м}.
    \\
    &\text{380 нм---фиол---440---син---485---гол---500---зел---565---жёл---590---оранж---625---крас---780 нм}
    \end{align*}
}
\solutionspace{180pt}

\tasknumber{2}%
\task{%
    Установка для наблюдения интерференции состоит
    из двух когерентных источников света и экрана.
    Расстояние между источниками $l = 500\,\text{мкм}$,
    а от каждого источника до экрана — $L = 4\,\text{м}$.
    Сделайте рисунок и укажите положение нулевого максимума освещенности,
    а также определите расстояние между первым максимумом и нулевым максимумом.
    Длина волны падающего света составляет $\lambda = 400\,\text{нм}$.
}
\solutionspace{180pt}

\tasknumber{3}%
\task{%
    На стеклянную пластинку ($\hat n = 1{,}6$) нанесена прозрачная пленка ($n = 1{,}7$).
    На плёнку нормально к поверхности падает монохроматический свет с длиной волны $\lambda = 600\,\text{нм}$.
    Какова должна быть минимальная толщина пленки, если в результате интерференции свет имеет наименьшую интенсивность?
}

\variantsplitter

\addpersonalvariant{Фёдор Гнутов}

\tasknumber{1}%
\task{%
    Длина волны света в~вакууме $\lambda = 500\,\text{нм}$.
    Какова частота этой световой волны?
    Какова длина этой волны в среде с показателем преломления $n = 1{,}6$?
    Может ли человек увидеть такую волну света, и если да, то какой именно цвет соответствует этим волнам в вакууме и в этой среде?
}
\answer{%
    \begin{align*}
    \nu &= \frac 1T = \frac 1{\lambda/c} = \frac c\lambda = \frac{3 \cdot 10^{8}\,\frac{\text{м}}{\text{с}}}{500\,\text{нм}} \approx 6{,}00 \cdot 10^{14}\,\text{Гц}, \\
    \nu' = \nu &\cbr{\text{или } T' = T} \implies \lambda' = v'T' = \frac vn T = \frac{ vt }n = \frac \lambda n = \frac{500\,\text{нм}}{1{,}6} \approx 313 \cdot 10^{-9}\,\text{м}.
    \\
    &\text{380 нм---фиол---440---син---485---гол---500---зел---565---жёл---590---оранж---625---крас---780 нм}
    \end{align*}
}
\solutionspace{180pt}

\tasknumber{2}%
\task{%
    Установка для наблюдения интерференции состоит
    из двух когерентных источников света и экрана.
    Расстояние между источниками $l = 500\,\text{мкм}$,
    а от каждого источника до экрана — $L = 5\,\text{м}$.
    Сделайте рисунок и укажите положение нулевого максимума освещенности,
    а также определите расстояние между первым максимумом и нулевым максимумом.
    Длина волны падающего света составляет $\lambda = 500\,\text{нм}$.
}
\solutionspace{180pt}

\tasknumber{3}%
\task{%
    На стеклянную пластинку ($\hat n = 1{,}6$) нанесена прозрачная пленка ($n = 1{,}7$).
    На плёнку нормально к поверхности падает монохроматический свет с длиной волны $\lambda = 400\,\text{нм}$.
    Какова должна быть минимальная толщина пленки, если в результате интерференции свет имеет наибольшую интенсивность?
}

\variantsplitter

\addpersonalvariant{Илья Гримберг}

\tasknumber{1}%
\task{%
    Длина волны света в~вакууме $\lambda = 600\,\text{нм}$.
    Какова частота этой световой волны?
    Какова длина этой волны в среде с показателем преломления $n = 1{,}5$?
    Может ли человек увидеть такую волну света, и если да, то какой именно цвет соответствует этим волнам в вакууме и в этой среде?
}
\answer{%
    \begin{align*}
    \nu &= \frac 1T = \frac 1{\lambda/c} = \frac c\lambda = \frac{3 \cdot 10^{8}\,\frac{\text{м}}{\text{с}}}{600\,\text{нм}} \approx 5{,}00 \cdot 10^{14}\,\text{Гц}, \\
    \nu' = \nu &\cbr{\text{или } T' = T} \implies \lambda' = v'T' = \frac vn T = \frac{ vt }n = \frac \lambda n = \frac{600\,\text{нм}}{1{,}5} \approx 400 \cdot 10^{-9}\,\text{м}.
    \\
    &\text{380 нм---фиол---440---син---485---гол---500---зел---565---жёл---590---оранж---625---крас---780 нм}
    \end{align*}
}
\solutionspace{180pt}

\tasknumber{2}%
\task{%
    Установка для наблюдения интерференции состоит
    из двух когерентных источников света и экрана.
    Расстояние между источниками $l = 400\,\text{мкм}$,
    а от каждого источника до экрана — $L = 6\,\text{м}$.
    Сделайте рисунок и укажите положение нулевого максимума освещенности,
    а также определите расстояние между первым минимумом и нулевым максимумом.
    Длина волны падающего света составляет $\lambda = 500\,\text{нм}$.
}
\solutionspace{180pt}

\tasknumber{3}%
\task{%
    На стеклянную пластинку ($\hat n = 1{,}5$) нанесена прозрачная пленка ($n = 1{,}7$).
    На плёнку нормально к поверхности падает монохроматический свет с длиной волны $\lambda = 600\,\text{нм}$.
    Какова должна быть минимальная толщина пленки, если в результате интерференции свет имеет наименьшую интенсивность?
}

\variantsplitter

\addpersonalvariant{Иван Гурьянов}

\tasknumber{1}%
\task{%
    Длина волны света в~вакууме $\lambda = 500\,\text{нм}$.
    Какова частота этой световой волны?
    Какова длина этой волны в среде с показателем преломления $n = 1{,}5$?
    Может ли человек увидеть такую волну света, и если да, то какой именно цвет соответствует этим волнам в вакууме и в этой среде?
}
\answer{%
    \begin{align*}
    \nu &= \frac 1T = \frac 1{\lambda/c} = \frac c\lambda = \frac{3 \cdot 10^{8}\,\frac{\text{м}}{\text{с}}}{500\,\text{нм}} \approx 6{,}00 \cdot 10^{14}\,\text{Гц}, \\
    \nu' = \nu &\cbr{\text{или } T' = T} \implies \lambda' = v'T' = \frac vn T = \frac{ vt }n = \frac \lambda n = \frac{500\,\text{нм}}{1{,}5} \approx 333 \cdot 10^{-9}\,\text{м}.
    \\
    &\text{380 нм---фиол---440---син---485---гол---500---зел---565---жёл---590---оранж---625---крас---780 нм}
    \end{align*}
}
\solutionspace{180pt}

\tasknumber{2}%
\task{%
    Установка для наблюдения интерференции состоит
    из двух когерентных источников света и экрана.
    Расстояние между источниками $l = 500\,\text{мкм}$,
    а от каждого источника до экрана — $L = 2\,\text{м}$.
    Сделайте рисунок и укажите положение нулевого максимума освещенности,
    а также определите расстояние между первым минимумом и нулевым максимумом.
    Длина волны падающего света составляет $\lambda = 500\,\text{нм}$.
}
\solutionspace{180pt}

\tasknumber{3}%
\task{%
    На стеклянную пластинку ($\hat n = 1{,}5$) нанесена прозрачная пленка ($n = 1{,}8$).
    На плёнку нормально к поверхности падает монохроматический свет с длиной волны $\lambda = 500\,\text{нм}$.
    Какова должна быть минимальная толщина пленки, если в результате интерференции свет имеет наименьшую интенсивность?
}

\variantsplitter

\addpersonalvariant{Артём Денежкин}

\tasknumber{1}%
\task{%
    Длина волны света в~вакууме $\lambda = 500\,\text{нм}$.
    Какова частота этой световой волны?
    Какова длина этой волны в среде с показателем преломления $n = 1{,}4$?
    Может ли человек увидеть такую волну света, и если да, то какой именно цвет соответствует этим волнам в вакууме и в этой среде?
}
\answer{%
    \begin{align*}
    \nu &= \frac 1T = \frac 1{\lambda/c} = \frac c\lambda = \frac{3 \cdot 10^{8}\,\frac{\text{м}}{\text{с}}}{500\,\text{нм}} \approx 6{,}00 \cdot 10^{14}\,\text{Гц}, \\
    \nu' = \nu &\cbr{\text{или } T' = T} \implies \lambda' = v'T' = \frac vn T = \frac{ vt }n = \frac \lambda n = \frac{500\,\text{нм}}{1{,}4} \approx 357 \cdot 10^{-9}\,\text{м}.
    \\
    &\text{380 нм---фиол---440---син---485---гол---500---зел---565---жёл---590---оранж---625---крас---780 нм}
    \end{align*}
}
\solutionspace{180pt}

\tasknumber{2}%
\task{%
    Установка для наблюдения интерференции состоит
    из двух когерентных источников света и экрана.
    Расстояние между источниками $l = 600\,\text{мкм}$,
    а от каждого источника до экрана — $L = 2\,\text{м}$.
    Сделайте рисунок и укажите положение нулевого максимума освещенности,
    а также определите расстояние между первым максимумом и нулевым максимумом.
    Длина волны падающего света составляет $\lambda = 400\,\text{нм}$.
}
\solutionspace{180pt}

\tasknumber{3}%
\task{%
    На стеклянную пластинку ($\hat n = 1{,}6$) нанесена прозрачная пленка ($n = 1{,}2$).
    На плёнку нормально к поверхности падает монохроматический свет с длиной волны $\lambda = 400\,\text{нм}$.
    Какова должна быть минимальная толщина пленки, если в результате интерференции свет имеет наименьшую интенсивность?
}

\variantsplitter

\addpersonalvariant{Виктор Жилин}

\tasknumber{1}%
\task{%
    Длина волны света в~вакууме $\lambda = 400\,\text{нм}$.
    Какова частота этой световой волны?
    Какова длина этой волны в среде с показателем преломления $n = 1{,}3$?
    Может ли человек увидеть такую волну света, и если да, то какой именно цвет соответствует этим волнам в вакууме и в этой среде?
}
\answer{%
    \begin{align*}
    \nu &= \frac 1T = \frac 1{\lambda/c} = \frac c\lambda = \frac{3 \cdot 10^{8}\,\frac{\text{м}}{\text{с}}}{400\,\text{нм}} \approx 7{,}50 \cdot 10^{14}\,\text{Гц}, \\
    \nu' = \nu &\cbr{\text{или } T' = T} \implies \lambda' = v'T' = \frac vn T = \frac{ vt }n = \frac \lambda n = \frac{400\,\text{нм}}{1{,}3} \approx 308 \cdot 10^{-9}\,\text{м}.
    \\
    &\text{380 нм---фиол---440---син---485---гол---500---зел---565---жёл---590---оранж---625---крас---780 нм}
    \end{align*}
}
\solutionspace{180pt}

\tasknumber{2}%
\task{%
    Установка для наблюдения интерференции состоит
    из двух когерентных источников света и экрана.
    Расстояние между источниками $l = 400\,\text{мкм}$,
    а от каждого источника до экрана — $L = 6\,\text{м}$.
    Сделайте рисунок и укажите положение нулевого максимума освещенности,
    а также определите расстояние между первым минимумом и нулевым максимумом.
    Длина волны падающего света составляет $\lambda = 600\,\text{нм}$.
}
\solutionspace{180pt}

\tasknumber{3}%
\task{%
    На стеклянную пластинку ($\hat n = 1{,}5$) нанесена прозрачная пленка ($n = 1{,}4$).
    На плёнку нормально к поверхности падает монохроматический свет с длиной волны $\lambda = 500\,\text{нм}$.
    Какова должна быть минимальная толщина пленки, если в результате интерференции свет имеет наименьшую интенсивность?
}

\variantsplitter

\addpersonalvariant{Дмитрий Иванов}

\tasknumber{1}%
\task{%
    Длина волны света в~вакууме $\lambda = 500\,\text{нм}$.
    Какова частота этой световой волны?
    Какова длина этой волны в среде с показателем преломления $n = 1{,}7$?
    Может ли человек увидеть такую волну света, и если да, то какой именно цвет соответствует этим волнам в вакууме и в этой среде?
}
\answer{%
    \begin{align*}
    \nu &= \frac 1T = \frac 1{\lambda/c} = \frac c\lambda = \frac{3 \cdot 10^{8}\,\frac{\text{м}}{\text{с}}}{500\,\text{нм}} \approx 6{,}00 \cdot 10^{14}\,\text{Гц}, \\
    \nu' = \nu &\cbr{\text{или } T' = T} \implies \lambda' = v'T' = \frac vn T = \frac{ vt }n = \frac \lambda n = \frac{500\,\text{нм}}{1{,}7} \approx 294 \cdot 10^{-9}\,\text{м}.
    \\
    &\text{380 нм---фиол---440---син---485---гол---500---зел---565---жёл---590---оранж---625---крас---780 нм}
    \end{align*}
}
\solutionspace{180pt}

\tasknumber{2}%
\task{%
    Установка для наблюдения интерференции состоит
    из двух когерентных источников света и экрана.
    Расстояние между источниками $l = 600\,\text{мкм}$,
    а от каждого источника до экрана — $L = 4\,\text{м}$.
    Сделайте рисунок и укажите положение нулевого максимума освещенности,
    а также определите расстояние между первым максимумом и нулевым максимумом.
    Длина волны падающего света составляет $\lambda = 500\,\text{нм}$.
}
\solutionspace{180pt}

\tasknumber{3}%
\task{%
    На стеклянную пластинку ($\hat n = 1{,}5$) нанесена прозрачная пленка ($n = 1{,}7$).
    На плёнку нормально к поверхности падает монохроматический свет с длиной волны $\lambda = 600\,\text{нм}$.
    Какова должна быть минимальная толщина пленки, если в результате интерференции свет имеет наибольшую интенсивность?
}

\variantsplitter

\addpersonalvariant{Олег Климов}

\tasknumber{1}%
\task{%
    Длина волны света в~вакууме $\lambda = 500\,\text{нм}$.
    Какова частота этой световой волны?
    Какова длина этой волны в среде с показателем преломления $n = 1{,}3$?
    Может ли человек увидеть такую волну света, и если да, то какой именно цвет соответствует этим волнам в вакууме и в этой среде?
}
\answer{%
    \begin{align*}
    \nu &= \frac 1T = \frac 1{\lambda/c} = \frac c\lambda = \frac{3 \cdot 10^{8}\,\frac{\text{м}}{\text{с}}}{500\,\text{нм}} \approx 6{,}00 \cdot 10^{14}\,\text{Гц}, \\
    \nu' = \nu &\cbr{\text{или } T' = T} \implies \lambda' = v'T' = \frac vn T = \frac{ vt }n = \frac \lambda n = \frac{500\,\text{нм}}{1{,}3} \approx 385 \cdot 10^{-9}\,\text{м}.
    \\
    &\text{380 нм---фиол---440---син---485---гол---500---зел---565---жёл---590---оранж---625---крас---780 нм}
    \end{align*}
}
\solutionspace{180pt}

\tasknumber{2}%
\task{%
    Установка для наблюдения интерференции состоит
    из двух когерентных источников света и экрана.
    Расстояние между источниками $l = 400\,\text{мкм}$,
    а от каждого источника до экрана — $L = 4\,\text{м}$.
    Сделайте рисунок и укажите положение нулевого максимума освещенности,
    а также определите расстояние между первым минимумом и нулевым максимумом.
    Длина волны падающего света составляет $\lambda = 400\,\text{нм}$.
}
\solutionspace{180pt}

\tasknumber{3}%
\task{%
    На стеклянную пластинку ($\hat n = 1{,}6$) нанесена прозрачная пленка ($n = 1{,}2$).
    На плёнку нормально к поверхности падает монохроматический свет с длиной волны $\lambda = 500\,\text{нм}$.
    Какова должна быть минимальная толщина пленки, если в результате интерференции свет имеет наибольшую интенсивность?
}

\variantsplitter

\addpersonalvariant{Анна Ковалева}

\tasknumber{1}%
\task{%
    Длина волны света в~вакууме $\lambda = 600\,\text{нм}$.
    Какова частота этой световой волны?
    Какова длина этой волны в среде с показателем преломления $n = 1{,}5$?
    Может ли человек увидеть такую волну света, и если да, то какой именно цвет соответствует этим волнам в вакууме и в этой среде?
}
\answer{%
    \begin{align*}
    \nu &= \frac 1T = \frac 1{\lambda/c} = \frac c\lambda = \frac{3 \cdot 10^{8}\,\frac{\text{м}}{\text{с}}}{600\,\text{нм}} \approx 5{,}00 \cdot 10^{14}\,\text{Гц}, \\
    \nu' = \nu &\cbr{\text{или } T' = T} \implies \lambda' = v'T' = \frac vn T = \frac{ vt }n = \frac \lambda n = \frac{600\,\text{нм}}{1{,}5} \approx 400 \cdot 10^{-9}\,\text{м}.
    \\
    &\text{380 нм---фиол---440---син---485---гол---500---зел---565---жёл---590---оранж---625---крас---780 нм}
    \end{align*}
}
\solutionspace{180pt}

\tasknumber{2}%
\task{%
    Установка для наблюдения интерференции состоит
    из двух когерентных источников света и экрана.
    Расстояние между источниками $l = 500\,\text{мкм}$,
    а от каждого источника до экрана — $L = 4\,\text{м}$.
    Сделайте рисунок и укажите положение нулевого максимума освещенности,
    а также определите расстояние между первым минимумом и нулевым максимумом.
    Длина волны падающего света составляет $\lambda = 600\,\text{нм}$.
}
\solutionspace{180pt}

\tasknumber{3}%
\task{%
    На стеклянную пластинку ($\hat n = 1{,}5$) нанесена прозрачная пленка ($n = 1{,}8$).
    На плёнку нормально к поверхности падает монохроматический свет с длиной волны $\lambda = 600\,\text{нм}$.
    Какова должна быть минимальная толщина пленки, если в результате интерференции свет имеет наименьшую интенсивность?
}

\variantsplitter

\addpersonalvariant{Глеб Ковылин}

\tasknumber{1}%
\task{%
    Длина волны света в~вакууме $\lambda = 700\,\text{нм}$.
    Какова частота этой световой волны?
    Какова длина этой волны в среде с показателем преломления $n = 1{,}7$?
    Может ли человек увидеть такую волну света, и если да, то какой именно цвет соответствует этим волнам в вакууме и в этой среде?
}
\answer{%
    \begin{align*}
    \nu &= \frac 1T = \frac 1{\lambda/c} = \frac c\lambda = \frac{3 \cdot 10^{8}\,\frac{\text{м}}{\text{с}}}{700\,\text{нм}} \approx 4{,}29 \cdot 10^{14}\,\text{Гц}, \\
    \nu' = \nu &\cbr{\text{или } T' = T} \implies \lambda' = v'T' = \frac vn T = \frac{ vt }n = \frac \lambda n = \frac{700\,\text{нм}}{1{,}7} \approx 412 \cdot 10^{-9}\,\text{м}.
    \\
    &\text{380 нм---фиол---440---син---485---гол---500---зел---565---жёл---590---оранж---625---крас---780 нм}
    \end{align*}
}
\solutionspace{180pt}

\tasknumber{2}%
\task{%
    Установка для наблюдения интерференции состоит
    из двух когерентных источников света и экрана.
    Расстояние между источниками $l = 400\,\text{мкм}$,
    а от каждого источника до экрана — $L = 6\,\text{м}$.
    Сделайте рисунок и укажите положение нулевого максимума освещенности,
    а также определите расстояние между первым минимумом и нулевым максимумом.
    Длина волны падающего света составляет $\lambda = 500\,\text{нм}$.
}
\solutionspace{180pt}

\tasknumber{3}%
\task{%
    На стеклянную пластинку ($\hat n = 1{,}6$) нанесена прозрачная пленка ($n = 1{,}8$).
    На плёнку нормально к поверхности падает монохроматический свет с длиной волны $\lambda = 400\,\text{нм}$.
    Какова должна быть минимальная толщина пленки, если в результате интерференции свет имеет наименьшую интенсивность?
}

\variantsplitter

\addpersonalvariant{Даниил Космынин}

\tasknumber{1}%
\task{%
    Длина волны света в~вакууме $\lambda = 400\,\text{нм}$.
    Какова частота этой световой волны?
    Какова длина этой волны в среде с показателем преломления $n = 1{,}3$?
    Может ли человек увидеть такую волну света, и если да, то какой именно цвет соответствует этим волнам в вакууме и в этой среде?
}
\answer{%
    \begin{align*}
    \nu &= \frac 1T = \frac 1{\lambda/c} = \frac c\lambda = \frac{3 \cdot 10^{8}\,\frac{\text{м}}{\text{с}}}{400\,\text{нм}} \approx 7{,}50 \cdot 10^{14}\,\text{Гц}, \\
    \nu' = \nu &\cbr{\text{или } T' = T} \implies \lambda' = v'T' = \frac vn T = \frac{ vt }n = \frac \lambda n = \frac{400\,\text{нм}}{1{,}3} \approx 308 \cdot 10^{-9}\,\text{м}.
    \\
    &\text{380 нм---фиол---440---син---485---гол---500---зел---565---жёл---590---оранж---625---крас---780 нм}
    \end{align*}
}
\solutionspace{180pt}

\tasknumber{2}%
\task{%
    Установка для наблюдения интерференции состоит
    из двух когерентных источников света и экрана.
    Расстояние между источниками $l = 600\,\text{мкм}$,
    а от каждого источника до экрана — $L = 5\,\text{м}$.
    Сделайте рисунок и укажите положение нулевого максимума освещенности,
    а также определите расстояние между первым минимумом и нулевым максимумом.
    Длина волны падающего света составляет $\lambda = 500\,\text{нм}$.
}
\solutionspace{180pt}

\tasknumber{3}%
\task{%
    На стеклянную пластинку ($\hat n = 1{,}6$) нанесена прозрачная пленка ($n = 1{,}2$).
    На плёнку нормально к поверхности падает монохроматический свет с длиной волны $\lambda = 500\,\text{нм}$.
    Какова должна быть минимальная толщина пленки, если в результате интерференции свет имеет наименьшую интенсивность?
}

\variantsplitter

\addpersonalvariant{Алина Леоничева}

\tasknumber{1}%
\task{%
    Длина волны света в~вакууме $\lambda = 600\,\text{нм}$.
    Какова частота этой световой волны?
    Какова длина этой волны в среде с показателем преломления $n = 1{,}3$?
    Может ли человек увидеть такую волну света, и если да, то какой именно цвет соответствует этим волнам в вакууме и в этой среде?
}
\answer{%
    \begin{align*}
    \nu &= \frac 1T = \frac 1{\lambda/c} = \frac c\lambda = \frac{3 \cdot 10^{8}\,\frac{\text{м}}{\text{с}}}{600\,\text{нм}} \approx 5{,}00 \cdot 10^{14}\,\text{Гц}, \\
    \nu' = \nu &\cbr{\text{или } T' = T} \implies \lambda' = v'T' = \frac vn T = \frac{ vt }n = \frac \lambda n = \frac{600\,\text{нм}}{1{,}3} \approx 462 \cdot 10^{-9}\,\text{м}.
    \\
    &\text{380 нм---фиол---440---син---485---гол---500---зел---565---жёл---590---оранж---625---крас---780 нм}
    \end{align*}
}
\solutionspace{180pt}

\tasknumber{2}%
\task{%
    Установка для наблюдения интерференции состоит
    из двух когерентных источников света и экрана.
    Расстояние между источниками $l = 500\,\text{мкм}$,
    а от каждого источника до экрана — $L = 4\,\text{м}$.
    Сделайте рисунок и укажите положение нулевого максимума освещенности,
    а также определите расстояние между первым минимумом и нулевым максимумом.
    Длина волны падающего света составляет $\lambda = 600\,\text{нм}$.
}
\solutionspace{180pt}

\tasknumber{3}%
\task{%
    На стеклянную пластинку ($\hat n = 1{,}5$) нанесена прозрачная пленка ($n = 1{,}4$).
    На плёнку нормально к поверхности падает монохроматический свет с длиной волны $\lambda = 500\,\text{нм}$.
    Какова должна быть минимальная толщина пленки, если в результате интерференции свет имеет наименьшую интенсивность?
}

\variantsplitter

\addpersonalvariant{Ирина Лин}

\tasknumber{1}%
\task{%
    Длина волны света в~вакууме $\lambda = 700\,\text{нм}$.
    Какова частота этой световой волны?
    Какова длина этой волны в среде с показателем преломления $n = 1{,}7$?
    Может ли человек увидеть такую волну света, и если да, то какой именно цвет соответствует этим волнам в вакууме и в этой среде?
}
\answer{%
    \begin{align*}
    \nu &= \frac 1T = \frac 1{\lambda/c} = \frac c\lambda = \frac{3 \cdot 10^{8}\,\frac{\text{м}}{\text{с}}}{700\,\text{нм}} \approx 4{,}29 \cdot 10^{14}\,\text{Гц}, \\
    \nu' = \nu &\cbr{\text{или } T' = T} \implies \lambda' = v'T' = \frac vn T = \frac{ vt }n = \frac \lambda n = \frac{700\,\text{нм}}{1{,}7} \approx 412 \cdot 10^{-9}\,\text{м}.
    \\
    &\text{380 нм---фиол---440---син---485---гол---500---зел---565---жёл---590---оранж---625---крас---780 нм}
    \end{align*}
}
\solutionspace{180pt}

\tasknumber{2}%
\task{%
    Установка для наблюдения интерференции состоит
    из двух когерентных источников света и экрана.
    Расстояние между источниками $l = 500\,\text{мкм}$,
    а от каждого источника до экрана — $L = 5\,\text{м}$.
    Сделайте рисунок и укажите положение нулевого максимума освещенности,
    а также определите расстояние между первым минимумом и нулевым максимумом.
    Длина волны падающего света составляет $\lambda = 400\,\text{нм}$.
}
\solutionspace{180pt}

\tasknumber{3}%
\task{%
    На стеклянную пластинку ($\hat n = 1{,}6$) нанесена прозрачная пленка ($n = 1{,}2$).
    На плёнку нормально к поверхности падает монохроматический свет с длиной волны $\lambda = 500\,\text{нм}$.
    Какова должна быть минимальная толщина пленки, если в результате интерференции свет имеет наибольшую интенсивность?
}

\variantsplitter

\addpersonalvariant{Ислам Мунаев}

\tasknumber{1}%
\task{%
    Длина волны света в~вакууме $\lambda = 700\,\text{нм}$.
    Какова частота этой световой волны?
    Какова длина этой волны в среде с показателем преломления $n = 1{,}4$?
    Может ли человек увидеть такую волну света, и если да, то какой именно цвет соответствует этим волнам в вакууме и в этой среде?
}
\answer{%
    \begin{align*}
    \nu &= \frac 1T = \frac 1{\lambda/c} = \frac c\lambda = \frac{3 \cdot 10^{8}\,\frac{\text{м}}{\text{с}}}{700\,\text{нм}} \approx 4{,}29 \cdot 10^{14}\,\text{Гц}, \\
    \nu' = \nu &\cbr{\text{или } T' = T} \implies \lambda' = v'T' = \frac vn T = \frac{ vt }n = \frac \lambda n = \frac{700\,\text{нм}}{1{,}4} \approx 500 \cdot 10^{-9}\,\text{м}.
    \\
    &\text{380 нм---фиол---440---син---485---гол---500---зел---565---жёл---590---оранж---625---крас---780 нм}
    \end{align*}
}
\solutionspace{180pt}

\tasknumber{2}%
\task{%
    Установка для наблюдения интерференции состоит
    из двух когерентных источников света и экрана.
    Расстояние между источниками $l = 500\,\text{мкм}$,
    а от каждого источника до экрана — $L = 2\,\text{м}$.
    Сделайте рисунок и укажите положение нулевого максимума освещенности,
    а также определите расстояние между первым максимумом и нулевым максимумом.
    Длина волны падающего света составляет $\lambda = 600\,\text{нм}$.
}
\solutionspace{180pt}

\tasknumber{3}%
\task{%
    На стеклянную пластинку ($\hat n = 1{,}5$) нанесена прозрачная пленка ($n = 1{,}2$).
    На плёнку нормально к поверхности падает монохроматический свет с длиной волны $\lambda = 400\,\text{нм}$.
    Какова должна быть минимальная толщина пленки, если в результате интерференции свет имеет наименьшую интенсивность?
}

\variantsplitter

\addpersonalvariant{Александр Наумов}

\tasknumber{1}%
\task{%
    Длина волны света в~вакууме $\lambda = 700\,\text{нм}$.
    Какова частота этой световой волны?
    Какова длина этой волны в среде с показателем преломления $n = 1{,}3$?
    Может ли человек увидеть такую волну света, и если да, то какой именно цвет соответствует этим волнам в вакууме и в этой среде?
}
\answer{%
    \begin{align*}
    \nu &= \frac 1T = \frac 1{\lambda/c} = \frac c\lambda = \frac{3 \cdot 10^{8}\,\frac{\text{м}}{\text{с}}}{700\,\text{нм}} \approx 4{,}29 \cdot 10^{14}\,\text{Гц}, \\
    \nu' = \nu &\cbr{\text{или } T' = T} \implies \lambda' = v'T' = \frac vn T = \frac{ vt }n = \frac \lambda n = \frac{700\,\text{нм}}{1{,}3} \approx 538 \cdot 10^{-9}\,\text{м}.
    \\
    &\text{380 нм---фиол---440---син---485---гол---500---зел---565---жёл---590---оранж---625---крас---780 нм}
    \end{align*}
}
\solutionspace{180pt}

\tasknumber{2}%
\task{%
    Установка для наблюдения интерференции состоит
    из двух когерентных источников света и экрана.
    Расстояние между источниками $l = 400\,\text{мкм}$,
    а от каждого источника до экрана — $L = 4\,\text{м}$.
    Сделайте рисунок и укажите положение нулевого максимума освещенности,
    а также определите расстояние между первым минимумом и нулевым максимумом.
    Длина волны падающего света составляет $\lambda = 400\,\text{нм}$.
}
\solutionspace{180pt}

\tasknumber{3}%
\task{%
    На стеклянную пластинку ($\hat n = 1{,}5$) нанесена прозрачная пленка ($n = 1{,}2$).
    На плёнку нормально к поверхности падает монохроматический свет с длиной волны $\lambda = 600\,\text{нм}$.
    Какова должна быть минимальная толщина пленки, если в результате интерференции свет имеет наименьшую интенсивность?
}

\variantsplitter

\addpersonalvariant{Георгий Новиков}

\tasknumber{1}%
\task{%
    Длина волны света в~вакууме $\lambda = 700\,\text{нм}$.
    Какова частота этой световой волны?
    Какова длина этой волны в среде с показателем преломления $n = 1{,}7$?
    Может ли человек увидеть такую волну света, и если да, то какой именно цвет соответствует этим волнам в вакууме и в этой среде?
}
\answer{%
    \begin{align*}
    \nu &= \frac 1T = \frac 1{\lambda/c} = \frac c\lambda = \frac{3 \cdot 10^{8}\,\frac{\text{м}}{\text{с}}}{700\,\text{нм}} \approx 4{,}29 \cdot 10^{14}\,\text{Гц}, \\
    \nu' = \nu &\cbr{\text{или } T' = T} \implies \lambda' = v'T' = \frac vn T = \frac{ vt }n = \frac \lambda n = \frac{700\,\text{нм}}{1{,}7} \approx 412 \cdot 10^{-9}\,\text{м}.
    \\
    &\text{380 нм---фиол---440---син---485---гол---500---зел---565---жёл---590---оранж---625---крас---780 нм}
    \end{align*}
}
\solutionspace{180pt}

\tasknumber{2}%
\task{%
    Установка для наблюдения интерференции состоит
    из двух когерентных источников света и экрана.
    Расстояние между источниками $l = 600\,\text{мкм}$,
    а от каждого источника до экрана — $L = 2\,\text{м}$.
    Сделайте рисунок и укажите положение нулевого максимума освещенности,
    а также определите расстояние между первым максимумом и нулевым максимумом.
    Длина волны падающего света составляет $\lambda = 500\,\text{нм}$.
}
\solutionspace{180pt}

\tasknumber{3}%
\task{%
    На стеклянную пластинку ($\hat n = 1{,}5$) нанесена прозрачная пленка ($n = 1{,}8$).
    На плёнку нормально к поверхности падает монохроматический свет с длиной волны $\lambda = 500\,\text{нм}$.
    Какова должна быть минимальная толщина пленки, если в результате интерференции свет имеет наименьшую интенсивность?
}

\variantsplitter

\addpersonalvariant{Егор Осипов}

\tasknumber{1}%
\task{%
    Длина волны света в~вакууме $\lambda = 600\,\text{нм}$.
    Какова частота этой световой волны?
    Какова длина этой волны в среде с показателем преломления $n = 1{,}5$?
    Может ли человек увидеть такую волну света, и если да, то какой именно цвет соответствует этим волнам в вакууме и в этой среде?
}
\answer{%
    \begin{align*}
    \nu &= \frac 1T = \frac 1{\lambda/c} = \frac c\lambda = \frac{3 \cdot 10^{8}\,\frac{\text{м}}{\text{с}}}{600\,\text{нм}} \approx 5{,}00 \cdot 10^{14}\,\text{Гц}, \\
    \nu' = \nu &\cbr{\text{или } T' = T} \implies \lambda' = v'T' = \frac vn T = \frac{ vt }n = \frac \lambda n = \frac{600\,\text{нм}}{1{,}5} \approx 400 \cdot 10^{-9}\,\text{м}.
    \\
    &\text{380 нм---фиол---440---син---485---гол---500---зел---565---жёл---590---оранж---625---крас---780 нм}
    \end{align*}
}
\solutionspace{180pt}

\tasknumber{2}%
\task{%
    Установка для наблюдения интерференции состоит
    из двух когерентных источников света и экрана.
    Расстояние между источниками $l = 500\,\text{мкм}$,
    а от каждого источника до экрана — $L = 5\,\text{м}$.
    Сделайте рисунок и укажите положение нулевого максимума освещенности,
    а также определите расстояние между первым минимумом и нулевым максимумом.
    Длина волны падающего света составляет $\lambda = 600\,\text{нм}$.
}
\solutionspace{180pt}

\tasknumber{3}%
\task{%
    На стеклянную пластинку ($\hat n = 1{,}6$) нанесена прозрачная пленка ($n = 1{,}4$).
    На плёнку нормально к поверхности падает монохроматический свет с длиной волны $\lambda = 400\,\text{нм}$.
    Какова должна быть минимальная толщина пленки, если в результате интерференции свет имеет наибольшую интенсивность?
}

\variantsplitter

\addpersonalvariant{Руслан Перепелица}

\tasknumber{1}%
\task{%
    Длина волны света в~вакууме $\lambda = 700\,\text{нм}$.
    Какова частота этой световой волны?
    Какова длина этой волны в среде с показателем преломления $n = 1{,}5$?
    Может ли человек увидеть такую волну света, и если да, то какой именно цвет соответствует этим волнам в вакууме и в этой среде?
}
\answer{%
    \begin{align*}
    \nu &= \frac 1T = \frac 1{\lambda/c} = \frac c\lambda = \frac{3 \cdot 10^{8}\,\frac{\text{м}}{\text{с}}}{700\,\text{нм}} \approx 4{,}29 \cdot 10^{14}\,\text{Гц}, \\
    \nu' = \nu &\cbr{\text{или } T' = T} \implies \lambda' = v'T' = \frac vn T = \frac{ vt }n = \frac \lambda n = \frac{700\,\text{нм}}{1{,}5} \approx 467 \cdot 10^{-9}\,\text{м}.
    \\
    &\text{380 нм---фиол---440---син---485---гол---500---зел---565---жёл---590---оранж---625---крас---780 нм}
    \end{align*}
}
\solutionspace{180pt}

\tasknumber{2}%
\task{%
    Установка для наблюдения интерференции состоит
    из двух когерентных источников света и экрана.
    Расстояние между источниками $l = 600\,\text{мкм}$,
    а от каждого источника до экрана — $L = 5\,\text{м}$.
    Сделайте рисунок и укажите положение нулевого максимума освещенности,
    а также определите расстояние между первым минимумом и нулевым максимумом.
    Длина волны падающего света составляет $\lambda = 600\,\text{нм}$.
}
\solutionspace{180pt}

\tasknumber{3}%
\task{%
    На стеклянную пластинку ($\hat n = 1{,}6$) нанесена прозрачная пленка ($n = 1{,}7$).
    На плёнку нормально к поверхности падает монохроматический свет с длиной волны $\lambda = 600\,\text{нм}$.
    Какова должна быть минимальная толщина пленки, если в результате интерференции свет имеет наибольшую интенсивность?
}

\variantsplitter

\addpersonalvariant{Михаил Перин}

\tasknumber{1}%
\task{%
    Длина волны света в~вакууме $\lambda = 400\,\text{нм}$.
    Какова частота этой световой волны?
    Какова длина этой волны в среде с показателем преломления $n = 1{,}6$?
    Может ли человек увидеть такую волну света, и если да, то какой именно цвет соответствует этим волнам в вакууме и в этой среде?
}
\answer{%
    \begin{align*}
    \nu &= \frac 1T = \frac 1{\lambda/c} = \frac c\lambda = \frac{3 \cdot 10^{8}\,\frac{\text{м}}{\text{с}}}{400\,\text{нм}} \approx 7{,}50 \cdot 10^{14}\,\text{Гц}, \\
    \nu' = \nu &\cbr{\text{или } T' = T} \implies \lambda' = v'T' = \frac vn T = \frac{ vt }n = \frac \lambda n = \frac{400\,\text{нм}}{1{,}6} \approx 250 \cdot 10^{-9}\,\text{м}.
    \\
    &\text{380 нм---фиол---440---син---485---гол---500---зел---565---жёл---590---оранж---625---крас---780 нм}
    \end{align*}
}
\solutionspace{180pt}

\tasknumber{2}%
\task{%
    Установка для наблюдения интерференции состоит
    из двух когерентных источников света и экрана.
    Расстояние между источниками $l = 400\,\text{мкм}$,
    а от каждого источника до экрана — $L = 2\,\text{м}$.
    Сделайте рисунок и укажите положение нулевого максимума освещенности,
    а также определите расстояние между первым максимумом и нулевым максимумом.
    Длина волны падающего света составляет $\lambda = 500\,\text{нм}$.
}
\solutionspace{180pt}

\tasknumber{3}%
\task{%
    На стеклянную пластинку ($\hat n = 1{,}5$) нанесена прозрачная пленка ($n = 1{,}8$).
    На плёнку нормально к поверхности падает монохроматический свет с длиной волны $\lambda = 500\,\text{нм}$.
    Какова должна быть минимальная толщина пленки, если в результате интерференции свет имеет наименьшую интенсивность?
}

\variantsplitter

\addpersonalvariant{Егор Подуровский}

\tasknumber{1}%
\task{%
    Длина волны света в~вакууме $\lambda = 400\,\text{нм}$.
    Какова частота этой световой волны?
    Какова длина этой волны в среде с показателем преломления $n = 1{,}3$?
    Может ли человек увидеть такую волну света, и если да, то какой именно цвет соответствует этим волнам в вакууме и в этой среде?
}
\answer{%
    \begin{align*}
    \nu &= \frac 1T = \frac 1{\lambda/c} = \frac c\lambda = \frac{3 \cdot 10^{8}\,\frac{\text{м}}{\text{с}}}{400\,\text{нм}} \approx 7{,}50 \cdot 10^{14}\,\text{Гц}, \\
    \nu' = \nu &\cbr{\text{или } T' = T} \implies \lambda' = v'T' = \frac vn T = \frac{ vt }n = \frac \lambda n = \frac{400\,\text{нм}}{1{,}3} \approx 308 \cdot 10^{-9}\,\text{м}.
    \\
    &\text{380 нм---фиол---440---син---485---гол---500---зел---565---жёл---590---оранж---625---крас---780 нм}
    \end{align*}
}
\solutionspace{180pt}

\tasknumber{2}%
\task{%
    Установка для наблюдения интерференции состоит
    из двух когерентных источников света и экрана.
    Расстояние между источниками $l = 500\,\text{мкм}$,
    а от каждого источника до экрана — $L = 2\,\text{м}$.
    Сделайте рисунок и укажите положение нулевого максимума освещенности,
    а также определите расстояние между первым минимумом и нулевым максимумом.
    Длина волны падающего света составляет $\lambda = 600\,\text{нм}$.
}
\solutionspace{180pt}

\tasknumber{3}%
\task{%
    На стеклянную пластинку ($\hat n = 1{,}6$) нанесена прозрачная пленка ($n = 1{,}8$).
    На плёнку нормально к поверхности падает монохроматический свет с длиной волны $\lambda = 500\,\text{нм}$.
    Какова должна быть минимальная толщина пленки, если в результате интерференции свет имеет наименьшую интенсивность?
}

\variantsplitter

\addpersonalvariant{Роман Прибылов}

\tasknumber{1}%
\task{%
    Длина волны света в~вакууме $\lambda = 500\,\text{нм}$.
    Какова частота этой световой волны?
    Какова длина этой волны в среде с показателем преломления $n = 1{,}5$?
    Может ли человек увидеть такую волну света, и если да, то какой именно цвет соответствует этим волнам в вакууме и в этой среде?
}
\answer{%
    \begin{align*}
    \nu &= \frac 1T = \frac 1{\lambda/c} = \frac c\lambda = \frac{3 \cdot 10^{8}\,\frac{\text{м}}{\text{с}}}{500\,\text{нм}} \approx 6{,}00 \cdot 10^{14}\,\text{Гц}, \\
    \nu' = \nu &\cbr{\text{или } T' = T} \implies \lambda' = v'T' = \frac vn T = \frac{ vt }n = \frac \lambda n = \frac{500\,\text{нм}}{1{,}5} \approx 333 \cdot 10^{-9}\,\text{м}.
    \\
    &\text{380 нм---фиол---440---син---485---гол---500---зел---565---жёл---590---оранж---625---крас---780 нм}
    \end{align*}
}
\solutionspace{180pt}

\tasknumber{2}%
\task{%
    Установка для наблюдения интерференции состоит
    из двух когерентных источников света и экрана.
    Расстояние между источниками $l = 500\,\text{мкм}$,
    а от каждого источника до экрана — $L = 6\,\text{м}$.
    Сделайте рисунок и укажите положение нулевого максимума освещенности,
    а также определите расстояние между первым минимумом и нулевым максимумом.
    Длина волны падающего света составляет $\lambda = 600\,\text{нм}$.
}
\solutionspace{180pt}

\tasknumber{3}%
\task{%
    На стеклянную пластинку ($\hat n = 1{,}5$) нанесена прозрачная пленка ($n = 1{,}4$).
    На плёнку нормально к поверхности падает монохроматический свет с длиной волны $\lambda = 500\,\text{нм}$.
    Какова должна быть минимальная толщина пленки, если в результате интерференции свет имеет наименьшую интенсивность?
}

\variantsplitter

\addpersonalvariant{Александр Селехметьев}

\tasknumber{1}%
\task{%
    Длина волны света в~вакууме $\lambda = 500\,\text{нм}$.
    Какова частота этой световой волны?
    Какова длина этой волны в среде с показателем преломления $n = 1{,}5$?
    Может ли человек увидеть такую волну света, и если да, то какой именно цвет соответствует этим волнам в вакууме и в этой среде?
}
\answer{%
    \begin{align*}
    \nu &= \frac 1T = \frac 1{\lambda/c} = \frac c\lambda = \frac{3 \cdot 10^{8}\,\frac{\text{м}}{\text{с}}}{500\,\text{нм}} \approx 6{,}00 \cdot 10^{14}\,\text{Гц}, \\
    \nu' = \nu &\cbr{\text{или } T' = T} \implies \lambda' = v'T' = \frac vn T = \frac{ vt }n = \frac \lambda n = \frac{500\,\text{нм}}{1{,}5} \approx 333 \cdot 10^{-9}\,\text{м}.
    \\
    &\text{380 нм---фиол---440---син---485---гол---500---зел---565---жёл---590---оранж---625---крас---780 нм}
    \end{align*}
}
\solutionspace{180pt}

\tasknumber{2}%
\task{%
    Установка для наблюдения интерференции состоит
    из двух когерентных источников света и экрана.
    Расстояние между источниками $l = 400\,\text{мкм}$,
    а от каждого источника до экрана — $L = 6\,\text{м}$.
    Сделайте рисунок и укажите положение нулевого максимума освещенности,
    а также определите расстояние между первым минимумом и нулевым максимумом.
    Длина волны падающего света составляет $\lambda = 400\,\text{нм}$.
}
\solutionspace{180pt}

\tasknumber{3}%
\task{%
    На стеклянную пластинку ($\hat n = 1{,}5$) нанесена прозрачная пленка ($n = 1{,}7$).
    На плёнку нормально к поверхности падает монохроматический свет с длиной волны $\lambda = 500\,\text{нм}$.
    Какова должна быть минимальная толщина пленки, если в результате интерференции свет имеет наибольшую интенсивность?
}

\variantsplitter

\addpersonalvariant{Алексей Тихонов}

\tasknumber{1}%
\task{%
    Длина волны света в~вакууме $\lambda = 600\,\text{нм}$.
    Какова частота этой световой волны?
    Какова длина этой волны в среде с показателем преломления $n = 1{,}7$?
    Может ли человек увидеть такую волну света, и если да, то какой именно цвет соответствует этим волнам в вакууме и в этой среде?
}
\answer{%
    \begin{align*}
    \nu &= \frac 1T = \frac 1{\lambda/c} = \frac c\lambda = \frac{3 \cdot 10^{8}\,\frac{\text{м}}{\text{с}}}{600\,\text{нм}} \approx 5{,}00 \cdot 10^{14}\,\text{Гц}, \\
    \nu' = \nu &\cbr{\text{или } T' = T} \implies \lambda' = v'T' = \frac vn T = \frac{ vt }n = \frac \lambda n = \frac{600\,\text{нм}}{1{,}7} \approx 353 \cdot 10^{-9}\,\text{м}.
    \\
    &\text{380 нм---фиол---440---син---485---гол---500---зел---565---жёл---590---оранж---625---крас---780 нм}
    \end{align*}
}
\solutionspace{180pt}

\tasknumber{2}%
\task{%
    Установка для наблюдения интерференции состоит
    из двух когерентных источников света и экрана.
    Расстояние между источниками $l = 500\,\text{мкм}$,
    а от каждого источника до экрана — $L = 6\,\text{м}$.
    Сделайте рисунок и укажите положение нулевого максимума освещенности,
    а также определите расстояние между первым максимумом и нулевым максимумом.
    Длина волны падающего света составляет $\lambda = 500\,\text{нм}$.
}
\solutionspace{180pt}

\tasknumber{3}%
\task{%
    На стеклянную пластинку ($\hat n = 1{,}6$) нанесена прозрачная пленка ($n = 1{,}4$).
    На плёнку нормально к поверхности падает монохроматический свет с длиной волны $\lambda = 500\,\text{нм}$.
    Какова должна быть минимальная толщина пленки, если в результате интерференции свет имеет наибольшую интенсивность?
}

\variantsplitter

\addpersonalvariant{Алина Филиппова}

\tasknumber{1}%
\task{%
    Длина волны света в~вакууме $\lambda = 700\,\text{нм}$.
    Какова частота этой световой волны?
    Какова длина этой волны в среде с показателем преломления $n = 1{,}3$?
    Может ли человек увидеть такую волну света, и если да, то какой именно цвет соответствует этим волнам в вакууме и в этой среде?
}
\answer{%
    \begin{align*}
    \nu &= \frac 1T = \frac 1{\lambda/c} = \frac c\lambda = \frac{3 \cdot 10^{8}\,\frac{\text{м}}{\text{с}}}{700\,\text{нм}} \approx 4{,}29 \cdot 10^{14}\,\text{Гц}, \\
    \nu' = \nu &\cbr{\text{или } T' = T} \implies \lambda' = v'T' = \frac vn T = \frac{ vt }n = \frac \lambda n = \frac{700\,\text{нм}}{1{,}3} \approx 538 \cdot 10^{-9}\,\text{м}.
    \\
    &\text{380 нм---фиол---440---син---485---гол---500---зел---565---жёл---590---оранж---625---крас---780 нм}
    \end{align*}
}
\solutionspace{180pt}

\tasknumber{2}%
\task{%
    Установка для наблюдения интерференции состоит
    из двух когерентных источников света и экрана.
    Расстояние между источниками $l = 400\,\text{мкм}$,
    а от каждого источника до экрана — $L = 5\,\text{м}$.
    Сделайте рисунок и укажите положение нулевого максимума освещенности,
    а также определите расстояние между первым максимумом и нулевым максимумом.
    Длина волны падающего света составляет $\lambda = 400\,\text{нм}$.
}
\solutionspace{180pt}

\tasknumber{3}%
\task{%
    На стеклянную пластинку ($\hat n = 1{,}6$) нанесена прозрачная пленка ($n = 1{,}8$).
    На плёнку нормально к поверхности падает монохроматический свет с длиной волны $\lambda = 500\,\text{нм}$.
    Какова должна быть минимальная толщина пленки, если в результате интерференции свет имеет наименьшую интенсивность?
}

\variantsplitter

\addpersonalvariant{Дарья Шашкова}

\tasknumber{1}%
\task{%
    Длина волны света в~вакууме $\lambda = 600\,\text{нм}$.
    Какова частота этой световой волны?
    Какова длина этой волны в среде с показателем преломления $n = 1{,}5$?
    Может ли человек увидеть такую волну света, и если да, то какой именно цвет соответствует этим волнам в вакууме и в этой среде?
}
\answer{%
    \begin{align*}
    \nu &= \frac 1T = \frac 1{\lambda/c} = \frac c\lambda = \frac{3 \cdot 10^{8}\,\frac{\text{м}}{\text{с}}}{600\,\text{нм}} \approx 5{,}00 \cdot 10^{14}\,\text{Гц}, \\
    \nu' = \nu &\cbr{\text{или } T' = T} \implies \lambda' = v'T' = \frac vn T = \frac{ vt }n = \frac \lambda n = \frac{600\,\text{нм}}{1{,}5} \approx 400 \cdot 10^{-9}\,\text{м}.
    \\
    &\text{380 нм---фиол---440---син---485---гол---500---зел---565---жёл---590---оранж---625---крас---780 нм}
    \end{align*}
}
\solutionspace{180pt}

\tasknumber{2}%
\task{%
    Установка для наблюдения интерференции состоит
    из двух когерентных источников света и экрана.
    Расстояние между источниками $l = 500\,\text{мкм}$,
    а от каждого источника до экрана — $L = 5\,\text{м}$.
    Сделайте рисунок и укажите положение нулевого максимума освещенности,
    а также определите расстояние между первым максимумом и нулевым максимумом.
    Длина волны падающего света составляет $\lambda = 500\,\text{нм}$.
}
\solutionspace{180pt}

\tasknumber{3}%
\task{%
    На стеклянную пластинку ($\hat n = 1{,}6$) нанесена прозрачная пленка ($n = 1{,}7$).
    На плёнку нормально к поверхности падает монохроматический свет с длиной волны $\lambda = 500\,\text{нм}$.
    Какова должна быть минимальная толщина пленки, если в результате интерференции свет имеет наибольшую интенсивность?
}

\variantsplitter

\addpersonalvariant{Алина Яшина}

\tasknumber{1}%
\task{%
    Длина волны света в~вакууме $\lambda = 600\,\text{нм}$.
    Какова частота этой световой волны?
    Какова длина этой волны в среде с показателем преломления $n = 1{,}5$?
    Может ли человек увидеть такую волну света, и если да, то какой именно цвет соответствует этим волнам в вакууме и в этой среде?
}
\answer{%
    \begin{align*}
    \nu &= \frac 1T = \frac 1{\lambda/c} = \frac c\lambda = \frac{3 \cdot 10^{8}\,\frac{\text{м}}{\text{с}}}{600\,\text{нм}} \approx 5{,}00 \cdot 10^{14}\,\text{Гц}, \\
    \nu' = \nu &\cbr{\text{или } T' = T} \implies \lambda' = v'T' = \frac vn T = \frac{ vt }n = \frac \lambda n = \frac{600\,\text{нм}}{1{,}5} \approx 400 \cdot 10^{-9}\,\text{м}.
    \\
    &\text{380 нм---фиол---440---син---485---гол---500---зел---565---жёл---590---оранж---625---крас---780 нм}
    \end{align*}
}
\solutionspace{180pt}

\tasknumber{2}%
\task{%
    Установка для наблюдения интерференции состоит
    из двух когерентных источников света и экрана.
    Расстояние между источниками $l = 600\,\text{мкм}$,
    а от каждого источника до экрана — $L = 2\,\text{м}$.
    Сделайте рисунок и укажите положение нулевого максимума освещенности,
    а также определите расстояние между первым минимумом и нулевым максимумом.
    Длина волны падающего света составляет $\lambda = 600\,\text{нм}$.
}
\solutionspace{180pt}

\tasknumber{3}%
\task{%
    На стеклянную пластинку ($\hat n = 1{,}5$) нанесена прозрачная пленка ($n = 1{,}7$).
    На плёнку нормально к поверхности падает монохроматический свет с длиной волны $\lambda = 500\,\text{нм}$.
    Какова должна быть минимальная толщина пленки, если в результате интерференции свет имеет наименьшую интенсивность?
}
% autogenerated
