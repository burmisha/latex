\setdate{11~сентября~2019}
\setclass{11«Т»}

\addpersonalvariant{Михаил Бурмистров}

\tasknumber{1}%
\task{%
    Укажите, верны ли утверждения («да» или «нет» слева от каждого утверждения):
    \begin{itemize}
        \item  Если распилить постоянный магнит на 2, то мы получим 2 магнита:
                один только с южным полюсом, а второй — только с северным.
        \item  Полосовой магнит можно распилить 3 разрезами на 4 магнита поменьше.
        \item  Между линиями индукции магнитного поля величина этого поля пренебрежимо мала.
        \item  Линии магнитного поля всегда замкнуты.
        \item  Линии магнитного поля могут пересекаться в полюсах магнитов.
        \item  Линии магнитного поля начинаются у северного полюса и заканчиваются у южного.
        \item  Чем гуще линии — тем сильнее магнитное поле.
        \item  Северный географический полюс Земли в точности совпадает с южным магнитным полюсом Земли.
        \item  Если в компасе установить сильный магнит, то его не удастся отклонить магнитным полем неподалёку.
                Так не делают лишь потому, что компас станет слишком неудобным в бытовом использовании.
        \item  Внутри магнита есть магнитное поле, поэтому для честности мы обязаны рисовать поле как снаружи, так и внутри него.
    \end{itemize}
}
\answer{%
    $\text{нет, да, нет, да, нет, нет, да, нет, нет, да}$
}

\tasknumber{2}%
\task{%
    Для постоянного магнита, изображённого на рис.
    1г)
    изобразите линии индукции магнитного поля
    и укажите, как соориентируются магнитные стрелки в точках A и C.
}
\solutionspace{80pt}

\tasknumber{3}%
\task{%
    Магнитная стрелка вблизи длинного прямолинейного проводника
    повёрнута в точке $B$ северным полюсом направо (см.
    рис.
    2б).
    Сделайте рисунок, укажите направление протекания электрического тока,
    изобразите линии индукции магнитного поля.
}
\solutionspace{80pt}

\tasknumber{4}%
\task{%
    Кольцевой ток (виток с током) ориентирован как указано на рисунке 3б).
    Изобразите линии индукции магнитного поля.
    Укажите, как соориентируется магнитная стрелка в точках B и C.
    Как нужно расположить ещё один кольцевой ток, чтобы между ними возникло притяжение
    (сделайте отдельный рисунок с витками, укажите направления протекания тока и направление сил)?
}

\variantsplitter

\addpersonalvariant{Гагик Аракелян}

\tasknumber{1}%
\task{%
    Укажите, верны ли утверждения («да» или «нет» слева от каждого утверждения):
    \begin{itemize}
        \item  Если распилить постоянный магнит на 2, то мы получим 2 магнита:
                один только с южным полюсом, а второй — только с северным.
        \item  Полосовой магнит можно распилить 2 разрезами на 3 магнита поменьше.
        \item  Между линиями индукции магнитного поля величина этого поля пренебрежимо мала.
        \item  Линии магнитного поля всегда замкнуты.
        \item  Линии магнитного поля могут пересекаться в полюсах магнитов.
        \item  Линии магнитного поля начинаются у северного полюса и заканчиваются у южного.
        \item  Чем гуще линии — тем слабее магнитное поле.
        \item  Северный географический полюс Земли в точности совпадает с северным магнитным полюсом Земли.
        \item  Если в компасе установить сильный магнит, то его не удастся отклонить магнитным полем неподалёку.
                Так не делают лишь потому, что компас станет слишком неудобным в бытовом использовании.
        \item  Внутри магнита есть магнитное поле, поэтому для честности мы обязаны рисовать поле как снаружи, так и внутри него.
    \end{itemize}
}
\answer{%
    $\text{нет, да, нет, да, нет, нет, нет, нет, нет, да}$
}

\tasknumber{2}%
\task{%
    Для постоянного магнита, изображённого на рис.
    1а)
    изобразите линии индукции магнитного поля
    и укажите, как соориентируются магнитные стрелки в точках C и D.
}
\solutionspace{80pt}

\tasknumber{3}%
\task{%
    Магнитная стрелка вблизи длинного прямолинейного проводника
    повёрнута в точке $B$ северным полюсом вниз (см.
    рис.
    2в).
    Сделайте рисунок, укажите направление протекания электрического тока,
    изобразите линии индукции магнитного поля.
}
\solutionspace{80pt}

\tasknumber{4}%
\task{%
    Кольцевой ток (виток с током) ориентирован как указано на рисунке 3в).
    Изобразите линии индукции магнитного поля.
    Укажите, как соориентируется магнитная стрелка в точках A и D.
    Как нужно расположить ещё один кольцевой ток, чтобы между ними возникло отталкивание
    (сделайте отдельный рисунок с витками, укажите направления протекания тока и направление сил)?
}

\variantsplitter

\addpersonalvariant{Ирен Аракелян}

\tasknumber{1}%
\task{%
    Укажите, верны ли утверждения («да» или «нет» слева от каждого утверждения):
    \begin{itemize}
        \item  Если распилить постоянный магнит на 2, то мы получим 2 магнита:
                один только с южным полюсом, а второй — только с северным.
        \item  Полосовой магнит можно распилить 2 разрезами на 3 магнита поменьше.
        \item  Между линиями индукции магнитного поля величина этого поля пренебрежимо мала.
        \item  Линии магнитного поля всегда замкнуты.
        \item  Линии магнитного поля могут пересекаться в полюсах магнитов.
        \item  Линии магнитного поля начинаются у северного полюса и заканчиваются у южного.
        \item  Чем гуще линии — тем сильнее магнитное поле.
        \item  Северный географический полюс Земли в точности совпадает с южным магнитным полюсом Земли.
        \item  Если в компасе установить сильный магнит, то его не удастся отклонить магнитным полем неподалёку.
                Так не делают лишь потому, что компас станет слишком неудобным в бытовом использовании.
        \item  Внутри магнита есть магнитное поле, поэтому для честности мы обязаны рисовать поле как снаружи, так и внутри него.
    \end{itemize}
}
\answer{%
    $\text{нет, да, нет, да, нет, нет, да, нет, нет, да}$
}

\tasknumber{2}%
\task{%
    Для постоянного магнита, изображённого на рис.
    1а)
    изобразите линии индукции магнитного поля
    и укажите, как соориентируются магнитные стрелки в точках A и C.
}
\solutionspace{80pt}

\tasknumber{3}%
\task{%
    Магнитная стрелка вблизи длинного прямолинейного проводника
    повёрнута в точке $B$ северным полюсом направо (см.
    рис.
    2а).
    Сделайте рисунок, укажите направление протекания электрического тока,
    изобразите линии индукции магнитного поля.
}
\solutionspace{80pt}

\tasknumber{4}%
\task{%
    Кольцевой ток (виток с током) ориентирован как указано на рисунке 3г).
    Изобразите линии индукции магнитного поля.
    Укажите, как соориентируется магнитная стрелка в точках A и B.
    Как нужно расположить ещё один кольцевой ток, чтобы между ними возникло отталкивание
    (сделайте отдельный рисунок с витками, укажите направления протекания тока и направление сил)?
}

\variantsplitter

\addpersonalvariant{Сабина Асадуллаева}

\tasknumber{1}%
\task{%
    Укажите, верны ли утверждения («да» или «нет» слева от каждого утверждения):
    \begin{itemize}
        \item  Если распилить постоянный магнит на 2, то мы получим 2 магнита:
                один только с южным полюсом, а второй — только с северным.
        \item  Полосовой магнит можно распилить 2 разрезами на 3 магнита поменьше.
        \item  Между линиями индукции магнитного поля величина этого поля пренебрежимо мала.
        \item  Линии магнитного поля всегда замкнуты.
        \item  Линии магнитного поля могут пересекаться в полюсах магнитов.
        \item  Линии магнитного поля начинаются у северного полюса и заканчиваются у южного.
        \item  Чем гуще линии — тем сильнее магнитное поле.
        \item  Северный географический полюс Земли в точности совпадает с южным магнитным полюсом Земли.
        \item  Если в компасе установить сильный магнит, то его не удастся отклонить магнитным полем неподалёку.
                Так не делают лишь потому, что компас станет слишком неудобным в бытовом использовании.
        \item  Внутри магнита есть магнитное поле, поэтому для честности мы обязаны рисовать поле как снаружи, так и внутри него.
    \end{itemize}
}
\answer{%
    $\text{нет, да, нет, да, нет, нет, да, нет, нет, да}$
}

\tasknumber{2}%
\task{%
    Для постоянного магнита, изображённого на рис.
    1в)
    изобразите линии индукции магнитного поля
    и укажите, как соориентируются магнитные стрелки в точках A и C.
}
\solutionspace{80pt}

\tasknumber{3}%
\task{%
    Магнитная стрелка вблизи длинного прямолинейного проводника
    повёрнута в точке $A$ северным полюсом вверх (см.
    рис.
    2г).
    Сделайте рисунок, укажите направление протекания электрического тока,
    изобразите линии индукции магнитного поля.
}
\solutionspace{80pt}

\tasknumber{4}%
\task{%
    Кольцевой ток (виток с током) ориентирован как указано на рисунке 3б).
    Изобразите линии индукции магнитного поля.
    Укажите, как соориентируется магнитная стрелка в точках B и D.
    Как нужно расположить ещё один кольцевой ток, чтобы между ними возникло отталкивание
    (сделайте отдельный рисунок с витками, укажите направления протекания тока и направление сил)?
}

\variantsplitter

\addpersonalvariant{Вероника Битерякова}

\tasknumber{1}%
\task{%
    Укажите, верны ли утверждения («да» или «нет» слева от каждого утверждения):
    \begin{itemize}
        \item  Если распилить постоянный магнит на 2, то мы получим 2 магнита:
                один только с южным полюсом, а второй — только с северным.
        \item  Полосовой магнит можно распилить 2 разрезами на 3 магнита поменьше.
        \item  Между линиями индукции магнитного поля величина этого поля пренебрежимо мала.
        \item  Линии магнитного поля всегда замкнуты.
        \item  Линии магнитного поля могут пересекаться в полюсах магнитов.
        \item  Линии магнитного поля начинаются у северного полюса и заканчиваются у южного.
        \item  Чем гуще линии — тем сильнее магнитное поле.
        \item  Северный географический полюс Земли в точности совпадает с южным магнитным полюсом Земли.
        \item  Если в компасе установить сильный магнит, то его не удастся отклонить магнитным полем неподалёку.
                Так не делают лишь потому, что компас станет слишком неудобным в бытовом использовании.
        \item  Внутри магнита есть магнитное поле, поэтому для честности мы обязаны рисовать поле как снаружи, так и внутри него.
    \end{itemize}
}
\answer{%
    $\text{нет, да, нет, да, нет, нет, да, нет, нет, да}$
}

\tasknumber{2}%
\task{%
    Для постоянного магнита, изображённого на рис.
    1г)
    изобразите линии индукции магнитного поля
    и укажите, как соориентируются магнитные стрелки в точках B и C.
}
\solutionspace{80pt}

\tasknumber{3}%
\task{%
    Магнитная стрелка вблизи длинного прямолинейного проводника
    повёрнута в точке $B$ северным полюсом направо (см.
    рис.
    2б).
    Сделайте рисунок, укажите направление протекания электрического тока,
    изобразите линии индукции магнитного поля.
}
\solutionspace{80pt}

\tasknumber{4}%
\task{%
    Кольцевой ток (виток с током) ориентирован как указано на рисунке 3г).
    Изобразите линии индукции магнитного поля.
    Укажите, как соориентируется магнитная стрелка в точках B и D.
    Как нужно расположить ещё один кольцевой ток, чтобы между ними возникло отталкивание
    (сделайте отдельный рисунок с витками, укажите направления протекания тока и направление сил)?
}

\variantsplitter

\addpersonalvariant{Юлия Буянова}

\tasknumber{1}%
\task{%
    Укажите, верны ли утверждения («да» или «нет» слева от каждого утверждения):
    \begin{itemize}
        \item  Если распилить постоянный магнит на 2, то мы получим 2 магнита:
                один только с южным полюсом, а второй — только с северным.
        \item  Полосовой магнит можно распилить 2 разрезами на 3 магнита поменьше.
        \item  Между линиями индукции магнитного поля величина этого поля пренебрежимо мала.
        \item  Линии магнитного поля всегда замкнуты.
        \item  Линии магнитного поля могут пересекаться в полюсах магнитов.
        \item  Линии магнитного поля заканчиваются у северного полюса и начинаются у южного.
        \item  Чем гуще линии — тем сильнее магнитное поле.
        \item  Северный географический полюс Земли в точности совпадает с северным магнитным полюсом Земли.
        \item  Если в компасе установить сильный магнит, то его не удастся отклонить магнитным полем неподалёку.
                Так не делают лишь потому, что компас станет слишком неудобным в бытовом использовании.
        \item  Внутри магнита есть магнитное поле, поэтому для честности мы обязаны рисовать поле как снаружи, так и внутри него.
    \end{itemize}
}
\answer{%
    $\text{нет, да, нет, да, нет, нет, да, нет, нет, да}$
}

\tasknumber{2}%
\task{%
    Для постоянного магнита, изображённого на рис.
    1б)
    изобразите линии индукции магнитного поля
    и укажите, как соориентируются магнитные стрелки в точках C и D.
}
\solutionspace{80pt}

\tasknumber{3}%
\task{%
    Магнитная стрелка вблизи длинного прямолинейного проводника
    повёрнута в точке $B$ северным полюсом вверх (см.
    рис.
    2г).
    Сделайте рисунок, укажите направление протекания электрического тока,
    изобразите линии индукции магнитного поля.
}
\solutionspace{80pt}

\tasknumber{4}%
\task{%
    Кольцевой ток (виток с током) ориентирован как указано на рисунке 3а).
    Изобразите линии индукции магнитного поля.
    Укажите, как соориентируется магнитная стрелка в точках A и D.
    Как нужно расположить ещё один кольцевой ток, чтобы между ними возникло отталкивание
    (сделайте отдельный рисунок с витками, укажите направления протекания тока и направление сил)?
}

\variantsplitter

\addpersonalvariant{Пелагея Вдовина}

\tasknumber{1}%
\task{%
    Укажите, верны ли утверждения («да» или «нет» слева от каждого утверждения):
    \begin{itemize}
        \item  Если распилить постоянный магнит на 2, то мы получим 2 магнита:
                один только с южным полюсом, а второй — только с северным.
        \item  Полосовой магнит можно распилить 2 разрезами на 3 магнита поменьше.
        \item  Между линиями индукции магнитного поля величина этого поля пренебрежимо мала.
        \item  Линии магнитного поля всегда замкнуты.
        \item  Линии магнитного поля могут пересекаться в полюсах магнитов.
        \item  Линии магнитного поля начинаются у северного полюса и заканчиваются у южного.
        \item  Чем гуще линии — тем сильнее магнитное поле.
        \item  Северный географический полюс Земли в точности совпадает с южным магнитным полюсом Земли.
        \item  Если в компасе установить сильный магнит, то его не удастся отклонить магнитным полем неподалёку.
                Так не делают лишь потому, что компас станет слишком неудобным в бытовом использовании.
        \item  Внутри магнита есть магнитное поле, поэтому для честности мы обязаны рисовать поле как снаружи, так и внутри него.
    \end{itemize}
}
\answer{%
    $\text{нет, да, нет, да, нет, нет, да, нет, нет, да}$
}

\tasknumber{2}%
\task{%
    Для постоянного магнита, изображённого на рис.
    1а)
    изобразите линии индукции магнитного поля
    и укажите, как соориентируются магнитные стрелки в точках C и D.
}
\solutionspace{80pt}

\tasknumber{3}%
\task{%
    Магнитная стрелка вблизи длинного прямолинейного проводника
    повёрнута в точке $B$ северным полюсом вверх (см.
    рис.
    2в).
    Сделайте рисунок, укажите направление протекания электрического тока,
    изобразите линии индукции магнитного поля.
}
\solutionspace{80pt}

\tasknumber{4}%
\task{%
    Кольцевой ток (виток с током) ориентирован как указано на рисунке 3а).
    Изобразите линии индукции магнитного поля.
    Укажите, как соориентируется магнитная стрелка в точках C и D.
    Как нужно расположить ещё один кольцевой ток, чтобы между ними возникло отталкивание
    (сделайте отдельный рисунок с витками, укажите направления протекания тока и направление сил)?
}

\variantsplitter

\addpersonalvariant{Леонид Викторов}

\tasknumber{1}%
\task{%
    Укажите, верны ли утверждения («да» или «нет» слева от каждого утверждения):
    \begin{itemize}
        \item  Если распилить постоянный магнит на 2, то мы получим 2 магнита:
                один только с южным полюсом, а второй — только с северным.
        \item  Полосовой магнит можно распилить 2 разрезами на 3 магнита поменьше.
        \item  Между линиями индукции магнитного поля величина этого поля пренебрежимо мала.
        \item  Линии магнитного поля всегда замкнуты.
        \item  Линии магнитного поля могут пересекаться в полюсах магнитов.
        \item  Линии магнитного поля заканчиваются у северного полюса и начинаются у южного.
        \item  Чем гуще линии — тем сильнее магнитное поле.
        \item  Северный географический полюс Земли в точности совпадает с северным магнитным полюсом Земли.
        \item  Если в компасе установить сильный магнит, то его не удастся отклонить магнитным полем неподалёку.
                Так не делают лишь потому, что компас станет слишком неудобным в бытовом использовании.
        \item  Внутри магнита есть магнитное поле, поэтому для честности мы обязаны рисовать поле как снаружи, так и внутри него.
    \end{itemize}
}
\answer{%
    $\text{нет, да, нет, да, нет, нет, да, нет, нет, да}$
}

\tasknumber{2}%
\task{%
    Для постоянного магнита, изображённого на рис.
    1в)
    изобразите линии индукции магнитного поля
    и укажите, как соориентируются магнитные стрелки в точках A и C.
}
\solutionspace{80pt}

\tasknumber{3}%
\task{%
    Магнитная стрелка вблизи длинного прямолинейного проводника
    повёрнута в точке $B$ северным полюсом налево (см.
    рис.
    2б).
    Сделайте рисунок, укажите направление протекания электрического тока,
    изобразите линии индукции магнитного поля.
}
\solutionspace{80pt}

\tasknumber{4}%
\task{%
    Кольцевой ток (виток с током) ориентирован как указано на рисунке 3г).
    Изобразите линии индукции магнитного поля.
    Укажите, как соориентируется магнитная стрелка в точках B и D.
    Как нужно расположить ещё один кольцевой ток, чтобы между ними возникло отталкивание
    (сделайте отдельный рисунок с витками, укажите направления протекания тока и направление сил)?
}

\variantsplitter

\addpersonalvariant{Фёдор Гнутов}

\tasknumber{1}%
\task{%
    Укажите, верны ли утверждения («да» или «нет» слева от каждого утверждения):
    \begin{itemize}
        \item  Если распилить постоянный магнит на 2, то мы получим 2 магнита:
                один только с южным полюсом, а второй — только с северным.
        \item  Полосовой магнит можно распилить 3 разрезами на 4 магнита поменьше.
        \item  Между линиями индукции магнитного поля величина этого поля пренебрежимо мала.
        \item  Линии магнитного поля всегда замкнуты.
        \item  Линии магнитного поля могут пересекаться в полюсах магнитов.
        \item  Линии магнитного поля начинаются у северного полюса и заканчиваются у южного.
        \item  Чем гуще линии — тем слабее магнитное поле.
        \item  Северный географический полюс Земли в точности совпадает с южным магнитным полюсом Земли.
        \item  Если в компасе установить сильный магнит, то его не удастся отклонить магнитным полем неподалёку.
                Так не делают лишь потому, что компас станет слишком неудобным в бытовом использовании.
        \item  Внутри магнита есть магнитное поле, поэтому для честности мы обязаны рисовать поле как снаружи, так и внутри него.
    \end{itemize}
}
\answer{%
    $\text{нет, да, нет, да, нет, нет, нет, нет, нет, да}$
}

\tasknumber{2}%
\task{%
    Для постоянного магнита, изображённого на рис.
    1г)
    изобразите линии индукции магнитного поля
    и укажите, как соориентируются магнитные стрелки в точках A и D.
}
\solutionspace{80pt}

\tasknumber{3}%
\task{%
    Магнитная стрелка вблизи длинного прямолинейного проводника
    повёрнута в точке $B$ северным полюсом вверх (см.
    рис.
    2г).
    Сделайте рисунок, укажите направление протекания электрического тока,
    изобразите линии индукции магнитного поля.
}
\solutionspace{80pt}

\tasknumber{4}%
\task{%
    Кольцевой ток (виток с током) ориентирован как указано на рисунке 3а).
    Изобразите линии индукции магнитного поля.
    Укажите, как соориентируется магнитная стрелка в точках A и D.
    Как нужно расположить ещё один кольцевой ток, чтобы между ними возникло отталкивание
    (сделайте отдельный рисунок с витками, укажите направления протекания тока и направление сил)?
}

\variantsplitter

\addpersonalvariant{Илья Гримберг}

\tasknumber{1}%
\task{%
    Укажите, верны ли утверждения («да» или «нет» слева от каждого утверждения):
    \begin{itemize}
        \item  Если распилить постоянный магнит на 2, то мы получим 2 магнита:
                один только с южным полюсом, а второй — только с северным.
        \item  Полосовой магнит можно распилить 2 разрезами на 3 магнита поменьше.
        \item  Между линиями индукции магнитного поля величина этого поля пренебрежимо мала.
        \item  Линии магнитного поля всегда замкнуты.
        \item  Линии магнитного поля могут пересекаться в полюсах магнитов.
        \item  Линии магнитного поля заканчиваются у северного полюса и начинаются у южного.
        \item  Чем гуще линии — тем сильнее магнитное поле.
        \item  Северный географический полюс Земли в точности совпадает с северным магнитным полюсом Земли.
        \item  Если в компасе установить сильный магнит, то его не удастся отклонить магнитным полем неподалёку.
                Так не делают лишь потому, что компас станет слишком неудобным в бытовом использовании.
        \item  Внутри магнита есть магнитное поле, поэтому для честности мы обязаны рисовать поле как снаружи, так и внутри него.
    \end{itemize}
}
\answer{%
    $\text{нет, да, нет, да, нет, нет, да, нет, нет, да}$
}

\tasknumber{2}%
\task{%
    Для постоянного магнита, изображённого на рис.
    1в)
    изобразите линии индукции магнитного поля
    и укажите, как соориентируются магнитные стрелки в точках C и D.
}
\solutionspace{80pt}

\tasknumber{3}%
\task{%
    Магнитная стрелка вблизи длинного прямолинейного проводника
    повёрнута в точке $B$ северным полюсом вниз (см.
    рис.
    2в).
    Сделайте рисунок, укажите направление протекания электрического тока,
    изобразите линии индукции магнитного поля.
}
\solutionspace{80pt}

\tasknumber{4}%
\task{%
    Кольцевой ток (виток с током) ориентирован как указано на рисунке 3г).
    Изобразите линии индукции магнитного поля.
    Укажите, как соориентируется магнитная стрелка в точках A и D.
    Как нужно расположить ещё один кольцевой ток, чтобы между ними возникло отталкивание
    (сделайте отдельный рисунок с витками, укажите направления протекания тока и направление сил)?
}

\variantsplitter

\addpersonalvariant{Иван Гурьянов}

\tasknumber{1}%
\task{%
    Укажите, верны ли утверждения («да» или «нет» слева от каждого утверждения):
    \begin{itemize}
        \item  Если распилить постоянный магнит на 2, то мы получим 2 магнита:
                один только с южным полюсом, а второй — только с северным.
        \item  Полосовой магнит можно распилить 2 разрезами на 3 магнита поменьше.
        \item  Между линиями индукции магнитного поля величина этого поля пренебрежимо мала.
        \item  Линии магнитного поля всегда замкнуты.
        \item  Линии магнитного поля могут пересекаться в полюсах магнитов.
        \item  Линии магнитного поля начинаются у северного полюса и заканчиваются у южного.
        \item  Чем гуще линии — тем слабее магнитное поле.
        \item  Северный географический полюс Земли в точности совпадает с южным магнитным полюсом Земли.
        \item  Если в компасе установить сильный магнит, то его не удастся отклонить магнитным полем неподалёку.
                Так не делают лишь потому, что компас станет слишком неудобным в бытовом использовании.
        \item  Внутри магнита есть магнитное поле, поэтому для честности мы обязаны рисовать поле как снаружи, так и внутри него.
    \end{itemize}
}
\answer{%
    $\text{нет, да, нет, да, нет, нет, нет, нет, нет, да}$
}

\tasknumber{2}%
\task{%
    Для постоянного магнита, изображённого на рис.
    1а)
    изобразите линии индукции магнитного поля
    и укажите, как соориентируются магнитные стрелки в точках A и D.
}
\solutionspace{80pt}

\tasknumber{3}%
\task{%
    Магнитная стрелка вблизи длинного прямолинейного проводника
    повёрнута в точке $B$ северным полюсом направо (см.
    рис.
    2б).
    Сделайте рисунок, укажите направление протекания электрического тока,
    изобразите линии индукции магнитного поля.
}
\solutionspace{80pt}

\tasknumber{4}%
\task{%
    Кольцевой ток (виток с током) ориентирован как указано на рисунке 3б).
    Изобразите линии индукции магнитного поля.
    Укажите, как соориентируется магнитная стрелка в точках B и C.
    Как нужно расположить ещё один кольцевой ток, чтобы между ними возникло отталкивание
    (сделайте отдельный рисунок с витками, укажите направления протекания тока и направление сил)?
}

\variantsplitter

\addpersonalvariant{Артём Денежкин}

\tasknumber{1}%
\task{%
    Укажите, верны ли утверждения («да» или «нет» слева от каждого утверждения):
    \begin{itemize}
        \item  Если распилить постоянный магнит на 2, то мы получим 2 магнита:
                один только с южным полюсом, а второй — только с северным.
        \item  Полосовой магнит можно распилить 3 разрезами на 4 магнита поменьше.
        \item  Между линиями индукции магнитного поля величина этого поля пренебрежимо мала.
        \item  Линии магнитного поля всегда замкнуты.
        \item  Линии магнитного поля могут пересекаться в полюсах магнитов.
        \item  Линии магнитного поля начинаются у северного полюса и заканчиваются у южного.
        \item  Чем гуще линии — тем сильнее магнитное поле.
        \item  Северный географический полюс Земли в точности совпадает с южным магнитным полюсом Земли.
        \item  Если в компасе установить сильный магнит, то его не удастся отклонить магнитным полем неподалёку.
                Так не делают лишь потому, что компас станет слишком неудобным в бытовом использовании.
        \item  Внутри магнита есть магнитное поле, поэтому для честности мы обязаны рисовать поле как снаружи, так и внутри него.
    \end{itemize}
}
\answer{%
    $\text{нет, да, нет, да, нет, нет, да, нет, нет, да}$
}

\tasknumber{2}%
\task{%
    Для постоянного магнита, изображённого на рис.
    1а)
    изобразите линии индукции магнитного поля
    и укажите, как соориентируются магнитные стрелки в точках A и D.
}
\solutionspace{80pt}

\tasknumber{3}%
\task{%
    Магнитная стрелка вблизи длинного прямолинейного проводника
    повёрнута в точке $A$ северным полюсом направо (см.
    рис.
    2б).
    Сделайте рисунок, укажите направление протекания электрического тока,
    изобразите линии индукции магнитного поля.
}
\solutionspace{80pt}

\tasknumber{4}%
\task{%
    Кольцевой ток (виток с током) ориентирован как указано на рисунке 3г).
    Изобразите линии индукции магнитного поля.
    Укажите, как соориентируется магнитная стрелка в точках A и D.
    Как нужно расположить ещё один кольцевой ток, чтобы между ними возникло отталкивание
    (сделайте отдельный рисунок с витками, укажите направления протекания тока и направление сил)?
}

\variantsplitter

\addpersonalvariant{Виктор Жилин}

\tasknumber{1}%
\task{%
    Укажите, верны ли утверждения («да» или «нет» слева от каждого утверждения):
    \begin{itemize}
        \item  Если распилить постоянный магнит на 2, то мы получим 2 магнита:
                один только с южным полюсом, а второй — только с северным.
        \item  Полосовой магнит можно распилить 2 разрезами на 3 магнита поменьше.
        \item  Между линиями индукции магнитного поля величина этого поля пренебрежимо мала.
        \item  Линии магнитного поля всегда замкнуты.
        \item  Линии магнитного поля могут пересекаться в полюсах магнитов.
        \item  Линии магнитного поля начинаются у северного полюса и заканчиваются у южного.
        \item  Чем гуще линии — тем слабее магнитное поле.
        \item  Северный географический полюс Земли в точности совпадает с южным магнитным полюсом Земли.
        \item  Если в компасе установить сильный магнит, то его не удастся отклонить магнитным полем неподалёку.
                Так не делают лишь потому, что компас станет слишком неудобным в бытовом использовании.
        \item  Внутри магнита есть магнитное поле, поэтому для честности мы обязаны рисовать поле как снаружи, так и внутри него.
    \end{itemize}
}
\answer{%
    $\text{нет, да, нет, да, нет, нет, нет, нет, нет, да}$
}

\tasknumber{2}%
\task{%
    Для постоянного магнита, изображённого на рис.
    1а)
    изобразите линии индукции магнитного поля
    и укажите, как соориентируются магнитные стрелки в точках C и D.
}
\solutionspace{80pt}

\tasknumber{3}%
\task{%
    Магнитная стрелка вблизи длинного прямолинейного проводника
    повёрнута в точке $B$ северным полюсом налево (см.
    рис.
    2б).
    Сделайте рисунок, укажите направление протекания электрического тока,
    изобразите линии индукции магнитного поля.
}
\solutionspace{80pt}

\tasknumber{4}%
\task{%
    Кольцевой ток (виток с током) ориентирован как указано на рисунке 3б).
    Изобразите линии индукции магнитного поля.
    Укажите, как соориентируется магнитная стрелка в точках A и C.
    Как нужно расположить ещё один кольцевой ток, чтобы между ними возникло отталкивание
    (сделайте отдельный рисунок с витками, укажите направления протекания тока и направление сил)?
}

\variantsplitter

\addpersonalvariant{Дмитрий Иванов}

\tasknumber{1}%
\task{%
    Укажите, верны ли утверждения («да» или «нет» слева от каждого утверждения):
    \begin{itemize}
        \item  Если распилить постоянный магнит на 2, то мы получим 2 магнита:
                один только с южным полюсом, а второй — только с северным.
        \item  Полосовой магнит можно распилить 2 разрезами на 3 магнита поменьше.
        \item  Между линиями индукции магнитного поля величина этого поля пренебрежимо мала.
        \item  Линии магнитного поля всегда замкнуты.
        \item  Линии магнитного поля могут пересекаться в полюсах магнитов.
        \item  Линии магнитного поля заканчиваются у северного полюса и начинаются у южного.
        \item  Чем гуще линии — тем сильнее магнитное поле.
        \item  Северный географический полюс Земли в точности совпадает с северным магнитным полюсом Земли.
        \item  Если в компасе установить сильный магнит, то его не удастся отклонить магнитным полем неподалёку.
                Так не делают лишь потому, что компас станет слишком неудобным в бытовом использовании.
        \item  Внутри магнита есть магнитное поле, поэтому для честности мы обязаны рисовать поле как снаружи, так и внутри него.
    \end{itemize}
}
\answer{%
    $\text{нет, да, нет, да, нет, нет, да, нет, нет, да}$
}

\tasknumber{2}%
\task{%
    Для постоянного магнита, изображённого на рис.
    1в)
    изобразите линии индукции магнитного поля
    и укажите, как соориентируются магнитные стрелки в точках B и C.
}
\solutionspace{80pt}

\tasknumber{3}%
\task{%
    Магнитная стрелка вблизи длинного прямолинейного проводника
    повёрнута в точке $B$ северным полюсом направо (см.
    рис.
    2а).
    Сделайте рисунок, укажите направление протекания электрического тока,
    изобразите линии индукции магнитного поля.
}
\solutionspace{80pt}

\tasknumber{4}%
\task{%
    Кольцевой ток (виток с током) ориентирован как указано на рисунке 3б).
    Изобразите линии индукции магнитного поля.
    Укажите, как соориентируется магнитная стрелка в точках A и D.
    Как нужно расположить ещё один кольцевой ток, чтобы между ними возникло отталкивание
    (сделайте отдельный рисунок с витками, укажите направления протекания тока и направление сил)?
}

\variantsplitter

\addpersonalvariant{Олег Климов}

\tasknumber{1}%
\task{%
    Укажите, верны ли утверждения («да» или «нет» слева от каждого утверждения):
    \begin{itemize}
        \item  Если распилить постоянный магнит на 2, то мы получим 2 магнита:
                один только с южным полюсом, а второй — только с северным.
        \item  Полосовой магнит можно распилить 2 разрезами на 3 магнита поменьше.
        \item  Между линиями индукции магнитного поля величина этого поля пренебрежимо мала.
        \item  Линии магнитного поля всегда замкнуты.
        \item  Линии магнитного поля могут пересекаться в полюсах магнитов.
        \item  Линии магнитного поля заканчиваются у северного полюса и начинаются у южного.
        \item  Чем гуще линии — тем слабее магнитное поле.
        \item  Северный географический полюс Земли в точности совпадает с южным магнитным полюсом Земли.
        \item  Если в компасе установить сильный магнит, то его не удастся отклонить магнитным полем неподалёку.
                Так не делают лишь потому, что компас станет слишком неудобным в бытовом использовании.
        \item  Внутри магнита есть магнитное поле, поэтому для честности мы обязаны рисовать поле как снаружи, так и внутри него.
    \end{itemize}
}
\answer{%
    $\text{нет, да, нет, да, нет, нет, нет, нет, нет, да}$
}

\tasknumber{2}%
\task{%
    Для постоянного магнита, изображённого на рис.
    1б)
    изобразите линии индукции магнитного поля
    и укажите, как соориентируются магнитные стрелки в точках A и D.
}
\solutionspace{80pt}

\tasknumber{3}%
\task{%
    Магнитная стрелка вблизи длинного прямолинейного проводника
    повёрнута в точке $A$ северным полюсом направо (см.
    рис.
    2а).
    Сделайте рисунок, укажите направление протекания электрического тока,
    изобразите линии индукции магнитного поля.
}
\solutionspace{80pt}

\tasknumber{4}%
\task{%
    Кольцевой ток (виток с током) ориентирован как указано на рисунке 3г).
    Изобразите линии индукции магнитного поля.
    Укажите, как соориентируется магнитная стрелка в точках C и D.
    Как нужно расположить ещё один кольцевой ток, чтобы между ними возникло отталкивание
    (сделайте отдельный рисунок с витками, укажите направления протекания тока и направление сил)?
}

\variantsplitter

\addpersonalvariant{Анна Ковалева}

\tasknumber{1}%
\task{%
    Укажите, верны ли утверждения («да» или «нет» слева от каждого утверждения):
    \begin{itemize}
        \item  Если распилить постоянный магнит на 2, то мы получим 2 магнита:
                один только с южным полюсом, а второй — только с северным.
        \item  Полосовой магнит можно распилить 3 разрезами на 4 магнита поменьше.
        \item  Между линиями индукции магнитного поля величина этого поля пренебрежимо мала.
        \item  Линии магнитного поля всегда замкнуты.
        \item  Линии магнитного поля могут пересекаться в полюсах магнитов.
        \item  Линии магнитного поля начинаются у северного полюса и заканчиваются у южного.
        \item  Чем гуще линии — тем сильнее магнитное поле.
        \item  Северный географический полюс Земли в точности совпадает с северным магнитным полюсом Земли.
        \item  Если в компасе установить сильный магнит, то его не удастся отклонить магнитным полем неподалёку.
                Так не делают лишь потому, что компас станет слишком неудобным в бытовом использовании.
        \item  Внутри магнита есть магнитное поле, поэтому для честности мы обязаны рисовать поле как снаружи, так и внутри него.
    \end{itemize}
}
\answer{%
    $\text{нет, да, нет, да, нет, нет, да, нет, нет, да}$
}

\tasknumber{2}%
\task{%
    Для постоянного магнита, изображённого на рис.
    1а)
    изобразите линии индукции магнитного поля
    и укажите, как соориентируются магнитные стрелки в точках C и D.
}
\solutionspace{80pt}

\tasknumber{3}%
\task{%
    Магнитная стрелка вблизи длинного прямолинейного проводника
    повёрнута в точке $B$ северным полюсом вверх (см.
    рис.
    2в).
    Сделайте рисунок, укажите направление протекания электрического тока,
    изобразите линии индукции магнитного поля.
}
\solutionspace{80pt}

\tasknumber{4}%
\task{%
    Кольцевой ток (виток с током) ориентирован как указано на рисунке 3а).
    Изобразите линии индукции магнитного поля.
    Укажите, как соориентируется магнитная стрелка в точках C и D.
    Как нужно расположить ещё один кольцевой ток, чтобы между ними возникло притяжение
    (сделайте отдельный рисунок с витками, укажите направления протекания тока и направление сил)?
}

\variantsplitter

\addpersonalvariant{Глеб Ковылин}

\tasknumber{1}%
\task{%
    Укажите, верны ли утверждения («да» или «нет» слева от каждого утверждения):
    \begin{itemize}
        \item  Если распилить постоянный магнит на 2, то мы получим 2 магнита:
                один только с южным полюсом, а второй — только с северным.
        \item  Полосовой магнит можно распилить 2 разрезами на 3 магнита поменьше.
        \item  Между линиями индукции магнитного поля величина этого поля пренебрежимо мала.
        \item  Линии магнитного поля всегда замкнуты.
        \item  Линии магнитного поля могут пересекаться в полюсах магнитов.
        \item  Линии магнитного поля начинаются у северного полюса и заканчиваются у южного.
        \item  Чем гуще линии — тем слабее магнитное поле.
        \item  Северный географический полюс Земли в точности совпадает с южным магнитным полюсом Земли.
        \item  Если в компасе установить сильный магнит, то его не удастся отклонить магнитным полем неподалёку.
                Так не делают лишь потому, что компас станет слишком неудобным в бытовом использовании.
        \item  Внутри магнита есть магнитное поле, поэтому для честности мы обязаны рисовать поле как снаружи, так и внутри него.
    \end{itemize}
}
\answer{%
    $\text{нет, да, нет, да, нет, нет, нет, нет, нет, да}$
}

\tasknumber{2}%
\task{%
    Для постоянного магнита, изображённого на рис.
    1а)
    изобразите линии индукции магнитного поля
    и укажите, как соориентируются магнитные стрелки в точках C и D.
}
\solutionspace{80pt}

\tasknumber{3}%
\task{%
    Магнитная стрелка вблизи длинного прямолинейного проводника
    повёрнута в точке $A$ северным полюсом вниз (см.
    рис.
    2в).
    Сделайте рисунок, укажите направление протекания электрического тока,
    изобразите линии индукции магнитного поля.
}
\solutionspace{80pt}

\tasknumber{4}%
\task{%
    Кольцевой ток (виток с током) ориентирован как указано на рисунке 3в).
    Изобразите линии индукции магнитного поля.
    Укажите, как соориентируется магнитная стрелка в точках A и C.
    Как нужно расположить ещё один кольцевой ток, чтобы между ними возникло притяжение
    (сделайте отдельный рисунок с витками, укажите направления протекания тока и направление сил)?
}

\variantsplitter

\addpersonalvariant{Даниил Космынин}

\tasknumber{1}%
\task{%
    Укажите, верны ли утверждения («да» или «нет» слева от каждого утверждения):
    \begin{itemize}
        \item  Если распилить постоянный магнит на 2, то мы получим 2 магнита:
                один только с южным полюсом, а второй — только с северным.
        \item  Полосовой магнит можно распилить 3 разрезами на 4 магнита поменьше.
        \item  Между линиями индукции магнитного поля величина этого поля пренебрежимо мала.
        \item  Линии магнитного поля всегда замкнуты.
        \item  Линии магнитного поля могут пересекаться в полюсах магнитов.
        \item  Линии магнитного поля заканчиваются у северного полюса и начинаются у южного.
        \item  Чем гуще линии — тем сильнее магнитное поле.
        \item  Северный географический полюс Земли в точности совпадает с южным магнитным полюсом Земли.
        \item  Если в компасе установить сильный магнит, то его не удастся отклонить магнитным полем неподалёку.
                Так не делают лишь потому, что компас станет слишком неудобным в бытовом использовании.
        \item  Внутри магнита есть магнитное поле, поэтому для честности мы обязаны рисовать поле как снаружи, так и внутри него.
    \end{itemize}
}
\answer{%
    $\text{нет, да, нет, да, нет, нет, да, нет, нет, да}$
}

\tasknumber{2}%
\task{%
    Для постоянного магнита, изображённого на рис.
    1б)
    изобразите линии индукции магнитного поля
    и укажите, как соориентируются магнитные стрелки в точках A и D.
}
\solutionspace{80pt}

\tasknumber{3}%
\task{%
    Магнитная стрелка вблизи длинного прямолинейного проводника
    повёрнута в точке $A$ северным полюсом вниз (см.
    рис.
    2г).
    Сделайте рисунок, укажите направление протекания электрического тока,
    изобразите линии индукции магнитного поля.
}
\solutionspace{80pt}

\tasknumber{4}%
\task{%
    Кольцевой ток (виток с током) ориентирован как указано на рисунке 3б).
    Изобразите линии индукции магнитного поля.
    Укажите, как соориентируется магнитная стрелка в точках B и D.
    Как нужно расположить ещё один кольцевой ток, чтобы между ними возникло притяжение
    (сделайте отдельный рисунок с витками, укажите направления протекания тока и направление сил)?
}

\variantsplitter

\addpersonalvariant{Алина Леоничева}

\tasknumber{1}%
\task{%
    Укажите, верны ли утверждения («да» или «нет» слева от каждого утверждения):
    \begin{itemize}
        \item  Если распилить постоянный магнит на 2, то мы получим 2 магнита:
                один только с южным полюсом, а второй — только с северным.
        \item  Полосовой магнит можно распилить 2 разрезами на 3 магнита поменьше.
        \item  Между линиями индукции магнитного поля величина этого поля пренебрежимо мала.
        \item  Линии магнитного поля всегда замкнуты.
        \item  Линии магнитного поля могут пересекаться в полюсах магнитов.
        \item  Линии магнитного поля заканчиваются у северного полюса и начинаются у южного.
        \item  Чем гуще линии — тем слабее магнитное поле.
        \item  Северный географический полюс Земли в точности совпадает с северным магнитным полюсом Земли.
        \item  Если в компасе установить сильный магнит, то его не удастся отклонить магнитным полем неподалёку.
                Так не делают лишь потому, что компас станет слишком неудобным в бытовом использовании.
        \item  Внутри магнита есть магнитное поле, поэтому для честности мы обязаны рисовать поле как снаружи, так и внутри него.
    \end{itemize}
}
\answer{%
    $\text{нет, да, нет, да, нет, нет, нет, нет, нет, да}$
}

\tasknumber{2}%
\task{%
    Для постоянного магнита, изображённого на рис.
    1б)
    изобразите линии индукции магнитного поля
    и укажите, как соориентируются магнитные стрелки в точках C и D.
}
\solutionspace{80pt}

\tasknumber{3}%
\task{%
    Магнитная стрелка вблизи длинного прямолинейного проводника
    повёрнута в точке $A$ северным полюсом направо (см.
    рис.
    2а).
    Сделайте рисунок, укажите направление протекания электрического тока,
    изобразите линии индукции магнитного поля.
}
\solutionspace{80pt}

\tasknumber{4}%
\task{%
    Кольцевой ток (виток с током) ориентирован как указано на рисунке 3б).
    Изобразите линии индукции магнитного поля.
    Укажите, как соориентируется магнитная стрелка в точках C и D.
    Как нужно расположить ещё один кольцевой ток, чтобы между ними возникло отталкивание
    (сделайте отдельный рисунок с витками, укажите направления протекания тока и направление сил)?
}

\variantsplitter

\addpersonalvariant{Ирина Лин}

\tasknumber{1}%
\task{%
    Укажите, верны ли утверждения («да» или «нет» слева от каждого утверждения):
    \begin{itemize}
        \item  Если распилить постоянный магнит на 2, то мы получим 2 магнита:
                один только с южным полюсом, а второй — только с северным.
        \item  Полосовой магнит можно распилить 2 разрезами на 3 магнита поменьше.
        \item  Между линиями индукции магнитного поля величина этого поля пренебрежимо мала.
        \item  Линии магнитного поля всегда замкнуты.
        \item  Линии магнитного поля могут пересекаться в полюсах магнитов.
        \item  Линии магнитного поля начинаются у северного полюса и заканчиваются у южного.
        \item  Чем гуще линии — тем сильнее магнитное поле.
        \item  Северный географический полюс Земли в точности совпадает с южным магнитным полюсом Земли.
        \item  Если в компасе установить сильный магнит, то его не удастся отклонить магнитным полем неподалёку.
                Так не делают лишь потому, что компас станет слишком неудобным в бытовом использовании.
        \item  Внутри магнита есть магнитное поле, поэтому для честности мы обязаны рисовать поле как снаружи, так и внутри него.
    \end{itemize}
}
\answer{%
    $\text{нет, да, нет, да, нет, нет, да, нет, нет, да}$
}

\tasknumber{2}%
\task{%
    Для постоянного магнита, изображённого на рис.
    1а)
    изобразите линии индукции магнитного поля
    и укажите, как соориентируются магнитные стрелки в точках A и C.
}
\solutionspace{80pt}

\tasknumber{3}%
\task{%
    Магнитная стрелка вблизи длинного прямолинейного проводника
    повёрнута в точке $B$ северным полюсом направо (см.
    рис.
    2а).
    Сделайте рисунок, укажите направление протекания электрического тока,
    изобразите линии индукции магнитного поля.
}
\solutionspace{80pt}

\tasknumber{4}%
\task{%
    Кольцевой ток (виток с током) ориентирован как указано на рисунке 3б).
    Изобразите линии индукции магнитного поля.
    Укажите, как соориентируется магнитная стрелка в точках A и B.
    Как нужно расположить ещё один кольцевой ток, чтобы между ними возникло притяжение
    (сделайте отдельный рисунок с витками, укажите направления протекания тока и направление сил)?
}

\variantsplitter

\addpersonalvariant{Ислам Мунаев}

\tasknumber{1}%
\task{%
    Укажите, верны ли утверждения («да» или «нет» слева от каждого утверждения):
    \begin{itemize}
        \item  Если распилить постоянный магнит на 2, то мы получим 2 магнита:
                один только с южным полюсом, а второй — только с северным.
        \item  Полосовой магнит можно распилить 2 разрезами на 3 магнита поменьше.
        \item  Между линиями индукции магнитного поля величина этого поля пренебрежимо мала.
        \item  Линии магнитного поля всегда замкнуты.
        \item  Линии магнитного поля могут пересекаться в полюсах магнитов.
        \item  Линии магнитного поля заканчиваются у северного полюса и начинаются у южного.
        \item  Чем гуще линии — тем слабее магнитное поле.
        \item  Северный географический полюс Земли в точности совпадает с северным магнитным полюсом Земли.
        \item  Если в компасе установить сильный магнит, то его не удастся отклонить магнитным полем неподалёку.
                Так не делают лишь потому, что компас станет слишком неудобным в бытовом использовании.
        \item  Внутри магнита есть магнитное поле, поэтому для честности мы обязаны рисовать поле как снаружи, так и внутри него.
    \end{itemize}
}
\answer{%
    $\text{нет, да, нет, да, нет, нет, нет, нет, нет, да}$
}

\tasknumber{2}%
\task{%
    Для постоянного магнита, изображённого на рис.
    1б)
    изобразите линии индукции магнитного поля
    и укажите, как соориентируются магнитные стрелки в точках B и D.
}
\solutionspace{80pt}

\tasknumber{3}%
\task{%
    Магнитная стрелка вблизи длинного прямолинейного проводника
    повёрнута в точке $B$ северным полюсом налево (см.
    рис.
    2б).
    Сделайте рисунок, укажите направление протекания электрического тока,
    изобразите линии индукции магнитного поля.
}
\solutionspace{80pt}

\tasknumber{4}%
\task{%
    Кольцевой ток (виток с током) ориентирован как указано на рисунке 3б).
    Изобразите линии индукции магнитного поля.
    Укажите, как соориентируется магнитная стрелка в точках A и D.
    Как нужно расположить ещё один кольцевой ток, чтобы между ними возникло притяжение
    (сделайте отдельный рисунок с витками, укажите направления протекания тока и направление сил)?
}

\variantsplitter

\addpersonalvariant{Александр Наумов}

\tasknumber{1}%
\task{%
    Укажите, верны ли утверждения («да» или «нет» слева от каждого утверждения):
    \begin{itemize}
        \item  Если распилить постоянный магнит на 2, то мы получим 2 магнита:
                один только с южным полюсом, а второй — только с северным.
        \item  Полосовой магнит можно распилить 3 разрезами на 4 магнита поменьше.
        \item  Между линиями индукции магнитного поля величина этого поля пренебрежимо мала.
        \item  Линии магнитного поля всегда замкнуты.
        \item  Линии магнитного поля могут пересекаться в полюсах магнитов.
        \item  Линии магнитного поля заканчиваются у северного полюса и начинаются у южного.
        \item  Чем гуще линии — тем слабее магнитное поле.
        \item  Северный географический полюс Земли в точности совпадает с северным магнитным полюсом Земли.
        \item  Если в компасе установить сильный магнит, то его не удастся отклонить магнитным полем неподалёку.
                Так не делают лишь потому, что компас станет слишком неудобным в бытовом использовании.
        \item  Внутри магнита есть магнитное поле, поэтому для честности мы обязаны рисовать поле как снаружи, так и внутри него.
    \end{itemize}
}
\answer{%
    $\text{нет, да, нет, да, нет, нет, нет, нет, нет, да}$
}

\tasknumber{2}%
\task{%
    Для постоянного магнита, изображённого на рис.
    1б)
    изобразите линии индукции магнитного поля
    и укажите, как соориентируются магнитные стрелки в точках C и D.
}
\solutionspace{80pt}

\tasknumber{3}%
\task{%
    Магнитная стрелка вблизи длинного прямолинейного проводника
    повёрнута в точке $B$ северным полюсом налево (см.
    рис.
    2а).
    Сделайте рисунок, укажите направление протекания электрического тока,
    изобразите линии индукции магнитного поля.
}
\solutionspace{80pt}

\tasknumber{4}%
\task{%
    Кольцевой ток (виток с током) ориентирован как указано на рисунке 3в).
    Изобразите линии индукции магнитного поля.
    Укажите, как соориентируется магнитная стрелка в точках A и B.
    Как нужно расположить ещё один кольцевой ток, чтобы между ними возникло притяжение
    (сделайте отдельный рисунок с витками, укажите направления протекания тока и направление сил)?
}

\variantsplitter

\addpersonalvariant{Георгий Новиков}

\tasknumber{1}%
\task{%
    Укажите, верны ли утверждения («да» или «нет» слева от каждого утверждения):
    \begin{itemize}
        \item  Если распилить постоянный магнит на 2, то мы получим 2 магнита:
                один только с южным полюсом, а второй — только с северным.
        \item  Полосовой магнит можно распилить 2 разрезами на 3 магнита поменьше.
        \item  Между линиями индукции магнитного поля величина этого поля пренебрежимо мала.
        \item  Линии магнитного поля всегда замкнуты.
        \item  Линии магнитного поля могут пересекаться в полюсах магнитов.
        \item  Линии магнитного поля начинаются у северного полюса и заканчиваются у южного.
        \item  Чем гуще линии — тем сильнее магнитное поле.
        \item  Северный географический полюс Земли в точности совпадает с северным магнитным полюсом Земли.
        \item  Если в компасе установить сильный магнит, то его не удастся отклонить магнитным полем неподалёку.
                Так не делают лишь потому, что компас станет слишком неудобным в бытовом использовании.
        \item  Внутри магнита есть магнитное поле, поэтому для честности мы обязаны рисовать поле как снаружи, так и внутри него.
    \end{itemize}
}
\answer{%
    $\text{нет, да, нет, да, нет, нет, да, нет, нет, да}$
}

\tasknumber{2}%
\task{%
    Для постоянного магнита, изображённого на рис.
    1г)
    изобразите линии индукции магнитного поля
    и укажите, как соориентируются магнитные стрелки в точках B и D.
}
\solutionspace{80pt}

\tasknumber{3}%
\task{%
    Магнитная стрелка вблизи длинного прямолинейного проводника
    повёрнута в точке $B$ северным полюсом вверх (см.
    рис.
    2г).
    Сделайте рисунок, укажите направление протекания электрического тока,
    изобразите линии индукции магнитного поля.
}
\solutionspace{80pt}

\tasknumber{4}%
\task{%
    Кольцевой ток (виток с током) ориентирован как указано на рисунке 3г).
    Изобразите линии индукции магнитного поля.
    Укажите, как соориентируется магнитная стрелка в точках B и D.
    Как нужно расположить ещё один кольцевой ток, чтобы между ними возникло притяжение
    (сделайте отдельный рисунок с витками, укажите направления протекания тока и направление сил)?
}

\variantsplitter

\addpersonalvariant{Егор Осипов}

\tasknumber{1}%
\task{%
    Укажите, верны ли утверждения («да» или «нет» слева от каждого утверждения):
    \begin{itemize}
        \item  Если распилить постоянный магнит на 2, то мы получим 2 магнита:
                один только с южным полюсом, а второй — только с северным.
        \item  Полосовой магнит можно распилить 2 разрезами на 3 магнита поменьше.
        \item  Между линиями индукции магнитного поля величина этого поля пренебрежимо мала.
        \item  Линии магнитного поля всегда замкнуты.
        \item  Линии магнитного поля могут пересекаться в полюсах магнитов.
        \item  Линии магнитного поля начинаются у северного полюса и заканчиваются у южного.
        \item  Чем гуще линии — тем сильнее магнитное поле.
        \item  Северный географический полюс Земли в точности совпадает с южным магнитным полюсом Земли.
        \item  Если в компасе установить сильный магнит, то его не удастся отклонить магнитным полем неподалёку.
                Так не делают лишь потому, что компас станет слишком неудобным в бытовом использовании.
        \item  Внутри магнита есть магнитное поле, поэтому для честности мы обязаны рисовать поле как снаружи, так и внутри него.
    \end{itemize}
}
\answer{%
    $\text{нет, да, нет, да, нет, нет, да, нет, нет, да}$
}

\tasknumber{2}%
\task{%
    Для постоянного магнита, изображённого на рис.
    1а)
    изобразите линии индукции магнитного поля
    и укажите, как соориентируются магнитные стрелки в точках B и C.
}
\solutionspace{80pt}

\tasknumber{3}%
\task{%
    Магнитная стрелка вблизи длинного прямолинейного проводника
    повёрнута в точке $B$ северным полюсом налево (см.
    рис.
    2а).
    Сделайте рисунок, укажите направление протекания электрического тока,
    изобразите линии индукции магнитного поля.
}
\solutionspace{80pt}

\tasknumber{4}%
\task{%
    Кольцевой ток (виток с током) ориентирован как указано на рисунке 3г).
    Изобразите линии индукции магнитного поля.
    Укажите, как соориентируется магнитная стрелка в точках B и C.
    Как нужно расположить ещё один кольцевой ток, чтобы между ними возникло отталкивание
    (сделайте отдельный рисунок с витками, укажите направления протекания тока и направление сил)?
}

\variantsplitter

\addpersonalvariant{Руслан Перепелица}

\tasknumber{1}%
\task{%
    Укажите, верны ли утверждения («да» или «нет» слева от каждого утверждения):
    \begin{itemize}
        \item  Если распилить постоянный магнит на 2, то мы получим 2 магнита:
                один только с южным полюсом, а второй — только с северным.
        \item  Полосовой магнит можно распилить 2 разрезами на 3 магнита поменьше.
        \item  Между линиями индукции магнитного поля величина этого поля пренебрежимо мала.
        \item  Линии магнитного поля всегда замкнуты.
        \item  Линии магнитного поля могут пересекаться в полюсах магнитов.
        \item  Линии магнитного поля начинаются у северного полюса и заканчиваются у южного.
        \item  Чем гуще линии — тем слабее магнитное поле.
        \item  Северный географический полюс Земли в точности совпадает с северным магнитным полюсом Земли.
        \item  Если в компасе установить сильный магнит, то его не удастся отклонить магнитным полем неподалёку.
                Так не делают лишь потому, что компас станет слишком неудобным в бытовом использовании.
        \item  Внутри магнита есть магнитное поле, поэтому для честности мы обязаны рисовать поле как снаружи, так и внутри него.
    \end{itemize}
}
\answer{%
    $\text{нет, да, нет, да, нет, нет, нет, нет, нет, да}$
}

\tasknumber{2}%
\task{%
    Для постоянного магнита, изображённого на рис.
    1г)
    изобразите линии индукции магнитного поля
    и укажите, как соориентируются магнитные стрелки в точках A и D.
}
\solutionspace{80pt}

\tasknumber{3}%
\task{%
    Магнитная стрелка вблизи длинного прямолинейного проводника
    повёрнута в точке $B$ северным полюсом вверх (см.
    рис.
    2г).
    Сделайте рисунок, укажите направление протекания электрического тока,
    изобразите линии индукции магнитного поля.
}
\solutionspace{80pt}

\tasknumber{4}%
\task{%
    Кольцевой ток (виток с током) ориентирован как указано на рисунке 3б).
    Изобразите линии индукции магнитного поля.
    Укажите, как соориентируется магнитная стрелка в точках B и C.
    Как нужно расположить ещё один кольцевой ток, чтобы между ними возникло отталкивание
    (сделайте отдельный рисунок с витками, укажите направления протекания тока и направление сил)?
}

\variantsplitter

\addpersonalvariant{Михаил Перин}

\tasknumber{1}%
\task{%
    Укажите, верны ли утверждения («да» или «нет» слева от каждого утверждения):
    \begin{itemize}
        \item  Если распилить постоянный магнит на 2, то мы получим 2 магнита:
                один только с южным полюсом, а второй — только с северным.
        \item  Полосовой магнит можно распилить 2 разрезами на 3 магнита поменьше.
        \item  Между линиями индукции магнитного поля величина этого поля пренебрежимо мала.
        \item  Линии магнитного поля всегда замкнуты.
        \item  Линии магнитного поля могут пересекаться в полюсах магнитов.
        \item  Линии магнитного поля заканчиваются у северного полюса и начинаются у южного.
        \item  Чем гуще линии — тем слабее магнитное поле.
        \item  Северный географический полюс Земли в точности совпадает с северным магнитным полюсом Земли.
        \item  Если в компасе установить сильный магнит, то его не удастся отклонить магнитным полем неподалёку.
                Так не делают лишь потому, что компас станет слишком неудобным в бытовом использовании.
        \item  Внутри магнита есть магнитное поле, поэтому для честности мы обязаны рисовать поле как снаружи, так и внутри него.
    \end{itemize}
}
\answer{%
    $\text{нет, да, нет, да, нет, нет, нет, нет, нет, да}$
}

\tasknumber{2}%
\task{%
    Для постоянного магнита, изображённого на рис.
    1г)
    изобразите линии индукции магнитного поля
    и укажите, как соориентируются магнитные стрелки в точках B и C.
}
\solutionspace{80pt}

\tasknumber{3}%
\task{%
    Магнитная стрелка вблизи длинного прямолинейного проводника
    повёрнута в точке $B$ северным полюсом вверх (см.
    рис.
    2в).
    Сделайте рисунок, укажите направление протекания электрического тока,
    изобразите линии индукции магнитного поля.
}
\solutionspace{80pt}

\tasknumber{4}%
\task{%
    Кольцевой ток (виток с током) ориентирован как указано на рисунке 3г).
    Изобразите линии индукции магнитного поля.
    Укажите, как соориентируется магнитная стрелка в точках A и D.
    Как нужно расположить ещё один кольцевой ток, чтобы между ними возникло притяжение
    (сделайте отдельный рисунок с витками, укажите направления протекания тока и направление сил)?
}

\variantsplitter

\addpersonalvariant{Егор Подуровский}

\tasknumber{1}%
\task{%
    Укажите, верны ли утверждения («да» или «нет» слева от каждого утверждения):
    \begin{itemize}
        \item  Если распилить постоянный магнит на 2, то мы получим 2 магнита:
                один только с южным полюсом, а второй — только с северным.
        \item  Полосовой магнит можно распилить 2 разрезами на 3 магнита поменьше.
        \item  Между линиями индукции магнитного поля величина этого поля пренебрежимо мала.
        \item  Линии магнитного поля всегда замкнуты.
        \item  Линии магнитного поля могут пересекаться в полюсах магнитов.
        \item  Линии магнитного поля заканчиваются у северного полюса и начинаются у южного.
        \item  Чем гуще линии — тем сильнее магнитное поле.
        \item  Северный географический полюс Земли в точности совпадает с южным магнитным полюсом Земли.
        \item  Если в компасе установить сильный магнит, то его не удастся отклонить магнитным полем неподалёку.
                Так не делают лишь потому, что компас станет слишком неудобным в бытовом использовании.
        \item  Внутри магнита есть магнитное поле, поэтому для честности мы обязаны рисовать поле как снаружи, так и внутри него.
    \end{itemize}
}
\answer{%
    $\text{нет, да, нет, да, нет, нет, да, нет, нет, да}$
}

\tasknumber{2}%
\task{%
    Для постоянного магнита, изображённого на рис.
    1б)
    изобразите линии индукции магнитного поля
    и укажите, как соориентируются магнитные стрелки в точках B и D.
}
\solutionspace{80pt}

\tasknumber{3}%
\task{%
    Магнитная стрелка вблизи длинного прямолинейного проводника
    повёрнута в точке $A$ северным полюсом налево (см.
    рис.
    2б).
    Сделайте рисунок, укажите направление протекания электрического тока,
    изобразите линии индукции магнитного поля.
}
\solutionspace{80pt}

\tasknumber{4}%
\task{%
    Кольцевой ток (виток с током) ориентирован как указано на рисунке 3г).
    Изобразите линии индукции магнитного поля.
    Укажите, как соориентируется магнитная стрелка в точках A и D.
    Как нужно расположить ещё один кольцевой ток, чтобы между ними возникло отталкивание
    (сделайте отдельный рисунок с витками, укажите направления протекания тока и направление сил)?
}

\variantsplitter

\addpersonalvariant{Роман Прибылов}

\tasknumber{1}%
\task{%
    Укажите, верны ли утверждения («да» или «нет» слева от каждого утверждения):
    \begin{itemize}
        \item  Если распилить постоянный магнит на 2, то мы получим 2 магнита:
                один только с южным полюсом, а второй — только с северным.
        \item  Полосовой магнит можно распилить 3 разрезами на 4 магнита поменьше.
        \item  Между линиями индукции магнитного поля величина этого поля пренебрежимо мала.
        \item  Линии магнитного поля всегда замкнуты.
        \item  Линии магнитного поля могут пересекаться в полюсах магнитов.
        \item  Линии магнитного поля начинаются у северного полюса и заканчиваются у южного.
        \item  Чем гуще линии — тем слабее магнитное поле.
        \item  Северный географический полюс Земли в точности совпадает с южным магнитным полюсом Земли.
        \item  Если в компасе установить сильный магнит, то его не удастся отклонить магнитным полем неподалёку.
                Так не делают лишь потому, что компас станет слишком неудобным в бытовом использовании.
        \item  Внутри магнита есть магнитное поле, поэтому для честности мы обязаны рисовать поле как снаружи, так и внутри него.
    \end{itemize}
}
\answer{%
    $\text{нет, да, нет, да, нет, нет, нет, нет, нет, да}$
}

\tasknumber{2}%
\task{%
    Для постоянного магнита, изображённого на рис.
    1б)
    изобразите линии индукции магнитного поля
    и укажите, как соориентируются магнитные стрелки в точках C и D.
}
\solutionspace{80pt}

\tasknumber{3}%
\task{%
    Магнитная стрелка вблизи длинного прямолинейного проводника
    повёрнута в точке $B$ северным полюсом направо (см.
    рис.
    2б).
    Сделайте рисунок, укажите направление протекания электрического тока,
    изобразите линии индукции магнитного поля.
}
\solutionspace{80pt}

\tasknumber{4}%
\task{%
    Кольцевой ток (виток с током) ориентирован как указано на рисунке 3г).
    Изобразите линии индукции магнитного поля.
    Укажите, как соориентируется магнитная стрелка в точках A и B.
    Как нужно расположить ещё один кольцевой ток, чтобы между ними возникло отталкивание
    (сделайте отдельный рисунок с витками, укажите направления протекания тока и направление сил)?
}

\variantsplitter

\addpersonalvariant{Александр Селехметьев}

\tasknumber{1}%
\task{%
    Укажите, верны ли утверждения («да» или «нет» слева от каждого утверждения):
    \begin{itemize}
        \item  Если распилить постоянный магнит на 2, то мы получим 2 магнита:
                один только с южным полюсом, а второй — только с северным.
        \item  Полосовой магнит можно распилить 3 разрезами на 4 магнита поменьше.
        \item  Между линиями индукции магнитного поля величина этого поля пренебрежимо мала.
        \item  Линии магнитного поля всегда замкнуты.
        \item  Линии магнитного поля могут пересекаться в полюсах магнитов.
        \item  Линии магнитного поля начинаются у северного полюса и заканчиваются у южного.
        \item  Чем гуще линии — тем сильнее магнитное поле.
        \item  Северный географический полюс Земли в точности совпадает с южным магнитным полюсом Земли.
        \item  Если в компасе установить сильный магнит, то его не удастся отклонить магнитным полем неподалёку.
                Так не делают лишь потому, что компас станет слишком неудобным в бытовом использовании.
        \item  Внутри магнита есть магнитное поле, поэтому для честности мы обязаны рисовать поле как снаружи, так и внутри него.
    \end{itemize}
}
\answer{%
    $\text{нет, да, нет, да, нет, нет, да, нет, нет, да}$
}

\tasknumber{2}%
\task{%
    Для постоянного магнита, изображённого на рис.
    1б)
    изобразите линии индукции магнитного поля
    и укажите, как соориентируются магнитные стрелки в точках B и C.
}
\solutionspace{80pt}

\tasknumber{3}%
\task{%
    Магнитная стрелка вблизи длинного прямолинейного проводника
    повёрнута в точке $B$ северным полюсом направо (см.
    рис.
    2а).
    Сделайте рисунок, укажите направление протекания электрического тока,
    изобразите линии индукции магнитного поля.
}
\solutionspace{80pt}

\tasknumber{4}%
\task{%
    Кольцевой ток (виток с током) ориентирован как указано на рисунке 3б).
    Изобразите линии индукции магнитного поля.
    Укажите, как соориентируется магнитная стрелка в точках B и D.
    Как нужно расположить ещё один кольцевой ток, чтобы между ними возникло отталкивание
    (сделайте отдельный рисунок с витками, укажите направления протекания тока и направление сил)?
}

\variantsplitter

\addpersonalvariant{Алексей Тихонов}

\tasknumber{1}%
\task{%
    Укажите, верны ли утверждения («да» или «нет» слева от каждого утверждения):
    \begin{itemize}
        \item  Если распилить постоянный магнит на 2, то мы получим 2 магнита:
                один только с южным полюсом, а второй — только с северным.
        \item  Полосовой магнит можно распилить 3 разрезами на 4 магнита поменьше.
        \item  Между линиями индукции магнитного поля величина этого поля пренебрежимо мала.
        \item  Линии магнитного поля всегда замкнуты.
        \item  Линии магнитного поля могут пересекаться в полюсах магнитов.
        \item  Линии магнитного поля начинаются у северного полюса и заканчиваются у южного.
        \item  Чем гуще линии — тем сильнее магнитное поле.
        \item  Северный географический полюс Земли в точности совпадает с северным магнитным полюсом Земли.
        \item  Если в компасе установить сильный магнит, то его не удастся отклонить магнитным полем неподалёку.
                Так не делают лишь потому, что компас станет слишком неудобным в бытовом использовании.
        \item  Внутри магнита есть магнитное поле, поэтому для честности мы обязаны рисовать поле как снаружи, так и внутри него.
    \end{itemize}
}
\answer{%
    $\text{нет, да, нет, да, нет, нет, да, нет, нет, да}$
}

\tasknumber{2}%
\task{%
    Для постоянного магнита, изображённого на рис.
    1б)
    изобразите линии индукции магнитного поля
    и укажите, как соориентируются магнитные стрелки в точках B и D.
}
\solutionspace{80pt}

\tasknumber{3}%
\task{%
    Магнитная стрелка вблизи длинного прямолинейного проводника
    повёрнута в точке $B$ северным полюсом вниз (см.
    рис.
    2г).
    Сделайте рисунок, укажите направление протекания электрического тока,
    изобразите линии индукции магнитного поля.
}
\solutionspace{80pt}

\tasknumber{4}%
\task{%
    Кольцевой ток (виток с током) ориентирован как указано на рисунке 3г).
    Изобразите линии индукции магнитного поля.
    Укажите, как соориентируется магнитная стрелка в точках A и B.
    Как нужно расположить ещё один кольцевой ток, чтобы между ними возникло притяжение
    (сделайте отдельный рисунок с витками, укажите направления протекания тока и направление сил)?
}

\variantsplitter

\addpersonalvariant{Алина Филиппова}

\tasknumber{1}%
\task{%
    Укажите, верны ли утверждения («да» или «нет» слева от каждого утверждения):
    \begin{itemize}
        \item  Если распилить постоянный магнит на 2, то мы получим 2 магнита:
                один только с южным полюсом, а второй — только с северным.
        \item  Полосовой магнит можно распилить 2 разрезами на 3 магнита поменьше.
        \item  Между линиями индукции магнитного поля величина этого поля пренебрежимо мала.
        \item  Линии магнитного поля всегда замкнуты.
        \item  Линии магнитного поля могут пересекаться в полюсах магнитов.
        \item  Линии магнитного поля начинаются у северного полюса и заканчиваются у южного.
        \item  Чем гуще линии — тем слабее магнитное поле.
        \item  Северный географический полюс Земли в точности совпадает с северным магнитным полюсом Земли.
        \item  Если в компасе установить сильный магнит, то его не удастся отклонить магнитным полем неподалёку.
                Так не делают лишь потому, что компас станет слишком неудобным в бытовом использовании.
        \item  Внутри магнита есть магнитное поле, поэтому для честности мы обязаны рисовать поле как снаружи, так и внутри него.
    \end{itemize}
}
\answer{%
    $\text{нет, да, нет, да, нет, нет, нет, нет, нет, да}$
}

\tasknumber{2}%
\task{%
    Для постоянного магнита, изображённого на рис.
    1а)
    изобразите линии индукции магнитного поля
    и укажите, как соориентируются магнитные стрелки в точках A и D.
}
\solutionspace{80pt}

\tasknumber{3}%
\task{%
    Магнитная стрелка вблизи длинного прямолинейного проводника
    повёрнута в точке $B$ северным полюсом вниз (см.
    рис.
    2г).
    Сделайте рисунок, укажите направление протекания электрического тока,
    изобразите линии индукции магнитного поля.
}
\solutionspace{80pt}

\tasknumber{4}%
\task{%
    Кольцевой ток (виток с током) ориентирован как указано на рисунке 3в).
    Изобразите линии индукции магнитного поля.
    Укажите, как соориентируется магнитная стрелка в точках C и D.
    Как нужно расположить ещё один кольцевой ток, чтобы между ними возникло притяжение
    (сделайте отдельный рисунок с витками, укажите направления протекания тока и направление сил)?
}

\variantsplitter

\addpersonalvariant{Дарья Шашкова}

\tasknumber{1}%
\task{%
    Укажите, верны ли утверждения («да» или «нет» слева от каждого утверждения):
    \begin{itemize}
        \item  Если распилить постоянный магнит на 2, то мы получим 2 магнита:
                один только с южным полюсом, а второй — только с северным.
        \item  Полосовой магнит можно распилить 2 разрезами на 3 магнита поменьше.
        \item  Между линиями индукции магнитного поля величина этого поля пренебрежимо мала.
        \item  Линии магнитного поля всегда замкнуты.
        \item  Линии магнитного поля могут пересекаться в полюсах магнитов.
        \item  Линии магнитного поля начинаются у северного полюса и заканчиваются у южного.
        \item  Чем гуще линии — тем слабее магнитное поле.
        \item  Северный географический полюс Земли в точности совпадает с северным магнитным полюсом Земли.
        \item  Если в компасе установить сильный магнит, то его не удастся отклонить магнитным полем неподалёку.
                Так не делают лишь потому, что компас станет слишком неудобным в бытовом использовании.
        \item  Внутри магнита есть магнитное поле, поэтому для честности мы обязаны рисовать поле как снаружи, так и внутри него.
    \end{itemize}
}
\answer{%
    $\text{нет, да, нет, да, нет, нет, нет, нет, нет, да}$
}

\tasknumber{2}%
\task{%
    Для постоянного магнита, изображённого на рис.
    1г)
    изобразите линии индукции магнитного поля
    и укажите, как соориентируются магнитные стрелки в точках A и B.
}
\solutionspace{80pt}

\tasknumber{3}%
\task{%
    Магнитная стрелка вблизи длинного прямолинейного проводника
    повёрнута в точке $B$ северным полюсом вниз (см.
    рис.
    2г).
    Сделайте рисунок, укажите направление протекания электрического тока,
    изобразите линии индукции магнитного поля.
}
\solutionspace{80pt}

\tasknumber{4}%
\task{%
    Кольцевой ток (виток с током) ориентирован как указано на рисунке 3в).
    Изобразите линии индукции магнитного поля.
    Укажите, как соориентируется магнитная стрелка в точках A и D.
    Как нужно расположить ещё один кольцевой ток, чтобы между ними возникло притяжение
    (сделайте отдельный рисунок с витками, укажите направления протекания тока и направление сил)?
}

\variantsplitter

\addpersonalvariant{Алина Яшина}

\tasknumber{1}%
\task{%
    Укажите, верны ли утверждения («да» или «нет» слева от каждого утверждения):
    \begin{itemize}
        \item  Если распилить постоянный магнит на 2, то мы получим 2 магнита:
                один только с южным полюсом, а второй — только с северным.
        \item  Полосовой магнит можно распилить 2 разрезами на 3 магнита поменьше.
        \item  Между линиями индукции магнитного поля величина этого поля пренебрежимо мала.
        \item  Линии магнитного поля всегда замкнуты.
        \item  Линии магнитного поля могут пересекаться в полюсах магнитов.
        \item  Линии магнитного поля начинаются у северного полюса и заканчиваются у южного.
        \item  Чем гуще линии — тем сильнее магнитное поле.
        \item  Северный географический полюс Земли в точности совпадает с южным магнитным полюсом Земли.
        \item  Если в компасе установить сильный магнит, то его не удастся отклонить магнитным полем неподалёку.
                Так не делают лишь потому, что компас станет слишком неудобным в бытовом использовании.
        \item  Внутри магнита есть магнитное поле, поэтому для честности мы обязаны рисовать поле как снаружи, так и внутри него.
    \end{itemize}
}
\answer{%
    $\text{нет, да, нет, да, нет, нет, да, нет, нет, да}$
}

\tasknumber{2}%
\task{%
    Для постоянного магнита, изображённого на рис.
    1б)
    изобразите линии индукции магнитного поля
    и укажите, как соориентируются магнитные стрелки в точках A и B.
}
\solutionspace{80pt}

\tasknumber{3}%
\task{%
    Магнитная стрелка вблизи длинного прямолинейного проводника
    повёрнута в точке $B$ северным полюсом вниз (см.
    рис.
    2в).
    Сделайте рисунок, укажите направление протекания электрического тока,
    изобразите линии индукции магнитного поля.
}
\solutionspace{80pt}

\tasknumber{4}%
\task{%
    Кольцевой ток (виток с током) ориентирован как указано на рисунке 3а).
    Изобразите линии индукции магнитного поля.
    Укажите, как соориентируется магнитная стрелка в точках A и B.
    Как нужно расположить ещё один кольцевой ток, чтобы между ними возникло отталкивание
    (сделайте отдельный рисунок с витками, укажите направления протекания тока и направление сил)?
}
% autogenerated
