\setdate{29~апреля~2020}
\setclass{11«Т»}

\addpersonalvariant{Михаил Бурмистров}

\tasknumber{1}%
\task{%
    Для частицы, движущейся с релятивистской скоростью,
    выразите $v$ и $p$ через $c$, $E_\text{кин}$ и $E_0$,
    где $E_\text{кин}$~--- кинетическая энергия частицы,
    а $E_0$, $p$ и $v$~--- её энергия покоя импульс и скорость.
}
\answer{%
    \begin{align*}
    E_\text{кин}, E_0:\quad&E = E_\text{кин} + E_0 = \frac{E_0}{\sqrt{1 - \frac{v^2}{c^2}}} \implies \sqrt{1 - \frac{v^2}{c^2}} = \frac{E_0}{{E_0} + {E_\text{кин}}} \implies v = c\sqrt{1 - \sqr{\frac{E_0}{{E_0} + {E_\text{кин}}}}} \\
    &p = \frac{mv}{\sqrt{1 - \frac{v^2}{c^2}}} = \frac{E_0}{c^2} \cdot \sqrt{1 - \sqr{\frac{E_0}{{E_0} + {E_\text{кин}}}}} \cdot \frac{{E_\text{кин}} + {E_0}}{E_0} = \frac{E_0}{c^2} \cdot \sqrt{\sqr{\frac{{E_\text{кин}} + {E_0}}{E_0}} - 1}.
    \\
    E_\text{кин}, p:\quad&E_\text{кин} = E - E_0 = mc^2\cbr{\frac 1{\sqrt{1 - \frac{v^2}{c^2}}} - 1}, p = \frac{mv}{\sqrt{1 - \frac{v^2}{c^2}}} \implies \frac{E_\text{кин}}{p} = \frac{\frac 1{\sqrt{1 - \frac{v^2}{c^2}}} - 1}{\sqrt{1 - \frac{v^2}{c^2}}} \implies v = \ldots \\
    &E_0 = E - E_\text{кин} = \frac{E_0}{\sqrt{1 - \frac{v^2}{c^2}}} - E_\text{кин} \implies E_0 = \frac{E_\text{кин}}{\frac 1{\sqrt{1 - \frac{v^2}{c^2}}} - 1} = \ldots \\
    E_\text{кин}, v:\quad&E_\text{кин} = E - E_0 = mc^2\cbr{\frac 1{\sqrt{1 - \frac{v^2}{c^2}}} - 1} \implies m = \frac{E_\text{кин}}{c^2\cbr{\frac 1{\sqrt{1 - \frac{v^2}{c^2}}} - 1}} \\
    &E_0 = mc^2 = \frac{E_\text{кин}}{\frac 1{\sqrt{1 - \frac{v^2}{c^2}}} - 1} \\
    &p = \frac{mv}{\sqrt{1 - \frac{v^2}{c^2}}} = \frac{E_\text{кин}}{c^2\cbr{\frac 1{\sqrt{1 - \frac{v^2}{c^2}}} - 1}} \cdot \frac{v}{\sqrt{1 - \frac{v^2}{c^2}}} = \frac{{E_\text{кин}} v}{c^2\cbr{1 - {\sqrt{1 - \frac{v^2}{c^2}}}}} \\
    E_0, p:\quad&E_0 = mc^2, \quad p = \frac{mv}{\sqrt{1 - \frac{v^2}{c^2}}} \implies \frac{E_0}{p} = \frac{c^2}v{\sqrt{1 - \frac{v^2}{c^2}}} = c\sqrt{\frac{c^2}{v^2} - 1} \\
    &\sqr{\frac{E_0}{pc}} = \frac{c^2}{v^2} - 1 \implies \frac{v^2}{c^2} = \frac 1{1 + \frac{E_0^2}{p^2c^2}} \implies v = \frac c{\sqrt{1 + \frac{E_0^2}{p^2c^2}}} \\
    &{E_\text{кин}} = E - E_0 = \sqrt{E_0^2 + p^2c^2} - E_0 \\
    E_0, v:\quad&E_0 = mc^2 \implies m = \frac{E_0}{c^2} \qquad p = \frac{mv}{\sqrt{1 - \frac{v^2}{c^2}}} = \frac{E_0}{c^2} \cdot \frac{v}{\sqrt{1 - \frac{v^2}{c^2}}} \\
    &E_\text{кин}= mc^2\cbr{\frac 1{\sqrt{1 - \frac{v^2}{c^2}}} - 1} = \frac{E_0}{c^2}\cbr{\frac 1{\sqrt{1 - \frac{v^2}{c^2}}} - 1} \\
    p, v:\quad&p = \frac{mv}{\sqrt{1 - \frac{v^2}{c^2}}} \implies m = \frac p v {\sqrt{1 - \frac{v^2}{c^2}}} \implies E_0 = mc^2 =\frac {pc^2} v {\sqrt{1 - \frac{v^2}{c^2}}} \\
    &E_\text{кин} = mc^2\cbr{\frac 1{\sqrt{1 - \frac{v^2}{c^2}}} - 1} = \frac p v {\sqrt{1 - \frac{v^2}{c^2}}}\cbr{\frac 1{\sqrt{1 - \frac{v^2}{c^2}}} - 1} = \frac p v \cbr{1 - {\sqrt{1 - \frac{v^2}{c^2}}}}
    \end{align*}
}
\solutionspace{200pt}

\tasknumber{2}%
\task{%
    Позитрон движется со скоростью $0{,}6\,c$, где $c$~--- скорость света в вакууме.
    Каково при этом отношение кинетической энергии частицы $E_\text{кин.}$ к его энергии покоя $E_0$?
}
\answer{%
    \begin{align*}
    E &= \frac{E_0}{\sqrt{1 - \frac{v^2}{c^2}}}
            \implies \frac E{E_0}
                = \frac 1{\sqrt{1 - \frac{v^2}{c^2}}}
                = \frac 1{\sqrt{1 - \sqr{0{,}6}}}
                \approx 1{,}250,
         \\
        E_{\text{кин}} &= E - E_0
            \implies \frac{E_{\text{кин}}}{E_0}
                = \frac E{E_0} - 1
                = \frac 1{\sqrt{1 - \frac{v^2}{c^2}}} - 1
                = \frac 1{\sqrt{1 - \sqr{0{,}6}}} - 1
                \approx 0{,}250.
    \end{align*}
}
\solutionspace{150pt}

\tasknumber{3}%
\task{%
    Электрон движется со скоростью $0{,}85\,c$, где $c$~--- скорость света в вакууме.
    Определите его полную энергию (в ответе приведите формулу и укажите численное значение).
}
\answer{%
    \begin{align*}
    E &= \frac{mc^2}{\sqrt{1 - \frac{v^2}{c^2}}}
            \approx \frac{9{,}1 \cdot 10^{-31}\,\text{кг} \cdot \sqr{3 \cdot 10^{8}\,\frac{\text{м}}{\text{с}}}}{\sqrt{1 - 0{,}85^2}}
            \approx 0{,}155 \cdot 10^{-12}\,\text{Дж},
         \\
        E_{\text{кин}} &= \frac{mc^2}{\sqrt{1 - \frac{v^2}{c^2}}} - mc^2
            = mc^2 \cbr{\frac 1{\sqrt{1 - \frac{v^2}{c^2}}} - 1} \approx \\
            &\approx \cbr{9{,}1 \cdot 10^{-31}\,\text{кг} \cdot \sqr{3 \cdot 10^{8}\,\frac{\text{м}}{\text{с}}}}
            \cdot \cbr{\frac 1{\sqrt{1 - 0{,}85^2}} - 1}
            \approx 0{,}074 \cdot 10^{-12}\,\text{Дж},
         \\
        p &= \frac{mv}{\sqrt{1 - \frac{v^2}{c^2}}}
            \approx \frac{9{,}1 \cdot 10^{-31}\,\text{кг} \cdot 0{,}85 \cdot 3 \cdot 10^{8}\,\frac{\text{м}}{\text{с}}}{\sqrt{1 - 0{,}85^2}}
            \approx 0{,}441 \cdot 10^{-21}\,\frac{\text{кг}\cdot\text{м}}{\text{с}}.
    \end{align*}
}
\solutionspace{150pt}

\tasknumber{4}%
\task{%
    При какой скорости движения (в м/с) релятивистское сокращение длины движущегося тела
    составит 50\%?
}
\answer{%
    \begin{align*}
    l_0 &= \frac l{\sqrt{1 - \frac{v^2}{c^2}}}
        \implies 1 - \frac{v^2}{c^2} = \sqr{\frac l{l_0}}
        \implies \frac v c = \sqrt{1 - \sqr{\frac l{l_0}}} \implies
         \\
        \implies v &= c\sqrt{1 - \sqr{\frac l{l_0}}}
        = 3 \cdot 10^{8}\,\frac{\text{м}}{\text{с}} \cdot \sqrt{1 - \sqr{\frac {l_0 - 0{,}50l_0}{l_0}}}
        = 3 \cdot 10^{8}\,\frac{\text{м}}{\text{с}} \cdot \sqrt{1 - \sqr{1 - 0{,}50}} \approx  \\
        &\approx 0{,}866c
        \approx 260 \cdot 10^{6}\,\frac{\text{м}}{\text{с}}
        \approx 935 \cdot 10^{6}\,\frac{\text{км}}{\text{ч}}.
    \end{align*}
}
\solutionspace{150pt}

\tasknumber{5}%
\task{%
    При переходе электрона в атоме с одной стационарной орбиты на другую
    излучается фотон с энергией $2{,}02 \cdot 10^{-19}\,\text{Дж}$.
    Какова длина волны этой линии спектра?
    Постоянная Планка $h = 6{,}626 \cdot 10^{-34}\,\text{Дж}\cdot\text{с}$, скорость света $c = 3 \cdot 10^{8}\,\frac{\text{м}}{\text{с}}$.
}
\answer{%
    $
        E = h\nu = h \frac c\lambda
        \implies \lambda = \frac{hc}E
            = \frac{6{,}626 \cdot 10^{-34}\,\text{Дж}\cdot\text{с} \cdot {3 \cdot 10^{8}\,\frac{\text{м}}{\text{с}}}}{2{,}02 \cdot 10^{-19}\,\text{Дж}}
            = 984{,}06\,\text{нм}.
    $
}
\solutionspace{150pt}

\tasknumber{6}%
\task{%
    Излучение какой длины волны поглотил атом водорода, если полная энергия в атоме увеличилась на $2 \cdot 10^{-19}\,\text{Дж}$?
    Постоянная Планка $h = 6{,}626 \cdot 10^{-34}\,\text{Дж}\cdot\text{с}$, скорость света $c = 3 \cdot 10^{8}\,\frac{\text{м}}{\text{с}}$.
}
\answer{%
    $
        E = h\nu = h \frac c\lambda
        \implies \lambda = \frac{hc}E
            = \frac{6{,}626 \cdot 10^{-34}\,\text{Дж}\cdot\text{с} \cdot {3 \cdot 10^{8}\,\frac{\text{м}}{\text{с}}}}{2 \cdot 10^{-19}\,\text{Дж}}
            = 994\,\text{нм}.
    $
}
\solutionspace{150pt}

\tasknumber{7}%
\task{%
    Сделайте схематичный рисунок энергетических уровней атома водорода
    и отметьте на нём первый (основной) уровень и последующие.
    Сколько различных длин волн может испустить атом водорода,
    находящийся в 4-м возбуждённом состоянии?
    Отметьте все соответствующие переходы на рисунке и укажите,
    при каком переходе (среди отмеченных) энергия излучённого фотона максимальна.
}
\answer{%
    $N = 6{,}0, \text{самая длинная линия}$
}
\solutionspace{150pt}

\tasknumber{8}%
\task{%
    Сколько фотонов испускает за $20\,\text{мин}$ лазер,
    если мощность его излучения $200\,\text{мВт}$?
    Длина волны излучения $600\,\text{нм}$.
    $h = 6{,}626 \cdot 10^{-34}\,\text{Дж}\cdot\text{с}$.
}
\answer{%
    $
        N
            = \frac{E_{\text{общая}}}{E_{\text{одного фотона}}}
            = \frac{Pt}{h\nu} = \frac{Pt}{h \frac c\lambda}
            = \frac{Pt\lambda}{hc}
            = \frac{200\,\text{мВт} \cdot 20\,\text{мин} \cdot 600\,\text{нм}}{6{,}626 \cdot 10^{-34}\,\text{Дж}\cdot\text{с} \cdot 3 \cdot 10^{8}\,\frac{\text{м}}{\text{с}}}
            \approx 7{,}24 \cdot 10^{20}\units{фотонов}
    $
}
\solutionspace{120pt}

\tasknumber{9}%
\task{%
    Какая доля (от начального количества) радиоактивных ядер распадётся через время,
    равное четырём периодам полураспада? Ответ выразить в процентах.
}
\answer{%
    \begin{align*}
    N &= N_0 \cdot 2^{- \frac t{T_{1/2}}} \implies
        \frac N{N_0} = 2^{- \frac t{T_{1/2}}}
        = 2^{-4} \approx 0{,}06 \approx 6\% \\
    N_\text{расп.} &= N_0 - N = N_0 - N_0 \cdot 2^{-\frac t{T_{1/2}}}
        = N_0\cbr{1 - 2^{-\frac t{T_{1/2}}}} \implies
        \frac{N_\text{расп.}}{N_0} = 1 - 2^{-\frac t{T_{1/2}}}
        = 1 - 2^{-4} \approx 0{,}94 \approx 94\%
    \end{align*}
}
\solutionspace{150pt}

\tasknumber{10}%
\task{%
    Сколько процентов ядер радиоактивного железа $\ce{^{59}Fe}$
    останется через $182{,}4\,\text{суток}$, если период его полураспада составляет $45{,}6\,\text{суток}$?
}
\answer{%
    \begin{align*}
    N &= N_0 \cdot 2^{-\frac t{T_{1/2}}}
        = 2^{-\frac{182{,}4\,\text{суток}}{45{,}6\,\text{суток}}}
        \approx 0{,}0625 = 6{,}25\%
    \end{align*}
}
\solutionspace{150pt}

\tasknumber{11}%
\task{%
    За $4\,\text{суток}$ от начального количества ядер радиоизотопа осталась четверть.
    Каков период полураспада этого изотопа (ответ приведите в сутках)?
    Какая ещё доля (также от начального количества) распадётся, если подождать ещё столько же?
}
\answer{%
    \begin{align*}
            N &= N_0 \cdot 2^{-\frac t{T_{1/2}}}
            \implies \frac N{N_0} = 2^{-\frac t{T_{1/2}}}
            \implies \frac 1{4} = 2^{-\frac {4\,\text{суток}}{T_{1/2}}}
            \implies 2 = \frac {4\,\text{суток}}{T_{1/2}}
            \implies T_{1/2} = \frac {4\,\text{суток}}2 \approx 2\,\text{суток}.
         \\
            \delta &= \frac{N(t)}{N_0} - \frac{N(2t)}{N_0}
            = 2^{-\frac t{T_{1/2}}} - 2^{-\frac {2t}{T_{1/2}}}
            = 2^{-\frac t{T_{1/2}}}\cbr{1 - 2^{-\frac t{T_{1/2}}}}
            = \frac 1{4} \cdot \cbr{1-\frac 1{4}} \approx 0{,}188
    \end{align*}
}
\solutionspace{150pt}

\tasknumber{12}%
\task{%
    Энергия связи ядра азота \ce{^{14}_{7}N} равна $115{,}5\,\text{МэВ}$.
    Найти дефект массы этого ядра.
    Ответ выразите в а.е.м.
    и кг.
    Скорость света $c = 2{,}998 \cdot 10^{8}\,\frac{\text{м}}{\text{с}}$, элементарный заряд $e = 1{,}6 \cdot 10^{-19}\,\text{Кл}$.
}
\answer{%
    \begin{align*}
    E_\text{св.} &= \Delta m c^2 \implies \\
    \implies
            \Delta m &= \frac {E_\text{св.}}{c^2} = \frac{115{,}5\,\text{МэВ}}{\sqr{2{,}998 \cdot 10^{8}\,\frac{\text{м}}{\text{с}}}}
            = \frac{115{,}5 \cdot 10^6 \cdot 1{,}6 \cdot 10^{-19}\,\text{Дж}}{\sqr{2{,}998 \cdot 10^{8}\,\frac{\text{м}}{\text{с}}}}
            \approx 0{,}206 \cdot 10^{-27}\,\text{кг} \approx 0{,}1238\,\text{а.е.м.}
    \end{align*}
}

\variantsplitter

\addpersonalvariant{Гагик Аракелян}

\tasknumber{1}%
\task{%
    Для частицы, движущейся с релятивистской скоростью,
    выразите $E_0$ и $p$ через $c$, $E_\text{кин}$ и $v$,
    где $E_\text{кин}$~--- кинетическая энергия частицы,
    а $E_0$, $p$ и $v$~--- её энергия покоя импульс и скорость.
}
\answer{%
    \begin{align*}
    E_\text{кин}, E_0:\quad&E = E_\text{кин} + E_0 = \frac{E_0}{\sqrt{1 - \frac{v^2}{c^2}}} \implies \sqrt{1 - \frac{v^2}{c^2}} = \frac{E_0}{{E_0} + {E_\text{кин}}} \implies v = c\sqrt{1 - \sqr{\frac{E_0}{{E_0} + {E_\text{кин}}}}} \\
    &p = \frac{mv}{\sqrt{1 - \frac{v^2}{c^2}}} = \frac{E_0}{c^2} \cdot \sqrt{1 - \sqr{\frac{E_0}{{E_0} + {E_\text{кин}}}}} \cdot \frac{{E_\text{кин}} + {E_0}}{E_0} = \frac{E_0}{c^2} \cdot \sqrt{\sqr{\frac{{E_\text{кин}} + {E_0}}{E_0}} - 1}.
    \\
    E_\text{кин}, p:\quad&E_\text{кин} = E - E_0 = mc^2\cbr{\frac 1{\sqrt{1 - \frac{v^2}{c^2}}} - 1}, p = \frac{mv}{\sqrt{1 - \frac{v^2}{c^2}}} \implies \frac{E_\text{кин}}{p} = \frac{\frac 1{\sqrt{1 - \frac{v^2}{c^2}}} - 1}{\sqrt{1 - \frac{v^2}{c^2}}} \implies v = \ldots \\
    &E_0 = E - E_\text{кин} = \frac{E_0}{\sqrt{1 - \frac{v^2}{c^2}}} - E_\text{кин} \implies E_0 = \frac{E_\text{кин}}{\frac 1{\sqrt{1 - \frac{v^2}{c^2}}} - 1} = \ldots \\
    E_\text{кин}, v:\quad&E_\text{кин} = E - E_0 = mc^2\cbr{\frac 1{\sqrt{1 - \frac{v^2}{c^2}}} - 1} \implies m = \frac{E_\text{кин}}{c^2\cbr{\frac 1{\sqrt{1 - \frac{v^2}{c^2}}} - 1}} \\
    &E_0 = mc^2 = \frac{E_\text{кин}}{\frac 1{\sqrt{1 - \frac{v^2}{c^2}}} - 1} \\
    &p = \frac{mv}{\sqrt{1 - \frac{v^2}{c^2}}} = \frac{E_\text{кин}}{c^2\cbr{\frac 1{\sqrt{1 - \frac{v^2}{c^2}}} - 1}} \cdot \frac{v}{\sqrt{1 - \frac{v^2}{c^2}}} = \frac{{E_\text{кин}} v}{c^2\cbr{1 - {\sqrt{1 - \frac{v^2}{c^2}}}}} \\
    E_0, p:\quad&E_0 = mc^2, \quad p = \frac{mv}{\sqrt{1 - \frac{v^2}{c^2}}} \implies \frac{E_0}{p} = \frac{c^2}v{\sqrt{1 - \frac{v^2}{c^2}}} = c\sqrt{\frac{c^2}{v^2} - 1} \\
    &\sqr{\frac{E_0}{pc}} = \frac{c^2}{v^2} - 1 \implies \frac{v^2}{c^2} = \frac 1{1 + \frac{E_0^2}{p^2c^2}} \implies v = \frac c{\sqrt{1 + \frac{E_0^2}{p^2c^2}}} \\
    &{E_\text{кин}} = E - E_0 = \sqrt{E_0^2 + p^2c^2} - E_0 \\
    E_0, v:\quad&E_0 = mc^2 \implies m = \frac{E_0}{c^2} \qquad p = \frac{mv}{\sqrt{1 - \frac{v^2}{c^2}}} = \frac{E_0}{c^2} \cdot \frac{v}{\sqrt{1 - \frac{v^2}{c^2}}} \\
    &E_\text{кин}= mc^2\cbr{\frac 1{\sqrt{1 - \frac{v^2}{c^2}}} - 1} = \frac{E_0}{c^2}\cbr{\frac 1{\sqrt{1 - \frac{v^2}{c^2}}} - 1} \\
    p, v:\quad&p = \frac{mv}{\sqrt{1 - \frac{v^2}{c^2}}} \implies m = \frac p v {\sqrt{1 - \frac{v^2}{c^2}}} \implies E_0 = mc^2 =\frac {pc^2} v {\sqrt{1 - \frac{v^2}{c^2}}} \\
    &E_\text{кин} = mc^2\cbr{\frac 1{\sqrt{1 - \frac{v^2}{c^2}}} - 1} = \frac p v {\sqrt{1 - \frac{v^2}{c^2}}}\cbr{\frac 1{\sqrt{1 - \frac{v^2}{c^2}}} - 1} = \frac p v \cbr{1 - {\sqrt{1 - \frac{v^2}{c^2}}}}
    \end{align*}
}
\solutionspace{200pt}

\tasknumber{2}%
\task{%
    Электрон движется со скоростью $0{,}6\,c$, где $c$~--- скорость света в вакууме.
    Каково при этом отношение кинетической энергии частицы $E_\text{кин.}$ к его энергии покоя $E_0$?
}
\answer{%
    \begin{align*}
    E &= \frac{E_0}{\sqrt{1 - \frac{v^2}{c^2}}}
            \implies \frac E{E_0}
                = \frac 1{\sqrt{1 - \frac{v^2}{c^2}}}
                = \frac 1{\sqrt{1 - \sqr{0{,}6}}}
                \approx 1{,}250,
         \\
        E_{\text{кин}} &= E - E_0
            \implies \frac{E_{\text{кин}}}{E_0}
                = \frac E{E_0} - 1
                = \frac 1{\sqrt{1 - \frac{v^2}{c^2}}} - 1
                = \frac 1{\sqrt{1 - \sqr{0{,}6}}} - 1
                \approx 0{,}250.
    \end{align*}
}
\solutionspace{150pt}

\tasknumber{3}%
\task{%
    Электрон движется со скоростью $0{,}65\,c$, где $c$~--- скорость света в вакууме.
    Определите его импульс (в ответе приведите формулу и укажите численное значение).
}
\answer{%
    \begin{align*}
    E &= \frac{mc^2}{\sqrt{1 - \frac{v^2}{c^2}}}
            \approx \frac{9{,}1 \cdot 10^{-31}\,\text{кг} \cdot \sqr{3 \cdot 10^{8}\,\frac{\text{м}}{\text{с}}}}{\sqrt{1 - 0{,}65^2}}
            \approx 0{,}108 \cdot 10^{-12}\,\text{Дж},
         \\
        E_{\text{кин}} &= \frac{mc^2}{\sqrt{1 - \frac{v^2}{c^2}}} - mc^2
            = mc^2 \cbr{\frac 1{\sqrt{1 - \frac{v^2}{c^2}}} - 1} \approx \\
            &\approx \cbr{9{,}1 \cdot 10^{-31}\,\text{кг} \cdot \sqr{3 \cdot 10^{8}\,\frac{\text{м}}{\text{с}}}}
            \cdot \cbr{\frac 1{\sqrt{1 - 0{,}65^2}} - 1}
            \approx 0{,}026 \cdot 10^{-12}\,\text{Дж},
         \\
        p &= \frac{mv}{\sqrt{1 - \frac{v^2}{c^2}}}
            \approx \frac{9{,}1 \cdot 10^{-31}\,\text{кг} \cdot 0{,}65 \cdot 3 \cdot 10^{8}\,\frac{\text{м}}{\text{с}}}{\sqrt{1 - 0{,}65^2}}
            \approx 0{,}234 \cdot 10^{-21}\,\frac{\text{кг}\cdot\text{м}}{\text{с}}.
    \end{align*}
}
\solutionspace{150pt}

\tasknumber{4}%
\task{%
    При какой скорости движения (в м/с) релятивистское сокращение длины движущегося тела
    составит 30\%?
}
\answer{%
    \begin{align*}
    l_0 &= \frac l{\sqrt{1 - \frac{v^2}{c^2}}}
        \implies 1 - \frac{v^2}{c^2} = \sqr{\frac l{l_0}}
        \implies \frac v c = \sqrt{1 - \sqr{\frac l{l_0}}} \implies
         \\
        \implies v &= c\sqrt{1 - \sqr{\frac l{l_0}}}
        = 3 \cdot 10^{8}\,\frac{\text{м}}{\text{с}} \cdot \sqrt{1 - \sqr{\frac {l_0 - 0{,}30l_0}{l_0}}}
        = 3 \cdot 10^{8}\,\frac{\text{м}}{\text{с}} \cdot \sqrt{1 - \sqr{1 - 0{,}30}} \approx  \\
        &\approx 0{,}714c
        \approx 214 \cdot 10^{6}\,\frac{\text{м}}{\text{с}}
        \approx 771 \cdot 10^{6}\,\frac{\text{км}}{\text{ч}}.
    \end{align*}
}
\solutionspace{150pt}

\tasknumber{5}%
\task{%
    При переходе электрона в атоме с одной стационарной орбиты на другую
    излучается фотон с энергией $4{,}04 \cdot 10^{-19}\,\text{Дж}$.
    Какова длина волны этой линии спектра?
    Постоянная Планка $h = 6{,}626 \cdot 10^{-34}\,\text{Дж}\cdot\text{с}$, скорость света $c = 3 \cdot 10^{8}\,\frac{\text{м}}{\text{с}}$.
}
\answer{%
    $
        E = h\nu = h \frac c\lambda
        \implies \lambda = \frac{hc}E
            = \frac{6{,}626 \cdot 10^{-34}\,\text{Дж}\cdot\text{с} \cdot {3 \cdot 10^{8}\,\frac{\text{м}}{\text{с}}}}{4{,}04 \cdot 10^{-19}\,\text{Дж}}
            = 492{,}03\,\text{нм}.
    $
}
\solutionspace{150pt}

\tasknumber{6}%
\task{%
    Излучение какой длины волны поглотил атом водорода, если полная энергия в атоме увеличилась на $4 \cdot 10^{-19}\,\text{Дж}$?
    Постоянная Планка $h = 6{,}626 \cdot 10^{-34}\,\text{Дж}\cdot\text{с}$, скорость света $c = 3 \cdot 10^{8}\,\frac{\text{м}}{\text{с}}$.
}
\answer{%
    $
        E = h\nu = h \frac c\lambda
        \implies \lambda = \frac{hc}E
            = \frac{6{,}626 \cdot 10^{-34}\,\text{Дж}\cdot\text{с} \cdot {3 \cdot 10^{8}\,\frac{\text{м}}{\text{с}}}}{4 \cdot 10^{-19}\,\text{Дж}}
            = 497\,\text{нм}.
    $
}
\solutionspace{150pt}

\tasknumber{7}%
\task{%
    Сделайте схематичный рисунок энергетических уровней атома водорода
    и отметьте на нём первый (основной) уровень и последующие.
    Сколько различных длин волн может испустить атом водорода,
    находящийся в 5-м возбуждённом состоянии?
    Отметьте все соответствующие переходы на рисунке и укажите,
    при каком переходе (среди отмеченных) частота излучённого фотона минимальна.
}
\answer{%
    $N = 10{,}0, \text{самая короткая линия}$
}
\solutionspace{150pt}

\tasknumber{8}%
\task{%
    Сколько фотонов испускает за $10\,\text{мин}$ лазер,
    если мощность его излучения $75\,\text{мВт}$?
    Длина волны излучения $600\,\text{нм}$.
    $h = 6{,}626 \cdot 10^{-34}\,\text{Дж}\cdot\text{с}$.
}
\answer{%
    $
        N
            = \frac{E_{\text{общая}}}{E_{\text{одного фотона}}}
            = \frac{Pt}{h\nu} = \frac{Pt}{h \frac c\lambda}
            = \frac{Pt\lambda}{hc}
            = \frac{75\,\text{мВт} \cdot 10\,\text{мин} \cdot 600\,\text{нм}}{6{,}626 \cdot 10^{-34}\,\text{Дж}\cdot\text{с} \cdot 3 \cdot 10^{8}\,\frac{\text{м}}{\text{с}}}
            \approx 1{,}36 \cdot 10^{20}\units{фотонов}
    $
}
\solutionspace{120pt}

\tasknumber{9}%
\task{%
    Какая доля (от начального количества) радиоактивных ядер останется через время,
    равное четырём периодам полураспада? Ответ выразить в процентах.
}
\answer{%
    \begin{align*}
    N &= N_0 \cdot 2^{- \frac t{T_{1/2}}} \implies
        \frac N{N_0} = 2^{- \frac t{T_{1/2}}}
        = 2^{-4} \approx 0{,}06 \approx 6\% \\
    N_\text{расп.} &= N_0 - N = N_0 - N_0 \cdot 2^{-\frac t{T_{1/2}}}
        = N_0\cbr{1 - 2^{-\frac t{T_{1/2}}}} \implies
        \frac{N_\text{расп.}}{N_0} = 1 - 2^{-\frac t{T_{1/2}}}
        = 1 - 2^{-4} \approx 0{,}94 \approx 94\%
    \end{align*}
}
\solutionspace{150pt}

\tasknumber{10}%
\task{%
    Сколько процентов ядер радиоактивного железа $\ce{^{59}Fe}$
    останется через $182{,}4\,\text{суток}$, если период его полураспада составляет $45{,}6\,\text{суток}$?
}
\answer{%
    \begin{align*}
    N &= N_0 \cdot 2^{-\frac t{T_{1/2}}}
        = 2^{-\frac{182{,}4\,\text{суток}}{45{,}6\,\text{суток}}}
        \approx 0{,}0625 = 6{,}25\%
    \end{align*}
}
\solutionspace{150pt}

\tasknumber{11}%
\task{%
    За $2\,\text{суток}$ от начального количества ядер радиоизотопа осталась четверть.
    Каков период полураспада этого изотопа (ответ приведите в сутках)?
    Какая ещё доля (также от начального количества) распадётся, если подождать ещё столько же?
}
\answer{%
    \begin{align*}
            N &= N_0 \cdot 2^{-\frac t{T_{1/2}}}
            \implies \frac N{N_0} = 2^{-\frac t{T_{1/2}}}
            \implies \frac 1{4} = 2^{-\frac {2\,\text{суток}}{T_{1/2}}}
            \implies 2 = \frac {2\,\text{суток}}{T_{1/2}}
            \implies T_{1/2} = \frac {2\,\text{суток}}2 \approx 1\,\text{суток}.
         \\
            \delta &= \frac{N(t)}{N_0} - \frac{N(2t)}{N_0}
            = 2^{-\frac t{T_{1/2}}} - 2^{-\frac {2t}{T_{1/2}}}
            = 2^{-\frac t{T_{1/2}}}\cbr{1 - 2^{-\frac t{T_{1/2}}}}
            = \frac 1{4} \cdot \cbr{1-\frac 1{4}} \approx 0{,}188
    \end{align*}
}
\solutionspace{150pt}

\tasknumber{12}%
\task{%
    Энергия связи ядра бора \ce{^{11}_{5}B} равна $76{,}2\,\text{МэВ}$.
    Найти дефект массы этого ядра.
    Ответ выразите в а.е.м.
    и кг.
    Скорость света $c = 2{,}998 \cdot 10^{8}\,\frac{\text{м}}{\text{с}}$, элементарный заряд $e = 1{,}6 \cdot 10^{-19}\,\text{Кл}$.
}
\answer{%
    \begin{align*}
    E_\text{св.} &= \Delta m c^2 \implies \\
    \implies
            \Delta m &= \frac {E_\text{св.}}{c^2} = \frac{76{,}2\,\text{МэВ}}{\sqr{2{,}998 \cdot 10^{8}\,\frac{\text{м}}{\text{с}}}}
            = \frac{76{,}2 \cdot 10^6 \cdot 1{,}6 \cdot 10^{-19}\,\text{Дж}}{\sqr{2{,}998 \cdot 10^{8}\,\frac{\text{м}}{\text{с}}}}
            \approx 0{,}1356 \cdot 10^{-27}\,\text{кг} \approx 0{,}0817\,\text{а.е.м.}
    \end{align*}
}

\variantsplitter

\addpersonalvariant{Ирен Аракелян}

\tasknumber{1}%
\task{%
    Для частицы, движущейся с релятивистской скоростью,
    выразите $E_\text{кин}$ и $v$ через $c$, $p$ и $E_0$,
    где $E_\text{кин}$~--- кинетическая энергия частицы,
    а $E_0$, $p$ и $v$~--- её энергия покоя импульс и скорость.
}
\answer{%
    \begin{align*}
    E_\text{кин}, E_0:\quad&E = E_\text{кин} + E_0 = \frac{E_0}{\sqrt{1 - \frac{v^2}{c^2}}} \implies \sqrt{1 - \frac{v^2}{c^2}} = \frac{E_0}{{E_0} + {E_\text{кин}}} \implies v = c\sqrt{1 - \sqr{\frac{E_0}{{E_0} + {E_\text{кин}}}}} \\
    &p = \frac{mv}{\sqrt{1 - \frac{v^2}{c^2}}} = \frac{E_0}{c^2} \cdot \sqrt{1 - \sqr{\frac{E_0}{{E_0} + {E_\text{кин}}}}} \cdot \frac{{E_\text{кин}} + {E_0}}{E_0} = \frac{E_0}{c^2} \cdot \sqrt{\sqr{\frac{{E_\text{кин}} + {E_0}}{E_0}} - 1}.
    \\
    E_\text{кин}, p:\quad&E_\text{кин} = E - E_0 = mc^2\cbr{\frac 1{\sqrt{1 - \frac{v^2}{c^2}}} - 1}, p = \frac{mv}{\sqrt{1 - \frac{v^2}{c^2}}} \implies \frac{E_\text{кин}}{p} = \frac{\frac 1{\sqrt{1 - \frac{v^2}{c^2}}} - 1}{\sqrt{1 - \frac{v^2}{c^2}}} \implies v = \ldots \\
    &E_0 = E - E_\text{кин} = \frac{E_0}{\sqrt{1 - \frac{v^2}{c^2}}} - E_\text{кин} \implies E_0 = \frac{E_\text{кин}}{\frac 1{\sqrt{1 - \frac{v^2}{c^2}}} - 1} = \ldots \\
    E_\text{кин}, v:\quad&E_\text{кин} = E - E_0 = mc^2\cbr{\frac 1{\sqrt{1 - \frac{v^2}{c^2}}} - 1} \implies m = \frac{E_\text{кин}}{c^2\cbr{\frac 1{\sqrt{1 - \frac{v^2}{c^2}}} - 1}} \\
    &E_0 = mc^2 = \frac{E_\text{кин}}{\frac 1{\sqrt{1 - \frac{v^2}{c^2}}} - 1} \\
    &p = \frac{mv}{\sqrt{1 - \frac{v^2}{c^2}}} = \frac{E_\text{кин}}{c^2\cbr{\frac 1{\sqrt{1 - \frac{v^2}{c^2}}} - 1}} \cdot \frac{v}{\sqrt{1 - \frac{v^2}{c^2}}} = \frac{{E_\text{кин}} v}{c^2\cbr{1 - {\sqrt{1 - \frac{v^2}{c^2}}}}} \\
    E_0, p:\quad&E_0 = mc^2, \quad p = \frac{mv}{\sqrt{1 - \frac{v^2}{c^2}}} \implies \frac{E_0}{p} = \frac{c^2}v{\sqrt{1 - \frac{v^2}{c^2}}} = c\sqrt{\frac{c^2}{v^2} - 1} \\
    &\sqr{\frac{E_0}{pc}} = \frac{c^2}{v^2} - 1 \implies \frac{v^2}{c^2} = \frac 1{1 + \frac{E_0^2}{p^2c^2}} \implies v = \frac c{\sqrt{1 + \frac{E_0^2}{p^2c^2}}} \\
    &{E_\text{кин}} = E - E_0 = \sqrt{E_0^2 + p^2c^2} - E_0 \\
    E_0, v:\quad&E_0 = mc^2 \implies m = \frac{E_0}{c^2} \qquad p = \frac{mv}{\sqrt{1 - \frac{v^2}{c^2}}} = \frac{E_0}{c^2} \cdot \frac{v}{\sqrt{1 - \frac{v^2}{c^2}}} \\
    &E_\text{кин}= mc^2\cbr{\frac 1{\sqrt{1 - \frac{v^2}{c^2}}} - 1} = \frac{E_0}{c^2}\cbr{\frac 1{\sqrt{1 - \frac{v^2}{c^2}}} - 1} \\
    p, v:\quad&p = \frac{mv}{\sqrt{1 - \frac{v^2}{c^2}}} \implies m = \frac p v {\sqrt{1 - \frac{v^2}{c^2}}} \implies E_0 = mc^2 =\frac {pc^2} v {\sqrt{1 - \frac{v^2}{c^2}}} \\
    &E_\text{кин} = mc^2\cbr{\frac 1{\sqrt{1 - \frac{v^2}{c^2}}} - 1} = \frac p v {\sqrt{1 - \frac{v^2}{c^2}}}\cbr{\frac 1{\sqrt{1 - \frac{v^2}{c^2}}} - 1} = \frac p v \cbr{1 - {\sqrt{1 - \frac{v^2}{c^2}}}}
    \end{align*}
}
\solutionspace{200pt}

\tasknumber{2}%
\task{%
    Электрон движется со скоростью $0{,}6\,c$, где $c$~--- скорость света в вакууме.
    Каково при этом отношение кинетической энергии частицы $E_\text{кин.}$ к его энергии покоя $E_0$?
}
\answer{%
    \begin{align*}
    E &= \frac{E_0}{\sqrt{1 - \frac{v^2}{c^2}}}
            \implies \frac E{E_0}
                = \frac 1{\sqrt{1 - \frac{v^2}{c^2}}}
                = \frac 1{\sqrt{1 - \sqr{0{,}6}}}
                \approx 1{,}250,
         \\
        E_{\text{кин}} &= E - E_0
            \implies \frac{E_{\text{кин}}}{E_0}
                = \frac E{E_0} - 1
                = \frac 1{\sqrt{1 - \frac{v^2}{c^2}}} - 1
                = \frac 1{\sqrt{1 - \sqr{0{,}6}}} - 1
                \approx 0{,}250.
    \end{align*}
}
\solutionspace{150pt}

\tasknumber{3}%
\task{%
    Электрон движется со скоростью $0{,}65\,c$, где $c$~--- скорость света в вакууме.
    Определите его кинетическую энергию (в ответе приведите формулу и укажите численное значение).
}
\answer{%
    \begin{align*}
    E &= \frac{mc^2}{\sqrt{1 - \frac{v^2}{c^2}}}
            \approx \frac{9{,}1 \cdot 10^{-31}\,\text{кг} \cdot \sqr{3 \cdot 10^{8}\,\frac{\text{м}}{\text{с}}}}{\sqrt{1 - 0{,}65^2}}
            \approx 0{,}108 \cdot 10^{-12}\,\text{Дж},
         \\
        E_{\text{кин}} &= \frac{mc^2}{\sqrt{1 - \frac{v^2}{c^2}}} - mc^2
            = mc^2 \cbr{\frac 1{\sqrt{1 - \frac{v^2}{c^2}}} - 1} \approx \\
            &\approx \cbr{9{,}1 \cdot 10^{-31}\,\text{кг} \cdot \sqr{3 \cdot 10^{8}\,\frac{\text{м}}{\text{с}}}}
            \cdot \cbr{\frac 1{\sqrt{1 - 0{,}65^2}} - 1}
            \approx 0{,}026 \cdot 10^{-12}\,\text{Дж},
         \\
        p &= \frac{mv}{\sqrt{1 - \frac{v^2}{c^2}}}
            \approx \frac{9{,}1 \cdot 10^{-31}\,\text{кг} \cdot 0{,}65 \cdot 3 \cdot 10^{8}\,\frac{\text{м}}{\text{с}}}{\sqrt{1 - 0{,}65^2}}
            \approx 0{,}234 \cdot 10^{-21}\,\frac{\text{кг}\cdot\text{м}}{\text{с}}.
    \end{align*}
}
\solutionspace{150pt}

\tasknumber{4}%
\task{%
    При какой скорости движения (в м/с) релятивистское сокращение длины движущегося тела
    составит 50\%?
}
\answer{%
    \begin{align*}
    l_0 &= \frac l{\sqrt{1 - \frac{v^2}{c^2}}}
        \implies 1 - \frac{v^2}{c^2} = \sqr{\frac l{l_0}}
        \implies \frac v c = \sqrt{1 - \sqr{\frac l{l_0}}} \implies
         \\
        \implies v &= c\sqrt{1 - \sqr{\frac l{l_0}}}
        = 3 \cdot 10^{8}\,\frac{\text{м}}{\text{с}} \cdot \sqrt{1 - \sqr{\frac {l_0 - 0{,}50l_0}{l_0}}}
        = 3 \cdot 10^{8}\,\frac{\text{м}}{\text{с}} \cdot \sqrt{1 - \sqr{1 - 0{,}50}} \approx  \\
        &\approx 0{,}866c
        \approx 260 \cdot 10^{6}\,\frac{\text{м}}{\text{с}}
        \approx 935 \cdot 10^{6}\,\frac{\text{км}}{\text{ч}}.
    \end{align*}
}
\solutionspace{150pt}

\tasknumber{5}%
\task{%
    При переходе электрона в атоме с одной стационарной орбиты на другую
    излучается фотон с энергией $1{,}01 \cdot 10^{-19}\,\text{Дж}$.
    Какова длина волны этой линии спектра?
    Постоянная Планка $h = 6{,}626 \cdot 10^{-34}\,\text{Дж}\cdot\text{с}$, скорость света $c = 3 \cdot 10^{8}\,\frac{\text{м}}{\text{с}}$.
}
\answer{%
    $
        E = h\nu = h \frac c\lambda
        \implies \lambda = \frac{hc}E
            = \frac{6{,}626 \cdot 10^{-34}\,\text{Дж}\cdot\text{с} \cdot {3 \cdot 10^{8}\,\frac{\text{м}}{\text{с}}}}{1{,}01 \cdot 10^{-19}\,\text{Дж}}
            = 1968{,}1\,\text{нм}.
    $
}
\solutionspace{150pt}

\tasknumber{6}%
\task{%
    Излучение какой длины волны поглотил атом водорода, если полная энергия в атоме увеличилась на $4 \cdot 10^{-19}\,\text{Дж}$?
    Постоянная Планка $h = 6{,}626 \cdot 10^{-34}\,\text{Дж}\cdot\text{с}$, скорость света $c = 3 \cdot 10^{8}\,\frac{\text{м}}{\text{с}}$.
}
\answer{%
    $
        E = h\nu = h \frac c\lambda
        \implies \lambda = \frac{hc}E
            = \frac{6{,}626 \cdot 10^{-34}\,\text{Дж}\cdot\text{с} \cdot {3 \cdot 10^{8}\,\frac{\text{м}}{\text{с}}}}{4 \cdot 10^{-19}\,\text{Дж}}
            = 497\,\text{нм}.
    $
}
\solutionspace{150pt}

\tasknumber{7}%
\task{%
    Сделайте схематичный рисунок энергетических уровней атома водорода
    и отметьте на нём первый (основной) уровень и последующие.
    Сколько различных длин волн может испустить атом водорода,
    находящийся в 5-м возбуждённом состоянии?
    Отметьте все соответствующие переходы на рисунке и укажите,
    при каком переходе (среди отмеченных) частота излучённого фотона максимальна.
}
\answer{%
    $N = 10{,}0, \text{самая длинная линия}$
}
\solutionspace{150pt}

\tasknumber{8}%
\task{%
    Сколько фотонов испускает за $10\,\text{мин}$ лазер,
    если мощность его излучения $15\,\text{мВт}$?
    Длина волны излучения $750\,\text{нм}$.
    $h = 6{,}626 \cdot 10^{-34}\,\text{Дж}\cdot\text{с}$.
}
\answer{%
    $
        N
            = \frac{E_{\text{общая}}}{E_{\text{одного фотона}}}
            = \frac{Pt}{h\nu} = \frac{Pt}{h \frac c\lambda}
            = \frac{Pt\lambda}{hc}
            = \frac{15\,\text{мВт} \cdot 10\,\text{мин} \cdot 750\,\text{нм}}{6{,}626 \cdot 10^{-34}\,\text{Дж}\cdot\text{с} \cdot 3 \cdot 10^{8}\,\frac{\text{м}}{\text{с}}}
            \approx 0{,}34 \cdot 10^{20}\units{фотонов}
    $
}
\solutionspace{120pt}

\tasknumber{9}%
\task{%
    Какая доля (от начального количества) радиоактивных ядер останется через время,
    равное двум периодам полураспада? Ответ выразить в процентах.
}
\answer{%
    \begin{align*}
    N &= N_0 \cdot 2^{- \frac t{T_{1/2}}} \implies
        \frac N{N_0} = 2^{- \frac t{T_{1/2}}}
        = 2^{-2} \approx 0{,}25 \approx 25\% \\
    N_\text{расп.} &= N_0 - N = N_0 - N_0 \cdot 2^{-\frac t{T_{1/2}}}
        = N_0\cbr{1 - 2^{-\frac t{T_{1/2}}}} \implies
        \frac{N_\text{расп.}}{N_0} = 1 - 2^{-\frac t{T_{1/2}}}
        = 1 - 2^{-2} \approx 0{,}75 \approx 75\%
    \end{align*}
}
\solutionspace{150pt}

\tasknumber{10}%
\task{%
    Сколько процентов ядер радиоактивного железа $\ce{^{59}Fe}$
    останется через $182{,}4\,\text{суток}$, если период его полураспада составляет $45{,}6\,\text{суток}$?
}
\answer{%
    \begin{align*}
    N &= N_0 \cdot 2^{-\frac t{T_{1/2}}}
        = 2^{-\frac{182{,}4\,\text{суток}}{45{,}6\,\text{суток}}}
        \approx 0{,}0625 = 6{,}25\%
    \end{align*}
}
\solutionspace{150pt}

\tasknumber{11}%
\task{%
    За $4\,\text{суток}$ от начального количества ядер радиоизотопа осталась четверть.
    Каков период полураспада этого изотопа (ответ приведите в сутках)?
    Какая ещё доля (также от начального количества) распадётся, если подождать ещё столько же?
}
\answer{%
    \begin{align*}
            N &= N_0 \cdot 2^{-\frac t{T_{1/2}}}
            \implies \frac N{N_0} = 2^{-\frac t{T_{1/2}}}
            \implies \frac 1{4} = 2^{-\frac {4\,\text{суток}}{T_{1/2}}}
            \implies 2 = \frac {4\,\text{суток}}{T_{1/2}}
            \implies T_{1/2} = \frac {4\,\text{суток}}2 \approx 2\,\text{суток}.
         \\
            \delta &= \frac{N(t)}{N_0} - \frac{N(2t)}{N_0}
            = 2^{-\frac t{T_{1/2}}} - 2^{-\frac {2t}{T_{1/2}}}
            = 2^{-\frac t{T_{1/2}}}\cbr{1 - 2^{-\frac t{T_{1/2}}}}
            = \frac 1{4} \cdot \cbr{1-\frac 1{4}} \approx 0{,}188
    \end{align*}
}
\solutionspace{150pt}

\tasknumber{12}%
\task{%
    Энергия связи ядра азота \ce{^{14}_{7}N} равна $104{,}7\,\text{МэВ}$.
    Найти дефект массы этого ядра.
    Ответ выразите в а.е.м.
    и кг.
    Скорость света $c = 2{,}998 \cdot 10^{8}\,\frac{\text{м}}{\text{с}}$, элементарный заряд $e = 1{,}6 \cdot 10^{-19}\,\text{Кл}$.
}
\answer{%
    \begin{align*}
    E_\text{св.} &= \Delta m c^2 \implies \\
    \implies
            \Delta m &= \frac {E_\text{св.}}{c^2} = \frac{104{,}7\,\text{МэВ}}{\sqr{2{,}998 \cdot 10^{8}\,\frac{\text{м}}{\text{с}}}}
            = \frac{104{,}7 \cdot 10^6 \cdot 1{,}6 \cdot 10^{-19}\,\text{Дж}}{\sqr{2{,}998 \cdot 10^{8}\,\frac{\text{м}}{\text{с}}}}
            \approx 0{,}1864 \cdot 10^{-27}\,\text{кг} \approx 0{,}1122\,\text{а.е.м.}
    \end{align*}
}

\variantsplitter

\addpersonalvariant{Сабина Асадуллаева}

\tasknumber{1}%
\task{%
    Для частицы, движущейся с релятивистской скоростью,
    выразите $v$ и $p$ через $c$, $E_0$ и $E_\text{кин}$,
    где $E_\text{кин}$~--- кинетическая энергия частицы,
    а $E_0$, $p$ и $v$~--- её энергия покоя импульс и скорость.
}
\answer{%
    \begin{align*}
    E_\text{кин}, E_0:\quad&E = E_\text{кин} + E_0 = \frac{E_0}{\sqrt{1 - \frac{v^2}{c^2}}} \implies \sqrt{1 - \frac{v^2}{c^2}} = \frac{E_0}{{E_0} + {E_\text{кин}}} \implies v = c\sqrt{1 - \sqr{\frac{E_0}{{E_0} + {E_\text{кин}}}}} \\
    &p = \frac{mv}{\sqrt{1 - \frac{v^2}{c^2}}} = \frac{E_0}{c^2} \cdot \sqrt{1 - \sqr{\frac{E_0}{{E_0} + {E_\text{кин}}}}} \cdot \frac{{E_\text{кин}} + {E_0}}{E_0} = \frac{E_0}{c^2} \cdot \sqrt{\sqr{\frac{{E_\text{кин}} + {E_0}}{E_0}} - 1}.
    \\
    E_\text{кин}, p:\quad&E_\text{кин} = E - E_0 = mc^2\cbr{\frac 1{\sqrt{1 - \frac{v^2}{c^2}}} - 1}, p = \frac{mv}{\sqrt{1 - \frac{v^2}{c^2}}} \implies \frac{E_\text{кин}}{p} = \frac{\frac 1{\sqrt{1 - \frac{v^2}{c^2}}} - 1}{\sqrt{1 - \frac{v^2}{c^2}}} \implies v = \ldots \\
    &E_0 = E - E_\text{кин} = \frac{E_0}{\sqrt{1 - \frac{v^2}{c^2}}} - E_\text{кин} \implies E_0 = \frac{E_\text{кин}}{\frac 1{\sqrt{1 - \frac{v^2}{c^2}}} - 1} = \ldots \\
    E_\text{кин}, v:\quad&E_\text{кин} = E - E_0 = mc^2\cbr{\frac 1{\sqrt{1 - \frac{v^2}{c^2}}} - 1} \implies m = \frac{E_\text{кин}}{c^2\cbr{\frac 1{\sqrt{1 - \frac{v^2}{c^2}}} - 1}} \\
    &E_0 = mc^2 = \frac{E_\text{кин}}{\frac 1{\sqrt{1 - \frac{v^2}{c^2}}} - 1} \\
    &p = \frac{mv}{\sqrt{1 - \frac{v^2}{c^2}}} = \frac{E_\text{кин}}{c^2\cbr{\frac 1{\sqrt{1 - \frac{v^2}{c^2}}} - 1}} \cdot \frac{v}{\sqrt{1 - \frac{v^2}{c^2}}} = \frac{{E_\text{кин}} v}{c^2\cbr{1 - {\sqrt{1 - \frac{v^2}{c^2}}}}} \\
    E_0, p:\quad&E_0 = mc^2, \quad p = \frac{mv}{\sqrt{1 - \frac{v^2}{c^2}}} \implies \frac{E_0}{p} = \frac{c^2}v{\sqrt{1 - \frac{v^2}{c^2}}} = c\sqrt{\frac{c^2}{v^2} - 1} \\
    &\sqr{\frac{E_0}{pc}} = \frac{c^2}{v^2} - 1 \implies \frac{v^2}{c^2} = \frac 1{1 + \frac{E_0^2}{p^2c^2}} \implies v = \frac c{\sqrt{1 + \frac{E_0^2}{p^2c^2}}} \\
    &{E_\text{кин}} = E - E_0 = \sqrt{E_0^2 + p^2c^2} - E_0 \\
    E_0, v:\quad&E_0 = mc^2 \implies m = \frac{E_0}{c^2} \qquad p = \frac{mv}{\sqrt{1 - \frac{v^2}{c^2}}} = \frac{E_0}{c^2} \cdot \frac{v}{\sqrt{1 - \frac{v^2}{c^2}}} \\
    &E_\text{кин}= mc^2\cbr{\frac 1{\sqrt{1 - \frac{v^2}{c^2}}} - 1} = \frac{E_0}{c^2}\cbr{\frac 1{\sqrt{1 - \frac{v^2}{c^2}}} - 1} \\
    p, v:\quad&p = \frac{mv}{\sqrt{1 - \frac{v^2}{c^2}}} \implies m = \frac p v {\sqrt{1 - \frac{v^2}{c^2}}} \implies E_0 = mc^2 =\frac {pc^2} v {\sqrt{1 - \frac{v^2}{c^2}}} \\
    &E_\text{кин} = mc^2\cbr{\frac 1{\sqrt{1 - \frac{v^2}{c^2}}} - 1} = \frac p v {\sqrt{1 - \frac{v^2}{c^2}}}\cbr{\frac 1{\sqrt{1 - \frac{v^2}{c^2}}} - 1} = \frac p v \cbr{1 - {\sqrt{1 - \frac{v^2}{c^2}}}}
    \end{align*}
}
\solutionspace{200pt}

\tasknumber{2}%
\task{%
    Электрон движется со скоростью $0{,}7\,c$, где $c$~--- скорость света в вакууме.
    Каково при этом отношение кинетической энергии частицы $E_\text{кин.}$ к его энергии покоя $E_0$?
}
\answer{%
    \begin{align*}
    E &= \frac{E_0}{\sqrt{1 - \frac{v^2}{c^2}}}
            \implies \frac E{E_0}
                = \frac 1{\sqrt{1 - \frac{v^2}{c^2}}}
                = \frac 1{\sqrt{1 - \sqr{0{,}7}}}
                \approx 1{,}400,
         \\
        E_{\text{кин}} &= E - E_0
            \implies \frac{E_{\text{кин}}}{E_0}
                = \frac E{E_0} - 1
                = \frac 1{\sqrt{1 - \frac{v^2}{c^2}}} - 1
                = \frac 1{\sqrt{1 - \sqr{0{,}7}}} - 1
                \approx 0{,}400.
    \end{align*}
}
\solutionspace{150pt}

\tasknumber{3}%
\task{%
    Протон движется со скоростью $0{,}75\,c$, где $c$~--- скорость света в вакууме.
    Определите его полную энергию (в ответе приведите формулу и укажите численное значение).
}
\answer{%
    \begin{align*}
    E &= \frac{mc^2}{\sqrt{1 - \frac{v^2}{c^2}}}
            \approx \frac{1{,}673 \cdot 10^{-27}\,\text{кг} \cdot \sqr{3 \cdot 10^{8}\,\frac{\text{м}}{\text{с}}}}{\sqrt{1 - 0{,}75^2}}
            \approx 227{,}589 \cdot 10^{-12}\,\text{Дж},
         \\
        E_{\text{кин}} &= \frac{mc^2}{\sqrt{1 - \frac{v^2}{c^2}}} - mc^2
            = mc^2 \cbr{\frac 1{\sqrt{1 - \frac{v^2}{c^2}}} - 1} \approx \\
            &\approx \cbr{1{,}673 \cdot 10^{-27}\,\text{кг} \cdot \sqr{3 \cdot 10^{8}\,\frac{\text{м}}{\text{с}}}}
            \cdot \cbr{\frac 1{\sqrt{1 - 0{,}75^2}} - 1}
            \approx 77{,}053 \cdot 10^{-12}\,\text{Дж},
         \\
        p &= \frac{mv}{\sqrt{1 - \frac{v^2}{c^2}}}
            \approx \frac{1{,}673 \cdot 10^{-27}\,\text{кг} \cdot 0{,}75 \cdot 3 \cdot 10^{8}\,\frac{\text{м}}{\text{с}}}{\sqrt{1 - 0{,}75^2}}
            \approx 568{,}972 \cdot 10^{-21}\,\frac{\text{кг}\cdot\text{м}}{\text{с}}.
    \end{align*}
}
\solutionspace{150pt}

\tasknumber{4}%
\task{%
    При какой скорости движения (в долях скорости света) релятивистское сокращение длины движущегося тела
    составит 30\%?
}
\answer{%
    \begin{align*}
    l_0 &= \frac l{\sqrt{1 - \frac{v^2}{c^2}}}
        \implies 1 - \frac{v^2}{c^2} = \sqr{\frac l{l_0}}
        \implies \frac v c = \sqrt{1 - \sqr{\frac l{l_0}}} \implies
         \\
        \implies v &= c\sqrt{1 - \sqr{\frac l{l_0}}}
        = 3 \cdot 10^{8}\,\frac{\text{м}}{\text{с}} \cdot \sqrt{1 - \sqr{\frac {l_0 - 0{,}30l_0}{l_0}}}
        = 3 \cdot 10^{8}\,\frac{\text{м}}{\text{с}} \cdot \sqrt{1 - \sqr{1 - 0{,}30}} \approx  \\
        &\approx 0{,}714c
        \approx 214 \cdot 10^{6}\,\frac{\text{м}}{\text{с}}
        \approx 771 \cdot 10^{6}\,\frac{\text{км}}{\text{ч}}.
    \end{align*}
}
\solutionspace{150pt}

\tasknumber{5}%
\task{%
    При переходе электрона в атоме с одной стационарной орбиты на другую
    излучается фотон с энергией $7{,}07 \cdot 10^{-19}\,\text{Дж}$.
    Какова длина волны этой линии спектра?
    Постоянная Планка $h = 6{,}626 \cdot 10^{-34}\,\text{Дж}\cdot\text{с}$, скорость света $c = 3 \cdot 10^{8}\,\frac{\text{м}}{\text{с}}$.
}
\answer{%
    $
        E = h\nu = h \frac c\lambda
        \implies \lambda = \frac{hc}E
            = \frac{6{,}626 \cdot 10^{-34}\,\text{Дж}\cdot\text{с} \cdot {3 \cdot 10^{8}\,\frac{\text{м}}{\text{с}}}}{7{,}07 \cdot 10^{-19}\,\text{Дж}}
            = 281{,}16\,\text{нм}.
    $
}
\solutionspace{150pt}

\tasknumber{6}%
\task{%
    Излучение какой длины волны поглотил атом водорода, если полная энергия в атоме увеличилась на $3 \cdot 10^{-19}\,\text{Дж}$?
    Постоянная Планка $h = 6{,}626 \cdot 10^{-34}\,\text{Дж}\cdot\text{с}$, скорость света $c = 3 \cdot 10^{8}\,\frac{\text{м}}{\text{с}}$.
}
\answer{%
    $
        E = h\nu = h \frac c\lambda
        \implies \lambda = \frac{hc}E
            = \frac{6{,}626 \cdot 10^{-34}\,\text{Дж}\cdot\text{с} \cdot {3 \cdot 10^{8}\,\frac{\text{м}}{\text{с}}}}{3 \cdot 10^{-19}\,\text{Дж}}
            = 663\,\text{нм}.
    $
}
\solutionspace{150pt}

\tasknumber{7}%
\task{%
    Сделайте схематичный рисунок энергетических уровней атома водорода
    и отметьте на нём первый (основной) уровень и последующие.
    Сколько различных длин волн может испустить атом водорода,
    находящийся в 3-м возбуждённом состоянии?
    Отметьте все соответствующие переходы на рисунке и укажите,
    при каком переходе (среди отмеченных) длина волны излучённого фотона минимальна.
}
\answer{%
    $N = 3{,}0, \text{самая длинная линия}$
}
\solutionspace{150pt}

\tasknumber{8}%
\task{%
    Сколько фотонов испускает за $5\,\text{мин}$ лазер,
    если мощность его излучения $15\,\text{мВт}$?
    Длина волны излучения $600\,\text{нм}$.
    $h = 6{,}626 \cdot 10^{-34}\,\text{Дж}\cdot\text{с}$.
}
\answer{%
    $
        N
            = \frac{E_{\text{общая}}}{E_{\text{одного фотона}}}
            = \frac{Pt}{h\nu} = \frac{Pt}{h \frac c\lambda}
            = \frac{Pt\lambda}{hc}
            = \frac{15\,\text{мВт} \cdot 5\,\text{мин} \cdot 600\,\text{нм}}{6{,}626 \cdot 10^{-34}\,\text{Дж}\cdot\text{с} \cdot 3 \cdot 10^{8}\,\frac{\text{м}}{\text{с}}}
            \approx 0{,}14 \cdot 10^{20}\units{фотонов}
    $
}
\solutionspace{120pt}

\tasknumber{9}%
\task{%
    Какая доля (от начального количества) радиоактивных ядер распадётся через время,
    равное четырём периодам полураспада? Ответ выразить в процентах.
}
\answer{%
    \begin{align*}
    N &= N_0 \cdot 2^{- \frac t{T_{1/2}}} \implies
        \frac N{N_0} = 2^{- \frac t{T_{1/2}}}
        = 2^{-4} \approx 0{,}06 \approx 6\% \\
    N_\text{расп.} &= N_0 - N = N_0 - N_0 \cdot 2^{-\frac t{T_{1/2}}}
        = N_0\cbr{1 - 2^{-\frac t{T_{1/2}}}} \implies
        \frac{N_\text{расп.}}{N_0} = 1 - 2^{-\frac t{T_{1/2}}}
        = 1 - 2^{-4} \approx 0{,}94 \approx 94\%
    \end{align*}
}
\solutionspace{150pt}

\tasknumber{10}%
\task{%
    Сколько процентов ядер радиоактивного железа $\ce{^{59}Fe}$
    останется через $182{,}4\,\text{суток}$, если период его полураспада составляет $45{,}6\,\text{суток}$?
}
\answer{%
    \begin{align*}
    N &= N_0 \cdot 2^{-\frac t{T_{1/2}}}
        = 2^{-\frac{182{,}4\,\text{суток}}{45{,}6\,\text{суток}}}
        \approx 0{,}0625 = 6{,}25\%
    \end{align*}
}
\solutionspace{150pt}

\tasknumber{11}%
\task{%
    За $3\,\text{суток}$ от начального количества ядер радиоизотопа осталась одна восьмая.
    Каков период полураспада этого изотопа (ответ приведите в сутках)?
    Какая ещё доля (также от начального количества) распадётся, если подождать ещё столько же?
}
\answer{%
    \begin{align*}
            N &= N_0 \cdot 2^{-\frac t{T_{1/2}}}
            \implies \frac N{N_0} = 2^{-\frac t{T_{1/2}}}
            \implies \frac 1{8} = 2^{-\frac {3\,\text{суток}}{T_{1/2}}}
            \implies 3 = \frac {3\,\text{суток}}{T_{1/2}}
            \implies T_{1/2} = \frac {3\,\text{суток}}3 \approx 1\,\text{суток}.
         \\
            \delta &= \frac{N(t)}{N_0} - \frac{N(2t)}{N_0}
            = 2^{-\frac t{T_{1/2}}} - 2^{-\frac {2t}{T_{1/2}}}
            = 2^{-\frac t{T_{1/2}}}\cbr{1 - 2^{-\frac t{T_{1/2}}}}
            = \frac 1{8} \cdot \cbr{1-\frac 1{8}} \approx 0{,}109
    \end{align*}
}
\solutionspace{150pt}

\tasknumber{12}%
\task{%
    Энергия связи ядра кислорода \ce{^{17}_{8}O} равна $131{,}8\,\text{МэВ}$.
    Найти дефект массы этого ядра.
    Ответ выразите в а.е.м.
    и кг.
    Скорость света $c = 2{,}998 \cdot 10^{8}\,\frac{\text{м}}{\text{с}}$, элементарный заряд $e = 1{,}6 \cdot 10^{-19}\,\text{Кл}$.
}
\answer{%
    \begin{align*}
    E_\text{св.} &= \Delta m c^2 \implies \\
    \implies
            \Delta m &= \frac {E_\text{св.}}{c^2} = \frac{131{,}8\,\text{МэВ}}{\sqr{2{,}998 \cdot 10^{8}\,\frac{\text{м}}{\text{с}}}}
            = \frac{131{,}8 \cdot 10^6 \cdot 1{,}6 \cdot 10^{-19}\,\text{Дж}}{\sqr{2{,}998 \cdot 10^{8}\,\frac{\text{м}}{\text{с}}}}
            \approx 0{,}235 \cdot 10^{-27}\,\text{кг} \approx 0{,}1413\,\text{а.е.м.}
    \end{align*}
}

\variantsplitter

\addpersonalvariant{Вероника Битерякова}

\tasknumber{1}%
\task{%
    Для частицы, движущейся с релятивистской скоростью,
    выразите $p$ и $E_\text{кин}$ через $c$, $v$ и $E_0$,
    где $E_\text{кин}$~--- кинетическая энергия частицы,
    а $E_0$, $p$ и $v$~--- её энергия покоя импульс и скорость.
}
\answer{%
    \begin{align*}
    E_\text{кин}, E_0:\quad&E = E_\text{кин} + E_0 = \frac{E_0}{\sqrt{1 - \frac{v^2}{c^2}}} \implies \sqrt{1 - \frac{v^2}{c^2}} = \frac{E_0}{{E_0} + {E_\text{кин}}} \implies v = c\sqrt{1 - \sqr{\frac{E_0}{{E_0} + {E_\text{кин}}}}} \\
    &p = \frac{mv}{\sqrt{1 - \frac{v^2}{c^2}}} = \frac{E_0}{c^2} \cdot \sqrt{1 - \sqr{\frac{E_0}{{E_0} + {E_\text{кин}}}}} \cdot \frac{{E_\text{кин}} + {E_0}}{E_0} = \frac{E_0}{c^2} \cdot \sqrt{\sqr{\frac{{E_\text{кин}} + {E_0}}{E_0}} - 1}.
    \\
    E_\text{кин}, p:\quad&E_\text{кин} = E - E_0 = mc^2\cbr{\frac 1{\sqrt{1 - \frac{v^2}{c^2}}} - 1}, p = \frac{mv}{\sqrt{1 - \frac{v^2}{c^2}}} \implies \frac{E_\text{кин}}{p} = \frac{\frac 1{\sqrt{1 - \frac{v^2}{c^2}}} - 1}{\sqrt{1 - \frac{v^2}{c^2}}} \implies v = \ldots \\
    &E_0 = E - E_\text{кин} = \frac{E_0}{\sqrt{1 - \frac{v^2}{c^2}}} - E_\text{кин} \implies E_0 = \frac{E_\text{кин}}{\frac 1{\sqrt{1 - \frac{v^2}{c^2}}} - 1} = \ldots \\
    E_\text{кин}, v:\quad&E_\text{кин} = E - E_0 = mc^2\cbr{\frac 1{\sqrt{1 - \frac{v^2}{c^2}}} - 1} \implies m = \frac{E_\text{кин}}{c^2\cbr{\frac 1{\sqrt{1 - \frac{v^2}{c^2}}} - 1}} \\
    &E_0 = mc^2 = \frac{E_\text{кин}}{\frac 1{\sqrt{1 - \frac{v^2}{c^2}}} - 1} \\
    &p = \frac{mv}{\sqrt{1 - \frac{v^2}{c^2}}} = \frac{E_\text{кин}}{c^2\cbr{\frac 1{\sqrt{1 - \frac{v^2}{c^2}}} - 1}} \cdot \frac{v}{\sqrt{1 - \frac{v^2}{c^2}}} = \frac{{E_\text{кин}} v}{c^2\cbr{1 - {\sqrt{1 - \frac{v^2}{c^2}}}}} \\
    E_0, p:\quad&E_0 = mc^2, \quad p = \frac{mv}{\sqrt{1 - \frac{v^2}{c^2}}} \implies \frac{E_0}{p} = \frac{c^2}v{\sqrt{1 - \frac{v^2}{c^2}}} = c\sqrt{\frac{c^2}{v^2} - 1} \\
    &\sqr{\frac{E_0}{pc}} = \frac{c^2}{v^2} - 1 \implies \frac{v^2}{c^2} = \frac 1{1 + \frac{E_0^2}{p^2c^2}} \implies v = \frac c{\sqrt{1 + \frac{E_0^2}{p^2c^2}}} \\
    &{E_\text{кин}} = E - E_0 = \sqrt{E_0^2 + p^2c^2} - E_0 \\
    E_0, v:\quad&E_0 = mc^2 \implies m = \frac{E_0}{c^2} \qquad p = \frac{mv}{\sqrt{1 - \frac{v^2}{c^2}}} = \frac{E_0}{c^2} \cdot \frac{v}{\sqrt{1 - \frac{v^2}{c^2}}} \\
    &E_\text{кин}= mc^2\cbr{\frac 1{\sqrt{1 - \frac{v^2}{c^2}}} - 1} = \frac{E_0}{c^2}\cbr{\frac 1{\sqrt{1 - \frac{v^2}{c^2}}} - 1} \\
    p, v:\quad&p = \frac{mv}{\sqrt{1 - \frac{v^2}{c^2}}} \implies m = \frac p v {\sqrt{1 - \frac{v^2}{c^2}}} \implies E_0 = mc^2 =\frac {pc^2} v {\sqrt{1 - \frac{v^2}{c^2}}} \\
    &E_\text{кин} = mc^2\cbr{\frac 1{\sqrt{1 - \frac{v^2}{c^2}}} - 1} = \frac p v {\sqrt{1 - \frac{v^2}{c^2}}}\cbr{\frac 1{\sqrt{1 - \frac{v^2}{c^2}}} - 1} = \frac p v \cbr{1 - {\sqrt{1 - \frac{v^2}{c^2}}}}
    \end{align*}
}
\solutionspace{200pt}

\tasknumber{2}%
\task{%
    Электрон движется со скоростью $0{,}7\,c$, где $c$~--- скорость света в вакууме.
    Каково при этом отношение кинетической энергии частицы $E_\text{кин.}$ к его энергии покоя $E_0$?
}
\answer{%
    \begin{align*}
    E &= \frac{E_0}{\sqrt{1 - \frac{v^2}{c^2}}}
            \implies \frac E{E_0}
                = \frac 1{\sqrt{1 - \frac{v^2}{c^2}}}
                = \frac 1{\sqrt{1 - \sqr{0{,}7}}}
                \approx 1{,}400,
         \\
        E_{\text{кин}} &= E - E_0
            \implies \frac{E_{\text{кин}}}{E_0}
                = \frac E{E_0} - 1
                = \frac 1{\sqrt{1 - \frac{v^2}{c^2}}} - 1
                = \frac 1{\sqrt{1 - \sqr{0{,}7}}} - 1
                \approx 0{,}400.
    \end{align*}
}
\solutionspace{150pt}

\tasknumber{3}%
\task{%
    Протон движется со скоростью $0{,}75\,c$, где $c$~--- скорость света в вакууме.
    Определите его полную энергию (в ответе приведите формулу и укажите численное значение).
}
\answer{%
    \begin{align*}
    E &= \frac{mc^2}{\sqrt{1 - \frac{v^2}{c^2}}}
            \approx \frac{1{,}673 \cdot 10^{-27}\,\text{кг} \cdot \sqr{3 \cdot 10^{8}\,\frac{\text{м}}{\text{с}}}}{\sqrt{1 - 0{,}75^2}}
            \approx 227{,}589 \cdot 10^{-12}\,\text{Дж},
         \\
        E_{\text{кин}} &= \frac{mc^2}{\sqrt{1 - \frac{v^2}{c^2}}} - mc^2
            = mc^2 \cbr{\frac 1{\sqrt{1 - \frac{v^2}{c^2}}} - 1} \approx \\
            &\approx \cbr{1{,}673 \cdot 10^{-27}\,\text{кг} \cdot \sqr{3 \cdot 10^{8}\,\frac{\text{м}}{\text{с}}}}
            \cdot \cbr{\frac 1{\sqrt{1 - 0{,}75^2}} - 1}
            \approx 77{,}053 \cdot 10^{-12}\,\text{Дж},
         \\
        p &= \frac{mv}{\sqrt{1 - \frac{v^2}{c^2}}}
            \approx \frac{1{,}673 \cdot 10^{-27}\,\text{кг} \cdot 0{,}75 \cdot 3 \cdot 10^{8}\,\frac{\text{м}}{\text{с}}}{\sqrt{1 - 0{,}75^2}}
            \approx 568{,}972 \cdot 10^{-21}\,\frac{\text{кг}\cdot\text{м}}{\text{с}}.
    \end{align*}
}
\solutionspace{150pt}

\tasknumber{4}%
\task{%
    При какой скорости движения (в м/с) релятивистское сокращение длины движущегося тела
    составит 30\%?
}
\answer{%
    \begin{align*}
    l_0 &= \frac l{\sqrt{1 - \frac{v^2}{c^2}}}
        \implies 1 - \frac{v^2}{c^2} = \sqr{\frac l{l_0}}
        \implies \frac v c = \sqrt{1 - \sqr{\frac l{l_0}}} \implies
         \\
        \implies v &= c\sqrt{1 - \sqr{\frac l{l_0}}}
        = 3 \cdot 10^{8}\,\frac{\text{м}}{\text{с}} \cdot \sqrt{1 - \sqr{\frac {l_0 - 0{,}30l_0}{l_0}}}
        = 3 \cdot 10^{8}\,\frac{\text{м}}{\text{с}} \cdot \sqrt{1 - \sqr{1 - 0{,}30}} \approx  \\
        &\approx 0{,}714c
        \approx 214 \cdot 10^{6}\,\frac{\text{м}}{\text{с}}
        \approx 771 \cdot 10^{6}\,\frac{\text{км}}{\text{ч}}.
    \end{align*}
}
\solutionspace{150pt}

\tasknumber{5}%
\task{%
    При переходе электрона в атоме с одной стационарной орбиты на другую
    излучается фотон с энергией $4{,}04 \cdot 10^{-19}\,\text{Дж}$.
    Какова длина волны этой линии спектра?
    Постоянная Планка $h = 6{,}626 \cdot 10^{-34}\,\text{Дж}\cdot\text{с}$, скорость света $c = 3 \cdot 10^{8}\,\frac{\text{м}}{\text{с}}$.
}
\answer{%
    $
        E = h\nu = h \frac c\lambda
        \implies \lambda = \frac{hc}E
            = \frac{6{,}626 \cdot 10^{-34}\,\text{Дж}\cdot\text{с} \cdot {3 \cdot 10^{8}\,\frac{\text{м}}{\text{с}}}}{4{,}04 \cdot 10^{-19}\,\text{Дж}}
            = 492{,}03\,\text{нм}.
    $
}
\solutionspace{150pt}

\tasknumber{6}%
\task{%
    Излучение какой длины волны поглотил атом водорода, если полная энергия в атоме увеличилась на $2 \cdot 10^{-19}\,\text{Дж}$?
    Постоянная Планка $h = 6{,}626 \cdot 10^{-34}\,\text{Дж}\cdot\text{с}$, скорость света $c = 3 \cdot 10^{8}\,\frac{\text{м}}{\text{с}}$.
}
\answer{%
    $
        E = h\nu = h \frac c\lambda
        \implies \lambda = \frac{hc}E
            = \frac{6{,}626 \cdot 10^{-34}\,\text{Дж}\cdot\text{с} \cdot {3 \cdot 10^{8}\,\frac{\text{м}}{\text{с}}}}{2 \cdot 10^{-19}\,\text{Дж}}
            = 994\,\text{нм}.
    $
}
\solutionspace{150pt}

\tasknumber{7}%
\task{%
    Сделайте схематичный рисунок энергетических уровней атома водорода
    и отметьте на нём первый (основной) уровень и последующие.
    Сколько различных длин волн может испустить атом водорода,
    находящийся в 3-м возбуждённом состоянии?
    Отметьте все соответствующие переходы на рисунке и укажите,
    при каком переходе (среди отмеченных) частота излучённого фотона максимальна.
}
\answer{%
    $N = 3{,}0, \text{самая длинная линия}$
}
\solutionspace{150pt}

\tasknumber{8}%
\task{%
    Сколько фотонов испускает за $10\,\text{мин}$ лазер,
    если мощность его излучения $200\,\text{мВт}$?
    Длина волны излучения $750\,\text{нм}$.
    $h = 6{,}626 \cdot 10^{-34}\,\text{Дж}\cdot\text{с}$.
}
\answer{%
    $
        N
            = \frac{E_{\text{общая}}}{E_{\text{одного фотона}}}
            = \frac{Pt}{h\nu} = \frac{Pt}{h \frac c\lambda}
            = \frac{Pt\lambda}{hc}
            = \frac{200\,\text{мВт} \cdot 10\,\text{мин} \cdot 750\,\text{нм}}{6{,}626 \cdot 10^{-34}\,\text{Дж}\cdot\text{с} \cdot 3 \cdot 10^{8}\,\frac{\text{м}}{\text{с}}}
            \approx 4{,}53 \cdot 10^{20}\units{фотонов}
    $
}
\solutionspace{120pt}

\tasknumber{9}%
\task{%
    Какая доля (от начального количества) радиоактивных ядер останется через время,
    равное двум периодам полураспада? Ответ выразить в процентах.
}
\answer{%
    \begin{align*}
    N &= N_0 \cdot 2^{- \frac t{T_{1/2}}} \implies
        \frac N{N_0} = 2^{- \frac t{T_{1/2}}}
        = 2^{-2} \approx 0{,}25 \approx 25\% \\
    N_\text{расп.} &= N_0 - N = N_0 - N_0 \cdot 2^{-\frac t{T_{1/2}}}
        = N_0\cbr{1 - 2^{-\frac t{T_{1/2}}}} \implies
        \frac{N_\text{расп.}}{N_0} = 1 - 2^{-\frac t{T_{1/2}}}
        = 1 - 2^{-2} \approx 0{,}75 \approx 75\%
    \end{align*}
}
\solutionspace{150pt}

\tasknumber{10}%
\task{%
    Сколько процентов ядер радиоактивного железа $\ce{^{59}Fe}$
    останется через $91{,}2\,\text{суток}$, если период его полураспада составляет $45{,}6\,\text{суток}$?
}
\answer{%
    \begin{align*}
    N &= N_0 \cdot 2^{-\frac t{T_{1/2}}}
        = 2^{-\frac{91{,}2\,\text{суток}}{45{,}6\,\text{суток}}}
        \approx 0{,}2500 = 25{,}00\%
    \end{align*}
}
\solutionspace{150pt}

\tasknumber{11}%
\task{%
    За $5\,\text{суток}$ от начального количества ядер радиоизотопа осталась половина.
    Каков период полураспада этого изотопа (ответ приведите в сутках)?
    Какая ещё доля (также от начального количества) распадётся, если подождать ещё столько же?
}
\answer{%
    \begin{align*}
            N &= N_0 \cdot 2^{-\frac t{T_{1/2}}}
            \implies \frac N{N_0} = 2^{-\frac t{T_{1/2}}}
            \implies \frac 1{2} = 2^{-\frac {5\,\text{суток}}{T_{1/2}}}
            \implies 1 = \frac {5\,\text{суток}}{T_{1/2}}
            \implies T_{1/2} = \frac {5\,\text{суток}}1 \approx 5\,\text{суток}.
         \\
            \delta &= \frac{N(t)}{N_0} - \frac{N(2t)}{N_0}
            = 2^{-\frac t{T_{1/2}}} - 2^{-\frac {2t}{T_{1/2}}}
            = 2^{-\frac t{T_{1/2}}}\cbr{1 - 2^{-\frac t{T_{1/2}}}}
            = \frac 1{2} \cdot \cbr{1-\frac 1{2}} \approx 0{,}250
    \end{align*}
}
\solutionspace{150pt}

\tasknumber{12}%
\task{%
    Энергия связи ядра бора \ce{^{11}_{5}B} равна $76{,}2\,\text{МэВ}$.
    Найти дефект массы этого ядра.
    Ответ выразите в а.е.м.
    и кг.
    Скорость света $c = 2{,}998 \cdot 10^{8}\,\frac{\text{м}}{\text{с}}$, элементарный заряд $e = 1{,}6 \cdot 10^{-19}\,\text{Кл}$.
}
\answer{%
    \begin{align*}
    E_\text{св.} &= \Delta m c^2 \implies \\
    \implies
            \Delta m &= \frac {E_\text{св.}}{c^2} = \frac{76{,}2\,\text{МэВ}}{\sqr{2{,}998 \cdot 10^{8}\,\frac{\text{м}}{\text{с}}}}
            = \frac{76{,}2 \cdot 10^6 \cdot 1{,}6 \cdot 10^{-19}\,\text{Дж}}{\sqr{2{,}998 \cdot 10^{8}\,\frac{\text{м}}{\text{с}}}}
            \approx 0{,}1356 \cdot 10^{-27}\,\text{кг} \approx 0{,}0817\,\text{а.е.м.}
    \end{align*}
}

\variantsplitter

\addpersonalvariant{Юлия Буянова}

\tasknumber{1}%
\task{%
    Для частицы, движущейся с релятивистской скоростью,
    выразите $p$ и $E_\text{кин}$ через $c$, $E_0$ и $v$,
    где $E_\text{кин}$~--- кинетическая энергия частицы,
    а $E_0$, $p$ и $v$~--- её энергия покоя импульс и скорость.
}
\answer{%
    \begin{align*}
    E_\text{кин}, E_0:\quad&E = E_\text{кин} + E_0 = \frac{E_0}{\sqrt{1 - \frac{v^2}{c^2}}} \implies \sqrt{1 - \frac{v^2}{c^2}} = \frac{E_0}{{E_0} + {E_\text{кин}}} \implies v = c\sqrt{1 - \sqr{\frac{E_0}{{E_0} + {E_\text{кин}}}}} \\
    &p = \frac{mv}{\sqrt{1 - \frac{v^2}{c^2}}} = \frac{E_0}{c^2} \cdot \sqrt{1 - \sqr{\frac{E_0}{{E_0} + {E_\text{кин}}}}} \cdot \frac{{E_\text{кин}} + {E_0}}{E_0} = \frac{E_0}{c^2} \cdot \sqrt{\sqr{\frac{{E_\text{кин}} + {E_0}}{E_0}} - 1}.
    \\
    E_\text{кин}, p:\quad&E_\text{кин} = E - E_0 = mc^2\cbr{\frac 1{\sqrt{1 - \frac{v^2}{c^2}}} - 1}, p = \frac{mv}{\sqrt{1 - \frac{v^2}{c^2}}} \implies \frac{E_\text{кин}}{p} = \frac{\frac 1{\sqrt{1 - \frac{v^2}{c^2}}} - 1}{\sqrt{1 - \frac{v^2}{c^2}}} \implies v = \ldots \\
    &E_0 = E - E_\text{кин} = \frac{E_0}{\sqrt{1 - \frac{v^2}{c^2}}} - E_\text{кин} \implies E_0 = \frac{E_\text{кин}}{\frac 1{\sqrt{1 - \frac{v^2}{c^2}}} - 1} = \ldots \\
    E_\text{кин}, v:\quad&E_\text{кин} = E - E_0 = mc^2\cbr{\frac 1{\sqrt{1 - \frac{v^2}{c^2}}} - 1} \implies m = \frac{E_\text{кин}}{c^2\cbr{\frac 1{\sqrt{1 - \frac{v^2}{c^2}}} - 1}} \\
    &E_0 = mc^2 = \frac{E_\text{кин}}{\frac 1{\sqrt{1 - \frac{v^2}{c^2}}} - 1} \\
    &p = \frac{mv}{\sqrt{1 - \frac{v^2}{c^2}}} = \frac{E_\text{кин}}{c^2\cbr{\frac 1{\sqrt{1 - \frac{v^2}{c^2}}} - 1}} \cdot \frac{v}{\sqrt{1 - \frac{v^2}{c^2}}} = \frac{{E_\text{кин}} v}{c^2\cbr{1 - {\sqrt{1 - \frac{v^2}{c^2}}}}} \\
    E_0, p:\quad&E_0 = mc^2, \quad p = \frac{mv}{\sqrt{1 - \frac{v^2}{c^2}}} \implies \frac{E_0}{p} = \frac{c^2}v{\sqrt{1 - \frac{v^2}{c^2}}} = c\sqrt{\frac{c^2}{v^2} - 1} \\
    &\sqr{\frac{E_0}{pc}} = \frac{c^2}{v^2} - 1 \implies \frac{v^2}{c^2} = \frac 1{1 + \frac{E_0^2}{p^2c^2}} \implies v = \frac c{\sqrt{1 + \frac{E_0^2}{p^2c^2}}} \\
    &{E_\text{кин}} = E - E_0 = \sqrt{E_0^2 + p^2c^2} - E_0 \\
    E_0, v:\quad&E_0 = mc^2 \implies m = \frac{E_0}{c^2} \qquad p = \frac{mv}{\sqrt{1 - \frac{v^2}{c^2}}} = \frac{E_0}{c^2} \cdot \frac{v}{\sqrt{1 - \frac{v^2}{c^2}}} \\
    &E_\text{кин}= mc^2\cbr{\frac 1{\sqrt{1 - \frac{v^2}{c^2}}} - 1} = \frac{E_0}{c^2}\cbr{\frac 1{\sqrt{1 - \frac{v^2}{c^2}}} - 1} \\
    p, v:\quad&p = \frac{mv}{\sqrt{1 - \frac{v^2}{c^2}}} \implies m = \frac p v {\sqrt{1 - \frac{v^2}{c^2}}} \implies E_0 = mc^2 =\frac {pc^2} v {\sqrt{1 - \frac{v^2}{c^2}}} \\
    &E_\text{кин} = mc^2\cbr{\frac 1{\sqrt{1 - \frac{v^2}{c^2}}} - 1} = \frac p v {\sqrt{1 - \frac{v^2}{c^2}}}\cbr{\frac 1{\sqrt{1 - \frac{v^2}{c^2}}} - 1} = \frac p v \cbr{1 - {\sqrt{1 - \frac{v^2}{c^2}}}}
    \end{align*}
}
\solutionspace{200pt}

\tasknumber{2}%
\task{%
    Позитрон движется со скоростью $0{,}6\,c$, где $c$~--- скорость света в вакууме.
    Каково при этом отношение кинетической энергии частицы $E_\text{кин.}$ к его энергии покоя $E_0$?
}
\answer{%
    \begin{align*}
    E &= \frac{E_0}{\sqrt{1 - \frac{v^2}{c^2}}}
            \implies \frac E{E_0}
                = \frac 1{\sqrt{1 - \frac{v^2}{c^2}}}
                = \frac 1{\sqrt{1 - \sqr{0{,}6}}}
                \approx 1{,}250,
         \\
        E_{\text{кин}} &= E - E_0
            \implies \frac{E_{\text{кин}}}{E_0}
                = \frac E{E_0} - 1
                = \frac 1{\sqrt{1 - \frac{v^2}{c^2}}} - 1
                = \frac 1{\sqrt{1 - \sqr{0{,}6}}} - 1
                \approx 0{,}250.
    \end{align*}
}
\solutionspace{150pt}

\tasknumber{3}%
\task{%
    Электрон движется со скоростью $0{,}75\,c$, где $c$~--- скорость света в вакууме.
    Определите его импульс (в ответе приведите формулу и укажите численное значение).
}
\answer{%
    \begin{align*}
    E &= \frac{mc^2}{\sqrt{1 - \frac{v^2}{c^2}}}
            \approx \frac{9{,}1 \cdot 10^{-31}\,\text{кг} \cdot \sqr{3 \cdot 10^{8}\,\frac{\text{м}}{\text{с}}}}{\sqrt{1 - 0{,}75^2}}
            \approx 0{,}124 \cdot 10^{-12}\,\text{Дж},
         \\
        E_{\text{кин}} &= \frac{mc^2}{\sqrt{1 - \frac{v^2}{c^2}}} - mc^2
            = mc^2 \cbr{\frac 1{\sqrt{1 - \frac{v^2}{c^2}}} - 1} \approx \\
            &\approx \cbr{9{,}1 \cdot 10^{-31}\,\text{кг} \cdot \sqr{3 \cdot 10^{8}\,\frac{\text{м}}{\text{с}}}}
            \cdot \cbr{\frac 1{\sqrt{1 - 0{,}75^2}} - 1}
            \approx 0{,}042 \cdot 10^{-12}\,\text{Дж},
         \\
        p &= \frac{mv}{\sqrt{1 - \frac{v^2}{c^2}}}
            \approx \frac{9{,}1 \cdot 10^{-31}\,\text{кг} \cdot 0{,}75 \cdot 3 \cdot 10^{8}\,\frac{\text{м}}{\text{с}}}{\sqrt{1 - 0{,}75^2}}
            \approx 0{,}310 \cdot 10^{-21}\,\frac{\text{кг}\cdot\text{м}}{\text{с}}.
    \end{align*}
}
\solutionspace{150pt}

\tasknumber{4}%
\task{%
    При какой скорости движения (в м/с) релятивистское сокращение длины движущегося тела
    составит 30\%?
}
\answer{%
    \begin{align*}
    l_0 &= \frac l{\sqrt{1 - \frac{v^2}{c^2}}}
        \implies 1 - \frac{v^2}{c^2} = \sqr{\frac l{l_0}}
        \implies \frac v c = \sqrt{1 - \sqr{\frac l{l_0}}} \implies
         \\
        \implies v &= c\sqrt{1 - \sqr{\frac l{l_0}}}
        = 3 \cdot 10^{8}\,\frac{\text{м}}{\text{с}} \cdot \sqrt{1 - \sqr{\frac {l_0 - 0{,}30l_0}{l_0}}}
        = 3 \cdot 10^{8}\,\frac{\text{м}}{\text{с}} \cdot \sqrt{1 - \sqr{1 - 0{,}30}} \approx  \\
        &\approx 0{,}714c
        \approx 214 \cdot 10^{6}\,\frac{\text{м}}{\text{с}}
        \approx 771 \cdot 10^{6}\,\frac{\text{км}}{\text{ч}}.
    \end{align*}
}
\solutionspace{150pt}

\tasknumber{5}%
\task{%
    При переходе электрона в атоме с одной стационарной орбиты на другую
    излучается фотон с энергией $4{,}04 \cdot 10^{-19}\,\text{Дж}$.
    Какова длина волны этой линии спектра?
    Постоянная Планка $h = 6{,}626 \cdot 10^{-34}\,\text{Дж}\cdot\text{с}$, скорость света $c = 3 \cdot 10^{8}\,\frac{\text{м}}{\text{с}}$.
}
\answer{%
    $
        E = h\nu = h \frac c\lambda
        \implies \lambda = \frac{hc}E
            = \frac{6{,}626 \cdot 10^{-34}\,\text{Дж}\cdot\text{с} \cdot {3 \cdot 10^{8}\,\frac{\text{м}}{\text{с}}}}{4{,}04 \cdot 10^{-19}\,\text{Дж}}
            = 492{,}03\,\text{нм}.
    $
}
\solutionspace{150pt}

\tasknumber{6}%
\task{%
    Излучение какой длины волны поглотил атом водорода, если полная энергия в атоме увеличилась на $4 \cdot 10^{-19}\,\text{Дж}$?
    Постоянная Планка $h = 6{,}626 \cdot 10^{-34}\,\text{Дж}\cdot\text{с}$, скорость света $c = 3 \cdot 10^{8}\,\frac{\text{м}}{\text{с}}$.
}
\answer{%
    $
        E = h\nu = h \frac c\lambda
        \implies \lambda = \frac{hc}E
            = \frac{6{,}626 \cdot 10^{-34}\,\text{Дж}\cdot\text{с} \cdot {3 \cdot 10^{8}\,\frac{\text{м}}{\text{с}}}}{4 \cdot 10^{-19}\,\text{Дж}}
            = 497\,\text{нм}.
    $
}
\solutionspace{150pt}

\tasknumber{7}%
\task{%
    Сделайте схематичный рисунок энергетических уровней атома водорода
    и отметьте на нём первый (основной) уровень и последующие.
    Сколько различных длин волн может испустить атом водорода,
    находящийся в 4-м возбуждённом состоянии?
    Отметьте все соответствующие переходы на рисунке и укажите,
    при каком переходе (среди отмеченных) энергия излучённого фотона максимальна.
}
\answer{%
    $N = 6{,}0, \text{самая длинная линия}$
}
\solutionspace{150pt}

\tasknumber{8}%
\task{%
    Сколько фотонов испускает за $60\,\text{мин}$ лазер,
    если мощность его излучения $15\,\text{мВт}$?
    Длина волны излучения $500\,\text{нм}$.
    $h = 6{,}626 \cdot 10^{-34}\,\text{Дж}\cdot\text{с}$.
}
\answer{%
    $
        N
            = \frac{E_{\text{общая}}}{E_{\text{одного фотона}}}
            = \frac{Pt}{h\nu} = \frac{Pt}{h \frac c\lambda}
            = \frac{Pt\lambda}{hc}
            = \frac{15\,\text{мВт} \cdot 60\,\text{мин} \cdot 500\,\text{нм}}{6{,}626 \cdot 10^{-34}\,\text{Дж}\cdot\text{с} \cdot 3 \cdot 10^{8}\,\frac{\text{м}}{\text{с}}}
            \approx 1{,}36 \cdot 10^{20}\units{фотонов}
    $
}
\solutionspace{120pt}

\tasknumber{9}%
\task{%
    Какая доля (от начального количества) радиоактивных ядер останется через время,
    равное двум периодам полураспада? Ответ выразить в процентах.
}
\answer{%
    \begin{align*}
    N &= N_0 \cdot 2^{- \frac t{T_{1/2}}} \implies
        \frac N{N_0} = 2^{- \frac t{T_{1/2}}}
        = 2^{-2} \approx 0{,}25 \approx 25\% \\
    N_\text{расп.} &= N_0 - N = N_0 - N_0 \cdot 2^{-\frac t{T_{1/2}}}
        = N_0\cbr{1 - 2^{-\frac t{T_{1/2}}}} \implies
        \frac{N_\text{расп.}}{N_0} = 1 - 2^{-\frac t{T_{1/2}}}
        = 1 - 2^{-2} \approx 0{,}75 \approx 75\%
    \end{align*}
}
\solutionspace{150pt}

\tasknumber{10}%
\task{%
    Сколько процентов ядер радиоактивного железа $\ce{^{59}Fe}$
    останется через $91{,}2\,\text{суток}$, если период его полураспада составляет $45{,}6\,\text{суток}$?
}
\answer{%
    \begin{align*}
    N &= N_0 \cdot 2^{-\frac t{T_{1/2}}}
        = 2^{-\frac{91{,}2\,\text{суток}}{45{,}6\,\text{суток}}}
        \approx 0{,}2500 = 25{,}00\%
    \end{align*}
}
\solutionspace{150pt}

\tasknumber{11}%
\task{%
    За $2\,\text{суток}$ от начального количества ядер радиоизотопа осталась четверть.
    Каков период полураспада этого изотопа (ответ приведите в сутках)?
    Какая ещё доля (также от начального количества) распадётся, если подождать ещё столько же?
}
\answer{%
    \begin{align*}
            N &= N_0 \cdot 2^{-\frac t{T_{1/2}}}
            \implies \frac N{N_0} = 2^{-\frac t{T_{1/2}}}
            \implies \frac 1{4} = 2^{-\frac {2\,\text{суток}}{T_{1/2}}}
            \implies 2 = \frac {2\,\text{суток}}{T_{1/2}}
            \implies T_{1/2} = \frac {2\,\text{суток}}2 \approx 1\,\text{суток}.
         \\
            \delta &= \frac{N(t)}{N_0} - \frac{N(2t)}{N_0}
            = 2^{-\frac t{T_{1/2}}} - 2^{-\frac {2t}{T_{1/2}}}
            = 2^{-\frac t{T_{1/2}}}\cbr{1 - 2^{-\frac t{T_{1/2}}}}
            = \frac 1{4} \cdot \cbr{1-\frac 1{4}} \approx 0{,}188
    \end{align*}
}
\solutionspace{150pt}

\tasknumber{12}%
\task{%
    Энергия связи ядра лития \ce{^{7}_{3}Li} равна $39{,}2\,\text{МэВ}$.
    Найти дефект массы этого ядра.
    Ответ выразите в а.е.м.
    и кг.
    Скорость света $c = 2{,}998 \cdot 10^{8}\,\frac{\text{м}}{\text{с}}$, элементарный заряд $e = 1{,}6 \cdot 10^{-19}\,\text{Кл}$.
}
\answer{%
    \begin{align*}
    E_\text{св.} &= \Delta m c^2 \implies \\
    \implies
            \Delta m &= \frac {E_\text{св.}}{c^2} = \frac{39{,}2\,\text{МэВ}}{\sqr{2{,}998 \cdot 10^{8}\,\frac{\text{м}}{\text{с}}}}
            = \frac{39{,}2 \cdot 10^6 \cdot 1{,}6 \cdot 10^{-19}\,\text{Дж}}{\sqr{2{,}998 \cdot 10^{8}\,\frac{\text{м}}{\text{с}}}}
            \approx 69{,}8 \cdot 10^{-30}\,\text{кг} \approx 0{,}0420\,\text{а.е.м.}
    \end{align*}
}

\variantsplitter

\addpersonalvariant{Пелагея Вдовина}

\tasknumber{1}%
\task{%
    Для частицы, движущейся с релятивистской скоростью,
    выразите $v$ и $E_0$ через $c$, $p$ и $E_\text{кин}$,
    где $E_\text{кин}$~--- кинетическая энергия частицы,
    а $E_0$, $p$ и $v$~--- её энергия покоя импульс и скорость.
}
\answer{%
    \begin{align*}
    E_\text{кин}, E_0:\quad&E = E_\text{кин} + E_0 = \frac{E_0}{\sqrt{1 - \frac{v^2}{c^2}}} \implies \sqrt{1 - \frac{v^2}{c^2}} = \frac{E_0}{{E_0} + {E_\text{кин}}} \implies v = c\sqrt{1 - \sqr{\frac{E_0}{{E_0} + {E_\text{кин}}}}} \\
    &p = \frac{mv}{\sqrt{1 - \frac{v^2}{c^2}}} = \frac{E_0}{c^2} \cdot \sqrt{1 - \sqr{\frac{E_0}{{E_0} + {E_\text{кин}}}}} \cdot \frac{{E_\text{кин}} + {E_0}}{E_0} = \frac{E_0}{c^2} \cdot \sqrt{\sqr{\frac{{E_\text{кин}} + {E_0}}{E_0}} - 1}.
    \\
    E_\text{кин}, p:\quad&E_\text{кин} = E - E_0 = mc^2\cbr{\frac 1{\sqrt{1 - \frac{v^2}{c^2}}} - 1}, p = \frac{mv}{\sqrt{1 - \frac{v^2}{c^2}}} \implies \frac{E_\text{кин}}{p} = \frac{\frac 1{\sqrt{1 - \frac{v^2}{c^2}}} - 1}{\sqrt{1 - \frac{v^2}{c^2}}} \implies v = \ldots \\
    &E_0 = E - E_\text{кин} = \frac{E_0}{\sqrt{1 - \frac{v^2}{c^2}}} - E_\text{кин} \implies E_0 = \frac{E_\text{кин}}{\frac 1{\sqrt{1 - \frac{v^2}{c^2}}} - 1} = \ldots \\
    E_\text{кин}, v:\quad&E_\text{кин} = E - E_0 = mc^2\cbr{\frac 1{\sqrt{1 - \frac{v^2}{c^2}}} - 1} \implies m = \frac{E_\text{кин}}{c^2\cbr{\frac 1{\sqrt{1 - \frac{v^2}{c^2}}} - 1}} \\
    &E_0 = mc^2 = \frac{E_\text{кин}}{\frac 1{\sqrt{1 - \frac{v^2}{c^2}}} - 1} \\
    &p = \frac{mv}{\sqrt{1 - \frac{v^2}{c^2}}} = \frac{E_\text{кин}}{c^2\cbr{\frac 1{\sqrt{1 - \frac{v^2}{c^2}}} - 1}} \cdot \frac{v}{\sqrt{1 - \frac{v^2}{c^2}}} = \frac{{E_\text{кин}} v}{c^2\cbr{1 - {\sqrt{1 - \frac{v^2}{c^2}}}}} \\
    E_0, p:\quad&E_0 = mc^2, \quad p = \frac{mv}{\sqrt{1 - \frac{v^2}{c^2}}} \implies \frac{E_0}{p} = \frac{c^2}v{\sqrt{1 - \frac{v^2}{c^2}}} = c\sqrt{\frac{c^2}{v^2} - 1} \\
    &\sqr{\frac{E_0}{pc}} = \frac{c^2}{v^2} - 1 \implies \frac{v^2}{c^2} = \frac 1{1 + \frac{E_0^2}{p^2c^2}} \implies v = \frac c{\sqrt{1 + \frac{E_0^2}{p^2c^2}}} \\
    &{E_\text{кин}} = E - E_0 = \sqrt{E_0^2 + p^2c^2} - E_0 \\
    E_0, v:\quad&E_0 = mc^2 \implies m = \frac{E_0}{c^2} \qquad p = \frac{mv}{\sqrt{1 - \frac{v^2}{c^2}}} = \frac{E_0}{c^2} \cdot \frac{v}{\sqrt{1 - \frac{v^2}{c^2}}} \\
    &E_\text{кин}= mc^2\cbr{\frac 1{\sqrt{1 - \frac{v^2}{c^2}}} - 1} = \frac{E_0}{c^2}\cbr{\frac 1{\sqrt{1 - \frac{v^2}{c^2}}} - 1} \\
    p, v:\quad&p = \frac{mv}{\sqrt{1 - \frac{v^2}{c^2}}} \implies m = \frac p v {\sqrt{1 - \frac{v^2}{c^2}}} \implies E_0 = mc^2 =\frac {pc^2} v {\sqrt{1 - \frac{v^2}{c^2}}} \\
    &E_\text{кин} = mc^2\cbr{\frac 1{\sqrt{1 - \frac{v^2}{c^2}}} - 1} = \frac p v {\sqrt{1 - \frac{v^2}{c^2}}}\cbr{\frac 1{\sqrt{1 - \frac{v^2}{c^2}}} - 1} = \frac p v \cbr{1 - {\sqrt{1 - \frac{v^2}{c^2}}}}
    \end{align*}
}
\solutionspace{200pt}

\tasknumber{2}%
\task{%
    Позитрон движется со скоростью $0{,}9\,c$, где $c$~--- скорость света в вакууме.
    Каково при этом отношение полной энергии частицы $E$ к его энергии покоя $E_0$?
}
\answer{%
    \begin{align*}
    E &= \frac{E_0}{\sqrt{1 - \frac{v^2}{c^2}}}
            \implies \frac E{E_0}
                = \frac 1{\sqrt{1 - \frac{v^2}{c^2}}}
                = \frac 1{\sqrt{1 - \sqr{0{,}9}}}
                \approx 2{,}294,
         \\
        E_{\text{кин}} &= E - E_0
            \implies \frac{E_{\text{кин}}}{E_0}
                = \frac E{E_0} - 1
                = \frac 1{\sqrt{1 - \frac{v^2}{c^2}}} - 1
                = \frac 1{\sqrt{1 - \sqr{0{,}9}}} - 1
                \approx 1{,}294.
    \end{align*}
}
\solutionspace{150pt}

\tasknumber{3}%
\task{%
    Протон движется со скоростью $0{,}75\,c$, где $c$~--- скорость света в вакууме.
    Определите его полную энергию (в ответе приведите формулу и укажите численное значение).
}
\answer{%
    \begin{align*}
    E &= \frac{mc^2}{\sqrt{1 - \frac{v^2}{c^2}}}
            \approx \frac{1{,}673 \cdot 10^{-27}\,\text{кг} \cdot \sqr{3 \cdot 10^{8}\,\frac{\text{м}}{\text{с}}}}{\sqrt{1 - 0{,}75^2}}
            \approx 227{,}589 \cdot 10^{-12}\,\text{Дж},
         \\
        E_{\text{кин}} &= \frac{mc^2}{\sqrt{1 - \frac{v^2}{c^2}}} - mc^2
            = mc^2 \cbr{\frac 1{\sqrt{1 - \frac{v^2}{c^2}}} - 1} \approx \\
            &\approx \cbr{1{,}673 \cdot 10^{-27}\,\text{кг} \cdot \sqr{3 \cdot 10^{8}\,\frac{\text{м}}{\text{с}}}}
            \cdot \cbr{\frac 1{\sqrt{1 - 0{,}75^2}} - 1}
            \approx 77{,}053 \cdot 10^{-12}\,\text{Дж},
         \\
        p &= \frac{mv}{\sqrt{1 - \frac{v^2}{c^2}}}
            \approx \frac{1{,}673 \cdot 10^{-27}\,\text{кг} \cdot 0{,}75 \cdot 3 \cdot 10^{8}\,\frac{\text{м}}{\text{с}}}{\sqrt{1 - 0{,}75^2}}
            \approx 568{,}972 \cdot 10^{-21}\,\frac{\text{кг}\cdot\text{м}}{\text{с}}.
    \end{align*}
}
\solutionspace{150pt}

\tasknumber{4}%
\task{%
    При какой скорости движения (в долях скорости света) релятивистское сокращение длины движущегося тела
    составит 10\%?
}
\answer{%
    \begin{align*}
    l_0 &= \frac l{\sqrt{1 - \frac{v^2}{c^2}}}
        \implies 1 - \frac{v^2}{c^2} = \sqr{\frac l{l_0}}
        \implies \frac v c = \sqrt{1 - \sqr{\frac l{l_0}}} \implies
         \\
        \implies v &= c\sqrt{1 - \sqr{\frac l{l_0}}}
        = 3 \cdot 10^{8}\,\frac{\text{м}}{\text{с}} \cdot \sqrt{1 - \sqr{\frac {l_0 - 0{,}10l_0}{l_0}}}
        = 3 \cdot 10^{8}\,\frac{\text{м}}{\text{с}} \cdot \sqrt{1 - \sqr{1 - 0{,}10}} \approx  \\
        &\approx 0{,}436c
        \approx 130{,}8 \cdot 10^{6}\,\frac{\text{м}}{\text{с}}
        \approx 471 \cdot 10^{6}\,\frac{\text{км}}{\text{ч}}.
    \end{align*}
}
\solutionspace{150pt}

\tasknumber{5}%
\task{%
    При переходе электрона в атоме с одной стационарной орбиты на другую
    излучается фотон с энергией $7{,}07 \cdot 10^{-19}\,\text{Дж}$.
    Какова длина волны этой линии спектра?
    Постоянная Планка $h = 6{,}626 \cdot 10^{-34}\,\text{Дж}\cdot\text{с}$, скорость света $c = 3 \cdot 10^{8}\,\frac{\text{м}}{\text{с}}$.
}
\answer{%
    $
        E = h\nu = h \frac c\lambda
        \implies \lambda = \frac{hc}E
            = \frac{6{,}626 \cdot 10^{-34}\,\text{Дж}\cdot\text{с} \cdot {3 \cdot 10^{8}\,\frac{\text{м}}{\text{с}}}}{7{,}07 \cdot 10^{-19}\,\text{Дж}}
            = 281{,}16\,\text{нм}.
    $
}
\solutionspace{150pt}

\tasknumber{6}%
\task{%
    Излучение какой длины волны поглотил атом водорода, если полная энергия в атоме увеличилась на $6 \cdot 10^{-19}\,\text{Дж}$?
    Постоянная Планка $h = 6{,}626 \cdot 10^{-34}\,\text{Дж}\cdot\text{с}$, скорость света $c = 3 \cdot 10^{8}\,\frac{\text{м}}{\text{с}}$.
}
\answer{%
    $
        E = h\nu = h \frac c\lambda
        \implies \lambda = \frac{hc}E
            = \frac{6{,}626 \cdot 10^{-34}\,\text{Дж}\cdot\text{с} \cdot {3 \cdot 10^{8}\,\frac{\text{м}}{\text{с}}}}{6 \cdot 10^{-19}\,\text{Дж}}
            = 331\,\text{нм}.
    $
}
\solutionspace{150pt}

\tasknumber{7}%
\task{%
    Сделайте схематичный рисунок энергетических уровней атома водорода
    и отметьте на нём первый (основной) уровень и последующие.
    Сколько различных длин волн может испустить атом водорода,
    находящийся в 4-м возбуждённом состоянии?
    Отметьте все соответствующие переходы на рисунке и укажите,
    при каком переходе (среди отмеченных) длина волны излучённого фотона максимальна.
}
\answer{%
    $N = 6{,}0, \text{самая короткая линия}$
}
\solutionspace{150pt}

\tasknumber{8}%
\task{%
    Сколько фотонов испускает за $5\,\text{мин}$ лазер,
    если мощность его излучения $15\,\text{мВт}$?
    Длина волны излучения $600\,\text{нм}$.
    $h = 6{,}626 \cdot 10^{-34}\,\text{Дж}\cdot\text{с}$.
}
\answer{%
    $
        N
            = \frac{E_{\text{общая}}}{E_{\text{одного фотона}}}
            = \frac{Pt}{h\nu} = \frac{Pt}{h \frac c\lambda}
            = \frac{Pt\lambda}{hc}
            = \frac{15\,\text{мВт} \cdot 5\,\text{мин} \cdot 600\,\text{нм}}{6{,}626 \cdot 10^{-34}\,\text{Дж}\cdot\text{с} \cdot 3 \cdot 10^{8}\,\frac{\text{м}}{\text{с}}}
            \approx 0{,}14 \cdot 10^{20}\units{фотонов}
    $
}
\solutionspace{120pt}

\tasknumber{9}%
\task{%
    Какая доля (от начального количества) радиоактивных ядер распадётся через время,
    равное двум периодам полураспада? Ответ выразить в процентах.
}
\answer{%
    \begin{align*}
    N &= N_0 \cdot 2^{- \frac t{T_{1/2}}} \implies
        \frac N{N_0} = 2^{- \frac t{T_{1/2}}}
        = 2^{-2} \approx 0{,}25 \approx 25\% \\
    N_\text{расп.} &= N_0 - N = N_0 - N_0 \cdot 2^{-\frac t{T_{1/2}}}
        = N_0\cbr{1 - 2^{-\frac t{T_{1/2}}}} \implies
        \frac{N_\text{расп.}}{N_0} = 1 - 2^{-\frac t{T_{1/2}}}
        = 1 - 2^{-2} \approx 0{,}75 \approx 75\%
    \end{align*}
}
\solutionspace{150pt}

\tasknumber{10}%
\task{%
    Сколько процентов ядер радиоактивного железа $\ce{^{59}Fe}$
    останется через $136{,}8\,\text{суток}$, если период его полураспада составляет $45{,}6\,\text{суток}$?
}
\answer{%
    \begin{align*}
    N &= N_0 \cdot 2^{-\frac t{T_{1/2}}}
        = 2^{-\frac{136{,}8\,\text{суток}}{45{,}6\,\text{суток}}}
        \approx 0{,}1250 = 12{,}50\%
    \end{align*}
}
\solutionspace{150pt}

\tasknumber{11}%
\task{%
    За $2\,\text{суток}$ от начального количества ядер радиоизотопа осталась четверть.
    Каков период полураспада этого изотопа (ответ приведите в сутках)?
    Какая ещё доля (также от начального количества) распадётся, если подождать ещё столько же?
}
\answer{%
    \begin{align*}
            N &= N_0 \cdot 2^{-\frac t{T_{1/2}}}
            \implies \frac N{N_0} = 2^{-\frac t{T_{1/2}}}
            \implies \frac 1{4} = 2^{-\frac {2\,\text{суток}}{T_{1/2}}}
            \implies 2 = \frac {2\,\text{суток}}{T_{1/2}}
            \implies T_{1/2} = \frac {2\,\text{суток}}2 \approx 1\,\text{суток}.
         \\
            \delta &= \frac{N(t)}{N_0} - \frac{N(2t)}{N_0}
            = 2^{-\frac t{T_{1/2}}} - 2^{-\frac {2t}{T_{1/2}}}
            = 2^{-\frac t{T_{1/2}}}\cbr{1 - 2^{-\frac t{T_{1/2}}}}
            = \frac 1{4} \cdot \cbr{1-\frac 1{4}} \approx 0{,}188
    \end{align*}
}
\solutionspace{150pt}

\tasknumber{12}%
\task{%
    Энергия связи ядра углерода \ce{^{12}_{6}C} равна $92{,}2\,\text{МэВ}$.
    Найти дефект массы этого ядра.
    Ответ выразите в а.е.м.
    и кг.
    Скорость света $c = 2{,}998 \cdot 10^{8}\,\frac{\text{м}}{\text{с}}$, элементарный заряд $e = 1{,}6 \cdot 10^{-19}\,\text{Кл}$.
}
\answer{%
    \begin{align*}
    E_\text{св.} &= \Delta m c^2 \implies \\
    \implies
            \Delta m &= \frac {E_\text{св.}}{c^2} = \frac{92{,}2\,\text{МэВ}}{\sqr{2{,}998 \cdot 10^{8}\,\frac{\text{м}}{\text{с}}}}
            = \frac{92{,}2 \cdot 10^6 \cdot 1{,}6 \cdot 10^{-19}\,\text{Дж}}{\sqr{2{,}998 \cdot 10^{8}\,\frac{\text{м}}{\text{с}}}}
            \approx 0{,}1641 \cdot 10^{-27}\,\text{кг} \approx 0{,}0988\,\text{а.е.м.}
    \end{align*}
}

\variantsplitter

\addpersonalvariant{Леонид Викторов}

\tasknumber{1}%
\task{%
    Для частицы, движущейся с релятивистской скоростью,
    выразите $E_0$ и $v$ через $c$, $p$ и $E_\text{кин}$,
    где $E_\text{кин}$~--- кинетическая энергия частицы,
    а $E_0$, $p$ и $v$~--- её энергия покоя импульс и скорость.
}
\answer{%
    \begin{align*}
    E_\text{кин}, E_0:\quad&E = E_\text{кин} + E_0 = \frac{E_0}{\sqrt{1 - \frac{v^2}{c^2}}} \implies \sqrt{1 - \frac{v^2}{c^2}} = \frac{E_0}{{E_0} + {E_\text{кин}}} \implies v = c\sqrt{1 - \sqr{\frac{E_0}{{E_0} + {E_\text{кин}}}}} \\
    &p = \frac{mv}{\sqrt{1 - \frac{v^2}{c^2}}} = \frac{E_0}{c^2} \cdot \sqrt{1 - \sqr{\frac{E_0}{{E_0} + {E_\text{кин}}}}} \cdot \frac{{E_\text{кин}} + {E_0}}{E_0} = \frac{E_0}{c^2} \cdot \sqrt{\sqr{\frac{{E_\text{кин}} + {E_0}}{E_0}} - 1}.
    \\
    E_\text{кин}, p:\quad&E_\text{кин} = E - E_0 = mc^2\cbr{\frac 1{\sqrt{1 - \frac{v^2}{c^2}}} - 1}, p = \frac{mv}{\sqrt{1 - \frac{v^2}{c^2}}} \implies \frac{E_\text{кин}}{p} = \frac{\frac 1{\sqrt{1 - \frac{v^2}{c^2}}} - 1}{\sqrt{1 - \frac{v^2}{c^2}}} \implies v = \ldots \\
    &E_0 = E - E_\text{кин} = \frac{E_0}{\sqrt{1 - \frac{v^2}{c^2}}} - E_\text{кин} \implies E_0 = \frac{E_\text{кин}}{\frac 1{\sqrt{1 - \frac{v^2}{c^2}}} - 1} = \ldots \\
    E_\text{кин}, v:\quad&E_\text{кин} = E - E_0 = mc^2\cbr{\frac 1{\sqrt{1 - \frac{v^2}{c^2}}} - 1} \implies m = \frac{E_\text{кин}}{c^2\cbr{\frac 1{\sqrt{1 - \frac{v^2}{c^2}}} - 1}} \\
    &E_0 = mc^2 = \frac{E_\text{кин}}{\frac 1{\sqrt{1 - \frac{v^2}{c^2}}} - 1} \\
    &p = \frac{mv}{\sqrt{1 - \frac{v^2}{c^2}}} = \frac{E_\text{кин}}{c^2\cbr{\frac 1{\sqrt{1 - \frac{v^2}{c^2}}} - 1}} \cdot \frac{v}{\sqrt{1 - \frac{v^2}{c^2}}} = \frac{{E_\text{кин}} v}{c^2\cbr{1 - {\sqrt{1 - \frac{v^2}{c^2}}}}} \\
    E_0, p:\quad&E_0 = mc^2, \quad p = \frac{mv}{\sqrt{1 - \frac{v^2}{c^2}}} \implies \frac{E_0}{p} = \frac{c^2}v{\sqrt{1 - \frac{v^2}{c^2}}} = c\sqrt{\frac{c^2}{v^2} - 1} \\
    &\sqr{\frac{E_0}{pc}} = \frac{c^2}{v^2} - 1 \implies \frac{v^2}{c^2} = \frac 1{1 + \frac{E_0^2}{p^2c^2}} \implies v = \frac c{\sqrt{1 + \frac{E_0^2}{p^2c^2}}} \\
    &{E_\text{кин}} = E - E_0 = \sqrt{E_0^2 + p^2c^2} - E_0 \\
    E_0, v:\quad&E_0 = mc^2 \implies m = \frac{E_0}{c^2} \qquad p = \frac{mv}{\sqrt{1 - \frac{v^2}{c^2}}} = \frac{E_0}{c^2} \cdot \frac{v}{\sqrt{1 - \frac{v^2}{c^2}}} \\
    &E_\text{кин}= mc^2\cbr{\frac 1{\sqrt{1 - \frac{v^2}{c^2}}} - 1} = \frac{E_0}{c^2}\cbr{\frac 1{\sqrt{1 - \frac{v^2}{c^2}}} - 1} \\
    p, v:\quad&p = \frac{mv}{\sqrt{1 - \frac{v^2}{c^2}}} \implies m = \frac p v {\sqrt{1 - \frac{v^2}{c^2}}} \implies E_0 = mc^2 =\frac {pc^2} v {\sqrt{1 - \frac{v^2}{c^2}}} \\
    &E_\text{кин} = mc^2\cbr{\frac 1{\sqrt{1 - \frac{v^2}{c^2}}} - 1} = \frac p v {\sqrt{1 - \frac{v^2}{c^2}}}\cbr{\frac 1{\sqrt{1 - \frac{v^2}{c^2}}} - 1} = \frac p v \cbr{1 - {\sqrt{1 - \frac{v^2}{c^2}}}}
    \end{align*}
}
\solutionspace{200pt}

\tasknumber{2}%
\task{%
    Позитрон движется со скоростью $0{,}8\,c$, где $c$~--- скорость света в вакууме.
    Каково при этом отношение полной энергии частицы $E$ к его энергии покоя $E_0$?
}
\answer{%
    \begin{align*}
    E &= \frac{E_0}{\sqrt{1 - \frac{v^2}{c^2}}}
            \implies \frac E{E_0}
                = \frac 1{\sqrt{1 - \frac{v^2}{c^2}}}
                = \frac 1{\sqrt{1 - \sqr{0{,}8}}}
                \approx 1{,}667,
         \\
        E_{\text{кин}} &= E - E_0
            \implies \frac{E_{\text{кин}}}{E_0}
                = \frac E{E_0} - 1
                = \frac 1{\sqrt{1 - \frac{v^2}{c^2}}} - 1
                = \frac 1{\sqrt{1 - \sqr{0{,}8}}} - 1
                \approx 0{,}667.
    \end{align*}
}
\solutionspace{150pt}

\tasknumber{3}%
\task{%
    Протон движется со скоростью $0{,}65\,c$, где $c$~--- скорость света в вакууме.
    Определите его полную энергию (в ответе приведите формулу и укажите численное значение).
}
\answer{%
    \begin{align*}
    E &= \frac{mc^2}{\sqrt{1 - \frac{v^2}{c^2}}}
            \approx \frac{1{,}673 \cdot 10^{-27}\,\text{кг} \cdot \sqr{3 \cdot 10^{8}\,\frac{\text{м}}{\text{с}}}}{\sqrt{1 - 0{,}65^2}}
            \approx 198{,}091 \cdot 10^{-12}\,\text{Дж},
         \\
        E_{\text{кин}} &= \frac{mc^2}{\sqrt{1 - \frac{v^2}{c^2}}} - mc^2
            = mc^2 \cbr{\frac 1{\sqrt{1 - \frac{v^2}{c^2}}} - 1} \approx \\
            &\approx \cbr{1{,}673 \cdot 10^{-27}\,\text{кг} \cdot \sqr{3 \cdot 10^{8}\,\frac{\text{м}}{\text{с}}}}
            \cdot \cbr{\frac 1{\sqrt{1 - 0{,}65^2}} - 1}
            \approx 47{,}555 \cdot 10^{-12}\,\text{Дж},
         \\
        p &= \frac{mv}{\sqrt{1 - \frac{v^2}{c^2}}}
            \approx \frac{1{,}673 \cdot 10^{-27}\,\text{кг} \cdot 0{,}65 \cdot 3 \cdot 10^{8}\,\frac{\text{м}}{\text{с}}}{\sqrt{1 - 0{,}65^2}}
            \approx 429{,}196 \cdot 10^{-21}\,\frac{\text{кг}\cdot\text{м}}{\text{с}}.
    \end{align*}
}
\solutionspace{150pt}

\tasknumber{4}%
\task{%
    При какой скорости движения (в км/ч) релятивистское сокращение длины движущегося тела
    составит 30\%?
}
\answer{%
    \begin{align*}
    l_0 &= \frac l{\sqrt{1 - \frac{v^2}{c^2}}}
        \implies 1 - \frac{v^2}{c^2} = \sqr{\frac l{l_0}}
        \implies \frac v c = \sqrt{1 - \sqr{\frac l{l_0}}} \implies
         \\
        \implies v &= c\sqrt{1 - \sqr{\frac l{l_0}}}
        = 3 \cdot 10^{8}\,\frac{\text{м}}{\text{с}} \cdot \sqrt{1 - \sqr{\frac {l_0 - 0{,}30l_0}{l_0}}}
        = 3 \cdot 10^{8}\,\frac{\text{м}}{\text{с}} \cdot \sqrt{1 - \sqr{1 - 0{,}30}} \approx  \\
        &\approx 0{,}714c
        \approx 214 \cdot 10^{6}\,\frac{\text{м}}{\text{с}}
        \approx 771 \cdot 10^{6}\,\frac{\text{км}}{\text{ч}}.
    \end{align*}
}
\solutionspace{150pt}

\tasknumber{5}%
\task{%
    При переходе электрона в атоме с одной стационарной орбиты на другую
    излучается фотон с энергией $4{,}04 \cdot 10^{-19}\,\text{Дж}$.
    Какова длина волны этой линии спектра?
    Постоянная Планка $h = 6{,}626 \cdot 10^{-34}\,\text{Дж}\cdot\text{с}$, скорость света $c = 3 \cdot 10^{8}\,\frac{\text{м}}{\text{с}}$.
}
\answer{%
    $
        E = h\nu = h \frac c\lambda
        \implies \lambda = \frac{hc}E
            = \frac{6{,}626 \cdot 10^{-34}\,\text{Дж}\cdot\text{с} \cdot {3 \cdot 10^{8}\,\frac{\text{м}}{\text{с}}}}{4{,}04 \cdot 10^{-19}\,\text{Дж}}
            = 492{,}03\,\text{нм}.
    $
}
\solutionspace{150pt}

\tasknumber{6}%
\task{%
    Излучение какой длины волны поглотил атом водорода, если полная энергия в атоме увеличилась на $4 \cdot 10^{-19}\,\text{Дж}$?
    Постоянная Планка $h = 6{,}626 \cdot 10^{-34}\,\text{Дж}\cdot\text{с}$, скорость света $c = 3 \cdot 10^{8}\,\frac{\text{м}}{\text{с}}$.
}
\answer{%
    $
        E = h\nu = h \frac c\lambda
        \implies \lambda = \frac{hc}E
            = \frac{6{,}626 \cdot 10^{-34}\,\text{Дж}\cdot\text{с} \cdot {3 \cdot 10^{8}\,\frac{\text{м}}{\text{с}}}}{4 \cdot 10^{-19}\,\text{Дж}}
            = 497\,\text{нм}.
    $
}
\solutionspace{150pt}

\tasknumber{7}%
\task{%
    Сделайте схематичный рисунок энергетических уровней атома водорода
    и отметьте на нём первый (основной) уровень и последующие.
    Сколько различных длин волн может испустить атом водорода,
    находящийся в 4-м возбуждённом состоянии?
    Отметьте все соответствующие переходы на рисунке и укажите,
    при каком переходе (среди отмеченных) длина волны излучённого фотона минимальна.
}
\answer{%
    $N = 6{,}0, \text{самая длинная линия}$
}
\solutionspace{150pt}

\tasknumber{8}%
\task{%
    Сколько фотонов испускает за $30\,\text{мин}$ лазер,
    если мощность его излучения $15\,\text{мВт}$?
    Длина волны излучения $500\,\text{нм}$.
    $h = 6{,}626 \cdot 10^{-34}\,\text{Дж}\cdot\text{с}$.
}
\answer{%
    $
        N
            = \frac{E_{\text{общая}}}{E_{\text{одного фотона}}}
            = \frac{Pt}{h\nu} = \frac{Pt}{h \frac c\lambda}
            = \frac{Pt\lambda}{hc}
            = \frac{15\,\text{мВт} \cdot 30\,\text{мин} \cdot 500\,\text{нм}}{6{,}626 \cdot 10^{-34}\,\text{Дж}\cdot\text{с} \cdot 3 \cdot 10^{8}\,\frac{\text{м}}{\text{с}}}
            \approx 0{,}68 \cdot 10^{20}\units{фотонов}
    $
}
\solutionspace{120pt}

\tasknumber{9}%
\task{%
    Какая доля (от начального количества) радиоактивных ядер останется через время,
    равное двум периодам полураспада? Ответ выразить в процентах.
}
\answer{%
    \begin{align*}
    N &= N_0 \cdot 2^{- \frac t{T_{1/2}}} \implies
        \frac N{N_0} = 2^{- \frac t{T_{1/2}}}
        = 2^{-2} \approx 0{,}25 \approx 25\% \\
    N_\text{расп.} &= N_0 - N = N_0 - N_0 \cdot 2^{-\frac t{T_{1/2}}}
        = N_0\cbr{1 - 2^{-\frac t{T_{1/2}}}} \implies
        \frac{N_\text{расп.}}{N_0} = 1 - 2^{-\frac t{T_{1/2}}}
        = 1 - 2^{-2} \approx 0{,}75 \approx 75\%
    \end{align*}
}
\solutionspace{150pt}

\tasknumber{10}%
\task{%
    Сколько процентов ядер радиоактивного железа $\ce{^{59}Fe}$
    останется через $182{,}4\,\text{суток}$, если период его полураспада составляет $45{,}6\,\text{суток}$?
}
\answer{%
    \begin{align*}
    N &= N_0 \cdot 2^{-\frac t{T_{1/2}}}
        = 2^{-\frac{182{,}4\,\text{суток}}{45{,}6\,\text{суток}}}
        \approx 0{,}0625 = 6{,}25\%
    \end{align*}
}
\solutionspace{150pt}

\tasknumber{11}%
\task{%
    За $2\,\text{суток}$ от начального количества ядер радиоизотопа осталась одна шестнадцатая.
    Каков период полураспада этого изотопа (ответ приведите в сутках)?
    Какая ещё доля (также от начального количества) распадётся, если подождать ещё столько же?
}
\answer{%
    \begin{align*}
            N &= N_0 \cdot 2^{-\frac t{T_{1/2}}}
            \implies \frac N{N_0} = 2^{-\frac t{T_{1/2}}}
            \implies \frac 1{16} = 2^{-\frac {2\,\text{суток}}{T_{1/2}}}
            \implies 4 = \frac {2\,\text{суток}}{T_{1/2}}
            \implies T_{1/2} = \frac {2\,\text{суток}}4 \approx 0{,}50\,\text{суток}.
         \\
            \delta &= \frac{N(t)}{N_0} - \frac{N(2t)}{N_0}
            = 2^{-\frac t{T_{1/2}}} - 2^{-\frac {2t}{T_{1/2}}}
            = 2^{-\frac t{T_{1/2}}}\cbr{1 - 2^{-\frac t{T_{1/2}}}}
            = \frac 1{16} \cdot \cbr{1-\frac 1{16}} \approx 0{,}059
    \end{align*}
}
\solutionspace{150pt}

\tasknumber{12}%
\task{%
    Энергия связи ядра азота \ce{^{14}_{7}N} равна $115{,}5\,\text{МэВ}$.
    Найти дефект массы этого ядра.
    Ответ выразите в а.е.м.
    и кг.
    Скорость света $c = 2{,}998 \cdot 10^{8}\,\frac{\text{м}}{\text{с}}$, элементарный заряд $e = 1{,}6 \cdot 10^{-19}\,\text{Кл}$.
}
\answer{%
    \begin{align*}
    E_\text{св.} &= \Delta m c^2 \implies \\
    \implies
            \Delta m &= \frac {E_\text{св.}}{c^2} = \frac{115{,}5\,\text{МэВ}}{\sqr{2{,}998 \cdot 10^{8}\,\frac{\text{м}}{\text{с}}}}
            = \frac{115{,}5 \cdot 10^6 \cdot 1{,}6 \cdot 10^{-19}\,\text{Дж}}{\sqr{2{,}998 \cdot 10^{8}\,\frac{\text{м}}{\text{с}}}}
            \approx 0{,}206 \cdot 10^{-27}\,\text{кг} \approx 0{,}1238\,\text{а.е.м.}
    \end{align*}
}

\variantsplitter

\addpersonalvariant{Фёдор Гнутов}

\tasknumber{1}%
\task{%
    Для частицы, движущейся с релятивистской скоростью,
    выразите $v$ и $E_\text{кин}$ через $c$, $p$ и $E_0$,
    где $E_\text{кин}$~--- кинетическая энергия частицы,
    а $E_0$, $p$ и $v$~--- её энергия покоя импульс и скорость.
}
\answer{%
    \begin{align*}
    E_\text{кин}, E_0:\quad&E = E_\text{кин} + E_0 = \frac{E_0}{\sqrt{1 - \frac{v^2}{c^2}}} \implies \sqrt{1 - \frac{v^2}{c^2}} = \frac{E_0}{{E_0} + {E_\text{кин}}} \implies v = c\sqrt{1 - \sqr{\frac{E_0}{{E_0} + {E_\text{кин}}}}} \\
    &p = \frac{mv}{\sqrt{1 - \frac{v^2}{c^2}}} = \frac{E_0}{c^2} \cdot \sqrt{1 - \sqr{\frac{E_0}{{E_0} + {E_\text{кин}}}}} \cdot \frac{{E_\text{кин}} + {E_0}}{E_0} = \frac{E_0}{c^2} \cdot \sqrt{\sqr{\frac{{E_\text{кин}} + {E_0}}{E_0}} - 1}.
    \\
    E_\text{кин}, p:\quad&E_\text{кин} = E - E_0 = mc^2\cbr{\frac 1{\sqrt{1 - \frac{v^2}{c^2}}} - 1}, p = \frac{mv}{\sqrt{1 - \frac{v^2}{c^2}}} \implies \frac{E_\text{кин}}{p} = \frac{\frac 1{\sqrt{1 - \frac{v^2}{c^2}}} - 1}{\sqrt{1 - \frac{v^2}{c^2}}} \implies v = \ldots \\
    &E_0 = E - E_\text{кин} = \frac{E_0}{\sqrt{1 - \frac{v^2}{c^2}}} - E_\text{кин} \implies E_0 = \frac{E_\text{кин}}{\frac 1{\sqrt{1 - \frac{v^2}{c^2}}} - 1} = \ldots \\
    E_\text{кин}, v:\quad&E_\text{кин} = E - E_0 = mc^2\cbr{\frac 1{\sqrt{1 - \frac{v^2}{c^2}}} - 1} \implies m = \frac{E_\text{кин}}{c^2\cbr{\frac 1{\sqrt{1 - \frac{v^2}{c^2}}} - 1}} \\
    &E_0 = mc^2 = \frac{E_\text{кин}}{\frac 1{\sqrt{1 - \frac{v^2}{c^2}}} - 1} \\
    &p = \frac{mv}{\sqrt{1 - \frac{v^2}{c^2}}} = \frac{E_\text{кин}}{c^2\cbr{\frac 1{\sqrt{1 - \frac{v^2}{c^2}}} - 1}} \cdot \frac{v}{\sqrt{1 - \frac{v^2}{c^2}}} = \frac{{E_\text{кин}} v}{c^2\cbr{1 - {\sqrt{1 - \frac{v^2}{c^2}}}}} \\
    E_0, p:\quad&E_0 = mc^2, \quad p = \frac{mv}{\sqrt{1 - \frac{v^2}{c^2}}} \implies \frac{E_0}{p} = \frac{c^2}v{\sqrt{1 - \frac{v^2}{c^2}}} = c\sqrt{\frac{c^2}{v^2} - 1} \\
    &\sqr{\frac{E_0}{pc}} = \frac{c^2}{v^2} - 1 \implies \frac{v^2}{c^2} = \frac 1{1 + \frac{E_0^2}{p^2c^2}} \implies v = \frac c{\sqrt{1 + \frac{E_0^2}{p^2c^2}}} \\
    &{E_\text{кин}} = E - E_0 = \sqrt{E_0^2 + p^2c^2} - E_0 \\
    E_0, v:\quad&E_0 = mc^2 \implies m = \frac{E_0}{c^2} \qquad p = \frac{mv}{\sqrt{1 - \frac{v^2}{c^2}}} = \frac{E_0}{c^2} \cdot \frac{v}{\sqrt{1 - \frac{v^2}{c^2}}} \\
    &E_\text{кин}= mc^2\cbr{\frac 1{\sqrt{1 - \frac{v^2}{c^2}}} - 1} = \frac{E_0}{c^2}\cbr{\frac 1{\sqrt{1 - \frac{v^2}{c^2}}} - 1} \\
    p, v:\quad&p = \frac{mv}{\sqrt{1 - \frac{v^2}{c^2}}} \implies m = \frac p v {\sqrt{1 - \frac{v^2}{c^2}}} \implies E_0 = mc^2 =\frac {pc^2} v {\sqrt{1 - \frac{v^2}{c^2}}} \\
    &E_\text{кин} = mc^2\cbr{\frac 1{\sqrt{1 - \frac{v^2}{c^2}}} - 1} = \frac p v {\sqrt{1 - \frac{v^2}{c^2}}}\cbr{\frac 1{\sqrt{1 - \frac{v^2}{c^2}}} - 1} = \frac p v \cbr{1 - {\sqrt{1 - \frac{v^2}{c^2}}}}
    \end{align*}
}
\solutionspace{200pt}

\tasknumber{2}%
\task{%
    Электрон движется со скоростью $0{,}7\,c$, где $c$~--- скорость света в вакууме.
    Каково при этом отношение полной энергии частицы $E$ к его энергии покоя $E_0$?
}
\answer{%
    \begin{align*}
    E &= \frac{E_0}{\sqrt{1 - \frac{v^2}{c^2}}}
            \implies \frac E{E_0}
                = \frac 1{\sqrt{1 - \frac{v^2}{c^2}}}
                = \frac 1{\sqrt{1 - \sqr{0{,}7}}}
                \approx 1{,}400,
         \\
        E_{\text{кин}} &= E - E_0
            \implies \frac{E_{\text{кин}}}{E_0}
                = \frac E{E_0} - 1
                = \frac 1{\sqrt{1 - \frac{v^2}{c^2}}} - 1
                = \frac 1{\sqrt{1 - \sqr{0{,}7}}} - 1
                \approx 0{,}400.
    \end{align*}
}
\solutionspace{150pt}

\tasknumber{3}%
\task{%
    Электрон движется со скоростью $0{,}75\,c$, где $c$~--- скорость света в вакууме.
    Определите его импульс (в ответе приведите формулу и укажите численное значение).
}
\answer{%
    \begin{align*}
    E &= \frac{mc^2}{\sqrt{1 - \frac{v^2}{c^2}}}
            \approx \frac{9{,}1 \cdot 10^{-31}\,\text{кг} \cdot \sqr{3 \cdot 10^{8}\,\frac{\text{м}}{\text{с}}}}{\sqrt{1 - 0{,}75^2}}
            \approx 0{,}124 \cdot 10^{-12}\,\text{Дж},
         \\
        E_{\text{кин}} &= \frac{mc^2}{\sqrt{1 - \frac{v^2}{c^2}}} - mc^2
            = mc^2 \cbr{\frac 1{\sqrt{1 - \frac{v^2}{c^2}}} - 1} \approx \\
            &\approx \cbr{9{,}1 \cdot 10^{-31}\,\text{кг} \cdot \sqr{3 \cdot 10^{8}\,\frac{\text{м}}{\text{с}}}}
            \cdot \cbr{\frac 1{\sqrt{1 - 0{,}75^2}} - 1}
            \approx 0{,}042 \cdot 10^{-12}\,\text{Дж},
         \\
        p &= \frac{mv}{\sqrt{1 - \frac{v^2}{c^2}}}
            \approx \frac{9{,}1 \cdot 10^{-31}\,\text{кг} \cdot 0{,}75 \cdot 3 \cdot 10^{8}\,\frac{\text{м}}{\text{с}}}{\sqrt{1 - 0{,}75^2}}
            \approx 0{,}310 \cdot 10^{-21}\,\frac{\text{кг}\cdot\text{м}}{\text{с}}.
    \end{align*}
}
\solutionspace{150pt}

\tasknumber{4}%
\task{%
    При какой скорости движения (в м/с) релятивистское сокращение длины движущегося тела
    составит 50\%?
}
\answer{%
    \begin{align*}
    l_0 &= \frac l{\sqrt{1 - \frac{v^2}{c^2}}}
        \implies 1 - \frac{v^2}{c^2} = \sqr{\frac l{l_0}}
        \implies \frac v c = \sqrt{1 - \sqr{\frac l{l_0}}} \implies
         \\
        \implies v &= c\sqrt{1 - \sqr{\frac l{l_0}}}
        = 3 \cdot 10^{8}\,\frac{\text{м}}{\text{с}} \cdot \sqrt{1 - \sqr{\frac {l_0 - 0{,}50l_0}{l_0}}}
        = 3 \cdot 10^{8}\,\frac{\text{м}}{\text{с}} \cdot \sqrt{1 - \sqr{1 - 0{,}50}} \approx  \\
        &\approx 0{,}866c
        \approx 260 \cdot 10^{6}\,\frac{\text{м}}{\text{с}}
        \approx 935 \cdot 10^{6}\,\frac{\text{км}}{\text{ч}}.
    \end{align*}
}
\solutionspace{150pt}

\tasknumber{5}%
\task{%
    При переходе электрона в атоме с одной стационарной орбиты на другую
    излучается фотон с энергией $0{,}55 \cdot 10^{-19}\,\text{Дж}$.
    Какова длина волны этой линии спектра?
    Постоянная Планка $h = 6{,}626 \cdot 10^{-34}\,\text{Дж}\cdot\text{с}$, скорость света $c = 3 \cdot 10^{8}\,\frac{\text{м}}{\text{с}}$.
}
\answer{%
    $
        E = h\nu = h \frac c\lambda
        \implies \lambda = \frac{hc}E
            = \frac{6{,}626 \cdot 10^{-34}\,\text{Дж}\cdot\text{с} \cdot {3 \cdot 10^{8}\,\frac{\text{м}}{\text{с}}}}{0{,}55 \cdot 10^{-19}\,\text{Дж}}
            = 3614\,\text{нм}.
    $
}
\solutionspace{150pt}

\tasknumber{6}%
\task{%
    Излучение какой длины волны поглотил атом водорода, если полная энергия в атоме увеличилась на $3 \cdot 10^{-19}\,\text{Дж}$?
    Постоянная Планка $h = 6{,}626 \cdot 10^{-34}\,\text{Дж}\cdot\text{с}$, скорость света $c = 3 \cdot 10^{8}\,\frac{\text{м}}{\text{с}}$.
}
\answer{%
    $
        E = h\nu = h \frac c\lambda
        \implies \lambda = \frac{hc}E
            = \frac{6{,}626 \cdot 10^{-34}\,\text{Дж}\cdot\text{с} \cdot {3 \cdot 10^{8}\,\frac{\text{м}}{\text{с}}}}{3 \cdot 10^{-19}\,\text{Дж}}
            = 663\,\text{нм}.
    $
}
\solutionspace{150pt}

\tasknumber{7}%
\task{%
    Сделайте схематичный рисунок энергетических уровней атома водорода
    и отметьте на нём первый (основной) уровень и последующие.
    Сколько различных длин волн может испустить атом водорода,
    находящийся в 3-м возбуждённом состоянии?
    Отметьте все соответствующие переходы на рисунке и укажите,
    при каком переходе (среди отмеченных) длина волны излучённого фотона максимальна.
}
\answer{%
    $N = 3{,}0, \text{самая короткая линия}$
}
\solutionspace{150pt}

\tasknumber{8}%
\task{%
    Сколько фотонов испускает за $40\,\text{мин}$ лазер,
    если мощность его излучения $15\,\text{мВт}$?
    Длина волны излучения $750\,\text{нм}$.
    $h = 6{,}626 \cdot 10^{-34}\,\text{Дж}\cdot\text{с}$.
}
\answer{%
    $
        N
            = \frac{E_{\text{общая}}}{E_{\text{одного фотона}}}
            = \frac{Pt}{h\nu} = \frac{Pt}{h \frac c\lambda}
            = \frac{Pt\lambda}{hc}
            = \frac{15\,\text{мВт} \cdot 40\,\text{мин} \cdot 750\,\text{нм}}{6{,}626 \cdot 10^{-34}\,\text{Дж}\cdot\text{с} \cdot 3 \cdot 10^{8}\,\frac{\text{м}}{\text{с}}}
            \approx 1{,}36 \cdot 10^{20}\units{фотонов}
    $
}
\solutionspace{120pt}

\tasknumber{9}%
\task{%
    Какая доля (от начального количества) радиоактивных ядер останется через время,
    равное трём периодам полураспада? Ответ выразить в процентах.
}
\answer{%
    \begin{align*}
    N &= N_0 \cdot 2^{- \frac t{T_{1/2}}} \implies
        \frac N{N_0} = 2^{- \frac t{T_{1/2}}}
        = 2^{-3} \approx 0{,}12 \approx 12\% \\
    N_\text{расп.} &= N_0 - N = N_0 - N_0 \cdot 2^{-\frac t{T_{1/2}}}
        = N_0\cbr{1 - 2^{-\frac t{T_{1/2}}}} \implies
        \frac{N_\text{расп.}}{N_0} = 1 - 2^{-\frac t{T_{1/2}}}
        = 1 - 2^{-3} \approx 0{,}88 \approx 88\%
    \end{align*}
}
\solutionspace{150pt}

\tasknumber{10}%
\task{%
    Сколько процентов ядер радиоактивного железа $\ce{^{59}Fe}$
    останется через $91{,}2\,\text{суток}$, если период его полураспада составляет $45{,}6\,\text{суток}$?
}
\answer{%
    \begin{align*}
    N &= N_0 \cdot 2^{-\frac t{T_{1/2}}}
        = 2^{-\frac{91{,}2\,\text{суток}}{45{,}6\,\text{суток}}}
        \approx 0{,}2500 = 25{,}00\%
    \end{align*}
}
\solutionspace{150pt}

\tasknumber{11}%
\task{%
    За $3\,\text{суток}$ от начального количества ядер радиоизотопа осталась половина.
    Каков период полураспада этого изотопа (ответ приведите в сутках)?
    Какая ещё доля (также от начального количества) распадётся, если подождать ещё столько же?
}
\answer{%
    \begin{align*}
            N &= N_0 \cdot 2^{-\frac t{T_{1/2}}}
            \implies \frac N{N_0} = 2^{-\frac t{T_{1/2}}}
            \implies \frac 1{2} = 2^{-\frac {3\,\text{суток}}{T_{1/2}}}
            \implies 1 = \frac {3\,\text{суток}}{T_{1/2}}
            \implies T_{1/2} = \frac {3\,\text{суток}}1 \approx 3\,\text{суток}.
         \\
            \delta &= \frac{N(t)}{N_0} - \frac{N(2t)}{N_0}
            = 2^{-\frac t{T_{1/2}}} - 2^{-\frac {2t}{T_{1/2}}}
            = 2^{-\frac t{T_{1/2}}}\cbr{1 - 2^{-\frac t{T_{1/2}}}}
            = \frac 1{2} \cdot \cbr{1-\frac 1{2}} \approx 0{,}250
    \end{align*}
}
\solutionspace{150pt}

\tasknumber{12}%
\task{%
    Энергия связи ядра бериллия \ce{^{9}_{4}Be} равна $58{,}2\,\text{МэВ}$.
    Найти дефект массы этого ядра.
    Ответ выразите в а.е.м.
    и кг.
    Скорость света $c = 2{,}998 \cdot 10^{8}\,\frac{\text{м}}{\text{с}}$, элементарный заряд $e = 1{,}6 \cdot 10^{-19}\,\text{Кл}$.
}
\answer{%
    \begin{align*}
    E_\text{св.} &= \Delta m c^2 \implies \\
    \implies
            \Delta m &= \frac {E_\text{св.}}{c^2} = \frac{58{,}2\,\text{МэВ}}{\sqr{2{,}998 \cdot 10^{8}\,\frac{\text{м}}{\text{с}}}}
            = \frac{58{,}2 \cdot 10^6 \cdot 1{,}6 \cdot 10^{-19}\,\text{Дж}}{\sqr{2{,}998 \cdot 10^{8}\,\frac{\text{м}}{\text{с}}}}
            \approx 0{,}1036 \cdot 10^{-27}\,\text{кг} \approx 0{,}0624\,\text{а.е.м.}
    \end{align*}
}

\variantsplitter

\addpersonalvariant{Илья Гримберг}

\tasknumber{1}%
\task{%
    Для частицы, движущейся с релятивистской скоростью,
    выразите $p$ и $E_0$ через $c$, $v$ и $E_\text{кин}$,
    где $E_\text{кин}$~--- кинетическая энергия частицы,
    а $E_0$, $p$ и $v$~--- её энергия покоя импульс и скорость.
}
\answer{%
    \begin{align*}
    E_\text{кин}, E_0:\quad&E = E_\text{кин} + E_0 = \frac{E_0}{\sqrt{1 - \frac{v^2}{c^2}}} \implies \sqrt{1 - \frac{v^2}{c^2}} = \frac{E_0}{{E_0} + {E_\text{кин}}} \implies v = c\sqrt{1 - \sqr{\frac{E_0}{{E_0} + {E_\text{кин}}}}} \\
    &p = \frac{mv}{\sqrt{1 - \frac{v^2}{c^2}}} = \frac{E_0}{c^2} \cdot \sqrt{1 - \sqr{\frac{E_0}{{E_0} + {E_\text{кин}}}}} \cdot \frac{{E_\text{кин}} + {E_0}}{E_0} = \frac{E_0}{c^2} \cdot \sqrt{\sqr{\frac{{E_\text{кин}} + {E_0}}{E_0}} - 1}.
    \\
    E_\text{кин}, p:\quad&E_\text{кин} = E - E_0 = mc^2\cbr{\frac 1{\sqrt{1 - \frac{v^2}{c^2}}} - 1}, p = \frac{mv}{\sqrt{1 - \frac{v^2}{c^2}}} \implies \frac{E_\text{кин}}{p} = \frac{\frac 1{\sqrt{1 - \frac{v^2}{c^2}}} - 1}{\sqrt{1 - \frac{v^2}{c^2}}} \implies v = \ldots \\
    &E_0 = E - E_\text{кин} = \frac{E_0}{\sqrt{1 - \frac{v^2}{c^2}}} - E_\text{кин} \implies E_0 = \frac{E_\text{кин}}{\frac 1{\sqrt{1 - \frac{v^2}{c^2}}} - 1} = \ldots \\
    E_\text{кин}, v:\quad&E_\text{кин} = E - E_0 = mc^2\cbr{\frac 1{\sqrt{1 - \frac{v^2}{c^2}}} - 1} \implies m = \frac{E_\text{кин}}{c^2\cbr{\frac 1{\sqrt{1 - \frac{v^2}{c^2}}} - 1}} \\
    &E_0 = mc^2 = \frac{E_\text{кин}}{\frac 1{\sqrt{1 - \frac{v^2}{c^2}}} - 1} \\
    &p = \frac{mv}{\sqrt{1 - \frac{v^2}{c^2}}} = \frac{E_\text{кин}}{c^2\cbr{\frac 1{\sqrt{1 - \frac{v^2}{c^2}}} - 1}} \cdot \frac{v}{\sqrt{1 - \frac{v^2}{c^2}}} = \frac{{E_\text{кин}} v}{c^2\cbr{1 - {\sqrt{1 - \frac{v^2}{c^2}}}}} \\
    E_0, p:\quad&E_0 = mc^2, \quad p = \frac{mv}{\sqrt{1 - \frac{v^2}{c^2}}} \implies \frac{E_0}{p} = \frac{c^2}v{\sqrt{1 - \frac{v^2}{c^2}}} = c\sqrt{\frac{c^2}{v^2} - 1} \\
    &\sqr{\frac{E_0}{pc}} = \frac{c^2}{v^2} - 1 \implies \frac{v^2}{c^2} = \frac 1{1 + \frac{E_0^2}{p^2c^2}} \implies v = \frac c{\sqrt{1 + \frac{E_0^2}{p^2c^2}}} \\
    &{E_\text{кин}} = E - E_0 = \sqrt{E_0^2 + p^2c^2} - E_0 \\
    E_0, v:\quad&E_0 = mc^2 \implies m = \frac{E_0}{c^2} \qquad p = \frac{mv}{\sqrt{1 - \frac{v^2}{c^2}}} = \frac{E_0}{c^2} \cdot \frac{v}{\sqrt{1 - \frac{v^2}{c^2}}} \\
    &E_\text{кин}= mc^2\cbr{\frac 1{\sqrt{1 - \frac{v^2}{c^2}}} - 1} = \frac{E_0}{c^2}\cbr{\frac 1{\sqrt{1 - \frac{v^2}{c^2}}} - 1} \\
    p, v:\quad&p = \frac{mv}{\sqrt{1 - \frac{v^2}{c^2}}} \implies m = \frac p v {\sqrt{1 - \frac{v^2}{c^2}}} \implies E_0 = mc^2 =\frac {pc^2} v {\sqrt{1 - \frac{v^2}{c^2}}} \\
    &E_\text{кин} = mc^2\cbr{\frac 1{\sqrt{1 - \frac{v^2}{c^2}}} - 1} = \frac p v {\sqrt{1 - \frac{v^2}{c^2}}}\cbr{\frac 1{\sqrt{1 - \frac{v^2}{c^2}}} - 1} = \frac p v \cbr{1 - {\sqrt{1 - \frac{v^2}{c^2}}}}
    \end{align*}
}
\solutionspace{200pt}

\tasknumber{2}%
\task{%
    Протон движется со скоростью $0{,}6\,c$, где $c$~--- скорость света в вакууме.
    Каково при этом отношение кинетической энергии частицы $E_\text{кин.}$ к его энергии покоя $E_0$?
}
\answer{%
    \begin{align*}
    E &= \frac{E_0}{\sqrt{1 - \frac{v^2}{c^2}}}
            \implies \frac E{E_0}
                = \frac 1{\sqrt{1 - \frac{v^2}{c^2}}}
                = \frac 1{\sqrt{1 - \sqr{0{,}6}}}
                \approx 1{,}250,
         \\
        E_{\text{кин}} &= E - E_0
            \implies \frac{E_{\text{кин}}}{E_0}
                = \frac E{E_0} - 1
                = \frac 1{\sqrt{1 - \frac{v^2}{c^2}}} - 1
                = \frac 1{\sqrt{1 - \sqr{0{,}6}}} - 1
                \approx 0{,}250.
    \end{align*}
}
\solutionspace{150pt}

\tasknumber{3}%
\task{%
    Протон движется со скоростью $0{,}75\,c$, где $c$~--- скорость света в вакууме.
    Определите его кинетическую энергию (в ответе приведите формулу и укажите численное значение).
}
\answer{%
    \begin{align*}
    E &= \frac{mc^2}{\sqrt{1 - \frac{v^2}{c^2}}}
            \approx \frac{1{,}673 \cdot 10^{-27}\,\text{кг} \cdot \sqr{3 \cdot 10^{8}\,\frac{\text{м}}{\text{с}}}}{\sqrt{1 - 0{,}75^2}}
            \approx 227{,}589 \cdot 10^{-12}\,\text{Дж},
         \\
        E_{\text{кин}} &= \frac{mc^2}{\sqrt{1 - \frac{v^2}{c^2}}} - mc^2
            = mc^2 \cbr{\frac 1{\sqrt{1 - \frac{v^2}{c^2}}} - 1} \approx \\
            &\approx \cbr{1{,}673 \cdot 10^{-27}\,\text{кг} \cdot \sqr{3 \cdot 10^{8}\,\frac{\text{м}}{\text{с}}}}
            \cdot \cbr{\frac 1{\sqrt{1 - 0{,}75^2}} - 1}
            \approx 77{,}053 \cdot 10^{-12}\,\text{Дж},
         \\
        p &= \frac{mv}{\sqrt{1 - \frac{v^2}{c^2}}}
            \approx \frac{1{,}673 \cdot 10^{-27}\,\text{кг} \cdot 0{,}75 \cdot 3 \cdot 10^{8}\,\frac{\text{м}}{\text{с}}}{\sqrt{1 - 0{,}75^2}}
            \approx 568{,}972 \cdot 10^{-21}\,\frac{\text{кг}\cdot\text{м}}{\text{с}}.
    \end{align*}
}
\solutionspace{150pt}

\tasknumber{4}%
\task{%
    При какой скорости движения (в м/с) релятивистское сокращение длины движущегося тела
    составит 10\%?
}
\answer{%
    \begin{align*}
    l_0 &= \frac l{\sqrt{1 - \frac{v^2}{c^2}}}
        \implies 1 - \frac{v^2}{c^2} = \sqr{\frac l{l_0}}
        \implies \frac v c = \sqrt{1 - \sqr{\frac l{l_0}}} \implies
         \\
        \implies v &= c\sqrt{1 - \sqr{\frac l{l_0}}}
        = 3 \cdot 10^{8}\,\frac{\text{м}}{\text{с}} \cdot \sqrt{1 - \sqr{\frac {l_0 - 0{,}10l_0}{l_0}}}
        = 3 \cdot 10^{8}\,\frac{\text{м}}{\text{с}} \cdot \sqrt{1 - \sqr{1 - 0{,}10}} \approx  \\
        &\approx 0{,}436c
        \approx 130{,}8 \cdot 10^{6}\,\frac{\text{м}}{\text{с}}
        \approx 471 \cdot 10^{6}\,\frac{\text{км}}{\text{ч}}.
    \end{align*}
}
\solutionspace{150pt}

\tasknumber{5}%
\task{%
    При переходе электрона в атоме с одной стационарной орбиты на другую
    излучается фотон с энергией $7{,}07 \cdot 10^{-19}\,\text{Дж}$.
    Какова длина волны этой линии спектра?
    Постоянная Планка $h = 6{,}626 \cdot 10^{-34}\,\text{Дж}\cdot\text{с}$, скорость света $c = 3 \cdot 10^{8}\,\frac{\text{м}}{\text{с}}$.
}
\answer{%
    $
        E = h\nu = h \frac c\lambda
        \implies \lambda = \frac{hc}E
            = \frac{6{,}626 \cdot 10^{-34}\,\text{Дж}\cdot\text{с} \cdot {3 \cdot 10^{8}\,\frac{\text{м}}{\text{с}}}}{7{,}07 \cdot 10^{-19}\,\text{Дж}}
            = 281{,}16\,\text{нм}.
    $
}
\solutionspace{150pt}

\tasknumber{6}%
\task{%
    Излучение какой длины волны поглотил атом водорода, если полная энергия в атоме увеличилась на $6 \cdot 10^{-19}\,\text{Дж}$?
    Постоянная Планка $h = 6{,}626 \cdot 10^{-34}\,\text{Дж}\cdot\text{с}$, скорость света $c = 3 \cdot 10^{8}\,\frac{\text{м}}{\text{с}}$.
}
\answer{%
    $
        E = h\nu = h \frac c\lambda
        \implies \lambda = \frac{hc}E
            = \frac{6{,}626 \cdot 10^{-34}\,\text{Дж}\cdot\text{с} \cdot {3 \cdot 10^{8}\,\frac{\text{м}}{\text{с}}}}{6 \cdot 10^{-19}\,\text{Дж}}
            = 331\,\text{нм}.
    $
}
\solutionspace{150pt}

\tasknumber{7}%
\task{%
    Сделайте схематичный рисунок энергетических уровней атома водорода
    и отметьте на нём первый (основной) уровень и последующие.
    Сколько различных длин волн может испустить атом водорода,
    находящийся в 5-м возбуждённом состоянии?
    Отметьте все соответствующие переходы на рисунке и укажите,
    при каком переходе (среди отмеченных) длина волны излучённого фотона минимальна.
}
\answer{%
    $N = 10{,}0, \text{самая длинная линия}$
}
\solutionspace{150pt}

\tasknumber{8}%
\task{%
    Сколько фотонов испускает за $40\,\text{мин}$ лазер,
    если мощность его излучения $75\,\text{мВт}$?
    Длина волны излучения $500\,\text{нм}$.
    $h = 6{,}626 \cdot 10^{-34}\,\text{Дж}\cdot\text{с}$.
}
\answer{%
    $
        N
            = \frac{E_{\text{общая}}}{E_{\text{одного фотона}}}
            = \frac{Pt}{h\nu} = \frac{Pt}{h \frac c\lambda}
            = \frac{Pt\lambda}{hc}
            = \frac{75\,\text{мВт} \cdot 40\,\text{мин} \cdot 500\,\text{нм}}{6{,}626 \cdot 10^{-34}\,\text{Дж}\cdot\text{с} \cdot 3 \cdot 10^{8}\,\frac{\text{м}}{\text{с}}}
            \approx 4{,}53 \cdot 10^{20}\units{фотонов}
    $
}
\solutionspace{120pt}

\tasknumber{9}%
\task{%
    Какая доля (от начального количества) радиоактивных ядер останется через время,
    равное двум периодам полураспада? Ответ выразить в процентах.
}
\answer{%
    \begin{align*}
    N &= N_0 \cdot 2^{- \frac t{T_{1/2}}} \implies
        \frac N{N_0} = 2^{- \frac t{T_{1/2}}}
        = 2^{-2} \approx 0{,}25 \approx 25\% \\
    N_\text{расп.} &= N_0 - N = N_0 - N_0 \cdot 2^{-\frac t{T_{1/2}}}
        = N_0\cbr{1 - 2^{-\frac t{T_{1/2}}}} \implies
        \frac{N_\text{расп.}}{N_0} = 1 - 2^{-\frac t{T_{1/2}}}
        = 1 - 2^{-2} \approx 0{,}75 \approx 75\%
    \end{align*}
}
\solutionspace{150pt}

\tasknumber{10}%
\task{%
    Сколько процентов ядер радиоактивного железа $\ce{^{59}Fe}$
    останется через $182{,}4\,\text{суток}$, если период его полураспада составляет $45{,}6\,\text{суток}$?
}
\answer{%
    \begin{align*}
    N &= N_0 \cdot 2^{-\frac t{T_{1/2}}}
        = 2^{-\frac{182{,}4\,\text{суток}}{45{,}6\,\text{суток}}}
        \approx 0{,}0625 = 6{,}25\%
    \end{align*}
}
\solutionspace{150pt}

\tasknumber{11}%
\task{%
    За $5\,\text{суток}$ от начального количества ядер радиоизотопа осталась четверть.
    Каков период полураспада этого изотопа (ответ приведите в сутках)?
    Какая ещё доля (также от начального количества) распадётся, если подождать ещё столько же?
}
\answer{%
    \begin{align*}
            N &= N_0 \cdot 2^{-\frac t{T_{1/2}}}
            \implies \frac N{N_0} = 2^{-\frac t{T_{1/2}}}
            \implies \frac 1{4} = 2^{-\frac {5\,\text{суток}}{T_{1/2}}}
            \implies 2 = \frac {5\,\text{суток}}{T_{1/2}}
            \implies T_{1/2} = \frac {5\,\text{суток}}2 \approx 2{,}5\,\text{суток}.
         \\
            \delta &= \frac{N(t)}{N_0} - \frac{N(2t)}{N_0}
            = 2^{-\frac t{T_{1/2}}} - 2^{-\frac {2t}{T_{1/2}}}
            = 2^{-\frac t{T_{1/2}}}\cbr{1 - 2^{-\frac t{T_{1/2}}}}
            = \frac 1{4} \cdot \cbr{1-\frac 1{4}} \approx 0{,}188
    \end{align*}
}
\solutionspace{150pt}

\tasknumber{12}%
\task{%
    Энергия связи ядра азота \ce{^{14}_{7}N} равна $115{,}5\,\text{МэВ}$.
    Найти дефект массы этого ядра.
    Ответ выразите в а.е.м.
    и кг.
    Скорость света $c = 2{,}998 \cdot 10^{8}\,\frac{\text{м}}{\text{с}}$, элементарный заряд $e = 1{,}6 \cdot 10^{-19}\,\text{Кл}$.
}
\answer{%
    \begin{align*}
    E_\text{св.} &= \Delta m c^2 \implies \\
    \implies
            \Delta m &= \frac {E_\text{св.}}{c^2} = \frac{115{,}5\,\text{МэВ}}{\sqr{2{,}998 \cdot 10^{8}\,\frac{\text{м}}{\text{с}}}}
            = \frac{115{,}5 \cdot 10^6 \cdot 1{,}6 \cdot 10^{-19}\,\text{Дж}}{\sqr{2{,}998 \cdot 10^{8}\,\frac{\text{м}}{\text{с}}}}
            \approx 0{,}206 \cdot 10^{-27}\,\text{кг} \approx 0{,}1238\,\text{а.е.м.}
    \end{align*}
}

\variantsplitter

\addpersonalvariant{Иван Гурьянов}

\tasknumber{1}%
\task{%
    Для частицы, движущейся с релятивистской скоростью,
    выразите $p$ и $E_0$ через $c$, $E_\text{кин}$ и $v$,
    где $E_\text{кин}$~--- кинетическая энергия частицы,
    а $E_0$, $p$ и $v$~--- её энергия покоя импульс и скорость.
}
\answer{%
    \begin{align*}
    E_\text{кин}, E_0:\quad&E = E_\text{кин} + E_0 = \frac{E_0}{\sqrt{1 - \frac{v^2}{c^2}}} \implies \sqrt{1 - \frac{v^2}{c^2}} = \frac{E_0}{{E_0} + {E_\text{кин}}} \implies v = c\sqrt{1 - \sqr{\frac{E_0}{{E_0} + {E_\text{кин}}}}} \\
    &p = \frac{mv}{\sqrt{1 - \frac{v^2}{c^2}}} = \frac{E_0}{c^2} \cdot \sqrt{1 - \sqr{\frac{E_0}{{E_0} + {E_\text{кин}}}}} \cdot \frac{{E_\text{кин}} + {E_0}}{E_0} = \frac{E_0}{c^2} \cdot \sqrt{\sqr{\frac{{E_\text{кин}} + {E_0}}{E_0}} - 1}.
    \\
    E_\text{кин}, p:\quad&E_\text{кин} = E - E_0 = mc^2\cbr{\frac 1{\sqrt{1 - \frac{v^2}{c^2}}} - 1}, p = \frac{mv}{\sqrt{1 - \frac{v^2}{c^2}}} \implies \frac{E_\text{кин}}{p} = \frac{\frac 1{\sqrt{1 - \frac{v^2}{c^2}}} - 1}{\sqrt{1 - \frac{v^2}{c^2}}} \implies v = \ldots \\
    &E_0 = E - E_\text{кин} = \frac{E_0}{\sqrt{1 - \frac{v^2}{c^2}}} - E_\text{кин} \implies E_0 = \frac{E_\text{кин}}{\frac 1{\sqrt{1 - \frac{v^2}{c^2}}} - 1} = \ldots \\
    E_\text{кин}, v:\quad&E_\text{кин} = E - E_0 = mc^2\cbr{\frac 1{\sqrt{1 - \frac{v^2}{c^2}}} - 1} \implies m = \frac{E_\text{кин}}{c^2\cbr{\frac 1{\sqrt{1 - \frac{v^2}{c^2}}} - 1}} \\
    &E_0 = mc^2 = \frac{E_\text{кин}}{\frac 1{\sqrt{1 - \frac{v^2}{c^2}}} - 1} \\
    &p = \frac{mv}{\sqrt{1 - \frac{v^2}{c^2}}} = \frac{E_\text{кин}}{c^2\cbr{\frac 1{\sqrt{1 - \frac{v^2}{c^2}}} - 1}} \cdot \frac{v}{\sqrt{1 - \frac{v^2}{c^2}}} = \frac{{E_\text{кин}} v}{c^2\cbr{1 - {\sqrt{1 - \frac{v^2}{c^2}}}}} \\
    E_0, p:\quad&E_0 = mc^2, \quad p = \frac{mv}{\sqrt{1 - \frac{v^2}{c^2}}} \implies \frac{E_0}{p} = \frac{c^2}v{\sqrt{1 - \frac{v^2}{c^2}}} = c\sqrt{\frac{c^2}{v^2} - 1} \\
    &\sqr{\frac{E_0}{pc}} = \frac{c^2}{v^2} - 1 \implies \frac{v^2}{c^2} = \frac 1{1 + \frac{E_0^2}{p^2c^2}} \implies v = \frac c{\sqrt{1 + \frac{E_0^2}{p^2c^2}}} \\
    &{E_\text{кин}} = E - E_0 = \sqrt{E_0^2 + p^2c^2} - E_0 \\
    E_0, v:\quad&E_0 = mc^2 \implies m = \frac{E_0}{c^2} \qquad p = \frac{mv}{\sqrt{1 - \frac{v^2}{c^2}}} = \frac{E_0}{c^2} \cdot \frac{v}{\sqrt{1 - \frac{v^2}{c^2}}} \\
    &E_\text{кин}= mc^2\cbr{\frac 1{\sqrt{1 - \frac{v^2}{c^2}}} - 1} = \frac{E_0}{c^2}\cbr{\frac 1{\sqrt{1 - \frac{v^2}{c^2}}} - 1} \\
    p, v:\quad&p = \frac{mv}{\sqrt{1 - \frac{v^2}{c^2}}} \implies m = \frac p v {\sqrt{1 - \frac{v^2}{c^2}}} \implies E_0 = mc^2 =\frac {pc^2} v {\sqrt{1 - \frac{v^2}{c^2}}} \\
    &E_\text{кин} = mc^2\cbr{\frac 1{\sqrt{1 - \frac{v^2}{c^2}}} - 1} = \frac p v {\sqrt{1 - \frac{v^2}{c^2}}}\cbr{\frac 1{\sqrt{1 - \frac{v^2}{c^2}}} - 1} = \frac p v \cbr{1 - {\sqrt{1 - \frac{v^2}{c^2}}}}
    \end{align*}
}
\solutionspace{200pt}

\tasknumber{2}%
\task{%
    Протон движется со скоростью $0{,}6\,c$, где $c$~--- скорость света в вакууме.
    Каково при этом отношение кинетической энергии частицы $E_\text{кин.}$ к его энергии покоя $E_0$?
}
\answer{%
    \begin{align*}
    E &= \frac{E_0}{\sqrt{1 - \frac{v^2}{c^2}}}
            \implies \frac E{E_0}
                = \frac 1{\sqrt{1 - \frac{v^2}{c^2}}}
                = \frac 1{\sqrt{1 - \sqr{0{,}6}}}
                \approx 1{,}250,
         \\
        E_{\text{кин}} &= E - E_0
            \implies \frac{E_{\text{кин}}}{E_0}
                = \frac E{E_0} - 1
                = \frac 1{\sqrt{1 - \frac{v^2}{c^2}}} - 1
                = \frac 1{\sqrt{1 - \sqr{0{,}6}}} - 1
                \approx 0{,}250.
    \end{align*}
}
\solutionspace{150pt}

\tasknumber{3}%
\task{%
    Электрон движется со скоростью $0{,}65\,c$, где $c$~--- скорость света в вакууме.
    Определите его полную энергию (в ответе приведите формулу и укажите численное значение).
}
\answer{%
    \begin{align*}
    E &= \frac{mc^2}{\sqrt{1 - \frac{v^2}{c^2}}}
            \approx \frac{9{,}1 \cdot 10^{-31}\,\text{кг} \cdot \sqr{3 \cdot 10^{8}\,\frac{\text{м}}{\text{с}}}}{\sqrt{1 - 0{,}65^2}}
            \approx 0{,}108 \cdot 10^{-12}\,\text{Дж},
         \\
        E_{\text{кин}} &= \frac{mc^2}{\sqrt{1 - \frac{v^2}{c^2}}} - mc^2
            = mc^2 \cbr{\frac 1{\sqrt{1 - \frac{v^2}{c^2}}} - 1} \approx \\
            &\approx \cbr{9{,}1 \cdot 10^{-31}\,\text{кг} \cdot \sqr{3 \cdot 10^{8}\,\frac{\text{м}}{\text{с}}}}
            \cdot \cbr{\frac 1{\sqrt{1 - 0{,}65^2}} - 1}
            \approx 0{,}026 \cdot 10^{-12}\,\text{Дж},
         \\
        p &= \frac{mv}{\sqrt{1 - \frac{v^2}{c^2}}}
            \approx \frac{9{,}1 \cdot 10^{-31}\,\text{кг} \cdot 0{,}65 \cdot 3 \cdot 10^{8}\,\frac{\text{м}}{\text{с}}}{\sqrt{1 - 0{,}65^2}}
            \approx 0{,}234 \cdot 10^{-21}\,\frac{\text{кг}\cdot\text{м}}{\text{с}}.
    \end{align*}
}
\solutionspace{150pt}

\tasknumber{4}%
\task{%
    При какой скорости движения (в м/с) релятивистское сокращение длины движущегося тела
    составит 10\%?
}
\answer{%
    \begin{align*}
    l_0 &= \frac l{\sqrt{1 - \frac{v^2}{c^2}}}
        \implies 1 - \frac{v^2}{c^2} = \sqr{\frac l{l_0}}
        \implies \frac v c = \sqrt{1 - \sqr{\frac l{l_0}}} \implies
         \\
        \implies v &= c\sqrt{1 - \sqr{\frac l{l_0}}}
        = 3 \cdot 10^{8}\,\frac{\text{м}}{\text{с}} \cdot \sqrt{1 - \sqr{\frac {l_0 - 0{,}10l_0}{l_0}}}
        = 3 \cdot 10^{8}\,\frac{\text{м}}{\text{с}} \cdot \sqrt{1 - \sqr{1 - 0{,}10}} \approx  \\
        &\approx 0{,}436c
        \approx 130{,}8 \cdot 10^{6}\,\frac{\text{м}}{\text{с}}
        \approx 471 \cdot 10^{6}\,\frac{\text{км}}{\text{ч}}.
    \end{align*}
}
\solutionspace{150pt}

\tasknumber{5}%
\task{%
    При переходе электрона в атоме с одной стационарной орбиты на другую
    излучается фотон с энергией $7{,}07 \cdot 10^{-19}\,\text{Дж}$.
    Какова длина волны этой линии спектра?
    Постоянная Планка $h = 6{,}626 \cdot 10^{-34}\,\text{Дж}\cdot\text{с}$, скорость света $c = 3 \cdot 10^{8}\,\frac{\text{м}}{\text{с}}$.
}
\answer{%
    $
        E = h\nu = h \frac c\lambda
        \implies \lambda = \frac{hc}E
            = \frac{6{,}626 \cdot 10^{-34}\,\text{Дж}\cdot\text{с} \cdot {3 \cdot 10^{8}\,\frac{\text{м}}{\text{с}}}}{7{,}07 \cdot 10^{-19}\,\text{Дж}}
            = 281{,}16\,\text{нм}.
    $
}
\solutionspace{150pt}

\tasknumber{6}%
\task{%
    Излучение какой длины волны поглотил атом водорода, если полная энергия в атоме увеличилась на $6 \cdot 10^{-19}\,\text{Дж}$?
    Постоянная Планка $h = 6{,}626 \cdot 10^{-34}\,\text{Дж}\cdot\text{с}$, скорость света $c = 3 \cdot 10^{8}\,\frac{\text{м}}{\text{с}}$.
}
\answer{%
    $
        E = h\nu = h \frac c\lambda
        \implies \lambda = \frac{hc}E
            = \frac{6{,}626 \cdot 10^{-34}\,\text{Дж}\cdot\text{с} \cdot {3 \cdot 10^{8}\,\frac{\text{м}}{\text{с}}}}{6 \cdot 10^{-19}\,\text{Дж}}
            = 331\,\text{нм}.
    $
}
\solutionspace{150pt}

\tasknumber{7}%
\task{%
    Сделайте схематичный рисунок энергетических уровней атома водорода
    и отметьте на нём первый (основной) уровень и последующие.
    Сколько различных длин волн может испустить атом водорода,
    находящийся в 3-м возбуждённом состоянии?
    Отметьте все соответствующие переходы на рисунке и укажите,
    при каком переходе (среди отмеченных) частота излучённого фотона максимальна.
}
\answer{%
    $N = 3{,}0, \text{самая длинная линия}$
}
\solutionspace{150pt}

\tasknumber{8}%
\task{%
    Сколько фотонов испускает за $20\,\text{мин}$ лазер,
    если мощность его излучения $200\,\text{мВт}$?
    Длина волны излучения $500\,\text{нм}$.
    $h = 6{,}626 \cdot 10^{-34}\,\text{Дж}\cdot\text{с}$.
}
\answer{%
    $
        N
            = \frac{E_{\text{общая}}}{E_{\text{одного фотона}}}
            = \frac{Pt}{h\nu} = \frac{Pt}{h \frac c\lambda}
            = \frac{Pt\lambda}{hc}
            = \frac{200\,\text{мВт} \cdot 20\,\text{мин} \cdot 500\,\text{нм}}{6{,}626 \cdot 10^{-34}\,\text{Дж}\cdot\text{с} \cdot 3 \cdot 10^{8}\,\frac{\text{м}}{\text{с}}}
            \approx 6{,}04 \cdot 10^{20}\units{фотонов}
    $
}
\solutionspace{120pt}

\tasknumber{9}%
\task{%
    Какая доля (от начального количества) радиоактивных ядер останется через время,
    равное трём периодам полураспада? Ответ выразить в процентах.
}
\answer{%
    \begin{align*}
    N &= N_0 \cdot 2^{- \frac t{T_{1/2}}} \implies
        \frac N{N_0} = 2^{- \frac t{T_{1/2}}}
        = 2^{-3} \approx 0{,}12 \approx 12\% \\
    N_\text{расп.} &= N_0 - N = N_0 - N_0 \cdot 2^{-\frac t{T_{1/2}}}
        = N_0\cbr{1 - 2^{-\frac t{T_{1/2}}}} \implies
        \frac{N_\text{расп.}}{N_0} = 1 - 2^{-\frac t{T_{1/2}}}
        = 1 - 2^{-3} \approx 0{,}88 \approx 88\%
    \end{align*}
}
\solutionspace{150pt}

\tasknumber{10}%
\task{%
    Сколько процентов ядер радиоактивного железа $\ce{^{59}Fe}$
    останется через $136{,}8\,\text{суток}$, если период его полураспада составляет $45{,}6\,\text{суток}$?
}
\answer{%
    \begin{align*}
    N &= N_0 \cdot 2^{-\frac t{T_{1/2}}}
        = 2^{-\frac{136{,}8\,\text{суток}}{45{,}6\,\text{суток}}}
        \approx 0{,}1250 = 12{,}50\%
    \end{align*}
}
\solutionspace{150pt}

\tasknumber{11}%
\task{%
    За $2\,\text{суток}$ от начального количества ядер радиоизотопа осталась половина.
    Каков период полураспада этого изотопа (ответ приведите в сутках)?
    Какая ещё доля (также от начального количества) распадётся, если подождать ещё столько же?
}
\answer{%
    \begin{align*}
            N &= N_0 \cdot 2^{-\frac t{T_{1/2}}}
            \implies \frac N{N_0} = 2^{-\frac t{T_{1/2}}}
            \implies \frac 1{2} = 2^{-\frac {2\,\text{суток}}{T_{1/2}}}
            \implies 1 = \frac {2\,\text{суток}}{T_{1/2}}
            \implies T_{1/2} = \frac {2\,\text{суток}}1 \approx 2\,\text{суток}.
         \\
            \delta &= \frac{N(t)}{N_0} - \frac{N(2t)}{N_0}
            = 2^{-\frac t{T_{1/2}}} - 2^{-\frac {2t}{T_{1/2}}}
            = 2^{-\frac t{T_{1/2}}}\cbr{1 - 2^{-\frac t{T_{1/2}}}}
            = \frac 1{2} \cdot \cbr{1-\frac 1{2}} \approx 0{,}250
    \end{align*}
}
\solutionspace{150pt}

\tasknumber{12}%
\task{%
    Энергия связи ядра лития \ce{^{6}_{3}Li} равна $31{,}99\,\text{МэВ}$.
    Найти дефект массы этого ядра.
    Ответ выразите в а.е.м.
    и кг.
    Скорость света $c = 2{,}998 \cdot 10^{8}\,\frac{\text{м}}{\text{с}}$, элементарный заряд $e = 1{,}6 \cdot 10^{-19}\,\text{Кл}$.
}
\answer{%
    \begin{align*}
    E_\text{св.} &= \Delta m c^2 \implies \\
    \implies
            \Delta m &= \frac {E_\text{св.}}{c^2} = \frac{31{,}99\,\text{МэВ}}{\sqr{2{,}998 \cdot 10^{8}\,\frac{\text{м}}{\text{с}}}}
            = \frac{31{,}99 \cdot 10^6 \cdot 1{,}6 \cdot 10^{-19}\,\text{Дж}}{\sqr{2{,}998 \cdot 10^{8}\,\frac{\text{м}}{\text{с}}}}
            \approx 56{,}95 \cdot 10^{-30}\,\text{кг} \approx 0{,}03429\,\text{а.е.м.}
    \end{align*}
}

\variantsplitter

\addpersonalvariant{Артём Денежкин}

\tasknumber{1}%
\task{%
    Для частицы, движущейся с релятивистской скоростью,
    выразите $v$ и $p$ через $c$, $E_0$ и $E_\text{кин}$,
    где $E_\text{кин}$~--- кинетическая энергия частицы,
    а $E_0$, $p$ и $v$~--- её энергия покоя импульс и скорость.
}
\answer{%
    \begin{align*}
    E_\text{кин}, E_0:\quad&E = E_\text{кин} + E_0 = \frac{E_0}{\sqrt{1 - \frac{v^2}{c^2}}} \implies \sqrt{1 - \frac{v^2}{c^2}} = \frac{E_0}{{E_0} + {E_\text{кин}}} \implies v = c\sqrt{1 - \sqr{\frac{E_0}{{E_0} + {E_\text{кин}}}}} \\
    &p = \frac{mv}{\sqrt{1 - \frac{v^2}{c^2}}} = \frac{E_0}{c^2} \cdot \sqrt{1 - \sqr{\frac{E_0}{{E_0} + {E_\text{кин}}}}} \cdot \frac{{E_\text{кин}} + {E_0}}{E_0} = \frac{E_0}{c^2} \cdot \sqrt{\sqr{\frac{{E_\text{кин}} + {E_0}}{E_0}} - 1}.
    \\
    E_\text{кин}, p:\quad&E_\text{кин} = E - E_0 = mc^2\cbr{\frac 1{\sqrt{1 - \frac{v^2}{c^2}}} - 1}, p = \frac{mv}{\sqrt{1 - \frac{v^2}{c^2}}} \implies \frac{E_\text{кин}}{p} = \frac{\frac 1{\sqrt{1 - \frac{v^2}{c^2}}} - 1}{\sqrt{1 - \frac{v^2}{c^2}}} \implies v = \ldots \\
    &E_0 = E - E_\text{кин} = \frac{E_0}{\sqrt{1 - \frac{v^2}{c^2}}} - E_\text{кин} \implies E_0 = \frac{E_\text{кин}}{\frac 1{\sqrt{1 - \frac{v^2}{c^2}}} - 1} = \ldots \\
    E_\text{кин}, v:\quad&E_\text{кин} = E - E_0 = mc^2\cbr{\frac 1{\sqrt{1 - \frac{v^2}{c^2}}} - 1} \implies m = \frac{E_\text{кин}}{c^2\cbr{\frac 1{\sqrt{1 - \frac{v^2}{c^2}}} - 1}} \\
    &E_0 = mc^2 = \frac{E_\text{кин}}{\frac 1{\sqrt{1 - \frac{v^2}{c^2}}} - 1} \\
    &p = \frac{mv}{\sqrt{1 - \frac{v^2}{c^2}}} = \frac{E_\text{кин}}{c^2\cbr{\frac 1{\sqrt{1 - \frac{v^2}{c^2}}} - 1}} \cdot \frac{v}{\sqrt{1 - \frac{v^2}{c^2}}} = \frac{{E_\text{кин}} v}{c^2\cbr{1 - {\sqrt{1 - \frac{v^2}{c^2}}}}} \\
    E_0, p:\quad&E_0 = mc^2, \quad p = \frac{mv}{\sqrt{1 - \frac{v^2}{c^2}}} \implies \frac{E_0}{p} = \frac{c^2}v{\sqrt{1 - \frac{v^2}{c^2}}} = c\sqrt{\frac{c^2}{v^2} - 1} \\
    &\sqr{\frac{E_0}{pc}} = \frac{c^2}{v^2} - 1 \implies \frac{v^2}{c^2} = \frac 1{1 + \frac{E_0^2}{p^2c^2}} \implies v = \frac c{\sqrt{1 + \frac{E_0^2}{p^2c^2}}} \\
    &{E_\text{кин}} = E - E_0 = \sqrt{E_0^2 + p^2c^2} - E_0 \\
    E_0, v:\quad&E_0 = mc^2 \implies m = \frac{E_0}{c^2} \qquad p = \frac{mv}{\sqrt{1 - \frac{v^2}{c^2}}} = \frac{E_0}{c^2} \cdot \frac{v}{\sqrt{1 - \frac{v^2}{c^2}}} \\
    &E_\text{кин}= mc^2\cbr{\frac 1{\sqrt{1 - \frac{v^2}{c^2}}} - 1} = \frac{E_0}{c^2}\cbr{\frac 1{\sqrt{1 - \frac{v^2}{c^2}}} - 1} \\
    p, v:\quad&p = \frac{mv}{\sqrt{1 - \frac{v^2}{c^2}}} \implies m = \frac p v {\sqrt{1 - \frac{v^2}{c^2}}} \implies E_0 = mc^2 =\frac {pc^2} v {\sqrt{1 - \frac{v^2}{c^2}}} \\
    &E_\text{кин} = mc^2\cbr{\frac 1{\sqrt{1 - \frac{v^2}{c^2}}} - 1} = \frac p v {\sqrt{1 - \frac{v^2}{c^2}}}\cbr{\frac 1{\sqrt{1 - \frac{v^2}{c^2}}} - 1} = \frac p v \cbr{1 - {\sqrt{1 - \frac{v^2}{c^2}}}}
    \end{align*}
}
\solutionspace{200pt}

\tasknumber{2}%
\task{%
    Электрон движется со скоростью $0{,}6\,c$, где $c$~--- скорость света в вакууме.
    Каково при этом отношение полной энергии частицы $E$ к его энергии покоя $E_0$?
}
\answer{%
    \begin{align*}
    E &= \frac{E_0}{\sqrt{1 - \frac{v^2}{c^2}}}
            \implies \frac E{E_0}
                = \frac 1{\sqrt{1 - \frac{v^2}{c^2}}}
                = \frac 1{\sqrt{1 - \sqr{0{,}6}}}
                \approx 1{,}250,
         \\
        E_{\text{кин}} &= E - E_0
            \implies \frac{E_{\text{кин}}}{E_0}
                = \frac E{E_0} - 1
                = \frac 1{\sqrt{1 - \frac{v^2}{c^2}}} - 1
                = \frac 1{\sqrt{1 - \sqr{0{,}6}}} - 1
                \approx 0{,}250.
    \end{align*}
}
\solutionspace{150pt}

\tasknumber{3}%
\task{%
    Электрон движется со скоростью $0{,}85\,c$, где $c$~--- скорость света в вакууме.
    Определите его полную энергию (в ответе приведите формулу и укажите численное значение).
}
\answer{%
    \begin{align*}
    E &= \frac{mc^2}{\sqrt{1 - \frac{v^2}{c^2}}}
            \approx \frac{9{,}1 \cdot 10^{-31}\,\text{кг} \cdot \sqr{3 \cdot 10^{8}\,\frac{\text{м}}{\text{с}}}}{\sqrt{1 - 0{,}85^2}}
            \approx 0{,}155 \cdot 10^{-12}\,\text{Дж},
         \\
        E_{\text{кин}} &= \frac{mc^2}{\sqrt{1 - \frac{v^2}{c^2}}} - mc^2
            = mc^2 \cbr{\frac 1{\sqrt{1 - \frac{v^2}{c^2}}} - 1} \approx \\
            &\approx \cbr{9{,}1 \cdot 10^{-31}\,\text{кг} \cdot \sqr{3 \cdot 10^{8}\,\frac{\text{м}}{\text{с}}}}
            \cdot \cbr{\frac 1{\sqrt{1 - 0{,}85^2}} - 1}
            \approx 0{,}074 \cdot 10^{-12}\,\text{Дж},
         \\
        p &= \frac{mv}{\sqrt{1 - \frac{v^2}{c^2}}}
            \approx \frac{9{,}1 \cdot 10^{-31}\,\text{кг} \cdot 0{,}85 \cdot 3 \cdot 10^{8}\,\frac{\text{м}}{\text{с}}}{\sqrt{1 - 0{,}85^2}}
            \approx 0{,}441 \cdot 10^{-21}\,\frac{\text{кг}\cdot\text{м}}{\text{с}}.
    \end{align*}
}
\solutionspace{150pt}

\tasknumber{4}%
\task{%
    При какой скорости движения (в м/с) релятивистское сокращение длины движущегося тела
    составит 10\%?
}
\answer{%
    \begin{align*}
    l_0 &= \frac l{\sqrt{1 - \frac{v^2}{c^2}}}
        \implies 1 - \frac{v^2}{c^2} = \sqr{\frac l{l_0}}
        \implies \frac v c = \sqrt{1 - \sqr{\frac l{l_0}}} \implies
         \\
        \implies v &= c\sqrt{1 - \sqr{\frac l{l_0}}}
        = 3 \cdot 10^{8}\,\frac{\text{м}}{\text{с}} \cdot \sqrt{1 - \sqr{\frac {l_0 - 0{,}10l_0}{l_0}}}
        = 3 \cdot 10^{8}\,\frac{\text{м}}{\text{с}} \cdot \sqrt{1 - \sqr{1 - 0{,}10}} \approx  \\
        &\approx 0{,}436c
        \approx 130{,}8 \cdot 10^{6}\,\frac{\text{м}}{\text{с}}
        \approx 471 \cdot 10^{6}\,\frac{\text{км}}{\text{ч}}.
    \end{align*}
}
\solutionspace{150pt}

\tasknumber{5}%
\task{%
    При переходе электрона в атоме с одной стационарной орбиты на другую
    излучается фотон с энергией $0{,}55 \cdot 10^{-19}\,\text{Дж}$.
    Какова длина волны этой линии спектра?
    Постоянная Планка $h = 6{,}626 \cdot 10^{-34}\,\text{Дж}\cdot\text{с}$, скорость света $c = 3 \cdot 10^{8}\,\frac{\text{м}}{\text{с}}$.
}
\answer{%
    $
        E = h\nu = h \frac c\lambda
        \implies \lambda = \frac{hc}E
            = \frac{6{,}626 \cdot 10^{-34}\,\text{Дж}\cdot\text{с} \cdot {3 \cdot 10^{8}\,\frac{\text{м}}{\text{с}}}}{0{,}55 \cdot 10^{-19}\,\text{Дж}}
            = 3614\,\text{нм}.
    $
}
\solutionspace{150pt}

\tasknumber{6}%
\task{%
    Излучение какой длины волны поглотил атом водорода, если полная энергия в атоме увеличилась на $6 \cdot 10^{-19}\,\text{Дж}$?
    Постоянная Планка $h = 6{,}626 \cdot 10^{-34}\,\text{Дж}\cdot\text{с}$, скорость света $c = 3 \cdot 10^{8}\,\frac{\text{м}}{\text{с}}$.
}
\answer{%
    $
        E = h\nu = h \frac c\lambda
        \implies \lambda = \frac{hc}E
            = \frac{6{,}626 \cdot 10^{-34}\,\text{Дж}\cdot\text{с} \cdot {3 \cdot 10^{8}\,\frac{\text{м}}{\text{с}}}}{6 \cdot 10^{-19}\,\text{Дж}}
            = 331\,\text{нм}.
    $
}
\solutionspace{150pt}

\tasknumber{7}%
\task{%
    Сделайте схематичный рисунок энергетических уровней атома водорода
    и отметьте на нём первый (основной) уровень и последующие.
    Сколько различных длин волн может испустить атом водорода,
    находящийся в 3-м возбуждённом состоянии?
    Отметьте все соответствующие переходы на рисунке и укажите,
    при каком переходе (среди отмеченных) энергия излучённого фотона максимальна.
}
\answer{%
    $N = 3{,}0, \text{самая длинная линия}$
}
\solutionspace{150pt}

\tasknumber{8}%
\task{%
    Сколько фотонов испускает за $5\,\text{мин}$ лазер,
    если мощность его излучения $15\,\text{мВт}$?
    Длина волны излучения $500\,\text{нм}$.
    $h = 6{,}626 \cdot 10^{-34}\,\text{Дж}\cdot\text{с}$.
}
\answer{%
    $
        N
            = \frac{E_{\text{общая}}}{E_{\text{одного фотона}}}
            = \frac{Pt}{h\nu} = \frac{Pt}{h \frac c\lambda}
            = \frac{Pt\lambda}{hc}
            = \frac{15\,\text{мВт} \cdot 5\,\text{мин} \cdot 500\,\text{нм}}{6{,}626 \cdot 10^{-34}\,\text{Дж}\cdot\text{с} \cdot 3 \cdot 10^{8}\,\frac{\text{м}}{\text{с}}}
            \approx 0{,}11 \cdot 10^{20}\units{фотонов}
    $
}
\solutionspace{120pt}

\tasknumber{9}%
\task{%
    Какая доля (от начального количества) радиоактивных ядер останется через время,
    равное трём периодам полураспада? Ответ выразить в процентах.
}
\answer{%
    \begin{align*}
    N &= N_0 \cdot 2^{- \frac t{T_{1/2}}} \implies
        \frac N{N_0} = 2^{- \frac t{T_{1/2}}}
        = 2^{-3} \approx 0{,}12 \approx 12\% \\
    N_\text{расп.} &= N_0 - N = N_0 - N_0 \cdot 2^{-\frac t{T_{1/2}}}
        = N_0\cbr{1 - 2^{-\frac t{T_{1/2}}}} \implies
        \frac{N_\text{расп.}}{N_0} = 1 - 2^{-\frac t{T_{1/2}}}
        = 1 - 2^{-3} \approx 0{,}88 \approx 88\%
    \end{align*}
}
\solutionspace{150pt}

\tasknumber{10}%
\task{%
    Сколько процентов ядер радиоактивного железа $\ce{^{59}Fe}$
    останется через $91{,}2\,\text{суток}$, если период его полураспада составляет $45{,}6\,\text{суток}$?
}
\answer{%
    \begin{align*}
    N &= N_0 \cdot 2^{-\frac t{T_{1/2}}}
        = 2^{-\frac{91{,}2\,\text{суток}}{45{,}6\,\text{суток}}}
        \approx 0{,}2500 = 25{,}00\%
    \end{align*}
}
\solutionspace{150pt}

\tasknumber{11}%
\task{%
    За $3\,\text{суток}$ от начального количества ядер радиоизотопа осталась половина.
    Каков период полураспада этого изотопа (ответ приведите в сутках)?
    Какая ещё доля (также от начального количества) распадётся, если подождать ещё столько же?
}
\answer{%
    \begin{align*}
            N &= N_0 \cdot 2^{-\frac t{T_{1/2}}}
            \implies \frac N{N_0} = 2^{-\frac t{T_{1/2}}}
            \implies \frac 1{2} = 2^{-\frac {3\,\text{суток}}{T_{1/2}}}
            \implies 1 = \frac {3\,\text{суток}}{T_{1/2}}
            \implies T_{1/2} = \frac {3\,\text{суток}}1 \approx 3\,\text{суток}.
         \\
            \delta &= \frac{N(t)}{N_0} - \frac{N(2t)}{N_0}
            = 2^{-\frac t{T_{1/2}}} - 2^{-\frac {2t}{T_{1/2}}}
            = 2^{-\frac t{T_{1/2}}}\cbr{1 - 2^{-\frac t{T_{1/2}}}}
            = \frac 1{2} \cdot \cbr{1-\frac 1{2}} \approx 0{,}250
    \end{align*}
}
\solutionspace{150pt}

\tasknumber{12}%
\task{%
    Энергия связи ядра азота \ce{^{14}_{7}N} равна $115{,}5\,\text{МэВ}$.
    Найти дефект массы этого ядра.
    Ответ выразите в а.е.м.
    и кг.
    Скорость света $c = 2{,}998 \cdot 10^{8}\,\frac{\text{м}}{\text{с}}$, элементарный заряд $e = 1{,}6 \cdot 10^{-19}\,\text{Кл}$.
}
\answer{%
    \begin{align*}
    E_\text{св.} &= \Delta m c^2 \implies \\
    \implies
            \Delta m &= \frac {E_\text{св.}}{c^2} = \frac{115{,}5\,\text{МэВ}}{\sqr{2{,}998 \cdot 10^{8}\,\frac{\text{м}}{\text{с}}}}
            = \frac{115{,}5 \cdot 10^6 \cdot 1{,}6 \cdot 10^{-19}\,\text{Дж}}{\sqr{2{,}998 \cdot 10^{8}\,\frac{\text{м}}{\text{с}}}}
            \approx 0{,}206 \cdot 10^{-27}\,\text{кг} \approx 0{,}1238\,\text{а.е.м.}
    \end{align*}
}

\variantsplitter

\addpersonalvariant{Виктор Жилин}

\tasknumber{1}%
\task{%
    Для частицы, движущейся с релятивистской скоростью,
    выразите $p$ и $E_0$ через $c$, $E_\text{кин}$ и $v$,
    где $E_\text{кин}$~--- кинетическая энергия частицы,
    а $E_0$, $p$ и $v$~--- её энергия покоя импульс и скорость.
}
\answer{%
    \begin{align*}
    E_\text{кин}, E_0:\quad&E = E_\text{кин} + E_0 = \frac{E_0}{\sqrt{1 - \frac{v^2}{c^2}}} \implies \sqrt{1 - \frac{v^2}{c^2}} = \frac{E_0}{{E_0} + {E_\text{кин}}} \implies v = c\sqrt{1 - \sqr{\frac{E_0}{{E_0} + {E_\text{кин}}}}} \\
    &p = \frac{mv}{\sqrt{1 - \frac{v^2}{c^2}}} = \frac{E_0}{c^2} \cdot \sqrt{1 - \sqr{\frac{E_0}{{E_0} + {E_\text{кин}}}}} \cdot \frac{{E_\text{кин}} + {E_0}}{E_0} = \frac{E_0}{c^2} \cdot \sqrt{\sqr{\frac{{E_\text{кин}} + {E_0}}{E_0}} - 1}.
    \\
    E_\text{кин}, p:\quad&E_\text{кин} = E - E_0 = mc^2\cbr{\frac 1{\sqrt{1 - \frac{v^2}{c^2}}} - 1}, p = \frac{mv}{\sqrt{1 - \frac{v^2}{c^2}}} \implies \frac{E_\text{кин}}{p} = \frac{\frac 1{\sqrt{1 - \frac{v^2}{c^2}}} - 1}{\sqrt{1 - \frac{v^2}{c^2}}} \implies v = \ldots \\
    &E_0 = E - E_\text{кин} = \frac{E_0}{\sqrt{1 - \frac{v^2}{c^2}}} - E_\text{кин} \implies E_0 = \frac{E_\text{кин}}{\frac 1{\sqrt{1 - \frac{v^2}{c^2}}} - 1} = \ldots \\
    E_\text{кин}, v:\quad&E_\text{кин} = E - E_0 = mc^2\cbr{\frac 1{\sqrt{1 - \frac{v^2}{c^2}}} - 1} \implies m = \frac{E_\text{кин}}{c^2\cbr{\frac 1{\sqrt{1 - \frac{v^2}{c^2}}} - 1}} \\
    &E_0 = mc^2 = \frac{E_\text{кин}}{\frac 1{\sqrt{1 - \frac{v^2}{c^2}}} - 1} \\
    &p = \frac{mv}{\sqrt{1 - \frac{v^2}{c^2}}} = \frac{E_\text{кин}}{c^2\cbr{\frac 1{\sqrt{1 - \frac{v^2}{c^2}}} - 1}} \cdot \frac{v}{\sqrt{1 - \frac{v^2}{c^2}}} = \frac{{E_\text{кин}} v}{c^2\cbr{1 - {\sqrt{1 - \frac{v^2}{c^2}}}}} \\
    E_0, p:\quad&E_0 = mc^2, \quad p = \frac{mv}{\sqrt{1 - \frac{v^2}{c^2}}} \implies \frac{E_0}{p} = \frac{c^2}v{\sqrt{1 - \frac{v^2}{c^2}}} = c\sqrt{\frac{c^2}{v^2} - 1} \\
    &\sqr{\frac{E_0}{pc}} = \frac{c^2}{v^2} - 1 \implies \frac{v^2}{c^2} = \frac 1{1 + \frac{E_0^2}{p^2c^2}} \implies v = \frac c{\sqrt{1 + \frac{E_0^2}{p^2c^2}}} \\
    &{E_\text{кин}} = E - E_0 = \sqrt{E_0^2 + p^2c^2} - E_0 \\
    E_0, v:\quad&E_0 = mc^2 \implies m = \frac{E_0}{c^2} \qquad p = \frac{mv}{\sqrt{1 - \frac{v^2}{c^2}}} = \frac{E_0}{c^2} \cdot \frac{v}{\sqrt{1 - \frac{v^2}{c^2}}} \\
    &E_\text{кин}= mc^2\cbr{\frac 1{\sqrt{1 - \frac{v^2}{c^2}}} - 1} = \frac{E_0}{c^2}\cbr{\frac 1{\sqrt{1 - \frac{v^2}{c^2}}} - 1} \\
    p, v:\quad&p = \frac{mv}{\sqrt{1 - \frac{v^2}{c^2}}} \implies m = \frac p v {\sqrt{1 - \frac{v^2}{c^2}}} \implies E_0 = mc^2 =\frac {pc^2} v {\sqrt{1 - \frac{v^2}{c^2}}} \\
    &E_\text{кин} = mc^2\cbr{\frac 1{\sqrt{1 - \frac{v^2}{c^2}}} - 1} = \frac p v {\sqrt{1 - \frac{v^2}{c^2}}}\cbr{\frac 1{\sqrt{1 - \frac{v^2}{c^2}}} - 1} = \frac p v \cbr{1 - {\sqrt{1 - \frac{v^2}{c^2}}}}
    \end{align*}
}
\solutionspace{200pt}

\tasknumber{2}%
\task{%
    Протон движется со скоростью $0{,}8\,c$, где $c$~--- скорость света в вакууме.
    Каково при этом отношение кинетической энергии частицы $E_\text{кин.}$ к его энергии покоя $E_0$?
}
\answer{%
    \begin{align*}
    E &= \frac{E_0}{\sqrt{1 - \frac{v^2}{c^2}}}
            \implies \frac E{E_0}
                = \frac 1{\sqrt{1 - \frac{v^2}{c^2}}}
                = \frac 1{\sqrt{1 - \sqr{0{,}8}}}
                \approx 1{,}667,
         \\
        E_{\text{кин}} &= E - E_0
            \implies \frac{E_{\text{кин}}}{E_0}
                = \frac E{E_0} - 1
                = \frac 1{\sqrt{1 - \frac{v^2}{c^2}}} - 1
                = \frac 1{\sqrt{1 - \sqr{0{,}8}}} - 1
                \approx 0{,}667.
    \end{align*}
}
\solutionspace{150pt}

\tasknumber{3}%
\task{%
    Протон движется со скоростью $0{,}75\,c$, где $c$~--- скорость света в вакууме.
    Определите его импульс (в ответе приведите формулу и укажите численное значение).
}
\answer{%
    \begin{align*}
    E &= \frac{mc^2}{\sqrt{1 - \frac{v^2}{c^2}}}
            \approx \frac{1{,}673 \cdot 10^{-27}\,\text{кг} \cdot \sqr{3 \cdot 10^{8}\,\frac{\text{м}}{\text{с}}}}{\sqrt{1 - 0{,}75^2}}
            \approx 227{,}589 \cdot 10^{-12}\,\text{Дж},
         \\
        E_{\text{кин}} &= \frac{mc^2}{\sqrt{1 - \frac{v^2}{c^2}}} - mc^2
            = mc^2 \cbr{\frac 1{\sqrt{1 - \frac{v^2}{c^2}}} - 1} \approx \\
            &\approx \cbr{1{,}673 \cdot 10^{-27}\,\text{кг} \cdot \sqr{3 \cdot 10^{8}\,\frac{\text{м}}{\text{с}}}}
            \cdot \cbr{\frac 1{\sqrt{1 - 0{,}75^2}} - 1}
            \approx 77{,}053 \cdot 10^{-12}\,\text{Дж},
         \\
        p &= \frac{mv}{\sqrt{1 - \frac{v^2}{c^2}}}
            \approx \frac{1{,}673 \cdot 10^{-27}\,\text{кг} \cdot 0{,}75 \cdot 3 \cdot 10^{8}\,\frac{\text{м}}{\text{с}}}{\sqrt{1 - 0{,}75^2}}
            \approx 568{,}972 \cdot 10^{-21}\,\frac{\text{кг}\cdot\text{м}}{\text{с}}.
    \end{align*}
}
\solutionspace{150pt}

\tasknumber{4}%
\task{%
    При какой скорости движения (в м/с) релятивистское сокращение длины движущегося тела
    составит 10\%?
}
\answer{%
    \begin{align*}
    l_0 &= \frac l{\sqrt{1 - \frac{v^2}{c^2}}}
        \implies 1 - \frac{v^2}{c^2} = \sqr{\frac l{l_0}}
        \implies \frac v c = \sqrt{1 - \sqr{\frac l{l_0}}} \implies
         \\
        \implies v &= c\sqrt{1 - \sqr{\frac l{l_0}}}
        = 3 \cdot 10^{8}\,\frac{\text{м}}{\text{с}} \cdot \sqrt{1 - \sqr{\frac {l_0 - 0{,}10l_0}{l_0}}}
        = 3 \cdot 10^{8}\,\frac{\text{м}}{\text{с}} \cdot \sqrt{1 - \sqr{1 - 0{,}10}} \approx  \\
        &\approx 0{,}436c
        \approx 130{,}8 \cdot 10^{6}\,\frac{\text{м}}{\text{с}}
        \approx 471 \cdot 10^{6}\,\frac{\text{км}}{\text{ч}}.
    \end{align*}
}
\solutionspace{150pt}

\tasknumber{5}%
\task{%
    При переходе электрона в атоме с одной стационарной орбиты на другую
    излучается фотон с энергией $1{,}01 \cdot 10^{-19}\,\text{Дж}$.
    Какова длина волны этой линии спектра?
    Постоянная Планка $h = 6{,}626 \cdot 10^{-34}\,\text{Дж}\cdot\text{с}$, скорость света $c = 3 \cdot 10^{8}\,\frac{\text{м}}{\text{с}}$.
}
\answer{%
    $
        E = h\nu = h \frac c\lambda
        \implies \lambda = \frac{hc}E
            = \frac{6{,}626 \cdot 10^{-34}\,\text{Дж}\cdot\text{с} \cdot {3 \cdot 10^{8}\,\frac{\text{м}}{\text{с}}}}{1{,}01 \cdot 10^{-19}\,\text{Дж}}
            = 1968{,}1\,\text{нм}.
    $
}
\solutionspace{150pt}

\tasknumber{6}%
\task{%
    Излучение какой длины волны поглотил атом водорода, если полная энергия в атоме увеличилась на $2 \cdot 10^{-19}\,\text{Дж}$?
    Постоянная Планка $h = 6{,}626 \cdot 10^{-34}\,\text{Дж}\cdot\text{с}$, скорость света $c = 3 \cdot 10^{8}\,\frac{\text{м}}{\text{с}}$.
}
\answer{%
    $
        E = h\nu = h \frac c\lambda
        \implies \lambda = \frac{hc}E
            = \frac{6{,}626 \cdot 10^{-34}\,\text{Дж}\cdot\text{с} \cdot {3 \cdot 10^{8}\,\frac{\text{м}}{\text{с}}}}{2 \cdot 10^{-19}\,\text{Дж}}
            = 994\,\text{нм}.
    $
}
\solutionspace{150pt}

\tasknumber{7}%
\task{%
    Сделайте схематичный рисунок энергетических уровней атома водорода
    и отметьте на нём первый (основной) уровень и последующие.
    Сколько различных длин волн может испустить атом водорода,
    находящийся в 3-м возбуждённом состоянии?
    Отметьте все соответствующие переходы на рисунке и укажите,
    при каком переходе (среди отмеченных) длина волны излучённого фотона минимальна.
}
\answer{%
    $N = 3{,}0, \text{самая длинная линия}$
}
\solutionspace{150pt}

\tasknumber{8}%
\task{%
    Сколько фотонов испускает за $10\,\text{мин}$ лазер,
    если мощность его излучения $15\,\text{мВт}$?
    Длина волны излучения $600\,\text{нм}$.
    $h = 6{,}626 \cdot 10^{-34}\,\text{Дж}\cdot\text{с}$.
}
\answer{%
    $
        N
            = \frac{E_{\text{общая}}}{E_{\text{одного фотона}}}
            = \frac{Pt}{h\nu} = \frac{Pt}{h \frac c\lambda}
            = \frac{Pt\lambda}{hc}
            = \frac{15\,\text{мВт} \cdot 10\,\text{мин} \cdot 600\,\text{нм}}{6{,}626 \cdot 10^{-34}\,\text{Дж}\cdot\text{с} \cdot 3 \cdot 10^{8}\,\frac{\text{м}}{\text{с}}}
            \approx 0{,}27 \cdot 10^{20}\units{фотонов}
    $
}
\solutionspace{120pt}

\tasknumber{9}%
\task{%
    Какая доля (от начального количества) радиоактивных ядер распадётся через время,
    равное трём периодам полураспада? Ответ выразить в процентах.
}
\answer{%
    \begin{align*}
    N &= N_0 \cdot 2^{- \frac t{T_{1/2}}} \implies
        \frac N{N_0} = 2^{- \frac t{T_{1/2}}}
        = 2^{-3} \approx 0{,}12 \approx 12\% \\
    N_\text{расп.} &= N_0 - N = N_0 - N_0 \cdot 2^{-\frac t{T_{1/2}}}
        = N_0\cbr{1 - 2^{-\frac t{T_{1/2}}}} \implies
        \frac{N_\text{расп.}}{N_0} = 1 - 2^{-\frac t{T_{1/2}}}
        = 1 - 2^{-3} \approx 0{,}88 \approx 88\%
    \end{align*}
}
\solutionspace{150pt}

\tasknumber{10}%
\task{%
    Сколько процентов ядер радиоактивного железа $\ce{^{59}Fe}$
    останется через $91{,}2\,\text{суток}$, если период его полураспада составляет $45{,}6\,\text{суток}$?
}
\answer{%
    \begin{align*}
    N &= N_0 \cdot 2^{-\frac t{T_{1/2}}}
        = 2^{-\frac{91{,}2\,\text{суток}}{45{,}6\,\text{суток}}}
        \approx 0{,}2500 = 25{,}00\%
    \end{align*}
}
\solutionspace{150pt}

\tasknumber{11}%
\task{%
    За $2\,\text{суток}$ от начального количества ядер радиоизотопа осталась одна восьмая.
    Каков период полураспада этого изотопа (ответ приведите в сутках)?
    Какая ещё доля (также от начального количества) распадётся, если подождать ещё столько же?
}
\answer{%
    \begin{align*}
            N &= N_0 \cdot 2^{-\frac t{T_{1/2}}}
            \implies \frac N{N_0} = 2^{-\frac t{T_{1/2}}}
            \implies \frac 1{8} = 2^{-\frac {2\,\text{суток}}{T_{1/2}}}
            \implies 3 = \frac {2\,\text{суток}}{T_{1/2}}
            \implies T_{1/2} = \frac {2\,\text{суток}}3 \approx 0{,}67\,\text{суток}.
         \\
            \delta &= \frac{N(t)}{N_0} - \frac{N(2t)}{N_0}
            = 2^{-\frac t{T_{1/2}}} - 2^{-\frac {2t}{T_{1/2}}}
            = 2^{-\frac t{T_{1/2}}}\cbr{1 - 2^{-\frac t{T_{1/2}}}}
            = \frac 1{8} \cdot \cbr{1-\frac 1{8}} \approx 0{,}109
    \end{align*}
}
\solutionspace{150pt}

\tasknumber{12}%
\task{%
    Энергия связи ядра лития \ce{^{7}_{3}Li} равна $39{,}2\,\text{МэВ}$.
    Найти дефект массы этого ядра.
    Ответ выразите в а.е.м.
    и кг.
    Скорость света $c = 2{,}998 \cdot 10^{8}\,\frac{\text{м}}{\text{с}}$, элементарный заряд $e = 1{,}6 \cdot 10^{-19}\,\text{Кл}$.
}
\answer{%
    \begin{align*}
    E_\text{св.} &= \Delta m c^2 \implies \\
    \implies
            \Delta m &= \frac {E_\text{св.}}{c^2} = \frac{39{,}2\,\text{МэВ}}{\sqr{2{,}998 \cdot 10^{8}\,\frac{\text{м}}{\text{с}}}}
            = \frac{39{,}2 \cdot 10^6 \cdot 1{,}6 \cdot 10^{-19}\,\text{Дж}}{\sqr{2{,}998 \cdot 10^{8}\,\frac{\text{м}}{\text{с}}}}
            \approx 69{,}8 \cdot 10^{-30}\,\text{кг} \approx 0{,}0420\,\text{а.е.м.}
    \end{align*}
}

\variantsplitter

\addpersonalvariant{Дмитрий Иванов}

\tasknumber{1}%
\task{%
    Для частицы, движущейся с релятивистской скоростью,
    выразите $E_0$ и $v$ через $c$, $E_\text{кин}$ и $p$,
    где $E_\text{кин}$~--- кинетическая энергия частицы,
    а $E_0$, $p$ и $v$~--- её энергия покоя импульс и скорость.
}
\answer{%
    \begin{align*}
    E_\text{кин}, E_0:\quad&E = E_\text{кин} + E_0 = \frac{E_0}{\sqrt{1 - \frac{v^2}{c^2}}} \implies \sqrt{1 - \frac{v^2}{c^2}} = \frac{E_0}{{E_0} + {E_\text{кин}}} \implies v = c\sqrt{1 - \sqr{\frac{E_0}{{E_0} + {E_\text{кин}}}}} \\
    &p = \frac{mv}{\sqrt{1 - \frac{v^2}{c^2}}} = \frac{E_0}{c^2} \cdot \sqrt{1 - \sqr{\frac{E_0}{{E_0} + {E_\text{кин}}}}} \cdot \frac{{E_\text{кин}} + {E_0}}{E_0} = \frac{E_0}{c^2} \cdot \sqrt{\sqr{\frac{{E_\text{кин}} + {E_0}}{E_0}} - 1}.
    \\
    E_\text{кин}, p:\quad&E_\text{кин} = E - E_0 = mc^2\cbr{\frac 1{\sqrt{1 - \frac{v^2}{c^2}}} - 1}, p = \frac{mv}{\sqrt{1 - \frac{v^2}{c^2}}} \implies \frac{E_\text{кин}}{p} = \frac{\frac 1{\sqrt{1 - \frac{v^2}{c^2}}} - 1}{\sqrt{1 - \frac{v^2}{c^2}}} \implies v = \ldots \\
    &E_0 = E - E_\text{кин} = \frac{E_0}{\sqrt{1 - \frac{v^2}{c^2}}} - E_\text{кин} \implies E_0 = \frac{E_\text{кин}}{\frac 1{\sqrt{1 - \frac{v^2}{c^2}}} - 1} = \ldots \\
    E_\text{кин}, v:\quad&E_\text{кин} = E - E_0 = mc^2\cbr{\frac 1{\sqrt{1 - \frac{v^2}{c^2}}} - 1} \implies m = \frac{E_\text{кин}}{c^2\cbr{\frac 1{\sqrt{1 - \frac{v^2}{c^2}}} - 1}} \\
    &E_0 = mc^2 = \frac{E_\text{кин}}{\frac 1{\sqrt{1 - \frac{v^2}{c^2}}} - 1} \\
    &p = \frac{mv}{\sqrt{1 - \frac{v^2}{c^2}}} = \frac{E_\text{кин}}{c^2\cbr{\frac 1{\sqrt{1 - \frac{v^2}{c^2}}} - 1}} \cdot \frac{v}{\sqrt{1 - \frac{v^2}{c^2}}} = \frac{{E_\text{кин}} v}{c^2\cbr{1 - {\sqrt{1 - \frac{v^2}{c^2}}}}} \\
    E_0, p:\quad&E_0 = mc^2, \quad p = \frac{mv}{\sqrt{1 - \frac{v^2}{c^2}}} \implies \frac{E_0}{p} = \frac{c^2}v{\sqrt{1 - \frac{v^2}{c^2}}} = c\sqrt{\frac{c^2}{v^2} - 1} \\
    &\sqr{\frac{E_0}{pc}} = \frac{c^2}{v^2} - 1 \implies \frac{v^2}{c^2} = \frac 1{1 + \frac{E_0^2}{p^2c^2}} \implies v = \frac c{\sqrt{1 + \frac{E_0^2}{p^2c^2}}} \\
    &{E_\text{кин}} = E - E_0 = \sqrt{E_0^2 + p^2c^2} - E_0 \\
    E_0, v:\quad&E_0 = mc^2 \implies m = \frac{E_0}{c^2} \qquad p = \frac{mv}{\sqrt{1 - \frac{v^2}{c^2}}} = \frac{E_0}{c^2} \cdot \frac{v}{\sqrt{1 - \frac{v^2}{c^2}}} \\
    &E_\text{кин}= mc^2\cbr{\frac 1{\sqrt{1 - \frac{v^2}{c^2}}} - 1} = \frac{E_0}{c^2}\cbr{\frac 1{\sqrt{1 - \frac{v^2}{c^2}}} - 1} \\
    p, v:\quad&p = \frac{mv}{\sqrt{1 - \frac{v^2}{c^2}}} \implies m = \frac p v {\sqrt{1 - \frac{v^2}{c^2}}} \implies E_0 = mc^2 =\frac {pc^2} v {\sqrt{1 - \frac{v^2}{c^2}}} \\
    &E_\text{кин} = mc^2\cbr{\frac 1{\sqrt{1 - \frac{v^2}{c^2}}} - 1} = \frac p v {\sqrt{1 - \frac{v^2}{c^2}}}\cbr{\frac 1{\sqrt{1 - \frac{v^2}{c^2}}} - 1} = \frac p v \cbr{1 - {\sqrt{1 - \frac{v^2}{c^2}}}}
    \end{align*}
}
\solutionspace{200pt}

\tasknumber{2}%
\task{%
    Электрон движется со скоростью $0{,}8\,c$, где $c$~--- скорость света в вакууме.
    Каково при этом отношение кинетической энергии частицы $E_\text{кин.}$ к его энергии покоя $E_0$?
}
\answer{%
    \begin{align*}
    E &= \frac{E_0}{\sqrt{1 - \frac{v^2}{c^2}}}
            \implies \frac E{E_0}
                = \frac 1{\sqrt{1 - \frac{v^2}{c^2}}}
                = \frac 1{\sqrt{1 - \sqr{0{,}8}}}
                \approx 1{,}667,
         \\
        E_{\text{кин}} &= E - E_0
            \implies \frac{E_{\text{кин}}}{E_0}
                = \frac E{E_0} - 1
                = \frac 1{\sqrt{1 - \frac{v^2}{c^2}}} - 1
                = \frac 1{\sqrt{1 - \sqr{0{,}8}}} - 1
                \approx 0{,}667.
    \end{align*}
}
\solutionspace{150pt}

\tasknumber{3}%
\task{%
    Протон движется со скоростью $0{,}65\,c$, где $c$~--- скорость света в вакууме.
    Определите его полную энергию (в ответе приведите формулу и укажите численное значение).
}
\answer{%
    \begin{align*}
    E &= \frac{mc^2}{\sqrt{1 - \frac{v^2}{c^2}}}
            \approx \frac{1{,}673 \cdot 10^{-27}\,\text{кг} \cdot \sqr{3 \cdot 10^{8}\,\frac{\text{м}}{\text{с}}}}{\sqrt{1 - 0{,}65^2}}
            \approx 198{,}091 \cdot 10^{-12}\,\text{Дж},
         \\
        E_{\text{кин}} &= \frac{mc^2}{\sqrt{1 - \frac{v^2}{c^2}}} - mc^2
            = mc^2 \cbr{\frac 1{\sqrt{1 - \frac{v^2}{c^2}}} - 1} \approx \\
            &\approx \cbr{1{,}673 \cdot 10^{-27}\,\text{кг} \cdot \sqr{3 \cdot 10^{8}\,\frac{\text{м}}{\text{с}}}}
            \cdot \cbr{\frac 1{\sqrt{1 - 0{,}65^2}} - 1}
            \approx 47{,}555 \cdot 10^{-12}\,\text{Дж},
         \\
        p &= \frac{mv}{\sqrt{1 - \frac{v^2}{c^2}}}
            \approx \frac{1{,}673 \cdot 10^{-27}\,\text{кг} \cdot 0{,}65 \cdot 3 \cdot 10^{8}\,\frac{\text{м}}{\text{с}}}{\sqrt{1 - 0{,}65^2}}
            \approx 429{,}196 \cdot 10^{-21}\,\frac{\text{кг}\cdot\text{м}}{\text{с}}.
    \end{align*}
}
\solutionspace{150pt}

\tasknumber{4}%
\task{%
    При какой скорости движения (в м/с) релятивистское сокращение длины движущегося тела
    составит 10\%?
}
\answer{%
    \begin{align*}
    l_0 &= \frac l{\sqrt{1 - \frac{v^2}{c^2}}}
        \implies 1 - \frac{v^2}{c^2} = \sqr{\frac l{l_0}}
        \implies \frac v c = \sqrt{1 - \sqr{\frac l{l_0}}} \implies
         \\
        \implies v &= c\sqrt{1 - \sqr{\frac l{l_0}}}
        = 3 \cdot 10^{8}\,\frac{\text{м}}{\text{с}} \cdot \sqrt{1 - \sqr{\frac {l_0 - 0{,}10l_0}{l_0}}}
        = 3 \cdot 10^{8}\,\frac{\text{м}}{\text{с}} \cdot \sqrt{1 - \sqr{1 - 0{,}10}} \approx  \\
        &\approx 0{,}436c
        \approx 130{,}8 \cdot 10^{6}\,\frac{\text{м}}{\text{с}}
        \approx 471 \cdot 10^{6}\,\frac{\text{км}}{\text{ч}}.
    \end{align*}
}
\solutionspace{150pt}

\tasknumber{5}%
\task{%
    При переходе электрона в атоме с одной стационарной орбиты на другую
    излучается фотон с энергией $1{,}01 \cdot 10^{-19}\,\text{Дж}$.
    Какова длина волны этой линии спектра?
    Постоянная Планка $h = 6{,}626 \cdot 10^{-34}\,\text{Дж}\cdot\text{с}$, скорость света $c = 3 \cdot 10^{8}\,\frac{\text{м}}{\text{с}}$.
}
\answer{%
    $
        E = h\nu = h \frac c\lambda
        \implies \lambda = \frac{hc}E
            = \frac{6{,}626 \cdot 10^{-34}\,\text{Дж}\cdot\text{с} \cdot {3 \cdot 10^{8}\,\frac{\text{м}}{\text{с}}}}{1{,}01 \cdot 10^{-19}\,\text{Дж}}
            = 1968{,}1\,\text{нм}.
    $
}
\solutionspace{150pt}

\tasknumber{6}%
\task{%
    Излучение какой длины волны поглотил атом водорода, если полная энергия в атоме увеличилась на $2 \cdot 10^{-19}\,\text{Дж}$?
    Постоянная Планка $h = 6{,}626 \cdot 10^{-34}\,\text{Дж}\cdot\text{с}$, скорость света $c = 3 \cdot 10^{8}\,\frac{\text{м}}{\text{с}}$.
}
\answer{%
    $
        E = h\nu = h \frac c\lambda
        \implies \lambda = \frac{hc}E
            = \frac{6{,}626 \cdot 10^{-34}\,\text{Дж}\cdot\text{с} \cdot {3 \cdot 10^{8}\,\frac{\text{м}}{\text{с}}}}{2 \cdot 10^{-19}\,\text{Дж}}
            = 994\,\text{нм}.
    $
}
\solutionspace{150pt}

\tasknumber{7}%
\task{%
    Сделайте схематичный рисунок энергетических уровней атома водорода
    и отметьте на нём первый (основной) уровень и последующие.
    Сколько различных длин волн может испустить атом водорода,
    находящийся в 4-м возбуждённом состоянии?
    Отметьте все соответствующие переходы на рисунке и укажите,
    при каком переходе (среди отмеченных) частота излучённого фотона минимальна.
}
\answer{%
    $N = 6{,}0, \text{самая короткая линия}$
}
\solutionspace{150pt}

\tasknumber{8}%
\task{%
    Сколько фотонов испускает за $10\,\text{мин}$ лазер,
    если мощность его излучения $40\,\text{мВт}$?
    Длина волны излучения $750\,\text{нм}$.
    $h = 6{,}626 \cdot 10^{-34}\,\text{Дж}\cdot\text{с}$.
}
\answer{%
    $
        N
            = \frac{E_{\text{общая}}}{E_{\text{одного фотона}}}
            = \frac{Pt}{h\nu} = \frac{Pt}{h \frac c\lambda}
            = \frac{Pt\lambda}{hc}
            = \frac{40\,\text{мВт} \cdot 10\,\text{мин} \cdot 750\,\text{нм}}{6{,}626 \cdot 10^{-34}\,\text{Дж}\cdot\text{с} \cdot 3 \cdot 10^{8}\,\frac{\text{м}}{\text{с}}}
            \approx 0{,}91 \cdot 10^{20}\units{фотонов}
    $
}
\solutionspace{120pt}

\tasknumber{9}%
\task{%
    Какая доля (от начального количества) радиоактивных ядер распадётся через время,
    равное двум периодам полураспада? Ответ выразить в процентах.
}
\answer{%
    \begin{align*}
    N &= N_0 \cdot 2^{- \frac t{T_{1/2}}} \implies
        \frac N{N_0} = 2^{- \frac t{T_{1/2}}}
        = 2^{-2} \approx 0{,}25 \approx 25\% \\
    N_\text{расп.} &= N_0 - N = N_0 - N_0 \cdot 2^{-\frac t{T_{1/2}}}
        = N_0\cbr{1 - 2^{-\frac t{T_{1/2}}}} \implies
        \frac{N_\text{расп.}}{N_0} = 1 - 2^{-\frac t{T_{1/2}}}
        = 1 - 2^{-2} \approx 0{,}75 \approx 75\%
    \end{align*}
}
\solutionspace{150pt}

\tasknumber{10}%
\task{%
    Сколько процентов ядер радиоактивного железа $\ce{^{59}Fe}$
    останется через $91{,}2\,\text{суток}$, если период его полураспада составляет $45{,}6\,\text{суток}$?
}
\answer{%
    \begin{align*}
    N &= N_0 \cdot 2^{-\frac t{T_{1/2}}}
        = 2^{-\frac{91{,}2\,\text{суток}}{45{,}6\,\text{суток}}}
        \approx 0{,}2500 = 25{,}00\%
    \end{align*}
}
\solutionspace{150pt}

\tasknumber{11}%
\task{%
    За $4\,\text{суток}$ от начального количества ядер радиоизотопа осталась одна восьмая.
    Каков период полураспада этого изотопа (ответ приведите в сутках)?
    Какая ещё доля (также от начального количества) распадётся, если подождать ещё столько же?
}
\answer{%
    \begin{align*}
            N &= N_0 \cdot 2^{-\frac t{T_{1/2}}}
            \implies \frac N{N_0} = 2^{-\frac t{T_{1/2}}}
            \implies \frac 1{8} = 2^{-\frac {4\,\text{суток}}{T_{1/2}}}
            \implies 3 = \frac {4\,\text{суток}}{T_{1/2}}
            \implies T_{1/2} = \frac {4\,\text{суток}}3 \approx 1{,}33\,\text{суток}.
         \\
            \delta &= \frac{N(t)}{N_0} - \frac{N(2t)}{N_0}
            = 2^{-\frac t{T_{1/2}}} - 2^{-\frac {2t}{T_{1/2}}}
            = 2^{-\frac t{T_{1/2}}}\cbr{1 - 2^{-\frac t{T_{1/2}}}}
            = \frac 1{8} \cdot \cbr{1-\frac 1{8}} \approx 0{,}109
    \end{align*}
}
\solutionspace{150pt}

\tasknumber{12}%
\task{%
    Энергия связи ядра кислорода \ce{^{17}_{8}O} равна $131{,}8\,\text{МэВ}$.
    Найти дефект массы этого ядра.
    Ответ выразите в а.е.м.
    и кг.
    Скорость света $c = 2{,}998 \cdot 10^{8}\,\frac{\text{м}}{\text{с}}$, элементарный заряд $e = 1{,}6 \cdot 10^{-19}\,\text{Кл}$.
}
\answer{%
    \begin{align*}
    E_\text{св.} &= \Delta m c^2 \implies \\
    \implies
            \Delta m &= \frac {E_\text{св.}}{c^2} = \frac{131{,}8\,\text{МэВ}}{\sqr{2{,}998 \cdot 10^{8}\,\frac{\text{м}}{\text{с}}}}
            = \frac{131{,}8 \cdot 10^6 \cdot 1{,}6 \cdot 10^{-19}\,\text{Дж}}{\sqr{2{,}998 \cdot 10^{8}\,\frac{\text{м}}{\text{с}}}}
            \approx 0{,}235 \cdot 10^{-27}\,\text{кг} \approx 0{,}1413\,\text{а.е.м.}
    \end{align*}
}

\variantsplitter

\addpersonalvariant{Олег Климов}

\tasknumber{1}%
\task{%
    Для частицы, движущейся с релятивистской скоростью,
    выразите $v$ и $E_\text{кин}$ через $c$, $p$ и $E_0$,
    где $E_\text{кин}$~--- кинетическая энергия частицы,
    а $E_0$, $p$ и $v$~--- её энергия покоя импульс и скорость.
}
\answer{%
    \begin{align*}
    E_\text{кин}, E_0:\quad&E = E_\text{кин} + E_0 = \frac{E_0}{\sqrt{1 - \frac{v^2}{c^2}}} \implies \sqrt{1 - \frac{v^2}{c^2}} = \frac{E_0}{{E_0} + {E_\text{кин}}} \implies v = c\sqrt{1 - \sqr{\frac{E_0}{{E_0} + {E_\text{кин}}}}} \\
    &p = \frac{mv}{\sqrt{1 - \frac{v^2}{c^2}}} = \frac{E_0}{c^2} \cdot \sqrt{1 - \sqr{\frac{E_0}{{E_0} + {E_\text{кин}}}}} \cdot \frac{{E_\text{кин}} + {E_0}}{E_0} = \frac{E_0}{c^2} \cdot \sqrt{\sqr{\frac{{E_\text{кин}} + {E_0}}{E_0}} - 1}.
    \\
    E_\text{кин}, p:\quad&E_\text{кин} = E - E_0 = mc^2\cbr{\frac 1{\sqrt{1 - \frac{v^2}{c^2}}} - 1}, p = \frac{mv}{\sqrt{1 - \frac{v^2}{c^2}}} \implies \frac{E_\text{кин}}{p} = \frac{\frac 1{\sqrt{1 - \frac{v^2}{c^2}}} - 1}{\sqrt{1 - \frac{v^2}{c^2}}} \implies v = \ldots \\
    &E_0 = E - E_\text{кин} = \frac{E_0}{\sqrt{1 - \frac{v^2}{c^2}}} - E_\text{кин} \implies E_0 = \frac{E_\text{кин}}{\frac 1{\sqrt{1 - \frac{v^2}{c^2}}} - 1} = \ldots \\
    E_\text{кин}, v:\quad&E_\text{кин} = E - E_0 = mc^2\cbr{\frac 1{\sqrt{1 - \frac{v^2}{c^2}}} - 1} \implies m = \frac{E_\text{кин}}{c^2\cbr{\frac 1{\sqrt{1 - \frac{v^2}{c^2}}} - 1}} \\
    &E_0 = mc^2 = \frac{E_\text{кин}}{\frac 1{\sqrt{1 - \frac{v^2}{c^2}}} - 1} \\
    &p = \frac{mv}{\sqrt{1 - \frac{v^2}{c^2}}} = \frac{E_\text{кин}}{c^2\cbr{\frac 1{\sqrt{1 - \frac{v^2}{c^2}}} - 1}} \cdot \frac{v}{\sqrt{1 - \frac{v^2}{c^2}}} = \frac{{E_\text{кин}} v}{c^2\cbr{1 - {\sqrt{1 - \frac{v^2}{c^2}}}}} \\
    E_0, p:\quad&E_0 = mc^2, \quad p = \frac{mv}{\sqrt{1 - \frac{v^2}{c^2}}} \implies \frac{E_0}{p} = \frac{c^2}v{\sqrt{1 - \frac{v^2}{c^2}}} = c\sqrt{\frac{c^2}{v^2} - 1} \\
    &\sqr{\frac{E_0}{pc}} = \frac{c^2}{v^2} - 1 \implies \frac{v^2}{c^2} = \frac 1{1 + \frac{E_0^2}{p^2c^2}} \implies v = \frac c{\sqrt{1 + \frac{E_0^2}{p^2c^2}}} \\
    &{E_\text{кин}} = E - E_0 = \sqrt{E_0^2 + p^2c^2} - E_0 \\
    E_0, v:\quad&E_0 = mc^2 \implies m = \frac{E_0}{c^2} \qquad p = \frac{mv}{\sqrt{1 - \frac{v^2}{c^2}}} = \frac{E_0}{c^2} \cdot \frac{v}{\sqrt{1 - \frac{v^2}{c^2}}} \\
    &E_\text{кин}= mc^2\cbr{\frac 1{\sqrt{1 - \frac{v^2}{c^2}}} - 1} = \frac{E_0}{c^2}\cbr{\frac 1{\sqrt{1 - \frac{v^2}{c^2}}} - 1} \\
    p, v:\quad&p = \frac{mv}{\sqrt{1 - \frac{v^2}{c^2}}} \implies m = \frac p v {\sqrt{1 - \frac{v^2}{c^2}}} \implies E_0 = mc^2 =\frac {pc^2} v {\sqrt{1 - \frac{v^2}{c^2}}} \\
    &E_\text{кин} = mc^2\cbr{\frac 1{\sqrt{1 - \frac{v^2}{c^2}}} - 1} = \frac p v {\sqrt{1 - \frac{v^2}{c^2}}}\cbr{\frac 1{\sqrt{1 - \frac{v^2}{c^2}}} - 1} = \frac p v \cbr{1 - {\sqrt{1 - \frac{v^2}{c^2}}}}
    \end{align*}
}
\solutionspace{200pt}

\tasknumber{2}%
\task{%
    Электрон движется со скоростью $0{,}8\,c$, где $c$~--- скорость света в вакууме.
    Каково при этом отношение кинетической энергии частицы $E_\text{кин.}$ к его энергии покоя $E_0$?
}
\answer{%
    \begin{align*}
    E &= \frac{E_0}{\sqrt{1 - \frac{v^2}{c^2}}}
            \implies \frac E{E_0}
                = \frac 1{\sqrt{1 - \frac{v^2}{c^2}}}
                = \frac 1{\sqrt{1 - \sqr{0{,}8}}}
                \approx 1{,}667,
         \\
        E_{\text{кин}} &= E - E_0
            \implies \frac{E_{\text{кин}}}{E_0}
                = \frac E{E_0} - 1
                = \frac 1{\sqrt{1 - \frac{v^2}{c^2}}} - 1
                = \frac 1{\sqrt{1 - \sqr{0{,}8}}} - 1
                \approx 0{,}667.
    \end{align*}
}
\solutionspace{150pt}

\tasknumber{3}%
\task{%
    Протон движется со скоростью $0{,}75\,c$, где $c$~--- скорость света в вакууме.
    Определите его импульс (в ответе приведите формулу и укажите численное значение).
}
\answer{%
    \begin{align*}
    E &= \frac{mc^2}{\sqrt{1 - \frac{v^2}{c^2}}}
            \approx \frac{1{,}673 \cdot 10^{-27}\,\text{кг} \cdot \sqr{3 \cdot 10^{8}\,\frac{\text{м}}{\text{с}}}}{\sqrt{1 - 0{,}75^2}}
            \approx 227{,}589 \cdot 10^{-12}\,\text{Дж},
         \\
        E_{\text{кин}} &= \frac{mc^2}{\sqrt{1 - \frac{v^2}{c^2}}} - mc^2
            = mc^2 \cbr{\frac 1{\sqrt{1 - \frac{v^2}{c^2}}} - 1} \approx \\
            &\approx \cbr{1{,}673 \cdot 10^{-27}\,\text{кг} \cdot \sqr{3 \cdot 10^{8}\,\frac{\text{м}}{\text{с}}}}
            \cdot \cbr{\frac 1{\sqrt{1 - 0{,}75^2}} - 1}
            \approx 77{,}053 \cdot 10^{-12}\,\text{Дж},
         \\
        p &= \frac{mv}{\sqrt{1 - \frac{v^2}{c^2}}}
            \approx \frac{1{,}673 \cdot 10^{-27}\,\text{кг} \cdot 0{,}75 \cdot 3 \cdot 10^{8}\,\frac{\text{м}}{\text{с}}}{\sqrt{1 - 0{,}75^2}}
            \approx 568{,}972 \cdot 10^{-21}\,\frac{\text{кг}\cdot\text{м}}{\text{с}}.
    \end{align*}
}
\solutionspace{150pt}

\tasknumber{4}%
\task{%
    При какой скорости движения (в м/с) релятивистское сокращение длины движущегося тела
    составит 50\%?
}
\answer{%
    \begin{align*}
    l_0 &= \frac l{\sqrt{1 - \frac{v^2}{c^2}}}
        \implies 1 - \frac{v^2}{c^2} = \sqr{\frac l{l_0}}
        \implies \frac v c = \sqrt{1 - \sqr{\frac l{l_0}}} \implies
         \\
        \implies v &= c\sqrt{1 - \sqr{\frac l{l_0}}}
        = 3 \cdot 10^{8}\,\frac{\text{м}}{\text{с}} \cdot \sqrt{1 - \sqr{\frac {l_0 - 0{,}50l_0}{l_0}}}
        = 3 \cdot 10^{8}\,\frac{\text{м}}{\text{с}} \cdot \sqrt{1 - \sqr{1 - 0{,}50}} \approx  \\
        &\approx 0{,}866c
        \approx 260 \cdot 10^{6}\,\frac{\text{м}}{\text{с}}
        \approx 935 \cdot 10^{6}\,\frac{\text{км}}{\text{ч}}.
    \end{align*}
}
\solutionspace{150pt}

\tasknumber{5}%
\task{%
    При переходе электрона в атоме с одной стационарной орбиты на другую
    излучается фотон с энергией $4{,}04 \cdot 10^{-19}\,\text{Дж}$.
    Какова длина волны этой линии спектра?
    Постоянная Планка $h = 6{,}626 \cdot 10^{-34}\,\text{Дж}\cdot\text{с}$, скорость света $c = 3 \cdot 10^{8}\,\frac{\text{м}}{\text{с}}$.
}
\answer{%
    $
        E = h\nu = h \frac c\lambda
        \implies \lambda = \frac{hc}E
            = \frac{6{,}626 \cdot 10^{-34}\,\text{Дж}\cdot\text{с} \cdot {3 \cdot 10^{8}\,\frac{\text{м}}{\text{с}}}}{4{,}04 \cdot 10^{-19}\,\text{Дж}}
            = 492{,}03\,\text{нм}.
    $
}
\solutionspace{150pt}

\tasknumber{6}%
\task{%
    Излучение какой длины волны поглотил атом водорода, если полная энергия в атоме увеличилась на $4 \cdot 10^{-19}\,\text{Дж}$?
    Постоянная Планка $h = 6{,}626 \cdot 10^{-34}\,\text{Дж}\cdot\text{с}$, скорость света $c = 3 \cdot 10^{8}\,\frac{\text{м}}{\text{с}}$.
}
\answer{%
    $
        E = h\nu = h \frac c\lambda
        \implies \lambda = \frac{hc}E
            = \frac{6{,}626 \cdot 10^{-34}\,\text{Дж}\cdot\text{с} \cdot {3 \cdot 10^{8}\,\frac{\text{м}}{\text{с}}}}{4 \cdot 10^{-19}\,\text{Дж}}
            = 497\,\text{нм}.
    $
}
\solutionspace{150pt}

\tasknumber{7}%
\task{%
    Сделайте схематичный рисунок энергетических уровней атома водорода
    и отметьте на нём первый (основной) уровень и последующие.
    Сколько различных длин волн может испустить атом водорода,
    находящийся в 5-м возбуждённом состоянии?
    Отметьте все соответствующие переходы на рисунке и укажите,
    при каком переходе (среди отмеченных) энергия излучённого фотона максимальна.
}
\answer{%
    $N = 10{,}0, \text{самая длинная линия}$
}
\solutionspace{150pt}

\tasknumber{8}%
\task{%
    Сколько фотонов испускает за $10\,\text{мин}$ лазер,
    если мощность его излучения $200\,\text{мВт}$?
    Длина волны излучения $750\,\text{нм}$.
    $h = 6{,}626 \cdot 10^{-34}\,\text{Дж}\cdot\text{с}$.
}
\answer{%
    $
        N
            = \frac{E_{\text{общая}}}{E_{\text{одного фотона}}}
            = \frac{Pt}{h\nu} = \frac{Pt}{h \frac c\lambda}
            = \frac{Pt\lambda}{hc}
            = \frac{200\,\text{мВт} \cdot 10\,\text{мин} \cdot 750\,\text{нм}}{6{,}626 \cdot 10^{-34}\,\text{Дж}\cdot\text{с} \cdot 3 \cdot 10^{8}\,\frac{\text{м}}{\text{с}}}
            \approx 4{,}53 \cdot 10^{20}\units{фотонов}
    $
}
\solutionspace{120pt}

\tasknumber{9}%
\task{%
    Какая доля (от начального количества) радиоактивных ядер распадётся через время,
    равное трём периодам полураспада? Ответ выразить в процентах.
}
\answer{%
    \begin{align*}
    N &= N_0 \cdot 2^{- \frac t{T_{1/2}}} \implies
        \frac N{N_0} = 2^{- \frac t{T_{1/2}}}
        = 2^{-3} \approx 0{,}12 \approx 12\% \\
    N_\text{расп.} &= N_0 - N = N_0 - N_0 \cdot 2^{-\frac t{T_{1/2}}}
        = N_0\cbr{1 - 2^{-\frac t{T_{1/2}}}} \implies
        \frac{N_\text{расп.}}{N_0} = 1 - 2^{-\frac t{T_{1/2}}}
        = 1 - 2^{-3} \approx 0{,}88 \approx 88\%
    \end{align*}
}
\solutionspace{150pt}

\tasknumber{10}%
\task{%
    Сколько процентов ядер радиоактивного железа $\ce{^{59}Fe}$
    останется через $91{,}2\,\text{суток}$, если период его полураспада составляет $45{,}6\,\text{суток}$?
}
\answer{%
    \begin{align*}
    N &= N_0 \cdot 2^{-\frac t{T_{1/2}}}
        = 2^{-\frac{91{,}2\,\text{суток}}{45{,}6\,\text{суток}}}
        \approx 0{,}2500 = 25{,}00\%
    \end{align*}
}
\solutionspace{150pt}

\tasknumber{11}%
\task{%
    За $3\,\text{суток}$ от начального количества ядер радиоизотопа осталась одна восьмая.
    Каков период полураспада этого изотопа (ответ приведите в сутках)?
    Какая ещё доля (также от начального количества) распадётся, если подождать ещё столько же?
}
\answer{%
    \begin{align*}
            N &= N_0 \cdot 2^{-\frac t{T_{1/2}}}
            \implies \frac N{N_0} = 2^{-\frac t{T_{1/2}}}
            \implies \frac 1{8} = 2^{-\frac {3\,\text{суток}}{T_{1/2}}}
            \implies 3 = \frac {3\,\text{суток}}{T_{1/2}}
            \implies T_{1/2} = \frac {3\,\text{суток}}3 \approx 1\,\text{суток}.
         \\
            \delta &= \frac{N(t)}{N_0} - \frac{N(2t)}{N_0}
            = 2^{-\frac t{T_{1/2}}} - 2^{-\frac {2t}{T_{1/2}}}
            = 2^{-\frac t{T_{1/2}}}\cbr{1 - 2^{-\frac t{T_{1/2}}}}
            = \frac 1{8} \cdot \cbr{1-\frac 1{8}} \approx 0{,}109
    \end{align*}
}
\solutionspace{150pt}

\tasknumber{12}%
\task{%
    Энергия связи ядра лития \ce{^{7}_{3}Li} равна $39{,}2\,\text{МэВ}$.
    Найти дефект массы этого ядра.
    Ответ выразите в а.е.м.
    и кг.
    Скорость света $c = 2{,}998 \cdot 10^{8}\,\frac{\text{м}}{\text{с}}$, элементарный заряд $e = 1{,}6 \cdot 10^{-19}\,\text{Кл}$.
}
\answer{%
    \begin{align*}
    E_\text{св.} &= \Delta m c^2 \implies \\
    \implies
            \Delta m &= \frac {E_\text{св.}}{c^2} = \frac{39{,}2\,\text{МэВ}}{\sqr{2{,}998 \cdot 10^{8}\,\frac{\text{м}}{\text{с}}}}
            = \frac{39{,}2 \cdot 10^6 \cdot 1{,}6 \cdot 10^{-19}\,\text{Дж}}{\sqr{2{,}998 \cdot 10^{8}\,\frac{\text{м}}{\text{с}}}}
            \approx 69{,}8 \cdot 10^{-30}\,\text{кг} \approx 0{,}0420\,\text{а.е.м.}
    \end{align*}
}

\variantsplitter

\addpersonalvariant{Анна Ковалева}

\tasknumber{1}%
\task{%
    Для частицы, движущейся с релятивистской скоростью,
    выразите $E_0$ и $p$ через $c$, $v$ и $E_\text{кин}$,
    где $E_\text{кин}$~--- кинетическая энергия частицы,
    а $E_0$, $p$ и $v$~--- её энергия покоя импульс и скорость.
}
\answer{%
    \begin{align*}
    E_\text{кин}, E_0:\quad&E = E_\text{кин} + E_0 = \frac{E_0}{\sqrt{1 - \frac{v^2}{c^2}}} \implies \sqrt{1 - \frac{v^2}{c^2}} = \frac{E_0}{{E_0} + {E_\text{кин}}} \implies v = c\sqrt{1 - \sqr{\frac{E_0}{{E_0} + {E_\text{кин}}}}} \\
    &p = \frac{mv}{\sqrt{1 - \frac{v^2}{c^2}}} = \frac{E_0}{c^2} \cdot \sqrt{1 - \sqr{\frac{E_0}{{E_0} + {E_\text{кин}}}}} \cdot \frac{{E_\text{кин}} + {E_0}}{E_0} = \frac{E_0}{c^2} \cdot \sqrt{\sqr{\frac{{E_\text{кин}} + {E_0}}{E_0}} - 1}.
    \\
    E_\text{кин}, p:\quad&E_\text{кин} = E - E_0 = mc^2\cbr{\frac 1{\sqrt{1 - \frac{v^2}{c^2}}} - 1}, p = \frac{mv}{\sqrt{1 - \frac{v^2}{c^2}}} \implies \frac{E_\text{кин}}{p} = \frac{\frac 1{\sqrt{1 - \frac{v^2}{c^2}}} - 1}{\sqrt{1 - \frac{v^2}{c^2}}} \implies v = \ldots \\
    &E_0 = E - E_\text{кин} = \frac{E_0}{\sqrt{1 - \frac{v^2}{c^2}}} - E_\text{кин} \implies E_0 = \frac{E_\text{кин}}{\frac 1{\sqrt{1 - \frac{v^2}{c^2}}} - 1} = \ldots \\
    E_\text{кин}, v:\quad&E_\text{кин} = E - E_0 = mc^2\cbr{\frac 1{\sqrt{1 - \frac{v^2}{c^2}}} - 1} \implies m = \frac{E_\text{кин}}{c^2\cbr{\frac 1{\sqrt{1 - \frac{v^2}{c^2}}} - 1}} \\
    &E_0 = mc^2 = \frac{E_\text{кин}}{\frac 1{\sqrt{1 - \frac{v^2}{c^2}}} - 1} \\
    &p = \frac{mv}{\sqrt{1 - \frac{v^2}{c^2}}} = \frac{E_\text{кин}}{c^2\cbr{\frac 1{\sqrt{1 - \frac{v^2}{c^2}}} - 1}} \cdot \frac{v}{\sqrt{1 - \frac{v^2}{c^2}}} = \frac{{E_\text{кин}} v}{c^2\cbr{1 - {\sqrt{1 - \frac{v^2}{c^2}}}}} \\
    E_0, p:\quad&E_0 = mc^2, \quad p = \frac{mv}{\sqrt{1 - \frac{v^2}{c^2}}} \implies \frac{E_0}{p} = \frac{c^2}v{\sqrt{1 - \frac{v^2}{c^2}}} = c\sqrt{\frac{c^2}{v^2} - 1} \\
    &\sqr{\frac{E_0}{pc}} = \frac{c^2}{v^2} - 1 \implies \frac{v^2}{c^2} = \frac 1{1 + \frac{E_0^2}{p^2c^2}} \implies v = \frac c{\sqrt{1 + \frac{E_0^2}{p^2c^2}}} \\
    &{E_\text{кин}} = E - E_0 = \sqrt{E_0^2 + p^2c^2} - E_0 \\
    E_0, v:\quad&E_0 = mc^2 \implies m = \frac{E_0}{c^2} \qquad p = \frac{mv}{\sqrt{1 - \frac{v^2}{c^2}}} = \frac{E_0}{c^2} \cdot \frac{v}{\sqrt{1 - \frac{v^2}{c^2}}} \\
    &E_\text{кин}= mc^2\cbr{\frac 1{\sqrt{1 - \frac{v^2}{c^2}}} - 1} = \frac{E_0}{c^2}\cbr{\frac 1{\sqrt{1 - \frac{v^2}{c^2}}} - 1} \\
    p, v:\quad&p = \frac{mv}{\sqrt{1 - \frac{v^2}{c^2}}} \implies m = \frac p v {\sqrt{1 - \frac{v^2}{c^2}}} \implies E_0 = mc^2 =\frac {pc^2} v {\sqrt{1 - \frac{v^2}{c^2}}} \\
    &E_\text{кин} = mc^2\cbr{\frac 1{\sqrt{1 - \frac{v^2}{c^2}}} - 1} = \frac p v {\sqrt{1 - \frac{v^2}{c^2}}}\cbr{\frac 1{\sqrt{1 - \frac{v^2}{c^2}}} - 1} = \frac p v \cbr{1 - {\sqrt{1 - \frac{v^2}{c^2}}}}
    \end{align*}
}
\solutionspace{200pt}

\tasknumber{2}%
\task{%
    Протон движется со скоростью $0{,}9\,c$, где $c$~--- скорость света в вакууме.
    Каково при этом отношение кинетической энергии частицы $E_\text{кин.}$ к его энергии покоя $E_0$?
}
\answer{%
    \begin{align*}
    E &= \frac{E_0}{\sqrt{1 - \frac{v^2}{c^2}}}
            \implies \frac E{E_0}
                = \frac 1{\sqrt{1 - \frac{v^2}{c^2}}}
                = \frac 1{\sqrt{1 - \sqr{0{,}9}}}
                \approx 2{,}294,
         \\
        E_{\text{кин}} &= E - E_0
            \implies \frac{E_{\text{кин}}}{E_0}
                = \frac E{E_0} - 1
                = \frac 1{\sqrt{1 - \frac{v^2}{c^2}}} - 1
                = \frac 1{\sqrt{1 - \sqr{0{,}9}}} - 1
                \approx 1{,}294.
    \end{align*}
}
\solutionspace{150pt}

\tasknumber{3}%
\task{%
    Протон движется со скоростью $0{,}75\,c$, где $c$~--- скорость света в вакууме.
    Определите его импульс (в ответе приведите формулу и укажите численное значение).
}
\answer{%
    \begin{align*}
    E &= \frac{mc^2}{\sqrt{1 - \frac{v^2}{c^2}}}
            \approx \frac{1{,}673 \cdot 10^{-27}\,\text{кг} \cdot \sqr{3 \cdot 10^{8}\,\frac{\text{м}}{\text{с}}}}{\sqrt{1 - 0{,}75^2}}
            \approx 227{,}589 \cdot 10^{-12}\,\text{Дж},
         \\
        E_{\text{кин}} &= \frac{mc^2}{\sqrt{1 - \frac{v^2}{c^2}}} - mc^2
            = mc^2 \cbr{\frac 1{\sqrt{1 - \frac{v^2}{c^2}}} - 1} \approx \\
            &\approx \cbr{1{,}673 \cdot 10^{-27}\,\text{кг} \cdot \sqr{3 \cdot 10^{8}\,\frac{\text{м}}{\text{с}}}}
            \cdot \cbr{\frac 1{\sqrt{1 - 0{,}75^2}} - 1}
            \approx 77{,}053 \cdot 10^{-12}\,\text{Дж},
         \\
        p &= \frac{mv}{\sqrt{1 - \frac{v^2}{c^2}}}
            \approx \frac{1{,}673 \cdot 10^{-27}\,\text{кг} \cdot 0{,}75 \cdot 3 \cdot 10^{8}\,\frac{\text{м}}{\text{с}}}{\sqrt{1 - 0{,}75^2}}
            \approx 568{,}972 \cdot 10^{-21}\,\frac{\text{кг}\cdot\text{м}}{\text{с}}.
    \end{align*}
}
\solutionspace{150pt}

\tasknumber{4}%
\task{%
    При какой скорости движения (в долях скорости света) релятивистское сокращение длины движущегося тела
    составит 50\%?
}
\answer{%
    \begin{align*}
    l_0 &= \frac l{\sqrt{1 - \frac{v^2}{c^2}}}
        \implies 1 - \frac{v^2}{c^2} = \sqr{\frac l{l_0}}
        \implies \frac v c = \sqrt{1 - \sqr{\frac l{l_0}}} \implies
         \\
        \implies v &= c\sqrt{1 - \sqr{\frac l{l_0}}}
        = 3 \cdot 10^{8}\,\frac{\text{м}}{\text{с}} \cdot \sqrt{1 - \sqr{\frac {l_0 - 0{,}50l_0}{l_0}}}
        = 3 \cdot 10^{8}\,\frac{\text{м}}{\text{с}} \cdot \sqrt{1 - \sqr{1 - 0{,}50}} \approx  \\
        &\approx 0{,}866c
        \approx 260 \cdot 10^{6}\,\frac{\text{м}}{\text{с}}
        \approx 935 \cdot 10^{6}\,\frac{\text{км}}{\text{ч}}.
    \end{align*}
}
\solutionspace{150pt}

\tasknumber{5}%
\task{%
    При переходе электрона в атоме с одной стационарной орбиты на другую
    излучается фотон с энергией $1{,}01 \cdot 10^{-19}\,\text{Дж}$.
    Какова длина волны этой линии спектра?
    Постоянная Планка $h = 6{,}626 \cdot 10^{-34}\,\text{Дж}\cdot\text{с}$, скорость света $c = 3 \cdot 10^{8}\,\frac{\text{м}}{\text{с}}$.
}
\answer{%
    $
        E = h\nu = h \frac c\lambda
        \implies \lambda = \frac{hc}E
            = \frac{6{,}626 \cdot 10^{-34}\,\text{Дж}\cdot\text{с} \cdot {3 \cdot 10^{8}\,\frac{\text{м}}{\text{с}}}}{1{,}01 \cdot 10^{-19}\,\text{Дж}}
            = 1968{,}1\,\text{нм}.
    $
}
\solutionspace{150pt}

\tasknumber{6}%
\task{%
    Излучение какой длины волны поглотил атом водорода, если полная энергия в атоме увеличилась на $2 \cdot 10^{-19}\,\text{Дж}$?
    Постоянная Планка $h = 6{,}626 \cdot 10^{-34}\,\text{Дж}\cdot\text{с}$, скорость света $c = 3 \cdot 10^{8}\,\frac{\text{м}}{\text{с}}$.
}
\answer{%
    $
        E = h\nu = h \frac c\lambda
        \implies \lambda = \frac{hc}E
            = \frac{6{,}626 \cdot 10^{-34}\,\text{Дж}\cdot\text{с} \cdot {3 \cdot 10^{8}\,\frac{\text{м}}{\text{с}}}}{2 \cdot 10^{-19}\,\text{Дж}}
            = 994\,\text{нм}.
    $
}
\solutionspace{150pt}

\tasknumber{7}%
\task{%
    Сделайте схематичный рисунок энергетических уровней атома водорода
    и отметьте на нём первый (основной) уровень и последующие.
    Сколько различных длин волн может испустить атом водорода,
    находящийся в 5-м возбуждённом состоянии?
    Отметьте все соответствующие переходы на рисунке и укажите,
    при каком переходе (среди отмеченных) длина волны излучённого фотона максимальна.
}
\answer{%
    $N = 10{,}0, \text{самая короткая линия}$
}
\solutionspace{150pt}

\tasknumber{8}%
\task{%
    Сколько фотонов испускает за $40\,\text{мин}$ лазер,
    если мощность его излучения $200\,\text{мВт}$?
    Длина волны излучения $500\,\text{нм}$.
    $h = 6{,}626 \cdot 10^{-34}\,\text{Дж}\cdot\text{с}$.
}
\answer{%
    $
        N
            = \frac{E_{\text{общая}}}{E_{\text{одного фотона}}}
            = \frac{Pt}{h\nu} = \frac{Pt}{h \frac c\lambda}
            = \frac{Pt\lambda}{hc}
            = \frac{200\,\text{мВт} \cdot 40\,\text{мин} \cdot 500\,\text{нм}}{6{,}626 \cdot 10^{-34}\,\text{Дж}\cdot\text{с} \cdot 3 \cdot 10^{8}\,\frac{\text{м}}{\text{с}}}
            \approx 12{,}07 \cdot 10^{20}\units{фотонов}
    $
}
\solutionspace{120pt}

\tasknumber{9}%
\task{%
    Какая доля (от начального количества) радиоактивных ядер останется через время,
    равное двум периодам полураспада? Ответ выразить в процентах.
}
\answer{%
    \begin{align*}
    N &= N_0 \cdot 2^{- \frac t{T_{1/2}}} \implies
        \frac N{N_0} = 2^{- \frac t{T_{1/2}}}
        = 2^{-2} \approx 0{,}25 \approx 25\% \\
    N_\text{расп.} &= N_0 - N = N_0 - N_0 \cdot 2^{-\frac t{T_{1/2}}}
        = N_0\cbr{1 - 2^{-\frac t{T_{1/2}}}} \implies
        \frac{N_\text{расп.}}{N_0} = 1 - 2^{-\frac t{T_{1/2}}}
        = 1 - 2^{-2} \approx 0{,}75 \approx 75\%
    \end{align*}
}
\solutionspace{150pt}

\tasknumber{10}%
\task{%
    Сколько процентов ядер радиоактивного железа $\ce{^{59}Fe}$
    останется через $91{,}2\,\text{суток}$, если период его полураспада составляет $45{,}6\,\text{суток}$?
}
\answer{%
    \begin{align*}
    N &= N_0 \cdot 2^{-\frac t{T_{1/2}}}
        = 2^{-\frac{91{,}2\,\text{суток}}{45{,}6\,\text{суток}}}
        \approx 0{,}2500 = 25{,}00\%
    \end{align*}
}
\solutionspace{150pt}

\tasknumber{11}%
\task{%
    За $2\,\text{суток}$ от начального количества ядер радиоизотопа осталась половина.
    Каков период полураспада этого изотопа (ответ приведите в сутках)?
    Какая ещё доля (также от начального количества) распадётся, если подождать ещё столько же?
}
\answer{%
    \begin{align*}
            N &= N_0 \cdot 2^{-\frac t{T_{1/2}}}
            \implies \frac N{N_0} = 2^{-\frac t{T_{1/2}}}
            \implies \frac 1{2} = 2^{-\frac {2\,\text{суток}}{T_{1/2}}}
            \implies 1 = \frac {2\,\text{суток}}{T_{1/2}}
            \implies T_{1/2} = \frac {2\,\text{суток}}1 \approx 2\,\text{суток}.
         \\
            \delta &= \frac{N(t)}{N_0} - \frac{N(2t)}{N_0}
            = 2^{-\frac t{T_{1/2}}} - 2^{-\frac {2t}{T_{1/2}}}
            = 2^{-\frac t{T_{1/2}}}\cbr{1 - 2^{-\frac t{T_{1/2}}}}
            = \frac 1{2} \cdot \cbr{1-\frac 1{2}} \approx 0{,}250
    \end{align*}
}
\solutionspace{150pt}

\tasknumber{12}%
\task{%
    Энергия связи ядра азота \ce{^{14}_{7}N} равна $104{,}7\,\text{МэВ}$.
    Найти дефект массы этого ядра.
    Ответ выразите в а.е.м.
    и кг.
    Скорость света $c = 2{,}998 \cdot 10^{8}\,\frac{\text{м}}{\text{с}}$, элементарный заряд $e = 1{,}6 \cdot 10^{-19}\,\text{Кл}$.
}
\answer{%
    \begin{align*}
    E_\text{св.} &= \Delta m c^2 \implies \\
    \implies
            \Delta m &= \frac {E_\text{св.}}{c^2} = \frac{104{,}7\,\text{МэВ}}{\sqr{2{,}998 \cdot 10^{8}\,\frac{\text{м}}{\text{с}}}}
            = \frac{104{,}7 \cdot 10^6 \cdot 1{,}6 \cdot 10^{-19}\,\text{Дж}}{\sqr{2{,}998 \cdot 10^{8}\,\frac{\text{м}}{\text{с}}}}
            \approx 0{,}1864 \cdot 10^{-27}\,\text{кг} \approx 0{,}1122\,\text{а.е.м.}
    \end{align*}
}

\variantsplitter

\addpersonalvariant{Глеб Ковылин}

\tasknumber{1}%
\task{%
    Для частицы, движущейся с релятивистской скоростью,
    выразите $v$ и $p$ через $c$, $E_\text{кин}$ и $E_0$,
    где $E_\text{кин}$~--- кинетическая энергия частицы,
    а $E_0$, $p$ и $v$~--- её энергия покоя импульс и скорость.
}
\answer{%
    \begin{align*}
    E_\text{кин}, E_0:\quad&E = E_\text{кин} + E_0 = \frac{E_0}{\sqrt{1 - \frac{v^2}{c^2}}} \implies \sqrt{1 - \frac{v^2}{c^2}} = \frac{E_0}{{E_0} + {E_\text{кин}}} \implies v = c\sqrt{1 - \sqr{\frac{E_0}{{E_0} + {E_\text{кин}}}}} \\
    &p = \frac{mv}{\sqrt{1 - \frac{v^2}{c^2}}} = \frac{E_0}{c^2} \cdot \sqrt{1 - \sqr{\frac{E_0}{{E_0} + {E_\text{кин}}}}} \cdot \frac{{E_\text{кин}} + {E_0}}{E_0} = \frac{E_0}{c^2} \cdot \sqrt{\sqr{\frac{{E_\text{кин}} + {E_0}}{E_0}} - 1}.
    \\
    E_\text{кин}, p:\quad&E_\text{кин} = E - E_0 = mc^2\cbr{\frac 1{\sqrt{1 - \frac{v^2}{c^2}}} - 1}, p = \frac{mv}{\sqrt{1 - \frac{v^2}{c^2}}} \implies \frac{E_\text{кин}}{p} = \frac{\frac 1{\sqrt{1 - \frac{v^2}{c^2}}} - 1}{\sqrt{1 - \frac{v^2}{c^2}}} \implies v = \ldots \\
    &E_0 = E - E_\text{кин} = \frac{E_0}{\sqrt{1 - \frac{v^2}{c^2}}} - E_\text{кин} \implies E_0 = \frac{E_\text{кин}}{\frac 1{\sqrt{1 - \frac{v^2}{c^2}}} - 1} = \ldots \\
    E_\text{кин}, v:\quad&E_\text{кин} = E - E_0 = mc^2\cbr{\frac 1{\sqrt{1 - \frac{v^2}{c^2}}} - 1} \implies m = \frac{E_\text{кин}}{c^2\cbr{\frac 1{\sqrt{1 - \frac{v^2}{c^2}}} - 1}} \\
    &E_0 = mc^2 = \frac{E_\text{кин}}{\frac 1{\sqrt{1 - \frac{v^2}{c^2}}} - 1} \\
    &p = \frac{mv}{\sqrt{1 - \frac{v^2}{c^2}}} = \frac{E_\text{кин}}{c^2\cbr{\frac 1{\sqrt{1 - \frac{v^2}{c^2}}} - 1}} \cdot \frac{v}{\sqrt{1 - \frac{v^2}{c^2}}} = \frac{{E_\text{кин}} v}{c^2\cbr{1 - {\sqrt{1 - \frac{v^2}{c^2}}}}} \\
    E_0, p:\quad&E_0 = mc^2, \quad p = \frac{mv}{\sqrt{1 - \frac{v^2}{c^2}}} \implies \frac{E_0}{p} = \frac{c^2}v{\sqrt{1 - \frac{v^2}{c^2}}} = c\sqrt{\frac{c^2}{v^2} - 1} \\
    &\sqr{\frac{E_0}{pc}} = \frac{c^2}{v^2} - 1 \implies \frac{v^2}{c^2} = \frac 1{1 + \frac{E_0^2}{p^2c^2}} \implies v = \frac c{\sqrt{1 + \frac{E_0^2}{p^2c^2}}} \\
    &{E_\text{кин}} = E - E_0 = \sqrt{E_0^2 + p^2c^2} - E_0 \\
    E_0, v:\quad&E_0 = mc^2 \implies m = \frac{E_0}{c^2} \qquad p = \frac{mv}{\sqrt{1 - \frac{v^2}{c^2}}} = \frac{E_0}{c^2} \cdot \frac{v}{\sqrt{1 - \frac{v^2}{c^2}}} \\
    &E_\text{кин}= mc^2\cbr{\frac 1{\sqrt{1 - \frac{v^2}{c^2}}} - 1} = \frac{E_0}{c^2}\cbr{\frac 1{\sqrt{1 - \frac{v^2}{c^2}}} - 1} \\
    p, v:\quad&p = \frac{mv}{\sqrt{1 - \frac{v^2}{c^2}}} \implies m = \frac p v {\sqrt{1 - \frac{v^2}{c^2}}} \implies E_0 = mc^2 =\frac {pc^2} v {\sqrt{1 - \frac{v^2}{c^2}}} \\
    &E_\text{кин} = mc^2\cbr{\frac 1{\sqrt{1 - \frac{v^2}{c^2}}} - 1} = \frac p v {\sqrt{1 - \frac{v^2}{c^2}}}\cbr{\frac 1{\sqrt{1 - \frac{v^2}{c^2}}} - 1} = \frac p v \cbr{1 - {\sqrt{1 - \frac{v^2}{c^2}}}}
    \end{align*}
}
\solutionspace{200pt}

\tasknumber{2}%
\task{%
    Протон движется со скоростью $0{,}7\,c$, где $c$~--- скорость света в вакууме.
    Каково при этом отношение полной энергии частицы $E$ к его энергии покоя $E_0$?
}
\answer{%
    \begin{align*}
    E &= \frac{E_0}{\sqrt{1 - \frac{v^2}{c^2}}}
            \implies \frac E{E_0}
                = \frac 1{\sqrt{1 - \frac{v^2}{c^2}}}
                = \frac 1{\sqrt{1 - \sqr{0{,}7}}}
                \approx 1{,}400,
         \\
        E_{\text{кин}} &= E - E_0
            \implies \frac{E_{\text{кин}}}{E_0}
                = \frac E{E_0} - 1
                = \frac 1{\sqrt{1 - \frac{v^2}{c^2}}} - 1
                = \frac 1{\sqrt{1 - \sqr{0{,}7}}} - 1
                \approx 0{,}400.
    \end{align*}
}
\solutionspace{150pt}

\tasknumber{3}%
\task{%
    Электрон движется со скоростью $0{,}65\,c$, где $c$~--- скорость света в вакууме.
    Определите его полную энергию (в ответе приведите формулу и укажите численное значение).
}
\answer{%
    \begin{align*}
    E &= \frac{mc^2}{\sqrt{1 - \frac{v^2}{c^2}}}
            \approx \frac{9{,}1 \cdot 10^{-31}\,\text{кг} \cdot \sqr{3 \cdot 10^{8}\,\frac{\text{м}}{\text{с}}}}{\sqrt{1 - 0{,}65^2}}
            \approx 0{,}108 \cdot 10^{-12}\,\text{Дж},
         \\
        E_{\text{кин}} &= \frac{mc^2}{\sqrt{1 - \frac{v^2}{c^2}}} - mc^2
            = mc^2 \cbr{\frac 1{\sqrt{1 - \frac{v^2}{c^2}}} - 1} \approx \\
            &\approx \cbr{9{,}1 \cdot 10^{-31}\,\text{кг} \cdot \sqr{3 \cdot 10^{8}\,\frac{\text{м}}{\text{с}}}}
            \cdot \cbr{\frac 1{\sqrt{1 - 0{,}65^2}} - 1}
            \approx 0{,}026 \cdot 10^{-12}\,\text{Дж},
         \\
        p &= \frac{mv}{\sqrt{1 - \frac{v^2}{c^2}}}
            \approx \frac{9{,}1 \cdot 10^{-31}\,\text{кг} \cdot 0{,}65 \cdot 3 \cdot 10^{8}\,\frac{\text{м}}{\text{с}}}{\sqrt{1 - 0{,}65^2}}
            \approx 0{,}234 \cdot 10^{-21}\,\frac{\text{кг}\cdot\text{м}}{\text{с}}.
    \end{align*}
}
\solutionspace{150pt}

\tasknumber{4}%
\task{%
    При какой скорости движения (в долях скорости света) релятивистское сокращение длины движущегося тела
    составит 30\%?
}
\answer{%
    \begin{align*}
    l_0 &= \frac l{\sqrt{1 - \frac{v^2}{c^2}}}
        \implies 1 - \frac{v^2}{c^2} = \sqr{\frac l{l_0}}
        \implies \frac v c = \sqrt{1 - \sqr{\frac l{l_0}}} \implies
         \\
        \implies v &= c\sqrt{1 - \sqr{\frac l{l_0}}}
        = 3 \cdot 10^{8}\,\frac{\text{м}}{\text{с}} \cdot \sqrt{1 - \sqr{\frac {l_0 - 0{,}30l_0}{l_0}}}
        = 3 \cdot 10^{8}\,\frac{\text{м}}{\text{с}} \cdot \sqrt{1 - \sqr{1 - 0{,}30}} \approx  \\
        &\approx 0{,}714c
        \approx 214 \cdot 10^{6}\,\frac{\text{м}}{\text{с}}
        \approx 771 \cdot 10^{6}\,\frac{\text{км}}{\text{ч}}.
    \end{align*}
}
\solutionspace{150pt}

\tasknumber{5}%
\task{%
    При переходе электрона в атоме с одной стационарной орбиты на другую
    излучается фотон с энергией $7{,}07 \cdot 10^{-19}\,\text{Дж}$.
    Какова длина волны этой линии спектра?
    Постоянная Планка $h = 6{,}626 \cdot 10^{-34}\,\text{Дж}\cdot\text{с}$, скорость света $c = 3 \cdot 10^{8}\,\frac{\text{м}}{\text{с}}$.
}
\answer{%
    $
        E = h\nu = h \frac c\lambda
        \implies \lambda = \frac{hc}E
            = \frac{6{,}626 \cdot 10^{-34}\,\text{Дж}\cdot\text{с} \cdot {3 \cdot 10^{8}\,\frac{\text{м}}{\text{с}}}}{7{,}07 \cdot 10^{-19}\,\text{Дж}}
            = 281{,}16\,\text{нм}.
    $
}
\solutionspace{150pt}

\tasknumber{6}%
\task{%
    Излучение какой длины волны поглотил атом водорода, если полная энергия в атоме увеличилась на $3 \cdot 10^{-19}\,\text{Дж}$?
    Постоянная Планка $h = 6{,}626 \cdot 10^{-34}\,\text{Дж}\cdot\text{с}$, скорость света $c = 3 \cdot 10^{8}\,\frac{\text{м}}{\text{с}}$.
}
\answer{%
    $
        E = h\nu = h \frac c\lambda
        \implies \lambda = \frac{hc}E
            = \frac{6{,}626 \cdot 10^{-34}\,\text{Дж}\cdot\text{с} \cdot {3 \cdot 10^{8}\,\frac{\text{м}}{\text{с}}}}{3 \cdot 10^{-19}\,\text{Дж}}
            = 663\,\text{нм}.
    $
}
\solutionspace{150pt}

\tasknumber{7}%
\task{%
    Сделайте схематичный рисунок энергетических уровней атома водорода
    и отметьте на нём первый (основной) уровень и последующие.
    Сколько различных длин волн может испустить атом водорода,
    находящийся в 3-м возбуждённом состоянии?
    Отметьте все соответствующие переходы на рисунке и укажите,
    при каком переходе (среди отмеченных) энергия излучённого фотона максимальна.
}
\answer{%
    $N = 3{,}0, \text{самая длинная линия}$
}
\solutionspace{150pt}

\tasknumber{8}%
\task{%
    Сколько фотонов испускает за $5\,\text{мин}$ лазер,
    если мощность его излучения $200\,\text{мВт}$?
    Длина волны излучения $500\,\text{нм}$.
    $h = 6{,}626 \cdot 10^{-34}\,\text{Дж}\cdot\text{с}$.
}
\answer{%
    $
        N
            = \frac{E_{\text{общая}}}{E_{\text{одного фотона}}}
            = \frac{Pt}{h\nu} = \frac{Pt}{h \frac c\lambda}
            = \frac{Pt\lambda}{hc}
            = \frac{200\,\text{мВт} \cdot 5\,\text{мин} \cdot 500\,\text{нм}}{6{,}626 \cdot 10^{-34}\,\text{Дж}\cdot\text{с} \cdot 3 \cdot 10^{8}\,\frac{\text{м}}{\text{с}}}
            \approx 1{,}51 \cdot 10^{20}\units{фотонов}
    $
}
\solutionspace{120pt}

\tasknumber{9}%
\task{%
    Какая доля (от начального количества) радиоактивных ядер распадётся через время,
    равное четырём периодам полураспада? Ответ выразить в процентах.
}
\answer{%
    \begin{align*}
    N &= N_0 \cdot 2^{- \frac t{T_{1/2}}} \implies
        \frac N{N_0} = 2^{- \frac t{T_{1/2}}}
        = 2^{-4} \approx 0{,}06 \approx 6\% \\
    N_\text{расп.} &= N_0 - N = N_0 - N_0 \cdot 2^{-\frac t{T_{1/2}}}
        = N_0\cbr{1 - 2^{-\frac t{T_{1/2}}}} \implies
        \frac{N_\text{расп.}}{N_0} = 1 - 2^{-\frac t{T_{1/2}}}
        = 1 - 2^{-4} \approx 0{,}94 \approx 94\%
    \end{align*}
}
\solutionspace{150pt}

\tasknumber{10}%
\task{%
    Сколько процентов ядер радиоактивного железа $\ce{^{59}Fe}$
    останется через $136{,}8\,\text{суток}$, если период его полураспада составляет $45{,}6\,\text{суток}$?
}
\answer{%
    \begin{align*}
    N &= N_0 \cdot 2^{-\frac t{T_{1/2}}}
        = 2^{-\frac{136{,}8\,\text{суток}}{45{,}6\,\text{суток}}}
        \approx 0{,}1250 = 12{,}50\%
    \end{align*}
}
\solutionspace{150pt}

\tasknumber{11}%
\task{%
    За $5\,\text{суток}$ от начального количества ядер радиоизотопа осталась одна шестнадцатая.
    Каков период полураспада этого изотопа (ответ приведите в сутках)?
    Какая ещё доля (также от начального количества) распадётся, если подождать ещё столько же?
}
\answer{%
    \begin{align*}
            N &= N_0 \cdot 2^{-\frac t{T_{1/2}}}
            \implies \frac N{N_0} = 2^{-\frac t{T_{1/2}}}
            \implies \frac 1{16} = 2^{-\frac {5\,\text{суток}}{T_{1/2}}}
            \implies 4 = \frac {5\,\text{суток}}{T_{1/2}}
            \implies T_{1/2} = \frac {5\,\text{суток}}4 \approx 1{,}25\,\text{суток}.
         \\
            \delta &= \frac{N(t)}{N_0} - \frac{N(2t)}{N_0}
            = 2^{-\frac t{T_{1/2}}} - 2^{-\frac {2t}{T_{1/2}}}
            = 2^{-\frac t{T_{1/2}}}\cbr{1 - 2^{-\frac t{T_{1/2}}}}
            = \frac 1{16} \cdot \cbr{1-\frac 1{16}} \approx 0{,}059
    \end{align*}
}
\solutionspace{150pt}

\tasknumber{12}%
\task{%
    Энергия связи ядра бора \ce{^{11}_{5}B} равна $76{,}2\,\text{МэВ}$.
    Найти дефект массы этого ядра.
    Ответ выразите в а.е.м.
    и кг.
    Скорость света $c = 2{,}998 \cdot 10^{8}\,\frac{\text{м}}{\text{с}}$, элементарный заряд $e = 1{,}6 \cdot 10^{-19}\,\text{Кл}$.
}
\answer{%
    \begin{align*}
    E_\text{св.} &= \Delta m c^2 \implies \\
    \implies
            \Delta m &= \frac {E_\text{св.}}{c^2} = \frac{76{,}2\,\text{МэВ}}{\sqr{2{,}998 \cdot 10^{8}\,\frac{\text{м}}{\text{с}}}}
            = \frac{76{,}2 \cdot 10^6 \cdot 1{,}6 \cdot 10^{-19}\,\text{Дж}}{\sqr{2{,}998 \cdot 10^{8}\,\frac{\text{м}}{\text{с}}}}
            \approx 0{,}1356 \cdot 10^{-27}\,\text{кг} \approx 0{,}0817\,\text{а.е.м.}
    \end{align*}
}

\variantsplitter

\addpersonalvariant{Даниил Космынин}

\tasknumber{1}%
\task{%
    Для частицы, движущейся с релятивистской скоростью,
    выразите $E_0$ и $p$ через $c$, $E_\text{кин}$ и $v$,
    где $E_\text{кин}$~--- кинетическая энергия частицы,
    а $E_0$, $p$ и $v$~--- её энергия покоя импульс и скорость.
}
\answer{%
    \begin{align*}
    E_\text{кин}, E_0:\quad&E = E_\text{кин} + E_0 = \frac{E_0}{\sqrt{1 - \frac{v^2}{c^2}}} \implies \sqrt{1 - \frac{v^2}{c^2}} = \frac{E_0}{{E_0} + {E_\text{кин}}} \implies v = c\sqrt{1 - \sqr{\frac{E_0}{{E_0} + {E_\text{кин}}}}} \\
    &p = \frac{mv}{\sqrt{1 - \frac{v^2}{c^2}}} = \frac{E_0}{c^2} \cdot \sqrt{1 - \sqr{\frac{E_0}{{E_0} + {E_\text{кин}}}}} \cdot \frac{{E_\text{кин}} + {E_0}}{E_0} = \frac{E_0}{c^2} \cdot \sqrt{\sqr{\frac{{E_\text{кин}} + {E_0}}{E_0}} - 1}.
    \\
    E_\text{кин}, p:\quad&E_\text{кин} = E - E_0 = mc^2\cbr{\frac 1{\sqrt{1 - \frac{v^2}{c^2}}} - 1}, p = \frac{mv}{\sqrt{1 - \frac{v^2}{c^2}}} \implies \frac{E_\text{кин}}{p} = \frac{\frac 1{\sqrt{1 - \frac{v^2}{c^2}}} - 1}{\sqrt{1 - \frac{v^2}{c^2}}} \implies v = \ldots \\
    &E_0 = E - E_\text{кин} = \frac{E_0}{\sqrt{1 - \frac{v^2}{c^2}}} - E_\text{кин} \implies E_0 = \frac{E_\text{кин}}{\frac 1{\sqrt{1 - \frac{v^2}{c^2}}} - 1} = \ldots \\
    E_\text{кин}, v:\quad&E_\text{кин} = E - E_0 = mc^2\cbr{\frac 1{\sqrt{1 - \frac{v^2}{c^2}}} - 1} \implies m = \frac{E_\text{кин}}{c^2\cbr{\frac 1{\sqrt{1 - \frac{v^2}{c^2}}} - 1}} \\
    &E_0 = mc^2 = \frac{E_\text{кин}}{\frac 1{\sqrt{1 - \frac{v^2}{c^2}}} - 1} \\
    &p = \frac{mv}{\sqrt{1 - \frac{v^2}{c^2}}} = \frac{E_\text{кин}}{c^2\cbr{\frac 1{\sqrt{1 - \frac{v^2}{c^2}}} - 1}} \cdot \frac{v}{\sqrt{1 - \frac{v^2}{c^2}}} = \frac{{E_\text{кин}} v}{c^2\cbr{1 - {\sqrt{1 - \frac{v^2}{c^2}}}}} \\
    E_0, p:\quad&E_0 = mc^2, \quad p = \frac{mv}{\sqrt{1 - \frac{v^2}{c^2}}} \implies \frac{E_0}{p} = \frac{c^2}v{\sqrt{1 - \frac{v^2}{c^2}}} = c\sqrt{\frac{c^2}{v^2} - 1} \\
    &\sqr{\frac{E_0}{pc}} = \frac{c^2}{v^2} - 1 \implies \frac{v^2}{c^2} = \frac 1{1 + \frac{E_0^2}{p^2c^2}} \implies v = \frac c{\sqrt{1 + \frac{E_0^2}{p^2c^2}}} \\
    &{E_\text{кин}} = E - E_0 = \sqrt{E_0^2 + p^2c^2} - E_0 \\
    E_0, v:\quad&E_0 = mc^2 \implies m = \frac{E_0}{c^2} \qquad p = \frac{mv}{\sqrt{1 - \frac{v^2}{c^2}}} = \frac{E_0}{c^2} \cdot \frac{v}{\sqrt{1 - \frac{v^2}{c^2}}} \\
    &E_\text{кин}= mc^2\cbr{\frac 1{\sqrt{1 - \frac{v^2}{c^2}}} - 1} = \frac{E_0}{c^2}\cbr{\frac 1{\sqrt{1 - \frac{v^2}{c^2}}} - 1} \\
    p, v:\quad&p = \frac{mv}{\sqrt{1 - \frac{v^2}{c^2}}} \implies m = \frac p v {\sqrt{1 - \frac{v^2}{c^2}}} \implies E_0 = mc^2 =\frac {pc^2} v {\sqrt{1 - \frac{v^2}{c^2}}} \\
    &E_\text{кин} = mc^2\cbr{\frac 1{\sqrt{1 - \frac{v^2}{c^2}}} - 1} = \frac p v {\sqrt{1 - \frac{v^2}{c^2}}}\cbr{\frac 1{\sqrt{1 - \frac{v^2}{c^2}}} - 1} = \frac p v \cbr{1 - {\sqrt{1 - \frac{v^2}{c^2}}}}
    \end{align*}
}
\solutionspace{200pt}

\tasknumber{2}%
\task{%
    Позитрон движется со скоростью $0{,}7\,c$, где $c$~--- скорость света в вакууме.
    Каково при этом отношение кинетической энергии частицы $E_\text{кин.}$ к его энергии покоя $E_0$?
}
\answer{%
    \begin{align*}
    E &= \frac{E_0}{\sqrt{1 - \frac{v^2}{c^2}}}
            \implies \frac E{E_0}
                = \frac 1{\sqrt{1 - \frac{v^2}{c^2}}}
                = \frac 1{\sqrt{1 - \sqr{0{,}7}}}
                \approx 1{,}400,
         \\
        E_{\text{кин}} &= E - E_0
            \implies \frac{E_{\text{кин}}}{E_0}
                = \frac E{E_0} - 1
                = \frac 1{\sqrt{1 - \frac{v^2}{c^2}}} - 1
                = \frac 1{\sqrt{1 - \sqr{0{,}7}}} - 1
                \approx 0{,}400.
    \end{align*}
}
\solutionspace{150pt}

\tasknumber{3}%
\task{%
    Протон движется со скоростью $0{,}75\,c$, где $c$~--- скорость света в вакууме.
    Определите его полную энергию (в ответе приведите формулу и укажите численное значение).
}
\answer{%
    \begin{align*}
    E &= \frac{mc^2}{\sqrt{1 - \frac{v^2}{c^2}}}
            \approx \frac{1{,}673 \cdot 10^{-27}\,\text{кг} \cdot \sqr{3 \cdot 10^{8}\,\frac{\text{м}}{\text{с}}}}{\sqrt{1 - 0{,}75^2}}
            \approx 227{,}589 \cdot 10^{-12}\,\text{Дж},
         \\
        E_{\text{кин}} &= \frac{mc^2}{\sqrt{1 - \frac{v^2}{c^2}}} - mc^2
            = mc^2 \cbr{\frac 1{\sqrt{1 - \frac{v^2}{c^2}}} - 1} \approx \\
            &\approx \cbr{1{,}673 \cdot 10^{-27}\,\text{кг} \cdot \sqr{3 \cdot 10^{8}\,\frac{\text{м}}{\text{с}}}}
            \cdot \cbr{\frac 1{\sqrt{1 - 0{,}75^2}} - 1}
            \approx 77{,}053 \cdot 10^{-12}\,\text{Дж},
         \\
        p &= \frac{mv}{\sqrt{1 - \frac{v^2}{c^2}}}
            \approx \frac{1{,}673 \cdot 10^{-27}\,\text{кг} \cdot 0{,}75 \cdot 3 \cdot 10^{8}\,\frac{\text{м}}{\text{с}}}{\sqrt{1 - 0{,}75^2}}
            \approx 568{,}972 \cdot 10^{-21}\,\frac{\text{кг}\cdot\text{м}}{\text{с}}.
    \end{align*}
}
\solutionspace{150pt}

\tasknumber{4}%
\task{%
    При какой скорости движения (в долях скорости света) релятивистское сокращение длины движущегося тела
    составит 30\%?
}
\answer{%
    \begin{align*}
    l_0 &= \frac l{\sqrt{1 - \frac{v^2}{c^2}}}
        \implies 1 - \frac{v^2}{c^2} = \sqr{\frac l{l_0}}
        \implies \frac v c = \sqrt{1 - \sqr{\frac l{l_0}}} \implies
         \\
        \implies v &= c\sqrt{1 - \sqr{\frac l{l_0}}}
        = 3 \cdot 10^{8}\,\frac{\text{м}}{\text{с}} \cdot \sqrt{1 - \sqr{\frac {l_0 - 0{,}30l_0}{l_0}}}
        = 3 \cdot 10^{8}\,\frac{\text{м}}{\text{с}} \cdot \sqrt{1 - \sqr{1 - 0{,}30}} \approx  \\
        &\approx 0{,}714c
        \approx 214 \cdot 10^{6}\,\frac{\text{м}}{\text{с}}
        \approx 771 \cdot 10^{6}\,\frac{\text{км}}{\text{ч}}.
    \end{align*}
}
\solutionspace{150pt}

\tasknumber{5}%
\task{%
    При переходе электрона в атоме с одной стационарной орбиты на другую
    излучается фотон с энергией $0{,}55 \cdot 10^{-19}\,\text{Дж}$.
    Какова длина волны этой линии спектра?
    Постоянная Планка $h = 6{,}626 \cdot 10^{-34}\,\text{Дж}\cdot\text{с}$, скорость света $c = 3 \cdot 10^{8}\,\frac{\text{м}}{\text{с}}$.
}
\answer{%
    $
        E = h\nu = h \frac c\lambda
        \implies \lambda = \frac{hc}E
            = \frac{6{,}626 \cdot 10^{-34}\,\text{Дж}\cdot\text{с} \cdot {3 \cdot 10^{8}\,\frac{\text{м}}{\text{с}}}}{0{,}55 \cdot 10^{-19}\,\text{Дж}}
            = 3614\,\text{нм}.
    $
}
\solutionspace{150pt}

\tasknumber{6}%
\task{%
    Излучение какой длины волны поглотил атом водорода, если полная энергия в атоме увеличилась на $3 \cdot 10^{-19}\,\text{Дж}$?
    Постоянная Планка $h = 6{,}626 \cdot 10^{-34}\,\text{Дж}\cdot\text{с}$, скорость света $c = 3 \cdot 10^{8}\,\frac{\text{м}}{\text{с}}$.
}
\answer{%
    $
        E = h\nu = h \frac c\lambda
        \implies \lambda = \frac{hc}E
            = \frac{6{,}626 \cdot 10^{-34}\,\text{Дж}\cdot\text{с} \cdot {3 \cdot 10^{8}\,\frac{\text{м}}{\text{с}}}}{3 \cdot 10^{-19}\,\text{Дж}}
            = 663\,\text{нм}.
    $
}
\solutionspace{150pt}

\tasknumber{7}%
\task{%
    Сделайте схематичный рисунок энергетических уровней атома водорода
    и отметьте на нём первый (основной) уровень и последующие.
    Сколько различных длин волн может испустить атом водорода,
    находящийся в 3-м возбуждённом состоянии?
    Отметьте все соответствующие переходы на рисунке и укажите,
    при каком переходе (среди отмеченных) энергия излучённого фотона максимальна.
}
\answer{%
    $N = 3{,}0, \text{самая длинная линия}$
}
\solutionspace{150pt}

\tasknumber{8}%
\task{%
    Сколько фотонов испускает за $40\,\text{мин}$ лазер,
    если мощность его излучения $75\,\text{мВт}$?
    Длина волны излучения $600\,\text{нм}$.
    $h = 6{,}626 \cdot 10^{-34}\,\text{Дж}\cdot\text{с}$.
}
\answer{%
    $
        N
            = \frac{E_{\text{общая}}}{E_{\text{одного фотона}}}
            = \frac{Pt}{h\nu} = \frac{Pt}{h \frac c\lambda}
            = \frac{Pt\lambda}{hc}
            = \frac{75\,\text{мВт} \cdot 40\,\text{мин} \cdot 600\,\text{нм}}{6{,}626 \cdot 10^{-34}\,\text{Дж}\cdot\text{с} \cdot 3 \cdot 10^{8}\,\frac{\text{м}}{\text{с}}}
            \approx 5{,}43 \cdot 10^{20}\units{фотонов}
    $
}
\solutionspace{120pt}

\tasknumber{9}%
\task{%
    Какая доля (от начального количества) радиоактивных ядер распадётся через время,
    равное четырём периодам полураспада? Ответ выразить в процентах.
}
\answer{%
    \begin{align*}
    N &= N_0 \cdot 2^{- \frac t{T_{1/2}}} \implies
        \frac N{N_0} = 2^{- \frac t{T_{1/2}}}
        = 2^{-4} \approx 0{,}06 \approx 6\% \\
    N_\text{расп.} &= N_0 - N = N_0 - N_0 \cdot 2^{-\frac t{T_{1/2}}}
        = N_0\cbr{1 - 2^{-\frac t{T_{1/2}}}} \implies
        \frac{N_\text{расп.}}{N_0} = 1 - 2^{-\frac t{T_{1/2}}}
        = 1 - 2^{-4} \approx 0{,}94 \approx 94\%
    \end{align*}
}
\solutionspace{150pt}

\tasknumber{10}%
\task{%
    Сколько процентов ядер радиоактивного железа $\ce{^{59}Fe}$
    останется через $182{,}4\,\text{суток}$, если период его полураспада составляет $45{,}6\,\text{суток}$?
}
\answer{%
    \begin{align*}
    N &= N_0 \cdot 2^{-\frac t{T_{1/2}}}
        = 2^{-\frac{182{,}4\,\text{суток}}{45{,}6\,\text{суток}}}
        \approx 0{,}0625 = 6{,}25\%
    \end{align*}
}
\solutionspace{150pt}

\tasknumber{11}%
\task{%
    За $4\,\text{суток}$ от начального количества ядер радиоизотопа осталась половина.
    Каков период полураспада этого изотопа (ответ приведите в сутках)?
    Какая ещё доля (также от начального количества) распадётся, если подождать ещё столько же?
}
\answer{%
    \begin{align*}
            N &= N_0 \cdot 2^{-\frac t{T_{1/2}}}
            \implies \frac N{N_0} = 2^{-\frac t{T_{1/2}}}
            \implies \frac 1{2} = 2^{-\frac {4\,\text{суток}}{T_{1/2}}}
            \implies 1 = \frac {4\,\text{суток}}{T_{1/2}}
            \implies T_{1/2} = \frac {4\,\text{суток}}1 \approx 4\,\text{суток}.
         \\
            \delta &= \frac{N(t)}{N_0} - \frac{N(2t)}{N_0}
            = 2^{-\frac t{T_{1/2}}} - 2^{-\frac {2t}{T_{1/2}}}
            = 2^{-\frac t{T_{1/2}}}\cbr{1 - 2^{-\frac t{T_{1/2}}}}
            = \frac 1{2} \cdot \cbr{1-\frac 1{2}} \approx 0{,}250
    \end{align*}
}
\solutionspace{150pt}

\tasknumber{12}%
\task{%
    Энергия связи ядра кислорода \ce{^{17}_{8}O} равна $131{,}8\,\text{МэВ}$.
    Найти дефект массы этого ядра.
    Ответ выразите в а.е.м.
    и кг.
    Скорость света $c = 2{,}998 \cdot 10^{8}\,\frac{\text{м}}{\text{с}}$, элементарный заряд $e = 1{,}6 \cdot 10^{-19}\,\text{Кл}$.
}
\answer{%
    \begin{align*}
    E_\text{св.} &= \Delta m c^2 \implies \\
    \implies
            \Delta m &= \frac {E_\text{св.}}{c^2} = \frac{131{,}8\,\text{МэВ}}{\sqr{2{,}998 \cdot 10^{8}\,\frac{\text{м}}{\text{с}}}}
            = \frac{131{,}8 \cdot 10^6 \cdot 1{,}6 \cdot 10^{-19}\,\text{Дж}}{\sqr{2{,}998 \cdot 10^{8}\,\frac{\text{м}}{\text{с}}}}
            \approx 0{,}235 \cdot 10^{-27}\,\text{кг} \approx 0{,}1413\,\text{а.е.м.}
    \end{align*}
}

\variantsplitter

\addpersonalvariant{Алина Леоничева}

\tasknumber{1}%
\task{%
    Для частицы, движущейся с релятивистской скоростью,
    выразите $E_\text{кин}$ и $E_0$ через $c$, $v$ и $p$,
    где $E_\text{кин}$~--- кинетическая энергия частицы,
    а $E_0$, $p$ и $v$~--- её энергия покоя импульс и скорость.
}
\answer{%
    \begin{align*}
    E_\text{кин}, E_0:\quad&E = E_\text{кин} + E_0 = \frac{E_0}{\sqrt{1 - \frac{v^2}{c^2}}} \implies \sqrt{1 - \frac{v^2}{c^2}} = \frac{E_0}{{E_0} + {E_\text{кин}}} \implies v = c\sqrt{1 - \sqr{\frac{E_0}{{E_0} + {E_\text{кин}}}}} \\
    &p = \frac{mv}{\sqrt{1 - \frac{v^2}{c^2}}} = \frac{E_0}{c^2} \cdot \sqrt{1 - \sqr{\frac{E_0}{{E_0} + {E_\text{кин}}}}} \cdot \frac{{E_\text{кин}} + {E_0}}{E_0} = \frac{E_0}{c^2} \cdot \sqrt{\sqr{\frac{{E_\text{кин}} + {E_0}}{E_0}} - 1}.
    \\
    E_\text{кин}, p:\quad&E_\text{кин} = E - E_0 = mc^2\cbr{\frac 1{\sqrt{1 - \frac{v^2}{c^2}}} - 1}, p = \frac{mv}{\sqrt{1 - \frac{v^2}{c^2}}} \implies \frac{E_\text{кин}}{p} = \frac{\frac 1{\sqrt{1 - \frac{v^2}{c^2}}} - 1}{\sqrt{1 - \frac{v^2}{c^2}}} \implies v = \ldots \\
    &E_0 = E - E_\text{кин} = \frac{E_0}{\sqrt{1 - \frac{v^2}{c^2}}} - E_\text{кин} \implies E_0 = \frac{E_\text{кин}}{\frac 1{\sqrt{1 - \frac{v^2}{c^2}}} - 1} = \ldots \\
    E_\text{кин}, v:\quad&E_\text{кин} = E - E_0 = mc^2\cbr{\frac 1{\sqrt{1 - \frac{v^2}{c^2}}} - 1} \implies m = \frac{E_\text{кин}}{c^2\cbr{\frac 1{\sqrt{1 - \frac{v^2}{c^2}}} - 1}} \\
    &E_0 = mc^2 = \frac{E_\text{кин}}{\frac 1{\sqrt{1 - \frac{v^2}{c^2}}} - 1} \\
    &p = \frac{mv}{\sqrt{1 - \frac{v^2}{c^2}}} = \frac{E_\text{кин}}{c^2\cbr{\frac 1{\sqrt{1 - \frac{v^2}{c^2}}} - 1}} \cdot \frac{v}{\sqrt{1 - \frac{v^2}{c^2}}} = \frac{{E_\text{кин}} v}{c^2\cbr{1 - {\sqrt{1 - \frac{v^2}{c^2}}}}} \\
    E_0, p:\quad&E_0 = mc^2, \quad p = \frac{mv}{\sqrt{1 - \frac{v^2}{c^2}}} \implies \frac{E_0}{p} = \frac{c^2}v{\sqrt{1 - \frac{v^2}{c^2}}} = c\sqrt{\frac{c^2}{v^2} - 1} \\
    &\sqr{\frac{E_0}{pc}} = \frac{c^2}{v^2} - 1 \implies \frac{v^2}{c^2} = \frac 1{1 + \frac{E_0^2}{p^2c^2}} \implies v = \frac c{\sqrt{1 + \frac{E_0^2}{p^2c^2}}} \\
    &{E_\text{кин}} = E - E_0 = \sqrt{E_0^2 + p^2c^2} - E_0 \\
    E_0, v:\quad&E_0 = mc^2 \implies m = \frac{E_0}{c^2} \qquad p = \frac{mv}{\sqrt{1 - \frac{v^2}{c^2}}} = \frac{E_0}{c^2} \cdot \frac{v}{\sqrt{1 - \frac{v^2}{c^2}}} \\
    &E_\text{кин}= mc^2\cbr{\frac 1{\sqrt{1 - \frac{v^2}{c^2}}} - 1} = \frac{E_0}{c^2}\cbr{\frac 1{\sqrt{1 - \frac{v^2}{c^2}}} - 1} \\
    p, v:\quad&p = \frac{mv}{\sqrt{1 - \frac{v^2}{c^2}}} \implies m = \frac p v {\sqrt{1 - \frac{v^2}{c^2}}} \implies E_0 = mc^2 =\frac {pc^2} v {\sqrt{1 - \frac{v^2}{c^2}}} \\
    &E_\text{кин} = mc^2\cbr{\frac 1{\sqrt{1 - \frac{v^2}{c^2}}} - 1} = \frac p v {\sqrt{1 - \frac{v^2}{c^2}}}\cbr{\frac 1{\sqrt{1 - \frac{v^2}{c^2}}} - 1} = \frac p v \cbr{1 - {\sqrt{1 - \frac{v^2}{c^2}}}}
    \end{align*}
}
\solutionspace{200pt}

\tasknumber{2}%
\task{%
    Позитрон движется со скоростью $0{,}7\,c$, где $c$~--- скорость света в вакууме.
    Каково при этом отношение кинетической энергии частицы $E_\text{кин.}$ к его энергии покоя $E_0$?
}
\answer{%
    \begin{align*}
    E &= \frac{E_0}{\sqrt{1 - \frac{v^2}{c^2}}}
            \implies \frac E{E_0}
                = \frac 1{\sqrt{1 - \frac{v^2}{c^2}}}
                = \frac 1{\sqrt{1 - \sqr{0{,}7}}}
                \approx 1{,}400,
         \\
        E_{\text{кин}} &= E - E_0
            \implies \frac{E_{\text{кин}}}{E_0}
                = \frac E{E_0} - 1
                = \frac 1{\sqrt{1 - \frac{v^2}{c^2}}} - 1
                = \frac 1{\sqrt{1 - \sqr{0{,}7}}} - 1
                \approx 0{,}400.
    \end{align*}
}
\solutionspace{150pt}

\tasknumber{3}%
\task{%
    Протон движется со скоростью $0{,}85\,c$, где $c$~--- скорость света в вакууме.
    Определите его кинетическую энергию (в ответе приведите формулу и укажите численное значение).
}
\answer{%
    \begin{align*}
    E &= \frac{mc^2}{\sqrt{1 - \frac{v^2}{c^2}}}
            \approx \frac{1{,}673 \cdot 10^{-27}\,\text{кг} \cdot \sqr{3 \cdot 10^{8}\,\frac{\text{м}}{\text{с}}}}{\sqrt{1 - 0{,}85^2}}
            \approx 285{,}765 \cdot 10^{-12}\,\text{Дж},
         \\
        E_{\text{кин}} &= \frac{mc^2}{\sqrt{1 - \frac{v^2}{c^2}}} - mc^2
            = mc^2 \cbr{\frac 1{\sqrt{1 - \frac{v^2}{c^2}}} - 1} \approx \\
            &\approx \cbr{1{,}673 \cdot 10^{-27}\,\text{кг} \cdot \sqr{3 \cdot 10^{8}\,\frac{\text{м}}{\text{с}}}}
            \cdot \cbr{\frac 1{\sqrt{1 - 0{,}85^2}} - 1}
            \approx 135{,}229 \cdot 10^{-12}\,\text{Дж},
         \\
        p &= \frac{mv}{\sqrt{1 - \frac{v^2}{c^2}}}
            \approx \frac{1{,}673 \cdot 10^{-27}\,\text{кг} \cdot 0{,}85 \cdot 3 \cdot 10^{8}\,\frac{\text{м}}{\text{с}}}{\sqrt{1 - 0{,}85^2}}
            \approx 809{,}666 \cdot 10^{-21}\,\frac{\text{кг}\cdot\text{м}}{\text{с}}.
    \end{align*}
}
\solutionspace{150pt}

\tasknumber{4}%
\task{%
    При какой скорости движения (в долях скорости света) релятивистское сокращение длины движущегося тела
    составит 50\%?
}
\answer{%
    \begin{align*}
    l_0 &= \frac l{\sqrt{1 - \frac{v^2}{c^2}}}
        \implies 1 - \frac{v^2}{c^2} = \sqr{\frac l{l_0}}
        \implies \frac v c = \sqrt{1 - \sqr{\frac l{l_0}}} \implies
         \\
        \implies v &= c\sqrt{1 - \sqr{\frac l{l_0}}}
        = 3 \cdot 10^{8}\,\frac{\text{м}}{\text{с}} \cdot \sqrt{1 - \sqr{\frac {l_0 - 0{,}50l_0}{l_0}}}
        = 3 \cdot 10^{8}\,\frac{\text{м}}{\text{с}} \cdot \sqrt{1 - \sqr{1 - 0{,}50}} \approx  \\
        &\approx 0{,}866c
        \approx 260 \cdot 10^{6}\,\frac{\text{м}}{\text{с}}
        \approx 935 \cdot 10^{6}\,\frac{\text{км}}{\text{ч}}.
    \end{align*}
}
\solutionspace{150pt}

\tasknumber{5}%
\task{%
    При переходе электрона в атоме с одной стационарной орбиты на другую
    излучается фотон с энергией $0{,}55 \cdot 10^{-19}\,\text{Дж}$.
    Какова длина волны этой линии спектра?
    Постоянная Планка $h = 6{,}626 \cdot 10^{-34}\,\text{Дж}\cdot\text{с}$, скорость света $c = 3 \cdot 10^{8}\,\frac{\text{м}}{\text{с}}$.
}
\answer{%
    $
        E = h\nu = h \frac c\lambda
        \implies \lambda = \frac{hc}E
            = \frac{6{,}626 \cdot 10^{-34}\,\text{Дж}\cdot\text{с} \cdot {3 \cdot 10^{8}\,\frac{\text{м}}{\text{с}}}}{0{,}55 \cdot 10^{-19}\,\text{Дж}}
            = 3614\,\text{нм}.
    $
}
\solutionspace{150pt}

\tasknumber{6}%
\task{%
    Излучение какой длины волны поглотил атом водорода, если полная энергия в атоме увеличилась на $3 \cdot 10^{-19}\,\text{Дж}$?
    Постоянная Планка $h = 6{,}626 \cdot 10^{-34}\,\text{Дж}\cdot\text{с}$, скорость света $c = 3 \cdot 10^{8}\,\frac{\text{м}}{\text{с}}$.
}
\answer{%
    $
        E = h\nu = h \frac c\lambda
        \implies \lambda = \frac{hc}E
            = \frac{6{,}626 \cdot 10^{-34}\,\text{Дж}\cdot\text{с} \cdot {3 \cdot 10^{8}\,\frac{\text{м}}{\text{с}}}}{3 \cdot 10^{-19}\,\text{Дж}}
            = 663\,\text{нм}.
    $
}
\solutionspace{150pt}

\tasknumber{7}%
\task{%
    Сделайте схематичный рисунок энергетических уровней атома водорода
    и отметьте на нём первый (основной) уровень и последующие.
    Сколько различных длин волн может испустить атом водорода,
    находящийся в 4-м возбуждённом состоянии?
    Отметьте все соответствующие переходы на рисунке и укажите,
    при каком переходе (среди отмеченных) энергия излучённого фотона минимальна.
}
\answer{%
    $N = 6{,}0, \text{самая короткая линия}$
}
\solutionspace{150pt}

\tasknumber{8}%
\task{%
    Сколько фотонов испускает за $40\,\text{мин}$ лазер,
    если мощность его излучения $200\,\text{мВт}$?
    Длина волны излучения $750\,\text{нм}$.
    $h = 6{,}626 \cdot 10^{-34}\,\text{Дж}\cdot\text{с}$.
}
\answer{%
    $
        N
            = \frac{E_{\text{общая}}}{E_{\text{одного фотона}}}
            = \frac{Pt}{h\nu} = \frac{Pt}{h \frac c\lambda}
            = \frac{Pt\lambda}{hc}
            = \frac{200\,\text{мВт} \cdot 40\,\text{мин} \cdot 750\,\text{нм}}{6{,}626 \cdot 10^{-34}\,\text{Дж}\cdot\text{с} \cdot 3 \cdot 10^{8}\,\frac{\text{м}}{\text{с}}}
            \approx 18{,}11 \cdot 10^{20}\units{фотонов}
    $
}
\solutionspace{120pt}

\tasknumber{9}%
\task{%
    Какая доля (от начального количества) радиоактивных ядер распадётся через время,
    равное трём периодам полураспада? Ответ выразить в процентах.
}
\answer{%
    \begin{align*}
    N &= N_0 \cdot 2^{- \frac t{T_{1/2}}} \implies
        \frac N{N_0} = 2^{- \frac t{T_{1/2}}}
        = 2^{-3} \approx 0{,}12 \approx 12\% \\
    N_\text{расп.} &= N_0 - N = N_0 - N_0 \cdot 2^{-\frac t{T_{1/2}}}
        = N_0\cbr{1 - 2^{-\frac t{T_{1/2}}}} \implies
        \frac{N_\text{расп.}}{N_0} = 1 - 2^{-\frac t{T_{1/2}}}
        = 1 - 2^{-3} \approx 0{,}88 \approx 88\%
    \end{align*}
}
\solutionspace{150pt}

\tasknumber{10}%
\task{%
    Сколько процентов ядер радиоактивного железа $\ce{^{59}Fe}$
    останется через $136{,}8\,\text{суток}$, если период его полураспада составляет $45{,}6\,\text{суток}$?
}
\answer{%
    \begin{align*}
    N &= N_0 \cdot 2^{-\frac t{T_{1/2}}}
        = 2^{-\frac{136{,}8\,\text{суток}}{45{,}6\,\text{суток}}}
        \approx 0{,}1250 = 12{,}50\%
    \end{align*}
}
\solutionspace{150pt}

\tasknumber{11}%
\task{%
    За $2\,\text{суток}$ от начального количества ядер радиоизотопа осталась половина.
    Каков период полураспада этого изотопа (ответ приведите в сутках)?
    Какая ещё доля (также от начального количества) распадётся, если подождать ещё столько же?
}
\answer{%
    \begin{align*}
            N &= N_0 \cdot 2^{-\frac t{T_{1/2}}}
            \implies \frac N{N_0} = 2^{-\frac t{T_{1/2}}}
            \implies \frac 1{2} = 2^{-\frac {2\,\text{суток}}{T_{1/2}}}
            \implies 1 = \frac {2\,\text{суток}}{T_{1/2}}
            \implies T_{1/2} = \frac {2\,\text{суток}}1 \approx 2\,\text{суток}.
         \\
            \delta &= \frac{N(t)}{N_0} - \frac{N(2t)}{N_0}
            = 2^{-\frac t{T_{1/2}}} - 2^{-\frac {2t}{T_{1/2}}}
            = 2^{-\frac t{T_{1/2}}}\cbr{1 - 2^{-\frac t{T_{1/2}}}}
            = \frac 1{2} \cdot \cbr{1-\frac 1{2}} \approx 0{,}250
    \end{align*}
}
\solutionspace{150pt}

\tasknumber{12}%
\task{%
    Энергия связи ядра бора \ce{^{11}_{5}B} равна $76{,}2\,\text{МэВ}$.
    Найти дефект массы этого ядра.
    Ответ выразите в а.е.м.
    и кг.
    Скорость света $c = 2{,}998 \cdot 10^{8}\,\frac{\text{м}}{\text{с}}$, элементарный заряд $e = 1{,}6 \cdot 10^{-19}\,\text{Кл}$.
}
\answer{%
    \begin{align*}
    E_\text{св.} &= \Delta m c^2 \implies \\
    \implies
            \Delta m &= \frac {E_\text{св.}}{c^2} = \frac{76{,}2\,\text{МэВ}}{\sqr{2{,}998 \cdot 10^{8}\,\frac{\text{м}}{\text{с}}}}
            = \frac{76{,}2 \cdot 10^6 \cdot 1{,}6 \cdot 10^{-19}\,\text{Дж}}{\sqr{2{,}998 \cdot 10^{8}\,\frac{\text{м}}{\text{с}}}}
            \approx 0{,}1356 \cdot 10^{-27}\,\text{кг} \approx 0{,}0817\,\text{а.е.м.}
    \end{align*}
}

\variantsplitter

\addpersonalvariant{Ирина Лин}

\tasknumber{1}%
\task{%
    Для частицы, движущейся с релятивистской скоростью,
    выразите $p$ и $v$ через $c$, $E_\text{кин}$ и $E_0$,
    где $E_\text{кин}$~--- кинетическая энергия частицы,
    а $E_0$, $p$ и $v$~--- её энергия покоя импульс и скорость.
}
\answer{%
    \begin{align*}
    E_\text{кин}, E_0:\quad&E = E_\text{кин} + E_0 = \frac{E_0}{\sqrt{1 - \frac{v^2}{c^2}}} \implies \sqrt{1 - \frac{v^2}{c^2}} = \frac{E_0}{{E_0} + {E_\text{кин}}} \implies v = c\sqrt{1 - \sqr{\frac{E_0}{{E_0} + {E_\text{кин}}}}} \\
    &p = \frac{mv}{\sqrt{1 - \frac{v^2}{c^2}}} = \frac{E_0}{c^2} \cdot \sqrt{1 - \sqr{\frac{E_0}{{E_0} + {E_\text{кин}}}}} \cdot \frac{{E_\text{кин}} + {E_0}}{E_0} = \frac{E_0}{c^2} \cdot \sqrt{\sqr{\frac{{E_\text{кин}} + {E_0}}{E_0}} - 1}.
    \\
    E_\text{кин}, p:\quad&E_\text{кин} = E - E_0 = mc^2\cbr{\frac 1{\sqrt{1 - \frac{v^2}{c^2}}} - 1}, p = \frac{mv}{\sqrt{1 - \frac{v^2}{c^2}}} \implies \frac{E_\text{кин}}{p} = \frac{\frac 1{\sqrt{1 - \frac{v^2}{c^2}}} - 1}{\sqrt{1 - \frac{v^2}{c^2}}} \implies v = \ldots \\
    &E_0 = E - E_\text{кин} = \frac{E_0}{\sqrt{1 - \frac{v^2}{c^2}}} - E_\text{кин} \implies E_0 = \frac{E_\text{кин}}{\frac 1{\sqrt{1 - \frac{v^2}{c^2}}} - 1} = \ldots \\
    E_\text{кин}, v:\quad&E_\text{кин} = E - E_0 = mc^2\cbr{\frac 1{\sqrt{1 - \frac{v^2}{c^2}}} - 1} \implies m = \frac{E_\text{кин}}{c^2\cbr{\frac 1{\sqrt{1 - \frac{v^2}{c^2}}} - 1}} \\
    &E_0 = mc^2 = \frac{E_\text{кин}}{\frac 1{\sqrt{1 - \frac{v^2}{c^2}}} - 1} \\
    &p = \frac{mv}{\sqrt{1 - \frac{v^2}{c^2}}} = \frac{E_\text{кин}}{c^2\cbr{\frac 1{\sqrt{1 - \frac{v^2}{c^2}}} - 1}} \cdot \frac{v}{\sqrt{1 - \frac{v^2}{c^2}}} = \frac{{E_\text{кин}} v}{c^2\cbr{1 - {\sqrt{1 - \frac{v^2}{c^2}}}}} \\
    E_0, p:\quad&E_0 = mc^2, \quad p = \frac{mv}{\sqrt{1 - \frac{v^2}{c^2}}} \implies \frac{E_0}{p} = \frac{c^2}v{\sqrt{1 - \frac{v^2}{c^2}}} = c\sqrt{\frac{c^2}{v^2} - 1} \\
    &\sqr{\frac{E_0}{pc}} = \frac{c^2}{v^2} - 1 \implies \frac{v^2}{c^2} = \frac 1{1 + \frac{E_0^2}{p^2c^2}} \implies v = \frac c{\sqrt{1 + \frac{E_0^2}{p^2c^2}}} \\
    &{E_\text{кин}} = E - E_0 = \sqrt{E_0^2 + p^2c^2} - E_0 \\
    E_0, v:\quad&E_0 = mc^2 \implies m = \frac{E_0}{c^2} \qquad p = \frac{mv}{\sqrt{1 - \frac{v^2}{c^2}}} = \frac{E_0}{c^2} \cdot \frac{v}{\sqrt{1 - \frac{v^2}{c^2}}} \\
    &E_\text{кин}= mc^2\cbr{\frac 1{\sqrt{1 - \frac{v^2}{c^2}}} - 1} = \frac{E_0}{c^2}\cbr{\frac 1{\sqrt{1 - \frac{v^2}{c^2}}} - 1} \\
    p, v:\quad&p = \frac{mv}{\sqrt{1 - \frac{v^2}{c^2}}} \implies m = \frac p v {\sqrt{1 - \frac{v^2}{c^2}}} \implies E_0 = mc^2 =\frac {pc^2} v {\sqrt{1 - \frac{v^2}{c^2}}} \\
    &E_\text{кин} = mc^2\cbr{\frac 1{\sqrt{1 - \frac{v^2}{c^2}}} - 1} = \frac p v {\sqrt{1 - \frac{v^2}{c^2}}}\cbr{\frac 1{\sqrt{1 - \frac{v^2}{c^2}}} - 1} = \frac p v \cbr{1 - {\sqrt{1 - \frac{v^2}{c^2}}}}
    \end{align*}
}
\solutionspace{200pt}

\tasknumber{2}%
\task{%
    Протон движется со скоростью $0{,}7\,c$, где $c$~--- скорость света в вакууме.
    Каково при этом отношение кинетической энергии частицы $E_\text{кин.}$ к его энергии покоя $E_0$?
}
\answer{%
    \begin{align*}
    E &= \frac{E_0}{\sqrt{1 - \frac{v^2}{c^2}}}
            \implies \frac E{E_0}
                = \frac 1{\sqrt{1 - \frac{v^2}{c^2}}}
                = \frac 1{\sqrt{1 - \sqr{0{,}7}}}
                \approx 1{,}400,
         \\
        E_{\text{кин}} &= E - E_0
            \implies \frac{E_{\text{кин}}}{E_0}
                = \frac E{E_0} - 1
                = \frac 1{\sqrt{1 - \frac{v^2}{c^2}}} - 1
                = \frac 1{\sqrt{1 - \sqr{0{,}7}}} - 1
                \approx 0{,}400.
    \end{align*}
}
\solutionspace{150pt}

\tasknumber{3}%
\task{%
    Протон движется со скоростью $0{,}65\,c$, где $c$~--- скорость света в вакууме.
    Определите его кинетическую энергию (в ответе приведите формулу и укажите численное значение).
}
\answer{%
    \begin{align*}
    E &= \frac{mc^2}{\sqrt{1 - \frac{v^2}{c^2}}}
            \approx \frac{1{,}673 \cdot 10^{-27}\,\text{кг} \cdot \sqr{3 \cdot 10^{8}\,\frac{\text{м}}{\text{с}}}}{\sqrt{1 - 0{,}65^2}}
            \approx 198{,}091 \cdot 10^{-12}\,\text{Дж},
         \\
        E_{\text{кин}} &= \frac{mc^2}{\sqrt{1 - \frac{v^2}{c^2}}} - mc^2
            = mc^2 \cbr{\frac 1{\sqrt{1 - \frac{v^2}{c^2}}} - 1} \approx \\
            &\approx \cbr{1{,}673 \cdot 10^{-27}\,\text{кг} \cdot \sqr{3 \cdot 10^{8}\,\frac{\text{м}}{\text{с}}}}
            \cdot \cbr{\frac 1{\sqrt{1 - 0{,}65^2}} - 1}
            \approx 47{,}555 \cdot 10^{-12}\,\text{Дж},
         \\
        p &= \frac{mv}{\sqrt{1 - \frac{v^2}{c^2}}}
            \approx \frac{1{,}673 \cdot 10^{-27}\,\text{кг} \cdot 0{,}65 \cdot 3 \cdot 10^{8}\,\frac{\text{м}}{\text{с}}}{\sqrt{1 - 0{,}65^2}}
            \approx 429{,}196 \cdot 10^{-21}\,\frac{\text{кг}\cdot\text{м}}{\text{с}}.
    \end{align*}
}
\solutionspace{150pt}

\tasknumber{4}%
\task{%
    При какой скорости движения (в м/с) релятивистское сокращение длины движущегося тела
    составит 50\%?
}
\answer{%
    \begin{align*}
    l_0 &= \frac l{\sqrt{1 - \frac{v^2}{c^2}}}
        \implies 1 - \frac{v^2}{c^2} = \sqr{\frac l{l_0}}
        \implies \frac v c = \sqrt{1 - \sqr{\frac l{l_0}}} \implies
         \\
        \implies v &= c\sqrt{1 - \sqr{\frac l{l_0}}}
        = 3 \cdot 10^{8}\,\frac{\text{м}}{\text{с}} \cdot \sqrt{1 - \sqr{\frac {l_0 - 0{,}50l_0}{l_0}}}
        = 3 \cdot 10^{8}\,\frac{\text{м}}{\text{с}} \cdot \sqrt{1 - \sqr{1 - 0{,}50}} \approx  \\
        &\approx 0{,}866c
        \approx 260 \cdot 10^{6}\,\frac{\text{м}}{\text{с}}
        \approx 935 \cdot 10^{6}\,\frac{\text{км}}{\text{ч}}.
    \end{align*}
}
\solutionspace{150pt}

\tasknumber{5}%
\task{%
    При переходе электрона в атоме с одной стационарной орбиты на другую
    излучается фотон с энергией $5{,}05 \cdot 10^{-19}\,\text{Дж}$.
    Какова длина волны этой линии спектра?
    Постоянная Планка $h = 6{,}626 \cdot 10^{-34}\,\text{Дж}\cdot\text{с}$, скорость света $c = 3 \cdot 10^{8}\,\frac{\text{м}}{\text{с}}$.
}
\answer{%
    $
        E = h\nu = h \frac c\lambda
        \implies \lambda = \frac{hc}E
            = \frac{6{,}626 \cdot 10^{-34}\,\text{Дж}\cdot\text{с} \cdot {3 \cdot 10^{8}\,\frac{\text{м}}{\text{с}}}}{5{,}05 \cdot 10^{-19}\,\text{Дж}}
            = 393{,}62\,\text{нм}.
    $
}
\solutionspace{150pt}

\tasknumber{6}%
\task{%
    Излучение какой длины волны поглотил атом водорода, если полная энергия в атоме увеличилась на $3 \cdot 10^{-19}\,\text{Дж}$?
    Постоянная Планка $h = 6{,}626 \cdot 10^{-34}\,\text{Дж}\cdot\text{с}$, скорость света $c = 3 \cdot 10^{8}\,\frac{\text{м}}{\text{с}}$.
}
\answer{%
    $
        E = h\nu = h \frac c\lambda
        \implies \lambda = \frac{hc}E
            = \frac{6{,}626 \cdot 10^{-34}\,\text{Дж}\cdot\text{с} \cdot {3 \cdot 10^{8}\,\frac{\text{м}}{\text{с}}}}{3 \cdot 10^{-19}\,\text{Дж}}
            = 663\,\text{нм}.
    $
}
\solutionspace{150pt}

\tasknumber{7}%
\task{%
    Сделайте схематичный рисунок энергетических уровней атома водорода
    и отметьте на нём первый (основной) уровень и последующие.
    Сколько различных длин волн может испустить атом водорода,
    находящийся в 3-м возбуждённом состоянии?
    Отметьте все соответствующие переходы на рисунке и укажите,
    при каком переходе (среди отмеченных) частота излучённого фотона минимальна.
}
\answer{%
    $N = 3{,}0, \text{самая короткая линия}$
}
\solutionspace{150pt}

\tasknumber{8}%
\task{%
    Сколько фотонов испускает за $30\,\text{мин}$ лазер,
    если мощность его излучения $200\,\text{мВт}$?
    Длина волны излучения $600\,\text{нм}$.
    $h = 6{,}626 \cdot 10^{-34}\,\text{Дж}\cdot\text{с}$.
}
\answer{%
    $
        N
            = \frac{E_{\text{общая}}}{E_{\text{одного фотона}}}
            = \frac{Pt}{h\nu} = \frac{Pt}{h \frac c\lambda}
            = \frac{Pt\lambda}{hc}
            = \frac{200\,\text{мВт} \cdot 30\,\text{мин} \cdot 600\,\text{нм}}{6{,}626 \cdot 10^{-34}\,\text{Дж}\cdot\text{с} \cdot 3 \cdot 10^{8}\,\frac{\text{м}}{\text{с}}}
            \approx 10{,}87 \cdot 10^{20}\units{фотонов}
    $
}
\solutionspace{120pt}

\tasknumber{9}%
\task{%
    Какая доля (от начального количества) радиоактивных ядер распадётся через время,
    равное трём периодам полураспада? Ответ выразить в процентах.
}
\answer{%
    \begin{align*}
    N &= N_0 \cdot 2^{- \frac t{T_{1/2}}} \implies
        \frac N{N_0} = 2^{- \frac t{T_{1/2}}}
        = 2^{-3} \approx 0{,}12 \approx 12\% \\
    N_\text{расп.} &= N_0 - N = N_0 - N_0 \cdot 2^{-\frac t{T_{1/2}}}
        = N_0\cbr{1 - 2^{-\frac t{T_{1/2}}}} \implies
        \frac{N_\text{расп.}}{N_0} = 1 - 2^{-\frac t{T_{1/2}}}
        = 1 - 2^{-3} \approx 0{,}88 \approx 88\%
    \end{align*}
}
\solutionspace{150pt}

\tasknumber{10}%
\task{%
    Сколько процентов ядер радиоактивного железа $\ce{^{59}Fe}$
    останется через $136{,}8\,\text{суток}$, если период его полураспада составляет $45{,}6\,\text{суток}$?
}
\answer{%
    \begin{align*}
    N &= N_0 \cdot 2^{-\frac t{T_{1/2}}}
        = 2^{-\frac{136{,}8\,\text{суток}}{45{,}6\,\text{суток}}}
        \approx 0{,}1250 = 12{,}50\%
    \end{align*}
}
\solutionspace{150pt}

\tasknumber{11}%
\task{%
    За $3\,\text{суток}$ от начального количества ядер радиоизотопа осталась одна шестнадцатая.
    Каков период полураспада этого изотопа (ответ приведите в сутках)?
    Какая ещё доля (также от начального количества) распадётся, если подождать ещё столько же?
}
\answer{%
    \begin{align*}
            N &= N_0 \cdot 2^{-\frac t{T_{1/2}}}
            \implies \frac N{N_0} = 2^{-\frac t{T_{1/2}}}
            \implies \frac 1{16} = 2^{-\frac {3\,\text{суток}}{T_{1/2}}}
            \implies 4 = \frac {3\,\text{суток}}{T_{1/2}}
            \implies T_{1/2} = \frac {3\,\text{суток}}4 \approx 0{,}75\,\text{суток}.
         \\
            \delta &= \frac{N(t)}{N_0} - \frac{N(2t)}{N_0}
            = 2^{-\frac t{T_{1/2}}} - 2^{-\frac {2t}{T_{1/2}}}
            = 2^{-\frac t{T_{1/2}}}\cbr{1 - 2^{-\frac t{T_{1/2}}}}
            = \frac 1{16} \cdot \cbr{1-\frac 1{16}} \approx 0{,}059
    \end{align*}
}
\solutionspace{150pt}

\tasknumber{12}%
\task{%
    Энергия связи ядра углерода \ce{^{12}_{6}C} равна $92{,}2\,\text{МэВ}$.
    Найти дефект массы этого ядра.
    Ответ выразите в а.е.м.
    и кг.
    Скорость света $c = 2{,}998 \cdot 10^{8}\,\frac{\text{м}}{\text{с}}$, элементарный заряд $e = 1{,}6 \cdot 10^{-19}\,\text{Кл}$.
}
\answer{%
    \begin{align*}
    E_\text{св.} &= \Delta m c^2 \implies \\
    \implies
            \Delta m &= \frac {E_\text{св.}}{c^2} = \frac{92{,}2\,\text{МэВ}}{\sqr{2{,}998 \cdot 10^{8}\,\frac{\text{м}}{\text{с}}}}
            = \frac{92{,}2 \cdot 10^6 \cdot 1{,}6 \cdot 10^{-19}\,\text{Дж}}{\sqr{2{,}998 \cdot 10^{8}\,\frac{\text{м}}{\text{с}}}}
            \approx 0{,}1641 \cdot 10^{-27}\,\text{кг} \approx 0{,}0988\,\text{а.е.м.}
    \end{align*}
}

\variantsplitter

\addpersonalvariant{Ислам Мунаев}

\tasknumber{1}%
\task{%
    Для частицы, движущейся с релятивистской скоростью,
    выразите $E_0$ и $v$ через $c$, $E_\text{кин}$ и $p$,
    где $E_\text{кин}$~--- кинетическая энергия частицы,
    а $E_0$, $p$ и $v$~--- её энергия покоя импульс и скорость.
}
\answer{%
    \begin{align*}
    E_\text{кин}, E_0:\quad&E = E_\text{кин} + E_0 = \frac{E_0}{\sqrt{1 - \frac{v^2}{c^2}}} \implies \sqrt{1 - \frac{v^2}{c^2}} = \frac{E_0}{{E_0} + {E_\text{кин}}} \implies v = c\sqrt{1 - \sqr{\frac{E_0}{{E_0} + {E_\text{кин}}}}} \\
    &p = \frac{mv}{\sqrt{1 - \frac{v^2}{c^2}}} = \frac{E_0}{c^2} \cdot \sqrt{1 - \sqr{\frac{E_0}{{E_0} + {E_\text{кин}}}}} \cdot \frac{{E_\text{кин}} + {E_0}}{E_0} = \frac{E_0}{c^2} \cdot \sqrt{\sqr{\frac{{E_\text{кин}} + {E_0}}{E_0}} - 1}.
    \\
    E_\text{кин}, p:\quad&E_\text{кин} = E - E_0 = mc^2\cbr{\frac 1{\sqrt{1 - \frac{v^2}{c^2}}} - 1}, p = \frac{mv}{\sqrt{1 - \frac{v^2}{c^2}}} \implies \frac{E_\text{кин}}{p} = \frac{\frac 1{\sqrt{1 - \frac{v^2}{c^2}}} - 1}{\sqrt{1 - \frac{v^2}{c^2}}} \implies v = \ldots \\
    &E_0 = E - E_\text{кин} = \frac{E_0}{\sqrt{1 - \frac{v^2}{c^2}}} - E_\text{кин} \implies E_0 = \frac{E_\text{кин}}{\frac 1{\sqrt{1 - \frac{v^2}{c^2}}} - 1} = \ldots \\
    E_\text{кин}, v:\quad&E_\text{кин} = E - E_0 = mc^2\cbr{\frac 1{\sqrt{1 - \frac{v^2}{c^2}}} - 1} \implies m = \frac{E_\text{кин}}{c^2\cbr{\frac 1{\sqrt{1 - \frac{v^2}{c^2}}} - 1}} \\
    &E_0 = mc^2 = \frac{E_\text{кин}}{\frac 1{\sqrt{1 - \frac{v^2}{c^2}}} - 1} \\
    &p = \frac{mv}{\sqrt{1 - \frac{v^2}{c^2}}} = \frac{E_\text{кин}}{c^2\cbr{\frac 1{\sqrt{1 - \frac{v^2}{c^2}}} - 1}} \cdot \frac{v}{\sqrt{1 - \frac{v^2}{c^2}}} = \frac{{E_\text{кин}} v}{c^2\cbr{1 - {\sqrt{1 - \frac{v^2}{c^2}}}}} \\
    E_0, p:\quad&E_0 = mc^2, \quad p = \frac{mv}{\sqrt{1 - \frac{v^2}{c^2}}} \implies \frac{E_0}{p} = \frac{c^2}v{\sqrt{1 - \frac{v^2}{c^2}}} = c\sqrt{\frac{c^2}{v^2} - 1} \\
    &\sqr{\frac{E_0}{pc}} = \frac{c^2}{v^2} - 1 \implies \frac{v^2}{c^2} = \frac 1{1 + \frac{E_0^2}{p^2c^2}} \implies v = \frac c{\sqrt{1 + \frac{E_0^2}{p^2c^2}}} \\
    &{E_\text{кин}} = E - E_0 = \sqrt{E_0^2 + p^2c^2} - E_0 \\
    E_0, v:\quad&E_0 = mc^2 \implies m = \frac{E_0}{c^2} \qquad p = \frac{mv}{\sqrt{1 - \frac{v^2}{c^2}}} = \frac{E_0}{c^2} \cdot \frac{v}{\sqrt{1 - \frac{v^2}{c^2}}} \\
    &E_\text{кин}= mc^2\cbr{\frac 1{\sqrt{1 - \frac{v^2}{c^2}}} - 1} = \frac{E_0}{c^2}\cbr{\frac 1{\sqrt{1 - \frac{v^2}{c^2}}} - 1} \\
    p, v:\quad&p = \frac{mv}{\sqrt{1 - \frac{v^2}{c^2}}} \implies m = \frac p v {\sqrt{1 - \frac{v^2}{c^2}}} \implies E_0 = mc^2 =\frac {pc^2} v {\sqrt{1 - \frac{v^2}{c^2}}} \\
    &E_\text{кин} = mc^2\cbr{\frac 1{\sqrt{1 - \frac{v^2}{c^2}}} - 1} = \frac p v {\sqrt{1 - \frac{v^2}{c^2}}}\cbr{\frac 1{\sqrt{1 - \frac{v^2}{c^2}}} - 1} = \frac p v \cbr{1 - {\sqrt{1 - \frac{v^2}{c^2}}}}
    \end{align*}
}
\solutionspace{200pt}

\tasknumber{2}%
\task{%
    Позитрон движется со скоростью $0{,}6\,c$, где $c$~--- скорость света в вакууме.
    Каково при этом отношение полной энергии частицы $E$ к его энергии покоя $E_0$?
}
\answer{%
    \begin{align*}
    E &= \frac{E_0}{\sqrt{1 - \frac{v^2}{c^2}}}
            \implies \frac E{E_0}
                = \frac 1{\sqrt{1 - \frac{v^2}{c^2}}}
                = \frac 1{\sqrt{1 - \sqr{0{,}6}}}
                \approx 1{,}250,
         \\
        E_{\text{кин}} &= E - E_0
            \implies \frac{E_{\text{кин}}}{E_0}
                = \frac E{E_0} - 1
                = \frac 1{\sqrt{1 - \frac{v^2}{c^2}}} - 1
                = \frac 1{\sqrt{1 - \sqr{0{,}6}}} - 1
                \approx 0{,}250.
    \end{align*}
}
\solutionspace{150pt}

\tasknumber{3}%
\task{%
    Протон движется со скоростью $0{,}65\,c$, где $c$~--- скорость света в вакууме.
    Определите его кинетическую энергию (в ответе приведите формулу и укажите численное значение).
}
\answer{%
    \begin{align*}
    E &= \frac{mc^2}{\sqrt{1 - \frac{v^2}{c^2}}}
            \approx \frac{1{,}673 \cdot 10^{-27}\,\text{кг} \cdot \sqr{3 \cdot 10^{8}\,\frac{\text{м}}{\text{с}}}}{\sqrt{1 - 0{,}65^2}}
            \approx 198{,}091 \cdot 10^{-12}\,\text{Дж},
         \\
        E_{\text{кин}} &= \frac{mc^2}{\sqrt{1 - \frac{v^2}{c^2}}} - mc^2
            = mc^2 \cbr{\frac 1{\sqrt{1 - \frac{v^2}{c^2}}} - 1} \approx \\
            &\approx \cbr{1{,}673 \cdot 10^{-27}\,\text{кг} \cdot \sqr{3 \cdot 10^{8}\,\frac{\text{м}}{\text{с}}}}
            \cdot \cbr{\frac 1{\sqrt{1 - 0{,}65^2}} - 1}
            \approx 47{,}555 \cdot 10^{-12}\,\text{Дж},
         \\
        p &= \frac{mv}{\sqrt{1 - \frac{v^2}{c^2}}}
            \approx \frac{1{,}673 \cdot 10^{-27}\,\text{кг} \cdot 0{,}65 \cdot 3 \cdot 10^{8}\,\frac{\text{м}}{\text{с}}}{\sqrt{1 - 0{,}65^2}}
            \approx 429{,}196 \cdot 10^{-21}\,\frac{\text{кг}\cdot\text{м}}{\text{с}}.
    \end{align*}
}
\solutionspace{150pt}

\tasknumber{4}%
\task{%
    При какой скорости движения (в м/с) релятивистское сокращение длины движущегося тела
    составит 50\%?
}
\answer{%
    \begin{align*}
    l_0 &= \frac l{\sqrt{1 - \frac{v^2}{c^2}}}
        \implies 1 - \frac{v^2}{c^2} = \sqr{\frac l{l_0}}
        \implies \frac v c = \sqrt{1 - \sqr{\frac l{l_0}}} \implies
         \\
        \implies v &= c\sqrt{1 - \sqr{\frac l{l_0}}}
        = 3 \cdot 10^{8}\,\frac{\text{м}}{\text{с}} \cdot \sqrt{1 - \sqr{\frac {l_0 - 0{,}50l_0}{l_0}}}
        = 3 \cdot 10^{8}\,\frac{\text{м}}{\text{с}} \cdot \sqrt{1 - \sqr{1 - 0{,}50}} \approx  \\
        &\approx 0{,}866c
        \approx 260 \cdot 10^{6}\,\frac{\text{м}}{\text{с}}
        \approx 935 \cdot 10^{6}\,\frac{\text{км}}{\text{ч}}.
    \end{align*}
}
\solutionspace{150pt}

\tasknumber{5}%
\task{%
    При переходе электрона в атоме с одной стационарной орбиты на другую
    излучается фотон с энергией $0{,}55 \cdot 10^{-19}\,\text{Дж}$.
    Какова длина волны этой линии спектра?
    Постоянная Планка $h = 6{,}626 \cdot 10^{-34}\,\text{Дж}\cdot\text{с}$, скорость света $c = 3 \cdot 10^{8}\,\frac{\text{м}}{\text{с}}$.
}
\answer{%
    $
        E = h\nu = h \frac c\lambda
        \implies \lambda = \frac{hc}E
            = \frac{6{,}626 \cdot 10^{-34}\,\text{Дж}\cdot\text{с} \cdot {3 \cdot 10^{8}\,\frac{\text{м}}{\text{с}}}}{0{,}55 \cdot 10^{-19}\,\text{Дж}}
            = 3614\,\text{нм}.
    $
}
\solutionspace{150pt}

\tasknumber{6}%
\task{%
    Излучение какой длины волны поглотил атом водорода, если полная энергия в атоме увеличилась на $6 \cdot 10^{-19}\,\text{Дж}$?
    Постоянная Планка $h = 6{,}626 \cdot 10^{-34}\,\text{Дж}\cdot\text{с}$, скорость света $c = 3 \cdot 10^{8}\,\frac{\text{м}}{\text{с}}$.
}
\answer{%
    $
        E = h\nu = h \frac c\lambda
        \implies \lambda = \frac{hc}E
            = \frac{6{,}626 \cdot 10^{-34}\,\text{Дж}\cdot\text{с} \cdot {3 \cdot 10^{8}\,\frac{\text{м}}{\text{с}}}}{6 \cdot 10^{-19}\,\text{Дж}}
            = 331\,\text{нм}.
    $
}
\solutionspace{150pt}

\tasknumber{7}%
\task{%
    Сделайте схематичный рисунок энергетических уровней атома водорода
    и отметьте на нём первый (основной) уровень и последующие.
    Сколько различных длин волн может испустить атом водорода,
    находящийся в 3-м возбуждённом состоянии?
    Отметьте все соответствующие переходы на рисунке и укажите,
    при каком переходе (среди отмеченных) энергия излучённого фотона минимальна.
}
\answer{%
    $N = 3{,}0, \text{самая короткая линия}$
}
\solutionspace{150pt}

\tasknumber{8}%
\task{%
    Сколько фотонов испускает за $40\,\text{мин}$ лазер,
    если мощность его излучения $75\,\text{мВт}$?
    Длина волны излучения $500\,\text{нм}$.
    $h = 6{,}626 \cdot 10^{-34}\,\text{Дж}\cdot\text{с}$.
}
\answer{%
    $
        N
            = \frac{E_{\text{общая}}}{E_{\text{одного фотона}}}
            = \frac{Pt}{h\nu} = \frac{Pt}{h \frac c\lambda}
            = \frac{Pt\lambda}{hc}
            = \frac{75\,\text{мВт} \cdot 40\,\text{мин} \cdot 500\,\text{нм}}{6{,}626 \cdot 10^{-34}\,\text{Дж}\cdot\text{с} \cdot 3 \cdot 10^{8}\,\frac{\text{м}}{\text{с}}}
            \approx 4{,}53 \cdot 10^{20}\units{фотонов}
    $
}
\solutionspace{120pt}

\tasknumber{9}%
\task{%
    Какая доля (от начального количества) радиоактивных ядер распадётся через время,
    равное трём периодам полураспада? Ответ выразить в процентах.
}
\answer{%
    \begin{align*}
    N &= N_0 \cdot 2^{- \frac t{T_{1/2}}} \implies
        \frac N{N_0} = 2^{- \frac t{T_{1/2}}}
        = 2^{-3} \approx 0{,}12 \approx 12\% \\
    N_\text{расп.} &= N_0 - N = N_0 - N_0 \cdot 2^{-\frac t{T_{1/2}}}
        = N_0\cbr{1 - 2^{-\frac t{T_{1/2}}}} \implies
        \frac{N_\text{расп.}}{N_0} = 1 - 2^{-\frac t{T_{1/2}}}
        = 1 - 2^{-3} \approx 0{,}88 \approx 88\%
    \end{align*}
}
\solutionspace{150pt}

\tasknumber{10}%
\task{%
    Сколько процентов ядер радиоактивного железа $\ce{^{59}Fe}$
    останется через $136{,}8\,\text{суток}$, если период его полураспада составляет $45{,}6\,\text{суток}$?
}
\answer{%
    \begin{align*}
    N &= N_0 \cdot 2^{-\frac t{T_{1/2}}}
        = 2^{-\frac{136{,}8\,\text{суток}}{45{,}6\,\text{суток}}}
        \approx 0{,}1250 = 12{,}50\%
    \end{align*}
}
\solutionspace{150pt}

\tasknumber{11}%
\task{%
    За $5\,\text{суток}$ от начального количества ядер радиоизотопа осталась одна восьмая.
    Каков период полураспада этого изотопа (ответ приведите в сутках)?
    Какая ещё доля (также от начального количества) распадётся, если подождать ещё столько же?
}
\answer{%
    \begin{align*}
            N &= N_0 \cdot 2^{-\frac t{T_{1/2}}}
            \implies \frac N{N_0} = 2^{-\frac t{T_{1/2}}}
            \implies \frac 1{8} = 2^{-\frac {5\,\text{суток}}{T_{1/2}}}
            \implies 3 = \frac {5\,\text{суток}}{T_{1/2}}
            \implies T_{1/2} = \frac {5\,\text{суток}}3 \approx 1{,}67\,\text{суток}.
         \\
            \delta &= \frac{N(t)}{N_0} - \frac{N(2t)}{N_0}
            = 2^{-\frac t{T_{1/2}}} - 2^{-\frac {2t}{T_{1/2}}}
            = 2^{-\frac t{T_{1/2}}}\cbr{1 - 2^{-\frac t{T_{1/2}}}}
            = \frac 1{8} \cdot \cbr{1-\frac 1{8}} \approx 0{,}109
    \end{align*}
}
\solutionspace{150pt}

\tasknumber{12}%
\task{%
    Энергия связи ядра дейтерия \ce{^{2}_{1}H} (D) равна $2{,}22\,\text{МэВ}$.
    Найти дефект массы этого ядра.
    Ответ выразите в а.е.м.
    и кг.
    Скорость света $c = 2{,}998 \cdot 10^{8}\,\frac{\text{м}}{\text{с}}$, элементарный заряд $e = 1{,}6 \cdot 10^{-19}\,\text{Кл}$.
}
\answer{%
    \begin{align*}
    E_\text{св.} &= \Delta m c^2 \implies \\
    \implies
            \Delta m &= \frac {E_\text{св.}}{c^2} = \frac{2{,}22\,\text{МэВ}}{\sqr{2{,}998 \cdot 10^{8}\,\frac{\text{м}}{\text{с}}}}
            = \frac{2{,}22 \cdot 10^6 \cdot 1{,}6 \cdot 10^{-19}\,\text{Дж}}{\sqr{2{,}998 \cdot 10^{8}\,\frac{\text{м}}{\text{с}}}}
            \approx 3{,}95 \cdot 10^{-30}\,\text{кг} \approx 0{,}00238\,\text{а.е.м.}
    \end{align*}
}

\variantsplitter

\addpersonalvariant{Александр Наумов}

\tasknumber{1}%
\task{%
    Для частицы, движущейся с релятивистской скоростью,
    выразите $E_0$ и $p$ через $c$, $E_\text{кин}$ и $v$,
    где $E_\text{кин}$~--- кинетическая энергия частицы,
    а $E_0$, $p$ и $v$~--- её энергия покоя импульс и скорость.
}
\answer{%
    \begin{align*}
    E_\text{кин}, E_0:\quad&E = E_\text{кин} + E_0 = \frac{E_0}{\sqrt{1 - \frac{v^2}{c^2}}} \implies \sqrt{1 - \frac{v^2}{c^2}} = \frac{E_0}{{E_0} + {E_\text{кин}}} \implies v = c\sqrt{1 - \sqr{\frac{E_0}{{E_0} + {E_\text{кин}}}}} \\
    &p = \frac{mv}{\sqrt{1 - \frac{v^2}{c^2}}} = \frac{E_0}{c^2} \cdot \sqrt{1 - \sqr{\frac{E_0}{{E_0} + {E_\text{кин}}}}} \cdot \frac{{E_\text{кин}} + {E_0}}{E_0} = \frac{E_0}{c^2} \cdot \sqrt{\sqr{\frac{{E_\text{кин}} + {E_0}}{E_0}} - 1}.
    \\
    E_\text{кин}, p:\quad&E_\text{кин} = E - E_0 = mc^2\cbr{\frac 1{\sqrt{1 - \frac{v^2}{c^2}}} - 1}, p = \frac{mv}{\sqrt{1 - \frac{v^2}{c^2}}} \implies \frac{E_\text{кин}}{p} = \frac{\frac 1{\sqrt{1 - \frac{v^2}{c^2}}} - 1}{\sqrt{1 - \frac{v^2}{c^2}}} \implies v = \ldots \\
    &E_0 = E - E_\text{кин} = \frac{E_0}{\sqrt{1 - \frac{v^2}{c^2}}} - E_\text{кин} \implies E_0 = \frac{E_\text{кин}}{\frac 1{\sqrt{1 - \frac{v^2}{c^2}}} - 1} = \ldots \\
    E_\text{кин}, v:\quad&E_\text{кин} = E - E_0 = mc^2\cbr{\frac 1{\sqrt{1 - \frac{v^2}{c^2}}} - 1} \implies m = \frac{E_\text{кин}}{c^2\cbr{\frac 1{\sqrt{1 - \frac{v^2}{c^2}}} - 1}} \\
    &E_0 = mc^2 = \frac{E_\text{кин}}{\frac 1{\sqrt{1 - \frac{v^2}{c^2}}} - 1} \\
    &p = \frac{mv}{\sqrt{1 - \frac{v^2}{c^2}}} = \frac{E_\text{кин}}{c^2\cbr{\frac 1{\sqrt{1 - \frac{v^2}{c^2}}} - 1}} \cdot \frac{v}{\sqrt{1 - \frac{v^2}{c^2}}} = \frac{{E_\text{кин}} v}{c^2\cbr{1 - {\sqrt{1 - \frac{v^2}{c^2}}}}} \\
    E_0, p:\quad&E_0 = mc^2, \quad p = \frac{mv}{\sqrt{1 - \frac{v^2}{c^2}}} \implies \frac{E_0}{p} = \frac{c^2}v{\sqrt{1 - \frac{v^2}{c^2}}} = c\sqrt{\frac{c^2}{v^2} - 1} \\
    &\sqr{\frac{E_0}{pc}} = \frac{c^2}{v^2} - 1 \implies \frac{v^2}{c^2} = \frac 1{1 + \frac{E_0^2}{p^2c^2}} \implies v = \frac c{\sqrt{1 + \frac{E_0^2}{p^2c^2}}} \\
    &{E_\text{кин}} = E - E_0 = \sqrt{E_0^2 + p^2c^2} - E_0 \\
    E_0, v:\quad&E_0 = mc^2 \implies m = \frac{E_0}{c^2} \qquad p = \frac{mv}{\sqrt{1 - \frac{v^2}{c^2}}} = \frac{E_0}{c^2} \cdot \frac{v}{\sqrt{1 - \frac{v^2}{c^2}}} \\
    &E_\text{кин}= mc^2\cbr{\frac 1{\sqrt{1 - \frac{v^2}{c^2}}} - 1} = \frac{E_0}{c^2}\cbr{\frac 1{\sqrt{1 - \frac{v^2}{c^2}}} - 1} \\
    p, v:\quad&p = \frac{mv}{\sqrt{1 - \frac{v^2}{c^2}}} \implies m = \frac p v {\sqrt{1 - \frac{v^2}{c^2}}} \implies E_0 = mc^2 =\frac {pc^2} v {\sqrt{1 - \frac{v^2}{c^2}}} \\
    &E_\text{кин} = mc^2\cbr{\frac 1{\sqrt{1 - \frac{v^2}{c^2}}} - 1} = \frac p v {\sqrt{1 - \frac{v^2}{c^2}}}\cbr{\frac 1{\sqrt{1 - \frac{v^2}{c^2}}} - 1} = \frac p v \cbr{1 - {\sqrt{1 - \frac{v^2}{c^2}}}}
    \end{align*}
}
\solutionspace{200pt}

\tasknumber{2}%
\task{%
    Позитрон движется со скоростью $0{,}8\,c$, где $c$~--- скорость света в вакууме.
    Каково при этом отношение кинетической энергии частицы $E_\text{кин.}$ к его энергии покоя $E_0$?
}
\answer{%
    \begin{align*}
    E &= \frac{E_0}{\sqrt{1 - \frac{v^2}{c^2}}}
            \implies \frac E{E_0}
                = \frac 1{\sqrt{1 - \frac{v^2}{c^2}}}
                = \frac 1{\sqrt{1 - \sqr{0{,}8}}}
                \approx 1{,}667,
         \\
        E_{\text{кин}} &= E - E_0
            \implies \frac{E_{\text{кин}}}{E_0}
                = \frac E{E_0} - 1
                = \frac 1{\sqrt{1 - \frac{v^2}{c^2}}} - 1
                = \frac 1{\sqrt{1 - \sqr{0{,}8}}} - 1
                \approx 0{,}667.
    \end{align*}
}
\solutionspace{150pt}

\tasknumber{3}%
\task{%
    Протон движется со скоростью $0{,}65\,c$, где $c$~--- скорость света в вакууме.
    Определите его кинетическую энергию (в ответе приведите формулу и укажите численное значение).
}
\answer{%
    \begin{align*}
    E &= \frac{mc^2}{\sqrt{1 - \frac{v^2}{c^2}}}
            \approx \frac{1{,}673 \cdot 10^{-27}\,\text{кг} \cdot \sqr{3 \cdot 10^{8}\,\frac{\text{м}}{\text{с}}}}{\sqrt{1 - 0{,}65^2}}
            \approx 198{,}091 \cdot 10^{-12}\,\text{Дж},
         \\
        E_{\text{кин}} &= \frac{mc^2}{\sqrt{1 - \frac{v^2}{c^2}}} - mc^2
            = mc^2 \cbr{\frac 1{\sqrt{1 - \frac{v^2}{c^2}}} - 1} \approx \\
            &\approx \cbr{1{,}673 \cdot 10^{-27}\,\text{кг} \cdot \sqr{3 \cdot 10^{8}\,\frac{\text{м}}{\text{с}}}}
            \cdot \cbr{\frac 1{\sqrt{1 - 0{,}65^2}} - 1}
            \approx 47{,}555 \cdot 10^{-12}\,\text{Дж},
         \\
        p &= \frac{mv}{\sqrt{1 - \frac{v^2}{c^2}}}
            \approx \frac{1{,}673 \cdot 10^{-27}\,\text{кг} \cdot 0{,}65 \cdot 3 \cdot 10^{8}\,\frac{\text{м}}{\text{с}}}{\sqrt{1 - 0{,}65^2}}
            \approx 429{,}196 \cdot 10^{-21}\,\frac{\text{кг}\cdot\text{м}}{\text{с}}.
    \end{align*}
}
\solutionspace{150pt}

\tasknumber{4}%
\task{%
    При какой скорости движения (в м/с) релятивистское сокращение длины движущегося тела
    составит 30\%?
}
\answer{%
    \begin{align*}
    l_0 &= \frac l{\sqrt{1 - \frac{v^2}{c^2}}}
        \implies 1 - \frac{v^2}{c^2} = \sqr{\frac l{l_0}}
        \implies \frac v c = \sqrt{1 - \sqr{\frac l{l_0}}} \implies
         \\
        \implies v &= c\sqrt{1 - \sqr{\frac l{l_0}}}
        = 3 \cdot 10^{8}\,\frac{\text{м}}{\text{с}} \cdot \sqrt{1 - \sqr{\frac {l_0 - 0{,}30l_0}{l_0}}}
        = 3 \cdot 10^{8}\,\frac{\text{м}}{\text{с}} \cdot \sqrt{1 - \sqr{1 - 0{,}30}} \approx  \\
        &\approx 0{,}714c
        \approx 214 \cdot 10^{6}\,\frac{\text{м}}{\text{с}}
        \approx 771 \cdot 10^{6}\,\frac{\text{км}}{\text{ч}}.
    \end{align*}
}
\solutionspace{150pt}

\tasknumber{5}%
\task{%
    При переходе электрона в атоме с одной стационарной орбиты на другую
    излучается фотон с энергией $0{,}55 \cdot 10^{-19}\,\text{Дж}$.
    Какова длина волны этой линии спектра?
    Постоянная Планка $h = 6{,}626 \cdot 10^{-34}\,\text{Дж}\cdot\text{с}$, скорость света $c = 3 \cdot 10^{8}\,\frac{\text{м}}{\text{с}}$.
}
\answer{%
    $
        E = h\nu = h \frac c\lambda
        \implies \lambda = \frac{hc}E
            = \frac{6{,}626 \cdot 10^{-34}\,\text{Дж}\cdot\text{с} \cdot {3 \cdot 10^{8}\,\frac{\text{м}}{\text{с}}}}{0{,}55 \cdot 10^{-19}\,\text{Дж}}
            = 3614\,\text{нм}.
    $
}
\solutionspace{150pt}

\tasknumber{6}%
\task{%
    Излучение какой длины волны поглотил атом водорода, если полная энергия в атоме увеличилась на $3 \cdot 10^{-19}\,\text{Дж}$?
    Постоянная Планка $h = 6{,}626 \cdot 10^{-34}\,\text{Дж}\cdot\text{с}$, скорость света $c = 3 \cdot 10^{8}\,\frac{\text{м}}{\text{с}}$.
}
\answer{%
    $
        E = h\nu = h \frac c\lambda
        \implies \lambda = \frac{hc}E
            = \frac{6{,}626 \cdot 10^{-34}\,\text{Дж}\cdot\text{с} \cdot {3 \cdot 10^{8}\,\frac{\text{м}}{\text{с}}}}{3 \cdot 10^{-19}\,\text{Дж}}
            = 663\,\text{нм}.
    $
}
\solutionspace{150pt}

\tasknumber{7}%
\task{%
    Сделайте схематичный рисунок энергетических уровней атома водорода
    и отметьте на нём первый (основной) уровень и последующие.
    Сколько различных длин волн может испустить атом водорода,
    находящийся в 5-м возбуждённом состоянии?
    Отметьте все соответствующие переходы на рисунке и укажите,
    при каком переходе (среди отмеченных) энергия излучённого фотона максимальна.
}
\answer{%
    $N = 10{,}0, \text{самая длинная линия}$
}
\solutionspace{150pt}

\tasknumber{8}%
\task{%
    Сколько фотонов испускает за $120\,\text{мин}$ лазер,
    если мощность его излучения $40\,\text{мВт}$?
    Длина волны излучения $600\,\text{нм}$.
    $h = 6{,}626 \cdot 10^{-34}\,\text{Дж}\cdot\text{с}$.
}
\answer{%
    $
        N
            = \frac{E_{\text{общая}}}{E_{\text{одного фотона}}}
            = \frac{Pt}{h\nu} = \frac{Pt}{h \frac c\lambda}
            = \frac{Pt\lambda}{hc}
            = \frac{40\,\text{мВт} \cdot 120\,\text{мин} \cdot 600\,\text{нм}}{6{,}626 \cdot 10^{-34}\,\text{Дж}\cdot\text{с} \cdot 3 \cdot 10^{8}\,\frac{\text{м}}{\text{с}}}
            \approx 8{,}69 \cdot 10^{20}\units{фотонов}
    $
}
\solutionspace{120pt}

\tasknumber{9}%
\task{%
    Какая доля (от начального количества) радиоактивных ядер останется через время,
    равное трём периодам полураспада? Ответ выразить в процентах.
}
\answer{%
    \begin{align*}
    N &= N_0 \cdot 2^{- \frac t{T_{1/2}}} \implies
        \frac N{N_0} = 2^{- \frac t{T_{1/2}}}
        = 2^{-3} \approx 0{,}12 \approx 12\% \\
    N_\text{расп.} &= N_0 - N = N_0 - N_0 \cdot 2^{-\frac t{T_{1/2}}}
        = N_0\cbr{1 - 2^{-\frac t{T_{1/2}}}} \implies
        \frac{N_\text{расп.}}{N_0} = 1 - 2^{-\frac t{T_{1/2}}}
        = 1 - 2^{-3} \approx 0{,}88 \approx 88\%
    \end{align*}
}
\solutionspace{150pt}

\tasknumber{10}%
\task{%
    Сколько процентов ядер радиоактивного железа $\ce{^{59}Fe}$
    останется через $136{,}8\,\text{суток}$, если период его полураспада составляет $45{,}6\,\text{суток}$?
}
\answer{%
    \begin{align*}
    N &= N_0 \cdot 2^{-\frac t{T_{1/2}}}
        = 2^{-\frac{136{,}8\,\text{суток}}{45{,}6\,\text{суток}}}
        \approx 0{,}1250 = 12{,}50\%
    \end{align*}
}
\solutionspace{150pt}

\tasknumber{11}%
\task{%
    За $2\,\text{суток}$ от начального количества ядер радиоизотопа осталась одна восьмая.
    Каков период полураспада этого изотопа (ответ приведите в сутках)?
    Какая ещё доля (также от начального количества) распадётся, если подождать ещё столько же?
}
\answer{%
    \begin{align*}
            N &= N_0 \cdot 2^{-\frac t{T_{1/2}}}
            \implies \frac N{N_0} = 2^{-\frac t{T_{1/2}}}
            \implies \frac 1{8} = 2^{-\frac {2\,\text{суток}}{T_{1/2}}}
            \implies 3 = \frac {2\,\text{суток}}{T_{1/2}}
            \implies T_{1/2} = \frac {2\,\text{суток}}3 \approx 0{,}67\,\text{суток}.
         \\
            \delta &= \frac{N(t)}{N_0} - \frac{N(2t)}{N_0}
            = 2^{-\frac t{T_{1/2}}} - 2^{-\frac {2t}{T_{1/2}}}
            = 2^{-\frac t{T_{1/2}}}\cbr{1 - 2^{-\frac t{T_{1/2}}}}
            = \frac 1{8} \cdot \cbr{1-\frac 1{8}} \approx 0{,}109
    \end{align*}
}
\solutionspace{150pt}

\tasknumber{12}%
\task{%
    Энергия связи ядра гелия \ce{^{3}_{2}He} равна $28{,}29\,\text{МэВ}$.
    Найти дефект массы этого ядра.
    Ответ выразите в а.е.м.
    и кг.
    Скорость света $c = 2{,}998 \cdot 10^{8}\,\frac{\text{м}}{\text{с}}$, элементарный заряд $e = 1{,}6 \cdot 10^{-19}\,\text{Кл}$.
}
\answer{%
    \begin{align*}
    E_\text{св.} &= \Delta m c^2 \implies \\
    \implies
            \Delta m &= \frac {E_\text{св.}}{c^2} = \frac{28{,}29\,\text{МэВ}}{\sqr{2{,}998 \cdot 10^{8}\,\frac{\text{м}}{\text{с}}}}
            = \frac{28{,}29 \cdot 10^6 \cdot 1{,}6 \cdot 10^{-19}\,\text{Дж}}{\sqr{2{,}998 \cdot 10^{8}\,\frac{\text{м}}{\text{с}}}}
            \approx 50{,}36 \cdot 10^{-30}\,\text{кг} \approx 0{,}03033\,\text{а.е.м.}
    \end{align*}
}

\variantsplitter

\addpersonalvariant{Георгий Новиков}

\tasknumber{1}%
\task{%
    Для частицы, движущейся с релятивистской скоростью,
    выразите $p$ и $v$ через $c$, $E_0$ и $E_\text{кин}$,
    где $E_\text{кин}$~--- кинетическая энергия частицы,
    а $E_0$, $p$ и $v$~--- её энергия покоя импульс и скорость.
}
\answer{%
    \begin{align*}
    E_\text{кин}, E_0:\quad&E = E_\text{кин} + E_0 = \frac{E_0}{\sqrt{1 - \frac{v^2}{c^2}}} \implies \sqrt{1 - \frac{v^2}{c^2}} = \frac{E_0}{{E_0} + {E_\text{кин}}} \implies v = c\sqrt{1 - \sqr{\frac{E_0}{{E_0} + {E_\text{кин}}}}} \\
    &p = \frac{mv}{\sqrt{1 - \frac{v^2}{c^2}}} = \frac{E_0}{c^2} \cdot \sqrt{1 - \sqr{\frac{E_0}{{E_0} + {E_\text{кин}}}}} \cdot \frac{{E_\text{кин}} + {E_0}}{E_0} = \frac{E_0}{c^2} \cdot \sqrt{\sqr{\frac{{E_\text{кин}} + {E_0}}{E_0}} - 1}.
    \\
    E_\text{кин}, p:\quad&E_\text{кин} = E - E_0 = mc^2\cbr{\frac 1{\sqrt{1 - \frac{v^2}{c^2}}} - 1}, p = \frac{mv}{\sqrt{1 - \frac{v^2}{c^2}}} \implies \frac{E_\text{кин}}{p} = \frac{\frac 1{\sqrt{1 - \frac{v^2}{c^2}}} - 1}{\sqrt{1 - \frac{v^2}{c^2}}} \implies v = \ldots \\
    &E_0 = E - E_\text{кин} = \frac{E_0}{\sqrt{1 - \frac{v^2}{c^2}}} - E_\text{кин} \implies E_0 = \frac{E_\text{кин}}{\frac 1{\sqrt{1 - \frac{v^2}{c^2}}} - 1} = \ldots \\
    E_\text{кин}, v:\quad&E_\text{кин} = E - E_0 = mc^2\cbr{\frac 1{\sqrt{1 - \frac{v^2}{c^2}}} - 1} \implies m = \frac{E_\text{кин}}{c^2\cbr{\frac 1{\sqrt{1 - \frac{v^2}{c^2}}} - 1}} \\
    &E_0 = mc^2 = \frac{E_\text{кин}}{\frac 1{\sqrt{1 - \frac{v^2}{c^2}}} - 1} \\
    &p = \frac{mv}{\sqrt{1 - \frac{v^2}{c^2}}} = \frac{E_\text{кин}}{c^2\cbr{\frac 1{\sqrt{1 - \frac{v^2}{c^2}}} - 1}} \cdot \frac{v}{\sqrt{1 - \frac{v^2}{c^2}}} = \frac{{E_\text{кин}} v}{c^2\cbr{1 - {\sqrt{1 - \frac{v^2}{c^2}}}}} \\
    E_0, p:\quad&E_0 = mc^2, \quad p = \frac{mv}{\sqrt{1 - \frac{v^2}{c^2}}} \implies \frac{E_0}{p} = \frac{c^2}v{\sqrt{1 - \frac{v^2}{c^2}}} = c\sqrt{\frac{c^2}{v^2} - 1} \\
    &\sqr{\frac{E_0}{pc}} = \frac{c^2}{v^2} - 1 \implies \frac{v^2}{c^2} = \frac 1{1 + \frac{E_0^2}{p^2c^2}} \implies v = \frac c{\sqrt{1 + \frac{E_0^2}{p^2c^2}}} \\
    &{E_\text{кин}} = E - E_0 = \sqrt{E_0^2 + p^2c^2} - E_0 \\
    E_0, v:\quad&E_0 = mc^2 \implies m = \frac{E_0}{c^2} \qquad p = \frac{mv}{\sqrt{1 - \frac{v^2}{c^2}}} = \frac{E_0}{c^2} \cdot \frac{v}{\sqrt{1 - \frac{v^2}{c^2}}} \\
    &E_\text{кин}= mc^2\cbr{\frac 1{\sqrt{1 - \frac{v^2}{c^2}}} - 1} = \frac{E_0}{c^2}\cbr{\frac 1{\sqrt{1 - \frac{v^2}{c^2}}} - 1} \\
    p, v:\quad&p = \frac{mv}{\sqrt{1 - \frac{v^2}{c^2}}} \implies m = \frac p v {\sqrt{1 - \frac{v^2}{c^2}}} \implies E_0 = mc^2 =\frac {pc^2} v {\sqrt{1 - \frac{v^2}{c^2}}} \\
    &E_\text{кин} = mc^2\cbr{\frac 1{\sqrt{1 - \frac{v^2}{c^2}}} - 1} = \frac p v {\sqrt{1 - \frac{v^2}{c^2}}}\cbr{\frac 1{\sqrt{1 - \frac{v^2}{c^2}}} - 1} = \frac p v \cbr{1 - {\sqrt{1 - \frac{v^2}{c^2}}}}
    \end{align*}
}
\solutionspace{200pt}

\tasknumber{2}%
\task{%
    Позитрон движется со скоростью $0{,}9\,c$, где $c$~--- скорость света в вакууме.
    Каково при этом отношение полной энергии частицы $E$ к его энергии покоя $E_0$?
}
\answer{%
    \begin{align*}
    E &= \frac{E_0}{\sqrt{1 - \frac{v^2}{c^2}}}
            \implies \frac E{E_0}
                = \frac 1{\sqrt{1 - \frac{v^2}{c^2}}}
                = \frac 1{\sqrt{1 - \sqr{0{,}9}}}
                \approx 2{,}294,
         \\
        E_{\text{кин}} &= E - E_0
            \implies \frac{E_{\text{кин}}}{E_0}
                = \frac E{E_0} - 1
                = \frac 1{\sqrt{1 - \frac{v^2}{c^2}}} - 1
                = \frac 1{\sqrt{1 - \sqr{0{,}9}}} - 1
                \approx 1{,}294.
    \end{align*}
}
\solutionspace{150pt}

\tasknumber{3}%
\task{%
    Протон движется со скоростью $0{,}75\,c$, где $c$~--- скорость света в вакууме.
    Определите его полную энергию (в ответе приведите формулу и укажите численное значение).
}
\answer{%
    \begin{align*}
    E &= \frac{mc^2}{\sqrt{1 - \frac{v^2}{c^2}}}
            \approx \frac{1{,}673 \cdot 10^{-27}\,\text{кг} \cdot \sqr{3 \cdot 10^{8}\,\frac{\text{м}}{\text{с}}}}{\sqrt{1 - 0{,}75^2}}
            \approx 227{,}589 \cdot 10^{-12}\,\text{Дж},
         \\
        E_{\text{кин}} &= \frac{mc^2}{\sqrt{1 - \frac{v^2}{c^2}}} - mc^2
            = mc^2 \cbr{\frac 1{\sqrt{1 - \frac{v^2}{c^2}}} - 1} \approx \\
            &\approx \cbr{1{,}673 \cdot 10^{-27}\,\text{кг} \cdot \sqr{3 \cdot 10^{8}\,\frac{\text{м}}{\text{с}}}}
            \cdot \cbr{\frac 1{\sqrt{1 - 0{,}75^2}} - 1}
            \approx 77{,}053 \cdot 10^{-12}\,\text{Дж},
         \\
        p &= \frac{mv}{\sqrt{1 - \frac{v^2}{c^2}}}
            \approx \frac{1{,}673 \cdot 10^{-27}\,\text{кг} \cdot 0{,}75 \cdot 3 \cdot 10^{8}\,\frac{\text{м}}{\text{с}}}{\sqrt{1 - 0{,}75^2}}
            \approx 568{,}972 \cdot 10^{-21}\,\frac{\text{кг}\cdot\text{м}}{\text{с}}.
    \end{align*}
}
\solutionspace{150pt}

\tasknumber{4}%
\task{%
    При какой скорости движения (в долях скорости света) релятивистское сокращение длины движущегося тела
    составит 30\%?
}
\answer{%
    \begin{align*}
    l_0 &= \frac l{\sqrt{1 - \frac{v^2}{c^2}}}
        \implies 1 - \frac{v^2}{c^2} = \sqr{\frac l{l_0}}
        \implies \frac v c = \sqrt{1 - \sqr{\frac l{l_0}}} \implies
         \\
        \implies v &= c\sqrt{1 - \sqr{\frac l{l_0}}}
        = 3 \cdot 10^{8}\,\frac{\text{м}}{\text{с}} \cdot \sqrt{1 - \sqr{\frac {l_0 - 0{,}30l_0}{l_0}}}
        = 3 \cdot 10^{8}\,\frac{\text{м}}{\text{с}} \cdot \sqrt{1 - \sqr{1 - 0{,}30}} \approx  \\
        &\approx 0{,}714c
        \approx 214 \cdot 10^{6}\,\frac{\text{м}}{\text{с}}
        \approx 771 \cdot 10^{6}\,\frac{\text{км}}{\text{ч}}.
    \end{align*}
}
\solutionspace{150pt}

\tasknumber{5}%
\task{%
    При переходе электрона в атоме с одной стационарной орбиты на другую
    излучается фотон с энергией $0{,}55 \cdot 10^{-19}\,\text{Дж}$.
    Какова длина волны этой линии спектра?
    Постоянная Планка $h = 6{,}626 \cdot 10^{-34}\,\text{Дж}\cdot\text{с}$, скорость света $c = 3 \cdot 10^{8}\,\frac{\text{м}}{\text{с}}$.
}
\answer{%
    $
        E = h\nu = h \frac c\lambda
        \implies \lambda = \frac{hc}E
            = \frac{6{,}626 \cdot 10^{-34}\,\text{Дж}\cdot\text{с} \cdot {3 \cdot 10^{8}\,\frac{\text{м}}{\text{с}}}}{0{,}55 \cdot 10^{-19}\,\text{Дж}}
            = 3614\,\text{нм}.
    $
}
\solutionspace{150pt}

\tasknumber{6}%
\task{%
    Излучение какой длины волны поглотил атом водорода, если полная энергия в атоме увеличилась на $6 \cdot 10^{-19}\,\text{Дж}$?
    Постоянная Планка $h = 6{,}626 \cdot 10^{-34}\,\text{Дж}\cdot\text{с}$, скорость света $c = 3 \cdot 10^{8}\,\frac{\text{м}}{\text{с}}$.
}
\answer{%
    $
        E = h\nu = h \frac c\lambda
        \implies \lambda = \frac{hc}E
            = \frac{6{,}626 \cdot 10^{-34}\,\text{Дж}\cdot\text{с} \cdot {3 \cdot 10^{8}\,\frac{\text{м}}{\text{с}}}}{6 \cdot 10^{-19}\,\text{Дж}}
            = 331\,\text{нм}.
    $
}
\solutionspace{150pt}

\tasknumber{7}%
\task{%
    Сделайте схематичный рисунок энергетических уровней атома водорода
    и отметьте на нём первый (основной) уровень и последующие.
    Сколько различных длин волн может испустить атом водорода,
    находящийся в 5-м возбуждённом состоянии?
    Отметьте все соответствующие переходы на рисунке и укажите,
    при каком переходе (среди отмеченных) частота излучённого фотона максимальна.
}
\answer{%
    $N = 10{,}0, \text{самая длинная линия}$
}
\solutionspace{150pt}

\tasknumber{8}%
\task{%
    Сколько фотонов испускает за $60\,\text{мин}$ лазер,
    если мощность его излучения $75\,\text{мВт}$?
    Длина волны излучения $500\,\text{нм}$.
    $h = 6{,}626 \cdot 10^{-34}\,\text{Дж}\cdot\text{с}$.
}
\answer{%
    $
        N
            = \frac{E_{\text{общая}}}{E_{\text{одного фотона}}}
            = \frac{Pt}{h\nu} = \frac{Pt}{h \frac c\lambda}
            = \frac{Pt\lambda}{hc}
            = \frac{75\,\text{мВт} \cdot 60\,\text{мин} \cdot 500\,\text{нм}}{6{,}626 \cdot 10^{-34}\,\text{Дж}\cdot\text{с} \cdot 3 \cdot 10^{8}\,\frac{\text{м}}{\text{с}}}
            \approx 6{,}79 \cdot 10^{20}\units{фотонов}
    $
}
\solutionspace{120pt}

\tasknumber{9}%
\task{%
    Какая доля (от начального количества) радиоактивных ядер распадётся через время,
    равное трём периодам полураспада? Ответ выразить в процентах.
}
\answer{%
    \begin{align*}
    N &= N_0 \cdot 2^{- \frac t{T_{1/2}}} \implies
        \frac N{N_0} = 2^{- \frac t{T_{1/2}}}
        = 2^{-3} \approx 0{,}12 \approx 12\% \\
    N_\text{расп.} &= N_0 - N = N_0 - N_0 \cdot 2^{-\frac t{T_{1/2}}}
        = N_0\cbr{1 - 2^{-\frac t{T_{1/2}}}} \implies
        \frac{N_\text{расп.}}{N_0} = 1 - 2^{-\frac t{T_{1/2}}}
        = 1 - 2^{-3} \approx 0{,}88 \approx 88\%
    \end{align*}
}
\solutionspace{150pt}

\tasknumber{10}%
\task{%
    Сколько процентов ядер радиоактивного железа $\ce{^{59}Fe}$
    останется через $182{,}4\,\text{суток}$, если период его полураспада составляет $45{,}6\,\text{суток}$?
}
\answer{%
    \begin{align*}
    N &= N_0 \cdot 2^{-\frac t{T_{1/2}}}
        = 2^{-\frac{182{,}4\,\text{суток}}{45{,}6\,\text{суток}}}
        \approx 0{,}0625 = 6{,}25\%
    \end{align*}
}
\solutionspace{150pt}

\tasknumber{11}%
\task{%
    За $4\,\text{суток}$ от начального количества ядер радиоизотопа осталась одна восьмая.
    Каков период полураспада этого изотопа (ответ приведите в сутках)?
    Какая ещё доля (также от начального количества) распадётся, если подождать ещё столько же?
}
\answer{%
    \begin{align*}
            N &= N_0 \cdot 2^{-\frac t{T_{1/2}}}
            \implies \frac N{N_0} = 2^{-\frac t{T_{1/2}}}
            \implies \frac 1{8} = 2^{-\frac {4\,\text{суток}}{T_{1/2}}}
            \implies 3 = \frac {4\,\text{суток}}{T_{1/2}}
            \implies T_{1/2} = \frac {4\,\text{суток}}3 \approx 1{,}33\,\text{суток}.
         \\
            \delta &= \frac{N(t)}{N_0} - \frac{N(2t)}{N_0}
            = 2^{-\frac t{T_{1/2}}} - 2^{-\frac {2t}{T_{1/2}}}
            = 2^{-\frac t{T_{1/2}}}\cbr{1 - 2^{-\frac t{T_{1/2}}}}
            = \frac 1{8} \cdot \cbr{1-\frac 1{8}} \approx 0{,}109
    \end{align*}
}
\solutionspace{150pt}

\tasknumber{12}%
\task{%
    Энергия связи ядра бора \ce{^{10}_{5}B} равна $64{,}7\,\text{МэВ}$.
    Найти дефект массы этого ядра.
    Ответ выразите в а.е.м.
    и кг.
    Скорость света $c = 2{,}998 \cdot 10^{8}\,\frac{\text{м}}{\text{с}}$, элементарный заряд $e = 1{,}6 \cdot 10^{-19}\,\text{Кл}$.
}
\answer{%
    \begin{align*}
    E_\text{св.} &= \Delta m c^2 \implies \\
    \implies
            \Delta m &= \frac {E_\text{св.}}{c^2} = \frac{64{,}7\,\text{МэВ}}{\sqr{2{,}998 \cdot 10^{8}\,\frac{\text{м}}{\text{с}}}}
            = \frac{64{,}7 \cdot 10^6 \cdot 1{,}6 \cdot 10^{-19}\,\text{Дж}}{\sqr{2{,}998 \cdot 10^{8}\,\frac{\text{м}}{\text{с}}}}
            \approx 0{,}1152 \cdot 10^{-27}\,\text{кг} \approx 0{,}0694\,\text{а.е.м.}
    \end{align*}
}

\variantsplitter

\addpersonalvariant{Егор Осипов}

\tasknumber{1}%
\task{%
    Для частицы, движущейся с релятивистской скоростью,
    выразите $v$ и $p$ через $c$, $E_0$ и $E_\text{кин}$,
    где $E_\text{кин}$~--- кинетическая энергия частицы,
    а $E_0$, $p$ и $v$~--- её энергия покоя импульс и скорость.
}
\answer{%
    \begin{align*}
    E_\text{кин}, E_0:\quad&E = E_\text{кин} + E_0 = \frac{E_0}{\sqrt{1 - \frac{v^2}{c^2}}} \implies \sqrt{1 - \frac{v^2}{c^2}} = \frac{E_0}{{E_0} + {E_\text{кин}}} \implies v = c\sqrt{1 - \sqr{\frac{E_0}{{E_0} + {E_\text{кин}}}}} \\
    &p = \frac{mv}{\sqrt{1 - \frac{v^2}{c^2}}} = \frac{E_0}{c^2} \cdot \sqrt{1 - \sqr{\frac{E_0}{{E_0} + {E_\text{кин}}}}} \cdot \frac{{E_\text{кин}} + {E_0}}{E_0} = \frac{E_0}{c^2} \cdot \sqrt{\sqr{\frac{{E_\text{кин}} + {E_0}}{E_0}} - 1}.
    \\
    E_\text{кин}, p:\quad&E_\text{кин} = E - E_0 = mc^2\cbr{\frac 1{\sqrt{1 - \frac{v^2}{c^2}}} - 1}, p = \frac{mv}{\sqrt{1 - \frac{v^2}{c^2}}} \implies \frac{E_\text{кин}}{p} = \frac{\frac 1{\sqrt{1 - \frac{v^2}{c^2}}} - 1}{\sqrt{1 - \frac{v^2}{c^2}}} \implies v = \ldots \\
    &E_0 = E - E_\text{кин} = \frac{E_0}{\sqrt{1 - \frac{v^2}{c^2}}} - E_\text{кин} \implies E_0 = \frac{E_\text{кин}}{\frac 1{\sqrt{1 - \frac{v^2}{c^2}}} - 1} = \ldots \\
    E_\text{кин}, v:\quad&E_\text{кин} = E - E_0 = mc^2\cbr{\frac 1{\sqrt{1 - \frac{v^2}{c^2}}} - 1} \implies m = \frac{E_\text{кин}}{c^2\cbr{\frac 1{\sqrt{1 - \frac{v^2}{c^2}}} - 1}} \\
    &E_0 = mc^2 = \frac{E_\text{кин}}{\frac 1{\sqrt{1 - \frac{v^2}{c^2}}} - 1} \\
    &p = \frac{mv}{\sqrt{1 - \frac{v^2}{c^2}}} = \frac{E_\text{кин}}{c^2\cbr{\frac 1{\sqrt{1 - \frac{v^2}{c^2}}} - 1}} \cdot \frac{v}{\sqrt{1 - \frac{v^2}{c^2}}} = \frac{{E_\text{кин}} v}{c^2\cbr{1 - {\sqrt{1 - \frac{v^2}{c^2}}}}} \\
    E_0, p:\quad&E_0 = mc^2, \quad p = \frac{mv}{\sqrt{1 - \frac{v^2}{c^2}}} \implies \frac{E_0}{p} = \frac{c^2}v{\sqrt{1 - \frac{v^2}{c^2}}} = c\sqrt{\frac{c^2}{v^2} - 1} \\
    &\sqr{\frac{E_0}{pc}} = \frac{c^2}{v^2} - 1 \implies \frac{v^2}{c^2} = \frac 1{1 + \frac{E_0^2}{p^2c^2}} \implies v = \frac c{\sqrt{1 + \frac{E_0^2}{p^2c^2}}} \\
    &{E_\text{кин}} = E - E_0 = \sqrt{E_0^2 + p^2c^2} - E_0 \\
    E_0, v:\quad&E_0 = mc^2 \implies m = \frac{E_0}{c^2} \qquad p = \frac{mv}{\sqrt{1 - \frac{v^2}{c^2}}} = \frac{E_0}{c^2} \cdot \frac{v}{\sqrt{1 - \frac{v^2}{c^2}}} \\
    &E_\text{кин}= mc^2\cbr{\frac 1{\sqrt{1 - \frac{v^2}{c^2}}} - 1} = \frac{E_0}{c^2}\cbr{\frac 1{\sqrt{1 - \frac{v^2}{c^2}}} - 1} \\
    p, v:\quad&p = \frac{mv}{\sqrt{1 - \frac{v^2}{c^2}}} \implies m = \frac p v {\sqrt{1 - \frac{v^2}{c^2}}} \implies E_0 = mc^2 =\frac {pc^2} v {\sqrt{1 - \frac{v^2}{c^2}}} \\
    &E_\text{кин} = mc^2\cbr{\frac 1{\sqrt{1 - \frac{v^2}{c^2}}} - 1} = \frac p v {\sqrt{1 - \frac{v^2}{c^2}}}\cbr{\frac 1{\sqrt{1 - \frac{v^2}{c^2}}} - 1} = \frac p v \cbr{1 - {\sqrt{1 - \frac{v^2}{c^2}}}}
    \end{align*}
}
\solutionspace{200pt}

\tasknumber{2}%
\task{%
    Позитрон движется со скоростью $0{,}8\,c$, где $c$~--- скорость света в вакууме.
    Каково при этом отношение полной энергии частицы $E$ к его энергии покоя $E_0$?
}
\answer{%
    \begin{align*}
    E &= \frac{E_0}{\sqrt{1 - \frac{v^2}{c^2}}}
            \implies \frac E{E_0}
                = \frac 1{\sqrt{1 - \frac{v^2}{c^2}}}
                = \frac 1{\sqrt{1 - \sqr{0{,}8}}}
                \approx 1{,}667,
         \\
        E_{\text{кин}} &= E - E_0
            \implies \frac{E_{\text{кин}}}{E_0}
                = \frac E{E_0} - 1
                = \frac 1{\sqrt{1 - \frac{v^2}{c^2}}} - 1
                = \frac 1{\sqrt{1 - \sqr{0{,}8}}} - 1
                \approx 0{,}667.
    \end{align*}
}
\solutionspace{150pt}

\tasknumber{3}%
\task{%
    Протон движется со скоростью $0{,}65\,c$, где $c$~--- скорость света в вакууме.
    Определите его импульс (в ответе приведите формулу и укажите численное значение).
}
\answer{%
    \begin{align*}
    E &= \frac{mc^2}{\sqrt{1 - \frac{v^2}{c^2}}}
            \approx \frac{1{,}673 \cdot 10^{-27}\,\text{кг} \cdot \sqr{3 \cdot 10^{8}\,\frac{\text{м}}{\text{с}}}}{\sqrt{1 - 0{,}65^2}}
            \approx 198{,}091 \cdot 10^{-12}\,\text{Дж},
         \\
        E_{\text{кин}} &= \frac{mc^2}{\sqrt{1 - \frac{v^2}{c^2}}} - mc^2
            = mc^2 \cbr{\frac 1{\sqrt{1 - \frac{v^2}{c^2}}} - 1} \approx \\
            &\approx \cbr{1{,}673 \cdot 10^{-27}\,\text{кг} \cdot \sqr{3 \cdot 10^{8}\,\frac{\text{м}}{\text{с}}}}
            \cdot \cbr{\frac 1{\sqrt{1 - 0{,}65^2}} - 1}
            \approx 47{,}555 \cdot 10^{-12}\,\text{Дж},
         \\
        p &= \frac{mv}{\sqrt{1 - \frac{v^2}{c^2}}}
            \approx \frac{1{,}673 \cdot 10^{-27}\,\text{кг} \cdot 0{,}65 \cdot 3 \cdot 10^{8}\,\frac{\text{м}}{\text{с}}}{\sqrt{1 - 0{,}65^2}}
            \approx 429{,}196 \cdot 10^{-21}\,\frac{\text{кг}\cdot\text{м}}{\text{с}}.
    \end{align*}
}
\solutionspace{150pt}

\tasknumber{4}%
\task{%
    При какой скорости движения (в км/ч) релятивистское сокращение длины движущегося тела
    составит 30\%?
}
\answer{%
    \begin{align*}
    l_0 &= \frac l{\sqrt{1 - \frac{v^2}{c^2}}}
        \implies 1 - \frac{v^2}{c^2} = \sqr{\frac l{l_0}}
        \implies \frac v c = \sqrt{1 - \sqr{\frac l{l_0}}} \implies
         \\
        \implies v &= c\sqrt{1 - \sqr{\frac l{l_0}}}
        = 3 \cdot 10^{8}\,\frac{\text{м}}{\text{с}} \cdot \sqrt{1 - \sqr{\frac {l_0 - 0{,}30l_0}{l_0}}}
        = 3 \cdot 10^{8}\,\frac{\text{м}}{\text{с}} \cdot \sqrt{1 - \sqr{1 - 0{,}30}} \approx  \\
        &\approx 0{,}714c
        \approx 214 \cdot 10^{6}\,\frac{\text{м}}{\text{с}}
        \approx 771 \cdot 10^{6}\,\frac{\text{км}}{\text{ч}}.
    \end{align*}
}
\solutionspace{150pt}

\tasknumber{5}%
\task{%
    При переходе электрона в атоме с одной стационарной орбиты на другую
    излучается фотон с энергией $4{,}04 \cdot 10^{-19}\,\text{Дж}$.
    Какова длина волны этой линии спектра?
    Постоянная Планка $h = 6{,}626 \cdot 10^{-34}\,\text{Дж}\cdot\text{с}$, скорость света $c = 3 \cdot 10^{8}\,\frac{\text{м}}{\text{с}}$.
}
\answer{%
    $
        E = h\nu = h \frac c\lambda
        \implies \lambda = \frac{hc}E
            = \frac{6{,}626 \cdot 10^{-34}\,\text{Дж}\cdot\text{с} \cdot {3 \cdot 10^{8}\,\frac{\text{м}}{\text{с}}}}{4{,}04 \cdot 10^{-19}\,\text{Дж}}
            = 492{,}03\,\text{нм}.
    $
}
\solutionspace{150pt}

\tasknumber{6}%
\task{%
    Излучение какой длины волны поглотил атом водорода, если полная энергия в атоме увеличилась на $2 \cdot 10^{-19}\,\text{Дж}$?
    Постоянная Планка $h = 6{,}626 \cdot 10^{-34}\,\text{Дж}\cdot\text{с}$, скорость света $c = 3 \cdot 10^{8}\,\frac{\text{м}}{\text{с}}$.
}
\answer{%
    $
        E = h\nu = h \frac c\lambda
        \implies \lambda = \frac{hc}E
            = \frac{6{,}626 \cdot 10^{-34}\,\text{Дж}\cdot\text{с} \cdot {3 \cdot 10^{8}\,\frac{\text{м}}{\text{с}}}}{2 \cdot 10^{-19}\,\text{Дж}}
            = 994\,\text{нм}.
    $
}
\solutionspace{150pt}

\tasknumber{7}%
\task{%
    Сделайте схематичный рисунок энергетических уровней атома водорода
    и отметьте на нём первый (основной) уровень и последующие.
    Сколько различных длин волн может испустить атом водорода,
    находящийся в 4-м возбуждённом состоянии?
    Отметьте все соответствующие переходы на рисунке и укажите,
    при каком переходе (среди отмеченных) частота излучённого фотона минимальна.
}
\answer{%
    $N = 6{,}0, \text{самая короткая линия}$
}
\solutionspace{150pt}

\tasknumber{8}%
\task{%
    Сколько фотонов испускает за $40\,\text{мин}$ лазер,
    если мощность его излучения $15\,\text{мВт}$?
    Длина волны излучения $500\,\text{нм}$.
    $h = 6{,}626 \cdot 10^{-34}\,\text{Дж}\cdot\text{с}$.
}
\answer{%
    $
        N
            = \frac{E_{\text{общая}}}{E_{\text{одного фотона}}}
            = \frac{Pt}{h\nu} = \frac{Pt}{h \frac c\lambda}
            = \frac{Pt\lambda}{hc}
            = \frac{15\,\text{мВт} \cdot 40\,\text{мин} \cdot 500\,\text{нм}}{6{,}626 \cdot 10^{-34}\,\text{Дж}\cdot\text{с} \cdot 3 \cdot 10^{8}\,\frac{\text{м}}{\text{с}}}
            \approx 0{,}91 \cdot 10^{20}\units{фотонов}
    $
}
\solutionspace{120pt}

\tasknumber{9}%
\task{%
    Какая доля (от начального количества) радиоактивных ядер останется через время,
    равное трём периодам полураспада? Ответ выразить в процентах.
}
\answer{%
    \begin{align*}
    N &= N_0 \cdot 2^{- \frac t{T_{1/2}}} \implies
        \frac N{N_0} = 2^{- \frac t{T_{1/2}}}
        = 2^{-3} \approx 0{,}12 \approx 12\% \\
    N_\text{расп.} &= N_0 - N = N_0 - N_0 \cdot 2^{-\frac t{T_{1/2}}}
        = N_0\cbr{1 - 2^{-\frac t{T_{1/2}}}} \implies
        \frac{N_\text{расп.}}{N_0} = 1 - 2^{-\frac t{T_{1/2}}}
        = 1 - 2^{-3} \approx 0{,}88 \approx 88\%
    \end{align*}
}
\solutionspace{150pt}

\tasknumber{10}%
\task{%
    Сколько процентов ядер радиоактивного железа $\ce{^{59}Fe}$
    останется через $91{,}2\,\text{суток}$, если период его полураспада составляет $45{,}6\,\text{суток}$?
}
\answer{%
    \begin{align*}
    N &= N_0 \cdot 2^{-\frac t{T_{1/2}}}
        = 2^{-\frac{91{,}2\,\text{суток}}{45{,}6\,\text{суток}}}
        \approx 0{,}2500 = 25{,}00\%
    \end{align*}
}
\solutionspace{150pt}

\tasknumber{11}%
\task{%
    За $2\,\text{суток}$ от начального количества ядер радиоизотопа осталась одна шестнадцатая.
    Каков период полураспада этого изотопа (ответ приведите в сутках)?
    Какая ещё доля (также от начального количества) распадётся, если подождать ещё столько же?
}
\answer{%
    \begin{align*}
            N &= N_0 \cdot 2^{-\frac t{T_{1/2}}}
            \implies \frac N{N_0} = 2^{-\frac t{T_{1/2}}}
            \implies \frac 1{16} = 2^{-\frac {2\,\text{суток}}{T_{1/2}}}
            \implies 4 = \frac {2\,\text{суток}}{T_{1/2}}
            \implies T_{1/2} = \frac {2\,\text{суток}}4 \approx 0{,}50\,\text{суток}.
         \\
            \delta &= \frac{N(t)}{N_0} - \frac{N(2t)}{N_0}
            = 2^{-\frac t{T_{1/2}}} - 2^{-\frac {2t}{T_{1/2}}}
            = 2^{-\frac t{T_{1/2}}}\cbr{1 - 2^{-\frac t{T_{1/2}}}}
            = \frac 1{16} \cdot \cbr{1-\frac 1{16}} \approx 0{,}059
    \end{align*}
}
\solutionspace{150pt}

\tasknumber{12}%
\task{%
    Энергия связи ядра бора \ce{^{11}_{5}B} равна $76{,}2\,\text{МэВ}$.
    Найти дефект массы этого ядра.
    Ответ выразите в а.е.м.
    и кг.
    Скорость света $c = 2{,}998 \cdot 10^{8}\,\frac{\text{м}}{\text{с}}$, элементарный заряд $e = 1{,}6 \cdot 10^{-19}\,\text{Кл}$.
}
\answer{%
    \begin{align*}
    E_\text{св.} &= \Delta m c^2 \implies \\
    \implies
            \Delta m &= \frac {E_\text{св.}}{c^2} = \frac{76{,}2\,\text{МэВ}}{\sqr{2{,}998 \cdot 10^{8}\,\frac{\text{м}}{\text{с}}}}
            = \frac{76{,}2 \cdot 10^6 \cdot 1{,}6 \cdot 10^{-19}\,\text{Дж}}{\sqr{2{,}998 \cdot 10^{8}\,\frac{\text{м}}{\text{с}}}}
            \approx 0{,}1356 \cdot 10^{-27}\,\text{кг} \approx 0{,}0817\,\text{а.е.м.}
    \end{align*}
}

\variantsplitter

\addpersonalvariant{Руслан Перепелица}

\tasknumber{1}%
\task{%
    Для частицы, движущейся с релятивистской скоростью,
    выразите $v$ и $p$ через $c$, $E_0$ и $E_\text{кин}$,
    где $E_\text{кин}$~--- кинетическая энергия частицы,
    а $E_0$, $p$ и $v$~--- её энергия покоя импульс и скорость.
}
\answer{%
    \begin{align*}
    E_\text{кин}, E_0:\quad&E = E_\text{кин} + E_0 = \frac{E_0}{\sqrt{1 - \frac{v^2}{c^2}}} \implies \sqrt{1 - \frac{v^2}{c^2}} = \frac{E_0}{{E_0} + {E_\text{кин}}} \implies v = c\sqrt{1 - \sqr{\frac{E_0}{{E_0} + {E_\text{кин}}}}} \\
    &p = \frac{mv}{\sqrt{1 - \frac{v^2}{c^2}}} = \frac{E_0}{c^2} \cdot \sqrt{1 - \sqr{\frac{E_0}{{E_0} + {E_\text{кин}}}}} \cdot \frac{{E_\text{кин}} + {E_0}}{E_0} = \frac{E_0}{c^2} \cdot \sqrt{\sqr{\frac{{E_\text{кин}} + {E_0}}{E_0}} - 1}.
    \\
    E_\text{кин}, p:\quad&E_\text{кин} = E - E_0 = mc^2\cbr{\frac 1{\sqrt{1 - \frac{v^2}{c^2}}} - 1}, p = \frac{mv}{\sqrt{1 - \frac{v^2}{c^2}}} \implies \frac{E_\text{кин}}{p} = \frac{\frac 1{\sqrt{1 - \frac{v^2}{c^2}}} - 1}{\sqrt{1 - \frac{v^2}{c^2}}} \implies v = \ldots \\
    &E_0 = E - E_\text{кин} = \frac{E_0}{\sqrt{1 - \frac{v^2}{c^2}}} - E_\text{кин} \implies E_0 = \frac{E_\text{кин}}{\frac 1{\sqrt{1 - \frac{v^2}{c^2}}} - 1} = \ldots \\
    E_\text{кин}, v:\quad&E_\text{кин} = E - E_0 = mc^2\cbr{\frac 1{\sqrt{1 - \frac{v^2}{c^2}}} - 1} \implies m = \frac{E_\text{кин}}{c^2\cbr{\frac 1{\sqrt{1 - \frac{v^2}{c^2}}} - 1}} \\
    &E_0 = mc^2 = \frac{E_\text{кин}}{\frac 1{\sqrt{1 - \frac{v^2}{c^2}}} - 1} \\
    &p = \frac{mv}{\sqrt{1 - \frac{v^2}{c^2}}} = \frac{E_\text{кин}}{c^2\cbr{\frac 1{\sqrt{1 - \frac{v^2}{c^2}}} - 1}} \cdot \frac{v}{\sqrt{1 - \frac{v^2}{c^2}}} = \frac{{E_\text{кин}} v}{c^2\cbr{1 - {\sqrt{1 - \frac{v^2}{c^2}}}}} \\
    E_0, p:\quad&E_0 = mc^2, \quad p = \frac{mv}{\sqrt{1 - \frac{v^2}{c^2}}} \implies \frac{E_0}{p} = \frac{c^2}v{\sqrt{1 - \frac{v^2}{c^2}}} = c\sqrt{\frac{c^2}{v^2} - 1} \\
    &\sqr{\frac{E_0}{pc}} = \frac{c^2}{v^2} - 1 \implies \frac{v^2}{c^2} = \frac 1{1 + \frac{E_0^2}{p^2c^2}} \implies v = \frac c{\sqrt{1 + \frac{E_0^2}{p^2c^2}}} \\
    &{E_\text{кин}} = E - E_0 = \sqrt{E_0^2 + p^2c^2} - E_0 \\
    E_0, v:\quad&E_0 = mc^2 \implies m = \frac{E_0}{c^2} \qquad p = \frac{mv}{\sqrt{1 - \frac{v^2}{c^2}}} = \frac{E_0}{c^2} \cdot \frac{v}{\sqrt{1 - \frac{v^2}{c^2}}} \\
    &E_\text{кин}= mc^2\cbr{\frac 1{\sqrt{1 - \frac{v^2}{c^2}}} - 1} = \frac{E_0}{c^2}\cbr{\frac 1{\sqrt{1 - \frac{v^2}{c^2}}} - 1} \\
    p, v:\quad&p = \frac{mv}{\sqrt{1 - \frac{v^2}{c^2}}} \implies m = \frac p v {\sqrt{1 - \frac{v^2}{c^2}}} \implies E_0 = mc^2 =\frac {pc^2} v {\sqrt{1 - \frac{v^2}{c^2}}} \\
    &E_\text{кин} = mc^2\cbr{\frac 1{\sqrt{1 - \frac{v^2}{c^2}}} - 1} = \frac p v {\sqrt{1 - \frac{v^2}{c^2}}}\cbr{\frac 1{\sqrt{1 - \frac{v^2}{c^2}}} - 1} = \frac p v \cbr{1 - {\sqrt{1 - \frac{v^2}{c^2}}}}
    \end{align*}
}
\solutionspace{200pt}

\tasknumber{2}%
\task{%
    Позитрон движется со скоростью $0{,}9\,c$, где $c$~--- скорость света в вакууме.
    Каково при этом отношение полной энергии частицы $E$ к его энергии покоя $E_0$?
}
\answer{%
    \begin{align*}
    E &= \frac{E_0}{\sqrt{1 - \frac{v^2}{c^2}}}
            \implies \frac E{E_0}
                = \frac 1{\sqrt{1 - \frac{v^2}{c^2}}}
                = \frac 1{\sqrt{1 - \sqr{0{,}9}}}
                \approx 2{,}294,
         \\
        E_{\text{кин}} &= E - E_0
            \implies \frac{E_{\text{кин}}}{E_0}
                = \frac E{E_0} - 1
                = \frac 1{\sqrt{1 - \frac{v^2}{c^2}}} - 1
                = \frac 1{\sqrt{1 - \sqr{0{,}9}}} - 1
                \approx 1{,}294.
    \end{align*}
}
\solutionspace{150pt}

\tasknumber{3}%
\task{%
    Протон движется со скоростью $0{,}75\,c$, где $c$~--- скорость света в вакууме.
    Определите его кинетическую энергию (в ответе приведите формулу и укажите численное значение).
}
\answer{%
    \begin{align*}
    E &= \frac{mc^2}{\sqrt{1 - \frac{v^2}{c^2}}}
            \approx \frac{1{,}673 \cdot 10^{-27}\,\text{кг} \cdot \sqr{3 \cdot 10^{8}\,\frac{\text{м}}{\text{с}}}}{\sqrt{1 - 0{,}75^2}}
            \approx 227{,}589 \cdot 10^{-12}\,\text{Дж},
         \\
        E_{\text{кин}} &= \frac{mc^2}{\sqrt{1 - \frac{v^2}{c^2}}} - mc^2
            = mc^2 \cbr{\frac 1{\sqrt{1 - \frac{v^2}{c^2}}} - 1} \approx \\
            &\approx \cbr{1{,}673 \cdot 10^{-27}\,\text{кг} \cdot \sqr{3 \cdot 10^{8}\,\frac{\text{м}}{\text{с}}}}
            \cdot \cbr{\frac 1{\sqrt{1 - 0{,}75^2}} - 1}
            \approx 77{,}053 \cdot 10^{-12}\,\text{Дж},
         \\
        p &= \frac{mv}{\sqrt{1 - \frac{v^2}{c^2}}}
            \approx \frac{1{,}673 \cdot 10^{-27}\,\text{кг} \cdot 0{,}75 \cdot 3 \cdot 10^{8}\,\frac{\text{м}}{\text{с}}}{\sqrt{1 - 0{,}75^2}}
            \approx 568{,}972 \cdot 10^{-21}\,\frac{\text{кг}\cdot\text{м}}{\text{с}}.
    \end{align*}
}
\solutionspace{150pt}

\tasknumber{4}%
\task{%
    При какой скорости движения (в долях скорости света) релятивистское сокращение длины движущегося тела
    составит 50\%?
}
\answer{%
    \begin{align*}
    l_0 &= \frac l{\sqrt{1 - \frac{v^2}{c^2}}}
        \implies 1 - \frac{v^2}{c^2} = \sqr{\frac l{l_0}}
        \implies \frac v c = \sqrt{1 - \sqr{\frac l{l_0}}} \implies
         \\
        \implies v &= c\sqrt{1 - \sqr{\frac l{l_0}}}
        = 3 \cdot 10^{8}\,\frac{\text{м}}{\text{с}} \cdot \sqrt{1 - \sqr{\frac {l_0 - 0{,}50l_0}{l_0}}}
        = 3 \cdot 10^{8}\,\frac{\text{м}}{\text{с}} \cdot \sqrt{1 - \sqr{1 - 0{,}50}} \approx  \\
        &\approx 0{,}866c
        \approx 260 \cdot 10^{6}\,\frac{\text{м}}{\text{с}}
        \approx 935 \cdot 10^{6}\,\frac{\text{км}}{\text{ч}}.
    \end{align*}
}
\solutionspace{150pt}

\tasknumber{5}%
\task{%
    При переходе электрона в атоме с одной стационарной орбиты на другую
    излучается фотон с энергией $1{,}01 \cdot 10^{-19}\,\text{Дж}$.
    Какова длина волны этой линии спектра?
    Постоянная Планка $h = 6{,}626 \cdot 10^{-34}\,\text{Дж}\cdot\text{с}$, скорость света $c = 3 \cdot 10^{8}\,\frac{\text{м}}{\text{с}}$.
}
\answer{%
    $
        E = h\nu = h \frac c\lambda
        \implies \lambda = \frac{hc}E
            = \frac{6{,}626 \cdot 10^{-34}\,\text{Дж}\cdot\text{с} \cdot {3 \cdot 10^{8}\,\frac{\text{м}}{\text{с}}}}{1{,}01 \cdot 10^{-19}\,\text{Дж}}
            = 1968{,}1\,\text{нм}.
    $
}
\solutionspace{150pt}

\tasknumber{6}%
\task{%
    Излучение какой длины волны поглотил атом водорода, если полная энергия в атоме увеличилась на $2 \cdot 10^{-19}\,\text{Дж}$?
    Постоянная Планка $h = 6{,}626 \cdot 10^{-34}\,\text{Дж}\cdot\text{с}$, скорость света $c = 3 \cdot 10^{8}\,\frac{\text{м}}{\text{с}}$.
}
\answer{%
    $
        E = h\nu = h \frac c\lambda
        \implies \lambda = \frac{hc}E
            = \frac{6{,}626 \cdot 10^{-34}\,\text{Дж}\cdot\text{с} \cdot {3 \cdot 10^{8}\,\frac{\text{м}}{\text{с}}}}{2 \cdot 10^{-19}\,\text{Дж}}
            = 994\,\text{нм}.
    $
}
\solutionspace{150pt}

\tasknumber{7}%
\task{%
    Сделайте схематичный рисунок энергетических уровней атома водорода
    и отметьте на нём первый (основной) уровень и последующие.
    Сколько различных длин волн может испустить атом водорода,
    находящийся в 3-м возбуждённом состоянии?
    Отметьте все соответствующие переходы на рисунке и укажите,
    при каком переходе (среди отмеченных) энергия излучённого фотона максимальна.
}
\answer{%
    $N = 3{,}0, \text{самая длинная линия}$
}
\solutionspace{150pt}

\tasknumber{8}%
\task{%
    Сколько фотонов испускает за $20\,\text{мин}$ лазер,
    если мощность его излучения $75\,\text{мВт}$?
    Длина волны излучения $500\,\text{нм}$.
    $h = 6{,}626 \cdot 10^{-34}\,\text{Дж}\cdot\text{с}$.
}
\answer{%
    $
        N
            = \frac{E_{\text{общая}}}{E_{\text{одного фотона}}}
            = \frac{Pt}{h\nu} = \frac{Pt}{h \frac c\lambda}
            = \frac{Pt\lambda}{hc}
            = \frac{75\,\text{мВт} \cdot 20\,\text{мин} \cdot 500\,\text{нм}}{6{,}626 \cdot 10^{-34}\,\text{Дж}\cdot\text{с} \cdot 3 \cdot 10^{8}\,\frac{\text{м}}{\text{с}}}
            \approx 2{,}26 \cdot 10^{20}\units{фотонов}
    $
}
\solutionspace{120pt}

\tasknumber{9}%
\task{%
    Какая доля (от начального количества) радиоактивных ядер распадётся через время,
    равное двум периодам полураспада? Ответ выразить в процентах.
}
\answer{%
    \begin{align*}
    N &= N_0 \cdot 2^{- \frac t{T_{1/2}}} \implies
        \frac N{N_0} = 2^{- \frac t{T_{1/2}}}
        = 2^{-2} \approx 0{,}25 \approx 25\% \\
    N_\text{расп.} &= N_0 - N = N_0 - N_0 \cdot 2^{-\frac t{T_{1/2}}}
        = N_0\cbr{1 - 2^{-\frac t{T_{1/2}}}} \implies
        \frac{N_\text{расп.}}{N_0} = 1 - 2^{-\frac t{T_{1/2}}}
        = 1 - 2^{-2} \approx 0{,}75 \approx 75\%
    \end{align*}
}
\solutionspace{150pt}

\tasknumber{10}%
\task{%
    Сколько процентов ядер радиоактивного железа $\ce{^{59}Fe}$
    останется через $91{,}2\,\text{суток}$, если период его полураспада составляет $45{,}6\,\text{суток}$?
}
\answer{%
    \begin{align*}
    N &= N_0 \cdot 2^{-\frac t{T_{1/2}}}
        = 2^{-\frac{91{,}2\,\text{суток}}{45{,}6\,\text{суток}}}
        \approx 0{,}2500 = 25{,}00\%
    \end{align*}
}
\solutionspace{150pt}

\tasknumber{11}%
\task{%
    За $2\,\text{суток}$ от начального количества ядер радиоизотопа осталась четверть.
    Каков период полураспада этого изотопа (ответ приведите в сутках)?
    Какая ещё доля (также от начального количества) распадётся, если подождать ещё столько же?
}
\answer{%
    \begin{align*}
            N &= N_0 \cdot 2^{-\frac t{T_{1/2}}}
            \implies \frac N{N_0} = 2^{-\frac t{T_{1/2}}}
            \implies \frac 1{4} = 2^{-\frac {2\,\text{суток}}{T_{1/2}}}
            \implies 2 = \frac {2\,\text{суток}}{T_{1/2}}
            \implies T_{1/2} = \frac {2\,\text{суток}}2 \approx 1\,\text{суток}.
         \\
            \delta &= \frac{N(t)}{N_0} - \frac{N(2t)}{N_0}
            = 2^{-\frac t{T_{1/2}}} - 2^{-\frac {2t}{T_{1/2}}}
            = 2^{-\frac t{T_{1/2}}}\cbr{1 - 2^{-\frac t{T_{1/2}}}}
            = \frac 1{4} \cdot \cbr{1-\frac 1{4}} \approx 0{,}188
    \end{align*}
}
\solutionspace{150pt}

\tasknumber{12}%
\task{%
    Энергия связи ядра бериллия \ce{^{9}_{4}Be} равна $58{,}2\,\text{МэВ}$.
    Найти дефект массы этого ядра.
    Ответ выразите в а.е.м.
    и кг.
    Скорость света $c = 2{,}998 \cdot 10^{8}\,\frac{\text{м}}{\text{с}}$, элементарный заряд $e = 1{,}6 \cdot 10^{-19}\,\text{Кл}$.
}
\answer{%
    \begin{align*}
    E_\text{св.} &= \Delta m c^2 \implies \\
    \implies
            \Delta m &= \frac {E_\text{св.}}{c^2} = \frac{58{,}2\,\text{МэВ}}{\sqr{2{,}998 \cdot 10^{8}\,\frac{\text{м}}{\text{с}}}}
            = \frac{58{,}2 \cdot 10^6 \cdot 1{,}6 \cdot 10^{-19}\,\text{Дж}}{\sqr{2{,}998 \cdot 10^{8}\,\frac{\text{м}}{\text{с}}}}
            \approx 0{,}1036 \cdot 10^{-27}\,\text{кг} \approx 0{,}0624\,\text{а.е.м.}
    \end{align*}
}

\variantsplitter

\addpersonalvariant{Михаил Перин}

\tasknumber{1}%
\task{%
    Для частицы, движущейся с релятивистской скоростью,
    выразите $E_\text{кин}$ и $v$ через $c$, $E_0$ и $p$,
    где $E_\text{кин}$~--- кинетическая энергия частицы,
    а $E_0$, $p$ и $v$~--- её энергия покоя импульс и скорость.
}
\answer{%
    \begin{align*}
    E_\text{кин}, E_0:\quad&E = E_\text{кин} + E_0 = \frac{E_0}{\sqrt{1 - \frac{v^2}{c^2}}} \implies \sqrt{1 - \frac{v^2}{c^2}} = \frac{E_0}{{E_0} + {E_\text{кин}}} \implies v = c\sqrt{1 - \sqr{\frac{E_0}{{E_0} + {E_\text{кин}}}}} \\
    &p = \frac{mv}{\sqrt{1 - \frac{v^2}{c^2}}} = \frac{E_0}{c^2} \cdot \sqrt{1 - \sqr{\frac{E_0}{{E_0} + {E_\text{кин}}}}} \cdot \frac{{E_\text{кин}} + {E_0}}{E_0} = \frac{E_0}{c^2} \cdot \sqrt{\sqr{\frac{{E_\text{кин}} + {E_0}}{E_0}} - 1}.
    \\
    E_\text{кин}, p:\quad&E_\text{кин} = E - E_0 = mc^2\cbr{\frac 1{\sqrt{1 - \frac{v^2}{c^2}}} - 1}, p = \frac{mv}{\sqrt{1 - \frac{v^2}{c^2}}} \implies \frac{E_\text{кин}}{p} = \frac{\frac 1{\sqrt{1 - \frac{v^2}{c^2}}} - 1}{\sqrt{1 - \frac{v^2}{c^2}}} \implies v = \ldots \\
    &E_0 = E - E_\text{кин} = \frac{E_0}{\sqrt{1 - \frac{v^2}{c^2}}} - E_\text{кин} \implies E_0 = \frac{E_\text{кин}}{\frac 1{\sqrt{1 - \frac{v^2}{c^2}}} - 1} = \ldots \\
    E_\text{кин}, v:\quad&E_\text{кин} = E - E_0 = mc^2\cbr{\frac 1{\sqrt{1 - \frac{v^2}{c^2}}} - 1} \implies m = \frac{E_\text{кин}}{c^2\cbr{\frac 1{\sqrt{1 - \frac{v^2}{c^2}}} - 1}} \\
    &E_0 = mc^2 = \frac{E_\text{кин}}{\frac 1{\sqrt{1 - \frac{v^2}{c^2}}} - 1} \\
    &p = \frac{mv}{\sqrt{1 - \frac{v^2}{c^2}}} = \frac{E_\text{кин}}{c^2\cbr{\frac 1{\sqrt{1 - \frac{v^2}{c^2}}} - 1}} \cdot \frac{v}{\sqrt{1 - \frac{v^2}{c^2}}} = \frac{{E_\text{кин}} v}{c^2\cbr{1 - {\sqrt{1 - \frac{v^2}{c^2}}}}} \\
    E_0, p:\quad&E_0 = mc^2, \quad p = \frac{mv}{\sqrt{1 - \frac{v^2}{c^2}}} \implies \frac{E_0}{p} = \frac{c^2}v{\sqrt{1 - \frac{v^2}{c^2}}} = c\sqrt{\frac{c^2}{v^2} - 1} \\
    &\sqr{\frac{E_0}{pc}} = \frac{c^2}{v^2} - 1 \implies \frac{v^2}{c^2} = \frac 1{1 + \frac{E_0^2}{p^2c^2}} \implies v = \frac c{\sqrt{1 + \frac{E_0^2}{p^2c^2}}} \\
    &{E_\text{кин}} = E - E_0 = \sqrt{E_0^2 + p^2c^2} - E_0 \\
    E_0, v:\quad&E_0 = mc^2 \implies m = \frac{E_0}{c^2} \qquad p = \frac{mv}{\sqrt{1 - \frac{v^2}{c^2}}} = \frac{E_0}{c^2} \cdot \frac{v}{\sqrt{1 - \frac{v^2}{c^2}}} \\
    &E_\text{кин}= mc^2\cbr{\frac 1{\sqrt{1 - \frac{v^2}{c^2}}} - 1} = \frac{E_0}{c^2}\cbr{\frac 1{\sqrt{1 - \frac{v^2}{c^2}}} - 1} \\
    p, v:\quad&p = \frac{mv}{\sqrt{1 - \frac{v^2}{c^2}}} \implies m = \frac p v {\sqrt{1 - \frac{v^2}{c^2}}} \implies E_0 = mc^2 =\frac {pc^2} v {\sqrt{1 - \frac{v^2}{c^2}}} \\
    &E_\text{кин} = mc^2\cbr{\frac 1{\sqrt{1 - \frac{v^2}{c^2}}} - 1} = \frac p v {\sqrt{1 - \frac{v^2}{c^2}}}\cbr{\frac 1{\sqrt{1 - \frac{v^2}{c^2}}} - 1} = \frac p v \cbr{1 - {\sqrt{1 - \frac{v^2}{c^2}}}}
    \end{align*}
}
\solutionspace{200pt}

\tasknumber{2}%
\task{%
    Протон движется со скоростью $0{,}7\,c$, где $c$~--- скорость света в вакууме.
    Каково при этом отношение кинетической энергии частицы $E_\text{кин.}$ к его энергии покоя $E_0$?
}
\answer{%
    \begin{align*}
    E &= \frac{E_0}{\sqrt{1 - \frac{v^2}{c^2}}}
            \implies \frac E{E_0}
                = \frac 1{\sqrt{1 - \frac{v^2}{c^2}}}
                = \frac 1{\sqrt{1 - \sqr{0{,}7}}}
                \approx 1{,}400,
         \\
        E_{\text{кин}} &= E - E_0
            \implies \frac{E_{\text{кин}}}{E_0}
                = \frac E{E_0} - 1
                = \frac 1{\sqrt{1 - \frac{v^2}{c^2}}} - 1
                = \frac 1{\sqrt{1 - \sqr{0{,}7}}} - 1
                \approx 0{,}400.
    \end{align*}
}
\solutionspace{150pt}

\tasknumber{3}%
\task{%
    Электрон движется со скоростью $0{,}75\,c$, где $c$~--- скорость света в вакууме.
    Определите его полную энергию (в ответе приведите формулу и укажите численное значение).
}
\answer{%
    \begin{align*}
    E &= \frac{mc^2}{\sqrt{1 - \frac{v^2}{c^2}}}
            \approx \frac{9{,}1 \cdot 10^{-31}\,\text{кг} \cdot \sqr{3 \cdot 10^{8}\,\frac{\text{м}}{\text{с}}}}{\sqrt{1 - 0{,}75^2}}
            \approx 0{,}124 \cdot 10^{-12}\,\text{Дж},
         \\
        E_{\text{кин}} &= \frac{mc^2}{\sqrt{1 - \frac{v^2}{c^2}}} - mc^2
            = mc^2 \cbr{\frac 1{\sqrt{1 - \frac{v^2}{c^2}}} - 1} \approx \\
            &\approx \cbr{9{,}1 \cdot 10^{-31}\,\text{кг} \cdot \sqr{3 \cdot 10^{8}\,\frac{\text{м}}{\text{с}}}}
            \cdot \cbr{\frac 1{\sqrt{1 - 0{,}75^2}} - 1}
            \approx 0{,}042 \cdot 10^{-12}\,\text{Дж},
         \\
        p &= \frac{mv}{\sqrt{1 - \frac{v^2}{c^2}}}
            \approx \frac{9{,}1 \cdot 10^{-31}\,\text{кг} \cdot 0{,}75 \cdot 3 \cdot 10^{8}\,\frac{\text{м}}{\text{с}}}{\sqrt{1 - 0{,}75^2}}
            \approx 0{,}310 \cdot 10^{-21}\,\frac{\text{кг}\cdot\text{м}}{\text{с}}.
    \end{align*}
}
\solutionspace{150pt}

\tasknumber{4}%
\task{%
    При какой скорости движения (в км/ч) релятивистское сокращение длины движущегося тела
    составит 10\%?
}
\answer{%
    \begin{align*}
    l_0 &= \frac l{\sqrt{1 - \frac{v^2}{c^2}}}
        \implies 1 - \frac{v^2}{c^2} = \sqr{\frac l{l_0}}
        \implies \frac v c = \sqrt{1 - \sqr{\frac l{l_0}}} \implies
         \\
        \implies v &= c\sqrt{1 - \sqr{\frac l{l_0}}}
        = 3 \cdot 10^{8}\,\frac{\text{м}}{\text{с}} \cdot \sqrt{1 - \sqr{\frac {l_0 - 0{,}10l_0}{l_0}}}
        = 3 \cdot 10^{8}\,\frac{\text{м}}{\text{с}} \cdot \sqrt{1 - \sqr{1 - 0{,}10}} \approx  \\
        &\approx 0{,}436c
        \approx 130{,}8 \cdot 10^{6}\,\frac{\text{м}}{\text{с}}
        \approx 471 \cdot 10^{6}\,\frac{\text{км}}{\text{ч}}.
    \end{align*}
}
\solutionspace{150pt}

\tasknumber{5}%
\task{%
    При переходе электрона в атоме с одной стационарной орбиты на другую
    излучается фотон с энергией $4{,}04 \cdot 10^{-19}\,\text{Дж}$.
    Какова длина волны этой линии спектра?
    Постоянная Планка $h = 6{,}626 \cdot 10^{-34}\,\text{Дж}\cdot\text{с}$, скорость света $c = 3 \cdot 10^{8}\,\frac{\text{м}}{\text{с}}$.
}
\answer{%
    $
        E = h\nu = h \frac c\lambda
        \implies \lambda = \frac{hc}E
            = \frac{6{,}626 \cdot 10^{-34}\,\text{Дж}\cdot\text{с} \cdot {3 \cdot 10^{8}\,\frac{\text{м}}{\text{с}}}}{4{,}04 \cdot 10^{-19}\,\text{Дж}}
            = 492{,}03\,\text{нм}.
    $
}
\solutionspace{150pt}

\tasknumber{6}%
\task{%
    Излучение какой длины волны поглотил атом водорода, если полная энергия в атоме увеличилась на $2 \cdot 10^{-19}\,\text{Дж}$?
    Постоянная Планка $h = 6{,}626 \cdot 10^{-34}\,\text{Дж}\cdot\text{с}$, скорость света $c = 3 \cdot 10^{8}\,\frac{\text{м}}{\text{с}}$.
}
\answer{%
    $
        E = h\nu = h \frac c\lambda
        \implies \lambda = \frac{hc}E
            = \frac{6{,}626 \cdot 10^{-34}\,\text{Дж}\cdot\text{с} \cdot {3 \cdot 10^{8}\,\frac{\text{м}}{\text{с}}}}{2 \cdot 10^{-19}\,\text{Дж}}
            = 994\,\text{нм}.
    $
}
\solutionspace{150pt}

\tasknumber{7}%
\task{%
    Сделайте схематичный рисунок энергетических уровней атома водорода
    и отметьте на нём первый (основной) уровень и последующие.
    Сколько различных длин волн может испустить атом водорода,
    находящийся в 5-м возбуждённом состоянии?
    Отметьте все соответствующие переходы на рисунке и укажите,
    при каком переходе (среди отмеченных) энергия излучённого фотона максимальна.
}
\answer{%
    $N = 10{,}0, \text{самая длинная линия}$
}
\solutionspace{150pt}

\tasknumber{8}%
\task{%
    Сколько фотонов испускает за $5\,\text{мин}$ лазер,
    если мощность его излучения $200\,\text{мВт}$?
    Длина волны излучения $750\,\text{нм}$.
    $h = 6{,}626 \cdot 10^{-34}\,\text{Дж}\cdot\text{с}$.
}
\answer{%
    $
        N
            = \frac{E_{\text{общая}}}{E_{\text{одного фотона}}}
            = \frac{Pt}{h\nu} = \frac{Pt}{h \frac c\lambda}
            = \frac{Pt\lambda}{hc}
            = \frac{200\,\text{мВт} \cdot 5\,\text{мин} \cdot 750\,\text{нм}}{6{,}626 \cdot 10^{-34}\,\text{Дж}\cdot\text{с} \cdot 3 \cdot 10^{8}\,\frac{\text{м}}{\text{с}}}
            \approx 2{,}26 \cdot 10^{20}\units{фотонов}
    $
}
\solutionspace{120pt}

\tasknumber{9}%
\task{%
    Какая доля (от начального количества) радиоактивных ядер распадётся через время,
    равное двум периодам полураспада? Ответ выразить в процентах.
}
\answer{%
    \begin{align*}
    N &= N_0 \cdot 2^{- \frac t{T_{1/2}}} \implies
        \frac N{N_0} = 2^{- \frac t{T_{1/2}}}
        = 2^{-2} \approx 0{,}25 \approx 25\% \\
    N_\text{расп.} &= N_0 - N = N_0 - N_0 \cdot 2^{-\frac t{T_{1/2}}}
        = N_0\cbr{1 - 2^{-\frac t{T_{1/2}}}} \implies
        \frac{N_\text{расп.}}{N_0} = 1 - 2^{-\frac t{T_{1/2}}}
        = 1 - 2^{-2} \approx 0{,}75 \approx 75\%
    \end{align*}
}
\solutionspace{150pt}

\tasknumber{10}%
\task{%
    Сколько процентов ядер радиоактивного железа $\ce{^{59}Fe}$
    останется через $182{,}4\,\text{суток}$, если период его полураспада составляет $45{,}6\,\text{суток}$?
}
\answer{%
    \begin{align*}
    N &= N_0 \cdot 2^{-\frac t{T_{1/2}}}
        = 2^{-\frac{182{,}4\,\text{суток}}{45{,}6\,\text{суток}}}
        \approx 0{,}0625 = 6{,}25\%
    \end{align*}
}
\solutionspace{150pt}

\tasknumber{11}%
\task{%
    За $2\,\text{суток}$ от начального количества ядер радиоизотопа осталась одна шестнадцатая.
    Каков период полураспада этого изотопа (ответ приведите в сутках)?
    Какая ещё доля (также от начального количества) распадётся, если подождать ещё столько же?
}
\answer{%
    \begin{align*}
            N &= N_0 \cdot 2^{-\frac t{T_{1/2}}}
            \implies \frac N{N_0} = 2^{-\frac t{T_{1/2}}}
            \implies \frac 1{16} = 2^{-\frac {2\,\text{суток}}{T_{1/2}}}
            \implies 4 = \frac {2\,\text{суток}}{T_{1/2}}
            \implies T_{1/2} = \frac {2\,\text{суток}}4 \approx 0{,}50\,\text{суток}.
         \\
            \delta &= \frac{N(t)}{N_0} - \frac{N(2t)}{N_0}
            = 2^{-\frac t{T_{1/2}}} - 2^{-\frac {2t}{T_{1/2}}}
            = 2^{-\frac t{T_{1/2}}}\cbr{1 - 2^{-\frac t{T_{1/2}}}}
            = \frac 1{16} \cdot \cbr{1-\frac 1{16}} \approx 0{,}059
    \end{align*}
}
\solutionspace{150pt}

\tasknumber{12}%
\task{%
    Энергия связи ядра лития \ce{^{6}_{3}Li} равна $31{,}99\,\text{МэВ}$.
    Найти дефект массы этого ядра.
    Ответ выразите в а.е.м.
    и кг.
    Скорость света $c = 2{,}998 \cdot 10^{8}\,\frac{\text{м}}{\text{с}}$, элементарный заряд $e = 1{,}6 \cdot 10^{-19}\,\text{Кл}$.
}
\answer{%
    \begin{align*}
    E_\text{св.} &= \Delta m c^2 \implies \\
    \implies
            \Delta m &= \frac {E_\text{св.}}{c^2} = \frac{31{,}99\,\text{МэВ}}{\sqr{2{,}998 \cdot 10^{8}\,\frac{\text{м}}{\text{с}}}}
            = \frac{31{,}99 \cdot 10^6 \cdot 1{,}6 \cdot 10^{-19}\,\text{Дж}}{\sqr{2{,}998 \cdot 10^{8}\,\frac{\text{м}}{\text{с}}}}
            \approx 56{,}95 \cdot 10^{-30}\,\text{кг} \approx 0{,}03429\,\text{а.е.м.}
    \end{align*}
}

\variantsplitter

\addpersonalvariant{Егор Подуровский}

\tasknumber{1}%
\task{%
    Для частицы, движущейся с релятивистской скоростью,
    выразите $E_\text{кин}$ и $v$ через $c$, $p$ и $E_0$,
    где $E_\text{кин}$~--- кинетическая энергия частицы,
    а $E_0$, $p$ и $v$~--- её энергия покоя импульс и скорость.
}
\answer{%
    \begin{align*}
    E_\text{кин}, E_0:\quad&E = E_\text{кин} + E_0 = \frac{E_0}{\sqrt{1 - \frac{v^2}{c^2}}} \implies \sqrt{1 - \frac{v^2}{c^2}} = \frac{E_0}{{E_0} + {E_\text{кин}}} \implies v = c\sqrt{1 - \sqr{\frac{E_0}{{E_0} + {E_\text{кин}}}}} \\
    &p = \frac{mv}{\sqrt{1 - \frac{v^2}{c^2}}} = \frac{E_0}{c^2} \cdot \sqrt{1 - \sqr{\frac{E_0}{{E_0} + {E_\text{кин}}}}} \cdot \frac{{E_\text{кин}} + {E_0}}{E_0} = \frac{E_0}{c^2} \cdot \sqrt{\sqr{\frac{{E_\text{кин}} + {E_0}}{E_0}} - 1}.
    \\
    E_\text{кин}, p:\quad&E_\text{кин} = E - E_0 = mc^2\cbr{\frac 1{\sqrt{1 - \frac{v^2}{c^2}}} - 1}, p = \frac{mv}{\sqrt{1 - \frac{v^2}{c^2}}} \implies \frac{E_\text{кин}}{p} = \frac{\frac 1{\sqrt{1 - \frac{v^2}{c^2}}} - 1}{\sqrt{1 - \frac{v^2}{c^2}}} \implies v = \ldots \\
    &E_0 = E - E_\text{кин} = \frac{E_0}{\sqrt{1 - \frac{v^2}{c^2}}} - E_\text{кин} \implies E_0 = \frac{E_\text{кин}}{\frac 1{\sqrt{1 - \frac{v^2}{c^2}}} - 1} = \ldots \\
    E_\text{кин}, v:\quad&E_\text{кин} = E - E_0 = mc^2\cbr{\frac 1{\sqrt{1 - \frac{v^2}{c^2}}} - 1} \implies m = \frac{E_\text{кин}}{c^2\cbr{\frac 1{\sqrt{1 - \frac{v^2}{c^2}}} - 1}} \\
    &E_0 = mc^2 = \frac{E_\text{кин}}{\frac 1{\sqrt{1 - \frac{v^2}{c^2}}} - 1} \\
    &p = \frac{mv}{\sqrt{1 - \frac{v^2}{c^2}}} = \frac{E_\text{кин}}{c^2\cbr{\frac 1{\sqrt{1 - \frac{v^2}{c^2}}} - 1}} \cdot \frac{v}{\sqrt{1 - \frac{v^2}{c^2}}} = \frac{{E_\text{кин}} v}{c^2\cbr{1 - {\sqrt{1 - \frac{v^2}{c^2}}}}} \\
    E_0, p:\quad&E_0 = mc^2, \quad p = \frac{mv}{\sqrt{1 - \frac{v^2}{c^2}}} \implies \frac{E_0}{p} = \frac{c^2}v{\sqrt{1 - \frac{v^2}{c^2}}} = c\sqrt{\frac{c^2}{v^2} - 1} \\
    &\sqr{\frac{E_0}{pc}} = \frac{c^2}{v^2} - 1 \implies \frac{v^2}{c^2} = \frac 1{1 + \frac{E_0^2}{p^2c^2}} \implies v = \frac c{\sqrt{1 + \frac{E_0^2}{p^2c^2}}} \\
    &{E_\text{кин}} = E - E_0 = \sqrt{E_0^2 + p^2c^2} - E_0 \\
    E_0, v:\quad&E_0 = mc^2 \implies m = \frac{E_0}{c^2} \qquad p = \frac{mv}{\sqrt{1 - \frac{v^2}{c^2}}} = \frac{E_0}{c^2} \cdot \frac{v}{\sqrt{1 - \frac{v^2}{c^2}}} \\
    &E_\text{кин}= mc^2\cbr{\frac 1{\sqrt{1 - \frac{v^2}{c^2}}} - 1} = \frac{E_0}{c^2}\cbr{\frac 1{\sqrt{1 - \frac{v^2}{c^2}}} - 1} \\
    p, v:\quad&p = \frac{mv}{\sqrt{1 - \frac{v^2}{c^2}}} \implies m = \frac p v {\sqrt{1 - \frac{v^2}{c^2}}} \implies E_0 = mc^2 =\frac {pc^2} v {\sqrt{1 - \frac{v^2}{c^2}}} \\
    &E_\text{кин} = mc^2\cbr{\frac 1{\sqrt{1 - \frac{v^2}{c^2}}} - 1} = \frac p v {\sqrt{1 - \frac{v^2}{c^2}}}\cbr{\frac 1{\sqrt{1 - \frac{v^2}{c^2}}} - 1} = \frac p v \cbr{1 - {\sqrt{1 - \frac{v^2}{c^2}}}}
    \end{align*}
}
\solutionspace{200pt}

\tasknumber{2}%
\task{%
    Позитрон движется со скоростью $0{,}6\,c$, где $c$~--- скорость света в вакууме.
    Каково при этом отношение кинетической энергии частицы $E_\text{кин.}$ к его энергии покоя $E_0$?
}
\answer{%
    \begin{align*}
    E &= \frac{E_0}{\sqrt{1 - \frac{v^2}{c^2}}}
            \implies \frac E{E_0}
                = \frac 1{\sqrt{1 - \frac{v^2}{c^2}}}
                = \frac 1{\sqrt{1 - \sqr{0{,}6}}}
                \approx 1{,}250,
         \\
        E_{\text{кин}} &= E - E_0
            \implies \frac{E_{\text{кин}}}{E_0}
                = \frac E{E_0} - 1
                = \frac 1{\sqrt{1 - \frac{v^2}{c^2}}} - 1
                = \frac 1{\sqrt{1 - \sqr{0{,}6}}} - 1
                \approx 0{,}250.
    \end{align*}
}
\solutionspace{150pt}

\tasknumber{3}%
\task{%
    Протон движется со скоростью $0{,}65\,c$, где $c$~--- скорость света в вакууме.
    Определите его кинетическую энергию (в ответе приведите формулу и укажите численное значение).
}
\answer{%
    \begin{align*}
    E &= \frac{mc^2}{\sqrt{1 - \frac{v^2}{c^2}}}
            \approx \frac{1{,}673 \cdot 10^{-27}\,\text{кг} \cdot \sqr{3 \cdot 10^{8}\,\frac{\text{м}}{\text{с}}}}{\sqrt{1 - 0{,}65^2}}
            \approx 198{,}091 \cdot 10^{-12}\,\text{Дж},
         \\
        E_{\text{кин}} &= \frac{mc^2}{\sqrt{1 - \frac{v^2}{c^2}}} - mc^2
            = mc^2 \cbr{\frac 1{\sqrt{1 - \frac{v^2}{c^2}}} - 1} \approx \\
            &\approx \cbr{1{,}673 \cdot 10^{-27}\,\text{кг} \cdot \sqr{3 \cdot 10^{8}\,\frac{\text{м}}{\text{с}}}}
            \cdot \cbr{\frac 1{\sqrt{1 - 0{,}65^2}} - 1}
            \approx 47{,}555 \cdot 10^{-12}\,\text{Дж},
         \\
        p &= \frac{mv}{\sqrt{1 - \frac{v^2}{c^2}}}
            \approx \frac{1{,}673 \cdot 10^{-27}\,\text{кг} \cdot 0{,}65 \cdot 3 \cdot 10^{8}\,\frac{\text{м}}{\text{с}}}{\sqrt{1 - 0{,}65^2}}
            \approx 429{,}196 \cdot 10^{-21}\,\frac{\text{кг}\cdot\text{м}}{\text{с}}.
    \end{align*}
}
\solutionspace{150pt}

\tasknumber{4}%
\task{%
    При какой скорости движения (в км/ч) релятивистское сокращение длины движущегося тела
    составит 50\%?
}
\answer{%
    \begin{align*}
    l_0 &= \frac l{\sqrt{1 - \frac{v^2}{c^2}}}
        \implies 1 - \frac{v^2}{c^2} = \sqr{\frac l{l_0}}
        \implies \frac v c = \sqrt{1 - \sqr{\frac l{l_0}}} \implies
         \\
        \implies v &= c\sqrt{1 - \sqr{\frac l{l_0}}}
        = 3 \cdot 10^{8}\,\frac{\text{м}}{\text{с}} \cdot \sqrt{1 - \sqr{\frac {l_0 - 0{,}50l_0}{l_0}}}
        = 3 \cdot 10^{8}\,\frac{\text{м}}{\text{с}} \cdot \sqrt{1 - \sqr{1 - 0{,}50}} \approx  \\
        &\approx 0{,}866c
        \approx 260 \cdot 10^{6}\,\frac{\text{м}}{\text{с}}
        \approx 935 \cdot 10^{6}\,\frac{\text{км}}{\text{ч}}.
    \end{align*}
}
\solutionspace{150pt}

\tasknumber{5}%
\task{%
    При переходе электрона в атоме с одной стационарной орбиты на другую
    излучается фотон с энергией $5{,}05 \cdot 10^{-19}\,\text{Дж}$.
    Какова длина волны этой линии спектра?
    Постоянная Планка $h = 6{,}626 \cdot 10^{-34}\,\text{Дж}\cdot\text{с}$, скорость света $c = 3 \cdot 10^{8}\,\frac{\text{м}}{\text{с}}$.
}
\answer{%
    $
        E = h\nu = h \frac c\lambda
        \implies \lambda = \frac{hc}E
            = \frac{6{,}626 \cdot 10^{-34}\,\text{Дж}\cdot\text{с} \cdot {3 \cdot 10^{8}\,\frac{\text{м}}{\text{с}}}}{5{,}05 \cdot 10^{-19}\,\text{Дж}}
            = 393{,}62\,\text{нм}.
    $
}
\solutionspace{150pt}

\tasknumber{6}%
\task{%
    Излучение какой длины волны поглотил атом водорода, если полная энергия в атоме увеличилась на $3 \cdot 10^{-19}\,\text{Дж}$?
    Постоянная Планка $h = 6{,}626 \cdot 10^{-34}\,\text{Дж}\cdot\text{с}$, скорость света $c = 3 \cdot 10^{8}\,\frac{\text{м}}{\text{с}}$.
}
\answer{%
    $
        E = h\nu = h \frac c\lambda
        \implies \lambda = \frac{hc}E
            = \frac{6{,}626 \cdot 10^{-34}\,\text{Дж}\cdot\text{с} \cdot {3 \cdot 10^{8}\,\frac{\text{м}}{\text{с}}}}{3 \cdot 10^{-19}\,\text{Дж}}
            = 663\,\text{нм}.
    $
}
\solutionspace{150pt}

\tasknumber{7}%
\task{%
    Сделайте схематичный рисунок энергетических уровней атома водорода
    и отметьте на нём первый (основной) уровень и последующие.
    Сколько различных длин волн может испустить атом водорода,
    находящийся в 4-м возбуждённом состоянии?
    Отметьте все соответствующие переходы на рисунке и укажите,
    при каком переходе (среди отмеченных) длина волны излучённого фотона минимальна.
}
\answer{%
    $N = 6{,}0, \text{самая длинная линия}$
}
\solutionspace{150pt}

\tasknumber{8}%
\task{%
    Сколько фотонов испускает за $120\,\text{мин}$ лазер,
    если мощность его излучения $40\,\text{мВт}$?
    Длина волны излучения $750\,\text{нм}$.
    $h = 6{,}626 \cdot 10^{-34}\,\text{Дж}\cdot\text{с}$.
}
\answer{%
    $
        N
            = \frac{E_{\text{общая}}}{E_{\text{одного фотона}}}
            = \frac{Pt}{h\nu} = \frac{Pt}{h \frac c\lambda}
            = \frac{Pt\lambda}{hc}
            = \frac{40\,\text{мВт} \cdot 120\,\text{мин} \cdot 750\,\text{нм}}{6{,}626 \cdot 10^{-34}\,\text{Дж}\cdot\text{с} \cdot 3 \cdot 10^{8}\,\frac{\text{м}}{\text{с}}}
            \approx 10{,}87 \cdot 10^{20}\units{фотонов}
    $
}
\solutionspace{120pt}

\tasknumber{9}%
\task{%
    Какая доля (от начального количества) радиоактивных ядер останется через время,
    равное четырём периодам полураспада? Ответ выразить в процентах.
}
\answer{%
    \begin{align*}
    N &= N_0 \cdot 2^{- \frac t{T_{1/2}}} \implies
        \frac N{N_0} = 2^{- \frac t{T_{1/2}}}
        = 2^{-4} \approx 0{,}06 \approx 6\% \\
    N_\text{расп.} &= N_0 - N = N_0 - N_0 \cdot 2^{-\frac t{T_{1/2}}}
        = N_0\cbr{1 - 2^{-\frac t{T_{1/2}}}} \implies
        \frac{N_\text{расп.}}{N_0} = 1 - 2^{-\frac t{T_{1/2}}}
        = 1 - 2^{-4} \approx 0{,}94 \approx 94\%
    \end{align*}
}
\solutionspace{150pt}

\tasknumber{10}%
\task{%
    Сколько процентов ядер радиоактивного железа $\ce{^{59}Fe}$
    останется через $91{,}2\,\text{суток}$, если период его полураспада составляет $45{,}6\,\text{суток}$?
}
\answer{%
    \begin{align*}
    N &= N_0 \cdot 2^{-\frac t{T_{1/2}}}
        = 2^{-\frac{91{,}2\,\text{суток}}{45{,}6\,\text{суток}}}
        \approx 0{,}2500 = 25{,}00\%
    \end{align*}
}
\solutionspace{150pt}

\tasknumber{11}%
\task{%
    За $5\,\text{суток}$ от начального количества ядер радиоизотопа осталась половина.
    Каков период полураспада этого изотопа (ответ приведите в сутках)?
    Какая ещё доля (также от начального количества) распадётся, если подождать ещё столько же?
}
\answer{%
    \begin{align*}
            N &= N_0 \cdot 2^{-\frac t{T_{1/2}}}
            \implies \frac N{N_0} = 2^{-\frac t{T_{1/2}}}
            \implies \frac 1{2} = 2^{-\frac {5\,\text{суток}}{T_{1/2}}}
            \implies 1 = \frac {5\,\text{суток}}{T_{1/2}}
            \implies T_{1/2} = \frac {5\,\text{суток}}1 \approx 5\,\text{суток}.
         \\
            \delta &= \frac{N(t)}{N_0} - \frac{N(2t)}{N_0}
            = 2^{-\frac t{T_{1/2}}} - 2^{-\frac {2t}{T_{1/2}}}
            = 2^{-\frac t{T_{1/2}}}\cbr{1 - 2^{-\frac t{T_{1/2}}}}
            = \frac 1{2} \cdot \cbr{1-\frac 1{2}} \approx 0{,}250
    \end{align*}
}
\solutionspace{150pt}

\tasknumber{12}%
\task{%
    Энергия связи ядра лития \ce{^{6}_{3}Li} равна $31{,}99\,\text{МэВ}$.
    Найти дефект массы этого ядра.
    Ответ выразите в а.е.м.
    и кг.
    Скорость света $c = 2{,}998 \cdot 10^{8}\,\frac{\text{м}}{\text{с}}$, элементарный заряд $e = 1{,}6 \cdot 10^{-19}\,\text{Кл}$.
}
\answer{%
    \begin{align*}
    E_\text{св.} &= \Delta m c^2 \implies \\
    \implies
            \Delta m &= \frac {E_\text{св.}}{c^2} = \frac{31{,}99\,\text{МэВ}}{\sqr{2{,}998 \cdot 10^{8}\,\frac{\text{м}}{\text{с}}}}
            = \frac{31{,}99 \cdot 10^6 \cdot 1{,}6 \cdot 10^{-19}\,\text{Дж}}{\sqr{2{,}998 \cdot 10^{8}\,\frac{\text{м}}{\text{с}}}}
            \approx 56{,}95 \cdot 10^{-30}\,\text{кг} \approx 0{,}03429\,\text{а.е.м.}
    \end{align*}
}

\variantsplitter

\addpersonalvariant{Роман Прибылов}

\tasknumber{1}%
\task{%
    Для частицы, движущейся с релятивистской скоростью,
    выразите $E_\text{кин}$ и $p$ через $c$, $E_0$ и $v$,
    где $E_\text{кин}$~--- кинетическая энергия частицы,
    а $E_0$, $p$ и $v$~--- её энергия покоя импульс и скорость.
}
\answer{%
    \begin{align*}
    E_\text{кин}, E_0:\quad&E = E_\text{кин} + E_0 = \frac{E_0}{\sqrt{1 - \frac{v^2}{c^2}}} \implies \sqrt{1 - \frac{v^2}{c^2}} = \frac{E_0}{{E_0} + {E_\text{кин}}} \implies v = c\sqrt{1 - \sqr{\frac{E_0}{{E_0} + {E_\text{кин}}}}} \\
    &p = \frac{mv}{\sqrt{1 - \frac{v^2}{c^2}}} = \frac{E_0}{c^2} \cdot \sqrt{1 - \sqr{\frac{E_0}{{E_0} + {E_\text{кин}}}}} \cdot \frac{{E_\text{кин}} + {E_0}}{E_0} = \frac{E_0}{c^2} \cdot \sqrt{\sqr{\frac{{E_\text{кин}} + {E_0}}{E_0}} - 1}.
    \\
    E_\text{кин}, p:\quad&E_\text{кин} = E - E_0 = mc^2\cbr{\frac 1{\sqrt{1 - \frac{v^2}{c^2}}} - 1}, p = \frac{mv}{\sqrt{1 - \frac{v^2}{c^2}}} \implies \frac{E_\text{кин}}{p} = \frac{\frac 1{\sqrt{1 - \frac{v^2}{c^2}}} - 1}{\sqrt{1 - \frac{v^2}{c^2}}} \implies v = \ldots \\
    &E_0 = E - E_\text{кин} = \frac{E_0}{\sqrt{1 - \frac{v^2}{c^2}}} - E_\text{кин} \implies E_0 = \frac{E_\text{кин}}{\frac 1{\sqrt{1 - \frac{v^2}{c^2}}} - 1} = \ldots \\
    E_\text{кин}, v:\quad&E_\text{кин} = E - E_0 = mc^2\cbr{\frac 1{\sqrt{1 - \frac{v^2}{c^2}}} - 1} \implies m = \frac{E_\text{кин}}{c^2\cbr{\frac 1{\sqrt{1 - \frac{v^2}{c^2}}} - 1}} \\
    &E_0 = mc^2 = \frac{E_\text{кин}}{\frac 1{\sqrt{1 - \frac{v^2}{c^2}}} - 1} \\
    &p = \frac{mv}{\sqrt{1 - \frac{v^2}{c^2}}} = \frac{E_\text{кин}}{c^2\cbr{\frac 1{\sqrt{1 - \frac{v^2}{c^2}}} - 1}} \cdot \frac{v}{\sqrt{1 - \frac{v^2}{c^2}}} = \frac{{E_\text{кин}} v}{c^2\cbr{1 - {\sqrt{1 - \frac{v^2}{c^2}}}}} \\
    E_0, p:\quad&E_0 = mc^2, \quad p = \frac{mv}{\sqrt{1 - \frac{v^2}{c^2}}} \implies \frac{E_0}{p} = \frac{c^2}v{\sqrt{1 - \frac{v^2}{c^2}}} = c\sqrt{\frac{c^2}{v^2} - 1} \\
    &\sqr{\frac{E_0}{pc}} = \frac{c^2}{v^2} - 1 \implies \frac{v^2}{c^2} = \frac 1{1 + \frac{E_0^2}{p^2c^2}} \implies v = \frac c{\sqrt{1 + \frac{E_0^2}{p^2c^2}}} \\
    &{E_\text{кин}} = E - E_0 = \sqrt{E_0^2 + p^2c^2} - E_0 \\
    E_0, v:\quad&E_0 = mc^2 \implies m = \frac{E_0}{c^2} \qquad p = \frac{mv}{\sqrt{1 - \frac{v^2}{c^2}}} = \frac{E_0}{c^2} \cdot \frac{v}{\sqrt{1 - \frac{v^2}{c^2}}} \\
    &E_\text{кин}= mc^2\cbr{\frac 1{\sqrt{1 - \frac{v^2}{c^2}}} - 1} = \frac{E_0}{c^2}\cbr{\frac 1{\sqrt{1 - \frac{v^2}{c^2}}} - 1} \\
    p, v:\quad&p = \frac{mv}{\sqrt{1 - \frac{v^2}{c^2}}} \implies m = \frac p v {\sqrt{1 - \frac{v^2}{c^2}}} \implies E_0 = mc^2 =\frac {pc^2} v {\sqrt{1 - \frac{v^2}{c^2}}} \\
    &E_\text{кин} = mc^2\cbr{\frac 1{\sqrt{1 - \frac{v^2}{c^2}}} - 1} = \frac p v {\sqrt{1 - \frac{v^2}{c^2}}}\cbr{\frac 1{\sqrt{1 - \frac{v^2}{c^2}}} - 1} = \frac p v \cbr{1 - {\sqrt{1 - \frac{v^2}{c^2}}}}
    \end{align*}
}
\solutionspace{200pt}

\tasknumber{2}%
\task{%
    Электрон движется со скоростью $0{,}7\,c$, где $c$~--- скорость света в вакууме.
    Каково при этом отношение кинетической энергии частицы $E_\text{кин.}$ к его энергии покоя $E_0$?
}
\answer{%
    \begin{align*}
    E &= \frac{E_0}{\sqrt{1 - \frac{v^2}{c^2}}}
            \implies \frac E{E_0}
                = \frac 1{\sqrt{1 - \frac{v^2}{c^2}}}
                = \frac 1{\sqrt{1 - \sqr{0{,}7}}}
                \approx 1{,}400,
         \\
        E_{\text{кин}} &= E - E_0
            \implies \frac{E_{\text{кин}}}{E_0}
                = \frac E{E_0} - 1
                = \frac 1{\sqrt{1 - \frac{v^2}{c^2}}} - 1
                = \frac 1{\sqrt{1 - \sqr{0{,}7}}} - 1
                \approx 0{,}400.
    \end{align*}
}
\solutionspace{150pt}

\tasknumber{3}%
\task{%
    Электрон движется со скоростью $0{,}65\,c$, где $c$~--- скорость света в вакууме.
    Определите его полную энергию (в ответе приведите формулу и укажите численное значение).
}
\answer{%
    \begin{align*}
    E &= \frac{mc^2}{\sqrt{1 - \frac{v^2}{c^2}}}
            \approx \frac{9{,}1 \cdot 10^{-31}\,\text{кг} \cdot \sqr{3 \cdot 10^{8}\,\frac{\text{м}}{\text{с}}}}{\sqrt{1 - 0{,}65^2}}
            \approx 0{,}108 \cdot 10^{-12}\,\text{Дж},
         \\
        E_{\text{кин}} &= \frac{mc^2}{\sqrt{1 - \frac{v^2}{c^2}}} - mc^2
            = mc^2 \cbr{\frac 1{\sqrt{1 - \frac{v^2}{c^2}}} - 1} \approx \\
            &\approx \cbr{9{,}1 \cdot 10^{-31}\,\text{кг} \cdot \sqr{3 \cdot 10^{8}\,\frac{\text{м}}{\text{с}}}}
            \cdot \cbr{\frac 1{\sqrt{1 - 0{,}65^2}} - 1}
            \approx 0{,}026 \cdot 10^{-12}\,\text{Дж},
         \\
        p &= \frac{mv}{\sqrt{1 - \frac{v^2}{c^2}}}
            \approx \frac{9{,}1 \cdot 10^{-31}\,\text{кг} \cdot 0{,}65 \cdot 3 \cdot 10^{8}\,\frac{\text{м}}{\text{с}}}{\sqrt{1 - 0{,}65^2}}
            \approx 0{,}234 \cdot 10^{-21}\,\frac{\text{кг}\cdot\text{м}}{\text{с}}.
    \end{align*}
}
\solutionspace{150pt}

\tasknumber{4}%
\task{%
    При какой скорости движения (в км/ч) релятивистское сокращение длины движущегося тела
    составит 50\%?
}
\answer{%
    \begin{align*}
    l_0 &= \frac l{\sqrt{1 - \frac{v^2}{c^2}}}
        \implies 1 - \frac{v^2}{c^2} = \sqr{\frac l{l_0}}
        \implies \frac v c = \sqrt{1 - \sqr{\frac l{l_0}}} \implies
         \\
        \implies v &= c\sqrt{1 - \sqr{\frac l{l_0}}}
        = 3 \cdot 10^{8}\,\frac{\text{м}}{\text{с}} \cdot \sqrt{1 - \sqr{\frac {l_0 - 0{,}50l_0}{l_0}}}
        = 3 \cdot 10^{8}\,\frac{\text{м}}{\text{с}} \cdot \sqrt{1 - \sqr{1 - 0{,}50}} \approx  \\
        &\approx 0{,}866c
        \approx 260 \cdot 10^{6}\,\frac{\text{м}}{\text{с}}
        \approx 935 \cdot 10^{6}\,\frac{\text{км}}{\text{ч}}.
    \end{align*}
}
\solutionspace{150pt}

\tasknumber{5}%
\task{%
    При переходе электрона в атоме с одной стационарной орбиты на другую
    излучается фотон с энергией $0{,}55 \cdot 10^{-19}\,\text{Дж}$.
    Какова длина волны этой линии спектра?
    Постоянная Планка $h = 6{,}626 \cdot 10^{-34}\,\text{Дж}\cdot\text{с}$, скорость света $c = 3 \cdot 10^{8}\,\frac{\text{м}}{\text{с}}$.
}
\answer{%
    $
        E = h\nu = h \frac c\lambda
        \implies \lambda = \frac{hc}E
            = \frac{6{,}626 \cdot 10^{-34}\,\text{Дж}\cdot\text{с} \cdot {3 \cdot 10^{8}\,\frac{\text{м}}{\text{с}}}}{0{,}55 \cdot 10^{-19}\,\text{Дж}}
            = 3614\,\text{нм}.
    $
}
\solutionspace{150pt}

\tasknumber{6}%
\task{%
    Излучение какой длины волны поглотил атом водорода, если полная энергия в атоме увеличилась на $3 \cdot 10^{-19}\,\text{Дж}$?
    Постоянная Планка $h = 6{,}626 \cdot 10^{-34}\,\text{Дж}\cdot\text{с}$, скорость света $c = 3 \cdot 10^{8}\,\frac{\text{м}}{\text{с}}$.
}
\answer{%
    $
        E = h\nu = h \frac c\lambda
        \implies \lambda = \frac{hc}E
            = \frac{6{,}626 \cdot 10^{-34}\,\text{Дж}\cdot\text{с} \cdot {3 \cdot 10^{8}\,\frac{\text{м}}{\text{с}}}}{3 \cdot 10^{-19}\,\text{Дж}}
            = 663\,\text{нм}.
    $
}
\solutionspace{150pt}

\tasknumber{7}%
\task{%
    Сделайте схематичный рисунок энергетических уровней атома водорода
    и отметьте на нём первый (основной) уровень и последующие.
    Сколько различных длин волн может испустить атом водорода,
    находящийся в 4-м возбуждённом состоянии?
    Отметьте все соответствующие переходы на рисунке и укажите,
    при каком переходе (среди отмеченных) частота излучённого фотона минимальна.
}
\answer{%
    $N = 6{,}0, \text{самая короткая линия}$
}
\solutionspace{150pt}

\tasknumber{8}%
\task{%
    Сколько фотонов испускает за $10\,\text{мин}$ лазер,
    если мощность его излучения $200\,\text{мВт}$?
    Длина волны излучения $500\,\text{нм}$.
    $h = 6{,}626 \cdot 10^{-34}\,\text{Дж}\cdot\text{с}$.
}
\answer{%
    $
        N
            = \frac{E_{\text{общая}}}{E_{\text{одного фотона}}}
            = \frac{Pt}{h\nu} = \frac{Pt}{h \frac c\lambda}
            = \frac{Pt\lambda}{hc}
            = \frac{200\,\text{мВт} \cdot 10\,\text{мин} \cdot 500\,\text{нм}}{6{,}626 \cdot 10^{-34}\,\text{Дж}\cdot\text{с} \cdot 3 \cdot 10^{8}\,\frac{\text{м}}{\text{с}}}
            \approx 3{,}02 \cdot 10^{20}\units{фотонов}
    $
}
\solutionspace{120pt}

\tasknumber{9}%
\task{%
    Какая доля (от начального количества) радиоактивных ядер распадётся через время,
    равное трём периодам полураспада? Ответ выразить в процентах.
}
\answer{%
    \begin{align*}
    N &= N_0 \cdot 2^{- \frac t{T_{1/2}}} \implies
        \frac N{N_0} = 2^{- \frac t{T_{1/2}}}
        = 2^{-3} \approx 0{,}12 \approx 12\% \\
    N_\text{расп.} &= N_0 - N = N_0 - N_0 \cdot 2^{-\frac t{T_{1/2}}}
        = N_0\cbr{1 - 2^{-\frac t{T_{1/2}}}} \implies
        \frac{N_\text{расп.}}{N_0} = 1 - 2^{-\frac t{T_{1/2}}}
        = 1 - 2^{-3} \approx 0{,}88 \approx 88\%
    \end{align*}
}
\solutionspace{150pt}

\tasknumber{10}%
\task{%
    Сколько процентов ядер радиоактивного железа $\ce{^{59}Fe}$
    останется через $91{,}2\,\text{суток}$, если период его полураспада составляет $45{,}6\,\text{суток}$?
}
\answer{%
    \begin{align*}
    N &= N_0 \cdot 2^{-\frac t{T_{1/2}}}
        = 2^{-\frac{91{,}2\,\text{суток}}{45{,}6\,\text{суток}}}
        \approx 0{,}2500 = 25{,}00\%
    \end{align*}
}
\solutionspace{150pt}

\tasknumber{11}%
\task{%
    За $3\,\text{суток}$ от начального количества ядер радиоизотопа осталась одна шестнадцатая.
    Каков период полураспада этого изотопа (ответ приведите в сутках)?
    Какая ещё доля (также от начального количества) распадётся, если подождать ещё столько же?
}
\answer{%
    \begin{align*}
            N &= N_0 \cdot 2^{-\frac t{T_{1/2}}}
            \implies \frac N{N_0} = 2^{-\frac t{T_{1/2}}}
            \implies \frac 1{16} = 2^{-\frac {3\,\text{суток}}{T_{1/2}}}
            \implies 4 = \frac {3\,\text{суток}}{T_{1/2}}
            \implies T_{1/2} = \frac {3\,\text{суток}}4 \approx 0{,}75\,\text{суток}.
         \\
            \delta &= \frac{N(t)}{N_0} - \frac{N(2t)}{N_0}
            = 2^{-\frac t{T_{1/2}}} - 2^{-\frac {2t}{T_{1/2}}}
            = 2^{-\frac t{T_{1/2}}}\cbr{1 - 2^{-\frac t{T_{1/2}}}}
            = \frac 1{16} \cdot \cbr{1-\frac 1{16}} \approx 0{,}059
    \end{align*}
}
\solutionspace{150pt}

\tasknumber{12}%
\task{%
    Энергия связи ядра азота \ce{^{14}_{7}N} равна $115{,}5\,\text{МэВ}$.
    Найти дефект массы этого ядра.
    Ответ выразите в а.е.м.
    и кг.
    Скорость света $c = 2{,}998 \cdot 10^{8}\,\frac{\text{м}}{\text{с}}$, элементарный заряд $e = 1{,}6 \cdot 10^{-19}\,\text{Кл}$.
}
\answer{%
    \begin{align*}
    E_\text{св.} &= \Delta m c^2 \implies \\
    \implies
            \Delta m &= \frac {E_\text{св.}}{c^2} = \frac{115{,}5\,\text{МэВ}}{\sqr{2{,}998 \cdot 10^{8}\,\frac{\text{м}}{\text{с}}}}
            = \frac{115{,}5 \cdot 10^6 \cdot 1{,}6 \cdot 10^{-19}\,\text{Дж}}{\sqr{2{,}998 \cdot 10^{8}\,\frac{\text{м}}{\text{с}}}}
            \approx 0{,}206 \cdot 10^{-27}\,\text{кг} \approx 0{,}1238\,\text{а.е.м.}
    \end{align*}
}

\variantsplitter

\addpersonalvariant{Александр Селехметьев}

\tasknumber{1}%
\task{%
    Для частицы, движущейся с релятивистской скоростью,
    выразите $v$ и $E_0$ через $c$, $E_\text{кин}$ и $p$,
    где $E_\text{кин}$~--- кинетическая энергия частицы,
    а $E_0$, $p$ и $v$~--- её энергия покоя импульс и скорость.
}
\answer{%
    \begin{align*}
    E_\text{кин}, E_0:\quad&E = E_\text{кин} + E_0 = \frac{E_0}{\sqrt{1 - \frac{v^2}{c^2}}} \implies \sqrt{1 - \frac{v^2}{c^2}} = \frac{E_0}{{E_0} + {E_\text{кин}}} \implies v = c\sqrt{1 - \sqr{\frac{E_0}{{E_0} + {E_\text{кин}}}}} \\
    &p = \frac{mv}{\sqrt{1 - \frac{v^2}{c^2}}} = \frac{E_0}{c^2} \cdot \sqrt{1 - \sqr{\frac{E_0}{{E_0} + {E_\text{кин}}}}} \cdot \frac{{E_\text{кин}} + {E_0}}{E_0} = \frac{E_0}{c^2} \cdot \sqrt{\sqr{\frac{{E_\text{кин}} + {E_0}}{E_0}} - 1}.
    \\
    E_\text{кин}, p:\quad&E_\text{кин} = E - E_0 = mc^2\cbr{\frac 1{\sqrt{1 - \frac{v^2}{c^2}}} - 1}, p = \frac{mv}{\sqrt{1 - \frac{v^2}{c^2}}} \implies \frac{E_\text{кин}}{p} = \frac{\frac 1{\sqrt{1 - \frac{v^2}{c^2}}} - 1}{\sqrt{1 - \frac{v^2}{c^2}}} \implies v = \ldots \\
    &E_0 = E - E_\text{кин} = \frac{E_0}{\sqrt{1 - \frac{v^2}{c^2}}} - E_\text{кин} \implies E_0 = \frac{E_\text{кин}}{\frac 1{\sqrt{1 - \frac{v^2}{c^2}}} - 1} = \ldots \\
    E_\text{кин}, v:\quad&E_\text{кин} = E - E_0 = mc^2\cbr{\frac 1{\sqrt{1 - \frac{v^2}{c^2}}} - 1} \implies m = \frac{E_\text{кин}}{c^2\cbr{\frac 1{\sqrt{1 - \frac{v^2}{c^2}}} - 1}} \\
    &E_0 = mc^2 = \frac{E_\text{кин}}{\frac 1{\sqrt{1 - \frac{v^2}{c^2}}} - 1} \\
    &p = \frac{mv}{\sqrt{1 - \frac{v^2}{c^2}}} = \frac{E_\text{кин}}{c^2\cbr{\frac 1{\sqrt{1 - \frac{v^2}{c^2}}} - 1}} \cdot \frac{v}{\sqrt{1 - \frac{v^2}{c^2}}} = \frac{{E_\text{кин}} v}{c^2\cbr{1 - {\sqrt{1 - \frac{v^2}{c^2}}}}} \\
    E_0, p:\quad&E_0 = mc^2, \quad p = \frac{mv}{\sqrt{1 - \frac{v^2}{c^2}}} \implies \frac{E_0}{p} = \frac{c^2}v{\sqrt{1 - \frac{v^2}{c^2}}} = c\sqrt{\frac{c^2}{v^2} - 1} \\
    &\sqr{\frac{E_0}{pc}} = \frac{c^2}{v^2} - 1 \implies \frac{v^2}{c^2} = \frac 1{1 + \frac{E_0^2}{p^2c^2}} \implies v = \frac c{\sqrt{1 + \frac{E_0^2}{p^2c^2}}} \\
    &{E_\text{кин}} = E - E_0 = \sqrt{E_0^2 + p^2c^2} - E_0 \\
    E_0, v:\quad&E_0 = mc^2 \implies m = \frac{E_0}{c^2} \qquad p = \frac{mv}{\sqrt{1 - \frac{v^2}{c^2}}} = \frac{E_0}{c^2} \cdot \frac{v}{\sqrt{1 - \frac{v^2}{c^2}}} \\
    &E_\text{кин}= mc^2\cbr{\frac 1{\sqrt{1 - \frac{v^2}{c^2}}} - 1} = \frac{E_0}{c^2}\cbr{\frac 1{\sqrt{1 - \frac{v^2}{c^2}}} - 1} \\
    p, v:\quad&p = \frac{mv}{\sqrt{1 - \frac{v^2}{c^2}}} \implies m = \frac p v {\sqrt{1 - \frac{v^2}{c^2}}} \implies E_0 = mc^2 =\frac {pc^2} v {\sqrt{1 - \frac{v^2}{c^2}}} \\
    &E_\text{кин} = mc^2\cbr{\frac 1{\sqrt{1 - \frac{v^2}{c^2}}} - 1} = \frac p v {\sqrt{1 - \frac{v^2}{c^2}}}\cbr{\frac 1{\sqrt{1 - \frac{v^2}{c^2}}} - 1} = \frac p v \cbr{1 - {\sqrt{1 - \frac{v^2}{c^2}}}}
    \end{align*}
}
\solutionspace{200pt}

\tasknumber{2}%
\task{%
    Протон движется со скоростью $0{,}8\,c$, где $c$~--- скорость света в вакууме.
    Каково при этом отношение полной энергии частицы $E$ к его энергии покоя $E_0$?
}
\answer{%
    \begin{align*}
    E &= \frac{E_0}{\sqrt{1 - \frac{v^2}{c^2}}}
            \implies \frac E{E_0}
                = \frac 1{\sqrt{1 - \frac{v^2}{c^2}}}
                = \frac 1{\sqrt{1 - \sqr{0{,}8}}}
                \approx 1{,}667,
         \\
        E_{\text{кин}} &= E - E_0
            \implies \frac{E_{\text{кин}}}{E_0}
                = \frac E{E_0} - 1
                = \frac 1{\sqrt{1 - \frac{v^2}{c^2}}} - 1
                = \frac 1{\sqrt{1 - \sqr{0{,}8}}} - 1
                \approx 0{,}667.
    \end{align*}
}
\solutionspace{150pt}

\tasknumber{3}%
\task{%
    Протон движется со скоростью $0{,}85\,c$, где $c$~--- скорость света в вакууме.
    Определите его полную энергию (в ответе приведите формулу и укажите численное значение).
}
\answer{%
    \begin{align*}
    E &= \frac{mc^2}{\sqrt{1 - \frac{v^2}{c^2}}}
            \approx \frac{1{,}673 \cdot 10^{-27}\,\text{кг} \cdot \sqr{3 \cdot 10^{8}\,\frac{\text{м}}{\text{с}}}}{\sqrt{1 - 0{,}85^2}}
            \approx 285{,}765 \cdot 10^{-12}\,\text{Дж},
         \\
        E_{\text{кин}} &= \frac{mc^2}{\sqrt{1 - \frac{v^2}{c^2}}} - mc^2
            = mc^2 \cbr{\frac 1{\sqrt{1 - \frac{v^2}{c^2}}} - 1} \approx \\
            &\approx \cbr{1{,}673 \cdot 10^{-27}\,\text{кг} \cdot \sqr{3 \cdot 10^{8}\,\frac{\text{м}}{\text{с}}}}
            \cdot \cbr{\frac 1{\sqrt{1 - 0{,}85^2}} - 1}
            \approx 135{,}229 \cdot 10^{-12}\,\text{Дж},
         \\
        p &= \frac{mv}{\sqrt{1 - \frac{v^2}{c^2}}}
            \approx \frac{1{,}673 \cdot 10^{-27}\,\text{кг} \cdot 0{,}85 \cdot 3 \cdot 10^{8}\,\frac{\text{м}}{\text{с}}}{\sqrt{1 - 0{,}85^2}}
            \approx 809{,}666 \cdot 10^{-21}\,\frac{\text{кг}\cdot\text{м}}{\text{с}}.
    \end{align*}
}
\solutionspace{150pt}

\tasknumber{4}%
\task{%
    При какой скорости движения (в м/с) релятивистское сокращение длины движущегося тела
    составит 30\%?
}
\answer{%
    \begin{align*}
    l_0 &= \frac l{\sqrt{1 - \frac{v^2}{c^2}}}
        \implies 1 - \frac{v^2}{c^2} = \sqr{\frac l{l_0}}
        \implies \frac v c = \sqrt{1 - \sqr{\frac l{l_0}}} \implies
         \\
        \implies v &= c\sqrt{1 - \sqr{\frac l{l_0}}}
        = 3 \cdot 10^{8}\,\frac{\text{м}}{\text{с}} \cdot \sqrt{1 - \sqr{\frac {l_0 - 0{,}30l_0}{l_0}}}
        = 3 \cdot 10^{8}\,\frac{\text{м}}{\text{с}} \cdot \sqrt{1 - \sqr{1 - 0{,}30}} \approx  \\
        &\approx 0{,}714c
        \approx 214 \cdot 10^{6}\,\frac{\text{м}}{\text{с}}
        \approx 771 \cdot 10^{6}\,\frac{\text{км}}{\text{ч}}.
    \end{align*}
}
\solutionspace{150pt}

\tasknumber{5}%
\task{%
    При переходе электрона в атоме с одной стационарной орбиты на другую
    излучается фотон с энергией $5{,}05 \cdot 10^{-19}\,\text{Дж}$.
    Какова длина волны этой линии спектра?
    Постоянная Планка $h = 6{,}626 \cdot 10^{-34}\,\text{Дж}\cdot\text{с}$, скорость света $c = 3 \cdot 10^{8}\,\frac{\text{м}}{\text{с}}$.
}
\answer{%
    $
        E = h\nu = h \frac c\lambda
        \implies \lambda = \frac{hc}E
            = \frac{6{,}626 \cdot 10^{-34}\,\text{Дж}\cdot\text{с} \cdot {3 \cdot 10^{8}\,\frac{\text{м}}{\text{с}}}}{5{,}05 \cdot 10^{-19}\,\text{Дж}}
            = 393{,}62\,\text{нм}.
    $
}
\solutionspace{150pt}

\tasknumber{6}%
\task{%
    Излучение какой длины волны поглотил атом водорода, если полная энергия в атоме увеличилась на $3 \cdot 10^{-19}\,\text{Дж}$?
    Постоянная Планка $h = 6{,}626 \cdot 10^{-34}\,\text{Дж}\cdot\text{с}$, скорость света $c = 3 \cdot 10^{8}\,\frac{\text{м}}{\text{с}}$.
}
\answer{%
    $
        E = h\nu = h \frac c\lambda
        \implies \lambda = \frac{hc}E
            = \frac{6{,}626 \cdot 10^{-34}\,\text{Дж}\cdot\text{с} \cdot {3 \cdot 10^{8}\,\frac{\text{м}}{\text{с}}}}{3 \cdot 10^{-19}\,\text{Дж}}
            = 663\,\text{нм}.
    $
}
\solutionspace{150pt}

\tasknumber{7}%
\task{%
    Сделайте схематичный рисунок энергетических уровней атома водорода
    и отметьте на нём первый (основной) уровень и последующие.
    Сколько различных длин волн может испустить атом водорода,
    находящийся в 3-м возбуждённом состоянии?
    Отметьте все соответствующие переходы на рисунке и укажите,
    при каком переходе (среди отмеченных) частота излучённого фотона максимальна.
}
\answer{%
    $N = 3{,}0, \text{самая длинная линия}$
}
\solutionspace{150pt}

\tasknumber{8}%
\task{%
    Сколько фотонов испускает за $40\,\text{мин}$ лазер,
    если мощность его излучения $200\,\text{мВт}$?
    Длина волны излучения $600\,\text{нм}$.
    $h = 6{,}626 \cdot 10^{-34}\,\text{Дж}\cdot\text{с}$.
}
\answer{%
    $
        N
            = \frac{E_{\text{общая}}}{E_{\text{одного фотона}}}
            = \frac{Pt}{h\nu} = \frac{Pt}{h \frac c\lambda}
            = \frac{Pt\lambda}{hc}
            = \frac{200\,\text{мВт} \cdot 40\,\text{мин} \cdot 600\,\text{нм}}{6{,}626 \cdot 10^{-34}\,\text{Дж}\cdot\text{с} \cdot 3 \cdot 10^{8}\,\frac{\text{м}}{\text{с}}}
            \approx 14{,}49 \cdot 10^{20}\units{фотонов}
    $
}
\solutionspace{120pt}

\tasknumber{9}%
\task{%
    Какая доля (от начального количества) радиоактивных ядер останется через время,
    равное четырём периодам полураспада? Ответ выразить в процентах.
}
\answer{%
    \begin{align*}
    N &= N_0 \cdot 2^{- \frac t{T_{1/2}}} \implies
        \frac N{N_0} = 2^{- \frac t{T_{1/2}}}
        = 2^{-4} \approx 0{,}06 \approx 6\% \\
    N_\text{расп.} &= N_0 - N = N_0 - N_0 \cdot 2^{-\frac t{T_{1/2}}}
        = N_0\cbr{1 - 2^{-\frac t{T_{1/2}}}} \implies
        \frac{N_\text{расп.}}{N_0} = 1 - 2^{-\frac t{T_{1/2}}}
        = 1 - 2^{-4} \approx 0{,}94 \approx 94\%
    \end{align*}
}
\solutionspace{150pt}

\tasknumber{10}%
\task{%
    Сколько процентов ядер радиоактивного железа $\ce{^{59}Fe}$
    останется через $136{,}8\,\text{суток}$, если период его полураспада составляет $45{,}6\,\text{суток}$?
}
\answer{%
    \begin{align*}
    N &= N_0 \cdot 2^{-\frac t{T_{1/2}}}
        = 2^{-\frac{136{,}8\,\text{суток}}{45{,}6\,\text{суток}}}
        \approx 0{,}1250 = 12{,}50\%
    \end{align*}
}
\solutionspace{150pt}

\tasknumber{11}%
\task{%
    За $5\,\text{суток}$ от начального количества ядер радиоизотопа осталась половина.
    Каков период полураспада этого изотопа (ответ приведите в сутках)?
    Какая ещё доля (также от начального количества) распадётся, если подождать ещё столько же?
}
\answer{%
    \begin{align*}
            N &= N_0 \cdot 2^{-\frac t{T_{1/2}}}
            \implies \frac N{N_0} = 2^{-\frac t{T_{1/2}}}
            \implies \frac 1{2} = 2^{-\frac {5\,\text{суток}}{T_{1/2}}}
            \implies 1 = \frac {5\,\text{суток}}{T_{1/2}}
            \implies T_{1/2} = \frac {5\,\text{суток}}1 \approx 5\,\text{суток}.
         \\
            \delta &= \frac{N(t)}{N_0} - \frac{N(2t)}{N_0}
            = 2^{-\frac t{T_{1/2}}} - 2^{-\frac {2t}{T_{1/2}}}
            = 2^{-\frac t{T_{1/2}}}\cbr{1 - 2^{-\frac t{T_{1/2}}}}
            = \frac 1{2} \cdot \cbr{1-\frac 1{2}} \approx 0{,}250
    \end{align*}
}
\solutionspace{150pt}

\tasknumber{12}%
\task{%
    Энергия связи ядра гелия \ce{^{3}_{2}He} равна $28{,}29\,\text{МэВ}$.
    Найти дефект массы этого ядра.
    Ответ выразите в а.е.м.
    и кг.
    Скорость света $c = 2{,}998 \cdot 10^{8}\,\frac{\text{м}}{\text{с}}$, элементарный заряд $e = 1{,}6 \cdot 10^{-19}\,\text{Кл}$.
}
\answer{%
    \begin{align*}
    E_\text{св.} &= \Delta m c^2 \implies \\
    \implies
            \Delta m &= \frac {E_\text{св.}}{c^2} = \frac{28{,}29\,\text{МэВ}}{\sqr{2{,}998 \cdot 10^{8}\,\frac{\text{м}}{\text{с}}}}
            = \frac{28{,}29 \cdot 10^6 \cdot 1{,}6 \cdot 10^{-19}\,\text{Дж}}{\sqr{2{,}998 \cdot 10^{8}\,\frac{\text{м}}{\text{с}}}}
            \approx 50{,}36 \cdot 10^{-30}\,\text{кг} \approx 0{,}03033\,\text{а.е.м.}
    \end{align*}
}

\variantsplitter

\addpersonalvariant{Алексей Тихонов}

\tasknumber{1}%
\task{%
    Для частицы, движущейся с релятивистской скоростью,
    выразите $E_0$ и $v$ через $c$, $E_\text{кин}$ и $p$,
    где $E_\text{кин}$~--- кинетическая энергия частицы,
    а $E_0$, $p$ и $v$~--- её энергия покоя импульс и скорость.
}
\answer{%
    \begin{align*}
    E_\text{кин}, E_0:\quad&E = E_\text{кин} + E_0 = \frac{E_0}{\sqrt{1 - \frac{v^2}{c^2}}} \implies \sqrt{1 - \frac{v^2}{c^2}} = \frac{E_0}{{E_0} + {E_\text{кин}}} \implies v = c\sqrt{1 - \sqr{\frac{E_0}{{E_0} + {E_\text{кин}}}}} \\
    &p = \frac{mv}{\sqrt{1 - \frac{v^2}{c^2}}} = \frac{E_0}{c^2} \cdot \sqrt{1 - \sqr{\frac{E_0}{{E_0} + {E_\text{кин}}}}} \cdot \frac{{E_\text{кин}} + {E_0}}{E_0} = \frac{E_0}{c^2} \cdot \sqrt{\sqr{\frac{{E_\text{кин}} + {E_0}}{E_0}} - 1}.
    \\
    E_\text{кин}, p:\quad&E_\text{кин} = E - E_0 = mc^2\cbr{\frac 1{\sqrt{1 - \frac{v^2}{c^2}}} - 1}, p = \frac{mv}{\sqrt{1 - \frac{v^2}{c^2}}} \implies \frac{E_\text{кин}}{p} = \frac{\frac 1{\sqrt{1 - \frac{v^2}{c^2}}} - 1}{\sqrt{1 - \frac{v^2}{c^2}}} \implies v = \ldots \\
    &E_0 = E - E_\text{кин} = \frac{E_0}{\sqrt{1 - \frac{v^2}{c^2}}} - E_\text{кин} \implies E_0 = \frac{E_\text{кин}}{\frac 1{\sqrt{1 - \frac{v^2}{c^2}}} - 1} = \ldots \\
    E_\text{кин}, v:\quad&E_\text{кин} = E - E_0 = mc^2\cbr{\frac 1{\sqrt{1 - \frac{v^2}{c^2}}} - 1} \implies m = \frac{E_\text{кин}}{c^2\cbr{\frac 1{\sqrt{1 - \frac{v^2}{c^2}}} - 1}} \\
    &E_0 = mc^2 = \frac{E_\text{кин}}{\frac 1{\sqrt{1 - \frac{v^2}{c^2}}} - 1} \\
    &p = \frac{mv}{\sqrt{1 - \frac{v^2}{c^2}}} = \frac{E_\text{кин}}{c^2\cbr{\frac 1{\sqrt{1 - \frac{v^2}{c^2}}} - 1}} \cdot \frac{v}{\sqrt{1 - \frac{v^2}{c^2}}} = \frac{{E_\text{кин}} v}{c^2\cbr{1 - {\sqrt{1 - \frac{v^2}{c^2}}}}} \\
    E_0, p:\quad&E_0 = mc^2, \quad p = \frac{mv}{\sqrt{1 - \frac{v^2}{c^2}}} \implies \frac{E_0}{p} = \frac{c^2}v{\sqrt{1 - \frac{v^2}{c^2}}} = c\sqrt{\frac{c^2}{v^2} - 1} \\
    &\sqr{\frac{E_0}{pc}} = \frac{c^2}{v^2} - 1 \implies \frac{v^2}{c^2} = \frac 1{1 + \frac{E_0^2}{p^2c^2}} \implies v = \frac c{\sqrt{1 + \frac{E_0^2}{p^2c^2}}} \\
    &{E_\text{кин}} = E - E_0 = \sqrt{E_0^2 + p^2c^2} - E_0 \\
    E_0, v:\quad&E_0 = mc^2 \implies m = \frac{E_0}{c^2} \qquad p = \frac{mv}{\sqrt{1 - \frac{v^2}{c^2}}} = \frac{E_0}{c^2} \cdot \frac{v}{\sqrt{1 - \frac{v^2}{c^2}}} \\
    &E_\text{кин}= mc^2\cbr{\frac 1{\sqrt{1 - \frac{v^2}{c^2}}} - 1} = \frac{E_0}{c^2}\cbr{\frac 1{\sqrt{1 - \frac{v^2}{c^2}}} - 1} \\
    p, v:\quad&p = \frac{mv}{\sqrt{1 - \frac{v^2}{c^2}}} \implies m = \frac p v {\sqrt{1 - \frac{v^2}{c^2}}} \implies E_0 = mc^2 =\frac {pc^2} v {\sqrt{1 - \frac{v^2}{c^2}}} \\
    &E_\text{кин} = mc^2\cbr{\frac 1{\sqrt{1 - \frac{v^2}{c^2}}} - 1} = \frac p v {\sqrt{1 - \frac{v^2}{c^2}}}\cbr{\frac 1{\sqrt{1 - \frac{v^2}{c^2}}} - 1} = \frac p v \cbr{1 - {\sqrt{1 - \frac{v^2}{c^2}}}}
    \end{align*}
}
\solutionspace{200pt}

\tasknumber{2}%
\task{%
    Протон движется со скоростью $0{,}8\,c$, где $c$~--- скорость света в вакууме.
    Каково при этом отношение кинетической энергии частицы $E_\text{кин.}$ к его энергии покоя $E_0$?
}
\answer{%
    \begin{align*}
    E &= \frac{E_0}{\sqrt{1 - \frac{v^2}{c^2}}}
            \implies \frac E{E_0}
                = \frac 1{\sqrt{1 - \frac{v^2}{c^2}}}
                = \frac 1{\sqrt{1 - \sqr{0{,}8}}}
                \approx 1{,}667,
         \\
        E_{\text{кин}} &= E - E_0
            \implies \frac{E_{\text{кин}}}{E_0}
                = \frac E{E_0} - 1
                = \frac 1{\sqrt{1 - \frac{v^2}{c^2}}} - 1
                = \frac 1{\sqrt{1 - \sqr{0{,}8}}} - 1
                \approx 0{,}667.
    \end{align*}
}
\solutionspace{150pt}

\tasknumber{3}%
\task{%
    Электрон движется со скоростью $0{,}65\,c$, где $c$~--- скорость света в вакууме.
    Определите его полную энергию (в ответе приведите формулу и укажите численное значение).
}
\answer{%
    \begin{align*}
    E &= \frac{mc^2}{\sqrt{1 - \frac{v^2}{c^2}}}
            \approx \frac{9{,}1 \cdot 10^{-31}\,\text{кг} \cdot \sqr{3 \cdot 10^{8}\,\frac{\text{м}}{\text{с}}}}{\sqrt{1 - 0{,}65^2}}
            \approx 0{,}108 \cdot 10^{-12}\,\text{Дж},
         \\
        E_{\text{кин}} &= \frac{mc^2}{\sqrt{1 - \frac{v^2}{c^2}}} - mc^2
            = mc^2 \cbr{\frac 1{\sqrt{1 - \frac{v^2}{c^2}}} - 1} \approx \\
            &\approx \cbr{9{,}1 \cdot 10^{-31}\,\text{кг} \cdot \sqr{3 \cdot 10^{8}\,\frac{\text{м}}{\text{с}}}}
            \cdot \cbr{\frac 1{\sqrt{1 - 0{,}65^2}} - 1}
            \approx 0{,}026 \cdot 10^{-12}\,\text{Дж},
         \\
        p &= \frac{mv}{\sqrt{1 - \frac{v^2}{c^2}}}
            \approx \frac{9{,}1 \cdot 10^{-31}\,\text{кг} \cdot 0{,}65 \cdot 3 \cdot 10^{8}\,\frac{\text{м}}{\text{с}}}{\sqrt{1 - 0{,}65^2}}
            \approx 0{,}234 \cdot 10^{-21}\,\frac{\text{кг}\cdot\text{м}}{\text{с}}.
    \end{align*}
}
\solutionspace{150pt}

\tasknumber{4}%
\task{%
    При какой скорости движения (в км/ч) релятивистское сокращение длины движущегося тела
    составит 50\%?
}
\answer{%
    \begin{align*}
    l_0 &= \frac l{\sqrt{1 - \frac{v^2}{c^2}}}
        \implies 1 - \frac{v^2}{c^2} = \sqr{\frac l{l_0}}
        \implies \frac v c = \sqrt{1 - \sqr{\frac l{l_0}}} \implies
         \\
        \implies v &= c\sqrt{1 - \sqr{\frac l{l_0}}}
        = 3 \cdot 10^{8}\,\frac{\text{м}}{\text{с}} \cdot \sqrt{1 - \sqr{\frac {l_0 - 0{,}50l_0}{l_0}}}
        = 3 \cdot 10^{8}\,\frac{\text{м}}{\text{с}} \cdot \sqrt{1 - \sqr{1 - 0{,}50}} \approx  \\
        &\approx 0{,}866c
        \approx 260 \cdot 10^{6}\,\frac{\text{м}}{\text{с}}
        \approx 935 \cdot 10^{6}\,\frac{\text{км}}{\text{ч}}.
    \end{align*}
}
\solutionspace{150pt}

\tasknumber{5}%
\task{%
    При переходе электрона в атоме с одной стационарной орбиты на другую
    излучается фотон с энергией $0{,}55 \cdot 10^{-19}\,\text{Дж}$.
    Какова длина волны этой линии спектра?
    Постоянная Планка $h = 6{,}626 \cdot 10^{-34}\,\text{Дж}\cdot\text{с}$, скорость света $c = 3 \cdot 10^{8}\,\frac{\text{м}}{\text{с}}$.
}
\answer{%
    $
        E = h\nu = h \frac c\lambda
        \implies \lambda = \frac{hc}E
            = \frac{6{,}626 \cdot 10^{-34}\,\text{Дж}\cdot\text{с} \cdot {3 \cdot 10^{8}\,\frac{\text{м}}{\text{с}}}}{0{,}55 \cdot 10^{-19}\,\text{Дж}}
            = 3614\,\text{нм}.
    $
}
\solutionspace{150pt}

\tasknumber{6}%
\task{%
    Излучение какой длины волны поглотил атом водорода, если полная энергия в атоме увеличилась на $3 \cdot 10^{-19}\,\text{Дж}$?
    Постоянная Планка $h = 6{,}626 \cdot 10^{-34}\,\text{Дж}\cdot\text{с}$, скорость света $c = 3 \cdot 10^{8}\,\frac{\text{м}}{\text{с}}$.
}
\answer{%
    $
        E = h\nu = h \frac c\lambda
        \implies \lambda = \frac{hc}E
            = \frac{6{,}626 \cdot 10^{-34}\,\text{Дж}\cdot\text{с} \cdot {3 \cdot 10^{8}\,\frac{\text{м}}{\text{с}}}}{3 \cdot 10^{-19}\,\text{Дж}}
            = 663\,\text{нм}.
    $
}
\solutionspace{150pt}

\tasknumber{7}%
\task{%
    Сделайте схематичный рисунок энергетических уровней атома водорода
    и отметьте на нём первый (основной) уровень и последующие.
    Сколько различных длин волн может испустить атом водорода,
    находящийся в 5-м возбуждённом состоянии?
    Отметьте все соответствующие переходы на рисунке и укажите,
    при каком переходе (среди отмеченных) длина волны излучённого фотона максимальна.
}
\answer{%
    $N = 10{,}0, \text{самая короткая линия}$
}
\solutionspace{150pt}

\tasknumber{8}%
\task{%
    Сколько фотонов испускает за $30\,\text{мин}$ лазер,
    если мощность его излучения $40\,\text{мВт}$?
    Длина волны излучения $500\,\text{нм}$.
    $h = 6{,}626 \cdot 10^{-34}\,\text{Дж}\cdot\text{с}$.
}
\answer{%
    $
        N
            = \frac{E_{\text{общая}}}{E_{\text{одного фотона}}}
            = \frac{Pt}{h\nu} = \frac{Pt}{h \frac c\lambda}
            = \frac{Pt\lambda}{hc}
            = \frac{40\,\text{мВт} \cdot 30\,\text{мин} \cdot 500\,\text{нм}}{6{,}626 \cdot 10^{-34}\,\text{Дж}\cdot\text{с} \cdot 3 \cdot 10^{8}\,\frac{\text{м}}{\text{с}}}
            \approx 1{,}81 \cdot 10^{20}\units{фотонов}
    $
}
\solutionspace{120pt}

\tasknumber{9}%
\task{%
    Какая доля (от начального количества) радиоактивных ядер распадётся через время,
    равное четырём периодам полураспада? Ответ выразить в процентах.
}
\answer{%
    \begin{align*}
    N &= N_0 \cdot 2^{- \frac t{T_{1/2}}} \implies
        \frac N{N_0} = 2^{- \frac t{T_{1/2}}}
        = 2^{-4} \approx 0{,}06 \approx 6\% \\
    N_\text{расп.} &= N_0 - N = N_0 - N_0 \cdot 2^{-\frac t{T_{1/2}}}
        = N_0\cbr{1 - 2^{-\frac t{T_{1/2}}}} \implies
        \frac{N_\text{расп.}}{N_0} = 1 - 2^{-\frac t{T_{1/2}}}
        = 1 - 2^{-4} \approx 0{,}94 \approx 94\%
    \end{align*}
}
\solutionspace{150pt}

\tasknumber{10}%
\task{%
    Сколько процентов ядер радиоактивного железа $\ce{^{59}Fe}$
    останется через $91{,}2\,\text{суток}$, если период его полураспада составляет $45{,}6\,\text{суток}$?
}
\answer{%
    \begin{align*}
    N &= N_0 \cdot 2^{-\frac t{T_{1/2}}}
        = 2^{-\frac{91{,}2\,\text{суток}}{45{,}6\,\text{суток}}}
        \approx 0{,}2500 = 25{,}00\%
    \end{align*}
}
\solutionspace{150pt}

\tasknumber{11}%
\task{%
    За $5\,\text{суток}$ от начального количества ядер радиоизотопа осталась одна восьмая.
    Каков период полураспада этого изотопа (ответ приведите в сутках)?
    Какая ещё доля (также от начального количества) распадётся, если подождать ещё столько же?
}
\answer{%
    \begin{align*}
            N &= N_0 \cdot 2^{-\frac t{T_{1/2}}}
            \implies \frac N{N_0} = 2^{-\frac t{T_{1/2}}}
            \implies \frac 1{8} = 2^{-\frac {5\,\text{суток}}{T_{1/2}}}
            \implies 3 = \frac {5\,\text{суток}}{T_{1/2}}
            \implies T_{1/2} = \frac {5\,\text{суток}}3 \approx 1{,}67\,\text{суток}.
         \\
            \delta &= \frac{N(t)}{N_0} - \frac{N(2t)}{N_0}
            = 2^{-\frac t{T_{1/2}}} - 2^{-\frac {2t}{T_{1/2}}}
            = 2^{-\frac t{T_{1/2}}}\cbr{1 - 2^{-\frac t{T_{1/2}}}}
            = \frac 1{8} \cdot \cbr{1-\frac 1{8}} \approx 0{,}109
    \end{align*}
}
\solutionspace{150pt}

\tasknumber{12}%
\task{%
    Энергия связи ядра азота \ce{^{14}_{7}N} равна $115{,}5\,\text{МэВ}$.
    Найти дефект массы этого ядра.
    Ответ выразите в а.е.м.
    и кг.
    Скорость света $c = 2{,}998 \cdot 10^{8}\,\frac{\text{м}}{\text{с}}$, элементарный заряд $e = 1{,}6 \cdot 10^{-19}\,\text{Кл}$.
}
\answer{%
    \begin{align*}
    E_\text{св.} &= \Delta m c^2 \implies \\
    \implies
            \Delta m &= \frac {E_\text{св.}}{c^2} = \frac{115{,}5\,\text{МэВ}}{\sqr{2{,}998 \cdot 10^{8}\,\frac{\text{м}}{\text{с}}}}
            = \frac{115{,}5 \cdot 10^6 \cdot 1{,}6 \cdot 10^{-19}\,\text{Дж}}{\sqr{2{,}998 \cdot 10^{8}\,\frac{\text{м}}{\text{с}}}}
            \approx 0{,}206 \cdot 10^{-27}\,\text{кг} \approx 0{,}1238\,\text{а.е.м.}
    \end{align*}
}

\variantsplitter

\addpersonalvariant{Алина Филиппова}

\tasknumber{1}%
\task{%
    Для частицы, движущейся с релятивистской скоростью,
    выразите $v$ и $E_\text{кин}$ через $c$, $E_0$ и $p$,
    где $E_\text{кин}$~--- кинетическая энергия частицы,
    а $E_0$, $p$ и $v$~--- её энергия покоя импульс и скорость.
}
\answer{%
    \begin{align*}
    E_\text{кин}, E_0:\quad&E = E_\text{кин} + E_0 = \frac{E_0}{\sqrt{1 - \frac{v^2}{c^2}}} \implies \sqrt{1 - \frac{v^2}{c^2}} = \frac{E_0}{{E_0} + {E_\text{кин}}} \implies v = c\sqrt{1 - \sqr{\frac{E_0}{{E_0} + {E_\text{кин}}}}} \\
    &p = \frac{mv}{\sqrt{1 - \frac{v^2}{c^2}}} = \frac{E_0}{c^2} \cdot \sqrt{1 - \sqr{\frac{E_0}{{E_0} + {E_\text{кин}}}}} \cdot \frac{{E_\text{кин}} + {E_0}}{E_0} = \frac{E_0}{c^2} \cdot \sqrt{\sqr{\frac{{E_\text{кин}} + {E_0}}{E_0}} - 1}.
    \\
    E_\text{кин}, p:\quad&E_\text{кин} = E - E_0 = mc^2\cbr{\frac 1{\sqrt{1 - \frac{v^2}{c^2}}} - 1}, p = \frac{mv}{\sqrt{1 - \frac{v^2}{c^2}}} \implies \frac{E_\text{кин}}{p} = \frac{\frac 1{\sqrt{1 - \frac{v^2}{c^2}}} - 1}{\sqrt{1 - \frac{v^2}{c^2}}} \implies v = \ldots \\
    &E_0 = E - E_\text{кин} = \frac{E_0}{\sqrt{1 - \frac{v^2}{c^2}}} - E_\text{кин} \implies E_0 = \frac{E_\text{кин}}{\frac 1{\sqrt{1 - \frac{v^2}{c^2}}} - 1} = \ldots \\
    E_\text{кин}, v:\quad&E_\text{кин} = E - E_0 = mc^2\cbr{\frac 1{\sqrt{1 - \frac{v^2}{c^2}}} - 1} \implies m = \frac{E_\text{кин}}{c^2\cbr{\frac 1{\sqrt{1 - \frac{v^2}{c^2}}} - 1}} \\
    &E_0 = mc^2 = \frac{E_\text{кин}}{\frac 1{\sqrt{1 - \frac{v^2}{c^2}}} - 1} \\
    &p = \frac{mv}{\sqrt{1 - \frac{v^2}{c^2}}} = \frac{E_\text{кин}}{c^2\cbr{\frac 1{\sqrt{1 - \frac{v^2}{c^2}}} - 1}} \cdot \frac{v}{\sqrt{1 - \frac{v^2}{c^2}}} = \frac{{E_\text{кин}} v}{c^2\cbr{1 - {\sqrt{1 - \frac{v^2}{c^2}}}}} \\
    E_0, p:\quad&E_0 = mc^2, \quad p = \frac{mv}{\sqrt{1 - \frac{v^2}{c^2}}} \implies \frac{E_0}{p} = \frac{c^2}v{\sqrt{1 - \frac{v^2}{c^2}}} = c\sqrt{\frac{c^2}{v^2} - 1} \\
    &\sqr{\frac{E_0}{pc}} = \frac{c^2}{v^2} - 1 \implies \frac{v^2}{c^2} = \frac 1{1 + \frac{E_0^2}{p^2c^2}} \implies v = \frac c{\sqrt{1 + \frac{E_0^2}{p^2c^2}}} \\
    &{E_\text{кин}} = E - E_0 = \sqrt{E_0^2 + p^2c^2} - E_0 \\
    E_0, v:\quad&E_0 = mc^2 \implies m = \frac{E_0}{c^2} \qquad p = \frac{mv}{\sqrt{1 - \frac{v^2}{c^2}}} = \frac{E_0}{c^2} \cdot \frac{v}{\sqrt{1 - \frac{v^2}{c^2}}} \\
    &E_\text{кин}= mc^2\cbr{\frac 1{\sqrt{1 - \frac{v^2}{c^2}}} - 1} = \frac{E_0}{c^2}\cbr{\frac 1{\sqrt{1 - \frac{v^2}{c^2}}} - 1} \\
    p, v:\quad&p = \frac{mv}{\sqrt{1 - \frac{v^2}{c^2}}} \implies m = \frac p v {\sqrt{1 - \frac{v^2}{c^2}}} \implies E_0 = mc^2 =\frac {pc^2} v {\sqrt{1 - \frac{v^2}{c^2}}} \\
    &E_\text{кин} = mc^2\cbr{\frac 1{\sqrt{1 - \frac{v^2}{c^2}}} - 1} = \frac p v {\sqrt{1 - \frac{v^2}{c^2}}}\cbr{\frac 1{\sqrt{1 - \frac{v^2}{c^2}}} - 1} = \frac p v \cbr{1 - {\sqrt{1 - \frac{v^2}{c^2}}}}
    \end{align*}
}
\solutionspace{200pt}

\tasknumber{2}%
\task{%
    Позитрон движется со скоростью $0{,}6\,c$, где $c$~--- скорость света в вакууме.
    Каково при этом отношение полной энергии частицы $E$ к его энергии покоя $E_0$?
}
\answer{%
    \begin{align*}
    E &= \frac{E_0}{\sqrt{1 - \frac{v^2}{c^2}}}
            \implies \frac E{E_0}
                = \frac 1{\sqrt{1 - \frac{v^2}{c^2}}}
                = \frac 1{\sqrt{1 - \sqr{0{,}6}}}
                \approx 1{,}250,
         \\
        E_{\text{кин}} &= E - E_0
            \implies \frac{E_{\text{кин}}}{E_0}
                = \frac E{E_0} - 1
                = \frac 1{\sqrt{1 - \frac{v^2}{c^2}}} - 1
                = \frac 1{\sqrt{1 - \sqr{0{,}6}}} - 1
                \approx 0{,}250.
    \end{align*}
}
\solutionspace{150pt}

\tasknumber{3}%
\task{%
    Протон движется со скоростью $0{,}85\,c$, где $c$~--- скорость света в вакууме.
    Определите его полную энергию (в ответе приведите формулу и укажите численное значение).
}
\answer{%
    \begin{align*}
    E &= \frac{mc^2}{\sqrt{1 - \frac{v^2}{c^2}}}
            \approx \frac{1{,}673 \cdot 10^{-27}\,\text{кг} \cdot \sqr{3 \cdot 10^{8}\,\frac{\text{м}}{\text{с}}}}{\sqrt{1 - 0{,}85^2}}
            \approx 285{,}765 \cdot 10^{-12}\,\text{Дж},
         \\
        E_{\text{кин}} &= \frac{mc^2}{\sqrt{1 - \frac{v^2}{c^2}}} - mc^2
            = mc^2 \cbr{\frac 1{\sqrt{1 - \frac{v^2}{c^2}}} - 1} \approx \\
            &\approx \cbr{1{,}673 \cdot 10^{-27}\,\text{кг} \cdot \sqr{3 \cdot 10^{8}\,\frac{\text{м}}{\text{с}}}}
            \cdot \cbr{\frac 1{\sqrt{1 - 0{,}85^2}} - 1}
            \approx 135{,}229 \cdot 10^{-12}\,\text{Дж},
         \\
        p &= \frac{mv}{\sqrt{1 - \frac{v^2}{c^2}}}
            \approx \frac{1{,}673 \cdot 10^{-27}\,\text{кг} \cdot 0{,}85 \cdot 3 \cdot 10^{8}\,\frac{\text{м}}{\text{с}}}{\sqrt{1 - 0{,}85^2}}
            \approx 809{,}666 \cdot 10^{-21}\,\frac{\text{кг}\cdot\text{м}}{\text{с}}.
    \end{align*}
}
\solutionspace{150pt}

\tasknumber{4}%
\task{%
    При какой скорости движения (в долях скорости света) релятивистское сокращение длины движущегося тела
    составит 30\%?
}
\answer{%
    \begin{align*}
    l_0 &= \frac l{\sqrt{1 - \frac{v^2}{c^2}}}
        \implies 1 - \frac{v^2}{c^2} = \sqr{\frac l{l_0}}
        \implies \frac v c = \sqrt{1 - \sqr{\frac l{l_0}}} \implies
         \\
        \implies v &= c\sqrt{1 - \sqr{\frac l{l_0}}}
        = 3 \cdot 10^{8}\,\frac{\text{м}}{\text{с}} \cdot \sqrt{1 - \sqr{\frac {l_0 - 0{,}30l_0}{l_0}}}
        = 3 \cdot 10^{8}\,\frac{\text{м}}{\text{с}} \cdot \sqrt{1 - \sqr{1 - 0{,}30}} \approx  \\
        &\approx 0{,}714c
        \approx 214 \cdot 10^{6}\,\frac{\text{м}}{\text{с}}
        \approx 771 \cdot 10^{6}\,\frac{\text{км}}{\text{ч}}.
    \end{align*}
}
\solutionspace{150pt}

\tasknumber{5}%
\task{%
    При переходе электрона в атоме с одной стационарной орбиты на другую
    излучается фотон с энергией $4{,}04 \cdot 10^{-19}\,\text{Дж}$.
    Какова длина волны этой линии спектра?
    Постоянная Планка $h = 6{,}626 \cdot 10^{-34}\,\text{Дж}\cdot\text{с}$, скорость света $c = 3 \cdot 10^{8}\,\frac{\text{м}}{\text{с}}$.
}
\answer{%
    $
        E = h\nu = h \frac c\lambda
        \implies \lambda = \frac{hc}E
            = \frac{6{,}626 \cdot 10^{-34}\,\text{Дж}\cdot\text{с} \cdot {3 \cdot 10^{8}\,\frac{\text{м}}{\text{с}}}}{4{,}04 \cdot 10^{-19}\,\text{Дж}}
            = 492{,}03\,\text{нм}.
    $
}
\solutionspace{150pt}

\tasknumber{6}%
\task{%
    Излучение какой длины волны поглотил атом водорода, если полная энергия в атоме увеличилась на $2 \cdot 10^{-19}\,\text{Дж}$?
    Постоянная Планка $h = 6{,}626 \cdot 10^{-34}\,\text{Дж}\cdot\text{с}$, скорость света $c = 3 \cdot 10^{8}\,\frac{\text{м}}{\text{с}}$.
}
\answer{%
    $
        E = h\nu = h \frac c\lambda
        \implies \lambda = \frac{hc}E
            = \frac{6{,}626 \cdot 10^{-34}\,\text{Дж}\cdot\text{с} \cdot {3 \cdot 10^{8}\,\frac{\text{м}}{\text{с}}}}{2 \cdot 10^{-19}\,\text{Дж}}
            = 994\,\text{нм}.
    $
}
\solutionspace{150pt}

\tasknumber{7}%
\task{%
    Сделайте схематичный рисунок энергетических уровней атома водорода
    и отметьте на нём первый (основной) уровень и последующие.
    Сколько различных длин волн может испустить атом водорода,
    находящийся в 3-м возбуждённом состоянии?
    Отметьте все соответствующие переходы на рисунке и укажите,
    при каком переходе (среди отмеченных) энергия излучённого фотона минимальна.
}
\answer{%
    $N = 3{,}0, \text{самая короткая линия}$
}
\solutionspace{150pt}

\tasknumber{8}%
\task{%
    Сколько фотонов испускает за $10\,\text{мин}$ лазер,
    если мощность его излучения $40\,\text{мВт}$?
    Длина волны излучения $750\,\text{нм}$.
    $h = 6{,}626 \cdot 10^{-34}\,\text{Дж}\cdot\text{с}$.
}
\answer{%
    $
        N
            = \frac{E_{\text{общая}}}{E_{\text{одного фотона}}}
            = \frac{Pt}{h\nu} = \frac{Pt}{h \frac c\lambda}
            = \frac{Pt\lambda}{hc}
            = \frac{40\,\text{мВт} \cdot 10\,\text{мин} \cdot 750\,\text{нм}}{6{,}626 \cdot 10^{-34}\,\text{Дж}\cdot\text{с} \cdot 3 \cdot 10^{8}\,\frac{\text{м}}{\text{с}}}
            \approx 0{,}91 \cdot 10^{20}\units{фотонов}
    $
}
\solutionspace{120pt}

\tasknumber{9}%
\task{%
    Какая доля (от начального количества) радиоактивных ядер останется через время,
    равное двум периодам полураспада? Ответ выразить в процентах.
}
\answer{%
    \begin{align*}
    N &= N_0 \cdot 2^{- \frac t{T_{1/2}}} \implies
        \frac N{N_0} = 2^{- \frac t{T_{1/2}}}
        = 2^{-2} \approx 0{,}25 \approx 25\% \\
    N_\text{расп.} &= N_0 - N = N_0 - N_0 \cdot 2^{-\frac t{T_{1/2}}}
        = N_0\cbr{1 - 2^{-\frac t{T_{1/2}}}} \implies
        \frac{N_\text{расп.}}{N_0} = 1 - 2^{-\frac t{T_{1/2}}}
        = 1 - 2^{-2} \approx 0{,}75 \approx 75\%
    \end{align*}
}
\solutionspace{150pt}

\tasknumber{10}%
\task{%
    Сколько процентов ядер радиоактивного железа $\ce{^{59}Fe}$
    останется через $91{,}2\,\text{суток}$, если период его полураспада составляет $45{,}6\,\text{суток}$?
}
\answer{%
    \begin{align*}
    N &= N_0 \cdot 2^{-\frac t{T_{1/2}}}
        = 2^{-\frac{91{,}2\,\text{суток}}{45{,}6\,\text{суток}}}
        \approx 0{,}2500 = 25{,}00\%
    \end{align*}
}
\solutionspace{150pt}

\tasknumber{11}%
\task{%
    За $5\,\text{суток}$ от начального количества ядер радиоизотопа осталась одна шестнадцатая.
    Каков период полураспада этого изотопа (ответ приведите в сутках)?
    Какая ещё доля (также от начального количества) распадётся, если подождать ещё столько же?
}
\answer{%
    \begin{align*}
            N &= N_0 \cdot 2^{-\frac t{T_{1/2}}}
            \implies \frac N{N_0} = 2^{-\frac t{T_{1/2}}}
            \implies \frac 1{16} = 2^{-\frac {5\,\text{суток}}{T_{1/2}}}
            \implies 4 = \frac {5\,\text{суток}}{T_{1/2}}
            \implies T_{1/2} = \frac {5\,\text{суток}}4 \approx 1{,}25\,\text{суток}.
         \\
            \delta &= \frac{N(t)}{N_0} - \frac{N(2t)}{N_0}
            = 2^{-\frac t{T_{1/2}}} - 2^{-\frac {2t}{T_{1/2}}}
            = 2^{-\frac t{T_{1/2}}}\cbr{1 - 2^{-\frac t{T_{1/2}}}}
            = \frac 1{16} \cdot \cbr{1-\frac 1{16}} \approx 0{,}059
    \end{align*}
}
\solutionspace{150pt}

\tasknumber{12}%
\task{%
    Энергия связи ядра трития \ce{^{3}_{1}H} (T) равна $8{,}48\,\text{МэВ}$.
    Найти дефект массы этого ядра.
    Ответ выразите в а.е.м.
    и кг.
    Скорость света $c = 2{,}998 \cdot 10^{8}\,\frac{\text{м}}{\text{с}}$, элементарный заряд $e = 1{,}6 \cdot 10^{-19}\,\text{Кл}$.
}
\answer{%
    \begin{align*}
    E_\text{св.} &= \Delta m c^2 \implies \\
    \implies
            \Delta m &= \frac {E_\text{св.}}{c^2} = \frac{8{,}48\,\text{МэВ}}{\sqr{2{,}998 \cdot 10^{8}\,\frac{\text{м}}{\text{с}}}}
            = \frac{8{,}48 \cdot 10^6 \cdot 1{,}6 \cdot 10^{-19}\,\text{Дж}}{\sqr{2{,}998 \cdot 10^{8}\,\frac{\text{м}}{\text{с}}}}
            \approx 15{,}10 \cdot 10^{-30}\,\text{кг} \approx 0{,}00909\,\text{а.е.м.}
    \end{align*}
}

\variantsplitter

\addpersonalvariant{Дарья Шашкова}

\tasknumber{1}%
\task{%
    Для частицы, движущейся с релятивистской скоростью,
    выразите $p$ и $E_\text{кин}$ через $c$, $E_0$ и $v$,
    где $E_\text{кин}$~--- кинетическая энергия частицы,
    а $E_0$, $p$ и $v$~--- её энергия покоя импульс и скорость.
}
\answer{%
    \begin{align*}
    E_\text{кин}, E_0:\quad&E = E_\text{кин} + E_0 = \frac{E_0}{\sqrt{1 - \frac{v^2}{c^2}}} \implies \sqrt{1 - \frac{v^2}{c^2}} = \frac{E_0}{{E_0} + {E_\text{кин}}} \implies v = c\sqrt{1 - \sqr{\frac{E_0}{{E_0} + {E_\text{кин}}}}} \\
    &p = \frac{mv}{\sqrt{1 - \frac{v^2}{c^2}}} = \frac{E_0}{c^2} \cdot \sqrt{1 - \sqr{\frac{E_0}{{E_0} + {E_\text{кин}}}}} \cdot \frac{{E_\text{кин}} + {E_0}}{E_0} = \frac{E_0}{c^2} \cdot \sqrt{\sqr{\frac{{E_\text{кин}} + {E_0}}{E_0}} - 1}.
    \\
    E_\text{кин}, p:\quad&E_\text{кин} = E - E_0 = mc^2\cbr{\frac 1{\sqrt{1 - \frac{v^2}{c^2}}} - 1}, p = \frac{mv}{\sqrt{1 - \frac{v^2}{c^2}}} \implies \frac{E_\text{кин}}{p} = \frac{\frac 1{\sqrt{1 - \frac{v^2}{c^2}}} - 1}{\sqrt{1 - \frac{v^2}{c^2}}} \implies v = \ldots \\
    &E_0 = E - E_\text{кин} = \frac{E_0}{\sqrt{1 - \frac{v^2}{c^2}}} - E_\text{кин} \implies E_0 = \frac{E_\text{кин}}{\frac 1{\sqrt{1 - \frac{v^2}{c^2}}} - 1} = \ldots \\
    E_\text{кин}, v:\quad&E_\text{кин} = E - E_0 = mc^2\cbr{\frac 1{\sqrt{1 - \frac{v^2}{c^2}}} - 1} \implies m = \frac{E_\text{кин}}{c^2\cbr{\frac 1{\sqrt{1 - \frac{v^2}{c^2}}} - 1}} \\
    &E_0 = mc^2 = \frac{E_\text{кин}}{\frac 1{\sqrt{1 - \frac{v^2}{c^2}}} - 1} \\
    &p = \frac{mv}{\sqrt{1 - \frac{v^2}{c^2}}} = \frac{E_\text{кин}}{c^2\cbr{\frac 1{\sqrt{1 - \frac{v^2}{c^2}}} - 1}} \cdot \frac{v}{\sqrt{1 - \frac{v^2}{c^2}}} = \frac{{E_\text{кин}} v}{c^2\cbr{1 - {\sqrt{1 - \frac{v^2}{c^2}}}}} \\
    E_0, p:\quad&E_0 = mc^2, \quad p = \frac{mv}{\sqrt{1 - \frac{v^2}{c^2}}} \implies \frac{E_0}{p} = \frac{c^2}v{\sqrt{1 - \frac{v^2}{c^2}}} = c\sqrt{\frac{c^2}{v^2} - 1} \\
    &\sqr{\frac{E_0}{pc}} = \frac{c^2}{v^2} - 1 \implies \frac{v^2}{c^2} = \frac 1{1 + \frac{E_0^2}{p^2c^2}} \implies v = \frac c{\sqrt{1 + \frac{E_0^2}{p^2c^2}}} \\
    &{E_\text{кин}} = E - E_0 = \sqrt{E_0^2 + p^2c^2} - E_0 \\
    E_0, v:\quad&E_0 = mc^2 \implies m = \frac{E_0}{c^2} \qquad p = \frac{mv}{\sqrt{1 - \frac{v^2}{c^2}}} = \frac{E_0}{c^2} \cdot \frac{v}{\sqrt{1 - \frac{v^2}{c^2}}} \\
    &E_\text{кин}= mc^2\cbr{\frac 1{\sqrt{1 - \frac{v^2}{c^2}}} - 1} = \frac{E_0}{c^2}\cbr{\frac 1{\sqrt{1 - \frac{v^2}{c^2}}} - 1} \\
    p, v:\quad&p = \frac{mv}{\sqrt{1 - \frac{v^2}{c^2}}} \implies m = \frac p v {\sqrt{1 - \frac{v^2}{c^2}}} \implies E_0 = mc^2 =\frac {pc^2} v {\sqrt{1 - \frac{v^2}{c^2}}} \\
    &E_\text{кин} = mc^2\cbr{\frac 1{\sqrt{1 - \frac{v^2}{c^2}}} - 1} = \frac p v {\sqrt{1 - \frac{v^2}{c^2}}}\cbr{\frac 1{\sqrt{1 - \frac{v^2}{c^2}}} - 1} = \frac p v \cbr{1 - {\sqrt{1 - \frac{v^2}{c^2}}}}
    \end{align*}
}
\solutionspace{200pt}

\tasknumber{2}%
\task{%
    Позитрон движется со скоростью $0{,}9\,c$, где $c$~--- скорость света в вакууме.
    Каково при этом отношение кинетической энергии частицы $E_\text{кин.}$ к его энергии покоя $E_0$?
}
\answer{%
    \begin{align*}
    E &= \frac{E_0}{\sqrt{1 - \frac{v^2}{c^2}}}
            \implies \frac E{E_0}
                = \frac 1{\sqrt{1 - \frac{v^2}{c^2}}}
                = \frac 1{\sqrt{1 - \sqr{0{,}9}}}
                \approx 2{,}294,
         \\
        E_{\text{кин}} &= E - E_0
            \implies \frac{E_{\text{кин}}}{E_0}
                = \frac E{E_0} - 1
                = \frac 1{\sqrt{1 - \frac{v^2}{c^2}}} - 1
                = \frac 1{\sqrt{1 - \sqr{0{,}9}}} - 1
                \approx 1{,}294.
    \end{align*}
}
\solutionspace{150pt}

\tasknumber{3}%
\task{%
    Протон движется со скоростью $0{,}85\,c$, где $c$~--- скорость света в вакууме.
    Определите его импульс (в ответе приведите формулу и укажите численное значение).
}
\answer{%
    \begin{align*}
    E &= \frac{mc^2}{\sqrt{1 - \frac{v^2}{c^2}}}
            \approx \frac{1{,}673 \cdot 10^{-27}\,\text{кг} \cdot \sqr{3 \cdot 10^{8}\,\frac{\text{м}}{\text{с}}}}{\sqrt{1 - 0{,}85^2}}
            \approx 285{,}765 \cdot 10^{-12}\,\text{Дж},
         \\
        E_{\text{кин}} &= \frac{mc^2}{\sqrt{1 - \frac{v^2}{c^2}}} - mc^2
            = mc^2 \cbr{\frac 1{\sqrt{1 - \frac{v^2}{c^2}}} - 1} \approx \\
            &\approx \cbr{1{,}673 \cdot 10^{-27}\,\text{кг} \cdot \sqr{3 \cdot 10^{8}\,\frac{\text{м}}{\text{с}}}}
            \cdot \cbr{\frac 1{\sqrt{1 - 0{,}85^2}} - 1}
            \approx 135{,}229 \cdot 10^{-12}\,\text{Дж},
         \\
        p &= \frac{mv}{\sqrt{1 - \frac{v^2}{c^2}}}
            \approx \frac{1{,}673 \cdot 10^{-27}\,\text{кг} \cdot 0{,}85 \cdot 3 \cdot 10^{8}\,\frac{\text{м}}{\text{с}}}{\sqrt{1 - 0{,}85^2}}
            \approx 809{,}666 \cdot 10^{-21}\,\frac{\text{кг}\cdot\text{м}}{\text{с}}.
    \end{align*}
}
\solutionspace{150pt}

\tasknumber{4}%
\task{%
    При какой скорости движения (в долях скорости света) релятивистское сокращение длины движущегося тела
    составит 30\%?
}
\answer{%
    \begin{align*}
    l_0 &= \frac l{\sqrt{1 - \frac{v^2}{c^2}}}
        \implies 1 - \frac{v^2}{c^2} = \sqr{\frac l{l_0}}
        \implies \frac v c = \sqrt{1 - \sqr{\frac l{l_0}}} \implies
         \\
        \implies v &= c\sqrt{1 - \sqr{\frac l{l_0}}}
        = 3 \cdot 10^{8}\,\frac{\text{м}}{\text{с}} \cdot \sqrt{1 - \sqr{\frac {l_0 - 0{,}30l_0}{l_0}}}
        = 3 \cdot 10^{8}\,\frac{\text{м}}{\text{с}} \cdot \sqrt{1 - \sqr{1 - 0{,}30}} \approx  \\
        &\approx 0{,}714c
        \approx 214 \cdot 10^{6}\,\frac{\text{м}}{\text{с}}
        \approx 771 \cdot 10^{6}\,\frac{\text{км}}{\text{ч}}.
    \end{align*}
}
\solutionspace{150pt}

\tasknumber{5}%
\task{%
    При переходе электрона в атоме с одной стационарной орбиты на другую
    излучается фотон с энергией $5{,}05 \cdot 10^{-19}\,\text{Дж}$.
    Какова длина волны этой линии спектра?
    Постоянная Планка $h = 6{,}626 \cdot 10^{-34}\,\text{Дж}\cdot\text{с}$, скорость света $c = 3 \cdot 10^{8}\,\frac{\text{м}}{\text{с}}$.
}
\answer{%
    $
        E = h\nu = h \frac c\lambda
        \implies \lambda = \frac{hc}E
            = \frac{6{,}626 \cdot 10^{-34}\,\text{Дж}\cdot\text{с} \cdot {3 \cdot 10^{8}\,\frac{\text{м}}{\text{с}}}}{5{,}05 \cdot 10^{-19}\,\text{Дж}}
            = 393{,}62\,\text{нм}.
    $
}
\solutionspace{150pt}

\tasknumber{6}%
\task{%
    Излучение какой длины волны поглотил атом водорода, если полная энергия в атоме увеличилась на $6 \cdot 10^{-19}\,\text{Дж}$?
    Постоянная Планка $h = 6{,}626 \cdot 10^{-34}\,\text{Дж}\cdot\text{с}$, скорость света $c = 3 \cdot 10^{8}\,\frac{\text{м}}{\text{с}}$.
}
\answer{%
    $
        E = h\nu = h \frac c\lambda
        \implies \lambda = \frac{hc}E
            = \frac{6{,}626 \cdot 10^{-34}\,\text{Дж}\cdot\text{с} \cdot {3 \cdot 10^{8}\,\frac{\text{м}}{\text{с}}}}{6 \cdot 10^{-19}\,\text{Дж}}
            = 331\,\text{нм}.
    $
}
\solutionspace{150pt}

\tasknumber{7}%
\task{%
    Сделайте схематичный рисунок энергетических уровней атома водорода
    и отметьте на нём первый (основной) уровень и последующие.
    Сколько различных длин волн может испустить атом водорода,
    находящийся в 3-м возбуждённом состоянии?
    Отметьте все соответствующие переходы на рисунке и укажите,
    при каком переходе (среди отмеченных) частота излучённого фотона минимальна.
}
\answer{%
    $N = 3{,}0, \text{самая короткая линия}$
}
\solutionspace{150pt}

\tasknumber{8}%
\task{%
    Сколько фотонов испускает за $60\,\text{мин}$ лазер,
    если мощность его излучения $40\,\text{мВт}$?
    Длина волны излучения $600\,\text{нм}$.
    $h = 6{,}626 \cdot 10^{-34}\,\text{Дж}\cdot\text{с}$.
}
\answer{%
    $
        N
            = \frac{E_{\text{общая}}}{E_{\text{одного фотона}}}
            = \frac{Pt}{h\nu} = \frac{Pt}{h \frac c\lambda}
            = \frac{Pt\lambda}{hc}
            = \frac{40\,\text{мВт} \cdot 60\,\text{мин} \cdot 600\,\text{нм}}{6{,}626 \cdot 10^{-34}\,\text{Дж}\cdot\text{с} \cdot 3 \cdot 10^{8}\,\frac{\text{м}}{\text{с}}}
            \approx 4{,}35 \cdot 10^{20}\units{фотонов}
    $
}
\solutionspace{120pt}

\tasknumber{9}%
\task{%
    Какая доля (от начального количества) радиоактивных ядер останется через время,
    равное трём периодам полураспада? Ответ выразить в процентах.
}
\answer{%
    \begin{align*}
    N &= N_0 \cdot 2^{- \frac t{T_{1/2}}} \implies
        \frac N{N_0} = 2^{- \frac t{T_{1/2}}}
        = 2^{-3} \approx 0{,}12 \approx 12\% \\
    N_\text{расп.} &= N_0 - N = N_0 - N_0 \cdot 2^{-\frac t{T_{1/2}}}
        = N_0\cbr{1 - 2^{-\frac t{T_{1/2}}}} \implies
        \frac{N_\text{расп.}}{N_0} = 1 - 2^{-\frac t{T_{1/2}}}
        = 1 - 2^{-3} \approx 0{,}88 \approx 88\%
    \end{align*}
}
\solutionspace{150pt}

\tasknumber{10}%
\task{%
    Сколько процентов ядер радиоактивного железа $\ce{^{59}Fe}$
    останется через $136{,}8\,\text{суток}$, если период его полураспада составляет $45{,}6\,\text{суток}$?
}
\answer{%
    \begin{align*}
    N &= N_0 \cdot 2^{-\frac t{T_{1/2}}}
        = 2^{-\frac{136{,}8\,\text{суток}}{45{,}6\,\text{суток}}}
        \approx 0{,}1250 = 12{,}50\%
    \end{align*}
}
\solutionspace{150pt}

\tasknumber{11}%
\task{%
    За $4\,\text{суток}$ от начального количества ядер радиоизотопа осталась одна шестнадцатая.
    Каков период полураспада этого изотопа (ответ приведите в сутках)?
    Какая ещё доля (также от начального количества) распадётся, если подождать ещё столько же?
}
\answer{%
    \begin{align*}
            N &= N_0 \cdot 2^{-\frac t{T_{1/2}}}
            \implies \frac N{N_0} = 2^{-\frac t{T_{1/2}}}
            \implies \frac 1{16} = 2^{-\frac {4\,\text{суток}}{T_{1/2}}}
            \implies 4 = \frac {4\,\text{суток}}{T_{1/2}}
            \implies T_{1/2} = \frac {4\,\text{суток}}4 \approx 1\,\text{суток}.
         \\
            \delta &= \frac{N(t)}{N_0} - \frac{N(2t)}{N_0}
            = 2^{-\frac t{T_{1/2}}} - 2^{-\frac {2t}{T_{1/2}}}
            = 2^{-\frac t{T_{1/2}}}\cbr{1 - 2^{-\frac t{T_{1/2}}}}
            = \frac 1{16} \cdot \cbr{1-\frac 1{16}} \approx 0{,}059
    \end{align*}
}
\solutionspace{150pt}

\tasknumber{12}%
\task{%
    Энергия связи ядра дейтерия \ce{^{2}_{1}H} (D) равна $2{,}22\,\text{МэВ}$.
    Найти дефект массы этого ядра.
    Ответ выразите в а.е.м.
    и кг.
    Скорость света $c = 2{,}998 \cdot 10^{8}\,\frac{\text{м}}{\text{с}}$, элементарный заряд $e = 1{,}6 \cdot 10^{-19}\,\text{Кл}$.
}
\answer{%
    \begin{align*}
    E_\text{св.} &= \Delta m c^2 \implies \\
    \implies
            \Delta m &= \frac {E_\text{св.}}{c^2} = \frac{2{,}22\,\text{МэВ}}{\sqr{2{,}998 \cdot 10^{8}\,\frac{\text{м}}{\text{с}}}}
            = \frac{2{,}22 \cdot 10^6 \cdot 1{,}6 \cdot 10^{-19}\,\text{Дж}}{\sqr{2{,}998 \cdot 10^{8}\,\frac{\text{м}}{\text{с}}}}
            \approx 3{,}95 \cdot 10^{-30}\,\text{кг} \approx 0{,}00238\,\text{а.е.м.}
    \end{align*}
}

\variantsplitter

\addpersonalvariant{Алина Яшина}

\tasknumber{1}%
\task{%
    Для частицы, движущейся с релятивистской скоростью,
    выразите $E_0$ и $E_\text{кин}$ через $c$, $p$ и $v$,
    где $E_\text{кин}$~--- кинетическая энергия частицы,
    а $E_0$, $p$ и $v$~--- её энергия покоя импульс и скорость.
}
\answer{%
    \begin{align*}
    E_\text{кин}, E_0:\quad&E = E_\text{кин} + E_0 = \frac{E_0}{\sqrt{1 - \frac{v^2}{c^2}}} \implies \sqrt{1 - \frac{v^2}{c^2}} = \frac{E_0}{{E_0} + {E_\text{кин}}} \implies v = c\sqrt{1 - \sqr{\frac{E_0}{{E_0} + {E_\text{кин}}}}} \\
    &p = \frac{mv}{\sqrt{1 - \frac{v^2}{c^2}}} = \frac{E_0}{c^2} \cdot \sqrt{1 - \sqr{\frac{E_0}{{E_0} + {E_\text{кин}}}}} \cdot \frac{{E_\text{кин}} + {E_0}}{E_0} = \frac{E_0}{c^2} \cdot \sqrt{\sqr{\frac{{E_\text{кин}} + {E_0}}{E_0}} - 1}.
    \\
    E_\text{кин}, p:\quad&E_\text{кин} = E - E_0 = mc^2\cbr{\frac 1{\sqrt{1 - \frac{v^2}{c^2}}} - 1}, p = \frac{mv}{\sqrt{1 - \frac{v^2}{c^2}}} \implies \frac{E_\text{кин}}{p} = \frac{\frac 1{\sqrt{1 - \frac{v^2}{c^2}}} - 1}{\sqrt{1 - \frac{v^2}{c^2}}} \implies v = \ldots \\
    &E_0 = E - E_\text{кин} = \frac{E_0}{\sqrt{1 - \frac{v^2}{c^2}}} - E_\text{кин} \implies E_0 = \frac{E_\text{кин}}{\frac 1{\sqrt{1 - \frac{v^2}{c^2}}} - 1} = \ldots \\
    E_\text{кин}, v:\quad&E_\text{кин} = E - E_0 = mc^2\cbr{\frac 1{\sqrt{1 - \frac{v^2}{c^2}}} - 1} \implies m = \frac{E_\text{кин}}{c^2\cbr{\frac 1{\sqrt{1 - \frac{v^2}{c^2}}} - 1}} \\
    &E_0 = mc^2 = \frac{E_\text{кин}}{\frac 1{\sqrt{1 - \frac{v^2}{c^2}}} - 1} \\
    &p = \frac{mv}{\sqrt{1 - \frac{v^2}{c^2}}} = \frac{E_\text{кин}}{c^2\cbr{\frac 1{\sqrt{1 - \frac{v^2}{c^2}}} - 1}} \cdot \frac{v}{\sqrt{1 - \frac{v^2}{c^2}}} = \frac{{E_\text{кин}} v}{c^2\cbr{1 - {\sqrt{1 - \frac{v^2}{c^2}}}}} \\
    E_0, p:\quad&E_0 = mc^2, \quad p = \frac{mv}{\sqrt{1 - \frac{v^2}{c^2}}} \implies \frac{E_0}{p} = \frac{c^2}v{\sqrt{1 - \frac{v^2}{c^2}}} = c\sqrt{\frac{c^2}{v^2} - 1} \\
    &\sqr{\frac{E_0}{pc}} = \frac{c^2}{v^2} - 1 \implies \frac{v^2}{c^2} = \frac 1{1 + \frac{E_0^2}{p^2c^2}} \implies v = \frac c{\sqrt{1 + \frac{E_0^2}{p^2c^2}}} \\
    &{E_\text{кин}} = E - E_0 = \sqrt{E_0^2 + p^2c^2} - E_0 \\
    E_0, v:\quad&E_0 = mc^2 \implies m = \frac{E_0}{c^2} \qquad p = \frac{mv}{\sqrt{1 - \frac{v^2}{c^2}}} = \frac{E_0}{c^2} \cdot \frac{v}{\sqrt{1 - \frac{v^2}{c^2}}} \\
    &E_\text{кин}= mc^2\cbr{\frac 1{\sqrt{1 - \frac{v^2}{c^2}}} - 1} = \frac{E_0}{c^2}\cbr{\frac 1{\sqrt{1 - \frac{v^2}{c^2}}} - 1} \\
    p, v:\quad&p = \frac{mv}{\sqrt{1 - \frac{v^2}{c^2}}} \implies m = \frac p v {\sqrt{1 - \frac{v^2}{c^2}}} \implies E_0 = mc^2 =\frac {pc^2} v {\sqrt{1 - \frac{v^2}{c^2}}} \\
    &E_\text{кин} = mc^2\cbr{\frac 1{\sqrt{1 - \frac{v^2}{c^2}}} - 1} = \frac p v {\sqrt{1 - \frac{v^2}{c^2}}}\cbr{\frac 1{\sqrt{1 - \frac{v^2}{c^2}}} - 1} = \frac p v \cbr{1 - {\sqrt{1 - \frac{v^2}{c^2}}}}
    \end{align*}
}
\solutionspace{200pt}

\tasknumber{2}%
\task{%
    Электрон движется со скоростью $0{,}7\,c$, где $c$~--- скорость света в вакууме.
    Каково при этом отношение кинетической энергии частицы $E_\text{кин.}$ к его энергии покоя $E_0$?
}
\answer{%
    \begin{align*}
    E &= \frac{E_0}{\sqrt{1 - \frac{v^2}{c^2}}}
            \implies \frac E{E_0}
                = \frac 1{\sqrt{1 - \frac{v^2}{c^2}}}
                = \frac 1{\sqrt{1 - \sqr{0{,}7}}}
                \approx 1{,}400,
         \\
        E_{\text{кин}} &= E - E_0
            \implies \frac{E_{\text{кин}}}{E_0}
                = \frac E{E_0} - 1
                = \frac 1{\sqrt{1 - \frac{v^2}{c^2}}} - 1
                = \frac 1{\sqrt{1 - \sqr{0{,}7}}} - 1
                \approx 0{,}400.
    \end{align*}
}
\solutionspace{150pt}

\tasknumber{3}%
\task{%
    Электрон движется со скоростью $0{,}75\,c$, где $c$~--- скорость света в вакууме.
    Определите его кинетическую энергию (в ответе приведите формулу и укажите численное значение).
}
\answer{%
    \begin{align*}
    E &= \frac{mc^2}{\sqrt{1 - \frac{v^2}{c^2}}}
            \approx \frac{9{,}1 \cdot 10^{-31}\,\text{кг} \cdot \sqr{3 \cdot 10^{8}\,\frac{\text{м}}{\text{с}}}}{\sqrt{1 - 0{,}75^2}}
            \approx 0{,}124 \cdot 10^{-12}\,\text{Дж},
         \\
        E_{\text{кин}} &= \frac{mc^2}{\sqrt{1 - \frac{v^2}{c^2}}} - mc^2
            = mc^2 \cbr{\frac 1{\sqrt{1 - \frac{v^2}{c^2}}} - 1} \approx \\
            &\approx \cbr{9{,}1 \cdot 10^{-31}\,\text{кг} \cdot \sqr{3 \cdot 10^{8}\,\frac{\text{м}}{\text{с}}}}
            \cdot \cbr{\frac 1{\sqrt{1 - 0{,}75^2}} - 1}
            \approx 0{,}042 \cdot 10^{-12}\,\text{Дж},
         \\
        p &= \frac{mv}{\sqrt{1 - \frac{v^2}{c^2}}}
            \approx \frac{9{,}1 \cdot 10^{-31}\,\text{кг} \cdot 0{,}75 \cdot 3 \cdot 10^{8}\,\frac{\text{м}}{\text{с}}}{\sqrt{1 - 0{,}75^2}}
            \approx 0{,}310 \cdot 10^{-21}\,\frac{\text{кг}\cdot\text{м}}{\text{с}}.
    \end{align*}
}
\solutionspace{150pt}

\tasknumber{4}%
\task{%
    При какой скорости движения (в долях скорости света) релятивистское сокращение длины движущегося тела
    составит 10\%?
}
\answer{%
    \begin{align*}
    l_0 &= \frac l{\sqrt{1 - \frac{v^2}{c^2}}}
        \implies 1 - \frac{v^2}{c^2} = \sqr{\frac l{l_0}}
        \implies \frac v c = \sqrt{1 - \sqr{\frac l{l_0}}} \implies
         \\
        \implies v &= c\sqrt{1 - \sqr{\frac l{l_0}}}
        = 3 \cdot 10^{8}\,\frac{\text{м}}{\text{с}} \cdot \sqrt{1 - \sqr{\frac {l_0 - 0{,}10l_0}{l_0}}}
        = 3 \cdot 10^{8}\,\frac{\text{м}}{\text{с}} \cdot \sqrt{1 - \sqr{1 - 0{,}10}} \approx  \\
        &\approx 0{,}436c
        \approx 130{,}8 \cdot 10^{6}\,\frac{\text{м}}{\text{с}}
        \approx 471 \cdot 10^{6}\,\frac{\text{км}}{\text{ч}}.
    \end{align*}
}
\solutionspace{150pt}

\tasknumber{5}%
\task{%
    При переходе электрона в атоме с одной стационарной орбиты на другую
    излучается фотон с энергией $2{,}02 \cdot 10^{-19}\,\text{Дж}$.
    Какова длина волны этой линии спектра?
    Постоянная Планка $h = 6{,}626 \cdot 10^{-34}\,\text{Дж}\cdot\text{с}$, скорость света $c = 3 \cdot 10^{8}\,\frac{\text{м}}{\text{с}}$.
}
\answer{%
    $
        E = h\nu = h \frac c\lambda
        \implies \lambda = \frac{hc}E
            = \frac{6{,}626 \cdot 10^{-34}\,\text{Дж}\cdot\text{с} \cdot {3 \cdot 10^{8}\,\frac{\text{м}}{\text{с}}}}{2{,}02 \cdot 10^{-19}\,\text{Дж}}
            = 984{,}06\,\text{нм}.
    $
}
\solutionspace{150pt}

\tasknumber{6}%
\task{%
    Излучение какой длины волны поглотил атом водорода, если полная энергия в атоме увеличилась на $4 \cdot 10^{-19}\,\text{Дж}$?
    Постоянная Планка $h = 6{,}626 \cdot 10^{-34}\,\text{Дж}\cdot\text{с}$, скорость света $c = 3 \cdot 10^{8}\,\frac{\text{м}}{\text{с}}$.
}
\answer{%
    $
        E = h\nu = h \frac c\lambda
        \implies \lambda = \frac{hc}E
            = \frac{6{,}626 \cdot 10^{-34}\,\text{Дж}\cdot\text{с} \cdot {3 \cdot 10^{8}\,\frac{\text{м}}{\text{с}}}}{4 \cdot 10^{-19}\,\text{Дж}}
            = 497\,\text{нм}.
    $
}
\solutionspace{150pt}

\tasknumber{7}%
\task{%
    Сделайте схематичный рисунок энергетических уровней атома водорода
    и отметьте на нём первый (основной) уровень и последующие.
    Сколько различных длин волн может испустить атом водорода,
    находящийся в 5-м возбуждённом состоянии?
    Отметьте все соответствующие переходы на рисунке и укажите,
    при каком переходе (среди отмеченных) частота излучённого фотона минимальна.
}
\answer{%
    $N = 10{,}0, \text{самая короткая линия}$
}
\solutionspace{150pt}

\tasknumber{8}%
\task{%
    Сколько фотонов испускает за $10\,\text{мин}$ лазер,
    если мощность его излучения $40\,\text{мВт}$?
    Длина волны излучения $500\,\text{нм}$.
    $h = 6{,}626 \cdot 10^{-34}\,\text{Дж}\cdot\text{с}$.
}
\answer{%
    $
        N
            = \frac{E_{\text{общая}}}{E_{\text{одного фотона}}}
            = \frac{Pt}{h\nu} = \frac{Pt}{h \frac c\lambda}
            = \frac{Pt\lambda}{hc}
            = \frac{40\,\text{мВт} \cdot 10\,\text{мин} \cdot 500\,\text{нм}}{6{,}626 \cdot 10^{-34}\,\text{Дж}\cdot\text{с} \cdot 3 \cdot 10^{8}\,\frac{\text{м}}{\text{с}}}
            \approx 0{,}60 \cdot 10^{20}\units{фотонов}
    $
}
\solutionspace{120pt}

\tasknumber{9}%
\task{%
    Какая доля (от начального количества) радиоактивных ядер останется через время,
    равное трём периодам полураспада? Ответ выразить в процентах.
}
\answer{%
    \begin{align*}
    N &= N_0 \cdot 2^{- \frac t{T_{1/2}}} \implies
        \frac N{N_0} = 2^{- \frac t{T_{1/2}}}
        = 2^{-3} \approx 0{,}12 \approx 12\% \\
    N_\text{расп.} &= N_0 - N = N_0 - N_0 \cdot 2^{-\frac t{T_{1/2}}}
        = N_0\cbr{1 - 2^{-\frac t{T_{1/2}}}} \implies
        \frac{N_\text{расп.}}{N_0} = 1 - 2^{-\frac t{T_{1/2}}}
        = 1 - 2^{-3} \approx 0{,}88 \approx 88\%
    \end{align*}
}
\solutionspace{150pt}

\tasknumber{10}%
\task{%
    Сколько процентов ядер радиоактивного железа $\ce{^{59}Fe}$
    останется через $182{,}4\,\text{суток}$, если период его полураспада составляет $45{,}6\,\text{суток}$?
}
\answer{%
    \begin{align*}
    N &= N_0 \cdot 2^{-\frac t{T_{1/2}}}
        = 2^{-\frac{182{,}4\,\text{суток}}{45{,}6\,\text{суток}}}
        \approx 0{,}0625 = 6{,}25\%
    \end{align*}
}
\solutionspace{150pt}

\tasknumber{11}%
\task{%
    За $4\,\text{суток}$ от начального количества ядер радиоизотопа осталась четверть.
    Каков период полураспада этого изотопа (ответ приведите в сутках)?
    Какая ещё доля (также от начального количества) распадётся, если подождать ещё столько же?
}
\answer{%
    \begin{align*}
            N &= N_0 \cdot 2^{-\frac t{T_{1/2}}}
            \implies \frac N{N_0} = 2^{-\frac t{T_{1/2}}}
            \implies \frac 1{4} = 2^{-\frac {4\,\text{суток}}{T_{1/2}}}
            \implies 2 = \frac {4\,\text{суток}}{T_{1/2}}
            \implies T_{1/2} = \frac {4\,\text{суток}}2 \approx 2\,\text{суток}.
         \\
            \delta &= \frac{N(t)}{N_0} - \frac{N(2t)}{N_0}
            = 2^{-\frac t{T_{1/2}}} - 2^{-\frac {2t}{T_{1/2}}}
            = 2^{-\frac t{T_{1/2}}}\cbr{1 - 2^{-\frac t{T_{1/2}}}}
            = \frac 1{4} \cdot \cbr{1-\frac 1{4}} \approx 0{,}188
    \end{align*}
}
\solutionspace{150pt}

\tasknumber{12}%
\task{%
    Энергия связи ядра углерода \ce{^{13}_{6}C} равна $97{,}1\,\text{МэВ}$.
    Найти дефект массы этого ядра.
    Ответ выразите в а.е.м.
    и кг.
    Скорость света $c = 2{,}998 \cdot 10^{8}\,\frac{\text{м}}{\text{с}}$, элементарный заряд $e = 1{,}6 \cdot 10^{-19}\,\text{Кл}$.
}
\answer{%
    \begin{align*}
    E_\text{св.} &= \Delta m c^2 \implies \\
    \implies
            \Delta m &= \frac {E_\text{св.}}{c^2} = \frac{97{,}1\,\text{МэВ}}{\sqr{2{,}998 \cdot 10^{8}\,\frac{\text{м}}{\text{с}}}}
            = \frac{97{,}1 \cdot 10^6 \cdot 1{,}6 \cdot 10^{-19}\,\text{Дж}}{\sqr{2{,}998 \cdot 10^{8}\,\frac{\text{м}}{\text{с}}}}
            \approx 0{,}1729 \cdot 10^{-27}\,\text{кг} \approx 0{,}1041\,\text{а.е.м.}
    \end{align*}
}
% autogenerated
