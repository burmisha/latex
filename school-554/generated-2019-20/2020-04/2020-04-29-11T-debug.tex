\newcommand\rootpath{../../..}
\documentclass[12pt,a4paper]{amsart}%DVI-mode.
\usepackage{graphics,graphicx,epsfig}%DVI-mode.
% \documentclass[pdftex,12pt]{amsart} %PDF-mode.
% \usepackage[pdftex]{graphicx}       %PDF-mode.
% \usepackage[babel=true]{microtype}
% \usepackage[T1]{fontenc}
% \usepackage{lmodern}

\usepackage{cmap}
%\usepackage{a4wide}                 % Fit the text to A4 page tightly.
% \usepackage[utf8]{inputenc}
\usepackage[T2A]{fontenc}
\usepackage[english,russian]{babel} % Download Russian fonts.
\usepackage{amsmath,amsfonts,amssymb,amsthm,amscd,mathrsfs} % Use AMS symbols.
\usepackage{tikz}
\usetikzlibrary{circuits.ee.IEC}
\usetikzlibrary{shapes.geometric}
\usetikzlibrary{decorations.markings}
%\usetikzlibrary{dashs}
%\usetikzlibrary{info}


\textheight=28cm % высота текста
\textwidth=18cm % ширина текста
\topmargin=-2.5cm % отступ от верхнего края
\parskip=2pt % интервал между абзацами
\oddsidemargin=-1.5cm
\evensidemargin=-1.5cm 

\parindent=0pt % абзацный отступ
\tolerance=500 % терпимость к "жидким" строкам
\binoppenalty=10000 % штраф за перенос формул - 10000 - абсолютный запрет
\relpenalty=10000
\flushbottom % выравнивание высоты страниц
\pagenumbering{gobble}

\newcommand\bivec[2]{\begin{pmatrix} #1 \\ #2 \end{pmatrix}}

\newcommand\ol[1]{\overline{#1}}

\newcommand\p[1]{\Prob\!\left(#1\right)}
\newcommand\e[1]{\mathsf{E}\!\left(#1\right)}
\newcommand\disp[1]{\mathsf{D}\!\left(#1\right)}
%\newcommand\norm[2]{\mathcal{N}\!\cbr{#1,#2}}
\newcommand\sign{\text{ sign }}

\newcommand\al[1]{\begin{align*} #1 \end{align*}}
\newcommand\begcas[1]{\begin{cases}#1\end{cases}}
\newcommand\tab[2]{	\vspace{-#1pt}
						\begin{tabbing} 
						#2
						\end{tabbing}
					\vspace{-#1pt}
					}

\newcommand\maintext[1]{{\bfseries\sffamily{#1}}}
\newcommand\skipped[1]{ {\ensuremath{\text{\small{\sffamily{Пропущено:} #1} } } } }
\newcommand\simpletitle[1]{\begin{center} \maintext{#1} \end{center}}

\def\le{\leqslant}
\def\ge{\geqslant}
\def\Ell{\mathcal{L}}
\def\eps{{\varepsilon}}
\def\Rn{\mathbb{R}^n}
\def\RSS{\mathsf{RSS}}

\newcommand\foral[1]{\forall\,#1\:}
\newcommand\exist[1]{\exists\,#1\:\colon}

\newcommand\cbr[1]{\left(#1\right)} %circled brackets
\newcommand\fbr[1]{\left\{#1\right\}} %figure brackets
\newcommand\sbr[1]{\left[#1\right]} %square brackets
\newcommand\modul[1]{\left|#1\right|}

\newcommand\sqr[1]{\cbr{#1}^2}
\newcommand\inv[1]{\cbr{#1}^{-1}}

\newcommand\cdf[2]{\cdot\frac{#1}{#2}}
\newcommand\dd[2]{\frac{\partial#1}{\partial#2}}

\newcommand\integr[2]{\int\limits_{#1}^{#2}}
\newcommand\suml[2]{\sum\limits_{#1}^{#2}}
\newcommand\isum[2]{\sum\limits_{#1=#2}^{+\infty}}
\newcommand\idots[3]{#1_{#2},\ldots,#1_{#3}}
\newcommand\fdots[5]{#4{#1_{#2}}#5\ldots#5#4{#1_{#3}}}

\newcommand\obol[1]{O\!\cbr{#1}}
\newcommand\omal[1]{o\!\cbr{#1}}

\newcommand\addeps[2]{
	\begin{figure} [!ht] %lrp
		\centering
		\includegraphics[height=320px]{#1.eps}
		\vspace{-10pt}
		\caption{#2}
		\label{eps:#1}
	\end{figure}
}

\newcommand\addepssize[3]{
	\begin{figure} [!ht] %lrp hp
		\centering
		\includegraphics[height=#3px]{#1.eps}
		\vspace{-10pt}
		\caption{#2}
		\label{eps:#1}
	\end{figure}
}


\newcommand\norm[1]{\ensuremath{\left\|{#1}\right\|}}
\newcommand\ort{\bot}
\newcommand\theorem[1]{{\sffamily Теорема #1\ }}
\newcommand\lemma[1]{{\sffamily Лемма #1\ }}
\newcommand\difflim[2]{\frac{#1\cbr{#2 + \Delta#2} - #1\cbr{#2}}{\Delta #2}}
\renewcommand\proof[1]{\par\noindent$\square$ #1 \hfill$\blacksquare$\par}
\newcommand\defenition[1]{{\sffamilyОпределение #1\ }}

% \begin{document}
% %\raggedright
% \addclassdate{7}{20 апреля 2018}

\task 1
Площадь большого поршня гидравлического домкрата $S_1 = 20\units{см}^2$, а малого $S_2 = 0{,}5\units{см}^2.$ Груз какой максимальной массы можно поднять этим домкратом, если на малый поршень давить с силой не более $F=200\units{Н}?$ Силой трения от стенки цилиндров пренебречь.

\task 2
В сосуд налита вода. Расстояние от поверхности воды до дна $H = 0{,}5\units{м},$ площадь дна $S = 0{,}1\units{м}^2.$ Найти гидростатическое давление $P_1$ и полное давление $P_2$ вблизи дна. Найти силу давления воды на дно. Плотность воды \rhowater

\task 3
На лёгкий поршень площадью $S=900\units{см}^2,$ касающийся поверхности воды, поставили гирю массы $m=3\units{кг}$. Высота слоя воды в сосуде с вертикальными стенками $H = 20\units{см}$. Определить давление жидкости вблизи дна, если плотность воды \rhowater

\task 4
Давление газов в конце сгорания в цилиндре дизельного двигателя трактора $P = 9\units{МПа}.$ Диаметр цилиндра $d = 130\units{мм}.$ С какой силой газы давят на поршень в цилиндре? Площадь круга диаметром $D$ равна $S = \cfrac{\pi D^2}4.$

\task 5
Площадь малого поршня гидравлического подъёмника $S_1 = 0{,}8\units{см}^2$, а большого $S_2 = 40\units{см}^2.$ Какую силу $F$ надо приложить к малому поршню, чтобы поднять груз весом $P = 8\units{кН}?$

\task 6
Герметичный сосуд полностью заполнен водой и стоит на столе. На небольшой поршень площадью $S$ давят рукой с силой $F$. Поршень находится ниже крышки сосуда на $H_1$, выше дна на $H_2$ и может свободно перемещаться. Плотность воды $\rho$, атмосферное давление $P_A$. Найти давления $P_1$ и $P_2$ в воде вблизи крышки и дна сосуда.
\\ \\
\addclassdate{7}{20 апреля 2018}

\task 1
Площадь большого поршня гидравлического домкрата $S_1 = 20\units{см}^2$, а малого $S_2 = 0{,}5\units{см}^2.$ Груз какой максимальной массы можно поднять этим домкратом, если на малый поршень давить с силой не более $F=200\units{Н}?$ Силой трения от стенки цилиндров пренебречь.

\task 2
В сосуд налита вода. Расстояние от поверхности воды до дна $H = 0{,}5\units{м},$ площадь дна $S = 0{,}1\units{м}^2.$ Найти гидростатическое давление $P_1$ и полное давление $P_2$ вблизи дна. Найти силу давления воды на дно. Плотность воды \rhowater

\task 3
На лёгкий поршень площадью $S=900\units{см}^2,$ касающийся поверхности воды, поставили гирю массы $m=3\units{кг}$. Высота слоя воды в сосуде с вертикальными стенками $H = 20\units{см}$. Определить давление жидкости вблизи дна, если плотность воды \rhowater

\task 4
Давление газов в конце сгорания в цилиндре дизельного двигателя трактора $P = 9\units{МПа}.$ Диаметр цилиндра $d = 130\units{мм}.$ С какой силой газы давят на поршень в цилиндре? Площадь круга диаметром $D$ равна $S = \cfrac{\pi D^2}4.$

\task 5
Площадь малого поршня гидравлического подъёмника $S_1 = 0{,}8\units{см}^2$, а большого $S_2 = 40\units{см}^2.$ Какую силу $F$ надо приложить к малому поршню, чтобы поднять груз весом $P = 8\units{кН}?$

\task 6
Герметичный сосуд полностью заполнен водой и стоит на столе. На небольшой поршень площадью $S$ давят рукой с силой $F$. Поршень находится ниже крышки сосуда на $H_1$, выше дна на $H_2$ и может свободно перемещаться. Плотность воды $\rho$, атмосферное давление $P_A$. Найти давления $P_1$ и $P_2$ в воде вблизи крышки и дна сосуда.

\newpage

\adddate{8 класс. 20 апреля 2018}

\task 1
Между точками $A$ и $B$ электрической цепи подключены последовательно резисторы $R_1 = 10\units{Ом}$ и $R_2 = 20\units{Ом}$ и параллельно им $R_3 = 30\units{Ом}.$ Найдите эквивалентное сопротивление $R_{AB}$ этого участка цепи.

\task 2
Электрическая цепь состоит из последовательности $N$ одинаковых звеньев, в которых каждый резистор имеет сопротивление $r$. Последнее звено замкнуто резистором сопротивлением $R$. При каком соотношении $\cfrac{R}{r}$ сопротивление цепи не зависит от числа звеньев?

\task 3
Для измерения сопротивления $R$ проводника собрана электрическая цепь. Вольтметр $V$ показывает напряжение $U_V = 5\units{В},$ показание амперметра $A$ равно $I_A = 25\units{мА}.$ Найдите величину $R$ сопротивления проводника. Внутреннее сопротивление вольтметра $R_V = 1{,}0\units{кОм},$ внутреннее сопротивление амперметра $R_A = 2{,}0\units{Ом}.$

\task 4
Шкала гальванометра имеет $N=100$ делений, цена деления $\delta = 1\units{мкА}$. Внутреннее сопротивление гальванометра $R_G = 1{,}0\units{кОм}.$ Как из этого прибора сделать вольтметр для измерения напряжений до $U = 100\units{В}$ или амперметр для измерения токов силой до $I = 1\units{А}?$

\\ \\ \\ \\ \\ \\ \\ \\
\adddate{8 класс. 20 апреля 2018}

\task 1
Между точками $A$ и $B$ электрической цепи подключены последовательно резисторы $R_1 = 10\units{Ом}$ и $R_2 = 20\units{Ом}$ и параллельно им $R_3 = 30\units{Ом}.$ Найдите эквивалентное сопротивление $R_{AB}$ этого участка цепи.

\task 2
Электрическая цепь состоит из последовательности $N$ одинаковых звеньев, в которых каждый резистор имеет сопротивление $r$. Последнее звено замкнуто резистором сопротивлением $R$. При каком соотношении $\cfrac{R}{r}$ сопротивление цепи не зависит от числа звеньев?

\task 3
Для измерения сопротивления $R$ проводника собрана электрическая цепь. Вольтметр $V$ показывает напряжение $U_V = 5\units{В},$ показание амперметра $A$ равно $I_A = 25\units{мА}.$ Найдите величину $R$ сопротивления проводника. Внутреннее сопротивление вольтметра $R_V = 1{,}0\units{кОм},$ внутреннее сопротивление амперметра $R_A = 2{,}0\units{Ом}.$

\task 4
Шкала гальванометра имеет $N=100$ делений, цена деления $\delta = 1\units{мкА}$. Внутреннее сопротивление гальванометра $R_G = 1{,}0\units{кОм}.$ Как из этого прибора сделать вольтметр для измерения напряжений до $U = 100\units{В}$ или амперметр для измерения токов силой до $I = 1\units{А}?$


% % \begin{flushright}
\textsc{ГБОУ школа №554, 20 ноября 2018\,г.}
\end{flushright}

\begin{center}
\LARGE \textsc{Математический бой, 8 класс}
\end{center}

\problem{1} Есть тридцать карточек, на каждой написано по одному числу: на десяти карточках~–~$a$,  на десяти других~–~$b$ и на десяти оставшихся~–~$c$ (числа  различны). Известно, что к любым пяти карточкам можно подобрать ещё пять так, что сумма чисел на этих десяти карточках будет равна нулю. Докажите, что~одно из~чисел~$a, b, c$ равно нулю.

\problem{2} Вокруг стола стола пустили пакет с орешками. Первый взял один орешек, второй — 2, третий — 3 и так далее: каждый следующий брал на 1 орешек больше. Известно, что на втором круге было взято в сумме на 100 орешков больше, чем на первом. Сколько человек сидело за столом?

% \problem{2} Натуральное число разрешено увеличить на любое целое число процентов от 1 до 100, если при этом получаем натуральное число. Найдите наименьшее натуральное число, которое нельзя при помощи таких операций получить из~числа 1.

% \problem{3} Найти сумму $1^2 - 2^2 + 3^2 - 4^2 + 5^2 + \ldots - 2018^2$.

\problem{3} В кружке рукоделия, где занимается Валя, более 93\% участников~—~девочки. Какое наименьшее число участников может быть в таком кружке?

\problem{4} Произведение 2018 целых чисел равно 1. Может ли их сумма оказаться равной~0?

% \problem{4} Можно ли все натуральные числа от~1 до~9 записать в~клетки таблицы~$3\times3$ так, чтобы сумма в~любых двух соседних (по~вертикали или горизонтали) клетках равнялось простому числу?

\problem{5} На доске написано 2018 нулей и 2019 единиц. Женя стирает 2 числа и, если они были одинаковы, дописывает к оставшимся один ноль, а~если разные — единицу. Потом Женя повторяет эту операцию снова, потом ещё и~так далее. В~результате на~доске останется только одно число. Что это за~число?

\problem{6} Докажите, что в~любой компании людей найдутся 2~человека, имеющие равное число знакомых в этой компании (если $A$~знаком с~$B$, то~и $B$~знаком с~$A$).

\problem{7} Три колокола начинают бить одновременно. Интервалы между ударами колоколов соответственно составляют $\cfrac43$~секунды, $\cfrac53$~секунды и $2$~секунды. Совпавшие по времени удары воспринимаются за~один. Сколько ударов будет услышано за 1~минуту, включая первый и последний удары?

\problem{8} Восемь одинаковых момент расположены по кругу. Известно, что три из~них~— фальшивые, и они расположены рядом друг с~другом. Вес фальшивой монеты отличается от~веса настоящей. Все фальшивые монеты весят одинаково, но неизвестно, тяжелее или легче фальшивая монета настоящей. Покажите, что за~3~взвешивания на~чашечных весах без~гирь можно определить все фальшивые монеты.

% \end{document}

\begin{document}

\setdate{29~апреля~2020}
\setclass{11«Т»}

\addpersonalvariant{Михаил Бурмистров}

\tasknumber{1}%
\task{%
    Для частицы, движущейся с релятивистской скоростью,
    выразите $v$ и $p$ через $c$, $E_\text{кин}$ и $E_0$,
    где $E_\text{кин}$~--- кинетическая энергия частицы,
    а $E_0$, $p$ и $v$~--- её энергия покоя импульс и скорость.
}
\answer{%
    \begin{align*}
    E_\text{кин}, E_0:\quad&E = E_\text{кин} + E_0 = \frac{E_0}{\sqrt{1 - \frac{v^2}{c^2}}} \implies \sqrt{1 - \frac{v^2}{c^2}} = \frac{E_0}{{E_0} + {E_\text{кин}}} \implies v = c\sqrt{1 - \sqr{\frac{E_0}{{E_0} + {E_\text{кин}}}}} \\
    &p = \frac{mv}{\sqrt{1 - \frac{v^2}{c^2}}} = \frac{E_0}{c^2} \cdot \sqrt{1 - \sqr{\frac{E_0}{{E_0} + {E_\text{кин}}}}} \cdot \frac{{E_\text{кин}} + {E_0}}{E_0} = \frac{E_0}{c^2} \cdot \sqrt{\sqr{\frac{{E_\text{кин}} + {E_0}}{E_0}} - 1}.
    \\
    E_\text{кин}, p:\quad&E_\text{кин} = E - E_0 = mc^2\cbr{\frac 1{\sqrt{1 - \frac{v^2}{c^2}}} - 1}, p = \frac{mv}{\sqrt{1 - \frac{v^2}{c^2}}} \implies \frac{E_\text{кин}}{p} = \frac{\frac 1{\sqrt{1 - \frac{v^2}{c^2}}} - 1}{\sqrt{1 - \frac{v^2}{c^2}}} \implies v = \ldots \\
    &E_0 = E - E_\text{кин} = \frac{E_0}{\sqrt{1 - \frac{v^2}{c^2}}} - E_\text{кин} \implies E_0 = \frac{E_\text{кин}}{\frac 1{\sqrt{1 - \frac{v^2}{c^2}}} - 1} = \ldots \\
    E_\text{кин}, v:\quad&E_\text{кин} = E - E_0 = mc^2\cbr{\frac 1{\sqrt{1 - \frac{v^2}{c^2}}} - 1} \implies m = \frac{E_\text{кин}}{c^2\cbr{\frac 1{\sqrt{1 - \frac{v^2}{c^2}}} - 1}} \\
    &E_0 = mc^2 = \frac{E_\text{кин}}{\frac 1{\sqrt{1 - \frac{v^2}{c^2}}} - 1} \\
    &p = \frac{mv}{\sqrt{1 - \frac{v^2}{c^2}}} = \frac{E_\text{кин}}{c^2\cbr{\frac 1{\sqrt{1 - \frac{v^2}{c^2}}} - 1}} \cdot \frac{v}{\sqrt{1 - \frac{v^2}{c^2}}} = \frac{{E_\text{кин}} v}{c^2\cbr{1 - {\sqrt{1 - \frac{v^2}{c^2}}}}} \\
    E_0, p:\quad&E_0 = mc^2, \quad p = \frac{mv}{\sqrt{1 - \frac{v^2}{c^2}}} \implies \frac{E_0}{p} = \frac{c^2}v{\sqrt{1 - \frac{v^2}{c^2}}} = c\sqrt{\frac{c^2}{v^2} - 1} \\
    &\sqr{\frac{E_0}{pc}} = \frac{c^2}{v^2} - 1 \implies \frac{v^2}{c^2} = \frac 1{1 + \frac{E_0^2}{p^2c^2}} \implies v = \frac c{\sqrt{1 + \frac{E_0^2}{p^2c^2}}} \\
    &{E_\text{кин}} = E - E_0 = \sqrt{E_0^2 + p^2c^2} - E_0 \\
    E_0, v:\quad&E_0 = mc^2 \implies m = \frac{E_0}{c^2} \qquad p = \frac{mv}{\sqrt{1 - \frac{v^2}{c^2}}} = \frac{E_0}{c^2} \cdot \frac{v}{\sqrt{1 - \frac{v^2}{c^2}}} \\
    &E_\text{кин}= mc^2\cbr{\frac 1{\sqrt{1 - \frac{v^2}{c^2}}} - 1} = \frac{E_0}{c^2}\cbr{\frac 1{\sqrt{1 - \frac{v^2}{c^2}}} - 1} \\
    p, v:\quad&p = \frac{mv}{\sqrt{1 - \frac{v^2}{c^2}}} \implies m = \frac p v {\sqrt{1 - \frac{v^2}{c^2}}} \implies E_0 = mc^2 =\frac {pc^2} v {\sqrt{1 - \frac{v^2}{c^2}}} \\
    &E_\text{кин} = mc^2\cbr{\frac 1{\sqrt{1 - \frac{v^2}{c^2}}} - 1} = \frac p v {\sqrt{1 - \frac{v^2}{c^2}}}\cbr{\frac 1{\sqrt{1 - \frac{v^2}{c^2}}} - 1} = \frac p v \cbr{1 - {\sqrt{1 - \frac{v^2}{c^2}}}}
    \end{align*}
}
\solutionspace{200pt}

\tasknumber{2}%
\task{%
    Протон движется со скоростью $0{,}8\,c$, где $c$~--- скорость света в вакууме.
    Каково при этом отношение кинетической энергии частицы $E_\text{кин.}$ к его энергии покоя $E_0$?
}
\answer{%
    \begin{align*}
    E &= \frac{E_0}{\sqrt{1 - \frac{v^2}{c^2}}}
            \implies \frac E{E_0}
                = \frac 1{\sqrt{1 - \frac{v^2}{c^2}}}
                = \frac 1{\sqrt{1 - \sqr{0{,}8}}}
                \approx 1{,}667,
         \\
        E_{\text{кин}} &= E - E_0
            \implies \frac{E_{\text{кин}}}{E_0}
                = \frac E{E_0} - 1
                = \frac 1{\sqrt{1 - \frac{v^2}{c^2}}} - 1
                = \frac 1{\sqrt{1 - \sqr{0{,}8}}} - 1
                \approx 0{,}667.
    \end{align*}
}
\solutionspace{150pt}

\tasknumber{3}%
\task{%
    Электрон движется со скоростью $0{,}85\,c$, где $c$~--- скорость света в вакууме.
    Определите его полную энергию (в ответе приведите формулу и укажите численное значение).
}
\answer{%
    \begin{align*}
    E &= \frac{mc^2}{\sqrt{1 - \frac{v^2}{c^2}}}
            \approx \frac{9{,}1 \cdot 10^{-31}\,\text{кг} \cdot \sqr{3 \cdot 10^{8}\,\frac{\text{м}}{\text{с}}}}{\sqrt{1 - 0{,}85^2}}
            \approx 0{,}155 \cdot 10^{-12}\,\text{Дж},
         \\
        E_{\text{кин}} &= \frac{mc^2}{\sqrt{1 - \frac{v^2}{c^2}}} - mc^2
            = mc^2 \cbr{\frac 1{\sqrt{1 - \frac{v^2}{c^2}}} - 1} \approx \\
            &\approx \cbr{9{,}1 \cdot 10^{-31}\,\text{кг} \cdot \sqr{3 \cdot 10^{8}\,\frac{\text{м}}{\text{с}}}}
            \cdot \cbr{\frac 1{\sqrt{1 - 0{,}85^2}} - 1}
            \approx 0{,}074 \cdot 10^{-12}\,\text{Дж},
         \\
        p &= \frac{mv}{\sqrt{1 - \frac{v^2}{c^2}}}
            \approx \frac{9{,}1 \cdot 10^{-31}\,\text{кг} \cdot 0{,}85 \cdot 3 \cdot 10^{8}\,\frac{\text{м}}{\text{с}}}{\sqrt{1 - 0{,}85^2}}
            \approx 0{,}441 \cdot 10^{-21}\,\frac{\text{кг}\cdot\text{м}}{\text{с}}.
    \end{align*}
}
\solutionspace{150pt}

\tasknumber{4}%
\task{%
    При какой скорости движения (в м/с) релятивистское сокращение длины движущегося тела
    составит 50\%?
}
\answer{%
    \begin{align*}
    l_0 &= \frac l{\sqrt{1 - \frac{v^2}{c^2}}}
        \implies 1 - \frac{v^2}{c^2} = \sqr{\frac l{l_0}}
        \implies \frac v c = \sqrt{1 - \sqr{\frac l{l_0}}} \implies
         \\
        \implies v &= c\sqrt{1 - \sqr{\frac l{l_0}}}
        = 3 \cdot 10^{8}\,\frac{\text{м}}{\text{с}} \cdot \sqrt{1 - \sqr{\frac {l_0 - 0{,}50l_0}{l_0}}}
        = 3 \cdot 10^{8}\,\frac{\text{м}}{\text{с}} \cdot \sqrt{1 - \sqr{1 - 0{,}50}} \approx  \\
        &\approx 0{,}866c
        \approx 260 \cdot 10^{6}\,\frac{\text{м}}{\text{с}}
        \approx 935 \cdot 10^{6}\,\frac{\text{км}}{\text{ч}}.
    \end{align*}
}
\solutionspace{150pt}

\tasknumber{5}%
\task{%
    При переходе электрона в атоме с одной стационарной орбиты на другую
    излучается фотон с энергией $2{,}02 \cdot 10^{-19}\,\text{Дж}$.
    Какова длина волны этой линии спектра?
    Постоянная Планка $h = 6{,}626 \cdot 10^{-34}\,\text{Дж}\cdot\text{с}$, скорость света $c = 3 \cdot 10^{8}\,\frac{\text{м}}{\text{с}}$.
}
\answer{%
    $
        E = h\nu = h \frac c\lambda
        \implies \lambda = \frac{hc}E
            = \frac{6{,}626 \cdot 10^{-34}\,\text{Дж}\cdot\text{с} \cdot {3 \cdot 10^{8}\,\frac{\text{м}}{\text{с}}}}{2{,}02 \cdot 10^{-19}\,\text{Дж}}
            = 984{,}06\,\text{нм}.
    $
}
\solutionspace{150pt}

\tasknumber{6}%
\task{%
    Излучение какой длины волны поглотил атом водорода, если полная энергия в атоме увеличилась на $2 \cdot 10^{-19}\,\text{Дж}$?
    Постоянная Планка $h = 6{,}626 \cdot 10^{-34}\,\text{Дж}\cdot\text{с}$, скорость света $c = 3 \cdot 10^{8}\,\frac{\text{м}}{\text{с}}$.
}
\answer{%
    $
        E = h\nu = h \frac c\lambda
        \implies \lambda = \frac{hc}E
            = \frac{6{,}626 \cdot 10^{-34}\,\text{Дж}\cdot\text{с} \cdot {3 \cdot 10^{8}\,\frac{\text{м}}{\text{с}}}}{2 \cdot 10^{-19}\,\text{Дж}}
            = 994\,\text{нм}.
    $
}
\solutionspace{150pt}

\tasknumber{7}%
\task{%
    Сделайте схематичный рисунок энергетических уровней атома водорода
    и отметьте на нём первый (основной) уровень и последующие.
    Сколько различных длин волн может испустить атом водорода,
    находящийся в 4-м возбуждённом состоянии?
    Отметьте все соответствующие переходы на рисунке и укажите,
    при каком переходе (среди отмеченных) энергия излучённого фотона максимальна.
}
\answer{%
    $N = 6{,}0, \text{самая длинная линия}$
}
\solutionspace{150pt}

\tasknumber{8}%
\task{%
    Сколько фотонов испускает за $20\,\text{мин}$ лазер,
    если мощность его излучения $200\,\text{мВт}$?
    Длина волны излучения $600\,\text{нм}$.
    $h = 6{,}626 \cdot 10^{-34}\,\text{Дж}\cdot\text{с}$.
}
\answer{%
    $
        N
            = \frac{E_{\text{общая}}}{E_{\text{одного фотона}}}
            = \frac{Pt}{h\nu} = \frac{Pt}{h \frac c\lambda}
            = \frac{Pt\lambda}{hc}
            = \frac{200\,\text{мВт} \cdot 20\,\text{мин} \cdot 600\,\text{нм}}{6{,}626 \cdot 10^{-34}\,\text{Дж}\cdot\text{с} \cdot 3 \cdot 10^{8}\,\frac{\text{м}}{\text{с}}}
            \approx 7{,}24 \cdot 10^{20}\units{фотонов}
    $
}
\solutionspace{120pt}

\tasknumber{9}%
\task{%
    Какая доля (от начального количества) радиоактивных ядер распадётся через время,
    равное четырём периодам полураспада? Ответ выразить в процентах.
}
\answer{%
    \begin{align*}
    N &= N_0 \cdot 2^{- \frac t{T_{1/2}}} \implies
        \frac N{N_0} = 2^{- \frac t{T_{1/2}}}
        = 2^{-4} \approx 0{,}06 \approx 6\% \\
    N_\text{расп.} &= N_0 - N = N_0 - N_0 \cdot 2^{-\frac t{T_{1/2}}}
        = N_0\cbr{1 - 2^{-\frac t{T_{1/2}}}} \implies
        \frac{N_\text{расп.}}{N_0} = 1 - 2^{-\frac t{T_{1/2}}}
        = 1 - 2^{-4} \approx 0{,}94 \approx 94\%
    \end{align*}
}
\solutionspace{150pt}

\tasknumber{10}%
\task{%
    Сколько процентов ядер радиоактивного железа $\ce{^{59}Fe}$
    останется через $182{,}4\,\text{суток}$, если период его полураспада составляет $45{,}6\,\text{суток}$?
}
\answer{%
    \begin{align*}
    N &= N_0 \cdot 2^{-\frac t{T_{1/2}}}
        = 2^{-\frac{182{,}4\,\text{суток}}{45{,}6\,\text{суток}}}
        \approx 0{,}0625 = 6{,}25\%
    \end{align*}
}
\solutionspace{150pt}

\tasknumber{11}%
\task{%
    За $4\,\text{суток}$ от начального количества ядер радиоизотопа осталась четверть.
    Каков период полураспада этого изотопа (ответ приведите в сутках)?
    Какая ещё доля (также от начального количества) распадётся, если подождать ещё столько же?
}
\answer{%
    \begin{align*}
            N &= N_0 \cdot 2^{-\frac t{T_{1/2}}}
            \implies \frac N{N_0} = 2^{-\frac t{T_{1/2}}}
            \implies \frac 1{4} = 2^{-\frac {4\,\text{суток}}{T_{1/2}}}
            \implies 2 = \frac {4\,\text{суток}}{T_{1/2}}
            \implies T_{1/2} = \frac {4\,\text{суток}}2 \approx 2\,\text{суток}.
         \\
            \delta &= \frac{N(t)}{N_0} - \frac{N(2t)}{N_0}
            = 2^{-\frac t{T_{1/2}}} - 2^{-\frac {2t}{T_{1/2}}}
            = 2^{-\frac t{T_{1/2}}}\cbr{1 - 2^{-\frac t{T_{1/2}}}}
            = \frac 1{4} \cdot \cbr{1-\frac 1{4}} \approx 0{,}188
    \end{align*}
}
\solutionspace{150pt}

\tasknumber{12}%
\task{%
    Энергия связи ядра азота \ce{^{14}_{7}N} равна $115{,}5\,\text{МэВ}$.
    Найти дефект массы этого ядра.
    Ответ выразите в а.е.м.
    и кг.
    Скорость света $c = 2{,}998 \cdot 10^{8}\,\frac{\text{м}}{\text{с}}$, элементарный заряд $e = 1{,}6 \cdot 10^{-19}\,\text{Кл}$.
}
\answer{%
    \begin{align*}
    E_\text{св.} &= \Delta m c^2 \implies \\
    \implies
            \Delta m &= \frac {E_\text{св.}}{c^2} = \frac{115{,}5\,\text{МэВ}}{\sqr{2{,}998 \cdot 10^{8}\,\frac{\text{м}}{\text{с}}}}
            = \frac{115{,}5 \cdot 10^6 \cdot 1{,}6 \cdot 10^{-19}\,\text{Дж}}{\sqr{2{,}998 \cdot 10^{8}\,\frac{\text{м}}{\text{с}}}}
            \approx 0{,}206 \cdot 10^{-27}\,\text{кг} \approx 0{,}1238\,\text{а.е.м.}
    \end{align*}
}

\end{document}
% autogenerated
