\setdate{13~ноября~2019}
\setclass{11«Т»}

\addpersonalvariant{Михаил Бурмистров}

\tasknumber{1}%
\task{%
    Частота собственных малых колебаний пружинного маятника равна $\nu = 8\,\text{Гц}$.
    Чему станет равен период колебаний, если массу пружинного маятника увеличить в $4$ раз?
}
\answer{%
    $
        T'  = 2\pi\sqrt{\frac {m'}k}
            = 2\pi\sqrt{\frac {\alpha m}k}
            = \sqrt{\alpha} \cdot 2\pi\sqrt{\frac mk}
            = \sqrt{\alpha} \cdot T
            = T\sqrt{\alpha}
            = \frac 1\nu \cdot \sqrt{\alpha}
            = \frac{\sqrt{\alpha}}{\nu}
            = \frac{\sqrt{4}}{8\,\text{Гц}}
            = 0{,}25\,\text{с}
    $
}
\solutionspace{120pt}

\tasknumber{2}%
\task{%
    Сравните длины звуковой волны частотой $\nu_1 = 150\,\text{Гц}$ и радиоволны частотой $\nu_2 = 200\,\text{МГц}$.
    Какая больше, во сколько раз? Скорость звука примите равной $v = 320\,\frac{\text{м}}{\text{с}}$.
}
\answer{%
    $
        \lambda_1
            = v T_1 = v \cdot \frac 1{\nu_1} = \frac v{\nu_1}
            = \frac{320\,\frac{\text{м}}{\text{с}}}{150\,\text{Гц}} = 2{,}13\,\text{м},
        \quad
        \lambda_2
            = c T_2 = c \cdot \frac 1{\nu_2} = \frac c{\nu_2}
            = \frac{300\,\frac{\text{Мм}}{\text{с}}}{200\,\text{МГц}} = 1{,}50\,\text{м},
        \quad n = \frac{\lambda_2}{\lambda_1} \approx 0{,}70
    $
}
\solutionspace{120pt}

\tasknumber{3}%
\task{%
    Чему равна длина волны, если две точки среды, находящиеся на расстоянии $l = 50\,\text{см}$,
    совершают колебания с разностью фаз $\frac{\pi}{8}$?
}
\answer{%
    $
        \frac l\lambda = \frac \varphi{2\pi} + k (k\in\mathbb{N:L:s})
        \implies \lambda = \frac l{\frac \varphi{2\pi} + k} = \frac {2\pi l}{\varphi + 2\pi k},
        \quad \lambda_0 = \frac {2\pi l}{\varphi} = 800{,}00\,\text{см}
    $
}
\solutionspace{120pt}

\tasknumber{4}%
\task{%
    Тело массой $M = 100\,\text{г}$ совершает гармонические колебания.
    При этом амплитуда колебаний его скорости равна $v = 4\,\frac{\text{м}}{\text{с}}$.
    Определите запас полной механической энергии колебательной системы
    и амплитуду колебаний потенциальной энергии.
}
\answer{%
    \begin{align*}
            E_{\text{полная механическая}} &= E_{\text{max кинетическая}}
            = \frac{m v_{\max}^2}2 = \frac{100\,\text{г} \cdot \sqr{4\,\frac{\text{м}}{\text{с}}}}2 = 0{,}800\,\text{Дж}, \\
    A_{E_{\text{потенциальная}}} &= \frac{E_{\text{полная механическая}}}2 = 0{,}400\,\text{Дж}.
    \end{align*}
}

\variantsplitter

\addpersonalvariant{Гагик Аракелян}

\tasknumber{1}%
\task{%
    Частота собственных малых колебаний пружинного маятника равна $\nu = 4\,\text{Гц}$.
    Чему станет равен период колебаний, если массу пружинного маятника увеличить в $16$ раз?
}
\answer{%
    $
        T'  = 2\pi\sqrt{\frac {m'}k}
            = 2\pi\sqrt{\frac {\alpha m}k}
            = \sqrt{\alpha} \cdot 2\pi\sqrt{\frac mk}
            = \sqrt{\alpha} \cdot T
            = T\sqrt{\alpha}
            = \frac 1\nu \cdot \sqrt{\alpha}
            = \frac{\sqrt{\alpha}}{\nu}
            = \frac{\sqrt{16}}{4\,\text{Гц}}
            = 1{,}00\,\text{с}
    $
}
\solutionspace{120pt}

\tasknumber{2}%
\task{%
    Сравните длины звуковой волны частотой $\nu_1 = 150\,\text{Гц}$ и радиоволны частотой $\nu_2 = 800\,\text{МГц}$.
    Какая больше, во сколько раз? Скорость звука примите равной $v = 320\,\frac{\text{м}}{\text{с}}$.
}
\answer{%
    $
        \lambda_1
            = v T_1 = v \cdot \frac 1{\nu_1} = \frac v{\nu_1}
            = \frac{320\,\frac{\text{м}}{\text{с}}}{150\,\text{Гц}} = 2{,}13\,\text{м},
        \quad
        \lambda_2
            = c T_2 = c \cdot \frac 1{\nu_2} = \frac c{\nu_2}
            = \frac{300\,\frac{\text{Мм}}{\text{с}}}{800\,\text{МГц}} = 0{,}38\,\text{м},
        \quad n = \frac{\lambda_2}{\lambda_1} \approx 0{,}18
    $
}
\solutionspace{120pt}

\tasknumber{3}%
\task{%
    Чему равна длина волны, если две точки среды, находящиеся на расстоянии $l = 50\,\text{см}$,
    совершают колебания с разностью фаз $\frac{3\pi}{4}$?
}
\answer{%
    $
        \frac l\lambda = \frac \varphi{2\pi} + k (k\in\mathbb{N:L:s})
        \implies \lambda = \frac l{\frac \varphi{2\pi} + k} = \frac {2\pi l}{\varphi + 2\pi k},
        \quad \lambda_0 = \frac {2\pi l}{\varphi} = 133{,}33\,\text{см}
    $
}
\solutionspace{120pt}

\tasknumber{4}%
\task{%
    Тело массой $M = 250\,\text{г}$ совершает гармонические колебания.
    При этом амплитуда колебаний его скорости равна $v = 2\,\frac{\text{м}}{\text{с}}$.
    Определите запас полной механической энергии колебательной системы
    и амплитуду колебаний потенциальной энергии.
}
\answer{%
    \begin{align*}
            E_{\text{полная механическая}} &= E_{\text{max кинетическая}}
            = \frac{m v_{\max}^2}2 = \frac{250\,\text{г} \cdot \sqr{2\,\frac{\text{м}}{\text{с}}}}2 = 0{,}500\,\text{Дж}, \\
    A_{E_{\text{потенциальная}}} &= \frac{E_{\text{полная механическая}}}2 = 0{,}250\,\text{Дж}.
    \end{align*}
}

\variantsplitter

\addpersonalvariant{Ирен Аракелян}

\tasknumber{1}%
\task{%
    Частота собственных малых колебаний пружинного маятника равна $\nu = 8\,\text{Гц}$.
    Чему станет равен период колебаний, если массу пружинного маятника увеличить в $16$ раз?
}
\answer{%
    $
        T'  = 2\pi\sqrt{\frac {m'}k}
            = 2\pi\sqrt{\frac {\alpha m}k}
            = \sqrt{\alpha} \cdot 2\pi\sqrt{\frac mk}
            = \sqrt{\alpha} \cdot T
            = T\sqrt{\alpha}
            = \frac 1\nu \cdot \sqrt{\alpha}
            = \frac{\sqrt{\alpha}}{\nu}
            = \frac{\sqrt{16}}{8\,\text{Гц}}
            = 0{,}50\,\text{с}
    $
}
\solutionspace{120pt}

\tasknumber{2}%
\task{%
    Сравните длины звуковой волны частотой $\nu_1 = 300\,\text{Гц}$ и радиоволны частотой $\nu_2 = 200\,\text{МГц}$.
    Какая больше, во сколько раз? Скорость звука примите равной $v = 320\,\frac{\text{м}}{\text{с}}$.
}
\answer{%
    $
        \lambda_1
            = v T_1 = v \cdot \frac 1{\nu_1} = \frac v{\nu_1}
            = \frac{320\,\frac{\text{м}}{\text{с}}}{300\,\text{Гц}} = 1{,}07\,\text{м},
        \quad
        \lambda_2
            = c T_2 = c \cdot \frac 1{\nu_2} = \frac c{\nu_2}
            = \frac{300\,\frac{\text{Мм}}{\text{с}}}{200\,\text{МГц}} = 1{,}50\,\text{м},
        \quad n = \frac{\lambda_2}{\lambda_1} \approx 1{,}41
    $
}
\solutionspace{120pt}

\tasknumber{3}%
\task{%
    Чему равна длина волны, если две точки среды, находящиеся на расстоянии $l = 20\,\text{см}$,
    совершают колебания с разностью фаз $\frac{\pi}{8}$?
}
\answer{%
    $
        \frac l\lambda = \frac \varphi{2\pi} + k (k\in\mathbb{N:L:s})
        \implies \lambda = \frac l{\frac \varphi{2\pi} + k} = \frac {2\pi l}{\varphi + 2\pi k},
        \quad \lambda_0 = \frac {2\pi l}{\varphi} = 320{,}00\,\text{см}
    $
}
\solutionspace{120pt}

\tasknumber{4}%
\task{%
    Тело массой $M = 400\,\text{г}$ совершает гармонические колебания.
    При этом амплитуда колебаний его скорости равна $v = 5\,\frac{\text{м}}{\text{с}}$.
    Определите запас полной механической энергии колебательной системы
    и амплитуду колебаний потенциальной энергии.
}
\answer{%
    \begin{align*}
            E_{\text{полная механическая}} &= E_{\text{max кинетическая}}
            = \frac{m v_{\max}^2}2 = \frac{400\,\text{г} \cdot \sqr{5\,\frac{\text{м}}{\text{с}}}}2 = 5{,}000\,\text{Дж}, \\
    A_{E_{\text{потенциальная}}} &= \frac{E_{\text{полная механическая}}}2 = 2{,}500\,\text{Дж}.
    \end{align*}
}

\variantsplitter

\addpersonalvariant{Сабина Асадуллаева}

\tasknumber{1}%
\task{%
    Частота собственных малых колебаний пружинного маятника равна $\nu = 4\,\text{Гц}$.
    Чему станет равен период колебаний, если массу пружинного маятника увеличить в $25$ раз?
}
\answer{%
    $
        T'  = 2\pi\sqrt{\frac {m'}k}
            = 2\pi\sqrt{\frac {\alpha m}k}
            = \sqrt{\alpha} \cdot 2\pi\sqrt{\frac mk}
            = \sqrt{\alpha} \cdot T
            = T\sqrt{\alpha}
            = \frac 1\nu \cdot \sqrt{\alpha}
            = \frac{\sqrt{\alpha}}{\nu}
            = \frac{\sqrt{25}}{4\,\text{Гц}}
            = 1{,}25\,\text{с}
    $
}
\solutionspace{120pt}

\tasknumber{2}%
\task{%
    Сравните длины звуковой волны частотой $\nu_1 = 500\,\text{Гц}$ и радиоволны частотой $\nu_2 = 200\,\text{МГц}$.
    Какая больше, во сколько раз? Скорость звука примите равной $v = 320\,\frac{\text{м}}{\text{с}}$.
}
\answer{%
    $
        \lambda_1
            = v T_1 = v \cdot \frac 1{\nu_1} = \frac v{\nu_1}
            = \frac{320\,\frac{\text{м}}{\text{с}}}{500\,\text{Гц}} = 0{,}64\,\text{м},
        \quad
        \lambda_2
            = c T_2 = c \cdot \frac 1{\nu_2} = \frac c{\nu_2}
            = \frac{300\,\frac{\text{Мм}}{\text{с}}}{200\,\text{МГц}} = 1{,}50\,\text{м},
        \quad n = \frac{\lambda_2}{\lambda_1} \approx 2{,}34
    $
}
\solutionspace{120pt}

\tasknumber{3}%
\task{%
    Чему равна длина волны, если две точки среды, находящиеся на расстоянии $l = 50\,\text{см}$,
    совершают колебания с разностью фаз $\frac{\pi}{2}$?
}
\answer{%
    $
        \frac l\lambda = \frac \varphi{2\pi} + k (k\in\mathbb{N:L:s})
        \implies \lambda = \frac l{\frac \varphi{2\pi} + k} = \frac {2\pi l}{\varphi + 2\pi k},
        \quad \lambda_0 = \frac {2\pi l}{\varphi} = 200{,}00\,\text{см}
    $
}
\solutionspace{120pt}

\tasknumber{4}%
\task{%
    Тело массой $m = 200\,\text{г}$ совершает гармонические колебания.
    При этом амплитуда колебаний его скорости равна $v = 1\,\frac{\text{м}}{\text{с}}$.
    Определите запас полной механической энергии колебательной системы
    и амплитуду колебаний потенциальной энергии.
}
\answer{%
    \begin{align*}
            E_{\text{полная механическая}} &= E_{\text{max кинетическая}}
            = \frac{m v_{\max}^2}2 = \frac{200\,\text{г} \cdot \sqr{1\,\frac{\text{м}}{\text{с}}}}2 = 0{,}100\,\text{Дж}, \\
    A_{E_{\text{потенциальная}}} &= \frac{E_{\text{полная механическая}}}2 = 0{,}050\,\text{Дж}.
    \end{align*}
}

\variantsplitter

\addpersonalvariant{Вероника Битерякова}

\tasknumber{1}%
\task{%
    Частота собственных малых колебаний пружинного маятника равна $\nu = 8\,\text{Гц}$.
    Чему станет равен период колебаний, если массу пружинного маятника увеличить в $25$ раз?
}
\answer{%
    $
        T'  = 2\pi\sqrt{\frac {m'}k}
            = 2\pi\sqrt{\frac {\alpha m}k}
            = \sqrt{\alpha} \cdot 2\pi\sqrt{\frac mk}
            = \sqrt{\alpha} \cdot T
            = T\sqrt{\alpha}
            = \frac 1\nu \cdot \sqrt{\alpha}
            = \frac{\sqrt{\alpha}}{\nu}
            = \frac{\sqrt{25}}{8\,\text{Гц}}
            = 0{,}62\,\text{с}
    $
}
\solutionspace{120pt}

\tasknumber{2}%
\task{%
    Сравните длины звуковой волны частотой $\nu_1 = 150\,\text{Гц}$ и радиоволны частотой $\nu_2 = 200\,\text{МГц}$.
    Какая больше, во сколько раз? Скорость звука примите равной $v = 320\,\frac{\text{м}}{\text{с}}$.
}
\answer{%
    $
        \lambda_1
            = v T_1 = v \cdot \frac 1{\nu_1} = \frac v{\nu_1}
            = \frac{320\,\frac{\text{м}}{\text{с}}}{150\,\text{Гц}} = 2{,}13\,\text{м},
        \quad
        \lambda_2
            = c T_2 = c \cdot \frac 1{\nu_2} = \frac c{\nu_2}
            = \frac{300\,\frac{\text{Мм}}{\text{с}}}{200\,\text{МГц}} = 1{,}50\,\text{м},
        \quad n = \frac{\lambda_2}{\lambda_1} \approx 0{,}70
    $
}
\solutionspace{120pt}

\tasknumber{3}%
\task{%
    Чему равна длина волны, если две точки среды, находящиеся на расстоянии $l = 40\,\text{см}$,
    совершают колебания с разностью фаз $\frac{\pi}{2}$?
}
\answer{%
    $
        \frac l\lambda = \frac \varphi{2\pi} + k (k\in\mathbb{N:L:s})
        \implies \lambda = \frac l{\frac \varphi{2\pi} + k} = \frac {2\pi l}{\varphi + 2\pi k},
        \quad \lambda_0 = \frac {2\pi l}{\varphi} = 160{,}00\,\text{см}
    $
}
\solutionspace{120pt}

\tasknumber{4}%
\task{%
    Тело массой $M = 100\,\text{г}$ совершает гармонические колебания.
    При этом амплитуда колебаний его скорости равна $v = 1\,\frac{\text{м}}{\text{с}}$.
    Определите запас полной механической энергии колебательной системы
    и амплитуду колебаний потенциальной энергии.
}
\answer{%
    \begin{align*}
            E_{\text{полная механическая}} &= E_{\text{max кинетическая}}
            = \frac{m v_{\max}^2}2 = \frac{100\,\text{г} \cdot \sqr{1\,\frac{\text{м}}{\text{с}}}}2 = 0{,}050\,\text{Дж}, \\
    A_{E_{\text{потенциальная}}} &= \frac{E_{\text{полная механическая}}}2 = 0{,}025\,\text{Дж}.
    \end{align*}
}

\variantsplitter

\addpersonalvariant{Юлия Буянова}

\tasknumber{1}%
\task{%
    Частота собственных малых колебаний пружинного маятника равна $\nu = 2\,\text{Гц}$.
    Чему станет равен период колебаний, если массу пружинного маятника увеличить в $16$ раз?
}
\answer{%
    $
        T'  = 2\pi\sqrt{\frac {m'}k}
            = 2\pi\sqrt{\frac {\alpha m}k}
            = \sqrt{\alpha} \cdot 2\pi\sqrt{\frac mk}
            = \sqrt{\alpha} \cdot T
            = T\sqrt{\alpha}
            = \frac 1\nu \cdot \sqrt{\alpha}
            = \frac{\sqrt{\alpha}}{\nu}
            = \frac{\sqrt{16}}{2\,\text{Гц}}
            = 2{,}00\,\text{с}
    $
}
\solutionspace{120pt}

\tasknumber{2}%
\task{%
    Сравните длины звуковой волны частотой $\nu_1 = 500\,\text{Гц}$ и радиоволны частотой $\nu_2 = 800\,\text{МГц}$.
    Какая больше, во сколько раз? Скорость звука примите равной $v = 320\,\frac{\text{м}}{\text{с}}$.
}
\answer{%
    $
        \lambda_1
            = v T_1 = v \cdot \frac 1{\nu_1} = \frac v{\nu_1}
            = \frac{320\,\frac{\text{м}}{\text{с}}}{500\,\text{Гц}} = 0{,}64\,\text{м},
        \quad
        \lambda_2
            = c T_2 = c \cdot \frac 1{\nu_2} = \frac c{\nu_2}
            = \frac{300\,\frac{\text{Мм}}{\text{с}}}{800\,\text{МГц}} = 0{,}38\,\text{м},
        \quad n = \frac{\lambda_2}{\lambda_1} \approx 0{,}59
    $
}
\solutionspace{120pt}

\tasknumber{3}%
\task{%
    Чему равна длина волны, если две точки среды, находящиеся на расстоянии $l = 75\,\text{см}$,
    совершают колебания с разностью фаз $\frac{\pi}{2}$?
}
\answer{%
    $
        \frac l\lambda = \frac \varphi{2\pi} + k (k\in\mathbb{N:L:s})
        \implies \lambda = \frac l{\frac \varphi{2\pi} + k} = \frac {2\pi l}{\varphi + 2\pi k},
        \quad \lambda_0 = \frac {2\pi l}{\varphi} = 300{,}00\,\text{см}
    $
}
\solutionspace{120pt}

\tasknumber{4}%
\task{%
    Тело массой $m = 250\,\text{г}$ совершает гармонические колебания.
    При этом амплитуда колебаний его скорости равна $v = 5\,\frac{\text{м}}{\text{с}}$.
    Определите запас полной механической энергии колебательной системы
    и амплитуду колебаний потенциальной энергии.
}
\answer{%
    \begin{align*}
            E_{\text{полная механическая}} &= E_{\text{max кинетическая}}
            = \frac{m v_{\max}^2}2 = \frac{250\,\text{г} \cdot \sqr{5\,\frac{\text{м}}{\text{с}}}}2 = 3{,}125\,\text{Дж}, \\
    A_{E_{\text{потенциальная}}} &= \frac{E_{\text{полная механическая}}}2 = 1{,}562\,\text{Дж}.
    \end{align*}
}

\variantsplitter

\addpersonalvariant{Пелагея Вдовина}

\tasknumber{1}%
\task{%
    Частота собственных малых колебаний пружинного маятника равна $\nu = 8\,\text{Гц}$.
    Чему станет равен период колебаний, если массу пружинного маятника увеличить в $16$ раз?
}
\answer{%
    $
        T'  = 2\pi\sqrt{\frac {m'}k}
            = 2\pi\sqrt{\frac {\alpha m}k}
            = \sqrt{\alpha} \cdot 2\pi\sqrt{\frac mk}
            = \sqrt{\alpha} \cdot T
            = T\sqrt{\alpha}
            = \frac 1\nu \cdot \sqrt{\alpha}
            = \frac{\sqrt{\alpha}}{\nu}
            = \frac{\sqrt{16}}{8\,\text{Гц}}
            = 0{,}50\,\text{с}
    $
}
\solutionspace{120pt}

\tasknumber{2}%
\task{%
    Сравните длины звуковой волны частотой $\nu_1 = 300\,\text{Гц}$ и радиоволны частотой $\nu_2 = 800\,\text{МГц}$.
    Какая больше, во сколько раз? Скорость звука примите равной $v = 320\,\frac{\text{м}}{\text{с}}$.
}
\answer{%
    $
        \lambda_1
            = v T_1 = v \cdot \frac 1{\nu_1} = \frac v{\nu_1}
            = \frac{320\,\frac{\text{м}}{\text{с}}}{300\,\text{Гц}} = 1{,}07\,\text{м},
        \quad
        \lambda_2
            = c T_2 = c \cdot \frac 1{\nu_2} = \frac c{\nu_2}
            = \frac{300\,\frac{\text{Мм}}{\text{с}}}{800\,\text{МГц}} = 0{,}38\,\text{м},
        \quad n = \frac{\lambda_2}{\lambda_1} \approx 0{,}35
    $
}
\solutionspace{120pt}

\tasknumber{3}%
\task{%
    Чему равна длина волны, если две точки среды, находящиеся на расстоянии $l = 40\,\text{см}$,
    совершают колебания с разностью фаз $\frac{2\pi}{5}$?
}
\answer{%
    $
        \frac l\lambda = \frac \varphi{2\pi} + k (k\in\mathbb{N:L:s})
        \implies \lambda = \frac l{\frac \varphi{2\pi} + k} = \frac {2\pi l}{\varphi + 2\pi k},
        \quad \lambda_0 = \frac {2\pi l}{\varphi} = 200{,}00\,\text{см}
    $
}
\solutionspace{120pt}

\tasknumber{4}%
\task{%
    Тело массой $m = 400\,\text{г}$ совершает гармонические колебания.
    При этом амплитуда колебаний его скорости равна $v = 1\,\frac{\text{м}}{\text{с}}$.
    Определите запас полной механической энергии колебательной системы
    и амплитуду колебаний потенциальной энергии.
}
\answer{%
    \begin{align*}
            E_{\text{полная механическая}} &= E_{\text{max кинетическая}}
            = \frac{m v_{\max}^2}2 = \frac{400\,\text{г} \cdot \sqr{1\,\frac{\text{м}}{\text{с}}}}2 = 0{,}200\,\text{Дж}, \\
    A_{E_{\text{потенциальная}}} &= \frac{E_{\text{полная механическая}}}2 = 0{,}100\,\text{Дж}.
    \end{align*}
}

\variantsplitter

\addpersonalvariant{Леонид Викторов}

\tasknumber{1}%
\task{%
    Частота собственных малых колебаний пружинного маятника равна $\nu = 5\,\text{Гц}$.
    Чему станет равен период колебаний, если массу пружинного маятника увеличить в $25$ раз?
}
\answer{%
    $
        T'  = 2\pi\sqrt{\frac {m'}k}
            = 2\pi\sqrt{\frac {\alpha m}k}
            = \sqrt{\alpha} \cdot 2\pi\sqrt{\frac mk}
            = \sqrt{\alpha} \cdot T
            = T\sqrt{\alpha}
            = \frac 1\nu \cdot \sqrt{\alpha}
            = \frac{\sqrt{\alpha}}{\nu}
            = \frac{\sqrt{25}}{5\,\text{Гц}}
            = 1{,}00\,\text{с}
    $
}
\solutionspace{120pt}

\tasknumber{2}%
\task{%
    Сравните длины звуковой волны частотой $\nu_1 = 150\,\text{Гц}$ и радиоволны частотой $\nu_2 = 800\,\text{МГц}$.
    Какая больше, во сколько раз? Скорость звука примите равной $v = 320\,\frac{\text{м}}{\text{с}}$.
}
\answer{%
    $
        \lambda_1
            = v T_1 = v \cdot \frac 1{\nu_1} = \frac v{\nu_1}
            = \frac{320\,\frac{\text{м}}{\text{с}}}{150\,\text{Гц}} = 2{,}13\,\text{м},
        \quad
        \lambda_2
            = c T_2 = c \cdot \frac 1{\nu_2} = \frac c{\nu_2}
            = \frac{300\,\frac{\text{Мм}}{\text{с}}}{800\,\text{МГц}} = 0{,}38\,\text{м},
        \quad n = \frac{\lambda_2}{\lambda_1} \approx 0{,}18
    $
}
\solutionspace{120pt}

\tasknumber{3}%
\task{%
    Чему равна длина волны, если две точки среды, находящиеся на расстоянии $l = 20\,\text{см}$,
    совершают колебания с разностью фаз $\frac{3\pi}{4}$?
}
\answer{%
    $
        \frac l\lambda = \frac \varphi{2\pi} + k (k\in\mathbb{N:L:s})
        \implies \lambda = \frac l{\frac \varphi{2\pi} + k} = \frac {2\pi l}{\varphi + 2\pi k},
        \quad \lambda_0 = \frac {2\pi l}{\varphi} = 53{,}33\,\text{см}
    $
}
\solutionspace{120pt}

\tasknumber{4}%
\task{%
    Тело массой $M = 200\,\text{г}$ совершает гармонические колебания.
    При этом амплитуда колебаний его скорости равна $v = 1\,\frac{\text{м}}{\text{с}}$.
    Определите запас полной механической энергии колебательной системы
    и амплитуду колебаний потенциальной энергии.
}
\answer{%
    \begin{align*}
            E_{\text{полная механическая}} &= E_{\text{max кинетическая}}
            = \frac{m v_{\max}^2}2 = \frac{200\,\text{г} \cdot \sqr{1\,\frac{\text{м}}{\text{с}}}}2 = 0{,}100\,\text{Дж}, \\
    A_{E_{\text{потенциальная}}} &= \frac{E_{\text{полная механическая}}}2 = 0{,}050\,\text{Дж}.
    \end{align*}
}

\variantsplitter

\addpersonalvariant{Фёдор Гнутов}

\tasknumber{1}%
\task{%
    Частота собственных малых колебаний пружинного маятника равна $\nu = 5\,\text{Гц}$.
    Чему станет равен период колебаний, если массу пружинного маятника увеличить в $16$ раз?
}
\answer{%
    $
        T'  = 2\pi\sqrt{\frac {m'}k}
            = 2\pi\sqrt{\frac {\alpha m}k}
            = \sqrt{\alpha} \cdot 2\pi\sqrt{\frac mk}
            = \sqrt{\alpha} \cdot T
            = T\sqrt{\alpha}
            = \frac 1\nu \cdot \sqrt{\alpha}
            = \frac{\sqrt{\alpha}}{\nu}
            = \frac{\sqrt{16}}{5\,\text{Гц}}
            = 0{,}80\,\text{с}
    $
}
\solutionspace{120pt}

\tasknumber{2}%
\task{%
    Сравните длины звуковой волны частотой $\nu_1 = 200\,\text{Гц}$ и радиоволны частотой $\nu_2 = 200\,\text{МГц}$.
    Какая больше, во сколько раз? Скорость звука примите равной $v = 320\,\frac{\text{м}}{\text{с}}$.
}
\answer{%
    $
        \lambda_1
            = v T_1 = v \cdot \frac 1{\nu_1} = \frac v{\nu_1}
            = \frac{320\,\frac{\text{м}}{\text{с}}}{200\,\text{Гц}} = 1{,}60\,\text{м},
        \quad
        \lambda_2
            = c T_2 = c \cdot \frac 1{\nu_2} = \frac c{\nu_2}
            = \frac{300\,\frac{\text{Мм}}{\text{с}}}{200\,\text{МГц}} = 1{,}50\,\text{м},
        \quad n = \frac{\lambda_2}{\lambda_1} \approx 0{,}94
    $
}
\solutionspace{120pt}

\tasknumber{3}%
\task{%
    Чему равна длина волны, если две точки среды, находящиеся на расстоянии $l = 75\,\text{см}$,
    совершают колебания с разностью фаз $\frac{\pi}{2}$?
}
\answer{%
    $
        \frac l\lambda = \frac \varphi{2\pi} + k (k\in\mathbb{N:L:s})
        \implies \lambda = \frac l{\frac \varphi{2\pi} + k} = \frac {2\pi l}{\varphi + 2\pi k},
        \quad \lambda_0 = \frac {2\pi l}{\varphi} = 300{,}00\,\text{см}
    $
}
\solutionspace{120pt}

\tasknumber{4}%
\task{%
    Тело массой $m = 400\,\text{г}$ совершает гармонические колебания.
    При этом амплитуда колебаний его скорости равна $v = 4\,\frac{\text{м}}{\text{с}}$.
    Определите запас полной механической энергии колебательной системы
    и амплитуду колебаний потенциальной энергии.
}
\answer{%
    \begin{align*}
            E_{\text{полная механическая}} &= E_{\text{max кинетическая}}
            = \frac{m v_{\max}^2}2 = \frac{400\,\text{г} \cdot \sqr{4\,\frac{\text{м}}{\text{с}}}}2 = 3{,}200\,\text{Дж}, \\
    A_{E_{\text{потенциальная}}} &= \frac{E_{\text{полная механическая}}}2 = 1{,}600\,\text{Дж}.
    \end{align*}
}

\variantsplitter

\addpersonalvariant{Илья Гримберг}

\tasknumber{1}%
\task{%
    Частота собственных малых колебаний пружинного маятника равна $\nu = 2\,\text{Гц}$.
    Чему станет равен период колебаний, если массу пружинного маятника увеличить в $25$ раз?
}
\answer{%
    $
        T'  = 2\pi\sqrt{\frac {m'}k}
            = 2\pi\sqrt{\frac {\alpha m}k}
            = \sqrt{\alpha} \cdot 2\pi\sqrt{\frac mk}
            = \sqrt{\alpha} \cdot T
            = T\sqrt{\alpha}
            = \frac 1\nu \cdot \sqrt{\alpha}
            = \frac{\sqrt{\alpha}}{\nu}
            = \frac{\sqrt{25}}{2\,\text{Гц}}
            = 2{,}50\,\text{с}
    $
}
\solutionspace{120pt}

\tasknumber{2}%
\task{%
    Сравните длины звуковой волны частотой $\nu_1 = 300\,\text{Гц}$ и радиоволны частотой $\nu_2 = 800\,\text{МГц}$.
    Какая больше, во сколько раз? Скорость звука примите равной $v = 320\,\frac{\text{м}}{\text{с}}$.
}
\answer{%
    $
        \lambda_1
            = v T_1 = v \cdot \frac 1{\nu_1} = \frac v{\nu_1}
            = \frac{320\,\frac{\text{м}}{\text{с}}}{300\,\text{Гц}} = 1{,}07\,\text{м},
        \quad
        \lambda_2
            = c T_2 = c \cdot \frac 1{\nu_2} = \frac c{\nu_2}
            = \frac{300\,\frac{\text{Мм}}{\text{с}}}{800\,\text{МГц}} = 0{,}38\,\text{м},
        \quad n = \frac{\lambda_2}{\lambda_1} \approx 0{,}35
    $
}
\solutionspace{120pt}

\tasknumber{3}%
\task{%
    Чему равна длина волны, если две точки среды, находящиеся на расстоянии $l = 20\,\text{см}$,
    совершают колебания с разностью фаз $\frac{\pi}{8}$?
}
\answer{%
    $
        \frac l\lambda = \frac \varphi{2\pi} + k (k\in\mathbb{N:L:s})
        \implies \lambda = \frac l{\frac \varphi{2\pi} + k} = \frac {2\pi l}{\varphi + 2\pi k},
        \quad \lambda_0 = \frac {2\pi l}{\varphi} = 320{,}00\,\text{см}
    $
}
\solutionspace{120pt}

\tasknumber{4}%
\task{%
    Тело массой $M = 400\,\text{г}$ совершает гармонические колебания.
    При этом амплитуда колебаний его скорости равна $v = 1\,\frac{\text{м}}{\text{с}}$.
    Определите запас полной механической энергии колебательной системы
    и амплитуду колебаний потенциальной энергии.
}
\answer{%
    \begin{align*}
            E_{\text{полная механическая}} &= E_{\text{max кинетическая}}
            = \frac{m v_{\max}^2}2 = \frac{400\,\text{г} \cdot \sqr{1\,\frac{\text{м}}{\text{с}}}}2 = 0{,}200\,\text{Дж}, \\
    A_{E_{\text{потенциальная}}} &= \frac{E_{\text{полная механическая}}}2 = 0{,}100\,\text{Дж}.
    \end{align*}
}

\variantsplitter

\addpersonalvariant{Иван Гурьянов}

\tasknumber{1}%
\task{%
    Частота собственных малых колебаний пружинного маятника равна $\nu = 8\,\text{Гц}$.
    Чему станет равен период колебаний, если массу пружинного маятника увеличить в $16$ раз?
}
\answer{%
    $
        T'  = 2\pi\sqrt{\frac {m'}k}
            = 2\pi\sqrt{\frac {\alpha m}k}
            = \sqrt{\alpha} \cdot 2\pi\sqrt{\frac mk}
            = \sqrt{\alpha} \cdot T
            = T\sqrt{\alpha}
            = \frac 1\nu \cdot \sqrt{\alpha}
            = \frac{\sqrt{\alpha}}{\nu}
            = \frac{\sqrt{16}}{8\,\text{Гц}}
            = 0{,}50\,\text{с}
    $
}
\solutionspace{120pt}

\tasknumber{2}%
\task{%
    Сравните длины звуковой волны частотой $\nu_1 = 500\,\text{Гц}$ и радиоволны частотой $\nu_2 = 800\,\text{МГц}$.
    Какая больше, во сколько раз? Скорость звука примите равной $v = 320\,\frac{\text{м}}{\text{с}}$.
}
\answer{%
    $
        \lambda_1
            = v T_1 = v \cdot \frac 1{\nu_1} = \frac v{\nu_1}
            = \frac{320\,\frac{\text{м}}{\text{с}}}{500\,\text{Гц}} = 0{,}64\,\text{м},
        \quad
        \lambda_2
            = c T_2 = c \cdot \frac 1{\nu_2} = \frac c{\nu_2}
            = \frac{300\,\frac{\text{Мм}}{\text{с}}}{800\,\text{МГц}} = 0{,}38\,\text{м},
        \quad n = \frac{\lambda_2}{\lambda_1} \approx 0{,}59
    $
}
\solutionspace{120pt}

\tasknumber{3}%
\task{%
    Чему равна длина волны, если две точки среды, находящиеся на расстоянии $l = 50\,\text{см}$,
    совершают колебания с разностью фаз $\frac{2\pi}{5}$?
}
\answer{%
    $
        \frac l\lambda = \frac \varphi{2\pi} + k (k\in\mathbb{N:L:s})
        \implies \lambda = \frac l{\frac \varphi{2\pi} + k} = \frac {2\pi l}{\varphi + 2\pi k},
        \quad \lambda_0 = \frac {2\pi l}{\varphi} = 250{,}00\,\text{см}
    $
}
\solutionspace{120pt}

\tasknumber{4}%
\task{%
    Тело массой $m = 400\,\text{г}$ совершает гармонические колебания.
    При этом амплитуда колебаний его скорости равна $v = 1\,\frac{\text{м}}{\text{с}}$.
    Определите запас полной механической энергии колебательной системы
    и амплитуду колебаний потенциальной энергии.
}
\answer{%
    \begin{align*}
            E_{\text{полная механическая}} &= E_{\text{max кинетическая}}
            = \frac{m v_{\max}^2}2 = \frac{400\,\text{г} \cdot \sqr{1\,\frac{\text{м}}{\text{с}}}}2 = 0{,}200\,\text{Дж}, \\
    A_{E_{\text{потенциальная}}} &= \frac{E_{\text{полная механическая}}}2 = 0{,}100\,\text{Дж}.
    \end{align*}
}

\variantsplitter

\addpersonalvariant{Артём Денежкин}

\tasknumber{1}%
\task{%
    Частота собственных малых колебаний пружинного маятника равна $\nu = 5\,\text{Гц}$.
    Чему станет равен период колебаний, если массу пружинного маятника увеличить в $25$ раз?
}
\answer{%
    $
        T'  = 2\pi\sqrt{\frac {m'}k}
            = 2\pi\sqrt{\frac {\alpha m}k}
            = \sqrt{\alpha} \cdot 2\pi\sqrt{\frac mk}
            = \sqrt{\alpha} \cdot T
            = T\sqrt{\alpha}
            = \frac 1\nu \cdot \sqrt{\alpha}
            = \frac{\sqrt{\alpha}}{\nu}
            = \frac{\sqrt{25}}{5\,\text{Гц}}
            = 1{,}00\,\text{с}
    $
}
\solutionspace{120pt}

\tasknumber{2}%
\task{%
    Сравните длины звуковой волны частотой $\nu_1 = 300\,\text{Гц}$ и радиоволны частотой $\nu_2 = 800\,\text{МГц}$.
    Какая больше, во сколько раз? Скорость звука примите равной $v = 320\,\frac{\text{м}}{\text{с}}$.
}
\answer{%
    $
        \lambda_1
            = v T_1 = v \cdot \frac 1{\nu_1} = \frac v{\nu_1}
            = \frac{320\,\frac{\text{м}}{\text{с}}}{300\,\text{Гц}} = 1{,}07\,\text{м},
        \quad
        \lambda_2
            = c T_2 = c \cdot \frac 1{\nu_2} = \frac c{\nu_2}
            = \frac{300\,\frac{\text{Мм}}{\text{с}}}{800\,\text{МГц}} = 0{,}38\,\text{м},
        \quad n = \frac{\lambda_2}{\lambda_1} \approx 0{,}35
    $
}
\solutionspace{120pt}

\tasknumber{3}%
\task{%
    Чему равна длина волны, если две точки среды, находящиеся на расстоянии $l = 25\,\text{см}$,
    совершают колебания с разностью фаз $\frac{3\pi}{8}$?
}
\answer{%
    $
        \frac l\lambda = \frac \varphi{2\pi} + k (k\in\mathbb{N:L:s})
        \implies \lambda = \frac l{\frac \varphi{2\pi} + k} = \frac {2\pi l}{\varphi + 2\pi k},
        \quad \lambda_0 = \frac {2\pi l}{\varphi} = 133{,}33\,\text{см}
    $
}
\solutionspace{120pt}

\tasknumber{4}%
\task{%
    Тело массой $M = 400\,\text{г}$ совершает гармонические колебания.
    При этом амплитуда колебаний его скорости равна $v = 2\,\frac{\text{м}}{\text{с}}$.
    Определите запас полной механической энергии колебательной системы
    и амплитуду колебаний потенциальной энергии.
}
\answer{%
    \begin{align*}
            E_{\text{полная механическая}} &= E_{\text{max кинетическая}}
            = \frac{m v_{\max}^2}2 = \frac{400\,\text{г} \cdot \sqr{2\,\frac{\text{м}}{\text{с}}}}2 = 0{,}800\,\text{Дж}, \\
    A_{E_{\text{потенциальная}}} &= \frac{E_{\text{полная механическая}}}2 = 0{,}400\,\text{Дж}.
    \end{align*}
}

\variantsplitter

\addpersonalvariant{Виктор Жилин}

\tasknumber{1}%
\task{%
    Частота собственных малых колебаний пружинного маятника равна $\nu = 2\,\text{Гц}$.
    Чему станет равен период колебаний, если массу пружинного маятника увеличить в $16$ раз?
}
\answer{%
    $
        T'  = 2\pi\sqrt{\frac {m'}k}
            = 2\pi\sqrt{\frac {\alpha m}k}
            = \sqrt{\alpha} \cdot 2\pi\sqrt{\frac mk}
            = \sqrt{\alpha} \cdot T
            = T\sqrt{\alpha}
            = \frac 1\nu \cdot \sqrt{\alpha}
            = \frac{\sqrt{\alpha}}{\nu}
            = \frac{\sqrt{16}}{2\,\text{Гц}}
            = 2{,}00\,\text{с}
    $
}
\solutionspace{120pt}

\tasknumber{2}%
\task{%
    Сравните длины звуковой волны частотой $\nu_1 = 150\,\text{Гц}$ и радиоволны частотой $\nu_2 = 500\,\text{МГц}$.
    Какая больше, во сколько раз? Скорость звука примите равной $v = 320\,\frac{\text{м}}{\text{с}}$.
}
\answer{%
    $
        \lambda_1
            = v T_1 = v \cdot \frac 1{\nu_1} = \frac v{\nu_1}
            = \frac{320\,\frac{\text{м}}{\text{с}}}{150\,\text{Гц}} = 2{,}13\,\text{м},
        \quad
        \lambda_2
            = c T_2 = c \cdot \frac 1{\nu_2} = \frac c{\nu_2}
            = \frac{300\,\frac{\text{Мм}}{\text{с}}}{500\,\text{МГц}} = 0{,}60\,\text{м},
        \quad n = \frac{\lambda_2}{\lambda_1} \approx 0{,}28
    $
}
\solutionspace{120pt}

\tasknumber{3}%
\task{%
    Чему равна длина волны, если две точки среды, находящиеся на расстоянии $l = 50\,\text{см}$,
    совершают колебания с разностью фаз $\frac{\pi}{2}$?
}
\answer{%
    $
        \frac l\lambda = \frac \varphi{2\pi} + k (k\in\mathbb{N:L:s})
        \implies \lambda = \frac l{\frac \varphi{2\pi} + k} = \frac {2\pi l}{\varphi + 2\pi k},
        \quad \lambda_0 = \frac {2\pi l}{\varphi} = 200{,}00\,\text{см}
    $
}
\solutionspace{120pt}

\tasknumber{4}%
\task{%
    Тело массой $M = 100\,\text{г}$ совершает гармонические колебания.
    При этом амплитуда колебаний его скорости равна $v = 5\,\frac{\text{м}}{\text{с}}$.
    Определите запас полной механической энергии колебательной системы
    и амплитуду колебаний потенциальной энергии.
}
\answer{%
    \begin{align*}
            E_{\text{полная механическая}} &= E_{\text{max кинетическая}}
            = \frac{m v_{\max}^2}2 = \frac{100\,\text{г} \cdot \sqr{5\,\frac{\text{м}}{\text{с}}}}2 = 1{,}250\,\text{Дж}, \\
    A_{E_{\text{потенциальная}}} &= \frac{E_{\text{полная механическая}}}2 = 0{,}625\,\text{Дж}.
    \end{align*}
}

\variantsplitter

\addpersonalvariant{Дмитрий Иванов}

\tasknumber{1}%
\task{%
    Частота собственных малых колебаний пружинного маятника равна $\nu = 8\,\text{Гц}$.
    Чему станет равен период колебаний, если массу пружинного маятника увеличить в $25$ раз?
}
\answer{%
    $
        T'  = 2\pi\sqrt{\frac {m'}k}
            = 2\pi\sqrt{\frac {\alpha m}k}
            = \sqrt{\alpha} \cdot 2\pi\sqrt{\frac mk}
            = \sqrt{\alpha} \cdot T
            = T\sqrt{\alpha}
            = \frac 1\nu \cdot \sqrt{\alpha}
            = \frac{\sqrt{\alpha}}{\nu}
            = \frac{\sqrt{25}}{8\,\text{Гц}}
            = 0{,}62\,\text{с}
    $
}
\solutionspace{120pt}

\tasknumber{2}%
\task{%
    Сравните длины звуковой волны частотой $\nu_1 = 300\,\text{Гц}$ и радиоволны частотой $\nu_2 = 200\,\text{МГц}$.
    Какая больше, во сколько раз? Скорость звука примите равной $v = 320\,\frac{\text{м}}{\text{с}}$.
}
\answer{%
    $
        \lambda_1
            = v T_1 = v \cdot \frac 1{\nu_1} = \frac v{\nu_1}
            = \frac{320\,\frac{\text{м}}{\text{с}}}{300\,\text{Гц}} = 1{,}07\,\text{м},
        \quad
        \lambda_2
            = c T_2 = c \cdot \frac 1{\nu_2} = \frac c{\nu_2}
            = \frac{300\,\frac{\text{Мм}}{\text{с}}}{200\,\text{МГц}} = 1{,}50\,\text{м},
        \quad n = \frac{\lambda_2}{\lambda_1} \approx 1{,}41
    $
}
\solutionspace{120pt}

\tasknumber{3}%
\task{%
    Чему равна длина волны, если две точки среды, находящиеся на расстоянии $l = 75\,\text{см}$,
    совершают колебания с разностью фаз $\frac{\pi}{8}$?
}
\answer{%
    $
        \frac l\lambda = \frac \varphi{2\pi} + k (k\in\mathbb{N:L:s})
        \implies \lambda = \frac l{\frac \varphi{2\pi} + k} = \frac {2\pi l}{\varphi + 2\pi k},
        \quad \lambda_0 = \frac {2\pi l}{\varphi} = 1200{,}00\,\text{см}
    $
}
\solutionspace{120pt}

\tasknumber{4}%
\task{%
    Тело массой $m = 100\,\text{г}$ совершает гармонические колебания.
    При этом амплитуда колебаний его скорости равна $v = 4\,\frac{\text{м}}{\text{с}}$.
    Определите запас полной механической энергии колебательной системы
    и амплитуду колебаний потенциальной энергии.
}
\answer{%
    \begin{align*}
            E_{\text{полная механическая}} &= E_{\text{max кинетическая}}
            = \frac{m v_{\max}^2}2 = \frac{100\,\text{г} \cdot \sqr{4\,\frac{\text{м}}{\text{с}}}}2 = 0{,}800\,\text{Дж}, \\
    A_{E_{\text{потенциальная}}} &= \frac{E_{\text{полная механическая}}}2 = 0{,}400\,\text{Дж}.
    \end{align*}
}

\variantsplitter

\addpersonalvariant{Олег Климов}

\tasknumber{1}%
\task{%
    Частота собственных малых колебаний пружинного маятника равна $\nu = 2\,\text{Гц}$.
    Чему станет равен период колебаний, если массу пружинного маятника увеличить в $4$ раз?
}
\answer{%
    $
        T'  = 2\pi\sqrt{\frac {m'}k}
            = 2\pi\sqrt{\frac {\alpha m}k}
            = \sqrt{\alpha} \cdot 2\pi\sqrt{\frac mk}
            = \sqrt{\alpha} \cdot T
            = T\sqrt{\alpha}
            = \frac 1\nu \cdot \sqrt{\alpha}
            = \frac{\sqrt{\alpha}}{\nu}
            = \frac{\sqrt{4}}{2\,\text{Гц}}
            = 1{,}00\,\text{с}
    $
}
\solutionspace{120pt}

\tasknumber{2}%
\task{%
    Сравните длины звуковой волны частотой $\nu_1 = 300\,\text{Гц}$ и радиоволны частотой $\nu_2 = 800\,\text{МГц}$.
    Какая больше, во сколько раз? Скорость звука примите равной $v = 320\,\frac{\text{м}}{\text{с}}$.
}
\answer{%
    $
        \lambda_1
            = v T_1 = v \cdot \frac 1{\nu_1} = \frac v{\nu_1}
            = \frac{320\,\frac{\text{м}}{\text{с}}}{300\,\text{Гц}} = 1{,}07\,\text{м},
        \quad
        \lambda_2
            = c T_2 = c \cdot \frac 1{\nu_2} = \frac c{\nu_2}
            = \frac{300\,\frac{\text{Мм}}{\text{с}}}{800\,\text{МГц}} = 0{,}38\,\text{м},
        \quad n = \frac{\lambda_2}{\lambda_1} \approx 0{,}35
    $
}
\solutionspace{120pt}

\tasknumber{3}%
\task{%
    Чему равна длина волны, если две точки среды, находящиеся на расстоянии $l = 25\,\text{см}$,
    совершают колебания с разностью фаз $\frac{\pi}{2}$?
}
\answer{%
    $
        \frac l\lambda = \frac \varphi{2\pi} + k (k\in\mathbb{N:L:s})
        \implies \lambda = \frac l{\frac \varphi{2\pi} + k} = \frac {2\pi l}{\varphi + 2\pi k},
        \quad \lambda_0 = \frac {2\pi l}{\varphi} = 100{,}00\,\text{см}
    $
}
\solutionspace{120pt}

\tasknumber{4}%
\task{%
    Тело массой $M = 100\,\text{г}$ совершает гармонические колебания.
    При этом амплитуда колебаний его скорости равна $v = 4\,\frac{\text{м}}{\text{с}}$.
    Определите запас полной механической энергии колебательной системы
    и амплитуду колебаний потенциальной энергии.
}
\answer{%
    \begin{align*}
            E_{\text{полная механическая}} &= E_{\text{max кинетическая}}
            = \frac{m v_{\max}^2}2 = \frac{100\,\text{г} \cdot \sqr{4\,\frac{\text{м}}{\text{с}}}}2 = 0{,}800\,\text{Дж}, \\
    A_{E_{\text{потенциальная}}} &= \frac{E_{\text{полная механическая}}}2 = 0{,}400\,\text{Дж}.
    \end{align*}
}

\variantsplitter

\addpersonalvariant{Анна Ковалева}

\tasknumber{1}%
\task{%
    Частота собственных малых колебаний пружинного маятника равна $\nu = 4\,\text{Гц}$.
    Чему станет равен период колебаний, если массу пружинного маятника увеличить в $16$ раз?
}
\answer{%
    $
        T'  = 2\pi\sqrt{\frac {m'}k}
            = 2\pi\sqrt{\frac {\alpha m}k}
            = \sqrt{\alpha} \cdot 2\pi\sqrt{\frac mk}
            = \sqrt{\alpha} \cdot T
            = T\sqrt{\alpha}
            = \frac 1\nu \cdot \sqrt{\alpha}
            = \frac{\sqrt{\alpha}}{\nu}
            = \frac{\sqrt{16}}{4\,\text{Гц}}
            = 1{,}00\,\text{с}
    $
}
\solutionspace{120pt}

\tasknumber{2}%
\task{%
    Сравните длины звуковой волны частотой $\nu_1 = 200\,\text{Гц}$ и радиоволны частотой $\nu_2 = 500\,\text{МГц}$.
    Какая больше, во сколько раз? Скорость звука примите равной $v = 320\,\frac{\text{м}}{\text{с}}$.
}
\answer{%
    $
        \lambda_1
            = v T_1 = v \cdot \frac 1{\nu_1} = \frac v{\nu_1}
            = \frac{320\,\frac{\text{м}}{\text{с}}}{200\,\text{Гц}} = 1{,}60\,\text{м},
        \quad
        \lambda_2
            = c T_2 = c \cdot \frac 1{\nu_2} = \frac c{\nu_2}
            = \frac{300\,\frac{\text{Мм}}{\text{с}}}{500\,\text{МГц}} = 0{,}60\,\text{м},
        \quad n = \frac{\lambda_2}{\lambda_1} \approx 0{,}37
    $
}
\solutionspace{120pt}

\tasknumber{3}%
\task{%
    Чему равна длина волны, если две точки среды, находящиеся на расстоянии $l = 20\,\text{см}$,
    совершают колебания с разностью фаз $\frac{\pi}{8}$?
}
\answer{%
    $
        \frac l\lambda = \frac \varphi{2\pi} + k (k\in\mathbb{N:L:s})
        \implies \lambda = \frac l{\frac \varphi{2\pi} + k} = \frac {2\pi l}{\varphi + 2\pi k},
        \quad \lambda_0 = \frac {2\pi l}{\varphi} = 320{,}00\,\text{см}
    $
}
\solutionspace{120pt}

\tasknumber{4}%
\task{%
    Тело массой $M = 100\,\text{г}$ совершает гармонические колебания.
    При этом амплитуда колебаний его скорости равна $v = 5\,\frac{\text{м}}{\text{с}}$.
    Определите запас полной механической энергии колебательной системы
    и амплитуду колебаний потенциальной энергии.
}
\answer{%
    \begin{align*}
            E_{\text{полная механическая}} &= E_{\text{max кинетическая}}
            = \frac{m v_{\max}^2}2 = \frac{100\,\text{г} \cdot \sqr{5\,\frac{\text{м}}{\text{с}}}}2 = 1{,}250\,\text{Дж}, \\
    A_{E_{\text{потенциальная}}} &= \frac{E_{\text{полная механическая}}}2 = 0{,}625\,\text{Дж}.
    \end{align*}
}

\variantsplitter

\addpersonalvariant{Глеб Ковылин}

\tasknumber{1}%
\task{%
    Частота собственных малых колебаний пружинного маятника равна $\nu = 2\,\text{Гц}$.
    Чему станет равен период колебаний, если массу пружинного маятника увеличить в $25$ раз?
}
\answer{%
    $
        T'  = 2\pi\sqrt{\frac {m'}k}
            = 2\pi\sqrt{\frac {\alpha m}k}
            = \sqrt{\alpha} \cdot 2\pi\sqrt{\frac mk}
            = \sqrt{\alpha} \cdot T
            = T\sqrt{\alpha}
            = \frac 1\nu \cdot \sqrt{\alpha}
            = \frac{\sqrt{\alpha}}{\nu}
            = \frac{\sqrt{25}}{2\,\text{Гц}}
            = 2{,}50\,\text{с}
    $
}
\solutionspace{120pt}

\tasknumber{2}%
\task{%
    Сравните длины звуковой волны частотой $\nu_1 = 500\,\text{Гц}$ и радиоволны частотой $\nu_2 = 500\,\text{МГц}$.
    Какая больше, во сколько раз? Скорость звука примите равной $v = 320\,\frac{\text{м}}{\text{с}}$.
}
\answer{%
    $
        \lambda_1
            = v T_1 = v \cdot \frac 1{\nu_1} = \frac v{\nu_1}
            = \frac{320\,\frac{\text{м}}{\text{с}}}{500\,\text{Гц}} = 0{,}64\,\text{м},
        \quad
        \lambda_2
            = c T_2 = c \cdot \frac 1{\nu_2} = \frac c{\nu_2}
            = \frac{300\,\frac{\text{Мм}}{\text{с}}}{500\,\text{МГц}} = 0{,}60\,\text{м},
        \quad n = \frac{\lambda_2}{\lambda_1} \approx 0{,}94
    $
}
\solutionspace{120pt}

\tasknumber{3}%
\task{%
    Чему равна длина волны, если две точки среды, находящиеся на расстоянии $l = 50\,\text{см}$,
    совершают колебания с разностью фаз $\frac{\pi}{2}$?
}
\answer{%
    $
        \frac l\lambda = \frac \varphi{2\pi} + k (k\in\mathbb{N:L:s})
        \implies \lambda = \frac l{\frac \varphi{2\pi} + k} = \frac {2\pi l}{\varphi + 2\pi k},
        \quad \lambda_0 = \frac {2\pi l}{\varphi} = 200{,}00\,\text{см}
    $
}
\solutionspace{120pt}

\tasknumber{4}%
\task{%
    Тело массой $M = 250\,\text{г}$ совершает гармонические колебания.
    При этом амплитуда колебаний его скорости равна $v = 5\,\frac{\text{м}}{\text{с}}$.
    Определите запас полной механической энергии колебательной системы
    и амплитуду колебаний потенциальной энергии.
}
\answer{%
    \begin{align*}
            E_{\text{полная механическая}} &= E_{\text{max кинетическая}}
            = \frac{m v_{\max}^2}2 = \frac{250\,\text{г} \cdot \sqr{5\,\frac{\text{м}}{\text{с}}}}2 = 3{,}125\,\text{Дж}, \\
    A_{E_{\text{потенциальная}}} &= \frac{E_{\text{полная механическая}}}2 = 1{,}562\,\text{Дж}.
    \end{align*}
}

\variantsplitter

\addpersonalvariant{Даниил Космынин}

\tasknumber{1}%
\task{%
    Частота собственных малых колебаний пружинного маятника равна $\nu = 8\,\text{Гц}$.
    Чему станет равен период колебаний, если массу пружинного маятника увеличить в $16$ раз?
}
\answer{%
    $
        T'  = 2\pi\sqrt{\frac {m'}k}
            = 2\pi\sqrt{\frac {\alpha m}k}
            = \sqrt{\alpha} \cdot 2\pi\sqrt{\frac mk}
            = \sqrt{\alpha} \cdot T
            = T\sqrt{\alpha}
            = \frac 1\nu \cdot \sqrt{\alpha}
            = \frac{\sqrt{\alpha}}{\nu}
            = \frac{\sqrt{16}}{8\,\text{Гц}}
            = 0{,}50\,\text{с}
    $
}
\solutionspace{120pt}

\tasknumber{2}%
\task{%
    Сравните длины звуковой волны частотой $\nu_1 = 500\,\text{Гц}$ и радиоволны частотой $\nu_2 = 800\,\text{МГц}$.
    Какая больше, во сколько раз? Скорость звука примите равной $v = 320\,\frac{\text{м}}{\text{с}}$.
}
\answer{%
    $
        \lambda_1
            = v T_1 = v \cdot \frac 1{\nu_1} = \frac v{\nu_1}
            = \frac{320\,\frac{\text{м}}{\text{с}}}{500\,\text{Гц}} = 0{,}64\,\text{м},
        \quad
        \lambda_2
            = c T_2 = c \cdot \frac 1{\nu_2} = \frac c{\nu_2}
            = \frac{300\,\frac{\text{Мм}}{\text{с}}}{800\,\text{МГц}} = 0{,}38\,\text{м},
        \quad n = \frac{\lambda_2}{\lambda_1} \approx 0{,}59
    $
}
\solutionspace{120pt}

\tasknumber{3}%
\task{%
    Чему равна длина волны, если две точки среды, находящиеся на расстоянии $l = 50\,\text{см}$,
    совершают колебания с разностью фаз $\frac{2\pi}{5}$?
}
\answer{%
    $
        \frac l\lambda = \frac \varphi{2\pi} + k (k\in\mathbb{N:L:s})
        \implies \lambda = \frac l{\frac \varphi{2\pi} + k} = \frac {2\pi l}{\varphi + 2\pi k},
        \quad \lambda_0 = \frac {2\pi l}{\varphi} = 250{,}00\,\text{см}
    $
}
\solutionspace{120pt}

\tasknumber{4}%
\task{%
    Тело массой $M = 400\,\text{г}$ совершает гармонические колебания.
    При этом амплитуда колебаний его скорости равна $v = 1\,\frac{\text{м}}{\text{с}}$.
    Определите запас полной механической энергии колебательной системы
    и амплитуду колебаний потенциальной энергии.
}
\answer{%
    \begin{align*}
            E_{\text{полная механическая}} &= E_{\text{max кинетическая}}
            = \frac{m v_{\max}^2}2 = \frac{400\,\text{г} \cdot \sqr{1\,\frac{\text{м}}{\text{с}}}}2 = 0{,}200\,\text{Дж}, \\
    A_{E_{\text{потенциальная}}} &= \frac{E_{\text{полная механическая}}}2 = 0{,}100\,\text{Дж}.
    \end{align*}
}

\variantsplitter

\addpersonalvariant{Алина Леоничева}

\tasknumber{1}%
\task{%
    Частота собственных малых колебаний пружинного маятника равна $\nu = 5\,\text{Гц}$.
    Чему станет равен период колебаний, если массу пружинного маятника увеличить в $25$ раз?
}
\answer{%
    $
        T'  = 2\pi\sqrt{\frac {m'}k}
            = 2\pi\sqrt{\frac {\alpha m}k}
            = \sqrt{\alpha} \cdot 2\pi\sqrt{\frac mk}
            = \sqrt{\alpha} \cdot T
            = T\sqrt{\alpha}
            = \frac 1\nu \cdot \sqrt{\alpha}
            = \frac{\sqrt{\alpha}}{\nu}
            = \frac{\sqrt{25}}{5\,\text{Гц}}
            = 1{,}00\,\text{с}
    $
}
\solutionspace{120pt}

\tasknumber{2}%
\task{%
    Сравните длины звуковой волны частотой $\nu_1 = 150\,\text{Гц}$ и радиоволны частотой $\nu_2 = 800\,\text{МГц}$.
    Какая больше, во сколько раз? Скорость звука примите равной $v = 320\,\frac{\text{м}}{\text{с}}$.
}
\answer{%
    $
        \lambda_1
            = v T_1 = v \cdot \frac 1{\nu_1} = \frac v{\nu_1}
            = \frac{320\,\frac{\text{м}}{\text{с}}}{150\,\text{Гц}} = 2{,}13\,\text{м},
        \quad
        \lambda_2
            = c T_2 = c \cdot \frac 1{\nu_2} = \frac c{\nu_2}
            = \frac{300\,\frac{\text{Мм}}{\text{с}}}{800\,\text{МГц}} = 0{,}38\,\text{м},
        \quad n = \frac{\lambda_2}{\lambda_1} \approx 0{,}18
    $
}
\solutionspace{120pt}

\tasknumber{3}%
\task{%
    Чему равна длина волны, если две точки среды, находящиеся на расстоянии $l = 75\,\text{см}$,
    совершают колебания с разностью фаз $\frac{3\pi}{4}$?
}
\answer{%
    $
        \frac l\lambda = \frac \varphi{2\pi} + k (k\in\mathbb{N:L:s})
        \implies \lambda = \frac l{\frac \varphi{2\pi} + k} = \frac {2\pi l}{\varphi + 2\pi k},
        \quad \lambda_0 = \frac {2\pi l}{\varphi} = 200{,}00\,\text{см}
    $
}
\solutionspace{120pt}

\tasknumber{4}%
\task{%
    Тело массой $m = 100\,\text{г}$ совершает гармонические колебания.
    При этом амплитуда колебаний его скорости равна $v = 1\,\frac{\text{м}}{\text{с}}$.
    Определите запас полной механической энергии колебательной системы
    и амплитуду колебаний потенциальной энергии.
}
\answer{%
    \begin{align*}
            E_{\text{полная механическая}} &= E_{\text{max кинетическая}}
            = \frac{m v_{\max}^2}2 = \frac{100\,\text{г} \cdot \sqr{1\,\frac{\text{м}}{\text{с}}}}2 = 0{,}050\,\text{Дж}, \\
    A_{E_{\text{потенциальная}}} &= \frac{E_{\text{полная механическая}}}2 = 0{,}025\,\text{Дж}.
    \end{align*}
}

\variantsplitter

\addpersonalvariant{Ирина Лин}

\tasknumber{1}%
\task{%
    Частота собственных малых колебаний пружинного маятника равна $\nu = 5\,\text{Гц}$.
    Чему станет равен период колебаний, если массу пружинного маятника увеличить в $16$ раз?
}
\answer{%
    $
        T'  = 2\pi\sqrt{\frac {m'}k}
            = 2\pi\sqrt{\frac {\alpha m}k}
            = \sqrt{\alpha} \cdot 2\pi\sqrt{\frac mk}
            = \sqrt{\alpha} \cdot T
            = T\sqrt{\alpha}
            = \frac 1\nu \cdot \sqrt{\alpha}
            = \frac{\sqrt{\alpha}}{\nu}
            = \frac{\sqrt{16}}{5\,\text{Гц}}
            = 0{,}80\,\text{с}
    $
}
\solutionspace{120pt}

\tasknumber{2}%
\task{%
    Сравните длины звуковой волны частотой $\nu_1 = 150\,\text{Гц}$ и радиоволны частотой $\nu_2 = 200\,\text{МГц}$.
    Какая больше, во сколько раз? Скорость звука примите равной $v = 320\,\frac{\text{м}}{\text{с}}$.
}
\answer{%
    $
        \lambda_1
            = v T_1 = v \cdot \frac 1{\nu_1} = \frac v{\nu_1}
            = \frac{320\,\frac{\text{м}}{\text{с}}}{150\,\text{Гц}} = 2{,}13\,\text{м},
        \quad
        \lambda_2
            = c T_2 = c \cdot \frac 1{\nu_2} = \frac c{\nu_2}
            = \frac{300\,\frac{\text{Мм}}{\text{с}}}{200\,\text{МГц}} = 1{,}50\,\text{м},
        \quad n = \frac{\lambda_2}{\lambda_1} \approx 0{,}70
    $
}
\solutionspace{120pt}

\tasknumber{3}%
\task{%
    Чему равна длина волны, если две точки среды, находящиеся на расстоянии $l = 20\,\text{см}$,
    совершают колебания с разностью фаз $\frac{3\pi}{4}$?
}
\answer{%
    $
        \frac l\lambda = \frac \varphi{2\pi} + k (k\in\mathbb{N:L:s})
        \implies \lambda = \frac l{\frac \varphi{2\pi} + k} = \frac {2\pi l}{\varphi + 2\pi k},
        \quad \lambda_0 = \frac {2\pi l}{\varphi} = 53{,}33\,\text{см}
    $
}
\solutionspace{120pt}

\tasknumber{4}%
\task{%
    Тело массой $M = 200\,\text{г}$ совершает гармонические колебания.
    При этом амплитуда колебаний его скорости равна $v = 1\,\frac{\text{м}}{\text{с}}$.
    Определите запас полной механической энергии колебательной системы
    и амплитуду колебаний потенциальной энергии.
}
\answer{%
    \begin{align*}
            E_{\text{полная механическая}} &= E_{\text{max кинетическая}}
            = \frac{m v_{\max}^2}2 = \frac{200\,\text{г} \cdot \sqr{1\,\frac{\text{м}}{\text{с}}}}2 = 0{,}100\,\text{Дж}, \\
    A_{E_{\text{потенциальная}}} &= \frac{E_{\text{полная механическая}}}2 = 0{,}050\,\text{Дж}.
    \end{align*}
}

\variantsplitter

\addpersonalvariant{Ислам Мунаев}

\tasknumber{1}%
\task{%
    Частота собственных малых колебаний пружинного маятника равна $\nu = 2\,\text{Гц}$.
    Чему станет равен период колебаний, если массу пружинного маятника увеличить в $4$ раз?
}
\answer{%
    $
        T'  = 2\pi\sqrt{\frac {m'}k}
            = 2\pi\sqrt{\frac {\alpha m}k}
            = \sqrt{\alpha} \cdot 2\pi\sqrt{\frac mk}
            = \sqrt{\alpha} \cdot T
            = T\sqrt{\alpha}
            = \frac 1\nu \cdot \sqrt{\alpha}
            = \frac{\sqrt{\alpha}}{\nu}
            = \frac{\sqrt{4}}{2\,\text{Гц}}
            = 1{,}00\,\text{с}
    $
}
\solutionspace{120pt}

\tasknumber{2}%
\task{%
    Сравните длины звуковой волны частотой $\nu_1 = 200\,\text{Гц}$ и радиоволны частотой $\nu_2 = 500\,\text{МГц}$.
    Какая больше, во сколько раз? Скорость звука примите равной $v = 320\,\frac{\text{м}}{\text{с}}$.
}
\answer{%
    $
        \lambda_1
            = v T_1 = v \cdot \frac 1{\nu_1} = \frac v{\nu_1}
            = \frac{320\,\frac{\text{м}}{\text{с}}}{200\,\text{Гц}} = 1{,}60\,\text{м},
        \quad
        \lambda_2
            = c T_2 = c \cdot \frac 1{\nu_2} = \frac c{\nu_2}
            = \frac{300\,\frac{\text{Мм}}{\text{с}}}{500\,\text{МГц}} = 0{,}60\,\text{м},
        \quad n = \frac{\lambda_2}{\lambda_1} \approx 0{,}37
    $
}
\solutionspace{120pt}

\tasknumber{3}%
\task{%
    Чему равна длина волны, если две точки среды, находящиеся на расстоянии $l = 40\,\text{см}$,
    совершают колебания с разностью фаз $\frac{\pi}{8}$?
}
\answer{%
    $
        \frac l\lambda = \frac \varphi{2\pi} + k (k\in\mathbb{N:L:s})
        \implies \lambda = \frac l{\frac \varphi{2\pi} + k} = \frac {2\pi l}{\varphi + 2\pi k},
        \quad \lambda_0 = \frac {2\pi l}{\varphi} = 640{,}00\,\text{см}
    $
}
\solutionspace{120pt}

\tasknumber{4}%
\task{%
    Тело массой $m = 400\,\text{г}$ совершает гармонические колебания.
    При этом амплитуда колебаний его скорости равна $v = 4\,\frac{\text{м}}{\text{с}}$.
    Определите запас полной механической энергии колебательной системы
    и амплитуду колебаний потенциальной энергии.
}
\answer{%
    \begin{align*}
            E_{\text{полная механическая}} &= E_{\text{max кинетическая}}
            = \frac{m v_{\max}^2}2 = \frac{400\,\text{г} \cdot \sqr{4\,\frac{\text{м}}{\text{с}}}}2 = 3{,}200\,\text{Дж}, \\
    A_{E_{\text{потенциальная}}} &= \frac{E_{\text{полная механическая}}}2 = 1{,}600\,\text{Дж}.
    \end{align*}
}

\variantsplitter

\addpersonalvariant{Александр Наумов}

\tasknumber{1}%
\task{%
    Частота собственных малых колебаний пружинного маятника равна $\nu = 2\,\text{Гц}$.
    Чему станет равен период колебаний, если массу пружинного маятника увеличить в $4$ раз?
}
\answer{%
    $
        T'  = 2\pi\sqrt{\frac {m'}k}
            = 2\pi\sqrt{\frac {\alpha m}k}
            = \sqrt{\alpha} \cdot 2\pi\sqrt{\frac mk}
            = \sqrt{\alpha} \cdot T
            = T\sqrt{\alpha}
            = \frac 1\nu \cdot \sqrt{\alpha}
            = \frac{\sqrt{\alpha}}{\nu}
            = \frac{\sqrt{4}}{2\,\text{Гц}}
            = 1{,}00\,\text{с}
    $
}
\solutionspace{120pt}

\tasknumber{2}%
\task{%
    Сравните длины звуковой волны частотой $\nu_1 = 200\,\text{Гц}$ и радиоволны частотой $\nu_2 = 200\,\text{МГц}$.
    Какая больше, во сколько раз? Скорость звука примите равной $v = 320\,\frac{\text{м}}{\text{с}}$.
}
\answer{%
    $
        \lambda_1
            = v T_1 = v \cdot \frac 1{\nu_1} = \frac v{\nu_1}
            = \frac{320\,\frac{\text{м}}{\text{с}}}{200\,\text{Гц}} = 1{,}60\,\text{м},
        \quad
        \lambda_2
            = c T_2 = c \cdot \frac 1{\nu_2} = \frac c{\nu_2}
            = \frac{300\,\frac{\text{Мм}}{\text{с}}}{200\,\text{МГц}} = 1{,}50\,\text{м},
        \quad n = \frac{\lambda_2}{\lambda_1} \approx 0{,}94
    $
}
\solutionspace{120pt}

\tasknumber{3}%
\task{%
    Чему равна длина волны, если две точки среды, находящиеся на расстоянии $l = 50\,\text{см}$,
    совершают колебания с разностью фаз $\frac{3\pi}{8}$?
}
\answer{%
    $
        \frac l\lambda = \frac \varphi{2\pi} + k (k\in\mathbb{N:L:s})
        \implies \lambda = \frac l{\frac \varphi{2\pi} + k} = \frac {2\pi l}{\varphi + 2\pi k},
        \quad \lambda_0 = \frac {2\pi l}{\varphi} = 266{,}67\,\text{см}
    $
}
\solutionspace{120pt}

\tasknumber{4}%
\task{%
    Тело массой $m = 250\,\text{г}$ совершает гармонические колебания.
    При этом амплитуда колебаний его скорости равна $v = 2\,\frac{\text{м}}{\text{с}}$.
    Определите запас полной механической энергии колебательной системы
    и амплитуду колебаний потенциальной энергии.
}
\answer{%
    \begin{align*}
            E_{\text{полная механическая}} &= E_{\text{max кинетическая}}
            = \frac{m v_{\max}^2}2 = \frac{250\,\text{г} \cdot \sqr{2\,\frac{\text{м}}{\text{с}}}}2 = 0{,}500\,\text{Дж}, \\
    A_{E_{\text{потенциальная}}} &= \frac{E_{\text{полная механическая}}}2 = 0{,}250\,\text{Дж}.
    \end{align*}
}

\variantsplitter

\addpersonalvariant{Георгий Новиков}

\tasknumber{1}%
\task{%
    Частота собственных малых колебаний пружинного маятника равна $\nu = 2\,\text{Гц}$.
    Чему станет равен период колебаний, если массу пружинного маятника увеличить в $4$ раз?
}
\answer{%
    $
        T'  = 2\pi\sqrt{\frac {m'}k}
            = 2\pi\sqrt{\frac {\alpha m}k}
            = \sqrt{\alpha} \cdot 2\pi\sqrt{\frac mk}
            = \sqrt{\alpha} \cdot T
            = T\sqrt{\alpha}
            = \frac 1\nu \cdot \sqrt{\alpha}
            = \frac{\sqrt{\alpha}}{\nu}
            = \frac{\sqrt{4}}{2\,\text{Гц}}
            = 1{,}00\,\text{с}
    $
}
\solutionspace{120pt}

\tasknumber{2}%
\task{%
    Сравните длины звуковой волны частотой $\nu_1 = 150\,\text{Гц}$ и радиоволны частотой $\nu_2 = 500\,\text{МГц}$.
    Какая больше, во сколько раз? Скорость звука примите равной $v = 320\,\frac{\text{м}}{\text{с}}$.
}
\answer{%
    $
        \lambda_1
            = v T_1 = v \cdot \frac 1{\nu_1} = \frac v{\nu_1}
            = \frac{320\,\frac{\text{м}}{\text{с}}}{150\,\text{Гц}} = 2{,}13\,\text{м},
        \quad
        \lambda_2
            = c T_2 = c \cdot \frac 1{\nu_2} = \frac c{\nu_2}
            = \frac{300\,\frac{\text{Мм}}{\text{с}}}{500\,\text{МГц}} = 0{,}60\,\text{м},
        \quad n = \frac{\lambda_2}{\lambda_1} \approx 0{,}28
    $
}
\solutionspace{120pt}

\tasknumber{3}%
\task{%
    Чему равна длина волны, если две точки среды, находящиеся на расстоянии $l = 50\,\text{см}$,
    совершают колебания с разностью фаз $\frac{\pi}{8}$?
}
\answer{%
    $
        \frac l\lambda = \frac \varphi{2\pi} + k (k\in\mathbb{N:L:s})
        \implies \lambda = \frac l{\frac \varphi{2\pi} + k} = \frac {2\pi l}{\varphi + 2\pi k},
        \quad \lambda_0 = \frac {2\pi l}{\varphi} = 800{,}00\,\text{см}
    $
}
\solutionspace{120pt}

\tasknumber{4}%
\task{%
    Тело массой $m = 400\,\text{г}$ совершает гармонические колебания.
    При этом амплитуда колебаний его скорости равна $v = 4\,\frac{\text{м}}{\text{с}}$.
    Определите запас полной механической энергии колебательной системы
    и амплитуду колебаний потенциальной энергии.
}
\answer{%
    \begin{align*}
            E_{\text{полная механическая}} &= E_{\text{max кинетическая}}
            = \frac{m v_{\max}^2}2 = \frac{400\,\text{г} \cdot \sqr{4\,\frac{\text{м}}{\text{с}}}}2 = 3{,}200\,\text{Дж}, \\
    A_{E_{\text{потенциальная}}} &= \frac{E_{\text{полная механическая}}}2 = 1{,}600\,\text{Дж}.
    \end{align*}
}

\variantsplitter

\addpersonalvariant{Егор Осипов}

\tasknumber{1}%
\task{%
    Частота собственных малых колебаний пружинного маятника равна $\nu = 5\,\text{Гц}$.
    Чему станет равен период колебаний, если массу пружинного маятника увеличить в $25$ раз?
}
\answer{%
    $
        T'  = 2\pi\sqrt{\frac {m'}k}
            = 2\pi\sqrt{\frac {\alpha m}k}
            = \sqrt{\alpha} \cdot 2\pi\sqrt{\frac mk}
            = \sqrt{\alpha} \cdot T
            = T\sqrt{\alpha}
            = \frac 1\nu \cdot \sqrt{\alpha}
            = \frac{\sqrt{\alpha}}{\nu}
            = \frac{\sqrt{25}}{5\,\text{Гц}}
            = 1{,}00\,\text{с}
    $
}
\solutionspace{120pt}

\tasknumber{2}%
\task{%
    Сравните длины звуковой волны частотой $\nu_1 = 300\,\text{Гц}$ и радиоволны частотой $\nu_2 = 800\,\text{МГц}$.
    Какая больше, во сколько раз? Скорость звука примите равной $v = 320\,\frac{\text{м}}{\text{с}}$.
}
\answer{%
    $
        \lambda_1
            = v T_1 = v \cdot \frac 1{\nu_1} = \frac v{\nu_1}
            = \frac{320\,\frac{\text{м}}{\text{с}}}{300\,\text{Гц}} = 1{,}07\,\text{м},
        \quad
        \lambda_2
            = c T_2 = c \cdot \frac 1{\nu_2} = \frac c{\nu_2}
            = \frac{300\,\frac{\text{Мм}}{\text{с}}}{800\,\text{МГц}} = 0{,}38\,\text{м},
        \quad n = \frac{\lambda_2}{\lambda_1} \approx 0{,}35
    $
}
\solutionspace{120pt}

\tasknumber{3}%
\task{%
    Чему равна длина волны, если две точки среды, находящиеся на расстоянии $l = 40\,\text{см}$,
    совершают колебания с разностью фаз $\frac{\pi}{2}$?
}
\answer{%
    $
        \frac l\lambda = \frac \varphi{2\pi} + k (k\in\mathbb{N:L:s})
        \implies \lambda = \frac l{\frac \varphi{2\pi} + k} = \frac {2\pi l}{\varphi + 2\pi k},
        \quad \lambda_0 = \frac {2\pi l}{\varphi} = 160{,}00\,\text{см}
    $
}
\solutionspace{120pt}

\tasknumber{4}%
\task{%
    Тело массой $m = 200\,\text{г}$ совершает гармонические колебания.
    При этом амплитуда колебаний его скорости равна $v = 4\,\frac{\text{м}}{\text{с}}$.
    Определите запас полной механической энергии колебательной системы
    и амплитуду колебаний потенциальной энергии.
}
\answer{%
    \begin{align*}
            E_{\text{полная механическая}} &= E_{\text{max кинетическая}}
            = \frac{m v_{\max}^2}2 = \frac{200\,\text{г} \cdot \sqr{4\,\frac{\text{м}}{\text{с}}}}2 = 1{,}600\,\text{Дж}, \\
    A_{E_{\text{потенциальная}}} &= \frac{E_{\text{полная механическая}}}2 = 0{,}800\,\text{Дж}.
    \end{align*}
}

\variantsplitter

\addpersonalvariant{Руслан Перепелица}

\tasknumber{1}%
\task{%
    Частота собственных малых колебаний пружинного маятника равна $\nu = 2\,\text{Гц}$.
    Чему станет равен период колебаний, если массу пружинного маятника увеличить в $16$ раз?
}
\answer{%
    $
        T'  = 2\pi\sqrt{\frac {m'}k}
            = 2\pi\sqrt{\frac {\alpha m}k}
            = \sqrt{\alpha} \cdot 2\pi\sqrt{\frac mk}
            = \sqrt{\alpha} \cdot T
            = T\sqrt{\alpha}
            = \frac 1\nu \cdot \sqrt{\alpha}
            = \frac{\sqrt{\alpha}}{\nu}
            = \frac{\sqrt{16}}{2\,\text{Гц}}
            = 2{,}00\,\text{с}
    $
}
\solutionspace{120pt}

\tasknumber{2}%
\task{%
    Сравните длины звуковой волны частотой $\nu_1 = 300\,\text{Гц}$ и радиоволны частотой $\nu_2 = 200\,\text{МГц}$.
    Какая больше, во сколько раз? Скорость звука примите равной $v = 320\,\frac{\text{м}}{\text{с}}$.
}
\answer{%
    $
        \lambda_1
            = v T_1 = v \cdot \frac 1{\nu_1} = \frac v{\nu_1}
            = \frac{320\,\frac{\text{м}}{\text{с}}}{300\,\text{Гц}} = 1{,}07\,\text{м},
        \quad
        \lambda_2
            = c T_2 = c \cdot \frac 1{\nu_2} = \frac c{\nu_2}
            = \frac{300\,\frac{\text{Мм}}{\text{с}}}{200\,\text{МГц}} = 1{,}50\,\text{м},
        \quad n = \frac{\lambda_2}{\lambda_1} \approx 1{,}41
    $
}
\solutionspace{120pt}

\tasknumber{3}%
\task{%
    Чему равна длина волны, если две точки среды, находящиеся на расстоянии $l = 40\,\text{см}$,
    совершают колебания с разностью фаз $\frac{\pi}{2}$?
}
\answer{%
    $
        \frac l\lambda = \frac \varphi{2\pi} + k (k\in\mathbb{N:L:s})
        \implies \lambda = \frac l{\frac \varphi{2\pi} + k} = \frac {2\pi l}{\varphi + 2\pi k},
        \quad \lambda_0 = \frac {2\pi l}{\varphi} = 160{,}00\,\text{см}
    $
}
\solutionspace{120pt}

\tasknumber{4}%
\task{%
    Тело массой $m = 250\,\text{г}$ совершает гармонические колебания.
    При этом амплитуда колебаний его скорости равна $v = 1\,\frac{\text{м}}{\text{с}}$.
    Определите запас полной механической энергии колебательной системы
    и амплитуду колебаний потенциальной энергии.
}
\answer{%
    \begin{align*}
            E_{\text{полная механическая}} &= E_{\text{max кинетическая}}
            = \frac{m v_{\max}^2}2 = \frac{250\,\text{г} \cdot \sqr{1\,\frac{\text{м}}{\text{с}}}}2 = 0{,}125\,\text{Дж}, \\
    A_{E_{\text{потенциальная}}} &= \frac{E_{\text{полная механическая}}}2 = 0{,}062\,\text{Дж}.
    \end{align*}
}

\variantsplitter

\addpersonalvariant{Михаил Перин}

\tasknumber{1}%
\task{%
    Частота собственных малых колебаний пружинного маятника равна $\nu = 8\,\text{Гц}$.
    Чему станет равен период колебаний, если массу пружинного маятника увеличить в $4$ раз?
}
\answer{%
    $
        T'  = 2\pi\sqrt{\frac {m'}k}
            = 2\pi\sqrt{\frac {\alpha m}k}
            = \sqrt{\alpha} \cdot 2\pi\sqrt{\frac mk}
            = \sqrt{\alpha} \cdot T
            = T\sqrt{\alpha}
            = \frac 1\nu \cdot \sqrt{\alpha}
            = \frac{\sqrt{\alpha}}{\nu}
            = \frac{\sqrt{4}}{8\,\text{Гц}}
            = 0{,}25\,\text{с}
    $
}
\solutionspace{120pt}

\tasknumber{2}%
\task{%
    Сравните длины звуковой волны частотой $\nu_1 = 300\,\text{Гц}$ и радиоволны частотой $\nu_2 = 500\,\text{МГц}$.
    Какая больше, во сколько раз? Скорость звука примите равной $v = 320\,\frac{\text{м}}{\text{с}}$.
}
\answer{%
    $
        \lambda_1
            = v T_1 = v \cdot \frac 1{\nu_1} = \frac v{\nu_1}
            = \frac{320\,\frac{\text{м}}{\text{с}}}{300\,\text{Гц}} = 1{,}07\,\text{м},
        \quad
        \lambda_2
            = c T_2 = c \cdot \frac 1{\nu_2} = \frac c{\nu_2}
            = \frac{300\,\frac{\text{Мм}}{\text{с}}}{500\,\text{МГц}} = 0{,}60\,\text{м},
        \quad n = \frac{\lambda_2}{\lambda_1} \approx 0{,}56
    $
}
\solutionspace{120pt}

\tasknumber{3}%
\task{%
    Чему равна длина волны, если две точки среды, находящиеся на расстоянии $l = 40\,\text{см}$,
    совершают колебания с разностью фаз $\frac{\pi}{2}$?
}
\answer{%
    $
        \frac l\lambda = \frac \varphi{2\pi} + k (k\in\mathbb{N:L:s})
        \implies \lambda = \frac l{\frac \varphi{2\pi} + k} = \frac {2\pi l}{\varphi + 2\pi k},
        \quad \lambda_0 = \frac {2\pi l}{\varphi} = 160{,}00\,\text{см}
    $
}
\solutionspace{120pt}

\tasknumber{4}%
\task{%
    Тело массой $M = 100\,\text{г}$ совершает гармонические колебания.
    При этом амплитуда колебаний его скорости равна $v = 4\,\frac{\text{м}}{\text{с}}$.
    Определите запас полной механической энергии колебательной системы
    и амплитуду колебаний потенциальной энергии.
}
\answer{%
    \begin{align*}
            E_{\text{полная механическая}} &= E_{\text{max кинетическая}}
            = \frac{m v_{\max}^2}2 = \frac{100\,\text{г} \cdot \sqr{4\,\frac{\text{м}}{\text{с}}}}2 = 0{,}800\,\text{Дж}, \\
    A_{E_{\text{потенциальная}}} &= \frac{E_{\text{полная механическая}}}2 = 0{,}400\,\text{Дж}.
    \end{align*}
}

\variantsplitter

\addpersonalvariant{Егор Подуровский}

\tasknumber{1}%
\task{%
    Частота собственных малых колебаний пружинного маятника равна $\nu = 8\,\text{Гц}$.
    Чему станет равен период колебаний, если массу пружинного маятника увеличить в $25$ раз?
}
\answer{%
    $
        T'  = 2\pi\sqrt{\frac {m'}k}
            = 2\pi\sqrt{\frac {\alpha m}k}
            = \sqrt{\alpha} \cdot 2\pi\sqrt{\frac mk}
            = \sqrt{\alpha} \cdot T
            = T\sqrt{\alpha}
            = \frac 1\nu \cdot \sqrt{\alpha}
            = \frac{\sqrt{\alpha}}{\nu}
            = \frac{\sqrt{25}}{8\,\text{Гц}}
            = 0{,}62\,\text{с}
    $
}
\solutionspace{120pt}

\tasknumber{2}%
\task{%
    Сравните длины звуковой волны частотой $\nu_1 = 500\,\text{Гц}$ и радиоволны частотой $\nu_2 = 200\,\text{МГц}$.
    Какая больше, во сколько раз? Скорость звука примите равной $v = 320\,\frac{\text{м}}{\text{с}}$.
}
\answer{%
    $
        \lambda_1
            = v T_1 = v \cdot \frac 1{\nu_1} = \frac v{\nu_1}
            = \frac{320\,\frac{\text{м}}{\text{с}}}{500\,\text{Гц}} = 0{,}64\,\text{м},
        \quad
        \lambda_2
            = c T_2 = c \cdot \frac 1{\nu_2} = \frac c{\nu_2}
            = \frac{300\,\frac{\text{Мм}}{\text{с}}}{200\,\text{МГц}} = 1{,}50\,\text{м},
        \quad n = \frac{\lambda_2}{\lambda_1} \approx 2{,}34
    $
}
\solutionspace{120pt}

\tasknumber{3}%
\task{%
    Чему равна длина волны, если две точки среды, находящиеся на расстоянии $l = 75\,\text{см}$,
    совершают колебания с разностью фаз $\frac{\pi}{8}$?
}
\answer{%
    $
        \frac l\lambda = \frac \varphi{2\pi} + k (k\in\mathbb{N:L:s})
        \implies \lambda = \frac l{\frac \varphi{2\pi} + k} = \frac {2\pi l}{\varphi + 2\pi k},
        \quad \lambda_0 = \frac {2\pi l}{\varphi} = 1200{,}00\,\text{см}
    $
}
\solutionspace{120pt}

\tasknumber{4}%
\task{%
    Тело массой $M = 250\,\text{г}$ совершает гармонические колебания.
    При этом амплитуда колебаний его скорости равна $v = 2\,\frac{\text{м}}{\text{с}}$.
    Определите запас полной механической энергии колебательной системы
    и амплитуду колебаний потенциальной энергии.
}
\answer{%
    \begin{align*}
            E_{\text{полная механическая}} &= E_{\text{max кинетическая}}
            = \frac{m v_{\max}^2}2 = \frac{250\,\text{г} \cdot \sqr{2\,\frac{\text{м}}{\text{с}}}}2 = 0{,}500\,\text{Дж}, \\
    A_{E_{\text{потенциальная}}} &= \frac{E_{\text{полная механическая}}}2 = 0{,}250\,\text{Дж}.
    \end{align*}
}

\variantsplitter

\addpersonalvariant{Роман Прибылов}

\tasknumber{1}%
\task{%
    Частота собственных малых колебаний пружинного маятника равна $\nu = 4\,\text{Гц}$.
    Чему станет равен период колебаний, если массу пружинного маятника увеличить в $25$ раз?
}
\answer{%
    $
        T'  = 2\pi\sqrt{\frac {m'}k}
            = 2\pi\sqrt{\frac {\alpha m}k}
            = \sqrt{\alpha} \cdot 2\pi\sqrt{\frac mk}
            = \sqrt{\alpha} \cdot T
            = T\sqrt{\alpha}
            = \frac 1\nu \cdot \sqrt{\alpha}
            = \frac{\sqrt{\alpha}}{\nu}
            = \frac{\sqrt{25}}{4\,\text{Гц}}
            = 1{,}25\,\text{с}
    $
}
\solutionspace{120pt}

\tasknumber{2}%
\task{%
    Сравните длины звуковой волны частотой $\nu_1 = 200\,\text{Гц}$ и радиоволны частотой $\nu_2 = 800\,\text{МГц}$.
    Какая больше, во сколько раз? Скорость звука примите равной $v = 320\,\frac{\text{м}}{\text{с}}$.
}
\answer{%
    $
        \lambda_1
            = v T_1 = v \cdot \frac 1{\nu_1} = \frac v{\nu_1}
            = \frac{320\,\frac{\text{м}}{\text{с}}}{200\,\text{Гц}} = 1{,}60\,\text{м},
        \quad
        \lambda_2
            = c T_2 = c \cdot \frac 1{\nu_2} = \frac c{\nu_2}
            = \frac{300\,\frac{\text{Мм}}{\text{с}}}{800\,\text{МГц}} = 0{,}38\,\text{м},
        \quad n = \frac{\lambda_2}{\lambda_1} \approx 0{,}23
    $
}
\solutionspace{120pt}

\tasknumber{3}%
\task{%
    Чему равна длина волны, если две точки среды, находящиеся на расстоянии $l = 20\,\text{см}$,
    совершают колебания с разностью фаз $\frac{2\pi}{5}$?
}
\answer{%
    $
        \frac l\lambda = \frac \varphi{2\pi} + k (k\in\mathbb{N:L:s})
        \implies \lambda = \frac l{\frac \varphi{2\pi} + k} = \frac {2\pi l}{\varphi + 2\pi k},
        \quad \lambda_0 = \frac {2\pi l}{\varphi} = 100{,}00\,\text{см}
    $
}
\solutionspace{120pt}

\tasknumber{4}%
\task{%
    Тело массой $M = 250\,\text{г}$ совершает гармонические колебания.
    При этом амплитуда колебаний его скорости равна $v = 5\,\frac{\text{м}}{\text{с}}$.
    Определите запас полной механической энергии колебательной системы
    и амплитуду колебаний потенциальной энергии.
}
\answer{%
    \begin{align*}
            E_{\text{полная механическая}} &= E_{\text{max кинетическая}}
            = \frac{m v_{\max}^2}2 = \frac{250\,\text{г} \cdot \sqr{5\,\frac{\text{м}}{\text{с}}}}2 = 3{,}125\,\text{Дж}, \\
    A_{E_{\text{потенциальная}}} &= \frac{E_{\text{полная механическая}}}2 = 1{,}562\,\text{Дж}.
    \end{align*}
}

\variantsplitter

\addpersonalvariant{Александр Селехметьев}

\tasknumber{1}%
\task{%
    Частота собственных малых колебаний пружинного маятника равна $\nu = 2\,\text{Гц}$.
    Чему станет равен период колебаний, если массу пружинного маятника увеличить в $16$ раз?
}
\answer{%
    $
        T'  = 2\pi\sqrt{\frac {m'}k}
            = 2\pi\sqrt{\frac {\alpha m}k}
            = \sqrt{\alpha} \cdot 2\pi\sqrt{\frac mk}
            = \sqrt{\alpha} \cdot T
            = T\sqrt{\alpha}
            = \frac 1\nu \cdot \sqrt{\alpha}
            = \frac{\sqrt{\alpha}}{\nu}
            = \frac{\sqrt{16}}{2\,\text{Гц}}
            = 2{,}00\,\text{с}
    $
}
\solutionspace{120pt}

\tasknumber{2}%
\task{%
    Сравните длины звуковой волны частотой $\nu_1 = 500\,\text{Гц}$ и радиоволны частотой $\nu_2 = 800\,\text{МГц}$.
    Какая больше, во сколько раз? Скорость звука примите равной $v = 320\,\frac{\text{м}}{\text{с}}$.
}
\answer{%
    $
        \lambda_1
            = v T_1 = v \cdot \frac 1{\nu_1} = \frac v{\nu_1}
            = \frac{320\,\frac{\text{м}}{\text{с}}}{500\,\text{Гц}} = 0{,}64\,\text{м},
        \quad
        \lambda_2
            = c T_2 = c \cdot \frac 1{\nu_2} = \frac c{\nu_2}
            = \frac{300\,\frac{\text{Мм}}{\text{с}}}{800\,\text{МГц}} = 0{,}38\,\text{м},
        \quad n = \frac{\lambda_2}{\lambda_1} \approx 0{,}59
    $
}
\solutionspace{120pt}

\tasknumber{3}%
\task{%
    Чему равна длина волны, если две точки среды, находящиеся на расстоянии $l = 25\,\text{см}$,
    совершают колебания с разностью фаз $\frac{2\pi}{5}$?
}
\answer{%
    $
        \frac l\lambda = \frac \varphi{2\pi} + k (k\in\mathbb{N:L:s})
        \implies \lambda = \frac l{\frac \varphi{2\pi} + k} = \frac {2\pi l}{\varphi + 2\pi k},
        \quad \lambda_0 = \frac {2\pi l}{\varphi} = 125{,}00\,\text{см}
    $
}
\solutionspace{120pt}

\tasknumber{4}%
\task{%
    Тело массой $m = 100\,\text{г}$ совершает гармонические колебания.
    При этом амплитуда колебаний его скорости равна $v = 5\,\frac{\text{м}}{\text{с}}$.
    Определите запас полной механической энергии колебательной системы
    и амплитуду колебаний потенциальной энергии.
}
\answer{%
    \begin{align*}
            E_{\text{полная механическая}} &= E_{\text{max кинетическая}}
            = \frac{m v_{\max}^2}2 = \frac{100\,\text{г} \cdot \sqr{5\,\frac{\text{м}}{\text{с}}}}2 = 1{,}250\,\text{Дж}, \\
    A_{E_{\text{потенциальная}}} &= \frac{E_{\text{полная механическая}}}2 = 0{,}625\,\text{Дж}.
    \end{align*}
}

\variantsplitter

\addpersonalvariant{Алексей Тихонов}

\tasknumber{1}%
\task{%
    Частота собственных малых колебаний пружинного маятника равна $\nu = 8\,\text{Гц}$.
    Чему станет равен период колебаний, если массу пружинного маятника увеличить в $4$ раз?
}
\answer{%
    $
        T'  = 2\pi\sqrt{\frac {m'}k}
            = 2\pi\sqrt{\frac {\alpha m}k}
            = \sqrt{\alpha} \cdot 2\pi\sqrt{\frac mk}
            = \sqrt{\alpha} \cdot T
            = T\sqrt{\alpha}
            = \frac 1\nu \cdot \sqrt{\alpha}
            = \frac{\sqrt{\alpha}}{\nu}
            = \frac{\sqrt{4}}{8\,\text{Гц}}
            = 0{,}25\,\text{с}
    $
}
\solutionspace{120pt}

\tasknumber{2}%
\task{%
    Сравните длины звуковой волны частотой $\nu_1 = 500\,\text{Гц}$ и радиоволны частотой $\nu_2 = 500\,\text{МГц}$.
    Какая больше, во сколько раз? Скорость звука примите равной $v = 320\,\frac{\text{м}}{\text{с}}$.
}
\answer{%
    $
        \lambda_1
            = v T_1 = v \cdot \frac 1{\nu_1} = \frac v{\nu_1}
            = \frac{320\,\frac{\text{м}}{\text{с}}}{500\,\text{Гц}} = 0{,}64\,\text{м},
        \quad
        \lambda_2
            = c T_2 = c \cdot \frac 1{\nu_2} = \frac c{\nu_2}
            = \frac{300\,\frac{\text{Мм}}{\text{с}}}{500\,\text{МГц}} = 0{,}60\,\text{м},
        \quad n = \frac{\lambda_2}{\lambda_1} \approx 0{,}94
    $
}
\solutionspace{120pt}

\tasknumber{3}%
\task{%
    Чему равна длина волны, если две точки среды, находящиеся на расстоянии $l = 75\,\text{см}$,
    совершают колебания с разностью фаз $\frac{\pi}{8}$?
}
\answer{%
    $
        \frac l\lambda = \frac \varphi{2\pi} + k (k\in\mathbb{N:L:s})
        \implies \lambda = \frac l{\frac \varphi{2\pi} + k} = \frac {2\pi l}{\varphi + 2\pi k},
        \quad \lambda_0 = \frac {2\pi l}{\varphi} = 1200{,}00\,\text{см}
    $
}
\solutionspace{120pt}

\tasknumber{4}%
\task{%
    Тело массой $m = 100\,\text{г}$ совершает гармонические колебания.
    При этом амплитуда колебаний его скорости равна $v = 1\,\frac{\text{м}}{\text{с}}$.
    Определите запас полной механической энергии колебательной системы
    и амплитуду колебаний потенциальной энергии.
}
\answer{%
    \begin{align*}
            E_{\text{полная механическая}} &= E_{\text{max кинетическая}}
            = \frac{m v_{\max}^2}2 = \frac{100\,\text{г} \cdot \sqr{1\,\frac{\text{м}}{\text{с}}}}2 = 0{,}050\,\text{Дж}, \\
    A_{E_{\text{потенциальная}}} &= \frac{E_{\text{полная механическая}}}2 = 0{,}025\,\text{Дж}.
    \end{align*}
}

\variantsplitter

\addpersonalvariant{Алина Филиппова}

\tasknumber{1}%
\task{%
    Частота собственных малых колебаний пружинного маятника равна $\nu = 2\,\text{Гц}$.
    Чему станет равен период колебаний, если массу пружинного маятника увеличить в $25$ раз?
}
\answer{%
    $
        T'  = 2\pi\sqrt{\frac {m'}k}
            = 2\pi\sqrt{\frac {\alpha m}k}
            = \sqrt{\alpha} \cdot 2\pi\sqrt{\frac mk}
            = \sqrt{\alpha} \cdot T
            = T\sqrt{\alpha}
            = \frac 1\nu \cdot \sqrt{\alpha}
            = \frac{\sqrt{\alpha}}{\nu}
            = \frac{\sqrt{25}}{2\,\text{Гц}}
            = 2{,}50\,\text{с}
    $
}
\solutionspace{120pt}

\tasknumber{2}%
\task{%
    Сравните длины звуковой волны частотой $\nu_1 = 150\,\text{Гц}$ и радиоволны частотой $\nu_2 = 200\,\text{МГц}$.
    Какая больше, во сколько раз? Скорость звука примите равной $v = 320\,\frac{\text{м}}{\text{с}}$.
}
\answer{%
    $
        \lambda_1
            = v T_1 = v \cdot \frac 1{\nu_1} = \frac v{\nu_1}
            = \frac{320\,\frac{\text{м}}{\text{с}}}{150\,\text{Гц}} = 2{,}13\,\text{м},
        \quad
        \lambda_2
            = c T_2 = c \cdot \frac 1{\nu_2} = \frac c{\nu_2}
            = \frac{300\,\frac{\text{Мм}}{\text{с}}}{200\,\text{МГц}} = 1{,}50\,\text{м},
        \quad n = \frac{\lambda_2}{\lambda_1} \approx 0{,}70
    $
}
\solutionspace{120pt}

\tasknumber{3}%
\task{%
    Чему равна длина волны, если две точки среды, находящиеся на расстоянии $l = 75\,\text{см}$,
    совершают колебания с разностью фаз $\frac{2\pi}{5}$?
}
\answer{%
    $
        \frac l\lambda = \frac \varphi{2\pi} + k (k\in\mathbb{N:L:s})
        \implies \lambda = \frac l{\frac \varphi{2\pi} + k} = \frac {2\pi l}{\varphi + 2\pi k},
        \quad \lambda_0 = \frac {2\pi l}{\varphi} = 375{,}00\,\text{см}
    $
}
\solutionspace{120pt}

\tasknumber{4}%
\task{%
    Тело массой $M = 200\,\text{г}$ совершает гармонические колебания.
    При этом амплитуда колебаний его скорости равна $v = 4\,\frac{\text{м}}{\text{с}}$.
    Определите запас полной механической энергии колебательной системы
    и амплитуду колебаний потенциальной энергии.
}
\answer{%
    \begin{align*}
            E_{\text{полная механическая}} &= E_{\text{max кинетическая}}
            = \frac{m v_{\max}^2}2 = \frac{200\,\text{г} \cdot \sqr{4\,\frac{\text{м}}{\text{с}}}}2 = 1{,}600\,\text{Дж}, \\
    A_{E_{\text{потенциальная}}} &= \frac{E_{\text{полная механическая}}}2 = 0{,}800\,\text{Дж}.
    \end{align*}
}

\variantsplitter

\addpersonalvariant{Дарья Шашкова}

\tasknumber{1}%
\task{%
    Частота собственных малых колебаний пружинного маятника равна $\nu = 2\,\text{Гц}$.
    Чему станет равен период колебаний, если массу пружинного маятника увеличить в $16$ раз?
}
\answer{%
    $
        T'  = 2\pi\sqrt{\frac {m'}k}
            = 2\pi\sqrt{\frac {\alpha m}k}
            = \sqrt{\alpha} \cdot 2\pi\sqrt{\frac mk}
            = \sqrt{\alpha} \cdot T
            = T\sqrt{\alpha}
            = \frac 1\nu \cdot \sqrt{\alpha}
            = \frac{\sqrt{\alpha}}{\nu}
            = \frac{\sqrt{16}}{2\,\text{Гц}}
            = 2{,}00\,\text{с}
    $
}
\solutionspace{120pt}

\tasknumber{2}%
\task{%
    Сравните длины звуковой волны частотой $\nu_1 = 150\,\text{Гц}$ и радиоволны частотой $\nu_2 = 200\,\text{МГц}$.
    Какая больше, во сколько раз? Скорость звука примите равной $v = 320\,\frac{\text{м}}{\text{с}}$.
}
\answer{%
    $
        \lambda_1
            = v T_1 = v \cdot \frac 1{\nu_1} = \frac v{\nu_1}
            = \frac{320\,\frac{\text{м}}{\text{с}}}{150\,\text{Гц}} = 2{,}13\,\text{м},
        \quad
        \lambda_2
            = c T_2 = c \cdot \frac 1{\nu_2} = \frac c{\nu_2}
            = \frac{300\,\frac{\text{Мм}}{\text{с}}}{200\,\text{МГц}} = 1{,}50\,\text{м},
        \quad n = \frac{\lambda_2}{\lambda_1} \approx 0{,}70
    $
}
\solutionspace{120pt}

\tasknumber{3}%
\task{%
    Чему равна длина волны, если две точки среды, находящиеся на расстоянии $l = 20\,\text{см}$,
    совершают колебания с разностью фаз $\frac{\pi}{8}$?
}
\answer{%
    $
        \frac l\lambda = \frac \varphi{2\pi} + k (k\in\mathbb{N:L:s})
        \implies \lambda = \frac l{\frac \varphi{2\pi} + k} = \frac {2\pi l}{\varphi + 2\pi k},
        \quad \lambda_0 = \frac {2\pi l}{\varphi} = 320{,}00\,\text{см}
    $
}
\solutionspace{120pt}

\tasknumber{4}%
\task{%
    Тело массой $m = 100\,\text{г}$ совершает гармонические колебания.
    При этом амплитуда колебаний его скорости равна $v = 4\,\frac{\text{м}}{\text{с}}$.
    Определите запас полной механической энергии колебательной системы
    и амплитуду колебаний потенциальной энергии.
}
\answer{%
    \begin{align*}
            E_{\text{полная механическая}} &= E_{\text{max кинетическая}}
            = \frac{m v_{\max}^2}2 = \frac{100\,\text{г} \cdot \sqr{4\,\frac{\text{м}}{\text{с}}}}2 = 0{,}800\,\text{Дж}, \\
    A_{E_{\text{потенциальная}}} &= \frac{E_{\text{полная механическая}}}2 = 0{,}400\,\text{Дж}.
    \end{align*}
}

\variantsplitter

\addpersonalvariant{Алина Яшина}

\tasknumber{1}%
\task{%
    Частота собственных малых колебаний пружинного маятника равна $\nu = 8\,\text{Гц}$.
    Чему станет равен период колебаний, если массу пружинного маятника увеличить в $25$ раз?
}
\answer{%
    $
        T'  = 2\pi\sqrt{\frac {m'}k}
            = 2\pi\sqrt{\frac {\alpha m}k}
            = \sqrt{\alpha} \cdot 2\pi\sqrt{\frac mk}
            = \sqrt{\alpha} \cdot T
            = T\sqrt{\alpha}
            = \frac 1\nu \cdot \sqrt{\alpha}
            = \frac{\sqrt{\alpha}}{\nu}
            = \frac{\sqrt{25}}{8\,\text{Гц}}
            = 0{,}62\,\text{с}
    $
}
\solutionspace{120pt}

\tasknumber{2}%
\task{%
    Сравните длины звуковой волны частотой $\nu_1 = 300\,\text{Гц}$ и радиоволны частотой $\nu_2 = 200\,\text{МГц}$.
    Какая больше, во сколько раз? Скорость звука примите равной $v = 320\,\frac{\text{м}}{\text{с}}$.
}
\answer{%
    $
        \lambda_1
            = v T_1 = v \cdot \frac 1{\nu_1} = \frac v{\nu_1}
            = \frac{320\,\frac{\text{м}}{\text{с}}}{300\,\text{Гц}} = 1{,}07\,\text{м},
        \quad
        \lambda_2
            = c T_2 = c \cdot \frac 1{\nu_2} = \frac c{\nu_2}
            = \frac{300\,\frac{\text{Мм}}{\text{с}}}{200\,\text{МГц}} = 1{,}50\,\text{м},
        \quad n = \frac{\lambda_2}{\lambda_1} \approx 1{,}41
    $
}
\solutionspace{120pt}

\tasknumber{3}%
\task{%
    Чему равна длина волны, если две точки среды, находящиеся на расстоянии $l = 50\,\text{см}$,
    совершают колебания с разностью фаз $\frac{3\pi}{4}$?
}
\answer{%
    $
        \frac l\lambda = \frac \varphi{2\pi} + k (k\in\mathbb{N:L:s})
        \implies \lambda = \frac l{\frac \varphi{2\pi} + k} = \frac {2\pi l}{\varphi + 2\pi k},
        \quad \lambda_0 = \frac {2\pi l}{\varphi} = 133{,}33\,\text{см}
    $
}
\solutionspace{120pt}

\tasknumber{4}%
\task{%
    Тело массой $M = 400\,\text{г}$ совершает гармонические колебания.
    При этом амплитуда колебаний его скорости равна $v = 2\,\frac{\text{м}}{\text{с}}$.
    Определите запас полной механической энергии колебательной системы
    и амплитуду колебаний потенциальной энергии.
}
\answer{%
    \begin{align*}
            E_{\text{полная механическая}} &= E_{\text{max кинетическая}}
            = \frac{m v_{\max}^2}2 = \frac{400\,\text{г} \cdot \sqr{2\,\frac{\text{м}}{\text{с}}}}2 = 0{,}800\,\text{Дж}, \\
    A_{E_{\text{потенциальная}}} &= \frac{E_{\text{полная механическая}}}2 = 0{,}400\,\text{Дж}.
    \end{align*}
}
% autogenerated
