\setdate{4~марта~2020}
\setclass{11«Т»}

\addpersonalvariant{Михаил Бурмистров}

\tasknumber{1}%
\task{%
    Длина волны света в~вакууме $\lambda = 500\,\text{нм}$.
    Какова частота этой световой волны?
    Какова длина этой волны в среде с показателем преломления $n = 1{,}4$?
    Может ли человек увидеть такую волну света, и если да, то какой именно цвет соответствует этим волнам в вакууме и в этой среде?
}
\answer{%
    \begin{align*}
    \nu &= \frac 1T = \frac 1{\lambda/c} = \frac c\lambda = \frac{3 \cdot 10^{8}\,\frac{\text{м}}{\text{с}}}{500\,\text{нм}} \approx 600 \cdot 10^{12}\,\text{Гц}, \\
    \nu' &= \nu \cbr{\text{или } T' = T} \implies \lambda' = v'T' = \frac vn T = \frac{ vt }n = \frac \lambda n = \frac{500\,\text{нм}}{1{,}4} \approx 360\,\text{нм}.
    \\
    &\text{380 нм---фиол---440---син---485---гол---500---зел---565---жёл---590---оранж---625---крас---780 нм}, \text{увидит}
    \end{align*}
}
\solutionspace{60pt}

\tasknumber{2}%
\task{%
    Установка для наблюдения интерференции состоит
    из двух когерентных источников света и экрана.
    Расстояние между источниками $l = 1{,}5\,\text{мм}$,
    а от каждого источника до экрана — $L = 3\,\text{м}$.
    Сделайте рисунок и укажите положение нулевого максимума освещенности,
    а также определите расстояние между вторым максимумом и нулевым максимумом.
    Длина волны падающего света составляет $\lambda = 500\,\text{нм}$.
}
\answer{%
    \begin{align*}
    l_1^2 &= L^2 + \sqr{x - \frac \ell 2} \\
    l_2^2 &= L^2 + \sqr{x + \frac \ell 2} \\
    l_2^2 - l_1^2 &= 2x\ell \implies (l_2 - l_1)(l_2 + l_1) = 2x\ell \implies n\lambda \cdot 2L \approx 2x_n\ell \implies x_n = \frac{\lambda L}{\ell} n, n\in \mathbb{N} \\
    x &= \frac{\lambda L}{\ell} \cdot 2 = \frac{500\,\text{нм} \cdot 3\,\text{м}}{1{,}5\,\text{мм}} \cdot 2 \approx 2\,\text{мм}
    \end{align*}
}
\solutionspace{180pt}

\tasknumber{3}%
\task{%
    На стеклянную пластинку ($\hat n = 1{,}5$) нанесена прозрачная пленка ($n = 1{,}4$).
    На плёнку нормально к поверхности падает монохроматический свет с длиной волны $480\,\text{нм}$.
    Какова должна быть минимальная толщина пленки, чтобы в результате интерференции отражённый свет имел наибольшую интенсивность?
}
\answer{%
    $2 \cdot h \cdot n = 1 \lambda \implies h \approx 171\,\text{нм}$
}

\variantsplitter

\addpersonalvariant{Гагик Аракелян}

\tasknumber{1}%
\task{%
    Длина волны света в~вакууме $\lambda = 400\,\text{нм}$.
    Какова частота этой световой волны?
    Какова длина этой волны в среде с показателем преломления $n = 1{,}3$?
    Может ли человек увидеть такую волну света, и если да, то какой именно цвет соответствует этим волнам в вакууме и в этой среде?
}
\answer{%
    \begin{align*}
    \nu &= \frac 1T = \frac 1{\lambda/c} = \frac c\lambda = \frac{3 \cdot 10^{8}\,\frac{\text{м}}{\text{с}}}{400\,\text{нм}} \approx 750 \cdot 10^{12}\,\text{Гц}, \\
    \nu' &= \nu \cbr{\text{или } T' = T} \implies \lambda' = v'T' = \frac vn T = \frac{ vt }n = \frac \lambda n = \frac{400\,\text{нм}}{1{,}3} \approx 310\,\text{нм}.
    \\
    &\text{380 нм---фиол---440---син---485---гол---500---зел---565---жёл---590---оранж---625---крас---780 нм}, \text{увидит}
    \end{align*}
}
\solutionspace{60pt}

\tasknumber{2}%
\task{%
    Установка для наблюдения интерференции состоит
    из двух когерентных источников света и экрана.
    Расстояние между источниками $l = 2{,}4\,\text{мм}$,
    а от каждого источника до экрана — $L = 4\,\text{м}$.
    Сделайте рисунок и укажите положение нулевого максимума освещенности,
    а также определите расстояние между четвёртым минимумом и нулевым максимумом.
    Длина волны падающего света составляет $\lambda = 550\,\text{нм}$.
}
\answer{%
    \begin{align*}
    l_1^2 &= L^2 + \sqr{x - \frac \ell 2} \\
    l_2^2 &= L^2 + \sqr{x + \frac \ell 2} \\
    l_2^2 - l_1^2 &= 2x\ell \implies (l_2 - l_1)(l_2 + l_1) = 2x\ell \implies n\lambda \cdot 2L \approx 2x_n\ell \implies x_n = \frac{\lambda L}{\ell} n, n\in \mathbb{N} \\
    x &= \frac{\lambda L}{\ell} \cdot \frac72 = \frac{550\,\text{нм} \cdot 4\,\text{м}}{2{,}4\,\text{мм}} \cdot \frac72 \approx 3{,}2\,\text{мм}
    \end{align*}
}
\solutionspace{180pt}

\tasknumber{3}%
\task{%
    На стеклянную пластинку ($\hat n = 1{,}5$) нанесена прозрачная пленка ($n = 1{,}3$).
    На плёнку нормально к поверхности падает монохроматический свет с длиной волны $640\,\text{нм}$.
    Какова должна быть минимальная толщина пленки, чтобы в результате интерференции отражённый свет имел наименьшую интенсивность?
}
\answer{%
    $2 \cdot h \cdot n = \frac12 \lambda \implies h \approx 123\,\text{нм}$
}

\variantsplitter

\addpersonalvariant{Ирен Аракелян}

\tasknumber{1}%
\task{%
    Длина волны света в~вакууме $\lambda = 400\,\text{нм}$.
    Какова частота этой световой волны?
    Какова длина этой волны в среде с показателем преломления $n = 1{,}6$?
    Может ли человек увидеть такую волну света, и если да, то какой именно цвет соответствует этим волнам в вакууме и в этой среде?
}
\answer{%
    \begin{align*}
    \nu &= \frac 1T = \frac 1{\lambda/c} = \frac c\lambda = \frac{3 \cdot 10^{8}\,\frac{\text{м}}{\text{с}}}{400\,\text{нм}} \approx 750 \cdot 10^{12}\,\text{Гц}, \\
    \nu' &= \nu \cbr{\text{или } T' = T} \implies \lambda' = v'T' = \frac vn T = \frac{ vt }n = \frac \lambda n = \frac{400\,\text{нм}}{1{,}6} \approx 250\,\text{нм}.
    \\
    &\text{380 нм---фиол---440---син---485---гол---500---зел---565---жёл---590---оранж---625---крас---780 нм}, \text{увидит}
    \end{align*}
}
\solutionspace{60pt}

\tasknumber{2}%
\task{%
    Установка для наблюдения интерференции состоит
    из двух когерентных источников света и экрана.
    Расстояние между источниками $l = 1{,}5\,\text{мм}$,
    а от каждого источника до экрана — $L = 4\,\text{м}$.
    Сделайте рисунок и укажите положение нулевого максимума освещенности,
    а также определите расстояние между четвёртым максимумом и нулевым максимумом.
    Длина волны падающего света составляет $\lambda = 550\,\text{нм}$.
}
\answer{%
    \begin{align*}
    l_1^2 &= L^2 + \sqr{x - \frac \ell 2} \\
    l_2^2 &= L^2 + \sqr{x + \frac \ell 2} \\
    l_2^2 - l_1^2 &= 2x\ell \implies (l_2 - l_1)(l_2 + l_1) = 2x\ell \implies n\lambda \cdot 2L \approx 2x_n\ell \implies x_n = \frac{\lambda L}{\ell} n, n\in \mathbb{N} \\
    x &= \frac{\lambda L}{\ell} \cdot 4 = \frac{550\,\text{нм} \cdot 4\,\text{м}}{1{,}5\,\text{мм}} \cdot 4 \approx 5{,}9\,\text{мм}
    \end{align*}
}
\solutionspace{180pt}

\tasknumber{3}%
\task{%
    На стеклянную пластинку ($\hat n = 1{,}6$) нанесена прозрачная пленка ($n = 1{,}4$).
    На плёнку нормально к поверхности падает монохроматический свет с длиной волны $420\,\text{нм}$.
    Какова должна быть минимальная толщина пленки, чтобы в результате интерференции отражённый свет имел наибольшую интенсивность?
}
\answer{%
    $2 \cdot h \cdot n = 1 \lambda \implies h \approx 150\,\text{нм}$
}

\variantsplitter

\addpersonalvariant{Сабина Асадуллаева}

\tasknumber{1}%
\task{%
    Длина волны света в~вакууме $\lambda = 600\,\text{нм}$.
    Какова частота этой световой волны?
    Какова длина этой волны в среде с показателем преломления $n = 1{,}7$?
    Может ли человек увидеть такую волну света, и если да, то какой именно цвет соответствует этим волнам в вакууме и в этой среде?
}
\answer{%
    \begin{align*}
    \nu &= \frac 1T = \frac 1{\lambda/c} = \frac c\lambda = \frac{3 \cdot 10^{8}\,\frac{\text{м}}{\text{с}}}{600\,\text{нм}} \approx 500 \cdot 10^{12}\,\text{Гц}, \\
    \nu' &= \nu \cbr{\text{или } T' = T} \implies \lambda' = v'T' = \frac vn T = \frac{ vt }n = \frac \lambda n = \frac{600\,\text{нм}}{1{,}7} \approx 350\,\text{нм}.
    \\
    &\text{380 нм---фиол---440---син---485---гол---500---зел---565---жёл---590---оранж---625---крас---780 нм}, \text{увидит}
    \end{align*}
}
\solutionspace{60pt}

\tasknumber{2}%
\task{%
    Установка для наблюдения интерференции состоит
    из двух когерентных источников света и экрана.
    Расстояние между источниками $l = 1{,}5\,\text{мм}$,
    а от каждого источника до экрана — $L = 2\,\text{м}$.
    Сделайте рисунок и укажите положение нулевого максимума освещенности,
    а также определите расстояние между третьим максимумом и нулевым максимумом.
    Длина волны падающего света составляет $\lambda = 400\,\text{нм}$.
}
\answer{%
    \begin{align*}
    l_1^2 &= L^2 + \sqr{x - \frac \ell 2} \\
    l_2^2 &= L^2 + \sqr{x + \frac \ell 2} \\
    l_2^2 - l_1^2 &= 2x\ell \implies (l_2 - l_1)(l_2 + l_1) = 2x\ell \implies n\lambda \cdot 2L \approx 2x_n\ell \implies x_n = \frac{\lambda L}{\ell} n, n\in \mathbb{N} \\
    x &= \frac{\lambda L}{\ell} \cdot 3 = \frac{400\,\text{нм} \cdot 2\,\text{м}}{1{,}5\,\text{мм}} \cdot 3 \approx 1{,}60\,\text{мм}
    \end{align*}
}
\solutionspace{180pt}

\tasknumber{3}%
\task{%
    На стеклянную пластинку ($\hat n = 1{,}6$) нанесена прозрачная пленка ($n = 1{,}4$).
    На плёнку нормально к поверхности падает монохроматический свет с длиной волны $540\,\text{нм}$.
    Какова должна быть минимальная толщина пленки, чтобы в результате интерференции отражённый свет имел наименьшую интенсивность?
}
\answer{%
    $2 \cdot h \cdot n = \frac12 \lambda \implies h \approx 96\,\text{нм}$
}

\variantsplitter

\addpersonalvariant{Вероника Битерякова}

\tasknumber{1}%
\task{%
    Длина волны света в~вакууме $\lambda = 500\,\text{нм}$.
    Какова частота этой световой волны?
    Какова длина этой волны в среде с показателем преломления $n = 1{,}7$?
    Может ли человек увидеть такую волну света, и если да, то какой именно цвет соответствует этим волнам в вакууме и в этой среде?
}
\answer{%
    \begin{align*}
    \nu &= \frac 1T = \frac 1{\lambda/c} = \frac c\lambda = \frac{3 \cdot 10^{8}\,\frac{\text{м}}{\text{с}}}{500\,\text{нм}} \approx 600 \cdot 10^{12}\,\text{Гц}, \\
    \nu' &= \nu \cbr{\text{или } T' = T} \implies \lambda' = v'T' = \frac vn T = \frac{ vt }n = \frac \lambda n = \frac{500\,\text{нм}}{1{,}7} \approx 290\,\text{нм}.
    \\
    &\text{380 нм---фиол---440---син---485---гол---500---зел---565---жёл---590---оранж---625---крас---780 нм}, \text{увидит}
    \end{align*}
}
\solutionspace{60pt}

\tasknumber{2}%
\task{%
    Установка для наблюдения интерференции состоит
    из двух когерентных источников света и экрана.
    Расстояние между источниками $l = 1{,}2\,\text{мм}$,
    а от каждого источника до экрана — $L = 3\,\text{м}$.
    Сделайте рисунок и укажите положение нулевого максимума освещенности,
    а также определите расстояние между четвёртым минимумом и нулевым максимумом.
    Длина волны падающего света составляет $\lambda = 600\,\text{нм}$.
}
\answer{%
    \begin{align*}
    l_1^2 &= L^2 + \sqr{x - \frac \ell 2} \\
    l_2^2 &= L^2 + \sqr{x + \frac \ell 2} \\
    l_2^2 - l_1^2 &= 2x\ell \implies (l_2 - l_1)(l_2 + l_1) = 2x\ell \implies n\lambda \cdot 2L \approx 2x_n\ell \implies x_n = \frac{\lambda L}{\ell} n, n\in \mathbb{N} \\
    x &= \frac{\lambda L}{\ell} \cdot \frac72 = \frac{600\,\text{нм} \cdot 3\,\text{м}}{1{,}2\,\text{мм}} \cdot \frac72 \approx 5{,}3\,\text{мм}
    \end{align*}
}
\solutionspace{180pt}

\tasknumber{3}%
\task{%
    На стеклянную пластинку ($\hat n = 1{,}6$) нанесена прозрачная пленка ($n = 1{,}7$).
    На плёнку нормально к поверхности падает монохроматический свет с длиной волны $540\,\text{нм}$.
    Какова должна быть минимальная толщина пленки, чтобы в результате интерференции отражённый свет имел наименьшую интенсивность?
}
\answer{%
    $2 \cdot h \cdot n = 1 \lambda \implies h \approx 159\,\text{нм}$
}

\variantsplitter

\addpersonalvariant{Юлия Буянова}

\tasknumber{1}%
\task{%
    Длина волны света в~вакууме $\lambda = 400\,\text{нм}$.
    Какова частота этой световой волны?
    Какова длина этой волны в среде с показателем преломления $n = 1{,}5$?
    Может ли человек увидеть такую волну света, и если да, то какой именно цвет соответствует этим волнам в вакууме и в этой среде?
}
\answer{%
    \begin{align*}
    \nu &= \frac 1T = \frac 1{\lambda/c} = \frac c\lambda = \frac{3 \cdot 10^{8}\,\frac{\text{м}}{\text{с}}}{400\,\text{нм}} \approx 750 \cdot 10^{12}\,\text{Гц}, \\
    \nu' &= \nu \cbr{\text{или } T' = T} \implies \lambda' = v'T' = \frac vn T = \frac{ vt }n = \frac \lambda n = \frac{400\,\text{нм}}{1{,}5} \approx 270\,\text{нм}.
    \\
    &\text{380 нм---фиол---440---син---485---гол---500---зел---565---жёл---590---оранж---625---крас---780 нм}, \text{увидит}
    \end{align*}
}
\solutionspace{60pt}

\tasknumber{2}%
\task{%
    Установка для наблюдения интерференции состоит
    из двух когерентных источников света и экрана.
    Расстояние между источниками $l = 1{,}5\,\text{мм}$,
    а от каждого источника до экрана — $L = 4\,\text{м}$.
    Сделайте рисунок и укажите положение нулевого максимума освещенности,
    а также определите расстояние между четвёртым максимумом и нулевым максимумом.
    Длина волны падающего света составляет $\lambda = 500\,\text{нм}$.
}
\answer{%
    \begin{align*}
    l_1^2 &= L^2 + \sqr{x - \frac \ell 2} \\
    l_2^2 &= L^2 + \sqr{x + \frac \ell 2} \\
    l_2^2 - l_1^2 &= 2x\ell \implies (l_2 - l_1)(l_2 + l_1) = 2x\ell \implies n\lambda \cdot 2L \approx 2x_n\ell \implies x_n = \frac{\lambda L}{\ell} n, n\in \mathbb{N} \\
    x &= \frac{\lambda L}{\ell} \cdot 4 = \frac{500\,\text{нм} \cdot 4\,\text{м}}{1{,}5\,\text{мм}} \cdot 4 \approx 5{,}3\,\text{мм}
    \end{align*}
}
\solutionspace{180pt}

\tasknumber{3}%
\task{%
    На стеклянную пластинку ($\hat n = 1{,}6$) нанесена прозрачная пленка ($n = 1{,}4$).
    На плёнку нормально к поверхности падает монохроматический свет с длиной волны $420\,\text{нм}$.
    Какова должна быть минимальная толщина пленки, чтобы в результате интерференции отражённый свет имел наибольшую интенсивность?
}
\answer{%
    $2 \cdot h \cdot n = 1 \lambda \implies h \approx 150\,\text{нм}$
}

\variantsplitter

\addpersonalvariant{Пелагея Вдовина}

\tasknumber{1}%
\task{%
    Длина волны света в~вакууме $\lambda = 600\,\text{нм}$.
    Какова частота этой световой волны?
    Какова длина этой волны в среде с показателем преломления $n = 1{,}4$?
    Может ли человек увидеть такую волну света, и если да, то какой именно цвет соответствует этим волнам в вакууме и в этой среде?
}
\answer{%
    \begin{align*}
    \nu &= \frac 1T = \frac 1{\lambda/c} = \frac c\lambda = \frac{3 \cdot 10^{8}\,\frac{\text{м}}{\text{с}}}{600\,\text{нм}} \approx 500 \cdot 10^{12}\,\text{Гц}, \\
    \nu' &= \nu \cbr{\text{или } T' = T} \implies \lambda' = v'T' = \frac vn T = \frac{ vt }n = \frac \lambda n = \frac{600\,\text{нм}}{1{,}4} \approx 430\,\text{нм}.
    \\
    &\text{380 нм---фиол---440---син---485---гол---500---зел---565---жёл---590---оранж---625---крас---780 нм}, \text{увидит}
    \end{align*}
}
\solutionspace{60pt}

\tasknumber{2}%
\task{%
    Установка для наблюдения интерференции состоит
    из двух когерентных источников света и экрана.
    Расстояние между источниками $l = 0{,}8\,\text{мм}$,
    а от каждого источника до экрана — $L = 4\,\text{м}$.
    Сделайте рисунок и укажите положение нулевого максимума освещенности,
    а также определите расстояние между вторым максимумом и нулевым максимумом.
    Длина волны падающего света составляет $\lambda = 400\,\text{нм}$.
}
\answer{%
    \begin{align*}
    l_1^2 &= L^2 + \sqr{x - \frac \ell 2} \\
    l_2^2 &= L^2 + \sqr{x + \frac \ell 2} \\
    l_2^2 - l_1^2 &= 2x\ell \implies (l_2 - l_1)(l_2 + l_1) = 2x\ell \implies n\lambda \cdot 2L \approx 2x_n\ell \implies x_n = \frac{\lambda L}{\ell} n, n\in \mathbb{N} \\
    x &= \frac{\lambda L}{\ell} \cdot 2 = \frac{400\,\text{нм} \cdot 4\,\text{м}}{0{,}8\,\text{мм}} \cdot 2 \approx 4\,\text{мм}
    \end{align*}
}
\solutionspace{180pt}

\tasknumber{3}%
\task{%
    На стеклянную пластинку ($\hat n = 1{,}5$) нанесена прозрачная пленка ($n = 1{,}8$).
    На плёнку нормально к поверхности падает монохроматический свет с длиной волны $480\,\text{нм}$.
    Какова должна быть минимальная толщина пленки, чтобы в результате интерференции отражённый свет имел наименьшую интенсивность?
}
\answer{%
    $2 \cdot h \cdot n = 1 \lambda \implies h \approx 133\,\text{нм}$
}

\variantsplitter

\addpersonalvariant{Леонид Викторов}

\tasknumber{1}%
\task{%
    Длина волны света в~вакууме $\lambda = 500\,\text{нм}$.
    Какова частота этой световой волны?
    Какова длина этой волны в среде с показателем преломления $n = 1{,}4$?
    Может ли человек увидеть такую волну света, и если да, то какой именно цвет соответствует этим волнам в вакууме и в этой среде?
}
\answer{%
    \begin{align*}
    \nu &= \frac 1T = \frac 1{\lambda/c} = \frac c\lambda = \frac{3 \cdot 10^{8}\,\frac{\text{м}}{\text{с}}}{500\,\text{нм}} \approx 600 \cdot 10^{12}\,\text{Гц}, \\
    \nu' &= \nu \cbr{\text{или } T' = T} \implies \lambda' = v'T' = \frac vn T = \frac{ vt }n = \frac \lambda n = \frac{500\,\text{нм}}{1{,}4} \approx 360\,\text{нм}.
    \\
    &\text{380 нм---фиол---440---син---485---гол---500---зел---565---жёл---590---оранж---625---крас---780 нм}, \text{увидит}
    \end{align*}
}
\solutionspace{60pt}

\tasknumber{2}%
\task{%
    Установка для наблюдения интерференции состоит
    из двух когерентных источников света и экрана.
    Расстояние между источниками $l = 1{,}2\,\text{мм}$,
    а от каждого источника до экрана — $L = 3\,\text{м}$.
    Сделайте рисунок и укажите положение нулевого максимума освещенности,
    а также определите расстояние между четвёртым максимумом и нулевым максимумом.
    Длина волны падающего света составляет $\lambda = 550\,\text{нм}$.
}
\answer{%
    \begin{align*}
    l_1^2 &= L^2 + \sqr{x - \frac \ell 2} \\
    l_2^2 &= L^2 + \sqr{x + \frac \ell 2} \\
    l_2^2 - l_1^2 &= 2x\ell \implies (l_2 - l_1)(l_2 + l_1) = 2x\ell \implies n\lambda \cdot 2L \approx 2x_n\ell \implies x_n = \frac{\lambda L}{\ell} n, n\in \mathbb{N} \\
    x &= \frac{\lambda L}{\ell} \cdot 4 = \frac{550\,\text{нм} \cdot 3\,\text{м}}{1{,}2\,\text{мм}} \cdot 4 \approx 5{,}5\,\text{мм}
    \end{align*}
}
\solutionspace{180pt}

\tasknumber{3}%
\task{%
    На стеклянную пластинку ($\hat n = 1{,}5$) нанесена прозрачная пленка ($n = 1{,}4$).
    На плёнку нормально к поверхности падает монохроматический свет с длиной волны $480\,\text{нм}$.
    Какова должна быть минимальная толщина пленки, чтобы в результате интерференции отражённый свет имел наименьшую интенсивность?
}
\answer{%
    $2 \cdot h \cdot n = \frac12 \lambda \implies h \approx 86\,\text{нм}$
}

\variantsplitter

\addpersonalvariant{Фёдор Гнутов}

\tasknumber{1}%
\task{%
    Длина волны света в~вакууме $\lambda = 500\,\text{нм}$.
    Какова частота этой световой волны?
    Какова длина этой волны в среде с показателем преломления $n = 1{,}6$?
    Может ли человек увидеть такую волну света, и если да, то какой именно цвет соответствует этим волнам в вакууме и в этой среде?
}
\answer{%
    \begin{align*}
    \nu &= \frac 1T = \frac 1{\lambda/c} = \frac c\lambda = \frac{3 \cdot 10^{8}\,\frac{\text{м}}{\text{с}}}{500\,\text{нм}} \approx 600 \cdot 10^{12}\,\text{Гц}, \\
    \nu' &= \nu \cbr{\text{или } T' = T} \implies \lambda' = v'T' = \frac vn T = \frac{ vt }n = \frac \lambda n = \frac{500\,\text{нм}}{1{,}6} \approx 310\,\text{нм}.
    \\
    &\text{380 нм---фиол---440---син---485---гол---500---зел---565---жёл---590---оранж---625---крас---780 нм}, \text{увидит}
    \end{align*}
}
\solutionspace{60pt}

\tasknumber{2}%
\task{%
    Установка для наблюдения интерференции состоит
    из двух когерентных источников света и экрана.
    Расстояние между источниками $l = 2{,}4\,\text{мм}$,
    а от каждого источника до экрана — $L = 4\,\text{м}$.
    Сделайте рисунок и укажите положение нулевого максимума освещенности,
    а также определите расстояние между четвёртым минимумом и нулевым максимумом.
    Длина волны падающего света составляет $\lambda = 450\,\text{нм}$.
}
\answer{%
    \begin{align*}
    l_1^2 &= L^2 + \sqr{x - \frac \ell 2} \\
    l_2^2 &= L^2 + \sqr{x + \frac \ell 2} \\
    l_2^2 - l_1^2 &= 2x\ell \implies (l_2 - l_1)(l_2 + l_1) = 2x\ell \implies n\lambda \cdot 2L \approx 2x_n\ell \implies x_n = \frac{\lambda L}{\ell} n, n\in \mathbb{N} \\
    x &= \frac{\lambda L}{\ell} \cdot \frac72 = \frac{450\,\text{нм} \cdot 4\,\text{м}}{2{,}4\,\text{мм}} \cdot \frac72 \approx 2{,}6\,\text{мм}
    \end{align*}
}
\solutionspace{180pt}

\tasknumber{3}%
\task{%
    На стеклянную пластинку ($\hat n = 1{,}6$) нанесена прозрачная пленка ($n = 1{,}8$).
    На плёнку нормально к поверхности падает монохроматический свет с длиной волны $420\,\text{нм}$.
    Какова должна быть минимальная толщина пленки, чтобы в результате интерференции отражённый свет имел наименьшую интенсивность?
}
\answer{%
    $2 \cdot h \cdot n = 1 \lambda \implies h \approx 117\,\text{нм}$
}

\variantsplitter

\addpersonalvariant{Илья Гримберг}

\tasknumber{1}%
\task{%
    Длина волны света в~вакууме $\lambda = 600\,\text{нм}$.
    Какова частота этой световой волны?
    Какова длина этой волны в среде с показателем преломления $n = 1{,}5$?
    Может ли человек увидеть такую волну света, и если да, то какой именно цвет соответствует этим волнам в вакууме и в этой среде?
}
\answer{%
    \begin{align*}
    \nu &= \frac 1T = \frac 1{\lambda/c} = \frac c\lambda = \frac{3 \cdot 10^{8}\,\frac{\text{м}}{\text{с}}}{600\,\text{нм}} \approx 500 \cdot 10^{12}\,\text{Гц}, \\
    \nu' &= \nu \cbr{\text{или } T' = T} \implies \lambda' = v'T' = \frac vn T = \frac{ vt }n = \frac \lambda n = \frac{600\,\text{нм}}{1{,}5} \approx 400\,\text{нм}.
    \\
    &\text{380 нм---фиол---440---син---485---гол---500---зел---565---жёл---590---оранж---625---крас---780 нм}, \text{увидит}
    \end{align*}
}
\solutionspace{60pt}

\tasknumber{2}%
\task{%
    Установка для наблюдения интерференции состоит
    из двух когерентных источников света и экрана.
    Расстояние между источниками $l = 2{,}4\,\text{мм}$,
    а от каждого источника до экрана — $L = 4\,\text{м}$.
    Сделайте рисунок и укажите положение нулевого максимума освещенности,
    а также определите расстояние между вторым минимумом и нулевым максимумом.
    Длина волны падающего света составляет $\lambda = 450\,\text{нм}$.
}
\answer{%
    \begin{align*}
    l_1^2 &= L^2 + \sqr{x - \frac \ell 2} \\
    l_2^2 &= L^2 + \sqr{x + \frac \ell 2} \\
    l_2^2 - l_1^2 &= 2x\ell \implies (l_2 - l_1)(l_2 + l_1) = 2x\ell \implies n\lambda \cdot 2L \approx 2x_n\ell \implies x_n = \frac{\lambda L}{\ell} n, n\in \mathbb{N} \\
    x &= \frac{\lambda L}{\ell} \cdot \frac32 = \frac{450\,\text{нм} \cdot 4\,\text{м}}{2{,}4\,\text{мм}} \cdot \frac32 \approx 1{,}13\,\text{мм}
    \end{align*}
}
\solutionspace{180pt}

\tasknumber{3}%
\task{%
    На стеклянную пластинку ($\hat n = 1{,}5$) нанесена прозрачная пленка ($n = 1{,}8$).
    На плёнку нормально к поверхности падает монохроматический свет с длиной волны $540\,\text{нм}$.
    Какова должна быть минимальная толщина пленки, чтобы в результате интерференции отражённый свет имел наименьшую интенсивность?
}
\answer{%
    $2 \cdot h \cdot n = 1 \lambda \implies h \approx 150\,\text{нм}$
}

\variantsplitter

\addpersonalvariant{Иван Гурьянов}

\tasknumber{1}%
\task{%
    Длина волны света в~вакууме $\lambda = 500\,\text{нм}$.
    Какова частота этой световой волны?
    Какова длина этой волны в среде с показателем преломления $n = 1{,}5$?
    Может ли человек увидеть такую волну света, и если да, то какой именно цвет соответствует этим волнам в вакууме и в этой среде?
}
\answer{%
    \begin{align*}
    \nu &= \frac 1T = \frac 1{\lambda/c} = \frac c\lambda = \frac{3 \cdot 10^{8}\,\frac{\text{м}}{\text{с}}}{500\,\text{нм}} \approx 600 \cdot 10^{12}\,\text{Гц}, \\
    \nu' &= \nu \cbr{\text{или } T' = T} \implies \lambda' = v'T' = \frac vn T = \frac{ vt }n = \frac \lambda n = \frac{500\,\text{нм}}{1{,}5} \approx 330\,\text{нм}.
    \\
    &\text{380 нм---фиол---440---син---485---гол---500---зел---565---жёл---590---оранж---625---крас---780 нм}, \text{увидит}
    \end{align*}
}
\solutionspace{60pt}

\tasknumber{2}%
\task{%
    Установка для наблюдения интерференции состоит
    из двух когерентных источников света и экрана.
    Расстояние между источниками $l = 1{,}5\,\text{мм}$,
    а от каждого источника до экрана — $L = 3\,\text{м}$.
    Сделайте рисунок и укажите положение нулевого максимума освещенности,
    а также определите расстояние между вторым максимумом и нулевым максимумом.
    Длина волны падающего света составляет $\lambda = 450\,\text{нм}$.
}
\answer{%
    \begin{align*}
    l_1^2 &= L^2 + \sqr{x - \frac \ell 2} \\
    l_2^2 &= L^2 + \sqr{x + \frac \ell 2} \\
    l_2^2 - l_1^2 &= 2x\ell \implies (l_2 - l_1)(l_2 + l_1) = 2x\ell \implies n\lambda \cdot 2L \approx 2x_n\ell \implies x_n = \frac{\lambda L}{\ell} n, n\in \mathbb{N} \\
    x &= \frac{\lambda L}{\ell} \cdot 2 = \frac{450\,\text{нм} \cdot 3\,\text{м}}{1{,}5\,\text{мм}} \cdot 2 \approx 1{,}80\,\text{мм}
    \end{align*}
}
\solutionspace{180pt}

\tasknumber{3}%
\task{%
    На стеклянную пластинку ($\hat n = 1{,}6$) нанесена прозрачная пленка ($n = 1{,}3$).
    На плёнку нормально к поверхности падает монохроматический свет с длиной волны $420\,\text{нм}$.
    Какова должна быть минимальная толщина пленки, чтобы в результате интерференции отражённый свет имел наибольшую интенсивность?
}
\answer{%
    $2 \cdot h \cdot n = 1 \lambda \implies h \approx 162\,\text{нм}$
}

\variantsplitter

\addpersonalvariant{Артём Денежкин}

\tasknumber{1}%
\task{%
    Длина волны света в~вакууме $\lambda = 500\,\text{нм}$.
    Какова частота этой световой волны?
    Какова длина этой волны в среде с показателем преломления $n = 1{,}4$?
    Может ли человек увидеть такую волну света, и если да, то какой именно цвет соответствует этим волнам в вакууме и в этой среде?
}
\answer{%
    \begin{align*}
    \nu &= \frac 1T = \frac 1{\lambda/c} = \frac c\lambda = \frac{3 \cdot 10^{8}\,\frac{\text{м}}{\text{с}}}{500\,\text{нм}} \approx 600 \cdot 10^{12}\,\text{Гц}, \\
    \nu' &= \nu \cbr{\text{или } T' = T} \implies \lambda' = v'T' = \frac vn T = \frac{ vt }n = \frac \lambda n = \frac{500\,\text{нм}}{1{,}4} \approx 360\,\text{нм}.
    \\
    &\text{380 нм---фиол---440---син---485---гол---500---зел---565---жёл---590---оранж---625---крас---780 нм}, \text{увидит}
    \end{align*}
}
\solutionspace{60pt}

\tasknumber{2}%
\task{%
    Установка для наблюдения интерференции состоит
    из двух когерентных источников света и экрана.
    Расстояние между источниками $l = 0{,}8\,\text{мм}$,
    а от каждого источника до экрана — $L = 4\,\text{м}$.
    Сделайте рисунок и укажите положение нулевого максимума освещенности,
    а также определите расстояние между четвёртым минимумом и нулевым максимумом.
    Длина волны падающего света составляет $\lambda = 500\,\text{нм}$.
}
\answer{%
    \begin{align*}
    l_1^2 &= L^2 + \sqr{x - \frac \ell 2} \\
    l_2^2 &= L^2 + \sqr{x + \frac \ell 2} \\
    l_2^2 - l_1^2 &= 2x\ell \implies (l_2 - l_1)(l_2 + l_1) = 2x\ell \implies n\lambda \cdot 2L \approx 2x_n\ell \implies x_n = \frac{\lambda L}{\ell} n, n\in \mathbb{N} \\
    x &= \frac{\lambda L}{\ell} \cdot \frac72 = \frac{500\,\text{нм} \cdot 4\,\text{м}}{0{,}8\,\text{мм}} \cdot \frac72 \approx 8{,}8\,\text{мм}
    \end{align*}
}
\solutionspace{180pt}

\tasknumber{3}%
\task{%
    На стеклянную пластинку ($\hat n = 1{,}6$) нанесена прозрачная пленка ($n = 1{,}4$).
    На плёнку нормально к поверхности падает монохроматический свет с длиной волны $640\,\text{нм}$.
    Какова должна быть минимальная толщина пленки, чтобы в результате интерференции отражённый свет имел наименьшую интенсивность?
}
\answer{%
    $2 \cdot h \cdot n = \frac12 \lambda \implies h \approx 114\,\text{нм}$
}

\variantsplitter

\addpersonalvariant{Виктор Жилин}

\tasknumber{1}%
\task{%
    Длина волны света в~вакууме $\lambda = 400\,\text{нм}$.
    Какова частота этой световой волны?
    Какова длина этой волны в среде с показателем преломления $n = 1{,}3$?
    Может ли человек увидеть такую волну света, и если да, то какой именно цвет соответствует этим волнам в вакууме и в этой среде?
}
\answer{%
    \begin{align*}
    \nu &= \frac 1T = \frac 1{\lambda/c} = \frac c\lambda = \frac{3 \cdot 10^{8}\,\frac{\text{м}}{\text{с}}}{400\,\text{нм}} \approx 750 \cdot 10^{12}\,\text{Гц}, \\
    \nu' &= \nu \cbr{\text{или } T' = T} \implies \lambda' = v'T' = \frac vn T = \frac{ vt }n = \frac \lambda n = \frac{400\,\text{нм}}{1{,}3} \approx 310\,\text{нм}.
    \\
    &\text{380 нм---фиол---440---син---485---гол---500---зел---565---жёл---590---оранж---625---крас---780 нм}, \text{увидит}
    \end{align*}
}
\solutionspace{60pt}

\tasknumber{2}%
\task{%
    Установка для наблюдения интерференции состоит
    из двух когерентных источников света и экрана.
    Расстояние между источниками $l = 1{,}5\,\text{мм}$,
    а от каждого источника до экрана — $L = 4\,\text{м}$.
    Сделайте рисунок и укажите положение нулевого максимума освещенности,
    а также определите расстояние между третьим максимумом и нулевым максимумом.
    Длина волны падающего света составляет $\lambda = 550\,\text{нм}$.
}
\answer{%
    \begin{align*}
    l_1^2 &= L^2 + \sqr{x - \frac \ell 2} \\
    l_2^2 &= L^2 + \sqr{x + \frac \ell 2} \\
    l_2^2 - l_1^2 &= 2x\ell \implies (l_2 - l_1)(l_2 + l_1) = 2x\ell \implies n\lambda \cdot 2L \approx 2x_n\ell \implies x_n = \frac{\lambda L}{\ell} n, n\in \mathbb{N} \\
    x &= \frac{\lambda L}{\ell} \cdot 3 = \frac{550\,\text{нм} \cdot 4\,\text{м}}{1{,}5\,\text{мм}} \cdot 3 \approx 4{,}4\,\text{мм}
    \end{align*}
}
\solutionspace{180pt}

\tasknumber{3}%
\task{%
    На стеклянную пластинку ($\hat n = 1{,}5$) нанесена прозрачная пленка ($n = 1{,}7$).
    На плёнку нормально к поверхности падает монохроматический свет с длиной волны $540\,\text{нм}$.
    Какова должна быть минимальная толщина пленки, чтобы в результате интерференции отражённый свет имел наибольшую интенсивность?
}
\answer{%
    $2 \cdot h \cdot n = \frac12 \lambda \implies h \approx 79\,\text{нм}$
}

\variantsplitter

\addpersonalvariant{Дмитрий Иванов}

\tasknumber{1}%
\task{%
    Длина волны света в~вакууме $\lambda = 500\,\text{нм}$.
    Какова частота этой световой волны?
    Какова длина этой волны в среде с показателем преломления $n = 1{,}7$?
    Может ли человек увидеть такую волну света, и если да, то какой именно цвет соответствует этим волнам в вакууме и в этой среде?
}
\answer{%
    \begin{align*}
    \nu &= \frac 1T = \frac 1{\lambda/c} = \frac c\lambda = \frac{3 \cdot 10^{8}\,\frac{\text{м}}{\text{с}}}{500\,\text{нм}} \approx 600 \cdot 10^{12}\,\text{Гц}, \\
    \nu' &= \nu \cbr{\text{или } T' = T} \implies \lambda' = v'T' = \frac vn T = \frac{ vt }n = \frac \lambda n = \frac{500\,\text{нм}}{1{,}7} \approx 290\,\text{нм}.
    \\
    &\text{380 нм---фиол---440---син---485---гол---500---зел---565---жёл---590---оранж---625---крас---780 нм}, \text{увидит}
    \end{align*}
}
\solutionspace{60pt}

\tasknumber{2}%
\task{%
    Установка для наблюдения интерференции состоит
    из двух когерентных источников света и экрана.
    Расстояние между источниками $l = 1{,}2\,\text{мм}$,
    а от каждого источника до экрана — $L = 4\,\text{м}$.
    Сделайте рисунок и укажите положение нулевого максимума освещенности,
    а также определите расстояние между четвёртым максимумом и нулевым максимумом.
    Длина волны падающего света составляет $\lambda = 500\,\text{нм}$.
}
\answer{%
    \begin{align*}
    l_1^2 &= L^2 + \sqr{x - \frac \ell 2} \\
    l_2^2 &= L^2 + \sqr{x + \frac \ell 2} \\
    l_2^2 - l_1^2 &= 2x\ell \implies (l_2 - l_1)(l_2 + l_1) = 2x\ell \implies n\lambda \cdot 2L \approx 2x_n\ell \implies x_n = \frac{\lambda L}{\ell} n, n\in \mathbb{N} \\
    x &= \frac{\lambda L}{\ell} \cdot 4 = \frac{500\,\text{нм} \cdot 4\,\text{м}}{1{,}2\,\text{мм}} \cdot 4 \approx 6{,}7\,\text{мм}
    \end{align*}
}
\solutionspace{180pt}

\tasknumber{3}%
\task{%
    На стеклянную пластинку ($\hat n = 1{,}6$) нанесена прозрачная пленка ($n = 1{,}7$).
    На плёнку нормально к поверхности падает монохроматический свет с длиной волны $540\,\text{нм}$.
    Какова должна быть минимальная толщина пленки, чтобы в результате интерференции отражённый свет имел наименьшую интенсивность?
}
\answer{%
    $2 \cdot h \cdot n = 1 \lambda \implies h \approx 159\,\text{нм}$
}

\variantsplitter

\addpersonalvariant{Олег Климов}

\tasknumber{1}%
\task{%
    Длина волны света в~вакууме $\lambda = 500\,\text{нм}$.
    Какова частота этой световой волны?
    Какова длина этой волны в среде с показателем преломления $n = 1{,}3$?
    Может ли человек увидеть такую волну света, и если да, то какой именно цвет соответствует этим волнам в вакууме и в этой среде?
}
\answer{%
    \begin{align*}
    \nu &= \frac 1T = \frac 1{\lambda/c} = \frac c\lambda = \frac{3 \cdot 10^{8}\,\frac{\text{м}}{\text{с}}}{500\,\text{нм}} \approx 600 \cdot 10^{12}\,\text{Гц}, \\
    \nu' &= \nu \cbr{\text{или } T' = T} \implies \lambda' = v'T' = \frac vn T = \frac{ vt }n = \frac \lambda n = \frac{500\,\text{нм}}{1{,}3} \approx 380\,\text{нм}.
    \\
    &\text{380 нм---фиол---440---син---485---гол---500---зел---565---жёл---590---оранж---625---крас---780 нм}, \text{увидит}
    \end{align*}
}
\solutionspace{60pt}

\tasknumber{2}%
\task{%
    Установка для наблюдения интерференции состоит
    из двух когерентных источников света и экрана.
    Расстояние между источниками $l = 1{,}2\,\text{мм}$,
    а от каждого источника до экрана — $L = 2\,\text{м}$.
    Сделайте рисунок и укажите положение нулевого максимума освещенности,
    а также определите расстояние между третьим максимумом и нулевым максимумом.
    Длина волны падающего света составляет $\lambda = 450\,\text{нм}$.
}
\answer{%
    \begin{align*}
    l_1^2 &= L^2 + \sqr{x - \frac \ell 2} \\
    l_2^2 &= L^2 + \sqr{x + \frac \ell 2} \\
    l_2^2 - l_1^2 &= 2x\ell \implies (l_2 - l_1)(l_2 + l_1) = 2x\ell \implies n\lambda \cdot 2L \approx 2x_n\ell \implies x_n = \frac{\lambda L}{\ell} n, n\in \mathbb{N} \\
    x &= \frac{\lambda L}{\ell} \cdot 3 = \frac{450\,\text{нм} \cdot 2\,\text{м}}{1{,}2\,\text{мм}} \cdot 3 \approx 2{,}3\,\text{мм}
    \end{align*}
}
\solutionspace{180pt}

\tasknumber{3}%
\task{%
    На стеклянную пластинку ($\hat n = 1{,}5$) нанесена прозрачная пленка ($n = 1{,}3$).
    На плёнку нормально к поверхности падает монохроматический свет с длиной волны $640\,\text{нм}$.
    Какова должна быть минимальная толщина пленки, чтобы в результате интерференции отражённый свет имел наименьшую интенсивность?
}
\answer{%
    $2 \cdot h \cdot n = \frac12 \lambda \implies h \approx 123\,\text{нм}$
}

\variantsplitter

\addpersonalvariant{Анна Ковалева}

\tasknumber{1}%
\task{%
    Длина волны света в~вакууме $\lambda = 600\,\text{нм}$.
    Какова частота этой световой волны?
    Какова длина этой волны в среде с показателем преломления $n = 1{,}5$?
    Может ли человек увидеть такую волну света, и если да, то какой именно цвет соответствует этим волнам в вакууме и в этой среде?
}
\answer{%
    \begin{align*}
    \nu &= \frac 1T = \frac 1{\lambda/c} = \frac c\lambda = \frac{3 \cdot 10^{8}\,\frac{\text{м}}{\text{с}}}{600\,\text{нм}} \approx 500 \cdot 10^{12}\,\text{Гц}, \\
    \nu' &= \nu \cbr{\text{или } T' = T} \implies \lambda' = v'T' = \frac vn T = \frac{ vt }n = \frac \lambda n = \frac{600\,\text{нм}}{1{,}5} \approx 400\,\text{нм}.
    \\
    &\text{380 нм---фиол---440---син---485---гол---500---зел---565---жёл---590---оранж---625---крас---780 нм}, \text{увидит}
    \end{align*}
}
\solutionspace{60pt}

\tasknumber{2}%
\task{%
    Установка для наблюдения интерференции состоит
    из двух когерентных источников света и экрана.
    Расстояние между источниками $l = 0{,}8\,\text{мм}$,
    а от каждого источника до экрана — $L = 2\,\text{м}$.
    Сделайте рисунок и укажите положение нулевого максимума освещенности,
    а также определите расстояние между четвёртым максимумом и нулевым максимумом.
    Длина волны падающего света составляет $\lambda = 400\,\text{нм}$.
}
\answer{%
    \begin{align*}
    l_1^2 &= L^2 + \sqr{x - \frac \ell 2} \\
    l_2^2 &= L^2 + \sqr{x + \frac \ell 2} \\
    l_2^2 - l_1^2 &= 2x\ell \implies (l_2 - l_1)(l_2 + l_1) = 2x\ell \implies n\lambda \cdot 2L \approx 2x_n\ell \implies x_n = \frac{\lambda L}{\ell} n, n\in \mathbb{N} \\
    x &= \frac{\lambda L}{\ell} \cdot 4 = \frac{400\,\text{нм} \cdot 2\,\text{м}}{0{,}8\,\text{мм}} \cdot 4 \approx 4\,\text{мм}
    \end{align*}
}
\solutionspace{180pt}

\tasknumber{3}%
\task{%
    На стеклянную пластинку ($\hat n = 1{,}5$) нанесена прозрачная пленка ($n = 1{,}4$).
    На плёнку нормально к поверхности падает монохроматический свет с длиной волны $480\,\text{нм}$.
    Какова должна быть минимальная толщина пленки, чтобы в результате интерференции отражённый свет имел наибольшую интенсивность?
}
\answer{%
    $2 \cdot h \cdot n = 1 \lambda \implies h \approx 171\,\text{нм}$
}

\variantsplitter

\addpersonalvariant{Глеб Ковылин}

\tasknumber{1}%
\task{%
    Длина волны света в~вакууме $\lambda = 700\,\text{нм}$.
    Какова частота этой световой волны?
    Какова длина этой волны в среде с показателем преломления $n = 1{,}7$?
    Может ли человек увидеть такую волну света, и если да, то какой именно цвет соответствует этим волнам в вакууме и в этой среде?
}
\answer{%
    \begin{align*}
    \nu &= \frac 1T = \frac 1{\lambda/c} = \frac c\lambda = \frac{3 \cdot 10^{8}\,\frac{\text{м}}{\text{с}}}{700\,\text{нм}} \approx 429 \cdot 10^{12}\,\text{Гц}, \\
    \nu' &= \nu \cbr{\text{или } T' = T} \implies \lambda' = v'T' = \frac vn T = \frac{ vt }n = \frac \lambda n = \frac{700\,\text{нм}}{1{,}7} \approx 410\,\text{нм}.
    \\
    &\text{380 нм---фиол---440---син---485---гол---500---зел---565---жёл---590---оранж---625---крас---780 нм}, \text{увидит}
    \end{align*}
}
\solutionspace{60pt}

\tasknumber{2}%
\task{%
    Установка для наблюдения интерференции состоит
    из двух когерентных источников света и экрана.
    Расстояние между источниками $l = 0{,}8\,\text{мм}$,
    а от каждого источника до экрана — $L = 4\,\text{м}$.
    Сделайте рисунок и укажите положение нулевого максимума освещенности,
    а также определите расстояние между четвёртым максимумом и нулевым максимумом.
    Длина волны падающего света составляет $\lambda = 550\,\text{нм}$.
}
\answer{%
    \begin{align*}
    l_1^2 &= L^2 + \sqr{x - \frac \ell 2} \\
    l_2^2 &= L^2 + \sqr{x + \frac \ell 2} \\
    l_2^2 - l_1^2 &= 2x\ell \implies (l_2 - l_1)(l_2 + l_1) = 2x\ell \implies n\lambda \cdot 2L \approx 2x_n\ell \implies x_n = \frac{\lambda L}{\ell} n, n\in \mathbb{N} \\
    x &= \frac{\lambda L}{\ell} \cdot 4 = \frac{550\,\text{нм} \cdot 4\,\text{м}}{0{,}8\,\text{мм}} \cdot 4 \approx 11\,\text{мм}
    \end{align*}
}
\solutionspace{180pt}

\tasknumber{3}%
\task{%
    На стеклянную пластинку ($\hat n = 1{,}5$) нанесена прозрачная пленка ($n = 1{,}8$).
    На плёнку нормально к поверхности падает монохроматический свет с длиной волны $640\,\text{нм}$.
    Какова должна быть минимальная толщина пленки, чтобы в результате интерференции отражённый свет имел наименьшую интенсивность?
}
\answer{%
    $2 \cdot h \cdot n = 1 \lambda \implies h \approx 178\,\text{нм}$
}

\variantsplitter

\addpersonalvariant{Даниил Космынин}

\tasknumber{1}%
\task{%
    Длина волны света в~вакууме $\lambda = 400\,\text{нм}$.
    Какова частота этой световой волны?
    Какова длина этой волны в среде с показателем преломления $n = 1{,}3$?
    Может ли человек увидеть такую волну света, и если да, то какой именно цвет соответствует этим волнам в вакууме и в этой среде?
}
\answer{%
    \begin{align*}
    \nu &= \frac 1T = \frac 1{\lambda/c} = \frac c\lambda = \frac{3 \cdot 10^{8}\,\frac{\text{м}}{\text{с}}}{400\,\text{нм}} \approx 750 \cdot 10^{12}\,\text{Гц}, \\
    \nu' &= \nu \cbr{\text{или } T' = T} \implies \lambda' = v'T' = \frac vn T = \frac{ vt }n = \frac \lambda n = \frac{400\,\text{нм}}{1{,}3} \approx 310\,\text{нм}.
    \\
    &\text{380 нм---фиол---440---син---485---гол---500---зел---565---жёл---590---оранж---625---крас---780 нм}, \text{увидит}
    \end{align*}
}
\solutionspace{60pt}

\tasknumber{2}%
\task{%
    Установка для наблюдения интерференции состоит
    из двух когерентных источников света и экрана.
    Расстояние между источниками $l = 1{,}2\,\text{мм}$,
    а от каждого источника до экрана — $L = 3\,\text{м}$.
    Сделайте рисунок и укажите положение нулевого максимума освещенности,
    а также определите расстояние между третьим минимумом и нулевым максимумом.
    Длина волны падающего света составляет $\lambda = 500\,\text{нм}$.
}
\answer{%
    \begin{align*}
    l_1^2 &= L^2 + \sqr{x - \frac \ell 2} \\
    l_2^2 &= L^2 + \sqr{x + \frac \ell 2} \\
    l_2^2 - l_1^2 &= 2x\ell \implies (l_2 - l_1)(l_2 + l_1) = 2x\ell \implies n\lambda \cdot 2L \approx 2x_n\ell \implies x_n = \frac{\lambda L}{\ell} n, n\in \mathbb{N} \\
    x &= \frac{\lambda L}{\ell} \cdot \frac52 = \frac{500\,\text{нм} \cdot 3\,\text{м}}{1{,}2\,\text{мм}} \cdot \frac52 \approx 3{,}1\,\text{мм}
    \end{align*}
}
\solutionspace{180pt}

\tasknumber{3}%
\task{%
    На стеклянную пластинку ($\hat n = 1{,}6$) нанесена прозрачная пленка ($n = 1{,}4$).
    На плёнку нормально к поверхности падает монохроматический свет с длиной волны $640\,\text{нм}$.
    Какова должна быть минимальная толщина пленки, чтобы в результате интерференции отражённый свет имел наибольшую интенсивность?
}
\answer{%
    $2 \cdot h \cdot n = 1 \lambda \implies h \approx 230\,\text{нм}$
}

\variantsplitter

\addpersonalvariant{Алина Леоничева}

\tasknumber{1}%
\task{%
    Длина волны света в~вакууме $\lambda = 600\,\text{нм}$.
    Какова частота этой световой волны?
    Какова длина этой волны в среде с показателем преломления $n = 1{,}3$?
    Может ли человек увидеть такую волну света, и если да, то какой именно цвет соответствует этим волнам в вакууме и в этой среде?
}
\answer{%
    \begin{align*}
    \nu &= \frac 1T = \frac 1{\lambda/c} = \frac c\lambda = \frac{3 \cdot 10^{8}\,\frac{\text{м}}{\text{с}}}{600\,\text{нм}} \approx 500 \cdot 10^{12}\,\text{Гц}, \\
    \nu' &= \nu \cbr{\text{или } T' = T} \implies \lambda' = v'T' = \frac vn T = \frac{ vt }n = \frac \lambda n = \frac{600\,\text{нм}}{1{,}3} \approx 460\,\text{нм}.
    \\
    &\text{380 нм---фиол---440---син---485---гол---500---зел---565---жёл---590---оранж---625---крас---780 нм}, \text{увидит}
    \end{align*}
}
\solutionspace{60pt}

\tasknumber{2}%
\task{%
    Установка для наблюдения интерференции состоит
    из двух когерентных источников света и экрана.
    Расстояние между источниками $l = 1{,}5\,\text{мм}$,
    а от каждого источника до экрана — $L = 3\,\text{м}$.
    Сделайте рисунок и укажите положение нулевого максимума освещенности,
    а также определите расстояние между вторым максимумом и нулевым максимумом.
    Длина волны падающего света составляет $\lambda = 500\,\text{нм}$.
}
\answer{%
    \begin{align*}
    l_1^2 &= L^2 + \sqr{x - \frac \ell 2} \\
    l_2^2 &= L^2 + \sqr{x + \frac \ell 2} \\
    l_2^2 - l_1^2 &= 2x\ell \implies (l_2 - l_1)(l_2 + l_1) = 2x\ell \implies n\lambda \cdot 2L \approx 2x_n\ell \implies x_n = \frac{\lambda L}{\ell} n, n\in \mathbb{N} \\
    x &= \frac{\lambda L}{\ell} \cdot 2 = \frac{500\,\text{нм} \cdot 3\,\text{м}}{1{,}5\,\text{мм}} \cdot 2 \approx 2\,\text{мм}
    \end{align*}
}
\solutionspace{180pt}

\tasknumber{3}%
\task{%
    На стеклянную пластинку ($\hat n = 1{,}5$) нанесена прозрачная пленка ($n = 1{,}4$).
    На плёнку нормально к поверхности падает монохроматический свет с длиной волны $420\,\text{нм}$.
    Какова должна быть минимальная толщина пленки, чтобы в результате интерференции отражённый свет имел наибольшую интенсивность?
}
\answer{%
    $2 \cdot h \cdot n = 1 \lambda \implies h \approx 150\,\text{нм}$
}

\variantsplitter

\addpersonalvariant{Ирина Лин}

\tasknumber{1}%
\task{%
    Длина волны света в~вакууме $\lambda = 700\,\text{нм}$.
    Какова частота этой световой волны?
    Какова длина этой волны в среде с показателем преломления $n = 1{,}7$?
    Может ли человек увидеть такую волну света, и если да, то какой именно цвет соответствует этим волнам в вакууме и в этой среде?
}
\answer{%
    \begin{align*}
    \nu &= \frac 1T = \frac 1{\lambda/c} = \frac c\lambda = \frac{3 \cdot 10^{8}\,\frac{\text{м}}{\text{с}}}{700\,\text{нм}} \approx 429 \cdot 10^{12}\,\text{Гц}, \\
    \nu' &= \nu \cbr{\text{или } T' = T} \implies \lambda' = v'T' = \frac vn T = \frac{ vt }n = \frac \lambda n = \frac{700\,\text{нм}}{1{,}7} \approx 410\,\text{нм}.
    \\
    &\text{380 нм---фиол---440---син---485---гол---500---зел---565---жёл---590---оранж---625---крас---780 нм}, \text{увидит}
    \end{align*}
}
\solutionspace{60pt}

\tasknumber{2}%
\task{%
    Установка для наблюдения интерференции состоит
    из двух когерентных источников света и экрана.
    Расстояние между источниками $l = 2{,}4\,\text{мм}$,
    а от каждого источника до экрана — $L = 4\,\text{м}$.
    Сделайте рисунок и укажите положение нулевого максимума освещенности,
    а также определите расстояние между третьим максимумом и нулевым максимумом.
    Длина волны падающего света составляет $\lambda = 600\,\text{нм}$.
}
\answer{%
    \begin{align*}
    l_1^2 &= L^2 + \sqr{x - \frac \ell 2} \\
    l_2^2 &= L^2 + \sqr{x + \frac \ell 2} \\
    l_2^2 - l_1^2 &= 2x\ell \implies (l_2 - l_1)(l_2 + l_1) = 2x\ell \implies n\lambda \cdot 2L \approx 2x_n\ell \implies x_n = \frac{\lambda L}{\ell} n, n\in \mathbb{N} \\
    x &= \frac{\lambda L}{\ell} \cdot 3 = \frac{600\,\text{нм} \cdot 4\,\text{м}}{2{,}4\,\text{мм}} \cdot 3 \approx 3\,\text{мм}
    \end{align*}
}
\solutionspace{180pt}

\tasknumber{3}%
\task{%
    На стеклянную пластинку ($\hat n = 1{,}5$) нанесена прозрачная пленка ($n = 1{,}8$).
    На плёнку нормально к поверхности падает монохроматический свет с длиной волны $540\,\text{нм}$.
    Какова должна быть минимальная толщина пленки, чтобы в результате интерференции отражённый свет имел наибольшую интенсивность?
}
\answer{%
    $2 \cdot h \cdot n = \frac12 \lambda \implies h \approx 75\,\text{нм}$
}

\variantsplitter

\addpersonalvariant{Ислам Мунаев}

\tasknumber{1}%
\task{%
    Длина волны света в~вакууме $\lambda = 700\,\text{нм}$.
    Какова частота этой световой волны?
    Какова длина этой волны в среде с показателем преломления $n = 1{,}4$?
    Может ли человек увидеть такую волну света, и если да, то какой именно цвет соответствует этим волнам в вакууме и в этой среде?
}
\answer{%
    \begin{align*}
    \nu &= \frac 1T = \frac 1{\lambda/c} = \frac c\lambda = \frac{3 \cdot 10^{8}\,\frac{\text{м}}{\text{с}}}{700\,\text{нм}} \approx 429 \cdot 10^{12}\,\text{Гц}, \\
    \nu' &= \nu \cbr{\text{или } T' = T} \implies \lambda' = v'T' = \frac vn T = \frac{ vt }n = \frac \lambda n = \frac{700\,\text{нм}}{1{,}4} \approx 500\,\text{нм}.
    \\
    &\text{380 нм---фиол---440---син---485---гол---500---зел---565---жёл---590---оранж---625---крас---780 нм}, \text{увидит}
    \end{align*}
}
\solutionspace{60pt}

\tasknumber{2}%
\task{%
    Установка для наблюдения интерференции состоит
    из двух когерентных источников света и экрана.
    Расстояние между источниками $l = 0{,}8\,\text{мм}$,
    а от каждого источника до экрана — $L = 3\,\text{м}$.
    Сделайте рисунок и укажите положение нулевого максимума освещенности,
    а также определите расстояние между третьим максимумом и нулевым максимумом.
    Длина волны падающего света составляет $\lambda = 600\,\text{нм}$.
}
\answer{%
    \begin{align*}
    l_1^2 &= L^2 + \sqr{x - \frac \ell 2} \\
    l_2^2 &= L^2 + \sqr{x + \frac \ell 2} \\
    l_2^2 - l_1^2 &= 2x\ell \implies (l_2 - l_1)(l_2 + l_1) = 2x\ell \implies n\lambda \cdot 2L \approx 2x_n\ell \implies x_n = \frac{\lambda L}{\ell} n, n\in \mathbb{N} \\
    x &= \frac{\lambda L}{\ell} \cdot 3 = \frac{600\,\text{нм} \cdot 3\,\text{м}}{0{,}8\,\text{мм}} \cdot 3 \approx 6{,}8\,\text{мм}
    \end{align*}
}
\solutionspace{180pt}

\tasknumber{3}%
\task{%
    На стеклянную пластинку ($\hat n = 1{,}6$) нанесена прозрачная пленка ($n = 1{,}7$).
    На плёнку нормально к поверхности падает монохроматический свет с длиной волны $480\,\text{нм}$.
    Какова должна быть минимальная толщина пленки, чтобы в результате интерференции отражённый свет имел наибольшую интенсивность?
}
\answer{%
    $2 \cdot h \cdot n = \frac12 \lambda \implies h \approx 71\,\text{нм}$
}

\variantsplitter

\addpersonalvariant{Александр Наумов}

\tasknumber{1}%
\task{%
    Длина волны света в~вакууме $\lambda = 700\,\text{нм}$.
    Какова частота этой световой волны?
    Какова длина этой волны в среде с показателем преломления $n = 1{,}3$?
    Может ли человек увидеть такую волну света, и если да, то какой именно цвет соответствует этим волнам в вакууме и в этой среде?
}
\answer{%
    \begin{align*}
    \nu &= \frac 1T = \frac 1{\lambda/c} = \frac c\lambda = \frac{3 \cdot 10^{8}\,\frac{\text{м}}{\text{с}}}{700\,\text{нм}} \approx 429 \cdot 10^{12}\,\text{Гц}, \\
    \nu' &= \nu \cbr{\text{или } T' = T} \implies \lambda' = v'T' = \frac vn T = \frac{ vt }n = \frac \lambda n = \frac{700\,\text{нм}}{1{,}3} \approx 540\,\text{нм}.
    \\
    &\text{380 нм---фиол---440---син---485---гол---500---зел---565---жёл---590---оранж---625---крас---780 нм}, \text{увидит}
    \end{align*}
}
\solutionspace{60pt}

\tasknumber{2}%
\task{%
    Установка для наблюдения интерференции состоит
    из двух когерентных источников света и экрана.
    Расстояние между источниками $l = 2{,}4\,\text{мм}$,
    а от каждого источника до экрана — $L = 3\,\text{м}$.
    Сделайте рисунок и укажите положение нулевого максимума освещенности,
    а также определите расстояние между третьим максимумом и нулевым максимумом.
    Длина волны падающего света составляет $\lambda = 600\,\text{нм}$.
}
\answer{%
    \begin{align*}
    l_1^2 &= L^2 + \sqr{x - \frac \ell 2} \\
    l_2^2 &= L^2 + \sqr{x + \frac \ell 2} \\
    l_2^2 - l_1^2 &= 2x\ell \implies (l_2 - l_1)(l_2 + l_1) = 2x\ell \implies n\lambda \cdot 2L \approx 2x_n\ell \implies x_n = \frac{\lambda L}{\ell} n, n\in \mathbb{N} \\
    x &= \frac{\lambda L}{\ell} \cdot 3 = \frac{600\,\text{нм} \cdot 3\,\text{м}}{2{,}4\,\text{мм}} \cdot 3 \approx 2{,}3\,\text{мм}
    \end{align*}
}
\solutionspace{180pt}

\tasknumber{3}%
\task{%
    На стеклянную пластинку ($\hat n = 1{,}5$) нанесена прозрачная пленка ($n = 1{,}4$).
    На плёнку нормально к поверхности падает монохроматический свет с длиной волны $540\,\text{нм}$.
    Какова должна быть минимальная толщина пленки, чтобы в результате интерференции отражённый свет имел наименьшую интенсивность?
}
\answer{%
    $2 \cdot h \cdot n = \frac12 \lambda \implies h \approx 96\,\text{нм}$
}

\variantsplitter

\addpersonalvariant{Георгий Новиков}

\tasknumber{1}%
\task{%
    Длина волны света в~вакууме $\lambda = 700\,\text{нм}$.
    Какова частота этой световой волны?
    Какова длина этой волны в среде с показателем преломления $n = 1{,}7$?
    Может ли человек увидеть такую волну света, и если да, то какой именно цвет соответствует этим волнам в вакууме и в этой среде?
}
\answer{%
    \begin{align*}
    \nu &= \frac 1T = \frac 1{\lambda/c} = \frac c\lambda = \frac{3 \cdot 10^{8}\,\frac{\text{м}}{\text{с}}}{700\,\text{нм}} \approx 429 \cdot 10^{12}\,\text{Гц}, \\
    \nu' &= \nu \cbr{\text{или } T' = T} \implies \lambda' = v'T' = \frac vn T = \frac{ vt }n = \frac \lambda n = \frac{700\,\text{нм}}{1{,}7} \approx 410\,\text{нм}.
    \\
    &\text{380 нм---фиол---440---син---485---гол---500---зел---565---жёл---590---оранж---625---крас---780 нм}, \text{увидит}
    \end{align*}
}
\solutionspace{60pt}

\tasknumber{2}%
\task{%
    Установка для наблюдения интерференции состоит
    из двух когерентных источников света и экрана.
    Расстояние между источниками $l = 1{,}5\,\text{мм}$,
    а от каждого источника до экрана — $L = 4\,\text{м}$.
    Сделайте рисунок и укажите положение нулевого максимума освещенности,
    а также определите расстояние между третьим максимумом и нулевым максимумом.
    Длина волны падающего света составляет $\lambda = 600\,\text{нм}$.
}
\answer{%
    \begin{align*}
    l_1^2 &= L^2 + \sqr{x - \frac \ell 2} \\
    l_2^2 &= L^2 + \sqr{x + \frac \ell 2} \\
    l_2^2 - l_1^2 &= 2x\ell \implies (l_2 - l_1)(l_2 + l_1) = 2x\ell \implies n\lambda \cdot 2L \approx 2x_n\ell \implies x_n = \frac{\lambda L}{\ell} n, n\in \mathbb{N} \\
    x &= \frac{\lambda L}{\ell} \cdot 3 = \frac{600\,\text{нм} \cdot 4\,\text{м}}{1{,}5\,\text{мм}} \cdot 3 \approx 4{,}8\,\text{мм}
    \end{align*}
}
\solutionspace{180pt}

\tasknumber{3}%
\task{%
    На стеклянную пластинку ($\hat n = 1{,}6$) нанесена прозрачная пленка ($n = 1{,}7$).
    На плёнку нормально к поверхности падает монохроматический свет с длиной волны $480\,\text{нм}$.
    Какова должна быть минимальная толщина пленки, чтобы в результате интерференции отражённый свет имел наименьшую интенсивность?
}
\answer{%
    $2 \cdot h \cdot n = 1 \lambda \implies h \approx 141\,\text{нм}$
}

\variantsplitter

\addpersonalvariant{Егор Осипов}

\tasknumber{1}%
\task{%
    Длина волны света в~вакууме $\lambda = 600\,\text{нм}$.
    Какова частота этой световой волны?
    Какова длина этой волны в среде с показателем преломления $n = 1{,}5$?
    Может ли человек увидеть такую волну света, и если да, то какой именно цвет соответствует этим волнам в вакууме и в этой среде?
}
\answer{%
    \begin{align*}
    \nu &= \frac 1T = \frac 1{\lambda/c} = \frac c\lambda = \frac{3 \cdot 10^{8}\,\frac{\text{м}}{\text{с}}}{600\,\text{нм}} \approx 500 \cdot 10^{12}\,\text{Гц}, \\
    \nu' &= \nu \cbr{\text{или } T' = T} \implies \lambda' = v'T' = \frac vn T = \frac{ vt }n = \frac \lambda n = \frac{600\,\text{нм}}{1{,}5} \approx 400\,\text{нм}.
    \\
    &\text{380 нм---фиол---440---син---485---гол---500---зел---565---жёл---590---оранж---625---крас---780 нм}, \text{увидит}
    \end{align*}
}
\solutionspace{60pt}

\tasknumber{2}%
\task{%
    Установка для наблюдения интерференции состоит
    из двух когерентных источников света и экрана.
    Расстояние между источниками $l = 1{,}2\,\text{мм}$,
    а от каждого источника до экрана — $L = 3\,\text{м}$.
    Сделайте рисунок и укажите положение нулевого максимума освещенности,
    а также определите расстояние между третьим минимумом и нулевым максимумом.
    Длина волны падающего света составляет $\lambda = 600\,\text{нм}$.
}
\answer{%
    \begin{align*}
    l_1^2 &= L^2 + \sqr{x - \frac \ell 2} \\
    l_2^2 &= L^2 + \sqr{x + \frac \ell 2} \\
    l_2^2 - l_1^2 &= 2x\ell \implies (l_2 - l_1)(l_2 + l_1) = 2x\ell \implies n\lambda \cdot 2L \approx 2x_n\ell \implies x_n = \frac{\lambda L}{\ell} n, n\in \mathbb{N} \\
    x &= \frac{\lambda L}{\ell} \cdot \frac52 = \frac{600\,\text{нм} \cdot 3\,\text{м}}{1{,}2\,\text{мм}} \cdot \frac52 \approx 3{,}8\,\text{мм}
    \end{align*}
}
\solutionspace{180pt}

\tasknumber{3}%
\task{%
    На стеклянную пластинку ($\hat n = 1{,}5$) нанесена прозрачная пленка ($n = 1{,}7$).
    На плёнку нормально к поверхности падает монохроматический свет с длиной волны $540\,\text{нм}$.
    Какова должна быть минимальная толщина пленки, чтобы в результате интерференции отражённый свет имел наименьшую интенсивность?
}
\answer{%
    $2 \cdot h \cdot n = 1 \lambda \implies h \approx 159\,\text{нм}$
}

\variantsplitter

\addpersonalvariant{Руслан Перепелица}

\tasknumber{1}%
\task{%
    Длина волны света в~вакууме $\lambda = 700\,\text{нм}$.
    Какова частота этой световой волны?
    Какова длина этой волны в среде с показателем преломления $n = 1{,}5$?
    Может ли человек увидеть такую волну света, и если да, то какой именно цвет соответствует этим волнам в вакууме и в этой среде?
}
\answer{%
    \begin{align*}
    \nu &= \frac 1T = \frac 1{\lambda/c} = \frac c\lambda = \frac{3 \cdot 10^{8}\,\frac{\text{м}}{\text{с}}}{700\,\text{нм}} \approx 429 \cdot 10^{12}\,\text{Гц}, \\
    \nu' &= \nu \cbr{\text{или } T' = T} \implies \lambda' = v'T' = \frac vn T = \frac{ vt }n = \frac \lambda n = \frac{700\,\text{нм}}{1{,}5} \approx 470\,\text{нм}.
    \\
    &\text{380 нм---фиол---440---син---485---гол---500---зел---565---жёл---590---оранж---625---крас---780 нм}, \text{увидит}
    \end{align*}
}
\solutionspace{60pt}

\tasknumber{2}%
\task{%
    Установка для наблюдения интерференции состоит
    из двух когерентных источников света и экрана.
    Расстояние между источниками $l = 0{,}8\,\text{мм}$,
    а от каждого источника до экрана — $L = 2\,\text{м}$.
    Сделайте рисунок и укажите положение нулевого максимума освещенности,
    а также определите расстояние между четвёртым максимумом и нулевым максимумом.
    Длина волны падающего света составляет $\lambda = 500\,\text{нм}$.
}
\answer{%
    \begin{align*}
    l_1^2 &= L^2 + \sqr{x - \frac \ell 2} \\
    l_2^2 &= L^2 + \sqr{x + \frac \ell 2} \\
    l_2^2 - l_1^2 &= 2x\ell \implies (l_2 - l_1)(l_2 + l_1) = 2x\ell \implies n\lambda \cdot 2L \approx 2x_n\ell \implies x_n = \frac{\lambda L}{\ell} n, n\in \mathbb{N} \\
    x &= \frac{\lambda L}{\ell} \cdot 4 = \frac{500\,\text{нм} \cdot 2\,\text{м}}{0{,}8\,\text{мм}} \cdot 4 \approx 5\,\text{мм}
    \end{align*}
}
\solutionspace{180pt}

\tasknumber{3}%
\task{%
    На стеклянную пластинку ($\hat n = 1{,}6$) нанесена прозрачная пленка ($n = 1{,}7$).
    На плёнку нормально к поверхности падает монохроматический свет с длиной волны $640\,\text{нм}$.
    Какова должна быть минимальная толщина пленки, чтобы в результате интерференции отражённый свет имел наибольшую интенсивность?
}
\answer{%
    $2 \cdot h \cdot n = \frac12 \lambda \implies h \approx 94\,\text{нм}$
}

\variantsplitter

\addpersonalvariant{Михаил Перин}

\tasknumber{1}%
\task{%
    Длина волны света в~вакууме $\lambda = 400\,\text{нм}$.
    Какова частота этой световой волны?
    Какова длина этой волны в среде с показателем преломления $n = 1{,}6$?
    Может ли человек увидеть такую волну света, и если да, то какой именно цвет соответствует этим волнам в вакууме и в этой среде?
}
\answer{%
    \begin{align*}
    \nu &= \frac 1T = \frac 1{\lambda/c} = \frac c\lambda = \frac{3 \cdot 10^{8}\,\frac{\text{м}}{\text{с}}}{400\,\text{нм}} \approx 750 \cdot 10^{12}\,\text{Гц}, \\
    \nu' &= \nu \cbr{\text{или } T' = T} \implies \lambda' = v'T' = \frac vn T = \frac{ vt }n = \frac \lambda n = \frac{400\,\text{нм}}{1{,}6} \approx 250\,\text{нм}.
    \\
    &\text{380 нм---фиол---440---син---485---гол---500---зел---565---жёл---590---оранж---625---крас---780 нм}, \text{увидит}
    \end{align*}
}
\solutionspace{60pt}

\tasknumber{2}%
\task{%
    Установка для наблюдения интерференции состоит
    из двух когерентных источников света и экрана.
    Расстояние между источниками $l = 0{,}8\,\text{мм}$,
    а от каждого источника до экрана — $L = 4\,\text{м}$.
    Сделайте рисунок и укажите положение нулевого максимума освещенности,
    а также определите расстояние между вторым максимумом и нулевым максимумом.
    Длина волны падающего света составляет $\lambda = 450\,\text{нм}$.
}
\answer{%
    \begin{align*}
    l_1^2 &= L^2 + \sqr{x - \frac \ell 2} \\
    l_2^2 &= L^2 + \sqr{x + \frac \ell 2} \\
    l_2^2 - l_1^2 &= 2x\ell \implies (l_2 - l_1)(l_2 + l_1) = 2x\ell \implies n\lambda \cdot 2L \approx 2x_n\ell \implies x_n = \frac{\lambda L}{\ell} n, n\in \mathbb{N} \\
    x &= \frac{\lambda L}{\ell} \cdot 2 = \frac{450\,\text{нм} \cdot 4\,\text{м}}{0{,}8\,\text{мм}} \cdot 2 \approx 4{,}5\,\text{мм}
    \end{align*}
}
\solutionspace{180pt}

\tasknumber{3}%
\task{%
    На стеклянную пластинку ($\hat n = 1{,}5$) нанесена прозрачная пленка ($n = 1{,}4$).
    На плёнку нормально к поверхности падает монохроматический свет с длиной волны $420\,\text{нм}$.
    Какова должна быть минимальная толщина пленки, чтобы в результате интерференции отражённый свет имел наименьшую интенсивность?
}
\answer{%
    $2 \cdot h \cdot n = \frac12 \lambda \implies h \approx 75\,\text{нм}$
}

\variantsplitter

\addpersonalvariant{Егор Подуровский}

\tasknumber{1}%
\task{%
    Длина волны света в~вакууме $\lambda = 400\,\text{нм}$.
    Какова частота этой световой волны?
    Какова длина этой волны в среде с показателем преломления $n = 1{,}3$?
    Может ли человек увидеть такую волну света, и если да, то какой именно цвет соответствует этим волнам в вакууме и в этой среде?
}
\answer{%
    \begin{align*}
    \nu &= \frac 1T = \frac 1{\lambda/c} = \frac c\lambda = \frac{3 \cdot 10^{8}\,\frac{\text{м}}{\text{с}}}{400\,\text{нм}} \approx 750 \cdot 10^{12}\,\text{Гц}, \\
    \nu' &= \nu \cbr{\text{или } T' = T} \implies \lambda' = v'T' = \frac vn T = \frac{ vt }n = \frac \lambda n = \frac{400\,\text{нм}}{1{,}3} \approx 310\,\text{нм}.
    \\
    &\text{380 нм---фиол---440---син---485---гол---500---зел---565---жёл---590---оранж---625---крас---780 нм}, \text{увидит}
    \end{align*}
}
\solutionspace{60pt}

\tasknumber{2}%
\task{%
    Установка для наблюдения интерференции состоит
    из двух когерентных источников света и экрана.
    Расстояние между источниками $l = 2{,}4\,\text{мм}$,
    а от каждого источника до экрана — $L = 2\,\text{м}$.
    Сделайте рисунок и укажите положение нулевого максимума освещенности,
    а также определите расстояние между четвёртым минимумом и нулевым максимумом.
    Длина волны падающего света составляет $\lambda = 450\,\text{нм}$.
}
\answer{%
    \begin{align*}
    l_1^2 &= L^2 + \sqr{x - \frac \ell 2} \\
    l_2^2 &= L^2 + \sqr{x + \frac \ell 2} \\
    l_2^2 - l_1^2 &= 2x\ell \implies (l_2 - l_1)(l_2 + l_1) = 2x\ell \implies n\lambda \cdot 2L \approx 2x_n\ell \implies x_n = \frac{\lambda L}{\ell} n, n\in \mathbb{N} \\
    x &= \frac{\lambda L}{\ell} \cdot \frac72 = \frac{450\,\text{нм} \cdot 2\,\text{м}}{2{,}4\,\text{мм}} \cdot \frac72 \approx 1{,}31\,\text{мм}
    \end{align*}
}
\solutionspace{180pt}

\tasknumber{3}%
\task{%
    На стеклянную пластинку ($\hat n = 1{,}5$) нанесена прозрачная пленка ($n = 1{,}8$).
    На плёнку нормально к поверхности падает монохроматический свет с длиной волны $640\,\text{нм}$.
    Какова должна быть минимальная толщина пленки, чтобы в результате интерференции отражённый свет имел наименьшую интенсивность?
}
\answer{%
    $2 \cdot h \cdot n = 1 \lambda \implies h \approx 178\,\text{нм}$
}

\variantsplitter

\addpersonalvariant{Роман Прибылов}

\tasknumber{1}%
\task{%
    Длина волны света в~вакууме $\lambda = 500\,\text{нм}$.
    Какова частота этой световой волны?
    Какова длина этой волны в среде с показателем преломления $n = 1{,}5$?
    Может ли человек увидеть такую волну света, и если да, то какой именно цвет соответствует этим волнам в вакууме и в этой среде?
}
\answer{%
    \begin{align*}
    \nu &= \frac 1T = \frac 1{\lambda/c} = \frac c\lambda = \frac{3 \cdot 10^{8}\,\frac{\text{м}}{\text{с}}}{500\,\text{нм}} \approx 600 \cdot 10^{12}\,\text{Гц}, \\
    \nu' &= \nu \cbr{\text{или } T' = T} \implies \lambda' = v'T' = \frac vn T = \frac{ vt }n = \frac \lambda n = \frac{500\,\text{нм}}{1{,}5} \approx 330\,\text{нм}.
    \\
    &\text{380 нм---фиол---440---син---485---гол---500---зел---565---жёл---590---оранж---625---крас---780 нм}, \text{увидит}
    \end{align*}
}
\solutionspace{60pt}

\tasknumber{2}%
\task{%
    Установка для наблюдения интерференции состоит
    из двух когерентных источников света и экрана.
    Расстояние между источниками $l = 2{,}4\,\text{мм}$,
    а от каждого источника до экрана — $L = 2\,\text{м}$.
    Сделайте рисунок и укажите положение нулевого максимума освещенности,
    а также определите расстояние между вторым максимумом и нулевым максимумом.
    Длина волны падающего света составляет $\lambda = 500\,\text{нм}$.
}
\answer{%
    \begin{align*}
    l_1^2 &= L^2 + \sqr{x - \frac \ell 2} \\
    l_2^2 &= L^2 + \sqr{x + \frac \ell 2} \\
    l_2^2 - l_1^2 &= 2x\ell \implies (l_2 - l_1)(l_2 + l_1) = 2x\ell \implies n\lambda \cdot 2L \approx 2x_n\ell \implies x_n = \frac{\lambda L}{\ell} n, n\in \mathbb{N} \\
    x &= \frac{\lambda L}{\ell} \cdot 2 = \frac{500\,\text{нм} \cdot 2\,\text{м}}{2{,}4\,\text{мм}} \cdot 2 \approx 0{,}83\,\text{мм}
    \end{align*}
}
\solutionspace{180pt}

\tasknumber{3}%
\task{%
    На стеклянную пластинку ($\hat n = 1{,}6$) нанесена прозрачная пленка ($n = 1{,}8$).
    На плёнку нормально к поверхности падает монохроматический свет с длиной волны $640\,\text{нм}$.
    Какова должна быть минимальная толщина пленки, чтобы в результате интерференции отражённый свет имел наибольшую интенсивность?
}
\answer{%
    $2 \cdot h \cdot n = \frac12 \lambda \implies h \approx 89\,\text{нм}$
}

\variantsplitter

\addpersonalvariant{Александр Селехметьев}

\tasknumber{1}%
\task{%
    Длина волны света в~вакууме $\lambda = 500\,\text{нм}$.
    Какова частота этой световой волны?
    Какова длина этой волны в среде с показателем преломления $n = 1{,}5$?
    Может ли человек увидеть такую волну света, и если да, то какой именно цвет соответствует этим волнам в вакууме и в этой среде?
}
\answer{%
    \begin{align*}
    \nu &= \frac 1T = \frac 1{\lambda/c} = \frac c\lambda = \frac{3 \cdot 10^{8}\,\frac{\text{м}}{\text{с}}}{500\,\text{нм}} \approx 600 \cdot 10^{12}\,\text{Гц}, \\
    \nu' &= \nu \cbr{\text{или } T' = T} \implies \lambda' = v'T' = \frac vn T = \frac{ vt }n = \frac \lambda n = \frac{500\,\text{нм}}{1{,}5} \approx 330\,\text{нм}.
    \\
    &\text{380 нм---фиол---440---син---485---гол---500---зел---565---жёл---590---оранж---625---крас---780 нм}, \text{увидит}
    \end{align*}
}
\solutionspace{60pt}

\tasknumber{2}%
\task{%
    Установка для наблюдения интерференции состоит
    из двух когерентных источников света и экрана.
    Расстояние между источниками $l = 1{,}5\,\text{мм}$,
    а от каждого источника до экрана — $L = 2\,\text{м}$.
    Сделайте рисунок и укажите положение нулевого максимума освещенности,
    а также определите расстояние между третьим минимумом и нулевым максимумом.
    Длина волны падающего света составляет $\lambda = 450\,\text{нм}$.
}
\answer{%
    \begin{align*}
    l_1^2 &= L^2 + \sqr{x - \frac \ell 2} \\
    l_2^2 &= L^2 + \sqr{x + \frac \ell 2} \\
    l_2^2 - l_1^2 &= 2x\ell \implies (l_2 - l_1)(l_2 + l_1) = 2x\ell \implies n\lambda \cdot 2L \approx 2x_n\ell \implies x_n = \frac{\lambda L}{\ell} n, n\in \mathbb{N} \\
    x &= \frac{\lambda L}{\ell} \cdot \frac52 = \frac{450\,\text{нм} \cdot 2\,\text{м}}{1{,}5\,\text{мм}} \cdot \frac52 \approx 1{,}50\,\text{мм}
    \end{align*}
}
\solutionspace{180pt}

\tasknumber{3}%
\task{%
    На стеклянную пластинку ($\hat n = 1{,}5$) нанесена прозрачная пленка ($n = 1{,}3$).
    На плёнку нормально к поверхности падает монохроматический свет с длиной волны $540\,\text{нм}$.
    Какова должна быть минимальная толщина пленки, чтобы в результате интерференции отражённый свет имел наименьшую интенсивность?
}
\answer{%
    $2 \cdot h \cdot n = \frac12 \lambda \implies h \approx 104\,\text{нм}$
}

\variantsplitter

\addpersonalvariant{Алексей Тихонов}

\tasknumber{1}%
\task{%
    Длина волны света в~вакууме $\lambda = 600\,\text{нм}$.
    Какова частота этой световой волны?
    Какова длина этой волны в среде с показателем преломления $n = 1{,}7$?
    Может ли человек увидеть такую волну света, и если да, то какой именно цвет соответствует этим волнам в вакууме и в этой среде?
}
\answer{%
    \begin{align*}
    \nu &= \frac 1T = \frac 1{\lambda/c} = \frac c\lambda = \frac{3 \cdot 10^{8}\,\frac{\text{м}}{\text{с}}}{600\,\text{нм}} \approx 500 \cdot 10^{12}\,\text{Гц}, \\
    \nu' &= \nu \cbr{\text{или } T' = T} \implies \lambda' = v'T' = \frac vn T = \frac{ vt }n = \frac \lambda n = \frac{600\,\text{нм}}{1{,}7} \approx 350\,\text{нм}.
    \\
    &\text{380 нм---фиол---440---син---485---гол---500---зел---565---жёл---590---оранж---625---крас---780 нм}, \text{увидит}
    \end{align*}
}
\solutionspace{60pt}

\tasknumber{2}%
\task{%
    Установка для наблюдения интерференции состоит
    из двух когерентных источников света и экрана.
    Расстояние между источниками $l = 1{,}2\,\text{мм}$,
    а от каждого источника до экрана — $L = 3\,\text{м}$.
    Сделайте рисунок и укажите положение нулевого максимума освещенности,
    а также определите расстояние между третьим максимумом и нулевым максимумом.
    Длина волны падающего света составляет $\lambda = 550\,\text{нм}$.
}
\answer{%
    \begin{align*}
    l_1^2 &= L^2 + \sqr{x - \frac \ell 2} \\
    l_2^2 &= L^2 + \sqr{x + \frac \ell 2} \\
    l_2^2 - l_1^2 &= 2x\ell \implies (l_2 - l_1)(l_2 + l_1) = 2x\ell \implies n\lambda \cdot 2L \approx 2x_n\ell \implies x_n = \frac{\lambda L}{\ell} n, n\in \mathbb{N} \\
    x &= \frac{\lambda L}{\ell} \cdot 3 = \frac{550\,\text{нм} \cdot 3\,\text{м}}{1{,}2\,\text{мм}} \cdot 3 \approx 4{,}1\,\text{мм}
    \end{align*}
}
\solutionspace{180pt}

\tasknumber{3}%
\task{%
    На стеклянную пластинку ($\hat n = 1{,}6$) нанесена прозрачная пленка ($n = 1{,}8$).
    На плёнку нормально к поверхности падает монохроматический свет с длиной волны $480\,\text{нм}$.
    Какова должна быть минимальная толщина пленки, чтобы в результате интерференции отражённый свет имел наименьшую интенсивность?
}
\answer{%
    $2 \cdot h \cdot n = 1 \lambda \implies h \approx 133\,\text{нм}$
}

\variantsplitter

\addpersonalvariant{Алина Филиппова}

\tasknumber{1}%
\task{%
    Длина волны света в~вакууме $\lambda = 700\,\text{нм}$.
    Какова частота этой световой волны?
    Какова длина этой волны в среде с показателем преломления $n = 1{,}3$?
    Может ли человек увидеть такую волну света, и если да, то какой именно цвет соответствует этим волнам в вакууме и в этой среде?
}
\answer{%
    \begin{align*}
    \nu &= \frac 1T = \frac 1{\lambda/c} = \frac c\lambda = \frac{3 \cdot 10^{8}\,\frac{\text{м}}{\text{с}}}{700\,\text{нм}} \approx 429 \cdot 10^{12}\,\text{Гц}, \\
    \nu' &= \nu \cbr{\text{или } T' = T} \implies \lambda' = v'T' = \frac vn T = \frac{ vt }n = \frac \lambda n = \frac{700\,\text{нм}}{1{,}3} \approx 540\,\text{нм}.
    \\
    &\text{380 нм---фиол---440---син---485---гол---500---зел---565---жёл---590---оранж---625---крас---780 нм}, \text{увидит}
    \end{align*}
}
\solutionspace{60pt}

\tasknumber{2}%
\task{%
    Установка для наблюдения интерференции состоит
    из двух когерентных источников света и экрана.
    Расстояние между источниками $l = 2{,}4\,\text{мм}$,
    а от каждого источника до экрана — $L = 4\,\text{м}$.
    Сделайте рисунок и укажите положение нулевого максимума освещенности,
    а также определите расстояние между третьим максимумом и нулевым максимумом.
    Длина волны падающего света составляет $\lambda = 400\,\text{нм}$.
}
\answer{%
    \begin{align*}
    l_1^2 &= L^2 + \sqr{x - \frac \ell 2} \\
    l_2^2 &= L^2 + \sqr{x + \frac \ell 2} \\
    l_2^2 - l_1^2 &= 2x\ell \implies (l_2 - l_1)(l_2 + l_1) = 2x\ell \implies n\lambda \cdot 2L \approx 2x_n\ell \implies x_n = \frac{\lambda L}{\ell} n, n\in \mathbb{N} \\
    x &= \frac{\lambda L}{\ell} \cdot 3 = \frac{400\,\text{нм} \cdot 4\,\text{м}}{2{,}4\,\text{мм}} \cdot 3 \approx 2\,\text{мм}
    \end{align*}
}
\solutionspace{180pt}

\tasknumber{3}%
\task{%
    На стеклянную пластинку ($\hat n = 1{,}5$) нанесена прозрачная пленка ($n = 1{,}3$).
    На плёнку нормально к поверхности падает монохроматический свет с длиной волны $640\,\text{нм}$.
    Какова должна быть минимальная толщина пленки, чтобы в результате интерференции отражённый свет имел наименьшую интенсивность?
}
\answer{%
    $2 \cdot h \cdot n = \frac12 \lambda \implies h \approx 123\,\text{нм}$
}

\variantsplitter

\addpersonalvariant{Дарья Шашкова}

\tasknumber{1}%
\task{%
    Длина волны света в~вакууме $\lambda = 600\,\text{нм}$.
    Какова частота этой световой волны?
    Какова длина этой волны в среде с показателем преломления $n = 1{,}5$?
    Может ли человек увидеть такую волну света, и если да, то какой именно цвет соответствует этим волнам в вакууме и в этой среде?
}
\answer{%
    \begin{align*}
    \nu &= \frac 1T = \frac 1{\lambda/c} = \frac c\lambda = \frac{3 \cdot 10^{8}\,\frac{\text{м}}{\text{с}}}{600\,\text{нм}} \approx 500 \cdot 10^{12}\,\text{Гц}, \\
    \nu' &= \nu \cbr{\text{или } T' = T} \implies \lambda' = v'T' = \frac vn T = \frac{ vt }n = \frac \lambda n = \frac{600\,\text{нм}}{1{,}5} \approx 400\,\text{нм}.
    \\
    &\text{380 нм---фиол---440---син---485---гол---500---зел---565---жёл---590---оранж---625---крас---780 нм}, \text{увидит}
    \end{align*}
}
\solutionspace{60pt}

\tasknumber{2}%
\task{%
    Установка для наблюдения интерференции состоит
    из двух когерентных источников света и экрана.
    Расстояние между источниками $l = 1{,}5\,\text{мм}$,
    а от каждого источника до экрана — $L = 4\,\text{м}$.
    Сделайте рисунок и укажите положение нулевого максимума освещенности,
    а также определите расстояние между третьим максимумом и нулевым максимумом.
    Длина волны падающего света составляет $\lambda = 600\,\text{нм}$.
}
\answer{%
    \begin{align*}
    l_1^2 &= L^2 + \sqr{x - \frac \ell 2} \\
    l_2^2 &= L^2 + \sqr{x + \frac \ell 2} \\
    l_2^2 - l_1^2 &= 2x\ell \implies (l_2 - l_1)(l_2 + l_1) = 2x\ell \implies n\lambda \cdot 2L \approx 2x_n\ell \implies x_n = \frac{\lambda L}{\ell} n, n\in \mathbb{N} \\
    x &= \frac{\lambda L}{\ell} \cdot 3 = \frac{600\,\text{нм} \cdot 4\,\text{м}}{1{,}5\,\text{мм}} \cdot 3 \approx 4{,}8\,\text{мм}
    \end{align*}
}
\solutionspace{180pt}

\tasknumber{3}%
\task{%
    На стеклянную пластинку ($\hat n = 1{,}5$) нанесена прозрачная пленка ($n = 1{,}8$).
    На плёнку нормально к поверхности падает монохроматический свет с длиной волны $420\,\text{нм}$.
    Какова должна быть минимальная толщина пленки, чтобы в результате интерференции отражённый свет имел наибольшую интенсивность?
}
\answer{%
    $2 \cdot h \cdot n = \frac12 \lambda \implies h \approx 58\,\text{нм}$
}

\variantsplitter

\addpersonalvariant{Алина Яшина}

\tasknumber{1}%
\task{%
    Длина волны света в~вакууме $\lambda = 600\,\text{нм}$.
    Какова частота этой световой волны?
    Какова длина этой волны в среде с показателем преломления $n = 1{,}5$?
    Может ли человек увидеть такую волну света, и если да, то какой именно цвет соответствует этим волнам в вакууме и в этой среде?
}
\answer{%
    \begin{align*}
    \nu &= \frac 1T = \frac 1{\lambda/c} = \frac c\lambda = \frac{3 \cdot 10^{8}\,\frac{\text{м}}{\text{с}}}{600\,\text{нм}} \approx 500 \cdot 10^{12}\,\text{Гц}, \\
    \nu' &= \nu \cbr{\text{или } T' = T} \implies \lambda' = v'T' = \frac vn T = \frac{ vt }n = \frac \lambda n = \frac{600\,\text{нм}}{1{,}5} \approx 400\,\text{нм}.
    \\
    &\text{380 нм---фиол---440---син---485---гол---500---зел---565---жёл---590---оранж---625---крас---780 нм}, \text{увидит}
    \end{align*}
}
\solutionspace{60pt}

\tasknumber{2}%
\task{%
    Установка для наблюдения интерференции состоит
    из двух когерентных источников света и экрана.
    Расстояние между источниками $l = 0{,}8\,\text{мм}$,
    а от каждого источника до экрана — $L = 3\,\text{м}$.
    Сделайте рисунок и укажите положение нулевого максимума освещенности,
    а также определите расстояние между вторым минимумом и нулевым максимумом.
    Длина волны падающего света составляет $\lambda = 500\,\text{нм}$.
}
\answer{%
    \begin{align*}
    l_1^2 &= L^2 + \sqr{x - \frac \ell 2} \\
    l_2^2 &= L^2 + \sqr{x + \frac \ell 2} \\
    l_2^2 - l_1^2 &= 2x\ell \implies (l_2 - l_1)(l_2 + l_1) = 2x\ell \implies n\lambda \cdot 2L \approx 2x_n\ell \implies x_n = \frac{\lambda L}{\ell} n, n\in \mathbb{N} \\
    x &= \frac{\lambda L}{\ell} \cdot \frac32 = \frac{500\,\text{нм} \cdot 3\,\text{м}}{0{,}8\,\text{мм}} \cdot \frac32 \approx 2{,}8\,\text{мм}
    \end{align*}
}
\solutionspace{180pt}

\tasknumber{3}%
\task{%
    На стеклянную пластинку ($\hat n = 1{,}5$) нанесена прозрачная пленка ($n = 1{,}8$).
    На плёнку нормально к поверхности падает монохроматический свет с длиной волны $540\,\text{нм}$.
    Какова должна быть минимальная толщина пленки, чтобы в результате интерференции отражённый свет имел наименьшую интенсивность?
}
\answer{%
    $2 \cdot h \cdot n = 1 \lambda \implies h \approx 150\,\text{нм}$
}
% autogenerated
