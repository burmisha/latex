\documentclass[12pt,a4paper]{amsart}%DVI-mode.
\usepackage{graphics,graphicx,epsfig}%DVI-mode.
% \documentclass[pdftex,12pt]{amsart} %PDF-mode.
% \usepackage[pdftex]{graphicx}       %PDF-mode.
% \usepackage[babel=true]{microtype}
% \usepackage[T1]{fontenc}
% \usepackage{lmodern}

\usepackage{cmap}
%\usepackage{a4wide}                 % Fit the text to A4 page tightly.
% \usepackage[utf8]{inputenc}
\usepackage[T2A]{fontenc}
\usepackage[english,russian]{babel} % Download Russian fonts.
\usepackage{amsmath,amsfonts,amssymb,amsthm,amscd,mathrsfs} % Use AMS symbols.
\usepackage{tikz}
\usetikzlibrary{circuits.ee.IEC}
\usetikzlibrary{shapes.geometric}
\usetikzlibrary{decorations.markings}
%\usetikzlibrary{dashs}
%\usetikzlibrary{info}


\textheight=28cm % высота текста
\textwidth=18cm % ширина текста
\topmargin=-2.5cm % отступ от верхнего края
\parskip=2pt % интервал между абзацами
\oddsidemargin=-1.5cm
\evensidemargin=-1.5cm 

\parindent=0pt % абзацный отступ
\tolerance=500 % терпимость к "жидким" строкам
\binoppenalty=10000 % штраф за перенос формул - 10000 - абсолютный запрет
\relpenalty=10000
\flushbottom % выравнивание высоты страниц
\pagenumbering{gobble}

\newcommand\bivec[2]{\begin{pmatrix} #1 \\ #2 \end{pmatrix}}

\newcommand\ol[1]{\overline{#1}}

\newcommand\p[1]{\Prob\!\left(#1\right)}
\newcommand\e[1]{\mathsf{E}\!\left(#1\right)}
\newcommand\disp[1]{\mathsf{D}\!\left(#1\right)}
%\newcommand\norm[2]{\mathcal{N}\!\cbr{#1,#2}}
\newcommand\sign{\text{ sign }}

\newcommand\al[1]{\begin{align*} #1 \end{align*}}
\newcommand\begcas[1]{\begin{cases}#1\end{cases}}
\newcommand\tab[2]{	\vspace{-#1pt}
						\begin{tabbing} 
						#2
						\end{tabbing}
					\vspace{-#1pt}
					}

\newcommand\maintext[1]{{\bfseries\sffamily{#1}}}
\newcommand\skipped[1]{ {\ensuremath{\text{\small{\sffamily{Пропущено:} #1} } } } }
\newcommand\simpletitle[1]{\begin{center} \maintext{#1} \end{center}}

\def\le{\leqslant}
\def\ge{\geqslant}
\def\Ell{\mathcal{L}}
\def\eps{{\varepsilon}}
\def\Rn{\mathbb{R}^n}
\def\RSS{\mathsf{RSS}}

\newcommand\foral[1]{\forall\,#1\:}
\newcommand\exist[1]{\exists\,#1\:\colon}

\newcommand\cbr[1]{\left(#1\right)} %circled brackets
\newcommand\fbr[1]{\left\{#1\right\}} %figure brackets
\newcommand\sbr[1]{\left[#1\right]} %square brackets
\newcommand\modul[1]{\left|#1\right|}

\newcommand\sqr[1]{\cbr{#1}^2}
\newcommand\inv[1]{\cbr{#1}^{-1}}

\newcommand\cdf[2]{\cdot\frac{#1}{#2}}
\newcommand\dd[2]{\frac{\partial#1}{\partial#2}}

\newcommand\integr[2]{\int\limits_{#1}^{#2}}
\newcommand\suml[2]{\sum\limits_{#1}^{#2}}
\newcommand\isum[2]{\sum\limits_{#1=#2}^{+\infty}}
\newcommand\idots[3]{#1_{#2},\ldots,#1_{#3}}
\newcommand\fdots[5]{#4{#1_{#2}}#5\ldots#5#4{#1_{#3}}}

\newcommand\obol[1]{O\!\cbr{#1}}
\newcommand\omal[1]{o\!\cbr{#1}}

\newcommand\addeps[2]{
	\begin{figure} [!ht] %lrp
		\centering
		\includegraphics[height=320px]{#1.eps}
		\vspace{-10pt}
		\caption{#2}
		\label{eps:#1}
	\end{figure}
}

\newcommand\addepssize[3]{
	\begin{figure} [!ht] %lrp hp
		\centering
		\includegraphics[height=#3px]{#1.eps}
		\vspace{-10pt}
		\caption{#2}
		\label{eps:#1}
	\end{figure}
}


\newcommand\norm[1]{\ensuremath{\left\|{#1}\right\|}}
\newcommand\ort{\bot}
\newcommand\theorem[1]{{\sffamily Теорема #1\ }}
\newcommand\lemma[1]{{\sffamily Лемма #1\ }}
\newcommand\difflim[2]{\frac{#1\cbr{#2 + \Delta#2} - #1\cbr{#2}}{\Delta #2}}
\renewcommand\proof[1]{\par\noindent$\square$ #1 \hfill$\blacksquare$\par}
\newcommand\defenition[1]{{\sffamilyОпределение #1\ }}

% \begin{document}
% %\raggedright
% \addclassdate{7}{20 апреля 2018}

\task 1
Площадь большого поршня гидравлического домкрата $S_1 = 20\units{см}^2$, а малого $S_2 = 0{,}5\units{см}^2.$ Груз какой максимальной массы можно поднять этим домкратом, если на малый поршень давить с силой не более $F=200\units{Н}?$ Силой трения от стенки цилиндров пренебречь.

\task 2
В сосуд налита вода. Расстояние от поверхности воды до дна $H = 0{,}5\units{м},$ площадь дна $S = 0{,}1\units{м}^2.$ Найти гидростатическое давление $P_1$ и полное давление $P_2$ вблизи дна. Найти силу давления воды на дно. Плотность воды \rhowater

\task 3
На лёгкий поршень площадью $S=900\units{см}^2,$ касающийся поверхности воды, поставили гирю массы $m=3\units{кг}$. Высота слоя воды в сосуде с вертикальными стенками $H = 20\units{см}$. Определить давление жидкости вблизи дна, если плотность воды \rhowater

\task 4
Давление газов в конце сгорания в цилиндре дизельного двигателя трактора $P = 9\units{МПа}.$ Диаметр цилиндра $d = 130\units{мм}.$ С какой силой газы давят на поршень в цилиндре? Площадь круга диаметром $D$ равна $S = \cfrac{\pi D^2}4.$

\task 5
Площадь малого поршня гидравлического подъёмника $S_1 = 0{,}8\units{см}^2$, а большого $S_2 = 40\units{см}^2.$ Какую силу $F$ надо приложить к малому поршню, чтобы поднять груз весом $P = 8\units{кН}?$

\task 6
Герметичный сосуд полностью заполнен водой и стоит на столе. На небольшой поршень площадью $S$ давят рукой с силой $F$. Поршень находится ниже крышки сосуда на $H_1$, выше дна на $H_2$ и может свободно перемещаться. Плотность воды $\rho$, атмосферное давление $P_A$. Найти давления $P_1$ и $P_2$ в воде вблизи крышки и дна сосуда.
\\ \\
\addclassdate{7}{20 апреля 2018}

\task 1
Площадь большого поршня гидравлического домкрата $S_1 = 20\units{см}^2$, а малого $S_2 = 0{,}5\units{см}^2.$ Груз какой максимальной массы можно поднять этим домкратом, если на малый поршень давить с силой не более $F=200\units{Н}?$ Силой трения от стенки цилиндров пренебречь.

\task 2
В сосуд налита вода. Расстояние от поверхности воды до дна $H = 0{,}5\units{м},$ площадь дна $S = 0{,}1\units{м}^2.$ Найти гидростатическое давление $P_1$ и полное давление $P_2$ вблизи дна. Найти силу давления воды на дно. Плотность воды \rhowater

\task 3
На лёгкий поршень площадью $S=900\units{см}^2,$ касающийся поверхности воды, поставили гирю массы $m=3\units{кг}$. Высота слоя воды в сосуде с вертикальными стенками $H = 20\units{см}$. Определить давление жидкости вблизи дна, если плотность воды \rhowater

\task 4
Давление газов в конце сгорания в цилиндре дизельного двигателя трактора $P = 9\units{МПа}.$ Диаметр цилиндра $d = 130\units{мм}.$ С какой силой газы давят на поршень в цилиндре? Площадь круга диаметром $D$ равна $S = \cfrac{\pi D^2}4.$

\task 5
Площадь малого поршня гидравлического подъёмника $S_1 = 0{,}8\units{см}^2$, а большого $S_2 = 40\units{см}^2.$ Какую силу $F$ надо приложить к малому поршню, чтобы поднять груз весом $P = 8\units{кН}?$

\task 6
Герметичный сосуд полностью заполнен водой и стоит на столе. На небольшой поршень площадью $S$ давят рукой с силой $F$. Поршень находится ниже крышки сосуда на $H_1$, выше дна на $H_2$ и может свободно перемещаться. Плотность воды $\rho$, атмосферное давление $P_A$. Найти давления $P_1$ и $P_2$ в воде вблизи крышки и дна сосуда.

\newpage

\adddate{8 класс. 20 апреля 2018}

\task 1
Между точками $A$ и $B$ электрической цепи подключены последовательно резисторы $R_1 = 10\units{Ом}$ и $R_2 = 20\units{Ом}$ и параллельно им $R_3 = 30\units{Ом}.$ Найдите эквивалентное сопротивление $R_{AB}$ этого участка цепи.

\task 2
Электрическая цепь состоит из последовательности $N$ одинаковых звеньев, в которых каждый резистор имеет сопротивление $r$. Последнее звено замкнуто резистором сопротивлением $R$. При каком соотношении $\cfrac{R}{r}$ сопротивление цепи не зависит от числа звеньев?

\task 3
Для измерения сопротивления $R$ проводника собрана электрическая цепь. Вольтметр $V$ показывает напряжение $U_V = 5\units{В},$ показание амперметра $A$ равно $I_A = 25\units{мА}.$ Найдите величину $R$ сопротивления проводника. Внутреннее сопротивление вольтметра $R_V = 1{,}0\units{кОм},$ внутреннее сопротивление амперметра $R_A = 2{,}0\units{Ом}.$

\task 4
Шкала гальванометра имеет $N=100$ делений, цена деления $\delta = 1\units{мкА}$. Внутреннее сопротивление гальванометра $R_G = 1{,}0\units{кОм}.$ Как из этого прибора сделать вольтметр для измерения напряжений до $U = 100\units{В}$ или амперметр для измерения токов силой до $I = 1\units{А}?$

\\ \\ \\ \\ \\ \\ \\ \\
\adddate{8 класс. 20 апреля 2018}

\task 1
Между точками $A$ и $B$ электрической цепи подключены последовательно резисторы $R_1 = 10\units{Ом}$ и $R_2 = 20\units{Ом}$ и параллельно им $R_3 = 30\units{Ом}.$ Найдите эквивалентное сопротивление $R_{AB}$ этого участка цепи.

\task 2
Электрическая цепь состоит из последовательности $N$ одинаковых звеньев, в которых каждый резистор имеет сопротивление $r$. Последнее звено замкнуто резистором сопротивлением $R$. При каком соотношении $\cfrac{R}{r}$ сопротивление цепи не зависит от числа звеньев?

\task 3
Для измерения сопротивления $R$ проводника собрана электрическая цепь. Вольтметр $V$ показывает напряжение $U_V = 5\units{В},$ показание амперметра $A$ равно $I_A = 25\units{мА}.$ Найдите величину $R$ сопротивления проводника. Внутреннее сопротивление вольтметра $R_V = 1{,}0\units{кОм},$ внутреннее сопротивление амперметра $R_A = 2{,}0\units{Ом}.$

\task 4
Шкала гальванометра имеет $N=100$ делений, цена деления $\delta = 1\units{мкА}$. Внутреннее сопротивление гальванометра $R_G = 1{,}0\units{кОм}.$ Как из этого прибора сделать вольтметр для измерения напряжений до $U = 100\units{В}$ или амперметр для измерения токов силой до $I = 1\units{А}?$


% % \begin{flushright}
\textsc{ГБОУ школа №554, 20 ноября 2018\,г.}
\end{flushright}

\begin{center}
\LARGE \textsc{Математический бой, 8 класс}
\end{center}

\problem{1} Есть тридцать карточек, на каждой написано по одному числу: на десяти карточках~–~$a$,  на десяти других~–~$b$ и на десяти оставшихся~–~$c$ (числа  различны). Известно, что к любым пяти карточкам можно подобрать ещё пять так, что сумма чисел на этих десяти карточках будет равна нулю. Докажите, что~одно из~чисел~$a, b, c$ равно нулю.

\problem{2} Вокруг стола стола пустили пакет с орешками. Первый взял один орешек, второй — 2, третий — 3 и так далее: каждый следующий брал на 1 орешек больше. Известно, что на втором круге было взято в сумме на 100 орешков больше, чем на первом. Сколько человек сидело за столом?

% \problem{2} Натуральное число разрешено увеличить на любое целое число процентов от 1 до 100, если при этом получаем натуральное число. Найдите наименьшее натуральное число, которое нельзя при помощи таких операций получить из~числа 1.

% \problem{3} Найти сумму $1^2 - 2^2 + 3^2 - 4^2 + 5^2 + \ldots - 2018^2$.

\problem{3} В кружке рукоделия, где занимается Валя, более 93\% участников~—~девочки. Какое наименьшее число участников может быть в таком кружке?

\problem{4} Произведение 2018 целых чисел равно 1. Может ли их сумма оказаться равной~0?

% \problem{4} Можно ли все натуральные числа от~1 до~9 записать в~клетки таблицы~$3\times3$ так, чтобы сумма в~любых двух соседних (по~вертикали или горизонтали) клетках равнялось простому числу?

\problem{5} На доске написано 2018 нулей и 2019 единиц. Женя стирает 2 числа и, если они были одинаковы, дописывает к оставшимся один ноль, а~если разные — единицу. Потом Женя повторяет эту операцию снова, потом ещё и~так далее. В~результате на~доске останется только одно число. Что это за~число?

\problem{6} Докажите, что в~любой компании людей найдутся 2~человека, имеющие равное число знакомых в этой компании (если $A$~знаком с~$B$, то~и $B$~знаком с~$A$).

\problem{7} Три колокола начинают бить одновременно. Интервалы между ударами колоколов соответственно составляют $\cfrac43$~секунды, $\cfrac53$~секунды и $2$~секунды. Совпавшие по времени удары воспринимаются за~один. Сколько ударов будет услышано за 1~минуту, включая первый и последний удары?

\problem{8} Восемь одинаковых момент расположены по кругу. Известно, что три из~них~— фальшивые, и они расположены рядом друг с~другом. Вес фальшивой монеты отличается от~веса настоящей. Все фальшивые монеты весят одинаково, но неизвестно, тяжелее или легче фальшивая монета настоящей. Покажите, что за~3~взвешивания на~чашечных весах без~гирь можно определить все фальшивые монеты.

% \end{document}

\begin{document}
\noanswers

\setdate{5~марта~2021}
\setclass{10«АБ»}

\addpersonalvariant{Михаил Бурмистров}

\tasknumber{1}%
\task{%
    Из уравнения состояния идеального газа выведите или выразите...
    \begin{enumerate}
        \item объём,
        \item молярную массу,
        \item концентрацию молекул газа.
    \end{enumerate}
}
\solutionspace{40pt}

\tasknumber{2}%
\task{%
    Изобразите в координатах $PV$, соблюдая масштаб, процесс 1234,
    в котором 12 — изохорическое нагревание в 2 раза,
    23 — изотермическое сжатие в 2 раза,
    34 — изобарическое нагревание в 2 раза.
}
\solutionspace{160pt}

\tasknumber{3}%
\task{%
    Небольшую цилиндрическую пробирку с воздухом погружают на некоторую глубину в глубокое пресное озеро,
    после чего воздух занимает в ней лишь третью часть от общего объема.
    Определите глубину, на которую погрузили пробирку.
    Температуру считать постоянной $T = 288\,\text{К}$, давлением паров воды пренебречь,
    атмосферное давление принять равным $p_{\text{aтм}} = 100\,\text{кПа}$.
}
\answer{%
    \begin{align*}
    T\text{ — const } &\implies P_1V_1 = \nu RT = P_2V_2.
    \\
    V_2 = \frac 13 V_1 &\implies P_1V_1 = P_2\cdot \frac 13V_1 \implies P_2 = 3P_1 = 3p_{\text{aтм}}.
    \\
    P_2 = p_{\text{aтм}} + \rho_{\text{в}} g h \implies h = \frac{ P_2 - p_{\text{aтм}} }{ \rho_{\text{в}} g } &= \frac{ 3p_{\text{aтм}} - p_{\text{aтм}} }{ \rho_{\text{в}} g } = \frac{ 2 \cdot p_{\text{aтм}} }{ \rho_{\text{в}} g } =  \\
     &= \frac{ 2 \cdot 100\,\text{кПа} }{ 1000\,\frac{\text{кг}}{\text{м}^{3}} \cdot 10\,\frac{\text{м}}{\text{с}^{2}} } \approx 20\,\text{м}.
    \end{align*}
}
\solutionspace{120pt}

\tasknumber{4}%
\task{%
    В замкнутом сосуде объёмом $4\,\text{л}$ находится неон ($\mu = 20\,\frac{\text{г}}{\text{моль}}$) под давлением $4{,}5\units{ атм }$.
    Определите массу газа в сосуде и выразите её в граммах, приняв температуру газа равной $27\celsius$.
}
\answer{%
    $
        PV = \frac m\mu RT \implies m = \frac{ PV \mu }{ RT } =
        \frac{ 4{,}5\,\text{атм} \cdot4\,\text{л} \cdot 20\,\frac{\text{г}}{\text{моль}} }{ 8{,}31\,\frac{\text{Дж}}{\text{моль}\cdot\text{К}} \cdot \cbr{ 27 + 273 }\units{К} }
        \approx 14{,}44\,\text{г}.
    $
}
\solutionspace{120pt}

\tasknumber{5}%
\task{%
    Идеальный газ в экспериментальной установке подвергут политропному процессу $PV^n\text{ — const }$
    с показателем политропы $n=1{,}8$.
    В одном из экспериментов объём газа уменьшился в $3$ раза.
    Как при этом изменилась температура газа (выросла или уменьшилась, на сколько или во сколько раз)?
}
\answer{%
    \begin{align*}
    P_1V_1^n &= P_2V_2^n, P_1V_1 = \nu R T_1, P_2V_2 = \nu R T_2 \implies\frac{ \nu R T_1 }{ V_1 } V_1^n = \frac{ \nu R T_2 }{ V_2 } V_2^n \implies \\
    \implies T_1V_1^{ n-1 } &= T_2V_2^{ n - 1 } \implies\frac{ T_2 }{ T_1 } = \cbr{ \frac{ V_1 }{ V_2 } }^{ n-1 } \approx 2{,}408
    \end{align*}
}

\variantsplitter

\addpersonalvariant{Ирина Ан}

\tasknumber{1}%
\task{%
    Из уравнения состояния идеального газа выведите или выразите...
    \begin{enumerate}
        \item давление,
        \item молярную массу,
        \item концентрацию молекул газа.
    \end{enumerate}
}
\solutionspace{40pt}

\tasknumber{2}%
\task{%
    Изобразите в координатах $PV$, соблюдая масштаб, процесс 1234,
    в котором 12 — изобарическое охлаждение в 3 раза,
    23 — изотермическое расширение в 3 раза,
    34 — изохорическое нагревание в 2 раза.
}
\solutionspace{160pt}

\tasknumber{3}%
\task{%
    Небольшую цилиндрическую пробирку с воздухом погружают на некоторую глубину в глубокое пресное озеро,
    после чего воздух занимает в ней лишь третью часть от общего объема.
    Определите глубину, на которую погрузили пробирку.
    Температуру считать постоянной $T = 280\,\text{К}$, давлением паров воды пренебречь,
    атмосферное давление принять равным $p_{\text{aтм}} = 100\,\text{кПа}$.
}
\answer{%
    \begin{align*}
    T\text{ — const } &\implies P_1V_1 = \nu RT = P_2V_2.
    \\
    V_2 = \frac 13 V_1 &\implies P_1V_1 = P_2\cdot \frac 13V_1 \implies P_2 = 3P_1 = 3p_{\text{aтм}}.
    \\
    P_2 = p_{\text{aтм}} + \rho_{\text{в}} g h \implies h = \frac{ P_2 - p_{\text{aтм}} }{ \rho_{\text{в}} g } &= \frac{ 3p_{\text{aтм}} - p_{\text{aтм}} }{ \rho_{\text{в}} g } = \frac{ 2 \cdot p_{\text{aтм}} }{ \rho_{\text{в}} g } =  \\
     &= \frac{ 2 \cdot 100\,\text{кПа} }{ 1000\,\frac{\text{кг}}{\text{м}^{3}} \cdot 10\,\frac{\text{м}}{\text{с}^{2}} } \approx 20\,\text{м}.
    \end{align*}
}
\solutionspace{120pt}

\tasknumber{4}%
\task{%
    В замкнутом сосуде объёмом $2\,\text{л}$ находится азот ($\mu = 28\,\frac{\text{г}}{\text{моль}}$) под давлением $4\units{ атм }$.
    Определите массу газа в сосуде и выразите её в граммах, приняв температуру газа равной $7\celsius$.
}
\answer{%
    $
        PV = \frac m\mu RT \implies m = \frac{ PV \mu }{ RT } =
        \frac{ 4{,}0\,\text{атм} \cdot2\,\text{л} \cdot 28\,\frac{\text{г}}{\text{моль}} }{ 8{,}31\,\frac{\text{Дж}}{\text{моль}\cdot\text{К}} \cdot \cbr{ 7 + 273 }\units{К} }
        \approx 9{,}63\,\text{г}.
    $
}
\solutionspace{120pt}

\tasknumber{5}%
\task{%
    Идеальный газ в экспериментальной установке подвергут политропному процессу $PV^n\text{ — const }$
    с показателем политропы $n=1{,}8$.
    В одном из экспериментов объём газа увеличился в $4$ раза.
    Как при этом изменилась температура газа (выросла или уменьшилась, на сколько или во сколько раз)?
}
\answer{%
    \begin{align*}
    P_1V_1^n &= P_2V_2^n, P_1V_1 = \nu R T_1, P_2V_2 = \nu R T_2 \implies\frac{ \nu R T_1 }{ V_1 } V_1^n = \frac{ \nu R T_2 }{ V_2 } V_2^n \implies \\
    \implies T_1V_1^{ n-1 } &= T_2V_2^{ n - 1 } \implies\frac{ T_2 }{ T_1 } = \cbr{ \frac{ V_1 }{ V_2 } }^{ n-1 } \approx 0{,}330
    \end{align*}
}

\variantsplitter

\addpersonalvariant{Софья Андрианова}

\tasknumber{1}%
\task{%
    Из уравнения состояния идеального газа выведите или выразите...
    \begin{enumerate}
        \item давление,
        \item температуру,
        \item концентрацию молекул газа.
    \end{enumerate}
}
\solutionspace{40pt}

\tasknumber{2}%
\task{%
    Изобразите в координатах $PV$, соблюдая масштаб, процесс 1234,
    в котором 12 — изобарическое охлаждение в 2 раза,
    23 — изотермическое сжатие в 2 раза,
    34 — изохорическое охлаждение в 2 раза.
}
\solutionspace{160pt}

\tasknumber{3}%
\task{%
    Небольшую цилиндрическую пробирку с воздухом погружают на некоторую глубину в глубокое пресное озеро,
    после чего воздух занимает в ней лишь шестую часть от общего объема.
    Определите глубину, на которую погрузили пробирку.
    Температуру считать постоянной $T = 290\,\text{К}$, давлением паров воды пренебречь,
    атмосферное давление принять равным $p_{\text{aтм}} = 100\,\text{кПа}$.
}
\answer{%
    \begin{align*}
    T\text{ — const } &\implies P_1V_1 = \nu RT = P_2V_2.
    \\
    V_2 = \frac 16 V_1 &\implies P_1V_1 = P_2\cdot \frac 16V_1 \implies P_2 = 6P_1 = 6p_{\text{aтм}}.
    \\
    P_2 = p_{\text{aтм}} + \rho_{\text{в}} g h \implies h = \frac{ P_2 - p_{\text{aтм}} }{ \rho_{\text{в}} g } &= \frac{ 6p_{\text{aтм}} - p_{\text{aтм}} }{ \rho_{\text{в}} g } = \frac{ 5 \cdot p_{\text{aтм}} }{ \rho_{\text{в}} g } =  \\
     &= \frac{ 5 \cdot 100\,\text{кПа} }{ 1000\,\frac{\text{кг}}{\text{м}^{3}} \cdot 10\,\frac{\text{м}}{\text{с}^{2}} } \approx 50\,\text{м}.
    \end{align*}
}
\solutionspace{120pt}

\tasknumber{4}%
\task{%
    В замкнутом сосуде объёмом $5\,\text{л}$ находится аргон ($\mu = 40\,\frac{\text{г}}{\text{моль}}$) под давлением $2{,}5\units{ атм }$.
    Определите массу газа в сосуде и выразите её в граммах, приняв температуру газа равной $47\celsius$.
}
\answer{%
    $
        PV = \frac m\mu RT \implies m = \frac{ PV \mu }{ RT } =
        \frac{ 2{,}5\,\text{атм} \cdot5\,\text{л} \cdot 40\,\frac{\text{г}}{\text{моль}} }{ 8{,}31\,\frac{\text{Дж}}{\text{моль}\cdot\text{К}} \cdot \cbr{ 47 + 273 }\units{К} }
        \approx 18{,}80\,\text{г}.
    $
}
\solutionspace{120pt}

\tasknumber{5}%
\task{%
    Идеальный газ в экспериментальной установке подвергут политропному процессу $PV^n\text{ — const }$
    с показателем политропы $n=1{,}8$.
    В одном из экспериментов объём газа увеличился в $4$ раза.
    Как при этом изменилась температура газа (выросла или уменьшилась, на сколько или во сколько раз)?
}
\answer{%
    \begin{align*}
    P_1V_1^n &= P_2V_2^n, P_1V_1 = \nu R T_1, P_2V_2 = \nu R T_2 \implies\frac{ \nu R T_1 }{ V_1 } V_1^n = \frac{ \nu R T_2 }{ V_2 } V_2^n \implies \\
    \implies T_1V_1^{ n-1 } &= T_2V_2^{ n - 1 } \implies\frac{ T_2 }{ T_1 } = \cbr{ \frac{ V_1 }{ V_2 } }^{ n-1 } \approx 0{,}330
    \end{align*}
}

\variantsplitter

\addpersonalvariant{Владимир Артемчук}

\tasknumber{1}%
\task{%
    Из уравнения состояния идеального газа выведите или выразите...
    \begin{enumerate}
        \item объём,
        \item молярную массу,
        \item концентрацию молекул газа.
    \end{enumerate}
}
\solutionspace{40pt}

\tasknumber{2}%
\task{%
    Изобразите в координатах $PV$, соблюдая масштаб, процесс 1234,
    в котором 12 — изобарическое нагревание в 3 раза,
    23 — изотермическое сжатие в 3 раза,
    34 — изобарическое нагревание в 3 раза.
}
\solutionspace{160pt}

\tasknumber{3}%
\task{%
    Небольшую цилиндрическую пробирку с воздухом погружают на некоторую глубину в глубокое пресное озеро,
    после чего воздух занимает в ней лишь третью часть от общего объема.
    Определите глубину, на которую погрузили пробирку.
    Температуру считать постоянной $T = 291\,\text{К}$, давлением паров воды пренебречь,
    атмосферное давление принять равным $p_{\text{aтм}} = 100\,\text{кПа}$.
}
\answer{%
    \begin{align*}
    T\text{ — const } &\implies P_1V_1 = \nu RT = P_2V_2.
    \\
    V_2 = \frac 13 V_1 &\implies P_1V_1 = P_2\cdot \frac 13V_1 \implies P_2 = 3P_1 = 3p_{\text{aтм}}.
    \\
    P_2 = p_{\text{aтм}} + \rho_{\text{в}} g h \implies h = \frac{ P_2 - p_{\text{aтм}} }{ \rho_{\text{в}} g } &= \frac{ 3p_{\text{aтм}} - p_{\text{aтм}} }{ \rho_{\text{в}} g } = \frac{ 2 \cdot p_{\text{aтм}} }{ \rho_{\text{в}} g } =  \\
     &= \frac{ 2 \cdot 100\,\text{кПа} }{ 1000\,\frac{\text{кг}}{\text{м}^{3}} \cdot 10\,\frac{\text{м}}{\text{с}^{2}} } \approx 20\,\text{м}.
    \end{align*}
}
\solutionspace{120pt}

\tasknumber{4}%
\task{%
    В замкнутом сосуде объёмом $4\,\text{л}$ находится кислород ($\mu = 32\,\frac{\text{г}}{\text{моль}}$) под давлением $3\units{ атм }$.
    Определите массу газа в сосуде и выразите её в граммах, приняв температуру газа равной $7\celsius$.
}
\answer{%
    $
        PV = \frac m\mu RT \implies m = \frac{ PV \mu }{ RT } =
        \frac{ 3{,}0\,\text{атм} \cdot4\,\text{л} \cdot 32\,\frac{\text{г}}{\text{моль}} }{ 8{,}31\,\frac{\text{Дж}}{\text{моль}\cdot\text{К}} \cdot \cbr{ 7 + 273 }\units{К} }
        \approx 16{,}50\,\text{г}.
    $
}
\solutionspace{120pt}

\tasknumber{5}%
\task{%
    Идеальный газ в экспериментальной установке подвергут политропному процессу $PV^n\text{ — const }$
    с показателем политропы $n=1{,}5$.
    В одном из экспериментов объём газа увеличился в $2$ раза.
    Как при этом изменилась температура газа (выросла или уменьшилась, на сколько или во сколько раз)?
}
\answer{%
    \begin{align*}
    P_1V_1^n &= P_2V_2^n, P_1V_1 = \nu R T_1, P_2V_2 = \nu R T_2 \implies\frac{ \nu R T_1 }{ V_1 } V_1^n = \frac{ \nu R T_2 }{ V_2 } V_2^n \implies \\
    \implies T_1V_1^{ n-1 } &= T_2V_2^{ n - 1 } \implies\frac{ T_2 }{ T_1 } = \cbr{ \frac{ V_1 }{ V_2 } }^{ n-1 } \approx 0{,}707
    \end{align*}
}

\variantsplitter

\addpersonalvariant{Софья Белянкина}

\tasknumber{1}%
\task{%
    Из уравнения состояния идеального газа выведите или выразите...
    \begin{enumerate}
        \item объём,
        \item молярную массу,
        \item плотность газа.
    \end{enumerate}
}
\solutionspace{40pt}

\tasknumber{2}%
\task{%
    Изобразите в координатах $PV$, соблюдая масштаб, процесс 1234,
    в котором 12 — изобарическое нагревание в 2 раза,
    23 — изотермическое сжатие в 2 раза,
    34 — изобарическое охлаждение в 2 раза.
}
\solutionspace{160pt}

\tasknumber{3}%
\task{%
    Небольшую цилиндрическую пробирку с воздухом погружают на некоторую глубину в глубокое пресное озеро,
    после чего воздух занимает в ней лишь третью часть от общего объема.
    Определите глубину, на которую погрузили пробирку.
    Температуру считать постоянной $T = 286\,\text{К}$, давлением паров воды пренебречь,
    атмосферное давление принять равным $p_{\text{aтм}} = 100\,\text{кПа}$.
}
\answer{%
    \begin{align*}
    T\text{ — const } &\implies P_1V_1 = \nu RT = P_2V_2.
    \\
    V_2 = \frac 13 V_1 &\implies P_1V_1 = P_2\cdot \frac 13V_1 \implies P_2 = 3P_1 = 3p_{\text{aтм}}.
    \\
    P_2 = p_{\text{aтм}} + \rho_{\text{в}} g h \implies h = \frac{ P_2 - p_{\text{aтм}} }{ \rho_{\text{в}} g } &= \frac{ 3p_{\text{aтм}} - p_{\text{aтм}} }{ \rho_{\text{в}} g } = \frac{ 2 \cdot p_{\text{aтм}} }{ \rho_{\text{в}} g } =  \\
     &= \frac{ 2 \cdot 100\,\text{кПа} }{ 1000\,\frac{\text{кг}}{\text{м}^{3}} \cdot 10\,\frac{\text{м}}{\text{с}^{2}} } \approx 20\,\text{м}.
    \end{align*}
}
\solutionspace{120pt}

\tasknumber{4}%
\task{%
    В замкнутом сосуде объёмом $3\,\text{л}$ находится неон ($\mu = 20\,\frac{\text{г}}{\text{моль}}$) под давлением $2{,}5\units{ атм }$.
    Определите массу газа в сосуде и выразите её в граммах, приняв температуру газа равной $37\celsius$.
}
\answer{%
    $
        PV = \frac m\mu RT \implies m = \frac{ PV \mu }{ RT } =
        \frac{ 2{,}5\,\text{атм} \cdot3\,\text{л} \cdot 20\,\frac{\text{г}}{\text{моль}} }{ 8{,}31\,\frac{\text{Дж}}{\text{моль}\cdot\text{К}} \cdot \cbr{ 37 + 273 }\units{К} }
        \approx 5{,}82\,\text{г}.
    $
}
\solutionspace{120pt}

\tasknumber{5}%
\task{%
    Идеальный газ в экспериментальной установке подвергут политропному процессу $PV^n\text{ — const }$
    с показателем политропы $n=1{,}2$.
    В одном из экспериментов объём газа увеличился в $2$ раза.
    Как при этом изменилась температура газа (выросла или уменьшилась, на сколько или во сколько раз)?
}
\answer{%
    \begin{align*}
    P_1V_1^n &= P_2V_2^n, P_1V_1 = \nu R T_1, P_2V_2 = \nu R T_2 \implies\frac{ \nu R T_1 }{ V_1 } V_1^n = \frac{ \nu R T_2 }{ V_2 } V_2^n \implies \\
    \implies T_1V_1^{ n-1 } &= T_2V_2^{ n - 1 } \implies\frac{ T_2 }{ T_1 } = \cbr{ \frac{ V_1 }{ V_2 } }^{ n-1 } \approx 0{,}871
    \end{align*}
}

\variantsplitter

\addpersonalvariant{Варвара Егиазарян}

\tasknumber{1}%
\task{%
    Из уравнения состояния идеального газа выведите или выразите...
    \begin{enumerate}
        \item объём,
        \item молярную массу,
        \item концентрацию молекул газа.
    \end{enumerate}
}
\solutionspace{40pt}

\tasknumber{2}%
\task{%
    Изобразите в координатах $PV$, соблюдая масштаб, процесс 1234,
    в котором 12 — изохорическое охлаждение в 2 раза,
    23 — изотермическое сжатие в 2 раза,
    34 — изохорическое охлаждение в 3 раза.
}
\solutionspace{160pt}

\tasknumber{3}%
\task{%
    Небольшую цилиндрическую пробирку с воздухом погружают на некоторую глубину в глубокое пресное озеро,
    после чего воздух занимает в ней лишь третью часть от общего объема.
    Определите глубину, на которую погрузили пробирку.
    Температуру считать постоянной $T = 281\,\text{К}$, давлением паров воды пренебречь,
    атмосферное давление принять равным $p_{\text{aтм}} = 100\,\text{кПа}$.
}
\answer{%
    \begin{align*}
    T\text{ — const } &\implies P_1V_1 = \nu RT = P_2V_2.
    \\
    V_2 = \frac 13 V_1 &\implies P_1V_1 = P_2\cdot \frac 13V_1 \implies P_2 = 3P_1 = 3p_{\text{aтм}}.
    \\
    P_2 = p_{\text{aтм}} + \rho_{\text{в}} g h \implies h = \frac{ P_2 - p_{\text{aтм}} }{ \rho_{\text{в}} g } &= \frac{ 3p_{\text{aтм}} - p_{\text{aтм}} }{ \rho_{\text{в}} g } = \frac{ 2 \cdot p_{\text{aтм}} }{ \rho_{\text{в}} g } =  \\
     &= \frac{ 2 \cdot 100\,\text{кПа} }{ 1000\,\frac{\text{кг}}{\text{м}^{3}} \cdot 10\,\frac{\text{м}}{\text{с}^{2}} } \approx 20\,\text{м}.
    \end{align*}
}
\solutionspace{120pt}

\tasknumber{4}%
\task{%
    В замкнутом сосуде объёмом $5\,\text{л}$ находится аргон ($\mu = 40\,\frac{\text{г}}{\text{моль}}$) под давлением $2{,}5\units{ атм }$.
    Определите массу газа в сосуде и выразите её в граммах, приняв температуру газа равной $7\celsius$.
}
\answer{%
    $
        PV = \frac m\mu RT \implies m = \frac{ PV \mu }{ RT } =
        \frac{ 2{,}5\,\text{атм} \cdot5\,\text{л} \cdot 40\,\frac{\text{г}}{\text{моль}} }{ 8{,}31\,\frac{\text{Дж}}{\text{моль}\cdot\text{К}} \cdot \cbr{ 7 + 273 }\units{К} }
        \approx 21{,}49\,\text{г}.
    $
}
\solutionspace{120pt}

\tasknumber{5}%
\task{%
    Идеальный газ в экспериментальной установке подвергут политропному процессу $PV^n\text{ — const }$
    с показателем политропы $n=1{,}2$.
    В одном из экспериментов объём газа увеличился в $2$ раза.
    Как при этом изменилась температура газа (выросла или уменьшилась, на сколько или во сколько раз)?
}
\answer{%
    \begin{align*}
    P_1V_1^n &= P_2V_2^n, P_1V_1 = \nu R T_1, P_2V_2 = \nu R T_2 \implies\frac{ \nu R T_1 }{ V_1 } V_1^n = \frac{ \nu R T_2 }{ V_2 } V_2^n \implies \\
    \implies T_1V_1^{ n-1 } &= T_2V_2^{ n - 1 } \implies\frac{ T_2 }{ T_1 } = \cbr{ \frac{ V_1 }{ V_2 } }^{ n-1 } \approx 0{,}871
    \end{align*}
}

\variantsplitter

\addpersonalvariant{Владислав Емелин}

\tasknumber{1}%
\task{%
    Из уравнения состояния идеального газа выведите или выразите...
    \begin{enumerate}
        \item объём,
        \item молярную массу,
        \item концентрацию молекул газа.
    \end{enumerate}
}
\solutionspace{40pt}

\tasknumber{2}%
\task{%
    Изобразите в координатах $PV$, соблюдая масштаб, процесс 1234,
    в котором 12 — изобарическое охлаждение в 3 раза,
    23 — изотермическое расширение в 3 раза,
    34 — изохорическое охлаждение в 3 раза.
}
\solutionspace{160pt}

\tasknumber{3}%
\task{%
    Небольшую цилиндрическую пробирку с воздухом погружают на некоторую глубину в глубокое пресное озеро,
    после чего воздух занимает в ней лишь третью часть от общего объема.
    Определите глубину, на которую погрузили пробирку.
    Температуру считать постоянной $T = 290\,\text{К}$, давлением паров воды пренебречь,
    атмосферное давление принять равным $p_{\text{aтм}} = 100\,\text{кПа}$.
}
\answer{%
    \begin{align*}
    T\text{ — const } &\implies P_1V_1 = \nu RT = P_2V_2.
    \\
    V_2 = \frac 13 V_1 &\implies P_1V_1 = P_2\cdot \frac 13V_1 \implies P_2 = 3P_1 = 3p_{\text{aтм}}.
    \\
    P_2 = p_{\text{aтм}} + \rho_{\text{в}} g h \implies h = \frac{ P_2 - p_{\text{aтм}} }{ \rho_{\text{в}} g } &= \frac{ 3p_{\text{aтм}} - p_{\text{aтм}} }{ \rho_{\text{в}} g } = \frac{ 2 \cdot p_{\text{aтм}} }{ \rho_{\text{в}} g } =  \\
     &= \frac{ 2 \cdot 100\,\text{кПа} }{ 1000\,\frac{\text{кг}}{\text{м}^{3}} \cdot 10\,\frac{\text{м}}{\text{с}^{2}} } \approx 20\,\text{м}.
    \end{align*}
}
\solutionspace{120pt}

\tasknumber{4}%
\task{%
    В замкнутом сосуде объёмом $4\,\text{л}$ находится кислород ($\mu = 32\,\frac{\text{г}}{\text{моль}}$) под давлением $3{,}5\units{ атм }$.
    Определите массу газа в сосуде и выразите её в граммах, приняв температуру газа равной $37\celsius$.
}
\answer{%
    $
        PV = \frac m\mu RT \implies m = \frac{ PV \mu }{ RT } =
        \frac{ 3{,}5\,\text{атм} \cdot4\,\text{л} \cdot 32\,\frac{\text{г}}{\text{моль}} }{ 8{,}31\,\frac{\text{Дж}}{\text{моль}\cdot\text{К}} \cdot \cbr{ 37 + 273 }\units{К} }
        \approx 17{,}39\,\text{г}.
    $
}
\solutionspace{120pt}

\tasknumber{5}%
\task{%
    Идеальный газ в экспериментальной установке подвергут политропному процессу $PV^n\text{ — const }$
    с показателем политропы $n=0{,}4$.
    В одном из экспериментов объём газа уменьшился в $2$ раза.
    Как при этом изменилась температура газа (выросла или уменьшилась, на сколько или во сколько раз)?
}
\answer{%
    \begin{align*}
    P_1V_1^n &= P_2V_2^n, P_1V_1 = \nu R T_1, P_2V_2 = \nu R T_2 \implies\frac{ \nu R T_1 }{ V_1 } V_1^n = \frac{ \nu R T_2 }{ V_2 } V_2^n \implies \\
    \implies T_1V_1^{ n-1 } &= T_2V_2^{ n - 1 } \implies\frac{ T_2 }{ T_1 } = \cbr{ \frac{ V_1 }{ V_2 } }^{ n-1 } \approx 0{,}660
    \end{align*}
}

\variantsplitter

\addpersonalvariant{Артём Жичин}

\tasknumber{1}%
\task{%
    Из уравнения состояния идеального газа выведите или выразите...
    \begin{enumerate}
        \item давление,
        \item молярную массу,
        \item концентрацию молекул газа.
    \end{enumerate}
}
\solutionspace{40pt}

\tasknumber{2}%
\task{%
    Изобразите в координатах $PV$, соблюдая масштаб, процесс 1234,
    в котором 12 — изохорическое нагревание в 2 раза,
    23 — изотермическое сжатие в 2 раза,
    34 — изобарическое охлаждение в 2 раза.
}
\solutionspace{160pt}

\tasknumber{3}%
\task{%
    Небольшую цилиндрическую пробирку с воздухом погружают на некоторую глубину в глубокое пресное озеро,
    после чего воздух занимает в ней лишь пятую часть от общего объема.
    Определите глубину, на которую погрузили пробирку.
    Температуру считать постоянной $T = 289\,\text{К}$, давлением паров воды пренебречь,
    атмосферное давление принять равным $p_{\text{aтм}} = 100\,\text{кПа}$.
}
\answer{%
    \begin{align*}
    T\text{ — const } &\implies P_1V_1 = \nu RT = P_2V_2.
    \\
    V_2 = \frac 15 V_1 &\implies P_1V_1 = P_2\cdot \frac 15V_1 \implies P_2 = 5P_1 = 5p_{\text{aтм}}.
    \\
    P_2 = p_{\text{aтм}} + \rho_{\text{в}} g h \implies h = \frac{ P_2 - p_{\text{aтм}} }{ \rho_{\text{в}} g } &= \frac{ 5p_{\text{aтм}} - p_{\text{aтм}} }{ \rho_{\text{в}} g } = \frac{ 4 \cdot p_{\text{aтм}} }{ \rho_{\text{в}} g } =  \\
     &= \frac{ 4 \cdot 100\,\text{кПа} }{ 1000\,\frac{\text{кг}}{\text{м}^{3}} \cdot 10\,\frac{\text{м}}{\text{с}^{2}} } \approx 40\,\text{м}.
    \end{align*}
}
\solutionspace{120pt}

\tasknumber{4}%
\task{%
    В замкнутом сосуде объёмом $3\,\text{л}$ находится воздух ($\mu = 29\,\frac{\text{г}}{\text{моль}}$) под давлением $2{,}5\units{ атм }$.
    Определите массу газа в сосуде и выразите её в граммах, приняв температуру газа равной $37\celsius$.
}
\answer{%
    $
        PV = \frac m\mu RT \implies m = \frac{ PV \mu }{ RT } =
        \frac{ 2{,}5\,\text{атм} \cdot3\,\text{л} \cdot 29\,\frac{\text{г}}{\text{моль}} }{ 8{,}31\,\frac{\text{Дж}}{\text{моль}\cdot\text{К}} \cdot \cbr{ 37 + 273 }\units{К} }
        \approx 8{,}44\,\text{г}.
    $
}
\solutionspace{120pt}

\tasknumber{5}%
\task{%
    Идеальный газ в экспериментальной установке подвергут политропному процессу $PV^n\text{ — const }$
    с показателем политропы $n=1{,}5$.
    В одном из экспериментов объём газа уменьшился в $2$ раза.
    Как при этом изменилась температура газа (выросла или уменьшилась, на сколько или во сколько раз)?
}
\answer{%
    \begin{align*}
    P_1V_1^n &= P_2V_2^n, P_1V_1 = \nu R T_1, P_2V_2 = \nu R T_2 \implies\frac{ \nu R T_1 }{ V_1 } V_1^n = \frac{ \nu R T_2 }{ V_2 } V_2^n \implies \\
    \implies T_1V_1^{ n-1 } &= T_2V_2^{ n - 1 } \implies\frac{ T_2 }{ T_1 } = \cbr{ \frac{ V_1 }{ V_2 } }^{ n-1 } \approx 1{,}414
    \end{align*}
}

\variantsplitter

\addpersonalvariant{Дарья Кошман}

\tasknumber{1}%
\task{%
    Из уравнения состояния идеального газа выведите или выразите...
    \begin{enumerate}
        \item объём,
        \item температуру,
        \item плотность газа.
    \end{enumerate}
}
\solutionspace{40pt}

\tasknumber{2}%
\task{%
    Изобразите в координатах $PV$, соблюдая масштаб, процесс 1234,
    в котором 12 — изобарическое охлаждение в 3 раза,
    23 — изотермическое расширение в 3 раза,
    34 — изохорическое охлаждение в 3 раза.
}
\solutionspace{160pt}

\tasknumber{3}%
\task{%
    Небольшую цилиндрическую пробирку с воздухом погружают на некоторую глубину в глубокое пресное озеро,
    после чего воздух занимает в ней лишь шестую часть от общего объема.
    Определите глубину, на которую погрузили пробирку.
    Температуру считать постоянной $T = 284\,\text{К}$, давлением паров воды пренебречь,
    атмосферное давление принять равным $p_{\text{aтм}} = 100\,\text{кПа}$.
}
\answer{%
    \begin{align*}
    T\text{ — const } &\implies P_1V_1 = \nu RT = P_2V_2.
    \\
    V_2 = \frac 16 V_1 &\implies P_1V_1 = P_2\cdot \frac 16V_1 \implies P_2 = 6P_1 = 6p_{\text{aтм}}.
    \\
    P_2 = p_{\text{aтм}} + \rho_{\text{в}} g h \implies h = \frac{ P_2 - p_{\text{aтм}} }{ \rho_{\text{в}} g } &= \frac{ 6p_{\text{aтм}} - p_{\text{aтм}} }{ \rho_{\text{в}} g } = \frac{ 5 \cdot p_{\text{aтм}} }{ \rho_{\text{в}} g } =  \\
     &= \frac{ 5 \cdot 100\,\text{кПа} }{ 1000\,\frac{\text{кг}}{\text{м}^{3}} \cdot 10\,\frac{\text{м}}{\text{с}^{2}} } \approx 50\,\text{м}.
    \end{align*}
}
\solutionspace{120pt}

\tasknumber{4}%
\task{%
    В замкнутом сосуде объёмом $4\,\text{л}$ находится азот ($\mu = 28\,\frac{\text{г}}{\text{моль}}$) под давлением $4{,}5\units{ атм }$.
    Определите массу газа в сосуде и выразите её в граммах, приняв температуру газа равной $27\celsius$.
}
\answer{%
    $
        PV = \frac m\mu RT \implies m = \frac{ PV \mu }{ RT } =
        \frac{ 4{,}5\,\text{атм} \cdot4\,\text{л} \cdot 28\,\frac{\text{г}}{\text{моль}} }{ 8{,}31\,\frac{\text{Дж}}{\text{моль}\cdot\text{К}} \cdot \cbr{ 27 + 273 }\units{К} }
        \approx 20{,}22\,\text{г}.
    $
}
\solutionspace{120pt}

\tasknumber{5}%
\task{%
    Идеальный газ в экспериментальной установке подвергут политропному процессу $PV^n\text{ — const }$
    с показателем политропы $n=0{,}7$.
    В одном из экспериментов объём газа увеличился в $4$ раза.
    Как при этом изменилась температура газа (выросла или уменьшилась, на сколько или во сколько раз)?
}
\answer{%
    \begin{align*}
    P_1V_1^n &= P_2V_2^n, P_1V_1 = \nu R T_1, P_2V_2 = \nu R T_2 \implies\frac{ \nu R T_1 }{ V_1 } V_1^n = \frac{ \nu R T_2 }{ V_2 } V_2^n \implies \\
    \implies T_1V_1^{ n-1 } &= T_2V_2^{ n - 1 } \implies\frac{ T_2 }{ T_1 } = \cbr{ \frac{ V_1 }{ V_2 } }^{ n-1 } \approx 1{,}516
    \end{align*}
}

\variantsplitter

\addpersonalvariant{Анна Кузьмичёва}

\tasknumber{1}%
\task{%
    Из уравнения состояния идеального газа выведите или выразите...
    \begin{enumerate}
        \item объём,
        \item температуру,
        \item плотность газа.
    \end{enumerate}
}
\solutionspace{40pt}

\tasknumber{2}%
\task{%
    Изобразите в координатах $PV$, соблюдая масштаб, процесс 1234,
    в котором 12 — изохорическое нагревание в 3 раза,
    23 — изотермическое расширение в 2 раза,
    34 — изобарическое охлаждение в 3 раза.
}
\solutionspace{160pt}

\tasknumber{3}%
\task{%
    Небольшую цилиндрическую пробирку с воздухом погружают на некоторую глубину в глубокое пресное озеро,
    после чего воздух занимает в ней лишь пятую часть от общего объема.
    Определите глубину, на которую погрузили пробирку.
    Температуру считать постоянной $T = 284\,\text{К}$, давлением паров воды пренебречь,
    атмосферное давление принять равным $p_{\text{aтм}} = 100\,\text{кПа}$.
}
\answer{%
    \begin{align*}
    T\text{ — const } &\implies P_1V_1 = \nu RT = P_2V_2.
    \\
    V_2 = \frac 15 V_1 &\implies P_1V_1 = P_2\cdot \frac 15V_1 \implies P_2 = 5P_1 = 5p_{\text{aтм}}.
    \\
    P_2 = p_{\text{aтм}} + \rho_{\text{в}} g h \implies h = \frac{ P_2 - p_{\text{aтм}} }{ \rho_{\text{в}} g } &= \frac{ 5p_{\text{aтм}} - p_{\text{aтм}} }{ \rho_{\text{в}} g } = \frac{ 4 \cdot p_{\text{aтм}} }{ \rho_{\text{в}} g } =  \\
     &= \frac{ 4 \cdot 100\,\text{кПа} }{ 1000\,\frac{\text{кг}}{\text{м}^{3}} \cdot 10\,\frac{\text{м}}{\text{с}^{2}} } \approx 40\,\text{м}.
    \end{align*}
}
\solutionspace{120pt}

\tasknumber{4}%
\task{%
    В замкнутом сосуде объёмом $2\,\text{л}$ находится воздух ($\mu = 29\,\frac{\text{г}}{\text{моль}}$) под давлением $3\units{ атм }$.
    Определите массу газа в сосуде и выразите её в граммах, приняв температуру газа равной $47\celsius$.
}
\answer{%
    $
        PV = \frac m\mu RT \implies m = \frac{ PV \mu }{ RT } =
        \frac{ 3{,}0\,\text{атм} \cdot2\,\text{л} \cdot 29\,\frac{\text{г}}{\text{моль}} }{ 8{,}31\,\frac{\text{Дж}}{\text{моль}\cdot\text{К}} \cdot \cbr{ 47 + 273 }\units{К} }
        \approx 6{,}54\,\text{г}.
    $
}
\solutionspace{120pt}

\tasknumber{5}%
\task{%
    Идеальный газ в экспериментальной установке подвергут политропному процессу $PV^n\text{ — const }$
    с показателем политропы $n=1{,}8$.
    В одном из экспериментов объём газа увеличился в $3$ раза.
    Как при этом изменилась температура газа (выросла или уменьшилась, на сколько или во сколько раз)?
}
\answer{%
    \begin{align*}
    P_1V_1^n &= P_2V_2^n, P_1V_1 = \nu R T_1, P_2V_2 = \nu R T_2 \implies\frac{ \nu R T_1 }{ V_1 } V_1^n = \frac{ \nu R T_2 }{ V_2 } V_2^n \implies \\
    \implies T_1V_1^{ n-1 } &= T_2V_2^{ n - 1 } \implies\frac{ T_2 }{ T_1 } = \cbr{ \frac{ V_1 }{ V_2 } }^{ n-1 } \approx 0{,}415
    \end{align*}
}

\variantsplitter

\addpersonalvariant{Алёна Куприянова}

\tasknumber{1}%
\task{%
    Из уравнения состояния идеального газа выведите или выразите...
    \begin{enumerate}
        \item давление,
        \item молярную массу,
        \item концентрацию молекул газа.
    \end{enumerate}
}
\solutionspace{40pt}

\tasknumber{2}%
\task{%
    Изобразите в координатах $PV$, соблюдая масштаб, процесс 1234,
    в котором 12 — изохорическое нагревание в 2 раза,
    23 — изотермическое сжатие в 3 раза,
    34 — изобарическое нагревание в 3 раза.
}
\solutionspace{160pt}

\tasknumber{3}%
\task{%
    Небольшую цилиндрическую пробирку с воздухом погружают на некоторую глубину в глубокое пресное озеро,
    после чего воздух занимает в ней лишь шестую часть от общего объема.
    Определите глубину, на которую погрузили пробирку.
    Температуру считать постоянной $T = 285\,\text{К}$, давлением паров воды пренебречь,
    атмосферное давление принять равным $p_{\text{aтм}} = 100\,\text{кПа}$.
}
\answer{%
    \begin{align*}
    T\text{ — const } &\implies P_1V_1 = \nu RT = P_2V_2.
    \\
    V_2 = \frac 16 V_1 &\implies P_1V_1 = P_2\cdot \frac 16V_1 \implies P_2 = 6P_1 = 6p_{\text{aтм}}.
    \\
    P_2 = p_{\text{aтм}} + \rho_{\text{в}} g h \implies h = \frac{ P_2 - p_{\text{aтм}} }{ \rho_{\text{в}} g } &= \frac{ 6p_{\text{aтм}} - p_{\text{aтм}} }{ \rho_{\text{в}} g } = \frac{ 5 \cdot p_{\text{aтм}} }{ \rho_{\text{в}} g } =  \\
     &= \frac{ 5 \cdot 100\,\text{кПа} }{ 1000\,\frac{\text{кг}}{\text{м}^{3}} \cdot 10\,\frac{\text{м}}{\text{с}^{2}} } \approx 50\,\text{м}.
    \end{align*}
}
\solutionspace{120pt}

\tasknumber{4}%
\task{%
    В замкнутом сосуде объёмом $5\,\text{л}$ находится воздух ($\mu = 29\,\frac{\text{г}}{\text{моль}}$) под давлением $3{,}5\units{ атм }$.
    Определите массу газа в сосуде и выразите её в граммах, приняв температуру газа равной $27\celsius$.
}
\answer{%
    $
        PV = \frac m\mu RT \implies m = \frac{ PV \mu }{ RT } =
        \frac{ 3{,}5\,\text{атм} \cdot5\,\text{л} \cdot 29\,\frac{\text{г}}{\text{моль}} }{ 8{,}31\,\frac{\text{Дж}}{\text{моль}\cdot\text{К}} \cdot \cbr{ 27 + 273 }\units{К} }
        \approx 20{,}36\,\text{г}.
    $
}
\solutionspace{120pt}

\tasknumber{5}%
\task{%
    Идеальный газ в экспериментальной установке подвергут политропному процессу $PV^n\text{ — const }$
    с показателем политропы $n=0{,}4$.
    В одном из экспериментов объём газа уменьшился в $3$ раза.
    Как при этом изменилась температура газа (выросла или уменьшилась, на сколько или во сколько раз)?
}
\answer{%
    \begin{align*}
    P_1V_1^n &= P_2V_2^n, P_1V_1 = \nu R T_1, P_2V_2 = \nu R T_2 \implies\frac{ \nu R T_1 }{ V_1 } V_1^n = \frac{ \nu R T_2 }{ V_2 } V_2^n \implies \\
    \implies T_1V_1^{ n-1 } &= T_2V_2^{ n - 1 } \implies\frac{ T_2 }{ T_1 } = \cbr{ \frac{ V_1 }{ V_2 } }^{ n-1 } \approx 0{,}517
    \end{align*}
}

\variantsplitter

\addpersonalvariant{Ярослав Лавровский}

\tasknumber{1}%
\task{%
    Из уравнения состояния идеального газа выведите или выразите...
    \begin{enumerate}
        \item объём,
        \item молярную массу,
        \item плотность газа.
    \end{enumerate}
}
\solutionspace{40pt}

\tasknumber{2}%
\task{%
    Изобразите в координатах $PV$, соблюдая масштаб, процесс 1234,
    в котором 12 — изохорическое нагревание в 2 раза,
    23 — изотермическое расширение в 3 раза,
    34 — изохорическое охлаждение в 2 раза.
}
\solutionspace{160pt}

\tasknumber{3}%
\task{%
    Небольшую цилиндрическую пробирку с воздухом погружают на некоторую глубину в глубокое пресное озеро,
    после чего воздух занимает в ней лишь шестую часть от общего объема.
    Определите глубину, на которую погрузили пробирку.
    Температуру считать постоянной $T = 292\,\text{К}$, давлением паров воды пренебречь,
    атмосферное давление принять равным $p_{\text{aтм}} = 100\,\text{кПа}$.
}
\answer{%
    \begin{align*}
    T\text{ — const } &\implies P_1V_1 = \nu RT = P_2V_2.
    \\
    V_2 = \frac 16 V_1 &\implies P_1V_1 = P_2\cdot \frac 16V_1 \implies P_2 = 6P_1 = 6p_{\text{aтм}}.
    \\
    P_2 = p_{\text{aтм}} + \rho_{\text{в}} g h \implies h = \frac{ P_2 - p_{\text{aтм}} }{ \rho_{\text{в}} g } &= \frac{ 6p_{\text{aтм}} - p_{\text{aтм}} }{ \rho_{\text{в}} g } = \frac{ 5 \cdot p_{\text{aтм}} }{ \rho_{\text{в}} g } =  \\
     &= \frac{ 5 \cdot 100\,\text{кПа} }{ 1000\,\frac{\text{кг}}{\text{м}^{3}} \cdot 10\,\frac{\text{м}}{\text{с}^{2}} } \approx 50\,\text{м}.
    \end{align*}
}
\solutionspace{120pt}

\tasknumber{4}%
\task{%
    В замкнутом сосуде объёмом $3\,\text{л}$ находится воздух ($\mu = 29\,\frac{\text{г}}{\text{моль}}$) под давлением $5\units{ атм }$.
    Определите массу газа в сосуде и выразите её в граммах, приняв температуру газа равной $7\celsius$.
}
\answer{%
    $
        PV = \frac m\mu RT \implies m = \frac{ PV \mu }{ RT } =
        \frac{ 5{,}0\,\text{атм} \cdot3\,\text{л} \cdot 29\,\frac{\text{г}}{\text{моль}} }{ 8{,}31\,\frac{\text{Дж}}{\text{моль}\cdot\text{К}} \cdot \cbr{ 7 + 273 }\units{К} }
        \approx 18{,}70\,\text{г}.
    $
}
\solutionspace{120pt}

\tasknumber{5}%
\task{%
    Идеальный газ в экспериментальной установке подвергут политропному процессу $PV^n\text{ — const }$
    с показателем политропы $n=0{,}4$.
    В одном из экспериментов объём газа увеличился в $2$ раза.
    Как при этом изменилась температура газа (выросла или уменьшилась, на сколько или во сколько раз)?
}
\answer{%
    \begin{align*}
    P_1V_1^n &= P_2V_2^n, P_1V_1 = \nu R T_1, P_2V_2 = \nu R T_2 \implies\frac{ \nu R T_1 }{ V_1 } V_1^n = \frac{ \nu R T_2 }{ V_2 } V_2^n \implies \\
    \implies T_1V_1^{ n-1 } &= T_2V_2^{ n - 1 } \implies\frac{ T_2 }{ T_1 } = \cbr{ \frac{ V_1 }{ V_2 } }^{ n-1 } \approx 1{,}516
    \end{align*}
}

\variantsplitter

\addpersonalvariant{Анастасия Ламанова}

\tasknumber{1}%
\task{%
    Из уравнения состояния идеального газа выведите или выразите...
    \begin{enumerate}
        \item давление,
        \item температуру,
        \item концентрацию молекул газа.
    \end{enumerate}
}
\solutionspace{40pt}

\tasknumber{2}%
\task{%
    Изобразите в координатах $PV$, соблюдая масштаб, процесс 1234,
    в котором 12 — изохорическое охлаждение в 2 раза,
    23 — изотермическое расширение в 3 раза,
    34 — изобарическое охлаждение в 2 раза.
}
\solutionspace{160pt}

\tasknumber{3}%
\task{%
    Небольшую цилиндрическую пробирку с воздухом погружают на некоторую глубину в глубокое пресное озеро,
    после чего воздух занимает в ней лишь четвертую часть от общего объема.
    Определите глубину, на которую погрузили пробирку.
    Температуру считать постоянной $T = 286\,\text{К}$, давлением паров воды пренебречь,
    атмосферное давление принять равным $p_{\text{aтм}} = 100\,\text{кПа}$.
}
\answer{%
    \begin{align*}
    T\text{ — const } &\implies P_1V_1 = \nu RT = P_2V_2.
    \\
    V_2 = \frac 14 V_1 &\implies P_1V_1 = P_2\cdot \frac 14V_1 \implies P_2 = 4P_1 = 4p_{\text{aтм}}.
    \\
    P_2 = p_{\text{aтм}} + \rho_{\text{в}} g h \implies h = \frac{ P_2 - p_{\text{aтм}} }{ \rho_{\text{в}} g } &= \frac{ 4p_{\text{aтм}} - p_{\text{aтм}} }{ \rho_{\text{в}} g } = \frac{ 3 \cdot p_{\text{aтм}} }{ \rho_{\text{в}} g } =  \\
     &= \frac{ 3 \cdot 100\,\text{кПа} }{ 1000\,\frac{\text{кг}}{\text{м}^{3}} \cdot 10\,\frac{\text{м}}{\text{с}^{2}} } \approx 30\,\text{м}.
    \end{align*}
}
\solutionspace{120pt}

\tasknumber{4}%
\task{%
    В замкнутом сосуде объёмом $3\,\text{л}$ находится неон ($\mu = 20\,\frac{\text{г}}{\text{моль}}$) под давлением $3\units{ атм }$.
    Определите массу газа в сосуде и выразите её в граммах, приняв температуру газа равной $17\celsius$.
}
\answer{%
    $
        PV = \frac m\mu RT \implies m = \frac{ PV \mu }{ RT } =
        \frac{ 3{,}0\,\text{атм} \cdot3\,\text{л} \cdot 20\,\frac{\text{г}}{\text{моль}} }{ 8{,}31\,\frac{\text{Дж}}{\text{моль}\cdot\text{К}} \cdot \cbr{ 17 + 273 }\units{К} }
        \approx 7{,}47\,\text{г}.
    $
}
\solutionspace{120pt}

\tasknumber{5}%
\task{%
    Идеальный газ в экспериментальной установке подвергут политропному процессу $PV^n\text{ — const }$
    с показателем политропы $n=1{,}8$.
    В одном из экспериментов объём газа увеличился в $3$ раза.
    Как при этом изменилась температура газа (выросла или уменьшилась, на сколько или во сколько раз)?
}
\answer{%
    \begin{align*}
    P_1V_1^n &= P_2V_2^n, P_1V_1 = \nu R T_1, P_2V_2 = \nu R T_2 \implies\frac{ \nu R T_1 }{ V_1 } V_1^n = \frac{ \nu R T_2 }{ V_2 } V_2^n \implies \\
    \implies T_1V_1^{ n-1 } &= T_2V_2^{ n - 1 } \implies\frac{ T_2 }{ T_1 } = \cbr{ \frac{ V_1 }{ V_2 } }^{ n-1 } \approx 0{,}415
    \end{align*}
}

\variantsplitter

\addpersonalvariant{Виктория Легонькова}

\tasknumber{1}%
\task{%
    Из уравнения состояния идеального газа выведите или выразите...
    \begin{enumerate}
        \item объём,
        \item молярную массу,
        \item концентрацию молекул газа.
    \end{enumerate}
}
\solutionspace{40pt}

\tasknumber{2}%
\task{%
    Изобразите в координатах $PV$, соблюдая масштаб, процесс 1234,
    в котором 12 — изохорическое охлаждение в 2 раза,
    23 — изотермическое сжатие в 3 раза,
    34 — изохорическое нагревание в 2 раза.
}
\solutionspace{160pt}

\tasknumber{3}%
\task{%
    Небольшую цилиндрическую пробирку с воздухом погружают на некоторую глубину в глубокое пресное озеро,
    после чего воздух занимает в ней лишь шестую часть от общего объема.
    Определите глубину, на которую погрузили пробирку.
    Температуру считать постоянной $T = 291\,\text{К}$, давлением паров воды пренебречь,
    атмосферное давление принять равным $p_{\text{aтм}} = 100\,\text{кПа}$.
}
\answer{%
    \begin{align*}
    T\text{ — const } &\implies P_1V_1 = \nu RT = P_2V_2.
    \\
    V_2 = \frac 16 V_1 &\implies P_1V_1 = P_2\cdot \frac 16V_1 \implies P_2 = 6P_1 = 6p_{\text{aтм}}.
    \\
    P_2 = p_{\text{aтм}} + \rho_{\text{в}} g h \implies h = \frac{ P_2 - p_{\text{aтм}} }{ \rho_{\text{в}} g } &= \frac{ 6p_{\text{aтм}} - p_{\text{aтм}} }{ \rho_{\text{в}} g } = \frac{ 5 \cdot p_{\text{aтм}} }{ \rho_{\text{в}} g } =  \\
     &= \frac{ 5 \cdot 100\,\text{кПа} }{ 1000\,\frac{\text{кг}}{\text{м}^{3}} \cdot 10\,\frac{\text{м}}{\text{с}^{2}} } \approx 50\,\text{м}.
    \end{align*}
}
\solutionspace{120pt}

\tasknumber{4}%
\task{%
    В замкнутом сосуде объёмом $5\,\text{л}$ находится азот ($\mu = 28\,\frac{\text{г}}{\text{моль}}$) под давлением $3\units{ атм }$.
    Определите массу газа в сосуде и выразите её в граммах, приняв температуру газа равной $37\celsius$.
}
\answer{%
    $
        PV = \frac m\mu RT \implies m = \frac{ PV \mu }{ RT } =
        \frac{ 3{,}0\,\text{атм} \cdot5\,\text{л} \cdot 28\,\frac{\text{г}}{\text{моль}} }{ 8{,}31\,\frac{\text{Дж}}{\text{моль}\cdot\text{К}} \cdot \cbr{ 37 + 273 }\units{К} }
        \approx 16{,}30\,\text{г}.
    $
}
\solutionspace{120pt}

\tasknumber{5}%
\task{%
    Идеальный газ в экспериментальной установке подвергут политропному процессу $PV^n\text{ — const }$
    с показателем политропы $n=0{,}4$.
    В одном из экспериментов объём газа увеличился в $2$ раза.
    Как при этом изменилась температура газа (выросла или уменьшилась, на сколько или во сколько раз)?
}
\answer{%
    \begin{align*}
    P_1V_1^n &= P_2V_2^n, P_1V_1 = \nu R T_1, P_2V_2 = \nu R T_2 \implies\frac{ \nu R T_1 }{ V_1 } V_1^n = \frac{ \nu R T_2 }{ V_2 } V_2^n \implies \\
    \implies T_1V_1^{ n-1 } &= T_2V_2^{ n - 1 } \implies\frac{ T_2 }{ T_1 } = \cbr{ \frac{ V_1 }{ V_2 } }^{ n-1 } \approx 1{,}516
    \end{align*}
}

\variantsplitter

\addpersonalvariant{Семён Мартынов}

\tasknumber{1}%
\task{%
    Из уравнения состояния идеального газа выведите или выразите...
    \begin{enumerate}
        \item давление,
        \item температуру,
        \item концентрацию молекул газа.
    \end{enumerate}
}
\solutionspace{40pt}

\tasknumber{2}%
\task{%
    Изобразите в координатах $PV$, соблюдая масштаб, процесс 1234,
    в котором 12 — изобарическое охлаждение в 3 раза,
    23 — изотермическое сжатие в 3 раза,
    34 — изохорическое нагревание в 2 раза.
}
\solutionspace{160pt}

\tasknumber{3}%
\task{%
    Небольшую цилиндрическую пробирку с воздухом погружают на некоторую глубину в глубокое пресное озеро,
    после чего воздух занимает в ней лишь четвертую часть от общего объема.
    Определите глубину, на которую погрузили пробирку.
    Температуру считать постоянной $T = 287\,\text{К}$, давлением паров воды пренебречь,
    атмосферное давление принять равным $p_{\text{aтм}} = 100\,\text{кПа}$.
}
\answer{%
    \begin{align*}
    T\text{ — const } &\implies P_1V_1 = \nu RT = P_2V_2.
    \\
    V_2 = \frac 14 V_1 &\implies P_1V_1 = P_2\cdot \frac 14V_1 \implies P_2 = 4P_1 = 4p_{\text{aтм}}.
    \\
    P_2 = p_{\text{aтм}} + \rho_{\text{в}} g h \implies h = \frac{ P_2 - p_{\text{aтм}} }{ \rho_{\text{в}} g } &= \frac{ 4p_{\text{aтм}} - p_{\text{aтм}} }{ \rho_{\text{в}} g } = \frac{ 3 \cdot p_{\text{aтм}} }{ \rho_{\text{в}} g } =  \\
     &= \frac{ 3 \cdot 100\,\text{кПа} }{ 1000\,\frac{\text{кг}}{\text{м}^{3}} \cdot 10\,\frac{\text{м}}{\text{с}^{2}} } \approx 30\,\text{м}.
    \end{align*}
}
\solutionspace{120pt}

\tasknumber{4}%
\task{%
    В замкнутом сосуде объёмом $5\,\text{л}$ находится углекислый газ ($\mu = 44\,\frac{\text{г}}{\text{моль}}$) под давлением $5\units{ атм }$.
    Определите массу газа в сосуде и выразите её в граммах, приняв температуру газа равной $37\celsius$.
}
\answer{%
    $
        PV = \frac m\mu RT \implies m = \frac{ PV \mu }{ RT } =
        \frac{ 5{,}0\,\text{атм} \cdot5\,\text{л} \cdot 44\,\frac{\text{г}}{\text{моль}} }{ 8{,}31\,\frac{\text{Дж}}{\text{моль}\cdot\text{К}} \cdot \cbr{ 37 + 273 }\units{К} }
        \approx 42{,}70\,\text{г}.
    $
}
\solutionspace{120pt}

\tasknumber{5}%
\task{%
    Идеальный газ в экспериментальной установке подвергут политропному процессу $PV^n\text{ — const }$
    с показателем политропы $n=1{,}2$.
    В одном из экспериментов объём газа увеличился в $3$ раза.
    Как при этом изменилась температура газа (выросла или уменьшилась, на сколько или во сколько раз)?
}
\answer{%
    \begin{align*}
    P_1V_1^n &= P_2V_2^n, P_1V_1 = \nu R T_1, P_2V_2 = \nu R T_2 \implies\frac{ \nu R T_1 }{ V_1 } V_1^n = \frac{ \nu R T_2 }{ V_2 } V_2^n \implies \\
    \implies T_1V_1^{ n-1 } &= T_2V_2^{ n - 1 } \implies\frac{ T_2 }{ T_1 } = \cbr{ \frac{ V_1 }{ V_2 } }^{ n-1 } \approx 0{,}803
    \end{align*}
}

\variantsplitter

\addpersonalvariant{Варвара Минаева}

\tasknumber{1}%
\task{%
    Из уравнения состояния идеального газа выведите или выразите...
    \begin{enumerate}
        \item давление,
        \item температуру,
        \item концентрацию молекул газа.
    \end{enumerate}
}
\solutionspace{40pt}

\tasknumber{2}%
\task{%
    Изобразите в координатах $PV$, соблюдая масштаб, процесс 1234,
    в котором 12 — изобарическое нагревание в 3 раза,
    23 — изотермическое расширение в 3 раза,
    34 — изобарическое нагревание в 3 раза.
}
\solutionspace{160pt}

\tasknumber{3}%
\task{%
    Небольшую цилиндрическую пробирку с воздухом погружают на некоторую глубину в глубокое пресное озеро,
    после чего воздух занимает в ней лишь четвертую часть от общего объема.
    Определите глубину, на которую погрузили пробирку.
    Температуру считать постоянной $T = 285\,\text{К}$, давлением паров воды пренебречь,
    атмосферное давление принять равным $p_{\text{aтм}} = 100\,\text{кПа}$.
}
\answer{%
    \begin{align*}
    T\text{ — const } &\implies P_1V_1 = \nu RT = P_2V_2.
    \\
    V_2 = \frac 14 V_1 &\implies P_1V_1 = P_2\cdot \frac 14V_1 \implies P_2 = 4P_1 = 4p_{\text{aтм}}.
    \\
    P_2 = p_{\text{aтм}} + \rho_{\text{в}} g h \implies h = \frac{ P_2 - p_{\text{aтм}} }{ \rho_{\text{в}} g } &= \frac{ 4p_{\text{aтм}} - p_{\text{aтм}} }{ \rho_{\text{в}} g } = \frac{ 3 \cdot p_{\text{aтм}} }{ \rho_{\text{в}} g } =  \\
     &= \frac{ 3 \cdot 100\,\text{кПа} }{ 1000\,\frac{\text{кг}}{\text{м}^{3}} \cdot 10\,\frac{\text{м}}{\text{с}^{2}} } \approx 30\,\text{м}.
    \end{align*}
}
\solutionspace{120pt}

\tasknumber{4}%
\task{%
    В замкнутом сосуде объёмом $2\,\text{л}$ находится углекислый газ ($\mu = 44\,\frac{\text{г}}{\text{моль}}$) под давлением $3{,}5\units{ атм }$.
    Определите массу газа в сосуде и выразите её в граммах, приняв температуру газа равной $47\celsius$.
}
\answer{%
    $
        PV = \frac m\mu RT \implies m = \frac{ PV \mu }{ RT } =
        \frac{ 3{,}5\,\text{атм} \cdot2\,\text{л} \cdot 44\,\frac{\text{г}}{\text{моль}} }{ 8{,}31\,\frac{\text{Дж}}{\text{моль}\cdot\text{К}} \cdot \cbr{ 47 + 273 }\units{К} }
        \approx 11{,}58\,\text{г}.
    $
}
\solutionspace{120pt}

\tasknumber{5}%
\task{%
    Идеальный газ в экспериментальной установке подвергут политропному процессу $PV^n\text{ — const }$
    с показателем политропы $n=1{,}8$.
    В одном из экспериментов объём газа уменьшился в $2$ раза.
    Как при этом изменилась температура газа (выросла или уменьшилась, на сколько или во сколько раз)?
}
\answer{%
    \begin{align*}
    P_1V_1^n &= P_2V_2^n, P_1V_1 = \nu R T_1, P_2V_2 = \nu R T_2 \implies\frac{ \nu R T_1 }{ V_1 } V_1^n = \frac{ \nu R T_2 }{ V_2 } V_2^n \implies \\
    \implies T_1V_1^{ n-1 } &= T_2V_2^{ n - 1 } \implies\frac{ T_2 }{ T_1 } = \cbr{ \frac{ V_1 }{ V_2 } }^{ n-1 } \approx 1{,}741
    \end{align*}
}

\variantsplitter

\addpersonalvariant{Леонид Никитин}

\tasknumber{1}%
\task{%
    Из уравнения состояния идеального газа выведите или выразите...
    \begin{enumerate}
        \item давление,
        \item молярную массу,
        \item концентрацию молекул газа.
    \end{enumerate}
}
\solutionspace{40pt}

\tasknumber{2}%
\task{%
    Изобразите в координатах $PV$, соблюдая масштаб, процесс 1234,
    в котором 12 — изобарическое нагревание в 3 раза,
    23 — изотермическое сжатие в 2 раза,
    34 — изобарическое охлаждение в 2 раза.
}
\solutionspace{160pt}

\tasknumber{3}%
\task{%
    Небольшую цилиндрическую пробирку с воздухом погружают на некоторую глубину в глубокое пресное озеро,
    после чего воздух занимает в ней лишь шестую часть от общего объема.
    Определите глубину, на которую погрузили пробирку.
    Температуру считать постоянной $T = 292\,\text{К}$, давлением паров воды пренебречь,
    атмосферное давление принять равным $p_{\text{aтм}} = 100\,\text{кПа}$.
}
\answer{%
    \begin{align*}
    T\text{ — const } &\implies P_1V_1 = \nu RT = P_2V_2.
    \\
    V_2 = \frac 16 V_1 &\implies P_1V_1 = P_2\cdot \frac 16V_1 \implies P_2 = 6P_1 = 6p_{\text{aтм}}.
    \\
    P_2 = p_{\text{aтм}} + \rho_{\text{в}} g h \implies h = \frac{ P_2 - p_{\text{aтм}} }{ \rho_{\text{в}} g } &= \frac{ 6p_{\text{aтм}} - p_{\text{aтм}} }{ \rho_{\text{в}} g } = \frac{ 5 \cdot p_{\text{aтм}} }{ \rho_{\text{в}} g } =  \\
     &= \frac{ 5 \cdot 100\,\text{кПа} }{ 1000\,\frac{\text{кг}}{\text{м}^{3}} \cdot 10\,\frac{\text{м}}{\text{с}^{2}} } \approx 50\,\text{м}.
    \end{align*}
}
\solutionspace{120pt}

\tasknumber{4}%
\task{%
    В замкнутом сосуде объёмом $5\,\text{л}$ находится кислород ($\mu = 32\,\frac{\text{г}}{\text{моль}}$) под давлением $4\units{ атм }$.
    Определите массу газа в сосуде и выразите её в граммах, приняв температуру газа равной $47\celsius$.
}
\answer{%
    $
        PV = \frac m\mu RT \implies m = \frac{ PV \mu }{ RT } =
        \frac{ 4{,}0\,\text{атм} \cdot5\,\text{л} \cdot 32\,\frac{\text{г}}{\text{моль}} }{ 8{,}31\,\frac{\text{Дж}}{\text{моль}\cdot\text{К}} \cdot \cbr{ 47 + 273 }\units{К} }
        \approx 24{,}07\,\text{г}.
    $
}
\solutionspace{120pt}

\tasknumber{5}%
\task{%
    Идеальный газ в экспериментальной установке подвергут политропному процессу $PV^n\text{ — const }$
    с показателем политропы $n=1{,}5$.
    В одном из экспериментов объём газа увеличился в $4$ раза.
    Как при этом изменилась температура газа (выросла или уменьшилась, на сколько или во сколько раз)?
}
\answer{%
    \begin{align*}
    P_1V_1^n &= P_2V_2^n, P_1V_1 = \nu R T_1, P_2V_2 = \nu R T_2 \implies\frac{ \nu R T_1 }{ V_1 } V_1^n = \frac{ \nu R T_2 }{ V_2 } V_2^n \implies \\
    \implies T_1V_1^{ n-1 } &= T_2V_2^{ n - 1 } \implies\frac{ T_2 }{ T_1 } = \cbr{ \frac{ V_1 }{ V_2 } }^{ n-1 } \approx 0{,}500
    \end{align*}
}

\variantsplitter

\addpersonalvariant{Тимофей Полетаев}

\tasknumber{1}%
\task{%
    Из уравнения состояния идеального газа выведите или выразите...
    \begin{enumerate}
        \item давление,
        \item температуру,
        \item плотность газа.
    \end{enumerate}
}
\solutionspace{40pt}

\tasknumber{2}%
\task{%
    Изобразите в координатах $PV$, соблюдая масштаб, процесс 1234,
    в котором 12 — изохорическое охлаждение в 2 раза,
    23 — изотермическое сжатие в 2 раза,
    34 — изохорическое нагревание в 2 раза.
}
\solutionspace{160pt}

\tasknumber{3}%
\task{%
    Небольшую цилиндрическую пробирку с воздухом погружают на некоторую глубину в глубокое пресное озеро,
    после чего воздух занимает в ней лишь третью часть от общего объема.
    Определите глубину, на которую погрузили пробирку.
    Температуру считать постоянной $T = 291\,\text{К}$, давлением паров воды пренебречь,
    атмосферное давление принять равным $p_{\text{aтм}} = 100\,\text{кПа}$.
}
\answer{%
    \begin{align*}
    T\text{ — const } &\implies P_1V_1 = \nu RT = P_2V_2.
    \\
    V_2 = \frac 13 V_1 &\implies P_1V_1 = P_2\cdot \frac 13V_1 \implies P_2 = 3P_1 = 3p_{\text{aтм}}.
    \\
    P_2 = p_{\text{aтм}} + \rho_{\text{в}} g h \implies h = \frac{ P_2 - p_{\text{aтм}} }{ \rho_{\text{в}} g } &= \frac{ 3p_{\text{aтм}} - p_{\text{aтм}} }{ \rho_{\text{в}} g } = \frac{ 2 \cdot p_{\text{aтм}} }{ \rho_{\text{в}} g } =  \\
     &= \frac{ 2 \cdot 100\,\text{кПа} }{ 1000\,\frac{\text{кг}}{\text{м}^{3}} \cdot 10\,\frac{\text{м}}{\text{с}^{2}} } \approx 20\,\text{м}.
    \end{align*}
}
\solutionspace{120pt}

\tasknumber{4}%
\task{%
    В замкнутом сосуде объёмом $3\,\text{л}$ находится воздух ($\mu = 29\,\frac{\text{г}}{\text{моль}}$) под давлением $5\units{ атм }$.
    Определите массу газа в сосуде и выразите её в граммах, приняв температуру газа равной $27\celsius$.
}
\answer{%
    $
        PV = \frac m\mu RT \implies m = \frac{ PV \mu }{ RT } =
        \frac{ 5{,}0\,\text{атм} \cdot3\,\text{л} \cdot 29\,\frac{\text{г}}{\text{моль}} }{ 8{,}31\,\frac{\text{Дж}}{\text{моль}\cdot\text{К}} \cdot \cbr{ 27 + 273 }\units{К} }
        \approx 17{,}45\,\text{г}.
    $
}
\solutionspace{120pt}

\tasknumber{5}%
\task{%
    Идеальный газ в экспериментальной установке подвергут политропному процессу $PV^n\text{ — const }$
    с показателем политропы $n=1{,}5$.
    В одном из экспериментов объём газа уменьшился в $4$ раза.
    Как при этом изменилась температура газа (выросла или уменьшилась, на сколько или во сколько раз)?
}
\answer{%
    \begin{align*}
    P_1V_1^n &= P_2V_2^n, P_1V_1 = \nu R T_1, P_2V_2 = \nu R T_2 \implies\frac{ \nu R T_1 }{ V_1 } V_1^n = \frac{ \nu R T_2 }{ V_2 } V_2^n \implies \\
    \implies T_1V_1^{ n-1 } &= T_2V_2^{ n - 1 } \implies\frac{ T_2 }{ T_1 } = \cbr{ \frac{ V_1 }{ V_2 } }^{ n-1 } \approx 2{,}000
    \end{align*}
}

\variantsplitter

\addpersonalvariant{Андрей Рожков}

\tasknumber{1}%
\task{%
    Из уравнения состояния идеального газа выведите или выразите...
    \begin{enumerate}
        \item давление,
        \item молярную массу,
        \item концентрацию молекул газа.
    \end{enumerate}
}
\solutionspace{40pt}

\tasknumber{2}%
\task{%
    Изобразите в координатах $PV$, соблюдая масштаб, процесс 1234,
    в котором 12 — изохорическое нагревание в 2 раза,
    23 — изотермическое сжатие в 3 раза,
    34 — изобарическое нагревание в 2 раза.
}
\solutionspace{160pt}

\tasknumber{3}%
\task{%
    Небольшую цилиндрическую пробирку с воздухом погружают на некоторую глубину в глубокое пресное озеро,
    после чего воздух занимает в ней лишь пятую часть от общего объема.
    Определите глубину, на которую погрузили пробирку.
    Температуру считать постоянной $T = 286\,\text{К}$, давлением паров воды пренебречь,
    атмосферное давление принять равным $p_{\text{aтм}} = 100\,\text{кПа}$.
}
\answer{%
    \begin{align*}
    T\text{ — const } &\implies P_1V_1 = \nu RT = P_2V_2.
    \\
    V_2 = \frac 15 V_1 &\implies P_1V_1 = P_2\cdot \frac 15V_1 \implies P_2 = 5P_1 = 5p_{\text{aтм}}.
    \\
    P_2 = p_{\text{aтм}} + \rho_{\text{в}} g h \implies h = \frac{ P_2 - p_{\text{aтм}} }{ \rho_{\text{в}} g } &= \frac{ 5p_{\text{aтм}} - p_{\text{aтм}} }{ \rho_{\text{в}} g } = \frac{ 4 \cdot p_{\text{aтм}} }{ \rho_{\text{в}} g } =  \\
     &= \frac{ 4 \cdot 100\,\text{кПа} }{ 1000\,\frac{\text{кг}}{\text{м}^{3}} \cdot 10\,\frac{\text{м}}{\text{с}^{2}} } \approx 40\,\text{м}.
    \end{align*}
}
\solutionspace{120pt}

\tasknumber{4}%
\task{%
    В замкнутом сосуде объёмом $2\,\text{л}$ находится неон ($\mu = 20\,\frac{\text{г}}{\text{моль}}$) под давлением $2{,}5\units{ атм }$.
    Определите массу газа в сосуде и выразите её в граммах, приняв температуру газа равной $17\celsius$.
}
\answer{%
    $
        PV = \frac m\mu RT \implies m = \frac{ PV \mu }{ RT } =
        \frac{ 2{,}5\,\text{атм} \cdot2\,\text{л} \cdot 20\,\frac{\text{г}}{\text{моль}} }{ 8{,}31\,\frac{\text{Дж}}{\text{моль}\cdot\text{К}} \cdot \cbr{ 17 + 273 }\units{К} }
        \approx 4{,}15\,\text{г}.
    $
}
\solutionspace{120pt}

\tasknumber{5}%
\task{%
    Идеальный газ в экспериментальной установке подвергут политропному процессу $PV^n\text{ — const }$
    с показателем политропы $n=0{,}4$.
    В одном из экспериментов объём газа уменьшился в $4$ раза.
    Как при этом изменилась температура газа (выросла или уменьшилась, на сколько или во сколько раз)?
}
\answer{%
    \begin{align*}
    P_1V_1^n &= P_2V_2^n, P_1V_1 = \nu R T_1, P_2V_2 = \nu R T_2 \implies\frac{ \nu R T_1 }{ V_1 } V_1^n = \frac{ \nu R T_2 }{ V_2 } V_2^n \implies \\
    \implies T_1V_1^{ n-1 } &= T_2V_2^{ n - 1 } \implies\frac{ T_2 }{ T_1 } = \cbr{ \frac{ V_1 }{ V_2 } }^{ n-1 } \approx 0{,}435
    \end{align*}
}

\variantsplitter

\addpersonalvariant{Рената Таржиманова}

\tasknumber{1}%
\task{%
    Из уравнения состояния идеального газа выведите или выразите...
    \begin{enumerate}
        \item объём,
        \item молярную массу,
        \item концентрацию молекул газа.
    \end{enumerate}
}
\solutionspace{40pt}

\tasknumber{2}%
\task{%
    Изобразите в координатах $PV$, соблюдая масштаб, процесс 1234,
    в котором 12 — изохорическое охлаждение в 2 раза,
    23 — изотермическое сжатие в 3 раза,
    34 — изохорическое нагревание в 2 раза.
}
\solutionspace{160pt}

\tasknumber{3}%
\task{%
    Небольшую цилиндрическую пробирку с воздухом погружают на некоторую глубину в глубокое пресное озеро,
    после чего воздух занимает в ней лишь пятую часть от общего объема.
    Определите глубину, на которую погрузили пробирку.
    Температуру считать постоянной $T = 285\,\text{К}$, давлением паров воды пренебречь,
    атмосферное давление принять равным $p_{\text{aтм}} = 100\,\text{кПа}$.
}
\answer{%
    \begin{align*}
    T\text{ — const } &\implies P_1V_1 = \nu RT = P_2V_2.
    \\
    V_2 = \frac 15 V_1 &\implies P_1V_1 = P_2\cdot \frac 15V_1 \implies P_2 = 5P_1 = 5p_{\text{aтм}}.
    \\
    P_2 = p_{\text{aтм}} + \rho_{\text{в}} g h \implies h = \frac{ P_2 - p_{\text{aтм}} }{ \rho_{\text{в}} g } &= \frac{ 5p_{\text{aтм}} - p_{\text{aтм}} }{ \rho_{\text{в}} g } = \frac{ 4 \cdot p_{\text{aтм}} }{ \rho_{\text{в}} g } =  \\
     &= \frac{ 4 \cdot 100\,\text{кПа} }{ 1000\,\frac{\text{кг}}{\text{м}^{3}} \cdot 10\,\frac{\text{м}}{\text{с}^{2}} } \approx 40\,\text{м}.
    \end{align*}
}
\solutionspace{120pt}

\tasknumber{4}%
\task{%
    В замкнутом сосуде объёмом $2\,\text{л}$ находится углекислый газ ($\mu = 44\,\frac{\text{г}}{\text{моль}}$) под давлением $3{,}5\units{ атм }$.
    Определите массу газа в сосуде и выразите её в граммах, приняв температуру газа равной $27\celsius$.
}
\answer{%
    $
        PV = \frac m\mu RT \implies m = \frac{ PV \mu }{ RT } =
        \frac{ 3{,}5\,\text{атм} \cdot2\,\text{л} \cdot 44\,\frac{\text{г}}{\text{моль}} }{ 8{,}31\,\frac{\text{Дж}}{\text{моль}\cdot\text{К}} \cdot \cbr{ 27 + 273 }\units{К} }
        \approx 12{,}35\,\text{г}.
    $
}
\solutionspace{120pt}

\tasknumber{5}%
\task{%
    Идеальный газ в экспериментальной установке подвергут политропному процессу $PV^n\text{ — const }$
    с показателем политропы $n=1{,}5$.
    В одном из экспериментов объём газа увеличился в $4$ раза.
    Как при этом изменилась температура газа (выросла или уменьшилась, на сколько или во сколько раз)?
}
\answer{%
    \begin{align*}
    P_1V_1^n &= P_2V_2^n, P_1V_1 = \nu R T_1, P_2V_2 = \nu R T_2 \implies\frac{ \nu R T_1 }{ V_1 } V_1^n = \frac{ \nu R T_2 }{ V_2 } V_2^n \implies \\
    \implies T_1V_1^{ n-1 } &= T_2V_2^{ n - 1 } \implies\frac{ T_2 }{ T_1 } = \cbr{ \frac{ V_1 }{ V_2 } }^{ n-1 } \approx 0{,}500
    \end{align*}
}

\variantsplitter

\addpersonalvariant{Арсений Трофимов}

\tasknumber{1}%
\task{%
    Из уравнения состояния идеального газа выведите или выразите...
    \begin{enumerate}
        \item объём,
        \item молярную массу,
        \item концентрацию молекул газа.
    \end{enumerate}
}
\solutionspace{40pt}

\tasknumber{2}%
\task{%
    Изобразите в координатах $PV$, соблюдая масштаб, процесс 1234,
    в котором 12 — изобарическое нагревание в 2 раза,
    23 — изотермическое сжатие в 3 раза,
    34 — изохорическое охлаждение в 3 раза.
}
\solutionspace{160pt}

\tasknumber{3}%
\task{%
    Небольшую цилиндрическую пробирку с воздухом погружают на некоторую глубину в глубокое пресное озеро,
    после чего воздух занимает в ней лишь шестую часть от общего объема.
    Определите глубину, на которую погрузили пробирку.
    Температуру считать постоянной $T = 280\,\text{К}$, давлением паров воды пренебречь,
    атмосферное давление принять равным $p_{\text{aтм}} = 100\,\text{кПа}$.
}
\answer{%
    \begin{align*}
    T\text{ — const } &\implies P_1V_1 = \nu RT = P_2V_2.
    \\
    V_2 = \frac 16 V_1 &\implies P_1V_1 = P_2\cdot \frac 16V_1 \implies P_2 = 6P_1 = 6p_{\text{aтм}}.
    \\
    P_2 = p_{\text{aтм}} + \rho_{\text{в}} g h \implies h = \frac{ P_2 - p_{\text{aтм}} }{ \rho_{\text{в}} g } &= \frac{ 6p_{\text{aтм}} - p_{\text{aтм}} }{ \rho_{\text{в}} g } = \frac{ 5 \cdot p_{\text{aтм}} }{ \rho_{\text{в}} g } =  \\
     &= \frac{ 5 \cdot 100\,\text{кПа} }{ 1000\,\frac{\text{кг}}{\text{м}^{3}} \cdot 10\,\frac{\text{м}}{\text{с}^{2}} } \approx 50\,\text{м}.
    \end{align*}
}
\solutionspace{120pt}

\tasknumber{4}%
\task{%
    В замкнутом сосуде объёмом $5\,\text{л}$ находится кислород ($\mu = 32\,\frac{\text{г}}{\text{моль}}$) под давлением $4{,}5\units{ атм }$.
    Определите массу газа в сосуде и выразите её в граммах, приняв температуру газа равной $37\celsius$.
}
\answer{%
    $
        PV = \frac m\mu RT \implies m = \frac{ PV \mu }{ RT } =
        \frac{ 4{,}5\,\text{атм} \cdot5\,\text{л} \cdot 32\,\frac{\text{г}}{\text{моль}} }{ 8{,}31\,\frac{\text{Дж}}{\text{моль}\cdot\text{К}} \cdot \cbr{ 37 + 273 }\units{К} }
        \approx 27{,}95\,\text{г}.
    $
}
\solutionspace{120pt}

\tasknumber{5}%
\task{%
    Идеальный газ в экспериментальной установке подвергут политропному процессу $PV^n\text{ — const }$
    с показателем политропы $n=1{,}2$.
    В одном из экспериментов объём газа увеличился в $4$ раза.
    Как при этом изменилась температура газа (выросла или уменьшилась, на сколько или во сколько раз)?
}
\answer{%
    \begin{align*}
    P_1V_1^n &= P_2V_2^n, P_1V_1 = \nu R T_1, P_2V_2 = \nu R T_2 \implies\frac{ \nu R T_1 }{ V_1 } V_1^n = \frac{ \nu R T_2 }{ V_2 } V_2^n \implies \\
    \implies T_1V_1^{ n-1 } &= T_2V_2^{ n - 1 } \implies\frac{ T_2 }{ T_1 } = \cbr{ \frac{ V_1 }{ V_2 } }^{ n-1 } \approx 0{,}758
    \end{align*}
}

\variantsplitter

\addpersonalvariant{Андрей Щербаков}

\tasknumber{1}%
\task{%
    Из уравнения состояния идеального газа выведите или выразите...
    \begin{enumerate}
        \item объём,
        \item молярную массу,
        \item плотность газа.
    \end{enumerate}
}
\solutionspace{40pt}

\tasknumber{2}%
\task{%
    Изобразите в координатах $PV$, соблюдая масштаб, процесс 1234,
    в котором 12 — изобарическое нагревание в 2 раза,
    23 — изотермическое расширение в 3 раза,
    34 — изохорическое охлаждение в 3 раза.
}
\solutionspace{160pt}

\tasknumber{3}%
\task{%
    Небольшую цилиндрическую пробирку с воздухом погружают на некоторую глубину в глубокое пресное озеро,
    после чего воздух занимает в ней лишь шестую часть от общего объема.
    Определите глубину, на которую погрузили пробирку.
    Температуру считать постоянной $T = 279\,\text{К}$, давлением паров воды пренебречь,
    атмосферное давление принять равным $p_{\text{aтм}} = 100\,\text{кПа}$.
}
\answer{%
    \begin{align*}
    T\text{ — const } &\implies P_1V_1 = \nu RT = P_2V_2.
    \\
    V_2 = \frac 16 V_1 &\implies P_1V_1 = P_2\cdot \frac 16V_1 \implies P_2 = 6P_1 = 6p_{\text{aтм}}.
    \\
    P_2 = p_{\text{aтм}} + \rho_{\text{в}} g h \implies h = \frac{ P_2 - p_{\text{aтм}} }{ \rho_{\text{в}} g } &= \frac{ 6p_{\text{aтм}} - p_{\text{aтм}} }{ \rho_{\text{в}} g } = \frac{ 5 \cdot p_{\text{aтм}} }{ \rho_{\text{в}} g } =  \\
     &= \frac{ 5 \cdot 100\,\text{кПа} }{ 1000\,\frac{\text{кг}}{\text{м}^{3}} \cdot 10\,\frac{\text{м}}{\text{с}^{2}} } \approx 50\,\text{м}.
    \end{align*}
}
\solutionspace{120pt}

\tasknumber{4}%
\task{%
    В замкнутом сосуде объёмом $2\,\text{л}$ находится неон ($\mu = 20\,\frac{\text{г}}{\text{моль}}$) под давлением $5\units{ атм }$.
    Определите массу газа в сосуде и выразите её в граммах, приняв температуру газа равной $47\celsius$.
}
\answer{%
    $
        PV = \frac m\mu RT \implies m = \frac{ PV \mu }{ RT } =
        \frac{ 5{,}0\,\text{атм} \cdot2\,\text{л} \cdot 20\,\frac{\text{г}}{\text{моль}} }{ 8{,}31\,\frac{\text{Дж}}{\text{моль}\cdot\text{К}} \cdot \cbr{ 47 + 273 }\units{К} }
        \approx 7{,}52\,\text{г}.
    $
}
\solutionspace{120pt}

\tasknumber{5}%
\task{%
    Идеальный газ в экспериментальной установке подвергут политропному процессу $PV^n\text{ — const }$
    с показателем политропы $n=0{,}7$.
    В одном из экспериментов объём газа уменьшился в $4$ раза.
    Как при этом изменилась температура газа (выросла или уменьшилась, на сколько или во сколько раз)?
}
\answer{%
    \begin{align*}
    P_1V_1^n &= P_2V_2^n, P_1V_1 = \nu R T_1, P_2V_2 = \nu R T_2 \implies\frac{ \nu R T_1 }{ V_1 } V_1^n = \frac{ \nu R T_2 }{ V_2 } V_2^n \implies \\
    \implies T_1V_1^{ n-1 } &= T_2V_2^{ n - 1 } \implies\frac{ T_2 }{ T_1 } = \cbr{ \frac{ V_1 }{ V_2 } }^{ n-1 } \approx 0{,}660
    \end{align*}
}

\variantsplitter

\addpersonalvariant{Михаил Ярошевский}

\tasknumber{1}%
\task{%
    Из уравнения состояния идеального газа выведите или выразите...
    \begin{enumerate}
        \item давление,
        \item температуру,
        \item плотность газа.
    \end{enumerate}
}
\solutionspace{40pt}

\tasknumber{2}%
\task{%
    Изобразите в координатах $PV$, соблюдая масштаб, процесс 1234,
    в котором 12 — изобарическое нагревание в 2 раза,
    23 — изотермическое сжатие в 2 раза,
    34 — изобарическое нагревание в 2 раза.
}
\solutionspace{160pt}

\tasknumber{3}%
\task{%
    Небольшую цилиндрическую пробирку с воздухом погружают на некоторую глубину в глубокое пресное озеро,
    после чего воздух занимает в ней лишь третью часть от общего объема.
    Определите глубину, на которую погрузили пробирку.
    Температуру считать постоянной $T = 288\,\text{К}$, давлением паров воды пренебречь,
    атмосферное давление принять равным $p_{\text{aтм}} = 100\,\text{кПа}$.
}
\answer{%
    \begin{align*}
    T\text{ — const } &\implies P_1V_1 = \nu RT = P_2V_2.
    \\
    V_2 = \frac 13 V_1 &\implies P_1V_1 = P_2\cdot \frac 13V_1 \implies P_2 = 3P_1 = 3p_{\text{aтм}}.
    \\
    P_2 = p_{\text{aтм}} + \rho_{\text{в}} g h \implies h = \frac{ P_2 - p_{\text{aтм}} }{ \rho_{\text{в}} g } &= \frac{ 3p_{\text{aтм}} - p_{\text{aтм}} }{ \rho_{\text{в}} g } = \frac{ 2 \cdot p_{\text{aтм}} }{ \rho_{\text{в}} g } =  \\
     &= \frac{ 2 \cdot 100\,\text{кПа} }{ 1000\,\frac{\text{кг}}{\text{м}^{3}} \cdot 10\,\frac{\text{м}}{\text{с}^{2}} } \approx 20\,\text{м}.
    \end{align*}
}
\solutionspace{120pt}

\tasknumber{4}%
\task{%
    В замкнутом сосуде объёмом $5\,\text{л}$ находится кислород ($\mu = 32\,\frac{\text{г}}{\text{моль}}$) под давлением $3\units{ атм }$.
    Определите массу газа в сосуде и выразите её в граммах, приняв температуру газа равной $37\celsius$.
}
\answer{%
    $
        PV = \frac m\mu RT \implies m = \frac{ PV \mu }{ RT } =
        \frac{ 3{,}0\,\text{атм} \cdot5\,\text{л} \cdot 32\,\frac{\text{г}}{\text{моль}} }{ 8{,}31\,\frac{\text{Дж}}{\text{моль}\cdot\text{К}} \cdot \cbr{ 37 + 273 }\units{К} }
        \approx 18{,}63\,\text{г}.
    $
}
\solutionspace{120pt}

\tasknumber{5}%
\task{%
    Идеальный газ в экспериментальной установке подвергут политропному процессу $PV^n\text{ — const }$
    с показателем политропы $n=1{,}2$.
    В одном из экспериментов объём газа увеличился в $3$ раза.
    Как при этом изменилась температура газа (выросла или уменьшилась, на сколько или во сколько раз)?
}
\answer{%
    \begin{align*}
    P_1V_1^n &= P_2V_2^n, P_1V_1 = \nu R T_1, P_2V_2 = \nu R T_2 \implies\frac{ \nu R T_1 }{ V_1 } V_1^n = \frac{ \nu R T_2 }{ V_2 } V_2^n \implies \\
    \implies T_1V_1^{ n-1 } &= T_2V_2^{ n - 1 } \implies\frac{ T_2 }{ T_1 } = \cbr{ \frac{ V_1 }{ V_2 } }^{ n-1 } \approx 0{,}803
    \end{align*}
}

\variantsplitter

\addpersonalvariant{Алексей Алимпиев}

\tasknumber{1}%
\task{%
    Из уравнения состояния идеального газа выведите или выразите...
    \begin{enumerate}
        \item объём,
        \item молярную массу,
        \item плотность газа.
    \end{enumerate}
}
\solutionspace{40pt}

\tasknumber{2}%
\task{%
    Изобразите в координатах $PV$, соблюдая масштаб, процесс 1234,
    в котором 12 — изохорическое охлаждение в 2 раза,
    23 — изотермическое расширение в 3 раза,
    34 — изобарическое охлаждение в 2 раза.
}
\solutionspace{160pt}

\tasknumber{3}%
\task{%
    Небольшую цилиндрическую пробирку с воздухом погружают на некоторую глубину в глубокое пресное озеро,
    после чего воздух занимает в ней лишь пятую часть от общего объема.
    Определите глубину, на которую погрузили пробирку.
    Температуру считать постоянной $T = 278\,\text{К}$, давлением паров воды пренебречь,
    атмосферное давление принять равным $p_{\text{aтм}} = 100\,\text{кПа}$.
}
\answer{%
    \begin{align*}
    T\text{ — const } &\implies P_1V_1 = \nu RT = P_2V_2.
    \\
    V_2 = \frac 15 V_1 &\implies P_1V_1 = P_2\cdot \frac 15V_1 \implies P_2 = 5P_1 = 5p_{\text{aтм}}.
    \\
    P_2 = p_{\text{aтм}} + \rho_{\text{в}} g h \implies h = \frac{ P_2 - p_{\text{aтм}} }{ \rho_{\text{в}} g } &= \frac{ 5p_{\text{aтм}} - p_{\text{aтм}} }{ \rho_{\text{в}} g } = \frac{ 4 \cdot p_{\text{aтм}} }{ \rho_{\text{в}} g } =  \\
     &= \frac{ 4 \cdot 100\,\text{кПа} }{ 1000\,\frac{\text{кг}}{\text{м}^{3}} \cdot 10\,\frac{\text{м}}{\text{с}^{2}} } \approx 40\,\text{м}.
    \end{align*}
}
\solutionspace{120pt}

\tasknumber{4}%
\task{%
    В замкнутом сосуде объёмом $2\,\text{л}$ находится аргон ($\mu = 40\,\frac{\text{г}}{\text{моль}}$) под давлением $3{,}5\units{ атм }$.
    Определите массу газа в сосуде и выразите её в граммах, приняв температуру газа равной $7\celsius$.
}
\answer{%
    $
        PV = \frac m\mu RT \implies m = \frac{ PV \mu }{ RT } =
        \frac{ 3{,}5\,\text{атм} \cdot2\,\text{л} \cdot 40\,\frac{\text{г}}{\text{моль}} }{ 8{,}31\,\frac{\text{Дж}}{\text{моль}\cdot\text{К}} \cdot \cbr{ 7 + 273 }\units{К} }
        \approx 12{,}03\,\text{г}.
    $
}
\solutionspace{120pt}

\tasknumber{5}%
\task{%
    Идеальный газ в экспериментальной установке подвергут политропному процессу $PV^n\text{ — const }$
    с показателем политропы $n=1{,}2$.
    В одном из экспериментов объём газа уменьшился в $4$ раза.
    Как при этом изменилась температура газа (выросла или уменьшилась, на сколько или во сколько раз)?
}
\answer{%
    \begin{align*}
    P_1V_1^n &= P_2V_2^n, P_1V_1 = \nu R T_1, P_2V_2 = \nu R T_2 \implies\frac{ \nu R T_1 }{ V_1 } V_1^n = \frac{ \nu R T_2 }{ V_2 } V_2^n \implies \\
    \implies T_1V_1^{ n-1 } &= T_2V_2^{ n - 1 } \implies\frac{ T_2 }{ T_1 } = \cbr{ \frac{ V_1 }{ V_2 } }^{ n-1 } \approx 1{,}320
    \end{align*}
}

\variantsplitter

\addpersonalvariant{Евгений Васин}

\tasknumber{1}%
\task{%
    Из уравнения состояния идеального газа выведите или выразите...
    \begin{enumerate}
        \item давление,
        \item температуру,
        \item концентрацию молекул газа.
    \end{enumerate}
}
\solutionspace{40pt}

\tasknumber{2}%
\task{%
    Изобразите в координатах $PV$, соблюдая масштаб, процесс 1234,
    в котором 12 — изобарическое нагревание в 2 раза,
    23 — изотермическое расширение в 3 раза,
    34 — изобарическое нагревание в 2 раза.
}
\solutionspace{160pt}

\tasknumber{3}%
\task{%
    Небольшую цилиндрическую пробирку с воздухом погружают на некоторую глубину в глубокое пресное озеро,
    после чего воздух занимает в ней лишь третью часть от общего объема.
    Определите глубину, на которую погрузили пробирку.
    Температуру считать постоянной $T = 285\,\text{К}$, давлением паров воды пренебречь,
    атмосферное давление принять равным $p_{\text{aтм}} = 100\,\text{кПа}$.
}
\answer{%
    \begin{align*}
    T\text{ — const } &\implies P_1V_1 = \nu RT = P_2V_2.
    \\
    V_2 = \frac 13 V_1 &\implies P_1V_1 = P_2\cdot \frac 13V_1 \implies P_2 = 3P_1 = 3p_{\text{aтм}}.
    \\
    P_2 = p_{\text{aтм}} + \rho_{\text{в}} g h \implies h = \frac{ P_2 - p_{\text{aтм}} }{ \rho_{\text{в}} g } &= \frac{ 3p_{\text{aтм}} - p_{\text{aтм}} }{ \rho_{\text{в}} g } = \frac{ 2 \cdot p_{\text{aтм}} }{ \rho_{\text{в}} g } =  \\
     &= \frac{ 2 \cdot 100\,\text{кПа} }{ 1000\,\frac{\text{кг}}{\text{м}^{3}} \cdot 10\,\frac{\text{м}}{\text{с}^{2}} } \approx 20\,\text{м}.
    \end{align*}
}
\solutionspace{120pt}

\tasknumber{4}%
\task{%
    В замкнутом сосуде объёмом $4\,\text{л}$ находится азот ($\mu = 28\,\frac{\text{г}}{\text{моль}}$) под давлением $2{,}5\units{ атм }$.
    Определите массу газа в сосуде и выразите её в граммах, приняв температуру газа равной $27\celsius$.
}
\answer{%
    $
        PV = \frac m\mu RT \implies m = \frac{ PV \mu }{ RT } =
        \frac{ 2{,}5\,\text{атм} \cdot4\,\text{л} \cdot 28\,\frac{\text{г}}{\text{моль}} }{ 8{,}31\,\frac{\text{Дж}}{\text{моль}\cdot\text{К}} \cdot \cbr{ 27 + 273 }\units{К} }
        \approx 11{,}23\,\text{г}.
    $
}
\solutionspace{120pt}

\tasknumber{5}%
\task{%
    Идеальный газ в экспериментальной установке подвергут политропному процессу $PV^n\text{ — const }$
    с показателем политропы $n=0{,}7$.
    В одном из экспериментов объём газа уменьшился в $4$ раза.
    Как при этом изменилась температура газа (выросла или уменьшилась, на сколько или во сколько раз)?
}
\answer{%
    \begin{align*}
    P_1V_1^n &= P_2V_2^n, P_1V_1 = \nu R T_1, P_2V_2 = \nu R T_2 \implies\frac{ \nu R T_1 }{ V_1 } V_1^n = \frac{ \nu R T_2 }{ V_2 } V_2^n \implies \\
    \implies T_1V_1^{ n-1 } &= T_2V_2^{ n - 1 } \implies\frac{ T_2 }{ T_1 } = \cbr{ \frac{ V_1 }{ V_2 } }^{ n-1 } \approx 0{,}660
    \end{align*}
}

\variantsplitter

\addpersonalvariant{Волохов Вячеслав}

\tasknumber{1}%
\task{%
    Из уравнения состояния идеального газа выведите или выразите...
    \begin{enumerate}
        \item давление,
        \item молярную массу,
        \item плотность газа.
    \end{enumerate}
}
\solutionspace{40pt}

\tasknumber{2}%
\task{%
    Изобразите в координатах $PV$, соблюдая масштаб, процесс 1234,
    в котором 12 — изохорическое охлаждение в 3 раза,
    23 — изотермическое расширение в 2 раза,
    34 — изобарическое охлаждение в 3 раза.
}
\solutionspace{160pt}

\tasknumber{3}%
\task{%
    Небольшую цилиндрическую пробирку с воздухом погружают на некоторую глубину в глубокое пресное озеро,
    после чего воздух занимает в ней лишь третью часть от общего объема.
    Определите глубину, на которую погрузили пробирку.
    Температуру считать постоянной $T = 280\,\text{К}$, давлением паров воды пренебречь,
    атмосферное давление принять равным $p_{\text{aтм}} = 100\,\text{кПа}$.
}
\answer{%
    \begin{align*}
    T\text{ — const } &\implies P_1V_1 = \nu RT = P_2V_2.
    \\
    V_2 = \frac 13 V_1 &\implies P_1V_1 = P_2\cdot \frac 13V_1 \implies P_2 = 3P_1 = 3p_{\text{aтм}}.
    \\
    P_2 = p_{\text{aтм}} + \rho_{\text{в}} g h \implies h = \frac{ P_2 - p_{\text{aтм}} }{ \rho_{\text{в}} g } &= \frac{ 3p_{\text{aтм}} - p_{\text{aтм}} }{ \rho_{\text{в}} g } = \frac{ 2 \cdot p_{\text{aтм}} }{ \rho_{\text{в}} g } =  \\
     &= \frac{ 2 \cdot 100\,\text{кПа} }{ 1000\,\frac{\text{кг}}{\text{м}^{3}} \cdot 10\,\frac{\text{м}}{\text{с}^{2}} } \approx 20\,\text{м}.
    \end{align*}
}
\solutionspace{120pt}

\tasknumber{4}%
\task{%
    В замкнутом сосуде объёмом $2\,\text{л}$ находится воздух ($\mu = 29\,\frac{\text{г}}{\text{моль}}$) под давлением $4{,}5\units{ атм }$.
    Определите массу газа в сосуде и выразите её в граммах, приняв температуру газа равной $7\celsius$.
}
\answer{%
    $
        PV = \frac m\mu RT \implies m = \frac{ PV \mu }{ RT } =
        \frac{ 4{,}5\,\text{атм} \cdot2\,\text{л} \cdot 29\,\frac{\text{г}}{\text{моль}} }{ 8{,}31\,\frac{\text{Дж}}{\text{моль}\cdot\text{К}} \cdot \cbr{ 7 + 273 }\units{К} }
        \approx 11{,}22\,\text{г}.
    $
}
\solutionspace{120pt}

\tasknumber{5}%
\task{%
    Идеальный газ в экспериментальной установке подвергут политропному процессу $PV^n\text{ — const }$
    с показателем политропы $n=0{,}4$.
    В одном из экспериментов объём газа уменьшился в $2$ раза.
    Как при этом изменилась температура газа (выросла или уменьшилась, на сколько или во сколько раз)?
}
\answer{%
    \begin{align*}
    P_1V_1^n &= P_2V_2^n, P_1V_1 = \nu R T_1, P_2V_2 = \nu R T_2 \implies\frac{ \nu R T_1 }{ V_1 } V_1^n = \frac{ \nu R T_2 }{ V_2 } V_2^n \implies \\
    \implies T_1V_1^{ n-1 } &= T_2V_2^{ n - 1 } \implies\frac{ T_2 }{ T_1 } = \cbr{ \frac{ V_1 }{ V_2 } }^{ n-1 } \approx 0{,}660
    \end{align*}
}

\variantsplitter

\addpersonalvariant{Герман Говоров}

\tasknumber{1}%
\task{%
    Из уравнения состояния идеального газа выведите или выразите...
    \begin{enumerate}
        \item давление,
        \item температуру,
        \item плотность газа.
    \end{enumerate}
}
\solutionspace{40pt}

\tasknumber{2}%
\task{%
    Изобразите в координатах $PV$, соблюдая масштаб, процесс 1234,
    в котором 12 — изобарическое охлаждение в 3 раза,
    23 — изотермическое расширение в 3 раза,
    34 — изобарическое охлаждение в 3 раза.
}
\solutionspace{160pt}

\tasknumber{3}%
\task{%
    Небольшую цилиндрическую пробирку с воздухом погружают на некоторую глубину в глубокое пресное озеро,
    после чего воздух занимает в ней лишь третью часть от общего объема.
    Определите глубину, на которую погрузили пробирку.
    Температуру считать постоянной $T = 292\,\text{К}$, давлением паров воды пренебречь,
    атмосферное давление принять равным $p_{\text{aтм}} = 100\,\text{кПа}$.
}
\answer{%
    \begin{align*}
    T\text{ — const } &\implies P_1V_1 = \nu RT = P_2V_2.
    \\
    V_2 = \frac 13 V_1 &\implies P_1V_1 = P_2\cdot \frac 13V_1 \implies P_2 = 3P_1 = 3p_{\text{aтм}}.
    \\
    P_2 = p_{\text{aтм}} + \rho_{\text{в}} g h \implies h = \frac{ P_2 - p_{\text{aтм}} }{ \rho_{\text{в}} g } &= \frac{ 3p_{\text{aтм}} - p_{\text{aтм}} }{ \rho_{\text{в}} g } = \frac{ 2 \cdot p_{\text{aтм}} }{ \rho_{\text{в}} g } =  \\
     &= \frac{ 2 \cdot 100\,\text{кПа} }{ 1000\,\frac{\text{кг}}{\text{м}^{3}} \cdot 10\,\frac{\text{м}}{\text{с}^{2}} } \approx 20\,\text{м}.
    \end{align*}
}
\solutionspace{120pt}

\tasknumber{4}%
\task{%
    В замкнутом сосуде объёмом $3\,\text{л}$ находится воздух ($\mu = 29\,\frac{\text{г}}{\text{моль}}$) под давлением $4{,}5\units{ атм }$.
    Определите массу газа в сосуде и выразите её в граммах, приняв температуру газа равной $7\celsius$.
}
\answer{%
    $
        PV = \frac m\mu RT \implies m = \frac{ PV \mu }{ RT } =
        \frac{ 4{,}5\,\text{атм} \cdot3\,\text{л} \cdot 29\,\frac{\text{г}}{\text{моль}} }{ 8{,}31\,\frac{\text{Дж}}{\text{моль}\cdot\text{К}} \cdot \cbr{ 7 + 273 }\units{К} }
        \approx 16{,}83\,\text{г}.
    $
}
\solutionspace{120pt}

\tasknumber{5}%
\task{%
    Идеальный газ в экспериментальной установке подвергут политропному процессу $PV^n\text{ — const }$
    с показателем политропы $n=0{,}7$.
    В одном из экспериментов объём газа увеличился в $2$ раза.
    Как при этом изменилась температура газа (выросла или уменьшилась, на сколько или во сколько раз)?
}
\answer{%
    \begin{align*}
    P_1V_1^n &= P_2V_2^n, P_1V_1 = \nu R T_1, P_2V_2 = \nu R T_2 \implies\frac{ \nu R T_1 }{ V_1 } V_1^n = \frac{ \nu R T_2 }{ V_2 } V_2^n \implies \\
    \implies T_1V_1^{ n-1 } &= T_2V_2^{ n - 1 } \implies\frac{ T_2 }{ T_1 } = \cbr{ \frac{ V_1 }{ V_2 } }^{ n-1 } \approx 1{,}231
    \end{align*}
}

\variantsplitter

\addpersonalvariant{София Журавлева}

\tasknumber{1}%
\task{%
    Из уравнения состояния идеального газа выведите или выразите...
    \begin{enumerate}
        \item объём,
        \item молярную массу,
        \item концентрацию молекул газа.
    \end{enumerate}
}
\solutionspace{40pt}

\tasknumber{2}%
\task{%
    Изобразите в координатах $PV$, соблюдая масштаб, процесс 1234,
    в котором 12 — изохорическое охлаждение в 2 раза,
    23 — изотермическое расширение в 2 раза,
    34 — изобарическое нагревание в 2 раза.
}
\solutionspace{160pt}

\tasknumber{3}%
\task{%
    Небольшую цилиндрическую пробирку с воздухом погружают на некоторую глубину в глубокое пресное озеро,
    после чего воздух занимает в ней лишь четвертую часть от общего объема.
    Определите глубину, на которую погрузили пробирку.
    Температуру считать постоянной $T = 289\,\text{К}$, давлением паров воды пренебречь,
    атмосферное давление принять равным $p_{\text{aтм}} = 100\,\text{кПа}$.
}
\answer{%
    \begin{align*}
    T\text{ — const } &\implies P_1V_1 = \nu RT = P_2V_2.
    \\
    V_2 = \frac 14 V_1 &\implies P_1V_1 = P_2\cdot \frac 14V_1 \implies P_2 = 4P_1 = 4p_{\text{aтм}}.
    \\
    P_2 = p_{\text{aтм}} + \rho_{\text{в}} g h \implies h = \frac{ P_2 - p_{\text{aтм}} }{ \rho_{\text{в}} g } &= \frac{ 4p_{\text{aтм}} - p_{\text{aтм}} }{ \rho_{\text{в}} g } = \frac{ 3 \cdot p_{\text{aтм}} }{ \rho_{\text{в}} g } =  \\
     &= \frac{ 3 \cdot 100\,\text{кПа} }{ 1000\,\frac{\text{кг}}{\text{м}^{3}} \cdot 10\,\frac{\text{м}}{\text{с}^{2}} } \approx 30\,\text{м}.
    \end{align*}
}
\solutionspace{120pt}

\tasknumber{4}%
\task{%
    В замкнутом сосуде объёмом $5\,\text{л}$ находится кислород ($\mu = 32\,\frac{\text{г}}{\text{моль}}$) под давлением $4{,}5\units{ атм }$.
    Определите массу газа в сосуде и выразите её в граммах, приняв температуру газа равной $27\celsius$.
}
\answer{%
    $
        PV = \frac m\mu RT \implies m = \frac{ PV \mu }{ RT } =
        \frac{ 4{,}5\,\text{атм} \cdot5\,\text{л} \cdot 32\,\frac{\text{г}}{\text{моль}} }{ 8{,}31\,\frac{\text{Дж}}{\text{моль}\cdot\text{К}} \cdot \cbr{ 27 + 273 }\units{К} }
        \approx 28{,}88\,\text{г}.
    $
}
\solutionspace{120pt}

\tasknumber{5}%
\task{%
    Идеальный газ в экспериментальной установке подвергут политропному процессу $PV^n\text{ — const }$
    с показателем политропы $n=1{,}5$.
    В одном из экспериментов объём газа увеличился в $3$ раза.
    Как при этом изменилась температура газа (выросла или уменьшилась, на сколько или во сколько раз)?
}
\answer{%
    \begin{align*}
    P_1V_1^n &= P_2V_2^n, P_1V_1 = \nu R T_1, P_2V_2 = \nu R T_2 \implies\frac{ \nu R T_1 }{ V_1 } V_1^n = \frac{ \nu R T_2 }{ V_2 } V_2^n \implies \\
    \implies T_1V_1^{ n-1 } &= T_2V_2^{ n - 1 } \implies\frac{ T_2 }{ T_1 } = \cbr{ \frac{ V_1 }{ V_2 } }^{ n-1 } \approx 0{,}577
    \end{align*}
}

\variantsplitter

\addpersonalvariant{Константин Козлов}

\tasknumber{1}%
\task{%
    Из уравнения состояния идеального газа выведите или выразите...
    \begin{enumerate}
        \item давление,
        \item молярную массу,
        \item концентрацию молекул газа.
    \end{enumerate}
}
\solutionspace{40pt}

\tasknumber{2}%
\task{%
    Изобразите в координатах $PV$, соблюдая масштаб, процесс 1234,
    в котором 12 — изобарическое охлаждение в 3 раза,
    23 — изотермическое расширение в 3 раза,
    34 — изохорическое охлаждение в 3 раза.
}
\solutionspace{160pt}

\tasknumber{3}%
\task{%
    Небольшую цилиндрическую пробирку с воздухом погружают на некоторую глубину в глубокое пресное озеро,
    после чего воздух занимает в ней лишь пятую часть от общего объема.
    Определите глубину, на которую погрузили пробирку.
    Температуру считать постоянной $T = 292\,\text{К}$, давлением паров воды пренебречь,
    атмосферное давление принять равным $p_{\text{aтм}} = 100\,\text{кПа}$.
}
\answer{%
    \begin{align*}
    T\text{ — const } &\implies P_1V_1 = \nu RT = P_2V_2.
    \\
    V_2 = \frac 15 V_1 &\implies P_1V_1 = P_2\cdot \frac 15V_1 \implies P_2 = 5P_1 = 5p_{\text{aтм}}.
    \\
    P_2 = p_{\text{aтм}} + \rho_{\text{в}} g h \implies h = \frac{ P_2 - p_{\text{aтм}} }{ \rho_{\text{в}} g } &= \frac{ 5p_{\text{aтм}} - p_{\text{aтм}} }{ \rho_{\text{в}} g } = \frac{ 4 \cdot p_{\text{aтм}} }{ \rho_{\text{в}} g } =  \\
     &= \frac{ 4 \cdot 100\,\text{кПа} }{ 1000\,\frac{\text{кг}}{\text{м}^{3}} \cdot 10\,\frac{\text{м}}{\text{с}^{2}} } \approx 40\,\text{м}.
    \end{align*}
}
\solutionspace{120pt}

\tasknumber{4}%
\task{%
    В замкнутом сосуде объёмом $5\,\text{л}$ находится кислород ($\mu = 32\,\frac{\text{г}}{\text{моль}}$) под давлением $3\units{ атм }$.
    Определите массу газа в сосуде и выразите её в граммах, приняв температуру газа равной $7\celsius$.
}
\answer{%
    $
        PV = \frac m\mu RT \implies m = \frac{ PV \mu }{ RT } =
        \frac{ 3{,}0\,\text{атм} \cdot5\,\text{л} \cdot 32\,\frac{\text{г}}{\text{моль}} }{ 8{,}31\,\frac{\text{Дж}}{\text{моль}\cdot\text{К}} \cdot \cbr{ 7 + 273 }\units{К} }
        \approx 20{,}63\,\text{г}.
    $
}
\solutionspace{120pt}

\tasknumber{5}%
\task{%
    Идеальный газ в экспериментальной установке подвергут политропному процессу $PV^n\text{ — const }$
    с показателем политропы $n=1{,}8$.
    В одном из экспериментов объём газа уменьшился в $3$ раза.
    Как при этом изменилась температура газа (выросла или уменьшилась, на сколько или во сколько раз)?
}
\answer{%
    \begin{align*}
    P_1V_1^n &= P_2V_2^n, P_1V_1 = \nu R T_1, P_2V_2 = \nu R T_2 \implies\frac{ \nu R T_1 }{ V_1 } V_1^n = \frac{ \nu R T_2 }{ V_2 } V_2^n \implies \\
    \implies T_1V_1^{ n-1 } &= T_2V_2^{ n - 1 } \implies\frac{ T_2 }{ T_1 } = \cbr{ \frac{ V_1 }{ V_2 } }^{ n-1 } \approx 2{,}408
    \end{align*}
}

\variantsplitter

\addpersonalvariant{Кузьмин Матвей}

\tasknumber{1}%
\task{%
    Из уравнения состояния идеального газа выведите или выразите...
    \begin{enumerate}
        \item давление,
        \item температуру,
        \item концентрацию молекул газа.
    \end{enumerate}
}
\solutionspace{40pt}

\tasknumber{2}%
\task{%
    Изобразите в координатах $PV$, соблюдая масштаб, процесс 1234,
    в котором 12 — изохорическое охлаждение в 3 раза,
    23 — изотермическое расширение в 3 раза,
    34 — изобарическое охлаждение в 2 раза.
}
\solutionspace{160pt}

\tasknumber{3}%
\task{%
    Небольшую цилиндрическую пробирку с воздухом погружают на некоторую глубину в глубокое пресное озеро,
    после чего воздух занимает в ней лишь третью часть от общего объема.
    Определите глубину, на которую погрузили пробирку.
    Температуру считать постоянной $T = 282\,\text{К}$, давлением паров воды пренебречь,
    атмосферное давление принять равным $p_{\text{aтм}} = 100\,\text{кПа}$.
}
\answer{%
    \begin{align*}
    T\text{ — const } &\implies P_1V_1 = \nu RT = P_2V_2.
    \\
    V_2 = \frac 13 V_1 &\implies P_1V_1 = P_2\cdot \frac 13V_1 \implies P_2 = 3P_1 = 3p_{\text{aтм}}.
    \\
    P_2 = p_{\text{aтм}} + \rho_{\text{в}} g h \implies h = \frac{ P_2 - p_{\text{aтм}} }{ \rho_{\text{в}} g } &= \frac{ 3p_{\text{aтм}} - p_{\text{aтм}} }{ \rho_{\text{в}} g } = \frac{ 2 \cdot p_{\text{aтм}} }{ \rho_{\text{в}} g } =  \\
     &= \frac{ 2 \cdot 100\,\text{кПа} }{ 1000\,\frac{\text{кг}}{\text{м}^{3}} \cdot 10\,\frac{\text{м}}{\text{с}^{2}} } \approx 20\,\text{м}.
    \end{align*}
}
\solutionspace{120pt}

\tasknumber{4}%
\task{%
    В замкнутом сосуде объёмом $2\,\text{л}$ находится азот ($\mu = 28\,\frac{\text{г}}{\text{моль}}$) под давлением $3\units{ атм }$.
    Определите массу газа в сосуде и выразите её в граммах, приняв температуру газа равной $17\celsius$.
}
\answer{%
    $
        PV = \frac m\mu RT \implies m = \frac{ PV \mu }{ RT } =
        \frac{ 3{,}0\,\text{атм} \cdot2\,\text{л} \cdot 28\,\frac{\text{г}}{\text{моль}} }{ 8{,}31\,\frac{\text{Дж}}{\text{моль}\cdot\text{К}} \cdot \cbr{ 17 + 273 }\units{К} }
        \approx 6{,}97\,\text{г}.
    $
}
\solutionspace{120pt}

\tasknumber{5}%
\task{%
    Идеальный газ в экспериментальной установке подвергут политропному процессу $PV^n\text{ — const }$
    с показателем политропы $n=0{,}7$.
    В одном из экспериментов объём газа уменьшился в $4$ раза.
    Как при этом изменилась температура газа (выросла или уменьшилась, на сколько или во сколько раз)?
}
\answer{%
    \begin{align*}
    P_1V_1^n &= P_2V_2^n, P_1V_1 = \nu R T_1, P_2V_2 = \nu R T_2 \implies\frac{ \nu R T_1 }{ V_1 } V_1^n = \frac{ \nu R T_2 }{ V_2 } V_2^n \implies \\
    \implies T_1V_1^{ n-1 } &= T_2V_2^{ n - 1 } \implies\frac{ T_2 }{ T_1 } = \cbr{ \frac{ V_1 }{ V_2 } }^{ n-1 } \approx 0{,}660
    \end{align*}
}

\variantsplitter

\addpersonalvariant{Наталья Кравченко}

\tasknumber{1}%
\task{%
    Из уравнения состояния идеального газа выведите или выразите...
    \begin{enumerate}
        \item объём,
        \item температуру,
        \item плотность газа.
    \end{enumerate}
}
\solutionspace{40pt}

\tasknumber{2}%
\task{%
    Изобразите в координатах $PV$, соблюдая масштаб, процесс 1234,
    в котором 12 — изобарическое охлаждение в 2 раза,
    23 — изотермическое расширение в 3 раза,
    34 — изобарическое нагревание в 3 раза.
}
\solutionspace{160pt}

\tasknumber{3}%
\task{%
    Небольшую цилиндрическую пробирку с воздухом погружают на некоторую глубину в глубокое пресное озеро,
    после чего воздух занимает в ней лишь четвертую часть от общего объема.
    Определите глубину, на которую погрузили пробирку.
    Температуру считать постоянной $T = 288\,\text{К}$, давлением паров воды пренебречь,
    атмосферное давление принять равным $p_{\text{aтм}} = 100\,\text{кПа}$.
}
\answer{%
    \begin{align*}
    T\text{ — const } &\implies P_1V_1 = \nu RT = P_2V_2.
    \\
    V_2 = \frac 14 V_1 &\implies P_1V_1 = P_2\cdot \frac 14V_1 \implies P_2 = 4P_1 = 4p_{\text{aтм}}.
    \\
    P_2 = p_{\text{aтм}} + \rho_{\text{в}} g h \implies h = \frac{ P_2 - p_{\text{aтм}} }{ \rho_{\text{в}} g } &= \frac{ 4p_{\text{aтм}} - p_{\text{aтм}} }{ \rho_{\text{в}} g } = \frac{ 3 \cdot p_{\text{aтм}} }{ \rho_{\text{в}} g } =  \\
     &= \frac{ 3 \cdot 100\,\text{кПа} }{ 1000\,\frac{\text{кг}}{\text{м}^{3}} \cdot 10\,\frac{\text{м}}{\text{с}^{2}} } \approx 30\,\text{м}.
    \end{align*}
}
\solutionspace{120pt}

\tasknumber{4}%
\task{%
    В замкнутом сосуде объёмом $4\,\text{л}$ находится аргон ($\mu = 40\,\frac{\text{г}}{\text{моль}}$) под давлением $5\units{ атм }$.
    Определите массу газа в сосуде и выразите её в граммах, приняв температуру газа равной $17\celsius$.
}
\answer{%
    $
        PV = \frac m\mu RT \implies m = \frac{ PV \mu }{ RT } =
        \frac{ 5{,}0\,\text{атм} \cdot4\,\text{л} \cdot 40\,\frac{\text{г}}{\text{моль}} }{ 8{,}31\,\frac{\text{Дж}}{\text{моль}\cdot\text{К}} \cdot \cbr{ 17 + 273 }\units{К} }
        \approx 33{,}20\,\text{г}.
    $
}
\solutionspace{120pt}

\tasknumber{5}%
\task{%
    Идеальный газ в экспериментальной установке подвергут политропному процессу $PV^n\text{ — const }$
    с показателем политропы $n=1{,}2$.
    В одном из экспериментов объём газа уменьшился в $4$ раза.
    Как при этом изменилась температура газа (выросла или уменьшилась, на сколько или во сколько раз)?
}
\answer{%
    \begin{align*}
    P_1V_1^n &= P_2V_2^n, P_1V_1 = \nu R T_1, P_2V_2 = \nu R T_2 \implies\frac{ \nu R T_1 }{ V_1 } V_1^n = \frac{ \nu R T_2 }{ V_2 } V_2^n \implies \\
    \implies T_1V_1^{ n-1 } &= T_2V_2^{ n - 1 } \implies\frac{ T_2 }{ T_1 } = \cbr{ \frac{ V_1 }{ V_2 } }^{ n-1 } \approx 1{,}320
    \end{align*}
}

\variantsplitter

\addpersonalvariant{Сергей Малышев}

\tasknumber{1}%
\task{%
    Из уравнения состояния идеального газа выведите или выразите...
    \begin{enumerate}
        \item объём,
        \item молярную массу,
        \item плотность газа.
    \end{enumerate}
}
\solutionspace{40pt}

\tasknumber{2}%
\task{%
    Изобразите в координатах $PV$, соблюдая масштаб, процесс 1234,
    в котором 12 — изохорическое охлаждение в 3 раза,
    23 — изотермическое сжатие в 2 раза,
    34 — изохорическое охлаждение в 2 раза.
}
\solutionspace{160pt}

\tasknumber{3}%
\task{%
    Небольшую цилиндрическую пробирку с воздухом погружают на некоторую глубину в глубокое пресное озеро,
    после чего воздух занимает в ней лишь третью часть от общего объема.
    Определите глубину, на которую погрузили пробирку.
    Температуру считать постоянной $T = 279\,\text{К}$, давлением паров воды пренебречь,
    атмосферное давление принять равным $p_{\text{aтм}} = 100\,\text{кПа}$.
}
\answer{%
    \begin{align*}
    T\text{ — const } &\implies P_1V_1 = \nu RT = P_2V_2.
    \\
    V_2 = \frac 13 V_1 &\implies P_1V_1 = P_2\cdot \frac 13V_1 \implies P_2 = 3P_1 = 3p_{\text{aтм}}.
    \\
    P_2 = p_{\text{aтм}} + \rho_{\text{в}} g h \implies h = \frac{ P_2 - p_{\text{aтм}} }{ \rho_{\text{в}} g } &= \frac{ 3p_{\text{aтм}} - p_{\text{aтм}} }{ \rho_{\text{в}} g } = \frac{ 2 \cdot p_{\text{aтм}} }{ \rho_{\text{в}} g } =  \\
     &= \frac{ 2 \cdot 100\,\text{кПа} }{ 1000\,\frac{\text{кг}}{\text{м}^{3}} \cdot 10\,\frac{\text{м}}{\text{с}^{2}} } \approx 20\,\text{м}.
    \end{align*}
}
\solutionspace{120pt}

\tasknumber{4}%
\task{%
    В замкнутом сосуде объёмом $4\,\text{л}$ находится углекислый газ ($\mu = 44\,\frac{\text{г}}{\text{моль}}$) под давлением $2{,}5\units{ атм }$.
    Определите массу газа в сосуде и выразите её в граммах, приняв температуру газа равной $37\celsius$.
}
\answer{%
    $
        PV = \frac m\mu RT \implies m = \frac{ PV \mu }{ RT } =
        \frac{ 2{,}5\,\text{атм} \cdot4\,\text{л} \cdot 44\,\frac{\text{г}}{\text{моль}} }{ 8{,}31\,\frac{\text{Дж}}{\text{моль}\cdot\text{К}} \cdot \cbr{ 37 + 273 }\units{К} }
        \approx 17{,}08\,\text{г}.
    $
}
\solutionspace{120pt}

\tasknumber{5}%
\task{%
    Идеальный газ в экспериментальной установке подвергут политропному процессу $PV^n\text{ — const }$
    с показателем политропы $n=1{,}2$.
    В одном из экспериментов объём газа увеличился в $4$ раза.
    Как при этом изменилась температура газа (выросла или уменьшилась, на сколько или во сколько раз)?
}
\answer{%
    \begin{align*}
    P_1V_1^n &= P_2V_2^n, P_1V_1 = \nu R T_1, P_2V_2 = \nu R T_2 \implies\frac{ \nu R T_1 }{ V_1 } V_1^n = \frac{ \nu R T_2 }{ V_2 } V_2^n \implies \\
    \implies T_1V_1^{ n-1 } &= T_2V_2^{ n - 1 } \implies\frac{ T_2 }{ T_1 } = \cbr{ \frac{ V_1 }{ V_2 } }^{ n-1 } \approx 0{,}758
    \end{align*}
}

\variantsplitter

\addpersonalvariant{Алина Полканова}

\tasknumber{1}%
\task{%
    Из уравнения состояния идеального газа выведите или выразите...
    \begin{enumerate}
        \item давление,
        \item температуру,
        \item плотность газа.
    \end{enumerate}
}
\solutionspace{40pt}

\tasknumber{2}%
\task{%
    Изобразите в координатах $PV$, соблюдая масштаб, процесс 1234,
    в котором 12 — изобарическое нагревание в 3 раза,
    23 — изотермическое расширение в 3 раза,
    34 — изохорическое нагревание в 3 раза.
}
\solutionspace{160pt}

\tasknumber{3}%
\task{%
    Небольшую цилиндрическую пробирку с воздухом погружают на некоторую глубину в глубокое пресное озеро,
    после чего воздух занимает в ней лишь третью часть от общего объема.
    Определите глубину, на которую погрузили пробирку.
    Температуру считать постоянной $T = 290\,\text{К}$, давлением паров воды пренебречь,
    атмосферное давление принять равным $p_{\text{aтм}} = 100\,\text{кПа}$.
}
\answer{%
    \begin{align*}
    T\text{ — const } &\implies P_1V_1 = \nu RT = P_2V_2.
    \\
    V_2 = \frac 13 V_1 &\implies P_1V_1 = P_2\cdot \frac 13V_1 \implies P_2 = 3P_1 = 3p_{\text{aтм}}.
    \\
    P_2 = p_{\text{aтм}} + \rho_{\text{в}} g h \implies h = \frac{ P_2 - p_{\text{aтм}} }{ \rho_{\text{в}} g } &= \frac{ 3p_{\text{aтм}} - p_{\text{aтм}} }{ \rho_{\text{в}} g } = \frac{ 2 \cdot p_{\text{aтм}} }{ \rho_{\text{в}} g } =  \\
     &= \frac{ 2 \cdot 100\,\text{кПа} }{ 1000\,\frac{\text{кг}}{\text{м}^{3}} \cdot 10\,\frac{\text{м}}{\text{с}^{2}} } \approx 20\,\text{м}.
    \end{align*}
}
\solutionspace{120pt}

\tasknumber{4}%
\task{%
    В замкнутом сосуде объёмом $3\,\text{л}$ находится азот ($\mu = 28\,\frac{\text{г}}{\text{моль}}$) под давлением $3\units{ атм }$.
    Определите массу газа в сосуде и выразите её в граммах, приняв температуру газа равной $37\celsius$.
}
\answer{%
    $
        PV = \frac m\mu RT \implies m = \frac{ PV \mu }{ RT } =
        \frac{ 3{,}0\,\text{атм} \cdot3\,\text{л} \cdot 28\,\frac{\text{г}}{\text{моль}} }{ 8{,}31\,\frac{\text{Дж}}{\text{моль}\cdot\text{К}} \cdot \cbr{ 37 + 273 }\units{К} }
        \approx 9{,}78\,\text{г}.
    $
}
\solutionspace{120pt}

\tasknumber{5}%
\task{%
    Идеальный газ в экспериментальной установке подвергут политропному процессу $PV^n\text{ — const }$
    с показателем политропы $n=1{,}8$.
    В одном из экспериментов объём газа уменьшился в $3$ раза.
    Как при этом изменилась температура газа (выросла или уменьшилась, на сколько или во сколько раз)?
}
\answer{%
    \begin{align*}
    P_1V_1^n &= P_2V_2^n, P_1V_1 = \nu R T_1, P_2V_2 = \nu R T_2 \implies\frac{ \nu R T_1 }{ V_1 } V_1^n = \frac{ \nu R T_2 }{ V_2 } V_2^n \implies \\
    \implies T_1V_1^{ n-1 } &= T_2V_2^{ n - 1 } \implies\frac{ T_2 }{ T_1 } = \cbr{ \frac{ V_1 }{ V_2 } }^{ n-1 } \approx 2{,}408
    \end{align*}
}

\variantsplitter

\addpersonalvariant{Сергей Пономарёв}

\tasknumber{1}%
\task{%
    Из уравнения состояния идеального газа выведите или выразите...
    \begin{enumerate}
        \item давление,
        \item температуру,
        \item плотность газа.
    \end{enumerate}
}
\solutionspace{40pt}

\tasknumber{2}%
\task{%
    Изобразите в координатах $PV$, соблюдая масштаб, процесс 1234,
    в котором 12 — изобарическое охлаждение в 2 раза,
    23 — изотермическое сжатие в 2 раза,
    34 — изобарическое охлаждение в 2 раза.
}
\solutionspace{160pt}

\tasknumber{3}%
\task{%
    Небольшую цилиндрическую пробирку с воздухом погружают на некоторую глубину в глубокое пресное озеро,
    после чего воздух занимает в ней лишь четвертую часть от общего объема.
    Определите глубину, на которую погрузили пробирку.
    Температуру считать постоянной $T = 282\,\text{К}$, давлением паров воды пренебречь,
    атмосферное давление принять равным $p_{\text{aтм}} = 100\,\text{кПа}$.
}
\answer{%
    \begin{align*}
    T\text{ — const } &\implies P_1V_1 = \nu RT = P_2V_2.
    \\
    V_2 = \frac 14 V_1 &\implies P_1V_1 = P_2\cdot \frac 14V_1 \implies P_2 = 4P_1 = 4p_{\text{aтм}}.
    \\
    P_2 = p_{\text{aтм}} + \rho_{\text{в}} g h \implies h = \frac{ P_2 - p_{\text{aтм}} }{ \rho_{\text{в}} g } &= \frac{ 4p_{\text{aтм}} - p_{\text{aтм}} }{ \rho_{\text{в}} g } = \frac{ 3 \cdot p_{\text{aтм}} }{ \rho_{\text{в}} g } =  \\
     &= \frac{ 3 \cdot 100\,\text{кПа} }{ 1000\,\frac{\text{кг}}{\text{м}^{3}} \cdot 10\,\frac{\text{м}}{\text{с}^{2}} } \approx 30\,\text{м}.
    \end{align*}
}
\solutionspace{120pt}

\tasknumber{4}%
\task{%
    В замкнутом сосуде объёмом $4\,\text{л}$ находится неон ($\mu = 20\,\frac{\text{г}}{\text{моль}}$) под давлением $2{,}5\units{ атм }$.
    Определите массу газа в сосуде и выразите её в граммах, приняв температуру газа равной $37\celsius$.
}
\answer{%
    $
        PV = \frac m\mu RT \implies m = \frac{ PV \mu }{ RT } =
        \frac{ 2{,}5\,\text{атм} \cdot4\,\text{л} \cdot 20\,\frac{\text{г}}{\text{моль}} }{ 8{,}31\,\frac{\text{Дж}}{\text{моль}\cdot\text{К}} \cdot \cbr{ 37 + 273 }\units{К} }
        \approx 7{,}76\,\text{г}.
    $
}
\solutionspace{120pt}

\tasknumber{5}%
\task{%
    Идеальный газ в экспериментальной установке подвергут политропному процессу $PV^n\text{ — const }$
    с показателем политропы $n=1{,}5$.
    В одном из экспериментов объём газа увеличился в $4$ раза.
    Как при этом изменилась температура газа (выросла или уменьшилась, на сколько или во сколько раз)?
}
\answer{%
    \begin{align*}
    P_1V_1^n &= P_2V_2^n, P_1V_1 = \nu R T_1, P_2V_2 = \nu R T_2 \implies\frac{ \nu R T_1 }{ V_1 } V_1^n = \frac{ \nu R T_2 }{ V_2 } V_2^n \implies \\
    \implies T_1V_1^{ n-1 } &= T_2V_2^{ n - 1 } \implies\frac{ T_2 }{ T_1 } = \cbr{ \frac{ V_1 }{ V_2 } }^{ n-1 } \approx 0{,}500
    \end{align*}
}

\variantsplitter

\addpersonalvariant{Егор Свистушкин}

\tasknumber{1}%
\task{%
    Из уравнения состояния идеального газа выведите или выразите...
    \begin{enumerate}
        \item объём,
        \item молярную массу,
        \item плотность газа.
    \end{enumerate}
}
\solutionspace{40pt}

\tasknumber{2}%
\task{%
    Изобразите в координатах $PV$, соблюдая масштаб, процесс 1234,
    в котором 12 — изобарическое нагревание в 3 раза,
    23 — изотермическое расширение в 3 раза,
    34 — изобарическое охлаждение в 2 раза.
}
\solutionspace{160pt}

\tasknumber{3}%
\task{%
    Небольшую цилиндрическую пробирку с воздухом погружают на некоторую глубину в глубокое пресное озеро,
    после чего воздух занимает в ней лишь шестую часть от общего объема.
    Определите глубину, на которую погрузили пробирку.
    Температуру считать постоянной $T = 292\,\text{К}$, давлением паров воды пренебречь,
    атмосферное давление принять равным $p_{\text{aтм}} = 100\,\text{кПа}$.
}
\answer{%
    \begin{align*}
    T\text{ — const } &\implies P_1V_1 = \nu RT = P_2V_2.
    \\
    V_2 = \frac 16 V_1 &\implies P_1V_1 = P_2\cdot \frac 16V_1 \implies P_2 = 6P_1 = 6p_{\text{aтм}}.
    \\
    P_2 = p_{\text{aтм}} + \rho_{\text{в}} g h \implies h = \frac{ P_2 - p_{\text{aтм}} }{ \rho_{\text{в}} g } &= \frac{ 6p_{\text{aтм}} - p_{\text{aтм}} }{ \rho_{\text{в}} g } = \frac{ 5 \cdot p_{\text{aтм}} }{ \rho_{\text{в}} g } =  \\
     &= \frac{ 5 \cdot 100\,\text{кПа} }{ 1000\,\frac{\text{кг}}{\text{м}^{3}} \cdot 10\,\frac{\text{м}}{\text{с}^{2}} } \approx 50\,\text{м}.
    \end{align*}
}
\solutionspace{120pt}

\tasknumber{4}%
\task{%
    В замкнутом сосуде объёмом $5\,\text{л}$ находится неон ($\mu = 20\,\frac{\text{г}}{\text{моль}}$) под давлением $5\units{ атм }$.
    Определите массу газа в сосуде и выразите её в граммах, приняв температуру газа равной $37\celsius$.
}
\answer{%
    $
        PV = \frac m\mu RT \implies m = \frac{ PV \mu }{ RT } =
        \frac{ 5{,}0\,\text{атм} \cdot5\,\text{л} \cdot 20\,\frac{\text{г}}{\text{моль}} }{ 8{,}31\,\frac{\text{Дж}}{\text{моль}\cdot\text{К}} \cdot \cbr{ 37 + 273 }\units{К} }
        \approx 19{,}41\,\text{г}.
    $
}
\solutionspace{120pt}

\tasknumber{5}%
\task{%
    Идеальный газ в экспериментальной установке подвергут политропному процессу $PV^n\text{ — const }$
    с показателем политропы $n=1{,}2$.
    В одном из экспериментов объём газа увеличился в $4$ раза.
    Как при этом изменилась температура газа (выросла или уменьшилась, на сколько или во сколько раз)?
}
\answer{%
    \begin{align*}
    P_1V_1^n &= P_2V_2^n, P_1V_1 = \nu R T_1, P_2V_2 = \nu R T_2 \implies\frac{ \nu R T_1 }{ V_1 } V_1^n = \frac{ \nu R T_2 }{ V_2 } V_2^n \implies \\
    \implies T_1V_1^{ n-1 } &= T_2V_2^{ n - 1 } \implies\frac{ T_2 }{ T_1 } = \cbr{ \frac{ V_1 }{ V_2 } }^{ n-1 } \approx 0{,}758
    \end{align*}
}

\variantsplitter

\addpersonalvariant{Дмитрий Соколов}

\tasknumber{1}%
\task{%
    Из уравнения состояния идеального газа выведите или выразите...
    \begin{enumerate}
        \item давление,
        \item молярную массу,
        \item концентрацию молекул газа.
    \end{enumerate}
}
\solutionspace{40pt}

\tasknumber{2}%
\task{%
    Изобразите в координатах $PV$, соблюдая масштаб, процесс 1234,
    в котором 12 — изобарическое нагревание в 2 раза,
    23 — изотермическое расширение в 3 раза,
    34 — изобарическое нагревание в 2 раза.
}
\solutionspace{160pt}

\tasknumber{3}%
\task{%
    Небольшую цилиндрическую пробирку с воздухом погружают на некоторую глубину в глубокое пресное озеро,
    после чего воздух занимает в ней лишь шестую часть от общего объема.
    Определите глубину, на которую погрузили пробирку.
    Температуру считать постоянной $T = 281\,\text{К}$, давлением паров воды пренебречь,
    атмосферное давление принять равным $p_{\text{aтм}} = 100\,\text{кПа}$.
}
\answer{%
    \begin{align*}
    T\text{ — const } &\implies P_1V_1 = \nu RT = P_2V_2.
    \\
    V_2 = \frac 16 V_1 &\implies P_1V_1 = P_2\cdot \frac 16V_1 \implies P_2 = 6P_1 = 6p_{\text{aтм}}.
    \\
    P_2 = p_{\text{aтм}} + \rho_{\text{в}} g h \implies h = \frac{ P_2 - p_{\text{aтм}} }{ \rho_{\text{в}} g } &= \frac{ 6p_{\text{aтм}} - p_{\text{aтм}} }{ \rho_{\text{в}} g } = \frac{ 5 \cdot p_{\text{aтм}} }{ \rho_{\text{в}} g } =  \\
     &= \frac{ 5 \cdot 100\,\text{кПа} }{ 1000\,\frac{\text{кг}}{\text{м}^{3}} \cdot 10\,\frac{\text{м}}{\text{с}^{2}} } \approx 50\,\text{м}.
    \end{align*}
}
\solutionspace{120pt}

\tasknumber{4}%
\task{%
    В замкнутом сосуде объёмом $3\,\text{л}$ находится кислород ($\mu = 32\,\frac{\text{г}}{\text{моль}}$) под давлением $3{,}5\units{ атм }$.
    Определите массу газа в сосуде и выразите её в граммах, приняв температуру газа равной $47\celsius$.
}
\answer{%
    $
        PV = \frac m\mu RT \implies m = \frac{ PV \mu }{ RT } =
        \frac{ 3{,}5\,\text{атм} \cdot3\,\text{л} \cdot 32\,\frac{\text{г}}{\text{моль}} }{ 8{,}31\,\frac{\text{Дж}}{\text{моль}\cdot\text{К}} \cdot \cbr{ 47 + 273 }\units{К} }
        \approx 12{,}64\,\text{г}.
    $
}
\solutionspace{120pt}

\tasknumber{5}%
\task{%
    Идеальный газ в экспериментальной установке подвергут политропному процессу $PV^n\text{ — const }$
    с показателем политропы $n=0{,}4$.
    В одном из экспериментов объём газа увеличился в $2$ раза.
    Как при этом изменилась температура газа (выросла или уменьшилась, на сколько или во сколько раз)?
}
\answer{%
    \begin{align*}
    P_1V_1^n &= P_2V_2^n, P_1V_1 = \nu R T_1, P_2V_2 = \nu R T_2 \implies\frac{ \nu R T_1 }{ V_1 } V_1^n = \frac{ \nu R T_2 }{ V_2 } V_2^n \implies \\
    \implies T_1V_1^{ n-1 } &= T_2V_2^{ n - 1 } \implies\frac{ T_2 }{ T_1 } = \cbr{ \frac{ V_1 }{ V_2 } }^{ n-1 } \approx 1{,}516
    \end{align*}
}

\end{document}
% autogenerated
