\setdate{18~марта~2021}
\setclass{10«АБ»}

\addpersonalvariant{Михаил Бурмистров}

\tasknumber{1}%
\task{%
    Определите КПД цикла 12341, рабочим телом которого является идеальный одноатомный газ, если
    12 — изохорический нагрев в шесть раз,
    23 — изобарическое расширение, при котором температура растёт в четыре раза,
    34 — изохора, 41 — изобара.

    Определите КПД цикла Карно, температура нагревателя которого равна максимальной температуре в цикле 12341, а холодильника — минимальной.
    Ответы в обоих случаях оставьте точными в виде нескоратимой дроби, никаких округлений.
}
\answer{%
    \begin{align*}
    A_{12} &= 0, \Delta U_{12} > 0, \implies Q_{12} = A_{12} + \Delta U_{12} > 0.
    \\
    A_{23} &> 0, \Delta U_{23} > 0, \implies Q_{23} = A_{23} + \Delta U_{23} > 0, \\
    A_{34} &= 0, \Delta U_{34} < 0, \implies Q_{34} = A_{34} + \Delta U_{34} < 0, \\
    A_{41} &< 0, \Delta U_{41} < 0, \implies Q_{41} = A_{41} + \Delta U_{41} < 0.
    \\
    P_1V_1 &= \nu R T_1, P_2V_2 = \nu R T_2, P_3V_3 = \nu R T_3, P_4V_4 = \nu R T_4 \text{ — уравнения состояния идеального газа}, \\
    &\text{Пусть $P_0$, $V_0$, $T_0$ — давление, объём и температура в точке 1 (минимальные во всём цикле):} \\
    P_1 &= P_4 = P_0, P_2 = P_3, V_1 = V_2 = V_0, V_3 = V_4, \text{остальные соотношения между объёмами и давлениями не даны, нужно считать} \\
    T_2 &= 6T_1 = 6T_0 \text{(по условию)} \implies \frac{P_2}{P_1} = \frac{P_2V_0}{P_1V_0} = \frac{P_2 V_2}{P_1 V_1}= \frac{\nu R T_2}{\nu R T_1} = \frac{T_2}{T_1} = 6 \implies P_2 = P_3 = 6 P_1 = 6 P_0, \\
    T_3 &= 4T_2 = 24T_0 \text{(по условию)} \implies \frac{V_3}{V_2} = \frac{P_3V_3}{P_2V_2}= \frac{\nu R T_3}{\nu R T_2} = \frac{T_3}{T_2} = 4 \implies V_3 = V_4 = 4 V_2 = 4 V_0.
    \\
    A_\text{цикл} &= (4P_0 - P_0)(6V_0 - V_0) = 15P_0V_0, \\
    A_{23} &= 6P_0 \cdot (4V_0 - V_0) = 18P_0V_0, \\
    \Delta U_{23} &= \frac 32 \nu R T_3 - \frac 32 \nu R T_3 = \frac 32 P_3 V_3 - \frac 32 P_2 V_2 = \frac 32 \cdot 6 P_0 \cdot 4 V_0 -  \frac 32 \cdot 6 P_0 \cdot V_0 = \frac 32 \cdot 18 \cdot P_0V_0, \\
    \Delta U_{12} &= \frac 32 \nu R T_2 - \frac 32 \nu R T_1 = \frac 32 P_2 V_2 - \frac 32 P_1 V_1 = \frac 32 \cdot 6 P_0 V_0 - \frac 32 P_0 V_0 = \frac 32 \cdot 5 \cdot P_0V_0.
    \\
    \eta &= \frac{A_\text{цикл}}{Q_+} = \frac{A_\text{цикл}}{Q_{12} + Q_{23}}  = \frac{A_\text{цикл}}{A_{12} + \Delta U_{12} + A_{23} + \Delta U_{23}} =  \\
     &= \frac{15P_0V_0}{0 + \frac 32 \cdot 5 \cdot P_0V_0 + 18P_0V_0 + \frac 32 \cdot 18 \cdot P_0V_0} = \frac{15}{\frac 32 \cdot 5 + 18 + \frac 32 \cdot 18} = \frac27 \approx 0{,}286.
     \\
    \eta_\text{Карно} &= 1 - \frac{T_\text{х}}{T_\text{н}} = 1 - \frac{T_\text{1}}{T_\text{3}} = 1 - \frac{T_0}{24T_0} = 1 - \frac 1{24}  = \frac{23}{24} \approx 0{,}958.
    \end{align*}
}
\solutionspace{360pt}

\tasknumber{2}%
\task{%
    Изобразите в координатах $PV$/$VT$/$PT$ графики изобарического сжатия в 2 раза (все 3 графика).
    Не забудьте указать оси и масштаб, начальную и конечную точки, направление движения на графике.
}
\solutionspace{100pt}

\tasknumber{3}%
\task{%
    Укажите, верны ли утверждения («да» или «нет» слева от каждого утверждения):
    \begin{enumerate}
        \item При адиабатическом расширении идеальный газ совершает ровно столько работы, сколько внутренней энергии теряет.
        % \item В силу третьего закона Ньютона, совершённая газом работа и работа, совершённая над ним, всегда равны по модулю и противоположны по знаку.
        \item Работу газа в некотором процессе можно вычислять как площадь под графиком в системе координат $PV$, главное лишь правильно расположить оси.
        % \item Дважды два три.
        \item При изотермическом процессе внутренняя энергия идеального одноатомного газа не изменяется, даже если ему подводят тепло.
        \item Газ может совершить ненулевую работу в изотермическом процессе.
        % \item Адиабатический процесс лишь по воле случая не имеет приставки «изо»: в нём изменяются давление, температура и объём, но это не все макропараметры идеального газа.
        \item Полученное выражение для внутренней энергии идеального газа ($\frac 32 \nu RT$) применимо к трёхатомному газу, при этом, например, уравнение состояния идеального газа применимо независимо от числа атомов в молекулах газа.
    \end{enumerate}
}
\answer{%
    $\text{да, да, да, да, нет}$
}

\variantsplitter

\addpersonalvariant{Ирина Ан}

\tasknumber{1}%
\task{%
    Определите КПД цикла 12341, рабочим телом которого является идеальный одноатомный газ, если
    12 — изохорический нагрев в шесть раз,
    23 — изобарическое расширение, при котором температура растёт в четыре раза,
    34 — изохора, 41 — изобара.

    Определите КПД цикла Карно, температура нагревателя которого равна максимальной температуре в цикле 12341, а холодильника — минимальной.
    Ответы в обоих случаях оставьте точными в виде нескоратимой дроби, никаких округлений.
}
\answer{%
    \begin{align*}
    A_{12} &= 0, \Delta U_{12} > 0, \implies Q_{12} = A_{12} + \Delta U_{12} > 0.
    \\
    A_{23} &> 0, \Delta U_{23} > 0, \implies Q_{23} = A_{23} + \Delta U_{23} > 0, \\
    A_{34} &= 0, \Delta U_{34} < 0, \implies Q_{34} = A_{34} + \Delta U_{34} < 0, \\
    A_{41} &< 0, \Delta U_{41} < 0, \implies Q_{41} = A_{41} + \Delta U_{41} < 0.
    \\
    P_1V_1 &= \nu R T_1, P_2V_2 = \nu R T_2, P_3V_3 = \nu R T_3, P_4V_4 = \nu R T_4 \text{ — уравнения состояния идеального газа}, \\
    &\text{Пусть $P_0$, $V_0$, $T_0$ — давление, объём и температура в точке 1 (минимальные во всём цикле):} \\
    P_1 &= P_4 = P_0, P_2 = P_3, V_1 = V_2 = V_0, V_3 = V_4, \text{остальные соотношения между объёмами и давлениями не даны, нужно считать} \\
    T_2 &= 6T_1 = 6T_0 \text{(по условию)} \implies \frac{P_2}{P_1} = \frac{P_2V_0}{P_1V_0} = \frac{P_2 V_2}{P_1 V_1}= \frac{\nu R T_2}{\nu R T_1} = \frac{T_2}{T_1} = 6 \implies P_2 = P_3 = 6 P_1 = 6 P_0, \\
    T_3 &= 4T_2 = 24T_0 \text{(по условию)} \implies \frac{V_3}{V_2} = \frac{P_3V_3}{P_2V_2}= \frac{\nu R T_3}{\nu R T_2} = \frac{T_3}{T_2} = 4 \implies V_3 = V_4 = 4 V_2 = 4 V_0.
    \\
    A_\text{цикл} &= (4P_0 - P_0)(6V_0 - V_0) = 15P_0V_0, \\
    A_{23} &= 6P_0 \cdot (4V_0 - V_0) = 18P_0V_0, \\
    \Delta U_{23} &= \frac 32 \nu R T_3 - \frac 32 \nu R T_3 = \frac 32 P_3 V_3 - \frac 32 P_2 V_2 = \frac 32 \cdot 6 P_0 \cdot 4 V_0 -  \frac 32 \cdot 6 P_0 \cdot V_0 = \frac 32 \cdot 18 \cdot P_0V_0, \\
    \Delta U_{12} &= \frac 32 \nu R T_2 - \frac 32 \nu R T_1 = \frac 32 P_2 V_2 - \frac 32 P_1 V_1 = \frac 32 \cdot 6 P_0 V_0 - \frac 32 P_0 V_0 = \frac 32 \cdot 5 \cdot P_0V_0.
    \\
    \eta &= \frac{A_\text{цикл}}{Q_+} = \frac{A_\text{цикл}}{Q_{12} + Q_{23}}  = \frac{A_\text{цикл}}{A_{12} + \Delta U_{12} + A_{23} + \Delta U_{23}} =  \\
     &= \frac{15P_0V_0}{0 + \frac 32 \cdot 5 \cdot P_0V_0 + 18P_0V_0 + \frac 32 \cdot 18 \cdot P_0V_0} = \frac{15}{\frac 32 \cdot 5 + 18 + \frac 32 \cdot 18} = \frac27 \approx 0{,}286.
     \\
    \eta_\text{Карно} &= 1 - \frac{T_\text{х}}{T_\text{н}} = 1 - \frac{T_\text{1}}{T_\text{3}} = 1 - \frac{T_0}{24T_0} = 1 - \frac 1{24}  = \frac{23}{24} \approx 0{,}958.
    \end{align*}
}
\solutionspace{360pt}

\tasknumber{2}%
\task{%
    Изобразите в координатах $PV$/$VT$/$PT$ графики изотермического повышения давления в 3 раза (все 3 графика).
    Не забудьте указать оси и масштаб, начальную и конечную точки, направление движения на графике.
}
\solutionspace{100pt}

\tasknumber{3}%
\task{%
    Укажите, верны ли утверждения («да» или «нет» слева от каждого утверждения):
    \begin{enumerate}
        \item При изобарном расширении идеальный газ совершает ровно столько работы, сколько внутренней энергии теряет.
        % \item В силу третьего закона Ньютона, совершённая газом работа и работа, совершённая над ним, всегда равны по модулю и противоположны по знаку.
        \item Работу газа в некотором процессе можно вычислять как площадь под графиком в системе координат $PV$, главное лишь правильно расположить оси.
        % \item Дважды два четыре.
        \item При изотермическом процессе внутренняя энергия идеального одноатомного газа не изменяется, даже если ему подводят тепло.
        \item Газ может совершить ненулевую работу в изобарном процессе.
        % \item Адиабатический процесс лишь по воле случая не имеет приставки «изо»: в нём изменяются давление, температура и объём, но это не все макропараметры идеального газа.
        \item Полученное выражение для внутренней энергии идеального газа ($\frac 32 \nu RT$) применимо к двухоатомному газу, при этом, например, уравнение состояния идеального газа применимо независимо от числа атомов в молекулах газа.
    \end{enumerate}
}
\answer{%
    $\text{нет, да, да, да, нет}$
}

\variantsplitter

\addpersonalvariant{Софья Андрианова}

\tasknumber{1}%
\task{%
    Определите КПД цикла 12341, рабочим телом которого является идеальный одноатомный газ, если
    12 — изохорический нагрев в пять раз,
    23 — изобарическое расширение, при котором температура растёт в два раза,
    34 — изохора, 41 — изобара.

    Определите КПД цикла Карно, температура нагревателя которого равна максимальной температуре в цикле 12341, а холодильника — минимальной.
    Ответы в обоих случаях оставьте точными в виде нескоратимой дроби, никаких округлений.
}
\answer{%
    \begin{align*}
    A_{12} &= 0, \Delta U_{12} > 0, \implies Q_{12} = A_{12} + \Delta U_{12} > 0.
    \\
    A_{23} &> 0, \Delta U_{23} > 0, \implies Q_{23} = A_{23} + \Delta U_{23} > 0, \\
    A_{34} &= 0, \Delta U_{34} < 0, \implies Q_{34} = A_{34} + \Delta U_{34} < 0, \\
    A_{41} &< 0, \Delta U_{41} < 0, \implies Q_{41} = A_{41} + \Delta U_{41} < 0.
    \\
    P_1V_1 &= \nu R T_1, P_2V_2 = \nu R T_2, P_3V_3 = \nu R T_3, P_4V_4 = \nu R T_4 \text{ — уравнения состояния идеального газа}, \\
    &\text{Пусть $P_0$, $V_0$, $T_0$ — давление, объём и температура в точке 1 (минимальные во всём цикле):} \\
    P_1 &= P_4 = P_0, P_2 = P_3, V_1 = V_2 = V_0, V_3 = V_4, \text{остальные соотношения между объёмами и давлениями не даны, нужно считать} \\
    T_2 &= 5T_1 = 5T_0 \text{(по условию)} \implies \frac{P_2}{P_1} = \frac{P_2V_0}{P_1V_0} = \frac{P_2 V_2}{P_1 V_1}= \frac{\nu R T_2}{\nu R T_1} = \frac{T_2}{T_1} = 5 \implies P_2 = P_3 = 5 P_1 = 5 P_0, \\
    T_3 &= 2T_2 = 10T_0 \text{(по условию)} \implies \frac{V_3}{V_2} = \frac{P_3V_3}{P_2V_2}= \frac{\nu R T_3}{\nu R T_2} = \frac{T_3}{T_2} = 2 \implies V_3 = V_4 = 2 V_2 = 2 V_0.
    \\
    A_\text{цикл} &= (2P_0 - P_0)(5V_0 - V_0) = 4P_0V_0, \\
    A_{23} &= 5P_0 \cdot (2V_0 - V_0) = 5P_0V_0, \\
    \Delta U_{23} &= \frac 32 \nu R T_3 - \frac 32 \nu R T_3 = \frac 32 P_3 V_3 - \frac 32 P_2 V_2 = \frac 32 \cdot 5 P_0 \cdot 2 V_0 -  \frac 32 \cdot 5 P_0 \cdot V_0 = \frac 32 \cdot 5 \cdot P_0V_0, \\
    \Delta U_{12} &= \frac 32 \nu R T_2 - \frac 32 \nu R T_1 = \frac 32 P_2 V_2 - \frac 32 P_1 V_1 = \frac 32 \cdot 5 P_0 V_0 - \frac 32 P_0 V_0 = \frac 32 \cdot 4 \cdot P_0V_0.
    \\
    \eta &= \frac{A_\text{цикл}}{Q_+} = \frac{A_\text{цикл}}{Q_{12} + Q_{23}}  = \frac{A_\text{цикл}}{A_{12} + \Delta U_{12} + A_{23} + \Delta U_{23}} =  \\
     &= \frac{4P_0V_0}{0 + \frac 32 \cdot 4 \cdot P_0V_0 + 5P_0V_0 + \frac 32 \cdot 5 \cdot P_0V_0} = \frac{4}{\frac 32 \cdot 4 + 5 + \frac 32 \cdot 5} = \frac8{37} \approx 0{,}216.
     \\
    \eta_\text{Карно} &= 1 - \frac{T_\text{х}}{T_\text{н}} = 1 - \frac{T_\text{1}}{T_\text{3}} = 1 - \frac{T_0}{10T_0} = 1 - \frac 1{10}  = \frac9{10} \approx 0{,}900.
    \end{align*}
}
\solutionspace{360pt}

\tasknumber{2}%
\task{%
    Изобразите в координатах $PV$/$VT$/$PT$ графики изохорического нагрева в 3 раза (все 3 графика).
    Не забудьте указать оси и масштаб, начальную и конечную точки, направление движения на графике.
}
\solutionspace{100pt}

\tasknumber{3}%
\task{%
    Укажите, верны ли утверждения («да» или «нет» слева от каждого утверждения):
    \begin{enumerate}
        \item При адиабатическом расширении идеальный газ совершает ровно столько работы, сколько внутренней энергии теряет.
        % \item В силу третьего закона Ньютона, совершённая газом работа и работа, совершённая над ним, всегда равны по модулю и противоположны по знаку.
        \item Работу газа в некотором процессе можно вычислять как площадь под графиком в системе координат $PT$, главное лишь правильно расположить оси.
        % \item Дважды два три.
        \item При изохорном процессе внутренняя энергия идеального одноатомного газа не изменяется, даже если ему подводят тепло.
        \item Газ может совершить ненулевую работу в изобарном процессе.
        % \item Адиабатический процесс лишь по воле случая не имеет приставки «изо»: в нём изменяются давление, температура и объём, но это не все макропараметры идеального газа.
        \item Полученное выражение для внутренней энергии идеального газа ($\frac 32 \nu RT$) применимо к трёхатомному газу, при этом, например, уравнение состояния идеального газа применимо независимо от числа атомов в молекулах газа.
    \end{enumerate}
}
\answer{%
    $\text{да, нет, нет, да, нет}$
}

\variantsplitter

\addpersonalvariant{Владимир Артемчук}

\tasknumber{1}%
\task{%
    Определите КПД цикла 12341, рабочим телом которого является идеальный одноатомный газ, если
    12 — изохорический нагрев в пять раз,
    23 — изобарическое расширение, при котором температура растёт в шесть раз,
    34 — изохора, 41 — изобара.

    Определите КПД цикла Карно, температура нагревателя которого равна максимальной температуре в цикле 12341, а холодильника — минимальной.
    Ответы в обоих случаях оставьте точными в виде нескоратимой дроби, никаких округлений.
}
\answer{%
    \begin{align*}
    A_{12} &= 0, \Delta U_{12} > 0, \implies Q_{12} = A_{12} + \Delta U_{12} > 0.
    \\
    A_{23} &> 0, \Delta U_{23} > 0, \implies Q_{23} = A_{23} + \Delta U_{23} > 0, \\
    A_{34} &= 0, \Delta U_{34} < 0, \implies Q_{34} = A_{34} + \Delta U_{34} < 0, \\
    A_{41} &< 0, \Delta U_{41} < 0, \implies Q_{41} = A_{41} + \Delta U_{41} < 0.
    \\
    P_1V_1 &= \nu R T_1, P_2V_2 = \nu R T_2, P_3V_3 = \nu R T_3, P_4V_4 = \nu R T_4 \text{ — уравнения состояния идеального газа}, \\
    &\text{Пусть $P_0$, $V_0$, $T_0$ — давление, объём и температура в точке 1 (минимальные во всём цикле):} \\
    P_1 &= P_4 = P_0, P_2 = P_3, V_1 = V_2 = V_0, V_3 = V_4, \text{остальные соотношения между объёмами и давлениями не даны, нужно считать} \\
    T_2 &= 5T_1 = 5T_0 \text{(по условию)} \implies \frac{P_2}{P_1} = \frac{P_2V_0}{P_1V_0} = \frac{P_2 V_2}{P_1 V_1}= \frac{\nu R T_2}{\nu R T_1} = \frac{T_2}{T_1} = 5 \implies P_2 = P_3 = 5 P_1 = 5 P_0, \\
    T_3 &= 6T_2 = 30T_0 \text{(по условию)} \implies \frac{V_3}{V_2} = \frac{P_3V_3}{P_2V_2}= \frac{\nu R T_3}{\nu R T_2} = \frac{T_3}{T_2} = 6 \implies V_3 = V_4 = 6 V_2 = 6 V_0.
    \\
    A_\text{цикл} &= (6P_0 - P_0)(5V_0 - V_0) = 20P_0V_0, \\
    A_{23} &= 5P_0 \cdot (6V_0 - V_0) = 25P_0V_0, \\
    \Delta U_{23} &= \frac 32 \nu R T_3 - \frac 32 \nu R T_3 = \frac 32 P_3 V_3 - \frac 32 P_2 V_2 = \frac 32 \cdot 5 P_0 \cdot 6 V_0 -  \frac 32 \cdot 5 P_0 \cdot V_0 = \frac 32 \cdot 25 \cdot P_0V_0, \\
    \Delta U_{12} &= \frac 32 \nu R T_2 - \frac 32 \nu R T_1 = \frac 32 P_2 V_2 - \frac 32 P_1 V_1 = \frac 32 \cdot 5 P_0 V_0 - \frac 32 P_0 V_0 = \frac 32 \cdot 4 \cdot P_0V_0.
    \\
    \eta &= \frac{A_\text{цикл}}{Q_+} = \frac{A_\text{цикл}}{Q_{12} + Q_{23}}  = \frac{A_\text{цикл}}{A_{12} + \Delta U_{12} + A_{23} + \Delta U_{23}} =  \\
     &= \frac{20P_0V_0}{0 + \frac 32 \cdot 4 \cdot P_0V_0 + 25P_0V_0 + \frac 32 \cdot 25 \cdot P_0V_0} = \frac{20}{\frac 32 \cdot 4 + 25 + \frac 32 \cdot 25} = \frac{40}{137} \approx 0{,}292.
     \\
    \eta_\text{Карно} &= 1 - \frac{T_\text{х}}{T_\text{н}} = 1 - \frac{T_\text{1}}{T_\text{3}} = 1 - \frac{T_0}{30T_0} = 1 - \frac 1{30}  = \frac{29}{30} \approx 0{,}967.
    \end{align*}
}
\solutionspace{360pt}

\tasknumber{2}%
\task{%
    Изобразите в координатах $PV$/$VT$/$PT$ графики изохорического нагрева в 3 раза (все 3 графика).
    Не забудьте указать оси и масштаб, начальную и конечную точки, направление движения на графике.
}
\solutionspace{100pt}

\tasknumber{3}%
\task{%
    Укажите, верны ли утверждения («да» или «нет» слева от каждого утверждения):
    \begin{enumerate}
        \item При изобарном расширении идеальный газ совершает ровно столько работы, сколько внутренней энергии теряет.
        % \item В силу третьего закона Ньютона, совершённая газом работа и работа, совершённая над ним, всегда равны по модулю и противоположны по знаку.
        \item Работу газа в некотором процессе можно вычислять как площадь под графиком в системе координат $PV$, главное лишь правильно расположить оси.
        % \item Дважды два четыре.
        \item При изотермическом процессе внутренняя энергия идеального одноатомного газа не изменяется, даже если ему подводят тепло.
        \item Газ может совершить ненулевую работу в изотермическом процессе.
        % \item Адиабатический процесс лишь по воле случая не имеет приставки «изо»: в нём изменяются давление, температура и объём, но это не все макропараметры идеального газа.
        \item Полученное выражение для внутренней энергии идеального газа ($\frac 32 \nu RT$) применимо к двухоатомному газу, при этом, например, уравнение состояния идеального газа применимо независимо от числа атомов в молекулах газа.
    \end{enumerate}
}
\answer{%
    $\text{нет, да, да, да, нет}$
}

\variantsplitter

\addpersonalvariant{Софья Белянкина}

\tasknumber{1}%
\task{%
    Определите КПД цикла 12341, рабочим телом которого является идеальный одноатомный газ, если
    12 — изохорический нагрев в пять раз,
    23 — изобарическое расширение, при котором температура растёт в два раза,
    34 — изохора, 41 — изобара.

    Определите КПД цикла Карно, температура нагревателя которого равна максимальной температуре в цикле 12341, а холодильника — минимальной.
    Ответы в обоих случаях оставьте точными в виде нескоратимой дроби, никаких округлений.
}
\answer{%
    \begin{align*}
    A_{12} &= 0, \Delta U_{12} > 0, \implies Q_{12} = A_{12} + \Delta U_{12} > 0.
    \\
    A_{23} &> 0, \Delta U_{23} > 0, \implies Q_{23} = A_{23} + \Delta U_{23} > 0, \\
    A_{34} &= 0, \Delta U_{34} < 0, \implies Q_{34} = A_{34} + \Delta U_{34} < 0, \\
    A_{41} &< 0, \Delta U_{41} < 0, \implies Q_{41} = A_{41} + \Delta U_{41} < 0.
    \\
    P_1V_1 &= \nu R T_1, P_2V_2 = \nu R T_2, P_3V_3 = \nu R T_3, P_4V_4 = \nu R T_4 \text{ — уравнения состояния идеального газа}, \\
    &\text{Пусть $P_0$, $V_0$, $T_0$ — давление, объём и температура в точке 1 (минимальные во всём цикле):} \\
    P_1 &= P_4 = P_0, P_2 = P_3, V_1 = V_2 = V_0, V_3 = V_4, \text{остальные соотношения между объёмами и давлениями не даны, нужно считать} \\
    T_2 &= 5T_1 = 5T_0 \text{(по условию)} \implies \frac{P_2}{P_1} = \frac{P_2V_0}{P_1V_0} = \frac{P_2 V_2}{P_1 V_1}= \frac{\nu R T_2}{\nu R T_1} = \frac{T_2}{T_1} = 5 \implies P_2 = P_3 = 5 P_1 = 5 P_0, \\
    T_3 &= 2T_2 = 10T_0 \text{(по условию)} \implies \frac{V_3}{V_2} = \frac{P_3V_3}{P_2V_2}= \frac{\nu R T_3}{\nu R T_2} = \frac{T_3}{T_2} = 2 \implies V_3 = V_4 = 2 V_2 = 2 V_0.
    \\
    A_\text{цикл} &= (2P_0 - P_0)(5V_0 - V_0) = 4P_0V_0, \\
    A_{23} &= 5P_0 \cdot (2V_0 - V_0) = 5P_0V_0, \\
    \Delta U_{23} &= \frac 32 \nu R T_3 - \frac 32 \nu R T_3 = \frac 32 P_3 V_3 - \frac 32 P_2 V_2 = \frac 32 \cdot 5 P_0 \cdot 2 V_0 -  \frac 32 \cdot 5 P_0 \cdot V_0 = \frac 32 \cdot 5 \cdot P_0V_0, \\
    \Delta U_{12} &= \frac 32 \nu R T_2 - \frac 32 \nu R T_1 = \frac 32 P_2 V_2 - \frac 32 P_1 V_1 = \frac 32 \cdot 5 P_0 V_0 - \frac 32 P_0 V_0 = \frac 32 \cdot 4 \cdot P_0V_0.
    \\
    \eta &= \frac{A_\text{цикл}}{Q_+} = \frac{A_\text{цикл}}{Q_{12} + Q_{23}}  = \frac{A_\text{цикл}}{A_{12} + \Delta U_{12} + A_{23} + \Delta U_{23}} =  \\
     &= \frac{4P_0V_0}{0 + \frac 32 \cdot 4 \cdot P_0V_0 + 5P_0V_0 + \frac 32 \cdot 5 \cdot P_0V_0} = \frac{4}{\frac 32 \cdot 4 + 5 + \frac 32 \cdot 5} = \frac8{37} \approx 0{,}216.
     \\
    \eta_\text{Карно} &= 1 - \frac{T_\text{х}}{T_\text{н}} = 1 - \frac{T_\text{1}}{T_\text{3}} = 1 - \frac{T_0}{10T_0} = 1 - \frac 1{10}  = \frac9{10} \approx 0{,}900.
    \end{align*}
}
\solutionspace{360pt}

\tasknumber{2}%
\task{%
    Изобразите в координатах $PV$/$VT$/$PT$ графики изобарического расширения в 2 раза (все 3 графика).
    Не забудьте указать оси и масштаб, начальную и конечную точки, направление движения на графике.
}
\solutionspace{100pt}

\tasknumber{3}%
\task{%
    Укажите, верны ли утверждения («да» или «нет» слева от каждого утверждения):
    \begin{enumerate}
        \item При изобарном расширении идеальный газ совершает ровно столько работы, сколько внутренней энергии теряет.
        % \item В силу третьего закона Ньютона, совершённая газом работа и работа, совершённая над ним, всегда равны по модулю и противоположны по знаку.
        \item Работу газа в некотором процессе можно вычислять как площадь под графиком в системе координат $PV$, главное лишь правильно расположить оси.
        % \item Дважды два пять.
        \item При изохорном процессе внутренняя энергия идеального одноатомного газа не изменяется, даже если ему подводят тепло.
        \item Газ может совершить ненулевую работу в изохорном процессе.
        % \item Адиабатический процесс лишь по воле случая не имеет приставки «изо»: в нём изменяются давление, температура и объём, но это не все макропараметры идеального газа.
        \item Полученное выражение для внутренней энергии идеального газа ($\frac 32 \nu RT$) применимо к одноатомному газу, при этом, например, уравнение состояния идеального газа применимо независимо от числа атомов в молекулах газа.
    \end{enumerate}
}
\answer{%
    $\text{нет, да, нет, нет, да}$
}

\variantsplitter

\addpersonalvariant{Варвара Егиазарян}

\tasknumber{1}%
\task{%
    Определите КПД цикла 12341, рабочим телом которого является идеальный одноатомный газ, если
    12 — изохорический нагрев в три раза,
    23 — изобарическое расширение, при котором температура растёт в четыре раза,
    34 — изохора, 41 — изобара.

    Определите КПД цикла Карно, температура нагревателя которого равна максимальной температуре в цикле 12341, а холодильника — минимальной.
    Ответы в обоих случаях оставьте точными в виде нескоратимой дроби, никаких округлений.
}
\answer{%
    \begin{align*}
    A_{12} &= 0, \Delta U_{12} > 0, \implies Q_{12} = A_{12} + \Delta U_{12} > 0.
    \\
    A_{23} &> 0, \Delta U_{23} > 0, \implies Q_{23} = A_{23} + \Delta U_{23} > 0, \\
    A_{34} &= 0, \Delta U_{34} < 0, \implies Q_{34} = A_{34} + \Delta U_{34} < 0, \\
    A_{41} &< 0, \Delta U_{41} < 0, \implies Q_{41} = A_{41} + \Delta U_{41} < 0.
    \\
    P_1V_1 &= \nu R T_1, P_2V_2 = \nu R T_2, P_3V_3 = \nu R T_3, P_4V_4 = \nu R T_4 \text{ — уравнения состояния идеального газа}, \\
    &\text{Пусть $P_0$, $V_0$, $T_0$ — давление, объём и температура в точке 1 (минимальные во всём цикле):} \\
    P_1 &= P_4 = P_0, P_2 = P_3, V_1 = V_2 = V_0, V_3 = V_4, \text{остальные соотношения между объёмами и давлениями не даны, нужно считать} \\
    T_2 &= 3T_1 = 3T_0 \text{(по условию)} \implies \frac{P_2}{P_1} = \frac{P_2V_0}{P_1V_0} = \frac{P_2 V_2}{P_1 V_1}= \frac{\nu R T_2}{\nu R T_1} = \frac{T_2}{T_1} = 3 \implies P_2 = P_3 = 3 P_1 = 3 P_0, \\
    T_3 &= 4T_2 = 12T_0 \text{(по условию)} \implies \frac{V_3}{V_2} = \frac{P_3V_3}{P_2V_2}= \frac{\nu R T_3}{\nu R T_2} = \frac{T_3}{T_2} = 4 \implies V_3 = V_4 = 4 V_2 = 4 V_0.
    \\
    A_\text{цикл} &= (4P_0 - P_0)(3V_0 - V_0) = 6P_0V_0, \\
    A_{23} &= 3P_0 \cdot (4V_0 - V_0) = 9P_0V_0, \\
    \Delta U_{23} &= \frac 32 \nu R T_3 - \frac 32 \nu R T_3 = \frac 32 P_3 V_3 - \frac 32 P_2 V_2 = \frac 32 \cdot 3 P_0 \cdot 4 V_0 -  \frac 32 \cdot 3 P_0 \cdot V_0 = \frac 32 \cdot 9 \cdot P_0V_0, \\
    \Delta U_{12} &= \frac 32 \nu R T_2 - \frac 32 \nu R T_1 = \frac 32 P_2 V_2 - \frac 32 P_1 V_1 = \frac 32 \cdot 3 P_0 V_0 - \frac 32 P_0 V_0 = \frac 32 \cdot 2 \cdot P_0V_0.
    \\
    \eta &= \frac{A_\text{цикл}}{Q_+} = \frac{A_\text{цикл}}{Q_{12} + Q_{23}}  = \frac{A_\text{цикл}}{A_{12} + \Delta U_{12} + A_{23} + \Delta U_{23}} =  \\
     &= \frac{6P_0V_0}{0 + \frac 32 \cdot 2 \cdot P_0V_0 + 9P_0V_0 + \frac 32 \cdot 9 \cdot P_0V_0} = \frac{6}{\frac 32 \cdot 2 + 9 + \frac 32 \cdot 9} = \frac4{17} \approx 0{,}235.
     \\
    \eta_\text{Карно} &= 1 - \frac{T_\text{х}}{T_\text{н}} = 1 - \frac{T_\text{1}}{T_\text{3}} = 1 - \frac{T_0}{12T_0} = 1 - \frac 1{12}  = \frac{11}{12} \approx 0{,}917.
    \end{align*}
}
\solutionspace{360pt}

\tasknumber{2}%
\task{%
    Изобразите в координатах $PV$/$VT$/$PT$ графики изобарического расширения в 4 раза (все 3 графика).
    Не забудьте указать оси и масштаб, начальную и конечную точки, направление движения на графике.
}
\solutionspace{100pt}

\tasknumber{3}%
\task{%
    Укажите, верны ли утверждения («да» или «нет» слева от каждого утверждения):
    \begin{enumerate}
        \item При изобарном расширении идеальный газ совершает ровно столько работы, сколько внутренней энергии теряет.
        % \item В силу третьего закона Ньютона, совершённая газом работа и работа, совершённая над ним, всегда равны по модулю и противоположны по знаку.
        \item Работу газа в некотором процессе можно вычислять как площадь под графиком в системе координат $PV$, главное лишь правильно расположить оси.
        % \item Дважды два три.
        \item При изотермическом процессе внутренняя энергия идеального одноатомного газа не изменяется, даже если ему подводят тепло.
        \item Газ может совершить ненулевую работу в изобарном процессе.
        % \item Адиабатический процесс лишь по воле случая не имеет приставки «изо»: в нём изменяются давление, температура и объём, но это не все макропараметры идеального газа.
        \item Полученное выражение для внутренней энергии идеального газа ($\frac 32 \nu RT$) применимо к одноатомному газу, при этом, например, уравнение состояния идеального газа применимо независимо от числа атомов в молекулах газа.
    \end{enumerate}
}
\answer{%
    $\text{нет, да, да, да, да}$
}

\variantsplitter

\addpersonalvariant{Владислав Емелин}

\tasknumber{1}%
\task{%
    Определите КПД цикла 12341, рабочим телом которого является идеальный одноатомный газ, если
    12 — изохорический нагрев в четыре раза,
    23 — изобарическое расширение, при котором температура растёт в два раза,
    34 — изохора, 41 — изобара.

    Определите КПД цикла Карно, температура нагревателя которого равна максимальной температуре в цикле 12341, а холодильника — минимальной.
    Ответы в обоих случаях оставьте точными в виде нескоратимой дроби, никаких округлений.
}
\answer{%
    \begin{align*}
    A_{12} &= 0, \Delta U_{12} > 0, \implies Q_{12} = A_{12} + \Delta U_{12} > 0.
    \\
    A_{23} &> 0, \Delta U_{23} > 0, \implies Q_{23} = A_{23} + \Delta U_{23} > 0, \\
    A_{34} &= 0, \Delta U_{34} < 0, \implies Q_{34} = A_{34} + \Delta U_{34} < 0, \\
    A_{41} &< 0, \Delta U_{41} < 0, \implies Q_{41} = A_{41} + \Delta U_{41} < 0.
    \\
    P_1V_1 &= \nu R T_1, P_2V_2 = \nu R T_2, P_3V_3 = \nu R T_3, P_4V_4 = \nu R T_4 \text{ — уравнения состояния идеального газа}, \\
    &\text{Пусть $P_0$, $V_0$, $T_0$ — давление, объём и температура в точке 1 (минимальные во всём цикле):} \\
    P_1 &= P_4 = P_0, P_2 = P_3, V_1 = V_2 = V_0, V_3 = V_4, \text{остальные соотношения между объёмами и давлениями не даны, нужно считать} \\
    T_2 &= 4T_1 = 4T_0 \text{(по условию)} \implies \frac{P_2}{P_1} = \frac{P_2V_0}{P_1V_0} = \frac{P_2 V_2}{P_1 V_1}= \frac{\nu R T_2}{\nu R T_1} = \frac{T_2}{T_1} = 4 \implies P_2 = P_3 = 4 P_1 = 4 P_0, \\
    T_3 &= 2T_2 = 8T_0 \text{(по условию)} \implies \frac{V_3}{V_2} = \frac{P_3V_3}{P_2V_2}= \frac{\nu R T_3}{\nu R T_2} = \frac{T_3}{T_2} = 2 \implies V_3 = V_4 = 2 V_2 = 2 V_0.
    \\
    A_\text{цикл} &= (2P_0 - P_0)(4V_0 - V_0) = 3P_0V_0, \\
    A_{23} &= 4P_0 \cdot (2V_0 - V_0) = 4P_0V_0, \\
    \Delta U_{23} &= \frac 32 \nu R T_3 - \frac 32 \nu R T_3 = \frac 32 P_3 V_3 - \frac 32 P_2 V_2 = \frac 32 \cdot 4 P_0 \cdot 2 V_0 -  \frac 32 \cdot 4 P_0 \cdot V_0 = \frac 32 \cdot 4 \cdot P_0V_0, \\
    \Delta U_{12} &= \frac 32 \nu R T_2 - \frac 32 \nu R T_1 = \frac 32 P_2 V_2 - \frac 32 P_1 V_1 = \frac 32 \cdot 4 P_0 V_0 - \frac 32 P_0 V_0 = \frac 32 \cdot 3 \cdot P_0V_0.
    \\
    \eta &= \frac{A_\text{цикл}}{Q_+} = \frac{A_\text{цикл}}{Q_{12} + Q_{23}}  = \frac{A_\text{цикл}}{A_{12} + \Delta U_{12} + A_{23} + \Delta U_{23}} =  \\
     &= \frac{3P_0V_0}{0 + \frac 32 \cdot 3 \cdot P_0V_0 + 4P_0V_0 + \frac 32 \cdot 4 \cdot P_0V_0} = \frac{3}{\frac 32 \cdot 3 + 4 + \frac 32 \cdot 4} = \frac6{29} \approx 0{,}207.
     \\
    \eta_\text{Карно} &= 1 - \frac{T_\text{х}}{T_\text{н}} = 1 - \frac{T_\text{1}}{T_\text{3}} = 1 - \frac{T_0}{8T_0} = 1 - \frac 1{8}  = \frac78 \approx 0{,}875.
    \end{align*}
}
\solutionspace{360pt}

\tasknumber{2}%
\task{%
    Изобразите в координатах $PV$/$VT$/$PT$ графики изобарического расширения в 4 раза (все 3 графика).
    Не забудьте указать оси и масштаб, начальную и конечную точки, направление движения на графике.
}
\solutionspace{100pt}

\tasknumber{3}%
\task{%
    Укажите, верны ли утверждения («да» или «нет» слева от каждого утверждения):
    \begin{enumerate}
        \item При адиабатическом расширении идеальный газ совершает ровно столько работы, сколько внутренней энергии теряет.
        % \item В силу третьего закона Ньютона, совершённая газом работа и работа, совершённая над ним, всегда равны по модулю и противоположны по знаку.
        \item Работу газа в некотором процессе можно вычислять как площадь под графиком в системе координат $PT$, главное лишь правильно расположить оси.
        % \item Дважды два пять.
        \item При изобарном процессе внутренняя энергия идеального одноатомного газа не изменяется, даже если ему подводят тепло.
        \item Газ может совершить ненулевую работу в изобарном процессе.
        % \item Адиабатический процесс лишь по воле случая не имеет приставки «изо»: в нём изменяются давление, температура и объём, но это не все макропараметры идеального газа.
        \item Полученное выражение для внутренней энергии идеального газа ($\frac 32 \nu RT$) применимо к одноатомному газу, при этом, например, уравнение состояния идеального газа применимо независимо от числа атомов в молекулах газа.
    \end{enumerate}
}
\answer{%
    $\text{да, нет, нет, да, да}$
}

\variantsplitter

\addpersonalvariant{Артём Жичин}

\tasknumber{1}%
\task{%
    Определите КПД цикла 12341, рабочим телом которого является идеальный одноатомный газ, если
    12 — изохорический нагрев в шесть раз,
    23 — изобарическое расширение, при котором температура растёт в три раза,
    34 — изохора, 41 — изобара.

    Определите КПД цикла Карно, температура нагревателя которого равна максимальной температуре в цикле 12341, а холодильника — минимальной.
    Ответы в обоих случаях оставьте точными в виде нескоратимой дроби, никаких округлений.
}
\answer{%
    \begin{align*}
    A_{12} &= 0, \Delta U_{12} > 0, \implies Q_{12} = A_{12} + \Delta U_{12} > 0.
    \\
    A_{23} &> 0, \Delta U_{23} > 0, \implies Q_{23} = A_{23} + \Delta U_{23} > 0, \\
    A_{34} &= 0, \Delta U_{34} < 0, \implies Q_{34} = A_{34} + \Delta U_{34} < 0, \\
    A_{41} &< 0, \Delta U_{41} < 0, \implies Q_{41} = A_{41} + \Delta U_{41} < 0.
    \\
    P_1V_1 &= \nu R T_1, P_2V_2 = \nu R T_2, P_3V_3 = \nu R T_3, P_4V_4 = \nu R T_4 \text{ — уравнения состояния идеального газа}, \\
    &\text{Пусть $P_0$, $V_0$, $T_0$ — давление, объём и температура в точке 1 (минимальные во всём цикле):} \\
    P_1 &= P_4 = P_0, P_2 = P_3, V_1 = V_2 = V_0, V_3 = V_4, \text{остальные соотношения между объёмами и давлениями не даны, нужно считать} \\
    T_2 &= 6T_1 = 6T_0 \text{(по условию)} \implies \frac{P_2}{P_1} = \frac{P_2V_0}{P_1V_0} = \frac{P_2 V_2}{P_1 V_1}= \frac{\nu R T_2}{\nu R T_1} = \frac{T_2}{T_1} = 6 \implies P_2 = P_3 = 6 P_1 = 6 P_0, \\
    T_3 &= 3T_2 = 18T_0 \text{(по условию)} \implies \frac{V_3}{V_2} = \frac{P_3V_3}{P_2V_2}= \frac{\nu R T_3}{\nu R T_2} = \frac{T_3}{T_2} = 3 \implies V_3 = V_4 = 3 V_2 = 3 V_0.
    \\
    A_\text{цикл} &= (3P_0 - P_0)(6V_0 - V_0) = 10P_0V_0, \\
    A_{23} &= 6P_0 \cdot (3V_0 - V_0) = 12P_0V_0, \\
    \Delta U_{23} &= \frac 32 \nu R T_3 - \frac 32 \nu R T_3 = \frac 32 P_3 V_3 - \frac 32 P_2 V_2 = \frac 32 \cdot 6 P_0 \cdot 3 V_0 -  \frac 32 \cdot 6 P_0 \cdot V_0 = \frac 32 \cdot 12 \cdot P_0V_0, \\
    \Delta U_{12} &= \frac 32 \nu R T_2 - \frac 32 \nu R T_1 = \frac 32 P_2 V_2 - \frac 32 P_1 V_1 = \frac 32 \cdot 6 P_0 V_0 - \frac 32 P_0 V_0 = \frac 32 \cdot 5 \cdot P_0V_0.
    \\
    \eta &= \frac{A_\text{цикл}}{Q_+} = \frac{A_\text{цикл}}{Q_{12} + Q_{23}}  = \frac{A_\text{цикл}}{A_{12} + \Delta U_{12} + A_{23} + \Delta U_{23}} =  \\
     &= \frac{10P_0V_0}{0 + \frac 32 \cdot 5 \cdot P_0V_0 + 12P_0V_0 + \frac 32 \cdot 12 \cdot P_0V_0} = \frac{10}{\frac 32 \cdot 5 + 12 + \frac 32 \cdot 12} = \frac4{15} \approx 0{,}267.
     \\
    \eta_\text{Карно} &= 1 - \frac{T_\text{х}}{T_\text{н}} = 1 - \frac{T_\text{1}}{T_\text{3}} = 1 - \frac{T_0}{18T_0} = 1 - \frac 1{18}  = \frac{17}{18} \approx 0{,}944.
    \end{align*}
}
\solutionspace{360pt}

\tasknumber{2}%
\task{%
    Изобразите в координатах $PV$/$VT$/$PT$ графики изобарического сжатия в 4 раза (все 3 графика).
    Не забудьте указать оси и масштаб, начальную и конечную точки, направление движения на графике.
}
\solutionspace{100pt}

\tasknumber{3}%
\task{%
    Укажите, верны ли утверждения («да» или «нет» слева от каждого утверждения):
    \begin{enumerate}
        \item При изобарном расширении идеальный газ совершает ровно столько работы, сколько внутренней энергии теряет.
        % \item В силу третьего закона Ньютона, совершённая газом работа и работа, совершённая над ним, всегда равны по модулю и противоположны по знаку.
        \item Работу газа в некотором процессе можно вычислять как площадь под графиком в системе координат $PT$, главное лишь правильно расположить оси.
        % \item Дважды два три.
        \item При изобарном процессе внутренняя энергия идеального одноатомного газа не изменяется, даже если ему подводят тепло.
        \item Газ может совершить ненулевую работу в изобарном процессе.
        % \item Адиабатический процесс лишь по воле случая не имеет приставки «изо»: в нём изменяются давление, температура и объём, но это не все макропараметры идеального газа.
        \item Полученное выражение для внутренней энергии идеального газа ($\frac 32 \nu RT$) применимо к двухоатомному газу, при этом, например, уравнение состояния идеального газа применимо независимо от числа атомов в молекулах газа.
    \end{enumerate}
}
\answer{%
    $\text{нет, нет, нет, да, нет}$
}

\variantsplitter

\addpersonalvariant{Дарья Кошман}

\tasknumber{1}%
\task{%
    Определите КПД цикла 12341, рабочим телом которого является идеальный одноатомный газ, если
    12 — изохорический нагрев в пять раз,
    23 — изобарическое расширение, при котором температура растёт в четыре раза,
    34 — изохора, 41 — изобара.

    Определите КПД цикла Карно, температура нагревателя которого равна максимальной температуре в цикле 12341, а холодильника — минимальной.
    Ответы в обоих случаях оставьте точными в виде нескоратимой дроби, никаких округлений.
}
\answer{%
    \begin{align*}
    A_{12} &= 0, \Delta U_{12} > 0, \implies Q_{12} = A_{12} + \Delta U_{12} > 0.
    \\
    A_{23} &> 0, \Delta U_{23} > 0, \implies Q_{23} = A_{23} + \Delta U_{23} > 0, \\
    A_{34} &= 0, \Delta U_{34} < 0, \implies Q_{34} = A_{34} + \Delta U_{34} < 0, \\
    A_{41} &< 0, \Delta U_{41} < 0, \implies Q_{41} = A_{41} + \Delta U_{41} < 0.
    \\
    P_1V_1 &= \nu R T_1, P_2V_2 = \nu R T_2, P_3V_3 = \nu R T_3, P_4V_4 = \nu R T_4 \text{ — уравнения состояния идеального газа}, \\
    &\text{Пусть $P_0$, $V_0$, $T_0$ — давление, объём и температура в точке 1 (минимальные во всём цикле):} \\
    P_1 &= P_4 = P_0, P_2 = P_3, V_1 = V_2 = V_0, V_3 = V_4, \text{остальные соотношения между объёмами и давлениями не даны, нужно считать} \\
    T_2 &= 5T_1 = 5T_0 \text{(по условию)} \implies \frac{P_2}{P_1} = \frac{P_2V_0}{P_1V_0} = \frac{P_2 V_2}{P_1 V_1}= \frac{\nu R T_2}{\nu R T_1} = \frac{T_2}{T_1} = 5 \implies P_2 = P_3 = 5 P_1 = 5 P_0, \\
    T_3 &= 4T_2 = 20T_0 \text{(по условию)} \implies \frac{V_3}{V_2} = \frac{P_3V_3}{P_2V_2}= \frac{\nu R T_3}{\nu R T_2} = \frac{T_3}{T_2} = 4 \implies V_3 = V_4 = 4 V_2 = 4 V_0.
    \\
    A_\text{цикл} &= (4P_0 - P_0)(5V_0 - V_0) = 12P_0V_0, \\
    A_{23} &= 5P_0 \cdot (4V_0 - V_0) = 15P_0V_0, \\
    \Delta U_{23} &= \frac 32 \nu R T_3 - \frac 32 \nu R T_3 = \frac 32 P_3 V_3 - \frac 32 P_2 V_2 = \frac 32 \cdot 5 P_0 \cdot 4 V_0 -  \frac 32 \cdot 5 P_0 \cdot V_0 = \frac 32 \cdot 15 \cdot P_0V_0, \\
    \Delta U_{12} &= \frac 32 \nu R T_2 - \frac 32 \nu R T_1 = \frac 32 P_2 V_2 - \frac 32 P_1 V_1 = \frac 32 \cdot 5 P_0 V_0 - \frac 32 P_0 V_0 = \frac 32 \cdot 4 \cdot P_0V_0.
    \\
    \eta &= \frac{A_\text{цикл}}{Q_+} = \frac{A_\text{цикл}}{Q_{12} + Q_{23}}  = \frac{A_\text{цикл}}{A_{12} + \Delta U_{12} + A_{23} + \Delta U_{23}} =  \\
     &= \frac{12P_0V_0}{0 + \frac 32 \cdot 4 \cdot P_0V_0 + 15P_0V_0 + \frac 32 \cdot 15 \cdot P_0V_0} = \frac{12}{\frac 32 \cdot 4 + 15 + \frac 32 \cdot 15} = \frac8{29} \approx 0{,}276.
     \\
    \eta_\text{Карно} &= 1 - \frac{T_\text{х}}{T_\text{н}} = 1 - \frac{T_\text{1}}{T_\text{3}} = 1 - \frac{T_0}{20T_0} = 1 - \frac 1{20}  = \frac{19}{20} \approx 0{,}950.
    \end{align*}
}
\solutionspace{360pt}

\tasknumber{2}%
\task{%
    Изобразите в координатах $PV$/$VT$/$PT$ графики изобарического расширения в 2 раза (все 3 графика).
    Не забудьте указать оси и масштаб, начальную и конечную точки, направление движения на графике.
}
\solutionspace{100pt}

\tasknumber{3}%
\task{%
    Укажите, верны ли утверждения («да» или «нет» слева от каждого утверждения):
    \begin{enumerate}
        \item При адиабатическом расширении идеальный газ совершает ровно столько работы, сколько внутренней энергии теряет.
        % \item В силу третьего закона Ньютона, совершённая газом работа и работа, совершённая над ним, всегда равны по модулю и противоположны по знаку.
        \item Работу газа в некотором процессе можно вычислять как площадь под графиком в системе координат $VT$, главное лишь правильно расположить оси.
        % \item Дважды два четыре.
        \item При изобарном процессе внутренняя энергия идеального одноатомного газа не изменяется, даже если ему подводят тепло.
        \item Газ может совершить ненулевую работу в изотермическом процессе.
        % \item Адиабатический процесс лишь по воле случая не имеет приставки «изо»: в нём изменяются давление, температура и объём, но это не все макропараметры идеального газа.
        \item Полученное выражение для внутренней энергии идеального газа ($\frac 32 \nu RT$) применимо к двухоатомному газу, при этом, например, уравнение состояния идеального газа применимо независимо от числа атомов в молекулах газа.
    \end{enumerate}
}
\answer{%
    $\text{да, нет, нет, да, нет}$
}

\variantsplitter

\addpersonalvariant{Анна Кузьмичёва}

\tasknumber{1}%
\task{%
    Определите КПД цикла 12341, рабочим телом которого является идеальный одноатомный газ, если
    12 — изохорический нагрев в шесть раз,
    23 — изобарическое расширение, при котором температура растёт в четыре раза,
    34 — изохора, 41 — изобара.

    Определите КПД цикла Карно, температура нагревателя которого равна максимальной температуре в цикле 12341, а холодильника — минимальной.
    Ответы в обоих случаях оставьте точными в виде нескоратимой дроби, никаких округлений.
}
\answer{%
    \begin{align*}
    A_{12} &= 0, \Delta U_{12} > 0, \implies Q_{12} = A_{12} + \Delta U_{12} > 0.
    \\
    A_{23} &> 0, \Delta U_{23} > 0, \implies Q_{23} = A_{23} + \Delta U_{23} > 0, \\
    A_{34} &= 0, \Delta U_{34} < 0, \implies Q_{34} = A_{34} + \Delta U_{34} < 0, \\
    A_{41} &< 0, \Delta U_{41} < 0, \implies Q_{41} = A_{41} + \Delta U_{41} < 0.
    \\
    P_1V_1 &= \nu R T_1, P_2V_2 = \nu R T_2, P_3V_3 = \nu R T_3, P_4V_4 = \nu R T_4 \text{ — уравнения состояния идеального газа}, \\
    &\text{Пусть $P_0$, $V_0$, $T_0$ — давление, объём и температура в точке 1 (минимальные во всём цикле):} \\
    P_1 &= P_4 = P_0, P_2 = P_3, V_1 = V_2 = V_0, V_3 = V_4, \text{остальные соотношения между объёмами и давлениями не даны, нужно считать} \\
    T_2 &= 6T_1 = 6T_0 \text{(по условию)} \implies \frac{P_2}{P_1} = \frac{P_2V_0}{P_1V_0} = \frac{P_2 V_2}{P_1 V_1}= \frac{\nu R T_2}{\nu R T_1} = \frac{T_2}{T_1} = 6 \implies P_2 = P_3 = 6 P_1 = 6 P_0, \\
    T_3 &= 4T_2 = 24T_0 \text{(по условию)} \implies \frac{V_3}{V_2} = \frac{P_3V_3}{P_2V_2}= \frac{\nu R T_3}{\nu R T_2} = \frac{T_3}{T_2} = 4 \implies V_3 = V_4 = 4 V_2 = 4 V_0.
    \\
    A_\text{цикл} &= (4P_0 - P_0)(6V_0 - V_0) = 15P_0V_0, \\
    A_{23} &= 6P_0 \cdot (4V_0 - V_0) = 18P_0V_0, \\
    \Delta U_{23} &= \frac 32 \nu R T_3 - \frac 32 \nu R T_3 = \frac 32 P_3 V_3 - \frac 32 P_2 V_2 = \frac 32 \cdot 6 P_0 \cdot 4 V_0 -  \frac 32 \cdot 6 P_0 \cdot V_0 = \frac 32 \cdot 18 \cdot P_0V_0, \\
    \Delta U_{12} &= \frac 32 \nu R T_2 - \frac 32 \nu R T_1 = \frac 32 P_2 V_2 - \frac 32 P_1 V_1 = \frac 32 \cdot 6 P_0 V_0 - \frac 32 P_0 V_0 = \frac 32 \cdot 5 \cdot P_0V_0.
    \\
    \eta &= \frac{A_\text{цикл}}{Q_+} = \frac{A_\text{цикл}}{Q_{12} + Q_{23}}  = \frac{A_\text{цикл}}{A_{12} + \Delta U_{12} + A_{23} + \Delta U_{23}} =  \\
     &= \frac{15P_0V_0}{0 + \frac 32 \cdot 5 \cdot P_0V_0 + 18P_0V_0 + \frac 32 \cdot 18 \cdot P_0V_0} = \frac{15}{\frac 32 \cdot 5 + 18 + \frac 32 \cdot 18} = \frac27 \approx 0{,}286.
     \\
    \eta_\text{Карно} &= 1 - \frac{T_\text{х}}{T_\text{н}} = 1 - \frac{T_\text{1}}{T_\text{3}} = 1 - \frac{T_0}{24T_0} = 1 - \frac 1{24}  = \frac{23}{24} \approx 0{,}958.
    \end{align*}
}
\solutionspace{360pt}

\tasknumber{2}%
\task{%
    Изобразите в координатах $PV$/$VT$/$PT$ графики изохорического нагрева в 4 раза (все 3 графика).
    Не забудьте указать оси и масштаб, начальную и конечную точки, направление движения на графике.
}
\solutionspace{100pt}

\tasknumber{3}%
\task{%
    Укажите, верны ли утверждения («да» или «нет» слева от каждого утверждения):
    \begin{enumerate}
        \item При изобарном расширении идеальный газ совершает ровно столько работы, сколько внутренней энергии теряет.
        % \item В силу третьего закона Ньютона, совершённая газом работа и работа, совершённая над ним, всегда равны по модулю и противоположны по знаку.
        \item Работу газа в некотором процессе можно вычислять как площадь под графиком в системе координат $PV$, главное лишь правильно расположить оси.
        % \item Дважды два три.
        \item При изобарном процессе внутренняя энергия идеального одноатомного газа не изменяется, даже если ему подводят тепло.
        \item Газ может совершить ненулевую работу в изотермическом процессе.
        % \item Адиабатический процесс лишь по воле случая не имеет приставки «изо»: в нём изменяются давление, температура и объём, но это не все макропараметры идеального газа.
        \item Полученное выражение для внутренней энергии идеального газа ($\frac 32 \nu RT$) применимо к двухоатомному газу, при этом, например, уравнение состояния идеального газа применимо независимо от числа атомов в молекулах газа.
    \end{enumerate}
}
\answer{%
    $\text{нет, да, нет, да, нет}$
}

\variantsplitter

\addpersonalvariant{Алёна Куприянова}

\tasknumber{1}%
\task{%
    Определите КПД цикла 12341, рабочим телом которого является идеальный одноатомный газ, если
    12 — изохорический нагрев в четыре раза,
    23 — изобарическое расширение, при котором температура растёт в три раза,
    34 — изохора, 41 — изобара.

    Определите КПД цикла Карно, температура нагревателя которого равна максимальной температуре в цикле 12341, а холодильника — минимальной.
    Ответы в обоих случаях оставьте точными в виде нескоратимой дроби, никаких округлений.
}
\answer{%
    \begin{align*}
    A_{12} &= 0, \Delta U_{12} > 0, \implies Q_{12} = A_{12} + \Delta U_{12} > 0.
    \\
    A_{23} &> 0, \Delta U_{23} > 0, \implies Q_{23} = A_{23} + \Delta U_{23} > 0, \\
    A_{34} &= 0, \Delta U_{34} < 0, \implies Q_{34} = A_{34} + \Delta U_{34} < 0, \\
    A_{41} &< 0, \Delta U_{41} < 0, \implies Q_{41} = A_{41} + \Delta U_{41} < 0.
    \\
    P_1V_1 &= \nu R T_1, P_2V_2 = \nu R T_2, P_3V_3 = \nu R T_3, P_4V_4 = \nu R T_4 \text{ — уравнения состояния идеального газа}, \\
    &\text{Пусть $P_0$, $V_0$, $T_0$ — давление, объём и температура в точке 1 (минимальные во всём цикле):} \\
    P_1 &= P_4 = P_0, P_2 = P_3, V_1 = V_2 = V_0, V_3 = V_4, \text{остальные соотношения между объёмами и давлениями не даны, нужно считать} \\
    T_2 &= 4T_1 = 4T_0 \text{(по условию)} \implies \frac{P_2}{P_1} = \frac{P_2V_0}{P_1V_0} = \frac{P_2 V_2}{P_1 V_1}= \frac{\nu R T_2}{\nu R T_1} = \frac{T_2}{T_1} = 4 \implies P_2 = P_3 = 4 P_1 = 4 P_0, \\
    T_3 &= 3T_2 = 12T_0 \text{(по условию)} \implies \frac{V_3}{V_2} = \frac{P_3V_3}{P_2V_2}= \frac{\nu R T_3}{\nu R T_2} = \frac{T_3}{T_2} = 3 \implies V_3 = V_4 = 3 V_2 = 3 V_0.
    \\
    A_\text{цикл} &= (3P_0 - P_0)(4V_0 - V_0) = 6P_0V_0, \\
    A_{23} &= 4P_0 \cdot (3V_0 - V_0) = 8P_0V_0, \\
    \Delta U_{23} &= \frac 32 \nu R T_3 - \frac 32 \nu R T_3 = \frac 32 P_3 V_3 - \frac 32 P_2 V_2 = \frac 32 \cdot 4 P_0 \cdot 3 V_0 -  \frac 32 \cdot 4 P_0 \cdot V_0 = \frac 32 \cdot 8 \cdot P_0V_0, \\
    \Delta U_{12} &= \frac 32 \nu R T_2 - \frac 32 \nu R T_1 = \frac 32 P_2 V_2 - \frac 32 P_1 V_1 = \frac 32 \cdot 4 P_0 V_0 - \frac 32 P_0 V_0 = \frac 32 \cdot 3 \cdot P_0V_0.
    \\
    \eta &= \frac{A_\text{цикл}}{Q_+} = \frac{A_\text{цикл}}{Q_{12} + Q_{23}}  = \frac{A_\text{цикл}}{A_{12} + \Delta U_{12} + A_{23} + \Delta U_{23}} =  \\
     &= \frac{6P_0V_0}{0 + \frac 32 \cdot 3 \cdot P_0V_0 + 8P_0V_0 + \frac 32 \cdot 8 \cdot P_0V_0} = \frac{6}{\frac 32 \cdot 3 + 8 + \frac 32 \cdot 8} = \frac{12}{49} \approx 0{,}245.
     \\
    \eta_\text{Карно} &= 1 - \frac{T_\text{х}}{T_\text{н}} = 1 - \frac{T_\text{1}}{T_\text{3}} = 1 - \frac{T_0}{12T_0} = 1 - \frac 1{12}  = \frac{11}{12} \approx 0{,}917.
    \end{align*}
}
\solutionspace{360pt}

\tasknumber{2}%
\task{%
    Изобразите в координатах $PV$/$VT$/$PT$ графики изохорического охлаждения в 4 раза (все 3 графика).
    Не забудьте указать оси и масштаб, начальную и конечную точки, направление движения на графике.
}
\solutionspace{100pt}

\tasknumber{3}%
\task{%
    Укажите, верны ли утверждения («да» или «нет» слева от каждого утверждения):
    \begin{enumerate}
        \item При адиабатическом расширении идеальный газ совершает ровно столько работы, сколько внутренней энергии теряет.
        % \item В силу третьего закона Ньютона, совершённая газом работа и работа, совершённая над ним, всегда равны по модулю и противоположны по знаку.
        \item Работу газа в некотором процессе можно вычислять как площадь под графиком в системе координат $VT$, главное лишь правильно расположить оси.
        % \item Дважды два пять.
        \item При изотермическом процессе внутренняя энергия идеального одноатомного газа не изменяется, даже если ему подводят тепло.
        \item Газ может совершить ненулевую работу в изохорном процессе.
        % \item Адиабатический процесс лишь по воле случая не имеет приставки «изо»: в нём изменяются давление, температура и объём, но это не все макропараметры идеального газа.
        \item Полученное выражение для внутренней энергии идеального газа ($\frac 32 \nu RT$) применимо к одноатомному газу, при этом, например, уравнение состояния идеального газа применимо независимо от числа атомов в молекулах газа.
    \end{enumerate}
}
\answer{%
    $\text{да, нет, да, нет, да}$
}

\variantsplitter

\addpersonalvariant{Ярослав Лавровский}

\tasknumber{1}%
\task{%
    Определите КПД цикла 12341, рабочим телом которого является идеальный одноатомный газ, если
    12 — изохорический нагрев в шесть раз,
    23 — изобарическое расширение, при котором температура растёт в пять раз,
    34 — изохора, 41 — изобара.

    Определите КПД цикла Карно, температура нагревателя которого равна максимальной температуре в цикле 12341, а холодильника — минимальной.
    Ответы в обоих случаях оставьте точными в виде нескоратимой дроби, никаких округлений.
}
\answer{%
    \begin{align*}
    A_{12} &= 0, \Delta U_{12} > 0, \implies Q_{12} = A_{12} + \Delta U_{12} > 0.
    \\
    A_{23} &> 0, \Delta U_{23} > 0, \implies Q_{23} = A_{23} + \Delta U_{23} > 0, \\
    A_{34} &= 0, \Delta U_{34} < 0, \implies Q_{34} = A_{34} + \Delta U_{34} < 0, \\
    A_{41} &< 0, \Delta U_{41} < 0, \implies Q_{41} = A_{41} + \Delta U_{41} < 0.
    \\
    P_1V_1 &= \nu R T_1, P_2V_2 = \nu R T_2, P_3V_3 = \nu R T_3, P_4V_4 = \nu R T_4 \text{ — уравнения состояния идеального газа}, \\
    &\text{Пусть $P_0$, $V_0$, $T_0$ — давление, объём и температура в точке 1 (минимальные во всём цикле):} \\
    P_1 &= P_4 = P_0, P_2 = P_3, V_1 = V_2 = V_0, V_3 = V_4, \text{остальные соотношения между объёмами и давлениями не даны, нужно считать} \\
    T_2 &= 6T_1 = 6T_0 \text{(по условию)} \implies \frac{P_2}{P_1} = \frac{P_2V_0}{P_1V_0} = \frac{P_2 V_2}{P_1 V_1}= \frac{\nu R T_2}{\nu R T_1} = \frac{T_2}{T_1} = 6 \implies P_2 = P_3 = 6 P_1 = 6 P_0, \\
    T_3 &= 5T_2 = 30T_0 \text{(по условию)} \implies \frac{V_3}{V_2} = \frac{P_3V_3}{P_2V_2}= \frac{\nu R T_3}{\nu R T_2} = \frac{T_3}{T_2} = 5 \implies V_3 = V_4 = 5 V_2 = 5 V_0.
    \\
    A_\text{цикл} &= (5P_0 - P_0)(6V_0 - V_0) = 20P_0V_0, \\
    A_{23} &= 6P_0 \cdot (5V_0 - V_0) = 24P_0V_0, \\
    \Delta U_{23} &= \frac 32 \nu R T_3 - \frac 32 \nu R T_3 = \frac 32 P_3 V_3 - \frac 32 P_2 V_2 = \frac 32 \cdot 6 P_0 \cdot 5 V_0 -  \frac 32 \cdot 6 P_0 \cdot V_0 = \frac 32 \cdot 24 \cdot P_0V_0, \\
    \Delta U_{12} &= \frac 32 \nu R T_2 - \frac 32 \nu R T_1 = \frac 32 P_2 V_2 - \frac 32 P_1 V_1 = \frac 32 \cdot 6 P_0 V_0 - \frac 32 P_0 V_0 = \frac 32 \cdot 5 \cdot P_0V_0.
    \\
    \eta &= \frac{A_\text{цикл}}{Q_+} = \frac{A_\text{цикл}}{Q_{12} + Q_{23}}  = \frac{A_\text{цикл}}{A_{12} + \Delta U_{12} + A_{23} + \Delta U_{23}} =  \\
     &= \frac{20P_0V_0}{0 + \frac 32 \cdot 5 \cdot P_0V_0 + 24P_0V_0 + \frac 32 \cdot 24 \cdot P_0V_0} = \frac{20}{\frac 32 \cdot 5 + 24 + \frac 32 \cdot 24} = \frac8{27} \approx 0{,}296.
     \\
    \eta_\text{Карно} &= 1 - \frac{T_\text{х}}{T_\text{н}} = 1 - \frac{T_\text{1}}{T_\text{3}} = 1 - \frac{T_0}{30T_0} = 1 - \frac 1{30}  = \frac{29}{30} \approx 0{,}967.
    \end{align*}
}
\solutionspace{360pt}

\tasknumber{2}%
\task{%
    Изобразите в координатах $PV$/$VT$/$PT$ графики изохорического охлаждения в 3 раза (все 3 графика).
    Не забудьте указать оси и масштаб, начальную и конечную точки, направление движения на графике.
}
\solutionspace{100pt}

\tasknumber{3}%
\task{%
    Укажите, верны ли утверждения («да» или «нет» слева от каждого утверждения):
    \begin{enumerate}
        \item При изобарном расширении идеальный газ совершает ровно столько работы, сколько внутренней энергии теряет.
        % \item В силу третьего закона Ньютона, совершённая газом работа и работа, совершённая над ним, всегда равны по модулю и противоположны по знаку.
        \item Работу газа в некотором процессе можно вычислять как площадь под графиком в системе координат $VT$, главное лишь правильно расположить оси.
        % \item Дважды два три.
        \item При изобарном процессе внутренняя энергия идеального одноатомного газа не изменяется, даже если ему подводят тепло.
        \item Газ может совершить ненулевую работу в изобарном процессе.
        % \item Адиабатический процесс лишь по воле случая не имеет приставки «изо»: в нём изменяются давление, температура и объём, но это не все макропараметры идеального газа.
        \item Полученное выражение для внутренней энергии идеального газа ($\frac 32 \nu RT$) применимо к трёхатомному газу, при этом, например, уравнение состояния идеального газа применимо независимо от числа атомов в молекулах газа.
    \end{enumerate}
}
\answer{%
    $\text{нет, нет, нет, да, нет}$
}

\variantsplitter

\addpersonalvariant{Анастасия Ламанова}

\tasknumber{1}%
\task{%
    Определите КПД цикла 12341, рабочим телом которого является идеальный одноатомный газ, если
    12 — изохорический нагрев в три раза,
    23 — изобарическое расширение, при котором температура растёт в три раза,
    34 — изохора, 41 — изобара.

    Определите КПД цикла Карно, температура нагревателя которого равна максимальной температуре в цикле 12341, а холодильника — минимальной.
    Ответы в обоих случаях оставьте точными в виде нескоратимой дроби, никаких округлений.
}
\answer{%
    \begin{align*}
    A_{12} &= 0, \Delta U_{12} > 0, \implies Q_{12} = A_{12} + \Delta U_{12} > 0.
    \\
    A_{23} &> 0, \Delta U_{23} > 0, \implies Q_{23} = A_{23} + \Delta U_{23} > 0, \\
    A_{34} &= 0, \Delta U_{34} < 0, \implies Q_{34} = A_{34} + \Delta U_{34} < 0, \\
    A_{41} &< 0, \Delta U_{41} < 0, \implies Q_{41} = A_{41} + \Delta U_{41} < 0.
    \\
    P_1V_1 &= \nu R T_1, P_2V_2 = \nu R T_2, P_3V_3 = \nu R T_3, P_4V_4 = \nu R T_4 \text{ — уравнения состояния идеального газа}, \\
    &\text{Пусть $P_0$, $V_0$, $T_0$ — давление, объём и температура в точке 1 (минимальные во всём цикле):} \\
    P_1 &= P_4 = P_0, P_2 = P_3, V_1 = V_2 = V_0, V_3 = V_4, \text{остальные соотношения между объёмами и давлениями не даны, нужно считать} \\
    T_2 &= 3T_1 = 3T_0 \text{(по условию)} \implies \frac{P_2}{P_1} = \frac{P_2V_0}{P_1V_0} = \frac{P_2 V_2}{P_1 V_1}= \frac{\nu R T_2}{\nu R T_1} = \frac{T_2}{T_1} = 3 \implies P_2 = P_3 = 3 P_1 = 3 P_0, \\
    T_3 &= 3T_2 = 9T_0 \text{(по условию)} \implies \frac{V_3}{V_2} = \frac{P_3V_3}{P_2V_2}= \frac{\nu R T_3}{\nu R T_2} = \frac{T_3}{T_2} = 3 \implies V_3 = V_4 = 3 V_2 = 3 V_0.
    \\
    A_\text{цикл} &= (3P_0 - P_0)(3V_0 - V_0) = 4P_0V_0, \\
    A_{23} &= 3P_0 \cdot (3V_0 - V_0) = 6P_0V_0, \\
    \Delta U_{23} &= \frac 32 \nu R T_3 - \frac 32 \nu R T_3 = \frac 32 P_3 V_3 - \frac 32 P_2 V_2 = \frac 32 \cdot 3 P_0 \cdot 3 V_0 -  \frac 32 \cdot 3 P_0 \cdot V_0 = \frac 32 \cdot 6 \cdot P_0V_0, \\
    \Delta U_{12} &= \frac 32 \nu R T_2 - \frac 32 \nu R T_1 = \frac 32 P_2 V_2 - \frac 32 P_1 V_1 = \frac 32 \cdot 3 P_0 V_0 - \frac 32 P_0 V_0 = \frac 32 \cdot 2 \cdot P_0V_0.
    \\
    \eta &= \frac{A_\text{цикл}}{Q_+} = \frac{A_\text{цикл}}{Q_{12} + Q_{23}}  = \frac{A_\text{цикл}}{A_{12} + \Delta U_{12} + A_{23} + \Delta U_{23}} =  \\
     &= \frac{4P_0V_0}{0 + \frac 32 \cdot 2 \cdot P_0V_0 + 6P_0V_0 + \frac 32 \cdot 6 \cdot P_0V_0} = \frac{4}{\frac 32 \cdot 2 + 6 + \frac 32 \cdot 6} = \frac29 \approx 0{,}222.
     \\
    \eta_\text{Карно} &= 1 - \frac{T_\text{х}}{T_\text{н}} = 1 - \frac{T_\text{1}}{T_\text{3}} = 1 - \frac{T_0}{9T_0} = 1 - \frac 1{9}  = \frac89 \approx 0{,}889.
    \end{align*}
}
\solutionspace{360pt}

\tasknumber{2}%
\task{%
    Изобразите в координатах $PV$/$VT$/$PT$ графики изобарического расширения в 3 раза (все 3 графика).
    Не забудьте указать оси и масштаб, начальную и конечную точки, направление движения на графике.
}
\solutionspace{100pt}

\tasknumber{3}%
\task{%
    Укажите, верны ли утверждения («да» или «нет» слева от каждого утверждения):
    \begin{enumerate}
        \item При адиабатическом расширении идеальный газ совершает ровно столько работы, сколько внутренней энергии теряет.
        % \item В силу третьего закона Ньютона, совершённая газом работа и работа, совершённая над ним, всегда равны по модулю и противоположны по знаку.
        \item Работу газа в некотором процессе можно вычислять как площадь под графиком в системе координат $PT$, главное лишь правильно расположить оси.
        % \item Дважды два три.
        \item При изотермическом процессе внутренняя энергия идеального одноатомного газа не изменяется, даже если ему подводят тепло.
        \item Газ может совершить ненулевую работу в изохорном процессе.
        % \item Адиабатический процесс лишь по воле случая не имеет приставки «изо»: в нём изменяются давление, температура и объём, но это не все макропараметры идеального газа.
        \item Полученное выражение для внутренней энергии идеального газа ($\frac 32 \nu RT$) применимо к двухоатомному газу, при этом, например, уравнение состояния идеального газа применимо независимо от числа атомов в молекулах газа.
    \end{enumerate}
}
\answer{%
    $\text{да, нет, да, нет, нет}$
}

\variantsplitter

\addpersonalvariant{Виктория Легонькова}

\tasknumber{1}%
\task{%
    Определите КПД цикла 12341, рабочим телом которого является идеальный одноатомный газ, если
    12 — изохорический нагрев в три раза,
    23 — изобарическое расширение, при котором температура растёт в три раза,
    34 — изохора, 41 — изобара.

    Определите КПД цикла Карно, температура нагревателя которого равна максимальной температуре в цикле 12341, а холодильника — минимальной.
    Ответы в обоих случаях оставьте точными в виде нескоратимой дроби, никаких округлений.
}
\answer{%
    \begin{align*}
    A_{12} &= 0, \Delta U_{12} > 0, \implies Q_{12} = A_{12} + \Delta U_{12} > 0.
    \\
    A_{23} &> 0, \Delta U_{23} > 0, \implies Q_{23} = A_{23} + \Delta U_{23} > 0, \\
    A_{34} &= 0, \Delta U_{34} < 0, \implies Q_{34} = A_{34} + \Delta U_{34} < 0, \\
    A_{41} &< 0, \Delta U_{41} < 0, \implies Q_{41} = A_{41} + \Delta U_{41} < 0.
    \\
    P_1V_1 &= \nu R T_1, P_2V_2 = \nu R T_2, P_3V_3 = \nu R T_3, P_4V_4 = \nu R T_4 \text{ — уравнения состояния идеального газа}, \\
    &\text{Пусть $P_0$, $V_0$, $T_0$ — давление, объём и температура в точке 1 (минимальные во всём цикле):} \\
    P_1 &= P_4 = P_0, P_2 = P_3, V_1 = V_2 = V_0, V_3 = V_4, \text{остальные соотношения между объёмами и давлениями не даны, нужно считать} \\
    T_2 &= 3T_1 = 3T_0 \text{(по условию)} \implies \frac{P_2}{P_1} = \frac{P_2V_0}{P_1V_0} = \frac{P_2 V_2}{P_1 V_1}= \frac{\nu R T_2}{\nu R T_1} = \frac{T_2}{T_1} = 3 \implies P_2 = P_3 = 3 P_1 = 3 P_0, \\
    T_3 &= 3T_2 = 9T_0 \text{(по условию)} \implies \frac{V_3}{V_2} = \frac{P_3V_3}{P_2V_2}= \frac{\nu R T_3}{\nu R T_2} = \frac{T_3}{T_2} = 3 \implies V_3 = V_4 = 3 V_2 = 3 V_0.
    \\
    A_\text{цикл} &= (3P_0 - P_0)(3V_0 - V_0) = 4P_0V_0, \\
    A_{23} &= 3P_0 \cdot (3V_0 - V_0) = 6P_0V_0, \\
    \Delta U_{23} &= \frac 32 \nu R T_3 - \frac 32 \nu R T_3 = \frac 32 P_3 V_3 - \frac 32 P_2 V_2 = \frac 32 \cdot 3 P_0 \cdot 3 V_0 -  \frac 32 \cdot 3 P_0 \cdot V_0 = \frac 32 \cdot 6 \cdot P_0V_0, \\
    \Delta U_{12} &= \frac 32 \nu R T_2 - \frac 32 \nu R T_1 = \frac 32 P_2 V_2 - \frac 32 P_1 V_1 = \frac 32 \cdot 3 P_0 V_0 - \frac 32 P_0 V_0 = \frac 32 \cdot 2 \cdot P_0V_0.
    \\
    \eta &= \frac{A_\text{цикл}}{Q_+} = \frac{A_\text{цикл}}{Q_{12} + Q_{23}}  = \frac{A_\text{цикл}}{A_{12} + \Delta U_{12} + A_{23} + \Delta U_{23}} =  \\
     &= \frac{4P_0V_0}{0 + \frac 32 \cdot 2 \cdot P_0V_0 + 6P_0V_0 + \frac 32 \cdot 6 \cdot P_0V_0} = \frac{4}{\frac 32 \cdot 2 + 6 + \frac 32 \cdot 6} = \frac29 \approx 0{,}222.
     \\
    \eta_\text{Карно} &= 1 - \frac{T_\text{х}}{T_\text{н}} = 1 - \frac{T_\text{1}}{T_\text{3}} = 1 - \frac{T_0}{9T_0} = 1 - \frac 1{9}  = \frac89 \approx 0{,}889.
    \end{align*}
}
\solutionspace{360pt}

\tasknumber{2}%
\task{%
    Изобразите в координатах $PV$/$VT$/$PT$ графики изотермического понижения давления в 3 раза (все 3 графика).
    Не забудьте указать оси и масштаб, начальную и конечную точки, направление движения на графике.
}
\solutionspace{100pt}

\tasknumber{3}%
\task{%
    Укажите, верны ли утверждения («да» или «нет» слева от каждого утверждения):
    \begin{enumerate}
        \item При адиабатическом расширении идеальный газ совершает ровно столько работы, сколько внутренней энергии теряет.
        % \item В силу третьего закона Ньютона, совершённая газом работа и работа, совершённая над ним, всегда равны по модулю и противоположны по знаку.
        \item Работу газа в некотором процессе можно вычислять как площадь под графиком в системе координат $PV$, главное лишь правильно расположить оси.
        % \item Дважды два пять.
        \item При изотермическом процессе внутренняя энергия идеального одноатомного газа не изменяется, даже если ему подводят тепло.
        \item Газ может совершить ненулевую работу в изобарном процессе.
        % \item Адиабатический процесс лишь по воле случая не имеет приставки «изо»: в нём изменяются давление, температура и объём, но это не все макропараметры идеального газа.
        \item Полученное выражение для внутренней энергии идеального газа ($\frac 32 \nu RT$) применимо к двухоатомному газу, при этом, например, уравнение состояния идеального газа применимо независимо от числа атомов в молекулах газа.
    \end{enumerate}
}
\answer{%
    $\text{да, да, да, да, нет}$
}

\variantsplitter

\addpersonalvariant{Семён Мартынов}

\tasknumber{1}%
\task{%
    Определите КПД цикла 12341, рабочим телом которого является идеальный одноатомный газ, если
    12 — изохорический нагрев в четыре раза,
    23 — изобарическое расширение, при котором температура растёт в два раза,
    34 — изохора, 41 — изобара.

    Определите КПД цикла Карно, температура нагревателя которого равна максимальной температуре в цикле 12341, а холодильника — минимальной.
    Ответы в обоих случаях оставьте точными в виде нескоратимой дроби, никаких округлений.
}
\answer{%
    \begin{align*}
    A_{12} &= 0, \Delta U_{12} > 0, \implies Q_{12} = A_{12} + \Delta U_{12} > 0.
    \\
    A_{23} &> 0, \Delta U_{23} > 0, \implies Q_{23} = A_{23} + \Delta U_{23} > 0, \\
    A_{34} &= 0, \Delta U_{34} < 0, \implies Q_{34} = A_{34} + \Delta U_{34} < 0, \\
    A_{41} &< 0, \Delta U_{41} < 0, \implies Q_{41} = A_{41} + \Delta U_{41} < 0.
    \\
    P_1V_1 &= \nu R T_1, P_2V_2 = \nu R T_2, P_3V_3 = \nu R T_3, P_4V_4 = \nu R T_4 \text{ — уравнения состояния идеального газа}, \\
    &\text{Пусть $P_0$, $V_0$, $T_0$ — давление, объём и температура в точке 1 (минимальные во всём цикле):} \\
    P_1 &= P_4 = P_0, P_2 = P_3, V_1 = V_2 = V_0, V_3 = V_4, \text{остальные соотношения между объёмами и давлениями не даны, нужно считать} \\
    T_2 &= 4T_1 = 4T_0 \text{(по условию)} \implies \frac{P_2}{P_1} = \frac{P_2V_0}{P_1V_0} = \frac{P_2 V_2}{P_1 V_1}= \frac{\nu R T_2}{\nu R T_1} = \frac{T_2}{T_1} = 4 \implies P_2 = P_3 = 4 P_1 = 4 P_0, \\
    T_3 &= 2T_2 = 8T_0 \text{(по условию)} \implies \frac{V_3}{V_2} = \frac{P_3V_3}{P_2V_2}= \frac{\nu R T_3}{\nu R T_2} = \frac{T_3}{T_2} = 2 \implies V_3 = V_4 = 2 V_2 = 2 V_0.
    \\
    A_\text{цикл} &= (2P_0 - P_0)(4V_0 - V_0) = 3P_0V_0, \\
    A_{23} &= 4P_0 \cdot (2V_0 - V_0) = 4P_0V_0, \\
    \Delta U_{23} &= \frac 32 \nu R T_3 - \frac 32 \nu R T_3 = \frac 32 P_3 V_3 - \frac 32 P_2 V_2 = \frac 32 \cdot 4 P_0 \cdot 2 V_0 -  \frac 32 \cdot 4 P_0 \cdot V_0 = \frac 32 \cdot 4 \cdot P_0V_0, \\
    \Delta U_{12} &= \frac 32 \nu R T_2 - \frac 32 \nu R T_1 = \frac 32 P_2 V_2 - \frac 32 P_1 V_1 = \frac 32 \cdot 4 P_0 V_0 - \frac 32 P_0 V_0 = \frac 32 \cdot 3 \cdot P_0V_0.
    \\
    \eta &= \frac{A_\text{цикл}}{Q_+} = \frac{A_\text{цикл}}{Q_{12} + Q_{23}}  = \frac{A_\text{цикл}}{A_{12} + \Delta U_{12} + A_{23} + \Delta U_{23}} =  \\
     &= \frac{3P_0V_0}{0 + \frac 32 \cdot 3 \cdot P_0V_0 + 4P_0V_0 + \frac 32 \cdot 4 \cdot P_0V_0} = \frac{3}{\frac 32 \cdot 3 + 4 + \frac 32 \cdot 4} = \frac6{29} \approx 0{,}207.
     \\
    \eta_\text{Карно} &= 1 - \frac{T_\text{х}}{T_\text{н}} = 1 - \frac{T_\text{1}}{T_\text{3}} = 1 - \frac{T_0}{8T_0} = 1 - \frac 1{8}  = \frac78 \approx 0{,}875.
    \end{align*}
}
\solutionspace{360pt}

\tasknumber{2}%
\task{%
    Изобразите в координатах $PV$/$VT$/$PT$ графики изотермического повышения давления в 4 раза (все 3 графика).
    Не забудьте указать оси и масштаб, начальную и конечную точки, направление движения на графике.
}
\solutionspace{100pt}

\tasknumber{3}%
\task{%
    Укажите, верны ли утверждения («да» или «нет» слева от каждого утверждения):
    \begin{enumerate}
        \item При изобарном расширении идеальный газ совершает ровно столько работы, сколько внутренней энергии теряет.
        % \item В силу третьего закона Ньютона, совершённая газом работа и работа, совершённая над ним, всегда равны по модулю и противоположны по знаку.
        \item Работу газа в некотором процессе можно вычислять как площадь под графиком в системе координат $VT$, главное лишь правильно расположить оси.
        % \item Дважды два пять.
        \item При изотермическом процессе внутренняя энергия идеального одноатомного газа не изменяется, даже если ему подводят тепло.
        \item Газ может совершить ненулевую работу в изохорном процессе.
        % \item Адиабатический процесс лишь по воле случая не имеет приставки «изо»: в нём изменяются давление, температура и объём, но это не все макропараметры идеального газа.
        \item Полученное выражение для внутренней энергии идеального газа ($\frac 32 \nu RT$) применимо к двухоатомному газу, при этом, например, уравнение состояния идеального газа применимо независимо от числа атомов в молекулах газа.
    \end{enumerate}
}
\answer{%
    $\text{нет, нет, да, нет, нет}$
}

\variantsplitter

\addpersonalvariant{Варвара Минаева}

\tasknumber{1}%
\task{%
    Определите КПД цикла 12341, рабочим телом которого является идеальный одноатомный газ, если
    12 — изохорический нагрев в четыре раза,
    23 — изобарическое расширение, при котором температура растёт в три раза,
    34 — изохора, 41 — изобара.

    Определите КПД цикла Карно, температура нагревателя которого равна максимальной температуре в цикле 12341, а холодильника — минимальной.
    Ответы в обоих случаях оставьте точными в виде нескоратимой дроби, никаких округлений.
}
\answer{%
    \begin{align*}
    A_{12} &= 0, \Delta U_{12} > 0, \implies Q_{12} = A_{12} + \Delta U_{12} > 0.
    \\
    A_{23} &> 0, \Delta U_{23} > 0, \implies Q_{23} = A_{23} + \Delta U_{23} > 0, \\
    A_{34} &= 0, \Delta U_{34} < 0, \implies Q_{34} = A_{34} + \Delta U_{34} < 0, \\
    A_{41} &< 0, \Delta U_{41} < 0, \implies Q_{41} = A_{41} + \Delta U_{41} < 0.
    \\
    P_1V_1 &= \nu R T_1, P_2V_2 = \nu R T_2, P_3V_3 = \nu R T_3, P_4V_4 = \nu R T_4 \text{ — уравнения состояния идеального газа}, \\
    &\text{Пусть $P_0$, $V_0$, $T_0$ — давление, объём и температура в точке 1 (минимальные во всём цикле):} \\
    P_1 &= P_4 = P_0, P_2 = P_3, V_1 = V_2 = V_0, V_3 = V_4, \text{остальные соотношения между объёмами и давлениями не даны, нужно считать} \\
    T_2 &= 4T_1 = 4T_0 \text{(по условию)} \implies \frac{P_2}{P_1} = \frac{P_2V_0}{P_1V_0} = \frac{P_2 V_2}{P_1 V_1}= \frac{\nu R T_2}{\nu R T_1} = \frac{T_2}{T_1} = 4 \implies P_2 = P_3 = 4 P_1 = 4 P_0, \\
    T_3 &= 3T_2 = 12T_0 \text{(по условию)} \implies \frac{V_3}{V_2} = \frac{P_3V_3}{P_2V_2}= \frac{\nu R T_3}{\nu R T_2} = \frac{T_3}{T_2} = 3 \implies V_3 = V_4 = 3 V_2 = 3 V_0.
    \\
    A_\text{цикл} &= (3P_0 - P_0)(4V_0 - V_0) = 6P_0V_0, \\
    A_{23} &= 4P_0 \cdot (3V_0 - V_0) = 8P_0V_0, \\
    \Delta U_{23} &= \frac 32 \nu R T_3 - \frac 32 \nu R T_3 = \frac 32 P_3 V_3 - \frac 32 P_2 V_2 = \frac 32 \cdot 4 P_0 \cdot 3 V_0 -  \frac 32 \cdot 4 P_0 \cdot V_0 = \frac 32 \cdot 8 \cdot P_0V_0, \\
    \Delta U_{12} &= \frac 32 \nu R T_2 - \frac 32 \nu R T_1 = \frac 32 P_2 V_2 - \frac 32 P_1 V_1 = \frac 32 \cdot 4 P_0 V_0 - \frac 32 P_0 V_0 = \frac 32 \cdot 3 \cdot P_0V_0.
    \\
    \eta &= \frac{A_\text{цикл}}{Q_+} = \frac{A_\text{цикл}}{Q_{12} + Q_{23}}  = \frac{A_\text{цикл}}{A_{12} + \Delta U_{12} + A_{23} + \Delta U_{23}} =  \\
     &= \frac{6P_0V_0}{0 + \frac 32 \cdot 3 \cdot P_0V_0 + 8P_0V_0 + \frac 32 \cdot 8 \cdot P_0V_0} = \frac{6}{\frac 32 \cdot 3 + 8 + \frac 32 \cdot 8} = \frac{12}{49} \approx 0{,}245.
     \\
    \eta_\text{Карно} &= 1 - \frac{T_\text{х}}{T_\text{н}} = 1 - \frac{T_\text{1}}{T_\text{3}} = 1 - \frac{T_0}{12T_0} = 1 - \frac 1{12}  = \frac{11}{12} \approx 0{,}917.
    \end{align*}
}
\solutionspace{360pt}

\tasknumber{2}%
\task{%
    Изобразите в координатах $PV$/$VT$/$PT$ графики изотермического повышения давления в 3 раза (все 3 графика).
    Не забудьте указать оси и масштаб, начальную и конечную точки, направление движения на графике.
}
\solutionspace{100pt}

\tasknumber{3}%
\task{%
    Укажите, верны ли утверждения («да» или «нет» слева от каждого утверждения):
    \begin{enumerate}
        \item При изобарном расширении идеальный газ совершает ровно столько работы, сколько внутренней энергии теряет.
        % \item В силу третьего закона Ньютона, совершённая газом работа и работа, совершённая над ним, всегда равны по модулю и противоположны по знаку.
        \item Работу газа в некотором процессе можно вычислять как площадь под графиком в системе координат $PT$, главное лишь правильно расположить оси.
        % \item Дважды два пять.
        \item При изохорном процессе внутренняя энергия идеального одноатомного газа не изменяется, даже если ему подводят тепло.
        \item Газ может совершить ненулевую работу в изотермическом процессе.
        % \item Адиабатический процесс лишь по воле случая не имеет приставки «изо»: в нём изменяются давление, температура и объём, но это не все макропараметры идеального газа.
        \item Полученное выражение для внутренней энергии идеального газа ($\frac 32 \nu RT$) применимо к трёхатомному газу, при этом, например, уравнение состояния идеального газа применимо независимо от числа атомов в молекулах газа.
    \end{enumerate}
}
\answer{%
    $\text{нет, нет, нет, да, нет}$
}

\variantsplitter

\addpersonalvariant{Леонид Никитин}

\tasknumber{1}%
\task{%
    Определите КПД цикла 12341, рабочим телом которого является идеальный одноатомный газ, если
    12 — изохорический нагрев в три раза,
    23 — изобарическое расширение, при котором температура растёт в два раза,
    34 — изохора, 41 — изобара.

    Определите КПД цикла Карно, температура нагревателя которого равна максимальной температуре в цикле 12341, а холодильника — минимальной.
    Ответы в обоих случаях оставьте точными в виде нескоратимой дроби, никаких округлений.
}
\answer{%
    \begin{align*}
    A_{12} &= 0, \Delta U_{12} > 0, \implies Q_{12} = A_{12} + \Delta U_{12} > 0.
    \\
    A_{23} &> 0, \Delta U_{23} > 0, \implies Q_{23} = A_{23} + \Delta U_{23} > 0, \\
    A_{34} &= 0, \Delta U_{34} < 0, \implies Q_{34} = A_{34} + \Delta U_{34} < 0, \\
    A_{41} &< 0, \Delta U_{41} < 0, \implies Q_{41} = A_{41} + \Delta U_{41} < 0.
    \\
    P_1V_1 &= \nu R T_1, P_2V_2 = \nu R T_2, P_3V_3 = \nu R T_3, P_4V_4 = \nu R T_4 \text{ — уравнения состояния идеального газа}, \\
    &\text{Пусть $P_0$, $V_0$, $T_0$ — давление, объём и температура в точке 1 (минимальные во всём цикле):} \\
    P_1 &= P_4 = P_0, P_2 = P_3, V_1 = V_2 = V_0, V_3 = V_4, \text{остальные соотношения между объёмами и давлениями не даны, нужно считать} \\
    T_2 &= 3T_1 = 3T_0 \text{(по условию)} \implies \frac{P_2}{P_1} = \frac{P_2V_0}{P_1V_0} = \frac{P_2 V_2}{P_1 V_1}= \frac{\nu R T_2}{\nu R T_1} = \frac{T_2}{T_1} = 3 \implies P_2 = P_3 = 3 P_1 = 3 P_0, \\
    T_3 &= 2T_2 = 6T_0 \text{(по условию)} \implies \frac{V_3}{V_2} = \frac{P_3V_3}{P_2V_2}= \frac{\nu R T_3}{\nu R T_2} = \frac{T_3}{T_2} = 2 \implies V_3 = V_4 = 2 V_2 = 2 V_0.
    \\
    A_\text{цикл} &= (2P_0 - P_0)(3V_0 - V_0) = 2P_0V_0, \\
    A_{23} &= 3P_0 \cdot (2V_0 - V_0) = 3P_0V_0, \\
    \Delta U_{23} &= \frac 32 \nu R T_3 - \frac 32 \nu R T_3 = \frac 32 P_3 V_3 - \frac 32 P_2 V_2 = \frac 32 \cdot 3 P_0 \cdot 2 V_0 -  \frac 32 \cdot 3 P_0 \cdot V_0 = \frac 32 \cdot 3 \cdot P_0V_0, \\
    \Delta U_{12} &= \frac 32 \nu R T_2 - \frac 32 \nu R T_1 = \frac 32 P_2 V_2 - \frac 32 P_1 V_1 = \frac 32 \cdot 3 P_0 V_0 - \frac 32 P_0 V_0 = \frac 32 \cdot 2 \cdot P_0V_0.
    \\
    \eta &= \frac{A_\text{цикл}}{Q_+} = \frac{A_\text{цикл}}{Q_{12} + Q_{23}}  = \frac{A_\text{цикл}}{A_{12} + \Delta U_{12} + A_{23} + \Delta U_{23}} =  \\
     &= \frac{2P_0V_0}{0 + \frac 32 \cdot 2 \cdot P_0V_0 + 3P_0V_0 + \frac 32 \cdot 3 \cdot P_0V_0} = \frac{2}{\frac 32 \cdot 2 + 3 + \frac 32 \cdot 3} = \frac4{21} \approx 0{,}190.
     \\
    \eta_\text{Карно} &= 1 - \frac{T_\text{х}}{T_\text{н}} = 1 - \frac{T_\text{1}}{T_\text{3}} = 1 - \frac{T_0}{6T_0} = 1 - \frac 1{6}  = \frac56 \approx 0{,}833.
    \end{align*}
}
\solutionspace{360pt}

\tasknumber{2}%
\task{%
    Изобразите в координатах $PV$/$VT$/$PT$ графики изотермического понижения давления в 4 раза (все 3 графика).
    Не забудьте указать оси и масштаб, начальную и конечную точки, направление движения на графике.
}
\solutionspace{100pt}

\tasknumber{3}%
\task{%
    Укажите, верны ли утверждения («да» или «нет» слева от каждого утверждения):
    \begin{enumerate}
        \item При изобарном расширении идеальный газ совершает ровно столько работы, сколько внутренней энергии теряет.
        % \item В силу третьего закона Ньютона, совершённая газом работа и работа, совершённая над ним, всегда равны по модулю и противоположны по знаку.
        \item Работу газа в некотором процессе можно вычислять как площадь под графиком в системе координат $PV$, главное лишь правильно расположить оси.
        % \item Дважды два пять.
        \item При изобарном процессе внутренняя энергия идеального одноатомного газа не изменяется, даже если ему подводят тепло.
        \item Газ может совершить ненулевую работу в изотермическом процессе.
        % \item Адиабатический процесс лишь по воле случая не имеет приставки «изо»: в нём изменяются давление, температура и объём, но это не все макропараметры идеального газа.
        \item Полученное выражение для внутренней энергии идеального газа ($\frac 32 \nu RT$) применимо к одноатомному газу, при этом, например, уравнение состояния идеального газа применимо независимо от числа атомов в молекулах газа.
    \end{enumerate}
}
\answer{%
    $\text{нет, да, нет, да, да}$
}

\variantsplitter

\addpersonalvariant{Тимофей Полетаев}

\tasknumber{1}%
\task{%
    Определите КПД цикла 12341, рабочим телом которого является идеальный одноатомный газ, если
    12 — изохорический нагрев в четыре раза,
    23 — изобарическое расширение, при котором температура растёт в четыре раза,
    34 — изохора, 41 — изобара.

    Определите КПД цикла Карно, температура нагревателя которого равна максимальной температуре в цикле 12341, а холодильника — минимальной.
    Ответы в обоих случаях оставьте точными в виде нескоратимой дроби, никаких округлений.
}
\answer{%
    \begin{align*}
    A_{12} &= 0, \Delta U_{12} > 0, \implies Q_{12} = A_{12} + \Delta U_{12} > 0.
    \\
    A_{23} &> 0, \Delta U_{23} > 0, \implies Q_{23} = A_{23} + \Delta U_{23} > 0, \\
    A_{34} &= 0, \Delta U_{34} < 0, \implies Q_{34} = A_{34} + \Delta U_{34} < 0, \\
    A_{41} &< 0, \Delta U_{41} < 0, \implies Q_{41} = A_{41} + \Delta U_{41} < 0.
    \\
    P_1V_1 &= \nu R T_1, P_2V_2 = \nu R T_2, P_3V_3 = \nu R T_3, P_4V_4 = \nu R T_4 \text{ — уравнения состояния идеального газа}, \\
    &\text{Пусть $P_0$, $V_0$, $T_0$ — давление, объём и температура в точке 1 (минимальные во всём цикле):} \\
    P_1 &= P_4 = P_0, P_2 = P_3, V_1 = V_2 = V_0, V_3 = V_4, \text{остальные соотношения между объёмами и давлениями не даны, нужно считать} \\
    T_2 &= 4T_1 = 4T_0 \text{(по условию)} \implies \frac{P_2}{P_1} = \frac{P_2V_0}{P_1V_0} = \frac{P_2 V_2}{P_1 V_1}= \frac{\nu R T_2}{\nu R T_1} = \frac{T_2}{T_1} = 4 \implies P_2 = P_3 = 4 P_1 = 4 P_0, \\
    T_3 &= 4T_2 = 16T_0 \text{(по условию)} \implies \frac{V_3}{V_2} = \frac{P_3V_3}{P_2V_2}= \frac{\nu R T_3}{\nu R T_2} = \frac{T_3}{T_2} = 4 \implies V_3 = V_4 = 4 V_2 = 4 V_0.
    \\
    A_\text{цикл} &= (4P_0 - P_0)(4V_0 - V_0) = 9P_0V_0, \\
    A_{23} &= 4P_0 \cdot (4V_0 - V_0) = 12P_0V_0, \\
    \Delta U_{23} &= \frac 32 \nu R T_3 - \frac 32 \nu R T_3 = \frac 32 P_3 V_3 - \frac 32 P_2 V_2 = \frac 32 \cdot 4 P_0 \cdot 4 V_0 -  \frac 32 \cdot 4 P_0 \cdot V_0 = \frac 32 \cdot 12 \cdot P_0V_0, \\
    \Delta U_{12} &= \frac 32 \nu R T_2 - \frac 32 \nu R T_1 = \frac 32 P_2 V_2 - \frac 32 P_1 V_1 = \frac 32 \cdot 4 P_0 V_0 - \frac 32 P_0 V_0 = \frac 32 \cdot 3 \cdot P_0V_0.
    \\
    \eta &= \frac{A_\text{цикл}}{Q_+} = \frac{A_\text{цикл}}{Q_{12} + Q_{23}}  = \frac{A_\text{цикл}}{A_{12} + \Delta U_{12} + A_{23} + \Delta U_{23}} =  \\
     &= \frac{9P_0V_0}{0 + \frac 32 \cdot 3 \cdot P_0V_0 + 12P_0V_0 + \frac 32 \cdot 12 \cdot P_0V_0} = \frac{9}{\frac 32 \cdot 3 + 12 + \frac 32 \cdot 12} = \frac6{23} \approx 0{,}261.
     \\
    \eta_\text{Карно} &= 1 - \frac{T_\text{х}}{T_\text{н}} = 1 - \frac{T_\text{1}}{T_\text{3}} = 1 - \frac{T_0}{16T_0} = 1 - \frac 1{16}  = \frac{15}{16} \approx 0{,}938.
    \end{align*}
}
\solutionspace{360pt}

\tasknumber{2}%
\task{%
    Изобразите в координатах $PV$/$VT$/$PT$ графики изохорического нагрева в 4 раза (все 3 графика).
    Не забудьте указать оси и масштаб, начальную и конечную точки, направление движения на графике.
}
\solutionspace{100pt}

\tasknumber{3}%
\task{%
    Укажите, верны ли утверждения («да» или «нет» слева от каждого утверждения):
    \begin{enumerate}
        \item При адиабатическом расширении идеальный газ совершает ровно столько работы, сколько внутренней энергии теряет.
        % \item В силу третьего закона Ньютона, совершённая газом работа и работа, совершённая над ним, всегда равны по модулю и противоположны по знаку.
        \item Работу газа в некотором процессе можно вычислять как площадь под графиком в системе координат $PV$, главное лишь правильно расположить оси.
        % \item Дважды два три.
        \item При изотермическом процессе внутренняя энергия идеального одноатомного газа не изменяется, даже если ему подводят тепло.
        \item Газ может совершить ненулевую работу в изобарном процессе.
        % \item Адиабатический процесс лишь по воле случая не имеет приставки «изо»: в нём изменяются давление, температура и объём, но это не все макропараметры идеального газа.
        \item Полученное выражение для внутренней энергии идеального газа ($\frac 32 \nu RT$) применимо к трёхатомному газу, при этом, например, уравнение состояния идеального газа применимо независимо от числа атомов в молекулах газа.
    \end{enumerate}
}
\answer{%
    $\text{да, да, да, да, нет}$
}

\variantsplitter

\addpersonalvariant{Андрей Рожков}

\tasknumber{1}%
\task{%
    Определите КПД цикла 12341, рабочим телом которого является идеальный одноатомный газ, если
    12 — изохорический нагрев в шесть раз,
    23 — изобарическое расширение, при котором температура растёт в пять раз,
    34 — изохора, 41 — изобара.

    Определите КПД цикла Карно, температура нагревателя которого равна максимальной температуре в цикле 12341, а холодильника — минимальной.
    Ответы в обоих случаях оставьте точными в виде нескоратимой дроби, никаких округлений.
}
\answer{%
    \begin{align*}
    A_{12} &= 0, \Delta U_{12} > 0, \implies Q_{12} = A_{12} + \Delta U_{12} > 0.
    \\
    A_{23} &> 0, \Delta U_{23} > 0, \implies Q_{23} = A_{23} + \Delta U_{23} > 0, \\
    A_{34} &= 0, \Delta U_{34} < 0, \implies Q_{34} = A_{34} + \Delta U_{34} < 0, \\
    A_{41} &< 0, \Delta U_{41} < 0, \implies Q_{41} = A_{41} + \Delta U_{41} < 0.
    \\
    P_1V_1 &= \nu R T_1, P_2V_2 = \nu R T_2, P_3V_3 = \nu R T_3, P_4V_4 = \nu R T_4 \text{ — уравнения состояния идеального газа}, \\
    &\text{Пусть $P_0$, $V_0$, $T_0$ — давление, объём и температура в точке 1 (минимальные во всём цикле):} \\
    P_1 &= P_4 = P_0, P_2 = P_3, V_1 = V_2 = V_0, V_3 = V_4, \text{остальные соотношения между объёмами и давлениями не даны, нужно считать} \\
    T_2 &= 6T_1 = 6T_0 \text{(по условию)} \implies \frac{P_2}{P_1} = \frac{P_2V_0}{P_1V_0} = \frac{P_2 V_2}{P_1 V_1}= \frac{\nu R T_2}{\nu R T_1} = \frac{T_2}{T_1} = 6 \implies P_2 = P_3 = 6 P_1 = 6 P_0, \\
    T_3 &= 5T_2 = 30T_0 \text{(по условию)} \implies \frac{V_3}{V_2} = \frac{P_3V_3}{P_2V_2}= \frac{\nu R T_3}{\nu R T_2} = \frac{T_3}{T_2} = 5 \implies V_3 = V_4 = 5 V_2 = 5 V_0.
    \\
    A_\text{цикл} &= (5P_0 - P_0)(6V_0 - V_0) = 20P_0V_0, \\
    A_{23} &= 6P_0 \cdot (5V_0 - V_0) = 24P_0V_0, \\
    \Delta U_{23} &= \frac 32 \nu R T_3 - \frac 32 \nu R T_3 = \frac 32 P_3 V_3 - \frac 32 P_2 V_2 = \frac 32 \cdot 6 P_0 \cdot 5 V_0 -  \frac 32 \cdot 6 P_0 \cdot V_0 = \frac 32 \cdot 24 \cdot P_0V_0, \\
    \Delta U_{12} &= \frac 32 \nu R T_2 - \frac 32 \nu R T_1 = \frac 32 P_2 V_2 - \frac 32 P_1 V_1 = \frac 32 \cdot 6 P_0 V_0 - \frac 32 P_0 V_0 = \frac 32 \cdot 5 \cdot P_0V_0.
    \\
    \eta &= \frac{A_\text{цикл}}{Q_+} = \frac{A_\text{цикл}}{Q_{12} + Q_{23}}  = \frac{A_\text{цикл}}{A_{12} + \Delta U_{12} + A_{23} + \Delta U_{23}} =  \\
     &= \frac{20P_0V_0}{0 + \frac 32 \cdot 5 \cdot P_0V_0 + 24P_0V_0 + \frac 32 \cdot 24 \cdot P_0V_0} = \frac{20}{\frac 32 \cdot 5 + 24 + \frac 32 \cdot 24} = \frac8{27} \approx 0{,}296.
     \\
    \eta_\text{Карно} &= 1 - \frac{T_\text{х}}{T_\text{н}} = 1 - \frac{T_\text{1}}{T_\text{3}} = 1 - \frac{T_0}{30T_0} = 1 - \frac 1{30}  = \frac{29}{30} \approx 0{,}967.
    \end{align*}
}
\solutionspace{360pt}

\tasknumber{2}%
\task{%
    Изобразите в координатах $PV$/$VT$/$PT$ графики изобарического расширения в 2 раза (все 3 графика).
    Не забудьте указать оси и масштаб, начальную и конечную точки, направление движения на графике.
}
\solutionspace{100pt}

\tasknumber{3}%
\task{%
    Укажите, верны ли утверждения («да» или «нет» слева от каждого утверждения):
    \begin{enumerate}
        \item При адиабатическом расширении идеальный газ совершает ровно столько работы, сколько внутренней энергии теряет.
        % \item В силу третьего закона Ньютона, совершённая газом работа и работа, совершённая над ним, всегда равны по модулю и противоположны по знаку.
        \item Работу газа в некотором процессе можно вычислять как площадь под графиком в системе координат $VT$, главное лишь правильно расположить оси.
        % \item Дважды два четыре.
        \item При изохорном процессе внутренняя энергия идеального одноатомного газа не изменяется, даже если ему подводят тепло.
        \item Газ может совершить ненулевую работу в изохорном процессе.
        % \item Адиабатический процесс лишь по воле случая не имеет приставки «изо»: в нём изменяются давление, температура и объём, но это не все макропараметры идеального газа.
        \item Полученное выражение для внутренней энергии идеального газа ($\frac 32 \nu RT$) применимо к двухоатомному газу, при этом, например, уравнение состояния идеального газа применимо независимо от числа атомов в молекулах газа.
    \end{enumerate}
}
\answer{%
    $\text{да, нет, нет, нет, нет}$
}

\variantsplitter

\addpersonalvariant{Рената Таржиманова}

\tasknumber{1}%
\task{%
    Определите КПД цикла 12341, рабочим телом которого является идеальный одноатомный газ, если
    12 — изохорический нагрев в шесть раз,
    23 — изобарическое расширение, при котором температура растёт в три раза,
    34 — изохора, 41 — изобара.

    Определите КПД цикла Карно, температура нагревателя которого равна максимальной температуре в цикле 12341, а холодильника — минимальной.
    Ответы в обоих случаях оставьте точными в виде нескоратимой дроби, никаких округлений.
}
\answer{%
    \begin{align*}
    A_{12} &= 0, \Delta U_{12} > 0, \implies Q_{12} = A_{12} + \Delta U_{12} > 0.
    \\
    A_{23} &> 0, \Delta U_{23} > 0, \implies Q_{23} = A_{23} + \Delta U_{23} > 0, \\
    A_{34} &= 0, \Delta U_{34} < 0, \implies Q_{34} = A_{34} + \Delta U_{34} < 0, \\
    A_{41} &< 0, \Delta U_{41} < 0, \implies Q_{41} = A_{41} + \Delta U_{41} < 0.
    \\
    P_1V_1 &= \nu R T_1, P_2V_2 = \nu R T_2, P_3V_3 = \nu R T_3, P_4V_4 = \nu R T_4 \text{ — уравнения состояния идеального газа}, \\
    &\text{Пусть $P_0$, $V_0$, $T_0$ — давление, объём и температура в точке 1 (минимальные во всём цикле):} \\
    P_1 &= P_4 = P_0, P_2 = P_3, V_1 = V_2 = V_0, V_3 = V_4, \text{остальные соотношения между объёмами и давлениями не даны, нужно считать} \\
    T_2 &= 6T_1 = 6T_0 \text{(по условию)} \implies \frac{P_2}{P_1} = \frac{P_2V_0}{P_1V_0} = \frac{P_2 V_2}{P_1 V_1}= \frac{\nu R T_2}{\nu R T_1} = \frac{T_2}{T_1} = 6 \implies P_2 = P_3 = 6 P_1 = 6 P_0, \\
    T_3 &= 3T_2 = 18T_0 \text{(по условию)} \implies \frac{V_3}{V_2} = \frac{P_3V_3}{P_2V_2}= \frac{\nu R T_3}{\nu R T_2} = \frac{T_3}{T_2} = 3 \implies V_3 = V_4 = 3 V_2 = 3 V_0.
    \\
    A_\text{цикл} &= (3P_0 - P_0)(6V_0 - V_0) = 10P_0V_0, \\
    A_{23} &= 6P_0 \cdot (3V_0 - V_0) = 12P_0V_0, \\
    \Delta U_{23} &= \frac 32 \nu R T_3 - \frac 32 \nu R T_3 = \frac 32 P_3 V_3 - \frac 32 P_2 V_2 = \frac 32 \cdot 6 P_0 \cdot 3 V_0 -  \frac 32 \cdot 6 P_0 \cdot V_0 = \frac 32 \cdot 12 \cdot P_0V_0, \\
    \Delta U_{12} &= \frac 32 \nu R T_2 - \frac 32 \nu R T_1 = \frac 32 P_2 V_2 - \frac 32 P_1 V_1 = \frac 32 \cdot 6 P_0 V_0 - \frac 32 P_0 V_0 = \frac 32 \cdot 5 \cdot P_0V_0.
    \\
    \eta &= \frac{A_\text{цикл}}{Q_+} = \frac{A_\text{цикл}}{Q_{12} + Q_{23}}  = \frac{A_\text{цикл}}{A_{12} + \Delta U_{12} + A_{23} + \Delta U_{23}} =  \\
     &= \frac{10P_0V_0}{0 + \frac 32 \cdot 5 \cdot P_0V_0 + 12P_0V_0 + \frac 32 \cdot 12 \cdot P_0V_0} = \frac{10}{\frac 32 \cdot 5 + 12 + \frac 32 \cdot 12} = \frac4{15} \approx 0{,}267.
     \\
    \eta_\text{Карно} &= 1 - \frac{T_\text{х}}{T_\text{н}} = 1 - \frac{T_\text{1}}{T_\text{3}} = 1 - \frac{T_0}{18T_0} = 1 - \frac 1{18}  = \frac{17}{18} \approx 0{,}944.
    \end{align*}
}
\solutionspace{360pt}

\tasknumber{2}%
\task{%
    Изобразите в координатах $PV$/$VT$/$PT$ графики изотермического повышения давления в 4 раза (все 3 графика).
    Не забудьте указать оси и масштаб, начальную и конечную точки, направление движения на графике.
}
\solutionspace{100pt}

\tasknumber{3}%
\task{%
    Укажите, верны ли утверждения («да» или «нет» слева от каждого утверждения):
    \begin{enumerate}
        \item При адиабатическом расширении идеальный газ совершает ровно столько работы, сколько внутренней энергии теряет.
        % \item В силу третьего закона Ньютона, совершённая газом работа и работа, совершённая над ним, всегда равны по модулю и противоположны по знаку.
        \item Работу газа в некотором процессе можно вычислять как площадь под графиком в системе координат $PT$, главное лишь правильно расположить оси.
        % \item Дважды два пять.
        \item При изотермическом процессе внутренняя энергия идеального одноатомного газа не изменяется, даже если ему подводят тепло.
        \item Газ может совершить ненулевую работу в изохорном процессе.
        % \item Адиабатический процесс лишь по воле случая не имеет приставки «изо»: в нём изменяются давление, температура и объём, но это не все макропараметры идеального газа.
        \item Полученное выражение для внутренней энергии идеального газа ($\frac 32 \nu RT$) применимо к трёхатомному газу, при этом, например, уравнение состояния идеального газа применимо независимо от числа атомов в молекулах газа.
    \end{enumerate}
}
\answer{%
    $\text{да, нет, да, нет, нет}$
}

\variantsplitter

\addpersonalvariant{Андрей Щербаков}

\tasknumber{1}%
\task{%
    Определите КПД цикла 12341, рабочим телом которого является идеальный одноатомный газ, если
    12 — изохорический нагрев в три раза,
    23 — изобарическое расширение, при котором температура растёт в два раза,
    34 — изохора, 41 — изобара.

    Определите КПД цикла Карно, температура нагревателя которого равна максимальной температуре в цикле 12341, а холодильника — минимальной.
    Ответы в обоих случаях оставьте точными в виде нескоратимой дроби, никаких округлений.
}
\answer{%
    \begin{align*}
    A_{12} &= 0, \Delta U_{12} > 0, \implies Q_{12} = A_{12} + \Delta U_{12} > 0.
    \\
    A_{23} &> 0, \Delta U_{23} > 0, \implies Q_{23} = A_{23} + \Delta U_{23} > 0, \\
    A_{34} &= 0, \Delta U_{34} < 0, \implies Q_{34} = A_{34} + \Delta U_{34} < 0, \\
    A_{41} &< 0, \Delta U_{41} < 0, \implies Q_{41} = A_{41} + \Delta U_{41} < 0.
    \\
    P_1V_1 &= \nu R T_1, P_2V_2 = \nu R T_2, P_3V_3 = \nu R T_3, P_4V_4 = \nu R T_4 \text{ — уравнения состояния идеального газа}, \\
    &\text{Пусть $P_0$, $V_0$, $T_0$ — давление, объём и температура в точке 1 (минимальные во всём цикле):} \\
    P_1 &= P_4 = P_0, P_2 = P_3, V_1 = V_2 = V_0, V_3 = V_4, \text{остальные соотношения между объёмами и давлениями не даны, нужно считать} \\
    T_2 &= 3T_1 = 3T_0 \text{(по условию)} \implies \frac{P_2}{P_1} = \frac{P_2V_0}{P_1V_0} = \frac{P_2 V_2}{P_1 V_1}= \frac{\nu R T_2}{\nu R T_1} = \frac{T_2}{T_1} = 3 \implies P_2 = P_3 = 3 P_1 = 3 P_0, \\
    T_3 &= 2T_2 = 6T_0 \text{(по условию)} \implies \frac{V_3}{V_2} = \frac{P_3V_3}{P_2V_2}= \frac{\nu R T_3}{\nu R T_2} = \frac{T_3}{T_2} = 2 \implies V_3 = V_4 = 2 V_2 = 2 V_0.
    \\
    A_\text{цикл} &= (2P_0 - P_0)(3V_0 - V_0) = 2P_0V_0, \\
    A_{23} &= 3P_0 \cdot (2V_0 - V_0) = 3P_0V_0, \\
    \Delta U_{23} &= \frac 32 \nu R T_3 - \frac 32 \nu R T_3 = \frac 32 P_3 V_3 - \frac 32 P_2 V_2 = \frac 32 \cdot 3 P_0 \cdot 2 V_0 -  \frac 32 \cdot 3 P_0 \cdot V_0 = \frac 32 \cdot 3 \cdot P_0V_0, \\
    \Delta U_{12} &= \frac 32 \nu R T_2 - \frac 32 \nu R T_1 = \frac 32 P_2 V_2 - \frac 32 P_1 V_1 = \frac 32 \cdot 3 P_0 V_0 - \frac 32 P_0 V_0 = \frac 32 \cdot 2 \cdot P_0V_0.
    \\
    \eta &= \frac{A_\text{цикл}}{Q_+} = \frac{A_\text{цикл}}{Q_{12} + Q_{23}}  = \frac{A_\text{цикл}}{A_{12} + \Delta U_{12} + A_{23} + \Delta U_{23}} =  \\
     &= \frac{2P_0V_0}{0 + \frac 32 \cdot 2 \cdot P_0V_0 + 3P_0V_0 + \frac 32 \cdot 3 \cdot P_0V_0} = \frac{2}{\frac 32 \cdot 2 + 3 + \frac 32 \cdot 3} = \frac4{21} \approx 0{,}190.
     \\
    \eta_\text{Карно} &= 1 - \frac{T_\text{х}}{T_\text{н}} = 1 - \frac{T_\text{1}}{T_\text{3}} = 1 - \frac{T_0}{6T_0} = 1 - \frac 1{6}  = \frac56 \approx 0{,}833.
    \end{align*}
}
\solutionspace{360pt}

\tasknumber{2}%
\task{%
    Изобразите в координатах $PV$/$VT$/$PT$ графики изотермического повышения давления в 4 раза (все 3 графика).
    Не забудьте указать оси и масштаб, начальную и конечную точки, направление движения на графике.
}
\solutionspace{100pt}

\tasknumber{3}%
\task{%
    Укажите, верны ли утверждения («да» или «нет» слева от каждого утверждения):
    \begin{enumerate}
        \item При адиабатическом расширении идеальный газ совершает ровно столько работы, сколько внутренней энергии теряет.
        % \item В силу третьего закона Ньютона, совершённая газом работа и работа, совершённая над ним, всегда равны по модулю и противоположны по знаку.
        \item Работу газа в некотором процессе можно вычислять как площадь под графиком в системе координат $PV$, главное лишь правильно расположить оси.
        % \item Дважды два пять.
        \item При изохорном процессе внутренняя энергия идеального одноатомного газа не изменяется, даже если ему подводят тепло.
        \item Газ может совершить ненулевую работу в изотермическом процессе.
        % \item Адиабатический процесс лишь по воле случая не имеет приставки «изо»: в нём изменяются давление, температура и объём, но это не все макропараметры идеального газа.
        \item Полученное выражение для внутренней энергии идеального газа ($\frac 32 \nu RT$) применимо к трёхатомному газу, при этом, например, уравнение состояния идеального газа применимо независимо от числа атомов в молекулах газа.
    \end{enumerate}
}
\answer{%
    $\text{да, да, нет, да, нет}$
}

\variantsplitter

\addpersonalvariant{Михаил Ярошевский}

\tasknumber{1}%
\task{%
    Определите КПД цикла 12341, рабочим телом которого является идеальный одноатомный газ, если
    12 — изохорический нагрев в шесть раз,
    23 — изобарическое расширение, при котором температура растёт в три раза,
    34 — изохора, 41 — изобара.

    Определите КПД цикла Карно, температура нагревателя которого равна максимальной температуре в цикле 12341, а холодильника — минимальной.
    Ответы в обоих случаях оставьте точными в виде нескоратимой дроби, никаких округлений.
}
\answer{%
    \begin{align*}
    A_{12} &= 0, \Delta U_{12} > 0, \implies Q_{12} = A_{12} + \Delta U_{12} > 0.
    \\
    A_{23} &> 0, \Delta U_{23} > 0, \implies Q_{23} = A_{23} + \Delta U_{23} > 0, \\
    A_{34} &= 0, \Delta U_{34} < 0, \implies Q_{34} = A_{34} + \Delta U_{34} < 0, \\
    A_{41} &< 0, \Delta U_{41} < 0, \implies Q_{41} = A_{41} + \Delta U_{41} < 0.
    \\
    P_1V_1 &= \nu R T_1, P_2V_2 = \nu R T_2, P_3V_3 = \nu R T_3, P_4V_4 = \nu R T_4 \text{ — уравнения состояния идеального газа}, \\
    &\text{Пусть $P_0$, $V_0$, $T_0$ — давление, объём и температура в точке 1 (минимальные во всём цикле):} \\
    P_1 &= P_4 = P_0, P_2 = P_3, V_1 = V_2 = V_0, V_3 = V_4, \text{остальные соотношения между объёмами и давлениями не даны, нужно считать} \\
    T_2 &= 6T_1 = 6T_0 \text{(по условию)} \implies \frac{P_2}{P_1} = \frac{P_2V_0}{P_1V_0} = \frac{P_2 V_2}{P_1 V_1}= \frac{\nu R T_2}{\nu R T_1} = \frac{T_2}{T_1} = 6 \implies P_2 = P_3 = 6 P_1 = 6 P_0, \\
    T_3 &= 3T_2 = 18T_0 \text{(по условию)} \implies \frac{V_3}{V_2} = \frac{P_3V_3}{P_2V_2}= \frac{\nu R T_3}{\nu R T_2} = \frac{T_3}{T_2} = 3 \implies V_3 = V_4 = 3 V_2 = 3 V_0.
    \\
    A_\text{цикл} &= (3P_0 - P_0)(6V_0 - V_0) = 10P_0V_0, \\
    A_{23} &= 6P_0 \cdot (3V_0 - V_0) = 12P_0V_0, \\
    \Delta U_{23} &= \frac 32 \nu R T_3 - \frac 32 \nu R T_3 = \frac 32 P_3 V_3 - \frac 32 P_2 V_2 = \frac 32 \cdot 6 P_0 \cdot 3 V_0 -  \frac 32 \cdot 6 P_0 \cdot V_0 = \frac 32 \cdot 12 \cdot P_0V_0, \\
    \Delta U_{12} &= \frac 32 \nu R T_2 - \frac 32 \nu R T_1 = \frac 32 P_2 V_2 - \frac 32 P_1 V_1 = \frac 32 \cdot 6 P_0 V_0 - \frac 32 P_0 V_0 = \frac 32 \cdot 5 \cdot P_0V_0.
    \\
    \eta &= \frac{A_\text{цикл}}{Q_+} = \frac{A_\text{цикл}}{Q_{12} + Q_{23}}  = \frac{A_\text{цикл}}{A_{12} + \Delta U_{12} + A_{23} + \Delta U_{23}} =  \\
     &= \frac{10P_0V_0}{0 + \frac 32 \cdot 5 \cdot P_0V_0 + 12P_0V_0 + \frac 32 \cdot 12 \cdot P_0V_0} = \frac{10}{\frac 32 \cdot 5 + 12 + \frac 32 \cdot 12} = \frac4{15} \approx 0{,}267.
     \\
    \eta_\text{Карно} &= 1 - \frac{T_\text{х}}{T_\text{н}} = 1 - \frac{T_\text{1}}{T_\text{3}} = 1 - \frac{T_0}{18T_0} = 1 - \frac 1{18}  = \frac{17}{18} \approx 0{,}944.
    \end{align*}
}
\solutionspace{360pt}

\tasknumber{2}%
\task{%
    Изобразите в координатах $PV$/$VT$/$PT$ графики изотермического понижения давления в 2 раза (все 3 графика).
    Не забудьте указать оси и масштаб, начальную и конечную точки, направление движения на графике.
}
\solutionspace{100pt}

\tasknumber{3}%
\task{%
    Укажите, верны ли утверждения («да» или «нет» слева от каждого утверждения):
    \begin{enumerate}
        \item При изобарном расширении идеальный газ совершает ровно столько работы, сколько внутренней энергии теряет.
        % \item В силу третьего закона Ньютона, совершённая газом работа и работа, совершённая над ним, всегда равны по модулю и противоположны по знаку.
        \item Работу газа в некотором процессе можно вычислять как площадь под графиком в системе координат $VT$, главное лишь правильно расположить оси.
        % \item Дважды два три.
        \item При изобарном процессе внутренняя энергия идеального одноатомного газа не изменяется, даже если ему подводят тепло.
        \item Газ может совершить ненулевую работу в изотермическом процессе.
        % \item Адиабатический процесс лишь по воле случая не имеет приставки «изо»: в нём изменяются давление, температура и объём, но это не все макропараметры идеального газа.
        \item Полученное выражение для внутренней энергии идеального газа ($\frac 32 \nu RT$) применимо к двухоатомному газу, при этом, например, уравнение состояния идеального газа применимо независимо от числа атомов в молекулах газа.
    \end{enumerate}
}
\answer{%
    $\text{нет, нет, нет, да, нет}$
}

\variantsplitter

\addpersonalvariant{Алексей Алимпиев}

\tasknumber{1}%
\task{%
    Определите КПД цикла 12341, рабочим телом которого является идеальный одноатомный газ, если
    12 — изохорический нагрев в два раза,
    23 — изобарическое расширение, при котором температура растёт в три раза,
    34 — изохора, 41 — изобара.

    Определите КПД цикла Карно, температура нагревателя которого равна максимальной температуре в цикле 12341, а холодильника — минимальной.
    Ответы в обоих случаях оставьте точными в виде нескоратимой дроби, никаких округлений.
}
\answer{%
    \begin{align*}
    A_{12} &= 0, \Delta U_{12} > 0, \implies Q_{12} = A_{12} + \Delta U_{12} > 0.
    \\
    A_{23} &> 0, \Delta U_{23} > 0, \implies Q_{23} = A_{23} + \Delta U_{23} > 0, \\
    A_{34} &= 0, \Delta U_{34} < 0, \implies Q_{34} = A_{34} + \Delta U_{34} < 0, \\
    A_{41} &< 0, \Delta U_{41} < 0, \implies Q_{41} = A_{41} + \Delta U_{41} < 0.
    \\
    P_1V_1 &= \nu R T_1, P_2V_2 = \nu R T_2, P_3V_3 = \nu R T_3, P_4V_4 = \nu R T_4 \text{ — уравнения состояния идеального газа}, \\
    &\text{Пусть $P_0$, $V_0$, $T_0$ — давление, объём и температура в точке 1 (минимальные во всём цикле):} \\
    P_1 &= P_4 = P_0, P_2 = P_3, V_1 = V_2 = V_0, V_3 = V_4, \text{остальные соотношения между объёмами и давлениями не даны, нужно считать} \\
    T_2 &= 2T_1 = 2T_0 \text{(по условию)} \implies \frac{P_2}{P_1} = \frac{P_2V_0}{P_1V_0} = \frac{P_2 V_2}{P_1 V_1}= \frac{\nu R T_2}{\nu R T_1} = \frac{T_2}{T_1} = 2 \implies P_2 = P_3 = 2 P_1 = 2 P_0, \\
    T_3 &= 3T_2 = 6T_0 \text{(по условию)} \implies \frac{V_3}{V_2} = \frac{P_3V_3}{P_2V_2}= \frac{\nu R T_3}{\nu R T_2} = \frac{T_3}{T_2} = 3 \implies V_3 = V_4 = 3 V_2 = 3 V_0.
    \\
    A_\text{цикл} &= (3P_0 - P_0)(2V_0 - V_0) = 2P_0V_0, \\
    A_{23} &= 2P_0 \cdot (3V_0 - V_0) = 4P_0V_0, \\
    \Delta U_{23} &= \frac 32 \nu R T_3 - \frac 32 \nu R T_3 = \frac 32 P_3 V_3 - \frac 32 P_2 V_2 = \frac 32 \cdot 2 P_0 \cdot 3 V_0 -  \frac 32 \cdot 2 P_0 \cdot V_0 = \frac 32 \cdot 4 \cdot P_0V_0, \\
    \Delta U_{12} &= \frac 32 \nu R T_2 - \frac 32 \nu R T_1 = \frac 32 P_2 V_2 - \frac 32 P_1 V_1 = \frac 32 \cdot 2 P_0 V_0 - \frac 32 P_0 V_0 = \frac 32 \cdot 1 \cdot P_0V_0.
    \\
    \eta &= \frac{A_\text{цикл}}{Q_+} = \frac{A_\text{цикл}}{Q_{12} + Q_{23}}  = \frac{A_\text{цикл}}{A_{12} + \Delta U_{12} + A_{23} + \Delta U_{23}} =  \\
     &= \frac{2P_0V_0}{0 + \frac 32 \cdot 1 \cdot P_0V_0 + 4P_0V_0 + \frac 32 \cdot 4 \cdot P_0V_0} = \frac{2}{\frac 32 \cdot 1 + 4 + \frac 32 \cdot 4} = \frac4{23} \approx 0{,}174.
     \\
    \eta_\text{Карно} &= 1 - \frac{T_\text{х}}{T_\text{н}} = 1 - \frac{T_\text{1}}{T_\text{3}} = 1 - \frac{T_0}{6T_0} = 1 - \frac 1{6}  = \frac56 \approx 0{,}833.
    \end{align*}
}
\solutionspace{360pt}

\tasknumber{2}%
\task{%
    Изобразите в координатах $PV$/$VT$/$PT$ графики изобарического расширения в 3 раза (все 3 графика).
    Не забудьте указать оси и масштаб, начальную и конечную точки, направление движения на графике.
}
\solutionspace{100pt}

\tasknumber{3}%
\task{%
    Укажите, верны ли утверждения («да» или «нет» слева от каждого утверждения):
    \begin{enumerate}
        \item При адиабатическом расширении идеальный газ совершает ровно столько работы, сколько внутренней энергии теряет.
        % \item В силу третьего закона Ньютона, совершённая газом работа и работа, совершённая над ним, всегда равны по модулю и противоположны по знаку.
        \item Работу газа в некотором процессе можно вычислять как площадь под графиком в системе координат $VT$, главное лишь правильно расположить оси.
        % \item Дважды два четыре.
        \item При изотермическом процессе внутренняя энергия идеального одноатомного газа не изменяется, даже если ему подводят тепло.
        \item Газ может совершить ненулевую работу в изотермическом процессе.
        % \item Адиабатический процесс лишь по воле случая не имеет приставки «изо»: в нём изменяются давление, температура и объём, но это не все макропараметры идеального газа.
        \item Полученное выражение для внутренней энергии идеального газа ($\frac 32 \nu RT$) применимо к двухоатомному газу, при этом, например, уравнение состояния идеального газа применимо независимо от числа атомов в молекулах газа.
    \end{enumerate}
}
\answer{%
    $\text{да, нет, да, да, нет}$
}

\variantsplitter

\addpersonalvariant{Евгений Васин}

\tasknumber{1}%
\task{%
    Определите КПД цикла 12341, рабочим телом которого является идеальный одноатомный газ, если
    12 — изохорический нагрев в пять раз,
    23 — изобарическое расширение, при котором температура растёт в четыре раза,
    34 — изохора, 41 — изобара.

    Определите КПД цикла Карно, температура нагревателя которого равна максимальной температуре в цикле 12341, а холодильника — минимальной.
    Ответы в обоих случаях оставьте точными в виде нескоратимой дроби, никаких округлений.
}
\answer{%
    \begin{align*}
    A_{12} &= 0, \Delta U_{12} > 0, \implies Q_{12} = A_{12} + \Delta U_{12} > 0.
    \\
    A_{23} &> 0, \Delta U_{23} > 0, \implies Q_{23} = A_{23} + \Delta U_{23} > 0, \\
    A_{34} &= 0, \Delta U_{34} < 0, \implies Q_{34} = A_{34} + \Delta U_{34} < 0, \\
    A_{41} &< 0, \Delta U_{41} < 0, \implies Q_{41} = A_{41} + \Delta U_{41} < 0.
    \\
    P_1V_1 &= \nu R T_1, P_2V_2 = \nu R T_2, P_3V_3 = \nu R T_3, P_4V_4 = \nu R T_4 \text{ — уравнения состояния идеального газа}, \\
    &\text{Пусть $P_0$, $V_0$, $T_0$ — давление, объём и температура в точке 1 (минимальные во всём цикле):} \\
    P_1 &= P_4 = P_0, P_2 = P_3, V_1 = V_2 = V_0, V_3 = V_4, \text{остальные соотношения между объёмами и давлениями не даны, нужно считать} \\
    T_2 &= 5T_1 = 5T_0 \text{(по условию)} \implies \frac{P_2}{P_1} = \frac{P_2V_0}{P_1V_0} = \frac{P_2 V_2}{P_1 V_1}= \frac{\nu R T_2}{\nu R T_1} = \frac{T_2}{T_1} = 5 \implies P_2 = P_3 = 5 P_1 = 5 P_0, \\
    T_3 &= 4T_2 = 20T_0 \text{(по условию)} \implies \frac{V_3}{V_2} = \frac{P_3V_3}{P_2V_2}= \frac{\nu R T_3}{\nu R T_2} = \frac{T_3}{T_2} = 4 \implies V_3 = V_4 = 4 V_2 = 4 V_0.
    \\
    A_\text{цикл} &= (4P_0 - P_0)(5V_0 - V_0) = 12P_0V_0, \\
    A_{23} &= 5P_0 \cdot (4V_0 - V_0) = 15P_0V_0, \\
    \Delta U_{23} &= \frac 32 \nu R T_3 - \frac 32 \nu R T_3 = \frac 32 P_3 V_3 - \frac 32 P_2 V_2 = \frac 32 \cdot 5 P_0 \cdot 4 V_0 -  \frac 32 \cdot 5 P_0 \cdot V_0 = \frac 32 \cdot 15 \cdot P_0V_0, \\
    \Delta U_{12} &= \frac 32 \nu R T_2 - \frac 32 \nu R T_1 = \frac 32 P_2 V_2 - \frac 32 P_1 V_1 = \frac 32 \cdot 5 P_0 V_0 - \frac 32 P_0 V_0 = \frac 32 \cdot 4 \cdot P_0V_0.
    \\
    \eta &= \frac{A_\text{цикл}}{Q_+} = \frac{A_\text{цикл}}{Q_{12} + Q_{23}}  = \frac{A_\text{цикл}}{A_{12} + \Delta U_{12} + A_{23} + \Delta U_{23}} =  \\
     &= \frac{12P_0V_0}{0 + \frac 32 \cdot 4 \cdot P_0V_0 + 15P_0V_0 + \frac 32 \cdot 15 \cdot P_0V_0} = \frac{12}{\frac 32 \cdot 4 + 15 + \frac 32 \cdot 15} = \frac8{29} \approx 0{,}276.
     \\
    \eta_\text{Карно} &= 1 - \frac{T_\text{х}}{T_\text{н}} = 1 - \frac{T_\text{1}}{T_\text{3}} = 1 - \frac{T_0}{20T_0} = 1 - \frac 1{20}  = \frac{19}{20} \approx 0{,}950.
    \end{align*}
}
\solutionspace{360pt}

\tasknumber{2}%
\task{%
    Изобразите в координатах $PV$/$VT$/$PT$ графики изохорического охлаждения в 2 раза (все 3 графика).
    Не забудьте указать оси и масштаб, начальную и конечную точки, направление движения на графике.
}
\solutionspace{100pt}

\tasknumber{3}%
\task{%
    Укажите, верны ли утверждения («да» или «нет» слева от каждого утверждения):
    \begin{enumerate}
        \item При изобарном расширении идеальный газ совершает ровно столько работы, сколько внутренней энергии теряет.
        % \item В силу третьего закона Ньютона, совершённая газом работа и работа, совершённая над ним, всегда равны по модулю и противоположны по знаку.
        \item Работу газа в некотором процессе можно вычислять как площадь под графиком в системе координат $PT$, главное лишь правильно расположить оси.
        % \item Дважды два четыре.
        \item При изотермическом процессе внутренняя энергия идеального одноатомного газа не изменяется, даже если ему подводят тепло.
        \item Газ может совершить ненулевую работу в изобарном процессе.
        % \item Адиабатический процесс лишь по воле случая не имеет приставки «изо»: в нём изменяются давление, температура и объём, но это не все макропараметры идеального газа.
        \item Полученное выражение для внутренней энергии идеального газа ($\frac 32 \nu RT$) применимо к одноатомному газу, при этом, например, уравнение состояния идеального газа применимо независимо от числа атомов в молекулах газа.
    \end{enumerate}
}
\answer{%
    $\text{нет, нет, да, да, да}$
}

\variantsplitter

\addpersonalvariant{Вячеслав Волохов}

\tasknumber{1}%
\task{%
    Определите КПД цикла 12341, рабочим телом которого является идеальный одноатомный газ, если
    12 — изохорический нагрев в четыре раза,
    23 — изобарическое расширение, при котором температура растёт в пять раз,
    34 — изохора, 41 — изобара.

    Определите КПД цикла Карно, температура нагревателя которого равна максимальной температуре в цикле 12341, а холодильника — минимальной.
    Ответы в обоих случаях оставьте точными в виде нескоратимой дроби, никаких округлений.
}
\answer{%
    \begin{align*}
    A_{12} &= 0, \Delta U_{12} > 0, \implies Q_{12} = A_{12} + \Delta U_{12} > 0.
    \\
    A_{23} &> 0, \Delta U_{23} > 0, \implies Q_{23} = A_{23} + \Delta U_{23} > 0, \\
    A_{34} &= 0, \Delta U_{34} < 0, \implies Q_{34} = A_{34} + \Delta U_{34} < 0, \\
    A_{41} &< 0, \Delta U_{41} < 0, \implies Q_{41} = A_{41} + \Delta U_{41} < 0.
    \\
    P_1V_1 &= \nu R T_1, P_2V_2 = \nu R T_2, P_3V_3 = \nu R T_3, P_4V_4 = \nu R T_4 \text{ — уравнения состояния идеального газа}, \\
    &\text{Пусть $P_0$, $V_0$, $T_0$ — давление, объём и температура в точке 1 (минимальные во всём цикле):} \\
    P_1 &= P_4 = P_0, P_2 = P_3, V_1 = V_2 = V_0, V_3 = V_4, \text{остальные соотношения между объёмами и давлениями не даны, нужно считать} \\
    T_2 &= 4T_1 = 4T_0 \text{(по условию)} \implies \frac{P_2}{P_1} = \frac{P_2V_0}{P_1V_0} = \frac{P_2 V_2}{P_1 V_1}= \frac{\nu R T_2}{\nu R T_1} = \frac{T_2}{T_1} = 4 \implies P_2 = P_3 = 4 P_1 = 4 P_0, \\
    T_3 &= 5T_2 = 20T_0 \text{(по условию)} \implies \frac{V_3}{V_2} = \frac{P_3V_3}{P_2V_2}= \frac{\nu R T_3}{\nu R T_2} = \frac{T_3}{T_2} = 5 \implies V_3 = V_4 = 5 V_2 = 5 V_0.
    \\
    A_\text{цикл} &= (5P_0 - P_0)(4V_0 - V_0) = 12P_0V_0, \\
    A_{23} &= 4P_0 \cdot (5V_0 - V_0) = 16P_0V_0, \\
    \Delta U_{23} &= \frac 32 \nu R T_3 - \frac 32 \nu R T_3 = \frac 32 P_3 V_3 - \frac 32 P_2 V_2 = \frac 32 \cdot 4 P_0 \cdot 5 V_0 -  \frac 32 \cdot 4 P_0 \cdot V_0 = \frac 32 \cdot 16 \cdot P_0V_0, \\
    \Delta U_{12} &= \frac 32 \nu R T_2 - \frac 32 \nu R T_1 = \frac 32 P_2 V_2 - \frac 32 P_1 V_1 = \frac 32 \cdot 4 P_0 V_0 - \frac 32 P_0 V_0 = \frac 32 \cdot 3 \cdot P_0V_0.
    \\
    \eta &= \frac{A_\text{цикл}}{Q_+} = \frac{A_\text{цикл}}{Q_{12} + Q_{23}}  = \frac{A_\text{цикл}}{A_{12} + \Delta U_{12} + A_{23} + \Delta U_{23}} =  \\
     &= \frac{12P_0V_0}{0 + \frac 32 \cdot 3 \cdot P_0V_0 + 16P_0V_0 + \frac 32 \cdot 16 \cdot P_0V_0} = \frac{12}{\frac 32 \cdot 3 + 16 + \frac 32 \cdot 16} = \frac{24}{89} \approx 0{,}270.
     \\
    \eta_\text{Карно} &= 1 - \frac{T_\text{х}}{T_\text{н}} = 1 - \frac{T_\text{1}}{T_\text{3}} = 1 - \frac{T_0}{20T_0} = 1 - \frac 1{20}  = \frac{19}{20} \approx 0{,}950.
    \end{align*}
}
\solutionspace{360pt}

\tasknumber{2}%
\task{%
    Изобразите в координатах $PV$/$VT$/$PT$ графики изобарического расширения в 2 раза (все 3 графика).
    Не забудьте указать оси и масштаб, начальную и конечную точки, направление движения на графике.
}
\solutionspace{100pt}

\tasknumber{3}%
\task{%
    Укажите, верны ли утверждения («да» или «нет» слева от каждого утверждения):
    \begin{enumerate}
        \item При адиабатическом расширении идеальный газ совершает ровно столько работы, сколько внутренней энергии теряет.
        % \item В силу третьего закона Ньютона, совершённая газом работа и работа, совершённая над ним, всегда равны по модулю и противоположны по знаку.
        \item Работу газа в некотором процессе можно вычислять как площадь под графиком в системе координат $PT$, главное лишь правильно расположить оси.
        % \item Дважды два пять.
        \item При изотермическом процессе внутренняя энергия идеального одноатомного газа не изменяется, даже если ему подводят тепло.
        \item Газ может совершить ненулевую работу в изотермическом процессе.
        % \item Адиабатический процесс лишь по воле случая не имеет приставки «изо»: в нём изменяются давление, температура и объём, но это не все макропараметры идеального газа.
        \item Полученное выражение для внутренней энергии идеального газа ($\frac 32 \nu RT$) применимо к трёхатомному газу, при этом, например, уравнение состояния идеального газа применимо независимо от числа атомов в молекулах газа.
    \end{enumerate}
}
\answer{%
    $\text{да, нет, да, да, нет}$
}

\variantsplitter

\addpersonalvariant{Герман Говоров}

\tasknumber{1}%
\task{%
    Определите КПД цикла 12341, рабочим телом которого является идеальный одноатомный газ, если
    12 — изохорический нагрев в два раза,
    23 — изобарическое расширение, при котором температура растёт в четыре раза,
    34 — изохора, 41 — изобара.

    Определите КПД цикла Карно, температура нагревателя которого равна максимальной температуре в цикле 12341, а холодильника — минимальной.
    Ответы в обоих случаях оставьте точными в виде нескоратимой дроби, никаких округлений.
}
\answer{%
    \begin{align*}
    A_{12} &= 0, \Delta U_{12} > 0, \implies Q_{12} = A_{12} + \Delta U_{12} > 0.
    \\
    A_{23} &> 0, \Delta U_{23} > 0, \implies Q_{23} = A_{23} + \Delta U_{23} > 0, \\
    A_{34} &= 0, \Delta U_{34} < 0, \implies Q_{34} = A_{34} + \Delta U_{34} < 0, \\
    A_{41} &< 0, \Delta U_{41} < 0, \implies Q_{41} = A_{41} + \Delta U_{41} < 0.
    \\
    P_1V_1 &= \nu R T_1, P_2V_2 = \nu R T_2, P_3V_3 = \nu R T_3, P_4V_4 = \nu R T_4 \text{ — уравнения состояния идеального газа}, \\
    &\text{Пусть $P_0$, $V_0$, $T_0$ — давление, объём и температура в точке 1 (минимальные во всём цикле):} \\
    P_1 &= P_4 = P_0, P_2 = P_3, V_1 = V_2 = V_0, V_3 = V_4, \text{остальные соотношения между объёмами и давлениями не даны, нужно считать} \\
    T_2 &= 2T_1 = 2T_0 \text{(по условию)} \implies \frac{P_2}{P_1} = \frac{P_2V_0}{P_1V_0} = \frac{P_2 V_2}{P_1 V_1}= \frac{\nu R T_2}{\nu R T_1} = \frac{T_2}{T_1} = 2 \implies P_2 = P_3 = 2 P_1 = 2 P_0, \\
    T_3 &= 4T_2 = 8T_0 \text{(по условию)} \implies \frac{V_3}{V_2} = \frac{P_3V_3}{P_2V_2}= \frac{\nu R T_3}{\nu R T_2} = \frac{T_3}{T_2} = 4 \implies V_3 = V_4 = 4 V_2 = 4 V_0.
    \\
    A_\text{цикл} &= (4P_0 - P_0)(2V_0 - V_0) = 3P_0V_0, \\
    A_{23} &= 2P_0 \cdot (4V_0 - V_0) = 6P_0V_0, \\
    \Delta U_{23} &= \frac 32 \nu R T_3 - \frac 32 \nu R T_3 = \frac 32 P_3 V_3 - \frac 32 P_2 V_2 = \frac 32 \cdot 2 P_0 \cdot 4 V_0 -  \frac 32 \cdot 2 P_0 \cdot V_0 = \frac 32 \cdot 6 \cdot P_0V_0, \\
    \Delta U_{12} &= \frac 32 \nu R T_2 - \frac 32 \nu R T_1 = \frac 32 P_2 V_2 - \frac 32 P_1 V_1 = \frac 32 \cdot 2 P_0 V_0 - \frac 32 P_0 V_0 = \frac 32 \cdot 1 \cdot P_0V_0.
    \\
    \eta &= \frac{A_\text{цикл}}{Q_+} = \frac{A_\text{цикл}}{Q_{12} + Q_{23}}  = \frac{A_\text{цикл}}{A_{12} + \Delta U_{12} + A_{23} + \Delta U_{23}} =  \\
     &= \frac{3P_0V_0}{0 + \frac 32 \cdot 1 \cdot P_0V_0 + 6P_0V_0 + \frac 32 \cdot 6 \cdot P_0V_0} = \frac{3}{\frac 32 \cdot 1 + 6 + \frac 32 \cdot 6} = \frac2{11} \approx 0{,}182.
     \\
    \eta_\text{Карно} &= 1 - \frac{T_\text{х}}{T_\text{н}} = 1 - \frac{T_\text{1}}{T_\text{3}} = 1 - \frac{T_0}{8T_0} = 1 - \frac 1{8}  = \frac78 \approx 0{,}875.
    \end{align*}
}
\solutionspace{360pt}

\tasknumber{2}%
\task{%
    Изобразите в координатах $PV$/$VT$/$PT$ графики изотермического понижения давления в 2 раза (все 3 графика).
    Не забудьте указать оси и масштаб, начальную и конечную точки, направление движения на графике.
}
\solutionspace{100pt}

\tasknumber{3}%
\task{%
    Укажите, верны ли утверждения («да» или «нет» слева от каждого утверждения):
    \begin{enumerate}
        \item При изобарном расширении идеальный газ совершает ровно столько работы, сколько внутренней энергии теряет.
        % \item В силу третьего закона Ньютона, совершённая газом работа и работа, совершённая над ним, всегда равны по модулю и противоположны по знаку.
        \item Работу газа в некотором процессе можно вычислять как площадь под графиком в системе координат $VT$, главное лишь правильно расположить оси.
        % \item Дважды два три.
        \item При изохорном процессе внутренняя энергия идеального одноатомного газа не изменяется, даже если ему подводят тепло.
        \item Газ может совершить ненулевую работу в изобарном процессе.
        % \item Адиабатический процесс лишь по воле случая не имеет приставки «изо»: в нём изменяются давление, температура и объём, но это не все макропараметры идеального газа.
        \item Полученное выражение для внутренней энергии идеального газа ($\frac 32 \nu RT$) применимо к двухоатомному газу, при этом, например, уравнение состояния идеального газа применимо независимо от числа атомов в молекулах газа.
    \end{enumerate}
}
\answer{%
    $\text{нет, нет, нет, да, нет}$
}

\variantsplitter

\addpersonalvariant{София Журавлёва}

\tasknumber{1}%
\task{%
    Определите КПД цикла 12341, рабочим телом которого является идеальный одноатомный газ, если
    12 — изохорический нагрев в шесть раз,
    23 — изобарическое расширение, при котором температура растёт в шесть раз,
    34 — изохора, 41 — изобара.

    Определите КПД цикла Карно, температура нагревателя которого равна максимальной температуре в цикле 12341, а холодильника — минимальной.
    Ответы в обоих случаях оставьте точными в виде нескоратимой дроби, никаких округлений.
}
\answer{%
    \begin{align*}
    A_{12} &= 0, \Delta U_{12} > 0, \implies Q_{12} = A_{12} + \Delta U_{12} > 0.
    \\
    A_{23} &> 0, \Delta U_{23} > 0, \implies Q_{23} = A_{23} + \Delta U_{23} > 0, \\
    A_{34} &= 0, \Delta U_{34} < 0, \implies Q_{34} = A_{34} + \Delta U_{34} < 0, \\
    A_{41} &< 0, \Delta U_{41} < 0, \implies Q_{41} = A_{41} + \Delta U_{41} < 0.
    \\
    P_1V_1 &= \nu R T_1, P_2V_2 = \nu R T_2, P_3V_3 = \nu R T_3, P_4V_4 = \nu R T_4 \text{ — уравнения состояния идеального газа}, \\
    &\text{Пусть $P_0$, $V_0$, $T_0$ — давление, объём и температура в точке 1 (минимальные во всём цикле):} \\
    P_1 &= P_4 = P_0, P_2 = P_3, V_1 = V_2 = V_0, V_3 = V_4, \text{остальные соотношения между объёмами и давлениями не даны, нужно считать} \\
    T_2 &= 6T_1 = 6T_0 \text{(по условию)} \implies \frac{P_2}{P_1} = \frac{P_2V_0}{P_1V_0} = \frac{P_2 V_2}{P_1 V_1}= \frac{\nu R T_2}{\nu R T_1} = \frac{T_2}{T_1} = 6 \implies P_2 = P_3 = 6 P_1 = 6 P_0, \\
    T_3 &= 6T_2 = 36T_0 \text{(по условию)} \implies \frac{V_3}{V_2} = \frac{P_3V_3}{P_2V_2}= \frac{\nu R T_3}{\nu R T_2} = \frac{T_3}{T_2} = 6 \implies V_3 = V_4 = 6 V_2 = 6 V_0.
    \\
    A_\text{цикл} &= (6P_0 - P_0)(6V_0 - V_0) = 25P_0V_0, \\
    A_{23} &= 6P_0 \cdot (6V_0 - V_0) = 30P_0V_0, \\
    \Delta U_{23} &= \frac 32 \nu R T_3 - \frac 32 \nu R T_3 = \frac 32 P_3 V_3 - \frac 32 P_2 V_2 = \frac 32 \cdot 6 P_0 \cdot 6 V_0 -  \frac 32 \cdot 6 P_0 \cdot V_0 = \frac 32 \cdot 30 \cdot P_0V_0, \\
    \Delta U_{12} &= \frac 32 \nu R T_2 - \frac 32 \nu R T_1 = \frac 32 P_2 V_2 - \frac 32 P_1 V_1 = \frac 32 \cdot 6 P_0 V_0 - \frac 32 P_0 V_0 = \frac 32 \cdot 5 \cdot P_0V_0.
    \\
    \eta &= \frac{A_\text{цикл}}{Q_+} = \frac{A_\text{цикл}}{Q_{12} + Q_{23}}  = \frac{A_\text{цикл}}{A_{12} + \Delta U_{12} + A_{23} + \Delta U_{23}} =  \\
     &= \frac{25P_0V_0}{0 + \frac 32 \cdot 5 \cdot P_0V_0 + 30P_0V_0 + \frac 32 \cdot 30 \cdot P_0V_0} = \frac{25}{\frac 32 \cdot 5 + 30 + \frac 32 \cdot 30} = \frac{10}{33} \approx 0{,}303.
     \\
    \eta_\text{Карно} &= 1 - \frac{T_\text{х}}{T_\text{н}} = 1 - \frac{T_\text{1}}{T_\text{3}} = 1 - \frac{T_0}{36T_0} = 1 - \frac 1{36}  = \frac{35}{36} \approx 0{,}972.
    \end{align*}
}
\solutionspace{360pt}

\tasknumber{2}%
\task{%
    Изобразите в координатах $PV$/$VT$/$PT$ графики изохорического охлаждения в 3 раза (все 3 графика).
    Не забудьте указать оси и масштаб, начальную и конечную точки, направление движения на графике.
}
\solutionspace{100pt}

\tasknumber{3}%
\task{%
    Укажите, верны ли утверждения («да» или «нет» слева от каждого утверждения):
    \begin{enumerate}
        \item При адиабатическом расширении идеальный газ совершает ровно столько работы, сколько внутренней энергии теряет.
        % \item В силу третьего закона Ньютона, совершённая газом работа и работа, совершённая над ним, всегда равны по модулю и противоположны по знаку.
        \item Работу газа в некотором процессе можно вычислять как площадь под графиком в системе координат $VT$, главное лишь правильно расположить оси.
        % \item Дважды два пять.
        \item При изобарном процессе внутренняя энергия идеального одноатомного газа не изменяется, даже если ему подводят тепло.
        \item Газ может совершить ненулевую работу в изотермическом процессе.
        % \item Адиабатический процесс лишь по воле случая не имеет приставки «изо»: в нём изменяются давление, температура и объём, но это не все макропараметры идеального газа.
        \item Полученное выражение для внутренней энергии идеального газа ($\frac 32 \nu RT$) применимо к трёхатомному газу, при этом, например, уравнение состояния идеального газа применимо независимо от числа атомов в молекулах газа.
    \end{enumerate}
}
\answer{%
    $\text{да, нет, нет, да, нет}$
}

\variantsplitter

\addpersonalvariant{Константин Козлов}

\tasknumber{1}%
\task{%
    Определите КПД цикла 12341, рабочим телом которого является идеальный одноатомный газ, если
    12 — изохорический нагрев в шесть раз,
    23 — изобарическое расширение, при котором температура растёт в шесть раз,
    34 — изохора, 41 — изобара.

    Определите КПД цикла Карно, температура нагревателя которого равна максимальной температуре в цикле 12341, а холодильника — минимальной.
    Ответы в обоих случаях оставьте точными в виде нескоратимой дроби, никаких округлений.
}
\answer{%
    \begin{align*}
    A_{12} &= 0, \Delta U_{12} > 0, \implies Q_{12} = A_{12} + \Delta U_{12} > 0.
    \\
    A_{23} &> 0, \Delta U_{23} > 0, \implies Q_{23} = A_{23} + \Delta U_{23} > 0, \\
    A_{34} &= 0, \Delta U_{34} < 0, \implies Q_{34} = A_{34} + \Delta U_{34} < 0, \\
    A_{41} &< 0, \Delta U_{41} < 0, \implies Q_{41} = A_{41} + \Delta U_{41} < 0.
    \\
    P_1V_1 &= \nu R T_1, P_2V_2 = \nu R T_2, P_3V_3 = \nu R T_3, P_4V_4 = \nu R T_4 \text{ — уравнения состояния идеального газа}, \\
    &\text{Пусть $P_0$, $V_0$, $T_0$ — давление, объём и температура в точке 1 (минимальные во всём цикле):} \\
    P_1 &= P_4 = P_0, P_2 = P_3, V_1 = V_2 = V_0, V_3 = V_4, \text{остальные соотношения между объёмами и давлениями не даны, нужно считать} \\
    T_2 &= 6T_1 = 6T_0 \text{(по условию)} \implies \frac{P_2}{P_1} = \frac{P_2V_0}{P_1V_0} = \frac{P_2 V_2}{P_1 V_1}= \frac{\nu R T_2}{\nu R T_1} = \frac{T_2}{T_1} = 6 \implies P_2 = P_3 = 6 P_1 = 6 P_0, \\
    T_3 &= 6T_2 = 36T_0 \text{(по условию)} \implies \frac{V_3}{V_2} = \frac{P_3V_3}{P_2V_2}= \frac{\nu R T_3}{\nu R T_2} = \frac{T_3}{T_2} = 6 \implies V_3 = V_4 = 6 V_2 = 6 V_0.
    \\
    A_\text{цикл} &= (6P_0 - P_0)(6V_0 - V_0) = 25P_0V_0, \\
    A_{23} &= 6P_0 \cdot (6V_0 - V_0) = 30P_0V_0, \\
    \Delta U_{23} &= \frac 32 \nu R T_3 - \frac 32 \nu R T_3 = \frac 32 P_3 V_3 - \frac 32 P_2 V_2 = \frac 32 \cdot 6 P_0 \cdot 6 V_0 -  \frac 32 \cdot 6 P_0 \cdot V_0 = \frac 32 \cdot 30 \cdot P_0V_0, \\
    \Delta U_{12} &= \frac 32 \nu R T_2 - \frac 32 \nu R T_1 = \frac 32 P_2 V_2 - \frac 32 P_1 V_1 = \frac 32 \cdot 6 P_0 V_0 - \frac 32 P_0 V_0 = \frac 32 \cdot 5 \cdot P_0V_0.
    \\
    \eta &= \frac{A_\text{цикл}}{Q_+} = \frac{A_\text{цикл}}{Q_{12} + Q_{23}}  = \frac{A_\text{цикл}}{A_{12} + \Delta U_{12} + A_{23} + \Delta U_{23}} =  \\
     &= \frac{25P_0V_0}{0 + \frac 32 \cdot 5 \cdot P_0V_0 + 30P_0V_0 + \frac 32 \cdot 30 \cdot P_0V_0} = \frac{25}{\frac 32 \cdot 5 + 30 + \frac 32 \cdot 30} = \frac{10}{33} \approx 0{,}303.
     \\
    \eta_\text{Карно} &= 1 - \frac{T_\text{х}}{T_\text{н}} = 1 - \frac{T_\text{1}}{T_\text{3}} = 1 - \frac{T_0}{36T_0} = 1 - \frac 1{36}  = \frac{35}{36} \approx 0{,}972.
    \end{align*}
}
\solutionspace{360pt}

\tasknumber{2}%
\task{%
    Изобразите в координатах $PV$/$VT$/$PT$ графики изотермического понижения давления в 3 раза (все 3 графика).
    Не забудьте указать оси и масштаб, начальную и конечную точки, направление движения на графике.
}
\solutionspace{100pt}

\tasknumber{3}%
\task{%
    Укажите, верны ли утверждения («да» или «нет» слева от каждого утверждения):
    \begin{enumerate}
        \item При изобарном расширении идеальный газ совершает ровно столько работы, сколько внутренней энергии теряет.
        % \item В силу третьего закона Ньютона, совершённая газом работа и работа, совершённая над ним, всегда равны по модулю и противоположны по знаку.
        \item Работу газа в некотором процессе можно вычислять как площадь под графиком в системе координат $VT$, главное лишь правильно расположить оси.
        % \item Дважды два четыре.
        \item При изобарном процессе внутренняя энергия идеального одноатомного газа не изменяется, даже если ему подводят тепло.
        \item Газ может совершить ненулевую работу в изотермическом процессе.
        % \item Адиабатический процесс лишь по воле случая не имеет приставки «изо»: в нём изменяются давление, температура и объём, но это не все макропараметры идеального газа.
        \item Полученное выражение для внутренней энергии идеального газа ($\frac 32 \nu RT$) применимо к двухоатомному газу, при этом, например, уравнение состояния идеального газа применимо независимо от числа атомов в молекулах газа.
    \end{enumerate}
}
\answer{%
    $\text{нет, нет, нет, да, нет}$
}

\variantsplitter

\addpersonalvariant{Наталья Кравченко}

\tasknumber{1}%
\task{%
    Определите КПД цикла 12341, рабочим телом которого является идеальный одноатомный газ, если
    12 — изохорический нагрев в четыре раза,
    23 — изобарическое расширение, при котором температура растёт в два раза,
    34 — изохора, 41 — изобара.

    Определите КПД цикла Карно, температура нагревателя которого равна максимальной температуре в цикле 12341, а холодильника — минимальной.
    Ответы в обоих случаях оставьте точными в виде нескоратимой дроби, никаких округлений.
}
\answer{%
    \begin{align*}
    A_{12} &= 0, \Delta U_{12} > 0, \implies Q_{12} = A_{12} + \Delta U_{12} > 0.
    \\
    A_{23} &> 0, \Delta U_{23} > 0, \implies Q_{23} = A_{23} + \Delta U_{23} > 0, \\
    A_{34} &= 0, \Delta U_{34} < 0, \implies Q_{34} = A_{34} + \Delta U_{34} < 0, \\
    A_{41} &< 0, \Delta U_{41} < 0, \implies Q_{41} = A_{41} + \Delta U_{41} < 0.
    \\
    P_1V_1 &= \nu R T_1, P_2V_2 = \nu R T_2, P_3V_3 = \nu R T_3, P_4V_4 = \nu R T_4 \text{ — уравнения состояния идеального газа}, \\
    &\text{Пусть $P_0$, $V_0$, $T_0$ — давление, объём и температура в точке 1 (минимальные во всём цикле):} \\
    P_1 &= P_4 = P_0, P_2 = P_3, V_1 = V_2 = V_0, V_3 = V_4, \text{остальные соотношения между объёмами и давлениями не даны, нужно считать} \\
    T_2 &= 4T_1 = 4T_0 \text{(по условию)} \implies \frac{P_2}{P_1} = \frac{P_2V_0}{P_1V_0} = \frac{P_2 V_2}{P_1 V_1}= \frac{\nu R T_2}{\nu R T_1} = \frac{T_2}{T_1} = 4 \implies P_2 = P_3 = 4 P_1 = 4 P_0, \\
    T_3 &= 2T_2 = 8T_0 \text{(по условию)} \implies \frac{V_3}{V_2} = \frac{P_3V_3}{P_2V_2}= \frac{\nu R T_3}{\nu R T_2} = \frac{T_3}{T_2} = 2 \implies V_3 = V_4 = 2 V_2 = 2 V_0.
    \\
    A_\text{цикл} &= (2P_0 - P_0)(4V_0 - V_0) = 3P_0V_0, \\
    A_{23} &= 4P_0 \cdot (2V_0 - V_0) = 4P_0V_0, \\
    \Delta U_{23} &= \frac 32 \nu R T_3 - \frac 32 \nu R T_3 = \frac 32 P_3 V_3 - \frac 32 P_2 V_2 = \frac 32 \cdot 4 P_0 \cdot 2 V_0 -  \frac 32 \cdot 4 P_0 \cdot V_0 = \frac 32 \cdot 4 \cdot P_0V_0, \\
    \Delta U_{12} &= \frac 32 \nu R T_2 - \frac 32 \nu R T_1 = \frac 32 P_2 V_2 - \frac 32 P_1 V_1 = \frac 32 \cdot 4 P_0 V_0 - \frac 32 P_0 V_0 = \frac 32 \cdot 3 \cdot P_0V_0.
    \\
    \eta &= \frac{A_\text{цикл}}{Q_+} = \frac{A_\text{цикл}}{Q_{12} + Q_{23}}  = \frac{A_\text{цикл}}{A_{12} + \Delta U_{12} + A_{23} + \Delta U_{23}} =  \\
     &= \frac{3P_0V_0}{0 + \frac 32 \cdot 3 \cdot P_0V_0 + 4P_0V_0 + \frac 32 \cdot 4 \cdot P_0V_0} = \frac{3}{\frac 32 \cdot 3 + 4 + \frac 32 \cdot 4} = \frac6{29} \approx 0{,}207.
     \\
    \eta_\text{Карно} &= 1 - \frac{T_\text{х}}{T_\text{н}} = 1 - \frac{T_\text{1}}{T_\text{3}} = 1 - \frac{T_0}{8T_0} = 1 - \frac 1{8}  = \frac78 \approx 0{,}875.
    \end{align*}
}
\solutionspace{360pt}

\tasknumber{2}%
\task{%
    Изобразите в координатах $PV$/$VT$/$PT$ графики изохорического нагрева в 2 раза (все 3 графика).
    Не забудьте указать оси и масштаб, начальную и конечную точки, направление движения на графике.
}
\solutionspace{100pt}

\tasknumber{3}%
\task{%
    Укажите, верны ли утверждения («да» или «нет» слева от каждого утверждения):
    \begin{enumerate}
        \item При адиабатическом расширении идеальный газ совершает ровно столько работы, сколько внутренней энергии теряет.
        % \item В силу третьего закона Ньютона, совершённая газом работа и работа, совершённая над ним, всегда равны по модулю и противоположны по знаку.
        \item Работу газа в некотором процессе можно вычислять как площадь под графиком в системе координат $PV$, главное лишь правильно расположить оси.
        % \item Дважды два четыре.
        \item При изобарном процессе внутренняя энергия идеального одноатомного газа не изменяется, даже если ему подводят тепло.
        \item Газ может совершить ненулевую работу в изобарном процессе.
        % \item Адиабатический процесс лишь по воле случая не имеет приставки «изо»: в нём изменяются давление, температура и объём, но это не все макропараметры идеального газа.
        \item Полученное выражение для внутренней энергии идеального газа ($\frac 32 \nu RT$) применимо к трёхатомному газу, при этом, например, уравнение состояния идеального газа применимо независимо от числа атомов в молекулах газа.
    \end{enumerate}
}
\answer{%
    $\text{да, да, нет, да, нет}$
}

\variantsplitter

\addpersonalvariant{Матвей Кузьмин}

\tasknumber{1}%
\task{%
    Определите КПД цикла 12341, рабочим телом которого является идеальный одноатомный газ, если
    12 — изохорический нагрев в два раза,
    23 — изобарическое расширение, при котором температура растёт в два раза,
    34 — изохора, 41 — изобара.

    Определите КПД цикла Карно, температура нагревателя которого равна максимальной температуре в цикле 12341, а холодильника — минимальной.
    Ответы в обоих случаях оставьте точными в виде нескоратимой дроби, никаких округлений.
}
\answer{%
    \begin{align*}
    A_{12} &= 0, \Delta U_{12} > 0, \implies Q_{12} = A_{12} + \Delta U_{12} > 0.
    \\
    A_{23} &> 0, \Delta U_{23} > 0, \implies Q_{23} = A_{23} + \Delta U_{23} > 0, \\
    A_{34} &= 0, \Delta U_{34} < 0, \implies Q_{34} = A_{34} + \Delta U_{34} < 0, \\
    A_{41} &< 0, \Delta U_{41} < 0, \implies Q_{41} = A_{41} + \Delta U_{41} < 0.
    \\
    P_1V_1 &= \nu R T_1, P_2V_2 = \nu R T_2, P_3V_3 = \nu R T_3, P_4V_4 = \nu R T_4 \text{ — уравнения состояния идеального газа}, \\
    &\text{Пусть $P_0$, $V_0$, $T_0$ — давление, объём и температура в точке 1 (минимальные во всём цикле):} \\
    P_1 &= P_4 = P_0, P_2 = P_3, V_1 = V_2 = V_0, V_3 = V_4, \text{остальные соотношения между объёмами и давлениями не даны, нужно считать} \\
    T_2 &= 2T_1 = 2T_0 \text{(по условию)} \implies \frac{P_2}{P_1} = \frac{P_2V_0}{P_1V_0} = \frac{P_2 V_2}{P_1 V_1}= \frac{\nu R T_2}{\nu R T_1} = \frac{T_2}{T_1} = 2 \implies P_2 = P_3 = 2 P_1 = 2 P_0, \\
    T_3 &= 2T_2 = 4T_0 \text{(по условию)} \implies \frac{V_3}{V_2} = \frac{P_3V_3}{P_2V_2}= \frac{\nu R T_3}{\nu R T_2} = \frac{T_3}{T_2} = 2 \implies V_3 = V_4 = 2 V_2 = 2 V_0.
    \\
    A_\text{цикл} &= (2P_0 - P_0)(2V_0 - V_0) = 1P_0V_0, \\
    A_{23} &= 2P_0 \cdot (2V_0 - V_0) = 2P_0V_0, \\
    \Delta U_{23} &= \frac 32 \nu R T_3 - \frac 32 \nu R T_3 = \frac 32 P_3 V_3 - \frac 32 P_2 V_2 = \frac 32 \cdot 2 P_0 \cdot 2 V_0 -  \frac 32 \cdot 2 P_0 \cdot V_0 = \frac 32 \cdot 2 \cdot P_0V_0, \\
    \Delta U_{12} &= \frac 32 \nu R T_2 - \frac 32 \nu R T_1 = \frac 32 P_2 V_2 - \frac 32 P_1 V_1 = \frac 32 \cdot 2 P_0 V_0 - \frac 32 P_0 V_0 = \frac 32 \cdot 1 \cdot P_0V_0.
    \\
    \eta &= \frac{A_\text{цикл}}{Q_+} = \frac{A_\text{цикл}}{Q_{12} + Q_{23}}  = \frac{A_\text{цикл}}{A_{12} + \Delta U_{12} + A_{23} + \Delta U_{23}} =  \\
     &= \frac{1P_0V_0}{0 + \frac 32 \cdot 1 \cdot P_0V_0 + 2P_0V_0 + \frac 32 \cdot 2 \cdot P_0V_0} = \frac{1}{\frac 32 \cdot 1 + 2 + \frac 32 \cdot 2} = \frac2{13} \approx 0{,}154.
     \\
    \eta_\text{Карно} &= 1 - \frac{T_\text{х}}{T_\text{н}} = 1 - \frac{T_\text{1}}{T_\text{3}} = 1 - \frac{T_0}{4T_0} = 1 - \frac 1{4}  = \frac34 \approx 0{,}750.
    \end{align*}
}
\solutionspace{360pt}

\tasknumber{2}%
\task{%
    Изобразите в координатах $PV$/$VT$/$PT$ графики изотермического повышения давления в 4 раза (все 3 графика).
    Не забудьте указать оси и масштаб, начальную и конечную точки, направление движения на графике.
}
\solutionspace{100pt}

\tasknumber{3}%
\task{%
    Укажите, верны ли утверждения («да» или «нет» слева от каждого утверждения):
    \begin{enumerate}
        \item При изобарном расширении идеальный газ совершает ровно столько работы, сколько внутренней энергии теряет.
        % \item В силу третьего закона Ньютона, совершённая газом работа и работа, совершённая над ним, всегда равны по модулю и противоположны по знаку.
        \item Работу газа в некотором процессе можно вычислять как площадь под графиком в системе координат $VT$, главное лишь правильно расположить оси.
        % \item Дважды два четыре.
        \item При изобарном процессе внутренняя энергия идеального одноатомного газа не изменяется, даже если ему подводят тепло.
        \item Газ может совершить ненулевую работу в изобарном процессе.
        % \item Адиабатический процесс лишь по воле случая не имеет приставки «изо»: в нём изменяются давление, температура и объём, но это не все макропараметры идеального газа.
        \item Полученное выражение для внутренней энергии идеального газа ($\frac 32 \nu RT$) применимо к двухоатомному газу, при этом, например, уравнение состояния идеального газа применимо независимо от числа атомов в молекулах газа.
    \end{enumerate}
}
\answer{%
    $\text{нет, нет, нет, да, нет}$
}

\variantsplitter

\addpersonalvariant{Сергей Малышев}

\tasknumber{1}%
\task{%
    Определите КПД цикла 12341, рабочим телом которого является идеальный одноатомный газ, если
    12 — изохорический нагрев в шесть раз,
    23 — изобарическое расширение, при котором температура растёт в шесть раз,
    34 — изохора, 41 — изобара.

    Определите КПД цикла Карно, температура нагревателя которого равна максимальной температуре в цикле 12341, а холодильника — минимальной.
    Ответы в обоих случаях оставьте точными в виде нескоратимой дроби, никаких округлений.
}
\answer{%
    \begin{align*}
    A_{12} &= 0, \Delta U_{12} > 0, \implies Q_{12} = A_{12} + \Delta U_{12} > 0.
    \\
    A_{23} &> 0, \Delta U_{23} > 0, \implies Q_{23} = A_{23} + \Delta U_{23} > 0, \\
    A_{34} &= 0, \Delta U_{34} < 0, \implies Q_{34} = A_{34} + \Delta U_{34} < 0, \\
    A_{41} &< 0, \Delta U_{41} < 0, \implies Q_{41} = A_{41} + \Delta U_{41} < 0.
    \\
    P_1V_1 &= \nu R T_1, P_2V_2 = \nu R T_2, P_3V_3 = \nu R T_3, P_4V_4 = \nu R T_4 \text{ — уравнения состояния идеального газа}, \\
    &\text{Пусть $P_0$, $V_0$, $T_0$ — давление, объём и температура в точке 1 (минимальные во всём цикле):} \\
    P_1 &= P_4 = P_0, P_2 = P_3, V_1 = V_2 = V_0, V_3 = V_4, \text{остальные соотношения между объёмами и давлениями не даны, нужно считать} \\
    T_2 &= 6T_1 = 6T_0 \text{(по условию)} \implies \frac{P_2}{P_1} = \frac{P_2V_0}{P_1V_0} = \frac{P_2 V_2}{P_1 V_1}= \frac{\nu R T_2}{\nu R T_1} = \frac{T_2}{T_1} = 6 \implies P_2 = P_3 = 6 P_1 = 6 P_0, \\
    T_3 &= 6T_2 = 36T_0 \text{(по условию)} \implies \frac{V_3}{V_2} = \frac{P_3V_3}{P_2V_2}= \frac{\nu R T_3}{\nu R T_2} = \frac{T_3}{T_2} = 6 \implies V_3 = V_4 = 6 V_2 = 6 V_0.
    \\
    A_\text{цикл} &= (6P_0 - P_0)(6V_0 - V_0) = 25P_0V_0, \\
    A_{23} &= 6P_0 \cdot (6V_0 - V_0) = 30P_0V_0, \\
    \Delta U_{23} &= \frac 32 \nu R T_3 - \frac 32 \nu R T_3 = \frac 32 P_3 V_3 - \frac 32 P_2 V_2 = \frac 32 \cdot 6 P_0 \cdot 6 V_0 -  \frac 32 \cdot 6 P_0 \cdot V_0 = \frac 32 \cdot 30 \cdot P_0V_0, \\
    \Delta U_{12} &= \frac 32 \nu R T_2 - \frac 32 \nu R T_1 = \frac 32 P_2 V_2 - \frac 32 P_1 V_1 = \frac 32 \cdot 6 P_0 V_0 - \frac 32 P_0 V_0 = \frac 32 \cdot 5 \cdot P_0V_0.
    \\
    \eta &= \frac{A_\text{цикл}}{Q_+} = \frac{A_\text{цикл}}{Q_{12} + Q_{23}}  = \frac{A_\text{цикл}}{A_{12} + \Delta U_{12} + A_{23} + \Delta U_{23}} =  \\
     &= \frac{25P_0V_0}{0 + \frac 32 \cdot 5 \cdot P_0V_0 + 30P_0V_0 + \frac 32 \cdot 30 \cdot P_0V_0} = \frac{25}{\frac 32 \cdot 5 + 30 + \frac 32 \cdot 30} = \frac{10}{33} \approx 0{,}303.
     \\
    \eta_\text{Карно} &= 1 - \frac{T_\text{х}}{T_\text{н}} = 1 - \frac{T_\text{1}}{T_\text{3}} = 1 - \frac{T_0}{36T_0} = 1 - \frac 1{36}  = \frac{35}{36} \approx 0{,}972.
    \end{align*}
}
\solutionspace{360pt}

\tasknumber{2}%
\task{%
    Изобразите в координатах $PV$/$VT$/$PT$ графики изотермического повышения давления в 2 раза (все 3 графика).
    Не забудьте указать оси и масштаб, начальную и конечную точки, направление движения на графике.
}
\solutionspace{100pt}

\tasknumber{3}%
\task{%
    Укажите, верны ли утверждения («да» или «нет» слева от каждого утверждения):
    \begin{enumerate}
        \item При адиабатическом расширении идеальный газ совершает ровно столько работы, сколько внутренней энергии теряет.
        % \item В силу третьего закона Ньютона, совершённая газом работа и работа, совершённая над ним, всегда равны по модулю и противоположны по знаку.
        \item Работу газа в некотором процессе можно вычислять как площадь под графиком в системе координат $PV$, главное лишь правильно расположить оси.
        % \item Дважды два пять.
        \item При изобарном процессе внутренняя энергия идеального одноатомного газа не изменяется, даже если ему подводят тепло.
        \item Газ может совершить ненулевую работу в изобарном процессе.
        % \item Адиабатический процесс лишь по воле случая не имеет приставки «изо»: в нём изменяются давление, температура и объём, но это не все макропараметры идеального газа.
        \item Полученное выражение для внутренней энергии идеального газа ($\frac 32 \nu RT$) применимо к трёхатомному газу, при этом, например, уравнение состояния идеального газа применимо независимо от числа атомов в молекулах газа.
    \end{enumerate}
}
\answer{%
    $\text{да, да, нет, да, нет}$
}

\variantsplitter

\addpersonalvariant{Алина Полканова}

\tasknumber{1}%
\task{%
    Определите КПД цикла 12341, рабочим телом которого является идеальный одноатомный газ, если
    12 — изохорический нагрев в пять раз,
    23 — изобарическое расширение, при котором температура растёт в пять раз,
    34 — изохора, 41 — изобара.

    Определите КПД цикла Карно, температура нагревателя которого равна максимальной температуре в цикле 12341, а холодильника — минимальной.
    Ответы в обоих случаях оставьте точными в виде нескоратимой дроби, никаких округлений.
}
\answer{%
    \begin{align*}
    A_{12} &= 0, \Delta U_{12} > 0, \implies Q_{12} = A_{12} + \Delta U_{12} > 0.
    \\
    A_{23} &> 0, \Delta U_{23} > 0, \implies Q_{23} = A_{23} + \Delta U_{23} > 0, \\
    A_{34} &= 0, \Delta U_{34} < 0, \implies Q_{34} = A_{34} + \Delta U_{34} < 0, \\
    A_{41} &< 0, \Delta U_{41} < 0, \implies Q_{41} = A_{41} + \Delta U_{41} < 0.
    \\
    P_1V_1 &= \nu R T_1, P_2V_2 = \nu R T_2, P_3V_3 = \nu R T_3, P_4V_4 = \nu R T_4 \text{ — уравнения состояния идеального газа}, \\
    &\text{Пусть $P_0$, $V_0$, $T_0$ — давление, объём и температура в точке 1 (минимальные во всём цикле):} \\
    P_1 &= P_4 = P_0, P_2 = P_3, V_1 = V_2 = V_0, V_3 = V_4, \text{остальные соотношения между объёмами и давлениями не даны, нужно считать} \\
    T_2 &= 5T_1 = 5T_0 \text{(по условию)} \implies \frac{P_2}{P_1} = \frac{P_2V_0}{P_1V_0} = \frac{P_2 V_2}{P_1 V_1}= \frac{\nu R T_2}{\nu R T_1} = \frac{T_2}{T_1} = 5 \implies P_2 = P_3 = 5 P_1 = 5 P_0, \\
    T_3 &= 5T_2 = 25T_0 \text{(по условию)} \implies \frac{V_3}{V_2} = \frac{P_3V_3}{P_2V_2}= \frac{\nu R T_3}{\nu R T_2} = \frac{T_3}{T_2} = 5 \implies V_3 = V_4 = 5 V_2 = 5 V_0.
    \\
    A_\text{цикл} &= (5P_0 - P_0)(5V_0 - V_0) = 16P_0V_0, \\
    A_{23} &= 5P_0 \cdot (5V_0 - V_0) = 20P_0V_0, \\
    \Delta U_{23} &= \frac 32 \nu R T_3 - \frac 32 \nu R T_3 = \frac 32 P_3 V_3 - \frac 32 P_2 V_2 = \frac 32 \cdot 5 P_0 \cdot 5 V_0 -  \frac 32 \cdot 5 P_0 \cdot V_0 = \frac 32 \cdot 20 \cdot P_0V_0, \\
    \Delta U_{12} &= \frac 32 \nu R T_2 - \frac 32 \nu R T_1 = \frac 32 P_2 V_2 - \frac 32 P_1 V_1 = \frac 32 \cdot 5 P_0 V_0 - \frac 32 P_0 V_0 = \frac 32 \cdot 4 \cdot P_0V_0.
    \\
    \eta &= \frac{A_\text{цикл}}{Q_+} = \frac{A_\text{цикл}}{Q_{12} + Q_{23}}  = \frac{A_\text{цикл}}{A_{12} + \Delta U_{12} + A_{23} + \Delta U_{23}} =  \\
     &= \frac{16P_0V_0}{0 + \frac 32 \cdot 4 \cdot P_0V_0 + 20P_0V_0 + \frac 32 \cdot 20 \cdot P_0V_0} = \frac{16}{\frac 32 \cdot 4 + 20 + \frac 32 \cdot 20} = \frac27 \approx 0{,}286.
     \\
    \eta_\text{Карно} &= 1 - \frac{T_\text{х}}{T_\text{н}} = 1 - \frac{T_\text{1}}{T_\text{3}} = 1 - \frac{T_0}{25T_0} = 1 - \frac 1{25}  = \frac{24}{25} \approx 0{,}960.
    \end{align*}
}
\solutionspace{360pt}

\tasknumber{2}%
\task{%
    Изобразите в координатах $PV$/$VT$/$PT$ графики изотермического повышения давления в 2 раза (все 3 графика).
    Не забудьте указать оси и масштаб, начальную и конечную точки, направление движения на графике.
}
\solutionspace{100pt}

\tasknumber{3}%
\task{%
    Укажите, верны ли утверждения («да» или «нет» слева от каждого утверждения):
    \begin{enumerate}
        \item При изобарном расширении идеальный газ совершает ровно столько работы, сколько внутренней энергии теряет.
        % \item В силу третьего закона Ньютона, совершённая газом работа и работа, совершённая над ним, всегда равны по модулю и противоположны по знаку.
        \item Работу газа в некотором процессе можно вычислять как площадь под графиком в системе координат $PT$, главное лишь правильно расположить оси.
        % \item Дважды два четыре.
        \item При изобарном процессе внутренняя энергия идеального одноатомного газа не изменяется, даже если ему подводят тепло.
        \item Газ может совершить ненулевую работу в изобарном процессе.
        % \item Адиабатический процесс лишь по воле случая не имеет приставки «изо»: в нём изменяются давление, температура и объём, но это не все макропараметры идеального газа.
        \item Полученное выражение для внутренней энергии идеального газа ($\frac 32 \nu RT$) применимо к трёхатомному газу, при этом, например, уравнение состояния идеального газа применимо независимо от числа атомов в молекулах газа.
    \end{enumerate}
}
\answer{%
    $\text{нет, нет, нет, да, нет}$
}

\variantsplitter

\addpersonalvariant{Сергей Пономарёв}

\tasknumber{1}%
\task{%
    Определите КПД цикла 12341, рабочим телом которого является идеальный одноатомный газ, если
    12 — изохорический нагрев в пять раз,
    23 — изобарическое расширение, при котором температура растёт в два раза,
    34 — изохора, 41 — изобара.

    Определите КПД цикла Карно, температура нагревателя которого равна максимальной температуре в цикле 12341, а холодильника — минимальной.
    Ответы в обоих случаях оставьте точными в виде нескоратимой дроби, никаких округлений.
}
\answer{%
    \begin{align*}
    A_{12} &= 0, \Delta U_{12} > 0, \implies Q_{12} = A_{12} + \Delta U_{12} > 0.
    \\
    A_{23} &> 0, \Delta U_{23} > 0, \implies Q_{23} = A_{23} + \Delta U_{23} > 0, \\
    A_{34} &= 0, \Delta U_{34} < 0, \implies Q_{34} = A_{34} + \Delta U_{34} < 0, \\
    A_{41} &< 0, \Delta U_{41} < 0, \implies Q_{41} = A_{41} + \Delta U_{41} < 0.
    \\
    P_1V_1 &= \nu R T_1, P_2V_2 = \nu R T_2, P_3V_3 = \nu R T_3, P_4V_4 = \nu R T_4 \text{ — уравнения состояния идеального газа}, \\
    &\text{Пусть $P_0$, $V_0$, $T_0$ — давление, объём и температура в точке 1 (минимальные во всём цикле):} \\
    P_1 &= P_4 = P_0, P_2 = P_3, V_1 = V_2 = V_0, V_3 = V_4, \text{остальные соотношения между объёмами и давлениями не даны, нужно считать} \\
    T_2 &= 5T_1 = 5T_0 \text{(по условию)} \implies \frac{P_2}{P_1} = \frac{P_2V_0}{P_1V_0} = \frac{P_2 V_2}{P_1 V_1}= \frac{\nu R T_2}{\nu R T_1} = \frac{T_2}{T_1} = 5 \implies P_2 = P_3 = 5 P_1 = 5 P_0, \\
    T_3 &= 2T_2 = 10T_0 \text{(по условию)} \implies \frac{V_3}{V_2} = \frac{P_3V_3}{P_2V_2}= \frac{\nu R T_3}{\nu R T_2} = \frac{T_3}{T_2} = 2 \implies V_3 = V_4 = 2 V_2 = 2 V_0.
    \\
    A_\text{цикл} &= (2P_0 - P_0)(5V_0 - V_0) = 4P_0V_0, \\
    A_{23} &= 5P_0 \cdot (2V_0 - V_0) = 5P_0V_0, \\
    \Delta U_{23} &= \frac 32 \nu R T_3 - \frac 32 \nu R T_3 = \frac 32 P_3 V_3 - \frac 32 P_2 V_2 = \frac 32 \cdot 5 P_0 \cdot 2 V_0 -  \frac 32 \cdot 5 P_0 \cdot V_0 = \frac 32 \cdot 5 \cdot P_0V_0, \\
    \Delta U_{12} &= \frac 32 \nu R T_2 - \frac 32 \nu R T_1 = \frac 32 P_2 V_2 - \frac 32 P_1 V_1 = \frac 32 \cdot 5 P_0 V_0 - \frac 32 P_0 V_0 = \frac 32 \cdot 4 \cdot P_0V_0.
    \\
    \eta &= \frac{A_\text{цикл}}{Q_+} = \frac{A_\text{цикл}}{Q_{12} + Q_{23}}  = \frac{A_\text{цикл}}{A_{12} + \Delta U_{12} + A_{23} + \Delta U_{23}} =  \\
     &= \frac{4P_0V_0}{0 + \frac 32 \cdot 4 \cdot P_0V_0 + 5P_0V_0 + \frac 32 \cdot 5 \cdot P_0V_0} = \frac{4}{\frac 32 \cdot 4 + 5 + \frac 32 \cdot 5} = \frac8{37} \approx 0{,}216.
     \\
    \eta_\text{Карно} &= 1 - \frac{T_\text{х}}{T_\text{н}} = 1 - \frac{T_\text{1}}{T_\text{3}} = 1 - \frac{T_0}{10T_0} = 1 - \frac 1{10}  = \frac9{10} \approx 0{,}900.
    \end{align*}
}
\solutionspace{360pt}

\tasknumber{2}%
\task{%
    Изобразите в координатах $PV$/$VT$/$PT$ графики изохорического охлаждения в 3 раза (все 3 графика).
    Не забудьте указать оси и масштаб, начальную и конечную точки, направление движения на графике.
}
\solutionspace{100pt}

\tasknumber{3}%
\task{%
    Укажите, верны ли утверждения («да» или «нет» слева от каждого утверждения):
    \begin{enumerate}
        \item При изобарном расширении идеальный газ совершает ровно столько работы, сколько внутренней энергии теряет.
        % \item В силу третьего закона Ньютона, совершённая газом работа и работа, совершённая над ним, всегда равны по модулю и противоположны по знаку.
        \item Работу газа в некотором процессе можно вычислять как площадь под графиком в системе координат $PV$, главное лишь правильно расположить оси.
        % \item Дважды два пять.
        \item При изохорном процессе внутренняя энергия идеального одноатомного газа не изменяется, даже если ему подводят тепло.
        \item Газ может совершить ненулевую работу в изохорном процессе.
        % \item Адиабатический процесс лишь по воле случая не имеет приставки «изо»: в нём изменяются давление, температура и объём, но это не все макропараметры идеального газа.
        \item Полученное выражение для внутренней энергии идеального газа ($\frac 32 \nu RT$) применимо к двухоатомному газу, при этом, например, уравнение состояния идеального газа применимо независимо от числа атомов в молекулах газа.
    \end{enumerate}
}
\answer{%
    $\text{нет, да, нет, нет, нет}$
}

\variantsplitter

\addpersonalvariant{Егор Свистушкин}

\tasknumber{1}%
\task{%
    Определите КПД цикла 12341, рабочим телом которого является идеальный одноатомный газ, если
    12 — изохорический нагрев в шесть раз,
    23 — изобарическое расширение, при котором температура растёт в три раза,
    34 — изохора, 41 — изобара.

    Определите КПД цикла Карно, температура нагревателя которого равна максимальной температуре в цикле 12341, а холодильника — минимальной.
    Ответы в обоих случаях оставьте точными в виде нескоратимой дроби, никаких округлений.
}
\answer{%
    \begin{align*}
    A_{12} &= 0, \Delta U_{12} > 0, \implies Q_{12} = A_{12} + \Delta U_{12} > 0.
    \\
    A_{23} &> 0, \Delta U_{23} > 0, \implies Q_{23} = A_{23} + \Delta U_{23} > 0, \\
    A_{34} &= 0, \Delta U_{34} < 0, \implies Q_{34} = A_{34} + \Delta U_{34} < 0, \\
    A_{41} &< 0, \Delta U_{41} < 0, \implies Q_{41} = A_{41} + \Delta U_{41} < 0.
    \\
    P_1V_1 &= \nu R T_1, P_2V_2 = \nu R T_2, P_3V_3 = \nu R T_3, P_4V_4 = \nu R T_4 \text{ — уравнения состояния идеального газа}, \\
    &\text{Пусть $P_0$, $V_0$, $T_0$ — давление, объём и температура в точке 1 (минимальные во всём цикле):} \\
    P_1 &= P_4 = P_0, P_2 = P_3, V_1 = V_2 = V_0, V_3 = V_4, \text{остальные соотношения между объёмами и давлениями не даны, нужно считать} \\
    T_2 &= 6T_1 = 6T_0 \text{(по условию)} \implies \frac{P_2}{P_1} = \frac{P_2V_0}{P_1V_0} = \frac{P_2 V_2}{P_1 V_1}= \frac{\nu R T_2}{\nu R T_1} = \frac{T_2}{T_1} = 6 \implies P_2 = P_3 = 6 P_1 = 6 P_0, \\
    T_3 &= 3T_2 = 18T_0 \text{(по условию)} \implies \frac{V_3}{V_2} = \frac{P_3V_3}{P_2V_2}= \frac{\nu R T_3}{\nu R T_2} = \frac{T_3}{T_2} = 3 \implies V_3 = V_4 = 3 V_2 = 3 V_0.
    \\
    A_\text{цикл} &= (3P_0 - P_0)(6V_0 - V_0) = 10P_0V_0, \\
    A_{23} &= 6P_0 \cdot (3V_0 - V_0) = 12P_0V_0, \\
    \Delta U_{23} &= \frac 32 \nu R T_3 - \frac 32 \nu R T_3 = \frac 32 P_3 V_3 - \frac 32 P_2 V_2 = \frac 32 \cdot 6 P_0 \cdot 3 V_0 -  \frac 32 \cdot 6 P_0 \cdot V_0 = \frac 32 \cdot 12 \cdot P_0V_0, \\
    \Delta U_{12} &= \frac 32 \nu R T_2 - \frac 32 \nu R T_1 = \frac 32 P_2 V_2 - \frac 32 P_1 V_1 = \frac 32 \cdot 6 P_0 V_0 - \frac 32 P_0 V_0 = \frac 32 \cdot 5 \cdot P_0V_0.
    \\
    \eta &= \frac{A_\text{цикл}}{Q_+} = \frac{A_\text{цикл}}{Q_{12} + Q_{23}}  = \frac{A_\text{цикл}}{A_{12} + \Delta U_{12} + A_{23} + \Delta U_{23}} =  \\
     &= \frac{10P_0V_0}{0 + \frac 32 \cdot 5 \cdot P_0V_0 + 12P_0V_0 + \frac 32 \cdot 12 \cdot P_0V_0} = \frac{10}{\frac 32 \cdot 5 + 12 + \frac 32 \cdot 12} = \frac4{15} \approx 0{,}267.
     \\
    \eta_\text{Карно} &= 1 - \frac{T_\text{х}}{T_\text{н}} = 1 - \frac{T_\text{1}}{T_\text{3}} = 1 - \frac{T_0}{18T_0} = 1 - \frac 1{18}  = \frac{17}{18} \approx 0{,}944.
    \end{align*}
}
\solutionspace{360pt}

\tasknumber{2}%
\task{%
    Изобразите в координатах $PV$/$VT$/$PT$ графики изобарического сжатия в 4 раза (все 3 графика).
    Не забудьте указать оси и масштаб, начальную и конечную точки, направление движения на графике.
}
\solutionspace{100pt}

\tasknumber{3}%
\task{%
    Укажите, верны ли утверждения («да» или «нет» слева от каждого утверждения):
    \begin{enumerate}
        \item При адиабатическом расширении идеальный газ совершает ровно столько работы, сколько внутренней энергии теряет.
        % \item В силу третьего закона Ньютона, совершённая газом работа и работа, совершённая над ним, всегда равны по модулю и противоположны по знаку.
        \item Работу газа в некотором процессе можно вычислять как площадь под графиком в системе координат $PV$, главное лишь правильно расположить оси.
        % \item Дважды два пять.
        \item При изотермическом процессе внутренняя энергия идеального одноатомного газа не изменяется, даже если ему подводят тепло.
        \item Газ может совершить ненулевую работу в изобарном процессе.
        % \item Адиабатический процесс лишь по воле случая не имеет приставки «изо»: в нём изменяются давление, температура и объём, но это не все макропараметры идеального газа.
        \item Полученное выражение для внутренней энергии идеального газа ($\frac 32 \nu RT$) применимо к одноатомному газу, при этом, например, уравнение состояния идеального газа применимо независимо от числа атомов в молекулах газа.
    \end{enumerate}
}
\answer{%
    $\text{да, да, да, да, да}$
}

\variantsplitter

\addpersonalvariant{Дмитрий Соколов}

\tasknumber{1}%
\task{%
    Определите КПД цикла 12341, рабочим телом которого является идеальный одноатомный газ, если
    12 — изохорический нагрев в три раза,
    23 — изобарическое расширение, при котором температура растёт в три раза,
    34 — изохора, 41 — изобара.

    Определите КПД цикла Карно, температура нагревателя которого равна максимальной температуре в цикле 12341, а холодильника — минимальной.
    Ответы в обоих случаях оставьте точными в виде нескоратимой дроби, никаких округлений.
}
\answer{%
    \begin{align*}
    A_{12} &= 0, \Delta U_{12} > 0, \implies Q_{12} = A_{12} + \Delta U_{12} > 0.
    \\
    A_{23} &> 0, \Delta U_{23} > 0, \implies Q_{23} = A_{23} + \Delta U_{23} > 0, \\
    A_{34} &= 0, \Delta U_{34} < 0, \implies Q_{34} = A_{34} + \Delta U_{34} < 0, \\
    A_{41} &< 0, \Delta U_{41} < 0, \implies Q_{41} = A_{41} + \Delta U_{41} < 0.
    \\
    P_1V_1 &= \nu R T_1, P_2V_2 = \nu R T_2, P_3V_3 = \nu R T_3, P_4V_4 = \nu R T_4 \text{ — уравнения состояния идеального газа}, \\
    &\text{Пусть $P_0$, $V_0$, $T_0$ — давление, объём и температура в точке 1 (минимальные во всём цикле):} \\
    P_1 &= P_4 = P_0, P_2 = P_3, V_1 = V_2 = V_0, V_3 = V_4, \text{остальные соотношения между объёмами и давлениями не даны, нужно считать} \\
    T_2 &= 3T_1 = 3T_0 \text{(по условию)} \implies \frac{P_2}{P_1} = \frac{P_2V_0}{P_1V_0} = \frac{P_2 V_2}{P_1 V_1}= \frac{\nu R T_2}{\nu R T_1} = \frac{T_2}{T_1} = 3 \implies P_2 = P_3 = 3 P_1 = 3 P_0, \\
    T_3 &= 3T_2 = 9T_0 \text{(по условию)} \implies \frac{V_3}{V_2} = \frac{P_3V_3}{P_2V_2}= \frac{\nu R T_3}{\nu R T_2} = \frac{T_3}{T_2} = 3 \implies V_3 = V_4 = 3 V_2 = 3 V_0.
    \\
    A_\text{цикл} &= (3P_0 - P_0)(3V_0 - V_0) = 4P_0V_0, \\
    A_{23} &= 3P_0 \cdot (3V_0 - V_0) = 6P_0V_0, \\
    \Delta U_{23} &= \frac 32 \nu R T_3 - \frac 32 \nu R T_3 = \frac 32 P_3 V_3 - \frac 32 P_2 V_2 = \frac 32 \cdot 3 P_0 \cdot 3 V_0 -  \frac 32 \cdot 3 P_0 \cdot V_0 = \frac 32 \cdot 6 \cdot P_0V_0, \\
    \Delta U_{12} &= \frac 32 \nu R T_2 - \frac 32 \nu R T_1 = \frac 32 P_2 V_2 - \frac 32 P_1 V_1 = \frac 32 \cdot 3 P_0 V_0 - \frac 32 P_0 V_0 = \frac 32 \cdot 2 \cdot P_0V_0.
    \\
    \eta &= \frac{A_\text{цикл}}{Q_+} = \frac{A_\text{цикл}}{Q_{12} + Q_{23}}  = \frac{A_\text{цикл}}{A_{12} + \Delta U_{12} + A_{23} + \Delta U_{23}} =  \\
     &= \frac{4P_0V_0}{0 + \frac 32 \cdot 2 \cdot P_0V_0 + 6P_0V_0 + \frac 32 \cdot 6 \cdot P_0V_0} = \frac{4}{\frac 32 \cdot 2 + 6 + \frac 32 \cdot 6} = \frac29 \approx 0{,}222.
     \\
    \eta_\text{Карно} &= 1 - \frac{T_\text{х}}{T_\text{н}} = 1 - \frac{T_\text{1}}{T_\text{3}} = 1 - \frac{T_0}{9T_0} = 1 - \frac 1{9}  = \frac89 \approx 0{,}889.
    \end{align*}
}
\solutionspace{360pt}

\tasknumber{2}%
\task{%
    Изобразите в координатах $PV$/$VT$/$PT$ графики изотермического повышения давления в 2 раза (все 3 графика).
    Не забудьте указать оси и масштаб, начальную и конечную точки, направление движения на графике.
}
\solutionspace{100pt}

\tasknumber{3}%
\task{%
    Укажите, верны ли утверждения («да» или «нет» слева от каждого утверждения):
    \begin{enumerate}
        \item При изобарном расширении идеальный газ совершает ровно столько работы, сколько внутренней энергии теряет.
        % \item В силу третьего закона Ньютона, совершённая газом работа и работа, совершённая над ним, всегда равны по модулю и противоположны по знаку.
        \item Работу газа в некотором процессе можно вычислять как площадь под графиком в системе координат $PT$, главное лишь правильно расположить оси.
        % \item Дважды два пять.
        \item При изотермическом процессе внутренняя энергия идеального одноатомного газа не изменяется, даже если ему подводят тепло.
        \item Газ может совершить ненулевую работу в изотермическом процессе.
        % \item Адиабатический процесс лишь по воле случая не имеет приставки «изо»: в нём изменяются давление, температура и объём, но это не все макропараметры идеального газа.
        \item Полученное выражение для внутренней энергии идеального газа ($\frac 32 \nu RT$) применимо к трёхатомному газу, при этом, например, уравнение состояния идеального газа применимо независимо от числа атомов в молекулах газа.
    \end{enumerate}
}
\answer{%
    $\text{нет, нет, да, да, нет}$
}

\variantsplitter

\addpersonalvariant{Арсений Трофимов}

\tasknumber{1}%
\task{%
    Определите КПД цикла 12341, рабочим телом которого является идеальный одноатомный газ, если
    12 — изохорический нагрев в четыре раза,
    23 — изобарическое расширение, при котором температура растёт в шесть раз,
    34 — изохора, 41 — изобара.

    Определите КПД цикла Карно, температура нагревателя которого равна максимальной температуре в цикле 12341, а холодильника — минимальной.
    Ответы в обоих случаях оставьте точными в виде нескоратимой дроби, никаких округлений.
}
\answer{%
    \begin{align*}
    A_{12} &= 0, \Delta U_{12} > 0, \implies Q_{12} = A_{12} + \Delta U_{12} > 0.
    \\
    A_{23} &> 0, \Delta U_{23} > 0, \implies Q_{23} = A_{23} + \Delta U_{23} > 0, \\
    A_{34} &= 0, \Delta U_{34} < 0, \implies Q_{34} = A_{34} + \Delta U_{34} < 0, \\
    A_{41} &< 0, \Delta U_{41} < 0, \implies Q_{41} = A_{41} + \Delta U_{41} < 0.
    \\
    P_1V_1 &= \nu R T_1, P_2V_2 = \nu R T_2, P_3V_3 = \nu R T_3, P_4V_4 = \nu R T_4 \text{ — уравнения состояния идеального газа}, \\
    &\text{Пусть $P_0$, $V_0$, $T_0$ — давление, объём и температура в точке 1 (минимальные во всём цикле):} \\
    P_1 &= P_4 = P_0, P_2 = P_3, V_1 = V_2 = V_0, V_3 = V_4, \text{остальные соотношения между объёмами и давлениями не даны, нужно считать} \\
    T_2 &= 4T_1 = 4T_0 \text{(по условию)} \implies \frac{P_2}{P_1} = \frac{P_2V_0}{P_1V_0} = \frac{P_2 V_2}{P_1 V_1}= \frac{\nu R T_2}{\nu R T_1} = \frac{T_2}{T_1} = 4 \implies P_2 = P_3 = 4 P_1 = 4 P_0, \\
    T_3 &= 6T_2 = 24T_0 \text{(по условию)} \implies \frac{V_3}{V_2} = \frac{P_3V_3}{P_2V_2}= \frac{\nu R T_3}{\nu R T_2} = \frac{T_3}{T_2} = 6 \implies V_3 = V_4 = 6 V_2 = 6 V_0.
    \\
    A_\text{цикл} &= (6P_0 - P_0)(4V_0 - V_0) = 15P_0V_0, \\
    A_{23} &= 4P_0 \cdot (6V_0 - V_0) = 20P_0V_0, \\
    \Delta U_{23} &= \frac 32 \nu R T_3 - \frac 32 \nu R T_3 = \frac 32 P_3 V_3 - \frac 32 P_2 V_2 = \frac 32 \cdot 4 P_0 \cdot 6 V_0 -  \frac 32 \cdot 4 P_0 \cdot V_0 = \frac 32 \cdot 20 \cdot P_0V_0, \\
    \Delta U_{12} &= \frac 32 \nu R T_2 - \frac 32 \nu R T_1 = \frac 32 P_2 V_2 - \frac 32 P_1 V_1 = \frac 32 \cdot 4 P_0 V_0 - \frac 32 P_0 V_0 = \frac 32 \cdot 3 \cdot P_0V_0.
    \\
    \eta &= \frac{A_\text{цикл}}{Q_+} = \frac{A_\text{цикл}}{Q_{12} + Q_{23}}  = \frac{A_\text{цикл}}{A_{12} + \Delta U_{12} + A_{23} + \Delta U_{23}} =  \\
     &= \frac{15P_0V_0}{0 + \frac 32 \cdot 3 \cdot P_0V_0 + 20P_0V_0 + \frac 32 \cdot 20 \cdot P_0V_0} = \frac{15}{\frac 32 \cdot 3 + 20 + \frac 32 \cdot 20} = \frac{30}{109} \approx 0{,}275.
     \\
    \eta_\text{Карно} &= 1 - \frac{T_\text{х}}{T_\text{н}} = 1 - \frac{T_\text{1}}{T_\text{3}} = 1 - \frac{T_0}{24T_0} = 1 - \frac 1{24}  = \frac{23}{24} \approx 0{,}958.
    \end{align*}
}
\solutionspace{360pt}

\tasknumber{2}%
\task{%
    Изобразите в координатах $PV$/$VT$/$PT$ графики изобарического сжатия в 3 раза (все 3 графика).
    Не забудьте указать оси и масштаб, начальную и конечную точки, направление движения на графике.
}
\solutionspace{100pt}

\tasknumber{3}%
\task{%
    Укажите, верны ли утверждения («да» или «нет» слева от каждого утверждения):
    \begin{enumerate}
        \item При адиабатическом расширении идеальный газ совершает ровно столько работы, сколько внутренней энергии теряет.
        % \item В силу третьего закона Ньютона, совершённая газом работа и работа, совершённая над ним, всегда равны по модулю и противоположны по знаку.
        \item Работу газа в некотором процессе можно вычислять как площадь под графиком в системе координат $VT$, главное лишь правильно расположить оси.
        % \item Дважды два три.
        \item При изобарном процессе внутренняя энергия идеального одноатомного газа не изменяется, даже если ему подводят тепло.
        \item Газ может совершить ненулевую работу в изотермическом процессе.
        % \item Адиабатический процесс лишь по воле случая не имеет приставки «изо»: в нём изменяются давление, температура и объём, но это не все макропараметры идеального газа.
        \item Полученное выражение для внутренней энергии идеального газа ($\frac 32 \nu RT$) применимо к трёхатомному газу, при этом, например, уравнение состояния идеального газа применимо независимо от числа атомов в молекулах газа.
    \end{enumerate}
}
\answer{%
    $\text{да, нет, нет, да, нет}$
}
% autogenerated
