\adddate{8 класс. 18 сентября 2018}

\task 1
Почему в прудах, лунках, озёрах лёд вначале появляется на поверхности?

\task 2
Почему нагретые детали охлаждаются в воде быстрее, чем в воздухе?

\task 3
Почему риск обжечь губы при пользовании металлической кружкой больше, чем при использовании фарфоровой чашки?

\task 4
Стальное сверло массой $m = 10\units{г}$, при работе нагрелось от $T_1 = 15\celsius$ до $T_2 = 115\celsius$. Сколько энергии $E$ израсходовано двигателем непосредственно на нагревание сверла? 
\cother{ст}{500}.

\task 5
Какое количество теплоты $Q$ отдаст кирпичная печь массой $m = 0{,}35\units{т}$, если при её остывании температура изменилась на $\Delta T = 50\celsius$?
\cother{кирп}{880}.

\task 6
Какое количество теплоты $Q$ получила вода при нагревании от $T_1 = 17\celsius$ до $T_2 = 22\celsius$ в бассейне, длина которого $L = 100\units{м}$, ширина $d=6\units{м}$, а глубина $h=2\units{м}$?
\cother{воды}{4200}.

\task 7
Определите, теплоёмкость металла, из которого сделан брусок массой $m=300\units{г}$, если при нагревании от $T_1 = 18\celsius$ до $T_2 = 26\celsius$ его внутренняя энергия увеличилась на $Q = 912\units{Дж}$?

\task 8
Определите, из какого металла сделан брусок массой $m=100\units{г}$, если при его охлаждении от $T_1 = 24\celsius$ до $T_2 = 20\celsius$ его внутренняя энергия уменьшилась на $Q = 216\units{Дж}$?


