\documentclass[12pt,a4paper]{amsart}%DVI-mode.
\usepackage{graphics,graphicx,epsfig}%DVI-mode.
%\documentclass[pdftex,12pt]{amsart} %PDF-mode.
%\usepackage[pdftex]{graphicx}       %PDF-mode.

%\usepackage{a4wide}                 % Fit the text to A4 page tightly.
\usepackage[utf8]{inputenc}
\usepackage[T2A]{fontenc}
\usepackage[english,russian]{babel} % Download Russian fonts.
\usepackage{amsmath,amsfonts,amssymb,amsthm,amscd,mathrsfs} % Use AMS symbols.
\usepackage{tikz}
\usetikzlibrary{circuits.ee.IEC}
\usetikzlibrary{shapes.geometric}
\usetikzlibrary{decorations.markings}
%\usetikzlibrary{dashs}
%\usetikzlibrary{info}


\textheight=29cm % высота текста
\textwidth=18cm % ширина текста
\topmargin=-2.5cm % отступ от верхнего края
\parskip=6pt % интервал между абзацами
\oddsidemargin=-1.5cm
\evensidemargin=-1.5cm 

% wide docs
% \oddsidemargin=0cm
% \evensidemargin=0cm 
% \textheight=29cm % высота текста
% \textwidth=15cm % ширина текста
% \topmargin=-1.5cm % отступ от верхнего края
% \parskip=18pt % интервал между абзацами


\parindent=0pt % абзацный отступ
\tolerance=500 % терпимость к "жидким" строкам
\binoppenalty=10000 % штраф за перенос формул - 10000 - абсолютный запрет
\relpenalty=10000
\flushbottom % выравнивание высоты страниц
\def\baselinestretch{1.00}
\pagenumbering{gobble}

\begin{document}
\newcommand\bivec[2]{\begin{pmatrix} #1 \\ #2 \end{pmatrix}}

\newcommand\ol[1]{\overline{#1}}

\newcommand\p[1]{\ensuremath{\Prob\!\left(#1\right)}}
\def\cond{\,|\,}
\newcommand\e[1]{\mathsf{E}\!\left(#1\right)}
\newcommand\disp[1]{\mathsf{D}\!\left(#1\right)}
%\newcommand\norm[2]{\mathcal{N}\!\cbr{#1,#2}}
\newcommand\sign{\text{ sign }}

\newcommand\al[1]{\begin{align*} #1 \end{align*}}
\newcommand\begcas[1]{\begin{cases}#1\end{cases}}
\newcommand\tab[2]{	\vspace{-#1pt}
						\begin{tabbing} 
						#2
						\end{tabbing}
					\vspace{-#1pt}
					}


\newcommand\maintext[1]{{\bfseries\sffamily{#1}}}
\newcommand\simpletitle[1]{\begin{center} \maintext{#1} \end{center}}

\def\le{\leqslant}
\def\ge{\geqslant}
\def\Ell{\mathcal{L}}
\def\eps{\varepsilon}
\def\x{\ensuremath{\textbf{x}}}
\def\y{\ensuremath{\textbf{y}}}
\def\Rn{\ensuremath{\mathbb{R}^n}}
\def\RSS{\mathsf{RSS}}

\newcommand\mb[1]{\ensuremath{\boldsymbol{\mathbf{#1}}}}
\newcommand\argmax[1]{\arg\underset{#1}\max\,} % \operatornamewithlimits
%\newcommand{\prodl}{\mathop{\textstyle\prod}\limits}
\newcommand{\prodl}{\prod\limits}
\newcommand{\suml}{\sum\limits}
\newcommand\foral[1]{\forall\,#1\:}
\newcommand\exist[1]{\exists\,#1\:\colon}

\newcommand\cbr[1]{\left(#1\right)} %circled brackets
\newcommand\fbr[1]{\left\{#1\right\}} %figure brackets
\newcommand\sbr[1]{\left[#1\right]} %square brackets
\newcommand\modul[1]{\left|#1\right|}
\newcommand\cdf[2]{\cdot\frac{#1}{#2}}
\newcommand\integr[3]{\int\limits_{#1}^{#2}{#3}}
\newcommand\obol[1]{O\!\cbr{#1}}
\newcommand\norm[1]{\ensuremath{\left\|{#1}\right\|}}

\newcommand\dd[2]{\frac{\partial#1}{\partial#2}}

\newcommand\addeps[2]{
	\begin{figure} [!ht] %lrp
		\centering
		\includegraphics[height=240px]{#1.eps}
		\vspace{-10pt}
		\caption{#2}
		\label{eps:#1}
	\end{figure}
}

\newcommand\addtikz[4]{
	\begin{figure} [!ht] %lrp
		\centering
		\begin{tikzpicture}[x=#2cm,y=#2cm,#3]
			\input{#1.tikz}
		\end{tikzpicture}
		\vspace{-10pt}
		\caption{#4}
		\label{tikz:#1}	
	\end{figure}
}



\newcommand\addepssize[3]{
	\begin{figure} [!ht] %lrp hp
		\centering
		\includegraphics[height=#3px]{#1.eps}
		\vspace{-10pt}
		\caption{#2}
		\label{eps:#1}
	\end{figure}
}

\def\algorithmicrequire{\textbf{Вход:}}
\def\algorithmicensure{\textbf{Выход:}}
\def\algorithmicif{\textbf{если}}
\def\algorithmicthen{\textbf{то}}
\def\algorithmicelse{\textbf{иначе}}
\def\algorithmicelsif{\textbf{иначе если}}
\def\algorithmicfor{\textbf{для}}
\def\algorithmicforall{\textbf{для всех}}
\def\algorithmicdo{}
\def\algorithmicwhile{\textbf{пока}}
\def\algorithmicrepeat{\textbf{повторять}}
\def\algorithmicuntil{\textbf{пока}}
\def\algorithmicloop{\textbf{цикл}}
% переопределение стиля комментариев
\def\algorithmiccomment#1{\quad// {\sl #1}}
%\raggedright
\classdate{7}{20 апреля 2018}

\task 1
Площадь большого поршня гидравлического домкрата $S_1 = 20\units{см}^2$, а малого $S_2 = 0{,}5\units{см}^2.$ Груз какой максимальной массы можно поднять этим домкратом, если на малый поршень давить с силой не более $F=200\units{Н}?$ Силой трения от стенки цилиндров пренебречь.

\task 2
В сосуд налита вода. Расстояние от поверхности воды до дна $H = 0{,}5\units{м},$ площадь дна $S = 0{,}1\units{м}^2.$ Найти гидростатическое давление $P_1$ и полное давление $P_2$ вблизи дна. Найти силу давления воды на дно. Плотность воды \rhowater

\task 3
На лёгкий поршень площадью $S=900\units{см}^2,$ касающийся поверхности воды, поставили гирю массы $m=3\units{кг}$. Высота слоя воды в сосуде с вертикальными стенками $H = 20\units{см}$. Определить давление жидкости вблизи дна, если плотность воды \rhowater

\task 4
Давление газов в конце сгорания в цилиндре дизельного двигателя трактора $P = 9\units{МПа}.$ Диаметр цилиндра $d = 130\units{мм}.$ С какой силой газы давят на поршень в цилиндре? Площадь круга диаметром $D$ равна $S = \cfrac{\pi D^2}4.$

\task 5
Площадь малого поршня гидравлического подъёмника $S_1 = 0{,}8\units{см}^2$, а большого $S_2 = 40\units{см}^2.$ Какую силу $F$ надо приложить к малому поршню, чтобы поднять груз весом $P = 8\units{кН}?$

\task 6
Герметичный сосуд полностью заполнен водой и стоит на столе. На небольшой поршень площадью $S$ давят рукой с силой $F$. Поршень находится ниже крышки сосуда на $H_1$, выше дна на $H_2$ и может свободно перемещаться. Плотность воды $\rho$, атмосферное давление $P_A$. Найти давления $P_1$ и $P_2$ в воде вблизи крышки и дна сосуда.
\\ \\
\classdate{7}{20 апреля 2018}

\task 1
Площадь большого поршня гидравлического домкрата $S_1 = 20\units{см}^2$, а малого $S_2 = 0{,}5\units{см}^2.$ Груз какой максимальной массы можно поднять этим домкратом, если на малый поршень давить с силой не более $F=200\units{Н}?$ Силой трения от стенки цилиндров пренебречь.

\task 2
В сосуд налита вода. Расстояние от поверхности воды до дна $H = 0{,}5\units{м},$ площадь дна $S = 0{,}1\units{м}^2.$ Найти гидростатическое давление $P_1$ и полное давление $P_2$ вблизи дна. Найти силу давления воды на дно. Плотность воды \rhowater

\task 3
На лёгкий поршень площадью $S=900\units{см}^2,$ касающийся поверхности воды, поставили гирю массы $m=3\units{кг}$. Высота слоя воды в сосуде с вертикальными стенками $H = 20\units{см}$. Определить давление жидкости вблизи дна, если плотность воды \rhowater

\task 4
Давление газов в конце сгорания в цилиндре дизельного двигателя трактора $P = 9\units{МПа}.$ Диаметр цилиндра $d = 130\units{мм}.$ С какой силой газы давят на поршень в цилиндре? Площадь круга диаметром $D$ равна $S = \cfrac{\pi D^2}4.$

\task 5
Площадь малого поршня гидравлического подъёмника $S_1 = 0{,}8\units{см}^2$, а большого $S_2 = 40\units{см}^2.$ Какую силу $F$ надо приложить к малому поршню, чтобы поднять груз весом $P = 8\units{кН}?$

\task 6
Герметичный сосуд полностью заполнен водой и стоит на столе. На небольшой поршень площадью $S$ давят рукой с силой $F$. Поршень находится ниже крышки сосуда на $H_1$, выше дна на $H_2$ и может свободно перемещаться. Плотность воды $\rho$, атмосферное давление $P_A$. Найти давления $P_1$ и $P_2$ в воде вблизи крышки и дна сосуда.

\newpage

\adddate{8 класс. 20 апреля 2018}

\task 1
Между точками $A$ и $B$ электрической цепи подключены последовательно резисторы $R_1 = 10\units{Ом}$ и $R_2 = 20\units{Ом}$ и параллельно им $R_3 = 30\units{Ом}.$ Найдите эквивалентное сопротивление $R_{AB}$ этого участка цепи.

\task 2
Электрическая цепь состоит из последовательности $N$ одинаковых звеньев, в которых каждый резистор имеет сопротивление $r$. Последнее звено замкнуто резистором сопротивлением $R$. При каком соотношении $\cfrac{R}{r}$ сопротивление цепи не зависит от числа звеньев?

\task 3
Для измерения сопротивления $R$ проводника собрана электрическая цепь. Вольтметр $V$ показывает напряжение $U_V = 5\units{В},$ показание амперметра $A$ равно $I_A = 25\units{мА}.$ Найдите величину $R$ сопротивления проводника. Внутреннее сопротивление вольтметра $R_V = 1{,}0\units{кОм},$ внутреннее сопротивление амперметра $R_A = 2{,}0\units{Ом}.$

\task 4
Шкала гальванометра имеет $N=100$ делений, цена деления $\delta = 1\units{мкА}$. Внутреннее сопротивление гальванометра $R_G = 1{,}0\units{кОм}.$ Как из этого прибора сделать вольтметр для измерения напряжений до $U = 100\units{В}$ или амперметр для измерения токов силой до $I = 1\units{А}?$

\\ \\ \\ \\ \\ \\ \\ \\
\adddate{8 класс. 20 апреля 2018}

\task 1
Между точками $A$ и $B$ электрической цепи подключены последовательно резисторы $R_1 = 10\units{Ом}$ и $R_2 = 20\units{Ом}$ и параллельно им $R_3 = 30\units{Ом}.$ Найдите эквивалентное сопротивление $R_{AB}$ этого участка цепи.

\task 2
Электрическая цепь состоит из последовательности $N$ одинаковых звеньев, в которых каждый резистор имеет сопротивление $r$. Последнее звено замкнуто резистором сопротивлением $R$. При каком соотношении $\cfrac{R}{r}$ сопротивление цепи не зависит от числа звеньев?

\task 3
Для измерения сопротивления $R$ проводника собрана электрическая цепь. Вольтметр $V$ показывает напряжение $U_V = 5\units{В},$ показание амперметра $A$ равно $I_A = 25\units{мА}.$ Найдите величину $R$ сопротивления проводника. Внутреннее сопротивление вольтметра $R_V = 1{,}0\units{кОм},$ внутреннее сопротивление амперметра $R_A = 2{,}0\units{Ом}.$

\task 4
Шкала гальванометра имеет $N=100$ делений, цена деления $\delta = 1\units{мкА}$. Внутреннее сопротивление гальванометра $R_G = 1{,}0\units{кОм}.$ Как из этого прибора сделать вольтметр для измерения напряжений до $U = 100\units{В}$ или амперметр для измерения токов силой до $I = 1\units{А}?$


% \begin{flushright}
\textsc{ГБОУ школа №554, 20 ноября 2018\,г.}
\end{flushright}

\begin{center}
\LARGE \textsc{Математический бой, 8 класс}
\end{center}

\problem{1} Есть тридцать карточек, на каждой написано по одному числу: на десяти карточках~–~$a$,  на десяти других~–~$b$ и на десяти оставшихся~–~$c$ (числа  различны). Известно, что к любым пяти карточкам можно подобрать ещё пять так, что сумма чисел на этих десяти карточках будет равна нулю. Докажите, что~одно из~чисел~$a, b, c$ равно нулю.

\problem{2} Вокруг стола стола пустили пакет с орешками. Первый взял один орешек, второй — 2, третий — 3 и так далее: каждый следующий брал на 1 орешек больше. Известно, что на втором круге было взято в сумме на 100 орешков больше, чем на первом. Сколько человек сидело за столом?

% \problem{2} Натуральное число разрешено увеличить на любое целое число процентов от 1 до 100, если при этом получаем натуральное число. Найдите наименьшее натуральное число, которое нельзя при помощи таких операций получить из~числа 1.

% \problem{3} Найти сумму $1^2 - 2^2 + 3^2 - 4^2 + 5^2 + \ldots - 2018^2$.

\problem{3} В кружке рукоделия, где занимается Валя, более 93\% участников~—~девочки. Какое наименьшее число участников может быть в таком кружке?

\problem{4} Произведение 2018 целых чисел равно 1. Может ли их сумма оказаться равной~0?

% \problem{4} Можно ли все натуральные числа от~1 до~9 записать в~клетки таблицы~$3\times3$ так, чтобы сумма в~любых двух соседних (по~вертикали или горизонтали) клетках равнялось простому числу?

\problem{5} На доске написано 2018 нулей и 2019 единиц. Женя стирает 2 числа и, если они были одинаковы, дописывает к оставшимся один ноль, а~если разные — единицу. Потом Женя повторяет эту операцию снова, потом ещё и~так далее. В~результате на~доске останется только одно число. Что это за~число?

\problem{6} Докажите, что в~любой компании людей найдутся 2~человека, имеющие равное число знакомых в этой компании (если $A$~знаком с~$B$, то~и $B$~знаком с~$A$).

\problem{7} Три колокола начинают бить одновременно. Интервалы между ударами колоколов соответственно составляют $\cfrac43$~секунды, $\cfrac53$~секунды и $2$~секунды. Совпавшие по времени удары воспринимаются за~один. Сколько ударов будет услышано за 1~минуту, включая первый и последний удары?

\problem{8} Восемь одинаковых момент расположены по кругу. Известно, что три из~них~— фальшивые, и они расположены рядом друг с~другом. Вес фальшивой монеты отличается от~веса настоящей. Все фальшивые монеты весят одинаково, но неизвестно, тяжелее или легче фальшивая монета настоящей. Покажите, что за~3~взвешивания на~чашечных весах без~гирь можно определить все фальшивые монеты.

\end{document}

\narrow

\renewcommand{\theenumi}{\asbuk{enumi}}
\renewcommand{\labelenumi}{\asbuk{enumi})}

\begin{document}
\begin{flushright}
\textsc{Летняя научная школа \\ ГБОУ школа №554 \\ 20~июня 2020\,г.}
\end{flushright}

\begin{center}
\LARGE \textsc{Математический марафон}
\LARGE \textsc{5–7~классы}
\end{center}

Правила участия: \href{https://www.notion.so/Math-Marathon-8acf3ff3b2874cefabbfa78d2db4f07e}{https://notion.so/Math-Marathon-8acf3ff3b2874cefabbfa78d2db4f07e}.

% Добрынин
\problem{1.1}
Найдите значение выражение $6x - 8y$ при $x=\frac 23$, $y=\frac 58$.
\answer{-1}

\problem{1.2}
Сравните значения выражений $-0{,}8x - 1$ и $0{,}8x - 1$ при $x=6$.
\answer{левое меньше}

\problem{1.3}
Упростите варажения:
\begin{enumerate}
    \item $2x - 3y - 11x + 8y,$
    \item $5(2a+ 1) - 3,$
    \item $14x - (x-1) + (2x+6).$
\end{enumerate}
\answer{$5y - 9x$, $10 a + 2$, $15 x  + 7$}

\problem{1.4}
Упростите выражение и~найдите его~значение: 
$$-4(2{,}5a - 1{,}5) + 5{,}5a - 8, \text{при $a=-\frac 29$.}$$
\answer{-1}

\problem{1.5}
Из~двух городов, расстояние между которым $s\units{км}$, одновременно навстречу друг другу выехали 
легковой автомобиль и грузовик и~встретились через $t\units{ч}$. 
Скорость легкового автомобиля $v\fracunits{км}{}{ч}{}$. Найдите скорость грузовика. 
Ответьте на~вопрос задачи, если $s=200, t=2, v=60$.
\answer{40 км/ч}

\problem{1.6}
Раскройте скобки: $3x - (5x - (3x - 1))$.
\answer{$x - 1$}

\problem{1.7}
Решите уравнение:
\begin{enumerate}
    \item $\frac 12 x = 12,$
    \item $6x - 10{,}2 = 0,$
    \item $5x - 4{,}5 = 3x + 2{,}5,$
    \item $2x - (6x - 5) = 45.$
\end{enumerate}
\answer{24; 1{,}7; 3{,}5; -10}

\problem{1.8}
Таня в~школу сначала едет на~автобусе, а~потом идёт пешком. Вся~дорога у~неё занимает 26~мин. Идёт она~на 6~мин~дольше,
чем~едет на~автобусе. Сколько минут она~едет на~автобусе?
\answer{10 минут}

\problem{1.9}
В~двух сараях сложено сено, причём в~первом сарае сена в 3~раза больше, чем~во втором. 
После того как~из первого сарая увезли 20~т~сена, а~во второй привезли 10~т, 
а~обоих сараях сена стало поровну. Сколько всего тонн сеня было в~двух сараях первоначально?
\answer{60 т}

\problem{1.10}
Решите уравнение $7x - (x+3) = 3(2x - 1).$
\answer{x - любое число}

\problem{1.11}
Функция задана формулой $y = 6x + 19$. Определите
\begin{enumerate}
    \item значение $y$, если $x=0{,}5$;
    \item значение $x$, при~котором $y = 1$;
    \item проходит ли~график функции через точку $A(-2;7)$.
\end{enumerate}
\answer{y=22, x=-3, проходит}


\problem{1.12}
Используя график функции $y=2x-4$, опредите значение $y$ при $x=1{,}5$.
\answer{-1}
% skip graph

\problem{1.14}
Найдите координаты точки пересечения графиков функций $y = 47x - 37$ и $y=-13 x + 23$.
\answer{(1; 10)}

\problem{1.14}
Задайте формулой линейную функцию, график которой параллелен прямой $y=3x-7$ и~проходит через начало координат.
\answer{y=3x}

% K-4
\problem{1.15}
Найдите значение выражения $x - 5x^2$ при $x = -4$.
\answer{-84}

\problem{1.16}
Выполните действия:
\begin{enumerate}
    \item $y^7\cdot y^{12}$;
    \item $y^{20} : y^5$;
    \item $\cbr{y^{2}}^8$;
    \item $\cbr{2y}^4$.
\end{enumerate}
\answer{$y^{19}; y^{15}; y^{16}; 16y^4$}

\problem{1.17}
Упростите выражение
\begin{enumerate}
    \item $-2ab^3 \cdot 3a^2 \cdot b^4$;
    \item $\cbr{-2a^5b^2}^3$.
\end{enumerate}
\answer{$-6a^3b^7; -8a^{15}b^6$}

\problem{1.18}
Используя график функции $y=x^2$, определите значение $y$ при $x=1{,}5$ и $x=-1{,}5$.
\answer{2{,}25}

\problem{1.19}
Вычислите $\frac{25^2\cdot 5^5}{5^7}.$
\answer{25}

\problem{1.20}
Упростите выражение:
\begin{enumerate}
    \item $2\frac 23 x^2y^8\cdot\cbr{-1\frac 12 xy^3}^4$;
    \item $x^{n-2} \cdot x^{3-n} \cdot x$.
\end{enumerate}
\answer{$13{,}5x^6y^{20}; x^2$}

% K-5
\problem{1.21}
Выполните действия 
\begin{enumerate}
    \item $(3a - 4ax + 2) - (11a - 14 ax)$;
    \item $3y^2\cbr{y^3 + 1}$.
\end{enumerate}
\answer{$10ax - 8a + 2; 3y^5 + 3y^2$}

\problem{1.22}
Вынесите общий множитель за~скобки
\begin{enumerate}
    \item $10ab - 15b^2$;
    \item $18a^3 + 6a^2$.
\end{enumerate}
\answer{$5b(2a - 3b); 6a^2(3a + 1)$}

\problem{1.23}
Решите уравнение $9x - 6(x-1) = 5(x+2).$
\answer{-2}

\problem{1.24}
Пассажирский поезд за 4~ч~прошёл такое же~расстояние, какое товарный за 6~ч. 
Найдите скорость пассажирского поезда, если известно, что~скорость товарного на $20\fracunits{км}{}{ч}{}$ меньше.
\answer{60 км/ч}

\problem{1.25}
Решите уравнение $\frac{3x-1}6 - \frac x3 = \frac{5-x}9.$
\answer{2{,}6}

\problem{1.26}
Упростите выражение 
$$
2a(a+b-c) - 2b(a - b - c) + 2c(a-b+c).
$$
\answer{$2a^2 + 2b^2 + 2c^2$}

% K-6
\problem{1.27}
Выполните умножение:
\begin{enumerate}
    \item $(c+2)(c-3)$;
    \item $(2a-1)(3a+4)$;
    \item $(5x-2)(4x-y)$;
    \item $(a-2)(a^2 - 3a + 6)$.
\end{enumerate}
\answer{$c^2 - c -6; 6a^2 + 5a - 4; 20x^2 - 8x - 5xy + 2y; a^3 - 5a^2 + 12 a - 12$}

\problem{1.28}
Разложите на~множители:
\begin{enumerate}
    \item $a(a+3) - 2(a+3)$;
    \item $ax - ay + 5x - 5y$.
\end{enumerate}
\answer{$(a-2)(a+3); (a+5)(x-y)$}

\problem{1.29}
Упростите выражение: $-0{,}1x(2x^2 + 6)(5-4x^2)$.
\answer{$0{,}8x^5 + 1{,}4x^3 - 3x$}

\problem{1.30}
Представьте многочлен в~виде произведения:
\begin{enumerate}
    \item $x^2 - xy - 4x + 4y$;
    \item $ab - ac - bx + cx + c - b$.
\end{enumerate}
\answer{$(x-y)(x-4); (b-c)(a+1-x)$}

\problem{1.31}
Из~прямоугольного листа фанеры вырезали квадратную пластинку, для~чего с~одной стороны листа фанеры 
отрезали полосу шириной 2~см, а~с другой, соседней, — 3~см. Найдите сторону получишегося квадрата, если известно, 
что~его площадь на $51\dunits{см}2$ меньше площади прямоугольника.
\answer{9 см}

% Мария Сергеевна
\problem{2.1}
Кто-то~принес букет цветов. Ребята стали гадать и~высказали предположения, что~букет принесли:
\renewcommand{\labelenumi}{\arabic{enumi})}
\begin{enumerate}
    \item Андрей или~Борис;
    \item Андрей или~Сергей;
    \item Борис или~Даша;
    \item Борис или~Володя;
    \item Володя или~Галя;
    \item Галя или~Даша.
\end{enumerate}
Учительница сказала, что~в одном из~этих предположений одно имя~названо правильно, а~второе – неправильно. 
Во~всех же~остальных предположениях оба~имени названы неправильно. 
Кто~принес цветы? (В~ответе запишите имя)
\answer{Сергей}

\problem{2.2}
Перед судом стоят три~человека, из~которых только один может быть преступником. 
Известно, что~преступник, отвечая на~вопросы, всегда лжет. А~те, кто не~совершал преступления, всегда говорят правду. 
Получив ответ одного из~подозреваемых на~вопрос «Виновны ли~вы?», 
судья задал двум другим один и~тот же~вопрос «Прав ли~первый?» и~получил следующие ответы.
Второй подозреваемый ответил: «Первый прав».
Третий подозреваемый ответил: «Первый солгал».
Кто~же преступник? (В~ответе запишите номер подозреваемого цифрой)
\answer{3}

\problem{2.3}
Пять девочек – Света, Таня, Аня, Лена и~Юля – писали контрольную работу. 
Три~девочки написали на~«5», две~девочки – на~«4». Известно, что у~Светы с~Таней разные оценки, 
у~Светы с~Аней – разные, у~Светы с~Леной – разные. 
Какую оценку получила Аня? (В~ответе укажите цифру)
\answer{5}

\problem{2.4}
Три~дочери писательницы – Джуди, Айрис и~Линда — тоже очень талантливы. 
Они~приобрели известность в~разных видах искусств: пении, балете и~кино. 
Все~они живут в~разных городах, поэтому мама часто звонит им в~Париж, Рим и~Чикаго. 
Известно, что:
\renewcommand{\labelenumi}{\arabic{enumi})}
\begin{enumerate}
    \item Джуди живет не~в Париже, а~Линда — не~в Риме;
    \item Парижанка не~снимается в~кино;
    \item Та, кто~живет в~Риме, певица;
    \item Линда равнодушна к~балету.
\end{enumerate}
Где~живет Айрис и~какова ее~профессия? (В~ответе укажите город и~название профессии через «;»)
\answer{Париж; балерина}

\problem{2.5}
Витя, Саша, Вика и~Толя собирали модели самолетов. Двое делали модели из~белой пластмассы, а~двое – из~серой. 
Саша и~Вика, Саша и~Толя делали модели из~пластмассы разного цвета. 
Получились три~модели самолетов фирмы «Ту»~и одна – фирмы «Су». 
Какую модель из~какого материала делал Витя, если Саша делал модель фирмы «Су»~из белой пластмассы? 
(В~ответ укажите модель фирмы без~кавычек и~цвет пластмасса через «;»)
\answer{Ту; белый}

\problem{2.6}
Студенты Васильев, Алексеев, Дмитриева и~Сергеев обучаются на~физическом, математическом, 
филологическом и~искусствоведческом факультетах в~четырехэтажном здании университета. 
Согласно расписанию, им~нужно посетить занятия по~истории, литературе, иностранному языку и~черчению. 
Все~эти занятия проводятся в~аудиториях, расположенных на~разных этажах.
Известно, что:
\renewcommand{\labelenumi}{\arabic{enumi})}
\begin{enumerate}
    \item Занятия по~иностранному языку проводятся этажом выше занятий по~истории, а~занятия по~литературе – этажом выше занятий по~черчению;
    \item Студентка Дмитриева не~занимается ни~литературой, ни~иностранным языком;
    \item На~физическом факультете изучают черчение;
    \item Студент Сергеев занимается на 1-м~этаже;
    \item На 3-м~этаже работает искусствоведческий факультет;
    \item Дмитриева обучается на~более высоком этаже, чем~Алексеев, но~более низком, чем~Васильев;
    \item Васильев не~учится на~математическом факультете.
\end{enumerate}
Где~учится, какие занятия и~на каком этаже посещает Сергеев? 
(В~ответе записать название факультета, этаж цифрой и~предмет через «;»)
\answer{физический; 1; черчение}

\problem{2.7}
Пять друзей, Варя, Валя, Саша, Егор, Женя, собрались играть в~приставку. 
Всего есть четыре джойстика, поэтому, когда четверо играют, один следит за~ходом игры. 
Ребята постоянно менялись, и~вечером посчитали, что~Варя играла 7~раз, Валя — 8~раз, Саша — 6~раз, Егор — 5~раз, Женя — 6~раз. 
Сколько раз~Егор следил за~ходом игры? 
\answer{3}

\problem{2.8}
Возраст нескольких друзей в~сумме составляет 29~лет. Через три~года он~будет равен 41~году. Сколько этих друзей?
\answer{4}

\problem{2.9}
Три~друга – Алеша, Коля и~Саша – сели на~скамейку в~один ряд. Сколькими способами они~могли это~сделать?
\answer{6}

\problem{2.10}
Представьте число 111 с~помощью четырех двоек и~арифметических операций. (В~ответе запишите получившееся выражение)
\answer{222:2}

\problem{2.11}
Шестеро друзей в~ожидании электрички заскочили в~буфет, в~котором:
\renewcommand{\labelenumi}{\arabic{enumi})}
\begin{enumerate}
    \item Маша купила то~же, что~Егор, и~вдобавок еще~бутерброд с~сыром;
    \item Аня~купила то~же, что~Саша, но~не стала покупать шоколадное печенье;
    \item Кирилл ел~то же, что~Наташа, но~без луковых чипсов;
    \item Егор завтракал тем~же, что~и Наташа, но~бутерброду с~котлетой предпочел картофельные чипсы;
    \item Саша ела~то же, что~Наташа, но~вместо молочного коктейля пила лимонад.
\end{enumerate}
Из~чего состоял завтрак Ани? (В~ответе укажите наименования блюд через «;»)
\answer{бутерброд с~котлетой; луковые чипсы; лимонад}

\problem{2.12}
Отец с~двумя сыновьями отправился в~поход. На~их пути встретилась река, у~берега которой находится плот. 
Он~выдерживает на~воде или~отца, или~двух сыновей. Всем нужно переправиться на~другой берег. 
Сколько времени (минимально) потребуется на~переправу, если каждая поездка через реку занимает 3~минуты? 
Ответ дайте в~минутах.
\answer{15}

\problem{2.13}
Сколько слонов и~верблюдов в~зоопарке города С., если известно, что всего у~этих животных 22~уха, 
а~горбов в 9~раз~больше, чем~хоботов. (В~ответе укажите число верблюдов и~число слонов через «;»)
\answer{9; 2}

\problem{2.14}
У~Юры есть два~автомобиля, четыре медвежонка и~три мяча. Он~хочет выбрать из~этих игрушек один автомобиль, 
одного медвежонка и~один мяч. Сколько у~него есть вариантов выбора?
\answer{24}

\problem{2.15}
В~автобусе было несколько пассажиров. На~первой остановке вышло 7~и~вошло 4, а~на второй вышло 6~и~вошло 13~пассажиров. 
Сколько пассажиров было в~автобусе до~первой остановки, если после второй остановки их~стало 38?
\answer{34}

\problem{2.16}
Девочка купила 2~марки по 50~коп~и 3~открытки по 65~коп. Какую сдачу она~должна получить с~пяти рублей? 
Ответ дайте в~копейках.
\answer{205}

\problem{2.17}
Для~награждения победителей математической олимпиады купили 10~книг по 9~руб. 
и 12~комплектов головоломок – общая сумма составила 222~руб. 
Сколько стоит один комплект головоломок? Ответ укажите в~рублях.
\answer{11}

\problem{2.18}
Некто утверждает, что~знает 4~натуральных числа, произведение и~сумма которых нечетные числа. 
Не~ошибается ли~он? (В~ответе укажите «Прав»~или «Ошибается»)
\answer{Ошибается}

\problem{2.19}
Имеется 9~листов бумаги. Некоторые из~них разорвали на 3~или 5~частей. 
Некоторые из~образовавшихся частей разорвали на 3~или 5~частей и~так несколько раз. 
Можно ли~после нескольких таких операций получить 100~частей?
\answer{Нет}

\problem{2.20}
Приходит пастух с 70~быками. Его~спрашивают:

"--* Сколько приводишь ты~из своего многочисленного стада?

Пастух отвечает:

"--* Я~привожу две~трети от~трети скота. Сочти!

Сколько быков в~стаде?
\answer{315}

\problem{2.21}
Прохожий, догнавший другого, спросил: «Как~далеко до~деревни, которая у~нас впереди?» 
Ответил прохожий: «Расстояние от~той деревни, от~которой ты~идешь, равно третьей части всего расстояния 
между деревнями, а~если ты~пройдешь еще~две версты, то~будешь ровно посередине между деревнями». 
Сколько верст осталось идти первому прохожему?
\answer{8}

\problem{2.22}
Лодка проплыла некоторое расстояние по~озеру за 4~часа. Такое же~расстояние плот проплывает по~реке за 12~часов. 
Сколько времени затратит лодка на~тот же~путь по~течению реки?
\answer{3}

\problem{2.23}
Два~почтальона А~и В~находятся друг от~друга на~расстоянии 59~миль. Утром они~отправляются навстречу друг другу. 
А~проходит в 2~часа 7~миль, а~В — в 3~часа 8~миль, но~В выходит часом позднее, чем~А. 
Сколько миль пройдет А~до встречи с~В?
\answer{35}

\problem{2.24}
Некто купил 96~гусей. Половину гусей он~купил по 2~алтына и 7~полушек за~каждого гуся. За~каждого из~остальных 
гусей он~заплатил по 2~алтына без~полушки. Сколько стоит покупка? Ответ дайте в~копейках.
\answer{648}

\problem{2.25}
Первая и~вторая бригады могли бы~выполнить задание за 9~дней; вторая и~третья бригады — за 18~дней; 
первая и~третья — за 12~дней. За~сколько дней это~задание могут выполнить три~бригады, работая вместе?
\answer{8}

\problem{2.26}
Трое рабочих могут выполнить некоторую работу, при~этом А~может выполнить ее~один раз~за 3~недели, 
в~три раза за 8~недель, С~пять раз~за 12~недель. Спрашивается, за~какое время они~смогут выполнить эту~работу 
все~вместе. (Считайте, что~в неделе 6~рабочих дней по 12~ч.) Ответ дайте в~часах.
\answer{64}

\problem{2.27}
Из~верхнего угла комнаты вниз по~стене поползли две~мухи. Спустившись до~пола, они~поползли обратно. 
Первая муха ползла вниз и~вверх с~одинаковой скоростью, а~вторая муха хоть и~поднималась в~два раза медленнее, 
но~зато спускалась вдвое быстрее. Какая из~мух раньше приползет обратно? (В~ответе укажите номер мухи)
\answer{1}

\problem{2.28}
Выполните действия:
$$(24\,347\,420 : 8105 + 572\,580 : 180)\cdot 504 + 18\,999\,380:9223$$
\answer{3119300}

\problem{2.29}
Два~конькобежца одновременно стартовали на~дистанцию 10 000~м~по замкнутой дорожке, длина которой 400~м. 
Скорость первого конькобежца 20~км/ч, а~скорость второго конькобежца 21~км/ч. 
Обгонит ли~второй конькобежец первого на~два круга до~конца дистанции?
\answer{Нет}

\problem{2.30}
Все~четырехзначные числа, записанные цифрами 1, 2, 3, 4~без~повторения, занумеровали в~порядке возрастания чисел. 
Какой номер имеет число 4312?
\answer{23}

\problem{2.31}
Из 28~костей домино выбирают наугад одну кость. Какова вероятность выбрать кость с~суммой очков 6? 
(В~ответе запишите числитель и~знаменатель получившейся дроби через «;»)
\answer{1; 7}

\problem{2.32}
Цена товара в 100~условных единиц понизилась сначала на 10\%, потом еще~на 10\%. 
На~сколько процентов понизилась цена товара за 2~раза? (В~ответе запишите количество \%)
\answer{19}

\problem{2.33}
Вася возвел натуральное число в~квадрат и~получил число, оканчивающееся цифрой 2. Не~ошибся ли~Вася? 
(В~ответе укажите «Не~ошибся»~или «Ошибся»)
\answer{Ошибся}

\problem{2.34}
Ослица и~мул шли~вместе, нагруженные мешками равного веса. Ослица жаловалась на~тяжесть ноши. 
«Чего ты~жалуешься, — сказал мул, — если ты~мне дашь один твой мешок, моя~ноша будет вдвое больше твоей, 
а~если я~дам тебе один мешок, наши грузы только сравняются». Сколько мешков было у~каждого? 
(В~ответе запишите в~порядке возрастания)
\answer{5; 7}

\problem{2.35}
Двое очистили 400~штук картофеля: один очищал 3~штуки в~минуту, второй — 2. 
Второй работал на 25~мин~больше, чем~первый. Сколько времени работал второй? Ответ дайте в~минутах.
\answer{95}

\problem{2.36}
В~классе мальчиков и~девочек поровну. На~встречу пришла половина всех мальчиков и~еще 3~мальчика, 
реть всех девочек и~еще 6~девочек. Оказалось, что~на встречу пришло мальчиков и~девочек поровну. 
Сколько всего учащихся в~классе?
\answer{36}

\problem{2.37}
Тракторист может вспахать поле за 5~дней. Увеличив выработку на 2{,}5~га~в день, он~выполнил работу за 4~дня. 
Какова площадь поля? Ответ дайте в~гектарах.
\answer{50}

\problem{2.38}
Косцы должны выкосить два~луга. Начав с~утра косить большой луг, они~после полудня разделились: 
одна половина осталась на~первом луге и~к вечеру его~докосила, а~другая – перешла косить на~второй луг, 
площадью вдвое меньше первого. Сколько было косцов, если известно, 
что~в течение следующего дня~оставшуюся часть работы выполнил один косец?
\answer{8}


\problem{2.39}
Первый турист может пройти расстояние между городами за 4~ч, а~второй – за 6~ч. 
Как-то~раз они~вышли одновременно из~этих городов навстречу друг другу. 
Хватит ли~им 2{,}5~ч~на движение до~встречи? 
(В~ответе запишите «Хватит»~или «Не~хватит»)
\answer{Хватит}

\problem{2.40}
Переписка доклада поручена двум машинисткам. Более опытная из~них могла бы~выполнить всю~работу в 2~ч, 
менее опытная – в 3~ч. Во~сколько времени перепишут они~доклад, если разделят между собой работу так, 
чтобы выполнить ее~в кратчайший срок? Ответ дайте в~часах.
\answer{1{,}2}


% Шелаков-1
\problem{3.1}
Придумайте трехзначное число, у~которого с~любым из~чисел 914, 213~и 973~совпадает один из~разрядов, 
а~два других не~совпадают.
\answer{274}

\problem{3.2}
Разбейте число 150~на~три не~равных друг другу натуральных слагаемых, сумма любых двух из~которых делится на~третье. 
\answer{25 + 50 + 75}

\problem{3.3}
У~Васи, Пети, Бори, Миши и~Вани было поровну мандаринов. Вася, Петя и~Боря съели по 3~мандарина каждый. 
В~итоге у~них вместе оказалось столько же~мандаринов, сколько у~Миши и~Вани вместе. Сколько мандаринов было у~Пети? 
\answer{9}

\problem{3.4}
От~бревна длиной 5~метров отпиливают кусок длины 1~метр за 45~секунд. За~какое время распилят все~бревно? 
\answer{3~минуты (или 180~секунд)}

\problem{3.5}
На~двух полках 37~книг. На~верхней на 3~книги больше, чем~на нижней. Сколько книг на~верхней полке?  
\answer{20}

\problem{3.6}
В~ящике лежат 17~зеленых и 17~синих шари¬ков. Сколько ша¬риков нужно вытащить из~коробки, 
чтобы среди них~обязательно нашлись два~шарика одного цвета? 
\answer{3}

\problem{3.7}
Какое наименьшее количество карандашей надо взять в~темноте из~коробки с 7~красными и 12~синими карандашами, 
чтобы среди взятых было гарантированно не~меньше трех красных и~не меньше пяти синих? 
Из-за~темноты, цвет карандашей не~виден. 
\answer{15}

\problem{3.8}
Какое наименьшее количество ботинок нужно взять из~ящика, не~заглядывая в~него, 
если в~нем лежат 8~пар~чёрных и 8~пар~синих ботинок и~требуется составить пару черного цвета. 
Ботинки, в~отличие от~носков, бывают левыми и~правыми, из~ящика они~вытаскиваются по~одному
\answer{25}

\problem{3.9}
Летела стая одноголовых сороконожек и~трехглавых дра¬конов. Вместе у~них: 48~голов и 296~ног. 
Сколько ног~у каждого из~таких драконов?
\answer{4}

\problem{3.10}
Сумма пяти подряд идущих целых чисел равна 795. Най¬дите эти~числа.
\answer{157, 158, 159, 160, 161}

\problem{3.11}
Пете 28~лет, а~Вове 4~года. Через сколько лет~Петя будет втрое старше Вовы ?
\answer{8}

\problem{3.12}
Моему брату через 4~года будет втрое больше лет, чем~ему было 4~года назад, 
а~моя сестра через 4~года будет вдвое старше, чем 4~года назад. Кто~из них~старше и~на сколько?
\answer{сестра на 4~года~старше}

\problem{3.13}
Мама положила на~стол сливы и~сказала детям, чтобы они, вернувшись из~школы, разделили их~поровну. 
Первой пришла Аня, взяла треть слив и~ушла. Потом вернулся из~школы Боря, взял треть оставшихся слив и~ушел. 
Затем пришел Витя и~взял 4~сливы — треть от~числа слив, которые он~увидел. Сколько слив оставила мама?
 \answer{27}

\problem{3.14}
Малыш съедает 1200~г~варенья за 8~мин. Карлсон делает это~вдвое быстрее. 
За~сколько минут они~вместе съедят 2~кг 700~г~варенья? 
\answer{6~минут}

\problem{3.15}
На~уборке снега работают две, машины. Одна из~них может убрать всю~улицу за 2~ч, а~другая — за 1~час 30~мин. 
Начав уборку одновременно, обе~машины проработали вместе 40~мин, после чего первая машина сломалась. 
Сколько надо времени, чтобы вторая машина закончила работу? 
\answer{20~минут}

\problem{3.16}
Через кран вода заполняет бак~за 6~ч, а~через сливное отверстие вся~вода из~бака выливается за 10~ч. 
За~какое время вода заполнит бак~при открытых кране и~отверстии? 
(Считайте, что~скорость вытекания воды из~бака не~зависит от~его наполненности.) 
\answer{за 15~часов}

\problem{3.17}
Двое рабочих выполнили вместе некоторую работу за 24~дня. Если бы~сначала первый сделал половину работы, 
а~затем второй другую половину, то~вся работа была бы~выполнена за 50~дней. 
За~какое время мог~выполнить эту~работу каждый в~отдельности? 
\answer{первый за 40~дней, второй — за 60~дней (или наоборот)}

\problem{3.18}
Бутылка и~стакан весят столько же, сколько кувшин. Бутылка весит столько же, сколько стакан и~тарелка. 
Два~кувшина весят столько же, сколько три~тарелки. Сколько стаканов уравновешивают одну бутылку? 
\answer{5}

\problem{3.19}
Четверо товарищей покупают лодку. Первый вносит половину суммы, вносимой остальными, второй — треть суммы, 
вносимой остальными, третий — четверть суммы, вносимой остальными, а~четвертый — 3900~р. Сколько стоит лодка? 
\answer{18000~рублей}

\problem{3.20}
На~остров рыцарей и~лжецов приехал путешественник и~нанял себе проводника. 
Каждый рыцарь всегда говорит правду, каждый лжец всегда врет. 
Однажды, увидев вдали туземца, путешественник сказал проводнику: «Пойди и~спроси у~того человека: рыцарь он~или лжец». 
Вскоре проводник вернулся и~сказал: «Этот человек сказал, что~он лжец». Кем~был проводник, рыцарем или~лжецом?
\answer{лжецом}

\problem{3.21}
Один из~попугаев всегда говорит правду, другой всегда врет, а~третий — хитрец — иногда говорит правду, иногда врет. 
На~вопрос: «Кто~Гоша?» — они~ответили:
\begin{itemize}
    \item Рома: — Он — лжец.
    \item Гоша: — Я~хитрец!
    \item Кеша: — Он — абсолютно честный попугай.
\end{itemize}
Кем~является Гоша?
\answer{лжецом}

\problem{3.22}
На~острове рыцарей и~лжецов вы~встретили двух друзей и~спросили: «Кто~вы?». Первый сказал: «Мы~оба лжецы». 
Определите, кем~является его~друг. Каждый рыцарь всегда говорит правду, каждый лжец всегда врет.
\answer{рыцарем}

\problem{3.23}
На~острове рыцарей и~лжецов встретилось трое человек и~каждый заявил всем остальным: «Вы~все — лжецы». 
Сколько рыцарей было среди них?
\answer{1}

\problem{3.24}
За~круглым столом сидят 15~человек: рыцари и~лжецы. Рыцарь всегда говорит правду, лжец всегда врет. 
Каждый из~них произнес такую фразу: «Мои~соседи — лжец и~рыцарь». 
Сколько за~столом рыцарей, если известно, что~одновременно все 15~не~могут быть лжецами?
\answer{10}

\problem{3.25}
В~корзине лежат 30~грибов. Среди любых 8~из~них имеется хотя бы~один рыжик, 
а~среди любых 24~грибов — хотя бы~один груздь. Сколько рыжиков в~корзине? 
\answer{23}

\problem{3.26}
В~забеге участвовали 3~бегуна: Иванов, Петров и~Сидоров. Перед забегом 4~болельщика дали такие 4~прогноза: 
\begin{itemize}
    \item Победит Петров,
    \item Иванов обгонит Сидорова,
    \item Сидоров финиширует следующим после Петрова,
    \item Иванов не~победит.  
\end{itemize}
После забега оказалось, что~среди прогнозов было чётное число верных. В~каком порядке финишировали бегуны? 
\answer{1 - Иванов, 2 - Петров, 3 - Сидоров}

% Шелаков-2
\problem{4.1}
Сколькими способами можно поставить на~шахматную доску белую и~черную ладьи так, чтобы они~не били друг друга? 
\answer{$64 \cdot 49 = 3136$}

\problem{4.2}
Сколько трехзначных чисел можно составить, используя только цифры 1, 2, 4, 7, 9, 
если цифры в~записи числа могут повторяться (например, 714, 277 и~т.~д.)?
\answer{125}

\problem{4.3}
Сколько разных чисел можно получить, переставляя цифры числа 3751. 
\answer{24}

\problem{4.4}
Сколько разных чисел можно получить, переставляя цифры числа 75170. 
\answer{48}

\problem{4.5}
Сколько существует семизначных чисел, цифры которых расположены в~порядке убывания, 
если могут использоваться только цифры от 0~до 7? 
\answer{8}

\problem{4.6}
Сколько имеется четырехзначных чисел, в~запись которых входит ровно одна цифра 7? 
\answer{2673}

\problem{4.7}
Шириной прямоугольника назовем длину наименьшей из~его сторон (квадраты в~данной задаче не~рассматриваются). 
Сколькими способами можно вырезать из~шахматной доски $8 \times 8$~прямоугольник шириной 3? 
(Разрезы должны идти только по~границам клеток.) 
\answer{180}

% Жичина
\problem{5.1}
Один килограмм огурцов стоит 15~рублей. Мама купила 2~кг 400~г~огурцов. 
Сколько рублей сдачи она~должна получить со 100~рублей?
\answer{64}

\problem{5.2}
Андрей Петрович купил американский автомобиль, на~спидометре которого скорость измеряется в~милях в~час. 
Американская миля равна 1609~м. Какова скорость автомобиля в~километрах в~час, 
если спидометр показывает 42~мили в~час? Ответ округлите до~целого числа.
\answer{68}

\problem{5.3}
1~киловатт-час~электроэнергии стоит 3~рубля 60~копеек. Счетчик электроэнергии 1~ноября показывал 32\,544~киловаттчаса, 
а 1~декабря — 32\,726~киловаттчасов. Сколько рублей нужно заплатить за~электроэнергию за~ноябрь?
\answer{655{,}2}

\problem{5.4}
Пакет сока стоит 32~рубля. Какое наибольшее количество пакетов сока можно купить на~200~рублей?
\answer{6}

\problem{5.5}
В~пачке бумаги 500~листов. За~неделю в~офисе расходуется 1200~листов. 
Какое наименьшее количество пачек бумаги нужно купить в~офис на 8~недель?
\answer{20}

\problem{5.6}
В~супермаркете проходит рекламная акция: заплатив за~две шоколадки, покупатель получает три~шоколадки 
(одна шоколадка в~подарок). Шоколадка стоит 35~рублей. Какое наибольшее число шоколадок можно получить на 200~рублей?
\answer{7}

\problem{5.7}
Рубашка стоит 450~рублей. Во~время распродажи скидка на~все товары составляет 20\%. 
Сколько рублей стоит рубашка во~время распродажи?
\answer{360}

\problem{5.8}
В~сентябре 1~кг~слив стоил 60~рублей. В~октябре сливы подорожали на~25\%. 
Сколько рублей стоил 1~кг~слив после подорожания в~октябре?
\answer{75}

\problem{5.9}
Магазин делает пенсионерам скидку на~определенное количество процентов от~цены покупки. 
Пакет кефира стоит в~магазине 40~рублей. Пенсионер заплатил за~пакет кефира 38~рублей. 
Сколько процентов составляет скидка для~пенсионеров?
\answer{5}

\problem{5.10}
В~июне 1~кг~огурцов стоил 50~рублей. В~июле огурцы подешевели на~20\%, а~в августе ещё~на~50\%. 
Сколько рублей стоил 1~кг~огурцов после снижения цены в~августе?
\answer{20}

\problem{5.11}
В~городе $\mathbb{N}$ живет 300 000~жителей. Среди них 20\% детей и~подростков. 
Среди взрослых 35\% не~работает (пенсионеры, студенты, домохозяйки и~т. п.). 
Сколько взрослых жителей города работает?
\answer{156000}

\problem{5.12}
В~кафе действует следующее правило: на~ту часть заказа, которая превышает 1000~рублей, действует скидка 25\%. 
После игры в~футбол студенческая компания из 20~человек сделала в~кафе заказ на~3400~рублей. 
Все~платят поровну. Сколько рублей заплатит каждый?
\answer{140}

% Губина
\problem{6.1}
Полный бидон с~молоком весит 33~кг. Бидон, заполненный наполовину, весит 17~кг. Какова масса пустого бидона в~кг?
\answer{1}

\problem{6.2}
Улитка Даша, длиной 10~мм~и удав Саша, длиной 2{,}2~м, устроили соревнование по~скоростному ползанию. 
Кто~из участников финиширует раньше, если финиш регистрируется по~кончику хвоста? 
Скорость Даши $1\fracunits{см}{}{c}{}$, скорость Саши $40\fracunits{см}{}{c}{}$. Расстояние от~старта до~финиша 1~м.
В~ответ запишите имя.
\answer{Саша}

\problem{6.3}
Антилопа проскакала половину дистанции со~скоростью $v_1 = 10\fracunits{м}{}{c}{}$, 
другую половину – со~скоростью $v_2 = 15\fracunits{м}{}{c}{}$. Гепард половину времени, 
затраченного на~преодоление той~же дистанции, бежал со~скоростью $v_3 = 15\fracunits{м}{}{c}{}$, 
а~вторую половину времени – со~скоростью $v_4 = 10\fracunits{м}{}{c}{}$. Кто~финишировал раньше?
\answer{Гепард}

\problem{6.4}
Собственная скорость теплохода $27\fracunits{км}{}{ч}{}$, скорость течения реки $3\fracunits{км}{}{ч}{}$. 
Сколько часов затратит теплоход на~путь по~течению реки между двумя причалами, если расстояние между ними 120~км?
\answer{4}

\problem{6.5}
За 2~часа автомобиль проехал 96~км, а~велосипедист за 6~часов проехал 72~км. 
Во~сколько раз~автомобиль двигался быстрее велосипедиста?
\answer{4}

\problem{6.6}
Расстояние от~Перми до~Казани, равное 723~км, автомобиль проехал за 13~часов. 
Первые 9~часов он~ехал со~скоростью $55\fracunits{км}{}{ч}{}$. 
Определить скорость автомобиля в~оставшееся время.
\answer{57}

\problem{6.7}
Воробей гонится за~мухой, которая летит со~скоростью $150\fracunits{см}{}{с}{}$ на~расстоянии 12~м~впереди воробья. 
Какое расстояние в~метрах он~пролетит, чтобы поймать муху, если его~скорость $27\fracunits{км}{}{ч}{}$?
\answer{15}

\end{document}