\setdate{14~мая~2019}
\setclass{10«Т»}

\addpersonalvariant{Михаил Бурмистров}

\tasknumber{1}%
\task{%
    На резистор сопротивлением $r = 18\,\text{Ом}$ подали напряжение $V = 120\,\text{В}$.
    Определите ток, который потечёт через резистор, и мощность, выделяющуюся на нём.
}
\answer{%
    \begin{align*}
    \mathcal{I} &= \frac{ V }{ r } = \frac{ 120\,\text{В} }{ 18\,\text{Ом} } = 6{,}67\,\text{А},  \\
    P &= \frac{V^2}{ r } = \frac{ \sqr{ 120\,\text{В} } }{ 18\,\text{Ом} } = 800{,}00\,\text{Вт}
    \end{align*}
}
\solutionspace{120pt}

\tasknumber{2}%
\task{%
    Замкнутая электрическая цепь состоит из ЭДС $\mathcal{E} = 1\,\text{В}$ и сопротивлением $r$
    и резистора $R = 30\,\text{Ом}$.
    Определите ток, протекающий в цепи.
    Какая тепловая энергия выделится на резисторе за время
    $\tau = 10\,\text{с}$? Какая работа будет совершена ЭДС за это время? Каков знак этой работы? Чему равен КПД цепи? Вычислите значения для 2 случаев:
    $r=0$ и $r = 10\,\text{Ом}$.
}
\answer{%
    \begin{align*}
    \mathcal{I}_1 &= \frac{ \mathcal{E} }{ R } = \frac{ 1\,\text{В} }{ 30\,\text{Ом} } = 0{,}03\,\text{А},  \\
    \mathcal{I}_2 &= \frac{ \mathcal{E} }{R + r} = \frac{ 1\,\text{В} }{30\,\text{Ом} + 10\,\text{Ом}} = 0{,}03\,\text{А},  \\
    Q_1 &= \mathcal{I}_1^2R\tau = \sqr{\frac{ \mathcal{E} }{ R }} R \tau
            = \sqr{\frac{ 1\,\text{В} }{ 30\,\text{Ом} }} \cdot 30\,\text{Ом} \cdot 10\,\text{с} = 0{,}270\,\text{Дж},  \\
    Q_2 &= \mathcal{I}_2^2R\tau = \sqr{\frac{ \mathcal{E} }{R + r}} R \tau
            = \sqr{\frac{ 1\,\text{В} }{30\,\text{Ом} + 10\,\text{Ом}}} \cdot 30\,\text{Ом} \cdot 10\,\text{с} = 0{,}270\,\text{Дж},  \\
    A_1 &= \mathcal{I}_1\tau\mathcal{E} = \frac{ \mathcal{E} }{R} \tau \mathcal{E}
            = \frac{\mathcal{E}^2 \tau}{ R } = \frac{\sqr{ 1\,\text{В} } \cdot 10\,\text{с}}{ 30\,\text{Ом} }
            = 0{,}300\,\text{Дж}, \text{положительна},  \\
    A_2 &= \mathcal{I}_2\tau\mathcal{E} = \frac{ \mathcal{E} }{R + r} \tau \mathcal{E}
            = \frac{\mathcal{E}^2 \tau}{R + r} = \frac{\sqr{ 1\,\text{В} } \cdot 10\,\text{с}}{30\,\text{Ом} + 10\,\text{Ом}}
            = 0{,}300\,\text{Дж}, \text{положительна},  \\
    \eta_1 &= \frac{ Q_1 }{ A_1 } = \ldots = \frac{ R }{ R } = 1,  \\
    \eta_2 &= \frac{ Q_2 }{ A_2 } = \ldots = \frac{ R }{R + r} = 0{,}90
    \end{align*}
}
\solutionspace{120pt}

\tasknumber{3}%
\task{%
    Лампочки, сопротивления которых $R_1 = 3{,}00\,\text{Ом}$ и $R_2 = 48{,}00\,\text{Ом}$, поочерёдно подключённные к некоторому источнику тока,
    потребляют одинаковую мощность.
    Найти внутреннее сопротивление источника и КПД цепи в каждом случае.
}
\answer{%
    \begin{align*}
        P_1 &= \sqr{\frac{ \mathcal{E} }{R_1 + r}}R_1,
        P_2  = \sqr{\frac{ \mathcal{E} }{R_2 + r}}R_2,
        P_1 = P_2 \implies  \\
        &\implies R_1 \sqr{R_2 + r} = R_2 \sqr{R_1 + r} \implies  \\
        &\implies R_1 R_2^2 + 2 R_1 R_2 r + R_1 r^2 =
                    R_2 R_1^2 + 2 R_2 R_1 r + R_2 r^2  \implies  \\
    &\implies r^2 (R_2 - R_1) = R_2^2 R_2 - R_1^2 R_2 \implies  \\
    &\implies r
            = \sqrt{R_1 R_2 \frac{R_2 - R_1}{R_2 - R_1}}
            = \sqrt{R_1 R_2}
            = \sqrt{3{,}00\,\text{Ом} \cdot 48{,}00\,\text{Ом}}
            = 12{,}00\,\text{Ом}.
            \\
    \eta_1
            &= \frac{ R_1 }{R_1 + r}
            = \frac{\sqrt{ R_1 }}{\sqrt{ R_1 } + \sqrt{ R_2 }}
            = 0{,}200,  \\
    \eta_2
            &= \frac{ R_2 }{R_2 + r}
            = \frac{ \sqrt{ R_2 } }{\sqrt{ R_2 } + \sqrt{ R_1 }}
            = 0{,}800
    \end{align*}
}
\solutionspace{120pt}

\tasknumber{4}%
\task{%
    Определите ток, протекающий через резистор $R = 10\,\text{Ом}$ и разность потенциалов на нём (см.
    рис.
    на доске),
    если $r_1 = 2\,\text{Ом}$, $r_2 = 1\,\text{Ом}$, $\mathcal{E}_1 = 30\,\text{В}$, $\mathcal{E}_2 = 20\,\text{В}$
}

\variantsplitter

\addpersonalvariant{Гагик Аракелян}

\tasknumber{1}%
\task{%
    На резистор сопротивлением $R = 30\,\text{Ом}$ подали напряжение $V = 240\,\text{В}$.
    Определите ток, который потечёт через резистор, и мощность, выделяющуюся на нём.
}
\answer{%
    \begin{align*}
    \mathcal{I} &= \frac{ V }{ R } = \frac{ 240\,\text{В} }{ 30\,\text{Ом} } = 8{,}00\,\text{А},  \\
    P &= \frac{V^2}{ R } = \frac{ \sqr{ 240\,\text{В} } }{ 30\,\text{Ом} } = 1920{,}00\,\text{Вт}
    \end{align*}
}
\solutionspace{120pt}

\tasknumber{2}%
\task{%
    Замкнутая электрическая цепь состоит из ЭДС $\mathcal{E} = 2\,\text{В}$ и сопротивлением $r$
    и резистора $R = 15\,\text{Ом}$.
    Определите ток, протекающий в цепи.
    Какая тепловая энергия выделится на резисторе за время
    $\tau = 5\,\text{с}$? Какая работа будет совершена ЭДС за это время? Каков знак этой работы? Чему равен КПД цепи? Вычислите значения для 2 случаев:
    $r=0$ и $r = 20\,\text{Ом}$.
}
\answer{%
    \begin{align*}
    \mathcal{I}_1 &= \frac{ \mathcal{E} }{ R } = \frac{ 2\,\text{В} }{ 15\,\text{Ом} } = 0{,}13\,\text{А},  \\
    \mathcal{I}_2 &= \frac{ \mathcal{E} }{R + r} = \frac{ 2\,\text{В} }{15\,\text{Ом} + 20\,\text{Ом}} = 0{,}06\,\text{А},  \\
    Q_1 &= \mathcal{I}_1^2R\tau = \sqr{\frac{ \mathcal{E} }{ R }} R \tau
            = \sqr{\frac{ 2\,\text{В} }{ 15\,\text{Ом} }} \cdot 15\,\text{Ом} \cdot 5\,\text{с} = 1{,}268\,\text{Дж},  \\
    Q_2 &= \mathcal{I}_2^2R\tau = \sqr{\frac{ \mathcal{E} }{R + r}} R \tau
            = \sqr{\frac{ 2\,\text{В} }{15\,\text{Ом} + 20\,\text{Ом}}} \cdot 15\,\text{Ом} \cdot 5\,\text{с} = 0{,}270\,\text{Дж},  \\
    A_1 &= \mathcal{I}_1\tau\mathcal{E} = \frac{ \mathcal{E} }{R} \tau \mathcal{E}
            = \frac{\mathcal{E}^2 \tau}{ R } = \frac{\sqr{ 2\,\text{В} } \cdot 5\,\text{с}}{ 15\,\text{Ом} }
            = 1{,}300\,\text{Дж}, \text{положительна},  \\
    A_2 &= \mathcal{I}_2\tau\mathcal{E} = \frac{ \mathcal{E} }{R + r} \tau \mathcal{E}
            = \frac{\mathcal{E}^2 \tau}{R + r} = \frac{\sqr{ 2\,\text{В} } \cdot 5\,\text{с}}{15\,\text{Ом} + 20\,\text{Ом}}
            = 0{,}600\,\text{Дж}, \text{положительна},  \\
    \eta_1 &= \frac{ Q_1 }{ A_1 } = \ldots = \frac{ R }{ R } = 1,  \\
    \eta_2 &= \frac{ Q_2 }{ A_2 } = \ldots = \frac{ R }{R + r} = 0{,}45
    \end{align*}
}
\solutionspace{120pt}

\tasknumber{3}%
\task{%
    Лампочки, сопротивления которых $R_1 = 1{,}00\,\text{Ом}$ и $R_2 = 4{,}00\,\text{Ом}$, поочерёдно подключённные к некоторому источнику тока,
    потребляют одинаковую мощность.
    Найти внутреннее сопротивление источника и КПД цепи в каждом случае.
}
\answer{%
    \begin{align*}
        P_1 &= \sqr{\frac{ \mathcal{E} }{R_1 + r}}R_1,
        P_2  = \sqr{\frac{ \mathcal{E} }{R_2 + r}}R_2,
        P_1 = P_2 \implies  \\
        &\implies R_1 \sqr{R_2 + r} = R_2 \sqr{R_1 + r} \implies  \\
        &\implies R_1 R_2^2 + 2 R_1 R_2 r + R_1 r^2 =
                    R_2 R_1^2 + 2 R_2 R_1 r + R_2 r^2  \implies  \\
    &\implies r^2 (R_2 - R_1) = R_2^2 R_2 - R_1^2 R_2 \implies  \\
    &\implies r
            = \sqrt{R_1 R_2 \frac{R_2 - R_1}{R_2 - R_1}}
            = \sqrt{R_1 R_2}
            = \sqrt{1{,}00\,\text{Ом} \cdot 4{,}00\,\text{Ом}}
            = 2{,}00\,\text{Ом}.
            \\
    \eta_1
            &= \frac{ R_1 }{R_1 + r}
            = \frac{\sqrt{ R_1 }}{\sqrt{ R_1 } + \sqrt{ R_2 }}
            = 0{,}333,  \\
    \eta_2
            &= \frac{ R_2 }{R_2 + r}
            = \frac{ \sqrt{ R_2 } }{\sqrt{ R_2 } + \sqrt{ R_1 }}
            = 0{,}667
    \end{align*}
}
\solutionspace{120pt}

\tasknumber{4}%
\task{%
    Определите ток, протекающий через резистор $R = 15\,\text{Ом}$ и разность потенциалов на нём (см.
    рис.
    на доске),
    если $r_1 = 1\,\text{Ом}$, $r_2 = 3\,\text{Ом}$, $\mathcal{E}_1 = 40\,\text{В}$, $\mathcal{E}_2 = 60\,\text{В}$
}

\variantsplitter

\addpersonalvariant{Ирен Аракелян}

\tasknumber{1}%
\task{%
    На резистор сопротивлением $r = 30\,\text{Ом}$ подали напряжение $V = 240\,\text{В}$.
    Определите ток, который потечёт через резистор, и мощность, выделяющуюся на нём.
}
\answer{%
    \begin{align*}
    \mathcal{I} &= \frac{ V }{ r } = \frac{ 240\,\text{В} }{ 30\,\text{Ом} } = 8{,}00\,\text{А},  \\
    P &= \frac{V^2}{ r } = \frac{ \sqr{ 240\,\text{В} } }{ 30\,\text{Ом} } = 1920{,}00\,\text{Вт}
    \end{align*}
}
\solutionspace{120pt}

\tasknumber{2}%
\task{%
    Замкнутая электрическая цепь состоит из ЭДС $\mathcal{E} = 2\,\text{В}$ и сопротивлением $r$
    и резистора $R = 15\,\text{Ом}$.
    Определите ток, протекающий в цепи.
    Какая тепловая энергия выделится на резисторе за время
    $\tau = 2\,\text{с}$? Какая работа будет совершена ЭДС за это время? Каков знак этой работы? Чему равен КПД цепи? Вычислите значения для 2 случаев:
    $r=0$ и $r = 30\,\text{Ом}$.
}
\answer{%
    \begin{align*}
    \mathcal{I}_1 &= \frac{ \mathcal{E} }{ R } = \frac{ 2\,\text{В} }{ 15\,\text{Ом} } = 0{,}13\,\text{А},  \\
    \mathcal{I}_2 &= \frac{ \mathcal{E} }{R + r} = \frac{ 2\,\text{В} }{15\,\text{Ом} + 30\,\text{Ом}} = 0{,}04\,\text{А},  \\
    Q_1 &= \mathcal{I}_1^2R\tau = \sqr{\frac{ \mathcal{E} }{ R }} R \tau
            = \sqr{\frac{ 2\,\text{В} }{ 15\,\text{Ом} }} \cdot 15\,\text{Ом} \cdot 2\,\text{с} = 0{,}507\,\text{Дж},  \\
    Q_2 &= \mathcal{I}_2^2R\tau = \sqr{\frac{ \mathcal{E} }{R + r}} R \tau
            = \sqr{\frac{ 2\,\text{В} }{15\,\text{Ом} + 30\,\text{Ом}}} \cdot 15\,\text{Ом} \cdot 2\,\text{с} = 0{,}048\,\text{Дж},  \\
    A_1 &= \mathcal{I}_1\tau\mathcal{E} = \frac{ \mathcal{E} }{R} \tau \mathcal{E}
            = \frac{\mathcal{E}^2 \tau}{ R } = \frac{\sqr{ 2\,\text{В} } \cdot 2\,\text{с}}{ 15\,\text{Ом} }
            = 0{,}520\,\text{Дж}, \text{положительна},  \\
    A_2 &= \mathcal{I}_2\tau\mathcal{E} = \frac{ \mathcal{E} }{R + r} \tau \mathcal{E}
            = \frac{\mathcal{E}^2 \tau}{R + r} = \frac{\sqr{ 2\,\text{В} } \cdot 2\,\text{с}}{15\,\text{Ом} + 30\,\text{Ом}}
            = 0{,}160\,\text{Дж}, \text{положительна},  \\
    \eta_1 &= \frac{ Q_1 }{ A_1 } = \ldots = \frac{ R }{ R } = 1,  \\
    \eta_2 &= \frac{ Q_2 }{ A_2 } = \ldots = \frac{ R }{R + r} = 0{,}30
    \end{align*}
}
\solutionspace{120pt}

\tasknumber{3}%
\task{%
    Лампочки, сопротивления которых $R_1 = 4{,}00\,\text{Ом}$ и $R_2 = 36{,}00\,\text{Ом}$, поочерёдно подключённные к некоторому источнику тока,
    потребляют одинаковую мощность.
    Найти внутреннее сопротивление источника и КПД цепи в каждом случае.
}
\answer{%
    \begin{align*}
        P_1 &= \sqr{\frac{ \mathcal{E} }{R_1 + r}}R_1,
        P_2  = \sqr{\frac{ \mathcal{E} }{R_2 + r}}R_2,
        P_1 = P_2 \implies  \\
        &\implies R_1 \sqr{R_2 + r} = R_2 \sqr{R_1 + r} \implies  \\
        &\implies R_1 R_2^2 + 2 R_1 R_2 r + R_1 r^2 =
                    R_2 R_1^2 + 2 R_2 R_1 r + R_2 r^2  \implies  \\
    &\implies r^2 (R_2 - R_1) = R_2^2 R_2 - R_1^2 R_2 \implies  \\
    &\implies r
            = \sqrt{R_1 R_2 \frac{R_2 - R_1}{R_2 - R_1}}
            = \sqrt{R_1 R_2}
            = \sqrt{4{,}00\,\text{Ом} \cdot 36{,}00\,\text{Ом}}
            = 12{,}00\,\text{Ом}.
            \\
    \eta_1
            &= \frac{ R_1 }{R_1 + r}
            = \frac{\sqrt{ R_1 }}{\sqrt{ R_1 } + \sqrt{ R_2 }}
            = 0{,}250,  \\
    \eta_2
            &= \frac{ R_2 }{R_2 + r}
            = \frac{ \sqrt{ R_2 } }{\sqrt{ R_2 } + \sqrt{ R_1 }}
            = 0{,}750
    \end{align*}
}
\solutionspace{120pt}

\tasknumber{4}%
\task{%
    Определите ток, протекающий через резистор $R = 10\,\text{Ом}$ и разность потенциалов на нём (см.
    рис.
    на доске),
    если $r_1 = 2\,\text{Ом}$, $r_2 = 2\,\text{Ом}$, $\mathcal{E}_1 = 20\,\text{В}$, $\mathcal{E}_2 = 20\,\text{В}$
}

\variantsplitter

\addpersonalvariant{Сабина Асадуллаева}

\tasknumber{1}%
\task{%
    На резистор сопротивлением $r = 18\,\text{Ом}$ подали напряжение $V = 150\,\text{В}$.
    Определите ток, который потечёт через резистор, и мощность, выделяющуюся на нём.
}
\answer{%
    \begin{align*}
    \mathcal{I} &= \frac{ V }{ r } = \frac{ 150\,\text{В} }{ 18\,\text{Ом} } = 8{,}33\,\text{А},  \\
    P &= \frac{V^2}{ r } = \frac{ \sqr{ 150\,\text{В} } }{ 18\,\text{Ом} } = 1250{,}00\,\text{Вт}
    \end{align*}
}
\solutionspace{120pt}

\tasknumber{2}%
\task{%
    Замкнутая электрическая цепь состоит из ЭДС $\mathcal{E} = 1\,\text{В}$ и сопротивлением $r$
    и резистора $R = 15\,\text{Ом}$.
    Определите ток, протекающий в цепи.
    Какая тепловая энергия выделится на резисторе за время
    $\tau = 5\,\text{с}$? Какая работа будет совершена ЭДС за это время? Каков знак этой работы? Чему равен КПД цепи? Вычислите значения для 2 случаев:
    $r=0$ и $r = 10\,\text{Ом}$.
}
\answer{%
    \begin{align*}
    \mathcal{I}_1 &= \frac{ \mathcal{E} }{ R } = \frac{ 1\,\text{В} }{ 15\,\text{Ом} } = 0{,}07\,\text{А},  \\
    \mathcal{I}_2 &= \frac{ \mathcal{E} }{R + r} = \frac{ 1\,\text{В} }{15\,\text{Ом} + 10\,\text{Ом}} = 0{,}04\,\text{А},  \\
    Q_1 &= \mathcal{I}_1^2R\tau = \sqr{\frac{ \mathcal{E} }{ R }} R \tau
            = \sqr{\frac{ 1\,\text{В} }{ 15\,\text{Ом} }} \cdot 15\,\text{Ом} \cdot 5\,\text{с} = 0{,}368\,\text{Дж},  \\
    Q_2 &= \mathcal{I}_2^2R\tau = \sqr{\frac{ \mathcal{E} }{R + r}} R \tau
            = \sqr{\frac{ 1\,\text{В} }{15\,\text{Ом} + 10\,\text{Ом}}} \cdot 15\,\text{Ом} \cdot 5\,\text{с} = 0{,}120\,\text{Дж},  \\
    A_1 &= \mathcal{I}_1\tau\mathcal{E} = \frac{ \mathcal{E} }{R} \tau \mathcal{E}
            = \frac{\mathcal{E}^2 \tau}{ R } = \frac{\sqr{ 1\,\text{В} } \cdot 5\,\text{с}}{ 15\,\text{Ом} }
            = 0{,}350\,\text{Дж}, \text{положительна},  \\
    A_2 &= \mathcal{I}_2\tau\mathcal{E} = \frac{ \mathcal{E} }{R + r} \tau \mathcal{E}
            = \frac{\mathcal{E}^2 \tau}{R + r} = \frac{\sqr{ 1\,\text{В} } \cdot 5\,\text{с}}{15\,\text{Ом} + 10\,\text{Ом}}
            = 0{,}200\,\text{Дж}, \text{положительна},  \\
    \eta_1 &= \frac{ Q_1 }{ A_1 } = \ldots = \frac{ R }{ R } = 1,  \\
    \eta_2 &= \frac{ Q_2 }{ A_2 } = \ldots = \frac{ R }{R + r} = 0{,}60
    \end{align*}
}
\solutionspace{120pt}

\tasknumber{3}%
\task{%
    Лампочки, сопротивления которых $R_1 = 5{,}00\,\text{Ом}$ и $R_2 = 80{,}00\,\text{Ом}$, поочерёдно подключённные к некоторому источнику тока,
    потребляют одинаковую мощность.
    Найти внутреннее сопротивление источника и КПД цепи в каждом случае.
}
\answer{%
    \begin{align*}
        P_1 &= \sqr{\frac{ \mathcal{E} }{R_1 + r}}R_1,
        P_2  = \sqr{\frac{ \mathcal{E} }{R_2 + r}}R_2,
        P_1 = P_2 \implies  \\
        &\implies R_1 \sqr{R_2 + r} = R_2 \sqr{R_1 + r} \implies  \\
        &\implies R_1 R_2^2 + 2 R_1 R_2 r + R_1 r^2 =
                    R_2 R_1^2 + 2 R_2 R_1 r + R_2 r^2  \implies  \\
    &\implies r^2 (R_2 - R_1) = R_2^2 R_2 - R_1^2 R_2 \implies  \\
    &\implies r
            = \sqrt{R_1 R_2 \frac{R_2 - R_1}{R_2 - R_1}}
            = \sqrt{R_1 R_2}
            = \sqrt{5{,}00\,\text{Ом} \cdot 80{,}00\,\text{Ом}}
            = 20{,}00\,\text{Ом}.
            \\
    \eta_1
            &= \frac{ R_1 }{R_1 + r}
            = \frac{\sqrt{ R_1 }}{\sqrt{ R_1 } + \sqrt{ R_2 }}
            = 0{,}200,  \\
    \eta_2
            &= \frac{ R_2 }{R_2 + r}
            = \frac{ \sqrt{ R_2 } }{\sqrt{ R_2 } + \sqrt{ R_1 }}
            = 0{,}800
    \end{align*}
}
\solutionspace{120pt}

\tasknumber{4}%
\task{%
    Определите ток, протекающий через резистор $R = 10\,\text{Ом}$ и разность потенциалов на нём (см.
    рис.
    на доске),
    если $r_1 = 3\,\text{Ом}$, $r_2 = 3\,\text{Ом}$, $\mathcal{E}_1 = 20\,\text{В}$, $\mathcal{E}_2 = 40\,\text{В}$
}

\variantsplitter

\addpersonalvariant{Вероника Битерякова}

\tasknumber{1}%
\task{%
    На резистор сопротивлением $r = 12\,\text{Ом}$ подали напряжение $V = 180\,\text{В}$.
    Определите ток, который потечёт через резистор, и мощность, выделяющуюся на нём.
}
\answer{%
    \begin{align*}
    \mathcal{I} &= \frac{ V }{ r } = \frac{ 180\,\text{В} }{ 12\,\text{Ом} } = 15{,}00\,\text{А},  \\
    P &= \frac{V^2}{ r } = \frac{ \sqr{ 180\,\text{В} } }{ 12\,\text{Ом} } = 2700{,}00\,\text{Вт}
    \end{align*}
}
\solutionspace{120pt}

\tasknumber{2}%
\task{%
    Замкнутая электрическая цепь состоит из ЭДС $\mathcal{E} = 3\,\text{В}$ и сопротивлением $r$
    и резистора $R = 24\,\text{Ом}$.
    Определите ток, протекающий в цепи.
    Какая тепловая энергия выделится на резисторе за время
    $\tau = 5\,\text{с}$? Какая работа будет совершена ЭДС за это время? Каков знак этой работы? Чему равен КПД цепи? Вычислите значения для 2 случаев:
    $r=0$ и $r = 10\,\text{Ом}$.
}
\answer{%
    \begin{align*}
    \mathcal{I}_1 &= \frac{ \mathcal{E} }{ R } = \frac{ 3\,\text{В} }{ 24\,\text{Ом} } = 0{,}12\,\text{А},  \\
    \mathcal{I}_2 &= \frac{ \mathcal{E} }{R + r} = \frac{ 3\,\text{В} }{24\,\text{Ом} + 10\,\text{Ом}} = 0{,}09\,\text{А},  \\
    Q_1 &= \mathcal{I}_1^2R\tau = \sqr{\frac{ \mathcal{E} }{ R }} R \tau
            = \sqr{\frac{ 3\,\text{В} }{ 24\,\text{Ом} }} \cdot 24\,\text{Ом} \cdot 5\,\text{с} = 1{,}728\,\text{Дж},  \\
    Q_2 &= \mathcal{I}_2^2R\tau = \sqr{\frac{ \mathcal{E} }{R + r}} R \tau
            = \sqr{\frac{ 3\,\text{В} }{24\,\text{Ом} + 10\,\text{Ом}}} \cdot 24\,\text{Ом} \cdot 5\,\text{с} = 0{,}972\,\text{Дж},  \\
    A_1 &= \mathcal{I}_1\tau\mathcal{E} = \frac{ \mathcal{E} }{R} \tau \mathcal{E}
            = \frac{\mathcal{E}^2 \tau}{ R } = \frac{\sqr{ 3\,\text{В} } \cdot 5\,\text{с}}{ 24\,\text{Ом} }
            = 1{,}800\,\text{Дж}, \text{положительна},  \\
    A_2 &= \mathcal{I}_2\tau\mathcal{E} = \frac{ \mathcal{E} }{R + r} \tau \mathcal{E}
            = \frac{\mathcal{E}^2 \tau}{R + r} = \frac{\sqr{ 3\,\text{В} } \cdot 5\,\text{с}}{24\,\text{Ом} + 10\,\text{Ом}}
            = 1{,}350\,\text{Дж}, \text{положительна},  \\
    \eta_1 &= \frac{ Q_1 }{ A_1 } = \ldots = \frac{ R }{ R } = 1,  \\
    \eta_2 &= \frac{ Q_2 }{ A_2 } = \ldots = \frac{ R }{R + r} = 0{,}72
    \end{align*}
}
\solutionspace{120pt}

\tasknumber{3}%
\task{%
    Лампочки, сопротивления которых $R_1 = 6{,}00\,\text{Ом}$ и $R_2 = 54{,}00\,\text{Ом}$, поочерёдно подключённные к некоторому источнику тока,
    потребляют одинаковую мощность.
    Найти внутреннее сопротивление источника и КПД цепи в каждом случае.
}
\answer{%
    \begin{align*}
        P_1 &= \sqr{\frac{ \mathcal{E} }{R_1 + r}}R_1,
        P_2  = \sqr{\frac{ \mathcal{E} }{R_2 + r}}R_2,
        P_1 = P_2 \implies  \\
        &\implies R_1 \sqr{R_2 + r} = R_2 \sqr{R_1 + r} \implies  \\
        &\implies R_1 R_2^2 + 2 R_1 R_2 r + R_1 r^2 =
                    R_2 R_1^2 + 2 R_2 R_1 r + R_2 r^2  \implies  \\
    &\implies r^2 (R_2 - R_1) = R_2^2 R_2 - R_1^2 R_2 \implies  \\
    &\implies r
            = \sqrt{R_1 R_2 \frac{R_2 - R_1}{R_2 - R_1}}
            = \sqrt{R_1 R_2}
            = \sqrt{6{,}00\,\text{Ом} \cdot 54{,}00\,\text{Ом}}
            = 18{,}00\,\text{Ом}.
            \\
    \eta_1
            &= \frac{ R_1 }{R_1 + r}
            = \frac{\sqrt{ R_1 }}{\sqrt{ R_1 } + \sqrt{ R_2 }}
            = 0{,}250,  \\
    \eta_2
            &= \frac{ R_2 }{R_2 + r}
            = \frac{ \sqrt{ R_2 } }{\sqrt{ R_2 } + \sqrt{ R_1 }}
            = 0{,}750
    \end{align*}
}
\solutionspace{120pt}

\tasknumber{4}%
\task{%
    Определите ток, протекающий через резистор $R = 10\,\text{Ом}$ и разность потенциалов на нём (см.
    рис.
    на доске),
    если $r_1 = 3\,\text{Ом}$, $r_2 = 3\,\text{Ом}$, $\mathcal{E}_1 = 20\,\text{В}$, $\mathcal{E}_2 = 30\,\text{В}$
}

\variantsplitter

\addpersonalvariant{Юлия Буянова}

\tasknumber{1}%
\task{%
    На резистор сопротивлением $r = 30\,\text{Ом}$ подали напряжение $V = 180\,\text{В}$.
    Определите ток, который потечёт через резистор, и мощность, выделяющуюся на нём.
}
\answer{%
    \begin{align*}
    \mathcal{I} &= \frac{ V }{ r } = \frac{ 180\,\text{В} }{ 30\,\text{Ом} } = 6{,}00\,\text{А},  \\
    P &= \frac{V^2}{ r } = \frac{ \sqr{ 180\,\text{В} } }{ 30\,\text{Ом} } = 1080{,}00\,\text{Вт}
    \end{align*}
}
\solutionspace{120pt}

\tasknumber{2}%
\task{%
    Замкнутая электрическая цепь состоит из ЭДС $\mathcal{E} = 3\,\text{В}$ и сопротивлением $r$
    и резистора $R = 30\,\text{Ом}$.
    Определите ток, протекающий в цепи.
    Какая тепловая энергия выделится на резисторе за время
    $\tau = 10\,\text{с}$? Какая работа будет совершена ЭДС за это время? Каков знак этой работы? Чему равен КПД цепи? Вычислите значения для 2 случаев:
    $r=0$ и $r = 30\,\text{Ом}$.
}
\answer{%
    \begin{align*}
    \mathcal{I}_1 &= \frac{ \mathcal{E} }{ R } = \frac{ 3\,\text{В} }{ 30\,\text{Ом} } = 0{,}10\,\text{А},  \\
    \mathcal{I}_2 &= \frac{ \mathcal{E} }{R + r} = \frac{ 3\,\text{В} }{30\,\text{Ом} + 30\,\text{Ом}} = 0{,}05\,\text{А},  \\
    Q_1 &= \mathcal{I}_1^2R\tau = \sqr{\frac{ \mathcal{E} }{ R }} R \tau
            = \sqr{\frac{ 3\,\text{В} }{ 30\,\text{Ом} }} \cdot 30\,\text{Ом} \cdot 10\,\text{с} = 3{,}000\,\text{Дж},  \\
    Q_2 &= \mathcal{I}_2^2R\tau = \sqr{\frac{ \mathcal{E} }{R + r}} R \tau
            = \sqr{\frac{ 3\,\text{В} }{30\,\text{Ом} + 30\,\text{Ом}}} \cdot 30\,\text{Ом} \cdot 10\,\text{с} = 0{,}750\,\text{Дж},  \\
    A_1 &= \mathcal{I}_1\tau\mathcal{E} = \frac{ \mathcal{E} }{R} \tau \mathcal{E}
            = \frac{\mathcal{E}^2 \tau}{ R } = \frac{\sqr{ 3\,\text{В} } \cdot 10\,\text{с}}{ 30\,\text{Ом} }
            = 3{,}000\,\text{Дж}, \text{положительна},  \\
    A_2 &= \mathcal{I}_2\tau\mathcal{E} = \frac{ \mathcal{E} }{R + r} \tau \mathcal{E}
            = \frac{\mathcal{E}^2 \tau}{R + r} = \frac{\sqr{ 3\,\text{В} } \cdot 10\,\text{с}}{30\,\text{Ом} + 30\,\text{Ом}}
            = 1{,}500\,\text{Дж}, \text{положительна},  \\
    \eta_1 &= \frac{ Q_1 }{ A_1 } = \ldots = \frac{ R }{ R } = 1,  \\
    \eta_2 &= \frac{ Q_2 }{ A_2 } = \ldots = \frac{ R }{R + r} = 0{,}50
    \end{align*}
}
\solutionspace{120pt}

\tasknumber{3}%
\task{%
    Лампочки, сопротивления которых $R_1 = 0{,}50\,\text{Ом}$ и $R_2 = 18{,}00\,\text{Ом}$, поочерёдно подключённные к некоторому источнику тока,
    потребляют одинаковую мощность.
    Найти внутреннее сопротивление источника и КПД цепи в каждом случае.
}
\answer{%
    \begin{align*}
        P_1 &= \sqr{\frac{ \mathcal{E} }{R_1 + r}}R_1,
        P_2  = \sqr{\frac{ \mathcal{E} }{R_2 + r}}R_2,
        P_1 = P_2 \implies  \\
        &\implies R_1 \sqr{R_2 + r} = R_2 \sqr{R_1 + r} \implies  \\
        &\implies R_1 R_2^2 + 2 R_1 R_2 r + R_1 r^2 =
                    R_2 R_1^2 + 2 R_2 R_1 r + R_2 r^2  \implies  \\
    &\implies r^2 (R_2 - R_1) = R_2^2 R_2 - R_1^2 R_2 \implies  \\
    &\implies r
            = \sqrt{R_1 R_2 \frac{R_2 - R_1}{R_2 - R_1}}
            = \sqrt{R_1 R_2}
            = \sqrt{0{,}50\,\text{Ом} \cdot 18{,}00\,\text{Ом}}
            = 3{,}00\,\text{Ом}.
            \\
    \eta_1
            &= \frac{ R_1 }{R_1 + r}
            = \frac{\sqrt{ R_1 }}{\sqrt{ R_1 } + \sqrt{ R_2 }}
            = 0{,}143,  \\
    \eta_2
            &= \frac{ R_2 }{R_2 + r}
            = \frac{ \sqrt{ R_2 } }{\sqrt{ R_2 } + \sqrt{ R_1 }}
            = 0{,}857
    \end{align*}
}
\solutionspace{120pt}

\tasknumber{4}%
\task{%
    Определите ток, протекающий через резистор $R = 10\,\text{Ом}$ и разность потенциалов на нём (см.
    рис.
    на доске),
    если $r_1 = 1\,\text{Ом}$, $r_2 = 1\,\text{Ом}$, $\mathcal{E}_1 = 20\,\text{В}$, $\mathcal{E}_2 = 20\,\text{В}$
}

\variantsplitter

\addpersonalvariant{Пелагея Вдовина}

\tasknumber{1}%
\task{%
    На резистор сопротивлением $r = 12\,\text{Ом}$ подали напряжение $V = 120\,\text{В}$.
    Определите ток, который потечёт через резистор, и мощность, выделяющуюся на нём.
}
\answer{%
    \begin{align*}
    \mathcal{I} &= \frac{ V }{ r } = \frac{ 120\,\text{В} }{ 12\,\text{Ом} } = 10{,}00\,\text{А},  \\
    P &= \frac{V^2}{ r } = \frac{ \sqr{ 120\,\text{В} } }{ 12\,\text{Ом} } = 1200{,}00\,\text{Вт}
    \end{align*}
}
\solutionspace{120pt}

\tasknumber{2}%
\task{%
    Замкнутая электрическая цепь состоит из ЭДС $\mathcal{E} = 1\,\text{В}$ и сопротивлением $r$
    и резистора $R = 24\,\text{Ом}$.
    Определите ток, протекающий в цепи.
    Какая тепловая энергия выделится на резисторе за время
    $\tau = 10\,\text{с}$? Какая работа будет совершена ЭДС за это время? Каков знак этой работы? Чему равен КПД цепи? Вычислите значения для 2 случаев:
    $r=0$ и $r = 30\,\text{Ом}$.
}
\answer{%
    \begin{align*}
    \mathcal{I}_1 &= \frac{ \mathcal{E} }{ R } = \frac{ 1\,\text{В} }{ 24\,\text{Ом} } = 0{,}04\,\text{А},  \\
    \mathcal{I}_2 &= \frac{ \mathcal{E} }{R + r} = \frac{ 1\,\text{В} }{24\,\text{Ом} + 30\,\text{Ом}} = 0{,}02\,\text{А},  \\
    Q_1 &= \mathcal{I}_1^2R\tau = \sqr{\frac{ \mathcal{E} }{ R }} R \tau
            = \sqr{\frac{ 1\,\text{В} }{ 24\,\text{Ом} }} \cdot 24\,\text{Ом} \cdot 10\,\text{с} = 0{,}384\,\text{Дж},  \\
    Q_2 &= \mathcal{I}_2^2R\tau = \sqr{\frac{ \mathcal{E} }{R + r}} R \tau
            = \sqr{\frac{ 1\,\text{В} }{24\,\text{Ом} + 30\,\text{Ом}}} \cdot 24\,\text{Ом} \cdot 10\,\text{с} = 0{,}096\,\text{Дж},  \\
    A_1 &= \mathcal{I}_1\tau\mathcal{E} = \frac{ \mathcal{E} }{R} \tau \mathcal{E}
            = \frac{\mathcal{E}^2 \tau}{ R } = \frac{\sqr{ 1\,\text{В} } \cdot 10\,\text{с}}{ 24\,\text{Ом} }
            = 0{,}400\,\text{Дж}, \text{положительна},  \\
    A_2 &= \mathcal{I}_2\tau\mathcal{E} = \frac{ \mathcal{E} }{R + r} \tau \mathcal{E}
            = \frac{\mathcal{E}^2 \tau}{R + r} = \frac{\sqr{ 1\,\text{В} } \cdot 10\,\text{с}}{24\,\text{Ом} + 30\,\text{Ом}}
            = 0{,}200\,\text{Дж}, \text{положительна},  \\
    \eta_1 &= \frac{ Q_1 }{ A_1 } = \ldots = \frac{ R }{ R } = 1,  \\
    \eta_2 &= \frac{ Q_2 }{ A_2 } = \ldots = \frac{ R }{R + r} = 0{,}48
    \end{align*}
}
\solutionspace{120pt}

\tasknumber{3}%
\task{%
    Лампочки, сопротивления которых $R_1 = 0{,}50\,\text{Ом}$ и $R_2 = 18{,}00\,\text{Ом}$, поочерёдно подключённные к некоторому источнику тока,
    потребляют одинаковую мощность.
    Найти внутреннее сопротивление источника и КПД цепи в каждом случае.
}
\answer{%
    \begin{align*}
        P_1 &= \sqr{\frac{ \mathcal{E} }{R_1 + r}}R_1,
        P_2  = \sqr{\frac{ \mathcal{E} }{R_2 + r}}R_2,
        P_1 = P_2 \implies  \\
        &\implies R_1 \sqr{R_2 + r} = R_2 \sqr{R_1 + r} \implies  \\
        &\implies R_1 R_2^2 + 2 R_1 R_2 r + R_1 r^2 =
                    R_2 R_1^2 + 2 R_2 R_1 r + R_2 r^2  \implies  \\
    &\implies r^2 (R_2 - R_1) = R_2^2 R_2 - R_1^2 R_2 \implies  \\
    &\implies r
            = \sqrt{R_1 R_2 \frac{R_2 - R_1}{R_2 - R_1}}
            = \sqrt{R_1 R_2}
            = \sqrt{0{,}50\,\text{Ом} \cdot 18{,}00\,\text{Ом}}
            = 3{,}00\,\text{Ом}.
            \\
    \eta_1
            &= \frac{ R_1 }{R_1 + r}
            = \frac{\sqrt{ R_1 }}{\sqrt{ R_1 } + \sqrt{ R_2 }}
            = 0{,}143,  \\
    \eta_2
            &= \frac{ R_2 }{R_2 + r}
            = \frac{ \sqrt{ R_2 } }{\sqrt{ R_2 } + \sqrt{ R_1 }}
            = 0{,}857
    \end{align*}
}
\solutionspace{120pt}

\tasknumber{4}%
\task{%
    Определите ток, протекающий через резистор $R = 20\,\text{Ом}$ и разность потенциалов на нём (см.
    рис.
    на доске),
    если $r_1 = 3\,\text{Ом}$, $r_2 = 1\,\text{Ом}$, $\mathcal{E}_1 = 60\,\text{В}$, $\mathcal{E}_2 = 30\,\text{В}$
}

\variantsplitter

\addpersonalvariant{Леонид Викторов}

\tasknumber{1}%
\task{%
    На резистор сопротивлением $r = 30\,\text{Ом}$ подали напряжение $U = 240\,\text{В}$.
    Определите ток, который потечёт через резистор, и мощность, выделяющуюся на нём.
}
\answer{%
    \begin{align*}
    \mathcal{I} &= \frac{ U }{ r } = \frac{ 240\,\text{В} }{ 30\,\text{Ом} } = 8{,}00\,\text{А},  \\
    P &= \frac{U^2}{ r } = \frac{ \sqr{ 240\,\text{В} } }{ 30\,\text{Ом} } = 1920{,}00\,\text{Вт}
    \end{align*}
}
\solutionspace{120pt}

\tasknumber{2}%
\task{%
    Замкнутая электрическая цепь состоит из ЭДС $\mathcal{E} = 4\,\text{В}$ и сопротивлением $r$
    и резистора $R = 10\,\text{Ом}$.
    Определите ток, протекающий в цепи.
    Какая тепловая энергия выделится на резисторе за время
    $\tau = 5\,\text{с}$? Какая работа будет совершена ЭДС за это время? Каков знак этой работы? Чему равен КПД цепи? Вычислите значения для 2 случаев:
    $r=0$ и $r = 60\,\text{Ом}$.
}
\answer{%
    \begin{align*}
    \mathcal{I}_1 &= \frac{ \mathcal{E} }{ R } = \frac{ 4\,\text{В} }{ 10\,\text{Ом} } = 0{,}40\,\text{А},  \\
    \mathcal{I}_2 &= \frac{ \mathcal{E} }{R + r} = \frac{ 4\,\text{В} }{10\,\text{Ом} + 60\,\text{Ом}} = 0{,}06\,\text{А},  \\
    Q_1 &= \mathcal{I}_1^2R\tau = \sqr{\frac{ \mathcal{E} }{ R }} R \tau
            = \sqr{\frac{ 4\,\text{В} }{ 10\,\text{Ом} }} \cdot 10\,\text{Ом} \cdot 5\,\text{с} = 8{,}000\,\text{Дж},  \\
    Q_2 &= \mathcal{I}_2^2R\tau = \sqr{\frac{ \mathcal{E} }{R + r}} R \tau
            = \sqr{\frac{ 4\,\text{В} }{10\,\text{Ом} + 60\,\text{Ом}}} \cdot 10\,\text{Ом} \cdot 5\,\text{с} = 0{,}180\,\text{Дж},  \\
    A_1 &= \mathcal{I}_1\tau\mathcal{E} = \frac{ \mathcal{E} }{R} \tau \mathcal{E}
            = \frac{\mathcal{E}^2 \tau}{ R } = \frac{\sqr{ 4\,\text{В} } \cdot 5\,\text{с}}{ 10\,\text{Ом} }
            = 8{,}000\,\text{Дж}, \text{положительна},  \\
    A_2 &= \mathcal{I}_2\tau\mathcal{E} = \frac{ \mathcal{E} }{R + r} \tau \mathcal{E}
            = \frac{\mathcal{E}^2 \tau}{R + r} = \frac{\sqr{ 4\,\text{В} } \cdot 5\,\text{с}}{10\,\text{Ом} + 60\,\text{Ом}}
            = 1{,}200\,\text{Дж}, \text{положительна},  \\
    \eta_1 &= \frac{ Q_1 }{ A_1 } = \ldots = \frac{ R }{ R } = 1,  \\
    \eta_2 &= \frac{ Q_2 }{ A_2 } = \ldots = \frac{ R }{R + r} = 0{,}15
    \end{align*}
}
\solutionspace{120pt}

\tasknumber{3}%
\task{%
    Лампочки, сопротивления которых $R_1 = 1{,}00\,\text{Ом}$ и $R_2 = 9{,}00\,\text{Ом}$, поочерёдно подключённные к некоторому источнику тока,
    потребляют одинаковую мощность.
    Найти внутреннее сопротивление источника и КПД цепи в каждом случае.
}
\answer{%
    \begin{align*}
        P_1 &= \sqr{\frac{ \mathcal{E} }{R_1 + r}}R_1,
        P_2  = \sqr{\frac{ \mathcal{E} }{R_2 + r}}R_2,
        P_1 = P_2 \implies  \\
        &\implies R_1 \sqr{R_2 + r} = R_2 \sqr{R_1 + r} \implies  \\
        &\implies R_1 R_2^2 + 2 R_1 R_2 r + R_1 r^2 =
                    R_2 R_1^2 + 2 R_2 R_1 r + R_2 r^2  \implies  \\
    &\implies r^2 (R_2 - R_1) = R_2^2 R_2 - R_1^2 R_2 \implies  \\
    &\implies r
            = \sqrt{R_1 R_2 \frac{R_2 - R_1}{R_2 - R_1}}
            = \sqrt{R_1 R_2}
            = \sqrt{1{,}00\,\text{Ом} \cdot 9{,}00\,\text{Ом}}
            = 3{,}00\,\text{Ом}.
            \\
    \eta_1
            &= \frac{ R_1 }{R_1 + r}
            = \frac{\sqrt{ R_1 }}{\sqrt{ R_1 } + \sqrt{ R_2 }}
            = 0{,}250,  \\
    \eta_2
            &= \frac{ R_2 }{R_2 + r}
            = \frac{ \sqrt{ R_2 } }{\sqrt{ R_2 } + \sqrt{ R_1 }}
            = 0{,}750
    \end{align*}
}
\solutionspace{120pt}

\tasknumber{4}%
\task{%
    Определите ток, протекающий через резистор $R = 20\,\text{Ом}$ и разность потенциалов на нём (см.
    рис.
    на доске),
    если $r_1 = 3\,\text{Ом}$, $r_2 = 2\,\text{Ом}$, $\mathcal{E}_1 = 60\,\text{В}$, $\mathcal{E}_2 = 20\,\text{В}$
}

\variantsplitter

\addpersonalvariant{Фёдор Гнутов}

\tasknumber{1}%
\task{%
    На резистор сопротивлением $R = 30\,\text{Ом}$ подали напряжение $V = 150\,\text{В}$.
    Определите ток, который потечёт через резистор, и мощность, выделяющуюся на нём.
}
\answer{%
    \begin{align*}
    \mathcal{I} &= \frac{ V }{ R } = \frac{ 150\,\text{В} }{ 30\,\text{Ом} } = 5{,}00\,\text{А},  \\
    P &= \frac{V^2}{ R } = \frac{ \sqr{ 150\,\text{В} } }{ 30\,\text{Ом} } = 750{,}00\,\text{Вт}
    \end{align*}
}
\solutionspace{120pt}

\tasknumber{2}%
\task{%
    Замкнутая электрическая цепь состоит из ЭДС $\mathcal{E} = 1\,\text{В}$ и сопротивлением $r$
    и резистора $R = 15\,\text{Ом}$.
    Определите ток, протекающий в цепи.
    Какая тепловая энергия выделится на резисторе за время
    $\tau = 5\,\text{с}$? Какая работа будет совершена ЭДС за это время? Каков знак этой работы? Чему равен КПД цепи? Вычислите значения для 2 случаев:
    $r=0$ и $r = 20\,\text{Ом}$.
}
\answer{%
    \begin{align*}
    \mathcal{I}_1 &= \frac{ \mathcal{E} }{ R } = \frac{ 1\,\text{В} }{ 15\,\text{Ом} } = 0{,}07\,\text{А},  \\
    \mathcal{I}_2 &= \frac{ \mathcal{E} }{R + r} = \frac{ 1\,\text{В} }{15\,\text{Ом} + 20\,\text{Ом}} = 0{,}03\,\text{А},  \\
    Q_1 &= \mathcal{I}_1^2R\tau = \sqr{\frac{ \mathcal{E} }{ R }} R \tau
            = \sqr{\frac{ 1\,\text{В} }{ 15\,\text{Ом} }} \cdot 15\,\text{Ом} \cdot 5\,\text{с} = 0{,}368\,\text{Дж},  \\
    Q_2 &= \mathcal{I}_2^2R\tau = \sqr{\frac{ \mathcal{E} }{R + r}} R \tau
            = \sqr{\frac{ 1\,\text{В} }{15\,\text{Ом} + 20\,\text{Ом}}} \cdot 15\,\text{Ом} \cdot 5\,\text{с} = 0{,}068\,\text{Дж},  \\
    A_1 &= \mathcal{I}_1\tau\mathcal{E} = \frac{ \mathcal{E} }{R} \tau \mathcal{E}
            = \frac{\mathcal{E}^2 \tau}{ R } = \frac{\sqr{ 1\,\text{В} } \cdot 5\,\text{с}}{ 15\,\text{Ом} }
            = 0{,}350\,\text{Дж}, \text{положительна},  \\
    A_2 &= \mathcal{I}_2\tau\mathcal{E} = \frac{ \mathcal{E} }{R + r} \tau \mathcal{E}
            = \frac{\mathcal{E}^2 \tau}{R + r} = \frac{\sqr{ 1\,\text{В} } \cdot 5\,\text{с}}{15\,\text{Ом} + 20\,\text{Ом}}
            = 0{,}150\,\text{Дж}, \text{положительна},  \\
    \eta_1 &= \frac{ Q_1 }{ A_1 } = \ldots = \frac{ R }{ R } = 1,  \\
    \eta_2 &= \frac{ Q_2 }{ A_2 } = \ldots = \frac{ R }{R + r} = 0{,}45
    \end{align*}
}
\solutionspace{120pt}

\tasknumber{3}%
\task{%
    Лампочки, сопротивления которых $R_1 = 5{,}00\,\text{Ом}$ и $R_2 = 80{,}00\,\text{Ом}$, поочерёдно подключённные к некоторому источнику тока,
    потребляют одинаковую мощность.
    Найти внутреннее сопротивление источника и КПД цепи в каждом случае.
}
\answer{%
    \begin{align*}
        P_1 &= \sqr{\frac{ \mathcal{E} }{R_1 + r}}R_1,
        P_2  = \sqr{\frac{ \mathcal{E} }{R_2 + r}}R_2,
        P_1 = P_2 \implies  \\
        &\implies R_1 \sqr{R_2 + r} = R_2 \sqr{R_1 + r} \implies  \\
        &\implies R_1 R_2^2 + 2 R_1 R_2 r + R_1 r^2 =
                    R_2 R_1^2 + 2 R_2 R_1 r + R_2 r^2  \implies  \\
    &\implies r^2 (R_2 - R_1) = R_2^2 R_2 - R_1^2 R_2 \implies  \\
    &\implies r
            = \sqrt{R_1 R_2 \frac{R_2 - R_1}{R_2 - R_1}}
            = \sqrt{R_1 R_2}
            = \sqrt{5{,}00\,\text{Ом} \cdot 80{,}00\,\text{Ом}}
            = 20{,}00\,\text{Ом}.
            \\
    \eta_1
            &= \frac{ R_1 }{R_1 + r}
            = \frac{\sqrt{ R_1 }}{\sqrt{ R_1 } + \sqrt{ R_2 }}
            = 0{,}200,  \\
    \eta_2
            &= \frac{ R_2 }{R_2 + r}
            = \frac{ \sqrt{ R_2 } }{\sqrt{ R_2 } + \sqrt{ R_1 }}
            = 0{,}800
    \end{align*}
}
\solutionspace{120pt}

\tasknumber{4}%
\task{%
    Определите ток, протекающий через резистор $R = 15\,\text{Ом}$ и разность потенциалов на нём (см.
    рис.
    на доске),
    если $r_1 = 2\,\text{Ом}$, $r_2 = 2\,\text{Ом}$, $\mathcal{E}_1 = 40\,\text{В}$, $\mathcal{E}_2 = 20\,\text{В}$
}

\variantsplitter

\addpersonalvariant{Илья Гримберг}

\tasknumber{1}%
\task{%
    На резистор сопротивлением $r = 5\,\text{Ом}$ подали напряжение $U = 180\,\text{В}$.
    Определите ток, который потечёт через резистор, и мощность, выделяющуюся на нём.
}
\answer{%
    \begin{align*}
    \mathcal{I} &= \frac{ U }{ r } = \frac{ 180\,\text{В} }{ 5\,\text{Ом} } = 36{,}00\,\text{А},  \\
    P &= \frac{U^2}{ r } = \frac{ \sqr{ 180\,\text{В} } }{ 5\,\text{Ом} } = 6480{,}00\,\text{Вт}
    \end{align*}
}
\solutionspace{120pt}

\tasknumber{2}%
\task{%
    Замкнутая электрическая цепь состоит из ЭДС $\mathcal{E} = 2\,\text{В}$ и сопротивлением $r$
    и резистора $R = 10\,\text{Ом}$.
    Определите ток, протекающий в цепи.
    Какая тепловая энергия выделится на резисторе за время
    $\tau = 5\,\text{с}$? Какая работа будет совершена ЭДС за это время? Каков знак этой работы? Чему равен КПД цепи? Вычислите значения для 2 случаев:
    $r=0$ и $r = 20\,\text{Ом}$.
}
\answer{%
    \begin{align*}
    \mathcal{I}_1 &= \frac{ \mathcal{E} }{ R } = \frac{ 2\,\text{В} }{ 10\,\text{Ом} } = 0{,}20\,\text{А},  \\
    \mathcal{I}_2 &= \frac{ \mathcal{E} }{R + r} = \frac{ 2\,\text{В} }{10\,\text{Ом} + 20\,\text{Ом}} = 0{,}07\,\text{А},  \\
    Q_1 &= \mathcal{I}_1^2R\tau = \sqr{\frac{ \mathcal{E} }{ R }} R \tau
            = \sqr{\frac{ 2\,\text{В} }{ 10\,\text{Ом} }} \cdot 10\,\text{Ом} \cdot 5\,\text{с} = 2{,}000\,\text{Дж},  \\
    Q_2 &= \mathcal{I}_2^2R\tau = \sqr{\frac{ \mathcal{E} }{R + r}} R \tau
            = \sqr{\frac{ 2\,\text{В} }{10\,\text{Ом} + 20\,\text{Ом}}} \cdot 10\,\text{Ом} \cdot 5\,\text{с} = 0{,}245\,\text{Дж},  \\
    A_1 &= \mathcal{I}_1\tau\mathcal{E} = \frac{ \mathcal{E} }{R} \tau \mathcal{E}
            = \frac{\mathcal{E}^2 \tau}{ R } = \frac{\sqr{ 2\,\text{В} } \cdot 5\,\text{с}}{ 10\,\text{Ом} }
            = 2{,}000\,\text{Дж}, \text{положительна},  \\
    A_2 &= \mathcal{I}_2\tau\mathcal{E} = \frac{ \mathcal{E} }{R + r} \tau \mathcal{E}
            = \frac{\mathcal{E}^2 \tau}{R + r} = \frac{\sqr{ 2\,\text{В} } \cdot 5\,\text{с}}{10\,\text{Ом} + 20\,\text{Ом}}
            = 0{,}700\,\text{Дж}, \text{положительна},  \\
    \eta_1 &= \frac{ Q_1 }{ A_1 } = \ldots = \frac{ R }{ R } = 1,  \\
    \eta_2 &= \frac{ Q_2 }{ A_2 } = \ldots = \frac{ R }{R + r} = 0{,}35
    \end{align*}
}
\solutionspace{120pt}

\tasknumber{3}%
\task{%
    Лампочки, сопротивления которых $R_1 = 0{,}50\,\text{Ом}$ и $R_2 = 18{,}00\,\text{Ом}$, поочерёдно подключённные к некоторому источнику тока,
    потребляют одинаковую мощность.
    Найти внутреннее сопротивление источника и КПД цепи в каждом случае.
}
\answer{%
    \begin{align*}
        P_1 &= \sqr{\frac{ \mathcal{E} }{R_1 + r}}R_1,
        P_2  = \sqr{\frac{ \mathcal{E} }{R_2 + r}}R_2,
        P_1 = P_2 \implies  \\
        &\implies R_1 \sqr{R_2 + r} = R_2 \sqr{R_1 + r} \implies  \\
        &\implies R_1 R_2^2 + 2 R_1 R_2 r + R_1 r^2 =
                    R_2 R_1^2 + 2 R_2 R_1 r + R_2 r^2  \implies  \\
    &\implies r^2 (R_2 - R_1) = R_2^2 R_2 - R_1^2 R_2 \implies  \\
    &\implies r
            = \sqrt{R_1 R_2 \frac{R_2 - R_1}{R_2 - R_1}}
            = \sqrt{R_1 R_2}
            = \sqrt{0{,}50\,\text{Ом} \cdot 18{,}00\,\text{Ом}}
            = 3{,}00\,\text{Ом}.
            \\
    \eta_1
            &= \frac{ R_1 }{R_1 + r}
            = \frac{\sqrt{ R_1 }}{\sqrt{ R_1 } + \sqrt{ R_2 }}
            = 0{,}143,  \\
    \eta_2
            &= \frac{ R_2 }{R_2 + r}
            = \frac{ \sqrt{ R_2 } }{\sqrt{ R_2 } + \sqrt{ R_1 }}
            = 0{,}857
    \end{align*}
}
\solutionspace{120pt}

\tasknumber{4}%
\task{%
    Определите ток, протекающий через резистор $R = 10\,\text{Ом}$ и разность потенциалов на нём (см.
    рис.
    на доске),
    если $r_1 = 3\,\text{Ом}$, $r_2 = 3\,\text{Ом}$, $\mathcal{E}_1 = 60\,\text{В}$, $\mathcal{E}_2 = 30\,\text{В}$
}

\variantsplitter

\addpersonalvariant{Иван Гурьянов}

\tasknumber{1}%
\task{%
    На резистор сопротивлением $r = 30\,\text{Ом}$ подали напряжение $U = 240\,\text{В}$.
    Определите ток, который потечёт через резистор, и мощность, выделяющуюся на нём.
}
\answer{%
    \begin{align*}
    \mathcal{I} &= \frac{ U }{ r } = \frac{ 240\,\text{В} }{ 30\,\text{Ом} } = 8{,}00\,\text{А},  \\
    P &= \frac{U^2}{ r } = \frac{ \sqr{ 240\,\text{В} } }{ 30\,\text{Ом} } = 1920{,}00\,\text{Вт}
    \end{align*}
}
\solutionspace{120pt}

\tasknumber{2}%
\task{%
    Замкнутая электрическая цепь состоит из ЭДС $\mathcal{E} = 3\,\text{В}$ и сопротивлением $r$
    и резистора $R = 15\,\text{Ом}$.
    Определите ток, протекающий в цепи.
    Какая тепловая энергия выделится на резисторе за время
    $\tau = 10\,\text{с}$? Какая работа будет совершена ЭДС за это время? Каков знак этой работы? Чему равен КПД цепи? Вычислите значения для 2 случаев:
    $r=0$ и $r = 60\,\text{Ом}$.
}
\answer{%
    \begin{align*}
    \mathcal{I}_1 &= \frac{ \mathcal{E} }{ R } = \frac{ 3\,\text{В} }{ 15\,\text{Ом} } = 0{,}20\,\text{А},  \\
    \mathcal{I}_2 &= \frac{ \mathcal{E} }{R + r} = \frac{ 3\,\text{В} }{15\,\text{Ом} + 60\,\text{Ом}} = 0{,}04\,\text{А},  \\
    Q_1 &= \mathcal{I}_1^2R\tau = \sqr{\frac{ \mathcal{E} }{ R }} R \tau
            = \sqr{\frac{ 3\,\text{В} }{ 15\,\text{Ом} }} \cdot 15\,\text{Ом} \cdot 10\,\text{с} = 6{,}000\,\text{Дж},  \\
    Q_2 &= \mathcal{I}_2^2R\tau = \sqr{\frac{ \mathcal{E} }{R + r}} R \tau
            = \sqr{\frac{ 3\,\text{В} }{15\,\text{Ом} + 60\,\text{Ом}}} \cdot 15\,\text{Ом} \cdot 10\,\text{с} = 0{,}240\,\text{Дж},  \\
    A_1 &= \mathcal{I}_1\tau\mathcal{E} = \frac{ \mathcal{E} }{R} \tau \mathcal{E}
            = \frac{\mathcal{E}^2 \tau}{ R } = \frac{\sqr{ 3\,\text{В} } \cdot 10\,\text{с}}{ 15\,\text{Ом} }
            = 6{,}000\,\text{Дж}, \text{положительна},  \\
    A_2 &= \mathcal{I}_2\tau\mathcal{E} = \frac{ \mathcal{E} }{R + r} \tau \mathcal{E}
            = \frac{\mathcal{E}^2 \tau}{R + r} = \frac{\sqr{ 3\,\text{В} } \cdot 10\,\text{с}}{15\,\text{Ом} + 60\,\text{Ом}}
            = 1{,}200\,\text{Дж}, \text{положительна},  \\
    \eta_1 &= \frac{ Q_1 }{ A_1 } = \ldots = \frac{ R }{ R } = 1,  \\
    \eta_2 &= \frac{ Q_2 }{ A_2 } = \ldots = \frac{ R }{R + r} = 0{,}20
    \end{align*}
}
\solutionspace{120pt}

\tasknumber{3}%
\task{%
    Лампочки, сопротивления которых $R_1 = 0{,}50\,\text{Ом}$ и $R_2 = 2{,}00\,\text{Ом}$, поочерёдно подключённные к некоторому источнику тока,
    потребляют одинаковую мощность.
    Найти внутреннее сопротивление источника и КПД цепи в каждом случае.
}
\answer{%
    \begin{align*}
        P_1 &= \sqr{\frac{ \mathcal{E} }{R_1 + r}}R_1,
        P_2  = \sqr{\frac{ \mathcal{E} }{R_2 + r}}R_2,
        P_1 = P_2 \implies  \\
        &\implies R_1 \sqr{R_2 + r} = R_2 \sqr{R_1 + r} \implies  \\
        &\implies R_1 R_2^2 + 2 R_1 R_2 r + R_1 r^2 =
                    R_2 R_1^2 + 2 R_2 R_1 r + R_2 r^2  \implies  \\
    &\implies r^2 (R_2 - R_1) = R_2^2 R_2 - R_1^2 R_2 \implies  \\
    &\implies r
            = \sqrt{R_1 R_2 \frac{R_2 - R_1}{R_2 - R_1}}
            = \sqrt{R_1 R_2}
            = \sqrt{0{,}50\,\text{Ом} \cdot 2{,}00\,\text{Ом}}
            = 1{,}00\,\text{Ом}.
            \\
    \eta_1
            &= \frac{ R_1 }{R_1 + r}
            = \frac{\sqrt{ R_1 }}{\sqrt{ R_1 } + \sqrt{ R_2 }}
            = 0{,}333,  \\
    \eta_2
            &= \frac{ R_2 }{R_2 + r}
            = \frac{ \sqrt{ R_2 } }{\sqrt{ R_2 } + \sqrt{ R_1 }}
            = 0{,}667
    \end{align*}
}
\solutionspace{120pt}

\tasknumber{4}%
\task{%
    Определите ток, протекающий через резистор $R = 15\,\text{Ом}$ и разность потенциалов на нём (см.
    рис.
    на доске),
    если $r_1 = 1\,\text{Ом}$, $r_2 = 2\,\text{Ом}$, $\mathcal{E}_1 = 20\,\text{В}$, $\mathcal{E}_2 = 60\,\text{В}$
}

\variantsplitter

\addpersonalvariant{Артём Денежкин}

\tasknumber{1}%
\task{%
    На резистор сопротивлением $R = 5\,\text{Ом}$ подали напряжение $U = 150\,\text{В}$.
    Определите ток, который потечёт через резистор, и мощность, выделяющуюся на нём.
}
\answer{%
    \begin{align*}
    \mathcal{I} &= \frac{ U }{ R } = \frac{ 150\,\text{В} }{ 5\,\text{Ом} } = 30{,}00\,\text{А},  \\
    P &= \frac{U^2}{ R } = \frac{ \sqr{ 150\,\text{В} } }{ 5\,\text{Ом} } = 4500{,}00\,\text{Вт}
    \end{align*}
}
\solutionspace{120pt}

\tasknumber{2}%
\task{%
    Замкнутая электрическая цепь состоит из ЭДС $\mathcal{E} = 4\,\text{В}$ и сопротивлением $r$
    и резистора $R = 24\,\text{Ом}$.
    Определите ток, протекающий в цепи.
    Какая тепловая энергия выделится на резисторе за время
    $\tau = 10\,\text{с}$? Какая работа будет совершена ЭДС за это время? Каков знак этой работы? Чему равен КПД цепи? Вычислите значения для 2 случаев:
    $r=0$ и $r = 10\,\text{Ом}$.
}
\answer{%
    \begin{align*}
    \mathcal{I}_1 &= \frac{ \mathcal{E} }{ R } = \frac{ 4\,\text{В} }{ 24\,\text{Ом} } = 0{,}17\,\text{А},  \\
    \mathcal{I}_2 &= \frac{ \mathcal{E} }{R + r} = \frac{ 4\,\text{В} }{24\,\text{Ом} + 10\,\text{Ом}} = 0{,}12\,\text{А},  \\
    Q_1 &= \mathcal{I}_1^2R\tau = \sqr{\frac{ \mathcal{E} }{ R }} R \tau
            = \sqr{\frac{ 4\,\text{В} }{ 24\,\text{Ом} }} \cdot 24\,\text{Ом} \cdot 10\,\text{с} = 6{,}936\,\text{Дж},  \\
    Q_2 &= \mathcal{I}_2^2R\tau = \sqr{\frac{ \mathcal{E} }{R + r}} R \tau
            = \sqr{\frac{ 4\,\text{В} }{24\,\text{Ом} + 10\,\text{Ом}}} \cdot 24\,\text{Ом} \cdot 10\,\text{с} = 3{,}456\,\text{Дж},  \\
    A_1 &= \mathcal{I}_1\tau\mathcal{E} = \frac{ \mathcal{E} }{R} \tau \mathcal{E}
            = \frac{\mathcal{E}^2 \tau}{ R } = \frac{\sqr{ 4\,\text{В} } \cdot 10\,\text{с}}{ 24\,\text{Ом} }
            = 6{,}800\,\text{Дж}, \text{положительна},  \\
    A_2 &= \mathcal{I}_2\tau\mathcal{E} = \frac{ \mathcal{E} }{R + r} \tau \mathcal{E}
            = \frac{\mathcal{E}^2 \tau}{R + r} = \frac{\sqr{ 4\,\text{В} } \cdot 10\,\text{с}}{24\,\text{Ом} + 10\,\text{Ом}}
            = 4{,}800\,\text{Дж}, \text{положительна},  \\
    \eta_1 &= \frac{ Q_1 }{ A_1 } = \ldots = \frac{ R }{ R } = 1,  \\
    \eta_2 &= \frac{ Q_2 }{ A_2 } = \ldots = \frac{ R }{R + r} = 0{,}72
    \end{align*}
}
\solutionspace{120pt}

\tasknumber{3}%
\task{%
    Лампочки, сопротивления которых $R_1 = 0{,}50\,\text{Ом}$ и $R_2 = 2{,}00\,\text{Ом}$, поочерёдно подключённные к некоторому источнику тока,
    потребляют одинаковую мощность.
    Найти внутреннее сопротивление источника и КПД цепи в каждом случае.
}
\answer{%
    \begin{align*}
        P_1 &= \sqr{\frac{ \mathcal{E} }{R_1 + r}}R_1,
        P_2  = \sqr{\frac{ \mathcal{E} }{R_2 + r}}R_2,
        P_1 = P_2 \implies  \\
        &\implies R_1 \sqr{R_2 + r} = R_2 \sqr{R_1 + r} \implies  \\
        &\implies R_1 R_2^2 + 2 R_1 R_2 r + R_1 r^2 =
                    R_2 R_1^2 + 2 R_2 R_1 r + R_2 r^2  \implies  \\
    &\implies r^2 (R_2 - R_1) = R_2^2 R_2 - R_1^2 R_2 \implies  \\
    &\implies r
            = \sqrt{R_1 R_2 \frac{R_2 - R_1}{R_2 - R_1}}
            = \sqrt{R_1 R_2}
            = \sqrt{0{,}50\,\text{Ом} \cdot 2{,}00\,\text{Ом}}
            = 1{,}00\,\text{Ом}.
            \\
    \eta_1
            &= \frac{ R_1 }{R_1 + r}
            = \frac{\sqrt{ R_1 }}{\sqrt{ R_1 } + \sqrt{ R_2 }}
            = 0{,}333,  \\
    \eta_2
            &= \frac{ R_2 }{R_2 + r}
            = \frac{ \sqrt{ R_2 } }{\sqrt{ R_2 } + \sqrt{ R_1 }}
            = 0{,}667
    \end{align*}
}
\solutionspace{120pt}

\tasknumber{4}%
\task{%
    Определите ток, протекающий через резистор $R = 18\,\text{Ом}$ и разность потенциалов на нём (см.
    рис.
    на доске),
    если $r_1 = 2\,\text{Ом}$, $r_2 = 2\,\text{Ом}$, $\mathcal{E}_1 = 60\,\text{В}$, $\mathcal{E}_2 = 20\,\text{В}$
}

\variantsplitter

\addpersonalvariant{Виктор Жилин}

\tasknumber{1}%
\task{%
    На резистор сопротивлением $R = 12\,\text{Ом}$ подали напряжение $U = 240\,\text{В}$.
    Определите ток, который потечёт через резистор, и мощность, выделяющуюся на нём.
}
\answer{%
    \begin{align*}
    \mathcal{I} &= \frac{ U }{ R } = \frac{ 240\,\text{В} }{ 12\,\text{Ом} } = 20{,}00\,\text{А},  \\
    P &= \frac{U^2}{ R } = \frac{ \sqr{ 240\,\text{В} } }{ 12\,\text{Ом} } = 4800{,}00\,\text{Вт}
    \end{align*}
}
\solutionspace{120pt}

\tasknumber{2}%
\task{%
    Замкнутая электрическая цепь состоит из ЭДС $\mathcal{E} = 4\,\text{В}$ и сопротивлением $r$
    и резистора $R = 24\,\text{Ом}$.
    Определите ток, протекающий в цепи.
    Какая тепловая энергия выделится на резисторе за время
    $\tau = 2\,\text{с}$? Какая работа будет совершена ЭДС за это время? Каков знак этой работы? Чему равен КПД цепи? Вычислите значения для 2 случаев:
    $r=0$ и $r = 60\,\text{Ом}$.
}
\answer{%
    \begin{align*}
    \mathcal{I}_1 &= \frac{ \mathcal{E} }{ R } = \frac{ 4\,\text{В} }{ 24\,\text{Ом} } = 0{,}17\,\text{А},  \\
    \mathcal{I}_2 &= \frac{ \mathcal{E} }{R + r} = \frac{ 4\,\text{В} }{24\,\text{Ом} + 60\,\text{Ом}} = 0{,}05\,\text{А},  \\
    Q_1 &= \mathcal{I}_1^2R\tau = \sqr{\frac{ \mathcal{E} }{ R }} R \tau
            = \sqr{\frac{ 4\,\text{В} }{ 24\,\text{Ом} }} \cdot 24\,\text{Ом} \cdot 2\,\text{с} = 1{,}387\,\text{Дж},  \\
    Q_2 &= \mathcal{I}_2^2R\tau = \sqr{\frac{ \mathcal{E} }{R + r}} R \tau
            = \sqr{\frac{ 4\,\text{В} }{24\,\text{Ом} + 60\,\text{Ом}}} \cdot 24\,\text{Ом} \cdot 2\,\text{с} = 0{,}120\,\text{Дж},  \\
    A_1 &= \mathcal{I}_1\tau\mathcal{E} = \frac{ \mathcal{E} }{R} \tau \mathcal{E}
            = \frac{\mathcal{E}^2 \tau}{ R } = \frac{\sqr{ 4\,\text{В} } \cdot 2\,\text{с}}{ 24\,\text{Ом} }
            = 1{,}360\,\text{Дж}, \text{положительна},  \\
    A_2 &= \mathcal{I}_2\tau\mathcal{E} = \frac{ \mathcal{E} }{R + r} \tau \mathcal{E}
            = \frac{\mathcal{E}^2 \tau}{R + r} = \frac{\sqr{ 4\,\text{В} } \cdot 2\,\text{с}}{24\,\text{Ом} + 60\,\text{Ом}}
            = 0{,}400\,\text{Дж}, \text{положительна},  \\
    \eta_1 &= \frac{ Q_1 }{ A_1 } = \ldots = \frac{ R }{ R } = 1,  \\
    \eta_2 &= \frac{ Q_2 }{ A_2 } = \ldots = \frac{ R }{R + r} = 0{,}30
    \end{align*}
}
\solutionspace{120pt}

\tasknumber{3}%
\task{%
    Лампочки, сопротивления которых $R_1 = 0{,}50\,\text{Ом}$ и $R_2 = 18{,}00\,\text{Ом}$, поочерёдно подключённные к некоторому источнику тока,
    потребляют одинаковую мощность.
    Найти внутреннее сопротивление источника и КПД цепи в каждом случае.
}
\answer{%
    \begin{align*}
        P_1 &= \sqr{\frac{ \mathcal{E} }{R_1 + r}}R_1,
        P_2  = \sqr{\frac{ \mathcal{E} }{R_2 + r}}R_2,
        P_1 = P_2 \implies  \\
        &\implies R_1 \sqr{R_2 + r} = R_2 \sqr{R_1 + r} \implies  \\
        &\implies R_1 R_2^2 + 2 R_1 R_2 r + R_1 r^2 =
                    R_2 R_1^2 + 2 R_2 R_1 r + R_2 r^2  \implies  \\
    &\implies r^2 (R_2 - R_1) = R_2^2 R_2 - R_1^2 R_2 \implies  \\
    &\implies r
            = \sqrt{R_1 R_2 \frac{R_2 - R_1}{R_2 - R_1}}
            = \sqrt{R_1 R_2}
            = \sqrt{0{,}50\,\text{Ом} \cdot 18{,}00\,\text{Ом}}
            = 3{,}00\,\text{Ом}.
            \\
    \eta_1
            &= \frac{ R_1 }{R_1 + r}
            = \frac{\sqrt{ R_1 }}{\sqrt{ R_1 } + \sqrt{ R_2 }}
            = 0{,}143,  \\
    \eta_2
            &= \frac{ R_2 }{R_2 + r}
            = \frac{ \sqrt{ R_2 } }{\sqrt{ R_2 } + \sqrt{ R_1 }}
            = 0{,}857
    \end{align*}
}
\solutionspace{120pt}

\tasknumber{4}%
\task{%
    Определите ток, протекающий через резистор $R = 20\,\text{Ом}$ и разность потенциалов на нём (см.
    рис.
    на доске),
    если $r_1 = 3\,\text{Ом}$, $r_2 = 2\,\text{Ом}$, $\mathcal{E}_1 = 20\,\text{В}$, $\mathcal{E}_2 = 40\,\text{В}$
}

\variantsplitter

\addpersonalvariant{Дмитрий Иванов}

\tasknumber{1}%
\task{%
    На резистор сопротивлением $r = 5\,\text{Ом}$ подали напряжение $V = 180\,\text{В}$.
    Определите ток, который потечёт через резистор, и мощность, выделяющуюся на нём.
}
\answer{%
    \begin{align*}
    \mathcal{I} &= \frac{ V }{ r } = \frac{ 180\,\text{В} }{ 5\,\text{Ом} } = 36{,}00\,\text{А},  \\
    P &= \frac{V^2}{ r } = \frac{ \sqr{ 180\,\text{В} } }{ 5\,\text{Ом} } = 6480{,}00\,\text{Вт}
    \end{align*}
}
\solutionspace{120pt}

\tasknumber{2}%
\task{%
    Замкнутая электрическая цепь состоит из ЭДС $\mathcal{E} = 4\,\text{В}$ и сопротивлением $r$
    и резистора $R = 15\,\text{Ом}$.
    Определите ток, протекающий в цепи.
    Какая тепловая энергия выделится на резисторе за время
    $\tau = 10\,\text{с}$? Какая работа будет совершена ЭДС за это время? Каков знак этой работы? Чему равен КПД цепи? Вычислите значения для 2 случаев:
    $r=0$ и $r = 20\,\text{Ом}$.
}
\answer{%
    \begin{align*}
    \mathcal{I}_1 &= \frac{ \mathcal{E} }{ R } = \frac{ 4\,\text{В} }{ 15\,\text{Ом} } = 0{,}27\,\text{А},  \\
    \mathcal{I}_2 &= \frac{ \mathcal{E} }{R + r} = \frac{ 4\,\text{В} }{15\,\text{Ом} + 20\,\text{Ом}} = 0{,}11\,\text{А},  \\
    Q_1 &= \mathcal{I}_1^2R\tau = \sqr{\frac{ \mathcal{E} }{ R }} R \tau
            = \sqr{\frac{ 4\,\text{В} }{ 15\,\text{Ом} }} \cdot 15\,\text{Ом} \cdot 10\,\text{с} = 10{,}935\,\text{Дж},  \\
    Q_2 &= \mathcal{I}_2^2R\tau = \sqr{\frac{ \mathcal{E} }{R + r}} R \tau
            = \sqr{\frac{ 4\,\text{В} }{15\,\text{Ом} + 20\,\text{Ом}}} \cdot 15\,\text{Ом} \cdot 10\,\text{с} = 1{,}815\,\text{Дж},  \\
    A_1 &= \mathcal{I}_1\tau\mathcal{E} = \frac{ \mathcal{E} }{R} \tau \mathcal{E}
            = \frac{\mathcal{E}^2 \tau}{ R } = \frac{\sqr{ 4\,\text{В} } \cdot 10\,\text{с}}{ 15\,\text{Ом} }
            = 10{,}800\,\text{Дж}, \text{положительна},  \\
    A_2 &= \mathcal{I}_2\tau\mathcal{E} = \frac{ \mathcal{E} }{R + r} \tau \mathcal{E}
            = \frac{\mathcal{E}^2 \tau}{R + r} = \frac{\sqr{ 4\,\text{В} } \cdot 10\,\text{с}}{15\,\text{Ом} + 20\,\text{Ом}}
            = 4{,}400\,\text{Дж}, \text{положительна},  \\
    \eta_1 &= \frac{ Q_1 }{ A_1 } = \ldots = \frac{ R }{ R } = 1,  \\
    \eta_2 &= \frac{ Q_2 }{ A_2 } = \ldots = \frac{ R }{R + r} = 0{,}41
    \end{align*}
}
\solutionspace{120pt}

\tasknumber{3}%
\task{%
    Лампочки, сопротивления которых $R_1 = 5{,}00\,\text{Ом}$ и $R_2 = 80{,}00\,\text{Ом}$, поочерёдно подключённные к некоторому источнику тока,
    потребляют одинаковую мощность.
    Найти внутреннее сопротивление источника и КПД цепи в каждом случае.
}
\answer{%
    \begin{align*}
        P_1 &= \sqr{\frac{ \mathcal{E} }{R_1 + r}}R_1,
        P_2  = \sqr{\frac{ \mathcal{E} }{R_2 + r}}R_2,
        P_1 = P_2 \implies  \\
        &\implies R_1 \sqr{R_2 + r} = R_2 \sqr{R_1 + r} \implies  \\
        &\implies R_1 R_2^2 + 2 R_1 R_2 r + R_1 r^2 =
                    R_2 R_1^2 + 2 R_2 R_1 r + R_2 r^2  \implies  \\
    &\implies r^2 (R_2 - R_1) = R_2^2 R_2 - R_1^2 R_2 \implies  \\
    &\implies r
            = \sqrt{R_1 R_2 \frac{R_2 - R_1}{R_2 - R_1}}
            = \sqrt{R_1 R_2}
            = \sqrt{5{,}00\,\text{Ом} \cdot 80{,}00\,\text{Ом}}
            = 20{,}00\,\text{Ом}.
            \\
    \eta_1
            &= \frac{ R_1 }{R_1 + r}
            = \frac{\sqrt{ R_1 }}{\sqrt{ R_1 } + \sqrt{ R_2 }}
            = 0{,}200,  \\
    \eta_2
            &= \frac{ R_2 }{R_2 + r}
            = \frac{ \sqrt{ R_2 } }{\sqrt{ R_2 } + \sqrt{ R_1 }}
            = 0{,}800
    \end{align*}
}
\solutionspace{120pt}

\tasknumber{4}%
\task{%
    Определите ток, протекающий через резистор $R = 10\,\text{Ом}$ и разность потенциалов на нём (см.
    рис.
    на доске),
    если $r_1 = 2\,\text{Ом}$, $r_2 = 1\,\text{Ом}$, $\mathcal{E}_1 = 60\,\text{В}$, $\mathcal{E}_2 = 20\,\text{В}$
}

\variantsplitter

\addpersonalvariant{Олег Климов}

\tasknumber{1}%
\task{%
    На резистор сопротивлением $r = 30\,\text{Ом}$ подали напряжение $V = 240\,\text{В}$.
    Определите ток, который потечёт через резистор, и мощность, выделяющуюся на нём.
}
\answer{%
    \begin{align*}
    \mathcal{I} &= \frac{ V }{ r } = \frac{ 240\,\text{В} }{ 30\,\text{Ом} } = 8{,}00\,\text{А},  \\
    P &= \frac{V^2}{ r } = \frac{ \sqr{ 240\,\text{В} } }{ 30\,\text{Ом} } = 1920{,}00\,\text{Вт}
    \end{align*}
}
\solutionspace{120pt}

\tasknumber{2}%
\task{%
    Замкнутая электрическая цепь состоит из ЭДС $\mathcal{E} = 4\,\text{В}$ и сопротивлением $r$
    и резистора $R = 15\,\text{Ом}$.
    Определите ток, протекающий в цепи.
    Какая тепловая энергия выделится на резисторе за время
    $\tau = 10\,\text{с}$? Какая работа будет совершена ЭДС за это время? Каков знак этой работы? Чему равен КПД цепи? Вычислите значения для 2 случаев:
    $r=0$ и $r = 20\,\text{Ом}$.
}
\answer{%
    \begin{align*}
    \mathcal{I}_1 &= \frac{ \mathcal{E} }{ R } = \frac{ 4\,\text{В} }{ 15\,\text{Ом} } = 0{,}27\,\text{А},  \\
    \mathcal{I}_2 &= \frac{ \mathcal{E} }{R + r} = \frac{ 4\,\text{В} }{15\,\text{Ом} + 20\,\text{Ом}} = 0{,}11\,\text{А},  \\
    Q_1 &= \mathcal{I}_1^2R\tau = \sqr{\frac{ \mathcal{E} }{ R }} R \tau
            = \sqr{\frac{ 4\,\text{В} }{ 15\,\text{Ом} }} \cdot 15\,\text{Ом} \cdot 10\,\text{с} = 10{,}935\,\text{Дж},  \\
    Q_2 &= \mathcal{I}_2^2R\tau = \sqr{\frac{ \mathcal{E} }{R + r}} R \tau
            = \sqr{\frac{ 4\,\text{В} }{15\,\text{Ом} + 20\,\text{Ом}}} \cdot 15\,\text{Ом} \cdot 10\,\text{с} = 1{,}815\,\text{Дж},  \\
    A_1 &= \mathcal{I}_1\tau\mathcal{E} = \frac{ \mathcal{E} }{R} \tau \mathcal{E}
            = \frac{\mathcal{E}^2 \tau}{ R } = \frac{\sqr{ 4\,\text{В} } \cdot 10\,\text{с}}{ 15\,\text{Ом} }
            = 10{,}800\,\text{Дж}, \text{положительна},  \\
    A_2 &= \mathcal{I}_2\tau\mathcal{E} = \frac{ \mathcal{E} }{R + r} \tau \mathcal{E}
            = \frac{\mathcal{E}^2 \tau}{R + r} = \frac{\sqr{ 4\,\text{В} } \cdot 10\,\text{с}}{15\,\text{Ом} + 20\,\text{Ом}}
            = 4{,}400\,\text{Дж}, \text{положительна},  \\
    \eta_1 &= \frac{ Q_1 }{ A_1 } = \ldots = \frac{ R }{ R } = 1,  \\
    \eta_2 &= \frac{ Q_2 }{ A_2 } = \ldots = \frac{ R }{R + r} = 0{,}41
    \end{align*}
}
\solutionspace{120pt}

\tasknumber{3}%
\task{%
    Лампочки, сопротивления которых $R_1 = 4{,}00\,\text{Ом}$ и $R_2 = 36{,}00\,\text{Ом}$, поочерёдно подключённные к некоторому источнику тока,
    потребляют одинаковую мощность.
    Найти внутреннее сопротивление источника и КПД цепи в каждом случае.
}
\answer{%
    \begin{align*}
        P_1 &= \sqr{\frac{ \mathcal{E} }{R_1 + r}}R_1,
        P_2  = \sqr{\frac{ \mathcal{E} }{R_2 + r}}R_2,
        P_1 = P_2 \implies  \\
        &\implies R_1 \sqr{R_2 + r} = R_2 \sqr{R_1 + r} \implies  \\
        &\implies R_1 R_2^2 + 2 R_1 R_2 r + R_1 r^2 =
                    R_2 R_1^2 + 2 R_2 R_1 r + R_2 r^2  \implies  \\
    &\implies r^2 (R_2 - R_1) = R_2^2 R_2 - R_1^2 R_2 \implies  \\
    &\implies r
            = \sqrt{R_1 R_2 \frac{R_2 - R_1}{R_2 - R_1}}
            = \sqrt{R_1 R_2}
            = \sqrt{4{,}00\,\text{Ом} \cdot 36{,}00\,\text{Ом}}
            = 12{,}00\,\text{Ом}.
            \\
    \eta_1
            &= \frac{ R_1 }{R_1 + r}
            = \frac{\sqrt{ R_1 }}{\sqrt{ R_1 } + \sqrt{ R_2 }}
            = 0{,}250,  \\
    \eta_2
            &= \frac{ R_2 }{R_2 + r}
            = \frac{ \sqrt{ R_2 } }{\sqrt{ R_2 } + \sqrt{ R_1 }}
            = 0{,}750
    \end{align*}
}
\solutionspace{120pt}

\tasknumber{4}%
\task{%
    Определите ток, протекающий через резистор $R = 15\,\text{Ом}$ и разность потенциалов на нём (см.
    рис.
    на доске),
    если $r_1 = 2\,\text{Ом}$, $r_2 = 2\,\text{Ом}$, $\mathcal{E}_1 = 60\,\text{В}$, $\mathcal{E}_2 = 40\,\text{В}$
}

\variantsplitter

\addpersonalvariant{Анна Ковалева}

\tasknumber{1}%
\task{%
    На резистор сопротивлением $R = 12\,\text{Ом}$ подали напряжение $U = 150\,\text{В}$.
    Определите ток, который потечёт через резистор, и мощность, выделяющуюся на нём.
}
\answer{%
    \begin{align*}
    \mathcal{I} &= \frac{ U }{ R } = \frac{ 150\,\text{В} }{ 12\,\text{Ом} } = 12{,}50\,\text{А},  \\
    P &= \frac{U^2}{ R } = \frac{ \sqr{ 150\,\text{В} } }{ 12\,\text{Ом} } = 1875{,}00\,\text{Вт}
    \end{align*}
}
\solutionspace{120pt}

\tasknumber{2}%
\task{%
    Замкнутая электрическая цепь состоит из ЭДС $\mathcal{E} = 1\,\text{В}$ и сопротивлением $r$
    и резистора $R = 30\,\text{Ом}$.
    Определите ток, протекающий в цепи.
    Какая тепловая энергия выделится на резисторе за время
    $\tau = 10\,\text{с}$? Какая работа будет совершена ЭДС за это время? Каков знак этой работы? Чему равен КПД цепи? Вычислите значения для 2 случаев:
    $r=0$ и $r = 10\,\text{Ом}$.
}
\answer{%
    \begin{align*}
    \mathcal{I}_1 &= \frac{ \mathcal{E} }{ R } = \frac{ 1\,\text{В} }{ 30\,\text{Ом} } = 0{,}03\,\text{А},  \\
    \mathcal{I}_2 &= \frac{ \mathcal{E} }{R + r} = \frac{ 1\,\text{В} }{30\,\text{Ом} + 10\,\text{Ом}} = 0{,}03\,\text{А},  \\
    Q_1 &= \mathcal{I}_1^2R\tau = \sqr{\frac{ \mathcal{E} }{ R }} R \tau
            = \sqr{\frac{ 1\,\text{В} }{ 30\,\text{Ом} }} \cdot 30\,\text{Ом} \cdot 10\,\text{с} = 0{,}270\,\text{Дж},  \\
    Q_2 &= \mathcal{I}_2^2R\tau = \sqr{\frac{ \mathcal{E} }{R + r}} R \tau
            = \sqr{\frac{ 1\,\text{В} }{30\,\text{Ом} + 10\,\text{Ом}}} \cdot 30\,\text{Ом} \cdot 10\,\text{с} = 0{,}270\,\text{Дж},  \\
    A_1 &= \mathcal{I}_1\tau\mathcal{E} = \frac{ \mathcal{E} }{R} \tau \mathcal{E}
            = \frac{\mathcal{E}^2 \tau}{ R } = \frac{\sqr{ 1\,\text{В} } \cdot 10\,\text{с}}{ 30\,\text{Ом} }
            = 0{,}300\,\text{Дж}, \text{положительна},  \\
    A_2 &= \mathcal{I}_2\tau\mathcal{E} = \frac{ \mathcal{E} }{R + r} \tau \mathcal{E}
            = \frac{\mathcal{E}^2 \tau}{R + r} = \frac{\sqr{ 1\,\text{В} } \cdot 10\,\text{с}}{30\,\text{Ом} + 10\,\text{Ом}}
            = 0{,}300\,\text{Дж}, \text{положительна},  \\
    \eta_1 &= \frac{ Q_1 }{ A_1 } = \ldots = \frac{ R }{ R } = 1,  \\
    \eta_2 &= \frac{ Q_2 }{ A_2 } = \ldots = \frac{ R }{R + r} = 0{,}90
    \end{align*}
}
\solutionspace{120pt}

\tasknumber{3}%
\task{%
    Лампочки, сопротивления которых $R_1 = 1{,}00\,\text{Ом}$ и $R_2 = 49{,}00\,\text{Ом}$, поочерёдно подключённные к некоторому источнику тока,
    потребляют одинаковую мощность.
    Найти внутреннее сопротивление источника и КПД цепи в каждом случае.
}
\answer{%
    \begin{align*}
        P_1 &= \sqr{\frac{ \mathcal{E} }{R_1 + r}}R_1,
        P_2  = \sqr{\frac{ \mathcal{E} }{R_2 + r}}R_2,
        P_1 = P_2 \implies  \\
        &\implies R_1 \sqr{R_2 + r} = R_2 \sqr{R_1 + r} \implies  \\
        &\implies R_1 R_2^2 + 2 R_1 R_2 r + R_1 r^2 =
                    R_2 R_1^2 + 2 R_2 R_1 r + R_2 r^2  \implies  \\
    &\implies r^2 (R_2 - R_1) = R_2^2 R_2 - R_1^2 R_2 \implies  \\
    &\implies r
            = \sqrt{R_1 R_2 \frac{R_2 - R_1}{R_2 - R_1}}
            = \sqrt{R_1 R_2}
            = \sqrt{1{,}00\,\text{Ом} \cdot 49{,}00\,\text{Ом}}
            = 7{,}00\,\text{Ом}.
            \\
    \eta_1
            &= \frac{ R_1 }{R_1 + r}
            = \frac{\sqrt{ R_1 }}{\sqrt{ R_1 } + \sqrt{ R_2 }}
            = 0{,}125,  \\
    \eta_2
            &= \frac{ R_2 }{R_2 + r}
            = \frac{ \sqrt{ R_2 } }{\sqrt{ R_2 } + \sqrt{ R_1 }}
            = 0{,}875
    \end{align*}
}
\solutionspace{120pt}

\tasknumber{4}%
\task{%
    Определите ток, протекающий через резистор $R = 10\,\text{Ом}$ и разность потенциалов на нём (см.
    рис.
    на доске),
    если $r_1 = 1\,\text{Ом}$, $r_2 = 3\,\text{Ом}$, $\mathcal{E}_1 = 40\,\text{В}$, $\mathcal{E}_2 = 20\,\text{В}$
}

\variantsplitter

\addpersonalvariant{Глеб Ковылин}

\tasknumber{1}%
\task{%
    На резистор сопротивлением $r = 30\,\text{Ом}$ подали напряжение $U = 240\,\text{В}$.
    Определите ток, который потечёт через резистор, и мощность, выделяющуюся на нём.
}
\answer{%
    \begin{align*}
    \mathcal{I} &= \frac{ U }{ r } = \frac{ 240\,\text{В} }{ 30\,\text{Ом} } = 8{,}00\,\text{А},  \\
    P &= \frac{U^2}{ r } = \frac{ \sqr{ 240\,\text{В} } }{ 30\,\text{Ом} } = 1920{,}00\,\text{Вт}
    \end{align*}
}
\solutionspace{120pt}

\tasknumber{2}%
\task{%
    Замкнутая электрическая цепь состоит из ЭДС $\mathcal{E} = 4\,\text{В}$ и сопротивлением $r$
    и резистора $R = 10\,\text{Ом}$.
    Определите ток, протекающий в цепи.
    Какая тепловая энергия выделится на резисторе за время
    $\tau = 10\,\text{с}$? Какая работа будет совершена ЭДС за это время? Каков знак этой работы? Чему равен КПД цепи? Вычислите значения для 2 случаев:
    $r=0$ и $r = 60\,\text{Ом}$.
}
\answer{%
    \begin{align*}
    \mathcal{I}_1 &= \frac{ \mathcal{E} }{ R } = \frac{ 4\,\text{В} }{ 10\,\text{Ом} } = 0{,}40\,\text{А},  \\
    \mathcal{I}_2 &= \frac{ \mathcal{E} }{R + r} = \frac{ 4\,\text{В} }{10\,\text{Ом} + 60\,\text{Ом}} = 0{,}06\,\text{А},  \\
    Q_1 &= \mathcal{I}_1^2R\tau = \sqr{\frac{ \mathcal{E} }{ R }} R \tau
            = \sqr{\frac{ 4\,\text{В} }{ 10\,\text{Ом} }} \cdot 10\,\text{Ом} \cdot 10\,\text{с} = 16{,}000\,\text{Дж},  \\
    Q_2 &= \mathcal{I}_2^2R\tau = \sqr{\frac{ \mathcal{E} }{R + r}} R \tau
            = \sqr{\frac{ 4\,\text{В} }{10\,\text{Ом} + 60\,\text{Ом}}} \cdot 10\,\text{Ом} \cdot 10\,\text{с} = 0{,}360\,\text{Дж},  \\
    A_1 &= \mathcal{I}_1\tau\mathcal{E} = \frac{ \mathcal{E} }{R} \tau \mathcal{E}
            = \frac{\mathcal{E}^2 \tau}{ R } = \frac{\sqr{ 4\,\text{В} } \cdot 10\,\text{с}}{ 10\,\text{Ом} }
            = 16{,}000\,\text{Дж}, \text{положительна},  \\
    A_2 &= \mathcal{I}_2\tau\mathcal{E} = \frac{ \mathcal{E} }{R + r} \tau \mathcal{E}
            = \frac{\mathcal{E}^2 \tau}{R + r} = \frac{\sqr{ 4\,\text{В} } \cdot 10\,\text{с}}{10\,\text{Ом} + 60\,\text{Ом}}
            = 2{,}400\,\text{Дж}, \text{положительна},  \\
    \eta_1 &= \frac{ Q_1 }{ A_1 } = \ldots = \frac{ R }{ R } = 1,  \\
    \eta_2 &= \frac{ Q_2 }{ A_2 } = \ldots = \frac{ R }{R + r} = 0{,}15
    \end{align*}
}
\solutionspace{120pt}

\tasknumber{3}%
\task{%
    Лампочки, сопротивления которых $R_1 = 0{,}25\,\text{Ом}$ и $R_2 = 64{,}00\,\text{Ом}$, поочерёдно подключённные к некоторому источнику тока,
    потребляют одинаковую мощность.
    Найти внутреннее сопротивление источника и КПД цепи в каждом случае.
}
\answer{%
    \begin{align*}
        P_1 &= \sqr{\frac{ \mathcal{E} }{R_1 + r}}R_1,
        P_2  = \sqr{\frac{ \mathcal{E} }{R_2 + r}}R_2,
        P_1 = P_2 \implies  \\
        &\implies R_1 \sqr{R_2 + r} = R_2 \sqr{R_1 + r} \implies  \\
        &\implies R_1 R_2^2 + 2 R_1 R_2 r + R_1 r^2 =
                    R_2 R_1^2 + 2 R_2 R_1 r + R_2 r^2  \implies  \\
    &\implies r^2 (R_2 - R_1) = R_2^2 R_2 - R_1^2 R_2 \implies  \\
    &\implies r
            = \sqrt{R_1 R_2 \frac{R_2 - R_1}{R_2 - R_1}}
            = \sqrt{R_1 R_2}
            = \sqrt{0{,}25\,\text{Ом} \cdot 64{,}00\,\text{Ом}}
            = 4{,}00\,\text{Ом}.
            \\
    \eta_1
            &= \frac{ R_1 }{R_1 + r}
            = \frac{\sqrt{ R_1 }}{\sqrt{ R_1 } + \sqrt{ R_2 }}
            = 0{,}059,  \\
    \eta_2
            &= \frac{ R_2 }{R_2 + r}
            = \frac{ \sqrt{ R_2 } }{\sqrt{ R_2 } + \sqrt{ R_1 }}
            = 0{,}941
    \end{align*}
}
\solutionspace{120pt}

\tasknumber{4}%
\task{%
    Определите ток, протекающий через резистор $R = 10\,\text{Ом}$ и разность потенциалов на нём (см.
    рис.
    на доске),
    если $r_1 = 2\,\text{Ом}$, $r_2 = 2\,\text{Ом}$, $\mathcal{E}_1 = 60\,\text{В}$, $\mathcal{E}_2 = 30\,\text{В}$
}

\variantsplitter

\addpersonalvariant{Даниил Космынин}

\tasknumber{1}%
\task{%
    На резистор сопротивлением $r = 5\,\text{Ом}$ подали напряжение $U = 150\,\text{В}$.
    Определите ток, который потечёт через резистор, и мощность, выделяющуюся на нём.
}
\answer{%
    \begin{align*}
    \mathcal{I} &= \frac{ U }{ r } = \frac{ 150\,\text{В} }{ 5\,\text{Ом} } = 30{,}00\,\text{А},  \\
    P &= \frac{U^2}{ r } = \frac{ \sqr{ 150\,\text{В} } }{ 5\,\text{Ом} } = 4500{,}00\,\text{Вт}
    \end{align*}
}
\solutionspace{120pt}

\tasknumber{2}%
\task{%
    Замкнутая электрическая цепь состоит из ЭДС $\mathcal{E} = 3\,\text{В}$ и сопротивлением $r$
    и резистора $R = 10\,\text{Ом}$.
    Определите ток, протекающий в цепи.
    Какая тепловая энергия выделится на резисторе за время
    $\tau = 2\,\text{с}$? Какая работа будет совершена ЭДС за это время? Каков знак этой работы? Чему равен КПД цепи? Вычислите значения для 2 случаев:
    $r=0$ и $r = 10\,\text{Ом}$.
}
\answer{%
    \begin{align*}
    \mathcal{I}_1 &= \frac{ \mathcal{E} }{ R } = \frac{ 3\,\text{В} }{ 10\,\text{Ом} } = 0{,}30\,\text{А},  \\
    \mathcal{I}_2 &= \frac{ \mathcal{E} }{R + r} = \frac{ 3\,\text{В} }{10\,\text{Ом} + 10\,\text{Ом}} = 0{,}15\,\text{А},  \\
    Q_1 &= \mathcal{I}_1^2R\tau = \sqr{\frac{ \mathcal{E} }{ R }} R \tau
            = \sqr{\frac{ 3\,\text{В} }{ 10\,\text{Ом} }} \cdot 10\,\text{Ом} \cdot 2\,\text{с} = 1{,}800\,\text{Дж},  \\
    Q_2 &= \mathcal{I}_2^2R\tau = \sqr{\frac{ \mathcal{E} }{R + r}} R \tau
            = \sqr{\frac{ 3\,\text{В} }{10\,\text{Ом} + 10\,\text{Ом}}} \cdot 10\,\text{Ом} \cdot 2\,\text{с} = 0{,}450\,\text{Дж},  \\
    A_1 &= \mathcal{I}_1\tau\mathcal{E} = \frac{ \mathcal{E} }{R} \tau \mathcal{E}
            = \frac{\mathcal{E}^2 \tau}{ R } = \frac{\sqr{ 3\,\text{В} } \cdot 2\,\text{с}}{ 10\,\text{Ом} }
            = 1{,}800\,\text{Дж}, \text{положительна},  \\
    A_2 &= \mathcal{I}_2\tau\mathcal{E} = \frac{ \mathcal{E} }{R + r} \tau \mathcal{E}
            = \frac{\mathcal{E}^2 \tau}{R + r} = \frac{\sqr{ 3\,\text{В} } \cdot 2\,\text{с}}{10\,\text{Ом} + 10\,\text{Ом}}
            = 0{,}900\,\text{Дж}, \text{положительна},  \\
    \eta_1 &= \frac{ Q_1 }{ A_1 } = \ldots = \frac{ R }{ R } = 1,  \\
    \eta_2 &= \frac{ Q_2 }{ A_2 } = \ldots = \frac{ R }{R + r} = 0{,}50
    \end{align*}
}
\solutionspace{120pt}

\tasknumber{3}%
\task{%
    Лампочки, сопротивления которых $R_1 = 0{,}50\,\text{Ом}$ и $R_2 = 18{,}00\,\text{Ом}$, поочерёдно подключённные к некоторому источнику тока,
    потребляют одинаковую мощность.
    Найти внутреннее сопротивление источника и КПД цепи в каждом случае.
}
\answer{%
    \begin{align*}
        P_1 &= \sqr{\frac{ \mathcal{E} }{R_1 + r}}R_1,
        P_2  = \sqr{\frac{ \mathcal{E} }{R_2 + r}}R_2,
        P_1 = P_2 \implies  \\
        &\implies R_1 \sqr{R_2 + r} = R_2 \sqr{R_1 + r} \implies  \\
        &\implies R_1 R_2^2 + 2 R_1 R_2 r + R_1 r^2 =
                    R_2 R_1^2 + 2 R_2 R_1 r + R_2 r^2  \implies  \\
    &\implies r^2 (R_2 - R_1) = R_2^2 R_2 - R_1^2 R_2 \implies  \\
    &\implies r
            = \sqrt{R_1 R_2 \frac{R_2 - R_1}{R_2 - R_1}}
            = \sqrt{R_1 R_2}
            = \sqrt{0{,}50\,\text{Ом} \cdot 18{,}00\,\text{Ом}}
            = 3{,}00\,\text{Ом}.
            \\
    \eta_1
            &= \frac{ R_1 }{R_1 + r}
            = \frac{\sqrt{ R_1 }}{\sqrt{ R_1 } + \sqrt{ R_2 }}
            = 0{,}143,  \\
    \eta_2
            &= \frac{ R_2 }{R_2 + r}
            = \frac{ \sqrt{ R_2 } }{\sqrt{ R_2 } + \sqrt{ R_1 }}
            = 0{,}857
    \end{align*}
}
\solutionspace{120pt}

\tasknumber{4}%
\task{%
    Определите ток, протекающий через резистор $R = 20\,\text{Ом}$ и разность потенциалов на нём (см.
    рис.
    на доске),
    если $r_1 = 2\,\text{Ом}$, $r_2 = 2\,\text{Ом}$, $\mathcal{E}_1 = 20\,\text{В}$, $\mathcal{E}_2 = 40\,\text{В}$
}

\variantsplitter

\addpersonalvariant{Алина Леоничева}

\tasknumber{1}%
\task{%
    На резистор сопротивлением $r = 12\,\text{Ом}$ подали напряжение $V = 180\,\text{В}$.
    Определите ток, который потечёт через резистор, и мощность, выделяющуюся на нём.
}
\answer{%
    \begin{align*}
    \mathcal{I} &= \frac{ V }{ r } = \frac{ 180\,\text{В} }{ 12\,\text{Ом} } = 15{,}00\,\text{А},  \\
    P &= \frac{V^2}{ r } = \frac{ \sqr{ 180\,\text{В} } }{ 12\,\text{Ом} } = 2700{,}00\,\text{Вт}
    \end{align*}
}
\solutionspace{120pt}

\tasknumber{2}%
\task{%
    Замкнутая электрическая цепь состоит из ЭДС $\mathcal{E} = 4\,\text{В}$ и сопротивлением $r$
    и резистора $R = 10\,\text{Ом}$.
    Определите ток, протекающий в цепи.
    Какая тепловая энергия выделится на резисторе за время
    $\tau = 2\,\text{с}$? Какая работа будет совершена ЭДС за это время? Каков знак этой работы? Чему равен КПД цепи? Вычислите значения для 2 случаев:
    $r=0$ и $r = 60\,\text{Ом}$.
}
\answer{%
    \begin{align*}
    \mathcal{I}_1 &= \frac{ \mathcal{E} }{ R } = \frac{ 4\,\text{В} }{ 10\,\text{Ом} } = 0{,}40\,\text{А},  \\
    \mathcal{I}_2 &= \frac{ \mathcal{E} }{R + r} = \frac{ 4\,\text{В} }{10\,\text{Ом} + 60\,\text{Ом}} = 0{,}06\,\text{А},  \\
    Q_1 &= \mathcal{I}_1^2R\tau = \sqr{\frac{ \mathcal{E} }{ R }} R \tau
            = \sqr{\frac{ 4\,\text{В} }{ 10\,\text{Ом} }} \cdot 10\,\text{Ом} \cdot 2\,\text{с} = 3{,}200\,\text{Дж},  \\
    Q_2 &= \mathcal{I}_2^2R\tau = \sqr{\frac{ \mathcal{E} }{R + r}} R \tau
            = \sqr{\frac{ 4\,\text{В} }{10\,\text{Ом} + 60\,\text{Ом}}} \cdot 10\,\text{Ом} \cdot 2\,\text{с} = 0{,}072\,\text{Дж},  \\
    A_1 &= \mathcal{I}_1\tau\mathcal{E} = \frac{ \mathcal{E} }{R} \tau \mathcal{E}
            = \frac{\mathcal{E}^2 \tau}{ R } = \frac{\sqr{ 4\,\text{В} } \cdot 2\,\text{с}}{ 10\,\text{Ом} }
            = 3{,}200\,\text{Дж}, \text{положительна},  \\
    A_2 &= \mathcal{I}_2\tau\mathcal{E} = \frac{ \mathcal{E} }{R + r} \tau \mathcal{E}
            = \frac{\mathcal{E}^2 \tau}{R + r} = \frac{\sqr{ 4\,\text{В} } \cdot 2\,\text{с}}{10\,\text{Ом} + 60\,\text{Ом}}
            = 0{,}480\,\text{Дж}, \text{положительна},  \\
    \eta_1 &= \frac{ Q_1 }{ A_1 } = \ldots = \frac{ R }{ R } = 1,  \\
    \eta_2 &= \frac{ Q_2 }{ A_2 } = \ldots = \frac{ R }{R + r} = 0{,}15
    \end{align*}
}
\solutionspace{120pt}

\tasknumber{3}%
\task{%
    Лампочки, сопротивления которых $R_1 = 0{,}25\,\text{Ом}$ и $R_2 = 16{,}00\,\text{Ом}$, поочерёдно подключённные к некоторому источнику тока,
    потребляют одинаковую мощность.
    Найти внутреннее сопротивление источника и КПД цепи в каждом случае.
}
\answer{%
    \begin{align*}
        P_1 &= \sqr{\frac{ \mathcal{E} }{R_1 + r}}R_1,
        P_2  = \sqr{\frac{ \mathcal{E} }{R_2 + r}}R_2,
        P_1 = P_2 \implies  \\
        &\implies R_1 \sqr{R_2 + r} = R_2 \sqr{R_1 + r} \implies  \\
        &\implies R_1 R_2^2 + 2 R_1 R_2 r + R_1 r^2 =
                    R_2 R_1^2 + 2 R_2 R_1 r + R_2 r^2  \implies  \\
    &\implies r^2 (R_2 - R_1) = R_2^2 R_2 - R_1^2 R_2 \implies  \\
    &\implies r
            = \sqrt{R_1 R_2 \frac{R_2 - R_1}{R_2 - R_1}}
            = \sqrt{R_1 R_2}
            = \sqrt{0{,}25\,\text{Ом} \cdot 16{,}00\,\text{Ом}}
            = 2{,}00\,\text{Ом}.
            \\
    \eta_1
            &= \frac{ R_1 }{R_1 + r}
            = \frac{\sqrt{ R_1 }}{\sqrt{ R_1 } + \sqrt{ R_2 }}
            = 0{,}111,  \\
    \eta_2
            &= \frac{ R_2 }{R_2 + r}
            = \frac{ \sqrt{ R_2 } }{\sqrt{ R_2 } + \sqrt{ R_1 }}
            = 0{,}889
    \end{align*}
}
\solutionspace{120pt}

\tasknumber{4}%
\task{%
    Определите ток, протекающий через резистор $R = 12\,\text{Ом}$ и разность потенциалов на нём (см.
    рис.
    на доске),
    если $r_1 = 3\,\text{Ом}$, $r_2 = 3\,\text{Ом}$, $\mathcal{E}_1 = 60\,\text{В}$, $\mathcal{E}_2 = 20\,\text{В}$
}

\variantsplitter

\addpersonalvariant{Ирина Лин}

\tasknumber{1}%
\task{%
    На резистор сопротивлением $r = 30\,\text{Ом}$ подали напряжение $V = 120\,\text{В}$.
    Определите ток, который потечёт через резистор, и мощность, выделяющуюся на нём.
}
\answer{%
    \begin{align*}
    \mathcal{I} &= \frac{ V }{ r } = \frac{ 120\,\text{В} }{ 30\,\text{Ом} } = 4{,}00\,\text{А},  \\
    P &= \frac{V^2}{ r } = \frac{ \sqr{ 120\,\text{В} } }{ 30\,\text{Ом} } = 480{,}00\,\text{Вт}
    \end{align*}
}
\solutionspace{120pt}

\tasknumber{2}%
\task{%
    Замкнутая электрическая цепь состоит из ЭДС $\mathcal{E} = 4\,\text{В}$ и сопротивлением $r$
    и резистора $R = 15\,\text{Ом}$.
    Определите ток, протекающий в цепи.
    Какая тепловая энергия выделится на резисторе за время
    $\tau = 5\,\text{с}$? Какая работа будет совершена ЭДС за это время? Каков знак этой работы? Чему равен КПД цепи? Вычислите значения для 2 случаев:
    $r=0$ и $r = 30\,\text{Ом}$.
}
\answer{%
    \begin{align*}
    \mathcal{I}_1 &= \frac{ \mathcal{E} }{ R } = \frac{ 4\,\text{В} }{ 15\,\text{Ом} } = 0{,}27\,\text{А},  \\
    \mathcal{I}_2 &= \frac{ \mathcal{E} }{R + r} = \frac{ 4\,\text{В} }{15\,\text{Ом} + 30\,\text{Ом}} = 0{,}09\,\text{А},  \\
    Q_1 &= \mathcal{I}_1^2R\tau = \sqr{\frac{ \mathcal{E} }{ R }} R \tau
            = \sqr{\frac{ 4\,\text{В} }{ 15\,\text{Ом} }} \cdot 15\,\text{Ом} \cdot 5\,\text{с} = 5{,}468\,\text{Дж},  \\
    Q_2 &= \mathcal{I}_2^2R\tau = \sqr{\frac{ \mathcal{E} }{R + r}} R \tau
            = \sqr{\frac{ 4\,\text{В} }{15\,\text{Ом} + 30\,\text{Ом}}} \cdot 15\,\text{Ом} \cdot 5\,\text{с} = 0{,}607\,\text{Дж},  \\
    A_1 &= \mathcal{I}_1\tau\mathcal{E} = \frac{ \mathcal{E} }{R} \tau \mathcal{E}
            = \frac{\mathcal{E}^2 \tau}{ R } = \frac{\sqr{ 4\,\text{В} } \cdot 5\,\text{с}}{ 15\,\text{Ом} }
            = 5{,}400\,\text{Дж}, \text{положительна},  \\
    A_2 &= \mathcal{I}_2\tau\mathcal{E} = \frac{ \mathcal{E} }{R + r} \tau \mathcal{E}
            = \frac{\mathcal{E}^2 \tau}{R + r} = \frac{\sqr{ 4\,\text{В} } \cdot 5\,\text{с}}{15\,\text{Ом} + 30\,\text{Ом}}
            = 1{,}800\,\text{Дж}, \text{положительна},  \\
    \eta_1 &= \frac{ Q_1 }{ A_1 } = \ldots = \frac{ R }{ R } = 1,  \\
    \eta_2 &= \frac{ Q_2 }{ A_2 } = \ldots = \frac{ R }{R + r} = 0{,}34
    \end{align*}
}
\solutionspace{120pt}

\tasknumber{3}%
\task{%
    Лампочки, сопротивления которых $R_1 = 0{,}50\,\text{Ом}$ и $R_2 = 4{,}50\,\text{Ом}$, поочерёдно подключённные к некоторому источнику тока,
    потребляют одинаковую мощность.
    Найти внутреннее сопротивление источника и КПД цепи в каждом случае.
}
\answer{%
    \begin{align*}
        P_1 &= \sqr{\frac{ \mathcal{E} }{R_1 + r}}R_1,
        P_2  = \sqr{\frac{ \mathcal{E} }{R_2 + r}}R_2,
        P_1 = P_2 \implies  \\
        &\implies R_1 \sqr{R_2 + r} = R_2 \sqr{R_1 + r} \implies  \\
        &\implies R_1 R_2^2 + 2 R_1 R_2 r + R_1 r^2 =
                    R_2 R_1^2 + 2 R_2 R_1 r + R_2 r^2  \implies  \\
    &\implies r^2 (R_2 - R_1) = R_2^2 R_2 - R_1^2 R_2 \implies  \\
    &\implies r
            = \sqrt{R_1 R_2 \frac{R_2 - R_1}{R_2 - R_1}}
            = \sqrt{R_1 R_2}
            = \sqrt{0{,}50\,\text{Ом} \cdot 4{,}50\,\text{Ом}}
            = 1{,}50\,\text{Ом}.
            \\
    \eta_1
            &= \frac{ R_1 }{R_1 + r}
            = \frac{\sqrt{ R_1 }}{\sqrt{ R_1 } + \sqrt{ R_2 }}
            = 0{,}250,  \\
    \eta_2
            &= \frac{ R_2 }{R_2 + r}
            = \frac{ \sqrt{ R_2 } }{\sqrt{ R_2 } + \sqrt{ R_1 }}
            = 0{,}750
    \end{align*}
}
\solutionspace{120pt}

\tasknumber{4}%
\task{%
    Определите ток, протекающий через резистор $R = 18\,\text{Ом}$ и разность потенциалов на нём (см.
    рис.
    на доске),
    если $r_1 = 2\,\text{Ом}$, $r_2 = 3\,\text{Ом}$, $\mathcal{E}_1 = 30\,\text{В}$, $\mathcal{E}_2 = 30\,\text{В}$
}

\variantsplitter

\addpersonalvariant{Олег Мальцев}

\tasknumber{1}%
\task{%
    На резистор сопротивлением $R = 5\,\text{Ом}$ подали напряжение $U = 180\,\text{В}$.
    Определите ток, который потечёт через резистор, и мощность, выделяющуюся на нём.
}
\answer{%
    \begin{align*}
    \mathcal{I} &= \frac{ U }{ R } = \frac{ 180\,\text{В} }{ 5\,\text{Ом} } = 36{,}00\,\text{А},  \\
    P &= \frac{U^2}{ R } = \frac{ \sqr{ 180\,\text{В} } }{ 5\,\text{Ом} } = 6480{,}00\,\text{Вт}
    \end{align*}
}
\solutionspace{120pt}

\tasknumber{2}%
\task{%
    Замкнутая электрическая цепь состоит из ЭДС $\mathcal{E} = 3\,\text{В}$ и сопротивлением $r$
    и резистора $R = 30\,\text{Ом}$.
    Определите ток, протекающий в цепи.
    Какая тепловая энергия выделится на резисторе за время
    $\tau = 10\,\text{с}$? Какая работа будет совершена ЭДС за это время? Каков знак этой работы? Чему равен КПД цепи? Вычислите значения для 2 случаев:
    $r=0$ и $r = 60\,\text{Ом}$.
}
\answer{%
    \begin{align*}
    \mathcal{I}_1 &= \frac{ \mathcal{E} }{ R } = \frac{ 3\,\text{В} }{ 30\,\text{Ом} } = 0{,}10\,\text{А},  \\
    \mathcal{I}_2 &= \frac{ \mathcal{E} }{R + r} = \frac{ 3\,\text{В} }{30\,\text{Ом} + 60\,\text{Ом}} = 0{,}03\,\text{А},  \\
    Q_1 &= \mathcal{I}_1^2R\tau = \sqr{\frac{ \mathcal{E} }{ R }} R \tau
            = \sqr{\frac{ 3\,\text{В} }{ 30\,\text{Ом} }} \cdot 30\,\text{Ом} \cdot 10\,\text{с} = 3{,}000\,\text{Дж},  \\
    Q_2 &= \mathcal{I}_2^2R\tau = \sqr{\frac{ \mathcal{E} }{R + r}} R \tau
            = \sqr{\frac{ 3\,\text{В} }{30\,\text{Ом} + 60\,\text{Ом}}} \cdot 30\,\text{Ом} \cdot 10\,\text{с} = 0{,}270\,\text{Дж},  \\
    A_1 &= \mathcal{I}_1\tau\mathcal{E} = \frac{ \mathcal{E} }{R} \tau \mathcal{E}
            = \frac{\mathcal{E}^2 \tau}{ R } = \frac{\sqr{ 3\,\text{В} } \cdot 10\,\text{с}}{ 30\,\text{Ом} }
            = 3{,}000\,\text{Дж}, \text{положительна},  \\
    A_2 &= \mathcal{I}_2\tau\mathcal{E} = \frac{ \mathcal{E} }{R + r} \tau \mathcal{E}
            = \frac{\mathcal{E}^2 \tau}{R + r} = \frac{\sqr{ 3\,\text{В} } \cdot 10\,\text{с}}{30\,\text{Ом} + 60\,\text{Ом}}
            = 0{,}900\,\text{Дж}, \text{положительна},  \\
    \eta_1 &= \frac{ Q_1 }{ A_1 } = \ldots = \frac{ R }{ R } = 1,  \\
    \eta_2 &= \frac{ Q_2 }{ A_2 } = \ldots = \frac{ R }{R + r} = 0{,}30
    \end{align*}
}
\solutionspace{120pt}

\tasknumber{3}%
\task{%
    Лампочки, сопротивления которых $R_1 = 0{,}50\,\text{Ом}$ и $R_2 = 2{,}00\,\text{Ом}$, поочерёдно подключённные к некоторому источнику тока,
    потребляют одинаковую мощность.
    Найти внутреннее сопротивление источника и КПД цепи в каждом случае.
}
\answer{%
    \begin{align*}
        P_1 &= \sqr{\frac{ \mathcal{E} }{R_1 + r}}R_1,
        P_2  = \sqr{\frac{ \mathcal{E} }{R_2 + r}}R_2,
        P_1 = P_2 \implies  \\
        &\implies R_1 \sqr{R_2 + r} = R_2 \sqr{R_1 + r} \implies  \\
        &\implies R_1 R_2^2 + 2 R_1 R_2 r + R_1 r^2 =
                    R_2 R_1^2 + 2 R_2 R_1 r + R_2 r^2  \implies  \\
    &\implies r^2 (R_2 - R_1) = R_2^2 R_2 - R_1^2 R_2 \implies  \\
    &\implies r
            = \sqrt{R_1 R_2 \frac{R_2 - R_1}{R_2 - R_1}}
            = \sqrt{R_1 R_2}
            = \sqrt{0{,}50\,\text{Ом} \cdot 2{,}00\,\text{Ом}}
            = 1{,}00\,\text{Ом}.
            \\
    \eta_1
            &= \frac{ R_1 }{R_1 + r}
            = \frac{\sqrt{ R_1 }}{\sqrt{ R_1 } + \sqrt{ R_2 }}
            = 0{,}333,  \\
    \eta_2
            &= \frac{ R_2 }{R_2 + r}
            = \frac{ \sqrt{ R_2 } }{\sqrt{ R_2 } + \sqrt{ R_1 }}
            = 0{,}667
    \end{align*}
}
\solutionspace{120pt}

\tasknumber{4}%
\task{%
    Определите ток, протекающий через резистор $R = 12\,\text{Ом}$ и разность потенциалов на нём (см.
    рис.
    на доске),
    если $r_1 = 3\,\text{Ом}$, $r_2 = 3\,\text{Ом}$, $\mathcal{E}_1 = 60\,\text{В}$, $\mathcal{E}_2 = 30\,\text{В}$
}

\variantsplitter

\addpersonalvariant{Ислам Мунаев}

\tasknumber{1}%
\task{%
    На резистор сопротивлением $R = 30\,\text{Ом}$ подали напряжение $U = 240\,\text{В}$.
    Определите ток, который потечёт через резистор, и мощность, выделяющуюся на нём.
}
\answer{%
    \begin{align*}
    \mathcal{I} &= \frac{ U }{ R } = \frac{ 240\,\text{В} }{ 30\,\text{Ом} } = 8{,}00\,\text{А},  \\
    P &= \frac{U^2}{ R } = \frac{ \sqr{ 240\,\text{В} } }{ 30\,\text{Ом} } = 1920{,}00\,\text{Вт}
    \end{align*}
}
\solutionspace{120pt}

\tasknumber{2}%
\task{%
    Замкнутая электрическая цепь состоит из ЭДС $\mathcal{E} = 3\,\text{В}$ и сопротивлением $r$
    и резистора $R = 10\,\text{Ом}$.
    Определите ток, протекающий в цепи.
    Какая тепловая энергия выделится на резисторе за время
    $\tau = 10\,\text{с}$? Какая работа будет совершена ЭДС за это время? Каков знак этой работы? Чему равен КПД цепи? Вычислите значения для 2 случаев:
    $r=0$ и $r = 60\,\text{Ом}$.
}
\answer{%
    \begin{align*}
    \mathcal{I}_1 &= \frac{ \mathcal{E} }{ R } = \frac{ 3\,\text{В} }{ 10\,\text{Ом} } = 0{,}30\,\text{А},  \\
    \mathcal{I}_2 &= \frac{ \mathcal{E} }{R + r} = \frac{ 3\,\text{В} }{10\,\text{Ом} + 60\,\text{Ом}} = 0{,}04\,\text{А},  \\
    Q_1 &= \mathcal{I}_1^2R\tau = \sqr{\frac{ \mathcal{E} }{ R }} R \tau
            = \sqr{\frac{ 3\,\text{В} }{ 10\,\text{Ом} }} \cdot 10\,\text{Ом} \cdot 10\,\text{с} = 9{,}000\,\text{Дж},  \\
    Q_2 &= \mathcal{I}_2^2R\tau = \sqr{\frac{ \mathcal{E} }{R + r}} R \tau
            = \sqr{\frac{ 3\,\text{В} }{10\,\text{Ом} + 60\,\text{Ом}}} \cdot 10\,\text{Ом} \cdot 10\,\text{с} = 0{,}160\,\text{Дж},  \\
    A_1 &= \mathcal{I}_1\tau\mathcal{E} = \frac{ \mathcal{E} }{R} \tau \mathcal{E}
            = \frac{\mathcal{E}^2 \tau}{ R } = \frac{\sqr{ 3\,\text{В} } \cdot 10\,\text{с}}{ 10\,\text{Ом} }
            = 9{,}000\,\text{Дж}, \text{положительна},  \\
    A_2 &= \mathcal{I}_2\tau\mathcal{E} = \frac{ \mathcal{E} }{R + r} \tau \mathcal{E}
            = \frac{\mathcal{E}^2 \tau}{R + r} = \frac{\sqr{ 3\,\text{В} } \cdot 10\,\text{с}}{10\,\text{Ом} + 60\,\text{Ом}}
            = 1{,}200\,\text{Дж}, \text{положительна},  \\
    \eta_1 &= \frac{ Q_1 }{ A_1 } = \ldots = \frac{ R }{ R } = 1,  \\
    \eta_2 &= \frac{ Q_2 }{ A_2 } = \ldots = \frac{ R }{R + r} = 0{,}13
    \end{align*}
}
\solutionspace{120pt}

\tasknumber{3}%
\task{%
    Лампочки, сопротивления которых $R_1 = 0{,}50\,\text{Ом}$ и $R_2 = 18{,}00\,\text{Ом}$, поочерёдно подключённные к некоторому источнику тока,
    потребляют одинаковую мощность.
    Найти внутреннее сопротивление источника и КПД цепи в каждом случае.
}
\answer{%
    \begin{align*}
        P_1 &= \sqr{\frac{ \mathcal{E} }{R_1 + r}}R_1,
        P_2  = \sqr{\frac{ \mathcal{E} }{R_2 + r}}R_2,
        P_1 = P_2 \implies  \\
        &\implies R_1 \sqr{R_2 + r} = R_2 \sqr{R_1 + r} \implies  \\
        &\implies R_1 R_2^2 + 2 R_1 R_2 r + R_1 r^2 =
                    R_2 R_1^2 + 2 R_2 R_1 r + R_2 r^2  \implies  \\
    &\implies r^2 (R_2 - R_1) = R_2^2 R_2 - R_1^2 R_2 \implies  \\
    &\implies r
            = \sqrt{R_1 R_2 \frac{R_2 - R_1}{R_2 - R_1}}
            = \sqrt{R_1 R_2}
            = \sqrt{0{,}50\,\text{Ом} \cdot 18{,}00\,\text{Ом}}
            = 3{,}00\,\text{Ом}.
            \\
    \eta_1
            &= \frac{ R_1 }{R_1 + r}
            = \frac{\sqrt{ R_1 }}{\sqrt{ R_1 } + \sqrt{ R_2 }}
            = 0{,}143,  \\
    \eta_2
            &= \frac{ R_2 }{R_2 + r}
            = \frac{ \sqrt{ R_2 } }{\sqrt{ R_2 } + \sqrt{ R_1 }}
            = 0{,}857
    \end{align*}
}
\solutionspace{120pt}

\tasknumber{4}%
\task{%
    Определите ток, протекающий через резистор $R = 15\,\text{Ом}$ и разность потенциалов на нём (см.
    рис.
    на доске),
    если $r_1 = 1\,\text{Ом}$, $r_2 = 3\,\text{Ом}$, $\mathcal{E}_1 = 20\,\text{В}$, $\mathcal{E}_2 = 60\,\text{В}$
}

\variantsplitter

\addpersonalvariant{Александр Наумов}

\tasknumber{1}%
\task{%
    На резистор сопротивлением $R = 5\,\text{Ом}$ подали напряжение $U = 150\,\text{В}$.
    Определите ток, который потечёт через резистор, и мощность, выделяющуюся на нём.
}
\answer{%
    \begin{align*}
    \mathcal{I} &= \frac{ U }{ R } = \frac{ 150\,\text{В} }{ 5\,\text{Ом} } = 30{,}00\,\text{А},  \\
    P &= \frac{U^2}{ R } = \frac{ \sqr{ 150\,\text{В} } }{ 5\,\text{Ом} } = 4500{,}00\,\text{Вт}
    \end{align*}
}
\solutionspace{120pt}

\tasknumber{2}%
\task{%
    Замкнутая электрическая цепь состоит из ЭДС $\mathcal{E} = 2\,\text{В}$ и сопротивлением $r$
    и резистора $R = 24\,\text{Ом}$.
    Определите ток, протекающий в цепи.
    Какая тепловая энергия выделится на резисторе за время
    $\tau = 2\,\text{с}$? Какая работа будет совершена ЭДС за это время? Каков знак этой работы? Чему равен КПД цепи? Вычислите значения для 2 случаев:
    $r=0$ и $r = 30\,\text{Ом}$.
}
\answer{%
    \begin{align*}
    \mathcal{I}_1 &= \frac{ \mathcal{E} }{ R } = \frac{ 2\,\text{В} }{ 24\,\text{Ом} } = 0{,}08\,\text{А},  \\
    \mathcal{I}_2 &= \frac{ \mathcal{E} }{R + r} = \frac{ 2\,\text{В} }{24\,\text{Ом} + 30\,\text{Ом}} = 0{,}04\,\text{А},  \\
    Q_1 &= \mathcal{I}_1^2R\tau = \sqr{\frac{ \mathcal{E} }{ R }} R \tau
            = \sqr{\frac{ 2\,\text{В} }{ 24\,\text{Ом} }} \cdot 24\,\text{Ом} \cdot 2\,\text{с} = 0{,}307\,\text{Дж},  \\
    Q_2 &= \mathcal{I}_2^2R\tau = \sqr{\frac{ \mathcal{E} }{R + r}} R \tau
            = \sqr{\frac{ 2\,\text{В} }{24\,\text{Ом} + 30\,\text{Ом}}} \cdot 24\,\text{Ом} \cdot 2\,\text{с} = 0{,}077\,\text{Дж},  \\
    A_1 &= \mathcal{I}_1\tau\mathcal{E} = \frac{ \mathcal{E} }{R} \tau \mathcal{E}
            = \frac{\mathcal{E}^2 \tau}{ R } = \frac{\sqr{ 2\,\text{В} } \cdot 2\,\text{с}}{ 24\,\text{Ом} }
            = 0{,}320\,\text{Дж}, \text{положительна},  \\
    A_2 &= \mathcal{I}_2\tau\mathcal{E} = \frac{ \mathcal{E} }{R + r} \tau \mathcal{E}
            = \frac{\mathcal{E}^2 \tau}{R + r} = \frac{\sqr{ 2\,\text{В} } \cdot 2\,\text{с}}{24\,\text{Ом} + 30\,\text{Ом}}
            = 0{,}160\,\text{Дж}, \text{положительна},  \\
    \eta_1 &= \frac{ Q_1 }{ A_1 } = \ldots = \frac{ R }{ R } = 1,  \\
    \eta_2 &= \frac{ Q_2 }{ A_2 } = \ldots = \frac{ R }{R + r} = 0{,}48
    \end{align*}
}
\solutionspace{120pt}

\tasknumber{3}%
\task{%
    Лампочки, сопротивления которых $R_1 = 3{,}00\,\text{Ом}$ и $R_2 = 12{,}00\,\text{Ом}$, поочерёдно подключённные к некоторому источнику тока,
    потребляют одинаковую мощность.
    Найти внутреннее сопротивление источника и КПД цепи в каждом случае.
}
\answer{%
    \begin{align*}
        P_1 &= \sqr{\frac{ \mathcal{E} }{R_1 + r}}R_1,
        P_2  = \sqr{\frac{ \mathcal{E} }{R_2 + r}}R_2,
        P_1 = P_2 \implies  \\
        &\implies R_1 \sqr{R_2 + r} = R_2 \sqr{R_1 + r} \implies  \\
        &\implies R_1 R_2^2 + 2 R_1 R_2 r + R_1 r^2 =
                    R_2 R_1^2 + 2 R_2 R_1 r + R_2 r^2  \implies  \\
    &\implies r^2 (R_2 - R_1) = R_2^2 R_2 - R_1^2 R_2 \implies  \\
    &\implies r
            = \sqrt{R_1 R_2 \frac{R_2 - R_1}{R_2 - R_1}}
            = \sqrt{R_1 R_2}
            = \sqrt{3{,}00\,\text{Ом} \cdot 12{,}00\,\text{Ом}}
            = 6{,}00\,\text{Ом}.
            \\
    \eta_1
            &= \frac{ R_1 }{R_1 + r}
            = \frac{\sqrt{ R_1 }}{\sqrt{ R_1 } + \sqrt{ R_2 }}
            = 0{,}333,  \\
    \eta_2
            &= \frac{ R_2 }{R_2 + r}
            = \frac{ \sqrt{ R_2 } }{\sqrt{ R_2 } + \sqrt{ R_1 }}
            = 0{,}667
    \end{align*}
}
\solutionspace{120pt}

\tasknumber{4}%
\task{%
    Определите ток, протекающий через резистор $R = 20\,\text{Ом}$ и разность потенциалов на нём (см.
    рис.
    на доске),
    если $r_1 = 2\,\text{Ом}$, $r_2 = 1\,\text{Ом}$, $\mathcal{E}_1 = 60\,\text{В}$, $\mathcal{E}_2 = 60\,\text{В}$
}

\variantsplitter

\addpersonalvariant{Георгий Новиков}

\tasknumber{1}%
\task{%
    На резистор сопротивлением $r = 5\,\text{Ом}$ подали напряжение $V = 180\,\text{В}$.
    Определите ток, который потечёт через резистор, и мощность, выделяющуюся на нём.
}
\answer{%
    \begin{align*}
    \mathcal{I} &= \frac{ V }{ r } = \frac{ 180\,\text{В} }{ 5\,\text{Ом} } = 36{,}00\,\text{А},  \\
    P &= \frac{V^2}{ r } = \frac{ \sqr{ 180\,\text{В} } }{ 5\,\text{Ом} } = 6480{,}00\,\text{Вт}
    \end{align*}
}
\solutionspace{120pt}

\tasknumber{2}%
\task{%
    Замкнутая электрическая цепь состоит из ЭДС $\mathcal{E} = 4\,\text{В}$ и сопротивлением $r$
    и резистора $R = 30\,\text{Ом}$.
    Определите ток, протекающий в цепи.
    Какая тепловая энергия выделится на резисторе за время
    $\tau = 10\,\text{с}$? Какая работа будет совершена ЭДС за это время? Каков знак этой работы? Чему равен КПД цепи? Вычислите значения для 2 случаев:
    $r=0$ и $r = 20\,\text{Ом}$.
}
\answer{%
    \begin{align*}
    \mathcal{I}_1 &= \frac{ \mathcal{E} }{ R } = \frac{ 4\,\text{В} }{ 30\,\text{Ом} } = 0{,}13\,\text{А},  \\
    \mathcal{I}_2 &= \frac{ \mathcal{E} }{R + r} = \frac{ 4\,\text{В} }{30\,\text{Ом} + 20\,\text{Ом}} = 0{,}08\,\text{А},  \\
    Q_1 &= \mathcal{I}_1^2R\tau = \sqr{\frac{ \mathcal{E} }{ R }} R \tau
            = \sqr{\frac{ 4\,\text{В} }{ 30\,\text{Ом} }} \cdot 30\,\text{Ом} \cdot 10\,\text{с} = 5{,}070\,\text{Дж},  \\
    Q_2 &= \mathcal{I}_2^2R\tau = \sqr{\frac{ \mathcal{E} }{R + r}} R \tau
            = \sqr{\frac{ 4\,\text{В} }{30\,\text{Ом} + 20\,\text{Ом}}} \cdot 30\,\text{Ом} \cdot 10\,\text{с} = 1{,}920\,\text{Дж},  \\
    A_1 &= \mathcal{I}_1\tau\mathcal{E} = \frac{ \mathcal{E} }{R} \tau \mathcal{E}
            = \frac{\mathcal{E}^2 \tau}{ R } = \frac{\sqr{ 4\,\text{В} } \cdot 10\,\text{с}}{ 30\,\text{Ом} }
            = 5{,}200\,\text{Дж}, \text{положительна},  \\
    A_2 &= \mathcal{I}_2\tau\mathcal{E} = \frac{ \mathcal{E} }{R + r} \tau \mathcal{E}
            = \frac{\mathcal{E}^2 \tau}{R + r} = \frac{\sqr{ 4\,\text{В} } \cdot 10\,\text{с}}{30\,\text{Ом} + 20\,\text{Ом}}
            = 3{,}200\,\text{Дж}, \text{положительна},  \\
    \eta_1 &= \frac{ Q_1 }{ A_1 } = \ldots = \frac{ R }{ R } = 1,  \\
    \eta_2 &= \frac{ Q_2 }{ A_2 } = \ldots = \frac{ R }{R + r} = 0{,}60
    \end{align*}
}
\solutionspace{120pt}

\tasknumber{3}%
\task{%
    Лампочки, сопротивления которых $R_1 = 1{,}00\,\text{Ом}$ и $R_2 = 9{,}00\,\text{Ом}$, поочерёдно подключённные к некоторому источнику тока,
    потребляют одинаковую мощность.
    Найти внутреннее сопротивление источника и КПД цепи в каждом случае.
}
\answer{%
    \begin{align*}
        P_1 &= \sqr{\frac{ \mathcal{E} }{R_1 + r}}R_1,
        P_2  = \sqr{\frac{ \mathcal{E} }{R_2 + r}}R_2,
        P_1 = P_2 \implies  \\
        &\implies R_1 \sqr{R_2 + r} = R_2 \sqr{R_1 + r} \implies  \\
        &\implies R_1 R_2^2 + 2 R_1 R_2 r + R_1 r^2 =
                    R_2 R_1^2 + 2 R_2 R_1 r + R_2 r^2  \implies  \\
    &\implies r^2 (R_2 - R_1) = R_2^2 R_2 - R_1^2 R_2 \implies  \\
    &\implies r
            = \sqrt{R_1 R_2 \frac{R_2 - R_1}{R_2 - R_1}}
            = \sqrt{R_1 R_2}
            = \sqrt{1{,}00\,\text{Ом} \cdot 9{,}00\,\text{Ом}}
            = 3{,}00\,\text{Ом}.
            \\
    \eta_1
            &= \frac{ R_1 }{R_1 + r}
            = \frac{\sqrt{ R_1 }}{\sqrt{ R_1 } + \sqrt{ R_2 }}
            = 0{,}250,  \\
    \eta_2
            &= \frac{ R_2 }{R_2 + r}
            = \frac{ \sqrt{ R_2 } }{\sqrt{ R_2 } + \sqrt{ R_1 }}
            = 0{,}750
    \end{align*}
}
\solutionspace{120pt}

\tasknumber{4}%
\task{%
    Определите ток, протекающий через резистор $R = 12\,\text{Ом}$ и разность потенциалов на нём (см.
    рис.
    на доске),
    если $r_1 = 1\,\text{Ом}$, $r_2 = 3\,\text{Ом}$, $\mathcal{E}_1 = 30\,\text{В}$, $\mathcal{E}_2 = 30\,\text{В}$
}

\variantsplitter

\addpersonalvariant{Егор Осипов}

\tasknumber{1}%
\task{%
    На резистор сопротивлением $R = 18\,\text{Ом}$ подали напряжение $U = 120\,\text{В}$.
    Определите ток, который потечёт через резистор, и мощность, выделяющуюся на нём.
}
\answer{%
    \begin{align*}
    \mathcal{I} &= \frac{ U }{ R } = \frac{ 120\,\text{В} }{ 18\,\text{Ом} } = 6{,}67\,\text{А},  \\
    P &= \frac{U^2}{ R } = \frac{ \sqr{ 120\,\text{В} } }{ 18\,\text{Ом} } = 800{,}00\,\text{Вт}
    \end{align*}
}
\solutionspace{120pt}

\tasknumber{2}%
\task{%
    Замкнутая электрическая цепь состоит из ЭДС $\mathcal{E} = 1\,\text{В}$ и сопротивлением $r$
    и резистора $R = 30\,\text{Ом}$.
    Определите ток, протекающий в цепи.
    Какая тепловая энергия выделится на резисторе за время
    $\tau = 10\,\text{с}$? Какая работа будет совершена ЭДС за это время? Каков знак этой работы? Чему равен КПД цепи? Вычислите значения для 2 случаев:
    $r=0$ и $r = 60\,\text{Ом}$.
}
\answer{%
    \begin{align*}
    \mathcal{I}_1 &= \frac{ \mathcal{E} }{ R } = \frac{ 1\,\text{В} }{ 30\,\text{Ом} } = 0{,}03\,\text{А},  \\
    \mathcal{I}_2 &= \frac{ \mathcal{E} }{R + r} = \frac{ 1\,\text{В} }{30\,\text{Ом} + 60\,\text{Ом}} = 0{,}010\,\text{А},  \\
    Q_1 &= \mathcal{I}_1^2R\tau = \sqr{\frac{ \mathcal{E} }{ R }} R \tau
            = \sqr{\frac{ 1\,\text{В} }{ 30\,\text{Ом} }} \cdot 30\,\text{Ом} \cdot 10\,\text{с} = 0{,}270\,\text{Дж},  \\
    Q_2 &= \mathcal{I}_2^2R\tau = \sqr{\frac{ \mathcal{E} }{R + r}} R \tau
            = \sqr{\frac{ 1\,\text{В} }{30\,\text{Ом} + 60\,\text{Ом}}} \cdot 30\,\text{Ом} \cdot 10\,\text{с} = 0{,}030\,\text{Дж},  \\
    A_1 &= \mathcal{I}_1\tau\mathcal{E} = \frac{ \mathcal{E} }{R} \tau \mathcal{E}
            = \frac{\mathcal{E}^2 \tau}{ R } = \frac{\sqr{ 1\,\text{В} } \cdot 10\,\text{с}}{ 30\,\text{Ом} }
            = 0{,}300\,\text{Дж}, \text{положительна},  \\
    A_2 &= \mathcal{I}_2\tau\mathcal{E} = \frac{ \mathcal{E} }{R + r} \tau \mathcal{E}
            = \frac{\mathcal{E}^2 \tau}{R + r} = \frac{\sqr{ 1\,\text{В} } \cdot 10\,\text{с}}{30\,\text{Ом} + 60\,\text{Ом}}
            = 0{,}100\,\text{Дж}, \text{положительна},  \\
    \eta_1 &= \frac{ Q_1 }{ A_1 } = \ldots = \frac{ R }{ R } = 1,  \\
    \eta_2 &= \frac{ Q_2 }{ A_2 } = \ldots = \frac{ R }{R + r} = 0{,}30
    \end{align*}
}
\solutionspace{120pt}

\tasknumber{3}%
\task{%
    Лампочки, сопротивления которых $R_1 = 6{,}00\,\text{Ом}$ и $R_2 = 24{,}00\,\text{Ом}$, поочерёдно подключённные к некоторому источнику тока,
    потребляют одинаковую мощность.
    Найти внутреннее сопротивление источника и КПД цепи в каждом случае.
}
\answer{%
    \begin{align*}
        P_1 &= \sqr{\frac{ \mathcal{E} }{R_1 + r}}R_1,
        P_2  = \sqr{\frac{ \mathcal{E} }{R_2 + r}}R_2,
        P_1 = P_2 \implies  \\
        &\implies R_1 \sqr{R_2 + r} = R_2 \sqr{R_1 + r} \implies  \\
        &\implies R_1 R_2^2 + 2 R_1 R_2 r + R_1 r^2 =
                    R_2 R_1^2 + 2 R_2 R_1 r + R_2 r^2  \implies  \\
    &\implies r^2 (R_2 - R_1) = R_2^2 R_2 - R_1^2 R_2 \implies  \\
    &\implies r
            = \sqrt{R_1 R_2 \frac{R_2 - R_1}{R_2 - R_1}}
            = \sqrt{R_1 R_2}
            = \sqrt{6{,}00\,\text{Ом} \cdot 24{,}00\,\text{Ом}}
            = 12{,}00\,\text{Ом}.
            \\
    \eta_1
            &= \frac{ R_1 }{R_1 + r}
            = \frac{\sqrt{ R_1 }}{\sqrt{ R_1 } + \sqrt{ R_2 }}
            = 0{,}333,  \\
    \eta_2
            &= \frac{ R_2 }{R_2 + r}
            = \frac{ \sqrt{ R_2 } }{\sqrt{ R_2 } + \sqrt{ R_1 }}
            = 0{,}667
    \end{align*}
}
\solutionspace{120pt}

\tasknumber{4}%
\task{%
    Определите ток, протекающий через резистор $R = 10\,\text{Ом}$ и разность потенциалов на нём (см.
    рис.
    на доске),
    если $r_1 = 1\,\text{Ом}$, $r_2 = 1\,\text{Ом}$, $\mathcal{E}_1 = 60\,\text{В}$, $\mathcal{E}_2 = 30\,\text{В}$
}

\variantsplitter

\addpersonalvariant{Руслан Перепелица}

\tasknumber{1}%
\task{%
    На резистор сопротивлением $r = 5\,\text{Ом}$ подали напряжение $U = 240\,\text{В}$.
    Определите ток, который потечёт через резистор, и мощность, выделяющуюся на нём.
}
\answer{%
    \begin{align*}
    \mathcal{I} &= \frac{ U }{ r } = \frac{ 240\,\text{В} }{ 5\,\text{Ом} } = 48{,}00\,\text{А},  \\
    P &= \frac{U^2}{ r } = \frac{ \sqr{ 240\,\text{В} } }{ 5\,\text{Ом} } = 11520{,}00\,\text{Вт}
    \end{align*}
}
\solutionspace{120pt}

\tasknumber{2}%
\task{%
    Замкнутая электрическая цепь состоит из ЭДС $\mathcal{E} = 1\,\text{В}$ и сопротивлением $r$
    и резистора $R = 30\,\text{Ом}$.
    Определите ток, протекающий в цепи.
    Какая тепловая энергия выделится на резисторе за время
    $\tau = 2\,\text{с}$? Какая работа будет совершена ЭДС за это время? Каков знак этой работы? Чему равен КПД цепи? Вычислите значения для 2 случаев:
    $r=0$ и $r = 30\,\text{Ом}$.
}
\answer{%
    \begin{align*}
    \mathcal{I}_1 &= \frac{ \mathcal{E} }{ R } = \frac{ 1\,\text{В} }{ 30\,\text{Ом} } = 0{,}03\,\text{А},  \\
    \mathcal{I}_2 &= \frac{ \mathcal{E} }{R + r} = \frac{ 1\,\text{В} }{30\,\text{Ом} + 30\,\text{Ом}} = 0{,}02\,\text{А},  \\
    Q_1 &= \mathcal{I}_1^2R\tau = \sqr{\frac{ \mathcal{E} }{ R }} R \tau
            = \sqr{\frac{ 1\,\text{В} }{ 30\,\text{Ом} }} \cdot 30\,\text{Ом} \cdot 2\,\text{с} = 0{,}054\,\text{Дж},  \\
    Q_2 &= \mathcal{I}_2^2R\tau = \sqr{\frac{ \mathcal{E} }{R + r}} R \tau
            = \sqr{\frac{ 1\,\text{В} }{30\,\text{Ом} + 30\,\text{Ом}}} \cdot 30\,\text{Ом} \cdot 2\,\text{с} = 0{,}024\,\text{Дж},  \\
    A_1 &= \mathcal{I}_1\tau\mathcal{E} = \frac{ \mathcal{E} }{R} \tau \mathcal{E}
            = \frac{\mathcal{E}^2 \tau}{ R } = \frac{\sqr{ 1\,\text{В} } \cdot 2\,\text{с}}{ 30\,\text{Ом} }
            = 0{,}060\,\text{Дж}, \text{положительна},  \\
    A_2 &= \mathcal{I}_2\tau\mathcal{E} = \frac{ \mathcal{E} }{R + r} \tau \mathcal{E}
            = \frac{\mathcal{E}^2 \tau}{R + r} = \frac{\sqr{ 1\,\text{В} } \cdot 2\,\text{с}}{30\,\text{Ом} + 30\,\text{Ом}}
            = 0{,}040\,\text{Дж}, \text{положительна},  \\
    \eta_1 &= \frac{ Q_1 }{ A_1 } = \ldots = \frac{ R }{ R } = 1,  \\
    \eta_2 &= \frac{ Q_2 }{ A_2 } = \ldots = \frac{ R }{R + r} = 0{,}60
    \end{align*}
}
\solutionspace{120pt}

\tasknumber{3}%
\task{%
    Лампочки, сопротивления которых $R_1 = 0{,}25\,\text{Ом}$ и $R_2 = 4{,}00\,\text{Ом}$, поочерёдно подключённные к некоторому источнику тока,
    потребляют одинаковую мощность.
    Найти внутреннее сопротивление источника и КПД цепи в каждом случае.
}
\answer{%
    \begin{align*}
        P_1 &= \sqr{\frac{ \mathcal{E} }{R_1 + r}}R_1,
        P_2  = \sqr{\frac{ \mathcal{E} }{R_2 + r}}R_2,
        P_1 = P_2 \implies  \\
        &\implies R_1 \sqr{R_2 + r} = R_2 \sqr{R_1 + r} \implies  \\
        &\implies R_1 R_2^2 + 2 R_1 R_2 r + R_1 r^2 =
                    R_2 R_1^2 + 2 R_2 R_1 r + R_2 r^2  \implies  \\
    &\implies r^2 (R_2 - R_1) = R_2^2 R_2 - R_1^2 R_2 \implies  \\
    &\implies r
            = \sqrt{R_1 R_2 \frac{R_2 - R_1}{R_2 - R_1}}
            = \sqrt{R_1 R_2}
            = \sqrt{0{,}25\,\text{Ом} \cdot 4{,}00\,\text{Ом}}
            = 1{,}00\,\text{Ом}.
            \\
    \eta_1
            &= \frac{ R_1 }{R_1 + r}
            = \frac{\sqrt{ R_1 }}{\sqrt{ R_1 } + \sqrt{ R_2 }}
            = 0{,}200,  \\
    \eta_2
            &= \frac{ R_2 }{R_2 + r}
            = \frac{ \sqrt{ R_2 } }{\sqrt{ R_2 } + \sqrt{ R_1 }}
            = 0{,}800
    \end{align*}
}
\solutionspace{120pt}

\tasknumber{4}%
\task{%
    Определите ток, протекающий через резистор $R = 18\,\text{Ом}$ и разность потенциалов на нём (см.
    рис.
    на доске),
    если $r_1 = 2\,\text{Ом}$, $r_2 = 2\,\text{Ом}$, $\mathcal{E}_1 = 30\,\text{В}$, $\mathcal{E}_2 = 30\,\text{В}$
}

\variantsplitter

\addpersonalvariant{Михаил Перин}

\tasknumber{1}%
\task{%
    На резистор сопротивлением $R = 5\,\text{Ом}$ подали напряжение $U = 120\,\text{В}$.
    Определите ток, который потечёт через резистор, и мощность, выделяющуюся на нём.
}
\answer{%
    \begin{align*}
    \mathcal{I} &= \frac{ U }{ R } = \frac{ 120\,\text{В} }{ 5\,\text{Ом} } = 24{,}00\,\text{А},  \\
    P &= \frac{U^2}{ R } = \frac{ \sqr{ 120\,\text{В} } }{ 5\,\text{Ом} } = 2880{,}00\,\text{Вт}
    \end{align*}
}
\solutionspace{120pt}

\tasknumber{2}%
\task{%
    Замкнутая электрическая цепь состоит из ЭДС $\mathcal{E} = 4\,\text{В}$ и сопротивлением $r$
    и резистора $R = 15\,\text{Ом}$.
    Определите ток, протекающий в цепи.
    Какая тепловая энергия выделится на резисторе за время
    $\tau = 2\,\text{с}$? Какая работа будет совершена ЭДС за это время? Каков знак этой работы? Чему равен КПД цепи? Вычислите значения для 2 случаев:
    $r=0$ и $r = 60\,\text{Ом}$.
}
\answer{%
    \begin{align*}
    \mathcal{I}_1 &= \frac{ \mathcal{E} }{ R } = \frac{ 4\,\text{В} }{ 15\,\text{Ом} } = 0{,}27\,\text{А},  \\
    \mathcal{I}_2 &= \frac{ \mathcal{E} }{R + r} = \frac{ 4\,\text{В} }{15\,\text{Ом} + 60\,\text{Ом}} = 0{,}05\,\text{А},  \\
    Q_1 &= \mathcal{I}_1^2R\tau = \sqr{\frac{ \mathcal{E} }{ R }} R \tau
            = \sqr{\frac{ 4\,\text{В} }{ 15\,\text{Ом} }} \cdot 15\,\text{Ом} \cdot 2\,\text{с} = 2{,}187\,\text{Дж},  \\
    Q_2 &= \mathcal{I}_2^2R\tau = \sqr{\frac{ \mathcal{E} }{R + r}} R \tau
            = \sqr{\frac{ 4\,\text{В} }{15\,\text{Ом} + 60\,\text{Ом}}} \cdot 15\,\text{Ом} \cdot 2\,\text{с} = 0{,}075\,\text{Дж},  \\
    A_1 &= \mathcal{I}_1\tau\mathcal{E} = \frac{ \mathcal{E} }{R} \tau \mathcal{E}
            = \frac{\mathcal{E}^2 \tau}{ R } = \frac{\sqr{ 4\,\text{В} } \cdot 2\,\text{с}}{ 15\,\text{Ом} }
            = 2{,}160\,\text{Дж}, \text{положительна},  \\
    A_2 &= \mathcal{I}_2\tau\mathcal{E} = \frac{ \mathcal{E} }{R + r} \tau \mathcal{E}
            = \frac{\mathcal{E}^2 \tau}{R + r} = \frac{\sqr{ 4\,\text{В} } \cdot 2\,\text{с}}{15\,\text{Ом} + 60\,\text{Ом}}
            = 0{,}400\,\text{Дж}, \text{положительна},  \\
    \eta_1 &= \frac{ Q_1 }{ A_1 } = \ldots = \frac{ R }{ R } = 1,  \\
    \eta_2 &= \frac{ Q_2 }{ A_2 } = \ldots = \frac{ R }{R + r} = 0{,}19
    \end{align*}
}
\solutionspace{120pt}

\tasknumber{3}%
\task{%
    Лампочки, сопротивления которых $R_1 = 0{,}25\,\text{Ом}$ и $R_2 = 64{,}00\,\text{Ом}$, поочерёдно подключённные к некоторому источнику тока,
    потребляют одинаковую мощность.
    Найти внутреннее сопротивление источника и КПД цепи в каждом случае.
}
\answer{%
    \begin{align*}
        P_1 &= \sqr{\frac{ \mathcal{E} }{R_1 + r}}R_1,
        P_2  = \sqr{\frac{ \mathcal{E} }{R_2 + r}}R_2,
        P_1 = P_2 \implies  \\
        &\implies R_1 \sqr{R_2 + r} = R_2 \sqr{R_1 + r} \implies  \\
        &\implies R_1 R_2^2 + 2 R_1 R_2 r + R_1 r^2 =
                    R_2 R_1^2 + 2 R_2 R_1 r + R_2 r^2  \implies  \\
    &\implies r^2 (R_2 - R_1) = R_2^2 R_2 - R_1^2 R_2 \implies  \\
    &\implies r
            = \sqrt{R_1 R_2 \frac{R_2 - R_1}{R_2 - R_1}}
            = \sqrt{R_1 R_2}
            = \sqrt{0{,}25\,\text{Ом} \cdot 64{,}00\,\text{Ом}}
            = 4{,}00\,\text{Ом}.
            \\
    \eta_1
            &= \frac{ R_1 }{R_1 + r}
            = \frac{\sqrt{ R_1 }}{\sqrt{ R_1 } + \sqrt{ R_2 }}
            = 0{,}059,  \\
    \eta_2
            &= \frac{ R_2 }{R_2 + r}
            = \frac{ \sqrt{ R_2 } }{\sqrt{ R_2 } + \sqrt{ R_1 }}
            = 0{,}941
    \end{align*}
}
\solutionspace{120pt}

\tasknumber{4}%
\task{%
    Определите ток, протекающий через резистор $R = 12\,\text{Ом}$ и разность потенциалов на нём (см.
    рис.
    на доске),
    если $r_1 = 1\,\text{Ом}$, $r_2 = 2\,\text{Ом}$, $\mathcal{E}_1 = 30\,\text{В}$, $\mathcal{E}_2 = 40\,\text{В}$
}

\variantsplitter

\addpersonalvariant{Егор Подуровский}

\tasknumber{1}%
\task{%
    На резистор сопротивлением $R = 12\,\text{Ом}$ подали напряжение $V = 180\,\text{В}$.
    Определите ток, который потечёт через резистор, и мощность, выделяющуюся на нём.
}
\answer{%
    \begin{align*}
    \mathcal{I} &= \frac{ V }{ R } = \frac{ 180\,\text{В} }{ 12\,\text{Ом} } = 15{,}00\,\text{А},  \\
    P &= \frac{V^2}{ R } = \frac{ \sqr{ 180\,\text{В} } }{ 12\,\text{Ом} } = 2700{,}00\,\text{Вт}
    \end{align*}
}
\solutionspace{120pt}

\tasknumber{2}%
\task{%
    Замкнутая электрическая цепь состоит из ЭДС $\mathcal{E} = 3\,\text{В}$ и сопротивлением $r$
    и резистора $R = 30\,\text{Ом}$.
    Определите ток, протекающий в цепи.
    Какая тепловая энергия выделится на резисторе за время
    $\tau = 2\,\text{с}$? Какая работа будет совершена ЭДС за это время? Каков знак этой работы? Чему равен КПД цепи? Вычислите значения для 2 случаев:
    $r=0$ и $r = 60\,\text{Ом}$.
}
\answer{%
    \begin{align*}
    \mathcal{I}_1 &= \frac{ \mathcal{E} }{ R } = \frac{ 3\,\text{В} }{ 30\,\text{Ом} } = 0{,}10\,\text{А},  \\
    \mathcal{I}_2 &= \frac{ \mathcal{E} }{R + r} = \frac{ 3\,\text{В} }{30\,\text{Ом} + 60\,\text{Ом}} = 0{,}03\,\text{А},  \\
    Q_1 &= \mathcal{I}_1^2R\tau = \sqr{\frac{ \mathcal{E} }{ R }} R \tau
            = \sqr{\frac{ 3\,\text{В} }{ 30\,\text{Ом} }} \cdot 30\,\text{Ом} \cdot 2\,\text{с} = 0{,}600\,\text{Дж},  \\
    Q_2 &= \mathcal{I}_2^2R\tau = \sqr{\frac{ \mathcal{E} }{R + r}} R \tau
            = \sqr{\frac{ 3\,\text{В} }{30\,\text{Ом} + 60\,\text{Ом}}} \cdot 30\,\text{Ом} \cdot 2\,\text{с} = 0{,}054\,\text{Дж},  \\
    A_1 &= \mathcal{I}_1\tau\mathcal{E} = \frac{ \mathcal{E} }{R} \tau \mathcal{E}
            = \frac{\mathcal{E}^2 \tau}{ R } = \frac{\sqr{ 3\,\text{В} } \cdot 2\,\text{с}}{ 30\,\text{Ом} }
            = 0{,}600\,\text{Дж}, \text{положительна},  \\
    A_2 &= \mathcal{I}_2\tau\mathcal{E} = \frac{ \mathcal{E} }{R + r} \tau \mathcal{E}
            = \frac{\mathcal{E}^2 \tau}{R + r} = \frac{\sqr{ 3\,\text{В} } \cdot 2\,\text{с}}{30\,\text{Ом} + 60\,\text{Ом}}
            = 0{,}180\,\text{Дж}, \text{положительна},  \\
    \eta_1 &= \frac{ Q_1 }{ A_1 } = \ldots = \frac{ R }{ R } = 1,  \\
    \eta_2 &= \frac{ Q_2 }{ A_2 } = \ldots = \frac{ R }{R + r} = 0{,}30
    \end{align*}
}
\solutionspace{120pt}

\tasknumber{3}%
\task{%
    Лампочки, сопротивления которых $R_1 = 0{,}25\,\text{Ом}$ и $R_2 = 16{,}00\,\text{Ом}$, поочерёдно подключённные к некоторому источнику тока,
    потребляют одинаковую мощность.
    Найти внутреннее сопротивление источника и КПД цепи в каждом случае.
}
\answer{%
    \begin{align*}
        P_1 &= \sqr{\frac{ \mathcal{E} }{R_1 + r}}R_1,
        P_2  = \sqr{\frac{ \mathcal{E} }{R_2 + r}}R_2,
        P_1 = P_2 \implies  \\
        &\implies R_1 \sqr{R_2 + r} = R_2 \sqr{R_1 + r} \implies  \\
        &\implies R_1 R_2^2 + 2 R_1 R_2 r + R_1 r^2 =
                    R_2 R_1^2 + 2 R_2 R_1 r + R_2 r^2  \implies  \\
    &\implies r^2 (R_2 - R_1) = R_2^2 R_2 - R_1^2 R_2 \implies  \\
    &\implies r
            = \sqrt{R_1 R_2 \frac{R_2 - R_1}{R_2 - R_1}}
            = \sqrt{R_1 R_2}
            = \sqrt{0{,}25\,\text{Ом} \cdot 16{,}00\,\text{Ом}}
            = 2{,}00\,\text{Ом}.
            \\
    \eta_1
            &= \frac{ R_1 }{R_1 + r}
            = \frac{\sqrt{ R_1 }}{\sqrt{ R_1 } + \sqrt{ R_2 }}
            = 0{,}111,  \\
    \eta_2
            &= \frac{ R_2 }{R_2 + r}
            = \frac{ \sqrt{ R_2 } }{\sqrt{ R_2 } + \sqrt{ R_1 }}
            = 0{,}889
    \end{align*}
}
\solutionspace{120pt}

\tasknumber{4}%
\task{%
    Определите ток, протекающий через резистор $R = 12\,\text{Ом}$ и разность потенциалов на нём (см.
    рис.
    на доске),
    если $r_1 = 1\,\text{Ом}$, $r_2 = 2\,\text{Ом}$, $\mathcal{E}_1 = 60\,\text{В}$, $\mathcal{E}_2 = 60\,\text{В}$
}

\variantsplitter

\addpersonalvariant{Роман Прибылов}

\tasknumber{1}%
\task{%
    На резистор сопротивлением $r = 18\,\text{Ом}$ подали напряжение $U = 120\,\text{В}$.
    Определите ток, который потечёт через резистор, и мощность, выделяющуюся на нём.
}
\answer{%
    \begin{align*}
    \mathcal{I} &= \frac{ U }{ r } = \frac{ 120\,\text{В} }{ 18\,\text{Ом} } = 6{,}67\,\text{А},  \\
    P &= \frac{U^2}{ r } = \frac{ \sqr{ 120\,\text{В} } }{ 18\,\text{Ом} } = 800{,}00\,\text{Вт}
    \end{align*}
}
\solutionspace{120pt}

\tasknumber{2}%
\task{%
    Замкнутая электрическая цепь состоит из ЭДС $\mathcal{E} = 1\,\text{В}$ и сопротивлением $r$
    и резистора $R = 24\,\text{Ом}$.
    Определите ток, протекающий в цепи.
    Какая тепловая энергия выделится на резисторе за время
    $\tau = 5\,\text{с}$? Какая работа будет совершена ЭДС за это время? Каков знак этой работы? Чему равен КПД цепи? Вычислите значения для 2 случаев:
    $r=0$ и $r = 20\,\text{Ом}$.
}
\answer{%
    \begin{align*}
    \mathcal{I}_1 &= \frac{ \mathcal{E} }{ R } = \frac{ 1\,\text{В} }{ 24\,\text{Ом} } = 0{,}04\,\text{А},  \\
    \mathcal{I}_2 &= \frac{ \mathcal{E} }{R + r} = \frac{ 1\,\text{В} }{24\,\text{Ом} + 20\,\text{Ом}} = 0{,}02\,\text{А},  \\
    Q_1 &= \mathcal{I}_1^2R\tau = \sqr{\frac{ \mathcal{E} }{ R }} R \tau
            = \sqr{\frac{ 1\,\text{В} }{ 24\,\text{Ом} }} \cdot 24\,\text{Ом} \cdot 5\,\text{с} = 0{,}192\,\text{Дж},  \\
    Q_2 &= \mathcal{I}_2^2R\tau = \sqr{\frac{ \mathcal{E} }{R + r}} R \tau
            = \sqr{\frac{ 1\,\text{В} }{24\,\text{Ом} + 20\,\text{Ом}}} \cdot 24\,\text{Ом} \cdot 5\,\text{с} = 0{,}048\,\text{Дж},  \\
    A_1 &= \mathcal{I}_1\tau\mathcal{E} = \frac{ \mathcal{E} }{R} \tau \mathcal{E}
            = \frac{\mathcal{E}^2 \tau}{ R } = \frac{\sqr{ 1\,\text{В} } \cdot 5\,\text{с}}{ 24\,\text{Ом} }
            = 0{,}200\,\text{Дж}, \text{положительна},  \\
    A_2 &= \mathcal{I}_2\tau\mathcal{E} = \frac{ \mathcal{E} }{R + r} \tau \mathcal{E}
            = \frac{\mathcal{E}^2 \tau}{R + r} = \frac{\sqr{ 1\,\text{В} } \cdot 5\,\text{с}}{24\,\text{Ом} + 20\,\text{Ом}}
            = 0{,}100\,\text{Дж}, \text{положительна},  \\
    \eta_1 &= \frac{ Q_1 }{ A_1 } = \ldots = \frac{ R }{ R } = 1,  \\
    \eta_2 &= \frac{ Q_2 }{ A_2 } = \ldots = \frac{ R }{R + r} = 0{,}48
    \end{align*}
}
\solutionspace{120pt}

\tasknumber{3}%
\task{%
    Лампочки, сопротивления которых $R_1 = 1{,}00\,\text{Ом}$ и $R_2 = 4{,}00\,\text{Ом}$, поочерёдно подключённные к некоторому источнику тока,
    потребляют одинаковую мощность.
    Найти внутреннее сопротивление источника и КПД цепи в каждом случае.
}
\answer{%
    \begin{align*}
        P_1 &= \sqr{\frac{ \mathcal{E} }{R_1 + r}}R_1,
        P_2  = \sqr{\frac{ \mathcal{E} }{R_2 + r}}R_2,
        P_1 = P_2 \implies  \\
        &\implies R_1 \sqr{R_2 + r} = R_2 \sqr{R_1 + r} \implies  \\
        &\implies R_1 R_2^2 + 2 R_1 R_2 r + R_1 r^2 =
                    R_2 R_1^2 + 2 R_2 R_1 r + R_2 r^2  \implies  \\
    &\implies r^2 (R_2 - R_1) = R_2^2 R_2 - R_1^2 R_2 \implies  \\
    &\implies r
            = \sqrt{R_1 R_2 \frac{R_2 - R_1}{R_2 - R_1}}
            = \sqrt{R_1 R_2}
            = \sqrt{1{,}00\,\text{Ом} \cdot 4{,}00\,\text{Ом}}
            = 2{,}00\,\text{Ом}.
            \\
    \eta_1
            &= \frac{ R_1 }{R_1 + r}
            = \frac{\sqrt{ R_1 }}{\sqrt{ R_1 } + \sqrt{ R_2 }}
            = 0{,}333,  \\
    \eta_2
            &= \frac{ R_2 }{R_2 + r}
            = \frac{ \sqrt{ R_2 } }{\sqrt{ R_2 } + \sqrt{ R_1 }}
            = 0{,}667
    \end{align*}
}
\solutionspace{120pt}

\tasknumber{4}%
\task{%
    Определите ток, протекающий через резистор $R = 10\,\text{Ом}$ и разность потенциалов на нём (см.
    рис.
    на доске),
    если $r_1 = 1\,\text{Ом}$, $r_2 = 2\,\text{Ом}$, $\mathcal{E}_1 = 20\,\text{В}$, $\mathcal{E}_2 = 60\,\text{В}$
}

\variantsplitter

\addpersonalvariant{Александр Селехметьев}

\tasknumber{1}%
\task{%
    На резистор сопротивлением $r = 12\,\text{Ом}$ подали напряжение $U = 150\,\text{В}$.
    Определите ток, который потечёт через резистор, и мощность, выделяющуюся на нём.
}
\answer{%
    \begin{align*}
    \mathcal{I} &= \frac{ U }{ r } = \frac{ 150\,\text{В} }{ 12\,\text{Ом} } = 12{,}50\,\text{А},  \\
    P &= \frac{U^2}{ r } = \frac{ \sqr{ 150\,\text{В} } }{ 12\,\text{Ом} } = 1875{,}00\,\text{Вт}
    \end{align*}
}
\solutionspace{120pt}

\tasknumber{2}%
\task{%
    Замкнутая электрическая цепь состоит из ЭДС $\mathcal{E} = 3\,\text{В}$ и сопротивлением $r$
    и резистора $R = 10\,\text{Ом}$.
    Определите ток, протекающий в цепи.
    Какая тепловая энергия выделится на резисторе за время
    $\tau = 5\,\text{с}$? Какая работа будет совершена ЭДС за это время? Каков знак этой работы? Чему равен КПД цепи? Вычислите значения для 2 случаев:
    $r=0$ и $r = 30\,\text{Ом}$.
}
\answer{%
    \begin{align*}
    \mathcal{I}_1 &= \frac{ \mathcal{E} }{ R } = \frac{ 3\,\text{В} }{ 10\,\text{Ом} } = 0{,}30\,\text{А},  \\
    \mathcal{I}_2 &= \frac{ \mathcal{E} }{R + r} = \frac{ 3\,\text{В} }{10\,\text{Ом} + 30\,\text{Ом}} = 0{,}07\,\text{А},  \\
    Q_1 &= \mathcal{I}_1^2R\tau = \sqr{\frac{ \mathcal{E} }{ R }} R \tau
            = \sqr{\frac{ 3\,\text{В} }{ 10\,\text{Ом} }} \cdot 10\,\text{Ом} \cdot 5\,\text{с} = 4{,}500\,\text{Дж},  \\
    Q_2 &= \mathcal{I}_2^2R\tau = \sqr{\frac{ \mathcal{E} }{R + r}} R \tau
            = \sqr{\frac{ 3\,\text{В} }{10\,\text{Ом} + 30\,\text{Ом}}} \cdot 10\,\text{Ом} \cdot 5\,\text{с} = 0{,}245\,\text{Дж},  \\
    A_1 &= \mathcal{I}_1\tau\mathcal{E} = \frac{ \mathcal{E} }{R} \tau \mathcal{E}
            = \frac{\mathcal{E}^2 \tau}{ R } = \frac{\sqr{ 3\,\text{В} } \cdot 5\,\text{с}}{ 10\,\text{Ом} }
            = 4{,}500\,\text{Дж}, \text{положительна},  \\
    A_2 &= \mathcal{I}_2\tau\mathcal{E} = \frac{ \mathcal{E} }{R + r} \tau \mathcal{E}
            = \frac{\mathcal{E}^2 \tau}{R + r} = \frac{\sqr{ 3\,\text{В} } \cdot 5\,\text{с}}{10\,\text{Ом} + 30\,\text{Ом}}
            = 1{,}050\,\text{Дж}, \text{положительна},  \\
    \eta_1 &= \frac{ Q_1 }{ A_1 } = \ldots = \frac{ R }{ R } = 1,  \\
    \eta_2 &= \frac{ Q_2 }{ A_2 } = \ldots = \frac{ R }{R + r} = 0{,}23
    \end{align*}
}
\solutionspace{120pt}

\tasknumber{3}%
\task{%
    Лампочки, сопротивления которых $R_1 = 4{,}00\,\text{Ом}$ и $R_2 = 100{,}00\,\text{Ом}$, поочерёдно подключённные к некоторому источнику тока,
    потребляют одинаковую мощность.
    Найти внутреннее сопротивление источника и КПД цепи в каждом случае.
}
\answer{%
    \begin{align*}
        P_1 &= \sqr{\frac{ \mathcal{E} }{R_1 + r}}R_1,
        P_2  = \sqr{\frac{ \mathcal{E} }{R_2 + r}}R_2,
        P_1 = P_2 \implies  \\
        &\implies R_1 \sqr{R_2 + r} = R_2 \sqr{R_1 + r} \implies  \\
        &\implies R_1 R_2^2 + 2 R_1 R_2 r + R_1 r^2 =
                    R_2 R_1^2 + 2 R_2 R_1 r + R_2 r^2  \implies  \\
    &\implies r^2 (R_2 - R_1) = R_2^2 R_2 - R_1^2 R_2 \implies  \\
    &\implies r
            = \sqrt{R_1 R_2 \frac{R_2 - R_1}{R_2 - R_1}}
            = \sqrt{R_1 R_2}
            = \sqrt{4{,}00\,\text{Ом} \cdot 100{,}00\,\text{Ом}}
            = 20{,}00\,\text{Ом}.
            \\
    \eta_1
            &= \frac{ R_1 }{R_1 + r}
            = \frac{\sqrt{ R_1 }}{\sqrt{ R_1 } + \sqrt{ R_2 }}
            = 0{,}167,  \\
    \eta_2
            &= \frac{ R_2 }{R_2 + r}
            = \frac{ \sqrt{ R_2 } }{\sqrt{ R_2 } + \sqrt{ R_1 }}
            = 0{,}833
    \end{align*}
}
\solutionspace{120pt}

\tasknumber{4}%
\task{%
    Определите ток, протекающий через резистор $R = 12\,\text{Ом}$ и разность потенциалов на нём (см.
    рис.
    на доске),
    если $r_1 = 3\,\text{Ом}$, $r_2 = 1\,\text{Ом}$, $\mathcal{E}_1 = 40\,\text{В}$, $\mathcal{E}_2 = 20\,\text{В}$
}

\variantsplitter

\addpersonalvariant{Алексей Тихонов}

\tasknumber{1}%
\task{%
    На резистор сопротивлением $R = 18\,\text{Ом}$ подали напряжение $U = 180\,\text{В}$.
    Определите ток, который потечёт через резистор, и мощность, выделяющуюся на нём.
}
\answer{%
    \begin{align*}
    \mathcal{I} &= \frac{ U }{ R } = \frac{ 180\,\text{В} }{ 18\,\text{Ом} } = 10{,}00\,\text{А},  \\
    P &= \frac{U^2}{ R } = \frac{ \sqr{ 180\,\text{В} } }{ 18\,\text{Ом} } = 1800{,}00\,\text{Вт}
    \end{align*}
}
\solutionspace{120pt}

\tasknumber{2}%
\task{%
    Замкнутая электрическая цепь состоит из ЭДС $\mathcal{E} = 4\,\text{В}$ и сопротивлением $r$
    и резистора $R = 15\,\text{Ом}$.
    Определите ток, протекающий в цепи.
    Какая тепловая энергия выделится на резисторе за время
    $\tau = 10\,\text{с}$? Какая работа будет совершена ЭДС за это время? Каков знак этой работы? Чему равен КПД цепи? Вычислите значения для 2 случаев:
    $r=0$ и $r = 30\,\text{Ом}$.
}
\answer{%
    \begin{align*}
    \mathcal{I}_1 &= \frac{ \mathcal{E} }{ R } = \frac{ 4\,\text{В} }{ 15\,\text{Ом} } = 0{,}27\,\text{А},  \\
    \mathcal{I}_2 &= \frac{ \mathcal{E} }{R + r} = \frac{ 4\,\text{В} }{15\,\text{Ом} + 30\,\text{Ом}} = 0{,}09\,\text{А},  \\
    Q_1 &= \mathcal{I}_1^2R\tau = \sqr{\frac{ \mathcal{E} }{ R }} R \tau
            = \sqr{\frac{ 4\,\text{В} }{ 15\,\text{Ом} }} \cdot 15\,\text{Ом} \cdot 10\,\text{с} = 10{,}935\,\text{Дж},  \\
    Q_2 &= \mathcal{I}_2^2R\tau = \sqr{\frac{ \mathcal{E} }{R + r}} R \tau
            = \sqr{\frac{ 4\,\text{В} }{15\,\text{Ом} + 30\,\text{Ом}}} \cdot 15\,\text{Ом} \cdot 10\,\text{с} = 1{,}215\,\text{Дж},  \\
    A_1 &= \mathcal{I}_1\tau\mathcal{E} = \frac{ \mathcal{E} }{R} \tau \mathcal{E}
            = \frac{\mathcal{E}^2 \tau}{ R } = \frac{\sqr{ 4\,\text{В} } \cdot 10\,\text{с}}{ 15\,\text{Ом} }
            = 10{,}800\,\text{Дж}, \text{положительна},  \\
    A_2 &= \mathcal{I}_2\tau\mathcal{E} = \frac{ \mathcal{E} }{R + r} \tau \mathcal{E}
            = \frac{\mathcal{E}^2 \tau}{R + r} = \frac{\sqr{ 4\,\text{В} } \cdot 10\,\text{с}}{15\,\text{Ом} + 30\,\text{Ом}}
            = 3{,}600\,\text{Дж}, \text{положительна},  \\
    \eta_1 &= \frac{ Q_1 }{ A_1 } = \ldots = \frac{ R }{ R } = 1,  \\
    \eta_2 &= \frac{ Q_2 }{ A_2 } = \ldots = \frac{ R }{R + r} = 0{,}34
    \end{align*}
}
\solutionspace{120pt}

\tasknumber{3}%
\task{%
    Лампочки, сопротивления которых $R_1 = 6{,}00\,\text{Ом}$ и $R_2 = 54{,}00\,\text{Ом}$, поочерёдно подключённные к некоторому источнику тока,
    потребляют одинаковую мощность.
    Найти внутреннее сопротивление источника и КПД цепи в каждом случае.
}
\answer{%
    \begin{align*}
        P_1 &= \sqr{\frac{ \mathcal{E} }{R_1 + r}}R_1,
        P_2  = \sqr{\frac{ \mathcal{E} }{R_2 + r}}R_2,
        P_1 = P_2 \implies  \\
        &\implies R_1 \sqr{R_2 + r} = R_2 \sqr{R_1 + r} \implies  \\
        &\implies R_1 R_2^2 + 2 R_1 R_2 r + R_1 r^2 =
                    R_2 R_1^2 + 2 R_2 R_1 r + R_2 r^2  \implies  \\
    &\implies r^2 (R_2 - R_1) = R_2^2 R_2 - R_1^2 R_2 \implies  \\
    &\implies r
            = \sqrt{R_1 R_2 \frac{R_2 - R_1}{R_2 - R_1}}
            = \sqrt{R_1 R_2}
            = \sqrt{6{,}00\,\text{Ом} \cdot 54{,}00\,\text{Ом}}
            = 18{,}00\,\text{Ом}.
            \\
    \eta_1
            &= \frac{ R_1 }{R_1 + r}
            = \frac{\sqrt{ R_1 }}{\sqrt{ R_1 } + \sqrt{ R_2 }}
            = 0{,}250,  \\
    \eta_2
            &= \frac{ R_2 }{R_2 + r}
            = \frac{ \sqrt{ R_2 } }{\sqrt{ R_2 } + \sqrt{ R_1 }}
            = 0{,}750
    \end{align*}
}
\solutionspace{120pt}

\tasknumber{4}%
\task{%
    Определите ток, протекающий через резистор $R = 15\,\text{Ом}$ и разность потенциалов на нём (см.
    рис.
    на доске),
    если $r_1 = 2\,\text{Ом}$, $r_2 = 1\,\text{Ом}$, $\mathcal{E}_1 = 20\,\text{В}$, $\mathcal{E}_2 = 60\,\text{В}$
}

\variantsplitter

\addpersonalvariant{Алина Филиппова}

\tasknumber{1}%
\task{%
    На резистор сопротивлением $r = 5\,\text{Ом}$ подали напряжение $U = 180\,\text{В}$.
    Определите ток, который потечёт через резистор, и мощность, выделяющуюся на нём.
}
\answer{%
    \begin{align*}
    \mathcal{I} &= \frac{ U }{ r } = \frac{ 180\,\text{В} }{ 5\,\text{Ом} } = 36{,}00\,\text{А},  \\
    P &= \frac{U^2}{ r } = \frac{ \sqr{ 180\,\text{В} } }{ 5\,\text{Ом} } = 6480{,}00\,\text{Вт}
    \end{align*}
}
\solutionspace{120pt}

\tasknumber{2}%
\task{%
    Замкнутая электрическая цепь состоит из ЭДС $\mathcal{E} = 4\,\text{В}$ и сопротивлением $r$
    и резистора $R = 10\,\text{Ом}$.
    Определите ток, протекающий в цепи.
    Какая тепловая энергия выделится на резисторе за время
    $\tau = 10\,\text{с}$? Какая работа будет совершена ЭДС за это время? Каков знак этой работы? Чему равен КПД цепи? Вычислите значения для 2 случаев:
    $r=0$ и $r = 10\,\text{Ом}$.
}
\answer{%
    \begin{align*}
    \mathcal{I}_1 &= \frac{ \mathcal{E} }{ R } = \frac{ 4\,\text{В} }{ 10\,\text{Ом} } = 0{,}40\,\text{А},  \\
    \mathcal{I}_2 &= \frac{ \mathcal{E} }{R + r} = \frac{ 4\,\text{В} }{10\,\text{Ом} + 10\,\text{Ом}} = 0{,}20\,\text{А},  \\
    Q_1 &= \mathcal{I}_1^2R\tau = \sqr{\frac{ \mathcal{E} }{ R }} R \tau
            = \sqr{\frac{ 4\,\text{В} }{ 10\,\text{Ом} }} \cdot 10\,\text{Ом} \cdot 10\,\text{с} = 16{,}000\,\text{Дж},  \\
    Q_2 &= \mathcal{I}_2^2R\tau = \sqr{\frac{ \mathcal{E} }{R + r}} R \tau
            = \sqr{\frac{ 4\,\text{В} }{10\,\text{Ом} + 10\,\text{Ом}}} \cdot 10\,\text{Ом} \cdot 10\,\text{с} = 4{,}000\,\text{Дж},  \\
    A_1 &= \mathcal{I}_1\tau\mathcal{E} = \frac{ \mathcal{E} }{R} \tau \mathcal{E}
            = \frac{\mathcal{E}^2 \tau}{ R } = \frac{\sqr{ 4\,\text{В} } \cdot 10\,\text{с}}{ 10\,\text{Ом} }
            = 16{,}000\,\text{Дж}, \text{положительна},  \\
    A_2 &= \mathcal{I}_2\tau\mathcal{E} = \frac{ \mathcal{E} }{R + r} \tau \mathcal{E}
            = \frac{\mathcal{E}^2 \tau}{R + r} = \frac{\sqr{ 4\,\text{В} } \cdot 10\,\text{с}}{10\,\text{Ом} + 10\,\text{Ом}}
            = 8{,}000\,\text{Дж}, \text{положительна},  \\
    \eta_1 &= \frac{ Q_1 }{ A_1 } = \ldots = \frac{ R }{ R } = 1,  \\
    \eta_2 &= \frac{ Q_2 }{ A_2 } = \ldots = \frac{ R }{R + r} = 0{,}50
    \end{align*}
}
\solutionspace{120pt}

\tasknumber{3}%
\task{%
    Лампочки, сопротивления которых $R_1 = 3{,}00\,\text{Ом}$ и $R_2 = 12{,}00\,\text{Ом}$, поочерёдно подключённные к некоторому источнику тока,
    потребляют одинаковую мощность.
    Найти внутреннее сопротивление источника и КПД цепи в каждом случае.
}
\answer{%
    \begin{align*}
        P_1 &= \sqr{\frac{ \mathcal{E} }{R_1 + r}}R_1,
        P_2  = \sqr{\frac{ \mathcal{E} }{R_2 + r}}R_2,
        P_1 = P_2 \implies  \\
        &\implies R_1 \sqr{R_2 + r} = R_2 \sqr{R_1 + r} \implies  \\
        &\implies R_1 R_2^2 + 2 R_1 R_2 r + R_1 r^2 =
                    R_2 R_1^2 + 2 R_2 R_1 r + R_2 r^2  \implies  \\
    &\implies r^2 (R_2 - R_1) = R_2^2 R_2 - R_1^2 R_2 \implies  \\
    &\implies r
            = \sqrt{R_1 R_2 \frac{R_2 - R_1}{R_2 - R_1}}
            = \sqrt{R_1 R_2}
            = \sqrt{3{,}00\,\text{Ом} \cdot 12{,}00\,\text{Ом}}
            = 6{,}00\,\text{Ом}.
            \\
    \eta_1
            &= \frac{ R_1 }{R_1 + r}
            = \frac{\sqrt{ R_1 }}{\sqrt{ R_1 } + \sqrt{ R_2 }}
            = 0{,}333,  \\
    \eta_2
            &= \frac{ R_2 }{R_2 + r}
            = \frac{ \sqrt{ R_2 } }{\sqrt{ R_2 } + \sqrt{ R_1 }}
            = 0{,}667
    \end{align*}
}
\solutionspace{120pt}

\tasknumber{4}%
\task{%
    Определите ток, протекающий через резистор $R = 15\,\text{Ом}$ и разность потенциалов на нём (см.
    рис.
    на доске),
    если $r_1 = 2\,\text{Ом}$, $r_2 = 1\,\text{Ом}$, $\mathcal{E}_1 = 20\,\text{В}$, $\mathcal{E}_2 = 20\,\text{В}$
}

\variantsplitter

\addpersonalvariant{Алина Яшина}

\tasknumber{1}%
\task{%
    На резистор сопротивлением $R = 12\,\text{Ом}$ подали напряжение $U = 150\,\text{В}$.
    Определите ток, который потечёт через резистор, и мощность, выделяющуюся на нём.
}
\answer{%
    \begin{align*}
    \mathcal{I} &= \frac{ U }{ R } = \frac{ 150\,\text{В} }{ 12\,\text{Ом} } = 12{,}50\,\text{А},  \\
    P &= \frac{U^2}{ R } = \frac{ \sqr{ 150\,\text{В} } }{ 12\,\text{Ом} } = 1875{,}00\,\text{Вт}
    \end{align*}
}
\solutionspace{120pt}

\tasknumber{2}%
\task{%
    Замкнутая электрическая цепь состоит из ЭДС $\mathcal{E} = 2\,\text{В}$ и сопротивлением $r$
    и резистора $R = 10\,\text{Ом}$.
    Определите ток, протекающий в цепи.
    Какая тепловая энергия выделится на резисторе за время
    $\tau = 2\,\text{с}$? Какая работа будет совершена ЭДС за это время? Каков знак этой работы? Чему равен КПД цепи? Вычислите значения для 2 случаев:
    $r=0$ и $r = 20\,\text{Ом}$.
}
\answer{%
    \begin{align*}
    \mathcal{I}_1 &= \frac{ \mathcal{E} }{ R } = \frac{ 2\,\text{В} }{ 10\,\text{Ом} } = 0{,}20\,\text{А},  \\
    \mathcal{I}_2 &= \frac{ \mathcal{E} }{R + r} = \frac{ 2\,\text{В} }{10\,\text{Ом} + 20\,\text{Ом}} = 0{,}07\,\text{А},  \\
    Q_1 &= \mathcal{I}_1^2R\tau = \sqr{\frac{ \mathcal{E} }{ R }} R \tau
            = \sqr{\frac{ 2\,\text{В} }{ 10\,\text{Ом} }} \cdot 10\,\text{Ом} \cdot 2\,\text{с} = 0{,}800\,\text{Дж},  \\
    Q_2 &= \mathcal{I}_2^2R\tau = \sqr{\frac{ \mathcal{E} }{R + r}} R \tau
            = \sqr{\frac{ 2\,\text{В} }{10\,\text{Ом} + 20\,\text{Ом}}} \cdot 10\,\text{Ом} \cdot 2\,\text{с} = 0{,}098\,\text{Дж},  \\
    A_1 &= \mathcal{I}_1\tau\mathcal{E} = \frac{ \mathcal{E} }{R} \tau \mathcal{E}
            = \frac{\mathcal{E}^2 \tau}{ R } = \frac{\sqr{ 2\,\text{В} } \cdot 2\,\text{с}}{ 10\,\text{Ом} }
            = 0{,}800\,\text{Дж}, \text{положительна},  \\
    A_2 &= \mathcal{I}_2\tau\mathcal{E} = \frac{ \mathcal{E} }{R + r} \tau \mathcal{E}
            = \frac{\mathcal{E}^2 \tau}{R + r} = \frac{\sqr{ 2\,\text{В} } \cdot 2\,\text{с}}{10\,\text{Ом} + 20\,\text{Ом}}
            = 0{,}280\,\text{Дж}, \text{положительна},  \\
    \eta_1 &= \frac{ Q_1 }{ A_1 } = \ldots = \frac{ R }{ R } = 1,  \\
    \eta_2 &= \frac{ Q_2 }{ A_2 } = \ldots = \frac{ R }{R + r} = 0{,}35
    \end{align*}
}
\solutionspace{120pt}

\tasknumber{3}%
\task{%
    Лампочки, сопротивления которых $R_1 = 4{,}00\,\text{Ом}$ и $R_2 = 36{,}00\,\text{Ом}$, поочерёдно подключённные к некоторому источнику тока,
    потребляют одинаковую мощность.
    Найти внутреннее сопротивление источника и КПД цепи в каждом случае.
}
\answer{%
    \begin{align*}
        P_1 &= \sqr{\frac{ \mathcal{E} }{R_1 + r}}R_1,
        P_2  = \sqr{\frac{ \mathcal{E} }{R_2 + r}}R_2,
        P_1 = P_2 \implies  \\
        &\implies R_1 \sqr{R_2 + r} = R_2 \sqr{R_1 + r} \implies  \\
        &\implies R_1 R_2^2 + 2 R_1 R_2 r + R_1 r^2 =
                    R_2 R_1^2 + 2 R_2 R_1 r + R_2 r^2  \implies  \\
    &\implies r^2 (R_2 - R_1) = R_2^2 R_2 - R_1^2 R_2 \implies  \\
    &\implies r
            = \sqrt{R_1 R_2 \frac{R_2 - R_1}{R_2 - R_1}}
            = \sqrt{R_1 R_2}
            = \sqrt{4{,}00\,\text{Ом} \cdot 36{,}00\,\text{Ом}}
            = 12{,}00\,\text{Ом}.
            \\
    \eta_1
            &= \frac{ R_1 }{R_1 + r}
            = \frac{\sqrt{ R_1 }}{\sqrt{ R_1 } + \sqrt{ R_2 }}
            = 0{,}250,  \\
    \eta_2
            &= \frac{ R_2 }{R_2 + r}
            = \frac{ \sqrt{ R_2 } }{\sqrt{ R_2 } + \sqrt{ R_1 }}
            = 0{,}750
    \end{align*}
}
\solutionspace{120pt}

\tasknumber{4}%
\task{%
    Определите ток, протекающий через резистор $R = 10\,\text{Ом}$ и разность потенциалов на нём (см.
    рис.
    на доске),
    если $r_1 = 3\,\text{Ом}$, $r_2 = 3\,\text{Ом}$, $\mathcal{E}_1 = 40\,\text{В}$, $\mathcal{E}_2 = 60\,\text{В}$
}
% autogenerated
