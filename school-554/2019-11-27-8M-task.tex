\setdate{27~ноября~2019}
\setclass{8«М»}

\addpersonalvariant{Михаил Бурмистров}

\tasknumber{1}%
\task{%
    Сколько льда при температуре $0\celsius$ можно расплавить,
    сообщив ему энергию $8\,\text{МДж}$?
    Здесь (и во всех следующих задачах) используйте табличные значения из учебника.
}
\answer{%
    $
        Q
            = \lambda m \implies m
            = \frac Q{\lambda}
            = \frac { 8\,\text{МДж} }{ 340\,\frac{\text{кДж}}{\text{кг}} }
            \approx 23{,}5\,\text{кг}
    $
}
\solutionspace{120pt}

\tasknumber{2}%
\task{%
    Какое количество теплоты выделится при затвердевании $25\,\text{кг}$ расплавленного стали при температуре плавления?
}
\answer{%
    $
        Q
            = - \lambda m
            = - 84\,\frac{\text{кДж}}{\text{кг}} \cdot 25\,\text{кг}
            = - 2{,}1\,\text{МДж} \implies \abs{Q} = 2{,}1\,\text{МДж}
    $
}
\solutionspace{120pt}

\tasknumber{3}%
\task{%
    Какое количество теплоты необходимо для превращения воды массой $2\,\text{кг}$ при $t = 50\celsius$
    в пар при температуре $t_{100} = 100\celsius$?
}
\answer{%
    $
        Q
            = cm\Delta t + Lm
            = m\cbr{c(t_{100} - t) + L}
            = 2\,\text{кг} \cdot \cbr{4200\,\frac{\text{Дж}}{\text{кг}\cdot\text{К}}\cbr{100\celsius - 50\celsius} + 2{,}3\,\frac{\text{МДж}}{\text{кг}}}
            = 5{,}02\,\text{МДж}
    $
}
\solutionspace{120pt}

\tasknumber{4}%
\task{%
    Воду температурой $t = 40\celsius$ нагрели и превратили в пар при температуре $t_{100} = 100\celsius$,
    потратив $2500\,\text{кДж}$.
    Определите массу воды.
}
\answer{%
    $
        Q
            = cm\Delta t + Lm
            = m\cbr{c(t_{100} - t) + L}
        \implies
        m = \frac{Q}{c(t_{100} - t) + L}
            = \frac { 2500\,\text{кДж} }{4200\,\frac{\text{Дж}}{\text{кг}\cdot\text{К}}\cbr{100\celsius - 40\celsius} + 2{,}3\,\frac{\text{МДж}}{\text{кг}}}
            \approx 0{,}98\,\text{кг}
    $
}
\solutionspace{120pt}

\tasknumber{5}%
\task{%
    Цинковое тело температурой $T = 80\celsius$ опустили
    в воду температурой $t = 30\celsius$, масса которой равна массе тела.
    Определите, какая температура установится в сосуде.
}
\answer{%
    \begin{align*}
    Q_1 + Q_2 &= 0,  \\
    Q_1 &= c_1 m_1 \Delta t_1 = c_1 m (\theta - t_1),  \\
    Q_2 &= c_2 m_2 \Delta t_2 = c_2 m (\theta - t_2),  \\
    c_1 m (\theta - t_1) + c_2 m (\theta - t_2) &= 0,  \\
    c_1 (\theta - t_1) + c_2 (\theta - t_2) &= 0,  \\
    c_1 \theta - c_1 t_1 + c_2 \theta - c_2 t_2 &= 0,  \\
    (c_1 + c_2)\theta &= c_1 t_1 + c_2 t_2,  \\
    \theta &= \frac{c_1 t_1 + c_2 t_2}{c_1 + c_2}
            = \frac{4200\,\frac{\text{Дж}}{\text{кг}\cdot\text{К}} \cdot 30\celsius + 400\,\frac{\text{Дж}}{\text{кг}\cdot\text{К}} \cdot  80\celsius}{4200\,\frac{\text{Дж}}{\text{кг}\cdot\text{К}} + 400\,\frac{\text{Дж}}{\text{кг}\cdot\text{К}}}
            \approx 34{,}3 \celsius.
    \end{align*}
}

\variantsplitter

\addpersonalvariant{Максим Аксенов}

\tasknumber{1}%
\task{%
    Сколько льда при температуре $0\celsius$ можно расплавить,
    сообщив ему энергию $3\,\text{МДж}$?
    Здесь (и во всех следующих задачах) используйте табличные значения из учебника.
}
\answer{%
    $
        Q
            = \lambda m \implies m
            = \frac Q{\lambda}
            = \frac { 3\,\text{МДж} }{ 340\,\frac{\text{кДж}}{\text{кг}} }
            \approx 8{,}8\,\text{кг}
    $
}
\solutionspace{120pt}

\tasknumber{2}%
\task{%
    Какое количество теплоты выделится при затвердевании $15\,\text{кг}$ расплавленного свинца при температуре плавления?
}
\answer{%
    $
        Q
            = - \lambda m
            = - 25\,\frac{\text{кДж}}{\text{кг}} \cdot 15\,\text{кг}
            = - 0{,}4\,\text{МДж} \implies \abs{Q} = 0{,}4\,\text{МДж}
    $
}
\solutionspace{120pt}

\tasknumber{3}%
\task{%
    Какое количество теплоты необходимо для превращения воды массой $3\,\text{кг}$ при $t = 40\celsius$
    в пар при температуре $t_{100} = 100\celsius$?
}
\answer{%
    $
        Q
            = cm\Delta t + Lm
            = m\cbr{c(t_{100} - t) + L}
            = 3\,\text{кг} \cdot \cbr{4200\,\frac{\text{Дж}}{\text{кг}\cdot\text{К}}\cbr{100\celsius - 40\celsius} + 2{,}3\,\frac{\text{МДж}}{\text{кг}}}
            = 7{,}66\,\text{МДж}
    $
}
\solutionspace{120pt}

\tasknumber{4}%
\task{%
    Воду температурой $t = 10\celsius$ нагрели и превратили в пар при температуре $t_{100} = 100\celsius$,
    потратив $2000\,\text{кДж}$.
    Определите массу воды.
}
\answer{%
    $
        Q
            = cm\Delta t + Lm
            = m\cbr{c(t_{100} - t) + L}
        \implies
        m = \frac{Q}{c(t_{100} - t) + L}
            = \frac { 2000\,\text{кДж} }{4200\,\frac{\text{Дж}}{\text{кг}\cdot\text{К}}\cbr{100\celsius - 10\celsius} + 2{,}3\,\frac{\text{МДж}}{\text{кг}}}
            \approx 0{,}75\,\text{кг}
    $
}
\solutionspace{120pt}

\tasknumber{5}%
\task{%
    Цинковое тело температурой $T = 100\celsius$ опустили
    в воду температурой $t = 10\celsius$, масса которой равна массе тела.
    Определите, какая температура установится в сосуде.
}
\answer{%
    \begin{align*}
    Q_1 + Q_2 &= 0,  \\
    Q_1 &= c_1 m_1 \Delta t_1 = c_1 m (\theta - t_1),  \\
    Q_2 &= c_2 m_2 \Delta t_2 = c_2 m (\theta - t_2),  \\
    c_1 m (\theta - t_1) + c_2 m (\theta - t_2) &= 0,  \\
    c_1 (\theta - t_1) + c_2 (\theta - t_2) &= 0,  \\
    c_1 \theta - c_1 t_1 + c_2 \theta - c_2 t_2 &= 0,  \\
    (c_1 + c_2)\theta &= c_1 t_1 + c_2 t_2,  \\
    \theta &= \frac{c_1 t_1 + c_2 t_2}{c_1 + c_2}
            = \frac{4200\,\frac{\text{Дж}}{\text{кг}\cdot\text{К}} \cdot 10\celsius + 400\,\frac{\text{Дж}}{\text{кг}\cdot\text{К}} \cdot  100\celsius}{4200\,\frac{\text{Дж}}{\text{кг}\cdot\text{К}} + 400\,\frac{\text{Дж}}{\text{кг}\cdot\text{К}}}
            \approx 17{,}8 \celsius.
    \end{align*}
}

\variantsplitter

\addpersonalvariant{Маргарита Ахметова}

\tasknumber{1}%
\task{%
    Сколько льда при температуре $0\celsius$ можно расплавить,
    сообщив ему энергию $6\,\text{МДж}$?
    Здесь (и во всех следующих задачах) используйте табличные значения из учебника.
}
\answer{%
    $
        Q
            = \lambda m \implies m
            = \frac Q{\lambda}
            = \frac { 6\,\text{МДж} }{ 340\,\frac{\text{кДж}}{\text{кг}} }
            \approx 17{,}6\,\text{кг}
    $
}
\solutionspace{120pt}

\tasknumber{2}%
\task{%
    Какое количество теплоты выделится при затвердевании $25\,\text{кг}$ расплавленного меди при температуре плавления?
}
\answer{%
    $
        Q
            = - \lambda m
            = - 210\,\frac{\text{кДж}}{\text{кг}} \cdot 25\,\text{кг}
            = - 5{,}2\,\text{МДж} \implies \abs{Q} = 5{,}2\,\text{МДж}
    $
}
\solutionspace{120pt}

\tasknumber{3}%
\task{%
    Какое количество теплоты необходимо для превращения воды массой $2\,\text{кг}$ при $t = 50\celsius$
    в пар при температуре $t_{100} = 100\celsius$?
}
\answer{%
    $
        Q
            = cm\Delta t + Lm
            = m\cbr{c(t_{100} - t) + L}
            = 2\,\text{кг} \cdot \cbr{4200\,\frac{\text{Дж}}{\text{кг}\cdot\text{К}}\cbr{100\celsius - 50\celsius} + 2{,}3\,\frac{\text{МДж}}{\text{кг}}}
            = 5{,}02\,\text{МДж}
    $
}
\solutionspace{120pt}

\tasknumber{4}%
\task{%
    Воду температурой $t = 10\celsius$ нагрели и превратили в пар при температуре $t_{100} = 100\celsius$,
    потратив $4000\,\text{кДж}$.
    Определите массу воды.
}
\answer{%
    $
        Q
            = cm\Delta t + Lm
            = m\cbr{c(t_{100} - t) + L}
        \implies
        m = \frac{Q}{c(t_{100} - t) + L}
            = \frac { 4000\,\text{кДж} }{4200\,\frac{\text{Дж}}{\text{кг}\cdot\text{К}}\cbr{100\celsius - 10\celsius} + 2{,}3\,\frac{\text{МДж}}{\text{кг}}}
            \approx 1{,}49\,\text{кг}
    $
}
\solutionspace{120pt}

\tasknumber{5}%
\task{%
    Цинковое тело температурой $T = 100\celsius$ опустили
    в воду температурой $t = 10\celsius$, масса которой равна массе тела.
    Определите, какая температура установится в сосуде.
}
\answer{%
    \begin{align*}
    Q_1 + Q_2 &= 0,  \\
    Q_1 &= c_1 m_1 \Delta t_1 = c_1 m (\theta - t_1),  \\
    Q_2 &= c_2 m_2 \Delta t_2 = c_2 m (\theta - t_2),  \\
    c_1 m (\theta - t_1) + c_2 m (\theta - t_2) &= 0,  \\
    c_1 (\theta - t_1) + c_2 (\theta - t_2) &= 0,  \\
    c_1 \theta - c_1 t_1 + c_2 \theta - c_2 t_2 &= 0,  \\
    (c_1 + c_2)\theta &= c_1 t_1 + c_2 t_2,  \\
    \theta &= \frac{c_1 t_1 + c_2 t_2}{c_1 + c_2}
            = \frac{4200\,\frac{\text{Дж}}{\text{кг}\cdot\text{К}} \cdot 10\celsius + 400\,\frac{\text{Дж}}{\text{кг}\cdot\text{К}} \cdot  100\celsius}{4200\,\frac{\text{Дж}}{\text{кг}\cdot\text{К}} + 400\,\frac{\text{Дж}}{\text{кг}\cdot\text{К}}}
            \approx 17{,}8 \celsius.
    \end{align*}
}

\variantsplitter

\addpersonalvariant{Артём Глембо}

\tasknumber{1}%
\task{%
    Сколько льда при температуре $0\celsius$ можно расплавить,
    сообщив ему энергию $5\,\text{МДж}$?
    Здесь (и во всех следующих задачах) используйте табличные значения из учебника.
}
\answer{%
    $
        Q
            = \lambda m \implies m
            = \frac Q{\lambda}
            = \frac { 5\,\text{МДж} }{ 340\,\frac{\text{кДж}}{\text{кг}} }
            \approx 14{,}7\,\text{кг}
    $
}
\solutionspace{120pt}

\tasknumber{2}%
\task{%
    Какое количество теплоты выделится при затвердевании $20\,\text{кг}$ расплавленного стали при температуре плавления?
}
\answer{%
    $
        Q
            = - \lambda m
            = - 84\,\frac{\text{кДж}}{\text{кг}} \cdot 20\,\text{кг}
            = - 1{,}7\,\text{МДж} \implies \abs{Q} = 1{,}7\,\text{МДж}
    $
}
\solutionspace{120pt}

\tasknumber{3}%
\task{%
    Какое количество теплоты необходимо для превращения воды массой $5\,\text{кг}$ при $t = 30\celsius$
    в пар при температуре $t_{100} = 100\celsius$?
}
\answer{%
    $
        Q
            = cm\Delta t + Lm
            = m\cbr{c(t_{100} - t) + L}
            = 5\,\text{кг} \cdot \cbr{4200\,\frac{\text{Дж}}{\text{кг}\cdot\text{К}}\cbr{100\celsius - 30\celsius} + 2{,}3\,\frac{\text{МДж}}{\text{кг}}}
            = 12{,}97\,\text{МДж}
    $
}
\solutionspace{120pt}

\tasknumber{4}%
\task{%
    Воду температурой $t = 50\celsius$ нагрели и превратили в пар при температуре $t_{100} = 100\celsius$,
    потратив $2000\,\text{кДж}$.
    Определите массу воды.
}
\answer{%
    $
        Q
            = cm\Delta t + Lm
            = m\cbr{c(t_{100} - t) + L}
        \implies
        m = \frac{Q}{c(t_{100} - t) + L}
            = \frac { 2000\,\text{кДж} }{4200\,\frac{\text{Дж}}{\text{кг}\cdot\text{К}}\cbr{100\celsius - 50\celsius} + 2{,}3\,\frac{\text{МДж}}{\text{кг}}}
            \approx 0{,}80\,\text{кг}
    $
}
\solutionspace{120pt}

\tasknumber{5}%
\task{%
    Алюминиевое тело температурой $T = 100\celsius$ опустили
    в воду температурой $t = 30\celsius$, масса которой равна массе тела.
    Определите, какая температура установится в сосуде.
}
\answer{%
    \begin{align*}
    Q_1 + Q_2 &= 0,  \\
    Q_1 &= c_1 m_1 \Delta t_1 = c_1 m (\theta - t_1),  \\
    Q_2 &= c_2 m_2 \Delta t_2 = c_2 m (\theta - t_2),  \\
    c_1 m (\theta - t_1) + c_2 m (\theta - t_2) &= 0,  \\
    c_1 (\theta - t_1) + c_2 (\theta - t_2) &= 0,  \\
    c_1 \theta - c_1 t_1 + c_2 \theta - c_2 t_2 &= 0,  \\
    (c_1 + c_2)\theta &= c_1 t_1 + c_2 t_2,  \\
    \theta &= \frac{c_1 t_1 + c_2 t_2}{c_1 + c_2}
            = \frac{4200\,\frac{\text{Дж}}{\text{кг}\cdot\text{К}} \cdot 30\celsius + 920\,\frac{\text{Дж}}{\text{кг}\cdot\text{К}} \cdot  100\celsius}{4200\,\frac{\text{Дж}}{\text{кг}\cdot\text{К}} + 920\,\frac{\text{Дж}}{\text{кг}\cdot\text{К}}}
            \approx 42{,}6 \celsius.
    \end{align*}
}

\variantsplitter

\addpersonalvariant{Наталья Гончарова}

\tasknumber{1}%
\task{%
    Сколько льда при температуре $0\celsius$ можно расплавить,
    сообщив ему энергию $9\,\text{МДж}$?
    Здесь (и во всех следующих задачах) используйте табличные значения из учебника.
}
\answer{%
    $
        Q
            = \lambda m \implies m
            = \frac Q{\lambda}
            = \frac { 9\,\text{МДж} }{ 340\,\frac{\text{кДж}}{\text{кг}} }
            \approx 26{,}5\,\text{кг}
    $
}
\solutionspace{120pt}

\tasknumber{2}%
\task{%
    Какое количество теплоты выделится при затвердевании $15\,\text{кг}$ расплавленного свинца при температуре плавления?
}
\answer{%
    $
        Q
            = - \lambda m
            = - 25\,\frac{\text{кДж}}{\text{кг}} \cdot 15\,\text{кг}
            = - 0{,}4\,\text{МДж} \implies \abs{Q} = 0{,}4\,\text{МДж}
    $
}
\solutionspace{120pt}

\tasknumber{3}%
\task{%
    Какое количество теплоты необходимо для превращения воды массой $3\,\text{кг}$ при $t = 70\celsius$
    в пар при температуре $t_{100} = 100\celsius$?
}
\answer{%
    $
        Q
            = cm\Delta t + Lm
            = m\cbr{c(t_{100} - t) + L}
            = 3\,\text{кг} \cdot \cbr{4200\,\frac{\text{Дж}}{\text{кг}\cdot\text{К}}\cbr{100\celsius - 70\celsius} + 2{,}3\,\frac{\text{МДж}}{\text{кг}}}
            = 7{,}28\,\text{МДж}
    $
}
\solutionspace{120pt}

\tasknumber{4}%
\task{%
    Воду температурой $t = 50\celsius$ нагрели и превратили в пар при температуре $t_{100} = 100\celsius$,
    потратив $4000\,\text{кДж}$.
    Определите массу воды.
}
\answer{%
    $
        Q
            = cm\Delta t + Lm
            = m\cbr{c(t_{100} - t) + L}
        \implies
        m = \frac{Q}{c(t_{100} - t) + L}
            = \frac { 4000\,\text{кДж} }{4200\,\frac{\text{Дж}}{\text{кг}\cdot\text{К}}\cbr{100\celsius - 50\celsius} + 2{,}3\,\frac{\text{МДж}}{\text{кг}}}
            \approx 1{,}59\,\text{кг}
    $
}
\solutionspace{120pt}

\tasknumber{5}%
\task{%
    Стальное тело температурой $T = 100\celsius$ опустили
    в воду температурой $t = 30\celsius$, масса которой равна массе тела.
    Определите, какая температура установится в сосуде.
}
\answer{%
    \begin{align*}
    Q_1 + Q_2 &= 0,  \\
    Q_1 &= c_1 m_1 \Delta t_1 = c_1 m (\theta - t_1),  \\
    Q_2 &= c_2 m_2 \Delta t_2 = c_2 m (\theta - t_2),  \\
    c_1 m (\theta - t_1) + c_2 m (\theta - t_2) &= 0,  \\
    c_1 (\theta - t_1) + c_2 (\theta - t_2) &= 0,  \\
    c_1 \theta - c_1 t_1 + c_2 \theta - c_2 t_2 &= 0,  \\
    (c_1 + c_2)\theta &= c_1 t_1 + c_2 t_2,  \\
    \theta &= \frac{c_1 t_1 + c_2 t_2}{c_1 + c_2}
            = \frac{4200\,\frac{\text{Дж}}{\text{кг}\cdot\text{К}} \cdot 30\celsius + 500\,\frac{\text{Дж}}{\text{кг}\cdot\text{К}} \cdot  100\celsius}{4200\,\frac{\text{Дж}}{\text{кг}\cdot\text{К}} + 500\,\frac{\text{Дж}}{\text{кг}\cdot\text{К}}}
            \approx 37{,}4 \celsius.
    \end{align*}
}

\variantsplitter

\addpersonalvariant{Файёзбек Касымов}

\tasknumber{1}%
\task{%
    Сколько льда при температуре $0\celsius$ можно расплавить,
    сообщив ему энергию $8\,\text{МДж}$?
    Здесь (и во всех следующих задачах) используйте табличные значения из учебника.
}
\answer{%
    $
        Q
            = \lambda m \implies m
            = \frac Q{\lambda}
            = \frac { 8\,\text{МДж} }{ 340\,\frac{\text{кДж}}{\text{кг}} }
            \approx 23{,}5\,\text{кг}
    $
}
\solutionspace{120pt}

\tasknumber{2}%
\task{%
    Какое количество теплоты выделится при затвердевании $15\,\text{кг}$ расплавленного свинца при температуре плавления?
}
\answer{%
    $
        Q
            = - \lambda m
            = - 25\,\frac{\text{кДж}}{\text{кг}} \cdot 15\,\text{кг}
            = - 0{,}4\,\text{МДж} \implies \abs{Q} = 0{,}4\,\text{МДж}
    $
}
\solutionspace{120pt}

\tasknumber{3}%
\task{%
    Какое количество теплоты необходимо для превращения воды массой $5\,\text{кг}$ при $t = 50\celsius$
    в пар при температуре $t_{100} = 100\celsius$?
}
\answer{%
    $
        Q
            = cm\Delta t + Lm
            = m\cbr{c(t_{100} - t) + L}
            = 5\,\text{кг} \cdot \cbr{4200\,\frac{\text{Дж}}{\text{кг}\cdot\text{К}}\cbr{100\celsius - 50\celsius} + 2{,}3\,\frac{\text{МДж}}{\text{кг}}}
            = 12{,}55\,\text{МДж}
    $
}
\solutionspace{120pt}

\tasknumber{4}%
\task{%
    Воду температурой $t = 30\celsius$ нагрели и превратили в пар при температуре $t_{100} = 100\celsius$,
    потратив $2000\,\text{кДж}$.
    Определите массу воды.
}
\answer{%
    $
        Q
            = cm\Delta t + Lm
            = m\cbr{c(t_{100} - t) + L}
        \implies
        m = \frac{Q}{c(t_{100} - t) + L}
            = \frac { 2000\,\text{кДж} }{4200\,\frac{\text{Дж}}{\text{кг}\cdot\text{К}}\cbr{100\celsius - 30\celsius} + 2{,}3\,\frac{\text{МДж}}{\text{кг}}}
            \approx 0{,}77\,\text{кг}
    $
}
\solutionspace{120pt}

\tasknumber{5}%
\task{%
    Стальное тело температурой $T = 70\celsius$ опустили
    в воду температурой $t = 30\celsius$, масса которой равна массе тела.
    Определите, какая температура установится в сосуде.
}
\answer{%
    \begin{align*}
    Q_1 + Q_2 &= 0,  \\
    Q_1 &= c_1 m_1 \Delta t_1 = c_1 m (\theta - t_1),  \\
    Q_2 &= c_2 m_2 \Delta t_2 = c_2 m (\theta - t_2),  \\
    c_1 m (\theta - t_1) + c_2 m (\theta - t_2) &= 0,  \\
    c_1 (\theta - t_1) + c_2 (\theta - t_2) &= 0,  \\
    c_1 \theta - c_1 t_1 + c_2 \theta - c_2 t_2 &= 0,  \\
    (c_1 + c_2)\theta &= c_1 t_1 + c_2 t_2,  \\
    \theta &= \frac{c_1 t_1 + c_2 t_2}{c_1 + c_2}
            = \frac{4200\,\frac{\text{Дж}}{\text{кг}\cdot\text{К}} \cdot 30\celsius + 500\,\frac{\text{Дж}}{\text{кг}\cdot\text{К}} \cdot  70\celsius}{4200\,\frac{\text{Дж}}{\text{кг}\cdot\text{К}} + 500\,\frac{\text{Дж}}{\text{кг}\cdot\text{К}}}
            \approx 34{,}3 \celsius.
    \end{align*}
}

\variantsplitter

\addpersonalvariant{Александр Козинец}

\tasknumber{1}%
\task{%
    Сколько льда при температуре $0\celsius$ можно расплавить,
    сообщив ему энергию $9\,\text{МДж}$?
    Здесь (и во всех следующих задачах) используйте табличные значения из учебника.
}
\answer{%
    $
        Q
            = \lambda m \implies m
            = \frac Q{\lambda}
            = \frac { 9\,\text{МДж} }{ 340\,\frac{\text{кДж}}{\text{кг}} }
            \approx 26{,}5\,\text{кг}
    $
}
\solutionspace{120pt}

\tasknumber{2}%
\task{%
    Какое количество теплоты выделится при затвердевании $75\,\text{кг}$ расплавленного свинца при температуре плавления?
}
\answer{%
    $
        Q
            = - \lambda m
            = - 25\,\frac{\text{кДж}}{\text{кг}} \cdot 75\,\text{кг}
            = - 1{,}9\,\text{МДж} \implies \abs{Q} = 1{,}9\,\text{МДж}
    $
}
\solutionspace{120pt}

\tasknumber{3}%
\task{%
    Какое количество теплоты необходимо для превращения воды массой $2\,\text{кг}$ при $t = 20\celsius$
    в пар при температуре $t_{100} = 100\celsius$?
}
\answer{%
    $
        Q
            = cm\Delta t + Lm
            = m\cbr{c(t_{100} - t) + L}
            = 2\,\text{кг} \cdot \cbr{4200\,\frac{\text{Дж}}{\text{кг}\cdot\text{К}}\cbr{100\celsius - 20\celsius} + 2{,}3\,\frac{\text{МДж}}{\text{кг}}}
            = 5{,}27\,\text{МДж}
    $
}
\solutionspace{120pt}

\tasknumber{4}%
\task{%
    Воду температурой $t = 40\celsius$ нагрели и превратили в пар при температуре $t_{100} = 100\celsius$,
    потратив $5000\,\text{кДж}$.
    Определите массу воды.
}
\answer{%
    $
        Q
            = cm\Delta t + Lm
            = m\cbr{c(t_{100} - t) + L}
        \implies
        m = \frac{Q}{c(t_{100} - t) + L}
            = \frac { 5000\,\text{кДж} }{4200\,\frac{\text{Дж}}{\text{кг}\cdot\text{К}}\cbr{100\celsius - 40\celsius} + 2{,}3\,\frac{\text{МДж}}{\text{кг}}}
            \approx 1{,}96\,\text{кг}
    $
}
\solutionspace{120pt}

\tasknumber{5}%
\task{%
    Стальное тело температурой $T = 100\celsius$ опустили
    в воду температурой $t = 30\celsius$, масса которой равна массе тела.
    Определите, какая температура установится в сосуде.
}
\answer{%
    \begin{align*}
    Q_1 + Q_2 &= 0,  \\
    Q_1 &= c_1 m_1 \Delta t_1 = c_1 m (\theta - t_1),  \\
    Q_2 &= c_2 m_2 \Delta t_2 = c_2 m (\theta - t_2),  \\
    c_1 m (\theta - t_1) + c_2 m (\theta - t_2) &= 0,  \\
    c_1 (\theta - t_1) + c_2 (\theta - t_2) &= 0,  \\
    c_1 \theta - c_1 t_1 + c_2 \theta - c_2 t_2 &= 0,  \\
    (c_1 + c_2)\theta &= c_1 t_1 + c_2 t_2,  \\
    \theta &= \frac{c_1 t_1 + c_2 t_2}{c_1 + c_2}
            = \frac{4200\,\frac{\text{Дж}}{\text{кг}\cdot\text{К}} \cdot 30\celsius + 500\,\frac{\text{Дж}}{\text{кг}\cdot\text{К}} \cdot  100\celsius}{4200\,\frac{\text{Дж}}{\text{кг}\cdot\text{К}} + 500\,\frac{\text{Дж}}{\text{кг}\cdot\text{К}}}
            \approx 37{,}4 \celsius.
    \end{align*}
}

\variantsplitter

\addpersonalvariant{Екатерина Медведева}

\tasknumber{1}%
\task{%
    Сколько льда при температуре $0\celsius$ можно расплавить,
    сообщив ему энергию $2\,\text{МДж}$?
    Здесь (и во всех следующих задачах) используйте табличные значения из учебника.
}
\answer{%
    $
        Q
            = \lambda m \implies m
            = \frac Q{\lambda}
            = \frac { 2\,\text{МДж} }{ 340\,\frac{\text{кДж}}{\text{кг}} }
            \approx 5{,}9\,\text{кг}
    $
}
\solutionspace{120pt}

\tasknumber{2}%
\task{%
    Какое количество теплоты выделится при затвердевании $20\,\text{кг}$ расплавленного алюминия при температуре плавления?
}
\answer{%
    $
        Q
            = - \lambda m
            = - 390\,\frac{\text{кДж}}{\text{кг}} \cdot 20\,\text{кг}
            = - 7{,}8\,\text{МДж} \implies \abs{Q} = 7{,}8\,\text{МДж}
    $
}
\solutionspace{120pt}

\tasknumber{3}%
\task{%
    Какое количество теплоты необходимо для превращения воды массой $5\,\text{кг}$ при $t = 30\celsius$
    в пар при температуре $t_{100} = 100\celsius$?
}
\answer{%
    $
        Q
            = cm\Delta t + Lm
            = m\cbr{c(t_{100} - t) + L}
            = 5\,\text{кг} \cdot \cbr{4200\,\frac{\text{Дж}}{\text{кг}\cdot\text{К}}\cbr{100\celsius - 30\celsius} + 2{,}3\,\frac{\text{МДж}}{\text{кг}}}
            = 12{,}97\,\text{МДж}
    $
}
\solutionspace{120pt}

\tasknumber{4}%
\task{%
    Воду температурой $t = 30\celsius$ нагрели и превратили в пар при температуре $t_{100} = 100\celsius$,
    потратив $5000\,\text{кДж}$.
    Определите массу воды.
}
\answer{%
    $
        Q
            = cm\Delta t + Lm
            = m\cbr{c(t_{100} - t) + L}
        \implies
        m = \frac{Q}{c(t_{100} - t) + L}
            = \frac { 5000\,\text{кДж} }{4200\,\frac{\text{Дж}}{\text{кг}\cdot\text{К}}\cbr{100\celsius - 30\celsius} + 2{,}3\,\frac{\text{МДж}}{\text{кг}}}
            \approx 1{,}93\,\text{кг}
    $
}
\solutionspace{120pt}

\tasknumber{5}%
\task{%
    Стальное тело температурой $T = 80\celsius$ опустили
    в воду температурой $t = 10\celsius$, масса которой равна массе тела.
    Определите, какая температура установится в сосуде.
}
\answer{%
    \begin{align*}
    Q_1 + Q_2 &= 0,  \\
    Q_1 &= c_1 m_1 \Delta t_1 = c_1 m (\theta - t_1),  \\
    Q_2 &= c_2 m_2 \Delta t_2 = c_2 m (\theta - t_2),  \\
    c_1 m (\theta - t_1) + c_2 m (\theta - t_2) &= 0,  \\
    c_1 (\theta - t_1) + c_2 (\theta - t_2) &= 0,  \\
    c_1 \theta - c_1 t_1 + c_2 \theta - c_2 t_2 &= 0,  \\
    (c_1 + c_2)\theta &= c_1 t_1 + c_2 t_2,  \\
    \theta &= \frac{c_1 t_1 + c_2 t_2}{c_1 + c_2}
            = \frac{4200\,\frac{\text{Дж}}{\text{кг}\cdot\text{К}} \cdot 10\celsius + 500\,\frac{\text{Дж}}{\text{кг}\cdot\text{К}} \cdot  80\celsius}{4200\,\frac{\text{Дж}}{\text{кг}\cdot\text{К}} + 500\,\frac{\text{Дж}}{\text{кг}\cdot\text{К}}}
            \approx 17{,}4 \celsius.
    \end{align*}
}

\variantsplitter

\addpersonalvariant{Константин Мельник}

\tasknumber{1}%
\task{%
    Сколько льда при температуре $0\celsius$ можно расплавить,
    сообщив ему энергию $5\,\text{МДж}$?
    Здесь (и во всех следующих задачах) используйте табличные значения из учебника.
}
\answer{%
    $
        Q
            = \lambda m \implies m
            = \frac Q{\lambda}
            = \frac { 5\,\text{МДж} }{ 340\,\frac{\text{кДж}}{\text{кг}} }
            \approx 14{,}7\,\text{кг}
    $
}
\solutionspace{120pt}

\tasknumber{2}%
\task{%
    Какое количество теплоты выделится при затвердевании $25\,\text{кг}$ расплавленного стали при температуре плавления?
}
\answer{%
    $
        Q
            = - \lambda m
            = - 84\,\frac{\text{кДж}}{\text{кг}} \cdot 25\,\text{кг}
            = - 2{,}1\,\text{МДж} \implies \abs{Q} = 2{,}1\,\text{МДж}
    $
}
\solutionspace{120pt}

\tasknumber{3}%
\task{%
    Какое количество теплоты необходимо для превращения воды массой $2\,\text{кг}$ при $t = 40\celsius$
    в пар при температуре $t_{100} = 100\celsius$?
}
\answer{%
    $
        Q
            = cm\Delta t + Lm
            = m\cbr{c(t_{100} - t) + L}
            = 2\,\text{кг} \cdot \cbr{4200\,\frac{\text{Дж}}{\text{кг}\cdot\text{К}}\cbr{100\celsius - 40\celsius} + 2{,}3\,\frac{\text{МДж}}{\text{кг}}}
            = 5{,}10\,\text{МДж}
    $
}
\solutionspace{120pt}

\tasknumber{4}%
\task{%
    Воду температурой $t = 30\celsius$ нагрели и превратили в пар при температуре $t_{100} = 100\celsius$,
    потратив $2000\,\text{кДж}$.
    Определите массу воды.
}
\answer{%
    $
        Q
            = cm\Delta t + Lm
            = m\cbr{c(t_{100} - t) + L}
        \implies
        m = \frac{Q}{c(t_{100} - t) + L}
            = \frac { 2000\,\text{кДж} }{4200\,\frac{\text{Дж}}{\text{кг}\cdot\text{К}}\cbr{100\celsius - 30\celsius} + 2{,}3\,\frac{\text{МДж}}{\text{кг}}}
            \approx 0{,}77\,\text{кг}
    $
}
\solutionspace{120pt}

\tasknumber{5}%
\task{%
    Стальное тело температурой $T = 80\celsius$ опустили
    в воду температурой $t = 10\celsius$, масса которой равна массе тела.
    Определите, какая температура установится в сосуде.
}
\answer{%
    \begin{align*}
    Q_1 + Q_2 &= 0,  \\
    Q_1 &= c_1 m_1 \Delta t_1 = c_1 m (\theta - t_1),  \\
    Q_2 &= c_2 m_2 \Delta t_2 = c_2 m (\theta - t_2),  \\
    c_1 m (\theta - t_1) + c_2 m (\theta - t_2) &= 0,  \\
    c_1 (\theta - t_1) + c_2 (\theta - t_2) &= 0,  \\
    c_1 \theta - c_1 t_1 + c_2 \theta - c_2 t_2 &= 0,  \\
    (c_1 + c_2)\theta &= c_1 t_1 + c_2 t_2,  \\
    \theta &= \frac{c_1 t_1 + c_2 t_2}{c_1 + c_2}
            = \frac{4200\,\frac{\text{Дж}}{\text{кг}\cdot\text{К}} \cdot 10\celsius + 500\,\frac{\text{Дж}}{\text{кг}\cdot\text{К}} \cdot  80\celsius}{4200\,\frac{\text{Дж}}{\text{кг}\cdot\text{К}} + 500\,\frac{\text{Дж}}{\text{кг}\cdot\text{К}}}
            \approx 17{,}4 \celsius.
    \end{align*}
}

\variantsplitter

\addpersonalvariant{Степан Небоваренков}

\tasknumber{1}%
\task{%
    Сколько льда при температуре $0\celsius$ можно расплавить,
    сообщив ему энергию $8\,\text{МДж}$?
    Здесь (и во всех следующих задачах) используйте табличные значения из учебника.
}
\answer{%
    $
        Q
            = \lambda m \implies m
            = \frac Q{\lambda}
            = \frac { 8\,\text{МДж} }{ 340\,\frac{\text{кДж}}{\text{кг}} }
            \approx 23{,}5\,\text{кг}
    $
}
\solutionspace{120pt}

\tasknumber{2}%
\task{%
    Какое количество теплоты выделится при затвердевании $30\,\text{кг}$ расплавленного меди при температуре плавления?
}
\answer{%
    $
        Q
            = - \lambda m
            = - 210\,\frac{\text{кДж}}{\text{кг}} \cdot 30\,\text{кг}
            = - 6{,}3\,\text{МДж} \implies \abs{Q} = 6{,}3\,\text{МДж}
    $
}
\solutionspace{120pt}

\tasknumber{3}%
\task{%
    Какое количество теплоты необходимо для превращения воды массой $2\,\text{кг}$ при $t = 50\celsius$
    в пар при температуре $t_{100} = 100\celsius$?
}
\answer{%
    $
        Q
            = cm\Delta t + Lm
            = m\cbr{c(t_{100} - t) + L}
            = 2\,\text{кг} \cdot \cbr{4200\,\frac{\text{Дж}}{\text{кг}\cdot\text{К}}\cbr{100\celsius - 50\celsius} + 2{,}3\,\frac{\text{МДж}}{\text{кг}}}
            = 5{,}02\,\text{МДж}
    $
}
\solutionspace{120pt}

\tasknumber{4}%
\task{%
    Воду температурой $t = 50\celsius$ нагрели и превратили в пар при температуре $t_{100} = 100\celsius$,
    потратив $4000\,\text{кДж}$.
    Определите массу воды.
}
\answer{%
    $
        Q
            = cm\Delta t + Lm
            = m\cbr{c(t_{100} - t) + L}
        \implies
        m = \frac{Q}{c(t_{100} - t) + L}
            = \frac { 4000\,\text{кДж} }{4200\,\frac{\text{Дж}}{\text{кг}\cdot\text{К}}\cbr{100\celsius - 50\celsius} + 2{,}3\,\frac{\text{МДж}}{\text{кг}}}
            \approx 1{,}59\,\text{кг}
    $
}
\solutionspace{120pt}

\tasknumber{5}%
\task{%
    Стальное тело температурой $T = 70\celsius$ опустили
    в воду температурой $t = 10\celsius$, масса которой равна массе тела.
    Определите, какая температура установится в сосуде.
}
\answer{%
    \begin{align*}
    Q_1 + Q_2 &= 0,  \\
    Q_1 &= c_1 m_1 \Delta t_1 = c_1 m (\theta - t_1),  \\
    Q_2 &= c_2 m_2 \Delta t_2 = c_2 m (\theta - t_2),  \\
    c_1 m (\theta - t_1) + c_2 m (\theta - t_2) &= 0,  \\
    c_1 (\theta - t_1) + c_2 (\theta - t_2) &= 0,  \\
    c_1 \theta - c_1 t_1 + c_2 \theta - c_2 t_2 &= 0,  \\
    (c_1 + c_2)\theta &= c_1 t_1 + c_2 t_2,  \\
    \theta &= \frac{c_1 t_1 + c_2 t_2}{c_1 + c_2}
            = \frac{4200\,\frac{\text{Дж}}{\text{кг}\cdot\text{К}} \cdot 10\celsius + 500\,\frac{\text{Дж}}{\text{кг}\cdot\text{К}} \cdot  70\celsius}{4200\,\frac{\text{Дж}}{\text{кг}\cdot\text{К}} + 500\,\frac{\text{Дж}}{\text{кг}\cdot\text{К}}}
            \approx 16{,}4 \celsius.
    \end{align*}
}

\variantsplitter

\addpersonalvariant{Матвей Неретин}

\tasknumber{1}%
\task{%
    Сколько льда при температуре $0\celsius$ можно расплавить,
    сообщив ему энергию $9\,\text{МДж}$?
    Здесь (и во всех следующих задачах) используйте табличные значения из учебника.
}
\answer{%
    $
        Q
            = \lambda m \implies m
            = \frac Q{\lambda}
            = \frac { 9\,\text{МДж} }{ 340\,\frac{\text{кДж}}{\text{кг}} }
            \approx 26{,}5\,\text{кг}
    $
}
\solutionspace{120pt}

\tasknumber{2}%
\task{%
    Какое количество теплоты выделится при затвердевании $50\,\text{кг}$ расплавленного свинца при температуре плавления?
}
\answer{%
    $
        Q
            = - \lambda m
            = - 25\,\frac{\text{кДж}}{\text{кг}} \cdot 50\,\text{кг}
            = - 1{,}2\,\text{МДж} \implies \abs{Q} = 1{,}2\,\text{МДж}
    $
}
\solutionspace{120pt}

\tasknumber{3}%
\task{%
    Какое количество теплоты необходимо для превращения воды массой $3\,\text{кг}$ при $t = 60\celsius$
    в пар при температуре $t_{100} = 100\celsius$?
}
\answer{%
    $
        Q
            = cm\Delta t + Lm
            = m\cbr{c(t_{100} - t) + L}
            = 3\,\text{кг} \cdot \cbr{4200\,\frac{\text{Дж}}{\text{кг}\cdot\text{К}}\cbr{100\celsius - 60\celsius} + 2{,}3\,\frac{\text{МДж}}{\text{кг}}}
            = 7{,}40\,\text{МДж}
    $
}
\solutionspace{120pt}

\tasknumber{4}%
\task{%
    Воду температурой $t = 70\celsius$ нагрели и превратили в пар при температуре $t_{100} = 100\celsius$,
    потратив $2500\,\text{кДж}$.
    Определите массу воды.
}
\answer{%
    $
        Q
            = cm\Delta t + Lm
            = m\cbr{c(t_{100} - t) + L}
        \implies
        m = \frac{Q}{c(t_{100} - t) + L}
            = \frac { 2500\,\text{кДж} }{4200\,\frac{\text{Дж}}{\text{кг}\cdot\text{К}}\cbr{100\celsius - 70\celsius} + 2{,}3\,\frac{\text{МДж}}{\text{кг}}}
            \approx 1{,}03\,\text{кг}
    $
}
\solutionspace{120pt}

\tasknumber{5}%
\task{%
    Стальное тело температурой $T = 70\celsius$ опустили
    в воду температурой $t = 30\celsius$, масса которой равна массе тела.
    Определите, какая температура установится в сосуде.
}
\answer{%
    \begin{align*}
    Q_1 + Q_2 &= 0,  \\
    Q_1 &= c_1 m_1 \Delta t_1 = c_1 m (\theta - t_1),  \\
    Q_2 &= c_2 m_2 \Delta t_2 = c_2 m (\theta - t_2),  \\
    c_1 m (\theta - t_1) + c_2 m (\theta - t_2) &= 0,  \\
    c_1 (\theta - t_1) + c_2 (\theta - t_2) &= 0,  \\
    c_1 \theta - c_1 t_1 + c_2 \theta - c_2 t_2 &= 0,  \\
    (c_1 + c_2)\theta &= c_1 t_1 + c_2 t_2,  \\
    \theta &= \frac{c_1 t_1 + c_2 t_2}{c_1 + c_2}
            = \frac{4200\,\frac{\text{Дж}}{\text{кг}\cdot\text{К}} \cdot 30\celsius + 500\,\frac{\text{Дж}}{\text{кг}\cdot\text{К}} \cdot  70\celsius}{4200\,\frac{\text{Дж}}{\text{кг}\cdot\text{К}} + 500\,\frac{\text{Дж}}{\text{кг}\cdot\text{К}}}
            \approx 34{,}3 \celsius.
    \end{align*}
}

\variantsplitter

\addpersonalvariant{Мария Никонова}

\tasknumber{1}%
\task{%
    Сколько льда при температуре $0\celsius$ можно расплавить,
    сообщив ему энергию $4\,\text{МДж}$?
    Здесь (и во всех следующих задачах) используйте табличные значения из учебника.
}
\answer{%
    $
        Q
            = \lambda m \implies m
            = \frac Q{\lambda}
            = \frac { 4\,\text{МДж} }{ 340\,\frac{\text{кДж}}{\text{кг}} }
            \approx 11{,}8\,\text{кг}
    $
}
\solutionspace{120pt}

\tasknumber{2}%
\task{%
    Какое количество теплоты выделится при затвердевании $25\,\text{кг}$ расплавленного меди при температуре плавления?
}
\answer{%
    $
        Q
            = - \lambda m
            = - 210\,\frac{\text{кДж}}{\text{кг}} \cdot 25\,\text{кг}
            = - 5{,}2\,\text{МДж} \implies \abs{Q} = 5{,}2\,\text{МДж}
    $
}
\solutionspace{120pt}

\tasknumber{3}%
\task{%
    Какое количество теплоты необходимо для превращения воды массой $3\,\text{кг}$ при $t = 20\celsius$
    в пар при температуре $t_{100} = 100\celsius$?
}
\answer{%
    $
        Q
            = cm\Delta t + Lm
            = m\cbr{c(t_{100} - t) + L}
            = 3\,\text{кг} \cdot \cbr{4200\,\frac{\text{Дж}}{\text{кг}\cdot\text{К}}\cbr{100\celsius - 20\celsius} + 2{,}3\,\frac{\text{МДж}}{\text{кг}}}
            = 7{,}91\,\text{МДж}
    $
}
\solutionspace{120pt}

\tasknumber{4}%
\task{%
    Воду температурой $t = 50\celsius$ нагрели и превратили в пар при температуре $t_{100} = 100\celsius$,
    потратив $2000\,\text{кДж}$.
    Определите массу воды.
}
\answer{%
    $
        Q
            = cm\Delta t + Lm
            = m\cbr{c(t_{100} - t) + L}
        \implies
        m = \frac{Q}{c(t_{100} - t) + L}
            = \frac { 2000\,\text{кДж} }{4200\,\frac{\text{Дж}}{\text{кг}\cdot\text{К}}\cbr{100\celsius - 50\celsius} + 2{,}3\,\frac{\text{МДж}}{\text{кг}}}
            \approx 0{,}80\,\text{кг}
    $
}
\solutionspace{120pt}

\tasknumber{5}%
\task{%
    Цинковое тело температурой $T = 80\celsius$ опустили
    в воду температурой $t = 10\celsius$, масса которой равна массе тела.
    Определите, какая температура установится в сосуде.
}
\answer{%
    \begin{align*}
    Q_1 + Q_2 &= 0,  \\
    Q_1 &= c_1 m_1 \Delta t_1 = c_1 m (\theta - t_1),  \\
    Q_2 &= c_2 m_2 \Delta t_2 = c_2 m (\theta - t_2),  \\
    c_1 m (\theta - t_1) + c_2 m (\theta - t_2) &= 0,  \\
    c_1 (\theta - t_1) + c_2 (\theta - t_2) &= 0,  \\
    c_1 \theta - c_1 t_1 + c_2 \theta - c_2 t_2 &= 0,  \\
    (c_1 + c_2)\theta &= c_1 t_1 + c_2 t_2,  \\
    \theta &= \frac{c_1 t_1 + c_2 t_2}{c_1 + c_2}
            = \frac{4200\,\frac{\text{Дж}}{\text{кг}\cdot\text{К}} \cdot 10\celsius + 400\,\frac{\text{Дж}}{\text{кг}\cdot\text{К}} \cdot  80\celsius}{4200\,\frac{\text{Дж}}{\text{кг}\cdot\text{К}} + 400\,\frac{\text{Дж}}{\text{кг}\cdot\text{К}}}
            \approx 16{,}1 \celsius.
    \end{align*}
}

\variantsplitter

\addpersonalvariant{Даниил Палаткин}

\tasknumber{1}%
\task{%
    Сколько льда при температуре $0\celsius$ можно расплавить,
    сообщив ему энергию $6\,\text{МДж}$?
    Здесь (и во всех следующих задачах) используйте табличные значения из учебника.
}
\answer{%
    $
        Q
            = \lambda m \implies m
            = \frac Q{\lambda}
            = \frac { 6\,\text{МДж} }{ 340\,\frac{\text{кДж}}{\text{кг}} }
            \approx 17{,}6\,\text{кг}
    $
}
\solutionspace{120pt}

\tasknumber{2}%
\task{%
    Какое количество теплоты выделится при затвердевании $15\,\text{кг}$ расплавленного алюминия при температуре плавления?
}
\answer{%
    $
        Q
            = - \lambda m
            = - 390\,\frac{\text{кДж}}{\text{кг}} \cdot 15\,\text{кг}
            = - 5{,}8\,\text{МДж} \implies \abs{Q} = 5{,}8\,\text{МДж}
    $
}
\solutionspace{120pt}

\tasknumber{3}%
\task{%
    Какое количество теплоты необходимо для превращения воды массой $3\,\text{кг}$ при $t = 20\celsius$
    в пар при температуре $t_{100} = 100\celsius$?
}
\answer{%
    $
        Q
            = cm\Delta t + Lm
            = m\cbr{c(t_{100} - t) + L}
            = 3\,\text{кг} \cdot \cbr{4200\,\frac{\text{Дж}}{\text{кг}\cdot\text{К}}\cbr{100\celsius - 20\celsius} + 2{,}3\,\frac{\text{МДж}}{\text{кг}}}
            = 7{,}91\,\text{МДж}
    $
}
\solutionspace{120pt}

\tasknumber{4}%
\task{%
    Воду температурой $t = 60\celsius$ нагрели и превратили в пар при температуре $t_{100} = 100\celsius$,
    потратив $5000\,\text{кДж}$.
    Определите массу воды.
}
\answer{%
    $
        Q
            = cm\Delta t + Lm
            = m\cbr{c(t_{100} - t) + L}
        \implies
        m = \frac{Q}{c(t_{100} - t) + L}
            = \frac { 5000\,\text{кДж} }{4200\,\frac{\text{Дж}}{\text{кг}\cdot\text{К}}\cbr{100\celsius - 60\celsius} + 2{,}3\,\frac{\text{МДж}}{\text{кг}}}
            \approx 2{,}03\,\text{кг}
    $
}
\solutionspace{120pt}

\tasknumber{5}%
\task{%
    Стальное тело температурой $T = 100\celsius$ опустили
    в воду температурой $t = 20\celsius$, масса которой равна массе тела.
    Определите, какая температура установится в сосуде.
}
\answer{%
    \begin{align*}
    Q_1 + Q_2 &= 0,  \\
    Q_1 &= c_1 m_1 \Delta t_1 = c_1 m (\theta - t_1),  \\
    Q_2 &= c_2 m_2 \Delta t_2 = c_2 m (\theta - t_2),  \\
    c_1 m (\theta - t_1) + c_2 m (\theta - t_2) &= 0,  \\
    c_1 (\theta - t_1) + c_2 (\theta - t_2) &= 0,  \\
    c_1 \theta - c_1 t_1 + c_2 \theta - c_2 t_2 &= 0,  \\
    (c_1 + c_2)\theta &= c_1 t_1 + c_2 t_2,  \\
    \theta &= \frac{c_1 t_1 + c_2 t_2}{c_1 + c_2}
            = \frac{4200\,\frac{\text{Дж}}{\text{кг}\cdot\text{К}} \cdot 20\celsius + 500\,\frac{\text{Дж}}{\text{кг}\cdot\text{К}} \cdot  100\celsius}{4200\,\frac{\text{Дж}}{\text{кг}\cdot\text{К}} + 500\,\frac{\text{Дж}}{\text{кг}\cdot\text{К}}}
            \approx 28{,}5 \celsius.
    \end{align*}
}

\variantsplitter

\addpersonalvariant{Илья Пичугин}

\tasknumber{1}%
\task{%
    Сколько льда при температуре $0\celsius$ можно расплавить,
    сообщив ему энергию $9\,\text{МДж}$?
    Здесь (и во всех следующих задачах) используйте табличные значения из учебника.
}
\answer{%
    $
        Q
            = \lambda m \implies m
            = \frac Q{\lambda}
            = \frac { 9\,\text{МДж} }{ 340\,\frac{\text{кДж}}{\text{кг}} }
            \approx 26{,}5\,\text{кг}
    $
}
\solutionspace{120pt}

\tasknumber{2}%
\task{%
    Какое количество теплоты выделится при затвердевании $30\,\text{кг}$ расплавленного алюминия при температуре плавления?
}
\answer{%
    $
        Q
            = - \lambda m
            = - 390\,\frac{\text{кДж}}{\text{кг}} \cdot 30\,\text{кг}
            = - 11{,}7\,\text{МДж} \implies \abs{Q} = 11{,}7\,\text{МДж}
    $
}
\solutionspace{120pt}

\tasknumber{3}%
\task{%
    Какое количество теплоты необходимо для превращения воды массой $15\,\text{кг}$ при $t = 50\celsius$
    в пар при температуре $t_{100} = 100\celsius$?
}
\answer{%
    $
        Q
            = cm\Delta t + Lm
            = m\cbr{c(t_{100} - t) + L}
            = 15\,\text{кг} \cdot \cbr{4200\,\frac{\text{Дж}}{\text{кг}\cdot\text{К}}\cbr{100\celsius - 50\celsius} + 2{,}3\,\frac{\text{МДж}}{\text{кг}}}
            = 37{,}65\,\text{МДж}
    $
}
\solutionspace{120pt}

\tasknumber{4}%
\task{%
    Воду температурой $t = 60\celsius$ нагрели и превратили в пар при температуре $t_{100} = 100\celsius$,
    потратив $5000\,\text{кДж}$.
    Определите массу воды.
}
\answer{%
    $
        Q
            = cm\Delta t + Lm
            = m\cbr{c(t_{100} - t) + L}
        \implies
        m = \frac{Q}{c(t_{100} - t) + L}
            = \frac { 5000\,\text{кДж} }{4200\,\frac{\text{Дж}}{\text{кг}\cdot\text{К}}\cbr{100\celsius - 60\celsius} + 2{,}3\,\frac{\text{МДж}}{\text{кг}}}
            \approx 2{,}03\,\text{кг}
    $
}
\solutionspace{120pt}

\tasknumber{5}%
\task{%
    Стальное тело температурой $T = 70\celsius$ опустили
    в воду температурой $t = 20\celsius$, масса которой равна массе тела.
    Определите, какая температура установится в сосуде.
}
\answer{%
    \begin{align*}
    Q_1 + Q_2 &= 0,  \\
    Q_1 &= c_1 m_1 \Delta t_1 = c_1 m (\theta - t_1),  \\
    Q_2 &= c_2 m_2 \Delta t_2 = c_2 m (\theta - t_2),  \\
    c_1 m (\theta - t_1) + c_2 m (\theta - t_2) &= 0,  \\
    c_1 (\theta - t_1) + c_2 (\theta - t_2) &= 0,  \\
    c_1 \theta - c_1 t_1 + c_2 \theta - c_2 t_2 &= 0,  \\
    (c_1 + c_2)\theta &= c_1 t_1 + c_2 t_2,  \\
    \theta &= \frac{c_1 t_1 + c_2 t_2}{c_1 + c_2}
            = \frac{4200\,\frac{\text{Дж}}{\text{кг}\cdot\text{К}} \cdot 20\celsius + 500\,\frac{\text{Дж}}{\text{кг}\cdot\text{К}} \cdot  70\celsius}{4200\,\frac{\text{Дж}}{\text{кг}\cdot\text{К}} + 500\,\frac{\text{Дж}}{\text{кг}\cdot\text{К}}}
            \approx 25{,}3 \celsius.
    \end{align*}
}

\variantsplitter

\addpersonalvariant{Илья Стратонников}

\tasknumber{1}%
\task{%
    Сколько льда при температуре $0\celsius$ можно расплавить,
    сообщив ему энергию $3\,\text{МДж}$?
    Здесь (и во всех следующих задачах) используйте табличные значения из учебника.
}
\answer{%
    $
        Q
            = \lambda m \implies m
            = \frac Q{\lambda}
            = \frac { 3\,\text{МДж} }{ 340\,\frac{\text{кДж}}{\text{кг}} }
            \approx 8{,}8\,\text{кг}
    $
}
\solutionspace{120pt}

\tasknumber{2}%
\task{%
    Какое количество теплоты выделится при затвердевании $15\,\text{кг}$ расплавленного стали при температуре плавления?
}
\answer{%
    $
        Q
            = - \lambda m
            = - 84\,\frac{\text{кДж}}{\text{кг}} \cdot 15\,\text{кг}
            = - 1{,}3\,\text{МДж} \implies \abs{Q} = 1{,}3\,\text{МДж}
    $
}
\solutionspace{120pt}

\tasknumber{3}%
\task{%
    Какое количество теплоты необходимо для превращения воды массой $4\,\text{кг}$ при $t = 20\celsius$
    в пар при температуре $t_{100} = 100\celsius$?
}
\answer{%
    $
        Q
            = cm\Delta t + Lm
            = m\cbr{c(t_{100} - t) + L}
            = 4\,\text{кг} \cdot \cbr{4200\,\frac{\text{Дж}}{\text{кг}\cdot\text{К}}\cbr{100\celsius - 20\celsius} + 2{,}3\,\frac{\text{МДж}}{\text{кг}}}
            = 10{,}54\,\text{МДж}
    $
}
\solutionspace{120pt}

\tasknumber{4}%
\task{%
    Воду температурой $t = 10\celsius$ нагрели и превратили в пар при температуре $t_{100} = 100\celsius$,
    потратив $2000\,\text{кДж}$.
    Определите массу воды.
}
\answer{%
    $
        Q
            = cm\Delta t + Lm
            = m\cbr{c(t_{100} - t) + L}
        \implies
        m = \frac{Q}{c(t_{100} - t) + L}
            = \frac { 2000\,\text{кДж} }{4200\,\frac{\text{Дж}}{\text{кг}\cdot\text{К}}\cbr{100\celsius - 10\celsius} + 2{,}3\,\frac{\text{МДж}}{\text{кг}}}
            \approx 0{,}75\,\text{кг}
    $
}
\solutionspace{120pt}

\tasknumber{5}%
\task{%
    Цинковое тело температурой $T = 100\celsius$ опустили
    в воду температурой $t = 30\celsius$, масса которой равна массе тела.
    Определите, какая температура установится в сосуде.
}
\answer{%
    \begin{align*}
    Q_1 + Q_2 &= 0,  \\
    Q_1 &= c_1 m_1 \Delta t_1 = c_1 m (\theta - t_1),  \\
    Q_2 &= c_2 m_2 \Delta t_2 = c_2 m (\theta - t_2),  \\
    c_1 m (\theta - t_1) + c_2 m (\theta - t_2) &= 0,  \\
    c_1 (\theta - t_1) + c_2 (\theta - t_2) &= 0,  \\
    c_1 \theta - c_1 t_1 + c_2 \theta - c_2 t_2 &= 0,  \\
    (c_1 + c_2)\theta &= c_1 t_1 + c_2 t_2,  \\
    \theta &= \frac{c_1 t_1 + c_2 t_2}{c_1 + c_2}
            = \frac{4200\,\frac{\text{Дж}}{\text{кг}\cdot\text{К}} \cdot 30\celsius + 400\,\frac{\text{Дж}}{\text{кг}\cdot\text{К}} \cdot  100\celsius}{4200\,\frac{\text{Дж}}{\text{кг}\cdot\text{К}} + 400\,\frac{\text{Дж}}{\text{кг}\cdot\text{К}}}
            \approx 36{,}1 \celsius.
    \end{align*}
}

\variantsplitter

\addpersonalvariant{Федотова Дарья}

\tasknumber{1}%
\task{%
    Сколько льда при температуре $0\celsius$ можно расплавить,
    сообщив ему энергию $9\,\text{МДж}$?
    Здесь (и во всех следующих задачах) используйте табличные значения из учебника.
}
\answer{%
    $
        Q
            = \lambda m \implies m
            = \frac Q{\lambda}
            = \frac { 9\,\text{МДж} }{ 340\,\frac{\text{кДж}}{\text{кг}} }
            \approx 26{,}5\,\text{кг}
    $
}
\solutionspace{120pt}

\tasknumber{2}%
\task{%
    Какое количество теплоты выделится при затвердевании $25\,\text{кг}$ расплавленного свинца при температуре плавления?
}
\answer{%
    $
        Q
            = - \lambda m
            = - 25\,\frac{\text{кДж}}{\text{кг}} \cdot 25\,\text{кг}
            = - 0{,}6\,\text{МДж} \implies \abs{Q} = 0{,}6\,\text{МДж}
    $
}
\solutionspace{120pt}

\tasknumber{3}%
\task{%
    Какое количество теплоты необходимо для превращения воды массой $2\,\text{кг}$ при $t = 50\celsius$
    в пар при температуре $t_{100} = 100\celsius$?
}
\answer{%
    $
        Q
            = cm\Delta t + Lm
            = m\cbr{c(t_{100} - t) + L}
            = 2\,\text{кг} \cdot \cbr{4200\,\frac{\text{Дж}}{\text{кг}\cdot\text{К}}\cbr{100\celsius - 50\celsius} + 2{,}3\,\frac{\text{МДж}}{\text{кг}}}
            = 5{,}02\,\text{МДж}
    $
}
\solutionspace{120pt}

\tasknumber{4}%
\task{%
    Воду температурой $t = 70\celsius$ нагрели и превратили в пар при температуре $t_{100} = 100\celsius$,
    потратив $2500\,\text{кДж}$.
    Определите массу воды.
}
\answer{%
    $
        Q
            = cm\Delta t + Lm
            = m\cbr{c(t_{100} - t) + L}
        \implies
        m = \frac{Q}{c(t_{100} - t) + L}
            = \frac { 2500\,\text{кДж} }{4200\,\frac{\text{Дж}}{\text{кг}\cdot\text{К}}\cbr{100\celsius - 70\celsius} + 2{,}3\,\frac{\text{МДж}}{\text{кг}}}
            \approx 1{,}03\,\text{кг}
    $
}
\solutionspace{120pt}

\tasknumber{5}%
\task{%
    Алюминиевое тело температурой $T = 90\celsius$ опустили
    в воду температурой $t = 10\celsius$, масса которой равна массе тела.
    Определите, какая температура установится в сосуде.
}
\answer{%
    \begin{align*}
    Q_1 + Q_2 &= 0,  \\
    Q_1 &= c_1 m_1 \Delta t_1 = c_1 m (\theta - t_1),  \\
    Q_2 &= c_2 m_2 \Delta t_2 = c_2 m (\theta - t_2),  \\
    c_1 m (\theta - t_1) + c_2 m (\theta - t_2) &= 0,  \\
    c_1 (\theta - t_1) + c_2 (\theta - t_2) &= 0,  \\
    c_1 \theta - c_1 t_1 + c_2 \theta - c_2 t_2 &= 0,  \\
    (c_1 + c_2)\theta &= c_1 t_1 + c_2 t_2,  \\
    \theta &= \frac{c_1 t_1 + c_2 t_2}{c_1 + c_2}
            = \frac{4200\,\frac{\text{Дж}}{\text{кг}\cdot\text{К}} \cdot 10\celsius + 920\,\frac{\text{Дж}}{\text{кг}\cdot\text{К}} \cdot  90\celsius}{4200\,\frac{\text{Дж}}{\text{кг}\cdot\text{К}} + 920\,\frac{\text{Дж}}{\text{кг}\cdot\text{К}}}
            \approx 24{,}4 \celsius.
    \end{align*}
}

\variantsplitter

\addpersonalvariant{Арсений Храмов}

\tasknumber{1}%
\task{%
    Сколько льда при температуре $0\celsius$ можно расплавить,
    сообщив ему энергию $2\,\text{МДж}$?
    Здесь (и во всех следующих задачах) используйте табличные значения из учебника.
}
\answer{%
    $
        Q
            = \lambda m \implies m
            = \frac Q{\lambda}
            = \frac { 2\,\text{МДж} }{ 340\,\frac{\text{кДж}}{\text{кг}} }
            \approx 5{,}9\,\text{кг}
    $
}
\solutionspace{120pt}

\tasknumber{2}%
\task{%
    Какое количество теплоты выделится при затвердевании $25\,\text{кг}$ расплавленного стали при температуре плавления?
}
\answer{%
    $
        Q
            = - \lambda m
            = - 84\,\frac{\text{кДж}}{\text{кг}} \cdot 25\,\text{кг}
            = - 2{,}1\,\text{МДж} \implies \abs{Q} = 2{,}1\,\text{МДж}
    $
}
\solutionspace{120pt}

\tasknumber{3}%
\task{%
    Какое количество теплоты необходимо для превращения воды массой $15\,\text{кг}$ при $t = 60\celsius$
    в пар при температуре $t_{100} = 100\celsius$?
}
\answer{%
    $
        Q
            = cm\Delta t + Lm
            = m\cbr{c(t_{100} - t) + L}
            = 15\,\text{кг} \cdot \cbr{4200\,\frac{\text{Дж}}{\text{кг}\cdot\text{К}}\cbr{100\celsius - 60\celsius} + 2{,}3\,\frac{\text{МДж}}{\text{кг}}}
            = 37{,}02\,\text{МДж}
    $
}
\solutionspace{120pt}

\tasknumber{4}%
\task{%
    Воду температурой $t = 50\celsius$ нагрели и превратили в пар при температуре $t_{100} = 100\celsius$,
    потратив $4000\,\text{кДж}$.
    Определите массу воды.
}
\answer{%
    $
        Q
            = cm\Delta t + Lm
            = m\cbr{c(t_{100} - t) + L}
        \implies
        m = \frac{Q}{c(t_{100} - t) + L}
            = \frac { 4000\,\text{кДж} }{4200\,\frac{\text{Дж}}{\text{кг}\cdot\text{К}}\cbr{100\celsius - 50\celsius} + 2{,}3\,\frac{\text{МДж}}{\text{кг}}}
            \approx 1{,}59\,\text{кг}
    $
}
\solutionspace{120pt}

\tasknumber{5}%
\task{%
    Стальное тело температурой $T = 100\celsius$ опустили
    в воду температурой $t = 10\celsius$, масса которой равна массе тела.
    Определите, какая температура установится в сосуде.
}
\answer{%
    \begin{align*}
    Q_1 + Q_2 &= 0,  \\
    Q_1 &= c_1 m_1 \Delta t_1 = c_1 m (\theta - t_1),  \\
    Q_2 &= c_2 m_2 \Delta t_2 = c_2 m (\theta - t_2),  \\
    c_1 m (\theta - t_1) + c_2 m (\theta - t_2) &= 0,  \\
    c_1 (\theta - t_1) + c_2 (\theta - t_2) &= 0,  \\
    c_1 \theta - c_1 t_1 + c_2 \theta - c_2 t_2 &= 0,  \\
    (c_1 + c_2)\theta &= c_1 t_1 + c_2 t_2,  \\
    \theta &= \frac{c_1 t_1 + c_2 t_2}{c_1 + c_2}
            = \frac{4200\,\frac{\text{Дж}}{\text{кг}\cdot\text{К}} \cdot 10\celsius + 500\,\frac{\text{Дж}}{\text{кг}\cdot\text{К}} \cdot  100\celsius}{4200\,\frac{\text{Дж}}{\text{кг}\cdot\text{К}} + 500\,\frac{\text{Дж}}{\text{кг}\cdot\text{К}}}
            \approx 19{,}6 \celsius.
    \end{align*}
}

\variantsplitter

\addpersonalvariant{Иван Шустов}

\tasknumber{1}%
\task{%
    Сколько льда при температуре $0\celsius$ можно расплавить,
    сообщив ему энергию $9\,\text{МДж}$?
    Здесь (и во всех следующих задачах) используйте табличные значения из учебника.
}
\answer{%
    $
        Q
            = \lambda m \implies m
            = \frac Q{\lambda}
            = \frac { 9\,\text{МДж} }{ 340\,\frac{\text{кДж}}{\text{кг}} }
            \approx 26{,}5\,\text{кг}
    $
}
\solutionspace{120pt}

\tasknumber{2}%
\task{%
    Какое количество теплоты выделится при затвердевании $15\,\text{кг}$ расплавленного стали при температуре плавления?
}
\answer{%
    $
        Q
            = - \lambda m
            = - 84\,\frac{\text{кДж}}{\text{кг}} \cdot 15\,\text{кг}
            = - 1{,}3\,\text{МДж} \implies \abs{Q} = 1{,}3\,\text{МДж}
    $
}
\solutionspace{120pt}

\tasknumber{3}%
\task{%
    Какое количество теплоты необходимо для превращения воды массой $15\,\text{кг}$ при $t = 40\celsius$
    в пар при температуре $t_{100} = 100\celsius$?
}
\answer{%
    $
        Q
            = cm\Delta t + Lm
            = m\cbr{c(t_{100} - t) + L}
            = 15\,\text{кг} \cdot \cbr{4200\,\frac{\text{Дж}}{\text{кг}\cdot\text{К}}\cbr{100\celsius - 40\celsius} + 2{,}3\,\frac{\text{МДж}}{\text{кг}}}
            = 38{,}28\,\text{МДж}
    $
}
\solutionspace{120pt}

\tasknumber{4}%
\task{%
    Воду температурой $t = 70\celsius$ нагрели и превратили в пар при температуре $t_{100} = 100\celsius$,
    потратив $5000\,\text{кДж}$.
    Определите массу воды.
}
\answer{%
    $
        Q
            = cm\Delta t + Lm
            = m\cbr{c(t_{100} - t) + L}
        \implies
        m = \frac{Q}{c(t_{100} - t) + L}
            = \frac { 5000\,\text{кДж} }{4200\,\frac{\text{Дж}}{\text{кг}\cdot\text{К}}\cbr{100\celsius - 70\celsius} + 2{,}3\,\frac{\text{МДж}}{\text{кг}}}
            \approx 2{,}06\,\text{кг}
    $
}
\solutionspace{120pt}

\tasknumber{5}%
\task{%
    Стальное тело температурой $T = 70\celsius$ опустили
    в воду температурой $t = 30\celsius$, масса которой равна массе тела.
    Определите, какая температура установится в сосуде.
}
\answer{%
    \begin{align*}
    Q_1 + Q_2 &= 0,  \\
    Q_1 &= c_1 m_1 \Delta t_1 = c_1 m (\theta - t_1),  \\
    Q_2 &= c_2 m_2 \Delta t_2 = c_2 m (\theta - t_2),  \\
    c_1 m (\theta - t_1) + c_2 m (\theta - t_2) &= 0,  \\
    c_1 (\theta - t_1) + c_2 (\theta - t_2) &= 0,  \\
    c_1 \theta - c_1 t_1 + c_2 \theta - c_2 t_2 &= 0,  \\
    (c_1 + c_2)\theta &= c_1 t_1 + c_2 t_2,  \\
    \theta &= \frac{c_1 t_1 + c_2 t_2}{c_1 + c_2}
            = \frac{4200\,\frac{\text{Дж}}{\text{кг}\cdot\text{К}} \cdot 30\celsius + 500\,\frac{\text{Дж}}{\text{кг}\cdot\text{К}} \cdot  70\celsius}{4200\,\frac{\text{Дж}}{\text{кг}\cdot\text{К}} + 500\,\frac{\text{Дж}}{\text{кг}\cdot\text{К}}}
            \approx 34{,}3 \celsius.
    \end{align*}
}
% autogenerated
