\setdate{26~марта~2021}
\setclass{10«АБ»}

\addpersonalvariant{Михаил Бурмистров}

\tasknumber{1}%
\task{%
    Сколько молекул водяного пара содержится в сосуде объёмом $3\,\text{л}$ при температуре $90\celsius$,
    и влажности воздуха $35\%$?
}
\solutionspace{160pt}

\tasknumber{2}%
\task{%
    В герметичном сосуде находится влажный воздух при температуре $25\celsius$ и относительной влажности $25\%$.
    \begin{enumerate}
        \item Чему равно парциальное давление насыщенного водяного пара при этой температуре?
        \item Чему равно парциальное давление водяного пара?
        \item Определите точку росы этого пара?
        \item Каким станет парциальное давление водяного пара, если сосуд нагреть до  $80\celsius$?
        \item Чему будет равна относительная влажность воздуха, если сосуд нагреть до $80\celsius$?
        \item Получите ответ на предыдущий вопрос, используя плотности, а не давления.
    \end{enumerate}
}
\solutionspace{200pt}

\tasknumber{3}%
\task{%
    Закрытый сосуд объёмом $15\,\text{л}$ заполнен сухим воздухом при давлении $100\,\text{кПа}$ и температуре $30\celsius$.
    Каким станет давление в сосуде, если в него налить $10\,\text{г}$ воды и нагреть содержимое сосуда до $90\celsius$?
}
\answer{%
    Конечное давление газа в сосуде складывается по закону Дальтона из давления нагретого сухого воздуха $P'_\text{ воздуха }$ и
    давления насыщенного пара $P_\text{ пара }$:
    $$P' = P'_\text{ воздуха } + P_\text{ пара }.$$

    Сперва определим новое давление сухого воздуха из уравнения состояния идеального газа:
    $$\frac{P'_\text{ воздуха } \cdot V}{T'} = \nu R = \frac{P \cdot V}{T} \implies P'_\text{ воздуха } = P \cdot \frac{T'}{T}.$$

    Чтобы найти давление пара, нужно понять, будет ли он насыщенным после нагрева или нет.

    Плотность насыщенного пара при температуре равна $424\,\frac{\text{г}}{\text{м}^{3}}$, тогда для того,
    чтобы весь сосуд был заполнен насыщенным водяным паром нужно
    $m_\text{ н.
    п.
    } = \rho_\text{ н.
    п.
    90 $\celsius$ } \cdot V = 424\,\frac{\text{г}}{\text{м}^{3}} \cdot 15\,\text{л} = 6{,}4\,\text{г}$ воды.
    Сравнивая эту массу с массой воды из условия, получаем массу жидкости, которая испарится: $6{,}4\,\text{г}$.
    Осталось определить давление этого пара:
    $$P_\text{ пара } = \frac{m_\text{ пара }RT}{\mu V} = \frac{6{,}4\,\text{г} \cdot 8{,}31\,\frac{\text{Дж}}{\text{моль}\cdot\text{К}} \cdot 363\,\text{К}}{18\,\frac{\text{г}}{\text{моль}} \cdot 15\,\text{л}} \approx 71\,\text{кПа}.$$

    Получаем ответ: $P'_\text{ пара } = 190{,}9\,\text{кПа}$.

    Другой вариант решения для давления пара:
    Определим давление пара, если бы вся вода испарилась (что не факт):
    $$P_\text{ max } = \frac{mRT'}{\mu V} = \frac{10\,\text{г} \cdot 8{,}31\,\frac{\text{Дж}}{\text{моль}\cdot\text{К}} \cdot 363\,\text{К}}{18\,\frac{\text{г}}{\text{моль}} \cdot 15\,\text{л}} = 111\,\text{кПа}.$$
    Сравниваем это давление с давлением насыщенного пара при этой температуре $P_\text{ н.
    п.
    90 $\celsius$ } = 70\,\text{кПа}$:
     если у нас получилось меньше табличного значения,
    то вся вода испарилась, если же больше — испарилась лишь часть, а пар является насыщенным.
    Отсюда сразу получаем давление пара: $P'_\text{ пара } = 70{,}1$.
    Сравните этот результат с первым вариантом решения.

    Тут получаем ответ: $P'_\text{ пара } = 189{,}9\,\text{кПа}$.
}
\solutionspace{150pt}

\tasknumber{4}%
\task{%
    Напротив физических величин запишите определение, обозначение и единицы измерения в системе СИ (если есть):
    \begin{enumerate}
        \item относительная влажность,
        \item динамическое равновесие.
    \end{enumerate}
}

\variantsplitter

\addpersonalvariant{Ирина Ан}

\tasknumber{1}%
\task{%
    Сколько молекул водяного пара содержится в сосуде объёмом $3\,\text{л}$ при температуре $60\celsius$,
    и влажности воздуха $55\%$?
}
\solutionspace{160pt}

\tasknumber{2}%
\task{%
    В герметичном сосуде находится влажный воздух при температуре $20\celsius$ и относительной влажности $40\%$.
    \begin{enumerate}
        \item Чему равно парциальное давление насыщенного водяного пара при этой температуре?
        \item Чему равно парциальное давление водяного пара?
        \item Определите точку росы этого пара?
        \item Каким станет парциальное давление водяного пара, если сосуд нагреть до  $70\celsius$?
        \item Чему будет равна относительная влажность воздуха, если сосуд нагреть до $70\celsius$?
        \item Получите ответ на предыдущий вопрос, используя плотности, а не давления.
    \end{enumerate}
}
\solutionspace{200pt}

\tasknumber{3}%
\task{%
    Закрытый сосуд объёмом $10\,\text{л}$ заполнен сухим воздухом при давлении $100\,\text{кПа}$ и температуре $30\celsius$.
    Каким станет давление в сосуде, если в него налить $20\,\text{г}$ воды и нагреть содержимое сосуда до $80\celsius$?
}
\answer{%
    Конечное давление газа в сосуде складывается по закону Дальтона из давления нагретого сухого воздуха $P'_\text{ воздуха }$ и
    давления насыщенного пара $P_\text{ пара }$:
    $$P' = P'_\text{ воздуха } + P_\text{ пара }.$$

    Сперва определим новое давление сухого воздуха из уравнения состояния идеального газа:
    $$\frac{P'_\text{ воздуха } \cdot V}{T'} = \nu R = \frac{P \cdot V}{T} \implies P'_\text{ воздуха } = P \cdot \frac{T'}{T}.$$

    Чтобы найти давление пара, нужно понять, будет ли он насыщенным после нагрева или нет.

    Плотность насыщенного пара при температуре равна $293\,\frac{\text{г}}{\text{м}^{3}}$, тогда для того,
    чтобы весь сосуд был заполнен насыщенным водяным паром нужно
    $m_\text{ н.
    п.
    } = \rho_\text{ н.
    п.
    80 $\celsius$ } \cdot V = 293\,\frac{\text{г}}{\text{м}^{3}} \cdot 10\,\text{л} = 2{,}9\,\text{г}$ воды.
    Сравнивая эту массу с массой воды из условия, получаем массу жидкости, которая испарится: $2{,}9\,\text{г}$.
    Осталось определить давление этого пара:
    $$P_\text{ пара } = \frac{m_\text{ пара }RT}{\mu V} = \frac{2{,}9\,\text{г} \cdot 8{,}31\,\frac{\text{Дж}}{\text{моль}\cdot\text{К}} \cdot 353\,\text{К}}{18\,\frac{\text{г}}{\text{моль}} \cdot 10\,\text{л}} \approx 47\,\text{кПа}.$$

    Получаем ответ: $P'_\text{ пара } = 164{,}3\,\text{кПа}$.

    Другой вариант решения для давления пара:
    Определим давление пара, если бы вся вода испарилась (что не факт):
    $$P_\text{ max } = \frac{mRT'}{\mu V} = \frac{20\,\text{г} \cdot 8{,}31\,\frac{\text{Дж}}{\text{моль}\cdot\text{К}} \cdot 353\,\text{К}}{18\,\frac{\text{г}}{\text{моль}} \cdot 10\,\text{л}} = 325\,\text{кПа}.$$
    Сравниваем это давление с давлением насыщенного пара при этой температуре $P_\text{ н.
    п.
    80 $\celsius$ } = 47\,\text{кПа}$:
     если у нас получилось меньше табличного значения,
    то вся вода испарилась, если же больше — испарилась лишь часть, а пар является насыщенным.
    Отсюда сразу получаем давление пара: $P'_\text{ пара } = 47{,}3$.
    Сравните этот результат с первым вариантом решения.

    Тут получаем ответ: $P'_\text{ пара } = 163{,}8\,\text{кПа}$.
}
\solutionspace{150pt}

\tasknumber{4}%
\task{%
    Напротив физических величин запишите определение, обозначение и единицы измерения в системе СИ (если есть):
    \begin{enumerate}
        \item относительная влажность,
        \item насыщенный пар.
    \end{enumerate}
}

\variantsplitter

\addpersonalvariant{Софья Андрианова}

\tasknumber{1}%
\task{%
    Сколько молекул водяного пара содержится в сосуде объёмом $6\,\text{л}$ при температуре $50\celsius$,
    и влажности воздуха $30\%$?
}
\solutionspace{160pt}

\tasknumber{2}%
\task{%
    В герметичном сосуде находится влажный воздух при температуре $30\celsius$ и относительной влажности $60\%$.
    \begin{enumerate}
        \item Чему равно парциальное давление насыщенного водяного пара при этой температуре?
        \item Чему равно парциальное давление водяного пара?
        \item Определите точку росы этого пара?
        \item Каким станет парциальное давление водяного пара, если сосуд нагреть до  $80\celsius$?
        \item Чему будет равна относительная влажность воздуха, если сосуд нагреть до $80\celsius$?
        \item Получите ответ на предыдущий вопрос, используя плотности, а не давления.
    \end{enumerate}
}
\solutionspace{200pt}

\tasknumber{3}%
\task{%
    Закрытый сосуд объёмом $10\,\text{л}$ заполнен сухим воздухом при давлении $100\,\text{кПа}$ и температуре $10\celsius$.
    Каким станет давление в сосуде, если в него налить $5\,\text{г}$ воды и нагреть содержимое сосуда до $90\celsius$?
}
\answer{%
    Конечное давление газа в сосуде складывается по закону Дальтона из давления нагретого сухого воздуха $P'_\text{ воздуха }$ и
    давления насыщенного пара $P_\text{ пара }$:
    $$P' = P'_\text{ воздуха } + P_\text{ пара }.$$

    Сперва определим новое давление сухого воздуха из уравнения состояния идеального газа:
    $$\frac{P'_\text{ воздуха } \cdot V}{T'} = \nu R = \frac{P \cdot V}{T} \implies P'_\text{ воздуха } = P \cdot \frac{T'}{T}.$$

    Чтобы найти давление пара, нужно понять, будет ли он насыщенным после нагрева или нет.

    Плотность насыщенного пара при температуре равна $424\,\frac{\text{г}}{\text{м}^{3}}$, тогда для того,
    чтобы весь сосуд был заполнен насыщенным водяным паром нужно
    $m_\text{ н.
    п.
    } = \rho_\text{ н.
    п.
    90 $\celsius$ } \cdot V = 424\,\frac{\text{г}}{\text{м}^{3}} \cdot 10\,\text{л} = 4{,}2\,\text{г}$ воды.
    Сравнивая эту массу с массой воды из условия, получаем массу жидкости, которая испарится: $4{,}2\,\text{г}$.
    Осталось определить давление этого пара:
    $$P_\text{ пара } = \frac{m_\text{ пара }RT}{\mu V} = \frac{4{,}2\,\text{г} \cdot 8{,}31\,\frac{\text{Дж}}{\text{моль}\cdot\text{К}} \cdot 363\,\text{К}}{18\,\frac{\text{г}}{\text{моль}} \cdot 10\,\text{л}} \approx 71\,\text{кПа}.$$

    Получаем ответ: $P'_\text{ пара } = 199{,}3\,\text{кПа}$.

    Другой вариант решения для давления пара:
    Определим давление пара, если бы вся вода испарилась (что не факт):
    $$P_\text{ max } = \frac{mRT'}{\mu V} = \frac{5\,\text{г} \cdot 8{,}31\,\frac{\text{Дж}}{\text{моль}\cdot\text{К}} \cdot 363\,\text{К}}{18\,\frac{\text{г}}{\text{моль}} \cdot 10\,\text{л}} = 83\,\text{кПа}.$$
    Сравниваем это давление с давлением насыщенного пара при этой температуре $P_\text{ н.
    п.
    90 $\celsius$ } = 70\,\text{кПа}$:
     если у нас получилось меньше табличного значения,
    то вся вода испарилась, если же больше — испарилась лишь часть, а пар является насыщенным.
    Отсюда сразу получаем давление пара: $P'_\text{ пара } = 70{,}1$.
    Сравните этот результат с первым вариантом решения.

    Тут получаем ответ: $P'_\text{ пара } = 198{,}4\,\text{кПа}$.
}
\solutionspace{150pt}

\tasknumber{4}%
\task{%
    Напротив физических величин запишите определение, обозначение и единицы измерения в системе СИ (если есть):
    \begin{enumerate}
        \item относительная влажность,
        \item насыщенный пар.
    \end{enumerate}
}

\variantsplitter

\addpersonalvariant{Владимир Артемчук}

\tasknumber{1}%
\task{%
    Сколько молекул водяного пара содержится в сосуде объёмом $15\,\text{л}$ при температуре $60\celsius$,
    и влажности воздуха $80\%$?
}
\solutionspace{160pt}

\tasknumber{2}%
\task{%
    В герметичном сосуде находится влажный воздух при температуре $25\celsius$ и относительной влажности $40\%$.
    \begin{enumerate}
        \item Чему равно парциальное давление насыщенного водяного пара при этой температуре?
        \item Чему равно парциальное давление водяного пара?
        \item Определите точку росы этого пара?
        \item Каким станет парциальное давление водяного пара, если сосуд нагреть до  $90\celsius$?
        \item Чему будет равна относительная влажность воздуха, если сосуд нагреть до $90\celsius$?
        \item Получите ответ на предыдущий вопрос, используя плотности, а не давления.
    \end{enumerate}
}
\solutionspace{200pt}

\tasknumber{3}%
\task{%
    Закрытый сосуд объёмом $15\,\text{л}$ заполнен сухим воздухом при давлении $100\,\text{кПа}$ и температуре $20\celsius$.
    Каким станет давление в сосуде, если в него налить $30\,\text{г}$ воды и нагреть содержимое сосуда до $80\celsius$?
}
\answer{%
    Конечное давление газа в сосуде складывается по закону Дальтона из давления нагретого сухого воздуха $P'_\text{ воздуха }$ и
    давления насыщенного пара $P_\text{ пара }$:
    $$P' = P'_\text{ воздуха } + P_\text{ пара }.$$

    Сперва определим новое давление сухого воздуха из уравнения состояния идеального газа:
    $$\frac{P'_\text{ воздуха } \cdot V}{T'} = \nu R = \frac{P \cdot V}{T} \implies P'_\text{ воздуха } = P \cdot \frac{T'}{T}.$$

    Чтобы найти давление пара, нужно понять, будет ли он насыщенным после нагрева или нет.

    Плотность насыщенного пара при температуре равна $293\,\frac{\text{г}}{\text{м}^{3}}$, тогда для того,
    чтобы весь сосуд был заполнен насыщенным водяным паром нужно
    $m_\text{ н.
    п.
    } = \rho_\text{ н.
    п.
    80 $\celsius$ } \cdot V = 293\,\frac{\text{г}}{\text{м}^{3}} \cdot 15\,\text{л} = 4{,}4\,\text{г}$ воды.
    Сравнивая эту массу с массой воды из условия, получаем массу жидкости, которая испарится: $4{,}4\,\text{г}$.
    Осталось определить давление этого пара:
    $$P_\text{ пара } = \frac{m_\text{ пара }RT}{\mu V} = \frac{4{,}4\,\text{г} \cdot 8{,}31\,\frac{\text{Дж}}{\text{моль}\cdot\text{К}} \cdot 353\,\text{К}}{18\,\frac{\text{г}}{\text{моль}} \cdot 15\,\text{л}} \approx 47\,\text{кПа}.$$

    Получаем ответ: $P'_\text{ пара } = 168{,}2\,\text{кПа}$.

    Другой вариант решения для давления пара:
    Определим давление пара, если бы вся вода испарилась (что не факт):
    $$P_\text{ max } = \frac{mRT'}{\mu V} = \frac{30\,\text{г} \cdot 8{,}31\,\frac{\text{Дж}}{\text{моль}\cdot\text{К}} \cdot 353\,\text{К}}{18\,\frac{\text{г}}{\text{моль}} \cdot 15\,\text{л}} = 325\,\text{кПа}.$$
    Сравниваем это давление с давлением насыщенного пара при этой температуре $P_\text{ н.
    п.
    80 $\celsius$ } = 47\,\text{кПа}$:
     если у нас получилось меньше табличного значения,
    то вся вода испарилась, если же больше — испарилась лишь часть, а пар является насыщенным.
    Отсюда сразу получаем давление пара: $P'_\text{ пара } = 47{,}3$.
    Сравните этот результат с первым вариантом решения.

    Тут получаем ответ: $P'_\text{ пара } = 167{,}8\,\text{кПа}$.
}
\solutionspace{150pt}

\tasknumber{4}%
\task{%
    Напротив физических величин запишите определение, обозначение и единицы измерения в системе СИ (если есть):
    \begin{enumerate}
        \item абсолютная влажность,
        \item насыщенный пар.
    \end{enumerate}
}

\variantsplitter

\addpersonalvariant{Софья Белянкина}

\tasknumber{1}%
\task{%
    Сколько молекул водяного пара содержится в сосуде объёмом $3\,\text{л}$ при температуре $30\celsius$,
    и влажности воздуха $20\%$?
}
\solutionspace{160pt}

\tasknumber{2}%
\task{%
    В герметичном сосуде находится влажный воздух при температуре $20\celsius$ и относительной влажности $80\%$.
    \begin{enumerate}
        \item Чему равно парциальное давление насыщенного водяного пара при этой температуре?
        \item Чему равно парциальное давление водяного пара?
        \item Определите точку росы этого пара?
        \item Каким станет парциальное давление водяного пара, если сосуд нагреть до  $80\celsius$?
        \item Чему будет равна относительная влажность воздуха, если сосуд нагреть до $80\celsius$?
        \item Получите ответ на предыдущий вопрос, используя плотности, а не давления.
    \end{enumerate}
}
\solutionspace{200pt}

\tasknumber{3}%
\task{%
    Закрытый сосуд объёмом $10\,\text{л}$ заполнен сухим воздухом при давлении $100\,\text{кПа}$ и температуре $10\celsius$.
    Каким станет давление в сосуде, если в него налить $10\,\text{г}$ воды и нагреть содержимое сосуда до $90\celsius$?
}
\answer{%
    Конечное давление газа в сосуде складывается по закону Дальтона из давления нагретого сухого воздуха $P'_\text{ воздуха }$ и
    давления насыщенного пара $P_\text{ пара }$:
    $$P' = P'_\text{ воздуха } + P_\text{ пара }.$$

    Сперва определим новое давление сухого воздуха из уравнения состояния идеального газа:
    $$\frac{P'_\text{ воздуха } \cdot V}{T'} = \nu R = \frac{P \cdot V}{T} \implies P'_\text{ воздуха } = P \cdot \frac{T'}{T}.$$

    Чтобы найти давление пара, нужно понять, будет ли он насыщенным после нагрева или нет.

    Плотность насыщенного пара при температуре равна $424\,\frac{\text{г}}{\text{м}^{3}}$, тогда для того,
    чтобы весь сосуд был заполнен насыщенным водяным паром нужно
    $m_\text{ н.
    п.
    } = \rho_\text{ н.
    п.
    90 $\celsius$ } \cdot V = 424\,\frac{\text{г}}{\text{м}^{3}} \cdot 10\,\text{л} = 4{,}2\,\text{г}$ воды.
    Сравнивая эту массу с массой воды из условия, получаем массу жидкости, которая испарится: $4{,}2\,\text{г}$.
    Осталось определить давление этого пара:
    $$P_\text{ пара } = \frac{m_\text{ пара }RT}{\mu V} = \frac{4{,}2\,\text{г} \cdot 8{,}31\,\frac{\text{Дж}}{\text{моль}\cdot\text{К}} \cdot 363\,\text{К}}{18\,\frac{\text{г}}{\text{моль}} \cdot 10\,\text{л}} \approx 71\,\text{кПа}.$$

    Получаем ответ: $P'_\text{ пара } = 199{,}3\,\text{кПа}$.

    Другой вариант решения для давления пара:
    Определим давление пара, если бы вся вода испарилась (что не факт):
    $$P_\text{ max } = \frac{mRT'}{\mu V} = \frac{10\,\text{г} \cdot 8{,}31\,\frac{\text{Дж}}{\text{моль}\cdot\text{К}} \cdot 363\,\text{К}}{18\,\frac{\text{г}}{\text{моль}} \cdot 10\,\text{л}} = 167\,\text{кПа}.$$
    Сравниваем это давление с давлением насыщенного пара при этой температуре $P_\text{ н.
    п.
    90 $\celsius$ } = 70\,\text{кПа}$:
     если у нас получилось меньше табличного значения,
    то вся вода испарилась, если же больше — испарилась лишь часть, а пар является насыщенным.
    Отсюда сразу получаем давление пара: $P'_\text{ пара } = 70{,}1$.
    Сравните этот результат с первым вариантом решения.

    Тут получаем ответ: $P'_\text{ пара } = 198{,}4\,\text{кПа}$.
}
\solutionspace{150pt}

\tasknumber{4}%
\task{%
    Напротив физических величин запишите определение, обозначение и единицы измерения в системе СИ (если есть):
    \begin{enumerate}
        \item абсолютная влажность,
        \item динамическое равновесие.
    \end{enumerate}
}

\variantsplitter

\addpersonalvariant{Варвара Егиазарян}

\tasknumber{1}%
\task{%
    Сколько молекул водяного пара содержится в сосуде объёмом $12\,\text{л}$ при температуре $100\celsius$,
    и влажности воздуха $60\%$?
}
\solutionspace{160pt}

\tasknumber{2}%
\task{%
    В герметичном сосуде находится влажный воздух при температуре $30\celsius$ и относительной влажности $30\%$.
    \begin{enumerate}
        \item Чему равно парциальное давление насыщенного водяного пара при этой температуре?
        \item Чему равно парциальное давление водяного пара?
        \item Определите точку росы этого пара?
        \item Каким станет парциальное давление водяного пара, если сосуд нагреть до  $90\celsius$?
        \item Чему будет равна относительная влажность воздуха, если сосуд нагреть до $90\celsius$?
        \item Получите ответ на предыдущий вопрос, используя плотности, а не давления.
    \end{enumerate}
}
\solutionspace{200pt}

\tasknumber{3}%
\task{%
    Закрытый сосуд объёмом $20\,\text{л}$ заполнен сухим воздухом при давлении $100\,\text{кПа}$ и температуре $20\celsius$.
    Каким станет давление в сосуде, если в него налить $20\,\text{г}$ воды и нагреть содержимое сосуда до $90\celsius$?
}
\answer{%
    Конечное давление газа в сосуде складывается по закону Дальтона из давления нагретого сухого воздуха $P'_\text{ воздуха }$ и
    давления насыщенного пара $P_\text{ пара }$:
    $$P' = P'_\text{ воздуха } + P_\text{ пара }.$$

    Сперва определим новое давление сухого воздуха из уравнения состояния идеального газа:
    $$\frac{P'_\text{ воздуха } \cdot V}{T'} = \nu R = \frac{P \cdot V}{T} \implies P'_\text{ воздуха } = P \cdot \frac{T'}{T}.$$

    Чтобы найти давление пара, нужно понять, будет ли он насыщенным после нагрева или нет.

    Плотность насыщенного пара при температуре равна $424\,\frac{\text{г}}{\text{м}^{3}}$, тогда для того,
    чтобы весь сосуд был заполнен насыщенным водяным паром нужно
    $m_\text{ н.
    п.
    } = \rho_\text{ н.
    п.
    90 $\celsius$ } \cdot V = 424\,\frac{\text{г}}{\text{м}^{3}} \cdot 20\,\text{л} = 8{,}5\,\text{г}$ воды.
    Сравнивая эту массу с массой воды из условия, получаем массу жидкости, которая испарится: $8{,}5\,\text{г}$.
    Осталось определить давление этого пара:
    $$P_\text{ пара } = \frac{m_\text{ пара }RT}{\mu V} = \frac{8{,}5\,\text{г} \cdot 8{,}31\,\frac{\text{Дж}}{\text{моль}\cdot\text{К}} \cdot 363\,\text{К}}{18\,\frac{\text{г}}{\text{моль}} \cdot 20\,\text{л}} \approx 71\,\text{кПа}.$$

    Получаем ответ: $P'_\text{ пара } = 194{,}9\,\text{кПа}$.

    Другой вариант решения для давления пара:
    Определим давление пара, если бы вся вода испарилась (что не факт):
    $$P_\text{ max } = \frac{mRT'}{\mu V} = \frac{20\,\text{г} \cdot 8{,}31\,\frac{\text{Дж}}{\text{моль}\cdot\text{К}} \cdot 363\,\text{К}}{18\,\frac{\text{г}}{\text{моль}} \cdot 20\,\text{л}} = 167\,\text{кПа}.$$
    Сравниваем это давление с давлением насыщенного пара при этой температуре $P_\text{ н.
    п.
    90 $\celsius$ } = 70\,\text{кПа}$:
     если у нас получилось меньше табличного значения,
    то вся вода испарилась, если же больше — испарилась лишь часть, а пар является насыщенным.
    Отсюда сразу получаем давление пара: $P'_\text{ пара } = 70{,}1$.
    Сравните этот результат с первым вариантом решения.

    Тут получаем ответ: $P'_\text{ пара } = 194{,}0\,\text{кПа}$.
}
\solutionspace{150pt}

\tasknumber{4}%
\task{%
    Напротив физических величин запишите определение, обозначение и единицы измерения в системе СИ (если есть):
    \begin{enumerate}
        \item относительная влажность,
        \item насыщенный пар.
    \end{enumerate}
}

\variantsplitter

\addpersonalvariant{Владислав Емелин}

\tasknumber{1}%
\task{%
    Сколько молекул водяного пара содержится в сосуде объёмом $12\,\text{л}$ при температуре $100\celsius$,
    и влажности воздуха $25\%$?
}
\solutionspace{160pt}

\tasknumber{2}%
\task{%
    В герметичном сосуде находится влажный воздух при температуре $15\celsius$ и относительной влажности $65\%$.
    \begin{enumerate}
        \item Чему равно парциальное давление насыщенного водяного пара при этой температуре?
        \item Чему равно парциальное давление водяного пара?
        \item Определите точку росы этого пара?
        \item Каким станет парциальное давление водяного пара, если сосуд нагреть до  $80\celsius$?
        \item Чему будет равна относительная влажность воздуха, если сосуд нагреть до $80\celsius$?
        \item Получите ответ на предыдущий вопрос, используя плотности, а не давления.
    \end{enumerate}
}
\solutionspace{200pt}

\tasknumber{3}%
\task{%
    Закрытый сосуд объёмом $15\,\text{л}$ заполнен сухим воздухом при давлении $100\,\text{кПа}$ и температуре $30\celsius$.
    Каким станет давление в сосуде, если в него налить $20\,\text{г}$ воды и нагреть содержимое сосуда до $100\celsius$?
}
\answer{%
    Конечное давление газа в сосуде складывается по закону Дальтона из давления нагретого сухого воздуха $P'_\text{ воздуха }$ и
    давления насыщенного пара $P_\text{ пара }$:
    $$P' = P'_\text{ воздуха } + P_\text{ пара }.$$

    Сперва определим новое давление сухого воздуха из уравнения состояния идеального газа:
    $$\frac{P'_\text{ воздуха } \cdot V}{T'} = \nu R = \frac{P \cdot V}{T} \implies P'_\text{ воздуха } = P \cdot \frac{T'}{T}.$$

    Чтобы найти давление пара, нужно понять, будет ли он насыщенным после нагрева или нет.

    Плотность насыщенного пара при температуре равна $598\,\frac{\text{г}}{\text{м}^{3}}$, тогда для того,
    чтобы весь сосуд был заполнен насыщенным водяным паром нужно
    $m_\text{ н.
    п.
    } = \rho_\text{ н.
    п.
    100 $\celsius$ } \cdot V = 598\,\frac{\text{г}}{\text{м}^{3}} \cdot 15\,\text{л} = 9{,}0\,\text{г}$ воды.
    Сравнивая эту массу с массой воды из условия, получаем массу жидкости, которая испарится: $9{,}0\,\text{г}$.
    Осталось определить давление этого пара:
    $$P_\text{ пара } = \frac{m_\text{ пара }RT}{\mu V} = \frac{9{,}0\,\text{г} \cdot 8{,}31\,\frac{\text{Дж}}{\text{моль}\cdot\text{К}} \cdot 373\,\text{К}}{18\,\frac{\text{г}}{\text{моль}} \cdot 15\,\text{л}} \approx 102\,\text{кПа}.$$

    Получаем ответ: $P'_\text{ пара } = 226{,}1\,\text{кПа}$.

    Другой вариант решения для давления пара:
    Определим давление пара, если бы вся вода испарилась (что не факт):
    $$P_\text{ max } = \frac{mRT'}{\mu V} = \frac{20\,\text{г} \cdot 8{,}31\,\frac{\text{Дж}}{\text{моль}\cdot\text{К}} \cdot 373\,\text{К}}{18\,\frac{\text{г}}{\text{моль}} \cdot 15\,\text{л}} = 229\,\text{кПа}.$$
    Сравниваем это давление с давлением насыщенного пара при этой температуре $P_\text{ н.
    п.
    100 $\celsius$ } = 101\,\text{кПа}$:
     если у нас получилось меньше табличного значения,
    то вся вода испарилась, если же больше — испарилась лишь часть, а пар является насыщенным.
    Отсюда сразу получаем давление пара: $P'_\text{ пара } = 101{,}3$.
    Сравните этот результат с первым вариантом решения.

    Тут получаем ответ: $P'_\text{ пара } = 224{,}4\,\text{кПа}$.
}
\solutionspace{150pt}

\tasknumber{4}%
\task{%
    Напротив физических величин запишите определение, обозначение и единицы измерения в системе СИ (если есть):
    \begin{enumerate}
        \item относительная влажность,
        \item динамическое равновесие.
    \end{enumerate}
}

\variantsplitter

\addpersonalvariant{Артём Жичин}

\tasknumber{1}%
\task{%
    Сколько молекул водяного пара содержится в сосуде объёмом $15\,\text{л}$ при температуре $100\celsius$,
    и влажности воздуха $55\%$?
}
\solutionspace{160pt}

\tasknumber{2}%
\task{%
    В герметичном сосуде находится влажный воздух при температуре $20\celsius$ и относительной влажности $40\%$.
    \begin{enumerate}
        \item Чему равно парциальное давление насыщенного водяного пара при этой температуре?
        \item Чему равно парциальное давление водяного пара?
        \item Определите точку росы этого пара?
        \item Каким станет парциальное давление водяного пара, если сосуд нагреть до  $90\celsius$?
        \item Чему будет равна относительная влажность воздуха, если сосуд нагреть до $90\celsius$?
        \item Получите ответ на предыдущий вопрос, используя плотности, а не давления.
    \end{enumerate}
}
\solutionspace{200pt}

\tasknumber{3}%
\task{%
    Закрытый сосуд объёмом $10\,\text{л}$ заполнен сухим воздухом при давлении $100\,\text{кПа}$ и температуре $20\celsius$.
    Каким станет давление в сосуде, если в него налить $20\,\text{г}$ воды и нагреть содержимое сосуда до $90\celsius$?
}
\answer{%
    Конечное давление газа в сосуде складывается по закону Дальтона из давления нагретого сухого воздуха $P'_\text{ воздуха }$ и
    давления насыщенного пара $P_\text{ пара }$:
    $$P' = P'_\text{ воздуха } + P_\text{ пара }.$$

    Сперва определим новое давление сухого воздуха из уравнения состояния идеального газа:
    $$\frac{P'_\text{ воздуха } \cdot V}{T'} = \nu R = \frac{P \cdot V}{T} \implies P'_\text{ воздуха } = P \cdot \frac{T'}{T}.$$

    Чтобы найти давление пара, нужно понять, будет ли он насыщенным после нагрева или нет.

    Плотность насыщенного пара при температуре равна $424\,\frac{\text{г}}{\text{м}^{3}}$, тогда для того,
    чтобы весь сосуд был заполнен насыщенным водяным паром нужно
    $m_\text{ н.
    п.
    } = \rho_\text{ н.
    п.
    90 $\celsius$ } \cdot V = 424\,\frac{\text{г}}{\text{м}^{3}} \cdot 10\,\text{л} = 4{,}2\,\text{г}$ воды.
    Сравнивая эту массу с массой воды из условия, получаем массу жидкости, которая испарится: $4{,}2\,\text{г}$.
    Осталось определить давление этого пара:
    $$P_\text{ пара } = \frac{m_\text{ пара }RT}{\mu V} = \frac{4{,}2\,\text{г} \cdot 8{,}31\,\frac{\text{Дж}}{\text{моль}\cdot\text{К}} \cdot 363\,\text{К}}{18\,\frac{\text{г}}{\text{моль}} \cdot 10\,\text{л}} \approx 71\,\text{кПа}.$$

    Получаем ответ: $P'_\text{ пара } = 194{,}9\,\text{кПа}$.

    Другой вариант решения для давления пара:
    Определим давление пара, если бы вся вода испарилась (что не факт):
    $$P_\text{ max } = \frac{mRT'}{\mu V} = \frac{20\,\text{г} \cdot 8{,}31\,\frac{\text{Дж}}{\text{моль}\cdot\text{К}} \cdot 363\,\text{К}}{18\,\frac{\text{г}}{\text{моль}} \cdot 10\,\text{л}} = 335\,\text{кПа}.$$
    Сравниваем это давление с давлением насыщенного пара при этой температуре $P_\text{ н.
    п.
    90 $\celsius$ } = 70\,\text{кПа}$:
     если у нас получилось меньше табличного значения,
    то вся вода испарилась, если же больше — испарилась лишь часть, а пар является насыщенным.
    Отсюда сразу получаем давление пара: $P'_\text{ пара } = 70{,}1$.
    Сравните этот результат с первым вариантом решения.

    Тут получаем ответ: $P'_\text{ пара } = 194{,}0\,\text{кПа}$.
}
\solutionspace{150pt}

\tasknumber{4}%
\task{%
    Напротив физических величин запишите определение, обозначение и единицы измерения в системе СИ (если есть):
    \begin{enumerate}
        \item абсолютная влажность,
        \item динамическое равновесие.
    \end{enumerate}
}

\variantsplitter

\addpersonalvariant{Дарья Кошман}

\tasknumber{1}%
\task{%
    Сколько молекул водяного пара содержится в сосуде объёмом $9\,\text{л}$ при температуре $90\celsius$,
    и влажности воздуха $60\%$?
}
\solutionspace{160pt}

\tasknumber{2}%
\task{%
    В герметичном сосуде находится влажный воздух при температуре $20\celsius$ и относительной влажности $20\%$.
    \begin{enumerate}
        \item Чему равно парциальное давление насыщенного водяного пара при этой температуре?
        \item Чему равно парциальное давление водяного пара?
        \item Определите точку росы этого пара?
        \item Каким станет парциальное давление водяного пара, если сосуд нагреть до  $70\celsius$?
        \item Чему будет равна относительная влажность воздуха, если сосуд нагреть до $70\celsius$?
        \item Получите ответ на предыдущий вопрос, используя плотности, а не давления.
    \end{enumerate}
}
\solutionspace{200pt}

\tasknumber{3}%
\task{%
    Закрытый сосуд объёмом $20\,\text{л}$ заполнен сухим воздухом при давлении $100\,\text{кПа}$ и температуре $10\celsius$.
    Каким станет давление в сосуде, если в него налить $5\,\text{г}$ воды и нагреть содержимое сосуда до $80\celsius$?
}
\answer{%
    Конечное давление газа в сосуде складывается по закону Дальтона из давления нагретого сухого воздуха $P'_\text{ воздуха }$ и
    давления насыщенного пара $P_\text{ пара }$:
    $$P' = P'_\text{ воздуха } + P_\text{ пара }.$$

    Сперва определим новое давление сухого воздуха из уравнения состояния идеального газа:
    $$\frac{P'_\text{ воздуха } \cdot V}{T'} = \nu R = \frac{P \cdot V}{T} \implies P'_\text{ воздуха } = P \cdot \frac{T'}{T}.$$

    Чтобы найти давление пара, нужно понять, будет ли он насыщенным после нагрева или нет.

    Плотность насыщенного пара при температуре равна $293\,\frac{\text{г}}{\text{м}^{3}}$, тогда для того,
    чтобы весь сосуд был заполнен насыщенным водяным паром нужно
    $m_\text{ н.
    п.
    } = \rho_\text{ н.
    п.
    80 $\celsius$ } \cdot V = 293\,\frac{\text{г}}{\text{м}^{3}} \cdot 20\,\text{л} = 5{,}9\,\text{г}$ воды.
    Сравнивая эту массу с массой воды из условия, получаем массу жидкости, которая испарится: $5{,}0\,\text{г}$.
    Осталось определить давление этого пара:
    $$P_\text{ пара } = \frac{m_\text{ пара }RT}{\mu V} = \frac{5{,}0\,\text{г} \cdot 8{,}31\,\frac{\text{Дж}}{\text{моль}\cdot\text{К}} \cdot 353\,\text{К}}{18\,\frac{\text{г}}{\text{моль}} \cdot 20\,\text{л}} \approx 40\,\text{кПа}.$$

    Получаем ответ: $P'_\text{ пара } = 165{,}5\,\text{кПа}$.

    Другой вариант решения для давления пара:
    Определим давление пара, если бы вся вода испарилась (что не факт):
    $$P_\text{ max } = \frac{mRT'}{\mu V} = \frac{5\,\text{г} \cdot 8{,}31\,\frac{\text{Дж}}{\text{моль}\cdot\text{К}} \cdot 353\,\text{К}}{18\,\frac{\text{г}}{\text{моль}} \cdot 20\,\text{л}} = 40\,\text{кПа}.$$
    Сравниваем это давление с давлением насыщенного пара при этой температуре $P_\text{ н.
    п.
    80 $\celsius$ } = 47\,\text{кПа}$:
     если у нас получилось меньше табличного значения,
    то вся вода испарилась, если же больше — испарилась лишь часть, а пар является насыщенным.
    Отсюда сразу получаем давление пара: $P'_\text{ пара } = 40{,}7$.
    Сравните этот результат с первым вариантом решения.

    Тут получаем ответ: $P'_\text{ пара } = 165{,}5\,\text{кПа}$.
}
\solutionspace{150pt}

\tasknumber{4}%
\task{%
    Напротив физических величин запишите определение, обозначение и единицы измерения в системе СИ (если есть):
    \begin{enumerate}
        \item абсолютная влажность,
        \item динамическое равновесие.
    \end{enumerate}
}

\variantsplitter

\addpersonalvariant{Анна Кузьмичёва}

\tasknumber{1}%
\task{%
    Сколько молекул водяного пара содержится в сосуде объёмом $3\,\text{л}$ при температуре $50\celsius$,
    и влажности воздуха $25\%$?
}
\solutionspace{160pt}

\tasknumber{2}%
\task{%
    В герметичном сосуде находится влажный воздух при температуре $15\celsius$ и относительной влажности $35\%$.
    \begin{enumerate}
        \item Чему равно парциальное давление насыщенного водяного пара при этой температуре?
        \item Чему равно парциальное давление водяного пара?
        \item Определите точку росы этого пара?
        \item Каким станет парциальное давление водяного пара, если сосуд нагреть до  $90\celsius$?
        \item Чему будет равна относительная влажность воздуха, если сосуд нагреть до $90\celsius$?
        \item Получите ответ на предыдущий вопрос, используя плотности, а не давления.
    \end{enumerate}
}
\solutionspace{200pt}

\tasknumber{3}%
\task{%
    Закрытый сосуд объёмом $15\,\text{л}$ заполнен сухим воздухом при давлении $100\,\text{кПа}$ и температуре $20\celsius$.
    Каким станет давление в сосуде, если в него налить $30\,\text{г}$ воды и нагреть содержимое сосуда до $100\celsius$?
}
\answer{%
    Конечное давление газа в сосуде складывается по закону Дальтона из давления нагретого сухого воздуха $P'_\text{ воздуха }$ и
    давления насыщенного пара $P_\text{ пара }$:
    $$P' = P'_\text{ воздуха } + P_\text{ пара }.$$

    Сперва определим новое давление сухого воздуха из уравнения состояния идеального газа:
    $$\frac{P'_\text{ воздуха } \cdot V}{T'} = \nu R = \frac{P \cdot V}{T} \implies P'_\text{ воздуха } = P \cdot \frac{T'}{T}.$$

    Чтобы найти давление пара, нужно понять, будет ли он насыщенным после нагрева или нет.

    Плотность насыщенного пара при температуре равна $598\,\frac{\text{г}}{\text{м}^{3}}$, тогда для того,
    чтобы весь сосуд был заполнен насыщенным водяным паром нужно
    $m_\text{ н.
    п.
    } = \rho_\text{ н.
    п.
    100 $\celsius$ } \cdot V = 598\,\frac{\text{г}}{\text{м}^{3}} \cdot 15\,\text{л} = 9{,}0\,\text{г}$ воды.
    Сравнивая эту массу с массой воды из условия, получаем массу жидкости, которая испарится: $9{,}0\,\text{г}$.
    Осталось определить давление этого пара:
    $$P_\text{ пара } = \frac{m_\text{ пара }RT}{\mu V} = \frac{9{,}0\,\text{г} \cdot 8{,}31\,\frac{\text{Дж}}{\text{моль}\cdot\text{К}} \cdot 373\,\text{К}}{18\,\frac{\text{г}}{\text{моль}} \cdot 15\,\text{л}} \approx 102\,\text{кПа}.$$

    Получаем ответ: $P'_\text{ пара } = 230{,}3\,\text{кПа}$.

    Другой вариант решения для давления пара:
    Определим давление пара, если бы вся вода испарилась (что не факт):
    $$P_\text{ max } = \frac{mRT'}{\mu V} = \frac{30\,\text{г} \cdot 8{,}31\,\frac{\text{Дж}}{\text{моль}\cdot\text{К}} \cdot 373\,\text{К}}{18\,\frac{\text{г}}{\text{моль}} \cdot 15\,\text{л}} = 344\,\text{кПа}.$$
    Сравниваем это давление с давлением насыщенного пара при этой температуре $P_\text{ н.
    п.
    100 $\celsius$ } = 101\,\text{кПа}$:
     если у нас получилось меньше табличного значения,
    то вся вода испарилась, если же больше — испарилась лишь часть, а пар является насыщенным.
    Отсюда сразу получаем давление пара: $P'_\text{ пара } = 101{,}3$.
    Сравните этот результат с первым вариантом решения.

    Тут получаем ответ: $P'_\text{ пара } = 228{,}6\,\text{кПа}$.
}
\solutionspace{150pt}

\tasknumber{4}%
\task{%
    Напротив физических величин запишите определение, обозначение и единицы измерения в системе СИ (если есть):
    \begin{enumerate}
        \item абсолютная влажность,
        \item динамическое равновесие.
    \end{enumerate}
}

\variantsplitter

\addpersonalvariant{Алёна Куприянова}

\tasknumber{1}%
\task{%
    Сколько молекул водяного пара содержится в сосуде объёмом $15\,\text{л}$ при температуре $40\celsius$,
    и влажности воздуха $60\%$?
}
\solutionspace{160pt}

\tasknumber{2}%
\task{%
    В герметичном сосуде находится влажный воздух при температуре $25\celsius$ и относительной влажности $50\%$.
    \begin{enumerate}
        \item Чему равно парциальное давление насыщенного водяного пара при этой температуре?
        \item Чему равно парциальное давление водяного пара?
        \item Определите точку росы этого пара?
        \item Каким станет парциальное давление водяного пара, если сосуд нагреть до  $90\celsius$?
        \item Чему будет равна относительная влажность воздуха, если сосуд нагреть до $90\celsius$?
        \item Получите ответ на предыдущий вопрос, используя плотности, а не давления.
    \end{enumerate}
}
\solutionspace{200pt}

\tasknumber{3}%
\task{%
    Закрытый сосуд объёмом $15\,\text{л}$ заполнен сухим воздухом при давлении $100\,\text{кПа}$ и температуре $10\celsius$.
    Каким станет давление в сосуде, если в него налить $5\,\text{г}$ воды и нагреть содержимое сосуда до $90\celsius$?
}
\answer{%
    Конечное давление газа в сосуде складывается по закону Дальтона из давления нагретого сухого воздуха $P'_\text{ воздуха }$ и
    давления насыщенного пара $P_\text{ пара }$:
    $$P' = P'_\text{ воздуха } + P_\text{ пара }.$$

    Сперва определим новое давление сухого воздуха из уравнения состояния идеального газа:
    $$\frac{P'_\text{ воздуха } \cdot V}{T'} = \nu R = \frac{P \cdot V}{T} \implies P'_\text{ воздуха } = P \cdot \frac{T'}{T}.$$

    Чтобы найти давление пара, нужно понять, будет ли он насыщенным после нагрева или нет.

    Плотность насыщенного пара при температуре равна $424\,\frac{\text{г}}{\text{м}^{3}}$, тогда для того,
    чтобы весь сосуд был заполнен насыщенным водяным паром нужно
    $m_\text{ н.
    п.
    } = \rho_\text{ н.
    п.
    90 $\celsius$ } \cdot V = 424\,\frac{\text{г}}{\text{м}^{3}} \cdot 15\,\text{л} = 6{,}4\,\text{г}$ воды.
    Сравнивая эту массу с массой воды из условия, получаем массу жидкости, которая испарится: $5{,}0\,\text{г}$.
    Осталось определить давление этого пара:
    $$P_\text{ пара } = \frac{m_\text{ пара }RT}{\mu V} = \frac{5{,}0\,\text{г} \cdot 8{,}31\,\frac{\text{Дж}}{\text{моль}\cdot\text{К}} \cdot 363\,\text{К}}{18\,\frac{\text{г}}{\text{моль}} \cdot 15\,\text{л}} \approx 55\,\text{кПа}.$$

    Получаем ответ: $P'_\text{ пара } = 184{,}1\,\text{кПа}$.

    Другой вариант решения для давления пара:
    Определим давление пара, если бы вся вода испарилась (что не факт):
    $$P_\text{ max } = \frac{mRT'}{\mu V} = \frac{5\,\text{г} \cdot 8{,}31\,\frac{\text{Дж}}{\text{моль}\cdot\text{К}} \cdot 363\,\text{К}}{18\,\frac{\text{г}}{\text{моль}} \cdot 15\,\text{л}} = 55\,\text{кПа}.$$
    Сравниваем это давление с давлением насыщенного пара при этой температуре $P_\text{ н.
    п.
    90 $\celsius$ } = 70\,\text{кПа}$:
     если у нас получилось меньше табличного значения,
    то вся вода испарилась, если же больше — испарилась лишь часть, а пар является насыщенным.
    Отсюда сразу получаем давление пара: $P'_\text{ пара } = 55{,}9$.
    Сравните этот результат с первым вариантом решения.

    Тут получаем ответ: $P'_\text{ пара } = 184{,}1\,\text{кПа}$.
}
\solutionspace{150pt}

\tasknumber{4}%
\task{%
    Напротив физических величин запишите определение, обозначение и единицы измерения в системе СИ (если есть):
    \begin{enumerate}
        \item относительная влажность,
        \item насыщенный пар.
    \end{enumerate}
}

\variantsplitter

\addpersonalvariant{Ярослав Лавровский}

\tasknumber{1}%
\task{%
    Сколько молекул водяного пара содержится в сосуде объёмом $9\,\text{л}$ при температуре $100\celsius$,
    и влажности воздуха $40\%$?
}
\solutionspace{160pt}

\tasknumber{2}%
\task{%
    В герметичном сосуде находится влажный воздух при температуре $15\celsius$ и относительной влажности $40\%$.
    \begin{enumerate}
        \item Чему равно парциальное давление насыщенного водяного пара при этой температуре?
        \item Чему равно парциальное давление водяного пара?
        \item Определите точку росы этого пара?
        \item Каким станет парциальное давление водяного пара, если сосуд нагреть до  $90\celsius$?
        \item Чему будет равна относительная влажность воздуха, если сосуд нагреть до $90\celsius$?
        \item Получите ответ на предыдущий вопрос, используя плотности, а не давления.
    \end{enumerate}
}
\solutionspace{200pt}

\tasknumber{3}%
\task{%
    Закрытый сосуд объёмом $15\,\text{л}$ заполнен сухим воздухом при давлении $100\,\text{кПа}$ и температуре $10\celsius$.
    Каким станет давление в сосуде, если в него налить $5\,\text{г}$ воды и нагреть содержимое сосуда до $80\celsius$?
}
\answer{%
    Конечное давление газа в сосуде складывается по закону Дальтона из давления нагретого сухого воздуха $P'_\text{ воздуха }$ и
    давления насыщенного пара $P_\text{ пара }$:
    $$P' = P'_\text{ воздуха } + P_\text{ пара }.$$

    Сперва определим новое давление сухого воздуха из уравнения состояния идеального газа:
    $$\frac{P'_\text{ воздуха } \cdot V}{T'} = \nu R = \frac{P \cdot V}{T} \implies P'_\text{ воздуха } = P \cdot \frac{T'}{T}.$$

    Чтобы найти давление пара, нужно понять, будет ли он насыщенным после нагрева или нет.

    Плотность насыщенного пара при температуре равна $293\,\frac{\text{г}}{\text{м}^{3}}$, тогда для того,
    чтобы весь сосуд был заполнен насыщенным водяным паром нужно
    $m_\text{ н.
    п.
    } = \rho_\text{ н.
    п.
    80 $\celsius$ } \cdot V = 293\,\frac{\text{г}}{\text{м}^{3}} \cdot 15\,\text{л} = 4{,}4\,\text{г}$ воды.
    Сравнивая эту массу с массой воды из условия, получаем массу жидкости, которая испарится: $4{,}4\,\text{г}$.
    Осталось определить давление этого пара:
    $$P_\text{ пара } = \frac{m_\text{ пара }RT}{\mu V} = \frac{4{,}4\,\text{г} \cdot 8{,}31\,\frac{\text{Дж}}{\text{моль}\cdot\text{К}} \cdot 353\,\text{К}}{18\,\frac{\text{г}}{\text{моль}} \cdot 15\,\text{л}} \approx 47\,\text{кПа}.$$

    Получаем ответ: $P'_\text{ пара } = 172{,}5\,\text{кПа}$.

    Другой вариант решения для давления пара:
    Определим давление пара, если бы вся вода испарилась (что не факт):
    $$P_\text{ max } = \frac{mRT'}{\mu V} = \frac{5\,\text{г} \cdot 8{,}31\,\frac{\text{Дж}}{\text{моль}\cdot\text{К}} \cdot 353\,\text{К}}{18\,\frac{\text{г}}{\text{моль}} \cdot 15\,\text{л}} = 54\,\text{кПа}.$$
    Сравниваем это давление с давлением насыщенного пара при этой температуре $P_\text{ н.
    п.
    80 $\celsius$ } = 47\,\text{кПа}$:
     если у нас получилось меньше табличного значения,
    то вся вода испарилась, если же больше — испарилась лишь часть, а пар является насыщенным.
    Отсюда сразу получаем давление пара: $P'_\text{ пара } = 47{,}3$.
    Сравните этот результат с первым вариантом решения.

    Тут получаем ответ: $P'_\text{ пара } = 172{,}0\,\text{кПа}$.
}
\solutionspace{150pt}

\tasknumber{4}%
\task{%
    Напротив физических величин запишите определение, обозначение и единицы измерения в системе СИ (если есть):
    \begin{enumerate}
        \item относительная влажность,
        \item насыщенный пар.
    \end{enumerate}
}

\variantsplitter

\addpersonalvariant{Анастасия Ламанова}

\tasknumber{1}%
\task{%
    Сколько молекул водяного пара содержится в сосуде объёмом $15\,\text{л}$ при температуре $15\celsius$,
    и влажности воздуха $35\%$?
}
\solutionspace{160pt}

\tasknumber{2}%
\task{%
    В герметичном сосуде находится влажный воздух при температуре $15\celsius$ и относительной влажности $35\%$.
    \begin{enumerate}
        \item Чему равно парциальное давление насыщенного водяного пара при этой температуре?
        \item Чему равно парциальное давление водяного пара?
        \item Определите точку росы этого пара?
        \item Каким станет парциальное давление водяного пара, если сосуд нагреть до  $70\celsius$?
        \item Чему будет равна относительная влажность воздуха, если сосуд нагреть до $70\celsius$?
        \item Получите ответ на предыдущий вопрос, используя плотности, а не давления.
    \end{enumerate}
}
\solutionspace{200pt}

\tasknumber{3}%
\task{%
    Закрытый сосуд объёмом $20\,\text{л}$ заполнен сухим воздухом при давлении $100\,\text{кПа}$ и температуре $30\celsius$.
    Каким станет давление в сосуде, если в него налить $30\,\text{г}$ воды и нагреть содержимое сосуда до $90\celsius$?
}
\answer{%
    Конечное давление газа в сосуде складывается по закону Дальтона из давления нагретого сухого воздуха $P'_\text{ воздуха }$ и
    давления насыщенного пара $P_\text{ пара }$:
    $$P' = P'_\text{ воздуха } + P_\text{ пара }.$$

    Сперва определим новое давление сухого воздуха из уравнения состояния идеального газа:
    $$\frac{P'_\text{ воздуха } \cdot V}{T'} = \nu R = \frac{P \cdot V}{T} \implies P'_\text{ воздуха } = P \cdot \frac{T'}{T}.$$

    Чтобы найти давление пара, нужно понять, будет ли он насыщенным после нагрева или нет.

    Плотность насыщенного пара при температуре равна $424\,\frac{\text{г}}{\text{м}^{3}}$, тогда для того,
    чтобы весь сосуд был заполнен насыщенным водяным паром нужно
    $m_\text{ н.
    п.
    } = \rho_\text{ н.
    п.
    90 $\celsius$ } \cdot V = 424\,\frac{\text{г}}{\text{м}^{3}} \cdot 20\,\text{л} = 8{,}5\,\text{г}$ воды.
    Сравнивая эту массу с массой воды из условия, получаем массу жидкости, которая испарится: $8{,}5\,\text{г}$.
    Осталось определить давление этого пара:
    $$P_\text{ пара } = \frac{m_\text{ пара }RT}{\mu V} = \frac{8{,}5\,\text{г} \cdot 8{,}31\,\frac{\text{Дж}}{\text{моль}\cdot\text{К}} \cdot 363\,\text{К}}{18\,\frac{\text{г}}{\text{моль}} \cdot 20\,\text{л}} \approx 71\,\text{кПа}.$$

    Получаем ответ: $P'_\text{ пара } = 190{,}9\,\text{кПа}$.

    Другой вариант решения для давления пара:
    Определим давление пара, если бы вся вода испарилась (что не факт):
    $$P_\text{ max } = \frac{mRT'}{\mu V} = \frac{30\,\text{г} \cdot 8{,}31\,\frac{\text{Дж}}{\text{моль}\cdot\text{К}} \cdot 363\,\text{К}}{18\,\frac{\text{г}}{\text{моль}} \cdot 20\,\text{л}} = 251\,\text{кПа}.$$
    Сравниваем это давление с давлением насыщенного пара при этой температуре $P_\text{ н.
    п.
    90 $\celsius$ } = 70\,\text{кПа}$:
     если у нас получилось меньше табличного значения,
    то вся вода испарилась, если же больше — испарилась лишь часть, а пар является насыщенным.
    Отсюда сразу получаем давление пара: $P'_\text{ пара } = 70{,}1$.
    Сравните этот результат с первым вариантом решения.

    Тут получаем ответ: $P'_\text{ пара } = 189{,}9\,\text{кПа}$.
}
\solutionspace{150pt}

\tasknumber{4}%
\task{%
    Напротив физических величин запишите определение, обозначение и единицы измерения в системе СИ (если есть):
    \begin{enumerate}
        \item относительная влажность,
        \item динамическое равновесие.
    \end{enumerate}
}

\variantsplitter

\addpersonalvariant{Виктория Легонькова}

\tasknumber{1}%
\task{%
    Сколько молекул водяного пара содержится в сосуде объёмом $6\,\text{л}$ при температуре $20\celsius$,
    и влажности воздуха $70\%$?
}
\solutionspace{160pt}

\tasknumber{2}%
\task{%
    В герметичном сосуде находится влажный воздух при температуре $30\celsius$ и относительной влажности $25\%$.
    \begin{enumerate}
        \item Чему равно парциальное давление насыщенного водяного пара при этой температуре?
        \item Чему равно парциальное давление водяного пара?
        \item Определите точку росы этого пара?
        \item Каким станет парциальное давление водяного пара, если сосуд нагреть до  $90\celsius$?
        \item Чему будет равна относительная влажность воздуха, если сосуд нагреть до $90\celsius$?
        \item Получите ответ на предыдущий вопрос, используя плотности, а не давления.
    \end{enumerate}
}
\solutionspace{200pt}

\tasknumber{3}%
\task{%
    Закрытый сосуд объёмом $10\,\text{л}$ заполнен сухим воздухом при давлении $100\,\text{кПа}$ и температуре $20\celsius$.
    Каким станет давление в сосуде, если в него налить $30\,\text{г}$ воды и нагреть содержимое сосуда до $90\celsius$?
}
\answer{%
    Конечное давление газа в сосуде складывается по закону Дальтона из давления нагретого сухого воздуха $P'_\text{ воздуха }$ и
    давления насыщенного пара $P_\text{ пара }$:
    $$P' = P'_\text{ воздуха } + P_\text{ пара }.$$

    Сперва определим новое давление сухого воздуха из уравнения состояния идеального газа:
    $$\frac{P'_\text{ воздуха } \cdot V}{T'} = \nu R = \frac{P \cdot V}{T} \implies P'_\text{ воздуха } = P \cdot \frac{T'}{T}.$$

    Чтобы найти давление пара, нужно понять, будет ли он насыщенным после нагрева или нет.

    Плотность насыщенного пара при температуре равна $424\,\frac{\text{г}}{\text{м}^{3}}$, тогда для того,
    чтобы весь сосуд был заполнен насыщенным водяным паром нужно
    $m_\text{ н.
    п.
    } = \rho_\text{ н.
    п.
    90 $\celsius$ } \cdot V = 424\,\frac{\text{г}}{\text{м}^{3}} \cdot 10\,\text{л} = 4{,}2\,\text{г}$ воды.
    Сравнивая эту массу с массой воды из условия, получаем массу жидкости, которая испарится: $4{,}2\,\text{г}$.
    Осталось определить давление этого пара:
    $$P_\text{ пара } = \frac{m_\text{ пара }RT}{\mu V} = \frac{4{,}2\,\text{г} \cdot 8{,}31\,\frac{\text{Дж}}{\text{моль}\cdot\text{К}} \cdot 363\,\text{К}}{18\,\frac{\text{г}}{\text{моль}} \cdot 10\,\text{л}} \approx 71\,\text{кПа}.$$

    Получаем ответ: $P'_\text{ пара } = 194{,}9\,\text{кПа}$.

    Другой вариант решения для давления пара:
    Определим давление пара, если бы вся вода испарилась (что не факт):
    $$P_\text{ max } = \frac{mRT'}{\mu V} = \frac{30\,\text{г} \cdot 8{,}31\,\frac{\text{Дж}}{\text{моль}\cdot\text{К}} \cdot 363\,\text{К}}{18\,\frac{\text{г}}{\text{моль}} \cdot 10\,\text{л}} = 502\,\text{кПа}.$$
    Сравниваем это давление с давлением насыщенного пара при этой температуре $P_\text{ н.
    п.
    90 $\celsius$ } = 70\,\text{кПа}$:
     если у нас получилось меньше табличного значения,
    то вся вода испарилась, если же больше — испарилась лишь часть, а пар является насыщенным.
    Отсюда сразу получаем давление пара: $P'_\text{ пара } = 70{,}1$.
    Сравните этот результат с первым вариантом решения.

    Тут получаем ответ: $P'_\text{ пара } = 194{,}0\,\text{кПа}$.
}
\solutionspace{150pt}

\tasknumber{4}%
\task{%
    Напротив физических величин запишите определение, обозначение и единицы измерения в системе СИ (если есть):
    \begin{enumerate}
        \item относительная влажность,
        \item динамическое равновесие.
    \end{enumerate}
}

\variantsplitter

\addpersonalvariant{Семён Мартынов}

\tasknumber{1}%
\task{%
    Сколько молекул водяного пара содержится в сосуде объёмом $6\,\text{л}$ при температуре $100\celsius$,
    и влажности воздуха $35\%$?
}
\solutionspace{160pt}

\tasknumber{2}%
\task{%
    В герметичном сосуде находится влажный воздух при температуре $15\celsius$ и относительной влажности $65\%$.
    \begin{enumerate}
        \item Чему равно парциальное давление насыщенного водяного пара при этой температуре?
        \item Чему равно парциальное давление водяного пара?
        \item Определите точку росы этого пара?
        \item Каким станет парциальное давление водяного пара, если сосуд нагреть до  $70\celsius$?
        \item Чему будет равна относительная влажность воздуха, если сосуд нагреть до $70\celsius$?
        \item Получите ответ на предыдущий вопрос, используя плотности, а не давления.
    \end{enumerate}
}
\solutionspace{200pt}

\tasknumber{3}%
\task{%
    Закрытый сосуд объёмом $10\,\text{л}$ заполнен сухим воздухом при давлении $100\,\text{кПа}$ и температуре $20\celsius$.
    Каким станет давление в сосуде, если в него налить $20\,\text{г}$ воды и нагреть содержимое сосуда до $80\celsius$?
}
\answer{%
    Конечное давление газа в сосуде складывается по закону Дальтона из давления нагретого сухого воздуха $P'_\text{ воздуха }$ и
    давления насыщенного пара $P_\text{ пара }$:
    $$P' = P'_\text{ воздуха } + P_\text{ пара }.$$

    Сперва определим новое давление сухого воздуха из уравнения состояния идеального газа:
    $$\frac{P'_\text{ воздуха } \cdot V}{T'} = \nu R = \frac{P \cdot V}{T} \implies P'_\text{ воздуха } = P \cdot \frac{T'}{T}.$$

    Чтобы найти давление пара, нужно понять, будет ли он насыщенным после нагрева или нет.

    Плотность насыщенного пара при температуре равна $293\,\frac{\text{г}}{\text{м}^{3}}$, тогда для того,
    чтобы весь сосуд был заполнен насыщенным водяным паром нужно
    $m_\text{ н.
    п.
    } = \rho_\text{ н.
    п.
    80 $\celsius$ } \cdot V = 293\,\frac{\text{г}}{\text{м}^{3}} \cdot 10\,\text{л} = 2{,}9\,\text{г}$ воды.
    Сравнивая эту массу с массой воды из условия, получаем массу жидкости, которая испарится: $2{,}9\,\text{г}$.
    Осталось определить давление этого пара:
    $$P_\text{ пара } = \frac{m_\text{ пара }RT}{\mu V} = \frac{2{,}9\,\text{г} \cdot 8{,}31\,\frac{\text{Дж}}{\text{моль}\cdot\text{К}} \cdot 353\,\text{К}}{18\,\frac{\text{г}}{\text{моль}} \cdot 10\,\text{л}} \approx 47\,\text{кПа}.$$

    Получаем ответ: $P'_\text{ пара } = 168{,}2\,\text{кПа}$.

    Другой вариант решения для давления пара:
    Определим давление пара, если бы вся вода испарилась (что не факт):
    $$P_\text{ max } = \frac{mRT'}{\mu V} = \frac{20\,\text{г} \cdot 8{,}31\,\frac{\text{Дж}}{\text{моль}\cdot\text{К}} \cdot 353\,\text{К}}{18\,\frac{\text{г}}{\text{моль}} \cdot 10\,\text{л}} = 325\,\text{кПа}.$$
    Сравниваем это давление с давлением насыщенного пара при этой температуре $P_\text{ н.
    п.
    80 $\celsius$ } = 47\,\text{кПа}$:
     если у нас получилось меньше табличного значения,
    то вся вода испарилась, если же больше — испарилась лишь часть, а пар является насыщенным.
    Отсюда сразу получаем давление пара: $P'_\text{ пара } = 47{,}3$.
    Сравните этот результат с первым вариантом решения.

    Тут получаем ответ: $P'_\text{ пара } = 167{,}8\,\text{кПа}$.
}
\solutionspace{150pt}

\tasknumber{4}%
\task{%
    Напротив физических величин запишите определение, обозначение и единицы измерения в системе СИ (если есть):
    \begin{enumerate}
        \item относительная влажность,
        \item насыщенный пар.
    \end{enumerate}
}

\variantsplitter

\addpersonalvariant{Варвара Минаева}

\tasknumber{1}%
\task{%
    Сколько молекул водяного пара содержится в сосуде объёмом $9\,\text{л}$ при температуре $100\celsius$,
    и влажности воздуха $25\%$?
}
\solutionspace{160pt}

\tasknumber{2}%
\task{%
    В герметичном сосуде находится влажный воздух при температуре $25\celsius$ и относительной влажности $25\%$.
    \begin{enumerate}
        \item Чему равно парциальное давление насыщенного водяного пара при этой температуре?
        \item Чему равно парциальное давление водяного пара?
        \item Определите точку росы этого пара?
        \item Каким станет парциальное давление водяного пара, если сосуд нагреть до  $90\celsius$?
        \item Чему будет равна относительная влажность воздуха, если сосуд нагреть до $90\celsius$?
        \item Получите ответ на предыдущий вопрос, используя плотности, а не давления.
    \end{enumerate}
}
\solutionspace{200pt}

\tasknumber{3}%
\task{%
    Закрытый сосуд объёмом $15\,\text{л}$ заполнен сухим воздухом при давлении $100\,\text{кПа}$ и температуре $10\celsius$.
    Каким станет давление в сосуде, если в него налить $5\,\text{г}$ воды и нагреть содержимое сосуда до $100\celsius$?
}
\answer{%
    Конечное давление газа в сосуде складывается по закону Дальтона из давления нагретого сухого воздуха $P'_\text{ воздуха }$ и
    давления насыщенного пара $P_\text{ пара }$:
    $$P' = P'_\text{ воздуха } + P_\text{ пара }.$$

    Сперва определим новое давление сухого воздуха из уравнения состояния идеального газа:
    $$\frac{P'_\text{ воздуха } \cdot V}{T'} = \nu R = \frac{P \cdot V}{T} \implies P'_\text{ воздуха } = P \cdot \frac{T'}{T}.$$

    Чтобы найти давление пара, нужно понять, будет ли он насыщенным после нагрева или нет.

    Плотность насыщенного пара при температуре равна $598\,\frac{\text{г}}{\text{м}^{3}}$, тогда для того,
    чтобы весь сосуд был заполнен насыщенным водяным паром нужно
    $m_\text{ н.
    п.
    } = \rho_\text{ н.
    п.
    100 $\celsius$ } \cdot V = 598\,\frac{\text{г}}{\text{м}^{3}} \cdot 15\,\text{л} = 9{,}0\,\text{г}$ воды.
    Сравнивая эту массу с массой воды из условия, получаем массу жидкости, которая испарится: $5{,}0\,\text{г}$.
    Осталось определить давление этого пара:
    $$P_\text{ пара } = \frac{m_\text{ пара }RT}{\mu V} = \frac{5{,}0\,\text{г} \cdot 8{,}31\,\frac{\text{Дж}}{\text{моль}\cdot\text{К}} \cdot 373\,\text{К}}{18\,\frac{\text{г}}{\text{моль}} \cdot 15\,\text{л}} \approx 57\,\text{кПа}.$$

    Получаем ответ: $P'_\text{ пара } = 189{,}2\,\text{кПа}$.

    Другой вариант решения для давления пара:
    Определим давление пара, если бы вся вода испарилась (что не факт):
    $$P_\text{ max } = \frac{mRT'}{\mu V} = \frac{5\,\text{г} \cdot 8{,}31\,\frac{\text{Дж}}{\text{моль}\cdot\text{К}} \cdot 373\,\text{К}}{18\,\frac{\text{г}}{\text{моль}} \cdot 15\,\text{л}} = 57\,\text{кПа}.$$
    Сравниваем это давление с давлением насыщенного пара при этой температуре $P_\text{ н.
    п.
    100 $\celsius$ } = 101\,\text{кПа}$:
     если у нас получилось меньше табличного значения,
    то вся вода испарилась, если же больше — испарилась лишь часть, а пар является насыщенным.
    Отсюда сразу получаем давление пара: $P'_\text{ пара } = 57{,}4$.
    Сравните этот результат с первым вариантом решения.

    Тут получаем ответ: $P'_\text{ пара } = 189{,}2\,\text{кПа}$.
}
\solutionspace{150pt}

\tasknumber{4}%
\task{%
    Напротив физических величин запишите определение, обозначение и единицы измерения в системе СИ (если есть):
    \begin{enumerate}
        \item абсолютная влажность,
        \item динамическое равновесие.
    \end{enumerate}
}

\variantsplitter

\addpersonalvariant{Леонид Никитин}

\tasknumber{1}%
\task{%
    Сколько молекул водяного пара содержится в сосуде объёмом $9\,\text{л}$ при температуре $80\celsius$,
    и влажности воздуха $40\%$?
}
\solutionspace{160pt}

\tasknumber{2}%
\task{%
    В герметичном сосуде находится влажный воздух при температуре $25\celsius$ и относительной влажности $20\%$.
    \begin{enumerate}
        \item Чему равно парциальное давление насыщенного водяного пара при этой температуре?
        \item Чему равно парциальное давление водяного пара?
        \item Определите точку росы этого пара?
        \item Каким станет парциальное давление водяного пара, если сосуд нагреть до  $80\celsius$?
        \item Чему будет равна относительная влажность воздуха, если сосуд нагреть до $80\celsius$?
        \item Получите ответ на предыдущий вопрос, используя плотности, а не давления.
    \end{enumerate}
}
\solutionspace{200pt}

\tasknumber{3}%
\task{%
    Закрытый сосуд объёмом $10\,\text{л}$ заполнен сухим воздухом при давлении $100\,\text{кПа}$ и температуре $20\celsius$.
    Каким станет давление в сосуде, если в него налить $20\,\text{г}$ воды и нагреть содержимое сосуда до $90\celsius$?
}
\answer{%
    Конечное давление газа в сосуде складывается по закону Дальтона из давления нагретого сухого воздуха $P'_\text{ воздуха }$ и
    давления насыщенного пара $P_\text{ пара }$:
    $$P' = P'_\text{ воздуха } + P_\text{ пара }.$$

    Сперва определим новое давление сухого воздуха из уравнения состояния идеального газа:
    $$\frac{P'_\text{ воздуха } \cdot V}{T'} = \nu R = \frac{P \cdot V}{T} \implies P'_\text{ воздуха } = P \cdot \frac{T'}{T}.$$

    Чтобы найти давление пара, нужно понять, будет ли он насыщенным после нагрева или нет.

    Плотность насыщенного пара при температуре равна $424\,\frac{\text{г}}{\text{м}^{3}}$, тогда для того,
    чтобы весь сосуд был заполнен насыщенным водяным паром нужно
    $m_\text{ н.
    п.
    } = \rho_\text{ н.
    п.
    90 $\celsius$ } \cdot V = 424\,\frac{\text{г}}{\text{м}^{3}} \cdot 10\,\text{л} = 4{,}2\,\text{г}$ воды.
    Сравнивая эту массу с массой воды из условия, получаем массу жидкости, которая испарится: $4{,}2\,\text{г}$.
    Осталось определить давление этого пара:
    $$P_\text{ пара } = \frac{m_\text{ пара }RT}{\mu V} = \frac{4{,}2\,\text{г} \cdot 8{,}31\,\frac{\text{Дж}}{\text{моль}\cdot\text{К}} \cdot 363\,\text{К}}{18\,\frac{\text{г}}{\text{моль}} \cdot 10\,\text{л}} \approx 71\,\text{кПа}.$$

    Получаем ответ: $P'_\text{ пара } = 194{,}9\,\text{кПа}$.

    Другой вариант решения для давления пара:
    Определим давление пара, если бы вся вода испарилась (что не факт):
    $$P_\text{ max } = \frac{mRT'}{\mu V} = \frac{20\,\text{г} \cdot 8{,}31\,\frac{\text{Дж}}{\text{моль}\cdot\text{К}} \cdot 363\,\text{К}}{18\,\frac{\text{г}}{\text{моль}} \cdot 10\,\text{л}} = 335\,\text{кПа}.$$
    Сравниваем это давление с давлением насыщенного пара при этой температуре $P_\text{ н.
    п.
    90 $\celsius$ } = 70\,\text{кПа}$:
     если у нас получилось меньше табличного значения,
    то вся вода испарилась, если же больше — испарилась лишь часть, а пар является насыщенным.
    Отсюда сразу получаем давление пара: $P'_\text{ пара } = 70{,}1$.
    Сравните этот результат с первым вариантом решения.

    Тут получаем ответ: $P'_\text{ пара } = 194{,}0\,\text{кПа}$.
}
\solutionspace{150pt}

\tasknumber{4}%
\task{%
    Напротив физических величин запишите определение, обозначение и единицы измерения в системе СИ (если есть):
    \begin{enumerate}
        \item абсолютная влажность,
        \item динамическое равновесие.
    \end{enumerate}
}

\variantsplitter

\addpersonalvariant{Тимофей Полетаев}

\tasknumber{1}%
\task{%
    Сколько молекул водяного пара содержится в сосуде объёмом $15\,\text{л}$ при температуре $100\celsius$,
    и влажности воздуха $20\%$?
}
\solutionspace{160pt}

\tasknumber{2}%
\task{%
    В герметичном сосуде находится влажный воздух при температуре $15\celsius$ и относительной влажности $25\%$.
    \begin{enumerate}
        \item Чему равно парциальное давление насыщенного водяного пара при этой температуре?
        \item Чему равно парциальное давление водяного пара?
        \item Определите точку росы этого пара?
        \item Каким станет парциальное давление водяного пара, если сосуд нагреть до  $70\celsius$?
        \item Чему будет равна относительная влажность воздуха, если сосуд нагреть до $70\celsius$?
        \item Получите ответ на предыдущий вопрос, используя плотности, а не давления.
    \end{enumerate}
}
\solutionspace{200pt}

\tasknumber{3}%
\task{%
    Закрытый сосуд объёмом $15\,\text{л}$ заполнен сухим воздухом при давлении $100\,\text{кПа}$ и температуре $20\celsius$.
    Каким станет давление в сосуде, если в него налить $5\,\text{г}$ воды и нагреть содержимое сосуда до $90\celsius$?
}
\answer{%
    Конечное давление газа в сосуде складывается по закону Дальтона из давления нагретого сухого воздуха $P'_\text{ воздуха }$ и
    давления насыщенного пара $P_\text{ пара }$:
    $$P' = P'_\text{ воздуха } + P_\text{ пара }.$$

    Сперва определим новое давление сухого воздуха из уравнения состояния идеального газа:
    $$\frac{P'_\text{ воздуха } \cdot V}{T'} = \nu R = \frac{P \cdot V}{T} \implies P'_\text{ воздуха } = P \cdot \frac{T'}{T}.$$

    Чтобы найти давление пара, нужно понять, будет ли он насыщенным после нагрева или нет.

    Плотность насыщенного пара при температуре равна $424\,\frac{\text{г}}{\text{м}^{3}}$, тогда для того,
    чтобы весь сосуд был заполнен насыщенным водяным паром нужно
    $m_\text{ н.
    п.
    } = \rho_\text{ н.
    п.
    90 $\celsius$ } \cdot V = 424\,\frac{\text{г}}{\text{м}^{3}} \cdot 15\,\text{л} = 6{,}4\,\text{г}$ воды.
    Сравнивая эту массу с массой воды из условия, получаем массу жидкости, которая испарится: $5{,}0\,\text{г}$.
    Осталось определить давление этого пара:
    $$P_\text{ пара } = \frac{m_\text{ пара }RT}{\mu V} = \frac{5{,}0\,\text{г} \cdot 8{,}31\,\frac{\text{Дж}}{\text{моль}\cdot\text{К}} \cdot 363\,\text{К}}{18\,\frac{\text{г}}{\text{моль}} \cdot 15\,\text{л}} \approx 55\,\text{кПа}.$$

    Получаем ответ: $P'_\text{ пара } = 179{,}8\,\text{кПа}$.

    Другой вариант решения для давления пара:
    Определим давление пара, если бы вся вода испарилась (что не факт):
    $$P_\text{ max } = \frac{mRT'}{\mu V} = \frac{5\,\text{г} \cdot 8{,}31\,\frac{\text{Дж}}{\text{моль}\cdot\text{К}} \cdot 363\,\text{К}}{18\,\frac{\text{г}}{\text{моль}} \cdot 15\,\text{л}} = 55\,\text{кПа}.$$
    Сравниваем это давление с давлением насыщенного пара при этой температуре $P_\text{ н.
    п.
    90 $\celsius$ } = 70\,\text{кПа}$:
     если у нас получилось меньше табличного значения,
    то вся вода испарилась, если же больше — испарилась лишь часть, а пар является насыщенным.
    Отсюда сразу получаем давление пара: $P'_\text{ пара } = 55{,}9$.
    Сравните этот результат с первым вариантом решения.

    Тут получаем ответ: $P'_\text{ пара } = 179{,}8\,\text{кПа}$.
}
\solutionspace{150pt}

\tasknumber{4}%
\task{%
    Напротив физических величин запишите определение, обозначение и единицы измерения в системе СИ (если есть):
    \begin{enumerate}
        \item относительная влажность,
        \item динамическое равновесие.
    \end{enumerate}
}

\variantsplitter

\addpersonalvariant{Андрей Рожков}

\tasknumber{1}%
\task{%
    Сколько молекул водяного пара содержится в сосуде объёмом $12\,\text{л}$ при температуре $30\celsius$,
    и влажности воздуха $75\%$?
}
\solutionspace{160pt}

\tasknumber{2}%
\task{%
    В герметичном сосуде находится влажный воздух при температуре $40\celsius$ и относительной влажности $40\%$.
    \begin{enumerate}
        \item Чему равно парциальное давление насыщенного водяного пара при этой температуре?
        \item Чему равно парциальное давление водяного пара?
        \item Определите точку росы этого пара?
        \item Каким станет парциальное давление водяного пара, если сосуд нагреть до  $90\celsius$?
        \item Чему будет равна относительная влажность воздуха, если сосуд нагреть до $90\celsius$?
        \item Получите ответ на предыдущий вопрос, используя плотности, а не давления.
    \end{enumerate}
}
\solutionspace{200pt}

\tasknumber{3}%
\task{%
    Закрытый сосуд объёмом $20\,\text{л}$ заполнен сухим воздухом при давлении $100\,\text{кПа}$ и температуре $20\celsius$.
    Каким станет давление в сосуде, если в него налить $20\,\text{г}$ воды и нагреть содержимое сосуда до $80\celsius$?
}
\answer{%
    Конечное давление газа в сосуде складывается по закону Дальтона из давления нагретого сухого воздуха $P'_\text{ воздуха }$ и
    давления насыщенного пара $P_\text{ пара }$:
    $$P' = P'_\text{ воздуха } + P_\text{ пара }.$$

    Сперва определим новое давление сухого воздуха из уравнения состояния идеального газа:
    $$\frac{P'_\text{ воздуха } \cdot V}{T'} = \nu R = \frac{P \cdot V}{T} \implies P'_\text{ воздуха } = P \cdot \frac{T'}{T}.$$

    Чтобы найти давление пара, нужно понять, будет ли он насыщенным после нагрева или нет.

    Плотность насыщенного пара при температуре равна $293\,\frac{\text{г}}{\text{м}^{3}}$, тогда для того,
    чтобы весь сосуд был заполнен насыщенным водяным паром нужно
    $m_\text{ н.
    п.
    } = \rho_\text{ н.
    п.
    80 $\celsius$ } \cdot V = 293\,\frac{\text{г}}{\text{м}^{3}} \cdot 20\,\text{л} = 5{,}9\,\text{г}$ воды.
    Сравнивая эту массу с массой воды из условия, получаем массу жидкости, которая испарится: $5{,}9\,\text{г}$.
    Осталось определить давление этого пара:
    $$P_\text{ пара } = \frac{m_\text{ пара }RT}{\mu V} = \frac{5{,}9\,\text{г} \cdot 8{,}31\,\frac{\text{Дж}}{\text{моль}\cdot\text{К}} \cdot 353\,\text{К}}{18\,\frac{\text{г}}{\text{моль}} \cdot 20\,\text{л}} \approx 47\,\text{кПа}.$$

    Получаем ответ: $P'_\text{ пара } = 168{,}2\,\text{кПа}$.

    Другой вариант решения для давления пара:
    Определим давление пара, если бы вся вода испарилась (что не факт):
    $$P_\text{ max } = \frac{mRT'}{\mu V} = \frac{20\,\text{г} \cdot 8{,}31\,\frac{\text{Дж}}{\text{моль}\cdot\text{К}} \cdot 353\,\text{К}}{18\,\frac{\text{г}}{\text{моль}} \cdot 20\,\text{л}} = 162\,\text{кПа}.$$
    Сравниваем это давление с давлением насыщенного пара при этой температуре $P_\text{ н.
    п.
    80 $\celsius$ } = 47\,\text{кПа}$:
     если у нас получилось меньше табличного значения,
    то вся вода испарилась, если же больше — испарилась лишь часть, а пар является насыщенным.
    Отсюда сразу получаем давление пара: $P'_\text{ пара } = 47{,}3$.
    Сравните этот результат с первым вариантом решения.

    Тут получаем ответ: $P'_\text{ пара } = 167{,}8\,\text{кПа}$.
}
\solutionspace{150pt}

\tasknumber{4}%
\task{%
    Напротив физических величин запишите определение, обозначение и единицы измерения в системе СИ (если есть):
    \begin{enumerate}
        \item относительная влажность,
        \item динамическое равновесие.
    \end{enumerate}
}

\variantsplitter

\addpersonalvariant{Рената Таржиманова}

\tasknumber{1}%
\task{%
    Сколько молекул водяного пара содержится в сосуде объёмом $12\,\text{л}$ при температуре $20\celsius$,
    и влажности воздуха $35\%$?
}
\solutionspace{160pt}

\tasknumber{2}%
\task{%
    В герметичном сосуде находится влажный воздух при температуре $15\celsius$ и относительной влажности $55\%$.
    \begin{enumerate}
        \item Чему равно парциальное давление насыщенного водяного пара при этой температуре?
        \item Чему равно парциальное давление водяного пара?
        \item Определите точку росы этого пара?
        \item Каким станет парциальное давление водяного пара, если сосуд нагреть до  $80\celsius$?
        \item Чему будет равна относительная влажность воздуха, если сосуд нагреть до $80\celsius$?
        \item Получите ответ на предыдущий вопрос, используя плотности, а не давления.
    \end{enumerate}
}
\solutionspace{200pt}

\tasknumber{3}%
\task{%
    Закрытый сосуд объёмом $20\,\text{л}$ заполнен сухим воздухом при давлении $100\,\text{кПа}$ и температуре $20\celsius$.
    Каким станет давление в сосуде, если в него налить $10\,\text{г}$ воды и нагреть содержимое сосуда до $90\celsius$?
}
\answer{%
    Конечное давление газа в сосуде складывается по закону Дальтона из давления нагретого сухого воздуха $P'_\text{ воздуха }$ и
    давления насыщенного пара $P_\text{ пара }$:
    $$P' = P'_\text{ воздуха } + P_\text{ пара }.$$

    Сперва определим новое давление сухого воздуха из уравнения состояния идеального газа:
    $$\frac{P'_\text{ воздуха } \cdot V}{T'} = \nu R = \frac{P \cdot V}{T} \implies P'_\text{ воздуха } = P \cdot \frac{T'}{T}.$$

    Чтобы найти давление пара, нужно понять, будет ли он насыщенным после нагрева или нет.

    Плотность насыщенного пара при температуре равна $424\,\frac{\text{г}}{\text{м}^{3}}$, тогда для того,
    чтобы весь сосуд был заполнен насыщенным водяным паром нужно
    $m_\text{ н.
    п.
    } = \rho_\text{ н.
    п.
    90 $\celsius$ } \cdot V = 424\,\frac{\text{г}}{\text{м}^{3}} \cdot 20\,\text{л} = 8{,}5\,\text{г}$ воды.
    Сравнивая эту массу с массой воды из условия, получаем массу жидкости, которая испарится: $8{,}5\,\text{г}$.
    Осталось определить давление этого пара:
    $$P_\text{ пара } = \frac{m_\text{ пара }RT}{\mu V} = \frac{8{,}5\,\text{г} \cdot 8{,}31\,\frac{\text{Дж}}{\text{моль}\cdot\text{К}} \cdot 363\,\text{К}}{18\,\frac{\text{г}}{\text{моль}} \cdot 20\,\text{л}} \approx 71\,\text{кПа}.$$

    Получаем ответ: $P'_\text{ пара } = 194{,}9\,\text{кПа}$.

    Другой вариант решения для давления пара:
    Определим давление пара, если бы вся вода испарилась (что не факт):
    $$P_\text{ max } = \frac{mRT'}{\mu V} = \frac{10\,\text{г} \cdot 8{,}31\,\frac{\text{Дж}}{\text{моль}\cdot\text{К}} \cdot 363\,\text{К}}{18\,\frac{\text{г}}{\text{моль}} \cdot 20\,\text{л}} = 83\,\text{кПа}.$$
    Сравниваем это давление с давлением насыщенного пара при этой температуре $P_\text{ н.
    п.
    90 $\celsius$ } = 70\,\text{кПа}$:
     если у нас получилось меньше табличного значения,
    то вся вода испарилась, если же больше — испарилась лишь часть, а пар является насыщенным.
    Отсюда сразу получаем давление пара: $P'_\text{ пара } = 70{,}1$.
    Сравните этот результат с первым вариантом решения.

    Тут получаем ответ: $P'_\text{ пара } = 194{,}0\,\text{кПа}$.
}
\solutionspace{150pt}

\tasknumber{4}%
\task{%
    Напротив физических величин запишите определение, обозначение и единицы измерения в системе СИ (если есть):
    \begin{enumerate}
        \item абсолютная влажность,
        \item насыщенный пар.
    \end{enumerate}
}

\variantsplitter

\addpersonalvariant{Андрей Щербаков}

\tasknumber{1}%
\task{%
    Сколько молекул водяного пара содержится в сосуде объёмом $6\,\text{л}$ при температуре $20\celsius$,
    и влажности воздуха $65\%$?
}
\solutionspace{160pt}

\tasknumber{2}%
\task{%
    В герметичном сосуде находится влажный воздух при температуре $25\celsius$ и относительной влажности $40\%$.
    \begin{enumerate}
        \item Чему равно парциальное давление насыщенного водяного пара при этой температуре?
        \item Чему равно парциальное давление водяного пара?
        \item Определите точку росы этого пара?
        \item Каким станет парциальное давление водяного пара, если сосуд нагреть до  $90\celsius$?
        \item Чему будет равна относительная влажность воздуха, если сосуд нагреть до $90\celsius$?
        \item Получите ответ на предыдущий вопрос, используя плотности, а не давления.
    \end{enumerate}
}
\solutionspace{200pt}

\tasknumber{3}%
\task{%
    Закрытый сосуд объёмом $15\,\text{л}$ заполнен сухим воздухом при давлении $100\,\text{кПа}$ и температуре $20\celsius$.
    Каким станет давление в сосуде, если в него налить $5\,\text{г}$ воды и нагреть содержимое сосуда до $90\celsius$?
}
\answer{%
    Конечное давление газа в сосуде складывается по закону Дальтона из давления нагретого сухого воздуха $P'_\text{ воздуха }$ и
    давления насыщенного пара $P_\text{ пара }$:
    $$P' = P'_\text{ воздуха } + P_\text{ пара }.$$

    Сперва определим новое давление сухого воздуха из уравнения состояния идеального газа:
    $$\frac{P'_\text{ воздуха } \cdot V}{T'} = \nu R = \frac{P \cdot V}{T} \implies P'_\text{ воздуха } = P \cdot \frac{T'}{T}.$$

    Чтобы найти давление пара, нужно понять, будет ли он насыщенным после нагрева или нет.

    Плотность насыщенного пара при температуре равна $424\,\frac{\text{г}}{\text{м}^{3}}$, тогда для того,
    чтобы весь сосуд был заполнен насыщенным водяным паром нужно
    $m_\text{ н.
    п.
    } = \rho_\text{ н.
    п.
    90 $\celsius$ } \cdot V = 424\,\frac{\text{г}}{\text{м}^{3}} \cdot 15\,\text{л} = 6{,}4\,\text{г}$ воды.
    Сравнивая эту массу с массой воды из условия, получаем массу жидкости, которая испарится: $5{,}0\,\text{г}$.
    Осталось определить давление этого пара:
    $$P_\text{ пара } = \frac{m_\text{ пара }RT}{\mu V} = \frac{5{,}0\,\text{г} \cdot 8{,}31\,\frac{\text{Дж}}{\text{моль}\cdot\text{К}} \cdot 363\,\text{К}}{18\,\frac{\text{г}}{\text{моль}} \cdot 15\,\text{л}} \approx 55\,\text{кПа}.$$

    Получаем ответ: $P'_\text{ пара } = 179{,}8\,\text{кПа}$.

    Другой вариант решения для давления пара:
    Определим давление пара, если бы вся вода испарилась (что не факт):
    $$P_\text{ max } = \frac{mRT'}{\mu V} = \frac{5\,\text{г} \cdot 8{,}31\,\frac{\text{Дж}}{\text{моль}\cdot\text{К}} \cdot 363\,\text{К}}{18\,\frac{\text{г}}{\text{моль}} \cdot 15\,\text{л}} = 55\,\text{кПа}.$$
    Сравниваем это давление с давлением насыщенного пара при этой температуре $P_\text{ н.
    п.
    90 $\celsius$ } = 70\,\text{кПа}$:
     если у нас получилось меньше табличного значения,
    то вся вода испарилась, если же больше — испарилась лишь часть, а пар является насыщенным.
    Отсюда сразу получаем давление пара: $P'_\text{ пара } = 55{,}9$.
    Сравните этот результат с первым вариантом решения.

    Тут получаем ответ: $P'_\text{ пара } = 179{,}8\,\text{кПа}$.
}
\solutionspace{150pt}

\tasknumber{4}%
\task{%
    Напротив физических величин запишите определение, обозначение и единицы измерения в системе СИ (если есть):
    \begin{enumerate}
        \item абсолютная влажность,
        \item насыщенный пар.
    \end{enumerate}
}

\variantsplitter

\addpersonalvariant{Михаил Ярошевский}

\tasknumber{1}%
\task{%
    Сколько молекул водяного пара содержится в сосуде объёмом $3\,\text{л}$ при температуре $60\celsius$,
    и влажности воздуха $25\%$?
}
\solutionspace{160pt}

\tasknumber{2}%
\task{%
    В герметичном сосуде находится влажный воздух при температуре $30\celsius$ и относительной влажности $55\%$.
    \begin{enumerate}
        \item Чему равно парциальное давление насыщенного водяного пара при этой температуре?
        \item Чему равно парциальное давление водяного пара?
        \item Определите точку росы этого пара?
        \item Каким станет парциальное давление водяного пара, если сосуд нагреть до  $90\celsius$?
        \item Чему будет равна относительная влажность воздуха, если сосуд нагреть до $90\celsius$?
        \item Получите ответ на предыдущий вопрос, используя плотности, а не давления.
    \end{enumerate}
}
\solutionspace{200pt}

\tasknumber{3}%
\task{%
    Закрытый сосуд объёмом $10\,\text{л}$ заполнен сухим воздухом при давлении $100\,\text{кПа}$ и температуре $10\celsius$.
    Каким станет давление в сосуде, если в него налить $30\,\text{г}$ воды и нагреть содержимое сосуда до $100\celsius$?
}
\answer{%
    Конечное давление газа в сосуде складывается по закону Дальтона из давления нагретого сухого воздуха $P'_\text{ воздуха }$ и
    давления насыщенного пара $P_\text{ пара }$:
    $$P' = P'_\text{ воздуха } + P_\text{ пара }.$$

    Сперва определим новое давление сухого воздуха из уравнения состояния идеального газа:
    $$\frac{P'_\text{ воздуха } \cdot V}{T'} = \nu R = \frac{P \cdot V}{T} \implies P'_\text{ воздуха } = P \cdot \frac{T'}{T}.$$

    Чтобы найти давление пара, нужно понять, будет ли он насыщенным после нагрева или нет.

    Плотность насыщенного пара при температуре равна $598\,\frac{\text{г}}{\text{м}^{3}}$, тогда для того,
    чтобы весь сосуд был заполнен насыщенным водяным паром нужно
    $m_\text{ н.
    п.
    } = \rho_\text{ н.
    п.
    100 $\celsius$ } \cdot V = 598\,\frac{\text{г}}{\text{м}^{3}} \cdot 10\,\text{л} = 6{,}0\,\text{г}$ воды.
    Сравнивая эту массу с массой воды из условия, получаем массу жидкости, которая испарится: $6{,}0\,\text{г}$.
    Осталось определить давление этого пара:
    $$P_\text{ пара } = \frac{m_\text{ пара }RT}{\mu V} = \frac{6{,}0\,\text{г} \cdot 8{,}31\,\frac{\text{Дж}}{\text{моль}\cdot\text{К}} \cdot 373\,\text{К}}{18\,\frac{\text{г}}{\text{моль}} \cdot 10\,\text{л}} \approx 102\,\text{кПа}.$$

    Получаем ответ: $P'_\text{ пара } = 234{,}8\,\text{кПа}$.

    Другой вариант решения для давления пара:
    Определим давление пара, если бы вся вода испарилась (что не факт):
    $$P_\text{ max } = \frac{mRT'}{\mu V} = \frac{30\,\text{г} \cdot 8{,}31\,\frac{\text{Дж}}{\text{моль}\cdot\text{К}} \cdot 373\,\text{К}}{18\,\frac{\text{г}}{\text{моль}} \cdot 10\,\text{л}} = 516\,\text{кПа}.$$
    Сравниваем это давление с давлением насыщенного пара при этой температуре $P_\text{ н.
    п.
    100 $\celsius$ } = 101\,\text{кПа}$:
     если у нас получилось меньше табличного значения,
    то вся вода испарилась, если же больше — испарилась лишь часть, а пар является насыщенным.
    Отсюда сразу получаем давление пара: $P'_\text{ пара } = 101{,}3$.
    Сравните этот результат с первым вариантом решения.

    Тут получаем ответ: $P'_\text{ пара } = 233{,}1\,\text{кПа}$.
}
\solutionspace{150pt}

\tasknumber{4}%
\task{%
    Напротив физических величин запишите определение, обозначение и единицы измерения в системе СИ (если есть):
    \begin{enumerate}
        \item абсолютная влажность,
        \item насыщенный пар.
    \end{enumerate}
}

\variantsplitter

\addpersonalvariant{Алексей Алимпиев}

\tasknumber{1}%
\task{%
    Сколько молекул водяного пара содержится в сосуде объёмом $12\,\text{л}$ при температуре $70\celsius$,
    и влажности воздуха $30\%$?
}
\solutionspace{160pt}

\tasknumber{2}%
\task{%
    В герметичном сосуде находится влажный воздух при температуре $25\celsius$ и относительной влажности $35\%$.
    \begin{enumerate}
        \item Чему равно парциальное давление насыщенного водяного пара при этой температуре?
        \item Чему равно парциальное давление водяного пара?
        \item Определите точку росы этого пара?
        \item Каким станет парциальное давление водяного пара, если сосуд нагреть до  $70\celsius$?
        \item Чему будет равна относительная влажность воздуха, если сосуд нагреть до $70\celsius$?
        \item Получите ответ на предыдущий вопрос, используя плотности, а не давления.
    \end{enumerate}
}
\solutionspace{200pt}

\tasknumber{3}%
\task{%
    Закрытый сосуд объёмом $10\,\text{л}$ заполнен сухим воздухом при давлении $100\,\text{кПа}$ и температуре $20\celsius$.
    Каким станет давление в сосуде, если в него налить $20\,\text{г}$ воды и нагреть содержимое сосуда до $90\celsius$?
}
\answer{%
    Конечное давление газа в сосуде складывается по закону Дальтона из давления нагретого сухого воздуха $P'_\text{ воздуха }$ и
    давления насыщенного пара $P_\text{ пара }$:
    $$P' = P'_\text{ воздуха } + P_\text{ пара }.$$

    Сперва определим новое давление сухого воздуха из уравнения состояния идеального газа:
    $$\frac{P'_\text{ воздуха } \cdot V}{T'} = \nu R = \frac{P \cdot V}{T} \implies P'_\text{ воздуха } = P \cdot \frac{T'}{T}.$$

    Чтобы найти давление пара, нужно понять, будет ли он насыщенным после нагрева или нет.

    Плотность насыщенного пара при температуре равна $424\,\frac{\text{г}}{\text{м}^{3}}$, тогда для того,
    чтобы весь сосуд был заполнен насыщенным водяным паром нужно
    $m_\text{ н.
    п.
    } = \rho_\text{ н.
    п.
    90 $\celsius$ } \cdot V = 424\,\frac{\text{г}}{\text{м}^{3}} \cdot 10\,\text{л} = 4{,}2\,\text{г}$ воды.
    Сравнивая эту массу с массой воды из условия, получаем массу жидкости, которая испарится: $4{,}2\,\text{г}$.
    Осталось определить давление этого пара:
    $$P_\text{ пара } = \frac{m_\text{ пара }RT}{\mu V} = \frac{4{,}2\,\text{г} \cdot 8{,}31\,\frac{\text{Дж}}{\text{моль}\cdot\text{К}} \cdot 363\,\text{К}}{18\,\frac{\text{г}}{\text{моль}} \cdot 10\,\text{л}} \approx 71\,\text{кПа}.$$

    Получаем ответ: $P'_\text{ пара } = 194{,}9\,\text{кПа}$.

    Другой вариант решения для давления пара:
    Определим давление пара, если бы вся вода испарилась (что не факт):
    $$P_\text{ max } = \frac{mRT'}{\mu V} = \frac{20\,\text{г} \cdot 8{,}31\,\frac{\text{Дж}}{\text{моль}\cdot\text{К}} \cdot 363\,\text{К}}{18\,\frac{\text{г}}{\text{моль}} \cdot 10\,\text{л}} = 335\,\text{кПа}.$$
    Сравниваем это давление с давлением насыщенного пара при этой температуре $P_\text{ н.
    п.
    90 $\celsius$ } = 70\,\text{кПа}$:
     если у нас получилось меньше табличного значения,
    то вся вода испарилась, если же больше — испарилась лишь часть, а пар является насыщенным.
    Отсюда сразу получаем давление пара: $P'_\text{ пара } = 70{,}1$.
    Сравните этот результат с первым вариантом решения.

    Тут получаем ответ: $P'_\text{ пара } = 194{,}0\,\text{кПа}$.
}
\solutionspace{150pt}

\tasknumber{4}%
\task{%
    Напротив физических величин запишите определение, обозначение и единицы измерения в системе СИ (если есть):
    \begin{enumerate}
        \item относительная влажность,
        \item насыщенный пар.
    \end{enumerate}
}

\variantsplitter

\addpersonalvariant{Евгений Васин}

\tasknumber{1}%
\task{%
    Сколько молекул водяного пара содержится в сосуде объёмом $6\,\text{л}$ при температуре $40\celsius$,
    и влажности воздуха $75\%$?
}
\solutionspace{160pt}

\tasknumber{2}%
\task{%
    В герметичном сосуде находится влажный воздух при температуре $25\celsius$ и относительной влажности $45\%$.
    \begin{enumerate}
        \item Чему равно парциальное давление насыщенного водяного пара при этой температуре?
        \item Чему равно парциальное давление водяного пара?
        \item Определите точку росы этого пара?
        \item Каким станет парциальное давление водяного пара, если сосуд нагреть до  $70\celsius$?
        \item Чему будет равна относительная влажность воздуха, если сосуд нагреть до $70\celsius$?
        \item Получите ответ на предыдущий вопрос, используя плотности, а не давления.
    \end{enumerate}
}
\solutionspace{200pt}

\tasknumber{3}%
\task{%
    Закрытый сосуд объёмом $10\,\text{л}$ заполнен сухим воздухом при давлении $100\,\text{кПа}$ и температуре $30\celsius$.
    Каким станет давление в сосуде, если в него налить $5\,\text{г}$ воды и нагреть содержимое сосуда до $80\celsius$?
}
\answer{%
    Конечное давление газа в сосуде складывается по закону Дальтона из давления нагретого сухого воздуха $P'_\text{ воздуха }$ и
    давления насыщенного пара $P_\text{ пара }$:
    $$P' = P'_\text{ воздуха } + P_\text{ пара }.$$

    Сперва определим новое давление сухого воздуха из уравнения состояния идеального газа:
    $$\frac{P'_\text{ воздуха } \cdot V}{T'} = \nu R = \frac{P \cdot V}{T} \implies P'_\text{ воздуха } = P \cdot \frac{T'}{T}.$$

    Чтобы найти давление пара, нужно понять, будет ли он насыщенным после нагрева или нет.

    Плотность насыщенного пара при температуре равна $293\,\frac{\text{г}}{\text{м}^{3}}$, тогда для того,
    чтобы весь сосуд был заполнен насыщенным водяным паром нужно
    $m_\text{ н.
    п.
    } = \rho_\text{ н.
    п.
    80 $\celsius$ } \cdot V = 293\,\frac{\text{г}}{\text{м}^{3}} \cdot 10\,\text{л} = 2{,}9\,\text{г}$ воды.
    Сравнивая эту массу с массой воды из условия, получаем массу жидкости, которая испарится: $2{,}9\,\text{г}$.
    Осталось определить давление этого пара:
    $$P_\text{ пара } = \frac{m_\text{ пара }RT}{\mu V} = \frac{2{,}9\,\text{г} \cdot 8{,}31\,\frac{\text{Дж}}{\text{моль}\cdot\text{К}} \cdot 353\,\text{К}}{18\,\frac{\text{г}}{\text{моль}} \cdot 10\,\text{л}} \approx 47\,\text{кПа}.$$

    Получаем ответ: $P'_\text{ пара } = 164{,}3\,\text{кПа}$.

    Другой вариант решения для давления пара:
    Определим давление пара, если бы вся вода испарилась (что не факт):
    $$P_\text{ max } = \frac{mRT'}{\mu V} = \frac{5\,\text{г} \cdot 8{,}31\,\frac{\text{Дж}}{\text{моль}\cdot\text{К}} \cdot 353\,\text{К}}{18\,\frac{\text{г}}{\text{моль}} \cdot 10\,\text{л}} = 81\,\text{кПа}.$$
    Сравниваем это давление с давлением насыщенного пара при этой температуре $P_\text{ н.
    п.
    80 $\celsius$ } = 47\,\text{кПа}$:
     если у нас получилось меньше табличного значения,
    то вся вода испарилась, если же больше — испарилась лишь часть, а пар является насыщенным.
    Отсюда сразу получаем давление пара: $P'_\text{ пара } = 47{,}3$.
    Сравните этот результат с первым вариантом решения.

    Тут получаем ответ: $P'_\text{ пара } = 163{,}8\,\text{кПа}$.
}
\solutionspace{150pt}

\tasknumber{4}%
\task{%
    Напротив физических величин запишите определение, обозначение и единицы измерения в системе СИ (если есть):
    \begin{enumerate}
        \item абсолютная влажность,
        \item динамическое равновесие.
    \end{enumerate}
}

\variantsplitter

\addpersonalvariant{Вячеслав Волохов}

\tasknumber{1}%
\task{%
    Сколько молекул водяного пара содержится в сосуде объёмом $12\,\text{л}$ при температуре $40\celsius$,
    и влажности воздуха $75\%$?
}
\solutionspace{160pt}

\tasknumber{2}%
\task{%
    В герметичном сосуде находится влажный воздух при температуре $15\celsius$ и относительной влажности $65\%$.
    \begin{enumerate}
        \item Чему равно парциальное давление насыщенного водяного пара при этой температуре?
        \item Чему равно парциальное давление водяного пара?
        \item Определите точку росы этого пара?
        \item Каким станет парциальное давление водяного пара, если сосуд нагреть до  $90\celsius$?
        \item Чему будет равна относительная влажность воздуха, если сосуд нагреть до $90\celsius$?
        \item Получите ответ на предыдущий вопрос, используя плотности, а не давления.
    \end{enumerate}
}
\solutionspace{200pt}

\tasknumber{3}%
\task{%
    Закрытый сосуд объёмом $10\,\text{л}$ заполнен сухим воздухом при давлении $100\,\text{кПа}$ и температуре $20\celsius$.
    Каким станет давление в сосуде, если в него налить $5\,\text{г}$ воды и нагреть содержимое сосуда до $80\celsius$?
}
\answer{%
    Конечное давление газа в сосуде складывается по закону Дальтона из давления нагретого сухого воздуха $P'_\text{ воздуха }$ и
    давления насыщенного пара $P_\text{ пара }$:
    $$P' = P'_\text{ воздуха } + P_\text{ пара }.$$

    Сперва определим новое давление сухого воздуха из уравнения состояния идеального газа:
    $$\frac{P'_\text{ воздуха } \cdot V}{T'} = \nu R = \frac{P \cdot V}{T} \implies P'_\text{ воздуха } = P \cdot \frac{T'}{T}.$$

    Чтобы найти давление пара, нужно понять, будет ли он насыщенным после нагрева или нет.

    Плотность насыщенного пара при температуре равна $293\,\frac{\text{г}}{\text{м}^{3}}$, тогда для того,
    чтобы весь сосуд был заполнен насыщенным водяным паром нужно
    $m_\text{ н.
    п.
    } = \rho_\text{ н.
    п.
    80 $\celsius$ } \cdot V = 293\,\frac{\text{г}}{\text{м}^{3}} \cdot 10\,\text{л} = 2{,}9\,\text{г}$ воды.
    Сравнивая эту массу с массой воды из условия, получаем массу жидкости, которая испарится: $2{,}9\,\text{г}$.
    Осталось определить давление этого пара:
    $$P_\text{ пара } = \frac{m_\text{ пара }RT}{\mu V} = \frac{2{,}9\,\text{г} \cdot 8{,}31\,\frac{\text{Дж}}{\text{моль}\cdot\text{К}} \cdot 353\,\text{К}}{18\,\frac{\text{г}}{\text{моль}} \cdot 10\,\text{л}} \approx 47\,\text{кПа}.$$

    Получаем ответ: $P'_\text{ пара } = 168{,}2\,\text{кПа}$.

    Другой вариант решения для давления пара:
    Определим давление пара, если бы вся вода испарилась (что не факт):
    $$P_\text{ max } = \frac{mRT'}{\mu V} = \frac{5\,\text{г} \cdot 8{,}31\,\frac{\text{Дж}}{\text{моль}\cdot\text{К}} \cdot 353\,\text{К}}{18\,\frac{\text{г}}{\text{моль}} \cdot 10\,\text{л}} = 81\,\text{кПа}.$$
    Сравниваем это давление с давлением насыщенного пара при этой температуре $P_\text{ н.
    п.
    80 $\celsius$ } = 47\,\text{кПа}$:
     если у нас получилось меньше табличного значения,
    то вся вода испарилась, если же больше — испарилась лишь часть, а пар является насыщенным.
    Отсюда сразу получаем давление пара: $P'_\text{ пара } = 47{,}3$.
    Сравните этот результат с первым вариантом решения.

    Тут получаем ответ: $P'_\text{ пара } = 167{,}8\,\text{кПа}$.
}
\solutionspace{150pt}

\tasknumber{4}%
\task{%
    Напротив физических величин запишите определение, обозначение и единицы измерения в системе СИ (если есть):
    \begin{enumerate}
        \item абсолютная влажность,
        \item насыщенный пар.
    \end{enumerate}
}

\variantsplitter

\addpersonalvariant{Герман Говоров}

\tasknumber{1}%
\task{%
    Сколько молекул водяного пара содержится в сосуде объёмом $12\,\text{л}$ при температуре $20\celsius$,
    и влажности воздуха $55\%$?
}
\solutionspace{160pt}

\tasknumber{2}%
\task{%
    В герметичном сосуде находится влажный воздух при температуре $30\celsius$ и относительной влажности $50\%$.
    \begin{enumerate}
        \item Чему равно парциальное давление насыщенного водяного пара при этой температуре?
        \item Чему равно парциальное давление водяного пара?
        \item Определите точку росы этого пара?
        \item Каким станет парциальное давление водяного пара, если сосуд нагреть до  $80\celsius$?
        \item Чему будет равна относительная влажность воздуха, если сосуд нагреть до $80\celsius$?
        \item Получите ответ на предыдущий вопрос, используя плотности, а не давления.
    \end{enumerate}
}
\solutionspace{200pt}

\tasknumber{3}%
\task{%
    Закрытый сосуд объёмом $10\,\text{л}$ заполнен сухим воздухом при давлении $100\,\text{кПа}$ и температуре $20\celsius$.
    Каким станет давление в сосуде, если в него налить $5\,\text{г}$ воды и нагреть содержимое сосуда до $90\celsius$?
}
\answer{%
    Конечное давление газа в сосуде складывается по закону Дальтона из давления нагретого сухого воздуха $P'_\text{ воздуха }$ и
    давления насыщенного пара $P_\text{ пара }$:
    $$P' = P'_\text{ воздуха } + P_\text{ пара }.$$

    Сперва определим новое давление сухого воздуха из уравнения состояния идеального газа:
    $$\frac{P'_\text{ воздуха } \cdot V}{T'} = \nu R = \frac{P \cdot V}{T} \implies P'_\text{ воздуха } = P \cdot \frac{T'}{T}.$$

    Чтобы найти давление пара, нужно понять, будет ли он насыщенным после нагрева или нет.

    Плотность насыщенного пара при температуре равна $424\,\frac{\text{г}}{\text{м}^{3}}$, тогда для того,
    чтобы весь сосуд был заполнен насыщенным водяным паром нужно
    $m_\text{ н.
    п.
    } = \rho_\text{ н.
    п.
    90 $\celsius$ } \cdot V = 424\,\frac{\text{г}}{\text{м}^{3}} \cdot 10\,\text{л} = 4{,}2\,\text{г}$ воды.
    Сравнивая эту массу с массой воды из условия, получаем массу жидкости, которая испарится: $4{,}2\,\text{г}$.
    Осталось определить давление этого пара:
    $$P_\text{ пара } = \frac{m_\text{ пара }RT}{\mu V} = \frac{4{,}2\,\text{г} \cdot 8{,}31\,\frac{\text{Дж}}{\text{моль}\cdot\text{К}} \cdot 363\,\text{К}}{18\,\frac{\text{г}}{\text{моль}} \cdot 10\,\text{л}} \approx 71\,\text{кПа}.$$

    Получаем ответ: $P'_\text{ пара } = 194{,}9\,\text{кПа}$.

    Другой вариант решения для давления пара:
    Определим давление пара, если бы вся вода испарилась (что не факт):
    $$P_\text{ max } = \frac{mRT'}{\mu V} = \frac{5\,\text{г} \cdot 8{,}31\,\frac{\text{Дж}}{\text{моль}\cdot\text{К}} \cdot 363\,\text{К}}{18\,\frac{\text{г}}{\text{моль}} \cdot 10\,\text{л}} = 83\,\text{кПа}.$$
    Сравниваем это давление с давлением насыщенного пара при этой температуре $P_\text{ н.
    п.
    90 $\celsius$ } = 70\,\text{кПа}$:
     если у нас получилось меньше табличного значения,
    то вся вода испарилась, если же больше — испарилась лишь часть, а пар является насыщенным.
    Отсюда сразу получаем давление пара: $P'_\text{ пара } = 70{,}1$.
    Сравните этот результат с первым вариантом решения.

    Тут получаем ответ: $P'_\text{ пара } = 194{,}0\,\text{кПа}$.
}
\solutionspace{150pt}

\tasknumber{4}%
\task{%
    Напротив физических величин запишите определение, обозначение и единицы измерения в системе СИ (если есть):
    \begin{enumerate}
        \item относительная влажность,
        \item насыщенный пар.
    \end{enumerate}
}

\variantsplitter

\addpersonalvariant{София Журавлёва}

\tasknumber{1}%
\task{%
    Сколько молекул водяного пара содержится в сосуде объёмом $12\,\text{л}$ при температуре $40\celsius$,
    и влажности воздуха $30\%$?
}
\solutionspace{160pt}

\tasknumber{2}%
\task{%
    В герметичном сосуде находится влажный воздух при температуре $25\celsius$ и относительной влажности $75\%$.
    \begin{enumerate}
        \item Чему равно парциальное давление насыщенного водяного пара при этой температуре?
        \item Чему равно парциальное давление водяного пара?
        \item Определите точку росы этого пара?
        \item Каким станет парциальное давление водяного пара, если сосуд нагреть до  $70\celsius$?
        \item Чему будет равна относительная влажность воздуха, если сосуд нагреть до $70\celsius$?
        \item Получите ответ на предыдущий вопрос, используя плотности, а не давления.
    \end{enumerate}
}
\solutionspace{200pt}

\tasknumber{3}%
\task{%
    Закрытый сосуд объёмом $15\,\text{л}$ заполнен сухим воздухом при давлении $100\,\text{кПа}$ и температуре $30\celsius$.
    Каким станет давление в сосуде, если в него налить $5\,\text{г}$ воды и нагреть содержимое сосуда до $100\celsius$?
}
\answer{%
    Конечное давление газа в сосуде складывается по закону Дальтона из давления нагретого сухого воздуха $P'_\text{ воздуха }$ и
    давления насыщенного пара $P_\text{ пара }$:
    $$P' = P'_\text{ воздуха } + P_\text{ пара }.$$

    Сперва определим новое давление сухого воздуха из уравнения состояния идеального газа:
    $$\frac{P'_\text{ воздуха } \cdot V}{T'} = \nu R = \frac{P \cdot V}{T} \implies P'_\text{ воздуха } = P \cdot \frac{T'}{T}.$$

    Чтобы найти давление пара, нужно понять, будет ли он насыщенным после нагрева или нет.

    Плотность насыщенного пара при температуре равна $598\,\frac{\text{г}}{\text{м}^{3}}$, тогда для того,
    чтобы весь сосуд был заполнен насыщенным водяным паром нужно
    $m_\text{ н.
    п.
    } = \rho_\text{ н.
    п.
    100 $\celsius$ } \cdot V = 598\,\frac{\text{г}}{\text{м}^{3}} \cdot 15\,\text{л} = 9{,}0\,\text{г}$ воды.
    Сравнивая эту массу с массой воды из условия, получаем массу жидкости, которая испарится: $5{,}0\,\text{г}$.
    Осталось определить давление этого пара:
    $$P_\text{ пара } = \frac{m_\text{ пара }RT}{\mu V} = \frac{5{,}0\,\text{г} \cdot 8{,}31\,\frac{\text{Дж}}{\text{моль}\cdot\text{К}} \cdot 373\,\text{К}}{18\,\frac{\text{г}}{\text{моль}} \cdot 15\,\text{л}} \approx 57\,\text{кПа}.$$

    Получаем ответ: $P'_\text{ пара } = 180{,}5\,\text{кПа}$.

    Другой вариант решения для давления пара:
    Определим давление пара, если бы вся вода испарилась (что не факт):
    $$P_\text{ max } = \frac{mRT'}{\mu V} = \frac{5\,\text{г} \cdot 8{,}31\,\frac{\text{Дж}}{\text{моль}\cdot\text{К}} \cdot 373\,\text{К}}{18\,\frac{\text{г}}{\text{моль}} \cdot 15\,\text{л}} = 57\,\text{кПа}.$$
    Сравниваем это давление с давлением насыщенного пара при этой температуре $P_\text{ н.
    п.
    100 $\celsius$ } = 101\,\text{кПа}$:
     если у нас получилось меньше табличного значения,
    то вся вода испарилась, если же больше — испарилась лишь часть, а пар является насыщенным.
    Отсюда сразу получаем давление пара: $P'_\text{ пара } = 57{,}4$.
    Сравните этот результат с первым вариантом решения.

    Тут получаем ответ: $P'_\text{ пара } = 180{,}5\,\text{кПа}$.
}
\solutionspace{150pt}

\tasknumber{4}%
\task{%
    Напротив физических величин запишите определение, обозначение и единицы измерения в системе СИ (если есть):
    \begin{enumerate}
        \item относительная влажность,
        \item динамическое равновесие.
    \end{enumerate}
}

\variantsplitter

\addpersonalvariant{Константин Козлов}

\tasknumber{1}%
\task{%
    Сколько молекул водяного пара содержится в сосуде объёмом $12\,\text{л}$ при температуре $70\celsius$,
    и влажности воздуха $55\%$?
}
\solutionspace{160pt}

\tasknumber{2}%
\task{%
    В герметичном сосуде находится влажный воздух при температуре $40\celsius$ и относительной влажности $75\%$.
    \begin{enumerate}
        \item Чему равно парциальное давление насыщенного водяного пара при этой температуре?
        \item Чему равно парциальное давление водяного пара?
        \item Определите точку росы этого пара?
        \item Каким станет парциальное давление водяного пара, если сосуд нагреть до  $80\celsius$?
        \item Чему будет равна относительная влажность воздуха, если сосуд нагреть до $80\celsius$?
        \item Получите ответ на предыдущий вопрос, используя плотности, а не давления.
    \end{enumerate}
}
\solutionspace{200pt}

\tasknumber{3}%
\task{%
    Закрытый сосуд объёмом $10\,\text{л}$ заполнен сухим воздухом при давлении $100\,\text{кПа}$ и температуре $20\celsius$.
    Каким станет давление в сосуде, если в него налить $30\,\text{г}$ воды и нагреть содержимое сосуда до $100\celsius$?
}
\answer{%
    Конечное давление газа в сосуде складывается по закону Дальтона из давления нагретого сухого воздуха $P'_\text{ воздуха }$ и
    давления насыщенного пара $P_\text{ пара }$:
    $$P' = P'_\text{ воздуха } + P_\text{ пара }.$$

    Сперва определим новое давление сухого воздуха из уравнения состояния идеального газа:
    $$\frac{P'_\text{ воздуха } \cdot V}{T'} = \nu R = \frac{P \cdot V}{T} \implies P'_\text{ воздуха } = P \cdot \frac{T'}{T}.$$

    Чтобы найти давление пара, нужно понять, будет ли он насыщенным после нагрева или нет.

    Плотность насыщенного пара при температуре равна $598\,\frac{\text{г}}{\text{м}^{3}}$, тогда для того,
    чтобы весь сосуд был заполнен насыщенным водяным паром нужно
    $m_\text{ н.
    п.
    } = \rho_\text{ н.
    п.
    100 $\celsius$ } \cdot V = 598\,\frac{\text{г}}{\text{м}^{3}} \cdot 10\,\text{л} = 6{,}0\,\text{г}$ воды.
    Сравнивая эту массу с массой воды из условия, получаем массу жидкости, которая испарится: $6{,}0\,\text{г}$.
    Осталось определить давление этого пара:
    $$P_\text{ пара } = \frac{m_\text{ пара }RT}{\mu V} = \frac{6{,}0\,\text{г} \cdot 8{,}31\,\frac{\text{Дж}}{\text{моль}\cdot\text{К}} \cdot 373\,\text{К}}{18\,\frac{\text{г}}{\text{моль}} \cdot 10\,\text{л}} \approx 102\,\text{кПа}.$$

    Получаем ответ: $P'_\text{ пара } = 230{,}3\,\text{кПа}$.

    Другой вариант решения для давления пара:
    Определим давление пара, если бы вся вода испарилась (что не факт):
    $$P_\text{ max } = \frac{mRT'}{\mu V} = \frac{30\,\text{г} \cdot 8{,}31\,\frac{\text{Дж}}{\text{моль}\cdot\text{К}} \cdot 373\,\text{К}}{18\,\frac{\text{г}}{\text{моль}} \cdot 10\,\text{л}} = 516\,\text{кПа}.$$
    Сравниваем это давление с давлением насыщенного пара при этой температуре $P_\text{ н.
    п.
    100 $\celsius$ } = 101\,\text{кПа}$:
     если у нас получилось меньше табличного значения,
    то вся вода испарилась, если же больше — испарилась лишь часть, а пар является насыщенным.
    Отсюда сразу получаем давление пара: $P'_\text{ пара } = 101{,}3$.
    Сравните этот результат с первым вариантом решения.

    Тут получаем ответ: $P'_\text{ пара } = 228{,}6\,\text{кПа}$.
}
\solutionspace{150pt}

\tasknumber{4}%
\task{%
    Напротив физических величин запишите определение, обозначение и единицы измерения в системе СИ (если есть):
    \begin{enumerate}
        \item абсолютная влажность,
        \item динамическое равновесие.
    \end{enumerate}
}

\variantsplitter

\addpersonalvariant{Наталья Кравченко}

\tasknumber{1}%
\task{%
    Сколько молекул водяного пара содержится в сосуде объёмом $15\,\text{л}$ при температуре $60\celsius$,
    и влажности воздуха $75\%$?
}
\solutionspace{160pt}

\tasknumber{2}%
\task{%
    В герметичном сосуде находится влажный воздух при температуре $40\celsius$ и относительной влажности $70\%$.
    \begin{enumerate}
        \item Чему равно парциальное давление насыщенного водяного пара при этой температуре?
        \item Чему равно парциальное давление водяного пара?
        \item Определите точку росы этого пара?
        \item Каким станет парциальное давление водяного пара, если сосуд нагреть до  $90\celsius$?
        \item Чему будет равна относительная влажность воздуха, если сосуд нагреть до $90\celsius$?
        \item Получите ответ на предыдущий вопрос, используя плотности, а не давления.
    \end{enumerate}
}
\solutionspace{200pt}

\tasknumber{3}%
\task{%
    Закрытый сосуд объёмом $15\,\text{л}$ заполнен сухим воздухом при давлении $100\,\text{кПа}$ и температуре $10\celsius$.
    Каким станет давление в сосуде, если в него налить $5\,\text{г}$ воды и нагреть содержимое сосуда до $80\celsius$?
}
\answer{%
    Конечное давление газа в сосуде складывается по закону Дальтона из давления нагретого сухого воздуха $P'_\text{ воздуха }$ и
    давления насыщенного пара $P_\text{ пара }$:
    $$P' = P'_\text{ воздуха } + P_\text{ пара }.$$

    Сперва определим новое давление сухого воздуха из уравнения состояния идеального газа:
    $$\frac{P'_\text{ воздуха } \cdot V}{T'} = \nu R = \frac{P \cdot V}{T} \implies P'_\text{ воздуха } = P \cdot \frac{T'}{T}.$$

    Чтобы найти давление пара, нужно понять, будет ли он насыщенным после нагрева или нет.

    Плотность насыщенного пара при температуре равна $293\,\frac{\text{г}}{\text{м}^{3}}$, тогда для того,
    чтобы весь сосуд был заполнен насыщенным водяным паром нужно
    $m_\text{ н.
    п.
    } = \rho_\text{ н.
    п.
    80 $\celsius$ } \cdot V = 293\,\frac{\text{г}}{\text{м}^{3}} \cdot 15\,\text{л} = 4{,}4\,\text{г}$ воды.
    Сравнивая эту массу с массой воды из условия, получаем массу жидкости, которая испарится: $4{,}4\,\text{г}$.
    Осталось определить давление этого пара:
    $$P_\text{ пара } = \frac{m_\text{ пара }RT}{\mu V} = \frac{4{,}4\,\text{г} \cdot 8{,}31\,\frac{\text{Дж}}{\text{моль}\cdot\text{К}} \cdot 353\,\text{К}}{18\,\frac{\text{г}}{\text{моль}} \cdot 15\,\text{л}} \approx 47\,\text{кПа}.$$

    Получаем ответ: $P'_\text{ пара } = 172{,}5\,\text{кПа}$.

    Другой вариант решения для давления пара:
    Определим давление пара, если бы вся вода испарилась (что не факт):
    $$P_\text{ max } = \frac{mRT'}{\mu V} = \frac{5\,\text{г} \cdot 8{,}31\,\frac{\text{Дж}}{\text{моль}\cdot\text{К}} \cdot 353\,\text{К}}{18\,\frac{\text{г}}{\text{моль}} \cdot 15\,\text{л}} = 54\,\text{кПа}.$$
    Сравниваем это давление с давлением насыщенного пара при этой температуре $P_\text{ н.
    п.
    80 $\celsius$ } = 47\,\text{кПа}$:
     если у нас получилось меньше табличного значения,
    то вся вода испарилась, если же больше — испарилась лишь часть, а пар является насыщенным.
    Отсюда сразу получаем давление пара: $P'_\text{ пара } = 47{,}3$.
    Сравните этот результат с первым вариантом решения.

    Тут получаем ответ: $P'_\text{ пара } = 172{,}0\,\text{кПа}$.
}
\solutionspace{150pt}

\tasknumber{4}%
\task{%
    Напротив физических величин запишите определение, обозначение и единицы измерения в системе СИ (если есть):
    \begin{enumerate}
        \item относительная влажность,
        \item насыщенный пар.
    \end{enumerate}
}

\variantsplitter

\addpersonalvariant{Матвей Кузьмин}

\tasknumber{1}%
\task{%
    Сколько молекул водяного пара содержится в сосуде объёмом $15\,\text{л}$ при температуре $20\celsius$,
    и влажности воздуха $20\%$?
}
\solutionspace{160pt}

\tasknumber{2}%
\task{%
    В герметичном сосуде находится влажный воздух при температуре $25\celsius$ и относительной влажности $45\%$.
    \begin{enumerate}
        \item Чему равно парциальное давление насыщенного водяного пара при этой температуре?
        \item Чему равно парциальное давление водяного пара?
        \item Определите точку росы этого пара?
        \item Каким станет парциальное давление водяного пара, если сосуд нагреть до  $70\celsius$?
        \item Чему будет равна относительная влажность воздуха, если сосуд нагреть до $70\celsius$?
        \item Получите ответ на предыдущий вопрос, используя плотности, а не давления.
    \end{enumerate}
}
\solutionspace{200pt}

\tasknumber{3}%
\task{%
    Закрытый сосуд объёмом $10\,\text{л}$ заполнен сухим воздухом при давлении $100\,\text{кПа}$ и температуре $30\celsius$.
    Каким станет давление в сосуде, если в него налить $5\,\text{г}$ воды и нагреть содержимое сосуда до $90\celsius$?
}
\answer{%
    Конечное давление газа в сосуде складывается по закону Дальтона из давления нагретого сухого воздуха $P'_\text{ воздуха }$ и
    давления насыщенного пара $P_\text{ пара }$:
    $$P' = P'_\text{ воздуха } + P_\text{ пара }.$$

    Сперва определим новое давление сухого воздуха из уравнения состояния идеального газа:
    $$\frac{P'_\text{ воздуха } \cdot V}{T'} = \nu R = \frac{P \cdot V}{T} \implies P'_\text{ воздуха } = P \cdot \frac{T'}{T}.$$

    Чтобы найти давление пара, нужно понять, будет ли он насыщенным после нагрева или нет.

    Плотность насыщенного пара при температуре равна $424\,\frac{\text{г}}{\text{м}^{3}}$, тогда для того,
    чтобы весь сосуд был заполнен насыщенным водяным паром нужно
    $m_\text{ н.
    п.
    } = \rho_\text{ н.
    п.
    90 $\celsius$ } \cdot V = 424\,\frac{\text{г}}{\text{м}^{3}} \cdot 10\,\text{л} = 4{,}2\,\text{г}$ воды.
    Сравнивая эту массу с массой воды из условия, получаем массу жидкости, которая испарится: $4{,}2\,\text{г}$.
    Осталось определить давление этого пара:
    $$P_\text{ пара } = \frac{m_\text{ пара }RT}{\mu V} = \frac{4{,}2\,\text{г} \cdot 8{,}31\,\frac{\text{Дж}}{\text{моль}\cdot\text{К}} \cdot 363\,\text{К}}{18\,\frac{\text{г}}{\text{моль}} \cdot 10\,\text{л}} \approx 71\,\text{кПа}.$$

    Получаем ответ: $P'_\text{ пара } = 190{,}9\,\text{кПа}$.

    Другой вариант решения для давления пара:
    Определим давление пара, если бы вся вода испарилась (что не факт):
    $$P_\text{ max } = \frac{mRT'}{\mu V} = \frac{5\,\text{г} \cdot 8{,}31\,\frac{\text{Дж}}{\text{моль}\cdot\text{К}} \cdot 363\,\text{К}}{18\,\frac{\text{г}}{\text{моль}} \cdot 10\,\text{л}} = 83\,\text{кПа}.$$
    Сравниваем это давление с давлением насыщенного пара при этой температуре $P_\text{ н.
    п.
    90 $\celsius$ } = 70\,\text{кПа}$:
     если у нас получилось меньше табличного значения,
    то вся вода испарилась, если же больше — испарилась лишь часть, а пар является насыщенным.
    Отсюда сразу получаем давление пара: $P'_\text{ пара } = 70{,}1$.
    Сравните этот результат с первым вариантом решения.

    Тут получаем ответ: $P'_\text{ пара } = 189{,}9\,\text{кПа}$.
}
\solutionspace{150pt}

\tasknumber{4}%
\task{%
    Напротив физических величин запишите определение, обозначение и единицы измерения в системе СИ (если есть):
    \begin{enumerate}
        \item относительная влажность,
        \item динамическое равновесие.
    \end{enumerate}
}

\variantsplitter

\addpersonalvariant{Сергей Малышев}

\tasknumber{1}%
\task{%
    Сколько молекул водяного пара содержится в сосуде объёмом $3\,\text{л}$ при температуре $25\celsius$,
    и влажности воздуха $75\%$?
}
\solutionspace{160pt}

\tasknumber{2}%
\task{%
    В герметичном сосуде находится влажный воздух при температуре $20\celsius$ и относительной влажности $55\%$.
    \begin{enumerate}
        \item Чему равно парциальное давление насыщенного водяного пара при этой температуре?
        \item Чему равно парциальное давление водяного пара?
        \item Определите точку росы этого пара?
        \item Каким станет парциальное давление водяного пара, если сосуд нагреть до  $80\celsius$?
        \item Чему будет равна относительная влажность воздуха, если сосуд нагреть до $80\celsius$?
        \item Получите ответ на предыдущий вопрос, используя плотности, а не давления.
    \end{enumerate}
}
\solutionspace{200pt}

\tasknumber{3}%
\task{%
    Закрытый сосуд объёмом $10\,\text{л}$ заполнен сухим воздухом при давлении $100\,\text{кПа}$ и температуре $20\celsius$.
    Каким станет давление в сосуде, если в него налить $30\,\text{г}$ воды и нагреть содержимое сосуда до $80\celsius$?
}
\answer{%
    Конечное давление газа в сосуде складывается по закону Дальтона из давления нагретого сухого воздуха $P'_\text{ воздуха }$ и
    давления насыщенного пара $P_\text{ пара }$:
    $$P' = P'_\text{ воздуха } + P_\text{ пара }.$$

    Сперва определим новое давление сухого воздуха из уравнения состояния идеального газа:
    $$\frac{P'_\text{ воздуха } \cdot V}{T'} = \nu R = \frac{P \cdot V}{T} \implies P'_\text{ воздуха } = P \cdot \frac{T'}{T}.$$

    Чтобы найти давление пара, нужно понять, будет ли он насыщенным после нагрева или нет.

    Плотность насыщенного пара при температуре равна $293\,\frac{\text{г}}{\text{м}^{3}}$, тогда для того,
    чтобы весь сосуд был заполнен насыщенным водяным паром нужно
    $m_\text{ н.
    п.
    } = \rho_\text{ н.
    п.
    80 $\celsius$ } \cdot V = 293\,\frac{\text{г}}{\text{м}^{3}} \cdot 10\,\text{л} = 2{,}9\,\text{г}$ воды.
    Сравнивая эту массу с массой воды из условия, получаем массу жидкости, которая испарится: $2{,}9\,\text{г}$.
    Осталось определить давление этого пара:
    $$P_\text{ пара } = \frac{m_\text{ пара }RT}{\mu V} = \frac{2{,}9\,\text{г} \cdot 8{,}31\,\frac{\text{Дж}}{\text{моль}\cdot\text{К}} \cdot 353\,\text{К}}{18\,\frac{\text{г}}{\text{моль}} \cdot 10\,\text{л}} \approx 47\,\text{кПа}.$$

    Получаем ответ: $P'_\text{ пара } = 168{,}2\,\text{кПа}$.

    Другой вариант решения для давления пара:
    Определим давление пара, если бы вся вода испарилась (что не факт):
    $$P_\text{ max } = \frac{mRT'}{\mu V} = \frac{30\,\text{г} \cdot 8{,}31\,\frac{\text{Дж}}{\text{моль}\cdot\text{К}} \cdot 353\,\text{К}}{18\,\frac{\text{г}}{\text{моль}} \cdot 10\,\text{л}} = 488\,\text{кПа}.$$
    Сравниваем это давление с давлением насыщенного пара при этой температуре $P_\text{ н.
    п.
    80 $\celsius$ } = 47\,\text{кПа}$:
     если у нас получилось меньше табличного значения,
    то вся вода испарилась, если же больше — испарилась лишь часть, а пар является насыщенным.
    Отсюда сразу получаем давление пара: $P'_\text{ пара } = 47{,}3$.
    Сравните этот результат с первым вариантом решения.

    Тут получаем ответ: $P'_\text{ пара } = 167{,}8\,\text{кПа}$.
}
\solutionspace{150pt}

\tasknumber{4}%
\task{%
    Напротив физических величин запишите определение, обозначение и единицы измерения в системе СИ (если есть):
    \begin{enumerate}
        \item абсолютная влажность,
        \item динамическое равновесие.
    \end{enumerate}
}

\variantsplitter

\addpersonalvariant{Алина Полканова}

\tasknumber{1}%
\task{%
    Сколько молекул водяного пара содержится в сосуде объёмом $3\,\text{л}$ при температуре $90\celsius$,
    и влажности воздуха $20\%$?
}
\solutionspace{160pt}

\tasknumber{2}%
\task{%
    В герметичном сосуде находится влажный воздух при температуре $25\celsius$ и относительной влажности $35\%$.
    \begin{enumerate}
        \item Чему равно парциальное давление насыщенного водяного пара при этой температуре?
        \item Чему равно парциальное давление водяного пара?
        \item Определите точку росы этого пара?
        \item Каким станет парциальное давление водяного пара, если сосуд нагреть до  $80\celsius$?
        \item Чему будет равна относительная влажность воздуха, если сосуд нагреть до $80\celsius$?
        \item Получите ответ на предыдущий вопрос, используя плотности, а не давления.
    \end{enumerate}
}
\solutionspace{200pt}

\tasknumber{3}%
\task{%
    Закрытый сосуд объёмом $10\,\text{л}$ заполнен сухим воздухом при давлении $100\,\text{кПа}$ и температуре $10\celsius$.
    Каким станет давление в сосуде, если в него налить $5\,\text{г}$ воды и нагреть содержимое сосуда до $80\celsius$?
}
\answer{%
    Конечное давление газа в сосуде складывается по закону Дальтона из давления нагретого сухого воздуха $P'_\text{ воздуха }$ и
    давления насыщенного пара $P_\text{ пара }$:
    $$P' = P'_\text{ воздуха } + P_\text{ пара }.$$

    Сперва определим новое давление сухого воздуха из уравнения состояния идеального газа:
    $$\frac{P'_\text{ воздуха } \cdot V}{T'} = \nu R = \frac{P \cdot V}{T} \implies P'_\text{ воздуха } = P \cdot \frac{T'}{T}.$$

    Чтобы найти давление пара, нужно понять, будет ли он насыщенным после нагрева или нет.

    Плотность насыщенного пара при температуре равна $293\,\frac{\text{г}}{\text{м}^{3}}$, тогда для того,
    чтобы весь сосуд был заполнен насыщенным водяным паром нужно
    $m_\text{ н.
    п.
    } = \rho_\text{ н.
    п.
    80 $\celsius$ } \cdot V = 293\,\frac{\text{г}}{\text{м}^{3}} \cdot 10\,\text{л} = 2{,}9\,\text{г}$ воды.
    Сравнивая эту массу с массой воды из условия, получаем массу жидкости, которая испарится: $2{,}9\,\text{г}$.
    Осталось определить давление этого пара:
    $$P_\text{ пара } = \frac{m_\text{ пара }RT}{\mu V} = \frac{2{,}9\,\text{г} \cdot 8{,}31\,\frac{\text{Дж}}{\text{моль}\cdot\text{К}} \cdot 353\,\text{К}}{18\,\frac{\text{г}}{\text{моль}} \cdot 10\,\text{л}} \approx 47\,\text{кПа}.$$

    Получаем ответ: $P'_\text{ пара } = 172{,}5\,\text{кПа}$.

    Другой вариант решения для давления пара:
    Определим давление пара, если бы вся вода испарилась (что не факт):
    $$P_\text{ max } = \frac{mRT'}{\mu V} = \frac{5\,\text{г} \cdot 8{,}31\,\frac{\text{Дж}}{\text{моль}\cdot\text{К}} \cdot 353\,\text{К}}{18\,\frac{\text{г}}{\text{моль}} \cdot 10\,\text{л}} = 81\,\text{кПа}.$$
    Сравниваем это давление с давлением насыщенного пара при этой температуре $P_\text{ н.
    п.
    80 $\celsius$ } = 47\,\text{кПа}$:
     если у нас получилось меньше табличного значения,
    то вся вода испарилась, если же больше — испарилась лишь часть, а пар является насыщенным.
    Отсюда сразу получаем давление пара: $P'_\text{ пара } = 47{,}3$.
    Сравните этот результат с первым вариантом решения.

    Тут получаем ответ: $P'_\text{ пара } = 172{,}0\,\text{кПа}$.
}
\solutionspace{150pt}

\tasknumber{4}%
\task{%
    Напротив физических величин запишите определение, обозначение и единицы измерения в системе СИ (если есть):
    \begin{enumerate}
        \item абсолютная влажность,
        \item насыщенный пар.
    \end{enumerate}
}

\variantsplitter

\addpersonalvariant{Сергей Пономарёв}

\tasknumber{1}%
\task{%
    Сколько молекул водяного пара содержится в сосуде объёмом $15\,\text{л}$ при температуре $60\celsius$,
    и влажности воздуха $75\%$?
}
\solutionspace{160pt}

\tasknumber{2}%
\task{%
    В герметичном сосуде находится влажный воздух при температуре $30\celsius$ и относительной влажности $40\%$.
    \begin{enumerate}
        \item Чему равно парциальное давление насыщенного водяного пара при этой температуре?
        \item Чему равно парциальное давление водяного пара?
        \item Определите точку росы этого пара?
        \item Каким станет парциальное давление водяного пара, если сосуд нагреть до  $90\celsius$?
        \item Чему будет равна относительная влажность воздуха, если сосуд нагреть до $90\celsius$?
        \item Получите ответ на предыдущий вопрос, используя плотности, а не давления.
    \end{enumerate}
}
\solutionspace{200pt}

\tasknumber{3}%
\task{%
    Закрытый сосуд объёмом $10\,\text{л}$ заполнен сухим воздухом при давлении $100\,\text{кПа}$ и температуре $20\celsius$.
    Каким станет давление в сосуде, если в него налить $10\,\text{г}$ воды и нагреть содержимое сосуда до $80\celsius$?
}
\answer{%
    Конечное давление газа в сосуде складывается по закону Дальтона из давления нагретого сухого воздуха $P'_\text{ воздуха }$ и
    давления насыщенного пара $P_\text{ пара }$:
    $$P' = P'_\text{ воздуха } + P_\text{ пара }.$$

    Сперва определим новое давление сухого воздуха из уравнения состояния идеального газа:
    $$\frac{P'_\text{ воздуха } \cdot V}{T'} = \nu R = \frac{P \cdot V}{T} \implies P'_\text{ воздуха } = P \cdot \frac{T'}{T}.$$

    Чтобы найти давление пара, нужно понять, будет ли он насыщенным после нагрева или нет.

    Плотность насыщенного пара при температуре равна $293\,\frac{\text{г}}{\text{м}^{3}}$, тогда для того,
    чтобы весь сосуд был заполнен насыщенным водяным паром нужно
    $m_\text{ н.
    п.
    } = \rho_\text{ н.
    п.
    80 $\celsius$ } \cdot V = 293\,\frac{\text{г}}{\text{м}^{3}} \cdot 10\,\text{л} = 2{,}9\,\text{г}$ воды.
    Сравнивая эту массу с массой воды из условия, получаем массу жидкости, которая испарится: $2{,}9\,\text{г}$.
    Осталось определить давление этого пара:
    $$P_\text{ пара } = \frac{m_\text{ пара }RT}{\mu V} = \frac{2{,}9\,\text{г} \cdot 8{,}31\,\frac{\text{Дж}}{\text{моль}\cdot\text{К}} \cdot 353\,\text{К}}{18\,\frac{\text{г}}{\text{моль}} \cdot 10\,\text{л}} \approx 47\,\text{кПа}.$$

    Получаем ответ: $P'_\text{ пара } = 168{,}2\,\text{кПа}$.

    Другой вариант решения для давления пара:
    Определим давление пара, если бы вся вода испарилась (что не факт):
    $$P_\text{ max } = \frac{mRT'}{\mu V} = \frac{10\,\text{г} \cdot 8{,}31\,\frac{\text{Дж}}{\text{моль}\cdot\text{К}} \cdot 353\,\text{К}}{18\,\frac{\text{г}}{\text{моль}} \cdot 10\,\text{л}} = 162\,\text{кПа}.$$
    Сравниваем это давление с давлением насыщенного пара при этой температуре $P_\text{ н.
    п.
    80 $\celsius$ } = 47\,\text{кПа}$:
     если у нас получилось меньше табличного значения,
    то вся вода испарилась, если же больше — испарилась лишь часть, а пар является насыщенным.
    Отсюда сразу получаем давление пара: $P'_\text{ пара } = 47{,}3$.
    Сравните этот результат с первым вариантом решения.

    Тут получаем ответ: $P'_\text{ пара } = 167{,}8\,\text{кПа}$.
}
\solutionspace{150pt}

\tasknumber{4}%
\task{%
    Напротив физических величин запишите определение, обозначение и единицы измерения в системе СИ (если есть):
    \begin{enumerate}
        \item относительная влажность,
        \item насыщенный пар.
    \end{enumerate}
}

\variantsplitter

\addpersonalvariant{Егор Свистушкин}

\tasknumber{1}%
\task{%
    Сколько молекул водяного пара содержится в сосуде объёмом $12\,\text{л}$ при температуре $80\celsius$,
    и влажности воздуха $25\%$?
}
\solutionspace{160pt}

\tasknumber{2}%
\task{%
    В герметичном сосуде находится влажный воздух при температуре $25\celsius$ и относительной влажности $45\%$.
    \begin{enumerate}
        \item Чему равно парциальное давление насыщенного водяного пара при этой температуре?
        \item Чему равно парциальное давление водяного пара?
        \item Определите точку росы этого пара?
        \item Каким станет парциальное давление водяного пара, если сосуд нагреть до  $90\celsius$?
        \item Чему будет равна относительная влажность воздуха, если сосуд нагреть до $90\celsius$?
        \item Получите ответ на предыдущий вопрос, используя плотности, а не давления.
    \end{enumerate}
}
\solutionspace{200pt}

\tasknumber{3}%
\task{%
    Закрытый сосуд объёмом $15\,\text{л}$ заполнен сухим воздухом при давлении $100\,\text{кПа}$ и температуре $10\celsius$.
    Каким станет давление в сосуде, если в него налить $30\,\text{г}$ воды и нагреть содержимое сосуда до $100\celsius$?
}
\answer{%
    Конечное давление газа в сосуде складывается по закону Дальтона из давления нагретого сухого воздуха $P'_\text{ воздуха }$ и
    давления насыщенного пара $P_\text{ пара }$:
    $$P' = P'_\text{ воздуха } + P_\text{ пара }.$$

    Сперва определим новое давление сухого воздуха из уравнения состояния идеального газа:
    $$\frac{P'_\text{ воздуха } \cdot V}{T'} = \nu R = \frac{P \cdot V}{T} \implies P'_\text{ воздуха } = P \cdot \frac{T'}{T}.$$

    Чтобы найти давление пара, нужно понять, будет ли он насыщенным после нагрева или нет.

    Плотность насыщенного пара при температуре равна $598\,\frac{\text{г}}{\text{м}^{3}}$, тогда для того,
    чтобы весь сосуд был заполнен насыщенным водяным паром нужно
    $m_\text{ н.
    п.
    } = \rho_\text{ н.
    п.
    100 $\celsius$ } \cdot V = 598\,\frac{\text{г}}{\text{м}^{3}} \cdot 15\,\text{л} = 9{,}0\,\text{г}$ воды.
    Сравнивая эту массу с массой воды из условия, получаем массу жидкости, которая испарится: $9{,}0\,\text{г}$.
    Осталось определить давление этого пара:
    $$P_\text{ пара } = \frac{m_\text{ пара }RT}{\mu V} = \frac{9{,}0\,\text{г} \cdot 8{,}31\,\frac{\text{Дж}}{\text{моль}\cdot\text{К}} \cdot 373\,\text{К}}{18\,\frac{\text{г}}{\text{моль}} \cdot 15\,\text{л}} \approx 102\,\text{кПа}.$$

    Получаем ответ: $P'_\text{ пара } = 234{,}8\,\text{кПа}$.

    Другой вариант решения для давления пара:
    Определим давление пара, если бы вся вода испарилась (что не факт):
    $$P_\text{ max } = \frac{mRT'}{\mu V} = \frac{30\,\text{г} \cdot 8{,}31\,\frac{\text{Дж}}{\text{моль}\cdot\text{К}} \cdot 373\,\text{К}}{18\,\frac{\text{г}}{\text{моль}} \cdot 15\,\text{л}} = 344\,\text{кПа}.$$
    Сравниваем это давление с давлением насыщенного пара при этой температуре $P_\text{ н.
    п.
    100 $\celsius$ } = 101\,\text{кПа}$:
     если у нас получилось меньше табличного значения,
    то вся вода испарилась, если же больше — испарилась лишь часть, а пар является насыщенным.
    Отсюда сразу получаем давление пара: $P'_\text{ пара } = 101{,}3$.
    Сравните этот результат с первым вариантом решения.

    Тут получаем ответ: $P'_\text{ пара } = 233{,}1\,\text{кПа}$.
}
\solutionspace{150pt}

\tasknumber{4}%
\task{%
    Напротив физических величин запишите определение, обозначение и единицы измерения в системе СИ (если есть):
    \begin{enumerate}
        \item абсолютная влажность,
        \item насыщенный пар.
    \end{enumerate}
}

\variantsplitter

\addpersonalvariant{Дмитрий Соколов}

\tasknumber{1}%
\task{%
    Сколько молекул водяного пара содержится в сосуде объёмом $15\,\text{л}$ при температуре $25\celsius$,
    и влажности воздуха $55\%$?
}
\solutionspace{160pt}

\tasknumber{2}%
\task{%
    В герметичном сосуде находится влажный воздух при температуре $40\celsius$ и относительной влажности $45\%$.
    \begin{enumerate}
        \item Чему равно парциальное давление насыщенного водяного пара при этой температуре?
        \item Чему равно парциальное давление водяного пара?
        \item Определите точку росы этого пара?
        \item Каким станет парциальное давление водяного пара, если сосуд нагреть до  $90\celsius$?
        \item Чему будет равна относительная влажность воздуха, если сосуд нагреть до $90\celsius$?
        \item Получите ответ на предыдущий вопрос, используя плотности, а не давления.
    \end{enumerate}
}
\solutionspace{200pt}

\tasknumber{3}%
\task{%
    Закрытый сосуд объёмом $20\,\text{л}$ заполнен сухим воздухом при давлении $100\,\text{кПа}$ и температуре $20\celsius$.
    Каким станет давление в сосуде, если в него налить $10\,\text{г}$ воды и нагреть содержимое сосуда до $90\celsius$?
}
\answer{%
    Конечное давление газа в сосуде складывается по закону Дальтона из давления нагретого сухого воздуха $P'_\text{ воздуха }$ и
    давления насыщенного пара $P_\text{ пара }$:
    $$P' = P'_\text{ воздуха } + P_\text{ пара }.$$

    Сперва определим новое давление сухого воздуха из уравнения состояния идеального газа:
    $$\frac{P'_\text{ воздуха } \cdot V}{T'} = \nu R = \frac{P \cdot V}{T} \implies P'_\text{ воздуха } = P \cdot \frac{T'}{T}.$$

    Чтобы найти давление пара, нужно понять, будет ли он насыщенным после нагрева или нет.

    Плотность насыщенного пара при температуре равна $424\,\frac{\text{г}}{\text{м}^{3}}$, тогда для того,
    чтобы весь сосуд был заполнен насыщенным водяным паром нужно
    $m_\text{ н.
    п.
    } = \rho_\text{ н.
    п.
    90 $\celsius$ } \cdot V = 424\,\frac{\text{г}}{\text{м}^{3}} \cdot 20\,\text{л} = 8{,}5\,\text{г}$ воды.
    Сравнивая эту массу с массой воды из условия, получаем массу жидкости, которая испарится: $8{,}5\,\text{г}$.
    Осталось определить давление этого пара:
    $$P_\text{ пара } = \frac{m_\text{ пара }RT}{\mu V} = \frac{8{,}5\,\text{г} \cdot 8{,}31\,\frac{\text{Дж}}{\text{моль}\cdot\text{К}} \cdot 363\,\text{К}}{18\,\frac{\text{г}}{\text{моль}} \cdot 20\,\text{л}} \approx 71\,\text{кПа}.$$

    Получаем ответ: $P'_\text{ пара } = 194{,}9\,\text{кПа}$.

    Другой вариант решения для давления пара:
    Определим давление пара, если бы вся вода испарилась (что не факт):
    $$P_\text{ max } = \frac{mRT'}{\mu V} = \frac{10\,\text{г} \cdot 8{,}31\,\frac{\text{Дж}}{\text{моль}\cdot\text{К}} \cdot 363\,\text{К}}{18\,\frac{\text{г}}{\text{моль}} \cdot 20\,\text{л}} = 83\,\text{кПа}.$$
    Сравниваем это давление с давлением насыщенного пара при этой температуре $P_\text{ н.
    п.
    90 $\celsius$ } = 70\,\text{кПа}$:
     если у нас получилось меньше табличного значения,
    то вся вода испарилась, если же больше — испарилась лишь часть, а пар является насыщенным.
    Отсюда сразу получаем давление пара: $P'_\text{ пара } = 70{,}1$.
    Сравните этот результат с первым вариантом решения.

    Тут получаем ответ: $P'_\text{ пара } = 194{,}0\,\text{кПа}$.
}
\solutionspace{150pt}

\tasknumber{4}%
\task{%
    Напротив физических величин запишите определение, обозначение и единицы измерения в системе СИ (если есть):
    \begin{enumerate}
        \item относительная влажность,
        \item динамическое равновесие.
    \end{enumerate}
}

\variantsplitter

\addpersonalvariant{Арсений Трофимов}

\tasknumber{1}%
\task{%
    Сколько молекул водяного пара содержится в сосуде объёмом $9\,\text{л}$ при температуре $100\celsius$,
    и влажности воздуха $30\%$?
}
\solutionspace{160pt}

\tasknumber{2}%
\task{%
    В герметичном сосуде находится влажный воздух при температуре $25\celsius$ и относительной влажности $45\%$.
    \begin{enumerate}
        \item Чему равно парциальное давление насыщенного водяного пара при этой температуре?
        \item Чему равно парциальное давление водяного пара?
        \item Определите точку росы этого пара?
        \item Каким станет парциальное давление водяного пара, если сосуд нагреть до  $70\celsius$?
        \item Чему будет равна относительная влажность воздуха, если сосуд нагреть до $70\celsius$?
        \item Получите ответ на предыдущий вопрос, используя плотности, а не давления.
    \end{enumerate}
}
\solutionspace{200pt}

\tasknumber{3}%
\task{%
    Закрытый сосуд объёмом $15\,\text{л}$ заполнен сухим воздухом при давлении $100\,\text{кПа}$ и температуре $20\celsius$.
    Каким станет давление в сосуде, если в него налить $10\,\text{г}$ воды и нагреть содержимое сосуда до $100\celsius$?
}
\answer{%
    Конечное давление газа в сосуде складывается по закону Дальтона из давления нагретого сухого воздуха $P'_\text{ воздуха }$ и
    давления насыщенного пара $P_\text{ пара }$:
    $$P' = P'_\text{ воздуха } + P_\text{ пара }.$$

    Сперва определим новое давление сухого воздуха из уравнения состояния идеального газа:
    $$\frac{P'_\text{ воздуха } \cdot V}{T'} = \nu R = \frac{P \cdot V}{T} \implies P'_\text{ воздуха } = P \cdot \frac{T'}{T}.$$

    Чтобы найти давление пара, нужно понять, будет ли он насыщенным после нагрева или нет.

    Плотность насыщенного пара при температуре равна $598\,\frac{\text{г}}{\text{м}^{3}}$, тогда для того,
    чтобы весь сосуд был заполнен насыщенным водяным паром нужно
    $m_\text{ н.
    п.
    } = \rho_\text{ н.
    п.
    100 $\celsius$ } \cdot V = 598\,\frac{\text{г}}{\text{м}^{3}} \cdot 15\,\text{л} = 9{,}0\,\text{г}$ воды.
    Сравнивая эту массу с массой воды из условия, получаем массу жидкости, которая испарится: $9{,}0\,\text{г}$.
    Осталось определить давление этого пара:
    $$P_\text{ пара } = \frac{m_\text{ пара }RT}{\mu V} = \frac{9{,}0\,\text{г} \cdot 8{,}31\,\frac{\text{Дж}}{\text{моль}\cdot\text{К}} \cdot 373\,\text{К}}{18\,\frac{\text{г}}{\text{моль}} \cdot 15\,\text{л}} \approx 102\,\text{кПа}.$$

    Получаем ответ: $P'_\text{ пара } = 230{,}3\,\text{кПа}$.

    Другой вариант решения для давления пара:
    Определим давление пара, если бы вся вода испарилась (что не факт):
    $$P_\text{ max } = \frac{mRT'}{\mu V} = \frac{10\,\text{г} \cdot 8{,}31\,\frac{\text{Дж}}{\text{моль}\cdot\text{К}} \cdot 373\,\text{К}}{18\,\frac{\text{г}}{\text{моль}} \cdot 15\,\text{л}} = 114\,\text{кПа}.$$
    Сравниваем это давление с давлением насыщенного пара при этой температуре $P_\text{ н.
    п.
    100 $\celsius$ } = 101\,\text{кПа}$:
     если у нас получилось меньше табличного значения,
    то вся вода испарилась, если же больше — испарилась лишь часть, а пар является насыщенным.
    Отсюда сразу получаем давление пара: $P'_\text{ пара } = 101{,}3$.
    Сравните этот результат с первым вариантом решения.

    Тут получаем ответ: $P'_\text{ пара } = 228{,}6\,\text{кПа}$.
}
\solutionspace{150pt}

\tasknumber{4}%
\task{%
    Напротив физических величин запишите определение, обозначение и единицы измерения в системе СИ (если есть):
    \begin{enumerate}
        \item абсолютная влажность,
        \item динамическое равновесие.
    \end{enumerate}
}
% autogenerated
