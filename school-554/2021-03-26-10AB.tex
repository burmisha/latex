\setdate{26~марта~2021}
\setclass{10«АБ»}

\addpersonalvariant{Михаил Бурмистров}

\tasknumber{1}%
\task{%
    Сколько молекул водяного пара содержится в сосуде объёмом $3\,\text{л}$ при температуре $90\celsius$,
    и влажности воздуха $35\%$?
}
\solutionspace{160pt}

\tasknumber{2}%
\task{%
    В герметичном сосуде находится влажный воздух при температуре $25\celsius$ и относительной влажности $25\%$.
    \begin{enumerate}
        \item Чему равно парциальное давление насыщенного водяного пара при этой температуре?
        \item Чему равно парциальное давление водяного пара?
        \item Определите точку росы этого пара?
        \item Каким станет парциальное давление водяного пара, если сосуд нагреть до  $80\celsius$?
        \item Чему будет равна относительная влажность воздуха, если сосуд нагреть до $80\celsius$?
        \item Получите ответ на предыдущий вопрос, используя плотности, а не давления.
    \end{enumerate}
}
\solutionspace{200pt}

\tasknumber{3}%
\task{%
    Закрытый сосуд объёмом $10\,\text{л}$ заполнен сухим воздухом при давлении $100\,\text{кПа}$ и температуре $30\celsius$.
    Каким станет давление в сосуде, если в него налить $5\,\text{г}$ воды и нагреть содержимое сосуда до $90\celsius$?
}
\solutionspace{150pt}

\tasknumber{4}%
\task{%
    Напротив физических величин запишите определение, обозначение и единицы измерения в системе СИ (если есть):
    \begin{enumerate}
        \item относительная влажность,
        \item динамическое равновесие.
    \end{enumerate}
}

\variantsplitter

\addpersonalvariant{Ирина Ан}

\tasknumber{1}%
\task{%
    Сколько молекул водяного пара содержится в сосуде объёмом $3\,\text{л}$ при температуре $60\celsius$,
    и влажности воздуха $55\%$?
}
\solutionspace{160pt}

\tasknumber{2}%
\task{%
    В герметичном сосуде находится влажный воздух при температуре $20\celsius$ и относительной влажности $40\%$.
    \begin{enumerate}
        \item Чему равно парциальное давление насыщенного водяного пара при этой температуре?
        \item Чему равно парциальное давление водяного пара?
        \item Определите точку росы этого пара?
        \item Каким станет парциальное давление водяного пара, если сосуд нагреть до  $70\celsius$?
        \item Чему будет равна относительная влажность воздуха, если сосуд нагреть до $70\celsius$?
        \item Получите ответ на предыдущий вопрос, используя плотности, а не давления.
    \end{enumerate}
}
\solutionspace{200pt}

\tasknumber{3}%
\task{%
    Закрытый сосуд объёмом $10\,\text{л}$ заполнен сухим воздухом при давлении $100\,\text{кПа}$ и температуре $10\celsius$.
    Каким станет давление в сосуде, если в него налить $10\,\text{г}$ воды и нагреть содержимое сосуда до $80\celsius$?
}
\solutionspace{150pt}

\tasknumber{4}%
\task{%
    Напротив физических величин запишите определение, обозначение и единицы измерения в системе СИ (если есть):
    \begin{enumerate}
        \item относительная влажность,
        \item насыщенный пар.
    \end{enumerate}
}

\variantsplitter

\addpersonalvariant{Софья Андрианова}

\tasknumber{1}%
\task{%
    Сколько молекул водяного пара содержится в сосуде объёмом $6\,\text{л}$ при температуре $50\celsius$,
    и влажности воздуха $30\%$?
}
\solutionspace{160pt}

\tasknumber{2}%
\task{%
    В герметичном сосуде находится влажный воздух при температуре $30\celsius$ и относительной влажности $60\%$.
    \begin{enumerate}
        \item Чему равно парциальное давление насыщенного водяного пара при этой температуре?
        \item Чему равно парциальное давление водяного пара?
        \item Определите точку росы этого пара?
        \item Каким станет парциальное давление водяного пара, если сосуд нагреть до  $80\celsius$?
        \item Чему будет равна относительная влажность воздуха, если сосуд нагреть до $80\celsius$?
        \item Получите ответ на предыдущий вопрос, используя плотности, а не давления.
    \end{enumerate}
}
\solutionspace{200pt}

\tasknumber{3}%
\task{%
    Закрытый сосуд объёмом $15\,\text{л}$ заполнен сухим воздухом при давлении $100\,\text{кПа}$ и температуре $20\celsius$.
    Каким станет давление в сосуде, если в него налить $5\,\text{г}$ воды и нагреть содержимое сосуда до $80\celsius$?
}
\solutionspace{150pt}

\tasknumber{4}%
\task{%
    Напротив физических величин запишите определение, обозначение и единицы измерения в системе СИ (если есть):
    \begin{enumerate}
        \item относительная влажность,
        \item насыщенный пар.
    \end{enumerate}
}

\variantsplitter

\addpersonalvariant{Владимир Артемчук}

\tasknumber{1}%
\task{%
    Сколько молекул водяного пара содержится в сосуде объёмом $15\,\text{л}$ при температуре $60\celsius$,
    и влажности воздуха $80\%$?
}
\solutionspace{160pt}

\tasknumber{2}%
\task{%
    В герметичном сосуде находится влажный воздух при температуре $25\celsius$ и относительной влажности $40\%$.
    \begin{enumerate}
        \item Чему равно парциальное давление насыщенного водяного пара при этой температуре?
        \item Чему равно парциальное давление водяного пара?
        \item Определите точку росы этого пара?
        \item Каким станет парциальное давление водяного пара, если сосуд нагреть до  $90\celsius$?
        \item Чему будет равна относительная влажность воздуха, если сосуд нагреть до $90\celsius$?
        \item Получите ответ на предыдущий вопрос, используя плотности, а не давления.
    \end{enumerate}
}
\solutionspace{200pt}

\tasknumber{3}%
\task{%
    Закрытый сосуд объёмом $20\,\text{л}$ заполнен сухим воздухом при давлении $100\,\text{кПа}$ и температуре $10\celsius$.
    Каким станет давление в сосуде, если в него налить $5\,\text{г}$ воды и нагреть содержимое сосуда до $80\celsius$?
}
\solutionspace{150pt}

\tasknumber{4}%
\task{%
    Напротив физических величин запишите определение, обозначение и единицы измерения в системе СИ (если есть):
    \begin{enumerate}
        \item абсолютная влажность,
        \item насыщенный пар.
    \end{enumerate}
}

\variantsplitter

\addpersonalvariant{Софья Белянкина}

\tasknumber{1}%
\task{%
    Сколько молекул водяного пара содержится в сосуде объёмом $3\,\text{л}$ при температуре $30\celsius$,
    и влажности воздуха $20\%$?
}
\solutionspace{160pt}

\tasknumber{2}%
\task{%
    В герметичном сосуде находится влажный воздух при температуре $20\celsius$ и относительной влажности $80\%$.
    \begin{enumerate}
        \item Чему равно парциальное давление насыщенного водяного пара при этой температуре?
        \item Чему равно парциальное давление водяного пара?
        \item Определите точку росы этого пара?
        \item Каким станет парциальное давление водяного пара, если сосуд нагреть до  $80\celsius$?
        \item Чему будет равна относительная влажность воздуха, если сосуд нагреть до $80\celsius$?
        \item Получите ответ на предыдущий вопрос, используя плотности, а не давления.
    \end{enumerate}
}
\solutionspace{200pt}

\tasknumber{3}%
\task{%
    Закрытый сосуд объёмом $15\,\text{л}$ заполнен сухим воздухом при давлении $100\,\text{кПа}$ и температуре $10\celsius$.
    Каким станет давление в сосуде, если в него налить $20\,\text{г}$ воды и нагреть содержимое сосуда до $90\celsius$?
}
\solutionspace{150pt}

\tasknumber{4}%
\task{%
    Напротив физических величин запишите определение, обозначение и единицы измерения в системе СИ (если есть):
    \begin{enumerate}
        \item абсолютная влажность,
        \item динамическое равновесие.
    \end{enumerate}
}

\variantsplitter

\addpersonalvariant{Варвара Егиазарян}

\tasknumber{1}%
\task{%
    Сколько молекул водяного пара содержится в сосуде объёмом $12\,\text{л}$ при температуре $100\celsius$,
    и влажности воздуха $60\%$?
}
\solutionspace{160pt}

\tasknumber{2}%
\task{%
    В герметичном сосуде находится влажный воздух при температуре $30\celsius$ и относительной влажности $30\%$.
    \begin{enumerate}
        \item Чему равно парциальное давление насыщенного водяного пара при этой температуре?
        \item Чему равно парциальное давление водяного пара?
        \item Определите точку росы этого пара?
        \item Каким станет парциальное давление водяного пара, если сосуд нагреть до  $90\celsius$?
        \item Чему будет равна относительная влажность воздуха, если сосуд нагреть до $90\celsius$?
        \item Получите ответ на предыдущий вопрос, используя плотности, а не давления.
    \end{enumerate}
}
\solutionspace{200pt}

\tasknumber{3}%
\task{%
    Закрытый сосуд объёмом $20\,\text{л}$ заполнен сухим воздухом при давлении $100\,\text{кПа}$ и температуре $30\celsius$.
    Каким станет давление в сосуде, если в него налить $20\,\text{г}$ воды и нагреть содержимое сосуда до $100\celsius$?
}
\solutionspace{150pt}

\tasknumber{4}%
\task{%
    Напротив физических величин запишите определение, обозначение и единицы измерения в системе СИ (если есть):
    \begin{enumerate}
        \item относительная влажность,
        \item насыщенный пар.
    \end{enumerate}
}

\variantsplitter

\addpersonalvariant{Владислав Емелин}

\tasknumber{1}%
\task{%
    Сколько молекул водяного пара содержится в сосуде объёмом $12\,\text{л}$ при температуре $100\celsius$,
    и влажности воздуха $25\%$?
}
\solutionspace{160pt}

\tasknumber{2}%
\task{%
    В герметичном сосуде находится влажный воздух при температуре $15\celsius$ и относительной влажности $65\%$.
    \begin{enumerate}
        \item Чему равно парциальное давление насыщенного водяного пара при этой температуре?
        \item Чему равно парциальное давление водяного пара?
        \item Определите точку росы этого пара?
        \item Каким станет парциальное давление водяного пара, если сосуд нагреть до  $80\celsius$?
        \item Чему будет равна относительная влажность воздуха, если сосуд нагреть до $80\celsius$?
        \item Получите ответ на предыдущий вопрос, используя плотности, а не давления.
    \end{enumerate}
}
\solutionspace{200pt}

\tasknumber{3}%
\task{%
    Закрытый сосуд объёмом $10\,\text{л}$ заполнен сухим воздухом при давлении $100\,\text{кПа}$ и температуре $20\celsius$.
    Каким станет давление в сосуде, если в него налить $10\,\text{г}$ воды и нагреть содержимое сосуда до $100\celsius$?
}
\solutionspace{150pt}

\tasknumber{4}%
\task{%
    Напротив физических величин запишите определение, обозначение и единицы измерения в системе СИ (если есть):
    \begin{enumerate}
        \item относительная влажность,
        \item динамическое равновесие.
    \end{enumerate}
}

\variantsplitter

\addpersonalvariant{Артём Жичин}

\tasknumber{1}%
\task{%
    Сколько молекул водяного пара содержится в сосуде объёмом $15\,\text{л}$ при температуре $100\celsius$,
    и влажности воздуха $55\%$?
}
\solutionspace{160pt}

\tasknumber{2}%
\task{%
    В герметичном сосуде находится влажный воздух при температуре $20\celsius$ и относительной влажности $40\%$.
    \begin{enumerate}
        \item Чему равно парциальное давление насыщенного водяного пара при этой температуре?
        \item Чему равно парциальное давление водяного пара?
        \item Определите точку росы этого пара?
        \item Каким станет парциальное давление водяного пара, если сосуд нагреть до  $90\celsius$?
        \item Чему будет равна относительная влажность воздуха, если сосуд нагреть до $90\celsius$?
        \item Получите ответ на предыдущий вопрос, используя плотности, а не давления.
    \end{enumerate}
}
\solutionspace{200pt}

\tasknumber{3}%
\task{%
    Закрытый сосуд объёмом $20\,\text{л}$ заполнен сухим воздухом при давлении $100\,\text{кПа}$ и температуре $30\celsius$.
    Каким станет давление в сосуде, если в него налить $10\,\text{г}$ воды и нагреть содержимое сосуда до $90\celsius$?
}
\solutionspace{150pt}

\tasknumber{4}%
\task{%
    Напротив физических величин запишите определение, обозначение и единицы измерения в системе СИ (если есть):
    \begin{enumerate}
        \item абсолютная влажность,
        \item динамическое равновесие.
    \end{enumerate}
}

\variantsplitter

\addpersonalvariant{Дарья Кошман}

\tasknumber{1}%
\task{%
    Сколько молекул водяного пара содержится в сосуде объёмом $9\,\text{л}$ при температуре $90\celsius$,
    и влажности воздуха $60\%$?
}
\solutionspace{160pt}

\tasknumber{2}%
\task{%
    В герметичном сосуде находится влажный воздух при температуре $20\celsius$ и относительной влажности $20\%$.
    \begin{enumerate}
        \item Чему равно парциальное давление насыщенного водяного пара при этой температуре?
        \item Чему равно парциальное давление водяного пара?
        \item Определите точку росы этого пара?
        \item Каким станет парциальное давление водяного пара, если сосуд нагреть до  $70\celsius$?
        \item Чему будет равна относительная влажность воздуха, если сосуд нагреть до $70\celsius$?
        \item Получите ответ на предыдущий вопрос, используя плотности, а не давления.
    \end{enumerate}
}
\solutionspace{200pt}

\tasknumber{3}%
\task{%
    Закрытый сосуд объёмом $10\,\text{л}$ заполнен сухим воздухом при давлении $100\,\text{кПа}$ и температуре $10\celsius$.
    Каким станет давление в сосуде, если в него налить $10\,\text{г}$ воды и нагреть содержимое сосуда до $80\celsius$?
}
\solutionspace{150pt}

\tasknumber{4}%
\task{%
    Напротив физических величин запишите определение, обозначение и единицы измерения в системе СИ (если есть):
    \begin{enumerate}
        \item абсолютная влажность,
        \item динамическое равновесие.
    \end{enumerate}
}

\variantsplitter

\addpersonalvariant{Анна Кузьмичёва}

\tasknumber{1}%
\task{%
    Сколько молекул водяного пара содержится в сосуде объёмом $3\,\text{л}$ при температуре $50\celsius$,
    и влажности воздуха $25\%$?
}
\solutionspace{160pt}

\tasknumber{2}%
\task{%
    В герметичном сосуде находится влажный воздух при температуре $15\celsius$ и относительной влажности $35\%$.
    \begin{enumerate}
        \item Чему равно парциальное давление насыщенного водяного пара при этой температуре?
        \item Чему равно парциальное давление водяного пара?
        \item Определите точку росы этого пара?
        \item Каким станет парциальное давление водяного пара, если сосуд нагреть до  $90\celsius$?
        \item Чему будет равна относительная влажность воздуха, если сосуд нагреть до $90\celsius$?
        \item Получите ответ на предыдущий вопрос, используя плотности, а не давления.
    \end{enumerate}
}
\solutionspace{200pt}

\tasknumber{3}%
\task{%
    Закрытый сосуд объёмом $15\,\text{л}$ заполнен сухим воздухом при давлении $100\,\text{кПа}$ и температуре $10\celsius$.
    Каким станет давление в сосуде, если в него налить $20\,\text{г}$ воды и нагреть содержимое сосуда до $90\celsius$?
}
\solutionspace{150pt}

\tasknumber{4}%
\task{%
    Напротив физических величин запишите определение, обозначение и единицы измерения в системе СИ (если есть):
    \begin{enumerate}
        \item абсолютная влажность,
        \item динамическое равновесие.
    \end{enumerate}
}

\variantsplitter

\addpersonalvariant{Алёна Куприянова}

\tasknumber{1}%
\task{%
    Сколько молекул водяного пара содержится в сосуде объёмом $15\,\text{л}$ при температуре $40\celsius$,
    и влажности воздуха $60\%$?
}
\solutionspace{160pt}

\tasknumber{2}%
\task{%
    В герметичном сосуде находится влажный воздух при температуре $25\celsius$ и относительной влажности $50\%$.
    \begin{enumerate}
        \item Чему равно парциальное давление насыщенного водяного пара при этой температуре?
        \item Чему равно парциальное давление водяного пара?
        \item Определите точку росы этого пара?
        \item Каким станет парциальное давление водяного пара, если сосуд нагреть до  $90\celsius$?
        \item Чему будет равна относительная влажность воздуха, если сосуд нагреть до $90\celsius$?
        \item Получите ответ на предыдущий вопрос, используя плотности, а не давления.
    \end{enumerate}
}
\solutionspace{200pt}

\tasknumber{3}%
\task{%
    Закрытый сосуд объёмом $10\,\text{л}$ заполнен сухим воздухом при давлении $100\,\text{кПа}$ и температуре $20\celsius$.
    Каким станет давление в сосуде, если в него налить $20\,\text{г}$ воды и нагреть содержимое сосуда до $80\celsius$?
}
\solutionspace{150pt}

\tasknumber{4}%
\task{%
    Напротив физических величин запишите определение, обозначение и единицы измерения в системе СИ (если есть):
    \begin{enumerate}
        \item относительная влажность,
        \item насыщенный пар.
    \end{enumerate}
}

\variantsplitter

\addpersonalvariant{Ярослав Лавровский}

\tasknumber{1}%
\task{%
    Сколько молекул водяного пара содержится в сосуде объёмом $9\,\text{л}$ при температуре $100\celsius$,
    и влажности воздуха $40\%$?
}
\solutionspace{160pt}

\tasknumber{2}%
\task{%
    В герметичном сосуде находится влажный воздух при температуре $15\celsius$ и относительной влажности $40\%$.
    \begin{enumerate}
        \item Чему равно парциальное давление насыщенного водяного пара при этой температуре?
        \item Чему равно парциальное давление водяного пара?
        \item Определите точку росы этого пара?
        \item Каким станет парциальное давление водяного пара, если сосуд нагреть до  $90\celsius$?
        \item Чему будет равна относительная влажность воздуха, если сосуд нагреть до $90\celsius$?
        \item Получите ответ на предыдущий вопрос, используя плотности, а не давления.
    \end{enumerate}
}
\solutionspace{200pt}

\tasknumber{3}%
\task{%
    Закрытый сосуд объёмом $10\,\text{л}$ заполнен сухим воздухом при давлении $100\,\text{кПа}$ и температуре $30\celsius$.
    Каким станет давление в сосуде, если в него налить $5\,\text{г}$ воды и нагреть содержимое сосуда до $90\celsius$?
}
\solutionspace{150pt}

\tasknumber{4}%
\task{%
    Напротив физических величин запишите определение, обозначение и единицы измерения в системе СИ (если есть):
    \begin{enumerate}
        \item относительная влажность,
        \item насыщенный пар.
    \end{enumerate}
}

\variantsplitter

\addpersonalvariant{Анастасия Ламанова}

\tasknumber{1}%
\task{%
    Сколько молекул водяного пара содержится в сосуде объёмом $15\,\text{л}$ при температуре $15\celsius$,
    и влажности воздуха $35\%$?
}
\solutionspace{160pt}

\tasknumber{2}%
\task{%
    В герметичном сосуде находится влажный воздух при температуре $15\celsius$ и относительной влажности $35\%$.
    \begin{enumerate}
        \item Чему равно парциальное давление насыщенного водяного пара при этой температуре?
        \item Чему равно парциальное давление водяного пара?
        \item Определите точку росы этого пара?
        \item Каким станет парциальное давление водяного пара, если сосуд нагреть до  $70\celsius$?
        \item Чему будет равна относительная влажность воздуха, если сосуд нагреть до $70\celsius$?
        \item Получите ответ на предыдущий вопрос, используя плотности, а не давления.
    \end{enumerate}
}
\solutionspace{200pt}

\tasknumber{3}%
\task{%
    Закрытый сосуд объёмом $15\,\text{л}$ заполнен сухим воздухом при давлении $100\,\text{кПа}$ и температуре $30\celsius$.
    Каким станет давление в сосуде, если в него налить $5\,\text{г}$ воды и нагреть содержимое сосуда до $100\celsius$?
}
\solutionspace{150pt}

\tasknumber{4}%
\task{%
    Напротив физических величин запишите определение, обозначение и единицы измерения в системе СИ (если есть):
    \begin{enumerate}
        \item относительная влажность,
        \item динамическое равновесие.
    \end{enumerate}
}

\variantsplitter

\addpersonalvariant{Виктория Легонькова}

\tasknumber{1}%
\task{%
    Сколько молекул водяного пара содержится в сосуде объёмом $6\,\text{л}$ при температуре $20\celsius$,
    и влажности воздуха $70\%$?
}
\solutionspace{160pt}

\tasknumber{2}%
\task{%
    В герметичном сосуде находится влажный воздух при температуре $30\celsius$ и относительной влажности $25\%$.
    \begin{enumerate}
        \item Чему равно парциальное давление насыщенного водяного пара при этой температуре?
        \item Чему равно парциальное давление водяного пара?
        \item Определите точку росы этого пара?
        \item Каким станет парциальное давление водяного пара, если сосуд нагреть до  $90\celsius$?
        \item Чему будет равна относительная влажность воздуха, если сосуд нагреть до $90\celsius$?
        \item Получите ответ на предыдущий вопрос, используя плотности, а не давления.
    \end{enumerate}
}
\solutionspace{200pt}

\tasknumber{3}%
\task{%
    Закрытый сосуд объёмом $10\,\text{л}$ заполнен сухим воздухом при давлении $100\,\text{кПа}$ и температуре $10\celsius$.
    Каким станет давление в сосуде, если в него налить $5\,\text{г}$ воды и нагреть содержимое сосуда до $90\celsius$?
}
\solutionspace{150pt}

\tasknumber{4}%
\task{%
    Напротив физических величин запишите определение, обозначение и единицы измерения в системе СИ (если есть):
    \begin{enumerate}
        \item относительная влажность,
        \item динамическое равновесие.
    \end{enumerate}
}

\variantsplitter

\addpersonalvariant{Семён Мартынов}

\tasknumber{1}%
\task{%
    Сколько молекул водяного пара содержится в сосуде объёмом $6\,\text{л}$ при температуре $100\celsius$,
    и влажности воздуха $35\%$?
}
\solutionspace{160pt}

\tasknumber{2}%
\task{%
    В герметичном сосуде находится влажный воздух при температуре $15\celsius$ и относительной влажности $65\%$.
    \begin{enumerate}
        \item Чему равно парциальное давление насыщенного водяного пара при этой температуре?
        \item Чему равно парциальное давление водяного пара?
        \item Определите точку росы этого пара?
        \item Каким станет парциальное давление водяного пара, если сосуд нагреть до  $70\celsius$?
        \item Чему будет равна относительная влажность воздуха, если сосуд нагреть до $70\celsius$?
        \item Получите ответ на предыдущий вопрос, используя плотности, а не давления.
    \end{enumerate}
}
\solutionspace{200pt}

\tasknumber{3}%
\task{%
    Закрытый сосуд объёмом $15\,\text{л}$ заполнен сухим воздухом при давлении $100\,\text{кПа}$ и температуре $30\celsius$.
    Каким станет давление в сосуде, если в него налить $30\,\text{г}$ воды и нагреть содержимое сосуда до $80\celsius$?
}
\solutionspace{150pt}

\tasknumber{4}%
\task{%
    Напротив физических величин запишите определение, обозначение и единицы измерения в системе СИ (если есть):
    \begin{enumerate}
        \item относительная влажность,
        \item насыщенный пар.
    \end{enumerate}
}

\variantsplitter

\addpersonalvariant{Варвара Минаева}

\tasknumber{1}%
\task{%
    Сколько молекул водяного пара содержится в сосуде объёмом $9\,\text{л}$ при температуре $100\celsius$,
    и влажности воздуха $25\%$?
}
\solutionspace{160pt}

\tasknumber{2}%
\task{%
    В герметичном сосуде находится влажный воздух при температуре $25\celsius$ и относительной влажности $25\%$.
    \begin{enumerate}
        \item Чему равно парциальное давление насыщенного водяного пара при этой температуре?
        \item Чему равно парциальное давление водяного пара?
        \item Определите точку росы этого пара?
        \item Каким станет парциальное давление водяного пара, если сосуд нагреть до  $90\celsius$?
        \item Чему будет равна относительная влажность воздуха, если сосуд нагреть до $90\celsius$?
        \item Получите ответ на предыдущий вопрос, используя плотности, а не давления.
    \end{enumerate}
}
\solutionspace{200pt}

\tasknumber{3}%
\task{%
    Закрытый сосуд объёмом $20\,\text{л}$ заполнен сухим воздухом при давлении $100\,\text{кПа}$ и температуре $30\celsius$.
    Каким станет давление в сосуде, если в него налить $30\,\text{г}$ воды и нагреть содержимое сосуда до $80\celsius$?
}
\solutionspace{150pt}

\tasknumber{4}%
\task{%
    Напротив физических величин запишите определение, обозначение и единицы измерения в системе СИ (если есть):
    \begin{enumerate}
        \item абсолютная влажность,
        \item динамическое равновесие.
    \end{enumerate}
}

\variantsplitter

\addpersonalvariant{Леонид Никитин}

\tasknumber{1}%
\task{%
    Сколько молекул водяного пара содержится в сосуде объёмом $9\,\text{л}$ при температуре $80\celsius$,
    и влажности воздуха $40\%$?
}
\solutionspace{160pt}

\tasknumber{2}%
\task{%
    В герметичном сосуде находится влажный воздух при температуре $25\celsius$ и относительной влажности $20\%$.
    \begin{enumerate}
        \item Чему равно парциальное давление насыщенного водяного пара при этой температуре?
        \item Чему равно парциальное давление водяного пара?
        \item Определите точку росы этого пара?
        \item Каким станет парциальное давление водяного пара, если сосуд нагреть до  $80\celsius$?
        \item Чему будет равна относительная влажность воздуха, если сосуд нагреть до $80\celsius$?
        \item Получите ответ на предыдущий вопрос, используя плотности, а не давления.
    \end{enumerate}
}
\solutionspace{200pt}

\tasknumber{3}%
\task{%
    Закрытый сосуд объёмом $10\,\text{л}$ заполнен сухим воздухом при давлении $100\,\text{кПа}$ и температуре $30\celsius$.
    Каким станет давление в сосуде, если в него налить $10\,\text{г}$ воды и нагреть содержимое сосуда до $90\celsius$?
}
\solutionspace{150pt}

\tasknumber{4}%
\task{%
    Напротив физических величин запишите определение, обозначение и единицы измерения в системе СИ (если есть):
    \begin{enumerate}
        \item абсолютная влажность,
        \item динамическое равновесие.
    \end{enumerate}
}

\variantsplitter

\addpersonalvariant{Тимофей Полетаев}

\tasknumber{1}%
\task{%
    Сколько молекул водяного пара содержится в сосуде объёмом $15\,\text{л}$ при температуре $100\celsius$,
    и влажности воздуха $20\%$?
}
\solutionspace{160pt}

\tasknumber{2}%
\task{%
    В герметичном сосуде находится влажный воздух при температуре $15\celsius$ и относительной влажности $25\%$.
    \begin{enumerate}
        \item Чему равно парциальное давление насыщенного водяного пара при этой температуре?
        \item Чему равно парциальное давление водяного пара?
        \item Определите точку росы этого пара?
        \item Каким станет парциальное давление водяного пара, если сосуд нагреть до  $70\celsius$?
        \item Чему будет равна относительная влажность воздуха, если сосуд нагреть до $70\celsius$?
        \item Получите ответ на предыдущий вопрос, используя плотности, а не давления.
    \end{enumerate}
}
\solutionspace{200pt}

\tasknumber{3}%
\task{%
    Закрытый сосуд объёмом $15\,\text{л}$ заполнен сухим воздухом при давлении $100\,\text{кПа}$ и температуре $20\celsius$.
    Каким станет давление в сосуде, если в него налить $20\,\text{г}$ воды и нагреть содержимое сосуда до $100\celsius$?
}
\solutionspace{150pt}

\tasknumber{4}%
\task{%
    Напротив физических величин запишите определение, обозначение и единицы измерения в системе СИ (если есть):
    \begin{enumerate}
        \item относительная влажность,
        \item динамическое равновесие.
    \end{enumerate}
}

\variantsplitter

\addpersonalvariant{Андрей Рожков}

\tasknumber{1}%
\task{%
    Сколько молекул водяного пара содержится в сосуде объёмом $12\,\text{л}$ при температуре $30\celsius$,
    и влажности воздуха $75\%$?
}
\solutionspace{160pt}

\tasknumber{2}%
\task{%
    В герметичном сосуде находится влажный воздух при температуре $40\celsius$ и относительной влажности $40\%$.
    \begin{enumerate}
        \item Чему равно парциальное давление насыщенного водяного пара при этой температуре?
        \item Чему равно парциальное давление водяного пара?
        \item Определите точку росы этого пара?
        \item Каким станет парциальное давление водяного пара, если сосуд нагреть до  $90\celsius$?
        \item Чему будет равна относительная влажность воздуха, если сосуд нагреть до $90\celsius$?
        \item Получите ответ на предыдущий вопрос, используя плотности, а не давления.
    \end{enumerate}
}
\solutionspace{200pt}

\tasknumber{3}%
\task{%
    Закрытый сосуд объёмом $10\,\text{л}$ заполнен сухим воздухом при давлении $100\,\text{кПа}$ и температуре $20\celsius$.
    Каким станет давление в сосуде, если в него налить $5\,\text{г}$ воды и нагреть содержимое сосуда до $80\celsius$?
}
\solutionspace{150pt}

\tasknumber{4}%
\task{%
    Напротив физических величин запишите определение, обозначение и единицы измерения в системе СИ (если есть):
    \begin{enumerate}
        \item относительная влажность,
        \item динамическое равновесие.
    \end{enumerate}
}

\variantsplitter

\addpersonalvariant{Рената Таржиманова}

\tasknumber{1}%
\task{%
    Сколько молекул водяного пара содержится в сосуде объёмом $12\,\text{л}$ при температуре $20\celsius$,
    и влажности воздуха $35\%$?
}
\solutionspace{160pt}

\tasknumber{2}%
\task{%
    В герметичном сосуде находится влажный воздух при температуре $15\celsius$ и относительной влажности $55\%$.
    \begin{enumerate}
        \item Чему равно парциальное давление насыщенного водяного пара при этой температуре?
        \item Чему равно парциальное давление водяного пара?
        \item Определите точку росы этого пара?
        \item Каким станет парциальное давление водяного пара, если сосуд нагреть до  $80\celsius$?
        \item Чему будет равна относительная влажность воздуха, если сосуд нагреть до $80\celsius$?
        \item Получите ответ на предыдущий вопрос, используя плотности, а не давления.
    \end{enumerate}
}
\solutionspace{200pt}

\tasknumber{3}%
\task{%
    Закрытый сосуд объёмом $10\,\text{л}$ заполнен сухим воздухом при давлении $100\,\text{кПа}$ и температуре $10\celsius$.
    Каким станет давление в сосуде, если в него налить $20\,\text{г}$ воды и нагреть содержимое сосуда до $80\celsius$?
}
\solutionspace{150pt}

\tasknumber{4}%
\task{%
    Напротив физических величин запишите определение, обозначение и единицы измерения в системе СИ (если есть):
    \begin{enumerate}
        \item абсолютная влажность,
        \item насыщенный пар.
    \end{enumerate}
}

\variantsplitter

\addpersonalvariant{Андрей Щербаков}

\tasknumber{1}%
\task{%
    Сколько молекул водяного пара содержится в сосуде объёмом $6\,\text{л}$ при температуре $20\celsius$,
    и влажности воздуха $65\%$?
}
\solutionspace{160pt}

\tasknumber{2}%
\task{%
    В герметичном сосуде находится влажный воздух при температуре $25\celsius$ и относительной влажности $40\%$.
    \begin{enumerate}
        \item Чему равно парциальное давление насыщенного водяного пара при этой температуре?
        \item Чему равно парциальное давление водяного пара?
        \item Определите точку росы этого пара?
        \item Каким станет парциальное давление водяного пара, если сосуд нагреть до  $90\celsius$?
        \item Чему будет равна относительная влажность воздуха, если сосуд нагреть до $90\celsius$?
        \item Получите ответ на предыдущий вопрос, используя плотности, а не давления.
    \end{enumerate}
}
\solutionspace{200pt}

\tasknumber{3}%
\task{%
    Закрытый сосуд объёмом $20\,\text{л}$ заполнен сухим воздухом при давлении $100\,\text{кПа}$ и температуре $30\celsius$.
    Каким станет давление в сосуде, если в него налить $20\,\text{г}$ воды и нагреть содержимое сосуда до $100\celsius$?
}
\solutionspace{150pt}

\tasknumber{4}%
\task{%
    Напротив физических величин запишите определение, обозначение и единицы измерения в системе СИ (если есть):
    \begin{enumerate}
        \item абсолютная влажность,
        \item насыщенный пар.
    \end{enumerate}
}

\variantsplitter

\addpersonalvariant{Михаил Ярошевский}

\tasknumber{1}%
\task{%
    Сколько молекул водяного пара содержится в сосуде объёмом $3\,\text{л}$ при температуре $60\celsius$,
    и влажности воздуха $25\%$?
}
\solutionspace{160pt}

\tasknumber{2}%
\task{%
    В герметичном сосуде находится влажный воздух при температуре $30\celsius$ и относительной влажности $55\%$.
    \begin{enumerate}
        \item Чему равно парциальное давление насыщенного водяного пара при этой температуре?
        \item Чему равно парциальное давление водяного пара?
        \item Определите точку росы этого пара?
        \item Каким станет парциальное давление водяного пара, если сосуд нагреть до  $90\celsius$?
        \item Чему будет равна относительная влажность воздуха, если сосуд нагреть до $90\celsius$?
        \item Получите ответ на предыдущий вопрос, используя плотности, а не давления.
    \end{enumerate}
}
\solutionspace{200pt}

\tasknumber{3}%
\task{%
    Закрытый сосуд объёмом $15\,\text{л}$ заполнен сухим воздухом при давлении $100\,\text{кПа}$ и температуре $30\celsius$.
    Каким станет давление в сосуде, если в него налить $10\,\text{г}$ воды и нагреть содержимое сосуда до $90\celsius$?
}
\solutionspace{150pt}

\tasknumber{4}%
\task{%
    Напротив физических величин запишите определение, обозначение и единицы измерения в системе СИ (если есть):
    \begin{enumerate}
        \item абсолютная влажность,
        \item насыщенный пар.
    \end{enumerate}
}

\variantsplitter

\addpersonalvariant{Алексей Алимпиев}

\tasknumber{1}%
\task{%
    Сколько молекул водяного пара содержится в сосуде объёмом $12\,\text{л}$ при температуре $70\celsius$,
    и влажности воздуха $30\%$?
}
\solutionspace{160pt}

\tasknumber{2}%
\task{%
    В герметичном сосуде находится влажный воздух при температуре $25\celsius$ и относительной влажности $35\%$.
    \begin{enumerate}
        \item Чему равно парциальное давление насыщенного водяного пара при этой температуре?
        \item Чему равно парциальное давление водяного пара?
        \item Определите точку росы этого пара?
        \item Каким станет парциальное давление водяного пара, если сосуд нагреть до  $70\celsius$?
        \item Чему будет равна относительная влажность воздуха, если сосуд нагреть до $70\celsius$?
        \item Получите ответ на предыдущий вопрос, используя плотности, а не давления.
    \end{enumerate}
}
\solutionspace{200pt}

\tasknumber{3}%
\task{%
    Закрытый сосуд объёмом $10\,\text{л}$ заполнен сухим воздухом при давлении $100\,\text{кПа}$ и температуре $20\celsius$.
    Каким станет давление в сосуде, если в него налить $5\,\text{г}$ воды и нагреть содержимое сосуда до $100\celsius$?
}
\solutionspace{150pt}

\tasknumber{4}%
\task{%
    Напротив физических величин запишите определение, обозначение и единицы измерения в системе СИ (если есть):
    \begin{enumerate}
        \item относительная влажность,
        \item насыщенный пар.
    \end{enumerate}
}

\variantsplitter

\addpersonalvariant{Евгений Васин}

\tasknumber{1}%
\task{%
    Сколько молекул водяного пара содержится в сосуде объёмом $6\,\text{л}$ при температуре $40\celsius$,
    и влажности воздуха $75\%$?
}
\solutionspace{160pt}

\tasknumber{2}%
\task{%
    В герметичном сосуде находится влажный воздух при температуре $25\celsius$ и относительной влажности $45\%$.
    \begin{enumerate}
        \item Чему равно парциальное давление насыщенного водяного пара при этой температуре?
        \item Чему равно парциальное давление водяного пара?
        \item Определите точку росы этого пара?
        \item Каким станет парциальное давление водяного пара, если сосуд нагреть до  $70\celsius$?
        \item Чему будет равна относительная влажность воздуха, если сосуд нагреть до $70\celsius$?
        \item Получите ответ на предыдущий вопрос, используя плотности, а не давления.
    \end{enumerate}
}
\solutionspace{200pt}

\tasknumber{3}%
\task{%
    Закрытый сосуд объёмом $15\,\text{л}$ заполнен сухим воздухом при давлении $100\,\text{кПа}$ и температуре $20\celsius$.
    Каким станет давление в сосуде, если в него налить $20\,\text{г}$ воды и нагреть содержимое сосуда до $90\celsius$?
}
\solutionspace{150pt}

\tasknumber{4}%
\task{%
    Напротив физических величин запишите определение, обозначение и единицы измерения в системе СИ (если есть):
    \begin{enumerate}
        \item абсолютная влажность,
        \item динамическое равновесие.
    \end{enumerate}
}

\variantsplitter

\addpersonalvariant{Вячеслав Волохов}

\tasknumber{1}%
\task{%
    Сколько молекул водяного пара содержится в сосуде объёмом $12\,\text{л}$ при температуре $40\celsius$,
    и влажности воздуха $75\%$?
}
\solutionspace{160pt}

\tasknumber{2}%
\task{%
    В герметичном сосуде находится влажный воздух при температуре $15\celsius$ и относительной влажности $65\%$.
    \begin{enumerate}
        \item Чему равно парциальное давление насыщенного водяного пара при этой температуре?
        \item Чему равно парциальное давление водяного пара?
        \item Определите точку росы этого пара?
        \item Каким станет парциальное давление водяного пара, если сосуд нагреть до  $90\celsius$?
        \item Чему будет равна относительная влажность воздуха, если сосуд нагреть до $90\celsius$?
        \item Получите ответ на предыдущий вопрос, используя плотности, а не давления.
    \end{enumerate}
}
\solutionspace{200pt}

\tasknumber{3}%
\task{%
    Закрытый сосуд объёмом $20\,\text{л}$ заполнен сухим воздухом при давлении $100\,\text{кПа}$ и температуре $30\celsius$.
    Каким станет давление в сосуде, если в него налить $10\,\text{г}$ воды и нагреть содержимое сосуда до $90\celsius$?
}
\solutionspace{150pt}

\tasknumber{4}%
\task{%
    Напротив физических величин запишите определение, обозначение и единицы измерения в системе СИ (если есть):
    \begin{enumerate}
        \item абсолютная влажность,
        \item насыщенный пар.
    \end{enumerate}
}

\variantsplitter

\addpersonalvariant{Герман Говоров}

\tasknumber{1}%
\task{%
    Сколько молекул водяного пара содержится в сосуде объёмом $12\,\text{л}$ при температуре $20\celsius$,
    и влажности воздуха $55\%$?
}
\solutionspace{160pt}

\tasknumber{2}%
\task{%
    В герметичном сосуде находится влажный воздух при температуре $30\celsius$ и относительной влажности $50\%$.
    \begin{enumerate}
        \item Чему равно парциальное давление насыщенного водяного пара при этой температуре?
        \item Чему равно парциальное давление водяного пара?
        \item Определите точку росы этого пара?
        \item Каким станет парциальное давление водяного пара, если сосуд нагреть до  $80\celsius$?
        \item Чему будет равна относительная влажность воздуха, если сосуд нагреть до $80\celsius$?
        \item Получите ответ на предыдущий вопрос, используя плотности, а не давления.
    \end{enumerate}
}
\solutionspace{200pt}

\tasknumber{3}%
\task{%
    Закрытый сосуд объёмом $10\,\text{л}$ заполнен сухим воздухом при давлении $100\,\text{кПа}$ и температуре $30\celsius$.
    Каким станет давление в сосуде, если в него налить $20\,\text{г}$ воды и нагреть содержимое сосуда до $90\celsius$?
}
\solutionspace{150pt}

\tasknumber{4}%
\task{%
    Напротив физических величин запишите определение, обозначение и единицы измерения в системе СИ (если есть):
    \begin{enumerate}
        \item относительная влажность,
        \item насыщенный пар.
    \end{enumerate}
}

\variantsplitter

\addpersonalvariant{София Журавлёва}

\tasknumber{1}%
\task{%
    Сколько молекул водяного пара содержится в сосуде объёмом $12\,\text{л}$ при температуре $40\celsius$,
    и влажности воздуха $30\%$?
}
\solutionspace{160pt}

\tasknumber{2}%
\task{%
    В герметичном сосуде находится влажный воздух при температуре $25\celsius$ и относительной влажности $75\%$.
    \begin{enumerate}
        \item Чему равно парциальное давление насыщенного водяного пара при этой температуре?
        \item Чему равно парциальное давление водяного пара?
        \item Определите точку росы этого пара?
        \item Каким станет парциальное давление водяного пара, если сосуд нагреть до  $70\celsius$?
        \item Чему будет равна относительная влажность воздуха, если сосуд нагреть до $70\celsius$?
        \item Получите ответ на предыдущий вопрос, используя плотности, а не давления.
    \end{enumerate}
}
\solutionspace{200pt}

\tasknumber{3}%
\task{%
    Закрытый сосуд объёмом $15\,\text{л}$ заполнен сухим воздухом при давлении $100\,\text{кПа}$ и температуре $30\celsius$.
    Каким станет давление в сосуде, если в него налить $10\,\text{г}$ воды и нагреть содержимое сосуда до $80\celsius$?
}
\solutionspace{150pt}

\tasknumber{4}%
\task{%
    Напротив физических величин запишите определение, обозначение и единицы измерения в системе СИ (если есть):
    \begin{enumerate}
        \item относительная влажность,
        \item динамическое равновесие.
    \end{enumerate}
}

\variantsplitter

\addpersonalvariant{Константин Козлов}

\tasknumber{1}%
\task{%
    Сколько молекул водяного пара содержится в сосуде объёмом $12\,\text{л}$ при температуре $70\celsius$,
    и влажности воздуха $55\%$?
}
\solutionspace{160pt}

\tasknumber{2}%
\task{%
    В герметичном сосуде находится влажный воздух при температуре $40\celsius$ и относительной влажности $75\%$.
    \begin{enumerate}
        \item Чему равно парциальное давление насыщенного водяного пара при этой температуре?
        \item Чему равно парциальное давление водяного пара?
        \item Определите точку росы этого пара?
        \item Каким станет парциальное давление водяного пара, если сосуд нагреть до  $80\celsius$?
        \item Чему будет равна относительная влажность воздуха, если сосуд нагреть до $80\celsius$?
        \item Получите ответ на предыдущий вопрос, используя плотности, а не давления.
    \end{enumerate}
}
\solutionspace{200pt}

\tasknumber{3}%
\task{%
    Закрытый сосуд объёмом $20\,\text{л}$ заполнен сухим воздухом при давлении $100\,\text{кПа}$ и температуре $20\celsius$.
    Каким станет давление в сосуде, если в него налить $5\,\text{г}$ воды и нагреть содержимое сосуда до $100\celsius$?
}
\solutionspace{150pt}

\tasknumber{4}%
\task{%
    Напротив физических величин запишите определение, обозначение и единицы измерения в системе СИ (если есть):
    \begin{enumerate}
        \item абсолютная влажность,
        \item динамическое равновесие.
    \end{enumerate}
}

\variantsplitter

\addpersonalvariant{Наталья Кравченко}

\tasknumber{1}%
\task{%
    Сколько молекул водяного пара содержится в сосуде объёмом $15\,\text{л}$ при температуре $60\celsius$,
    и влажности воздуха $75\%$?
}
\solutionspace{160pt}

\tasknumber{2}%
\task{%
    В герметичном сосуде находится влажный воздух при температуре $40\celsius$ и относительной влажности $70\%$.
    \begin{enumerate}
        \item Чему равно парциальное давление насыщенного водяного пара при этой температуре?
        \item Чему равно парциальное давление водяного пара?
        \item Определите точку росы этого пара?
        \item Каким станет парциальное давление водяного пара, если сосуд нагреть до  $90\celsius$?
        \item Чему будет равна относительная влажность воздуха, если сосуд нагреть до $90\celsius$?
        \item Получите ответ на предыдущий вопрос, используя плотности, а не давления.
    \end{enumerate}
}
\solutionspace{200pt}

\tasknumber{3}%
\task{%
    Закрытый сосуд объёмом $15\,\text{л}$ заполнен сухим воздухом при давлении $100\,\text{кПа}$ и температуре $30\celsius$.
    Каким станет давление в сосуде, если в него налить $5\,\text{г}$ воды и нагреть содержимое сосуда до $90\celsius$?
}
\solutionspace{150pt}

\tasknumber{4}%
\task{%
    Напротив физических величин запишите определение, обозначение и единицы измерения в системе СИ (если есть):
    \begin{enumerate}
        \item относительная влажность,
        \item насыщенный пар.
    \end{enumerate}
}

\variantsplitter

\addpersonalvariant{Матвей Кузьмин}

\tasknumber{1}%
\task{%
    Сколько молекул водяного пара содержится в сосуде объёмом $15\,\text{л}$ при температуре $20\celsius$,
    и влажности воздуха $20\%$?
}
\solutionspace{160pt}

\tasknumber{2}%
\task{%
    В герметичном сосуде находится влажный воздух при температуре $25\celsius$ и относительной влажности $45\%$.
    \begin{enumerate}
        \item Чему равно парциальное давление насыщенного водяного пара при этой температуре?
        \item Чему равно парциальное давление водяного пара?
        \item Определите точку росы этого пара?
        \item Каким станет парциальное давление водяного пара, если сосуд нагреть до  $70\celsius$?
        \item Чему будет равна относительная влажность воздуха, если сосуд нагреть до $70\celsius$?
        \item Получите ответ на предыдущий вопрос, используя плотности, а не давления.
    \end{enumerate}
}
\solutionspace{200pt}

\tasknumber{3}%
\task{%
    Закрытый сосуд объёмом $10\,\text{л}$ заполнен сухим воздухом при давлении $100\,\text{кПа}$ и температуре $30\celsius$.
    Каким станет давление в сосуде, если в него налить $30\,\text{г}$ воды и нагреть содержимое сосуда до $90\celsius$?
}
\solutionspace{150pt}

\tasknumber{4}%
\task{%
    Напротив физических величин запишите определение, обозначение и единицы измерения в системе СИ (если есть):
    \begin{enumerate}
        \item относительная влажность,
        \item динамическое равновесие.
    \end{enumerate}
}

\variantsplitter

\addpersonalvariant{Сергей Малышев}

\tasknumber{1}%
\task{%
    Сколько молекул водяного пара содержится в сосуде объёмом $3\,\text{л}$ при температуре $25\celsius$,
    и влажности воздуха $75\%$?
}
\solutionspace{160pt}

\tasknumber{2}%
\task{%
    В герметичном сосуде находится влажный воздух при температуре $20\celsius$ и относительной влажности $55\%$.
    \begin{enumerate}
        \item Чему равно парциальное давление насыщенного водяного пара при этой температуре?
        \item Чему равно парциальное давление водяного пара?
        \item Определите точку росы этого пара?
        \item Каким станет парциальное давление водяного пара, если сосуд нагреть до  $80\celsius$?
        \item Чему будет равна относительная влажность воздуха, если сосуд нагреть до $80\celsius$?
        \item Получите ответ на предыдущий вопрос, используя плотности, а не давления.
    \end{enumerate}
}
\solutionspace{200pt}

\tasknumber{3}%
\task{%
    Закрытый сосуд объёмом $10\,\text{л}$ заполнен сухим воздухом при давлении $100\,\text{кПа}$ и температуре $10\celsius$.
    Каким станет давление в сосуде, если в него налить $20\,\text{г}$ воды и нагреть содержимое сосуда до $100\celsius$?
}
\solutionspace{150pt}

\tasknumber{4}%
\task{%
    Напротив физических величин запишите определение, обозначение и единицы измерения в системе СИ (если есть):
    \begin{enumerate}
        \item абсолютная влажность,
        \item динамическое равновесие.
    \end{enumerate}
}

\variantsplitter

\addpersonalvariant{Алина Полканова}

\tasknumber{1}%
\task{%
    Сколько молекул водяного пара содержится в сосуде объёмом $3\,\text{л}$ при температуре $90\celsius$,
    и влажности воздуха $20\%$?
}
\solutionspace{160pt}

\tasknumber{2}%
\task{%
    В герметичном сосуде находится влажный воздух при температуре $25\celsius$ и относительной влажности $35\%$.
    \begin{enumerate}
        \item Чему равно парциальное давление насыщенного водяного пара при этой температуре?
        \item Чему равно парциальное давление водяного пара?
        \item Определите точку росы этого пара?
        \item Каким станет парциальное давление водяного пара, если сосуд нагреть до  $80\celsius$?
        \item Чему будет равна относительная влажность воздуха, если сосуд нагреть до $80\celsius$?
        \item Получите ответ на предыдущий вопрос, используя плотности, а не давления.
    \end{enumerate}
}
\solutionspace{200pt}

\tasknumber{3}%
\task{%
    Закрытый сосуд объёмом $20\,\text{л}$ заполнен сухим воздухом при давлении $100\,\text{кПа}$ и температуре $20\celsius$.
    Каким станет давление в сосуде, если в него налить $30\,\text{г}$ воды и нагреть содержимое сосуда до $80\celsius$?
}
\solutionspace{150pt}

\tasknumber{4}%
\task{%
    Напротив физических величин запишите определение, обозначение и единицы измерения в системе СИ (если есть):
    \begin{enumerate}
        \item абсолютная влажность,
        \item насыщенный пар.
    \end{enumerate}
}

\variantsplitter

\addpersonalvariant{Сергей Пономарёв}

\tasknumber{1}%
\task{%
    Сколько молекул водяного пара содержится в сосуде объёмом $15\,\text{л}$ при температуре $60\celsius$,
    и влажности воздуха $75\%$?
}
\solutionspace{160pt}

\tasknumber{2}%
\task{%
    В герметичном сосуде находится влажный воздух при температуре $30\celsius$ и относительной влажности $40\%$.
    \begin{enumerate}
        \item Чему равно парциальное давление насыщенного водяного пара при этой температуре?
        \item Чему равно парциальное давление водяного пара?
        \item Определите точку росы этого пара?
        \item Каким станет парциальное давление водяного пара, если сосуд нагреть до  $90\celsius$?
        \item Чему будет равна относительная влажность воздуха, если сосуд нагреть до $90\celsius$?
        \item Получите ответ на предыдущий вопрос, используя плотности, а не давления.
    \end{enumerate}
}
\solutionspace{200pt}

\tasknumber{3}%
\task{%
    Закрытый сосуд объёмом $20\,\text{л}$ заполнен сухим воздухом при давлении $100\,\text{кПа}$ и температуре $20\celsius$.
    Каким станет давление в сосуде, если в него налить $20\,\text{г}$ воды и нагреть содержимое сосуда до $100\celsius$?
}
\solutionspace{150pt}

\tasknumber{4}%
\task{%
    Напротив физических величин запишите определение, обозначение и единицы измерения в системе СИ (если есть):
    \begin{enumerate}
        \item относительная влажность,
        \item насыщенный пар.
    \end{enumerate}
}

\variantsplitter

\addpersonalvariant{Егор Свистушкин}

\tasknumber{1}%
\task{%
    Сколько молекул водяного пара содержится в сосуде объёмом $12\,\text{л}$ при температуре $80\celsius$,
    и влажности воздуха $25\%$?
}
\solutionspace{160pt}

\tasknumber{2}%
\task{%
    В герметичном сосуде находится влажный воздух при температуре $25\celsius$ и относительной влажности $45\%$.
    \begin{enumerate}
        \item Чему равно парциальное давление насыщенного водяного пара при этой температуре?
        \item Чему равно парциальное давление водяного пара?
        \item Определите точку росы этого пара?
        \item Каким станет парциальное давление водяного пара, если сосуд нагреть до  $90\celsius$?
        \item Чему будет равна относительная влажность воздуха, если сосуд нагреть до $90\celsius$?
        \item Получите ответ на предыдущий вопрос, используя плотности, а не давления.
    \end{enumerate}
}
\solutionspace{200pt}

\tasknumber{3}%
\task{%
    Закрытый сосуд объёмом $10\,\text{л}$ заполнен сухим воздухом при давлении $100\,\text{кПа}$ и температуре $10\celsius$.
    Каким станет давление в сосуде, если в него налить $20\,\text{г}$ воды и нагреть содержимое сосуда до $90\celsius$?
}
\solutionspace{150pt}

\tasknumber{4}%
\task{%
    Напротив физических величин запишите определение, обозначение и единицы измерения в системе СИ (если есть):
    \begin{enumerate}
        \item абсолютная влажность,
        \item насыщенный пар.
    \end{enumerate}
}

\variantsplitter

\addpersonalvariant{Дмитрий Соколов}

\tasknumber{1}%
\task{%
    Сколько молекул водяного пара содержится в сосуде объёмом $15\,\text{л}$ при температуре $25\celsius$,
    и влажности воздуха $55\%$?
}
\solutionspace{160pt}

\tasknumber{2}%
\task{%
    В герметичном сосуде находится влажный воздух при температуре $40\celsius$ и относительной влажности $45\%$.
    \begin{enumerate}
        \item Чему равно парциальное давление насыщенного водяного пара при этой температуре?
        \item Чему равно парциальное давление водяного пара?
        \item Определите точку росы этого пара?
        \item Каким станет парциальное давление водяного пара, если сосуд нагреть до  $90\celsius$?
        \item Чему будет равна относительная влажность воздуха, если сосуд нагреть до $90\celsius$?
        \item Получите ответ на предыдущий вопрос, используя плотности, а не давления.
    \end{enumerate}
}
\solutionspace{200pt}

\tasknumber{3}%
\task{%
    Закрытый сосуд объёмом $20\,\text{л}$ заполнен сухим воздухом при давлении $100\,\text{кПа}$ и температуре $30\celsius$.
    Каким станет давление в сосуде, если в него налить $30\,\text{г}$ воды и нагреть содержимое сосуда до $90\celsius$?
}
\solutionspace{150pt}

\tasknumber{4}%
\task{%
    Напротив физических величин запишите определение, обозначение и единицы измерения в системе СИ (если есть):
    \begin{enumerate}
        \item относительная влажность,
        \item динамическое равновесие.
    \end{enumerate}
}

\variantsplitter

\addpersonalvariant{Арсений Трофимов}

\tasknumber{1}%
\task{%
    Сколько молекул водяного пара содержится в сосуде объёмом $9\,\text{л}$ при температуре $100\celsius$,
    и влажности воздуха $30\%$?
}
\solutionspace{160pt}

\tasknumber{2}%
\task{%
    В герметичном сосуде находится влажный воздух при температуре $25\celsius$ и относительной влажности $45\%$.
    \begin{enumerate}
        \item Чему равно парциальное давление насыщенного водяного пара при этой температуре?
        \item Чему равно парциальное давление водяного пара?
        \item Определите точку росы этого пара?
        \item Каким станет парциальное давление водяного пара, если сосуд нагреть до  $70\celsius$?
        \item Чему будет равна относительная влажность воздуха, если сосуд нагреть до $70\celsius$?
        \item Получите ответ на предыдущий вопрос, используя плотности, а не давления.
    \end{enumerate}
}
\solutionspace{200pt}

\tasknumber{3}%
\task{%
    Закрытый сосуд объёмом $15\,\text{л}$ заполнен сухим воздухом при давлении $100\,\text{кПа}$ и температуре $30\celsius$.
    Каким станет давление в сосуде, если в него налить $10\,\text{г}$ воды и нагреть содержимое сосуда до $80\celsius$?
}
\solutionspace{150pt}

\tasknumber{4}%
\task{%
    Напротив физических величин запишите определение, обозначение и единицы измерения в системе СИ (если есть):
    \begin{enumerate}
        \item абсолютная влажность,
        \item динамическое равновесие.
    \end{enumerate}
}
% autogenerated
