\setdate{25~марта~2021}
\setclass{10«АБ»}

\addpersonalvariant{Михаил Бурмистров}

\tasknumber{1}%
\task{%
    С какой силой взаимодействуют 2 точечных заряда $q_1 = 3\,\text{нКл}$ и $q_2 = 4\,\text{нКл}$,
    находящиеся на расстоянии $r = 6\,\text{см}$?
}
\answer{%
    $
        F
            = k\frac{q_1q_2}{r^2}
            = 9 \cdot 10^{9}\,\frac{\text{Н}\cdot\text{м}^{2}}{\text{Кл}^{2}} \cdot \frac{3\,\text{нКл} *4\,\text{нКл}}{\sqr{ 6\,\text{см} }}
            = 3 \cdot 10^{31}\units{Н}
              \approx {3{,}00} \cdot 10^{31}\units{Н}
    $
}
\solutionspace{120pt}

\tasknumber{2}%
\task{%
    Два одинаковых маленьких проводящих заряженных шарика находятся на расстоянии~$l$ друг от друга.
    Заряд первого равен~$+3Q$, второго~--- $+8Q$.
    Шарики приводят в соприкосновение, а после опять разводят на расстояние~$2l$.
    \begin{itemize}
        \item Каким стал заряд каждого из шариков?
        \item Определите характер (притяжение или отталкивание) и силу взаимодействия шариков до и после соприкосновения.
        \item Как изменилась сила взаимодействия шариков после соприкосновения?
    \end{itemize}
}
\answer{%
    \begin{align*}
    F &= k\frac{q_1 q_2}{\sqr{2 l}} = k\frac{(+3Q) \cdot (+8Q)}{\sqr{2 l}}, \text{отталкивание}; \\
        q'_1 = q'_2 = \frac{q_1 + q_2}2 = \frac{+3Q + +8Q}2 \implies
        F'  &= k\frac{q'_1 q'_2}{l^2}
            = k\frac{\sqr{\frac{(+3Q) + (+8Q)}2}}{2^2 \cdot l^2},
        \text{отталкивание}.
    \end{align*}
}
\solutionspace{120pt}

\tasknumber{3}%
\task{%
    На координатной плоскости в точках $(-d; 0)$ и $(d; 0)$
    находятся заряды, соответственно, $+Q$ и $+Q$.
    Сделайте рисунок, определите величину напряжённости электрического поля
    и укажите её направление в двух точках: $(0; -d)$ и $(2d; 0)$.
}
\solutionspace{120pt}

\tasknumber{4}%
\task{%
    Заряд $q_1$ создает в точке $A$ электрическое поле
    по величине равное~$E_1=300\funits{В}{м}$,
    а $q_2$~--- $E_2=400\funits{В}{м}$.
    Угол между векторами $\vect{E_1}$ и $\vect{E_2}$ равен $\varphi$.
    Определите величину суммарного электрического поля в точке $A$,
    создаваемого обоими зарядами $q_1$ и $q_2$.
    Сделайте рисунки и вычислите значение для двух значений угла $\varphi$:
    $\varphi_1=0^\circ$ и $\varphi_2=90^\circ$.
}

\variantsplitter

\addpersonalvariant{Ирина Ан}

\tasknumber{1}%
\task{%
    С какой силой взаимодействуют 2 точечных заряда $q_1 = 4\,\text{нКл}$ и $q_2 = 2\,\text{нКл}$,
    находящиеся на расстоянии $l = 6\,\text{см}$?
}
\answer{%
    $
        F
            = k\frac{q_1q_2}{l^2}
            = 9 \cdot 10^{9}\,\frac{\text{Н}\cdot\text{м}^{2}}{\text{Кл}^{2}} \cdot \frac{4\,\text{нКл} *2\,\text{нКл}}{\sqr{ 6\,\text{см} }}
            = 2 \cdot 10^{31}\units{Н}
              \approx {2{,}00} \cdot 10^{31}\units{Н}
    $
}
\solutionspace{120pt}

\tasknumber{2}%
\task{%
    Два одинаковых маленьких проводящих заряженных шарика находятся на расстоянии~$r$ друг от друга.
    Заряд первого равен~$-7q$, второго~--- $-2q$.
    Шарики приводят в соприкосновение, а после опять разводят на расстояние~$2r$.
    \begin{itemize}
        \item Каким стал заряд каждого из шариков?
        \item Определите характер (притяжение или отталкивание) и силу взаимодействия шариков до и после соприкосновения.
        \item Как изменилась сила взаимодействия шариков после соприкосновения?
    \end{itemize}
}
\answer{%
    \begin{align*}
    F &= k\frac{q_1 q_2}{\sqr{2 r}} = k\frac{(-7q) \cdot (-2q)}{\sqr{2 r}}, \text{отталкивание}; \\
        q'_1 = q'_2 = \frac{q_1 + q_2}2 = \frac{-7q -2q}2 \implies
        F'  &= k\frac{q'_1 q'_2}{r^2}
            = k\frac{\sqr{\frac{(-7q) + (-2q)}2}}{2^2 \cdot r^2},
        \text{отталкивание}.
    \end{align*}
}
\solutionspace{120pt}

\tasknumber{3}%
\task{%
    На координатной плоскости в точках $(-d; 0)$ и $(d; 0)$
    находятся заряды, соответственно, $+Q$ и $+Q$.
    Сделайте рисунок, определите величину напряжённости электрического поля
    и укажите её направление в двух точках: $(0; d)$ и $(2d; 0)$.
}
\solutionspace{120pt}

\tasknumber{4}%
\task{%
    Заряд $q_1$ создает в точке $A$ электрическое поле
    по величине равное~$E_1=300\funits{В}{м}$,
    а $q_2$~--- $E_2=400\funits{В}{м}$.
    Угол между векторами $\vect{E_1}$ и $\vect{E_2}$ равен $\varphi$.
    Определите величину суммарного электрического поля в точке $A$,
    создаваемого обоими зарядами $q_1$ и $q_2$.
    Сделайте рисунки и вычислите значение для двух значений угла $\varphi$:
    $\varphi_1=90^\circ$ и $\varphi_2=180^\circ$.
}

\variantsplitter

\addpersonalvariant{Софья Андрианова}

\tasknumber{1}%
\task{%
    С какой силой взаимодействуют 2 точечных заряда $q_1 = 3\,\text{нКл}$ и $q_2 = 4\,\text{нКл}$,
    находящиеся на расстоянии $l = 5\,\text{см}$?
}
\answer{%
    $
        F
            = k\frac{q_1q_2}{l^2}
            = 9 \cdot 10^{9}\,\frac{\text{Н}\cdot\text{м}^{2}}{\text{Кл}^{2}} \cdot \frac{3\,\text{нКл} *4\,\text{нКл}}{\sqr{ 5\,\text{см} }}
            = \frac{108}{25} \cdot 10^{31}\units{Н}
              \approx {4{,}32} \cdot 10^{31}\units{Н}
    $
}
\solutionspace{120pt}

\tasknumber{2}%
\task{%
    Два одинаковых маленьких проводящих заряженных шарика находятся на расстоянии~$l$ друг от друга.
    Заряд первого равен~$-5q$, второго~--- $+6q$.
    Шарики приводят в соприкосновение, а после опять разводят на расстояние~$4l$.
    \begin{itemize}
        \item Каким стал заряд каждого из шариков?
        \item Определите характер (притяжение или отталкивание) и силу взаимодействия шариков до и после соприкосновения.
        \item Как изменилась сила взаимодействия шариков после соприкосновения?
    \end{itemize}
}
\answer{%
    \begin{align*}
    F &= k\frac{q_1 q_2}{\sqr{4 l}} = k\frac{(-5q) \cdot (+6q)}{\sqr{4 l}}, \text{отталкивание}; \\
        q'_1 = q'_2 = \frac{q_1 + q_2}2 = \frac{-5q + +6q}2 \implies
        F'  &= k\frac{q'_1 q'_2}{l^2}
            = k\frac{\sqr{\frac{(-5q) + (+6q)}2}}{4^2 \cdot l^2},
        \text{отталкивание}.
    \end{align*}
}
\solutionspace{120pt}

\tasknumber{3}%
\task{%
    На координатной плоскости в точках $(-a; 0)$ и $(a; 0)$
    находятся заряды, соответственно, $-q$ и $-q$.
    Сделайте рисунок, определите величину напряжённости электрического поля
    и укажите её направление в двух точках: $(0; -a)$ и $(-2a; 0)$.
}
\solutionspace{120pt}

\tasknumber{4}%
\task{%
    Заряд $q_1$ создает в точке $A$ электрическое поле
    по величине равное~$E_1=200\funits{В}{м}$,
    а $q_2$~--- $E_2=200\funits{В}{м}$.
    Угол между векторами $\vect{E_1}$ и $\vect{E_2}$ равен $\alpha$.
    Определите величину суммарного электрического поля в точке $A$,
    создаваемого обоими зарядами $q_1$ и $q_2$.
    Сделайте рисунки и вычислите значение для двух значений угла $\alpha$:
    $\alpha_1=0^\circ$ и $\alpha_2=60^\circ$.
}

\variantsplitter

\addpersonalvariant{Владимир Артемчук}

\tasknumber{1}%
\task{%
    С какой силой взаимодействуют 2 точечных заряда $q_1 = 2\,\text{нКл}$ и $q_2 = 3\,\text{нКл}$,
    находящиеся на расстоянии $r = 2\,\text{см}$?
}
\answer{%
    $
        F
            = k\frac{q_1q_2}{r^2}
            = 9 \cdot 10^{9}\,\frac{\text{Н}\cdot\text{м}^{2}}{\text{Кл}^{2}} \cdot \frac{2\,\text{нКл} *3\,\text{нКл}}{\sqr{ 2\,\text{см} }}
            = \frac{27}2 \cdot 10^{31}\units{Н}
              \approx {13{,}50} \cdot 10^{31}\units{Н}
    $
}
\solutionspace{120pt}

\tasknumber{2}%
\task{%
    Два одинаковых маленьких проводящих заряженных шарика находятся на расстоянии~$d$ друг от друга.
    Заряд первого равен~$+7Q$, второго~--- $+6Q$.
    Шарики приводят в соприкосновение, а после опять разводят на расстояние~$4d$.
    \begin{itemize}
        \item Каким стал заряд каждого из шариков?
        \item Определите характер (притяжение или отталкивание) и силу взаимодействия шариков до и после соприкосновения.
        \item Как изменилась сила взаимодействия шариков после соприкосновения?
    \end{itemize}
}
\answer{%
    \begin{align*}
    F &= k\frac{q_1 q_2}{\sqr{4 d}} = k\frac{(+7Q) \cdot (+6Q)}{\sqr{4 d}}, \text{отталкивание}; \\
        q'_1 = q'_2 = \frac{q_1 + q_2}2 = \frac{+7Q + +6Q}2 \implies
        F'  &= k\frac{q'_1 q'_2}{d^2}
            = k\frac{\sqr{\frac{(+7Q) + (+6Q)}2}}{4^2 \cdot d^2},
        \text{отталкивание}.
    \end{align*}
}
\solutionspace{120pt}

\tasknumber{3}%
\task{%
    На координатной плоскости в точках $(-a; 0)$ и $(a; 0)$
    находятся заряды, соответственно, $+Q$ и $+Q$.
    Сделайте рисунок, определите величину напряжённости электрического поля
    и укажите её направление в двух точках: $(0; a)$ и $(-2a; 0)$.
}
\solutionspace{120pt}

\tasknumber{4}%
\task{%
    Заряд $q_1$ создает в точке $A$ электрическое поле
    по величине равное~$E_1=200\funits{В}{м}$,
    а $q_2$~--- $E_2=200\funits{В}{м}$.
    Угол между векторами $\vect{E_1}$ и $\vect{E_2}$ равен $\varphi$.
    Определите величину суммарного электрического поля в точке $A$,
    создаваемого обоими зарядами $q_1$ и $q_2$.
    Сделайте рисунки и вычислите значение для двух значений угла $\varphi$:
    $\varphi_1=0^\circ$ и $\varphi_2=60^\circ$.
}

\variantsplitter

\addpersonalvariant{Софья Белянкина}

\tasknumber{1}%
\task{%
    С какой силой взаимодействуют 2 точечных заряда $q_1 = 3\,\text{нКл}$ и $q_2 = 4\,\text{нКл}$,
    находящиеся на расстоянии $l = 5\,\text{см}$?
}
\answer{%
    $
        F
            = k\frac{q_1q_2}{l^2}
            = 9 \cdot 10^{9}\,\frac{\text{Н}\cdot\text{м}^{2}}{\text{Кл}^{2}} \cdot \frac{3\,\text{нКл} *4\,\text{нКл}}{\sqr{ 5\,\text{см} }}
            = \frac{108}{25} \cdot 10^{31}\units{Н}
              \approx {4{,}32} \cdot 10^{31}\units{Н}
    $
}
\solutionspace{120pt}

\tasknumber{2}%
\task{%
    Два одинаковых маленьких проводящих заряженных шарика находятся на расстоянии~$d$ друг от друга.
    Заряд первого равен~$+7Q$, второго~--- $+8Q$.
    Шарики приводят в соприкосновение, а после опять разводят на расстояние~$2d$.
    \begin{itemize}
        \item Каким стал заряд каждого из шариков?
        \item Определите характер (притяжение или отталкивание) и силу взаимодействия шариков до и после соприкосновения.
        \item Как изменилась сила взаимодействия шариков после соприкосновения?
    \end{itemize}
}
\answer{%
    \begin{align*}
    F &= k\frac{q_1 q_2}{\sqr{2 d}} = k\frac{(+7Q) \cdot (+8Q)}{\sqr{2 d}}, \text{отталкивание}; \\
        q'_1 = q'_2 = \frac{q_1 + q_2}2 = \frac{+7Q + +8Q}2 \implies
        F'  &= k\frac{q'_1 q'_2}{d^2}
            = k\frac{\sqr{\frac{(+7Q) + (+8Q)}2}}{2^2 \cdot d^2},
        \text{отталкивание}.
    \end{align*}
}
\solutionspace{120pt}

\tasknumber{3}%
\task{%
    На координатной плоскости в точках $(-l; 0)$ и $(l; 0)$
    находятся заряды, соответственно, $+Q$ и $+Q$.
    Сделайте рисунок, определите величину напряжённости электрического поля
    и укажите её направление в двух точках: $(0; l)$ и $(2l; 0)$.
}
\solutionspace{120pt}

\tasknumber{4}%
\task{%
    Заряд $q_1$ создает в точке $A$ электрическое поле
    по величине равное~$E_1=250\funits{В}{м}$,
    а $q_2$~--- $E_2=250\funits{В}{м}$.
    Угол между векторами $\vect{E_1}$ и $\vect{E_2}$ равен $\varphi$.
    Определите величину суммарного электрического поля в точке $A$,
    создаваемого обоими зарядами $q_1$ и $q_2$.
    Сделайте рисунки и вычислите значение для двух значений угла $\varphi$:
    $\varphi_1=0^\circ$ и $\varphi_2=60^\circ$.
}

\variantsplitter

\addpersonalvariant{Варвара Егиазарян}

\tasknumber{1}%
\task{%
    С какой силой взаимодействуют 2 точечных заряда $q_1 = 3\,\text{нКл}$ и $q_2 = 4\,\text{нКл}$,
    находящиеся на расстоянии $l = 3\,\text{см}$?
}
\answer{%
    $
        F
            = k\frac{q_1q_2}{l^2}
            = 9 \cdot 10^{9}\,\frac{\text{Н}\cdot\text{м}^{2}}{\text{Кл}^{2}} \cdot \frac{3\,\text{нКл} *4\,\text{нКл}}{\sqr{ 3\,\text{см} }}
            = 12 \cdot 10^{31}\units{Н}
              \approx {12{,}00} \cdot 10^{31}\units{Н}
    $
}
\solutionspace{120pt}

\tasknumber{2}%
\task{%
    Два одинаковых маленьких проводящих заряженных шарика находятся на расстоянии~$l$ друг от друга.
    Заряд первого равен~$-5Q$, второго~--- $+8Q$.
    Шарики приводят в соприкосновение, а после опять разводят на расстояние~$3l$.
    \begin{itemize}
        \item Каким стал заряд каждого из шариков?
        \item Определите характер (притяжение или отталкивание) и силу взаимодействия шариков до и после соприкосновения.
        \item Как изменилась сила взаимодействия шариков после соприкосновения?
    \end{itemize}
}
\answer{%
    \begin{align*}
    F &= k\frac{q_1 q_2}{\sqr{3 l}} = k\frac{(-5Q) \cdot (+8Q)}{\sqr{3 l}}, \text{отталкивание}; \\
        q'_1 = q'_2 = \frac{q_1 + q_2}2 = \frac{-5Q + +8Q}2 \implies
        F'  &= k\frac{q'_1 q'_2}{l^2}
            = k\frac{\sqr{\frac{(-5Q) + (+8Q)}2}}{3^2 \cdot l^2},
        \text{отталкивание}.
    \end{align*}
}
\solutionspace{120pt}

\tasknumber{3}%
\task{%
    На координатной плоскости в точках $(-a; 0)$ и $(a; 0)$
    находятся заряды, соответственно, $+Q$ и $+Q$.
    Сделайте рисунок, определите величину напряжённости электрического поля
    и укажите её направление в двух точках: $(0; -a)$ и $(2a; 0)$.
}
\solutionspace{120pt}

\tasknumber{4}%
\task{%
    Заряд $q_1$ создает в точке $A$ электрическое поле
    по величине равное~$E_1=300\funits{В}{м}$,
    а $q_2$~--- $E_2=400\funits{В}{м}$.
    Угол между векторами $\vect{E_1}$ и $\vect{E_2}$ равен $\alpha$.
    Определите величину суммарного электрического поля в точке $A$,
    создаваемого обоими зарядами $q_1$ и $q_2$.
    Сделайте рисунки и вычислите значение для двух значений угла $\alpha$:
    $\alpha_1=90^\circ$ и $\alpha_2=180^\circ$.
}

\variantsplitter

\addpersonalvariant{Владислав Емелин}

\tasknumber{1}%
\task{%
    С какой силой взаимодействуют 2 точечных заряда $q_1 = 4\,\text{нКл}$ и $q_2 = 3\,\text{нКл}$,
    находящиеся на расстоянии $r = 2\,\text{см}$?
}
\answer{%
    $
        F
            = k\frac{q_1q_2}{r^2}
            = 9 \cdot 10^{9}\,\frac{\text{Н}\cdot\text{м}^{2}}{\text{Кл}^{2}} \cdot \frac{4\,\text{нКл} *3\,\text{нКл}}{\sqr{ 2\,\text{см} }}
            = 27 \cdot 10^{31}\units{Н}
              \approx {27{,}00} \cdot 10^{31}\units{Н}
    $
}
\solutionspace{120pt}

\tasknumber{2}%
\task{%
    Два одинаковых маленьких проводящих заряженных шарика находятся на расстоянии~$d$ друг от друга.
    Заряд первого равен~$-7Q$, второго~--- $-2Q$.
    Шарики приводят в соприкосновение, а после опять разводят на расстояние~$3d$.
    \begin{itemize}
        \item Каким стал заряд каждого из шариков?
        \item Определите характер (притяжение или отталкивание) и силу взаимодействия шариков до и после соприкосновения.
        \item Как изменилась сила взаимодействия шариков после соприкосновения?
    \end{itemize}
}
\answer{%
    \begin{align*}
    F &= k\frac{q_1 q_2}{\sqr{3 d}} = k\frac{(-7Q) \cdot (-2Q)}{\sqr{3 d}}, \text{отталкивание}; \\
        q'_1 = q'_2 = \frac{q_1 + q_2}2 = \frac{-7Q -2Q}2 \implies
        F'  &= k\frac{q'_1 q'_2}{d^2}
            = k\frac{\sqr{\frac{(-7Q) + (-2Q)}2}}{3^2 \cdot d^2},
        \text{отталкивание}.
    \end{align*}
}
\solutionspace{120pt}

\tasknumber{3}%
\task{%
    На координатной плоскости в точках $(-r; 0)$ и $(r; 0)$
    находятся заряды, соответственно, $-Q$ и $+Q$.
    Сделайте рисунок, определите величину напряжённости электрического поля
    и укажите её направление в двух точках: $(0; r)$ и $(-2r; 0)$.
}
\solutionspace{120pt}

\tasknumber{4}%
\task{%
    Заряд $q_1$ создает в точке $A$ электрическое поле
    по величине равное~$E_1=120\funits{В}{м}$,
    а $q_2$~--- $E_2=50\funits{В}{м}$.
    Угол между векторами $\vect{E_1}$ и $\vect{E_2}$ равен $\varphi$.
    Определите величину суммарного электрического поля в точке $A$,
    создаваемого обоими зарядами $q_1$ и $q_2$.
    Сделайте рисунки и вычислите значение для двух значений угла $\varphi$:
    $\varphi_1=90^\circ$ и $\varphi_2=180^\circ$.
}

\variantsplitter

\addpersonalvariant{Артём Жичин}

\tasknumber{1}%
\task{%
    С какой силой взаимодействуют 2 точечных заряда $q_1 = 3\,\text{нКл}$ и $q_2 = 2\,\text{нКл}$,
    находящиеся на расстоянии $r = 6\,\text{см}$?
}
\answer{%
    $
        F
            = k\frac{q_1q_2}{r^2}
            = 9 \cdot 10^{9}\,\frac{\text{Н}\cdot\text{м}^{2}}{\text{Кл}^{2}} \cdot \frac{3\,\text{нКл} *2\,\text{нКл}}{\sqr{ 6\,\text{см} }}
            = \frac32 \cdot 10^{31}\units{Н}
              \approx {1{,}50} \cdot 10^{31}\units{Н}
    $
}
\solutionspace{120pt}

\tasknumber{2}%
\task{%
    Два одинаковых маленьких проводящих заряженных шарика находятся на расстоянии~$d$ друг от друга.
    Заряд первого равен~$-7q$, второго~--- $-4q$.
    Шарики приводят в соприкосновение, а после опять разводят на расстояние~$4d$.
    \begin{itemize}
        \item Каким стал заряд каждого из шариков?
        \item Определите характер (притяжение или отталкивание) и силу взаимодействия шариков до и после соприкосновения.
        \item Как изменилась сила взаимодействия шариков после соприкосновения?
    \end{itemize}
}
\answer{%
    \begin{align*}
    F &= k\frac{q_1 q_2}{\sqr{4 d}} = k\frac{(-7q) \cdot (-4q)}{\sqr{4 d}}, \text{отталкивание}; \\
        q'_1 = q'_2 = \frac{q_1 + q_2}2 = \frac{-7q -4q}2 \implies
        F'  &= k\frac{q'_1 q'_2}{d^2}
            = k\frac{\sqr{\frac{(-7q) + (-4q)}2}}{4^2 \cdot d^2},
        \text{отталкивание}.
    \end{align*}
}
\solutionspace{120pt}

\tasknumber{3}%
\task{%
    На координатной плоскости в точках $(-d; 0)$ и $(d; 0)$
    находятся заряды, соответственно, $+Q$ и $+Q$.
    Сделайте рисунок, определите величину напряжённости электрического поля
    и укажите её направление в двух точках: $(0; -d)$ и $(2d; 0)$.
}
\solutionspace{120pt}

\tasknumber{4}%
\task{%
    Заряд $q_1$ создает в точке $A$ электрическое поле
    по величине равное~$E_1=7\funits{В}{м}$,
    а $q_2$~--- $E_2=24\funits{В}{м}$.
    Угол между векторами $\vect{E_1}$ и $\vect{E_2}$ равен $\varphi$.
    Определите величину суммарного электрического поля в точке $A$,
    создаваемого обоими зарядами $q_1$ и $q_2$.
    Сделайте рисунки и вычислите значение для двух значений угла $\varphi$:
    $\varphi_1=0^\circ$ и $\varphi_2=90^\circ$.
}

\variantsplitter

\addpersonalvariant{Дарья Кошман}

\tasknumber{1}%
\task{%
    С какой силой взаимодействуют 2 точечных заряда $q_1 = 4\,\text{нКл}$ и $q_2 = 3\,\text{нКл}$,
    находящиеся на расстоянии $l = 6\,\text{см}$?
}
\answer{%
    $
        F
            = k\frac{q_1q_2}{l^2}
            = 9 \cdot 10^{9}\,\frac{\text{Н}\cdot\text{м}^{2}}{\text{Кл}^{2}} \cdot \frac{4\,\text{нКл} *3\,\text{нКл}}{\sqr{ 6\,\text{см} }}
            = 3 \cdot 10^{31}\units{Н}
              \approx {3{,}00} \cdot 10^{31}\units{Н}
    $
}
\solutionspace{120pt}

\tasknumber{2}%
\task{%
    Два одинаковых маленьких проводящих заряженных шарика находятся на расстоянии~$l$ друг от друга.
    Заряд первого равен~$-5q$, второго~--- $-2q$.
    Шарики приводят в соприкосновение, а после опять разводят на расстояние~$2l$.
    \begin{itemize}
        \item Каким стал заряд каждого из шариков?
        \item Определите характер (притяжение или отталкивание) и силу взаимодействия шариков до и после соприкосновения.
        \item Как изменилась сила взаимодействия шариков после соприкосновения?
    \end{itemize}
}
\answer{%
    \begin{align*}
    F &= k\frac{q_1 q_2}{\sqr{2 l}} = k\frac{(-5q) \cdot (-2q)}{\sqr{2 l}}, \text{отталкивание}; \\
        q'_1 = q'_2 = \frac{q_1 + q_2}2 = \frac{-5q -2q}2 \implies
        F'  &= k\frac{q'_1 q'_2}{l^2}
            = k\frac{\sqr{\frac{(-5q) + (-2q)}2}}{2^2 \cdot l^2},
        \text{отталкивание}.
    \end{align*}
}
\solutionspace{120pt}

\tasknumber{3}%
\task{%
    На координатной плоскости в точках $(-l; 0)$ и $(l; 0)$
    находятся заряды, соответственно, $-q$ и $-q$.
    Сделайте рисунок, определите величину напряжённости электрического поля
    и укажите её направление в двух точках: $(0; l)$ и $(2l; 0)$.
}
\solutionspace{120pt}

\tasknumber{4}%
\task{%
    Заряд $q_1$ создает в точке $A$ электрическое поле
    по величине равное~$E_1=24\funits{В}{м}$,
    а $q_2$~--- $E_2=7\funits{В}{м}$.
    Угол между векторами $\vect{E_1}$ и $\vect{E_2}$ равен $\alpha$.
    Определите величину суммарного электрического поля в точке $A$,
    создаваемого обоими зарядами $q_1$ и $q_2$.
    Сделайте рисунки и вычислите значение для двух значений угла $\alpha$:
    $\alpha_1=90^\circ$ и $\alpha_2=180^\circ$.
}

\variantsplitter

\addpersonalvariant{Анна Кузьмичёва}

\tasknumber{1}%
\task{%
    С какой силой взаимодействуют 2 точечных заряда $q_1 = 2\,\text{нКл}$ и $q_2 = 3\,\text{нКл}$,
    находящиеся на расстоянии $d = 3\,\text{см}$?
}
\answer{%
    $
        F
            = k\frac{q_1q_2}{d^2}
            = 9 \cdot 10^{9}\,\frac{\text{Н}\cdot\text{м}^{2}}{\text{Кл}^{2}} \cdot \frac{2\,\text{нКл} *3\,\text{нКл}}{\sqr{ 3\,\text{см} }}
            = 6 \cdot 10^{31}\units{Н}
              \approx {6{,}00} \cdot 10^{31}\units{Н}
    $
}
\solutionspace{120pt}

\tasknumber{2}%
\task{%
    Два одинаковых маленьких проводящих заряженных шарика находятся на расстоянии~$d$ друг от друга.
    Заряд первого равен~$-7q$, второго~--- $-2q$.
    Шарики приводят в соприкосновение, а после опять разводят на расстояние~$2d$.
    \begin{itemize}
        \item Каким стал заряд каждого из шариков?
        \item Определите характер (притяжение или отталкивание) и силу взаимодействия шариков до и после соприкосновения.
        \item Как изменилась сила взаимодействия шариков после соприкосновения?
    \end{itemize}
}
\answer{%
    \begin{align*}
    F &= k\frac{q_1 q_2}{\sqr{2 d}} = k\frac{(-7q) \cdot (-2q)}{\sqr{2 d}}, \text{отталкивание}; \\
        q'_1 = q'_2 = \frac{q_1 + q_2}2 = \frac{-7q -2q}2 \implies
        F'  &= k\frac{q'_1 q'_2}{d^2}
            = k\frac{\sqr{\frac{(-7q) + (-2q)}2}}{2^2 \cdot d^2},
        \text{отталкивание}.
    \end{align*}
}
\solutionspace{120pt}

\tasknumber{3}%
\task{%
    На координатной плоскости в точках $(-r; 0)$ и $(r; 0)$
    находятся заряды, соответственно, $-q$ и $-q$.
    Сделайте рисунок, определите величину напряжённости электрического поля
    и укажите её направление в двух точках: $(0; r)$ и $(-2r; 0)$.
}
\solutionspace{120pt}

\tasknumber{4}%
\task{%
    Заряд $q_1$ создает в точке $A$ электрическое поле
    по величине равное~$E_1=250\funits{В}{м}$,
    а $q_2$~--- $E_2=250\funits{В}{м}$.
    Угол между векторами $\vect{E_1}$ и $\vect{E_2}$ равен $\varphi$.
    Определите величину суммарного электрического поля в точке $A$,
    создаваемого обоими зарядами $q_1$ и $q_2$.
    Сделайте рисунки и вычислите значение для двух значений угла $\varphi$:
    $\varphi_1=0^\circ$ и $\varphi_2=60^\circ$.
}

\variantsplitter

\addpersonalvariant{Алёна Куприянова}

\tasknumber{1}%
\task{%
    С какой силой взаимодействуют 2 точечных заряда $q_1 = 2\,\text{нКл}$ и $q_2 = 3\,\text{нКл}$,
    находящиеся на расстоянии $r = 5\,\text{см}$?
}
\answer{%
    $
        F
            = k\frac{q_1q_2}{r^2}
            = 9 \cdot 10^{9}\,\frac{\text{Н}\cdot\text{м}^{2}}{\text{Кл}^{2}} \cdot \frac{2\,\text{нКл} *3\,\text{нКл}}{\sqr{ 5\,\text{см} }}
            = \frac{54}{25} \cdot 10^{31}\units{Н}
              \approx {2{,}16} \cdot 10^{31}\units{Н}
    $
}
\solutionspace{120pt}

\tasknumber{2}%
\task{%
    Два одинаковых маленьких проводящих заряженных шарика находятся на расстоянии~$r$ друг от друга.
    Заряд первого равен~$-3q$, второго~--- $+4q$.
    Шарики приводят в соприкосновение, а после опять разводят на расстояние~$4r$.
    \begin{itemize}
        \item Каким стал заряд каждого из шариков?
        \item Определите характер (притяжение или отталкивание) и силу взаимодействия шариков до и после соприкосновения.
        \item Как изменилась сила взаимодействия шариков после соприкосновения?
    \end{itemize}
}
\answer{%
    \begin{align*}
    F &= k\frac{q_1 q_2}{\sqr{4 r}} = k\frac{(-3q) \cdot (+4q)}{\sqr{4 r}}, \text{отталкивание}; \\
        q'_1 = q'_2 = \frac{q_1 + q_2}2 = \frac{-3q + +4q}2 \implies
        F'  &= k\frac{q'_1 q'_2}{r^2}
            = k\frac{\sqr{\frac{(-3q) + (+4q)}2}}{4^2 \cdot r^2},
        \text{отталкивание}.
    \end{align*}
}
\solutionspace{120pt}

\tasknumber{3}%
\task{%
    На координатной плоскости в точках $(-d; 0)$ и $(d; 0)$
    находятся заряды, соответственно, $-Q$ и $+Q$.
    Сделайте рисунок, определите величину напряжённости электрического поля
    и укажите её направление в двух точках: $(0; -d)$ и $(2d; 0)$.
}
\solutionspace{120pt}

\tasknumber{4}%
\task{%
    Заряд $q_1$ создает в точке $A$ электрическое поле
    по величине равное~$E_1=120\funits{В}{м}$,
    а $q_2$~--- $E_2=50\funits{В}{м}$.
    Угол между векторами $\vect{E_1}$ и $\vect{E_2}$ равен $\alpha$.
    Определите величину суммарного электрического поля в точке $A$,
    создаваемого обоими зарядами $q_1$ и $q_2$.
    Сделайте рисунки и вычислите значение для двух значений угла $\alpha$:
    $\alpha_1=90^\circ$ и $\alpha_2=180^\circ$.
}

\variantsplitter

\addpersonalvariant{Ярослав Лавровский}

\tasknumber{1}%
\task{%
    С какой силой взаимодействуют 2 точечных заряда $q_1 = 3\,\text{нКл}$ и $q_2 = 2\,\text{нКл}$,
    находящиеся на расстоянии $r = 5\,\text{см}$?
}
\answer{%
    $
        F
            = k\frac{q_1q_2}{r^2}
            = 9 \cdot 10^{9}\,\frac{\text{Н}\cdot\text{м}^{2}}{\text{Кл}^{2}} \cdot \frac{3\,\text{нКл} *2\,\text{нКл}}{\sqr{ 5\,\text{см} }}
            = \frac{54}{25} \cdot 10^{31}\units{Н}
              \approx {2{,}16} \cdot 10^{31}\units{Н}
    $
}
\solutionspace{120pt}

\tasknumber{2}%
\task{%
    Два одинаковых маленьких проводящих заряженных шарика находятся на расстоянии~$d$ друг от друга.
    Заряд первого равен~$-5Q$, второго~--- $-8Q$.
    Шарики приводят в соприкосновение, а после опять разводят на расстояние~$4d$.
    \begin{itemize}
        \item Каким стал заряд каждого из шариков?
        \item Определите характер (притяжение или отталкивание) и силу взаимодействия шариков до и после соприкосновения.
        \item Как изменилась сила взаимодействия шариков после соприкосновения?
    \end{itemize}
}
\answer{%
    \begin{align*}
    F &= k\frac{q_1 q_2}{\sqr{4 d}} = k\frac{(-5Q) \cdot (-8Q)}{\sqr{4 d}}, \text{отталкивание}; \\
        q'_1 = q'_2 = \frac{q_1 + q_2}2 = \frac{-5Q -8Q}2 \implies
        F'  &= k\frac{q'_1 q'_2}{d^2}
            = k\frac{\sqr{\frac{(-5Q) + (-8Q)}2}}{4^2 \cdot d^2},
        \text{отталкивание}.
    \end{align*}
}
\solutionspace{120pt}

\tasknumber{3}%
\task{%
    На координатной плоскости в точках $(-r; 0)$ и $(r; 0)$
    находятся заряды, соответственно, $-q$ и $-q$.
    Сделайте рисунок, определите величину напряжённости электрического поля
    и укажите её направление в двух точках: $(0; -r)$ и $(2r; 0)$.
}
\solutionspace{120pt}

\tasknumber{4}%
\task{%
    Заряд $q_1$ создает в точке $A$ электрическое поле
    по величине равное~$E_1=7\funits{В}{м}$,
    а $q_2$~--- $E_2=24\funits{В}{м}$.
    Угол между векторами $\vect{E_1}$ и $\vect{E_2}$ равен $\varphi$.
    Определите величину суммарного электрического поля в точке $A$,
    создаваемого обоими зарядами $q_1$ и $q_2$.
    Сделайте рисунки и вычислите значение для двух значений угла $\varphi$:
    $\varphi_1=0^\circ$ и $\varphi_2=90^\circ$.
}

\variantsplitter

\addpersonalvariant{Анастасия Ламанова}

\tasknumber{1}%
\task{%
    С какой силой взаимодействуют 2 точечных заряда $q_1 = 3\,\text{нКл}$ и $q_2 = 2\,\text{нКл}$,
    находящиеся на расстоянии $d = 5\,\text{см}$?
}
\answer{%
    $
        F
            = k\frac{q_1q_2}{d^2}
            = 9 \cdot 10^{9}\,\frac{\text{Н}\cdot\text{м}^{2}}{\text{Кл}^{2}} \cdot \frac{3\,\text{нКл} *2\,\text{нКл}}{\sqr{ 5\,\text{см} }}
            = \frac{54}{25} \cdot 10^{31}\units{Н}
              \approx {2{,}16} \cdot 10^{31}\units{Н}
    $
}
\solutionspace{120pt}

\tasknumber{2}%
\task{%
    Два одинаковых маленьких проводящих заряженных шарика находятся на расстоянии~$l$ друг от друга.
    Заряд первого равен~$+7Q$, второго~--- $-8Q$.
    Шарики приводят в соприкосновение, а после опять разводят на расстояние~$4l$.
    \begin{itemize}
        \item Каким стал заряд каждого из шариков?
        \item Определите характер (притяжение или отталкивание) и силу взаимодействия шариков до и после соприкосновения.
        \item Как изменилась сила взаимодействия шариков после соприкосновения?
    \end{itemize}
}
\answer{%
    \begin{align*}
    F &= k\frac{q_1 q_2}{\sqr{4 l}} = k\frac{(+7Q) \cdot (-8Q)}{\sqr{4 l}}, \text{отталкивание}; \\
        q'_1 = q'_2 = \frac{q_1 + q_2}2 = \frac{+7Q -8Q}2 \implies
        F'  &= k\frac{q'_1 q'_2}{l^2}
            = k\frac{\sqr{\frac{(+7Q) + (-8Q)}2}}{4^2 \cdot l^2},
        \text{отталкивание}.
    \end{align*}
}
\solutionspace{120pt}

\tasknumber{3}%
\task{%
    На координатной плоскости в точках $(-r; 0)$ и $(r; 0)$
    находятся заряды, соответственно, $-Q$ и $+Q$.
    Сделайте рисунок, определите величину напряжённости электрического поля
    и укажите её направление в двух точках: $(0; r)$ и $(-2r; 0)$.
}
\solutionspace{120pt}

\tasknumber{4}%
\task{%
    Заряд $q_1$ создает в точке $A$ электрическое поле
    по величине равное~$E_1=72\funits{В}{м}$,
    а $q_2$~--- $E_2=72\funits{В}{м}$.
    Угол между векторами $\vect{E_1}$ и $\vect{E_2}$ равен $\alpha$.
    Определите величину суммарного электрического поля в точке $A$,
    создаваемого обоими зарядами $q_1$ и $q_2$.
    Сделайте рисунки и вычислите значение для двух значений угла $\alpha$:
    $\alpha_1=0^\circ$ и $\alpha_2=120^\circ$.
}

\variantsplitter

\addpersonalvariant{Виктория Легонькова}

\tasknumber{1}%
\task{%
    С какой силой взаимодействуют 2 точечных заряда $q_1 = 4\,\text{нКл}$ и $q_2 = 3\,\text{нКл}$,
    находящиеся на расстоянии $l = 3\,\text{см}$?
}
\answer{%
    $
        F
            = k\frac{q_1q_2}{l^2}
            = 9 \cdot 10^{9}\,\frac{\text{Н}\cdot\text{м}^{2}}{\text{Кл}^{2}} \cdot \frac{4\,\text{нКл} *3\,\text{нКл}}{\sqr{ 3\,\text{см} }}
            = 12 \cdot 10^{31}\units{Н}
              \approx {12{,}00} \cdot 10^{31}\units{Н}
    $
}
\solutionspace{120pt}

\tasknumber{2}%
\task{%
    Два одинаковых маленьких проводящих заряженных шарика находятся на расстоянии~$d$ друг от друга.
    Заряд первого равен~$-3q$, второго~--- $-4q$.
    Шарики приводят в соприкосновение, а после опять разводят на расстояние~$2d$.
    \begin{itemize}
        \item Каким стал заряд каждого из шариков?
        \item Определите характер (притяжение или отталкивание) и силу взаимодействия шариков до и после соприкосновения.
        \item Как изменилась сила взаимодействия шариков после соприкосновения?
    \end{itemize}
}
\answer{%
    \begin{align*}
    F &= k\frac{q_1 q_2}{\sqr{2 d}} = k\frac{(-3q) \cdot (-4q)}{\sqr{2 d}}, \text{отталкивание}; \\
        q'_1 = q'_2 = \frac{q_1 + q_2}2 = \frac{-3q -4q}2 \implies
        F'  &= k\frac{q'_1 q'_2}{d^2}
            = k\frac{\sqr{\frac{(-3q) + (-4q)}2}}{2^2 \cdot d^2},
        \text{отталкивание}.
    \end{align*}
}
\solutionspace{120pt}

\tasknumber{3}%
\task{%
    На координатной плоскости в точках $(-r; 0)$ и $(r; 0)$
    находятся заряды, соответственно, $-q$ и $-q$.
    Сделайте рисунок, определите величину напряжённости электрического поля
    и укажите её направление в двух точках: $(0; -r)$ и $(2r; 0)$.
}
\solutionspace{120pt}

\tasknumber{4}%
\task{%
    Заряд $q_1$ создает в точке $A$ электрическое поле
    по величине равное~$E_1=300\funits{В}{м}$,
    а $q_2$~--- $E_2=400\funits{В}{м}$.
    Угол между векторами $\vect{E_1}$ и $\vect{E_2}$ равен $\varphi$.
    Определите величину суммарного электрического поля в точке $A$,
    создаваемого обоими зарядами $q_1$ и $q_2$.
    Сделайте рисунки и вычислите значение для двух значений угла $\varphi$:
    $\varphi_1=90^\circ$ и $\varphi_2=180^\circ$.
}

\variantsplitter

\addpersonalvariant{Семён Мартынов}

\tasknumber{1}%
\task{%
    С какой силой взаимодействуют 2 точечных заряда $q_1 = 2\,\text{нКл}$ и $q_2 = 4\,\text{нКл}$,
    находящиеся на расстоянии $r = 5\,\text{см}$?
}
\answer{%
    $
        F
            = k\frac{q_1q_2}{r^2}
            = 9 \cdot 10^{9}\,\frac{\text{Н}\cdot\text{м}^{2}}{\text{Кл}^{2}} \cdot \frac{2\,\text{нКл} *4\,\text{нКл}}{\sqr{ 5\,\text{см} }}
            = \frac{72}{25} \cdot 10^{31}\units{Н}
              \approx {2{,}88} \cdot 10^{31}\units{Н}
    $
}
\solutionspace{120pt}

\tasknumber{2}%
\task{%
    Два одинаковых маленьких проводящих заряженных шарика находятся на расстоянии~$l$ друг от друга.
    Заряд первого равен~$-5q$, второго~--- $+2q$.
    Шарики приводят в соприкосновение, а после опять разводят на расстояние~$2l$.
    \begin{itemize}
        \item Каким стал заряд каждого из шариков?
        \item Определите характер (притяжение или отталкивание) и силу взаимодействия шариков до и после соприкосновения.
        \item Как изменилась сила взаимодействия шариков после соприкосновения?
    \end{itemize}
}
\answer{%
    \begin{align*}
    F &= k\frac{q_1 q_2}{\sqr{2 l}} = k\frac{(-5q) \cdot (+2q)}{\sqr{2 l}}, \text{отталкивание}; \\
        q'_1 = q'_2 = \frac{q_1 + q_2}2 = \frac{-5q + +2q}2 \implies
        F'  &= k\frac{q'_1 q'_2}{l^2}
            = k\frac{\sqr{\frac{(-5q) + (+2q)}2}}{2^2 \cdot l^2},
        \text{отталкивание}.
    \end{align*}
}
\solutionspace{120pt}

\tasknumber{3}%
\task{%
    На координатной плоскости в точках $(-r; 0)$ и $(r; 0)$
    находятся заряды, соответственно, $+Q$ и $+Q$.
    Сделайте рисунок, определите величину напряжённости электрического поля
    и укажите её направление в двух точках: $(0; -r)$ и $(2r; 0)$.
}
\solutionspace{120pt}

\tasknumber{4}%
\task{%
    Заряд $q_1$ создает в точке $A$ электрическое поле
    по величине равное~$E_1=7\funits{В}{м}$,
    а $q_2$~--- $E_2=24\funits{В}{м}$.
    Угол между векторами $\vect{E_1}$ и $\vect{E_2}$ равен $\alpha$.
    Определите величину суммарного электрического поля в точке $A$,
    создаваемого обоими зарядами $q_1$ и $q_2$.
    Сделайте рисунки и вычислите значение для двух значений угла $\alpha$:
    $\alpha_1=0^\circ$ и $\alpha_2=90^\circ$.
}

\variantsplitter

\addpersonalvariant{Варвара Минаева}

\tasknumber{1}%
\task{%
    С какой силой взаимодействуют 2 точечных заряда $q_1 = 4\,\text{нКл}$ и $q_2 = 3\,\text{нКл}$,
    находящиеся на расстоянии $d = 5\,\text{см}$?
}
\answer{%
    $
        F
            = k\frac{q_1q_2}{d^2}
            = 9 \cdot 10^{9}\,\frac{\text{Н}\cdot\text{м}^{2}}{\text{Кл}^{2}} \cdot \frac{4\,\text{нКл} *3\,\text{нКл}}{\sqr{ 5\,\text{см} }}
            = \frac{108}{25} \cdot 10^{31}\units{Н}
              \approx {4{,}32} \cdot 10^{31}\units{Н}
    $
}
\solutionspace{120pt}

\tasknumber{2}%
\task{%
    Два одинаковых маленьких проводящих заряженных шарика находятся на расстоянии~$l$ друг от друга.
    Заряд первого равен~$+7q$, второго~--- $+2q$.
    Шарики приводят в соприкосновение, а после опять разводят на расстояние~$3l$.
    \begin{itemize}
        \item Каким стал заряд каждого из шариков?
        \item Определите характер (притяжение или отталкивание) и силу взаимодействия шариков до и после соприкосновения.
        \item Как изменилась сила взаимодействия шариков после соприкосновения?
    \end{itemize}
}
\answer{%
    \begin{align*}
    F &= k\frac{q_1 q_2}{\sqr{3 l}} = k\frac{(+7q) \cdot (+2q)}{\sqr{3 l}}, \text{отталкивание}; \\
        q'_1 = q'_2 = \frac{q_1 + q_2}2 = \frac{+7q + +2q}2 \implies
        F'  &= k\frac{q'_1 q'_2}{l^2}
            = k\frac{\sqr{\frac{(+7q) + (+2q)}2}}{3^2 \cdot l^2},
        \text{отталкивание}.
    \end{align*}
}
\solutionspace{120pt}

\tasknumber{3}%
\task{%
    На координатной плоскости в точках $(-l; 0)$ и $(l; 0)$
    находятся заряды, соответственно, $-q$ и $-q$.
    Сделайте рисунок, определите величину напряжённости электрического поля
    и укажите её направление в двух точках: $(0; l)$ и $(-2l; 0)$.
}
\solutionspace{120pt}

\tasknumber{4}%
\task{%
    Заряд $q_1$ создает в точке $A$ электрическое поле
    по величине равное~$E_1=300\funits{В}{м}$,
    а $q_2$~--- $E_2=400\funits{В}{м}$.
    Угол между векторами $\vect{E_1}$ и $\vect{E_2}$ равен $\alpha$.
    Определите величину суммарного электрического поля в точке $A$,
    создаваемого обоими зарядами $q_1$ и $q_2$.
    Сделайте рисунки и вычислите значение для двух значений угла $\alpha$:
    $\alpha_1=90^\circ$ и $\alpha_2=180^\circ$.
}

\variantsplitter

\addpersonalvariant{Леонид Никитин}

\tasknumber{1}%
\task{%
    С какой силой взаимодействуют 2 точечных заряда $q_1 = 3\,\text{нКл}$ и $q_2 = 4\,\text{нКл}$,
    находящиеся на расстоянии $d = 2\,\text{см}$?
}
\answer{%
    $
        F
            = k\frac{q_1q_2}{d^2}
            = 9 \cdot 10^{9}\,\frac{\text{Н}\cdot\text{м}^{2}}{\text{Кл}^{2}} \cdot \frac{3\,\text{нКл} *4\,\text{нКл}}{\sqr{ 2\,\text{см} }}
            = 27 \cdot 10^{31}\units{Н}
              \approx {27{,}00} \cdot 10^{31}\units{Н}
    $
}
\solutionspace{120pt}

\tasknumber{2}%
\task{%
    Два одинаковых маленьких проводящих заряженных шарика находятся на расстоянии~$d$ друг от друга.
    Заряд первого равен~$-3q$, второго~--- $+8q$.
    Шарики приводят в соприкосновение, а после опять разводят на расстояние~$2d$.
    \begin{itemize}
        \item Каким стал заряд каждого из шариков?
        \item Определите характер (притяжение или отталкивание) и силу взаимодействия шариков до и после соприкосновения.
        \item Как изменилась сила взаимодействия шариков после соприкосновения?
    \end{itemize}
}
\answer{%
    \begin{align*}
    F &= k\frac{q_1 q_2}{\sqr{2 d}} = k\frac{(-3q) \cdot (+8q)}{\sqr{2 d}}, \text{отталкивание}; \\
        q'_1 = q'_2 = \frac{q_1 + q_2}2 = \frac{-3q + +8q}2 \implies
        F'  &= k\frac{q'_1 q'_2}{d^2}
            = k\frac{\sqr{\frac{(-3q) + (+8q)}2}}{2^2 \cdot d^2},
        \text{отталкивание}.
    \end{align*}
}
\solutionspace{120pt}

\tasknumber{3}%
\task{%
    На координатной плоскости в точках $(-d; 0)$ и $(d; 0)$
    находятся заряды, соответственно, $-q$ и $-q$.
    Сделайте рисунок, определите величину напряжённости электрического поля
    и укажите её направление в двух точках: $(0; -d)$ и $(2d; 0)$.
}
\solutionspace{120pt}

\tasknumber{4}%
\task{%
    Заряд $q_1$ создает в точке $A$ электрическое поле
    по величине равное~$E_1=250\funits{В}{м}$,
    а $q_2$~--- $E_2=250\funits{В}{м}$.
    Угол между векторами $\vect{E_1}$ и $\vect{E_2}$ равен $\varphi$.
    Определите величину суммарного электрического поля в точке $A$,
    создаваемого обоими зарядами $q_1$ и $q_2$.
    Сделайте рисунки и вычислите значение для двух значений угла $\varphi$:
    $\varphi_1=0^\circ$ и $\varphi_2=60^\circ$.
}

\variantsplitter

\addpersonalvariant{Тимофей Полетаев}

\tasknumber{1}%
\task{%
    С какой силой взаимодействуют 2 точечных заряда $q_1 = 3\,\text{нКл}$ и $q_2 = 4\,\text{нКл}$,
    находящиеся на расстоянии $d = 3\,\text{см}$?
}
\answer{%
    $
        F
            = k\frac{q_1q_2}{d^2}
            = 9 \cdot 10^{9}\,\frac{\text{Н}\cdot\text{м}^{2}}{\text{Кл}^{2}} \cdot \frac{3\,\text{нКл} *4\,\text{нКл}}{\sqr{ 3\,\text{см} }}
            = 12 \cdot 10^{31}\units{Н}
              \approx {12{,}00} \cdot 10^{31}\units{Н}
    $
}
\solutionspace{120pt}

\tasknumber{2}%
\task{%
    Два одинаковых маленьких проводящих заряженных шарика находятся на расстоянии~$r$ друг от друга.
    Заряд первого равен~$-7Q$, второго~--- $+2Q$.
    Шарики приводят в соприкосновение, а после опять разводят на расстояние~$2r$.
    \begin{itemize}
        \item Каким стал заряд каждого из шариков?
        \item Определите характер (притяжение или отталкивание) и силу взаимодействия шариков до и после соприкосновения.
        \item Как изменилась сила взаимодействия шариков после соприкосновения?
    \end{itemize}
}
\answer{%
    \begin{align*}
    F &= k\frac{q_1 q_2}{\sqr{2 r}} = k\frac{(-7Q) \cdot (+2Q)}{\sqr{2 r}}, \text{отталкивание}; \\
        q'_1 = q'_2 = \frac{q_1 + q_2}2 = \frac{-7Q + +2Q}2 \implies
        F'  &= k\frac{q'_1 q'_2}{r^2}
            = k\frac{\sqr{\frac{(-7Q) + (+2Q)}2}}{2^2 \cdot r^2},
        \text{отталкивание}.
    \end{align*}
}
\solutionspace{120pt}

\tasknumber{3}%
\task{%
    На координатной плоскости в точках $(-d; 0)$ и $(d; 0)$
    находятся заряды, соответственно, $-q$ и $-q$.
    Сделайте рисунок, определите величину напряжённости электрического поля
    и укажите её направление в двух точках: $(0; d)$ и $(2d; 0)$.
}
\solutionspace{120pt}

\tasknumber{4}%
\task{%
    Заряд $q_1$ создает в точке $A$ электрическое поле
    по величине равное~$E_1=50\funits{В}{м}$,
    а $q_2$~--- $E_2=120\funits{В}{м}$.
    Угол между векторами $\vect{E_1}$ и $\vect{E_2}$ равен $\alpha$.
    Определите величину суммарного электрического поля в точке $A$,
    создаваемого обоими зарядами $q_1$ и $q_2$.
    Сделайте рисунки и вычислите значение для двух значений угла $\alpha$:
    $\alpha_1=0^\circ$ и $\alpha_2=90^\circ$.
}

\variantsplitter

\addpersonalvariant{Андрей Рожков}

\tasknumber{1}%
\task{%
    С какой силой взаимодействуют 2 точечных заряда $q_1 = 2\,\text{нКл}$ и $q_2 = 4\,\text{нКл}$,
    находящиеся на расстоянии $l = 2\,\text{см}$?
}
\answer{%
    $
        F
            = k\frac{q_1q_2}{l^2}
            = 9 \cdot 10^{9}\,\frac{\text{Н}\cdot\text{м}^{2}}{\text{Кл}^{2}} \cdot \frac{2\,\text{нКл} *4\,\text{нКл}}{\sqr{ 2\,\text{см} }}
            = 18 \cdot 10^{31}\units{Н}
              \approx {18{,}00} \cdot 10^{31}\units{Н}
    $
}
\solutionspace{120pt}

\tasknumber{2}%
\task{%
    Два одинаковых маленьких проводящих заряженных шарика находятся на расстоянии~$r$ друг от друга.
    Заряд первого равен~$+7q$, второго~--- $+8q$.
    Шарики приводят в соприкосновение, а после опять разводят на расстояние~$2r$.
    \begin{itemize}
        \item Каким стал заряд каждого из шариков?
        \item Определите характер (притяжение или отталкивание) и силу взаимодействия шариков до и после соприкосновения.
        \item Как изменилась сила взаимодействия шариков после соприкосновения?
    \end{itemize}
}
\answer{%
    \begin{align*}
    F &= k\frac{q_1 q_2}{\sqr{2 r}} = k\frac{(+7q) \cdot (+8q)}{\sqr{2 r}}, \text{отталкивание}; \\
        q'_1 = q'_2 = \frac{q_1 + q_2}2 = \frac{+7q + +8q}2 \implies
        F'  &= k\frac{q'_1 q'_2}{r^2}
            = k\frac{\sqr{\frac{(+7q) + (+8q)}2}}{2^2 \cdot r^2},
        \text{отталкивание}.
    \end{align*}
}
\solutionspace{120pt}

\tasknumber{3}%
\task{%
    На координатной плоскости в точках $(-r; 0)$ и $(r; 0)$
    находятся заряды, соответственно, $-Q$ и $+Q$.
    Сделайте рисунок, определите величину напряжённости электрического поля
    и укажите её направление в двух точках: $(0; -r)$ и $(-2r; 0)$.
}
\solutionspace{120pt}

\tasknumber{4}%
\task{%
    Заряд $q_1$ создает в точке $A$ электрическое поле
    по величине равное~$E_1=50\funits{В}{м}$,
    а $q_2$~--- $E_2=120\funits{В}{м}$.
    Угол между векторами $\vect{E_1}$ и $\vect{E_2}$ равен $\alpha$.
    Определите величину суммарного электрического поля в точке $A$,
    создаваемого обоими зарядами $q_1$ и $q_2$.
    Сделайте рисунки и вычислите значение для двух значений угла $\alpha$:
    $\alpha_1=0^\circ$ и $\alpha_2=90^\circ$.
}

\variantsplitter

\addpersonalvariant{Рената Таржиманова}

\tasknumber{1}%
\task{%
    С какой силой взаимодействуют 2 точечных заряда $q_1 = 2\,\text{нКл}$ и $q_2 = 4\,\text{нКл}$,
    находящиеся на расстоянии $l = 2\,\text{см}$?
}
\answer{%
    $
        F
            = k\frac{q_1q_2}{l^2}
            = 9 \cdot 10^{9}\,\frac{\text{Н}\cdot\text{м}^{2}}{\text{Кл}^{2}} \cdot \frac{2\,\text{нКл} *4\,\text{нКл}}{\sqr{ 2\,\text{см} }}
            = 18 \cdot 10^{31}\units{Н}
              \approx {18{,}00} \cdot 10^{31}\units{Н}
    $
}
\solutionspace{120pt}

\tasknumber{2}%
\task{%
    Два одинаковых маленьких проводящих заряженных шарика находятся на расстоянии~$r$ друг от друга.
    Заряд первого равен~$-3q$, второго~--- $+2q$.
    Шарики приводят в соприкосновение, а после опять разводят на расстояние~$3r$.
    \begin{itemize}
        \item Каким стал заряд каждого из шариков?
        \item Определите характер (притяжение или отталкивание) и силу взаимодействия шариков до и после соприкосновения.
        \item Как изменилась сила взаимодействия шариков после соприкосновения?
    \end{itemize}
}
\answer{%
    \begin{align*}
    F &= k\frac{q_1 q_2}{\sqr{3 r}} = k\frac{(-3q) \cdot (+2q)}{\sqr{3 r}}, \text{отталкивание}; \\
        q'_1 = q'_2 = \frac{q_1 + q_2}2 = \frac{-3q + +2q}2 \implies
        F'  &= k\frac{q'_1 q'_2}{r^2}
            = k\frac{\sqr{\frac{(-3q) + (+2q)}2}}{3^2 \cdot r^2},
        \text{отталкивание}.
    \end{align*}
}
\solutionspace{120pt}

\tasknumber{3}%
\task{%
    На координатной плоскости в точках $(-l; 0)$ и $(l; 0)$
    находятся заряды, соответственно, $+q$ и $-q$.
    Сделайте рисунок, определите величину напряжённости электрического поля
    и укажите её направление в двух точках: $(0; l)$ и $(-2l; 0)$.
}
\solutionspace{120pt}

\tasknumber{4}%
\task{%
    Заряд $q_1$ создает в точке $A$ электрическое поле
    по величине равное~$E_1=300\funits{В}{м}$,
    а $q_2$~--- $E_2=400\funits{В}{м}$.
    Угол между векторами $\vect{E_1}$ и $\vect{E_2}$ равен $\alpha$.
    Определите величину суммарного электрического поля в точке $A$,
    создаваемого обоими зарядами $q_1$ и $q_2$.
    Сделайте рисунки и вычислите значение для двух значений угла $\alpha$:
    $\alpha_1=0^\circ$ и $\alpha_2=90^\circ$.
}

\variantsplitter

\addpersonalvariant{Андрей Щербаков}

\tasknumber{1}%
\task{%
    С какой силой взаимодействуют 2 точечных заряда $q_1 = 4\,\text{нКл}$ и $q_2 = 2\,\text{нКл}$,
    находящиеся на расстоянии $l = 3\,\text{см}$?
}
\answer{%
    $
        F
            = k\frac{q_1q_2}{l^2}
            = 9 \cdot 10^{9}\,\frac{\text{Н}\cdot\text{м}^{2}}{\text{Кл}^{2}} \cdot \frac{4\,\text{нКл} *2\,\text{нКл}}{\sqr{ 3\,\text{см} }}
            = 8 \cdot 10^{31}\units{Н}
              \approx {8{,}00} \cdot 10^{31}\units{Н}
    $
}
\solutionspace{120pt}

\tasknumber{2}%
\task{%
    Два одинаковых маленьких проводящих заряженных шарика находятся на расстоянии~$r$ друг от друга.
    Заряд первого равен~$+3q$, второго~--- $+6q$.
    Шарики приводят в соприкосновение, а после опять разводят на расстояние~$2r$.
    \begin{itemize}
        \item Каким стал заряд каждого из шариков?
        \item Определите характер (притяжение или отталкивание) и силу взаимодействия шариков до и после соприкосновения.
        \item Как изменилась сила взаимодействия шариков после соприкосновения?
    \end{itemize}
}
\answer{%
    \begin{align*}
    F &= k\frac{q_1 q_2}{\sqr{2 r}} = k\frac{(+3q) \cdot (+6q)}{\sqr{2 r}}, \text{отталкивание}; \\
        q'_1 = q'_2 = \frac{q_1 + q_2}2 = \frac{+3q + +6q}2 \implies
        F'  &= k\frac{q'_1 q'_2}{r^2}
            = k\frac{\sqr{\frac{(+3q) + (+6q)}2}}{2^2 \cdot r^2},
        \text{отталкивание}.
    \end{align*}
}
\solutionspace{120pt}

\tasknumber{3}%
\task{%
    На координатной плоскости в точках $(-r; 0)$ и $(r; 0)$
    находятся заряды, соответственно, $-q$ и $-q$.
    Сделайте рисунок, определите величину напряжённости электрического поля
    и укажите её направление в двух точках: $(0; -r)$ и $(-2r; 0)$.
}
\solutionspace{120pt}

\tasknumber{4}%
\task{%
    Заряд $q_1$ создает в точке $A$ электрическое поле
    по величине равное~$E_1=7\funits{В}{м}$,
    а $q_2$~--- $E_2=24\funits{В}{м}$.
    Угол между векторами $\vect{E_1}$ и $\vect{E_2}$ равен $\varphi$.
    Определите величину суммарного электрического поля в точке $A$,
    создаваемого обоими зарядами $q_1$ и $q_2$.
    Сделайте рисунки и вычислите значение для двух значений угла $\varphi$:
    $\varphi_1=0^\circ$ и $\varphi_2=90^\circ$.
}

\variantsplitter

\addpersonalvariant{Михаил Ярошевский}

\tasknumber{1}%
\task{%
    С какой силой взаимодействуют 2 точечных заряда $q_1 = 2\,\text{нКл}$ и $q_2 = 3\,\text{нКл}$,
    находящиеся на расстоянии $d = 5\,\text{см}$?
}
\answer{%
    $
        F
            = k\frac{q_1q_2}{d^2}
            = 9 \cdot 10^{9}\,\frac{\text{Н}\cdot\text{м}^{2}}{\text{Кл}^{2}} \cdot \frac{2\,\text{нКл} *3\,\text{нКл}}{\sqr{ 5\,\text{см} }}
            = \frac{54}{25} \cdot 10^{31}\units{Н}
              \approx {2{,}16} \cdot 10^{31}\units{Н}
    $
}
\solutionspace{120pt}

\tasknumber{2}%
\task{%
    Два одинаковых маленьких проводящих заряженных шарика находятся на расстоянии~$r$ друг от друга.
    Заряд первого равен~$+3q$, второго~--- $+6q$.
    Шарики приводят в соприкосновение, а после опять разводят на расстояние~$2r$.
    \begin{itemize}
        \item Каким стал заряд каждого из шариков?
        \item Определите характер (притяжение или отталкивание) и силу взаимодействия шариков до и после соприкосновения.
        \item Как изменилась сила взаимодействия шариков после соприкосновения?
    \end{itemize}
}
\answer{%
    \begin{align*}
    F &= k\frac{q_1 q_2}{\sqr{2 r}} = k\frac{(+3q) \cdot (+6q)}{\sqr{2 r}}, \text{отталкивание}; \\
        q'_1 = q'_2 = \frac{q_1 + q_2}2 = \frac{+3q + +6q}2 \implies
        F'  &= k\frac{q'_1 q'_2}{r^2}
            = k\frac{\sqr{\frac{(+3q) + (+6q)}2}}{2^2 \cdot r^2},
        \text{отталкивание}.
    \end{align*}
}
\solutionspace{120pt}

\tasknumber{3}%
\task{%
    На координатной плоскости в точках $(-r; 0)$ и $(r; 0)$
    находятся заряды, соответственно, $-Q$ и $+Q$.
    Сделайте рисунок, определите величину напряжённости электрического поля
    и укажите её направление в двух точках: $(0; -r)$ и $(2r; 0)$.
}
\solutionspace{120pt}

\tasknumber{4}%
\task{%
    Заряд $q_1$ создает в точке $A$ электрическое поле
    по величине равное~$E_1=120\funits{В}{м}$,
    а $q_2$~--- $E_2=50\funits{В}{м}$.
    Угол между векторами $\vect{E_1}$ и $\vect{E_2}$ равен $\varphi$.
    Определите величину суммарного электрического поля в точке $A$,
    создаваемого обоими зарядами $q_1$ и $q_2$.
    Сделайте рисунки и вычислите значение для двух значений угла $\varphi$:
    $\varphi_1=90^\circ$ и $\varphi_2=180^\circ$.
}

\variantsplitter

\addpersonalvariant{Алексей Алимпиев}

\tasknumber{1}%
\task{%
    С какой силой взаимодействуют 2 точечных заряда $q_1 = 4\,\text{нКл}$ и $q_2 = 2\,\text{нКл}$,
    находящиеся на расстоянии $r = 2\,\text{см}$?
}
\answer{%
    $
        F
            = k\frac{q_1q_2}{r^2}
            = 9 \cdot 10^{9}\,\frac{\text{Н}\cdot\text{м}^{2}}{\text{Кл}^{2}} \cdot \frac{4\,\text{нКл} *2\,\text{нКл}}{\sqr{ 2\,\text{см} }}
            = 18 \cdot 10^{31}\units{Н}
              \approx {18{,}00} \cdot 10^{31}\units{Н}
    $
}
\solutionspace{120pt}

\tasknumber{2}%
\task{%
    Два одинаковых маленьких проводящих заряженных шарика находятся на расстоянии~$d$ друг от друга.
    Заряд первого равен~$+3Q$, второго~--- $+6Q$.
    Шарики приводят в соприкосновение, а после опять разводят на расстояние~$3d$.
    \begin{itemize}
        \item Каким стал заряд каждого из шариков?
        \item Определите характер (притяжение или отталкивание) и силу взаимодействия шариков до и после соприкосновения.
        \item Как изменилась сила взаимодействия шариков после соприкосновения?
    \end{itemize}
}
\answer{%
    \begin{align*}
    F &= k\frac{q_1 q_2}{\sqr{3 d}} = k\frac{(+3Q) \cdot (+6Q)}{\sqr{3 d}}, \text{отталкивание}; \\
        q'_1 = q'_2 = \frac{q_1 + q_2}2 = \frac{+3Q + +6Q}2 \implies
        F'  &= k\frac{q'_1 q'_2}{d^2}
            = k\frac{\sqr{\frac{(+3Q) + (+6Q)}2}}{3^2 \cdot d^2},
        \text{отталкивание}.
    \end{align*}
}
\solutionspace{120pt}

\tasknumber{3}%
\task{%
    На координатной плоскости в точках $(-r; 0)$ и $(r; 0)$
    находятся заряды, соответственно, $+Q$ и $+Q$.
    Сделайте рисунок, определите величину напряжённости электрического поля
    и укажите её направление в двух точках: $(0; r)$ и $(-2r; 0)$.
}
\solutionspace{120pt}

\tasknumber{4}%
\task{%
    Заряд $q_1$ создает в точке $A$ электрическое поле
    по величине равное~$E_1=7\funits{В}{м}$,
    а $q_2$~--- $E_2=24\funits{В}{м}$.
    Угол между векторами $\vect{E_1}$ и $\vect{E_2}$ равен $\alpha$.
    Определите величину суммарного электрического поля в точке $A$,
    создаваемого обоими зарядами $q_1$ и $q_2$.
    Сделайте рисунки и вычислите значение для двух значений угла $\alpha$:
    $\alpha_1=0^\circ$ и $\alpha_2=90^\circ$.
}

\variantsplitter

\addpersonalvariant{Евгений Васин}

\tasknumber{1}%
\task{%
    С какой силой взаимодействуют 2 точечных заряда $q_1 = 4\,\text{нКл}$ и $q_2 = 2\,\text{нКл}$,
    находящиеся на расстоянии $d = 2\,\text{см}$?
}
\answer{%
    $
        F
            = k\frac{q_1q_2}{d^2}
            = 9 \cdot 10^{9}\,\frac{\text{Н}\cdot\text{м}^{2}}{\text{Кл}^{2}} \cdot \frac{4\,\text{нКл} *2\,\text{нКл}}{\sqr{ 2\,\text{см} }}
            = 18 \cdot 10^{31}\units{Н}
              \approx {18{,}00} \cdot 10^{31}\units{Н}
    $
}
\solutionspace{120pt}

\tasknumber{2}%
\task{%
    Два одинаковых маленьких проводящих заряженных шарика находятся на расстоянии~$l$ друг от друга.
    Заряд первого равен~$+7Q$, второго~--- $-8Q$.
    Шарики приводят в соприкосновение, а после опять разводят на расстояние~$2l$.
    \begin{itemize}
        \item Каким стал заряд каждого из шариков?
        \item Определите характер (притяжение или отталкивание) и силу взаимодействия шариков до и после соприкосновения.
        \item Как изменилась сила взаимодействия шариков после соприкосновения?
    \end{itemize}
}
\answer{%
    \begin{align*}
    F &= k\frac{q_1 q_2}{\sqr{2 l}} = k\frac{(+7Q) \cdot (-8Q)}{\sqr{2 l}}, \text{отталкивание}; \\
        q'_1 = q'_2 = \frac{q_1 + q_2}2 = \frac{+7Q -8Q}2 \implies
        F'  &= k\frac{q'_1 q'_2}{l^2}
            = k\frac{\sqr{\frac{(+7Q) + (-8Q)}2}}{2^2 \cdot l^2},
        \text{отталкивание}.
    \end{align*}
}
\solutionspace{120pt}

\tasknumber{3}%
\task{%
    На координатной плоскости в точках $(-l; 0)$ и $(l; 0)$
    находятся заряды, соответственно, $-q$ и $-q$.
    Сделайте рисунок, определите величину напряжённости электрического поля
    и укажите её направление в двух точках: $(0; l)$ и $(-2l; 0)$.
}
\solutionspace{120pt}

\tasknumber{4}%
\task{%
    Заряд $q_1$ создает в точке $A$ электрическое поле
    по величине равное~$E_1=7\funits{В}{м}$,
    а $q_2$~--- $E_2=24\funits{В}{м}$.
    Угол между векторами $\vect{E_1}$ и $\vect{E_2}$ равен $\alpha$.
    Определите величину суммарного электрического поля в точке $A$,
    создаваемого обоими зарядами $q_1$ и $q_2$.
    Сделайте рисунки и вычислите значение для двух значений угла $\alpha$:
    $\alpha_1=0^\circ$ и $\alpha_2=90^\circ$.
}

\variantsplitter

\addpersonalvariant{Вячеслав Волохов}

\tasknumber{1}%
\task{%
    С какой силой взаимодействуют 2 точечных заряда $q_1 = 2\,\text{нКл}$ и $q_2 = 3\,\text{нКл}$,
    находящиеся на расстоянии $r = 5\,\text{см}$?
}
\answer{%
    $
        F
            = k\frac{q_1q_2}{r^2}
            = 9 \cdot 10^{9}\,\frac{\text{Н}\cdot\text{м}^{2}}{\text{Кл}^{2}} \cdot \frac{2\,\text{нКл} *3\,\text{нКл}}{\sqr{ 5\,\text{см} }}
            = \frac{54}{25} \cdot 10^{31}\units{Н}
              \approx {2{,}16} \cdot 10^{31}\units{Н}
    $
}
\solutionspace{120pt}

\tasknumber{2}%
\task{%
    Два одинаковых маленьких проводящих заряженных шарика находятся на расстоянии~$r$ друг от друга.
    Заряд первого равен~$+3Q$, второго~--- $-4Q$.
    Шарики приводят в соприкосновение, а после опять разводят на расстояние~$4r$.
    \begin{itemize}
        \item Каким стал заряд каждого из шариков?
        \item Определите характер (притяжение или отталкивание) и силу взаимодействия шариков до и после соприкосновения.
        \item Как изменилась сила взаимодействия шариков после соприкосновения?
    \end{itemize}
}
\answer{%
    \begin{align*}
    F &= k\frac{q_1 q_2}{\sqr{4 r}} = k\frac{(+3Q) \cdot (-4Q)}{\sqr{4 r}}, \text{отталкивание}; \\
        q'_1 = q'_2 = \frac{q_1 + q_2}2 = \frac{+3Q -4Q}2 \implies
        F'  &= k\frac{q'_1 q'_2}{r^2}
            = k\frac{\sqr{\frac{(+3Q) + (-4Q)}2}}{4^2 \cdot r^2},
        \text{отталкивание}.
    \end{align*}
}
\solutionspace{120pt}

\tasknumber{3}%
\task{%
    На координатной плоскости в точках $(-d; 0)$ и $(d; 0)$
    находятся заряды, соответственно, $-Q$ и $+Q$.
    Сделайте рисунок, определите величину напряжённости электрического поля
    и укажите её направление в двух точках: $(0; -d)$ и $(-2d; 0)$.
}
\solutionspace{120pt}

\tasknumber{4}%
\task{%
    Заряд $q_1$ создает в точке $A$ электрическое поле
    по величине равное~$E_1=120\funits{В}{м}$,
    а $q_2$~--- $E_2=50\funits{В}{м}$.
    Угол между векторами $\vect{E_1}$ и $\vect{E_2}$ равен $\varphi$.
    Определите величину суммарного электрического поля в точке $A$,
    создаваемого обоими зарядами $q_1$ и $q_2$.
    Сделайте рисунки и вычислите значение для двух значений угла $\varphi$:
    $\varphi_1=90^\circ$ и $\varphi_2=180^\circ$.
}

\variantsplitter

\addpersonalvariant{Герман Говоров}

\tasknumber{1}%
\task{%
    С какой силой взаимодействуют 2 точечных заряда $q_1 = 4\,\text{нКл}$ и $q_2 = 3\,\text{нКл}$,
    находящиеся на расстоянии $l = 3\,\text{см}$?
}
\answer{%
    $
        F
            = k\frac{q_1q_2}{l^2}
            = 9 \cdot 10^{9}\,\frac{\text{Н}\cdot\text{м}^{2}}{\text{Кл}^{2}} \cdot \frac{4\,\text{нКл} *3\,\text{нКл}}{\sqr{ 3\,\text{см} }}
            = 12 \cdot 10^{31}\units{Н}
              \approx {12{,}00} \cdot 10^{31}\units{Н}
    $
}
\solutionspace{120pt}

\tasknumber{2}%
\task{%
    Два одинаковых маленьких проводящих заряженных шарика находятся на расстоянии~$r$ друг от друга.
    Заряд первого равен~$-7q$, второго~--- $-6q$.
    Шарики приводят в соприкосновение, а после опять разводят на расстояние~$2r$.
    \begin{itemize}
        \item Каким стал заряд каждого из шариков?
        \item Определите характер (притяжение или отталкивание) и силу взаимодействия шариков до и после соприкосновения.
        \item Как изменилась сила взаимодействия шариков после соприкосновения?
    \end{itemize}
}
\answer{%
    \begin{align*}
    F &= k\frac{q_1 q_2}{\sqr{2 r}} = k\frac{(-7q) \cdot (-6q)}{\sqr{2 r}}, \text{отталкивание}; \\
        q'_1 = q'_2 = \frac{q_1 + q_2}2 = \frac{-7q -6q}2 \implies
        F'  &= k\frac{q'_1 q'_2}{r^2}
            = k\frac{\sqr{\frac{(-7q) + (-6q)}2}}{2^2 \cdot r^2},
        \text{отталкивание}.
    \end{align*}
}
\solutionspace{120pt}

\tasknumber{3}%
\task{%
    На координатной плоскости в точках $(-a; 0)$ и $(a; 0)$
    находятся заряды, соответственно, $-Q$ и $+Q$.
    Сделайте рисунок, определите величину напряжённости электрического поля
    и укажите её направление в двух точках: $(0; a)$ и $(-2a; 0)$.
}
\solutionspace{120pt}

\tasknumber{4}%
\task{%
    Заряд $q_1$ создает в точке $A$ электрическое поле
    по величине равное~$E_1=50\funits{В}{м}$,
    а $q_2$~--- $E_2=120\funits{В}{м}$.
    Угол между векторами $\vect{E_1}$ и $\vect{E_2}$ равен $\alpha$.
    Определите величину суммарного электрического поля в точке $A$,
    создаваемого обоими зарядами $q_1$ и $q_2$.
    Сделайте рисунки и вычислите значение для двух значений угла $\alpha$:
    $\alpha_1=0^\circ$ и $\alpha_2=90^\circ$.
}

\variantsplitter

\addpersonalvariant{София Журавлёва}

\tasknumber{1}%
\task{%
    С какой силой взаимодействуют 2 точечных заряда $q_1 = 4\,\text{нКл}$ и $q_2 = 3\,\text{нКл}$,
    находящиеся на расстоянии $l = 3\,\text{см}$?
}
\answer{%
    $
        F
            = k\frac{q_1q_2}{l^2}
            = 9 \cdot 10^{9}\,\frac{\text{Н}\cdot\text{м}^{2}}{\text{Кл}^{2}} \cdot \frac{4\,\text{нКл} *3\,\text{нКл}}{\sqr{ 3\,\text{см} }}
            = 12 \cdot 10^{31}\units{Н}
              \approx {12{,}00} \cdot 10^{31}\units{Н}
    $
}
\solutionspace{120pt}

\tasknumber{2}%
\task{%
    Два одинаковых маленьких проводящих заряженных шарика находятся на расстоянии~$d$ друг от друга.
    Заряд первого равен~$+5Q$, второго~--- $-4Q$.
    Шарики приводят в соприкосновение, а после опять разводят на расстояние~$3d$.
    \begin{itemize}
        \item Каким стал заряд каждого из шариков?
        \item Определите характер (притяжение или отталкивание) и силу взаимодействия шариков до и после соприкосновения.
        \item Как изменилась сила взаимодействия шариков после соприкосновения?
    \end{itemize}
}
\answer{%
    \begin{align*}
    F &= k\frac{q_1 q_2}{\sqr{3 d}} = k\frac{(+5Q) \cdot (-4Q)}{\sqr{3 d}}, \text{отталкивание}; \\
        q'_1 = q'_2 = \frac{q_1 + q_2}2 = \frac{+5Q -4Q}2 \implies
        F'  &= k\frac{q'_1 q'_2}{d^2}
            = k\frac{\sqr{\frac{(+5Q) + (-4Q)}2}}{3^2 \cdot d^2},
        \text{отталкивание}.
    \end{align*}
}
\solutionspace{120pt}

\tasknumber{3}%
\task{%
    На координатной плоскости в точках $(-d; 0)$ и $(d; 0)$
    находятся заряды, соответственно, $-q$ и $-q$.
    Сделайте рисунок, определите величину напряжённости электрического поля
    и укажите её направление в двух точках: $(0; -d)$ и $(-2d; 0)$.
}
\solutionspace{120pt}

\tasknumber{4}%
\task{%
    Заряд $q_1$ создает в точке $A$ электрическое поле
    по величине равное~$E_1=72\funits{В}{м}$,
    а $q_2$~--- $E_2=72\funits{В}{м}$.
    Угол между векторами $\vect{E_1}$ и $\vect{E_2}$ равен $\varphi$.
    Определите величину суммарного электрического поля в точке $A$,
    создаваемого обоими зарядами $q_1$ и $q_2$.
    Сделайте рисунки и вычислите значение для двух значений угла $\varphi$:
    $\varphi_1=0^\circ$ и $\varphi_2=120^\circ$.
}

\variantsplitter

\addpersonalvariant{Константин Козлов}

\tasknumber{1}%
\task{%
    С какой силой взаимодействуют 2 точечных заряда $q_1 = 4\,\text{нКл}$ и $q_2 = 2\,\text{нКл}$,
    находящиеся на расстоянии $l = 6\,\text{см}$?
}
\answer{%
    $
        F
            = k\frac{q_1q_2}{l^2}
            = 9 \cdot 10^{9}\,\frac{\text{Н}\cdot\text{м}^{2}}{\text{Кл}^{2}} \cdot \frac{4\,\text{нКл} *2\,\text{нКл}}{\sqr{ 6\,\text{см} }}
            = 2 \cdot 10^{31}\units{Н}
              \approx {2{,}00} \cdot 10^{31}\units{Н}
    $
}
\solutionspace{120pt}

\tasknumber{2}%
\task{%
    Два одинаковых маленьких проводящих заряженных шарика находятся на расстоянии~$l$ друг от друга.
    Заряд первого равен~$-3q$, второго~--- $-4q$.
    Шарики приводят в соприкосновение, а после опять разводят на расстояние~$2l$.
    \begin{itemize}
        \item Каким стал заряд каждого из шариков?
        \item Определите характер (притяжение или отталкивание) и силу взаимодействия шариков до и после соприкосновения.
        \item Как изменилась сила взаимодействия шариков после соприкосновения?
    \end{itemize}
}
\answer{%
    \begin{align*}
    F &= k\frac{q_1 q_2}{\sqr{2 l}} = k\frac{(-3q) \cdot (-4q)}{\sqr{2 l}}, \text{отталкивание}; \\
        q'_1 = q'_2 = \frac{q_1 + q_2}2 = \frac{-3q -4q}2 \implies
        F'  &= k\frac{q'_1 q'_2}{l^2}
            = k\frac{\sqr{\frac{(-3q) + (-4q)}2}}{2^2 \cdot l^2},
        \text{отталкивание}.
    \end{align*}
}
\solutionspace{120pt}

\tasknumber{3}%
\task{%
    На координатной плоскости в точках $(-a; 0)$ и $(a; 0)$
    находятся заряды, соответственно, $+q$ и $-q$.
    Сделайте рисунок, определите величину напряжённости электрического поля
    и укажите её направление в двух точках: $(0; a)$ и $(-2a; 0)$.
}
\solutionspace{120pt}

\tasknumber{4}%
\task{%
    Заряд $q_1$ создает в точке $A$ электрическое поле
    по величине равное~$E_1=300\funits{В}{м}$,
    а $q_2$~--- $E_2=400\funits{В}{м}$.
    Угол между векторами $\vect{E_1}$ и $\vect{E_2}$ равен $\alpha$.
    Определите величину суммарного электрического поля в точке $A$,
    создаваемого обоими зарядами $q_1$ и $q_2$.
    Сделайте рисунки и вычислите значение для двух значений угла $\alpha$:
    $\alpha_1=0^\circ$ и $\alpha_2=90^\circ$.
}

\variantsplitter

\addpersonalvariant{Наталья Кравченко}

\tasknumber{1}%
\task{%
    С какой силой взаимодействуют 2 точечных заряда $q_1 = 4\,\text{нКл}$ и $q_2 = 2\,\text{нКл}$,
    находящиеся на расстоянии $r = 3\,\text{см}$?
}
\answer{%
    $
        F
            = k\frac{q_1q_2}{r^2}
            = 9 \cdot 10^{9}\,\frac{\text{Н}\cdot\text{м}^{2}}{\text{Кл}^{2}} \cdot \frac{4\,\text{нКл} *2\,\text{нКл}}{\sqr{ 3\,\text{см} }}
            = 8 \cdot 10^{31}\units{Н}
              \approx {8{,}00} \cdot 10^{31}\units{Н}
    $
}
\solutionspace{120pt}

\tasknumber{2}%
\task{%
    Два одинаковых маленьких проводящих заряженных шарика находятся на расстоянии~$r$ друг от друга.
    Заряд первого равен~$-5Q$, второго~--- $-4Q$.
    Шарики приводят в соприкосновение, а после опять разводят на расстояние~$2r$.
    \begin{itemize}
        \item Каким стал заряд каждого из шариков?
        \item Определите характер (притяжение или отталкивание) и силу взаимодействия шариков до и после соприкосновения.
        \item Как изменилась сила взаимодействия шариков после соприкосновения?
    \end{itemize}
}
\answer{%
    \begin{align*}
    F &= k\frac{q_1 q_2}{\sqr{2 r}} = k\frac{(-5Q) \cdot (-4Q)}{\sqr{2 r}}, \text{отталкивание}; \\
        q'_1 = q'_2 = \frac{q_1 + q_2}2 = \frac{-5Q -4Q}2 \implies
        F'  &= k\frac{q'_1 q'_2}{r^2}
            = k\frac{\sqr{\frac{(-5Q) + (-4Q)}2}}{2^2 \cdot r^2},
        \text{отталкивание}.
    \end{align*}
}
\solutionspace{120pt}

\tasknumber{3}%
\task{%
    На координатной плоскости в точках $(-d; 0)$ и $(d; 0)$
    находятся заряды, соответственно, $+q$ и $-q$.
    Сделайте рисунок, определите величину напряжённости электрического поля
    и укажите её направление в двух точках: $(0; d)$ и $(2d; 0)$.
}
\solutionspace{120pt}

\tasknumber{4}%
\task{%
    Заряд $q_1$ создает в точке $A$ электрическое поле
    по величине равное~$E_1=50\funits{В}{м}$,
    а $q_2$~--- $E_2=120\funits{В}{м}$.
    Угол между векторами $\vect{E_1}$ и $\vect{E_2}$ равен $\varphi$.
    Определите величину суммарного электрического поля в точке $A$,
    создаваемого обоими зарядами $q_1$ и $q_2$.
    Сделайте рисунки и вычислите значение для двух значений угла $\varphi$:
    $\varphi_1=0^\circ$ и $\varphi_2=90^\circ$.
}

\variantsplitter

\addpersonalvariant{Матвей Кузьмин}

\tasknumber{1}%
\task{%
    С какой силой взаимодействуют 2 точечных заряда $q_1 = 2\,\text{нКл}$ и $q_2 = 3\,\text{нКл}$,
    находящиеся на расстоянии $r = 2\,\text{см}$?
}
\answer{%
    $
        F
            = k\frac{q_1q_2}{r^2}
            = 9 \cdot 10^{9}\,\frac{\text{Н}\cdot\text{м}^{2}}{\text{Кл}^{2}} \cdot \frac{2\,\text{нКл} *3\,\text{нКл}}{\sqr{ 2\,\text{см} }}
            = \frac{27}2 \cdot 10^{31}\units{Н}
              \approx {13{,}50} \cdot 10^{31}\units{Н}
    $
}
\solutionspace{120pt}

\tasknumber{2}%
\task{%
    Два одинаковых маленьких проводящих заряженных шарика находятся на расстоянии~$r$ друг от друга.
    Заряд первого равен~$+7q$, второго~--- $-6q$.
    Шарики приводят в соприкосновение, а после опять разводят на расстояние~$4r$.
    \begin{itemize}
        \item Каким стал заряд каждого из шариков?
        \item Определите характер (притяжение или отталкивание) и силу взаимодействия шариков до и после соприкосновения.
        \item Как изменилась сила взаимодействия шариков после соприкосновения?
    \end{itemize}
}
\answer{%
    \begin{align*}
    F &= k\frac{q_1 q_2}{\sqr{4 r}} = k\frac{(+7q) \cdot (-6q)}{\sqr{4 r}}, \text{отталкивание}; \\
        q'_1 = q'_2 = \frac{q_1 + q_2}2 = \frac{+7q -6q}2 \implies
        F'  &= k\frac{q'_1 q'_2}{r^2}
            = k\frac{\sqr{\frac{(+7q) + (-6q)}2}}{4^2 \cdot r^2},
        \text{отталкивание}.
    \end{align*}
}
\solutionspace{120pt}

\tasknumber{3}%
\task{%
    На координатной плоскости в точках $(-r; 0)$ и $(r; 0)$
    находятся заряды, соответственно, $+q$ и $-q$.
    Сделайте рисунок, определите величину напряжённости электрического поля
    и укажите её направление в двух точках: $(0; r)$ и $(2r; 0)$.
}
\solutionspace{120pt}

\tasknumber{4}%
\task{%
    Заряд $q_1$ создает в точке $A$ электрическое поле
    по величине равное~$E_1=500\funits{В}{м}$,
    а $q_2$~--- $E_2=500\funits{В}{м}$.
    Угол между векторами $\vect{E_1}$ и $\vect{E_2}$ равен $\varphi$.
    Определите величину суммарного электрического поля в точке $A$,
    создаваемого обоими зарядами $q_1$ и $q_2$.
    Сделайте рисунки и вычислите значение для двух значений угла $\varphi$:
    $\varphi_1=0^\circ$ и $\varphi_2=120^\circ$.
}

\variantsplitter

\addpersonalvariant{Сергей Малышев}

\tasknumber{1}%
\task{%
    С какой силой взаимодействуют 2 точечных заряда $q_1 = 4\,\text{нКл}$ и $q_2 = 3\,\text{нКл}$,
    находящиеся на расстоянии $d = 5\,\text{см}$?
}
\answer{%
    $
        F
            = k\frac{q_1q_2}{d^2}
            = 9 \cdot 10^{9}\,\frac{\text{Н}\cdot\text{м}^{2}}{\text{Кл}^{2}} \cdot \frac{4\,\text{нКл} *3\,\text{нКл}}{\sqr{ 5\,\text{см} }}
            = \frac{108}{25} \cdot 10^{31}\units{Н}
              \approx {4{,}32} \cdot 10^{31}\units{Н}
    $
}
\solutionspace{120pt}

\tasknumber{2}%
\task{%
    Два одинаковых маленьких проводящих заряженных шарика находятся на расстоянии~$r$ друг от друга.
    Заряд первого равен~$-5q$, второго~--- $+8q$.
    Шарики приводят в соприкосновение, а после опять разводят на расстояние~$3r$.
    \begin{itemize}
        \item Каким стал заряд каждого из шариков?
        \item Определите характер (притяжение или отталкивание) и силу взаимодействия шариков до и после соприкосновения.
        \item Как изменилась сила взаимодействия шариков после соприкосновения?
    \end{itemize}
}
\answer{%
    \begin{align*}
    F &= k\frac{q_1 q_2}{\sqr{3 r}} = k\frac{(-5q) \cdot (+8q)}{\sqr{3 r}}, \text{отталкивание}; \\
        q'_1 = q'_2 = \frac{q_1 + q_2}2 = \frac{-5q + +8q}2 \implies
        F'  &= k\frac{q'_1 q'_2}{r^2}
            = k\frac{\sqr{\frac{(-5q) + (+8q)}2}}{3^2 \cdot r^2},
        \text{отталкивание}.
    \end{align*}
}
\solutionspace{120pt}

\tasknumber{3}%
\task{%
    На координатной плоскости в точках $(-l; 0)$ и $(l; 0)$
    находятся заряды, соответственно, $-q$ и $-q$.
    Сделайте рисунок, определите величину напряжённости электрического поля
    и укажите её направление в двух точках: $(0; l)$ и $(-2l; 0)$.
}
\solutionspace{120pt}

\tasknumber{4}%
\task{%
    Заряд $q_1$ создает в точке $A$ электрическое поле
    по величине равное~$E_1=50\funits{В}{м}$,
    а $q_2$~--- $E_2=120\funits{В}{м}$.
    Угол между векторами $\vect{E_1}$ и $\vect{E_2}$ равен $\varphi$.
    Определите величину суммарного электрического поля в точке $A$,
    создаваемого обоими зарядами $q_1$ и $q_2$.
    Сделайте рисунки и вычислите значение для двух значений угла $\varphi$:
    $\varphi_1=0^\circ$ и $\varphi_2=90^\circ$.
}

\variantsplitter

\addpersonalvariant{Алина Полканова}

\tasknumber{1}%
\task{%
    С какой силой взаимодействуют 2 точечных заряда $q_1 = 4\,\text{нКл}$ и $q_2 = 3\,\text{нКл}$,
    находящиеся на расстоянии $l = 3\,\text{см}$?
}
\answer{%
    $
        F
            = k\frac{q_1q_2}{l^2}
            = 9 \cdot 10^{9}\,\frac{\text{Н}\cdot\text{м}^{2}}{\text{Кл}^{2}} \cdot \frac{4\,\text{нКл} *3\,\text{нКл}}{\sqr{ 3\,\text{см} }}
            = 12 \cdot 10^{31}\units{Н}
              \approx {12{,}00} \cdot 10^{31}\units{Н}
    $
}
\solutionspace{120pt}

\tasknumber{2}%
\task{%
    Два одинаковых маленьких проводящих заряженных шарика находятся на расстоянии~$r$ друг от друга.
    Заряд первого равен~$-3q$, второго~--- $-6q$.
    Шарики приводят в соприкосновение, а после опять разводят на расстояние~$2r$.
    \begin{itemize}
        \item Каким стал заряд каждого из шариков?
        \item Определите характер (притяжение или отталкивание) и силу взаимодействия шариков до и после соприкосновения.
        \item Как изменилась сила взаимодействия шариков после соприкосновения?
    \end{itemize}
}
\answer{%
    \begin{align*}
    F &= k\frac{q_1 q_2}{\sqr{2 r}} = k\frac{(-3q) \cdot (-6q)}{\sqr{2 r}}, \text{отталкивание}; \\
        q'_1 = q'_2 = \frac{q_1 + q_2}2 = \frac{-3q -6q}2 \implies
        F'  &= k\frac{q'_1 q'_2}{r^2}
            = k\frac{\sqr{\frac{(-3q) + (-6q)}2}}{2^2 \cdot r^2},
        \text{отталкивание}.
    \end{align*}
}
\solutionspace{120pt}

\tasknumber{3}%
\task{%
    На координатной плоскости в точках $(-a; 0)$ и $(a; 0)$
    находятся заряды, соответственно, $+q$ и $-q$.
    Сделайте рисунок, определите величину напряжённости электрического поля
    и укажите её направление в двух точках: $(0; -a)$ и $(2a; 0)$.
}
\solutionspace{120pt}

\tasknumber{4}%
\task{%
    Заряд $q_1$ создает в точке $A$ электрическое поле
    по величине равное~$E_1=200\funits{В}{м}$,
    а $q_2$~--- $E_2=200\funits{В}{м}$.
    Угол между векторами $\vect{E_1}$ и $\vect{E_2}$ равен $\alpha$.
    Определите величину суммарного электрического поля в точке $A$,
    создаваемого обоими зарядами $q_1$ и $q_2$.
    Сделайте рисунки и вычислите значение для двух значений угла $\alpha$:
    $\alpha_1=0^\circ$ и $\alpha_2=60^\circ$.
}

\variantsplitter

\addpersonalvariant{Сергей Пономарёв}

\tasknumber{1}%
\task{%
    С какой силой взаимодействуют 2 точечных заряда $q_1 = 2\,\text{нКл}$ и $q_2 = 4\,\text{нКл}$,
    находящиеся на расстоянии $d = 2\,\text{см}$?
}
\answer{%
    $
        F
            = k\frac{q_1q_2}{d^2}
            = 9 \cdot 10^{9}\,\frac{\text{Н}\cdot\text{м}^{2}}{\text{Кл}^{2}} \cdot \frac{2\,\text{нКл} *4\,\text{нКл}}{\sqr{ 2\,\text{см} }}
            = 18 \cdot 10^{31}\units{Н}
              \approx {18{,}00} \cdot 10^{31}\units{Н}
    $
}
\solutionspace{120pt}

\tasknumber{2}%
\task{%
    Два одинаковых маленьких проводящих заряженных шарика находятся на расстоянии~$r$ друг от друга.
    Заряд первого равен~$+7Q$, второго~--- $-8Q$.
    Шарики приводят в соприкосновение, а после опять разводят на расстояние~$4r$.
    \begin{itemize}
        \item Каким стал заряд каждого из шариков?
        \item Определите характер (притяжение или отталкивание) и силу взаимодействия шариков до и после соприкосновения.
        \item Как изменилась сила взаимодействия шариков после соприкосновения?
    \end{itemize}
}
\answer{%
    \begin{align*}
    F &= k\frac{q_1 q_2}{\sqr{4 r}} = k\frac{(+7Q) \cdot (-8Q)}{\sqr{4 r}}, \text{отталкивание}; \\
        q'_1 = q'_2 = \frac{q_1 + q_2}2 = \frac{+7Q -8Q}2 \implies
        F'  &= k\frac{q'_1 q'_2}{r^2}
            = k\frac{\sqr{\frac{(+7Q) + (-8Q)}2}}{4^2 \cdot r^2},
        \text{отталкивание}.
    \end{align*}
}
\solutionspace{120pt}

\tasknumber{3}%
\task{%
    На координатной плоскости в точках $(-l; 0)$ и $(l; 0)$
    находятся заряды, соответственно, $-q$ и $-q$.
    Сделайте рисунок, определите величину напряжённости электрического поля
    и укажите её направление в двух точках: $(0; -l)$ и $(2l; 0)$.
}
\solutionspace{120pt}

\tasknumber{4}%
\task{%
    Заряд $q_1$ создает в точке $A$ электрическое поле
    по величине равное~$E_1=24\funits{В}{м}$,
    а $q_2$~--- $E_2=7\funits{В}{м}$.
    Угол между векторами $\vect{E_1}$ и $\vect{E_2}$ равен $\alpha$.
    Определите величину суммарного электрического поля в точке $A$,
    создаваемого обоими зарядами $q_1$ и $q_2$.
    Сделайте рисунки и вычислите значение для двух значений угла $\alpha$:
    $\alpha_1=90^\circ$ и $\alpha_2=180^\circ$.
}

\variantsplitter

\addpersonalvariant{Егор Свистушкин}

\tasknumber{1}%
\task{%
    С какой силой взаимодействуют 2 точечных заряда $q_1 = 2\,\text{нКл}$ и $q_2 = 4\,\text{нКл}$,
    находящиеся на расстоянии $l = 6\,\text{см}$?
}
\answer{%
    $
        F
            = k\frac{q_1q_2}{l^2}
            = 9 \cdot 10^{9}\,\frac{\text{Н}\cdot\text{м}^{2}}{\text{Кл}^{2}} \cdot \frac{2\,\text{нКл} *4\,\text{нКл}}{\sqr{ 6\,\text{см} }}
            = 2 \cdot 10^{31}\units{Н}
              \approx {2{,}00} \cdot 10^{31}\units{Н}
    $
}
\solutionspace{120pt}

\tasknumber{2}%
\task{%
    Два одинаковых маленьких проводящих заряженных шарика находятся на расстоянии~$d$ друг от друга.
    Заряд первого равен~$-3Q$, второго~--- $-4Q$.
    Шарики приводят в соприкосновение, а после опять разводят на расстояние~$3d$.
    \begin{itemize}
        \item Каким стал заряд каждого из шариков?
        \item Определите характер (притяжение или отталкивание) и силу взаимодействия шариков до и после соприкосновения.
        \item Как изменилась сила взаимодействия шариков после соприкосновения?
    \end{itemize}
}
\answer{%
    \begin{align*}
    F &= k\frac{q_1 q_2}{\sqr{3 d}} = k\frac{(-3Q) \cdot (-4Q)}{\sqr{3 d}}, \text{отталкивание}; \\
        q'_1 = q'_2 = \frac{q_1 + q_2}2 = \frac{-3Q -4Q}2 \implies
        F'  &= k\frac{q'_1 q'_2}{d^2}
            = k\frac{\sqr{\frac{(-3Q) + (-4Q)}2}}{3^2 \cdot d^2},
        \text{отталкивание}.
    \end{align*}
}
\solutionspace{120pt}

\tasknumber{3}%
\task{%
    На координатной плоскости в точках $(-a; 0)$ и $(a; 0)$
    находятся заряды, соответственно, $-Q$ и $+Q$.
    Сделайте рисунок, определите величину напряжённости электрического поля
    и укажите её направление в двух точках: $(0; a)$ и $(-2a; 0)$.
}
\solutionspace{120pt}

\tasknumber{4}%
\task{%
    Заряд $q_1$ создает в точке $A$ электрическое поле
    по величине равное~$E_1=300\funits{В}{м}$,
    а $q_2$~--- $E_2=400\funits{В}{м}$.
    Угол между векторами $\vect{E_1}$ и $\vect{E_2}$ равен $\alpha$.
    Определите величину суммарного электрического поля в точке $A$,
    создаваемого обоими зарядами $q_1$ и $q_2$.
    Сделайте рисунки и вычислите значение для двух значений угла $\alpha$:
    $\alpha_1=90^\circ$ и $\alpha_2=180^\circ$.
}

\variantsplitter

\addpersonalvariant{Дмитрий Соколов}

\tasknumber{1}%
\task{%
    С какой силой взаимодействуют 2 точечных заряда $q_1 = 2\,\text{нКл}$ и $q_2 = 4\,\text{нКл}$,
    находящиеся на расстоянии $l = 3\,\text{см}$?
}
\answer{%
    $
        F
            = k\frac{q_1q_2}{l^2}
            = 9 \cdot 10^{9}\,\frac{\text{Н}\cdot\text{м}^{2}}{\text{Кл}^{2}} \cdot \frac{2\,\text{нКл} *4\,\text{нКл}}{\sqr{ 3\,\text{см} }}
            = 8 \cdot 10^{31}\units{Н}
              \approx {8{,}00} \cdot 10^{31}\units{Н}
    $
}
\solutionspace{120pt}

\tasknumber{2}%
\task{%
    Два одинаковых маленьких проводящих заряженных шарика находятся на расстоянии~$d$ друг от друга.
    Заряд первого равен~$-7q$, второго~--- $-4q$.
    Шарики приводят в соприкосновение, а после опять разводят на расстояние~$2d$.
    \begin{itemize}
        \item Каким стал заряд каждого из шариков?
        \item Определите характер (притяжение или отталкивание) и силу взаимодействия шариков до и после соприкосновения.
        \item Как изменилась сила взаимодействия шариков после соприкосновения?
    \end{itemize}
}
\answer{%
    \begin{align*}
    F &= k\frac{q_1 q_2}{\sqr{2 d}} = k\frac{(-7q) \cdot (-4q)}{\sqr{2 d}}, \text{отталкивание}; \\
        q'_1 = q'_2 = \frac{q_1 + q_2}2 = \frac{-7q -4q}2 \implies
        F'  &= k\frac{q'_1 q'_2}{d^2}
            = k\frac{\sqr{\frac{(-7q) + (-4q)}2}}{2^2 \cdot d^2},
        \text{отталкивание}.
    \end{align*}
}
\solutionspace{120pt}

\tasknumber{3}%
\task{%
    На координатной плоскости в точках $(-r; 0)$ и $(r; 0)$
    находятся заряды, соответственно, $+q$ и $-q$.
    Сделайте рисунок, определите величину напряжённости электрического поля
    и укажите её направление в двух точках: $(0; -r)$ и $(-2r; 0)$.
}
\solutionspace{120pt}

\tasknumber{4}%
\task{%
    Заряд $q_1$ создает в точке $A$ электрическое поле
    по величине равное~$E_1=50\funits{В}{м}$,
    а $q_2$~--- $E_2=120\funits{В}{м}$.
    Угол между векторами $\vect{E_1}$ и $\vect{E_2}$ равен $\varphi$.
    Определите величину суммарного электрического поля в точке $A$,
    создаваемого обоими зарядами $q_1$ и $q_2$.
    Сделайте рисунки и вычислите значение для двух значений угла $\varphi$:
    $\varphi_1=0^\circ$ и $\varphi_2=90^\circ$.
}

\variantsplitter

\addpersonalvariant{Арсений Трофимов}

\tasknumber{1}%
\task{%
    С какой силой взаимодействуют 2 точечных заряда $q_1 = 4\,\text{нКл}$ и $q_2 = 3\,\text{нКл}$,
    находящиеся на расстоянии $d = 6\,\text{см}$?
}
\answer{%
    $
        F
            = k\frac{q_1q_2}{d^2}
            = 9 \cdot 10^{9}\,\frac{\text{Н}\cdot\text{м}^{2}}{\text{Кл}^{2}} \cdot \frac{4\,\text{нКл} *3\,\text{нКл}}{\sqr{ 6\,\text{см} }}
            = 3 \cdot 10^{31}\units{Н}
              \approx {3{,}00} \cdot 10^{31}\units{Н}
    $
}
\solutionspace{120pt}

\tasknumber{2}%
\task{%
    Два одинаковых маленьких проводящих заряженных шарика находятся на расстоянии~$r$ друг от друга.
    Заряд первого равен~$-3q$, второго~--- $+4q$.
    Шарики приводят в соприкосновение, а после опять разводят на расстояние~$3r$.
    \begin{itemize}
        \item Каким стал заряд каждого из шариков?
        \item Определите характер (притяжение или отталкивание) и силу взаимодействия шариков до и после соприкосновения.
        \item Как изменилась сила взаимодействия шариков после соприкосновения?
    \end{itemize}
}
\answer{%
    \begin{align*}
    F &= k\frac{q_1 q_2}{\sqr{3 r}} = k\frac{(-3q) \cdot (+4q)}{\sqr{3 r}}, \text{отталкивание}; \\
        q'_1 = q'_2 = \frac{q_1 + q_2}2 = \frac{-3q + +4q}2 \implies
        F'  &= k\frac{q'_1 q'_2}{r^2}
            = k\frac{\sqr{\frac{(-3q) + (+4q)}2}}{3^2 \cdot r^2},
        \text{отталкивание}.
    \end{align*}
}
\solutionspace{120pt}

\tasknumber{3}%
\task{%
    На координатной плоскости в точках $(-d; 0)$ и $(d; 0)$
    находятся заряды, соответственно, $+q$ и $-q$.
    Сделайте рисунок, определите величину напряжённости электрического поля
    и укажите её направление в двух точках: $(0; -d)$ и $(2d; 0)$.
}
\solutionspace{120pt}

\tasknumber{4}%
\task{%
    Заряд $q_1$ создает в точке $A$ электрическое поле
    по величине равное~$E_1=120\funits{В}{м}$,
    а $q_2$~--- $E_2=50\funits{В}{м}$.
    Угол между векторами $\vect{E_1}$ и $\vect{E_2}$ равен $\alpha$.
    Определите величину суммарного электрического поля в точке $A$,
    создаваемого обоими зарядами $q_1$ и $q_2$.
    Сделайте рисунки и вычислите значение для двух значений угла $\alpha$:
    $\alpha_1=90^\circ$ и $\alpha_2=180^\circ$.
}
% autogenerated
