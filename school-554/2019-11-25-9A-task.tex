\setdate{25~ноября~2019}
\setclass{9«А»}

\addpersonalvariant{Михаил Бурмистров}

\tasknumber{1}%
\task{%
    Шарики массами $1\,\text{кг}$ и $4\,\text{кг}$
    движутся параллельно друг другу в одном направлении
    со скоростями $5\,\frac{\text{м}}{\text{с}}$ и $6\,\frac{\text{м}}{\text{с}}$ соответственно.
    Сделайте рисунок и укажите направления скоростей и импульсов.
    Определите импульс каждого из шариков, а также их суммарный импульс.
}
\answer{%
    \begin{align*}
    p_1 &= m_1v_1 = 1\,\text{кг} \cdot 5\,\frac{\text{м}}{\text{с}} = 5\,\frac{\text{кг}\cdot\text{м}}{\text{с}}, \\
    p_2 &= m_2v_2 = 4\,\text{кг} \cdot 6\,\frac{\text{м}}{\text{с}} = 24\,\frac{\text{кг}\cdot\text{м}}{\text{с}}, \\
    p &= p_1 + p_2 = m_1v_1 + m_2v_2 = 29\,\frac{\text{кг}\cdot\text{м}}{\text{с}}.
    \end{align*}
}
\solutionspace{120pt}

\tasknumber{2}%
\task{%
    Два шарика, масса каждого из которых составляет $2\,\text{кг}$,
    движутся навстречу друг другу.
    Скорость одного из них $5\,\frac{\text{м}}{\text{с}}$, а другого~--- $3\,\frac{\text{м}}{\text{с}}$.
    Сделайте рисунок, укажите направления скоростей и импульсов.
    Определите импульс каждого из шариков, а также их суммарный импульс.
}
\answer{%
    \begin{align*}
    p_1 &= mv_1 = 2\,\text{кг} \cdot 5\,\frac{\text{м}}{\text{с}} = 10\,\frac{\text{кг}\cdot\text{м}}{\text{с}}, \\
    p_2 &= mv_2 = 2\,\text{кг} \cdot 3\,\frac{\text{м}}{\text{с}} = 6\,\frac{\text{кг}\cdot\text{м}}{\text{с}}, \\
    p &= \abs{p_1 - p_2} = \abs{m(v_1 - v_2)}= 4\,\frac{\text{кг}\cdot\text{м}}{\text{с}}.
    \end{align*}
}
\solutionspace{120pt}

\tasknumber{3}%
\task{%
    Два одинаковых шарика массами по $2\,\text{кг}$
    движутся во взаимно перпендикулярных направлениях.
    Скорости шариков составляют $3\,\frac{\text{м}}{\text{с}}$ и $4\,\frac{\text{м}}{\text{с}}$.
    Сделайте рисунок, укажите направления скоростей и импульсов.
    Определите импульс каждого из шариков и полный импульс системы.
}
\answer{%
    \begin{align*}
    p_1 &= mv_1 = 2\,\text{кг} \cdot 3\,\frac{\text{м}}{\text{с}} = 6\,\frac{\text{кг}\cdot\text{м}}{\text{с}}, \\
    p_2 &= mv_2 = 2\,\text{кг} \cdot 4\,\frac{\text{м}}{\text{с}} = 8\,\frac{\text{кг}\cdot\text{м}}{\text{с}}, \\
    p &= \sqrt{p_1^2 + p_2^2} = m\sqrt{v_1^2 + v_2^2} = 10\,\frac{\text{кг}\cdot\text{м}}{\text{с}}.
    \end{align*}
}
\solutionspace{120pt}

\tasknumber{4}%
\task{%
    Паровоз массой $M = 150\,\text{т}$, скорость которого равна $v = 0{,}20\,\frac{\text{м}}{\text{с}}$,
    сталкивается с четыремя неподвижными вагонами массой $m = 40\,\text{т}$ каждый и сцепляется с ними.
    Запишите (формулами, не числами) импульсы каждого из тел до и после сцепки и после,
    а также определите скорость их совместного движения.
}
\answer{%
    \begin{align*}
    \text{ЗСИ:} &M \cdot v + 4 \cdot \cbr{m \cdot 0} =  M \cdot u + 4 \cdot \cbr{m \cdot u} \implies \\
    &\implies u = v \cdot \frac{M}{M + nm} = 0{,}20\,\frac{\text{м}}{\text{с}} \cdot  \frac{ 150\,\text{т} }{150\,\text{т} + 4 \cdot 40\,\text{т}} \approx 0{,}10\,\frac{\text{м}}{\text{с}}.
    \end{align*}
}
\solutionspace{120pt}

\tasknumber{5}%
\task{%
    Два тела двигаются навстречу друг другу.
    Скорость каждого из них составляет $9\,\frac{\text{м}}{\text{с}}$.
    После соударения тела слиплись и продолжили движение уже со скоростью $7\,\frac{\text{м}}{\text{с}}$.
    Определите отношение масс тел (большей к меньшей).
}
\answer{%
    \begin{align*}
    &\text{ЗСИ в проекции на ось, соединяющую центры тел:} m_1 v_1 - m_2 v_1 = (m_1 + m_2) v_2 \implies \\
    &\implies \frac{m_1}{m_2} v_1 - v_1 = \cbr{\frac{m_1}{m_2} + 1} v_2 \implies
        \frac{m_1}{m_2} (v_1 - v_2) = v_2 + v_1 \implies \frac{m_1}{m_2} = \frac{v_2 + v_1}{v_1 - v_2} = 8
    \end{align*}
}

\variantsplitter

\addpersonalvariant{Ирина Ан}

\tasknumber{1}%
\task{%
    Шарики массами $2\,\text{кг}$ и $3\,\text{кг}$
    движутся параллельно друг другу в одном направлении
    со скоростями $2\,\frac{\text{м}}{\text{с}}$ и $3\,\frac{\text{м}}{\text{с}}$ соответственно.
    Сделайте рисунок и укажите направления скоростей и импульсов.
    Определите импульс каждого из шариков, а также их суммарный импульс.
}
\answer{%
    \begin{align*}
    p_1 &= m_1v_1 = 2\,\text{кг} \cdot 2\,\frac{\text{м}}{\text{с}} = 4\,\frac{\text{кг}\cdot\text{м}}{\text{с}}, \\
    p_2 &= m_2v_2 = 3\,\text{кг} \cdot 3\,\frac{\text{м}}{\text{с}} = 9\,\frac{\text{кг}\cdot\text{м}}{\text{с}}, \\
    p &= p_1 + p_2 = m_1v_1 + m_2v_2 = 13\,\frac{\text{кг}\cdot\text{м}}{\text{с}}.
    \end{align*}
}
\solutionspace{120pt}

\tasknumber{2}%
\task{%
    Два шарика, масса каждого из которых составляет $5\,\text{кг}$,
    движутся навстречу друг другу.
    Скорость одного из них $2\,\frac{\text{м}}{\text{с}}$, а другого~--- $3\,\frac{\text{м}}{\text{с}}$.
    Сделайте рисунок, укажите направления скоростей и импульсов.
    Определите импульс каждого из шариков, а также их суммарный импульс.
}
\answer{%
    \begin{align*}
    p_1 &= mv_1 = 5\,\text{кг} \cdot 2\,\frac{\text{м}}{\text{с}} = 10\,\frac{\text{кг}\cdot\text{м}}{\text{с}}, \\
    p_2 &= mv_2 = 5\,\text{кг} \cdot 3\,\frac{\text{м}}{\text{с}} = 15\,\frac{\text{кг}\cdot\text{м}}{\text{с}}, \\
    p &= \abs{p_1 - p_2} = \abs{m(v_1 - v_2)}= 5\,\frac{\text{кг}\cdot\text{м}}{\text{с}}.
    \end{align*}
}
\solutionspace{120pt}

\tasknumber{3}%
\task{%
    Два одинаковых шарика массами по $10\,\text{кг}$
    движутся во взаимно перпендикулярных направлениях.
    Скорости шариков составляют $3\,\frac{\text{м}}{\text{с}}$ и $4\,\frac{\text{м}}{\text{с}}$.
    Сделайте рисунок, укажите направления скоростей и импульсов.
    Определите импульс каждого из шариков и полный импульс системы.
}
\answer{%
    \begin{align*}
    p_1 &= mv_1 = 10\,\text{кг} \cdot 3\,\frac{\text{м}}{\text{с}} = 30\,\frac{\text{кг}\cdot\text{м}}{\text{с}}, \\
    p_2 &= mv_2 = 10\,\text{кг} \cdot 4\,\frac{\text{м}}{\text{с}} = 40\,\frac{\text{кг}\cdot\text{м}}{\text{с}}, \\
    p &= \sqrt{p_1^2 + p_2^2} = m\sqrt{v_1^2 + v_2^2} = 50\,\frac{\text{кг}\cdot\text{м}}{\text{с}}.
    \end{align*}
}
\solutionspace{120pt}

\tasknumber{4}%
\task{%
    Паровоз массой $M = 150\,\text{т}$, скорость которого равна $v = 0{,}20\,\frac{\text{м}}{\text{с}}$,
    сталкивается с тремя неподвижными вагонами массой $m = 40\,\text{т}$ каждый и сцепляется с ними.
    Запишите (формулами, не числами) импульсы каждого из тел до и после сцепки и после,
    а также определите скорость их совместного движения.
}
\answer{%
    \begin{align*}
    \text{ЗСИ:} &M \cdot v + 3 \cdot \cbr{m \cdot 0} =  M \cdot u + 3 \cdot \cbr{m \cdot u} \implies \\
    &\implies u = v \cdot \frac{M}{M + nm} = 0{,}20\,\frac{\text{м}}{\text{с}} \cdot  \frac{ 150\,\text{т} }{150\,\text{т} + 3 \cdot 40\,\text{т}} \approx 0{,}11\,\frac{\text{м}}{\text{с}}.
    \end{align*}
}
\solutionspace{120pt}

\tasknumber{5}%
\task{%
    Два тела двигаются навстречу друг другу.
    Скорость каждого из них составляет $6\,\frac{\text{м}}{\text{с}}$.
    После соударения тела слиплись и продолжили движение уже со скоростью $3\,\frac{\text{м}}{\text{с}}$.
    Определите отношение масс тел (большей к меньшей).
}
\answer{%
    \begin{align*}
    &\text{ЗСИ в проекции на ось, соединяющую центры тел:} m_1 v_1 - m_2 v_1 = (m_1 + m_2) v_2 \implies \\
    &\implies \frac{m_1}{m_2} v_1 - v_1 = \cbr{\frac{m_1}{m_2} + 1} v_2 \implies
        \frac{m_1}{m_2} (v_1 - v_2) = v_2 + v_1 \implies \frac{m_1}{m_2} = \frac{v_2 + v_1}{v_1 - v_2} = 3
    \end{align*}
}

\variantsplitter

\addpersonalvariant{Софья Андрианова}

\tasknumber{1}%
\task{%
    Шарики массами $4\,\text{кг}$ и $1\,\text{кг}$
    движутся параллельно друг другу в одном направлении
    со скоростями $4\,\frac{\text{м}}{\text{с}}$ и $3\,\frac{\text{м}}{\text{с}}$ соответственно.
    Сделайте рисунок и укажите направления скоростей и импульсов.
    Определите импульс каждого из шариков, а также их суммарный импульс.
}
\answer{%
    \begin{align*}
    p_1 &= m_1v_1 = 4\,\text{кг} \cdot 4\,\frac{\text{м}}{\text{с}} = 16\,\frac{\text{кг}\cdot\text{м}}{\text{с}}, \\
    p_2 &= m_2v_2 = 1\,\text{кг} \cdot 3\,\frac{\text{м}}{\text{с}} = 3\,\frac{\text{кг}\cdot\text{м}}{\text{с}}, \\
    p &= p_1 + p_2 = m_1v_1 + m_2v_2 = 19\,\frac{\text{кг}\cdot\text{м}}{\text{с}}.
    \end{align*}
}
\solutionspace{120pt}

\tasknumber{2}%
\task{%
    Два шарика, масса каждого из которых составляет $10\,\text{кг}$,
    движутся навстречу друг другу.
    Скорость одного из них $10\,\frac{\text{м}}{\text{с}}$, а другого~--- $3\,\frac{\text{м}}{\text{с}}$.
    Сделайте рисунок, укажите направления скоростей и импульсов.
    Определите импульс каждого из шариков, а также их суммарный импульс.
}
\answer{%
    \begin{align*}
    p_1 &= mv_1 = 10\,\text{кг} \cdot 10\,\frac{\text{м}}{\text{с}} = 100\,\frac{\text{кг}\cdot\text{м}}{\text{с}}, \\
    p_2 &= mv_2 = 10\,\text{кг} \cdot 3\,\frac{\text{м}}{\text{с}} = 30\,\frac{\text{кг}\cdot\text{м}}{\text{с}}, \\
    p &= \abs{p_1 - p_2} = \abs{m(v_1 - v_2)}= 70\,\frac{\text{кг}\cdot\text{м}}{\text{с}}.
    \end{align*}
}
\solutionspace{120pt}

\tasknumber{3}%
\task{%
    Два одинаковых шарика массами по $2\,\text{кг}$
    движутся во взаимно перпендикулярных направлениях.
    Скорости шариков составляют $7\,\frac{\text{м}}{\text{с}}$ и $24\,\frac{\text{м}}{\text{с}}$.
    Сделайте рисунок, укажите направления скоростей и импульсов.
    Определите импульс каждого из шариков и полный импульс системы.
}
\answer{%
    \begin{align*}
    p_1 &= mv_1 = 2\,\text{кг} \cdot 7\,\frac{\text{м}}{\text{с}} = 14\,\frac{\text{кг}\cdot\text{м}}{\text{с}}, \\
    p_2 &= mv_2 = 2\,\text{кг} \cdot 24\,\frac{\text{м}}{\text{с}} = 48\,\frac{\text{кг}\cdot\text{м}}{\text{с}}, \\
    p &= \sqrt{p_1^2 + p_2^2} = m\sqrt{v_1^2 + v_2^2} = 50\,\frac{\text{кг}\cdot\text{м}}{\text{с}}.
    \end{align*}
}
\solutionspace{120pt}

\tasknumber{4}%
\task{%
    Паровоз массой $M = 120\,\text{т}$, скорость которого равна $v = 0{,}20\,\frac{\text{м}}{\text{с}}$,
    сталкивается с двумя неподвижными вагонами массой $m = 50\,\text{т}$ каждый и сцепляется с ними.
    Запишите (формулами, не числами) импульсы каждого из тел до и после сцепки и после,
    а также определите скорость их совместного движения.
}
\answer{%
    \begin{align*}
    \text{ЗСИ:} &M \cdot v + 2 \cdot \cbr{m \cdot 0} =  M \cdot u + 2 \cdot \cbr{m \cdot u} \implies \\
    &\implies u = v \cdot \frac{M}{M + nm} = 0{,}20\,\frac{\text{м}}{\text{с}} \cdot  \frac{ 120\,\text{т} }{120\,\text{т} + 2 \cdot 50\,\text{т}} \approx 0{,}11\,\frac{\text{м}}{\text{с}}.
    \end{align*}
}
\solutionspace{120pt}

\tasknumber{5}%
\task{%
    Два тела двигаются навстречу друг другу.
    Скорость каждого из них составляет $2\,\frac{\text{м}}{\text{с}}$.
    После соударения тела слиплись и продолжили движение уже со скоростью $1\,\frac{\text{м}}{\text{с}}$.
    Определите отношение масс тел (большей к меньшей).
}
\answer{%
    \begin{align*}
    &\text{ЗСИ в проекции на ось, соединяющую центры тел:} m_1 v_1 - m_2 v_1 = (m_1 + m_2) v_2 \implies \\
    &\implies \frac{m_1}{m_2} v_1 - v_1 = \cbr{\frac{m_1}{m_2} + 1} v_2 \implies
        \frac{m_1}{m_2} (v_1 - v_2) = v_2 + v_1 \implies \frac{m_1}{m_2} = \frac{v_2 + v_1}{v_1 - v_2} = 3
    \end{align*}
}

\variantsplitter

\addpersonalvariant{Владимир Артемчук}

\tasknumber{1}%
\task{%
    Шарики массами $2\,\text{кг}$ и $3\,\text{кг}$
    движутся параллельно друг другу в одном направлении
    со скоростями $5\,\frac{\text{м}}{\text{с}}$ и $3\,\frac{\text{м}}{\text{с}}$ соответственно.
    Сделайте рисунок и укажите направления скоростей и импульсов.
    Определите импульс каждого из шариков, а также их суммарный импульс.
}
\answer{%
    \begin{align*}
    p_1 &= m_1v_1 = 2\,\text{кг} \cdot 5\,\frac{\text{м}}{\text{с}} = 10\,\frac{\text{кг}\cdot\text{м}}{\text{с}}, \\
    p_2 &= m_2v_2 = 3\,\text{кг} \cdot 3\,\frac{\text{м}}{\text{с}} = 9\,\frac{\text{кг}\cdot\text{м}}{\text{с}}, \\
    p &= p_1 + p_2 = m_1v_1 + m_2v_2 = 19\,\frac{\text{кг}\cdot\text{м}}{\text{с}}.
    \end{align*}
}
\solutionspace{120pt}

\tasknumber{2}%
\task{%
    Два шарика, масса каждого из которых составляет $5\,\text{кг}$,
    движутся навстречу друг другу.
    Скорость одного из них $2\,\frac{\text{м}}{\text{с}}$, а другого~--- $3\,\frac{\text{м}}{\text{с}}$.
    Сделайте рисунок, укажите направления скоростей и импульсов.
    Определите импульс каждого из шариков, а также их суммарный импульс.
}
\answer{%
    \begin{align*}
    p_1 &= mv_1 = 5\,\text{кг} \cdot 2\,\frac{\text{м}}{\text{с}} = 10\,\frac{\text{кг}\cdot\text{м}}{\text{с}}, \\
    p_2 &= mv_2 = 5\,\text{кг} \cdot 3\,\frac{\text{м}}{\text{с}} = 15\,\frac{\text{кг}\cdot\text{м}}{\text{с}}, \\
    p &= \abs{p_1 - p_2} = \abs{m(v_1 - v_2)}= 5\,\frac{\text{кг}\cdot\text{м}}{\text{с}}.
    \end{align*}
}
\solutionspace{120pt}

\tasknumber{3}%
\task{%
    Два одинаковых шарика массами по $2\,\text{кг}$
    движутся во взаимно перпендикулярных направлениях.
    Скорости шариков составляют $7\,\frac{\text{м}}{\text{с}}$ и $24\,\frac{\text{м}}{\text{с}}$.
    Сделайте рисунок, укажите направления скоростей и импульсов.
    Определите импульс каждого из шариков и полный импульс системы.
}
\answer{%
    \begin{align*}
    p_1 &= mv_1 = 2\,\text{кг} \cdot 7\,\frac{\text{м}}{\text{с}} = 14\,\frac{\text{кг}\cdot\text{м}}{\text{с}}, \\
    p_2 &= mv_2 = 2\,\text{кг} \cdot 24\,\frac{\text{м}}{\text{с}} = 48\,\frac{\text{кг}\cdot\text{м}}{\text{с}}, \\
    p &= \sqrt{p_1^2 + p_2^2} = m\sqrt{v_1^2 + v_2^2} = 50\,\frac{\text{кг}\cdot\text{м}}{\text{с}}.
    \end{align*}
}
\solutionspace{120pt}

\tasknumber{4}%
\task{%
    Паровоз массой $M = 210\,\text{т}$, скорость которого равна $v = 0{,}20\,\frac{\text{м}}{\text{с}}$,
    сталкивается с четыремя неподвижными вагонами массой $m = 30\,\text{т}$ каждый и сцепляется с ними.
    Запишите (формулами, не числами) импульсы каждого из тел до и после сцепки и после,
    а также определите скорость их совместного движения.
}
\answer{%
    \begin{align*}
    \text{ЗСИ:} &M \cdot v + 4 \cdot \cbr{m \cdot 0} =  M \cdot u + 4 \cdot \cbr{m \cdot u} \implies \\
    &\implies u = v \cdot \frac{M}{M + nm} = 0{,}20\,\frac{\text{м}}{\text{с}} \cdot  \frac{ 210\,\text{т} }{210\,\text{т} + 4 \cdot 30\,\text{т}} \approx 0{,}13\,\frac{\text{м}}{\text{с}}.
    \end{align*}
}
\solutionspace{120pt}

\tasknumber{5}%
\task{%
    Два тела двигаются навстречу друг другу.
    Скорость каждого из них составляет $7\,\frac{\text{м}}{\text{с}}$.
    После соударения тела слиплись и продолжили движение уже со скоростью $5\,\frac{\text{м}}{\text{с}}$.
    Определите отношение масс тел (большей к меньшей).
}
\answer{%
    \begin{align*}
    &\text{ЗСИ в проекции на ось, соединяющую центры тел:} m_1 v_1 - m_2 v_1 = (m_1 + m_2) v_2 \implies \\
    &\implies \frac{m_1}{m_2} v_1 - v_1 = \cbr{\frac{m_1}{m_2} + 1} v_2 \implies
        \frac{m_1}{m_2} (v_1 - v_2) = v_2 + v_1 \implies \frac{m_1}{m_2} = \frac{v_2 + v_1}{v_1 - v_2} = 6
    \end{align*}
}

\variantsplitter

\addpersonalvariant{Софья Белянкина}

\tasknumber{1}%
\task{%
    Шарики массами $2\,\text{кг}$ и $3\,\text{кг}$
    движутся параллельно друг другу в одном направлении
    со скоростями $10\,\frac{\text{м}}{\text{с}}$ и $6\,\frac{\text{м}}{\text{с}}$ соответственно.
    Сделайте рисунок и укажите направления скоростей и импульсов.
    Определите импульс каждого из шариков, а также их суммарный импульс.
}
\answer{%
    \begin{align*}
    p_1 &= m_1v_1 = 2\,\text{кг} \cdot 10\,\frac{\text{м}}{\text{с}} = 20\,\frac{\text{кг}\cdot\text{м}}{\text{с}}, \\
    p_2 &= m_2v_2 = 3\,\text{кг} \cdot 6\,\frac{\text{м}}{\text{с}} = 18\,\frac{\text{кг}\cdot\text{м}}{\text{с}}, \\
    p &= p_1 + p_2 = m_1v_1 + m_2v_2 = 38\,\frac{\text{кг}\cdot\text{м}}{\text{с}}.
    \end{align*}
}
\solutionspace{120pt}

\tasknumber{2}%
\task{%
    Два шарика, масса каждого из которых составляет $2\,\text{кг}$,
    движутся навстречу друг другу.
    Скорость одного из них $5\,\frac{\text{м}}{\text{с}}$, а другого~--- $3\,\frac{\text{м}}{\text{с}}$.
    Сделайте рисунок, укажите направления скоростей и импульсов.
    Определите импульс каждого из шариков, а также их суммарный импульс.
}
\answer{%
    \begin{align*}
    p_1 &= mv_1 = 2\,\text{кг} \cdot 5\,\frac{\text{м}}{\text{с}} = 10\,\frac{\text{кг}\cdot\text{м}}{\text{с}}, \\
    p_2 &= mv_2 = 2\,\text{кг} \cdot 3\,\frac{\text{м}}{\text{с}} = 6\,\frac{\text{кг}\cdot\text{м}}{\text{с}}, \\
    p &= \abs{p_1 - p_2} = \abs{m(v_1 - v_2)}= 4\,\frac{\text{кг}\cdot\text{м}}{\text{с}}.
    \end{align*}
}
\solutionspace{120pt}

\tasknumber{3}%
\task{%
    Два одинаковых шарика массами по $5\,\text{кг}$
    движутся во взаимно перпендикулярных направлениях.
    Скорости шариков составляют $3\,\frac{\text{м}}{\text{с}}$ и $4\,\frac{\text{м}}{\text{с}}$.
    Сделайте рисунок, укажите направления скоростей и импульсов.
    Определите импульс каждого из шариков и полный импульс системы.
}
\answer{%
    \begin{align*}
    p_1 &= mv_1 = 5\,\text{кг} \cdot 3\,\frac{\text{м}}{\text{с}} = 15\,\frac{\text{кг}\cdot\text{м}}{\text{с}}, \\
    p_2 &= mv_2 = 5\,\text{кг} \cdot 4\,\frac{\text{м}}{\text{с}} = 20\,\frac{\text{кг}\cdot\text{м}}{\text{с}}, \\
    p &= \sqrt{p_1^2 + p_2^2} = m\sqrt{v_1^2 + v_2^2} = 25\,\frac{\text{кг}\cdot\text{м}}{\text{с}}.
    \end{align*}
}
\solutionspace{120pt}

\tasknumber{4}%
\task{%
    Паровоз массой $M = 210\,\text{т}$, скорость которого равна $v = 0{,}20\,\frac{\text{м}}{\text{с}}$,
    сталкивается с двумя неподвижными вагонами массой $m = 30\,\text{т}$ каждый и сцепляется с ними.
    Запишите (формулами, не числами) импульсы каждого из тел до и после сцепки и после,
    а также определите скорость их совместного движения.
}
\answer{%
    \begin{align*}
    \text{ЗСИ:} &M \cdot v + 2 \cdot \cbr{m \cdot 0} =  M \cdot u + 2 \cdot \cbr{m \cdot u} \implies \\
    &\implies u = v \cdot \frac{M}{M + nm} = 0{,}20\,\frac{\text{м}}{\text{с}} \cdot  \frac{ 210\,\text{т} }{210\,\text{т} + 2 \cdot 30\,\text{т}} \approx 0{,}16\,\frac{\text{м}}{\text{с}}.
    \end{align*}
}
\solutionspace{120pt}

\tasknumber{5}%
\task{%
    Два тела двигаются навстречу друг другу.
    Скорость каждого из них составляет $6\,\frac{\text{м}}{\text{с}}$.
    После соударения тела слиплись и продолжили движение уже со скоростью $4\,\frac{\text{м}}{\text{с}}$.
    Определите отношение масс тел (большей к меньшей).
}
\answer{%
    \begin{align*}
    &\text{ЗСИ в проекции на ось, соединяющую центры тел:} m_1 v_1 - m_2 v_1 = (m_1 + m_2) v_2 \implies \\
    &\implies \frac{m_1}{m_2} v_1 - v_1 = \cbr{\frac{m_1}{m_2} + 1} v_2 \implies
        \frac{m_1}{m_2} (v_1 - v_2) = v_2 + v_1 \implies \frac{m_1}{m_2} = \frac{v_2 + v_1}{v_1 - v_2} = 5
    \end{align*}
}

\variantsplitter

\addpersonalvariant{Варвара Егиазарян}

\tasknumber{1}%
\task{%
    Шарики массами $3\,\text{кг}$ и $2\,\text{кг}$
    движутся параллельно друг другу в одном направлении
    со скоростями $2\,\frac{\text{м}}{\text{с}}$ и $3\,\frac{\text{м}}{\text{с}}$ соответственно.
    Сделайте рисунок и укажите направления скоростей и импульсов.
    Определите импульс каждого из шариков, а также их суммарный импульс.
}
\answer{%
    \begin{align*}
    p_1 &= m_1v_1 = 3\,\text{кг} \cdot 2\,\frac{\text{м}}{\text{с}} = 6\,\frac{\text{кг}\cdot\text{м}}{\text{с}}, \\
    p_2 &= m_2v_2 = 2\,\text{кг} \cdot 3\,\frac{\text{м}}{\text{с}} = 6\,\frac{\text{кг}\cdot\text{м}}{\text{с}}, \\
    p &= p_1 + p_2 = m_1v_1 + m_2v_2 = 12\,\frac{\text{кг}\cdot\text{м}}{\text{с}}.
    \end{align*}
}
\solutionspace{120pt}

\tasknumber{2}%
\task{%
    Два шарика, масса каждого из которых составляет $5\,\text{кг}$,
    движутся навстречу друг другу.
    Скорость одного из них $5\,\frac{\text{м}}{\text{с}}$, а другого~--- $3\,\frac{\text{м}}{\text{с}}$.
    Сделайте рисунок, укажите направления скоростей и импульсов.
    Определите импульс каждого из шариков, а также их суммарный импульс.
}
\answer{%
    \begin{align*}
    p_1 &= mv_1 = 5\,\text{кг} \cdot 5\,\frac{\text{м}}{\text{с}} = 25\,\frac{\text{кг}\cdot\text{м}}{\text{с}}, \\
    p_2 &= mv_2 = 5\,\text{кг} \cdot 3\,\frac{\text{м}}{\text{с}} = 15\,\frac{\text{кг}\cdot\text{м}}{\text{с}}, \\
    p &= \abs{p_1 - p_2} = \abs{m(v_1 - v_2)}= 10\,\frac{\text{кг}\cdot\text{м}}{\text{с}}.
    \end{align*}
}
\solutionspace{120pt}

\tasknumber{3}%
\task{%
    Два одинаковых шарика массами по $5\,\text{кг}$
    движутся во взаимно перпендикулярных направлениях.
    Скорости шариков составляют $3\,\frac{\text{м}}{\text{с}}$ и $4\,\frac{\text{м}}{\text{с}}$.
    Сделайте рисунок, укажите направления скоростей и импульсов.
    Определите импульс каждого из шариков и полный импульс системы.
}
\answer{%
    \begin{align*}
    p_1 &= mv_1 = 5\,\text{кг} \cdot 3\,\frac{\text{м}}{\text{с}} = 15\,\frac{\text{кг}\cdot\text{м}}{\text{с}}, \\
    p_2 &= mv_2 = 5\,\text{кг} \cdot 4\,\frac{\text{м}}{\text{с}} = 20\,\frac{\text{кг}\cdot\text{м}}{\text{с}}, \\
    p &= \sqrt{p_1^2 + p_2^2} = m\sqrt{v_1^2 + v_2^2} = 25\,\frac{\text{кг}\cdot\text{м}}{\text{с}}.
    \end{align*}
}
\solutionspace{120pt}

\tasknumber{4}%
\task{%
    Паровоз массой $M = 120\,\text{т}$, скорость которого равна $v = 0{,}20\,\frac{\text{м}}{\text{с}}$,
    сталкивается с двумя неподвижными вагонами массой $m = 40\,\text{т}$ каждый и сцепляется с ними.
    Запишите (формулами, не числами) импульсы каждого из тел до и после сцепки и после,
    а также определите скорость их совместного движения.
}
\answer{%
    \begin{align*}
    \text{ЗСИ:} &M \cdot v + 2 \cdot \cbr{m \cdot 0} =  M \cdot u + 2 \cdot \cbr{m \cdot u} \implies \\
    &\implies u = v \cdot \frac{M}{M + nm} = 0{,}20\,\frac{\text{м}}{\text{с}} \cdot  \frac{ 120\,\text{т} }{120\,\text{т} + 2 \cdot 40\,\text{т}} \approx 0{,}12\,\frac{\text{м}}{\text{с}}.
    \end{align*}
}
\solutionspace{120pt}

\tasknumber{5}%
\task{%
    Два тела двигаются навстречу друг другу.
    Скорость каждого из них составляет $2\,\frac{\text{м}}{\text{с}}$.
    После соударения тела слиплись и продолжили движение уже со скоростью $1\,\frac{\text{м}}{\text{с}}$.
    Определите отношение масс тел (большей к меньшей).
}
\answer{%
    \begin{align*}
    &\text{ЗСИ в проекции на ось, соединяющую центры тел:} m_1 v_1 - m_2 v_1 = (m_1 + m_2) v_2 \implies \\
    &\implies \frac{m_1}{m_2} v_1 - v_1 = \cbr{\frac{m_1}{m_2} + 1} v_2 \implies
        \frac{m_1}{m_2} (v_1 - v_2) = v_2 + v_1 \implies \frac{m_1}{m_2} = \frac{v_2 + v_1}{v_1 - v_2} = 3
    \end{align*}
}

\variantsplitter

\addpersonalvariant{Владислав Емелин}

\tasknumber{1}%
\task{%
    Шарики массами $1\,\text{кг}$ и $4\,\text{кг}$
    движутся параллельно друг другу в одном направлении
    со скоростями $10\,\frac{\text{м}}{\text{с}}$ и $6\,\frac{\text{м}}{\text{с}}$ соответственно.
    Сделайте рисунок и укажите направления скоростей и импульсов.
    Определите импульс каждого из шариков, а также их суммарный импульс.
}
\answer{%
    \begin{align*}
    p_1 &= m_1v_1 = 1\,\text{кг} \cdot 10\,\frac{\text{м}}{\text{с}} = 10\,\frac{\text{кг}\cdot\text{м}}{\text{с}}, \\
    p_2 &= m_2v_2 = 4\,\text{кг} \cdot 6\,\frac{\text{м}}{\text{с}} = 24\,\frac{\text{кг}\cdot\text{м}}{\text{с}}, \\
    p &= p_1 + p_2 = m_1v_1 + m_2v_2 = 34\,\frac{\text{кг}\cdot\text{м}}{\text{с}}.
    \end{align*}
}
\solutionspace{120pt}

\tasknumber{2}%
\task{%
    Два шарика, масса каждого из которых составляет $10\,\text{кг}$,
    движутся навстречу друг другу.
    Скорость одного из них $10\,\frac{\text{м}}{\text{с}}$, а другого~--- $3\,\frac{\text{м}}{\text{с}}$.
    Сделайте рисунок, укажите направления скоростей и импульсов.
    Определите импульс каждого из шариков, а также их суммарный импульс.
}
\answer{%
    \begin{align*}
    p_1 &= mv_1 = 10\,\text{кг} \cdot 10\,\frac{\text{м}}{\text{с}} = 100\,\frac{\text{кг}\cdot\text{м}}{\text{с}}, \\
    p_2 &= mv_2 = 10\,\text{кг} \cdot 3\,\frac{\text{м}}{\text{с}} = 30\,\frac{\text{кг}\cdot\text{м}}{\text{с}}, \\
    p &= \abs{p_1 - p_2} = \abs{m(v_1 - v_2)}= 70\,\frac{\text{кг}\cdot\text{м}}{\text{с}}.
    \end{align*}
}
\solutionspace{120pt}

\tasknumber{3}%
\task{%
    Два одинаковых шарика массами по $5\,\text{кг}$
    движутся во взаимно перпендикулярных направлениях.
    Скорости шариков составляют $7\,\frac{\text{м}}{\text{с}}$ и $24\,\frac{\text{м}}{\text{с}}$.
    Сделайте рисунок, укажите направления скоростей и импульсов.
    Определите импульс каждого из шариков и полный импульс системы.
}
\answer{%
    \begin{align*}
    p_1 &= mv_1 = 5\,\text{кг} \cdot 7\,\frac{\text{м}}{\text{с}} = 35\,\frac{\text{кг}\cdot\text{м}}{\text{с}}, \\
    p_2 &= mv_2 = 5\,\text{кг} \cdot 24\,\frac{\text{м}}{\text{с}} = 120\,\frac{\text{кг}\cdot\text{м}}{\text{с}}, \\
    p &= \sqrt{p_1^2 + p_2^2} = m\sqrt{v_1^2 + v_2^2} = 125\,\frac{\text{кг}\cdot\text{м}}{\text{с}}.
    \end{align*}
}
\solutionspace{120pt}

\tasknumber{4}%
\task{%
    Паровоз массой $M = 120\,\text{т}$, скорость которого равна $v = 0{,}20\,\frac{\text{м}}{\text{с}}$,
    сталкивается с четыремя неподвижными вагонами массой $m = 30\,\text{т}$ каждый и сцепляется с ними.
    Запишите (формулами, не числами) импульсы каждого из тел до и после сцепки и после,
    а также определите скорость их совместного движения.
}
\answer{%
    \begin{align*}
    \text{ЗСИ:} &M \cdot v + 4 \cdot \cbr{m \cdot 0} =  M \cdot u + 4 \cdot \cbr{m \cdot u} \implies \\
    &\implies u = v \cdot \frac{M}{M + nm} = 0{,}20\,\frac{\text{м}}{\text{с}} \cdot  \frac{ 120\,\text{т} }{120\,\text{т} + 4 \cdot 30\,\text{т}} \approx 0{,}10\,\frac{\text{м}}{\text{с}}.
    \end{align*}
}
\solutionspace{120pt}

\tasknumber{5}%
\task{%
    Два тела двигаются навстречу друг другу.
    Скорость каждого из них составляет $6\,\frac{\text{м}}{\text{с}}$.
    После соударения тела слиплись и продолжили движение уже со скоростью $3\,\frac{\text{м}}{\text{с}}$.
    Определите отношение масс тел (большей к меньшей).
}
\answer{%
    \begin{align*}
    &\text{ЗСИ в проекции на ось, соединяющую центры тел:} m_1 v_1 - m_2 v_1 = (m_1 + m_2) v_2 \implies \\
    &\implies \frac{m_1}{m_2} v_1 - v_1 = \cbr{\frac{m_1}{m_2} + 1} v_2 \implies
        \frac{m_1}{m_2} (v_1 - v_2) = v_2 + v_1 \implies \frac{m_1}{m_2} = \frac{v_2 + v_1}{v_1 - v_2} = 3
    \end{align*}
}

\variantsplitter

\addpersonalvariant{Артём Жичин}

\tasknumber{1}%
\task{%
    Шарики массами $1\,\text{кг}$ и $4\,\text{кг}$
    движутся параллельно друг другу в одном направлении
    со скоростями $4\,\frac{\text{м}}{\text{с}}$ и $8\,\frac{\text{м}}{\text{с}}$ соответственно.
    Сделайте рисунок и укажите направления скоростей и импульсов.
    Определите импульс каждого из шариков, а также их суммарный импульс.
}
\answer{%
    \begin{align*}
    p_1 &= m_1v_1 = 1\,\text{кг} \cdot 4\,\frac{\text{м}}{\text{с}} = 4\,\frac{\text{кг}\cdot\text{м}}{\text{с}}, \\
    p_2 &= m_2v_2 = 4\,\text{кг} \cdot 8\,\frac{\text{м}}{\text{с}} = 32\,\frac{\text{кг}\cdot\text{м}}{\text{с}}, \\
    p &= p_1 + p_2 = m_1v_1 + m_2v_2 = 36\,\frac{\text{кг}\cdot\text{м}}{\text{с}}.
    \end{align*}
}
\solutionspace{120pt}

\tasknumber{2}%
\task{%
    Два шарика, масса каждого из которых составляет $2\,\text{кг}$,
    движутся навстречу друг другу.
    Скорость одного из них $10\,\frac{\text{м}}{\text{с}}$, а другого~--- $6\,\frac{\text{м}}{\text{с}}$.
    Сделайте рисунок, укажите направления скоростей и импульсов.
    Определите импульс каждого из шариков, а также их суммарный импульс.
}
\answer{%
    \begin{align*}
    p_1 &= mv_1 = 2\,\text{кг} \cdot 10\,\frac{\text{м}}{\text{с}} = 20\,\frac{\text{кг}\cdot\text{м}}{\text{с}}, \\
    p_2 &= mv_2 = 2\,\text{кг} \cdot 6\,\frac{\text{м}}{\text{с}} = 12\,\frac{\text{кг}\cdot\text{м}}{\text{с}}, \\
    p &= \abs{p_1 - p_2} = \abs{m(v_1 - v_2)}= 8\,\frac{\text{кг}\cdot\text{м}}{\text{с}}.
    \end{align*}
}
\solutionspace{120pt}

\tasknumber{3}%
\task{%
    Два одинаковых шарика массами по $10\,\text{кг}$
    движутся во взаимно перпендикулярных направлениях.
    Скорости шариков составляют $5\,\frac{\text{м}}{\text{с}}$ и $12\,\frac{\text{м}}{\text{с}}$.
    Сделайте рисунок, укажите направления скоростей и импульсов.
    Определите импульс каждого из шариков и полный импульс системы.
}
\answer{%
    \begin{align*}
    p_1 &= mv_1 = 10\,\text{кг} \cdot 5\,\frac{\text{м}}{\text{с}} = 50\,\frac{\text{кг}\cdot\text{м}}{\text{с}}, \\
    p_2 &= mv_2 = 10\,\text{кг} \cdot 12\,\frac{\text{м}}{\text{с}} = 120\,\frac{\text{кг}\cdot\text{м}}{\text{с}}, \\
    p &= \sqrt{p_1^2 + p_2^2} = m\sqrt{v_1^2 + v_2^2} = 130\,\frac{\text{кг}\cdot\text{м}}{\text{с}}.
    \end{align*}
}
\solutionspace{120pt}

\tasknumber{4}%
\task{%
    Паровоз массой $M = 210\,\text{т}$, скорость которого равна $v = 0{,}40\,\frac{\text{м}}{\text{с}}$,
    сталкивается с тремя неподвижными вагонами массой $m = 40\,\text{т}$ каждый и сцепляется с ними.
    Запишите (формулами, не числами) импульсы каждого из тел до и после сцепки и после,
    а также определите скорость их совместного движения.
}
\answer{%
    \begin{align*}
    \text{ЗСИ:} &M \cdot v + 3 \cdot \cbr{m \cdot 0} =  M \cdot u + 3 \cdot \cbr{m \cdot u} \implies \\
    &\implies u = v \cdot \frac{M}{M + nm} = 0{,}40\,\frac{\text{м}}{\text{с}} \cdot  \frac{ 210\,\text{т} }{210\,\text{т} + 3 \cdot 40\,\text{т}} \approx 0{,}25\,\frac{\text{м}}{\text{с}}.
    \end{align*}
}
\solutionspace{120pt}

\tasknumber{5}%
\task{%
    Два тела двигаются навстречу друг другу.
    Скорость каждого из них составляет $6\,\frac{\text{м}}{\text{с}}$.
    После соударения тела слиплись и продолжили движение уже со скоростью $3\,\frac{\text{м}}{\text{с}}$.
    Определите отношение масс тел (большей к меньшей).
}
\answer{%
    \begin{align*}
    &\text{ЗСИ в проекции на ось, соединяющую центры тел:} m_1 v_1 - m_2 v_1 = (m_1 + m_2) v_2 \implies \\
    &\implies \frac{m_1}{m_2} v_1 - v_1 = \cbr{\frac{m_1}{m_2} + 1} v_2 \implies
        \frac{m_1}{m_2} (v_1 - v_2) = v_2 + v_1 \implies \frac{m_1}{m_2} = \frac{v_2 + v_1}{v_1 - v_2} = 3
    \end{align*}
}

\variantsplitter

\addpersonalvariant{Елизавета Карманова}

\tasknumber{1}%
\task{%
    Шарики массами $4\,\text{кг}$ и $2\,\text{кг}$
    движутся параллельно друг другу в одном направлении
    со скоростями $4\,\frac{\text{м}}{\text{с}}$ и $3\,\frac{\text{м}}{\text{с}}$ соответственно.
    Сделайте рисунок и укажите направления скоростей и импульсов.
    Определите импульс каждого из шариков, а также их суммарный импульс.
}
\answer{%
    \begin{align*}
    p_1 &= m_1v_1 = 4\,\text{кг} \cdot 4\,\frac{\text{м}}{\text{с}} = 16\,\frac{\text{кг}\cdot\text{м}}{\text{с}}, \\
    p_2 &= m_2v_2 = 2\,\text{кг} \cdot 3\,\frac{\text{м}}{\text{с}} = 6\,\frac{\text{кг}\cdot\text{м}}{\text{с}}, \\
    p &= p_1 + p_2 = m_1v_1 + m_2v_2 = 22\,\frac{\text{кг}\cdot\text{м}}{\text{с}}.
    \end{align*}
}
\solutionspace{120pt}

\tasknumber{2}%
\task{%
    Два шарика, масса каждого из которых составляет $10\,\text{кг}$,
    движутся навстречу друг другу.
    Скорость одного из них $5\,\frac{\text{м}}{\text{с}}$, а другого~--- $6\,\frac{\text{м}}{\text{с}}$.
    Сделайте рисунок, укажите направления скоростей и импульсов.
    Определите импульс каждого из шариков, а также их суммарный импульс.
}
\answer{%
    \begin{align*}
    p_1 &= mv_1 = 10\,\text{кг} \cdot 5\,\frac{\text{м}}{\text{с}} = 50\,\frac{\text{кг}\cdot\text{м}}{\text{с}}, \\
    p_2 &= mv_2 = 10\,\text{кг} \cdot 6\,\frac{\text{м}}{\text{с}} = 60\,\frac{\text{кг}\cdot\text{м}}{\text{с}}, \\
    p &= \abs{p_1 - p_2} = \abs{m(v_1 - v_2)}= 10\,\frac{\text{кг}\cdot\text{м}}{\text{с}}.
    \end{align*}
}
\solutionspace{120pt}

\tasknumber{3}%
\task{%
    Два одинаковых шарика массами по $5\,\text{кг}$
    движутся во взаимно перпендикулярных направлениях.
    Скорости шариков составляют $7\,\frac{\text{м}}{\text{с}}$ и $24\,\frac{\text{м}}{\text{с}}$.
    Сделайте рисунок, укажите направления скоростей и импульсов.
    Определите импульс каждого из шариков и полный импульс системы.
}
\answer{%
    \begin{align*}
    p_1 &= mv_1 = 5\,\text{кг} \cdot 7\,\frac{\text{м}}{\text{с}} = 35\,\frac{\text{кг}\cdot\text{м}}{\text{с}}, \\
    p_2 &= mv_2 = 5\,\text{кг} \cdot 24\,\frac{\text{м}}{\text{с}} = 120\,\frac{\text{кг}\cdot\text{м}}{\text{с}}, \\
    p &= \sqrt{p_1^2 + p_2^2} = m\sqrt{v_1^2 + v_2^2} = 125\,\frac{\text{кг}\cdot\text{м}}{\text{с}}.
    \end{align*}
}
\solutionspace{120pt}

\tasknumber{4}%
\task{%
    Паровоз массой $M = 210\,\text{т}$, скорость которого равна $v = 0{,}40\,\frac{\text{м}}{\text{с}}$,
    сталкивается с двумя неподвижными вагонами массой $m = 50\,\text{т}$ каждый и сцепляется с ними.
    Запишите (формулами, не числами) импульсы каждого из тел до и после сцепки и после,
    а также определите скорость их совместного движения.
}
\answer{%
    \begin{align*}
    \text{ЗСИ:} &M \cdot v + 2 \cdot \cbr{m \cdot 0} =  M \cdot u + 2 \cdot \cbr{m \cdot u} \implies \\
    &\implies u = v \cdot \frac{M}{M + nm} = 0{,}40\,\frac{\text{м}}{\text{с}} \cdot  \frac{ 210\,\text{т} }{210\,\text{т} + 2 \cdot 50\,\text{т}} \approx 0{,}27\,\frac{\text{м}}{\text{с}}.
    \end{align*}
}
\solutionspace{120pt}

\tasknumber{5}%
\task{%
    Два тела двигаются навстречу друг другу.
    Скорость каждого из них составляет $6\,\frac{\text{м}}{\text{с}}$.
    После соударения тела слиплись и продолжили движение уже со скоростью $3\,\frac{\text{м}}{\text{с}}$.
    Определите отношение масс тел (большей к меньшей).
}
\answer{%
    \begin{align*}
    &\text{ЗСИ в проекции на ось, соединяющую центры тел:} m_1 v_1 - m_2 v_1 = (m_1 + m_2) v_2 \implies \\
    &\implies \frac{m_1}{m_2} v_1 - v_1 = \cbr{\frac{m_1}{m_2} + 1} v_2 \implies
        \frac{m_1}{m_2} (v_1 - v_2) = v_2 + v_1 \implies \frac{m_1}{m_2} = \frac{v_2 + v_1}{v_1 - v_2} = 3
    \end{align*}
}

\variantsplitter

\addpersonalvariant{Анна Кузьмичёва}

\tasknumber{1}%
\task{%
    Шарики массами $2\,\text{кг}$ и $4\,\text{кг}$
    движутся параллельно друг другу в одном направлении
    со скоростями $2\,\frac{\text{м}}{\text{с}}$ и $6\,\frac{\text{м}}{\text{с}}$ соответственно.
    Сделайте рисунок и укажите направления скоростей и импульсов.
    Определите импульс каждого из шариков, а также их суммарный импульс.
}
\answer{%
    \begin{align*}
    p_1 &= m_1v_1 = 2\,\text{кг} \cdot 2\,\frac{\text{м}}{\text{с}} = 4\,\frac{\text{кг}\cdot\text{м}}{\text{с}}, \\
    p_2 &= m_2v_2 = 4\,\text{кг} \cdot 6\,\frac{\text{м}}{\text{с}} = 24\,\frac{\text{кг}\cdot\text{м}}{\text{с}}, \\
    p &= p_1 + p_2 = m_1v_1 + m_2v_2 = 28\,\frac{\text{кг}\cdot\text{м}}{\text{с}}.
    \end{align*}
}
\solutionspace{120pt}

\tasknumber{2}%
\task{%
    Два шарика, масса каждого из которых составляет $5\,\text{кг}$,
    движутся навстречу друг другу.
    Скорость одного из них $1\,\frac{\text{м}}{\text{с}}$, а другого~--- $3\,\frac{\text{м}}{\text{с}}$.
    Сделайте рисунок, укажите направления скоростей и импульсов.
    Определите импульс каждого из шариков, а также их суммарный импульс.
}
\answer{%
    \begin{align*}
    p_1 &= mv_1 = 5\,\text{кг} \cdot 1\,\frac{\text{м}}{\text{с}} = 5\,\frac{\text{кг}\cdot\text{м}}{\text{с}}, \\
    p_2 &= mv_2 = 5\,\text{кг} \cdot 3\,\frac{\text{м}}{\text{с}} = 15\,\frac{\text{кг}\cdot\text{м}}{\text{с}}, \\
    p &= \abs{p_1 - p_2} = \abs{m(v_1 - v_2)}= 10\,\frac{\text{кг}\cdot\text{м}}{\text{с}}.
    \end{align*}
}
\solutionspace{120pt}

\tasknumber{3}%
\task{%
    Два одинаковых шарика массами по $2\,\text{кг}$
    движутся во взаимно перпендикулярных направлениях.
    Скорости шариков составляют $5\,\frac{\text{м}}{\text{с}}$ и $12\,\frac{\text{м}}{\text{с}}$.
    Сделайте рисунок, укажите направления скоростей и импульсов.
    Определите импульс каждого из шариков и полный импульс системы.
}
\answer{%
    \begin{align*}
    p_1 &= mv_1 = 2\,\text{кг} \cdot 5\,\frac{\text{м}}{\text{с}} = 10\,\frac{\text{кг}\cdot\text{м}}{\text{с}}, \\
    p_2 &= mv_2 = 2\,\text{кг} \cdot 12\,\frac{\text{м}}{\text{с}} = 24\,\frac{\text{кг}\cdot\text{м}}{\text{с}}, \\
    p &= \sqrt{p_1^2 + p_2^2} = m\sqrt{v_1^2 + v_2^2} = 26\,\frac{\text{кг}\cdot\text{м}}{\text{с}}.
    \end{align*}
}
\solutionspace{120pt}

\tasknumber{4}%
\task{%
    Паровоз массой $M = 150\,\text{т}$, скорость которого равна $v = 0{,}20\,\frac{\text{м}}{\text{с}}$,
    сталкивается с четыремя неподвижными вагонами массой $m = 30\,\text{т}$ каждый и сцепляется с ними.
    Запишите (формулами, не числами) импульсы каждого из тел до и после сцепки и после,
    а также определите скорость их совместного движения.
}
\answer{%
    \begin{align*}
    \text{ЗСИ:} &M \cdot v + 4 \cdot \cbr{m \cdot 0} =  M \cdot u + 4 \cdot \cbr{m \cdot u} \implies \\
    &\implies u = v \cdot \frac{M}{M + nm} = 0{,}20\,\frac{\text{м}}{\text{с}} \cdot  \frac{ 150\,\text{т} }{150\,\text{т} + 4 \cdot 30\,\text{т}} \approx 0{,}11\,\frac{\text{м}}{\text{с}}.
    \end{align*}
}
\solutionspace{120pt}

\tasknumber{5}%
\task{%
    Два тела двигаются навстречу друг другу.
    Скорость каждого из них составляет $6\,\frac{\text{м}}{\text{с}}$.
    После соударения тела слиплись и продолжили движение уже со скоростью $4\,\frac{\text{м}}{\text{с}}$.
    Определите отношение масс тел (большей к меньшей).
}
\answer{%
    \begin{align*}
    &\text{ЗСИ в проекции на ось, соединяющую центры тел:} m_1 v_1 - m_2 v_1 = (m_1 + m_2) v_2 \implies \\
    &\implies \frac{m_1}{m_2} v_1 - v_1 = \cbr{\frac{m_1}{m_2} + 1} v_2 \implies
        \frac{m_1}{m_2} (v_1 - v_2) = v_2 + v_1 \implies \frac{m_1}{m_2} = \frac{v_2 + v_1}{v_1 - v_2} = 5
    \end{align*}
}

\variantsplitter

\addpersonalvariant{Алёна Куприянова}

\tasknumber{1}%
\task{%
    Шарики массами $1\,\text{кг}$ и $3\,\text{кг}$
    движутся параллельно друг другу в одном направлении
    со скоростями $5\,\frac{\text{м}}{\text{с}}$ и $3\,\frac{\text{м}}{\text{с}}$ соответственно.
    Сделайте рисунок и укажите направления скоростей и импульсов.
    Определите импульс каждого из шариков, а также их суммарный импульс.
}
\answer{%
    \begin{align*}
    p_1 &= m_1v_1 = 1\,\text{кг} \cdot 5\,\frac{\text{м}}{\text{с}} = 5\,\frac{\text{кг}\cdot\text{м}}{\text{с}}, \\
    p_2 &= m_2v_2 = 3\,\text{кг} \cdot 3\,\frac{\text{м}}{\text{с}} = 9\,\frac{\text{кг}\cdot\text{м}}{\text{с}}, \\
    p &= p_1 + p_2 = m_1v_1 + m_2v_2 = 14\,\frac{\text{кг}\cdot\text{м}}{\text{с}}.
    \end{align*}
}
\solutionspace{120pt}

\tasknumber{2}%
\task{%
    Два шарика, масса каждого из которых составляет $2\,\text{кг}$,
    движутся навстречу друг другу.
    Скорость одного из них $10\,\frac{\text{м}}{\text{с}}$, а другого~--- $3\,\frac{\text{м}}{\text{с}}$.
    Сделайте рисунок, укажите направления скоростей и импульсов.
    Определите импульс каждого из шариков, а также их суммарный импульс.
}
\answer{%
    \begin{align*}
    p_1 &= mv_1 = 2\,\text{кг} \cdot 10\,\frac{\text{м}}{\text{с}} = 20\,\frac{\text{кг}\cdot\text{м}}{\text{с}}, \\
    p_2 &= mv_2 = 2\,\text{кг} \cdot 3\,\frac{\text{м}}{\text{с}} = 6\,\frac{\text{кг}\cdot\text{м}}{\text{с}}, \\
    p &= \abs{p_1 - p_2} = \abs{m(v_1 - v_2)}= 14\,\frac{\text{кг}\cdot\text{м}}{\text{с}}.
    \end{align*}
}
\solutionspace{120pt}

\tasknumber{3}%
\task{%
    Два одинаковых шарика массами по $5\,\text{кг}$
    движутся во взаимно перпендикулярных направлениях.
    Скорости шариков составляют $5\,\frac{\text{м}}{\text{с}}$ и $12\,\frac{\text{м}}{\text{с}}$.
    Сделайте рисунок, укажите направления скоростей и импульсов.
    Определите импульс каждого из шариков и полный импульс системы.
}
\answer{%
    \begin{align*}
    p_1 &= mv_1 = 5\,\text{кг} \cdot 5\,\frac{\text{м}}{\text{с}} = 25\,\frac{\text{кг}\cdot\text{м}}{\text{с}}, \\
    p_2 &= mv_2 = 5\,\text{кг} \cdot 12\,\frac{\text{м}}{\text{с}} = 60\,\frac{\text{кг}\cdot\text{м}}{\text{с}}, \\
    p &= \sqrt{p_1^2 + p_2^2} = m\sqrt{v_1^2 + v_2^2} = 65\,\frac{\text{кг}\cdot\text{м}}{\text{с}}.
    \end{align*}
}
\solutionspace{120pt}

\tasknumber{4}%
\task{%
    Паровоз массой $M = 150\,\text{т}$, скорость которого равна $v = 0{,}20\,\frac{\text{м}}{\text{с}}$,
    сталкивается с четыремя неподвижными вагонами массой $m = 40\,\text{т}$ каждый и сцепляется с ними.
    Запишите (формулами, не числами) импульсы каждого из тел до и после сцепки и после,
    а также определите скорость их совместного движения.
}
\answer{%
    \begin{align*}
    \text{ЗСИ:} &M \cdot v + 4 \cdot \cbr{m \cdot 0} =  M \cdot u + 4 \cdot \cbr{m \cdot u} \implies \\
    &\implies u = v \cdot \frac{M}{M + nm} = 0{,}20\,\frac{\text{м}}{\text{с}} \cdot  \frac{ 150\,\text{т} }{150\,\text{т} + 4 \cdot 40\,\text{т}} \approx 0{,}10\,\frac{\text{м}}{\text{с}}.
    \end{align*}
}
\solutionspace{120pt}

\tasknumber{5}%
\task{%
    Два тела двигаются навстречу друг другу.
    Скорость каждого из них составляет $6\,\frac{\text{м}}{\text{с}}$.
    После соударения тела слиплись и продолжили движение уже со скоростью $3\,\frac{\text{м}}{\text{с}}$.
    Определите отношение масс тел (большей к меньшей).
}
\answer{%
    \begin{align*}
    &\text{ЗСИ в проекции на ось, соединяющую центры тел:} m_1 v_1 - m_2 v_1 = (m_1 + m_2) v_2 \implies \\
    &\implies \frac{m_1}{m_2} v_1 - v_1 = \cbr{\frac{m_1}{m_2} + 1} v_2 \implies
        \frac{m_1}{m_2} (v_1 - v_2) = v_2 + v_1 \implies \frac{m_1}{m_2} = \frac{v_2 + v_1}{v_1 - v_2} = 3
    \end{align*}
}

\variantsplitter

\addpersonalvariant{Анастасия Ламанова}

\tasknumber{1}%
\task{%
    Шарики массами $3\,\text{кг}$ и $4\,\text{кг}$
    движутся параллельно друг другу в одном направлении
    со скоростями $2\,\frac{\text{м}}{\text{с}}$ и $8\,\frac{\text{м}}{\text{с}}$ соответственно.
    Сделайте рисунок и укажите направления скоростей и импульсов.
    Определите импульс каждого из шариков, а также их суммарный импульс.
}
\answer{%
    \begin{align*}
    p_1 &= m_1v_1 = 3\,\text{кг} \cdot 2\,\frac{\text{м}}{\text{с}} = 6\,\frac{\text{кг}\cdot\text{м}}{\text{с}}, \\
    p_2 &= m_2v_2 = 4\,\text{кг} \cdot 8\,\frac{\text{м}}{\text{с}} = 32\,\frac{\text{кг}\cdot\text{м}}{\text{с}}, \\
    p &= p_1 + p_2 = m_1v_1 + m_2v_2 = 38\,\frac{\text{кг}\cdot\text{м}}{\text{с}}.
    \end{align*}
}
\solutionspace{120pt}

\tasknumber{2}%
\task{%
    Два шарика, масса каждого из которых составляет $2\,\text{кг}$,
    движутся навстречу друг другу.
    Скорость одного из них $1\,\frac{\text{м}}{\text{с}}$, а другого~--- $3\,\frac{\text{м}}{\text{с}}$.
    Сделайте рисунок, укажите направления скоростей и импульсов.
    Определите импульс каждого из шариков, а также их суммарный импульс.
}
\answer{%
    \begin{align*}
    p_1 &= mv_1 = 2\,\text{кг} \cdot 1\,\frac{\text{м}}{\text{с}} = 2\,\frac{\text{кг}\cdot\text{м}}{\text{с}}, \\
    p_2 &= mv_2 = 2\,\text{кг} \cdot 3\,\frac{\text{м}}{\text{с}} = 6\,\frac{\text{кг}\cdot\text{м}}{\text{с}}, \\
    p &= \abs{p_1 - p_2} = \abs{m(v_1 - v_2)}= 4\,\frac{\text{кг}\cdot\text{м}}{\text{с}}.
    \end{align*}
}
\solutionspace{120pt}

\tasknumber{3}%
\task{%
    Два одинаковых шарика массами по $10\,\text{кг}$
    движутся во взаимно перпендикулярных направлениях.
    Скорости шариков составляют $7\,\frac{\text{м}}{\text{с}}$ и $24\,\frac{\text{м}}{\text{с}}$.
    Сделайте рисунок, укажите направления скоростей и импульсов.
    Определите импульс каждого из шариков и полный импульс системы.
}
\answer{%
    \begin{align*}
    p_1 &= mv_1 = 10\,\text{кг} \cdot 7\,\frac{\text{м}}{\text{с}} = 70\,\frac{\text{кг}\cdot\text{м}}{\text{с}}, \\
    p_2 &= mv_2 = 10\,\text{кг} \cdot 24\,\frac{\text{м}}{\text{с}} = 240\,\frac{\text{кг}\cdot\text{м}}{\text{с}}, \\
    p &= \sqrt{p_1^2 + p_2^2} = m\sqrt{v_1^2 + v_2^2} = 250\,\frac{\text{кг}\cdot\text{м}}{\text{с}}.
    \end{align*}
}
\solutionspace{120pt}

\tasknumber{4}%
\task{%
    Паровоз массой $M = 120\,\text{т}$, скорость которого равна $v = 0{,}60\,\frac{\text{м}}{\text{с}}$,
    сталкивается с тремя неподвижными вагонами массой $m = 50\,\text{т}$ каждый и сцепляется с ними.
    Запишите (формулами, не числами) импульсы каждого из тел до и после сцепки и после,
    а также определите скорость их совместного движения.
}
\answer{%
    \begin{align*}
    \text{ЗСИ:} &M \cdot v + 3 \cdot \cbr{m \cdot 0} =  M \cdot u + 3 \cdot \cbr{m \cdot u} \implies \\
    &\implies u = v \cdot \frac{M}{M + nm} = 0{,}60\,\frac{\text{м}}{\text{с}} \cdot  \frac{ 120\,\text{т} }{120\,\text{т} + 3 \cdot 50\,\text{т}} \approx 0{,}27\,\frac{\text{м}}{\text{с}}.
    \end{align*}
}
\solutionspace{120pt}

\tasknumber{5}%
\task{%
    Два тела двигаются навстречу друг другу.
    Скорость каждого из них составляет $9\,\frac{\text{м}}{\text{с}}$.
    После соударения тела слиплись и продолжили движение уже со скоростью $7\,\frac{\text{м}}{\text{с}}$.
    Определите отношение масс тел (большей к меньшей).
}
\answer{%
    \begin{align*}
    &\text{ЗСИ в проекции на ось, соединяющую центры тел:} m_1 v_1 - m_2 v_1 = (m_1 + m_2) v_2 \implies \\
    &\implies \frac{m_1}{m_2} v_1 - v_1 = \cbr{\frac{m_1}{m_2} + 1} v_2 \implies
        \frac{m_1}{m_2} (v_1 - v_2) = v_2 + v_1 \implies \frac{m_1}{m_2} = \frac{v_2 + v_1}{v_1 - v_2} = 8
    \end{align*}
}

\variantsplitter

\addpersonalvariant{Виктория Легонькова}

\tasknumber{1}%
\task{%
    Шарики массами $1\,\text{кг}$ и $4\,\text{кг}$
    движутся параллельно друг другу в одном направлении
    со скоростями $2\,\frac{\text{м}}{\text{с}}$ и $3\,\frac{\text{м}}{\text{с}}$ соответственно.
    Сделайте рисунок и укажите направления скоростей и импульсов.
    Определите импульс каждого из шариков, а также их суммарный импульс.
}
\answer{%
    \begin{align*}
    p_1 &= m_1v_1 = 1\,\text{кг} \cdot 2\,\frac{\text{м}}{\text{с}} = 2\,\frac{\text{кг}\cdot\text{м}}{\text{с}}, \\
    p_2 &= m_2v_2 = 4\,\text{кг} \cdot 3\,\frac{\text{м}}{\text{с}} = 12\,\frac{\text{кг}\cdot\text{м}}{\text{с}}, \\
    p &= p_1 + p_2 = m_1v_1 + m_2v_2 = 14\,\frac{\text{кг}\cdot\text{м}}{\text{с}}.
    \end{align*}
}
\solutionspace{120pt}

\tasknumber{2}%
\task{%
    Два шарика, масса каждого из которых составляет $5\,\text{кг}$,
    движутся навстречу друг другу.
    Скорость одного из них $5\,\frac{\text{м}}{\text{с}}$, а другого~--- $8\,\frac{\text{м}}{\text{с}}$.
    Сделайте рисунок, укажите направления скоростей и импульсов.
    Определите импульс каждого из шариков, а также их суммарный импульс.
}
\answer{%
    \begin{align*}
    p_1 &= mv_1 = 5\,\text{кг} \cdot 5\,\frac{\text{м}}{\text{с}} = 25\,\frac{\text{кг}\cdot\text{м}}{\text{с}}, \\
    p_2 &= mv_2 = 5\,\text{кг} \cdot 8\,\frac{\text{м}}{\text{с}} = 40\,\frac{\text{кг}\cdot\text{м}}{\text{с}}, \\
    p &= \abs{p_1 - p_2} = \abs{m(v_1 - v_2)}= 15\,\frac{\text{кг}\cdot\text{м}}{\text{с}}.
    \end{align*}
}
\solutionspace{120pt}

\tasknumber{3}%
\task{%
    Два одинаковых шарика массами по $2\,\text{кг}$
    движутся во взаимно перпендикулярных направлениях.
    Скорости шариков составляют $7\,\frac{\text{м}}{\text{с}}$ и $24\,\frac{\text{м}}{\text{с}}$.
    Сделайте рисунок, укажите направления скоростей и импульсов.
    Определите импульс каждого из шариков и полный импульс системы.
}
\answer{%
    \begin{align*}
    p_1 &= mv_1 = 2\,\text{кг} \cdot 7\,\frac{\text{м}}{\text{с}} = 14\,\frac{\text{кг}\cdot\text{м}}{\text{с}}, \\
    p_2 &= mv_2 = 2\,\text{кг} \cdot 24\,\frac{\text{м}}{\text{с}} = 48\,\frac{\text{кг}\cdot\text{м}}{\text{с}}, \\
    p &= \sqrt{p_1^2 + p_2^2} = m\sqrt{v_1^2 + v_2^2} = 50\,\frac{\text{кг}\cdot\text{м}}{\text{с}}.
    \end{align*}
}
\solutionspace{120pt}

\tasknumber{4}%
\task{%
    Паровоз массой $M = 120\,\text{т}$, скорость которого равна $v = 0{,}60\,\frac{\text{м}}{\text{с}}$,
    сталкивается с двумя неподвижными вагонами массой $m = 50\,\text{т}$ каждый и сцепляется с ними.
    Запишите (формулами, не числами) импульсы каждого из тел до и после сцепки и после,
    а также определите скорость их совместного движения.
}
\answer{%
    \begin{align*}
    \text{ЗСИ:} &M \cdot v + 2 \cdot \cbr{m \cdot 0} =  M \cdot u + 2 \cdot \cbr{m \cdot u} \implies \\
    &\implies u = v \cdot \frac{M}{M + nm} = 0{,}60\,\frac{\text{м}}{\text{с}} \cdot  \frac{ 120\,\text{т} }{120\,\text{т} + 2 \cdot 50\,\text{т}} \approx 0{,}33\,\frac{\text{м}}{\text{с}}.
    \end{align*}
}
\solutionspace{120pt}

\tasknumber{5}%
\task{%
    Два тела двигаются навстречу друг другу.
    Скорость каждого из них составляет $6\,\frac{\text{м}}{\text{с}}$.
    После соударения тела слиплись и продолжили движение уже со скоростью $3\,\frac{\text{м}}{\text{с}}$.
    Определите отношение масс тел (большей к меньшей).
}
\answer{%
    \begin{align*}
    &\text{ЗСИ в проекции на ось, соединяющую центры тел:} m_1 v_1 - m_2 v_1 = (m_1 + m_2) v_2 \implies \\
    &\implies \frac{m_1}{m_2} v_1 - v_1 = \cbr{\frac{m_1}{m_2} + 1} v_2 \implies
        \frac{m_1}{m_2} (v_1 - v_2) = v_2 + v_1 \implies \frac{m_1}{m_2} = \frac{v_2 + v_1}{v_1 - v_2} = 3
    \end{align*}
}

\variantsplitter

\addpersonalvariant{Семён Мартынов}

\tasknumber{1}%
\task{%
    Шарики массами $2\,\text{кг}$ и $1\,\text{кг}$
    движутся параллельно друг другу в одном направлении
    со скоростями $2\,\frac{\text{м}}{\text{с}}$ и $8\,\frac{\text{м}}{\text{с}}$ соответственно.
    Сделайте рисунок и укажите направления скоростей и импульсов.
    Определите импульс каждого из шариков, а также их суммарный импульс.
}
\answer{%
    \begin{align*}
    p_1 &= m_1v_1 = 2\,\text{кг} \cdot 2\,\frac{\text{м}}{\text{с}} = 4\,\frac{\text{кг}\cdot\text{м}}{\text{с}}, \\
    p_2 &= m_2v_2 = 1\,\text{кг} \cdot 8\,\frac{\text{м}}{\text{с}} = 8\,\frac{\text{кг}\cdot\text{м}}{\text{с}}, \\
    p &= p_1 + p_2 = m_1v_1 + m_2v_2 = 12\,\frac{\text{кг}\cdot\text{м}}{\text{с}}.
    \end{align*}
}
\solutionspace{120pt}

\tasknumber{2}%
\task{%
    Два шарика, масса каждого из которых составляет $10\,\text{кг}$,
    движутся навстречу друг другу.
    Скорость одного из них $1\,\frac{\text{м}}{\text{с}}$, а другого~--- $3\,\frac{\text{м}}{\text{с}}$.
    Сделайте рисунок, укажите направления скоростей и импульсов.
    Определите импульс каждого из шариков, а также их суммарный импульс.
}
\answer{%
    \begin{align*}
    p_1 &= mv_1 = 10\,\text{кг} \cdot 1\,\frac{\text{м}}{\text{с}} = 10\,\frac{\text{кг}\cdot\text{м}}{\text{с}}, \\
    p_2 &= mv_2 = 10\,\text{кг} \cdot 3\,\frac{\text{м}}{\text{с}} = 30\,\frac{\text{кг}\cdot\text{м}}{\text{с}}, \\
    p &= \abs{p_1 - p_2} = \abs{m(v_1 - v_2)}= 20\,\frac{\text{кг}\cdot\text{м}}{\text{с}}.
    \end{align*}
}
\solutionspace{120pt}

\tasknumber{3}%
\task{%
    Два одинаковых шарика массами по $10\,\text{кг}$
    движутся во взаимно перпендикулярных направлениях.
    Скорости шариков составляют $7\,\frac{\text{м}}{\text{с}}$ и $24\,\frac{\text{м}}{\text{с}}$.
    Сделайте рисунок, укажите направления скоростей и импульсов.
    Определите импульс каждого из шариков и полный импульс системы.
}
\answer{%
    \begin{align*}
    p_1 &= mv_1 = 10\,\text{кг} \cdot 7\,\frac{\text{м}}{\text{с}} = 70\,\frac{\text{кг}\cdot\text{м}}{\text{с}}, \\
    p_2 &= mv_2 = 10\,\text{кг} \cdot 24\,\frac{\text{м}}{\text{с}} = 240\,\frac{\text{кг}\cdot\text{м}}{\text{с}}, \\
    p &= \sqrt{p_1^2 + p_2^2} = m\sqrt{v_1^2 + v_2^2} = 250\,\frac{\text{кг}\cdot\text{м}}{\text{с}}.
    \end{align*}
}
\solutionspace{120pt}

\tasknumber{4}%
\task{%
    Паровоз массой $M = 210\,\text{т}$, скорость которого равна $v = 0{,}40\,\frac{\text{м}}{\text{с}}$,
    сталкивается с двумя неподвижными вагонами массой $m = 30\,\text{т}$ каждый и сцепляется с ними.
    Запишите (формулами, не числами) импульсы каждого из тел до и после сцепки и после,
    а также определите скорость их совместного движения.
}
\answer{%
    \begin{align*}
    \text{ЗСИ:} &M \cdot v + 2 \cdot \cbr{m \cdot 0} =  M \cdot u + 2 \cdot \cbr{m \cdot u} \implies \\
    &\implies u = v \cdot \frac{M}{M + nm} = 0{,}40\,\frac{\text{м}}{\text{с}} \cdot  \frac{ 210\,\text{т} }{210\,\text{т} + 2 \cdot 30\,\text{т}} \approx 0{,}31\,\frac{\text{м}}{\text{с}}.
    \end{align*}
}
\solutionspace{120pt}

\tasknumber{5}%
\task{%
    Два тела двигаются навстречу друг другу.
    Скорость каждого из них составляет $3\,\frac{\text{м}}{\text{с}}$.
    После соударения тела слиплись и продолжили движение уже со скоростью $1\,\frac{\text{м}}{\text{с}}$.
    Определите отношение масс тел (большей к меньшей).
}
\answer{%
    \begin{align*}
    &\text{ЗСИ в проекции на ось, соединяющую центры тел:} m_1 v_1 - m_2 v_1 = (m_1 + m_2) v_2 \implies \\
    &\implies \frac{m_1}{m_2} v_1 - v_1 = \cbr{\frac{m_1}{m_2} + 1} v_2 \implies
        \frac{m_1}{m_2} (v_1 - v_2) = v_2 + v_1 \implies \frac{m_1}{m_2} = \frac{v_2 + v_1}{v_1 - v_2} = 2
    \end{align*}
}

\variantsplitter

\addpersonalvariant{Варвара Минаева}

\tasknumber{1}%
\task{%
    Шарики массами $4\,\text{кг}$ и $1\,\text{кг}$
    движутся параллельно друг другу в одном направлении
    со скоростями $5\,\frac{\text{м}}{\text{с}}$ и $8\,\frac{\text{м}}{\text{с}}$ соответственно.
    Сделайте рисунок и укажите направления скоростей и импульсов.
    Определите импульс каждого из шариков, а также их суммарный импульс.
}
\answer{%
    \begin{align*}
    p_1 &= m_1v_1 = 4\,\text{кг} \cdot 5\,\frac{\text{м}}{\text{с}} = 20\,\frac{\text{кг}\cdot\text{м}}{\text{с}}, \\
    p_2 &= m_2v_2 = 1\,\text{кг} \cdot 8\,\frac{\text{м}}{\text{с}} = 8\,\frac{\text{кг}\cdot\text{м}}{\text{с}}, \\
    p &= p_1 + p_2 = m_1v_1 + m_2v_2 = 28\,\frac{\text{кг}\cdot\text{м}}{\text{с}}.
    \end{align*}
}
\solutionspace{120pt}

\tasknumber{2}%
\task{%
    Два шарика, масса каждого из которых составляет $10\,\text{кг}$,
    движутся навстречу друг другу.
    Скорость одного из них $5\,\frac{\text{м}}{\text{с}}$, а другого~--- $6\,\frac{\text{м}}{\text{с}}$.
    Сделайте рисунок, укажите направления скоростей и импульсов.
    Определите импульс каждого из шариков, а также их суммарный импульс.
}
\answer{%
    \begin{align*}
    p_1 &= mv_1 = 10\,\text{кг} \cdot 5\,\frac{\text{м}}{\text{с}} = 50\,\frac{\text{кг}\cdot\text{м}}{\text{с}}, \\
    p_2 &= mv_2 = 10\,\text{кг} \cdot 6\,\frac{\text{м}}{\text{с}} = 60\,\frac{\text{кг}\cdot\text{м}}{\text{с}}, \\
    p &= \abs{p_1 - p_2} = \abs{m(v_1 - v_2)}= 10\,\frac{\text{кг}\cdot\text{м}}{\text{с}}.
    \end{align*}
}
\solutionspace{120pt}

\tasknumber{3}%
\task{%
    Два одинаковых шарика массами по $10\,\text{кг}$
    движутся во взаимно перпендикулярных направлениях.
    Скорости шариков составляют $5\,\frac{\text{м}}{\text{с}}$ и $12\,\frac{\text{м}}{\text{с}}$.
    Сделайте рисунок, укажите направления скоростей и импульсов.
    Определите импульс каждого из шариков и полный импульс системы.
}
\answer{%
    \begin{align*}
    p_1 &= mv_1 = 10\,\text{кг} \cdot 5\,\frac{\text{м}}{\text{с}} = 50\,\frac{\text{кг}\cdot\text{м}}{\text{с}}, \\
    p_2 &= mv_2 = 10\,\text{кг} \cdot 12\,\frac{\text{м}}{\text{с}} = 120\,\frac{\text{кг}\cdot\text{м}}{\text{с}}, \\
    p &= \sqrt{p_1^2 + p_2^2} = m\sqrt{v_1^2 + v_2^2} = 130\,\frac{\text{кг}\cdot\text{м}}{\text{с}}.
    \end{align*}
}
\solutionspace{120pt}

\tasknumber{4}%
\task{%
    Паровоз массой $M = 150\,\text{т}$, скорость которого равна $v = 0{,}20\,\frac{\text{м}}{\text{с}}$,
    сталкивается с двумя неподвижными вагонами массой $m = 30\,\text{т}$ каждый и сцепляется с ними.
    Запишите (формулами, не числами) импульсы каждого из тел до и после сцепки и после,
    а также определите скорость их совместного движения.
}
\answer{%
    \begin{align*}
    \text{ЗСИ:} &M \cdot v + 2 \cdot \cbr{m \cdot 0} =  M \cdot u + 2 \cdot \cbr{m \cdot u} \implies \\
    &\implies u = v \cdot \frac{M}{M + nm} = 0{,}20\,\frac{\text{м}}{\text{с}} \cdot  \frac{ 150\,\text{т} }{150\,\text{т} + 2 \cdot 30\,\text{т}} \approx 0{,}14\,\frac{\text{м}}{\text{с}}.
    \end{align*}
}
\solutionspace{120pt}

\tasknumber{5}%
\task{%
    Два тела двигаются навстречу друг другу.
    Скорость каждого из них составляет $4\,\frac{\text{м}}{\text{с}}$.
    После соударения тела слиплись и продолжили движение уже со скоростью $3\,\frac{\text{м}}{\text{с}}$.
    Определите отношение масс тел (большей к меньшей).
}
\answer{%
    \begin{align*}
    &\text{ЗСИ в проекции на ось, соединяющую центры тел:} m_1 v_1 - m_2 v_1 = (m_1 + m_2) v_2 \implies \\
    &\implies \frac{m_1}{m_2} v_1 - v_1 = \cbr{\frac{m_1}{m_2} + 1} v_2 \implies
        \frac{m_1}{m_2} (v_1 - v_2) = v_2 + v_1 \implies \frac{m_1}{m_2} = \frac{v_2 + v_1}{v_1 - v_2} = 7
    \end{align*}
}

\variantsplitter

\addpersonalvariant{Тимофей Полетаев}

\tasknumber{1}%
\task{%
    Шарики массами $3\,\text{кг}$ и $4\,\text{кг}$
    движутся параллельно друг другу в одном направлении
    со скоростями $2\,\frac{\text{м}}{\text{с}}$ и $8\,\frac{\text{м}}{\text{с}}$ соответственно.
    Сделайте рисунок и укажите направления скоростей и импульсов.
    Определите импульс каждого из шариков, а также их суммарный импульс.
}
\answer{%
    \begin{align*}
    p_1 &= m_1v_1 = 3\,\text{кг} \cdot 2\,\frac{\text{м}}{\text{с}} = 6\,\frac{\text{кг}\cdot\text{м}}{\text{с}}, \\
    p_2 &= m_2v_2 = 4\,\text{кг} \cdot 8\,\frac{\text{м}}{\text{с}} = 32\,\frac{\text{кг}\cdot\text{м}}{\text{с}}, \\
    p &= p_1 + p_2 = m_1v_1 + m_2v_2 = 38\,\frac{\text{кг}\cdot\text{м}}{\text{с}}.
    \end{align*}
}
\solutionspace{120pt}

\tasknumber{2}%
\task{%
    Два шарика, масса каждого из которых составляет $2\,\text{кг}$,
    движутся навстречу друг другу.
    Скорость одного из них $2\,\frac{\text{м}}{\text{с}}$, а другого~--- $6\,\frac{\text{м}}{\text{с}}$.
    Сделайте рисунок, укажите направления скоростей и импульсов.
    Определите импульс каждого из шариков, а также их суммарный импульс.
}
\answer{%
    \begin{align*}
    p_1 &= mv_1 = 2\,\text{кг} \cdot 2\,\frac{\text{м}}{\text{с}} = 4\,\frac{\text{кг}\cdot\text{м}}{\text{с}}, \\
    p_2 &= mv_2 = 2\,\text{кг} \cdot 6\,\frac{\text{м}}{\text{с}} = 12\,\frac{\text{кг}\cdot\text{м}}{\text{с}}, \\
    p &= \abs{p_1 - p_2} = \abs{m(v_1 - v_2)}= 8\,\frac{\text{кг}\cdot\text{м}}{\text{с}}.
    \end{align*}
}
\solutionspace{120pt}

\tasknumber{3}%
\task{%
    Два одинаковых шарика массами по $10\,\text{кг}$
    движутся во взаимно перпендикулярных направлениях.
    Скорости шариков составляют $7\,\frac{\text{м}}{\text{с}}$ и $24\,\frac{\text{м}}{\text{с}}$.
    Сделайте рисунок, укажите направления скоростей и импульсов.
    Определите импульс каждого из шариков и полный импульс системы.
}
\answer{%
    \begin{align*}
    p_1 &= mv_1 = 10\,\text{кг} \cdot 7\,\frac{\text{м}}{\text{с}} = 70\,\frac{\text{кг}\cdot\text{м}}{\text{с}}, \\
    p_2 &= mv_2 = 10\,\text{кг} \cdot 24\,\frac{\text{м}}{\text{с}} = 240\,\frac{\text{кг}\cdot\text{м}}{\text{с}}, \\
    p &= \sqrt{p_1^2 + p_2^2} = m\sqrt{v_1^2 + v_2^2} = 250\,\frac{\text{кг}\cdot\text{м}}{\text{с}}.
    \end{align*}
}
\solutionspace{120pt}

\tasknumber{4}%
\task{%
    Паровоз массой $M = 120\,\text{т}$, скорость которого равна $v = 0{,}40\,\frac{\text{м}}{\text{с}}$,
    сталкивается с четыремя неподвижными вагонами массой $m = 30\,\text{т}$ каждый и сцепляется с ними.
    Запишите (формулами, не числами) импульсы каждого из тел до и после сцепки и после,
    а также определите скорость их совместного движения.
}
\answer{%
    \begin{align*}
    \text{ЗСИ:} &M \cdot v + 4 \cdot \cbr{m \cdot 0} =  M \cdot u + 4 \cdot \cbr{m \cdot u} \implies \\
    &\implies u = v \cdot \frac{M}{M + nm} = 0{,}40\,\frac{\text{м}}{\text{с}} \cdot  \frac{ 120\,\text{т} }{120\,\text{т} + 4 \cdot 30\,\text{т}} \approx 0{,}20\,\frac{\text{м}}{\text{с}}.
    \end{align*}
}
\solutionspace{120pt}

\tasknumber{5}%
\task{%
    Два тела двигаются навстречу друг другу.
    Скорость каждого из них составляет $2\,\frac{\text{м}}{\text{с}}$.
    После соударения тела слиплись и продолжили движение уже со скоростью $1\,\frac{\text{м}}{\text{с}}$.
    Определите отношение масс тел (большей к меньшей).
}
\answer{%
    \begin{align*}
    &\text{ЗСИ в проекции на ось, соединяющую центры тел:} m_1 v_1 - m_2 v_1 = (m_1 + m_2) v_2 \implies \\
    &\implies \frac{m_1}{m_2} v_1 - v_1 = \cbr{\frac{m_1}{m_2} + 1} v_2 \implies
        \frac{m_1}{m_2} (v_1 - v_2) = v_2 + v_1 \implies \frac{m_1}{m_2} = \frac{v_2 + v_1}{v_1 - v_2} = 3
    \end{align*}
}

\variantsplitter

\addpersonalvariant{Андрей Рожков}

\tasknumber{1}%
\task{%
    Шарики массами $2\,\text{кг}$ и $3\,\text{кг}$
    движутся параллельно друг другу в одном направлении
    со скоростями $5\,\frac{\text{м}}{\text{с}}$ и $8\,\frac{\text{м}}{\text{с}}$ соответственно.
    Сделайте рисунок и укажите направления скоростей и импульсов.
    Определите импульс каждого из шариков, а также их суммарный импульс.
}
\answer{%
    \begin{align*}
    p_1 &= m_1v_1 = 2\,\text{кг} \cdot 5\,\frac{\text{м}}{\text{с}} = 10\,\frac{\text{кг}\cdot\text{м}}{\text{с}}, \\
    p_2 &= m_2v_2 = 3\,\text{кг} \cdot 8\,\frac{\text{м}}{\text{с}} = 24\,\frac{\text{кг}\cdot\text{м}}{\text{с}}, \\
    p &= p_1 + p_2 = m_1v_1 + m_2v_2 = 34\,\frac{\text{кг}\cdot\text{м}}{\text{с}}.
    \end{align*}
}
\solutionspace{120pt}

\tasknumber{2}%
\task{%
    Два шарика, масса каждого из которых составляет $5\,\text{кг}$,
    движутся навстречу друг другу.
    Скорость одного из них $2\,\frac{\text{м}}{\text{с}}$, а другого~--- $6\,\frac{\text{м}}{\text{с}}$.
    Сделайте рисунок, укажите направления скоростей и импульсов.
    Определите импульс каждого из шариков, а также их суммарный импульс.
}
\answer{%
    \begin{align*}
    p_1 &= mv_1 = 5\,\text{кг} \cdot 2\,\frac{\text{м}}{\text{с}} = 10\,\frac{\text{кг}\cdot\text{м}}{\text{с}}, \\
    p_2 &= mv_2 = 5\,\text{кг} \cdot 6\,\frac{\text{м}}{\text{с}} = 30\,\frac{\text{кг}\cdot\text{м}}{\text{с}}, \\
    p &= \abs{p_1 - p_2} = \abs{m(v_1 - v_2)}= 20\,\frac{\text{кг}\cdot\text{м}}{\text{с}}.
    \end{align*}
}
\solutionspace{120pt}

\tasknumber{3}%
\task{%
    Два одинаковых шарика массами по $10\,\text{кг}$
    движутся во взаимно перпендикулярных направлениях.
    Скорости шариков составляют $7\,\frac{\text{м}}{\text{с}}$ и $24\,\frac{\text{м}}{\text{с}}$.
    Сделайте рисунок, укажите направления скоростей и импульсов.
    Определите импульс каждого из шариков и полный импульс системы.
}
\answer{%
    \begin{align*}
    p_1 &= mv_1 = 10\,\text{кг} \cdot 7\,\frac{\text{м}}{\text{с}} = 70\,\frac{\text{кг}\cdot\text{м}}{\text{с}}, \\
    p_2 &= mv_2 = 10\,\text{кг} \cdot 24\,\frac{\text{м}}{\text{с}} = 240\,\frac{\text{кг}\cdot\text{м}}{\text{с}}, \\
    p &= \sqrt{p_1^2 + p_2^2} = m\sqrt{v_1^2 + v_2^2} = 250\,\frac{\text{кг}\cdot\text{м}}{\text{с}}.
    \end{align*}
}
\solutionspace{120pt}

\tasknumber{4}%
\task{%
    Паровоз массой $M = 120\,\text{т}$, скорость которого равна $v = 0{,}60\,\frac{\text{м}}{\text{с}}$,
    сталкивается с двумя неподвижными вагонами массой $m = 40\,\text{т}$ каждый и сцепляется с ними.
    Запишите (формулами, не числами) импульсы каждого из тел до и после сцепки и после,
    а также определите скорость их совместного движения.
}
\answer{%
    \begin{align*}
    \text{ЗСИ:} &M \cdot v + 2 \cdot \cbr{m \cdot 0} =  M \cdot u + 2 \cdot \cbr{m \cdot u} \implies \\
    &\implies u = v \cdot \frac{M}{M + nm} = 0{,}60\,\frac{\text{м}}{\text{с}} \cdot  \frac{ 120\,\text{т} }{120\,\text{т} + 2 \cdot 40\,\text{т}} \approx 0{,}36\,\frac{\text{м}}{\text{с}}.
    \end{align*}
}
\solutionspace{120pt}

\tasknumber{5}%
\task{%
    Два тела двигаются навстречу друг другу.
    Скорость каждого из них составляет $4\,\frac{\text{м}}{\text{с}}$.
    После соударения тела слиплись и продолжили движение уже со скоростью $3\,\frac{\text{м}}{\text{с}}$.
    Определите отношение масс тел (большей к меньшей).
}
\answer{%
    \begin{align*}
    &\text{ЗСИ в проекции на ось, соединяющую центры тел:} m_1 v_1 - m_2 v_1 = (m_1 + m_2) v_2 \implies \\
    &\implies \frac{m_1}{m_2} v_1 - v_1 = \cbr{\frac{m_1}{m_2} + 1} v_2 \implies
        \frac{m_1}{m_2} (v_1 - v_2) = v_2 + v_1 \implies \frac{m_1}{m_2} = \frac{v_2 + v_1}{v_1 - v_2} = 7
    \end{align*}
}

\variantsplitter

\addpersonalvariant{Тимур Сидиков}

\tasknumber{1}%
\task{%
    Шарики массами $3\,\text{кг}$ и $2\,\text{кг}$
    движутся параллельно друг другу в одном направлении
    со скоростями $10\,\frac{\text{м}}{\text{с}}$ и $6\,\frac{\text{м}}{\text{с}}$ соответственно.
    Сделайте рисунок и укажите направления скоростей и импульсов.
    Определите импульс каждого из шариков, а также их суммарный импульс.
}
\answer{%
    \begin{align*}
    p_1 &= m_1v_1 = 3\,\text{кг} \cdot 10\,\frac{\text{м}}{\text{с}} = 30\,\frac{\text{кг}\cdot\text{м}}{\text{с}}, \\
    p_2 &= m_2v_2 = 2\,\text{кг} \cdot 6\,\frac{\text{м}}{\text{с}} = 12\,\frac{\text{кг}\cdot\text{м}}{\text{с}}, \\
    p &= p_1 + p_2 = m_1v_1 + m_2v_2 = 42\,\frac{\text{кг}\cdot\text{м}}{\text{с}}.
    \end{align*}
}
\solutionspace{120pt}

\tasknumber{2}%
\task{%
    Два шарика, масса каждого из которых составляет $2\,\text{кг}$,
    движутся навстречу друг другу.
    Скорость одного из них $2\,\frac{\text{м}}{\text{с}}$, а другого~--- $3\,\frac{\text{м}}{\text{с}}$.
    Сделайте рисунок, укажите направления скоростей и импульсов.
    Определите импульс каждого из шариков, а также их суммарный импульс.
}
\answer{%
    \begin{align*}
    p_1 &= mv_1 = 2\,\text{кг} \cdot 2\,\frac{\text{м}}{\text{с}} = 4\,\frac{\text{кг}\cdot\text{м}}{\text{с}}, \\
    p_2 &= mv_2 = 2\,\text{кг} \cdot 3\,\frac{\text{м}}{\text{с}} = 6\,\frac{\text{кг}\cdot\text{м}}{\text{с}}, \\
    p &= \abs{p_1 - p_2} = \abs{m(v_1 - v_2)}= 2\,\frac{\text{кг}\cdot\text{м}}{\text{с}}.
    \end{align*}
}
\solutionspace{120pt}

\tasknumber{3}%
\task{%
    Два одинаковых шарика массами по $10\,\text{кг}$
    движутся во взаимно перпендикулярных направлениях.
    Скорости шариков составляют $7\,\frac{\text{м}}{\text{с}}$ и $24\,\frac{\text{м}}{\text{с}}$.
    Сделайте рисунок, укажите направления скоростей и импульсов.
    Определите импульс каждого из шариков и полный импульс системы.
}
\answer{%
    \begin{align*}
    p_1 &= mv_1 = 10\,\text{кг} \cdot 7\,\frac{\text{м}}{\text{с}} = 70\,\frac{\text{кг}\cdot\text{м}}{\text{с}}, \\
    p_2 &= mv_2 = 10\,\text{кг} \cdot 24\,\frac{\text{м}}{\text{с}} = 240\,\frac{\text{кг}\cdot\text{м}}{\text{с}}, \\
    p &= \sqrt{p_1^2 + p_2^2} = m\sqrt{v_1^2 + v_2^2} = 250\,\frac{\text{кг}\cdot\text{м}}{\text{с}}.
    \end{align*}
}
\solutionspace{120pt}

\tasknumber{4}%
\task{%
    Паровоз массой $M = 120\,\text{т}$, скорость которого равна $v = 0{,}60\,\frac{\text{м}}{\text{с}}$,
    сталкивается с тремя неподвижными вагонами массой $m = 50\,\text{т}$ каждый и сцепляется с ними.
    Запишите (формулами, не числами) импульсы каждого из тел до и после сцепки и после,
    а также определите скорость их совместного движения.
}
\answer{%
    \begin{align*}
    \text{ЗСИ:} &M \cdot v + 3 \cdot \cbr{m \cdot 0} =  M \cdot u + 3 \cdot \cbr{m \cdot u} \implies \\
    &\implies u = v \cdot \frac{M}{M + nm} = 0{,}60\,\frac{\text{м}}{\text{с}} \cdot  \frac{ 120\,\text{т} }{120\,\text{т} + 3 \cdot 50\,\text{т}} \approx 0{,}27\,\frac{\text{м}}{\text{с}}.
    \end{align*}
}
\solutionspace{120pt}

\tasknumber{5}%
\task{%
    Два тела двигаются навстречу друг другу.
    Скорость каждого из них составляет $5\,\frac{\text{м}}{\text{с}}$.
    После соударения тела слиплись и продолжили движение уже со скоростью $4\,\frac{\text{м}}{\text{с}}$.
    Определите отношение масс тел (большей к меньшей).
}
\answer{%
    \begin{align*}
    &\text{ЗСИ в проекции на ось, соединяющую центры тел:} m_1 v_1 - m_2 v_1 = (m_1 + m_2) v_2 \implies \\
    &\implies \frac{m_1}{m_2} v_1 - v_1 = \cbr{\frac{m_1}{m_2} + 1} v_2 \implies
        \frac{m_1}{m_2} (v_1 - v_2) = v_2 + v_1 \implies \frac{m_1}{m_2} = \frac{v_2 + v_1}{v_1 - v_2} = 9
    \end{align*}
}

\variantsplitter

\addpersonalvariant{Рената Таржиманова}

\tasknumber{1}%
\task{%
    Шарики массами $1\,\text{кг}$ и $3\,\text{кг}$
    движутся параллельно друг другу в одном направлении
    со скоростями $4\,\frac{\text{м}}{\text{с}}$ и $6\,\frac{\text{м}}{\text{с}}$ соответственно.
    Сделайте рисунок и укажите направления скоростей и импульсов.
    Определите импульс каждого из шариков, а также их суммарный импульс.
}
\answer{%
    \begin{align*}
    p_1 &= m_1v_1 = 1\,\text{кг} \cdot 4\,\frac{\text{м}}{\text{с}} = 4\,\frac{\text{кг}\cdot\text{м}}{\text{с}}, \\
    p_2 &= m_2v_2 = 3\,\text{кг} \cdot 6\,\frac{\text{м}}{\text{с}} = 18\,\frac{\text{кг}\cdot\text{м}}{\text{с}}, \\
    p &= p_1 + p_2 = m_1v_1 + m_2v_2 = 22\,\frac{\text{кг}\cdot\text{м}}{\text{с}}.
    \end{align*}
}
\solutionspace{120pt}

\tasknumber{2}%
\task{%
    Два шарика, масса каждого из которых составляет $5\,\text{кг}$,
    движутся навстречу друг другу.
    Скорость одного из них $2\,\frac{\text{м}}{\text{с}}$, а другого~--- $6\,\frac{\text{м}}{\text{с}}$.
    Сделайте рисунок, укажите направления скоростей и импульсов.
    Определите импульс каждого из шариков, а также их суммарный импульс.
}
\answer{%
    \begin{align*}
    p_1 &= mv_1 = 5\,\text{кг} \cdot 2\,\frac{\text{м}}{\text{с}} = 10\,\frac{\text{кг}\cdot\text{м}}{\text{с}}, \\
    p_2 &= mv_2 = 5\,\text{кг} \cdot 6\,\frac{\text{м}}{\text{с}} = 30\,\frac{\text{кг}\cdot\text{м}}{\text{с}}, \\
    p &= \abs{p_1 - p_2} = \abs{m(v_1 - v_2)}= 20\,\frac{\text{кг}\cdot\text{м}}{\text{с}}.
    \end{align*}
}
\solutionspace{120pt}

\tasknumber{3}%
\task{%
    Два одинаковых шарика массами по $5\,\text{кг}$
    движутся во взаимно перпендикулярных направлениях.
    Скорости шариков составляют $5\,\frac{\text{м}}{\text{с}}$ и $12\,\frac{\text{м}}{\text{с}}$.
    Сделайте рисунок, укажите направления скоростей и импульсов.
    Определите импульс каждого из шариков и полный импульс системы.
}
\answer{%
    \begin{align*}
    p_1 &= mv_1 = 5\,\text{кг} \cdot 5\,\frac{\text{м}}{\text{с}} = 25\,\frac{\text{кг}\cdot\text{м}}{\text{с}}, \\
    p_2 &= mv_2 = 5\,\text{кг} \cdot 12\,\frac{\text{м}}{\text{с}} = 60\,\frac{\text{кг}\cdot\text{м}}{\text{с}}, \\
    p &= \sqrt{p_1^2 + p_2^2} = m\sqrt{v_1^2 + v_2^2} = 65\,\frac{\text{кг}\cdot\text{м}}{\text{с}}.
    \end{align*}
}
\solutionspace{120pt}

\tasknumber{4}%
\task{%
    Паровоз массой $M = 120\,\text{т}$, скорость которого равна $v = 0{,}40\,\frac{\text{м}}{\text{с}}$,
    сталкивается с тремя неподвижными вагонами массой $m = 50\,\text{т}$ каждый и сцепляется с ними.
    Запишите (формулами, не числами) импульсы каждого из тел до и после сцепки и после,
    а также определите скорость их совместного движения.
}
\answer{%
    \begin{align*}
    \text{ЗСИ:} &M \cdot v + 3 \cdot \cbr{m \cdot 0} =  M \cdot u + 3 \cdot \cbr{m \cdot u} \implies \\
    &\implies u = v \cdot \frac{M}{M + nm} = 0{,}40\,\frac{\text{м}}{\text{с}} \cdot  \frac{ 120\,\text{т} }{120\,\text{т} + 3 \cdot 50\,\text{т}} \approx 0{,}18\,\frac{\text{м}}{\text{с}}.
    \end{align*}
}
\solutionspace{120pt}

\tasknumber{5}%
\task{%
    Два тела двигаются навстречу друг другу.
    Скорость каждого из них составляет $5\,\frac{\text{м}}{\text{с}}$.
    После соударения тела слиплись и продолжили движение уже со скоростью $3\,\frac{\text{м}}{\text{с}}$.
    Определите отношение масс тел (большей к меньшей).
}
\answer{%
    \begin{align*}
    &\text{ЗСИ в проекции на ось, соединяющую центры тел:} m_1 v_1 - m_2 v_1 = (m_1 + m_2) v_2 \implies \\
    &\implies \frac{m_1}{m_2} v_1 - v_1 = \cbr{\frac{m_1}{m_2} + 1} v_2 \implies
        \frac{m_1}{m_2} (v_1 - v_2) = v_2 + v_1 \implies \frac{m_1}{m_2} = \frac{v_2 + v_1}{v_1 - v_2} = 4
    \end{align*}
}

\variantsplitter

\addpersonalvariant{Глеб Урбанский}

\tasknumber{1}%
\task{%
    Шарики массами $1\,\text{кг}$ и $3\,\text{кг}$
    движутся параллельно друг другу в одном направлении
    со скоростями $5\,\frac{\text{м}}{\text{с}}$ и $6\,\frac{\text{м}}{\text{с}}$ соответственно.
    Сделайте рисунок и укажите направления скоростей и импульсов.
    Определите импульс каждого из шариков, а также их суммарный импульс.
}
\answer{%
    \begin{align*}
    p_1 &= m_1v_1 = 1\,\text{кг} \cdot 5\,\frac{\text{м}}{\text{с}} = 5\,\frac{\text{кг}\cdot\text{м}}{\text{с}}, \\
    p_2 &= m_2v_2 = 3\,\text{кг} \cdot 6\,\frac{\text{м}}{\text{с}} = 18\,\frac{\text{кг}\cdot\text{м}}{\text{с}}, \\
    p &= p_1 + p_2 = m_1v_1 + m_2v_2 = 23\,\frac{\text{кг}\cdot\text{м}}{\text{с}}.
    \end{align*}
}
\solutionspace{120pt}

\tasknumber{2}%
\task{%
    Два шарика, масса каждого из которых составляет $10\,\text{кг}$,
    движутся навстречу друг другу.
    Скорость одного из них $10\,\frac{\text{м}}{\text{с}}$, а другого~--- $8\,\frac{\text{м}}{\text{с}}$.
    Сделайте рисунок, укажите направления скоростей и импульсов.
    Определите импульс каждого из шариков, а также их суммарный импульс.
}
\answer{%
    \begin{align*}
    p_1 &= mv_1 = 10\,\text{кг} \cdot 10\,\frac{\text{м}}{\text{с}} = 100\,\frac{\text{кг}\cdot\text{м}}{\text{с}}, \\
    p_2 &= mv_2 = 10\,\text{кг} \cdot 8\,\frac{\text{м}}{\text{с}} = 80\,\frac{\text{кг}\cdot\text{м}}{\text{с}}, \\
    p &= \abs{p_1 - p_2} = \abs{m(v_1 - v_2)}= 20\,\frac{\text{кг}\cdot\text{м}}{\text{с}}.
    \end{align*}
}
\solutionspace{120pt}

\tasknumber{3}%
\task{%
    Два одинаковых шарика массами по $10\,\text{кг}$
    движутся во взаимно перпендикулярных направлениях.
    Скорости шариков составляют $3\,\frac{\text{м}}{\text{с}}$ и $4\,\frac{\text{м}}{\text{с}}$.
    Сделайте рисунок, укажите направления скоростей и импульсов.
    Определите импульс каждого из шариков и полный импульс системы.
}
\answer{%
    \begin{align*}
    p_1 &= mv_1 = 10\,\text{кг} \cdot 3\,\frac{\text{м}}{\text{с}} = 30\,\frac{\text{кг}\cdot\text{м}}{\text{с}}, \\
    p_2 &= mv_2 = 10\,\text{кг} \cdot 4\,\frac{\text{м}}{\text{с}} = 40\,\frac{\text{кг}\cdot\text{м}}{\text{с}}, \\
    p &= \sqrt{p_1^2 + p_2^2} = m\sqrt{v_1^2 + v_2^2} = 50\,\frac{\text{кг}\cdot\text{м}}{\text{с}}.
    \end{align*}
}
\solutionspace{120pt}

\tasknumber{4}%
\task{%
    Паровоз массой $M = 120\,\text{т}$, скорость которого равна $v = 0{,}20\,\frac{\text{м}}{\text{с}}$,
    сталкивается с двумя неподвижными вагонами массой $m = 30\,\text{т}$ каждый и сцепляется с ними.
    Запишите (формулами, не числами) импульсы каждого из тел до и после сцепки и после,
    а также определите скорость их совместного движения.
}
\answer{%
    \begin{align*}
    \text{ЗСИ:} &M \cdot v + 2 \cdot \cbr{m \cdot 0} =  M \cdot u + 2 \cdot \cbr{m \cdot u} \implies \\
    &\implies u = v \cdot \frac{M}{M + nm} = 0{,}20\,\frac{\text{м}}{\text{с}} \cdot  \frac{ 120\,\text{т} }{120\,\text{т} + 2 \cdot 30\,\text{т}} \approx 0{,}13\,\frac{\text{м}}{\text{с}}.
    \end{align*}
}
\solutionspace{120pt}

\tasknumber{5}%
\task{%
    Два тела двигаются навстречу друг другу.
    Скорость каждого из них составляет $6\,\frac{\text{м}}{\text{с}}$.
    После соударения тела слиплись и продолжили движение уже со скоростью $3\,\frac{\text{м}}{\text{с}}$.
    Определите отношение масс тел (большей к меньшей).
}
\answer{%
    \begin{align*}
    &\text{ЗСИ в проекции на ось, соединяющую центры тел:} m_1 v_1 - m_2 v_1 = (m_1 + m_2) v_2 \implies \\
    &\implies \frac{m_1}{m_2} v_1 - v_1 = \cbr{\frac{m_1}{m_2} + 1} v_2 \implies
        \frac{m_1}{m_2} (v_1 - v_2) = v_2 + v_1 \implies \frac{m_1}{m_2} = \frac{v_2 + v_1}{v_1 - v_2} = 3
    \end{align*}
}

\variantsplitter

\addpersonalvariant{Кирилл Швец}

\tasknumber{1}%
\task{%
    Шарики массами $2\,\text{кг}$ и $1\,\text{кг}$
    движутся параллельно друг другу в одном направлении
    со скоростями $2\,\frac{\text{м}}{\text{с}}$ и $8\,\frac{\text{м}}{\text{с}}$ соответственно.
    Сделайте рисунок и укажите направления скоростей и импульсов.
    Определите импульс каждого из шариков, а также их суммарный импульс.
}
\answer{%
    \begin{align*}
    p_1 &= m_1v_1 = 2\,\text{кг} \cdot 2\,\frac{\text{м}}{\text{с}} = 4\,\frac{\text{кг}\cdot\text{м}}{\text{с}}, \\
    p_2 &= m_2v_2 = 1\,\text{кг} \cdot 8\,\frac{\text{м}}{\text{с}} = 8\,\frac{\text{кг}\cdot\text{м}}{\text{с}}, \\
    p &= p_1 + p_2 = m_1v_1 + m_2v_2 = 12\,\frac{\text{кг}\cdot\text{м}}{\text{с}}.
    \end{align*}
}
\solutionspace{120pt}

\tasknumber{2}%
\task{%
    Два шарика, масса каждого из которых составляет $5\,\text{кг}$,
    движутся навстречу друг другу.
    Скорость одного из них $10\,\frac{\text{м}}{\text{с}}$, а другого~--- $3\,\frac{\text{м}}{\text{с}}$.
    Сделайте рисунок, укажите направления скоростей и импульсов.
    Определите импульс каждого из шариков, а также их суммарный импульс.
}
\answer{%
    \begin{align*}
    p_1 &= mv_1 = 5\,\text{кг} \cdot 10\,\frac{\text{м}}{\text{с}} = 50\,\frac{\text{кг}\cdot\text{м}}{\text{с}}, \\
    p_2 &= mv_2 = 5\,\text{кг} \cdot 3\,\frac{\text{м}}{\text{с}} = 15\,\frac{\text{кг}\cdot\text{м}}{\text{с}}, \\
    p &= \abs{p_1 - p_2} = \abs{m(v_1 - v_2)}= 35\,\frac{\text{кг}\cdot\text{м}}{\text{с}}.
    \end{align*}
}
\solutionspace{120pt}

\tasknumber{3}%
\task{%
    Два одинаковых шарика массами по $10\,\text{кг}$
    движутся во взаимно перпендикулярных направлениях.
    Скорости шариков составляют $7\,\frac{\text{м}}{\text{с}}$ и $24\,\frac{\text{м}}{\text{с}}$.
    Сделайте рисунок, укажите направления скоростей и импульсов.
    Определите импульс каждого из шариков и полный импульс системы.
}
\answer{%
    \begin{align*}
    p_1 &= mv_1 = 10\,\text{кг} \cdot 7\,\frac{\text{м}}{\text{с}} = 70\,\frac{\text{кг}\cdot\text{м}}{\text{с}}, \\
    p_2 &= mv_2 = 10\,\text{кг} \cdot 24\,\frac{\text{м}}{\text{с}} = 240\,\frac{\text{кг}\cdot\text{м}}{\text{с}}, \\
    p &= \sqrt{p_1^2 + p_2^2} = m\sqrt{v_1^2 + v_2^2} = 250\,\frac{\text{кг}\cdot\text{м}}{\text{с}}.
    \end{align*}
}
\solutionspace{120pt}

\tasknumber{4}%
\task{%
    Паровоз массой $M = 150\,\text{т}$, скорость которого равна $v = 0{,}60\,\frac{\text{м}}{\text{с}}$,
    сталкивается с двумя неподвижными вагонами массой $m = 30\,\text{т}$ каждый и сцепляется с ними.
    Запишите (формулами, не числами) импульсы каждого из тел до и после сцепки и после,
    а также определите скорость их совместного движения.
}
\answer{%
    \begin{align*}
    \text{ЗСИ:} &M \cdot v + 2 \cdot \cbr{m \cdot 0} =  M \cdot u + 2 \cdot \cbr{m \cdot u} \implies \\
    &\implies u = v \cdot \frac{M}{M + nm} = 0{,}60\,\frac{\text{м}}{\text{с}} \cdot  \frac{ 150\,\text{т} }{150\,\text{т} + 2 \cdot 30\,\text{т}} \approx 0{,}43\,\frac{\text{м}}{\text{с}}.
    \end{align*}
}
\solutionspace{120pt}

\tasknumber{5}%
\task{%
    Два тела двигаются навстречу друг другу.
    Скорость каждого из них составляет $6\,\frac{\text{м}}{\text{с}}$.
    После соударения тела слиплись и продолжили движение уже со скоростью $4\,\frac{\text{м}}{\text{с}}$.
    Определите отношение масс тел (большей к меньшей).
}
\answer{%
    \begin{align*}
    &\text{ЗСИ в проекции на ось, соединяющую центры тел:} m_1 v_1 - m_2 v_1 = (m_1 + m_2) v_2 \implies \\
    &\implies \frac{m_1}{m_2} v_1 - v_1 = \cbr{\frac{m_1}{m_2} + 1} v_2 \implies
        \frac{m_1}{m_2} (v_1 - v_2) = v_2 + v_1 \implies \frac{m_1}{m_2} = \frac{v_2 + v_1}{v_1 - v_2} = 5
    \end{align*}
}

\variantsplitter

\addpersonalvariant{Андрей Щербаков}

\tasknumber{1}%
\task{%
    Шарики массами $2\,\text{кг}$ и $3\,\text{кг}$
    движутся параллельно друг другу в одном направлении
    со скоростями $2\,\frac{\text{м}}{\text{с}}$ и $3\,\frac{\text{м}}{\text{с}}$ соответственно.
    Сделайте рисунок и укажите направления скоростей и импульсов.
    Определите импульс каждого из шариков, а также их суммарный импульс.
}
\answer{%
    \begin{align*}
    p_1 &= m_1v_1 = 2\,\text{кг} \cdot 2\,\frac{\text{м}}{\text{с}} = 4\,\frac{\text{кг}\cdot\text{м}}{\text{с}}, \\
    p_2 &= m_2v_2 = 3\,\text{кг} \cdot 3\,\frac{\text{м}}{\text{с}} = 9\,\frac{\text{кг}\cdot\text{м}}{\text{с}}, \\
    p &= p_1 + p_2 = m_1v_1 + m_2v_2 = 13\,\frac{\text{кг}\cdot\text{м}}{\text{с}}.
    \end{align*}
}
\solutionspace{120pt}

\tasknumber{2}%
\task{%
    Два шарика, масса каждого из которых составляет $10\,\text{кг}$,
    движутся навстречу друг другу.
    Скорость одного из них $10\,\frac{\text{м}}{\text{с}}$, а другого~--- $6\,\frac{\text{м}}{\text{с}}$.
    Сделайте рисунок, укажите направления скоростей и импульсов.
    Определите импульс каждого из шариков, а также их суммарный импульс.
}
\answer{%
    \begin{align*}
    p_1 &= mv_1 = 10\,\text{кг} \cdot 10\,\frac{\text{м}}{\text{с}} = 100\,\frac{\text{кг}\cdot\text{м}}{\text{с}}, \\
    p_2 &= mv_2 = 10\,\text{кг} \cdot 6\,\frac{\text{м}}{\text{с}} = 60\,\frac{\text{кг}\cdot\text{м}}{\text{с}}, \\
    p &= \abs{p_1 - p_2} = \abs{m(v_1 - v_2)}= 40\,\frac{\text{кг}\cdot\text{м}}{\text{с}}.
    \end{align*}
}
\solutionspace{120pt}

\tasknumber{3}%
\task{%
    Два одинаковых шарика массами по $10\,\text{кг}$
    движутся во взаимно перпендикулярных направлениях.
    Скорости шариков составляют $3\,\frac{\text{м}}{\text{с}}$ и $4\,\frac{\text{м}}{\text{с}}$.
    Сделайте рисунок, укажите направления скоростей и импульсов.
    Определите импульс каждого из шариков и полный импульс системы.
}
\answer{%
    \begin{align*}
    p_1 &= mv_1 = 10\,\text{кг} \cdot 3\,\frac{\text{м}}{\text{с}} = 30\,\frac{\text{кг}\cdot\text{м}}{\text{с}}, \\
    p_2 &= mv_2 = 10\,\text{кг} \cdot 4\,\frac{\text{м}}{\text{с}} = 40\,\frac{\text{кг}\cdot\text{м}}{\text{с}}, \\
    p &= \sqrt{p_1^2 + p_2^2} = m\sqrt{v_1^2 + v_2^2} = 50\,\frac{\text{кг}\cdot\text{м}}{\text{с}}.
    \end{align*}
}
\solutionspace{120pt}

\tasknumber{4}%
\task{%
    Паровоз массой $M = 210\,\text{т}$, скорость которого равна $v = 0{,}40\,\frac{\text{м}}{\text{с}}$,
    сталкивается с двумя неподвижными вагонами массой $m = 30\,\text{т}$ каждый и сцепляется с ними.
    Запишите (формулами, не числами) импульсы каждого из тел до и после сцепки и после,
    а также определите скорость их совместного движения.
}
\answer{%
    \begin{align*}
    \text{ЗСИ:} &M \cdot v + 2 \cdot \cbr{m \cdot 0} =  M \cdot u + 2 \cdot \cbr{m \cdot u} \implies \\
    &\implies u = v \cdot \frac{M}{M + nm} = 0{,}40\,\frac{\text{м}}{\text{с}} \cdot  \frac{ 210\,\text{т} }{210\,\text{т} + 2 \cdot 30\,\text{т}} \approx 0{,}31\,\frac{\text{м}}{\text{с}}.
    \end{align*}
}
\solutionspace{120pt}

\tasknumber{5}%
\task{%
    Два тела двигаются навстречу друг другу.
    Скорость каждого из них составляет $7\,\frac{\text{м}}{\text{с}}$.
    После соударения тела слиплись и продолжили движение уже со скоростью $5\,\frac{\text{м}}{\text{с}}$.
    Определите отношение масс тел (большей к меньшей).
}
\answer{%
    \begin{align*}
    &\text{ЗСИ в проекции на ось, соединяющую центры тел:} m_1 v_1 - m_2 v_1 = (m_1 + m_2) v_2 \implies \\
    &\implies \frac{m_1}{m_2} v_1 - v_1 = \cbr{\frac{m_1}{m_2} + 1} v_2 \implies
        \frac{m_1}{m_2} (v_1 - v_2) = v_2 + v_1 \implies \frac{m_1}{m_2} = \frac{v_2 + v_1}{v_1 - v_2} = 6
    \end{align*}
}

\variantsplitter

\addpersonalvariant{Михаил Ярошевский}

\tasknumber{1}%
\task{%
    Шарики массами $1\,\text{кг}$ и $4\,\text{кг}$
    движутся параллельно друг другу в одном направлении
    со скоростями $10\,\frac{\text{м}}{\text{с}}$ и $3\,\frac{\text{м}}{\text{с}}$ соответственно.
    Сделайте рисунок и укажите направления скоростей и импульсов.
    Определите импульс каждого из шариков, а также их суммарный импульс.
}
\answer{%
    \begin{align*}
    p_1 &= m_1v_1 = 1\,\text{кг} \cdot 10\,\frac{\text{м}}{\text{с}} = 10\,\frac{\text{кг}\cdot\text{м}}{\text{с}}, \\
    p_2 &= m_2v_2 = 4\,\text{кг} \cdot 3\,\frac{\text{м}}{\text{с}} = 12\,\frac{\text{кг}\cdot\text{м}}{\text{с}}, \\
    p &= p_1 + p_2 = m_1v_1 + m_2v_2 = 22\,\frac{\text{кг}\cdot\text{м}}{\text{с}}.
    \end{align*}
}
\solutionspace{120pt}

\tasknumber{2}%
\task{%
    Два шарика, масса каждого из которых составляет $2\,\text{кг}$,
    движутся навстречу друг другу.
    Скорость одного из них $10\,\frac{\text{м}}{\text{с}}$, а другого~--- $6\,\frac{\text{м}}{\text{с}}$.
    Сделайте рисунок, укажите направления скоростей и импульсов.
    Определите импульс каждого из шариков, а также их суммарный импульс.
}
\answer{%
    \begin{align*}
    p_1 &= mv_1 = 2\,\text{кг} \cdot 10\,\frac{\text{м}}{\text{с}} = 20\,\frac{\text{кг}\cdot\text{м}}{\text{с}}, \\
    p_2 &= mv_2 = 2\,\text{кг} \cdot 6\,\frac{\text{м}}{\text{с}} = 12\,\frac{\text{кг}\cdot\text{м}}{\text{с}}, \\
    p &= \abs{p_1 - p_2} = \abs{m(v_1 - v_2)}= 8\,\frac{\text{кг}\cdot\text{м}}{\text{с}}.
    \end{align*}
}
\solutionspace{120pt}

\tasknumber{3}%
\task{%
    Два одинаковых шарика массами по $10\,\text{кг}$
    движутся во взаимно перпендикулярных направлениях.
    Скорости шариков составляют $7\,\frac{\text{м}}{\text{с}}$ и $24\,\frac{\text{м}}{\text{с}}$.
    Сделайте рисунок, укажите направления скоростей и импульсов.
    Определите импульс каждого из шариков и полный импульс системы.
}
\answer{%
    \begin{align*}
    p_1 &= mv_1 = 10\,\text{кг} \cdot 7\,\frac{\text{м}}{\text{с}} = 70\,\frac{\text{кг}\cdot\text{м}}{\text{с}}, \\
    p_2 &= mv_2 = 10\,\text{кг} \cdot 24\,\frac{\text{м}}{\text{с}} = 240\,\frac{\text{кг}\cdot\text{м}}{\text{с}}, \\
    p &= \sqrt{p_1^2 + p_2^2} = m\sqrt{v_1^2 + v_2^2} = 250\,\frac{\text{кг}\cdot\text{м}}{\text{с}}.
    \end{align*}
}
\solutionspace{120pt}

\tasknumber{4}%
\task{%
    Паровоз массой $M = 120\,\text{т}$, скорость которого равна $v = 0{,}20\,\frac{\text{м}}{\text{с}}$,
    сталкивается с двумя неподвижными вагонами массой $m = 30\,\text{т}$ каждый и сцепляется с ними.
    Запишите (формулами, не числами) импульсы каждого из тел до и после сцепки и после,
    а также определите скорость их совместного движения.
}
\answer{%
    \begin{align*}
    \text{ЗСИ:} &M \cdot v + 2 \cdot \cbr{m \cdot 0} =  M \cdot u + 2 \cdot \cbr{m \cdot u} \implies \\
    &\implies u = v \cdot \frac{M}{M + nm} = 0{,}20\,\frac{\text{м}}{\text{с}} \cdot  \frac{ 120\,\text{т} }{120\,\text{т} + 2 \cdot 30\,\text{т}} \approx 0{,}13\,\frac{\text{м}}{\text{с}}.
    \end{align*}
}
\solutionspace{120pt}

\tasknumber{5}%
\task{%
    Два тела двигаются навстречу друг другу.
    Скорость каждого из них составляет $7\,\frac{\text{м}}{\text{с}}$.
    После соударения тела слиплись и продолжили движение уже со скоростью $5\,\frac{\text{м}}{\text{с}}$.
    Определите отношение масс тел (большей к меньшей).
}
\answer{%
    \begin{align*}
    &\text{ЗСИ в проекции на ось, соединяющую центры тел:} m_1 v_1 - m_2 v_1 = (m_1 + m_2) v_2 \implies \\
    &\implies \frac{m_1}{m_2} v_1 - v_1 = \cbr{\frac{m_1}{m_2} + 1} v_2 \implies
        \frac{m_1}{m_2} (v_1 - v_2) = v_2 + v_1 \implies \frac{m_1}{m_2} = \frac{v_2 + v_1}{v_1 - v_2} = 6
    \end{align*}
}
% autogenerated
