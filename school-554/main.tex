\documentclass[12pt,a4paper]{amsart}%DVI-mode.
\usepackage{graphics,graphicx,epsfig}%DVI-mode.
% \documentclass[pdftex,12pt]{amsart} %PDF-mode.
% \usepackage[pdftex]{graphicx}       %PDF-mode.
\usepackage[babel=true]{microtype}
% \usepackage[T1]{fontenc}
% \usepackage{lmodern}

\usepackage{cmap}
\usepackage{multicol}
%\usepackage{a4wide}                 % Fit the text to A4 page tightly.
\usepackage[utf8]{inputenc}
\usepackage[T2A]{fontenc}
\usepackage[russian]{babel} % Download Russian fonts.
% \usepackage[english,russian]{babel} % Download Russian fonts.
\usepackage{amsmath,amsfonts,amssymb,amsthm,amscd,mathrsfs} % Use AMS symbols.
\usepackage{tikz}
\usepackage{tkz-euclide}  % https://gist.github.com/kpym/f08d1a326884e11dddb7d272359eb2bd
\usepackage[version=4]{mhchem}
\usepackage{wrapfig}  % for wrapfigure, see https://www.overleaf.com/learn/latex/Wrapping_text_around_figures
\usetikzlibrary{circuits.ee.IEC}
\usetikzlibrary{shapes.geometric}
\usetikzlibrary{decorations.markings}
\usetikzlibrary{arrows, arrows.meta}  % https://tex.stackexchange.com/questions/298777/how-to-draw-this-p-v-graph
\usetikzlibrary{patterns.meta}
%\usetikzlibrary{dashs}
%\usetikzlibrary{info}
% \usetkzobj{all}
\tkzSetUpPoint[fill=black]  % https://tex.stackexchange.com/questions/165803/how-can-i-style-tkz-euclide-points

\usepackage{xcolor}
\usepackage[colorlinks,urlcolor=blue]{hyperref}  % for \href

\textheight=28cm % высота текста
\textwidth=19cm % ширина текста
\topmargin=-2.5cm % отступ от верхнего края
\parskip=2pt % интервал между абзацами
\oddsidemargin=-1.5cm
\evensidemargin=-1.5cm

\parindent=0pt % абзацный отступ
\tolerance=500 % терпимость к "жидким" строкам
\binoppenalty=10000 % штраф за перенос формул - 10000 - абсолютный запрет
\relpenalty=10000
\flushbottom % выравнивание высоты страниц
\pagenumbering{gobble}

\providecommand{\rootpath}{..}
\newcommand\bivec[2]{\begin{pmatrix} #1 \\ #2 \end{pmatrix}}

\newcommand\ol[1]{\overline{#1}}

\newcommand\p[1]{\Prob\!\left(#1\right)}
\newcommand\e[1]{\mathsf{E}\!\left(#1\right)}
\newcommand\disp[1]{\mathsf{D}\!\left(#1\right)}
%\newcommand\norm[2]{\mathcal{N}\!\cbr{#1,#2}}
\newcommand\sign{\text{ sign }}

\newcommand\al[1]{\begin{align*} #1 \end{align*}}
\newcommand\begcas[1]{\begin{cases}#1\end{cases}}
\newcommand\tab[2]{	\vspace{-#1pt}
						\begin{tabbing} 
						#2
						\end{tabbing}
					\vspace{-#1pt}
					}

\newcommand\maintext[1]{{\bfseries\sffamily{#1}}}
\newcommand\skipped[1]{ {\ensuremath{\text{\small{\sffamily{Пропущено:} #1} } } } }
\newcommand\simpletitle[1]{\begin{center} \maintext{#1} \end{center}}

\def\le{\leqslant}
\def\ge{\geqslant}
\def\Ell{\mathcal{L}}
\def\eps{{\varepsilon}}
\def\Rn{\mathbb{R}^n}
\def\RSS{\mathsf{RSS}}

\newcommand\foral[1]{\forall\,#1\:}
\newcommand\exist[1]{\exists\,#1\:\colon}

\newcommand\cbr[1]{\left(#1\right)} %circled brackets
\newcommand\fbr[1]{\left\{#1\right\}} %figure brackets
\newcommand\sbr[1]{\left[#1\right]} %square brackets
\newcommand\modul[1]{\left|#1\right|}

\newcommand\sqr[1]{\cbr{#1}^2}
\newcommand\inv[1]{\cbr{#1}^{-1}}

\newcommand\cdf[2]{\cdot\frac{#1}{#2}}
\newcommand\dd[2]{\frac{\partial#1}{\partial#2}}

\newcommand\integr[2]{\int\limits_{#1}^{#2}}
\newcommand\suml[2]{\sum\limits_{#1}^{#2}}
\newcommand\isum[2]{\sum\limits_{#1=#2}^{+\infty}}
\newcommand\idots[3]{#1_{#2},\ldots,#1_{#3}}
\newcommand\fdots[5]{#4{#1_{#2}}#5\ldots#5#4{#1_{#3}}}

\newcommand\obol[1]{O\!\cbr{#1}}
\newcommand\omal[1]{o\!\cbr{#1}}

\newcommand\addeps[2]{
	\begin{figure} [!ht] %lrp
		\centering
		\includegraphics[height=320px]{#1.eps}
		\vspace{-10pt}
		\caption{#2}
		\label{eps:#1}
	\end{figure}
}

\newcommand\addepssize[3]{
	\begin{figure} [!ht] %lrp hp
		\centering
		\includegraphics[height=#3px]{#1.eps}
		\vspace{-10pt}
		\caption{#2}
		\label{eps:#1}
	\end{figure}
}


\newcommand\norm[1]{\ensuremath{\left\|{#1}\right\|}}
\newcommand\ort{\bot}
\newcommand\theorem[1]{{\sffamily Теорема #1\ }}
\newcommand\lemma[1]{{\sffamily Лемма #1\ }}
\newcommand\difflim[2]{\frac{#1\cbr{#2 + \Delta#2} - #1\cbr{#2}}{\Delta #2}}
\renewcommand\proof[1]{\par\noindent$\square$ #1 \hfill$\blacksquare$\par}
\newcommand\defenition[1]{{\sffamilyОпределение #1\ }}

\renewcommand{\theenumi}{\asbuk{enumi}}
\renewcommand{\labelenumi}{\asbuk{enumi})}

% \begin{document}
% %\raggedright
% % \begin{flushright}
\textsc{ГБОУ школа №554, 20 ноября 2018\,г.}
\end{flushright}

\begin{center}
\LARGE \textsc{Математический бой, 8 класс}
\end{center}

\problem{1} Есть тридцать карточек, на каждой написано по одному числу: на десяти карточках~–~$a$,  на десяти других~–~$b$ и на десяти оставшихся~–~$c$ (числа  различны). Известно, что к любым пяти карточкам можно подобрать ещё пять так, что сумма чисел на этих десяти карточках будет равна нулю. Докажите, что~одно из~чисел~$a, b, c$ равно нулю.

\problem{2} Вокруг стола стола пустили пакет с орешками. Первый взял один орешек, второй — 2, третий — 3 и так далее: каждый следующий брал на 1 орешек больше. Известно, что на втором круге было взято в сумме на 100 орешков больше, чем на первом. Сколько человек сидело за столом?

% \problem{2} Натуральное число разрешено увеличить на любое целое число процентов от 1 до 100, если при этом получаем натуральное число. Найдите наименьшее натуральное число, которое нельзя при помощи таких операций получить из~числа 1.

% \problem{3} Найти сумму $1^2 - 2^2 + 3^2 - 4^2 + 5^2 + \ldots - 2018^2$.

\problem{3} В кружке рукоделия, где занимается Валя, более 93\% участников~—~девочки. Какое наименьшее число участников может быть в таком кружке?

\problem{4} Произведение 2018 целых чисел равно 1. Может ли их сумма оказаться равной~0?

% \problem{4} Можно ли все натуральные числа от~1 до~9 записать в~клетки таблицы~$3\times3$ так, чтобы сумма в~любых двух соседних (по~вертикали или горизонтали) клетках равнялось простому числу?

\problem{5} На доске написано 2018 нулей и 2019 единиц. Женя стирает 2 числа и, если они были одинаковы, дописывает к оставшимся один ноль, а~если разные — единицу. Потом Женя повторяет эту операцию снова, потом ещё и~так далее. В~результате на~доске останется только одно число. Что это за~число?

\problem{6} Докажите, что в~любой компании людей найдутся 2~человека, имеющие равное число знакомых в этой компании (если $A$~знаком с~$B$, то~и $B$~знаком с~$A$).

\problem{7} Три колокола начинают бить одновременно. Интервалы между ударами колоколов соответственно составляют $\cfrac43$~секунды, $\cfrac53$~секунды и $2$~секунды. Совпавшие по времени удары воспринимаются за~один. Сколько ударов будет услышано за 1~минуту, включая первый и последний удары?

\problem{8} Восемь одинаковых момент расположены по кругу. Известно, что три из~них~— фальшивые, и они расположены рядом друг с~другом. Вес фальшивой монеты отличается от~веса настоящей. Все фальшивые монеты весят одинаково, но неизвестно, тяжелее или легче фальшивая монета настоящей. Покажите, что за~3~взвешивания на~чашечных весах без~гирь можно определить все фальшивые монеты.

% \end{document}
