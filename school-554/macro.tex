\newcommand\units[1]{{\,\text{#1}}}
\newcommand\powtext[2]{{\text{#1}^{#2}}}
\newcommand\dunits[2]{{\,\powtext{#1}{#2}}}
\newcommand\fracunits[4]{{\,\frac{\powtext{#1}{#2}}{\powtext{#3}{#4}}}}
\newcommand\cfracunits[4]{{\,\cfrac{\powtext{#1}{#2}}{\powtext{#3}{#4}}}}
\newcommand\dfracunits[4]{{\,\dfrac{\powtext{#1}{#2}}{\powtext{#3}{#4}}}}
\newcommand\tfracunits[4]{\,\tfrac{\powtext{#1}{#2}}{\powtext{#3}{#4}}}
% \newcommand\tasklong[1]{{\bfseries \sffamily{Задача #1.}}}
% \newcommand\task[1]{{\bfseries \sffamily{Задача #1.}}}

\newcommand\ounits[2]{~$#1\units{#2}$}

\newcommand\funits[2]{{\,\frac{\text{#1}}{\text{#2}}}}
% \newcommand\cfracunits[2]{{\,\cfrac{\text{#1}}{\text{#2}}}}
% \newcommand\tunits[2]{\,\tfrac{\text{#1}}{\text{#2}}}
% \newcommand\ukg{\text{кг}}
% \newcommand\ugg[2]{\text{г}}
% \newcommand\umt[1]{\text{м$^{#1}$}}

\newcommand\vect[1]{\overrightarrow{#1}}

\newcommand\problem[1]{{\fbox{\bfseries{#1}}}}
\newcommand\tasknumber[1]{{\problem{#1}}}
\newcommand\starnumber[1]{{\problem{#1${}^\text{*}$}}}

\def\rhowater{$\rho = 1000\,\cfrac{\text{кг}}{\text{м}^3}.$}

% \newcommand\adddate[1]{
%     \begin{center}
%     \bfseries \sffamily{#1\,г.}
%     \end{center}
% }

\newcommand\adddate[1]{
    \begin{flushright}
    \textsc{#1\,г.}
    \end{flushright}
}

\newcommand\classdate[2]{
    \hasanswerswarning
    \begin{flushright}
    \textsc{#1 класс, #2\,г.}
    \end{flushright}
}

\newcommand\classdatevariant[3]{
    \hasanswerswarning
    \begin{flushright}
    \textsc{#1 класс, #2\,г. Вариант #3}
    \end{flushright}
}


\newcommand\rhoother[2]{
    $\rho_\text{#1} = #2\,\cfrac{\text{кг}}{\text{м}^3}$
}

\newcommand\cother[2]{
    $c_\text{#1} = #2\,\cfrac{\text{Дж}}{\text{кг}\cdot^\circ \!\text{C}}$
}

\def\celsius{\,^\circ\text{C}}
\def\le{\leqslant}
\def\ge{\geqslant}
\def\eps{{\varepsilon}}

\newcommand\cbr[1]{\left(#1\right)} %circled brackets
\newcommand\fbr[1]{\left\{#1\right\}} %figure brackets
\newcommand\sbr[1]{\left[#1\right]} %square brackets
\newcommand\modul[1]{\left|#1\right|}

\newcommand\sqr[1]{\cbr{#1}^2}
\newcommand\inv[1]{\cbr{#1}^{-1}}

\newcommand\libproblempath{\rootpath/problems}

\newcommand\taskname[2]{{\qquad\tt(Задача #2 из #1)}}

\newcommand\libproblem[2]{
    \input{\rootpath/problems/#1/#2.tex}
    \taskname{#1}{#2}
}

\newcommand\task[1]{
    #1
}

\newcommand{\answer}[1]{
    \\ {\bfseries \sffamily{Ответ:}} #1.
}

\newcommand\hasanswerswarning{{\bfseries \sffamily{Вариант с ответами}}}
\newcommand{\noanswers}[0]{
    \renewcommand\hasanswerswarning{}
    \renewcommand{\answer}[1]{}
    \renewcommand{\taskname}[2]{}
}

\def\narrow{
    % \usepackage{beton}
    \oddsidemargin=1cm
    \evensidemargin=1cm
    \textheight=29cm % высота текста
    \textwidth=13cm % ширина текста
    \topmargin=-1.5cm % отступ от верхнего края
    \parskip=2pt % интервал между абзацами

    \def\baselinestretch{1.25}
}

\def\invert{
    \usepackage{xcolor}
    \pagecolor[rgb]{0,0,0}
    \color[rgb]{1,1,1}
}

\def\qqquad{\qquad\qquad}
\def\qqqquad{\qqquad\qqquad}
\def\qqqqquad{\qqqquad\qqqquad}


\newcommand\twocolumns[3]{
    \columnsep=#1
    \begin{multicols}{2}
        #2
        \columnbreak
        #3
    \end{multicols}
}

\newcommand\threecolumns[4]{
    \columnsep=#1
    \begin{multicols}{3}
        #2
        \columnbreak
        #3
        \columnbreak
        #4
    \end{multicols}
}

\newcommand\twovariants[2]{
    #2 \vspace{#1} #2
}

\newcommand\threevariants[2]{
    #2 \vspace{#1} #2 \vspace{#1} #2
}

\newcommand\fourvariants[2]{
    #2 \vspace{#1} #2 \vspace{#1} #2 \vspace{#1} #2
}

\newcommand\fivevariants[2]{
    #2 \vspace{#1} #2 \vspace{#1} #2 \vspace{#1} #2 \vspace{#1} #2
}

\newcommand\sixvariants[2]{
    #2 \vspace{#1} #2 \vspace{#1} #2 \vspace{#1} #2 \vspace{#1} #2 \vspace{#1} #2
}

\newcommand\customdate{}
\newcommand\customclass{}

\newcommand\setdate[1]{
    \renewcommand\customdate{{#1}}
}

\newcommand\setclass[1]{
    \renewcommand\customclass{{#1}}
}

\def\insertclassdate{
    \begin{flushright}
    \textsc{\customclass~класс, \customdate\,г.}
    \end{flushright}
}

\newcommand\addvariant[1]{
    \begin{flushright}
    \textsc{\customclass~класс, \customdate\,г. Вариант #1}
    \end{flushright}
}

\newcommand\addpersonalvariant[1]{
    \begin{flushright}
    \textsc{#1~---~\customclass~класс, \customdate\,г.}
    \end{flushright}
}
