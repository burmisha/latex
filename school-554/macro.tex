\newcommand\units[1]{{\,\text{#1}}}
\newcommand\powtext[2]{{\text{#1}^{#2}}}
\newcommand\dunits[2]{{\,\powtext{#1}{#2}}}
\newcommand\fracunits[4]{{\,\frac{\powtext{#1}{#2}}{\powtext{#3}{#4}}}}
\newcommand\cfracunits[4]{{\,\cfrac{\powtext{#1}{#2}}{\powtext{#3}{#4}}}}
\newcommand\dfracunits[4]{{\,\dfrac{\powtext{#1}{#2}}{\powtext{#3}{#4}}}}
\newcommand\tfracunits[4]{\,\tfrac{\powtext{#1}{#2}}{\powtext{#3}{#4}}}

\newcommand\ounits[2]{~$#1\units{#2}$}

\newcommand\funits[2]{{\,\frac{\text{#1}}{\text{#2}}}}

\newcommand\vect[1]{\overrightarrow{#1}}

\newcommand\problem[1]{\smallskip\par\noindent\hskip0mm\llap{\fbox{\bfseries{#1}}\kern.7em}}  % ium
\newcommand\tasknumber[1]{{\problem{#1}}}
\newcommand\starnumber[1]{{\problem{#1${}^\text{*}$}}}

\def\rhowater{$\rho = 1000\,\cfrac{\text{кг}}{\text{м}^3}.$}

\newcommand\adddate[1]{
    \begin{flushright}
    \textsc{#1\,г.}
    \end{flushright}
}

\newcommand\classdate[2]{
    \hasanswerswarning
    \begin{flushright}
    \textsc{#1 класс, #2\,г.}
    \end{flushright}
}

\newcommand\classdatevariant[3]{
    \hasanswerswarning
    \begin{flushright}
    \textsc{#1 класс, #2\,г. Вариант #3}
    \end{flushright}
}

\newcommand\cother[2]{
    $c_\text{#1} = #2\,\cfrac{\text{Дж}}{\text{кг}\cdot^\circ \!\text{C}}$
}

\def\degrees{^\circ}
\def\celsius{\,^\circ\text{C}}
\def\le{\leqslant}
\def\ge{\geqslant}
\def\eps{{\varepsilon}}
\def\ele{\mathscr{ E }}
\def\eli{\mathscr{ I }}

% brackets
\newcommand\cbr[1]{\left(#1\right)}     % circled
\newcommand\fbr[1]{\left\{#1\right\}}   % figure
\newcommand\sbr[1]{\left[#1\right]}     % square
\newcommand\modul[1]{\left|#1\right|}   % abs
\newcommand\abs[1]{\left|#1\right|}     % abs

\newcommand\sqr[1]{\cbr{#1}^2}
\newcommand\inv[1]{\cbr{#1}^{-1}}

\newcommand\libproblempath{\rootpath/problems}

\newcommand\taskname[2]{{\qquad\tt(Задача #2 из #1)}}

\newcommand\libproblem[2]{
    \input{\rootpath/problems/#1/#2.tex}
    \taskname{#1}{#2}
}

\newcommand\task[1]{%
    #1%
}

\newcommand{\answer}[1]{
\par
{\itshape{Указание:}} #1
}

\newcommand\solutionspace[1]{}

\def\variantsplitter{}

\newcommand\hasanswerswarning{{\bfseries \sffamily{Вариант с ответами}}}
\def\noanswers{
    \renewcommand\hasanswerswarning{}
    \renewcommand{\answer}[1]{}
    \renewcommand{\taskname}[2]{}
    % \renewcommand{\solutionspace}[1]{\vspace{##1}}  % old version
    \renewcommand{\solutionspace}[1]{

        \begin{tikzpicture}
            \tikzset{normal lines/.style={gray, very thin}}
            \tikzset{margin lines/.style={gray, thick}}
            \tikzset{mm lines/.style={gray, ultra thin}}
            \tikzset{strong lines/.style={black, very thin}}
            \tikzset{master lines/.style={black, very thick}}
            \tikzset{dashed master lines/.style={loosely dashed, black, very thick}}

            \begin{tikzpicture}
                \draw[style=mm lines,step=0.4cm] (1,-0.05) grid +(220mm,##1);
            \end{tikzpicture}%
        \end{tikzpicture}%
    }

    % \renewcommand{\solutionspace}[1]{ \vspace*{##1} }
%     \renewcommand{\solutionspace}[1]{

% \addvspace{##1}\par

% }
    % \renewcommand{\solutionspace}[1]{ \vskip ##1 }
    \def\variantsplitter{\newpage}
}

\def\variantsplitter{\newpage}

\def\narrow{
    % \usepackage{beton}
    \oddsidemargin=1cm
    \evensidemargin=1cm
    \textheight=28cm % высота текста
    \textwidth=13cm % ширина текста
    \topmargin=-2.5cm % отступ от верхнего края
    \parskip=2pt % интервал между абзацами

    \def\baselinestretch{1.25}
}

\def\invert{
    \pagecolor[rgb]{0,0,0}
    \color[rgb]{1,1,1}
}

\def\qqquad{\qquad\qquad}
\def\qqqquad{\qqquad\qqquad}
\def\qqqqquad{\qqqquad\qqqquad}


\newcommand\twocolumns[3]{
    \columnsep=#1
    \begin{multicols}{2}
        #2
        \columnbreak
        #3
    \end{multicols}
}

\newcommand\threecolumns[4]{
    \columnsep=#1
    \begin{multicols}{3}
        #2
        \columnbreak
        #3
        \columnbreak
        #4
    \end{multicols}
}

\newcommand\twovariants[2]{
    #2 \vspace{#1} #2
}

\newcommand\threevariants[2]{
    #2 \vspace{#1} #2 \vspace{#1} #2
}

\newcommand\fourvariants[2]{
    #2 \vspace{#1} #2 \vspace{#1} #2 \vspace{#1} #2
}

\newcommand\fivevariants[2]{
    #2 \vspace{#1} #2 \vspace{#1} #2 \vspace{#1} #2 \vspace{#1} #2
}

\newcommand\sixvariants[2]{
    #2 \vspace{#1} #2 \vspace{#1} #2 \vspace{#1} #2 \vspace{#1} #2 \vspace{#1} #2
}

\newcommand\customdate{}
\newcommand\customclass{}

\newcommand\setdate[1]{
    \renewcommand\customdate{{#1\,г.}}
}

\newcommand\setclass[1]{
    \renewcommand\customclass{{#1 класс}}
}

\def\insertclassdate{
    \begin{flushright}
    \textsc{\customclass, \customdate}
    \end{flushright}
}

\newcommand\addvariant[1]{
    \begin{flushright}
    \textsc{\customclass, \customdate, вариант #1}
    \end{flushright}
}

\newcommand\addpersonalvariant[1]{
    \begin{flushright}
    \textsc{#1~---~\customclass, \customdate}
    \end{flushright}
}

\tikzset{
    midar/.style 2 args={ 
            very thick,
            decoration={ name=markings,
                mark=at position .55 with { \arrow{ latex } },
                mark=at position 0 with { \fill circle (2pt); },
                mark=at position 1 with { \fill circle (2pt); }
            },
            postaction=decorate,
    },
}