\setdate{24~мая~2022}
\setclass{11«БА»}

\addpersonalvariant{Михаил Бурмистров}

\tasknumber{1}%
\task{%
    \begin{itemize}
        \item Как меня зовут?
        \item Как называется предмет?
        \item Какого цвета учебник?
        \item Чем заканчивается анекдот?
    \end{itemize}
}

\tasknumber{2}%
\task{%
    \begin{itemize}
        \item Как проведёте вечер перед экзаменом по математике?
        \item Куда съездите и как отдохнёте после?
    \end{itemize}
}
\solutionspace{100pt}

\tasknumber{3}%
\task{%
    О существовании какого населённого пункта вы узнали этой весной? Чем он вам запомнился?
}
\solutionspace{60pt}

\tasknumber{4}%
\task{%
    \begin{itemize}
        \item Какой ЕГЭ (кроме математики и русского) вы сдаете?
        \item Сколько времени у вас будет на финальную подготовку и повторение?
        \item Как вы распределите время на экзамене?
        \item Как искать ошибки и проверять себя?
    \end{itemize}
}
\solutionspace{120pt}

\tasknumber{5}%
\task{%
    Выберите и запишите из всего курса физики 2 формулы:
    \begin{itemize}
        \item наиболее удивительную,
        \item и наименее важную.
    \end{itemize}
}
\solutionspace{120pt}

\tasknumber{6}%
\task{%
    \begin{itemize}
        \item Какие физические открытия или достижения были сделаны на веку вас и ваших родителей?
        \item Как это повлияло на нашу жизнь?
    \end{itemize}
}


\variantsplitter


\addpersonalvariant{Михаил Бурмистров}

\tasknumber{7}%
\task{%
    Назовите 6 знаменитых деятелей и деятельниц физики и области науки, в которых они работали.
    Обеспечьте представительство различных групп.
}
\solutionspace{100pt}

\tasknumber{8}%
\task{%
    Назовите 3 причины для вас лично вернуться в школу
}
\solutionspace{60pt}

\tasknumber{9}%
\task{%
    Расскажите на какой работе вы уже поработали?
    Сколько часов в неделю и сколько недель или месяцев опыта? Норм по деньгам?
}
\solutionspace{120pt}

\tasknumber{10}%
\task{%
    Где бы вы хотели поработать, но ТЕПЕРЬ не получится?
    Какие новые возможности открылись за последние месяцы?
}
\solutionspace{60pt}

\tasknumber{11}%
\task{%
    Что нового вы попробуете в июле-августе?
    Куда надо съездить и где побывать?
    Укажите не менее 3 активностей и 7 мест.
}
\solutionspace{80pt}

\tasknumber{12}%
\task{%
    По какому принципу зарубежные университеты сохраняют партнёрство с российскими?
}
\solutionspace{80pt}

\tasknumber{13}%
\task{%
    Посоветуйте свой (или не свой) Тг/TT/YT/Ig.
    Точным названием или ссылкой, не более двух ответов.
}

\variantsplitter

\addpersonalvariant{Ирина Ан}

\tasknumber{1}%
\task{%
    \begin{itemize}
        \item Как меня зовут?
        \item Как называется предмет?
        \item Какого цвета учебник?
        \item Чем заканчивается анекдот?
    \end{itemize}
}

\tasknumber{2}%
\task{%
    \begin{itemize}
        \item Как проведёте вечер перед экзаменом по математике?
        \item Куда съездите и как отдохнёте после?
    \end{itemize}
}
\solutionspace{100pt}

\tasknumber{3}%
\task{%
    О существовании какого населённого пункта вы узнали этой весной? Чем он вам запомнился?
}
\solutionspace{60pt}

\tasknumber{4}%
\task{%
    \begin{itemize}
        \item Какой ЕГЭ (кроме математики и русского) вы сдаете?
        \item Сколько времени у вас будет на финальную подготовку и повторение?
        \item Как вы распределите время на экзамене?
        \item Как искать ошибки и проверять себя?
    \end{itemize}
}
\solutionspace{120pt}

\tasknumber{5}%
\task{%
    Выберите и запишите из всего курса физики 2 формулы:
    \begin{itemize}
        \item наиболее удивительную,
        \item и наименее интересную.
    \end{itemize}
}
\solutionspace{120pt}

\tasknumber{6}%
\task{%
    \begin{itemize}
        \item Какие физические открытия или достижения были сделаны на веку вас и ваших родителей?
        \item Как это повлияло на нашу жизнь?
    \end{itemize}
}


\variantsplitter


\addpersonalvariant{Ирина Ан}

\tasknumber{7}%
\task{%
    Назовите 6 знаменитых деятелей и деятельниц физики и области науки, в которых они работали.
    Обеспечьте представительство различных групп.
}
\solutionspace{100pt}

\tasknumber{8}%
\task{%
    Назовите 3 причины для вас лично вернуться в школу
}
\solutionspace{60pt}

\tasknumber{9}%
\task{%
    Расскажите на какой работе вы уже поработали?
    Сколько часов в неделю и сколько недель или месяцев опыта? Норм по деньгам?
}
\solutionspace{120pt}

\tasknumber{10}%
\task{%
    Где бы вы хотели поработать, но ТЕПЕРЬ не получится?
    Какие новые возможности открылись за последние месяцы?
}
\solutionspace{60pt}

\tasknumber{11}%
\task{%
    Что нового вы попробуете в июле-августе?
    Куда надо съездить и где побывать?
    Укажите не менее 3 активностей и 7 мест.
}
\solutionspace{80pt}

\tasknumber{12}%
\task{%
    По какому принципу зарубежные университеты сохраняют партнёрство с российскими?
}
\solutionspace{80pt}

\tasknumber{13}%
\task{%
    Посоветуйте свой (или не свой) Тг/TT/YT/Ig.
    Точным названием или ссылкой, не более двух ответов.
}

\variantsplitter

\addpersonalvariant{Софья Андрианова}

\tasknumber{1}%
\task{%
    \begin{itemize}
        \item Как меня зовут?
        \item Как называется предмет?
        \item Какого цвета учебник?
        \item Чем заканчивается анекдот?
    \end{itemize}
}

\tasknumber{2}%
\task{%
    \begin{itemize}
        \item Как проведёте утро перед экзаменом по математике?
        \item Куда съездите и как отдохнёте после?
    \end{itemize}
}
\solutionspace{100pt}

\tasknumber{3}%
\task{%
    О существовании какого населённого пункта вы узнали этой весной? Чем он вам запомнился?
}
\solutionspace{60pt}

\tasknumber{4}%
\task{%
    \begin{itemize}
        \item Какой ЕГЭ (кроме математики и русского) вы сдаете?
        \item Сколько времени у вас будет на финальную подготовку и повторение?
        \item Как вы распределите время на экзамене?
        \item Как искать ошибки и проверять себя?
    \end{itemize}
}
\solutionspace{120pt}

\tasknumber{5}%
\task{%
    Выберите и запишите из всего курса физики 2 формулы:
    \begin{itemize}
        \item наиболее полезную,
        \item и наименее важную.
    \end{itemize}
}
\solutionspace{120pt}

\tasknumber{6}%
\task{%
    \begin{itemize}
        \item Какие физические открытия или достижения были сделаны на веку вас и ваших родителей?
        \item Как это повлияло на нашу жизнь?
    \end{itemize}
}


\variantsplitter


\addpersonalvariant{Софья Андрианова}

\tasknumber{7}%
\task{%
    Назовите 6 знаменитых деятелей и деятельниц физики и области науки, в которых они работали.
    Обеспечьте представительство различных групп.
}
\solutionspace{100pt}

\tasknumber{8}%
\task{%
    Назовите 3 причины для вас лично вернуться в школу
}
\solutionspace{60pt}

\tasknumber{9}%
\task{%
    Расскажите на какой работе вы уже поработали?
    Сколько часов в неделю и сколько недель или месяцев опыта? Норм по деньгам?
}
\solutionspace{120pt}

\tasknumber{10}%
\task{%
    Где бы вы хотели поработать, но ТЕПЕРЬ не получится?
    Какие новые возможности открылись за последние месяцы?
}
\solutionspace{60pt}

\tasknumber{11}%
\task{%
    Что нового вы попробуете в июле-августе?
    Куда надо съездить и где побывать?
    Укажите не менее 3 активностей и 7 мест.
}
\solutionspace{80pt}

\tasknumber{12}%
\task{%
    По какому принципу зарубежные университеты сохраняют партнёрство с российскими?
}
\solutionspace{80pt}

\tasknumber{13}%
\task{%
    Посоветуйте свой (или не свой) Тг/TT/YT/Ig.
    Точным названием или ссылкой, не более двух ответов.
}

\variantsplitter

\addpersonalvariant{Владимир Артемчук}

\tasknumber{1}%
\task{%
    \begin{itemize}
        \item Как меня зовут?
        \item Как называется предмет?
        \item Какого цвета учебник?
        \item Чем заканчивается анекдот?
    \end{itemize}
}

\tasknumber{2}%
\task{%
    \begin{itemize}
        \item Как проведёте утро перед экзаменом по русскому языку?
        \item Куда съездите и как отдохнёте после?
    \end{itemize}
}
\solutionspace{100pt}

\tasknumber{3}%
\task{%
    О существовании какого населённого пункта вы узнали этой весной? Чем он вам запомнился?
}
\solutionspace{60pt}

\tasknumber{4}%
\task{%
    \begin{itemize}
        \item Какой ЕГЭ (кроме математики и русского) вы сдаете?
        \item Сколько времени у вас будет на финальную подготовку и повторение?
        \item Как вы распределите время на экзамене?
        \item Как искать ошибки и проверять себя?
    \end{itemize}
}
\solutionspace{120pt}

\tasknumber{5}%
\task{%
    Выберите и запишите из всего курса физики 2 формулы:
    \begin{itemize}
        \item наиболее полезную,
        \item и наименее интересную.
    \end{itemize}
}
\solutionspace{120pt}

\tasknumber{6}%
\task{%
    \begin{itemize}
        \item Какие физические открытия или достижения были сделаны на веку вас и ваших родителей?
        \item Как это повлияло на нашу жизнь?
    \end{itemize}
}


\variantsplitter


\addpersonalvariant{Владимир Артемчук}

\tasknumber{7}%
\task{%
    Назовите 6 знаменитых деятелей и деятельниц физики и области науки, в которых они работали.
    Обеспечьте представительство различных групп.
}
\solutionspace{100pt}

\tasknumber{8}%
\task{%
    Назовите 3 причины для вас лично вернуться в школу
}
\solutionspace{60pt}

\tasknumber{9}%
\task{%
    Расскажите на какой работе вы уже поработали?
    Сколько часов в неделю и сколько недель или месяцев опыта? Норм по деньгам?
}
\solutionspace{120pt}

\tasknumber{10}%
\task{%
    Где бы вы хотели поработать, но ТЕПЕРЬ не получится?
    Какие новые возможности открылись за последние месяцы?
}
\solutionspace{60pt}

\tasknumber{11}%
\task{%
    Что нового вы попробуете в июле-августе?
    Куда надо съездить и где побывать?
    Укажите не менее 3 активностей и 7 мест.
}
\solutionspace{80pt}

\tasknumber{12}%
\task{%
    По какому принципу зарубежные университеты сохраняют партнёрство с российскими?
}
\solutionspace{80pt}

\tasknumber{13}%
\task{%
    Посоветуйте свой (или не свой) Тг/TT/YT/Ig.
    Точным названием или ссылкой, не более двух ответов.
}

\variantsplitter

\addpersonalvariant{Софья Белянкина}

\tasknumber{1}%
\task{%
    \begin{itemize}
        \item Как меня зовут?
        \item Как называется предмет?
        \item Какого цвета учебник?
        \item Чем заканчивается анекдот?
    \end{itemize}
}

\tasknumber{2}%
\task{%
    \begin{itemize}
        \item Как проведёте утро перед экзаменом по математике?
        \item Куда съездите и как отдохнёте после?
    \end{itemize}
}
\solutionspace{100pt}

\tasknumber{3}%
\task{%
    О существовании какого населённого пункта вы узнали этой весной? Чем он вам запомнился?
}
\solutionspace{60pt}

\tasknumber{4}%
\task{%
    \begin{itemize}
        \item Какой ЕГЭ (кроме математики и русского) вы сдаете?
        \item Сколько времени у вас будет на финальную подготовку и повторение?
        \item Как вы распределите время на экзамене?
        \item Как искать ошибки и проверять себя?
    \end{itemize}
}
\solutionspace{120pt}

\tasknumber{5}%
\task{%
    Выберите и запишите из всего курса физики 2 формулы:
    \begin{itemize}
        \item наиболее запомнившуюсю,
        \item и наименее интересную.
    \end{itemize}
}
\solutionspace{120pt}

\tasknumber{6}%
\task{%
    \begin{itemize}
        \item Какие физические открытия или достижения были сделаны на веку вас и ваших родителей?
        \item Как это повлияло на нашу жизнь?
    \end{itemize}
}


\variantsplitter


\addpersonalvariant{Софья Белянкина}

\tasknumber{7}%
\task{%
    Назовите 6 знаменитых деятелей и деятельниц физики и области науки, в которых они работали.
    Обеспечьте представительство различных групп.
}
\solutionspace{100pt}

\tasknumber{8}%
\task{%
    Назовите 3 причины для вас лично вернуться в школу
}
\solutionspace{60pt}

\tasknumber{9}%
\task{%
    Расскажите на какой работе вы уже поработали?
    Сколько часов в неделю и сколько недель или месяцев опыта? Норм по деньгам?
}
\solutionspace{120pt}

\tasknumber{10}%
\task{%
    Где бы вы хотели поработать, но ТЕПЕРЬ не получится?
    Какие новые возможности открылись за последние месяцы?
}
\solutionspace{60pt}

\tasknumber{11}%
\task{%
    Что нового вы попробуете в июле-августе?
    Куда надо съездить и где побывать?
    Укажите не менее 3 активностей и 7 мест.
}
\solutionspace{80pt}

\tasknumber{12}%
\task{%
    По какому принципу зарубежные университеты сохраняют партнёрство с российскими?
}
\solutionspace{80pt}

\tasknumber{13}%
\task{%
    Посоветуйте свой (или не свой) Тг/TT/YT/Ig.
    Точным названием или ссылкой, не более двух ответов.
}

\variantsplitter

\addpersonalvariant{Варвара Егиазарян}

\tasknumber{1}%
\task{%
    \begin{itemize}
        \item Как меня зовут?
        \item Как называется предмет?
        \item Какого цвета учебник?
        \item Чем заканчивается анекдот?
    \end{itemize}
}

\tasknumber{2}%
\task{%
    \begin{itemize}
        \item Как проведёте утро перед экзаменом по математике?
        \item Куда съездите и как отдохнёте после?
    \end{itemize}
}
\solutionspace{100pt}

\tasknumber{3}%
\task{%
    О существовании какого населённого пункта вы узнали этой весной? Чем он вам запомнился?
}
\solutionspace{60pt}

\tasknumber{4}%
\task{%
    \begin{itemize}
        \item Какой ЕГЭ (кроме математики и русского) вы сдаете?
        \item Сколько времени у вас будет на финальную подготовку и повторение?
        \item Как вы распределите время на экзамене?
        \item Как искать ошибки и проверять себя?
    \end{itemize}
}
\solutionspace{120pt}

\tasknumber{5}%
\task{%
    Выберите и запишите из всего курса физики 2 формулы:
    \begin{itemize}
        \item наиболее смешную,
        \item и наименее интересную.
    \end{itemize}
}
\solutionspace{120pt}

\tasknumber{6}%
\task{%
    \begin{itemize}
        \item Какие физические открытия или достижения были сделаны на веку вас и ваших родителей?
        \item Как это повлияло на нашу жизнь?
    \end{itemize}
}


\variantsplitter


\addpersonalvariant{Варвара Егиазарян}

\tasknumber{7}%
\task{%
    Назовите 6 знаменитых деятелей и деятельниц физики и области науки, в которых они работали.
    Обеспечьте представительство различных групп.
}
\solutionspace{100pt}

\tasknumber{8}%
\task{%
    Назовите 3 причины для вас лично вернуться в школу
}
\solutionspace{60pt}

\tasknumber{9}%
\task{%
    Расскажите на какой работе вы уже поработали?
    Сколько часов в неделю и сколько недель или месяцев опыта? Норм по деньгам?
}
\solutionspace{120pt}

\tasknumber{10}%
\task{%
    Где бы вы хотели поработать, но ТЕПЕРЬ не получится?
    Какие новые возможности открылись за последние месяцы?
}
\solutionspace{60pt}

\tasknumber{11}%
\task{%
    Что нового вы попробуете в июле-августе?
    Куда надо съездить и где побывать?
    Укажите не менее 3 активностей и 7 мест.
}
\solutionspace{80pt}

\tasknumber{12}%
\task{%
    По какому принципу зарубежные университеты сохраняют партнёрство с российскими?
}
\solutionspace{80pt}

\tasknumber{13}%
\task{%
    Посоветуйте свой (или не свой) Тг/TT/YT/Ig.
    Точным названием или ссылкой, не более двух ответов.
}

\variantsplitter

\addpersonalvariant{Владислав Емелин}

\tasknumber{1}%
\task{%
    \begin{itemize}
        \item Как меня зовут?
        \item Как называется предмет?
        \item Какого цвета учебник?
        \item Чем заканчивается анекдот?
    \end{itemize}
}

\tasknumber{2}%
\task{%
    \begin{itemize}
        \item Как проведёте утро перед экзаменом по русскому языку?
        \item Куда съездите и как отдохнёте после?
    \end{itemize}
}
\solutionspace{100pt}

\tasknumber{3}%
\task{%
    О существовании какого населённого пункта вы узнали этой весной? Чем он вам запомнился?
}
\solutionspace{60pt}

\tasknumber{4}%
\task{%
    \begin{itemize}
        \item Какой ЕГЭ (кроме математики и русского) вы сдаете?
        \item Сколько времени у вас будет на финальную подготовку и повторение?
        \item Как вы распределите время на экзамене?
        \item Как искать ошибки и проверять себя?
    \end{itemize}
}
\solutionspace{120pt}

\tasknumber{5}%
\task{%
    Выберите и запишите из всего курса физики 2 формулы:
    \begin{itemize}
        \item наиболее смешную,
        \item и наименее важную.
    \end{itemize}
}
\solutionspace{120pt}

\tasknumber{6}%
\task{%
    \begin{itemize}
        \item Какие физические открытия или достижения были сделаны на веку вас и ваших родителей?
        \item Как это повлияло на нашу жизнь?
    \end{itemize}
}


\variantsplitter


\addpersonalvariant{Владислав Емелин}

\tasknumber{7}%
\task{%
    Назовите 6 знаменитых деятелей и деятельниц физики и области науки, в которых они работали.
    Обеспечьте представительство различных групп.
}
\solutionspace{100pt}

\tasknumber{8}%
\task{%
    Назовите 3 причины для вас лично вернуться в школу
}
\solutionspace{60pt}

\tasknumber{9}%
\task{%
    Расскажите на какой работе вы уже поработали?
    Сколько часов в неделю и сколько недель или месяцев опыта? Норм по деньгам?
}
\solutionspace{120pt}

\tasknumber{10}%
\task{%
    Где бы вы хотели поработать, но ТЕПЕРЬ не получится?
    Какие новые возможности открылись за последние месяцы?
}
\solutionspace{60pt}

\tasknumber{11}%
\task{%
    Что нового вы попробуете в июле-августе?
    Куда надо съездить и где побывать?
    Укажите не менее 3 активностей и 7 мест.
}
\solutionspace{80pt}

\tasknumber{12}%
\task{%
    По какому принципу зарубежные университеты сохраняют партнёрство с российскими?
}
\solutionspace{80pt}

\tasknumber{13}%
\task{%
    Посоветуйте свой (или не свой) Тг/TT/YT/Ig.
    Точным названием или ссылкой, не более двух ответов.
}

\variantsplitter

\addpersonalvariant{Артём Жичин}

\tasknumber{1}%
\task{%
    \begin{itemize}
        \item Как меня зовут?
        \item Как называется предмет?
        \item Какого цвета учебник?
        \item Чем заканчивается анекдот?
    \end{itemize}
}

\tasknumber{2}%
\task{%
    \begin{itemize}
        \item Как проведёте утро перед экзаменом по математике?
        \item Куда съездите и как отдохнёте после?
    \end{itemize}
}
\solutionspace{100pt}

\tasknumber{3}%
\task{%
    О существовании какого населённого пункта вы узнали этой весной? Чем он вам запомнился?
}
\solutionspace{60pt}

\tasknumber{4}%
\task{%
    \begin{itemize}
        \item Какой ЕГЭ (кроме математики и русского) вы сдаете?
        \item Сколько времени у вас будет на финальную подготовку и повторение?
        \item Как вы распределите время на экзамене?
        \item Как искать ошибки и проверять себя?
    \end{itemize}
}
\solutionspace{120pt}

\tasknumber{5}%
\task{%
    Выберите и запишите из всего курса физики 2 формулы:
    \begin{itemize}
        \item наиболее полезную,
        \item и наименее важную.
    \end{itemize}
}
\solutionspace{120pt}

\tasknumber{6}%
\task{%
    \begin{itemize}
        \item Какие физические открытия или достижения были сделаны на веку вас и ваших родителей?
        \item Как это повлияло на нашу жизнь?
    \end{itemize}
}


\variantsplitter


\addpersonalvariant{Артём Жичин}

\tasknumber{7}%
\task{%
    Назовите 6 знаменитых деятелей и деятельниц физики и области науки, в которых они работали.
    Обеспечьте представительство различных групп.
}
\solutionspace{100pt}

\tasknumber{8}%
\task{%
    Назовите 3 причины для вас лично вернуться в школу
}
\solutionspace{60pt}

\tasknumber{9}%
\task{%
    Расскажите на какой работе вы уже поработали?
    Сколько часов в неделю и сколько недель или месяцев опыта? Норм по деньгам?
}
\solutionspace{120pt}

\tasknumber{10}%
\task{%
    Где бы вы хотели поработать, но ТЕПЕРЬ не получится?
    Какие новые возможности открылись за последние месяцы?
}
\solutionspace{60pt}

\tasknumber{11}%
\task{%
    Что нового вы попробуете в июле-августе?
    Куда надо съездить и где побывать?
    Укажите не менее 3 активностей и 7 мест.
}
\solutionspace{80pt}

\tasknumber{12}%
\task{%
    По какому принципу зарубежные университеты сохраняют партнёрство с российскими?
}
\solutionspace{80pt}

\tasknumber{13}%
\task{%
    Посоветуйте свой (или не свой) Тг/TT/YT/Ig.
    Точным названием или ссылкой, не более двух ответов.
}

\variantsplitter

\addpersonalvariant{Дарья Кошман}

\tasknumber{1}%
\task{%
    \begin{itemize}
        \item Как меня зовут?
        \item Как называется предмет?
        \item Какого цвета учебник?
        \item Чем заканчивается анекдот?
    \end{itemize}
}

\tasknumber{2}%
\task{%
    \begin{itemize}
        \item Как проведёте вечер перед экзаменом по русскому языку?
        \item Куда съездите и как отдохнёте после?
    \end{itemize}
}
\solutionspace{100pt}

\tasknumber{3}%
\task{%
    О существовании какого населённого пункта вы узнали этой весной? Чем он вам запомнился?
}
\solutionspace{60pt}

\tasknumber{4}%
\task{%
    \begin{itemize}
        \item Какой ЕГЭ (кроме математики и русского) вы сдаете?
        \item Сколько времени у вас будет на финальную подготовку и повторение?
        \item Как вы распределите время на экзамене?
        \item Как искать ошибки и проверять себя?
    \end{itemize}
}
\solutionspace{120pt}

\tasknumber{5}%
\task{%
    Выберите и запишите из всего курса физики 2 формулы:
    \begin{itemize}
        \item наиболее смешную,
        \item и наименее важную.
    \end{itemize}
}
\solutionspace{120pt}

\tasknumber{6}%
\task{%
    \begin{itemize}
        \item Какие физические открытия или достижения были сделаны на веку вас и ваших родителей?
        \item Как это повлияло на нашу жизнь?
    \end{itemize}
}


\variantsplitter


\addpersonalvariant{Дарья Кошман}

\tasknumber{7}%
\task{%
    Назовите 6 знаменитых деятелей и деятельниц физики и области науки, в которых они работали.
    Обеспечьте представительство различных групп.
}
\solutionspace{100pt}

\tasknumber{8}%
\task{%
    Назовите 3 причины для вас лично вернуться в школу
}
\solutionspace{60pt}

\tasknumber{9}%
\task{%
    Расскажите на какой работе вы уже поработали?
    Сколько часов в неделю и сколько недель или месяцев опыта? Норм по деньгам?
}
\solutionspace{120pt}

\tasknumber{10}%
\task{%
    Где бы вы хотели поработать, но ТЕПЕРЬ не получится?
    Какие новые возможности открылись за последние месяцы?
}
\solutionspace{60pt}

\tasknumber{11}%
\task{%
    Что нового вы попробуете в июле-августе?
    Куда надо съездить и где побывать?
    Укажите не менее 3 активностей и 7 мест.
}
\solutionspace{80pt}

\tasknumber{12}%
\task{%
    По какому принципу зарубежные университеты сохраняют партнёрство с российскими?
}
\solutionspace{80pt}

\tasknumber{13}%
\task{%
    Посоветуйте свой (или не свой) Тг/TT/YT/Ig.
    Точным названием или ссылкой, не более двух ответов.
}

\variantsplitter

\addpersonalvariant{Анна Кузьмичёва}

\tasknumber{1}%
\task{%
    \begin{itemize}
        \item Как меня зовут?
        \item Как называется предмет?
        \item Какого цвета учебник?
        \item Чем заканчивается анекдот?
    \end{itemize}
}

\tasknumber{2}%
\task{%
    \begin{itemize}
        \item Как проведёте вечер перед экзаменом по математике?
        \item Куда съездите и как отдохнёте после?
    \end{itemize}
}
\solutionspace{100pt}

\tasknumber{3}%
\task{%
    О существовании какого населённого пункта вы узнали этой весной? Чем он вам запомнился?
}
\solutionspace{60pt}

\tasknumber{4}%
\task{%
    \begin{itemize}
        \item Какой ЕГЭ (кроме математики и русского) вы сдаете?
        \item Сколько времени у вас будет на финальную подготовку и повторение?
        \item Как вы распределите время на экзамене?
        \item Как искать ошибки и проверять себя?
    \end{itemize}
}
\solutionspace{120pt}

\tasknumber{5}%
\task{%
    Выберите и запишите из всего курса физики 2 формулы:
    \begin{itemize}
        \item наиболее запомнившуюсю,
        \item и наименее важную.
    \end{itemize}
}
\solutionspace{120pt}

\tasknumber{6}%
\task{%
    \begin{itemize}
        \item Какие физические открытия или достижения были сделаны на веку вас и ваших родителей?
        \item Как это повлияло на нашу жизнь?
    \end{itemize}
}


\variantsplitter


\addpersonalvariant{Анна Кузьмичёва}

\tasknumber{7}%
\task{%
    Назовите 6 знаменитых деятелей и деятельниц физики и области науки, в которых они работали.
    Обеспечьте представительство различных групп.
}
\solutionspace{100pt}

\tasknumber{8}%
\task{%
    Назовите 3 причины для вас лично вернуться в школу
}
\solutionspace{60pt}

\tasknumber{9}%
\task{%
    Расскажите на какой работе вы уже поработали?
    Сколько часов в неделю и сколько недель или месяцев опыта? Норм по деньгам?
}
\solutionspace{120pt}

\tasknumber{10}%
\task{%
    Где бы вы хотели поработать, но ТЕПЕРЬ не получится?
    Какие новые возможности открылись за последние месяцы?
}
\solutionspace{60pt}

\tasknumber{11}%
\task{%
    Что нового вы попробуете в июле-августе?
    Куда надо съездить и где побывать?
    Укажите не менее 3 активностей и 7 мест.
}
\solutionspace{80pt}

\tasknumber{12}%
\task{%
    По какому принципу зарубежные университеты сохраняют партнёрство с российскими?
}
\solutionspace{80pt}

\tasknumber{13}%
\task{%
    Посоветуйте свой (или не свой) Тг/TT/YT/Ig.
    Точным названием или ссылкой, не более двух ответов.
}

\variantsplitter

\addpersonalvariant{Алёна Куприянова}

\tasknumber{1}%
\task{%
    \begin{itemize}
        \item Как меня зовут?
        \item Как называется предмет?
        \item Какого цвета учебник?
        \item Чем заканчивается анекдот?
    \end{itemize}
}

\tasknumber{2}%
\task{%
    \begin{itemize}
        \item Как проведёте утро перед экзаменом по математике?
        \item Куда съездите и как отдохнёте после?
    \end{itemize}
}
\solutionspace{100pt}

\tasknumber{3}%
\task{%
    О существовании какого населённого пункта вы узнали этой весной? Чем он вам запомнился?
}
\solutionspace{60pt}

\tasknumber{4}%
\task{%
    \begin{itemize}
        \item Какой ЕГЭ (кроме математики и русского) вы сдаете?
        \item Сколько времени у вас будет на финальную подготовку и повторение?
        \item Как вы распределите время на экзамене?
        \item Как искать ошибки и проверять себя?
    \end{itemize}
}
\solutionspace{120pt}

\tasknumber{5}%
\task{%
    Выберите и запишите из всего курса физики 2 формулы:
    \begin{itemize}
        \item наиболее полезную,
        \item и наименее интересную.
    \end{itemize}
}
\solutionspace{120pt}

\tasknumber{6}%
\task{%
    \begin{itemize}
        \item Какие физические открытия или достижения были сделаны на веку вас и ваших родителей?
        \item Как это повлияло на нашу жизнь?
    \end{itemize}
}


\variantsplitter


\addpersonalvariant{Алёна Куприянова}

\tasknumber{7}%
\task{%
    Назовите 6 знаменитых деятелей и деятельниц физики и области науки, в которых они работали.
    Обеспечьте представительство различных групп.
}
\solutionspace{100pt}

\tasknumber{8}%
\task{%
    Назовите 3 причины для вас лично вернуться в школу
}
\solutionspace{60pt}

\tasknumber{9}%
\task{%
    Расскажите на какой работе вы уже поработали?
    Сколько часов в неделю и сколько недель или месяцев опыта? Норм по деньгам?
}
\solutionspace{120pt}

\tasknumber{10}%
\task{%
    Где бы вы хотели поработать, но ТЕПЕРЬ не получится?
    Какие новые возможности открылись за последние месяцы?
}
\solutionspace{60pt}

\tasknumber{11}%
\task{%
    Что нового вы попробуете в июле-августе?
    Куда надо съездить и где побывать?
    Укажите не менее 3 активностей и 7 мест.
}
\solutionspace{80pt}

\tasknumber{12}%
\task{%
    По какому принципу зарубежные университеты сохраняют партнёрство с российскими?
}
\solutionspace{80pt}

\tasknumber{13}%
\task{%
    Посоветуйте свой (или не свой) Тг/TT/YT/Ig.
    Точным названием или ссылкой, не более двух ответов.
}

\variantsplitter

\addpersonalvariant{Ярослав Лавровский}

\tasknumber{1}%
\task{%
    \begin{itemize}
        \item Как меня зовут?
        \item Как называется предмет?
        \item Какого цвета учебник?
        \item Чем заканчивается анекдот?
    \end{itemize}
}

\tasknumber{2}%
\task{%
    \begin{itemize}
        \item Как проведёте вечер перед экзаменом по русскому языку?
        \item Куда съездите и как отдохнёте после?
    \end{itemize}
}
\solutionspace{100pt}

\tasknumber{3}%
\task{%
    О существовании какого населённого пункта вы узнали этой весной? Чем он вам запомнился?
}
\solutionspace{60pt}

\tasknumber{4}%
\task{%
    \begin{itemize}
        \item Какой ЕГЭ (кроме математики и русского) вы сдаете?
        \item Сколько времени у вас будет на финальную подготовку и повторение?
        \item Как вы распределите время на экзамене?
        \item Как искать ошибки и проверять себя?
    \end{itemize}
}
\solutionspace{120pt}

\tasknumber{5}%
\task{%
    Выберите и запишите из всего курса физики 2 формулы:
    \begin{itemize}
        \item наиболее запомнившуюсю,
        \item и наименее интересную.
    \end{itemize}
}
\solutionspace{120pt}

\tasknumber{6}%
\task{%
    \begin{itemize}
        \item Какие физические открытия или достижения были сделаны на веку вас и ваших родителей?
        \item Как это повлияло на нашу жизнь?
    \end{itemize}
}


\variantsplitter


\addpersonalvariant{Ярослав Лавровский}

\tasknumber{7}%
\task{%
    Назовите 6 знаменитых деятелей и деятельниц физики и области науки, в которых они работали.
    Обеспечьте представительство различных групп.
}
\solutionspace{100pt}

\tasknumber{8}%
\task{%
    Назовите 3 причины для вас лично вернуться в школу
}
\solutionspace{60pt}

\tasknumber{9}%
\task{%
    Расскажите на какой работе вы уже поработали?
    Сколько часов в неделю и сколько недель или месяцев опыта? Норм по деньгам?
}
\solutionspace{120pt}

\tasknumber{10}%
\task{%
    Где бы вы хотели поработать, но ТЕПЕРЬ не получится?
    Какие новые возможности открылись за последние месяцы?
}
\solutionspace{60pt}

\tasknumber{11}%
\task{%
    Что нового вы попробуете в июле-августе?
    Куда надо съездить и где побывать?
    Укажите не менее 3 активностей и 7 мест.
}
\solutionspace{80pt}

\tasknumber{12}%
\task{%
    По какому принципу зарубежные университеты сохраняют партнёрство с российскими?
}
\solutionspace{80pt}

\tasknumber{13}%
\task{%
    Посоветуйте свой (или не свой) Тг/TT/YT/Ig.
    Точным названием или ссылкой, не более двух ответов.
}

\variantsplitter

\addpersonalvariant{Анастасия Ламанова}

\tasknumber{1}%
\task{%
    \begin{itemize}
        \item Как меня зовут?
        \item Как называется предмет?
        \item Какого цвета учебник?
        \item Чем заканчивается анекдот?
    \end{itemize}
}

\tasknumber{2}%
\task{%
    \begin{itemize}
        \item Как проведёте утро перед экзаменом по математике?
        \item Куда съездите и как отдохнёте после?
    \end{itemize}
}
\solutionspace{100pt}

\tasknumber{3}%
\task{%
    О существовании какого населённого пункта вы узнали этой весной? Чем он вам запомнился?
}
\solutionspace{60pt}

\tasknumber{4}%
\task{%
    \begin{itemize}
        \item Какой ЕГЭ (кроме математики и русского) вы сдаете?
        \item Сколько времени у вас будет на финальную подготовку и повторение?
        \item Как вы распределите время на экзамене?
        \item Как искать ошибки и проверять себя?
    \end{itemize}
}
\solutionspace{120pt}

\tasknumber{5}%
\task{%
    Выберите и запишите из всего курса физики 2 формулы:
    \begin{itemize}
        \item наиболее смешную,
        \item и наименее важную.
    \end{itemize}
}
\solutionspace{120pt}

\tasknumber{6}%
\task{%
    \begin{itemize}
        \item Какие физические открытия или достижения были сделаны на веку вас и ваших родителей?
        \item Как это повлияло на нашу жизнь?
    \end{itemize}
}


\variantsplitter


\addpersonalvariant{Анастасия Ламанова}

\tasknumber{7}%
\task{%
    Назовите 6 знаменитых деятелей и деятельниц физики и области науки, в которых они работали.
    Обеспечьте представительство различных групп.
}
\solutionspace{100pt}

\tasknumber{8}%
\task{%
    Назовите 3 причины для вас лично вернуться в школу
}
\solutionspace{60pt}

\tasknumber{9}%
\task{%
    Расскажите на какой работе вы уже поработали?
    Сколько часов в неделю и сколько недель или месяцев опыта? Норм по деньгам?
}
\solutionspace{120pt}

\tasknumber{10}%
\task{%
    Где бы вы хотели поработать, но ТЕПЕРЬ не получится?
    Какие новые возможности открылись за последние месяцы?
}
\solutionspace{60pt}

\tasknumber{11}%
\task{%
    Что нового вы попробуете в июле-августе?
    Куда надо съездить и где побывать?
    Укажите не менее 3 активностей и 7 мест.
}
\solutionspace{80pt}

\tasknumber{12}%
\task{%
    По какому принципу зарубежные университеты сохраняют партнёрство с российскими?
}
\solutionspace{80pt}

\tasknumber{13}%
\task{%
    Посоветуйте свой (или не свой) Тг/TT/YT/Ig.
    Точным названием или ссылкой, не более двух ответов.
}

\variantsplitter

\addpersonalvariant{Виктория Легонькова}

\tasknumber{1}%
\task{%
    \begin{itemize}
        \item Как меня зовут?
        \item Как называется предмет?
        \item Какого цвета учебник?
        \item Чем заканчивается анекдот?
    \end{itemize}
}

\tasknumber{2}%
\task{%
    \begin{itemize}
        \item Как проведёте вечер перед экзаменом по математике?
        \item Куда съездите и как отдохнёте после?
    \end{itemize}
}
\solutionspace{100pt}

\tasknumber{3}%
\task{%
    О существовании какого населённого пункта вы узнали этой весной? Чем он вам запомнился?
}
\solutionspace{60pt}

\tasknumber{4}%
\task{%
    \begin{itemize}
        \item Какой ЕГЭ (кроме математики и русского) вы сдаете?
        \item Сколько времени у вас будет на финальную подготовку и повторение?
        \item Как вы распределите время на экзамене?
        \item Как искать ошибки и проверять себя?
    \end{itemize}
}
\solutionspace{120pt}

\tasknumber{5}%
\task{%
    Выберите и запишите из всего курса физики 2 формулы:
    \begin{itemize}
        \item наиболее полезную,
        \item и наименее важную.
    \end{itemize}
}
\solutionspace{120pt}

\tasknumber{6}%
\task{%
    \begin{itemize}
        \item Какие физические открытия или достижения были сделаны на веку вас и ваших родителей?
        \item Как это повлияло на нашу жизнь?
    \end{itemize}
}


\variantsplitter


\addpersonalvariant{Виктория Легонькова}

\tasknumber{7}%
\task{%
    Назовите 6 знаменитых деятелей и деятельниц физики и области науки, в которых они работали.
    Обеспечьте представительство различных групп.
}
\solutionspace{100pt}

\tasknumber{8}%
\task{%
    Назовите 3 причины для вас лично вернуться в школу
}
\solutionspace{60pt}

\tasknumber{9}%
\task{%
    Расскажите на какой работе вы уже поработали?
    Сколько часов в неделю и сколько недель или месяцев опыта? Норм по деньгам?
}
\solutionspace{120pt}

\tasknumber{10}%
\task{%
    Где бы вы хотели поработать, но ТЕПЕРЬ не получится?
    Какие новые возможности открылись за последние месяцы?
}
\solutionspace{60pt}

\tasknumber{11}%
\task{%
    Что нового вы попробуете в июле-августе?
    Куда надо съездить и где побывать?
    Укажите не менее 3 активностей и 7 мест.
}
\solutionspace{80pt}

\tasknumber{12}%
\task{%
    По какому принципу зарубежные университеты сохраняют партнёрство с российскими?
}
\solutionspace{80pt}

\tasknumber{13}%
\task{%
    Посоветуйте свой (или не свой) Тг/TT/YT/Ig.
    Точным названием или ссылкой, не более двух ответов.
}

\variantsplitter

\addpersonalvariant{Семён Мартынов}

\tasknumber{1}%
\task{%
    \begin{itemize}
        \item Как меня зовут?
        \item Как называется предмет?
        \item Какого цвета учебник?
        \item Чем заканчивается анекдот?
    \end{itemize}
}

\tasknumber{2}%
\task{%
    \begin{itemize}
        \item Как проведёте утро перед экзаменом по русскому языку?
        \item Куда съездите и как отдохнёте после?
    \end{itemize}
}
\solutionspace{100pt}

\tasknumber{3}%
\task{%
    О существовании какого населённого пункта вы узнали этой весной? Чем он вам запомнился?
}
\solutionspace{60pt}

\tasknumber{4}%
\task{%
    \begin{itemize}
        \item Какой ЕГЭ (кроме математики и русского) вы сдаете?
        \item Сколько времени у вас будет на финальную подготовку и повторение?
        \item Как вы распределите время на экзамене?
        \item Как искать ошибки и проверять себя?
    \end{itemize}
}
\solutionspace{120pt}

\tasknumber{5}%
\task{%
    Выберите и запишите из всего курса физики 2 формулы:
    \begin{itemize}
        \item наиболее запомнившуюсю,
        \item и наименее важную.
    \end{itemize}
}
\solutionspace{120pt}

\tasknumber{6}%
\task{%
    \begin{itemize}
        \item Какие физические открытия или достижения были сделаны на веку вас и ваших родителей?
        \item Как это повлияло на нашу жизнь?
    \end{itemize}
}


\variantsplitter


\addpersonalvariant{Семён Мартынов}

\tasknumber{7}%
\task{%
    Назовите 6 знаменитых деятелей и деятельниц физики и области науки, в которых они работали.
    Обеспечьте представительство различных групп.
}
\solutionspace{100pt}

\tasknumber{8}%
\task{%
    Назовите 3 причины для вас лично вернуться в школу
}
\solutionspace{60pt}

\tasknumber{9}%
\task{%
    Расскажите на какой работе вы уже поработали?
    Сколько часов в неделю и сколько недель или месяцев опыта? Норм по деньгам?
}
\solutionspace{120pt}

\tasknumber{10}%
\task{%
    Где бы вы хотели поработать, но ТЕПЕРЬ не получится?
    Какие новые возможности открылись за последние месяцы?
}
\solutionspace{60pt}

\tasknumber{11}%
\task{%
    Что нового вы попробуете в июле-августе?
    Куда надо съездить и где побывать?
    Укажите не менее 3 активностей и 7 мест.
}
\solutionspace{80pt}

\tasknumber{12}%
\task{%
    По какому принципу зарубежные университеты сохраняют партнёрство с российскими?
}
\solutionspace{80pt}

\tasknumber{13}%
\task{%
    Посоветуйте свой (или не свой) Тг/TT/YT/Ig.
    Точным названием или ссылкой, не более двух ответов.
}

\variantsplitter

\addpersonalvariant{Варвара Минаева}

\tasknumber{1}%
\task{%
    \begin{itemize}
        \item Как меня зовут?
        \item Как называется предмет?
        \item Какого цвета учебник?
        \item Чем заканчивается анекдот?
    \end{itemize}
}

\tasknumber{2}%
\task{%
    \begin{itemize}
        \item Как проведёте утро перед экзаменом по математике?
        \item Куда съездите и как отдохнёте после?
    \end{itemize}
}
\solutionspace{100pt}

\tasknumber{3}%
\task{%
    О существовании какого населённого пункта вы узнали этой весной? Чем он вам запомнился?
}
\solutionspace{60pt}

\tasknumber{4}%
\task{%
    \begin{itemize}
        \item Какой ЕГЭ (кроме математики и русского) вы сдаете?
        \item Сколько времени у вас будет на финальную подготовку и повторение?
        \item Как вы распределите время на экзамене?
        \item Как искать ошибки и проверять себя?
    \end{itemize}
}
\solutionspace{120pt}

\tasknumber{5}%
\task{%
    Выберите и запишите из всего курса физики 2 формулы:
    \begin{itemize}
        \item наиболее удивительную,
        \item и наименее важную.
    \end{itemize}
}
\solutionspace{120pt}

\tasknumber{6}%
\task{%
    \begin{itemize}
        \item Какие физические открытия или достижения были сделаны на веку вас и ваших родителей?
        \item Как это повлияло на нашу жизнь?
    \end{itemize}
}


\variantsplitter


\addpersonalvariant{Варвара Минаева}

\tasknumber{7}%
\task{%
    Назовите 6 знаменитых деятелей и деятельниц физики и области науки, в которых они работали.
    Обеспечьте представительство различных групп.
}
\solutionspace{100pt}

\tasknumber{8}%
\task{%
    Назовите 3 причины для вас лично вернуться в школу
}
\solutionspace{60pt}

\tasknumber{9}%
\task{%
    Расскажите на какой работе вы уже поработали?
    Сколько часов в неделю и сколько недель или месяцев опыта? Норм по деньгам?
}
\solutionspace{120pt}

\tasknumber{10}%
\task{%
    Где бы вы хотели поработать, но ТЕПЕРЬ не получится?
    Какие новые возможности открылись за последние месяцы?
}
\solutionspace{60pt}

\tasknumber{11}%
\task{%
    Что нового вы попробуете в июле-августе?
    Куда надо съездить и где побывать?
    Укажите не менее 3 активностей и 7 мест.
}
\solutionspace{80pt}

\tasknumber{12}%
\task{%
    По какому принципу зарубежные университеты сохраняют партнёрство с российскими?
}
\solutionspace{80pt}

\tasknumber{13}%
\task{%
    Посоветуйте свой (или не свой) Тг/TT/YT/Ig.
    Точным названием или ссылкой, не более двух ответов.
}

\variantsplitter

\addpersonalvariant{Леонид Никитин}

\tasknumber{1}%
\task{%
    \begin{itemize}
        \item Как меня зовут?
        \item Как называется предмет?
        \item Какого цвета учебник?
        \item Чем заканчивается анекдот?
    \end{itemize}
}

\tasknumber{2}%
\task{%
    \begin{itemize}
        \item Как проведёте утро перед экзаменом по русскому языку?
        \item Куда съездите и как отдохнёте после?
    \end{itemize}
}
\solutionspace{100pt}

\tasknumber{3}%
\task{%
    О существовании какого населённого пункта вы узнали этой весной? Чем он вам запомнился?
}
\solutionspace{60pt}

\tasknumber{4}%
\task{%
    \begin{itemize}
        \item Какой ЕГЭ (кроме математики и русского) вы сдаете?
        \item Сколько времени у вас будет на финальную подготовку и повторение?
        \item Как вы распределите время на экзамене?
        \item Как искать ошибки и проверять себя?
    \end{itemize}
}
\solutionspace{120pt}

\tasknumber{5}%
\task{%
    Выберите и запишите из всего курса физики 2 формулы:
    \begin{itemize}
        \item наиболее смешную,
        \item и наименее интересную.
    \end{itemize}
}
\solutionspace{120pt}

\tasknumber{6}%
\task{%
    \begin{itemize}
        \item Какие физические открытия или достижения были сделаны на веку вас и ваших родителей?
        \item Как это повлияло на нашу жизнь?
    \end{itemize}
}


\variantsplitter


\addpersonalvariant{Леонид Никитин}

\tasknumber{7}%
\task{%
    Назовите 6 знаменитых деятелей и деятельниц физики и области науки, в которых они работали.
    Обеспечьте представительство различных групп.
}
\solutionspace{100pt}

\tasknumber{8}%
\task{%
    Назовите 3 причины для вас лично вернуться в школу
}
\solutionspace{60pt}

\tasknumber{9}%
\task{%
    Расскажите на какой работе вы уже поработали?
    Сколько часов в неделю и сколько недель или месяцев опыта? Норм по деньгам?
}
\solutionspace{120pt}

\tasknumber{10}%
\task{%
    Где бы вы хотели поработать, но ТЕПЕРЬ не получится?
    Какие новые возможности открылись за последние месяцы?
}
\solutionspace{60pt}

\tasknumber{11}%
\task{%
    Что нового вы попробуете в июле-августе?
    Куда надо съездить и где побывать?
    Укажите не менее 3 активностей и 7 мест.
}
\solutionspace{80pt}

\tasknumber{12}%
\task{%
    По какому принципу зарубежные университеты сохраняют партнёрство с российскими?
}
\solutionspace{80pt}

\tasknumber{13}%
\task{%
    Посоветуйте свой (или не свой) Тг/TT/YT/Ig.
    Точным названием или ссылкой, не более двух ответов.
}

\variantsplitter

\addpersonalvariant{Тимофей Полетаев}

\tasknumber{1}%
\task{%
    \begin{itemize}
        \item Как меня зовут?
        \item Как называется предмет?
        \item Какого цвета учебник?
        \item Чем заканчивается анекдот?
    \end{itemize}
}

\tasknumber{2}%
\task{%
    \begin{itemize}
        \item Как проведёте вечер перед экзаменом по математике?
        \item Куда съездите и как отдохнёте после?
    \end{itemize}
}
\solutionspace{100pt}

\tasknumber{3}%
\task{%
    О существовании какого населённого пункта вы узнали этой весной? Чем он вам запомнился?
}
\solutionspace{60pt}

\tasknumber{4}%
\task{%
    \begin{itemize}
        \item Какой ЕГЭ (кроме математики и русского) вы сдаете?
        \item Сколько времени у вас будет на финальную подготовку и повторение?
        \item Как вы распределите время на экзамене?
        \item Как искать ошибки и проверять себя?
    \end{itemize}
}
\solutionspace{120pt}

\tasknumber{5}%
\task{%
    Выберите и запишите из всего курса физики 2 формулы:
    \begin{itemize}
        \item наиболее запомнившуюсю,
        \item и наименее важную.
    \end{itemize}
}
\solutionspace{120pt}

\tasknumber{6}%
\task{%
    \begin{itemize}
        \item Какие физические открытия или достижения были сделаны на веку вас и ваших родителей?
        \item Как это повлияло на нашу жизнь?
    \end{itemize}
}


\variantsplitter


\addpersonalvariant{Тимофей Полетаев}

\tasknumber{7}%
\task{%
    Назовите 6 знаменитых деятелей и деятельниц физики и области науки, в которых они работали.
    Обеспечьте представительство различных групп.
}
\solutionspace{100pt}

\tasknumber{8}%
\task{%
    Назовите 3 причины для вас лично вернуться в школу
}
\solutionspace{60pt}

\tasknumber{9}%
\task{%
    Расскажите на какой работе вы уже поработали?
    Сколько часов в неделю и сколько недель или месяцев опыта? Норм по деньгам?
}
\solutionspace{120pt}

\tasknumber{10}%
\task{%
    Где бы вы хотели поработать, но ТЕПЕРЬ не получится?
    Какие новые возможности открылись за последние месяцы?
}
\solutionspace{60pt}

\tasknumber{11}%
\task{%
    Что нового вы попробуете в июле-августе?
    Куда надо съездить и где побывать?
    Укажите не менее 3 активностей и 7 мест.
}
\solutionspace{80pt}

\tasknumber{12}%
\task{%
    По какому принципу зарубежные университеты сохраняют партнёрство с российскими?
}
\solutionspace{80pt}

\tasknumber{13}%
\task{%
    Посоветуйте свой (или не свой) Тг/TT/YT/Ig.
    Точным названием или ссылкой, не более двух ответов.
}

\variantsplitter

\addpersonalvariant{Андрей Рожков}

\tasknumber{1}%
\task{%
    \begin{itemize}
        \item Как меня зовут?
        \item Как называется предмет?
        \item Какого цвета учебник?
        \item Чем заканчивается анекдот?
    \end{itemize}
}

\tasknumber{2}%
\task{%
    \begin{itemize}
        \item Как проведёте вечер перед экзаменом по математике?
        \item Куда съездите и как отдохнёте после?
    \end{itemize}
}
\solutionspace{100pt}

\tasknumber{3}%
\task{%
    О существовании какого населённого пункта вы узнали этой весной? Чем он вам запомнился?
}
\solutionspace{60pt}

\tasknumber{4}%
\task{%
    \begin{itemize}
        \item Какой ЕГЭ (кроме математики и русского) вы сдаете?
        \item Сколько времени у вас будет на финальную подготовку и повторение?
        \item Как вы распределите время на экзамене?
        \item Как искать ошибки и проверять себя?
    \end{itemize}
}
\solutionspace{120pt}

\tasknumber{5}%
\task{%
    Выберите и запишите из всего курса физики 2 формулы:
    \begin{itemize}
        \item наиболее полезную,
        \item и наименее важную.
    \end{itemize}
}
\solutionspace{120pt}

\tasknumber{6}%
\task{%
    \begin{itemize}
        \item Какие физические открытия или достижения были сделаны на веку вас и ваших родителей?
        \item Как это повлияло на нашу жизнь?
    \end{itemize}
}


\variantsplitter


\addpersonalvariant{Андрей Рожков}

\tasknumber{7}%
\task{%
    Назовите 6 знаменитых деятелей и деятельниц физики и области науки, в которых они работали.
    Обеспечьте представительство различных групп.
}
\solutionspace{100pt}

\tasknumber{8}%
\task{%
    Назовите 3 причины для вас лично вернуться в школу
}
\solutionspace{60pt}

\tasknumber{9}%
\task{%
    Расскажите на какой работе вы уже поработали?
    Сколько часов в неделю и сколько недель или месяцев опыта? Норм по деньгам?
}
\solutionspace{120pt}

\tasknumber{10}%
\task{%
    Где бы вы хотели поработать, но ТЕПЕРЬ не получится?
    Какие новые возможности открылись за последние месяцы?
}
\solutionspace{60pt}

\tasknumber{11}%
\task{%
    Что нового вы попробуете в июле-августе?
    Куда надо съездить и где побывать?
    Укажите не менее 3 активностей и 7 мест.
}
\solutionspace{80pt}

\tasknumber{12}%
\task{%
    По какому принципу зарубежные университеты сохраняют партнёрство с российскими?
}
\solutionspace{80pt}

\tasknumber{13}%
\task{%
    Посоветуйте свой (или не свой) Тг/TT/YT/Ig.
    Точным названием или ссылкой, не более двух ответов.
}

\variantsplitter

\addpersonalvariant{Рената Таржиманова}

\tasknumber{1}%
\task{%
    \begin{itemize}
        \item Как меня зовут?
        \item Как называется предмет?
        \item Какого цвета учебник?
        \item Чем заканчивается анекдот?
    \end{itemize}
}

\tasknumber{2}%
\task{%
    \begin{itemize}
        \item Как проведёте вечер перед экзаменом по математике?
        \item Куда съездите и как отдохнёте после?
    \end{itemize}
}
\solutionspace{100pt}

\tasknumber{3}%
\task{%
    О существовании какого населённого пункта вы узнали этой весной? Чем он вам запомнился?
}
\solutionspace{60pt}

\tasknumber{4}%
\task{%
    \begin{itemize}
        \item Какой ЕГЭ (кроме математики и русского) вы сдаете?
        \item Сколько времени у вас будет на финальную подготовку и повторение?
        \item Как вы распределите время на экзамене?
        \item Как искать ошибки и проверять себя?
    \end{itemize}
}
\solutionspace{120pt}

\tasknumber{5}%
\task{%
    Выберите и запишите из всего курса физики 2 формулы:
    \begin{itemize}
        \item наиболее удивительную,
        \item и наименее важную.
    \end{itemize}
}
\solutionspace{120pt}

\tasknumber{6}%
\task{%
    \begin{itemize}
        \item Какие физические открытия или достижения были сделаны на веку вас и ваших родителей?
        \item Как это повлияло на нашу жизнь?
    \end{itemize}
}


\variantsplitter


\addpersonalvariant{Рената Таржиманова}

\tasknumber{7}%
\task{%
    Назовите 6 знаменитых деятелей и деятельниц физики и области науки, в которых они работали.
    Обеспечьте представительство различных групп.
}
\solutionspace{100pt}

\tasknumber{8}%
\task{%
    Назовите 3 причины для вас лично вернуться в школу
}
\solutionspace{60pt}

\tasknumber{9}%
\task{%
    Расскажите на какой работе вы уже поработали?
    Сколько часов в неделю и сколько недель или месяцев опыта? Норм по деньгам?
}
\solutionspace{120pt}

\tasknumber{10}%
\task{%
    Где бы вы хотели поработать, но ТЕПЕРЬ не получится?
    Какие новые возможности открылись за последние месяцы?
}
\solutionspace{60pt}

\tasknumber{11}%
\task{%
    Что нового вы попробуете в июле-августе?
    Куда надо съездить и где побывать?
    Укажите не менее 3 активностей и 7 мест.
}
\solutionspace{80pt}

\tasknumber{12}%
\task{%
    По какому принципу зарубежные университеты сохраняют партнёрство с российскими?
}
\solutionspace{80pt}

\tasknumber{13}%
\task{%
    Посоветуйте свой (или не свой) Тг/TT/YT/Ig.
    Точным названием или ссылкой, не более двух ответов.
}

\variantsplitter

\addpersonalvariant{Андрей Щербаков}

\tasknumber{1}%
\task{%
    \begin{itemize}
        \item Как меня зовут?
        \item Как называется предмет?
        \item Какого цвета учебник?
        \item Чем заканчивается анекдот?
    \end{itemize}
}

\tasknumber{2}%
\task{%
    \begin{itemize}
        \item Как проведёте утро перед экзаменом по математике?
        \item Куда съездите и как отдохнёте после?
    \end{itemize}
}
\solutionspace{100pt}

\tasknumber{3}%
\task{%
    О существовании какого населённого пункта вы узнали этой весной? Чем он вам запомнился?
}
\solutionspace{60pt}

\tasknumber{4}%
\task{%
    \begin{itemize}
        \item Какой ЕГЭ (кроме математики и русского) вы сдаете?
        \item Сколько времени у вас будет на финальную подготовку и повторение?
        \item Как вы распределите время на экзамене?
        \item Как искать ошибки и проверять себя?
    \end{itemize}
}
\solutionspace{120pt}

\tasknumber{5}%
\task{%
    Выберите и запишите из всего курса физики 2 формулы:
    \begin{itemize}
        \item наиболее удивительную,
        \item и наименее важную.
    \end{itemize}
}
\solutionspace{120pt}

\tasknumber{6}%
\task{%
    \begin{itemize}
        \item Какие физические открытия или достижения были сделаны на веку вас и ваших родителей?
        \item Как это повлияло на нашу жизнь?
    \end{itemize}
}


\variantsplitter


\addpersonalvariant{Андрей Щербаков}

\tasknumber{7}%
\task{%
    Назовите 6 знаменитых деятелей и деятельниц физики и области науки, в которых они работали.
    Обеспечьте представительство различных групп.
}
\solutionspace{100pt}

\tasknumber{8}%
\task{%
    Назовите 3 причины для вас лично вернуться в школу
}
\solutionspace{60pt}

\tasknumber{9}%
\task{%
    Расскажите на какой работе вы уже поработали?
    Сколько часов в неделю и сколько недель или месяцев опыта? Норм по деньгам?
}
\solutionspace{120pt}

\tasknumber{10}%
\task{%
    Где бы вы хотели поработать, но ТЕПЕРЬ не получится?
    Какие новые возможности открылись за последние месяцы?
}
\solutionspace{60pt}

\tasknumber{11}%
\task{%
    Что нового вы попробуете в июле-августе?
    Куда надо съездить и где побывать?
    Укажите не менее 3 активностей и 7 мест.
}
\solutionspace{80pt}

\tasknumber{12}%
\task{%
    По какому принципу зарубежные университеты сохраняют партнёрство с российскими?
}
\solutionspace{80pt}

\tasknumber{13}%
\task{%
    Посоветуйте свой (или не свой) Тг/TT/YT/Ig.
    Точным названием или ссылкой, не более двух ответов.
}

\variantsplitter

\addpersonalvariant{Михаил Ярошевский}

\tasknumber{1}%
\task{%
    \begin{itemize}
        \item Как меня зовут?
        \item Как называется предмет?
        \item Какого цвета учебник?
        \item Чем заканчивается анекдот?
    \end{itemize}
}

\tasknumber{2}%
\task{%
    \begin{itemize}
        \item Как проведёте вечер перед экзаменом по русскому языку?
        \item Куда съездите и как отдохнёте после?
    \end{itemize}
}
\solutionspace{100pt}

\tasknumber{3}%
\task{%
    О существовании какого населённого пункта вы узнали этой весной? Чем он вам запомнился?
}
\solutionspace{60pt}

\tasknumber{4}%
\task{%
    \begin{itemize}
        \item Какой ЕГЭ (кроме математики и русского) вы сдаете?
        \item Сколько времени у вас будет на финальную подготовку и повторение?
        \item Как вы распределите время на экзамене?
        \item Как искать ошибки и проверять себя?
    \end{itemize}
}
\solutionspace{120pt}

\tasknumber{5}%
\task{%
    Выберите и запишите из всего курса физики 2 формулы:
    \begin{itemize}
        \item наиболее смешную,
        \item и наименее интересную.
    \end{itemize}
}
\solutionspace{120pt}

\tasknumber{6}%
\task{%
    \begin{itemize}
        \item Какие физические открытия или достижения были сделаны на веку вас и ваших родителей?
        \item Как это повлияло на нашу жизнь?
    \end{itemize}
}


\variantsplitter


\addpersonalvariant{Михаил Ярошевский}

\tasknumber{7}%
\task{%
    Назовите 6 знаменитых деятелей и деятельниц физики и области науки, в которых они работали.
    Обеспечьте представительство различных групп.
}
\solutionspace{100pt}

\tasknumber{8}%
\task{%
    Назовите 3 причины для вас лично вернуться в школу
}
\solutionspace{60pt}

\tasknumber{9}%
\task{%
    Расскажите на какой работе вы уже поработали?
    Сколько часов в неделю и сколько недель или месяцев опыта? Норм по деньгам?
}
\solutionspace{120pt}

\tasknumber{10}%
\task{%
    Где бы вы хотели поработать, но ТЕПЕРЬ не получится?
    Какие новые возможности открылись за последние месяцы?
}
\solutionspace{60pt}

\tasknumber{11}%
\task{%
    Что нового вы попробуете в июле-августе?
    Куда надо съездить и где побывать?
    Укажите не менее 3 активностей и 7 мест.
}
\solutionspace{80pt}

\tasknumber{12}%
\task{%
    По какому принципу зарубежные университеты сохраняют партнёрство с российскими?
}
\solutionspace{80pt}

\tasknumber{13}%
\task{%
    Посоветуйте свой (или не свой) Тг/TT/YT/Ig.
    Точным названием или ссылкой, не более двух ответов.
}

\variantsplitter

\addpersonalvariant{Алексей Алимпиев}

\tasknumber{1}%
\task{%
    \begin{itemize}
        \item Как меня зовут?
        \item Как называется предмет?
        \item Какого цвета учебник?
        \item Чем заканчивается анекдот?
    \end{itemize}
}

\tasknumber{2}%
\task{%
    \begin{itemize}
        \item Как проведёте утро перед экзаменом по математике?
        \item Куда съездите и как отдохнёте после?
    \end{itemize}
}
\solutionspace{100pt}

\tasknumber{3}%
\task{%
    О существовании какого населённого пункта вы узнали этой весной? Чем он вам запомнился?
}
\solutionspace{60pt}

\tasknumber{4}%
\task{%
    \begin{itemize}
        \item Какой ЕГЭ (кроме математики и русского) вы сдаете?
        \item Сколько времени у вас будет на финальную подготовку и повторение?
        \item Как вы распределите время на экзамене?
        \item Как искать ошибки и проверять себя?
    \end{itemize}
}
\solutionspace{120pt}

\tasknumber{5}%
\task{%
    Выберите и запишите из всего курса физики 2 формулы:
    \begin{itemize}
        \item наиболее смешную,
        \item и наименее важную.
    \end{itemize}
}
\solutionspace{120pt}

\tasknumber{6}%
\task{%
    \begin{itemize}
        \item Какие физические открытия или достижения были сделаны на веку вас и ваших родителей?
        \item Как это повлияло на нашу жизнь?
    \end{itemize}
}


\variantsplitter


\addpersonalvariant{Алексей Алимпиев}

\tasknumber{7}%
\task{%
    Назовите 6 знаменитых деятелей и деятельниц физики и области науки, в которых они работали.
    Обеспечьте представительство различных групп.
}
\solutionspace{100pt}

\tasknumber{8}%
\task{%
    Назовите 3 причины для вас лично вернуться в школу
}
\solutionspace{60pt}

\tasknumber{9}%
\task{%
    Расскажите на какой работе вы уже поработали?
    Сколько часов в неделю и сколько недель или месяцев опыта? Норм по деньгам?
}
\solutionspace{120pt}

\tasknumber{10}%
\task{%
    Где бы вы хотели поработать, но ТЕПЕРЬ не получится?
    Какие новые возможности открылись за последние месяцы?
}
\solutionspace{60pt}

\tasknumber{11}%
\task{%
    Что нового вы попробуете в июле-августе?
    Куда надо съездить и где побывать?
    Укажите не менее 3 активностей и 7 мест.
}
\solutionspace{80pt}

\tasknumber{12}%
\task{%
    По какому принципу зарубежные университеты сохраняют партнёрство с российскими?
}
\solutionspace{80pt}

\tasknumber{13}%
\task{%
    Посоветуйте свой (или не свой) Тг/TT/YT/Ig.
    Точным названием или ссылкой, не более двух ответов.
}

\variantsplitter

\addpersonalvariant{Евгений Васин}

\tasknumber{1}%
\task{%
    \begin{itemize}
        \item Как меня зовут?
        \item Как называется предмет?
        \item Какого цвета учебник?
        \item Чем заканчивается анекдот?
    \end{itemize}
}

\tasknumber{2}%
\task{%
    \begin{itemize}
        \item Как проведёте вечер перед экзаменом по русскому языку?
        \item Куда съездите и как отдохнёте после?
    \end{itemize}
}
\solutionspace{100pt}

\tasknumber{3}%
\task{%
    О существовании какого населённого пункта вы узнали этой весной? Чем он вам запомнился?
}
\solutionspace{60pt}

\tasknumber{4}%
\task{%
    \begin{itemize}
        \item Какой ЕГЭ (кроме математики и русского) вы сдаете?
        \item Сколько времени у вас будет на финальную подготовку и повторение?
        \item Как вы распределите время на экзамене?
        \item Как искать ошибки и проверять себя?
    \end{itemize}
}
\solutionspace{120pt}

\tasknumber{5}%
\task{%
    Выберите и запишите из всего курса физики 2 формулы:
    \begin{itemize}
        \item наиболее удивительную,
        \item и наименее важную.
    \end{itemize}
}
\solutionspace{120pt}

\tasknumber{6}%
\task{%
    \begin{itemize}
        \item Какие физические открытия или достижения были сделаны на веку вас и ваших родителей?
        \item Как это повлияло на нашу жизнь?
    \end{itemize}
}


\variantsplitter


\addpersonalvariant{Евгений Васин}

\tasknumber{7}%
\task{%
    Назовите 6 знаменитых деятелей и деятельниц физики и области науки, в которых они работали.
    Обеспечьте представительство различных групп.
}
\solutionspace{100pt}

\tasknumber{8}%
\task{%
    Назовите 3 причины для вас лично вернуться в школу
}
\solutionspace{60pt}

\tasknumber{9}%
\task{%
    Расскажите на какой работе вы уже поработали?
    Сколько часов в неделю и сколько недель или месяцев опыта? Норм по деньгам?
}
\solutionspace{120pt}

\tasknumber{10}%
\task{%
    Где бы вы хотели поработать, но ТЕПЕРЬ не получится?
    Какие новые возможности открылись за последние месяцы?
}
\solutionspace{60pt}

\tasknumber{11}%
\task{%
    Что нового вы попробуете в июле-августе?
    Куда надо съездить и где побывать?
    Укажите не менее 3 активностей и 7 мест.
}
\solutionspace{80pt}

\tasknumber{12}%
\task{%
    По какому принципу зарубежные университеты сохраняют партнёрство с российскими?
}
\solutionspace{80pt}

\tasknumber{13}%
\task{%
    Посоветуйте свой (или не свой) Тг/TT/YT/Ig.
    Точным названием или ссылкой, не более двух ответов.
}

\variantsplitter

\addpersonalvariant{Вячеслав Волохов}

\tasknumber{1}%
\task{%
    \begin{itemize}
        \item Как меня зовут?
        \item Как называется предмет?
        \item Какого цвета учебник?
        \item Чем заканчивается анекдот?
    \end{itemize}
}

\tasknumber{2}%
\task{%
    \begin{itemize}
        \item Как проведёте вечер перед экзаменом по математике?
        \item Куда съездите и как отдохнёте после?
    \end{itemize}
}
\solutionspace{100pt}

\tasknumber{3}%
\task{%
    О существовании какого населённого пункта вы узнали этой весной? Чем он вам запомнился?
}
\solutionspace{60pt}

\tasknumber{4}%
\task{%
    \begin{itemize}
        \item Какой ЕГЭ (кроме математики и русского) вы сдаете?
        \item Сколько времени у вас будет на финальную подготовку и повторение?
        \item Как вы распределите время на экзамене?
        \item Как искать ошибки и проверять себя?
    \end{itemize}
}
\solutionspace{120pt}

\tasknumber{5}%
\task{%
    Выберите и запишите из всего курса физики 2 формулы:
    \begin{itemize}
        \item наиболее запомнившуюсю,
        \item и наименее важную.
    \end{itemize}
}
\solutionspace{120pt}

\tasknumber{6}%
\task{%
    \begin{itemize}
        \item Какие физические открытия или достижения были сделаны на веку вас и ваших родителей?
        \item Как это повлияло на нашу жизнь?
    \end{itemize}
}


\variantsplitter


\addpersonalvariant{Вячеслав Волохов}

\tasknumber{7}%
\task{%
    Назовите 6 знаменитых деятелей и деятельниц физики и области науки, в которых они работали.
    Обеспечьте представительство различных групп.
}
\solutionspace{100pt}

\tasknumber{8}%
\task{%
    Назовите 3 причины для вас лично вернуться в школу
}
\solutionspace{60pt}

\tasknumber{9}%
\task{%
    Расскажите на какой работе вы уже поработали?
    Сколько часов в неделю и сколько недель или месяцев опыта? Норм по деньгам?
}
\solutionspace{120pt}

\tasknumber{10}%
\task{%
    Где бы вы хотели поработать, но ТЕПЕРЬ не получится?
    Какие новые возможности открылись за последние месяцы?
}
\solutionspace{60pt}

\tasknumber{11}%
\task{%
    Что нового вы попробуете в июле-августе?
    Куда надо съездить и где побывать?
    Укажите не менее 3 активностей и 7 мест.
}
\solutionspace{80pt}

\tasknumber{12}%
\task{%
    По какому принципу зарубежные университеты сохраняют партнёрство с российскими?
}
\solutionspace{80pt}

\tasknumber{13}%
\task{%
    Посоветуйте свой (или не свой) Тг/TT/YT/Ig.
    Точным названием или ссылкой, не более двух ответов.
}

\variantsplitter

\addpersonalvariant{Герман Говоров}

\tasknumber{1}%
\task{%
    \begin{itemize}
        \item Как меня зовут?
        \item Как называется предмет?
        \item Какого цвета учебник?
        \item Чем заканчивается анекдот?
    \end{itemize}
}

\tasknumber{2}%
\task{%
    \begin{itemize}
        \item Как проведёте утро перед экзаменом по математике?
        \item Куда съездите и как отдохнёте после?
    \end{itemize}
}
\solutionspace{100pt}

\tasknumber{3}%
\task{%
    О существовании какого населённого пункта вы узнали этой весной? Чем он вам запомнился?
}
\solutionspace{60pt}

\tasknumber{4}%
\task{%
    \begin{itemize}
        \item Какой ЕГЭ (кроме математики и русского) вы сдаете?
        \item Сколько времени у вас будет на финальную подготовку и повторение?
        \item Как вы распределите время на экзамене?
        \item Как искать ошибки и проверять себя?
    \end{itemize}
}
\solutionspace{120pt}

\tasknumber{5}%
\task{%
    Выберите и запишите из всего курса физики 2 формулы:
    \begin{itemize}
        \item наиболее запомнившуюсю,
        \item и наименее важную.
    \end{itemize}
}
\solutionspace{120pt}

\tasknumber{6}%
\task{%
    \begin{itemize}
        \item Какие физические открытия или достижения были сделаны на веку вас и ваших родителей?
        \item Как это повлияло на нашу жизнь?
    \end{itemize}
}


\variantsplitter


\addpersonalvariant{Герман Говоров}

\tasknumber{7}%
\task{%
    Назовите 6 знаменитых деятелей и деятельниц физики и области науки, в которых они работали.
    Обеспечьте представительство различных групп.
}
\solutionspace{100pt}

\tasknumber{8}%
\task{%
    Назовите 3 причины для вас лично вернуться в школу
}
\solutionspace{60pt}

\tasknumber{9}%
\task{%
    Расскажите на какой работе вы уже поработали?
    Сколько часов в неделю и сколько недель или месяцев опыта? Норм по деньгам?
}
\solutionspace{120pt}

\tasknumber{10}%
\task{%
    Где бы вы хотели поработать, но ТЕПЕРЬ не получится?
    Какие новые возможности открылись за последние месяцы?
}
\solutionspace{60pt}

\tasknumber{11}%
\task{%
    Что нового вы попробуете в июле-августе?
    Куда надо съездить и где побывать?
    Укажите не менее 3 активностей и 7 мест.
}
\solutionspace{80pt}

\tasknumber{12}%
\task{%
    По какому принципу зарубежные университеты сохраняют партнёрство с российскими?
}
\solutionspace{80pt}

\tasknumber{13}%
\task{%
    Посоветуйте свой (или не свой) Тг/TT/YT/Ig.
    Точным названием или ссылкой, не более двух ответов.
}

\variantsplitter

\addpersonalvariant{София Журавлёва}

\tasknumber{1}%
\task{%
    \begin{itemize}
        \item Как меня зовут?
        \item Как называется предмет?
        \item Какого цвета учебник?
        \item Чем заканчивается анекдот?
    \end{itemize}
}

\tasknumber{2}%
\task{%
    \begin{itemize}
        \item Как проведёте утро перед экзаменом по математике?
        \item Куда съездите и как отдохнёте после?
    \end{itemize}
}
\solutionspace{100pt}

\tasknumber{3}%
\task{%
    О существовании какого населённого пункта вы узнали этой весной? Чем он вам запомнился?
}
\solutionspace{60pt}

\tasknumber{4}%
\task{%
    \begin{itemize}
        \item Какой ЕГЭ (кроме математики и русского) вы сдаете?
        \item Сколько времени у вас будет на финальную подготовку и повторение?
        \item Как вы распределите время на экзамене?
        \item Как искать ошибки и проверять себя?
    \end{itemize}
}
\solutionspace{120pt}

\tasknumber{5}%
\task{%
    Выберите и запишите из всего курса физики 2 формулы:
    \begin{itemize}
        \item наиболее удивительную,
        \item и наименее важную.
    \end{itemize}
}
\solutionspace{120pt}

\tasknumber{6}%
\task{%
    \begin{itemize}
        \item Какие физические открытия или достижения были сделаны на веку вас и ваших родителей?
        \item Как это повлияло на нашу жизнь?
    \end{itemize}
}


\variantsplitter


\addpersonalvariant{София Журавлёва}

\tasknumber{7}%
\task{%
    Назовите 6 знаменитых деятелей и деятельниц физики и области науки, в которых они работали.
    Обеспечьте представительство различных групп.
}
\solutionspace{100pt}

\tasknumber{8}%
\task{%
    Назовите 3 причины для вас лично вернуться в школу
}
\solutionspace{60pt}

\tasknumber{9}%
\task{%
    Расскажите на какой работе вы уже поработали?
    Сколько часов в неделю и сколько недель или месяцев опыта? Норм по деньгам?
}
\solutionspace{120pt}

\tasknumber{10}%
\task{%
    Где бы вы хотели поработать, но ТЕПЕРЬ не получится?
    Какие новые возможности открылись за последние месяцы?
}
\solutionspace{60pt}

\tasknumber{11}%
\task{%
    Что нового вы попробуете в июле-августе?
    Куда надо съездить и где побывать?
    Укажите не менее 3 активностей и 7 мест.
}
\solutionspace{80pt}

\tasknumber{12}%
\task{%
    По какому принципу зарубежные университеты сохраняют партнёрство с российскими?
}
\solutionspace{80pt}

\tasknumber{13}%
\task{%
    Посоветуйте свой (или не свой) Тг/TT/YT/Ig.
    Точным названием или ссылкой, не более двух ответов.
}

\variantsplitter

\addpersonalvariant{Константин Козлов}

\tasknumber{1}%
\task{%
    \begin{itemize}
        \item Как меня зовут?
        \item Как называется предмет?
        \item Какого цвета учебник?
        \item Чем заканчивается анекдот?
    \end{itemize}
}

\tasknumber{2}%
\task{%
    \begin{itemize}
        \item Как проведёте утро перед экзаменом по русскому языку?
        \item Куда съездите и как отдохнёте после?
    \end{itemize}
}
\solutionspace{100pt}

\tasknumber{3}%
\task{%
    О существовании какого населённого пункта вы узнали этой весной? Чем он вам запомнился?
}
\solutionspace{60pt}

\tasknumber{4}%
\task{%
    \begin{itemize}
        \item Какой ЕГЭ (кроме математики и русского) вы сдаете?
        \item Сколько времени у вас будет на финальную подготовку и повторение?
        \item Как вы распределите время на экзамене?
        \item Как искать ошибки и проверять себя?
    \end{itemize}
}
\solutionspace{120pt}

\tasknumber{5}%
\task{%
    Выберите и запишите из всего курса физики 2 формулы:
    \begin{itemize}
        \item наиболее смешную,
        \item и наименее важную.
    \end{itemize}
}
\solutionspace{120pt}

\tasknumber{6}%
\task{%
    \begin{itemize}
        \item Какие физические открытия или достижения были сделаны на веку вас и ваших родителей?
        \item Как это повлияло на нашу жизнь?
    \end{itemize}
}


\variantsplitter


\addpersonalvariant{Константин Козлов}

\tasknumber{7}%
\task{%
    Назовите 6 знаменитых деятелей и деятельниц физики и области науки, в которых они работали.
    Обеспечьте представительство различных групп.
}
\solutionspace{100pt}

\tasknumber{8}%
\task{%
    Назовите 3 причины для вас лично вернуться в школу
}
\solutionspace{60pt}

\tasknumber{9}%
\task{%
    Расскажите на какой работе вы уже поработали?
    Сколько часов в неделю и сколько недель или месяцев опыта? Норм по деньгам?
}
\solutionspace{120pt}

\tasknumber{10}%
\task{%
    Где бы вы хотели поработать, но ТЕПЕРЬ не получится?
    Какие новые возможности открылись за последние месяцы?
}
\solutionspace{60pt}

\tasknumber{11}%
\task{%
    Что нового вы попробуете в июле-августе?
    Куда надо съездить и где побывать?
    Укажите не менее 3 активностей и 7 мест.
}
\solutionspace{80pt}

\tasknumber{12}%
\task{%
    По какому принципу зарубежные университеты сохраняют партнёрство с российскими?
}
\solutionspace{80pt}

\tasknumber{13}%
\task{%
    Посоветуйте свой (или не свой) Тг/TT/YT/Ig.
    Точным названием или ссылкой, не более двух ответов.
}

\variantsplitter

\addpersonalvariant{Наталья Кравченко}

\tasknumber{1}%
\task{%
    \begin{itemize}
        \item Как меня зовут?
        \item Как называется предмет?
        \item Какого цвета учебник?
        \item Чем заканчивается анекдот?
    \end{itemize}
}

\tasknumber{2}%
\task{%
    \begin{itemize}
        \item Как проведёте утро перед экзаменом по русскому языку?
        \item Куда съездите и как отдохнёте после?
    \end{itemize}
}
\solutionspace{100pt}

\tasknumber{3}%
\task{%
    О существовании какого населённого пункта вы узнали этой весной? Чем он вам запомнился?
}
\solutionspace{60pt}

\tasknumber{4}%
\task{%
    \begin{itemize}
        \item Какой ЕГЭ (кроме математики и русского) вы сдаете?
        \item Сколько времени у вас будет на финальную подготовку и повторение?
        \item Как вы распределите время на экзамене?
        \item Как искать ошибки и проверять себя?
    \end{itemize}
}
\solutionspace{120pt}

\tasknumber{5}%
\task{%
    Выберите и запишите из всего курса физики 2 формулы:
    \begin{itemize}
        \item наиболее удивительную,
        \item и наименее важную.
    \end{itemize}
}
\solutionspace{120pt}

\tasknumber{6}%
\task{%
    \begin{itemize}
        \item Какие физические открытия или достижения были сделаны на веку вас и ваших родителей?
        \item Как это повлияло на нашу жизнь?
    \end{itemize}
}


\variantsplitter


\addpersonalvariant{Наталья Кравченко}

\tasknumber{7}%
\task{%
    Назовите 6 знаменитых деятелей и деятельниц физики и области науки, в которых они работали.
    Обеспечьте представительство различных групп.
}
\solutionspace{100pt}

\tasknumber{8}%
\task{%
    Назовите 3 причины для вас лично вернуться в школу
}
\solutionspace{60pt}

\tasknumber{9}%
\task{%
    Расскажите на какой работе вы уже поработали?
    Сколько часов в неделю и сколько недель или месяцев опыта? Норм по деньгам?
}
\solutionspace{120pt}

\tasknumber{10}%
\task{%
    Где бы вы хотели поработать, но ТЕПЕРЬ не получится?
    Какие новые возможности открылись за последние месяцы?
}
\solutionspace{60pt}

\tasknumber{11}%
\task{%
    Что нового вы попробуете в июле-августе?
    Куда надо съездить и где побывать?
    Укажите не менее 3 активностей и 7 мест.
}
\solutionspace{80pt}

\tasknumber{12}%
\task{%
    По какому принципу зарубежные университеты сохраняют партнёрство с российскими?
}
\solutionspace{80pt}

\tasknumber{13}%
\task{%
    Посоветуйте свой (или не свой) Тг/TT/YT/Ig.
    Точным названием или ссылкой, не более двух ответов.
}

\variantsplitter

\addpersonalvariant{Матвей Кузьмин}

\tasknumber{1}%
\task{%
    \begin{itemize}
        \item Как меня зовут?
        \item Как называется предмет?
        \item Какого цвета учебник?
        \item Чем заканчивается анекдот?
    \end{itemize}
}

\tasknumber{2}%
\task{%
    \begin{itemize}
        \item Как проведёте утро перед экзаменом по русскому языку?
        \item Куда съездите и как отдохнёте после?
    \end{itemize}
}
\solutionspace{100pt}

\tasknumber{3}%
\task{%
    О существовании какого населённого пункта вы узнали этой весной? Чем он вам запомнился?
}
\solutionspace{60pt}

\tasknumber{4}%
\task{%
    \begin{itemize}
        \item Какой ЕГЭ (кроме математики и русского) вы сдаете?
        \item Сколько времени у вас будет на финальную подготовку и повторение?
        \item Как вы распределите время на экзамене?
        \item Как искать ошибки и проверять себя?
    \end{itemize}
}
\solutionspace{120pt}

\tasknumber{5}%
\task{%
    Выберите и запишите из всего курса физики 2 формулы:
    \begin{itemize}
        \item наиболее полезную,
        \item и наименее интересную.
    \end{itemize}
}
\solutionspace{120pt}

\tasknumber{6}%
\task{%
    \begin{itemize}
        \item Какие физические открытия или достижения были сделаны на веку вас и ваших родителей?
        \item Как это повлияло на нашу жизнь?
    \end{itemize}
}


\variantsplitter


\addpersonalvariant{Матвей Кузьмин}

\tasknumber{7}%
\task{%
    Назовите 6 знаменитых деятелей и деятельниц физики и области науки, в которых они работали.
    Обеспечьте представительство различных групп.
}
\solutionspace{100pt}

\tasknumber{8}%
\task{%
    Назовите 3 причины для вас лично вернуться в школу
}
\solutionspace{60pt}

\tasknumber{9}%
\task{%
    Расскажите на какой работе вы уже поработали?
    Сколько часов в неделю и сколько недель или месяцев опыта? Норм по деньгам?
}
\solutionspace{120pt}

\tasknumber{10}%
\task{%
    Где бы вы хотели поработать, но ТЕПЕРЬ не получится?
    Какие новые возможности открылись за последние месяцы?
}
\solutionspace{60pt}

\tasknumber{11}%
\task{%
    Что нового вы попробуете в июле-августе?
    Куда надо съездить и где побывать?
    Укажите не менее 3 активностей и 7 мест.
}
\solutionspace{80pt}

\tasknumber{12}%
\task{%
    По какому принципу зарубежные университеты сохраняют партнёрство с российскими?
}
\solutionspace{80pt}

\tasknumber{13}%
\task{%
    Посоветуйте свой (или не свой) Тг/TT/YT/Ig.
    Точным названием или ссылкой, не более двух ответов.
}

\variantsplitter

\addpersonalvariant{Сергей Малышев}

\tasknumber{1}%
\task{%
    \begin{itemize}
        \item Как меня зовут?
        \item Как называется предмет?
        \item Какого цвета учебник?
        \item Чем заканчивается анекдот?
    \end{itemize}
}

\tasknumber{2}%
\task{%
    \begin{itemize}
        \item Как проведёте вечер перед экзаменом по русскому языку?
        \item Куда съездите и как отдохнёте после?
    \end{itemize}
}
\solutionspace{100pt}

\tasknumber{3}%
\task{%
    О существовании какого населённого пункта вы узнали этой весной? Чем он вам запомнился?
}
\solutionspace{60pt}

\tasknumber{4}%
\task{%
    \begin{itemize}
        \item Какой ЕГЭ (кроме математики и русского) вы сдаете?
        \item Сколько времени у вас будет на финальную подготовку и повторение?
        \item Как вы распределите время на экзамене?
        \item Как искать ошибки и проверять себя?
    \end{itemize}
}
\solutionspace{120pt}

\tasknumber{5}%
\task{%
    Выберите и запишите из всего курса физики 2 формулы:
    \begin{itemize}
        \item наиболее удивительную,
        \item и наименее важную.
    \end{itemize}
}
\solutionspace{120pt}

\tasknumber{6}%
\task{%
    \begin{itemize}
        \item Какие физические открытия или достижения были сделаны на веку вас и ваших родителей?
        \item Как это повлияло на нашу жизнь?
    \end{itemize}
}


\variantsplitter


\addpersonalvariant{Сергей Малышев}

\tasknumber{7}%
\task{%
    Назовите 6 знаменитых деятелей и деятельниц физики и области науки, в которых они работали.
    Обеспечьте представительство различных групп.
}
\solutionspace{100pt}

\tasknumber{8}%
\task{%
    Назовите 3 причины для вас лично вернуться в школу
}
\solutionspace{60pt}

\tasknumber{9}%
\task{%
    Расскажите на какой работе вы уже поработали?
    Сколько часов в неделю и сколько недель или месяцев опыта? Норм по деньгам?
}
\solutionspace{120pt}

\tasknumber{10}%
\task{%
    Где бы вы хотели поработать, но ТЕПЕРЬ не получится?
    Какие новые возможности открылись за последние месяцы?
}
\solutionspace{60pt}

\tasknumber{11}%
\task{%
    Что нового вы попробуете в июле-августе?
    Куда надо съездить и где побывать?
    Укажите не менее 3 активностей и 7 мест.
}
\solutionspace{80pt}

\tasknumber{12}%
\task{%
    По какому принципу зарубежные университеты сохраняют партнёрство с российскими?
}
\solutionspace{80pt}

\tasknumber{13}%
\task{%
    Посоветуйте свой (или не свой) Тг/TT/YT/Ig.
    Точным названием или ссылкой, не более двух ответов.
}

\variantsplitter

\addpersonalvariant{Алина Полканова}

\tasknumber{1}%
\task{%
    \begin{itemize}
        \item Как меня зовут?
        \item Как называется предмет?
        \item Какого цвета учебник?
        \item Чем заканчивается анекдот?
    \end{itemize}
}

\tasknumber{2}%
\task{%
    \begin{itemize}
        \item Как проведёте вечер перед экзаменом по русскому языку?
        \item Куда съездите и как отдохнёте после?
    \end{itemize}
}
\solutionspace{100pt}

\tasknumber{3}%
\task{%
    О существовании какого населённого пункта вы узнали этой весной? Чем он вам запомнился?
}
\solutionspace{60pt}

\tasknumber{4}%
\task{%
    \begin{itemize}
        \item Какой ЕГЭ (кроме математики и русского) вы сдаете?
        \item Сколько времени у вас будет на финальную подготовку и повторение?
        \item Как вы распределите время на экзамене?
        \item Как искать ошибки и проверять себя?
    \end{itemize}
}
\solutionspace{120pt}

\tasknumber{5}%
\task{%
    Выберите и запишите из всего курса физики 2 формулы:
    \begin{itemize}
        \item наиболее запомнившуюсю,
        \item и наименее важную.
    \end{itemize}
}
\solutionspace{120pt}

\tasknumber{6}%
\task{%
    \begin{itemize}
        \item Какие физические открытия или достижения были сделаны на веку вас и ваших родителей?
        \item Как это повлияло на нашу жизнь?
    \end{itemize}
}


\variantsplitter


\addpersonalvariant{Алина Полканова}

\tasknumber{7}%
\task{%
    Назовите 6 знаменитых деятелей и деятельниц физики и области науки, в которых они работали.
    Обеспечьте представительство различных групп.
}
\solutionspace{100pt}

\tasknumber{8}%
\task{%
    Назовите 3 причины для вас лично вернуться в школу
}
\solutionspace{60pt}

\tasknumber{9}%
\task{%
    Расскажите на какой работе вы уже поработали?
    Сколько часов в неделю и сколько недель или месяцев опыта? Норм по деньгам?
}
\solutionspace{120pt}

\tasknumber{10}%
\task{%
    Где бы вы хотели поработать, но ТЕПЕРЬ не получится?
    Какие новые возможности открылись за последние месяцы?
}
\solutionspace{60pt}

\tasknumber{11}%
\task{%
    Что нового вы попробуете в июле-августе?
    Куда надо съездить и где побывать?
    Укажите не менее 3 активностей и 7 мест.
}
\solutionspace{80pt}

\tasknumber{12}%
\task{%
    По какому принципу зарубежные университеты сохраняют партнёрство с российскими?
}
\solutionspace{80pt}

\tasknumber{13}%
\task{%
    Посоветуйте свой (или не свой) Тг/TT/YT/Ig.
    Точным названием или ссылкой, не более двух ответов.
}

\variantsplitter

\addpersonalvariant{Сергей Пономарёв}

\tasknumber{1}%
\task{%
    \begin{itemize}
        \item Как меня зовут?
        \item Как называется предмет?
        \item Какого цвета учебник?
        \item Чем заканчивается анекдот?
    \end{itemize}
}

\tasknumber{2}%
\task{%
    \begin{itemize}
        \item Как проведёте вечер перед экзаменом по математике?
        \item Куда съездите и как отдохнёте после?
    \end{itemize}
}
\solutionspace{100pt}

\tasknumber{3}%
\task{%
    О существовании какого населённого пункта вы узнали этой весной? Чем он вам запомнился?
}
\solutionspace{60pt}

\tasknumber{4}%
\task{%
    \begin{itemize}
        \item Какой ЕГЭ (кроме математики и русского) вы сдаете?
        \item Сколько времени у вас будет на финальную подготовку и повторение?
        \item Как вы распределите время на экзамене?
        \item Как искать ошибки и проверять себя?
    \end{itemize}
}
\solutionspace{120pt}

\tasknumber{5}%
\task{%
    Выберите и запишите из всего курса физики 2 формулы:
    \begin{itemize}
        \item наиболее удивительную,
        \item и наименее интересную.
    \end{itemize}
}
\solutionspace{120pt}

\tasknumber{6}%
\task{%
    \begin{itemize}
        \item Какие физические открытия или достижения были сделаны на веку вас и ваших родителей?
        \item Как это повлияло на нашу жизнь?
    \end{itemize}
}


\variantsplitter


\addpersonalvariant{Сергей Пономарёв}

\tasknumber{7}%
\task{%
    Назовите 6 знаменитых деятелей и деятельниц физики и области науки, в которых они работали.
    Обеспечьте представительство различных групп.
}
\solutionspace{100pt}

\tasknumber{8}%
\task{%
    Назовите 3 причины для вас лично вернуться в школу
}
\solutionspace{60pt}

\tasknumber{9}%
\task{%
    Расскажите на какой работе вы уже поработали?
    Сколько часов в неделю и сколько недель или месяцев опыта? Норм по деньгам?
}
\solutionspace{120pt}

\tasknumber{10}%
\task{%
    Где бы вы хотели поработать, но ТЕПЕРЬ не получится?
    Какие новые возможности открылись за последние месяцы?
}
\solutionspace{60pt}

\tasknumber{11}%
\task{%
    Что нового вы попробуете в июле-августе?
    Куда надо съездить и где побывать?
    Укажите не менее 3 активностей и 7 мест.
}
\solutionspace{80pt}

\tasknumber{12}%
\task{%
    По какому принципу зарубежные университеты сохраняют партнёрство с российскими?
}
\solutionspace{80pt}

\tasknumber{13}%
\task{%
    Посоветуйте свой (или не свой) Тг/TT/YT/Ig.
    Точным названием или ссылкой, не более двух ответов.
}

\variantsplitter

\addpersonalvariant{Егор Свистушкин}

\tasknumber{1}%
\task{%
    \begin{itemize}
        \item Как меня зовут?
        \item Как называется предмет?
        \item Какого цвета учебник?
        \item Чем заканчивается анекдот?
    \end{itemize}
}

\tasknumber{2}%
\task{%
    \begin{itemize}
        \item Как проведёте вечер перед экзаменом по русскому языку?
        \item Куда съездите и как отдохнёте после?
    \end{itemize}
}
\solutionspace{100pt}

\tasknumber{3}%
\task{%
    О существовании какого населённого пункта вы узнали этой весной? Чем он вам запомнился?
}
\solutionspace{60pt}

\tasknumber{4}%
\task{%
    \begin{itemize}
        \item Какой ЕГЭ (кроме математики и русского) вы сдаете?
        \item Сколько времени у вас будет на финальную подготовку и повторение?
        \item Как вы распределите время на экзамене?
        \item Как искать ошибки и проверять себя?
    \end{itemize}
}
\solutionspace{120pt}

\tasknumber{5}%
\task{%
    Выберите и запишите из всего курса физики 2 формулы:
    \begin{itemize}
        \item наиболее запомнившуюсю,
        \item и наименее важную.
    \end{itemize}
}
\solutionspace{120pt}

\tasknumber{6}%
\task{%
    \begin{itemize}
        \item Какие физические открытия или достижения были сделаны на веку вас и ваших родителей?
        \item Как это повлияло на нашу жизнь?
    \end{itemize}
}


\variantsplitter


\addpersonalvariant{Егор Свистушкин}

\tasknumber{7}%
\task{%
    Назовите 6 знаменитых деятелей и деятельниц физики и области науки, в которых они работали.
    Обеспечьте представительство различных групп.
}
\solutionspace{100pt}

\tasknumber{8}%
\task{%
    Назовите 3 причины для вас лично вернуться в школу
}
\solutionspace{60pt}

\tasknumber{9}%
\task{%
    Расскажите на какой работе вы уже поработали?
    Сколько часов в неделю и сколько недель или месяцев опыта? Норм по деньгам?
}
\solutionspace{120pt}

\tasknumber{10}%
\task{%
    Где бы вы хотели поработать, но ТЕПЕРЬ не получится?
    Какие новые возможности открылись за последние месяцы?
}
\solutionspace{60pt}

\tasknumber{11}%
\task{%
    Что нового вы попробуете в июле-августе?
    Куда надо съездить и где побывать?
    Укажите не менее 3 активностей и 7 мест.
}
\solutionspace{80pt}

\tasknumber{12}%
\task{%
    По какому принципу зарубежные университеты сохраняют партнёрство с российскими?
}
\solutionspace{80pt}

\tasknumber{13}%
\task{%
    Посоветуйте свой (или не свой) Тг/TT/YT/Ig.
    Точным названием или ссылкой, не более двух ответов.
}

\variantsplitter

\addpersonalvariant{Дмитрий Соколов}

\tasknumber{1}%
\task{%
    \begin{itemize}
        \item Как меня зовут?
        \item Как называется предмет?
        \item Какого цвета учебник?
        \item Чем заканчивается анекдот?
    \end{itemize}
}

\tasknumber{2}%
\task{%
    \begin{itemize}
        \item Как проведёте вечер перед экзаменом по русскому языку?
        \item Куда съездите и как отдохнёте после?
    \end{itemize}
}
\solutionspace{100pt}

\tasknumber{3}%
\task{%
    О существовании какого населённого пункта вы узнали этой весной? Чем он вам запомнился?
}
\solutionspace{60pt}

\tasknumber{4}%
\task{%
    \begin{itemize}
        \item Какой ЕГЭ (кроме математики и русского) вы сдаете?
        \item Сколько времени у вас будет на финальную подготовку и повторение?
        \item Как вы распределите время на экзамене?
        \item Как искать ошибки и проверять себя?
    \end{itemize}
}
\solutionspace{120pt}

\tasknumber{5}%
\task{%
    Выберите и запишите из всего курса физики 2 формулы:
    \begin{itemize}
        \item наиболее запомнившуюсю,
        \item и наименее важную.
    \end{itemize}
}
\solutionspace{120pt}

\tasknumber{6}%
\task{%
    \begin{itemize}
        \item Какие физические открытия или достижения были сделаны на веку вас и ваших родителей?
        \item Как это повлияло на нашу жизнь?
    \end{itemize}
}


\variantsplitter


\addpersonalvariant{Дмитрий Соколов}

\tasknumber{7}%
\task{%
    Назовите 6 знаменитых деятелей и деятельниц физики и области науки, в которых они работали.
    Обеспечьте представительство различных групп.
}
\solutionspace{100pt}

\tasknumber{8}%
\task{%
    Назовите 3 причины для вас лично вернуться в школу
}
\solutionspace{60pt}

\tasknumber{9}%
\task{%
    Расскажите на какой работе вы уже поработали?
    Сколько часов в неделю и сколько недель или месяцев опыта? Норм по деньгам?
}
\solutionspace{120pt}

\tasknumber{10}%
\task{%
    Где бы вы хотели поработать, но ТЕПЕРЬ не получится?
    Какие новые возможности открылись за последние месяцы?
}
\solutionspace{60pt}

\tasknumber{11}%
\task{%
    Что нового вы попробуете в июле-августе?
    Куда надо съездить и где побывать?
    Укажите не менее 3 активностей и 7 мест.
}
\solutionspace{80pt}

\tasknumber{12}%
\task{%
    По какому принципу зарубежные университеты сохраняют партнёрство с российскими?
}
\solutionspace{80pt}

\tasknumber{13}%
\task{%
    Посоветуйте свой (или не свой) Тг/TT/YT/Ig.
    Точным названием или ссылкой, не более двух ответов.
}

\variantsplitter

\addpersonalvariant{Арсений Трофимов}

\tasknumber{1}%
\task{%
    \begin{itemize}
        \item Как меня зовут?
        \item Как называется предмет?
        \item Какого цвета учебник?
        \item Чем заканчивается анекдот?
    \end{itemize}
}

\tasknumber{2}%
\task{%
    \begin{itemize}
        \item Как проведёте вечер перед экзаменом по русскому языку?
        \item Куда съездите и как отдохнёте после?
    \end{itemize}
}
\solutionspace{100pt}

\tasknumber{3}%
\task{%
    О существовании какого населённого пункта вы узнали этой весной? Чем он вам запомнился?
}
\solutionspace{60pt}

\tasknumber{4}%
\task{%
    \begin{itemize}
        \item Какой ЕГЭ (кроме математики и русского) вы сдаете?
        \item Сколько времени у вас будет на финальную подготовку и повторение?
        \item Как вы распределите время на экзамене?
        \item Как искать ошибки и проверять себя?
    \end{itemize}
}
\solutionspace{120pt}

\tasknumber{5}%
\task{%
    Выберите и запишите из всего курса физики 2 формулы:
    \begin{itemize}
        \item наиболее удивительную,
        \item и наименее интересную.
    \end{itemize}
}
\solutionspace{120pt}

\tasknumber{6}%
\task{%
    \begin{itemize}
        \item Какие физические открытия или достижения были сделаны на веку вас и ваших родителей?
        \item Как это повлияло на нашу жизнь?
    \end{itemize}
}


\variantsplitter


\addpersonalvariant{Арсений Трофимов}

\tasknumber{7}%
\task{%
    Назовите 6 знаменитых деятелей и деятельниц физики и области науки, в которых они работали.
    Обеспечьте представительство различных групп.
}
\solutionspace{100pt}

\tasknumber{8}%
\task{%
    Назовите 3 причины для вас лично вернуться в школу
}
\solutionspace{60pt}

\tasknumber{9}%
\task{%
    Расскажите на какой работе вы уже поработали?
    Сколько часов в неделю и сколько недель или месяцев опыта? Норм по деньгам?
}
\solutionspace{120pt}

\tasknumber{10}%
\task{%
    Где бы вы хотели поработать, но ТЕПЕРЬ не получится?
    Какие новые возможности открылись за последние месяцы?
}
\solutionspace{60pt}

\tasknumber{11}%
\task{%
    Что нового вы попробуете в июле-августе?
    Куда надо съездить и где побывать?
    Укажите не менее 3 активностей и 7 мест.
}
\solutionspace{80pt}

\tasknumber{12}%
\task{%
    По какому принципу зарубежные университеты сохраняют партнёрство с российскими?
}
\solutionspace{80pt}

\tasknumber{13}%
\task{%
    Посоветуйте свой (или не свой) Тг/TT/YT/Ig.
    Точным названием или ссылкой, не более двух ответов.
}
% autogenerated
