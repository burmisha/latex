\setdate{20~октября~2021}
\setclass{11«Б»}

\addpersonalvariant{Михаил Бурмистров}

\tasknumber{1}%
\task{%
    Тело колеблется по гармоническому закону,
    амплитуда этих колебаний $18\,\text{см}$, период $6\,\text{c}$.
    Чему равно смещение тела относительно положения равновесия через $24\,\text{c}$
    после прохождения положения максимального отклонения?
}
\answer{%
    $x = A \cos \omega t = A \cos \cbr{ \frac {2\pi}T t } = A \cos \cbr{ 2\pi \frac tT } = 18\,\text{см} \cdot \cos \cbr{ 2\pi \cdot \frac {24\,\text{c}}{6\,\text{c}}}\approx 18\,\text{см}.$
}
\solutionspace{80pt}

\tasknumber{2}%
\task{%
    Запишите формулу для периода колебаний пружинного маятника и ...
    \begin{itemize}
        \item укажите названия всех физических величин в формуле,
        \item выразите из неё частоту колебаний
        \item выразите из неё жёсткость пружины.
    \end{itemize}
}
\answer{%
    \begin{align*}
    T &= 2\pi \sqrt{\frac lg} \implies \nu = \frac 1T = \frac 1{2\pi}\sqrt{\frac gl}, \omega = 2\pi\nu = \sqrt{\frac gl}, l = g\sqr{\frac T{2\pi}}, g = l\sqr{\frac {2\pi}T} \\
    T &= 2\pi \sqrt{\frac mk} \implies \nu = \frac 1T = \frac 1{2\pi}\sqrt{\frac km}, \omega = 2\pi\nu = \sqrt{\frac km}, m = k\sqr{\frac T{2\pi}}, k = m\sqr{\frac {2\pi}T}
    \end{align*}
}
\solutionspace{80pt}

\tasknumber{3}%
\task{%
    Период колебаний математического маятника равен $4\,\text{с}$,
    а их амплитуда — $10\,\text{см}$.
    Определите амплитуду колебаний скорости маятника.
}
\answer{%
    $
        T = \frac{2\pi}{\omega}
        \implies \omega = \frac{2\pi}{T}
        \implies v_{\max} = \omega A = \frac{2\pi}{T}A
        \approx 15{,}7\,\frac{\text{см}}{\text{с}}.
    $
}
\solutionspace{100pt}

\tasknumber{4}%
\task{%
    Математический маятник с нитью длиной $40\,\text{см}$ подвешен к потолку в лифте.
    За $30\,\text{с}$ маятник совершил $21$ колебаний.
    Определите модуль и направление ускорения лифта.
    Куда движется лифт?
}
\answer{%
    $
        T = 2\pi\sqrt{\frac\ell {a + g}}, T = \frac {t}{N}
        \implies a + g = \ell \cdot \frac{4 \pi ^ 2}{T^2},
        a = \ell \cdot \frac{4 \pi ^ 2}{T^2} - g = \ell \cdot \frac{4 \pi ^ 2 N^2}{t^2} - g \approx -2{,}26\,\frac{\text{м}}{\text{c}^{2}},
        \text{вверх}.
    $
}
\solutionspace{120pt}

\tasknumber{5}%
\task{%
    Масса груза в пружинном маятнике равна $500\,\text{г}$, при этом период его колебаний равен $1{,}2\,\text{с}$.
    Груз утяжеляют на $50\,\text{г}$.
    Определите новый период колебаний маятника.
}
\answer{%
    $
        T'
            = 2\pi\sqrt{\frac{M + m}{k}}
            = 2\pi\sqrt{\frac{M}{k} \cdot \frac{M + m}{M}}
            = T\sqrt{\frac{M + m}{M}} =  T\sqrt{1 + \frac{m}{M}} \approx 1{,}26\,\text{с}.
    $
}
\solutionspace{120pt}

\tasknumber{6}%
\task{%
    При какой длине нити математического маятника период колебаний груза массой $400\,\text{г}$
    окажется равен периоду колебаний этого же груза в пружинном маятнике с пружиной жёсткостью $50\,\frac{\text{Н}}{\text{м}}$?
}
\answer{%
    $
        2\pi \sqrt{\frac \ell g} = 2\pi \sqrt{\frac m k}
        \implies \frac \ell g = \frac m k
        \implies \ell = g \frac m k \approx 8\,\text{см}.
    $
}

\variantsplitter

\addpersonalvariant{Снежана Авдошина}

\tasknumber{1}%
\task{%
    Тело колеблется по гармоническому закону,
    амплитуда этих колебаний $20\,\text{см}$, период $2\,\text{c}$.
    Чему равно смещение тела относительно положения равновесия через $20\,\text{c}$
    после прохождения положения максимального отклонения?
}
\answer{%
    $x = A \cos \omega t = A \cos \cbr{ \frac {2\pi}T t } = A \cos \cbr{ 2\pi \frac tT } = 20\,\text{см} \cdot \cos \cbr{ 2\pi \cdot \frac {20\,\text{c}}{2\,\text{c}}}\approx 20\,\text{см}.$
}
\solutionspace{80pt}

\tasknumber{2}%
\task{%
    Запишите формулу для периода колебаний пружинного маятника и ...
    \begin{itemize}
        \item укажите названия всех физических величин в формуле,
        \item выразите из неё частоту колебаний
        \item выразите из неё массу груза.
    \end{itemize}
}
\answer{%
    \begin{align*}
    T &= 2\pi \sqrt{\frac lg} \implies \nu = \frac 1T = \frac 1{2\pi}\sqrt{\frac gl}, \omega = 2\pi\nu = \sqrt{\frac gl}, l = g\sqr{\frac T{2\pi}}, g = l\sqr{\frac {2\pi}T} \\
    T &= 2\pi \sqrt{\frac mk} \implies \nu = \frac 1T = \frac 1{2\pi}\sqrt{\frac km}, \omega = 2\pi\nu = \sqrt{\frac km}, m = k\sqr{\frac T{2\pi}}, k = m\sqr{\frac {2\pi}T}
    \end{align*}
}
\solutionspace{80pt}

\tasknumber{3}%
\task{%
    Период колебаний математического маятника равен $3\,\text{с}$,
    а их амплитуда — $20\,\text{см}$.
    Определите амплитуду колебаний скорости маятника.
}
\answer{%
    $
        T = \frac{2\pi}{\omega}
        \implies \omega = \frac{2\pi}{T}
        \implies v_{\max} = \omega A = \frac{2\pi}{T}A
        \approx 41{,}9\,\frac{\text{см}}{\text{с}}.
    $
}
\solutionspace{100pt}

\tasknumber{4}%
\task{%
    Математический маятник с нитью длиной $40\,\text{см}$ подвешен к потолку в лифте.
    За $20\,\text{с}$ маятник совершил $18$ колебаний.
    Определите модуль и направление ускорения лифта.
    Куда движется лифт?
}
\answer{%
    $
        T = 2\pi\sqrt{\frac\ell {a + g}}, T = \frac {t}{N}
        \implies a + g = \ell \cdot \frac{4 \pi ^ 2}{T^2},
        a = \ell \cdot \frac{4 \pi ^ 2}{T^2} - g = \ell \cdot \frac{4 \pi ^ 2 N^2}{t^2} - g \approx 2{,}79\,\frac{\text{м}}{\text{c}^{2}},
        \text{вниз}.
    $
}
\solutionspace{120pt}

\tasknumber{5}%
\task{%
    Масса груза в пружинном маятнике равна $400\,\text{г}$, при этом период его колебаний равен $1{,}4\,\text{с}$.
    Груз утяжеляют на $150\,\text{г}$.
    Определите новый период колебаний маятника.
}
\answer{%
    $
        T'
            = 2\pi\sqrt{\frac{M + m}{k}}
            = 2\pi\sqrt{\frac{M}{k} \cdot \frac{M + m}{M}}
            = T\sqrt{\frac{M + m}{M}} =  T\sqrt{1 + \frac{m}{M}} \approx 1{,}64\,\text{с}.
    $
}
\solutionspace{120pt}

\tasknumber{6}%
\task{%
    При какой длине нити математического маятника период колебаний груза массой $300\,\text{г}$
    окажется равен периоду колебаний этого же груза в пружинном маятнике с пружиной жёсткостью $50\,\frac{\text{Н}}{\text{м}}$?
}
\answer{%
    $
        2\pi \sqrt{\frac \ell g} = 2\pi \sqrt{\frac m k}
        \implies \frac \ell g = \frac m k
        \implies \ell = g \frac m k \approx 6\,\text{см}.
    $
}

\variantsplitter

\addpersonalvariant{Марьяна Аристова}

\tasknumber{1}%
\task{%
    Тело колеблется по гармоническому закону,
    амплитуда этих колебаний $10\,\text{см}$, период $2\,\text{c}$.
    Чему равно смещение тела относительно положения равновесия через $22\,\text{c}$
    после прохождения положения равновесия?
}
\answer{%
    $x = A \sin \omega t = A \sin \cbr{ \frac {2\pi}T t } = A \sin \cbr{ 2\pi \frac tT } = 10\,\text{см} \cdot \sin \cbr{ 2\pi \cdot \frac {22\,\text{c}}{2\,\text{c}}}\approx 0\,\text{см}.$
}
\solutionspace{80pt}

\tasknumber{2}%
\task{%
    Запишите формулу для периода колебаний пружинного маятника и ...
    \begin{itemize}
        \item укажите названия всех физических величин в формуле,
        \item выразите из неё частоту колебаний
        \item выразите из неё массу груза.
    \end{itemize}
}
\answer{%
    \begin{align*}
    T &= 2\pi \sqrt{\frac lg} \implies \nu = \frac 1T = \frac 1{2\pi}\sqrt{\frac gl}, \omega = 2\pi\nu = \sqrt{\frac gl}, l = g\sqr{\frac T{2\pi}}, g = l\sqr{\frac {2\pi}T} \\
    T &= 2\pi \sqrt{\frac mk} \implies \nu = \frac 1T = \frac 1{2\pi}\sqrt{\frac km}, \omega = 2\pi\nu = \sqrt{\frac km}, m = k\sqr{\frac T{2\pi}}, k = m\sqr{\frac {2\pi}T}
    \end{align*}
}
\solutionspace{80pt}

\tasknumber{3}%
\task{%
    Период колебаний математического маятника равен $4\,\text{с}$,
    а их амплитуда — $20\,\text{см}$.
    Определите максимальную скорость маятника.
}
\answer{%
    $
        T = \frac{2\pi}{\omega}
        \implies \omega = \frac{2\pi}{T}
        \implies v_{\max} = \omega A = \frac{2\pi}{T}A
        \approx 31{,}4\,\frac{\text{см}}{\text{с}}.
    $
}
\solutionspace{100pt}

\tasknumber{4}%
\task{%
    Математический маятник с нитью длиной $40\,\text{см}$ подвешен к потолку в лифте.
    За $20\,\text{с}$ маятник совершил $14$ колебаний.
    Определите модуль и направление ускорения лифта.
    Куда движется лифт?
}
\answer{%
    $
        T = 2\pi\sqrt{\frac\ell {a + g}}, T = \frac {t}{N}
        \implies a + g = \ell \cdot \frac{4 \pi ^ 2}{T^2},
        a = \ell \cdot \frac{4 \pi ^ 2}{T^2} - g = \ell \cdot \frac{4 \pi ^ 2 N^2}{t^2} - g \approx -2{,}26\,\frac{\text{м}}{\text{c}^{2}},
        \text{вверх}.
    $
}
\solutionspace{120pt}

\tasknumber{5}%
\task{%
    Масса груза в пружинном маятнике равна $500\,\text{г}$, при этом период его колебаний равен $1{,}4\,\text{с}$.
    Груз утяжеляют на $150\,\text{г}$.
    Определите новый период колебаний маятника.
}
\answer{%
    $
        T'
            = 2\pi\sqrt{\frac{M + m}{k}}
            = 2\pi\sqrt{\frac{M}{k} \cdot \frac{M + m}{M}}
            = T\sqrt{\frac{M + m}{M}} =  T\sqrt{1 + \frac{m}{M}} \approx 1{,}60\,\text{с}.
    $
}
\solutionspace{120pt}

\tasknumber{6}%
\task{%
    При какой длине нити математического маятника период колебаний груза массой $200\,\text{г}$
    окажется равен периоду колебаний этого же груза в пружинном маятнике с пружиной жёсткостью $50\,\frac{\text{Н}}{\text{м}}$?
}
\answer{%
    $
        2\pi \sqrt{\frac \ell g} = 2\pi \sqrt{\frac m k}
        \implies \frac \ell g = \frac m k
        \implies \ell = g \frac m k \approx 4\,\text{см}.
    $
}

\variantsplitter

\addpersonalvariant{Никита Иванов}

\tasknumber{1}%
\task{%
    Тело колеблется по гармоническому закону,
    амплитуда этих колебаний $14\,\text{см}$, период $6\,\text{c}$.
    Чему равно смещение тела относительно положения равновесия через $25\,\text{c}$
    после прохождения положения равновесия?
}
\answer{%
    $x = A \sin \omega t = A \sin \cbr{ \frac {2\pi}T t } = A \sin \cbr{ 2\pi \frac tT } = 14\,\text{см} \cdot \sin \cbr{ 2\pi \cdot \frac {25\,\text{c}}{6\,\text{c}}}\approx 12{,}1\,\text{см}.$
}
\solutionspace{80pt}

\tasknumber{2}%
\task{%
    Запишите формулу для периода колебаний пружинного маятника и ...
    \begin{itemize}
        \item укажите названия всех физических величин в формуле,
        \item выразите из неё частоту колебаний
        \item выразите из неё жёсткость пружины.
    \end{itemize}
}
\answer{%
    \begin{align*}
    T &= 2\pi \sqrt{\frac lg} \implies \nu = \frac 1T = \frac 1{2\pi}\sqrt{\frac gl}, \omega = 2\pi\nu = \sqrt{\frac gl}, l = g\sqr{\frac T{2\pi}}, g = l\sqr{\frac {2\pi}T} \\
    T &= 2\pi \sqrt{\frac mk} \implies \nu = \frac 1T = \frac 1{2\pi}\sqrt{\frac km}, \omega = 2\pi\nu = \sqrt{\frac km}, m = k\sqr{\frac T{2\pi}}, k = m\sqr{\frac {2\pi}T}
    \end{align*}
}
\solutionspace{80pt}

\tasknumber{3}%
\task{%
    Период колебаний математического маятника равен $4\,\text{с}$,
    а их амплитуда — $20\,\text{см}$.
    Определите максимальную скорость маятника.
}
\answer{%
    $
        T = \frac{2\pi}{\omega}
        \implies \omega = \frac{2\pi}{T}
        \implies v_{\max} = \omega A = \frac{2\pi}{T}A
        \approx 31{,}4\,\frac{\text{см}}{\text{с}}.
    $
}
\solutionspace{100pt}

\tasknumber{4}%
\task{%
    Математический маятник с нитью длиной $37\,\text{см}$ подвешен к потолку в лифте.
    За $30\,\text{с}$ маятник совершил $23$ колебаний.
    Определите модуль и направление ускорения лифта.
    Куда движется лифт?
}
\answer{%
    $
        T = 2\pi\sqrt{\frac\ell {a + g}}, T = \frac {t}{N}
        \implies a + g = \ell \cdot \frac{4 \pi ^ 2}{T^2},
        a = \ell \cdot \frac{4 \pi ^ 2}{T^2} - g = \ell \cdot \frac{4 \pi ^ 2 N^2}{t^2} - g \approx -1{,}41\,\frac{\text{м}}{\text{c}^{2}},
        \text{вверх}.
    $
}
\solutionspace{120pt}

\tasknumber{5}%
\task{%
    Масса груза в пружинном маятнике равна $500\,\text{г}$, при этом период его колебаний равен $1{,}4\,\text{с}$.
    Груз утяжеляют на $100\,\text{г}$.
    Определите новый период колебаний маятника.
}
\answer{%
    $
        T'
            = 2\pi\sqrt{\frac{M + m}{k}}
            = 2\pi\sqrt{\frac{M}{k} \cdot \frac{M + m}{M}}
            = T\sqrt{\frac{M + m}{M}} =  T\sqrt{1 + \frac{m}{M}} \approx 1{,}53\,\text{с}.
    $
}
\solutionspace{120pt}

\tasknumber{6}%
\task{%
    При какой длине нити математического маятника период колебаний груза массой $400\,\text{г}$
    окажется равен периоду колебаний этого же груза в пружинном маятнике с пружиной жёсткостью $60\,\frac{\text{Н}}{\text{м}}$?
}
\answer{%
    $
        2\pi \sqrt{\frac \ell g} = 2\pi \sqrt{\frac m k}
        \implies \frac \ell g = \frac m k
        \implies \ell = g \frac m k \approx 6{,}7\,\text{см}.
    $
}

\variantsplitter

\addpersonalvariant{Анастасия Князева}

\tasknumber{1}%
\task{%
    Тело колеблется по гармоническому закону,
    амплитуда этих колебаний $20\,\text{см}$, период $6\,\text{c}$.
    Чему равно смещение тела относительно положения равновесия через $23\,\text{c}$
    после прохождения положения равновесия?
}
\answer{%
    $x = A \sin \omega t = A \sin \cbr{ \frac {2\pi}T t } = A \sin \cbr{ 2\pi \frac tT } = 20\,\text{см} \cdot \sin \cbr{ 2\pi \cdot \frac {23\,\text{c}}{6\,\text{c}}}\approx -17{,}3\,\text{см}.$
}
\solutionspace{80pt}

\tasknumber{2}%
\task{%
    Запишите формулу для периода колебаний математического маятника и ...
    \begin{itemize}
        \item укажите названия всех физических величин в формуле,
        \item выразите из неё циклическую частоту колебаний
        \item выразите из неё ускорение свободного падения.
    \end{itemize}
}
\answer{%
    \begin{align*}
    T &= 2\pi \sqrt{\frac lg} \implies \nu = \frac 1T = \frac 1{2\pi}\sqrt{\frac gl}, \omega = 2\pi\nu = \sqrt{\frac gl}, l = g\sqr{\frac T{2\pi}}, g = l\sqr{\frac {2\pi}T} \\
    T &= 2\pi \sqrt{\frac mk} \implies \nu = \frac 1T = \frac 1{2\pi}\sqrt{\frac km}, \omega = 2\pi\nu = \sqrt{\frac km}, m = k\sqr{\frac T{2\pi}}, k = m\sqr{\frac {2\pi}T}
    \end{align*}
}
\solutionspace{80pt}

\tasknumber{3}%
\task{%
    Период колебаний математического маятника равен $2\,\text{с}$,
    а их амплитуда — $10\,\text{см}$.
    Определите максимальную скорость маятника.
}
\answer{%
    $
        T = \frac{2\pi}{\omega}
        \implies \omega = \frac{2\pi}{T}
        \implies v_{\max} = \omega A = \frac{2\pi}{T}A
        \approx 31{,}4\,\frac{\text{см}}{\text{с}}.
    $
}
\solutionspace{100pt}

\tasknumber{4}%
\task{%
    Математический маятник с нитью длиной $37\,\text{см}$ подвешен к потолку в лифте.
    За $25\,\text{с}$ маятник совершил $23$ колебаний.
    Определите модуль и направление ускорения лифта.
    Куда движется лифт?
}
\answer{%
    $
        T = 2\pi\sqrt{\frac\ell {a + g}}, T = \frac {t}{N}
        \implies a + g = \ell \cdot \frac{4 \pi ^ 2}{T^2},
        a = \ell \cdot \frac{4 \pi ^ 2}{T^2} - g = \ell \cdot \frac{4 \pi ^ 2 N^2}{t^2} - g \approx 2{,}36\,\frac{\text{м}}{\text{c}^{2}},
        \text{вниз}.
    $
}
\solutionspace{120pt}

\tasknumber{5}%
\task{%
    Масса груза в пружинном маятнике равна $400\,\text{г}$, при этом период его колебаний равен $1{,}5\,\text{с}$.
    Груз утяжеляют на $100\,\text{г}$.
    Определите новый период колебаний маятника.
}
\answer{%
    $
        T'
            = 2\pi\sqrt{\frac{M + m}{k}}
            = 2\pi\sqrt{\frac{M}{k} \cdot \frac{M + m}{M}}
            = T\sqrt{\frac{M + m}{M}} =  T\sqrt{1 + \frac{m}{M}} \approx 1{,}68\,\text{с}.
    $
}
\solutionspace{120pt}

\tasknumber{6}%
\task{%
    При какой длине нити математического маятника период колебаний груза массой $200\,\text{г}$
    окажется равен периоду колебаний этого же груза в пружинном маятнике с пружиной жёсткостью $50\,\frac{\text{Н}}{\text{м}}$?
}
\answer{%
    $
        2\pi \sqrt{\frac \ell g} = 2\pi \sqrt{\frac m k}
        \implies \frac \ell g = \frac m k
        \implies \ell = g \frac m k \approx 4\,\text{см}.
    $
}

\variantsplitter

\addpersonalvariant{Матвей Кузьмин}

\tasknumber{1}%
\task{%
    Тело колеблется по гармоническому закону,
    амплитуда этих колебаний $12\,\text{см}$, период $4\,\text{c}$.
    Чему равно смещение тела относительно положения равновесия через $20\,\text{c}$
    после прохождения положения максимального отклонения?
}
\answer{%
    $x = A \cos \omega t = A \cos \cbr{ \frac {2\pi}T t } = A \cos \cbr{ 2\pi \frac tT } = 12\,\text{см} \cdot \cos \cbr{ 2\pi \cdot \frac {20\,\text{c}}{4\,\text{c}}}\approx 12\,\text{см}.$
}
\solutionspace{80pt}

\tasknumber{2}%
\task{%
    Запишите формулу для периода колебаний пружинного маятника и ...
    \begin{itemize}
        \item укажите названия всех физических величин в формуле,
        \item выразите из неё циклическую частоту колебаний
        \item выразите из неё массу груза.
    \end{itemize}
}
\answer{%
    \begin{align*}
    T &= 2\pi \sqrt{\frac lg} \implies \nu = \frac 1T = \frac 1{2\pi}\sqrt{\frac gl}, \omega = 2\pi\nu = \sqrt{\frac gl}, l = g\sqr{\frac T{2\pi}}, g = l\sqr{\frac {2\pi}T} \\
    T &= 2\pi \sqrt{\frac mk} \implies \nu = \frac 1T = \frac 1{2\pi}\sqrt{\frac km}, \omega = 2\pi\nu = \sqrt{\frac km}, m = k\sqr{\frac T{2\pi}}, k = m\sqr{\frac {2\pi}T}
    \end{align*}
}
\solutionspace{80pt}

\tasknumber{3}%
\task{%
    Период колебаний математического маятника равен $5\,\text{с}$,
    а их амплитуда — $10\,\text{см}$.
    Определите максимальную скорость маятника.
}
\answer{%
    $
        T = \frac{2\pi}{\omega}
        \implies \omega = \frac{2\pi}{T}
        \implies v_{\max} = \omega A = \frac{2\pi}{T}A
        \approx 12{,}6\,\frac{\text{см}}{\text{с}}.
    $
}
\solutionspace{100pt}

\tasknumber{4}%
\task{%
    Математический маятник с нитью длиной $40\,\text{см}$ подвешен к потолку в лифте.
    За $20\,\text{с}$ маятник совершил $16$ колебаний.
    Определите модуль и направление ускорения лифта.
    Куда движется лифт?
}
\answer{%
    $
        T = 2\pi\sqrt{\frac\ell {a + g}}, T = \frac {t}{N}
        \implies a + g = \ell \cdot \frac{4 \pi ^ 2}{T^2},
        a = \ell \cdot \frac{4 \pi ^ 2}{T^2} - g = \ell \cdot \frac{4 \pi ^ 2 N^2}{t^2} - g \approx 0{,}11\,\frac{\text{м}}{\text{c}^{2}},
        \text{вниз}.
    $
}
\solutionspace{120pt}

\tasknumber{5}%
\task{%
    Масса груза в пружинном маятнике равна $400\,\text{г}$, при этом период его колебаний равен $1{,}5\,\text{с}$.
    Груз облегчают на $100\,\text{г}$.
    Определите новый период колебаний маятника.
}
\answer{%
    $
        T'
            = 2\pi\sqrt{\frac{M - m}{k}}
            = 2\pi\sqrt{\frac{M}{k} \cdot \frac{M - m}{M}}
            = T\sqrt{\frac{M - m}{M}} =  T\sqrt{1 - \frac{m}{M}} \approx 1{,}30\,\text{с}.
    $
}
\solutionspace{120pt}

\tasknumber{6}%
\task{%
    При какой длине нити математического маятника период колебаний груза массой $200\,\text{г}$
    окажется равен периоду колебаний этого же груза в пружинном маятнике с пружиной жёсткостью $60\,\frac{\text{Н}}{\text{м}}$?
}
\answer{%
    $
        2\pi \sqrt{\frac \ell g} = 2\pi \sqrt{\frac m k}
        \implies \frac \ell g = \frac m k
        \implies \ell = g \frac m k \approx 3{,}3\,\text{см}.
    $
}

\variantsplitter

\addpersonalvariant{Елизавета Кутумова}

\tasknumber{1}%
\task{%
    Тело колеблется по гармоническому закону,
    амплитуда этих колебаний $20\,\text{см}$, период $4\,\text{c}$.
    Чему равно смещение тела относительно положения равновесия через $25\,\text{c}$
    после прохождения положения равновесия?
}
\answer{%
    $x = A \sin \omega t = A \sin \cbr{ \frac {2\pi}T t } = A \sin \cbr{ 2\pi \frac tT } = 20\,\text{см} \cdot \sin \cbr{ 2\pi \cdot \frac {25\,\text{c}}{4\,\text{c}}}\approx 20\,\text{см}.$
}
\solutionspace{80pt}

\tasknumber{2}%
\task{%
    Запишите формулу для периода колебаний математического маятника и ...
    \begin{itemize}
        \item укажите названия всех физических величин в формуле,
        \item выразите из неё циклическую частоту колебаний
        \item выразите из неё длину маятника.
    \end{itemize}
}
\answer{%
    \begin{align*}
    T &= 2\pi \sqrt{\frac lg} \implies \nu = \frac 1T = \frac 1{2\pi}\sqrt{\frac gl}, \omega = 2\pi\nu = \sqrt{\frac gl}, l = g\sqr{\frac T{2\pi}}, g = l\sqr{\frac {2\pi}T} \\
    T &= 2\pi \sqrt{\frac mk} \implies \nu = \frac 1T = \frac 1{2\pi}\sqrt{\frac km}, \omega = 2\pi\nu = \sqrt{\frac km}, m = k\sqr{\frac T{2\pi}}, k = m\sqr{\frac {2\pi}T}
    \end{align*}
}
\solutionspace{80pt}

\tasknumber{3}%
\task{%
    Период колебаний математического маятника равен $4\,\text{с}$,
    а их амплитуда — $20\,\text{см}$.
    Определите амплитуду колебаний скорости маятника.
}
\answer{%
    $
        T = \frac{2\pi}{\omega}
        \implies \omega = \frac{2\pi}{T}
        \implies v_{\max} = \omega A = \frac{2\pi}{T}A
        \approx 31{,}4\,\frac{\text{см}}{\text{с}}.
    $
}
\solutionspace{100pt}

\tasknumber{4}%
\task{%
    Математический маятник с нитью длиной $37\,\text{см}$ подвешен к потолку в лифте.
    За $20\,\text{с}$ маятник совершил $15$ колебаний.
    Определите модуль и направление ускорения лифта.
    Куда движется лифт?
}
\answer{%
    $
        T = 2\pi\sqrt{\frac\ell {a + g}}, T = \frac {t}{N}
        \implies a + g = \ell \cdot \frac{4 \pi ^ 2}{T^2},
        a = \ell \cdot \frac{4 \pi ^ 2}{T^2} - g = \ell \cdot \frac{4 \pi ^ 2 N^2}{t^2} - g \approx -1{,}78\,\frac{\text{м}}{\text{c}^{2}},
        \text{вверх}.
    $
}
\solutionspace{120pt}

\tasknumber{5}%
\task{%
    Масса груза в пружинном маятнике равна $500\,\text{г}$, при этом период его колебаний равен $1{,}3\,\text{с}$.
    Груз облегчают на $50\,\text{г}$.
    Определите новый период колебаний маятника.
}
\answer{%
    $
        T'
            = 2\pi\sqrt{\frac{M - m}{k}}
            = 2\pi\sqrt{\frac{M}{k} \cdot \frac{M - m}{M}}
            = T\sqrt{\frac{M - m}{M}} =  T\sqrt{1 - \frac{m}{M}} \approx 1{,}23\,\text{с}.
    $
}
\solutionspace{120pt}

\tasknumber{6}%
\task{%
    При какой длине нити математического маятника период колебаний груза массой $200\,\text{г}$
    окажется равен периоду колебаний этого же груза в пружинном маятнике с пружиной жёсткостью $60\,\frac{\text{Н}}{\text{м}}$?
}
\answer{%
    $
        2\pi \sqrt{\frac \ell g} = 2\pi \sqrt{\frac m k}
        \implies \frac \ell g = \frac m k
        \implies \ell = g \frac m k \approx 3{,}3\,\text{см}.
    $
}

\variantsplitter

\addpersonalvariant{Роксана Мехтиева}

\tasknumber{1}%
\task{%
    Тело колеблется по гармоническому закону,
    амплитуда этих колебаний $12\,\text{см}$, период $4\,\text{c}$.
    Чему равно смещение тела относительно положения равновесия через $25\,\text{c}$
    после прохождения положения равновесия?
}
\answer{%
    $x = A \sin \omega t = A \sin \cbr{ \frac {2\pi}T t } = A \sin \cbr{ 2\pi \frac tT } = 12\,\text{см} \cdot \sin \cbr{ 2\pi \cdot \frac {25\,\text{c}}{4\,\text{c}}}\approx 12\,\text{см}.$
}
\solutionspace{80pt}

\tasknumber{2}%
\task{%
    Запишите формулу для периода колебаний математического маятника и ...
    \begin{itemize}
        \item укажите названия всех физических величин в формуле,
        \item выразите из неё циклическую частоту колебаний
        \item выразите из неё длину маятника.
    \end{itemize}
}
\answer{%
    \begin{align*}
    T &= 2\pi \sqrt{\frac lg} \implies \nu = \frac 1T = \frac 1{2\pi}\sqrt{\frac gl}, \omega = 2\pi\nu = \sqrt{\frac gl}, l = g\sqr{\frac T{2\pi}}, g = l\sqr{\frac {2\pi}T} \\
    T &= 2\pi \sqrt{\frac mk} \implies \nu = \frac 1T = \frac 1{2\pi}\sqrt{\frac km}, \omega = 2\pi\nu = \sqrt{\frac km}, m = k\sqr{\frac T{2\pi}}, k = m\sqr{\frac {2\pi}T}
    \end{align*}
}
\solutionspace{80pt}

\tasknumber{3}%
\task{%
    Период колебаний математического маятника равен $4\,\text{с}$,
    а их амплитуда — $10\,\text{см}$.
    Определите амплитуду колебаний скорости маятника.
}
\answer{%
    $
        T = \frac{2\pi}{\omega}
        \implies \omega = \frac{2\pi}{T}
        \implies v_{\max} = \omega A = \frac{2\pi}{T}A
        \approx 15{,}7\,\frac{\text{см}}{\text{с}}.
    $
}
\solutionspace{100pt}

\tasknumber{4}%
\task{%
    Математический маятник с нитью длиной $43\,\text{см}$ подвешен к потолку в лифте.
    За $20\,\text{с}$ маятник совершил $16$ колебаний.
    Определите модуль и направление ускорения лифта.
    Куда движется лифт?
}
\answer{%
    $
        T = 2\pi\sqrt{\frac\ell {a + g}}, T = \frac {t}{N}
        \implies a + g = \ell \cdot \frac{4 \pi ^ 2}{T^2},
        a = \ell \cdot \frac{4 \pi ^ 2}{T^2} - g = \ell \cdot \frac{4 \pi ^ 2 N^2}{t^2} - g \approx 0{,}86\,\frac{\text{м}}{\text{c}^{2}},
        \text{вниз}.
    $
}
\solutionspace{120pt}

\tasknumber{5}%
\task{%
    Масса груза в пружинном маятнике равна $500\,\text{г}$, при этом период его колебаний равен $1{,}3\,\text{с}$.
    Груз утяжеляют на $50\,\text{г}$.
    Определите новый период колебаний маятника.
}
\answer{%
    $
        T'
            = 2\pi\sqrt{\frac{M + m}{k}}
            = 2\pi\sqrt{\frac{M}{k} \cdot \frac{M + m}{M}}
            = T\sqrt{\frac{M + m}{M}} =  T\sqrt{1 + \frac{m}{M}} \approx 1{,}36\,\text{с}.
    $
}
\solutionspace{120pt}

\tasknumber{6}%
\task{%
    При какой длине нити математического маятника период колебаний груза массой $400\,\text{г}$
    окажется равен периоду колебаний этого же груза в пружинном маятнике с пружиной жёсткостью $40\,\frac{\text{Н}}{\text{м}}$?
}
\answer{%
    $
        2\pi \sqrt{\frac \ell g} = 2\pi \sqrt{\frac m k}
        \implies \frac \ell g = \frac m k
        \implies \ell = g \frac m k \approx 10\,\text{см}.
    $
}

\variantsplitter

\addpersonalvariant{Дилноза Нодиршоева}

\tasknumber{1}%
\task{%
    Тело колеблется по гармоническому закону,
    амплитуда этих колебаний $18\,\text{см}$, период $4\,\text{c}$.
    Чему равно смещение тела относительно положения равновесия через $26\,\text{c}$
    после прохождения положения равновесия?
}
\answer{%
    $x = A \sin \omega t = A \sin \cbr{ \frac {2\pi}T t } = A \sin \cbr{ 2\pi \frac tT } = 18\,\text{см} \cdot \sin \cbr{ 2\pi \cdot \frac {26\,\text{c}}{4\,\text{c}}}\approx 0\,\text{см}.$
}
\solutionspace{80pt}

\tasknumber{2}%
\task{%
    Запишите формулу для периода колебаний математического маятника и ...
    \begin{itemize}
        \item укажите названия всех физических величин в формуле,
        \item выразите из неё частоту колебаний
        \item выразите из неё ускорение свободного падения.
    \end{itemize}
}
\answer{%
    \begin{align*}
    T &= 2\pi \sqrt{\frac lg} \implies \nu = \frac 1T = \frac 1{2\pi}\sqrt{\frac gl}, \omega = 2\pi\nu = \sqrt{\frac gl}, l = g\sqr{\frac T{2\pi}}, g = l\sqr{\frac {2\pi}T} \\
    T &= 2\pi \sqrt{\frac mk} \implies \nu = \frac 1T = \frac 1{2\pi}\sqrt{\frac km}, \omega = 2\pi\nu = \sqrt{\frac km}, m = k\sqr{\frac T{2\pi}}, k = m\sqr{\frac {2\pi}T}
    \end{align*}
}
\solutionspace{80pt}

\tasknumber{3}%
\task{%
    Период колебаний математического маятника равен $2\,\text{с}$,
    а их амплитуда — $15\,\text{см}$.
    Определите амплитуду колебаний скорости маятника.
}
\answer{%
    $
        T = \frac{2\pi}{\omega}
        \implies \omega = \frac{2\pi}{T}
        \implies v_{\max} = \omega A = \frac{2\pi}{T}A
        \approx 47{,}1\,\frac{\text{см}}{\text{с}}.
    $
}
\solutionspace{100pt}

\tasknumber{4}%
\task{%
    Математический маятник с нитью длиной $37\,\text{см}$ подвешен к потолку в лифте.
    За $25\,\text{с}$ маятник совершил $22$ колебаний.
    Определите модуль и направление ускорения лифта.
    Куда движется лифт?
}
\answer{%
    $
        T = 2\pi\sqrt{\frac\ell {a + g}}, T = \frac {t}{N}
        \implies a + g = \ell \cdot \frac{4 \pi ^ 2}{T^2},
        a = \ell \cdot \frac{4 \pi ^ 2}{T^2} - g = \ell \cdot \frac{4 \pi ^ 2 N^2}{t^2} - g \approx 1{,}31\,\frac{\text{м}}{\text{c}^{2}},
        \text{вниз}.
    $
}
\solutionspace{120pt}

\tasknumber{5}%
\task{%
    Масса груза в пружинном маятнике равна $600\,\text{г}$, при этом период его колебаний равен $1{,}4\,\text{с}$.
    Груз утяжеляют на $100\,\text{г}$.
    Определите новый период колебаний маятника.
}
\answer{%
    $
        T'
            = 2\pi\sqrt{\frac{M + m}{k}}
            = 2\pi\sqrt{\frac{M}{k} \cdot \frac{M + m}{M}}
            = T\sqrt{\frac{M + m}{M}} =  T\sqrt{1 + \frac{m}{M}} \approx 1{,}51\,\text{с}.
    $
}
\solutionspace{120pt}

\tasknumber{6}%
\task{%
    При какой длине нити математического маятника период колебаний груза массой $400\,\text{г}$
    окажется равен периоду колебаний этого же груза в пружинном маятнике с пружиной жёсткостью $40\,\frac{\text{Н}}{\text{м}}$?
}
\answer{%
    $
        2\pi \sqrt{\frac \ell g} = 2\pi \sqrt{\frac m k}
        \implies \frac \ell g = \frac m k
        \implies \ell = g \frac m k \approx 10\,\text{см}.
    $
}

\variantsplitter

\addpersonalvariant{Артём Переверзев}

\tasknumber{1}%
\task{%
    Тело колеблется по гармоническому закону,
    амплитуда этих колебаний $20\,\text{см}$, период $2\,\text{c}$.
    Чему равно смещение тела относительно положения равновесия через $20\,\text{c}$
    после прохождения положения равновесия?
}
\answer{%
    $x = A \sin \omega t = A \sin \cbr{ \frac {2\pi}T t } = A \sin \cbr{ 2\pi \frac tT } = 20\,\text{см} \cdot \sin \cbr{ 2\pi \cdot \frac {20\,\text{c}}{2\,\text{c}}}\approx 0\,\text{см}.$
}
\solutionspace{80pt}

\tasknumber{2}%
\task{%
    Запишите формулу для периода колебаний математического маятника и ...
    \begin{itemize}
        \item укажите названия всех физических величин в формуле,
        \item выразите из неё частоту колебаний
        \item выразите из неё ускорение свободного падения.
    \end{itemize}
}
\answer{%
    \begin{align*}
    T &= 2\pi \sqrt{\frac lg} \implies \nu = \frac 1T = \frac 1{2\pi}\sqrt{\frac gl}, \omega = 2\pi\nu = \sqrt{\frac gl}, l = g\sqr{\frac T{2\pi}}, g = l\sqr{\frac {2\pi}T} \\
    T &= 2\pi \sqrt{\frac mk} \implies \nu = \frac 1T = \frac 1{2\pi}\sqrt{\frac km}, \omega = 2\pi\nu = \sqrt{\frac km}, m = k\sqr{\frac T{2\pi}}, k = m\sqr{\frac {2\pi}T}
    \end{align*}
}
\solutionspace{80pt}

\tasknumber{3}%
\task{%
    Период колебаний математического маятника равен $5\,\text{с}$,
    а их амплитуда — $20\,\text{см}$.
    Определите амплитуду колебаний скорости маятника.
}
\answer{%
    $
        T = \frac{2\pi}{\omega}
        \implies \omega = \frac{2\pi}{T}
        \implies v_{\max} = \omega A = \frac{2\pi}{T}A
        \approx 25{,}1\,\frac{\text{см}}{\text{с}}.
    $
}
\solutionspace{100pt}

\tasknumber{4}%
\task{%
    Математический маятник с нитью длиной $37\,\text{см}$ подвешен к потолку в лифте.
    За $30\,\text{с}$ маятник совершил $27$ колебаний.
    Определите модуль и направление ускорения лифта.
    Куда движется лифт?
}
\answer{%
    $
        T = 2\pi\sqrt{\frac\ell {a + g}}, T = \frac {t}{N}
        \implies a + g = \ell \cdot \frac{4 \pi ^ 2}{T^2},
        a = \ell \cdot \frac{4 \pi ^ 2}{T^2} - g = \ell \cdot \frac{4 \pi ^ 2 N^2}{t^2} - g \approx 1{,}83\,\frac{\text{м}}{\text{c}^{2}},
        \text{вниз}.
    $
}
\solutionspace{120pt}

\tasknumber{5}%
\task{%
    Масса груза в пружинном маятнике равна $500\,\text{г}$, при этом период его колебаний равен $1{,}4\,\text{с}$.
    Груз утяжеляют на $150\,\text{г}$.
    Определите новый период колебаний маятника.
}
\answer{%
    $
        T'
            = 2\pi\sqrt{\frac{M + m}{k}}
            = 2\pi\sqrt{\frac{M}{k} \cdot \frac{M + m}{M}}
            = T\sqrt{\frac{M + m}{M}} =  T\sqrt{1 + \frac{m}{M}} \approx 1{,}60\,\text{с}.
    $
}
\solutionspace{120pt}

\tasknumber{6}%
\task{%
    При какой длине нити математического маятника период колебаний груза массой $300\,\text{г}$
    окажется равен периоду колебаний этого же груза в пружинном маятнике с пружиной жёсткостью $60\,\frac{\text{Н}}{\text{м}}$?
}
\answer{%
    $
        2\pi \sqrt{\frac \ell g} = 2\pi \sqrt{\frac m k}
        \implies \frac \ell g = \frac m k
        \implies \ell = g \frac m k \approx 5\,\text{см}.
    $
}

\variantsplitter

\addpersonalvariant{Варвара Пранова}

\tasknumber{1}%
\task{%
    Тело колеблется по гармоническому закону,
    амплитуда этих колебаний $14\,\text{см}$, период $4\,\text{c}$.
    Чему равно смещение тела относительно положения равновесия через $23\,\text{c}$
    после прохождения положения максимального отклонения?
}
\answer{%
    $x = A \cos \omega t = A \cos \cbr{ \frac {2\pi}T t } = A \cos \cbr{ 2\pi \frac tT } = 14\,\text{см} \cdot \cos \cbr{ 2\pi \cdot \frac {23\,\text{c}}{4\,\text{c}}}\approx 0\,\text{см}.$
}
\solutionspace{80pt}

\tasknumber{2}%
\task{%
    Запишите формулу для периода колебаний математического маятника и ...
    \begin{itemize}
        \item укажите названия всех физических величин в формуле,
        \item выразите из неё частоту колебаний
        \item выразите из неё ускорение свободного падения.
    \end{itemize}
}
\answer{%
    \begin{align*}
    T &= 2\pi \sqrt{\frac lg} \implies \nu = \frac 1T = \frac 1{2\pi}\sqrt{\frac gl}, \omega = 2\pi\nu = \sqrt{\frac gl}, l = g\sqr{\frac T{2\pi}}, g = l\sqr{\frac {2\pi}T} \\
    T &= 2\pi \sqrt{\frac mk} \implies \nu = \frac 1T = \frac 1{2\pi}\sqrt{\frac km}, \omega = 2\pi\nu = \sqrt{\frac km}, m = k\sqr{\frac T{2\pi}}, k = m\sqr{\frac {2\pi}T}
    \end{align*}
}
\solutionspace{80pt}

\tasknumber{3}%
\task{%
    Период колебаний математического маятника равен $4\,\text{с}$,
    а их амплитуда — $20\,\text{см}$.
    Определите амплитуду колебаний скорости маятника.
}
\answer{%
    $
        T = \frac{2\pi}{\omega}
        \implies \omega = \frac{2\pi}{T}
        \implies v_{\max} = \omega A = \frac{2\pi}{T}A
        \approx 31{,}4\,\frac{\text{см}}{\text{с}}.
    $
}
\solutionspace{100pt}

\tasknumber{4}%
\task{%
    Математический маятник с нитью длиной $43\,\text{см}$ подвешен к потолку в лифте.
    За $25\,\text{с}$ маятник совершил $21$ колебаний.
    Определите модуль и направление ускорения лифта.
    Куда движется лифт?
}
\answer{%
    $
        T = 2\pi\sqrt{\frac\ell {a + g}}, T = \frac {t}{N}
        \implies a + g = \ell \cdot \frac{4 \pi ^ 2}{T^2},
        a = \ell \cdot \frac{4 \pi ^ 2}{T^2} - g = \ell \cdot \frac{4 \pi ^ 2 N^2}{t^2} - g \approx 1{,}98\,\frac{\text{м}}{\text{c}^{2}},
        \text{вниз}.
    $
}
\solutionspace{120pt}

\tasknumber{5}%
\task{%
    Масса груза в пружинном маятнике равна $500\,\text{г}$, при этом период его колебаний равен $1{,}5\,\text{с}$.
    Груз облегчают на $150\,\text{г}$.
    Определите новый период колебаний маятника.
}
\answer{%
    $
        T'
            = 2\pi\sqrt{\frac{M - m}{k}}
            = 2\pi\sqrt{\frac{M}{k} \cdot \frac{M - m}{M}}
            = T\sqrt{\frac{M - m}{M}} =  T\sqrt{1 - \frac{m}{M}} \approx 1{,}25\,\text{с}.
    $
}
\solutionspace{120pt}

\tasknumber{6}%
\task{%
    При какой длине нити математического маятника период колебаний груза массой $300\,\text{г}$
    окажется равен периоду колебаний этого же груза в пружинном маятнике с пружиной жёсткостью $60\,\frac{\text{Н}}{\text{м}}$?
}
\answer{%
    $
        2\pi \sqrt{\frac \ell g} = 2\pi \sqrt{\frac m k}
        \implies \frac \ell g = \frac m k
        \implies \ell = g \frac m k \approx 5\,\text{см}.
    $
}

\variantsplitter

\addpersonalvariant{Марьям Салимова}

\tasknumber{1}%
\task{%
    Тело колеблется по гармоническому закону,
    амплитуда этих колебаний $16\,\text{см}$, период $6\,\text{c}$.
    Чему равно смещение тела относительно положения равновесия через $26\,\text{c}$
    после прохождения положения максимального отклонения?
}
\answer{%
    $x = A \cos \omega t = A \cos \cbr{ \frac {2\pi}T t } = A \cos \cbr{ 2\pi \frac tT } = 16\,\text{см} \cdot \cos \cbr{ 2\pi \cdot \frac {26\,\text{c}}{6\,\text{c}}}\approx -8\,\text{см}.$
}
\solutionspace{80pt}

\tasknumber{2}%
\task{%
    Запишите формулу для периода колебаний пружинного маятника и ...
    \begin{itemize}
        \item укажите названия всех физических величин в формуле,
        \item выразите из неё частоту колебаний
        \item выразите из неё жёсткость пружины.
    \end{itemize}
}
\answer{%
    \begin{align*}
    T &= 2\pi \sqrt{\frac lg} \implies \nu = \frac 1T = \frac 1{2\pi}\sqrt{\frac gl}, \omega = 2\pi\nu = \sqrt{\frac gl}, l = g\sqr{\frac T{2\pi}}, g = l\sqr{\frac {2\pi}T} \\
    T &= 2\pi \sqrt{\frac mk} \implies \nu = \frac 1T = \frac 1{2\pi}\sqrt{\frac km}, \omega = 2\pi\nu = \sqrt{\frac km}, m = k\sqr{\frac T{2\pi}}, k = m\sqr{\frac {2\pi}T}
    \end{align*}
}
\solutionspace{80pt}

\tasknumber{3}%
\task{%
    Период колебаний математического маятника равен $2\,\text{с}$,
    а их амплитуда — $10\,\text{см}$.
    Определите максимальную скорость маятника.
}
\answer{%
    $
        T = \frac{2\pi}{\omega}
        \implies \omega = \frac{2\pi}{T}
        \implies v_{\max} = \omega A = \frac{2\pi}{T}A
        \approx 31{,}4\,\frac{\text{см}}{\text{с}}.
    $
}
\solutionspace{100pt}

\tasknumber{4}%
\task{%
    Математический маятник с нитью длиной $43\,\text{см}$ подвешен к потолку в лифте.
    За $25\,\text{с}$ маятник совершил $20$ колебаний.
    Определите модуль и направление ускорения лифта.
    Куда движется лифт?
}
\answer{%
    $
        T = 2\pi\sqrt{\frac\ell {a + g}}, T = \frac {t}{N}
        \implies a + g = \ell \cdot \frac{4 \pi ^ 2}{T^2},
        a = \ell \cdot \frac{4 \pi ^ 2}{T^2} - g = \ell \cdot \frac{4 \pi ^ 2 N^2}{t^2} - g \approx 0{,}86\,\frac{\text{м}}{\text{c}^{2}},
        \text{вниз}.
    $
}
\solutionspace{120pt}

\tasknumber{5}%
\task{%
    Масса груза в пружинном маятнике равна $500\,\text{г}$, при этом период его колебаний равен $1{,}3\,\text{с}$.
    Груз облегчают на $150\,\text{г}$.
    Определите новый период колебаний маятника.
}
\answer{%
    $
        T'
            = 2\pi\sqrt{\frac{M - m}{k}}
            = 2\pi\sqrt{\frac{M}{k} \cdot \frac{M - m}{M}}
            = T\sqrt{\frac{M - m}{M}} =  T\sqrt{1 - \frac{m}{M}} \approx 1{,}09\,\text{с}.
    $
}
\solutionspace{120pt}

\tasknumber{6}%
\task{%
    При какой длине нити математического маятника период колебаний груза массой $200\,\text{г}$
    окажется равен периоду колебаний этого же груза в пружинном маятнике с пружиной жёсткостью $50\,\frac{\text{Н}}{\text{м}}$?
}
\answer{%
    $
        2\pi \sqrt{\frac \ell g} = 2\pi \sqrt{\frac m k}
        \implies \frac \ell g = \frac m k
        \implies \ell = g \frac m k \approx 4\,\text{см}.
    $
}

\variantsplitter

\addpersonalvariant{Юлия Шевченко}

\tasknumber{1}%
\task{%
    Тело колеблется по гармоническому закону,
    амплитуда этих колебаний $14\,\text{см}$, период $2\,\text{c}$.
    Чему равно смещение тела относительно положения равновесия через $25\,\text{c}$
    после прохождения положения максимального отклонения?
}
\answer{%
    $x = A \cos \omega t = A \cos \cbr{ \frac {2\pi}T t } = A \cos \cbr{ 2\pi \frac tT } = 14\,\text{см} \cdot \cos \cbr{ 2\pi \cdot \frac {25\,\text{c}}{2\,\text{c}}}\approx -14\,\text{см}.$
}
\solutionspace{80pt}

\tasknumber{2}%
\task{%
    Запишите формулу для периода колебаний пружинного маятника и ...
    \begin{itemize}
        \item укажите названия всех физических величин в формуле,
        \item выразите из неё частоту колебаний
        \item выразите из неё жёсткость пружины.
    \end{itemize}
}
\answer{%
    \begin{align*}
    T &= 2\pi \sqrt{\frac lg} \implies \nu = \frac 1T = \frac 1{2\pi}\sqrt{\frac gl}, \omega = 2\pi\nu = \sqrt{\frac gl}, l = g\sqr{\frac T{2\pi}}, g = l\sqr{\frac {2\pi}T} \\
    T &= 2\pi \sqrt{\frac mk} \implies \nu = \frac 1T = \frac 1{2\pi}\sqrt{\frac km}, \omega = 2\pi\nu = \sqrt{\frac km}, m = k\sqr{\frac T{2\pi}}, k = m\sqr{\frac {2\pi}T}
    \end{align*}
}
\solutionspace{80pt}

\tasknumber{3}%
\task{%
    Период колебаний математического маятника равен $3\,\text{с}$,
    а их амплитуда — $20\,\text{см}$.
    Определите амплитуду колебаний скорости маятника.
}
\answer{%
    $
        T = \frac{2\pi}{\omega}
        \implies \omega = \frac{2\pi}{T}
        \implies v_{\max} = \omega A = \frac{2\pi}{T}A
        \approx 41{,}9\,\frac{\text{см}}{\text{с}}.
    $
}
\solutionspace{100pt}

\tasknumber{4}%
\task{%
    Математический маятник с нитью длиной $40\,\text{см}$ подвешен к потолку в лифте.
    За $25\,\text{с}$ маятник совершил $19$ колебаний.
    Определите модуль и направление ускорения лифта.
    Куда движется лифт?
}
\answer{%
    $
        T = 2\pi\sqrt{\frac\ell {a + g}}, T = \frac {t}{N}
        \implies a + g = \ell \cdot \frac{4 \pi ^ 2}{T^2},
        a = \ell \cdot \frac{4 \pi ^ 2}{T^2} - g = \ell \cdot \frac{4 \pi ^ 2 N^2}{t^2} - g \approx -0{,}88\,\frac{\text{м}}{\text{c}^{2}},
        \text{вверх}.
    $
}
\solutionspace{120pt}

\tasknumber{5}%
\task{%
    Масса груза в пружинном маятнике равна $600\,\text{г}$, при этом период его колебаний равен $1{,}2\,\text{с}$.
    Груз облегчают на $100\,\text{г}$.
    Определите новый период колебаний маятника.
}
\answer{%
    $
        T'
            = 2\pi\sqrt{\frac{M - m}{k}}
            = 2\pi\sqrt{\frac{M}{k} \cdot \frac{M - m}{M}}
            = T\sqrt{\frac{M - m}{M}} =  T\sqrt{1 - \frac{m}{M}} \approx 1{,}10\,\text{с}.
    $
}
\solutionspace{120pt}

\tasknumber{6}%
\task{%
    При какой длине нити математического маятника период колебаний груза массой $300\,\text{г}$
    окажется равен периоду колебаний этого же груза в пружинном маятнике с пружиной жёсткостью $40\,\frac{\text{Н}}{\text{м}}$?
}
\answer{%
    $
        2\pi \sqrt{\frac \ell g} = 2\pi \sqrt{\frac m k}
        \implies \frac \ell g = \frac m k
        \implies \ell = g \frac m k \approx 7{,}5\,\text{см}.
    $
}
% autogenerated
