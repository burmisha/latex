\setdate{7~октября~2021}
\setclass{11«БА»}

\addpersonalvariant{Михаил Бурмистров}

\tasknumber{1}%
\task{%
    Установите каждой букве в соответствие ровно одну цифру и запишите ответ (только цифры, без других символов).

    А) период колебаний, Б) число колебаний, В) время колебаний.

    1) $t$, 2) $T$, 3) $tN$, 4) $N$, 5) $\nu$.
}
\answer{%
    $241$
}

\tasknumber{2}%
\task{%
    Установите каждой букве в соответствие ровно одну цифру и запишите ответ (только цифры, без других символов).

    А) период колебаний, Б) частота колебаний.

    1) c, 2) рад / с, 3) Гц, 4) Гн.
}
\answer{%
    $13$
}

\tasknumber{3}%
\task{%
    \begin{itemize}
        \item Запишите линейное однородное дифференциальное уравнение второго порядка,
            описывающее свободные незатухающие колебания гармонического осциллятора,
        \item запишите общее решение этого уравнения,
        \item подпишите в выписанном решении фазу и амплитуду колебаний,
        \item запишите выражение для скорости,
        \item запишите выражение для ускорения.
    \end{itemize}
}
\answer{%
    \begin{align*}
    &\ddot x + \omega^2 x = 0 \Longleftrightarrow a_x + \omega^2 x = 0, \\
    &x = A \cos(\omega t + \varphi_0) \text{ или же } x = A \sin(\omega t + \varphi_0) \text{ или же } x = a \cos(\omega t) + b \sin(\omega t), \\
    &A \text{\, или \,} \sqrt{a^2 + b^2} \text{ --- это амплитуда}, \omega t + \varphi_0\text{ --- это фаза}, \\
    &v = \dot x = -\omega A \sin(\omega t + \varphi_0), \\
    &a = \dot v = \ddot x = -\omega^2 A \cos(\omega t + \varphi_0) = -\omega^2 x,
    \end{align*}
}
\solutionspace{135pt}

\tasknumber{4}%
\task{%
    Определите частоту колебаний, если их период составляет $T = 2\,\text{мс}$.
}
\answer{%
    $\nu = \frac 1T = \frac 1{2\,\text{мс}} = 500\,\text{Гц}$
}
\solutionspace{40pt}

\tasknumber{5}%
\task{%
    Определите период колебаний, если их частота составляет $\nu = 2\,\text{кГц}$.
    Сколько колебаний произойдёт за $t = 2\,\text{мин}$?
}
\answer{%
    \begin{align*}
    T &= \frac 1\nu = \frac 1{2\,\text{кГц}} = 0{,}500\,\text{мc}, \\
    N &= \nu t = 2\,\text{кГц} \cdot2\,\text{мин} = 240000\,\text{колебаний}.
    \end{align*}
}
\solutionspace{40pt}

\tasknumber{6}%
\task{%
    Амплитуда колебаний точки составляет $A = 3\,\text{см}$, а частота~--- $\nu = 20\,\text{Гц}$.
    Определите, какой путь преодолеет эта точка за $t = 80\,\text{с}$.
}
\answer{%
    $s = 4A \cdot N = 4A \cdot \frac tT = 4A \cdot t\nu = 4 \cdot 3\,\text{см} \cdot 80\,\text{с} \cdot 20\,\text{Гц} = 192{,}0\,\text{м}$
}
\solutionspace{120pt}

\tasknumber{7}%
\task{%
    Изобразите график гармонических колебаний,
    амплитуда которых составляла бы $A = 3\,\text{см}$, а период $T = 10\,\text{с}$.
}
\solutionspace{80pt}

\tasknumber{8}%
\task{%
    Координата материальной точки зависит от времени по закону $z = 0{,}05 \cdot \sin (6\pi t)$ (в СИ).
    Чему равен путь, пройденный точкой за $2\,\text{мин}$?
}
\answer{%
    $\omega = 6\pi \implies \nu = \frac62\,\units{Гц}, N = \nu t = 360{,}0, s = 4AN = 4 \cdot 0{,}05 \cdot 360{,}0 = 72{,}0 \text{(м)}$
}

\variantsplitter

\addpersonalvariant{Ирина Ан}

\tasknumber{1}%
\task{%
    Установите каждой букве в соответствие ровно одну цифру и запишите ответ (только цифры, без других символов).

    А) период колебаний, Б) циклическая частота, В) число колебаний.

    1) $N$, 2) $\omega$, 3) $T$, 4) $tN$, 5) $t$.
}
\answer{%
    $321$
}

\tasknumber{2}%
\task{%
    Установите каждой букве в соответствие ровно одну цифру и запишите ответ (только цифры, без других символов).

    А) частота колебаний, Б) циклическая частота.

    1) рад / с, 2) Гц, 3) м / с, 4) Гн.
}
\answer{%
    $21$
}

\tasknumber{3}%
\task{%
    \begin{itemize}
        \item Запишите линейное однородное дифференциальное уравнение второго порядка,
            описывающее свободные незатухающие колебания гармонического осциллятора,
        \item запишите общее решение этого уравнения,
        \item подпишите в выписанном решении фазу и амплитуду колебаний,
        \item запишите выражение для скорости,
        \item запишите выражение для ускорения.
    \end{itemize}
}
\answer{%
    \begin{align*}
    &\ddot x + \omega^2 x = 0 \Longleftrightarrow a_x + \omega^2 x = 0, \\
    &x = A \cos(\omega t + \varphi_0) \text{ или же } x = A \sin(\omega t + \varphi_0) \text{ или же } x = a \cos(\omega t) + b \sin(\omega t), \\
    &A \text{\, или \,} \sqrt{a^2 + b^2} \text{ --- это амплитуда}, \omega t + \varphi_0\text{ --- это фаза}, \\
    &v = \dot x = -\omega A \sin(\omega t + \varphi_0), \\
    &a = \dot v = \ddot x = -\omega^2 A \cos(\omega t + \varphi_0) = -\omega^2 x,
    \end{align*}
}
\solutionspace{135pt}

\tasknumber{4}%
\task{%
    Определите частоту колебаний, если их период составляет $T = 4\,\text{мс}$.
}
\answer{%
    $\nu = \frac 1T = \frac 1{4\,\text{мс}} = 250\,\text{Гц}$
}
\solutionspace{40pt}

\tasknumber{5}%
\task{%
    Определите период колебаний, если их частота составляет $\nu = 4\,\text{кГц}$.
    Сколько колебаний произойдёт за $t = 3\,\text{мин}$?
}
\answer{%
    \begin{align*}
    T &= \frac 1\nu = \frac 1{4\,\text{кГц}} = 0{,}250\,\text{мc}, \\
    N &= \nu t = 4\,\text{кГц} \cdot3\,\text{мин} = 720000\,\text{колебаний}.
    \end{align*}
}
\solutionspace{40pt}

\tasknumber{6}%
\task{%
    Амплитуда колебаний точки составляет $A = 2\,\text{см}$, а частота~--- $\nu = 10\,\text{Гц}$.
    Определите, какой путь преодолеет эта точка за $t = 10\,\text{с}$.
}
\answer{%
    $s = 4A \cdot N = 4A \cdot \frac tT = 4A \cdot t\nu = 4 \cdot 2\,\text{см} \cdot 10\,\text{с} \cdot 10\,\text{Гц} = 8{,}0\,\text{м}$
}
\solutionspace{120pt}

\tasknumber{7}%
\task{%
    Изобразите график гармонических колебаний,
    амплитуда которых составляла бы $A = 3\,\text{см}$, а период $T = 10\,\text{с}$.
}
\solutionspace{80pt}

\tasknumber{8}%
\task{%
    Координата материальной точки зависит от времени по закону $y = 0{,}02 \cdot \cos (5\pi t)$ (в СИ).
    Чему равен путь, пройденный точкой за $3\,\text{мин}$?
}
\answer{%
    $\omega = 5\pi \implies \nu = \frac52\,\units{Гц}, N = \nu t = 450{,}0, s = 4AN = 4 \cdot 0{,}02 \cdot 450{,}0 = 36{,}0 \text{(м)}$
}

\variantsplitter

\addpersonalvariant{Софья Андрианова}

\tasknumber{1}%
\task{%
    Установите каждой букве в соответствие ровно одну цифру и запишите ответ (только цифры, без других символов).

    А) время колебаний, Б) частота колебаний, В) период колебаний.

    1) $\frac{\nu}{2\pi}$, 2) $\nu$, 3) $T$, 4) $t$, 5) $N$.
}
\answer{%
    $423$
}

\tasknumber{2}%
\task{%
    Установите каждой букве в соответствие ровно одну цифру и запишите ответ (только цифры, без других символов).

    А) частота колебаний, Б) циклическая частота.

    1) рад / с, 2) c, 3) Гц, 4) м / с.
}
\answer{%
    $31$
}

\tasknumber{3}%
\task{%
    \begin{itemize}
        \item Запишите линейное однородное дифференциальное уравнение второго порядка,
            описывающее свободные незатухающие колебания гармонического осциллятора,
        \item запишите общее решение этого уравнения,
        \item подпишите в выписанном решении фазу и амплитуду колебаний,
        \item запишите выражение для скорости,
        \item запишите выражение для ускорения.
    \end{itemize}
}
\answer{%
    \begin{align*}
    &\ddot x + \omega^2 x = 0 \Longleftrightarrow a_x + \omega^2 x = 0, \\
    &x = A \cos(\omega t + \varphi_0) \text{ или же } x = A \sin(\omega t + \varphi_0) \text{ или же } x = a \cos(\omega t) + b \sin(\omega t), \\
    &A \text{\, или \,} \sqrt{a^2 + b^2} \text{ --- это амплитуда}, \omega t + \varphi_0\text{ --- это фаза}, \\
    &v = \dot x = -\omega A \sin(\omega t + \varphi_0), \\
    &a = \dot v = \ddot x = -\omega^2 A \cos(\omega t + \varphi_0) = -\omega^2 x,
    \end{align*}
}
\solutionspace{135pt}

\tasknumber{4}%
\task{%
    Определите частоту колебаний, если их период составляет $T = 4\,\text{мс}$.
}
\answer{%
    $\nu = \frac 1T = \frac 1{4\,\text{мс}} = 250\,\text{Гц}$
}
\solutionspace{40pt}

\tasknumber{5}%
\task{%
    Определите период колебаний, если их частота составляет $\nu = 4\,\text{кГц}$.
    Сколько колебаний произойдёт за $t = 3\,\text{мин}$?
}
\answer{%
    \begin{align*}
    T &= \frac 1\nu = \frac 1{4\,\text{кГц}} = 0{,}250\,\text{мc}, \\
    N &= \nu t = 4\,\text{кГц} \cdot3\,\text{мин} = 720000\,\text{колебаний}.
    \end{align*}
}
\solutionspace{40pt}

\tasknumber{6}%
\task{%
    Амплитуда колебаний точки составляет $A = 15\,\text{см}$, а частота~--- $\nu = 6\,\text{Гц}$.
    Определите, какой путь преодолеет эта точка за $t = 10\,\text{с}$.
}
\answer{%
    $s = 4A \cdot N = 4A \cdot \frac tT = 4A \cdot t\nu = 4 \cdot 15\,\text{см} \cdot 10\,\text{с} \cdot 6\,\text{Гц} = 36{,}0\,\text{м}$
}
\solutionspace{120pt}

\tasknumber{7}%
\task{%
    Изобразите график гармонических колебаний,
    амплитуда которых составляла бы $A = 1\,\text{см}$, а период $T = 8\,\text{с}$.
}
\solutionspace{80pt}

\tasknumber{8}%
\task{%
    Координата материальной точки зависит от времени по закону $z = 0{,}15 \cdot \sin (6\pi t)$ (в СИ).
    Чему равен путь, пройденный точкой за $2\,\text{мин}$?
}
\answer{%
    $\omega = 6\pi \implies \nu = \frac62\,\units{Гц}, N = \nu t = 360{,}0, s = 4AN = 4 \cdot 0{,}15 \cdot 360{,}0 = 216{,}0 \text{(м)}$
}

\variantsplitter

\addpersonalvariant{Владимир Артемчук}

\tasknumber{1}%
\task{%
    Установите каждой букве в соответствие ровно одну цифру и запишите ответ (только цифры, без других символов).

    А) частота колебаний, Б) время колебаний, В) циклическая частота.

    1) $N$, 2) $\omega$, 3) $\nu$, 4) $\frac{2\pi}{\nu}$, 5) $t$.
}
\answer{%
    $352$
}

\tasknumber{2}%
\task{%
    Установите каждой букве в соответствие ровно одну цифру и запишите ответ (только цифры, без других символов).

    А) циклическая частота, Б) период колебаний.

    1) Гн, 2) рад / с, 3) Гц, 4) c.
}
\answer{%
    $24$
}

\tasknumber{3}%
\task{%
    \begin{itemize}
        \item Запишите линейное однородное дифференциальное уравнение второго порядка,
            описывающее свободные незатухающие колебания гармонического осциллятора,
        \item запишите общее решение этого уравнения,
        \item подпишите в выписанном решении фазу и амплитуду колебаний,
        \item запишите выражение для скорости,
        \item запишите выражение для ускорения.
    \end{itemize}
}
\answer{%
    \begin{align*}
    &\ddot x + \omega^2 x = 0 \Longleftrightarrow a_x + \omega^2 x = 0, \\
    &x = A \cos(\omega t + \varphi_0) \text{ или же } x = A \sin(\omega t + \varphi_0) \text{ или же } x = a \cos(\omega t) + b \sin(\omega t), \\
    &A \text{\, или \,} \sqrt{a^2 + b^2} \text{ --- это амплитуда}, \omega t + \varphi_0\text{ --- это фаза}, \\
    &v = \dot x = -\omega A \sin(\omega t + \varphi_0), \\
    &a = \dot v = \ddot x = -\omega^2 A \cos(\omega t + \varphi_0) = -\omega^2 x,
    \end{align*}
}
\solutionspace{135pt}

\tasknumber{4}%
\task{%
    Определите частоту колебаний, если их период составляет $T = 4\,\text{мс}$.
}
\answer{%
    $\nu = \frac 1T = \frac 1{4\,\text{мс}} = 250\,\text{Гц}$
}
\solutionspace{40pt}

\tasknumber{5}%
\task{%
    Определите период колебаний, если их частота составляет $\nu = 10\,\text{кГц}$.
    Сколько колебаний произойдёт за $t = 2\,\text{мин}$?
}
\answer{%
    \begin{align*}
    T &= \frac 1\nu = \frac 1{10\,\text{кГц}} = 0{,}100\,\text{мc}, \\
    N &= \nu t = 10\,\text{кГц} \cdot2\,\text{мин} = 1200000\,\text{колебаний}.
    \end{align*}
}
\solutionspace{40pt}

\tasknumber{6}%
\task{%
    Амплитуда колебаний точки составляет $A = 2\,\text{см}$, а частота~--- $\nu = 5\,\text{Гц}$.
    Определите, какой путь преодолеет эта точка за $t = 80\,\text{с}$.
}
\answer{%
    $s = 4A \cdot N = 4A \cdot \frac tT = 4A \cdot t\nu = 4 \cdot 2\,\text{см} \cdot 80\,\text{с} \cdot 5\,\text{Гц} = 32{,}0\,\text{м}$
}
\solutionspace{120pt}

\tasknumber{7}%
\task{%
    Изобразите график гармонических колебаний,
    амплитуда которых составляла бы $A = 5\,\text{см}$, а период $T = 8\,\text{с}$.
}
\solutionspace{80pt}

\tasknumber{8}%
\task{%
    Координата материальной точки зависит от времени по закону $x = 0{,}15 \cdot \sin (4\pi t)$ (в СИ).
    Чему равен путь, пройденный точкой за $4\,\text{мин}$?
}
\answer{%
    $\omega = 4\pi \implies \nu = \frac42\,\units{Гц}, N = \nu t = 480{,}0, s = 4AN = 4 \cdot 0{,}15 \cdot 480{,}0 = 288{,}0 \text{(м)}$
}

\variantsplitter

\addpersonalvariant{Софья Белянкина}

\tasknumber{1}%
\task{%
    Установите каждой букве в соответствие ровно одну цифру и запишите ответ (только цифры, без других символов).

    А) время колебаний, Б) число колебаний, В) частота колебаний.

    1) $tN$, 2) $\nu$, 3) $\frac{\nu}{2\pi}$, 4) $t$, 5) $N$.
}
\answer{%
    $452$
}

\tasknumber{2}%
\task{%
    Установите каждой букве в соответствие ровно одну цифру и запишите ответ (только цифры, без других символов).

    А) циклическая частота, Б) период колебаний.

    1) Гц, 2) м / с, 3) рад / с, 4) c.
}
\answer{%
    $34$
}

\tasknumber{3}%
\task{%
    \begin{itemize}
        \item Запишите линейное однородное дифференциальное уравнение второго порядка,
            описывающее свободные незатухающие колебания гармонического осциллятора,
        \item запишите общее решение этого уравнения,
        \item подпишите в выписанном решении фазу и амплитуду колебаний,
        \item запишите выражение для скорости,
        \item запишите выражение для ускорения.
    \end{itemize}
}
\answer{%
    \begin{align*}
    &\ddot x + \omega^2 x = 0 \Longleftrightarrow a_x + \omega^2 x = 0, \\
    &x = A \cos(\omega t + \varphi_0) \text{ или же } x = A \sin(\omega t + \varphi_0) \text{ или же } x = a \cos(\omega t) + b \sin(\omega t), \\
    &A \text{\, или \,} \sqrt{a^2 + b^2} \text{ --- это амплитуда}, \omega t + \varphi_0\text{ --- это фаза}, \\
    &v = \dot x = -\omega A \sin(\omega t + \varphi_0), \\
    &a = \dot v = \ddot x = -\omega^2 A \cos(\omega t + \varphi_0) = -\omega^2 x,
    \end{align*}
}
\solutionspace{135pt}

\tasknumber{4}%
\task{%
    Определите частоту колебаний, если их период составляет $T = 2\,\text{мс}$.
}
\answer{%
    $\nu = \frac 1T = \frac 1{2\,\text{мс}} = 500\,\text{Гц}$
}
\solutionspace{40pt}

\tasknumber{5}%
\task{%
    Определите период колебаний, если их частота составляет $\nu = 5\,\text{кГц}$.
    Сколько колебаний произойдёт за $t = 3\,\text{мин}$?
}
\answer{%
    \begin{align*}
    T &= \frac 1\nu = \frac 1{5\,\text{кГц}} = 0{,}200\,\text{мc}, \\
    N &= \nu t = 5\,\text{кГц} \cdot3\,\text{мин} = 900000\,\text{колебаний}.
    \end{align*}
}
\solutionspace{40pt}

\tasknumber{6}%
\task{%
    Амплитуда колебаний точки составляет $A = 15\,\text{см}$, а частота~--- $\nu = 2\,\text{Гц}$.
    Определите, какой путь преодолеет эта точка за $t = 40\,\text{с}$.
}
\answer{%
    $s = 4A \cdot N = 4A \cdot \frac tT = 4A \cdot t\nu = 4 \cdot 15\,\text{см} \cdot 40\,\text{с} \cdot 2\,\text{Гц} = 48{,}0\,\text{м}$
}
\solutionspace{120pt}

\tasknumber{7}%
\task{%
    Изобразите график гармонических колебаний,
    амплитуда которых составляла бы $A = 3\,\text{см}$, а период $T = 4\,\text{с}$.
}
\solutionspace{80pt}

\tasknumber{8}%
\task{%
    Координата материальной точки зависит от времени по закону $x = 0{,}05 \cdot \cos (6\pi t)$ (в СИ).
    Чему равен путь, пройденный точкой за $3\,\text{мин}$?
}
\answer{%
    $\omega = 6\pi \implies \nu = \frac62\,\units{Гц}, N = \nu t = 540{,}0, s = 4AN = 4 \cdot 0{,}05 \cdot 540{,}0 = 108{,}0 \text{(м)}$
}

\variantsplitter

\addpersonalvariant{Варвара Егиазарян}

\tasknumber{1}%
\task{%
    Установите каждой букве в соответствие ровно одну цифру и запишите ответ (только цифры, без других символов).

    А) период колебаний, Б) число колебаний, В) циклическая частота.

    1) $\frac{2\pi}{\nu}$, 2) $\omega$, 3) $T$, 4) $N$, 5) $t$.
}
\answer{%
    $342$
}

\tasknumber{2}%
\task{%
    Установите каждой букве в соответствие ровно одну цифру и запишите ответ (только цифры, без других символов).

    А) период колебаний, Б) циклическая частота.

    1) Гц, 2) c, 3) рад / с, 4) Гн.
}
\answer{%
    $23$
}

\tasknumber{3}%
\task{%
    \begin{itemize}
        \item Запишите линейное однородное дифференциальное уравнение второго порядка,
            описывающее свободные незатухающие колебания гармонического осциллятора,
        \item запишите общее решение этого уравнения,
        \item подпишите в выписанном решении фазу и амплитуду колебаний,
        \item запишите выражение для скорости,
        \item запишите выражение для ускорения.
    \end{itemize}
}
\answer{%
    \begin{align*}
    &\ddot x + \omega^2 x = 0 \Longleftrightarrow a_x + \omega^2 x = 0, \\
    &x = A \cos(\omega t + \varphi_0) \text{ или же } x = A \sin(\omega t + \varphi_0) \text{ или же } x = a \cos(\omega t) + b \sin(\omega t), \\
    &A \text{\, или \,} \sqrt{a^2 + b^2} \text{ --- это амплитуда}, \omega t + \varphi_0\text{ --- это фаза}, \\
    &v = \dot x = -\omega A \sin(\omega t + \varphi_0), \\
    &a = \dot v = \ddot x = -\omega^2 A \cos(\omega t + \varphi_0) = -\omega^2 x,
    \end{align*}
}
\solutionspace{135pt}

\tasknumber{4}%
\task{%
    Определите частоту колебаний, если их период составляет $T = 2\,\text{мс}$.
}
\answer{%
    $\nu = \frac 1T = \frac 1{2\,\text{мс}} = 500\,\text{Гц}$
}
\solutionspace{40pt}

\tasknumber{5}%
\task{%
    Определите период колебаний, если их частота составляет $\nu = 20\,\text{кГц}$.
    Сколько колебаний произойдёт за $t = 2\,\text{мин}$?
}
\answer{%
    \begin{align*}
    T &= \frac 1\nu = \frac 1{20\,\text{кГц}} = 0{,}050\,\text{мc}, \\
    N &= \nu t = 20\,\text{кГц} \cdot2\,\text{мин} = 2400000\,\text{колебаний}.
    \end{align*}
}
\solutionspace{40pt}

\tasknumber{6}%
\task{%
    Амплитуда колебаний точки составляет $A = 3\,\text{см}$, а частота~--- $\nu = 2\,\text{Гц}$.
    Определите, какой путь преодолеет эта точка за $t = 10\,\text{с}$.
}
\answer{%
    $s = 4A \cdot N = 4A \cdot \frac tT = 4A \cdot t\nu = 4 \cdot 3\,\text{см} \cdot 10\,\text{с} \cdot 2\,\text{Гц} = 2{,}4\,\text{м}$
}
\solutionspace{120pt}

\tasknumber{7}%
\task{%
    Изобразите график гармонических колебаний,
    амплитуда которых составляла бы $A = 40\,\text{см}$, а период $T = 8\,\text{с}$.
}
\solutionspace{80pt}

\tasknumber{8}%
\task{%
    Координата материальной точки зависит от времени по закону $x = 0{,}05 \cdot \sin (4\pi t)$ (в СИ).
    Чему равен путь, пройденный точкой за $2\,\text{мин}$?
}
\answer{%
    $\omega = 4\pi \implies \nu = \frac42\,\units{Гц}, N = \nu t = 240{,}0, s = 4AN = 4 \cdot 0{,}05 \cdot 240{,}0 = 48{,}0 \text{(м)}$
}

\variantsplitter

\addpersonalvariant{Владислав Емелин}

\tasknumber{1}%
\task{%
    Установите каждой букве в соответствие ровно одну цифру и запишите ответ (только цифры, без других символов).

    А) циклическая частота, Б) период колебаний, В) число колебаний.

    1) $\omega$, 2) $tN$, 3) $\frac{\nu}{2\pi}$, 4) $N$, 5) $T$.
}
\answer{%
    $154$
}

\tasknumber{2}%
\task{%
    Установите каждой букве в соответствие ровно одну цифру и запишите ответ (только цифры, без других символов).

    А) циклическая частота, Б) период колебаний.

    1) м / с, 2) Гц, 3) c, 4) рад / с.
}
\answer{%
    $43$
}

\tasknumber{3}%
\task{%
    \begin{itemize}
        \item Запишите линейное однородное дифференциальное уравнение второго порядка,
            описывающее свободные незатухающие колебания гармонического осциллятора,
        \item запишите общее решение этого уравнения,
        \item подпишите в выписанном решении фазу и амплитуду колебаний,
        \item запишите выражение для скорости,
        \item запишите выражение для ускорения.
    \end{itemize}
}
\answer{%
    \begin{align*}
    &\ddot x + \omega^2 x = 0 \Longleftrightarrow a_x + \omega^2 x = 0, \\
    &x = A \cos(\omega t + \varphi_0) \text{ или же } x = A \sin(\omega t + \varphi_0) \text{ или же } x = a \cos(\omega t) + b \sin(\omega t), \\
    &A \text{\, или \,} \sqrt{a^2 + b^2} \text{ --- это амплитуда}, \omega t + \varphi_0\text{ --- это фаза}, \\
    &v = \dot x = -\omega A \sin(\omega t + \varphi_0), \\
    &a = \dot v = \ddot x = -\omega^2 A \cos(\omega t + \varphi_0) = -\omega^2 x,
    \end{align*}
}
\solutionspace{135pt}

\tasknumber{4}%
\task{%
    Определите частоту колебаний, если их период составляет $T = 50\,\text{мс}$.
}
\answer{%
    $\nu = \frac 1T = \frac 1{50\,\text{мс}} = 20\,\text{Гц}$
}
\solutionspace{40pt}

\tasknumber{5}%
\task{%
    Определите период колебаний, если их частота составляет $\nu = 40\,\text{кГц}$.
    Сколько колебаний произойдёт за $t = 1\,\text{мин}$?
}
\answer{%
    \begin{align*}
    T &= \frac 1\nu = \frac 1{40\,\text{кГц}} = 0{,}025\,\text{мc}, \\
    N &= \nu t = 40\,\text{кГц} \cdot1\,\text{мин} = 2400000\,\text{колебаний}.
    \end{align*}
}
\solutionspace{40pt}

\tasknumber{6}%
\task{%
    Амплитуда колебаний точки составляет $A = 3\,\text{см}$, а частота~--- $\nu = 20\,\text{Гц}$.
    Определите, какой путь преодолеет эта точка за $t = 40\,\text{с}$.
}
\answer{%
    $s = 4A \cdot N = 4A \cdot \frac tT = 4A \cdot t\nu = 4 \cdot 3\,\text{см} \cdot 40\,\text{с} \cdot 20\,\text{Гц} = 96{,}0\,\text{м}$
}
\solutionspace{120pt}

\tasknumber{7}%
\task{%
    Изобразите график гармонических колебаний,
    амплитуда которых составляла бы $A = 30\,\text{см}$, а период $T = 8\,\text{с}$.
}
\solutionspace{80pt}

\tasknumber{8}%
\task{%
    Координата материальной точки зависит от времени по закону $x = 0{,}05 \cdot \sin (5\pi t)$ (в СИ).
    Чему равен путь, пройденный точкой за $4\,\text{мин}$?
}
\answer{%
    $\omega = 5\pi \implies \nu = \frac52\,\units{Гц}, N = \nu t = 600{,}0, s = 4AN = 4 \cdot 0{,}05 \cdot 600{,}0 = 120{,}0 \text{(м)}$
}

\variantsplitter

\addpersonalvariant{Артём Жичин}

\tasknumber{1}%
\task{%
    Установите каждой букве в соответствие ровно одну цифру и запишите ответ (только цифры, без других символов).

    А) частота колебаний, Б) период колебаний, В) циклическая частота.

    1) $\frac{2\pi}{\nu}$, 2) $\frac{\nu}{2\pi}$, 3) $\omega$, 4) $T$, 5) $\nu$.
}
\answer{%
    $543$
}

\tasknumber{2}%
\task{%
    Установите каждой букве в соответствие ровно одну цифру и запишите ответ (только цифры, без других символов).

    А) период колебаний, Б) частота колебаний.

    1) рад / с, 2) Гн, 3) Гц, 4) c.
}
\answer{%
    $43$
}

\tasknumber{3}%
\task{%
    \begin{itemize}
        \item Запишите линейное однородное дифференциальное уравнение второго порядка,
            описывающее свободные незатухающие колебания гармонического осциллятора,
        \item запишите общее решение этого уравнения,
        \item подпишите в выписанном решении фазу и амплитуду колебаний,
        \item запишите выражение для скорости,
        \item запишите выражение для ускорения.
    \end{itemize}
}
\answer{%
    \begin{align*}
    &\ddot x + \omega^2 x = 0 \Longleftrightarrow a_x + \omega^2 x = 0, \\
    &x = A \cos(\omega t + \varphi_0) \text{ или же } x = A \sin(\omega t + \varphi_0) \text{ или же } x = a \cos(\omega t) + b \sin(\omega t), \\
    &A \text{\, или \,} \sqrt{a^2 + b^2} \text{ --- это амплитуда}, \omega t + \varphi_0\text{ --- это фаза}, \\
    &v = \dot x = -\omega A \sin(\omega t + \varphi_0), \\
    &a = \dot v = \ddot x = -\omega^2 A \cos(\omega t + \varphi_0) = -\omega^2 x,
    \end{align*}
}
\solutionspace{135pt}

\tasknumber{4}%
\task{%
    Определите частоту колебаний, если их период составляет $T = 4\,\text{мс}$.
}
\answer{%
    $\nu = \frac 1T = \frac 1{4\,\text{мс}} = 250\,\text{Гц}$
}
\solutionspace{40pt}

\tasknumber{5}%
\task{%
    Определите период колебаний, если их частота составляет $\nu = 50\,\text{кГц}$.
    Сколько колебаний произойдёт за $t = 2\,\text{мин}$?
}
\answer{%
    \begin{align*}
    T &= \frac 1\nu = \frac 1{50\,\text{кГц}} = 0{,}020\,\text{мc}, \\
    N &= \nu t = 50\,\text{кГц} \cdot2\,\text{мин} = 6000000\,\text{колебаний}.
    \end{align*}
}
\solutionspace{40pt}

\tasknumber{6}%
\task{%
    Амплитуда колебаний точки составляет $A = 5\,\text{см}$, а частота~--- $\nu = 10\,\text{Гц}$.
    Определите, какой путь преодолеет эта точка за $t = 10\,\text{с}$.
}
\answer{%
    $s = 4A \cdot N = 4A \cdot \frac tT = 4A \cdot t\nu = 4 \cdot 5\,\text{см} \cdot 10\,\text{с} \cdot 10\,\text{Гц} = 20{,}0\,\text{м}$
}
\solutionspace{120pt}

\tasknumber{7}%
\task{%
    Изобразите график гармонических колебаний,
    амплитуда которых составляла бы $A = 15\,\text{см}$, а период $T = 10\,\text{с}$.
}
\solutionspace{80pt}

\tasknumber{8}%
\task{%
    Координата материальной точки зависит от времени по закону $z = 0{,}02 \cdot \sin (6\pi t)$ (в СИ).
    Чему равен путь, пройденный точкой за $4\,\text{мин}$?
}
\answer{%
    $\omega = 6\pi \implies \nu = \frac62\,\units{Гц}, N = \nu t = 720{,}0, s = 4AN = 4 \cdot 0{,}02 \cdot 720{,}0 = 57{,}6 \text{(м)}$
}

\variantsplitter

\addpersonalvariant{Дарья Кошман}

\tasknumber{1}%
\task{%
    Установите каждой букве в соответствие ровно одну цифру и запишите ответ (только цифры, без других символов).

    А) время колебаний, Б) число колебаний, В) частота колебаний.

    1) $\nu$, 2) $\frac{\nu}{2\pi}$, 3) $tN$, 4) $N$, 5) $t$.
}
\answer{%
    $541$
}

\tasknumber{2}%
\task{%
    Установите каждой букве в соответствие ровно одну цифру и запишите ответ (только цифры, без других символов).

    А) циклическая частота, Б) период колебаний.

    1) Гц, 2) Гн, 3) c, 4) рад / с.
}
\answer{%
    $43$
}

\tasknumber{3}%
\task{%
    \begin{itemize}
        \item Запишите линейное однородное дифференциальное уравнение второго порядка,
            описывающее свободные незатухающие колебания гармонического осциллятора,
        \item запишите общее решение этого уравнения,
        \item подпишите в выписанном решении фазу и амплитуду колебаний,
        \item запишите выражение для скорости,
        \item запишите выражение для ускорения.
    \end{itemize}
}
\answer{%
    \begin{align*}
    &\ddot x + \omega^2 x = 0 \Longleftrightarrow a_x + \omega^2 x = 0, \\
    &x = A \cos(\omega t + \varphi_0) \text{ или же } x = A \sin(\omega t + \varphi_0) \text{ или же } x = a \cos(\omega t) + b \sin(\omega t), \\
    &A \text{\, или \,} \sqrt{a^2 + b^2} \text{ --- это амплитуда}, \omega t + \varphi_0\text{ --- это фаза}, \\
    &v = \dot x = -\omega A \sin(\omega t + \varphi_0), \\
    &a = \dot v = \ddot x = -\omega^2 A \cos(\omega t + \varphi_0) = -\omega^2 x,
    \end{align*}
}
\solutionspace{135pt}

\tasknumber{4}%
\task{%
    Определите частоту колебаний, если их период составляет $T = 10\,\text{мс}$.
}
\answer{%
    $\nu = \frac 1T = \frac 1{10\,\text{мс}} = 100\,\text{Гц}$
}
\solutionspace{40pt}

\tasknumber{5}%
\task{%
    Определите период колебаний, если их частота составляет $\nu = 4\,\text{кГц}$.
    Сколько колебаний произойдёт за $t = 2\,\text{мин}$?
}
\answer{%
    \begin{align*}
    T &= \frac 1\nu = \frac 1{4\,\text{кГц}} = 0{,}250\,\text{мc}, \\
    N &= \nu t = 4\,\text{кГц} \cdot2\,\text{мин} = 480000\,\text{колебаний}.
    \end{align*}
}
\solutionspace{40pt}

\tasknumber{6}%
\task{%
    Амплитуда колебаний точки составляет $A = 3\,\text{см}$, а частота~--- $\nu = 20\,\text{Гц}$.
    Определите, какой путь преодолеет эта точка за $t = 10\,\text{с}$.
}
\answer{%
    $s = 4A \cdot N = 4A \cdot \frac tT = 4A \cdot t\nu = 4 \cdot 3\,\text{см} \cdot 10\,\text{с} \cdot 20\,\text{Гц} = 24{,}0\,\text{м}$
}
\solutionspace{120pt}

\tasknumber{7}%
\task{%
    Изобразите график гармонических колебаний,
    амплитуда которых составляла бы $A = 2\,\text{см}$, а период $T = 10\,\text{с}$.
}
\solutionspace{80pt}

\tasknumber{8}%
\task{%
    Координата материальной точки зависит от времени по закону $y = 0{,}25 \cdot \cos (6\pi t)$ (в СИ).
    Чему равен путь, пройденный точкой за $2\,\text{мин}$?
}
\answer{%
    $\omega = 6\pi \implies \nu = \frac62\,\units{Гц}, N = \nu t = 360{,}0, s = 4AN = 4 \cdot 0{,}25 \cdot 360{,}0 = 360{,}0 \text{(м)}$
}

\variantsplitter

\addpersonalvariant{Анна Кузьмичёва}

\tasknumber{1}%
\task{%
    Установите каждой букве в соответствие ровно одну цифру и запишите ответ (только цифры, без других символов).

    А) циклическая частота, Б) время колебаний, В) период колебаний.

    1) $\omega$, 2) $N$, 3) $t$, 4) $T$, 5) $\frac{\nu}{2\pi}$.
}
\answer{%
    $134$
}

\tasknumber{2}%
\task{%
    Установите каждой букве в соответствие ровно одну цифру и запишите ответ (только цифры, без других символов).

    А) частота колебаний, Б) период колебаний.

    1) Гн, 2) Гц, 3) c, 4) рад / с.
}
\answer{%
    $23$
}

\tasknumber{3}%
\task{%
    \begin{itemize}
        \item Запишите линейное однородное дифференциальное уравнение второго порядка,
            описывающее свободные незатухающие колебания гармонического осциллятора,
        \item запишите общее решение этого уравнения,
        \item подпишите в выписанном решении фазу и амплитуду колебаний,
        \item запишите выражение для скорости,
        \item запишите выражение для ускорения.
    \end{itemize}
}
\answer{%
    \begin{align*}
    &\ddot x + \omega^2 x = 0 \Longleftrightarrow a_x + \omega^2 x = 0, \\
    &x = A \cos(\omega t + \varphi_0) \text{ или же } x = A \sin(\omega t + \varphi_0) \text{ или же } x = a \cos(\omega t) + b \sin(\omega t), \\
    &A \text{\, или \,} \sqrt{a^2 + b^2} \text{ --- это амплитуда}, \omega t + \varphi_0\text{ --- это фаза}, \\
    &v = \dot x = -\omega A \sin(\omega t + \varphi_0), \\
    &a = \dot v = \ddot x = -\omega^2 A \cos(\omega t + \varphi_0) = -\omega^2 x,
    \end{align*}
}
\solutionspace{135pt}

\tasknumber{4}%
\task{%
    Определите частоту колебаний, если их период составляет $T = 4\,\text{мс}$.
}
\answer{%
    $\nu = \frac 1T = \frac 1{4\,\text{мс}} = 250\,\text{Гц}$
}
\solutionspace{40pt}

\tasknumber{5}%
\task{%
    Определите период колебаний, если их частота составляет $\nu = 40\,\text{кГц}$.
    Сколько колебаний произойдёт за $t = 2\,\text{мин}$?
}
\answer{%
    \begin{align*}
    T &= \frac 1\nu = \frac 1{40\,\text{кГц}} = 0{,}025\,\text{мc}, \\
    N &= \nu t = 40\,\text{кГц} \cdot2\,\text{мин} = 4800000\,\text{колебаний}.
    \end{align*}
}
\solutionspace{40pt}

\tasknumber{6}%
\task{%
    Амплитуда колебаний точки составляет $A = 5\,\text{см}$, а частота~--- $\nu = 20\,\text{Гц}$.
    Определите, какой путь преодолеет эта точка за $t = 80\,\text{с}$.
}
\answer{%
    $s = 4A \cdot N = 4A \cdot \frac tT = 4A \cdot t\nu = 4 \cdot 5\,\text{см} \cdot 80\,\text{с} \cdot 20\,\text{Гц} = 320{,}0\,\text{м}$
}
\solutionspace{120pt}

\tasknumber{7}%
\task{%
    Изобразите график гармонических колебаний,
    амплитуда которых составляла бы $A = 2\,\text{см}$, а период $T = 8\,\text{с}$.
}
\solutionspace{80pt}

\tasknumber{8}%
\task{%
    Координата материальной точки зависит от времени по закону $x = 0{,}02 \cdot \cos (6\pi t)$ (в СИ).
    Чему равен путь, пройденный точкой за $2\,\text{мин}$?
}
\answer{%
    $\omega = 6\pi \implies \nu = \frac62\,\units{Гц}, N = \nu t = 360{,}0, s = 4AN = 4 \cdot 0{,}02 \cdot 360{,}0 = 28{,}8 \text{(м)}$
}

\variantsplitter

\addpersonalvariant{Алёна Куприянова}

\tasknumber{1}%
\task{%
    Установите каждой букве в соответствие ровно одну цифру и запишите ответ (только цифры, без других символов).

    А) циклическая частота, Б) число колебаний, В) частота колебаний.

    1) $\omega$, 2) $N$, 3) $\nu$, 4) $tN$, 5) $\frac{2\pi}{\nu}$.
}
\answer{%
    $123$
}

\tasknumber{2}%
\task{%
    Установите каждой букве в соответствие ровно одну цифру и запишите ответ (только цифры, без других символов).

    А) циклическая частота, Б) период колебаний.

    1) рад / с, 2) c, 3) м / с, 4) Гц.
}
\answer{%
    $12$
}

\tasknumber{3}%
\task{%
    \begin{itemize}
        \item Запишите линейное однородное дифференциальное уравнение второго порядка,
            описывающее свободные незатухающие колебания гармонического осциллятора,
        \item запишите общее решение этого уравнения,
        \item подпишите в выписанном решении фазу и амплитуду колебаний,
        \item запишите выражение для скорости,
        \item запишите выражение для ускорения.
    \end{itemize}
}
\answer{%
    \begin{align*}
    &\ddot x + \omega^2 x = 0 \Longleftrightarrow a_x + \omega^2 x = 0, \\
    &x = A \cos(\omega t + \varphi_0) \text{ или же } x = A \sin(\omega t + \varphi_0) \text{ или же } x = a \cos(\omega t) + b \sin(\omega t), \\
    &A \text{\, или \,} \sqrt{a^2 + b^2} \text{ --- это амплитуда}, \omega t + \varphi_0\text{ --- это фаза}, \\
    &v = \dot x = -\omega A \sin(\omega t + \varphi_0), \\
    &a = \dot v = \ddot x = -\omega^2 A \cos(\omega t + \varphi_0) = -\omega^2 x,
    \end{align*}
}
\solutionspace{135pt}

\tasknumber{4}%
\task{%
    Определите частоту колебаний, если их период составляет $T = 40\,\text{мс}$.
}
\answer{%
    $\nu = \frac 1T = \frac 1{40\,\text{мс}} = 25\,\text{Гц}$
}
\solutionspace{40pt}

\tasknumber{5}%
\task{%
    Определите период колебаний, если их частота составляет $\nu = 2\,\text{кГц}$.
    Сколько колебаний произойдёт за $t = 5\,\text{мин}$?
}
\answer{%
    \begin{align*}
    T &= \frac 1\nu = \frac 1{2\,\text{кГц}} = 0{,}500\,\text{мc}, \\
    N &= \nu t = 2\,\text{кГц} \cdot5\,\text{мин} = 600000\,\text{колебаний}.
    \end{align*}
}
\solutionspace{40pt}

\tasknumber{6}%
\task{%
    Амплитуда колебаний точки составляет $A = 3\,\text{см}$, а частота~--- $\nu = 20\,\text{Гц}$.
    Определите, какой путь преодолеет эта точка за $t = 10\,\text{с}$.
}
\answer{%
    $s = 4A \cdot N = 4A \cdot \frac tT = 4A \cdot t\nu = 4 \cdot 3\,\text{см} \cdot 10\,\text{с} \cdot 20\,\text{Гц} = 24{,}0\,\text{м}$
}
\solutionspace{120pt}

\tasknumber{7}%
\task{%
    Изобразите график гармонических колебаний,
    амплитуда которых составляла бы $A = 3\,\text{см}$, а период $T = 8\,\text{с}$.
}
\solutionspace{80pt}

\tasknumber{8}%
\task{%
    Координата материальной точки зависит от времени по закону $z = 0{,}02 \cdot \cos (6\pi t)$ (в СИ).
    Чему равен путь, пройденный точкой за $2\,\text{мин}$?
}
\answer{%
    $\omega = 6\pi \implies \nu = \frac62\,\units{Гц}, N = \nu t = 360{,}0, s = 4AN = 4 \cdot 0{,}02 \cdot 360{,}0 = 28{,}8 \text{(м)}$
}

\variantsplitter

\addpersonalvariant{Ярослав Лавровский}

\tasknumber{1}%
\task{%
    Установите каждой букве в соответствие ровно одну цифру и запишите ответ (только цифры, без других символов).

    А) циклическая частота, Б) число колебаний, В) период колебаний.

    1) $\omega$, 2) $T$, 3) $N$, 4) $tN$, 5) $\frac{\nu}{2\pi}$.
}
\answer{%
    $132$
}

\tasknumber{2}%
\task{%
    Установите каждой букве в соответствие ровно одну цифру и запишите ответ (только цифры, без других символов).

    А) частота колебаний, Б) период колебаний.

    1) c, 2) Гц, 3) Гн, 4) рад / с.
}
\answer{%
    $21$
}

\tasknumber{3}%
\task{%
    \begin{itemize}
        \item Запишите линейное однородное дифференциальное уравнение второго порядка,
            описывающее свободные незатухающие колебания гармонического осциллятора,
        \item запишите общее решение этого уравнения,
        \item подпишите в выписанном решении фазу и амплитуду колебаний,
        \item запишите выражение для скорости,
        \item запишите выражение для ускорения.
    \end{itemize}
}
\answer{%
    \begin{align*}
    &\ddot x + \omega^2 x = 0 \Longleftrightarrow a_x + \omega^2 x = 0, \\
    &x = A \cos(\omega t + \varphi_0) \text{ или же } x = A \sin(\omega t + \varphi_0) \text{ или же } x = a \cos(\omega t) + b \sin(\omega t), \\
    &A \text{\, или \,} \sqrt{a^2 + b^2} \text{ --- это амплитуда}, \omega t + \varphi_0\text{ --- это фаза}, \\
    &v = \dot x = -\omega A \sin(\omega t + \varphi_0), \\
    &a = \dot v = \ddot x = -\omega^2 A \cos(\omega t + \varphi_0) = -\omega^2 x,
    \end{align*}
}
\solutionspace{135pt}

\tasknumber{4}%
\task{%
    Определите частоту колебаний, если их период составляет $T = 50\,\text{мс}$.
}
\answer{%
    $\nu = \frac 1T = \frac 1{50\,\text{мс}} = 20\,\text{Гц}$
}
\solutionspace{40pt}

\tasknumber{5}%
\task{%
    Определите период колебаний, если их частота составляет $\nu = 10\,\text{кГц}$.
    Сколько колебаний произойдёт за $t = 5\,\text{мин}$?
}
\answer{%
    \begin{align*}
    T &= \frac 1\nu = \frac 1{10\,\text{кГц}} = 0{,}100\,\text{мc}, \\
    N &= \nu t = 10\,\text{кГц} \cdot5\,\text{мин} = 3000000\,\text{колебаний}.
    \end{align*}
}
\solutionspace{40pt}

\tasknumber{6}%
\task{%
    Амплитуда колебаний точки составляет $A = 15\,\text{см}$, а частота~--- $\nu = 5\,\text{Гц}$.
    Определите, какой путь преодолеет эта точка за $t = 80\,\text{с}$.
}
\answer{%
    $s = 4A \cdot N = 4A \cdot \frac tT = 4A \cdot t\nu = 4 \cdot 15\,\text{см} \cdot 80\,\text{с} \cdot 5\,\text{Гц} = 240{,}0\,\text{м}$
}
\solutionspace{120pt}

\tasknumber{7}%
\task{%
    Изобразите график гармонических колебаний,
    амплитуда которых составляла бы $A = 2\,\text{см}$, а период $T = 10\,\text{с}$.
}
\solutionspace{80pt}

\tasknumber{8}%
\task{%
    Координата материальной точки зависит от времени по закону $z = 0{,}02 \cdot \sin (3\pi t)$ (в СИ).
    Чему равен путь, пройденный точкой за $3\,\text{мин}$?
}
\answer{%
    $\omega = 3\pi \implies \nu = \frac32\,\units{Гц}, N = \nu t = 270{,}0, s = 4AN = 4 \cdot 0{,}02 \cdot 270{,}0 = 21{,}6 \text{(м)}$
}

\variantsplitter

\addpersonalvariant{Анастасия Ламанова}

\tasknumber{1}%
\task{%
    Установите каждой букве в соответствие ровно одну цифру и запишите ответ (только цифры, без других символов).

    А) частота колебаний, Б) период колебаний, В) циклическая частота.

    1) $T$, 2) $tN$, 3) $\omega$, 4) $\frac{\nu}{2\pi}$, 5) $\nu$.
}
\answer{%
    $513$
}

\tasknumber{2}%
\task{%
    Установите каждой букве в соответствие ровно одну цифру и запишите ответ (только цифры, без других символов).

    А) частота колебаний, Б) циклическая частота.

    1) c, 2) рад / с, 3) Гн, 4) Гц.
}
\answer{%
    $42$
}

\tasknumber{3}%
\task{%
    \begin{itemize}
        \item Запишите линейное однородное дифференциальное уравнение второго порядка,
            описывающее свободные незатухающие колебания гармонического осциллятора,
        \item запишите общее решение этого уравнения,
        \item подпишите в выписанном решении фазу и амплитуду колебаний,
        \item запишите выражение для скорости,
        \item запишите выражение для ускорения.
    \end{itemize}
}
\answer{%
    \begin{align*}
    &\ddot x + \omega^2 x = 0 \Longleftrightarrow a_x + \omega^2 x = 0, \\
    &x = A \cos(\omega t + \varphi_0) \text{ или же } x = A \sin(\omega t + \varphi_0) \text{ или же } x = a \cos(\omega t) + b \sin(\omega t), \\
    &A \text{\, или \,} \sqrt{a^2 + b^2} \text{ --- это амплитуда}, \omega t + \varphi_0\text{ --- это фаза}, \\
    &v = \dot x = -\omega A \sin(\omega t + \varphi_0), \\
    &a = \dot v = \ddot x = -\omega^2 A \cos(\omega t + \varphi_0) = -\omega^2 x,
    \end{align*}
}
\solutionspace{135pt}

\tasknumber{4}%
\task{%
    Определите частоту колебаний, если их период составляет $T = 5\,\text{мс}$.
}
\answer{%
    $\nu = \frac 1T = \frac 1{5\,\text{мс}} = 200\,\text{Гц}$
}
\solutionspace{40pt}

\tasknumber{5}%
\task{%
    Определите период колебаний, если их частота составляет $\nu = 40\,\text{кГц}$.
    Сколько колебаний произойдёт за $t = 5\,\text{мин}$?
}
\answer{%
    \begin{align*}
    T &= \frac 1\nu = \frac 1{40\,\text{кГц}} = 0{,}025\,\text{мc}, \\
    N &= \nu t = 40\,\text{кГц} \cdot5\,\text{мин} = 12000000\,\text{колебаний}.
    \end{align*}
}
\solutionspace{40pt}

\tasknumber{6}%
\task{%
    Амплитуда колебаний точки составляет $A = 2\,\text{см}$, а частота~--- $\nu = 5\,\text{Гц}$.
    Определите, какой путь преодолеет эта точка за $t = 40\,\text{с}$.
}
\answer{%
    $s = 4A \cdot N = 4A \cdot \frac tT = 4A \cdot t\nu = 4 \cdot 2\,\text{см} \cdot 40\,\text{с} \cdot 5\,\text{Гц} = 16{,}0\,\text{м}$
}
\solutionspace{120pt}

\tasknumber{7}%
\task{%
    Изобразите график гармонических колебаний,
    амплитуда которых составляла бы $A = 40\,\text{см}$, а период $T = 8\,\text{с}$.
}
\solutionspace{80pt}

\tasknumber{8}%
\task{%
    Координата материальной точки зависит от времени по закону $x = 0{,}05 \cdot \cos (3\pi t)$ (в СИ).
    Чему равен путь, пройденный точкой за $2\,\text{мин}$?
}
\answer{%
    $\omega = 3\pi \implies \nu = \frac32\,\units{Гц}, N = \nu t = 180{,}0, s = 4AN = 4 \cdot 0{,}05 \cdot 180{,}0 = 36{,}0 \text{(м)}$
}

\variantsplitter

\addpersonalvariant{Виктория Легонькова}

\tasknumber{1}%
\task{%
    Установите каждой букве в соответствие ровно одну цифру и запишите ответ (только цифры, без других символов).

    А) число колебаний, Б) время колебаний, В) период колебаний.

    1) $\frac{2\pi}{\nu}$, 2) $N$, 3) $T$, 4) $\omega$, 5) $t$.
}
\answer{%
    $253$
}

\tasknumber{2}%
\task{%
    Установите каждой букве в соответствие ровно одну цифру и запишите ответ (только цифры, без других символов).

    А) циклическая частота, Б) период колебаний.

    1) рад / с, 2) Гн, 3) м / с, 4) c.
}
\answer{%
    $14$
}

\tasknumber{3}%
\task{%
    \begin{itemize}
        \item Запишите линейное однородное дифференциальное уравнение второго порядка,
            описывающее свободные незатухающие колебания гармонического осциллятора,
        \item запишите общее решение этого уравнения,
        \item подпишите в выписанном решении фазу и амплитуду колебаний,
        \item запишите выражение для скорости,
        \item запишите выражение для ускорения.
    \end{itemize}
}
\answer{%
    \begin{align*}
    &\ddot x + \omega^2 x = 0 \Longleftrightarrow a_x + \omega^2 x = 0, \\
    &x = A \cos(\omega t + \varphi_0) \text{ или же } x = A \sin(\omega t + \varphi_0) \text{ или же } x = a \cos(\omega t) + b \sin(\omega t), \\
    &A \text{\, или \,} \sqrt{a^2 + b^2} \text{ --- это амплитуда}, \omega t + \varphi_0\text{ --- это фаза}, \\
    &v = \dot x = -\omega A \sin(\omega t + \varphi_0), \\
    &a = \dot v = \ddot x = -\omega^2 A \cos(\omega t + \varphi_0) = -\omega^2 x,
    \end{align*}
}
\solutionspace{135pt}

\tasknumber{4}%
\task{%
    Определите частоту колебаний, если их период составляет $T = 10\,\text{мс}$.
}
\answer{%
    $\nu = \frac 1T = \frac 1{10\,\text{мс}} = 100\,\text{Гц}$
}
\solutionspace{40pt}

\tasknumber{5}%
\task{%
    Определите период колебаний, если их частота составляет $\nu = 10\,\text{кГц}$.
    Сколько колебаний произойдёт за $t = 2\,\text{мин}$?
}
\answer{%
    \begin{align*}
    T &= \frac 1\nu = \frac 1{10\,\text{кГц}} = 0{,}100\,\text{мc}, \\
    N &= \nu t = 10\,\text{кГц} \cdot2\,\text{мин} = 1200000\,\text{колебаний}.
    \end{align*}
}
\solutionspace{40pt}

\tasknumber{6}%
\task{%
    Амплитуда колебаний точки составляет $A = 3\,\text{см}$, а частота~--- $\nu = 20\,\text{Гц}$.
    Определите, какой путь преодолеет эта точка за $t = 80\,\text{с}$.
}
\answer{%
    $s = 4A \cdot N = 4A \cdot \frac tT = 4A \cdot t\nu = 4 \cdot 3\,\text{см} \cdot 80\,\text{с} \cdot 20\,\text{Гц} = 192{,}0\,\text{м}$
}
\solutionspace{120pt}

\tasknumber{7}%
\task{%
    Изобразите график гармонических колебаний,
    амплитуда которых составляла бы $A = 2\,\text{см}$, а период $T = 10\,\text{с}$.
}
\solutionspace{80pt}

\tasknumber{8}%
\task{%
    Координата материальной точки зависит от времени по закону $z = 0{,}15 \cdot \cos (5\pi t)$ (в СИ).
    Чему равен путь, пройденный точкой за $3\,\text{мин}$?
}
\answer{%
    $\omega = 5\pi \implies \nu = \frac52\,\units{Гц}, N = \nu t = 450{,}0, s = 4AN = 4 \cdot 0{,}15 \cdot 450{,}0 = 270{,}0 \text{(м)}$
}

\variantsplitter

\addpersonalvariant{Семён Мартынов}

\tasknumber{1}%
\task{%
    Установите каждой букве в соответствие ровно одну цифру и запишите ответ (только цифры, без других символов).

    А) частота колебаний, Б) число колебаний, В) время колебаний.

    1) $\omega$, 2) $t$, 3) $\nu$, 4) $N$, 5) $tN$.
}
\answer{%
    $342$
}

\tasknumber{2}%
\task{%
    Установите каждой букве в соответствие ровно одну цифру и запишите ответ (только цифры, без других символов).

    А) частота колебаний, Б) циклическая частота.

    1) м / с, 2) Гц, 3) рад / с, 4) Гн.
}
\answer{%
    $23$
}

\tasknumber{3}%
\task{%
    \begin{itemize}
        \item Запишите линейное однородное дифференциальное уравнение второго порядка,
            описывающее свободные незатухающие колебания гармонического осциллятора,
        \item запишите общее решение этого уравнения,
        \item подпишите в выписанном решении фазу и амплитуду колебаний,
        \item запишите выражение для скорости,
        \item запишите выражение для ускорения.
    \end{itemize}
}
\answer{%
    \begin{align*}
    &\ddot x + \omega^2 x = 0 \Longleftrightarrow a_x + \omega^2 x = 0, \\
    &x = A \cos(\omega t + \varphi_0) \text{ или же } x = A \sin(\omega t + \varphi_0) \text{ или же } x = a \cos(\omega t) + b \sin(\omega t), \\
    &A \text{\, или \,} \sqrt{a^2 + b^2} \text{ --- это амплитуда}, \omega t + \varphi_0\text{ --- это фаза}, \\
    &v = \dot x = -\omega A \sin(\omega t + \varphi_0), \\
    &a = \dot v = \ddot x = -\omega^2 A \cos(\omega t + \varphi_0) = -\omega^2 x,
    \end{align*}
}
\solutionspace{135pt}

\tasknumber{4}%
\task{%
    Определите частоту колебаний, если их период составляет $T = 50\,\text{мс}$.
}
\answer{%
    $\nu = \frac 1T = \frac 1{50\,\text{мс}} = 20\,\text{Гц}$
}
\solutionspace{40pt}

\tasknumber{5}%
\task{%
    Определите период колебаний, если их частота составляет $\nu = 50\,\text{кГц}$.
    Сколько колебаний произойдёт за $t = 3\,\text{мин}$?
}
\answer{%
    \begin{align*}
    T &= \frac 1\nu = \frac 1{50\,\text{кГц}} = 0{,}020\,\text{мc}, \\
    N &= \nu t = 50\,\text{кГц} \cdot3\,\text{мин} = 9000000\,\text{колебаний}.
    \end{align*}
}
\solutionspace{40pt}

\tasknumber{6}%
\task{%
    Амплитуда колебаний точки составляет $A = 2\,\text{см}$, а частота~--- $\nu = 10\,\text{Гц}$.
    Определите, какой путь преодолеет эта точка за $t = 10\,\text{с}$.
}
\answer{%
    $s = 4A \cdot N = 4A \cdot \frac tT = 4A \cdot t\nu = 4 \cdot 2\,\text{см} \cdot 10\,\text{с} \cdot 10\,\text{Гц} = 8{,}0\,\text{м}$
}
\solutionspace{120pt}

\tasknumber{7}%
\task{%
    Изобразите график гармонических колебаний,
    амплитуда которых составляла бы $A = 5\,\text{см}$, а период $T = 6\,\text{с}$.
}
\solutionspace{80pt}

\tasknumber{8}%
\task{%
    Координата материальной точки зависит от времени по закону $y = 0{,}25 \cdot \sin (6\pi t)$ (в СИ).
    Чему равен путь, пройденный точкой за $2\,\text{мин}$?
}
\answer{%
    $\omega = 6\pi \implies \nu = \frac62\,\units{Гц}, N = \nu t = 360{,}0, s = 4AN = 4 \cdot 0{,}25 \cdot 360{,}0 = 360{,}0 \text{(м)}$
}

\variantsplitter

\addpersonalvariant{Варвара Минаева}

\tasknumber{1}%
\task{%
    Установите каждой букве в соответствие ровно одну цифру и запишите ответ (только цифры, без других символов).

    А) частота колебаний, Б) период колебаний, В) циклическая частота.

    1) $T$, 2) $N$, 3) $tN$, 4) $\omega$, 5) $\nu$.
}
\answer{%
    $514$
}

\tasknumber{2}%
\task{%
    Установите каждой букве в соответствие ровно одну цифру и запишите ответ (только цифры, без других символов).

    А) циклическая частота, Б) частота колебаний.

    1) м / с, 2) Гн, 3) Гц, 4) рад / с.
}
\answer{%
    $43$
}

\tasknumber{3}%
\task{%
    \begin{itemize}
        \item Запишите линейное однородное дифференциальное уравнение второго порядка,
            описывающее свободные незатухающие колебания гармонического осциллятора,
        \item запишите общее решение этого уравнения,
        \item подпишите в выписанном решении фазу и амплитуду колебаний,
        \item запишите выражение для скорости,
        \item запишите выражение для ускорения.
    \end{itemize}
}
\answer{%
    \begin{align*}
    &\ddot x + \omega^2 x = 0 \Longleftrightarrow a_x + \omega^2 x = 0, \\
    &x = A \cos(\omega t + \varphi_0) \text{ или же } x = A \sin(\omega t + \varphi_0) \text{ или же } x = a \cos(\omega t) + b \sin(\omega t), \\
    &A \text{\, или \,} \sqrt{a^2 + b^2} \text{ --- это амплитуда}, \omega t + \varphi_0\text{ --- это фаза}, \\
    &v = \dot x = -\omega A \sin(\omega t + \varphi_0), \\
    &a = \dot v = \ddot x = -\omega^2 A \cos(\omega t + \varphi_0) = -\omega^2 x,
    \end{align*}
}
\solutionspace{135pt}

\tasknumber{4}%
\task{%
    Определите частоту колебаний, если их период составляет $T = 10\,\text{мс}$.
}
\answer{%
    $\nu = \frac 1T = \frac 1{10\,\text{мс}} = 100\,\text{Гц}$
}
\solutionspace{40pt}

\tasknumber{5}%
\task{%
    Определите период колебаний, если их частота составляет $\nu = 10\,\text{кГц}$.
    Сколько колебаний произойдёт за $t = 5\,\text{мин}$?
}
\answer{%
    \begin{align*}
    T &= \frac 1\nu = \frac 1{10\,\text{кГц}} = 0{,}100\,\text{мc}, \\
    N &= \nu t = 10\,\text{кГц} \cdot5\,\text{мин} = 3000000\,\text{колебаний}.
    \end{align*}
}
\solutionspace{40pt}

\tasknumber{6}%
\task{%
    Амплитуда колебаний точки составляет $A = 3\,\text{см}$, а частота~--- $\nu = 6\,\text{Гц}$.
    Определите, какой путь преодолеет эта точка за $t = 40\,\text{с}$.
}
\answer{%
    $s = 4A \cdot N = 4A \cdot \frac tT = 4A \cdot t\nu = 4 \cdot 3\,\text{см} \cdot 40\,\text{с} \cdot 6\,\text{Гц} = 28{,}8\,\text{м}$
}
\solutionspace{120pt}

\tasknumber{7}%
\task{%
    Изобразите график гармонических колебаний,
    амплитуда которых составляла бы $A = 1\,\text{см}$, а период $T = 10\,\text{с}$.
}
\solutionspace{80pt}

\tasknumber{8}%
\task{%
    Координата материальной точки зависит от времени по закону $z = 0{,}25 \cdot \cos (6\pi t)$ (в СИ).
    Чему равен путь, пройденный точкой за $3\,\text{мин}$?
}
\answer{%
    $\omega = 6\pi \implies \nu = \frac62\,\units{Гц}, N = \nu t = 540{,}0, s = 4AN = 4 \cdot 0{,}25 \cdot 540{,}0 = 540{,}0 \text{(м)}$
}

\variantsplitter

\addpersonalvariant{Леонид Никитин}

\tasknumber{1}%
\task{%
    Установите каждой букве в соответствие ровно одну цифру и запишите ответ (только цифры, без других символов).

    А) циклическая частота, Б) число колебаний, В) период колебаний.

    1) $\frac{\nu}{2\pi}$, 2) $T$, 3) $N$, 4) $\omega$, 5) $\frac{2\pi}{\nu}$.
}
\answer{%
    $432$
}

\tasknumber{2}%
\task{%
    Установите каждой букве в соответствие ровно одну цифру и запишите ответ (только цифры, без других символов).

    А) период колебаний, Б) частота колебаний.

    1) рад / с, 2) Гц, 3) c, 4) Гн.
}
\answer{%
    $32$
}

\tasknumber{3}%
\task{%
    \begin{itemize}
        \item Запишите линейное однородное дифференциальное уравнение второго порядка,
            описывающее свободные незатухающие колебания гармонического осциллятора,
        \item запишите общее решение этого уравнения,
        \item подпишите в выписанном решении фазу и амплитуду колебаний,
        \item запишите выражение для скорости,
        \item запишите выражение для ускорения.
    \end{itemize}
}
\answer{%
    \begin{align*}
    &\ddot x + \omega^2 x = 0 \Longleftrightarrow a_x + \omega^2 x = 0, \\
    &x = A \cos(\omega t + \varphi_0) \text{ или же } x = A \sin(\omega t + \varphi_0) \text{ или же } x = a \cos(\omega t) + b \sin(\omega t), \\
    &A \text{\, или \,} \sqrt{a^2 + b^2} \text{ --- это амплитуда}, \omega t + \varphi_0\text{ --- это фаза}, \\
    &v = \dot x = -\omega A \sin(\omega t + \varphi_0), \\
    &a = \dot v = \ddot x = -\omega^2 A \cos(\omega t + \varphi_0) = -\omega^2 x,
    \end{align*}
}
\solutionspace{135pt}

\tasknumber{4}%
\task{%
    Определите частоту колебаний, если их период составляет $T = 40\,\text{мс}$.
}
\answer{%
    $\nu = \frac 1T = \frac 1{40\,\text{мс}} = 25\,\text{Гц}$
}
\solutionspace{40pt}

\tasknumber{5}%
\task{%
    Определите период колебаний, если их частота составляет $\nu = 5\,\text{кГц}$.
    Сколько колебаний произойдёт за $t = 10\,\text{мин}$?
}
\answer{%
    \begin{align*}
    T &= \frac 1\nu = \frac 1{5\,\text{кГц}} = 0{,}200\,\text{мc}, \\
    N &= \nu t = 5\,\text{кГц} \cdot10\,\text{мин} = 3000000\,\text{колебаний}.
    \end{align*}
}
\solutionspace{40pt}

\tasknumber{6}%
\task{%
    Амплитуда колебаний точки составляет $A = 3\,\text{см}$, а частота~--- $\nu = 10\,\text{Гц}$.
    Определите, какой путь преодолеет эта точка за $t = 40\,\text{с}$.
}
\answer{%
    $s = 4A \cdot N = 4A \cdot \frac tT = 4A \cdot t\nu = 4 \cdot 3\,\text{см} \cdot 40\,\text{с} \cdot 10\,\text{Гц} = 48{,}0\,\text{м}$
}
\solutionspace{120pt}

\tasknumber{7}%
\task{%
    Изобразите график гармонических колебаний,
    амплитуда которых составляла бы $A = 6\,\text{см}$, а период $T = 4\,\text{с}$.
}
\solutionspace{80pt}

\tasknumber{8}%
\task{%
    Координата материальной точки зависит от времени по закону $x = 0{,}15 \cdot \sin (4\pi t)$ (в СИ).
    Чему равен путь, пройденный точкой за $3\,\text{мин}$?
}
\answer{%
    $\omega = 4\pi \implies \nu = \frac42\,\units{Гц}, N = \nu t = 360{,}0, s = 4AN = 4 \cdot 0{,}15 \cdot 360{,}0 = 216{,}0 \text{(м)}$
}

\variantsplitter

\addpersonalvariant{Тимофей Полетаев}

\tasknumber{1}%
\task{%
    Установите каждой букве в соответствие ровно одну цифру и запишите ответ (только цифры, без других символов).

    А) время колебаний, Б) период колебаний, В) циклическая частота.

    1) $\omega$, 2) $t$, 3) $\frac{\nu}{2\pi}$, 4) $T$, 5) $\frac{2\pi}{\nu}$.
}
\answer{%
    $241$
}

\tasknumber{2}%
\task{%
    Установите каждой букве в соответствие ровно одну цифру и запишите ответ (только цифры, без других символов).

    А) циклическая частота, Б) частота колебаний.

    1) рад / с, 2) м / с, 3) Гц, 4) c.
}
\answer{%
    $13$
}

\tasknumber{3}%
\task{%
    \begin{itemize}
        \item Запишите линейное однородное дифференциальное уравнение второго порядка,
            описывающее свободные незатухающие колебания гармонического осциллятора,
        \item запишите общее решение этого уравнения,
        \item подпишите в выписанном решении фазу и амплитуду колебаний,
        \item запишите выражение для скорости,
        \item запишите выражение для ускорения.
    \end{itemize}
}
\answer{%
    \begin{align*}
    &\ddot x + \omega^2 x = 0 \Longleftrightarrow a_x + \omega^2 x = 0, \\
    &x = A \cos(\omega t + \varphi_0) \text{ или же } x = A \sin(\omega t + \varphi_0) \text{ или же } x = a \cos(\omega t) + b \sin(\omega t), \\
    &A \text{\, или \,} \sqrt{a^2 + b^2} \text{ --- это амплитуда}, \omega t + \varphi_0\text{ --- это фаза}, \\
    &v = \dot x = -\omega A \sin(\omega t + \varphi_0), \\
    &a = \dot v = \ddot x = -\omega^2 A \cos(\omega t + \varphi_0) = -\omega^2 x,
    \end{align*}
}
\solutionspace{135pt}

\tasknumber{4}%
\task{%
    Определите частоту колебаний, если их период составляет $T = 2\,\text{мс}$.
}
\answer{%
    $\nu = \frac 1T = \frac 1{2\,\text{мс}} = 500\,\text{Гц}$
}
\solutionspace{40pt}

\tasknumber{5}%
\task{%
    Определите период колебаний, если их частота составляет $\nu = 40\,\text{кГц}$.
    Сколько колебаний произойдёт за $t = 2\,\text{мин}$?
}
\answer{%
    \begin{align*}
    T &= \frac 1\nu = \frac 1{40\,\text{кГц}} = 0{,}025\,\text{мc}, \\
    N &= \nu t = 40\,\text{кГц} \cdot2\,\text{мин} = 4800000\,\text{колебаний}.
    \end{align*}
}
\solutionspace{40pt}

\tasknumber{6}%
\task{%
    Амплитуда колебаний точки составляет $A = 2\,\text{см}$, а частота~--- $\nu = 10\,\text{Гц}$.
    Определите, какой путь преодолеет эта точка за $t = 80\,\text{с}$.
}
\answer{%
    $s = 4A \cdot N = 4A \cdot \frac tT = 4A \cdot t\nu = 4 \cdot 2\,\text{см} \cdot 80\,\text{с} \cdot 10\,\text{Гц} = 64{,}0\,\text{м}$
}
\solutionspace{120pt}

\tasknumber{7}%
\task{%
    Изобразите график гармонических колебаний,
    амплитуда которых составляла бы $A = 3\,\text{см}$, а период $T = 8\,\text{с}$.
}
\solutionspace{80pt}

\tasknumber{8}%
\task{%
    Координата материальной точки зависит от времени по закону $x = 0{,}15 \cdot \sin (4\pi t)$ (в СИ).
    Чему равен путь, пройденный точкой за $3\,\text{мин}$?
}
\answer{%
    $\omega = 4\pi \implies \nu = \frac42\,\units{Гц}, N = \nu t = 360{,}0, s = 4AN = 4 \cdot 0{,}15 \cdot 360{,}0 = 216{,}0 \text{(м)}$
}

\variantsplitter

\addpersonalvariant{Андрей Рожков}

\tasknumber{1}%
\task{%
    Установите каждой букве в соответствие ровно одну цифру и запишите ответ (только цифры, без других символов).

    А) время колебаний, Б) циклическая частота, В) число колебаний.

    1) $t$, 2) $\frac{\nu}{2\pi}$, 3) $N$, 4) $\omega$, 5) $\frac{2\pi}{\nu}$.
}
\answer{%
    $143$
}

\tasknumber{2}%
\task{%
    Установите каждой букве в соответствие ровно одну цифру и запишите ответ (только цифры, без других символов).

    А) частота колебаний, Б) период колебаний.

    1) рад / с, 2) c, 3) Гц, 4) м / с.
}
\answer{%
    $32$
}

\tasknumber{3}%
\task{%
    \begin{itemize}
        \item Запишите линейное однородное дифференциальное уравнение второго порядка,
            описывающее свободные незатухающие колебания гармонического осциллятора,
        \item запишите общее решение этого уравнения,
        \item подпишите в выписанном решении фазу и амплитуду колебаний,
        \item запишите выражение для скорости,
        \item запишите выражение для ускорения.
    \end{itemize}
}
\answer{%
    \begin{align*}
    &\ddot x + \omega^2 x = 0 \Longleftrightarrow a_x + \omega^2 x = 0, \\
    &x = A \cos(\omega t + \varphi_0) \text{ или же } x = A \sin(\omega t + \varphi_0) \text{ или же } x = a \cos(\omega t) + b \sin(\omega t), \\
    &A \text{\, или \,} \sqrt{a^2 + b^2} \text{ --- это амплитуда}, \omega t + \varphi_0\text{ --- это фаза}, \\
    &v = \dot x = -\omega A \sin(\omega t + \varphi_0), \\
    &a = \dot v = \ddot x = -\omega^2 A \cos(\omega t + \varphi_0) = -\omega^2 x,
    \end{align*}
}
\solutionspace{135pt}

\tasknumber{4}%
\task{%
    Определите частоту колебаний, если их период составляет $T = 2\,\text{мс}$.
}
\answer{%
    $\nu = \frac 1T = \frac 1{2\,\text{мс}} = 500\,\text{Гц}$
}
\solutionspace{40pt}

\tasknumber{5}%
\task{%
    Определите период колебаний, если их частота составляет $\nu = 10\,\text{кГц}$.
    Сколько колебаний произойдёт за $t = 1\,\text{мин}$?
}
\answer{%
    \begin{align*}
    T &= \frac 1\nu = \frac 1{10\,\text{кГц}} = 0{,}100\,\text{мc}, \\
    N &= \nu t = 10\,\text{кГц} \cdot1\,\text{мин} = 600000\,\text{колебаний}.
    \end{align*}
}
\solutionspace{40pt}

\tasknumber{6}%
\task{%
    Амплитуда колебаний точки составляет $A = 15\,\text{см}$, а частота~--- $\nu = 20\,\text{Гц}$.
    Определите, какой путь преодолеет эта точка за $t = 80\,\text{с}$.
}
\answer{%
    $s = 4A \cdot N = 4A \cdot \frac tT = 4A \cdot t\nu = 4 \cdot 15\,\text{см} \cdot 80\,\text{с} \cdot 20\,\text{Гц} = 960{,}0\,\text{м}$
}
\solutionspace{120pt}

\tasknumber{7}%
\task{%
    Изобразите график гармонических колебаний,
    амплитуда которых составляла бы $A = 5\,\text{см}$, а период $T = 6\,\text{с}$.
}
\solutionspace{80pt}

\tasknumber{8}%
\task{%
    Координата материальной точки зависит от времени по закону $x = 0{,}15 \cdot \sin (3\pi t)$ (в СИ).
    Чему равен путь, пройденный точкой за $4\,\text{мин}$?
}
\answer{%
    $\omega = 3\pi \implies \nu = \frac32\,\units{Гц}, N = \nu t = 360{,}0, s = 4AN = 4 \cdot 0{,}15 \cdot 360{,}0 = 216{,}0 \text{(м)}$
}

\variantsplitter

\addpersonalvariant{Рената Таржиманова}

\tasknumber{1}%
\task{%
    Установите каждой букве в соответствие ровно одну цифру и запишите ответ (только цифры, без других символов).

    А) циклическая частота, Б) период колебаний, В) число колебаний.

    1) $\omega$, 2) $N$, 3) $tN$, 4) $\frac{\nu}{2\pi}$, 5) $T$.
}
\answer{%
    $152$
}

\tasknumber{2}%
\task{%
    Установите каждой букве в соответствие ровно одну цифру и запишите ответ (только цифры, без других символов).

    А) частота колебаний, Б) период колебаний.

    1) c, 2) рад / с, 3) м / с, 4) Гц.
}
\answer{%
    $41$
}

\tasknumber{3}%
\task{%
    \begin{itemize}
        \item Запишите линейное однородное дифференциальное уравнение второго порядка,
            описывающее свободные незатухающие колебания гармонического осциллятора,
        \item запишите общее решение этого уравнения,
        \item подпишите в выписанном решении фазу и амплитуду колебаний,
        \item запишите выражение для скорости,
        \item запишите выражение для ускорения.
    \end{itemize}
}
\answer{%
    \begin{align*}
    &\ddot x + \omega^2 x = 0 \Longleftrightarrow a_x + \omega^2 x = 0, \\
    &x = A \cos(\omega t + \varphi_0) \text{ или же } x = A \sin(\omega t + \varphi_0) \text{ или же } x = a \cos(\omega t) + b \sin(\omega t), \\
    &A \text{\, или \,} \sqrt{a^2 + b^2} \text{ --- это амплитуда}, \omega t + \varphi_0\text{ --- это фаза}, \\
    &v = \dot x = -\omega A \sin(\omega t + \varphi_0), \\
    &a = \dot v = \ddot x = -\omega^2 A \cos(\omega t + \varphi_0) = -\omega^2 x,
    \end{align*}
}
\solutionspace{135pt}

\tasknumber{4}%
\task{%
    Определите частоту колебаний, если их период составляет $T = 5\,\text{мс}$.
}
\answer{%
    $\nu = \frac 1T = \frac 1{5\,\text{мс}} = 200\,\text{Гц}$
}
\solutionspace{40pt}

\tasknumber{5}%
\task{%
    Определите период колебаний, если их частота составляет $\nu = 20\,\text{кГц}$.
    Сколько колебаний произойдёт за $t = 1\,\text{мин}$?
}
\answer{%
    \begin{align*}
    T &= \frac 1\nu = \frac 1{20\,\text{кГц}} = 0{,}050\,\text{мc}, \\
    N &= \nu t = 20\,\text{кГц} \cdot1\,\text{мин} = 1200000\,\text{колебаний}.
    \end{align*}
}
\solutionspace{40pt}

\tasknumber{6}%
\task{%
    Амплитуда колебаний точки составляет $A = 2\,\text{см}$, а частота~--- $\nu = 2\,\text{Гц}$.
    Определите, какой путь преодолеет эта точка за $t = 80\,\text{с}$.
}
\answer{%
    $s = 4A \cdot N = 4A \cdot \frac tT = 4A \cdot t\nu = 4 \cdot 2\,\text{см} \cdot 80\,\text{с} \cdot 2\,\text{Гц} = 12{,}8\,\text{м}$
}
\solutionspace{120pt}

\tasknumber{7}%
\task{%
    Изобразите график гармонических колебаний,
    амплитуда которых составляла бы $A = 6\,\text{см}$, а период $T = 6\,\text{с}$.
}
\solutionspace{80pt}

\tasknumber{8}%
\task{%
    Координата материальной точки зависит от времени по закону $z = 0{,}05 \cdot \cos (5\pi t)$ (в СИ).
    Чему равен путь, пройденный точкой за $4\,\text{мин}$?
}
\answer{%
    $\omega = 5\pi \implies \nu = \frac52\,\units{Гц}, N = \nu t = 600{,}0, s = 4AN = 4 \cdot 0{,}05 \cdot 600{,}0 = 120{,}0 \text{(м)}$
}

\variantsplitter

\addpersonalvariant{Андрей Щербаков}

\tasknumber{1}%
\task{%
    Установите каждой букве в соответствие ровно одну цифру и запишите ответ (только цифры, без других символов).

    А) число колебаний, Б) циклическая частота, В) время колебаний.

    1) $\omega$, 2) $T$, 3) $\frac{\nu}{2\pi}$, 4) $t$, 5) $N$.
}
\answer{%
    $514$
}

\tasknumber{2}%
\task{%
    Установите каждой букве в соответствие ровно одну цифру и запишите ответ (только цифры, без других символов).

    А) частота колебаний, Б) период колебаний.

    1) м / с, 2) Гн, 3) c, 4) Гц.
}
\answer{%
    $43$
}

\tasknumber{3}%
\task{%
    \begin{itemize}
        \item Запишите линейное однородное дифференциальное уравнение второго порядка,
            описывающее свободные незатухающие колебания гармонического осциллятора,
        \item запишите общее решение этого уравнения,
        \item подпишите в выписанном решении фазу и амплитуду колебаний,
        \item запишите выражение для скорости,
        \item запишите выражение для ускорения.
    \end{itemize}
}
\answer{%
    \begin{align*}
    &\ddot x + \omega^2 x = 0 \Longleftrightarrow a_x + \omega^2 x = 0, \\
    &x = A \cos(\omega t + \varphi_0) \text{ или же } x = A \sin(\omega t + \varphi_0) \text{ или же } x = a \cos(\omega t) + b \sin(\omega t), \\
    &A \text{\, или \,} \sqrt{a^2 + b^2} \text{ --- это амплитуда}, \omega t + \varphi_0\text{ --- это фаза}, \\
    &v = \dot x = -\omega A \sin(\omega t + \varphi_0), \\
    &a = \dot v = \ddot x = -\omega^2 A \cos(\omega t + \varphi_0) = -\omega^2 x,
    \end{align*}
}
\solutionspace{135pt}

\tasknumber{4}%
\task{%
    Определите частоту колебаний, если их период составляет $T = 40\,\text{мс}$.
}
\answer{%
    $\nu = \frac 1T = \frac 1{40\,\text{мс}} = 25\,\text{Гц}$
}
\solutionspace{40pt}

\tasknumber{5}%
\task{%
    Определите период колебаний, если их частота составляет $\nu = 10\,\text{кГц}$.
    Сколько колебаний произойдёт за $t = 2\,\text{мин}$?
}
\answer{%
    \begin{align*}
    T &= \frac 1\nu = \frac 1{10\,\text{кГц}} = 0{,}100\,\text{мc}, \\
    N &= \nu t = 10\,\text{кГц} \cdot2\,\text{мин} = 1200000\,\text{колебаний}.
    \end{align*}
}
\solutionspace{40pt}

\tasknumber{6}%
\task{%
    Амплитуда колебаний точки составляет $A = 3\,\text{см}$, а частота~--- $\nu = 20\,\text{Гц}$.
    Определите, какой путь преодолеет эта точка за $t = 80\,\text{с}$.
}
\answer{%
    $s = 4A \cdot N = 4A \cdot \frac tT = 4A \cdot t\nu = 4 \cdot 3\,\text{см} \cdot 80\,\text{с} \cdot 20\,\text{Гц} = 192{,}0\,\text{м}$
}
\solutionspace{120pt}

\tasknumber{7}%
\task{%
    Изобразите график гармонических колебаний,
    амплитуда которых составляла бы $A = 2\,\text{см}$, а период $T = 10\,\text{с}$.
}
\solutionspace{80pt}

\tasknumber{8}%
\task{%
    Координата материальной точки зависит от времени по закону $z = 0{,}05 \cdot \cos (6\pi t)$ (в СИ).
    Чему равен путь, пройденный точкой за $2\,\text{мин}$?
}
\answer{%
    $\omega = 6\pi \implies \nu = \frac62\,\units{Гц}, N = \nu t = 360{,}0, s = 4AN = 4 \cdot 0{,}05 \cdot 360{,}0 = 72{,}0 \text{(м)}$
}

\variantsplitter

\addpersonalvariant{Михаил Ярошевский}

\tasknumber{1}%
\task{%
    Установите каждой букве в соответствие ровно одну цифру и запишите ответ (только цифры, без других символов).

    А) циклическая частота, Б) частота колебаний, В) период колебаний.

    1) $\nu$, 2) $\omega$, 3) $\frac{2\pi}{\nu}$, 4) $T$, 5) $N$.
}
\answer{%
    $214$
}

\tasknumber{2}%
\task{%
    Установите каждой букве в соответствие ровно одну цифру и запишите ответ (только цифры, без других символов).

    А) период колебаний, Б) частота колебаний.

    1) c, 2) рад / с, 3) Гц, 4) Гн.
}
\answer{%
    $13$
}

\tasknumber{3}%
\task{%
    \begin{itemize}
        \item Запишите линейное однородное дифференциальное уравнение второго порядка,
            описывающее свободные незатухающие колебания гармонического осциллятора,
        \item запишите общее решение этого уравнения,
        \item подпишите в выписанном решении фазу и амплитуду колебаний,
        \item запишите выражение для скорости,
        \item запишите выражение для ускорения.
    \end{itemize}
}
\answer{%
    \begin{align*}
    &\ddot x + \omega^2 x = 0 \Longleftrightarrow a_x + \omega^2 x = 0, \\
    &x = A \cos(\omega t + \varphi_0) \text{ или же } x = A \sin(\omega t + \varphi_0) \text{ или же } x = a \cos(\omega t) + b \sin(\omega t), \\
    &A \text{\, или \,} \sqrt{a^2 + b^2} \text{ --- это амплитуда}, \omega t + \varphi_0\text{ --- это фаза}, \\
    &v = \dot x = -\omega A \sin(\omega t + \varphi_0), \\
    &a = \dot v = \ddot x = -\omega^2 A \cos(\omega t + \varphi_0) = -\omega^2 x,
    \end{align*}
}
\solutionspace{135pt}

\tasknumber{4}%
\task{%
    Определите частоту колебаний, если их период составляет $T = 20\,\text{мс}$.
}
\answer{%
    $\nu = \frac 1T = \frac 1{20\,\text{мс}} = 50\,\text{Гц}$
}
\solutionspace{40pt}

\tasknumber{5}%
\task{%
    Определите период колебаний, если их частота составляет $\nu = 10\,\text{кГц}$.
    Сколько колебаний произойдёт за $t = 5\,\text{мин}$?
}
\answer{%
    \begin{align*}
    T &= \frac 1\nu = \frac 1{10\,\text{кГц}} = 0{,}100\,\text{мc}, \\
    N &= \nu t = 10\,\text{кГц} \cdot5\,\text{мин} = 3000000\,\text{колебаний}.
    \end{align*}
}
\solutionspace{40pt}

\tasknumber{6}%
\task{%
    Амплитуда колебаний точки составляет $A = 15\,\text{см}$, а частота~--- $\nu = 6\,\text{Гц}$.
    Определите, какой путь преодолеет эта точка за $t = 40\,\text{с}$.
}
\answer{%
    $s = 4A \cdot N = 4A \cdot \frac tT = 4A \cdot t\nu = 4 \cdot 15\,\text{см} \cdot 40\,\text{с} \cdot 6\,\text{Гц} = 144{,}0\,\text{м}$
}
\solutionspace{120pt}

\tasknumber{7}%
\task{%
    Изобразите график гармонических колебаний,
    амплитуда которых составляла бы $A = 75\,\text{см}$, а период $T = 10\,\text{с}$.
}
\solutionspace{80pt}

\tasknumber{8}%
\task{%
    Координата материальной точки зависит от времени по закону $y = 0{,}02 \cdot \sin (6\pi t)$ (в СИ).
    Чему равен путь, пройденный точкой за $2\,\text{мин}$?
}
\answer{%
    $\omega = 6\pi \implies \nu = \frac62\,\units{Гц}, N = \nu t = 360{,}0, s = 4AN = 4 \cdot 0{,}02 \cdot 360{,}0 = 28{,}8 \text{(м)}$
}

\variantsplitter

\addpersonalvariant{Алексей Алимпиев}

\tasknumber{1}%
\task{%
    Установите каждой букве в соответствие ровно одну цифру и запишите ответ (только цифры, без других символов).

    А) время колебаний, Б) число колебаний, В) частота колебаний.

    1) $t$, 2) $tN$, 3) $N$, 4) $\nu$, 5) $T$.
}
\answer{%
    $134$
}

\tasknumber{2}%
\task{%
    Установите каждой букве в соответствие ровно одну цифру и запишите ответ (только цифры, без других символов).

    А) циклическая частота, Б) период колебаний.

    1) Гн, 2) рад / с, 3) c, 4) Гц.
}
\answer{%
    $23$
}

\tasknumber{3}%
\task{%
    \begin{itemize}
        \item Запишите линейное однородное дифференциальное уравнение второго порядка,
            описывающее свободные незатухающие колебания гармонического осциллятора,
        \item запишите общее решение этого уравнения,
        \item подпишите в выписанном решении фазу и амплитуду колебаний,
        \item запишите выражение для скорости,
        \item запишите выражение для ускорения.
    \end{itemize}
}
\answer{%
    \begin{align*}
    &\ddot x + \omega^2 x = 0 \Longleftrightarrow a_x + \omega^2 x = 0, \\
    &x = A \cos(\omega t + \varphi_0) \text{ или же } x = A \sin(\omega t + \varphi_0) \text{ или же } x = a \cos(\omega t) + b \sin(\omega t), \\
    &A \text{\, или \,} \sqrt{a^2 + b^2} \text{ --- это амплитуда}, \omega t + \varphi_0\text{ --- это фаза}, \\
    &v = \dot x = -\omega A \sin(\omega t + \varphi_0), \\
    &a = \dot v = \ddot x = -\omega^2 A \cos(\omega t + \varphi_0) = -\omega^2 x,
    \end{align*}
}
\solutionspace{135pt}

\tasknumber{4}%
\task{%
    Определите частоту колебаний, если их период составляет $T = 20\,\text{мс}$.
}
\answer{%
    $\nu = \frac 1T = \frac 1{20\,\text{мс}} = 50\,\text{Гц}$
}
\solutionspace{40pt}

\tasknumber{5}%
\task{%
    Определите период колебаний, если их частота составляет $\nu = 20\,\text{кГц}$.
    Сколько колебаний произойдёт за $t = 5\,\text{мин}$?
}
\answer{%
    \begin{align*}
    T &= \frac 1\nu = \frac 1{20\,\text{кГц}} = 0{,}050\,\text{мc}, \\
    N &= \nu t = 20\,\text{кГц} \cdot5\,\text{мин} = 6000000\,\text{колебаний}.
    \end{align*}
}
\solutionspace{40pt}

\tasknumber{6}%
\task{%
    Амплитуда колебаний точки составляет $A = 3\,\text{см}$, а частота~--- $\nu = 10\,\text{Гц}$.
    Определите, какой путь преодолеет эта точка за $t = 80\,\text{с}$.
}
\answer{%
    $s = 4A \cdot N = 4A \cdot \frac tT = 4A \cdot t\nu = 4 \cdot 3\,\text{см} \cdot 80\,\text{с} \cdot 10\,\text{Гц} = 96{,}0\,\text{м}$
}
\solutionspace{120pt}

\tasknumber{7}%
\task{%
    Изобразите график гармонических колебаний,
    амплитуда которых составляла бы $A = 5\,\text{см}$, а период $T = 8\,\text{с}$.
}
\solutionspace{80pt}

\tasknumber{8}%
\task{%
    Координата материальной точки зависит от времени по закону $y = 0{,}25 \cdot \cos (4\pi t)$ (в СИ).
    Чему равен путь, пройденный точкой за $3\,\text{мин}$?
}
\answer{%
    $\omega = 4\pi \implies \nu = \frac42\,\units{Гц}, N = \nu t = 360{,}0, s = 4AN = 4 \cdot 0{,}25 \cdot 360{,}0 = 360{,}0 \text{(м)}$
}

\variantsplitter

\addpersonalvariant{Евгений Васин}

\tasknumber{1}%
\task{%
    Установите каждой букве в соответствие ровно одну цифру и запишите ответ (только цифры, без других символов).

    А) частота колебаний, Б) период колебаний, В) циклическая частота.

    1) $T$, 2) $\frac{2\pi}{\nu}$, 3) $\omega$, 4) $\nu$, 5) $t$.
}
\answer{%
    $413$
}

\tasknumber{2}%
\task{%
    Установите каждой букве в соответствие ровно одну цифру и запишите ответ (только цифры, без других символов).

    А) циклическая частота, Б) частота колебаний.

    1) Гн, 2) Гц, 3) рад / с, 4) c.
}
\answer{%
    $32$
}

\tasknumber{3}%
\task{%
    \begin{itemize}
        \item Запишите линейное однородное дифференциальное уравнение второго порядка,
            описывающее свободные незатухающие колебания гармонического осциллятора,
        \item запишите общее решение этого уравнения,
        \item подпишите в выписанном решении фазу и амплитуду колебаний,
        \item запишите выражение для скорости,
        \item запишите выражение для ускорения.
    \end{itemize}
}
\answer{%
    \begin{align*}
    &\ddot x + \omega^2 x = 0 \Longleftrightarrow a_x + \omega^2 x = 0, \\
    &x = A \cos(\omega t + \varphi_0) \text{ или же } x = A \sin(\omega t + \varphi_0) \text{ или же } x = a \cos(\omega t) + b \sin(\omega t), \\
    &A \text{\, или \,} \sqrt{a^2 + b^2} \text{ --- это амплитуда}, \omega t + \varphi_0\text{ --- это фаза}, \\
    &v = \dot x = -\omega A \sin(\omega t + \varphi_0), \\
    &a = \dot v = \ddot x = -\omega^2 A \cos(\omega t + \varphi_0) = -\omega^2 x,
    \end{align*}
}
\solutionspace{135pt}

\tasknumber{4}%
\task{%
    Определите частоту колебаний, если их период составляет $T = 50\,\text{мс}$.
}
\answer{%
    $\nu = \frac 1T = \frac 1{50\,\text{мс}} = 20\,\text{Гц}$
}
\solutionspace{40pt}

\tasknumber{5}%
\task{%
    Определите период колебаний, если их частота составляет $\nu = 50\,\text{кГц}$.
    Сколько колебаний произойдёт за $t = 1\,\text{мин}$?
}
\answer{%
    \begin{align*}
    T &= \frac 1\nu = \frac 1{50\,\text{кГц}} = 0{,}020\,\text{мc}, \\
    N &= \nu t = 50\,\text{кГц} \cdot1\,\text{мин} = 3000000\,\text{колебаний}.
    \end{align*}
}
\solutionspace{40pt}

\tasknumber{6}%
\task{%
    Амплитуда колебаний точки составляет $A = 5\,\text{см}$, а частота~--- $\nu = 2\,\text{Гц}$.
    Определите, какой путь преодолеет эта точка за $t = 10\,\text{с}$.
}
\answer{%
    $s = 4A \cdot N = 4A \cdot \frac tT = 4A \cdot t\nu = 4 \cdot 5\,\text{см} \cdot 10\,\text{с} \cdot 2\,\text{Гц} = 4{,}0\,\text{м}$
}
\solutionspace{120pt}

\tasknumber{7}%
\task{%
    Изобразите график гармонических колебаний,
    амплитуда которых составляла бы $A = 30\,\text{см}$, а период $T = 2\,\text{с}$.
}
\solutionspace{80pt}

\tasknumber{8}%
\task{%
    Координата материальной точки зависит от времени по закону $z = 0{,}02 \cdot \cos (6\pi t)$ (в СИ).
    Чему равен путь, пройденный точкой за $4\,\text{мин}$?
}
\answer{%
    $\omega = 6\pi \implies \nu = \frac62\,\units{Гц}, N = \nu t = 720{,}0, s = 4AN = 4 \cdot 0{,}02 \cdot 720{,}0 = 57{,}6 \text{(м)}$
}

\variantsplitter

\addpersonalvariant{Вячеслав Волохов}

\tasknumber{1}%
\task{%
    Установите каждой букве в соответствие ровно одну цифру и запишите ответ (только цифры, без других символов).

    А) частота колебаний, Б) число колебаний, В) время колебаний.

    1) $\nu$, 2) $N$, 3) $\frac{\nu}{2\pi}$, 4) $T$, 5) $t$.
}
\answer{%
    $125$
}

\tasknumber{2}%
\task{%
    Установите каждой букве в соответствие ровно одну цифру и запишите ответ (только цифры, без других символов).

    А) циклическая частота, Б) период колебаний.

    1) рад / с, 2) м / с, 3) Гн, 4) c.
}
\answer{%
    $14$
}

\tasknumber{3}%
\task{%
    \begin{itemize}
        \item Запишите линейное однородное дифференциальное уравнение второго порядка,
            описывающее свободные незатухающие колебания гармонического осциллятора,
        \item запишите общее решение этого уравнения,
        \item подпишите в выписанном решении фазу и амплитуду колебаний,
        \item запишите выражение для скорости,
        \item запишите выражение для ускорения.
    \end{itemize}
}
\answer{%
    \begin{align*}
    &\ddot x + \omega^2 x = 0 \Longleftrightarrow a_x + \omega^2 x = 0, \\
    &x = A \cos(\omega t + \varphi_0) \text{ или же } x = A \sin(\omega t + \varphi_0) \text{ или же } x = a \cos(\omega t) + b \sin(\omega t), \\
    &A \text{\, или \,} \sqrt{a^2 + b^2} \text{ --- это амплитуда}, \omega t + \varphi_0\text{ --- это фаза}, \\
    &v = \dot x = -\omega A \sin(\omega t + \varphi_0), \\
    &a = \dot v = \ddot x = -\omega^2 A \cos(\omega t + \varphi_0) = -\omega^2 x,
    \end{align*}
}
\solutionspace{135pt}

\tasknumber{4}%
\task{%
    Определите частоту колебаний, если их период составляет $T = 5\,\text{мс}$.
}
\answer{%
    $\nu = \frac 1T = \frac 1{5\,\text{мс}} = 200\,\text{Гц}$
}
\solutionspace{40pt}

\tasknumber{5}%
\task{%
    Определите период колебаний, если их частота составляет $\nu = 50\,\text{кГц}$.
    Сколько колебаний произойдёт за $t = 1\,\text{мин}$?
}
\answer{%
    \begin{align*}
    T &= \frac 1\nu = \frac 1{50\,\text{кГц}} = 0{,}020\,\text{мc}, \\
    N &= \nu t = 50\,\text{кГц} \cdot1\,\text{мин} = 3000000\,\text{колебаний}.
    \end{align*}
}
\solutionspace{40pt}

\tasknumber{6}%
\task{%
    Амплитуда колебаний точки составляет $A = 15\,\text{см}$, а частота~--- $\nu = 6\,\text{Гц}$.
    Определите, какой путь преодолеет эта точка за $t = 80\,\text{с}$.
}
\answer{%
    $s = 4A \cdot N = 4A \cdot \frac tT = 4A \cdot t\nu = 4 \cdot 15\,\text{см} \cdot 80\,\text{с} \cdot 6\,\text{Гц} = 288{,}0\,\text{м}$
}
\solutionspace{120pt}

\tasknumber{7}%
\task{%
    Изобразите график гармонических колебаний,
    амплитуда которых составляла бы $A = 15\,\text{см}$, а период $T = 4\,\text{с}$.
}
\solutionspace{80pt}

\tasknumber{8}%
\task{%
    Координата материальной точки зависит от времени по закону $x = 0{,}05 \cdot \cos (4\pi t)$ (в СИ).
    Чему равен путь, пройденный точкой за $3\,\text{мин}$?
}
\answer{%
    $\omega = 4\pi \implies \nu = \frac42\,\units{Гц}, N = \nu t = 360{,}0, s = 4AN = 4 \cdot 0{,}05 \cdot 360{,}0 = 72{,}0 \text{(м)}$
}

\variantsplitter

\addpersonalvariant{Герман Говоров}

\tasknumber{1}%
\task{%
    Установите каждой букве в соответствие ровно одну цифру и запишите ответ (только цифры, без других символов).

    А) циклическая частота, Б) частота колебаний, В) число колебаний.

    1) $N$, 2) $t$, 3) $\frac{\nu}{2\pi}$, 4) $\nu$, 5) $\omega$.
}
\answer{%
    $541$
}

\tasknumber{2}%
\task{%
    Установите каждой букве в соответствие ровно одну цифру и запишите ответ (только цифры, без других символов).

    А) циклическая частота, Б) частота колебаний.

    1) c, 2) Гн, 3) Гц, 4) рад / с.
}
\answer{%
    $43$
}

\tasknumber{3}%
\task{%
    \begin{itemize}
        \item Запишите линейное однородное дифференциальное уравнение второго порядка,
            описывающее свободные незатухающие колебания гармонического осциллятора,
        \item запишите общее решение этого уравнения,
        \item подпишите в выписанном решении фазу и амплитуду колебаний,
        \item запишите выражение для скорости,
        \item запишите выражение для ускорения.
    \end{itemize}
}
\answer{%
    \begin{align*}
    &\ddot x + \omega^2 x = 0 \Longleftrightarrow a_x + \omega^2 x = 0, \\
    &x = A \cos(\omega t + \varphi_0) \text{ или же } x = A \sin(\omega t + \varphi_0) \text{ или же } x = a \cos(\omega t) + b \sin(\omega t), \\
    &A \text{\, или \,} \sqrt{a^2 + b^2} \text{ --- это амплитуда}, \omega t + \varphi_0\text{ --- это фаза}, \\
    &v = \dot x = -\omega A \sin(\omega t + \varphi_0), \\
    &a = \dot v = \ddot x = -\omega^2 A \cos(\omega t + \varphi_0) = -\omega^2 x,
    \end{align*}
}
\solutionspace{135pt}

\tasknumber{4}%
\task{%
    Определите частоту колебаний, если их период составляет $T = 20\,\text{мс}$.
}
\answer{%
    $\nu = \frac 1T = \frac 1{20\,\text{мс}} = 50\,\text{Гц}$
}
\solutionspace{40pt}

\tasknumber{5}%
\task{%
    Определите период колебаний, если их частота составляет $\nu = 20\,\text{кГц}$.
    Сколько колебаний произойдёт за $t = 1\,\text{мин}$?
}
\answer{%
    \begin{align*}
    T &= \frac 1\nu = \frac 1{20\,\text{кГц}} = 0{,}050\,\text{мc}, \\
    N &= \nu t = 20\,\text{кГц} \cdot1\,\text{мин} = 1200000\,\text{колебаний}.
    \end{align*}
}
\solutionspace{40pt}

\tasknumber{6}%
\task{%
    Амплитуда колебаний точки составляет $A = 5\,\text{см}$, а частота~--- $\nu = 5\,\text{Гц}$.
    Определите, какой путь преодолеет эта точка за $t = 10\,\text{с}$.
}
\answer{%
    $s = 4A \cdot N = 4A \cdot \frac tT = 4A \cdot t\nu = 4 \cdot 5\,\text{см} \cdot 10\,\text{с} \cdot 5\,\text{Гц} = 10{,}0\,\text{м}$
}
\solutionspace{120pt}

\tasknumber{7}%
\task{%
    Изобразите график гармонических колебаний,
    амплитуда которых составляла бы $A = 1\,\text{см}$, а период $T = 6\,\text{с}$.
}
\solutionspace{80pt}

\tasknumber{8}%
\task{%
    Координата материальной точки зависит от времени по закону $z = 0{,}05 \cdot \cos (3\pi t)$ (в СИ).
    Чему равен путь, пройденный точкой за $3\,\text{мин}$?
}
\answer{%
    $\omega = 3\pi \implies \nu = \frac32\,\units{Гц}, N = \nu t = 270{,}0, s = 4AN = 4 \cdot 0{,}05 \cdot 270{,}0 = 54{,}0 \text{(м)}$
}

\variantsplitter

\addpersonalvariant{София Журавлёва}

\tasknumber{1}%
\task{%
    Установите каждой букве в соответствие ровно одну цифру и запишите ответ (только цифры, без других символов).

    А) период колебаний, Б) частота колебаний, В) циклическая частота.

    1) $N$, 2) $\frac{\nu}{2\pi}$, 3) $\nu$, 4) $T$, 5) $\omega$.
}
\answer{%
    $435$
}

\tasknumber{2}%
\task{%
    Установите каждой букве в соответствие ровно одну цифру и запишите ответ (только цифры, без других символов).

    А) период колебаний, Б) циклическая частота.

    1) Гн, 2) рад / с, 3) c, 4) Гц.
}
\answer{%
    $32$
}

\tasknumber{3}%
\task{%
    \begin{itemize}
        \item Запишите линейное однородное дифференциальное уравнение второго порядка,
            описывающее свободные незатухающие колебания гармонического осциллятора,
        \item запишите общее решение этого уравнения,
        \item подпишите в выписанном решении фазу и амплитуду колебаний,
        \item запишите выражение для скорости,
        \item запишите выражение для ускорения.
    \end{itemize}
}
\answer{%
    \begin{align*}
    &\ddot x + \omega^2 x = 0 \Longleftrightarrow a_x + \omega^2 x = 0, \\
    &x = A \cos(\omega t + \varphi_0) \text{ или же } x = A \sin(\omega t + \varphi_0) \text{ или же } x = a \cos(\omega t) + b \sin(\omega t), \\
    &A \text{\, или \,} \sqrt{a^2 + b^2} \text{ --- это амплитуда}, \omega t + \varphi_0\text{ --- это фаза}, \\
    &v = \dot x = -\omega A \sin(\omega t + \varphi_0), \\
    &a = \dot v = \ddot x = -\omega^2 A \cos(\omega t + \varphi_0) = -\omega^2 x,
    \end{align*}
}
\solutionspace{135pt}

\tasknumber{4}%
\task{%
    Определите частоту колебаний, если их период составляет $T = 40\,\text{мс}$.
}
\answer{%
    $\nu = \frac 1T = \frac 1{40\,\text{мс}} = 25\,\text{Гц}$
}
\solutionspace{40pt}

\tasknumber{5}%
\task{%
    Определите период колебаний, если их частота составляет $\nu = 40\,\text{кГц}$.
    Сколько колебаний произойдёт за $t = 5\,\text{мин}$?
}
\answer{%
    \begin{align*}
    T &= \frac 1\nu = \frac 1{40\,\text{кГц}} = 0{,}025\,\text{мc}, \\
    N &= \nu t = 40\,\text{кГц} \cdot5\,\text{мин} = 12000000\,\text{колебаний}.
    \end{align*}
}
\solutionspace{40pt}

\tasknumber{6}%
\task{%
    Амплитуда колебаний точки составляет $A = 10\,\text{см}$, а частота~--- $\nu = 6\,\text{Гц}$.
    Определите, какой путь преодолеет эта точка за $t = 40\,\text{с}$.
}
\answer{%
    $s = 4A \cdot N = 4A \cdot \frac tT = 4A \cdot t\nu = 4 \cdot 10\,\text{см} \cdot 40\,\text{с} \cdot 6\,\text{Гц} = 96{,}0\,\text{м}$
}
\solutionspace{120pt}

\tasknumber{7}%
\task{%
    Изобразите график гармонических колебаний,
    амплитуда которых составляла бы $A = 1\,\text{см}$, а период $T = 10\,\text{с}$.
}
\solutionspace{80pt}

\tasknumber{8}%
\task{%
    Координата материальной точки зависит от времени по закону $z = 0{,}25 \cdot \sin (6\pi t)$ (в СИ).
    Чему равен путь, пройденный точкой за $2\,\text{мин}$?
}
\answer{%
    $\omega = 6\pi \implies \nu = \frac62\,\units{Гц}, N = \nu t = 360{,}0, s = 4AN = 4 \cdot 0{,}25 \cdot 360{,}0 = 360{,}0 \text{(м)}$
}

\variantsplitter

\addpersonalvariant{Константин Козлов}

\tasknumber{1}%
\task{%
    Установите каждой букве в соответствие ровно одну цифру и запишите ответ (только цифры, без других символов).

    А) число колебаний, Б) период колебаний, В) частота колебаний.

    1) $T$, 2) $\frac{\nu}{2\pi}$, 3) $\omega$, 4) $N$, 5) $\nu$.
}
\answer{%
    $415$
}

\tasknumber{2}%
\task{%
    Установите каждой букве в соответствие ровно одну цифру и запишите ответ (только цифры, без других символов).

    А) частота колебаний, Б) период колебаний.

    1) м / с, 2) Гн, 3) Гц, 4) c.
}
\answer{%
    $34$
}

\tasknumber{3}%
\task{%
    \begin{itemize}
        \item Запишите линейное однородное дифференциальное уравнение второго порядка,
            описывающее свободные незатухающие колебания гармонического осциллятора,
        \item запишите общее решение этого уравнения,
        \item подпишите в выписанном решении фазу и амплитуду колебаний,
        \item запишите выражение для скорости,
        \item запишите выражение для ускорения.
    \end{itemize}
}
\answer{%
    \begin{align*}
    &\ddot x + \omega^2 x = 0 \Longleftrightarrow a_x + \omega^2 x = 0, \\
    &x = A \cos(\omega t + \varphi_0) \text{ или же } x = A \sin(\omega t + \varphi_0) \text{ или же } x = a \cos(\omega t) + b \sin(\omega t), \\
    &A \text{\, или \,} \sqrt{a^2 + b^2} \text{ --- это амплитуда}, \omega t + \varphi_0\text{ --- это фаза}, \\
    &v = \dot x = -\omega A \sin(\omega t + \varphi_0), \\
    &a = \dot v = \ddot x = -\omega^2 A \cos(\omega t + \varphi_0) = -\omega^2 x,
    \end{align*}
}
\solutionspace{135pt}

\tasknumber{4}%
\task{%
    Определите частоту колебаний, если их период составляет $T = 40\,\text{мс}$.
}
\answer{%
    $\nu = \frac 1T = \frac 1{40\,\text{мс}} = 25\,\text{Гц}$
}
\solutionspace{40pt}

\tasknumber{5}%
\task{%
    Определите период колебаний, если их частота составляет $\nu = 50\,\text{кГц}$.
    Сколько колебаний произойдёт за $t = 1\,\text{мин}$?
}
\answer{%
    \begin{align*}
    T &= \frac 1\nu = \frac 1{50\,\text{кГц}} = 0{,}020\,\text{мc}, \\
    N &= \nu t = 50\,\text{кГц} \cdot1\,\text{мин} = 3000000\,\text{колебаний}.
    \end{align*}
}
\solutionspace{40pt}

\tasknumber{6}%
\task{%
    Амплитуда колебаний точки составляет $A = 3\,\text{см}$, а частота~--- $\nu = 10\,\text{Гц}$.
    Определите, какой путь преодолеет эта точка за $t = 40\,\text{с}$.
}
\answer{%
    $s = 4A \cdot N = 4A \cdot \frac tT = 4A \cdot t\nu = 4 \cdot 3\,\text{см} \cdot 40\,\text{с} \cdot 10\,\text{Гц} = 48{,}0\,\text{м}$
}
\solutionspace{120pt}

\tasknumber{7}%
\task{%
    Изобразите график гармонических колебаний,
    амплитуда которых составляла бы $A = 15\,\text{см}$, а период $T = 8\,\text{с}$.
}
\solutionspace{80pt}

\tasknumber{8}%
\task{%
    Координата материальной точки зависит от времени по закону $z = 0{,}15 \cdot \cos (4\pi t)$ (в СИ).
    Чему равен путь, пройденный точкой за $3\,\text{мин}$?
}
\answer{%
    $\omega = 4\pi \implies \nu = \frac42\,\units{Гц}, N = \nu t = 360{,}0, s = 4AN = 4 \cdot 0{,}15 \cdot 360{,}0 = 216{,}0 \text{(м)}$
}

\variantsplitter

\addpersonalvariant{Наталья Кравченко}

\tasknumber{1}%
\task{%
    Установите каждой букве в соответствие ровно одну цифру и запишите ответ (только цифры, без других символов).

    А) период колебаний, Б) время колебаний, В) циклическая частота.

    1) $\frac{2\pi}{\nu}$, 2) $t$, 3) $tN$, 4) $\omega$, 5) $T$.
}
\answer{%
    $524$
}

\tasknumber{2}%
\task{%
    Установите каждой букве в соответствие ровно одну цифру и запишите ответ (только цифры, без других символов).

    А) циклическая частота, Б) частота колебаний.

    1) Гц, 2) Гн, 3) c, 4) рад / с.
}
\answer{%
    $41$
}

\tasknumber{3}%
\task{%
    \begin{itemize}
        \item Запишите линейное однородное дифференциальное уравнение второго порядка,
            описывающее свободные незатухающие колебания гармонического осциллятора,
        \item запишите общее решение этого уравнения,
        \item подпишите в выписанном решении фазу и амплитуду колебаний,
        \item запишите выражение для скорости,
        \item запишите выражение для ускорения.
    \end{itemize}
}
\answer{%
    \begin{align*}
    &\ddot x + \omega^2 x = 0 \Longleftrightarrow a_x + \omega^2 x = 0, \\
    &x = A \cos(\omega t + \varphi_0) \text{ или же } x = A \sin(\omega t + \varphi_0) \text{ или же } x = a \cos(\omega t) + b \sin(\omega t), \\
    &A \text{\, или \,} \sqrt{a^2 + b^2} \text{ --- это амплитуда}, \omega t + \varphi_0\text{ --- это фаза}, \\
    &v = \dot x = -\omega A \sin(\omega t + \varphi_0), \\
    &a = \dot v = \ddot x = -\omega^2 A \cos(\omega t + \varphi_0) = -\omega^2 x,
    \end{align*}
}
\solutionspace{135pt}

\tasknumber{4}%
\task{%
    Определите частоту колебаний, если их период составляет $T = 10\,\text{мс}$.
}
\answer{%
    $\nu = \frac 1T = \frac 1{10\,\text{мс}} = 100\,\text{Гц}$
}
\solutionspace{40pt}

\tasknumber{5}%
\task{%
    Определите период колебаний, если их частота составляет $\nu = 40\,\text{кГц}$.
    Сколько колебаний произойдёт за $t = 2\,\text{мин}$?
}
\answer{%
    \begin{align*}
    T &= \frac 1\nu = \frac 1{40\,\text{кГц}} = 0{,}025\,\text{мc}, \\
    N &= \nu t = 40\,\text{кГц} \cdot2\,\text{мин} = 4800000\,\text{колебаний}.
    \end{align*}
}
\solutionspace{40pt}

\tasknumber{6}%
\task{%
    Амплитуда колебаний точки составляет $A = 5\,\text{см}$, а частота~--- $\nu = 5\,\text{Гц}$.
    Определите, какой путь преодолеет эта точка за $t = 80\,\text{с}$.
}
\answer{%
    $s = 4A \cdot N = 4A \cdot \frac tT = 4A \cdot t\nu = 4 \cdot 5\,\text{см} \cdot 80\,\text{с} \cdot 5\,\text{Гц} = 80{,}0\,\text{м}$
}
\solutionspace{120pt}

\tasknumber{7}%
\task{%
    Изобразите график гармонических колебаний,
    амплитуда которых составляла бы $A = 5\,\text{см}$, а период $T = 4\,\text{с}$.
}
\solutionspace{80pt}

\tasknumber{8}%
\task{%
    Координата материальной точки зависит от времени по закону $x = 0{,}02 \cdot \sin (4\pi t)$ (в СИ).
    Чему равен путь, пройденный точкой за $3\,\text{мин}$?
}
\answer{%
    $\omega = 4\pi \implies \nu = \frac42\,\units{Гц}, N = \nu t = 360{,}0, s = 4AN = 4 \cdot 0{,}02 \cdot 360{,}0 = 28{,}8 \text{(м)}$
}

\variantsplitter

\addpersonalvariant{Сергей Малышев}

\tasknumber{1}%
\task{%
    Установите каждой букве в соответствие ровно одну цифру и запишите ответ (только цифры, без других символов).

    А) число колебаний, Б) частота колебаний, В) период колебаний.

    1) $T$, 2) $t$, 3) $N$, 4) $\omega$, 5) $\nu$.
}
\answer{%
    $351$
}

\tasknumber{2}%
\task{%
    Установите каждой букве в соответствие ровно одну цифру и запишите ответ (только цифры, без других символов).

    А) период колебаний, Б) частота колебаний.

    1) Гн, 2) c, 3) рад / с, 4) Гц.
}
\answer{%
    $24$
}

\tasknumber{3}%
\task{%
    \begin{itemize}
        \item Запишите линейное однородное дифференциальное уравнение второго порядка,
            описывающее свободные незатухающие колебания гармонического осциллятора,
        \item запишите общее решение этого уравнения,
        \item подпишите в выписанном решении фазу и амплитуду колебаний,
        \item запишите выражение для скорости,
        \item запишите выражение для ускорения.
    \end{itemize}
}
\answer{%
    \begin{align*}
    &\ddot x + \omega^2 x = 0 \Longleftrightarrow a_x + \omega^2 x = 0, \\
    &x = A \cos(\omega t + \varphi_0) \text{ или же } x = A \sin(\omega t + \varphi_0) \text{ или же } x = a \cos(\omega t) + b \sin(\omega t), \\
    &A \text{\, или \,} \sqrt{a^2 + b^2} \text{ --- это амплитуда}, \omega t + \varphi_0\text{ --- это фаза}, \\
    &v = \dot x = -\omega A \sin(\omega t + \varphi_0), \\
    &a = \dot v = \ddot x = -\omega^2 A \cos(\omega t + \varphi_0) = -\omega^2 x,
    \end{align*}
}
\solutionspace{135pt}

\tasknumber{4}%
\task{%
    Определите частоту колебаний, если их период составляет $T = 4\,\text{мс}$.
}
\answer{%
    $\nu = \frac 1T = \frac 1{4\,\text{мс}} = 250\,\text{Гц}$
}
\solutionspace{40pt}

\tasknumber{5}%
\task{%
    Определите период колебаний, если их частота составляет $\nu = 5\,\text{кГц}$.
    Сколько колебаний произойдёт за $t = 1\,\text{мин}$?
}
\answer{%
    \begin{align*}
    T &= \frac 1\nu = \frac 1{5\,\text{кГц}} = 0{,}200\,\text{мc}, \\
    N &= \nu t = 5\,\text{кГц} \cdot1\,\text{мин} = 300000\,\text{колебаний}.
    \end{align*}
}
\solutionspace{40pt}

\tasknumber{6}%
\task{%
    Амплитуда колебаний точки составляет $A = 5\,\text{см}$, а частота~--- $\nu = 6\,\text{Гц}$.
    Определите, какой путь преодолеет эта точка за $t = 80\,\text{с}$.
}
\answer{%
    $s = 4A \cdot N = 4A \cdot \frac tT = 4A \cdot t\nu = 4 \cdot 5\,\text{см} \cdot 80\,\text{с} \cdot 6\,\text{Гц} = 96{,}0\,\text{м}$
}
\solutionspace{120pt}

\tasknumber{7}%
\task{%
    Изобразите график гармонических колебаний,
    амплитуда которых составляла бы $A = 30\,\text{см}$, а период $T = 4\,\text{с}$.
}
\solutionspace{80pt}

\tasknumber{8}%
\task{%
    Координата материальной точки зависит от времени по закону $y = 0{,}15 \cdot \cos (5\pi t)$ (в СИ).
    Чему равен путь, пройденный точкой за $3\,\text{мин}$?
}
\answer{%
    $\omega = 5\pi \implies \nu = \frac52\,\units{Гц}, N = \nu t = 450{,}0, s = 4AN = 4 \cdot 0{,}15 \cdot 450{,}0 = 270{,}0 \text{(м)}$
}

\variantsplitter

\addpersonalvariant{Алина Полканова}

\tasknumber{1}%
\task{%
    Установите каждой букве в соответствие ровно одну цифру и запишите ответ (только цифры, без других символов).

    А) число колебаний, Б) период колебаний, В) время колебаний.

    1) $T$, 2) $t$, 3) $\frac{\nu}{2\pi}$, 4) $N$, 5) $\frac{2\pi}{\nu}$.
}
\answer{%
    $412$
}

\tasknumber{2}%
\task{%
    Установите каждой букве в соответствие ровно одну цифру и запишите ответ (только цифры, без других символов).

    А) частота колебаний, Б) циклическая частота.

    1) рад / с, 2) м / с, 3) Гц, 4) Гн.
}
\answer{%
    $31$
}

\tasknumber{3}%
\task{%
    \begin{itemize}
        \item Запишите линейное однородное дифференциальное уравнение второго порядка,
            описывающее свободные незатухающие колебания гармонического осциллятора,
        \item запишите общее решение этого уравнения,
        \item подпишите в выписанном решении фазу и амплитуду колебаний,
        \item запишите выражение для скорости,
        \item запишите выражение для ускорения.
    \end{itemize}
}
\answer{%
    \begin{align*}
    &\ddot x + \omega^2 x = 0 \Longleftrightarrow a_x + \omega^2 x = 0, \\
    &x = A \cos(\omega t + \varphi_0) \text{ или же } x = A \sin(\omega t + \varphi_0) \text{ или же } x = a \cos(\omega t) + b \sin(\omega t), \\
    &A \text{\, или \,} \sqrt{a^2 + b^2} \text{ --- это амплитуда}, \omega t + \varphi_0\text{ --- это фаза}, \\
    &v = \dot x = -\omega A \sin(\omega t + \varphi_0), \\
    &a = \dot v = \ddot x = -\omega^2 A \cos(\omega t + \varphi_0) = -\omega^2 x,
    \end{align*}
}
\solutionspace{135pt}

\tasknumber{4}%
\task{%
    Определите частоту колебаний, если их период составляет $T = 10\,\text{мс}$.
}
\answer{%
    $\nu = \frac 1T = \frac 1{10\,\text{мс}} = 100\,\text{Гц}$
}
\solutionspace{40pt}

\tasknumber{5}%
\task{%
    Определите период колебаний, если их частота составляет $\nu = 2\,\text{кГц}$.
    Сколько колебаний произойдёт за $t = 3\,\text{мин}$?
}
\answer{%
    \begin{align*}
    T &= \frac 1\nu = \frac 1{2\,\text{кГц}} = 0{,}500\,\text{мc}, \\
    N &= \nu t = 2\,\text{кГц} \cdot3\,\text{мин} = 360000\,\text{колебаний}.
    \end{align*}
}
\solutionspace{40pt}

\tasknumber{6}%
\task{%
    Амплитуда колебаний точки составляет $A = 2\,\text{см}$, а частота~--- $\nu = 6\,\text{Гц}$.
    Определите, какой путь преодолеет эта точка за $t = 80\,\text{с}$.
}
\answer{%
    $s = 4A \cdot N = 4A \cdot \frac tT = 4A \cdot t\nu = 4 \cdot 2\,\text{см} \cdot 80\,\text{с} \cdot 6\,\text{Гц} = 38{,}4\,\text{м}$
}
\solutionspace{120pt}

\tasknumber{7}%
\task{%
    Изобразите график гармонических колебаний,
    амплитуда которых составляла бы $A = 1\,\text{см}$, а период $T = 10\,\text{с}$.
}
\solutionspace{80pt}

\tasknumber{8}%
\task{%
    Координата материальной точки зависит от времени по закону $y = 0{,}15 \cdot \sin (4\pi t)$ (в СИ).
    Чему равен путь, пройденный точкой за $4\,\text{мин}$?
}
\answer{%
    $\omega = 4\pi \implies \nu = \frac42\,\units{Гц}, N = \nu t = 480{,}0, s = 4AN = 4 \cdot 0{,}15 \cdot 480{,}0 = 288{,}0 \text{(м)}$
}

\variantsplitter

\addpersonalvariant{Сергей Пономарёв}

\tasknumber{1}%
\task{%
    Установите каждой букве в соответствие ровно одну цифру и запишите ответ (только цифры, без других символов).

    А) период колебаний, Б) время колебаний, В) частота колебаний.

    1) $t$, 2) $\frac{\nu}{2\pi}$, 3) $T$, 4) $\nu$, 5) $\omega$.
}
\answer{%
    $314$
}

\tasknumber{2}%
\task{%
    Установите каждой букве в соответствие ровно одну цифру и запишите ответ (только цифры, без других символов).

    А) период колебаний, Б) циклическая частота.

    1) Гц, 2) c, 3) рад / с, 4) м / с.
}
\answer{%
    $23$
}

\tasknumber{3}%
\task{%
    \begin{itemize}
        \item Запишите линейное однородное дифференциальное уравнение второго порядка,
            описывающее свободные незатухающие колебания гармонического осциллятора,
        \item запишите общее решение этого уравнения,
        \item подпишите в выписанном решении фазу и амплитуду колебаний,
        \item запишите выражение для скорости,
        \item запишите выражение для ускорения.
    \end{itemize}
}
\answer{%
    \begin{align*}
    &\ddot x + \omega^2 x = 0 \Longleftrightarrow a_x + \omega^2 x = 0, \\
    &x = A \cos(\omega t + \varphi_0) \text{ или же } x = A \sin(\omega t + \varphi_0) \text{ или же } x = a \cos(\omega t) + b \sin(\omega t), \\
    &A \text{\, или \,} \sqrt{a^2 + b^2} \text{ --- это амплитуда}, \omega t + \varphi_0\text{ --- это фаза}, \\
    &v = \dot x = -\omega A \sin(\omega t + \varphi_0), \\
    &a = \dot v = \ddot x = -\omega^2 A \cos(\omega t + \varphi_0) = -\omega^2 x,
    \end{align*}
}
\solutionspace{135pt}

\tasknumber{4}%
\task{%
    Определите частоту колебаний, если их период составляет $T = 10\,\text{мс}$.
}
\answer{%
    $\nu = \frac 1T = \frac 1{10\,\text{мс}} = 100\,\text{Гц}$
}
\solutionspace{40pt}

\tasknumber{5}%
\task{%
    Определите период колебаний, если их частота составляет $\nu = 4\,\text{кГц}$.
    Сколько колебаний произойдёт за $t = 2\,\text{мин}$?
}
\answer{%
    \begin{align*}
    T &= \frac 1\nu = \frac 1{4\,\text{кГц}} = 0{,}250\,\text{мc}, \\
    N &= \nu t = 4\,\text{кГц} \cdot2\,\text{мин} = 480000\,\text{колебаний}.
    \end{align*}
}
\solutionspace{40pt}

\tasknumber{6}%
\task{%
    Амплитуда колебаний точки составляет $A = 15\,\text{см}$, а частота~--- $\nu = 5\,\text{Гц}$.
    Определите, какой путь преодолеет эта точка за $t = 80\,\text{с}$.
}
\answer{%
    $s = 4A \cdot N = 4A \cdot \frac tT = 4A \cdot t\nu = 4 \cdot 15\,\text{см} \cdot 80\,\text{с} \cdot 5\,\text{Гц} = 240{,}0\,\text{м}$
}
\solutionspace{120pt}

\tasknumber{7}%
\task{%
    Изобразите график гармонических колебаний,
    амплитуда которых составляла бы $A = 30\,\text{см}$, а период $T = 4\,\text{с}$.
}
\solutionspace{80pt}

\tasknumber{8}%
\task{%
    Координата материальной точки зависит от времени по закону $y = 0{,}02 \cdot \sin (5\pi t)$ (в СИ).
    Чему равен путь, пройденный точкой за $2\,\text{мин}$?
}
\answer{%
    $\omega = 5\pi \implies \nu = \frac52\,\units{Гц}, N = \nu t = 300{,}0, s = 4AN = 4 \cdot 0{,}02 \cdot 300{,}0 = 24{,}0 \text{(м)}$
}

\variantsplitter

\addpersonalvariant{Егор Свистушкин}

\tasknumber{1}%
\task{%
    Установите каждой букве в соответствие ровно одну цифру и запишите ответ (только цифры, без других символов).

    А) число колебаний, Б) время колебаний, В) период колебаний.

    1) $N$, 2) $t$, 3) $\nu$, 4) $T$, 5) $\frac{2\pi}{\nu}$.
}
\answer{%
    $124$
}

\tasknumber{2}%
\task{%
    Установите каждой букве в соответствие ровно одну цифру и запишите ответ (только цифры, без других символов).

    А) частота колебаний, Б) период колебаний.

    1) Гц, 2) рад / с, 3) c, 4) Гн.
}
\answer{%
    $13$
}

\tasknumber{3}%
\task{%
    \begin{itemize}
        \item Запишите линейное однородное дифференциальное уравнение второго порядка,
            описывающее свободные незатухающие колебания гармонического осциллятора,
        \item запишите общее решение этого уравнения,
        \item подпишите в выписанном решении фазу и амплитуду колебаний,
        \item запишите выражение для скорости,
        \item запишите выражение для ускорения.
    \end{itemize}
}
\answer{%
    \begin{align*}
    &\ddot x + \omega^2 x = 0 \Longleftrightarrow a_x + \omega^2 x = 0, \\
    &x = A \cos(\omega t + \varphi_0) \text{ или же } x = A \sin(\omega t + \varphi_0) \text{ или же } x = a \cos(\omega t) + b \sin(\omega t), \\
    &A \text{\, или \,} \sqrt{a^2 + b^2} \text{ --- это амплитуда}, \omega t + \varphi_0\text{ --- это фаза}, \\
    &v = \dot x = -\omega A \sin(\omega t + \varphi_0), \\
    &a = \dot v = \ddot x = -\omega^2 A \cos(\omega t + \varphi_0) = -\omega^2 x,
    \end{align*}
}
\solutionspace{135pt}

\tasknumber{4}%
\task{%
    Определите частоту колебаний, если их период составляет $T = 4\,\text{мс}$.
}
\answer{%
    $\nu = \frac 1T = \frac 1{4\,\text{мс}} = 250\,\text{Гц}$
}
\solutionspace{40pt}

\tasknumber{5}%
\task{%
    Определите период колебаний, если их частота составляет $\nu = 2\,\text{кГц}$.
    Сколько колебаний произойдёт за $t = 2\,\text{мин}$?
}
\answer{%
    \begin{align*}
    T &= \frac 1\nu = \frac 1{2\,\text{кГц}} = 0{,}500\,\text{мc}, \\
    N &= \nu t = 2\,\text{кГц} \cdot2\,\text{мин} = 240000\,\text{колебаний}.
    \end{align*}
}
\solutionspace{40pt}

\tasknumber{6}%
\task{%
    Амплитуда колебаний точки составляет $A = 2\,\text{см}$, а частота~--- $\nu = 10\,\text{Гц}$.
    Определите, какой путь преодолеет эта точка за $t = 80\,\text{с}$.
}
\answer{%
    $s = 4A \cdot N = 4A \cdot \frac tT = 4A \cdot t\nu = 4 \cdot 2\,\text{см} \cdot 80\,\text{с} \cdot 10\,\text{Гц} = 64{,}0\,\text{м}$
}
\solutionspace{120pt}

\tasknumber{7}%
\task{%
    Изобразите график гармонических колебаний,
    амплитуда которых составляла бы $A = 40\,\text{см}$, а период $T = 8\,\text{с}$.
}
\solutionspace{80pt}

\tasknumber{8}%
\task{%
    Координата материальной точки зависит от времени по закону $x = 0{,}02 \cdot \sin (4\pi t)$ (в СИ).
    Чему равен путь, пройденный точкой за $4\,\text{мин}$?
}
\answer{%
    $\omega = 4\pi \implies \nu = \frac42\,\units{Гц}, N = \nu t = 480{,}0, s = 4AN = 4 \cdot 0{,}02 \cdot 480{,}0 = 38{,}4 \text{(м)}$
}

\variantsplitter

\addpersonalvariant{Дмитрий Соколов}

\tasknumber{1}%
\task{%
    Установите каждой букве в соответствие ровно одну цифру и запишите ответ (только цифры, без других символов).

    А) период колебаний, Б) число колебаний, В) циклическая частота.

    1) $N$, 2) $\frac{\nu}{2\pi}$, 3) $T$, 4) $tN$, 5) $\omega$.
}
\answer{%
    $315$
}

\tasknumber{2}%
\task{%
    Установите каждой букве в соответствие ровно одну цифру и запишите ответ (только цифры, без других символов).

    А) период колебаний, Б) частота колебаний.

    1) рад / с, 2) c, 3) Гн, 4) Гц.
}
\answer{%
    $24$
}

\tasknumber{3}%
\task{%
    \begin{itemize}
        \item Запишите линейное однородное дифференциальное уравнение второго порядка,
            описывающее свободные незатухающие колебания гармонического осциллятора,
        \item запишите общее решение этого уравнения,
        \item подпишите в выписанном решении фазу и амплитуду колебаний,
        \item запишите выражение для скорости,
        \item запишите выражение для ускорения.
    \end{itemize}
}
\answer{%
    \begin{align*}
    &\ddot x + \omega^2 x = 0 \Longleftrightarrow a_x + \omega^2 x = 0, \\
    &x = A \cos(\omega t + \varphi_0) \text{ или же } x = A \sin(\omega t + \varphi_0) \text{ или же } x = a \cos(\omega t) + b \sin(\omega t), \\
    &A \text{\, или \,} \sqrt{a^2 + b^2} \text{ --- это амплитуда}, \omega t + \varphi_0\text{ --- это фаза}, \\
    &v = \dot x = -\omega A \sin(\omega t + \varphi_0), \\
    &a = \dot v = \ddot x = -\omega^2 A \cos(\omega t + \varphi_0) = -\omega^2 x,
    \end{align*}
}
\solutionspace{135pt}

\tasknumber{4}%
\task{%
    Определите частоту колебаний, если их период составляет $T = 50\,\text{мс}$.
}
\answer{%
    $\nu = \frac 1T = \frac 1{50\,\text{мс}} = 20\,\text{Гц}$
}
\solutionspace{40pt}

\tasknumber{5}%
\task{%
    Определите период колебаний, если их частота составляет $\nu = 5\,\text{кГц}$.
    Сколько колебаний произойдёт за $t = 10\,\text{мин}$?
}
\answer{%
    \begin{align*}
    T &= \frac 1\nu = \frac 1{5\,\text{кГц}} = 0{,}200\,\text{мc}, \\
    N &= \nu t = 5\,\text{кГц} \cdot10\,\text{мин} = 3000000\,\text{колебаний}.
    \end{align*}
}
\solutionspace{40pt}

\tasknumber{6}%
\task{%
    Амплитуда колебаний точки составляет $A = 2\,\text{см}$, а частота~--- $\nu = 5\,\text{Гц}$.
    Определите, какой путь преодолеет эта точка за $t = 80\,\text{с}$.
}
\answer{%
    $s = 4A \cdot N = 4A \cdot \frac tT = 4A \cdot t\nu = 4 \cdot 2\,\text{см} \cdot 80\,\text{с} \cdot 5\,\text{Гц} = 32{,}0\,\text{м}$
}
\solutionspace{120pt}

\tasknumber{7}%
\task{%
    Изобразите график гармонических колебаний,
    амплитуда которых составляла бы $A = 75\,\text{см}$, а период $T = 8\,\text{с}$.
}
\solutionspace{80pt}

\tasknumber{8}%
\task{%
    Координата материальной точки зависит от времени по закону $y = 0{,}05 \cdot \cos (3\pi t)$ (в СИ).
    Чему равен путь, пройденный точкой за $3\,\text{мин}$?
}
\answer{%
    $\omega = 3\pi \implies \nu = \frac32\,\units{Гц}, N = \nu t = 270{,}0, s = 4AN = 4 \cdot 0{,}05 \cdot 270{,}0 = 54{,}0 \text{(м)}$
}

\variantsplitter

\addpersonalvariant{Арсений Трофимов}

\tasknumber{1}%
\task{%
    Установите каждой букве в соответствие ровно одну цифру и запишите ответ (только цифры, без других символов).

    А) время колебаний, Б) циклическая частота, В) число колебаний.

    1) $\frac{2\pi}{\nu}$, 2) $\nu$, 3) $t$, 4) $N$, 5) $\omega$.
}
\answer{%
    $354$
}

\tasknumber{2}%
\task{%
    Установите каждой букве в соответствие ровно одну цифру и запишите ответ (только цифры, без других символов).

    А) период колебаний, Б) частота колебаний.

    1) м / с, 2) c, 3) рад / с, 4) Гц.
}
\answer{%
    $24$
}

\tasknumber{3}%
\task{%
    \begin{itemize}
        \item Запишите линейное однородное дифференциальное уравнение второго порядка,
            описывающее свободные незатухающие колебания гармонического осциллятора,
        \item запишите общее решение этого уравнения,
        \item подпишите в выписанном решении фазу и амплитуду колебаний,
        \item запишите выражение для скорости,
        \item запишите выражение для ускорения.
    \end{itemize}
}
\answer{%
    \begin{align*}
    &\ddot x + \omega^2 x = 0 \Longleftrightarrow a_x + \omega^2 x = 0, \\
    &x = A \cos(\omega t + \varphi_0) \text{ или же } x = A \sin(\omega t + \varphi_0) \text{ или же } x = a \cos(\omega t) + b \sin(\omega t), \\
    &A \text{\, или \,} \sqrt{a^2 + b^2} \text{ --- это амплитуда}, \omega t + \varphi_0\text{ --- это фаза}, \\
    &v = \dot x = -\omega A \sin(\omega t + \varphi_0), \\
    &a = \dot v = \ddot x = -\omega^2 A \cos(\omega t + \varphi_0) = -\omega^2 x,
    \end{align*}
}
\solutionspace{135pt}

\tasknumber{4}%
\task{%
    Определите частоту колебаний, если их период составляет $T = 10\,\text{мс}$.
}
\answer{%
    $\nu = \frac 1T = \frac 1{10\,\text{мс}} = 100\,\text{Гц}$
}
\solutionspace{40pt}

\tasknumber{5}%
\task{%
    Определите период колебаний, если их частота составляет $\nu = 10\,\text{кГц}$.
    Сколько колебаний произойдёт за $t = 1\,\text{мин}$?
}
\answer{%
    \begin{align*}
    T &= \frac 1\nu = \frac 1{10\,\text{кГц}} = 0{,}100\,\text{мc}, \\
    N &= \nu t = 10\,\text{кГц} \cdot1\,\text{мин} = 600000\,\text{колебаний}.
    \end{align*}
}
\solutionspace{40pt}

\tasknumber{6}%
\task{%
    Амплитуда колебаний точки составляет $A = 15\,\text{см}$, а частота~--- $\nu = 6\,\text{Гц}$.
    Определите, какой путь преодолеет эта точка за $t = 80\,\text{с}$.
}
\answer{%
    $s = 4A \cdot N = 4A \cdot \frac tT = 4A \cdot t\nu = 4 \cdot 15\,\text{см} \cdot 80\,\text{с} \cdot 6\,\text{Гц} = 288{,}0\,\text{м}$
}
\solutionspace{120pt}

\tasknumber{7}%
\task{%
    Изобразите график гармонических колебаний,
    амплитуда которых составляла бы $A = 6\,\text{см}$, а период $T = 6\,\text{с}$.
}
\solutionspace{80pt}

\tasknumber{8}%
\task{%
    Координата материальной точки зависит от времени по закону $x = 0{,}02 \cdot \cos (5\pi t)$ (в СИ).
    Чему равен путь, пройденный точкой за $4\,\text{мин}$?
}
\answer{%
    $\omega = 5\pi \implies \nu = \frac52\,\units{Гц}, N = \nu t = 600{,}0, s = 4AN = 4 \cdot 0{,}02 \cdot 600{,}0 = 48{,}0 \text{(м)}$
}
% autogenerated
