\setdate{7~октября~2021}
\setclass{11«Б»}

\addpersonalvariant{Михаил Бурмистров}

\tasknumber{1}%
\task{%
    Установите каждой букве в соответствие ровно одну цифру и запишите ответ (только цифры, без других символов).

    А) число колебаний, Б) частота колебаний, В) циклическая частота.

    1) $N$, 2) $\omega$, 3) $\nu$, 4) $T$, 5) $t$.
}
\answer{%
    $132$
}
\solutionspace{20pt}

\tasknumber{2}%
\task{%
    Установите каждой букве в соответствие ровно одну цифру и запишите ответ (только цифры, без других символов).

    А) циклическая частота, Б) период колебаний.

    1) c, 2) рад / с, 3) Гн, 4) м / с.
}
\answer{%
    $21$
}
\solutionspace{20pt}

\tasknumber{3}%
\task{%
    Определите частоту колебаний, если их период составляет $T = 2\,\text{мс}$.
}
\answer{%
    $\nu = \frac 1T = \frac 1{2\,\text{мс}} = 500\,\text{Гц}$
}
\solutionspace{40pt}

\tasknumber{4}%
\task{%
    Определите период колебаний, если их частота составляет $\nu = 20\,\text{кГц}$.
    Сколько колебаний произойдёт за $t = 5\,\text{мин}$?
}
\answer{%
    \begin{align*}
    T &= \frac 1\nu = \frac 1{20\,\text{кГц}} = 0{,}050\,\text{мc}, \\
    N &= \nu t = 20\,\text{кГц} \cdot5\,\text{мин} = 6000000\,\text{колебаний}.
    \end{align*}
}
\solutionspace{40pt}

\tasknumber{5}%
\task{%
    Амплитуда колебаний точки составляет $A = 15\,\text{см}$, а частота~--- $\nu = 2\,\text{Гц}$.
    Определите, какой путь преодолеет эта точка за $t = 80\,\text{с}$.
}
\answer{%
    $s = 4A \cdot N = 4A \cdot \frac tT = 4A \cdot t\nu = 4 \cdot 15\,\text{см} \cdot 80\,\text{с} \cdot 2\,\text{Гц} = 96\,\text{м}$
}
\solutionspace{120pt}

\tasknumber{6}%
\task{%
    Координата материальной точки зависит от времени по закону $x = 0{,}15 \cdot \sin (4\pi t)$ (в СИ).
    Чему равен путь, пройденный точкой за $4\,\text{мин}$?
}
\answer{%
    $\omega = 4\pi \implies \nu = \frac42\,\units{Гц}, N = \nu t = 480, s = 4AN = 4 \cdot 0{,}15 \cdot 480 = 288{,}00 \text{(м)}$
}

\variantsplitter

\addpersonalvariant{Снежана Авдошина}

\tasknumber{1}%
\task{%
    Установите каждой букве в соответствие ровно одну цифру и запишите ответ (только цифры, без других символов).

    А) циклическая частота, Б) время колебаний, В) число колебаний.

    1) $\nu$, 2) $T$, 3) $\omega$, 4) $t$, 5) $N$.
}
\answer{%
    $345$
}
\solutionspace{20pt}

\tasknumber{2}%
\task{%
    Установите каждой букве в соответствие ровно одну цифру и запишите ответ (только цифры, без других символов).

    А) частота колебаний, Б) циклическая частота.

    1) c, 2) Гц, 3) рад / с, 4) м / с.
}
\answer{%
    $23$
}
\solutionspace{20pt}

\tasknumber{3}%
\task{%
    Определите частоту колебаний, если их период составляет $T = 2\,\text{мс}$.
}
\answer{%
    $\nu = \frac 1T = \frac 1{2\,\text{мс}} = 500\,\text{Гц}$
}
\solutionspace{40pt}

\tasknumber{4}%
\task{%
    Определите период колебаний, если их частота составляет $\nu = 10\,\text{кГц}$.
    Сколько колебаний произойдёт за $t = 2\,\text{мин}$?
}
\answer{%
    \begin{align*}
    T &= \frac 1\nu = \frac 1{10\,\text{кГц}} = 0{,}100\,\text{мc}, \\
    N &= \nu t = 10\,\text{кГц} \cdot2\,\text{мин} = 1200000\,\text{колебаний}.
    \end{align*}
}
\solutionspace{40pt}

\tasknumber{5}%
\task{%
    Амплитуда колебаний точки составляет $A = 15\,\text{см}$, а частота~--- $\nu = 2\,\text{Гц}$.
    Определите, какой путь преодолеет эта точка за $t = 10\,\text{с}$.
}
\answer{%
    $s = 4A \cdot N = 4A \cdot \frac tT = 4A \cdot t\nu = 4 \cdot 15\,\text{см} \cdot 10\,\text{с} \cdot 2\,\text{Гц} = 12\,\text{м}$
}
\solutionspace{120pt}

\tasknumber{6}%
\task{%
    Координата материальной точки зависит от времени по закону $z = 0{,}25 \cdot \sin (5\pi t)$ (в СИ).
    Чему равен путь, пройденный точкой за $2\,\text{мин}$?
}
\answer{%
    $\omega = 5\pi \implies \nu = \frac52\,\units{Гц}, N = \nu t = 300, s = 4AN = 4 \cdot 0{,}25 \cdot 300 = 300{,}00 \text{(м)}$
}

\variantsplitter

\addpersonalvariant{Марьяна Аристова}

\tasknumber{1}%
\task{%
    Установите каждой букве в соответствие ровно одну цифру и запишите ответ (только цифры, без других символов).

    А) время колебаний, Б) частота колебаний, В) циклическая частота.

    1) $T$, 2) $t$, 3) $\omega$, 4) $\frac{\nu}{2\pi}$, 5) $\nu$.
}
\answer{%
    $253$
}
\solutionspace{20pt}

\tasknumber{2}%
\task{%
    Установите каждой букве в соответствие ровно одну цифру и запишите ответ (только цифры, без других символов).

    А) частота колебаний, Б) период колебаний.

    1) Гц, 2) Гн, 3) рад / с, 4) c.
}
\answer{%
    $14$
}
\solutionspace{20pt}

\tasknumber{3}%
\task{%
    Определите частоту колебаний, если их период составляет $T = 2\,\text{мс}$.
}
\answer{%
    $\nu = \frac 1T = \frac 1{2\,\text{мс}} = 500\,\text{Гц}$
}
\solutionspace{40pt}

\tasknumber{4}%
\task{%
    Определите период колебаний, если их частота составляет $\nu = 5\,\text{кГц}$.
    Сколько колебаний произойдёт за $t = 1\,\text{мин}$?
}
\answer{%
    \begin{align*}
    T &= \frac 1\nu = \frac 1{5\,\text{кГц}} = 0{,}200\,\text{мc}, \\
    N &= \nu t = 5\,\text{кГц} \cdot1\,\text{мин} = 300000\,\text{колебаний}.
    \end{align*}
}
\solutionspace{40pt}

\tasknumber{5}%
\task{%
    Амплитуда колебаний точки составляет $A = 2\,\text{см}$, а частота~--- $\nu = 5\,\text{Гц}$.
    Определите, какой путь преодолеет эта точка за $t = 40\,\text{с}$.
}
\answer{%
    $s = 4A \cdot N = 4A \cdot \frac tT = 4A \cdot t\nu = 4 \cdot 2\,\text{см} \cdot 40\,\text{с} \cdot 5\,\text{Гц} = 16\,\text{м}$
}
\solutionspace{120pt}

\tasknumber{6}%
\task{%
    Координата материальной точки зависит от времени по закону $y = 0{,}15 \cdot \sin (3\pi t)$ (в СИ).
    Чему равен путь, пройденный точкой за $2\,\text{мин}$?
}
\answer{%
    $\omega = 3\pi \implies \nu = \frac32\,\units{Гц}, N = \nu t = 180, s = 4AN = 4 \cdot 0{,}15 \cdot 180 = 108{,}00 \text{(м)}$
}

\variantsplitter

\addpersonalvariant{Никита Иванов}

\tasknumber{1}%
\task{%
    Установите каждой букве в соответствие ровно одну цифру и запишите ответ (только цифры, без других символов).

    А) циклическая частота, Б) период колебаний, В) число колебаний.

    1) $\frac{2\pi}{\nu}$, 2) $\frac{\nu}{2\pi}$, 3) $T$, 4) $\omega$, 5) $N$.
}
\answer{%
    $435$
}
\solutionspace{20pt}

\tasknumber{2}%
\task{%
    Установите каждой букве в соответствие ровно одну цифру и запишите ответ (только цифры, без других символов).

    А) циклическая частота, Б) частота колебаний.

    1) м / с, 2) Гц, 3) рад / с, 4) c.
}
\answer{%
    $32$
}
\solutionspace{20pt}

\tasknumber{3}%
\task{%
    Определите частоту колебаний, если их период составляет $T = 50\,\text{мс}$.
}
\answer{%
    $\nu = \frac 1T = \frac 1{50\,\text{мс}} = 20\,\text{Гц}$
}
\solutionspace{40pt}

\tasknumber{4}%
\task{%
    Определите период колебаний, если их частота составляет $\nu = 40\,\text{кГц}$.
    Сколько колебаний произойдёт за $t = 1\,\text{мин}$?
}
\answer{%
    \begin{align*}
    T &= \frac 1\nu = \frac 1{40\,\text{кГц}} = 0{,}025\,\text{мc}, \\
    N &= \nu t = 40\,\text{кГц} \cdot1\,\text{мин} = 2400000\,\text{колебаний}.
    \end{align*}
}
\solutionspace{40pt}

\tasknumber{5}%
\task{%
    Амплитуда колебаний точки составляет $A = 2\,\text{см}$, а частота~--- $\nu = 10\,\text{Гц}$.
    Определите, какой путь преодолеет эта точка за $t = 80\,\text{с}$.
}
\answer{%
    $s = 4A \cdot N = 4A \cdot \frac tT = 4A \cdot t\nu = 4 \cdot 2\,\text{см} \cdot 80\,\text{с} \cdot 10\,\text{Гц} = 64\,\text{м}$
}
\solutionspace{120pt}

\tasknumber{6}%
\task{%
    Координата материальной точки зависит от времени по закону $y = 0{,}25 \cdot \sin (4\pi t)$ (в СИ).
    Чему равен путь, пройденный точкой за $2\,\text{мин}$?
}
\answer{%
    $\omega = 4\pi \implies \nu = \frac42\,\units{Гц}, N = \nu t = 240, s = 4AN = 4 \cdot 0{,}25 \cdot 240 = 240{,}00 \text{(м)}$
}

\variantsplitter

\addpersonalvariant{Анастасия Князева}

\tasknumber{1}%
\task{%
    Установите каждой букве в соответствие ровно одну цифру и запишите ответ (только цифры, без других символов).

    А) период колебаний, Б) циклическая частота, В) частота колебаний.

    1) $\frac{\nu}{2\pi}$, 2) $T$, 3) $\omega$, 4) $\nu$, 5) $tN$.
}
\answer{%
    $234$
}
\solutionspace{20pt}

\tasknumber{2}%
\task{%
    Установите каждой букве в соответствие ровно одну цифру и запишите ответ (только цифры, без других символов).

    А) частота колебаний, Б) циклическая частота.

    1) Гц, 2) рад / с, 3) м / с, 4) c.
}
\answer{%
    $12$
}
\solutionspace{20pt}

\tasknumber{3}%
\task{%
    Определите частоту колебаний, если их период составляет $T = 4\,\text{мс}$.
}
\answer{%
    $\nu = \frac 1T = \frac 1{4\,\text{мс}} = 250\,\text{Гц}$
}
\solutionspace{40pt}

\tasknumber{4}%
\task{%
    Определите период колебаний, если их частота составляет $\nu = 5\,\text{кГц}$.
    Сколько колебаний произойдёт за $t = 1\,\text{мин}$?
}
\answer{%
    \begin{align*}
    T &= \frac 1\nu = \frac 1{5\,\text{кГц}} = 0{,}200\,\text{мc}, \\
    N &= \nu t = 5\,\text{кГц} \cdot1\,\text{мин} = 300000\,\text{колебаний}.
    \end{align*}
}
\solutionspace{40pt}

\tasknumber{5}%
\task{%
    Амплитуда колебаний точки составляет $A = 2\,\text{см}$, а частота~--- $\nu = 10\,\text{Гц}$.
    Определите, какой путь преодолеет эта точка за $t = 10\,\text{с}$.
}
\answer{%
    $s = 4A \cdot N = 4A \cdot \frac tT = 4A \cdot t\nu = 4 \cdot 2\,\text{см} \cdot 10\,\text{с} \cdot 10\,\text{Гц} = 8\,\text{м}$
}
\solutionspace{120pt}

\tasknumber{6}%
\task{%
    Координата материальной точки зависит от времени по закону $x = 0{,}02 \cdot \sin (6\pi t)$ (в СИ).
    Чему равен путь, пройденный точкой за $4\,\text{мин}$?
}
\answer{%
    $\omega = 6\pi \implies \nu = \frac62\,\units{Гц}, N = \nu t = 720, s = 4AN = 4 \cdot 0{,}02 \cdot 720 = 57{,}60 \text{(м)}$
}

\variantsplitter

\addpersonalvariant{Елизавета Кутумова}

\tasknumber{1}%
\task{%
    Установите каждой букве в соответствие ровно одну цифру и запишите ответ (только цифры, без других символов).

    А) период колебаний, Б) циклическая частота, В) число колебаний.

    1) $\omega$, 2) $tN$, 3) $\frac{2\pi}{\nu}$, 4) $N$, 5) $T$.
}
\answer{%
    $514$
}
\solutionspace{20pt}

\tasknumber{2}%
\task{%
    Установите каждой букве в соответствие ровно одну цифру и запишите ответ (только цифры, без других символов).

    А) частота колебаний, Б) циклическая частота.

    1) c, 2) Гн, 3) рад / с, 4) Гц.
}
\answer{%
    $43$
}
\solutionspace{20pt}

\tasknumber{3}%
\task{%
    Определите частоту колебаний, если их период составляет $T = 10\,\text{мс}$.
}
\answer{%
    $\nu = \frac 1T = \frac 1{10\,\text{мс}} = 100\,\text{Гц}$
}
\solutionspace{40pt}

\tasknumber{4}%
\task{%
    Определите период колебаний, если их частота составляет $\nu = 10\,\text{кГц}$.
    Сколько колебаний произойдёт за $t = 3\,\text{мин}$?
}
\answer{%
    \begin{align*}
    T &= \frac 1\nu = \frac 1{10\,\text{кГц}} = 0{,}100\,\text{мc}, \\
    N &= \nu t = 10\,\text{кГц} \cdot3\,\text{мин} = 1800000\,\text{колебаний}.
    \end{align*}
}
\solutionspace{40pt}

\tasknumber{5}%
\task{%
    Амплитуда колебаний точки составляет $A = 2\,\text{см}$, а частота~--- $\nu = 5\,\text{Гц}$.
    Определите, какой путь преодолеет эта точка за $t = 40\,\text{с}$.
}
\answer{%
    $s = 4A \cdot N = 4A \cdot \frac tT = 4A \cdot t\nu = 4 \cdot 2\,\text{см} \cdot 40\,\text{с} \cdot 5\,\text{Гц} = 16\,\text{м}$
}
\solutionspace{120pt}

\tasknumber{6}%
\task{%
    Координата материальной точки зависит от времени по закону $x = 0{,}05 \cdot \cos (5\pi t)$ (в СИ).
    Чему равен путь, пройденный точкой за $4\,\text{мин}$?
}
\answer{%
    $\omega = 5\pi \implies \nu = \frac52\,\units{Гц}, N = \nu t = 600, s = 4AN = 4 \cdot 0{,}05 \cdot 600 = 120{,}00 \text{(м)}$
}

\variantsplitter

\addpersonalvariant{Роксана Мехтиева}

\tasknumber{1}%
\task{%
    Установите каждой букве в соответствие ровно одну цифру и запишите ответ (только цифры, без других символов).

    А) частота колебаний, Б) циклическая частота, В) число колебаний.

    1) $tN$, 2) $\omega$, 3) $\frac{2\pi}{\nu}$, 4) $\nu$, 5) $N$.
}
\answer{%
    $425$
}
\solutionspace{20pt}

\tasknumber{2}%
\task{%
    Установите каждой букве в соответствие ровно одну цифру и запишите ответ (только цифры, без других символов).

    А) период колебаний, Б) циклическая частота.

    1) рад / с, 2) м / с, 3) c, 4) Гн.
}
\answer{%
    $31$
}
\solutionspace{20pt}

\tasknumber{3}%
\task{%
    Определите частоту колебаний, если их период составляет $T = 5\,\text{мс}$.
}
\answer{%
    $\nu = \frac 1T = \frac 1{5\,\text{мс}} = 200\,\text{Гц}$
}
\solutionspace{40pt}

\tasknumber{4}%
\task{%
    Определите период колебаний, если их частота составляет $\nu = 2\,\text{кГц}$.
    Сколько колебаний произойдёт за $t = 5\,\text{мин}$?
}
\answer{%
    \begin{align*}
    T &= \frac 1\nu = \frac 1{2\,\text{кГц}} = 0{,}500\,\text{мc}, \\
    N &= \nu t = 2\,\text{кГц} \cdot5\,\text{мин} = 600000\,\text{колебаний}.
    \end{align*}
}
\solutionspace{40pt}

\tasknumber{5}%
\task{%
    Амплитуда колебаний точки составляет $A = 5\,\text{см}$, а частота~--- $\nu = 10\,\text{Гц}$.
    Определите, какой путь преодолеет эта точка за $t = 10\,\text{с}$.
}
\answer{%
    $s = 4A \cdot N = 4A \cdot \frac tT = 4A \cdot t\nu = 4 \cdot 5\,\text{см} \cdot 10\,\text{с} \cdot 10\,\text{Гц} = 20\,\text{м}$
}
\solutionspace{120pt}

\tasknumber{6}%
\task{%
    Координата материальной точки зависит от времени по закону $y = 0{,}05 \cdot \cos (6\pi t)$ (в СИ).
    Чему равен путь, пройденный точкой за $2\,\text{мин}$?
}
\answer{%
    $\omega = 6\pi \implies \nu = \frac62\,\units{Гц}, N = \nu t = 360, s = 4AN = 4 \cdot 0{,}05 \cdot 360 = 72{,}00 \text{(м)}$
}

\variantsplitter

\addpersonalvariant{Дилноза Нодиршоева}

\tasknumber{1}%
\task{%
    Установите каждой букве в соответствие ровно одну цифру и запишите ответ (только цифры, без других символов).

    А) циклическая частота, Б) число колебаний, В) частота колебаний.

    1) $N$, 2) $\omega$, 3) $\frac{2\pi}{\nu}$, 4) $t$, 5) $\nu$.
}
\answer{%
    $215$
}
\solutionspace{20pt}

\tasknumber{2}%
\task{%
    Установите каждой букве в соответствие ровно одну цифру и запишите ответ (только цифры, без других символов).

    А) период колебаний, Б) циклическая частота.

    1) c, 2) м / с, 3) Гц, 4) рад / с.
}
\answer{%
    $14$
}
\solutionspace{20pt}

\tasknumber{3}%
\task{%
    Определите частоту колебаний, если их период составляет $T = 2\,\text{мс}$.
}
\answer{%
    $\nu = \frac 1T = \frac 1{2\,\text{мс}} = 500\,\text{Гц}$
}
\solutionspace{40pt}

\tasknumber{4}%
\task{%
    Определите период колебаний, если их частота составляет $\nu = 2\,\text{кГц}$.
    Сколько колебаний произойдёт за $t = 2\,\text{мин}$?
}
\answer{%
    \begin{align*}
    T &= \frac 1\nu = \frac 1{2\,\text{кГц}} = 0{,}500\,\text{мc}, \\
    N &= \nu t = 2\,\text{кГц} \cdot2\,\text{мин} = 240000\,\text{колебаний}.
    \end{align*}
}
\solutionspace{40pt}

\tasknumber{5}%
\task{%
    Амплитуда колебаний точки составляет $A = 15\,\text{см}$, а частота~--- $\nu = 2\,\text{Гц}$.
    Определите, какой путь преодолеет эта точка за $t = 80\,\text{с}$.
}
\answer{%
    $s = 4A \cdot N = 4A \cdot \frac tT = 4A \cdot t\nu = 4 \cdot 15\,\text{см} \cdot 80\,\text{с} \cdot 2\,\text{Гц} = 96\,\text{м}$
}
\solutionspace{120pt}

\tasknumber{6}%
\task{%
    Координата материальной точки зависит от времени по закону $x = 0{,}05 \cdot \sin (5\pi t)$ (в СИ).
    Чему равен путь, пройденный точкой за $4\,\text{мин}$?
}
\answer{%
    $\omega = 5\pi \implies \nu = \frac52\,\units{Гц}, N = \nu t = 600, s = 4AN = 4 \cdot 0{,}05 \cdot 600 = 120{,}00 \text{(м)}$
}

\variantsplitter

\addpersonalvariant{Жаклин Пантелеева}

\tasknumber{1}%
\task{%
    Установите каждой букве в соответствие ровно одну цифру и запишите ответ (только цифры, без других символов).

    А) циклическая частота, Б) частота колебаний, В) число колебаний.

    1) $\omega$, 2) $\frac{2\pi}{\nu}$, 3) $\nu$, 4) $t$, 5) $N$.
}
\answer{%
    $135$
}
\solutionspace{20pt}

\tasknumber{2}%
\task{%
    Установите каждой букве в соответствие ровно одну цифру и запишите ответ (только цифры, без других символов).

    А) период колебаний, Б) циклическая частота.

    1) м / с, 2) c, 3) Гн, 4) рад / с.
}
\answer{%
    $24$
}
\solutionspace{20pt}

\tasknumber{3}%
\task{%
    Определите частоту колебаний, если их период составляет $T = 20\,\text{мс}$.
}
\answer{%
    $\nu = \frac 1T = \frac 1{20\,\text{мс}} = 50\,\text{Гц}$
}
\solutionspace{40pt}

\tasknumber{4}%
\task{%
    Определите период колебаний, если их частота составляет $\nu = 2\,\text{кГц}$.
    Сколько колебаний произойдёт за $t = 2\,\text{мин}$?
}
\answer{%
    \begin{align*}
    T &= \frac 1\nu = \frac 1{2\,\text{кГц}} = 0{,}500\,\text{мc}, \\
    N &= \nu t = 2\,\text{кГц} \cdot2\,\text{мин} = 240000\,\text{колебаний}.
    \end{align*}
}
\solutionspace{40pt}

\tasknumber{5}%
\task{%
    Амплитуда колебаний точки составляет $A = 15\,\text{см}$, а частота~--- $\nu = 20\,\text{Гц}$.
    Определите, какой путь преодолеет эта точка за $t = 80\,\text{с}$.
}
\answer{%
    $s = 4A \cdot N = 4A \cdot \frac tT = 4A \cdot t\nu = 4 \cdot 15\,\text{см} \cdot 80\,\text{с} \cdot 20\,\text{Гц} = 960\,\text{м}$
}
\solutionspace{120pt}

\tasknumber{6}%
\task{%
    Координата материальной точки зависит от времени по закону $z = 0{,}25 \cdot \sin (6\pi t)$ (в СИ).
    Чему равен путь, пройденный точкой за $4\,\text{мин}$?
}
\answer{%
    $\omega = 6\pi \implies \nu = \frac62\,\units{Гц}, N = \nu t = 720, s = 4AN = 4 \cdot 0{,}25 \cdot 720 = 720{,}00 \text{(м)}$
}

\variantsplitter

\addpersonalvariant{Артём Переверзев}

\tasknumber{1}%
\task{%
    Установите каждой букве в соответствие ровно одну цифру и запишите ответ (только цифры, без других символов).

    А) период колебаний, Б) время колебаний, В) частота колебаний.

    1) $\frac{\nu}{2\pi}$, 2) $T$, 3) $t$, 4) $\omega$, 5) $\nu$.
}
\answer{%
    $235$
}
\solutionspace{20pt}

\tasknumber{2}%
\task{%
    Установите каждой букве в соответствие ровно одну цифру и запишите ответ (только цифры, без других символов).

    А) частота колебаний, Б) период колебаний.

    1) Гц, 2) c, 3) м / с, 4) рад / с.
}
\answer{%
    $12$
}
\solutionspace{20pt}

\tasknumber{3}%
\task{%
    Определите частоту колебаний, если их период составляет $T = 20\,\text{мс}$.
}
\answer{%
    $\nu = \frac 1T = \frac 1{20\,\text{мс}} = 50\,\text{Гц}$
}
\solutionspace{40pt}

\tasknumber{4}%
\task{%
    Определите период колебаний, если их частота составляет $\nu = 20\,\text{кГц}$.
    Сколько колебаний произойдёт за $t = 10\,\text{мин}$?
}
\answer{%
    \begin{align*}
    T &= \frac 1\nu = \frac 1{20\,\text{кГц}} = 0{,}050\,\text{мc}, \\
    N &= \nu t = 20\,\text{кГц} \cdot10\,\text{мин} = 12000000\,\text{колебаний}.
    \end{align*}
}
\solutionspace{40pt}

\tasknumber{5}%
\task{%
    Амплитуда колебаний точки составляет $A = 10\,\text{см}$, а частота~--- $\nu = 20\,\text{Гц}$.
    Определите, какой путь преодолеет эта точка за $t = 80\,\text{с}$.
}
\answer{%
    $s = 4A \cdot N = 4A \cdot \frac tT = 4A \cdot t\nu = 4 \cdot 10\,\text{см} \cdot 80\,\text{с} \cdot 20\,\text{Гц} = 640\,\text{м}$
}
\solutionspace{120pt}

\tasknumber{6}%
\task{%
    Координата материальной точки зависит от времени по закону $z = 0{,}15 \cdot \cos (3\pi t)$ (в СИ).
    Чему равен путь, пройденный точкой за $3\,\text{мин}$?
}
\answer{%
    $\omega = 3\pi \implies \nu = \frac32\,\units{Гц}, N = \nu t = 270, s = 4AN = 4 \cdot 0{,}15 \cdot 270 = 162{,}00 \text{(м)}$
}

\variantsplitter

\addpersonalvariant{Варвара Пранова}

\tasknumber{1}%
\task{%
    Установите каждой букве в соответствие ровно одну цифру и запишите ответ (только цифры, без других символов).

    А) число колебаний, Б) период колебаний, В) частота колебаний.

    1) $t$, 2) $\frac{2\pi}{\nu}$, 3) $\nu$, 4) $N$, 5) $T$.
}
\answer{%
    $453$
}
\solutionspace{20pt}

\tasknumber{2}%
\task{%
    Установите каждой букве в соответствие ровно одну цифру и запишите ответ (только цифры, без других символов).

    А) циклическая частота, Б) период колебаний.

    1) Гн, 2) Гц, 3) рад / с, 4) c.
}
\answer{%
    $34$
}
\solutionspace{20pt}

\tasknumber{3}%
\task{%
    Определите частоту колебаний, если их период составляет $T = 2\,\text{мс}$.
}
\answer{%
    $\nu = \frac 1T = \frac 1{2\,\text{мс}} = 500\,\text{Гц}$
}
\solutionspace{40pt}

\tasknumber{4}%
\task{%
    Определите период колебаний, если их частота составляет $\nu = 50\,\text{кГц}$.
    Сколько колебаний произойдёт за $t = 10\,\text{мин}$?
}
\answer{%
    \begin{align*}
    T &= \frac 1\nu = \frac 1{50\,\text{кГц}} = 0{,}020\,\text{мc}, \\
    N &= \nu t = 50\,\text{кГц} \cdot10\,\text{мин} = 30000000\,\text{колебаний}.
    \end{align*}
}
\solutionspace{40pt}

\tasknumber{5}%
\task{%
    Амплитуда колебаний точки составляет $A = 3\,\text{см}$, а частота~--- $\nu = 6\,\text{Гц}$.
    Определите, какой путь преодолеет эта точка за $t = 80\,\text{с}$.
}
\answer{%
    $s = 4A \cdot N = 4A \cdot \frac tT = 4A \cdot t\nu = 4 \cdot 3\,\text{см} \cdot 80\,\text{с} \cdot 6\,\text{Гц} = 57{,}6\,\text{м}$
}
\solutionspace{120pt}

\tasknumber{6}%
\task{%
    Координата материальной точки зависит от времени по закону $y = 0{,}25 \cdot \sin (5\pi t)$ (в СИ).
    Чему равен путь, пройденный точкой за $4\,\text{мин}$?
}
\answer{%
    $\omega = 5\pi \implies \nu = \frac52\,\units{Гц}, N = \nu t = 600, s = 4AN = 4 \cdot 0{,}25 \cdot 600 = 600{,}00 \text{(м)}$
}

\variantsplitter

\addpersonalvariant{Марьям Салимова}

\tasknumber{1}%
\task{%
    Установите каждой букве в соответствие ровно одну цифру и запишите ответ (только цифры, без других символов).

    А) циклическая частота, Б) частота колебаний, В) период колебаний.

    1) $\frac{2\pi}{\nu}$, 2) $\frac{\nu}{2\pi}$, 3) $T$, 4) $\omega$, 5) $\nu$.
}
\answer{%
    $453$
}
\solutionspace{20pt}

\tasknumber{2}%
\task{%
    Установите каждой букве в соответствие ровно одну цифру и запишите ответ (только цифры, без других символов).

    А) период колебаний, Б) частота колебаний.

    1) Гн, 2) рад / с, 3) c, 4) Гц.
}
\answer{%
    $34$
}
\solutionspace{20pt}

\tasknumber{3}%
\task{%
    Определите частоту колебаний, если их период составляет $T = 50\,\text{мс}$.
}
\answer{%
    $\nu = \frac 1T = \frac 1{50\,\text{мс}} = 20\,\text{Гц}$
}
\solutionspace{40pt}

\tasknumber{4}%
\task{%
    Определите период колебаний, если их частота составляет $\nu = 40\,\text{кГц}$.
    Сколько колебаний произойдёт за $t = 5\,\text{мин}$?
}
\answer{%
    \begin{align*}
    T &= \frac 1\nu = \frac 1{40\,\text{кГц}} = 0{,}025\,\text{мc}, \\
    N &= \nu t = 40\,\text{кГц} \cdot5\,\text{мин} = 12000000\,\text{колебаний}.
    \end{align*}
}
\solutionspace{40pt}

\tasknumber{5}%
\task{%
    Амплитуда колебаний точки составляет $A = 2\,\text{см}$, а частота~--- $\nu = 5\,\text{Гц}$.
    Определите, какой путь преодолеет эта точка за $t = 10\,\text{с}$.
}
\answer{%
    $s = 4A \cdot N = 4A \cdot \frac tT = 4A \cdot t\nu = 4 \cdot 2\,\text{см} \cdot 10\,\text{с} \cdot 5\,\text{Гц} = 4\,\text{м}$
}
\solutionspace{120pt}

\tasknumber{6}%
\task{%
    Координата материальной точки зависит от времени по закону $x = 0{,}05 \cdot \sin (3\pi t)$ (в СИ).
    Чему равен путь, пройденный точкой за $4\,\text{мин}$?
}
\answer{%
    $\omega = 3\pi \implies \nu = \frac32\,\units{Гц}, N = \nu t = 360, s = 4AN = 4 \cdot 0{,}05 \cdot 360 = 72{,}00 \text{(м)}$
}

\variantsplitter

\addpersonalvariant{Юлия Шевченко}

\tasknumber{1}%
\task{%
    Установите каждой букве в соответствие ровно одну цифру и запишите ответ (только цифры, без других символов).

    А) циклическая частота, Б) число колебаний, В) период колебаний.

    1) $\omega$, 2) $N$, 3) $\frac{2\pi}{\nu}$, 4) $T$, 5) $t$.
}
\answer{%
    $124$
}
\solutionspace{20pt}

\tasknumber{2}%
\task{%
    Установите каждой букве в соответствие ровно одну цифру и запишите ответ (только цифры, без других символов).

    А) частота колебаний, Б) период колебаний.

    1) Гц, 2) м / с, 3) c, 4) Гн.
}
\answer{%
    $13$
}
\solutionspace{20pt}

\tasknumber{3}%
\task{%
    Определите частоту колебаний, если их период составляет $T = 4\,\text{мс}$.
}
\answer{%
    $\nu = \frac 1T = \frac 1{4\,\text{мс}} = 250\,\text{Гц}$
}
\solutionspace{40pt}

\tasknumber{4}%
\task{%
    Определите период колебаний, если их частота составляет $\nu = 40\,\text{кГц}$.
    Сколько колебаний произойдёт за $t = 3\,\text{мин}$?
}
\answer{%
    \begin{align*}
    T &= \frac 1\nu = \frac 1{40\,\text{кГц}} = 0{,}025\,\text{мc}, \\
    N &= \nu t = 40\,\text{кГц} \cdot3\,\text{мин} = 7200000\,\text{колебаний}.
    \end{align*}
}
\solutionspace{40pt}

\tasknumber{5}%
\task{%
    Амплитуда колебаний точки составляет $A = 2\,\text{см}$, а частота~--- $\nu = 5\,\text{Гц}$.
    Определите, какой путь преодолеет эта точка за $t = 80\,\text{с}$.
}
\answer{%
    $s = 4A \cdot N = 4A \cdot \frac tT = 4A \cdot t\nu = 4 \cdot 2\,\text{см} \cdot 80\,\text{с} \cdot 5\,\text{Гц} = 32\,\text{м}$
}
\solutionspace{120pt}

\tasknumber{6}%
\task{%
    Координата материальной точки зависит от времени по закону $x = 0{,}02 \cdot \sin (3\pi t)$ (в СИ).
    Чему равен путь, пройденный точкой за $2\,\text{мин}$?
}
\answer{%
    $\omega = 3\pi \implies \nu = \frac32\,\units{Гц}, N = \nu t = 180, s = 4AN = 4 \cdot 0{,}02 \cdot 180 = 14{,}40 \text{(м)}$
}
% autogenerated
