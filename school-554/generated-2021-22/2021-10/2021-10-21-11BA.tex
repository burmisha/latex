\setdate{21~октября~2021}
\setclass{11«БА»}

\addpersonalvariant{Михаил Бурмистров}

\tasknumber{1}%
\task{%
    Схематично изобразите колебательный контур.
    Запишите формулу для периода колебаний в колебательном контуре и ...
    \begin{itemize}
        \item подпишите все физические величины,
        \item укажите их единицы измерения,
        \item выразите из формулы периода циклическую частоту,
        \item выразите из формулы периода выразите емкость конденсатора.
    \end{itemize}
}
\answer{%
    \begin{align*}
    T &= 2\pi\sqrt{LC} \\
    \nu &= \frac 1{2\pi\sqrt{LC}}, \\
    \omega &= \frac 1{\sqrt{LC}}, \\
    L &= \frac 1C \sqr{\frac T{2\pi}}, \\
    C &= \frac 1L \sqr{\frac T{2\pi}}.
    \end{align*}
}
\solutionspace{80pt}

\tasknumber{2}%
\task{%
    Оказалось, что наибольший заряд конденсатора в колебательном контуре равен $60\,\text{мкКл}$,
    а максимальный ток — $270\,\text{мА}$.
    Определите частоту колебаний.
}
\answer{%
    $
        \eli_{\max} = q_{\max}\omega \implies \nu = \frac{\omega}{2\pi} = \frac{\eli_{\max}}{2\pi q} \approx 0{,}716\,\text{кГц}
    $
}
\solutionspace{80pt}

\tasknumber{3}%
\task{%
    В колебательном контура сила тока изменяется
    по закону $\eli=0{,}25\sin(12t)$ (в СИ).
    Индуктивность катушки при этом равна $50\,\text{мГн}$.
    Определите:
    \begin{itemize}
        \item период колебаний,
        \item ёмкость конденсатора,
        \item максимальный заряд конденсатора.
    \end{itemize}
}
\answer{%
    \begin{align*}
    \omega &= 12\units{c}^-1, \\
    T &= \frac{2\pi}\omega \approx 523{,}6\,\text{мc}, \\
    C &= \frac 1{\omega^2 L} \approx 138{,}9\,\text{мФ}, \\
    q &= \frac{\eli_{\max}}\omega  \approx 20{,}8\,\text{мКл}.
    \end{align*}
}
\solutionspace{80pt}

\tasknumber{4}%
\task{%
    Электрический колебательный контур состоит
    из катушки индуктивностью $L$ и конденсатора ёмкостью $C$.
    параллельно катушке подключают ещё одну катушку индуктивностью $\frac13L$.
    Как изменится период сводобных колебаний в контуре?
}
\answer{%
    \begin{align*}
    T &= 2\pi\sqrt{LC} \\
    T' &= 2\pi\sqrt{L'C'}= T \sqrt{\frac{L'}L \cdot \frac{C'}C} = T \sqrt{ \frac14 \cdot 1 } \\
    &\frac{T'}T = \sqrt{ \frac14 \cdot 1 } \approx 0{,}062
    \end{align*}
}

\variantsplitter

\addpersonalvariant{Ирина Ан}

\tasknumber{1}%
\task{%
    Схематично изобразите колебательный контур.
    Запишите формулу для периода колебаний в колебательном контуре и ...
    \begin{itemize}
        \item подпишите все физические величины,
        \item укажите их единицы измерения,
        \item выразите из формулы периода циклическую частоту,
        \item выразите из формулы периода выразите емкость конденсатора.
    \end{itemize}
}
\answer{%
    \begin{align*}
    T &= 2\pi\sqrt{LC} \\
    \nu &= \frac 1{2\pi\sqrt{LC}}, \\
    \omega &= \frac 1{\sqrt{LC}}, \\
    L &= \frac 1C \sqr{\frac T{2\pi}}, \\
    C &= \frac 1L \sqr{\frac T{2\pi}}.
    \end{align*}
}
\solutionspace{80pt}

\tasknumber{2}%
\task{%
    Оказалось, что наибольший заряд конденсатора в колебательном контуре равен $40\,\text{мкКл}$,
    а максимальный ток — $240\,\text{мА}$.
    Определите частоту колебаний.
}
\answer{%
    $
        \eli_{\max} = q_{\max}\omega \implies \nu = \frac{\omega}{2\pi} = \frac{\eli_{\max}}{2\pi q} \approx 0{,}955\,\text{кГц}
    $
}
\solutionspace{80pt}

\tasknumber{3}%
\task{%
    В колебательном контура сила тока изменяется
    по закону $\eli=0{,}05\cos(12t)$ (в СИ).
    Индуктивность катушки при этом равна $60\,\text{мГн}$.
    Определите:
    \begin{itemize}
        \item период колебаний,
        \item ёмкость конденсатора,
        \item максимальный заряд конденсатора.
    \end{itemize}
}
\answer{%
    \begin{align*}
    \omega &= 12\units{c}^-1, \\
    T &= \frac{2\pi}\omega \approx 523{,}6\,\text{мc}, \\
    C &= \frac 1{\omega^2 L} \approx 115{,}7\,\text{мФ}, \\
    q &= \frac{\eli_{\max}}\omega  \approx 4{,}2\,\text{мКл}.
    \end{align*}
}
\solutionspace{80pt}

\tasknumber{4}%
\task{%
    Электрический колебательный контур состоит
    из катушки индуктивностью $L$ и конденсатора ёмкостью $C$.
    последовательно катушке подключают ещё одну катушку индуктивностью $\frac12L$.
    Как изменится период сводобных колебаний в контуре?
}
\answer{%
    \begin{align*}
    T &= 2\pi\sqrt{LC} \\
    T' &= 2\pi\sqrt{L'C'}= T \sqrt{\frac{L'}L \cdot \frac{C'}C} = T \sqrt{ \frac32 \cdot 1 } \\
    &\frac{T'}T = \sqrt{ \frac32 \cdot 1 } \approx 2{,}250
    \end{align*}
}

\variantsplitter

\addpersonalvariant{Софья Андрианова}

\tasknumber{1}%
\task{%
    Схематично изобразите колебательный контур.
    Запишите формулу для периода колебаний в колебательном контуре и ...
    \begin{itemize}
        \item подпишите все физические величины,
        \item укажите их единицы измерения,
        \item выразите из формулы периода циклическую частоту,
        \item выразите из формулы периода выразите емкость конденсатора.
    \end{itemize}
}
\answer{%
    \begin{align*}
    T &= 2\pi\sqrt{LC} \\
    \nu &= \frac 1{2\pi\sqrt{LC}}, \\
    \omega &= \frac 1{\sqrt{LC}}, \\
    L &= \frac 1C \sqr{\frac T{2\pi}}, \\
    C &= \frac 1L \sqr{\frac T{2\pi}}.
    \end{align*}
}
\solutionspace{80pt}

\tasknumber{2}%
\task{%
    Оказалось, что наибольший заряд конденсатора в колебательном контуре равен $80\,\text{мкКл}$,
    а максимальный ток — $270\,\text{мА}$.
    Определите частоту колебаний.
}
\answer{%
    $
        \eli_{\max} = q_{\max}\omega \implies \nu = \frac{\omega}{2\pi} = \frac{\eli_{\max}}{2\pi q} \approx 0{,}537\,\text{кГц}
    $
}
\solutionspace{80pt}

\tasknumber{3}%
\task{%
    В колебательном контура сила тока изменяется
    по закону $\eli=0{,}05\sin(15t)$ (в СИ).
    Индуктивность катушки при этом равна $70\,\text{мГн}$.
    Определите:
    \begin{itemize}
        \item период колебаний,
        \item ёмкость конденсатора,
        \item максимальный заряд конденсатора.
    \end{itemize}
}
\answer{%
    \begin{align*}
    \omega &= 15\units{c}^-1, \\
    T &= \frac{2\pi}\omega \approx 418{,}9\,\text{мc}, \\
    C &= \frac 1{\omega^2 L} \approx 63{,}5\,\text{мФ}, \\
    q &= \frac{\eli_{\max}}\omega  \approx 3{,}3\,\text{мКл}.
    \end{align*}
}
\solutionspace{80pt}

\tasknumber{4}%
\task{%
    Электрический колебательный контур состоит
    из катушки индуктивностью $L$ и конденсатора ёмкостью $C$.
    параллельно катушке подключают ещё одну катушку индуктивностью $3L$.
    Как изменится период сводобных колебаний в контуре?
}
\answer{%
    \begin{align*}
    T &= 2\pi\sqrt{LC} \\
    T' &= 2\pi\sqrt{L'C'}= T \sqrt{\frac{L'}L \cdot \frac{C'}C} = T \sqrt{ \frac34 \cdot 1 } \\
    &\frac{T'}T = \sqrt{ \frac34 \cdot 1 } \approx 0{,}562
    \end{align*}
}

\variantsplitter

\addpersonalvariant{Владимир Артемчук}

\tasknumber{1}%
\task{%
    Схематично изобразите колебательный контур.
    Запишите формулу для периода колебаний в колебательном контуре и ...
    \begin{itemize}
        \item подпишите все физические величины,
        \item укажите их единицы измерения,
        \item выразите из формулы периода циклическую частоту,
        \item выразите из формулы периода выразите емкость конденсатора.
    \end{itemize}
}
\answer{%
    \begin{align*}
    T &= 2\pi\sqrt{LC} \\
    \nu &= \frac 1{2\pi\sqrt{LC}}, \\
    \omega &= \frac 1{\sqrt{LC}}, \\
    L &= \frac 1C \sqr{\frac T{2\pi}}, \\
    C &= \frac 1L \sqr{\frac T{2\pi}}.
    \end{align*}
}
\solutionspace{80pt}

\tasknumber{2}%
\task{%
    Оказалось, что наибольший заряд конденсатора в колебательном контуре равен $80\,\text{мкКл}$,
    а максимальный ток — $120\,\text{мА}$.
    Определите частоту колебаний.
}
\answer{%
    $
        \eli_{\max} = q_{\max}\omega \implies \nu = \frac{\omega}{2\pi} = \frac{\eli_{\max}}{2\pi q} \approx 0{,}239\,\text{кГц}
    $
}
\solutionspace{80pt}

\tasknumber{3}%
\task{%
    В колебательном контура сила тока изменяется
    по закону $\eli=0{,}30\sin(12t)$ (в СИ).
    Индуктивность катушки при этом равна $60\,\text{мГн}$.
    Определите:
    \begin{itemize}
        \item период колебаний,
        \item ёмкость конденсатора,
        \item максимальный заряд конденсатора.
    \end{itemize}
}
\answer{%
    \begin{align*}
    \omega &= 12\units{c}^-1, \\
    T &= \frac{2\pi}\omega \approx 523{,}6\,\text{мc}, \\
    C &= \frac 1{\omega^2 L} \approx 115{,}7\,\text{мФ}, \\
    q &= \frac{\eli_{\max}}\omega  \approx 25{,}0\,\text{мКл}.
    \end{align*}
}
\solutionspace{80pt}

\tasknumber{4}%
\task{%
    Электрический колебательный контур состоит
    из катушки индуктивностью $L$ и конденсатора ёмкостью $C$.
    последовательно конденсатору подключают ещё один конденсатор емкостью $\frac13L$.
    Как изменится период сводобных колебаний в контуре?
}
\answer{%
    \begin{align*}
    T &= 2\pi\sqrt{LC} \\
    T' &= 2\pi\sqrt{L'C'}= T \sqrt{\frac{L'}L \cdot \frac{C'}C} = T \sqrt{ 1 \cdot \frac14 } \\
    &\frac{T'}T = \sqrt{ 1 \cdot \frac14 } \approx 0{,}062
    \end{align*}
}

\variantsplitter

\addpersonalvariant{Софья Белянкина}

\tasknumber{1}%
\task{%
    Схематично изобразите колебательный контур.
    Запишите формулу для периода колебаний в колебательном контуре и ...
    \begin{itemize}
        \item подпишите все физические величины,
        \item укажите их единицы измерения,
        \item выразите из формулы периода частоту,
        \item выразите из формулы периода индуктивность катушки..
    \end{itemize}
}
\answer{%
    \begin{align*}
    T &= 2\pi\sqrt{LC} \\
    \nu &= \frac 1{2\pi\sqrt{LC}}, \\
    \omega &= \frac 1{\sqrt{LC}}, \\
    L &= \frac 1C \sqr{\frac T{2\pi}}, \\
    C &= \frac 1L \sqr{\frac T{2\pi}}.
    \end{align*}
}
\solutionspace{80pt}

\tasknumber{2}%
\task{%
    Оказалось, что наибольший заряд конденсатора в колебательном контуре равен $40\,\text{мкКл}$,
    а максимальный ток — $180\,\text{мА}$.
    Определите частоту колебаний.
}
\answer{%
    $
        \eli_{\max} = q_{\max}\omega \implies \nu = \frac{\omega}{2\pi} = \frac{\eli_{\max}}{2\pi q} \approx 0{,}716\,\text{кГц}
    $
}
\solutionspace{80pt}

\tasknumber{3}%
\task{%
    В колебательном контура сила тока изменяется
    по закону $\eli=0{,}25\sin(18t)$ (в СИ).
    Индуктивность катушки при этом равна $80\,\text{мГн}$.
    Определите:
    \begin{itemize}
        \item период колебаний,
        \item ёмкость конденсатора,
        \item максимальный заряд конденсатора.
    \end{itemize}
}
\answer{%
    \begin{align*}
    \omega &= 18\units{c}^-1, \\
    T &= \frac{2\pi}\omega \approx 349{,}1\,\text{мc}, \\
    C &= \frac 1{\omega^2 L} \approx 38{,}6\,\text{мФ}, \\
    q &= \frac{\eli_{\max}}\omega  \approx 13{,}9\,\text{мКл}.
    \end{align*}
}
\solutionspace{80pt}

\tasknumber{4}%
\task{%
    Электрический колебательный контур состоит
    из катушки индуктивностью $L$ и конденсатора ёмкостью $C$.
    последовательно конденсатору подключают ещё один конденсатор емкостью $2L$.
    Как изменится период сводобных колебаний в контуре?
}
\answer{%
    \begin{align*}
    T &= 2\pi\sqrt{LC} \\
    T' &= 2\pi\sqrt{L'C'}= T \sqrt{\frac{L'}L \cdot \frac{C'}C} = T \sqrt{ 1 \cdot \frac23 } \\
    &\frac{T'}T = \sqrt{ 1 \cdot \frac23 } \approx 0{,}444
    \end{align*}
}

\variantsplitter

\addpersonalvariant{Варвара Егиазарян}

\tasknumber{1}%
\task{%
    Схематично изобразите колебательный контур.
    Запишите формулу для периода колебаний в колебательном контуре и ...
    \begin{itemize}
        \item подпишите все физические величины,
        \item укажите их единицы измерения,
        \item выразите из формулы периода частоту,
        \item выразите из формулы периода индуктивность катушки..
    \end{itemize}
}
\answer{%
    \begin{align*}
    T &= 2\pi\sqrt{LC} \\
    \nu &= \frac 1{2\pi\sqrt{LC}}, \\
    \omega &= \frac 1{\sqrt{LC}}, \\
    L &= \frac 1C \sqr{\frac T{2\pi}}, \\
    C &= \frac 1L \sqr{\frac T{2\pi}}.
    \end{align*}
}
\solutionspace{80pt}

\tasknumber{2}%
\task{%
    Оказалось, что наибольший заряд конденсатора в колебательном контуре равен $60\,\text{мкКл}$,
    а максимальный ток — $270\,\text{мА}$.
    Определите частоту колебаний.
}
\answer{%
    $
        \eli_{\max} = q_{\max}\omega \implies \nu = \frac{\omega}{2\pi} = \frac{\eli_{\max}}{2\pi q} \approx 0{,}716\,\text{кГц}
    $
}
\solutionspace{80pt}

\tasknumber{3}%
\task{%
    В колебательном контура сила тока изменяется
    по закону $\eli=0{,}25\sin(12t)$ (в СИ).
    Индуктивность катушки при этом равна $50\,\text{мГн}$.
    Определите:
    \begin{itemize}
        \item период колебаний,
        \item ёмкость конденсатора,
        \item максимальный заряд конденсатора.
    \end{itemize}
}
\answer{%
    \begin{align*}
    \omega &= 12\units{c}^-1, \\
    T &= \frac{2\pi}\omega \approx 523{,}6\,\text{мc}, \\
    C &= \frac 1{\omega^2 L} \approx 138{,}9\,\text{мФ}, \\
    q &= \frac{\eli_{\max}}\omega  \approx 20{,}8\,\text{мКл}.
    \end{align*}
}
\solutionspace{80pt}

\tasknumber{4}%
\task{%
    Электрический колебательный контур состоит
    из катушки индуктивностью $L$ и конденсатора ёмкостью $C$.
    параллельно конденсатору подключают ещё один конденсатор емкостью $2L$.
    Как изменится период сводобных колебаний в контуре?
}
\answer{%
    \begin{align*}
    T &= 2\pi\sqrt{LC} \\
    T' &= 2\pi\sqrt{L'C'}= T \sqrt{\frac{L'}L \cdot \frac{C'}C} = T \sqrt{ 1 \cdot 3 } \\
    &\frac{T'}T = \sqrt{ 1 \cdot 3 } \approx 9{,}000
    \end{align*}
}

\variantsplitter

\addpersonalvariant{Владислав Емелин}

\tasknumber{1}%
\task{%
    Схематично изобразите колебательный контур.
    Запишите формулу для периода колебаний в колебательном контуре и ...
    \begin{itemize}
        \item подпишите все физические величины,
        \item укажите их единицы измерения,
        \item выразите из формулы периода циклическую частоту,
        \item выразите из формулы периода выразите емкость конденсатора.
    \end{itemize}
}
\answer{%
    \begin{align*}
    T &= 2\pi\sqrt{LC} \\
    \nu &= \frac 1{2\pi\sqrt{LC}}, \\
    \omega &= \frac 1{\sqrt{LC}}, \\
    L &= \frac 1C \sqr{\frac T{2\pi}}, \\
    C &= \frac 1L \sqr{\frac T{2\pi}}.
    \end{align*}
}
\solutionspace{80pt}

\tasknumber{2}%
\task{%
    Оказалось, что наибольший заряд конденсатора в колебательном контуре равен $60\,\text{мкКл}$,
    а максимальный ток — $270\,\text{мА}$.
    Определите частоту колебаний.
}
\answer{%
    $
        \eli_{\max} = q_{\max}\omega \implies \nu = \frac{\omega}{2\pi} = \frac{\eli_{\max}}{2\pi q} \approx 0{,}716\,\text{кГц}
    $
}
\solutionspace{80pt}

\tasknumber{3}%
\task{%
    В колебательном контура сила тока изменяется
    по закону $\eli=0{,}25\sin(15t)$ (в СИ).
    Индуктивность катушки при этом равна $50\,\text{мГн}$.
    Определите:
    \begin{itemize}
        \item период колебаний,
        \item ёмкость конденсатора,
        \item максимальный заряд конденсатора.
    \end{itemize}
}
\answer{%
    \begin{align*}
    \omega &= 15\units{c}^-1, \\
    T &= \frac{2\pi}\omega \approx 418{,}9\,\text{мc}, \\
    C &= \frac 1{\omega^2 L} \approx 88{,}9\,\text{мФ}, \\
    q &= \frac{\eli_{\max}}\omega  \approx 16{,}7\,\text{мКл}.
    \end{align*}
}
\solutionspace{80pt}

\tasknumber{4}%
\task{%
    Электрический колебательный контур состоит
    из катушки индуктивностью $L$ и конденсатора ёмкостью $C$.
    последовательно катушке подключают ещё одну катушку индуктивностью $\frac13L$.
    Как изменится период сводобных колебаний в контуре?
}
\answer{%
    \begin{align*}
    T &= 2\pi\sqrt{LC} \\
    T' &= 2\pi\sqrt{L'C'}= T \sqrt{\frac{L'}L \cdot \frac{C'}C} = T \sqrt{ \frac43 \cdot 1 } \\
    &\frac{T'}T = \sqrt{ \frac43 \cdot 1 } \approx 1{,}778
    \end{align*}
}

\variantsplitter

\addpersonalvariant{Артём Жичин}

\tasknumber{1}%
\task{%
    Схематично изобразите колебательный контур.
    Запишите формулу для периода колебаний в колебательном контуре и ...
    \begin{itemize}
        \item подпишите все физические величины,
        \item укажите их единицы измерения,
        \item выразите из формулы периода частоту,
        \item выразите из формулы периода индуктивность катушки..
    \end{itemize}
}
\answer{%
    \begin{align*}
    T &= 2\pi\sqrt{LC} \\
    \nu &= \frac 1{2\pi\sqrt{LC}}, \\
    \omega &= \frac 1{\sqrt{LC}}, \\
    L &= \frac 1C \sqr{\frac T{2\pi}}, \\
    C &= \frac 1L \sqr{\frac T{2\pi}}.
    \end{align*}
}
\solutionspace{80pt}

\tasknumber{2}%
\task{%
    Оказалось, что наибольший заряд конденсатора в колебательном контуре равен $80\,\text{мкКл}$,
    а максимальный ток — $120\,\text{мА}$.
    Определите частоту колебаний.
}
\answer{%
    $
        \eli_{\max} = q_{\max}\omega \implies \nu = \frac{\omega}{2\pi} = \frac{\eli_{\max}}{2\pi q} \approx 0{,}239\,\text{кГц}
    $
}
\solutionspace{80pt}

\tasknumber{3}%
\task{%
    В колебательном контура сила тока изменяется
    по закону $\eli=0{,}30\cos(18t)$ (в СИ).
    Индуктивность катушки при этом равна $50\,\text{мГн}$.
    Определите:
    \begin{itemize}
        \item период колебаний,
        \item ёмкость конденсатора,
        \item максимальный заряд конденсатора.
    \end{itemize}
}
\answer{%
    \begin{align*}
    \omega &= 18\units{c}^-1, \\
    T &= \frac{2\pi}\omega \approx 349{,}1\,\text{мc}, \\
    C &= \frac 1{\omega^2 L} \approx 61{,}7\,\text{мФ}, \\
    q &= \frac{\eli_{\max}}\omega  \approx 16{,}7\,\text{мКл}.
    \end{align*}
}
\solutionspace{80pt}

\tasknumber{4}%
\task{%
    Электрический колебательный контур состоит
    из катушки индуктивностью $L$ и конденсатора ёмкостью $C$.
    последовательно конденсатору подключают ещё один конденсатор емкостью $\frac13L$.
    Как изменится период сводобных колебаний в контуре?
}
\answer{%
    \begin{align*}
    T &= 2\pi\sqrt{LC} \\
    T' &= 2\pi\sqrt{L'C'}= T \sqrt{\frac{L'}L \cdot \frac{C'}C} = T \sqrt{ 1 \cdot \frac14 } \\
    &\frac{T'}T = \sqrt{ 1 \cdot \frac14 } \approx 0{,}062
    \end{align*}
}

\variantsplitter

\addpersonalvariant{Дарья Кошман}

\tasknumber{1}%
\task{%
    Схематично изобразите колебательный контур.
    Запишите формулу для периода колебаний в колебательном контуре и ...
    \begin{itemize}
        \item подпишите все физические величины,
        \item укажите их единицы измерения,
        \item выразите из формулы периода частоту,
        \item выразите из формулы периода индуктивность катушки..
    \end{itemize}
}
\answer{%
    \begin{align*}
    T &= 2\pi\sqrt{LC} \\
    \nu &= \frac 1{2\pi\sqrt{LC}}, \\
    \omega &= \frac 1{\sqrt{LC}}, \\
    L &= \frac 1C \sqr{\frac T{2\pi}}, \\
    C &= \frac 1L \sqr{\frac T{2\pi}}.
    \end{align*}
}
\solutionspace{80pt}

\tasknumber{2}%
\task{%
    Оказалось, что наибольший заряд конденсатора в колебательном контуре равен $60\,\text{мкКл}$,
    а максимальный ток — $180\,\text{мА}$.
    Определите частоту колебаний.
}
\answer{%
    $
        \eli_{\max} = q_{\max}\omega \implies \nu = \frac{\omega}{2\pi} = \frac{\eli_{\max}}{2\pi q} \approx 0{,}477\,\text{кГц}
    $
}
\solutionspace{80pt}

\tasknumber{3}%
\task{%
    В колебательном контура сила тока изменяется
    по закону $\eli=0{,}30\cos(18t)$ (в СИ).
    Индуктивность катушки при этом равна $80\,\text{мГн}$.
    Определите:
    \begin{itemize}
        \item период колебаний,
        \item ёмкость конденсатора,
        \item максимальный заряд конденсатора.
    \end{itemize}
}
\answer{%
    \begin{align*}
    \omega &= 18\units{c}^-1, \\
    T &= \frac{2\pi}\omega \approx 349{,}1\,\text{мc}, \\
    C &= \frac 1{\omega^2 L} \approx 38{,}6\,\text{мФ}, \\
    q &= \frac{\eli_{\max}}\omega  \approx 16{,}7\,\text{мКл}.
    \end{align*}
}
\solutionspace{80pt}

\tasknumber{4}%
\task{%
    Электрический колебательный контур состоит
    из катушки индуктивностью $L$ и конденсатора ёмкостью $C$.
    последовательно катушке подключают ещё одну катушку индуктивностью $\frac13L$.
    Как изменится период сводобных колебаний в контуре?
}
\answer{%
    \begin{align*}
    T &= 2\pi\sqrt{LC} \\
    T' &= 2\pi\sqrt{L'C'}= T \sqrt{\frac{L'}L \cdot \frac{C'}C} = T \sqrt{ \frac43 \cdot 1 } \\
    &\frac{T'}T = \sqrt{ \frac43 \cdot 1 } \approx 1{,}778
    \end{align*}
}

\variantsplitter

\addpersonalvariant{Анна Кузьмичёва}

\tasknumber{1}%
\task{%
    Схематично изобразите колебательный контур.
    Запишите формулу для периода колебаний в колебательном контуре и ...
    \begin{itemize}
        \item подпишите все физические величины,
        \item укажите их единицы измерения,
        \item выразите из формулы периода частоту,
        \item выразите из формулы периода выразите емкость конденсатора.
    \end{itemize}
}
\answer{%
    \begin{align*}
    T &= 2\pi\sqrt{LC} \\
    \nu &= \frac 1{2\pi\sqrt{LC}}, \\
    \omega &= \frac 1{\sqrt{LC}}, \\
    L &= \frac 1C \sqr{\frac T{2\pi}}, \\
    C &= \frac 1L \sqr{\frac T{2\pi}}.
    \end{align*}
}
\solutionspace{80pt}

\tasknumber{2}%
\task{%
    Оказалось, что наибольший заряд конденсатора в колебательном контуре равен $60\,\text{мкКл}$,
    а максимальный ток — $240\,\text{мА}$.
    Определите частоту колебаний.
}
\answer{%
    $
        \eli_{\max} = q_{\max}\omega \implies \nu = \frac{\omega}{2\pi} = \frac{\eli_{\max}}{2\pi q} \approx 0{,}637\,\text{кГц}
    $
}
\solutionspace{80pt}

\tasknumber{3}%
\task{%
    В колебательном контура сила тока изменяется
    по закону $\eli=0{,}25\cos(15t)$ (в СИ).
    Индуктивность катушки при этом равна $50\,\text{мГн}$.
    Определите:
    \begin{itemize}
        \item период колебаний,
        \item ёмкость конденсатора,
        \item максимальный заряд конденсатора.
    \end{itemize}
}
\answer{%
    \begin{align*}
    \omega &= 15\units{c}^-1, \\
    T &= \frac{2\pi}\omega \approx 418{,}9\,\text{мc}, \\
    C &= \frac 1{\omega^2 L} \approx 88{,}9\,\text{мФ}, \\
    q &= \frac{\eli_{\max}}\omega  \approx 16{,}7\,\text{мКл}.
    \end{align*}
}
\solutionspace{80pt}

\tasknumber{4}%
\task{%
    Электрический колебательный контур состоит
    из катушки индуктивностью $L$ и конденсатора ёмкостью $C$.
    последовательно конденсатору подключают ещё один конденсатор емкостью $\frac13L$.
    Как изменится период сводобных колебаний в контуре?
}
\answer{%
    \begin{align*}
    T &= 2\pi\sqrt{LC} \\
    T' &= 2\pi\sqrt{L'C'}= T \sqrt{\frac{L'}L \cdot \frac{C'}C} = T \sqrt{ 1 \cdot \frac14 } \\
    &\frac{T'}T = \sqrt{ 1 \cdot \frac14 } \approx 0{,}062
    \end{align*}
}

\variantsplitter

\addpersonalvariant{Алёна Куприянова}

\tasknumber{1}%
\task{%
    Схематично изобразите колебательный контур.
    Запишите формулу для периода колебаний в колебательном контуре и ...
    \begin{itemize}
        \item подпишите все физические величины,
        \item укажите их единицы измерения,
        \item выразите из формулы периода частоту,
        \item выразите из формулы периода индуктивность катушки..
    \end{itemize}
}
\answer{%
    \begin{align*}
    T &= 2\pi\sqrt{LC} \\
    \nu &= \frac 1{2\pi\sqrt{LC}}, \\
    \omega &= \frac 1{\sqrt{LC}}, \\
    L &= \frac 1C \sqr{\frac T{2\pi}}, \\
    C &= \frac 1L \sqr{\frac T{2\pi}}.
    \end{align*}
}
\solutionspace{80pt}

\tasknumber{2}%
\task{%
    Оказалось, что наибольший заряд конденсатора в колебательном контуре равен $80\,\text{мкКл}$,
    а максимальный ток — $180\,\text{мА}$.
    Определите частоту колебаний.
}
\answer{%
    $
        \eli_{\max} = q_{\max}\omega \implies \nu = \frac{\omega}{2\pi} = \frac{\eli_{\max}}{2\pi q} \approx 0{,}358\,\text{кГц}
    $
}
\solutionspace{80pt}

\tasknumber{3}%
\task{%
    В колебательном контура сила тока изменяется
    по закону $\eli=0{,}30\cos(12t)$ (в СИ).
    Индуктивность катушки при этом равна $50\,\text{мГн}$.
    Определите:
    \begin{itemize}
        \item период колебаний,
        \item ёмкость конденсатора,
        \item максимальный заряд конденсатора.
    \end{itemize}
}
\answer{%
    \begin{align*}
    \omega &= 12\units{c}^-1, \\
    T &= \frac{2\pi}\omega \approx 523{,}6\,\text{мc}, \\
    C &= \frac 1{\omega^2 L} \approx 138{,}9\,\text{мФ}, \\
    q &= \frac{\eli_{\max}}\omega  \approx 25{,}0\,\text{мКл}.
    \end{align*}
}
\solutionspace{80pt}

\tasknumber{4}%
\task{%
    Электрический колебательный контур состоит
    из катушки индуктивностью $L$ и конденсатора ёмкостью $C$.
    параллельно катушке подключают ещё одну катушку индуктивностью $2L$.
    Как изменится период сводобных колебаний в контуре?
}
\answer{%
    \begin{align*}
    T &= 2\pi\sqrt{LC} \\
    T' &= 2\pi\sqrt{L'C'}= T \sqrt{\frac{L'}L \cdot \frac{C'}C} = T \sqrt{ \frac23 \cdot 1 } \\
    &\frac{T'}T = \sqrt{ \frac23 \cdot 1 } \approx 0{,}444
    \end{align*}
}

\variantsplitter

\addpersonalvariant{Ярослав Лавровский}

\tasknumber{1}%
\task{%
    Схематично изобразите колебательный контур.
    Запишите формулу для периода колебаний в колебательном контуре и ...
    \begin{itemize}
        \item подпишите все физические величины,
        \item укажите их единицы измерения,
        \item выразите из формулы периода циклическую частоту,
        \item выразите из формулы периода выразите емкость конденсатора.
    \end{itemize}
}
\answer{%
    \begin{align*}
    T &= 2\pi\sqrt{LC} \\
    \nu &= \frac 1{2\pi\sqrt{LC}}, \\
    \omega &= \frac 1{\sqrt{LC}}, \\
    L &= \frac 1C \sqr{\frac T{2\pi}}, \\
    C &= \frac 1L \sqr{\frac T{2\pi}}.
    \end{align*}
}
\solutionspace{80pt}

\tasknumber{2}%
\task{%
    Оказалось, что наибольший заряд конденсатора в колебательном контуре равен $40\,\text{мкКл}$,
    а максимальный ток — $270\,\text{мА}$.
    Определите частоту колебаний.
}
\answer{%
    $
        \eli_{\max} = q_{\max}\omega \implies \nu = \frac{\omega}{2\pi} = \frac{\eli_{\max}}{2\pi q} \approx 1{,}074\,\text{кГц}
    $
}
\solutionspace{80pt}

\tasknumber{3}%
\task{%
    В колебательном контура сила тока изменяется
    по закону $\eli=0{,}25\cos(18t)$ (в СИ).
    Индуктивность катушки при этом равна $80\,\text{мГн}$.
    Определите:
    \begin{itemize}
        \item период колебаний,
        \item ёмкость конденсатора,
        \item максимальный заряд конденсатора.
    \end{itemize}
}
\answer{%
    \begin{align*}
    \omega &= 18\units{c}^-1, \\
    T &= \frac{2\pi}\omega \approx 349{,}1\,\text{мc}, \\
    C &= \frac 1{\omega^2 L} \approx 38{,}6\,\text{мФ}, \\
    q &= \frac{\eli_{\max}}\omega  \approx 13{,}9\,\text{мКл}.
    \end{align*}
}
\solutionspace{80pt}

\tasknumber{4}%
\task{%
    Электрический колебательный контур состоит
    из катушки индуктивностью $L$ и конденсатора ёмкостью $C$.
    параллельно катушке подключают ещё одну катушку индуктивностью $3L$.
    Как изменится период сводобных колебаний в контуре?
}
\answer{%
    \begin{align*}
    T &= 2\pi\sqrt{LC} \\
    T' &= 2\pi\sqrt{L'C'}= T \sqrt{\frac{L'}L \cdot \frac{C'}C} = T \sqrt{ \frac34 \cdot 1 } \\
    &\frac{T'}T = \sqrt{ \frac34 \cdot 1 } \approx 0{,}562
    \end{align*}
}

\variantsplitter

\addpersonalvariant{Анастасия Ламанова}

\tasknumber{1}%
\task{%
    Схематично изобразите колебательный контур.
    Запишите формулу для периода колебаний в колебательном контуре и ...
    \begin{itemize}
        \item подпишите все физические величины,
        \item укажите их единицы измерения,
        \item выразите из формулы периода частоту,
        \item выразите из формулы периода индуктивность катушки..
    \end{itemize}
}
\answer{%
    \begin{align*}
    T &= 2\pi\sqrt{LC} \\
    \nu &= \frac 1{2\pi\sqrt{LC}}, \\
    \omega &= \frac 1{\sqrt{LC}}, \\
    L &= \frac 1C \sqr{\frac T{2\pi}}, \\
    C &= \frac 1L \sqr{\frac T{2\pi}}.
    \end{align*}
}
\solutionspace{80pt}

\tasknumber{2}%
\task{%
    Оказалось, что наибольший заряд конденсатора в колебательном контуре равен $40\,\text{мкКл}$,
    а максимальный ток — $120\,\text{мА}$.
    Определите частоту колебаний.
}
\answer{%
    $
        \eli_{\max} = q_{\max}\omega \implies \nu = \frac{\omega}{2\pi} = \frac{\eli_{\max}}{2\pi q} \approx 0{,}477\,\text{кГц}
    $
}
\solutionspace{80pt}

\tasknumber{3}%
\task{%
    В колебательном контура сила тока изменяется
    по закону $\eli=0{,}30\cos(12t)$ (в СИ).
    Индуктивность катушки при этом равна $70\,\text{мГн}$.
    Определите:
    \begin{itemize}
        \item период колебаний,
        \item ёмкость конденсатора,
        \item максимальный заряд конденсатора.
    \end{itemize}
}
\answer{%
    \begin{align*}
    \omega &= 12\units{c}^-1, \\
    T &= \frac{2\pi}\omega \approx 523{,}6\,\text{мc}, \\
    C &= \frac 1{\omega^2 L} \approx 99{,}2\,\text{мФ}, \\
    q &= \frac{\eli_{\max}}\omega  \approx 25{,}0\,\text{мКл}.
    \end{align*}
}
\solutionspace{80pt}

\tasknumber{4}%
\task{%
    Электрический колебательный контур состоит
    из катушки индуктивностью $L$ и конденсатора ёмкостью $C$.
    последовательно конденсатору подключают ещё один конденсатор емкостью $\frac13L$.
    Как изменится период сводобных колебаний в контуре?
}
\answer{%
    \begin{align*}
    T &= 2\pi\sqrt{LC} \\
    T' &= 2\pi\sqrt{L'C'}= T \sqrt{\frac{L'}L \cdot \frac{C'}C} = T \sqrt{ 1 \cdot \frac14 } \\
    &\frac{T'}T = \sqrt{ 1 \cdot \frac14 } \approx 0{,}062
    \end{align*}
}

\variantsplitter

\addpersonalvariant{Виктория Легонькова}

\tasknumber{1}%
\task{%
    Схематично изобразите колебательный контур.
    Запишите формулу для периода колебаний в колебательном контуре и ...
    \begin{itemize}
        \item подпишите все физические величины,
        \item укажите их единицы измерения,
        \item выразите из формулы периода циклическую частоту,
        \item выразите из формулы периода индуктивность катушки..
    \end{itemize}
}
\answer{%
    \begin{align*}
    T &= 2\pi\sqrt{LC} \\
    \nu &= \frac 1{2\pi\sqrt{LC}}, \\
    \omega &= \frac 1{\sqrt{LC}}, \\
    L &= \frac 1C \sqr{\frac T{2\pi}}, \\
    C &= \frac 1L \sqr{\frac T{2\pi}}.
    \end{align*}
}
\solutionspace{80pt}

\tasknumber{2}%
\task{%
    Оказалось, что наибольший заряд конденсатора в колебательном контуре равен $80\,\text{мкКл}$,
    а максимальный ток — $180\,\text{мА}$.
    Определите частоту колебаний.
}
\answer{%
    $
        \eli_{\max} = q_{\max}\omega \implies \nu = \frac{\omega}{2\pi} = \frac{\eli_{\max}}{2\pi q} \approx 0{,}358\,\text{кГц}
    $
}
\solutionspace{80pt}

\tasknumber{3}%
\task{%
    В колебательном контура сила тока изменяется
    по закону $\eli=0{,}25\cos(15t)$ (в СИ).
    Индуктивность катушки при этом равна $70\,\text{мГн}$.
    Определите:
    \begin{itemize}
        \item период колебаний,
        \item ёмкость конденсатора,
        \item максимальный заряд конденсатора.
    \end{itemize}
}
\answer{%
    \begin{align*}
    \omega &= 15\units{c}^-1, \\
    T &= \frac{2\pi}\omega \approx 418{,}9\,\text{мc}, \\
    C &= \frac 1{\omega^2 L} \approx 63{,}5\,\text{мФ}, \\
    q &= \frac{\eli_{\max}}\omega  \approx 16{,}7\,\text{мКл}.
    \end{align*}
}
\solutionspace{80pt}

\tasknumber{4}%
\task{%
    Электрический колебательный контур состоит
    из катушки индуктивностью $L$ и конденсатора ёмкостью $C$.
    параллельно катушке подключают ещё одну катушку индуктивностью $\frac13L$.
    Как изменится период сводобных колебаний в контуре?
}
\answer{%
    \begin{align*}
    T &= 2\pi\sqrt{LC} \\
    T' &= 2\pi\sqrt{L'C'}= T \sqrt{\frac{L'}L \cdot \frac{C'}C} = T \sqrt{ \frac14 \cdot 1 } \\
    &\frac{T'}T = \sqrt{ \frac14 \cdot 1 } \approx 0{,}062
    \end{align*}
}

\variantsplitter

\addpersonalvariant{Семён Мартынов}

\tasknumber{1}%
\task{%
    Схематично изобразите колебательный контур.
    Запишите формулу для периода колебаний в колебательном контуре и ...
    \begin{itemize}
        \item подпишите все физические величины,
        \item укажите их единицы измерения,
        \item выразите из формулы периода частоту,
        \item выразите из формулы периода индуктивность катушки..
    \end{itemize}
}
\answer{%
    \begin{align*}
    T &= 2\pi\sqrt{LC} \\
    \nu &= \frac 1{2\pi\sqrt{LC}}, \\
    \omega &= \frac 1{\sqrt{LC}}, \\
    L &= \frac 1C \sqr{\frac T{2\pi}}, \\
    C &= \frac 1L \sqr{\frac T{2\pi}}.
    \end{align*}
}
\solutionspace{80pt}

\tasknumber{2}%
\task{%
    Оказалось, что наибольший заряд конденсатора в колебательном контуре равен $60\,\text{мкКл}$,
    а максимальный ток — $180\,\text{мА}$.
    Определите частоту колебаний.
}
\answer{%
    $
        \eli_{\max} = q_{\max}\omega \implies \nu = \frac{\omega}{2\pi} = \frac{\eli_{\max}}{2\pi q} \approx 0{,}477\,\text{кГц}
    $
}
\solutionspace{80pt}

\tasknumber{3}%
\task{%
    В колебательном контура сила тока изменяется
    по закону $\eli=0{,}05\cos(15t)$ (в СИ).
    Индуктивность катушки при этом равна $70\,\text{мГн}$.
    Определите:
    \begin{itemize}
        \item период колебаний,
        \item ёмкость конденсатора,
        \item максимальный заряд конденсатора.
    \end{itemize}
}
\answer{%
    \begin{align*}
    \omega &= 15\units{c}^-1, \\
    T &= \frac{2\pi}\omega \approx 418{,}9\,\text{мc}, \\
    C &= \frac 1{\omega^2 L} \approx 63{,}5\,\text{мФ}, \\
    q &= \frac{\eli_{\max}}\omega  \approx 3{,}3\,\text{мКл}.
    \end{align*}
}
\solutionspace{80pt}

\tasknumber{4}%
\task{%
    Электрический колебательный контур состоит
    из катушки индуктивностью $L$ и конденсатора ёмкостью $C$.
    параллельно конденсатору подключают ещё один конденсатор емкостью $\frac12L$.
    Как изменится период сводобных колебаний в контуре?
}
\answer{%
    \begin{align*}
    T &= 2\pi\sqrt{LC} \\
    T' &= 2\pi\sqrt{L'C'}= T \sqrt{\frac{L'}L \cdot \frac{C'}C} = T \sqrt{ 1 \cdot \frac32 } \\
    &\frac{T'}T = \sqrt{ 1 \cdot \frac32 } \approx 2{,}250
    \end{align*}
}

\variantsplitter

\addpersonalvariant{Варвара Минаева}

\tasknumber{1}%
\task{%
    Схематично изобразите колебательный контур.
    Запишите формулу для периода колебаний в колебательном контуре и ...
    \begin{itemize}
        \item подпишите все физические величины,
        \item укажите их единицы измерения,
        \item выразите из формулы периода циклическую частоту,
        \item выразите из формулы периода индуктивность катушки..
    \end{itemize}
}
\answer{%
    \begin{align*}
    T &= 2\pi\sqrt{LC} \\
    \nu &= \frac 1{2\pi\sqrt{LC}}, \\
    \omega &= \frac 1{\sqrt{LC}}, \\
    L &= \frac 1C \sqr{\frac T{2\pi}}, \\
    C &= \frac 1L \sqr{\frac T{2\pi}}.
    \end{align*}
}
\solutionspace{80pt}

\tasknumber{2}%
\task{%
    Оказалось, что наибольший заряд конденсатора в колебательном контуре равен $60\,\text{мкКл}$,
    а максимальный ток — $120\,\text{мА}$.
    Определите частоту колебаний.
}
\answer{%
    $
        \eli_{\max} = q_{\max}\omega \implies \nu = \frac{\omega}{2\pi} = \frac{\eli_{\max}}{2\pi q} \approx 0{,}318\,\text{кГц}
    $
}
\solutionspace{80pt}

\tasknumber{3}%
\task{%
    В колебательном контура сила тока изменяется
    по закону $\eli=0{,}25\sin(12t)$ (в СИ).
    Индуктивность катушки при этом равна $50\,\text{мГн}$.
    Определите:
    \begin{itemize}
        \item период колебаний,
        \item ёмкость конденсатора,
        \item максимальный заряд конденсатора.
    \end{itemize}
}
\answer{%
    \begin{align*}
    \omega &= 12\units{c}^-1, \\
    T &= \frac{2\pi}\omega \approx 523{,}6\,\text{мc}, \\
    C &= \frac 1{\omega^2 L} \approx 138{,}9\,\text{мФ}, \\
    q &= \frac{\eli_{\max}}\omega  \approx 20{,}8\,\text{мКл}.
    \end{align*}
}
\solutionspace{80pt}

\tasknumber{4}%
\task{%
    Электрический колебательный контур состоит
    из катушки индуктивностью $L$ и конденсатора ёмкостью $C$.
    последовательно конденсатору подключают ещё один конденсатор емкостью $3L$.
    Как изменится период сводобных колебаний в контуре?
}
\answer{%
    \begin{align*}
    T &= 2\pi\sqrt{LC} \\
    T' &= 2\pi\sqrt{L'C'}= T \sqrt{\frac{L'}L \cdot \frac{C'}C} = T \sqrt{ 1 \cdot \frac34 } \\
    &\frac{T'}T = \sqrt{ 1 \cdot \frac34 } \approx 0{,}562
    \end{align*}
}

\variantsplitter

\addpersonalvariant{Леонид Никитин}

\tasknumber{1}%
\task{%
    Схематично изобразите колебательный контур.
    Запишите формулу для периода колебаний в колебательном контуре и ...
    \begin{itemize}
        \item подпишите все физические величины,
        \item укажите их единицы измерения,
        \item выразите из формулы периода циклическую частоту,
        \item выразите из формулы периода выразите емкость конденсатора.
    \end{itemize}
}
\answer{%
    \begin{align*}
    T &= 2\pi\sqrt{LC} \\
    \nu &= \frac 1{2\pi\sqrt{LC}}, \\
    \omega &= \frac 1{\sqrt{LC}}, \\
    L &= \frac 1C \sqr{\frac T{2\pi}}, \\
    C &= \frac 1L \sqr{\frac T{2\pi}}.
    \end{align*}
}
\solutionspace{80pt}

\tasknumber{2}%
\task{%
    Оказалось, что наибольший заряд конденсатора в колебательном контуре равен $60\,\text{мкКл}$,
    а максимальный ток — $270\,\text{мА}$.
    Определите частоту колебаний.
}
\answer{%
    $
        \eli_{\max} = q_{\max}\omega \implies \nu = \frac{\omega}{2\pi} = \frac{\eli_{\max}}{2\pi q} \approx 0{,}716\,\text{кГц}
    $
}
\solutionspace{80pt}

\tasknumber{3}%
\task{%
    В колебательном контура сила тока изменяется
    по закону $\eli=0{,}25\sin(18t)$ (в СИ).
    Индуктивность катушки при этом равна $60\,\text{мГн}$.
    Определите:
    \begin{itemize}
        \item период колебаний,
        \item ёмкость конденсатора,
        \item максимальный заряд конденсатора.
    \end{itemize}
}
\answer{%
    \begin{align*}
    \omega &= 18\units{c}^-1, \\
    T &= \frac{2\pi}\omega \approx 349{,}1\,\text{мc}, \\
    C &= \frac 1{\omega^2 L} \approx 51{,}4\,\text{мФ}, \\
    q &= \frac{\eli_{\max}}\omega  \approx 13{,}9\,\text{мКл}.
    \end{align*}
}
\solutionspace{80pt}

\tasknumber{4}%
\task{%
    Электрический колебательный контур состоит
    из катушки индуктивностью $L$ и конденсатора ёмкостью $C$.
    параллельно конденсатору подключают ещё один конденсатор емкостью $2L$.
    Как изменится период сводобных колебаний в контуре?
}
\answer{%
    \begin{align*}
    T &= 2\pi\sqrt{LC} \\
    T' &= 2\pi\sqrt{L'C'}= T \sqrt{\frac{L'}L \cdot \frac{C'}C} = T \sqrt{ 1 \cdot 3 } \\
    &\frac{T'}T = \sqrt{ 1 \cdot 3 } \approx 9{,}000
    \end{align*}
}

\variantsplitter

\addpersonalvariant{Тимофей Полетаев}

\tasknumber{1}%
\task{%
    Схематично изобразите колебательный контур.
    Запишите формулу для периода колебаний в колебательном контуре и ...
    \begin{itemize}
        \item подпишите все физические величины,
        \item укажите их единицы измерения,
        \item выразите из формулы периода частоту,
        \item выразите из формулы периода индуктивность катушки..
    \end{itemize}
}
\answer{%
    \begin{align*}
    T &= 2\pi\sqrt{LC} \\
    \nu &= \frac 1{2\pi\sqrt{LC}}, \\
    \omega &= \frac 1{\sqrt{LC}}, \\
    L &= \frac 1C \sqr{\frac T{2\pi}}, \\
    C &= \frac 1L \sqr{\frac T{2\pi}}.
    \end{align*}
}
\solutionspace{80pt}

\tasknumber{2}%
\task{%
    Оказалось, что наибольший заряд конденсатора в колебательном контуре равен $40\,\text{мкКл}$,
    а максимальный ток — $120\,\text{мА}$.
    Определите частоту колебаний.
}
\answer{%
    $
        \eli_{\max} = q_{\max}\omega \implies \nu = \frac{\omega}{2\pi} = \frac{\eli_{\max}}{2\pi q} \approx 0{,}477\,\text{кГц}
    $
}
\solutionspace{80pt}

\tasknumber{3}%
\task{%
    В колебательном контура сила тока изменяется
    по закону $\eli=0{,}25\cos(15t)$ (в СИ).
    Индуктивность катушки при этом равна $80\,\text{мГн}$.
    Определите:
    \begin{itemize}
        \item период колебаний,
        \item ёмкость конденсатора,
        \item максимальный заряд конденсатора.
    \end{itemize}
}
\answer{%
    \begin{align*}
    \omega &= 15\units{c}^-1, \\
    T &= \frac{2\pi}\omega \approx 418{,}9\,\text{мc}, \\
    C &= \frac 1{\omega^2 L} \approx 55{,}6\,\text{мФ}, \\
    q &= \frac{\eli_{\max}}\omega  \approx 16{,}7\,\text{мКл}.
    \end{align*}
}
\solutionspace{80pt}

\tasknumber{4}%
\task{%
    Электрический колебательный контур состоит
    из катушки индуктивностью $L$ и конденсатора ёмкостью $C$.
    последовательно катушке подключают ещё одну катушку индуктивностью $\frac13L$.
    Как изменится период сводобных колебаний в контуре?
}
\answer{%
    \begin{align*}
    T &= 2\pi\sqrt{LC} \\
    T' &= 2\pi\sqrt{L'C'}= T \sqrt{\frac{L'}L \cdot \frac{C'}C} = T \sqrt{ \frac43 \cdot 1 } \\
    &\frac{T'}T = \sqrt{ \frac43 \cdot 1 } \approx 1{,}778
    \end{align*}
}

\variantsplitter

\addpersonalvariant{Андрей Рожков}

\tasknumber{1}%
\task{%
    Схематично изобразите колебательный контур.
    Запишите формулу для периода колебаний в колебательном контуре и ...
    \begin{itemize}
        \item подпишите все физические величины,
        \item укажите их единицы измерения,
        \item выразите из формулы периода циклическую частоту,
        \item выразите из формулы периода индуктивность катушки..
    \end{itemize}
}
\answer{%
    \begin{align*}
    T &= 2\pi\sqrt{LC} \\
    \nu &= \frac 1{2\pi\sqrt{LC}}, \\
    \omega &= \frac 1{\sqrt{LC}}, \\
    L &= \frac 1C \sqr{\frac T{2\pi}}, \\
    C &= \frac 1L \sqr{\frac T{2\pi}}.
    \end{align*}
}
\solutionspace{80pt}

\tasknumber{2}%
\task{%
    Оказалось, что наибольший заряд конденсатора в колебательном контуре равен $80\,\text{мкКл}$,
    а максимальный ток — $120\,\text{мА}$.
    Определите частоту колебаний.
}
\answer{%
    $
        \eli_{\max} = q_{\max}\omega \implies \nu = \frac{\omega}{2\pi} = \frac{\eli_{\max}}{2\pi q} \approx 0{,}239\,\text{кГц}
    $
}
\solutionspace{80pt}

\tasknumber{3}%
\task{%
    В колебательном контура сила тока изменяется
    по закону $\eli=0{,}05\sin(12t)$ (в СИ).
    Индуктивность катушки при этом равна $50\,\text{мГн}$.
    Определите:
    \begin{itemize}
        \item период колебаний,
        \item ёмкость конденсатора,
        \item максимальный заряд конденсатора.
    \end{itemize}
}
\answer{%
    \begin{align*}
    \omega &= 12\units{c}^-1, \\
    T &= \frac{2\pi}\omega \approx 523{,}6\,\text{мc}, \\
    C &= \frac 1{\omega^2 L} \approx 138{,}9\,\text{мФ}, \\
    q &= \frac{\eli_{\max}}\omega  \approx 4{,}2\,\text{мКл}.
    \end{align*}
}
\solutionspace{80pt}

\tasknumber{4}%
\task{%
    Электрический колебательный контур состоит
    из катушки индуктивностью $L$ и конденсатора ёмкостью $C$.
    параллельно конденсатору подключают ещё один конденсатор емкостью $\frac12L$.
    Как изменится период сводобных колебаний в контуре?
}
\answer{%
    \begin{align*}
    T &= 2\pi\sqrt{LC} \\
    T' &= 2\pi\sqrt{L'C'}= T \sqrt{\frac{L'}L \cdot \frac{C'}C} = T \sqrt{ 1 \cdot \frac32 } \\
    &\frac{T'}T = \sqrt{ 1 \cdot \frac32 } \approx 2{,}250
    \end{align*}
}

\variantsplitter

\addpersonalvariant{Рената Таржиманова}

\tasknumber{1}%
\task{%
    Схематично изобразите колебательный контур.
    Запишите формулу для периода колебаний в колебательном контуре и ...
    \begin{itemize}
        \item подпишите все физические величины,
        \item укажите их единицы измерения,
        \item выразите из формулы периода частоту,
        \item выразите из формулы периода индуктивность катушки..
    \end{itemize}
}
\answer{%
    \begin{align*}
    T &= 2\pi\sqrt{LC} \\
    \nu &= \frac 1{2\pi\sqrt{LC}}, \\
    \omega &= \frac 1{\sqrt{LC}}, \\
    L &= \frac 1C \sqr{\frac T{2\pi}}, \\
    C &= \frac 1L \sqr{\frac T{2\pi}}.
    \end{align*}
}
\solutionspace{80pt}

\tasknumber{2}%
\task{%
    Оказалось, что наибольший заряд конденсатора в колебательном контуре равен $80\,\text{мкКл}$,
    а максимальный ток — $150\,\text{мА}$.
    Определите частоту колебаний.
}
\answer{%
    $
        \eli_{\max} = q_{\max}\omega \implies \nu = \frac{\omega}{2\pi} = \frac{\eli_{\max}}{2\pi q} \approx 0{,}298\,\text{кГц}
    $
}
\solutionspace{80pt}

\tasknumber{3}%
\task{%
    В колебательном контура сила тока изменяется
    по закону $\eli=0{,}25\cos(18t)$ (в СИ).
    Индуктивность катушки при этом равна $70\,\text{мГн}$.
    Определите:
    \begin{itemize}
        \item период колебаний,
        \item ёмкость конденсатора,
        \item максимальный заряд конденсатора.
    \end{itemize}
}
\answer{%
    \begin{align*}
    \omega &= 18\units{c}^-1, \\
    T &= \frac{2\pi}\omega \approx 349{,}1\,\text{мc}, \\
    C &= \frac 1{\omega^2 L} \approx 44{,}1\,\text{мФ}, \\
    q &= \frac{\eli_{\max}}\omega  \approx 13{,}9\,\text{мКл}.
    \end{align*}
}
\solutionspace{80pt}

\tasknumber{4}%
\task{%
    Электрический колебательный контур состоит
    из катушки индуктивностью $L$ и конденсатора ёмкостью $C$.
    параллельно катушке подключают ещё одну катушку индуктивностью $3L$.
    Как изменится период сводобных колебаний в контуре?
}
\answer{%
    \begin{align*}
    T &= 2\pi\sqrt{LC} \\
    T' &= 2\pi\sqrt{L'C'}= T \sqrt{\frac{L'}L \cdot \frac{C'}C} = T \sqrt{ \frac34 \cdot 1 } \\
    &\frac{T'}T = \sqrt{ \frac34 \cdot 1 } \approx 0{,}562
    \end{align*}
}

\variantsplitter

\addpersonalvariant{Андрей Щербаков}

\tasknumber{1}%
\task{%
    Схематично изобразите колебательный контур.
    Запишите формулу для периода колебаний в колебательном контуре и ...
    \begin{itemize}
        \item подпишите все физические величины,
        \item укажите их единицы измерения,
        \item выразите из формулы периода циклическую частоту,
        \item выразите из формулы периода выразите емкость конденсатора.
    \end{itemize}
}
\answer{%
    \begin{align*}
    T &= 2\pi\sqrt{LC} \\
    \nu &= \frac 1{2\pi\sqrt{LC}}, \\
    \omega &= \frac 1{\sqrt{LC}}, \\
    L &= \frac 1C \sqr{\frac T{2\pi}}, \\
    C &= \frac 1L \sqr{\frac T{2\pi}}.
    \end{align*}
}
\solutionspace{80pt}

\tasknumber{2}%
\task{%
    Оказалось, что наибольший заряд конденсатора в колебательном контуре равен $60\,\text{мкКл}$,
    а максимальный ток — $150\,\text{мА}$.
    Определите частоту колебаний.
}
\answer{%
    $
        \eli_{\max} = q_{\max}\omega \implies \nu = \frac{\omega}{2\pi} = \frac{\eli_{\max}}{2\pi q} \approx 0{,}398\,\text{кГц}
    $
}
\solutionspace{80pt}

\tasknumber{3}%
\task{%
    В колебательном контура сила тока изменяется
    по закону $\eli=0{,}25\sin(18t)$ (в СИ).
    Индуктивность катушки при этом равна $70\,\text{мГн}$.
    Определите:
    \begin{itemize}
        \item период колебаний,
        \item ёмкость конденсатора,
        \item максимальный заряд конденсатора.
    \end{itemize}
}
\answer{%
    \begin{align*}
    \omega &= 18\units{c}^-1, \\
    T &= \frac{2\pi}\omega \approx 349{,}1\,\text{мc}, \\
    C &= \frac 1{\omega^2 L} \approx 44{,}1\,\text{мФ}, \\
    q &= \frac{\eli_{\max}}\omega  \approx 13{,}9\,\text{мКл}.
    \end{align*}
}
\solutionspace{80pt}

\tasknumber{4}%
\task{%
    Электрический колебательный контур состоит
    из катушки индуктивностью $L$ и конденсатора ёмкостью $C$.
    последовательно катушке подключают ещё одну катушку индуктивностью $2L$.
    Как изменится период сводобных колебаний в контуре?
}
\answer{%
    \begin{align*}
    T &= 2\pi\sqrt{LC} \\
    T' &= 2\pi\sqrt{L'C'}= T \sqrt{\frac{L'}L \cdot \frac{C'}C} = T \sqrt{ 3 \cdot 1 } \\
    &\frac{T'}T = \sqrt{ 3 \cdot 1 } \approx 9{,}000
    \end{align*}
}

\variantsplitter

\addpersonalvariant{Михаил Ярошевский}

\tasknumber{1}%
\task{%
    Схематично изобразите колебательный контур.
    Запишите формулу для периода колебаний в колебательном контуре и ...
    \begin{itemize}
        \item подпишите все физические величины,
        \item укажите их единицы измерения,
        \item выразите из формулы периода циклическую частоту,
        \item выразите из формулы периода выразите емкость конденсатора.
    \end{itemize}
}
\answer{%
    \begin{align*}
    T &= 2\pi\sqrt{LC} \\
    \nu &= \frac 1{2\pi\sqrt{LC}}, \\
    \omega &= \frac 1{\sqrt{LC}}, \\
    L &= \frac 1C \sqr{\frac T{2\pi}}, \\
    C &= \frac 1L \sqr{\frac T{2\pi}}.
    \end{align*}
}
\solutionspace{80pt}

\tasknumber{2}%
\task{%
    Оказалось, что наибольший заряд конденсатора в колебательном контуре равен $80\,\text{мкКл}$,
    а максимальный ток — $180\,\text{мА}$.
    Определите частоту колебаний.
}
\answer{%
    $
        \eli_{\max} = q_{\max}\omega \implies \nu = \frac{\omega}{2\pi} = \frac{\eli_{\max}}{2\pi q} \approx 0{,}358\,\text{кГц}
    $
}
\solutionspace{80pt}

\tasknumber{3}%
\task{%
    В колебательном контура сила тока изменяется
    по закону $\eli=0{,}05\cos(18t)$ (в СИ).
    Индуктивность катушки при этом равна $80\,\text{мГн}$.
    Определите:
    \begin{itemize}
        \item период колебаний,
        \item ёмкость конденсатора,
        \item максимальный заряд конденсатора.
    \end{itemize}
}
\answer{%
    \begin{align*}
    \omega &= 18\units{c}^-1, \\
    T &= \frac{2\pi}\omega \approx 349{,}1\,\text{мc}, \\
    C &= \frac 1{\omega^2 L} \approx 38{,}6\,\text{мФ}, \\
    q &= \frac{\eli_{\max}}\omega  \approx 2{,}8\,\text{мКл}.
    \end{align*}
}
\solutionspace{80pt}

\tasknumber{4}%
\task{%
    Электрический колебательный контур состоит
    из катушки индуктивностью $L$ и конденсатора ёмкостью $C$.
    параллельно конденсатору подключают ещё один конденсатор емкостью $\frac12L$.
    Как изменится период сводобных колебаний в контуре?
}
\answer{%
    \begin{align*}
    T &= 2\pi\sqrt{LC} \\
    T' &= 2\pi\sqrt{L'C'}= T \sqrt{\frac{L'}L \cdot \frac{C'}C} = T \sqrt{ 1 \cdot \frac32 } \\
    &\frac{T'}T = \sqrt{ 1 \cdot \frac32 } \approx 2{,}250
    \end{align*}
}

\variantsplitter

\addpersonalvariant{Алексей Алимпиев}

\tasknumber{1}%
\task{%
    Схематично изобразите колебательный контур.
    Запишите формулу для периода колебаний в колебательном контуре и ...
    \begin{itemize}
        \item подпишите все физические величины,
        \item укажите их единицы измерения,
        \item выразите из формулы периода циклическую частоту,
        \item выразите из формулы периода индуктивность катушки..
    \end{itemize}
}
\answer{%
    \begin{align*}
    T &= 2\pi\sqrt{LC} \\
    \nu &= \frac 1{2\pi\sqrt{LC}}, \\
    \omega &= \frac 1{\sqrt{LC}}, \\
    L &= \frac 1C \sqr{\frac T{2\pi}}, \\
    C &= \frac 1L \sqr{\frac T{2\pi}}.
    \end{align*}
}
\solutionspace{80pt}

\tasknumber{2}%
\task{%
    Оказалось, что наибольший заряд конденсатора в колебательном контуре равен $60\,\text{мкКл}$,
    а максимальный ток — $270\,\text{мА}$.
    Определите частоту колебаний.
}
\answer{%
    $
        \eli_{\max} = q_{\max}\omega \implies \nu = \frac{\omega}{2\pi} = \frac{\eli_{\max}}{2\pi q} \approx 0{,}716\,\text{кГц}
    $
}
\solutionspace{80pt}

\tasknumber{3}%
\task{%
    В колебательном контура сила тока изменяется
    по закону $\eli=0{,}25\cos(18t)$ (в СИ).
    Индуктивность катушки при этом равна $60\,\text{мГн}$.
    Определите:
    \begin{itemize}
        \item период колебаний,
        \item ёмкость конденсатора,
        \item максимальный заряд конденсатора.
    \end{itemize}
}
\answer{%
    \begin{align*}
    \omega &= 18\units{c}^-1, \\
    T &= \frac{2\pi}\omega \approx 349{,}1\,\text{мc}, \\
    C &= \frac 1{\omega^2 L} \approx 51{,}4\,\text{мФ}, \\
    q &= \frac{\eli_{\max}}\omega  \approx 13{,}9\,\text{мКл}.
    \end{align*}
}
\solutionspace{80pt}

\tasknumber{4}%
\task{%
    Электрический колебательный контур состоит
    из катушки индуктивностью $L$ и конденсатора ёмкостью $C$.
    последовательно конденсатору подключают ещё один конденсатор емкостью $\frac12L$.
    Как изменится период сводобных колебаний в контуре?
}
\answer{%
    \begin{align*}
    T &= 2\pi\sqrt{LC} \\
    T' &= 2\pi\sqrt{L'C'}= T \sqrt{\frac{L'}L \cdot \frac{C'}C} = T \sqrt{ 1 \cdot \frac13 } \\
    &\frac{T'}T = \sqrt{ 1 \cdot \frac13 } \approx 0{,}111
    \end{align*}
}

\variantsplitter

\addpersonalvariant{Евгений Васин}

\tasknumber{1}%
\task{%
    Схематично изобразите колебательный контур.
    Запишите формулу для периода колебаний в колебательном контуре и ...
    \begin{itemize}
        \item подпишите все физические величины,
        \item укажите их единицы измерения,
        \item выразите из формулы периода циклическую частоту,
        \item выразите из формулы периода индуктивность катушки..
    \end{itemize}
}
\answer{%
    \begin{align*}
    T &= 2\pi\sqrt{LC} \\
    \nu &= \frac 1{2\pi\sqrt{LC}}, \\
    \omega &= \frac 1{\sqrt{LC}}, \\
    L &= \frac 1C \sqr{\frac T{2\pi}}, \\
    C &= \frac 1L \sqr{\frac T{2\pi}}.
    \end{align*}
}
\solutionspace{80pt}

\tasknumber{2}%
\task{%
    Оказалось, что наибольший заряд конденсатора в колебательном контуре равен $40\,\text{мкКл}$,
    а максимальный ток — $270\,\text{мА}$.
    Определите частоту колебаний.
}
\answer{%
    $
        \eli_{\max} = q_{\max}\omega \implies \nu = \frac{\omega}{2\pi} = \frac{\eli_{\max}}{2\pi q} \approx 1{,}074\,\text{кГц}
    $
}
\solutionspace{80pt}

\tasknumber{3}%
\task{%
    В колебательном контура сила тока изменяется
    по закону $\eli=0{,}05\sin(18t)$ (в СИ).
    Индуктивность катушки при этом равна $70\,\text{мГн}$.
    Определите:
    \begin{itemize}
        \item период колебаний,
        \item ёмкость конденсатора,
        \item максимальный заряд конденсатора.
    \end{itemize}
}
\answer{%
    \begin{align*}
    \omega &= 18\units{c}^-1, \\
    T &= \frac{2\pi}\omega \approx 349{,}1\,\text{мc}, \\
    C &= \frac 1{\omega^2 L} \approx 44{,}1\,\text{мФ}, \\
    q &= \frac{\eli_{\max}}\omega  \approx 2{,}8\,\text{мКл}.
    \end{align*}
}
\solutionspace{80pt}

\tasknumber{4}%
\task{%
    Электрический колебательный контур состоит
    из катушки индуктивностью $L$ и конденсатора ёмкостью $C$.
    параллельно катушке подключают ещё одну катушку индуктивностью $\frac13L$.
    Как изменится период сводобных колебаний в контуре?
}
\answer{%
    \begin{align*}
    T &= 2\pi\sqrt{LC} \\
    T' &= 2\pi\sqrt{L'C'}= T \sqrt{\frac{L'}L \cdot \frac{C'}C} = T \sqrt{ \frac14 \cdot 1 } \\
    &\frac{T'}T = \sqrt{ \frac14 \cdot 1 } \approx 0{,}062
    \end{align*}
}

\variantsplitter

\addpersonalvariant{Вячеслав Волохов}

\tasknumber{1}%
\task{%
    Схематично изобразите колебательный контур.
    Запишите формулу для периода колебаний в колебательном контуре и ...
    \begin{itemize}
        \item подпишите все физические величины,
        \item укажите их единицы измерения,
        \item выразите из формулы периода частоту,
        \item выразите из формулы периода индуктивность катушки..
    \end{itemize}
}
\answer{%
    \begin{align*}
    T &= 2\pi\sqrt{LC} \\
    \nu &= \frac 1{2\pi\sqrt{LC}}, \\
    \omega &= \frac 1{\sqrt{LC}}, \\
    L &= \frac 1C \sqr{\frac T{2\pi}}, \\
    C &= \frac 1L \sqr{\frac T{2\pi}}.
    \end{align*}
}
\solutionspace{80pt}

\tasknumber{2}%
\task{%
    Оказалось, что наибольший заряд конденсатора в колебательном контуре равен $40\,\text{мкКл}$,
    а максимальный ток — $270\,\text{мА}$.
    Определите частоту колебаний.
}
\answer{%
    $
        \eli_{\max} = q_{\max}\omega \implies \nu = \frac{\omega}{2\pi} = \frac{\eli_{\max}}{2\pi q} \approx 1{,}074\,\text{кГц}
    $
}
\solutionspace{80pt}

\tasknumber{3}%
\task{%
    В колебательном контура сила тока изменяется
    по закону $\eli=0{,}05\sin(18t)$ (в СИ).
    Индуктивность катушки при этом равна $80\,\text{мГн}$.
    Определите:
    \begin{itemize}
        \item период колебаний,
        \item ёмкость конденсатора,
        \item максимальный заряд конденсатора.
    \end{itemize}
}
\answer{%
    \begin{align*}
    \omega &= 18\units{c}^-1, \\
    T &= \frac{2\pi}\omega \approx 349{,}1\,\text{мc}, \\
    C &= \frac 1{\omega^2 L} \approx 38{,}6\,\text{мФ}, \\
    q &= \frac{\eli_{\max}}\omega  \approx 2{,}8\,\text{мКл}.
    \end{align*}
}
\solutionspace{80pt}

\tasknumber{4}%
\task{%
    Электрический колебательный контур состоит
    из катушки индуктивностью $L$ и конденсатора ёмкостью $C$.
    последовательно катушке подключают ещё одну катушку индуктивностью $\frac13L$.
    Как изменится период сводобных колебаний в контуре?
}
\answer{%
    \begin{align*}
    T &= 2\pi\sqrt{LC} \\
    T' &= 2\pi\sqrt{L'C'}= T \sqrt{\frac{L'}L \cdot \frac{C'}C} = T \sqrt{ \frac43 \cdot 1 } \\
    &\frac{T'}T = \sqrt{ \frac43 \cdot 1 } \approx 1{,}778
    \end{align*}
}

\variantsplitter

\addpersonalvariant{Герман Говоров}

\tasknumber{1}%
\task{%
    Схематично изобразите колебательный контур.
    Запишите формулу для периода колебаний в колебательном контуре и ...
    \begin{itemize}
        \item подпишите все физические величины,
        \item укажите их единицы измерения,
        \item выразите из формулы периода частоту,
        \item выразите из формулы периода выразите емкость конденсатора.
    \end{itemize}
}
\answer{%
    \begin{align*}
    T &= 2\pi\sqrt{LC} \\
    \nu &= \frac 1{2\pi\sqrt{LC}}, \\
    \omega &= \frac 1{\sqrt{LC}}, \\
    L &= \frac 1C \sqr{\frac T{2\pi}}, \\
    C &= \frac 1L \sqr{\frac T{2\pi}}.
    \end{align*}
}
\solutionspace{80pt}

\tasknumber{2}%
\task{%
    Оказалось, что наибольший заряд конденсатора в колебательном контуре равен $60\,\text{мкКл}$,
    а максимальный ток — $120\,\text{мА}$.
    Определите частоту колебаний.
}
\answer{%
    $
        \eli_{\max} = q_{\max}\omega \implies \nu = \frac{\omega}{2\pi} = \frac{\eli_{\max}}{2\pi q} \approx 0{,}318\,\text{кГц}
    $
}
\solutionspace{80pt}

\tasknumber{3}%
\task{%
    В колебательном контура сила тока изменяется
    по закону $\eli=0{,}30\sin(18t)$ (в СИ).
    Индуктивность катушки при этом равна $70\,\text{мГн}$.
    Определите:
    \begin{itemize}
        \item период колебаний,
        \item ёмкость конденсатора,
        \item максимальный заряд конденсатора.
    \end{itemize}
}
\answer{%
    \begin{align*}
    \omega &= 18\units{c}^-1, \\
    T &= \frac{2\pi}\omega \approx 349{,}1\,\text{мc}, \\
    C &= \frac 1{\omega^2 L} \approx 44{,}1\,\text{мФ}, \\
    q &= \frac{\eli_{\max}}\omega  \approx 16{,}7\,\text{мКл}.
    \end{align*}
}
\solutionspace{80pt}

\tasknumber{4}%
\task{%
    Электрический колебательный контур состоит
    из катушки индуктивностью $L$ и конденсатора ёмкостью $C$.
    последовательно конденсатору подключают ещё один конденсатор емкостью $\frac13L$.
    Как изменится период сводобных колебаний в контуре?
}
\answer{%
    \begin{align*}
    T &= 2\pi\sqrt{LC} \\
    T' &= 2\pi\sqrt{L'C'}= T \sqrt{\frac{L'}L \cdot \frac{C'}C} = T \sqrt{ 1 \cdot \frac14 } \\
    &\frac{T'}T = \sqrt{ 1 \cdot \frac14 } \approx 0{,}062
    \end{align*}
}

\variantsplitter

\addpersonalvariant{София Журавлёва}

\tasknumber{1}%
\task{%
    Схематично изобразите колебательный контур.
    Запишите формулу для периода колебаний в колебательном контуре и ...
    \begin{itemize}
        \item подпишите все физические величины,
        \item укажите их единицы измерения,
        \item выразите из формулы периода частоту,
        \item выразите из формулы периода индуктивность катушки..
    \end{itemize}
}
\answer{%
    \begin{align*}
    T &= 2\pi\sqrt{LC} \\
    \nu &= \frac 1{2\pi\sqrt{LC}}, \\
    \omega &= \frac 1{\sqrt{LC}}, \\
    L &= \frac 1C \sqr{\frac T{2\pi}}, \\
    C &= \frac 1L \sqr{\frac T{2\pi}}.
    \end{align*}
}
\solutionspace{80pt}

\tasknumber{2}%
\task{%
    Оказалось, что наибольший заряд конденсатора в колебательном контуре равен $80\,\text{мкКл}$,
    а максимальный ток — $120\,\text{мА}$.
    Определите частоту колебаний.
}
\answer{%
    $
        \eli_{\max} = q_{\max}\omega \implies \nu = \frac{\omega}{2\pi} = \frac{\eli_{\max}}{2\pi q} \approx 0{,}239\,\text{кГц}
    $
}
\solutionspace{80pt}

\tasknumber{3}%
\task{%
    В колебательном контура сила тока изменяется
    по закону $\eli=0{,}25\sin(15t)$ (в СИ).
    Индуктивность катушки при этом равна $60\,\text{мГн}$.
    Определите:
    \begin{itemize}
        \item период колебаний,
        \item ёмкость конденсатора,
        \item максимальный заряд конденсатора.
    \end{itemize}
}
\answer{%
    \begin{align*}
    \omega &= 15\units{c}^-1, \\
    T &= \frac{2\pi}\omega \approx 418{,}9\,\text{мc}, \\
    C &= \frac 1{\omega^2 L} \approx 74{,}1\,\text{мФ}, \\
    q &= \frac{\eli_{\max}}\omega  \approx 16{,}7\,\text{мКл}.
    \end{align*}
}
\solutionspace{80pt}

\tasknumber{4}%
\task{%
    Электрический колебательный контур состоит
    из катушки индуктивностью $L$ и конденсатора ёмкостью $C$.
    параллельно конденсатору подключают ещё один конденсатор емкостью $\frac12L$.
    Как изменится период сводобных колебаний в контуре?
}
\answer{%
    \begin{align*}
    T &= 2\pi\sqrt{LC} \\
    T' &= 2\pi\sqrt{L'C'}= T \sqrt{\frac{L'}L \cdot \frac{C'}C} = T \sqrt{ 1 \cdot \frac32 } \\
    &\frac{T'}T = \sqrt{ 1 \cdot \frac32 } \approx 2{,}250
    \end{align*}
}

\variantsplitter

\addpersonalvariant{Константин Козлов}

\tasknumber{1}%
\task{%
    Схематично изобразите колебательный контур.
    Запишите формулу для периода колебаний в колебательном контуре и ...
    \begin{itemize}
        \item подпишите все физические величины,
        \item укажите их единицы измерения,
        \item выразите из формулы периода частоту,
        \item выразите из формулы периода индуктивность катушки..
    \end{itemize}
}
\answer{%
    \begin{align*}
    T &= 2\pi\sqrt{LC} \\
    \nu &= \frac 1{2\pi\sqrt{LC}}, \\
    \omega &= \frac 1{\sqrt{LC}}, \\
    L &= \frac 1C \sqr{\frac T{2\pi}}, \\
    C &= \frac 1L \sqr{\frac T{2\pi}}.
    \end{align*}
}
\solutionspace{80pt}

\tasknumber{2}%
\task{%
    Оказалось, что наибольший заряд конденсатора в колебательном контуре равен $80\,\text{мкКл}$,
    а максимальный ток — $240\,\text{мА}$.
    Определите частоту колебаний.
}
\answer{%
    $
        \eli_{\max} = q_{\max}\omega \implies \nu = \frac{\omega}{2\pi} = \frac{\eli_{\max}}{2\pi q} \approx 0{,}477\,\text{кГц}
    $
}
\solutionspace{80pt}

\tasknumber{3}%
\task{%
    В колебательном контура сила тока изменяется
    по закону $\eli=0{,}25\cos(18t)$ (в СИ).
    Индуктивность катушки при этом равна $70\,\text{мГн}$.
    Определите:
    \begin{itemize}
        \item период колебаний,
        \item ёмкость конденсатора,
        \item максимальный заряд конденсатора.
    \end{itemize}
}
\answer{%
    \begin{align*}
    \omega &= 18\units{c}^-1, \\
    T &= \frac{2\pi}\omega \approx 349{,}1\,\text{мc}, \\
    C &= \frac 1{\omega^2 L} \approx 44{,}1\,\text{мФ}, \\
    q &= \frac{\eli_{\max}}\omega  \approx 13{,}9\,\text{мКл}.
    \end{align*}
}
\solutionspace{80pt}

\tasknumber{4}%
\task{%
    Электрический колебательный контур состоит
    из катушки индуктивностью $L$ и конденсатора ёмкостью $C$.
    последовательно конденсатору подключают ещё один конденсатор емкостью $\frac13L$.
    Как изменится период сводобных колебаний в контуре?
}
\answer{%
    \begin{align*}
    T &= 2\pi\sqrt{LC} \\
    T' &= 2\pi\sqrt{L'C'}= T \sqrt{\frac{L'}L \cdot \frac{C'}C} = T \sqrt{ 1 \cdot \frac14 } \\
    &\frac{T'}T = \sqrt{ 1 \cdot \frac14 } \approx 0{,}062
    \end{align*}
}

\variantsplitter

\addpersonalvariant{Наталья Кравченко}

\tasknumber{1}%
\task{%
    Схематично изобразите колебательный контур.
    Запишите формулу для периода колебаний в колебательном контуре и ...
    \begin{itemize}
        \item подпишите все физические величины,
        \item укажите их единицы измерения,
        \item выразите из формулы периода частоту,
        \item выразите из формулы периода индуктивность катушки..
    \end{itemize}
}
\answer{%
    \begin{align*}
    T &= 2\pi\sqrt{LC} \\
    \nu &= \frac 1{2\pi\sqrt{LC}}, \\
    \omega &= \frac 1{\sqrt{LC}}, \\
    L &= \frac 1C \sqr{\frac T{2\pi}}, \\
    C &= \frac 1L \sqr{\frac T{2\pi}}.
    \end{align*}
}
\solutionspace{80pt}

\tasknumber{2}%
\task{%
    Оказалось, что наибольший заряд конденсатора в колебательном контуре равен $40\,\text{мкКл}$,
    а максимальный ток — $270\,\text{мА}$.
    Определите частоту колебаний.
}
\answer{%
    $
        \eli_{\max} = q_{\max}\omega \implies \nu = \frac{\omega}{2\pi} = \frac{\eli_{\max}}{2\pi q} \approx 1{,}074\,\text{кГц}
    $
}
\solutionspace{80pt}

\tasknumber{3}%
\task{%
    В колебательном контура сила тока изменяется
    по закону $\eli=0{,}30\sin(18t)$ (в СИ).
    Индуктивность катушки при этом равна $50\,\text{мГн}$.
    Определите:
    \begin{itemize}
        \item период колебаний,
        \item ёмкость конденсатора,
        \item максимальный заряд конденсатора.
    \end{itemize}
}
\answer{%
    \begin{align*}
    \omega &= 18\units{c}^-1, \\
    T &= \frac{2\pi}\omega \approx 349{,}1\,\text{мc}, \\
    C &= \frac 1{\omega^2 L} \approx 61{,}7\,\text{мФ}, \\
    q &= \frac{\eli_{\max}}\omega  \approx 16{,}7\,\text{мКл}.
    \end{align*}
}
\solutionspace{80pt}

\tasknumber{4}%
\task{%
    Электрический колебательный контур состоит
    из катушки индуктивностью $L$ и конденсатора ёмкостью $C$.
    последовательно конденсатору подключают ещё один конденсатор емкостью $\frac12L$.
    Как изменится период сводобных колебаний в контуре?
}
\answer{%
    \begin{align*}
    T &= 2\pi\sqrt{LC} \\
    T' &= 2\pi\sqrt{L'C'}= T \sqrt{\frac{L'}L \cdot \frac{C'}C} = T \sqrt{ 1 \cdot \frac13 } \\
    &\frac{T'}T = \sqrt{ 1 \cdot \frac13 } \approx 0{,}111
    \end{align*}
}

\variantsplitter

\addpersonalvariant{Сергей Малышев}

\tasknumber{1}%
\task{%
    Схематично изобразите колебательный контур.
    Запишите формулу для периода колебаний в колебательном контуре и ...
    \begin{itemize}
        \item подпишите все физические величины,
        \item укажите их единицы измерения,
        \item выразите из формулы периода частоту,
        \item выразите из формулы периода выразите емкость конденсатора.
    \end{itemize}
}
\answer{%
    \begin{align*}
    T &= 2\pi\sqrt{LC} \\
    \nu &= \frac 1{2\pi\sqrt{LC}}, \\
    \omega &= \frac 1{\sqrt{LC}}, \\
    L &= \frac 1C \sqr{\frac T{2\pi}}, \\
    C &= \frac 1L \sqr{\frac T{2\pi}}.
    \end{align*}
}
\solutionspace{80pt}

\tasknumber{2}%
\task{%
    Оказалось, что наибольший заряд конденсатора в колебательном контуре равен $80\,\text{мкКл}$,
    а максимальный ток — $150\,\text{мА}$.
    Определите частоту колебаний.
}
\answer{%
    $
        \eli_{\max} = q_{\max}\omega \implies \nu = \frac{\omega}{2\pi} = \frac{\eli_{\max}}{2\pi q} \approx 0{,}298\,\text{кГц}
    $
}
\solutionspace{80pt}

\tasknumber{3}%
\task{%
    В колебательном контура сила тока изменяется
    по закону $\eli=0{,}30\cos(18t)$ (в СИ).
    Индуктивность катушки при этом равна $60\,\text{мГн}$.
    Определите:
    \begin{itemize}
        \item период колебаний,
        \item ёмкость конденсатора,
        \item максимальный заряд конденсатора.
    \end{itemize}
}
\answer{%
    \begin{align*}
    \omega &= 18\units{c}^-1, \\
    T &= \frac{2\pi}\omega \approx 349{,}1\,\text{мc}, \\
    C &= \frac 1{\omega^2 L} \approx 51{,}4\,\text{мФ}, \\
    q &= \frac{\eli_{\max}}\omega  \approx 16{,}7\,\text{мКл}.
    \end{align*}
}
\solutionspace{80pt}

\tasknumber{4}%
\task{%
    Электрический колебательный контур состоит
    из катушки индуктивностью $L$ и конденсатора ёмкостью $C$.
    параллельно катушке подключают ещё одну катушку индуктивностью $\frac12L$.
    Как изменится период сводобных колебаний в контуре?
}
\answer{%
    \begin{align*}
    T &= 2\pi\sqrt{LC} \\
    T' &= 2\pi\sqrt{L'C'}= T \sqrt{\frac{L'}L \cdot \frac{C'}C} = T \sqrt{ \frac13 \cdot 1 } \\
    &\frac{T'}T = \sqrt{ \frac13 \cdot 1 } \approx 0{,}111
    \end{align*}
}

\variantsplitter

\addpersonalvariant{Алина Полканова}

\tasknumber{1}%
\task{%
    Схематично изобразите колебательный контур.
    Запишите формулу для периода колебаний в колебательном контуре и ...
    \begin{itemize}
        \item подпишите все физические величины,
        \item укажите их единицы измерения,
        \item выразите из формулы периода циклическую частоту,
        \item выразите из формулы периода индуктивность катушки..
    \end{itemize}
}
\answer{%
    \begin{align*}
    T &= 2\pi\sqrt{LC} \\
    \nu &= \frac 1{2\pi\sqrt{LC}}, \\
    \omega &= \frac 1{\sqrt{LC}}, \\
    L &= \frac 1C \sqr{\frac T{2\pi}}, \\
    C &= \frac 1L \sqr{\frac T{2\pi}}.
    \end{align*}
}
\solutionspace{80pt}

\tasknumber{2}%
\task{%
    Оказалось, что наибольший заряд конденсатора в колебательном контуре равен $60\,\text{мкКл}$,
    а максимальный ток — $270\,\text{мА}$.
    Определите частоту колебаний.
}
\answer{%
    $
        \eli_{\max} = q_{\max}\omega \implies \nu = \frac{\omega}{2\pi} = \frac{\eli_{\max}}{2\pi q} \approx 0{,}716\,\text{кГц}
    $
}
\solutionspace{80pt}

\tasknumber{3}%
\task{%
    В колебательном контура сила тока изменяется
    по закону $\eli=0{,}30\sin(15t)$ (в СИ).
    Индуктивность катушки при этом равна $60\,\text{мГн}$.
    Определите:
    \begin{itemize}
        \item период колебаний,
        \item ёмкость конденсатора,
        \item максимальный заряд конденсатора.
    \end{itemize}
}
\answer{%
    \begin{align*}
    \omega &= 15\units{c}^-1, \\
    T &= \frac{2\pi}\omega \approx 418{,}9\,\text{мc}, \\
    C &= \frac 1{\omega^2 L} \approx 74{,}1\,\text{мФ}, \\
    q &= \frac{\eli_{\max}}\omega  \approx 20{,}0\,\text{мКл}.
    \end{align*}
}
\solutionspace{80pt}

\tasknumber{4}%
\task{%
    Электрический колебательный контур состоит
    из катушки индуктивностью $L$ и конденсатора ёмкостью $C$.
    параллельно конденсатору подключают ещё один конденсатор емкостью $\frac13L$.
    Как изменится период сводобных колебаний в контуре?
}
\answer{%
    \begin{align*}
    T &= 2\pi\sqrt{LC} \\
    T' &= 2\pi\sqrt{L'C'}= T \sqrt{\frac{L'}L \cdot \frac{C'}C} = T \sqrt{ 1 \cdot \frac43 } \\
    &\frac{T'}T = \sqrt{ 1 \cdot \frac43 } \approx 1{,}778
    \end{align*}
}

\variantsplitter

\addpersonalvariant{Сергей Пономарёв}

\tasknumber{1}%
\task{%
    Схематично изобразите колебательный контур.
    Запишите формулу для периода колебаний в колебательном контуре и ...
    \begin{itemize}
        \item подпишите все физические величины,
        \item укажите их единицы измерения,
        \item выразите из формулы периода циклическую частоту,
        \item выразите из формулы периода индуктивность катушки..
    \end{itemize}
}
\answer{%
    \begin{align*}
    T &= 2\pi\sqrt{LC} \\
    \nu &= \frac 1{2\pi\sqrt{LC}}, \\
    \omega &= \frac 1{\sqrt{LC}}, \\
    L &= \frac 1C \sqr{\frac T{2\pi}}, \\
    C &= \frac 1L \sqr{\frac T{2\pi}}.
    \end{align*}
}
\solutionspace{80pt}

\tasknumber{2}%
\task{%
    Оказалось, что наибольший заряд конденсатора в колебательном контуре равен $60\,\text{мкКл}$,
    а максимальный ток — $120\,\text{мА}$.
    Определите частоту колебаний.
}
\answer{%
    $
        \eli_{\max} = q_{\max}\omega \implies \nu = \frac{\omega}{2\pi} = \frac{\eli_{\max}}{2\pi q} \approx 0{,}318\,\text{кГц}
    $
}
\solutionspace{80pt}

\tasknumber{3}%
\task{%
    В колебательном контура сила тока изменяется
    по закону $\eli=0{,}05\cos(15t)$ (в СИ).
    Индуктивность катушки при этом равна $80\,\text{мГн}$.
    Определите:
    \begin{itemize}
        \item период колебаний,
        \item ёмкость конденсатора,
        \item максимальный заряд конденсатора.
    \end{itemize}
}
\answer{%
    \begin{align*}
    \omega &= 15\units{c}^-1, \\
    T &= \frac{2\pi}\omega \approx 418{,}9\,\text{мc}, \\
    C &= \frac 1{\omega^2 L} \approx 55{,}6\,\text{мФ}, \\
    q &= \frac{\eli_{\max}}\omega  \approx 3{,}3\,\text{мКл}.
    \end{align*}
}
\solutionspace{80pt}

\tasknumber{4}%
\task{%
    Электрический колебательный контур состоит
    из катушки индуктивностью $L$ и конденсатора ёмкостью $C$.
    параллельно катушке подключают ещё одну катушку индуктивностью $\frac13L$.
    Как изменится период сводобных колебаний в контуре?
}
\answer{%
    \begin{align*}
    T &= 2\pi\sqrt{LC} \\
    T' &= 2\pi\sqrt{L'C'}= T \sqrt{\frac{L'}L \cdot \frac{C'}C} = T \sqrt{ \frac14 \cdot 1 } \\
    &\frac{T'}T = \sqrt{ \frac14 \cdot 1 } \approx 0{,}062
    \end{align*}
}

\variantsplitter

\addpersonalvariant{Егор Свистушкин}

\tasknumber{1}%
\task{%
    Схематично изобразите колебательный контур.
    Запишите формулу для периода колебаний в колебательном контуре и ...
    \begin{itemize}
        \item подпишите все физические величины,
        \item укажите их единицы измерения,
        \item выразите из формулы периода частоту,
        \item выразите из формулы периода индуктивность катушки..
    \end{itemize}
}
\answer{%
    \begin{align*}
    T &= 2\pi\sqrt{LC} \\
    \nu &= \frac 1{2\pi\sqrt{LC}}, \\
    \omega &= \frac 1{\sqrt{LC}}, \\
    L &= \frac 1C \sqr{\frac T{2\pi}}, \\
    C &= \frac 1L \sqr{\frac T{2\pi}}.
    \end{align*}
}
\solutionspace{80pt}

\tasknumber{2}%
\task{%
    Оказалось, что наибольший заряд конденсатора в колебательном контуре равен $60\,\text{мкКл}$,
    а максимальный ток — $120\,\text{мА}$.
    Определите частоту колебаний.
}
\answer{%
    $
        \eli_{\max} = q_{\max}\omega \implies \nu = \frac{\omega}{2\pi} = \frac{\eli_{\max}}{2\pi q} \approx 0{,}318\,\text{кГц}
    $
}
\solutionspace{80pt}

\tasknumber{3}%
\task{%
    В колебательном контура сила тока изменяется
    по закону $\eli=0{,}25\sin(15t)$ (в СИ).
    Индуктивность катушки при этом равна $70\,\text{мГн}$.
    Определите:
    \begin{itemize}
        \item период колебаний,
        \item ёмкость конденсатора,
        \item максимальный заряд конденсатора.
    \end{itemize}
}
\answer{%
    \begin{align*}
    \omega &= 15\units{c}^-1, \\
    T &= \frac{2\pi}\omega \approx 418{,}9\,\text{мc}, \\
    C &= \frac 1{\omega^2 L} \approx 63{,}5\,\text{мФ}, \\
    q &= \frac{\eli_{\max}}\omega  \approx 16{,}7\,\text{мКл}.
    \end{align*}
}
\solutionspace{80pt}

\tasknumber{4}%
\task{%
    Электрический колебательный контур состоит
    из катушки индуктивностью $L$ и конденсатора ёмкостью $C$.
    последовательно катушке подключают ещё одну катушку индуктивностью $2L$.
    Как изменится период сводобных колебаний в контуре?
}
\answer{%
    \begin{align*}
    T &= 2\pi\sqrt{LC} \\
    T' &= 2\pi\sqrt{L'C'}= T \sqrt{\frac{L'}L \cdot \frac{C'}C} = T \sqrt{ 3 \cdot 1 } \\
    &\frac{T'}T = \sqrt{ 3 \cdot 1 } \approx 9{,}000
    \end{align*}
}

\variantsplitter

\addpersonalvariant{Дмитрий Соколов}

\tasknumber{1}%
\task{%
    Схематично изобразите колебательный контур.
    Запишите формулу для периода колебаний в колебательном контуре и ...
    \begin{itemize}
        \item подпишите все физические величины,
        \item укажите их единицы измерения,
        \item выразите из формулы периода частоту,
        \item выразите из формулы периода выразите емкость конденсатора.
    \end{itemize}
}
\answer{%
    \begin{align*}
    T &= 2\pi\sqrt{LC} \\
    \nu &= \frac 1{2\pi\sqrt{LC}}, \\
    \omega &= \frac 1{\sqrt{LC}}, \\
    L &= \frac 1C \sqr{\frac T{2\pi}}, \\
    C &= \frac 1L \sqr{\frac T{2\pi}}.
    \end{align*}
}
\solutionspace{80pt}

\tasknumber{2}%
\task{%
    Оказалось, что наибольший заряд конденсатора в колебательном контуре равен $80\,\text{мкКл}$,
    а максимальный ток — $120\,\text{мА}$.
    Определите частоту колебаний.
}
\answer{%
    $
        \eli_{\max} = q_{\max}\omega \implies \nu = \frac{\omega}{2\pi} = \frac{\eli_{\max}}{2\pi q} \approx 0{,}239\,\text{кГц}
    $
}
\solutionspace{80pt}

\tasknumber{3}%
\task{%
    В колебательном контура сила тока изменяется
    по закону $\eli=0{,}05\sin(15t)$ (в СИ).
    Индуктивность катушки при этом равна $80\,\text{мГн}$.
    Определите:
    \begin{itemize}
        \item период колебаний,
        \item ёмкость конденсатора,
        \item максимальный заряд конденсатора.
    \end{itemize}
}
\answer{%
    \begin{align*}
    \omega &= 15\units{c}^-1, \\
    T &= \frac{2\pi}\omega \approx 418{,}9\,\text{мc}, \\
    C &= \frac 1{\omega^2 L} \approx 55{,}6\,\text{мФ}, \\
    q &= \frac{\eli_{\max}}\omega  \approx 3{,}3\,\text{мКл}.
    \end{align*}
}
\solutionspace{80pt}

\tasknumber{4}%
\task{%
    Электрический колебательный контур состоит
    из катушки индуктивностью $L$ и конденсатора ёмкостью $C$.
    последовательно катушке подключают ещё одну катушку индуктивностью $\frac12L$.
    Как изменится период сводобных колебаний в контуре?
}
\answer{%
    \begin{align*}
    T &= 2\pi\sqrt{LC} \\
    T' &= 2\pi\sqrt{L'C'}= T \sqrt{\frac{L'}L \cdot \frac{C'}C} = T \sqrt{ \frac32 \cdot 1 } \\
    &\frac{T'}T = \sqrt{ \frac32 \cdot 1 } \approx 2{,}250
    \end{align*}
}

\variantsplitter

\addpersonalvariant{Арсений Трофимов}

\tasknumber{1}%
\task{%
    Схематично изобразите колебательный контур.
    Запишите формулу для периода колебаний в колебательном контуре и ...
    \begin{itemize}
        \item подпишите все физические величины,
        \item укажите их единицы измерения,
        \item выразите из формулы периода частоту,
        \item выразите из формулы периода индуктивность катушки..
    \end{itemize}
}
\answer{%
    \begin{align*}
    T &= 2\pi\sqrt{LC} \\
    \nu &= \frac 1{2\pi\sqrt{LC}}, \\
    \omega &= \frac 1{\sqrt{LC}}, \\
    L &= \frac 1C \sqr{\frac T{2\pi}}, \\
    C &= \frac 1L \sqr{\frac T{2\pi}}.
    \end{align*}
}
\solutionspace{80pt}

\tasknumber{2}%
\task{%
    Оказалось, что наибольший заряд конденсатора в колебательном контуре равен $60\,\text{мкКл}$,
    а максимальный ток — $150\,\text{мА}$.
    Определите частоту колебаний.
}
\answer{%
    $
        \eli_{\max} = q_{\max}\omega \implies \nu = \frac{\omega}{2\pi} = \frac{\eli_{\max}}{2\pi q} \approx 0{,}398\,\text{кГц}
    $
}
\solutionspace{80pt}

\tasknumber{3}%
\task{%
    В колебательном контура сила тока изменяется
    по закону $\eli=0{,}30\sin(12t)$ (в СИ).
    Индуктивность катушки при этом равна $60\,\text{мГн}$.
    Определите:
    \begin{itemize}
        \item период колебаний,
        \item ёмкость конденсатора,
        \item максимальный заряд конденсатора.
    \end{itemize}
}
\answer{%
    \begin{align*}
    \omega &= 12\units{c}^-1, \\
    T &= \frac{2\pi}\omega \approx 523{,}6\,\text{мc}, \\
    C &= \frac 1{\omega^2 L} \approx 115{,}7\,\text{мФ}, \\
    q &= \frac{\eli_{\max}}\omega  \approx 25{,}0\,\text{мКл}.
    \end{align*}
}
\solutionspace{80pt}

\tasknumber{4}%
\task{%
    Электрический колебательный контур состоит
    из катушки индуктивностью $L$ и конденсатора ёмкостью $C$.
    последовательно конденсатору подключают ещё один конденсатор емкостью $2L$.
    Как изменится период сводобных колебаний в контуре?
}
\answer{%
    \begin{align*}
    T &= 2\pi\sqrt{LC} \\
    T' &= 2\pi\sqrt{L'C'}= T \sqrt{\frac{L'}L \cdot \frac{C'}C} = T \sqrt{ 1 \cdot \frac23 } \\
    &\frac{T'}T = \sqrt{ 1 \cdot \frac23 } \approx 0{,}444
    \end{align*}
}
% autogenerated
