\setdate{21~октября~2021}
\setclass{11«БА»}

\addpersonalvariant{Михаил Бурмистров}

\tasknumber{1}%
\task{%
    Схематично изобразите колебательный контур.
    Запишите формулу для периода колебаний в колебательном контуре и ...
    \begin{itemize}
        \item подпишите все физические величины,
        \item укажите их единицы измерения,
        \item выразите из формулы периода циклическую частоту,
        \item выразите из формулы периода ёмкость конденсатора.
    \end{itemize}
}
\answer{%
    $
        T = 2\pi\sqrt{LC},
        \nu = \frac 1{2\pi\sqrt{LC}},
        \omega = \frac 1{\sqrt{LC}},
        L = \frac 1C \sqr{\frac T{2\pi}},
        C = \frac 1L \sqr{\frac T{2\pi}}.
    $
}
\solutionspace{80pt}

\tasknumber{2}%
\task{%
    Оказалось, что наибольший заряд конденсатора в колебательном контуре равен $60\,\text{мкКл}$,
    а максимальный ток — $270\,\text{мА}$.
    Определите частоту колебаний.
}
\answer{%
    $
        \eli_{\max} = q_{\max}\omega \implies \nu = \frac{\omega}{2\pi} = \frac{\eli_{\max}}{2\pi q} \approx 716{,}2\,\text{Гц}.
    $
}
\solutionspace{80pt}

\tasknumber{3}%
\task{%
    В колебательном контуре сила тока изменяется
    по закону $\eli=0{,}25\sin(12t)$ (в СИ).
    Индуктивность катушки при этом равна $50\,\text{мГн}$.
    Определите:
    \begin{itemize}
        \item период колебаний,
        \item ёмкость конденсатора,
        \item максимальный заряд конденсатора.
    \end{itemize}
}
\answer{%
    \begin{align*}
    \omega &= 12\funits{рад}{c}, \qquad \eli_{\max} = 0{,}25\,\text{A}, \\
    T &= \frac{2\pi}\omega \approx 523{,}6\,\text{мc}, \\
    C &= \frac 1{\omega^2 L} \approx 138{,}9\,\text{мФ}, \\
    q_{\max} &= \frac{\eli_{\max}}\omega  \approx 20{,}8\,\text{мКл}.
    \end{align*}
}
\solutionspace{80pt}

\tasknumber{4}%
\task{%
    Электрический колебательный контур состоит
    из катушки индуктивностью $L$ и конденсатора ёмкостью $C$.
    Параллельно катушке подключают ещё одну катушку индуктивностью $\frac13L$.
    Как изменится период свободных колебаний в контуре?
}
\answer{%
    $
        T = 2\pi\sqrt{LC}, \quad
        T' = 2\pi\sqrt{L'C'}
            = T \sqrt{\frac{L'}L \cdot \frac{C'}C}
            = T \sqrt{ \frac14 \cdot 1 }
        \implies \frac{T'}T = \sqrt{ \frac14 \cdot 1 } \approx 0{,}500.
    $
}
\solutionspace{100pt}

\tasknumber{5}%
\task{%
    В колебательном контуре частота собственных колебаний $80\,\text{Гц}$.
    После замены катушки индуктивности на другую катушку частота стала равной $50\,\text{Гц}$.
    А какой станет частота, если в контур установить обе эти катушки последовательно?
}
\answer{%
    \begin{align*}
    T &= 2\pi\sqrt{LC} \implies \nu = \frac 1T = \frac 1{2\pi\sqrt{LC}} \implies L = \frac1 {4\pi^2 \nu^2 C}, \\
    L_1 &= \frac1 {4\pi^2 \nu_1^2 C}, L_2 = \frac1 {4\pi^2 \nu_1^2 C}, \\
    \nu_\text{послед.}
            &= \frac 1{2\pi\sqrt{(L_1 + L_2)C}}
            = \frac 1{2\pi\sqrt{\cbr{\frac1 {4\pi^2 \nu_1^2 C} + \frac1 {4\pi^2 \nu_2^2 C}}C}}
            = \frac 1{\sqrt{\cbr{\frac1 {\nu_1^2 C} + \frac1 {\nu_2^2 C}}C}} =  \\
     &= \frac 1{\sqrt{\frac1 {\nu_1^2} + \frac1 {\nu_2^2}}}
            = \frac 1{\sqrt{ \frac1 {\sqr{80\,\text{Гц}}} + \frac1 {\sqr{50\,\text{Гц}}}}}
            \approx 42{,}40\,\text{Гц}, \\
    \nu_\text{паралл.}
            &= \frac 1{2\pi\sqrt{\frac 1{\frac 1{L_1} + \frac 1{L_2}}C}}
            = \frac 1{2\pi\sqrt{\frac 1{\frac 1{\frac1 {4\pi^2 \nu_1^2 C}} + \frac 1{\frac1 {4\pi^2 \nu_2^2 C}}}C}}
            = \frac 1{2\pi\sqrt{\frac 1{4\pi^2 \nu_1^2 C + 4\pi^2 \nu_2^2 C}C}} = \\
     &= \frac 1{\sqrt{\frac 1{\nu_1^2 + \nu_2^2}}}
            = \sqrt{\nu_1^2 + \nu_2^2} = \sqrt{\sqr{80\,\text{Гц}} + \sqr{50\,\text{Гц}}} \approx 94{,}34\,\text{Гц}.
    \end{align*}
}

\variantsplitter

\addpersonalvariant{Ирина Ан}

\tasknumber{1}%
\task{%
    Схематично изобразите колебательный контур.
    Запишите формулу для периода колебаний в колебательном контуре и ...
    \begin{itemize}
        \item подпишите все физические величины,
        \item укажите их единицы измерения,
        \item выразите из формулы периода циклическую частоту,
        \item выразите из формулы периода ёмкость конденсатора.
    \end{itemize}
}
\answer{%
    $
        T = 2\pi\sqrt{LC},
        \nu = \frac 1{2\pi\sqrt{LC}},
        \omega = \frac 1{\sqrt{LC}},
        L = \frac 1C \sqr{\frac T{2\pi}},
        C = \frac 1L \sqr{\frac T{2\pi}}.
    $
}
\solutionspace{80pt}

\tasknumber{2}%
\task{%
    Оказалось, что наибольший заряд конденсатора в колебательном контуре равен $40\,\text{мкКл}$,
    а максимальный ток — $240\,\text{мА}$.
    Определите частоту колебаний.
}
\answer{%
    $
        \eli_{\max} = q_{\max}\omega \implies \nu = \frac{\omega}{2\pi} = \frac{\eli_{\max}}{2\pi q} \approx 954{,}9\,\text{Гц}.
    $
}
\solutionspace{80pt}

\tasknumber{3}%
\task{%
    В колебательном контуре сила тока изменяется
    по закону $\eli=0{,}05\cos(12t)$ (в СИ).
    Индуктивность катушки при этом равна $60\,\text{мГн}$.
    Определите:
    \begin{itemize}
        \item период колебаний,
        \item ёмкость конденсатора,
        \item максимальный заряд конденсатора.
    \end{itemize}
}
\answer{%
    \begin{align*}
    \omega &= 12\funits{рад}{c}, \qquad \eli_{\max} = 0{,}05\,\text{A}, \\
    T &= \frac{2\pi}\omega \approx 523{,}6\,\text{мc}, \\
    C &= \frac 1{\omega^2 L} \approx 115{,}7\,\text{мФ}, \\
    q_{\max} &= \frac{\eli_{\max}}\omega  \approx 4{,}2\,\text{мКл}.
    \end{align*}
}
\solutionspace{80pt}

\tasknumber{4}%
\task{%
    Электрический колебательный контур состоит
    из катушки индуктивностью $L$ и конденсатора ёмкостью $C$.
    Последовательно катушке подключают ещё одну катушку индуктивностью $\frac12L$.
    Как изменится период свободных колебаний в контуре?
}
\answer{%
    $
        T = 2\pi\sqrt{LC}, \quad
        T' = 2\pi\sqrt{L'C'}
            = T \sqrt{\frac{L'}L \cdot \frac{C'}C}
            = T \sqrt{ \frac32 \cdot 1 }
        \implies \frac{T'}T = \sqrt{ \frac32 \cdot 1 } \approx 1{,}225.
    $
}
\solutionspace{100pt}

\tasknumber{5}%
\task{%
    В колебательном контуре частота собственных колебаний $40\,\text{Гц}$.
    После замены катушки индуктивности на другую катушку частота стала равной $30\,\text{Гц}$.
    А какой станет частота, если в контур установить обе эти катушки параллельно?
}
\answer{%
    \begin{align*}
    T &= 2\pi\sqrt{LC} \implies \nu = \frac 1T = \frac 1{2\pi\sqrt{LC}} \implies L = \frac1 {4\pi^2 \nu^2 C}, \\
    L_1 &= \frac1 {4\pi^2 \nu_1^2 C}, L_2 = \frac1 {4\pi^2 \nu_1^2 C}, \\
    \nu_\text{послед.}
            &= \frac 1{2\pi\sqrt{(L_1 + L_2)C}}
            = \frac 1{2\pi\sqrt{\cbr{\frac1 {4\pi^2 \nu_1^2 C} + \frac1 {4\pi^2 \nu_2^2 C}}C}}
            = \frac 1{\sqrt{\cbr{\frac1 {\nu_1^2 C} + \frac1 {\nu_2^2 C}}C}} =  \\
     &= \frac 1{\sqrt{\frac1 {\nu_1^2} + \frac1 {\nu_2^2}}}
            = \frac 1{\sqrt{ \frac1 {\sqr{40\,\text{Гц}}} + \frac1 {\sqr{30\,\text{Гц}}}}}
            \approx 24\,\text{Гц}, \\
    \nu_\text{паралл.}
            &= \frac 1{2\pi\sqrt{\frac 1{\frac 1{L_1} + \frac 1{L_2}}C}}
            = \frac 1{2\pi\sqrt{\frac 1{\frac 1{\frac1 {4\pi^2 \nu_1^2 C}} + \frac 1{\frac1 {4\pi^2 \nu_2^2 C}}}C}}
            = \frac 1{2\pi\sqrt{\frac 1{4\pi^2 \nu_1^2 C + 4\pi^2 \nu_2^2 C}C}} = \\
     &= \frac 1{\sqrt{\frac 1{\nu_1^2 + \nu_2^2}}}
            = \sqrt{\nu_1^2 + \nu_2^2} = \sqrt{\sqr{40\,\text{Гц}} + \sqr{30\,\text{Гц}}} \approx 50\,\text{Гц}.
    \end{align*}
}

\variantsplitter

\addpersonalvariant{Софья Андрианова}

\tasknumber{1}%
\task{%
    Схематично изобразите колебательный контур.
    Запишите формулу для периода колебаний в колебательном контуре и ...
    \begin{itemize}
        \item подпишите все физические величины,
        \item укажите их единицы измерения,
        \item выразите из формулы периода циклическую частоту,
        \item выразите из формулы периода ёмкость конденсатора.
    \end{itemize}
}
\answer{%
    $
        T = 2\pi\sqrt{LC},
        \nu = \frac 1{2\pi\sqrt{LC}},
        \omega = \frac 1{\sqrt{LC}},
        L = \frac 1C \sqr{\frac T{2\pi}},
        C = \frac 1L \sqr{\frac T{2\pi}}.
    $
}
\solutionspace{80pt}

\tasknumber{2}%
\task{%
    Оказалось, что наибольший заряд конденсатора в колебательном контуре равен $80\,\text{мкКл}$,
    а максимальный ток — $270\,\text{мА}$.
    Определите частоту колебаний.
}
\answer{%
    $
        \eli_{\max} = q_{\max}\omega \implies \nu = \frac{\omega}{2\pi} = \frac{\eli_{\max}}{2\pi q} \approx 537{,}1\,\text{Гц}.
    $
}
\solutionspace{80pt}

\tasknumber{3}%
\task{%
    В колебательном контуре сила тока изменяется
    по закону $\eli=0{,}05\sin(15t)$ (в СИ).
    Индуктивность катушки при этом равна $70\,\text{мГн}$.
    Определите:
    \begin{itemize}
        \item период колебаний,
        \item ёмкость конденсатора,
        \item максимальный заряд конденсатора.
    \end{itemize}
}
\answer{%
    \begin{align*}
    \omega &= 15\funits{рад}{c}, \qquad \eli_{\max} = 0{,}05\,\text{A}, \\
    T &= \frac{2\pi}\omega \approx 418{,}9\,\text{мc}, \\
    C &= \frac 1{\omega^2 L} \approx 63{,}5\,\text{мФ}, \\
    q_{\max} &= \frac{\eli_{\max}}\omega  \approx 3{,}3\,\text{мКл}.
    \end{align*}
}
\solutionspace{80pt}

\tasknumber{4}%
\task{%
    Электрический колебательный контур состоит
    из катушки индуктивностью $L$ и конденсатора ёмкостью $C$.
    Параллельно катушке подключают ещё одну катушку индуктивностью $3L$.
    Как изменится период свободных колебаний в контуре?
}
\answer{%
    $
        T = 2\pi\sqrt{LC}, \quad
        T' = 2\pi\sqrt{L'C'}
            = T \sqrt{\frac{L'}L \cdot \frac{C'}C}
            = T \sqrt{ \frac34 \cdot 1 }
        \implies \frac{T'}T = \sqrt{ \frac34 \cdot 1 } \approx 0{,}866.
    $
}
\solutionspace{100pt}

\tasknumber{5}%
\task{%
    В колебательном контуре частота собственных колебаний $80\,\text{Гц}$.
    После замены катушки индуктивности на другую катушку частота стала равной $70\,\text{Гц}$.
    А какой станет частота, если в контур установить обе эти катушки последовательно?
}
\answer{%
    \begin{align*}
    T &= 2\pi\sqrt{LC} \implies \nu = \frac 1T = \frac 1{2\pi\sqrt{LC}} \implies L = \frac1 {4\pi^2 \nu^2 C}, \\
    L_1 &= \frac1 {4\pi^2 \nu_1^2 C}, L_2 = \frac1 {4\pi^2 \nu_1^2 C}, \\
    \nu_\text{послед.}
            &= \frac 1{2\pi\sqrt{(L_1 + L_2)C}}
            = \frac 1{2\pi\sqrt{\cbr{\frac1 {4\pi^2 \nu_1^2 C} + \frac1 {4\pi^2 \nu_2^2 C}}C}}
            = \frac 1{\sqrt{\cbr{\frac1 {\nu_1^2 C} + \frac1 {\nu_2^2 C}}C}} =  \\
     &= \frac 1{\sqrt{\frac1 {\nu_1^2} + \frac1 {\nu_2^2}}}
            = \frac 1{\sqrt{ \frac1 {\sqr{80\,\text{Гц}}} + \frac1 {\sqr{70\,\text{Гц}}}}}
            \approx 52{,}68\,\text{Гц}, \\
    \nu_\text{паралл.}
            &= \frac 1{2\pi\sqrt{\frac 1{\frac 1{L_1} + \frac 1{L_2}}C}}
            = \frac 1{2\pi\sqrt{\frac 1{\frac 1{\frac1 {4\pi^2 \nu_1^2 C}} + \frac 1{\frac1 {4\pi^2 \nu_2^2 C}}}C}}
            = \frac 1{2\pi\sqrt{\frac 1{4\pi^2 \nu_1^2 C + 4\pi^2 \nu_2^2 C}C}} = \\
     &= \frac 1{\sqrt{\frac 1{\nu_1^2 + \nu_2^2}}}
            = \sqrt{\nu_1^2 + \nu_2^2} = \sqrt{\sqr{80\,\text{Гц}} + \sqr{70\,\text{Гц}}} \approx 106{,}30\,\text{Гц}.
    \end{align*}
}

\variantsplitter

\addpersonalvariant{Владимир Артемчук}

\tasknumber{1}%
\task{%
    Схематично изобразите колебательный контур.
    Запишите формулу для периода колебаний в колебательном контуре и ...
    \begin{itemize}
        \item подпишите все физические величины,
        \item укажите их единицы измерения,
        \item выразите из формулы периода циклическую частоту,
        \item выразите из формулы периода ёмкость конденсатора.
    \end{itemize}
}
\answer{%
    $
        T = 2\pi\sqrt{LC},
        \nu = \frac 1{2\pi\sqrt{LC}},
        \omega = \frac 1{\sqrt{LC}},
        L = \frac 1C \sqr{\frac T{2\pi}},
        C = \frac 1L \sqr{\frac T{2\pi}}.
    $
}
\solutionspace{80pt}

\tasknumber{2}%
\task{%
    Оказалось, что наибольший заряд конденсатора в колебательном контуре равен $80\,\text{мкКл}$,
    а максимальный ток — $120\,\text{мА}$.
    Определите частоту колебаний.
}
\answer{%
    $
        \eli_{\max} = q_{\max}\omega \implies \nu = \frac{\omega}{2\pi} = \frac{\eli_{\max}}{2\pi q} \approx 238{,}7\,\text{Гц}.
    $
}
\solutionspace{80pt}

\tasknumber{3}%
\task{%
    В колебательном контуре сила тока изменяется
    по закону $\eli=0{,}30\sin(12t)$ (в СИ).
    Индуктивность катушки при этом равна $60\,\text{мГн}$.
    Определите:
    \begin{itemize}
        \item период колебаний,
        \item ёмкость конденсатора,
        \item максимальный заряд конденсатора.
    \end{itemize}
}
\answer{%
    \begin{align*}
    \omega &= 12\funits{рад}{c}, \qquad \eli_{\max} = 0{,}30\,\text{A}, \\
    T &= \frac{2\pi}\omega \approx 523{,}6\,\text{мc}, \\
    C &= \frac 1{\omega^2 L} \approx 115{,}7\,\text{мФ}, \\
    q_{\max} &= \frac{\eli_{\max}}\omega  \approx 25\,\text{мКл}.
    \end{align*}
}
\solutionspace{80pt}

\tasknumber{4}%
\task{%
    Электрический колебательный контур состоит
    из катушки индуктивностью $L$ и конденсатора ёмкостью $C$.
    Последовательно конденсатору подключают ещё один конденсатор ёмкостью $\frac13C$.
    Как изменится период свободных колебаний в контуре?
}
\answer{%
    $
        T = 2\pi\sqrt{LC}, \quad
        T' = 2\pi\sqrt{L'C'}
            = T \sqrt{\frac{L'}L \cdot \frac{C'}C}
            = T \sqrt{ 1 \cdot \frac14 }
        \implies \frac{T'}T = \sqrt{ 1 \cdot \frac14 } \approx 0{,}500.
    $
}
\solutionspace{100pt}

\tasknumber{5}%
\task{%
    В колебательном контуре частота собственных колебаний $40\,\text{Гц}$.
    После замены катушки индуктивности на другую катушку частота стала равной $30\,\text{Гц}$.
    А какой станет частота, если в контур установить обе эти катушки параллельно?
}
\answer{%
    \begin{align*}
    T &= 2\pi\sqrt{LC} \implies \nu = \frac 1T = \frac 1{2\pi\sqrt{LC}} \implies L = \frac1 {4\pi^2 \nu^2 C}, \\
    L_1 &= \frac1 {4\pi^2 \nu_1^2 C}, L_2 = \frac1 {4\pi^2 \nu_1^2 C}, \\
    \nu_\text{послед.}
            &= \frac 1{2\pi\sqrt{(L_1 + L_2)C}}
            = \frac 1{2\pi\sqrt{\cbr{\frac1 {4\pi^2 \nu_1^2 C} + \frac1 {4\pi^2 \nu_2^2 C}}C}}
            = \frac 1{\sqrt{\cbr{\frac1 {\nu_1^2 C} + \frac1 {\nu_2^2 C}}C}} =  \\
     &= \frac 1{\sqrt{\frac1 {\nu_1^2} + \frac1 {\nu_2^2}}}
            = \frac 1{\sqrt{ \frac1 {\sqr{40\,\text{Гц}}} + \frac1 {\sqr{30\,\text{Гц}}}}}
            \approx 24\,\text{Гц}, \\
    \nu_\text{паралл.}
            &= \frac 1{2\pi\sqrt{\frac 1{\frac 1{L_1} + \frac 1{L_2}}C}}
            = \frac 1{2\pi\sqrt{\frac 1{\frac 1{\frac1 {4\pi^2 \nu_1^2 C}} + \frac 1{\frac1 {4\pi^2 \nu_2^2 C}}}C}}
            = \frac 1{2\pi\sqrt{\frac 1{4\pi^2 \nu_1^2 C + 4\pi^2 \nu_2^2 C}C}} = \\
     &= \frac 1{\sqrt{\frac 1{\nu_1^2 + \nu_2^2}}}
            = \sqrt{\nu_1^2 + \nu_2^2} = \sqrt{\sqr{40\,\text{Гц}} + \sqr{30\,\text{Гц}}} \approx 50\,\text{Гц}.
    \end{align*}
}

\variantsplitter

\addpersonalvariant{Софья Белянкина}

\tasknumber{1}%
\task{%
    Схематично изобразите колебательный контур.
    Запишите формулу для периода колебаний в колебательном контуре и ...
    \begin{itemize}
        \item подпишите все физические величины,
        \item укажите их единицы измерения,
        \item выразите из формулы периода частоту,
        \item выразите из формулы периода индуктивность катушки.
    \end{itemize}
}
\answer{%
    $
        T = 2\pi\sqrt{LC},
        \nu = \frac 1{2\pi\sqrt{LC}},
        \omega = \frac 1{\sqrt{LC}},
        L = \frac 1C \sqr{\frac T{2\pi}},
        C = \frac 1L \sqr{\frac T{2\pi}}.
    $
}
\solutionspace{80pt}

\tasknumber{2}%
\task{%
    Оказалось, что наибольший заряд конденсатора в колебательном контуре равен $40\,\text{мкКл}$,
    а максимальный ток — $180\,\text{мА}$.
    Определите частоту колебаний.
}
\answer{%
    $
        \eli_{\max} = q_{\max}\omega \implies \nu = \frac{\omega}{2\pi} = \frac{\eli_{\max}}{2\pi q} \approx 716{,}2\,\text{Гц}.
    $
}
\solutionspace{80pt}

\tasknumber{3}%
\task{%
    В колебательном контуре сила тока изменяется
    по закону $\eli=0{,}25\sin(18t)$ (в СИ).
    Индуктивность катушки при этом равна $80\,\text{мГн}$.
    Определите:
    \begin{itemize}
        \item период колебаний,
        \item ёмкость конденсатора,
        \item максимальный заряд конденсатора.
    \end{itemize}
}
\answer{%
    \begin{align*}
    \omega &= 18\funits{рад}{c}, \qquad \eli_{\max} = 0{,}25\,\text{A}, \\
    T &= \frac{2\pi}\omega \approx 349{,}1\,\text{мc}, \\
    C &= \frac 1{\omega^2 L} \approx 38{,}6\,\text{мФ}, \\
    q_{\max} &= \frac{\eli_{\max}}\omega  \approx 13{,}9\,\text{мКл}.
    \end{align*}
}
\solutionspace{80pt}

\tasknumber{4}%
\task{%
    Электрический колебательный контур состоит
    из катушки индуктивностью $L$ и конденсатора ёмкостью $C$.
    Последовательно конденсатору подключают ещё один конденсатор ёмкостью $2C$.
    Как изменится период свободных колебаний в контуре?
}
\answer{%
    $
        T = 2\pi\sqrt{LC}, \quad
        T' = 2\pi\sqrt{L'C'}
            = T \sqrt{\frac{L'}L \cdot \frac{C'}C}
            = T \sqrt{ 1 \cdot \frac23 }
        \implies \frac{T'}T = \sqrt{ 1 \cdot \frac23 } \approx 0{,}816.
    $
}
\solutionspace{100pt}

\tasknumber{5}%
\task{%
    В колебательном контуре частота собственных колебаний $80\,\text{Гц}$.
    После замены катушки индуктивности на другую катушку частота стала равной $30\,\text{Гц}$.
    А какой станет частота, если в контур установить обе эти катушки параллельно?
}
\answer{%
    \begin{align*}
    T &= 2\pi\sqrt{LC} \implies \nu = \frac 1T = \frac 1{2\pi\sqrt{LC}} \implies L = \frac1 {4\pi^2 \nu^2 C}, \\
    L_1 &= \frac1 {4\pi^2 \nu_1^2 C}, L_2 = \frac1 {4\pi^2 \nu_1^2 C}, \\
    \nu_\text{послед.}
            &= \frac 1{2\pi\sqrt{(L_1 + L_2)C}}
            = \frac 1{2\pi\sqrt{\cbr{\frac1 {4\pi^2 \nu_1^2 C} + \frac1 {4\pi^2 \nu_2^2 C}}C}}
            = \frac 1{\sqrt{\cbr{\frac1 {\nu_1^2 C} + \frac1 {\nu_2^2 C}}C}} =  \\
     &= \frac 1{\sqrt{\frac1 {\nu_1^2} + \frac1 {\nu_2^2}}}
            = \frac 1{\sqrt{ \frac1 {\sqr{80\,\text{Гц}}} + \frac1 {\sqr{30\,\text{Гц}}}}}
            \approx 28{,}09\,\text{Гц}, \\
    \nu_\text{паралл.}
            &= \frac 1{2\pi\sqrt{\frac 1{\frac 1{L_1} + \frac 1{L_2}}C}}
            = \frac 1{2\pi\sqrt{\frac 1{\frac 1{\frac1 {4\pi^2 \nu_1^2 C}} + \frac 1{\frac1 {4\pi^2 \nu_2^2 C}}}C}}
            = \frac 1{2\pi\sqrt{\frac 1{4\pi^2 \nu_1^2 C + 4\pi^2 \nu_2^2 C}C}} = \\
     &= \frac 1{\sqrt{\frac 1{\nu_1^2 + \nu_2^2}}}
            = \sqrt{\nu_1^2 + \nu_2^2} = \sqrt{\sqr{80\,\text{Гц}} + \sqr{30\,\text{Гц}}} \approx 85{,}44\,\text{Гц}.
    \end{align*}
}

\variantsplitter

\addpersonalvariant{Варвара Егиазарян}

\tasknumber{1}%
\task{%
    Схематично изобразите колебательный контур.
    Запишите формулу для периода колебаний в колебательном контуре и ...
    \begin{itemize}
        \item подпишите все физические величины,
        \item укажите их единицы измерения,
        \item выразите из формулы периода частоту,
        \item выразите из формулы периода индуктивность катушки.
    \end{itemize}
}
\answer{%
    $
        T = 2\pi\sqrt{LC},
        \nu = \frac 1{2\pi\sqrt{LC}},
        \omega = \frac 1{\sqrt{LC}},
        L = \frac 1C \sqr{\frac T{2\pi}},
        C = \frac 1L \sqr{\frac T{2\pi}}.
    $
}
\solutionspace{80pt}

\tasknumber{2}%
\task{%
    Оказалось, что наибольший заряд конденсатора в колебательном контуре равен $60\,\text{мкКл}$,
    а максимальный ток — $270\,\text{мА}$.
    Определите частоту колебаний.
}
\answer{%
    $
        \eli_{\max} = q_{\max}\omega \implies \nu = \frac{\omega}{2\pi} = \frac{\eli_{\max}}{2\pi q} \approx 716{,}2\,\text{Гц}.
    $
}
\solutionspace{80pt}

\tasknumber{3}%
\task{%
    В колебательном контуре сила тока изменяется
    по закону $\eli=0{,}25\sin(12t)$ (в СИ).
    Индуктивность катушки при этом равна $50\,\text{мГн}$.
    Определите:
    \begin{itemize}
        \item период колебаний,
        \item ёмкость конденсатора,
        \item максимальный заряд конденсатора.
    \end{itemize}
}
\answer{%
    \begin{align*}
    \omega &= 12\funits{рад}{c}, \qquad \eli_{\max} = 0{,}25\,\text{A}, \\
    T &= \frac{2\pi}\omega \approx 523{,}6\,\text{мc}, \\
    C &= \frac 1{\omega^2 L} \approx 138{,}9\,\text{мФ}, \\
    q_{\max} &= \frac{\eli_{\max}}\omega  \approx 20{,}8\,\text{мКл}.
    \end{align*}
}
\solutionspace{80pt}

\tasknumber{4}%
\task{%
    Электрический колебательный контур состоит
    из катушки индуктивностью $L$ и конденсатора ёмкостью $C$.
    Параллельно конденсатору подключают ещё один конденсатор ёмкостью $2C$.
    Как изменится период свободных колебаний в контуре?
}
\answer{%
    $
        T = 2\pi\sqrt{LC}, \quad
        T' = 2\pi\sqrt{L'C'}
            = T \sqrt{\frac{L'}L \cdot \frac{C'}C}
            = T \sqrt{ 1 \cdot 3 }
        \implies \frac{T'}T = \sqrt{ 1 \cdot 3 } \approx 1{,}732.
    $
}
\solutionspace{100pt}

\tasknumber{5}%
\task{%
    В колебательном контуре частота собственных колебаний $80\,\text{Гц}$.
    После замены катушки индуктивности на другую катушку частота стала равной $30\,\text{Гц}$.
    А какой станет частота, если в контур установить обе эти катушки последовательно?
}
\answer{%
    \begin{align*}
    T &= 2\pi\sqrt{LC} \implies \nu = \frac 1T = \frac 1{2\pi\sqrt{LC}} \implies L = \frac1 {4\pi^2 \nu^2 C}, \\
    L_1 &= \frac1 {4\pi^2 \nu_1^2 C}, L_2 = \frac1 {4\pi^2 \nu_1^2 C}, \\
    \nu_\text{послед.}
            &= \frac 1{2\pi\sqrt{(L_1 + L_2)C}}
            = \frac 1{2\pi\sqrt{\cbr{\frac1 {4\pi^2 \nu_1^2 C} + \frac1 {4\pi^2 \nu_2^2 C}}C}}
            = \frac 1{\sqrt{\cbr{\frac1 {\nu_1^2 C} + \frac1 {\nu_2^2 C}}C}} =  \\
     &= \frac 1{\sqrt{\frac1 {\nu_1^2} + \frac1 {\nu_2^2}}}
            = \frac 1{\sqrt{ \frac1 {\sqr{80\,\text{Гц}}} + \frac1 {\sqr{30\,\text{Гц}}}}}
            \approx 28{,}09\,\text{Гц}, \\
    \nu_\text{паралл.}
            &= \frac 1{2\pi\sqrt{\frac 1{\frac 1{L_1} + \frac 1{L_2}}C}}
            = \frac 1{2\pi\sqrt{\frac 1{\frac 1{\frac1 {4\pi^2 \nu_1^2 C}} + \frac 1{\frac1 {4\pi^2 \nu_2^2 C}}}C}}
            = \frac 1{2\pi\sqrt{\frac 1{4\pi^2 \nu_1^2 C + 4\pi^2 \nu_2^2 C}C}} = \\
     &= \frac 1{\sqrt{\frac 1{\nu_1^2 + \nu_2^2}}}
            = \sqrt{\nu_1^2 + \nu_2^2} = \sqrt{\sqr{80\,\text{Гц}} + \sqr{30\,\text{Гц}}} \approx 85{,}44\,\text{Гц}.
    \end{align*}
}

\variantsplitter

\addpersonalvariant{Владислав Емелин}

\tasknumber{1}%
\task{%
    Схематично изобразите колебательный контур.
    Запишите формулу для периода колебаний в колебательном контуре и ...
    \begin{itemize}
        \item подпишите все физические величины,
        \item укажите их единицы измерения,
        \item выразите из формулы периода циклическую частоту,
        \item выразите из формулы периода ёмкость конденсатора.
    \end{itemize}
}
\answer{%
    $
        T = 2\pi\sqrt{LC},
        \nu = \frac 1{2\pi\sqrt{LC}},
        \omega = \frac 1{\sqrt{LC}},
        L = \frac 1C \sqr{\frac T{2\pi}},
        C = \frac 1L \sqr{\frac T{2\pi}}.
    $
}
\solutionspace{80pt}

\tasknumber{2}%
\task{%
    Оказалось, что наибольший заряд конденсатора в колебательном контуре равен $60\,\text{мкКл}$,
    а максимальный ток — $270\,\text{мА}$.
    Определите частоту колебаний.
}
\answer{%
    $
        \eli_{\max} = q_{\max}\omega \implies \nu = \frac{\omega}{2\pi} = \frac{\eli_{\max}}{2\pi q} \approx 716{,}2\,\text{Гц}.
    $
}
\solutionspace{80pt}

\tasknumber{3}%
\task{%
    В колебательном контуре сила тока изменяется
    по закону $\eli=0{,}25\sin(15t)$ (в СИ).
    Индуктивность катушки при этом равна $50\,\text{мГн}$.
    Определите:
    \begin{itemize}
        \item период колебаний,
        \item ёмкость конденсатора,
        \item максимальный заряд конденсатора.
    \end{itemize}
}
\answer{%
    \begin{align*}
    \omega &= 15\funits{рад}{c}, \qquad \eli_{\max} = 0{,}25\,\text{A}, \\
    T &= \frac{2\pi}\omega \approx 418{,}9\,\text{мc}, \\
    C &= \frac 1{\omega^2 L} \approx 88{,}9\,\text{мФ}, \\
    q_{\max} &= \frac{\eli_{\max}}\omega  \approx 16{,}7\,\text{мКл}.
    \end{align*}
}
\solutionspace{80pt}

\tasknumber{4}%
\task{%
    Электрический колебательный контур состоит
    из катушки индуктивностью $L$ и конденсатора ёмкостью $C$.
    Последовательно катушке подключают ещё одну катушку индуктивностью $\frac13L$.
    Как изменится период свободных колебаний в контуре?
}
\answer{%
    $
        T = 2\pi\sqrt{LC}, \quad
        T' = 2\pi\sqrt{L'C'}
            = T \sqrt{\frac{L'}L \cdot \frac{C'}C}
            = T \sqrt{ \frac43 \cdot 1 }
        \implies \frac{T'}T = \sqrt{ \frac43 \cdot 1 } \approx 1{,}155.
    $
}
\solutionspace{100pt}

\tasknumber{5}%
\task{%
    В колебательном контуре частота собственных колебаний $60\,\text{Гц}$.
    После замены катушки индуктивности на другую катушку частота стала равной $70\,\text{Гц}$.
    А какой станет частота, если в контур установить обе эти катушки параллельно?
}
\answer{%
    \begin{align*}
    T &= 2\pi\sqrt{LC} \implies \nu = \frac 1T = \frac 1{2\pi\sqrt{LC}} \implies L = \frac1 {4\pi^2 \nu^2 C}, \\
    L_1 &= \frac1 {4\pi^2 \nu_1^2 C}, L_2 = \frac1 {4\pi^2 \nu_1^2 C}, \\
    \nu_\text{послед.}
            &= \frac 1{2\pi\sqrt{(L_1 + L_2)C}}
            = \frac 1{2\pi\sqrt{\cbr{\frac1 {4\pi^2 \nu_1^2 C} + \frac1 {4\pi^2 \nu_2^2 C}}C}}
            = \frac 1{\sqrt{\cbr{\frac1 {\nu_1^2 C} + \frac1 {\nu_2^2 C}}C}} =  \\
     &= \frac 1{\sqrt{\frac1 {\nu_1^2} + \frac1 {\nu_2^2}}}
            = \frac 1{\sqrt{ \frac1 {\sqr{60\,\text{Гц}}} + \frac1 {\sqr{70\,\text{Гц}}}}}
            \approx 45{,}56\,\text{Гц}, \\
    \nu_\text{паралл.}
            &= \frac 1{2\pi\sqrt{\frac 1{\frac 1{L_1} + \frac 1{L_2}}C}}
            = \frac 1{2\pi\sqrt{\frac 1{\frac 1{\frac1 {4\pi^2 \nu_1^2 C}} + \frac 1{\frac1 {4\pi^2 \nu_2^2 C}}}C}}
            = \frac 1{2\pi\sqrt{\frac 1{4\pi^2 \nu_1^2 C + 4\pi^2 \nu_2^2 C}C}} = \\
     &= \frac 1{\sqrt{\frac 1{\nu_1^2 + \nu_2^2}}}
            = \sqrt{\nu_1^2 + \nu_2^2} = \sqrt{\sqr{60\,\text{Гц}} + \sqr{70\,\text{Гц}}} \approx 92{,}20\,\text{Гц}.
    \end{align*}
}

\variantsplitter

\addpersonalvariant{Артём Жичин}

\tasknumber{1}%
\task{%
    Схематично изобразите колебательный контур.
    Запишите формулу для периода колебаний в колебательном контуре и ...
    \begin{itemize}
        \item подпишите все физические величины,
        \item укажите их единицы измерения,
        \item выразите из формулы периода частоту,
        \item выразите из формулы периода индуктивность катушки.
    \end{itemize}
}
\answer{%
    $
        T = 2\pi\sqrt{LC},
        \nu = \frac 1{2\pi\sqrt{LC}},
        \omega = \frac 1{\sqrt{LC}},
        L = \frac 1C \sqr{\frac T{2\pi}},
        C = \frac 1L \sqr{\frac T{2\pi}}.
    $
}
\solutionspace{80pt}

\tasknumber{2}%
\task{%
    Оказалось, что наибольший заряд конденсатора в колебательном контуре равен $80\,\text{мкКл}$,
    а максимальный ток — $120\,\text{мА}$.
    Определите частоту колебаний.
}
\answer{%
    $
        \eli_{\max} = q_{\max}\omega \implies \nu = \frac{\omega}{2\pi} = \frac{\eli_{\max}}{2\pi q} \approx 238{,}7\,\text{Гц}.
    $
}
\solutionspace{80pt}

\tasknumber{3}%
\task{%
    В колебательном контуре сила тока изменяется
    по закону $\eli=0{,}30\cos(18t)$ (в СИ).
    Индуктивность катушки при этом равна $50\,\text{мГн}$.
    Определите:
    \begin{itemize}
        \item период колебаний,
        \item ёмкость конденсатора,
        \item максимальный заряд конденсатора.
    \end{itemize}
}
\answer{%
    \begin{align*}
    \omega &= 18\funits{рад}{c}, \qquad \eli_{\max} = 0{,}30\,\text{A}, \\
    T &= \frac{2\pi}\omega \approx 349{,}1\,\text{мc}, \\
    C &= \frac 1{\omega^2 L} \approx 61{,}7\,\text{мФ}, \\
    q_{\max} &= \frac{\eli_{\max}}\omega  \approx 16{,}7\,\text{мКл}.
    \end{align*}
}
\solutionspace{80pt}

\tasknumber{4}%
\task{%
    Электрический колебательный контур состоит
    из катушки индуктивностью $L$ и конденсатора ёмкостью $C$.
    Последовательно конденсатору подключают ещё один конденсатор ёмкостью $\frac13C$.
    Как изменится период свободных колебаний в контуре?
}
\answer{%
    $
        T = 2\pi\sqrt{LC}, \quad
        T' = 2\pi\sqrt{L'C'}
            = T \sqrt{\frac{L'}L \cdot \frac{C'}C}
            = T \sqrt{ 1 \cdot \frac14 }
        \implies \frac{T'}T = \sqrt{ 1 \cdot \frac14 } \approx 0{,}500.
    $
}
\solutionspace{100pt}

\tasknumber{5}%
\task{%
    В колебательном контуре частота собственных колебаний $60\,\text{Гц}$.
    После замены катушки индуктивности на другую катушку частота стала равной $30\,\text{Гц}$.
    А какой станет частота, если в контур установить обе эти катушки параллельно?
}
\answer{%
    \begin{align*}
    T &= 2\pi\sqrt{LC} \implies \nu = \frac 1T = \frac 1{2\pi\sqrt{LC}} \implies L = \frac1 {4\pi^2 \nu^2 C}, \\
    L_1 &= \frac1 {4\pi^2 \nu_1^2 C}, L_2 = \frac1 {4\pi^2 \nu_1^2 C}, \\
    \nu_\text{послед.}
            &= \frac 1{2\pi\sqrt{(L_1 + L_2)C}}
            = \frac 1{2\pi\sqrt{\cbr{\frac1 {4\pi^2 \nu_1^2 C} + \frac1 {4\pi^2 \nu_2^2 C}}C}}
            = \frac 1{\sqrt{\cbr{\frac1 {\nu_1^2 C} + \frac1 {\nu_2^2 C}}C}} =  \\
     &= \frac 1{\sqrt{\frac1 {\nu_1^2} + \frac1 {\nu_2^2}}}
            = \frac 1{\sqrt{ \frac1 {\sqr{60\,\text{Гц}}} + \frac1 {\sqr{30\,\text{Гц}}}}}
            \approx 26{,}83\,\text{Гц}, \\
    \nu_\text{паралл.}
            &= \frac 1{2\pi\sqrt{\frac 1{\frac 1{L_1} + \frac 1{L_2}}C}}
            = \frac 1{2\pi\sqrt{\frac 1{\frac 1{\frac1 {4\pi^2 \nu_1^2 C}} + \frac 1{\frac1 {4\pi^2 \nu_2^2 C}}}C}}
            = \frac 1{2\pi\sqrt{\frac 1{4\pi^2 \nu_1^2 C + 4\pi^2 \nu_2^2 C}C}} = \\
     &= \frac 1{\sqrt{\frac 1{\nu_1^2 + \nu_2^2}}}
            = \sqrt{\nu_1^2 + \nu_2^2} = \sqrt{\sqr{60\,\text{Гц}} + \sqr{30\,\text{Гц}}} \approx 67{,}08\,\text{Гц}.
    \end{align*}
}

\variantsplitter

\addpersonalvariant{Дарья Кошман}

\tasknumber{1}%
\task{%
    Схематично изобразите колебательный контур.
    Запишите формулу для периода колебаний в колебательном контуре и ...
    \begin{itemize}
        \item подпишите все физические величины,
        \item укажите их единицы измерения,
        \item выразите из формулы периода частоту,
        \item выразите из формулы периода индуктивность катушки.
    \end{itemize}
}
\answer{%
    $
        T = 2\pi\sqrt{LC},
        \nu = \frac 1{2\pi\sqrt{LC}},
        \omega = \frac 1{\sqrt{LC}},
        L = \frac 1C \sqr{\frac T{2\pi}},
        C = \frac 1L \sqr{\frac T{2\pi}}.
    $
}
\solutionspace{80pt}

\tasknumber{2}%
\task{%
    Оказалось, что наибольший заряд конденсатора в колебательном контуре равен $60\,\text{мкКл}$,
    а максимальный ток — $180\,\text{мА}$.
    Определите частоту колебаний.
}
\answer{%
    $
        \eli_{\max} = q_{\max}\omega \implies \nu = \frac{\omega}{2\pi} = \frac{\eli_{\max}}{2\pi q} \approx 477{,}5\,\text{Гц}.
    $
}
\solutionspace{80pt}

\tasknumber{3}%
\task{%
    В колебательном контуре сила тока изменяется
    по закону $\eli=0{,}30\cos(18t)$ (в СИ).
    Индуктивность катушки при этом равна $80\,\text{мГн}$.
    Определите:
    \begin{itemize}
        \item период колебаний,
        \item ёмкость конденсатора,
        \item максимальный заряд конденсатора.
    \end{itemize}
}
\answer{%
    \begin{align*}
    \omega &= 18\funits{рад}{c}, \qquad \eli_{\max} = 0{,}30\,\text{A}, \\
    T &= \frac{2\pi}\omega \approx 349{,}1\,\text{мc}, \\
    C &= \frac 1{\omega^2 L} \approx 38{,}6\,\text{мФ}, \\
    q_{\max} &= \frac{\eli_{\max}}\omega  \approx 16{,}7\,\text{мКл}.
    \end{align*}
}
\solutionspace{80pt}

\tasknumber{4}%
\task{%
    Электрический колебательный контур состоит
    из катушки индуктивностью $L$ и конденсатора ёмкостью $C$.
    Последовательно катушке подключают ещё одну катушку индуктивностью $\frac13L$.
    Как изменится период свободных колебаний в контуре?
}
\answer{%
    $
        T = 2\pi\sqrt{LC}, \quad
        T' = 2\pi\sqrt{L'C'}
            = T \sqrt{\frac{L'}L \cdot \frac{C'}C}
            = T \sqrt{ \frac43 \cdot 1 }
        \implies \frac{T'}T = \sqrt{ \frac43 \cdot 1 } \approx 1{,}155.
    $
}
\solutionspace{100pt}

\tasknumber{5}%
\task{%
    В колебательном контуре частота собственных колебаний $40\,\text{Гц}$.
    После замены катушки индуктивности на другую катушку частота стала равной $30\,\text{Гц}$.
    А какой станет частота, если в контур установить обе эти катушки последовательно?
}
\answer{%
    \begin{align*}
    T &= 2\pi\sqrt{LC} \implies \nu = \frac 1T = \frac 1{2\pi\sqrt{LC}} \implies L = \frac1 {4\pi^2 \nu^2 C}, \\
    L_1 &= \frac1 {4\pi^2 \nu_1^2 C}, L_2 = \frac1 {4\pi^2 \nu_1^2 C}, \\
    \nu_\text{послед.}
            &= \frac 1{2\pi\sqrt{(L_1 + L_2)C}}
            = \frac 1{2\pi\sqrt{\cbr{\frac1 {4\pi^2 \nu_1^2 C} + \frac1 {4\pi^2 \nu_2^2 C}}C}}
            = \frac 1{\sqrt{\cbr{\frac1 {\nu_1^2 C} + \frac1 {\nu_2^2 C}}C}} =  \\
     &= \frac 1{\sqrt{\frac1 {\nu_1^2} + \frac1 {\nu_2^2}}}
            = \frac 1{\sqrt{ \frac1 {\sqr{40\,\text{Гц}}} + \frac1 {\sqr{30\,\text{Гц}}}}}
            \approx 24\,\text{Гц}, \\
    \nu_\text{паралл.}
            &= \frac 1{2\pi\sqrt{\frac 1{\frac 1{L_1} + \frac 1{L_2}}C}}
            = \frac 1{2\pi\sqrt{\frac 1{\frac 1{\frac1 {4\pi^2 \nu_1^2 C}} + \frac 1{\frac1 {4\pi^2 \nu_2^2 C}}}C}}
            = \frac 1{2\pi\sqrt{\frac 1{4\pi^2 \nu_1^2 C + 4\pi^2 \nu_2^2 C}C}} = \\
     &= \frac 1{\sqrt{\frac 1{\nu_1^2 + \nu_2^2}}}
            = \sqrt{\nu_1^2 + \nu_2^2} = \sqrt{\sqr{40\,\text{Гц}} + \sqr{30\,\text{Гц}}} \approx 50\,\text{Гц}.
    \end{align*}
}

\variantsplitter

\addpersonalvariant{Анна Кузьмичёва}

\tasknumber{1}%
\task{%
    Схематично изобразите колебательный контур.
    Запишите формулу для периода колебаний в колебательном контуре и ...
    \begin{itemize}
        \item подпишите все физические величины,
        \item укажите их единицы измерения,
        \item выразите из формулы периода частоту,
        \item выразите из формулы периода ёмкость конденсатора.
    \end{itemize}
}
\answer{%
    $
        T = 2\pi\sqrt{LC},
        \nu = \frac 1{2\pi\sqrt{LC}},
        \omega = \frac 1{\sqrt{LC}},
        L = \frac 1C \sqr{\frac T{2\pi}},
        C = \frac 1L \sqr{\frac T{2\pi}}.
    $
}
\solutionspace{80pt}

\tasknumber{2}%
\task{%
    Оказалось, что наибольший заряд конденсатора в колебательном контуре равен $60\,\text{мкКл}$,
    а максимальный ток — $240\,\text{мА}$.
    Определите частоту колебаний.
}
\answer{%
    $
        \eli_{\max} = q_{\max}\omega \implies \nu = \frac{\omega}{2\pi} = \frac{\eli_{\max}}{2\pi q} \approx 636{,}6\,\text{Гц}.
    $
}
\solutionspace{80pt}

\tasknumber{3}%
\task{%
    В колебательном контуре сила тока изменяется
    по закону $\eli=0{,}25\cos(15t)$ (в СИ).
    Индуктивность катушки при этом равна $50\,\text{мГн}$.
    Определите:
    \begin{itemize}
        \item период колебаний,
        \item ёмкость конденсатора,
        \item максимальный заряд конденсатора.
    \end{itemize}
}
\answer{%
    \begin{align*}
    \omega &= 15\funits{рад}{c}, \qquad \eli_{\max} = 0{,}25\,\text{A}, \\
    T &= \frac{2\pi}\omega \approx 418{,}9\,\text{мc}, \\
    C &= \frac 1{\omega^2 L} \approx 88{,}9\,\text{мФ}, \\
    q_{\max} &= \frac{\eli_{\max}}\omega  \approx 16{,}7\,\text{мКл}.
    \end{align*}
}
\solutionspace{80pt}

\tasknumber{4}%
\task{%
    Электрический колебательный контур состоит
    из катушки индуктивностью $L$ и конденсатора ёмкостью $C$.
    Последовательно конденсатору подключают ещё один конденсатор ёмкостью $\frac13C$.
    Как изменится период свободных колебаний в контуре?
}
\answer{%
    $
        T = 2\pi\sqrt{LC}, \quad
        T' = 2\pi\sqrt{L'C'}
            = T \sqrt{\frac{L'}L \cdot \frac{C'}C}
            = T \sqrt{ 1 \cdot \frac14 }
        \implies \frac{T'}T = \sqrt{ 1 \cdot \frac14 } \approx 0{,}500.
    $
}
\solutionspace{100pt}

\tasknumber{5}%
\task{%
    В колебательном контуре частота собственных колебаний $60\,\text{Гц}$.
    После замены катушки индуктивности на другую катушку частота стала равной $70\,\text{Гц}$.
    А какой станет частота, если в контур установить обе эти катушки последовательно?
}
\answer{%
    \begin{align*}
    T &= 2\pi\sqrt{LC} \implies \nu = \frac 1T = \frac 1{2\pi\sqrt{LC}} \implies L = \frac1 {4\pi^2 \nu^2 C}, \\
    L_1 &= \frac1 {4\pi^2 \nu_1^2 C}, L_2 = \frac1 {4\pi^2 \nu_1^2 C}, \\
    \nu_\text{послед.}
            &= \frac 1{2\pi\sqrt{(L_1 + L_2)C}}
            = \frac 1{2\pi\sqrt{\cbr{\frac1 {4\pi^2 \nu_1^2 C} + \frac1 {4\pi^2 \nu_2^2 C}}C}}
            = \frac 1{\sqrt{\cbr{\frac1 {\nu_1^2 C} + \frac1 {\nu_2^2 C}}C}} =  \\
     &= \frac 1{\sqrt{\frac1 {\nu_1^2} + \frac1 {\nu_2^2}}}
            = \frac 1{\sqrt{ \frac1 {\sqr{60\,\text{Гц}}} + \frac1 {\sqr{70\,\text{Гц}}}}}
            \approx 45{,}56\,\text{Гц}, \\
    \nu_\text{паралл.}
            &= \frac 1{2\pi\sqrt{\frac 1{\frac 1{L_1} + \frac 1{L_2}}C}}
            = \frac 1{2\pi\sqrt{\frac 1{\frac 1{\frac1 {4\pi^2 \nu_1^2 C}} + \frac 1{\frac1 {4\pi^2 \nu_2^2 C}}}C}}
            = \frac 1{2\pi\sqrt{\frac 1{4\pi^2 \nu_1^2 C + 4\pi^2 \nu_2^2 C}C}} = \\
     &= \frac 1{\sqrt{\frac 1{\nu_1^2 + \nu_2^2}}}
            = \sqrt{\nu_1^2 + \nu_2^2} = \sqrt{\sqr{60\,\text{Гц}} + \sqr{70\,\text{Гц}}} \approx 92{,}20\,\text{Гц}.
    \end{align*}
}

\variantsplitter

\addpersonalvariant{Алёна Куприянова}

\tasknumber{1}%
\task{%
    Схематично изобразите колебательный контур.
    Запишите формулу для периода колебаний в колебательном контуре и ...
    \begin{itemize}
        \item подпишите все физические величины,
        \item укажите их единицы измерения,
        \item выразите из формулы периода частоту,
        \item выразите из формулы периода индуктивность катушки.
    \end{itemize}
}
\answer{%
    $
        T = 2\pi\sqrt{LC},
        \nu = \frac 1{2\pi\sqrt{LC}},
        \omega = \frac 1{\sqrt{LC}},
        L = \frac 1C \sqr{\frac T{2\pi}},
        C = \frac 1L \sqr{\frac T{2\pi}}.
    $
}
\solutionspace{80pt}

\tasknumber{2}%
\task{%
    Оказалось, что наибольший заряд конденсатора в колебательном контуре равен $80\,\text{мкКл}$,
    а максимальный ток — $180\,\text{мА}$.
    Определите частоту колебаний.
}
\answer{%
    $
        \eli_{\max} = q_{\max}\omega \implies \nu = \frac{\omega}{2\pi} = \frac{\eli_{\max}}{2\pi q} \approx 358{,}1\,\text{Гц}.
    $
}
\solutionspace{80pt}

\tasknumber{3}%
\task{%
    В колебательном контуре сила тока изменяется
    по закону $\eli=0{,}30\cos(12t)$ (в СИ).
    Индуктивность катушки при этом равна $50\,\text{мГн}$.
    Определите:
    \begin{itemize}
        \item период колебаний,
        \item ёмкость конденсатора,
        \item максимальный заряд конденсатора.
    \end{itemize}
}
\answer{%
    \begin{align*}
    \omega &= 12\funits{рад}{c}, \qquad \eli_{\max} = 0{,}30\,\text{A}, \\
    T &= \frac{2\pi}\omega \approx 523{,}6\,\text{мc}, \\
    C &= \frac 1{\omega^2 L} \approx 138{,}9\,\text{мФ}, \\
    q_{\max} &= \frac{\eli_{\max}}\omega  \approx 25\,\text{мКл}.
    \end{align*}
}
\solutionspace{80pt}

\tasknumber{4}%
\task{%
    Электрический колебательный контур состоит
    из катушки индуктивностью $L$ и конденсатора ёмкостью $C$.
    Параллельно катушке подключают ещё одну катушку индуктивностью $2L$.
    Как изменится период свободных колебаний в контуре?
}
\answer{%
    $
        T = 2\pi\sqrt{LC}, \quad
        T' = 2\pi\sqrt{L'C'}
            = T \sqrt{\frac{L'}L \cdot \frac{C'}C}
            = T \sqrt{ \frac23 \cdot 1 }
        \implies \frac{T'}T = \sqrt{ \frac23 \cdot 1 } \approx 0{,}816.
    $
}
\solutionspace{100pt}

\tasknumber{5}%
\task{%
    В колебательном контуре частота собственных колебаний $60\,\text{Гц}$.
    После замены катушки индуктивности на другую катушку частота стала равной $70\,\text{Гц}$.
    А какой станет частота, если в контур установить обе эти катушки последовательно?
}
\answer{%
    \begin{align*}
    T &= 2\pi\sqrt{LC} \implies \nu = \frac 1T = \frac 1{2\pi\sqrt{LC}} \implies L = \frac1 {4\pi^2 \nu^2 C}, \\
    L_1 &= \frac1 {4\pi^2 \nu_1^2 C}, L_2 = \frac1 {4\pi^2 \nu_1^2 C}, \\
    \nu_\text{послед.}
            &= \frac 1{2\pi\sqrt{(L_1 + L_2)C}}
            = \frac 1{2\pi\sqrt{\cbr{\frac1 {4\pi^2 \nu_1^2 C} + \frac1 {4\pi^2 \nu_2^2 C}}C}}
            = \frac 1{\sqrt{\cbr{\frac1 {\nu_1^2 C} + \frac1 {\nu_2^2 C}}C}} =  \\
     &= \frac 1{\sqrt{\frac1 {\nu_1^2} + \frac1 {\nu_2^2}}}
            = \frac 1{\sqrt{ \frac1 {\sqr{60\,\text{Гц}}} + \frac1 {\sqr{70\,\text{Гц}}}}}
            \approx 45{,}56\,\text{Гц}, \\
    \nu_\text{паралл.}
            &= \frac 1{2\pi\sqrt{\frac 1{\frac 1{L_1} + \frac 1{L_2}}C}}
            = \frac 1{2\pi\sqrt{\frac 1{\frac 1{\frac1 {4\pi^2 \nu_1^2 C}} + \frac 1{\frac1 {4\pi^2 \nu_2^2 C}}}C}}
            = \frac 1{2\pi\sqrt{\frac 1{4\pi^2 \nu_1^2 C + 4\pi^2 \nu_2^2 C}C}} = \\
     &= \frac 1{\sqrt{\frac 1{\nu_1^2 + \nu_2^2}}}
            = \sqrt{\nu_1^2 + \nu_2^2} = \sqrt{\sqr{60\,\text{Гц}} + \sqr{70\,\text{Гц}}} \approx 92{,}20\,\text{Гц}.
    \end{align*}
}

\variantsplitter

\addpersonalvariant{Ярослав Лавровский}

\tasknumber{1}%
\task{%
    Схематично изобразите колебательный контур.
    Запишите формулу для периода колебаний в колебательном контуре и ...
    \begin{itemize}
        \item подпишите все физические величины,
        \item укажите их единицы измерения,
        \item выразите из формулы периода циклическую частоту,
        \item выразите из формулы периода ёмкость конденсатора.
    \end{itemize}
}
\answer{%
    $
        T = 2\pi\sqrt{LC},
        \nu = \frac 1{2\pi\sqrt{LC}},
        \omega = \frac 1{\sqrt{LC}},
        L = \frac 1C \sqr{\frac T{2\pi}},
        C = \frac 1L \sqr{\frac T{2\pi}}.
    $
}
\solutionspace{80pt}

\tasknumber{2}%
\task{%
    Оказалось, что наибольший заряд конденсатора в колебательном контуре равен $40\,\text{мкКл}$,
    а максимальный ток — $270\,\text{мА}$.
    Определите частоту колебаний.
}
\answer{%
    $
        \eli_{\max} = q_{\max}\omega \implies \nu = \frac{\omega}{2\pi} = \frac{\eli_{\max}}{2\pi q} \approx 1074{,}3\,\text{Гц}.
    $
}
\solutionspace{80pt}

\tasknumber{3}%
\task{%
    В колебательном контуре сила тока изменяется
    по закону $\eli=0{,}25\cos(18t)$ (в СИ).
    Индуктивность катушки при этом равна $80\,\text{мГн}$.
    Определите:
    \begin{itemize}
        \item период колебаний,
        \item ёмкость конденсатора,
        \item максимальный заряд конденсатора.
    \end{itemize}
}
\answer{%
    \begin{align*}
    \omega &= 18\funits{рад}{c}, \qquad \eli_{\max} = 0{,}25\,\text{A}, \\
    T &= \frac{2\pi}\omega \approx 349{,}1\,\text{мc}, \\
    C &= \frac 1{\omega^2 L} \approx 38{,}6\,\text{мФ}, \\
    q_{\max} &= \frac{\eli_{\max}}\omega  \approx 13{,}9\,\text{мКл}.
    \end{align*}
}
\solutionspace{80pt}

\tasknumber{4}%
\task{%
    Электрический колебательный контур состоит
    из катушки индуктивностью $L$ и конденсатора ёмкостью $C$.
    Параллельно катушке подключают ещё одну катушку индуктивностью $3L$.
    Как изменится период свободных колебаний в контуре?
}
\answer{%
    $
        T = 2\pi\sqrt{LC}, \quad
        T' = 2\pi\sqrt{L'C'}
            = T \sqrt{\frac{L'}L \cdot \frac{C'}C}
            = T \sqrt{ \frac34 \cdot 1 }
        \implies \frac{T'}T = \sqrt{ \frac34 \cdot 1 } \approx 0{,}866.
    $
}
\solutionspace{100pt}

\tasknumber{5}%
\task{%
    В колебательном контуре частота собственных колебаний $60\,\text{Гц}$.
    После замены катушки индуктивности на другую катушку частота стала равной $90\,\text{Гц}$.
    А какой станет частота, если в контур установить обе эти катушки параллельно?
}
\answer{%
    \begin{align*}
    T &= 2\pi\sqrt{LC} \implies \nu = \frac 1T = \frac 1{2\pi\sqrt{LC}} \implies L = \frac1 {4\pi^2 \nu^2 C}, \\
    L_1 &= \frac1 {4\pi^2 \nu_1^2 C}, L_2 = \frac1 {4\pi^2 \nu_1^2 C}, \\
    \nu_\text{послед.}
            &= \frac 1{2\pi\sqrt{(L_1 + L_2)C}}
            = \frac 1{2\pi\sqrt{\cbr{\frac1 {4\pi^2 \nu_1^2 C} + \frac1 {4\pi^2 \nu_2^2 C}}C}}
            = \frac 1{\sqrt{\cbr{\frac1 {\nu_1^2 C} + \frac1 {\nu_2^2 C}}C}} =  \\
     &= \frac 1{\sqrt{\frac1 {\nu_1^2} + \frac1 {\nu_2^2}}}
            = \frac 1{\sqrt{ \frac1 {\sqr{60\,\text{Гц}}} + \frac1 {\sqr{90\,\text{Гц}}}}}
            \approx 49{,}92\,\text{Гц}, \\
    \nu_\text{паралл.}
            &= \frac 1{2\pi\sqrt{\frac 1{\frac 1{L_1} + \frac 1{L_2}}C}}
            = \frac 1{2\pi\sqrt{\frac 1{\frac 1{\frac1 {4\pi^2 \nu_1^2 C}} + \frac 1{\frac1 {4\pi^2 \nu_2^2 C}}}C}}
            = \frac 1{2\pi\sqrt{\frac 1{4\pi^2 \nu_1^2 C + 4\pi^2 \nu_2^2 C}C}} = \\
     &= \frac 1{\sqrt{\frac 1{\nu_1^2 + \nu_2^2}}}
            = \sqrt{\nu_1^2 + \nu_2^2} = \sqrt{\sqr{60\,\text{Гц}} + \sqr{90\,\text{Гц}}} \approx 108{,}17\,\text{Гц}.
    \end{align*}
}

\variantsplitter

\addpersonalvariant{Анастасия Ламанова}

\tasknumber{1}%
\task{%
    Схематично изобразите колебательный контур.
    Запишите формулу для периода колебаний в колебательном контуре и ...
    \begin{itemize}
        \item подпишите все физические величины,
        \item укажите их единицы измерения,
        \item выразите из формулы периода частоту,
        \item выразите из формулы периода индуктивность катушки.
    \end{itemize}
}
\answer{%
    $
        T = 2\pi\sqrt{LC},
        \nu = \frac 1{2\pi\sqrt{LC}},
        \omega = \frac 1{\sqrt{LC}},
        L = \frac 1C \sqr{\frac T{2\pi}},
        C = \frac 1L \sqr{\frac T{2\pi}}.
    $
}
\solutionspace{80pt}

\tasknumber{2}%
\task{%
    Оказалось, что наибольший заряд конденсатора в колебательном контуре равен $40\,\text{мкКл}$,
    а максимальный ток — $120\,\text{мА}$.
    Определите частоту колебаний.
}
\answer{%
    $
        \eli_{\max} = q_{\max}\omega \implies \nu = \frac{\omega}{2\pi} = \frac{\eli_{\max}}{2\pi q} \approx 477{,}5\,\text{Гц}.
    $
}
\solutionspace{80pt}

\tasknumber{3}%
\task{%
    В колебательном контуре сила тока изменяется
    по закону $\eli=0{,}30\cos(12t)$ (в СИ).
    Индуктивность катушки при этом равна $70\,\text{мГн}$.
    Определите:
    \begin{itemize}
        \item период колебаний,
        \item ёмкость конденсатора,
        \item максимальный заряд конденсатора.
    \end{itemize}
}
\answer{%
    \begin{align*}
    \omega &= 12\funits{рад}{c}, \qquad \eli_{\max} = 0{,}30\,\text{A}, \\
    T &= \frac{2\pi}\omega \approx 523{,}6\,\text{мc}, \\
    C &= \frac 1{\omega^2 L} \approx 99{,}2\,\text{мФ}, \\
    q_{\max} &= \frac{\eli_{\max}}\omega  \approx 25\,\text{мКл}.
    \end{align*}
}
\solutionspace{80pt}

\tasknumber{4}%
\task{%
    Электрический колебательный контур состоит
    из катушки индуктивностью $L$ и конденсатора ёмкостью $C$.
    Последовательно конденсатору подключают ещё один конденсатор ёмкостью $\frac13C$.
    Как изменится период свободных колебаний в контуре?
}
\answer{%
    $
        T = 2\pi\sqrt{LC}, \quad
        T' = 2\pi\sqrt{L'C'}
            = T \sqrt{\frac{L'}L \cdot \frac{C'}C}
            = T \sqrt{ 1 \cdot \frac14 }
        \implies \frac{T'}T = \sqrt{ 1 \cdot \frac14 } \approx 0{,}500.
    $
}
\solutionspace{100pt}

\tasknumber{5}%
\task{%
    В колебательном контуре частота собственных колебаний $40\,\text{Гц}$.
    После замены катушки индуктивности на другую катушку частота стала равной $50\,\text{Гц}$.
    А какой станет частота, если в контур установить обе эти катушки параллельно?
}
\answer{%
    \begin{align*}
    T &= 2\pi\sqrt{LC} \implies \nu = \frac 1T = \frac 1{2\pi\sqrt{LC}} \implies L = \frac1 {4\pi^2 \nu^2 C}, \\
    L_1 &= \frac1 {4\pi^2 \nu_1^2 C}, L_2 = \frac1 {4\pi^2 \nu_1^2 C}, \\
    \nu_\text{послед.}
            &= \frac 1{2\pi\sqrt{(L_1 + L_2)C}}
            = \frac 1{2\pi\sqrt{\cbr{\frac1 {4\pi^2 \nu_1^2 C} + \frac1 {4\pi^2 \nu_2^2 C}}C}}
            = \frac 1{\sqrt{\cbr{\frac1 {\nu_1^2 C} + \frac1 {\nu_2^2 C}}C}} =  \\
     &= \frac 1{\sqrt{\frac1 {\nu_1^2} + \frac1 {\nu_2^2}}}
            = \frac 1{\sqrt{ \frac1 {\sqr{40\,\text{Гц}}} + \frac1 {\sqr{50\,\text{Гц}}}}}
            \approx 31{,}23\,\text{Гц}, \\
    \nu_\text{паралл.}
            &= \frac 1{2\pi\sqrt{\frac 1{\frac 1{L_1} + \frac 1{L_2}}C}}
            = \frac 1{2\pi\sqrt{\frac 1{\frac 1{\frac1 {4\pi^2 \nu_1^2 C}} + \frac 1{\frac1 {4\pi^2 \nu_2^2 C}}}C}}
            = \frac 1{2\pi\sqrt{\frac 1{4\pi^2 \nu_1^2 C + 4\pi^2 \nu_2^2 C}C}} = \\
     &= \frac 1{\sqrt{\frac 1{\nu_1^2 + \nu_2^2}}}
            = \sqrt{\nu_1^2 + \nu_2^2} = \sqrt{\sqr{40\,\text{Гц}} + \sqr{50\,\text{Гц}}} \approx 64{,}03\,\text{Гц}.
    \end{align*}
}

\variantsplitter

\addpersonalvariant{Виктория Легонькова}

\tasknumber{1}%
\task{%
    Схематично изобразите колебательный контур.
    Запишите формулу для периода колебаний в колебательном контуре и ...
    \begin{itemize}
        \item подпишите все физические величины,
        \item укажите их единицы измерения,
        \item выразите из формулы периода циклическую частоту,
        \item выразите из формулы периода индуктивность катушки.
    \end{itemize}
}
\answer{%
    $
        T = 2\pi\sqrt{LC},
        \nu = \frac 1{2\pi\sqrt{LC}},
        \omega = \frac 1{\sqrt{LC}},
        L = \frac 1C \sqr{\frac T{2\pi}},
        C = \frac 1L \sqr{\frac T{2\pi}}.
    $
}
\solutionspace{80pt}

\tasknumber{2}%
\task{%
    Оказалось, что наибольший заряд конденсатора в колебательном контуре равен $80\,\text{мкКл}$,
    а максимальный ток — $180\,\text{мА}$.
    Определите частоту колебаний.
}
\answer{%
    $
        \eli_{\max} = q_{\max}\omega \implies \nu = \frac{\omega}{2\pi} = \frac{\eli_{\max}}{2\pi q} \approx 358{,}1\,\text{Гц}.
    $
}
\solutionspace{80pt}

\tasknumber{3}%
\task{%
    В колебательном контуре сила тока изменяется
    по закону $\eli=0{,}25\cos(15t)$ (в СИ).
    Индуктивность катушки при этом равна $70\,\text{мГн}$.
    Определите:
    \begin{itemize}
        \item период колебаний,
        \item ёмкость конденсатора,
        \item максимальный заряд конденсатора.
    \end{itemize}
}
\answer{%
    \begin{align*}
    \omega &= 15\funits{рад}{c}, \qquad \eli_{\max} = 0{,}25\,\text{A}, \\
    T &= \frac{2\pi}\omega \approx 418{,}9\,\text{мc}, \\
    C &= \frac 1{\omega^2 L} \approx 63{,}5\,\text{мФ}, \\
    q_{\max} &= \frac{\eli_{\max}}\omega  \approx 16{,}7\,\text{мКл}.
    \end{align*}
}
\solutionspace{80pt}

\tasknumber{4}%
\task{%
    Электрический колебательный контур состоит
    из катушки индуктивностью $L$ и конденсатора ёмкостью $C$.
    Параллельно катушке подключают ещё одну катушку индуктивностью $\frac13L$.
    Как изменится период свободных колебаний в контуре?
}
\answer{%
    $
        T = 2\pi\sqrt{LC}, \quad
        T' = 2\pi\sqrt{L'C'}
            = T \sqrt{\frac{L'}L \cdot \frac{C'}C}
            = T \sqrt{ \frac14 \cdot 1 }
        \implies \frac{T'}T = \sqrt{ \frac14 \cdot 1 } \approx 0{,}500.
    $
}
\solutionspace{100pt}

\tasknumber{5}%
\task{%
    В колебательном контуре частота собственных колебаний $80\,\text{Гц}$.
    После замены катушки индуктивности на другую катушку частота стала равной $70\,\text{Гц}$.
    А какой станет частота, если в контур установить обе эти катушки параллельно?
}
\answer{%
    \begin{align*}
    T &= 2\pi\sqrt{LC} \implies \nu = \frac 1T = \frac 1{2\pi\sqrt{LC}} \implies L = \frac1 {4\pi^2 \nu^2 C}, \\
    L_1 &= \frac1 {4\pi^2 \nu_1^2 C}, L_2 = \frac1 {4\pi^2 \nu_1^2 C}, \\
    \nu_\text{послед.}
            &= \frac 1{2\pi\sqrt{(L_1 + L_2)C}}
            = \frac 1{2\pi\sqrt{\cbr{\frac1 {4\pi^2 \nu_1^2 C} + \frac1 {4\pi^2 \nu_2^2 C}}C}}
            = \frac 1{\sqrt{\cbr{\frac1 {\nu_1^2 C} + \frac1 {\nu_2^2 C}}C}} =  \\
     &= \frac 1{\sqrt{\frac1 {\nu_1^2} + \frac1 {\nu_2^2}}}
            = \frac 1{\sqrt{ \frac1 {\sqr{80\,\text{Гц}}} + \frac1 {\sqr{70\,\text{Гц}}}}}
            \approx 52{,}68\,\text{Гц}, \\
    \nu_\text{паралл.}
            &= \frac 1{2\pi\sqrt{\frac 1{\frac 1{L_1} + \frac 1{L_2}}C}}
            = \frac 1{2\pi\sqrt{\frac 1{\frac 1{\frac1 {4\pi^2 \nu_1^2 C}} + \frac 1{\frac1 {4\pi^2 \nu_2^2 C}}}C}}
            = \frac 1{2\pi\sqrt{\frac 1{4\pi^2 \nu_1^2 C + 4\pi^2 \nu_2^2 C}C}} = \\
     &= \frac 1{\sqrt{\frac 1{\nu_1^2 + \nu_2^2}}}
            = \sqrt{\nu_1^2 + \nu_2^2} = \sqrt{\sqr{80\,\text{Гц}} + \sqr{70\,\text{Гц}}} \approx 106{,}30\,\text{Гц}.
    \end{align*}
}

\variantsplitter

\addpersonalvariant{Семён Мартынов}

\tasknumber{1}%
\task{%
    Схематично изобразите колебательный контур.
    Запишите формулу для периода колебаний в колебательном контуре и ...
    \begin{itemize}
        \item подпишите все физические величины,
        \item укажите их единицы измерения,
        \item выразите из формулы периода частоту,
        \item выразите из формулы периода индуктивность катушки.
    \end{itemize}
}
\answer{%
    $
        T = 2\pi\sqrt{LC},
        \nu = \frac 1{2\pi\sqrt{LC}},
        \omega = \frac 1{\sqrt{LC}},
        L = \frac 1C \sqr{\frac T{2\pi}},
        C = \frac 1L \sqr{\frac T{2\pi}}.
    $
}
\solutionspace{80pt}

\tasknumber{2}%
\task{%
    Оказалось, что наибольший заряд конденсатора в колебательном контуре равен $60\,\text{мкКл}$,
    а максимальный ток — $180\,\text{мА}$.
    Определите частоту колебаний.
}
\answer{%
    $
        \eli_{\max} = q_{\max}\omega \implies \nu = \frac{\omega}{2\pi} = \frac{\eli_{\max}}{2\pi q} \approx 477{,}5\,\text{Гц}.
    $
}
\solutionspace{80pt}

\tasknumber{3}%
\task{%
    В колебательном контуре сила тока изменяется
    по закону $\eli=0{,}05\cos(15t)$ (в СИ).
    Индуктивность катушки при этом равна $70\,\text{мГн}$.
    Определите:
    \begin{itemize}
        \item период колебаний,
        \item ёмкость конденсатора,
        \item максимальный заряд конденсатора.
    \end{itemize}
}
\answer{%
    \begin{align*}
    \omega &= 15\funits{рад}{c}, \qquad \eli_{\max} = 0{,}05\,\text{A}, \\
    T &= \frac{2\pi}\omega \approx 418{,}9\,\text{мc}, \\
    C &= \frac 1{\omega^2 L} \approx 63{,}5\,\text{мФ}, \\
    q_{\max} &= \frac{\eli_{\max}}\omega  \approx 3{,}3\,\text{мКл}.
    \end{align*}
}
\solutionspace{80pt}

\tasknumber{4}%
\task{%
    Электрический колебательный контур состоит
    из катушки индуктивностью $L$ и конденсатора ёмкостью $C$.
    Параллельно конденсатору подключают ещё один конденсатор ёмкостью $\frac12C$.
    Как изменится период свободных колебаний в контуре?
}
\answer{%
    $
        T = 2\pi\sqrt{LC}, \quad
        T' = 2\pi\sqrt{L'C'}
            = T \sqrt{\frac{L'}L \cdot \frac{C'}C}
            = T \sqrt{ 1 \cdot \frac32 }
        \implies \frac{T'}T = \sqrt{ 1 \cdot \frac32 } \approx 1{,}225.
    $
}
\solutionspace{100pt}

\tasknumber{5}%
\task{%
    В колебательном контуре частота собственных колебаний $60\,\text{Гц}$.
    После замены катушки индуктивности на другую катушку частота стала равной $90\,\text{Гц}$.
    А какой станет частота, если в контур установить обе эти катушки параллельно?
}
\answer{%
    \begin{align*}
    T &= 2\pi\sqrt{LC} \implies \nu = \frac 1T = \frac 1{2\pi\sqrt{LC}} \implies L = \frac1 {4\pi^2 \nu^2 C}, \\
    L_1 &= \frac1 {4\pi^2 \nu_1^2 C}, L_2 = \frac1 {4\pi^2 \nu_1^2 C}, \\
    \nu_\text{послед.}
            &= \frac 1{2\pi\sqrt{(L_1 + L_2)C}}
            = \frac 1{2\pi\sqrt{\cbr{\frac1 {4\pi^2 \nu_1^2 C} + \frac1 {4\pi^2 \nu_2^2 C}}C}}
            = \frac 1{\sqrt{\cbr{\frac1 {\nu_1^2 C} + \frac1 {\nu_2^2 C}}C}} =  \\
     &= \frac 1{\sqrt{\frac1 {\nu_1^2} + \frac1 {\nu_2^2}}}
            = \frac 1{\sqrt{ \frac1 {\sqr{60\,\text{Гц}}} + \frac1 {\sqr{90\,\text{Гц}}}}}
            \approx 49{,}92\,\text{Гц}, \\
    \nu_\text{паралл.}
            &= \frac 1{2\pi\sqrt{\frac 1{\frac 1{L_1} + \frac 1{L_2}}C}}
            = \frac 1{2\pi\sqrt{\frac 1{\frac 1{\frac1 {4\pi^2 \nu_1^2 C}} + \frac 1{\frac1 {4\pi^2 \nu_2^2 C}}}C}}
            = \frac 1{2\pi\sqrt{\frac 1{4\pi^2 \nu_1^2 C + 4\pi^2 \nu_2^2 C}C}} = \\
     &= \frac 1{\sqrt{\frac 1{\nu_1^2 + \nu_2^2}}}
            = \sqrt{\nu_1^2 + \nu_2^2} = \sqrt{\sqr{60\,\text{Гц}} + \sqr{90\,\text{Гц}}} \approx 108{,}17\,\text{Гц}.
    \end{align*}
}

\variantsplitter

\addpersonalvariant{Варвара Минаева}

\tasknumber{1}%
\task{%
    Схематично изобразите колебательный контур.
    Запишите формулу для периода колебаний в колебательном контуре и ...
    \begin{itemize}
        \item подпишите все физические величины,
        \item укажите их единицы измерения,
        \item выразите из формулы периода циклическую частоту,
        \item выразите из формулы периода индуктивность катушки.
    \end{itemize}
}
\answer{%
    $
        T = 2\pi\sqrt{LC},
        \nu = \frac 1{2\pi\sqrt{LC}},
        \omega = \frac 1{\sqrt{LC}},
        L = \frac 1C \sqr{\frac T{2\pi}},
        C = \frac 1L \sqr{\frac T{2\pi}}.
    $
}
\solutionspace{80pt}

\tasknumber{2}%
\task{%
    Оказалось, что наибольший заряд конденсатора в колебательном контуре равен $60\,\text{мкКл}$,
    а максимальный ток — $120\,\text{мА}$.
    Определите частоту колебаний.
}
\answer{%
    $
        \eli_{\max} = q_{\max}\omega \implies \nu = \frac{\omega}{2\pi} = \frac{\eli_{\max}}{2\pi q} \approx 318{,}3\,\text{Гц}.
    $
}
\solutionspace{80pt}

\tasknumber{3}%
\task{%
    В колебательном контуре сила тока изменяется
    по закону $\eli=0{,}25\sin(12t)$ (в СИ).
    Индуктивность катушки при этом равна $50\,\text{мГн}$.
    Определите:
    \begin{itemize}
        \item период колебаний,
        \item ёмкость конденсатора,
        \item максимальный заряд конденсатора.
    \end{itemize}
}
\answer{%
    \begin{align*}
    \omega &= 12\funits{рад}{c}, \qquad \eli_{\max} = 0{,}25\,\text{A}, \\
    T &= \frac{2\pi}\omega \approx 523{,}6\,\text{мc}, \\
    C &= \frac 1{\omega^2 L} \approx 138{,}9\,\text{мФ}, \\
    q_{\max} &= \frac{\eli_{\max}}\omega  \approx 20{,}8\,\text{мКл}.
    \end{align*}
}
\solutionspace{80pt}

\tasknumber{4}%
\task{%
    Электрический колебательный контур состоит
    из катушки индуктивностью $L$ и конденсатора ёмкостью $C$.
    Последовательно конденсатору подключают ещё один конденсатор ёмкостью $3C$.
    Как изменится период свободных колебаний в контуре?
}
\answer{%
    $
        T = 2\pi\sqrt{LC}, \quad
        T' = 2\pi\sqrt{L'C'}
            = T \sqrt{\frac{L'}L \cdot \frac{C'}C}
            = T \sqrt{ 1 \cdot \frac34 }
        \implies \frac{T'}T = \sqrt{ 1 \cdot \frac34 } \approx 0{,}866.
    $
}
\solutionspace{100pt}

\tasknumber{5}%
\task{%
    В колебательном контуре частота собственных колебаний $60\,\text{Гц}$.
    После замены катушки индуктивности на другую катушку частота стала равной $30\,\text{Гц}$.
    А какой станет частота, если в контур установить обе эти катушки параллельно?
}
\answer{%
    \begin{align*}
    T &= 2\pi\sqrt{LC} \implies \nu = \frac 1T = \frac 1{2\pi\sqrt{LC}} \implies L = \frac1 {4\pi^2 \nu^2 C}, \\
    L_1 &= \frac1 {4\pi^2 \nu_1^2 C}, L_2 = \frac1 {4\pi^2 \nu_1^2 C}, \\
    \nu_\text{послед.}
            &= \frac 1{2\pi\sqrt{(L_1 + L_2)C}}
            = \frac 1{2\pi\sqrt{\cbr{\frac1 {4\pi^2 \nu_1^2 C} + \frac1 {4\pi^2 \nu_2^2 C}}C}}
            = \frac 1{\sqrt{\cbr{\frac1 {\nu_1^2 C} + \frac1 {\nu_2^2 C}}C}} =  \\
     &= \frac 1{\sqrt{\frac1 {\nu_1^2} + \frac1 {\nu_2^2}}}
            = \frac 1{\sqrt{ \frac1 {\sqr{60\,\text{Гц}}} + \frac1 {\sqr{30\,\text{Гц}}}}}
            \approx 26{,}83\,\text{Гц}, \\
    \nu_\text{паралл.}
            &= \frac 1{2\pi\sqrt{\frac 1{\frac 1{L_1} + \frac 1{L_2}}C}}
            = \frac 1{2\pi\sqrt{\frac 1{\frac 1{\frac1 {4\pi^2 \nu_1^2 C}} + \frac 1{\frac1 {4\pi^2 \nu_2^2 C}}}C}}
            = \frac 1{2\pi\sqrt{\frac 1{4\pi^2 \nu_1^2 C + 4\pi^2 \nu_2^2 C}C}} = \\
     &= \frac 1{\sqrt{\frac 1{\nu_1^2 + \nu_2^2}}}
            = \sqrt{\nu_1^2 + \nu_2^2} = \sqrt{\sqr{60\,\text{Гц}} + \sqr{30\,\text{Гц}}} \approx 67{,}08\,\text{Гц}.
    \end{align*}
}

\variantsplitter

\addpersonalvariant{Леонид Никитин}

\tasknumber{1}%
\task{%
    Схематично изобразите колебательный контур.
    Запишите формулу для периода колебаний в колебательном контуре и ...
    \begin{itemize}
        \item подпишите все физические величины,
        \item укажите их единицы измерения,
        \item выразите из формулы периода циклическую частоту,
        \item выразите из формулы периода ёмкость конденсатора.
    \end{itemize}
}
\answer{%
    $
        T = 2\pi\sqrt{LC},
        \nu = \frac 1{2\pi\sqrt{LC}},
        \omega = \frac 1{\sqrt{LC}},
        L = \frac 1C \sqr{\frac T{2\pi}},
        C = \frac 1L \sqr{\frac T{2\pi}}.
    $
}
\solutionspace{80pt}

\tasknumber{2}%
\task{%
    Оказалось, что наибольший заряд конденсатора в колебательном контуре равен $60\,\text{мкКл}$,
    а максимальный ток — $270\,\text{мА}$.
    Определите частоту колебаний.
}
\answer{%
    $
        \eli_{\max} = q_{\max}\omega \implies \nu = \frac{\omega}{2\pi} = \frac{\eli_{\max}}{2\pi q} \approx 716{,}2\,\text{Гц}.
    $
}
\solutionspace{80pt}

\tasknumber{3}%
\task{%
    В колебательном контуре сила тока изменяется
    по закону $\eli=0{,}25\sin(18t)$ (в СИ).
    Индуктивность катушки при этом равна $60\,\text{мГн}$.
    Определите:
    \begin{itemize}
        \item период колебаний,
        \item ёмкость конденсатора,
        \item максимальный заряд конденсатора.
    \end{itemize}
}
\answer{%
    \begin{align*}
    \omega &= 18\funits{рад}{c}, \qquad \eli_{\max} = 0{,}25\,\text{A}, \\
    T &= \frac{2\pi}\omega \approx 349{,}1\,\text{мc}, \\
    C &= \frac 1{\omega^2 L} \approx 51{,}4\,\text{мФ}, \\
    q_{\max} &= \frac{\eli_{\max}}\omega  \approx 13{,}9\,\text{мКл}.
    \end{align*}
}
\solutionspace{80pt}

\tasknumber{4}%
\task{%
    Электрический колебательный контур состоит
    из катушки индуктивностью $L$ и конденсатора ёмкостью $C$.
    Параллельно конденсатору подключают ещё один конденсатор ёмкостью $2C$.
    Как изменится период свободных колебаний в контуре?
}
\answer{%
    $
        T = 2\pi\sqrt{LC}, \quad
        T' = 2\pi\sqrt{L'C'}
            = T \sqrt{\frac{L'}L \cdot \frac{C'}C}
            = T \sqrt{ 1 \cdot 3 }
        \implies \frac{T'}T = \sqrt{ 1 \cdot 3 } \approx 1{,}732.
    $
}
\solutionspace{100pt}

\tasknumber{5}%
\task{%
    В колебательном контуре частота собственных колебаний $40\,\text{Гц}$.
    После замены катушки индуктивности на другую катушку частота стала равной $90\,\text{Гц}$.
    А какой станет частота, если в контур установить обе эти катушки последовательно?
}
\answer{%
    \begin{align*}
    T &= 2\pi\sqrt{LC} \implies \nu = \frac 1T = \frac 1{2\pi\sqrt{LC}} \implies L = \frac1 {4\pi^2 \nu^2 C}, \\
    L_1 &= \frac1 {4\pi^2 \nu_1^2 C}, L_2 = \frac1 {4\pi^2 \nu_1^2 C}, \\
    \nu_\text{послед.}
            &= \frac 1{2\pi\sqrt{(L_1 + L_2)C}}
            = \frac 1{2\pi\sqrt{\cbr{\frac1 {4\pi^2 \nu_1^2 C} + \frac1 {4\pi^2 \nu_2^2 C}}C}}
            = \frac 1{\sqrt{\cbr{\frac1 {\nu_1^2 C} + \frac1 {\nu_2^2 C}}C}} =  \\
     &= \frac 1{\sqrt{\frac1 {\nu_1^2} + \frac1 {\nu_2^2}}}
            = \frac 1{\sqrt{ \frac1 {\sqr{40\,\text{Гц}}} + \frac1 {\sqr{90\,\text{Гц}}}}}
            \approx 36{,}55\,\text{Гц}, \\
    \nu_\text{паралл.}
            &= \frac 1{2\pi\sqrt{\frac 1{\frac 1{L_1} + \frac 1{L_2}}C}}
            = \frac 1{2\pi\sqrt{\frac 1{\frac 1{\frac1 {4\pi^2 \nu_1^2 C}} + \frac 1{\frac1 {4\pi^2 \nu_2^2 C}}}C}}
            = \frac 1{2\pi\sqrt{\frac 1{4\pi^2 \nu_1^2 C + 4\pi^2 \nu_2^2 C}C}} = \\
     &= \frac 1{\sqrt{\frac 1{\nu_1^2 + \nu_2^2}}}
            = \sqrt{\nu_1^2 + \nu_2^2} = \sqrt{\sqr{40\,\text{Гц}} + \sqr{90\,\text{Гц}}} \approx 98{,}49\,\text{Гц}.
    \end{align*}
}

\variantsplitter

\addpersonalvariant{Тимофей Полетаев}

\tasknumber{1}%
\task{%
    Схематично изобразите колебательный контур.
    Запишите формулу для периода колебаний в колебательном контуре и ...
    \begin{itemize}
        \item подпишите все физические величины,
        \item укажите их единицы измерения,
        \item выразите из формулы периода частоту,
        \item выразите из формулы периода индуктивность катушки.
    \end{itemize}
}
\answer{%
    $
        T = 2\pi\sqrt{LC},
        \nu = \frac 1{2\pi\sqrt{LC}},
        \omega = \frac 1{\sqrt{LC}},
        L = \frac 1C \sqr{\frac T{2\pi}},
        C = \frac 1L \sqr{\frac T{2\pi}}.
    $
}
\solutionspace{80pt}

\tasknumber{2}%
\task{%
    Оказалось, что наибольший заряд конденсатора в колебательном контуре равен $40\,\text{мкКл}$,
    а максимальный ток — $120\,\text{мА}$.
    Определите частоту колебаний.
}
\answer{%
    $
        \eli_{\max} = q_{\max}\omega \implies \nu = \frac{\omega}{2\pi} = \frac{\eli_{\max}}{2\pi q} \approx 477{,}5\,\text{Гц}.
    $
}
\solutionspace{80pt}

\tasknumber{3}%
\task{%
    В колебательном контуре сила тока изменяется
    по закону $\eli=0{,}25\cos(15t)$ (в СИ).
    Индуктивность катушки при этом равна $80\,\text{мГн}$.
    Определите:
    \begin{itemize}
        \item период колебаний,
        \item ёмкость конденсатора,
        \item максимальный заряд конденсатора.
    \end{itemize}
}
\answer{%
    \begin{align*}
    \omega &= 15\funits{рад}{c}, \qquad \eli_{\max} = 0{,}25\,\text{A}, \\
    T &= \frac{2\pi}\omega \approx 418{,}9\,\text{мc}, \\
    C &= \frac 1{\omega^2 L} \approx 55{,}6\,\text{мФ}, \\
    q_{\max} &= \frac{\eli_{\max}}\omega  \approx 16{,}7\,\text{мКл}.
    \end{align*}
}
\solutionspace{80pt}

\tasknumber{4}%
\task{%
    Электрический колебательный контур состоит
    из катушки индуктивностью $L$ и конденсатора ёмкостью $C$.
    Последовательно катушке подключают ещё одну катушку индуктивностью $\frac13L$.
    Как изменится период свободных колебаний в контуре?
}
\answer{%
    $
        T = 2\pi\sqrt{LC}, \quad
        T' = 2\pi\sqrt{L'C'}
            = T \sqrt{\frac{L'}L \cdot \frac{C'}C}
            = T \sqrt{ \frac43 \cdot 1 }
        \implies \frac{T'}T = \sqrt{ \frac43 \cdot 1 } \approx 1{,}155.
    $
}
\solutionspace{100pt}

\tasknumber{5}%
\task{%
    В колебательном контуре частота собственных колебаний $40\,\text{Гц}$.
    После замены катушки индуктивности на другую катушку частота стала равной $90\,\text{Гц}$.
    А какой станет частота, если в контур установить обе эти катушки параллельно?
}
\answer{%
    \begin{align*}
    T &= 2\pi\sqrt{LC} \implies \nu = \frac 1T = \frac 1{2\pi\sqrt{LC}} \implies L = \frac1 {4\pi^2 \nu^2 C}, \\
    L_1 &= \frac1 {4\pi^2 \nu_1^2 C}, L_2 = \frac1 {4\pi^2 \nu_1^2 C}, \\
    \nu_\text{послед.}
            &= \frac 1{2\pi\sqrt{(L_1 + L_2)C}}
            = \frac 1{2\pi\sqrt{\cbr{\frac1 {4\pi^2 \nu_1^2 C} + \frac1 {4\pi^2 \nu_2^2 C}}C}}
            = \frac 1{\sqrt{\cbr{\frac1 {\nu_1^2 C} + \frac1 {\nu_2^2 C}}C}} =  \\
     &= \frac 1{\sqrt{\frac1 {\nu_1^2} + \frac1 {\nu_2^2}}}
            = \frac 1{\sqrt{ \frac1 {\sqr{40\,\text{Гц}}} + \frac1 {\sqr{90\,\text{Гц}}}}}
            \approx 36{,}55\,\text{Гц}, \\
    \nu_\text{паралл.}
            &= \frac 1{2\pi\sqrt{\frac 1{\frac 1{L_1} + \frac 1{L_2}}C}}
            = \frac 1{2\pi\sqrt{\frac 1{\frac 1{\frac1 {4\pi^2 \nu_1^2 C}} + \frac 1{\frac1 {4\pi^2 \nu_2^2 C}}}C}}
            = \frac 1{2\pi\sqrt{\frac 1{4\pi^2 \nu_1^2 C + 4\pi^2 \nu_2^2 C}C}} = \\
     &= \frac 1{\sqrt{\frac 1{\nu_1^2 + \nu_2^2}}}
            = \sqrt{\nu_1^2 + \nu_2^2} = \sqrt{\sqr{40\,\text{Гц}} + \sqr{90\,\text{Гц}}} \approx 98{,}49\,\text{Гц}.
    \end{align*}
}

\variantsplitter

\addpersonalvariant{Андрей Рожков}

\tasknumber{1}%
\task{%
    Схематично изобразите колебательный контур.
    Запишите формулу для периода колебаний в колебательном контуре и ...
    \begin{itemize}
        \item подпишите все физические величины,
        \item укажите их единицы измерения,
        \item выразите из формулы периода циклическую частоту,
        \item выразите из формулы периода индуктивность катушки.
    \end{itemize}
}
\answer{%
    $
        T = 2\pi\sqrt{LC},
        \nu = \frac 1{2\pi\sqrt{LC}},
        \omega = \frac 1{\sqrt{LC}},
        L = \frac 1C \sqr{\frac T{2\pi}},
        C = \frac 1L \sqr{\frac T{2\pi}}.
    $
}
\solutionspace{80pt}

\tasknumber{2}%
\task{%
    Оказалось, что наибольший заряд конденсатора в колебательном контуре равен $80\,\text{мкКл}$,
    а максимальный ток — $120\,\text{мА}$.
    Определите частоту колебаний.
}
\answer{%
    $
        \eli_{\max} = q_{\max}\omega \implies \nu = \frac{\omega}{2\pi} = \frac{\eli_{\max}}{2\pi q} \approx 238{,}7\,\text{Гц}.
    $
}
\solutionspace{80pt}

\tasknumber{3}%
\task{%
    В колебательном контуре сила тока изменяется
    по закону $\eli=0{,}05\sin(12t)$ (в СИ).
    Индуктивность катушки при этом равна $50\,\text{мГн}$.
    Определите:
    \begin{itemize}
        \item период колебаний,
        \item ёмкость конденсатора,
        \item максимальный заряд конденсатора.
    \end{itemize}
}
\answer{%
    \begin{align*}
    \omega &= 12\funits{рад}{c}, \qquad \eli_{\max} = 0{,}05\,\text{A}, \\
    T &= \frac{2\pi}\omega \approx 523{,}6\,\text{мc}, \\
    C &= \frac 1{\omega^2 L} \approx 138{,}9\,\text{мФ}, \\
    q_{\max} &= \frac{\eli_{\max}}\omega  \approx 4{,}2\,\text{мКл}.
    \end{align*}
}
\solutionspace{80pt}

\tasknumber{4}%
\task{%
    Электрический колебательный контур состоит
    из катушки индуктивностью $L$ и конденсатора ёмкостью $C$.
    Параллельно конденсатору подключают ещё один конденсатор ёмкостью $\frac12C$.
    Как изменится период свободных колебаний в контуре?
}
\answer{%
    $
        T = 2\pi\sqrt{LC}, \quad
        T' = 2\pi\sqrt{L'C'}
            = T \sqrt{\frac{L'}L \cdot \frac{C'}C}
            = T \sqrt{ 1 \cdot \frac32 }
        \implies \frac{T'}T = \sqrt{ 1 \cdot \frac32 } \approx 1{,}225.
    $
}
\solutionspace{100pt}

\tasknumber{5}%
\task{%
    В колебательном контуре частота собственных колебаний $60\,\text{Гц}$.
    После замены катушки индуктивности на другую катушку частота стала равной $90\,\text{Гц}$.
    А какой станет частота, если в контур установить обе эти катушки последовательно?
}
\answer{%
    \begin{align*}
    T &= 2\pi\sqrt{LC} \implies \nu = \frac 1T = \frac 1{2\pi\sqrt{LC}} \implies L = \frac1 {4\pi^2 \nu^2 C}, \\
    L_1 &= \frac1 {4\pi^2 \nu_1^2 C}, L_2 = \frac1 {4\pi^2 \nu_1^2 C}, \\
    \nu_\text{послед.}
            &= \frac 1{2\pi\sqrt{(L_1 + L_2)C}}
            = \frac 1{2\pi\sqrt{\cbr{\frac1 {4\pi^2 \nu_1^2 C} + \frac1 {4\pi^2 \nu_2^2 C}}C}}
            = \frac 1{\sqrt{\cbr{\frac1 {\nu_1^2 C} + \frac1 {\nu_2^2 C}}C}} =  \\
     &= \frac 1{\sqrt{\frac1 {\nu_1^2} + \frac1 {\nu_2^2}}}
            = \frac 1{\sqrt{ \frac1 {\sqr{60\,\text{Гц}}} + \frac1 {\sqr{90\,\text{Гц}}}}}
            \approx 49{,}92\,\text{Гц}, \\
    \nu_\text{паралл.}
            &= \frac 1{2\pi\sqrt{\frac 1{\frac 1{L_1} + \frac 1{L_2}}C}}
            = \frac 1{2\pi\sqrt{\frac 1{\frac 1{\frac1 {4\pi^2 \nu_1^2 C}} + \frac 1{\frac1 {4\pi^2 \nu_2^2 C}}}C}}
            = \frac 1{2\pi\sqrt{\frac 1{4\pi^2 \nu_1^2 C + 4\pi^2 \nu_2^2 C}C}} = \\
     &= \frac 1{\sqrt{\frac 1{\nu_1^2 + \nu_2^2}}}
            = \sqrt{\nu_1^2 + \nu_2^2} = \sqrt{\sqr{60\,\text{Гц}} + \sqr{90\,\text{Гц}}} \approx 108{,}17\,\text{Гц}.
    \end{align*}
}

\variantsplitter

\addpersonalvariant{Рената Таржиманова}

\tasknumber{1}%
\task{%
    Схематично изобразите колебательный контур.
    Запишите формулу для периода колебаний в колебательном контуре и ...
    \begin{itemize}
        \item подпишите все физические величины,
        \item укажите их единицы измерения,
        \item выразите из формулы периода частоту,
        \item выразите из формулы периода индуктивность катушки.
    \end{itemize}
}
\answer{%
    $
        T = 2\pi\sqrt{LC},
        \nu = \frac 1{2\pi\sqrt{LC}},
        \omega = \frac 1{\sqrt{LC}},
        L = \frac 1C \sqr{\frac T{2\pi}},
        C = \frac 1L \sqr{\frac T{2\pi}}.
    $
}
\solutionspace{80pt}

\tasknumber{2}%
\task{%
    Оказалось, что наибольший заряд конденсатора в колебательном контуре равен $80\,\text{мкКл}$,
    а максимальный ток — $150\,\text{мА}$.
    Определите частоту колебаний.
}
\answer{%
    $
        \eli_{\max} = q_{\max}\omega \implies \nu = \frac{\omega}{2\pi} = \frac{\eli_{\max}}{2\pi q} \approx 298{,}4\,\text{Гц}.
    $
}
\solutionspace{80pt}

\tasknumber{3}%
\task{%
    В колебательном контуре сила тока изменяется
    по закону $\eli=0{,}25\cos(18t)$ (в СИ).
    Индуктивность катушки при этом равна $70\,\text{мГн}$.
    Определите:
    \begin{itemize}
        \item период колебаний,
        \item ёмкость конденсатора,
        \item максимальный заряд конденсатора.
    \end{itemize}
}
\answer{%
    \begin{align*}
    \omega &= 18\funits{рад}{c}, \qquad \eli_{\max} = 0{,}25\,\text{A}, \\
    T &= \frac{2\pi}\omega \approx 349{,}1\,\text{мc}, \\
    C &= \frac 1{\omega^2 L} \approx 44{,}1\,\text{мФ}, \\
    q_{\max} &= \frac{\eli_{\max}}\omega  \approx 13{,}9\,\text{мКл}.
    \end{align*}
}
\solutionspace{80pt}

\tasknumber{4}%
\task{%
    Электрический колебательный контур состоит
    из катушки индуктивностью $L$ и конденсатора ёмкостью $C$.
    Параллельно катушке подключают ещё одну катушку индуктивностью $3L$.
    Как изменится период свободных колебаний в контуре?
}
\answer{%
    $
        T = 2\pi\sqrt{LC}, \quad
        T' = 2\pi\sqrt{L'C'}
            = T \sqrt{\frac{L'}L \cdot \frac{C'}C}
            = T \sqrt{ \frac34 \cdot 1 }
        \implies \frac{T'}T = \sqrt{ \frac34 \cdot 1 } \approx 0{,}866.
    $
}
\solutionspace{100pt}

\tasknumber{5}%
\task{%
    В колебательном контуре частота собственных колебаний $40\,\text{Гц}$.
    После замены катушки индуктивности на другую катушку частота стала равной $90\,\text{Гц}$.
    А какой станет частота, если в контур установить обе эти катушки параллельно?
}
\answer{%
    \begin{align*}
    T &= 2\pi\sqrt{LC} \implies \nu = \frac 1T = \frac 1{2\pi\sqrt{LC}} \implies L = \frac1 {4\pi^2 \nu^2 C}, \\
    L_1 &= \frac1 {4\pi^2 \nu_1^2 C}, L_2 = \frac1 {4\pi^2 \nu_1^2 C}, \\
    \nu_\text{послед.}
            &= \frac 1{2\pi\sqrt{(L_1 + L_2)C}}
            = \frac 1{2\pi\sqrt{\cbr{\frac1 {4\pi^2 \nu_1^2 C} + \frac1 {4\pi^2 \nu_2^2 C}}C}}
            = \frac 1{\sqrt{\cbr{\frac1 {\nu_1^2 C} + \frac1 {\nu_2^2 C}}C}} =  \\
     &= \frac 1{\sqrt{\frac1 {\nu_1^2} + \frac1 {\nu_2^2}}}
            = \frac 1{\sqrt{ \frac1 {\sqr{40\,\text{Гц}}} + \frac1 {\sqr{90\,\text{Гц}}}}}
            \approx 36{,}55\,\text{Гц}, \\
    \nu_\text{паралл.}
            &= \frac 1{2\pi\sqrt{\frac 1{\frac 1{L_1} + \frac 1{L_2}}C}}
            = \frac 1{2\pi\sqrt{\frac 1{\frac 1{\frac1 {4\pi^2 \nu_1^2 C}} + \frac 1{\frac1 {4\pi^2 \nu_2^2 C}}}C}}
            = \frac 1{2\pi\sqrt{\frac 1{4\pi^2 \nu_1^2 C + 4\pi^2 \nu_2^2 C}C}} = \\
     &= \frac 1{\sqrt{\frac 1{\nu_1^2 + \nu_2^2}}}
            = \sqrt{\nu_1^2 + \nu_2^2} = \sqrt{\sqr{40\,\text{Гц}} + \sqr{90\,\text{Гц}}} \approx 98{,}49\,\text{Гц}.
    \end{align*}
}

\variantsplitter

\addpersonalvariant{Андрей Щербаков}

\tasknumber{1}%
\task{%
    Схематично изобразите колебательный контур.
    Запишите формулу для периода колебаний в колебательном контуре и ...
    \begin{itemize}
        \item подпишите все физические величины,
        \item укажите их единицы измерения,
        \item выразите из формулы периода циклическую частоту,
        \item выразите из формулы периода ёмкость конденсатора.
    \end{itemize}
}
\answer{%
    $
        T = 2\pi\sqrt{LC},
        \nu = \frac 1{2\pi\sqrt{LC}},
        \omega = \frac 1{\sqrt{LC}},
        L = \frac 1C \sqr{\frac T{2\pi}},
        C = \frac 1L \sqr{\frac T{2\pi}}.
    $
}
\solutionspace{80pt}

\tasknumber{2}%
\task{%
    Оказалось, что наибольший заряд конденсатора в колебательном контуре равен $60\,\text{мкКл}$,
    а максимальный ток — $150\,\text{мА}$.
    Определите частоту колебаний.
}
\answer{%
    $
        \eli_{\max} = q_{\max}\omega \implies \nu = \frac{\omega}{2\pi} = \frac{\eli_{\max}}{2\pi q} \approx 397{,}9\,\text{Гц}.
    $
}
\solutionspace{80pt}

\tasknumber{3}%
\task{%
    В колебательном контуре сила тока изменяется
    по закону $\eli=0{,}25\sin(18t)$ (в СИ).
    Индуктивность катушки при этом равна $70\,\text{мГн}$.
    Определите:
    \begin{itemize}
        \item период колебаний,
        \item ёмкость конденсатора,
        \item максимальный заряд конденсатора.
    \end{itemize}
}
\answer{%
    \begin{align*}
    \omega &= 18\funits{рад}{c}, \qquad \eli_{\max} = 0{,}25\,\text{A}, \\
    T &= \frac{2\pi}\omega \approx 349{,}1\,\text{мc}, \\
    C &= \frac 1{\omega^2 L} \approx 44{,}1\,\text{мФ}, \\
    q_{\max} &= \frac{\eli_{\max}}\omega  \approx 13{,}9\,\text{мКл}.
    \end{align*}
}
\solutionspace{80pt}

\tasknumber{4}%
\task{%
    Электрический колебательный контур состоит
    из катушки индуктивностью $L$ и конденсатора ёмкостью $C$.
    Последовательно катушке подключают ещё одну катушку индуктивностью $2L$.
    Как изменится период свободных колебаний в контуре?
}
\answer{%
    $
        T = 2\pi\sqrt{LC}, \quad
        T' = 2\pi\sqrt{L'C'}
            = T \sqrt{\frac{L'}L \cdot \frac{C'}C}
            = T \sqrt{ 3 \cdot 1 }
        \implies \frac{T'}T = \sqrt{ 3 \cdot 1 } \approx 1{,}732.
    $
}
\solutionspace{100pt}

\tasknumber{5}%
\task{%
    В колебательном контуре частота собственных колебаний $40\,\text{Гц}$.
    После замены катушки индуктивности на другую катушку частота стала равной $90\,\text{Гц}$.
    А какой станет частота, если в контур установить обе эти катушки последовательно?
}
\answer{%
    \begin{align*}
    T &= 2\pi\sqrt{LC} \implies \nu = \frac 1T = \frac 1{2\pi\sqrt{LC}} \implies L = \frac1 {4\pi^2 \nu^2 C}, \\
    L_1 &= \frac1 {4\pi^2 \nu_1^2 C}, L_2 = \frac1 {4\pi^2 \nu_1^2 C}, \\
    \nu_\text{послед.}
            &= \frac 1{2\pi\sqrt{(L_1 + L_2)C}}
            = \frac 1{2\pi\sqrt{\cbr{\frac1 {4\pi^2 \nu_1^2 C} + \frac1 {4\pi^2 \nu_2^2 C}}C}}
            = \frac 1{\sqrt{\cbr{\frac1 {\nu_1^2 C} + \frac1 {\nu_2^2 C}}C}} =  \\
     &= \frac 1{\sqrt{\frac1 {\nu_1^2} + \frac1 {\nu_2^2}}}
            = \frac 1{\sqrt{ \frac1 {\sqr{40\,\text{Гц}}} + \frac1 {\sqr{90\,\text{Гц}}}}}
            \approx 36{,}55\,\text{Гц}, \\
    \nu_\text{паралл.}
            &= \frac 1{2\pi\sqrt{\frac 1{\frac 1{L_1} + \frac 1{L_2}}C}}
            = \frac 1{2\pi\sqrt{\frac 1{\frac 1{\frac1 {4\pi^2 \nu_1^2 C}} + \frac 1{\frac1 {4\pi^2 \nu_2^2 C}}}C}}
            = \frac 1{2\pi\sqrt{\frac 1{4\pi^2 \nu_1^2 C + 4\pi^2 \nu_2^2 C}C}} = \\
     &= \frac 1{\sqrt{\frac 1{\nu_1^2 + \nu_2^2}}}
            = \sqrt{\nu_1^2 + \nu_2^2} = \sqrt{\sqr{40\,\text{Гц}} + \sqr{90\,\text{Гц}}} \approx 98{,}49\,\text{Гц}.
    \end{align*}
}

\variantsplitter

\addpersonalvariant{Михаил Ярошевский}

\tasknumber{1}%
\task{%
    Схематично изобразите колебательный контур.
    Запишите формулу для периода колебаний в колебательном контуре и ...
    \begin{itemize}
        \item подпишите все физические величины,
        \item укажите их единицы измерения,
        \item выразите из формулы периода циклическую частоту,
        \item выразите из формулы периода ёмкость конденсатора.
    \end{itemize}
}
\answer{%
    $
        T = 2\pi\sqrt{LC},
        \nu = \frac 1{2\pi\sqrt{LC}},
        \omega = \frac 1{\sqrt{LC}},
        L = \frac 1C \sqr{\frac T{2\pi}},
        C = \frac 1L \sqr{\frac T{2\pi}}.
    $
}
\solutionspace{80pt}

\tasknumber{2}%
\task{%
    Оказалось, что наибольший заряд конденсатора в колебательном контуре равен $80\,\text{мкКл}$,
    а максимальный ток — $180\,\text{мА}$.
    Определите частоту колебаний.
}
\answer{%
    $
        \eli_{\max} = q_{\max}\omega \implies \nu = \frac{\omega}{2\pi} = \frac{\eli_{\max}}{2\pi q} \approx 358{,}1\,\text{Гц}.
    $
}
\solutionspace{80pt}

\tasknumber{3}%
\task{%
    В колебательном контуре сила тока изменяется
    по закону $\eli=0{,}05\cos(18t)$ (в СИ).
    Индуктивность катушки при этом равна $80\,\text{мГн}$.
    Определите:
    \begin{itemize}
        \item период колебаний,
        \item ёмкость конденсатора,
        \item максимальный заряд конденсатора.
    \end{itemize}
}
\answer{%
    \begin{align*}
    \omega &= 18\funits{рад}{c}, \qquad \eli_{\max} = 0{,}05\,\text{A}, \\
    T &= \frac{2\pi}\omega \approx 349{,}1\,\text{мc}, \\
    C &= \frac 1{\omega^2 L} \approx 38{,}6\,\text{мФ}, \\
    q_{\max} &= \frac{\eli_{\max}}\omega  \approx 2{,}8\,\text{мКл}.
    \end{align*}
}
\solutionspace{80pt}

\tasknumber{4}%
\task{%
    Электрический колебательный контур состоит
    из катушки индуктивностью $L$ и конденсатора ёмкостью $C$.
    Параллельно конденсатору подключают ещё один конденсатор ёмкостью $\frac12C$.
    Как изменится период свободных колебаний в контуре?
}
\answer{%
    $
        T = 2\pi\sqrt{LC}, \quad
        T' = 2\pi\sqrt{L'C'}
            = T \sqrt{\frac{L'}L \cdot \frac{C'}C}
            = T \sqrt{ 1 \cdot \frac32 }
        \implies \frac{T'}T = \sqrt{ 1 \cdot \frac32 } \approx 1{,}225.
    $
}
\solutionspace{100pt}

\tasknumber{5}%
\task{%
    В колебательном контуре частота собственных колебаний $40\,\text{Гц}$.
    После замены катушки индуктивности на другую катушку частота стала равной $50\,\text{Гц}$.
    А какой станет частота, если в контур установить обе эти катушки параллельно?
}
\answer{%
    \begin{align*}
    T &= 2\pi\sqrt{LC} \implies \nu = \frac 1T = \frac 1{2\pi\sqrt{LC}} \implies L = \frac1 {4\pi^2 \nu^2 C}, \\
    L_1 &= \frac1 {4\pi^2 \nu_1^2 C}, L_2 = \frac1 {4\pi^2 \nu_1^2 C}, \\
    \nu_\text{послед.}
            &= \frac 1{2\pi\sqrt{(L_1 + L_2)C}}
            = \frac 1{2\pi\sqrt{\cbr{\frac1 {4\pi^2 \nu_1^2 C} + \frac1 {4\pi^2 \nu_2^2 C}}C}}
            = \frac 1{\sqrt{\cbr{\frac1 {\nu_1^2 C} + \frac1 {\nu_2^2 C}}C}} =  \\
     &= \frac 1{\sqrt{\frac1 {\nu_1^2} + \frac1 {\nu_2^2}}}
            = \frac 1{\sqrt{ \frac1 {\sqr{40\,\text{Гц}}} + \frac1 {\sqr{50\,\text{Гц}}}}}
            \approx 31{,}23\,\text{Гц}, \\
    \nu_\text{паралл.}
            &= \frac 1{2\pi\sqrt{\frac 1{\frac 1{L_1} + \frac 1{L_2}}C}}
            = \frac 1{2\pi\sqrt{\frac 1{\frac 1{\frac1 {4\pi^2 \nu_1^2 C}} + \frac 1{\frac1 {4\pi^2 \nu_2^2 C}}}C}}
            = \frac 1{2\pi\sqrt{\frac 1{4\pi^2 \nu_1^2 C + 4\pi^2 \nu_2^2 C}C}} = \\
     &= \frac 1{\sqrt{\frac 1{\nu_1^2 + \nu_2^2}}}
            = \sqrt{\nu_1^2 + \nu_2^2} = \sqrt{\sqr{40\,\text{Гц}} + \sqr{50\,\text{Гц}}} \approx 64{,}03\,\text{Гц}.
    \end{align*}
}

\variantsplitter

\addpersonalvariant{Алексей Алимпиев}

\tasknumber{1}%
\task{%
    Схематично изобразите колебательный контур.
    Запишите формулу для периода колебаний в колебательном контуре и ...
    \begin{itemize}
        \item подпишите все физические величины,
        \item укажите их единицы измерения,
        \item выразите из формулы периода циклическую частоту,
        \item выразите из формулы периода индуктивность катушки.
    \end{itemize}
}
\answer{%
    $
        T = 2\pi\sqrt{LC},
        \nu = \frac 1{2\pi\sqrt{LC}},
        \omega = \frac 1{\sqrt{LC}},
        L = \frac 1C \sqr{\frac T{2\pi}},
        C = \frac 1L \sqr{\frac T{2\pi}}.
    $
}
\solutionspace{80pt}

\tasknumber{2}%
\task{%
    Оказалось, что наибольший заряд конденсатора в колебательном контуре равен $60\,\text{мкКл}$,
    а максимальный ток — $270\,\text{мА}$.
    Определите частоту колебаний.
}
\answer{%
    $
        \eli_{\max} = q_{\max}\omega \implies \nu = \frac{\omega}{2\pi} = \frac{\eli_{\max}}{2\pi q} \approx 716{,}2\,\text{Гц}.
    $
}
\solutionspace{80pt}

\tasknumber{3}%
\task{%
    В колебательном контуре сила тока изменяется
    по закону $\eli=0{,}25\cos(18t)$ (в СИ).
    Индуктивность катушки при этом равна $60\,\text{мГн}$.
    Определите:
    \begin{itemize}
        \item период колебаний,
        \item ёмкость конденсатора,
        \item максимальный заряд конденсатора.
    \end{itemize}
}
\answer{%
    \begin{align*}
    \omega &= 18\funits{рад}{c}, \qquad \eli_{\max} = 0{,}25\,\text{A}, \\
    T &= \frac{2\pi}\omega \approx 349{,}1\,\text{мc}, \\
    C &= \frac 1{\omega^2 L} \approx 51{,}4\,\text{мФ}, \\
    q_{\max} &= \frac{\eli_{\max}}\omega  \approx 13{,}9\,\text{мКл}.
    \end{align*}
}
\solutionspace{80pt}

\tasknumber{4}%
\task{%
    Электрический колебательный контур состоит
    из катушки индуктивностью $L$ и конденсатора ёмкостью $C$.
    Последовательно конденсатору подключают ещё один конденсатор ёмкостью $\frac12C$.
    Как изменится период свободных колебаний в контуре?
}
\answer{%
    $
        T = 2\pi\sqrt{LC}, \quad
        T' = 2\pi\sqrt{L'C'}
            = T \sqrt{\frac{L'}L \cdot \frac{C'}C}
            = T \sqrt{ 1 \cdot \frac13 }
        \implies \frac{T'}T = \sqrt{ 1 \cdot \frac13 } \approx 0{,}577.
    $
}
\solutionspace{100pt}

\tasknumber{5}%
\task{%
    В колебательном контуре частота собственных колебаний $40\,\text{Гц}$.
    После замены катушки индуктивности на другую катушку частота стала равной $30\,\text{Гц}$.
    А какой станет частота, если в контур установить обе эти катушки последовательно?
}
\answer{%
    \begin{align*}
    T &= 2\pi\sqrt{LC} \implies \nu = \frac 1T = \frac 1{2\pi\sqrt{LC}} \implies L = \frac1 {4\pi^2 \nu^2 C}, \\
    L_1 &= \frac1 {4\pi^2 \nu_1^2 C}, L_2 = \frac1 {4\pi^2 \nu_1^2 C}, \\
    \nu_\text{послед.}
            &= \frac 1{2\pi\sqrt{(L_1 + L_2)C}}
            = \frac 1{2\pi\sqrt{\cbr{\frac1 {4\pi^2 \nu_1^2 C} + \frac1 {4\pi^2 \nu_2^2 C}}C}}
            = \frac 1{\sqrt{\cbr{\frac1 {\nu_1^2 C} + \frac1 {\nu_2^2 C}}C}} =  \\
     &= \frac 1{\sqrt{\frac1 {\nu_1^2} + \frac1 {\nu_2^2}}}
            = \frac 1{\sqrt{ \frac1 {\sqr{40\,\text{Гц}}} + \frac1 {\sqr{30\,\text{Гц}}}}}
            \approx 24\,\text{Гц}, \\
    \nu_\text{паралл.}
            &= \frac 1{2\pi\sqrt{\frac 1{\frac 1{L_1} + \frac 1{L_2}}C}}
            = \frac 1{2\pi\sqrt{\frac 1{\frac 1{\frac1 {4\pi^2 \nu_1^2 C}} + \frac 1{\frac1 {4\pi^2 \nu_2^2 C}}}C}}
            = \frac 1{2\pi\sqrt{\frac 1{4\pi^2 \nu_1^2 C + 4\pi^2 \nu_2^2 C}C}} = \\
     &= \frac 1{\sqrt{\frac 1{\nu_1^2 + \nu_2^2}}}
            = \sqrt{\nu_1^2 + \nu_2^2} = \sqrt{\sqr{40\,\text{Гц}} + \sqr{30\,\text{Гц}}} \approx 50\,\text{Гц}.
    \end{align*}
}

\variantsplitter

\addpersonalvariant{Евгений Васин}

\tasknumber{1}%
\task{%
    Схематично изобразите колебательный контур.
    Запишите формулу для периода колебаний в колебательном контуре и ...
    \begin{itemize}
        \item подпишите все физические величины,
        \item укажите их единицы измерения,
        \item выразите из формулы периода циклическую частоту,
        \item выразите из формулы периода индуктивность катушки.
    \end{itemize}
}
\answer{%
    $
        T = 2\pi\sqrt{LC},
        \nu = \frac 1{2\pi\sqrt{LC}},
        \omega = \frac 1{\sqrt{LC}},
        L = \frac 1C \sqr{\frac T{2\pi}},
        C = \frac 1L \sqr{\frac T{2\pi}}.
    $
}
\solutionspace{80pt}

\tasknumber{2}%
\task{%
    Оказалось, что наибольший заряд конденсатора в колебательном контуре равен $40\,\text{мкКл}$,
    а максимальный ток — $270\,\text{мА}$.
    Определите частоту колебаний.
}
\answer{%
    $
        \eli_{\max} = q_{\max}\omega \implies \nu = \frac{\omega}{2\pi} = \frac{\eli_{\max}}{2\pi q} \approx 1074{,}3\,\text{Гц}.
    $
}
\solutionspace{80pt}

\tasknumber{3}%
\task{%
    В колебательном контуре сила тока изменяется
    по закону $\eli=0{,}05\sin(18t)$ (в СИ).
    Индуктивность катушки при этом равна $70\,\text{мГн}$.
    Определите:
    \begin{itemize}
        \item период колебаний,
        \item ёмкость конденсатора,
        \item максимальный заряд конденсатора.
    \end{itemize}
}
\answer{%
    \begin{align*}
    \omega &= 18\funits{рад}{c}, \qquad \eli_{\max} = 0{,}05\,\text{A}, \\
    T &= \frac{2\pi}\omega \approx 349{,}1\,\text{мc}, \\
    C &= \frac 1{\omega^2 L} \approx 44{,}1\,\text{мФ}, \\
    q_{\max} &= \frac{\eli_{\max}}\omega  \approx 2{,}8\,\text{мКл}.
    \end{align*}
}
\solutionspace{80pt}

\tasknumber{4}%
\task{%
    Электрический колебательный контур состоит
    из катушки индуктивностью $L$ и конденсатора ёмкостью $C$.
    Параллельно катушке подключают ещё одну катушку индуктивностью $\frac13L$.
    Как изменится период свободных колебаний в контуре?
}
\answer{%
    $
        T = 2\pi\sqrt{LC}, \quad
        T' = 2\pi\sqrt{L'C'}
            = T \sqrt{\frac{L'}L \cdot \frac{C'}C}
            = T \sqrt{ \frac14 \cdot 1 }
        \implies \frac{T'}T = \sqrt{ \frac14 \cdot 1 } \approx 0{,}500.
    $
}
\solutionspace{100pt}

\tasknumber{5}%
\task{%
    В колебательном контуре частота собственных колебаний $40\,\text{Гц}$.
    После замены катушки индуктивности на другую катушку частота стала равной $70\,\text{Гц}$.
    А какой станет частота, если в контур установить обе эти катушки последовательно?
}
\answer{%
    \begin{align*}
    T &= 2\pi\sqrt{LC} \implies \nu = \frac 1T = \frac 1{2\pi\sqrt{LC}} \implies L = \frac1 {4\pi^2 \nu^2 C}, \\
    L_1 &= \frac1 {4\pi^2 \nu_1^2 C}, L_2 = \frac1 {4\pi^2 \nu_1^2 C}, \\
    \nu_\text{послед.}
            &= \frac 1{2\pi\sqrt{(L_1 + L_2)C}}
            = \frac 1{2\pi\sqrt{\cbr{\frac1 {4\pi^2 \nu_1^2 C} + \frac1 {4\pi^2 \nu_2^2 C}}C}}
            = \frac 1{\sqrt{\cbr{\frac1 {\nu_1^2 C} + \frac1 {\nu_2^2 C}}C}} =  \\
     &= \frac 1{\sqrt{\frac1 {\nu_1^2} + \frac1 {\nu_2^2}}}
            = \frac 1{\sqrt{ \frac1 {\sqr{40\,\text{Гц}}} + \frac1 {\sqr{70\,\text{Гц}}}}}
            \approx 34{,}73\,\text{Гц}, \\
    \nu_\text{паралл.}
            &= \frac 1{2\pi\sqrt{\frac 1{\frac 1{L_1} + \frac 1{L_2}}C}}
            = \frac 1{2\pi\sqrt{\frac 1{\frac 1{\frac1 {4\pi^2 \nu_1^2 C}} + \frac 1{\frac1 {4\pi^2 \nu_2^2 C}}}C}}
            = \frac 1{2\pi\sqrt{\frac 1{4\pi^2 \nu_1^2 C + 4\pi^2 \nu_2^2 C}C}} = \\
     &= \frac 1{\sqrt{\frac 1{\nu_1^2 + \nu_2^2}}}
            = \sqrt{\nu_1^2 + \nu_2^2} = \sqrt{\sqr{40\,\text{Гц}} + \sqr{70\,\text{Гц}}} \approx 80{,}62\,\text{Гц}.
    \end{align*}
}

\variantsplitter

\addpersonalvariant{Вячеслав Волохов}

\tasknumber{1}%
\task{%
    Схематично изобразите колебательный контур.
    Запишите формулу для периода колебаний в колебательном контуре и ...
    \begin{itemize}
        \item подпишите все физические величины,
        \item укажите их единицы измерения,
        \item выразите из формулы периода частоту,
        \item выразите из формулы периода индуктивность катушки.
    \end{itemize}
}
\answer{%
    $
        T = 2\pi\sqrt{LC},
        \nu = \frac 1{2\pi\sqrt{LC}},
        \omega = \frac 1{\sqrt{LC}},
        L = \frac 1C \sqr{\frac T{2\pi}},
        C = \frac 1L \sqr{\frac T{2\pi}}.
    $
}
\solutionspace{80pt}

\tasknumber{2}%
\task{%
    Оказалось, что наибольший заряд конденсатора в колебательном контуре равен $40\,\text{мкКл}$,
    а максимальный ток — $270\,\text{мА}$.
    Определите частоту колебаний.
}
\answer{%
    $
        \eli_{\max} = q_{\max}\omega \implies \nu = \frac{\omega}{2\pi} = \frac{\eli_{\max}}{2\pi q} \approx 1074{,}3\,\text{Гц}.
    $
}
\solutionspace{80pt}

\tasknumber{3}%
\task{%
    В колебательном контуре сила тока изменяется
    по закону $\eli=0{,}05\sin(18t)$ (в СИ).
    Индуктивность катушки при этом равна $80\,\text{мГн}$.
    Определите:
    \begin{itemize}
        \item период колебаний,
        \item ёмкость конденсатора,
        \item максимальный заряд конденсатора.
    \end{itemize}
}
\answer{%
    \begin{align*}
    \omega &= 18\funits{рад}{c}, \qquad \eli_{\max} = 0{,}05\,\text{A}, \\
    T &= \frac{2\pi}\omega \approx 349{,}1\,\text{мc}, \\
    C &= \frac 1{\omega^2 L} \approx 38{,}6\,\text{мФ}, \\
    q_{\max} &= \frac{\eli_{\max}}\omega  \approx 2{,}8\,\text{мКл}.
    \end{align*}
}
\solutionspace{80pt}

\tasknumber{4}%
\task{%
    Электрический колебательный контур состоит
    из катушки индуктивностью $L$ и конденсатора ёмкостью $C$.
    Последовательно катушке подключают ещё одну катушку индуктивностью $\frac13L$.
    Как изменится период свободных колебаний в контуре?
}
\answer{%
    $
        T = 2\pi\sqrt{LC}, \quad
        T' = 2\pi\sqrt{L'C'}
            = T \sqrt{\frac{L'}L \cdot \frac{C'}C}
            = T \sqrt{ \frac43 \cdot 1 }
        \implies \frac{T'}T = \sqrt{ \frac43 \cdot 1 } \approx 1{,}155.
    $
}
\solutionspace{100pt}

\tasknumber{5}%
\task{%
    В колебательном контуре частота собственных колебаний $40\,\text{Гц}$.
    После замены катушки индуктивности на другую катушку частота стала равной $30\,\text{Гц}$.
    А какой станет частота, если в контур установить обе эти катушки параллельно?
}
\answer{%
    \begin{align*}
    T &= 2\pi\sqrt{LC} \implies \nu = \frac 1T = \frac 1{2\pi\sqrt{LC}} \implies L = \frac1 {4\pi^2 \nu^2 C}, \\
    L_1 &= \frac1 {4\pi^2 \nu_1^2 C}, L_2 = \frac1 {4\pi^2 \nu_1^2 C}, \\
    \nu_\text{послед.}
            &= \frac 1{2\pi\sqrt{(L_1 + L_2)C}}
            = \frac 1{2\pi\sqrt{\cbr{\frac1 {4\pi^2 \nu_1^2 C} + \frac1 {4\pi^2 \nu_2^2 C}}C}}
            = \frac 1{\sqrt{\cbr{\frac1 {\nu_1^2 C} + \frac1 {\nu_2^2 C}}C}} =  \\
     &= \frac 1{\sqrt{\frac1 {\nu_1^2} + \frac1 {\nu_2^2}}}
            = \frac 1{\sqrt{ \frac1 {\sqr{40\,\text{Гц}}} + \frac1 {\sqr{30\,\text{Гц}}}}}
            \approx 24\,\text{Гц}, \\
    \nu_\text{паралл.}
            &= \frac 1{2\pi\sqrt{\frac 1{\frac 1{L_1} + \frac 1{L_2}}C}}
            = \frac 1{2\pi\sqrt{\frac 1{\frac 1{\frac1 {4\pi^2 \nu_1^2 C}} + \frac 1{\frac1 {4\pi^2 \nu_2^2 C}}}C}}
            = \frac 1{2\pi\sqrt{\frac 1{4\pi^2 \nu_1^2 C + 4\pi^2 \nu_2^2 C}C}} = \\
     &= \frac 1{\sqrt{\frac 1{\nu_1^2 + \nu_2^2}}}
            = \sqrt{\nu_1^2 + \nu_2^2} = \sqrt{\sqr{40\,\text{Гц}} + \sqr{30\,\text{Гц}}} \approx 50\,\text{Гц}.
    \end{align*}
}

\variantsplitter

\addpersonalvariant{Герман Говоров}

\tasknumber{1}%
\task{%
    Схематично изобразите колебательный контур.
    Запишите формулу для периода колебаний в колебательном контуре и ...
    \begin{itemize}
        \item подпишите все физические величины,
        \item укажите их единицы измерения,
        \item выразите из формулы периода частоту,
        \item выразите из формулы периода ёмкость конденсатора.
    \end{itemize}
}
\answer{%
    $
        T = 2\pi\sqrt{LC},
        \nu = \frac 1{2\pi\sqrt{LC}},
        \omega = \frac 1{\sqrt{LC}},
        L = \frac 1C \sqr{\frac T{2\pi}},
        C = \frac 1L \sqr{\frac T{2\pi}}.
    $
}
\solutionspace{80pt}

\tasknumber{2}%
\task{%
    Оказалось, что наибольший заряд конденсатора в колебательном контуре равен $60\,\text{мкКл}$,
    а максимальный ток — $120\,\text{мА}$.
    Определите частоту колебаний.
}
\answer{%
    $
        \eli_{\max} = q_{\max}\omega \implies \nu = \frac{\omega}{2\pi} = \frac{\eli_{\max}}{2\pi q} \approx 318{,}3\,\text{Гц}.
    $
}
\solutionspace{80pt}

\tasknumber{3}%
\task{%
    В колебательном контуре сила тока изменяется
    по закону $\eli=0{,}30\sin(18t)$ (в СИ).
    Индуктивность катушки при этом равна $70\,\text{мГн}$.
    Определите:
    \begin{itemize}
        \item период колебаний,
        \item ёмкость конденсатора,
        \item максимальный заряд конденсатора.
    \end{itemize}
}
\answer{%
    \begin{align*}
    \omega &= 18\funits{рад}{c}, \qquad \eli_{\max} = 0{,}30\,\text{A}, \\
    T &= \frac{2\pi}\omega \approx 349{,}1\,\text{мc}, \\
    C &= \frac 1{\omega^2 L} \approx 44{,}1\,\text{мФ}, \\
    q_{\max} &= \frac{\eli_{\max}}\omega  \approx 16{,}7\,\text{мКл}.
    \end{align*}
}
\solutionspace{80pt}

\tasknumber{4}%
\task{%
    Электрический колебательный контур состоит
    из катушки индуктивностью $L$ и конденсатора ёмкостью $C$.
    Последовательно конденсатору подключают ещё один конденсатор ёмкостью $\frac13C$.
    Как изменится период свободных колебаний в контуре?
}
\answer{%
    $
        T = 2\pi\sqrt{LC}, \quad
        T' = 2\pi\sqrt{L'C'}
            = T \sqrt{\frac{L'}L \cdot \frac{C'}C}
            = T \sqrt{ 1 \cdot \frac14 }
        \implies \frac{T'}T = \sqrt{ 1 \cdot \frac14 } \approx 0{,}500.
    $
}
\solutionspace{100pt}

\tasknumber{5}%
\task{%
    В колебательном контуре частота собственных колебаний $80\,\text{Гц}$.
    После замены катушки индуктивности на другую катушку частота стала равной $30\,\text{Гц}$.
    А какой станет частота, если в контур установить обе эти катушки последовательно?
}
\answer{%
    \begin{align*}
    T &= 2\pi\sqrt{LC} \implies \nu = \frac 1T = \frac 1{2\pi\sqrt{LC}} \implies L = \frac1 {4\pi^2 \nu^2 C}, \\
    L_1 &= \frac1 {4\pi^2 \nu_1^2 C}, L_2 = \frac1 {4\pi^2 \nu_1^2 C}, \\
    \nu_\text{послед.}
            &= \frac 1{2\pi\sqrt{(L_1 + L_2)C}}
            = \frac 1{2\pi\sqrt{\cbr{\frac1 {4\pi^2 \nu_1^2 C} + \frac1 {4\pi^2 \nu_2^2 C}}C}}
            = \frac 1{\sqrt{\cbr{\frac1 {\nu_1^2 C} + \frac1 {\nu_2^2 C}}C}} =  \\
     &= \frac 1{\sqrt{\frac1 {\nu_1^2} + \frac1 {\nu_2^2}}}
            = \frac 1{\sqrt{ \frac1 {\sqr{80\,\text{Гц}}} + \frac1 {\sqr{30\,\text{Гц}}}}}
            \approx 28{,}09\,\text{Гц}, \\
    \nu_\text{паралл.}
            &= \frac 1{2\pi\sqrt{\frac 1{\frac 1{L_1} + \frac 1{L_2}}C}}
            = \frac 1{2\pi\sqrt{\frac 1{\frac 1{\frac1 {4\pi^2 \nu_1^2 C}} + \frac 1{\frac1 {4\pi^2 \nu_2^2 C}}}C}}
            = \frac 1{2\pi\sqrt{\frac 1{4\pi^2 \nu_1^2 C + 4\pi^2 \nu_2^2 C}C}} = \\
     &= \frac 1{\sqrt{\frac 1{\nu_1^2 + \nu_2^2}}}
            = \sqrt{\nu_1^2 + \nu_2^2} = \sqrt{\sqr{80\,\text{Гц}} + \sqr{30\,\text{Гц}}} \approx 85{,}44\,\text{Гц}.
    \end{align*}
}

\variantsplitter

\addpersonalvariant{София Журавлёва}

\tasknumber{1}%
\task{%
    Схематично изобразите колебательный контур.
    Запишите формулу для периода колебаний в колебательном контуре и ...
    \begin{itemize}
        \item подпишите все физические величины,
        \item укажите их единицы измерения,
        \item выразите из формулы периода частоту,
        \item выразите из формулы периода индуктивность катушки.
    \end{itemize}
}
\answer{%
    $
        T = 2\pi\sqrt{LC},
        \nu = \frac 1{2\pi\sqrt{LC}},
        \omega = \frac 1{\sqrt{LC}},
        L = \frac 1C \sqr{\frac T{2\pi}},
        C = \frac 1L \sqr{\frac T{2\pi}}.
    $
}
\solutionspace{80pt}

\tasknumber{2}%
\task{%
    Оказалось, что наибольший заряд конденсатора в колебательном контуре равен $80\,\text{мкКл}$,
    а максимальный ток — $120\,\text{мА}$.
    Определите частоту колебаний.
}
\answer{%
    $
        \eli_{\max} = q_{\max}\omega \implies \nu = \frac{\omega}{2\pi} = \frac{\eli_{\max}}{2\pi q} \approx 238{,}7\,\text{Гц}.
    $
}
\solutionspace{80pt}

\tasknumber{3}%
\task{%
    В колебательном контуре сила тока изменяется
    по закону $\eli=0{,}25\sin(15t)$ (в СИ).
    Индуктивность катушки при этом равна $60\,\text{мГн}$.
    Определите:
    \begin{itemize}
        \item период колебаний,
        \item ёмкость конденсатора,
        \item максимальный заряд конденсатора.
    \end{itemize}
}
\answer{%
    \begin{align*}
    \omega &= 15\funits{рад}{c}, \qquad \eli_{\max} = 0{,}25\,\text{A}, \\
    T &= \frac{2\pi}\omega \approx 418{,}9\,\text{мc}, \\
    C &= \frac 1{\omega^2 L} \approx 74{,}1\,\text{мФ}, \\
    q_{\max} &= \frac{\eli_{\max}}\omega  \approx 16{,}7\,\text{мКл}.
    \end{align*}
}
\solutionspace{80pt}

\tasknumber{4}%
\task{%
    Электрический колебательный контур состоит
    из катушки индуктивностью $L$ и конденсатора ёмкостью $C$.
    Параллельно конденсатору подключают ещё один конденсатор ёмкостью $\frac12C$.
    Как изменится период свободных колебаний в контуре?
}
\answer{%
    $
        T = 2\pi\sqrt{LC}, \quad
        T' = 2\pi\sqrt{L'C'}
            = T \sqrt{\frac{L'}L \cdot \frac{C'}C}
            = T \sqrt{ 1 \cdot \frac32 }
        \implies \frac{T'}T = \sqrt{ 1 \cdot \frac32 } \approx 1{,}225.
    $
}
\solutionspace{100pt}

\tasknumber{5}%
\task{%
    В колебательном контуре частота собственных колебаний $80\,\text{Гц}$.
    После замены катушки индуктивности на другую катушку частота стала равной $90\,\text{Гц}$.
    А какой станет частота, если в контур установить обе эти катушки параллельно?
}
\answer{%
    \begin{align*}
    T &= 2\pi\sqrt{LC} \implies \nu = \frac 1T = \frac 1{2\pi\sqrt{LC}} \implies L = \frac1 {4\pi^2 \nu^2 C}, \\
    L_1 &= \frac1 {4\pi^2 \nu_1^2 C}, L_2 = \frac1 {4\pi^2 \nu_1^2 C}, \\
    \nu_\text{послед.}
            &= \frac 1{2\pi\sqrt{(L_1 + L_2)C}}
            = \frac 1{2\pi\sqrt{\cbr{\frac1 {4\pi^2 \nu_1^2 C} + \frac1 {4\pi^2 \nu_2^2 C}}C}}
            = \frac 1{\sqrt{\cbr{\frac1 {\nu_1^2 C} + \frac1 {\nu_2^2 C}}C}} =  \\
     &= \frac 1{\sqrt{\frac1 {\nu_1^2} + \frac1 {\nu_2^2}}}
            = \frac 1{\sqrt{ \frac1 {\sqr{80\,\text{Гц}}} + \frac1 {\sqr{90\,\text{Гц}}}}}
            \approx 59{,}79\,\text{Гц}, \\
    \nu_\text{паралл.}
            &= \frac 1{2\pi\sqrt{\frac 1{\frac 1{L_1} + \frac 1{L_2}}C}}
            = \frac 1{2\pi\sqrt{\frac 1{\frac 1{\frac1 {4\pi^2 \nu_1^2 C}} + \frac 1{\frac1 {4\pi^2 \nu_2^2 C}}}C}}
            = \frac 1{2\pi\sqrt{\frac 1{4\pi^2 \nu_1^2 C + 4\pi^2 \nu_2^2 C}C}} = \\
     &= \frac 1{\sqrt{\frac 1{\nu_1^2 + \nu_2^2}}}
            = \sqrt{\nu_1^2 + \nu_2^2} = \sqrt{\sqr{80\,\text{Гц}} + \sqr{90\,\text{Гц}}} \approx 120{,}42\,\text{Гц}.
    \end{align*}
}

\variantsplitter

\addpersonalvariant{Константин Козлов}

\tasknumber{1}%
\task{%
    Схематично изобразите колебательный контур.
    Запишите формулу для периода колебаний в колебательном контуре и ...
    \begin{itemize}
        \item подпишите все физические величины,
        \item укажите их единицы измерения,
        \item выразите из формулы периода частоту,
        \item выразите из формулы периода индуктивность катушки.
    \end{itemize}
}
\answer{%
    $
        T = 2\pi\sqrt{LC},
        \nu = \frac 1{2\pi\sqrt{LC}},
        \omega = \frac 1{\sqrt{LC}},
        L = \frac 1C \sqr{\frac T{2\pi}},
        C = \frac 1L \sqr{\frac T{2\pi}}.
    $
}
\solutionspace{80pt}

\tasknumber{2}%
\task{%
    Оказалось, что наибольший заряд конденсатора в колебательном контуре равен $80\,\text{мкКл}$,
    а максимальный ток — $240\,\text{мА}$.
    Определите частоту колебаний.
}
\answer{%
    $
        \eli_{\max} = q_{\max}\omega \implies \nu = \frac{\omega}{2\pi} = \frac{\eli_{\max}}{2\pi q} \approx 477{,}5\,\text{Гц}.
    $
}
\solutionspace{80pt}

\tasknumber{3}%
\task{%
    В колебательном контуре сила тока изменяется
    по закону $\eli=0{,}25\cos(18t)$ (в СИ).
    Индуктивность катушки при этом равна $70\,\text{мГн}$.
    Определите:
    \begin{itemize}
        \item период колебаний,
        \item ёмкость конденсатора,
        \item максимальный заряд конденсатора.
    \end{itemize}
}
\answer{%
    \begin{align*}
    \omega &= 18\funits{рад}{c}, \qquad \eli_{\max} = 0{,}25\,\text{A}, \\
    T &= \frac{2\pi}\omega \approx 349{,}1\,\text{мc}, \\
    C &= \frac 1{\omega^2 L} \approx 44{,}1\,\text{мФ}, \\
    q_{\max} &= \frac{\eli_{\max}}\omega  \approx 13{,}9\,\text{мКл}.
    \end{align*}
}
\solutionspace{80pt}

\tasknumber{4}%
\task{%
    Электрический колебательный контур состоит
    из катушки индуктивностью $L$ и конденсатора ёмкостью $C$.
    Последовательно конденсатору подключают ещё один конденсатор ёмкостью $\frac13C$.
    Как изменится период свободных колебаний в контуре?
}
\answer{%
    $
        T = 2\pi\sqrt{LC}, \quad
        T' = 2\pi\sqrt{L'C'}
            = T \sqrt{\frac{L'}L \cdot \frac{C'}C}
            = T \sqrt{ 1 \cdot \frac14 }
        \implies \frac{T'}T = \sqrt{ 1 \cdot \frac14 } \approx 0{,}500.
    $
}
\solutionspace{100pt}

\tasknumber{5}%
\task{%
    В колебательном контуре частота собственных колебаний $60\,\text{Гц}$.
    После замены катушки индуктивности на другую катушку частота стала равной $90\,\text{Гц}$.
    А какой станет частота, если в контур установить обе эти катушки последовательно?
}
\answer{%
    \begin{align*}
    T &= 2\pi\sqrt{LC} \implies \nu = \frac 1T = \frac 1{2\pi\sqrt{LC}} \implies L = \frac1 {4\pi^2 \nu^2 C}, \\
    L_1 &= \frac1 {4\pi^2 \nu_1^2 C}, L_2 = \frac1 {4\pi^2 \nu_1^2 C}, \\
    \nu_\text{послед.}
            &= \frac 1{2\pi\sqrt{(L_1 + L_2)C}}
            = \frac 1{2\pi\sqrt{\cbr{\frac1 {4\pi^2 \nu_1^2 C} + \frac1 {4\pi^2 \nu_2^2 C}}C}}
            = \frac 1{\sqrt{\cbr{\frac1 {\nu_1^2 C} + \frac1 {\nu_2^2 C}}C}} =  \\
     &= \frac 1{\sqrt{\frac1 {\nu_1^2} + \frac1 {\nu_2^2}}}
            = \frac 1{\sqrt{ \frac1 {\sqr{60\,\text{Гц}}} + \frac1 {\sqr{90\,\text{Гц}}}}}
            \approx 49{,}92\,\text{Гц}, \\
    \nu_\text{паралл.}
            &= \frac 1{2\pi\sqrt{\frac 1{\frac 1{L_1} + \frac 1{L_2}}C}}
            = \frac 1{2\pi\sqrt{\frac 1{\frac 1{\frac1 {4\pi^2 \nu_1^2 C}} + \frac 1{\frac1 {4\pi^2 \nu_2^2 C}}}C}}
            = \frac 1{2\pi\sqrt{\frac 1{4\pi^2 \nu_1^2 C + 4\pi^2 \nu_2^2 C}C}} = \\
     &= \frac 1{\sqrt{\frac 1{\nu_1^2 + \nu_2^2}}}
            = \sqrt{\nu_1^2 + \nu_2^2} = \sqrt{\sqr{60\,\text{Гц}} + \sqr{90\,\text{Гц}}} \approx 108{,}17\,\text{Гц}.
    \end{align*}
}

\variantsplitter

\addpersonalvariant{Наталья Кравченко}

\tasknumber{1}%
\task{%
    Схематично изобразите колебательный контур.
    Запишите формулу для периода колебаний в колебательном контуре и ...
    \begin{itemize}
        \item подпишите все физические величины,
        \item укажите их единицы измерения,
        \item выразите из формулы периода частоту,
        \item выразите из формулы периода индуктивность катушки.
    \end{itemize}
}
\answer{%
    $
        T = 2\pi\sqrt{LC},
        \nu = \frac 1{2\pi\sqrt{LC}},
        \omega = \frac 1{\sqrt{LC}},
        L = \frac 1C \sqr{\frac T{2\pi}},
        C = \frac 1L \sqr{\frac T{2\pi}}.
    $
}
\solutionspace{80pt}

\tasknumber{2}%
\task{%
    Оказалось, что наибольший заряд конденсатора в колебательном контуре равен $40\,\text{мкКл}$,
    а максимальный ток — $270\,\text{мА}$.
    Определите частоту колебаний.
}
\answer{%
    $
        \eli_{\max} = q_{\max}\omega \implies \nu = \frac{\omega}{2\pi} = \frac{\eli_{\max}}{2\pi q} \approx 1074{,}3\,\text{Гц}.
    $
}
\solutionspace{80pt}

\tasknumber{3}%
\task{%
    В колебательном контуре сила тока изменяется
    по закону $\eli=0{,}30\sin(18t)$ (в СИ).
    Индуктивность катушки при этом равна $50\,\text{мГн}$.
    Определите:
    \begin{itemize}
        \item период колебаний,
        \item ёмкость конденсатора,
        \item максимальный заряд конденсатора.
    \end{itemize}
}
\answer{%
    \begin{align*}
    \omega &= 18\funits{рад}{c}, \qquad \eli_{\max} = 0{,}30\,\text{A}, \\
    T &= \frac{2\pi}\omega \approx 349{,}1\,\text{мc}, \\
    C &= \frac 1{\omega^2 L} \approx 61{,}7\,\text{мФ}, \\
    q_{\max} &= \frac{\eli_{\max}}\omega  \approx 16{,}7\,\text{мКл}.
    \end{align*}
}
\solutionspace{80pt}

\tasknumber{4}%
\task{%
    Электрический колебательный контур состоит
    из катушки индуктивностью $L$ и конденсатора ёмкостью $C$.
    Последовательно конденсатору подключают ещё один конденсатор ёмкостью $\frac12C$.
    Как изменится период свободных колебаний в контуре?
}
\answer{%
    $
        T = 2\pi\sqrt{LC}, \quad
        T' = 2\pi\sqrt{L'C'}
            = T \sqrt{\frac{L'}L \cdot \frac{C'}C}
            = T \sqrt{ 1 \cdot \frac13 }
        \implies \frac{T'}T = \sqrt{ 1 \cdot \frac13 } \approx 0{,}577.
    $
}
\solutionspace{100pt}

\tasknumber{5}%
\task{%
    В колебательном контуре частота собственных колебаний $80\,\text{Гц}$.
    После замены катушки индуктивности на другую катушку частота стала равной $90\,\text{Гц}$.
    А какой станет частота, если в контур установить обе эти катушки параллельно?
}
\answer{%
    \begin{align*}
    T &= 2\pi\sqrt{LC} \implies \nu = \frac 1T = \frac 1{2\pi\sqrt{LC}} \implies L = \frac1 {4\pi^2 \nu^2 C}, \\
    L_1 &= \frac1 {4\pi^2 \nu_1^2 C}, L_2 = \frac1 {4\pi^2 \nu_1^2 C}, \\
    \nu_\text{послед.}
            &= \frac 1{2\pi\sqrt{(L_1 + L_2)C}}
            = \frac 1{2\pi\sqrt{\cbr{\frac1 {4\pi^2 \nu_1^2 C} + \frac1 {4\pi^2 \nu_2^2 C}}C}}
            = \frac 1{\sqrt{\cbr{\frac1 {\nu_1^2 C} + \frac1 {\nu_2^2 C}}C}} =  \\
     &= \frac 1{\sqrt{\frac1 {\nu_1^2} + \frac1 {\nu_2^2}}}
            = \frac 1{\sqrt{ \frac1 {\sqr{80\,\text{Гц}}} + \frac1 {\sqr{90\,\text{Гц}}}}}
            \approx 59{,}79\,\text{Гц}, \\
    \nu_\text{паралл.}
            &= \frac 1{2\pi\sqrt{\frac 1{\frac 1{L_1} + \frac 1{L_2}}C}}
            = \frac 1{2\pi\sqrt{\frac 1{\frac 1{\frac1 {4\pi^2 \nu_1^2 C}} + \frac 1{\frac1 {4\pi^2 \nu_2^2 C}}}C}}
            = \frac 1{2\pi\sqrt{\frac 1{4\pi^2 \nu_1^2 C + 4\pi^2 \nu_2^2 C}C}} = \\
     &= \frac 1{\sqrt{\frac 1{\nu_1^2 + \nu_2^2}}}
            = \sqrt{\nu_1^2 + \nu_2^2} = \sqrt{\sqr{80\,\text{Гц}} + \sqr{90\,\text{Гц}}} \approx 120{,}42\,\text{Гц}.
    \end{align*}
}

\variantsplitter

\addpersonalvariant{Матвей Кузьмин}

\tasknumber{1}%
\task{%
    Схематично изобразите колебательный контур.
    Запишите формулу для периода колебаний в колебательном контуре и ...
    \begin{itemize}
        \item подпишите все физические величины,
        \item укажите их единицы измерения,
        \item выразите из формулы периода циклическую частоту,
        \item выразите из формулы периода ёмкость конденсатора.
    \end{itemize}
}
\answer{%
    $
        T = 2\pi\sqrt{LC},
        \nu = \frac 1{2\pi\sqrt{LC}},
        \omega = \frac 1{\sqrt{LC}},
        L = \frac 1C \sqr{\frac T{2\pi}},
        C = \frac 1L \sqr{\frac T{2\pi}}.
    $
}
\solutionspace{80pt}

\tasknumber{2}%
\task{%
    Оказалось, что наибольший заряд конденсатора в колебательном контуре равен $80\,\text{мкКл}$,
    а максимальный ток — $180\,\text{мА}$.
    Определите частоту колебаний.
}
\answer{%
    $
        \eli_{\max} = q_{\max}\omega \implies \nu = \frac{\omega}{2\pi} = \frac{\eli_{\max}}{2\pi q} \approx 358{,}1\,\text{Гц}.
    $
}
\solutionspace{80pt}

\tasknumber{3}%
\task{%
    В колебательном контуре сила тока изменяется
    по закону $\eli=0{,}05\sin(15t)$ (в СИ).
    Индуктивность катушки при этом равна $80\,\text{мГн}$.
    Определите:
    \begin{itemize}
        \item период колебаний,
        \item ёмкость конденсатора,
        \item максимальный заряд конденсатора.
    \end{itemize}
}
\answer{%
    \begin{align*}
    \omega &= 15\funits{рад}{c}, \qquad \eli_{\max} = 0{,}05\,\text{A}, \\
    T &= \frac{2\pi}\omega \approx 418{,}9\,\text{мc}, \\
    C &= \frac 1{\omega^2 L} \approx 55{,}6\,\text{мФ}, \\
    q_{\max} &= \frac{\eli_{\max}}\omega  \approx 3{,}3\,\text{мКл}.
    \end{align*}
}
\solutionspace{80pt}

\tasknumber{4}%
\task{%
    Электрический колебательный контур состоит
    из катушки индуктивностью $L$ и конденсатора ёмкостью $C$.
    Параллельно катушке подключают ещё одну катушку индуктивностью $\frac12L$.
    Как изменится период свободных колебаний в контуре?
}
\answer{%
    $
        T = 2\pi\sqrt{LC}, \quad
        T' = 2\pi\sqrt{L'C'}
            = T \sqrt{\frac{L'}L \cdot \frac{C'}C}
            = T \sqrt{ \frac13 \cdot 1 }
        \implies \frac{T'}T = \sqrt{ \frac13 \cdot 1 } \approx 0{,}577.
    $
}
\solutionspace{100pt}

\tasknumber{5}%
\task{%
    В колебательном контуре частота собственных колебаний $60\,\text{Гц}$.
    После замены катушки индуктивности на другую катушку частота стала равной $50\,\text{Гц}$.
    А какой станет частота, если в контур установить обе эти катушки последовательно?
}
\answer{%
    \begin{align*}
    T &= 2\pi\sqrt{LC} \implies \nu = \frac 1T = \frac 1{2\pi\sqrt{LC}} \implies L = \frac1 {4\pi^2 \nu^2 C}, \\
    L_1 &= \frac1 {4\pi^2 \nu_1^2 C}, L_2 = \frac1 {4\pi^2 \nu_1^2 C}, \\
    \nu_\text{послед.}
            &= \frac 1{2\pi\sqrt{(L_1 + L_2)C}}
            = \frac 1{2\pi\sqrt{\cbr{\frac1 {4\pi^2 \nu_1^2 C} + \frac1 {4\pi^2 \nu_2^2 C}}C}}
            = \frac 1{\sqrt{\cbr{\frac1 {\nu_1^2 C} + \frac1 {\nu_2^2 C}}C}} =  \\
     &= \frac 1{\sqrt{\frac1 {\nu_1^2} + \frac1 {\nu_2^2}}}
            = \frac 1{\sqrt{ \frac1 {\sqr{60\,\text{Гц}}} + \frac1 {\sqr{50\,\text{Гц}}}}}
            \approx 38{,}41\,\text{Гц}, \\
    \nu_\text{паралл.}
            &= \frac 1{2\pi\sqrt{\frac 1{\frac 1{L_1} + \frac 1{L_2}}C}}
            = \frac 1{2\pi\sqrt{\frac 1{\frac 1{\frac1 {4\pi^2 \nu_1^2 C}} + \frac 1{\frac1 {4\pi^2 \nu_2^2 C}}}C}}
            = \frac 1{2\pi\sqrt{\frac 1{4\pi^2 \nu_1^2 C + 4\pi^2 \nu_2^2 C}C}} = \\
     &= \frac 1{\sqrt{\frac 1{\nu_1^2 + \nu_2^2}}}
            = \sqrt{\nu_1^2 + \nu_2^2} = \sqrt{\sqr{60\,\text{Гц}} + \sqr{50\,\text{Гц}}} \approx 78{,}10\,\text{Гц}.
    \end{align*}
}

\variantsplitter

\addpersonalvariant{Сергей Малышев}

\tasknumber{1}%
\task{%
    Схематично изобразите колебательный контур.
    Запишите формулу для периода колебаний в колебательном контуре и ...
    \begin{itemize}
        \item подпишите все физические величины,
        \item укажите их единицы измерения,
        \item выразите из формулы периода частоту,
        \item выразите из формулы периода ёмкость конденсатора.
    \end{itemize}
}
\answer{%
    $
        T = 2\pi\sqrt{LC},
        \nu = \frac 1{2\pi\sqrt{LC}},
        \omega = \frac 1{\sqrt{LC}},
        L = \frac 1C \sqr{\frac T{2\pi}},
        C = \frac 1L \sqr{\frac T{2\pi}}.
    $
}
\solutionspace{80pt}

\tasknumber{2}%
\task{%
    Оказалось, что наибольший заряд конденсатора в колебательном контуре равен $80\,\text{мкКл}$,
    а максимальный ток — $150\,\text{мА}$.
    Определите частоту колебаний.
}
\answer{%
    $
        \eli_{\max} = q_{\max}\omega \implies \nu = \frac{\omega}{2\pi} = \frac{\eli_{\max}}{2\pi q} \approx 298{,}4\,\text{Гц}.
    $
}
\solutionspace{80pt}

\tasknumber{3}%
\task{%
    В колебательном контуре сила тока изменяется
    по закону $\eli=0{,}30\cos(18t)$ (в СИ).
    Индуктивность катушки при этом равна $60\,\text{мГн}$.
    Определите:
    \begin{itemize}
        \item период колебаний,
        \item ёмкость конденсатора,
        \item максимальный заряд конденсатора.
    \end{itemize}
}
\answer{%
    \begin{align*}
    \omega &= 18\funits{рад}{c}, \qquad \eli_{\max} = 0{,}30\,\text{A}, \\
    T &= \frac{2\pi}\omega \approx 349{,}1\,\text{мc}, \\
    C &= \frac 1{\omega^2 L} \approx 51{,}4\,\text{мФ}, \\
    q_{\max} &= \frac{\eli_{\max}}\omega  \approx 16{,}7\,\text{мКл}.
    \end{align*}
}
\solutionspace{80pt}

\tasknumber{4}%
\task{%
    Электрический колебательный контур состоит
    из катушки индуктивностью $L$ и конденсатора ёмкостью $C$.
    Параллельно катушке подключают ещё одну катушку индуктивностью $\frac12L$.
    Как изменится период свободных колебаний в контуре?
}
\answer{%
    $
        T = 2\pi\sqrt{LC}, \quad
        T' = 2\pi\sqrt{L'C'}
            = T \sqrt{\frac{L'}L \cdot \frac{C'}C}
            = T \sqrt{ \frac13 \cdot 1 }
        \implies \frac{T'}T = \sqrt{ \frac13 \cdot 1 } \approx 0{,}577.
    $
}
\solutionspace{100pt}

\tasknumber{5}%
\task{%
    В колебательном контуре частота собственных колебаний $60\,\text{Гц}$.
    После замены катушки индуктивности на другую катушку частота стала равной $30\,\text{Гц}$.
    А какой станет частота, если в контур установить обе эти катушки параллельно?
}
\answer{%
    \begin{align*}
    T &= 2\pi\sqrt{LC} \implies \nu = \frac 1T = \frac 1{2\pi\sqrt{LC}} \implies L = \frac1 {4\pi^2 \nu^2 C}, \\
    L_1 &= \frac1 {4\pi^2 \nu_1^2 C}, L_2 = \frac1 {4\pi^2 \nu_1^2 C}, \\
    \nu_\text{послед.}
            &= \frac 1{2\pi\sqrt{(L_1 + L_2)C}}
            = \frac 1{2\pi\sqrt{\cbr{\frac1 {4\pi^2 \nu_1^2 C} + \frac1 {4\pi^2 \nu_2^2 C}}C}}
            = \frac 1{\sqrt{\cbr{\frac1 {\nu_1^2 C} + \frac1 {\nu_2^2 C}}C}} =  \\
     &= \frac 1{\sqrt{\frac1 {\nu_1^2} + \frac1 {\nu_2^2}}}
            = \frac 1{\sqrt{ \frac1 {\sqr{60\,\text{Гц}}} + \frac1 {\sqr{30\,\text{Гц}}}}}
            \approx 26{,}83\,\text{Гц}, \\
    \nu_\text{паралл.}
            &= \frac 1{2\pi\sqrt{\frac 1{\frac 1{L_1} + \frac 1{L_2}}C}}
            = \frac 1{2\pi\sqrt{\frac 1{\frac 1{\frac1 {4\pi^2 \nu_1^2 C}} + \frac 1{\frac1 {4\pi^2 \nu_2^2 C}}}C}}
            = \frac 1{2\pi\sqrt{\frac 1{4\pi^2 \nu_1^2 C + 4\pi^2 \nu_2^2 C}C}} = \\
     &= \frac 1{\sqrt{\frac 1{\nu_1^2 + \nu_2^2}}}
            = \sqrt{\nu_1^2 + \nu_2^2} = \sqrt{\sqr{60\,\text{Гц}} + \sqr{30\,\text{Гц}}} \approx 67{,}08\,\text{Гц}.
    \end{align*}
}

\variantsplitter

\addpersonalvariant{Алина Полканова}

\tasknumber{1}%
\task{%
    Схематично изобразите колебательный контур.
    Запишите формулу для периода колебаний в колебательном контуре и ...
    \begin{itemize}
        \item подпишите все физические величины,
        \item укажите их единицы измерения,
        \item выразите из формулы периода циклическую частоту,
        \item выразите из формулы периода индуктивность катушки.
    \end{itemize}
}
\answer{%
    $
        T = 2\pi\sqrt{LC},
        \nu = \frac 1{2\pi\sqrt{LC}},
        \omega = \frac 1{\sqrt{LC}},
        L = \frac 1C \sqr{\frac T{2\pi}},
        C = \frac 1L \sqr{\frac T{2\pi}}.
    $
}
\solutionspace{80pt}

\tasknumber{2}%
\task{%
    Оказалось, что наибольший заряд конденсатора в колебательном контуре равен $60\,\text{мкКл}$,
    а максимальный ток — $270\,\text{мА}$.
    Определите частоту колебаний.
}
\answer{%
    $
        \eli_{\max} = q_{\max}\omega \implies \nu = \frac{\omega}{2\pi} = \frac{\eli_{\max}}{2\pi q} \approx 716{,}2\,\text{Гц}.
    $
}
\solutionspace{80pt}

\tasknumber{3}%
\task{%
    В колебательном контуре сила тока изменяется
    по закону $\eli=0{,}30\sin(15t)$ (в СИ).
    Индуктивность катушки при этом равна $60\,\text{мГн}$.
    Определите:
    \begin{itemize}
        \item период колебаний,
        \item ёмкость конденсатора,
        \item максимальный заряд конденсатора.
    \end{itemize}
}
\answer{%
    \begin{align*}
    \omega &= 15\funits{рад}{c}, \qquad \eli_{\max} = 0{,}30\,\text{A}, \\
    T &= \frac{2\pi}\omega \approx 418{,}9\,\text{мc}, \\
    C &= \frac 1{\omega^2 L} \approx 74{,}1\,\text{мФ}, \\
    q_{\max} &= \frac{\eli_{\max}}\omega  \approx 20\,\text{мКл}.
    \end{align*}
}
\solutionspace{80pt}

\tasknumber{4}%
\task{%
    Электрический колебательный контур состоит
    из катушки индуктивностью $L$ и конденсатора ёмкостью $C$.
    Параллельно конденсатору подключают ещё один конденсатор ёмкостью $\frac13C$.
    Как изменится период свободных колебаний в контуре?
}
\answer{%
    $
        T = 2\pi\sqrt{LC}, \quad
        T' = 2\pi\sqrt{L'C'}
            = T \sqrt{\frac{L'}L \cdot \frac{C'}C}
            = T \sqrt{ 1 \cdot \frac43 }
        \implies \frac{T'}T = \sqrt{ 1 \cdot \frac43 } \approx 1{,}155.
    $
}
\solutionspace{100pt}

\tasknumber{5}%
\task{%
    В колебательном контуре частота собственных колебаний $80\,\text{Гц}$.
    После замены катушки индуктивности на другую катушку частота стала равной $50\,\text{Гц}$.
    А какой станет частота, если в контур установить обе эти катушки параллельно?
}
\answer{%
    \begin{align*}
    T &= 2\pi\sqrt{LC} \implies \nu = \frac 1T = \frac 1{2\pi\sqrt{LC}} \implies L = \frac1 {4\pi^2 \nu^2 C}, \\
    L_1 &= \frac1 {4\pi^2 \nu_1^2 C}, L_2 = \frac1 {4\pi^2 \nu_1^2 C}, \\
    \nu_\text{послед.}
            &= \frac 1{2\pi\sqrt{(L_1 + L_2)C}}
            = \frac 1{2\pi\sqrt{\cbr{\frac1 {4\pi^2 \nu_1^2 C} + \frac1 {4\pi^2 \nu_2^2 C}}C}}
            = \frac 1{\sqrt{\cbr{\frac1 {\nu_1^2 C} + \frac1 {\nu_2^2 C}}C}} =  \\
     &= \frac 1{\sqrt{\frac1 {\nu_1^2} + \frac1 {\nu_2^2}}}
            = \frac 1{\sqrt{ \frac1 {\sqr{80\,\text{Гц}}} + \frac1 {\sqr{50\,\text{Гц}}}}}
            \approx 42{,}40\,\text{Гц}, \\
    \nu_\text{паралл.}
            &= \frac 1{2\pi\sqrt{\frac 1{\frac 1{L_1} + \frac 1{L_2}}C}}
            = \frac 1{2\pi\sqrt{\frac 1{\frac 1{\frac1 {4\pi^2 \nu_1^2 C}} + \frac 1{\frac1 {4\pi^2 \nu_2^2 C}}}C}}
            = \frac 1{2\pi\sqrt{\frac 1{4\pi^2 \nu_1^2 C + 4\pi^2 \nu_2^2 C}C}} = \\
     &= \frac 1{\sqrt{\frac 1{\nu_1^2 + \nu_2^2}}}
            = \sqrt{\nu_1^2 + \nu_2^2} = \sqrt{\sqr{80\,\text{Гц}} + \sqr{50\,\text{Гц}}} \approx 94{,}34\,\text{Гц}.
    \end{align*}
}

\variantsplitter

\addpersonalvariant{Сергей Пономарёв}

\tasknumber{1}%
\task{%
    Схематично изобразите колебательный контур.
    Запишите формулу для периода колебаний в колебательном контуре и ...
    \begin{itemize}
        \item подпишите все физические величины,
        \item укажите их единицы измерения,
        \item выразите из формулы периода циклическую частоту,
        \item выразите из формулы периода индуктивность катушки.
    \end{itemize}
}
\answer{%
    $
        T = 2\pi\sqrt{LC},
        \nu = \frac 1{2\pi\sqrt{LC}},
        \omega = \frac 1{\sqrt{LC}},
        L = \frac 1C \sqr{\frac T{2\pi}},
        C = \frac 1L \sqr{\frac T{2\pi}}.
    $
}
\solutionspace{80pt}

\tasknumber{2}%
\task{%
    Оказалось, что наибольший заряд конденсатора в колебательном контуре равен $60\,\text{мкКл}$,
    а максимальный ток — $120\,\text{мА}$.
    Определите частоту колебаний.
}
\answer{%
    $
        \eli_{\max} = q_{\max}\omega \implies \nu = \frac{\omega}{2\pi} = \frac{\eli_{\max}}{2\pi q} \approx 318{,}3\,\text{Гц}.
    $
}
\solutionspace{80pt}

\tasknumber{3}%
\task{%
    В колебательном контуре сила тока изменяется
    по закону $\eli=0{,}05\cos(15t)$ (в СИ).
    Индуктивность катушки при этом равна $80\,\text{мГн}$.
    Определите:
    \begin{itemize}
        \item период колебаний,
        \item ёмкость конденсатора,
        \item максимальный заряд конденсатора.
    \end{itemize}
}
\answer{%
    \begin{align*}
    \omega &= 15\funits{рад}{c}, \qquad \eli_{\max} = 0{,}05\,\text{A}, \\
    T &= \frac{2\pi}\omega \approx 418{,}9\,\text{мc}, \\
    C &= \frac 1{\omega^2 L} \approx 55{,}6\,\text{мФ}, \\
    q_{\max} &= \frac{\eli_{\max}}\omega  \approx 3{,}3\,\text{мКл}.
    \end{align*}
}
\solutionspace{80pt}

\tasknumber{4}%
\task{%
    Электрический колебательный контур состоит
    из катушки индуктивностью $L$ и конденсатора ёмкостью $C$.
    Параллельно катушке подключают ещё одну катушку индуктивностью $\frac13L$.
    Как изменится период свободных колебаний в контуре?
}
\answer{%
    $
        T = 2\pi\sqrt{LC}, \quad
        T' = 2\pi\sqrt{L'C'}
            = T \sqrt{\frac{L'}L \cdot \frac{C'}C}
            = T \sqrt{ \frac14 \cdot 1 }
        \implies \frac{T'}T = \sqrt{ \frac14 \cdot 1 } \approx 0{,}500.
    $
}
\solutionspace{100pt}

\tasknumber{5}%
\task{%
    В колебательном контуре частота собственных колебаний $60\,\text{Гц}$.
    После замены катушки индуктивности на другую катушку частота стала равной $30\,\text{Гц}$.
    А какой станет частота, если в контур установить обе эти катушки параллельно?
}
\answer{%
    \begin{align*}
    T &= 2\pi\sqrt{LC} \implies \nu = \frac 1T = \frac 1{2\pi\sqrt{LC}} \implies L = \frac1 {4\pi^2 \nu^2 C}, \\
    L_1 &= \frac1 {4\pi^2 \nu_1^2 C}, L_2 = \frac1 {4\pi^2 \nu_1^2 C}, \\
    \nu_\text{послед.}
            &= \frac 1{2\pi\sqrt{(L_1 + L_2)C}}
            = \frac 1{2\pi\sqrt{\cbr{\frac1 {4\pi^2 \nu_1^2 C} + \frac1 {4\pi^2 \nu_2^2 C}}C}}
            = \frac 1{\sqrt{\cbr{\frac1 {\nu_1^2 C} + \frac1 {\nu_2^2 C}}C}} =  \\
     &= \frac 1{\sqrt{\frac1 {\nu_1^2} + \frac1 {\nu_2^2}}}
            = \frac 1{\sqrt{ \frac1 {\sqr{60\,\text{Гц}}} + \frac1 {\sqr{30\,\text{Гц}}}}}
            \approx 26{,}83\,\text{Гц}, \\
    \nu_\text{паралл.}
            &= \frac 1{2\pi\sqrt{\frac 1{\frac 1{L_1} + \frac 1{L_2}}C}}
            = \frac 1{2\pi\sqrt{\frac 1{\frac 1{\frac1 {4\pi^2 \nu_1^2 C}} + \frac 1{\frac1 {4\pi^2 \nu_2^2 C}}}C}}
            = \frac 1{2\pi\sqrt{\frac 1{4\pi^2 \nu_1^2 C + 4\pi^2 \nu_2^2 C}C}} = \\
     &= \frac 1{\sqrt{\frac 1{\nu_1^2 + \nu_2^2}}}
            = \sqrt{\nu_1^2 + \nu_2^2} = \sqrt{\sqr{60\,\text{Гц}} + \sqr{30\,\text{Гц}}} \approx 67{,}08\,\text{Гц}.
    \end{align*}
}

\variantsplitter

\addpersonalvariant{Егор Свистушкин}

\tasknumber{1}%
\task{%
    Схематично изобразите колебательный контур.
    Запишите формулу для периода колебаний в колебательном контуре и ...
    \begin{itemize}
        \item подпишите все физические величины,
        \item укажите их единицы измерения,
        \item выразите из формулы периода частоту,
        \item выразите из формулы периода индуктивность катушки.
    \end{itemize}
}
\answer{%
    $
        T = 2\pi\sqrt{LC},
        \nu = \frac 1{2\pi\sqrt{LC}},
        \omega = \frac 1{\sqrt{LC}},
        L = \frac 1C \sqr{\frac T{2\pi}},
        C = \frac 1L \sqr{\frac T{2\pi}}.
    $
}
\solutionspace{80pt}

\tasknumber{2}%
\task{%
    Оказалось, что наибольший заряд конденсатора в колебательном контуре равен $60\,\text{мкКл}$,
    а максимальный ток — $120\,\text{мА}$.
    Определите частоту колебаний.
}
\answer{%
    $
        \eli_{\max} = q_{\max}\omega \implies \nu = \frac{\omega}{2\pi} = \frac{\eli_{\max}}{2\pi q} \approx 318{,}3\,\text{Гц}.
    $
}
\solutionspace{80pt}

\tasknumber{3}%
\task{%
    В колебательном контуре сила тока изменяется
    по закону $\eli=0{,}25\sin(15t)$ (в СИ).
    Индуктивность катушки при этом равна $70\,\text{мГн}$.
    Определите:
    \begin{itemize}
        \item период колебаний,
        \item ёмкость конденсатора,
        \item максимальный заряд конденсатора.
    \end{itemize}
}
\answer{%
    \begin{align*}
    \omega &= 15\funits{рад}{c}, \qquad \eli_{\max} = 0{,}25\,\text{A}, \\
    T &= \frac{2\pi}\omega \approx 418{,}9\,\text{мc}, \\
    C &= \frac 1{\omega^2 L} \approx 63{,}5\,\text{мФ}, \\
    q_{\max} &= \frac{\eli_{\max}}\omega  \approx 16{,}7\,\text{мКл}.
    \end{align*}
}
\solutionspace{80pt}

\tasknumber{4}%
\task{%
    Электрический колебательный контур состоит
    из катушки индуктивностью $L$ и конденсатора ёмкостью $C$.
    Последовательно катушке подключают ещё одну катушку индуктивностью $2L$.
    Как изменится период свободных колебаний в контуре?
}
\answer{%
    $
        T = 2\pi\sqrt{LC}, \quad
        T' = 2\pi\sqrt{L'C'}
            = T \sqrt{\frac{L'}L \cdot \frac{C'}C}
            = T \sqrt{ 3 \cdot 1 }
        \implies \frac{T'}T = \sqrt{ 3 \cdot 1 } \approx 1{,}732.
    $
}
\solutionspace{100pt}

\tasknumber{5}%
\task{%
    В колебательном контуре частота собственных колебаний $80\,\text{Гц}$.
    После замены катушки индуктивности на другую катушку частота стала равной $50\,\text{Гц}$.
    А какой станет частота, если в контур установить обе эти катушки параллельно?
}
\answer{%
    \begin{align*}
    T &= 2\pi\sqrt{LC} \implies \nu = \frac 1T = \frac 1{2\pi\sqrt{LC}} \implies L = \frac1 {4\pi^2 \nu^2 C}, \\
    L_1 &= \frac1 {4\pi^2 \nu_1^2 C}, L_2 = \frac1 {4\pi^2 \nu_1^2 C}, \\
    \nu_\text{послед.}
            &= \frac 1{2\pi\sqrt{(L_1 + L_2)C}}
            = \frac 1{2\pi\sqrt{\cbr{\frac1 {4\pi^2 \nu_1^2 C} + \frac1 {4\pi^2 \nu_2^2 C}}C}}
            = \frac 1{\sqrt{\cbr{\frac1 {\nu_1^2 C} + \frac1 {\nu_2^2 C}}C}} =  \\
     &= \frac 1{\sqrt{\frac1 {\nu_1^2} + \frac1 {\nu_2^2}}}
            = \frac 1{\sqrt{ \frac1 {\sqr{80\,\text{Гц}}} + \frac1 {\sqr{50\,\text{Гц}}}}}
            \approx 42{,}40\,\text{Гц}, \\
    \nu_\text{паралл.}
            &= \frac 1{2\pi\sqrt{\frac 1{\frac 1{L_1} + \frac 1{L_2}}C}}
            = \frac 1{2\pi\sqrt{\frac 1{\frac 1{\frac1 {4\pi^2 \nu_1^2 C}} + \frac 1{\frac1 {4\pi^2 \nu_2^2 C}}}C}}
            = \frac 1{2\pi\sqrt{\frac 1{4\pi^2 \nu_1^2 C + 4\pi^2 \nu_2^2 C}C}} = \\
     &= \frac 1{\sqrt{\frac 1{\nu_1^2 + \nu_2^2}}}
            = \sqrt{\nu_1^2 + \nu_2^2} = \sqrt{\sqr{80\,\text{Гц}} + \sqr{50\,\text{Гц}}} \approx 94{,}34\,\text{Гц}.
    \end{align*}
}

\variantsplitter

\addpersonalvariant{Дмитрий Соколов}

\tasknumber{1}%
\task{%
    Схематично изобразите колебательный контур.
    Запишите формулу для периода колебаний в колебательном контуре и ...
    \begin{itemize}
        \item подпишите все физические величины,
        \item укажите их единицы измерения,
        \item выразите из формулы периода частоту,
        \item выразите из формулы периода ёмкость конденсатора.
    \end{itemize}
}
\answer{%
    $
        T = 2\pi\sqrt{LC},
        \nu = \frac 1{2\pi\sqrt{LC}},
        \omega = \frac 1{\sqrt{LC}},
        L = \frac 1C \sqr{\frac T{2\pi}},
        C = \frac 1L \sqr{\frac T{2\pi}}.
    $
}
\solutionspace{80pt}

\tasknumber{2}%
\task{%
    Оказалось, что наибольший заряд конденсатора в колебательном контуре равен $80\,\text{мкКл}$,
    а максимальный ток — $120\,\text{мА}$.
    Определите частоту колебаний.
}
\answer{%
    $
        \eli_{\max} = q_{\max}\omega \implies \nu = \frac{\omega}{2\pi} = \frac{\eli_{\max}}{2\pi q} \approx 238{,}7\,\text{Гц}.
    $
}
\solutionspace{80pt}

\tasknumber{3}%
\task{%
    В колебательном контуре сила тока изменяется
    по закону $\eli=0{,}05\sin(15t)$ (в СИ).
    Индуктивность катушки при этом равна $80\,\text{мГн}$.
    Определите:
    \begin{itemize}
        \item период колебаний,
        \item ёмкость конденсатора,
        \item максимальный заряд конденсатора.
    \end{itemize}
}
\answer{%
    \begin{align*}
    \omega &= 15\funits{рад}{c}, \qquad \eli_{\max} = 0{,}05\,\text{A}, \\
    T &= \frac{2\pi}\omega \approx 418{,}9\,\text{мc}, \\
    C &= \frac 1{\omega^2 L} \approx 55{,}6\,\text{мФ}, \\
    q_{\max} &= \frac{\eli_{\max}}\omega  \approx 3{,}3\,\text{мКл}.
    \end{align*}
}
\solutionspace{80pt}

\tasknumber{4}%
\task{%
    Электрический колебательный контур состоит
    из катушки индуктивностью $L$ и конденсатора ёмкостью $C$.
    Последовательно катушке подключают ещё одну катушку индуктивностью $\frac12L$.
    Как изменится период свободных колебаний в контуре?
}
\answer{%
    $
        T = 2\pi\sqrt{LC}, \quad
        T' = 2\pi\sqrt{L'C'}
            = T \sqrt{\frac{L'}L \cdot \frac{C'}C}
            = T \sqrt{ \frac32 \cdot 1 }
        \implies \frac{T'}T = \sqrt{ \frac32 \cdot 1 } \approx 1{,}225.
    $
}
\solutionspace{100pt}

\tasknumber{5}%
\task{%
    В колебательном контуре частота собственных колебаний $40\,\text{Гц}$.
    После замены катушки индуктивности на другую катушку частота стала равной $50\,\text{Гц}$.
    А какой станет частота, если в контур установить обе эти катушки параллельно?
}
\answer{%
    \begin{align*}
    T &= 2\pi\sqrt{LC} \implies \nu = \frac 1T = \frac 1{2\pi\sqrt{LC}} \implies L = \frac1 {4\pi^2 \nu^2 C}, \\
    L_1 &= \frac1 {4\pi^2 \nu_1^2 C}, L_2 = \frac1 {4\pi^2 \nu_1^2 C}, \\
    \nu_\text{послед.}
            &= \frac 1{2\pi\sqrt{(L_1 + L_2)C}}
            = \frac 1{2\pi\sqrt{\cbr{\frac1 {4\pi^2 \nu_1^2 C} + \frac1 {4\pi^2 \nu_2^2 C}}C}}
            = \frac 1{\sqrt{\cbr{\frac1 {\nu_1^2 C} + \frac1 {\nu_2^2 C}}C}} =  \\
     &= \frac 1{\sqrt{\frac1 {\nu_1^2} + \frac1 {\nu_2^2}}}
            = \frac 1{\sqrt{ \frac1 {\sqr{40\,\text{Гц}}} + \frac1 {\sqr{50\,\text{Гц}}}}}
            \approx 31{,}23\,\text{Гц}, \\
    \nu_\text{паралл.}
            &= \frac 1{2\pi\sqrt{\frac 1{\frac 1{L_1} + \frac 1{L_2}}C}}
            = \frac 1{2\pi\sqrt{\frac 1{\frac 1{\frac1 {4\pi^2 \nu_1^2 C}} + \frac 1{\frac1 {4\pi^2 \nu_2^2 C}}}C}}
            = \frac 1{2\pi\sqrt{\frac 1{4\pi^2 \nu_1^2 C + 4\pi^2 \nu_2^2 C}C}} = \\
     &= \frac 1{\sqrt{\frac 1{\nu_1^2 + \nu_2^2}}}
            = \sqrt{\nu_1^2 + \nu_2^2} = \sqrt{\sqr{40\,\text{Гц}} + \sqr{50\,\text{Гц}}} \approx 64{,}03\,\text{Гц}.
    \end{align*}
}

\variantsplitter

\addpersonalvariant{Арсений Трофимов}

\tasknumber{1}%
\task{%
    Схематично изобразите колебательный контур.
    Запишите формулу для периода колебаний в колебательном контуре и ...
    \begin{itemize}
        \item подпишите все физические величины,
        \item укажите их единицы измерения,
        \item выразите из формулы периода частоту,
        \item выразите из формулы периода индуктивность катушки.
    \end{itemize}
}
\answer{%
    $
        T = 2\pi\sqrt{LC},
        \nu = \frac 1{2\pi\sqrt{LC}},
        \omega = \frac 1{\sqrt{LC}},
        L = \frac 1C \sqr{\frac T{2\pi}},
        C = \frac 1L \sqr{\frac T{2\pi}}.
    $
}
\solutionspace{80pt}

\tasknumber{2}%
\task{%
    Оказалось, что наибольший заряд конденсатора в колебательном контуре равен $60\,\text{мкКл}$,
    а максимальный ток — $150\,\text{мА}$.
    Определите частоту колебаний.
}
\answer{%
    $
        \eli_{\max} = q_{\max}\omega \implies \nu = \frac{\omega}{2\pi} = \frac{\eli_{\max}}{2\pi q} \approx 397{,}9\,\text{Гц}.
    $
}
\solutionspace{80pt}

\tasknumber{3}%
\task{%
    В колебательном контуре сила тока изменяется
    по закону $\eli=0{,}30\sin(12t)$ (в СИ).
    Индуктивность катушки при этом равна $60\,\text{мГн}$.
    Определите:
    \begin{itemize}
        \item период колебаний,
        \item ёмкость конденсатора,
        \item максимальный заряд конденсатора.
    \end{itemize}
}
\answer{%
    \begin{align*}
    \omega &= 12\funits{рад}{c}, \qquad \eli_{\max} = 0{,}30\,\text{A}, \\
    T &= \frac{2\pi}\omega \approx 523{,}6\,\text{мc}, \\
    C &= \frac 1{\omega^2 L} \approx 115{,}7\,\text{мФ}, \\
    q_{\max} &= \frac{\eli_{\max}}\omega  \approx 25\,\text{мКл}.
    \end{align*}
}
\solutionspace{80pt}

\tasknumber{4}%
\task{%
    Электрический колебательный контур состоит
    из катушки индуктивностью $L$ и конденсатора ёмкостью $C$.
    Последовательно конденсатору подключают ещё один конденсатор ёмкостью $2C$.
    Как изменится период свободных колебаний в контуре?
}
\answer{%
    $
        T = 2\pi\sqrt{LC}, \quad
        T' = 2\pi\sqrt{L'C'}
            = T \sqrt{\frac{L'}L \cdot \frac{C'}C}
            = T \sqrt{ 1 \cdot \frac23 }
        \implies \frac{T'}T = \sqrt{ 1 \cdot \frac23 } \approx 0{,}816.
    $
}
\solutionspace{100pt}

\tasknumber{5}%
\task{%
    В колебательном контуре частота собственных колебаний $40\,\text{Гц}$.
    После замены катушки индуктивности на другую катушку частота стала равной $70\,\text{Гц}$.
    А какой станет частота, если в контур установить обе эти катушки параллельно?
}
\answer{%
    \begin{align*}
    T &= 2\pi\sqrt{LC} \implies \nu = \frac 1T = \frac 1{2\pi\sqrt{LC}} \implies L = \frac1 {4\pi^2 \nu^2 C}, \\
    L_1 &= \frac1 {4\pi^2 \nu_1^2 C}, L_2 = \frac1 {4\pi^2 \nu_1^2 C}, \\
    \nu_\text{послед.}
            &= \frac 1{2\pi\sqrt{(L_1 + L_2)C}}
            = \frac 1{2\pi\sqrt{\cbr{\frac1 {4\pi^2 \nu_1^2 C} + \frac1 {4\pi^2 \nu_2^2 C}}C}}
            = \frac 1{\sqrt{\cbr{\frac1 {\nu_1^2 C} + \frac1 {\nu_2^2 C}}C}} =  \\
     &= \frac 1{\sqrt{\frac1 {\nu_1^2} + \frac1 {\nu_2^2}}}
            = \frac 1{\sqrt{ \frac1 {\sqr{40\,\text{Гц}}} + \frac1 {\sqr{70\,\text{Гц}}}}}
            \approx 34{,}73\,\text{Гц}, \\
    \nu_\text{паралл.}
            &= \frac 1{2\pi\sqrt{\frac 1{\frac 1{L_1} + \frac 1{L_2}}C}}
            = \frac 1{2\pi\sqrt{\frac 1{\frac 1{\frac1 {4\pi^2 \nu_1^2 C}} + \frac 1{\frac1 {4\pi^2 \nu_2^2 C}}}C}}
            = \frac 1{2\pi\sqrt{\frac 1{4\pi^2 \nu_1^2 C + 4\pi^2 \nu_2^2 C}C}} = \\
     &= \frac 1{\sqrt{\frac 1{\nu_1^2 + \nu_2^2}}}
            = \sqrt{\nu_1^2 + \nu_2^2} = \sqrt{\sqr{40\,\text{Гц}} + \sqr{70\,\text{Гц}}} \approx 80{,}62\,\text{Гц}.
    \end{align*}
}
% autogenerated
