\setdate{19~октября~2021}
\setclass{11«БА»}

\addpersonalvariant{Михаил Бурмистров}

\tasknumber{1}%
\task{%
    Определите период колебаний и массу груза в пружинном маятнике,
    если максимальная скорость груза равна $5\,\frac{\text{м}}{\text{с}}$,
    жёсткость пружины $180\,\frac{\text{Н}}{\text{м}}$, а амплитуда колебаний $15\,\text{см}$.
}
\answer{%
    \begin{align*}
    \frac{mv^2}2 &= \frac{kA^2}2 \implies m = k\sqr{\frac{A}{v}} \approx 162\,\text{г}, \\
    T &= \frac{2\pi}\omega = 2\pi\sqrt{\frac mk} = 2\pi\frac{A}{v} \approx 0{,}188\,\text{c}.
    \end{align*}
}
\solutionspace{80pt}

\tasknumber{2}%
\task{%
    Тело совершает гармонические колебания с периодом $T$ и амплитудой $A$.
    Определите, какую долю периода тело находится на расстоянии менее $\frac{\sqrt 2}2A$ от положения равновесия.
}
\answer{%
    $t = \frac12T \implies \frac tT = \frac12$
}
\solutionspace{80pt}

\tasknumber{3}%
\task{%
    Определите период малых колебаний маятника (см.
    рис.
    на доске),
    если $\ell = 0{,}75\,\text{м}$, а $h = 25\,\text{см}$.
}
\answer{%
    $
        T = 2 \cdot \cfrac{T_1}4 + 2 \cdot \cfrac{T_2}4 = \cfrac{T_1 + T_2}2
        = \cfrac{2\pi\sqrt{\frac {\ell}{g}} + 2\pi\sqrt{\frac {h}{g} } }2
        = \pi\cbr{\sqrt{\frac {\ell}{g}} + \sqrt{\frac {h}{g} } }
        \approx 1{,}357\,\text{c}.
    $
}
\solutionspace{80pt}

\tasknumber{4}%
\task{%
    Определите период колебаний жидкости в тонкой $U$-образной трубке постоянного сечения,
    если длина столбика жидкости равна $45\,\text{см}$ (см.
    рис.
    на доске).
    Вязкостью, силами трения и сопротивления, капиллярными эффектами и притяжением Юпитера, пренебречь.
}
\answer{%
    $
        \Delta F = -m \frac {2x}\ell g = ma
        \implies a + \frac{2g}\ell x = 0
        \implies \omega^2 = \frac{2g}\ell
        \implies T = \frac{2\pi}\omega = 2\pi\sqrt{\frac\ell{2g}}
        \approx $0{,}942\,\text{c}$.
    $
}
\solutionspace{100pt}

\tasknumber{5}%
\task{%
    2 маленьких шарика скреплены лёгкой пружиной жёсткостью $k$, масса каждого шарика $m$.
    Пружина недеформирована, система летит со скоростью $v$ к стенке,
    расстояние от ближайшего шарика до стены $\ell$ (см.
    рис.
    на доске).
    Все удары упругие, трением пренебречь, $\frac{mv^2}{kd^2} < \frac 12,$ где $d$ — длина пружины.
    Через какое время шарики вновь окажутся в том же положении?
}
\answer{%
    \begin{align*}
    F &= -k \cdot 2x = ma \implies a + \frac {2k}m x = 0 \implies \omega^2 = \frac {2k}m \implies T =\frac{2\pi}\omega = 2\pi\sqrt{\frac m{2k}}, \\
    \frac{mv^2}2 &< \frac{2k\sqr{\frac d2}}2 \implies \text{шарики между собой не ударятся}, \\
    T &= 2 \cdot \frac \ell v + \frac 12 \cdot 2 \pi\sqrt{\frac m{2k}}.
    \end{align*}
}
\solutionspace{150pt}

\tasknumber{6}%
\task{%
    Доска массой $m$ и длиной $\ell$ скользит по гладкой горизонтальной плоскости со скоростью $v$,
    а после въезжает на шероховатую поверхность с коэффициентом трения $\mu$ (см.
    рис.
    на доске).
    Определите, за какое время доска остановится (от начала въезжания) и на какое расстояние въедет.
    Ускорение свободного падения $g$, $v^2 < \mu g \ell.$
}
\answer{%
    \begin{align*}
    F_\text{трения} &= -\mu m \frac x\ell g = ma \implies a + \mu \frac g\ell x = 0 \implies \omega^2 = \frac {\mu g}\ell, \\
    T &= \frac{2\pi}{\omega} = 2\pi \sqrt{\frac \ell{\mu g}}\implies t = \frac T4 = \frac\pi2 \sqrt{\frac \ell{\mu g}}, \\
    A &= \frac v\omega = v \sqrt{\frac \ell{\mu g}} \quad (\text{отметим, что $A < \sqrt{\mu g \ell} \cdot \sqrt{\frac \ell{\mu g}} = \ell$}).
    \end{align*}
}

\variantsplitter

\addpersonalvariant{Ирина Ан}

\tasknumber{1}%
\task{%
    Определите период колебаний и массу груза в пружинном маятнике,
    если максимальная скорость груза равна $4\,\frac{\text{м}}{\text{с}}$,
    жёсткость пружины $180\,\frac{\text{Н}}{\text{м}}$, а амплитуда колебаний $12\,\text{см}$.
}
\answer{%
    \begin{align*}
    \frac{mv^2}2 &= \frac{kA^2}2 \implies m = k\sqr{\frac{A}{v}} \approx 162\,\text{г}, \\
    T &= \frac{2\pi}\omega = 2\pi\sqrt{\frac mk} = 2\pi\frac{A}{v} \approx 0{,}188\,\text{c}.
    \end{align*}
}
\solutionspace{80pt}

\tasknumber{2}%
\task{%
    Тело совершает гармонические колебания с периодом $T$ и амплитудой $A$.
    Определите, какую долю периода тело находится на расстоянии более $\frac{\sqrt 3}2A$ от положения равновесия.
}
\answer{%
    $t = \frac13T \implies \frac tT = \frac13$
}
\solutionspace{80pt}

\tasknumber{3}%
\task{%
    Определите период малых колебаний маятника (см.
    рис.
    на доске),
    если $\ell = 0{,}6\,\text{м}$, а $h = 35\,\text{см}$.
}
\answer{%
    $
        T = 2 \cdot \cfrac{T_1}4 + 2 \cdot \cfrac{T_2}4 = \cfrac{T_1 + T_2}2
        = \cfrac{2\pi\sqrt{\frac {\ell}{g}} + 2\pi\sqrt{\frac {h}{g} } }2
        = \pi\cbr{\sqrt{\frac {\ell}{g}} + \sqrt{\frac {h}{g} } }
        \approx 1{,}357\,\text{c}.
    $
}
\solutionspace{80pt}

\tasknumber{4}%
\task{%
    Определите период колебаний жидкости в тонкой $U$-образной трубке постоянного сечения,
    если длина столбика жидкости равна $85\,\text{см}$ (см.
    рис.
    на доске).
    Вязкостью, силами трения и сопротивления, капиллярными эффектами и притяжением Юпитера, пренебречь.
}
\answer{%
    $
        \Delta F = -m \frac {2x}\ell g = ma
        \implies a + \frac{2g}\ell x = 0
        \implies \omega^2 = \frac{2g}\ell
        \implies T = \frac{2\pi}\omega = 2\pi\sqrt{\frac\ell{2g}}
        \approx $1{,}295\,\text{c}$.
    $
}
\solutionspace{100pt}

\tasknumber{5}%
\task{%
    2 маленьких шарика скреплены лёгкой пружиной жёсткостью $k$, масса каждого шарика $m$.
    Пружина недеформирована, система летит со скоростью $v$ к стенке,
    расстояние от ближайшего шарика до стены $\ell$ (см.
    рис.
    на доске).
    Все удары упругие, трением пренебречь, $\frac{mv^2}{kd^2} < \frac 12,$ где $d$ — длина пружины.
    Через какое время шарики вновь окажутся в том же положении?
}
\answer{%
    \begin{align*}
    F &= -k \cdot 2x = ma \implies a + \frac {2k}m x = 0 \implies \omega^2 = \frac {2k}m \implies T =\frac{2\pi}\omega = 2\pi\sqrt{\frac m{2k}}, \\
    \frac{mv^2}2 &< \frac{2k\sqr{\frac d2}}2 \implies \text{шарики между собой не ударятся}, \\
    T &= 2 \cdot \frac \ell v + \frac 12 \cdot 2 \pi\sqrt{\frac m{2k}}.
    \end{align*}
}
\solutionspace{150pt}

\tasknumber{6}%
\task{%
    Доска массой $m$ и длиной $\ell$ скользит по гладкой горизонтальной плоскости со скоростью $v$,
    а после въезжает на шероховатую поверхность с коэффициентом трения $\mu$ (см.
    рис.
    на доске).
    Определите, за какое время доска остановится (от начала въезжания) и на какое расстояние въедет.
    Ускорение свободного падения $g$, $v^2 < \mu g \ell.$
}
\answer{%
    \begin{align*}
    F_\text{трения} &= -\mu m \frac x\ell g = ma \implies a + \mu \frac g\ell x = 0 \implies \omega^2 = \frac {\mu g}\ell, \\
    T &= \frac{2\pi}{\omega} = 2\pi \sqrt{\frac \ell{\mu g}}\implies t = \frac T4 = \frac\pi2 \sqrt{\frac \ell{\mu g}}, \\
    A &= \frac v\omega = v \sqrt{\frac \ell{\mu g}} \quad (\text{отметим, что $A < \sqrt{\mu g \ell} \cdot \sqrt{\frac \ell{\mu g}} = \ell$}).
    \end{align*}
}

\variantsplitter

\addpersonalvariant{Софья Андрианова}

\tasknumber{1}%
\task{%
    Определите период колебаний и массу груза в пружинном маятнике,
    если максимальная скорость груза равна $2\,\frac{\text{м}}{\text{с}}$,
    жёсткость пружины $180\,\frac{\text{Н}}{\text{м}}$, а амплитуда колебаний $15\,\text{см}$.
}
\answer{%
    \begin{align*}
    \frac{mv^2}2 &= \frac{kA^2}2 \implies m = k\sqr{\frac{A}{v}} \approx 1013\,\text{г}, \\
    T &= \frac{2\pi}\omega = 2\pi\sqrt{\frac mk} = 2\pi\frac{A}{v} \approx 0{,}471\,\text{c}.
    \end{align*}
}
\solutionspace{80pt}

\tasknumber{2}%
\task{%
    Тело совершает гармонические колебания с периодом $T$ и амплитудой $A$.
    Определите, какую долю периода тело находится на расстоянии более $\frac{\sqrt 3}2A$ от положения равновесия.
}
\answer{%
    $t = \frac13T \implies \frac tT = \frac13$
}
\solutionspace{80pt}

\tasknumber{3}%
\task{%
    Определите период малых колебаний маятника (см.
    рис.
    на доске),
    если $\ell = 0{,}9\,\text{м}$, а $h = 35\,\text{см}$.
}
\answer{%
    $
        T = 2 \cdot \cfrac{T_1}4 + 2 \cdot \cfrac{T_2}4 = \cfrac{T_1 + T_2}2
        = \cfrac{2\pi\sqrt{\frac {\ell}{g}} + 2\pi\sqrt{\frac {h}{g} } }2
        = \pi\cbr{\sqrt{\frac {\ell}{g}} + \sqrt{\frac {h}{g} } }
        \approx 1{,}530\,\text{c}.
    $
}
\solutionspace{80pt}

\tasknumber{4}%
\task{%
    Определите период колебаний жидкости в тонкой $U$-образной трубке постоянного сечения,
    если длина столбика жидкости равна $45\,\text{см}$ (см.
    рис.
    на доске).
    Вязкостью, силами трения и сопротивления, капиллярными эффектами и притяжением Сатурна, пренебречь.
}
\answer{%
    $
        \Delta F = -m \frac {2x}\ell g = ma
        \implies a + \frac{2g}\ell x = 0
        \implies \omega^2 = \frac{2g}\ell
        \implies T = \frac{2\pi}\omega = 2\pi\sqrt{\frac\ell{2g}}
        \approx $0{,}942\,\text{c}$.
    $
}
\solutionspace{100pt}

\tasknumber{5}%
\task{%
    2 маленьких шарика скреплены лёгкой пружиной жёсткостью $k$, масса каждого шарика $m$.
    Пружина недеформирована, система летит со скоростью $v$ к стенке,
    расстояние от ближайшего шарика до стены $\ell$ (см.
    рис.
    на доске).
    Все удары упругие, трением пренебречь, $\frac{mv^2}{kd^2} < \frac 12,$ где $d$ — длина пружины.
    Через какое время шарики вновь окажутся в том же положении?
}
\answer{%
    \begin{align*}
    F &= -k \cdot 2x = ma \implies a + \frac {2k}m x = 0 \implies \omega^2 = \frac {2k}m \implies T =\frac{2\pi}\omega = 2\pi\sqrt{\frac m{2k}}, \\
    \frac{mv^2}2 &< \frac{2k\sqr{\frac d2}}2 \implies \text{шарики между собой не ударятся}, \\
    T &= 2 \cdot \frac \ell v + \frac 12 \cdot 2 \pi\sqrt{\frac m{2k}}.
    \end{align*}
}
\solutionspace{150pt}

\tasknumber{6}%
\task{%
    Доска массой $m$ и длиной $\ell$ скользит по гладкой горизонтальной плоскости со скоростью $v$,
    а после въезжает на шероховатую поверхность с коэффициентом трения $\mu$ (см.
    рис.
    на доске).
    Определите, за какое время доска остановится (от начала въезжания) и на какое расстояние въедет.
    Ускорение свободного падения $g$, $v^2 < \mu g \ell.$
}
\answer{%
    \begin{align*}
    F_\text{трения} &= -\mu m \frac x\ell g = ma \implies a + \mu \frac g\ell x = 0 \implies \omega^2 = \frac {\mu g}\ell, \\
    T &= \frac{2\pi}{\omega} = 2\pi \sqrt{\frac \ell{\mu g}}\implies t = \frac T4 = \frac\pi2 \sqrt{\frac \ell{\mu g}}, \\
    A &= \frac v\omega = v \sqrt{\frac \ell{\mu g}} \quad (\text{отметим, что $A < \sqrt{\mu g \ell} \cdot \sqrt{\frac \ell{\mu g}} = \ell$}).
    \end{align*}
}

\variantsplitter

\addpersonalvariant{Владимир Артемчук}

\tasknumber{1}%
\task{%
    Определите период колебаний и массу груза в пружинном маятнике,
    если максимальная скорость груза равна $3\,\frac{\text{м}}{\text{с}}$,
    жёсткость пружины $240\,\frac{\text{Н}}{\text{м}}$, а амплитуда колебаний $12\,\text{см}$.
}
\answer{%
    \begin{align*}
    \frac{mv^2}2 &= \frac{kA^2}2 \implies m = k\sqr{\frac{A}{v}} \approx 384\,\text{г}, \\
    T &= \frac{2\pi}\omega = 2\pi\sqrt{\frac mk} = 2\pi\frac{A}{v} \approx 0{,}251\,\text{c}.
    \end{align*}
}
\solutionspace{80pt}

\tasknumber{2}%
\task{%
    Тело совершает гармонические колебания с периодом $T$ и амплитудой $A$.
    Определите, какую долю периода тело находится на расстоянии более $\frac 12A$ от положения равновесия.
}
\answer{%
    $t = \frac23T \implies \frac tT = \frac23$
}
\solutionspace{80pt}

\tasknumber{3}%
\task{%
    Определите период малых колебаний маятника (см.
    рис.
    на доске),
    если $\ell = 0{,}6\,\text{м}$, а $h = 35\,\text{см}$.
}
\answer{%
    $
        T = 2 \cdot \cfrac{T_1}4 + 2 \cdot \cfrac{T_2}4 = \cfrac{T_1 + T_2}2
        = \cfrac{2\pi\sqrt{\frac {\ell}{g}} + 2\pi\sqrt{\frac {h}{g} } }2
        = \pi\cbr{\sqrt{\frac {\ell}{g}} + \sqrt{\frac {h}{g} } }
        \approx 1{,}357\,\text{c}.
    $
}
\solutionspace{80pt}

\tasknumber{4}%
\task{%
    Определите период колебаний жидкости в тонкой $U$-образной трубке постоянного сечения,
    если длина столбика жидкости равна $65\,\text{см}$ (см.
    рис.
    на доске).
    Вязкостью, силами трения и сопротивления, капиллярными эффектами и притяжением Юпитера, пренебречь.
}
\answer{%
    $
        \Delta F = -m \frac {2x}\ell g = ma
        \implies a + \frac{2g}\ell x = 0
        \implies \omega^2 = \frac{2g}\ell
        \implies T = \frac{2\pi}\omega = 2\pi\sqrt{\frac\ell{2g}}
        \approx $1{,}133\,\text{c}$.
    $
}
\solutionspace{100pt}

\tasknumber{5}%
\task{%
    2 маленьких шарика скреплены лёгкой пружиной жёсткостью $k$, масса каждого шарика $m$.
    Пружина недеформирована, система летит со скоростью $v$ к стенке,
    расстояние от ближайшего шарика до стены $\ell$ (см.
    рис.
    на доске).
    Все удары упругие, трением пренебречь, $\frac{mv^2}{kd^2} < \frac 12,$ где $d$ — длина пружины.
    Через какое время шарики вновь окажутся в том же положении?
}
\answer{%
    \begin{align*}
    F &= -k \cdot 2x = ma \implies a + \frac {2k}m x = 0 \implies \omega^2 = \frac {2k}m \implies T =\frac{2\pi}\omega = 2\pi\sqrt{\frac m{2k}}, \\
    \frac{mv^2}2 &< \frac{2k\sqr{\frac d2}}2 \implies \text{шарики между собой не ударятся}, \\
    T &= 2 \cdot \frac \ell v + \frac 12 \cdot 2 \pi\sqrt{\frac m{2k}}.
    \end{align*}
}
\solutionspace{150pt}

\tasknumber{6}%
\task{%
    Доска массой $m$ и длиной $\ell$ скользит по гладкой горизонтальной плоскости со скоростью $v$,
    а после въезжает на шероховатую поверхность с коэффициентом трения $\mu$ (см.
    рис.
    на доске).
    Определите, за какое время доска остановится (от начала въезжания) и на какое расстояние въедет.
    Ускорение свободного падения $g$, $v^2 < \mu g \ell.$
}
\answer{%
    \begin{align*}
    F_\text{трения} &= -\mu m \frac x\ell g = ma \implies a + \mu \frac g\ell x = 0 \implies \omega^2 = \frac {\mu g}\ell, \\
    T &= \frac{2\pi}{\omega} = 2\pi \sqrt{\frac \ell{\mu g}}\implies t = \frac T4 = \frac\pi2 \sqrt{\frac \ell{\mu g}}, \\
    A &= \frac v\omega = v \sqrt{\frac \ell{\mu g}} \quad (\text{отметим, что $A < \sqrt{\mu g \ell} \cdot \sqrt{\frac \ell{\mu g}} = \ell$}).
    \end{align*}
}

\variantsplitter

\addpersonalvariant{Софья Белянкина}

\tasknumber{1}%
\task{%
    Определите период колебаний и массу груза в пружинном маятнике,
    если максимальная скорость груза равна $5\,\frac{\text{м}}{\text{с}}$,
    жёсткость пружины $180\,\frac{\text{Н}}{\text{м}}$, а амплитуда колебаний $10\,\text{см}$.
}
\answer{%
    \begin{align*}
    \frac{mv^2}2 &= \frac{kA^2}2 \implies m = k\sqr{\frac{A}{v}} \approx 72\,\text{г}, \\
    T &= \frac{2\pi}\omega = 2\pi\sqrt{\frac mk} = 2\pi\frac{A}{v} \approx 0{,}126\,\text{c}.
    \end{align*}
}
\solutionspace{80pt}

\tasknumber{2}%
\task{%
    Тело совершает гармонические колебания с периодом $T$ и амплитудой $A$.
    Определите, какую долю периода тело находится на расстоянии более $\frac{\sqrt 3}2A$ от положения равновесия.
}
\answer{%
    $t = \frac13T \implies \frac tT = \frac13$
}
\solutionspace{80pt}

\tasknumber{3}%
\task{%
    Определите период малых колебаний маятника (см.
    рис.
    на доске),
    если $\ell = 0{,}75\,\text{м}$, а $h = 20\,\text{см}$.
}
\answer{%
    $
        T = 2 \cdot \cfrac{T_1}4 + 2 \cdot \cfrac{T_2}4 = \cfrac{T_1 + T_2}2
        = \cfrac{2\pi\sqrt{\frac {\ell}{g}} + 2\pi\sqrt{\frac {h}{g} } }2
        = \pi\cbr{\sqrt{\frac {\ell}{g}} + \sqrt{\frac {h}{g} } }
        \approx 1{,}305\,\text{c}.
    $
}
\solutionspace{80pt}

\tasknumber{4}%
\task{%
    Определите период колебаний жидкости в тонкой $U$-образной трубке постоянного сечения,
    если длина столбика жидкости равна $75\,\text{см}$ (см.
    рис.
    на доске).
    Вязкостью, силами трения и сопротивления, капиллярными эффектами и притяжением Марса, пренебречь.
}
\answer{%
    $
        \Delta F = -m \frac {2x}\ell g = ma
        \implies a + \frac{2g}\ell x = 0
        \implies \omega^2 = \frac{2g}\ell
        \implies T = \frac{2\pi}\omega = 2\pi\sqrt{\frac\ell{2g}}
        \approx $1{,}217\,\text{c}$.
    $
}
\solutionspace{100pt}

\tasknumber{5}%
\task{%
    2 маленьких шарика скреплены лёгкой пружиной жёсткостью $k$, масса каждого шарика $m$.
    Пружина недеформирована, система летит со скоростью $v$ к стенке,
    расстояние от ближайшего шарика до стены $\ell$ (см.
    рис.
    на доске).
    Все удары упругие, трением пренебречь, $\frac{mv^2}{kd^2} < \frac 12,$ где $d$ — длина пружины.
    Через какое время шарики вновь окажутся в том же положении?
}
\answer{%
    \begin{align*}
    F &= -k \cdot 2x = ma \implies a + \frac {2k}m x = 0 \implies \omega^2 = \frac {2k}m \implies T =\frac{2\pi}\omega = 2\pi\sqrt{\frac m{2k}}, \\
    \frac{mv^2}2 &< \frac{2k\sqr{\frac d2}}2 \implies \text{шарики между собой не ударятся}, \\
    T &= 2 \cdot \frac \ell v + \frac 12 \cdot 2 \pi\sqrt{\frac m{2k}}.
    \end{align*}
}
\solutionspace{150pt}

\tasknumber{6}%
\task{%
    Доска массой $m$ и длиной $\ell$ скользит по гладкой горизонтальной плоскости со скоростью $v$,
    а после въезжает на шероховатую поверхность с коэффициентом трения $\mu$ (см.
    рис.
    на доске).
    Определите, за какое время доска остановится (от начала въезжания) и на какое расстояние въедет.
    Ускорение свободного падения $g$, $v^2 < \mu g \ell.$
}
\answer{%
    \begin{align*}
    F_\text{трения} &= -\mu m \frac x\ell g = ma \implies a + \mu \frac g\ell x = 0 \implies \omega^2 = \frac {\mu g}\ell, \\
    T &= \frac{2\pi}{\omega} = 2\pi \sqrt{\frac \ell{\mu g}}\implies t = \frac T4 = \frac\pi2 \sqrt{\frac \ell{\mu g}}, \\
    A &= \frac v\omega = v \sqrt{\frac \ell{\mu g}} \quad (\text{отметим, что $A < \sqrt{\mu g \ell} \cdot \sqrt{\frac \ell{\mu g}} = \ell$}).
    \end{align*}
}

\variantsplitter

\addpersonalvariant{Варвара Егиазарян}

\tasknumber{1}%
\task{%
    Определите период колебаний и массу груза в пружинном маятнике,
    если максимальная скорость груза равна $3\,\frac{\text{м}}{\text{с}}$,
    жёсткость пружины $240\,\frac{\text{Н}}{\text{м}}$, а амплитуда колебаний $12\,\text{см}$.
}
\answer{%
    \begin{align*}
    \frac{mv^2}2 &= \frac{kA^2}2 \implies m = k\sqr{\frac{A}{v}} \approx 384\,\text{г}, \\
    T &= \frac{2\pi}\omega = 2\pi\sqrt{\frac mk} = 2\pi\frac{A}{v} \approx 0{,}251\,\text{c}.
    \end{align*}
}
\solutionspace{80pt}

\tasknumber{2}%
\task{%
    Тело совершает гармонические колебания с периодом $T$ и амплитудой $A$.
    Определите, какую долю периода тело находится на расстоянии менее $\frac{\sqrt 3}2A$ от положения равновесия.
}
\answer{%
    $t = \frac23T \implies \frac tT = \frac23$
}
\solutionspace{80pt}

\tasknumber{3}%
\task{%
    Определите период малых колебаний маятника (см.
    рис.
    на доске),
    если $\ell = 0{,}75\,\text{м}$, а $h = 20\,\text{см}$.
}
\answer{%
    $
        T = 2 \cdot \cfrac{T_1}4 + 2 \cdot \cfrac{T_2}4 = \cfrac{T_1 + T_2}2
        = \cfrac{2\pi\sqrt{\frac {\ell}{g}} + 2\pi\sqrt{\frac {h}{g} } }2
        = \pi\cbr{\sqrt{\frac {\ell}{g}} + \sqrt{\frac {h}{g} } }
        \approx 1{,}305\,\text{c}.
    $
}
\solutionspace{80pt}

\tasknumber{4}%
\task{%
    Определите период колебаний жидкости в тонкой $U$-образной трубке постоянного сечения,
    если длина столбика жидкости равна $45\,\text{см}$ (см.
    рис.
    на доске).
    Вязкостью, силами трения и сопротивления, капиллярными эффектами и притяжением Юпитера, пренебречь.
}
\answer{%
    $
        \Delta F = -m \frac {2x}\ell g = ma
        \implies a + \frac{2g}\ell x = 0
        \implies \omega^2 = \frac{2g}\ell
        \implies T = \frac{2\pi}\omega = 2\pi\sqrt{\frac\ell{2g}}
        \approx $0{,}942\,\text{c}$.
    $
}
\solutionspace{100pt}

\tasknumber{5}%
\task{%
    2 маленьких шарика скреплены лёгкой пружиной жёсткостью $k$, масса каждого шарика $m$.
    Пружина недеформирована, система летит со скоростью $v$ к стенке,
    расстояние от ближайшего шарика до стены $\ell$ (см.
    рис.
    на доске).
    Все удары упругие, трением пренебречь, $\frac{mv^2}{kd^2} < \frac 12,$ где $d$ — длина пружины.
    Через какое время шарики вновь окажутся в том же положении?
}
\answer{%
    \begin{align*}
    F &= -k \cdot 2x = ma \implies a + \frac {2k}m x = 0 \implies \omega^2 = \frac {2k}m \implies T =\frac{2\pi}\omega = 2\pi\sqrt{\frac m{2k}}, \\
    \frac{mv^2}2 &< \frac{2k\sqr{\frac d2}}2 \implies \text{шарики между собой не ударятся}, \\
    T &= 2 \cdot \frac \ell v + \frac 12 \cdot 2 \pi\sqrt{\frac m{2k}}.
    \end{align*}
}
\solutionspace{150pt}

\tasknumber{6}%
\task{%
    Доска массой $m$ и длиной $\ell$ скользит по гладкой горизонтальной плоскости со скоростью $v$,
    а после въезжает на шероховатую поверхность с коэффициентом трения $\mu$ (см.
    рис.
    на доске).
    Определите, за какое время доска остановится (от начала въезжания) и на какое расстояние въедет.
    Ускорение свободного падения $g$, $v^2 < \mu g \ell.$
}
\answer{%
    \begin{align*}
    F_\text{трения} &= -\mu m \frac x\ell g = ma \implies a + \mu \frac g\ell x = 0 \implies \omega^2 = \frac {\mu g}\ell, \\
    T &= \frac{2\pi}{\omega} = 2\pi \sqrt{\frac \ell{\mu g}}\implies t = \frac T4 = \frac\pi2 \sqrt{\frac \ell{\mu g}}, \\
    A &= \frac v\omega = v \sqrt{\frac \ell{\mu g}} \quad (\text{отметим, что $A < \sqrt{\mu g \ell} \cdot \sqrt{\frac \ell{\mu g}} = \ell$}).
    \end{align*}
}

\variantsplitter

\addpersonalvariant{Владислав Емелин}

\tasknumber{1}%
\task{%
    Определите период колебаний и массу груза в пружинном маятнике,
    если максимальная скорость груза равна $5\,\frac{\text{м}}{\text{с}}$,
    жёсткость пружины $180\,\frac{\text{Н}}{\text{м}}$, а амплитуда колебаний $10\,\text{см}$.
}
\answer{%
    \begin{align*}
    \frac{mv^2}2 &= \frac{kA^2}2 \implies m = k\sqr{\frac{A}{v}} \approx 72\,\text{г}, \\
    T &= \frac{2\pi}\omega = 2\pi\sqrt{\frac mk} = 2\pi\frac{A}{v} \approx 0{,}126\,\text{c}.
    \end{align*}
}
\solutionspace{80pt}

\tasknumber{2}%
\task{%
    Тело совершает гармонические колебания с периодом $T$ и амплитудой $A$.
    Определите, какую долю периода тело находится на расстоянии более $\frac{\sqrt 2}2A$ от положения равновесия.
}
\answer{%
    $t = \frac12T \implies \frac tT = \frac12$
}
\solutionspace{80pt}

\tasknumber{3}%
\task{%
    Определите период малых колебаний маятника (см.
    рис.
    на доске),
    если $\ell = 0{,}9\,\text{м}$, а $h = 35\,\text{см}$.
}
\answer{%
    $
        T = 2 \cdot \cfrac{T_1}4 + 2 \cdot \cfrac{T_2}4 = \cfrac{T_1 + T_2}2
        = \cfrac{2\pi\sqrt{\frac {\ell}{g}} + 2\pi\sqrt{\frac {h}{g} } }2
        = \pi\cbr{\sqrt{\frac {\ell}{g}} + \sqrt{\frac {h}{g} } }
        \approx 1{,}530\,\text{c}.
    $
}
\solutionspace{80pt}

\tasknumber{4}%
\task{%
    Определите период колебаний жидкости в тонкой $U$-образной трубке постоянного сечения,
    если длина столбика жидкости равна $55\,\text{см}$ (см.
    рис.
    на доске).
    Вязкостью, силами трения и сопротивления, капиллярными эффектами и притяжением Юпитера, пренебречь.
}
\answer{%
    $
        \Delta F = -m \frac {2x}\ell g = ma
        \implies a + \frac{2g}\ell x = 0
        \implies \omega^2 = \frac{2g}\ell
        \implies T = \frac{2\pi}\omega = 2\pi\sqrt{\frac\ell{2g}}
        \approx $1{,}042\,\text{c}$.
    $
}
\solutionspace{100pt}

\tasknumber{5}%
\task{%
    2 маленьких шарика скреплены лёгкой пружиной жёсткостью $k$, масса каждого шарика $m$.
    Пружина недеформирована, система летит со скоростью $v$ к стенке,
    расстояние от ближайшего шарика до стены $\ell$ (см.
    рис.
    на доске).
    Все удары упругие, трением пренебречь, $\frac{mv^2}{kd^2} < \frac 12,$ где $d$ — длина пружины.
    Через какое время шарики вновь окажутся в том же положении?
}
\answer{%
    \begin{align*}
    F &= -k \cdot 2x = ma \implies a + \frac {2k}m x = 0 \implies \omega^2 = \frac {2k}m \implies T =\frac{2\pi}\omega = 2\pi\sqrt{\frac m{2k}}, \\
    \frac{mv^2}2 &< \frac{2k\sqr{\frac d2}}2 \implies \text{шарики между собой не ударятся}, \\
    T &= 2 \cdot \frac \ell v + \frac 12 \cdot 2 \pi\sqrt{\frac m{2k}}.
    \end{align*}
}
\solutionspace{150pt}

\tasknumber{6}%
\task{%
    Доска массой $m$ и длиной $\ell$ скользит по гладкой горизонтальной плоскости со скоростью $v$,
    а после въезжает на шероховатую поверхность с коэффициентом трения $\mu$ (см.
    рис.
    на доске).
    Определите, за какое время доска остановится (от начала въезжания) и на какое расстояние въедет.
    Ускорение свободного падения $g$, $v^2 < \mu g \ell.$
}
\answer{%
    \begin{align*}
    F_\text{трения} &= -\mu m \frac x\ell g = ma \implies a + \mu \frac g\ell x = 0 \implies \omega^2 = \frac {\mu g}\ell, \\
    T &= \frac{2\pi}{\omega} = 2\pi \sqrt{\frac \ell{\mu g}}\implies t = \frac T4 = \frac\pi2 \sqrt{\frac \ell{\mu g}}, \\
    A &= \frac v\omega = v \sqrt{\frac \ell{\mu g}} \quad (\text{отметим, что $A < \sqrt{\mu g \ell} \cdot \sqrt{\frac \ell{\mu g}} = \ell$}).
    \end{align*}
}

\variantsplitter

\addpersonalvariant{Артём Жичин}

\tasknumber{1}%
\task{%
    Определите период колебаний и массу груза в пружинном маятнике,
    если максимальная скорость груза равна $4\,\frac{\text{м}}{\text{с}}$,
    жёсткость пружины $300\,\frac{\text{Н}}{\text{м}}$, а амплитуда колебаний $12\,\text{см}$.
}
\answer{%
    \begin{align*}
    \frac{mv^2}2 &= \frac{kA^2}2 \implies m = k\sqr{\frac{A}{v}} \approx 270\,\text{г}, \\
    T &= \frac{2\pi}\omega = 2\pi\sqrt{\frac mk} = 2\pi\frac{A}{v} \approx 0{,}188\,\text{c}.
    \end{align*}
}
\solutionspace{80pt}

\tasknumber{2}%
\task{%
    Тело совершает гармонические колебания с периодом $T$ и амплитудой $A$.
    Определите, какую долю периода тело находится на расстоянии менее $\frac 12A$ от положения равновесия.
}
\answer{%
    $t = \frac13T \implies \frac tT = \frac13$
}
\solutionspace{80pt}

\tasknumber{3}%
\task{%
    Определите период малых колебаний маятника (см.
    рис.
    на доске),
    если $\ell = 0{,}6\,\text{м}$, а $h = 35\,\text{см}$.
}
\answer{%
    $
        T = 2 \cdot \cfrac{T_1}4 + 2 \cdot \cfrac{T_2}4 = \cfrac{T_1 + T_2}2
        = \cfrac{2\pi\sqrt{\frac {\ell}{g}} + 2\pi\sqrt{\frac {h}{g} } }2
        = \pi\cbr{\sqrt{\frac {\ell}{g}} + \sqrt{\frac {h}{g} } }
        \approx 1{,}357\,\text{c}.
    $
}
\solutionspace{80pt}

\tasknumber{4}%
\task{%
    Определите период колебаний жидкости в тонкой $U$-образной трубке постоянного сечения,
    если длина столбика жидкости равна $55\,\text{см}$ (см.
    рис.
    на доске).
    Вязкостью, силами трения и сопротивления, капиллярными эффектами и притяжением Марса, пренебречь.
}
\answer{%
    $
        \Delta F = -m \frac {2x}\ell g = ma
        \implies a + \frac{2g}\ell x = 0
        \implies \omega^2 = \frac{2g}\ell
        \implies T = \frac{2\pi}\omega = 2\pi\sqrt{\frac\ell{2g}}
        \approx $1{,}042\,\text{c}$.
    $
}
\solutionspace{100pt}

\tasknumber{5}%
\task{%
    2 маленьких шарика скреплены лёгкой пружиной жёсткостью $k$, масса каждого шарика $m$.
    Пружина недеформирована, система летит со скоростью $v$ к стенке,
    расстояние от ближайшего шарика до стены $\ell$ (см.
    рис.
    на доске).
    Все удары упругие, трением пренебречь, $\frac{mv^2}{kd^2} < \frac 12,$ где $d$ — длина пружины.
    Через какое время шарики вновь окажутся в том же положении?
}
\answer{%
    \begin{align*}
    F &= -k \cdot 2x = ma \implies a + \frac {2k}m x = 0 \implies \omega^2 = \frac {2k}m \implies T =\frac{2\pi}\omega = 2\pi\sqrt{\frac m{2k}}, \\
    \frac{mv^2}2 &< \frac{2k\sqr{\frac d2}}2 \implies \text{шарики между собой не ударятся}, \\
    T &= 2 \cdot \frac \ell v + \frac 12 \cdot 2 \pi\sqrt{\frac m{2k}}.
    \end{align*}
}
\solutionspace{150pt}

\tasknumber{6}%
\task{%
    Доска массой $m$ и длиной $\ell$ скользит по гладкой горизонтальной плоскости со скоростью $v$,
    а после въезжает на шероховатую поверхность с коэффициентом трения $\mu$ (см.
    рис.
    на доске).
    Определите, за какое время доска остановится (от начала въезжания) и на какое расстояние въедет.
    Ускорение свободного падения $g$, $v^2 < \mu g \ell.$
}
\answer{%
    \begin{align*}
    F_\text{трения} &= -\mu m \frac x\ell g = ma \implies a + \mu \frac g\ell x = 0 \implies \omega^2 = \frac {\mu g}\ell, \\
    T &= \frac{2\pi}{\omega} = 2\pi \sqrt{\frac \ell{\mu g}}\implies t = \frac T4 = \frac\pi2 \sqrt{\frac \ell{\mu g}}, \\
    A &= \frac v\omega = v \sqrt{\frac \ell{\mu g}} \quad (\text{отметим, что $A < \sqrt{\mu g \ell} \cdot \sqrt{\frac \ell{\mu g}} = \ell$}).
    \end{align*}
}

\variantsplitter

\addpersonalvariant{Дарья Кошман}

\tasknumber{1}%
\task{%
    Определите период колебаний и массу груза в пружинном маятнике,
    если максимальная скорость груза равна $2\,\frac{\text{м}}{\text{с}}$,
    жёсткость пружины $300\,\frac{\text{Н}}{\text{м}}$, а амплитуда колебаний $15\,\text{см}$.
}
\answer{%
    \begin{align*}
    \frac{mv^2}2 &= \frac{kA^2}2 \implies m = k\sqr{\frac{A}{v}} \approx 1688\,\text{г}, \\
    T &= \frac{2\pi}\omega = 2\pi\sqrt{\frac mk} = 2\pi\frac{A}{v} \approx 0{,}471\,\text{c}.
    \end{align*}
}
\solutionspace{80pt}

\tasknumber{2}%
\task{%
    Тело совершает гармонические колебания с периодом $T$ и амплитудой $A$.
    Определите, какую долю периода тело находится на расстоянии более $\frac{\sqrt 3}2A$ от положения равновесия.
}
\answer{%
    $t = \frac13T \implies \frac tT = \frac13$
}
\solutionspace{80pt}

\tasknumber{3}%
\task{%
    Определите период малых колебаний маятника (см.
    рис.
    на доске),
    если $\ell = 0{,}9\,\text{м}$, а $h = 35\,\text{см}$.
}
\answer{%
    $
        T = 2 \cdot \cfrac{T_1}4 + 2 \cdot \cfrac{T_2}4 = \cfrac{T_1 + T_2}2
        = \cfrac{2\pi\sqrt{\frac {\ell}{g}} + 2\pi\sqrt{\frac {h}{g} } }2
        = \pi\cbr{\sqrt{\frac {\ell}{g}} + \sqrt{\frac {h}{g} } }
        \approx 1{,}530\,\text{c}.
    $
}
\solutionspace{80pt}

\tasknumber{4}%
\task{%
    Определите период колебаний жидкости в тонкой $U$-образной трубке постоянного сечения,
    если длина столбика жидкости равна $85\,\text{см}$ (см.
    рис.
    на доске).
    Вязкостью, силами трения и сопротивления, капиллярными эффектами и притяжением Марса, пренебречь.
}
\answer{%
    $
        \Delta F = -m \frac {2x}\ell g = ma
        \implies a + \frac{2g}\ell x = 0
        \implies \omega^2 = \frac{2g}\ell
        \implies T = \frac{2\pi}\omega = 2\pi\sqrt{\frac\ell{2g}}
        \approx $1{,}295\,\text{c}$.
    $
}
\solutionspace{100pt}

\tasknumber{5}%
\task{%
    2 маленьких шарика скреплены лёгкой пружиной жёсткостью $k$, масса каждого шарика $m$.
    Пружина недеформирована, система летит со скоростью $v$ к стенке,
    расстояние от ближайшего шарика до стены $\ell$ (см.
    рис.
    на доске).
    Все удары упругие, трением пренебречь, $\frac{mv^2}{kd^2} < \frac 12,$ где $d$ — длина пружины.
    Через какое время шарики вновь окажутся в том же положении?
}
\answer{%
    \begin{align*}
    F &= -k \cdot 2x = ma \implies a + \frac {2k}m x = 0 \implies \omega^2 = \frac {2k}m \implies T =\frac{2\pi}\omega = 2\pi\sqrt{\frac m{2k}}, \\
    \frac{mv^2}2 &< \frac{2k\sqr{\frac d2}}2 \implies \text{шарики между собой не ударятся}, \\
    T &= 2 \cdot \frac \ell v + \frac 12 \cdot 2 \pi\sqrt{\frac m{2k}}.
    \end{align*}
}
\solutionspace{150pt}

\tasknumber{6}%
\task{%
    Доска массой $m$ и длиной $\ell$ скользит по гладкой горизонтальной плоскости со скоростью $v$,
    а после въезжает на шероховатую поверхность с коэффициентом трения $\mu$ (см.
    рис.
    на доске).
    Определите, за какое время доска остановится (от начала въезжания) и на какое расстояние въедет.
    Ускорение свободного падения $g$, $v^2 < \mu g \ell.$
}
\answer{%
    \begin{align*}
    F_\text{трения} &= -\mu m \frac x\ell g = ma \implies a + \mu \frac g\ell x = 0 \implies \omega^2 = \frac {\mu g}\ell, \\
    T &= \frac{2\pi}{\omega} = 2\pi \sqrt{\frac \ell{\mu g}}\implies t = \frac T4 = \frac\pi2 \sqrt{\frac \ell{\mu g}}, \\
    A &= \frac v\omega = v \sqrt{\frac \ell{\mu g}} \quad (\text{отметим, что $A < \sqrt{\mu g \ell} \cdot \sqrt{\frac \ell{\mu g}} = \ell$}).
    \end{align*}
}

\variantsplitter

\addpersonalvariant{Анна Кузьмичёва}

\tasknumber{1}%
\task{%
    Определите период колебаний и массу груза в пружинном маятнике,
    если максимальная скорость груза равна $3\,\frac{\text{м}}{\text{с}}$,
    жёсткость пружины $240\,\frac{\text{Н}}{\text{м}}$, а амплитуда колебаний $15\,\text{см}$.
}
\answer{%
    \begin{align*}
    \frac{mv^2}2 &= \frac{kA^2}2 \implies m = k\sqr{\frac{A}{v}} \approx 600\,\text{г}, \\
    T &= \frac{2\pi}\omega = 2\pi\sqrt{\frac mk} = 2\pi\frac{A}{v} \approx 0{,}314\,\text{c}.
    \end{align*}
}
\solutionspace{80pt}

\tasknumber{2}%
\task{%
    Тело совершает гармонические колебания с периодом $T$ и амплитудой $A$.
    Определите, какую долю периода тело находится на расстоянии менее $\frac{\sqrt 3}2A$ от положения равновесия.
}
\answer{%
    $t = \frac23T \implies \frac tT = \frac23$
}
\solutionspace{80pt}

\tasknumber{3}%
\task{%
    Определите период малых колебаний маятника (см.
    рис.
    на доске),
    если $\ell = 0{,}6\,\text{м}$, а $h = 25\,\text{см}$.
}
\answer{%
    $
        T = 2 \cdot \cfrac{T_1}4 + 2 \cdot \cfrac{T_2}4 = \cfrac{T_1 + T_2}2
        = \cfrac{2\pi\sqrt{\frac {\ell}{g}} + 2\pi\sqrt{\frac {h}{g} } }2
        = \pi\cbr{\sqrt{\frac {\ell}{g}} + \sqrt{\frac {h}{g} } }
        \approx 1{,}266\,\text{c}.
    $
}
\solutionspace{80pt}

\tasknumber{4}%
\task{%
    Определите период колебаний жидкости в тонкой $U$-образной трубке постоянного сечения,
    если длина столбика жидкости равна $75\,\text{см}$ (см.
    рис.
    на доске).
    Вязкостью, силами трения и сопротивления, капиллярными эффектами и притяжением Венеры, пренебречь.
}
\answer{%
    $
        \Delta F = -m \frac {2x}\ell g = ma
        \implies a + \frac{2g}\ell x = 0
        \implies \omega^2 = \frac{2g}\ell
        \implies T = \frac{2\pi}\omega = 2\pi\sqrt{\frac\ell{2g}}
        \approx $1{,}217\,\text{c}$.
    $
}
\solutionspace{100pt}

\tasknumber{5}%
\task{%
    2 маленьких шарика скреплены лёгкой пружиной жёсткостью $k$, масса каждого шарика $m$.
    Пружина недеформирована, система летит со скоростью $v$ к стенке,
    расстояние от ближайшего шарика до стены $\ell$ (см.
    рис.
    на доске).
    Все удары упругие, трением пренебречь, $\frac{mv^2}{kd^2} < \frac 12,$ где $d$ — длина пружины.
    Через какое время шарики вновь окажутся в том же положении?
}
\answer{%
    \begin{align*}
    F &= -k \cdot 2x = ma \implies a + \frac {2k}m x = 0 \implies \omega^2 = \frac {2k}m \implies T =\frac{2\pi}\omega = 2\pi\sqrt{\frac m{2k}}, \\
    \frac{mv^2}2 &< \frac{2k\sqr{\frac d2}}2 \implies \text{шарики между собой не ударятся}, \\
    T &= 2 \cdot \frac \ell v + \frac 12 \cdot 2 \pi\sqrt{\frac m{2k}}.
    \end{align*}
}
\solutionspace{150pt}

\tasknumber{6}%
\task{%
    Доска массой $m$ и длиной $\ell$ скользит по гладкой горизонтальной плоскости со скоростью $v$,
    а после въезжает на шероховатую поверхность с коэффициентом трения $\mu$ (см.
    рис.
    на доске).
    Определите, за какое время доска остановится (от начала въезжания) и на какое расстояние въедет.
    Ускорение свободного падения $g$, $v^2 < \mu g \ell.$
}
\answer{%
    \begin{align*}
    F_\text{трения} &= -\mu m \frac x\ell g = ma \implies a + \mu \frac g\ell x = 0 \implies \omega^2 = \frac {\mu g}\ell, \\
    T &= \frac{2\pi}{\omega} = 2\pi \sqrt{\frac \ell{\mu g}}\implies t = \frac T4 = \frac\pi2 \sqrt{\frac \ell{\mu g}}, \\
    A &= \frac v\omega = v \sqrt{\frac \ell{\mu g}} \quad (\text{отметим, что $A < \sqrt{\mu g \ell} \cdot \sqrt{\frac \ell{\mu g}} = \ell$}).
    \end{align*}
}

\variantsplitter

\addpersonalvariant{Алёна Куприянова}

\tasknumber{1}%
\task{%
    Определите период колебаний и массу груза в пружинном маятнике,
    если максимальная скорость груза равна $5\,\frac{\text{м}}{\text{с}}$,
    жёсткость пружины $180\,\frac{\text{Н}}{\text{м}}$, а амплитуда колебаний $15\,\text{см}$.
}
\answer{%
    \begin{align*}
    \frac{mv^2}2 &= \frac{kA^2}2 \implies m = k\sqr{\frac{A}{v}} \approx 162\,\text{г}, \\
    T &= \frac{2\pi}\omega = 2\pi\sqrt{\frac mk} = 2\pi\frac{A}{v} \approx 0{,}188\,\text{c}.
    \end{align*}
}
\solutionspace{80pt}

\tasknumber{2}%
\task{%
    Тело совершает гармонические колебания с периодом $T$ и амплитудой $A$.
    Определите, какую долю периода тело находится на расстоянии менее $\frac{\sqrt 3}2A$ от положения равновесия.
}
\answer{%
    $t = \frac23T \implies \frac tT = \frac23$
}
\solutionspace{80pt}

\tasknumber{3}%
\task{%
    Определите период малых колебаний маятника (см.
    рис.
    на доске),
    если $\ell = 0{,}9\,\text{м}$, а $h = 25\,\text{см}$.
}
\answer{%
    $
        T = 2 \cdot \cfrac{T_1}4 + 2 \cdot \cfrac{T_2}4 = \cfrac{T_1 + T_2}2
        = \cfrac{2\pi\sqrt{\frac {\ell}{g}} + 2\pi\sqrt{\frac {h}{g} } }2
        = \pi\cbr{\sqrt{\frac {\ell}{g}} + \sqrt{\frac {h}{g} } }
        \approx 1{,}439\,\text{c}.
    $
}
\solutionspace{80pt}

\tasknumber{4}%
\task{%
    Определите период колебаний жидкости в тонкой $U$-образной трубке постоянного сечения,
    если длина столбика жидкости равна $75\,\text{см}$ (см.
    рис.
    на доске).
    Вязкостью, силами трения и сопротивления, капиллярными эффектами и притяжением Марса, пренебречь.
}
\answer{%
    $
        \Delta F = -m \frac {2x}\ell g = ma
        \implies a + \frac{2g}\ell x = 0
        \implies \omega^2 = \frac{2g}\ell
        \implies T = \frac{2\pi}\omega = 2\pi\sqrt{\frac\ell{2g}}
        \approx $1{,}217\,\text{c}$.
    $
}
\solutionspace{100pt}

\tasknumber{5}%
\task{%
    2 маленьких шарика скреплены лёгкой пружиной жёсткостью $k$, масса каждого шарика $m$.
    Пружина недеформирована, система летит со скоростью $v$ к стенке,
    расстояние от ближайшего шарика до стены $\ell$ (см.
    рис.
    на доске).
    Все удары упругие, трением пренебречь, $\frac{mv^2}{kd^2} < \frac 12,$ где $d$ — длина пружины.
    Через какое время шарики вновь окажутся в том же положении?
}
\answer{%
    \begin{align*}
    F &= -k \cdot 2x = ma \implies a + \frac {2k}m x = 0 \implies \omega^2 = \frac {2k}m \implies T =\frac{2\pi}\omega = 2\pi\sqrt{\frac m{2k}}, \\
    \frac{mv^2}2 &< \frac{2k\sqr{\frac d2}}2 \implies \text{шарики между собой не ударятся}, \\
    T &= 2 \cdot \frac \ell v + \frac 12 \cdot 2 \pi\sqrt{\frac m{2k}}.
    \end{align*}
}
\solutionspace{150pt}

\tasknumber{6}%
\task{%
    Доска массой $m$ и длиной $\ell$ скользит по гладкой горизонтальной плоскости со скоростью $v$,
    а после въезжает на шероховатую поверхность с коэффициентом трения $\mu$ (см.
    рис.
    на доске).
    Определите, за какое время доска остановится (от начала въезжания) и на какое расстояние въедет.
    Ускорение свободного падения $g$, $v^2 < \mu g \ell.$
}
\answer{%
    \begin{align*}
    F_\text{трения} &= -\mu m \frac x\ell g = ma \implies a + \mu \frac g\ell x = 0 \implies \omega^2 = \frac {\mu g}\ell, \\
    T &= \frac{2\pi}{\omega} = 2\pi \sqrt{\frac \ell{\mu g}}\implies t = \frac T4 = \frac\pi2 \sqrt{\frac \ell{\mu g}}, \\
    A &= \frac v\omega = v \sqrt{\frac \ell{\mu g}} \quad (\text{отметим, что $A < \sqrt{\mu g \ell} \cdot \sqrt{\frac \ell{\mu g}} = \ell$}).
    \end{align*}
}

\variantsplitter

\addpersonalvariant{Ярослав Лавровский}

\tasknumber{1}%
\task{%
    Определите период колебаний и массу груза в пружинном маятнике,
    если максимальная скорость груза равна $3\,\frac{\text{м}}{\text{с}}$,
    жёсткость пружины $300\,\frac{\text{Н}}{\text{м}}$, а амплитуда колебаний $12\,\text{см}$.
}
\answer{%
    \begin{align*}
    \frac{mv^2}2 &= \frac{kA^2}2 \implies m = k\sqr{\frac{A}{v}} \approx 480\,\text{г}, \\
    T &= \frac{2\pi}\omega = 2\pi\sqrt{\frac mk} = 2\pi\frac{A}{v} \approx 0{,}251\,\text{c}.
    \end{align*}
}
\solutionspace{80pt}

\tasknumber{2}%
\task{%
    Тело совершает гармонические колебания с периодом $T$ и амплитудой $A$.
    Определите, какую долю периода тело находится на расстоянии менее $\frac{\sqrt 3}2A$ от положения равновесия.
}
\answer{%
    $t = \frac23T \implies \frac tT = \frac23$
}
\solutionspace{80pt}

\tasknumber{3}%
\task{%
    Определите период малых колебаний маятника (см.
    рис.
    на доске),
    если $\ell = 0{,}9\,\text{м}$, а $h = 35\,\text{см}$.
}
\answer{%
    $
        T = 2 \cdot \cfrac{T_1}4 + 2 \cdot \cfrac{T_2}4 = \cfrac{T_1 + T_2}2
        = \cfrac{2\pi\sqrt{\frac {\ell}{g}} + 2\pi\sqrt{\frac {h}{g} } }2
        = \pi\cbr{\sqrt{\frac {\ell}{g}} + \sqrt{\frac {h}{g} } }
        \approx 1{,}530\,\text{c}.
    $
}
\solutionspace{80pt}

\tasknumber{4}%
\task{%
    Определите период колебаний жидкости в тонкой $U$-образной трубке постоянного сечения,
    если длина столбика жидкости равна $55\,\text{см}$ (см.
    рис.
    на доске).
    Вязкостью, силами трения и сопротивления, капиллярными эффектами и притяжением Марса, пренебречь.
}
\answer{%
    $
        \Delta F = -m \frac {2x}\ell g = ma
        \implies a + \frac{2g}\ell x = 0
        \implies \omega^2 = \frac{2g}\ell
        \implies T = \frac{2\pi}\omega = 2\pi\sqrt{\frac\ell{2g}}
        \approx $1{,}042\,\text{c}$.
    $
}
\solutionspace{100pt}

\tasknumber{5}%
\task{%
    2 маленьких шарика скреплены лёгкой пружиной жёсткостью $k$, масса каждого шарика $m$.
    Пружина недеформирована, система летит со скоростью $v$ к стенке,
    расстояние от ближайшего шарика до стены $\ell$ (см.
    рис.
    на доске).
    Все удары упругие, трением пренебречь, $\frac{mv^2}{kd^2} < \frac 12,$ где $d$ — длина пружины.
    Через какое время шарики вновь окажутся в том же положении?
}
\answer{%
    \begin{align*}
    F &= -k \cdot 2x = ma \implies a + \frac {2k}m x = 0 \implies \omega^2 = \frac {2k}m \implies T =\frac{2\pi}\omega = 2\pi\sqrt{\frac m{2k}}, \\
    \frac{mv^2}2 &< \frac{2k\sqr{\frac d2}}2 \implies \text{шарики между собой не ударятся}, \\
    T &= 2 \cdot \frac \ell v + \frac 12 \cdot 2 \pi\sqrt{\frac m{2k}}.
    \end{align*}
}
\solutionspace{150pt}

\tasknumber{6}%
\task{%
    Доска массой $m$ и длиной $\ell$ скользит по гладкой горизонтальной плоскости со скоростью $v$,
    а после въезжает на шероховатую поверхность с коэффициентом трения $\mu$ (см.
    рис.
    на доске).
    Определите, за какое время доска остановится (от начала въезжания) и на какое расстояние въедет.
    Ускорение свободного падения $g$, $v^2 < \mu g \ell.$
}
\answer{%
    \begin{align*}
    F_\text{трения} &= -\mu m \frac x\ell g = ma \implies a + \mu \frac g\ell x = 0 \implies \omega^2 = \frac {\mu g}\ell, \\
    T &= \frac{2\pi}{\omega} = 2\pi \sqrt{\frac \ell{\mu g}}\implies t = \frac T4 = \frac\pi2 \sqrt{\frac \ell{\mu g}}, \\
    A &= \frac v\omega = v \sqrt{\frac \ell{\mu g}} \quad (\text{отметим, что $A < \sqrt{\mu g \ell} \cdot \sqrt{\frac \ell{\mu g}} = \ell$}).
    \end{align*}
}

\variantsplitter

\addpersonalvariant{Анастасия Ламанова}

\tasknumber{1}%
\task{%
    Определите период колебаний и массу груза в пружинном маятнике,
    если максимальная скорость груза равна $3\,\frac{\text{м}}{\text{с}}$,
    жёсткость пружины $240\,\frac{\text{Н}}{\text{м}}$, а амплитуда колебаний $15\,\text{см}$.
}
\answer{%
    \begin{align*}
    \frac{mv^2}2 &= \frac{kA^2}2 \implies m = k\sqr{\frac{A}{v}} \approx 600\,\text{г}, \\
    T &= \frac{2\pi}\omega = 2\pi\sqrt{\frac mk} = 2\pi\frac{A}{v} \approx 0{,}314\,\text{c}.
    \end{align*}
}
\solutionspace{80pt}

\tasknumber{2}%
\task{%
    Тело совершает гармонические колебания с периодом $T$ и амплитудой $A$.
    Определите, какую долю периода тело находится на расстоянии менее $\frac{\sqrt 3}2A$ от положения равновесия.
}
\answer{%
    $t = \frac23T \implies \frac tT = \frac23$
}
\solutionspace{80pt}

\tasknumber{3}%
\task{%
    Определите период малых колебаний маятника (см.
    рис.
    на доске),
    если $\ell = 0{,}9\,\text{м}$, а $h = 20\,\text{см}$.
}
\answer{%
    $
        T = 2 \cdot \cfrac{T_1}4 + 2 \cdot \cfrac{T_2}4 = \cfrac{T_1 + T_2}2
        = \cfrac{2\pi\sqrt{\frac {\ell}{g}} + 2\pi\sqrt{\frac {h}{g} } }2
        = \pi\cbr{\sqrt{\frac {\ell}{g}} + \sqrt{\frac {h}{g} } }
        \approx 1{,}387\,\text{c}.
    $
}
\solutionspace{80pt}

\tasknumber{4}%
\task{%
    Определите период колебаний жидкости в тонкой $U$-образной трубке постоянного сечения,
    если длина столбика жидкости равна $45\,\text{см}$ (см.
    рис.
    на доске).
    Вязкостью, силами трения и сопротивления, капиллярными эффектами и притяжением Сатурна, пренебречь.
}
\answer{%
    $
        \Delta F = -m \frac {2x}\ell g = ma
        \implies a + \frac{2g}\ell x = 0
        \implies \omega^2 = \frac{2g}\ell
        \implies T = \frac{2\pi}\omega = 2\pi\sqrt{\frac\ell{2g}}
        \approx $0{,}942\,\text{c}$.
    $
}
\solutionspace{100pt}

\tasknumber{5}%
\task{%
    2 маленьких шарика скреплены лёгкой пружиной жёсткостью $k$, масса каждого шарика $m$.
    Пружина недеформирована, система летит со скоростью $v$ к стенке,
    расстояние от ближайшего шарика до стены $\ell$ (см.
    рис.
    на доске).
    Все удары упругие, трением пренебречь, $\frac{mv^2}{kd^2} < \frac 12,$ где $d$ — длина пружины.
    Через какое время шарики вновь окажутся в том же положении?
}
\answer{%
    \begin{align*}
    F &= -k \cdot 2x = ma \implies a + \frac {2k}m x = 0 \implies \omega^2 = \frac {2k}m \implies T =\frac{2\pi}\omega = 2\pi\sqrt{\frac m{2k}}, \\
    \frac{mv^2}2 &< \frac{2k\sqr{\frac d2}}2 \implies \text{шарики между собой не ударятся}, \\
    T &= 2 \cdot \frac \ell v + \frac 12 \cdot 2 \pi\sqrt{\frac m{2k}}.
    \end{align*}
}
\solutionspace{150pt}

\tasknumber{6}%
\task{%
    Доска массой $m$ и длиной $\ell$ скользит по гладкой горизонтальной плоскости со скоростью $v$,
    а после въезжает на шероховатую поверхность с коэффициентом трения $\mu$ (см.
    рис.
    на доске).
    Определите, за какое время доска остановится (от начала въезжания) и на какое расстояние въедет.
    Ускорение свободного падения $g$, $v^2 < \mu g \ell.$
}
\answer{%
    \begin{align*}
    F_\text{трения} &= -\mu m \frac x\ell g = ma \implies a + \mu \frac g\ell x = 0 \implies \omega^2 = \frac {\mu g}\ell, \\
    T &= \frac{2\pi}{\omega} = 2\pi \sqrt{\frac \ell{\mu g}}\implies t = \frac T4 = \frac\pi2 \sqrt{\frac \ell{\mu g}}, \\
    A &= \frac v\omega = v \sqrt{\frac \ell{\mu g}} \quad (\text{отметим, что $A < \sqrt{\mu g \ell} \cdot \sqrt{\frac \ell{\mu g}} = \ell$}).
    \end{align*}
}

\variantsplitter

\addpersonalvariant{Виктория Легонькова}

\tasknumber{1}%
\task{%
    Определите период колебаний и массу груза в пружинном маятнике,
    если максимальная скорость груза равна $2\,\frac{\text{м}}{\text{с}}$,
    жёсткость пружины $300\,\frac{\text{Н}}{\text{м}}$, а амплитуда колебаний $10\,\text{см}$.
}
\answer{%
    \begin{align*}
    \frac{mv^2}2 &= \frac{kA^2}2 \implies m = k\sqr{\frac{A}{v}} \approx 750\,\text{г}, \\
    T &= \frac{2\pi}\omega = 2\pi\sqrt{\frac mk} = 2\pi\frac{A}{v} \approx 0{,}314\,\text{c}.
    \end{align*}
}
\solutionspace{80pt}

\tasknumber{2}%
\task{%
    Тело совершает гармонические колебания с периодом $T$ и амплитудой $A$.
    Определите, какую долю периода тело находится на расстоянии более $\frac{\sqrt 3}2A$ от положения равновесия.
}
\answer{%
    $t = \frac13T \implies \frac tT = \frac13$
}
\solutionspace{80pt}

\tasknumber{3}%
\task{%
    Определите период малых колебаний маятника (см.
    рис.
    на доске),
    если $\ell = 0{,}6\,\text{м}$, а $h = 25\,\text{см}$.
}
\answer{%
    $
        T = 2 \cdot \cfrac{T_1}4 + 2 \cdot \cfrac{T_2}4 = \cfrac{T_1 + T_2}2
        = \cfrac{2\pi\sqrt{\frac {\ell}{g}} + 2\pi\sqrt{\frac {h}{g} } }2
        = \pi\cbr{\sqrt{\frac {\ell}{g}} + \sqrt{\frac {h}{g} } }
        \approx 1{,}266\,\text{c}.
    $
}
\solutionspace{80pt}

\tasknumber{4}%
\task{%
    Определите период колебаний жидкости в тонкой $U$-образной трубке постоянного сечения,
    если длина столбика жидкости равна $85\,\text{см}$ (см.
    рис.
    на доске).
    Вязкостью, силами трения и сопротивления, капиллярными эффектами и притяжением Сатурна, пренебречь.
}
\answer{%
    $
        \Delta F = -m \frac {2x}\ell g = ma
        \implies a + \frac{2g}\ell x = 0
        \implies \omega^2 = \frac{2g}\ell
        \implies T = \frac{2\pi}\omega = 2\pi\sqrt{\frac\ell{2g}}
        \approx $1{,}295\,\text{c}$.
    $
}
\solutionspace{100pt}

\tasknumber{5}%
\task{%
    2 маленьких шарика скреплены лёгкой пружиной жёсткостью $k$, масса каждого шарика $m$.
    Пружина недеформирована, система летит со скоростью $v$ к стенке,
    расстояние от ближайшего шарика до стены $\ell$ (см.
    рис.
    на доске).
    Все удары упругие, трением пренебречь, $\frac{mv^2}{kd^2} < \frac 12,$ где $d$ — длина пружины.
    Через какое время шарики вновь окажутся в том же положении?
}
\answer{%
    \begin{align*}
    F &= -k \cdot 2x = ma \implies a + \frac {2k}m x = 0 \implies \omega^2 = \frac {2k}m \implies T =\frac{2\pi}\omega = 2\pi\sqrt{\frac m{2k}}, \\
    \frac{mv^2}2 &< \frac{2k\sqr{\frac d2}}2 \implies \text{шарики между собой не ударятся}, \\
    T &= 2 \cdot \frac \ell v + \frac 12 \cdot 2 \pi\sqrt{\frac m{2k}}.
    \end{align*}
}
\solutionspace{150pt}

\tasknumber{6}%
\task{%
    Доска массой $m$ и длиной $\ell$ скользит по гладкой горизонтальной плоскости со скоростью $v$,
    а после въезжает на шероховатую поверхность с коэффициентом трения $\mu$ (см.
    рис.
    на доске).
    Определите, за какое время доска остановится (от начала въезжания) и на какое расстояние въедет.
    Ускорение свободного падения $g$, $v^2 < \mu g \ell.$
}
\answer{%
    \begin{align*}
    F_\text{трения} &= -\mu m \frac x\ell g = ma \implies a + \mu \frac g\ell x = 0 \implies \omega^2 = \frac {\mu g}\ell, \\
    T &= \frac{2\pi}{\omega} = 2\pi \sqrt{\frac \ell{\mu g}}\implies t = \frac T4 = \frac\pi2 \sqrt{\frac \ell{\mu g}}, \\
    A &= \frac v\omega = v \sqrt{\frac \ell{\mu g}} \quad (\text{отметим, что $A < \sqrt{\mu g \ell} \cdot \sqrt{\frac \ell{\mu g}} = \ell$}).
    \end{align*}
}

\variantsplitter

\addpersonalvariant{Семён Мартынов}

\tasknumber{1}%
\task{%
    Определите период колебаний и массу груза в пружинном маятнике,
    если максимальная скорость груза равна $5\,\frac{\text{м}}{\text{с}}$,
    жёсткость пружины $240\,\frac{\text{Н}}{\text{м}}$, а амплитуда колебаний $20\,\text{см}$.
}
\answer{%
    \begin{align*}
    \frac{mv^2}2 &= \frac{kA^2}2 \implies m = k\sqr{\frac{A}{v}} \approx 384\,\text{г}, \\
    T &= \frac{2\pi}\omega = 2\pi\sqrt{\frac mk} = 2\pi\frac{A}{v} \approx 0{,}251\,\text{c}.
    \end{align*}
}
\solutionspace{80pt}

\tasknumber{2}%
\task{%
    Тело совершает гармонические колебания с периодом $T$ и амплитудой $A$.
    Определите, какую долю периода тело находится на расстоянии более $\frac 12A$ от положения равновесия.
}
\answer{%
    $t = \frac23T \implies \frac tT = \frac23$
}
\solutionspace{80pt}

\tasknumber{3}%
\task{%
    Определите период малых колебаний маятника (см.
    рис.
    на доске),
    если $\ell = 0{,}75\,\text{м}$, а $h = 20\,\text{см}$.
}
\answer{%
    $
        T = 2 \cdot \cfrac{T_1}4 + 2 \cdot \cfrac{T_2}4 = \cfrac{T_1 + T_2}2
        = \cfrac{2\pi\sqrt{\frac {\ell}{g}} + 2\pi\sqrt{\frac {h}{g} } }2
        = \pi\cbr{\sqrt{\frac {\ell}{g}} + \sqrt{\frac {h}{g} } }
        \approx 1{,}305\,\text{c}.
    $
}
\solutionspace{80pt}

\tasknumber{4}%
\task{%
    Определите период колебаний жидкости в тонкой $U$-образной трубке постоянного сечения,
    если длина столбика жидкости равна $45\,\text{см}$ (см.
    рис.
    на доске).
    Вязкостью, силами трения и сопротивления, капиллярными эффектами и притяжением Юпитера, пренебречь.
}
\answer{%
    $
        \Delta F = -m \frac {2x}\ell g = ma
        \implies a + \frac{2g}\ell x = 0
        \implies \omega^2 = \frac{2g}\ell
        \implies T = \frac{2\pi}\omega = 2\pi\sqrt{\frac\ell{2g}}
        \approx $0{,}942\,\text{c}$.
    $
}
\solutionspace{100pt}

\tasknumber{5}%
\task{%
    2 маленьких шарика скреплены лёгкой пружиной жёсткостью $k$, масса каждого шарика $m$.
    Пружина недеформирована, система летит со скоростью $v$ к стенке,
    расстояние от ближайшего шарика до стены $\ell$ (см.
    рис.
    на доске).
    Все удары упругие, трением пренебречь, $\frac{mv^2}{kd^2} < \frac 12,$ где $d$ — длина пружины.
    Через какое время шарики вновь окажутся в том же положении?
}
\answer{%
    \begin{align*}
    F &= -k \cdot 2x = ma \implies a + \frac {2k}m x = 0 \implies \omega^2 = \frac {2k}m \implies T =\frac{2\pi}\omega = 2\pi\sqrt{\frac m{2k}}, \\
    \frac{mv^2}2 &< \frac{2k\sqr{\frac d2}}2 \implies \text{шарики между собой не ударятся}, \\
    T &= 2 \cdot \frac \ell v + \frac 12 \cdot 2 \pi\sqrt{\frac m{2k}}.
    \end{align*}
}
\solutionspace{150pt}

\tasknumber{6}%
\task{%
    Доска массой $m$ и длиной $\ell$ скользит по гладкой горизонтальной плоскости со скоростью $v$,
    а после въезжает на шероховатую поверхность с коэффициентом трения $\mu$ (см.
    рис.
    на доске).
    Определите, за какое время доска остановится (от начала въезжания) и на какое расстояние въедет.
    Ускорение свободного падения $g$, $v^2 < \mu g \ell.$
}
\answer{%
    \begin{align*}
    F_\text{трения} &= -\mu m \frac x\ell g = ma \implies a + \mu \frac g\ell x = 0 \implies \omega^2 = \frac {\mu g}\ell, \\
    T &= \frac{2\pi}{\omega} = 2\pi \sqrt{\frac \ell{\mu g}}\implies t = \frac T4 = \frac\pi2 \sqrt{\frac \ell{\mu g}}, \\
    A &= \frac v\omega = v \sqrt{\frac \ell{\mu g}} \quad (\text{отметим, что $A < \sqrt{\mu g \ell} \cdot \sqrt{\frac \ell{\mu g}} = \ell$}).
    \end{align*}
}

\variantsplitter

\addpersonalvariant{Варвара Минаева}

\tasknumber{1}%
\task{%
    Определите период колебаний и массу груза в пружинном маятнике,
    если максимальная скорость груза равна $2\,\frac{\text{м}}{\text{с}}$,
    жёсткость пружины $180\,\frac{\text{Н}}{\text{м}}$, а амплитуда колебаний $10\,\text{см}$.
}
\answer{%
    \begin{align*}
    \frac{mv^2}2 &= \frac{kA^2}2 \implies m = k\sqr{\frac{A}{v}} \approx 450\,\text{г}, \\
    T &= \frac{2\pi}\omega = 2\pi\sqrt{\frac mk} = 2\pi\frac{A}{v} \approx 0{,}314\,\text{c}.
    \end{align*}
}
\solutionspace{80pt}

\tasknumber{2}%
\task{%
    Тело совершает гармонические колебания с периодом $T$ и амплитудой $A$.
    Определите, какую долю периода тело находится на расстоянии менее $\frac{\sqrt 2}2A$ от положения равновесия.
}
\answer{%
    $t = \frac12T \implies \frac tT = \frac12$
}
\solutionspace{80pt}

\tasknumber{3}%
\task{%
    Определите период малых колебаний маятника (см.
    рис.
    на доске),
    если $\ell = 0{,}9\,\text{м}$, а $h = 35\,\text{см}$.
}
\answer{%
    $
        T = 2 \cdot \cfrac{T_1}4 + 2 \cdot \cfrac{T_2}4 = \cfrac{T_1 + T_2}2
        = \cfrac{2\pi\sqrt{\frac {\ell}{g}} + 2\pi\sqrt{\frac {h}{g} } }2
        = \pi\cbr{\sqrt{\frac {\ell}{g}} + \sqrt{\frac {h}{g} } }
        \approx 1{,}530\,\text{c}.
    $
}
\solutionspace{80pt}

\tasknumber{4}%
\task{%
    Определите период колебаний жидкости в тонкой $U$-образной трубке постоянного сечения,
    если длина столбика жидкости равна $55\,\text{см}$ (см.
    рис.
    на доске).
    Вязкостью, силами трения и сопротивления, капиллярными эффектами и притяжением Юпитера, пренебречь.
}
\answer{%
    $
        \Delta F = -m \frac {2x}\ell g = ma
        \implies a + \frac{2g}\ell x = 0
        \implies \omega^2 = \frac{2g}\ell
        \implies T = \frac{2\pi}\omega = 2\pi\sqrt{\frac\ell{2g}}
        \approx $1{,}042\,\text{c}$.
    $
}
\solutionspace{100pt}

\tasknumber{5}%
\task{%
    2 маленьких шарика скреплены лёгкой пружиной жёсткостью $k$, масса каждого шарика $m$.
    Пружина недеформирована, система летит со скоростью $v$ к стенке,
    расстояние от ближайшего шарика до стены $\ell$ (см.
    рис.
    на доске).
    Все удары упругие, трением пренебречь, $\frac{mv^2}{kd^2} < \frac 12,$ где $d$ — длина пружины.
    Через какое время шарики вновь окажутся в том же положении?
}
\answer{%
    \begin{align*}
    F &= -k \cdot 2x = ma \implies a + \frac {2k}m x = 0 \implies \omega^2 = \frac {2k}m \implies T =\frac{2\pi}\omega = 2\pi\sqrt{\frac m{2k}}, \\
    \frac{mv^2}2 &< \frac{2k\sqr{\frac d2}}2 \implies \text{шарики между собой не ударятся}, \\
    T &= 2 \cdot \frac \ell v + \frac 12 \cdot 2 \pi\sqrt{\frac m{2k}}.
    \end{align*}
}
\solutionspace{150pt}

\tasknumber{6}%
\task{%
    Доска массой $m$ и длиной $\ell$ скользит по гладкой горизонтальной плоскости со скоростью $v$,
    а после въезжает на шероховатую поверхность с коэффициентом трения $\mu$ (см.
    рис.
    на доске).
    Определите, за какое время доска остановится (от начала въезжания) и на какое расстояние въедет.
    Ускорение свободного падения $g$, $v^2 < \mu g \ell.$
}
\answer{%
    \begin{align*}
    F_\text{трения} &= -\mu m \frac x\ell g = ma \implies a + \mu \frac g\ell x = 0 \implies \omega^2 = \frac {\mu g}\ell, \\
    T &= \frac{2\pi}{\omega} = 2\pi \sqrt{\frac \ell{\mu g}}\implies t = \frac T4 = \frac\pi2 \sqrt{\frac \ell{\mu g}}, \\
    A &= \frac v\omega = v \sqrt{\frac \ell{\mu g}} \quad (\text{отметим, что $A < \sqrt{\mu g \ell} \cdot \sqrt{\frac \ell{\mu g}} = \ell$}).
    \end{align*}
}

\variantsplitter

\addpersonalvariant{Леонид Никитин}

\tasknumber{1}%
\task{%
    Определите период колебаний и массу груза в пружинном маятнике,
    если максимальная скорость груза равна $5\,\frac{\text{м}}{\text{с}}$,
    жёсткость пружины $240\,\frac{\text{Н}}{\text{м}}$, а амплитуда колебаний $15\,\text{см}$.
}
\answer{%
    \begin{align*}
    \frac{mv^2}2 &= \frac{kA^2}2 \implies m = k\sqr{\frac{A}{v}} \approx 216\,\text{г}, \\
    T &= \frac{2\pi}\omega = 2\pi\sqrt{\frac mk} = 2\pi\frac{A}{v} \approx 0{,}188\,\text{c}.
    \end{align*}
}
\solutionspace{80pt}

\tasknumber{2}%
\task{%
    Тело совершает гармонические колебания с периодом $T$ и амплитудой $A$.
    Определите, какую долю периода тело находится на расстоянии менее $\frac{\sqrt 2}2A$ от положения равновесия.
}
\answer{%
    $t = \frac12T \implies \frac tT = \frac12$
}
\solutionspace{80pt}

\tasknumber{3}%
\task{%
    Определите период малых колебаний маятника (см.
    рис.
    на доске),
    если $\ell = 0{,}9\,\text{м}$, а $h = 35\,\text{см}$.
}
\answer{%
    $
        T = 2 \cdot \cfrac{T_1}4 + 2 \cdot \cfrac{T_2}4 = \cfrac{T_1 + T_2}2
        = \cfrac{2\pi\sqrt{\frac {\ell}{g}} + 2\pi\sqrt{\frac {h}{g} } }2
        = \pi\cbr{\sqrt{\frac {\ell}{g}} + \sqrt{\frac {h}{g} } }
        \approx 1{,}530\,\text{c}.
    $
}
\solutionspace{80pt}

\tasknumber{4}%
\task{%
    Определите период колебаний жидкости в тонкой $U$-образной трубке постоянного сечения,
    если длина столбика жидкости равна $85\,\text{см}$ (см.
    рис.
    на доске).
    Вязкостью, силами трения и сопротивления, капиллярными эффектами и притяжением Венеры, пренебречь.
}
\answer{%
    $
        \Delta F = -m \frac {2x}\ell g = ma
        \implies a + \frac{2g}\ell x = 0
        \implies \omega^2 = \frac{2g}\ell
        \implies T = \frac{2\pi}\omega = 2\pi\sqrt{\frac\ell{2g}}
        \approx $1{,}295\,\text{c}$.
    $
}
\solutionspace{100pt}

\tasknumber{5}%
\task{%
    2 маленьких шарика скреплены лёгкой пружиной жёсткостью $k$, масса каждого шарика $m$.
    Пружина недеформирована, система летит со скоростью $v$ к стенке,
    расстояние от ближайшего шарика до стены $\ell$ (см.
    рис.
    на доске).
    Все удары упругие, трением пренебречь, $\frac{mv^2}{kd^2} < \frac 12,$ где $d$ — длина пружины.
    Через какое время шарики вновь окажутся в том же положении?
}
\answer{%
    \begin{align*}
    F &= -k \cdot 2x = ma \implies a + \frac {2k}m x = 0 \implies \omega^2 = \frac {2k}m \implies T =\frac{2\pi}\omega = 2\pi\sqrt{\frac m{2k}}, \\
    \frac{mv^2}2 &< \frac{2k\sqr{\frac d2}}2 \implies \text{шарики между собой не ударятся}, \\
    T &= 2 \cdot \frac \ell v + \frac 12 \cdot 2 \pi\sqrt{\frac m{2k}}.
    \end{align*}
}
\solutionspace{150pt}

\tasknumber{6}%
\task{%
    Доска массой $m$ и длиной $\ell$ скользит по гладкой горизонтальной плоскости со скоростью $v$,
    а после въезжает на шероховатую поверхность с коэффициентом трения $\mu$ (см.
    рис.
    на доске).
    Определите, за какое время доска остановится (от начала въезжания) и на какое расстояние въедет.
    Ускорение свободного падения $g$, $v^2 < \mu g \ell.$
}
\answer{%
    \begin{align*}
    F_\text{трения} &= -\mu m \frac x\ell g = ma \implies a + \mu \frac g\ell x = 0 \implies \omega^2 = \frac {\mu g}\ell, \\
    T &= \frac{2\pi}{\omega} = 2\pi \sqrt{\frac \ell{\mu g}}\implies t = \frac T4 = \frac\pi2 \sqrt{\frac \ell{\mu g}}, \\
    A &= \frac v\omega = v \sqrt{\frac \ell{\mu g}} \quad (\text{отметим, что $A < \sqrt{\mu g \ell} \cdot \sqrt{\frac \ell{\mu g}} = \ell$}).
    \end{align*}
}

\variantsplitter

\addpersonalvariant{Тимофей Полетаев}

\tasknumber{1}%
\task{%
    Определите период колебаний и массу груза в пружинном маятнике,
    если максимальная скорость груза равна $4\,\frac{\text{м}}{\text{с}}$,
    жёсткость пружины $180\,\frac{\text{Н}}{\text{м}}$, а амплитуда колебаний $15\,\text{см}$.
}
\answer{%
    \begin{align*}
    \frac{mv^2}2 &= \frac{kA^2}2 \implies m = k\sqr{\frac{A}{v}} \approx 254\,\text{г}, \\
    T &= \frac{2\pi}\omega = 2\pi\sqrt{\frac mk} = 2\pi\frac{A}{v} \approx 0{,}236\,\text{c}.
    \end{align*}
}
\solutionspace{80pt}

\tasknumber{2}%
\task{%
    Тело совершает гармонические колебания с периодом $T$ и амплитудой $A$.
    Определите, какую долю периода тело находится на расстоянии менее $\frac{\sqrt 3}2A$ от положения равновесия.
}
\answer{%
    $t = \frac23T \implies \frac tT = \frac23$
}
\solutionspace{80pt}

\tasknumber{3}%
\task{%
    Определите период малых колебаний маятника (см.
    рис.
    на доске),
    если $\ell = 0{,}6\,\text{м}$, а $h = 25\,\text{см}$.
}
\answer{%
    $
        T = 2 \cdot \cfrac{T_1}4 + 2 \cdot \cfrac{T_2}4 = \cfrac{T_1 + T_2}2
        = \cfrac{2\pi\sqrt{\frac {\ell}{g}} + 2\pi\sqrt{\frac {h}{g} } }2
        = \pi\cbr{\sqrt{\frac {\ell}{g}} + \sqrt{\frac {h}{g} } }
        \approx 1{,}266\,\text{c}.
    $
}
\solutionspace{80pt}

\tasknumber{4}%
\task{%
    Определите период колебаний жидкости в тонкой $U$-образной трубке постоянного сечения,
    если длина столбика жидкости равна $45\,\text{см}$ (см.
    рис.
    на доске).
    Вязкостью, силами трения и сопротивления, капиллярными эффектами и притяжением Юпитера, пренебречь.
}
\answer{%
    $
        \Delta F = -m \frac {2x}\ell g = ma
        \implies a + \frac{2g}\ell x = 0
        \implies \omega^2 = \frac{2g}\ell
        \implies T = \frac{2\pi}\omega = 2\pi\sqrt{\frac\ell{2g}}
        \approx $0{,}942\,\text{c}$.
    $
}
\solutionspace{100pt}

\tasknumber{5}%
\task{%
    2 маленьких шарика скреплены лёгкой пружиной жёсткостью $k$, масса каждого шарика $m$.
    Пружина недеформирована, система летит со скоростью $v$ к стенке,
    расстояние от ближайшего шарика до стены $\ell$ (см.
    рис.
    на доске).
    Все удары упругие, трением пренебречь, $\frac{mv^2}{kd^2} < \frac 12,$ где $d$ — длина пружины.
    Через какое время шарики вновь окажутся в том же положении?
}
\answer{%
    \begin{align*}
    F &= -k \cdot 2x = ma \implies a + \frac {2k}m x = 0 \implies \omega^2 = \frac {2k}m \implies T =\frac{2\pi}\omega = 2\pi\sqrt{\frac m{2k}}, \\
    \frac{mv^2}2 &< \frac{2k\sqr{\frac d2}}2 \implies \text{шарики между собой не ударятся}, \\
    T &= 2 \cdot \frac \ell v + \frac 12 \cdot 2 \pi\sqrt{\frac m{2k}}.
    \end{align*}
}
\solutionspace{150pt}

\tasknumber{6}%
\task{%
    Доска массой $m$ и длиной $\ell$ скользит по гладкой горизонтальной плоскости со скоростью $v$,
    а после въезжает на шероховатую поверхность с коэффициентом трения $\mu$ (см.
    рис.
    на доске).
    Определите, за какое время доска остановится (от начала въезжания) и на какое расстояние въедет.
    Ускорение свободного падения $g$, $v^2 < \mu g \ell.$
}
\answer{%
    \begin{align*}
    F_\text{трения} &= -\mu m \frac x\ell g = ma \implies a + \mu \frac g\ell x = 0 \implies \omega^2 = \frac {\mu g}\ell, \\
    T &= \frac{2\pi}{\omega} = 2\pi \sqrt{\frac \ell{\mu g}}\implies t = \frac T4 = \frac\pi2 \sqrt{\frac \ell{\mu g}}, \\
    A &= \frac v\omega = v \sqrt{\frac \ell{\mu g}} \quad (\text{отметим, что $A < \sqrt{\mu g \ell} \cdot \sqrt{\frac \ell{\mu g}} = \ell$}).
    \end{align*}
}

\variantsplitter

\addpersonalvariant{Андрей Рожков}

\tasknumber{1}%
\task{%
    Определите период колебаний и массу груза в пружинном маятнике,
    если максимальная скорость груза равна $5\,\frac{\text{м}}{\text{с}}$,
    жёсткость пружины $240\,\frac{\text{Н}}{\text{м}}$, а амплитуда колебаний $15\,\text{см}$.
}
\answer{%
    \begin{align*}
    \frac{mv^2}2 &= \frac{kA^2}2 \implies m = k\sqr{\frac{A}{v}} \approx 216\,\text{г}, \\
    T &= \frac{2\pi}\omega = 2\pi\sqrt{\frac mk} = 2\pi\frac{A}{v} \approx 0{,}188\,\text{c}.
    \end{align*}
}
\solutionspace{80pt}

\tasknumber{2}%
\task{%
    Тело совершает гармонические колебания с периодом $T$ и амплитудой $A$.
    Определите, какую долю периода тело находится на расстоянии менее $\frac{\sqrt 2}2A$ от положения равновесия.
}
\answer{%
    $t = \frac12T \implies \frac tT = \frac12$
}
\solutionspace{80pt}

\tasknumber{3}%
\task{%
    Определите период малых колебаний маятника (см.
    рис.
    на доске),
    если $\ell = 0{,}6\,\text{м}$, а $h = 35\,\text{см}$.
}
\answer{%
    $
        T = 2 \cdot \cfrac{T_1}4 + 2 \cdot \cfrac{T_2}4 = \cfrac{T_1 + T_2}2
        = \cfrac{2\pi\sqrt{\frac {\ell}{g}} + 2\pi\sqrt{\frac {h}{g} } }2
        = \pi\cbr{\sqrt{\frac {\ell}{g}} + \sqrt{\frac {h}{g} } }
        \approx 1{,}357\,\text{c}.
    $
}
\solutionspace{80pt}

\tasknumber{4}%
\task{%
    Определите период колебаний жидкости в тонкой $U$-образной трубке постоянного сечения,
    если длина столбика жидкости равна $75\,\text{см}$ (см.
    рис.
    на доске).
    Вязкостью, силами трения и сопротивления, капиллярными эффектами и притяжением Юпитера, пренебречь.
}
\answer{%
    $
        \Delta F = -m \frac {2x}\ell g = ma
        \implies a + \frac{2g}\ell x = 0
        \implies \omega^2 = \frac{2g}\ell
        \implies T = \frac{2\pi}\omega = 2\pi\sqrt{\frac\ell{2g}}
        \approx $1{,}217\,\text{c}$.
    $
}
\solutionspace{100pt}

\tasknumber{5}%
\task{%
    2 маленьких шарика скреплены лёгкой пружиной жёсткостью $k$, масса каждого шарика $m$.
    Пружина недеформирована, система летит со скоростью $v$ к стенке,
    расстояние от ближайшего шарика до стены $\ell$ (см.
    рис.
    на доске).
    Все удары упругие, трением пренебречь, $\frac{mv^2}{kd^2} < \frac 12,$ где $d$ — длина пружины.
    Через какое время шарики вновь окажутся в том же положении?
}
\answer{%
    \begin{align*}
    F &= -k \cdot 2x = ma \implies a + \frac {2k}m x = 0 \implies \omega^2 = \frac {2k}m \implies T =\frac{2\pi}\omega = 2\pi\sqrt{\frac m{2k}}, \\
    \frac{mv^2}2 &< \frac{2k\sqr{\frac d2}}2 \implies \text{шарики между собой не ударятся}, \\
    T &= 2 \cdot \frac \ell v + \frac 12 \cdot 2 \pi\sqrt{\frac m{2k}}.
    \end{align*}
}
\solutionspace{150pt}

\tasknumber{6}%
\task{%
    Доска массой $m$ и длиной $\ell$ скользит по гладкой горизонтальной плоскости со скоростью $v$,
    а после въезжает на шероховатую поверхность с коэффициентом трения $\mu$ (см.
    рис.
    на доске).
    Определите, за какое время доска остановится (от начала въезжания) и на какое расстояние въедет.
    Ускорение свободного падения $g$, $v^2 < \mu g \ell.$
}
\answer{%
    \begin{align*}
    F_\text{трения} &= -\mu m \frac x\ell g = ma \implies a + \mu \frac g\ell x = 0 \implies \omega^2 = \frac {\mu g}\ell, \\
    T &= \frac{2\pi}{\omega} = 2\pi \sqrt{\frac \ell{\mu g}}\implies t = \frac T4 = \frac\pi2 \sqrt{\frac \ell{\mu g}}, \\
    A &= \frac v\omega = v \sqrt{\frac \ell{\mu g}} \quad (\text{отметим, что $A < \sqrt{\mu g \ell} \cdot \sqrt{\frac \ell{\mu g}} = \ell$}).
    \end{align*}
}

\variantsplitter

\addpersonalvariant{Рената Таржиманова}

\tasknumber{1}%
\task{%
    Определите период колебаний и массу груза в пружинном маятнике,
    если максимальная скорость груза равна $4\,\frac{\text{м}}{\text{с}}$,
    жёсткость пружины $300\,\frac{\text{Н}}{\text{м}}$, а амплитуда колебаний $10\,\text{см}$.
}
\answer{%
    \begin{align*}
    \frac{mv^2}2 &= \frac{kA^2}2 \implies m = k\sqr{\frac{A}{v}} \approx 188\,\text{г}, \\
    T &= \frac{2\pi}\omega = 2\pi\sqrt{\frac mk} = 2\pi\frac{A}{v} \approx 0{,}157\,\text{c}.
    \end{align*}
}
\solutionspace{80pt}

\tasknumber{2}%
\task{%
    Тело совершает гармонические колебания с периодом $T$ и амплитудой $A$.
    Определите, какую долю периода тело находится на расстоянии менее $\frac{\sqrt 3}2A$ от положения равновесия.
}
\answer{%
    $t = \frac23T \implies \frac tT = \frac23$
}
\solutionspace{80pt}

\tasknumber{3}%
\task{%
    Определите период малых колебаний маятника (см.
    рис.
    на доске),
    если $\ell = 0{,}9\,\text{м}$, а $h = 20\,\text{см}$.
}
\answer{%
    $
        T = 2 \cdot \cfrac{T_1}4 + 2 \cdot \cfrac{T_2}4 = \cfrac{T_1 + T_2}2
        = \cfrac{2\pi\sqrt{\frac {\ell}{g}} + 2\pi\sqrt{\frac {h}{g} } }2
        = \pi\cbr{\sqrt{\frac {\ell}{g}} + \sqrt{\frac {h}{g} } }
        \approx 1{,}387\,\text{c}.
    $
}
\solutionspace{80pt}

\tasknumber{4}%
\task{%
    Определите период колебаний жидкости в тонкой $U$-образной трубке постоянного сечения,
    если длина столбика жидкости равна $65\,\text{см}$ (см.
    рис.
    на доске).
    Вязкостью, силами трения и сопротивления, капиллярными эффектами и притяжением Юпитера, пренебречь.
}
\answer{%
    $
        \Delta F = -m \frac {2x}\ell g = ma
        \implies a + \frac{2g}\ell x = 0
        \implies \omega^2 = \frac{2g}\ell
        \implies T = \frac{2\pi}\omega = 2\pi\sqrt{\frac\ell{2g}}
        \approx $1{,}133\,\text{c}$.
    $
}
\solutionspace{100pt}

\tasknumber{5}%
\task{%
    2 маленьких шарика скреплены лёгкой пружиной жёсткостью $k$, масса каждого шарика $m$.
    Пружина недеформирована, система летит со скоростью $v$ к стенке,
    расстояние от ближайшего шарика до стены $\ell$ (см.
    рис.
    на доске).
    Все удары упругие, трением пренебречь, $\frac{mv^2}{kd^2} < \frac 12,$ где $d$ — длина пружины.
    Через какое время шарики вновь окажутся в том же положении?
}
\answer{%
    \begin{align*}
    F &= -k \cdot 2x = ma \implies a + \frac {2k}m x = 0 \implies \omega^2 = \frac {2k}m \implies T =\frac{2\pi}\omega = 2\pi\sqrt{\frac m{2k}}, \\
    \frac{mv^2}2 &< \frac{2k\sqr{\frac d2}}2 \implies \text{шарики между собой не ударятся}, \\
    T &= 2 \cdot \frac \ell v + \frac 12 \cdot 2 \pi\sqrt{\frac m{2k}}.
    \end{align*}
}
\solutionspace{150pt}

\tasknumber{6}%
\task{%
    Доска массой $m$ и длиной $\ell$ скользит по гладкой горизонтальной плоскости со скоростью $v$,
    а после въезжает на шероховатую поверхность с коэффициентом трения $\mu$ (см.
    рис.
    на доске).
    Определите, за какое время доска остановится (от начала въезжания) и на какое расстояние въедет.
    Ускорение свободного падения $g$, $v^2 < \mu g \ell.$
}
\answer{%
    \begin{align*}
    F_\text{трения} &= -\mu m \frac x\ell g = ma \implies a + \mu \frac g\ell x = 0 \implies \omega^2 = \frac {\mu g}\ell, \\
    T &= \frac{2\pi}{\omega} = 2\pi \sqrt{\frac \ell{\mu g}}\implies t = \frac T4 = \frac\pi2 \sqrt{\frac \ell{\mu g}}, \\
    A &= \frac v\omega = v \sqrt{\frac \ell{\mu g}} \quad (\text{отметим, что $A < \sqrt{\mu g \ell} \cdot \sqrt{\frac \ell{\mu g}} = \ell$}).
    \end{align*}
}

\variantsplitter

\addpersonalvariant{Андрей Щербаков}

\tasknumber{1}%
\task{%
    Определите период колебаний и массу груза в пружинном маятнике,
    если максимальная скорость груза равна $5\,\frac{\text{м}}{\text{с}}$,
    жёсткость пружины $240\,\frac{\text{Н}}{\text{м}}$, а амплитуда колебаний $15\,\text{см}$.
}
\answer{%
    \begin{align*}
    \frac{mv^2}2 &= \frac{kA^2}2 \implies m = k\sqr{\frac{A}{v}} \approx 216\,\text{г}, \\
    T &= \frac{2\pi}\omega = 2\pi\sqrt{\frac mk} = 2\pi\frac{A}{v} \approx 0{,}188\,\text{c}.
    \end{align*}
}
\solutionspace{80pt}

\tasknumber{2}%
\task{%
    Тело совершает гармонические колебания с периодом $T$ и амплитудой $A$.
    Определите, какую долю периода тело находится на расстоянии менее $\frac{\sqrt 3}2A$ от положения равновесия.
}
\answer{%
    $t = \frac23T \implies \frac tT = \frac23$
}
\solutionspace{80pt}

\tasknumber{3}%
\task{%
    Определите период малых колебаний маятника (см.
    рис.
    на доске),
    если $\ell = 0{,}6\,\text{м}$, а $h = 35\,\text{см}$.
}
\answer{%
    $
        T = 2 \cdot \cfrac{T_1}4 + 2 \cdot \cfrac{T_2}4 = \cfrac{T_1 + T_2}2
        = \cfrac{2\pi\sqrt{\frac {\ell}{g}} + 2\pi\sqrt{\frac {h}{g} } }2
        = \pi\cbr{\sqrt{\frac {\ell}{g}} + \sqrt{\frac {h}{g} } }
        \approx 1{,}357\,\text{c}.
    $
}
\solutionspace{80pt}

\tasknumber{4}%
\task{%
    Определите период колебаний жидкости в тонкой $U$-образной трубке постоянного сечения,
    если длина столбика жидкости равна $55\,\text{см}$ (см.
    рис.
    на доске).
    Вязкостью, силами трения и сопротивления, капиллярными эффектами и притяжением Венеры, пренебречь.
}
\answer{%
    $
        \Delta F = -m \frac {2x}\ell g = ma
        \implies a + \frac{2g}\ell x = 0
        \implies \omega^2 = \frac{2g}\ell
        \implies T = \frac{2\pi}\omega = 2\pi\sqrt{\frac\ell{2g}}
        \approx $1{,}042\,\text{c}$.
    $
}
\solutionspace{100pt}

\tasknumber{5}%
\task{%
    2 маленьких шарика скреплены лёгкой пружиной жёсткостью $k$, масса каждого шарика $m$.
    Пружина недеформирована, система летит со скоростью $v$ к стенке,
    расстояние от ближайшего шарика до стены $\ell$ (см.
    рис.
    на доске).
    Все удары упругие, трением пренебречь, $\frac{mv^2}{kd^2} < \frac 12,$ где $d$ — длина пружины.
    Через какое время шарики вновь окажутся в том же положении?
}
\answer{%
    \begin{align*}
    F &= -k \cdot 2x = ma \implies a + \frac {2k}m x = 0 \implies \omega^2 = \frac {2k}m \implies T =\frac{2\pi}\omega = 2\pi\sqrt{\frac m{2k}}, \\
    \frac{mv^2}2 &< \frac{2k\sqr{\frac d2}}2 \implies \text{шарики между собой не ударятся}, \\
    T &= 2 \cdot \frac \ell v + \frac 12 \cdot 2 \pi\sqrt{\frac m{2k}}.
    \end{align*}
}
\solutionspace{150pt}

\tasknumber{6}%
\task{%
    Доска массой $m$ и длиной $\ell$ скользит по гладкой горизонтальной плоскости со скоростью $v$,
    а после въезжает на шероховатую поверхность с коэффициентом трения $\mu$ (см.
    рис.
    на доске).
    Определите, за какое время доска остановится (от начала въезжания) и на какое расстояние въедет.
    Ускорение свободного падения $g$, $v^2 < \mu g \ell.$
}
\answer{%
    \begin{align*}
    F_\text{трения} &= -\mu m \frac x\ell g = ma \implies a + \mu \frac g\ell x = 0 \implies \omega^2 = \frac {\mu g}\ell, \\
    T &= \frac{2\pi}{\omega} = 2\pi \sqrt{\frac \ell{\mu g}}\implies t = \frac T4 = \frac\pi2 \sqrt{\frac \ell{\mu g}}, \\
    A &= \frac v\omega = v \sqrt{\frac \ell{\mu g}} \quad (\text{отметим, что $A < \sqrt{\mu g \ell} \cdot \sqrt{\frac \ell{\mu g}} = \ell$}).
    \end{align*}
}

\variantsplitter

\addpersonalvariant{Михаил Ярошевский}

\tasknumber{1}%
\task{%
    Определите период колебаний и массу груза в пружинном маятнике,
    если максимальная скорость груза равна $4\,\frac{\text{м}}{\text{с}}$,
    жёсткость пружины $180\,\frac{\text{Н}}{\text{м}}$, а амплитуда колебаний $10\,\text{см}$.
}
\answer{%
    \begin{align*}
    \frac{mv^2}2 &= \frac{kA^2}2 \implies m = k\sqr{\frac{A}{v}} \approx 113\,\text{г}, \\
    T &= \frac{2\pi}\omega = 2\pi\sqrt{\frac mk} = 2\pi\frac{A}{v} \approx 0{,}157\,\text{c}.
    \end{align*}
}
\solutionspace{80pt}

\tasknumber{2}%
\task{%
    Тело совершает гармонические колебания с периодом $T$ и амплитудой $A$.
    Определите, какую долю периода тело находится на расстоянии менее $\frac 12A$ от положения равновесия.
}
\answer{%
    $t = \frac13T \implies \frac tT = \frac13$
}
\solutionspace{80pt}

\tasknumber{3}%
\task{%
    Определите период малых колебаний маятника (см.
    рис.
    на доске),
    если $\ell = 0{,}75\,\text{м}$, а $h = 25\,\text{см}$.
}
\answer{%
    $
        T = 2 \cdot \cfrac{T_1}4 + 2 \cdot \cfrac{T_2}4 = \cfrac{T_1 + T_2}2
        = \cfrac{2\pi\sqrt{\frac {\ell}{g}} + 2\pi\sqrt{\frac {h}{g} } }2
        = \pi\cbr{\sqrt{\frac {\ell}{g}} + \sqrt{\frac {h}{g} } }
        \approx 1{,}357\,\text{c}.
    $
}
\solutionspace{80pt}

\tasknumber{4}%
\task{%
    Определите период колебаний жидкости в тонкой $U$-образной трубке постоянного сечения,
    если длина столбика жидкости равна $85\,\text{см}$ (см.
    рис.
    на доске).
    Вязкостью, силами трения и сопротивления, капиллярными эффектами и притяжением Сатурна, пренебречь.
}
\answer{%
    $
        \Delta F = -m \frac {2x}\ell g = ma
        \implies a + \frac{2g}\ell x = 0
        \implies \omega^2 = \frac{2g}\ell
        \implies T = \frac{2\pi}\omega = 2\pi\sqrt{\frac\ell{2g}}
        \approx $1{,}295\,\text{c}$.
    $
}
\solutionspace{100pt}

\tasknumber{5}%
\task{%
    2 маленьких шарика скреплены лёгкой пружиной жёсткостью $k$, масса каждого шарика $m$.
    Пружина недеформирована, система летит со скоростью $v$ к стенке,
    расстояние от ближайшего шарика до стены $\ell$ (см.
    рис.
    на доске).
    Все удары упругие, трением пренебречь, $\frac{mv^2}{kd^2} < \frac 12,$ где $d$ — длина пружины.
    Через какое время шарики вновь окажутся в том же положении?
}
\answer{%
    \begin{align*}
    F &= -k \cdot 2x = ma \implies a + \frac {2k}m x = 0 \implies \omega^2 = \frac {2k}m \implies T =\frac{2\pi}\omega = 2\pi\sqrt{\frac m{2k}}, \\
    \frac{mv^2}2 &< \frac{2k\sqr{\frac d2}}2 \implies \text{шарики между собой не ударятся}, \\
    T &= 2 \cdot \frac \ell v + \frac 12 \cdot 2 \pi\sqrt{\frac m{2k}}.
    \end{align*}
}
\solutionspace{150pt}

\tasknumber{6}%
\task{%
    Доска массой $m$ и длиной $\ell$ скользит по гладкой горизонтальной плоскости со скоростью $v$,
    а после въезжает на шероховатую поверхность с коэффициентом трения $\mu$ (см.
    рис.
    на доске).
    Определите, за какое время доска остановится (от начала въезжания) и на какое расстояние въедет.
    Ускорение свободного падения $g$, $v^2 < \mu g \ell.$
}
\answer{%
    \begin{align*}
    F_\text{трения} &= -\mu m \frac x\ell g = ma \implies a + \mu \frac g\ell x = 0 \implies \omega^2 = \frac {\mu g}\ell, \\
    T &= \frac{2\pi}{\omega} = 2\pi \sqrt{\frac \ell{\mu g}}\implies t = \frac T4 = \frac\pi2 \sqrt{\frac \ell{\mu g}}, \\
    A &= \frac v\omega = v \sqrt{\frac \ell{\mu g}} \quad (\text{отметим, что $A < \sqrt{\mu g \ell} \cdot \sqrt{\frac \ell{\mu g}} = \ell$}).
    \end{align*}
}

\variantsplitter

\addpersonalvariant{Алексей Алимпиев}

\tasknumber{1}%
\task{%
    Определите период колебаний и массу груза в пружинном маятнике,
    если максимальная скорость груза равна $3\,\frac{\text{м}}{\text{с}}$,
    жёсткость пружины $240\,\frac{\text{Н}}{\text{м}}$, а амплитуда колебаний $10\,\text{см}$.
}
\answer{%
    \begin{align*}
    \frac{mv^2}2 &= \frac{kA^2}2 \implies m = k\sqr{\frac{A}{v}} \approx 267\,\text{г}, \\
    T &= \frac{2\pi}\omega = 2\pi\sqrt{\frac mk} = 2\pi\frac{A}{v} \approx 0{,}209\,\text{c}.
    \end{align*}
}
\solutionspace{80pt}

\tasknumber{2}%
\task{%
    Тело совершает гармонические колебания с периодом $T$ и амплитудой $A$.
    Определите, какую долю периода тело находится на расстоянии менее $\frac 12A$ от положения равновесия.
}
\answer{%
    $t = \frac13T \implies \frac tT = \frac13$
}
\solutionspace{80pt}

\tasknumber{3}%
\task{%
    Определите период малых колебаний маятника (см.
    рис.
    на доске),
    если $\ell = 0{,}6\,\text{м}$, а $h = 25\,\text{см}$.
}
\answer{%
    $
        T = 2 \cdot \cfrac{T_1}4 + 2 \cdot \cfrac{T_2}4 = \cfrac{T_1 + T_2}2
        = \cfrac{2\pi\sqrt{\frac {\ell}{g}} + 2\pi\sqrt{\frac {h}{g} } }2
        = \pi\cbr{\sqrt{\frac {\ell}{g}} + \sqrt{\frac {h}{g} } }
        \approx 1{,}266\,\text{c}.
    $
}
\solutionspace{80pt}

\tasknumber{4}%
\task{%
    Определите период колебаний жидкости в тонкой $U$-образной трубке постоянного сечения,
    если длина столбика жидкости равна $75\,\text{см}$ (см.
    рис.
    на доске).
    Вязкостью, силами трения и сопротивления, капиллярными эффектами и притяжением Юпитера, пренебречь.
}
\answer{%
    $
        \Delta F = -m \frac {2x}\ell g = ma
        \implies a + \frac{2g}\ell x = 0
        \implies \omega^2 = \frac{2g}\ell
        \implies T = \frac{2\pi}\omega = 2\pi\sqrt{\frac\ell{2g}}
        \approx $1{,}217\,\text{c}$.
    $
}
\solutionspace{100pt}

\tasknumber{5}%
\task{%
    2 маленьких шарика скреплены лёгкой пружиной жёсткостью $k$, масса каждого шарика $m$.
    Пружина недеформирована, система летит со скоростью $v$ к стенке,
    расстояние от ближайшего шарика до стены $\ell$ (см.
    рис.
    на доске).
    Все удары упругие, трением пренебречь, $\frac{mv^2}{kd^2} < \frac 12,$ где $d$ — длина пружины.
    Через какое время шарики вновь окажутся в том же положении?
}
\answer{%
    \begin{align*}
    F &= -k \cdot 2x = ma \implies a + \frac {2k}m x = 0 \implies \omega^2 = \frac {2k}m \implies T =\frac{2\pi}\omega = 2\pi\sqrt{\frac m{2k}}, \\
    \frac{mv^2}2 &< \frac{2k\sqr{\frac d2}}2 \implies \text{шарики между собой не ударятся}, \\
    T &= 2 \cdot \frac \ell v + \frac 12 \cdot 2 \pi\sqrt{\frac m{2k}}.
    \end{align*}
}
\solutionspace{150pt}

\tasknumber{6}%
\task{%
    Доска массой $m$ и длиной $\ell$ скользит по гладкой горизонтальной плоскости со скоростью $v$,
    а после въезжает на шероховатую поверхность с коэффициентом трения $\mu$ (см.
    рис.
    на доске).
    Определите, за какое время доска остановится (от начала въезжания) и на какое расстояние въедет.
    Ускорение свободного падения $g$, $v^2 < \mu g \ell.$
}
\answer{%
    \begin{align*}
    F_\text{трения} &= -\mu m \frac x\ell g = ma \implies a + \mu \frac g\ell x = 0 \implies \omega^2 = \frac {\mu g}\ell, \\
    T &= \frac{2\pi}{\omega} = 2\pi \sqrt{\frac \ell{\mu g}}\implies t = \frac T4 = \frac\pi2 \sqrt{\frac \ell{\mu g}}, \\
    A &= \frac v\omega = v \sqrt{\frac \ell{\mu g}} \quad (\text{отметим, что $A < \sqrt{\mu g \ell} \cdot \sqrt{\frac \ell{\mu g}} = \ell$}).
    \end{align*}
}

\variantsplitter

\addpersonalvariant{Евгений Васин}

\tasknumber{1}%
\task{%
    Определите период колебаний и массу груза в пружинном маятнике,
    если максимальная скорость груза равна $4\,\frac{\text{м}}{\text{с}}$,
    жёсткость пружины $300\,\frac{\text{Н}}{\text{м}}$, а амплитуда колебаний $12\,\text{см}$.
}
\answer{%
    \begin{align*}
    \frac{mv^2}2 &= \frac{kA^2}2 \implies m = k\sqr{\frac{A}{v}} \approx 270\,\text{г}, \\
    T &= \frac{2\pi}\omega = 2\pi\sqrt{\frac mk} = 2\pi\frac{A}{v} \approx 0{,}188\,\text{c}.
    \end{align*}
}
\solutionspace{80pt}

\tasknumber{2}%
\task{%
    Тело совершает гармонические колебания с периодом $T$ и амплитудой $A$.
    Определите, какую долю периода тело находится на расстоянии менее $\frac 12A$ от положения равновесия.
}
\answer{%
    $t = \frac13T \implies \frac tT = \frac13$
}
\solutionspace{80pt}

\tasknumber{3}%
\task{%
    Определите период малых колебаний маятника (см.
    рис.
    на доске),
    если $\ell = 0{,}9\,\text{м}$, а $h = 25\,\text{см}$.
}
\answer{%
    $
        T = 2 \cdot \cfrac{T_1}4 + 2 \cdot \cfrac{T_2}4 = \cfrac{T_1 + T_2}2
        = \cfrac{2\pi\sqrt{\frac {\ell}{g}} + 2\pi\sqrt{\frac {h}{g} } }2
        = \pi\cbr{\sqrt{\frac {\ell}{g}} + \sqrt{\frac {h}{g} } }
        \approx 1{,}439\,\text{c}.
    $
}
\solutionspace{80pt}

\tasknumber{4}%
\task{%
    Определите период колебаний жидкости в тонкой $U$-образной трубке постоянного сечения,
    если длина столбика жидкости равна $85\,\text{см}$ (см.
    рис.
    на доске).
    Вязкостью, силами трения и сопротивления, капиллярными эффектами и притяжением Сатурна, пренебречь.
}
\answer{%
    $
        \Delta F = -m \frac {2x}\ell g = ma
        \implies a + \frac{2g}\ell x = 0
        \implies \omega^2 = \frac{2g}\ell
        \implies T = \frac{2\pi}\omega = 2\pi\sqrt{\frac\ell{2g}}
        \approx $1{,}295\,\text{c}$.
    $
}
\solutionspace{100pt}

\tasknumber{5}%
\task{%
    2 маленьких шарика скреплены лёгкой пружиной жёсткостью $k$, масса каждого шарика $m$.
    Пружина недеформирована, система летит со скоростью $v$ к стенке,
    расстояние от ближайшего шарика до стены $\ell$ (см.
    рис.
    на доске).
    Все удары упругие, трением пренебречь, $\frac{mv^2}{kd^2} < \frac 12,$ где $d$ — длина пружины.
    Через какое время шарики вновь окажутся в том же положении?
}
\answer{%
    \begin{align*}
    F &= -k \cdot 2x = ma \implies a + \frac {2k}m x = 0 \implies \omega^2 = \frac {2k}m \implies T =\frac{2\pi}\omega = 2\pi\sqrt{\frac m{2k}}, \\
    \frac{mv^2}2 &< \frac{2k\sqr{\frac d2}}2 \implies \text{шарики между собой не ударятся}, \\
    T &= 2 \cdot \frac \ell v + \frac 12 \cdot 2 \pi\sqrt{\frac m{2k}}.
    \end{align*}
}
\solutionspace{150pt}

\tasknumber{6}%
\task{%
    Доска массой $m$ и длиной $\ell$ скользит по гладкой горизонтальной плоскости со скоростью $v$,
    а после въезжает на шероховатую поверхность с коэффициентом трения $\mu$ (см.
    рис.
    на доске).
    Определите, за какое время доска остановится (от начала въезжания) и на какое расстояние въедет.
    Ускорение свободного падения $g$, $v^2 < \mu g \ell.$
}
\answer{%
    \begin{align*}
    F_\text{трения} &= -\mu m \frac x\ell g = ma \implies a + \mu \frac g\ell x = 0 \implies \omega^2 = \frac {\mu g}\ell, \\
    T &= \frac{2\pi}{\omega} = 2\pi \sqrt{\frac \ell{\mu g}}\implies t = \frac T4 = \frac\pi2 \sqrt{\frac \ell{\mu g}}, \\
    A &= \frac v\omega = v \sqrt{\frac \ell{\mu g}} \quad (\text{отметим, что $A < \sqrt{\mu g \ell} \cdot \sqrt{\frac \ell{\mu g}} = \ell$}).
    \end{align*}
}

\variantsplitter

\addpersonalvariant{Вячеслав Волохов}

\tasknumber{1}%
\task{%
    Определите период колебаний и массу груза в пружинном маятнике,
    если максимальная скорость груза равна $3\,\frac{\text{м}}{\text{с}}$,
    жёсткость пружины $240\,\frac{\text{Н}}{\text{м}}$, а амплитуда колебаний $20\,\text{см}$.
}
\answer{%
    \begin{align*}
    \frac{mv^2}2 &= \frac{kA^2}2 \implies m = k\sqr{\frac{A}{v}} \approx 1067\,\text{г}, \\
    T &= \frac{2\pi}\omega = 2\pi\sqrt{\frac mk} = 2\pi\frac{A}{v} \approx 0{,}419\,\text{c}.
    \end{align*}
}
\solutionspace{80pt}

\tasknumber{2}%
\task{%
    Тело совершает гармонические колебания с периодом $T$ и амплитудой $A$.
    Определите, какую долю периода тело находится на расстоянии менее $\frac 12A$ от положения равновесия.
}
\answer{%
    $t = \frac13T \implies \frac tT = \frac13$
}
\solutionspace{80pt}

\tasknumber{3}%
\task{%
    Определите период малых колебаний маятника (см.
    рис.
    на доске),
    если $\ell = 0{,}9\,\text{м}$, а $h = 20\,\text{см}$.
}
\answer{%
    $
        T = 2 \cdot \cfrac{T_1}4 + 2 \cdot \cfrac{T_2}4 = \cfrac{T_1 + T_2}2
        = \cfrac{2\pi\sqrt{\frac {\ell}{g}} + 2\pi\sqrt{\frac {h}{g} } }2
        = \pi\cbr{\sqrt{\frac {\ell}{g}} + \sqrt{\frac {h}{g} } }
        \approx 1{,}387\,\text{c}.
    $
}
\solutionspace{80pt}

\tasknumber{4}%
\task{%
    Определите период колебаний жидкости в тонкой $U$-образной трубке постоянного сечения,
    если длина столбика жидкости равна $45\,\text{см}$ (см.
    рис.
    на доске).
    Вязкостью, силами трения и сопротивления, капиллярными эффектами и притяжением Юпитера, пренебречь.
}
\answer{%
    $
        \Delta F = -m \frac {2x}\ell g = ma
        \implies a + \frac{2g}\ell x = 0
        \implies \omega^2 = \frac{2g}\ell
        \implies T = \frac{2\pi}\omega = 2\pi\sqrt{\frac\ell{2g}}
        \approx $0{,}942\,\text{c}$.
    $
}
\solutionspace{100pt}

\tasknumber{5}%
\task{%
    2 маленьких шарика скреплены лёгкой пружиной жёсткостью $k$, масса каждого шарика $m$.
    Пружина недеформирована, система летит со скоростью $v$ к стенке,
    расстояние от ближайшего шарика до стены $\ell$ (см.
    рис.
    на доске).
    Все удары упругие, трением пренебречь, $\frac{mv^2}{kd^2} < \frac 12,$ где $d$ — длина пружины.
    Через какое время шарики вновь окажутся в том же положении?
}
\answer{%
    \begin{align*}
    F &= -k \cdot 2x = ma \implies a + \frac {2k}m x = 0 \implies \omega^2 = \frac {2k}m \implies T =\frac{2\pi}\omega = 2\pi\sqrt{\frac m{2k}}, \\
    \frac{mv^2}2 &< \frac{2k\sqr{\frac d2}}2 \implies \text{шарики между собой не ударятся}, \\
    T &= 2 \cdot \frac \ell v + \frac 12 \cdot 2 \pi\sqrt{\frac m{2k}}.
    \end{align*}
}
\solutionspace{150pt}

\tasknumber{6}%
\task{%
    Доска массой $m$ и длиной $\ell$ скользит по гладкой горизонтальной плоскости со скоростью $v$,
    а после въезжает на шероховатую поверхность с коэффициентом трения $\mu$ (см.
    рис.
    на доске).
    Определите, за какое время доска остановится (от начала въезжания) и на какое расстояние въедет.
    Ускорение свободного падения $g$, $v^2 < \mu g \ell.$
}
\answer{%
    \begin{align*}
    F_\text{трения} &= -\mu m \frac x\ell g = ma \implies a + \mu \frac g\ell x = 0 \implies \omega^2 = \frac {\mu g}\ell, \\
    T &= \frac{2\pi}{\omega} = 2\pi \sqrt{\frac \ell{\mu g}}\implies t = \frac T4 = \frac\pi2 \sqrt{\frac \ell{\mu g}}, \\
    A &= \frac v\omega = v \sqrt{\frac \ell{\mu g}} \quad (\text{отметим, что $A < \sqrt{\mu g \ell} \cdot \sqrt{\frac \ell{\mu g}} = \ell$}).
    \end{align*}
}

\variantsplitter

\addpersonalvariant{Герман Говоров}

\tasknumber{1}%
\task{%
    Определите период колебаний и массу груза в пружинном маятнике,
    если максимальная скорость груза равна $3\,\frac{\text{м}}{\text{с}}$,
    жёсткость пружины $300\,\frac{\text{Н}}{\text{м}}$, а амплитуда колебаний $10\,\text{см}$.
}
\answer{%
    \begin{align*}
    \frac{mv^2}2 &= \frac{kA^2}2 \implies m = k\sqr{\frac{A}{v}} \approx 334\,\text{г}, \\
    T &= \frac{2\pi}\omega = 2\pi\sqrt{\frac mk} = 2\pi\frac{A}{v} \approx 0{,}209\,\text{c}.
    \end{align*}
}
\solutionspace{80pt}

\tasknumber{2}%
\task{%
    Тело совершает гармонические колебания с периодом $T$ и амплитудой $A$.
    Определите, какую долю периода тело находится на расстоянии более $\frac 12A$ от положения равновесия.
}
\answer{%
    $t = \frac23T \implies \frac tT = \frac23$
}
\solutionspace{80pt}

\tasknumber{3}%
\task{%
    Определите период малых колебаний маятника (см.
    рис.
    на доске),
    если $\ell = 0{,}75\,\text{м}$, а $h = 25\,\text{см}$.
}
\answer{%
    $
        T = 2 \cdot \cfrac{T_1}4 + 2 \cdot \cfrac{T_2}4 = \cfrac{T_1 + T_2}2
        = \cfrac{2\pi\sqrt{\frac {\ell}{g}} + 2\pi\sqrt{\frac {h}{g} } }2
        = \pi\cbr{\sqrt{\frac {\ell}{g}} + \sqrt{\frac {h}{g} } }
        \approx 1{,}357\,\text{c}.
    $
}
\solutionspace{80pt}

\tasknumber{4}%
\task{%
    Определите период колебаний жидкости в тонкой $U$-образной трубке постоянного сечения,
    если длина столбика жидкости равна $75\,\text{см}$ (см.
    рис.
    на доске).
    Вязкостью, силами трения и сопротивления, капиллярными эффектами и притяжением Марса, пренебречь.
}
\answer{%
    $
        \Delta F = -m \frac {2x}\ell g = ma
        \implies a + \frac{2g}\ell x = 0
        \implies \omega^2 = \frac{2g}\ell
        \implies T = \frac{2\pi}\omega = 2\pi\sqrt{\frac\ell{2g}}
        \approx $1{,}217\,\text{c}$.
    $
}
\solutionspace{100pt}

\tasknumber{5}%
\task{%
    2 маленьких шарика скреплены лёгкой пружиной жёсткостью $k$, масса каждого шарика $m$.
    Пружина недеформирована, система летит со скоростью $v$ к стенке,
    расстояние от ближайшего шарика до стены $\ell$ (см.
    рис.
    на доске).
    Все удары упругие, трением пренебречь, $\frac{mv^2}{kd^2} < \frac 12,$ где $d$ — длина пружины.
    Через какое время шарики вновь окажутся в том же положении?
}
\answer{%
    \begin{align*}
    F &= -k \cdot 2x = ma \implies a + \frac {2k}m x = 0 \implies \omega^2 = \frac {2k}m \implies T =\frac{2\pi}\omega = 2\pi\sqrt{\frac m{2k}}, \\
    \frac{mv^2}2 &< \frac{2k\sqr{\frac d2}}2 \implies \text{шарики между собой не ударятся}, \\
    T &= 2 \cdot \frac \ell v + \frac 12 \cdot 2 \pi\sqrt{\frac m{2k}}.
    \end{align*}
}
\solutionspace{150pt}

\tasknumber{6}%
\task{%
    Доска массой $m$ и длиной $\ell$ скользит по гладкой горизонтальной плоскости со скоростью $v$,
    а после въезжает на шероховатую поверхность с коэффициентом трения $\mu$ (см.
    рис.
    на доске).
    Определите, за какое время доска остановится (от начала въезжания) и на какое расстояние въедет.
    Ускорение свободного падения $g$, $v^2 < \mu g \ell.$
}
\answer{%
    \begin{align*}
    F_\text{трения} &= -\mu m \frac x\ell g = ma \implies a + \mu \frac g\ell x = 0 \implies \omega^2 = \frac {\mu g}\ell, \\
    T &= \frac{2\pi}{\omega} = 2\pi \sqrt{\frac \ell{\mu g}}\implies t = \frac T4 = \frac\pi2 \sqrt{\frac \ell{\mu g}}, \\
    A &= \frac v\omega = v \sqrt{\frac \ell{\mu g}} \quad (\text{отметим, что $A < \sqrt{\mu g \ell} \cdot \sqrt{\frac \ell{\mu g}} = \ell$}).
    \end{align*}
}

\variantsplitter

\addpersonalvariant{София Журавлёва}

\tasknumber{1}%
\task{%
    Определите период колебаний и массу груза в пружинном маятнике,
    если максимальная скорость груза равна $3\,\frac{\text{м}}{\text{с}}$,
    жёсткость пружины $180\,\frac{\text{Н}}{\text{м}}$, а амплитуда колебаний $20\,\text{см}$.
}
\answer{%
    \begin{align*}
    \frac{mv^2}2 &= \frac{kA^2}2 \implies m = k\sqr{\frac{A}{v}} \approx 800\,\text{г}, \\
    T &= \frac{2\pi}\omega = 2\pi\sqrt{\frac mk} = 2\pi\frac{A}{v} \approx 0{,}419\,\text{c}.
    \end{align*}
}
\solutionspace{80pt}

\tasknumber{2}%
\task{%
    Тело совершает гармонические колебания с периодом $T$ и амплитудой $A$.
    Определите, какую долю периода тело находится на расстоянии менее $\frac{\sqrt 2}2A$ от положения равновесия.
}
\answer{%
    $t = \frac12T \implies \frac tT = \frac12$
}
\solutionspace{80pt}

\tasknumber{3}%
\task{%
    Определите период малых колебаний маятника (см.
    рис.
    на доске),
    если $\ell = 0{,}9\,\text{м}$, а $h = 35\,\text{см}$.
}
\answer{%
    $
        T = 2 \cdot \cfrac{T_1}4 + 2 \cdot \cfrac{T_2}4 = \cfrac{T_1 + T_2}2
        = \cfrac{2\pi\sqrt{\frac {\ell}{g}} + 2\pi\sqrt{\frac {h}{g} } }2
        = \pi\cbr{\sqrt{\frac {\ell}{g}} + \sqrt{\frac {h}{g} } }
        \approx 1{,}530\,\text{c}.
    $
}
\solutionspace{80pt}

\tasknumber{4}%
\task{%
    Определите период колебаний жидкости в тонкой $U$-образной трубке постоянного сечения,
    если длина столбика жидкости равна $75\,\text{см}$ (см.
    рис.
    на доске).
    Вязкостью, силами трения и сопротивления, капиллярными эффектами и притяжением Юпитера, пренебречь.
}
\answer{%
    $
        \Delta F = -m \frac {2x}\ell g = ma
        \implies a + \frac{2g}\ell x = 0
        \implies \omega^2 = \frac{2g}\ell
        \implies T = \frac{2\pi}\omega = 2\pi\sqrt{\frac\ell{2g}}
        \approx $1{,}217\,\text{c}$.
    $
}
\solutionspace{100pt}

\tasknumber{5}%
\task{%
    2 маленьких шарика скреплены лёгкой пружиной жёсткостью $k$, масса каждого шарика $m$.
    Пружина недеформирована, система летит со скоростью $v$ к стенке,
    расстояние от ближайшего шарика до стены $\ell$ (см.
    рис.
    на доске).
    Все удары упругие, трением пренебречь, $\frac{mv^2}{kd^2} < \frac 12,$ где $d$ — длина пружины.
    Через какое время шарики вновь окажутся в том же положении?
}
\answer{%
    \begin{align*}
    F &= -k \cdot 2x = ma \implies a + \frac {2k}m x = 0 \implies \omega^2 = \frac {2k}m \implies T =\frac{2\pi}\omega = 2\pi\sqrt{\frac m{2k}}, \\
    \frac{mv^2}2 &< \frac{2k\sqr{\frac d2}}2 \implies \text{шарики между собой не ударятся}, \\
    T &= 2 \cdot \frac \ell v + \frac 12 \cdot 2 \pi\sqrt{\frac m{2k}}.
    \end{align*}
}
\solutionspace{150pt}

\tasknumber{6}%
\task{%
    Доска массой $m$ и длиной $\ell$ скользит по гладкой горизонтальной плоскости со скоростью $v$,
    а после въезжает на шероховатую поверхность с коэффициентом трения $\mu$ (см.
    рис.
    на доске).
    Определите, за какое время доска остановится (от начала въезжания) и на какое расстояние въедет.
    Ускорение свободного падения $g$, $v^2 < \mu g \ell.$
}
\answer{%
    \begin{align*}
    F_\text{трения} &= -\mu m \frac x\ell g = ma \implies a + \mu \frac g\ell x = 0 \implies \omega^2 = \frac {\mu g}\ell, \\
    T &= \frac{2\pi}{\omega} = 2\pi \sqrt{\frac \ell{\mu g}}\implies t = \frac T4 = \frac\pi2 \sqrt{\frac \ell{\mu g}}, \\
    A &= \frac v\omega = v \sqrt{\frac \ell{\mu g}} \quad (\text{отметим, что $A < \sqrt{\mu g \ell} \cdot \sqrt{\frac \ell{\mu g}} = \ell$}).
    \end{align*}
}

\variantsplitter

\addpersonalvariant{Константин Козлов}

\tasknumber{1}%
\task{%
    Определите период колебаний и массу груза в пружинном маятнике,
    если максимальная скорость груза равна $5\,\frac{\text{м}}{\text{с}}$,
    жёсткость пружины $180\,\frac{\text{Н}}{\text{м}}$, а амплитуда колебаний $20\,\text{см}$.
}
\answer{%
    \begin{align*}
    \frac{mv^2}2 &= \frac{kA^2}2 \implies m = k\sqr{\frac{A}{v}} \approx 288\,\text{г}, \\
    T &= \frac{2\pi}\omega = 2\pi\sqrt{\frac mk} = 2\pi\frac{A}{v} \approx 0{,}251\,\text{c}.
    \end{align*}
}
\solutionspace{80pt}

\tasknumber{2}%
\task{%
    Тело совершает гармонические колебания с периодом $T$ и амплитудой $A$.
    Определите, какую долю периода тело находится на расстоянии более $\frac 12A$ от положения равновесия.
}
\answer{%
    $t = \frac23T \implies \frac tT = \frac23$
}
\solutionspace{80pt}

\tasknumber{3}%
\task{%
    Определите период малых колебаний маятника (см.
    рис.
    на доске),
    если $\ell = 0{,}75\,\text{м}$, а $h = 20\,\text{см}$.
}
\answer{%
    $
        T = 2 \cdot \cfrac{T_1}4 + 2 \cdot \cfrac{T_2}4 = \cfrac{T_1 + T_2}2
        = \cfrac{2\pi\sqrt{\frac {\ell}{g}} + 2\pi\sqrt{\frac {h}{g} } }2
        = \pi\cbr{\sqrt{\frac {\ell}{g}} + \sqrt{\frac {h}{g} } }
        \approx 1{,}305\,\text{c}.
    $
}
\solutionspace{80pt}

\tasknumber{4}%
\task{%
    Определите период колебаний жидкости в тонкой $U$-образной трубке постоянного сечения,
    если длина столбика жидкости равна $85\,\text{см}$ (см.
    рис.
    на доске).
    Вязкостью, силами трения и сопротивления, капиллярными эффектами и притяжением Сатурна, пренебречь.
}
\answer{%
    $
        \Delta F = -m \frac {2x}\ell g = ma
        \implies a + \frac{2g}\ell x = 0
        \implies \omega^2 = \frac{2g}\ell
        \implies T = \frac{2\pi}\omega = 2\pi\sqrt{\frac\ell{2g}}
        \approx $1{,}295\,\text{c}$.
    $
}
\solutionspace{100pt}

\tasknumber{5}%
\task{%
    2 маленьких шарика скреплены лёгкой пружиной жёсткостью $k$, масса каждого шарика $m$.
    Пружина недеформирована, система летит со скоростью $v$ к стенке,
    расстояние от ближайшего шарика до стены $\ell$ (см.
    рис.
    на доске).
    Все удары упругие, трением пренебречь, $\frac{mv^2}{kd^2} < \frac 12,$ где $d$ — длина пружины.
    Через какое время шарики вновь окажутся в том же положении?
}
\answer{%
    \begin{align*}
    F &= -k \cdot 2x = ma \implies a + \frac {2k}m x = 0 \implies \omega^2 = \frac {2k}m \implies T =\frac{2\pi}\omega = 2\pi\sqrt{\frac m{2k}}, \\
    \frac{mv^2}2 &< \frac{2k\sqr{\frac d2}}2 \implies \text{шарики между собой не ударятся}, \\
    T &= 2 \cdot \frac \ell v + \frac 12 \cdot 2 \pi\sqrt{\frac m{2k}}.
    \end{align*}
}
\solutionspace{150pt}

\tasknumber{6}%
\task{%
    Доска массой $m$ и длиной $\ell$ скользит по гладкой горизонтальной плоскости со скоростью $v$,
    а после въезжает на шероховатую поверхность с коэффициентом трения $\mu$ (см.
    рис.
    на доске).
    Определите, за какое время доска остановится (от начала въезжания) и на какое расстояние въедет.
    Ускорение свободного падения $g$, $v^2 < \mu g \ell.$
}
\answer{%
    \begin{align*}
    F_\text{трения} &= -\mu m \frac x\ell g = ma \implies a + \mu \frac g\ell x = 0 \implies \omega^2 = \frac {\mu g}\ell, \\
    T &= \frac{2\pi}{\omega} = 2\pi \sqrt{\frac \ell{\mu g}}\implies t = \frac T4 = \frac\pi2 \sqrt{\frac \ell{\mu g}}, \\
    A &= \frac v\omega = v \sqrt{\frac \ell{\mu g}} \quad (\text{отметим, что $A < \sqrt{\mu g \ell} \cdot \sqrt{\frac \ell{\mu g}} = \ell$}).
    \end{align*}
}

\variantsplitter

\addpersonalvariant{Наталья Кравченко}

\tasknumber{1}%
\task{%
    Определите период колебаний и массу груза в пружинном маятнике,
    если максимальная скорость груза равна $3\,\frac{\text{м}}{\text{с}}$,
    жёсткость пружины $180\,\frac{\text{Н}}{\text{м}}$, а амплитуда колебаний $20\,\text{см}$.
}
\answer{%
    \begin{align*}
    \frac{mv^2}2 &= \frac{kA^2}2 \implies m = k\sqr{\frac{A}{v}} \approx 800\,\text{г}, \\
    T &= \frac{2\pi}\omega = 2\pi\sqrt{\frac mk} = 2\pi\frac{A}{v} \approx 0{,}419\,\text{c}.
    \end{align*}
}
\solutionspace{80pt}

\tasknumber{2}%
\task{%
    Тело совершает гармонические колебания с периодом $T$ и амплитудой $A$.
    Определите, какую долю периода тело находится на расстоянии менее $\frac 12A$ от положения равновесия.
}
\answer{%
    $t = \frac13T \implies \frac tT = \frac13$
}
\solutionspace{80pt}

\tasknumber{3}%
\task{%
    Определите период малых колебаний маятника (см.
    рис.
    на доске),
    если $\ell = 0{,}9\,\text{м}$, а $h = 35\,\text{см}$.
}
\answer{%
    $
        T = 2 \cdot \cfrac{T_1}4 + 2 \cdot \cfrac{T_2}4 = \cfrac{T_1 + T_2}2
        = \cfrac{2\pi\sqrt{\frac {\ell}{g}} + 2\pi\sqrt{\frac {h}{g} } }2
        = \pi\cbr{\sqrt{\frac {\ell}{g}} + \sqrt{\frac {h}{g} } }
        \approx 1{,}530\,\text{c}.
    $
}
\solutionspace{80pt}

\tasknumber{4}%
\task{%
    Определите период колебаний жидкости в тонкой $U$-образной трубке постоянного сечения,
    если длина столбика жидкости равна $45\,\text{см}$ (см.
    рис.
    на доске).
    Вязкостью, силами трения и сопротивления, капиллярными эффектами и притяжением Марса, пренебречь.
}
\answer{%
    $
        \Delta F = -m \frac {2x}\ell g = ma
        \implies a + \frac{2g}\ell x = 0
        \implies \omega^2 = \frac{2g}\ell
        \implies T = \frac{2\pi}\omega = 2\pi\sqrt{\frac\ell{2g}}
        \approx $0{,}942\,\text{c}$.
    $
}
\solutionspace{100pt}

\tasknumber{5}%
\task{%
    2 маленьких шарика скреплены лёгкой пружиной жёсткостью $k$, масса каждого шарика $m$.
    Пружина недеформирована, система летит со скоростью $v$ к стенке,
    расстояние от ближайшего шарика до стены $\ell$ (см.
    рис.
    на доске).
    Все удары упругие, трением пренебречь, $\frac{mv^2}{kd^2} < \frac 12,$ где $d$ — длина пружины.
    Через какое время шарики вновь окажутся в том же положении?
}
\answer{%
    \begin{align*}
    F &= -k \cdot 2x = ma \implies a + \frac {2k}m x = 0 \implies \omega^2 = \frac {2k}m \implies T =\frac{2\pi}\omega = 2\pi\sqrt{\frac m{2k}}, \\
    \frac{mv^2}2 &< \frac{2k\sqr{\frac d2}}2 \implies \text{шарики между собой не ударятся}, \\
    T &= 2 \cdot \frac \ell v + \frac 12 \cdot 2 \pi\sqrt{\frac m{2k}}.
    \end{align*}
}
\solutionspace{150pt}

\tasknumber{6}%
\task{%
    Доска массой $m$ и длиной $\ell$ скользит по гладкой горизонтальной плоскости со скоростью $v$,
    а после въезжает на шероховатую поверхность с коэффициентом трения $\mu$ (см.
    рис.
    на доске).
    Определите, за какое время доска остановится (от начала въезжания) и на какое расстояние въедет.
    Ускорение свободного падения $g$, $v^2 < \mu g \ell.$
}
\answer{%
    \begin{align*}
    F_\text{трения} &= -\mu m \frac x\ell g = ma \implies a + \mu \frac g\ell x = 0 \implies \omega^2 = \frac {\mu g}\ell, \\
    T &= \frac{2\pi}{\omega} = 2\pi \sqrt{\frac \ell{\mu g}}\implies t = \frac T4 = \frac\pi2 \sqrt{\frac \ell{\mu g}}, \\
    A &= \frac v\omega = v \sqrt{\frac \ell{\mu g}} \quad (\text{отметим, что $A < \sqrt{\mu g \ell} \cdot \sqrt{\frac \ell{\mu g}} = \ell$}).
    \end{align*}
}

\variantsplitter

\addpersonalvariant{Сергей Малышев}

\tasknumber{1}%
\task{%
    Определите период колебаний и массу груза в пружинном маятнике,
    если максимальная скорость груза равна $3\,\frac{\text{м}}{\text{с}}$,
    жёсткость пружины $240\,\frac{\text{Н}}{\text{м}}$, а амплитуда колебаний $12\,\text{см}$.
}
\answer{%
    \begin{align*}
    \frac{mv^2}2 &= \frac{kA^2}2 \implies m = k\sqr{\frac{A}{v}} \approx 384\,\text{г}, \\
    T &= \frac{2\pi}\omega = 2\pi\sqrt{\frac mk} = 2\pi\frac{A}{v} \approx 0{,}251\,\text{c}.
    \end{align*}
}
\solutionspace{80pt}

\tasknumber{2}%
\task{%
    Тело совершает гармонические колебания с периодом $T$ и амплитудой $A$.
    Определите, какую долю периода тело находится на расстоянии более $\frac{\sqrt 2}2A$ от положения равновесия.
}
\answer{%
    $t = \frac12T \implies \frac tT = \frac12$
}
\solutionspace{80pt}

\tasknumber{3}%
\task{%
    Определите период малых колебаний маятника (см.
    рис.
    на доске),
    если $\ell = 0{,}9\,\text{м}$, а $h = 25\,\text{см}$.
}
\answer{%
    $
        T = 2 \cdot \cfrac{T_1}4 + 2 \cdot \cfrac{T_2}4 = \cfrac{T_1 + T_2}2
        = \cfrac{2\pi\sqrt{\frac {\ell}{g}} + 2\pi\sqrt{\frac {h}{g} } }2
        = \pi\cbr{\sqrt{\frac {\ell}{g}} + \sqrt{\frac {h}{g} } }
        \approx 1{,}439\,\text{c}.
    $
}
\solutionspace{80pt}

\tasknumber{4}%
\task{%
    Определите период колебаний жидкости в тонкой $U$-образной трубке постоянного сечения,
    если длина столбика жидкости равна $65\,\text{см}$ (см.
    рис.
    на доске).
    Вязкостью, силами трения и сопротивления, капиллярными эффектами и притяжением Марса, пренебречь.
}
\answer{%
    $
        \Delta F = -m \frac {2x}\ell g = ma
        \implies a + \frac{2g}\ell x = 0
        \implies \omega^2 = \frac{2g}\ell
        \implies T = \frac{2\pi}\omega = 2\pi\sqrt{\frac\ell{2g}}
        \approx $1{,}133\,\text{c}$.
    $
}
\solutionspace{100pt}

\tasknumber{5}%
\task{%
    2 маленьких шарика скреплены лёгкой пружиной жёсткостью $k$, масса каждого шарика $m$.
    Пружина недеформирована, система летит со скоростью $v$ к стенке,
    расстояние от ближайшего шарика до стены $\ell$ (см.
    рис.
    на доске).
    Все удары упругие, трением пренебречь, $\frac{mv^2}{kd^2} < \frac 12,$ где $d$ — длина пружины.
    Через какое время шарики вновь окажутся в том же положении?
}
\answer{%
    \begin{align*}
    F &= -k \cdot 2x = ma \implies a + \frac {2k}m x = 0 \implies \omega^2 = \frac {2k}m \implies T =\frac{2\pi}\omega = 2\pi\sqrt{\frac m{2k}}, \\
    \frac{mv^2}2 &< \frac{2k\sqr{\frac d2}}2 \implies \text{шарики между собой не ударятся}, \\
    T &= 2 \cdot \frac \ell v + \frac 12 \cdot 2 \pi\sqrt{\frac m{2k}}.
    \end{align*}
}
\solutionspace{150pt}

\tasknumber{6}%
\task{%
    Доска массой $m$ и длиной $\ell$ скользит по гладкой горизонтальной плоскости со скоростью $v$,
    а после въезжает на шероховатую поверхность с коэффициентом трения $\mu$ (см.
    рис.
    на доске).
    Определите, за какое время доска остановится (от начала въезжания) и на какое расстояние въедет.
    Ускорение свободного падения $g$, $v^2 < \mu g \ell.$
}
\answer{%
    \begin{align*}
    F_\text{трения} &= -\mu m \frac x\ell g = ma \implies a + \mu \frac g\ell x = 0 \implies \omega^2 = \frac {\mu g}\ell, \\
    T &= \frac{2\pi}{\omega} = 2\pi \sqrt{\frac \ell{\mu g}}\implies t = \frac T4 = \frac\pi2 \sqrt{\frac \ell{\mu g}}, \\
    A &= \frac v\omega = v \sqrt{\frac \ell{\mu g}} \quad (\text{отметим, что $A < \sqrt{\mu g \ell} \cdot \sqrt{\frac \ell{\mu g}} = \ell$}).
    \end{align*}
}

\variantsplitter

\addpersonalvariant{Алина Полканова}

\tasknumber{1}%
\task{%
    Определите период колебаний и массу груза в пружинном маятнике,
    если максимальная скорость груза равна $2\,\frac{\text{м}}{\text{с}}$,
    жёсткость пружины $240\,\frac{\text{Н}}{\text{м}}$, а амплитуда колебаний $15\,\text{см}$.
}
\answer{%
    \begin{align*}
    \frac{mv^2}2 &= \frac{kA^2}2 \implies m = k\sqr{\frac{A}{v}} \approx 1350\,\text{г}, \\
    T &= \frac{2\pi}\omega = 2\pi\sqrt{\frac mk} = 2\pi\frac{A}{v} \approx 0{,}471\,\text{c}.
    \end{align*}
}
\solutionspace{80pt}

\tasknumber{2}%
\task{%
    Тело совершает гармонические колебания с периодом $T$ и амплитудой $A$.
    Определите, какую долю периода тело находится на расстоянии более $\frac{\sqrt 2}2A$ от положения равновесия.
}
\answer{%
    $t = \frac12T \implies \frac tT = \frac12$
}
\solutionspace{80pt}

\tasknumber{3}%
\task{%
    Определите период малых колебаний маятника (см.
    рис.
    на доске),
    если $\ell = 0{,}9\,\text{м}$, а $h = 25\,\text{см}$.
}
\answer{%
    $
        T = 2 \cdot \cfrac{T_1}4 + 2 \cdot \cfrac{T_2}4 = \cfrac{T_1 + T_2}2
        = \cfrac{2\pi\sqrt{\frac {\ell}{g}} + 2\pi\sqrt{\frac {h}{g} } }2
        = \pi\cbr{\sqrt{\frac {\ell}{g}} + \sqrt{\frac {h}{g} } }
        \approx 1{,}439\,\text{c}.
    $
}
\solutionspace{80pt}

\tasknumber{4}%
\task{%
    Определите период колебаний жидкости в тонкой $U$-образной трубке постоянного сечения,
    если длина столбика жидкости равна $55\,\text{см}$ (см.
    рис.
    на доске).
    Вязкостью, силами трения и сопротивления, капиллярными эффектами и притяжением Сатурна, пренебречь.
}
\answer{%
    $
        \Delta F = -m \frac {2x}\ell g = ma
        \implies a + \frac{2g}\ell x = 0
        \implies \omega^2 = \frac{2g}\ell
        \implies T = \frac{2\pi}\omega = 2\pi\sqrt{\frac\ell{2g}}
        \approx $1{,}042\,\text{c}$.
    $
}
\solutionspace{100pt}

\tasknumber{5}%
\task{%
    2 маленьких шарика скреплены лёгкой пружиной жёсткостью $k$, масса каждого шарика $m$.
    Пружина недеформирована, система летит со скоростью $v$ к стенке,
    расстояние от ближайшего шарика до стены $\ell$ (см.
    рис.
    на доске).
    Все удары упругие, трением пренебречь, $\frac{mv^2}{kd^2} < \frac 12,$ где $d$ — длина пружины.
    Через какое время шарики вновь окажутся в том же положении?
}
\answer{%
    \begin{align*}
    F &= -k \cdot 2x = ma \implies a + \frac {2k}m x = 0 \implies \omega^2 = \frac {2k}m \implies T =\frac{2\pi}\omega = 2\pi\sqrt{\frac m{2k}}, \\
    \frac{mv^2}2 &< \frac{2k\sqr{\frac d2}}2 \implies \text{шарики между собой не ударятся}, \\
    T &= 2 \cdot \frac \ell v + \frac 12 \cdot 2 \pi\sqrt{\frac m{2k}}.
    \end{align*}
}
\solutionspace{150pt}

\tasknumber{6}%
\task{%
    Доска массой $m$ и длиной $\ell$ скользит по гладкой горизонтальной плоскости со скоростью $v$,
    а после въезжает на шероховатую поверхность с коэффициентом трения $\mu$ (см.
    рис.
    на доске).
    Определите, за какое время доска остановится (от начала въезжания) и на какое расстояние въедет.
    Ускорение свободного падения $g$, $v^2 < \mu g \ell.$
}
\answer{%
    \begin{align*}
    F_\text{трения} &= -\mu m \frac x\ell g = ma \implies a + \mu \frac g\ell x = 0 \implies \omega^2 = \frac {\mu g}\ell, \\
    T &= \frac{2\pi}{\omega} = 2\pi \sqrt{\frac \ell{\mu g}}\implies t = \frac T4 = \frac\pi2 \sqrt{\frac \ell{\mu g}}, \\
    A &= \frac v\omega = v \sqrt{\frac \ell{\mu g}} \quad (\text{отметим, что $A < \sqrt{\mu g \ell} \cdot \sqrt{\frac \ell{\mu g}} = \ell$}).
    \end{align*}
}

\variantsplitter

\addpersonalvariant{Сергей Пономарёв}

\tasknumber{1}%
\task{%
    Определите период колебаний и массу груза в пружинном маятнике,
    если максимальная скорость груза равна $2\,\frac{\text{м}}{\text{с}}$,
    жёсткость пружины $240\,\frac{\text{Н}}{\text{м}}$, а амплитуда колебаний $20\,\text{см}$.
}
\answer{%
    \begin{align*}
    \frac{mv^2}2 &= \frac{kA^2}2 \implies m = k\sqr{\frac{A}{v}} \approx 2400\,\text{г}, \\
    T &= \frac{2\pi}\omega = 2\pi\sqrt{\frac mk} = 2\pi\frac{A}{v} \approx 0{,}628\,\text{c}.
    \end{align*}
}
\solutionspace{80pt}

\tasknumber{2}%
\task{%
    Тело совершает гармонические колебания с периодом $T$ и амплитудой $A$.
    Определите, какую долю периода тело находится на расстоянии менее $\frac 12A$ от положения равновесия.
}
\answer{%
    $t = \frac13T \implies \frac tT = \frac13$
}
\solutionspace{80pt}

\tasknumber{3}%
\task{%
    Определите период малых колебаний маятника (см.
    рис.
    на доске),
    если $\ell = 0{,}9\,\text{м}$, а $h = 25\,\text{см}$.
}
\answer{%
    $
        T = 2 \cdot \cfrac{T_1}4 + 2 \cdot \cfrac{T_2}4 = \cfrac{T_1 + T_2}2
        = \cfrac{2\pi\sqrt{\frac {\ell}{g}} + 2\pi\sqrt{\frac {h}{g} } }2
        = \pi\cbr{\sqrt{\frac {\ell}{g}} + \sqrt{\frac {h}{g} } }
        \approx 1{,}439\,\text{c}.
    $
}
\solutionspace{80pt}

\tasknumber{4}%
\task{%
    Определите период колебаний жидкости в тонкой $U$-образной трубке постоянного сечения,
    если длина столбика жидкости равна $55\,\text{см}$ (см.
    рис.
    на доске).
    Вязкостью, силами трения и сопротивления, капиллярными эффектами и притяжением Марса, пренебречь.
}
\answer{%
    $
        \Delta F = -m \frac {2x}\ell g = ma
        \implies a + \frac{2g}\ell x = 0
        \implies \omega^2 = \frac{2g}\ell
        \implies T = \frac{2\pi}\omega = 2\pi\sqrt{\frac\ell{2g}}
        \approx $1{,}042\,\text{c}$.
    $
}
\solutionspace{100pt}

\tasknumber{5}%
\task{%
    2 маленьких шарика скреплены лёгкой пружиной жёсткостью $k$, масса каждого шарика $m$.
    Пружина недеформирована, система летит со скоростью $v$ к стенке,
    расстояние от ближайшего шарика до стены $\ell$ (см.
    рис.
    на доске).
    Все удары упругие, трением пренебречь, $\frac{mv^2}{kd^2} < \frac 12,$ где $d$ — длина пружины.
    Через какое время шарики вновь окажутся в том же положении?
}
\answer{%
    \begin{align*}
    F &= -k \cdot 2x = ma \implies a + \frac {2k}m x = 0 \implies \omega^2 = \frac {2k}m \implies T =\frac{2\pi}\omega = 2\pi\sqrt{\frac m{2k}}, \\
    \frac{mv^2}2 &< \frac{2k\sqr{\frac d2}}2 \implies \text{шарики между собой не ударятся}, \\
    T &= 2 \cdot \frac \ell v + \frac 12 \cdot 2 \pi\sqrt{\frac m{2k}}.
    \end{align*}
}
\solutionspace{150pt}

\tasknumber{6}%
\task{%
    Доска массой $m$ и длиной $\ell$ скользит по гладкой горизонтальной плоскости со скоростью $v$,
    а после въезжает на шероховатую поверхность с коэффициентом трения $\mu$ (см.
    рис.
    на доске).
    Определите, за какое время доска остановится (от начала въезжания) и на какое расстояние въедет.
    Ускорение свободного падения $g$, $v^2 < \mu g \ell.$
}
\answer{%
    \begin{align*}
    F_\text{трения} &= -\mu m \frac x\ell g = ma \implies a + \mu \frac g\ell x = 0 \implies \omega^2 = \frac {\mu g}\ell, \\
    T &= \frac{2\pi}{\omega} = 2\pi \sqrt{\frac \ell{\mu g}}\implies t = \frac T4 = \frac\pi2 \sqrt{\frac \ell{\mu g}}, \\
    A &= \frac v\omega = v \sqrt{\frac \ell{\mu g}} \quad (\text{отметим, что $A < \sqrt{\mu g \ell} \cdot \sqrt{\frac \ell{\mu g}} = \ell$}).
    \end{align*}
}

\variantsplitter

\addpersonalvariant{Егор Свистушкин}

\tasknumber{1}%
\task{%
    Определите период колебаний и массу груза в пружинном маятнике,
    если максимальная скорость груза равна $5\,\frac{\text{м}}{\text{с}}$,
    жёсткость пружины $180\,\frac{\text{Н}}{\text{м}}$, а амплитуда колебаний $20\,\text{см}$.
}
\answer{%
    \begin{align*}
    \frac{mv^2}2 &= \frac{kA^2}2 \implies m = k\sqr{\frac{A}{v}} \approx 288\,\text{г}, \\
    T &= \frac{2\pi}\omega = 2\pi\sqrt{\frac mk} = 2\pi\frac{A}{v} \approx 0{,}251\,\text{c}.
    \end{align*}
}
\solutionspace{80pt}

\tasknumber{2}%
\task{%
    Тело совершает гармонические колебания с периодом $T$ и амплитудой $A$.
    Определите, какую долю периода тело находится на расстоянии более $\frac 12A$ от положения равновесия.
}
\answer{%
    $t = \frac23T \implies \frac tT = \frac23$
}
\solutionspace{80pt}

\tasknumber{3}%
\task{%
    Определите период малых колебаний маятника (см.
    рис.
    на доске),
    если $\ell = 0{,}6\,\text{м}$, а $h = 25\,\text{см}$.
}
\answer{%
    $
        T = 2 \cdot \cfrac{T_1}4 + 2 \cdot \cfrac{T_2}4 = \cfrac{T_1 + T_2}2
        = \cfrac{2\pi\sqrt{\frac {\ell}{g}} + 2\pi\sqrt{\frac {h}{g} } }2
        = \pi\cbr{\sqrt{\frac {\ell}{g}} + \sqrt{\frac {h}{g} } }
        \approx 1{,}266\,\text{c}.
    $
}
\solutionspace{80pt}

\tasknumber{4}%
\task{%
    Определите период колебаний жидкости в тонкой $U$-образной трубке постоянного сечения,
    если длина столбика жидкости равна $45\,\text{см}$ (см.
    рис.
    на доске).
    Вязкостью, силами трения и сопротивления, капиллярными эффектами и притяжением Сатурна, пренебречь.
}
\answer{%
    $
        \Delta F = -m \frac {2x}\ell g = ma
        \implies a + \frac{2g}\ell x = 0
        \implies \omega^2 = \frac{2g}\ell
        \implies T = \frac{2\pi}\omega = 2\pi\sqrt{\frac\ell{2g}}
        \approx $0{,}942\,\text{c}$.
    $
}
\solutionspace{100pt}

\tasknumber{5}%
\task{%
    2 маленьких шарика скреплены лёгкой пружиной жёсткостью $k$, масса каждого шарика $m$.
    Пружина недеформирована, система летит со скоростью $v$ к стенке,
    расстояние от ближайшего шарика до стены $\ell$ (см.
    рис.
    на доске).
    Все удары упругие, трением пренебречь, $\frac{mv^2}{kd^2} < \frac 12,$ где $d$ — длина пружины.
    Через какое время шарики вновь окажутся в том же положении?
}
\answer{%
    \begin{align*}
    F &= -k \cdot 2x = ma \implies a + \frac {2k}m x = 0 \implies \omega^2 = \frac {2k}m \implies T =\frac{2\pi}\omega = 2\pi\sqrt{\frac m{2k}}, \\
    \frac{mv^2}2 &< \frac{2k\sqr{\frac d2}}2 \implies \text{шарики между собой не ударятся}, \\
    T &= 2 \cdot \frac \ell v + \frac 12 \cdot 2 \pi\sqrt{\frac m{2k}}.
    \end{align*}
}
\solutionspace{150pt}

\tasknumber{6}%
\task{%
    Доска массой $m$ и длиной $\ell$ скользит по гладкой горизонтальной плоскости со скоростью $v$,
    а после въезжает на шероховатую поверхность с коэффициентом трения $\mu$ (см.
    рис.
    на доске).
    Определите, за какое время доска остановится (от начала въезжания) и на какое расстояние въедет.
    Ускорение свободного падения $g$, $v^2 < \mu g \ell.$
}
\answer{%
    \begin{align*}
    F_\text{трения} &= -\mu m \frac x\ell g = ma \implies a + \mu \frac g\ell x = 0 \implies \omega^2 = \frac {\mu g}\ell, \\
    T &= \frac{2\pi}{\omega} = 2\pi \sqrt{\frac \ell{\mu g}}\implies t = \frac T4 = \frac\pi2 \sqrt{\frac \ell{\mu g}}, \\
    A &= \frac v\omega = v \sqrt{\frac \ell{\mu g}} \quad (\text{отметим, что $A < \sqrt{\mu g \ell} \cdot \sqrt{\frac \ell{\mu g}} = \ell$}).
    \end{align*}
}

\variantsplitter

\addpersonalvariant{Дмитрий Соколов}

\tasknumber{1}%
\task{%
    Определите период колебаний и массу груза в пружинном маятнике,
    если максимальная скорость груза равна $4\,\frac{\text{м}}{\text{с}}$,
    жёсткость пружины $180\,\frac{\text{Н}}{\text{м}}$, а амплитуда колебаний $15\,\text{см}$.
}
\answer{%
    \begin{align*}
    \frac{mv^2}2 &= \frac{kA^2}2 \implies m = k\sqr{\frac{A}{v}} \approx 254\,\text{г}, \\
    T &= \frac{2\pi}\omega = 2\pi\sqrt{\frac mk} = 2\pi\frac{A}{v} \approx 0{,}236\,\text{c}.
    \end{align*}
}
\solutionspace{80pt}

\tasknumber{2}%
\task{%
    Тело совершает гармонические колебания с периодом $T$ и амплитудой $A$.
    Определите, какую долю периода тело находится на расстоянии более $\frac{\sqrt 3}2A$ от положения равновесия.
}
\answer{%
    $t = \frac13T \implies \frac tT = \frac13$
}
\solutionspace{80pt}

\tasknumber{3}%
\task{%
    Определите период малых колебаний маятника (см.
    рис.
    на доске),
    если $\ell = 0{,}75\,\text{м}$, а $h = 20\,\text{см}$.
}
\answer{%
    $
        T = 2 \cdot \cfrac{T_1}4 + 2 \cdot \cfrac{T_2}4 = \cfrac{T_1 + T_2}2
        = \cfrac{2\pi\sqrt{\frac {\ell}{g}} + 2\pi\sqrt{\frac {h}{g} } }2
        = \pi\cbr{\sqrt{\frac {\ell}{g}} + \sqrt{\frac {h}{g} } }
        \approx 1{,}305\,\text{c}.
    $
}
\solutionspace{80pt}

\tasknumber{4}%
\task{%
    Определите период колебаний жидкости в тонкой $U$-образной трубке постоянного сечения,
    если длина столбика жидкости равна $65\,\text{см}$ (см.
    рис.
    на доске).
    Вязкостью, силами трения и сопротивления, капиллярными эффектами и притяжением Венеры, пренебречь.
}
\answer{%
    $
        \Delta F = -m \frac {2x}\ell g = ma
        \implies a + \frac{2g}\ell x = 0
        \implies \omega^2 = \frac{2g}\ell
        \implies T = \frac{2\pi}\omega = 2\pi\sqrt{\frac\ell{2g}}
        \approx $1{,}133\,\text{c}$.
    $
}
\solutionspace{100pt}

\tasknumber{5}%
\task{%
    2 маленьких шарика скреплены лёгкой пружиной жёсткостью $k$, масса каждого шарика $m$.
    Пружина недеформирована, система летит со скоростью $v$ к стенке,
    расстояние от ближайшего шарика до стены $\ell$ (см.
    рис.
    на доске).
    Все удары упругие, трением пренебречь, $\frac{mv^2}{kd^2} < \frac 12,$ где $d$ — длина пружины.
    Через какое время шарики вновь окажутся в том же положении?
}
\answer{%
    \begin{align*}
    F &= -k \cdot 2x = ma \implies a + \frac {2k}m x = 0 \implies \omega^2 = \frac {2k}m \implies T =\frac{2\pi}\omega = 2\pi\sqrt{\frac m{2k}}, \\
    \frac{mv^2}2 &< \frac{2k\sqr{\frac d2}}2 \implies \text{шарики между собой не ударятся}, \\
    T &= 2 \cdot \frac \ell v + \frac 12 \cdot 2 \pi\sqrt{\frac m{2k}}.
    \end{align*}
}
\solutionspace{150pt}

\tasknumber{6}%
\task{%
    Доска массой $m$ и длиной $\ell$ скользит по гладкой горизонтальной плоскости со скоростью $v$,
    а после въезжает на шероховатую поверхность с коэффициентом трения $\mu$ (см.
    рис.
    на доске).
    Определите, за какое время доска остановится (от начала въезжания) и на какое расстояние въедет.
    Ускорение свободного падения $g$, $v^2 < \mu g \ell.$
}
\answer{%
    \begin{align*}
    F_\text{трения} &= -\mu m \frac x\ell g = ma \implies a + \mu \frac g\ell x = 0 \implies \omega^2 = \frac {\mu g}\ell, \\
    T &= \frac{2\pi}{\omega} = 2\pi \sqrt{\frac \ell{\mu g}}\implies t = \frac T4 = \frac\pi2 \sqrt{\frac \ell{\mu g}}, \\
    A &= \frac v\omega = v \sqrt{\frac \ell{\mu g}} \quad (\text{отметим, что $A < \sqrt{\mu g \ell} \cdot \sqrt{\frac \ell{\mu g}} = \ell$}).
    \end{align*}
}

\variantsplitter

\addpersonalvariant{Арсений Трофимов}

\tasknumber{1}%
\task{%
    Определите период колебаний и массу груза в пружинном маятнике,
    если максимальная скорость груза равна $2\,\frac{\text{м}}{\text{с}}$,
    жёсткость пружины $240\,\frac{\text{Н}}{\text{м}}$, а амплитуда колебаний $10\,\text{см}$.
}
\answer{%
    \begin{align*}
    \frac{mv^2}2 &= \frac{kA^2}2 \implies m = k\sqr{\frac{A}{v}} \approx 600\,\text{г}, \\
    T &= \frac{2\pi}\omega = 2\pi\sqrt{\frac mk} = 2\pi\frac{A}{v} \approx 0{,}314\,\text{c}.
    \end{align*}
}
\solutionspace{80pt}

\tasknumber{2}%
\task{%
    Тело совершает гармонические колебания с периодом $T$ и амплитудой $A$.
    Определите, какую долю периода тело находится на расстоянии менее $\frac 12A$ от положения равновесия.
}
\answer{%
    $t = \frac13T \implies \frac tT = \frac13$
}
\solutionspace{80pt}

\tasknumber{3}%
\task{%
    Определите период малых колебаний маятника (см.
    рис.
    на доске),
    если $\ell = 0{,}75\,\text{м}$, а $h = 20\,\text{см}$.
}
\answer{%
    $
        T = 2 \cdot \cfrac{T_1}4 + 2 \cdot \cfrac{T_2}4 = \cfrac{T_1 + T_2}2
        = \cfrac{2\pi\sqrt{\frac {\ell}{g}} + 2\pi\sqrt{\frac {h}{g} } }2
        = \pi\cbr{\sqrt{\frac {\ell}{g}} + \sqrt{\frac {h}{g} } }
        \approx 1{,}305\,\text{c}.
    $
}
\solutionspace{80pt}

\tasknumber{4}%
\task{%
    Определите период колебаний жидкости в тонкой $U$-образной трубке постоянного сечения,
    если длина столбика жидкости равна $75\,\text{см}$ (см.
    рис.
    на доске).
    Вязкостью, силами трения и сопротивления, капиллярными эффектами и притяжением Юпитера, пренебречь.
}
\answer{%
    $
        \Delta F = -m \frac {2x}\ell g = ma
        \implies a + \frac{2g}\ell x = 0
        \implies \omega^2 = \frac{2g}\ell
        \implies T = \frac{2\pi}\omega = 2\pi\sqrt{\frac\ell{2g}}
        \approx $1{,}217\,\text{c}$.
    $
}
\solutionspace{100pt}

\tasknumber{5}%
\task{%
    2 маленьких шарика скреплены лёгкой пружиной жёсткостью $k$, масса каждого шарика $m$.
    Пружина недеформирована, система летит со скоростью $v$ к стенке,
    расстояние от ближайшего шарика до стены $\ell$ (см.
    рис.
    на доске).
    Все удары упругие, трением пренебречь, $\frac{mv^2}{kd^2} < \frac 12,$ где $d$ — длина пружины.
    Через какое время шарики вновь окажутся в том же положении?
}
\answer{%
    \begin{align*}
    F &= -k \cdot 2x = ma \implies a + \frac {2k}m x = 0 \implies \omega^2 = \frac {2k}m \implies T =\frac{2\pi}\omega = 2\pi\sqrt{\frac m{2k}}, \\
    \frac{mv^2}2 &< \frac{2k\sqr{\frac d2}}2 \implies \text{шарики между собой не ударятся}, \\
    T &= 2 \cdot \frac \ell v + \frac 12 \cdot 2 \pi\sqrt{\frac m{2k}}.
    \end{align*}
}
\solutionspace{150pt}

\tasknumber{6}%
\task{%
    Доска массой $m$ и длиной $\ell$ скользит по гладкой горизонтальной плоскости со скоростью $v$,
    а после въезжает на шероховатую поверхность с коэффициентом трения $\mu$ (см.
    рис.
    на доске).
    Определите, за какое время доска остановится (от начала въезжания) и на какое расстояние въедет.
    Ускорение свободного падения $g$, $v^2 < \mu g \ell.$
}
\answer{%
    \begin{align*}
    F_\text{трения} &= -\mu m \frac x\ell g = ma \implies a + \mu \frac g\ell x = 0 \implies \omega^2 = \frac {\mu g}\ell, \\
    T &= \frac{2\pi}{\omega} = 2\pi \sqrt{\frac \ell{\mu g}}\implies t = \frac T4 = \frac\pi2 \sqrt{\frac \ell{\mu g}}, \\
    A &= \frac v\omega = v \sqrt{\frac \ell{\mu g}} \quad (\text{отметим, что $A < \sqrt{\mu g \ell} \cdot \sqrt{\frac \ell{\mu g}} = \ell$}).
    \end{align*}
}
% autogenerated
