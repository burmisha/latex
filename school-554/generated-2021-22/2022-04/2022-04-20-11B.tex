\setdate{20~апреля~2022}
\setclass{11«Б»}

\addpersonalvariant{Михаил Бурмистров}

\tasknumber{1}%
\task{%
    Определите число протонов в атоме $\ce{^{40}_{20}{Ca}}$.
}
\answer{%
    $Z = 20$ протонов и столько же электронов $A = 40$ нуклонов, $A - Z = 20$ нейтронов.
    Ответ: 20
}

\tasknumber{2}%
\task{%
    Определите число нейтронов в атоме $\text{аргон-36}$.
}
\answer{%
    $Z = 18$ протонов и столько же электронов, $A = 36$ нуклонов, $A - Z = 18$ нейтронов.
    Ответ: 18
}

\tasknumber{3}%
\task{%
    Энергия связи ядра углерода \ce{^{12}_{6}C} равна $92{,}2\,\text{МэВ}$.
    Найти дефект массы этого ядра.
    Ответ выразите в а.е.м.
    и кг.
    Скорость света $c = 2{,}998 \cdot 10^{8}\,\frac{\text{м}}{\text{с}}$, элементарный заряд $e = 1{,}6 \cdot 10^{-19}\,\text{Кл}$.
}
\answer{%
    \begin{align*}
    E_\text{св.} &= \Delta m c^2 \implies \\
    \implies
            \Delta m &= \frac {E_\text{св.}}{c^2} = \frac{92{,}2\,\text{МэВ}}{\sqr{2{,}998 \cdot 10^{8}\,\frac{\text{м}}{\text{с}}}}
            = \frac{92{,}2 \cdot 10^6 \cdot 1{,}6 \cdot 10^{-19}\,\text{Дж}}{\sqr{2{,}998 \cdot 10^{8}\,\frac{\text{м}}{\text{с}}}}
            \approx 0{,}1641 \cdot 10^{-27}\,\text{кг} \approx 0{,}0988\,\text{а.е.м.}
    \end{align*}
}
\solutionspace{100pt}

\tasknumber{4}%
\task{%
    Определите дефект массы (в а.е.м.) и энергию связи (в МэВ) ядра атома \ce{^{2}_{1}{D}},
    если его масса составляет $2{,}0141\,\text{а.е.м.}$.
    Считать $m_{p} = 1{,}00728\,\text{а.е.м.}$, $m_{n} = 1{,}00867\,\text{а.е.м.}$.
}
\answer{%
    \begin{align*}
    \Delta m &= (A - Z)m_{n} + Zm_{p} - m = 1 \cdot 1{,}00867\,\text{а.е.м.} + 1 \cdot 1{,}00728\,\text{а.е.м.} - 2{,}0141\,\text{а.е.м.} \approx 0{,}00185\,\text{а.е.м.} \\
    E_\text{св.} &= \Delta m c^2 \approx 0{,}0018 \cdot 931{,}5\,\text{МэВ} \approx 1{,}73\,\text{МэВ}
    \end{align*}
}
\solutionspace{90pt}

\tasknumber{5}%
\task{%
    В какое ядро превращается исходное в результате ядерного распада?
    Запишите уравнение реакции и явно укажите число протонов и нейтронов в получившемся ядре.
    \begin{itemize}
        \item ядро фосфора $\ce{^{30}_{15}{P}}$, $\beta^+$-распад,
        \item ядро урана $\ce{^{234}_{92}{U}}$, $\alpha$-распад,
        \item ядро свинца $\ce{^{214}_{82}{Pb}}$, $\alpha$-распад,
        \item ядро свинца $\ce{^{209}_{82}{Pb}}$, $\beta$-распад.
    \end{itemize}
}
\answer{%
    \begin{align*}
    &\ce{^{30}_{15}{P}} \to \ce{^{30}_{14}{Si}} + e^+ + \nu_e: \qquad \text{ядро кремния $\ce{^{30}_{14}{Si}}$}: 14\,p^+, 16\,n^0, \\
    &\ce{^{234}_{92}{U}} \to \ce{^{230}_{90}{Th}} + \ce{^4_2{He}}: \qquad \text{ядро тория $\ce{^{230}_{90}{Th}}$}: 90\,p^+, 140\,n^0, \\
    &\ce{^{214}_{82}{Pb}} \to \ce{^{210}_{80}{Hg}} + \ce{^4_2{He}}: \qquad \text{ядро ртути $\ce{^{210}_{80}{Hg}}$}: 80\,p^+, 130\,n^0, \\
    &\ce{^{209}_{82}{Pb}} \to \ce{^{209}_{83}{Bi}} + e^- + \tilde\nu_e: \qquad \text{ядро висмута $\ce{^{209}_{83}{Bi}}$}: 83\,p^+, 126\,n^0.
    \end{align*}
}
\solutionspace{80pt}

\tasknumber{6}%
\task{%
    Какая доля (от начального количества) радиоактивных ядер распадётся через время,
    равное двум периодам полураспада? Ответ выразить в процентах.
}
\answer{%
    \begin{align*}
    N &= N_0 \cdot 2^{- \frac t{T_{1/2}}} \implies
        \frac N{N_0} = 2^{- \frac t{T_{1/2}}}
        = 2^{-2} \approx 0{,}25 \approx 25\% \\
    N_\text{расп.} &= N_0 - N = N_0 - N_0 \cdot 2^{-\frac t{T_{1/2}}}
        = N_0\cbr{1 - 2^{-\frac t{T_{1/2}}}} \implies
        \frac{N_\text{расп.}}{N_0} = 1 - 2^{-\frac t{T_{1/2}}}
        = 1 - 2^{-2} \approx 0{,}75 \approx 75\%
    \end{align*}
}
\solutionspace{90pt}

\tasknumber{7}%
\task{%
    Сколько процентов ядер радиоактивного железа $\ce{^{59}Fe}$
    останется через $91{,}2\,\text{суток}$, если период его полураспада составляет $45{,}6\,\text{суток}$?
}
\answer{%
    \begin{align*}
    N &= N_0 \cdot 2^{-\frac t{T_{1/2}}}
        = 2^{-\frac{91{,}2\,\text{суток}}{45{,}6\,\text{суток}}}
        \approx 0{,}2500 = 25{,}00\%
    \end{align*}
}
\solutionspace{90pt}

\tasknumber{8}%
\task{%
    За $2\,\text{суток}$ от начального количества ядер радиоизотопа осталась одна шестнадцатая.
    Каков период полураспада этого изотопа (ответ приведите в сутках)?
    Какая ещё доля (также от начального количества) распадётся, если подождать ещё столько же?
}
\answer{%
    \begin{align*}
            N &= N_0 \cdot 2^{-\frac t{T_{1/2}}}
            \implies \frac N{N_0} = 2^{-\frac t{T_{1/2}}}
            \implies \frac 1{16} = 2^{-\frac {2\,\text{суток}}{T_{1/2}}}
            \implies 4 = \frac {2\,\text{суток}}{T_{1/2}}
            \implies T_{1/2} = \frac {2\,\text{суток}}4 \approx 0{,}50\,\text{суток}.
         \\
            \delta &= \frac{N(t)}{N_0} - \frac{N(2t)}{N_0}
            = 2^{-\frac t{T_{1/2}}} - 2^{-\frac {2t}{T_{1/2}}}
            = 2^{-\frac t{T_{1/2}}}\cbr{1 - 2^{-\frac t{T_{1/2}}}}
            = \frac 1{16} \cdot \cbr{1-\frac 1{16}} \approx 0{,}059
    \end{align*}
}

\variantsplitter

\addpersonalvariant{Снежана Авдошина}

\tasknumber{1}%
\task{%
    Определите число электронов в атоме $\ce{^{40}_{20}{Ca}}$.
}
\answer{%
    $Z = 20$ протонов и столько же электронов $A = 40$ нуклонов, $A - Z = 20$ нейтронов.
    Ответ: 20
}

\tasknumber{2}%
\task{%
    Определите число электронов в атоме $\text{аргон-42}$.
}
\answer{%
    $Z = 18$ протонов и столько же электронов, $A = 42$ нуклонов, $A - Z = 24$ нейтронов.
    Ответ: 18
}

\tasknumber{3}%
\task{%
    Энергия связи ядра гелия \ce{^{3}_{2}He} равна $28{,}29\,\text{МэВ}$.
    Найти дефект массы этого ядра.
    Ответ выразите в а.е.м.
    и кг.
    Скорость света $c = 2{,}998 \cdot 10^{8}\,\frac{\text{м}}{\text{с}}$, элементарный заряд $e = 1{,}6 \cdot 10^{-19}\,\text{Кл}$.
}
\answer{%
    \begin{align*}
    E_\text{св.} &= \Delta m c^2 \implies \\
    \implies
            \Delta m &= \frac {E_\text{св.}}{c^2} = \frac{28{,}29\,\text{МэВ}}{\sqr{2{,}998 \cdot 10^{8}\,\frac{\text{м}}{\text{с}}}}
            = \frac{28{,}29 \cdot 10^6 \cdot 1{,}6 \cdot 10^{-19}\,\text{Дж}}{\sqr{2{,}998 \cdot 10^{8}\,\frac{\text{м}}{\text{с}}}}
            \approx 50{,}36 \cdot 10^{-30}\,\text{кг} \approx 0{,}03033\,\text{а.е.м.}
    \end{align*}
}
\solutionspace{100pt}

\tasknumber{4}%
\task{%
    Определите дефект массы (в а.е.м.) и энергию связи (в МэВ) ядра атома \ce{^{8}_{2}{He}},
    если его масса составляет $8{,}0225\,\text{а.е.м.}$.
    Считать $m_{p} = 1{,}00728\,\text{а.е.м.}$, $m_{n} = 1{,}00867\,\text{а.е.м.}$.
}
\answer{%
    \begin{align*}
    \Delta m &= (A - Z)m_{n} + Zm_{p} - m = 6 \cdot 1{,}00867\,\text{а.е.м.} + 2 \cdot 1{,}00728\,\text{а.е.м.} - 8{,}0225\,\text{а.е.м.} \approx 0{,}0441\,\text{а.е.м.} \\
    E_\text{св.} &= \Delta m c^2 \approx 0{,}0441 \cdot 931{,}5\,\text{МэВ} \approx 41{,}2\,\text{МэВ}
    \end{align*}
}
\solutionspace{90pt}

\tasknumber{5}%
\task{%
    В какое ядро превращается исходное в результате ядерного распада?
    Запишите уравнение реакции и явно укажите число протонов и нейтронов в получившемся ядре.
    \begin{itemize}
        \item ядро фосфора $\ce{^{30}_{15}{P}}$, $\beta^+$-распад,
        \item ядро радона $\ce{^{222}_{86}{Rn}}$, $\alpha$-распад,
        \item ядро висмута $\ce{^{210}_{83}{Bi}}$, $\beta$-распад,
        \item ядро плутония $\ce{^{239}_{94}{Pu}}$, $\alpha$-распад.
    \end{itemize}
}
\answer{%
    \begin{align*}
    &\ce{^{30}_{15}{P}} \to \ce{^{30}_{14}{Si}} + e^+ + \nu_e: \qquad \text{ядро кремния $\ce{^{30}_{14}{Si}}$}: 14\,p^+, 16\,n^0, \\
    &\ce{^{222}_{86}{Rn}} \to \ce{^{218}_{84}{Po}} + \ce{^4_2{He}}: \qquad \text{ядро полония $\ce{^{218}_{84}{Po}}$}: 84\,p^+, 134\,n^0, \\
    &\ce{^{210}_{83}{Bi}} \to \ce{^{210}_{84}{Po}} + e^- + \tilde\nu_e: \qquad \text{ядро полония $\ce{^{210}_{84}{Po}}$}: 84\,p^+, 126\,n^0, \\
    &\ce{^{239}_{94}{Pu}} \to \ce{^{235}_{92}{U}} + \ce{^4_2{He}}: \qquad \text{ядро урана $\ce{^{235}_{92}{U}}$}: 92\,p^+, 143\,n^0.
    \end{align*}
}
\solutionspace{80pt}

\tasknumber{6}%
\task{%
    Какая доля (от начального количества) радиоактивных ядер останется через время,
    равное двум периодам полураспада? Ответ выразить в процентах.
}
\answer{%
    \begin{align*}
    N &= N_0 \cdot 2^{- \frac t{T_{1/2}}} \implies
        \frac N{N_0} = 2^{- \frac t{T_{1/2}}}
        = 2^{-2} \approx 0{,}25 \approx 25\% \\
    N_\text{расп.} &= N_0 - N = N_0 - N_0 \cdot 2^{-\frac t{T_{1/2}}}
        = N_0\cbr{1 - 2^{-\frac t{T_{1/2}}}} \implies
        \frac{N_\text{расп.}}{N_0} = 1 - 2^{-\frac t{T_{1/2}}}
        = 1 - 2^{-2} \approx 0{,}75 \approx 75\%
    \end{align*}
}
\solutionspace{90pt}

\tasknumber{7}%
\task{%
    Сколько процентов ядер радиоактивного железа $\ce{^{59}Fe}$
    останется через $136{,}8\,\text{суток}$, если период его полураспада составляет $45{,}6\,\text{суток}$?
}
\answer{%
    \begin{align*}
    N &= N_0 \cdot 2^{-\frac t{T_{1/2}}}
        = 2^{-\frac{136{,}8\,\text{суток}}{45{,}6\,\text{суток}}}
        \approx 0{,}1250 = 12{,}50\%
    \end{align*}
}
\solutionspace{90pt}

\tasknumber{8}%
\task{%
    За $2\,\text{суток}$ от начального количества ядер радиоизотопа осталась половина.
    Каков период полураспада этого изотопа (ответ приведите в сутках)?
    Какая ещё доля (также от начального количества) распадётся, если подождать ещё столько же?
}
\answer{%
    \begin{align*}
            N &= N_0 \cdot 2^{-\frac t{T_{1/2}}}
            \implies \frac N{N_0} = 2^{-\frac t{T_{1/2}}}
            \implies \frac 1{2} = 2^{-\frac {2\,\text{суток}}{T_{1/2}}}
            \implies 1 = \frac {2\,\text{суток}}{T_{1/2}}
            \implies T_{1/2} = \frac {2\,\text{суток}}1 \approx 2\,\text{суток}.
         \\
            \delta &= \frac{N(t)}{N_0} - \frac{N(2t)}{N_0}
            = 2^{-\frac t{T_{1/2}}} - 2^{-\frac {2t}{T_{1/2}}}
            = 2^{-\frac t{T_{1/2}}}\cbr{1 - 2^{-\frac t{T_{1/2}}}}
            = \frac 1{2} \cdot \cbr{1-\frac 1{2}} \approx 0{,}250
    \end{align*}
}

\variantsplitter

\addpersonalvariant{Марьяна Аристова}

\tasknumber{1}%
\task{%
    Определите число нейтронов в атоме $\ce{^{41}_{19}{K}}$.
}
\answer{%
    $Z = 19$ протонов и столько же электронов $A = 41$ нуклонов, $A - Z = 22$ нейтронов.
    Ответ: 22
}

\tasknumber{2}%
\task{%
    Определите число протонов в атоме $\text{аргон-40}$.
}
\answer{%
    $Z = 18$ протонов и столько же электронов, $A = 40$ нуклонов, $A - Z = 22$ нейтронов.
    Ответ: 18
}

\tasknumber{3}%
\task{%
    Энергия связи ядра лития \ce{^{7}_{3}Li} равна $39{,}2\,\text{МэВ}$.
    Найти дефект массы этого ядра.
    Ответ выразите в а.е.м.
    и кг.
    Скорость света $c = 2{,}998 \cdot 10^{8}\,\frac{\text{м}}{\text{с}}$, элементарный заряд $e = 1{,}6 \cdot 10^{-19}\,\text{Кл}$.
}
\answer{%
    \begin{align*}
    E_\text{св.} &= \Delta m c^2 \implies \\
    \implies
            \Delta m &= \frac {E_\text{св.}}{c^2} = \frac{39{,}2\,\text{МэВ}}{\sqr{2{,}998 \cdot 10^{8}\,\frac{\text{м}}{\text{с}}}}
            = \frac{39{,}2 \cdot 10^6 \cdot 1{,}6 \cdot 10^{-19}\,\text{Дж}}{\sqr{2{,}998 \cdot 10^{8}\,\frac{\text{м}}{\text{с}}}}
            \approx 69{,}8 \cdot 10^{-30}\,\text{кг} \approx 0{,}0420\,\text{а.е.м.}
    \end{align*}
}
\solutionspace{100pt}

\tasknumber{4}%
\task{%
    Определите дефект массы (в а.е.м.) и энергию связи (в МэВ) ядра атома \ce{^{4}_{2}{He}},
    если его масса составляет $4{,}0026\,\text{а.е.м.}$.
    Считать $m_{p} = 1{,}00728\,\text{а.е.м.}$, $m_{n} = 1{,}00867\,\text{а.е.м.}$.
}
\answer{%
    \begin{align*}
    \Delta m &= (A - Z)m_{n} + Zm_{p} - m = 2 \cdot 1{,}00867\,\text{а.е.м.} + 2 \cdot 1{,}00728\,\text{а.е.м.} - 4{,}0026\,\text{а.е.м.} \approx 0{,}0293\,\text{а.е.м.} \\
    E_\text{св.} &= \Delta m c^2 \approx 0{,}0293 \cdot 931{,}5\,\text{МэВ} \approx 27{,}4\,\text{МэВ}
    \end{align*}
}
\solutionspace{90pt}

\tasknumber{5}%
\task{%
    В какое ядро превращается исходное в результате ядерного распада?
    Запишите уравнение реакции и явно укажите число протонов и нейтронов в получившемся ядре.
    \begin{itemize}
        \item ядро фосфора $\ce{^{30}_{15}{P}}$, $\beta^+$-распад,
        \item ядро тория $\ce{^{230}_{90}{Th}}$, $\alpha$-распад,
        \item ядро свинца $\ce{^{210}_{82}{Pb}}$, $\beta$-распад,
        \item ядро плутония $\ce{^{239}_{94}{Pu}}$, $\alpha$-распад.
    \end{itemize}
}
\answer{%
    \begin{align*}
    &\ce{^{30}_{15}{P}} \to \ce{^{30}_{14}{Si}} + e^+ + \nu_e: \qquad \text{ядро кремния $\ce{^{30}_{14}{Si}}$}: 14\,p^+, 16\,n^0, \\
    &\ce{^{230}_{90}{Th}} \to \ce{^{226}_{88}{Ra}} + \ce{^4_2{He}}: \qquad \text{ядро радия $\ce{^{226}_{88}{Ra}}$}: 88\,p^+, 138\,n^0, \\
    &\ce{^{210}_{82}{Pb}} \to \ce{^{210}_{83}{Bi}} + e^- + \tilde\nu_e: \qquad \text{ядро висмута $\ce{^{210}_{83}{Bi}}$}: 83\,p^+, 127\,n^0, \\
    &\ce{^{239}_{94}{Pu}} \to \ce{^{235}_{92}{U}} + \ce{^4_2{He}}: \qquad \text{ядро урана $\ce{^{235}_{92}{U}}$}: 92\,p^+, 143\,n^0.
    \end{align*}
}
\solutionspace{80pt}

\tasknumber{6}%
\task{%
    Какая доля (от начального количества) радиоактивных ядер распадётся через время,
    равное трём периодам полураспада? Ответ выразить в процентах.
}
\answer{%
    \begin{align*}
    N &= N_0 \cdot 2^{- \frac t{T_{1/2}}} \implies
        \frac N{N_0} = 2^{- \frac t{T_{1/2}}}
        = 2^{-3} \approx 0{,}12 \approx 12\% \\
    N_\text{расп.} &= N_0 - N = N_0 - N_0 \cdot 2^{-\frac t{T_{1/2}}}
        = N_0\cbr{1 - 2^{-\frac t{T_{1/2}}}} \implies
        \frac{N_\text{расп.}}{N_0} = 1 - 2^{-\frac t{T_{1/2}}}
        = 1 - 2^{-3} \approx 0{,}88 \approx 88\%
    \end{align*}
}
\solutionspace{90pt}

\tasknumber{7}%
\task{%
    Сколько процентов ядер радиоактивного железа $\ce{^{59}Fe}$
    останется через $136{,}8\,\text{суток}$, если период его полураспада составляет $45{,}6\,\text{суток}$?
}
\answer{%
    \begin{align*}
    N &= N_0 \cdot 2^{-\frac t{T_{1/2}}}
        = 2^{-\frac{136{,}8\,\text{суток}}{45{,}6\,\text{суток}}}
        \approx 0{,}1250 = 12{,}50\%
    \end{align*}
}
\solutionspace{90pt}

\tasknumber{8}%
\task{%
    За $3\,\text{суток}$ от начального количества ядер радиоизотопа осталась одна шестнадцатая.
    Каков период полураспада этого изотопа (ответ приведите в сутках)?
    Какая ещё доля (также от начального количества) распадётся, если подождать ещё столько же?
}
\answer{%
    \begin{align*}
            N &= N_0 \cdot 2^{-\frac t{T_{1/2}}}
            \implies \frac N{N_0} = 2^{-\frac t{T_{1/2}}}
            \implies \frac 1{16} = 2^{-\frac {3\,\text{суток}}{T_{1/2}}}
            \implies 4 = \frac {3\,\text{суток}}{T_{1/2}}
            \implies T_{1/2} = \frac {3\,\text{суток}}4 \approx 0{,}75\,\text{суток}.
         \\
            \delta &= \frac{N(t)}{N_0} - \frac{N(2t)}{N_0}
            = 2^{-\frac t{T_{1/2}}} - 2^{-\frac {2t}{T_{1/2}}}
            = 2^{-\frac t{T_{1/2}}}\cbr{1 - 2^{-\frac t{T_{1/2}}}}
            = \frac 1{16} \cdot \cbr{1-\frac 1{16}} \approx 0{,}059
    \end{align*}
}

\variantsplitter

\addpersonalvariant{Никита Иванов}

\tasknumber{1}%
\task{%
    Определите число нейтронов в атоме $\ce{^{40}_{19}{K}}$.
}
\answer{%
    $Z = 19$ протонов и столько же электронов $A = 40$ нуклонов, $A - Z = 21$ нейтронов.
    Ответ: 21
}

\tasknumber{2}%
\task{%
    Определите число нуклонов в атоме $\text{хлор-37}$.
}
\answer{%
    $Z = 17$ протонов и столько же электронов, $A = 37$ нуклонов, $A - Z = 20$ нейтронов.
    Ответ: 37
}

\tasknumber{3}%
\task{%
    Энергия связи ядра лития \ce{^{7}_{3}Li} равна $39{,}2\,\text{МэВ}$.
    Найти дефект массы этого ядра.
    Ответ выразите в а.е.м.
    и кг.
    Скорость света $c = 2{,}998 \cdot 10^{8}\,\frac{\text{м}}{\text{с}}$, элементарный заряд $e = 1{,}6 \cdot 10^{-19}\,\text{Кл}$.
}
\answer{%
    \begin{align*}
    E_\text{св.} &= \Delta m c^2 \implies \\
    \implies
            \Delta m &= \frac {E_\text{св.}}{c^2} = \frac{39{,}2\,\text{МэВ}}{\sqr{2{,}998 \cdot 10^{8}\,\frac{\text{м}}{\text{с}}}}
            = \frac{39{,}2 \cdot 10^6 \cdot 1{,}6 \cdot 10^{-19}\,\text{Дж}}{\sqr{2{,}998 \cdot 10^{8}\,\frac{\text{м}}{\text{с}}}}
            \approx 69{,}8 \cdot 10^{-30}\,\text{кг} \approx 0{,}0420\,\text{а.е.м.}
    \end{align*}
}
\solutionspace{100pt}

\tasknumber{4}%
\task{%
    Определите дефект массы (в а.е.м.) и энергию связи (в МэВ) ядра атома \ce{^{3}_{2}{He}},
    если его масса составляет $3{,}01603\,\text{а.е.м.}$.
    Считать $m_{p} = 1{,}00728\,\text{а.е.м.}$, $m_{n} = 1{,}00867\,\text{а.е.м.}$.
}
\answer{%
    \begin{align*}
    \Delta m &= (A - Z)m_{n} + Zm_{p} - m = 1 \cdot 1{,}00867\,\text{а.е.м.} + 2 \cdot 1{,}00728\,\text{а.е.м.} - 3{,}01603\,\text{а.е.м.} \approx 0{,}00720\,\text{а.е.м.} \\
    E_\text{св.} &= \Delta m c^2 \approx 0{,}0072 \cdot 931{,}5\,\text{МэВ} \approx 6{,}73\,\text{МэВ}
    \end{align*}
}
\solutionspace{90pt}

\tasknumber{5}%
\task{%
    В какое ядро превращается исходное в результате ядерного распада?
    Запишите уравнение реакции и явно укажите число протонов и нейтронов в получившемся ядре.
    \begin{itemize}
        \item ядро протактиния $\ce{^{234}_{91}{Pa}}$, $\beta$-распад,
        \item ядро урана $\ce{^{234}_{92}{U}}$, $\alpha$-распад,
        \item ядро свинца $\ce{^{210}_{82}{Pb}}$, $\beta$-распад,
        \item ядро свинца $\ce{^{209}_{82}{Pb}}$, $\beta$-распад.
    \end{itemize}
}
\answer{%
    \begin{align*}
    &\ce{^{234}_{91}{Pa}} \to \ce{^{234}_{92}{U}} + e^- + \tilde\nu_e: \qquad \text{ядро урана $\ce{^{234}_{92}{U}}$}: 92\,p^+, 142\,n^0, \\
    &\ce{^{234}_{92}{U}} \to \ce{^{230}_{90}{Th}} + \ce{^4_2{He}}: \qquad \text{ядро тория $\ce{^{230}_{90}{Th}}$}: 90\,p^+, 140\,n^0, \\
    &\ce{^{210}_{82}{Pb}} \to \ce{^{210}_{83}{Bi}} + e^- + \tilde\nu_e: \qquad \text{ядро висмута $\ce{^{210}_{83}{Bi}}$}: 83\,p^+, 127\,n^0, \\
    &\ce{^{209}_{82}{Pb}} \to \ce{^{209}_{83}{Bi}} + e^- + \tilde\nu_e: \qquad \text{ядро висмута $\ce{^{209}_{83}{Bi}}$}: 83\,p^+, 126\,n^0.
    \end{align*}
}
\solutionspace{80pt}

\tasknumber{6}%
\task{%
    Какая доля (от начального количества) радиоактивных ядер останется через время,
    равное трём периодам полураспада? Ответ выразить в процентах.
}
\answer{%
    \begin{align*}
    N &= N_0 \cdot 2^{- \frac t{T_{1/2}}} \implies
        \frac N{N_0} = 2^{- \frac t{T_{1/2}}}
        = 2^{-3} \approx 0{,}12 \approx 12\% \\
    N_\text{расп.} &= N_0 - N = N_0 - N_0 \cdot 2^{-\frac t{T_{1/2}}}
        = N_0\cbr{1 - 2^{-\frac t{T_{1/2}}}} \implies
        \frac{N_\text{расп.}}{N_0} = 1 - 2^{-\frac t{T_{1/2}}}
        = 1 - 2^{-3} \approx 0{,}88 \approx 88\%
    \end{align*}
}
\solutionspace{90pt}

\tasknumber{7}%
\task{%
    Сколько процентов ядер радиоактивного железа $\ce{^{59}Fe}$
    останется через $91{,}2\,\text{суток}$, если период его полураспада составляет $45{,}6\,\text{суток}$?
}
\answer{%
    \begin{align*}
    N &= N_0 \cdot 2^{-\frac t{T_{1/2}}}
        = 2^{-\frac{91{,}2\,\text{суток}}{45{,}6\,\text{суток}}}
        \approx 0{,}2500 = 25{,}00\%
    \end{align*}
}
\solutionspace{90pt}

\tasknumber{8}%
\task{%
    За $5\,\text{суток}$ от начального количества ядер радиоизотопа осталась половина.
    Каков период полураспада этого изотопа (ответ приведите в сутках)?
    Какая ещё доля (также от начального количества) распадётся, если подождать ещё столько же?
}
\answer{%
    \begin{align*}
            N &= N_0 \cdot 2^{-\frac t{T_{1/2}}}
            \implies \frac N{N_0} = 2^{-\frac t{T_{1/2}}}
            \implies \frac 1{2} = 2^{-\frac {5\,\text{суток}}{T_{1/2}}}
            \implies 1 = \frac {5\,\text{суток}}{T_{1/2}}
            \implies T_{1/2} = \frac {5\,\text{суток}}1 \approx 5\,\text{суток}.
         \\
            \delta &= \frac{N(t)}{N_0} - \frac{N(2t)}{N_0}
            = 2^{-\frac t{T_{1/2}}} - 2^{-\frac {2t}{T_{1/2}}}
            = 2^{-\frac t{T_{1/2}}}\cbr{1 - 2^{-\frac t{T_{1/2}}}}
            = \frac 1{2} \cdot \cbr{1-\frac 1{2}} \approx 0{,}250
    \end{align*}
}

\variantsplitter

\addpersonalvariant{Анастасия Князева}

\tasknumber{1}%
\task{%
    Определите число протонов в атоме $\ce{^{40}_{19}{K}}$.
}
\answer{%
    $Z = 19$ протонов и столько же электронов $A = 40$ нуклонов, $A - Z = 21$ нейтронов.
    Ответ: 19
}

\tasknumber{2}%
\task{%
    Определите число электронов в атоме $\text{аргон-36}$.
}
\answer{%
    $Z = 18$ протонов и столько же электронов, $A = 36$ нуклонов, $A - Z = 18$ нейтронов.
    Ответ: 18
}

\tasknumber{3}%
\task{%
    Энергия связи ядра кислорода \ce{^{17}_{8}O} равна $131{,}8\,\text{МэВ}$.
    Найти дефект массы этого ядра.
    Ответ выразите в а.е.м.
    и кг.
    Скорость света $c = 2{,}998 \cdot 10^{8}\,\frac{\text{м}}{\text{с}}$, элементарный заряд $e = 1{,}6 \cdot 10^{-19}\,\text{Кл}$.
}
\answer{%
    \begin{align*}
    E_\text{св.} &= \Delta m c^2 \implies \\
    \implies
            \Delta m &= \frac {E_\text{св.}}{c^2} = \frac{131{,}8\,\text{МэВ}}{\sqr{2{,}998 \cdot 10^{8}\,\frac{\text{м}}{\text{с}}}}
            = \frac{131{,}8 \cdot 10^6 \cdot 1{,}6 \cdot 10^{-19}\,\text{Дж}}{\sqr{2{,}998 \cdot 10^{8}\,\frac{\text{м}}{\text{с}}}}
            \approx 0{,}235 \cdot 10^{-27}\,\text{кг} \approx 0{,}1413\,\text{а.е.м.}
    \end{align*}
}
\solutionspace{100pt}

\tasknumber{4}%
\task{%
    Определите дефект массы (в а.е.м.) и энергию связи (в МэВ) ядра атома \ce{^{8}_{2}{He}},
    если его масса составляет $8{,}0225\,\text{а.е.м.}$.
    Считать $m_{p} = 1{,}00728\,\text{а.е.м.}$, $m_{n} = 1{,}00867\,\text{а.е.м.}$.
}
\answer{%
    \begin{align*}
    \Delta m &= (A - Z)m_{n} + Zm_{p} - m = 6 \cdot 1{,}00867\,\text{а.е.м.} + 2 \cdot 1{,}00728\,\text{а.е.м.} - 8{,}0225\,\text{а.е.м.} \approx 0{,}0441\,\text{а.е.м.} \\
    E_\text{св.} &= \Delta m c^2 \approx 0{,}0441 \cdot 931{,}5\,\text{МэВ} \approx 41{,}2\,\text{МэВ}
    \end{align*}
}
\solutionspace{90pt}

\tasknumber{5}%
\task{%
    В какое ядро превращается исходное в результате ядерного распада?
    Запишите уравнение реакции и явно укажите число протонов и нейтронов в получившемся ядре.
    \begin{itemize}
        \item ядро фосфора $\ce{^{30}_{15}{P}}$, $\beta^+$-распад,
        \item ядро радия $\ce{^{226}_{88}{Ra}}$, $\alpha$-распад,
        \item ядро свинца $\ce{^{214}_{82}{Pb}}$, $\alpha$-распад,
        \item ядро свинца $\ce{^{209}_{82}{Pb}}$, $\beta$-распад.
    \end{itemize}
}
\answer{%
    \begin{align*}
    &\ce{^{30}_{15}{P}} \to \ce{^{30}_{14}{Si}} + e^+ + \nu_e: \qquad \text{ядро кремния $\ce{^{30}_{14}{Si}}$}: 14\,p^+, 16\,n^0, \\
    &\ce{^{226}_{88}{Ra}} \to \ce{^{222}_{86}{Rn}} + \ce{^4_2{He}}: \qquad \text{ядро радона $\ce{^{222}_{86}{Rn}}$}: 86\,p^+, 136\,n^0, \\
    &\ce{^{214}_{82}{Pb}} \to \ce{^{210}_{80}{Hg}} + \ce{^4_2{He}}: \qquad \text{ядро ртути $\ce{^{210}_{80}{Hg}}$}: 80\,p^+, 130\,n^0, \\
    &\ce{^{209}_{82}{Pb}} \to \ce{^{209}_{83}{Bi}} + e^- + \tilde\nu_e: \qquad \text{ядро висмута $\ce{^{209}_{83}{Bi}}$}: 83\,p^+, 126\,n^0.
    \end{align*}
}
\solutionspace{80pt}

\tasknumber{6}%
\task{%
    Какая доля (от начального количества) радиоактивных ядер останется через время,
    равное трём периодам полураспада? Ответ выразить в процентах.
}
\answer{%
    \begin{align*}
    N &= N_0 \cdot 2^{- \frac t{T_{1/2}}} \implies
        \frac N{N_0} = 2^{- \frac t{T_{1/2}}}
        = 2^{-3} \approx 0{,}12 \approx 12\% \\
    N_\text{расп.} &= N_0 - N = N_0 - N_0 \cdot 2^{-\frac t{T_{1/2}}}
        = N_0\cbr{1 - 2^{-\frac t{T_{1/2}}}} \implies
        \frac{N_\text{расп.}}{N_0} = 1 - 2^{-\frac t{T_{1/2}}}
        = 1 - 2^{-3} \approx 0{,}88 \approx 88\%
    \end{align*}
}
\solutionspace{90pt}

\tasknumber{7}%
\task{%
    Сколько процентов ядер радиоактивного железа $\ce{^{59}Fe}$
    останется через $136{,}8\,\text{суток}$, если период его полураспада составляет $45{,}6\,\text{суток}$?
}
\answer{%
    \begin{align*}
    N &= N_0 \cdot 2^{-\frac t{T_{1/2}}}
        = 2^{-\frac{136{,}8\,\text{суток}}{45{,}6\,\text{суток}}}
        \approx 0{,}1250 = 12{,}50\%
    \end{align*}
}
\solutionspace{90pt}

\tasknumber{8}%
\task{%
    За $3\,\text{суток}$ от начального количества ядер радиоизотопа осталась одна восьмая.
    Каков период полураспада этого изотопа (ответ приведите в сутках)?
    Какая ещё доля (также от начального количества) распадётся, если подождать ещё столько же?
}
\answer{%
    \begin{align*}
            N &= N_0 \cdot 2^{-\frac t{T_{1/2}}}
            \implies \frac N{N_0} = 2^{-\frac t{T_{1/2}}}
            \implies \frac 1{8} = 2^{-\frac {3\,\text{суток}}{T_{1/2}}}
            \implies 3 = \frac {3\,\text{суток}}{T_{1/2}}
            \implies T_{1/2} = \frac {3\,\text{суток}}3 \approx 1\,\text{суток}.
         \\
            \delta &= \frac{N(t)}{N_0} - \frac{N(2t)}{N_0}
            = 2^{-\frac t{T_{1/2}}} - 2^{-\frac {2t}{T_{1/2}}}
            = 2^{-\frac t{T_{1/2}}}\cbr{1 - 2^{-\frac t{T_{1/2}}}}
            = \frac 1{8} \cdot \cbr{1-\frac 1{8}} \approx 0{,}109
    \end{align*}
}

\variantsplitter

\addpersonalvariant{Елизавета Кутумова}

\tasknumber{1}%
\task{%
    Определите число нейтронов в атоме $\ce{^{47}_{20}{Ca}}$.
}
\answer{%
    $Z = 20$ протонов и столько же электронов $A = 47$ нуклонов, $A - Z = 27$ нейтронов.
    Ответ: 27
}

\tasknumber{2}%
\task{%
    Определите число электронов в атоме $\text{аргон-36}$.
}
\answer{%
    $Z = 18$ протонов и столько же электронов, $A = 36$ нуклонов, $A - Z = 18$ нейтронов.
    Ответ: 18
}

\tasknumber{3}%
\task{%
    Энергия связи ядра бора \ce{^{11}_{5}B} равна $76{,}2\,\text{МэВ}$.
    Найти дефект массы этого ядра.
    Ответ выразите в а.е.м.
    и кг.
    Скорость света $c = 2{,}998 \cdot 10^{8}\,\frac{\text{м}}{\text{с}}$, элементарный заряд $e = 1{,}6 \cdot 10^{-19}\,\text{Кл}$.
}
\answer{%
    \begin{align*}
    E_\text{св.} &= \Delta m c^2 \implies \\
    \implies
            \Delta m &= \frac {E_\text{св.}}{c^2} = \frac{76{,}2\,\text{МэВ}}{\sqr{2{,}998 \cdot 10^{8}\,\frac{\text{м}}{\text{с}}}}
            = \frac{76{,}2 \cdot 10^6 \cdot 1{,}6 \cdot 10^{-19}\,\text{Дж}}{\sqr{2{,}998 \cdot 10^{8}\,\frac{\text{м}}{\text{с}}}}
            \approx 0{,}1356 \cdot 10^{-27}\,\text{кг} \approx 0{,}0817\,\text{а.е.м.}
    \end{align*}
}
\solutionspace{100pt}

\tasknumber{4}%
\task{%
    Определите дефект массы (в а.е.м.) и энергию связи (в МэВ) ядра атома \ce{^{3}_{2}{He}},
    если его масса составляет $3{,}01603\,\text{а.е.м.}$.
    Считать $m_{p} = 1{,}00728\,\text{а.е.м.}$, $m_{n} = 1{,}00867\,\text{а.е.м.}$.
}
\answer{%
    \begin{align*}
    \Delta m &= (A - Z)m_{n} + Zm_{p} - m = 1 \cdot 1{,}00867\,\text{а.е.м.} + 2 \cdot 1{,}00728\,\text{а.е.м.} - 3{,}01603\,\text{а.е.м.} \approx 0{,}00720\,\text{а.е.м.} \\
    E_\text{св.} &= \Delta m c^2 \approx 0{,}0072 \cdot 931{,}5\,\text{МэВ} \approx 6{,}73\,\text{МэВ}
    \end{align*}
}
\solutionspace{90pt}

\tasknumber{5}%
\task{%
    В какое ядро превращается исходное в результате ядерного распада?
    Запишите уравнение реакции и явно укажите число протонов и нейтронов в получившемся ядре.
    \begin{itemize}
        \item ядро фосфора $\ce{^{30}_{15}{P}}$, $\beta^+$-распад,
        \item ядро радия $\ce{^{226}_{88}{Ra}}$, $\alpha$-распад,
        \item ядро свинца $\ce{^{210}_{82}{Pb}}$, $\beta$-распад,
        \item ядро плутония $\ce{^{239}_{94}{Pu}}$, $\alpha$-распад.
    \end{itemize}
}
\answer{%
    \begin{align*}
    &\ce{^{30}_{15}{P}} \to \ce{^{30}_{14}{Si}} + e^+ + \nu_e: \qquad \text{ядро кремния $\ce{^{30}_{14}{Si}}$}: 14\,p^+, 16\,n^0, \\
    &\ce{^{226}_{88}{Ra}} \to \ce{^{222}_{86}{Rn}} + \ce{^4_2{He}}: \qquad \text{ядро радона $\ce{^{222}_{86}{Rn}}$}: 86\,p^+, 136\,n^0, \\
    &\ce{^{210}_{82}{Pb}} \to \ce{^{210}_{83}{Bi}} + e^- + \tilde\nu_e: \qquad \text{ядро висмута $\ce{^{210}_{83}{Bi}}$}: 83\,p^+, 127\,n^0, \\
    &\ce{^{239}_{94}{Pu}} \to \ce{^{235}_{92}{U}} + \ce{^4_2{He}}: \qquad \text{ядро урана $\ce{^{235}_{92}{U}}$}: 92\,p^+, 143\,n^0.
    \end{align*}
}
\solutionspace{80pt}

\tasknumber{6}%
\task{%
    Какая доля (от начального количества) радиоактивных ядер распадётся через время,
    равное четырём периодам полураспада? Ответ выразить в процентах.
}
\answer{%
    \begin{align*}
    N &= N_0 \cdot 2^{- \frac t{T_{1/2}}} \implies
        \frac N{N_0} = 2^{- \frac t{T_{1/2}}}
        = 2^{-4} \approx 0{,}06 \approx 6\% \\
    N_\text{расп.} &= N_0 - N = N_0 - N_0 \cdot 2^{-\frac t{T_{1/2}}}
        = N_0\cbr{1 - 2^{-\frac t{T_{1/2}}}} \implies
        \frac{N_\text{расп.}}{N_0} = 1 - 2^{-\frac t{T_{1/2}}}
        = 1 - 2^{-4} \approx 0{,}94 \approx 94\%
    \end{align*}
}
\solutionspace{90pt}

\tasknumber{7}%
\task{%
    Сколько процентов ядер радиоактивного железа $\ce{^{59}Fe}$
    останется через $91{,}2\,\text{суток}$, если период его полураспада составляет $45{,}6\,\text{суток}$?
}
\answer{%
    \begin{align*}
    N &= N_0 \cdot 2^{-\frac t{T_{1/2}}}
        = 2^{-\frac{91{,}2\,\text{суток}}{45{,}6\,\text{суток}}}
        \approx 0{,}2500 = 25{,}00\%
    \end{align*}
}
\solutionspace{90pt}

\tasknumber{8}%
\task{%
    За $3\,\text{суток}$ от начального количества ядер радиоизотопа осталась одна шестнадцатая.
    Каков период полураспада этого изотопа (ответ приведите в сутках)?
    Какая ещё доля (также от начального количества) распадётся, если подождать ещё столько же?
}
\answer{%
    \begin{align*}
            N &= N_0 \cdot 2^{-\frac t{T_{1/2}}}
            \implies \frac N{N_0} = 2^{-\frac t{T_{1/2}}}
            \implies \frac 1{16} = 2^{-\frac {3\,\text{суток}}{T_{1/2}}}
            \implies 4 = \frac {3\,\text{суток}}{T_{1/2}}
            \implies T_{1/2} = \frac {3\,\text{суток}}4 \approx 0{,}75\,\text{суток}.
         \\
            \delta &= \frac{N(t)}{N_0} - \frac{N(2t)}{N_0}
            = 2^{-\frac t{T_{1/2}}} - 2^{-\frac {2t}{T_{1/2}}}
            = 2^{-\frac t{T_{1/2}}}\cbr{1 - 2^{-\frac t{T_{1/2}}}}
            = \frac 1{16} \cdot \cbr{1-\frac 1{16}} \approx 0{,}059
    \end{align*}
}

\variantsplitter

\addpersonalvariant{Роксана Мехтиева}

\tasknumber{1}%
\task{%
    Определите число протонов в атоме $\ce{^{44}_{20}{Ca}}$.
}
\answer{%
    $Z = 20$ протонов и столько же электронов $A = 44$ нуклонов, $A - Z = 24$ нейтронов.
    Ответ: 20
}

\tasknumber{2}%
\task{%
    Определите число протонов в атоме $\text{аргон-39}$.
}
\answer{%
    $Z = 18$ протонов и столько же электронов, $A = 39$ нуклонов, $A - Z = 21$ нейтронов.
    Ответ: 18
}

\tasknumber{3}%
\task{%
    Энергия связи ядра бора \ce{^{11}_{5}B} равна $76{,}2\,\text{МэВ}$.
    Найти дефект массы этого ядра.
    Ответ выразите в а.е.м.
    и кг.
    Скорость света $c = 2{,}998 \cdot 10^{8}\,\frac{\text{м}}{\text{с}}$, элементарный заряд $e = 1{,}6 \cdot 10^{-19}\,\text{Кл}$.
}
\answer{%
    \begin{align*}
    E_\text{св.} &= \Delta m c^2 \implies \\
    \implies
            \Delta m &= \frac {E_\text{св.}}{c^2} = \frac{76{,}2\,\text{МэВ}}{\sqr{2{,}998 \cdot 10^{8}\,\frac{\text{м}}{\text{с}}}}
            = \frac{76{,}2 \cdot 10^6 \cdot 1{,}6 \cdot 10^{-19}\,\text{Дж}}{\sqr{2{,}998 \cdot 10^{8}\,\frac{\text{м}}{\text{с}}}}
            \approx 0{,}1356 \cdot 10^{-27}\,\text{кг} \approx 0{,}0817\,\text{а.е.м.}
    \end{align*}
}
\solutionspace{100pt}

\tasknumber{4}%
\task{%
    Определите дефект массы (в а.е.м.) и энергию связи (в МэВ) ядра атома \ce{^{6}_{2}{He}},
    если его масса составляет $6{,}0189\,\text{а.е.м.}$.
    Считать $m_{p} = 1{,}00728\,\text{а.е.м.}$, $m_{n} = 1{,}00867\,\text{а.е.м.}$.
}
\answer{%
    \begin{align*}
    \Delta m &= (A - Z)m_{n} + Zm_{p} - m = 4 \cdot 1{,}00867\,\text{а.е.м.} + 2 \cdot 1{,}00728\,\text{а.е.м.} - 6{,}0189\,\text{а.е.м.} \approx 0{,}0303\,\text{а.е.м.} \\
    E_\text{св.} &= \Delta m c^2 \approx 0{,}0303 \cdot 931{,}5\,\text{МэВ} \approx 28{,}3\,\text{МэВ}
    \end{align*}
}
\solutionspace{90pt}

\tasknumber{5}%
\task{%
    В какое ядро превращается исходное в результате ядерного распада?
    Запишите уравнение реакции и явно укажите число протонов и нейтронов в получившемся ядре.
    \begin{itemize}
        \item ядро фосфора $\ce{^{30}_{15}{P}}$, $\beta^+$-распад,
        \item ядро тория $\ce{^{230}_{90}{Th}}$, $\alpha$-распад,
        \item ядро полония $\ce{^{210}_{84}{Po}}$, $\beta$-распад,
        \item ядро свинца $\ce{^{209}_{82}{Pb}}$, $\beta$-распад.
    \end{itemize}
}
\answer{%
    \begin{align*}
    &\ce{^{30}_{15}{P}} \to \ce{^{30}_{14}{Si}} + e^+ + \nu_e: \qquad \text{ядро кремния $\ce{^{30}_{14}{Si}}$}: 14\,p^+, 16\,n^0, \\
    &\ce{^{230}_{90}{Th}} \to \ce{^{226}_{88}{Ra}} + \ce{^4_2{He}}: \qquad \text{ядро радия $\ce{^{226}_{88}{Ra}}$}: 88\,p^+, 138\,n^0, \\
    &\ce{^{210}_{84}{Po}} \to \ce{^{210}_{85}{At}} + e^- + \tilde\nu_e: \qquad \text{ядро астата $\ce{^{210}_{85}{At}}$}: 85\,p^+, 125\,n^0, \\
    &\ce{^{209}_{82}{Pb}} \to \ce{^{209}_{83}{Bi}} + e^- + \tilde\nu_e: \qquad \text{ядро висмута $\ce{^{209}_{83}{Bi}}$}: 83\,p^+, 126\,n^0.
    \end{align*}
}
\solutionspace{80pt}

\tasknumber{6}%
\task{%
    Какая доля (от начального количества) радиоактивных ядер распадётся через время,
    равное четырём периодам полураспада? Ответ выразить в процентах.
}
\answer{%
    \begin{align*}
    N &= N_0 \cdot 2^{- \frac t{T_{1/2}}} \implies
        \frac N{N_0} = 2^{- \frac t{T_{1/2}}}
        = 2^{-4} \approx 0{,}06 \approx 6\% \\
    N_\text{расп.} &= N_0 - N = N_0 - N_0 \cdot 2^{-\frac t{T_{1/2}}}
        = N_0\cbr{1 - 2^{-\frac t{T_{1/2}}}} \implies
        \frac{N_\text{расп.}}{N_0} = 1 - 2^{-\frac t{T_{1/2}}}
        = 1 - 2^{-4} \approx 0{,}94 \approx 94\%
    \end{align*}
}
\solutionspace{90pt}

\tasknumber{7}%
\task{%
    Сколько процентов ядер радиоактивного железа $\ce{^{59}Fe}$
    останется через $91{,}2\,\text{суток}$, если период его полураспада составляет $45{,}6\,\text{суток}$?
}
\answer{%
    \begin{align*}
    N &= N_0 \cdot 2^{-\frac t{T_{1/2}}}
        = 2^{-\frac{91{,}2\,\text{суток}}{45{,}6\,\text{суток}}}
        \approx 0{,}2500 = 25{,}00\%
    \end{align*}
}
\solutionspace{90pt}

\tasknumber{8}%
\task{%
    За $4\,\text{суток}$ от начального количества ядер радиоизотопа осталась четверть.
    Каков период полураспада этого изотопа (ответ приведите в сутках)?
    Какая ещё доля (также от начального количества) распадётся, если подождать ещё столько же?
}
\answer{%
    \begin{align*}
            N &= N_0 \cdot 2^{-\frac t{T_{1/2}}}
            \implies \frac N{N_0} = 2^{-\frac t{T_{1/2}}}
            \implies \frac 1{4} = 2^{-\frac {4\,\text{суток}}{T_{1/2}}}
            \implies 2 = \frac {4\,\text{суток}}{T_{1/2}}
            \implies T_{1/2} = \frac {4\,\text{суток}}2 \approx 2\,\text{суток}.
         \\
            \delta &= \frac{N(t)}{N_0} - \frac{N(2t)}{N_0}
            = 2^{-\frac t{T_{1/2}}} - 2^{-\frac {2t}{T_{1/2}}}
            = 2^{-\frac t{T_{1/2}}}\cbr{1 - 2^{-\frac t{T_{1/2}}}}
            = \frac 1{4} \cdot \cbr{1-\frac 1{4}} \approx 0{,}188
    \end{align*}
}

\variantsplitter

\addpersonalvariant{Дилноза Нодиршоева}

\tasknumber{1}%
\task{%
    Определите число нуклонов в атоме $\ce{^{42}_{20}{Ca}}$.
}
\answer{%
    $Z = 20$ протонов и столько же электронов $A = 42$ нуклонов, $A - Z = 22$ нейтронов.
    Ответ: 42
}

\tasknumber{2}%
\task{%
    Определите число протонов в атоме $\text{хлор-37}$.
}
\answer{%
    $Z = 17$ протонов и столько же электронов, $A = 37$ нуклонов, $A - Z = 20$ нейтронов.
    Ответ: 17
}

\tasknumber{3}%
\task{%
    Энергия связи ядра гелия \ce{^{3}_{2}He} равна $28{,}29\,\text{МэВ}$.
    Найти дефект массы этого ядра.
    Ответ выразите в а.е.м.
    и кг.
    Скорость света $c = 2{,}998 \cdot 10^{8}\,\frac{\text{м}}{\text{с}}$, элементарный заряд $e = 1{,}6 \cdot 10^{-19}\,\text{Кл}$.
}
\answer{%
    \begin{align*}
    E_\text{св.} &= \Delta m c^2 \implies \\
    \implies
            \Delta m &= \frac {E_\text{св.}}{c^2} = \frac{28{,}29\,\text{МэВ}}{\sqr{2{,}998 \cdot 10^{8}\,\frac{\text{м}}{\text{с}}}}
            = \frac{28{,}29 \cdot 10^6 \cdot 1{,}6 \cdot 10^{-19}\,\text{Дж}}{\sqr{2{,}998 \cdot 10^{8}\,\frac{\text{м}}{\text{с}}}}
            \approx 50{,}36 \cdot 10^{-30}\,\text{кг} \approx 0{,}03033\,\text{а.е.м.}
    \end{align*}
}
\solutionspace{100pt}

\tasknumber{4}%
\task{%
    Определите дефект массы (в а.е.м.) и энергию связи (в МэВ) ядра атома \ce{^{8}_{2}{He}},
    если его масса составляет $8{,}0225\,\text{а.е.м.}$.
    Считать $m_{p} = 1{,}00728\,\text{а.е.м.}$, $m_{n} = 1{,}00867\,\text{а.е.м.}$.
}
\answer{%
    \begin{align*}
    \Delta m &= (A - Z)m_{n} + Zm_{p} - m = 6 \cdot 1{,}00867\,\text{а.е.м.} + 2 \cdot 1{,}00728\,\text{а.е.м.} - 8{,}0225\,\text{а.е.м.} \approx 0{,}0441\,\text{а.е.м.} \\
    E_\text{св.} &= \Delta m c^2 \approx 0{,}0441 \cdot 931{,}5\,\text{МэВ} \approx 41{,}2\,\text{МэВ}
    \end{align*}
}
\solutionspace{90pt}

\tasknumber{5}%
\task{%
    В какое ядро превращается исходное в результате ядерного распада?
    Запишите уравнение реакции и явно укажите число протонов и нейтронов в получившемся ядре.
    \begin{itemize}
        \item ядро фосфора $\ce{^{30}_{15}{P}}$, $\beta^+$-распад,
        \item ядро тория $\ce{^{230}_{90}{Th}}$, $\alpha$-распад,
        \item ядро висмута $\ce{^{214}_{83}{Bi}}$, $\beta$-распад,
        \item ядро плутония $\ce{^{239}_{94}{Pu}}$, $\alpha$-распад.
    \end{itemize}
}
\answer{%
    \begin{align*}
    &\ce{^{30}_{15}{P}} \to \ce{^{30}_{14}{Si}} + e^+ + \nu_e: \qquad \text{ядро кремния $\ce{^{30}_{14}{Si}}$}: 14\,p^+, 16\,n^0, \\
    &\ce{^{230}_{90}{Th}} \to \ce{^{226}_{88}{Ra}} + \ce{^4_2{He}}: \qquad \text{ядро радия $\ce{^{226}_{88}{Ra}}$}: 88\,p^+, 138\,n^0, \\
    &\ce{^{214}_{83}{Bi}} \to \ce{^{214}_{84}{Po}} + e^- + \tilde\nu_e: \qquad \text{ядро полония $\ce{^{214}_{84}{Po}}$}: 84\,p^+, 130\,n^0, \\
    &\ce{^{239}_{94}{Pu}} \to \ce{^{235}_{92}{U}} + \ce{^4_2{He}}: \qquad \text{ядро урана $\ce{^{235}_{92}{U}}$}: 92\,p^+, 143\,n^0.
    \end{align*}
}
\solutionspace{80pt}

\tasknumber{6}%
\task{%
    Какая доля (от начального количества) радиоактивных ядер останется через время,
    равное четырём периодам полураспада? Ответ выразить в процентах.
}
\answer{%
    \begin{align*}
    N &= N_0 \cdot 2^{- \frac t{T_{1/2}}} \implies
        \frac N{N_0} = 2^{- \frac t{T_{1/2}}}
        = 2^{-4} \approx 0{,}06 \approx 6\% \\
    N_\text{расп.} &= N_0 - N = N_0 - N_0 \cdot 2^{-\frac t{T_{1/2}}}
        = N_0\cbr{1 - 2^{-\frac t{T_{1/2}}}} \implies
        \frac{N_\text{расп.}}{N_0} = 1 - 2^{-\frac t{T_{1/2}}}
        = 1 - 2^{-4} \approx 0{,}94 \approx 94\%
    \end{align*}
}
\solutionspace{90pt}

\tasknumber{7}%
\task{%
    Сколько процентов ядер радиоактивного железа $\ce{^{59}Fe}$
    останется через $91{,}2\,\text{суток}$, если период его полураспада составляет $45{,}6\,\text{суток}$?
}
\answer{%
    \begin{align*}
    N &= N_0 \cdot 2^{-\frac t{T_{1/2}}}
        = 2^{-\frac{91{,}2\,\text{суток}}{45{,}6\,\text{суток}}}
        \approx 0{,}2500 = 25{,}00\%
    \end{align*}
}
\solutionspace{90pt}

\tasknumber{8}%
\task{%
    За $5\,\text{суток}$ от начального количества ядер радиоизотопа осталась одна восьмая.
    Каков период полураспада этого изотопа (ответ приведите в сутках)?
    Какая ещё доля (также от начального количества) распадётся, если подождать ещё столько же?
}
\answer{%
    \begin{align*}
            N &= N_0 \cdot 2^{-\frac t{T_{1/2}}}
            \implies \frac N{N_0} = 2^{-\frac t{T_{1/2}}}
            \implies \frac 1{8} = 2^{-\frac {5\,\text{суток}}{T_{1/2}}}
            \implies 3 = \frac {5\,\text{суток}}{T_{1/2}}
            \implies T_{1/2} = \frac {5\,\text{суток}}3 \approx 1{,}67\,\text{суток}.
         \\
            \delta &= \frac{N(t)}{N_0} - \frac{N(2t)}{N_0}
            = 2^{-\frac t{T_{1/2}}} - 2^{-\frac {2t}{T_{1/2}}}
            = 2^{-\frac t{T_{1/2}}}\cbr{1 - 2^{-\frac t{T_{1/2}}}}
            = \frac 1{8} \cdot \cbr{1-\frac 1{8}} \approx 0{,}109
    \end{align*}
}

\variantsplitter

\addpersonalvariant{Жаклин Пантелеева}

\tasknumber{1}%
\task{%
    Определите число протонов в атоме $\ce{^{44}_{20}{Ca}}$.
}
\answer{%
    $Z = 20$ протонов и столько же электронов $A = 44$ нуклонов, $A - Z = 24$ нейтронов.
    Ответ: 20
}

\tasknumber{2}%
\task{%
    Определите число протонов в атоме $\text{хлор-37}$.
}
\answer{%
    $Z = 17$ протонов и столько же электронов, $A = 37$ нуклонов, $A - Z = 20$ нейтронов.
    Ответ: 17
}

\tasknumber{3}%
\task{%
    Энергия связи ядра трития \ce{^{3}_{1}H} (T) равна $8{,}48\,\text{МэВ}$.
    Найти дефект массы этого ядра.
    Ответ выразите в а.е.м.
    и кг.
    Скорость света $c = 2{,}998 \cdot 10^{8}\,\frac{\text{м}}{\text{с}}$, элементарный заряд $e = 1{,}6 \cdot 10^{-19}\,\text{Кл}$.
}
\answer{%
    \begin{align*}
    E_\text{св.} &= \Delta m c^2 \implies \\
    \implies
            \Delta m &= \frac {E_\text{св.}}{c^2} = \frac{8{,}48\,\text{МэВ}}{\sqr{2{,}998 \cdot 10^{8}\,\frac{\text{м}}{\text{с}}}}
            = \frac{8{,}48 \cdot 10^6 \cdot 1{,}6 \cdot 10^{-19}\,\text{Дж}}{\sqr{2{,}998 \cdot 10^{8}\,\frac{\text{м}}{\text{с}}}}
            \approx 15{,}10 \cdot 10^{-30}\,\text{кг} \approx 0{,}00909\,\text{а.е.м.}
    \end{align*}
}
\solutionspace{100pt}

\tasknumber{4}%
\task{%
    Определите дефект массы (в а.е.м.) и энергию связи (в МэВ) ядра атома \ce{^{3}_{1}{T}},
    если его масса составляет $3{,}01605\,\text{а.е.м.}$.
    Считать $m_{p} = 1{,}00728\,\text{а.е.м.}$, $m_{n} = 1{,}00867\,\text{а.е.м.}$.
}
\answer{%
    \begin{align*}
    \Delta m &= (A - Z)m_{n} + Zm_{p} - m = 2 \cdot 1{,}00867\,\text{а.е.м.} + 1 \cdot 1{,}00728\,\text{а.е.м.} - 3{,}01605\,\text{а.е.м.} \approx 0{,}00857\,\text{а.е.м.} \\
    E_\text{св.} &= \Delta m c^2 \approx 0{,}0086 \cdot 931{,}5\,\text{МэВ} \approx 8{,}00\,\text{МэВ}
    \end{align*}
}
\solutionspace{90pt}

\tasknumber{5}%
\task{%
    В какое ядро превращается исходное в результате ядерного распада?
    Запишите уравнение реакции и явно укажите число протонов и нейтронов в получившемся ядре.
    \begin{itemize}
        \item ядро протактиния $\ce{^{234}_{91}{Pa}}$, $\beta$-распад,
        \item ядро тория $\ce{^{230}_{90}{Th}}$, $\alpha$-распад,
        \item ядро полония $\ce{^{218}_{84}{Po}}$, $\alpha$-распад,
        \item ядро плутония $\ce{^{239}_{94}{Pu}}$, $\alpha$-распад.
    \end{itemize}
}
\answer{%
    \begin{align*}
    &\ce{^{234}_{91}{Pa}} \to \ce{^{234}_{92}{U}} + e^- + \tilde\nu_e: \qquad \text{ядро урана $\ce{^{234}_{92}{U}}$}: 92\,p^+, 142\,n^0, \\
    &\ce{^{230}_{90}{Th}} \to \ce{^{226}_{88}{Ra}} + \ce{^4_2{He}}: \qquad \text{ядро радия $\ce{^{226}_{88}{Ra}}$}: 88\,p^+, 138\,n^0, \\
    &\ce{^{218}_{84}{Po}} \to \ce{^{214}_{82}{Pb}} + \ce{^4_2{He}}: \qquad \text{ядро свинца $\ce{^{214}_{82}{Pb}}$}: 82\,p^+, 132\,n^0, \\
    &\ce{^{239}_{94}{Pu}} \to \ce{^{235}_{92}{U}} + \ce{^4_2{He}}: \qquad \text{ядро урана $\ce{^{235}_{92}{U}}$}: 92\,p^+, 143\,n^0.
    \end{align*}
}
\solutionspace{80pt}

\tasknumber{6}%
\task{%
    Какая доля (от начального количества) радиоактивных ядер останется через время,
    равное трём периодам полураспада? Ответ выразить в процентах.
}
\answer{%
    \begin{align*}
    N &= N_0 \cdot 2^{- \frac t{T_{1/2}}} \implies
        \frac N{N_0} = 2^{- \frac t{T_{1/2}}}
        = 2^{-3} \approx 0{,}12 \approx 12\% \\
    N_\text{расп.} &= N_0 - N = N_0 - N_0 \cdot 2^{-\frac t{T_{1/2}}}
        = N_0\cbr{1 - 2^{-\frac t{T_{1/2}}}} \implies
        \frac{N_\text{расп.}}{N_0} = 1 - 2^{-\frac t{T_{1/2}}}
        = 1 - 2^{-3} \approx 0{,}88 \approx 88\%
    \end{align*}
}
\solutionspace{90pt}

\tasknumber{7}%
\task{%
    Сколько процентов ядер радиоактивного железа $\ce{^{59}Fe}$
    останется через $91{,}2\,\text{суток}$, если период его полураспада составляет $45{,}6\,\text{суток}$?
}
\answer{%
    \begin{align*}
    N &= N_0 \cdot 2^{-\frac t{T_{1/2}}}
        = 2^{-\frac{91{,}2\,\text{суток}}{45{,}6\,\text{суток}}}
        \approx 0{,}2500 = 25{,}00\%
    \end{align*}
}
\solutionspace{90pt}

\tasknumber{8}%
\task{%
    За $3\,\text{суток}$ от начального количества ядер радиоизотопа осталась четверть.
    Каков период полураспада этого изотопа (ответ приведите в сутках)?
    Какая ещё доля (также от начального количества) распадётся, если подождать ещё столько же?
}
\answer{%
    \begin{align*}
            N &= N_0 \cdot 2^{-\frac t{T_{1/2}}}
            \implies \frac N{N_0} = 2^{-\frac t{T_{1/2}}}
            \implies \frac 1{4} = 2^{-\frac {3\,\text{суток}}{T_{1/2}}}
            \implies 2 = \frac {3\,\text{суток}}{T_{1/2}}
            \implies T_{1/2} = \frac {3\,\text{суток}}2 \approx 1{,}50\,\text{суток}.
         \\
            \delta &= \frac{N(t)}{N_0} - \frac{N(2t)}{N_0}
            = 2^{-\frac t{T_{1/2}}} - 2^{-\frac {2t}{T_{1/2}}}
            = 2^{-\frac t{T_{1/2}}}\cbr{1 - 2^{-\frac t{T_{1/2}}}}
            = \frac 1{4} \cdot \cbr{1-\frac 1{4}} \approx 0{,}188
    \end{align*}
}

\variantsplitter

\addpersonalvariant{Артём Переверзев}

\tasknumber{1}%
\task{%
    Определите число нуклонов в атоме $\ce{^{43}_{20}{Ca}}$.
}
\answer{%
    $Z = 20$ протонов и столько же электронов $A = 43$ нуклонов, $A - Z = 23$ нейтронов.
    Ответ: 43
}

\tasknumber{2}%
\task{%
    Определите число электронов в атоме $\text{хлор-35}$.
}
\answer{%
    $Z = 17$ протонов и столько же электронов, $A = 35$ нуклонов, $A - Z = 18$ нейтронов.
    Ответ: 17
}

\tasknumber{3}%
\task{%
    Энергия связи ядра углерода \ce{^{13}_{6}C} равна $97{,}1\,\text{МэВ}$.
    Найти дефект массы этого ядра.
    Ответ выразите в а.е.м.
    и кг.
    Скорость света $c = 2{,}998 \cdot 10^{8}\,\frac{\text{м}}{\text{с}}$, элементарный заряд $e = 1{,}6 \cdot 10^{-19}\,\text{Кл}$.
}
\answer{%
    \begin{align*}
    E_\text{св.} &= \Delta m c^2 \implies \\
    \implies
            \Delta m &= \frac {E_\text{св.}}{c^2} = \frac{97{,}1\,\text{МэВ}}{\sqr{2{,}998 \cdot 10^{8}\,\frac{\text{м}}{\text{с}}}}
            = \frac{97{,}1 \cdot 10^6 \cdot 1{,}6 \cdot 10^{-19}\,\text{Дж}}{\sqr{2{,}998 \cdot 10^{8}\,\frac{\text{м}}{\text{с}}}}
            \approx 0{,}1729 \cdot 10^{-27}\,\text{кг} \approx 0{,}1041\,\text{а.е.м.}
    \end{align*}
}
\solutionspace{100pt}

\tasknumber{4}%
\task{%
    Определите дефект массы (в а.е.м.) и энергию связи (в МэВ) ядра атома \ce{^{4}_{2}{He}},
    если его масса составляет $4{,}0026\,\text{а.е.м.}$.
    Считать $m_{p} = 1{,}00728\,\text{а.е.м.}$, $m_{n} = 1{,}00867\,\text{а.е.м.}$.
}
\answer{%
    \begin{align*}
    \Delta m &= (A - Z)m_{n} + Zm_{p} - m = 2 \cdot 1{,}00867\,\text{а.е.м.} + 2 \cdot 1{,}00728\,\text{а.е.м.} - 4{,}0026\,\text{а.е.м.} \approx 0{,}0293\,\text{а.е.м.} \\
    E_\text{св.} &= \Delta m c^2 \approx 0{,}0293 \cdot 931{,}5\,\text{МэВ} \approx 27{,}4\,\text{МэВ}
    \end{align*}
}
\solutionspace{90pt}

\tasknumber{5}%
\task{%
    В какое ядро превращается исходное в результате ядерного распада?
    Запишите уравнение реакции и явно укажите число протонов и нейтронов в получившемся ядре.
    \begin{itemize}
        \item ядро тория $\ce{^{234}_{90}{Th}}$, $\beta$-распад,
        \item ядро радия $\ce{^{226}_{88}{Ra}}$, $\alpha$-распад,
        \item ядро висмута $\ce{^{214}_{83}{Bi}}$, $\beta$-распад,
        \item ядро плутония $\ce{^{239}_{94}{Pu}}$, $\alpha$-распад.
    \end{itemize}
}
\answer{%
    \begin{align*}
    &\ce{^{234}_{90}{Th}} \to \ce{^{234}_{91}{Pa}} + e^- + \tilde\nu_e: \qquad \text{ядро протактиния $\ce{^{234}_{91}{Pa}}$}: 91\,p^+, 143\,n^0, \\
    &\ce{^{226}_{88}{Ra}} \to \ce{^{222}_{86}{Rn}} + \ce{^4_2{He}}: \qquad \text{ядро радона $\ce{^{222}_{86}{Rn}}$}: 86\,p^+, 136\,n^0, \\
    &\ce{^{214}_{83}{Bi}} \to \ce{^{214}_{84}{Po}} + e^- + \tilde\nu_e: \qquad \text{ядро полония $\ce{^{214}_{84}{Po}}$}: 84\,p^+, 130\,n^0, \\
    &\ce{^{239}_{94}{Pu}} \to \ce{^{235}_{92}{U}} + \ce{^4_2{He}}: \qquad \text{ядро урана $\ce{^{235}_{92}{U}}$}: 92\,p^+, 143\,n^0.
    \end{align*}
}
\solutionspace{80pt}

\tasknumber{6}%
\task{%
    Какая доля (от начального количества) радиоактивных ядер распадётся через время,
    равное двум периодам полураспада? Ответ выразить в процентах.
}
\answer{%
    \begin{align*}
    N &= N_0 \cdot 2^{- \frac t{T_{1/2}}} \implies
        \frac N{N_0} = 2^{- \frac t{T_{1/2}}}
        = 2^{-2} \approx 0{,}25 \approx 25\% \\
    N_\text{расп.} &= N_0 - N = N_0 - N_0 \cdot 2^{-\frac t{T_{1/2}}}
        = N_0\cbr{1 - 2^{-\frac t{T_{1/2}}}} \implies
        \frac{N_\text{расп.}}{N_0} = 1 - 2^{-\frac t{T_{1/2}}}
        = 1 - 2^{-2} \approx 0{,}75 \approx 75\%
    \end{align*}
}
\solutionspace{90pt}

\tasknumber{7}%
\task{%
    Сколько процентов ядер радиоактивного железа $\ce{^{59}Fe}$
    останется через $136{,}8\,\text{суток}$, если период его полураспада составляет $45{,}6\,\text{суток}$?
}
\answer{%
    \begin{align*}
    N &= N_0 \cdot 2^{-\frac t{T_{1/2}}}
        = 2^{-\frac{136{,}8\,\text{суток}}{45{,}6\,\text{суток}}}
        \approx 0{,}1250 = 12{,}50\%
    \end{align*}
}
\solutionspace{90pt}

\tasknumber{8}%
\task{%
    За $5\,\text{суток}$ от начального количества ядер радиоизотопа осталась одна восьмая.
    Каков период полураспада этого изотопа (ответ приведите в сутках)?
    Какая ещё доля (также от начального количества) распадётся, если подождать ещё столько же?
}
\answer{%
    \begin{align*}
            N &= N_0 \cdot 2^{-\frac t{T_{1/2}}}
            \implies \frac N{N_0} = 2^{-\frac t{T_{1/2}}}
            \implies \frac 1{8} = 2^{-\frac {5\,\text{суток}}{T_{1/2}}}
            \implies 3 = \frac {5\,\text{суток}}{T_{1/2}}
            \implies T_{1/2} = \frac {5\,\text{суток}}3 \approx 1{,}67\,\text{суток}.
         \\
            \delta &= \frac{N(t)}{N_0} - \frac{N(2t)}{N_0}
            = 2^{-\frac t{T_{1/2}}} - 2^{-\frac {2t}{T_{1/2}}}
            = 2^{-\frac t{T_{1/2}}}\cbr{1 - 2^{-\frac t{T_{1/2}}}}
            = \frac 1{8} \cdot \cbr{1-\frac 1{8}} \approx 0{,}109
    \end{align*}
}

\variantsplitter

\addpersonalvariant{Варвара Пранова}

\tasknumber{1}%
\task{%
    Определите число протонов в атоме $\ce{^{47}_{20}{Ca}}$.
}
\answer{%
    $Z = 20$ протонов и столько же электронов $A = 47$ нуклонов, $A - Z = 27$ нейтронов.
    Ответ: 20
}

\tasknumber{2}%
\task{%
    Определите число электронов в атоме $\text{аргон-40}$.
}
\answer{%
    $Z = 18$ протонов и столько же электронов, $A = 40$ нуклонов, $A - Z = 22$ нейтронов.
    Ответ: 18
}

\tasknumber{3}%
\task{%
    Энергия связи ядра углерода \ce{^{13}_{6}C} равна $97{,}1\,\text{МэВ}$.
    Найти дефект массы этого ядра.
    Ответ выразите в а.е.м.
    и кг.
    Скорость света $c = 2{,}998 \cdot 10^{8}\,\frac{\text{м}}{\text{с}}$, элементарный заряд $e = 1{,}6 \cdot 10^{-19}\,\text{Кл}$.
}
\answer{%
    \begin{align*}
    E_\text{св.} &= \Delta m c^2 \implies \\
    \implies
            \Delta m &= \frac {E_\text{св.}}{c^2} = \frac{97{,}1\,\text{МэВ}}{\sqr{2{,}998 \cdot 10^{8}\,\frac{\text{м}}{\text{с}}}}
            = \frac{97{,}1 \cdot 10^6 \cdot 1{,}6 \cdot 10^{-19}\,\text{Дж}}{\sqr{2{,}998 \cdot 10^{8}\,\frac{\text{м}}{\text{с}}}}
            \approx 0{,}1729 \cdot 10^{-27}\,\text{кг} \approx 0{,}1041\,\text{а.е.м.}
    \end{align*}
}
\solutionspace{100pt}

\tasknumber{4}%
\task{%
    Определите дефект массы (в а.е.м.) и энергию связи (в МэВ) ядра атома \ce{^{4}_{2}{He}},
    если его масса составляет $4{,}0026\,\text{а.е.м.}$.
    Считать $m_{p} = 1{,}00728\,\text{а.е.м.}$, $m_{n} = 1{,}00867\,\text{а.е.м.}$.
}
\answer{%
    \begin{align*}
    \Delta m &= (A - Z)m_{n} + Zm_{p} - m = 2 \cdot 1{,}00867\,\text{а.е.м.} + 2 \cdot 1{,}00728\,\text{а.е.м.} - 4{,}0026\,\text{а.е.м.} \approx 0{,}0293\,\text{а.е.м.} \\
    E_\text{св.} &= \Delta m c^2 \approx 0{,}0293 \cdot 931{,}5\,\text{МэВ} \approx 27{,}4\,\text{МэВ}
    \end{align*}
}
\solutionspace{90pt}

\tasknumber{5}%
\task{%
    В какое ядро превращается исходное в результате ядерного распада?
    Запишите уравнение реакции и явно укажите число протонов и нейтронов в получившемся ядре.
    \begin{itemize}
        \item ядро углерода $\ce{^{14}_{6}{C}}$, $\beta^-$-распад,
        \item ядро урана $\ce{^{238}_{92}{U}}$, $\alpha$-распад,
        \item ядро свинца $\ce{^{214}_{82}{Pb}}$, $\alpha$-распад,
        \item ядро плутония $\ce{^{239}_{94}{Pu}}$, $\alpha$-распад.
    \end{itemize}
}
\answer{%
    \begin{align*}
    &\ce{^{14}_{6}{C}} \to \ce{^{14}_{7}{N}} + e^- + \tilde\nu_e: \qquad \text{ядро азота $\ce{^{14}_{7}{N}}$}: 7\,p^+, 7\,n^0, \\
    &\ce{^{238}_{92}{U}} \to \ce{^{234}_{90}{Th}} + \ce{^4_2{He}}: \qquad \text{ядро тория $\ce{^{234}_{90}{Th}}$}: 90\,p^+, 144\,n^0, \\
    &\ce{^{214}_{82}{Pb}} \to \ce{^{210}_{80}{Hg}} + \ce{^4_2{He}}: \qquad \text{ядро ртути $\ce{^{210}_{80}{Hg}}$}: 80\,p^+, 130\,n^0, \\
    &\ce{^{239}_{94}{Pu}} \to \ce{^{235}_{92}{U}} + \ce{^4_2{He}}: \qquad \text{ядро урана $\ce{^{235}_{92}{U}}$}: 92\,p^+, 143\,n^0.
    \end{align*}
}
\solutionspace{80pt}

\tasknumber{6}%
\task{%
    Какая доля (от начального количества) радиоактивных ядер распадётся через время,
    равное двум периодам полураспада? Ответ выразить в процентах.
}
\answer{%
    \begin{align*}
    N &= N_0 \cdot 2^{- \frac t{T_{1/2}}} \implies
        \frac N{N_0} = 2^{- \frac t{T_{1/2}}}
        = 2^{-2} \approx 0{,}25 \approx 25\% \\
    N_\text{расп.} &= N_0 - N = N_0 - N_0 \cdot 2^{-\frac t{T_{1/2}}}
        = N_0\cbr{1 - 2^{-\frac t{T_{1/2}}}} \implies
        \frac{N_\text{расп.}}{N_0} = 1 - 2^{-\frac t{T_{1/2}}}
        = 1 - 2^{-2} \approx 0{,}75 \approx 75\%
    \end{align*}
}
\solutionspace{90pt}

\tasknumber{7}%
\task{%
    Сколько процентов ядер радиоактивного железа $\ce{^{59}Fe}$
    останется через $91{,}2\,\text{суток}$, если период его полураспада составляет $45{,}6\,\text{суток}$?
}
\answer{%
    \begin{align*}
    N &= N_0 \cdot 2^{-\frac t{T_{1/2}}}
        = 2^{-\frac{91{,}2\,\text{суток}}{45{,}6\,\text{суток}}}
        \approx 0{,}2500 = 25{,}00\%
    \end{align*}
}
\solutionspace{90pt}

\tasknumber{8}%
\task{%
    За $4\,\text{суток}$ от начального количества ядер радиоизотопа осталась одна восьмая.
    Каков период полураспада этого изотопа (ответ приведите в сутках)?
    Какая ещё доля (также от начального количества) распадётся, если подождать ещё столько же?
}
\answer{%
    \begin{align*}
            N &= N_0 \cdot 2^{-\frac t{T_{1/2}}}
            \implies \frac N{N_0} = 2^{-\frac t{T_{1/2}}}
            \implies \frac 1{8} = 2^{-\frac {4\,\text{суток}}{T_{1/2}}}
            \implies 3 = \frac {4\,\text{суток}}{T_{1/2}}
            \implies T_{1/2} = \frac {4\,\text{суток}}3 \approx 1{,}33\,\text{суток}.
         \\
            \delta &= \frac{N(t)}{N_0} - \frac{N(2t)}{N_0}
            = 2^{-\frac t{T_{1/2}}} - 2^{-\frac {2t}{T_{1/2}}}
            = 2^{-\frac t{T_{1/2}}}\cbr{1 - 2^{-\frac t{T_{1/2}}}}
            = \frac 1{8} \cdot \cbr{1-\frac 1{8}} \approx 0{,}109
    \end{align*}
}

\variantsplitter

\addpersonalvariant{Марьям Салимова}

\tasknumber{1}%
\task{%
    Определите число протонов в атоме $\ce{^{44}_{20}{Ca}}$.
}
\answer{%
    $Z = 20$ протонов и столько же электронов $A = 44$ нуклонов, $A - Z = 24$ нейтронов.
    Ответ: 20
}

\tasknumber{2}%
\task{%
    Определите число протонов в атоме $\text{аргон-39}$.
}
\answer{%
    $Z = 18$ протонов и столько же электронов, $A = 39$ нуклонов, $A - Z = 21$ нейтронов.
    Ответ: 18
}

\tasknumber{3}%
\task{%
    Энергия связи ядра дейтерия \ce{^{2}_{1}H} (D) равна $2{,}22\,\text{МэВ}$.
    Найти дефект массы этого ядра.
    Ответ выразите в а.е.м.
    и кг.
    Скорость света $c = 2{,}998 \cdot 10^{8}\,\frac{\text{м}}{\text{с}}$, элементарный заряд $e = 1{,}6 \cdot 10^{-19}\,\text{Кл}$.
}
\answer{%
    \begin{align*}
    E_\text{св.} &= \Delta m c^2 \implies \\
    \implies
            \Delta m &= \frac {E_\text{св.}}{c^2} = \frac{2{,}22\,\text{МэВ}}{\sqr{2{,}998 \cdot 10^{8}\,\frac{\text{м}}{\text{с}}}}
            = \frac{2{,}22 \cdot 10^6 \cdot 1{,}6 \cdot 10^{-19}\,\text{Дж}}{\sqr{2{,}998 \cdot 10^{8}\,\frac{\text{м}}{\text{с}}}}
            \approx 3{,}95 \cdot 10^{-30}\,\text{кг} \approx 0{,}00238\,\text{а.е.м.}
    \end{align*}
}
\solutionspace{100pt}

\tasknumber{4}%
\task{%
    Определите дефект массы (в а.е.м.) и энергию связи (в МэВ) ядра атома \ce{^{6}_{2}{He}},
    если его масса составляет $6{,}0189\,\text{а.е.м.}$.
    Считать $m_{p} = 1{,}00728\,\text{а.е.м.}$, $m_{n} = 1{,}00867\,\text{а.е.м.}$.
}
\answer{%
    \begin{align*}
    \Delta m &= (A - Z)m_{n} + Zm_{p} - m = 4 \cdot 1{,}00867\,\text{а.е.м.} + 2 \cdot 1{,}00728\,\text{а.е.м.} - 6{,}0189\,\text{а.е.м.} \approx 0{,}0303\,\text{а.е.м.} \\
    E_\text{св.} &= \Delta m c^2 \approx 0{,}0303 \cdot 931{,}5\,\text{МэВ} \approx 28{,}3\,\text{МэВ}
    \end{align*}
}
\solutionspace{90pt}

\tasknumber{5}%
\task{%
    В какое ядро превращается исходное в результате ядерного распада?
    Запишите уравнение реакции и явно укажите число протонов и нейтронов в получившемся ядре.
    \begin{itemize}
        \item ядро тория $\ce{^{234}_{90}{Th}}$, $\beta$-распад,
        \item ядро урана $\ce{^{238}_{92}{U}}$, $\alpha$-распад,
        \item ядро свинца $\ce{^{206}_{82}{Pb}}$, $\alpha$-распад,
        \item ядро плутония $\ce{^{239}_{94}{Pu}}$, $\alpha$-распад.
    \end{itemize}
}
\answer{%
    \begin{align*}
    &\ce{^{234}_{90}{Th}} \to \ce{^{234}_{91}{Pa}} + e^- + \tilde\nu_e: \qquad \text{ядро протактиния $\ce{^{234}_{91}{Pa}}$}: 91\,p^+, 143\,n^0, \\
    &\ce{^{238}_{92}{U}} \to \ce{^{234}_{90}{Th}} + \ce{^4_2{He}}: \qquad \text{ядро тория $\ce{^{234}_{90}{Th}}$}: 90\,p^+, 144\,n^0, \\
    &\ce{^{206}_{82}{Pb}} \to \ce{^{202}_{80}{Hg}} + \ce{^4_2{He}}: \qquad \text{ядро ртути $\ce{^{202}_{80}{Hg}}$}: 80\,p^+, 122\,n^0, \\
    &\ce{^{239}_{94}{Pu}} \to \ce{^{235}_{92}{U}} + \ce{^4_2{He}}: \qquad \text{ядро урана $\ce{^{235}_{92}{U}}$}: 92\,p^+, 143\,n^0.
    \end{align*}
}
\solutionspace{80pt}

\tasknumber{6}%
\task{%
    Какая доля (от начального количества) радиоактивных ядер останется через время,
    равное четырём периодам полураспада? Ответ выразить в процентах.
}
\answer{%
    \begin{align*}
    N &= N_0 \cdot 2^{- \frac t{T_{1/2}}} \implies
        \frac N{N_0} = 2^{- \frac t{T_{1/2}}}
        = 2^{-4} \approx 0{,}06 \approx 6\% \\
    N_\text{расп.} &= N_0 - N = N_0 - N_0 \cdot 2^{-\frac t{T_{1/2}}}
        = N_0\cbr{1 - 2^{-\frac t{T_{1/2}}}} \implies
        \frac{N_\text{расп.}}{N_0} = 1 - 2^{-\frac t{T_{1/2}}}
        = 1 - 2^{-4} \approx 0{,}94 \approx 94\%
    \end{align*}
}
\solutionspace{90pt}

\tasknumber{7}%
\task{%
    Сколько процентов ядер радиоактивного железа $\ce{^{59}Fe}$
    останется через $91{,}2\,\text{суток}$, если период его полураспада составляет $45{,}6\,\text{суток}$?
}
\answer{%
    \begin{align*}
    N &= N_0 \cdot 2^{-\frac t{T_{1/2}}}
        = 2^{-\frac{91{,}2\,\text{суток}}{45{,}6\,\text{суток}}}
        \approx 0{,}2500 = 25{,}00\%
    \end{align*}
}
\solutionspace{90pt}

\tasknumber{8}%
\task{%
    За $4\,\text{суток}$ от начального количества ядер радиоизотопа осталась половина.
    Каков период полураспада этого изотопа (ответ приведите в сутках)?
    Какая ещё доля (также от начального количества) распадётся, если подождать ещё столько же?
}
\answer{%
    \begin{align*}
            N &= N_0 \cdot 2^{-\frac t{T_{1/2}}}
            \implies \frac N{N_0} = 2^{-\frac t{T_{1/2}}}
            \implies \frac 1{2} = 2^{-\frac {4\,\text{суток}}{T_{1/2}}}
            \implies 1 = \frac {4\,\text{суток}}{T_{1/2}}
            \implies T_{1/2} = \frac {4\,\text{суток}}1 \approx 4\,\text{суток}.
         \\
            \delta &= \frac{N(t)}{N_0} - \frac{N(2t)}{N_0}
            = 2^{-\frac t{T_{1/2}}} - 2^{-\frac {2t}{T_{1/2}}}
            = 2^{-\frac t{T_{1/2}}}\cbr{1 - 2^{-\frac t{T_{1/2}}}}
            = \frac 1{2} \cdot \cbr{1-\frac 1{2}} \approx 0{,}250
    \end{align*}
}

\variantsplitter

\addpersonalvariant{Юлия Шевченко}

\tasknumber{1}%
\task{%
    Определите число нейтронов в атоме $\ce{^{41}_{20}{Ca}}$.
}
\answer{%
    $Z = 20$ протонов и столько же электронов $A = 41$ нуклонов, $A - Z = 21$ нейтронов.
    Ответ: 21
}

\tasknumber{2}%
\task{%
    Определите число нуклонов в атоме $\text{хлор-35}$.
}
\answer{%
    $Z = 17$ протонов и столько же электронов, $A = 35$ нуклонов, $A - Z = 18$ нейтронов.
    Ответ: 35
}

\tasknumber{3}%
\task{%
    Энергия связи ядра азота \ce{^{14}_{7}N} равна $115{,}5\,\text{МэВ}$.
    Найти дефект массы этого ядра.
    Ответ выразите в а.е.м.
    и кг.
    Скорость света $c = 2{,}998 \cdot 10^{8}\,\frac{\text{м}}{\text{с}}$, элементарный заряд $e = 1{,}6 \cdot 10^{-19}\,\text{Кл}$.
}
\answer{%
    \begin{align*}
    E_\text{св.} &= \Delta m c^2 \implies \\
    \implies
            \Delta m &= \frac {E_\text{св.}}{c^2} = \frac{115{,}5\,\text{МэВ}}{\sqr{2{,}998 \cdot 10^{8}\,\frac{\text{м}}{\text{с}}}}
            = \frac{115{,}5 \cdot 10^6 \cdot 1{,}6 \cdot 10^{-19}\,\text{Дж}}{\sqr{2{,}998 \cdot 10^{8}\,\frac{\text{м}}{\text{с}}}}
            \approx 0{,}206 \cdot 10^{-27}\,\text{кг} \approx 0{,}1238\,\text{а.е.м.}
    \end{align*}
}
\solutionspace{100pt}

\tasknumber{4}%
\task{%
    Определите дефект массы (в а.е.м.) и энергию связи (в МэВ) ядра атома \ce{^{3}_{1}{T}},
    если его масса составляет $3{,}01605\,\text{а.е.м.}$.
    Считать $m_{p} = 1{,}00728\,\text{а.е.м.}$, $m_{n} = 1{,}00867\,\text{а.е.м.}$.
}
\answer{%
    \begin{align*}
    \Delta m &= (A - Z)m_{n} + Zm_{p} - m = 2 \cdot 1{,}00867\,\text{а.е.м.} + 1 \cdot 1{,}00728\,\text{а.е.м.} - 3{,}01605\,\text{а.е.м.} \approx 0{,}00857\,\text{а.е.м.} \\
    E_\text{св.} &= \Delta m c^2 \approx 0{,}0086 \cdot 931{,}5\,\text{МэВ} \approx 8{,}00\,\text{МэВ}
    \end{align*}
}
\solutionspace{90pt}

\tasknumber{5}%
\task{%
    В какое ядро превращается исходное в результате ядерного распада?
    Запишите уравнение реакции и явно укажите число протонов и нейтронов в получившемся ядре.
    \begin{itemize}
        \item ядро тория $\ce{^{234}_{90}{Th}}$, $\beta$-распад,
        \item ядро радия $\ce{^{226}_{88}{Ra}}$, $\alpha$-распад,
        \item ядро полония $\ce{^{218}_{84}{Po}}$, $\alpha$-распад,
        \item ядро плутония $\ce{^{239}_{94}{Pu}}$, $\alpha$-распад.
    \end{itemize}
}
\answer{%
    \begin{align*}
    &\ce{^{234}_{90}{Th}} \to \ce{^{234}_{91}{Pa}} + e^- + \tilde\nu_e: \qquad \text{ядро протактиния $\ce{^{234}_{91}{Pa}}$}: 91\,p^+, 143\,n^0, \\
    &\ce{^{226}_{88}{Ra}} \to \ce{^{222}_{86}{Rn}} + \ce{^4_2{He}}: \qquad \text{ядро радона $\ce{^{222}_{86}{Rn}}$}: 86\,p^+, 136\,n^0, \\
    &\ce{^{218}_{84}{Po}} \to \ce{^{214}_{82}{Pb}} + \ce{^4_2{He}}: \qquad \text{ядро свинца $\ce{^{214}_{82}{Pb}}$}: 82\,p^+, 132\,n^0, \\
    &\ce{^{239}_{94}{Pu}} \to \ce{^{235}_{92}{U}} + \ce{^4_2{He}}: \qquad \text{ядро урана $\ce{^{235}_{92}{U}}$}: 92\,p^+, 143\,n^0.
    \end{align*}
}
\solutionspace{80pt}

\tasknumber{6}%
\task{%
    Какая доля (от начального количества) радиоактивных ядер распадётся через время,
    равное двум периодам полураспада? Ответ выразить в процентах.
}
\answer{%
    \begin{align*}
    N &= N_0 \cdot 2^{- \frac t{T_{1/2}}} \implies
        \frac N{N_0} = 2^{- \frac t{T_{1/2}}}
        = 2^{-2} \approx 0{,}25 \approx 25\% \\
    N_\text{расп.} &= N_0 - N = N_0 - N_0 \cdot 2^{-\frac t{T_{1/2}}}
        = N_0\cbr{1 - 2^{-\frac t{T_{1/2}}}} \implies
        \frac{N_\text{расп.}}{N_0} = 1 - 2^{-\frac t{T_{1/2}}}
        = 1 - 2^{-2} \approx 0{,}75 \approx 75\%
    \end{align*}
}
\solutionspace{90pt}

\tasknumber{7}%
\task{%
    Сколько процентов ядер радиоактивного железа $\ce{^{59}Fe}$
    останется через $136{,}8\,\text{суток}$, если период его полураспада составляет $45{,}6\,\text{суток}$?
}
\answer{%
    \begin{align*}
    N &= N_0 \cdot 2^{-\frac t{T_{1/2}}}
        = 2^{-\frac{136{,}8\,\text{суток}}{45{,}6\,\text{суток}}}
        \approx 0{,}1250 = 12{,}50\%
    \end{align*}
}
\solutionspace{90pt}

\tasknumber{8}%
\task{%
    За $3\,\text{суток}$ от начального количества ядер радиоизотопа осталась одна восьмая.
    Каков период полураспада этого изотопа (ответ приведите в сутках)?
    Какая ещё доля (также от начального количества) распадётся, если подождать ещё столько же?
}
\answer{%
    \begin{align*}
            N &= N_0 \cdot 2^{-\frac t{T_{1/2}}}
            \implies \frac N{N_0} = 2^{-\frac t{T_{1/2}}}
            \implies \frac 1{8} = 2^{-\frac {3\,\text{суток}}{T_{1/2}}}
            \implies 3 = \frac {3\,\text{суток}}{T_{1/2}}
            \implies T_{1/2} = \frac {3\,\text{суток}}3 \approx 1\,\text{суток}.
         \\
            \delta &= \frac{N(t)}{N_0} - \frac{N(2t)}{N_0}
            = 2^{-\frac t{T_{1/2}}} - 2^{-\frac {2t}{T_{1/2}}}
            = 2^{-\frac t{T_{1/2}}}\cbr{1 - 2^{-\frac t{T_{1/2}}}}
            = \frac 1{8} \cdot \cbr{1-\frac 1{8}} \approx 0{,}109
    \end{align*}
}
% autogenerated
