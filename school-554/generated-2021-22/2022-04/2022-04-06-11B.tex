\setdate{6~апреля~2022}
\setclass{11«Б»}

\addpersonalvariant{Михаил Бурмистров}

\tasknumber{1}%
\task{%
    В ядре электрически нейтрального атома 108 частиц.
    Вокруг ядра обращается 47 электронов.
    Сколько в ядре этого атома протонов и нейтронов?
    Назовите этот элемент.
}
\answer{%
    $Z = 47$ протонов и $A - Z = 61$ нейтронов, так что это \text{серебро-108}: $\ce{^{108}_{47}{Ag}}$
}
\solutionspace{40pt}

\tasknumber{2}%
\task{%
    Энергия связи ядра углерода \ce{^{12}_{6}C} равна $92{,}2\,\text{МэВ}$.
    Найти дефект массы этого ядра.
    Ответ выразите в а.е.м.
    и кг.
    Скорость света $c = 2{,}998 \cdot 10^{8}\,\frac{\text{м}}{\text{с}}$, элементарный заряд $e = 1{,}6 \cdot 10^{-19}\,\text{Кл}$.
}
\answer{%
    \begin{align*}
    E_\text{св.} &= \Delta m c^2 \implies \\
    \implies
            \Delta m &= \frac {E_\text{св.}}{c^2} = \frac{92{,}2\,\text{МэВ}}{\sqr{2{,}998 \cdot 10^{8}\,\frac{\text{м}}{\text{с}}}}
            = \frac{92{,}2 \cdot 10^6 \cdot 1{,}6 \cdot 10^{-19}\,\text{Дж}}{\sqr{2{,}998 \cdot 10^{8}\,\frac{\text{м}}{\text{с}}}}
            \approx 0{,}1641 \cdot 10^{-27}\,\text{кг} \approx 0{,}0988\,\text{а.е.м.}
    \end{align*}
}
\solutionspace{100pt}

\tasknumber{3}%
\task{%
    Определите дефект массы (в а.е.м.) и энергию связи (в МэВ) ядра атома \ce{^{6}_{2}{He}},
    если его масса составляет $6{,}0189\,\text{а.е.м.}$.
    Считать $m_{p} = 1{,}00728\,\text{а.е.м.}$, $m_{n} = 1{,}00867\,\text{а.е.м.}$.
}
\answer{%
    \begin{align*}
    \Delta m &= (A - Z)m_{n} + Zm_{p} - m = 4 \cdot 1{,}00867\,\text{а.е.м.} + 2 \cdot 1{,}00728\,\text{а.е.м.} - 6{,}0189\,\text{а.е.м.} \approx 0{,}0303\,\text{а.е.м.} \\
    E_\text{св.} &= \Delta m c^2 \approx 0{,}0303 \cdot 931{,}5\,\text{МэВ} \approx 28{,}3\,\text{МэВ}
    \end{align*}
}
\solutionspace{90pt}

\tasknumber{4}%
\task{%
    Установите каждой букве в соответствие ровно одну цифру и запишите ответ (только цифры):

    А) $\beta$-излучение, Б) $\gamma$-излучение, В) $\alpha$-излучение.

    1) обладает положительным зарядом, 2) не несёт электрического заряда, 3) обладает отрицательным электрическим зарядом.
}
\answer{%
    $321$
}
\solutionspace{18pt}

\tasknumber{5}%
\task{%
    Установите каждой букве в соответствие ровно одну цифру и запишите ответ (только цифры):

    А) $\beta$-излучение, Б) $\gamma$-излучение, В) $\alpha$-излучение.

    1) ядра атомов гелия, 2) электромагнитное излучение, 3) электроны.
}
\answer{%
    $321$
}
\solutionspace{18pt}

\tasknumber{6}%
\task{%
    Установите каждой букве в соответствие ровно одну цифру и запишите ответ (только цифры):

    А) атом Томсона, Б) атом Резерфорда.

    1) планетарная модель атома, 2) «пудинг с изюмом».
}
\answer{%
    $21$
}
\solutionspace{18pt}

\tasknumber{7}%
\task{%
    Установите каждой букве в соответствие ровно одну цифру и запишите ответ (только цифры):

    А) размер ядра атома, Б) размер атома.

    1) $10^{-15}\units{см}$, 2) $10^{-8}\units{см}$, 3) $10^{-13}\units{см}$.
}
\answer{%
    $32$
}
\solutionspace{18pt}

\tasknumber{8}%
\task{%
    Установите каждой букве в соответствие ровно одну цифру и запишите ответ (только цифры):

    А) массовое число азота \ce{^{14}_{7}N}, Б) массовое число водорода \ce{^{1}_{1}H}.

    1) 12, 2) 3, 3) 1, 4) 14.
}
\answer{%
    $43$
}
\solutionspace{18pt}

\tasknumber{9}%
\task{%
    Установите каждой букве в соответствие ровно одну цифру и запишите ответ (только цифры):

    А) зарядовое число $\beta$-частицы, Б) массовое число $\alpha$-частицы, В) массовое число $\beta$-частицы.

    1) 1, 2) 2, 3) 0, 4) 4, 5) -1.
}
\answer{%
    $543$
}
\solutionspace{18pt}

\tasknumber{10}%
\task{%
    На какой максимальный угол (в градусах) отклонялись $\alpha$-частицы
    в опытах Резерфорда по их рассеянию на тонкой золотой фольге?
}
\answer{%
    $180\degrees$
}

\variantsplitter

\addpersonalvariant{Снежана Авдошина}

\tasknumber{1}%
\task{%
    В ядре электрически нейтрального атома 190 частиц.
    Вокруг ядра обращается 78 электронов.
    Сколько в ядре этого атома протонов и нейтронов?
    Назовите этот элемент.
}
\answer{%
    $Z = 78$ протонов и $A - Z = 112$ нейтронов, так что это \text{платина-190}: $\ce{^{190}_{78}{Pt}}$
}
\solutionspace{40pt}

\tasknumber{2}%
\task{%
    Энергия связи ядра гелия \ce{^{3}_{2}He} равна $7{,}72\,\text{МэВ}$.
    Найти дефект массы этого ядра.
    Ответ выразите в а.е.м.
    и кг.
    Скорость света $c = 2{,}998 \cdot 10^{8}\,\frac{\text{м}}{\text{с}}$, элементарный заряд $e = 1{,}6 \cdot 10^{-19}\,\text{Кл}$.
}
\answer{%
    \begin{align*}
    E_\text{св.} &= \Delta m c^2 \implies \\
    \implies
            \Delta m &= \frac {E_\text{св.}}{c^2} = \frac{7{,}72\,\text{МэВ}}{\sqr{2{,}998 \cdot 10^{8}\,\frac{\text{м}}{\text{с}}}}
            = \frac{7{,}72 \cdot 10^6 \cdot 1{,}6 \cdot 10^{-19}\,\text{Дж}}{\sqr{2{,}998 \cdot 10^{8}\,\frac{\text{м}}{\text{с}}}}
            \approx 13{,}74 \cdot 10^{-30}\,\text{кг} \approx 0{,}00828\,\text{а.е.м.}
    \end{align*}
}
\solutionspace{100pt}

\tasknumber{3}%
\task{%
    Определите дефект массы (в а.е.м.) и энергию связи (в МэВ) ядра атома \ce{^{6}_{2}{He}},
    если его масса составляет $6{,}0189\,\text{а.е.м.}$.
    Считать $m_{p} = 1{,}00728\,\text{а.е.м.}$, $m_{n} = 1{,}00867\,\text{а.е.м.}$.
}
\answer{%
    \begin{align*}
    \Delta m &= (A - Z)m_{n} + Zm_{p} - m = 4 \cdot 1{,}00867\,\text{а.е.м.} + 2 \cdot 1{,}00728\,\text{а.е.м.} - 6{,}0189\,\text{а.е.м.} \approx 0{,}0303\,\text{а.е.м.} \\
    E_\text{св.} &= \Delta m c^2 \approx 0{,}0303 \cdot 931{,}5\,\text{МэВ} \approx 28{,}3\,\text{МэВ}
    \end{align*}
}
\solutionspace{90pt}

\tasknumber{4}%
\task{%
    Установите каждой букве в соответствие ровно одну цифру и запишите ответ (только цифры):

    А) $\beta$-излучение, Б) $\gamma$-излучение, В) $\alpha$-излучение.

    1) обладает положительным зарядом, 2) не несёт электрического заряда, 3) обладает отрицательным электрическим зарядом.
}
\answer{%
    $321$
}
\solutionspace{18pt}

\tasknumber{5}%
\task{%
    Установите каждой букве в соответствие ровно одну цифру и запишите ответ (только цифры):

    А) $\beta$-излучение, Б) $\gamma$-излучение, В) $\alpha$-излучение.

    1) ядра атомов гелия, 2) электромагнитное излучение, 3) электроны.
}
\answer{%
    $321$
}
\solutionspace{18pt}

\tasknumber{6}%
\task{%
    Установите каждой букве в соответствие ровно одну цифру и запишите ответ (только цифры):

    А) атом Томсона, Б) атом Резерфорда.

    1) планетарная модель атома, 2) «пудинг с изюмом».
}
\answer{%
    $21$
}
\solutionspace{18pt}

\tasknumber{7}%
\task{%
    Установите каждой букве в соответствие ровно одну цифру и запишите ответ (только цифры):

    А) размер атома, Б) размер ядра атома.

    1) $10^{-15}\units{см}$, 2) $10^{-13}\units{см}$, 3) $10^{-8}\units{см}$.
}
\answer{%
    $32$
}
\solutionspace{18pt}

\tasknumber{8}%
\task{%
    Установите каждой букве в соответствие ровно одну цифру и запишите ответ (только цифры):

    А) массовое число углерода \ce{^{12}_{6}C}, Б) зарядовое число азота \ce{^{14}_{7}N}.

    1) 7, 2) 9, 3) 14, 4) 12.
}
\answer{%
    $41$
}
\solutionspace{18pt}

\tasknumber{9}%
\task{%
    Установите каждой букве в соответствие ровно одну цифру и запишите ответ (только цифры):

    А) зарядовое число $\beta$-частицы, Б) массовое число $\alpha$-частицы, В) массовое число $\beta$-частицы.

    1) 4, 2) 0, 3) 1, 4) 2, 5) -1.
}
\answer{%
    $512$
}
\solutionspace{18pt}

\tasknumber{10}%
\task{%
    На какой максимальный угол (в градусах) отклонялись $\alpha$-частицы
    в опытах Резерфорда по их рассеянию на тонкой золотой фольге?
}
\answer{%
    $180\degrees$
}

\variantsplitter

\addpersonalvariant{Марьяна Аристова}

\tasknumber{1}%
\task{%
    В ядре электрически нейтрального атома 123 частиц.
    Вокруг ядра обращается 51 электронов.
    Сколько в ядре этого атома протонов и нейтронов?
    Назовите этот элемент.
}
\answer{%
    $Z = 51$ протонов и $A - Z = 72$ нейтронов, так что это \text{сурьма-123}: $\ce{^{123}_{51}{Sb}}$
}
\solutionspace{40pt}

\tasknumber{2}%
\task{%
    Энергия связи ядра дейтерия \ce{^{2}_{1}H} (D) равна $2{,}22\,\text{МэВ}$.
    Найти дефект массы этого ядра.
    Ответ выразите в а.е.м.
    и кг.
    Скорость света $c = 2{,}998 \cdot 10^{8}\,\frac{\text{м}}{\text{с}}$, элементарный заряд $e = 1{,}6 \cdot 10^{-19}\,\text{Кл}$.
}
\answer{%
    \begin{align*}
    E_\text{св.} &= \Delta m c^2 \implies \\
    \implies
            \Delta m &= \frac {E_\text{св.}}{c^2} = \frac{2{,}22\,\text{МэВ}}{\sqr{2{,}998 \cdot 10^{8}\,\frac{\text{м}}{\text{с}}}}
            = \frac{2{,}22 \cdot 10^6 \cdot 1{,}6 \cdot 10^{-19}\,\text{Дж}}{\sqr{2{,}998 \cdot 10^{8}\,\frac{\text{м}}{\text{с}}}}
            \approx 3{,}95 \cdot 10^{-30}\,\text{кг} \approx 0{,}00238\,\text{а.е.м.}
    \end{align*}
}
\solutionspace{100pt}

\tasknumber{3}%
\task{%
    Определите дефект массы (в а.е.м.) и энергию связи (в МэВ) ядра атома \ce{^{8}_{2}{He}},
    если его масса составляет $8{,}0225\,\text{а.е.м.}$.
    Считать $m_{p} = 1{,}00728\,\text{а.е.м.}$, $m_{n} = 1{,}00867\,\text{а.е.м.}$.
}
\answer{%
    \begin{align*}
    \Delta m &= (A - Z)m_{n} + Zm_{p} - m = 6 \cdot 1{,}00867\,\text{а.е.м.} + 2 \cdot 1{,}00728\,\text{а.е.м.} - 8{,}0225\,\text{а.е.м.} \approx 0{,}0441\,\text{а.е.м.} \\
    E_\text{св.} &= \Delta m c^2 \approx 0{,}0441 \cdot 931{,}5\,\text{МэВ} \approx 41{,}2\,\text{МэВ}
    \end{align*}
}
\solutionspace{90pt}

\tasknumber{4}%
\task{%
    Установите каждой букве в соответствие ровно одну цифру и запишите ответ (только цифры):

    А) $\beta$-излучение, Б) $\alpha$-излучение, В) $\gamma$-излучение.

    1) не несёт электрического заряда, 2) обладает положительным зарядом, 3) обладает отрицательным электрическим зарядом.
}
\answer{%
    $321$
}
\solutionspace{18pt}

\tasknumber{5}%
\task{%
    Установите каждой букве в соответствие ровно одну цифру и запишите ответ (только цифры):

    А) $\beta$-излучение, Б) $\alpha$-излучение, В) $\gamma$-излучение.

    1) электромагнитное излучение, 2) ядра атомов гелия, 3) электроны.
}
\answer{%
    $321$
}
\solutionspace{18pt}

\tasknumber{6}%
\task{%
    Установите каждой букве в соответствие ровно одну цифру и запишите ответ (только цифры):

    А) атом Резерфорда, Б) атом Томсона.

    1) «пудинг с изюмом», 2) планетарная модель атома.
}
\answer{%
    $21$
}
\solutionspace{18pt}

\tasknumber{7}%
\task{%
    Установите каждой букве в соответствие ровно одну цифру и запишите ответ (только цифры):

    А) размер ядра атома, Б) размер атома.

    1) $10^{-10}\units{см }$, 2) $10^{-8}\units{см}$, 3) $10^{-13}\units{см}$.
}
\answer{%
    $32$
}
\solutionspace{18pt}

\tasknumber{8}%
\task{%
    Установите каждой букве в соответствие ровно одну цифру и запишите ответ (только цифры):

    А) зарядовое число углерода \ce{^{12}_{6}C}, Б) массовое число азота \ce{^{14}_{7}N}.

    1) 11, 2) 9, 3) 14, 4) 6.
}
\answer{%
    $43$
}
\solutionspace{18pt}

\tasknumber{9}%
\task{%
    Установите каждой букве в соответствие ровно одну цифру и запишите ответ (только цифры):

    А) массовое число $\beta$-частицы, Б) массовое число $\alpha$-частицы, В) зарядовое число $\alpha$-частицы.

    1) 4, 2) -2, 3) 1, 4) 2, 5) 0.
}
\answer{%
    $514$
}
\solutionspace{18pt}

\tasknumber{10}%
\task{%
    На какой максимальный угол (в градусах) отклонялись $\alpha$-частицы
    в опытах Резерфорда по их рассеянию на тонкой золотой фольге?
}
\answer{%
    $180\degrees$
}

\variantsplitter

\addpersonalvariant{Никита Иванов}

\tasknumber{1}%
\task{%
    В ядре электрически нейтрального атома 63 частиц.
    Вокруг ядра обращается 29 электронов.
    Сколько в ядре этого атома протонов и нейтронов?
    Назовите этот элемент.
}
\answer{%
    $Z = 29$ протонов и $A - Z = 34$ нейтронов, так что это \text{медь-63}: $\ce{^{63}_{29}{Cu}}$
}
\solutionspace{40pt}

\tasknumber{2}%
\task{%
    Энергия связи ядра углерода \ce{^{13}_{6}C} равна $97{,}1\,\text{МэВ}$.
    Найти дефект массы этого ядра.
    Ответ выразите в а.е.м.
    и кг.
    Скорость света $c = 2{,}998 \cdot 10^{8}\,\frac{\text{м}}{\text{с}}$, элементарный заряд $e = 1{,}6 \cdot 10^{-19}\,\text{Кл}$.
}
\answer{%
    \begin{align*}
    E_\text{св.} &= \Delta m c^2 \implies \\
    \implies
            \Delta m &= \frac {E_\text{св.}}{c^2} = \frac{97{,}1\,\text{МэВ}}{\sqr{2{,}998 \cdot 10^{8}\,\frac{\text{м}}{\text{с}}}}
            = \frac{97{,}1 \cdot 10^6 \cdot 1{,}6 \cdot 10^{-19}\,\text{Дж}}{\sqr{2{,}998 \cdot 10^{8}\,\frac{\text{м}}{\text{с}}}}
            \approx 0{,}1729 \cdot 10^{-27}\,\text{кг} \approx 0{,}1041\,\text{а.е.м.}
    \end{align*}
}
\solutionspace{100pt}

\tasknumber{3}%
\task{%
    Определите дефект массы (в а.е.м.) и энергию связи (в МэВ) ядра атома \ce{^{8}_{2}{He}},
    если его масса составляет $8{,}0225\,\text{а.е.м.}$.
    Считать $m_{p} = 1{,}00728\,\text{а.е.м.}$, $m_{n} = 1{,}00867\,\text{а.е.м.}$.
}
\answer{%
    \begin{align*}
    \Delta m &= (A - Z)m_{n} + Zm_{p} - m = 6 \cdot 1{,}00867\,\text{а.е.м.} + 2 \cdot 1{,}00728\,\text{а.е.м.} - 8{,}0225\,\text{а.е.м.} \approx 0{,}0441\,\text{а.е.м.} \\
    E_\text{св.} &= \Delta m c^2 \approx 0{,}0441 \cdot 931{,}5\,\text{МэВ} \approx 41{,}2\,\text{МэВ}
    \end{align*}
}
\solutionspace{90pt}

\tasknumber{4}%
\task{%
    Установите каждой букве в соответствие ровно одну цифру и запишите ответ (только цифры):

    А) $\beta$-излучение, Б) $\alpha$-излучение, В) $\gamma$-излучение.

    1) обладает положительным зарядом, 2) не несёт электрического заряда, 3) обладает отрицательным электрическим зарядом.
}
\answer{%
    $312$
}
\solutionspace{18pt}

\tasknumber{5}%
\task{%
    Установите каждой букве в соответствие ровно одну цифру и запишите ответ (только цифры):

    А) $\beta$-излучение, Б) $\alpha$-излучение, В) $\gamma$-излучение.

    1) ядра атомов гелия, 2) электромагнитное излучение, 3) электроны.
}
\answer{%
    $312$
}
\solutionspace{18pt}

\tasknumber{6}%
\task{%
    Установите каждой букве в соответствие ровно одну цифру и запишите ответ (только цифры):

    А) атом Томсона, Б) атом Резерфорда.

    1) планетарная модель атома, 2) «пудинг с изюмом».
}
\answer{%
    $21$
}
\solutionspace{18pt}

\tasknumber{7}%
\task{%
    Установите каждой букве в соответствие ровно одну цифру и запишите ответ (только цифры):

    А) размер ядра атома, Б) размер атома.

    1) $10^{-8}\units{см}$, 2) $10^{-10}\units{см }$, 3) $10^{-13}\units{см}$.
}
\answer{%
    $31$
}
\solutionspace{18pt}

\tasknumber{8}%
\task{%
    Установите каждой букве в соответствие ровно одну цифру и запишите ответ (только цифры):

    А) зарядовое число углерода \ce{^{12}_{6}C}, Б) массовое число кислорода \ce{^{16}_{8}O}.

    1) 10, 2) 16, 3) 3, 4) 6.
}
\answer{%
    $42$
}
\solutionspace{18pt}

\tasknumber{9}%
\task{%
    Установите каждой букве в соответствие ровно одну цифру и запишите ответ (только цифры):

    А) массовое число $\alpha$-частицы, Б) зарядовое число $\beta$-частицы, В) массовое число $\beta$-частицы.

    1) -1, 2) 2, 3) 0, 4) -2, 5) 4.
}
\answer{%
    $513$
}
\solutionspace{18pt}

\tasknumber{10}%
\task{%
    На какой минимальный угол (в градусах) отклонялись $\alpha$-частицы
    в опытах Резерфорда по их рассеянию на тонкой золотой фольге?
}
\answer{%
    $0\degrees$
}

\variantsplitter

\addpersonalvariant{Анастасия Князева}

\tasknumber{1}%
\task{%
    В ядре электрически нейтрального атома 63 частиц.
    Вокруг ядра обращается 29 электронов.
    Сколько в ядре этого атома протонов и нейтронов?
    Назовите этот элемент.
}
\answer{%
    $Z = 29$ протонов и $A - Z = 34$ нейтронов, так что это \text{медь-63}: $\ce{^{63}_{29}{Cu}}$
}
\solutionspace{40pt}

\tasknumber{2}%
\task{%
    Энергия связи ядра гелия \ce{^{3}_{2}He} равна $28{,}29\,\text{МэВ}$.
    Найти дефект массы этого ядра.
    Ответ выразите в а.е.м.
    и кг.
    Скорость света $c = 2{,}998 \cdot 10^{8}\,\frac{\text{м}}{\text{с}}$, элементарный заряд $e = 1{,}6 \cdot 10^{-19}\,\text{Кл}$.
}
\answer{%
    \begin{align*}
    E_\text{св.} &= \Delta m c^2 \implies \\
    \implies
            \Delta m &= \frac {E_\text{св.}}{c^2} = \frac{28{,}29\,\text{МэВ}}{\sqr{2{,}998 \cdot 10^{8}\,\frac{\text{м}}{\text{с}}}}
            = \frac{28{,}29 \cdot 10^6 \cdot 1{,}6 \cdot 10^{-19}\,\text{Дж}}{\sqr{2{,}998 \cdot 10^{8}\,\frac{\text{м}}{\text{с}}}}
            \approx 50{,}36 \cdot 10^{-30}\,\text{кг} \approx 0{,}03033\,\text{а.е.м.}
    \end{align*}
}
\solutionspace{100pt}

\tasknumber{3}%
\task{%
    Определите дефект массы (в а.е.м.) и энергию связи (в МэВ) ядра атома \ce{^{2}_{1}{D}},
    если его масса составляет $2{,}0141\,\text{а.е.м.}$.
    Считать $m_{p} = 1{,}00728\,\text{а.е.м.}$, $m_{n} = 1{,}00867\,\text{а.е.м.}$.
}
\answer{%
    \begin{align*}
    \Delta m &= (A - Z)m_{n} + Zm_{p} - m = 1 \cdot 1{,}00867\,\text{а.е.м.} + 1 \cdot 1{,}00728\,\text{а.е.м.} - 2{,}0141\,\text{а.е.м.} \approx 0{,}00185\,\text{а.е.м.} \\
    E_\text{св.} &= \Delta m c^2 \approx 0{,}0018 \cdot 931{,}5\,\text{МэВ} \approx 1{,}73\,\text{МэВ}
    \end{align*}
}
\solutionspace{90pt}

\tasknumber{4}%
\task{%
    Установите каждой букве в соответствие ровно одну цифру и запишите ответ (только цифры):

    А) $\gamma$-излучение, Б) $\alpha$-излучение, В) $\beta$-излучение.

    1) обладает положительным зарядом, 2) не несёт электрического заряда, 3) обладает отрицательным электрическим зарядом.
}
\answer{%
    $213$
}
\solutionspace{18pt}

\tasknumber{5}%
\task{%
    Установите каждой букве в соответствие ровно одну цифру и запишите ответ (только цифры):

    А) $\gamma$-излучение, Б) $\alpha$-излучение, В) $\beta$-излучение.

    1) ядра атомов гелия, 2) электромагнитное излучение, 3) электроны.
}
\answer{%
    $213$
}
\solutionspace{18pt}

\tasknumber{6}%
\task{%
    Установите каждой букве в соответствие ровно одну цифру и запишите ответ (только цифры):

    А) атом Томсона, Б) атом Резерфорда.

    1) «пудинг с изюмом», 2) планетарная модель атома.
}
\answer{%
    $12$
}
\solutionspace{18pt}

\tasknumber{7}%
\task{%
    Установите каждой букве в соответствие ровно одну цифру и запишите ответ (только цифры):

    А) размер ядра атома, Б) размер атома.

    1) $10^{-8}\units{см}$, 2) $10^{-13}\units{см}$, 3) $10^{-10}\units{см }$.
}
\answer{%
    $21$
}
\solutionspace{18pt}

\tasknumber{8}%
\task{%
    Установите каждой букве в соответствие ровно одну цифру и запишите ответ (только цифры):

    А) массовое число кислорода \ce{^{16}_{8}O}, Б) массовое число углерода \ce{^{12}_{6}C}.

    1) 7, 2) 12, 3) 16, 4) 1.
}
\answer{%
    $32$
}
\solutionspace{18pt}

\tasknumber{9}%
\task{%
    Установите каждой букве в соответствие ровно одну цифру и запишите ответ (только цифры):

    А) зарядовое число $\alpha$-частицы, Б) зарядовое число $\beta$-частицы, В) массовое число $\alpha$-частицы.

    1) 2, 2) 1, 3) 4, 4) -1, 5) -2.
}
\answer{%
    $143$
}
\solutionspace{18pt}

\tasknumber{10}%
\task{%
    На какой максимальный угол (в градусах) отклонялись $\alpha$-частицы
    в опытах Резерфорда по их рассеянию на тонкой золотой фольге?
}
\answer{%
    $180\degrees$
}

\variantsplitter

\addpersonalvariant{Елизавета Кутумова}

\tasknumber{1}%
\task{%
    В ядре электрически нейтрального атома 108 частиц.
    Вокруг ядра обращается 47 электронов.
    Сколько в ядре этого атома протонов и нейтронов?
    Назовите этот элемент.
}
\answer{%
    $Z = 47$ протонов и $A - Z = 61$ нейтронов, так что это \text{серебро-108}: $\ce{^{108}_{47}{Ag}}$
}
\solutionspace{40pt}

\tasknumber{2}%
\task{%
    Энергия связи ядра лития \ce{^{6}_{3}Li} равна $31{,}99\,\text{МэВ}$.
    Найти дефект массы этого ядра.
    Ответ выразите в а.е.м.
    и кг.
    Скорость света $c = 2{,}998 \cdot 10^{8}\,\frac{\text{м}}{\text{с}}$, элементарный заряд $e = 1{,}6 \cdot 10^{-19}\,\text{Кл}$.
}
\answer{%
    \begin{align*}
    E_\text{св.} &= \Delta m c^2 \implies \\
    \implies
            \Delta m &= \frac {E_\text{св.}}{c^2} = \frac{31{,}99\,\text{МэВ}}{\sqr{2{,}998 \cdot 10^{8}\,\frac{\text{м}}{\text{с}}}}
            = \frac{31{,}99 \cdot 10^6 \cdot 1{,}6 \cdot 10^{-19}\,\text{Дж}}{\sqr{2{,}998 \cdot 10^{8}\,\frac{\text{м}}{\text{с}}}}
            \approx 56{,}95 \cdot 10^{-30}\,\text{кг} \approx 0{,}03429\,\text{а.е.м.}
    \end{align*}
}
\solutionspace{100pt}

\tasknumber{3}%
\task{%
    Определите дефект массы (в а.е.м.) и энергию связи (в МэВ) ядра атома \ce{^{3}_{2}{He}},
    если его масса составляет $3{,}01603\,\text{а.е.м.}$.
    Считать $m_{p} = 1{,}00728\,\text{а.е.м.}$, $m_{n} = 1{,}00867\,\text{а.е.м.}$.
}
\answer{%
    \begin{align*}
    \Delta m &= (A - Z)m_{n} + Zm_{p} - m = 1 \cdot 1{,}00867\,\text{а.е.м.} + 2 \cdot 1{,}00728\,\text{а.е.м.} - 3{,}01603\,\text{а.е.м.} \approx 0{,}00720\,\text{а.е.м.} \\
    E_\text{св.} &= \Delta m c^2 \approx 0{,}0072 \cdot 931{,}5\,\text{МэВ} \approx 6{,}73\,\text{МэВ}
    \end{align*}
}
\solutionspace{90pt}

\tasknumber{4}%
\task{%
    Установите каждой букве в соответствие ровно одну цифру и запишите ответ (только цифры):

    А) $\beta$-излучение, Б) $\alpha$-излучение, В) $\gamma$-излучение.

    1) не несёт электрического заряда, 2) обладает отрицательным электрическим зарядом, 3) обладает положительным зарядом.
}
\answer{%
    $231$
}
\solutionspace{18pt}

\tasknumber{5}%
\task{%
    Установите каждой букве в соответствие ровно одну цифру и запишите ответ (только цифры):

    А) $\beta$-излучение, Б) $\alpha$-излучение, В) $\gamma$-излучение.

    1) электромагнитное излучение, 2) электроны, 3) ядра атомов гелия.
}
\answer{%
    $231$
}
\solutionspace{18pt}

\tasknumber{6}%
\task{%
    Установите каждой букве в соответствие ровно одну цифру и запишите ответ (только цифры):

    А) атом Резерфорда, Б) атом Томсона.

    1) планетарная модель атома, 2) «пудинг с изюмом».
}
\answer{%
    $12$
}
\solutionspace{18pt}

\tasknumber{7}%
\task{%
    Установите каждой букве в соответствие ровно одну цифру и запишите ответ (только цифры):

    А) размер атома, Б) размер ядра атома.

    1) $10^{-10}\units{см }$, 2) $10^{-8}\units{см}$, 3) $10^{-13}\units{см}$.
}
\answer{%
    $23$
}
\solutionspace{18pt}

\tasknumber{8}%
\task{%
    Установите каждой букве в соответствие ровно одну цифру и запишите ответ (только цифры):

    А) зарядовое число азота \ce{^{14}_{7}N}, Б) массовое число водорода \ce{^{1}_{1}H}.

    1) 7, 2) 14, 3) 1, 4) 10.
}
\answer{%
    $13$
}
\solutionspace{18pt}

\tasknumber{9}%
\task{%
    Установите каждой букве в соответствие ровно одну цифру и запишите ответ (только цифры):

    А) зарядовое число $\beta$-частицы, Б) массовое число $\alpha$-частицы, В) зарядовое число $\alpha$-частицы.

    1) -2, 2) -1, 3) 1, 4) 4, 5) 2.
}
\answer{%
    $245$
}
\solutionspace{18pt}

\tasknumber{10}%
\task{%
    На какой минимальный угол (в градусах) отклонялись $\alpha$-частицы
    в опытах Резерфорда по их рассеянию на тонкой золотой фольге?
}
\answer{%
    $0\degrees$
}

\variantsplitter

\addpersonalvariant{Роксана Мехтиева}

\tasknumber{1}%
\task{%
    В ядре электрически нейтрального атома 65 частиц.
    Вокруг ядра обращается 29 электронов.
    Сколько в ядре этого атома протонов и нейтронов?
    Назовите этот элемент.
}
\answer{%
    $Z = 29$ протонов и $A - Z = 36$ нейтронов, так что это \text{медь-65}: $\ce{^{65}_{29}{Cu}}$
}
\solutionspace{40pt}

\tasknumber{2}%
\task{%
    Энергия связи ядра бериллия \ce{^{9}_{4}Be} равна $58{,}2\,\text{МэВ}$.
    Найти дефект массы этого ядра.
    Ответ выразите в а.е.м.
    и кг.
    Скорость света $c = 2{,}998 \cdot 10^{8}\,\frac{\text{м}}{\text{с}}$, элементарный заряд $e = 1{,}6 \cdot 10^{-19}\,\text{Кл}$.
}
\answer{%
    \begin{align*}
    E_\text{св.} &= \Delta m c^2 \implies \\
    \implies
            \Delta m &= \frac {E_\text{св.}}{c^2} = \frac{58{,}2\,\text{МэВ}}{\sqr{2{,}998 \cdot 10^{8}\,\frac{\text{м}}{\text{с}}}}
            = \frac{58{,}2 \cdot 10^6 \cdot 1{,}6 \cdot 10^{-19}\,\text{Дж}}{\sqr{2{,}998 \cdot 10^{8}\,\frac{\text{м}}{\text{с}}}}
            \approx 0{,}1036 \cdot 10^{-27}\,\text{кг} \approx 0{,}0624\,\text{а.е.м.}
    \end{align*}
}
\solutionspace{100pt}

\tasknumber{3}%
\task{%
    Определите дефект массы (в а.е.м.) и энергию связи (в МэВ) ядра атома \ce{^{3}_{2}{He}},
    если его масса составляет $3{,}01603\,\text{а.е.м.}$.
    Считать $m_{p} = 1{,}00728\,\text{а.е.м.}$, $m_{n} = 1{,}00867\,\text{а.е.м.}$.
}
\answer{%
    \begin{align*}
    \Delta m &= (A - Z)m_{n} + Zm_{p} - m = 1 \cdot 1{,}00867\,\text{а.е.м.} + 2 \cdot 1{,}00728\,\text{а.е.м.} - 3{,}01603\,\text{а.е.м.} \approx 0{,}00720\,\text{а.е.м.} \\
    E_\text{св.} &= \Delta m c^2 \approx 0{,}0072 \cdot 931{,}5\,\text{МэВ} \approx 6{,}73\,\text{МэВ}
    \end{align*}
}
\solutionspace{90pt}

\tasknumber{4}%
\task{%
    Установите каждой букве в соответствие ровно одну цифру и запишите ответ (только цифры):

    А) $\beta$-излучение, Б) $\gamma$-излучение, В) $\alpha$-излучение.

    1) не несёт электрического заряда, 2) обладает положительным зарядом, 3) обладает отрицательным электрическим зарядом.
}
\answer{%
    $312$
}
\solutionspace{18pt}

\tasknumber{5}%
\task{%
    Установите каждой букве в соответствие ровно одну цифру и запишите ответ (только цифры):

    А) $\beta$-излучение, Б) $\gamma$-излучение, В) $\alpha$-излучение.

    1) электромагнитное излучение, 2) ядра атомов гелия, 3) электроны.
}
\answer{%
    $312$
}
\solutionspace{18pt}

\tasknumber{6}%
\task{%
    Установите каждой букве в соответствие ровно одну цифру и запишите ответ (только цифры):

    А) атом Резерфорда, Б) атом Томсона.

    1) «пудинг с изюмом», 2) планетарная модель атома.
}
\answer{%
    $21$
}
\solutionspace{18pt}

\tasknumber{7}%
\task{%
    Установите каждой букве в соответствие ровно одну цифру и запишите ответ (только цифры):

    А) размер атома, Б) размер ядра атома.

    1) $10^{-13}\units{см}$, 2) $10^{-15}\units{см}$, 3) $10^{-8}\units{см}$.
}
\answer{%
    $31$
}
\solutionspace{18pt}

\tasknumber{8}%
\task{%
    Установите каждой букве в соответствие ровно одну цифру и запишите ответ (только цифры):

    А) массовое число водорода \ce{^{1}_{1}H}, Б) зарядовое число лития \ce{^{6}_{3}Li}.

    1) 3, 2) 14, 3) 8, 4) 1.
}
\answer{%
    $41$
}
\solutionspace{18pt}

\tasknumber{9}%
\task{%
    Установите каждой букве в соответствие ровно одну цифру и запишите ответ (только цифры):

    А) зарядовое число $\beta$-частицы, Б) массовое число $\alpha$-частицы, В) зарядовое число $\alpha$-частицы.

    1) -2, 2) 4, 3) 2, 4) 0, 5) -1.
}
\answer{%
    $523$
}
\solutionspace{18pt}

\tasknumber{10}%
\task{%
    На какой минимальный угол (в градусах) отклонялись $\alpha$-частицы
    в опытах Резерфорда по их рассеянию на тонкой золотой фольге?
}
\answer{%
    $0\degrees$
}

\variantsplitter

\addpersonalvariant{Дилноза Нодиршоева}

\tasknumber{1}%
\task{%
    В ядре электрически нейтрального атома 63 частиц.
    Вокруг ядра обращается 29 электронов.
    Сколько в ядре этого атома протонов и нейтронов?
    Назовите этот элемент.
}
\answer{%
    $Z = 29$ протонов и $A - Z = 34$ нейтронов, так что это \text{медь-63}: $\ce{^{63}_{29}{Cu}}$
}
\solutionspace{40pt}

\tasknumber{2}%
\task{%
    Энергия связи ядра гелия \ce{^{3}_{2}He} равна $7{,}72\,\text{МэВ}$.
    Найти дефект массы этого ядра.
    Ответ выразите в а.е.м.
    и кг.
    Скорость света $c = 2{,}998 \cdot 10^{8}\,\frac{\text{м}}{\text{с}}$, элементарный заряд $e = 1{,}6 \cdot 10^{-19}\,\text{Кл}$.
}
\answer{%
    \begin{align*}
    E_\text{св.} &= \Delta m c^2 \implies \\
    \implies
            \Delta m &= \frac {E_\text{св.}}{c^2} = \frac{7{,}72\,\text{МэВ}}{\sqr{2{,}998 \cdot 10^{8}\,\frac{\text{м}}{\text{с}}}}
            = \frac{7{,}72 \cdot 10^6 \cdot 1{,}6 \cdot 10^{-19}\,\text{Дж}}{\sqr{2{,}998 \cdot 10^{8}\,\frac{\text{м}}{\text{с}}}}
            \approx 13{,}74 \cdot 10^{-30}\,\text{кг} \approx 0{,}00828\,\text{а.е.м.}
    \end{align*}
}
\solutionspace{100pt}

\tasknumber{3}%
\task{%
    Определите дефект массы (в а.е.м.) и энергию связи (в МэВ) ядра атома \ce{^{6}_{2}{He}},
    если его масса составляет $6{,}0189\,\text{а.е.м.}$.
    Считать $m_{p} = 1{,}00728\,\text{а.е.м.}$, $m_{n} = 1{,}00867\,\text{а.е.м.}$.
}
\answer{%
    \begin{align*}
    \Delta m &= (A - Z)m_{n} + Zm_{p} - m = 4 \cdot 1{,}00867\,\text{а.е.м.} + 2 \cdot 1{,}00728\,\text{а.е.м.} - 6{,}0189\,\text{а.е.м.} \approx 0{,}0303\,\text{а.е.м.} \\
    E_\text{св.} &= \Delta m c^2 \approx 0{,}0303 \cdot 931{,}5\,\text{МэВ} \approx 28{,}3\,\text{МэВ}
    \end{align*}
}
\solutionspace{90pt}

\tasknumber{4}%
\task{%
    Установите каждой букве в соответствие ровно одну цифру и запишите ответ (только цифры):

    А) $\beta$-излучение, Б) $\gamma$-излучение, В) $\alpha$-излучение.

    1) обладает положительным зарядом, 2) обладает отрицательным электрическим зарядом, 3) не несёт электрического заряда.
}
\answer{%
    $231$
}
\solutionspace{18pt}

\tasknumber{5}%
\task{%
    Установите каждой букве в соответствие ровно одну цифру и запишите ответ (только цифры):

    А) $\beta$-излучение, Б) $\gamma$-излучение, В) $\alpha$-излучение.

    1) ядра атомов гелия, 2) электроны, 3) электромагнитное излучение.
}
\answer{%
    $231$
}
\solutionspace{18pt}

\tasknumber{6}%
\task{%
    Установите каждой букве в соответствие ровно одну цифру и запишите ответ (только цифры):

    А) атом Томсона, Б) атом Резерфорда.

    1) «пудинг с изюмом», 2) планетарная модель атома.
}
\answer{%
    $12$
}
\solutionspace{18pt}

\tasknumber{7}%
\task{%
    Установите каждой букве в соответствие ровно одну цифру и запишите ответ (только цифры):

    А) размер атома, Б) размер ядра атома.

    1) $10^{-15}\units{см}$, 2) $10^{-8}\units{см}$, 3) $10^{-13}\units{см}$.
}
\answer{%
    $23$
}
\solutionspace{18pt}

\tasknumber{8}%
\task{%
    Установите каждой букве в соответствие ровно одну цифру и запишите ответ (только цифры):

    А) массовое число углерода \ce{^{12}_{6}C}, Б) массовое число водорода \ce{^{1}_{1}H}.

    1) 1, 2) 4, 3) 12, 4) 8.
}
\answer{%
    $31$
}
\solutionspace{18pt}

\tasknumber{9}%
\task{%
    Установите каждой букве в соответствие ровно одну цифру и запишите ответ (только цифры):

    А) массовое число $\beta$-частицы, Б) зарядовое число $\beta$-частицы, В) зарядовое число $\alpha$-частицы.

    1) -1, 2) 2, 3) -2, 4) 0, 5) 1.
}
\answer{%
    $412$
}
\solutionspace{18pt}

\tasknumber{10}%
\task{%
    На какой минимальный угол (в градусах) отклонялись $\alpha$-частицы
    в опытах Резерфорда по их рассеянию на тонкой золотой фольге?
}
\answer{%
    $0\degrees$
}

\variantsplitter

\addpersonalvariant{Жаклин Пантелеева}

\tasknumber{1}%
\task{%
    В ядре электрически нейтрального атома 190 частиц.
    Вокруг ядра обращается 78 электронов.
    Сколько в ядре этого атома протонов и нейтронов?
    Назовите этот элемент.
}
\answer{%
    $Z = 78$ протонов и $A - Z = 112$ нейтронов, так что это \text{платина-190}: $\ce{^{190}_{78}{Pt}}$
}
\solutionspace{40pt}

\tasknumber{2}%
\task{%
    Энергия связи ядра углерода \ce{^{12}_{6}C} равна $92{,}2\,\text{МэВ}$.
    Найти дефект массы этого ядра.
    Ответ выразите в а.е.м.
    и кг.
    Скорость света $c = 2{,}998 \cdot 10^{8}\,\frac{\text{м}}{\text{с}}$, элементарный заряд $e = 1{,}6 \cdot 10^{-19}\,\text{Кл}$.
}
\answer{%
    \begin{align*}
    E_\text{св.} &= \Delta m c^2 \implies \\
    \implies
            \Delta m &= \frac {E_\text{св.}}{c^2} = \frac{92{,}2\,\text{МэВ}}{\sqr{2{,}998 \cdot 10^{8}\,\frac{\text{м}}{\text{с}}}}
            = \frac{92{,}2 \cdot 10^6 \cdot 1{,}6 \cdot 10^{-19}\,\text{Дж}}{\sqr{2{,}998 \cdot 10^{8}\,\frac{\text{м}}{\text{с}}}}
            \approx 0{,}1641 \cdot 10^{-27}\,\text{кг} \approx 0{,}0988\,\text{а.е.м.}
    \end{align*}
}
\solutionspace{100pt}

\tasknumber{3}%
\task{%
    Определите дефект массы (в а.е.м.) и энергию связи (в МэВ) ядра атома \ce{^{4}_{2}{He}},
    если его масса составляет $4{,}0026\,\text{а.е.м.}$.
    Считать $m_{p} = 1{,}00728\,\text{а.е.м.}$, $m_{n} = 1{,}00867\,\text{а.е.м.}$.
}
\answer{%
    \begin{align*}
    \Delta m &= (A - Z)m_{n} + Zm_{p} - m = 2 \cdot 1{,}00867\,\text{а.е.м.} + 2 \cdot 1{,}00728\,\text{а.е.м.} - 4{,}0026\,\text{а.е.м.} \approx 0{,}0293\,\text{а.е.м.} \\
    E_\text{св.} &= \Delta m c^2 \approx 0{,}0293 \cdot 931{,}5\,\text{МэВ} \approx 27{,}4\,\text{МэВ}
    \end{align*}
}
\solutionspace{90pt}

\tasknumber{4}%
\task{%
    Установите каждой букве в соответствие ровно одну цифру и запишите ответ (только цифры):

    А) $\gamma$-излучение, Б) $\beta$-излучение, В) $\alpha$-излучение.

    1) обладает отрицательным электрическим зарядом, 2) обладает положительным зарядом, 3) не несёт электрического заряда.
}
\answer{%
    $312$
}
\solutionspace{18pt}

\tasknumber{5}%
\task{%
    Установите каждой букве в соответствие ровно одну цифру и запишите ответ (только цифры):

    А) $\gamma$-излучение, Б) $\beta$-излучение, В) $\alpha$-излучение.

    1) электроны, 2) ядра атомов гелия, 3) электромагнитное излучение.
}
\answer{%
    $312$
}
\solutionspace{18pt}

\tasknumber{6}%
\task{%
    Установите каждой букве в соответствие ровно одну цифру и запишите ответ (только цифры):

    А) атом Резерфорда, Б) атом Томсона.

    1) «пудинг с изюмом», 2) планетарная модель атома.
}
\answer{%
    $21$
}
\solutionspace{18pt}

\tasknumber{7}%
\task{%
    Установите каждой букве в соответствие ровно одну цифру и запишите ответ (только цифры):

    А) размер ядра атома, Б) размер атома.

    1) $10^{-8}\units{см}$, 2) $10^{-15}\units{см}$, 3) $10^{-13}\units{см}$.
}
\answer{%
    $31$
}
\solutionspace{18pt}

\tasknumber{8}%
\task{%
    Установите каждой букве в соответствие ровно одну цифру и запишите ответ (только цифры):

    А) зарядовое число кислорода \ce{^{16}_{8}O}, Б) массовое число азота \ce{^{14}_{7}N}.

    1) 11, 2) 14, 3) 9, 4) 8.
}
\answer{%
    $42$
}
\solutionspace{18pt}

\tasknumber{9}%
\task{%
    Установите каждой букве в соответствие ровно одну цифру и запишите ответ (только цифры):

    А) массовое число $\alpha$-частицы, Б) зарядовое число $\beta$-частицы, В) массовое число $\beta$-частицы.

    1) 0, 2) 2, 3) -1, 4) 1, 5) 4.
}
\answer{%
    $531$
}
\solutionspace{18pt}

\tasknumber{10}%
\task{%
    На какой минимальный угол (в градусах) отклонялись $\alpha$-частицы
    в опытах Резерфорда по их рассеянию на тонкой золотой фольге?
}
\answer{%
    $0\degrees$
}

\variantsplitter

\addpersonalvariant{Артём Переверзев}

\tasknumber{1}%
\task{%
    В ядре электрически нейтрального атома 190 частиц.
    Вокруг ядра обращается 78 электронов.
    Сколько в ядре этого атома протонов и нейтронов?
    Назовите этот элемент.
}
\answer{%
    $Z = 78$ протонов и $A - Z = 112$ нейтронов, так что это \text{платина-190}: $\ce{^{190}_{78}{Pt}}$
}
\solutionspace{40pt}

\tasknumber{2}%
\task{%
    Энергия связи ядра бериллия \ce{^{9}_{4}Be} равна $58{,}2\,\text{МэВ}$.
    Найти дефект массы этого ядра.
    Ответ выразите в а.е.м.
    и кг.
    Скорость света $c = 2{,}998 \cdot 10^{8}\,\frac{\text{м}}{\text{с}}$, элементарный заряд $e = 1{,}6 \cdot 10^{-19}\,\text{Кл}$.
}
\answer{%
    \begin{align*}
    E_\text{св.} &= \Delta m c^2 \implies \\
    \implies
            \Delta m &= \frac {E_\text{св.}}{c^2} = \frac{58{,}2\,\text{МэВ}}{\sqr{2{,}998 \cdot 10^{8}\,\frac{\text{м}}{\text{с}}}}
            = \frac{58{,}2 \cdot 10^6 \cdot 1{,}6 \cdot 10^{-19}\,\text{Дж}}{\sqr{2{,}998 \cdot 10^{8}\,\frac{\text{м}}{\text{с}}}}
            \approx 0{,}1036 \cdot 10^{-27}\,\text{кг} \approx 0{,}0624\,\text{а.е.м.}
    \end{align*}
}
\solutionspace{100pt}

\tasknumber{3}%
\task{%
    Определите дефект массы (в а.е.м.) и энергию связи (в МэВ) ядра атома \ce{^{2}_{1}{D}},
    если его масса составляет $2{,}0141\,\text{а.е.м.}$.
    Считать $m_{p} = 1{,}00728\,\text{а.е.м.}$, $m_{n} = 1{,}00867\,\text{а.е.м.}$.
}
\answer{%
    \begin{align*}
    \Delta m &= (A - Z)m_{n} + Zm_{p} - m = 1 \cdot 1{,}00867\,\text{а.е.м.} + 1 \cdot 1{,}00728\,\text{а.е.м.} - 2{,}0141\,\text{а.е.м.} \approx 0{,}00185\,\text{а.е.м.} \\
    E_\text{св.} &= \Delta m c^2 \approx 0{,}0018 \cdot 931{,}5\,\text{МэВ} \approx 1{,}73\,\text{МэВ}
    \end{align*}
}
\solutionspace{90pt}

\tasknumber{4}%
\task{%
    Установите каждой букве в соответствие ровно одну цифру и запишите ответ (только цифры):

    А) $\alpha$-излучение, Б) $\gamma$-излучение, В) $\beta$-излучение.

    1) обладает отрицательным электрическим зарядом, 2) не несёт электрического заряда, 3) обладает положительным зарядом.
}
\answer{%
    $321$
}
\solutionspace{18pt}

\tasknumber{5}%
\task{%
    Установите каждой букве в соответствие ровно одну цифру и запишите ответ (только цифры):

    А) $\alpha$-излучение, Б) $\gamma$-излучение, В) $\beta$-излучение.

    1) электроны, 2) электромагнитное излучение, 3) ядра атомов гелия.
}
\answer{%
    $321$
}
\solutionspace{18pt}

\tasknumber{6}%
\task{%
    Установите каждой букве в соответствие ровно одну цифру и запишите ответ (только цифры):

    А) атом Томсона, Б) атом Резерфорда.

    1) планетарная модель атома, 2) «пудинг с изюмом».
}
\answer{%
    $21$
}
\solutionspace{18pt}

\tasknumber{7}%
\task{%
    Установите каждой букве в соответствие ровно одну цифру и запишите ответ (только цифры):

    А) размер атома, Б) размер ядра атома.

    1) $10^{-15}\units{см}$, 2) $10^{-13}\units{см}$, 3) $10^{-8}\units{см}$.
}
\answer{%
    $32$
}
\solutionspace{18pt}

\tasknumber{8}%
\task{%
    Установите каждой букве в соответствие ровно одну цифру и запишите ответ (только цифры):

    А) массовое число углерода \ce{^{12}_{6}C}, Б) массовое число кислорода \ce{^{16}_{8}O}.

    1) 16, 2) 7, 3) 5, 4) 12.
}
\answer{%
    $41$
}
\solutionspace{18pt}

\tasknumber{9}%
\task{%
    Установите каждой букве в соответствие ровно одну цифру и запишите ответ (только цифры):

    А) массовое число $\beta$-частицы, Б) зарядовое число $\beta$-частицы, В) зарядовое число $\alpha$-частицы.

    1) -2, 2) -1, 3) 4, 4) 2, 5) 0.
}
\answer{%
    $524$
}
\solutionspace{18pt}

\tasknumber{10}%
\task{%
    На какой максимальный угол (в градусах) отклонялись $\alpha$-частицы
    в опытах Резерфорда по их рассеянию на тонкой золотой фольге?
}
\answer{%
    $180\degrees$
}

\variantsplitter

\addpersonalvariant{Варвара Пранова}

\tasknumber{1}%
\task{%
    В ядре электрически нейтрального атома 108 частиц.
    Вокруг ядра обращается 47 электронов.
    Сколько в ядре этого атома протонов и нейтронов?
    Назовите этот элемент.
}
\answer{%
    $Z = 47$ протонов и $A - Z = 61$ нейтронов, так что это \text{серебро-108}: $\ce{^{108}_{47}{Ag}}$
}
\solutionspace{40pt}

\tasknumber{2}%
\task{%
    Энергия связи ядра бора \ce{^{10}_{5}B} равна $64{,}7\,\text{МэВ}$.
    Найти дефект массы этого ядра.
    Ответ выразите в а.е.м.
    и кг.
    Скорость света $c = 2{,}998 \cdot 10^{8}\,\frac{\text{м}}{\text{с}}$, элементарный заряд $e = 1{,}6 \cdot 10^{-19}\,\text{Кл}$.
}
\answer{%
    \begin{align*}
    E_\text{св.} &= \Delta m c^2 \implies \\
    \implies
            \Delta m &= \frac {E_\text{св.}}{c^2} = \frac{64{,}7\,\text{МэВ}}{\sqr{2{,}998 \cdot 10^{8}\,\frac{\text{м}}{\text{с}}}}
            = \frac{64{,}7 \cdot 10^6 \cdot 1{,}6 \cdot 10^{-19}\,\text{Дж}}{\sqr{2{,}998 \cdot 10^{8}\,\frac{\text{м}}{\text{с}}}}
            \approx 0{,}1152 \cdot 10^{-27}\,\text{кг} \approx 0{,}0694\,\text{а.е.м.}
    \end{align*}
}
\solutionspace{100pt}

\tasknumber{3}%
\task{%
    Определите дефект массы (в а.е.м.) и энергию связи (в МэВ) ядра атома \ce{^{2}_{1}{D}},
    если его масса составляет $2{,}0141\,\text{а.е.м.}$.
    Считать $m_{p} = 1{,}00728\,\text{а.е.м.}$, $m_{n} = 1{,}00867\,\text{а.е.м.}$.
}
\answer{%
    \begin{align*}
    \Delta m &= (A - Z)m_{n} + Zm_{p} - m = 1 \cdot 1{,}00867\,\text{а.е.м.} + 1 \cdot 1{,}00728\,\text{а.е.м.} - 2{,}0141\,\text{а.е.м.} \approx 0{,}00185\,\text{а.е.м.} \\
    E_\text{св.} &= \Delta m c^2 \approx 0{,}0018 \cdot 931{,}5\,\text{МэВ} \approx 1{,}73\,\text{МэВ}
    \end{align*}
}
\solutionspace{90pt}

\tasknumber{4}%
\task{%
    Установите каждой букве в соответствие ровно одну цифру и запишите ответ (только цифры):

    А) $\gamma$-излучение, Б) $\beta$-излучение, В) $\alpha$-излучение.

    1) обладает отрицательным электрическим зарядом, 2) обладает положительным зарядом, 3) не несёт электрического заряда.
}
\answer{%
    $312$
}
\solutionspace{18pt}

\tasknumber{5}%
\task{%
    Установите каждой букве в соответствие ровно одну цифру и запишите ответ (только цифры):

    А) $\gamma$-излучение, Б) $\beta$-излучение, В) $\alpha$-излучение.

    1) электроны, 2) ядра атомов гелия, 3) электромагнитное излучение.
}
\answer{%
    $312$
}
\solutionspace{18pt}

\tasknumber{6}%
\task{%
    Установите каждой букве в соответствие ровно одну цифру и запишите ответ (только цифры):

    А) атом Резерфорда, Б) атом Томсона.

    1) «пудинг с изюмом», 2) планетарная модель атома.
}
\answer{%
    $21$
}
\solutionspace{18pt}

\tasknumber{7}%
\task{%
    Установите каждой букве в соответствие ровно одну цифру и запишите ответ (только цифры):

    А) размер ядра атома, Б) размер атома.

    1) $10^{-8}\units{см}$, 2) $10^{-15}\units{см}$, 3) $10^{-13}\units{см}$.
}
\answer{%
    $31$
}
\solutionspace{18pt}

\tasknumber{8}%
\task{%
    Установите каждой букве в соответствие ровно одну цифру и запишите ответ (только цифры):

    А) массовое число азота \ce{^{14}_{7}N}, Б) массовое число углерода \ce{^{12}_{6}C}.

    1) 7, 2) 12, 3) 4, 4) 14.
}
\answer{%
    $42$
}
\solutionspace{18pt}

\tasknumber{9}%
\task{%
    Установите каждой букве в соответствие ровно одну цифру и запишите ответ (только цифры):

    А) массовое число $\alpha$-частицы, Б) массовое число $\beta$-частицы, В) зарядовое число $\beta$-частицы.

    1) -2, 2) 1, 3) 0, 4) -1, 5) 4.
}
\answer{%
    $534$
}
\solutionspace{18pt}

\tasknumber{10}%
\task{%
    На какой минимальный угол (в градусах) отклонялись $\alpha$-частицы
    в опытах Резерфорда по их рассеянию на тонкой золотой фольге?
}
\answer{%
    $0\degrees$
}

\variantsplitter

\addpersonalvariant{Марьям Салимова}

\tasknumber{1}%
\task{%
    В ядре электрически нейтрального атома 65 частиц.
    Вокруг ядра обращается 29 электронов.
    Сколько в ядре этого атома протонов и нейтронов?
    Назовите этот элемент.
}
\answer{%
    $Z = 29$ протонов и $A - Z = 36$ нейтронов, так что это \text{медь-65}: $\ce{^{65}_{29}{Cu}}$
}
\solutionspace{40pt}

\tasknumber{2}%
\task{%
    Энергия связи ядра азота \ce{^{14}_{7}N} равна $115{,}5\,\text{МэВ}$.
    Найти дефект массы этого ядра.
    Ответ выразите в а.е.м.
    и кг.
    Скорость света $c = 2{,}998 \cdot 10^{8}\,\frac{\text{м}}{\text{с}}$, элементарный заряд $e = 1{,}6 \cdot 10^{-19}\,\text{Кл}$.
}
\answer{%
    \begin{align*}
    E_\text{св.} &= \Delta m c^2 \implies \\
    \implies
            \Delta m &= \frac {E_\text{св.}}{c^2} = \frac{115{,}5\,\text{МэВ}}{\sqr{2{,}998 \cdot 10^{8}\,\frac{\text{м}}{\text{с}}}}
            = \frac{115{,}5 \cdot 10^6 \cdot 1{,}6 \cdot 10^{-19}\,\text{Дж}}{\sqr{2{,}998 \cdot 10^{8}\,\frac{\text{м}}{\text{с}}}}
            \approx 0{,}206 \cdot 10^{-27}\,\text{кг} \approx 0{,}1238\,\text{а.е.м.}
    \end{align*}
}
\solutionspace{100pt}

\tasknumber{3}%
\task{%
    Определите дефект массы (в а.е.м.) и энергию связи (в МэВ) ядра атома \ce{^{4}_{2}{He}},
    если его масса составляет $4{,}0026\,\text{а.е.м.}$.
    Считать $m_{p} = 1{,}00728\,\text{а.е.м.}$, $m_{n} = 1{,}00867\,\text{а.е.м.}$.
}
\answer{%
    \begin{align*}
    \Delta m &= (A - Z)m_{n} + Zm_{p} - m = 2 \cdot 1{,}00867\,\text{а.е.м.} + 2 \cdot 1{,}00728\,\text{а.е.м.} - 4{,}0026\,\text{а.е.м.} \approx 0{,}0293\,\text{а.е.м.} \\
    E_\text{св.} &= \Delta m c^2 \approx 0{,}0293 \cdot 931{,}5\,\text{МэВ} \approx 27{,}4\,\text{МэВ}
    \end{align*}
}
\solutionspace{90pt}

\tasknumber{4}%
\task{%
    Установите каждой букве в соответствие ровно одну цифру и запишите ответ (только цифры):

    А) $\alpha$-излучение, Б) $\gamma$-излучение, В) $\beta$-излучение.

    1) не несёт электрического заряда, 2) обладает положительным зарядом, 3) обладает отрицательным электрическим зарядом.
}
\answer{%
    $213$
}
\solutionspace{18pt}

\tasknumber{5}%
\task{%
    Установите каждой букве в соответствие ровно одну цифру и запишите ответ (только цифры):

    А) $\alpha$-излучение, Б) $\gamma$-излучение, В) $\beta$-излучение.

    1) электромагнитное излучение, 2) ядра атомов гелия, 3) электроны.
}
\answer{%
    $213$
}
\solutionspace{18pt}

\tasknumber{6}%
\task{%
    Установите каждой букве в соответствие ровно одну цифру и запишите ответ (только цифры):

    А) атом Резерфорда, Б) атом Томсона.

    1) планетарная модель атома, 2) «пудинг с изюмом».
}
\answer{%
    $12$
}
\solutionspace{18pt}

\tasknumber{7}%
\task{%
    Установите каждой букве в соответствие ровно одну цифру и запишите ответ (только цифры):

    А) размер ядра атома, Б) размер атома.

    1) $10^{-8}\units{см}$, 2) $10^{-13}\units{см}$, 3) $10^{-15}\units{см}$.
}
\answer{%
    $21$
}
\solutionspace{18pt}

\tasknumber{8}%
\task{%
    Установите каждой букве в соответствие ровно одну цифру и запишите ответ (только цифры):

    А) массовое число кислорода \ce{^{16}_{8}O}, Б) зарядовое число лития \ce{^{6}_{3}Li}.

    1) 1, 2) 3, 3) 16, 4) 12.
}
\answer{%
    $32$
}
\solutionspace{18pt}

\tasknumber{9}%
\task{%
    Установите каждой букве в соответствие ровно одну цифру и запишите ответ (только цифры):

    А) зарядовое число $\alpha$-частицы, Б) зарядовое число $\beta$-частицы, В) массовое число $\beta$-частицы.

    1) 2, 2) -2, 3) 0, 4) -1, 5) 1.
}
\answer{%
    $143$
}
\solutionspace{18pt}

\tasknumber{10}%
\task{%
    На какой минимальный угол (в градусах) отклонялись $\alpha$-частицы
    в опытах Резерфорда по их рассеянию на тонкой золотой фольге?
}
\answer{%
    $0\degrees$
}

\variantsplitter

\addpersonalvariant{Юлия Шевченко}

\tasknumber{1}%
\task{%
    В ядре электрически нейтрального атома 63 частиц.
    Вокруг ядра обращается 29 электронов.
    Сколько в ядре этого атома протонов и нейтронов?
    Назовите этот элемент.
}
\answer{%
    $Z = 29$ протонов и $A - Z = 34$ нейтронов, так что это \text{медь-63}: $\ce{^{63}_{29}{Cu}}$
}
\solutionspace{40pt}

\tasknumber{2}%
\task{%
    Энергия связи ядра гелия \ce{^{3}_{2}He} равна $28{,}29\,\text{МэВ}$.
    Найти дефект массы этого ядра.
    Ответ выразите в а.е.м.
    и кг.
    Скорость света $c = 2{,}998 \cdot 10^{8}\,\frac{\text{м}}{\text{с}}$, элементарный заряд $e = 1{,}6 \cdot 10^{-19}\,\text{Кл}$.
}
\answer{%
    \begin{align*}
    E_\text{св.} &= \Delta m c^2 \implies \\
    \implies
            \Delta m &= \frac {E_\text{св.}}{c^2} = \frac{28{,}29\,\text{МэВ}}{\sqr{2{,}998 \cdot 10^{8}\,\frac{\text{м}}{\text{с}}}}
            = \frac{28{,}29 \cdot 10^6 \cdot 1{,}6 \cdot 10^{-19}\,\text{Дж}}{\sqr{2{,}998 \cdot 10^{8}\,\frac{\text{м}}{\text{с}}}}
            \approx 50{,}36 \cdot 10^{-30}\,\text{кг} \approx 0{,}03033\,\text{а.е.м.}
    \end{align*}
}
\solutionspace{100pt}

\tasknumber{3}%
\task{%
    Определите дефект массы (в а.е.м.) и энергию связи (в МэВ) ядра атома \ce{^{8}_{2}{He}},
    если его масса составляет $8{,}0225\,\text{а.е.м.}$.
    Считать $m_{p} = 1{,}00728\,\text{а.е.м.}$, $m_{n} = 1{,}00867\,\text{а.е.м.}$.
}
\answer{%
    \begin{align*}
    \Delta m &= (A - Z)m_{n} + Zm_{p} - m = 6 \cdot 1{,}00867\,\text{а.е.м.} + 2 \cdot 1{,}00728\,\text{а.е.м.} - 8{,}0225\,\text{а.е.м.} \approx 0{,}0441\,\text{а.е.м.} \\
    E_\text{св.} &= \Delta m c^2 \approx 0{,}0441 \cdot 931{,}5\,\text{МэВ} \approx 41{,}2\,\text{МэВ}
    \end{align*}
}
\solutionspace{90pt}

\tasknumber{4}%
\task{%
    Установите каждой букве в соответствие ровно одну цифру и запишите ответ (только цифры):

    А) $\gamma$-излучение, Б) $\beta$-излучение, В) $\alpha$-излучение.

    1) обладает положительным зарядом, 2) обладает отрицательным электрическим зарядом, 3) не несёт электрического заряда.
}
\answer{%
    $321$
}
\solutionspace{18pt}

\tasknumber{5}%
\task{%
    Установите каждой букве в соответствие ровно одну цифру и запишите ответ (только цифры):

    А) $\gamma$-излучение, Б) $\beta$-излучение, В) $\alpha$-излучение.

    1) ядра атомов гелия, 2) электроны, 3) электромагнитное излучение.
}
\answer{%
    $321$
}
\solutionspace{18pt}

\tasknumber{6}%
\task{%
    Установите каждой букве в соответствие ровно одну цифру и запишите ответ (только цифры):

    А) атом Томсона, Б) атом Резерфорда.

    1) планетарная модель атома, 2) «пудинг с изюмом».
}
\answer{%
    $21$
}
\solutionspace{18pt}

\tasknumber{7}%
\task{%
    Установите каждой букве в соответствие ровно одну цифру и запишите ответ (только цифры):

    А) размер ядра атома, Б) размер атома.

    1) $10^{-15}\units{см}$, 2) $10^{-8}\units{см}$, 3) $10^{-13}\units{см}$.
}
\answer{%
    $32$
}
\solutionspace{18pt}

\tasknumber{8}%
\task{%
    Установите каждой букве в соответствие ровно одну цифру и запишите ответ (только цифры):

    А) зарядовое число азота \ce{^{14}_{7}N}, Б) массовое число углерода \ce{^{12}_{6}C}.

    1) 5, 2) 3, 3) 12, 4) 7.
}
\answer{%
    $43$
}
\solutionspace{18pt}

\tasknumber{9}%
\task{%
    Установите каждой букве в соответствие ровно одну цифру и запишите ответ (только цифры):

    А) зарядовое число $\alpha$-частицы, Б) массовое число $\alpha$-частицы, В) зарядовое число $\beta$-частицы.

    1) -1, 2) 0, 3) -2, 4) 4, 5) 2.
}
\answer{%
    $541$
}
\solutionspace{18pt}

\tasknumber{10}%
\task{%
    На какой минимальный угол (в градусах) отклонялись $\alpha$-частицы
    в опытах Резерфорда по их рассеянию на тонкой золотой фольге?
}
\answer{%
    $0\degrees$
}
% autogenerated
