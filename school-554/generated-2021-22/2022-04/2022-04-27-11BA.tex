\setdate{27~апреля~2022}
\setclass{11«БА»}

\addpersonalvariant{Михаил Бурмистров}

\tasknumber{1}%
\task{%
    Положительно заряженная частица движется со скоростью $v$ в магнитном поле перпендикулярно линиям его индукции.
    Индукция магнитного поля равна $B$, масса частицы $m$, её заряд — $q$.
    Выведите из базовых физических законов формулы для радиуса траектории частицы и её частоты обращения.
}
\answer{%
    $
        F = ma, F = qvB, a = v^2 / R \implies R = \frac{mv}{qB}.
        \quad T = \frac{2\pi R}{v} = \frac{2\pi m}{qB}.
        \quad \omega = \frac vR = \frac{qB}{m}.
        \quad \nu = \frac 1T = \frac{qB}{2\pi m}.
    $
}
\solutionspace{80pt}

\tasknumber{2}%
\task{%
    В однородном горизонтальном магнитном поле с индукцией $B = 50\,\text{мТл}$ находится проводник,
    расположенный также горизонтально и перпендикулярно полю.
    Какой ток необходимо пустить по проводнику, чтобы он завис?
    Масса единицы длины проводника $\rho = 40\,\frac{\text{г}}{\text{м}}$, $g = 10\,\frac{\text{м}}{\text{с}^{2}}$.
}
\answer{%
    $
            mg = B\eli l, m=\rho l
            \implies \eli
                = \frac{g\rho}B
                = \frac{10\,\frac{\text{м}}{\text{с}^{2}} \cdot 40\,\frac{\text{г}}{\text{м}}}{50\,\text{мТл}}
                = 8\,\text{А}.
    $
}
\solutionspace{80pt}

\tasknumber{3}%
\task{%
    Протон, прошедший через ускоряющую разность потенциалов, оказывается в магнитном поле индукцией $50\,\text{мТл}$
    и движется по окружности диаметром $6\,\text{мм}$.
    Сделайте рисунок, определите значение разности потенциалов
    и укажите, в какой области потенциал больше, а где меньше.
}
\solutionspace{80pt}

\tasknumber{4}%
\task{%
    Проводник лежит на горизонтальных рельсах,
    замкнутых резистором сопротивлением $4\,\text{Ом}$ (см.
    рис.
    на доске).
    Расстояние между рельсами $50\,\text{см}$.
    Конструкция помещена в вертикальное однородное магнитное поле индукцией $150\,\text{мТл}$.
    Какую силу необходимо прикладывать к проводнику, чтобы двигать его вдоль рельс с постоянной скоростью $2\,\frac{\text{м}}{\text{c}}$?
    Трением пренебречь, сопротивления рельс и проводника малы по сравнению с сопротивлением резистора.
    Ответ выразите в миллиньютонах.
}
\answer{%
    $
        F
            = F_A
            = \eli B l
            = \frac{\ele}R \cdot B l
            = \frac{B v l}R \cdot B l
            = \frac{B^2 v l^2}R
            = \frac{\sqr{150\,\text{мТл}} \cdot 2\,\frac{\text{м}}{\text{c}} \cdot \sqr{50\,\text{см}}}{4\,\text{Ом}}
            \approx 2{,}81\,\text{мН}.
    $
}
\solutionspace{120pt}

\tasknumber{5}%
\task{%
    При изменении силы тока в проводнике по закону $\eli = 6 + 1{,}5t$ (в системе СИ),
    в нём возникает ЭДС самоиндукции $150\,\text{мВ}$.
    Чему равна индуктивность проводника?
    Ответ выразите в миллигенри и округлите до целого.
}
\answer{%
    $
        \ele = L\frac{\abs{\Delta \eli}}{\Delta t} = L \cdot \abs{ + 1{,}5 } \text{(в СИ)}
        \implies L = \frac{\ele}{ 1{,}5 } = \frac{150\,\text{мВ}}{ 1{,}5 } \approx {100\,\text{мГн}}
    $
}
\solutionspace{80pt}

\tasknumber{6}%
\task{%
    Резистор сопротивлением $R = 4\,\text{Ом}$ и катушка индуктивностью $L = 0{,}5\,\text{Гн}$ (и пренебрежимо малым сопротивлением)
    подключены параллельно к источнику тока с ЭДС $\ele = 12\,\text{В}$ и внутренним сопротивлением $r = 1\,\text{Ом}$ (см.
    рис.
    на доске).
    Какое количество теплоты выделится в цепи после размыкания ключа $K$?
}
\answer{%
    \begin{align*}
    &\text{закон Ома для полной цепи}: \eli = \frac{\ele}{r + R_\text{внешнее}} = \frac{\ele}{r + \frac{R \cdot 0}{R + 0}} = \frac{\ele}{r}, \\
    Q &= W_m = \frac{L\eli^2}2 = \frac{L\sqr{\frac{\ele}{r}}}2 = \frac L2\frac{\ele^2}{r^2} = \frac{0{,}5\,\text{Гн}}2 \cdot \sqr{\frac{12\,\text{В}}{1\,\text{Ом}}} \approx 36\,\text{Дж}.
    \end{align*}
}

\variantsplitter

\addpersonalvariant{Ирина Ан}

\tasknumber{1}%
\task{%
    Положительно заряженная частица движется со скоростью $v$ в магнитном поле перпендикулярно линиям его индукции.
    Индукция магнитного поля равна $B$, масса частицы $m$, её заряд — $q$.
    Выведите из базовых физических законов формулы для радиуса траектории частицы и её периода обращения.
}
\answer{%
    $
        F = ma, F = qvB, a = v^2 / R \implies R = \frac{mv}{qB}.
        \quad T = \frac{2\pi R}{v} = \frac{2\pi m}{qB}.
        \quad \omega = \frac vR = \frac{qB}{m}.
        \quad \nu = \frac 1T = \frac{qB}{2\pi m}.
    $
}
\solutionspace{80pt}

\tasknumber{2}%
\task{%
    В однородном горизонтальном магнитном поле с индукцией $B = 50\,\text{мТл}$ находится проводник,
    расположенный также горизонтально и перпендикулярно полю.
    Какой ток необходимо пустить по проводнику, чтобы он завис?
    Масса единицы длины проводника $\rho = 20\,\frac{\text{г}}{\text{м}}$, $g = 10\,\frac{\text{м}}{\text{с}^{2}}$.
}
\answer{%
    $
            mg = B\eli l, m=\rho l
            \implies \eli
                = \frac{g\rho}B
                = \frac{10\,\frac{\text{м}}{\text{с}^{2}} \cdot 20\,\frac{\text{г}}{\text{м}}}{50\,\text{мТл}}
                = 4\,\text{А}.
    $
}
\solutionspace{80pt}

\tasknumber{3}%
\task{%
    Электрон, прошедший через ускоряющую разность потенциалов, оказывается в магнитном поле индукцией $40\,\text{мТл}$
    и движется по окружности диаметром $4\,\text{мм}$.
    Сделайте рисунок, определите значение разности потенциалов
    и укажите, в какой области потенциал больше, а где меньше.
}
\solutionspace{80pt}

\tasknumber{4}%
\task{%
    Проводник лежит на горизонтальных рельсах,
    замкнутых резистором сопротивлением $3\,\text{Ом}$ (см.
    рис.
    на доске).
    Расстояние между рельсами $50\,\text{см}$.
    Конструкция помещена в вертикальное однородное магнитное поле индукцией $200\,\text{мТл}$.
    Какую силу необходимо прикладывать к проводнику, чтобы двигать его вдоль рельс с постоянной скоростью $4\,\frac{\text{м}}{\text{c}}$?
    Трением пренебречь, сопротивления рельс и проводника малы по сравнению с сопротивлением резистора.
    Ответ выразите в миллиньютонах.
}
\answer{%
    $
        F
            = F_A
            = \eli B l
            = \frac{\ele}R \cdot B l
            = \frac{B v l}R \cdot B l
            = \frac{B^2 v l^2}R
            = \frac{\sqr{200\,\text{мТл}} \cdot 4\,\frac{\text{м}}{\text{c}} \cdot \sqr{50\,\text{см}}}{3\,\text{Ом}}
            \approx 13{,}33\,\text{мН}.
    $
}
\solutionspace{120pt}

\tasknumber{5}%
\task{%
    При изменении силы тока в проводнике по закону $\eli = 5 - 0{,}8t$ (в системе СИ),
    в нём возникает ЭДС самоиндукции $400\,\text{мВ}$.
    Чему равна индуктивность проводника?
    Ответ выразите в миллигенри и округлите до целого.
}
\answer{%
    $
        \ele = L\frac{\abs{\Delta \eli}}{\Delta t} = L \cdot \abs{ - 0{,}8 } \text{(в СИ)}
        \implies L = \frac{\ele}{ 0{,}8 } = \frac{400\,\text{мВ}}{ 0{,}8 } \approx {500\,\text{мГн}}
    $
}
\solutionspace{80pt}

\tasknumber{6}%
\task{%
    Резистор сопротивлением $R = 5\,\text{Ом}$ и катушка индуктивностью $L = 0{,}5\,\text{Гн}$ (и пренебрежимо малым сопротивлением)
    подключены параллельно к источнику тока с ЭДС $\ele = 8\,\text{В}$ и внутренним сопротивлением $r = 2\,\text{Ом}$ (см.
    рис.
    на доске).
    Какое количество теплоты выделится в цепи после размыкания ключа $K$?
}
\answer{%
    \begin{align*}
    &\text{закон Ома для полной цепи}: \eli = \frac{\ele}{r + R_\text{внешнее}} = \frac{\ele}{r + \frac{R \cdot 0}{R + 0}} = \frac{\ele}{r}, \\
    Q &= W_m = \frac{L\eli^2}2 = \frac{L\sqr{\frac{\ele}{r}}}2 = \frac L2\frac{\ele^2}{r^2} = \frac{0{,}5\,\text{Гн}}2 \cdot \sqr{\frac{8\,\text{В}}{2\,\text{Ом}}} \approx 4\,\text{Дж}.
    \end{align*}
}

\variantsplitter

\addpersonalvariant{Софья Андрианова}

\tasknumber{1}%
\task{%
    Положительно заряженная частица движется со скоростью $v$ в магнитном поле перпендикулярно линиям его индукции.
    Индукция магнитного поля равна $B$, масса частицы $m$, её заряд — $q$.
    Выведите из базовых физических законов формулы для радиуса траектории частицы и её периода обращения.
}
\answer{%
    $
        F = ma, F = qvB, a = v^2 / R \implies R = \frac{mv}{qB}.
        \quad T = \frac{2\pi R}{v} = \frac{2\pi m}{qB}.
        \quad \omega = \frac vR = \frac{qB}{m}.
        \quad \nu = \frac 1T = \frac{qB}{2\pi m}.
    $
}
\solutionspace{80pt}

\tasknumber{2}%
\task{%
    В однородном горизонтальном магнитном поле с индукцией $B = 100\,\text{мТл}$ находится проводник,
    расположенный также горизонтально и перпендикулярно полю.
    Какой ток необходимо пустить по проводнику, чтобы он завис?
    Масса единицы длины проводника $\rho = 100\,\frac{\text{г}}{\text{м}}$, $g = 10\,\frac{\text{м}}{\text{с}^{2}}$.
}
\answer{%
    $
            mg = B\eli l, m=\rho l
            \implies \eli
                = \frac{g\rho}B
                = \frac{10\,\frac{\text{м}}{\text{с}^{2}} \cdot 100\,\frac{\text{г}}{\text{м}}}{100\,\text{мТл}}
                = 10\,\text{А}.
    $
}
\solutionspace{80pt}

\tasknumber{3}%
\task{%
    Электрон, прошедший через ускоряющую разность потенциалов, оказывается в магнитном поле индукцией $50\,\text{мТл}$
    и движется по окружности диаметром $8\,\text{мм}$.
    Сделайте рисунок, определите значение разности потенциалов
    и укажите, в какой области потенциал больше, а где меньше.
}
\solutionspace{80pt}

\tasknumber{4}%
\task{%
    Проводник лежит на горизонтальных рельсах,
    замкнутых резистором сопротивлением $2\,\text{Ом}$ (см.
    рис.
    на доске).
    Расстояние между рельсами $60\,\text{см}$.
    Конструкция помещена в вертикальное однородное магнитное поле индукцией $150\,\text{мТл}$.
    Какую силу необходимо прикладывать к проводнику, чтобы двигать его вдоль рельс с постоянной скоростью $5\,\frac{\text{м}}{\text{c}}$?
    Трением пренебречь, сопротивления рельс и проводника малы по сравнению с сопротивлением резистора.
    Ответ выразите в миллиньютонах.
}
\answer{%
    $
        F
            = F_A
            = \eli B l
            = \frac{\ele}R \cdot B l
            = \frac{B v l}R \cdot B l
            = \frac{B^2 v l^2}R
            = \frac{\sqr{150\,\text{мТл}} \cdot 5\,\frac{\text{м}}{\text{c}} \cdot \sqr{60\,\text{см}}}{2\,\text{Ом}}
            \approx 20{,}25\,\text{мН}.
    $
}
\solutionspace{120pt}

\tasknumber{5}%
\task{%
    При изменении силы тока в проводнике по закону $\eli = 4 + 0{,}5t$ (в системе СИ),
    в нём возникает ЭДС самоиндукции $400\,\text{мВ}$.
    Чему равна индуктивность проводника?
    Ответ выразите в миллигенри и округлите до целого.
}
\answer{%
    $
        \ele = L\frac{\abs{\Delta \eli}}{\Delta t} = L \cdot \abs{ + 0{,}5 } \text{(в СИ)}
        \implies L = \frac{\ele}{ 0{,}5 } = \frac{400\,\text{мВ}}{ 0{,}5 } \approx {800\,\text{мГн}}
    $
}
\solutionspace{80pt}

\tasknumber{6}%
\task{%
    Резистор сопротивлением $R = 4\,\text{Ом}$ и катушка индуктивностью $L = 0{,}4\,\text{Гн}$ (и пренебрежимо малым сопротивлением)
    подключены параллельно к источнику тока с ЭДС $\ele = 5\,\text{В}$ и внутренним сопротивлением $r = 1\,\text{Ом}$ (см.
    рис.
    на доске).
    Какое количество теплоты выделится в цепи после размыкания ключа $K$?
}
\answer{%
    \begin{align*}
    &\text{закон Ома для полной цепи}: \eli = \frac{\ele}{r + R_\text{внешнее}} = \frac{\ele}{r + \frac{R \cdot 0}{R + 0}} = \frac{\ele}{r}, \\
    Q &= W_m = \frac{L\eli^2}2 = \frac{L\sqr{\frac{\ele}{r}}}2 = \frac L2\frac{\ele^2}{r^2} = \frac{0{,}4\,\text{Гн}}2 \cdot \sqr{\frac{5\,\text{В}}{1\,\text{Ом}}} \approx 5\,\text{Дж}.
    \end{align*}
}

\variantsplitter

\addpersonalvariant{Владимир Артемчук}

\tasknumber{1}%
\task{%
    Положительно заряженная частица движется со скоростью $v$ в магнитном поле перпендикулярно линиям его индукции.
    Индукция магнитного поля равна $B$, масса частицы $m$, её заряд — $q$.
    Выведите из базовых физических законов формулы для радиуса траектории частицы и её частоты обращения.
}
\answer{%
    $
        F = ma, F = qvB, a = v^2 / R \implies R = \frac{mv}{qB}.
        \quad T = \frac{2\pi R}{v} = \frac{2\pi m}{qB}.
        \quad \omega = \frac vR = \frac{qB}{m}.
        \quad \nu = \frac 1T = \frac{qB}{2\pi m}.
    $
}
\solutionspace{80pt}

\tasknumber{2}%
\task{%
    В однородном горизонтальном магнитном поле с индукцией $B = 50\,\text{мТл}$ находится проводник,
    расположенный также горизонтально и перпендикулярно полю.
    Какой ток необходимо пустить по проводнику, чтобы он завис?
    Масса единицы длины проводника $\rho = 100\,\frac{\text{г}}{\text{м}}$, $g = 10\,\frac{\text{м}}{\text{с}^{2}}$.
}
\answer{%
    $
            mg = B\eli l, m=\rho l
            \implies \eli
                = \frac{g\rho}B
                = \frac{10\,\frac{\text{м}}{\text{с}^{2}} \cdot 100\,\frac{\text{г}}{\text{м}}}{50\,\text{мТл}}
                = 20\,\text{А}.
    $
}
\solutionspace{80pt}

\tasknumber{3}%
\task{%
    Электрон, прошедший через ускоряющую разность потенциалов, оказывается в магнитном поле индукцией $20\,\text{мТл}$
    и движется по окружности диаметром $8\,\text{мм}$.
    Сделайте рисунок, определите значение разности потенциалов
    и укажите, в какой области потенциал больше, а где меньше.
}
\solutionspace{80pt}

\tasknumber{4}%
\task{%
    Проводник лежит на горизонтальных рельсах,
    замкнутых резистором сопротивлением $3\,\text{Ом}$ (см.
    рис.
    на доске).
    Расстояние между рельсами $70\,\text{см}$.
    Конструкция помещена в вертикальное однородное магнитное поле индукцией $400\,\text{мТл}$.
    Какую силу необходимо прикладывать к проводнику, чтобы двигать его вдоль рельс с постоянной скоростью $3\,\frac{\text{м}}{\text{c}}$?
    Трением пренебречь, сопротивления рельс и проводника малы по сравнению с сопротивлением резистора.
    Ответ выразите в миллиньютонах.
}
\answer{%
    $
        F
            = F_A
            = \eli B l
            = \frac{\ele}R \cdot B l
            = \frac{B v l}R \cdot B l
            = \frac{B^2 v l^2}R
            = \frac{\sqr{400\,\text{мТл}} \cdot 3\,\frac{\text{м}}{\text{c}} \cdot \sqr{70\,\text{см}}}{3\,\text{Ом}}
            \approx 78{,}40\,\text{мН}.
    $
}
\solutionspace{120pt}

\tasknumber{5}%
\task{%
    При изменении силы тока в проводнике по закону $\eli = 7 - 1{,}5t$ (в системе СИ),
    в нём возникает ЭДС самоиндукции $200\,\text{мВ}$.
    Чему равна индуктивность проводника?
    Ответ выразите в миллигенри и округлите до целого.
}
\answer{%
    $
        \ele = L\frac{\abs{\Delta \eli}}{\Delta t} = L \cdot \abs{ - 1{,}5 } \text{(в СИ)}
        \implies L = \frac{\ele}{ 1{,}5 } = \frac{200\,\text{мВ}}{ 1{,}5 } \approx {133{,}3\,\text{мГн}}
    $
}
\solutionspace{80pt}

\tasknumber{6}%
\task{%
    Резистор сопротивлением $R = 5\,\text{Ом}$ и катушка индуктивностью $L = 0{,}2\,\text{Гн}$ (и пренебрежимо малым сопротивлением)
    подключены параллельно к источнику тока с ЭДС $\ele = 5\,\text{В}$ и внутренним сопротивлением $r = 1\,\text{Ом}$ (см.
    рис.
    на доске).
    Какое количество теплоты выделится в цепи после размыкания ключа $K$?
}
\answer{%
    \begin{align*}
    &\text{закон Ома для полной цепи}: \eli = \frac{\ele}{r + R_\text{внешнее}} = \frac{\ele}{r + \frac{R \cdot 0}{R + 0}} = \frac{\ele}{r}, \\
    Q &= W_m = \frac{L\eli^2}2 = \frac{L\sqr{\frac{\ele}{r}}}2 = \frac L2\frac{\ele^2}{r^2} = \frac{0{,}2\,\text{Гн}}2 \cdot \sqr{\frac{5\,\text{В}}{1\,\text{Ом}}} \approx 2{,}50\,\text{Дж}.
    \end{align*}
}

\variantsplitter

\addpersonalvariant{Софья Белянкина}

\tasknumber{1}%
\task{%
    Положительно заряженная частица движется со скоростью $v$ в магнитном поле перпендикулярно линиям его индукции.
    Индукция магнитного поля равна $B$, масса частицы $m$, её заряд — $q$.
    Выведите из базовых физических законов формулы для радиуса траектории частицы и её угловой скорости.
}
\answer{%
    $
        F = ma, F = qvB, a = v^2 / R \implies R = \frac{mv}{qB}.
        \quad T = \frac{2\pi R}{v} = \frac{2\pi m}{qB}.
        \quad \omega = \frac vR = \frac{qB}{m}.
        \quad \nu = \frac 1T = \frac{qB}{2\pi m}.
    $
}
\solutionspace{80pt}

\tasknumber{2}%
\task{%
    В однородном горизонтальном магнитном поле с индукцией $B = 20\,\text{мТл}$ находится проводник,
    расположенный также горизонтально и перпендикулярно полю.
    Какой ток необходимо пустить по проводнику, чтобы он завис?
    Масса единицы длины проводника $\rho = 10\,\frac{\text{г}}{\text{м}}$, $g = 10\,\frac{\text{м}}{\text{с}^{2}}$.
}
\answer{%
    $
            mg = B\eli l, m=\rho l
            \implies \eli
                = \frac{g\rho}B
                = \frac{10\,\frac{\text{м}}{\text{с}^{2}} \cdot 10\,\frac{\text{г}}{\text{м}}}{20\,\text{мТл}}
                = 5\,\text{А}.
    $
}
\solutionspace{80pt}

\tasknumber{3}%
\task{%
    Протон, прошедший через ускоряющую разность потенциалов, оказывается в магнитном поле индукцией $50\,\text{мТл}$
    и движется по окружности диаметром $8\,\text{мм}$.
    Сделайте рисунок, определите значение разности потенциалов
    и укажите, в какой области потенциал больше, а где меньше.
}
\solutionspace{80pt}

\tasknumber{4}%
\task{%
    Проводник лежит на горизонтальных рельсах,
    замкнутых резистором сопротивлением $2\,\text{Ом}$ (см.
    рис.
    на доске).
    Расстояние между рельсами $60\,\text{см}$.
    Конструкция помещена в вертикальное однородное магнитное поле индукцией $200\,\text{мТл}$.
    Какую силу необходимо прикладывать к проводнику, чтобы двигать его вдоль рельс с постоянной скоростью $2\,\frac{\text{м}}{\text{c}}$?
    Трением пренебречь, сопротивления рельс и проводника малы по сравнению с сопротивлением резистора.
    Ответ выразите в миллиньютонах.
}
\answer{%
    $
        F
            = F_A
            = \eli B l
            = \frac{\ele}R \cdot B l
            = \frac{B v l}R \cdot B l
            = \frac{B^2 v l^2}R
            = \frac{\sqr{200\,\text{мТл}} \cdot 2\,\frac{\text{м}}{\text{c}} \cdot \sqr{60\,\text{см}}}{2\,\text{Ом}}
            \approx 14{,}40\,\text{мН}.
    $
}
\solutionspace{120pt}

\tasknumber{5}%
\task{%
    При изменении силы тока в проводнике по закону $\eli = 4 - 1{,}5t$ (в системе СИ),
    в нём возникает ЭДС самоиндукции $150\,\text{мВ}$.
    Чему равна индуктивность проводника?
    Ответ выразите в миллигенри и округлите до целого.
}
\answer{%
    $
        \ele = L\frac{\abs{\Delta \eli}}{\Delta t} = L \cdot \abs{ - 1{,}5 } \text{(в СИ)}
        \implies L = \frac{\ele}{ 1{,}5 } = \frac{150\,\text{мВ}}{ 1{,}5 } \approx {100\,\text{мГн}}
    $
}
\solutionspace{80pt}

\tasknumber{6}%
\task{%
    Резистор сопротивлением $R = 4\,\text{Ом}$ и катушка индуктивностью $L = 0{,}5\,\text{Гн}$ (и пренебрежимо малым сопротивлением)
    подключены параллельно к источнику тока с ЭДС $\ele = 6\,\text{В}$ и внутренним сопротивлением $r = 1\,\text{Ом}$ (см.
    рис.
    на доске).
    Какое количество теплоты выделится в цепи после размыкания ключа $K$?
}
\answer{%
    \begin{align*}
    &\text{закон Ома для полной цепи}: \eli = \frac{\ele}{r + R_\text{внешнее}} = \frac{\ele}{r + \frac{R \cdot 0}{R + 0}} = \frac{\ele}{r}, \\
    Q &= W_m = \frac{L\eli^2}2 = \frac{L\sqr{\frac{\ele}{r}}}2 = \frac L2\frac{\ele^2}{r^2} = \frac{0{,}5\,\text{Гн}}2 \cdot \sqr{\frac{6\,\text{В}}{1\,\text{Ом}}} \approx 9\,\text{Дж}.
    \end{align*}
}

\variantsplitter

\addpersonalvariant{Варвара Егиазарян}

\tasknumber{1}%
\task{%
    Положительно заряженная частица движется со скоростью $v$ в магнитном поле перпендикулярно линиям его индукции.
    Индукция магнитного поля равна $B$, масса частицы $m$, её заряд — $q$.
    Выведите из базовых физических законов формулы для радиуса траектории частицы и её периода обращения.
}
\answer{%
    $
        F = ma, F = qvB, a = v^2 / R \implies R = \frac{mv}{qB}.
        \quad T = \frac{2\pi R}{v} = \frac{2\pi m}{qB}.
        \quad \omega = \frac vR = \frac{qB}{m}.
        \quad \nu = \frac 1T = \frac{qB}{2\pi m}.
    $
}
\solutionspace{80pt}

\tasknumber{2}%
\task{%
    В однородном горизонтальном магнитном поле с индукцией $B = 50\,\text{мТл}$ находится проводник,
    расположенный также горизонтально и перпендикулярно полю.
    Какой ток необходимо пустить по проводнику, чтобы он завис?
    Масса единицы длины проводника $\rho = 40\,\frac{\text{г}}{\text{м}}$, $g = 10\,\frac{\text{м}}{\text{с}^{2}}$.
}
\answer{%
    $
            mg = B\eli l, m=\rho l
            \implies \eli
                = \frac{g\rho}B
                = \frac{10\,\frac{\text{м}}{\text{с}^{2}} \cdot 40\,\frac{\text{г}}{\text{м}}}{50\,\text{мТл}}
                = 8\,\text{А}.
    $
}
\solutionspace{80pt}

\tasknumber{3}%
\task{%
    Протон, прошедший через ускоряющую разность потенциалов, оказывается в магнитном поле индукцией $40\,\text{мТл}$
    и движется по окружности диаметром $6\,\text{мм}$.
    Сделайте рисунок, определите значение разности потенциалов
    и укажите, в какой области потенциал больше, а где меньше.
}
\solutionspace{80pt}

\tasknumber{4}%
\task{%
    Проводник лежит на горизонтальных рельсах,
    замкнутых резистором сопротивлением $2\,\text{Ом}$ (см.
    рис.
    на доске).
    Расстояние между рельсами $80\,\text{см}$.
    Конструкция помещена в вертикальное однородное магнитное поле индукцией $400\,\text{мТл}$.
    Какую силу необходимо прикладывать к проводнику, чтобы двигать его вдоль рельс с постоянной скоростью $3\,\frac{\text{м}}{\text{c}}$?
    Трением пренебречь, сопротивления рельс и проводника малы по сравнению с сопротивлением резистора.
    Ответ выразите в миллиньютонах.
}
\answer{%
    $
        F
            = F_A
            = \eli B l
            = \frac{\ele}R \cdot B l
            = \frac{B v l}R \cdot B l
            = \frac{B^2 v l^2}R
            = \frac{\sqr{400\,\text{мТл}} \cdot 3\,\frac{\text{м}}{\text{c}} \cdot \sqr{80\,\text{см}}}{2\,\text{Ом}}
            \approx 153{,}60\,\text{мН}.
    $
}
\solutionspace{120pt}

\tasknumber{5}%
\task{%
    При изменении силы тока в проводнике по закону $\eli = 3 - 0{,}8t$ (в системе СИ),
    в нём возникает ЭДС самоиндукции $200\,\text{мВ}$.
    Чему равна индуктивность проводника?
    Ответ выразите в миллигенри и округлите до целого.
}
\answer{%
    $
        \ele = L\frac{\abs{\Delta \eli}}{\Delta t} = L \cdot \abs{ - 0{,}8 } \text{(в СИ)}
        \implies L = \frac{\ele}{ 0{,}8 } = \frac{200\,\text{мВ}}{ 0{,}8 } \approx {250\,\text{мГн}}
    $
}
\solutionspace{80pt}

\tasknumber{6}%
\task{%
    Резистор сопротивлением $R = 4\,\text{Ом}$ и катушка индуктивностью $L = 0{,}2\,\text{Гн}$ (и пренебрежимо малым сопротивлением)
    подключены параллельно к источнику тока с ЭДС $\ele = 6\,\text{В}$ и внутренним сопротивлением $r = 1\,\text{Ом}$ (см.
    рис.
    на доске).
    Какое количество теплоты выделится в цепи после размыкания ключа $K$?
}
\answer{%
    \begin{align*}
    &\text{закон Ома для полной цепи}: \eli = \frac{\ele}{r + R_\text{внешнее}} = \frac{\ele}{r + \frac{R \cdot 0}{R + 0}} = \frac{\ele}{r}, \\
    Q &= W_m = \frac{L\eli^2}2 = \frac{L\sqr{\frac{\ele}{r}}}2 = \frac L2\frac{\ele^2}{r^2} = \frac{0{,}2\,\text{Гн}}2 \cdot \sqr{\frac{6\,\text{В}}{1\,\text{Ом}}} \approx 3{,}60\,\text{Дж}.
    \end{align*}
}

\variantsplitter

\addpersonalvariant{Владислав Емелин}

\tasknumber{1}%
\task{%
    Положительно заряженная частица движется со скоростью $v$ в магнитном поле перпендикулярно линиям его индукции.
    Индукция магнитного поля равна $B$, масса частицы $m$, её заряд — $q$.
    Выведите из базовых физических законов формулы для радиуса траектории частицы и её частоты обращения.
}
\answer{%
    $
        F = ma, F = qvB, a = v^2 / R \implies R = \frac{mv}{qB}.
        \quad T = \frac{2\pi R}{v} = \frac{2\pi m}{qB}.
        \quad \omega = \frac vR = \frac{qB}{m}.
        \quad \nu = \frac 1T = \frac{qB}{2\pi m}.
    $
}
\solutionspace{80pt}

\tasknumber{2}%
\task{%
    В однородном горизонтальном магнитном поле с индукцией $B = 50\,\text{мТл}$ находится проводник,
    расположенный также горизонтально и перпендикулярно полю.
    Какой ток необходимо пустить по проводнику, чтобы он завис?
    Масса единицы длины проводника $\rho = 20\,\frac{\text{г}}{\text{м}}$, $g = 10\,\frac{\text{м}}{\text{с}^{2}}$.
}
\answer{%
    $
            mg = B\eli l, m=\rho l
            \implies \eli
                = \frac{g\rho}B
                = \frac{10\,\frac{\text{м}}{\text{с}^{2}} \cdot 20\,\frac{\text{г}}{\text{м}}}{50\,\text{мТл}}
                = 4\,\text{А}.
    $
}
\solutionspace{80pt}

\tasknumber{3}%
\task{%
    Протон, прошедший через ускоряющую разность потенциалов, оказывается в магнитном поле индукцией $40\,\text{мТл}$
    и движется по окружности диаметром $8\,\text{мм}$.
    Сделайте рисунок, определите значение разности потенциалов
    и укажите, в какой области потенциал больше, а где меньше.
}
\solutionspace{80pt}

\tasknumber{4}%
\task{%
    Проводник лежит на горизонтальных рельсах,
    замкнутых резистором сопротивлением $2\,\text{Ом}$ (см.
    рис.
    на доске).
    Расстояние между рельсами $80\,\text{см}$.
    Конструкция помещена в вертикальное однородное магнитное поле индукцией $400\,\text{мТл}$.
    Какую силу необходимо прикладывать к проводнику, чтобы двигать его вдоль рельс с постоянной скоростью $2\,\frac{\text{м}}{\text{c}}$?
    Трением пренебречь, сопротивления рельс и проводника малы по сравнению с сопротивлением резистора.
    Ответ выразите в миллиньютонах.
}
\answer{%
    $
        F
            = F_A
            = \eli B l
            = \frac{\ele}R \cdot B l
            = \frac{B v l}R \cdot B l
            = \frac{B^2 v l^2}R
            = \frac{\sqr{400\,\text{мТл}} \cdot 2\,\frac{\text{м}}{\text{c}} \cdot \sqr{80\,\text{см}}}{2\,\text{Ом}}
            \approx 102{,}40\,\text{мН}.
    $
}
\solutionspace{120pt}

\tasknumber{5}%
\task{%
    При изменении силы тока в проводнике по закону $\eli = 4 + 1{,}5t$ (в системе СИ),
    в нём возникает ЭДС самоиндукции $150\,\text{мВ}$.
    Чему равна индуктивность проводника?
    Ответ выразите в миллигенри и округлите до целого.
}
\answer{%
    $
        \ele = L\frac{\abs{\Delta \eli}}{\Delta t} = L \cdot \abs{ + 1{,}5 } \text{(в СИ)}
        \implies L = \frac{\ele}{ 1{,}5 } = \frac{150\,\text{мВ}}{ 1{,}5 } \approx {100\,\text{мГн}}
    $
}
\solutionspace{80pt}

\tasknumber{6}%
\task{%
    Резистор сопротивлением $R = 4\,\text{Ом}$ и катушка индуктивностью $L = 0{,}4\,\text{Гн}$ (и пренебрежимо малым сопротивлением)
    подключены параллельно к источнику тока с ЭДС $\ele = 6\,\text{В}$ и внутренним сопротивлением $r = 2\,\text{Ом}$ (см.
    рис.
    на доске).
    Какое количество теплоты выделится в цепи после размыкания ключа $K$?
}
\answer{%
    \begin{align*}
    &\text{закон Ома для полной цепи}: \eli = \frac{\ele}{r + R_\text{внешнее}} = \frac{\ele}{r + \frac{R \cdot 0}{R + 0}} = \frac{\ele}{r}, \\
    Q &= W_m = \frac{L\eli^2}2 = \frac{L\sqr{\frac{\ele}{r}}}2 = \frac L2\frac{\ele^2}{r^2} = \frac{0{,}4\,\text{Гн}}2 \cdot \sqr{\frac{6\,\text{В}}{2\,\text{Ом}}} \approx 1{,}80\,\text{Дж}.
    \end{align*}
}

\variantsplitter

\addpersonalvariant{Артём Жичин}

\tasknumber{1}%
\task{%
    Положительно заряженная частица движется со скоростью $v$ в магнитном поле перпендикулярно линиям его индукции.
    Индукция магнитного поля равна $B$, масса частицы $m$, её заряд — $q$.
    Выведите из базовых физических законов формулы для радиуса траектории частицы и её частоты обращения.
}
\answer{%
    $
        F = ma, F = qvB, a = v^2 / R \implies R = \frac{mv}{qB}.
        \quad T = \frac{2\pi R}{v} = \frac{2\pi m}{qB}.
        \quad \omega = \frac vR = \frac{qB}{m}.
        \quad \nu = \frac 1T = \frac{qB}{2\pi m}.
    $
}
\solutionspace{80pt}

\tasknumber{2}%
\task{%
    В однородном горизонтальном магнитном поле с индукцией $B = 10\,\text{мТл}$ находится проводник,
    расположенный также горизонтально и перпендикулярно полю.
    Какой ток необходимо пустить по проводнику, чтобы он завис?
    Масса единицы длины проводника $\rho = 20\,\frac{\text{г}}{\text{м}}$, $g = 10\,\frac{\text{м}}{\text{с}^{2}}$.
}
\answer{%
    $
            mg = B\eli l, m=\rho l
            \implies \eli
                = \frac{g\rho}B
                = \frac{10\,\frac{\text{м}}{\text{с}^{2}} \cdot 20\,\frac{\text{г}}{\text{м}}}{10\,\text{мТл}}
                = 20\,\text{А}.
    $
}
\solutionspace{80pt}

\tasknumber{3}%
\task{%
    Позитрон, прошедший через ускоряющую разность потенциалов, оказывается в магнитном поле индукцией $40\,\text{мТл}$
    и движется по окружности диаметром $4\,\text{мм}$.
    Сделайте рисунок, определите значение разности потенциалов
    и укажите, в какой области потенциал больше, а где меньше.
}
\solutionspace{80pt}

\tasknumber{4}%
\task{%
    Проводник лежит на горизонтальных рельсах,
    замкнутых резистором сопротивлением $2\,\text{Ом}$ (см.
    рис.
    на доске).
    Расстояние между рельсами $70\,\text{см}$.
    Конструкция помещена в вертикальное однородное магнитное поле индукцией $200\,\text{мТл}$.
    Какую силу необходимо прикладывать к проводнику, чтобы двигать его вдоль рельс с постоянной скоростью $2\,\frac{\text{м}}{\text{c}}$?
    Трением пренебречь, сопротивления рельс и проводника малы по сравнению с сопротивлением резистора.
    Ответ выразите в миллиньютонах.
}
\answer{%
    $
        F
            = F_A
            = \eli B l
            = \frac{\ele}R \cdot B l
            = \frac{B v l}R \cdot B l
            = \frac{B^2 v l^2}R
            = \frac{\sqr{200\,\text{мТл}} \cdot 2\,\frac{\text{м}}{\text{c}} \cdot \sqr{70\,\text{см}}}{2\,\text{Ом}}
            \approx 19{,}60\,\text{мН}.
    $
}
\solutionspace{120pt}

\tasknumber{5}%
\task{%
    При изменении силы тока в проводнике по закону $\eli = 6 + 0{,}8t$ (в системе СИ),
    в нём возникает ЭДС самоиндукции $200\,\text{мВ}$.
    Чему равна индуктивность проводника?
    Ответ выразите в миллигенри и округлите до целого.
}
\answer{%
    $
        \ele = L\frac{\abs{\Delta \eli}}{\Delta t} = L \cdot \abs{ + 0{,}8 } \text{(в СИ)}
        \implies L = \frac{\ele}{ 0{,}8 } = \frac{200\,\text{мВ}}{ 0{,}8 } \approx {250\,\text{мГн}}
    $
}
\solutionspace{80pt}

\tasknumber{6}%
\task{%
    Резистор сопротивлением $R = 4\,\text{Ом}$ и катушка индуктивностью $L = 0{,}2\,\text{Гн}$ (и пренебрежимо малым сопротивлением)
    подключены параллельно к источнику тока с ЭДС $\ele = 6\,\text{В}$ и внутренним сопротивлением $r = 1\,\text{Ом}$ (см.
    рис.
    на доске).
    Какое количество теплоты выделится в цепи после размыкания ключа $K$?
}
\answer{%
    \begin{align*}
    &\text{закон Ома для полной цепи}: \eli = \frac{\ele}{r + R_\text{внешнее}} = \frac{\ele}{r + \frac{R \cdot 0}{R + 0}} = \frac{\ele}{r}, \\
    Q &= W_m = \frac{L\eli^2}2 = \frac{L\sqr{\frac{\ele}{r}}}2 = \frac L2\frac{\ele^2}{r^2} = \frac{0{,}2\,\text{Гн}}2 \cdot \sqr{\frac{6\,\text{В}}{1\,\text{Ом}}} \approx 3{,}60\,\text{Дж}.
    \end{align*}
}

\variantsplitter

\addpersonalvariant{Дарья Кошман}

\tasknumber{1}%
\task{%
    Положительно заряженная частица движется со скоростью $v$ в магнитном поле перпендикулярно линиям его индукции.
    Индукция магнитного поля равна $B$, масса частицы $m$, её заряд — $q$.
    Выведите из базовых физических законов формулы для радиуса траектории частицы и её частоты обращения.
}
\answer{%
    $
        F = ma, F = qvB, a = v^2 / R \implies R = \frac{mv}{qB}.
        \quad T = \frac{2\pi R}{v} = \frac{2\pi m}{qB}.
        \quad \omega = \frac vR = \frac{qB}{m}.
        \quad \nu = \frac 1T = \frac{qB}{2\pi m}.
    $
}
\solutionspace{80pt}

\tasknumber{2}%
\task{%
    В однородном горизонтальном магнитном поле с индукцией $B = 100\,\text{мТл}$ находится проводник,
    расположенный также горизонтально и перпендикулярно полю.
    Какой ток необходимо пустить по проводнику, чтобы он завис?
    Масса единицы длины проводника $\rho = 10\,\frac{\text{г}}{\text{м}}$, $g = 10\,\frac{\text{м}}{\text{с}^{2}}$.
}
\answer{%
    $
            mg = B\eli l, m=\rho l
            \implies \eli
                = \frac{g\rho}B
                = \frac{10\,\frac{\text{м}}{\text{с}^{2}} \cdot 10\,\frac{\text{г}}{\text{м}}}{100\,\text{мТл}}
                = 1\,\text{А}.
    $
}
\solutionspace{80pt}

\tasknumber{3}%
\task{%
    Протон, прошедший через ускоряющую разность потенциалов, оказывается в магнитном поле индукцией $50\,\text{мТл}$
    и движется по окружности диаметром $6\,\text{мм}$.
    Сделайте рисунок, определите значение разности потенциалов
    и укажите, в какой области потенциал больше, а где меньше.
}
\solutionspace{80pt}

\tasknumber{4}%
\task{%
    Проводник лежит на горизонтальных рельсах,
    замкнутых резистором сопротивлением $2\,\text{Ом}$ (см.
    рис.
    на доске).
    Расстояние между рельсами $50\,\text{см}$.
    Конструкция помещена в вертикальное однородное магнитное поле индукцией $150\,\text{мТл}$.
    Какую силу необходимо прикладывать к проводнику, чтобы двигать его вдоль рельс с постоянной скоростью $4\,\frac{\text{м}}{\text{c}}$?
    Трением пренебречь, сопротивления рельс и проводника малы по сравнению с сопротивлением резистора.
    Ответ выразите в миллиньютонах.
}
\answer{%
    $
        F
            = F_A
            = \eli B l
            = \frac{\ele}R \cdot B l
            = \frac{B v l}R \cdot B l
            = \frac{B^2 v l^2}R
            = \frac{\sqr{150\,\text{мТл}} \cdot 4\,\frac{\text{м}}{\text{c}} \cdot \sqr{50\,\text{см}}}{2\,\text{Ом}}
            \approx 11{,}25\,\text{мН}.
    $
}
\solutionspace{120pt}

\tasknumber{5}%
\task{%
    При изменении силы тока в проводнике по закону $\eli = 5 - 1{,}5t$ (в системе СИ),
    в нём возникает ЭДС самоиндукции $200\,\text{мВ}$.
    Чему равна индуктивность проводника?
    Ответ выразите в миллигенри и округлите до целого.
}
\answer{%
    $
        \ele = L\frac{\abs{\Delta \eli}}{\Delta t} = L \cdot \abs{ - 1{,}5 } \text{(в СИ)}
        \implies L = \frac{\ele}{ 1{,}5 } = \frac{200\,\text{мВ}}{ 1{,}5 } \approx {133{,}3\,\text{мГн}}
    $
}
\solutionspace{80pt}

\tasknumber{6}%
\task{%
    Резистор сопротивлением $R = 5\,\text{Ом}$ и катушка индуктивностью $L = 0{,}5\,\text{Гн}$ (и пренебрежимо малым сопротивлением)
    подключены параллельно к источнику тока с ЭДС $\ele = 8\,\text{В}$ и внутренним сопротивлением $r = 1\,\text{Ом}$ (см.
    рис.
    на доске).
    Какое количество теплоты выделится в цепи после размыкания ключа $K$?
}
\answer{%
    \begin{align*}
    &\text{закон Ома для полной цепи}: \eli = \frac{\ele}{r + R_\text{внешнее}} = \frac{\ele}{r + \frac{R \cdot 0}{R + 0}} = \frac{\ele}{r}, \\
    Q &= W_m = \frac{L\eli^2}2 = \frac{L\sqr{\frac{\ele}{r}}}2 = \frac L2\frac{\ele^2}{r^2} = \frac{0{,}5\,\text{Гн}}2 \cdot \sqr{\frac{8\,\text{В}}{1\,\text{Ом}}} \approx 16\,\text{Дж}.
    \end{align*}
}

\variantsplitter

\addpersonalvariant{Анна Кузьмичёва}

\tasknumber{1}%
\task{%
    Положительно заряженная частица движется со скоростью $v$ в магнитном поле перпендикулярно линиям его индукции.
    Индукция магнитного поля равна $B$, масса частицы $m$, её заряд — $q$.
    Выведите из базовых физических законов формулы для радиуса траектории частицы и её частоты обращения.
}
\answer{%
    $
        F = ma, F = qvB, a = v^2 / R \implies R = \frac{mv}{qB}.
        \quad T = \frac{2\pi R}{v} = \frac{2\pi m}{qB}.
        \quad \omega = \frac vR = \frac{qB}{m}.
        \quad \nu = \frac 1T = \frac{qB}{2\pi m}.
    $
}
\solutionspace{80pt}

\tasknumber{2}%
\task{%
    В однородном горизонтальном магнитном поле с индукцией $B = 10\,\text{мТл}$ находится проводник,
    расположенный также горизонтально и перпендикулярно полю.
    Какой ток необходимо пустить по проводнику, чтобы он завис?
    Масса единицы длины проводника $\rho = 5\,\frac{\text{г}}{\text{м}}$, $g = 10\,\frac{\text{м}}{\text{с}^{2}}$.
}
\answer{%
    $
            mg = B\eli l, m=\rho l
            \implies \eli
                = \frac{g\rho}B
                = \frac{10\,\frac{\text{м}}{\text{с}^{2}} \cdot 5\,\frac{\text{г}}{\text{м}}}{10\,\text{мТл}}
                = 5\,\text{А}.
    $
}
\solutionspace{80pt}

\tasknumber{3}%
\task{%
    Протон, прошедший через ускоряющую разность потенциалов, оказывается в магнитном поле индукцией $20\,\text{мТл}$
    и движется по окружности диаметром $6\,\text{мм}$.
    Сделайте рисунок, определите значение разности потенциалов
    и укажите, в какой области потенциал больше, а где меньше.
}
\solutionspace{80pt}

\tasknumber{4}%
\task{%
    Проводник лежит на горизонтальных рельсах,
    замкнутых резистором сопротивлением $3\,\text{Ом}$ (см.
    рис.
    на доске).
    Расстояние между рельсами $60\,\text{см}$.
    Конструкция помещена в вертикальное однородное магнитное поле индукцией $300\,\text{мТл}$.
    Какую силу необходимо прикладывать к проводнику, чтобы двигать его вдоль рельс с постоянной скоростью $4\,\frac{\text{м}}{\text{c}}$?
    Трением пренебречь, сопротивления рельс и проводника малы по сравнению с сопротивлением резистора.
    Ответ выразите в миллиньютонах.
}
\answer{%
    $
        F
            = F_A
            = \eli B l
            = \frac{\ele}R \cdot B l
            = \frac{B v l}R \cdot B l
            = \frac{B^2 v l^2}R
            = \frac{\sqr{300\,\text{мТл}} \cdot 4\,\frac{\text{м}}{\text{c}} \cdot \sqr{60\,\text{см}}}{3\,\text{Ом}}
            \approx 43{,}20\,\text{мН}.
    $
}
\solutionspace{120pt}

\tasknumber{5}%
\task{%
    При изменении силы тока в проводнике по закону $\eli = 5 + 0{,}5t$ (в системе СИ),
    в нём возникает ЭДС самоиндукции $400\,\text{мВ}$.
    Чему равна индуктивность проводника?
    Ответ выразите в миллигенри и округлите до целого.
}
\answer{%
    $
        \ele = L\frac{\abs{\Delta \eli}}{\Delta t} = L \cdot \abs{ + 0{,}5 } \text{(в СИ)}
        \implies L = \frac{\ele}{ 0{,}5 } = \frac{400\,\text{мВ}}{ 0{,}5 } \approx {800\,\text{мГн}}
    $
}
\solutionspace{80pt}

\tasknumber{6}%
\task{%
    Резистор сопротивлением $R = 3\,\text{Ом}$ и катушка индуктивностью $L = 0{,}5\,\text{Гн}$ (и пренебрежимо малым сопротивлением)
    подключены параллельно к источнику тока с ЭДС $\ele = 12\,\text{В}$ и внутренним сопротивлением $r = 2\,\text{Ом}$ (см.
    рис.
    на доске).
    Какое количество теплоты выделится в цепи после размыкания ключа $K$?
}
\answer{%
    \begin{align*}
    &\text{закон Ома для полной цепи}: \eli = \frac{\ele}{r + R_\text{внешнее}} = \frac{\ele}{r + \frac{R \cdot 0}{R + 0}} = \frac{\ele}{r}, \\
    Q &= W_m = \frac{L\eli^2}2 = \frac{L\sqr{\frac{\ele}{r}}}2 = \frac L2\frac{\ele^2}{r^2} = \frac{0{,}5\,\text{Гн}}2 \cdot \sqr{\frac{12\,\text{В}}{2\,\text{Ом}}} \approx 9\,\text{Дж}.
    \end{align*}
}

\variantsplitter

\addpersonalvariant{Алёна Куприянова}

\tasknumber{1}%
\task{%
    Положительно заряженная частица движется со скоростью $v$ в магнитном поле перпендикулярно линиям его индукции.
    Индукция магнитного поля равна $B$, масса частицы $m$, её заряд — $q$.
    Выведите из базовых физических законов формулы для радиуса траектории частицы и её угловой скорости.
}
\answer{%
    $
        F = ma, F = qvB, a = v^2 / R \implies R = \frac{mv}{qB}.
        \quad T = \frac{2\pi R}{v} = \frac{2\pi m}{qB}.
        \quad \omega = \frac vR = \frac{qB}{m}.
        \quad \nu = \frac 1T = \frac{qB}{2\pi m}.
    $
}
\solutionspace{80pt}

\tasknumber{2}%
\task{%
    В однородном горизонтальном магнитном поле с индукцией $B = 10\,\text{мТл}$ находится проводник,
    расположенный также горизонтально и перпендикулярно полю.
    Какой ток необходимо пустить по проводнику, чтобы он завис?
    Масса единицы длины проводника $\rho = 20\,\frac{\text{г}}{\text{м}}$, $g = 10\,\frac{\text{м}}{\text{с}^{2}}$.
}
\answer{%
    $
            mg = B\eli l, m=\rho l
            \implies \eli
                = \frac{g\rho}B
                = \frac{10\,\frac{\text{м}}{\text{с}^{2}} \cdot 20\,\frac{\text{г}}{\text{м}}}{10\,\text{мТл}}
                = 20\,\text{А}.
    $
}
\solutionspace{80pt}

\tasknumber{3}%
\task{%
    Электрон, прошедший через ускоряющую разность потенциалов, оказывается в магнитном поле индукцией $50\,\text{мТл}$
    и движется по окружности диаметром $6\,\text{мм}$.
    Сделайте рисунок, определите значение разности потенциалов
    и укажите, в какой области потенциал больше, а где меньше.
}
\solutionspace{80pt}

\tasknumber{4}%
\task{%
    Проводник лежит на горизонтальных рельсах,
    замкнутых резистором сопротивлением $4\,\text{Ом}$ (см.
    рис.
    на доске).
    Расстояние между рельсами $60\,\text{см}$.
    Конструкция помещена в вертикальное однородное магнитное поле индукцией $300\,\text{мТл}$.
    Какую силу необходимо прикладывать к проводнику, чтобы двигать его вдоль рельс с постоянной скоростью $3\,\frac{\text{м}}{\text{c}}$?
    Трением пренебречь, сопротивления рельс и проводника малы по сравнению с сопротивлением резистора.
    Ответ выразите в миллиньютонах.
}
\answer{%
    $
        F
            = F_A
            = \eli B l
            = \frac{\ele}R \cdot B l
            = \frac{B v l}R \cdot B l
            = \frac{B^2 v l^2}R
            = \frac{\sqr{300\,\text{мТл}} \cdot 3\,\frac{\text{м}}{\text{c}} \cdot \sqr{60\,\text{см}}}{4\,\text{Ом}}
            \approx 24{,}30\,\text{мН}.
    $
}
\solutionspace{120pt}

\tasknumber{5}%
\task{%
    При изменении силы тока в проводнике по закону $\eli = 3 + 1{,}5t$ (в системе СИ),
    в нём возникает ЭДС самоиндукции $400\,\text{мВ}$.
    Чему равна индуктивность проводника?
    Ответ выразите в миллигенри и округлите до целого.
}
\answer{%
    $
        \ele = L\frac{\abs{\Delta \eli}}{\Delta t} = L \cdot \abs{ + 1{,}5 } \text{(в СИ)}
        \implies L = \frac{\ele}{ 1{,}5 } = \frac{400\,\text{мВ}}{ 1{,}5 } \approx {266{,}7\,\text{мГн}}
    $
}
\solutionspace{80pt}

\tasknumber{6}%
\task{%
    Резистор сопротивлением $R = 3\,\text{Ом}$ и катушка индуктивностью $L = 0{,}2\,\text{Гн}$ (и пренебрежимо малым сопротивлением)
    подключены параллельно к источнику тока с ЭДС $\ele = 8\,\text{В}$ и внутренним сопротивлением $r = 2\,\text{Ом}$ (см.
    рис.
    на доске).
    Какое количество теплоты выделится в цепи после размыкания ключа $K$?
}
\answer{%
    \begin{align*}
    &\text{закон Ома для полной цепи}: \eli = \frac{\ele}{r + R_\text{внешнее}} = \frac{\ele}{r + \frac{R \cdot 0}{R + 0}} = \frac{\ele}{r}, \\
    Q &= W_m = \frac{L\eli^2}2 = \frac{L\sqr{\frac{\ele}{r}}}2 = \frac L2\frac{\ele^2}{r^2} = \frac{0{,}2\,\text{Гн}}2 \cdot \sqr{\frac{8\,\text{В}}{2\,\text{Ом}}} \approx 1{,}60\,\text{Дж}.
    \end{align*}
}

\variantsplitter

\addpersonalvariant{Ярослав Лавровский}

\tasknumber{1}%
\task{%
    Положительно заряженная частица движется со скоростью $v$ в магнитном поле перпендикулярно линиям его индукции.
    Индукция магнитного поля равна $B$, масса частицы $m$, её заряд — $q$.
    Выведите из базовых физических законов формулы для радиуса траектории частицы и её частоты обращения.
}
\answer{%
    $
        F = ma, F = qvB, a = v^2 / R \implies R = \frac{mv}{qB}.
        \quad T = \frac{2\pi R}{v} = \frac{2\pi m}{qB}.
        \quad \omega = \frac vR = \frac{qB}{m}.
        \quad \nu = \frac 1T = \frac{qB}{2\pi m}.
    $
}
\solutionspace{80pt}

\tasknumber{2}%
\task{%
    В однородном горизонтальном магнитном поле с индукцией $B = 10\,\text{мТл}$ находится проводник,
    расположенный также горизонтально и перпендикулярно полю.
    Какой ток необходимо пустить по проводнику, чтобы он завис?
    Масса единицы длины проводника $\rho = 100\,\frac{\text{г}}{\text{м}}$, $g = 10\,\frac{\text{м}}{\text{с}^{2}}$.
}
\answer{%
    $
            mg = B\eli l, m=\rho l
            \implies \eli
                = \frac{g\rho}B
                = \frac{10\,\frac{\text{м}}{\text{с}^{2}} \cdot 100\,\frac{\text{г}}{\text{м}}}{10\,\text{мТл}}
                = 100\,\text{А}.
    $
}
\solutionspace{80pt}

\tasknumber{3}%
\task{%
    Позитрон, прошедший через ускоряющую разность потенциалов, оказывается в магнитном поле индукцией $50\,\text{мТл}$
    и движется по окружности диаметром $6\,\text{мм}$.
    Сделайте рисунок, определите значение разности потенциалов
    и укажите, в какой области потенциал больше, а где меньше.
}
\solutionspace{80pt}

\tasknumber{4}%
\task{%
    Проводник лежит на горизонтальных рельсах,
    замкнутых резистором сопротивлением $3\,\text{Ом}$ (см.
    рис.
    на доске).
    Расстояние между рельсами $60\,\text{см}$.
    Конструкция помещена в вертикальное однородное магнитное поле индукцией $150\,\text{мТл}$.
    Какую силу необходимо прикладывать к проводнику, чтобы двигать его вдоль рельс с постоянной скоростью $2\,\frac{\text{м}}{\text{c}}$?
    Трением пренебречь, сопротивления рельс и проводника малы по сравнению с сопротивлением резистора.
    Ответ выразите в миллиньютонах.
}
\answer{%
    $
        F
            = F_A
            = \eli B l
            = \frac{\ele}R \cdot B l
            = \frac{B v l}R \cdot B l
            = \frac{B^2 v l^2}R
            = \frac{\sqr{150\,\text{мТл}} \cdot 2\,\frac{\text{м}}{\text{c}} \cdot \sqr{60\,\text{см}}}{3\,\text{Ом}}
            \approx 5{,}40\,\text{мН}.
    $
}
\solutionspace{120pt}

\tasknumber{5}%
\task{%
    При изменении силы тока в проводнике по закону $\eli = 4 - 0{,}4t$ (в системе СИ),
    в нём возникает ЭДС самоиндукции $150\,\text{мВ}$.
    Чему равна индуктивность проводника?
    Ответ выразите в миллигенри и округлите до целого.
}
\answer{%
    $
        \ele = L\frac{\abs{\Delta \eli}}{\Delta t} = L \cdot \abs{ - 0{,}4 } \text{(в СИ)}
        \implies L = \frac{\ele}{ 0{,}4 } = \frac{150\,\text{мВ}}{ 0{,}4 } \approx {375\,\text{мГн}}
    $
}
\solutionspace{80pt}

\tasknumber{6}%
\task{%
    Резистор сопротивлением $R = 5\,\text{Ом}$ и катушка индуктивностью $L = 0{,}5\,\text{Гн}$ (и пренебрежимо малым сопротивлением)
    подключены параллельно к источнику тока с ЭДС $\ele = 12\,\text{В}$ и внутренним сопротивлением $r = 1\,\text{Ом}$ (см.
    рис.
    на доске).
    Какое количество теплоты выделится в цепи после размыкания ключа $K$?
}
\answer{%
    \begin{align*}
    &\text{закон Ома для полной цепи}: \eli = \frac{\ele}{r + R_\text{внешнее}} = \frac{\ele}{r + \frac{R \cdot 0}{R + 0}} = \frac{\ele}{r}, \\
    Q &= W_m = \frac{L\eli^2}2 = \frac{L\sqr{\frac{\ele}{r}}}2 = \frac L2\frac{\ele^2}{r^2} = \frac{0{,}5\,\text{Гн}}2 \cdot \sqr{\frac{12\,\text{В}}{1\,\text{Ом}}} \approx 36\,\text{Дж}.
    \end{align*}
}

\variantsplitter

\addpersonalvariant{Анастасия Ламанова}

\tasknumber{1}%
\task{%
    Положительно заряженная частица движется со скоростью $v$ в магнитном поле перпендикулярно линиям его индукции.
    Индукция магнитного поля равна $B$, масса частицы $m$, её заряд — $q$.
    Выведите из базовых физических законов формулы для радиуса траектории частицы и её частоты обращения.
}
\answer{%
    $
        F = ma, F = qvB, a = v^2 / R \implies R = \frac{mv}{qB}.
        \quad T = \frac{2\pi R}{v} = \frac{2\pi m}{qB}.
        \quad \omega = \frac vR = \frac{qB}{m}.
        \quad \nu = \frac 1T = \frac{qB}{2\pi m}.
    $
}
\solutionspace{80pt}

\tasknumber{2}%
\task{%
    В однородном горизонтальном магнитном поле с индукцией $B = 10\,\text{мТл}$ находится проводник,
    расположенный также горизонтально и перпендикулярно полю.
    Какой ток необходимо пустить по проводнику, чтобы он завис?
    Масса единицы длины проводника $\rho = 20\,\frac{\text{г}}{\text{м}}$, $g = 10\,\frac{\text{м}}{\text{с}^{2}}$.
}
\answer{%
    $
            mg = B\eli l, m=\rho l
            \implies \eli
                = \frac{g\rho}B
                = \frac{10\,\frac{\text{м}}{\text{с}^{2}} \cdot 20\,\frac{\text{г}}{\text{м}}}{10\,\text{мТл}}
                = 20\,\text{А}.
    $
}
\solutionspace{80pt}

\tasknumber{3}%
\task{%
    Позитрон, прошедший через ускоряющую разность потенциалов, оказывается в магнитном поле индукцией $20\,\text{мТл}$
    и движется по окружности диаметром $6\,\text{мм}$.
    Сделайте рисунок, определите значение разности потенциалов
    и укажите, в какой области потенциал больше, а где меньше.
}
\solutionspace{80pt}

\tasknumber{4}%
\task{%
    Проводник лежит на горизонтальных рельсах,
    замкнутых резистором сопротивлением $3\,\text{Ом}$ (см.
    рис.
    на доске).
    Расстояние между рельсами $80\,\text{см}$.
    Конструкция помещена в вертикальное однородное магнитное поле индукцией $200\,\text{мТл}$.
    Какую силу необходимо прикладывать к проводнику, чтобы двигать его вдоль рельс с постоянной скоростью $5\,\frac{\text{м}}{\text{c}}$?
    Трением пренебречь, сопротивления рельс и проводника малы по сравнению с сопротивлением резистора.
    Ответ выразите в миллиньютонах.
}
\answer{%
    $
        F
            = F_A
            = \eli B l
            = \frac{\ele}R \cdot B l
            = \frac{B v l}R \cdot B l
            = \frac{B^2 v l^2}R
            = \frac{\sqr{200\,\text{мТл}} \cdot 5\,\frac{\text{м}}{\text{c}} \cdot \sqr{80\,\text{см}}}{3\,\text{Ом}}
            \approx 42{,}67\,\text{мН}.
    $
}
\solutionspace{120pt}

\tasknumber{5}%
\task{%
    При изменении силы тока в проводнике по закону $\eli = 7 + 0{,}5t$ (в системе СИ),
    в нём возникает ЭДС самоиндукции $150\,\text{мВ}$.
    Чему равна индуктивность проводника?
    Ответ выразите в миллигенри и округлите до целого.
}
\answer{%
    $
        \ele = L\frac{\abs{\Delta \eli}}{\Delta t} = L \cdot \abs{ + 0{,}5 } \text{(в СИ)}
        \implies L = \frac{\ele}{ 0{,}5 } = \frac{150\,\text{мВ}}{ 0{,}5 } \approx {300\,\text{мГн}}
    $
}
\solutionspace{80pt}

\tasknumber{6}%
\task{%
    Резистор сопротивлением $R = 5\,\text{Ом}$ и катушка индуктивностью $L = 0{,}2\,\text{Гн}$ (и пренебрежимо малым сопротивлением)
    подключены параллельно к источнику тока с ЭДС $\ele = 6\,\text{В}$ и внутренним сопротивлением $r = 2\,\text{Ом}$ (см.
    рис.
    на доске).
    Какое количество теплоты выделится в цепи после размыкания ключа $K$?
}
\answer{%
    \begin{align*}
    &\text{закон Ома для полной цепи}: \eli = \frac{\ele}{r + R_\text{внешнее}} = \frac{\ele}{r + \frac{R \cdot 0}{R + 0}} = \frac{\ele}{r}, \\
    Q &= W_m = \frac{L\eli^2}2 = \frac{L\sqr{\frac{\ele}{r}}}2 = \frac L2\frac{\ele^2}{r^2} = \frac{0{,}2\,\text{Гн}}2 \cdot \sqr{\frac{6\,\text{В}}{2\,\text{Ом}}} \approx 0{,}90\,\text{Дж}.
    \end{align*}
}

\variantsplitter

\addpersonalvariant{Виктория Легонькова}

\tasknumber{1}%
\task{%
    Положительно заряженная частица движется со скоростью $v$ в магнитном поле перпендикулярно линиям его индукции.
    Индукция магнитного поля равна $B$, масса частицы $m$, её заряд — $q$.
    Выведите из базовых физических законов формулы для радиуса траектории частицы и её периода обращения.
}
\answer{%
    $
        F = ma, F = qvB, a = v^2 / R \implies R = \frac{mv}{qB}.
        \quad T = \frac{2\pi R}{v} = \frac{2\pi m}{qB}.
        \quad \omega = \frac vR = \frac{qB}{m}.
        \quad \nu = \frac 1T = \frac{qB}{2\pi m}.
    $
}
\solutionspace{80pt}

\tasknumber{2}%
\task{%
    В однородном горизонтальном магнитном поле с индукцией $B = 20\,\text{мТл}$ находится проводник,
    расположенный также горизонтально и перпендикулярно полю.
    Какой ток необходимо пустить по проводнику, чтобы он завис?
    Масса единицы длины проводника $\rho = 100\,\frac{\text{г}}{\text{м}}$, $g = 10\,\frac{\text{м}}{\text{с}^{2}}$.
}
\answer{%
    $
            mg = B\eli l, m=\rho l
            \implies \eli
                = \frac{g\rho}B
                = \frac{10\,\frac{\text{м}}{\text{с}^{2}} \cdot 100\,\frac{\text{г}}{\text{м}}}{20\,\text{мТл}}
                = 50\,\text{А}.
    $
}
\solutionspace{80pt}

\tasknumber{3}%
\task{%
    Позитрон, прошедший через ускоряющую разность потенциалов, оказывается в магнитном поле индукцией $40\,\text{мТл}$
    и движется по окружности диаметром $6\,\text{мм}$.
    Сделайте рисунок, определите значение разности потенциалов
    и укажите, в какой области потенциал больше, а где меньше.
}
\solutionspace{80pt}

\tasknumber{4}%
\task{%
    Проводник лежит на горизонтальных рельсах,
    замкнутых резистором сопротивлением $3\,\text{Ом}$ (см.
    рис.
    на доске).
    Расстояние между рельсами $60\,\text{см}$.
    Конструкция помещена в вертикальное однородное магнитное поле индукцией $200\,\text{мТл}$.
    Какую силу необходимо прикладывать к проводнику, чтобы двигать его вдоль рельс с постоянной скоростью $4\,\frac{\text{м}}{\text{c}}$?
    Трением пренебречь, сопротивления рельс и проводника малы по сравнению с сопротивлением резистора.
    Ответ выразите в миллиньютонах.
}
\answer{%
    $
        F
            = F_A
            = \eli B l
            = \frac{\ele}R \cdot B l
            = \frac{B v l}R \cdot B l
            = \frac{B^2 v l^2}R
            = \frac{\sqr{200\,\text{мТл}} \cdot 4\,\frac{\text{м}}{\text{c}} \cdot \sqr{60\,\text{см}}}{3\,\text{Ом}}
            \approx 19{,}20\,\text{мН}.
    $
}
\solutionspace{120pt}

\tasknumber{5}%
\task{%
    При изменении силы тока в проводнике по закону $\eli = 4 - 0{,}4t$ (в системе СИ),
    в нём возникает ЭДС самоиндукции $400\,\text{мВ}$.
    Чему равна индуктивность проводника?
    Ответ выразите в миллигенри и округлите до целого.
}
\answer{%
    $
        \ele = L\frac{\abs{\Delta \eli}}{\Delta t} = L \cdot \abs{ - 0{,}4 } \text{(в СИ)}
        \implies L = \frac{\ele}{ 0{,}4 } = \frac{400\,\text{мВ}}{ 0{,}4 } \approx {1000\,\text{мГн}}
    $
}
\solutionspace{80pt}

\tasknumber{6}%
\task{%
    Резистор сопротивлением $R = 4\,\text{Ом}$ и катушка индуктивностью $L = 0{,}2\,\text{Гн}$ (и пренебрежимо малым сопротивлением)
    подключены параллельно к источнику тока с ЭДС $\ele = 6\,\text{В}$ и внутренним сопротивлением $r = 1\,\text{Ом}$ (см.
    рис.
    на доске).
    Какое количество теплоты выделится в цепи после размыкания ключа $K$?
}
\answer{%
    \begin{align*}
    &\text{закон Ома для полной цепи}: \eli = \frac{\ele}{r + R_\text{внешнее}} = \frac{\ele}{r + \frac{R \cdot 0}{R + 0}} = \frac{\ele}{r}, \\
    Q &= W_m = \frac{L\eli^2}2 = \frac{L\sqr{\frac{\ele}{r}}}2 = \frac L2\frac{\ele^2}{r^2} = \frac{0{,}2\,\text{Гн}}2 \cdot \sqr{\frac{6\,\text{В}}{1\,\text{Ом}}} \approx 3{,}60\,\text{Дж}.
    \end{align*}
}

\variantsplitter

\addpersonalvariant{Семён Мартынов}

\tasknumber{1}%
\task{%
    Положительно заряженная частица движется со скоростью $v$ в магнитном поле перпендикулярно линиям его индукции.
    Индукция магнитного поля равна $B$, масса частицы $m$, её заряд — $q$.
    Выведите из базовых физических законов формулы для радиуса траектории частицы и её частоты обращения.
}
\answer{%
    $
        F = ma, F = qvB, a = v^2 / R \implies R = \frac{mv}{qB}.
        \quad T = \frac{2\pi R}{v} = \frac{2\pi m}{qB}.
        \quad \omega = \frac vR = \frac{qB}{m}.
        \quad \nu = \frac 1T = \frac{qB}{2\pi m}.
    $
}
\solutionspace{80pt}

\tasknumber{2}%
\task{%
    В однородном горизонтальном магнитном поле с индукцией $B = 20\,\text{мТл}$ находится проводник,
    расположенный также горизонтально и перпендикулярно полю.
    Какой ток необходимо пустить по проводнику, чтобы он завис?
    Масса единицы длины проводника $\rho = 40\,\frac{\text{г}}{\text{м}}$, $g = 10\,\frac{\text{м}}{\text{с}^{2}}$.
}
\answer{%
    $
            mg = B\eli l, m=\rho l
            \implies \eli
                = \frac{g\rho}B
                = \frac{10\,\frac{\text{м}}{\text{с}^{2}} \cdot 40\,\frac{\text{г}}{\text{м}}}{20\,\text{мТл}}
                = 20\,\text{А}.
    $
}
\solutionspace{80pt}

\tasknumber{3}%
\task{%
    Электрон, прошедший через ускоряющую разность потенциалов, оказывается в магнитном поле индукцией $50\,\text{мТл}$
    и движется по окружности диаметром $4\,\text{мм}$.
    Сделайте рисунок, определите значение разности потенциалов
    и укажите, в какой области потенциал больше, а где меньше.
}
\solutionspace{80pt}

\tasknumber{4}%
\task{%
    Проводник лежит на горизонтальных рельсах,
    замкнутых резистором сопротивлением $4\,\text{Ом}$ (см.
    рис.
    на доске).
    Расстояние между рельсами $60\,\text{см}$.
    Конструкция помещена в вертикальное однородное магнитное поле индукцией $400\,\text{мТл}$.
    Какую силу необходимо прикладывать к проводнику, чтобы двигать его вдоль рельс с постоянной скоростью $4\,\frac{\text{м}}{\text{c}}$?
    Трением пренебречь, сопротивления рельс и проводника малы по сравнению с сопротивлением резистора.
    Ответ выразите в миллиньютонах.
}
\answer{%
    $
        F
            = F_A
            = \eli B l
            = \frac{\ele}R \cdot B l
            = \frac{B v l}R \cdot B l
            = \frac{B^2 v l^2}R
            = \frac{\sqr{400\,\text{мТл}} \cdot 4\,\frac{\text{м}}{\text{c}} \cdot \sqr{60\,\text{см}}}{4\,\text{Ом}}
            \approx 57{,}60\,\text{мН}.
    $
}
\solutionspace{120pt}

\tasknumber{5}%
\task{%
    При изменении силы тока в проводнике по закону $\eli = 2 - 0{,}8t$ (в системе СИ),
    в нём возникает ЭДС самоиндукции $400\,\text{мВ}$.
    Чему равна индуктивность проводника?
    Ответ выразите в миллигенри и округлите до целого.
}
\answer{%
    $
        \ele = L\frac{\abs{\Delta \eli}}{\Delta t} = L \cdot \abs{ - 0{,}8 } \text{(в СИ)}
        \implies L = \frac{\ele}{ 0{,}8 } = \frac{400\,\text{мВ}}{ 0{,}8 } \approx {500\,\text{мГн}}
    $
}
\solutionspace{80pt}

\tasknumber{6}%
\task{%
    Резистор сопротивлением $R = 5\,\text{Ом}$ и катушка индуктивностью $L = 0{,}5\,\text{Гн}$ (и пренебрежимо малым сопротивлением)
    подключены параллельно к источнику тока с ЭДС $\ele = 8\,\text{В}$ и внутренним сопротивлением $r = 2\,\text{Ом}$ (см.
    рис.
    на доске).
    Какое количество теплоты выделится в цепи после размыкания ключа $K$?
}
\answer{%
    \begin{align*}
    &\text{закон Ома для полной цепи}: \eli = \frac{\ele}{r + R_\text{внешнее}} = \frac{\ele}{r + \frac{R \cdot 0}{R + 0}} = \frac{\ele}{r}, \\
    Q &= W_m = \frac{L\eli^2}2 = \frac{L\sqr{\frac{\ele}{r}}}2 = \frac L2\frac{\ele^2}{r^2} = \frac{0{,}5\,\text{Гн}}2 \cdot \sqr{\frac{8\,\text{В}}{2\,\text{Ом}}} \approx 4\,\text{Дж}.
    \end{align*}
}

\variantsplitter

\addpersonalvariant{Варвара Минаева}

\tasknumber{1}%
\task{%
    Положительно заряженная частица движется со скоростью $v$ в магнитном поле перпендикулярно линиям его индукции.
    Индукция магнитного поля равна $B$, масса частицы $m$, её заряд — $q$.
    Выведите из базовых физических законов формулы для радиуса траектории частицы и её частоты обращения.
}
\answer{%
    $
        F = ma, F = qvB, a = v^2 / R \implies R = \frac{mv}{qB}.
        \quad T = \frac{2\pi R}{v} = \frac{2\pi m}{qB}.
        \quad \omega = \frac vR = \frac{qB}{m}.
        \quad \nu = \frac 1T = \frac{qB}{2\pi m}.
    $
}
\solutionspace{80pt}

\tasknumber{2}%
\task{%
    В однородном горизонтальном магнитном поле с индукцией $B = 100\,\text{мТл}$ находится проводник,
    расположенный также горизонтально и перпендикулярно полю.
    Какой ток необходимо пустить по проводнику, чтобы он завис?
    Масса единицы длины проводника $\rho = 100\,\frac{\text{г}}{\text{м}}$, $g = 10\,\frac{\text{м}}{\text{с}^{2}}$.
}
\answer{%
    $
            mg = B\eli l, m=\rho l
            \implies \eli
                = \frac{g\rho}B
                = \frac{10\,\frac{\text{м}}{\text{с}^{2}} \cdot 100\,\frac{\text{г}}{\text{м}}}{100\,\text{мТл}}
                = 10\,\text{А}.
    $
}
\solutionspace{80pt}

\tasknumber{3}%
\task{%
    Электрон, прошедший через ускоряющую разность потенциалов, оказывается в магнитном поле индукцией $40\,\text{мТл}$
    и движется по окружности диаметром $4\,\text{мм}$.
    Сделайте рисунок, определите значение разности потенциалов
    и укажите, в какой области потенциал больше, а где меньше.
}
\solutionspace{80pt}

\tasknumber{4}%
\task{%
    Проводник лежит на горизонтальных рельсах,
    замкнутых резистором сопротивлением $3\,\text{Ом}$ (см.
    рис.
    на доске).
    Расстояние между рельсами $80\,\text{см}$.
    Конструкция помещена в вертикальное однородное магнитное поле индукцией $150\,\text{мТл}$.
    Какую силу необходимо прикладывать к проводнику, чтобы двигать его вдоль рельс с постоянной скоростью $5\,\frac{\text{м}}{\text{c}}$?
    Трением пренебречь, сопротивления рельс и проводника малы по сравнению с сопротивлением резистора.
    Ответ выразите в миллиньютонах.
}
\answer{%
    $
        F
            = F_A
            = \eli B l
            = \frac{\ele}R \cdot B l
            = \frac{B v l}R \cdot B l
            = \frac{B^2 v l^2}R
            = \frac{\sqr{150\,\text{мТл}} \cdot 5\,\frac{\text{м}}{\text{c}} \cdot \sqr{80\,\text{см}}}{3\,\text{Ом}}
            \approx 24\,\text{мН}.
    $
}
\solutionspace{120pt}

\tasknumber{5}%
\task{%
    При изменении силы тока в проводнике по закону $\eli = 5 + 0{,}5t$ (в системе СИ),
    в нём возникает ЭДС самоиндукции $400\,\text{мВ}$.
    Чему равна индуктивность проводника?
    Ответ выразите в миллигенри и округлите до целого.
}
\answer{%
    $
        \ele = L\frac{\abs{\Delta \eli}}{\Delta t} = L \cdot \abs{ + 0{,}5 } \text{(в СИ)}
        \implies L = \frac{\ele}{ 0{,}5 } = \frac{400\,\text{мВ}}{ 0{,}5 } \approx {800\,\text{мГн}}
    $
}
\solutionspace{80pt}

\tasknumber{6}%
\task{%
    Резистор сопротивлением $R = 4\,\text{Ом}$ и катушка индуктивностью $L = 0{,}2\,\text{Гн}$ (и пренебрежимо малым сопротивлением)
    подключены параллельно к источнику тока с ЭДС $\ele = 5\,\text{В}$ и внутренним сопротивлением $r = 1\,\text{Ом}$ (см.
    рис.
    на доске).
    Какое количество теплоты выделится в цепи после размыкания ключа $K$?
}
\answer{%
    \begin{align*}
    &\text{закон Ома для полной цепи}: \eli = \frac{\ele}{r + R_\text{внешнее}} = \frac{\ele}{r + \frac{R \cdot 0}{R + 0}} = \frac{\ele}{r}, \\
    Q &= W_m = \frac{L\eli^2}2 = \frac{L\sqr{\frac{\ele}{r}}}2 = \frac L2\frac{\ele^2}{r^2} = \frac{0{,}2\,\text{Гн}}2 \cdot \sqr{\frac{5\,\text{В}}{1\,\text{Ом}}} \approx 2{,}50\,\text{Дж}.
    \end{align*}
}

\variantsplitter

\addpersonalvariant{Леонид Никитин}

\tasknumber{1}%
\task{%
    Положительно заряженная частица движется со скоростью $v$ в магнитном поле перпендикулярно линиям его индукции.
    Индукция магнитного поля равна $B$, масса частицы $m$, её заряд — $q$.
    Выведите из базовых физических законов формулы для радиуса траектории частицы и её частоты обращения.
}
\answer{%
    $
        F = ma, F = qvB, a = v^2 / R \implies R = \frac{mv}{qB}.
        \quad T = \frac{2\pi R}{v} = \frac{2\pi m}{qB}.
        \quad \omega = \frac vR = \frac{qB}{m}.
        \quad \nu = \frac 1T = \frac{qB}{2\pi m}.
    $
}
\solutionspace{80pt}

\tasknumber{2}%
\task{%
    В однородном горизонтальном магнитном поле с индукцией $B = 10\,\text{мТл}$ находится проводник,
    расположенный также горизонтально и перпендикулярно полю.
    Какой ток необходимо пустить по проводнику, чтобы он завис?
    Масса единицы длины проводника $\rho = 10\,\frac{\text{г}}{\text{м}}$, $g = 10\,\frac{\text{м}}{\text{с}^{2}}$.
}
\answer{%
    $
            mg = B\eli l, m=\rho l
            \implies \eli
                = \frac{g\rho}B
                = \frac{10\,\frac{\text{м}}{\text{с}^{2}} \cdot 10\,\frac{\text{г}}{\text{м}}}{10\,\text{мТл}}
                = 10\,\text{А}.
    $
}
\solutionspace{80pt}

\tasknumber{3}%
\task{%
    Электрон, прошедший через ускоряющую разность потенциалов, оказывается в магнитном поле индукцией $50\,\text{мТл}$
    и движется по окружности диаметром $6\,\text{мм}$.
    Сделайте рисунок, определите значение разности потенциалов
    и укажите, в какой области потенциал больше, а где меньше.
}
\solutionspace{80pt}

\tasknumber{4}%
\task{%
    Проводник лежит на горизонтальных рельсах,
    замкнутых резистором сопротивлением $4\,\text{Ом}$ (см.
    рис.
    на доске).
    Расстояние между рельсами $80\,\text{см}$.
    Конструкция помещена в вертикальное однородное магнитное поле индукцией $300\,\text{мТл}$.
    Какую силу необходимо прикладывать к проводнику, чтобы двигать его вдоль рельс с постоянной скоростью $5\,\frac{\text{м}}{\text{c}}$?
    Трением пренебречь, сопротивления рельс и проводника малы по сравнению с сопротивлением резистора.
    Ответ выразите в миллиньютонах.
}
\answer{%
    $
        F
            = F_A
            = \eli B l
            = \frac{\ele}R \cdot B l
            = \frac{B v l}R \cdot B l
            = \frac{B^2 v l^2}R
            = \frac{\sqr{300\,\text{мТл}} \cdot 5\,\frac{\text{м}}{\text{c}} \cdot \sqr{80\,\text{см}}}{4\,\text{Ом}}
            \approx 72\,\text{мН}.
    $
}
\solutionspace{120pt}

\tasknumber{5}%
\task{%
    При изменении силы тока в проводнике по закону $\eli = 2 + 0{,}4t$ (в системе СИ),
    в нём возникает ЭДС самоиндукции $200\,\text{мВ}$.
    Чему равна индуктивность проводника?
    Ответ выразите в миллигенри и округлите до целого.
}
\answer{%
    $
        \ele = L\frac{\abs{\Delta \eli}}{\Delta t} = L \cdot \abs{ + 0{,}4 } \text{(в СИ)}
        \implies L = \frac{\ele}{ 0{,}4 } = \frac{200\,\text{мВ}}{ 0{,}4 } \approx {500\,\text{мГн}}
    $
}
\solutionspace{80pt}

\tasknumber{6}%
\task{%
    Резистор сопротивлением $R = 5\,\text{Ом}$ и катушка индуктивностью $L = 0{,}2\,\text{Гн}$ (и пренебрежимо малым сопротивлением)
    подключены параллельно к источнику тока с ЭДС $\ele = 12\,\text{В}$ и внутренним сопротивлением $r = 1\,\text{Ом}$ (см.
    рис.
    на доске).
    Какое количество теплоты выделится в цепи после размыкания ключа $K$?
}
\answer{%
    \begin{align*}
    &\text{закон Ома для полной цепи}: \eli = \frac{\ele}{r + R_\text{внешнее}} = \frac{\ele}{r + \frac{R \cdot 0}{R + 0}} = \frac{\ele}{r}, \\
    Q &= W_m = \frac{L\eli^2}2 = \frac{L\sqr{\frac{\ele}{r}}}2 = \frac L2\frac{\ele^2}{r^2} = \frac{0{,}2\,\text{Гн}}2 \cdot \sqr{\frac{12\,\text{В}}{1\,\text{Ом}}} \approx 14{,}40\,\text{Дж}.
    \end{align*}
}

\variantsplitter

\addpersonalvariant{Тимофей Полетаев}

\tasknumber{1}%
\task{%
    Положительно заряженная частица движется со скоростью $v$ в магнитном поле перпендикулярно линиям его индукции.
    Индукция магнитного поля равна $B$, масса частицы $m$, её заряд — $q$.
    Выведите из базовых физических законов формулы для радиуса траектории частицы и её угловой скорости.
}
\answer{%
    $
        F = ma, F = qvB, a = v^2 / R \implies R = \frac{mv}{qB}.
        \quad T = \frac{2\pi R}{v} = \frac{2\pi m}{qB}.
        \quad \omega = \frac vR = \frac{qB}{m}.
        \quad \nu = \frac 1T = \frac{qB}{2\pi m}.
    $
}
\solutionspace{80pt}

\tasknumber{2}%
\task{%
    В однородном горизонтальном магнитном поле с индукцией $B = 20\,\text{мТл}$ находится проводник,
    расположенный также горизонтально и перпендикулярно полю.
    Какой ток необходимо пустить по проводнику, чтобы он завис?
    Масса единицы длины проводника $\rho = 5\,\frac{\text{г}}{\text{м}}$, $g = 10\,\frac{\text{м}}{\text{с}^{2}}$.
}
\answer{%
    $
            mg = B\eli l, m=\rho l
            \implies \eli
                = \frac{g\rho}B
                = \frac{10\,\frac{\text{м}}{\text{с}^{2}} \cdot 5\,\frac{\text{г}}{\text{м}}}{20\,\text{мТл}}
                = 2{,}5\,\text{А}.
    $
}
\solutionspace{80pt}

\tasknumber{3}%
\task{%
    Электрон, прошедший через ускоряющую разность потенциалов, оказывается в магнитном поле индукцией $50\,\text{мТл}$
    и движется по окружности диаметром $4\,\text{мм}$.
    Сделайте рисунок, определите значение разности потенциалов
    и укажите, в какой области потенциал больше, а где меньше.
}
\solutionspace{80pt}

\tasknumber{4}%
\task{%
    Проводник лежит на горизонтальных рельсах,
    замкнутых резистором сопротивлением $4\,\text{Ом}$ (см.
    рис.
    на доске).
    Расстояние между рельсами $70\,\text{см}$.
    Конструкция помещена в вертикальное однородное магнитное поле индукцией $200\,\text{мТл}$.
    Какую силу необходимо прикладывать к проводнику, чтобы двигать его вдоль рельс с постоянной скоростью $4\,\frac{\text{м}}{\text{c}}$?
    Трением пренебречь, сопротивления рельс и проводника малы по сравнению с сопротивлением резистора.
    Ответ выразите в миллиньютонах.
}
\answer{%
    $
        F
            = F_A
            = \eli B l
            = \frac{\ele}R \cdot B l
            = \frac{B v l}R \cdot B l
            = \frac{B^2 v l^2}R
            = \frac{\sqr{200\,\text{мТл}} \cdot 4\,\frac{\text{м}}{\text{c}} \cdot \sqr{70\,\text{см}}}{4\,\text{Ом}}
            \approx 19{,}60\,\text{мН}.
    $
}
\solutionspace{120pt}

\tasknumber{5}%
\task{%
    При изменении силы тока в проводнике по закону $\eli = 7 + 0{,}8t$ (в системе СИ),
    в нём возникает ЭДС самоиндукции $200\,\text{мВ}$.
    Чему равна индуктивность проводника?
    Ответ выразите в миллигенри и округлите до целого.
}
\answer{%
    $
        \ele = L\frac{\abs{\Delta \eli}}{\Delta t} = L \cdot \abs{ + 0{,}8 } \text{(в СИ)}
        \implies L = \frac{\ele}{ 0{,}8 } = \frac{200\,\text{мВ}}{ 0{,}8 } \approx {250\,\text{мГн}}
    $
}
\solutionspace{80pt}

\tasknumber{6}%
\task{%
    Резистор сопротивлением $R = 5\,\text{Ом}$ и катушка индуктивностью $L = 0{,}4\,\text{Гн}$ (и пренебрежимо малым сопротивлением)
    подключены параллельно к источнику тока с ЭДС $\ele = 6\,\text{В}$ и внутренним сопротивлением $r = 1\,\text{Ом}$ (см.
    рис.
    на доске).
    Какое количество теплоты выделится в цепи после размыкания ключа $K$?
}
\answer{%
    \begin{align*}
    &\text{закон Ома для полной цепи}: \eli = \frac{\ele}{r + R_\text{внешнее}} = \frac{\ele}{r + \frac{R \cdot 0}{R + 0}} = \frac{\ele}{r}, \\
    Q &= W_m = \frac{L\eli^2}2 = \frac{L\sqr{\frac{\ele}{r}}}2 = \frac L2\frac{\ele^2}{r^2} = \frac{0{,}4\,\text{Гн}}2 \cdot \sqr{\frac{6\,\text{В}}{1\,\text{Ом}}} \approx 7{,}20\,\text{Дж}.
    \end{align*}
}

\variantsplitter

\addpersonalvariant{Андрей Рожков}

\tasknumber{1}%
\task{%
    Положительно заряженная частица движется со скоростью $v$ в магнитном поле перпендикулярно линиям его индукции.
    Индукция магнитного поля равна $B$, масса частицы $m$, её заряд — $q$.
    Выведите из базовых физических законов формулы для радиуса траектории частицы и её угловой скорости.
}
\answer{%
    $
        F = ma, F = qvB, a = v^2 / R \implies R = \frac{mv}{qB}.
        \quad T = \frac{2\pi R}{v} = \frac{2\pi m}{qB}.
        \quad \omega = \frac vR = \frac{qB}{m}.
        \quad \nu = \frac 1T = \frac{qB}{2\pi m}.
    $
}
\solutionspace{80pt}

\tasknumber{2}%
\task{%
    В однородном горизонтальном магнитном поле с индукцией $B = 100\,\text{мТл}$ находится проводник,
    расположенный также горизонтально и перпендикулярно полю.
    Какой ток необходимо пустить по проводнику, чтобы он завис?
    Масса единицы длины проводника $\rho = 10\,\frac{\text{г}}{\text{м}}$, $g = 10\,\frac{\text{м}}{\text{с}^{2}}$.
}
\answer{%
    $
            mg = B\eli l, m=\rho l
            \implies \eli
                = \frac{g\rho}B
                = \frac{10\,\frac{\text{м}}{\text{с}^{2}} \cdot 10\,\frac{\text{г}}{\text{м}}}{100\,\text{мТл}}
                = 1\,\text{А}.
    $
}
\solutionspace{80pt}

\tasknumber{3}%
\task{%
    Протон, прошедший через ускоряющую разность потенциалов, оказывается в магнитном поле индукцией $20\,\text{мТл}$
    и движется по окружности диаметром $4\,\text{мм}$.
    Сделайте рисунок, определите значение разности потенциалов
    и укажите, в какой области потенциал больше, а где меньше.
}
\solutionspace{80pt}

\tasknumber{4}%
\task{%
    Проводник лежит на горизонтальных рельсах,
    замкнутых резистором сопротивлением $4\,\text{Ом}$ (см.
    рис.
    на доске).
    Расстояние между рельсами $70\,\text{см}$.
    Конструкция помещена в вертикальное однородное магнитное поле индукцией $200\,\text{мТл}$.
    Какую силу необходимо прикладывать к проводнику, чтобы двигать его вдоль рельс с постоянной скоростью $2\,\frac{\text{м}}{\text{c}}$?
    Трением пренебречь, сопротивления рельс и проводника малы по сравнению с сопротивлением резистора.
    Ответ выразите в миллиньютонах.
}
\answer{%
    $
        F
            = F_A
            = \eli B l
            = \frac{\ele}R \cdot B l
            = \frac{B v l}R \cdot B l
            = \frac{B^2 v l^2}R
            = \frac{\sqr{200\,\text{мТл}} \cdot 2\,\frac{\text{м}}{\text{c}} \cdot \sqr{70\,\text{см}}}{4\,\text{Ом}}
            \approx 9{,}80\,\text{мН}.
    $
}
\solutionspace{120pt}

\tasknumber{5}%
\task{%
    При изменении силы тока в проводнике по закону $\eli = 7 + 0{,}4t$ (в системе СИ),
    в нём возникает ЭДС самоиндукции $300\,\text{мВ}$.
    Чему равна индуктивность проводника?
    Ответ выразите в миллигенри и округлите до целого.
}
\answer{%
    $
        \ele = L\frac{\abs{\Delta \eli}}{\Delta t} = L \cdot \abs{ + 0{,}4 } \text{(в СИ)}
        \implies L = \frac{\ele}{ 0{,}4 } = \frac{300\,\text{мВ}}{ 0{,}4 } \approx {750\,\text{мГн}}
    $
}
\solutionspace{80pt}

\tasknumber{6}%
\task{%
    Резистор сопротивлением $R = 3\,\text{Ом}$ и катушка индуктивностью $L = 0{,}5\,\text{Гн}$ (и пренебрежимо малым сопротивлением)
    подключены параллельно к источнику тока с ЭДС $\ele = 8\,\text{В}$ и внутренним сопротивлением $r = 2\,\text{Ом}$ (см.
    рис.
    на доске).
    Какое количество теплоты выделится в цепи после размыкания ключа $K$?
}
\answer{%
    \begin{align*}
    &\text{закон Ома для полной цепи}: \eli = \frac{\ele}{r + R_\text{внешнее}} = \frac{\ele}{r + \frac{R \cdot 0}{R + 0}} = \frac{\ele}{r}, \\
    Q &= W_m = \frac{L\eli^2}2 = \frac{L\sqr{\frac{\ele}{r}}}2 = \frac L2\frac{\ele^2}{r^2} = \frac{0{,}5\,\text{Гн}}2 \cdot \sqr{\frac{8\,\text{В}}{2\,\text{Ом}}} \approx 4\,\text{Дж}.
    \end{align*}
}

\variantsplitter

\addpersonalvariant{Рената Таржиманова}

\tasknumber{1}%
\task{%
    Положительно заряженная частица движется со скоростью $v$ в магнитном поле перпендикулярно линиям его индукции.
    Индукция магнитного поля равна $B$, масса частицы $m$, её заряд — $q$.
    Выведите из базовых физических законов формулы для радиуса траектории частицы и её частоты обращения.
}
\answer{%
    $
        F = ma, F = qvB, a = v^2 / R \implies R = \frac{mv}{qB}.
        \quad T = \frac{2\pi R}{v} = \frac{2\pi m}{qB}.
        \quad \omega = \frac vR = \frac{qB}{m}.
        \quad \nu = \frac 1T = \frac{qB}{2\pi m}.
    $
}
\solutionspace{80pt}

\tasknumber{2}%
\task{%
    В однородном горизонтальном магнитном поле с индукцией $B = 50\,\text{мТл}$ находится проводник,
    расположенный также горизонтально и перпендикулярно полю.
    Какой ток необходимо пустить по проводнику, чтобы он завис?
    Масса единицы длины проводника $\rho = 20\,\frac{\text{г}}{\text{м}}$, $g = 10\,\frac{\text{м}}{\text{с}^{2}}$.
}
\answer{%
    $
            mg = B\eli l, m=\rho l
            \implies \eli
                = \frac{g\rho}B
                = \frac{10\,\frac{\text{м}}{\text{с}^{2}} \cdot 20\,\frac{\text{г}}{\text{м}}}{50\,\text{мТл}}
                = 4\,\text{А}.
    $
}
\solutionspace{80pt}

\tasknumber{3}%
\task{%
    Электрон, прошедший через ускоряющую разность потенциалов, оказывается в магнитном поле индукцией $40\,\text{мТл}$
    и движется по окружности диаметром $4\,\text{мм}$.
    Сделайте рисунок, определите значение разности потенциалов
    и укажите, в какой области потенциал больше, а где меньше.
}
\solutionspace{80pt}

\tasknumber{4}%
\task{%
    Проводник лежит на горизонтальных рельсах,
    замкнутых резистором сопротивлением $3\,\text{Ом}$ (см.
    рис.
    на доске).
    Расстояние между рельсами $70\,\text{см}$.
    Конструкция помещена в вертикальное однородное магнитное поле индукцией $150\,\text{мТл}$.
    Какую силу необходимо прикладывать к проводнику, чтобы двигать его вдоль рельс с постоянной скоростью $4\,\frac{\text{м}}{\text{c}}$?
    Трением пренебречь, сопротивления рельс и проводника малы по сравнению с сопротивлением резистора.
    Ответ выразите в миллиньютонах.
}
\answer{%
    $
        F
            = F_A
            = \eli B l
            = \frac{\ele}R \cdot B l
            = \frac{B v l}R \cdot B l
            = \frac{B^2 v l^2}R
            = \frac{\sqr{150\,\text{мТл}} \cdot 4\,\frac{\text{м}}{\text{c}} \cdot \sqr{70\,\text{см}}}{3\,\text{Ом}}
            \approx 14{,}70\,\text{мН}.
    $
}
\solutionspace{120pt}

\tasknumber{5}%
\task{%
    При изменении силы тока в проводнике по закону $\eli = 3 - 1{,}5t$ (в системе СИ),
    в нём возникает ЭДС самоиндукции $400\,\text{мВ}$.
    Чему равна индуктивность проводника?
    Ответ выразите в миллигенри и округлите до целого.
}
\answer{%
    $
        \ele = L\frac{\abs{\Delta \eli}}{\Delta t} = L \cdot \abs{ - 1{,}5 } \text{(в СИ)}
        \implies L = \frac{\ele}{ 1{,}5 } = \frac{400\,\text{мВ}}{ 1{,}5 } \approx {266{,}7\,\text{мГн}}
    $
}
\solutionspace{80pt}

\tasknumber{6}%
\task{%
    Резистор сопротивлением $R = 4\,\text{Ом}$ и катушка индуктивностью $L = 0{,}4\,\text{Гн}$ (и пренебрежимо малым сопротивлением)
    подключены параллельно к источнику тока с ЭДС $\ele = 8\,\text{В}$ и внутренним сопротивлением $r = 2\,\text{Ом}$ (см.
    рис.
    на доске).
    Какое количество теплоты выделится в цепи после размыкания ключа $K$?
}
\answer{%
    \begin{align*}
    &\text{закон Ома для полной цепи}: \eli = \frac{\ele}{r + R_\text{внешнее}} = \frac{\ele}{r + \frac{R \cdot 0}{R + 0}} = \frac{\ele}{r}, \\
    Q &= W_m = \frac{L\eli^2}2 = \frac{L\sqr{\frac{\ele}{r}}}2 = \frac L2\frac{\ele^2}{r^2} = \frac{0{,}4\,\text{Гн}}2 \cdot \sqr{\frac{8\,\text{В}}{2\,\text{Ом}}} \approx 3{,}20\,\text{Дж}.
    \end{align*}
}

\variantsplitter

\addpersonalvariant{Андрей Щербаков}

\tasknumber{1}%
\task{%
    Положительно заряженная частица движется со скоростью $v$ в магнитном поле перпендикулярно линиям его индукции.
    Индукция магнитного поля равна $B$, масса частицы $m$, её заряд — $q$.
    Выведите из базовых физических законов формулы для радиуса траектории частицы и её периода обращения.
}
\answer{%
    $
        F = ma, F = qvB, a = v^2 / R \implies R = \frac{mv}{qB}.
        \quad T = \frac{2\pi R}{v} = \frac{2\pi m}{qB}.
        \quad \omega = \frac vR = \frac{qB}{m}.
        \quad \nu = \frac 1T = \frac{qB}{2\pi m}.
    $
}
\solutionspace{80pt}

\tasknumber{2}%
\task{%
    В однородном горизонтальном магнитном поле с индукцией $B = 10\,\text{мТл}$ находится проводник,
    расположенный также горизонтально и перпендикулярно полю.
    Какой ток необходимо пустить по проводнику, чтобы он завис?
    Масса единицы длины проводника $\rho = 10\,\frac{\text{г}}{\text{м}}$, $g = 10\,\frac{\text{м}}{\text{с}^{2}}$.
}
\answer{%
    $
            mg = B\eli l, m=\rho l
            \implies \eli
                = \frac{g\rho}B
                = \frac{10\,\frac{\text{м}}{\text{с}^{2}} \cdot 10\,\frac{\text{г}}{\text{м}}}{10\,\text{мТл}}
                = 10\,\text{А}.
    $
}
\solutionspace{80pt}

\tasknumber{3}%
\task{%
    Позитрон, прошедший через ускоряющую разность потенциалов, оказывается в магнитном поле индукцией $40\,\text{мТл}$
    и движется по окружности диаметром $4\,\text{мм}$.
    Сделайте рисунок, определите значение разности потенциалов
    и укажите, в какой области потенциал больше, а где меньше.
}
\solutionspace{80pt}

\tasknumber{4}%
\task{%
    Проводник лежит на горизонтальных рельсах,
    замкнутых резистором сопротивлением $4\,\text{Ом}$ (см.
    рис.
    на доске).
    Расстояние между рельсами $60\,\text{см}$.
    Конструкция помещена в вертикальное однородное магнитное поле индукцией $200\,\text{мТл}$.
    Какую силу необходимо прикладывать к проводнику, чтобы двигать его вдоль рельс с постоянной скоростью $4\,\frac{\text{м}}{\text{c}}$?
    Трением пренебречь, сопротивления рельс и проводника малы по сравнению с сопротивлением резистора.
    Ответ выразите в миллиньютонах.
}
\answer{%
    $
        F
            = F_A
            = \eli B l
            = \frac{\ele}R \cdot B l
            = \frac{B v l}R \cdot B l
            = \frac{B^2 v l^2}R
            = \frac{\sqr{200\,\text{мТл}} \cdot 4\,\frac{\text{м}}{\text{c}} \cdot \sqr{60\,\text{см}}}{4\,\text{Ом}}
            \approx 14{,}40\,\text{мН}.
    $
}
\solutionspace{120pt}

\tasknumber{5}%
\task{%
    При изменении силы тока в проводнике по закону $\eli = 4 - 0{,}5t$ (в системе СИ),
    в нём возникает ЭДС самоиндукции $300\,\text{мВ}$.
    Чему равна индуктивность проводника?
    Ответ выразите в миллигенри и округлите до целого.
}
\answer{%
    $
        \ele = L\frac{\abs{\Delta \eli}}{\Delta t} = L \cdot \abs{ - 0{,}5 } \text{(в СИ)}
        \implies L = \frac{\ele}{ 0{,}5 } = \frac{300\,\text{мВ}}{ 0{,}5 } \approx {600\,\text{мГн}}
    $
}
\solutionspace{80pt}

\tasknumber{6}%
\task{%
    Резистор сопротивлением $R = 4\,\text{Ом}$ и катушка индуктивностью $L = 0{,}2\,\text{Гн}$ (и пренебрежимо малым сопротивлением)
    подключены параллельно к источнику тока с ЭДС $\ele = 8\,\text{В}$ и внутренним сопротивлением $r = 2\,\text{Ом}$ (см.
    рис.
    на доске).
    Какое количество теплоты выделится в цепи после размыкания ключа $K$?
}
\answer{%
    \begin{align*}
    &\text{закон Ома для полной цепи}: \eli = \frac{\ele}{r + R_\text{внешнее}} = \frac{\ele}{r + \frac{R \cdot 0}{R + 0}} = \frac{\ele}{r}, \\
    Q &= W_m = \frac{L\eli^2}2 = \frac{L\sqr{\frac{\ele}{r}}}2 = \frac L2\frac{\ele^2}{r^2} = \frac{0{,}2\,\text{Гн}}2 \cdot \sqr{\frac{8\,\text{В}}{2\,\text{Ом}}} \approx 1{,}60\,\text{Дж}.
    \end{align*}
}

\variantsplitter

\addpersonalvariant{Михаил Ярошевский}

\tasknumber{1}%
\task{%
    Положительно заряженная частица движется со скоростью $v$ в магнитном поле перпендикулярно линиям его индукции.
    Индукция магнитного поля равна $B$, масса частицы $m$, её заряд — $q$.
    Выведите из базовых физических законов формулы для радиуса траектории частицы и её частоты обращения.
}
\answer{%
    $
        F = ma, F = qvB, a = v^2 / R \implies R = \frac{mv}{qB}.
        \quad T = \frac{2\pi R}{v} = \frac{2\pi m}{qB}.
        \quad \omega = \frac vR = \frac{qB}{m}.
        \quad \nu = \frac 1T = \frac{qB}{2\pi m}.
    $
}
\solutionspace{80pt}

\tasknumber{2}%
\task{%
    В однородном горизонтальном магнитном поле с индукцией $B = 50\,\text{мТл}$ находится проводник,
    расположенный также горизонтально и перпендикулярно полю.
    Какой ток необходимо пустить по проводнику, чтобы он завис?
    Масса единицы длины проводника $\rho = 5\,\frac{\text{г}}{\text{м}}$, $g = 10\,\frac{\text{м}}{\text{с}^{2}}$.
}
\answer{%
    $
            mg = B\eli l, m=\rho l
            \implies \eli
                = \frac{g\rho}B
                = \frac{10\,\frac{\text{м}}{\text{с}^{2}} \cdot 5\,\frac{\text{г}}{\text{м}}}{50\,\text{мТл}}
                = 1\,\text{А}.
    $
}
\solutionspace{80pt}

\tasknumber{3}%
\task{%
    Протон, прошедший через ускоряющую разность потенциалов, оказывается в магнитном поле индукцией $40\,\text{мТл}$
    и движется по окружности диаметром $4\,\text{мм}$.
    Сделайте рисунок, определите значение разности потенциалов
    и укажите, в какой области потенциал больше, а где меньше.
}
\solutionspace{80pt}

\tasknumber{4}%
\task{%
    Проводник лежит на горизонтальных рельсах,
    замкнутых резистором сопротивлением $3\,\text{Ом}$ (см.
    рис.
    на доске).
    Расстояние между рельсами $80\,\text{см}$.
    Конструкция помещена в вертикальное однородное магнитное поле индукцией $200\,\text{мТл}$.
    Какую силу необходимо прикладывать к проводнику, чтобы двигать его вдоль рельс с постоянной скоростью $4\,\frac{\text{м}}{\text{c}}$?
    Трением пренебречь, сопротивления рельс и проводника малы по сравнению с сопротивлением резистора.
    Ответ выразите в миллиньютонах.
}
\answer{%
    $
        F
            = F_A
            = \eli B l
            = \frac{\ele}R \cdot B l
            = \frac{B v l}R \cdot B l
            = \frac{B^2 v l^2}R
            = \frac{\sqr{200\,\text{мТл}} \cdot 4\,\frac{\text{м}}{\text{c}} \cdot \sqr{80\,\text{см}}}{3\,\text{Ом}}
            \approx 34{,}13\,\text{мН}.
    $
}
\solutionspace{120pt}

\tasknumber{5}%
\task{%
    При изменении силы тока в проводнике по закону $\eli = 4 + 0{,}4t$ (в системе СИ),
    в нём возникает ЭДС самоиндукции $400\,\text{мВ}$.
    Чему равна индуктивность проводника?
    Ответ выразите в миллигенри и округлите до целого.
}
\answer{%
    $
        \ele = L\frac{\abs{\Delta \eli}}{\Delta t} = L \cdot \abs{ + 0{,}4 } \text{(в СИ)}
        \implies L = \frac{\ele}{ 0{,}4 } = \frac{400\,\text{мВ}}{ 0{,}4 } \approx {1000\,\text{мГн}}
    $
}
\solutionspace{80pt}

\tasknumber{6}%
\task{%
    Резистор сопротивлением $R = 5\,\text{Ом}$ и катушка индуктивностью $L = 0{,}4\,\text{Гн}$ (и пренебрежимо малым сопротивлением)
    подключены параллельно к источнику тока с ЭДС $\ele = 6\,\text{В}$ и внутренним сопротивлением $r = 2\,\text{Ом}$ (см.
    рис.
    на доске).
    Какое количество теплоты выделится в цепи после размыкания ключа $K$?
}
\answer{%
    \begin{align*}
    &\text{закон Ома для полной цепи}: \eli = \frac{\ele}{r + R_\text{внешнее}} = \frac{\ele}{r + \frac{R \cdot 0}{R + 0}} = \frac{\ele}{r}, \\
    Q &= W_m = \frac{L\eli^2}2 = \frac{L\sqr{\frac{\ele}{r}}}2 = \frac L2\frac{\ele^2}{r^2} = \frac{0{,}4\,\text{Гн}}2 \cdot \sqr{\frac{6\,\text{В}}{2\,\text{Ом}}} \approx 1{,}80\,\text{Дж}.
    \end{align*}
}

\variantsplitter

\addpersonalvariant{Алексей Алимпиев}

\tasknumber{1}%
\task{%
    Положительно заряженная частица движется со скоростью $v$ в магнитном поле перпендикулярно линиям его индукции.
    Индукция магнитного поля равна $B$, масса частицы $m$, её заряд — $q$.
    Выведите из базовых физических законов формулы для радиуса траектории частицы и её угловой скорости.
}
\answer{%
    $
        F = ma, F = qvB, a = v^2 / R \implies R = \frac{mv}{qB}.
        \quad T = \frac{2\pi R}{v} = \frac{2\pi m}{qB}.
        \quad \omega = \frac vR = \frac{qB}{m}.
        \quad \nu = \frac 1T = \frac{qB}{2\pi m}.
    $
}
\solutionspace{80pt}

\tasknumber{2}%
\task{%
    В однородном горизонтальном магнитном поле с индукцией $B = 20\,\text{мТл}$ находится проводник,
    расположенный также горизонтально и перпендикулярно полю.
    Какой ток необходимо пустить по проводнику, чтобы он завис?
    Масса единицы длины проводника $\rho = 40\,\frac{\text{г}}{\text{м}}$, $g = 10\,\frac{\text{м}}{\text{с}^{2}}$.
}
\answer{%
    $
            mg = B\eli l, m=\rho l
            \implies \eli
                = \frac{g\rho}B
                = \frac{10\,\frac{\text{м}}{\text{с}^{2}} \cdot 40\,\frac{\text{г}}{\text{м}}}{20\,\text{мТл}}
                = 20\,\text{А}.
    $
}
\solutionspace{80pt}

\tasknumber{3}%
\task{%
    Электрон, прошедший через ускоряющую разность потенциалов, оказывается в магнитном поле индукцией $20\,\text{мТл}$
    и движется по окружности диаметром $4\,\text{мм}$.
    Сделайте рисунок, определите значение разности потенциалов
    и укажите, в какой области потенциал больше, а где меньше.
}
\solutionspace{80pt}

\tasknumber{4}%
\task{%
    Проводник лежит на горизонтальных рельсах,
    замкнутых резистором сопротивлением $2\,\text{Ом}$ (см.
    рис.
    на доске).
    Расстояние между рельсами $60\,\text{см}$.
    Конструкция помещена в вертикальное однородное магнитное поле индукцией $200\,\text{мТл}$.
    Какую силу необходимо прикладывать к проводнику, чтобы двигать его вдоль рельс с постоянной скоростью $3\,\frac{\text{м}}{\text{c}}$?
    Трением пренебречь, сопротивления рельс и проводника малы по сравнению с сопротивлением резистора.
    Ответ выразите в миллиньютонах.
}
\answer{%
    $
        F
            = F_A
            = \eli B l
            = \frac{\ele}R \cdot B l
            = \frac{B v l}R \cdot B l
            = \frac{B^2 v l^2}R
            = \frac{\sqr{200\,\text{мТл}} \cdot 3\,\frac{\text{м}}{\text{c}} \cdot \sqr{60\,\text{см}}}{2\,\text{Ом}}
            \approx 21{,}60\,\text{мН}.
    $
}
\solutionspace{120pt}

\tasknumber{5}%
\task{%
    При изменении силы тока в проводнике по закону $\eli = 4 + 0{,}8t$ (в системе СИ),
    в нём возникает ЭДС самоиндукции $300\,\text{мВ}$.
    Чему равна индуктивность проводника?
    Ответ выразите в миллигенри и округлите до целого.
}
\answer{%
    $
        \ele = L\frac{\abs{\Delta \eli}}{\Delta t} = L \cdot \abs{ + 0{,}8 } \text{(в СИ)}
        \implies L = \frac{\ele}{ 0{,}8 } = \frac{300\,\text{мВ}}{ 0{,}8 } \approx {375\,\text{мГн}}
    $
}
\solutionspace{80pt}

\tasknumber{6}%
\task{%
    Резистор сопротивлением $R = 4\,\text{Ом}$ и катушка индуктивностью $L = 0{,}2\,\text{Гн}$ (и пренебрежимо малым сопротивлением)
    подключены параллельно к источнику тока с ЭДС $\ele = 8\,\text{В}$ и внутренним сопротивлением $r = 1\,\text{Ом}$ (см.
    рис.
    на доске).
    Какое количество теплоты выделится в цепи после размыкания ключа $K$?
}
\answer{%
    \begin{align*}
    &\text{закон Ома для полной цепи}: \eli = \frac{\ele}{r + R_\text{внешнее}} = \frac{\ele}{r + \frac{R \cdot 0}{R + 0}} = \frac{\ele}{r}, \\
    Q &= W_m = \frac{L\eli^2}2 = \frac{L\sqr{\frac{\ele}{r}}}2 = \frac L2\frac{\ele^2}{r^2} = \frac{0{,}2\,\text{Гн}}2 \cdot \sqr{\frac{8\,\text{В}}{1\,\text{Ом}}} \approx 6{,}40\,\text{Дж}.
    \end{align*}
}

\variantsplitter

\addpersonalvariant{Евгений Васин}

\tasknumber{1}%
\task{%
    Положительно заряженная частица движется со скоростью $v$ в магнитном поле перпендикулярно линиям его индукции.
    Индукция магнитного поля равна $B$, масса частицы $m$, её заряд — $q$.
    Выведите из базовых физических законов формулы для радиуса траектории частицы и её угловой скорости.
}
\answer{%
    $
        F = ma, F = qvB, a = v^2 / R \implies R = \frac{mv}{qB}.
        \quad T = \frac{2\pi R}{v} = \frac{2\pi m}{qB}.
        \quad \omega = \frac vR = \frac{qB}{m}.
        \quad \nu = \frac 1T = \frac{qB}{2\pi m}.
    $
}
\solutionspace{80pt}

\tasknumber{2}%
\task{%
    В однородном горизонтальном магнитном поле с индукцией $B = 100\,\text{мТл}$ находится проводник,
    расположенный также горизонтально и перпендикулярно полю.
    Какой ток необходимо пустить по проводнику, чтобы он завис?
    Масса единицы длины проводника $\rho = 20\,\frac{\text{г}}{\text{м}}$, $g = 10\,\frac{\text{м}}{\text{с}^{2}}$.
}
\answer{%
    $
            mg = B\eli l, m=\rho l
            \implies \eli
                = \frac{g\rho}B
                = \frac{10\,\frac{\text{м}}{\text{с}^{2}} \cdot 20\,\frac{\text{г}}{\text{м}}}{100\,\text{мТл}}
                = 2\,\text{А}.
    $
}
\solutionspace{80pt}

\tasknumber{3}%
\task{%
    Протон, прошедший через ускоряющую разность потенциалов, оказывается в магнитном поле индукцией $20\,\text{мТл}$
    и движется по окружности диаметром $8\,\text{мм}$.
    Сделайте рисунок, определите значение разности потенциалов
    и укажите, в какой области потенциал больше, а где меньше.
}
\solutionspace{80pt}

\tasknumber{4}%
\task{%
    Проводник лежит на горизонтальных рельсах,
    замкнутых резистором сопротивлением $3\,\text{Ом}$ (см.
    рис.
    на доске).
    Расстояние между рельсами $70\,\text{см}$.
    Конструкция помещена в вертикальное однородное магнитное поле индукцией $400\,\text{мТл}$.
    Какую силу необходимо прикладывать к проводнику, чтобы двигать его вдоль рельс с постоянной скоростью $5\,\frac{\text{м}}{\text{c}}$?
    Трением пренебречь, сопротивления рельс и проводника малы по сравнению с сопротивлением резистора.
    Ответ выразите в миллиньютонах.
}
\answer{%
    $
        F
            = F_A
            = \eli B l
            = \frac{\ele}R \cdot B l
            = \frac{B v l}R \cdot B l
            = \frac{B^2 v l^2}R
            = \frac{\sqr{400\,\text{мТл}} \cdot 5\,\frac{\text{м}}{\text{c}} \cdot \sqr{70\,\text{см}}}{3\,\text{Ом}}
            \approx 130{,}67\,\text{мН}.
    $
}
\solutionspace{120pt}

\tasknumber{5}%
\task{%
    При изменении силы тока в проводнике по закону $\eli = 7 + 1{,}5t$ (в системе СИ),
    в нём возникает ЭДС самоиндукции $200\,\text{мВ}$.
    Чему равна индуктивность проводника?
    Ответ выразите в миллигенри и округлите до целого.
}
\answer{%
    $
        \ele = L\frac{\abs{\Delta \eli}}{\Delta t} = L \cdot \abs{ + 1{,}5 } \text{(в СИ)}
        \implies L = \frac{\ele}{ 1{,}5 } = \frac{200\,\text{мВ}}{ 1{,}5 } \approx {133{,}3\,\text{мГн}}
    $
}
\solutionspace{80pt}

\tasknumber{6}%
\task{%
    Резистор сопротивлением $R = 5\,\text{Ом}$ и катушка индуктивностью $L = 0{,}2\,\text{Гн}$ (и пренебрежимо малым сопротивлением)
    подключены параллельно к источнику тока с ЭДС $\ele = 12\,\text{В}$ и внутренним сопротивлением $r = 2\,\text{Ом}$ (см.
    рис.
    на доске).
    Какое количество теплоты выделится в цепи после размыкания ключа $K$?
}
\answer{%
    \begin{align*}
    &\text{закон Ома для полной цепи}: \eli = \frac{\ele}{r + R_\text{внешнее}} = \frac{\ele}{r + \frac{R \cdot 0}{R + 0}} = \frac{\ele}{r}, \\
    Q &= W_m = \frac{L\eli^2}2 = \frac{L\sqr{\frac{\ele}{r}}}2 = \frac L2\frac{\ele^2}{r^2} = \frac{0{,}2\,\text{Гн}}2 \cdot \sqr{\frac{12\,\text{В}}{2\,\text{Ом}}} \approx 3{,}60\,\text{Дж}.
    \end{align*}
}

\variantsplitter

\addpersonalvariant{Вячеслав Волохов}

\tasknumber{1}%
\task{%
    Положительно заряженная частица движется со скоростью $v$ в магнитном поле перпендикулярно линиям его индукции.
    Индукция магнитного поля равна $B$, масса частицы $m$, её заряд — $q$.
    Выведите из базовых физических законов формулы для радиуса траектории частицы и её частоты обращения.
}
\answer{%
    $
        F = ma, F = qvB, a = v^2 / R \implies R = \frac{mv}{qB}.
        \quad T = \frac{2\pi R}{v} = \frac{2\pi m}{qB}.
        \quad \omega = \frac vR = \frac{qB}{m}.
        \quad \nu = \frac 1T = \frac{qB}{2\pi m}.
    $
}
\solutionspace{80pt}

\tasknumber{2}%
\task{%
    В однородном горизонтальном магнитном поле с индукцией $B = 10\,\text{мТл}$ находится проводник,
    расположенный также горизонтально и перпендикулярно полю.
    Какой ток необходимо пустить по проводнику, чтобы он завис?
    Масса единицы длины проводника $\rho = 100\,\frac{\text{г}}{\text{м}}$, $g = 10\,\frac{\text{м}}{\text{с}^{2}}$.
}
\answer{%
    $
            mg = B\eli l, m=\rho l
            \implies \eli
                = \frac{g\rho}B
                = \frac{10\,\frac{\text{м}}{\text{с}^{2}} \cdot 100\,\frac{\text{г}}{\text{м}}}{10\,\text{мТл}}
                = 100\,\text{А}.
    $
}
\solutionspace{80pt}

\tasknumber{3}%
\task{%
    Позитрон, прошедший через ускоряющую разность потенциалов, оказывается в магнитном поле индукцией $20\,\text{мТл}$
    и движется по окружности диаметром $6\,\text{мм}$.
    Сделайте рисунок, определите значение разности потенциалов
    и укажите, в какой области потенциал больше, а где меньше.
}
\solutionspace{80pt}

\tasknumber{4}%
\task{%
    Проводник лежит на горизонтальных рельсах,
    замкнутых резистором сопротивлением $2\,\text{Ом}$ (см.
    рис.
    на доске).
    Расстояние между рельсами $50\,\text{см}$.
    Конструкция помещена в вертикальное однородное магнитное поле индукцией $400\,\text{мТл}$.
    Какую силу необходимо прикладывать к проводнику, чтобы двигать его вдоль рельс с постоянной скоростью $3\,\frac{\text{м}}{\text{c}}$?
    Трением пренебречь, сопротивления рельс и проводника малы по сравнению с сопротивлением резистора.
    Ответ выразите в миллиньютонах.
}
\answer{%
    $
        F
            = F_A
            = \eli B l
            = \frac{\ele}R \cdot B l
            = \frac{B v l}R \cdot B l
            = \frac{B^2 v l^2}R
            = \frac{\sqr{400\,\text{мТл}} \cdot 3\,\frac{\text{м}}{\text{c}} \cdot \sqr{50\,\text{см}}}{2\,\text{Ом}}
            \approx 60\,\text{мН}.
    $
}
\solutionspace{120pt}

\tasknumber{5}%
\task{%
    При изменении силы тока в проводнике по закону $\eli = 7 - 0{,}8t$ (в системе СИ),
    в нём возникает ЭДС самоиндукции $400\,\text{мВ}$.
    Чему равна индуктивность проводника?
    Ответ выразите в миллигенри и округлите до целого.
}
\answer{%
    $
        \ele = L\frac{\abs{\Delta \eli}}{\Delta t} = L \cdot \abs{ - 0{,}8 } \text{(в СИ)}
        \implies L = \frac{\ele}{ 0{,}8 } = \frac{400\,\text{мВ}}{ 0{,}8 } \approx {500\,\text{мГн}}
    $
}
\solutionspace{80pt}

\tasknumber{6}%
\task{%
    Резистор сопротивлением $R = 4\,\text{Ом}$ и катушка индуктивностью $L = 0{,}5\,\text{Гн}$ (и пренебрежимо малым сопротивлением)
    подключены параллельно к источнику тока с ЭДС $\ele = 6\,\text{В}$ и внутренним сопротивлением $r = 2\,\text{Ом}$ (см.
    рис.
    на доске).
    Какое количество теплоты выделится в цепи после размыкания ключа $K$?
}
\answer{%
    \begin{align*}
    &\text{закон Ома для полной цепи}: \eli = \frac{\ele}{r + R_\text{внешнее}} = \frac{\ele}{r + \frac{R \cdot 0}{R + 0}} = \frac{\ele}{r}, \\
    Q &= W_m = \frac{L\eli^2}2 = \frac{L\sqr{\frac{\ele}{r}}}2 = \frac L2\frac{\ele^2}{r^2} = \frac{0{,}5\,\text{Гн}}2 \cdot \sqr{\frac{6\,\text{В}}{2\,\text{Ом}}} \approx 2{,}25\,\text{Дж}.
    \end{align*}
}

\variantsplitter

\addpersonalvariant{Герман Говоров}

\tasknumber{1}%
\task{%
    Положительно заряженная частица движется со скоростью $v$ в магнитном поле перпендикулярно линиям его индукции.
    Индукция магнитного поля равна $B$, масса частицы $m$, её заряд — $q$.
    Выведите из базовых физических законов формулы для радиуса траектории частицы и её периода обращения.
}
\answer{%
    $
        F = ma, F = qvB, a = v^2 / R \implies R = \frac{mv}{qB}.
        \quad T = \frac{2\pi R}{v} = \frac{2\pi m}{qB}.
        \quad \omega = \frac vR = \frac{qB}{m}.
        \quad \nu = \frac 1T = \frac{qB}{2\pi m}.
    $
}
\solutionspace{80pt}

\tasknumber{2}%
\task{%
    В однородном горизонтальном магнитном поле с индукцией $B = 10\,\text{мТл}$ находится проводник,
    расположенный также горизонтально и перпендикулярно полю.
    Какой ток необходимо пустить по проводнику, чтобы он завис?
    Масса единицы длины проводника $\rho = 5\,\frac{\text{г}}{\text{м}}$, $g = 10\,\frac{\text{м}}{\text{с}^{2}}$.
}
\answer{%
    $
            mg = B\eli l, m=\rho l
            \implies \eli
                = \frac{g\rho}B
                = \frac{10\,\frac{\text{м}}{\text{с}^{2}} \cdot 5\,\frac{\text{г}}{\text{м}}}{10\,\text{мТл}}
                = 5\,\text{А}.
    $
}
\solutionspace{80pt}

\tasknumber{3}%
\task{%
    Протон, прошедший через ускоряющую разность потенциалов, оказывается в магнитном поле индукцией $50\,\text{мТл}$
    и движется по окружности диаметром $6\,\text{мм}$.
    Сделайте рисунок, определите значение разности потенциалов
    и укажите, в какой области потенциал больше, а где меньше.
}
\solutionspace{80pt}

\tasknumber{4}%
\task{%
    Проводник лежит на горизонтальных рельсах,
    замкнутых резистором сопротивлением $4\,\text{Ом}$ (см.
    рис.
    на доске).
    Расстояние между рельсами $60\,\text{см}$.
    Конструкция помещена в вертикальное однородное магнитное поле индукцией $150\,\text{мТл}$.
    Какую силу необходимо прикладывать к проводнику, чтобы двигать его вдоль рельс с постоянной скоростью $3\,\frac{\text{м}}{\text{c}}$?
    Трением пренебречь, сопротивления рельс и проводника малы по сравнению с сопротивлением резистора.
    Ответ выразите в миллиньютонах.
}
\answer{%
    $
        F
            = F_A
            = \eli B l
            = \frac{\ele}R \cdot B l
            = \frac{B v l}R \cdot B l
            = \frac{B^2 v l^2}R
            = \frac{\sqr{150\,\text{мТл}} \cdot 3\,\frac{\text{м}}{\text{c}} \cdot \sqr{60\,\text{см}}}{4\,\text{Ом}}
            \approx 6{,}08\,\text{мН}.
    $
}
\solutionspace{120pt}

\tasknumber{5}%
\task{%
    При изменении силы тока в проводнике по закону $\eli = 5 + 0{,}8t$ (в системе СИ),
    в нём возникает ЭДС самоиндукции $300\,\text{мВ}$.
    Чему равна индуктивность проводника?
    Ответ выразите в миллигенри и округлите до целого.
}
\answer{%
    $
        \ele = L\frac{\abs{\Delta \eli}}{\Delta t} = L \cdot \abs{ + 0{,}8 } \text{(в СИ)}
        \implies L = \frac{\ele}{ 0{,}8 } = \frac{300\,\text{мВ}}{ 0{,}8 } \approx {375\,\text{мГн}}
    $
}
\solutionspace{80pt}

\tasknumber{6}%
\task{%
    Резистор сопротивлением $R = 3\,\text{Ом}$ и катушка индуктивностью $L = 0{,}5\,\text{Гн}$ (и пренебрежимо малым сопротивлением)
    подключены параллельно к источнику тока с ЭДС $\ele = 8\,\text{В}$ и внутренним сопротивлением $r = 1\,\text{Ом}$ (см.
    рис.
    на доске).
    Какое количество теплоты выделится в цепи после размыкания ключа $K$?
}
\answer{%
    \begin{align*}
    &\text{закон Ома для полной цепи}: \eli = \frac{\ele}{r + R_\text{внешнее}} = \frac{\ele}{r + \frac{R \cdot 0}{R + 0}} = \frac{\ele}{r}, \\
    Q &= W_m = \frac{L\eli^2}2 = \frac{L\sqr{\frac{\ele}{r}}}2 = \frac L2\frac{\ele^2}{r^2} = \frac{0{,}5\,\text{Гн}}2 \cdot \sqr{\frac{8\,\text{В}}{1\,\text{Ом}}} \approx 16\,\text{Дж}.
    \end{align*}
}

\variantsplitter

\addpersonalvariant{София Журавлёва}

\tasknumber{1}%
\task{%
    Положительно заряженная частица движется со скоростью $v$ в магнитном поле перпендикулярно линиям его индукции.
    Индукция магнитного поля равна $B$, масса частицы $m$, её заряд — $q$.
    Выведите из базовых физических законов формулы для радиуса траектории частицы и её периода обращения.
}
\answer{%
    $
        F = ma, F = qvB, a = v^2 / R \implies R = \frac{mv}{qB}.
        \quad T = \frac{2\pi R}{v} = \frac{2\pi m}{qB}.
        \quad \omega = \frac vR = \frac{qB}{m}.
        \quad \nu = \frac 1T = \frac{qB}{2\pi m}.
    $
}
\solutionspace{80pt}

\tasknumber{2}%
\task{%
    В однородном горизонтальном магнитном поле с индукцией $B = 10\,\text{мТл}$ находится проводник,
    расположенный также горизонтально и перпендикулярно полю.
    Какой ток необходимо пустить по проводнику, чтобы он завис?
    Масса единицы длины проводника $\rho = 100\,\frac{\text{г}}{\text{м}}$, $g = 10\,\frac{\text{м}}{\text{с}^{2}}$.
}
\answer{%
    $
            mg = B\eli l, m=\rho l
            \implies \eli
                = \frac{g\rho}B
                = \frac{10\,\frac{\text{м}}{\text{с}^{2}} \cdot 100\,\frac{\text{г}}{\text{м}}}{10\,\text{мТл}}
                = 100\,\text{А}.
    $
}
\solutionspace{80pt}

\tasknumber{3}%
\task{%
    Позитрон, прошедший через ускоряющую разность потенциалов, оказывается в магнитном поле индукцией $50\,\text{мТл}$
    и движется по окружности диаметром $6\,\text{мм}$.
    Сделайте рисунок, определите значение разности потенциалов
    и укажите, в какой области потенциал больше, а где меньше.
}
\solutionspace{80pt}

\tasknumber{4}%
\task{%
    Проводник лежит на горизонтальных рельсах,
    замкнутых резистором сопротивлением $4\,\text{Ом}$ (см.
    рис.
    на доске).
    Расстояние между рельсами $50\,\text{см}$.
    Конструкция помещена в вертикальное однородное магнитное поле индукцией $400\,\text{мТл}$.
    Какую силу необходимо прикладывать к проводнику, чтобы двигать его вдоль рельс с постоянной скоростью $2\,\frac{\text{м}}{\text{c}}$?
    Трением пренебречь, сопротивления рельс и проводника малы по сравнению с сопротивлением резистора.
    Ответ выразите в миллиньютонах.
}
\answer{%
    $
        F
            = F_A
            = \eli B l
            = \frac{\ele}R \cdot B l
            = \frac{B v l}R \cdot B l
            = \frac{B^2 v l^2}R
            = \frac{\sqr{400\,\text{мТл}} \cdot 2\,\frac{\text{м}}{\text{c}} \cdot \sqr{50\,\text{см}}}{4\,\text{Ом}}
            \approx 20\,\text{мН}.
    $
}
\solutionspace{120pt}

\tasknumber{5}%
\task{%
    При изменении силы тока в проводнике по закону $\eli = 6 + 1{,}5t$ (в системе СИ),
    в нём возникает ЭДС самоиндукции $150\,\text{мВ}$.
    Чему равна индуктивность проводника?
    Ответ выразите в миллигенри и округлите до целого.
}
\answer{%
    $
        \ele = L\frac{\abs{\Delta \eli}}{\Delta t} = L \cdot \abs{ + 1{,}5 } \text{(в СИ)}
        \implies L = \frac{\ele}{ 1{,}5 } = \frac{150\,\text{мВ}}{ 1{,}5 } \approx {100\,\text{мГн}}
    $
}
\solutionspace{80pt}

\tasknumber{6}%
\task{%
    Резистор сопротивлением $R = 3\,\text{Ом}$ и катушка индуктивностью $L = 0{,}5\,\text{Гн}$ (и пренебрежимо малым сопротивлением)
    подключены параллельно к источнику тока с ЭДС $\ele = 12\,\text{В}$ и внутренним сопротивлением $r = 2\,\text{Ом}$ (см.
    рис.
    на доске).
    Какое количество теплоты выделится в цепи после размыкания ключа $K$?
}
\answer{%
    \begin{align*}
    &\text{закон Ома для полной цепи}: \eli = \frac{\ele}{r + R_\text{внешнее}} = \frac{\ele}{r + \frac{R \cdot 0}{R + 0}} = \frac{\ele}{r}, \\
    Q &= W_m = \frac{L\eli^2}2 = \frac{L\sqr{\frac{\ele}{r}}}2 = \frac L2\frac{\ele^2}{r^2} = \frac{0{,}5\,\text{Гн}}2 \cdot \sqr{\frac{12\,\text{В}}{2\,\text{Ом}}} \approx 9\,\text{Дж}.
    \end{align*}
}

\variantsplitter

\addpersonalvariant{Константин Козлов}

\tasknumber{1}%
\task{%
    Положительно заряженная частица движется со скоростью $v$ в магнитном поле перпендикулярно линиям его индукции.
    Индукция магнитного поля равна $B$, масса частицы $m$, её заряд — $q$.
    Выведите из базовых физических законов формулы для радиуса траектории частицы и её частоты обращения.
}
\answer{%
    $
        F = ma, F = qvB, a = v^2 / R \implies R = \frac{mv}{qB}.
        \quad T = \frac{2\pi R}{v} = \frac{2\pi m}{qB}.
        \quad \omega = \frac vR = \frac{qB}{m}.
        \quad \nu = \frac 1T = \frac{qB}{2\pi m}.
    $
}
\solutionspace{80pt}

\tasknumber{2}%
\task{%
    В однородном горизонтальном магнитном поле с индукцией $B = 20\,\text{мТл}$ находится проводник,
    расположенный также горизонтально и перпендикулярно полю.
    Какой ток необходимо пустить по проводнику, чтобы он завис?
    Масса единицы длины проводника $\rho = 20\,\frac{\text{г}}{\text{м}}$, $g = 10\,\frac{\text{м}}{\text{с}^{2}}$.
}
\answer{%
    $
            mg = B\eli l, m=\rho l
            \implies \eli
                = \frac{g\rho}B
                = \frac{10\,\frac{\text{м}}{\text{с}^{2}} \cdot 20\,\frac{\text{г}}{\text{м}}}{20\,\text{мТл}}
                = 10\,\text{А}.
    $
}
\solutionspace{80pt}

\tasknumber{3}%
\task{%
    Позитрон, прошедший через ускоряющую разность потенциалов, оказывается в магнитном поле индукцией $50\,\text{мТл}$
    и движется по окружности диаметром $4\,\text{мм}$.
    Сделайте рисунок, определите значение разности потенциалов
    и укажите, в какой области потенциал больше, а где меньше.
}
\solutionspace{80pt}

\tasknumber{4}%
\task{%
    Проводник лежит на горизонтальных рельсах,
    замкнутых резистором сопротивлением $2\,\text{Ом}$ (см.
    рис.
    на доске).
    Расстояние между рельсами $80\,\text{см}$.
    Конструкция помещена в вертикальное однородное магнитное поле индукцией $300\,\text{мТл}$.
    Какую силу необходимо прикладывать к проводнику, чтобы двигать его вдоль рельс с постоянной скоростью $5\,\frac{\text{м}}{\text{c}}$?
    Трением пренебречь, сопротивления рельс и проводника малы по сравнению с сопротивлением резистора.
    Ответ выразите в миллиньютонах.
}
\answer{%
    $
        F
            = F_A
            = \eli B l
            = \frac{\ele}R \cdot B l
            = \frac{B v l}R \cdot B l
            = \frac{B^2 v l^2}R
            = \frac{\sqr{300\,\text{мТл}} \cdot 5\,\frac{\text{м}}{\text{c}} \cdot \sqr{80\,\text{см}}}{2\,\text{Ом}}
            \approx 144\,\text{мН}.
    $
}
\solutionspace{120pt}

\tasknumber{5}%
\task{%
    При изменении силы тока в проводнике по закону $\eli = 7 + 1{,}5t$ (в системе СИ),
    в нём возникает ЭДС самоиндукции $300\,\text{мВ}$.
    Чему равна индуктивность проводника?
    Ответ выразите в миллигенри и округлите до целого.
}
\answer{%
    $
        \ele = L\frac{\abs{\Delta \eli}}{\Delta t} = L \cdot \abs{ + 1{,}5 } \text{(в СИ)}
        \implies L = \frac{\ele}{ 1{,}5 } = \frac{300\,\text{мВ}}{ 1{,}5 } \approx {200\,\text{мГн}}
    $
}
\solutionspace{80pt}

\tasknumber{6}%
\task{%
    Резистор сопротивлением $R = 4\,\text{Ом}$ и катушка индуктивностью $L = 0{,}5\,\text{Гн}$ (и пренебрежимо малым сопротивлением)
    подключены параллельно к источнику тока с ЭДС $\ele = 6\,\text{В}$ и внутренним сопротивлением $r = 2\,\text{Ом}$ (см.
    рис.
    на доске).
    Какое количество теплоты выделится в цепи после размыкания ключа $K$?
}
\answer{%
    \begin{align*}
    &\text{закон Ома для полной цепи}: \eli = \frac{\ele}{r + R_\text{внешнее}} = \frac{\ele}{r + \frac{R \cdot 0}{R + 0}} = \frac{\ele}{r}, \\
    Q &= W_m = \frac{L\eli^2}2 = \frac{L\sqr{\frac{\ele}{r}}}2 = \frac L2\frac{\ele^2}{r^2} = \frac{0{,}5\,\text{Гн}}2 \cdot \sqr{\frac{6\,\text{В}}{2\,\text{Ом}}} \approx 2{,}25\,\text{Дж}.
    \end{align*}
}

\variantsplitter

\addpersonalvariant{Наталья Кравченко}

\tasknumber{1}%
\task{%
    Положительно заряженная частица движется со скоростью $v$ в магнитном поле перпендикулярно линиям его индукции.
    Индукция магнитного поля равна $B$, масса частицы $m$, её заряд — $q$.
    Выведите из базовых физических законов формулы для радиуса траектории частицы и её угловой скорости.
}
\answer{%
    $
        F = ma, F = qvB, a = v^2 / R \implies R = \frac{mv}{qB}.
        \quad T = \frac{2\pi R}{v} = \frac{2\pi m}{qB}.
        \quad \omega = \frac vR = \frac{qB}{m}.
        \quad \nu = \frac 1T = \frac{qB}{2\pi m}.
    $
}
\solutionspace{80pt}

\tasknumber{2}%
\task{%
    В однородном горизонтальном магнитном поле с индукцией $B = 10\,\text{мТл}$ находится проводник,
    расположенный также горизонтально и перпендикулярно полю.
    Какой ток необходимо пустить по проводнику, чтобы он завис?
    Масса единицы длины проводника $\rho = 40\,\frac{\text{г}}{\text{м}}$, $g = 10\,\frac{\text{м}}{\text{с}^{2}}$.
}
\answer{%
    $
            mg = B\eli l, m=\rho l
            \implies \eli
                = \frac{g\rho}B
                = \frac{10\,\frac{\text{м}}{\text{с}^{2}} \cdot 40\,\frac{\text{г}}{\text{м}}}{10\,\text{мТл}}
                = 40\,\text{А}.
    $
}
\solutionspace{80pt}

\tasknumber{3}%
\task{%
    Позитрон, прошедший через ускоряющую разность потенциалов, оказывается в магнитном поле индукцией $40\,\text{мТл}$
    и движется по окружности диаметром $6\,\text{мм}$.
    Сделайте рисунок, определите значение разности потенциалов
    и укажите, в какой области потенциал больше, а где меньше.
}
\solutionspace{80pt}

\tasknumber{4}%
\task{%
    Проводник лежит на горизонтальных рельсах,
    замкнутых резистором сопротивлением $3\,\text{Ом}$ (см.
    рис.
    на доске).
    Расстояние между рельсами $70\,\text{см}$.
    Конструкция помещена в вертикальное однородное магнитное поле индукцией $300\,\text{мТл}$.
    Какую силу необходимо прикладывать к проводнику, чтобы двигать его вдоль рельс с постоянной скоростью $2\,\frac{\text{м}}{\text{c}}$?
    Трением пренебречь, сопротивления рельс и проводника малы по сравнению с сопротивлением резистора.
    Ответ выразите в миллиньютонах.
}
\answer{%
    $
        F
            = F_A
            = \eli B l
            = \frac{\ele}R \cdot B l
            = \frac{B v l}R \cdot B l
            = \frac{B^2 v l^2}R
            = \frac{\sqr{300\,\text{мТл}} \cdot 2\,\frac{\text{м}}{\text{c}} \cdot \sqr{70\,\text{см}}}{3\,\text{Ом}}
            \approx 29{,}40\,\text{мН}.
    $
}
\solutionspace{120pt}

\tasknumber{5}%
\task{%
    При изменении силы тока в проводнике по закону $\eli = 7 + 1{,}5t$ (в системе СИ),
    в нём возникает ЭДС самоиндукции $400\,\text{мВ}$.
    Чему равна индуктивность проводника?
    Ответ выразите в миллигенри и округлите до целого.
}
\answer{%
    $
        \ele = L\frac{\abs{\Delta \eli}}{\Delta t} = L \cdot \abs{ + 1{,}5 } \text{(в СИ)}
        \implies L = \frac{\ele}{ 1{,}5 } = \frac{400\,\text{мВ}}{ 1{,}5 } \approx {266{,}7\,\text{мГн}}
    $
}
\solutionspace{80pt}

\tasknumber{6}%
\task{%
    Резистор сопротивлением $R = 3\,\text{Ом}$ и катушка индуктивностью $L = 0{,}2\,\text{Гн}$ (и пренебрежимо малым сопротивлением)
    подключены параллельно к источнику тока с ЭДС $\ele = 5\,\text{В}$ и внутренним сопротивлением $r = 1\,\text{Ом}$ (см.
    рис.
    на доске).
    Какое количество теплоты выделится в цепи после размыкания ключа $K$?
}
\answer{%
    \begin{align*}
    &\text{закон Ома для полной цепи}: \eli = \frac{\ele}{r + R_\text{внешнее}} = \frac{\ele}{r + \frac{R \cdot 0}{R + 0}} = \frac{\ele}{r}, \\
    Q &= W_m = \frac{L\eli^2}2 = \frac{L\sqr{\frac{\ele}{r}}}2 = \frac L2\frac{\ele^2}{r^2} = \frac{0{,}2\,\text{Гн}}2 \cdot \sqr{\frac{5\,\text{В}}{1\,\text{Ом}}} \approx 2{,}50\,\text{Дж}.
    \end{align*}
}

\variantsplitter

\addpersonalvariant{Матвей Кузьмин}

\tasknumber{1}%
\task{%
    Положительно заряженная частица движется со скоростью $v$ в магнитном поле перпендикулярно линиям его индукции.
    Индукция магнитного поля равна $B$, масса частицы $m$, её заряд — $q$.
    Выведите из базовых физических законов формулы для радиуса траектории частицы и её частоты обращения.
}
\answer{%
    $
        F = ma, F = qvB, a = v^2 / R \implies R = \frac{mv}{qB}.
        \quad T = \frac{2\pi R}{v} = \frac{2\pi m}{qB}.
        \quad \omega = \frac vR = \frac{qB}{m}.
        \quad \nu = \frac 1T = \frac{qB}{2\pi m}.
    $
}
\solutionspace{80pt}

\tasknumber{2}%
\task{%
    В однородном горизонтальном магнитном поле с индукцией $B = 50\,\text{мТл}$ находится проводник,
    расположенный также горизонтально и перпендикулярно полю.
    Какой ток необходимо пустить по проводнику, чтобы он завис?
    Масса единицы длины проводника $\rho = 10\,\frac{\text{г}}{\text{м}}$, $g = 10\,\frac{\text{м}}{\text{с}^{2}}$.
}
\answer{%
    $
            mg = B\eli l, m=\rho l
            \implies \eli
                = \frac{g\rho}B
                = \frac{10\,\frac{\text{м}}{\text{с}^{2}} \cdot 10\,\frac{\text{г}}{\text{м}}}{50\,\text{мТл}}
                = 2\,\text{А}.
    $
}
\solutionspace{80pt}

\tasknumber{3}%
\task{%
    Протон, прошедший через ускоряющую разность потенциалов, оказывается в магнитном поле индукцией $20\,\text{мТл}$
    и движется по окружности диаметром $4\,\text{мм}$.
    Сделайте рисунок, определите значение разности потенциалов
    и укажите, в какой области потенциал больше, а где меньше.
}
\solutionspace{80pt}

\tasknumber{4}%
\task{%
    Проводник лежит на горизонтальных рельсах,
    замкнутых резистором сопротивлением $4\,\text{Ом}$ (см.
    рис.
    на доске).
    Расстояние между рельсами $80\,\text{см}$.
    Конструкция помещена в вертикальное однородное магнитное поле индукцией $150\,\text{мТл}$.
    Какую силу необходимо прикладывать к проводнику, чтобы двигать его вдоль рельс с постоянной скоростью $2\,\frac{\text{м}}{\text{c}}$?
    Трением пренебречь, сопротивления рельс и проводника малы по сравнению с сопротивлением резистора.
    Ответ выразите в миллиньютонах.
}
\answer{%
    $
        F
            = F_A
            = \eli B l
            = \frac{\ele}R \cdot B l
            = \frac{B v l}R \cdot B l
            = \frac{B^2 v l^2}R
            = \frac{\sqr{150\,\text{мТл}} \cdot 2\,\frac{\text{м}}{\text{c}} \cdot \sqr{80\,\text{см}}}{4\,\text{Ом}}
            \approx 7{,}20\,\text{мН}.
    $
}
\solutionspace{120pt}

\tasknumber{5}%
\task{%
    При изменении силы тока в проводнике по закону $\eli = 7 - 0{,}4t$ (в системе СИ),
    в нём возникает ЭДС самоиндукции $150\,\text{мВ}$.
    Чему равна индуктивность проводника?
    Ответ выразите в миллигенри и округлите до целого.
}
\answer{%
    $
        \ele = L\frac{\abs{\Delta \eli}}{\Delta t} = L \cdot \abs{ - 0{,}4 } \text{(в СИ)}
        \implies L = \frac{\ele}{ 0{,}4 } = \frac{150\,\text{мВ}}{ 0{,}4 } \approx {375\,\text{мГн}}
    $
}
\solutionspace{80pt}

\tasknumber{6}%
\task{%
    Резистор сопротивлением $R = 4\,\text{Ом}$ и катушка индуктивностью $L = 0{,}5\,\text{Гн}$ (и пренебрежимо малым сопротивлением)
    подключены параллельно к источнику тока с ЭДС $\ele = 8\,\text{В}$ и внутренним сопротивлением $r = 2\,\text{Ом}$ (см.
    рис.
    на доске).
    Какое количество теплоты выделится в цепи после размыкания ключа $K$?
}
\answer{%
    \begin{align*}
    &\text{закон Ома для полной цепи}: \eli = \frac{\ele}{r + R_\text{внешнее}} = \frac{\ele}{r + \frac{R \cdot 0}{R + 0}} = \frac{\ele}{r}, \\
    Q &= W_m = \frac{L\eli^2}2 = \frac{L\sqr{\frac{\ele}{r}}}2 = \frac L2\frac{\ele^2}{r^2} = \frac{0{,}5\,\text{Гн}}2 \cdot \sqr{\frac{8\,\text{В}}{2\,\text{Ом}}} \approx 4\,\text{Дж}.
    \end{align*}
}

\variantsplitter

\addpersonalvariant{Сергей Малышев}

\tasknumber{1}%
\task{%
    Положительно заряженная частица движется со скоростью $v$ в магнитном поле перпендикулярно линиям его индукции.
    Индукция магнитного поля равна $B$, масса частицы $m$, её заряд — $q$.
    Выведите из базовых физических законов формулы для радиуса траектории частицы и её частоты обращения.
}
\answer{%
    $
        F = ma, F = qvB, a = v^2 / R \implies R = \frac{mv}{qB}.
        \quad T = \frac{2\pi R}{v} = \frac{2\pi m}{qB}.
        \quad \omega = \frac vR = \frac{qB}{m}.
        \quad \nu = \frac 1T = \frac{qB}{2\pi m}.
    $
}
\solutionspace{80pt}

\tasknumber{2}%
\task{%
    В однородном горизонтальном магнитном поле с индукцией $B = 100\,\text{мТл}$ находится проводник,
    расположенный также горизонтально и перпендикулярно полю.
    Какой ток необходимо пустить по проводнику, чтобы он завис?
    Масса единицы длины проводника $\rho = 10\,\frac{\text{г}}{\text{м}}$, $g = 10\,\frac{\text{м}}{\text{с}^{2}}$.
}
\answer{%
    $
            mg = B\eli l, m=\rho l
            \implies \eli
                = \frac{g\rho}B
                = \frac{10\,\frac{\text{м}}{\text{с}^{2}} \cdot 10\,\frac{\text{г}}{\text{м}}}{100\,\text{мТл}}
                = 1\,\text{А}.
    $
}
\solutionspace{80pt}

\tasknumber{3}%
\task{%
    Протон, прошедший через ускоряющую разность потенциалов, оказывается в магнитном поле индукцией $50\,\text{мТл}$
    и движется по окружности диаметром $6\,\text{мм}$.
    Сделайте рисунок, определите значение разности потенциалов
    и укажите, в какой области потенциал больше, а где меньше.
}
\solutionspace{80pt}

\tasknumber{4}%
\task{%
    Проводник лежит на горизонтальных рельсах,
    замкнутых резистором сопротивлением $4\,\text{Ом}$ (см.
    рис.
    на доске).
    Расстояние между рельсами $80\,\text{см}$.
    Конструкция помещена в вертикальное однородное магнитное поле индукцией $200\,\text{мТл}$.
    Какую силу необходимо прикладывать к проводнику, чтобы двигать его вдоль рельс с постоянной скоростью $3\,\frac{\text{м}}{\text{c}}$?
    Трением пренебречь, сопротивления рельс и проводника малы по сравнению с сопротивлением резистора.
    Ответ выразите в миллиньютонах.
}
\answer{%
    $
        F
            = F_A
            = \eli B l
            = \frac{\ele}R \cdot B l
            = \frac{B v l}R \cdot B l
            = \frac{B^2 v l^2}R
            = \frac{\sqr{200\,\text{мТл}} \cdot 3\,\frac{\text{м}}{\text{c}} \cdot \sqr{80\,\text{см}}}{4\,\text{Ом}}
            \approx 19{,}20\,\text{мН}.
    $
}
\solutionspace{120pt}

\tasknumber{5}%
\task{%
    При изменении силы тока в проводнике по закону $\eli = 5 + 0{,}5t$ (в системе СИ),
    в нём возникает ЭДС самоиндукции $150\,\text{мВ}$.
    Чему равна индуктивность проводника?
    Ответ выразите в миллигенри и округлите до целого.
}
\answer{%
    $
        \ele = L\frac{\abs{\Delta \eli}}{\Delta t} = L \cdot \abs{ + 0{,}5 } \text{(в СИ)}
        \implies L = \frac{\ele}{ 0{,}5 } = \frac{150\,\text{мВ}}{ 0{,}5 } \approx {300\,\text{мГн}}
    $
}
\solutionspace{80pt}

\tasknumber{6}%
\task{%
    Резистор сопротивлением $R = 4\,\text{Ом}$ и катушка индуктивностью $L = 0{,}4\,\text{Гн}$ (и пренебрежимо малым сопротивлением)
    подключены параллельно к источнику тока с ЭДС $\ele = 6\,\text{В}$ и внутренним сопротивлением $r = 2\,\text{Ом}$ (см.
    рис.
    на доске).
    Какое количество теплоты выделится в цепи после размыкания ключа $K$?
}
\answer{%
    \begin{align*}
    &\text{закон Ома для полной цепи}: \eli = \frac{\ele}{r + R_\text{внешнее}} = \frac{\ele}{r + \frac{R \cdot 0}{R + 0}} = \frac{\ele}{r}, \\
    Q &= W_m = \frac{L\eli^2}2 = \frac{L\sqr{\frac{\ele}{r}}}2 = \frac L2\frac{\ele^2}{r^2} = \frac{0{,}4\,\text{Гн}}2 \cdot \sqr{\frac{6\,\text{В}}{2\,\text{Ом}}} \approx 1{,}80\,\text{Дж}.
    \end{align*}
}

\variantsplitter

\addpersonalvariant{Алина Полканова}

\tasknumber{1}%
\task{%
    Положительно заряженная частица движется со скоростью $v$ в магнитном поле перпендикулярно линиям его индукции.
    Индукция магнитного поля равна $B$, масса частицы $m$, её заряд — $q$.
    Выведите из базовых физических законов формулы для радиуса траектории частицы и её периода обращения.
}
\answer{%
    $
        F = ma, F = qvB, a = v^2 / R \implies R = \frac{mv}{qB}.
        \quad T = \frac{2\pi R}{v} = \frac{2\pi m}{qB}.
        \quad \omega = \frac vR = \frac{qB}{m}.
        \quad \nu = \frac 1T = \frac{qB}{2\pi m}.
    $
}
\solutionspace{80pt}

\tasknumber{2}%
\task{%
    В однородном горизонтальном магнитном поле с индукцией $B = 20\,\text{мТл}$ находится проводник,
    расположенный также горизонтально и перпендикулярно полю.
    Какой ток необходимо пустить по проводнику, чтобы он завис?
    Масса единицы длины проводника $\rho = 20\,\frac{\text{г}}{\text{м}}$, $g = 10\,\frac{\text{м}}{\text{с}^{2}}$.
}
\answer{%
    $
            mg = B\eli l, m=\rho l
            \implies \eli
                = \frac{g\rho}B
                = \frac{10\,\frac{\text{м}}{\text{с}^{2}} \cdot 20\,\frac{\text{г}}{\text{м}}}{20\,\text{мТл}}
                = 10\,\text{А}.
    $
}
\solutionspace{80pt}

\tasknumber{3}%
\task{%
    Протон, прошедший через ускоряющую разность потенциалов, оказывается в магнитном поле индукцией $40\,\text{мТл}$
    и движется по окружности диаметром $6\,\text{мм}$.
    Сделайте рисунок, определите значение разности потенциалов
    и укажите, в какой области потенциал больше, а где меньше.
}
\solutionspace{80pt}

\tasknumber{4}%
\task{%
    Проводник лежит на горизонтальных рельсах,
    замкнутых резистором сопротивлением $3\,\text{Ом}$ (см.
    рис.
    на доске).
    Расстояние между рельсами $60\,\text{см}$.
    Конструкция помещена в вертикальное однородное магнитное поле индукцией $200\,\text{мТл}$.
    Какую силу необходимо прикладывать к проводнику, чтобы двигать его вдоль рельс с постоянной скоростью $4\,\frac{\text{м}}{\text{c}}$?
    Трением пренебречь, сопротивления рельс и проводника малы по сравнению с сопротивлением резистора.
    Ответ выразите в миллиньютонах.
}
\answer{%
    $
        F
            = F_A
            = \eli B l
            = \frac{\ele}R \cdot B l
            = \frac{B v l}R \cdot B l
            = \frac{B^2 v l^2}R
            = \frac{\sqr{200\,\text{мТл}} \cdot 4\,\frac{\text{м}}{\text{c}} \cdot \sqr{60\,\text{см}}}{3\,\text{Ом}}
            \approx 19{,}20\,\text{мН}.
    $
}
\solutionspace{120pt}

\tasknumber{5}%
\task{%
    При изменении силы тока в проводнике по закону $\eli = 3 - 1{,}5t$ (в системе СИ),
    в нём возникает ЭДС самоиндукции $300\,\text{мВ}$.
    Чему равна индуктивность проводника?
    Ответ выразите в миллигенри и округлите до целого.
}
\answer{%
    $
        \ele = L\frac{\abs{\Delta \eli}}{\Delta t} = L \cdot \abs{ - 1{,}5 } \text{(в СИ)}
        \implies L = \frac{\ele}{ 1{,}5 } = \frac{300\,\text{мВ}}{ 1{,}5 } \approx {200\,\text{мГн}}
    $
}
\solutionspace{80pt}

\tasknumber{6}%
\task{%
    Резистор сопротивлением $R = 5\,\text{Ом}$ и катушка индуктивностью $L = 0{,}5\,\text{Гн}$ (и пренебрежимо малым сопротивлением)
    подключены параллельно к источнику тока с ЭДС $\ele = 12\,\text{В}$ и внутренним сопротивлением $r = 2\,\text{Ом}$ (см.
    рис.
    на доске).
    Какое количество теплоты выделится в цепи после размыкания ключа $K$?
}
\answer{%
    \begin{align*}
    &\text{закон Ома для полной цепи}: \eli = \frac{\ele}{r + R_\text{внешнее}} = \frac{\ele}{r + \frac{R \cdot 0}{R + 0}} = \frac{\ele}{r}, \\
    Q &= W_m = \frac{L\eli^2}2 = \frac{L\sqr{\frac{\ele}{r}}}2 = \frac L2\frac{\ele^2}{r^2} = \frac{0{,}5\,\text{Гн}}2 \cdot \sqr{\frac{12\,\text{В}}{2\,\text{Ом}}} \approx 9\,\text{Дж}.
    \end{align*}
}

\variantsplitter

\addpersonalvariant{Сергей Пономарёв}

\tasknumber{1}%
\task{%
    Положительно заряженная частица движется со скоростью $v$ в магнитном поле перпендикулярно линиям его индукции.
    Индукция магнитного поля равна $B$, масса частицы $m$, её заряд — $q$.
    Выведите из базовых физических законов формулы для радиуса траектории частицы и её угловой скорости.
}
\answer{%
    $
        F = ma, F = qvB, a = v^2 / R \implies R = \frac{mv}{qB}.
        \quad T = \frac{2\pi R}{v} = \frac{2\pi m}{qB}.
        \quad \omega = \frac vR = \frac{qB}{m}.
        \quad \nu = \frac 1T = \frac{qB}{2\pi m}.
    $
}
\solutionspace{80pt}

\tasknumber{2}%
\task{%
    В однородном горизонтальном магнитном поле с индукцией $B = 20\,\text{мТл}$ находится проводник,
    расположенный также горизонтально и перпендикулярно полю.
    Какой ток необходимо пустить по проводнику, чтобы он завис?
    Масса единицы длины проводника $\rho = 20\,\frac{\text{г}}{\text{м}}$, $g = 10\,\frac{\text{м}}{\text{с}^{2}}$.
}
\answer{%
    $
            mg = B\eli l, m=\rho l
            \implies \eli
                = \frac{g\rho}B
                = \frac{10\,\frac{\text{м}}{\text{с}^{2}} \cdot 20\,\frac{\text{г}}{\text{м}}}{20\,\text{мТл}}
                = 10\,\text{А}.
    $
}
\solutionspace{80pt}

\tasknumber{3}%
\task{%
    Протон, прошедший через ускоряющую разность потенциалов, оказывается в магнитном поле индукцией $50\,\text{мТл}$
    и движется по окружности диаметром $6\,\text{мм}$.
    Сделайте рисунок, определите значение разности потенциалов
    и укажите, в какой области потенциал больше, а где меньше.
}
\solutionspace{80pt}

\tasknumber{4}%
\task{%
    Проводник лежит на горизонтальных рельсах,
    замкнутых резистором сопротивлением $2\,\text{Ом}$ (см.
    рис.
    на доске).
    Расстояние между рельсами $50\,\text{см}$.
    Конструкция помещена в вертикальное однородное магнитное поле индукцией $200\,\text{мТл}$.
    Какую силу необходимо прикладывать к проводнику, чтобы двигать его вдоль рельс с постоянной скоростью $4\,\frac{\text{м}}{\text{c}}$?
    Трением пренебречь, сопротивления рельс и проводника малы по сравнению с сопротивлением резистора.
    Ответ выразите в миллиньютонах.
}
\answer{%
    $
        F
            = F_A
            = \eli B l
            = \frac{\ele}R \cdot B l
            = \frac{B v l}R \cdot B l
            = \frac{B^2 v l^2}R
            = \frac{\sqr{200\,\text{мТл}} \cdot 4\,\frac{\text{м}}{\text{c}} \cdot \sqr{50\,\text{см}}}{2\,\text{Ом}}
            \approx 20\,\text{мН}.
    $
}
\solutionspace{120pt}

\tasknumber{5}%
\task{%
    При изменении силы тока в проводнике по закону $\eli = 5 + 1{,}5t$ (в системе СИ),
    в нём возникает ЭДС самоиндукции $300\,\text{мВ}$.
    Чему равна индуктивность проводника?
    Ответ выразите в миллигенри и округлите до целого.
}
\answer{%
    $
        \ele = L\frac{\abs{\Delta \eli}}{\Delta t} = L \cdot \abs{ + 1{,}5 } \text{(в СИ)}
        \implies L = \frac{\ele}{ 1{,}5 } = \frac{300\,\text{мВ}}{ 1{,}5 } \approx {200\,\text{мГн}}
    $
}
\solutionspace{80pt}

\tasknumber{6}%
\task{%
    Резистор сопротивлением $R = 3\,\text{Ом}$ и катушка индуктивностью $L = 0{,}5\,\text{Гн}$ (и пренебрежимо малым сопротивлением)
    подключены параллельно к источнику тока с ЭДС $\ele = 12\,\text{В}$ и внутренним сопротивлением $r = 1\,\text{Ом}$ (см.
    рис.
    на доске).
    Какое количество теплоты выделится в цепи после размыкания ключа $K$?
}
\answer{%
    \begin{align*}
    &\text{закон Ома для полной цепи}: \eli = \frac{\ele}{r + R_\text{внешнее}} = \frac{\ele}{r + \frac{R \cdot 0}{R + 0}} = \frac{\ele}{r}, \\
    Q &= W_m = \frac{L\eli^2}2 = \frac{L\sqr{\frac{\ele}{r}}}2 = \frac L2\frac{\ele^2}{r^2} = \frac{0{,}5\,\text{Гн}}2 \cdot \sqr{\frac{12\,\text{В}}{1\,\text{Ом}}} \approx 36\,\text{Дж}.
    \end{align*}
}

\variantsplitter

\addpersonalvariant{Егор Свистушкин}

\tasknumber{1}%
\task{%
    Положительно заряженная частица движется со скоростью $v$ в магнитном поле перпендикулярно линиям его индукции.
    Индукция магнитного поля равна $B$, масса частицы $m$, её заряд — $q$.
    Выведите из базовых физических законов формулы для радиуса траектории частицы и её угловой скорости.
}
\answer{%
    $
        F = ma, F = qvB, a = v^2 / R \implies R = \frac{mv}{qB}.
        \quad T = \frac{2\pi R}{v} = \frac{2\pi m}{qB}.
        \quad \omega = \frac vR = \frac{qB}{m}.
        \quad \nu = \frac 1T = \frac{qB}{2\pi m}.
    $
}
\solutionspace{80pt}

\tasknumber{2}%
\task{%
    В однородном горизонтальном магнитном поле с индукцией $B = 20\,\text{мТл}$ находится проводник,
    расположенный также горизонтально и перпендикулярно полю.
    Какой ток необходимо пустить по проводнику, чтобы он завис?
    Масса единицы длины проводника $\rho = 40\,\frac{\text{г}}{\text{м}}$, $g = 10\,\frac{\text{м}}{\text{с}^{2}}$.
}
\answer{%
    $
            mg = B\eli l, m=\rho l
            \implies \eli
                = \frac{g\rho}B
                = \frac{10\,\frac{\text{м}}{\text{с}^{2}} \cdot 40\,\frac{\text{г}}{\text{м}}}{20\,\text{мТл}}
                = 20\,\text{А}.
    $
}
\solutionspace{80pt}

\tasknumber{3}%
\task{%
    Позитрон, прошедший через ускоряющую разность потенциалов, оказывается в магнитном поле индукцией $20\,\text{мТл}$
    и движется по окружности диаметром $8\,\text{мм}$.
    Сделайте рисунок, определите значение разности потенциалов
    и укажите, в какой области потенциал больше, а где меньше.
}
\solutionspace{80pt}

\tasknumber{4}%
\task{%
    Проводник лежит на горизонтальных рельсах,
    замкнутых резистором сопротивлением $2\,\text{Ом}$ (см.
    рис.
    на доске).
    Расстояние между рельсами $70\,\text{см}$.
    Конструкция помещена в вертикальное однородное магнитное поле индукцией $300\,\text{мТл}$.
    Какую силу необходимо прикладывать к проводнику, чтобы двигать его вдоль рельс с постоянной скоростью $4\,\frac{\text{м}}{\text{c}}$?
    Трением пренебречь, сопротивления рельс и проводника малы по сравнению с сопротивлением резистора.
    Ответ выразите в миллиньютонах.
}
\answer{%
    $
        F
            = F_A
            = \eli B l
            = \frac{\ele}R \cdot B l
            = \frac{B v l}R \cdot B l
            = \frac{B^2 v l^2}R
            = \frac{\sqr{300\,\text{мТл}} \cdot 4\,\frac{\text{м}}{\text{c}} \cdot \sqr{70\,\text{см}}}{2\,\text{Ом}}
            \approx 88{,}20\,\text{мН}.
    $
}
\solutionspace{120pt}

\tasknumber{5}%
\task{%
    При изменении силы тока в проводнике по закону $\eli = 2 + 0{,}4t$ (в системе СИ),
    в нём возникает ЭДС самоиндукции $300\,\text{мВ}$.
    Чему равна индуктивность проводника?
    Ответ выразите в миллигенри и округлите до целого.
}
\answer{%
    $
        \ele = L\frac{\abs{\Delta \eli}}{\Delta t} = L \cdot \abs{ + 0{,}4 } \text{(в СИ)}
        \implies L = \frac{\ele}{ 0{,}4 } = \frac{300\,\text{мВ}}{ 0{,}4 } \approx {750\,\text{мГн}}
    $
}
\solutionspace{80pt}

\tasknumber{6}%
\task{%
    Резистор сопротивлением $R = 3\,\text{Ом}$ и катушка индуктивностью $L = 0{,}2\,\text{Гн}$ (и пренебрежимо малым сопротивлением)
    подключены параллельно к источнику тока с ЭДС $\ele = 6\,\text{В}$ и внутренним сопротивлением $r = 1\,\text{Ом}$ (см.
    рис.
    на доске).
    Какое количество теплоты выделится в цепи после размыкания ключа $K$?
}
\answer{%
    \begin{align*}
    &\text{закон Ома для полной цепи}: \eli = \frac{\ele}{r + R_\text{внешнее}} = \frac{\ele}{r + \frac{R \cdot 0}{R + 0}} = \frac{\ele}{r}, \\
    Q &= W_m = \frac{L\eli^2}2 = \frac{L\sqr{\frac{\ele}{r}}}2 = \frac L2\frac{\ele^2}{r^2} = \frac{0{,}2\,\text{Гн}}2 \cdot \sqr{\frac{6\,\text{В}}{1\,\text{Ом}}} \approx 3{,}60\,\text{Дж}.
    \end{align*}
}

\variantsplitter

\addpersonalvariant{Дмитрий Соколов}

\tasknumber{1}%
\task{%
    Положительно заряженная частица движется со скоростью $v$ в магнитном поле перпендикулярно линиям его индукции.
    Индукция магнитного поля равна $B$, масса частицы $m$, её заряд — $q$.
    Выведите из базовых физических законов формулы для радиуса траектории частицы и её частоты обращения.
}
\answer{%
    $
        F = ma, F = qvB, a = v^2 / R \implies R = \frac{mv}{qB}.
        \quad T = \frac{2\pi R}{v} = \frac{2\pi m}{qB}.
        \quad \omega = \frac vR = \frac{qB}{m}.
        \quad \nu = \frac 1T = \frac{qB}{2\pi m}.
    $
}
\solutionspace{80pt}

\tasknumber{2}%
\task{%
    В однородном горизонтальном магнитном поле с индукцией $B = 100\,\text{мТл}$ находится проводник,
    расположенный также горизонтально и перпендикулярно полю.
    Какой ток необходимо пустить по проводнику, чтобы он завис?
    Масса единицы длины проводника $\rho = 100\,\frac{\text{г}}{\text{м}}$, $g = 10\,\frac{\text{м}}{\text{с}^{2}}$.
}
\answer{%
    $
            mg = B\eli l, m=\rho l
            \implies \eli
                = \frac{g\rho}B
                = \frac{10\,\frac{\text{м}}{\text{с}^{2}} \cdot 100\,\frac{\text{г}}{\text{м}}}{100\,\text{мТл}}
                = 10\,\text{А}.
    $
}
\solutionspace{80pt}

\tasknumber{3}%
\task{%
    Позитрон, прошедший через ускоряющую разность потенциалов, оказывается в магнитном поле индукцией $20\,\text{мТл}$
    и движется по окружности диаметром $4\,\text{мм}$.
    Сделайте рисунок, определите значение разности потенциалов
    и укажите, в какой области потенциал больше, а где меньше.
}
\solutionspace{80pt}

\tasknumber{4}%
\task{%
    Проводник лежит на горизонтальных рельсах,
    замкнутых резистором сопротивлением $3\,\text{Ом}$ (см.
    рис.
    на доске).
    Расстояние между рельсами $70\,\text{см}$.
    Конструкция помещена в вертикальное однородное магнитное поле индукцией $300\,\text{мТл}$.
    Какую силу необходимо прикладывать к проводнику, чтобы двигать его вдоль рельс с постоянной скоростью $2\,\frac{\text{м}}{\text{c}}$?
    Трением пренебречь, сопротивления рельс и проводника малы по сравнению с сопротивлением резистора.
    Ответ выразите в миллиньютонах.
}
\answer{%
    $
        F
            = F_A
            = \eli B l
            = \frac{\ele}R \cdot B l
            = \frac{B v l}R \cdot B l
            = \frac{B^2 v l^2}R
            = \frac{\sqr{300\,\text{мТл}} \cdot 2\,\frac{\text{м}}{\text{c}} \cdot \sqr{70\,\text{см}}}{3\,\text{Ом}}
            \approx 29{,}40\,\text{мН}.
    $
}
\solutionspace{120pt}

\tasknumber{5}%
\task{%
    При изменении силы тока в проводнике по закону $\eli = 7 - 0{,}8t$ (в системе СИ),
    в нём возникает ЭДС самоиндукции $300\,\text{мВ}$.
    Чему равна индуктивность проводника?
    Ответ выразите в миллигенри и округлите до целого.
}
\answer{%
    $
        \ele = L\frac{\abs{\Delta \eli}}{\Delta t} = L \cdot \abs{ - 0{,}8 } \text{(в СИ)}
        \implies L = \frac{\ele}{ 0{,}8 } = \frac{300\,\text{мВ}}{ 0{,}8 } \approx {375\,\text{мГн}}
    $
}
\solutionspace{80pt}

\tasknumber{6}%
\task{%
    Резистор сопротивлением $R = 5\,\text{Ом}$ и катушка индуктивностью $L = 0{,}2\,\text{Гн}$ (и пренебрежимо малым сопротивлением)
    подключены параллельно к источнику тока с ЭДС $\ele = 5\,\text{В}$ и внутренним сопротивлением $r = 1\,\text{Ом}$ (см.
    рис.
    на доске).
    Какое количество теплоты выделится в цепи после размыкания ключа $K$?
}
\answer{%
    \begin{align*}
    &\text{закон Ома для полной цепи}: \eli = \frac{\ele}{r + R_\text{внешнее}} = \frac{\ele}{r + \frac{R \cdot 0}{R + 0}} = \frac{\ele}{r}, \\
    Q &= W_m = \frac{L\eli^2}2 = \frac{L\sqr{\frac{\ele}{r}}}2 = \frac L2\frac{\ele^2}{r^2} = \frac{0{,}2\,\text{Гн}}2 \cdot \sqr{\frac{5\,\text{В}}{1\,\text{Ом}}} \approx 2{,}50\,\text{Дж}.
    \end{align*}
}

\variantsplitter

\addpersonalvariant{Арсений Трофимов}

\tasknumber{1}%
\task{%
    Положительно заряженная частица движется со скоростью $v$ в магнитном поле перпендикулярно линиям его индукции.
    Индукция магнитного поля равна $B$, масса частицы $m$, её заряд — $q$.
    Выведите из базовых физических законов формулы для радиуса траектории частицы и её угловой скорости.
}
\answer{%
    $
        F = ma, F = qvB, a = v^2 / R \implies R = \frac{mv}{qB}.
        \quad T = \frac{2\pi R}{v} = \frac{2\pi m}{qB}.
        \quad \omega = \frac vR = \frac{qB}{m}.
        \quad \nu = \frac 1T = \frac{qB}{2\pi m}.
    $
}
\solutionspace{80pt}

\tasknumber{2}%
\task{%
    В однородном горизонтальном магнитном поле с индукцией $B = 100\,\text{мТл}$ находится проводник,
    расположенный также горизонтально и перпендикулярно полю.
    Какой ток необходимо пустить по проводнику, чтобы он завис?
    Масса единицы длины проводника $\rho = 20\,\frac{\text{г}}{\text{м}}$, $g = 10\,\frac{\text{м}}{\text{с}^{2}}$.
}
\answer{%
    $
            mg = B\eli l, m=\rho l
            \implies \eli
                = \frac{g\rho}B
                = \frac{10\,\frac{\text{м}}{\text{с}^{2}} \cdot 20\,\frac{\text{г}}{\text{м}}}{100\,\text{мТл}}
                = 2\,\text{А}.
    $
}
\solutionspace{80pt}

\tasknumber{3}%
\task{%
    Электрон, прошедший через ускоряющую разность потенциалов, оказывается в магнитном поле индукцией $40\,\text{мТл}$
    и движется по окружности диаметром $6\,\text{мм}$.
    Сделайте рисунок, определите значение разности потенциалов
    и укажите, в какой области потенциал больше, а где меньше.
}
\solutionspace{80pt}

\tasknumber{4}%
\task{%
    Проводник лежит на горизонтальных рельсах,
    замкнутых резистором сопротивлением $4\,\text{Ом}$ (см.
    рис.
    на доске).
    Расстояние между рельсами $50\,\text{см}$.
    Конструкция помещена в вертикальное однородное магнитное поле индукцией $150\,\text{мТл}$.
    Какую силу необходимо прикладывать к проводнику, чтобы двигать его вдоль рельс с постоянной скоростью $4\,\frac{\text{м}}{\text{c}}$?
    Трением пренебречь, сопротивления рельс и проводника малы по сравнению с сопротивлением резистора.
    Ответ выразите в миллиньютонах.
}
\answer{%
    $
        F
            = F_A
            = \eli B l
            = \frac{\ele}R \cdot B l
            = \frac{B v l}R \cdot B l
            = \frac{B^2 v l^2}R
            = \frac{\sqr{150\,\text{мТл}} \cdot 4\,\frac{\text{м}}{\text{c}} \cdot \sqr{50\,\text{см}}}{4\,\text{Ом}}
            \approx 5{,}62\,\text{мН}.
    $
}
\solutionspace{120pt}

\tasknumber{5}%
\task{%
    При изменении силы тока в проводнике по закону $\eli = 7 + 0{,}4t$ (в системе СИ),
    в нём возникает ЭДС самоиндукции $400\,\text{мВ}$.
    Чему равна индуктивность проводника?
    Ответ выразите в миллигенри и округлите до целого.
}
\answer{%
    $
        \ele = L\frac{\abs{\Delta \eli}}{\Delta t} = L \cdot \abs{ + 0{,}4 } \text{(в СИ)}
        \implies L = \frac{\ele}{ 0{,}4 } = \frac{400\,\text{мВ}}{ 0{,}4 } \approx {1000\,\text{мГн}}
    $
}
\solutionspace{80pt}

\tasknumber{6}%
\task{%
    Резистор сопротивлением $R = 3\,\text{Ом}$ и катушка индуктивностью $L = 0{,}4\,\text{Гн}$ (и пренебрежимо малым сопротивлением)
    подключены параллельно к источнику тока с ЭДС $\ele = 5\,\text{В}$ и внутренним сопротивлением $r = 1\,\text{Ом}$ (см.
    рис.
    на доске).
    Какое количество теплоты выделится в цепи после размыкания ключа $K$?
}
\answer{%
    \begin{align*}
    &\text{закон Ома для полной цепи}: \eli = \frac{\ele}{r + R_\text{внешнее}} = \frac{\ele}{r + \frac{R \cdot 0}{R + 0}} = \frac{\ele}{r}, \\
    Q &= W_m = \frac{L\eli^2}2 = \frac{L\sqr{\frac{\ele}{r}}}2 = \frac L2\frac{\ele^2}{r^2} = \frac{0{,}4\,\text{Гн}}2 \cdot \sqr{\frac{5\,\text{В}}{1\,\text{Ом}}} \approx 5\,\text{Дж}.
    \end{align*}
}
% autogenerated
