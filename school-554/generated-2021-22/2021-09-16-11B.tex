\setdate{16~сентября~2021}
\setclass{11«Б»}

\addpersonalvariant{Михаил Бурмистров}

\tasknumber{1}%
\task{%
    Рядом с каждой единицей измерения укажите физическую величину, которая в ней измеряется, и один из вариантов обозначений этой физической величины.
    \begin{enumerate}
        \item Кл,
        \item м/с,
        \item А.
    \end{enumerate}
}
\solutionspace{40pt}

\tasknumber{2}%
\task{%
    Запишите формулой закон для вычисления модуля силы, действующей
    на заряженную частицу, движущуюся в магнитном поле, и выразите из него значение угла.
}
\solutionspace{40pt}

\tasknumber{3}%
\task{%
    Запишите формулой закон Ампера и укажите
    для каждой величины её название и единицы измерения в системе СИ.
}
\solutionspace{80pt}

\tasknumber{4}%
\task{%
    В однородном горизонтальном магнитном поле с индукцией $B = 10\,\text{мТл}$ находится проводник,
    расположенный также горизонтально и перпендикулярно полю.
    Какой ток необходимо пустить по проводнику, чтобы он завис?
    Масса единицы длины проводника $\rho = 100\,\frac{\text{г}}{\text{м}}$, $g = 10\,\frac{\text{м}}{\text{с}^{2}}$.
}
\answer{%
    $
            mg = B\eli l, m=\rho l
            \implies \eli
                = \frac{g\rho}B
                = \frac{10\,\frac{\text{м}}{\text{с}^{2}} \cdot 100\,\frac{\text{г}}{\text{м}}}{10\,\text{мТл}}
                = 100{,}0\,\text{А}.
    $
}
\solutionspace{120pt}

\tasknumber{5}%
\task{%
    Частица, обладающая массой $m$ и положительным зарядом $q$, движется со скоростью $v$
    в магнитном поле перпендикулярно линиям его индукции.
    Индукция магнитного поля равна $B$.
    Выведите из базовых физических законов формулы для радиуса траектории частицы: сделайте рисунок, укажите вид движения и названия физических законов.
}
\answer{%
    $F = ma, F = qvB, a = v^2 / R \implies R = \frac{mv}{qB}.$
}
\solutionspace{100pt}

\tasknumber{6}%
\task{%
    Частица массой $m$ и зарядом $q$ влетает со скоростью $v$ в однородное магнитное поле индукцией $B$ перпендикулярно его линиям.
    Определите, за какое время вектор скорости частицы повернётся на $90\degrees$ (впервые, если таких моментов будет несколько).
}

\variantsplitter

\addpersonalvariant{Снежана Авдошина}

\tasknumber{1}%
\task{%
    Рядом с каждой единицей измерения укажите физическую величину, которая в ней измеряется, и один из вариантов обозначений этой физической величины.
    \begin{enumerate}
        \item Тл,
        \item м,
        \item А.
    \end{enumerate}
}
\solutionspace{40pt}

\tasknumber{2}%
\task{%
    Запишите формулой закон для вычисления модуля силы, действующей
    на проводник, по которому течёт электрический ток, в магнитном поле, и выразите из него значение угла.
}
\solutionspace{40pt}

\tasknumber{3}%
\task{%
    Запишите формулой закон Лоренца и укажите
    для каждой величины её название и единицы измерения в системе СИ.
}
\solutionspace{80pt}

\tasknumber{4}%
\task{%
    В однородном горизонтальном магнитном поле с индукцией $B = 20\,\text{мТл}$ находится проводник,
    расположенный также горизонтально и перпендикулярно полю.
    Какой ток необходимо пустить по проводнику, чтобы он завис?
    Масса единицы длины проводника $\rho = 40\,\frac{\text{г}}{\text{м}}$, $g = 10\,\frac{\text{м}}{\text{с}^{2}}$.
}
\answer{%
    $
            mg = B\eli l, m=\rho l
            \implies \eli
                = \frac{g\rho}B
                = \frac{10\,\frac{\text{м}}{\text{с}^{2}} \cdot 40\,\frac{\text{г}}{\text{м}}}{20\,\text{мТл}}
                = 20{,}0\,\text{А}.
    $
}
\solutionspace{120pt}

\tasknumber{5}%
\task{%
    Частица, обладающая массой $m$ и положительным зарядом $q$, движется со скоростью $v$
    в магнитном поле перпендикулярно линиям его индукции.
    Индукция магнитного поля равна $B$.
    Выведите из базовых физических законов формулы для радиуса траектории частицы: сделайте рисунок, укажите вид движения и названия физических законов.
}
\answer{%
    $F = ma, F = qvB, a = v^2 / R \implies R = \frac{mv}{qB}.$
}
\solutionspace{100pt}

\tasknumber{6}%
\task{%
    Частица массой $m$ и зарядом $q$ влетает со скоростью $v$ в однородное магнитное поле индукцией $B$ перпендикулярно его линиям.
    Определите, за какое время вектор скорости частицы повернётся на $60\degrees$ (впервые, если таких моментов будет несколько).
}

\variantsplitter

\addpersonalvariant{Марьяна Аристова}

\tasknumber{1}%
\task{%
    Рядом с каждой единицей измерения укажите физическую величину, которая в ней измеряется, и один из вариантов обозначений этой физической величины.
    \begin{enumerate}
        \item Тл,
        \item м,
        \item А.
    \end{enumerate}
}
\solutionspace{40pt}

\tasknumber{2}%
\task{%
    Запишите формулой закон для вычисления модуля силы, действующей
    на проводник, по которому течёт электрический ток, в магнитном поле, и выразите из него значение угла.
}
\solutionspace{40pt}

\tasknumber{3}%
\task{%
    Запишите формулой закон Лоренца и укажите
    для каждой величины её название и единицы измерения в системе СИ.
}
\solutionspace{80pt}

\tasknumber{4}%
\task{%
    В однородном горизонтальном магнитном поле с индукцией $B = 50\,\text{мТл}$ находится проводник,
    расположенный также горизонтально и перпендикулярно полю.
    Какой ток необходимо пустить по проводнику, чтобы он завис?
    Масса единицы длины проводника $\rho = 100\,\frac{\text{г}}{\text{м}}$, $g = 10\,\frac{\text{м}}{\text{с}^{2}}$.
}
\answer{%
    $
            mg = B\eli l, m=\rho l
            \implies \eli
                = \frac{g\rho}B
                = \frac{10\,\frac{\text{м}}{\text{с}^{2}} \cdot 100\,\frac{\text{г}}{\text{м}}}{50\,\text{мТл}}
                = 20{,}0\,\text{А}.
    $
}
\solutionspace{120pt}

\tasknumber{5}%
\task{%
    Частица, обладающая массой $m$ и положительным зарядом $q$, движется со скоростью $v$
    в магнитном поле перпендикулярно линиям его индукции.
    Индукция магнитного поля равна $B$.
    Выведите из базовых физических законов формулы для радиуса траектории частицы: сделайте рисунок, укажите вид движения и названия физических законов.
}
\answer{%
    $F = ma, F = qvB, a = v^2 / R \implies R = \frac{mv}{qB}.$
}
\solutionspace{100pt}

\tasknumber{6}%
\task{%
    Частица массой $m$ и зарядом $q$ влетает со скоростью $v$ в однородное магнитное поле индукцией $B$ перпендикулярно его линиям.
    Определите, за какое время вектор скорости частицы повернётся на $135\degrees$ (впервые, если таких моментов будет несколько).
}

\variantsplitter

\addpersonalvariant{Никита Иванов}

\tasknumber{1}%
\task{%
    Рядом с каждой единицей измерения укажите физическую величину, которая в ней измеряется, и один из вариантов обозначений этой физической величины.
    \begin{enumerate}
        \item Кл,
        \item м/с,
        \item радиан.
    \end{enumerate}
}
\solutionspace{40pt}

\tasknumber{2}%
\task{%
    Запишите формулой закон для вычисления модуля силы, действующей
    на заряженную частицу, движущуюся в магнитном поле, и выразите из него индукцию магнитного поля.
}
\solutionspace{40pt}

\tasknumber{3}%
\task{%
    Запишите формулой закон Лоренца и укажите
    для каждой величины её название и единицы измерения в системе СИ.
}
\solutionspace{80pt}

\tasknumber{4}%
\task{%
    В однородном горизонтальном магнитном поле с индукцией $B = 20\,\text{мТл}$ находится проводник,
    расположенный также горизонтально и перпендикулярно полю.
    Какой ток необходимо пустить по проводнику, чтобы он завис?
    Масса единицы длины проводника $\rho = 5\,\frac{\text{г}}{\text{м}}$, $g = 10\,\frac{\text{м}}{\text{с}^{2}}$.
}
\answer{%
    $
            mg = B\eli l, m=\rho l
            \implies \eli
                = \frac{g\rho}B
                = \frac{10\,\frac{\text{м}}{\text{с}^{2}} \cdot 5\,\frac{\text{г}}{\text{м}}}{20\,\text{мТл}}
                = 2{,}5\,\text{А}.
    $
}
\solutionspace{120pt}

\tasknumber{5}%
\task{%
    Частица, обладающая массой $m$ и положительным зарядом $q$, движется со скоростью $v$
    в магнитном поле перпендикулярно линиям его индукции.
    Индукция магнитного поля равна $B$.
    Выведите из базовых физических законов формулы для радиуса траектории частицы: сделайте рисунок, укажите вид движения и названия физических законов.
}
\answer{%
    $F = ma, F = qvB, a = v^2 / R \implies R = \frac{mv}{qB}.$
}
\solutionspace{100pt}

\tasknumber{6}%
\task{%
    Частица массой $m$ и зарядом $q$ влетает со скоростью $v$ в однородное магнитное поле индукцией $B$ перпендикулярно его линиям.
    Определите, за какое время вектор скорости частицы повернётся на $60\degrees$ (впервые, если таких моментов будет несколько).
}

\variantsplitter

\addpersonalvariant{Анастасия Князева}

\tasknumber{1}%
\task{%
    Рядом с каждой единицей измерения укажите физическую величину, которая в ней измеряется, и один из вариантов обозначений этой физической величины.
    \begin{enumerate}
        \item Тл,
        \item м,
        \item А.
    \end{enumerate}
}
\solutionspace{40pt}

\tasknumber{2}%
\task{%
    Запишите формулой закон для вычисления модуля силы, действующей
    на проводник, по которому течёт электрический ток, в магнитном поле, и выразите из него значение угла.
}
\solutionspace{40pt}

\tasknumber{3}%
\task{%
    Запишите формулой закон Ампера и укажите
    для каждой величины её название и единицы измерения в системе СИ.
}
\solutionspace{80pt}

\tasknumber{4}%
\task{%
    В однородном горизонтальном магнитном поле с индукцией $B = 100\,\text{мТл}$ находится проводник,
    расположенный также горизонтально и перпендикулярно полю.
    Какой ток необходимо пустить по проводнику, чтобы он завис?
    Масса единицы длины проводника $\rho = 40\,\frac{\text{г}}{\text{м}}$, $g = 10\,\frac{\text{м}}{\text{с}^{2}}$.
}
\answer{%
    $
            mg = B\eli l, m=\rho l
            \implies \eli
                = \frac{g\rho}B
                = \frac{10\,\frac{\text{м}}{\text{с}^{2}} \cdot 40\,\frac{\text{г}}{\text{м}}}{100\,\text{мТл}}
                = 4{,}0\,\text{А}.
    $
}
\solutionspace{120pt}

\tasknumber{5}%
\task{%
    Частица, обладающая массой $m$ и положительным зарядом $q$, движется со скоростью $v$
    в магнитном поле перпендикулярно линиям его индукции.
    Индукция магнитного поля равна $B$.
    Выведите из базовых физических законов формулы для радиуса траектории частицы: сделайте рисунок, укажите вид движения и названия физических законов.
}
\answer{%
    $F = ma, F = qvB, a = v^2 / R \implies R = \frac{mv}{qB}.$
}
\solutionspace{100pt}

\tasknumber{6}%
\task{%
    Частица массой $m$ и зарядом $q$ влетает со скоростью $v$ в однородное магнитное поле индукцией $B$ перпендикулярно его линиям.
    Определите, за какое время вектор скорости частицы повернётся на $45\degrees$ (впервые, если таких моментов будет несколько).
}

\variantsplitter

\addpersonalvariant{Матвей Кузьмин}

\tasknumber{1}%
\task{%
    Рядом с каждой единицей измерения укажите физическую величину, которая в ней измеряется, и один из вариантов обозначений этой физической величины.
    \begin{enumerate}
        \item Тл,
        \item м/с,
        \item А.
    \end{enumerate}
}
\solutionspace{40pt}

\tasknumber{2}%
\task{%
    Запишите формулой закон для вычисления модуля силы, действующей
    на проводник, по которому течёт электрический ток, в магнитном поле, и выразите из него индукцию магнитного поля.
}
\solutionspace{40pt}

\tasknumber{3}%
\task{%
    Запишите формулой закон Лоренца и укажите
    для каждой величины её название и единицы измерения в системе СИ.
}
\solutionspace{80pt}

\tasknumber{4}%
\task{%
    В однородном горизонтальном магнитном поле с индукцией $B = 20\,\text{мТл}$ находится проводник,
    расположенный также горизонтально и перпендикулярно полю.
    Какой ток необходимо пустить по проводнику, чтобы он завис?
    Масса единицы длины проводника $\rho = 5\,\frac{\text{г}}{\text{м}}$, $g = 10\,\frac{\text{м}}{\text{с}^{2}}$.
}
\answer{%
    $
            mg = B\eli l, m=\rho l
            \implies \eli
                = \frac{g\rho}B
                = \frac{10\,\frac{\text{м}}{\text{с}^{2}} \cdot 5\,\frac{\text{г}}{\text{м}}}{20\,\text{мТл}}
                = 2{,}5\,\text{А}.
    $
}
\solutionspace{120pt}

\tasknumber{5}%
\task{%
    Частица, обладающая массой $m$ и положительным зарядом $q$, движется со скоростью $v$
    в магнитном поле перпендикулярно линиям его индукции.
    Индукция магнитного поля равна $B$.
    Выведите из базовых физических законов формулы для радиуса траектории частицы: сделайте рисунок, укажите вид движения и названия физических законов.
}
\answer{%
    $F = ma, F = qvB, a = v^2 / R \implies R = \frac{mv}{qB}.$
}
\solutionspace{100pt}

\tasknumber{6}%
\task{%
    Частица массой $m$ и зарядом $q$ влетает со скоростью $v$ в однородное магнитное поле индукцией $B$ перпендикулярно его линиям.
    Определите, за какое время вектор скорости частицы повернётся на $120\degrees$ (впервые, если таких моментов будет несколько).
}

\variantsplitter

\addpersonalvariant{Елизавета Кутумова}

\tasknumber{1}%
\task{%
    Рядом с каждой единицей измерения укажите физическую величину, которая в ней измеряется, и один из вариантов обозначений этой физической величины.
    \begin{enumerate}
        \item Тл,
        \item м/с,
        \item А.
    \end{enumerate}
}
\solutionspace{40pt}

\tasknumber{2}%
\task{%
    Запишите формулой закон для вычисления модуля силы, действующей
    на проводник, по которому течёт электрический ток, в магнитном поле, и выразите из него индукцию магнитного поля.
}
\solutionspace{40pt}

\tasknumber{3}%
\task{%
    Запишите формулой закон Лоренца и укажите
    для каждой величины её название и единицы измерения в системе СИ.
}
\solutionspace{80pt}

\tasknumber{4}%
\task{%
    В однородном горизонтальном магнитном поле с индукцией $B = 10\,\text{мТл}$ находится проводник,
    расположенный также горизонтально и перпендикулярно полю.
    Какой ток необходимо пустить по проводнику, чтобы он завис?
    Масса единицы длины проводника $\rho = 40\,\frac{\text{г}}{\text{м}}$, $g = 10\,\frac{\text{м}}{\text{с}^{2}}$.
}
\answer{%
    $
            mg = B\eli l, m=\rho l
            \implies \eli
                = \frac{g\rho}B
                = \frac{10\,\frac{\text{м}}{\text{с}^{2}} \cdot 40\,\frac{\text{г}}{\text{м}}}{10\,\text{мТл}}
                = 40{,}0\,\text{А}.
    $
}
\solutionspace{120pt}

\tasknumber{5}%
\task{%
    Частица, обладающая массой $m$ и положительным зарядом $q$, движется со скоростью $v$
    в магнитном поле перпендикулярно линиям его индукции.
    Индукция магнитного поля равна $B$.
    Выведите из базовых физических законов формулы для радиуса траектории частицы: сделайте рисунок, укажите вид движения и названия физических законов.
}
\answer{%
    $F = ma, F = qvB, a = v^2 / R \implies R = \frac{mv}{qB}.$
}
\solutionspace{100pt}

\tasknumber{6}%
\task{%
    Частица массой $m$ и зарядом $q$ влетает со скоростью $v$ в однородное магнитное поле индукцией $B$ перпендикулярно его линиям.
    Определите, за какое время вектор скорости частицы повернётся на $60\degrees$ (впервые, если таких моментов будет несколько).
}

\variantsplitter

\addpersonalvariant{Роксана Мехтиева}

\tasknumber{1}%
\task{%
    Рядом с каждой единицей измерения укажите физическую величину, которая в ней измеряется, и один из вариантов обозначений этой физической величины.
    \begin{enumerate}
        \item Кл,
        \item м/с,
        \item А.
    \end{enumerate}
}
\solutionspace{40pt}

\tasknumber{2}%
\task{%
    Запишите формулой закон для вычисления модуля силы, действующей
    на проводник, по которому течёт электрический ток, в магнитном поле, и выразите из него значение угла.
}
\solutionspace{40pt}

\tasknumber{3}%
\task{%
    Запишите формулой закон Ампера и укажите
    для каждой величины её название и единицы измерения в системе СИ.
}
\solutionspace{80pt}

\tasknumber{4}%
\task{%
    В однородном горизонтальном магнитном поле с индукцией $B = 50\,\text{мТл}$ находится проводник,
    расположенный также горизонтально и перпендикулярно полю.
    Какой ток необходимо пустить по проводнику, чтобы он завис?
    Масса единицы длины проводника $\rho = 40\,\frac{\text{г}}{\text{м}}$, $g = 10\,\frac{\text{м}}{\text{с}^{2}}$.
}
\answer{%
    $
            mg = B\eli l, m=\rho l
            \implies \eli
                = \frac{g\rho}B
                = \frac{10\,\frac{\text{м}}{\text{с}^{2}} \cdot 40\,\frac{\text{г}}{\text{м}}}{50\,\text{мТл}}
                = 8{,}0\,\text{А}.
    $
}
\solutionspace{120pt}

\tasknumber{5}%
\task{%
    Частица, обладающая массой $m$ и положительным зарядом $q$, движется со скоростью $v$
    в магнитном поле перпендикулярно линиям его индукции.
    Индукция магнитного поля равна $B$.
    Выведите из базовых физических законов формулы для радиуса траектории частицы: сделайте рисунок, укажите вид движения и названия физических законов.
}
\answer{%
    $F = ma, F = qvB, a = v^2 / R \implies R = \frac{mv}{qB}.$
}
\solutionspace{100pt}

\tasknumber{6}%
\task{%
    Частица массой $m$ и зарядом $q$ влетает со скоростью $v$ в однородное магнитное поле индукцией $B$ перпендикулярно его линиям.
    Определите, за какое время вектор скорости частицы повернётся на $135\degrees$ (впервые, если таких моментов будет несколько).
}

\variantsplitter

\addpersonalvariant{Дилноза Нодиршоева}

\tasknumber{1}%
\task{%
    Рядом с каждой единицей измерения укажите физическую величину, которая в ней измеряется, и один из вариантов обозначений этой физической величины.
    \begin{enumerate}
        \item Кл,
        \item м,
        \item радиан.
    \end{enumerate}
}
\solutionspace{40pt}

\tasknumber{2}%
\task{%
    Запишите формулой закон для вычисления модуля силы, действующей
    на проводник, по которому течёт электрический ток, в магнитном поле, и выразите из него значение угла.
}
\solutionspace{40pt}

\tasknumber{3}%
\task{%
    Запишите формулой закон Ампера и укажите
    для каждой величины её название и единицы измерения в системе СИ.
}
\solutionspace{80pt}

\tasknumber{4}%
\task{%
    В однородном горизонтальном магнитном поле с индукцией $B = 20\,\text{мТл}$ находится проводник,
    расположенный также горизонтально и перпендикулярно полю.
    Какой ток необходимо пустить по проводнику, чтобы он завис?
    Масса единицы длины проводника $\rho = 10\,\frac{\text{г}}{\text{м}}$, $g = 10\,\frac{\text{м}}{\text{с}^{2}}$.
}
\answer{%
    $
            mg = B\eli l, m=\rho l
            \implies \eli
                = \frac{g\rho}B
                = \frac{10\,\frac{\text{м}}{\text{с}^{2}} \cdot 10\,\frac{\text{г}}{\text{м}}}{20\,\text{мТл}}
                = 5{,}0\,\text{А}.
    $
}
\solutionspace{120pt}

\tasknumber{5}%
\task{%
    Частица, обладающая массой $m$ и положительным зарядом $q$, движется со скоростью $v$
    в магнитном поле перпендикулярно линиям его индукции.
    Индукция магнитного поля равна $B$.
    Выведите из базовых физических законов формулы для радиуса траектории частицы: сделайте рисунок, укажите вид движения и названия физических законов.
}
\answer{%
    $F = ma, F = qvB, a = v^2 / R \implies R = \frac{mv}{qB}.$
}
\solutionspace{100pt}

\tasknumber{6}%
\task{%
    Частица массой $m$ и зарядом $q$ влетает со скоростью $v$ в однородное магнитное поле индукцией $B$ перпендикулярно его линиям.
    Определите, за какое время вектор скорости частицы повернётся на $60\degrees$ (впервые, если таких моментов будет несколько).
}

\variantsplitter

\addpersonalvariant{Артём Переверзев}

\tasknumber{1}%
\task{%
    Рядом с каждой единицей измерения укажите физическую величину, которая в ней измеряется, и один из вариантов обозначений этой физической величины.
    \begin{enumerate}
        \item Кл,
        \item м/с,
        \item А.
    \end{enumerate}
}
\solutionspace{40pt}

\tasknumber{2}%
\task{%
    Запишите формулой закон для вычисления модуля силы, действующей
    на проводник, по которому течёт электрический ток, в магнитном поле, и выразите из него индукцию магнитного поля.
}
\solutionspace{40pt}

\tasknumber{3}%
\task{%
    Запишите формулой закон Лоренца и укажите
    для каждой величины её название и единицы измерения в системе СИ.
}
\solutionspace{80pt}

\tasknumber{4}%
\task{%
    В однородном горизонтальном магнитном поле с индукцией $B = 50\,\text{мТл}$ находится проводник,
    расположенный также горизонтально и перпендикулярно полю.
    Какой ток необходимо пустить по проводнику, чтобы он завис?
    Масса единицы длины проводника $\rho = 40\,\frac{\text{г}}{\text{м}}$, $g = 10\,\frac{\text{м}}{\text{с}^{2}}$.
}
\answer{%
    $
            mg = B\eli l, m=\rho l
            \implies \eli
                = \frac{g\rho}B
                = \frac{10\,\frac{\text{м}}{\text{с}^{2}} \cdot 40\,\frac{\text{г}}{\text{м}}}{50\,\text{мТл}}
                = 8{,}0\,\text{А}.
    $
}
\solutionspace{120pt}

\tasknumber{5}%
\task{%
    Частица, обладающая массой $m$ и положительным зарядом $q$, движется со скоростью $v$
    в магнитном поле перпендикулярно линиям его индукции.
    Индукция магнитного поля равна $B$.
    Выведите из базовых физических законов формулы для радиуса траектории частицы: сделайте рисунок, укажите вид движения и названия физических законов.
}
\answer{%
    $F = ma, F = qvB, a = v^2 / R \implies R = \frac{mv}{qB}.$
}
\solutionspace{100pt}

\tasknumber{6}%
\task{%
    Частица массой $m$ и зарядом $q$ влетает со скоростью $v$ в однородное магнитное поле индукцией $B$ перпендикулярно его линиям.
    Определите, за какое время вектор скорости частицы повернётся на $135\degrees$ (впервые, если таких моментов будет несколько).
}

\variantsplitter

\addpersonalvariant{Варвара Пранова}

\tasknumber{1}%
\task{%
    Рядом с каждой единицей измерения укажите физическую величину, которая в ней измеряется, и один из вариантов обозначений этой физической величины.
    \begin{enumerate}
        \item Кл,
        \item м,
        \item А.
    \end{enumerate}
}
\solutionspace{40pt}

\tasknumber{2}%
\task{%
    Запишите формулой закон для вычисления модуля силы, действующей
    на заряженную частицу, движущуюся в магнитном поле, и выразите из него значение угла.
}
\solutionspace{40pt}

\tasknumber{3}%
\task{%
    Запишите формулой закон Лоренца и укажите
    для каждой величины её название и единицы измерения в системе СИ.
}
\solutionspace{80pt}

\tasknumber{4}%
\task{%
    В однородном горизонтальном магнитном поле с индукцией $B = 20\,\text{мТл}$ находится проводник,
    расположенный также горизонтально и перпендикулярно полю.
    Какой ток необходимо пустить по проводнику, чтобы он завис?
    Масса единицы длины проводника $\rho = 20\,\frac{\text{г}}{\text{м}}$, $g = 10\,\frac{\text{м}}{\text{с}^{2}}$.
}
\answer{%
    $
            mg = B\eli l, m=\rho l
            \implies \eli
                = \frac{g\rho}B
                = \frac{10\,\frac{\text{м}}{\text{с}^{2}} \cdot 20\,\frac{\text{г}}{\text{м}}}{20\,\text{мТл}}
                = 10{,}0\,\text{А}.
    $
}
\solutionspace{120pt}

\tasknumber{5}%
\task{%
    Частица, обладающая массой $m$ и положительным зарядом $q$, движется со скоростью $v$
    в магнитном поле перпендикулярно линиям его индукции.
    Индукция магнитного поля равна $B$.
    Выведите из базовых физических законов формулы для радиуса траектории частицы: сделайте рисунок, укажите вид движения и названия физических законов.
}
\answer{%
    $F = ma, F = qvB, a = v^2 / R \implies R = \frac{mv}{qB}.$
}
\solutionspace{100pt}

\tasknumber{6}%
\task{%
    Частица массой $m$ и зарядом $q$ влетает со скоростью $v$ в однородное магнитное поле индукцией $B$ перпендикулярно его линиям.
    Определите, за какое время вектор скорости частицы повернётся на $135\degrees$ (впервые, если таких моментов будет несколько).
}

\variantsplitter

\addpersonalvariant{Марьям Салимова}

\tasknumber{1}%
\task{%
    Рядом с каждой единицей измерения укажите физическую величину, которая в ней измеряется, и один из вариантов обозначений этой физической величины.
    \begin{enumerate}
        \item Тл,
        \item м,
        \item радиан.
    \end{enumerate}
}
\solutionspace{40pt}

\tasknumber{2}%
\task{%
    Запишите формулой закон для вычисления модуля силы, действующей
    на заряженную частицу, движущуюся в магнитном поле, и выразите из него значение угла.
}
\solutionspace{40pt}

\tasknumber{3}%
\task{%
    Запишите формулой закон Лоренца и укажите
    для каждой величины её название и единицы измерения в системе СИ.
}
\solutionspace{80pt}

\tasknumber{4}%
\task{%
    В однородном горизонтальном магнитном поле с индукцией $B = 10\,\text{мТл}$ находится проводник,
    расположенный также горизонтально и перпендикулярно полю.
    Какой ток необходимо пустить по проводнику, чтобы он завис?
    Масса единицы длины проводника $\rho = 5\,\frac{\text{г}}{\text{м}}$, $g = 10\,\frac{\text{м}}{\text{с}^{2}}$.
}
\answer{%
    $
            mg = B\eli l, m=\rho l
            \implies \eli
                = \frac{g\rho}B
                = \frac{10\,\frac{\text{м}}{\text{с}^{2}} \cdot 5\,\frac{\text{г}}{\text{м}}}{10\,\text{мТл}}
                = 5{,}0\,\text{А}.
    $
}
\solutionspace{120pt}

\tasknumber{5}%
\task{%
    Частица, обладающая массой $m$ и положительным зарядом $q$, движется со скоростью $v$
    в магнитном поле перпендикулярно линиям его индукции.
    Индукция магнитного поля равна $B$.
    Выведите из базовых физических законов формулы для радиуса траектории частицы: сделайте рисунок, укажите вид движения и названия физических законов.
}
\answer{%
    $F = ma, F = qvB, a = v^2 / R \implies R = \frac{mv}{qB}.$
}
\solutionspace{100pt}

\tasknumber{6}%
\task{%
    Частица массой $m$ и зарядом $q$ влетает со скоростью $v$ в однородное магнитное поле индукцией $B$ перпендикулярно его линиям.
    Определите, за какое время вектор скорости частицы повернётся на $90\degrees$ (впервые, если таких моментов будет несколько).
}

\variantsplitter

\addpersonalvariant{Юлия Шевченко}

\tasknumber{1}%
\task{%
    Рядом с каждой единицей измерения укажите физическую величину, которая в ней измеряется, и один из вариантов обозначений этой физической величины.
    \begin{enumerate}
        \item Кл,
        \item м,
        \item А.
    \end{enumerate}
}
\solutionspace{40pt}

\tasknumber{2}%
\task{%
    Запишите формулой закон для вычисления модуля силы, действующей
    на заряженную частицу, движущуюся в магнитном поле, и выразите из него значение угла.
}
\solutionspace{40pt}

\tasknumber{3}%
\task{%
    Запишите формулой закон Лоренца и укажите
    для каждой величины её название и единицы измерения в системе СИ.
}
\solutionspace{80pt}

\tasknumber{4}%
\task{%
    В однородном горизонтальном магнитном поле с индукцией $B = 10\,\text{мТл}$ находится проводник,
    расположенный также горизонтально и перпендикулярно полю.
    Какой ток необходимо пустить по проводнику, чтобы он завис?
    Масса единицы длины проводника $\rho = 10\,\frac{\text{г}}{\text{м}}$, $g = 10\,\frac{\text{м}}{\text{с}^{2}}$.
}
\answer{%
    $
            mg = B\eli l, m=\rho l
            \implies \eli
                = \frac{g\rho}B
                = \frac{10\,\frac{\text{м}}{\text{с}^{2}} \cdot 10\,\frac{\text{г}}{\text{м}}}{10\,\text{мТл}}
                = 10{,}0\,\text{А}.
    $
}
\solutionspace{120pt}

\tasknumber{5}%
\task{%
    Частица, обладающая массой $m$ и положительным зарядом $q$, движется со скоростью $v$
    в магнитном поле перпендикулярно линиям его индукции.
    Индукция магнитного поля равна $B$.
    Выведите из базовых физических законов формулы для радиуса траектории частицы: сделайте рисунок, укажите вид движения и названия физических законов.
}
\answer{%
    $F = ma, F = qvB, a = v^2 / R \implies R = \frac{mv}{qB}.$
}
\solutionspace{100pt}

\tasknumber{6}%
\task{%
    Частица массой $m$ и зарядом $q$ влетает со скоростью $v$ в однородное магнитное поле индукцией $B$ перпендикулярно его линиям.
    Определите, за какое время вектор скорости частицы повернётся на $90\degrees$ (впервые, если таких моментов будет несколько).
}
% autogenerated
