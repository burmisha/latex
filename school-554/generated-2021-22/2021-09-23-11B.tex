\setdate{23~сентября~2021}
\setclass{11«Б»}

\addpersonalvariant{Михаил Бурмистров}

\tasknumber{1}%
\task{%
    Установите каждой букве в соответствие ровно одну цифру и запишите ответ (только цифры, без других символов).

    А) вектор нормали к поверхности, Б) ЭДС индукции.

    1) $\ele$, 2) $\vec n$, 3) $S$, 4) $D$.
}
\answer{%
    $21$
}
\solutionspace{20pt}

\tasknumber{2}%
\task{%
    Установите каждой букве в соответствие ровно одну цифру и запишите ответ (только цифры, без других символов).

    А) индукция магнитного поля, Б) ЭДС индукции, В) площадь контура.

    1) $\text{м}^2$, 2) В, 3) Тл, 4) Ом, 5) А.
}
\answer{%
    $321$
}
\solutionspace{20pt}

\tasknumber{3}%
\task{%
    Однородное магнитное поле пронизывает плоский контур площадью $600\,\text{см}^{2}$.
    Индукция магнитного поля равна $500\,\text{мТл}$.
    Чему равен магнитный поток через контур, если его плоскость
    расположена под углом $90\degrees$ к вектору магнитной индукции?
    Ответ выразите в милливеберах и округлите до целого, единицы измерения писать не нужно.
}
\answer{%
    $\alpha = 0\degrees, \Phi_B = BS\cos\alpha = 30{,}00\,\text{мВб} \to 30$
}
\solutionspace{100pt}

\tasknumber{4}%
\task{%
    Определите притягивается (А), не взаимодействует (Б) или отталкивается (В) металлическое кольцо к магниту,
    если выдвигать южным полюсом (см.
    рис).
}
\answer{%
    \text{А}
}

\tasknumber{5}%
\task{%
    Определите притягивается (А), не взаимодействует (Б) или отталкивается (В) кольцо из диэлектрика к магниту,
    если выдвигать магнит из кольца южным полюсом (см.
    рис).
}
\answer{%
    \text{Б}
}

\tasknumber{6}%
\task{%
    Магнитный поток, пронизывающий замкнутый контур, равномерно изменяется от $35\,\text{мВб}$ до $110\,\text{мВб}$ за $1{,}5\,\text{c}$.
    Чему равна ЭДС в контуре? Ответ выразите в милливольтах и округлите до целого, единицы измерения писать не нужно.
}
\answer{%
    $
        \ele
        = \frac{\abs{\Delta \Phi}}{\Delta t}
        = \frac{\abs{\Phi_2 - \Phi_1}}{\Delta t}
        = \frac{\abs{110\,\text{мВб} - 35\,\text{мВб}}}{\Delta t}
        = 50{,}000\,\text{мВ} \to 50
    $
}
\solutionspace{60pt}

\tasknumber{7}%
\task{%
    Определите магнитный поток через контур,
    находящийся в однородном магнитном поле индукцией $500\,\text{мТл}$.
    Контур имеет форму прямоугольного треугольника с катетами $40\,\text{см}$ и $80\,\text{см}$.
    Угол между плоскостью контура и вектором индукции магнитного поля
    составляет $80\degrees$.
    Ответ выразите в милливеберах и округлите до целого, единицы измерения писать не нужно.
}
\answer{%
    $\alpha=10\degrees, \Phi_B = BS\cos\alpha = 78{,}78\,\text{мВб} \to 79$
}

\variantsplitter

\addpersonalvariant{Снежана Авдошина}

\tasknumber{1}%
\task{%
    Установите каждой букве в соответствие ровно одну цифру и запишите ответ (только цифры, без других символов).

    А) площадь контура, Б) индукция магнитного поля.

    1) $R$, 2) $v$, 3) $S$, 4) $B$.
}
\answer{%
    $34$
}
\solutionspace{20pt}

\tasknumber{2}%
\task{%
    Установите каждой букве в соответствие ровно одну цифру и запишите ответ (только цифры, без других символов).

    А) магнитный поток, Б) ЭДС индукции, В) площадь контура.

    1) Вб, 2) Тл, 3) Кл, 4) В, 5) $\text{м}^2$.
}
\answer{%
    $145$
}
\solutionspace{20pt}

\tasknumber{3}%
\task{%
    Однородное магнитное поле пронизывает плоский контур площадью $400\,\text{см}^{2}$.
    Индукция магнитного поля равна $300\,\text{мТл}$.
    Чему равен магнитный поток через контур, если его плоскость
    расположена под углом $30\degrees$ к вектору магнитной индукции?
    Ответ выразите в милливеберах и округлите до целого, единицы измерения писать не нужно.
}
\answer{%
    $\alpha = 60\degrees, \Phi_B = BS\cos\alpha = 6{,}00\,\text{мВб} \to 6$
}
\solutionspace{100pt}

\tasknumber{4}%
\task{%
    Определите притягивается (А), не взаимодействует (Б) или отталкивается (В) металлическое кольцо к магниту,
    если вдвигать южным полюсом (см.
    рис).
}
\answer{%
    \text{В}
}

\tasknumber{5}%
\task{%
    Определите притягивается (А), не взаимодействует (Б) или отталкивается (В) кольцо из диэлектрика к магниту,
    если выдвигать магнит из кольца южным полюсом (см.
    рис).
}
\answer{%
    \text{Б}
}

\tasknumber{6}%
\task{%
    Магнитный поток, пронизывающий замкнутый контур, равномерно изменяется от $65\,\text{мВб}$ до $20\,\text{мВб}$ за $1{,}3\,\text{c}$.
    Чему равна ЭДС в контуре? Ответ выразите в милливольтах и округлите до целого, единицы измерения писать не нужно.
}
\answer{%
    $
        \ele
        = \frac{\abs{\Delta \Phi}}{\Delta t}
        = \frac{\abs{\Phi_2 - \Phi_1}}{\Delta t}
        = \frac{\abs{20\,\text{мВб} - 65\,\text{мВб}}}{\Delta t}
        = 34{,}615\,\text{мВ} \to 35
    $
}
\solutionspace{60pt}

\tasknumber{7}%
\task{%
    Определите магнитный поток через контур,
    находящийся в однородном магнитном поле индукцией $700\,\text{мТл}$.
    Контур имеет форму прямоугольного треугольника с катетами $50\,\text{см}$ и $75\,\text{см}$.
    Угол между нормалью к плоскости контура и вектором индукции магнитного поля
    составляет $80\degrees$.
    Ответ выразите в милливеберах и округлите до целого, единицы измерения писать не нужно.
}
\answer{%
    $\alpha=80\degrees, \Phi_B = BS\cos\alpha = 22{,}79\,\text{мВб} \to 23$
}

\variantsplitter

\addpersonalvariant{Марьяна Аристова}

\tasknumber{1}%
\task{%
    Установите каждой букве в соответствие ровно одну цифру и запишите ответ (только цифры, без других символов).

    А) индукция магнитного поля, Б) индукционый ток.

    1) $\eli$, 2) $v$, 3) $l$, 4) $B$.
}
\answer{%
    $41$
}
\solutionspace{20pt}

\tasknumber{2}%
\task{%
    Установите каждой букве в соответствие ровно одну цифру и запишите ответ (только цифры, без других символов).

    А) магнитный поток, Б) индукция магнитного поля, В) индукционый ток.

    1) Тл, 2) А, 3) $\text{м}^2$, 4) Гн, 5) Вб.
}
\answer{%
    $512$
}
\solutionspace{20pt}

\tasknumber{3}%
\task{%
    Однородное магнитное поле пронизывает плоский контур площадью $600\,\text{см}^{2}$.
    Индукция магнитного поля равна $500\,\text{мТл}$.
    Чему равен магнитный поток через контур, если его плоскость
    расположена под углом $90\degrees$ к вектору магнитной индукции?
    Ответ выразите в милливеберах и округлите до целого, единицы измерения писать не нужно.
}
\answer{%
    $\alpha = 0\degrees, \Phi_B = BS\cos\alpha = 30{,}00\,\text{мВб} \to 30$
}
\solutionspace{100pt}

\tasknumber{4}%
\task{%
    Определите притягивается (А), не взаимодействует (Б) или отталкивается (В) металлическое кольцо к магниту,
    если выдвигать северным полюсом (см.
    рис).
}
\answer{%
    \text{А}
}

\tasknumber{5}%
\task{%
    Определите притягивается (А), не взаимодействует (Б) или отталкивается (В) кольцо из диэлектрика к магниту,
    если вдвигать магнит в кольцо южным полюсом (см.
    рис).
}
\answer{%
    \text{Б}
}

\tasknumber{6}%
\task{%
    Магнитный поток, пронизывающий замкнутый контур, равномерно изменяется от $35\,\text{мВб}$ до $110\,\text{мВб}$ за $1{,}3\,\text{c}$.
    Чему равна ЭДС в контуре? Ответ выразите в милливольтах и округлите до целого, единицы измерения писать не нужно.
}
\answer{%
    $
        \ele
        = \frac{\abs{\Delta \Phi}}{\Delta t}
        = \frac{\abs{\Phi_2 - \Phi_1}}{\Delta t}
        = \frac{\abs{110\,\text{мВб} - 35\,\text{мВб}}}{\Delta t}
        = 57{,}692\,\text{мВ} \to 58
    $
}
\solutionspace{60pt}

\tasknumber{7}%
\task{%
    Определите магнитный поток через контур,
    находящийся в однородном магнитном поле индукцией $500\,\text{мТл}$.
    Контур имеет форму прямоугольного треугольника с катетами $40\,\text{см}$ и $45\,\text{см}$.
    Угол между плоскостью контура и вектором индукции магнитного поля
    составляет $40\degrees$.
    Ответ выразите в милливеберах и округлите до целого, единицы измерения писать не нужно.
}
\answer{%
    $\alpha=50\degrees, \Phi_B = BS\cos\alpha = 28{,}93\,\text{мВб} \to 29$
}

\variantsplitter

\addpersonalvariant{Никита Иванов}

\tasknumber{1}%
\task{%
    Установите каждой букве в соответствие ровно одну цифру и запишите ответ (только цифры, без других символов).

    А) сопротивление контура, Б) индукционый ток.

    1) $R$, 2) $l$, 3) $\eli$, 4) $S$.
}
\answer{%
    $13$
}
\solutionspace{20pt}

\tasknumber{2}%
\task{%
    Установите каждой букве в соответствие ровно одну цифру и запишите ответ (только цифры, без других символов).

    А) магнитный поток, Б) индукционый ток, В) площадь контура.

    1) Гц, 2) Вб, 3) Гн, 4) А, 5) $\text{м}^2$.
}
\answer{%
    $245$
}
\solutionspace{20pt}

\tasknumber{3}%
\task{%
    Однородное магнитное поле пронизывает плоский контур площадью $600\,\text{см}^{2}$.
    Индукция магнитного поля равна $300\,\text{мТл}$.
    Чему равен магнитный поток через контур, если его плоскость
    расположена под углом $30\degrees$ к вектору магнитной индукции?
    Ответ выразите в милливеберах и округлите до целого, единицы измерения писать не нужно.
}
\answer{%
    $\alpha = 60\degrees, \Phi_B = BS\cos\alpha = 9{,}00\,\text{мВб} \to 9$
}
\solutionspace{100pt}

\tasknumber{4}%
\task{%
    Определите притягивается (А), не взаимодействует (Б) или отталкивается (В) металлическое кольцо к магниту,
    если вдвигать северным полюсом (см.
    рис).
}
\answer{%
    \text{В}
}

\tasknumber{5}%
\task{%
    Определите притягивается (А), не взаимодействует (Б) или отталкивается (В) кольцо из диэлектрика к магниту,
    если выдвигать магнит из кольца южным полюсом (см.
    рис).
}
\answer{%
    \text{Б}
}

\tasknumber{6}%
\task{%
    Магнитный поток, пронизывающий замкнутый контур, равномерно изменяется от $35\,\text{мВб}$ до $110\,\text{мВб}$ за $1{,}5\,\text{c}$.
    Чему равна ЭДС в контуре? Ответ выразите в милливольтах и округлите до целого, единицы измерения писать не нужно.
}
\answer{%
    $
        \ele
        = \frac{\abs{\Delta \Phi}}{\Delta t}
        = \frac{\abs{\Phi_2 - \Phi_1}}{\Delta t}
        = \frac{\abs{110\,\text{мВб} - 35\,\text{мВб}}}{\Delta t}
        = 50{,}000\,\text{мВ} \to 50
    $
}
\solutionspace{60pt}

\tasknumber{7}%
\task{%
    Определите магнитный поток через контур,
    находящийся в однородном магнитном поле индукцией $500\,\text{мТл}$.
    Контур имеет форму прямоугольного треугольника с катетами $40\,\text{см}$ и $75\,\text{см}$.
    Угол между нормалью к плоскости контура и вектором индукции магнитного поля
    составляет $40\degrees$.
    Ответ выразите в милливеберах и округлите до целого, единицы измерения писать не нужно.
}
\answer{%
    $\alpha=40\degrees, \Phi_B = BS\cos\alpha = 57{,}45\,\text{мВб} \to 57$
}

\variantsplitter

\addpersonalvariant{Анастасия Князева}

\tasknumber{1}%
\task{%
    Установите каждой букве в соответствие ровно одну цифру и запишите ответ (только цифры, без других символов).

    А) площадь контура, Б) ЭДС индукции.

    1) $S$, 2) $U$, 3) $\ele$, 4) $v$.
}
\answer{%
    $13$
}
\solutionspace{20pt}

\tasknumber{2}%
\task{%
    Установите каждой букве в соответствие ровно одну цифру и запишите ответ (только цифры, без других символов).

    А) индукция магнитного поля, Б) ЭДС индукции, В) индукционый ток.

    1) Тл, 2) В, 3) Гн, 4) А, 5) Вб.
}
\answer{%
    $124$
}
\solutionspace{20pt}

\tasknumber{3}%
\task{%
    Однородное магнитное поле пронизывает плоский контур площадью $200\,\text{см}^{2}$.
    Индукция магнитного поля равна $700\,\text{мТл}$.
    Чему равен магнитный поток через контур, если его плоскость
    расположена под углом $60\degrees$ к вектору магнитной индукции?
    Ответ выразите в милливеберах и округлите до целого, единицы измерения писать не нужно.
}
\answer{%
    $\alpha = 30\degrees, \Phi_B = BS\cos\alpha = 12{,}12\,\text{мВб} \to 12$
}
\solutionspace{100pt}

\tasknumber{4}%
\task{%
    Определите притягивается (А), не взаимодействует (Б) или отталкивается (В) металлическое кольцо к магниту,
    если выдвигать южным полюсом (см.
    рис).
}
\answer{%
    \text{А}
}

\tasknumber{5}%
\task{%
    Определите притягивается (А), не взаимодействует (Б) или отталкивается (В) кольцо из диэлектрика к магниту,
    если выдвигать магнит из кольца южным полюсом (см.
    рис).
}
\answer{%
    \text{Б}
}

\tasknumber{6}%
\task{%
    Магнитный поток, пронизывающий замкнутый контур, равномерно изменяется от $35\,\text{мВб}$ до $20\,\text{мВб}$ за $1{,}5\,\text{c}$.
    Чему равна ЭДС в контуре? Ответ выразите в милливольтах и округлите до целого, единицы измерения писать не нужно.
}
\answer{%
    $
        \ele
        = \frac{\abs{\Delta \Phi}}{\Delta t}
        = \frac{\abs{\Phi_2 - \Phi_1}}{\Delta t}
        = \frac{\abs{20\,\text{мВб} - 35\,\text{мВб}}}{\Delta t}
        = 10{,}000\,\text{мВ} \to 10
    $
}
\solutionspace{60pt}

\tasknumber{7}%
\task{%
    Определите магнитный поток через контур,
    находящийся в однородном магнитном поле индукцией $300\,\text{мТл}$.
    Контур имеет форму прямоугольного треугольника с катетами $60\,\text{см}$ и $45\,\text{см}$.
    Угол между плоскостью контура и вектором индукции магнитного поля
    составляет $50\degrees$.
    Ответ выразите в милливеберах и округлите до целого, единицы измерения писать не нужно.
}
\answer{%
    $\alpha=40\degrees, \Phi_B = BS\cos\alpha = 31{,}02\,\text{мВб} \to 31$
}

\variantsplitter

\addpersonalvariant{Матвей Кузьмин}

\tasknumber{1}%
\task{%
    Установите каждой букве в соответствие ровно одну цифру и запишите ответ (только цифры, без других символов).

    А) площадь контура, Б) ЭДС индукции.

    1) $\Phi$, 2) $v$, 3) $\ele$, 4) $S$.
}
\answer{%
    $43$
}
\solutionspace{20pt}

\tasknumber{2}%
\task{%
    Установите каждой букве в соответствие ровно одну цифру и запишите ответ (только цифры, без других символов).

    А) индукция магнитного поля, Б) площадь контура, В) индукционый ток.

    1) Гн, 2) А, 3) В, 4) $\text{м}^2$, 5) Тл.
}
\answer{%
    $542$
}
\solutionspace{20pt}

\tasknumber{3}%
\task{%
    Однородное магнитное поле пронизывает плоский контур площадью $400\,\text{см}^{2}$.
    Индукция магнитного поля равна $500\,\text{мТл}$.
    Чему равен магнитный поток через контур, если его плоскость
    расположена под углом $30\degrees$ к вектору магнитной индукции?
    Ответ выразите в милливеберах и округлите до целого, единицы измерения писать не нужно.
}
\answer{%
    $\alpha = 60\degrees, \Phi_B = BS\cos\alpha = 10{,}00\,\text{мВб} \to 10$
}
\solutionspace{100pt}

\tasknumber{4}%
\task{%
    Определите притягивается (А), не взаимодействует (Б) или отталкивается (В) металлическое кольцо к магниту,
    если выдвигать южным полюсом (см.
    рис).
}
\answer{%
    \text{А}
}

\tasknumber{5}%
\task{%
    Определите притягивается (А), не взаимодействует (Б) или отталкивается (В) кольцо из диэлектрика к магниту,
    если вдвигать магнит в кольцо южным полюсом (см.
    рис).
}
\answer{%
    \text{Б}
}

\tasknumber{6}%
\task{%
    Магнитный поток, пронизывающий замкнутый контур, равномерно изменяется от $95\,\text{мВб}$ до $80\,\text{мВб}$ за $1{,}5\,\text{c}$.
    Чему равна ЭДС в контуре? Ответ выразите в милливольтах и округлите до целого, единицы измерения писать не нужно.
}
\answer{%
    $
        \ele
        = \frac{\abs{\Delta \Phi}}{\Delta t}
        = \frac{\abs{\Phi_2 - \Phi_1}}{\Delta t}
        = \frac{\abs{80\,\text{мВб} - 95\,\text{мВб}}}{\Delta t}
        = 10{,}000\,\text{мВ} \to 10
    $
}
\solutionspace{60pt}

\tasknumber{7}%
\task{%
    Определите магнитный поток через контур,
    находящийся в однородном магнитном поле индукцией $700\,\text{мТл}$.
    Контур имеет форму прямоугольного треугольника с катетами $50\,\text{см}$ и $75\,\text{см}$.
    Угол между нормалью к плоскости контура и вектором индукции магнитного поля
    составляет $50\degrees$.
    Ответ выразите в милливеберах и округлите до целого, единицы измерения писать не нужно.
}
\answer{%
    $\alpha=50\degrees, \Phi_B = BS\cos\alpha = 84{,}37\,\text{мВб} \to 84$
}

\variantsplitter

\addpersonalvariant{Елизавета Кутумова}

\tasknumber{1}%
\task{%
    Установите каждой букве в соответствие ровно одну цифру и запишите ответ (только цифры, без других символов).

    А) площадь контура, Б) индукционый ток.

    1) $S$, 2) $\eli$, 3) $B$, 4) $l$.
}
\answer{%
    $12$
}
\solutionspace{20pt}

\tasknumber{2}%
\task{%
    Установите каждой букве в соответствие ровно одну цифру и запишите ответ (только цифры, без других символов).

    А) площадь контура, Б) ЭДС индукции, В) индукция магнитного поля.

    1) Гн, 2) $\text{м}^2$, 3) В, 4) Тл, 5) Кл.
}
\answer{%
    $234$
}
\solutionspace{20pt}

\tasknumber{3}%
\task{%
    Однородное магнитное поле пронизывает плоский контур площадью $600\,\text{см}^{2}$.
    Индукция магнитного поля равна $500\,\text{мТл}$.
    Чему равен магнитный поток через контур, если его плоскость
    расположена под углом $30\degrees$ к вектору магнитной индукции?
    Ответ выразите в милливеберах и округлите до целого, единицы измерения писать не нужно.
}
\answer{%
    $\alpha = 60\degrees, \Phi_B = BS\cos\alpha = 15{,}00\,\text{мВб} \to 15$
}
\solutionspace{100pt}

\tasknumber{4}%
\task{%
    Определите притягивается (А), не взаимодействует (Б) или отталкивается (В) металлическое кольцо к магниту,
    если вдвигать северным полюсом (см.
    рис).
}
\answer{%
    \text{В}
}

\tasknumber{5}%
\task{%
    Определите притягивается (А), не взаимодействует (Б) или отталкивается (В) кольцо из диэлектрика к магниту,
    если выдвигать магнит из кольца южным полюсом (см.
    рис).
}
\answer{%
    \text{Б}
}

\tasknumber{6}%
\task{%
    Магнитный поток, пронизывающий замкнутый контур, равномерно изменяется от $35\,\text{мВб}$ до $20\,\text{мВб}$ за $1{,}1\,\text{c}$.
    Чему равна ЭДС в контуре? Ответ выразите в милливольтах и округлите до целого, единицы измерения писать не нужно.
}
\answer{%
    $
        \ele
        = \frac{\abs{\Delta \Phi}}{\Delta t}
        = \frac{\abs{\Phi_2 - \Phi_1}}{\Delta t}
        = \frac{\abs{20\,\text{мВб} - 35\,\text{мВб}}}{\Delta t}
        = 13{,}636\,\text{мВ} \to 14
    $
}
\solutionspace{60pt}

\tasknumber{7}%
\task{%
    Определите магнитный поток через контур,
    находящийся в однородном магнитном поле индукцией $500\,\text{мТл}$.
    Контур имеет форму прямоугольного треугольника с катетами $40\,\text{см}$ и $75\,\text{см}$.
    Угол между плоскостью контура и вектором индукции магнитного поля
    составляет $70\degrees$.
    Ответ выразите в милливеберах и округлите до целого, единицы измерения писать не нужно.
}
\answer{%
    $\alpha=20\degrees, \Phi_B = BS\cos\alpha = 70{,}48\,\text{мВб} \to 70$
}

\variantsplitter

\addpersonalvariant{Роксана Мехтиева}

\tasknumber{1}%
\task{%
    Установите каждой букве в соответствие ровно одну цифру и запишите ответ (только цифры, без других символов).

    А) индукция магнитного поля, Б) вектор нормали к поверхности.

    1) $R$, 2) $B$, 3) $\vec n$, 4) $U$.
}
\answer{%
    $23$
}
\solutionspace{20pt}

\tasknumber{2}%
\task{%
    Установите каждой букве в соответствие ровно одну цифру и запишите ответ (только цифры, без других символов).

    А) индукция магнитного поля, Б) магнитный поток, В) площадь контура.

    1) Тл, 2) А, 3) Вб, 4) $\text{м}^2$, 5) Гц.
}
\answer{%
    $134$
}
\solutionspace{20pt}

\tasknumber{3}%
\task{%
    Однородное магнитное поле пронизывает плоский контур площадью $400\,\text{см}^{2}$.
    Индукция магнитного поля равна $300\,\text{мТл}$.
    Чему равен магнитный поток через контур, если его плоскость
    расположена под углом $60\degrees$ к вектору магнитной индукции?
    Ответ выразите в милливеберах и округлите до целого, единицы измерения писать не нужно.
}
\answer{%
    $\alpha = 30\degrees, \Phi_B = BS\cos\alpha = 10{,}39\,\text{мВб} \to 10$
}
\solutionspace{100pt}

\tasknumber{4}%
\task{%
    Определите притягивается (А), не взаимодействует (Б) или отталкивается (В) металлическое кольцо к магниту,
    если вдвигать южным полюсом (см.
    рис).
}
\answer{%
    \text{В}
}

\tasknumber{5}%
\task{%
    Определите притягивается (А), не взаимодействует (Б) или отталкивается (В) кольцо из диэлектрика к магниту,
    если выдвигать магнит из кольца северным полюсом (см.
    рис).
}
\answer{%
    \text{Б}
}

\tasknumber{6}%
\task{%
    Магнитный поток, пронизывающий замкнутый контур, равномерно изменяется от $35\,\text{мВб}$ до $50\,\text{мВб}$ за $1{,}1\,\text{c}$.
    Чему равна ЭДС в контуре? Ответ выразите в милливольтах и округлите до целого, единицы измерения писать не нужно.
}
\answer{%
    $
        \ele
        = \frac{\abs{\Delta \Phi}}{\Delta t}
        = \frac{\abs{\Phi_2 - \Phi_1}}{\Delta t}
        = \frac{\abs{50\,\text{мВб} - 35\,\text{мВб}}}{\Delta t}
        = 13{,}636\,\text{мВ} \to 14
    $
}
\solutionspace{60pt}

\tasknumber{7}%
\task{%
    Определите магнитный поток через контур,
    находящийся в однородном магнитном поле индукцией $300\,\text{мТл}$.
    Контур имеет форму прямоугольного треугольника с катетами $50\,\text{см}$ и $80\,\text{см}$.
    Угол между плоскостью контура и вектором индукции магнитного поля
    составляет $70\degrees$.
    Ответ выразите в милливеберах и округлите до целого, единицы измерения писать не нужно.
}
\answer{%
    $\alpha=20\degrees, \Phi_B = BS\cos\alpha = 56{,}38\,\text{мВб} \to 56$
}

\variantsplitter

\addpersonalvariant{Дилноза Нодиршоева}

\tasknumber{1}%
\task{%
    Установите каждой букве в соответствие ровно одну цифру и запишите ответ (только цифры, без других символов).

    А) площадь контура, Б) магнитный поток.

    1) $v$, 2) $\Phi$, 3) $\vec n$, 4) $S$.
}
\answer{%
    $42$
}
\solutionspace{20pt}

\tasknumber{2}%
\task{%
    Установите каждой букве в соответствие ровно одну цифру и запишите ответ (только цифры, без других символов).

    А) магнитный поток, Б) площадь контура, В) индукция магнитного поля.

    1) В, 2) Кл, 3) $\text{м}^2$, 4) Тл, 5) Вб.
}
\answer{%
    $534$
}
\solutionspace{20pt}

\tasknumber{3}%
\task{%
    Однородное магнитное поле пронизывает плоский контур площадью $800\,\text{см}^{2}$.
    Индукция магнитного поля равна $300\,\text{мТл}$.
    Чему равен магнитный поток через контур, если его плоскость
    расположена под углом $30\degrees$ к вектору магнитной индукции?
    Ответ выразите в милливеберах и округлите до целого, единицы измерения писать не нужно.
}
\answer{%
    $\alpha = 60\degrees, \Phi_B = BS\cos\alpha = 12{,}00\,\text{мВб} \to 12$
}
\solutionspace{100pt}

\tasknumber{4}%
\task{%
    Определите притягивается (А), не взаимодействует (Б) или отталкивается (В) металлическое кольцо к магниту,
    если вдвигать северным полюсом (см.
    рис).
}
\answer{%
    \text{В}
}

\tasknumber{5}%
\task{%
    Определите притягивается (А), не взаимодействует (Б) или отталкивается (В) кольцо из диэлектрика к магниту,
    если выдвигать магнит из кольца южным полюсом (см.
    рис).
}
\answer{%
    \text{Б}
}

\tasknumber{6}%
\task{%
    Магнитный поток, пронизывающий замкнутый контур, равномерно изменяется от $95\,\text{мВб}$ до $50\,\text{мВб}$ за $1{,}1\,\text{c}$.
    Чему равна ЭДС в контуре? Ответ выразите в милливольтах и округлите до целого, единицы измерения писать не нужно.
}
\answer{%
    $
        \ele
        = \frac{\abs{\Delta \Phi}}{\Delta t}
        = \frac{\abs{\Phi_2 - \Phi_1}}{\Delta t}
        = \frac{\abs{50\,\text{мВб} - 95\,\text{мВб}}}{\Delta t}
        = 40{,}909\,\text{мВ} \to 41
    $
}
\solutionspace{60pt}

\tasknumber{7}%
\task{%
    Определите магнитный поток через контур,
    находящийся в однородном магнитном поле индукцией $300\,\text{мТл}$.
    Контур имеет форму прямоугольного треугольника с катетами $60\,\text{см}$ и $75\,\text{см}$.
    Угол между нормалью к плоскости контура и вектором индукции магнитного поля
    составляет $40\degrees$.
    Ответ выразите в милливеберах и округлите до целого, единицы измерения писать не нужно.
}
\answer{%
    $\alpha=40\degrees, \Phi_B = BS\cos\alpha = 51{,}71\,\text{мВб} \to 52$
}

\variantsplitter

\addpersonalvariant{Артём Переверзев}

\tasknumber{1}%
\task{%
    Установите каждой букве в соответствие ровно одну цифру и запишите ответ (только цифры, без других символов).

    А) вектор нормали к поверхности, Б) магнитный поток.

    1) $\vec n$, 2) $\Phi$, 3) $U$, 4) $B$.
}
\answer{%
    $12$
}
\solutionspace{20pt}

\tasknumber{2}%
\task{%
    Установите каждой букве в соответствие ровно одну цифру и запишите ответ (только цифры, без других символов).

    А) индукционый ток, Б) магнитный поток, В) ЭДС индукции.

    1) В, 2) А, 3) Вб, 4) $\text{м}^2$, 5) Гн.
}
\answer{%
    $231$
}
\solutionspace{20pt}

\tasknumber{3}%
\task{%
    Однородное магнитное поле пронизывает плоский контур площадью $200\,\text{см}^{2}$.
    Индукция магнитного поля равна $300\,\text{мТл}$.
    Чему равен магнитный поток через контур, если его плоскость
    расположена под углом $60\degrees$ к вектору магнитной индукции?
    Ответ выразите в милливеберах и округлите до целого, единицы измерения писать не нужно.
}
\answer{%
    $\alpha = 30\degrees, \Phi_B = BS\cos\alpha = 5{,}20\,\text{мВб} \to 5$
}
\solutionspace{100pt}

\tasknumber{4}%
\task{%
    Определите притягивается (А), не взаимодействует (Б) или отталкивается (В) металлическое кольцо к магниту,
    если выдвигать южным полюсом (см.
    рис).
}
\answer{%
    \text{А}
}

\tasknumber{5}%
\task{%
    Определите притягивается (А), не взаимодействует (Б) или отталкивается (В) кольцо из диэлектрика к магниту,
    если вдвигать магнит в кольцо южным полюсом (см.
    рис).
}
\answer{%
    \text{Б}
}

\tasknumber{6}%
\task{%
    Магнитный поток, пронизывающий замкнутый контур, равномерно изменяется от $95\,\text{мВб}$ до $20\,\text{мВб}$ за $1{,}1\,\text{c}$.
    Чему равна ЭДС в контуре? Ответ выразите в милливольтах и округлите до целого, единицы измерения писать не нужно.
}
\answer{%
    $
        \ele
        = \frac{\abs{\Delta \Phi}}{\Delta t}
        = \frac{\abs{\Phi_2 - \Phi_1}}{\Delta t}
        = \frac{\abs{20\,\text{мВб} - 95\,\text{мВб}}}{\Delta t}
        = 68{,}182\,\text{мВ} \to 68
    $
}
\solutionspace{60pt}

\tasknumber{7}%
\task{%
    Определите магнитный поток через контур,
    находящийся в однородном магнитном поле индукцией $500\,\text{мТл}$.
    Контур имеет форму прямоугольного треугольника с катетами $60\,\text{см}$ и $75\,\text{см}$.
    Угол между нормалью к плоскости контура и вектором индукции магнитного поля
    составляет $40\degrees$.
    Ответ выразите в милливеберах и округлите до целого, единицы измерения писать не нужно.
}
\answer{%
    $\alpha=40\degrees, \Phi_B = BS\cos\alpha = 86{,}18\,\text{мВб} \to 86$
}

\variantsplitter

\addpersonalvariant{Варвара Пранова}

\tasknumber{1}%
\task{%
    Установите каждой букве в соответствие ровно одну цифру и запишите ответ (только цифры, без других символов).

    А) сопротивление контура, Б) вектор нормали к поверхности.

    1) $\vec n$, 2) $U$, 3) $R$, 4) $B$.
}
\answer{%
    $31$
}
\solutionspace{20pt}

\tasknumber{2}%
\task{%
    Установите каждой букве в соответствие ровно одну цифру и запишите ответ (только цифры, без других символов).

    А) магнитный поток, Б) индукционый ток, В) индукция магнитного поля.

    1) Тл, 2) А, 3) В, 4) Вб, 5) Кл.
}
\answer{%
    $421$
}
\solutionspace{20pt}

\tasknumber{3}%
\task{%
    Однородное магнитное поле пронизывает плоский контур площадью $800\,\text{см}^{2}$.
    Индукция магнитного поля равна $300\,\text{мТл}$.
    Чему равен магнитный поток через контур, если его плоскость
    расположена под углом $90\degrees$ к вектору магнитной индукции?
    Ответ выразите в милливеберах и округлите до целого, единицы измерения писать не нужно.
}
\answer{%
    $\alpha = 0\degrees, \Phi_B = BS\cos\alpha = 24{,}00\,\text{мВб} \to 24$
}
\solutionspace{100pt}

\tasknumber{4}%
\task{%
    Определите притягивается (А), не взаимодействует (Б) или отталкивается (В) металлическое кольцо к магниту,
    если выдвигать северным полюсом (см.
    рис).
}
\answer{%
    \text{А}
}

\tasknumber{5}%
\task{%
    Определите притягивается (А), не взаимодействует (Б) или отталкивается (В) кольцо из диэлектрика к магниту,
    если вдвигать магнит в кольцо северным полюсом (см.
    рис).
}
\answer{%
    \text{Б}
}

\tasknumber{6}%
\task{%
    Магнитный поток, пронизывающий замкнутый контур, равномерно изменяется от $35\,\text{мВб}$ до $80\,\text{мВб}$ за $1{,}1\,\text{c}$.
    Чему равна ЭДС в контуре? Ответ выразите в милливольтах и округлите до целого, единицы измерения писать не нужно.
}
\answer{%
    $
        \ele
        = \frac{\abs{\Delta \Phi}}{\Delta t}
        = \frac{\abs{\Phi_2 - \Phi_1}}{\Delta t}
        = \frac{\abs{80\,\text{мВб} - 35\,\text{мВб}}}{\Delta t}
        = 40{,}909\,\text{мВ} \to 41
    $
}
\solutionspace{60pt}

\tasknumber{7}%
\task{%
    Определите магнитный поток через контур,
    находящийся в однородном магнитном поле индукцией $500\,\text{мТл}$.
    Контур имеет форму прямоугольного треугольника с катетами $40\,\text{см}$ и $45\,\text{см}$.
    Угол между нормалью к плоскости контура и вектором индукции магнитного поля
    составляет $20\degrees$.
    Ответ выразите в милливеберах и округлите до целого, единицы измерения писать не нужно.
}
\answer{%
    $\alpha=20\degrees, \Phi_B = BS\cos\alpha = 42{,}29\,\text{мВб} \to 42$
}

\variantsplitter

\addpersonalvariant{Марьям Салимова}

\tasknumber{1}%
\task{%
    Установите каждой букве в соответствие ровно одну цифру и запишите ответ (только цифры, без других символов).

    А) площадь контура, Б) индукционый ток.

    1) $\eli$, 2) $B$, 3) $D$, 4) $S$.
}
\answer{%
    $41$
}
\solutionspace{20pt}

\tasknumber{2}%
\task{%
    Установите каждой букве в соответствие ровно одну цифру и запишите ответ (только цифры, без других символов).

    А) индукция магнитного поля, Б) площадь контура, В) магнитный поток.

    1) В, 2) $\text{м}^2$, 3) Ом, 4) Вб, 5) Тл.
}
\answer{%
    $524$
}
\solutionspace{20pt}

\tasknumber{3}%
\task{%
    Однородное магнитное поле пронизывает плоский контур площадью $200\,\text{см}^{2}$.
    Индукция магнитного поля равна $500\,\text{мТл}$.
    Чему равен магнитный поток через контур, если его плоскость
    расположена под углом $0\degrees$ к вектору магнитной индукции?
    Ответ выразите в милливеберах и округлите до целого, единицы измерения писать не нужно.
}
\answer{%
    $\alpha = 90\degrees, \Phi_B = BS\cos\alpha = 0\,\text{мВб} \to 0$
}
\solutionspace{100pt}

\tasknumber{4}%
\task{%
    Определите притягивается (А), не взаимодействует (Б) или отталкивается (В) металлическое кольцо к магниту,
    если выдвигать южным полюсом (см.
    рис).
}
\answer{%
    \text{А}
}

\tasknumber{5}%
\task{%
    Определите притягивается (А), не взаимодействует (Б) или отталкивается (В) кольцо из диэлектрика к магниту,
    если выдвигать магнит из кольца южным полюсом (см.
    рис).
}
\answer{%
    \text{Б}
}

\tasknumber{6}%
\task{%
    Магнитный поток, пронизывающий замкнутый контур, равномерно изменяется от $65\,\text{мВб}$ до $80\,\text{мВб}$ за $1{,}3\,\text{c}$.
    Чему равна ЭДС в контуре? Ответ выразите в милливольтах и округлите до целого, единицы измерения писать не нужно.
}
\answer{%
    $
        \ele
        = \frac{\abs{\Delta \Phi}}{\Delta t}
        = \frac{\abs{\Phi_2 - \Phi_1}}{\Delta t}
        = \frac{\abs{80\,\text{мВб} - 65\,\text{мВб}}}{\Delta t}
        = 11{,}538\,\text{мВ} \to 12
    $
}
\solutionspace{60pt}

\tasknumber{7}%
\task{%
    Определите магнитный поток через контур,
    находящийся в однородном магнитном поле индукцией $300\,\text{мТл}$.
    Контур имеет форму прямоугольного треугольника с катетами $60\,\text{см}$ и $75\,\text{см}$.
    Угол между плоскостью контура и вектором индукции магнитного поля
    составляет $10\degrees$.
    Ответ выразите в милливеберах и округлите до целого, единицы измерения писать не нужно.
}
\answer{%
    $\alpha=80\degrees, \Phi_B = BS\cos\alpha = 11{,}72\,\text{мВб} \to 12$
}

\variantsplitter

\addpersonalvariant{Юлия Шевченко}

\tasknumber{1}%
\task{%
    Установите каждой букве в соответствие ровно одну цифру и запишите ответ (только цифры, без других символов).

    А) ЭДС индукции, Б) вектор нормали к поверхности.

    1) $\ele$, 2) $\vec n$, 3) $\Phi$, 4) $R$.
}
\answer{%
    $12$
}
\solutionspace{20pt}

\tasknumber{2}%
\task{%
    Установите каждой букве в соответствие ровно одну цифру и запишите ответ (только цифры, без других символов).

    А) ЭДС индукции, Б) магнитный поток, В) индукция магнитного поля.

    1) Вб, 2) В, 3) Тл, 4) Кл, 5) Ом.
}
\answer{%
    $213$
}
\solutionspace{20pt}

\tasknumber{3}%
\task{%
    Однородное магнитное поле пронизывает плоский контур площадью $800\,\text{см}^{2}$.
    Индукция магнитного поля равна $700\,\text{мТл}$.
    Чему равен магнитный поток через контур, если его плоскость
    расположена под углом $30\degrees$ к вектору магнитной индукции?
    Ответ выразите в милливеберах и округлите до целого, единицы измерения писать не нужно.
}
\answer{%
    $\alpha = 60\degrees, \Phi_B = BS\cos\alpha = 28{,}00\,\text{мВб} \to 28$
}
\solutionspace{100pt}

\tasknumber{4}%
\task{%
    Определите притягивается (А), не взаимодействует (Б) или отталкивается (В) металлическое кольцо к магниту,
    если вдвигать северным полюсом (см.
    рис).
}
\answer{%
    \text{В}
}

\tasknumber{5}%
\task{%
    Определите притягивается (А), не взаимодействует (Б) или отталкивается (В) кольцо из диэлектрика к магниту,
    если вдвигать магнит в кольцо северным полюсом (см.
    рис).
}
\answer{%
    \text{Б}
}

\tasknumber{6}%
\task{%
    Магнитный поток, пронизывающий замкнутый контур, равномерно изменяется от $95\,\text{мВб}$ до $20\,\text{мВб}$ за $1{,}3\,\text{c}$.
    Чему равна ЭДС в контуре? Ответ выразите в милливольтах и округлите до целого, единицы измерения писать не нужно.
}
\answer{%
    $
        \ele
        = \frac{\abs{\Delta \Phi}}{\Delta t}
        = \frac{\abs{\Phi_2 - \Phi_1}}{\Delta t}
        = \frac{\abs{20\,\text{мВб} - 95\,\text{мВб}}}{\Delta t}
        = 57{,}692\,\text{мВ} \to 58
    $
}
\solutionspace{60pt}

\tasknumber{7}%
\task{%
    Определите магнитный поток через контур,
    находящийся в однородном магнитном поле индукцией $500\,\text{мТл}$.
    Контур имеет форму прямоугольного треугольника с катетами $50\,\text{см}$ и $80\,\text{см}$.
    Угол между нормалью к плоскости контура и вектором индукции магнитного поля
    составляет $80\degrees$.
    Ответ выразите в милливеберах и округлите до целого, единицы измерения писать не нужно.
}
\answer{%
    $\alpha=80\degrees, \Phi_B = BS\cos\alpha = 17{,}36\,\text{мВб} \to 17$
}
% autogenerated
