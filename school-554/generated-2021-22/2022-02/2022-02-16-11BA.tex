\setdate{16~февраля~2022}
\setclass{11«БА»}

\addpersonalvariant{Михаил Бурмистров}

\tasknumber{1}%
\task{%
    Запишите
    \begin{itemize}
        \item постулаты специальной теории относительности,
        \item пример релятивистского эффекта, обнаружимый при скоростях гораздо меньше скорости света.
    \end{itemize}
}
\solutionspace{120pt}

\tasknumber{2}%
\task{%
    Запишите формулу для ...
    \begin{itemize}
        \item релятивистского сжатия,
        \item классического импульса,
        \item релятивистской энергии тела,
        \item энергии покоя тела,
        \item связь между релятивистским импульсом и релятивистской энергией.
    \end{itemize}
    Обязательно подпишите все физические величины.
}
\solutionspace{150pt}

\tasknumber{3}%
\task{%
    Протон движется со скоростью $0{,}7\,c$, где $c$~--- скорость света в вакууме.
    Каково при этом отношение кинетической энергии частицы $E_\text{кин.}$ к его энергии покоя $E_0$?
}
\answer{%
    \begin{align*}
    E &= \frac{E_0}{\sqrt{1 - \frac{v^2}{c^2}}}
            \implies \frac E{E_0}
                = \frac 1{\sqrt{1 - \frac{v^2}{c^2}}}
                = \frac 1{\sqrt{1 - \sqr{0{,}7}}}
                \approx 1{,}400,
         \\
        {E_{\text{кин}}} &= E - E_0
            \implies \frac{E_{\text{кин}}}{E_0}
                = \frac E{E_0} - 1
                = \frac 1{\sqrt{1 - \frac{v^2}{c^2}}} - 1
                = \frac 1{\sqrt{1 - \sqr{0{,}7}}} - 1
                \approx 0{,}400.
    \end{align*}
}
\solutionspace{80pt}

\tasknumber{4}%
\task{%
    Полная энергия релятивистской частицы в шесть раз больше её энергии покоя.
    Найти скорость этой частицы: в долях $c$ и численное значение.
    Скорость света в вакууме $c = 3 \cdot 10^{8}\,\frac{\text{м}}{\text{с}}$.
}
\answer{%
    \begin{align*}
    E &= \frac{E_0}{\sqrt{1 - \frac{v^2}{c^2}}}\implies \sqrt{1 - \frac{v^2}{c^2}} = \frac{E_0}{E}\implies \frac{v^2}{c^2} = 1 - \sqr{\frac{E_0}{E}}\implies v = c \sqrt{1 - \sqr{\frac{E_0}{E}}} \approx 0{,}986c \approx 296 \cdot 10^{6}\,\frac{\text{м}}{\text{с}}.
    \end{align*}
}
\solutionspace{80pt}

\tasknumber{5}%
\task{%
    Кинетическая энергия релятивистской частицы в шесть раз больше её энергии покоя.
    Найти скорость этой частицы.
    Скорость света в вакууме $c = 3 \cdot 10^{8}\,\frac{\text{м}}{\text{с}}$.
}
\answer{%
    \begin{align*}
    E &= E_0 + E_{\text{кин}} \\
    E &= \frac{E_0}{\sqrt{1 - \frac{v^2}{c^2}}}\implies \sqrt{1 - \frac{v^2}{c^2}} = \frac{E_0}{E}\implies \frac{v^2}{c^2} = 1 - \sqr{\frac{E_0}{E}} \implies \\
    \implies &v = c \sqrt{1 - \sqr{\frac{E_0}{E}}} = c \sqrt{1 - \sqr{\frac{E_0}{E_0 + E_{\text{кин}} }}} = c \sqrt{1 - \frac 1 {\sqr{ 1 + \frac{E_{\text{кин}}}{E_0} }} }\approx 0{,}990c \approx 297 \cdot 10^{6}\,\frac{\text{м}}{\text{с}}.
    \end{align*}
}


\variantsplitter


\addpersonalvariant{Михаил Бурмистров}

\tasknumber{6}%
\task{%
    Электрон движется со скоростью $0{,}65\,c$, где $c$~--- скорость света в вакууме.
    Определите его кинетическую энергию (в ответе приведите формулу и укажите численное значение).
}
\answer{%
    \begin{align*}
    E &= \frac{mc^2}{\sqrt{1 - \frac{v^2}{c^2}}}
            \approx \frac{9{,}1 \cdot 10^{-31}\,\text{кг} \cdot \sqr{3 \cdot 10^{8}\,\frac{\text{м}}{\text{с}}}}{\sqrt{1 - 0{,}65^2}}
            \approx 0{,}1078 \cdot 10^{-12}\,\text{Дж},
         \\
        E_{\text{кин}} &= \frac{mc^2}{\sqrt{1 - \frac{v^2}{c^2}}} - mc^2
            = mc^2 \cbr{\frac 1{\sqrt{1 - \frac{v^2}{c^2}}} - 1} \approx \\
            &\approx \cbr{9{,}1 \cdot 10^{-31}\,\text{кг} \cdot \sqr{3 \cdot 10^{8}\,\frac{\text{м}}{\text{с}}}}
            \cdot \cbr{\frac 1{\sqrt{1 - 0{,}65^2}} - 1}
            \approx 25{,}9 \cdot 10^{-15}\,\text{Дж},
         \\
        p &= \frac{mv}{\sqrt{1 - \frac{v^2}{c^2}}}
            \approx \frac{9{,}1 \cdot 10^{-31}\,\text{кг} \cdot 0{,}65 \cdot 3 \cdot 10^{8}\,\frac{\text{м}}{\text{с}}}{\sqrt{1 - 0{,}65^2}}
            \approx 0{,}234 \cdot 10^{-21}\,\frac{\text{кг}\cdot\text{м}}{\text{с}}.
    \end{align*}
}
\solutionspace{100pt}

\tasknumber{7}%
\task{%
    Кинетическая энергия частицы космических лучей в шесть раз превышает её энергию покоя.
    Определить отношение скорости частицы к скорости света.
}
\answer{%
    \begin{align*}
    E &= E_0 + E_{\text{кин}} \\
    E &= \frac{E_0}{\sqrt{1 - \frac{v^2}{c^2}}}\implies \sqrt{1 - \frac{v^2}{c^2}} = \frac{E_0}{E}\implies \frac{v^2}{c^2} = 1 - \sqr{\frac{E_0}{E}} \implies \\
    \implies \frac vc &= \sqrt{1 - \sqr{\frac{E_0}{E}}} = \sqrt{1 - \sqr{\frac{E_0}{E_0 + E_{\text{кин}} }}} \approx 0{,}990.
    \end{align*}
}
\solutionspace{80pt}

\tasknumber{8}%
\task{%
    Некоторая частица, пройдя ускоряющую разность потенциалов, приобрела импульс $3 \cdot 10^{-19}\,\frac{\text{кг}\cdot\text{м}}{\text{с}}$.
    Скорость частицы стала равной $2{,}4 \cdot 10^{8}\,\frac{\text{м}}{\text{с}}$.
    Найти массу частицы.
}
\answer{%
    $p = \frac{ mv }{\sqrt{1 - \frac{v^2}{c^2} }}\implies m = \frac pv \sqrt{1 - \frac{v^2}{c^2}}= \frac {3 \cdot 10^{-19}\,\frac{\text{кг}\cdot\text{м}}{\text{с}}}{2{,}4 \cdot 10^{8}\,\frac{\text{м}}{\text{с}}} \sqrt{1 - \sqr{\frac{2{,}4 \cdot 10^{8}\,\frac{\text{м}}{\text{с}}}{3 \cdot 10^{8}\,\frac{\text{м}}{\text{с}}}} } \approx 0{,}75 \cdot 10^{-27}\,\text{кг}.$
}
\solutionspace{80pt}

\tasknumber{9}%
\task{%
    При какой скорости движения (в км/ч) релятивистское сокращение длины движущегося тела
    составит 50\%?
}
\answer{%
    \begin{align*}
    l_0 &= \frac l{\sqrt{1 - \frac{v^2}{c^2}}}
        \implies 1 - \frac{v^2}{c^2} = \sqr{\frac l{l_0}}
        \implies \frac v c = \sqrt{1 - \sqr{\frac l{l_0}}} \implies
         \\
        \implies v &= c\sqrt{1 - \sqr{\frac l{l_0}}}
        = 3 \cdot 10^{8}\,\frac{\text{м}}{\text{с}} \cdot \sqrt{1 - \sqr{\frac {l_0 - 0{,}50l_0}{l_0}}}
        = 3 \cdot 10^{8}\,\frac{\text{м}}{\text{с}} \cdot \sqrt{1 - \sqr{1 - 0{,}50}} \approx  \\
        &\approx 0{,}866c
        \approx 260 \cdot 10^{6}\,\frac{\text{м}}{\text{с}}
        \approx 935 \cdot 10^{6}\,\frac{\text{км}}{\text{ч}}.
    \end{align*}
}
\solutionspace{80pt}

\tasknumber{10}%
\task{%
    Стержень движется в продольном направлении с постоянной скоростью относительно инерциальной системы отсчёта.
    При каком значении скорости (в долях скорости света) длина стержня в этой системе отсчёта
    будет в  3  раза меньше его собственной длины?
}
\answer{%
    $l_0 = \frac l{\sqrt{1 - \frac{v^2}{c^2}}}\implies \sqrt{1 - \frac{v^2}{c^2}} = \frac{ l }{ l_0 }\implies \frac v c = \sqrt{1 - \sqr{\frac{ l }{ l_0 }}} \approx 0{,}943.$
}
\solutionspace{80pt}

\tasknumber{11}%
\task{%
    Какую скорость должно иметь движущееся тело, чтобы его продольные размеры уменьшились в шесть раз?
    Скорость света $c = 3 \cdot 10^{8}\,\frac{\text{м}}{\text{с}}$.
}
\answer{%
    $l_0 = \frac l{\sqrt{1 - \frac{v^2}{c^2}}}\implies \sqrt{1 - \frac{v^2}{c^2}} = \frac{ l }{ l_0 }\implies v = c\sqrt{1 - \sqr{\frac{ l }{ l_0 }}} \approx 296 \cdot 10^{6}\,\frac{\text{м}}{\text{с}}.$
}


\variantsplitter


\addpersonalvariant{Михаил Бурмистров}

\tasknumber{12}%
\task{%
    Время жизни мюона, измеренное наблюдателем, относительно которого мюон покоился, равно $\tau_0$
    Какое расстояние пролетит мюон в системе отсчёта, относительно которой он движется со скоростью $v$,
    сравнимой со скоростью света в вакууме $c$?
}
\answer{%
    $\ell = v\tau = v \frac{\tau_0}{\sqrt{1 - \frac{v^2}{c^2}}}$
}
\solutionspace{80pt}

\tasknumber{13}%
\task{%
    Если $c$ — скорость света в вакууме, то с какой скоростью должна двигаться нестабильная частица относительно наблюдателя,
    чтобы её время жизни было в шесть раз больше, чем у такой же, но покоящейся относительно наблюдателя частицы?
}
\answer{%
    $\tau = \frac{\tau_0}{\sqrt{1 - \frac{v^2}{c^2}}}\implies \sqrt{1 - \frac{v^2}{c^2}} = \frac{\tau_0}{\tau}\implies v = c\sqrt{1 - \sqr{\frac{\tau_0}{\tau}} } \approx 296 \cdot 10^{6}\,\frac{\text{м}}{\text{с}}.$
}
\solutionspace{80pt}

\tasknumber{14}%
\task{%
    Время жизни нестабильной частицы, входящего в состав космических лучей, измеренное земным наблюдателем,
    относительно которого частица двигалась со скоростью, составляющей 75\% скорости света в вакууме, оказалось равным $4{,}8\,\text{мкс}$.
    Каково время жизни частицы, покоящейся относительно наблюдателя?
}
\answer{%
    $t = \frac{t_0}{\sqrt{1 - \frac{v^2}{c^2}}}\implies t_0 = t\sqrt{1 - \frac{v^2}{c^2}} \approx 3{,}2 \cdot 10^{-6}\,\text{с}.$
}
\solutionspace{80pt}

\tasknumber{15}%
\task{%
    Частица увеличила в ускорителе свою скорость с $0{,}05c$ до $0{,}80c$.
    Во сколько раз выросла её кинетическая энергия?
}
\answer{%
    \begin{align*}
    E_{\text{кин.}} &= E - E_0 = \frac{mc^2}{\sqrt{1 - \frac{v^2}{c^2}}} - mc^2 = mc^2\cbr{ \frac1{\sqrt{1 - \frac{v^2}{c^2} }} - 1}.
    \\
    \frac{E_{\text{кин.
    2}}}{E_{\text{кин.
    1}}} &= \frac{\frac1{\sqrt{1 - \frac{v_2^2}{c^2} }} - 1}{\frac1{\sqrt{1 - \frac{v_1^2}{c^2} }} - 1}\approx 532{,}33
    \end{align*}
}
\solutionspace{120pt}

\tasknumber{16}%
\task{%
    Для частицы, движущейся с релятивистской скоростью,
    выразите $E_\text{кин}$ и $p$ через $c$, $E_0$ и $v$,
    где $E_\text{кин}$~--- кинетическая энергия частицы,
    а $E_0$, $p$ и $v$~--- её энергия покоя, импульс и скорость.
}
\answer{%
    \begin{align*}
    E_\text{кин}, E_0:\quad&E = E_\text{кин} + E_0 = \frac{E_0}{\sqrt{1 - \frac{v^2}{c^2}}} \implies \sqrt{1 - \frac{v^2}{c^2}} = \frac{E_0}{{E_0} + {E_\text{кин}}} \implies v = c\sqrt{1 - \sqr{\frac{E_0}{{E_0} + {E_\text{кин}}}}} \\
    &p = \frac{mv}{\sqrt{1 - \frac{v^2}{c^2}}} = \frac{E_0}{c^2} \cdot \sqrt{1 - \sqr{\frac{E_0}{{E_0} + {E_\text{кин}}}}} \cdot \frac{{E_\text{кин}} + {E_0}}{E_0} = \frac{E_0}{c^2} \cdot \sqrt{\sqr{\frac{{E_\text{кин}} + {E_0}}{E_0}} - 1}.
    \\
    E_\text{кин}, p:\quad&E_\text{кин} = E - E_0 = mc^2\cbr{\frac 1{\sqrt{1 - \frac{v^2}{c^2}}} - 1}, p = \frac{mv}{\sqrt{1 - \frac{v^2}{c^2}}} \implies \frac{E_\text{кин}}{p} = \frac{\frac 1{\sqrt{1 - \frac{v^2}{c^2}}} - 1}{\sqrt{1 - \frac{v^2}{c^2}}} \implies v = \ldots \\
    &E_0 = E - E_\text{кин} = \frac{E_0}{\sqrt{1 - \frac{v^2}{c^2}}} - E_\text{кин} \implies E_0 = \frac{E_\text{кин}}{\frac 1{\sqrt{1 - \frac{v^2}{c^2}}} - 1} = \ldots \\
    E_\text{кин}, v:\quad&E_\text{кин} = E - E_0 = mc^2\cbr{\frac 1{\sqrt{1 - \frac{v^2}{c^2}}} - 1} \implies m = \frac{E_\text{кин}}{c^2\cbr{\frac 1{\sqrt{1 - \frac{v^2}{c^2}}} - 1}} \\
    &E_0 = mc^2 = \frac{E_\text{кин}}{\frac 1{\sqrt{1 - \frac{v^2}{c^2}}} - 1} \\
    &p = \frac{mv}{\sqrt{1 - \frac{v^2}{c^2}}} = \frac{E_\text{кин}}{c^2\cbr{\frac 1{\sqrt{1 - \frac{v^2}{c^2}}} - 1}} \cdot \frac{v}{\sqrt{1 - \frac{v^2}{c^2}}} = \frac{{E_\text{кин}} v}{c^2\cbr{1 - {\sqrt{1 - \frac{v^2}{c^2}}}}} \\
    E_0, p:\quad&E_0 = mc^2, \quad p = \frac{mv}{\sqrt{1 - \frac{v^2}{c^2}}} \implies \frac{E_0}{p} = \frac{c^2}v{\sqrt{1 - \frac{v^2}{c^2}}} = c\sqrt{\frac{c^2}{v^2} - 1} \\
    &\sqr{\frac{E_0}{pc}} = \frac{c^2}{v^2} - 1 \implies \frac{v^2}{c^2} = \frac 1{1 + \frac{E_0^2}{p^2c^2}} \implies v = \frac c{\sqrt{1 + \frac{E_0^2}{p^2c^2}}} \\
    &{E_\text{кин}} = E - E_0 = \sqrt{E_0^2 + p^2c^2} - E_0 \\
    E_0, v:\quad&E_0 = mc^2 \implies m = \frac{E_0}{c^2} \qquad p = \frac{mv}{\sqrt{1 - \frac{v^2}{c^2}}} = \frac{E_0}{c^2} \cdot \frac{v}{\sqrt{1 - \frac{v^2}{c^2}}} \\
    &E_\text{кин}= mc^2\cbr{\frac 1{\sqrt{1 - \frac{v^2}{c^2}}} - 1} = \frac{E_0}{c^2}\cbr{\frac 1{\sqrt{1 - \frac{v^2}{c^2}}} - 1} \\
    p, v:\quad&p = \frac{mv}{\sqrt{1 - \frac{v^2}{c^2}}} \implies m = \frac p v {\sqrt{1 - \frac{v^2}{c^2}}} \implies E_0 = mc^2 =\frac {pc^2} v {\sqrt{1 - \frac{v^2}{c^2}}} \\
    &E_\text{кин} = mc^2\cbr{\frac 1{\sqrt{1 - \frac{v^2}{c^2}}} - 1} = \frac p v {\sqrt{1 - \frac{v^2}{c^2}}}\cbr{\frac 1{\sqrt{1 - \frac{v^2}{c^2}}} - 1} = \frac p v \cbr{1 - {\sqrt{1 - \frac{v^2}{c^2}}}}
    \end{align*}
}

\variantsplitter

\addpersonalvariant{Ирина Ан}

\tasknumber{1}%
\task{%
    Запишите
    \begin{itemize}
        \item постулаты специальной теории относительности,
        \item пример релятивистского эффекта, обнаружимый при скоростях гораздо меньше скорости света.
    \end{itemize}
}
\solutionspace{120pt}

\tasknumber{2}%
\task{%
    Запишите формулу для ...
    \begin{itemize}
        \item релятивистского сжатия,
        \item классического импульса,
        \item релятивистской энергии тела,
        \item энергии покоя тела,
        \item связь между релятивистским импульсом и релятивистской энергией.
    \end{itemize}
    Обязательно подпишите все физические величины.
}
\solutionspace{150pt}

\tasknumber{3}%
\task{%
    Протон движется со скоростью $0{,}9\,c$, где $c$~--- скорость света в вакууме.
    Каково при этом отношение полной энергии частицы $E$ к его энергии покоя $E_0$?
}
\answer{%
    \begin{align*}
    E &= \frac{E_0}{\sqrt{1 - \frac{v^2}{c^2}}}
            \implies \frac E{E_0}
                = \frac 1{\sqrt{1 - \frac{v^2}{c^2}}}
                = \frac 1{\sqrt{1 - \sqr{0{,}9}}}
                \approx 2{,}294,
         \\
        {E_{\text{кин}}} &= E - E_0
            \implies \frac{E_{\text{кин}}}{E_0}
                = \frac E{E_0} - 1
                = \frac 1{\sqrt{1 - \frac{v^2}{c^2}}} - 1
                = \frac 1{\sqrt{1 - \sqr{0{,}9}}} - 1
                \approx 1{,}294.
    \end{align*}
}
\solutionspace{80pt}

\tasknumber{4}%
\task{%
    Полная энергия релятивистской частицы в четыре раза больше её энергии покоя.
    Найти скорость этой частицы: в долях $c$ и численное значение.
    Скорость света в вакууме $c = 3 \cdot 10^{8}\,\frac{\text{м}}{\text{с}}$.
}
\answer{%
    \begin{align*}
    E &= \frac{E_0}{\sqrt{1 - \frac{v^2}{c^2}}}\implies \sqrt{1 - \frac{v^2}{c^2}} = \frac{E_0}{E}\implies \frac{v^2}{c^2} = 1 - \sqr{\frac{E_0}{E}}\implies v = c \sqrt{1 - \sqr{\frac{E_0}{E}}} \approx 0{,}968c \approx 290 \cdot 10^{6}\,\frac{\text{м}}{\text{с}}.
    \end{align*}
}
\solutionspace{80pt}

\tasknumber{5}%
\task{%
    Кинетическая энергия релятивистской частицы в четыре раза больше её энергии покоя.
    Найти скорость этой частицы.
    Скорость света в вакууме $c = 3 \cdot 10^{8}\,\frac{\text{м}}{\text{с}}$.
}
\answer{%
    \begin{align*}
    E &= E_0 + E_{\text{кин}} \\
    E &= \frac{E_0}{\sqrt{1 - \frac{v^2}{c^2}}}\implies \sqrt{1 - \frac{v^2}{c^2}} = \frac{E_0}{E}\implies \frac{v^2}{c^2} = 1 - \sqr{\frac{E_0}{E}} \implies \\
    \implies &v = c \sqrt{1 - \sqr{\frac{E_0}{E}}} = c \sqrt{1 - \sqr{\frac{E_0}{E_0 + E_{\text{кин}} }}} = c \sqrt{1 - \frac 1 {\sqr{ 1 + \frac{E_{\text{кин}}}{E_0} }} }\approx 0{,}980c \approx 294 \cdot 10^{6}\,\frac{\text{м}}{\text{с}}.
    \end{align*}
}


\variantsplitter


\addpersonalvariant{Ирина Ан}

\tasknumber{6}%
\task{%
    Электрон движется со скоростью $0{,}85\,c$, где $c$~--- скорость света в вакууме.
    Определите его кинетическую энергию (в ответе приведите формулу и укажите численное значение).
}
\answer{%
    \begin{align*}
    E &= \frac{mc^2}{\sqrt{1 - \frac{v^2}{c^2}}}
            \approx \frac{9{,}1 \cdot 10^{-31}\,\text{кг} \cdot \sqr{3 \cdot 10^{8}\,\frac{\text{м}}{\text{с}}}}{\sqrt{1 - 0{,}85^2}}
            \approx 0{,}1555 \cdot 10^{-12}\,\text{Дж},
         \\
        E_{\text{кин}} &= \frac{mc^2}{\sqrt{1 - \frac{v^2}{c^2}}} - mc^2
            = mc^2 \cbr{\frac 1{\sqrt{1 - \frac{v^2}{c^2}}} - 1} \approx \\
            &\approx \cbr{9{,}1 \cdot 10^{-31}\,\text{кг} \cdot \sqr{3 \cdot 10^{8}\,\frac{\text{м}}{\text{с}}}}
            \cdot \cbr{\frac 1{\sqrt{1 - 0{,}85^2}} - 1}
            \approx 73{,}6 \cdot 10^{-15}\,\text{Дж},
         \\
        p &= \frac{mv}{\sqrt{1 - \frac{v^2}{c^2}}}
            \approx \frac{9{,}1 \cdot 10^{-31}\,\text{кг} \cdot 0{,}85 \cdot 3 \cdot 10^{8}\,\frac{\text{м}}{\text{с}}}{\sqrt{1 - 0{,}85^2}}
            \approx 0{,}441 \cdot 10^{-21}\,\frac{\text{кг}\cdot\text{м}}{\text{с}}.
    \end{align*}
}
\solutionspace{100pt}

\tasknumber{7}%
\task{%
    Кинетическая энергия частицы космических лучей в четыре раза превышает её энергию покоя.
    Определить отношение скорости частицы к скорости света.
}
\answer{%
    \begin{align*}
    E &= E_0 + E_{\text{кин}} \\
    E &= \frac{E_0}{\sqrt{1 - \frac{v^2}{c^2}}}\implies \sqrt{1 - \frac{v^2}{c^2}} = \frac{E_0}{E}\implies \frac{v^2}{c^2} = 1 - \sqr{\frac{E_0}{E}} \implies \\
    \implies \frac vc &= \sqrt{1 - \sqr{\frac{E_0}{E}}} = \sqrt{1 - \sqr{\frac{E_0}{E_0 + E_{\text{кин}} }}} \approx 0{,}980.
    \end{align*}
}
\solutionspace{80pt}

\tasknumber{8}%
\task{%
    Некоторая частица, пройдя ускоряющую разность потенциалов, приобрела импульс $3 \cdot 10^{-19}\,\frac{\text{кг}\cdot\text{м}}{\text{с}}$.
    Скорость частицы стала равной $2{,}4 \cdot 10^{8}\,\frac{\text{м}}{\text{с}}$.
    Найти массу частицы.
}
\answer{%
    $p = \frac{ mv }{\sqrt{1 - \frac{v^2}{c^2} }}\implies m = \frac pv \sqrt{1 - \frac{v^2}{c^2}}= \frac {3 \cdot 10^{-19}\,\frac{\text{кг}\cdot\text{м}}{\text{с}}}{2{,}4 \cdot 10^{8}\,\frac{\text{м}}{\text{с}}} \sqrt{1 - \sqr{\frac{2{,}4 \cdot 10^{8}\,\frac{\text{м}}{\text{с}}}{3 \cdot 10^{8}\,\frac{\text{м}}{\text{с}}}} } \approx 0{,}75 \cdot 10^{-27}\,\text{кг}.$
}
\solutionspace{80pt}

\tasknumber{9}%
\task{%
    При какой скорости движения (в м/с) релятивистское сокращение длины движущегося тела
    составит 50\%?
}
\answer{%
    \begin{align*}
    l_0 &= \frac l{\sqrt{1 - \frac{v^2}{c^2}}}
        \implies 1 - \frac{v^2}{c^2} = \sqr{\frac l{l_0}}
        \implies \frac v c = \sqrt{1 - \sqr{\frac l{l_0}}} \implies
         \\
        \implies v &= c\sqrt{1 - \sqr{\frac l{l_0}}}
        = 3 \cdot 10^{8}\,\frac{\text{м}}{\text{с}} \cdot \sqrt{1 - \sqr{\frac {l_0 - 0{,}50l_0}{l_0}}}
        = 3 \cdot 10^{8}\,\frac{\text{м}}{\text{с}} \cdot \sqrt{1 - \sqr{1 - 0{,}50}} \approx  \\
        &\approx 0{,}866c
        \approx 260 \cdot 10^{6}\,\frac{\text{м}}{\text{с}}
        \approx 935 \cdot 10^{6}\,\frac{\text{км}}{\text{ч}}.
    \end{align*}
}
\solutionspace{80pt}

\tasknumber{10}%
\task{%
    Стержень движется в продольном направлении с постоянной скоростью относительно инерциальной системы отсчёта.
    При каком значении скорости (в долях скорости света) длина стержня в этой системе отсчёта
    будет в  1{,}5  раза меньше его собственной длины?
}
\answer{%
    $l_0 = \frac l{\sqrt{1 - \frac{v^2}{c^2}}}\implies \sqrt{1 - \frac{v^2}{c^2}} = \frac{ l }{ l_0 }\implies \frac v c = \sqrt{1 - \sqr{\frac{ l }{ l_0 }}} \approx 0{,}745.$
}
\solutionspace{80pt}

\tasknumber{11}%
\task{%
    Какую скорость должно иметь движущееся тело, чтобы его продольные размеры уменьшились в четыре раза?
    Скорость света $c = 3 \cdot 10^{8}\,\frac{\text{м}}{\text{с}}$.
}
\answer{%
    $l_0 = \frac l{\sqrt{1 - \frac{v^2}{c^2}}}\implies \sqrt{1 - \frac{v^2}{c^2}} = \frac{ l }{ l_0 }\implies v = c\sqrt{1 - \sqr{\frac{ l }{ l_0 }}} \approx 290 \cdot 10^{6}\,\frac{\text{м}}{\text{с}}.$
}


\variantsplitter


\addpersonalvariant{Ирина Ан}

\tasknumber{12}%
\task{%
    Время жизни мюона, измеренное наблюдателем, относительно которого мюон покоился, равно $\tau_0$
    Какое расстояние пролетит мюон в системе отсчёта, относительно которой он движется со скоростью $v$,
    сравнимой со скоростью света в вакууме $c$?
}
\answer{%
    $\ell = v\tau = v \frac{\tau_0}{\sqrt{1 - \frac{v^2}{c^2}}}$
}
\solutionspace{80pt}

\tasknumber{13}%
\task{%
    Если $c$ — скорость света в вакууме, то с какой скоростью должна двигаться нестабильная частица относительно наблюдателя,
    чтобы её время жизни было в восемь раз больше, чем у такой же, но покоящейся относительно наблюдателя частицы?
}
\answer{%
    $\tau = \frac{\tau_0}{\sqrt{1 - \frac{v^2}{c^2}}}\implies \sqrt{1 - \frac{v^2}{c^2}} = \frac{\tau_0}{\tau}\implies v = c\sqrt{1 - \sqr{\frac{\tau_0}{\tau}} } \approx 298 \cdot 10^{6}\,\frac{\text{м}}{\text{с}}.$
}
\solutionspace{80pt}

\tasknumber{14}%
\task{%
    Время жизни нестабильной частицы, входящего в состав космических лучей, измеренное земным наблюдателем,
    относительно которого частица двигалась со скоростью, составляющей 85\% скорости света в вакууме, оказалось равным $7{,}1\,\text{мкс}$.
    Каково время жизни частицы, покоящейся относительно наблюдателя?
}
\answer{%
    $t = \frac{t_0}{\sqrt{1 - \frac{v^2}{c^2}}}\implies t_0 = t\sqrt{1 - \frac{v^2}{c^2}} \approx 3{,}7 \cdot 10^{-6}\,\text{с}.$
}
\solutionspace{80pt}

\tasknumber{15}%
\task{%
    Частица увеличила в ускорителе свою скорость с $0{,}03c$ до $0{,}80c$.
    Во сколько раз выросла её кинетическая энергия?
}
\answer{%
    \begin{align*}
    E_{\text{кин.}} &= E - E_0 = \frac{mc^2}{\sqrt{1 - \frac{v^2}{c^2}}} - mc^2 = mc^2\cbr{ \frac1{\sqrt{1 - \frac{v^2}{c^2} }} - 1}.
    \\
    \frac{E_{\text{кин.
    2}}}{E_{\text{кин.
    1}}} &= \frac{\frac1{\sqrt{1 - \frac{v_2^2}{c^2} }} - 1}{\frac1{\sqrt{1 - \frac{v_1^2}{c^2} }} - 1}\approx 1480{,}48
    \end{align*}
}
\solutionspace{120pt}

\tasknumber{16}%
\task{%
    Для частицы, движущейся с релятивистской скоростью,
    выразите $v$ и $E_\text{кин}$ через $c$, $E_0$ и $p$,
    где $E_\text{кин}$~--- кинетическая энергия частицы,
    а $E_0$, $p$ и $v$~--- её энергия покоя, импульс и скорость.
}
\answer{%
    \begin{align*}
    E_\text{кин}, E_0:\quad&E = E_\text{кин} + E_0 = \frac{E_0}{\sqrt{1 - \frac{v^2}{c^2}}} \implies \sqrt{1 - \frac{v^2}{c^2}} = \frac{E_0}{{E_0} + {E_\text{кин}}} \implies v = c\sqrt{1 - \sqr{\frac{E_0}{{E_0} + {E_\text{кин}}}}} \\
    &p = \frac{mv}{\sqrt{1 - \frac{v^2}{c^2}}} = \frac{E_0}{c^2} \cdot \sqrt{1 - \sqr{\frac{E_0}{{E_0} + {E_\text{кин}}}}} \cdot \frac{{E_\text{кин}} + {E_0}}{E_0} = \frac{E_0}{c^2} \cdot \sqrt{\sqr{\frac{{E_\text{кин}} + {E_0}}{E_0}} - 1}.
    \\
    E_\text{кин}, p:\quad&E_\text{кин} = E - E_0 = mc^2\cbr{\frac 1{\sqrt{1 - \frac{v^2}{c^2}}} - 1}, p = \frac{mv}{\sqrt{1 - \frac{v^2}{c^2}}} \implies \frac{E_\text{кин}}{p} = \frac{\frac 1{\sqrt{1 - \frac{v^2}{c^2}}} - 1}{\sqrt{1 - \frac{v^2}{c^2}}} \implies v = \ldots \\
    &E_0 = E - E_\text{кин} = \frac{E_0}{\sqrt{1 - \frac{v^2}{c^2}}} - E_\text{кин} \implies E_0 = \frac{E_\text{кин}}{\frac 1{\sqrt{1 - \frac{v^2}{c^2}}} - 1} = \ldots \\
    E_\text{кин}, v:\quad&E_\text{кин} = E - E_0 = mc^2\cbr{\frac 1{\sqrt{1 - \frac{v^2}{c^2}}} - 1} \implies m = \frac{E_\text{кин}}{c^2\cbr{\frac 1{\sqrt{1 - \frac{v^2}{c^2}}} - 1}} \\
    &E_0 = mc^2 = \frac{E_\text{кин}}{\frac 1{\sqrt{1 - \frac{v^2}{c^2}}} - 1} \\
    &p = \frac{mv}{\sqrt{1 - \frac{v^2}{c^2}}} = \frac{E_\text{кин}}{c^2\cbr{\frac 1{\sqrt{1 - \frac{v^2}{c^2}}} - 1}} \cdot \frac{v}{\sqrt{1 - \frac{v^2}{c^2}}} = \frac{{E_\text{кин}} v}{c^2\cbr{1 - {\sqrt{1 - \frac{v^2}{c^2}}}}} \\
    E_0, p:\quad&E_0 = mc^2, \quad p = \frac{mv}{\sqrt{1 - \frac{v^2}{c^2}}} \implies \frac{E_0}{p} = \frac{c^2}v{\sqrt{1 - \frac{v^2}{c^2}}} = c\sqrt{\frac{c^2}{v^2} - 1} \\
    &\sqr{\frac{E_0}{pc}} = \frac{c^2}{v^2} - 1 \implies \frac{v^2}{c^2} = \frac 1{1 + \frac{E_0^2}{p^2c^2}} \implies v = \frac c{\sqrt{1 + \frac{E_0^2}{p^2c^2}}} \\
    &{E_\text{кин}} = E - E_0 = \sqrt{E_0^2 + p^2c^2} - E_0 \\
    E_0, v:\quad&E_0 = mc^2 \implies m = \frac{E_0}{c^2} \qquad p = \frac{mv}{\sqrt{1 - \frac{v^2}{c^2}}} = \frac{E_0}{c^2} \cdot \frac{v}{\sqrt{1 - \frac{v^2}{c^2}}} \\
    &E_\text{кин}= mc^2\cbr{\frac 1{\sqrt{1 - \frac{v^2}{c^2}}} - 1} = \frac{E_0}{c^2}\cbr{\frac 1{\sqrt{1 - \frac{v^2}{c^2}}} - 1} \\
    p, v:\quad&p = \frac{mv}{\sqrt{1 - \frac{v^2}{c^2}}} \implies m = \frac p v {\sqrt{1 - \frac{v^2}{c^2}}} \implies E_0 = mc^2 =\frac {pc^2} v {\sqrt{1 - \frac{v^2}{c^2}}} \\
    &E_\text{кин} = mc^2\cbr{\frac 1{\sqrt{1 - \frac{v^2}{c^2}}} - 1} = \frac p v {\sqrt{1 - \frac{v^2}{c^2}}}\cbr{\frac 1{\sqrt{1 - \frac{v^2}{c^2}}} - 1} = \frac p v \cbr{1 - {\sqrt{1 - \frac{v^2}{c^2}}}}
    \end{align*}
}

\variantsplitter

\addpersonalvariant{Софья Андрианова}

\tasknumber{1}%
\task{%
    Запишите
    \begin{itemize}
        \item постулаты специальной теории относительности,
        \item пример релятивистского эффекта, обнаружимый при скоростях гораздо меньше скорости света.
    \end{itemize}
}
\solutionspace{120pt}

\tasknumber{2}%
\task{%
    Запишите формулу для ...
    \begin{itemize}
        \item релятивистского сжатия,
        \item классической полной механической энергии тела,
        \item релятивистской энергии тела,
        \item энергии покоя тела,
        \item связь между релятивистским импульсом и релятивистской энергией.
    \end{itemize}
    Обязательно подпишите все физические величины.
}
\solutionspace{150pt}

\tasknumber{3}%
\task{%
    Электрон движется со скоростью $0{,}7\,c$, где $c$~--- скорость света в вакууме.
    Каково при этом отношение кинетической энергии частицы $E_\text{кин.}$ к его энергии покоя $E_0$?
}
\answer{%
    \begin{align*}
    E &= \frac{E_0}{\sqrt{1 - \frac{v^2}{c^2}}}
            \implies \frac E{E_0}
                = \frac 1{\sqrt{1 - \frac{v^2}{c^2}}}
                = \frac 1{\sqrt{1 - \sqr{0{,}7}}}
                \approx 1{,}400,
         \\
        {E_{\text{кин}}} &= E - E_0
            \implies \frac{E_{\text{кин}}}{E_0}
                = \frac E{E_0} - 1
                = \frac 1{\sqrt{1 - \frac{v^2}{c^2}}} - 1
                = \frac 1{\sqrt{1 - \sqr{0{,}7}}} - 1
                \approx 0{,}400.
    \end{align*}
}
\solutionspace{80pt}

\tasknumber{4}%
\task{%
    Полная энергия релятивистской частицы в пять раз больше её энергии покоя.
    Найти скорость этой частицы: в долях $c$ и численное значение.
    Скорость света в вакууме $c = 3 \cdot 10^{8}\,\frac{\text{м}}{\text{с}}$.
}
\answer{%
    \begin{align*}
    E &= \frac{E_0}{\sqrt{1 - \frac{v^2}{c^2}}}\implies \sqrt{1 - \frac{v^2}{c^2}} = \frac{E_0}{E}\implies \frac{v^2}{c^2} = 1 - \sqr{\frac{E_0}{E}}\implies v = c \sqrt{1 - \sqr{\frac{E_0}{E}}} \approx 0{,}980c \approx 294 \cdot 10^{6}\,\frac{\text{м}}{\text{с}}.
    \end{align*}
}
\solutionspace{80pt}

\tasknumber{5}%
\task{%
    Кинетическая энергия релятивистской частицы в пять раз больше её энергии покоя.
    Найти скорость этой частицы.
    Скорость света в вакууме $c = 3 \cdot 10^{8}\,\frac{\text{м}}{\text{с}}$.
}
\answer{%
    \begin{align*}
    E &= E_0 + E_{\text{кин}} \\
    E &= \frac{E_0}{\sqrt{1 - \frac{v^2}{c^2}}}\implies \sqrt{1 - \frac{v^2}{c^2}} = \frac{E_0}{E}\implies \frac{v^2}{c^2} = 1 - \sqr{\frac{E_0}{E}} \implies \\
    \implies &v = c \sqrt{1 - \sqr{\frac{E_0}{E}}} = c \sqrt{1 - \sqr{\frac{E_0}{E_0 + E_{\text{кин}} }}} = c \sqrt{1 - \frac 1 {\sqr{ 1 + \frac{E_{\text{кин}}}{E_0} }} }\approx 0{,}986c \approx 296 \cdot 10^{6}\,\frac{\text{м}}{\text{с}}.
    \end{align*}
}


\variantsplitter


\addpersonalvariant{Софья Андрианова}

\tasknumber{6}%
\task{%
    Электрон движется со скоростью $0{,}75\,c$, где $c$~--- скорость света в вакууме.
    Определите его импульс (в ответе приведите формулу и укажите численное значение).
}
\answer{%
    \begin{align*}
    E &= \frac{mc^2}{\sqrt{1 - \frac{v^2}{c^2}}}
            \approx \frac{9{,}1 \cdot 10^{-31}\,\text{кг} \cdot \sqr{3 \cdot 10^{8}\,\frac{\text{м}}{\text{с}}}}{\sqrt{1 - 0{,}75^2}}
            \approx 0{,}1238 \cdot 10^{-12}\,\text{Дж},
         \\
        E_{\text{кин}} &= \frac{mc^2}{\sqrt{1 - \frac{v^2}{c^2}}} - mc^2
            = mc^2 \cbr{\frac 1{\sqrt{1 - \frac{v^2}{c^2}}} - 1} \approx \\
            &\approx \cbr{9{,}1 \cdot 10^{-31}\,\text{кг} \cdot \sqr{3 \cdot 10^{8}\,\frac{\text{м}}{\text{с}}}}
            \cdot \cbr{\frac 1{\sqrt{1 - 0{,}75^2}} - 1}
            \approx 41{,}9 \cdot 10^{-15}\,\text{Дж},
         \\
        p &= \frac{mv}{\sqrt{1 - \frac{v^2}{c^2}}}
            \approx \frac{9{,}1 \cdot 10^{-31}\,\text{кг} \cdot 0{,}75 \cdot 3 \cdot 10^{8}\,\frac{\text{м}}{\text{с}}}{\sqrt{1 - 0{,}75^2}}
            \approx 0{,}310 \cdot 10^{-21}\,\frac{\text{кг}\cdot\text{м}}{\text{с}}.
    \end{align*}
}
\solutionspace{100pt}

\tasknumber{7}%
\task{%
    Кинетическая энергия частицы космических лучей в пять раз превышает её энергию покоя.
    Определить отношение скорости частицы к скорости света.
}
\answer{%
    \begin{align*}
    E &= E_0 + E_{\text{кин}} \\
    E &= \frac{E_0}{\sqrt{1 - \frac{v^2}{c^2}}}\implies \sqrt{1 - \frac{v^2}{c^2}} = \frac{E_0}{E}\implies \frac{v^2}{c^2} = 1 - \sqr{\frac{E_0}{E}} \implies \\
    \implies \frac vc &= \sqrt{1 - \sqr{\frac{E_0}{E}}} = \sqrt{1 - \sqr{\frac{E_0}{E_0 + E_{\text{кин}} }}} \approx 0{,}986.
    \end{align*}
}
\solutionspace{80pt}

\tasknumber{8}%
\task{%
    Некоторая частица, пройдя ускоряющую разность потенциалов, приобрела импульс $3 \cdot 10^{-19}\,\frac{\text{кг}\cdot\text{м}}{\text{с}}$.
    Скорость частицы стала равной $2{,}4 \cdot 10^{8}\,\frac{\text{м}}{\text{с}}$.
    Найти массу частицы.
}
\answer{%
    $p = \frac{ mv }{\sqrt{1 - \frac{v^2}{c^2} }}\implies m = \frac pv \sqrt{1 - \frac{v^2}{c^2}}= \frac {3 \cdot 10^{-19}\,\frac{\text{кг}\cdot\text{м}}{\text{с}}}{2{,}4 \cdot 10^{8}\,\frac{\text{м}}{\text{с}}} \sqrt{1 - \sqr{\frac{2{,}4 \cdot 10^{8}\,\frac{\text{м}}{\text{с}}}{3 \cdot 10^{8}\,\frac{\text{м}}{\text{с}}}} } \approx 0{,}75 \cdot 10^{-27}\,\text{кг}.$
}
\solutionspace{80pt}

\tasknumber{9}%
\task{%
    При какой скорости движения (в м/с) релятивистское сокращение длины движущегося тела
    составит 50\%?
}
\answer{%
    \begin{align*}
    l_0 &= \frac l{\sqrt{1 - \frac{v^2}{c^2}}}
        \implies 1 - \frac{v^2}{c^2} = \sqr{\frac l{l_0}}
        \implies \frac v c = \sqrt{1 - \sqr{\frac l{l_0}}} \implies
         \\
        \implies v &= c\sqrt{1 - \sqr{\frac l{l_0}}}
        = 3 \cdot 10^{8}\,\frac{\text{м}}{\text{с}} \cdot \sqrt{1 - \sqr{\frac {l_0 - 0{,}50l_0}{l_0}}}
        = 3 \cdot 10^{8}\,\frac{\text{м}}{\text{с}} \cdot \sqrt{1 - \sqr{1 - 0{,}50}} \approx  \\
        &\approx 0{,}866c
        \approx 260 \cdot 10^{6}\,\frac{\text{м}}{\text{с}}
        \approx 935 \cdot 10^{6}\,\frac{\text{км}}{\text{ч}}.
    \end{align*}
}
\solutionspace{80pt}

\tasknumber{10}%
\task{%
    Стержень движется в продольном направлении с постоянной скоростью относительно инерциальной системы отсчёта.
    При каком значении скорости (в долях скорости света) длина стержня в этой системе отсчёта
    будет в  1{,}5  раза меньше его собственной длины?
}
\answer{%
    $l_0 = \frac l{\sqrt{1 - \frac{v^2}{c^2}}}\implies \sqrt{1 - \frac{v^2}{c^2}} = \frac{ l }{ l_0 }\implies \frac v c = \sqrt{1 - \sqr{\frac{ l }{ l_0 }}} \approx 0{,}745.$
}
\solutionspace{80pt}

\tasknumber{11}%
\task{%
    Какую скорость должно иметь движущееся тело, чтобы его продольные размеры уменьшились в четыре раза?
    Скорость света $c = 3 \cdot 10^{8}\,\frac{\text{м}}{\text{с}}$.
}
\answer{%
    $l_0 = \frac l{\sqrt{1 - \frac{v^2}{c^2}}}\implies \sqrt{1 - \frac{v^2}{c^2}} = \frac{ l }{ l_0 }\implies v = c\sqrt{1 - \sqr{\frac{ l }{ l_0 }}} \approx 290 \cdot 10^{6}\,\frac{\text{м}}{\text{с}}.$
}


\variantsplitter


\addpersonalvariant{Софья Андрианова}

\tasknumber{12}%
\task{%
    Время жизни мюона, измеренное наблюдателем, относительно которого мюон покоился, равно $\tau_0$
    Какое расстояние пролетит мюон в системе отсчёта, относительно которой он движется со скоростью $v$,
    сравнимой со скоростью света в вакууме $c$?
}
\answer{%
    $\ell = v\tau = v \frac{\tau_0}{\sqrt{1 - \frac{v^2}{c^2}}}$
}
\solutionspace{80pt}

\tasknumber{13}%
\task{%
    Если $c$ — скорость света в вакууме, то с какой скоростью должна двигаться нестабильная частица относительно наблюдателя,
    чтобы её время жизни было в пять раз больше, чем у такой же, но покоящейся относительно наблюдателя частицы?
}
\answer{%
    $\tau = \frac{\tau_0}{\sqrt{1 - \frac{v^2}{c^2}}}\implies \sqrt{1 - \frac{v^2}{c^2}} = \frac{\tau_0}{\tau}\implies v = c\sqrt{1 - \sqr{\frac{\tau_0}{\tau}} } \approx 294 \cdot 10^{6}\,\frac{\text{м}}{\text{с}}.$
}
\solutionspace{80pt}

\tasknumber{14}%
\task{%
    Время жизни нестабильной частицы, входящего в состав космических лучей, измеренное земным наблюдателем,
    относительно которого частица двигалась со скоростью, составляющей 65\% скорости света в вакууме, оказалось равным $4{,}8\,\text{мкс}$.
    Каково время жизни частицы, покоящейся относительно наблюдателя?
}
\answer{%
    $t = \frac{t_0}{\sqrt{1 - \frac{v^2}{c^2}}}\implies t_0 = t\sqrt{1 - \frac{v^2}{c^2}} \approx 3{,}6 \cdot 10^{-6}\,\text{с}.$
}
\solutionspace{80pt}

\tasknumber{15}%
\task{%
    Частица увеличила в ускорителе свою скорость с $0{,}04c$ до $0{,}70c$.
    Во сколько раз выросла её кинетическая энергия?
}
\answer{%
    \begin{align*}
    E_{\text{кин.}} &= E - E_0 = \frac{mc^2}{\sqrt{1 - \frac{v^2}{c^2}}} - mc^2 = mc^2\cbr{ \frac1{\sqrt{1 - \frac{v^2}{c^2} }} - 1}.
    \\
    \frac{E_{\text{кин.
    2}}}{E_{\text{кин.
    1}}} &= \frac{\frac1{\sqrt{1 - \frac{v_2^2}{c^2} }} - 1}{\frac1{\sqrt{1 - \frac{v_1^2}{c^2} }} - 1}\approx 499{,}75
    \end{align*}
}
\solutionspace{120pt}

\tasknumber{16}%
\task{%
    Для частицы, движущейся с релятивистской скоростью,
    выразите $E_\text{кин}$ и $E_0$ через $c$, $p$ и $v$,
    где $E_\text{кин}$~--- кинетическая энергия частицы,
    а $E_0$, $p$ и $v$~--- её энергия покоя, импульс и скорость.
}
\answer{%
    \begin{align*}
    E_\text{кин}, E_0:\quad&E = E_\text{кин} + E_0 = \frac{E_0}{\sqrt{1 - \frac{v^2}{c^2}}} \implies \sqrt{1 - \frac{v^2}{c^2}} = \frac{E_0}{{E_0} + {E_\text{кин}}} \implies v = c\sqrt{1 - \sqr{\frac{E_0}{{E_0} + {E_\text{кин}}}}} \\
    &p = \frac{mv}{\sqrt{1 - \frac{v^2}{c^2}}} = \frac{E_0}{c^2} \cdot \sqrt{1 - \sqr{\frac{E_0}{{E_0} + {E_\text{кин}}}}} \cdot \frac{{E_\text{кин}} + {E_0}}{E_0} = \frac{E_0}{c^2} \cdot \sqrt{\sqr{\frac{{E_\text{кин}} + {E_0}}{E_0}} - 1}.
    \\
    E_\text{кин}, p:\quad&E_\text{кин} = E - E_0 = mc^2\cbr{\frac 1{\sqrt{1 - \frac{v^2}{c^2}}} - 1}, p = \frac{mv}{\sqrt{1 - \frac{v^2}{c^2}}} \implies \frac{E_\text{кин}}{p} = \frac{\frac 1{\sqrt{1 - \frac{v^2}{c^2}}} - 1}{\sqrt{1 - \frac{v^2}{c^2}}} \implies v = \ldots \\
    &E_0 = E - E_\text{кин} = \frac{E_0}{\sqrt{1 - \frac{v^2}{c^2}}} - E_\text{кин} \implies E_0 = \frac{E_\text{кин}}{\frac 1{\sqrt{1 - \frac{v^2}{c^2}}} - 1} = \ldots \\
    E_\text{кин}, v:\quad&E_\text{кин} = E - E_0 = mc^2\cbr{\frac 1{\sqrt{1 - \frac{v^2}{c^2}}} - 1} \implies m = \frac{E_\text{кин}}{c^2\cbr{\frac 1{\sqrt{1 - \frac{v^2}{c^2}}} - 1}} \\
    &E_0 = mc^2 = \frac{E_\text{кин}}{\frac 1{\sqrt{1 - \frac{v^2}{c^2}}} - 1} \\
    &p = \frac{mv}{\sqrt{1 - \frac{v^2}{c^2}}} = \frac{E_\text{кин}}{c^2\cbr{\frac 1{\sqrt{1 - \frac{v^2}{c^2}}} - 1}} \cdot \frac{v}{\sqrt{1 - \frac{v^2}{c^2}}} = \frac{{E_\text{кин}} v}{c^2\cbr{1 - {\sqrt{1 - \frac{v^2}{c^2}}}}} \\
    E_0, p:\quad&E_0 = mc^2, \quad p = \frac{mv}{\sqrt{1 - \frac{v^2}{c^2}}} \implies \frac{E_0}{p} = \frac{c^2}v{\sqrt{1 - \frac{v^2}{c^2}}} = c\sqrt{\frac{c^2}{v^2} - 1} \\
    &\sqr{\frac{E_0}{pc}} = \frac{c^2}{v^2} - 1 \implies \frac{v^2}{c^2} = \frac 1{1 + \frac{E_0^2}{p^2c^2}} \implies v = \frac c{\sqrt{1 + \frac{E_0^2}{p^2c^2}}} \\
    &{E_\text{кин}} = E - E_0 = \sqrt{E_0^2 + p^2c^2} - E_0 \\
    E_0, v:\quad&E_0 = mc^2 \implies m = \frac{E_0}{c^2} \qquad p = \frac{mv}{\sqrt{1 - \frac{v^2}{c^2}}} = \frac{E_0}{c^2} \cdot \frac{v}{\sqrt{1 - \frac{v^2}{c^2}}} \\
    &E_\text{кин}= mc^2\cbr{\frac 1{\sqrt{1 - \frac{v^2}{c^2}}} - 1} = \frac{E_0}{c^2}\cbr{\frac 1{\sqrt{1 - \frac{v^2}{c^2}}} - 1} \\
    p, v:\quad&p = \frac{mv}{\sqrt{1 - \frac{v^2}{c^2}}} \implies m = \frac p v {\sqrt{1 - \frac{v^2}{c^2}}} \implies E_0 = mc^2 =\frac {pc^2} v {\sqrt{1 - \frac{v^2}{c^2}}} \\
    &E_\text{кин} = mc^2\cbr{\frac 1{\sqrt{1 - \frac{v^2}{c^2}}} - 1} = \frac p v {\sqrt{1 - \frac{v^2}{c^2}}}\cbr{\frac 1{\sqrt{1 - \frac{v^2}{c^2}}} - 1} = \frac p v \cbr{1 - {\sqrt{1 - \frac{v^2}{c^2}}}}
    \end{align*}
}

\variantsplitter

\addpersonalvariant{Владимир Артемчук}

\tasknumber{1}%
\task{%
    Запишите
    \begin{itemize}
        \item постулаты специальной теории относительности,
        \item пример релятивистского эффекта, обнаружимый при скоростях гораздо меньше скорости света.
    \end{itemize}
}
\solutionspace{120pt}

\tasknumber{2}%
\task{%
    Запишите формулу для ...
    \begin{itemize}
        \item релятивистского замедления времени,
        \item классической полной механической энергии тела,
        \item релятивистской энергии тела,
        \item релятивистской кинетической энергии,
        \item связь между релятивистским импульсом и релятивистской энергией.
    \end{itemize}
    Обязательно подпишите все физические величины.
}
\solutionspace{150pt}

\tasknumber{3}%
\task{%
    Электрон движется со скоростью $0{,}9\,c$, где $c$~--- скорость света в вакууме.
    Каково при этом отношение полной энергии частицы $E$ к его энергии покоя $E_0$?
}
\answer{%
    \begin{align*}
    E &= \frac{E_0}{\sqrt{1 - \frac{v^2}{c^2}}}
            \implies \frac E{E_0}
                = \frac 1{\sqrt{1 - \frac{v^2}{c^2}}}
                = \frac 1{\sqrt{1 - \sqr{0{,}9}}}
                \approx 2{,}294,
         \\
        {E_{\text{кин}}} &= E - E_0
            \implies \frac{E_{\text{кин}}}{E_0}
                = \frac E{E_0} - 1
                = \frac 1{\sqrt{1 - \frac{v^2}{c^2}}} - 1
                = \frac 1{\sqrt{1 - \sqr{0{,}9}}} - 1
                \approx 1{,}294.
    \end{align*}
}
\solutionspace{80pt}

\tasknumber{4}%
\task{%
    Полная энергия релятивистской частицы в четыре раза больше её энергии покоя.
    Найти скорость этой частицы: в долях $c$ и численное значение.
    Скорость света в вакууме $c = 3 \cdot 10^{8}\,\frac{\text{м}}{\text{с}}$.
}
\answer{%
    \begin{align*}
    E &= \frac{E_0}{\sqrt{1 - \frac{v^2}{c^2}}}\implies \sqrt{1 - \frac{v^2}{c^2}} = \frac{E_0}{E}\implies \frac{v^2}{c^2} = 1 - \sqr{\frac{E_0}{E}}\implies v = c \sqrt{1 - \sqr{\frac{E_0}{E}}} \approx 0{,}968c \approx 290 \cdot 10^{6}\,\frac{\text{м}}{\text{с}}.
    \end{align*}
}
\solutionspace{80pt}

\tasknumber{5}%
\task{%
    Кинетическая энергия релятивистской частицы в четыре раза больше её энергии покоя.
    Найти скорость этой частицы.
    Скорость света в вакууме $c = 3 \cdot 10^{8}\,\frac{\text{м}}{\text{с}}$.
}
\answer{%
    \begin{align*}
    E &= E_0 + E_{\text{кин}} \\
    E &= \frac{E_0}{\sqrt{1 - \frac{v^2}{c^2}}}\implies \sqrt{1 - \frac{v^2}{c^2}} = \frac{E_0}{E}\implies \frac{v^2}{c^2} = 1 - \sqr{\frac{E_0}{E}} \implies \\
    \implies &v = c \sqrt{1 - \sqr{\frac{E_0}{E}}} = c \sqrt{1 - \sqr{\frac{E_0}{E_0 + E_{\text{кин}} }}} = c \sqrt{1 - \frac 1 {\sqr{ 1 + \frac{E_{\text{кин}}}{E_0} }} }\approx 0{,}980c \approx 294 \cdot 10^{6}\,\frac{\text{м}}{\text{с}}.
    \end{align*}
}


\variantsplitter


\addpersonalvariant{Владимир Артемчук}

\tasknumber{6}%
\task{%
    Протон движется со скоростью $0{,}75\,c$, где $c$~--- скорость света в вакууме.
    Определите его кинетическую энергию (в ответе приведите формулу и укажите численное значение).
}
\answer{%
    \begin{align*}
    E &= \frac{mc^2}{\sqrt{1 - \frac{v^2}{c^2}}}
            \approx \frac{1{,}673 \cdot 10^{-27}\,\text{кг} \cdot \sqr{3 \cdot 10^{8}\,\frac{\text{м}}{\text{с}}}}{\sqrt{1 - 0{,}75^2}}
            \approx 0{,}2276 \cdot 10^{-9}\,\text{Дж},
         \\
        E_{\text{кин}} &= \frac{mc^2}{\sqrt{1 - \frac{v^2}{c^2}}} - mc^2
            = mc^2 \cbr{\frac 1{\sqrt{1 - \frac{v^2}{c^2}}} - 1} \approx \\
            &\approx \cbr{1{,}673 \cdot 10^{-27}\,\text{кг} \cdot \sqr{3 \cdot 10^{8}\,\frac{\text{м}}{\text{с}}}}
            \cdot \cbr{\frac 1{\sqrt{1 - 0{,}75^2}} - 1}
            \approx 77{,}05 \cdot 10^{-12}\,\text{Дж},
         \\
        p &= \frac{mv}{\sqrt{1 - \frac{v^2}{c^2}}}
            \approx \frac{1{,}673 \cdot 10^{-27}\,\text{кг} \cdot 0{,}75 \cdot 3 \cdot 10^{8}\,\frac{\text{м}}{\text{с}}}{\sqrt{1 - 0{,}75^2}}
            \approx 0{,}5690 \cdot 10^{-18}\,\frac{\text{кг}\cdot\text{м}}{\text{с}}.
    \end{align*}
}
\solutionspace{100pt}

\tasknumber{7}%
\task{%
    Кинетическая энергия частицы космических лучей в четыре раза превышает её энергию покоя.
    Определить отношение скорости частицы к скорости света.
}
\answer{%
    \begin{align*}
    E &= E_0 + E_{\text{кин}} \\
    E &= \frac{E_0}{\sqrt{1 - \frac{v^2}{c^2}}}\implies \sqrt{1 - \frac{v^2}{c^2}} = \frac{E_0}{E}\implies \frac{v^2}{c^2} = 1 - \sqr{\frac{E_0}{E}} \implies \\
    \implies \frac vc &= \sqrt{1 - \sqr{\frac{E_0}{E}}} = \sqrt{1 - \sqr{\frac{E_0}{E_0 + E_{\text{кин}} }}} \approx 0{,}980.
    \end{align*}
}
\solutionspace{80pt}

\tasknumber{8}%
\task{%
    Некоторая частица, пройдя ускоряющую разность потенциалов, приобрела импульс $3 \cdot 10^{-19}\,\frac{\text{кг}\cdot\text{м}}{\text{с}}$.
    Скорость частицы стала равной $1{,}8 \cdot 10^{8}\,\frac{\text{м}}{\text{с}}$.
    Найти массу частицы.
}
\answer{%
    $p = \frac{ mv }{\sqrt{1 - \frac{v^2}{c^2} }}\implies m = \frac pv \sqrt{1 - \frac{v^2}{c^2}}= \frac {3 \cdot 10^{-19}\,\frac{\text{кг}\cdot\text{м}}{\text{с}}}{1{,}8 \cdot 10^{8}\,\frac{\text{м}}{\text{с}}} \sqrt{1 - \sqr{\frac{1{,}8 \cdot 10^{8}\,\frac{\text{м}}{\text{с}}}{3 \cdot 10^{8}\,\frac{\text{м}}{\text{с}}}} } \approx 1{,}33 \cdot 10^{-27}\,\text{кг}.$
}
\solutionspace{80pt}

\tasknumber{9}%
\task{%
    При какой скорости движения (в м/с) релятивистское сокращение длины движущегося тела
    составит 30\%?
}
\answer{%
    \begin{align*}
    l_0 &= \frac l{\sqrt{1 - \frac{v^2}{c^2}}}
        \implies 1 - \frac{v^2}{c^2} = \sqr{\frac l{l_0}}
        \implies \frac v c = \sqrt{1 - \sqr{\frac l{l_0}}} \implies
         \\
        \implies v &= c\sqrt{1 - \sqr{\frac l{l_0}}}
        = 3 \cdot 10^{8}\,\frac{\text{м}}{\text{с}} \cdot \sqrt{1 - \sqr{\frac {l_0 - 0{,}30l_0}{l_0}}}
        = 3 \cdot 10^{8}\,\frac{\text{м}}{\text{с}} \cdot \sqrt{1 - \sqr{1 - 0{,}30}} \approx  \\
        &\approx 0{,}714c
        \approx 214 \cdot 10^{6}\,\frac{\text{м}}{\text{с}}
        \approx 771 \cdot 10^{6}\,\frac{\text{км}}{\text{ч}}.
    \end{align*}
}
\solutionspace{80pt}

\tasknumber{10}%
\task{%
    Стержень движется в продольном направлении с постоянной скоростью относительно инерциальной системы отсчёта.
    При каком значении скорости (в долях скорости света) длина стержня в этой системе отсчёта
    будет в  4  раза меньше его собственной длины?
}
\answer{%
    $l_0 = \frac l{\sqrt{1 - \frac{v^2}{c^2}}}\implies \sqrt{1 - \frac{v^2}{c^2}} = \frac{ l }{ l_0 }\implies \frac v c = \sqrt{1 - \sqr{\frac{ l }{ l_0 }}} \approx 0{,}968.$
}
\solutionspace{80pt}

\tasknumber{11}%
\task{%
    Какую скорость должно иметь движущееся тело, чтобы его продольные размеры уменьшились в три раза?
    Скорость света $c = 3 \cdot 10^{8}\,\frac{\text{м}}{\text{с}}$.
}
\answer{%
    $l_0 = \frac l{\sqrt{1 - \frac{v^2}{c^2}}}\implies \sqrt{1 - \frac{v^2}{c^2}} = \frac{ l }{ l_0 }\implies v = c\sqrt{1 - \sqr{\frac{ l }{ l_0 }}} \approx 283 \cdot 10^{6}\,\frac{\text{м}}{\text{с}}.$
}


\variantsplitter


\addpersonalvariant{Владимир Артемчук}

\tasknumber{12}%
\task{%
    Время жизни мюона, измеренное наблюдателем, относительно которого мюон покоился, равно $\tau_0$
    Какое расстояние пролетит мюон в системе отсчёта, относительно которой он движется со скоростью $v$,
    сравнимой со скоростью света в вакууме $c$?
}
\answer{%
    $\ell = v\tau = v \frac{\tau_0}{\sqrt{1 - \frac{v^2}{c^2}}}$
}
\solutionspace{80pt}

\tasknumber{13}%
\task{%
    Если $c$ — скорость света в вакууме, то с какой скоростью должна двигаться нестабильная частица относительно наблюдателя,
    чтобы её время жизни было в четыре раза больше, чем у такой же, но покоящейся относительно наблюдателя частицы?
}
\answer{%
    $\tau = \frac{\tau_0}{\sqrt{1 - \frac{v^2}{c^2}}}\implies \sqrt{1 - \frac{v^2}{c^2}} = \frac{\tau_0}{\tau}\implies v = c\sqrt{1 - \sqr{\frac{\tau_0}{\tau}} } \approx 290 \cdot 10^{6}\,\frac{\text{м}}{\text{с}}.$
}
\solutionspace{80pt}

\tasknumber{14}%
\task{%
    Время жизни нестабильной частицы, входящего в состав космических лучей, измеренное земным наблюдателем,
    относительно которого частица двигалась со скоростью, составляющей 65\% скорости света в вакууме, оказалось равным $4{,}8\,\text{мкс}$.
    Каково время жизни частицы, покоящейся относительно наблюдателя?
}
\answer{%
    $t = \frac{t_0}{\sqrt{1 - \frac{v^2}{c^2}}}\implies t_0 = t\sqrt{1 - \frac{v^2}{c^2}} \approx 3{,}6 \cdot 10^{-6}\,\text{с}.$
}
\solutionspace{80pt}

\tasknumber{15}%
\task{%
    Частица увеличила в ускорителе свою скорость с $0{,}03c$ до $0{,}60c$.
    Во сколько раз выросла её кинетическая энергия?
}
\answer{%
    \begin{align*}
    E_{\text{кин.}} &= E - E_0 = \frac{mc^2}{\sqrt{1 - \frac{v^2}{c^2}}} - mc^2 = mc^2\cbr{ \frac1{\sqrt{1 - \frac{v^2}{c^2} }} - 1}.
    \\
    \frac{E_{\text{кин.
    2}}}{E_{\text{кин.
    1}}} &= \frac{\frac1{\sqrt{1 - \frac{v_2^2}{c^2} }} - 1}{\frac1{\sqrt{1 - \frac{v_1^2}{c^2} }} - 1}\approx 555{,}18
    \end{align*}
}
\solutionspace{120pt}

\tasknumber{16}%
\task{%
    Для частицы, движущейся с релятивистской скоростью,
    выразите $p$ и $E_\text{кин}$ через $c$, $v$ и $E_0$,
    где $E_\text{кин}$~--- кинетическая энергия частицы,
    а $E_0$, $p$ и $v$~--- её энергия покоя, импульс и скорость.
}
\answer{%
    \begin{align*}
    E_\text{кин}, E_0:\quad&E = E_\text{кин} + E_0 = \frac{E_0}{\sqrt{1 - \frac{v^2}{c^2}}} \implies \sqrt{1 - \frac{v^2}{c^2}} = \frac{E_0}{{E_0} + {E_\text{кин}}} \implies v = c\sqrt{1 - \sqr{\frac{E_0}{{E_0} + {E_\text{кин}}}}} \\
    &p = \frac{mv}{\sqrt{1 - \frac{v^2}{c^2}}} = \frac{E_0}{c^2} \cdot \sqrt{1 - \sqr{\frac{E_0}{{E_0} + {E_\text{кин}}}}} \cdot \frac{{E_\text{кин}} + {E_0}}{E_0} = \frac{E_0}{c^2} \cdot \sqrt{\sqr{\frac{{E_\text{кин}} + {E_0}}{E_0}} - 1}.
    \\
    E_\text{кин}, p:\quad&E_\text{кин} = E - E_0 = mc^2\cbr{\frac 1{\sqrt{1 - \frac{v^2}{c^2}}} - 1}, p = \frac{mv}{\sqrt{1 - \frac{v^2}{c^2}}} \implies \frac{E_\text{кин}}{p} = \frac{\frac 1{\sqrt{1 - \frac{v^2}{c^2}}} - 1}{\sqrt{1 - \frac{v^2}{c^2}}} \implies v = \ldots \\
    &E_0 = E - E_\text{кин} = \frac{E_0}{\sqrt{1 - \frac{v^2}{c^2}}} - E_\text{кин} \implies E_0 = \frac{E_\text{кин}}{\frac 1{\sqrt{1 - \frac{v^2}{c^2}}} - 1} = \ldots \\
    E_\text{кин}, v:\quad&E_\text{кин} = E - E_0 = mc^2\cbr{\frac 1{\sqrt{1 - \frac{v^2}{c^2}}} - 1} \implies m = \frac{E_\text{кин}}{c^2\cbr{\frac 1{\sqrt{1 - \frac{v^2}{c^2}}} - 1}} \\
    &E_0 = mc^2 = \frac{E_\text{кин}}{\frac 1{\sqrt{1 - \frac{v^2}{c^2}}} - 1} \\
    &p = \frac{mv}{\sqrt{1 - \frac{v^2}{c^2}}} = \frac{E_\text{кин}}{c^2\cbr{\frac 1{\sqrt{1 - \frac{v^2}{c^2}}} - 1}} \cdot \frac{v}{\sqrt{1 - \frac{v^2}{c^2}}} = \frac{{E_\text{кин}} v}{c^2\cbr{1 - {\sqrt{1 - \frac{v^2}{c^2}}}}} \\
    E_0, p:\quad&E_0 = mc^2, \quad p = \frac{mv}{\sqrt{1 - \frac{v^2}{c^2}}} \implies \frac{E_0}{p} = \frac{c^2}v{\sqrt{1 - \frac{v^2}{c^2}}} = c\sqrt{\frac{c^2}{v^2} - 1} \\
    &\sqr{\frac{E_0}{pc}} = \frac{c^2}{v^2} - 1 \implies \frac{v^2}{c^2} = \frac 1{1 + \frac{E_0^2}{p^2c^2}} \implies v = \frac c{\sqrt{1 + \frac{E_0^2}{p^2c^2}}} \\
    &{E_\text{кин}} = E - E_0 = \sqrt{E_0^2 + p^2c^2} - E_0 \\
    E_0, v:\quad&E_0 = mc^2 \implies m = \frac{E_0}{c^2} \qquad p = \frac{mv}{\sqrt{1 - \frac{v^2}{c^2}}} = \frac{E_0}{c^2} \cdot \frac{v}{\sqrt{1 - \frac{v^2}{c^2}}} \\
    &E_\text{кин}= mc^2\cbr{\frac 1{\sqrt{1 - \frac{v^2}{c^2}}} - 1} = \frac{E_0}{c^2}\cbr{\frac 1{\sqrt{1 - \frac{v^2}{c^2}}} - 1} \\
    p, v:\quad&p = \frac{mv}{\sqrt{1 - \frac{v^2}{c^2}}} \implies m = \frac p v {\sqrt{1 - \frac{v^2}{c^2}}} \implies E_0 = mc^2 =\frac {pc^2} v {\sqrt{1 - \frac{v^2}{c^2}}} \\
    &E_\text{кин} = mc^2\cbr{\frac 1{\sqrt{1 - \frac{v^2}{c^2}}} - 1} = \frac p v {\sqrt{1 - \frac{v^2}{c^2}}}\cbr{\frac 1{\sqrt{1 - \frac{v^2}{c^2}}} - 1} = \frac p v \cbr{1 - {\sqrt{1 - \frac{v^2}{c^2}}}}
    \end{align*}
}

\variantsplitter

\addpersonalvariant{Софья Белянкина}

\tasknumber{1}%
\task{%
    Запишите
    \begin{itemize}
        \item постулаты специальной теории относительности,
        \item пример релятивистского эффекта, обнаружимый при скоростях гораздо меньше скорости света.
    \end{itemize}
}
\solutionspace{120pt}

\tasknumber{2}%
\task{%
    Запишите формулу для ...
    \begin{itemize}
        \item релятивистского замедления времени,
        \item классической полной механической энергии тела,
        \item релятивистского импульса тела,
        \item энергии покоя тела,
        \item связь между релятивистским импульсом и релятивистской энергией.
    \end{itemize}
    Обязательно подпишите все физические величины.
}
\solutionspace{150pt}

\tasknumber{3}%
\task{%
    Протон движется со скоростью $0{,}7\,c$, где $c$~--- скорость света в вакууме.
    Каково при этом отношение полной энергии частицы $E$ к его энергии покоя $E_0$?
}
\answer{%
    \begin{align*}
    E &= \frac{E_0}{\sqrt{1 - \frac{v^2}{c^2}}}
            \implies \frac E{E_0}
                = \frac 1{\sqrt{1 - \frac{v^2}{c^2}}}
                = \frac 1{\sqrt{1 - \sqr{0{,}7}}}
                \approx 1{,}400,
         \\
        {E_{\text{кин}}} &= E - E_0
            \implies \frac{E_{\text{кин}}}{E_0}
                = \frac E{E_0} - 1
                = \frac 1{\sqrt{1 - \frac{v^2}{c^2}}} - 1
                = \frac 1{\sqrt{1 - \sqr{0{,}7}}} - 1
                \approx 0{,}400.
    \end{align*}
}
\solutionspace{80pt}

\tasknumber{4}%
\task{%
    Полная энергия релятивистской частицы в четыре раза больше её энергии покоя.
    Найти скорость этой частицы: в долях $c$ и численное значение.
    Скорость света в вакууме $c = 3 \cdot 10^{8}\,\frac{\text{м}}{\text{с}}$.
}
\answer{%
    \begin{align*}
    E &= \frac{E_0}{\sqrt{1 - \frac{v^2}{c^2}}}\implies \sqrt{1 - \frac{v^2}{c^2}} = \frac{E_0}{E}\implies \frac{v^2}{c^2} = 1 - \sqr{\frac{E_0}{E}}\implies v = c \sqrt{1 - \sqr{\frac{E_0}{E}}} \approx 0{,}968c \approx 290 \cdot 10^{6}\,\frac{\text{м}}{\text{с}}.
    \end{align*}
}
\solutionspace{80pt}

\tasknumber{5}%
\task{%
    Кинетическая энергия релятивистской частицы в четыре раза больше её энергии покоя.
    Найти скорость этой частицы.
    Скорость света в вакууме $c = 3 \cdot 10^{8}\,\frac{\text{м}}{\text{с}}$.
}
\answer{%
    \begin{align*}
    E &= E_0 + E_{\text{кин}} \\
    E &= \frac{E_0}{\sqrt{1 - \frac{v^2}{c^2}}}\implies \sqrt{1 - \frac{v^2}{c^2}} = \frac{E_0}{E}\implies \frac{v^2}{c^2} = 1 - \sqr{\frac{E_0}{E}} \implies \\
    \implies &v = c \sqrt{1 - \sqr{\frac{E_0}{E}}} = c \sqrt{1 - \sqr{\frac{E_0}{E_0 + E_{\text{кин}} }}} = c \sqrt{1 - \frac 1 {\sqr{ 1 + \frac{E_{\text{кин}}}{E_0} }} }\approx 0{,}980c \approx 294 \cdot 10^{6}\,\frac{\text{м}}{\text{с}}.
    \end{align*}
}


\variantsplitter


\addpersonalvariant{Софья Белянкина}

\tasknumber{6}%
\task{%
    Протон движется со скоростью $0{,}75\,c$, где $c$~--- скорость света в вакууме.
    Определите его кинетическую энергию (в ответе приведите формулу и укажите численное значение).
}
\answer{%
    \begin{align*}
    E &= \frac{mc^2}{\sqrt{1 - \frac{v^2}{c^2}}}
            \approx \frac{1{,}673 \cdot 10^{-27}\,\text{кг} \cdot \sqr{3 \cdot 10^{8}\,\frac{\text{м}}{\text{с}}}}{\sqrt{1 - 0{,}75^2}}
            \approx 0{,}2276 \cdot 10^{-9}\,\text{Дж},
         \\
        E_{\text{кин}} &= \frac{mc^2}{\sqrt{1 - \frac{v^2}{c^2}}} - mc^2
            = mc^2 \cbr{\frac 1{\sqrt{1 - \frac{v^2}{c^2}}} - 1} \approx \\
            &\approx \cbr{1{,}673 \cdot 10^{-27}\,\text{кг} \cdot \sqr{3 \cdot 10^{8}\,\frac{\text{м}}{\text{с}}}}
            \cdot \cbr{\frac 1{\sqrt{1 - 0{,}75^2}} - 1}
            \approx 77{,}05 \cdot 10^{-12}\,\text{Дж},
         \\
        p &= \frac{mv}{\sqrt{1 - \frac{v^2}{c^2}}}
            \approx \frac{1{,}673 \cdot 10^{-27}\,\text{кг} \cdot 0{,}75 \cdot 3 \cdot 10^{8}\,\frac{\text{м}}{\text{с}}}{\sqrt{1 - 0{,}75^2}}
            \approx 0{,}5690 \cdot 10^{-18}\,\frac{\text{кг}\cdot\text{м}}{\text{с}}.
    \end{align*}
}
\solutionspace{100pt}

\tasknumber{7}%
\task{%
    Кинетическая энергия частицы космических лучей в четыре раза превышает её энергию покоя.
    Определить отношение скорости частицы к скорости света.
}
\answer{%
    \begin{align*}
    E &= E_0 + E_{\text{кин}} \\
    E &= \frac{E_0}{\sqrt{1 - \frac{v^2}{c^2}}}\implies \sqrt{1 - \frac{v^2}{c^2}} = \frac{E_0}{E}\implies \frac{v^2}{c^2} = 1 - \sqr{\frac{E_0}{E}} \implies \\
    \implies \frac vc &= \sqrt{1 - \sqr{\frac{E_0}{E}}} = \sqrt{1 - \sqr{\frac{E_0}{E_0 + E_{\text{кин}} }}} \approx 0{,}980.
    \end{align*}
}
\solutionspace{80pt}

\tasknumber{8}%
\task{%
    Некоторая частица, пройдя ускоряющую разность потенциалов, приобрела импульс $3{,}8 \cdot 10^{-19}\,\frac{\text{кг}\cdot\text{м}}{\text{с}}$.
    Скорость частицы стала равной $1{,}5 \cdot 10^{8}\,\frac{\text{м}}{\text{с}}$.
    Найти массу частицы.
}
\answer{%
    $p = \frac{ mv }{\sqrt{1 - \frac{v^2}{c^2} }}\implies m = \frac pv \sqrt{1 - \frac{v^2}{c^2}}= \frac {3{,}8 \cdot 10^{-19}\,\frac{\text{кг}\cdot\text{м}}{\text{с}}}{1{,}5 \cdot 10^{8}\,\frac{\text{м}}{\text{с}}} \sqrt{1 - \sqr{\frac{1{,}5 \cdot 10^{8}\,\frac{\text{м}}{\text{с}}}{3 \cdot 10^{8}\,\frac{\text{м}}{\text{с}}}} } \approx 2{,}2 \cdot 10^{-27}\,\text{кг}.$
}
\solutionspace{80pt}

\tasknumber{9}%
\task{%
    При какой скорости движения (в долях скорости света) релятивистское сокращение длины движущегося тела
    составит 50\%?
}
\answer{%
    \begin{align*}
    l_0 &= \frac l{\sqrt{1 - \frac{v^2}{c^2}}}
        \implies 1 - \frac{v^2}{c^2} = \sqr{\frac l{l_0}}
        \implies \frac v c = \sqrt{1 - \sqr{\frac l{l_0}}} \implies
         \\
        \implies v &= c\sqrt{1 - \sqr{\frac l{l_0}}}
        = 3 \cdot 10^{8}\,\frac{\text{м}}{\text{с}} \cdot \sqrt{1 - \sqr{\frac {l_0 - 0{,}50l_0}{l_0}}}
        = 3 \cdot 10^{8}\,\frac{\text{м}}{\text{с}} \cdot \sqrt{1 - \sqr{1 - 0{,}50}} \approx  \\
        &\approx 0{,}866c
        \approx 260 \cdot 10^{6}\,\frac{\text{м}}{\text{с}}
        \approx 935 \cdot 10^{6}\,\frac{\text{км}}{\text{ч}}.
    \end{align*}
}
\solutionspace{80pt}

\tasknumber{10}%
\task{%
    Стержень движется в продольном направлении с постоянной скоростью относительно инерциальной системы отсчёта.
    При каком значении скорости (в долях скорости света) длина стержня в этой системе отсчёта
    будет в  3  раза меньше его собственной длины?
}
\answer{%
    $l_0 = \frac l{\sqrt{1 - \frac{v^2}{c^2}}}\implies \sqrt{1 - \frac{v^2}{c^2}} = \frac{ l }{ l_0 }\implies \frac v c = \sqrt{1 - \sqr{\frac{ l }{ l_0 }}} \approx 0{,}943.$
}
\solutionspace{80pt}

\tasknumber{11}%
\task{%
    Какую скорость должно иметь движущееся тело, чтобы его продольные размеры уменьшились в четыре раза?
    Скорость света $c = 3 \cdot 10^{8}\,\frac{\text{м}}{\text{с}}$.
}
\answer{%
    $l_0 = \frac l{\sqrt{1 - \frac{v^2}{c^2}}}\implies \sqrt{1 - \frac{v^2}{c^2}} = \frac{ l }{ l_0 }\implies v = c\sqrt{1 - \sqr{\frac{ l }{ l_0 }}} \approx 290 \cdot 10^{6}\,\frac{\text{м}}{\text{с}}.$
}


\variantsplitter


\addpersonalvariant{Софья Белянкина}

\tasknumber{12}%
\task{%
    Время жизни мюона, измеренное наблюдателем, относительно которого мюон покоился, равно $\tau_0$
    Какое расстояние пролетит мюон в системе отсчёта, относительно которой он движется со скоростью $v$,
    сравнимой со скоростью света в вакууме $c$?
}
\answer{%
    $\ell = v\tau = v \frac{\tau_0}{\sqrt{1 - \frac{v^2}{c^2}}}$
}
\solutionspace{80pt}

\tasknumber{13}%
\task{%
    Если $c$ — скорость света в вакууме, то с какой скоростью должна двигаться нестабильная частица относительно наблюдателя,
    чтобы её время жизни было в восемь раз больше, чем у такой же, но покоящейся относительно наблюдателя частицы?
}
\answer{%
    $\tau = \frac{\tau_0}{\sqrt{1 - \frac{v^2}{c^2}}}\implies \sqrt{1 - \frac{v^2}{c^2}} = \frac{\tau_0}{\tau}\implies v = c\sqrt{1 - \sqr{\frac{\tau_0}{\tau}} } \approx 298 \cdot 10^{6}\,\frac{\text{м}}{\text{с}}.$
}
\solutionspace{80pt}

\tasknumber{14}%
\task{%
    Время жизни нестабильной частицы, входящего в состав космических лучей, измеренное земным наблюдателем,
    относительно которого частица двигалась со скоростью, составляющей 75\% скорости света в вакууме, оказалось равным $6{,}4\,\text{мкс}$.
    Каково время жизни частицы, покоящейся относительно наблюдателя?
}
\answer{%
    $t = \frac{t_0}{\sqrt{1 - \frac{v^2}{c^2}}}\implies t_0 = t\sqrt{1 - \frac{v^2}{c^2}} \approx 4{,}2 \cdot 10^{-6}\,\text{с}.$
}
\solutionspace{80pt}

\tasknumber{15}%
\task{%
    Частица увеличила в ускорителе свою скорость с $0{,}02c$ до $0{,}60c$.
    Во сколько раз выросла её кинетическая энергия?
}
\answer{%
    \begin{align*}
    E_{\text{кин.}} &= E - E_0 = \frac{mc^2}{\sqrt{1 - \frac{v^2}{c^2}}} - mc^2 = mc^2\cbr{ \frac1{\sqrt{1 - \frac{v^2}{c^2} }} - 1}.
    \\
    \frac{E_{\text{кин.
    2}}}{E_{\text{кин.
    1}}} &= \frac{\frac1{\sqrt{1 - \frac{v_2^2}{c^2} }} - 1}{\frac1{\sqrt{1 - \frac{v_1^2}{c^2} }} - 1}\approx 1249{,}62
    \end{align*}
}
\solutionspace{120pt}

\tasknumber{16}%
\task{%
    Для частицы, движущейся с релятивистской скоростью,
    выразите $E_\text{кин}$ и $E_0$ через $c$, $p$ и $v$,
    где $E_\text{кин}$~--- кинетическая энергия частицы,
    а $E_0$, $p$ и $v$~--- её энергия покоя, импульс и скорость.
}
\answer{%
    \begin{align*}
    E_\text{кин}, E_0:\quad&E = E_\text{кин} + E_0 = \frac{E_0}{\sqrt{1 - \frac{v^2}{c^2}}} \implies \sqrt{1 - \frac{v^2}{c^2}} = \frac{E_0}{{E_0} + {E_\text{кин}}} \implies v = c\sqrt{1 - \sqr{\frac{E_0}{{E_0} + {E_\text{кин}}}}} \\
    &p = \frac{mv}{\sqrt{1 - \frac{v^2}{c^2}}} = \frac{E_0}{c^2} \cdot \sqrt{1 - \sqr{\frac{E_0}{{E_0} + {E_\text{кин}}}}} \cdot \frac{{E_\text{кин}} + {E_0}}{E_0} = \frac{E_0}{c^2} \cdot \sqrt{\sqr{\frac{{E_\text{кин}} + {E_0}}{E_0}} - 1}.
    \\
    E_\text{кин}, p:\quad&E_\text{кин} = E - E_0 = mc^2\cbr{\frac 1{\sqrt{1 - \frac{v^2}{c^2}}} - 1}, p = \frac{mv}{\sqrt{1 - \frac{v^2}{c^2}}} \implies \frac{E_\text{кин}}{p} = \frac{\frac 1{\sqrt{1 - \frac{v^2}{c^2}}} - 1}{\sqrt{1 - \frac{v^2}{c^2}}} \implies v = \ldots \\
    &E_0 = E - E_\text{кин} = \frac{E_0}{\sqrt{1 - \frac{v^2}{c^2}}} - E_\text{кин} \implies E_0 = \frac{E_\text{кин}}{\frac 1{\sqrt{1 - \frac{v^2}{c^2}}} - 1} = \ldots \\
    E_\text{кин}, v:\quad&E_\text{кин} = E - E_0 = mc^2\cbr{\frac 1{\sqrt{1 - \frac{v^2}{c^2}}} - 1} \implies m = \frac{E_\text{кин}}{c^2\cbr{\frac 1{\sqrt{1 - \frac{v^2}{c^2}}} - 1}} \\
    &E_0 = mc^2 = \frac{E_\text{кин}}{\frac 1{\sqrt{1 - \frac{v^2}{c^2}}} - 1} \\
    &p = \frac{mv}{\sqrt{1 - \frac{v^2}{c^2}}} = \frac{E_\text{кин}}{c^2\cbr{\frac 1{\sqrt{1 - \frac{v^2}{c^2}}} - 1}} \cdot \frac{v}{\sqrt{1 - \frac{v^2}{c^2}}} = \frac{{E_\text{кин}} v}{c^2\cbr{1 - {\sqrt{1 - \frac{v^2}{c^2}}}}} \\
    E_0, p:\quad&E_0 = mc^2, \quad p = \frac{mv}{\sqrt{1 - \frac{v^2}{c^2}}} \implies \frac{E_0}{p} = \frac{c^2}v{\sqrt{1 - \frac{v^2}{c^2}}} = c\sqrt{\frac{c^2}{v^2} - 1} \\
    &\sqr{\frac{E_0}{pc}} = \frac{c^2}{v^2} - 1 \implies \frac{v^2}{c^2} = \frac 1{1 + \frac{E_0^2}{p^2c^2}} \implies v = \frac c{\sqrt{1 + \frac{E_0^2}{p^2c^2}}} \\
    &{E_\text{кин}} = E - E_0 = \sqrt{E_0^2 + p^2c^2} - E_0 \\
    E_0, v:\quad&E_0 = mc^2 \implies m = \frac{E_0}{c^2} \qquad p = \frac{mv}{\sqrt{1 - \frac{v^2}{c^2}}} = \frac{E_0}{c^2} \cdot \frac{v}{\sqrt{1 - \frac{v^2}{c^2}}} \\
    &E_\text{кин}= mc^2\cbr{\frac 1{\sqrt{1 - \frac{v^2}{c^2}}} - 1} = \frac{E_0}{c^2}\cbr{\frac 1{\sqrt{1 - \frac{v^2}{c^2}}} - 1} \\
    p, v:\quad&p = \frac{mv}{\sqrt{1 - \frac{v^2}{c^2}}} \implies m = \frac p v {\sqrt{1 - \frac{v^2}{c^2}}} \implies E_0 = mc^2 =\frac {pc^2} v {\sqrt{1 - \frac{v^2}{c^2}}} \\
    &E_\text{кин} = mc^2\cbr{\frac 1{\sqrt{1 - \frac{v^2}{c^2}}} - 1} = \frac p v {\sqrt{1 - \frac{v^2}{c^2}}}\cbr{\frac 1{\sqrt{1 - \frac{v^2}{c^2}}} - 1} = \frac p v \cbr{1 - {\sqrt{1 - \frac{v^2}{c^2}}}}
    \end{align*}
}

\variantsplitter

\addpersonalvariant{Варвара Егиазарян}

\tasknumber{1}%
\task{%
    Запишите
    \begin{itemize}
        \item постулаты специальной теории относительности,
        \item пример релятивистского эффекта, обнаружимый при скоростях гораздо меньше скорости света.
    \end{itemize}
}
\solutionspace{120pt}

\tasknumber{2}%
\task{%
    Запишите формулу для ...
    \begin{itemize}
        \item релятивистского сжатия,
        \item классической полной механической энергии тела,
        \item релятивистского импульса тела,
        \item энергии покоя тела,
        \item связь между релятивистским импульсом и релятивистской энергией.
    \end{itemize}
    Обязательно подпишите все физические величины.
}
\solutionspace{150pt}

\tasknumber{3}%
\task{%
    Протон движется со скоростью $0{,}8\,c$, где $c$~--- скорость света в вакууме.
    Каково при этом отношение полной энергии частицы $E$ к его энергии покоя $E_0$?
}
\answer{%
    \begin{align*}
    E &= \frac{E_0}{\sqrt{1 - \frac{v^2}{c^2}}}
            \implies \frac E{E_0}
                = \frac 1{\sqrt{1 - \frac{v^2}{c^2}}}
                = \frac 1{\sqrt{1 - \sqr{0{,}8}}}
                \approx 1{,}667,
         \\
        {E_{\text{кин}}} &= E - E_0
            \implies \frac{E_{\text{кин}}}{E_0}
                = \frac E{E_0} - 1
                = \frac 1{\sqrt{1 - \frac{v^2}{c^2}}} - 1
                = \frac 1{\sqrt{1 - \sqr{0{,}8}}} - 1
                \approx 0{,}667.
    \end{align*}
}
\solutionspace{80pt}

\tasknumber{4}%
\task{%
    Полная энергия релятивистской частицы в шесть раз больше её энергии покоя.
    Найти скорость этой частицы: в долях $c$ и численное значение.
    Скорость света в вакууме $c = 3 \cdot 10^{8}\,\frac{\text{м}}{\text{с}}$.
}
\answer{%
    \begin{align*}
    E &= \frac{E_0}{\sqrt{1 - \frac{v^2}{c^2}}}\implies \sqrt{1 - \frac{v^2}{c^2}} = \frac{E_0}{E}\implies \frac{v^2}{c^2} = 1 - \sqr{\frac{E_0}{E}}\implies v = c \sqrt{1 - \sqr{\frac{E_0}{E}}} \approx 0{,}986c \approx 296 \cdot 10^{6}\,\frac{\text{м}}{\text{с}}.
    \end{align*}
}
\solutionspace{80pt}

\tasknumber{5}%
\task{%
    Кинетическая энергия релятивистской частицы в шесть раз больше её энергии покоя.
    Найти скорость этой частицы.
    Скорость света в вакууме $c = 3 \cdot 10^{8}\,\frac{\text{м}}{\text{с}}$.
}
\answer{%
    \begin{align*}
    E &= E_0 + E_{\text{кин}} \\
    E &= \frac{E_0}{\sqrt{1 - \frac{v^2}{c^2}}}\implies \sqrt{1 - \frac{v^2}{c^2}} = \frac{E_0}{E}\implies \frac{v^2}{c^2} = 1 - \sqr{\frac{E_0}{E}} \implies \\
    \implies &v = c \sqrt{1 - \sqr{\frac{E_0}{E}}} = c \sqrt{1 - \sqr{\frac{E_0}{E_0 + E_{\text{кин}} }}} = c \sqrt{1 - \frac 1 {\sqr{ 1 + \frac{E_{\text{кин}}}{E_0} }} }\approx 0{,}990c \approx 297 \cdot 10^{6}\,\frac{\text{м}}{\text{с}}.
    \end{align*}
}


\variantsplitter


\addpersonalvariant{Варвара Егиазарян}

\tasknumber{6}%
\task{%
    Протон движется со скоростью $0{,}85\,c$, где $c$~--- скорость света в вакууме.
    Определите его импульс (в ответе приведите формулу и укажите численное значение).
}
\answer{%
    \begin{align*}
    E &= \frac{mc^2}{\sqrt{1 - \frac{v^2}{c^2}}}
            \approx \frac{1{,}673 \cdot 10^{-27}\,\text{кг} \cdot \sqr{3 \cdot 10^{8}\,\frac{\text{м}}{\text{с}}}}{\sqrt{1 - 0{,}85^2}}
            \approx 0{,}2858 \cdot 10^{-9}\,\text{Дж},
         \\
        E_{\text{кин}} &= \frac{mc^2}{\sqrt{1 - \frac{v^2}{c^2}}} - mc^2
            = mc^2 \cbr{\frac 1{\sqrt{1 - \frac{v^2}{c^2}}} - 1} \approx \\
            &\approx \cbr{1{,}673 \cdot 10^{-27}\,\text{кг} \cdot \sqr{3 \cdot 10^{8}\,\frac{\text{м}}{\text{с}}}}
            \cdot \cbr{\frac 1{\sqrt{1 - 0{,}85^2}} - 1}
            \approx 0{,}13523 \cdot 10^{-9}\,\text{Дж},
         \\
        p &= \frac{mv}{\sqrt{1 - \frac{v^2}{c^2}}}
            \approx \frac{1{,}673 \cdot 10^{-27}\,\text{кг} \cdot 0{,}85 \cdot 3 \cdot 10^{8}\,\frac{\text{м}}{\text{с}}}{\sqrt{1 - 0{,}85^2}}
            \approx 0{,}8097 \cdot 10^{-18}\,\frac{\text{кг}\cdot\text{м}}{\text{с}}.
    \end{align*}
}
\solutionspace{100pt}

\tasknumber{7}%
\task{%
    Кинетическая энергия частицы космических лучей в шесть раз превышает её энергию покоя.
    Определить отношение скорости частицы к скорости света.
}
\answer{%
    \begin{align*}
    E &= E_0 + E_{\text{кин}} \\
    E &= \frac{E_0}{\sqrt{1 - \frac{v^2}{c^2}}}\implies \sqrt{1 - \frac{v^2}{c^2}} = \frac{E_0}{E}\implies \frac{v^2}{c^2} = 1 - \sqr{\frac{E_0}{E}} \implies \\
    \implies \frac vc &= \sqrt{1 - \sqr{\frac{E_0}{E}}} = \sqrt{1 - \sqr{\frac{E_0}{E_0 + E_{\text{кин}} }}} \approx 0{,}990.
    \end{align*}
}
\solutionspace{80pt}

\tasknumber{8}%
\task{%
    Некоторая частица, пройдя ускоряющую разность потенциалов, приобрела импульс $3{,}5 \cdot 10^{-19}\,\frac{\text{кг}\cdot\text{м}}{\text{с}}$.
    Скорость частицы стала равной $2 \cdot 10^{8}\,\frac{\text{м}}{\text{с}}$.
    Найти массу частицы.
}
\answer{%
    $p = \frac{ mv }{\sqrt{1 - \frac{v^2}{c^2} }}\implies m = \frac pv \sqrt{1 - \frac{v^2}{c^2}}= \frac {3{,}5 \cdot 10^{-19}\,\frac{\text{кг}\cdot\text{м}}{\text{с}}}{2 \cdot 10^{8}\,\frac{\text{м}}{\text{с}}} \sqrt{1 - \sqr{\frac{2 \cdot 10^{8}\,\frac{\text{м}}{\text{с}}}{3 \cdot 10^{8}\,\frac{\text{м}}{\text{с}}}} } \approx 1{,}304 \cdot 10^{-27}\,\text{кг}.$
}
\solutionspace{80pt}

\tasknumber{9}%
\task{%
    При какой скорости движения (в м/с) релятивистское сокращение длины движущегося тела
    составит 10\%?
}
\answer{%
    \begin{align*}
    l_0 &= \frac l{\sqrt{1 - \frac{v^2}{c^2}}}
        \implies 1 - \frac{v^2}{c^2} = \sqr{\frac l{l_0}}
        \implies \frac v c = \sqrt{1 - \sqr{\frac l{l_0}}} \implies
         \\
        \implies v &= c\sqrt{1 - \sqr{\frac l{l_0}}}
        = 3 \cdot 10^{8}\,\frac{\text{м}}{\text{с}} \cdot \sqrt{1 - \sqr{\frac {l_0 - 0{,}10l_0}{l_0}}}
        = 3 \cdot 10^{8}\,\frac{\text{м}}{\text{с}} \cdot \sqrt{1 - \sqr{1 - 0{,}10}} \approx  \\
        &\approx 0{,}436c
        \approx 130{,}8 \cdot 10^{6}\,\frac{\text{м}}{\text{с}}
        \approx 471 \cdot 10^{6}\,\frac{\text{км}}{\text{ч}}.
    \end{align*}
}
\solutionspace{80pt}

\tasknumber{10}%
\task{%
    Стержень движется в продольном направлении с постоянной скоростью относительно инерциальной системы отсчёта.
    При каком значении скорости (в долях скорости света) длина стержня в этой системе отсчёта
    будет в  2{,}5  раза меньше его собственной длины?
}
\answer{%
    $l_0 = \frac l{\sqrt{1 - \frac{v^2}{c^2}}}\implies \sqrt{1 - \frac{v^2}{c^2}} = \frac{ l }{ l_0 }\implies \frac v c = \sqrt{1 - \sqr{\frac{ l }{ l_0 }}} \approx 0{,}917.$
}
\solutionspace{80pt}

\tasknumber{11}%
\task{%
    Какую скорость должно иметь движущееся тело, чтобы его продольные размеры уменьшились в два раза?
    Скорость света $c = 3 \cdot 10^{8}\,\frac{\text{м}}{\text{с}}$.
}
\answer{%
    $l_0 = \frac l{\sqrt{1 - \frac{v^2}{c^2}}}\implies \sqrt{1 - \frac{v^2}{c^2}} = \frac{ l }{ l_0 }\implies v = c\sqrt{1 - \sqr{\frac{ l }{ l_0 }}} \approx 260 \cdot 10^{6}\,\frac{\text{м}}{\text{с}}.$
}


\variantsplitter


\addpersonalvariant{Варвара Егиазарян}

\tasknumber{12}%
\task{%
    Время жизни мюона, измеренное наблюдателем, относительно которого мюон покоился, равно $\tau_0$
    Какое расстояние пролетит мюон в системе отсчёта, относительно которой он движется со скоростью $v$,
    сравнимой со скоростью света в вакууме $c$?
}
\answer{%
    $\ell = v\tau = v \frac{\tau_0}{\sqrt{1 - \frac{v^2}{c^2}}}$
}
\solutionspace{80pt}

\tasknumber{13}%
\task{%
    Если $c$ — скорость света в вакууме, то с какой скоростью должна двигаться нестабильная частица относительно наблюдателя,
    чтобы её время жизни было в десять раз больше, чем у такой же, но покоящейся относительно наблюдателя частицы?
}
\answer{%
    $\tau = \frac{\tau_0}{\sqrt{1 - \frac{v^2}{c^2}}}\implies \sqrt{1 - \frac{v^2}{c^2}} = \frac{\tau_0}{\tau}\implies v = c\sqrt{1 - \sqr{\frac{\tau_0}{\tau}} } \approx 298 \cdot 10^{6}\,\frac{\text{м}}{\text{с}}.$
}
\solutionspace{80pt}

\tasknumber{14}%
\task{%
    Время жизни нестабильной частицы, входящего в состав космических лучей, измеренное земным наблюдателем,
    относительно которого частица двигалась со скоростью, составляющей 85\% скорости света в вакууме, оказалось равным $6{,}4\,\text{мкс}$.
    Каково время жизни частицы, покоящейся относительно наблюдателя?
}
\answer{%
    $t = \frac{t_0}{\sqrt{1 - \frac{v^2}{c^2}}}\implies t_0 = t\sqrt{1 - \frac{v^2}{c^2}} \approx 3{,}4 \cdot 10^{-6}\,\text{с}.$
}
\solutionspace{80pt}

\tasknumber{15}%
\task{%
    Частица увеличила в ускорителе свою скорость с $0{,}03c$ до $0{,}90c$.
    Во сколько раз выросла её кинетическая энергия?
}
\answer{%
    \begin{align*}
    E_{\text{кин.}} &= E - E_0 = \frac{mc^2}{\sqrt{1 - \frac{v^2}{c^2}}} - mc^2 = mc^2\cbr{ \frac1{\sqrt{1 - \frac{v^2}{c^2} }} - 1}.
    \\
    \frac{E_{\text{кин.
    2}}}{E_{\text{кин.
    1}}} &= \frac{\frac1{\sqrt{1 - \frac{v_2^2}{c^2} }} - 1}{\frac1{\sqrt{1 - \frac{v_1^2}{c^2} }} - 1}\approx 2873{,}96
    \end{align*}
}
\solutionspace{120pt}

\tasknumber{16}%
\task{%
    Для частицы, движущейся с релятивистской скоростью,
    выразите $v$ и $E_\text{кин}$ через $c$, $E_0$ и $p$,
    где $E_\text{кин}$~--- кинетическая энергия частицы,
    а $E_0$, $p$ и $v$~--- её энергия покоя, импульс и скорость.
}
\answer{%
    \begin{align*}
    E_\text{кин}, E_0:\quad&E = E_\text{кин} + E_0 = \frac{E_0}{\sqrt{1 - \frac{v^2}{c^2}}} \implies \sqrt{1 - \frac{v^2}{c^2}} = \frac{E_0}{{E_0} + {E_\text{кин}}} \implies v = c\sqrt{1 - \sqr{\frac{E_0}{{E_0} + {E_\text{кин}}}}} \\
    &p = \frac{mv}{\sqrt{1 - \frac{v^2}{c^2}}} = \frac{E_0}{c^2} \cdot \sqrt{1 - \sqr{\frac{E_0}{{E_0} + {E_\text{кин}}}}} \cdot \frac{{E_\text{кин}} + {E_0}}{E_0} = \frac{E_0}{c^2} \cdot \sqrt{\sqr{\frac{{E_\text{кин}} + {E_0}}{E_0}} - 1}.
    \\
    E_\text{кин}, p:\quad&E_\text{кин} = E - E_0 = mc^2\cbr{\frac 1{\sqrt{1 - \frac{v^2}{c^2}}} - 1}, p = \frac{mv}{\sqrt{1 - \frac{v^2}{c^2}}} \implies \frac{E_\text{кин}}{p} = \frac{\frac 1{\sqrt{1 - \frac{v^2}{c^2}}} - 1}{\sqrt{1 - \frac{v^2}{c^2}}} \implies v = \ldots \\
    &E_0 = E - E_\text{кин} = \frac{E_0}{\sqrt{1 - \frac{v^2}{c^2}}} - E_\text{кин} \implies E_0 = \frac{E_\text{кин}}{\frac 1{\sqrt{1 - \frac{v^2}{c^2}}} - 1} = \ldots \\
    E_\text{кин}, v:\quad&E_\text{кин} = E - E_0 = mc^2\cbr{\frac 1{\sqrt{1 - \frac{v^2}{c^2}}} - 1} \implies m = \frac{E_\text{кин}}{c^2\cbr{\frac 1{\sqrt{1 - \frac{v^2}{c^2}}} - 1}} \\
    &E_0 = mc^2 = \frac{E_\text{кин}}{\frac 1{\sqrt{1 - \frac{v^2}{c^2}}} - 1} \\
    &p = \frac{mv}{\sqrt{1 - \frac{v^2}{c^2}}} = \frac{E_\text{кин}}{c^2\cbr{\frac 1{\sqrt{1 - \frac{v^2}{c^2}}} - 1}} \cdot \frac{v}{\sqrt{1 - \frac{v^2}{c^2}}} = \frac{{E_\text{кин}} v}{c^2\cbr{1 - {\sqrt{1 - \frac{v^2}{c^2}}}}} \\
    E_0, p:\quad&E_0 = mc^2, \quad p = \frac{mv}{\sqrt{1 - \frac{v^2}{c^2}}} \implies \frac{E_0}{p} = \frac{c^2}v{\sqrt{1 - \frac{v^2}{c^2}}} = c\sqrt{\frac{c^2}{v^2} - 1} \\
    &\sqr{\frac{E_0}{pc}} = \frac{c^2}{v^2} - 1 \implies \frac{v^2}{c^2} = \frac 1{1 + \frac{E_0^2}{p^2c^2}} \implies v = \frac c{\sqrt{1 + \frac{E_0^2}{p^2c^2}}} \\
    &{E_\text{кин}} = E - E_0 = \sqrt{E_0^2 + p^2c^2} - E_0 \\
    E_0, v:\quad&E_0 = mc^2 \implies m = \frac{E_0}{c^2} \qquad p = \frac{mv}{\sqrt{1 - \frac{v^2}{c^2}}} = \frac{E_0}{c^2} \cdot \frac{v}{\sqrt{1 - \frac{v^2}{c^2}}} \\
    &E_\text{кин}= mc^2\cbr{\frac 1{\sqrt{1 - \frac{v^2}{c^2}}} - 1} = \frac{E_0}{c^2}\cbr{\frac 1{\sqrt{1 - \frac{v^2}{c^2}}} - 1} \\
    p, v:\quad&p = \frac{mv}{\sqrt{1 - \frac{v^2}{c^2}}} \implies m = \frac p v {\sqrt{1 - \frac{v^2}{c^2}}} \implies E_0 = mc^2 =\frac {pc^2} v {\sqrt{1 - \frac{v^2}{c^2}}} \\
    &E_\text{кин} = mc^2\cbr{\frac 1{\sqrt{1 - \frac{v^2}{c^2}}} - 1} = \frac p v {\sqrt{1 - \frac{v^2}{c^2}}}\cbr{\frac 1{\sqrt{1 - \frac{v^2}{c^2}}} - 1} = \frac p v \cbr{1 - {\sqrt{1 - \frac{v^2}{c^2}}}}
    \end{align*}
}

\variantsplitter

\addpersonalvariant{Владислав Емелин}

\tasknumber{1}%
\task{%
    Запишите
    \begin{itemize}
        \item постулаты специальной теории относительности,
        \item пример релятивистского эффекта, обнаружимый при скоростях гораздо меньше скорости света.
    \end{itemize}
}
\solutionspace{120pt}

\tasknumber{2}%
\task{%
    Запишите формулу для ...
    \begin{itemize}
        \item релятивистского сжатия,
        \item классической полной механической энергии тела,
        \item релятивистского импульса тела,
        \item релятивистской кинетической энергии,
        \item связь между релятивистским импульсом и релятивистской энергией.
    \end{itemize}
    Обязательно подпишите все физические величины.
}
\solutionspace{150pt}

\tasknumber{3}%
\task{%
    Электрон движется со скоростью $0{,}6\,c$, где $c$~--- скорость света в вакууме.
    Каково при этом отношение полной энергии частицы $E$ к его энергии покоя $E_0$?
}
\answer{%
    \begin{align*}
    E &= \frac{E_0}{\sqrt{1 - \frac{v^2}{c^2}}}
            \implies \frac E{E_0}
                = \frac 1{\sqrt{1 - \frac{v^2}{c^2}}}
                = \frac 1{\sqrt{1 - \sqr{0{,}6}}}
                \approx 1{,}250,
         \\
        {E_{\text{кин}}} &= E - E_0
            \implies \frac{E_{\text{кин}}}{E_0}
                = \frac E{E_0} - 1
                = \frac 1{\sqrt{1 - \frac{v^2}{c^2}}} - 1
                = \frac 1{\sqrt{1 - \sqr{0{,}6}}} - 1
                \approx 0{,}250.
    \end{align*}
}
\solutionspace{80pt}

\tasknumber{4}%
\task{%
    Полная энергия релятивистской частицы в три раза больше её энергии покоя.
    Найти скорость этой частицы: в долях $c$ и численное значение.
    Скорость света в вакууме $c = 3 \cdot 10^{8}\,\frac{\text{м}}{\text{с}}$.
}
\answer{%
    \begin{align*}
    E &= \frac{E_0}{\sqrt{1 - \frac{v^2}{c^2}}}\implies \sqrt{1 - \frac{v^2}{c^2}} = \frac{E_0}{E}\implies \frac{v^2}{c^2} = 1 - \sqr{\frac{E_0}{E}}\implies v = c \sqrt{1 - \sqr{\frac{E_0}{E}}} \approx 0{,}943c \approx 283 \cdot 10^{6}\,\frac{\text{м}}{\text{с}}.
    \end{align*}
}
\solutionspace{80pt}

\tasknumber{5}%
\task{%
    Кинетическая энергия релятивистской частицы в три раза больше её энергии покоя.
    Найти скорость этой частицы.
    Скорость света в вакууме $c = 3 \cdot 10^{8}\,\frac{\text{м}}{\text{с}}$.
}
\answer{%
    \begin{align*}
    E &= E_0 + E_{\text{кин}} \\
    E &= \frac{E_0}{\sqrt{1 - \frac{v^2}{c^2}}}\implies \sqrt{1 - \frac{v^2}{c^2}} = \frac{E_0}{E}\implies \frac{v^2}{c^2} = 1 - \sqr{\frac{E_0}{E}} \implies \\
    \implies &v = c \sqrt{1 - \sqr{\frac{E_0}{E}}} = c \sqrt{1 - \sqr{\frac{E_0}{E_0 + E_{\text{кин}} }}} = c \sqrt{1 - \frac 1 {\sqr{ 1 + \frac{E_{\text{кин}}}{E_0} }} }\approx 0{,}968c \approx 290 \cdot 10^{6}\,\frac{\text{м}}{\text{с}}.
    \end{align*}
}


\variantsplitter


\addpersonalvariant{Владислав Емелин}

\tasknumber{6}%
\task{%
    Электрон движется со скоростью $0{,}85\,c$, где $c$~--- скорость света в вакууме.
    Определите его кинетическую энергию (в ответе приведите формулу и укажите численное значение).
}
\answer{%
    \begin{align*}
    E &= \frac{mc^2}{\sqrt{1 - \frac{v^2}{c^2}}}
            \approx \frac{9{,}1 \cdot 10^{-31}\,\text{кг} \cdot \sqr{3 \cdot 10^{8}\,\frac{\text{м}}{\text{с}}}}{\sqrt{1 - 0{,}85^2}}
            \approx 0{,}1555 \cdot 10^{-12}\,\text{Дж},
         \\
        E_{\text{кин}} &= \frac{mc^2}{\sqrt{1 - \frac{v^2}{c^2}}} - mc^2
            = mc^2 \cbr{\frac 1{\sqrt{1 - \frac{v^2}{c^2}}} - 1} \approx \\
            &\approx \cbr{9{,}1 \cdot 10^{-31}\,\text{кг} \cdot \sqr{3 \cdot 10^{8}\,\frac{\text{м}}{\text{с}}}}
            \cdot \cbr{\frac 1{\sqrt{1 - 0{,}85^2}} - 1}
            \approx 73{,}6 \cdot 10^{-15}\,\text{Дж},
         \\
        p &= \frac{mv}{\sqrt{1 - \frac{v^2}{c^2}}}
            \approx \frac{9{,}1 \cdot 10^{-31}\,\text{кг} \cdot 0{,}85 \cdot 3 \cdot 10^{8}\,\frac{\text{м}}{\text{с}}}{\sqrt{1 - 0{,}85^2}}
            \approx 0{,}441 \cdot 10^{-21}\,\frac{\text{кг}\cdot\text{м}}{\text{с}}.
    \end{align*}
}
\solutionspace{100pt}

\tasknumber{7}%
\task{%
    Кинетическая энергия частицы космических лучей в три раза превышает её энергию покоя.
    Определить отношение скорости частицы к скорости света.
}
\answer{%
    \begin{align*}
    E &= E_0 + E_{\text{кин}} \\
    E &= \frac{E_0}{\sqrt{1 - \frac{v^2}{c^2}}}\implies \sqrt{1 - \frac{v^2}{c^2}} = \frac{E_0}{E}\implies \frac{v^2}{c^2} = 1 - \sqr{\frac{E_0}{E}} \implies \\
    \implies \frac vc &= \sqrt{1 - \sqr{\frac{E_0}{E}}} = \sqrt{1 - \sqr{\frac{E_0}{E_0 + E_{\text{кин}} }}} \approx 0{,}968.
    \end{align*}
}
\solutionspace{80pt}

\tasknumber{8}%
\task{%
    Некоторая частица, пройдя ускоряющую разность потенциалов, приобрела импульс $4{,}2 \cdot 10^{-19}\,\frac{\text{кг}\cdot\text{м}}{\text{с}}$.
    Скорость частицы стала равной $1{,}5 \cdot 10^{8}\,\frac{\text{м}}{\text{с}}$.
    Найти массу частицы.
}
\answer{%
    $p = \frac{ mv }{\sqrt{1 - \frac{v^2}{c^2} }}\implies m = \frac pv \sqrt{1 - \frac{v^2}{c^2}}= \frac {4{,}2 \cdot 10^{-19}\,\frac{\text{кг}\cdot\text{м}}{\text{с}}}{1{,}5 \cdot 10^{8}\,\frac{\text{м}}{\text{с}}} \sqrt{1 - \sqr{\frac{1{,}5 \cdot 10^{8}\,\frac{\text{м}}{\text{с}}}{3 \cdot 10^{8}\,\frac{\text{м}}{\text{с}}}} } \approx 2{,}4 \cdot 10^{-27}\,\text{кг}.$
}
\solutionspace{80pt}

\tasknumber{9}%
\task{%
    При какой скорости движения (в долях скорости света) релятивистское сокращение длины движущегося тела
    составит 30\%?
}
\answer{%
    \begin{align*}
    l_0 &= \frac l{\sqrt{1 - \frac{v^2}{c^2}}}
        \implies 1 - \frac{v^2}{c^2} = \sqr{\frac l{l_0}}
        \implies \frac v c = \sqrt{1 - \sqr{\frac l{l_0}}} \implies
         \\
        \implies v &= c\sqrt{1 - \sqr{\frac l{l_0}}}
        = 3 \cdot 10^{8}\,\frac{\text{м}}{\text{с}} \cdot \sqrt{1 - \sqr{\frac {l_0 - 0{,}30l_0}{l_0}}}
        = 3 \cdot 10^{8}\,\frac{\text{м}}{\text{с}} \cdot \sqrt{1 - \sqr{1 - 0{,}30}} \approx  \\
        &\approx 0{,}714c
        \approx 214 \cdot 10^{6}\,\frac{\text{м}}{\text{с}}
        \approx 771 \cdot 10^{6}\,\frac{\text{км}}{\text{ч}}.
    \end{align*}
}
\solutionspace{80pt}

\tasknumber{10}%
\task{%
    Стержень движется в продольном направлении с постоянной скоростью относительно инерциальной системы отсчёта.
    При каком значении скорости (в долях скорости света) длина стержня в этой системе отсчёта
    будет в  2{,}5  раза меньше его собственной длины?
}
\answer{%
    $l_0 = \frac l{\sqrt{1 - \frac{v^2}{c^2}}}\implies \sqrt{1 - \frac{v^2}{c^2}} = \frac{ l }{ l_0 }\implies \frac v c = \sqrt{1 - \sqr{\frac{ l }{ l_0 }}} \approx 0{,}917.$
}
\solutionspace{80pt}

\tasknumber{11}%
\task{%
    Какую скорость должно иметь движущееся тело, чтобы его продольные размеры уменьшились в шесть раз?
    Скорость света $c = 3 \cdot 10^{8}\,\frac{\text{м}}{\text{с}}$.
}
\answer{%
    $l_0 = \frac l{\sqrt{1 - \frac{v^2}{c^2}}}\implies \sqrt{1 - \frac{v^2}{c^2}} = \frac{ l }{ l_0 }\implies v = c\sqrt{1 - \sqr{\frac{ l }{ l_0 }}} \approx 296 \cdot 10^{6}\,\frac{\text{м}}{\text{с}}.$
}


\variantsplitter


\addpersonalvariant{Владислав Емелин}

\tasknumber{12}%
\task{%
    Время жизни мюона, измеренное наблюдателем, относительно которого мюон покоился, равно $\tau_0$
    Какое расстояние пролетит мюон в системе отсчёта, относительно которой он движется со скоростью $v$,
    сравнимой со скоростью света в вакууме $c$?
}
\answer{%
    $\ell = v\tau = v \frac{\tau_0}{\sqrt{1 - \frac{v^2}{c^2}}}$
}
\solutionspace{80pt}

\tasknumber{13}%
\task{%
    Если $c$ — скорость света в вакууме, то с какой скоростью должна двигаться нестабильная частица относительно наблюдателя,
    чтобы её время жизни было в три раза больше, чем у такой же, но покоящейся относительно наблюдателя частицы?
}
\answer{%
    $\tau = \frac{\tau_0}{\sqrt{1 - \frac{v^2}{c^2}}}\implies \sqrt{1 - \frac{v^2}{c^2}} = \frac{\tau_0}{\tau}\implies v = c\sqrt{1 - \sqr{\frac{\tau_0}{\tau}} } \approx 283 \cdot 10^{6}\,\frac{\text{м}}{\text{с}}.$
}
\solutionspace{80pt}

\tasknumber{14}%
\task{%
    Время жизни нестабильной частицы, входящего в состав космических лучей, измеренное земным наблюдателем,
    относительно которого частица двигалась со скоростью, составляющей 85\% скорости света в вакууме, оказалось равным $7{,}1\,\text{мкс}$.
    Каково время жизни частицы, покоящейся относительно наблюдателя?
}
\answer{%
    $t = \frac{t_0}{\sqrt{1 - \frac{v^2}{c^2}}}\implies t_0 = t\sqrt{1 - \frac{v^2}{c^2}} \approx 3{,}7 \cdot 10^{-6}\,\text{с}.$
}
\solutionspace{80pt}

\tasknumber{15}%
\task{%
    Частица увеличила в ускорителе свою скорость с $0{,}01c$ до $0{,}90c$.
    Во сколько раз выросла её кинетическая энергия?
}
\answer{%
    \begin{align*}
    E_{\text{кин.}} &= E - E_0 = \frac{mc^2}{\sqrt{1 - \frac{v^2}{c^2}}} - mc^2 = mc^2\cbr{ \frac1{\sqrt{1 - \frac{v^2}{c^2} }} - 1}.
    \\
    \frac{E_{\text{кин.
    2}}}{E_{\text{кин.
    1}}} &= \frac{\frac1{\sqrt{1 - \frac{v_2^2}{c^2} }} - 1}{\frac1{\sqrt{1 - \frac{v_1^2}{c^2} }} - 1}\approx 25881{,}21
    \end{align*}
}
\solutionspace{120pt}

\tasknumber{16}%
\task{%
    Для частицы, движущейся с релятивистской скоростью,
    выразите $v$ и $p$ через $c$, $E_0$ и $E_\text{кин}$,
    где $E_\text{кин}$~--- кинетическая энергия частицы,
    а $E_0$, $p$ и $v$~--- её энергия покоя, импульс и скорость.
}
\answer{%
    \begin{align*}
    E_\text{кин}, E_0:\quad&E = E_\text{кин} + E_0 = \frac{E_0}{\sqrt{1 - \frac{v^2}{c^2}}} \implies \sqrt{1 - \frac{v^2}{c^2}} = \frac{E_0}{{E_0} + {E_\text{кин}}} \implies v = c\sqrt{1 - \sqr{\frac{E_0}{{E_0} + {E_\text{кин}}}}} \\
    &p = \frac{mv}{\sqrt{1 - \frac{v^2}{c^2}}} = \frac{E_0}{c^2} \cdot \sqrt{1 - \sqr{\frac{E_0}{{E_0} + {E_\text{кин}}}}} \cdot \frac{{E_\text{кин}} + {E_0}}{E_0} = \frac{E_0}{c^2} \cdot \sqrt{\sqr{\frac{{E_\text{кин}} + {E_0}}{E_0}} - 1}.
    \\
    E_\text{кин}, p:\quad&E_\text{кин} = E - E_0 = mc^2\cbr{\frac 1{\sqrt{1 - \frac{v^2}{c^2}}} - 1}, p = \frac{mv}{\sqrt{1 - \frac{v^2}{c^2}}} \implies \frac{E_\text{кин}}{p} = \frac{\frac 1{\sqrt{1 - \frac{v^2}{c^2}}} - 1}{\sqrt{1 - \frac{v^2}{c^2}}} \implies v = \ldots \\
    &E_0 = E - E_\text{кин} = \frac{E_0}{\sqrt{1 - \frac{v^2}{c^2}}} - E_\text{кин} \implies E_0 = \frac{E_\text{кин}}{\frac 1{\sqrt{1 - \frac{v^2}{c^2}}} - 1} = \ldots \\
    E_\text{кин}, v:\quad&E_\text{кин} = E - E_0 = mc^2\cbr{\frac 1{\sqrt{1 - \frac{v^2}{c^2}}} - 1} \implies m = \frac{E_\text{кин}}{c^2\cbr{\frac 1{\sqrt{1 - \frac{v^2}{c^2}}} - 1}} \\
    &E_0 = mc^2 = \frac{E_\text{кин}}{\frac 1{\sqrt{1 - \frac{v^2}{c^2}}} - 1} \\
    &p = \frac{mv}{\sqrt{1 - \frac{v^2}{c^2}}} = \frac{E_\text{кин}}{c^2\cbr{\frac 1{\sqrt{1 - \frac{v^2}{c^2}}} - 1}} \cdot \frac{v}{\sqrt{1 - \frac{v^2}{c^2}}} = \frac{{E_\text{кин}} v}{c^2\cbr{1 - {\sqrt{1 - \frac{v^2}{c^2}}}}} \\
    E_0, p:\quad&E_0 = mc^2, \quad p = \frac{mv}{\sqrt{1 - \frac{v^2}{c^2}}} \implies \frac{E_0}{p} = \frac{c^2}v{\sqrt{1 - \frac{v^2}{c^2}}} = c\sqrt{\frac{c^2}{v^2} - 1} \\
    &\sqr{\frac{E_0}{pc}} = \frac{c^2}{v^2} - 1 \implies \frac{v^2}{c^2} = \frac 1{1 + \frac{E_0^2}{p^2c^2}} \implies v = \frac c{\sqrt{1 + \frac{E_0^2}{p^2c^2}}} \\
    &{E_\text{кин}} = E - E_0 = \sqrt{E_0^2 + p^2c^2} - E_0 \\
    E_0, v:\quad&E_0 = mc^2 \implies m = \frac{E_0}{c^2} \qquad p = \frac{mv}{\sqrt{1 - \frac{v^2}{c^2}}} = \frac{E_0}{c^2} \cdot \frac{v}{\sqrt{1 - \frac{v^2}{c^2}}} \\
    &E_\text{кин}= mc^2\cbr{\frac 1{\sqrt{1 - \frac{v^2}{c^2}}} - 1} = \frac{E_0}{c^2}\cbr{\frac 1{\sqrt{1 - \frac{v^2}{c^2}}} - 1} \\
    p, v:\quad&p = \frac{mv}{\sqrt{1 - \frac{v^2}{c^2}}} \implies m = \frac p v {\sqrt{1 - \frac{v^2}{c^2}}} \implies E_0 = mc^2 =\frac {pc^2} v {\sqrt{1 - \frac{v^2}{c^2}}} \\
    &E_\text{кин} = mc^2\cbr{\frac 1{\sqrt{1 - \frac{v^2}{c^2}}} - 1} = \frac p v {\sqrt{1 - \frac{v^2}{c^2}}}\cbr{\frac 1{\sqrt{1 - \frac{v^2}{c^2}}} - 1} = \frac p v \cbr{1 - {\sqrt{1 - \frac{v^2}{c^2}}}}
    \end{align*}
}

\variantsplitter

\addpersonalvariant{Артём Жичин}

\tasknumber{1}%
\task{%
    Запишите
    \begin{itemize}
        \item постулаты специальной теории относительности,
        \item пример релятивистского эффекта, обнаружимый при скоростях гораздо меньше скорости света.
    \end{itemize}
}
\solutionspace{120pt}

\tasknumber{2}%
\task{%
    Запишите формулу для ...
    \begin{itemize}
        \item релятивистского сжатия,
        \item классической полной механической энергии тела,
        \item релятивистского импульса тела,
        \item энергии покоя тела,
        \item связь между релятивистским импульсом и релятивистской энергией.
    \end{itemize}
    Обязательно подпишите все физические величины.
}
\solutionspace{150pt}

\tasknumber{3}%
\task{%
    Протон движется со скоростью $0{,}6\,c$, где $c$~--- скорость света в вакууме.
    Каково при этом отношение полной энергии частицы $E$ к его энергии покоя $E_0$?
}
\answer{%
    \begin{align*}
    E &= \frac{E_0}{\sqrt{1 - \frac{v^2}{c^2}}}
            \implies \frac E{E_0}
                = \frac 1{\sqrt{1 - \frac{v^2}{c^2}}}
                = \frac 1{\sqrt{1 - \sqr{0{,}6}}}
                \approx 1{,}250,
         \\
        {E_{\text{кин}}} &= E - E_0
            \implies \frac{E_{\text{кин}}}{E_0}
                = \frac E{E_0} - 1
                = \frac 1{\sqrt{1 - \frac{v^2}{c^2}}} - 1
                = \frac 1{\sqrt{1 - \sqr{0{,}6}}} - 1
                \approx 0{,}250.
    \end{align*}
}
\solutionspace{80pt}

\tasknumber{4}%
\task{%
    Полная энергия релятивистской частицы в пять раз больше её энергии покоя.
    Найти скорость этой частицы: в долях $c$ и численное значение.
    Скорость света в вакууме $c = 3 \cdot 10^{8}\,\frac{\text{м}}{\text{с}}$.
}
\answer{%
    \begin{align*}
    E &= \frac{E_0}{\sqrt{1 - \frac{v^2}{c^2}}}\implies \sqrt{1 - \frac{v^2}{c^2}} = \frac{E_0}{E}\implies \frac{v^2}{c^2} = 1 - \sqr{\frac{E_0}{E}}\implies v = c \sqrt{1 - \sqr{\frac{E_0}{E}}} \approx 0{,}980c \approx 294 \cdot 10^{6}\,\frac{\text{м}}{\text{с}}.
    \end{align*}
}
\solutionspace{80pt}

\tasknumber{5}%
\task{%
    Кинетическая энергия релятивистской частицы в пять раз больше её энергии покоя.
    Найти скорость этой частицы.
    Скорость света в вакууме $c = 3 \cdot 10^{8}\,\frac{\text{м}}{\text{с}}$.
}
\answer{%
    \begin{align*}
    E &= E_0 + E_{\text{кин}} \\
    E &= \frac{E_0}{\sqrt{1 - \frac{v^2}{c^2}}}\implies \sqrt{1 - \frac{v^2}{c^2}} = \frac{E_0}{E}\implies \frac{v^2}{c^2} = 1 - \sqr{\frac{E_0}{E}} \implies \\
    \implies &v = c \sqrt{1 - \sqr{\frac{E_0}{E}}} = c \sqrt{1 - \sqr{\frac{E_0}{E_0 + E_{\text{кин}} }}} = c \sqrt{1 - \frac 1 {\sqr{ 1 + \frac{E_{\text{кин}}}{E_0} }} }\approx 0{,}986c \approx 296 \cdot 10^{6}\,\frac{\text{м}}{\text{с}}.
    \end{align*}
}


\variantsplitter


\addpersonalvariant{Артём Жичин}

\tasknumber{6}%
\task{%
    Протон движется со скоростью $0{,}65\,c$, где $c$~--- скорость света в вакууме.
    Определите его полную энергию (в ответе приведите формулу и укажите численное значение).
}
\answer{%
    \begin{align*}
    E &= \frac{mc^2}{\sqrt{1 - \frac{v^2}{c^2}}}
            \approx \frac{1{,}673 \cdot 10^{-27}\,\text{кг} \cdot \sqr{3 \cdot 10^{8}\,\frac{\text{м}}{\text{с}}}}{\sqrt{1 - 0{,}65^2}}
            \approx 0{,}19809 \cdot 10^{-9}\,\text{Дж},
         \\
        E_{\text{кин}} &= \frac{mc^2}{\sqrt{1 - \frac{v^2}{c^2}}} - mc^2
            = mc^2 \cbr{\frac 1{\sqrt{1 - \frac{v^2}{c^2}}} - 1} \approx \\
            &\approx \cbr{1{,}673 \cdot 10^{-27}\,\text{кг} \cdot \sqr{3 \cdot 10^{8}\,\frac{\text{м}}{\text{с}}}}
            \cdot \cbr{\frac 1{\sqrt{1 - 0{,}65^2}} - 1}
            \approx 47{,}55 \cdot 10^{-12}\,\text{Дж},
         \\
        p &= \frac{mv}{\sqrt{1 - \frac{v^2}{c^2}}}
            \approx \frac{1{,}673 \cdot 10^{-27}\,\text{кг} \cdot 0{,}65 \cdot 3 \cdot 10^{8}\,\frac{\text{м}}{\text{с}}}{\sqrt{1 - 0{,}65^2}}
            \approx 0{,}4292 \cdot 10^{-18}\,\frac{\text{кг}\cdot\text{м}}{\text{с}}.
    \end{align*}
}
\solutionspace{100pt}

\tasknumber{7}%
\task{%
    Кинетическая энергия частицы космических лучей в пять раз превышает её энергию покоя.
    Определить отношение скорости частицы к скорости света.
}
\answer{%
    \begin{align*}
    E &= E_0 + E_{\text{кин}} \\
    E &= \frac{E_0}{\sqrt{1 - \frac{v^2}{c^2}}}\implies \sqrt{1 - \frac{v^2}{c^2}} = \frac{E_0}{E}\implies \frac{v^2}{c^2} = 1 - \sqr{\frac{E_0}{E}} \implies \\
    \implies \frac vc &= \sqrt{1 - \sqr{\frac{E_0}{E}}} = \sqrt{1 - \sqr{\frac{E_0}{E_0 + E_{\text{кин}} }}} \approx 0{,}986.
    \end{align*}
}
\solutionspace{80pt}

\tasknumber{8}%
\task{%
    Некоторая частица, пройдя ускоряющую разность потенциалов, приобрела импульс $4{,}2 \cdot 10^{-19}\,\frac{\text{кг}\cdot\text{м}}{\text{с}}$.
    Скорость частицы стала равной $2{,}4 \cdot 10^{8}\,\frac{\text{м}}{\text{с}}$.
    Найти массу частицы.
}
\answer{%
    $p = \frac{ mv }{\sqrt{1 - \frac{v^2}{c^2} }}\implies m = \frac pv \sqrt{1 - \frac{v^2}{c^2}}= \frac {4{,}2 \cdot 10^{-19}\,\frac{\text{кг}\cdot\text{м}}{\text{с}}}{2{,}4 \cdot 10^{8}\,\frac{\text{м}}{\text{с}}} \sqrt{1 - \sqr{\frac{2{,}4 \cdot 10^{8}\,\frac{\text{м}}{\text{с}}}{3 \cdot 10^{8}\,\frac{\text{м}}{\text{с}}}} } \approx 1{,}050 \cdot 10^{-27}\,\text{кг}.$
}
\solutionspace{80pt}

\tasknumber{9}%
\task{%
    При какой скорости движения (в км/ч) релятивистское сокращение длины движущегося тела
    составит 10\%?
}
\answer{%
    \begin{align*}
    l_0 &= \frac l{\sqrt{1 - \frac{v^2}{c^2}}}
        \implies 1 - \frac{v^2}{c^2} = \sqr{\frac l{l_0}}
        \implies \frac v c = \sqrt{1 - \sqr{\frac l{l_0}}} \implies
         \\
        \implies v &= c\sqrt{1 - \sqr{\frac l{l_0}}}
        = 3 \cdot 10^{8}\,\frac{\text{м}}{\text{с}} \cdot \sqrt{1 - \sqr{\frac {l_0 - 0{,}10l_0}{l_0}}}
        = 3 \cdot 10^{8}\,\frac{\text{м}}{\text{с}} \cdot \sqrt{1 - \sqr{1 - 0{,}10}} \approx  \\
        &\approx 0{,}436c
        \approx 130{,}8 \cdot 10^{6}\,\frac{\text{м}}{\text{с}}
        \approx 471 \cdot 10^{6}\,\frac{\text{км}}{\text{ч}}.
    \end{align*}
}
\solutionspace{80pt}

\tasknumber{10}%
\task{%
    Стержень движется в продольном направлении с постоянной скоростью относительно инерциальной системы отсчёта.
    При каком значении скорости (в долях скорости света) длина стержня в этой системе отсчёта
    будет в  1{,}5  раза меньше его собственной длины?
}
\answer{%
    $l_0 = \frac l{\sqrt{1 - \frac{v^2}{c^2}}}\implies \sqrt{1 - \frac{v^2}{c^2}} = \frac{ l }{ l_0 }\implies \frac v c = \sqrt{1 - \sqr{\frac{ l }{ l_0 }}} \approx 0{,}745.$
}
\solutionspace{80pt}

\tasknumber{11}%
\task{%
    Какую скорость должно иметь движущееся тело, чтобы его продольные размеры уменьшились в пять раз?
    Скорость света $c = 3 \cdot 10^{8}\,\frac{\text{м}}{\text{с}}$.
}
\answer{%
    $l_0 = \frac l{\sqrt{1 - \frac{v^2}{c^2}}}\implies \sqrt{1 - \frac{v^2}{c^2}} = \frac{ l }{ l_0 }\implies v = c\sqrt{1 - \sqr{\frac{ l }{ l_0 }}} \approx 294 \cdot 10^{6}\,\frac{\text{м}}{\text{с}}.$
}


\variantsplitter


\addpersonalvariant{Артём Жичин}

\tasknumber{12}%
\task{%
    Время жизни мюона, измеренное наблюдателем, относительно которого мюон покоился, равно $\tau_0$
    Какое расстояние пролетит мюон в системе отсчёта, относительно которой он движется со скоростью $v$,
    сравнимой со скоростью света в вакууме $c$?
}
\answer{%
    $\ell = v\tau = v \frac{\tau_0}{\sqrt{1 - \frac{v^2}{c^2}}}$
}
\solutionspace{80pt}

\tasknumber{13}%
\task{%
    Если $c$ — скорость света в вакууме, то с какой скоростью должна двигаться нестабильная частица относительно наблюдателя,
    чтобы её время жизни было в девять раз больше, чем у такой же, но покоящейся относительно наблюдателя частицы?
}
\answer{%
    $\tau = \frac{\tau_0}{\sqrt{1 - \frac{v^2}{c^2}}}\implies \sqrt{1 - \frac{v^2}{c^2}} = \frac{\tau_0}{\tau}\implies v = c\sqrt{1 - \sqr{\frac{\tau_0}{\tau}} } \approx 298 \cdot 10^{6}\,\frac{\text{м}}{\text{с}}.$
}
\solutionspace{80pt}

\tasknumber{14}%
\task{%
    Время жизни нестабильной частицы, входящего в состав космических лучей, измеренное земным наблюдателем,
    относительно которого частица двигалась со скоростью, составляющей 75\% скорости света в вакууме, оказалось равным $6{,}4\,\text{мкс}$.
    Каково время жизни частицы, покоящейся относительно наблюдателя?
}
\answer{%
    $t = \frac{t_0}{\sqrt{1 - \frac{v^2}{c^2}}}\implies t_0 = t\sqrt{1 - \frac{v^2}{c^2}} \approx 4{,}2 \cdot 10^{-6}\,\text{с}.$
}
\solutionspace{80pt}

\tasknumber{15}%
\task{%
    Частица увеличила в ускорителе свою скорость с $0{,}04c$ до $0{,}90c$.
    Во сколько раз выросла её кинетическая энергия?
}
\answer{%
    \begin{align*}
    E_{\text{кин.}} &= E - E_0 = \frac{mc^2}{\sqrt{1 - \frac{v^2}{c^2}}} - mc^2 = mc^2\cbr{ \frac1{\sqrt{1 - \frac{v^2}{c^2} }} - 1}.
    \\
    \frac{E_{\text{кин.
    2}}}{E_{\text{кин.
    1}}} &= \frac{\frac1{\sqrt{1 - \frac{v_2^2}{c^2} }} - 1}{\frac1{\sqrt{1 - \frac{v_1^2}{c^2} }} - 1}\approx 1615{,}76
    \end{align*}
}
\solutionspace{120pt}

\tasknumber{16}%
\task{%
    Для частицы, движущейся с релятивистской скоростью,
    выразите $E_\text{кин}$ и $p$ через $c$, $v$ и $E_0$,
    где $E_\text{кин}$~--- кинетическая энергия частицы,
    а $E_0$, $p$ и $v$~--- её энергия покоя, импульс и скорость.
}
\answer{%
    \begin{align*}
    E_\text{кин}, E_0:\quad&E = E_\text{кин} + E_0 = \frac{E_0}{\sqrt{1 - \frac{v^2}{c^2}}} \implies \sqrt{1 - \frac{v^2}{c^2}} = \frac{E_0}{{E_0} + {E_\text{кин}}} \implies v = c\sqrt{1 - \sqr{\frac{E_0}{{E_0} + {E_\text{кин}}}}} \\
    &p = \frac{mv}{\sqrt{1 - \frac{v^2}{c^2}}} = \frac{E_0}{c^2} \cdot \sqrt{1 - \sqr{\frac{E_0}{{E_0} + {E_\text{кин}}}}} \cdot \frac{{E_\text{кин}} + {E_0}}{E_0} = \frac{E_0}{c^2} \cdot \sqrt{\sqr{\frac{{E_\text{кин}} + {E_0}}{E_0}} - 1}.
    \\
    E_\text{кин}, p:\quad&E_\text{кин} = E - E_0 = mc^2\cbr{\frac 1{\sqrt{1 - \frac{v^2}{c^2}}} - 1}, p = \frac{mv}{\sqrt{1 - \frac{v^2}{c^2}}} \implies \frac{E_\text{кин}}{p} = \frac{\frac 1{\sqrt{1 - \frac{v^2}{c^2}}} - 1}{\sqrt{1 - \frac{v^2}{c^2}}} \implies v = \ldots \\
    &E_0 = E - E_\text{кин} = \frac{E_0}{\sqrt{1 - \frac{v^2}{c^2}}} - E_\text{кин} \implies E_0 = \frac{E_\text{кин}}{\frac 1{\sqrt{1 - \frac{v^2}{c^2}}} - 1} = \ldots \\
    E_\text{кин}, v:\quad&E_\text{кин} = E - E_0 = mc^2\cbr{\frac 1{\sqrt{1 - \frac{v^2}{c^2}}} - 1} \implies m = \frac{E_\text{кин}}{c^2\cbr{\frac 1{\sqrt{1 - \frac{v^2}{c^2}}} - 1}} \\
    &E_0 = mc^2 = \frac{E_\text{кин}}{\frac 1{\sqrt{1 - \frac{v^2}{c^2}}} - 1} \\
    &p = \frac{mv}{\sqrt{1 - \frac{v^2}{c^2}}} = \frac{E_\text{кин}}{c^2\cbr{\frac 1{\sqrt{1 - \frac{v^2}{c^2}}} - 1}} \cdot \frac{v}{\sqrt{1 - \frac{v^2}{c^2}}} = \frac{{E_\text{кин}} v}{c^2\cbr{1 - {\sqrt{1 - \frac{v^2}{c^2}}}}} \\
    E_0, p:\quad&E_0 = mc^2, \quad p = \frac{mv}{\sqrt{1 - \frac{v^2}{c^2}}} \implies \frac{E_0}{p} = \frac{c^2}v{\sqrt{1 - \frac{v^2}{c^2}}} = c\sqrt{\frac{c^2}{v^2} - 1} \\
    &\sqr{\frac{E_0}{pc}} = \frac{c^2}{v^2} - 1 \implies \frac{v^2}{c^2} = \frac 1{1 + \frac{E_0^2}{p^2c^2}} \implies v = \frac c{\sqrt{1 + \frac{E_0^2}{p^2c^2}}} \\
    &{E_\text{кин}} = E - E_0 = \sqrt{E_0^2 + p^2c^2} - E_0 \\
    E_0, v:\quad&E_0 = mc^2 \implies m = \frac{E_0}{c^2} \qquad p = \frac{mv}{\sqrt{1 - \frac{v^2}{c^2}}} = \frac{E_0}{c^2} \cdot \frac{v}{\sqrt{1 - \frac{v^2}{c^2}}} \\
    &E_\text{кин}= mc^2\cbr{\frac 1{\sqrt{1 - \frac{v^2}{c^2}}} - 1} = \frac{E_0}{c^2}\cbr{\frac 1{\sqrt{1 - \frac{v^2}{c^2}}} - 1} \\
    p, v:\quad&p = \frac{mv}{\sqrt{1 - \frac{v^2}{c^2}}} \implies m = \frac p v {\sqrt{1 - \frac{v^2}{c^2}}} \implies E_0 = mc^2 =\frac {pc^2} v {\sqrt{1 - \frac{v^2}{c^2}}} \\
    &E_\text{кин} = mc^2\cbr{\frac 1{\sqrt{1 - \frac{v^2}{c^2}}} - 1} = \frac p v {\sqrt{1 - \frac{v^2}{c^2}}}\cbr{\frac 1{\sqrt{1 - \frac{v^2}{c^2}}} - 1} = \frac p v \cbr{1 - {\sqrt{1 - \frac{v^2}{c^2}}}}
    \end{align*}
}

\variantsplitter

\addpersonalvariant{Дарья Кошман}

\tasknumber{1}%
\task{%
    Запишите
    \begin{itemize}
        \item постулаты специальной теории относительности,
        \item пример релятивистского эффекта, обнаружимый при скоростях гораздо меньше скорости света.
    \end{itemize}
}
\solutionspace{120pt}

\tasknumber{2}%
\task{%
    Запишите формулу для ...
    \begin{itemize}
        \item релятивистского замедления времени,
        \item классической полной механической энергии тела,
        \item релятивистского импульса тела,
        \item релятивистской кинетической энергии,
        \item связь между релятивистским импульсом и релятивистской энергией.
    \end{itemize}
    Обязательно подпишите все физические величины.
}
\solutionspace{150pt}

\tasknumber{3}%
\task{%
    Позитрон движется со скоростью $0{,}8\,c$, где $c$~--- скорость света в вакууме.
    Каково при этом отношение полной энергии частицы $E$ к его энергии покоя $E_0$?
}
\answer{%
    \begin{align*}
    E &= \frac{E_0}{\sqrt{1 - \frac{v^2}{c^2}}}
            \implies \frac E{E_0}
                = \frac 1{\sqrt{1 - \frac{v^2}{c^2}}}
                = \frac 1{\sqrt{1 - \sqr{0{,}8}}}
                \approx 1{,}667,
         \\
        {E_{\text{кин}}} &= E - E_0
            \implies \frac{E_{\text{кин}}}{E_0}
                = \frac E{E_0} - 1
                = \frac 1{\sqrt{1 - \frac{v^2}{c^2}}} - 1
                = \frac 1{\sqrt{1 - \sqr{0{,}8}}} - 1
                \approx 0{,}667.
    \end{align*}
}
\solutionspace{80pt}

\tasknumber{4}%
\task{%
    Полная энергия релятивистской частицы в шесть раз больше её энергии покоя.
    Найти скорость этой частицы: в долях $c$ и численное значение.
    Скорость света в вакууме $c = 3 \cdot 10^{8}\,\frac{\text{м}}{\text{с}}$.
}
\answer{%
    \begin{align*}
    E &= \frac{E_0}{\sqrt{1 - \frac{v^2}{c^2}}}\implies \sqrt{1 - \frac{v^2}{c^2}} = \frac{E_0}{E}\implies \frac{v^2}{c^2} = 1 - \sqr{\frac{E_0}{E}}\implies v = c \sqrt{1 - \sqr{\frac{E_0}{E}}} \approx 0{,}986c \approx 296 \cdot 10^{6}\,\frac{\text{м}}{\text{с}}.
    \end{align*}
}
\solutionspace{80pt}

\tasknumber{5}%
\task{%
    Кинетическая энергия релятивистской частицы в шесть раз больше её энергии покоя.
    Найти скорость этой частицы.
    Скорость света в вакууме $c = 3 \cdot 10^{8}\,\frac{\text{м}}{\text{с}}$.
}
\answer{%
    \begin{align*}
    E &= E_0 + E_{\text{кин}} \\
    E &= \frac{E_0}{\sqrt{1 - \frac{v^2}{c^2}}}\implies \sqrt{1 - \frac{v^2}{c^2}} = \frac{E_0}{E}\implies \frac{v^2}{c^2} = 1 - \sqr{\frac{E_0}{E}} \implies \\
    \implies &v = c \sqrt{1 - \sqr{\frac{E_0}{E}}} = c \sqrt{1 - \sqr{\frac{E_0}{E_0 + E_{\text{кин}} }}} = c \sqrt{1 - \frac 1 {\sqr{ 1 + \frac{E_{\text{кин}}}{E_0} }} }\approx 0{,}990c \approx 297 \cdot 10^{6}\,\frac{\text{м}}{\text{с}}.
    \end{align*}
}


\variantsplitter


\addpersonalvariant{Дарья Кошман}

\tasknumber{6}%
\task{%
    Протон движется со скоростью $0{,}65\,c$, где $c$~--- скорость света в вакууме.
    Определите его импульс (в ответе приведите формулу и укажите численное значение).
}
\answer{%
    \begin{align*}
    E &= \frac{mc^2}{\sqrt{1 - \frac{v^2}{c^2}}}
            \approx \frac{1{,}673 \cdot 10^{-27}\,\text{кг} \cdot \sqr{3 \cdot 10^{8}\,\frac{\text{м}}{\text{с}}}}{\sqrt{1 - 0{,}65^2}}
            \approx 0{,}19809 \cdot 10^{-9}\,\text{Дж},
         \\
        E_{\text{кин}} &= \frac{mc^2}{\sqrt{1 - \frac{v^2}{c^2}}} - mc^2
            = mc^2 \cbr{\frac 1{\sqrt{1 - \frac{v^2}{c^2}}} - 1} \approx \\
            &\approx \cbr{1{,}673 \cdot 10^{-27}\,\text{кг} \cdot \sqr{3 \cdot 10^{8}\,\frac{\text{м}}{\text{с}}}}
            \cdot \cbr{\frac 1{\sqrt{1 - 0{,}65^2}} - 1}
            \approx 47{,}55 \cdot 10^{-12}\,\text{Дж},
         \\
        p &= \frac{mv}{\sqrt{1 - \frac{v^2}{c^2}}}
            \approx \frac{1{,}673 \cdot 10^{-27}\,\text{кг} \cdot 0{,}65 \cdot 3 \cdot 10^{8}\,\frac{\text{м}}{\text{с}}}{\sqrt{1 - 0{,}65^2}}
            \approx 0{,}4292 \cdot 10^{-18}\,\frac{\text{кг}\cdot\text{м}}{\text{с}}.
    \end{align*}
}
\solutionspace{100pt}

\tasknumber{7}%
\task{%
    Кинетическая энергия частицы космических лучей в шесть раз превышает её энергию покоя.
    Определить отношение скорости частицы к скорости света.
}
\answer{%
    \begin{align*}
    E &= E_0 + E_{\text{кин}} \\
    E &= \frac{E_0}{\sqrt{1 - \frac{v^2}{c^2}}}\implies \sqrt{1 - \frac{v^2}{c^2}} = \frac{E_0}{E}\implies \frac{v^2}{c^2} = 1 - \sqr{\frac{E_0}{E}} \implies \\
    \implies \frac vc &= \sqrt{1 - \sqr{\frac{E_0}{E}}} = \sqrt{1 - \sqr{\frac{E_0}{E_0 + E_{\text{кин}} }}} \approx 0{,}990.
    \end{align*}
}
\solutionspace{80pt}

\tasknumber{8}%
\task{%
    Некоторая частица, пройдя ускоряющую разность потенциалов, приобрела импульс $3 \cdot 10^{-19}\,\frac{\text{кг}\cdot\text{м}}{\text{с}}$.
    Скорость частицы стала равной $1{,}8 \cdot 10^{8}\,\frac{\text{м}}{\text{с}}$.
    Найти массу частицы.
}
\answer{%
    $p = \frac{ mv }{\sqrt{1 - \frac{v^2}{c^2} }}\implies m = \frac pv \sqrt{1 - \frac{v^2}{c^2}}= \frac {3 \cdot 10^{-19}\,\frac{\text{кг}\cdot\text{м}}{\text{с}}}{1{,}8 \cdot 10^{8}\,\frac{\text{м}}{\text{с}}} \sqrt{1 - \sqr{\frac{1{,}8 \cdot 10^{8}\,\frac{\text{м}}{\text{с}}}{3 \cdot 10^{8}\,\frac{\text{м}}{\text{с}}}} } \approx 1{,}33 \cdot 10^{-27}\,\text{кг}.$
}
\solutionspace{80pt}

\tasknumber{9}%
\task{%
    При какой скорости движения (в долях скорости света) релятивистское сокращение длины движущегося тела
    составит 10\%?
}
\answer{%
    \begin{align*}
    l_0 &= \frac l{\sqrt{1 - \frac{v^2}{c^2}}}
        \implies 1 - \frac{v^2}{c^2} = \sqr{\frac l{l_0}}
        \implies \frac v c = \sqrt{1 - \sqr{\frac l{l_0}}} \implies
         \\
        \implies v &= c\sqrt{1 - \sqr{\frac l{l_0}}}
        = 3 \cdot 10^{8}\,\frac{\text{м}}{\text{с}} \cdot \sqrt{1 - \sqr{\frac {l_0 - 0{,}10l_0}{l_0}}}
        = 3 \cdot 10^{8}\,\frac{\text{м}}{\text{с}} \cdot \sqrt{1 - \sqr{1 - 0{,}10}} \approx  \\
        &\approx 0{,}436c
        \approx 130{,}8 \cdot 10^{6}\,\frac{\text{м}}{\text{с}}
        \approx 471 \cdot 10^{6}\,\frac{\text{км}}{\text{ч}}.
    \end{align*}
}
\solutionspace{80pt}

\tasknumber{10}%
\task{%
    Стержень движется в продольном направлении с постоянной скоростью относительно инерциальной системы отсчёта.
    При каком значении скорости (в долях скорости света) длина стержня в этой системе отсчёта
    будет в  1{,}67  раза меньше его собственной длины?
}
\answer{%
    $l_0 = \frac l{\sqrt{1 - \frac{v^2}{c^2}}}\implies \sqrt{1 - \frac{v^2}{c^2}} = \frac{ l }{ l_0 }\implies \frac v c = \sqrt{1 - \sqr{\frac{ l }{ l_0 }}} \approx 0{,}801.$
}
\solutionspace{80pt}

\tasknumber{11}%
\task{%
    Какую скорость должно иметь движущееся тело, чтобы его продольные размеры уменьшились в три раза?
    Скорость света $c = 3 \cdot 10^{8}\,\frac{\text{м}}{\text{с}}$.
}
\answer{%
    $l_0 = \frac l{\sqrt{1 - \frac{v^2}{c^2}}}\implies \sqrt{1 - \frac{v^2}{c^2}} = \frac{ l }{ l_0 }\implies v = c\sqrt{1 - \sqr{\frac{ l }{ l_0 }}} \approx 283 \cdot 10^{6}\,\frac{\text{м}}{\text{с}}.$
}


\variantsplitter


\addpersonalvariant{Дарья Кошман}

\tasknumber{12}%
\task{%
    Время жизни мюона, измеренное наблюдателем, относительно которого мюон покоился, равно $\tau_0$
    Какое расстояние пролетит мюон в системе отсчёта, относительно которой он движется со скоростью $v$,
    сравнимой со скоростью света в вакууме $c$?
}
\answer{%
    $\ell = v\tau = v \frac{\tau_0}{\sqrt{1 - \frac{v^2}{c^2}}}$
}
\solutionspace{80pt}

\tasknumber{13}%
\task{%
    Если $c$ — скорость света в вакууме, то с какой скоростью должна двигаться нестабильная частица относительно наблюдателя,
    чтобы её время жизни было в шесть раз больше, чем у такой же, но покоящейся относительно наблюдателя частицы?
}
\answer{%
    $\tau = \frac{\tau_0}{\sqrt{1 - \frac{v^2}{c^2}}}\implies \sqrt{1 - \frac{v^2}{c^2}} = \frac{\tau_0}{\tau}\implies v = c\sqrt{1 - \sqr{\frac{\tau_0}{\tau}} } \approx 296 \cdot 10^{6}\,\frac{\text{м}}{\text{с}}.$
}
\solutionspace{80pt}

\tasknumber{14}%
\task{%
    Время жизни нестабильной частицы, входящего в состав космических лучей, измеренное земным наблюдателем,
    относительно которого частица двигалась со скоростью, составляющей 75\% скорости света в вакууме, оказалось равным $5{,}3\,\text{мкс}$.
    Каково время жизни частицы, покоящейся относительно наблюдателя?
}
\answer{%
    $t = \frac{t_0}{\sqrt{1 - \frac{v^2}{c^2}}}\implies t_0 = t\sqrt{1 - \frac{v^2}{c^2}} \approx 3{,}5 \cdot 10^{-6}\,\text{с}.$
}
\solutionspace{80pt}

\tasknumber{15}%
\task{%
    Частица увеличила в ускорителе свою скорость с $0{,}02c$ до $0{,}50c$.
    Во сколько раз выросла её кинетическая энергия?
}
\answer{%
    \begin{align*}
    E_{\text{кин.}} &= E - E_0 = \frac{mc^2}{\sqrt{1 - \frac{v^2}{c^2}}} - mc^2 = mc^2\cbr{ \frac1{\sqrt{1 - \frac{v^2}{c^2} }} - 1}.
    \\
    \frac{E_{\text{кин.
    2}}}{E_{\text{кин.
    1}}} &= \frac{\frac1{\sqrt{1 - \frac{v_2^2}{c^2} }} - 1}{\frac1{\sqrt{1 - \frac{v_1^2}{c^2} }} - 1}\approx 773{,}27
    \end{align*}
}
\solutionspace{120pt}

\tasknumber{16}%
\task{%
    Для частицы, движущейся с релятивистской скоростью,
    выразите $E_0$ и $v$ через $c$, $E_\text{кин}$ и $p$,
    где $E_\text{кин}$~--- кинетическая энергия частицы,
    а $E_0$, $p$ и $v$~--- её энергия покоя, импульс и скорость.
}
\answer{%
    \begin{align*}
    E_\text{кин}, E_0:\quad&E = E_\text{кин} + E_0 = \frac{E_0}{\sqrt{1 - \frac{v^2}{c^2}}} \implies \sqrt{1 - \frac{v^2}{c^2}} = \frac{E_0}{{E_0} + {E_\text{кин}}} \implies v = c\sqrt{1 - \sqr{\frac{E_0}{{E_0} + {E_\text{кин}}}}} \\
    &p = \frac{mv}{\sqrt{1 - \frac{v^2}{c^2}}} = \frac{E_0}{c^2} \cdot \sqrt{1 - \sqr{\frac{E_0}{{E_0} + {E_\text{кин}}}}} \cdot \frac{{E_\text{кин}} + {E_0}}{E_0} = \frac{E_0}{c^2} \cdot \sqrt{\sqr{\frac{{E_\text{кин}} + {E_0}}{E_0}} - 1}.
    \\
    E_\text{кин}, p:\quad&E_\text{кин} = E - E_0 = mc^2\cbr{\frac 1{\sqrt{1 - \frac{v^2}{c^2}}} - 1}, p = \frac{mv}{\sqrt{1 - \frac{v^2}{c^2}}} \implies \frac{E_\text{кин}}{p} = \frac{\frac 1{\sqrt{1 - \frac{v^2}{c^2}}} - 1}{\sqrt{1 - \frac{v^2}{c^2}}} \implies v = \ldots \\
    &E_0 = E - E_\text{кин} = \frac{E_0}{\sqrt{1 - \frac{v^2}{c^2}}} - E_\text{кин} \implies E_0 = \frac{E_\text{кин}}{\frac 1{\sqrt{1 - \frac{v^2}{c^2}}} - 1} = \ldots \\
    E_\text{кин}, v:\quad&E_\text{кин} = E - E_0 = mc^2\cbr{\frac 1{\sqrt{1 - \frac{v^2}{c^2}}} - 1} \implies m = \frac{E_\text{кин}}{c^2\cbr{\frac 1{\sqrt{1 - \frac{v^2}{c^2}}} - 1}} \\
    &E_0 = mc^2 = \frac{E_\text{кин}}{\frac 1{\sqrt{1 - \frac{v^2}{c^2}}} - 1} \\
    &p = \frac{mv}{\sqrt{1 - \frac{v^2}{c^2}}} = \frac{E_\text{кин}}{c^2\cbr{\frac 1{\sqrt{1 - \frac{v^2}{c^2}}} - 1}} \cdot \frac{v}{\sqrt{1 - \frac{v^2}{c^2}}} = \frac{{E_\text{кин}} v}{c^2\cbr{1 - {\sqrt{1 - \frac{v^2}{c^2}}}}} \\
    E_0, p:\quad&E_0 = mc^2, \quad p = \frac{mv}{\sqrt{1 - \frac{v^2}{c^2}}} \implies \frac{E_0}{p} = \frac{c^2}v{\sqrt{1 - \frac{v^2}{c^2}}} = c\sqrt{\frac{c^2}{v^2} - 1} \\
    &\sqr{\frac{E_0}{pc}} = \frac{c^2}{v^2} - 1 \implies \frac{v^2}{c^2} = \frac 1{1 + \frac{E_0^2}{p^2c^2}} \implies v = \frac c{\sqrt{1 + \frac{E_0^2}{p^2c^2}}} \\
    &{E_\text{кин}} = E - E_0 = \sqrt{E_0^2 + p^2c^2} - E_0 \\
    E_0, v:\quad&E_0 = mc^2 \implies m = \frac{E_0}{c^2} \qquad p = \frac{mv}{\sqrt{1 - \frac{v^2}{c^2}}} = \frac{E_0}{c^2} \cdot \frac{v}{\sqrt{1 - \frac{v^2}{c^2}}} \\
    &E_\text{кин}= mc^2\cbr{\frac 1{\sqrt{1 - \frac{v^2}{c^2}}} - 1} = \frac{E_0}{c^2}\cbr{\frac 1{\sqrt{1 - \frac{v^2}{c^2}}} - 1} \\
    p, v:\quad&p = \frac{mv}{\sqrt{1 - \frac{v^2}{c^2}}} \implies m = \frac p v {\sqrt{1 - \frac{v^2}{c^2}}} \implies E_0 = mc^2 =\frac {pc^2} v {\sqrt{1 - \frac{v^2}{c^2}}} \\
    &E_\text{кин} = mc^2\cbr{\frac 1{\sqrt{1 - \frac{v^2}{c^2}}} - 1} = \frac p v {\sqrt{1 - \frac{v^2}{c^2}}}\cbr{\frac 1{\sqrt{1 - \frac{v^2}{c^2}}} - 1} = \frac p v \cbr{1 - {\sqrt{1 - \frac{v^2}{c^2}}}}
    \end{align*}
}

\variantsplitter

\addpersonalvariant{Анна Кузьмичёва}

\tasknumber{1}%
\task{%
    Запишите
    \begin{itemize}
        \item постулаты специальной теории относительности,
        \item пример релятивистского эффекта, обнаружимый при скоростях гораздо меньше скорости света.
    \end{itemize}
}
\solutionspace{120pt}

\tasknumber{2}%
\task{%
    Запишите формулу для ...
    \begin{itemize}
        \item релятивистского замедления времени,
        \item классической полной механической энергии тела,
        \item релятивистской энергии тела,
        \item энергии покоя тела,
        \item связь между релятивистским импульсом и релятивистской энергией.
    \end{itemize}
    Обязательно подпишите все физические величины.
}
\solutionspace{150pt}

\tasknumber{3}%
\task{%
    Позитрон движется со скоростью $0{,}9\,c$, где $c$~--- скорость света в вакууме.
    Каково при этом отношение кинетической энергии частицы $E_\text{кин.}$ к его энергии покоя $E_0$?
}
\answer{%
    \begin{align*}
    E &= \frac{E_0}{\sqrt{1 - \frac{v^2}{c^2}}}
            \implies \frac E{E_0}
                = \frac 1{\sqrt{1 - \frac{v^2}{c^2}}}
                = \frac 1{\sqrt{1 - \sqr{0{,}9}}}
                \approx 2{,}294,
         \\
        {E_{\text{кин}}} &= E - E_0
            \implies \frac{E_{\text{кин}}}{E_0}
                = \frac E{E_0} - 1
                = \frac 1{\sqrt{1 - \frac{v^2}{c^2}}} - 1
                = \frac 1{\sqrt{1 - \sqr{0{,}9}}} - 1
                \approx 1{,}294.
    \end{align*}
}
\solutionspace{80pt}

\tasknumber{4}%
\task{%
    Полная энергия релятивистской частицы в четыре раза больше её энергии покоя.
    Найти скорость этой частицы: в долях $c$ и численное значение.
    Скорость света в вакууме $c = 3 \cdot 10^{8}\,\frac{\text{м}}{\text{с}}$.
}
\answer{%
    \begin{align*}
    E &= \frac{E_0}{\sqrt{1 - \frac{v^2}{c^2}}}\implies \sqrt{1 - \frac{v^2}{c^2}} = \frac{E_0}{E}\implies \frac{v^2}{c^2} = 1 - \sqr{\frac{E_0}{E}}\implies v = c \sqrt{1 - \sqr{\frac{E_0}{E}}} \approx 0{,}968c \approx 290 \cdot 10^{6}\,\frac{\text{м}}{\text{с}}.
    \end{align*}
}
\solutionspace{80pt}

\tasknumber{5}%
\task{%
    Кинетическая энергия релятивистской частицы в четыре раза больше её энергии покоя.
    Найти скорость этой частицы.
    Скорость света в вакууме $c = 3 \cdot 10^{8}\,\frac{\text{м}}{\text{с}}$.
}
\answer{%
    \begin{align*}
    E &= E_0 + E_{\text{кин}} \\
    E &= \frac{E_0}{\sqrt{1 - \frac{v^2}{c^2}}}\implies \sqrt{1 - \frac{v^2}{c^2}} = \frac{E_0}{E}\implies \frac{v^2}{c^2} = 1 - \sqr{\frac{E_0}{E}} \implies \\
    \implies &v = c \sqrt{1 - \sqr{\frac{E_0}{E}}} = c \sqrt{1 - \sqr{\frac{E_0}{E_0 + E_{\text{кин}} }}} = c \sqrt{1 - \frac 1 {\sqr{ 1 + \frac{E_{\text{кин}}}{E_0} }} }\approx 0{,}980c \approx 294 \cdot 10^{6}\,\frac{\text{м}}{\text{с}}.
    \end{align*}
}


\variantsplitter


\addpersonalvariant{Анна Кузьмичёва}

\tasknumber{6}%
\task{%
    Электрон движется со скоростью $0{,}75\,c$, где $c$~--- скорость света в вакууме.
    Определите его полную энергию (в ответе приведите формулу и укажите численное значение).
}
\answer{%
    \begin{align*}
    E &= \frac{mc^2}{\sqrt{1 - \frac{v^2}{c^2}}}
            \approx \frac{9{,}1 \cdot 10^{-31}\,\text{кг} \cdot \sqr{3 \cdot 10^{8}\,\frac{\text{м}}{\text{с}}}}{\sqrt{1 - 0{,}75^2}}
            \approx 0{,}1238 \cdot 10^{-12}\,\text{Дж},
         \\
        E_{\text{кин}} &= \frac{mc^2}{\sqrt{1 - \frac{v^2}{c^2}}} - mc^2
            = mc^2 \cbr{\frac 1{\sqrt{1 - \frac{v^2}{c^2}}} - 1} \approx \\
            &\approx \cbr{9{,}1 \cdot 10^{-31}\,\text{кг} \cdot \sqr{3 \cdot 10^{8}\,\frac{\text{м}}{\text{с}}}}
            \cdot \cbr{\frac 1{\sqrt{1 - 0{,}75^2}} - 1}
            \approx 41{,}9 \cdot 10^{-15}\,\text{Дж},
         \\
        p &= \frac{mv}{\sqrt{1 - \frac{v^2}{c^2}}}
            \approx \frac{9{,}1 \cdot 10^{-31}\,\text{кг} \cdot 0{,}75 \cdot 3 \cdot 10^{8}\,\frac{\text{м}}{\text{с}}}{\sqrt{1 - 0{,}75^2}}
            \approx 0{,}310 \cdot 10^{-21}\,\frac{\text{кг}\cdot\text{м}}{\text{с}}.
    \end{align*}
}
\solutionspace{100pt}

\tasknumber{7}%
\task{%
    Кинетическая энергия частицы космических лучей в четыре раза превышает её энергию покоя.
    Определить отношение скорости частицы к скорости света.
}
\answer{%
    \begin{align*}
    E &= E_0 + E_{\text{кин}} \\
    E &= \frac{E_0}{\sqrt{1 - \frac{v^2}{c^2}}}\implies \sqrt{1 - \frac{v^2}{c^2}} = \frac{E_0}{E}\implies \frac{v^2}{c^2} = 1 - \sqr{\frac{E_0}{E}} \implies \\
    \implies \frac vc &= \sqrt{1 - \sqr{\frac{E_0}{E}}} = \sqrt{1 - \sqr{\frac{E_0}{E_0 + E_{\text{кин}} }}} \approx 0{,}980.
    \end{align*}
}
\solutionspace{80pt}

\tasknumber{8}%
\task{%
    Некоторая частица, пройдя ускоряющую разность потенциалов, приобрела импульс $3{,}8 \cdot 10^{-19}\,\frac{\text{кг}\cdot\text{м}}{\text{с}}$.
    Скорость частицы стала равной $1{,}8 \cdot 10^{8}\,\frac{\text{м}}{\text{с}}$.
    Найти массу частицы.
}
\answer{%
    $p = \frac{ mv }{\sqrt{1 - \frac{v^2}{c^2} }}\implies m = \frac pv \sqrt{1 - \frac{v^2}{c^2}}= \frac {3{,}8 \cdot 10^{-19}\,\frac{\text{кг}\cdot\text{м}}{\text{с}}}{1{,}8 \cdot 10^{8}\,\frac{\text{м}}{\text{с}}} \sqrt{1 - \sqr{\frac{1{,}8 \cdot 10^{8}\,\frac{\text{м}}{\text{с}}}{3 \cdot 10^{8}\,\frac{\text{м}}{\text{с}}}} } \approx 1{,}69 \cdot 10^{-27}\,\text{кг}.$
}
\solutionspace{80pt}

\tasknumber{9}%
\task{%
    При какой скорости движения (в долях скорости света) релятивистское сокращение длины движущегося тела
    составит 10\%?
}
\answer{%
    \begin{align*}
    l_0 &= \frac l{\sqrt{1 - \frac{v^2}{c^2}}}
        \implies 1 - \frac{v^2}{c^2} = \sqr{\frac l{l_0}}
        \implies \frac v c = \sqrt{1 - \sqr{\frac l{l_0}}} \implies
         \\
        \implies v &= c\sqrt{1 - \sqr{\frac l{l_0}}}
        = 3 \cdot 10^{8}\,\frac{\text{м}}{\text{с}} \cdot \sqrt{1 - \sqr{\frac {l_0 - 0{,}10l_0}{l_0}}}
        = 3 \cdot 10^{8}\,\frac{\text{м}}{\text{с}} \cdot \sqrt{1 - \sqr{1 - 0{,}10}} \approx  \\
        &\approx 0{,}436c
        \approx 130{,}8 \cdot 10^{6}\,\frac{\text{м}}{\text{с}}
        \approx 471 \cdot 10^{6}\,\frac{\text{км}}{\text{ч}}.
    \end{align*}
}
\solutionspace{80pt}

\tasknumber{10}%
\task{%
    Стержень движется в продольном направлении с постоянной скоростью относительно инерциальной системы отсчёта.
    При каком значении скорости (в долях скорости света) длина стержня в этой системе отсчёта
    будет в  3  раза меньше его собственной длины?
}
\answer{%
    $l_0 = \frac l{\sqrt{1 - \frac{v^2}{c^2}}}\implies \sqrt{1 - \frac{v^2}{c^2}} = \frac{ l }{ l_0 }\implies \frac v c = \sqrt{1 - \sqr{\frac{ l }{ l_0 }}} \approx 0{,}943.$
}
\solutionspace{80pt}

\tasknumber{11}%
\task{%
    Какую скорость должно иметь движущееся тело, чтобы его продольные размеры уменьшились в четыре раза?
    Скорость света $c = 3 \cdot 10^{8}\,\frac{\text{м}}{\text{с}}$.
}
\answer{%
    $l_0 = \frac l{\sqrt{1 - \frac{v^2}{c^2}}}\implies \sqrt{1 - \frac{v^2}{c^2}} = \frac{ l }{ l_0 }\implies v = c\sqrt{1 - \sqr{\frac{ l }{ l_0 }}} \approx 290 \cdot 10^{6}\,\frac{\text{м}}{\text{с}}.$
}


\variantsplitter


\addpersonalvariant{Анна Кузьмичёва}

\tasknumber{12}%
\task{%
    Время жизни мюона, измеренное наблюдателем, относительно которого мюон покоился, равно $\tau_0$
    Какое расстояние пролетит мюон в системе отсчёта, относительно которой он движется со скоростью $v$,
    сравнимой со скоростью света в вакууме $c$?
}
\answer{%
    $\ell = v\tau = v \frac{\tau_0}{\sqrt{1 - \frac{v^2}{c^2}}}$
}
\solutionspace{80pt}

\tasknumber{13}%
\task{%
    Если $c$ — скорость света в вакууме, то с какой скоростью должна двигаться нестабильная частица относительно наблюдателя,
    чтобы её время жизни было в четыре раза больше, чем у такой же, но покоящейся относительно наблюдателя частицы?
}
\answer{%
    $\tau = \frac{\tau_0}{\sqrt{1 - \frac{v^2}{c^2}}}\implies \sqrt{1 - \frac{v^2}{c^2}} = \frac{\tau_0}{\tau}\implies v = c\sqrt{1 - \sqr{\frac{\tau_0}{\tau}} } \approx 290 \cdot 10^{6}\,\frac{\text{м}}{\text{с}}.$
}
\solutionspace{80pt}

\tasknumber{14}%
\task{%
    Время жизни нестабильной частицы, входящего в состав космических лучей, измеренное земным наблюдателем,
    относительно которого частица двигалась со скоростью, составляющей 65\% скорости света в вакууме, оказалось равным $7{,}1\,\text{мкс}$.
    Каково время жизни частицы, покоящейся относительно наблюдателя?
}
\answer{%
    $t = \frac{t_0}{\sqrt{1 - \frac{v^2}{c^2}}}\implies t_0 = t\sqrt{1 - \frac{v^2}{c^2}} \approx 5{,}4 \cdot 10^{-6}\,\text{с}.$
}
\solutionspace{80pt}

\tasknumber{15}%
\task{%
    Частица увеличила в ускорителе свою скорость с $0{,}02c$ до $0{,}80c$.
    Во сколько раз выросла её кинетическая энергия?
}
\answer{%
    \begin{align*}
    E_{\text{кин.}} &= E - E_0 = \frac{mc^2}{\sqrt{1 - \frac{v^2}{c^2}}} - mc^2 = mc^2\cbr{ \frac1{\sqrt{1 - \frac{v^2}{c^2} }} - 1}.
    \\
    \frac{E_{\text{кин.
    2}}}{E_{\text{кин.
    1}}} &= \frac{\frac1{\sqrt{1 - \frac{v_2^2}{c^2} }} - 1}{\frac1{\sqrt{1 - \frac{v_1^2}{c^2} }} - 1}\approx 3332{,}33
    \end{align*}
}
\solutionspace{120pt}

\tasknumber{16}%
\task{%
    Для частицы, движущейся с релятивистской скоростью,
    выразите $p$ и $E_0$ через $c$, $v$ и $E_\text{кин}$,
    где $E_\text{кин}$~--- кинетическая энергия частицы,
    а $E_0$, $p$ и $v$~--- её энергия покоя, импульс и скорость.
}
\answer{%
    \begin{align*}
    E_\text{кин}, E_0:\quad&E = E_\text{кин} + E_0 = \frac{E_0}{\sqrt{1 - \frac{v^2}{c^2}}} \implies \sqrt{1 - \frac{v^2}{c^2}} = \frac{E_0}{{E_0} + {E_\text{кин}}} \implies v = c\sqrt{1 - \sqr{\frac{E_0}{{E_0} + {E_\text{кин}}}}} \\
    &p = \frac{mv}{\sqrt{1 - \frac{v^2}{c^2}}} = \frac{E_0}{c^2} \cdot \sqrt{1 - \sqr{\frac{E_0}{{E_0} + {E_\text{кин}}}}} \cdot \frac{{E_\text{кин}} + {E_0}}{E_0} = \frac{E_0}{c^2} \cdot \sqrt{\sqr{\frac{{E_\text{кин}} + {E_0}}{E_0}} - 1}.
    \\
    E_\text{кин}, p:\quad&E_\text{кин} = E - E_0 = mc^2\cbr{\frac 1{\sqrt{1 - \frac{v^2}{c^2}}} - 1}, p = \frac{mv}{\sqrt{1 - \frac{v^2}{c^2}}} \implies \frac{E_\text{кин}}{p} = \frac{\frac 1{\sqrt{1 - \frac{v^2}{c^2}}} - 1}{\sqrt{1 - \frac{v^2}{c^2}}} \implies v = \ldots \\
    &E_0 = E - E_\text{кин} = \frac{E_0}{\sqrt{1 - \frac{v^2}{c^2}}} - E_\text{кин} \implies E_0 = \frac{E_\text{кин}}{\frac 1{\sqrt{1 - \frac{v^2}{c^2}}} - 1} = \ldots \\
    E_\text{кин}, v:\quad&E_\text{кин} = E - E_0 = mc^2\cbr{\frac 1{\sqrt{1 - \frac{v^2}{c^2}}} - 1} \implies m = \frac{E_\text{кин}}{c^2\cbr{\frac 1{\sqrt{1 - \frac{v^2}{c^2}}} - 1}} \\
    &E_0 = mc^2 = \frac{E_\text{кин}}{\frac 1{\sqrt{1 - \frac{v^2}{c^2}}} - 1} \\
    &p = \frac{mv}{\sqrt{1 - \frac{v^2}{c^2}}} = \frac{E_\text{кин}}{c^2\cbr{\frac 1{\sqrt{1 - \frac{v^2}{c^2}}} - 1}} \cdot \frac{v}{\sqrt{1 - \frac{v^2}{c^2}}} = \frac{{E_\text{кин}} v}{c^2\cbr{1 - {\sqrt{1 - \frac{v^2}{c^2}}}}} \\
    E_0, p:\quad&E_0 = mc^2, \quad p = \frac{mv}{\sqrt{1 - \frac{v^2}{c^2}}} \implies \frac{E_0}{p} = \frac{c^2}v{\sqrt{1 - \frac{v^2}{c^2}}} = c\sqrt{\frac{c^2}{v^2} - 1} \\
    &\sqr{\frac{E_0}{pc}} = \frac{c^2}{v^2} - 1 \implies \frac{v^2}{c^2} = \frac 1{1 + \frac{E_0^2}{p^2c^2}} \implies v = \frac c{\sqrt{1 + \frac{E_0^2}{p^2c^2}}} \\
    &{E_\text{кин}} = E - E_0 = \sqrt{E_0^2 + p^2c^2} - E_0 \\
    E_0, v:\quad&E_0 = mc^2 \implies m = \frac{E_0}{c^2} \qquad p = \frac{mv}{\sqrt{1 - \frac{v^2}{c^2}}} = \frac{E_0}{c^2} \cdot \frac{v}{\sqrt{1 - \frac{v^2}{c^2}}} \\
    &E_\text{кин}= mc^2\cbr{\frac 1{\sqrt{1 - \frac{v^2}{c^2}}} - 1} = \frac{E_0}{c^2}\cbr{\frac 1{\sqrt{1 - \frac{v^2}{c^2}}} - 1} \\
    p, v:\quad&p = \frac{mv}{\sqrt{1 - \frac{v^2}{c^2}}} \implies m = \frac p v {\sqrt{1 - \frac{v^2}{c^2}}} \implies E_0 = mc^2 =\frac {pc^2} v {\sqrt{1 - \frac{v^2}{c^2}}} \\
    &E_\text{кин} = mc^2\cbr{\frac 1{\sqrt{1 - \frac{v^2}{c^2}}} - 1} = \frac p v {\sqrt{1 - \frac{v^2}{c^2}}}\cbr{\frac 1{\sqrt{1 - \frac{v^2}{c^2}}} - 1} = \frac p v \cbr{1 - {\sqrt{1 - \frac{v^2}{c^2}}}}
    \end{align*}
}

\variantsplitter

\addpersonalvariant{Алёна Куприянова}

\tasknumber{1}%
\task{%
    Запишите
    \begin{itemize}
        \item постулаты специальной теории относительности,
        \item пример релятивистского эффекта, обнаружимый при скоростях гораздо меньше скорости света.
    \end{itemize}
}
\solutionspace{120pt}

\tasknumber{2}%
\task{%
    Запишите формулу для ...
    \begin{itemize}
        \item релятивистского сжатия,
        \item классического импульса,
        \item релятивистского импульса тела,
        \item энергии покоя тела,
        \item связь между релятивистским импульсом и релятивистской энергией.
    \end{itemize}
    Обязательно подпишите все физические величины.
}
\solutionspace{150pt}

\tasknumber{3}%
\task{%
    Протон движется со скоростью $0{,}6\,c$, где $c$~--- скорость света в вакууме.
    Каково при этом отношение кинетической энергии частицы $E_\text{кин.}$ к его энергии покоя $E_0$?
}
\answer{%
    \begin{align*}
    E &= \frac{E_0}{\sqrt{1 - \frac{v^2}{c^2}}}
            \implies \frac E{E_0}
                = \frac 1{\sqrt{1 - \frac{v^2}{c^2}}}
                = \frac 1{\sqrt{1 - \sqr{0{,}6}}}
                \approx 1{,}250,
         \\
        {E_{\text{кин}}} &= E - E_0
            \implies \frac{E_{\text{кин}}}{E_0}
                = \frac E{E_0} - 1
                = \frac 1{\sqrt{1 - \frac{v^2}{c^2}}} - 1
                = \frac 1{\sqrt{1 - \sqr{0{,}6}}} - 1
                \approx 0{,}250.
    \end{align*}
}
\solutionspace{80pt}

\tasknumber{4}%
\task{%
    Полная энергия релятивистской частицы в четыре раза больше её энергии покоя.
    Найти скорость этой частицы: в долях $c$ и численное значение.
    Скорость света в вакууме $c = 3 \cdot 10^{8}\,\frac{\text{м}}{\text{с}}$.
}
\answer{%
    \begin{align*}
    E &= \frac{E_0}{\sqrt{1 - \frac{v^2}{c^2}}}\implies \sqrt{1 - \frac{v^2}{c^2}} = \frac{E_0}{E}\implies \frac{v^2}{c^2} = 1 - \sqr{\frac{E_0}{E}}\implies v = c \sqrt{1 - \sqr{\frac{E_0}{E}}} \approx 0{,}968c \approx 290 \cdot 10^{6}\,\frac{\text{м}}{\text{с}}.
    \end{align*}
}
\solutionspace{80pt}

\tasknumber{5}%
\task{%
    Кинетическая энергия релятивистской частицы в четыре раза больше её энергии покоя.
    Найти скорость этой частицы.
    Скорость света в вакууме $c = 3 \cdot 10^{8}\,\frac{\text{м}}{\text{с}}$.
}
\answer{%
    \begin{align*}
    E &= E_0 + E_{\text{кин}} \\
    E &= \frac{E_0}{\sqrt{1 - \frac{v^2}{c^2}}}\implies \sqrt{1 - \frac{v^2}{c^2}} = \frac{E_0}{E}\implies \frac{v^2}{c^2} = 1 - \sqr{\frac{E_0}{E}} \implies \\
    \implies &v = c \sqrt{1 - \sqr{\frac{E_0}{E}}} = c \sqrt{1 - \sqr{\frac{E_0}{E_0 + E_{\text{кин}} }}} = c \sqrt{1 - \frac 1 {\sqr{ 1 + \frac{E_{\text{кин}}}{E_0} }} }\approx 0{,}980c \approx 294 \cdot 10^{6}\,\frac{\text{м}}{\text{с}}.
    \end{align*}
}


\variantsplitter


\addpersonalvariant{Алёна Куприянова}

\tasknumber{6}%
\task{%
    Электрон движется со скоростью $0{,}65\,c$, где $c$~--- скорость света в вакууме.
    Определите его импульс (в ответе приведите формулу и укажите численное значение).
}
\answer{%
    \begin{align*}
    E &= \frac{mc^2}{\sqrt{1 - \frac{v^2}{c^2}}}
            \approx \frac{9{,}1 \cdot 10^{-31}\,\text{кг} \cdot \sqr{3 \cdot 10^{8}\,\frac{\text{м}}{\text{с}}}}{\sqrt{1 - 0{,}65^2}}
            \approx 0{,}1078 \cdot 10^{-12}\,\text{Дж},
         \\
        E_{\text{кин}} &= \frac{mc^2}{\sqrt{1 - \frac{v^2}{c^2}}} - mc^2
            = mc^2 \cbr{\frac 1{\sqrt{1 - \frac{v^2}{c^2}}} - 1} \approx \\
            &\approx \cbr{9{,}1 \cdot 10^{-31}\,\text{кг} \cdot \sqr{3 \cdot 10^{8}\,\frac{\text{м}}{\text{с}}}}
            \cdot \cbr{\frac 1{\sqrt{1 - 0{,}65^2}} - 1}
            \approx 25{,}9 \cdot 10^{-15}\,\text{Дж},
         \\
        p &= \frac{mv}{\sqrt{1 - \frac{v^2}{c^2}}}
            \approx \frac{9{,}1 \cdot 10^{-31}\,\text{кг} \cdot 0{,}65 \cdot 3 \cdot 10^{8}\,\frac{\text{м}}{\text{с}}}{\sqrt{1 - 0{,}65^2}}
            \approx 0{,}234 \cdot 10^{-21}\,\frac{\text{кг}\cdot\text{м}}{\text{с}}.
    \end{align*}
}
\solutionspace{100pt}

\tasknumber{7}%
\task{%
    Кинетическая энергия частицы космических лучей в четыре раза превышает её энергию покоя.
    Определить отношение скорости частицы к скорости света.
}
\answer{%
    \begin{align*}
    E &= E_0 + E_{\text{кин}} \\
    E &= \frac{E_0}{\sqrt{1 - \frac{v^2}{c^2}}}\implies \sqrt{1 - \frac{v^2}{c^2}} = \frac{E_0}{E}\implies \frac{v^2}{c^2} = 1 - \sqr{\frac{E_0}{E}} \implies \\
    \implies \frac vc &= \sqrt{1 - \sqr{\frac{E_0}{E}}} = \sqrt{1 - \sqr{\frac{E_0}{E_0 + E_{\text{кин}} }}} \approx 0{,}980.
    \end{align*}
}
\solutionspace{80pt}

\tasknumber{8}%
\task{%
    Некоторая частица, пройдя ускоряющую разность потенциалов, приобрела импульс $3{,}8 \cdot 10^{-19}\,\frac{\text{кг}\cdot\text{м}}{\text{с}}$.
    Скорость частицы стала равной $1{,}8 \cdot 10^{8}\,\frac{\text{м}}{\text{с}}$.
    Найти массу частицы.
}
\answer{%
    $p = \frac{ mv }{\sqrt{1 - \frac{v^2}{c^2} }}\implies m = \frac pv \sqrt{1 - \frac{v^2}{c^2}}= \frac {3{,}8 \cdot 10^{-19}\,\frac{\text{кг}\cdot\text{м}}{\text{с}}}{1{,}8 \cdot 10^{8}\,\frac{\text{м}}{\text{с}}} \sqrt{1 - \sqr{\frac{1{,}8 \cdot 10^{8}\,\frac{\text{м}}{\text{с}}}{3 \cdot 10^{8}\,\frac{\text{м}}{\text{с}}}} } \approx 1{,}69 \cdot 10^{-27}\,\text{кг}.$
}
\solutionspace{80pt}

\tasknumber{9}%
\task{%
    При какой скорости движения (в м/с) релятивистское сокращение длины движущегося тела
    составит 30\%?
}
\answer{%
    \begin{align*}
    l_0 &= \frac l{\sqrt{1 - \frac{v^2}{c^2}}}
        \implies 1 - \frac{v^2}{c^2} = \sqr{\frac l{l_0}}
        \implies \frac v c = \sqrt{1 - \sqr{\frac l{l_0}}} \implies
         \\
        \implies v &= c\sqrt{1 - \sqr{\frac l{l_0}}}
        = 3 \cdot 10^{8}\,\frac{\text{м}}{\text{с}} \cdot \sqrt{1 - \sqr{\frac {l_0 - 0{,}30l_0}{l_0}}}
        = 3 \cdot 10^{8}\,\frac{\text{м}}{\text{с}} \cdot \sqrt{1 - \sqr{1 - 0{,}30}} \approx  \\
        &\approx 0{,}714c
        \approx 214 \cdot 10^{6}\,\frac{\text{м}}{\text{с}}
        \approx 771 \cdot 10^{6}\,\frac{\text{км}}{\text{ч}}.
    \end{align*}
}
\solutionspace{80pt}

\tasknumber{10}%
\task{%
    Стержень движется в продольном направлении с постоянной скоростью относительно инерциальной системы отсчёта.
    При каком значении скорости (в долях скорости света) длина стержня в этой системе отсчёта
    будет в  3  раза меньше его собственной длины?
}
\answer{%
    $l_0 = \frac l{\sqrt{1 - \frac{v^2}{c^2}}}\implies \sqrt{1 - \frac{v^2}{c^2}} = \frac{ l }{ l_0 }\implies \frac v c = \sqrt{1 - \sqr{\frac{ l }{ l_0 }}} \approx 0{,}943.$
}
\solutionspace{80pt}

\tasknumber{11}%
\task{%
    Какую скорость должно иметь движущееся тело, чтобы его продольные размеры уменьшились в три раза?
    Скорость света $c = 3 \cdot 10^{8}\,\frac{\text{м}}{\text{с}}$.
}
\answer{%
    $l_0 = \frac l{\sqrt{1 - \frac{v^2}{c^2}}}\implies \sqrt{1 - \frac{v^2}{c^2}} = \frac{ l }{ l_0 }\implies v = c\sqrt{1 - \sqr{\frac{ l }{ l_0 }}} \approx 283 \cdot 10^{6}\,\frac{\text{м}}{\text{с}}.$
}


\variantsplitter


\addpersonalvariant{Алёна Куприянова}

\tasknumber{12}%
\task{%
    Время жизни мюона, измеренное наблюдателем, относительно которого мюон покоился, равно $\tau_0$
    Какое расстояние пролетит мюон в системе отсчёта, относительно которой он движется со скоростью $v$,
    сравнимой со скоростью света в вакууме $c$?
}
\answer{%
    $\ell = v\tau = v \frac{\tau_0}{\sqrt{1 - \frac{v^2}{c^2}}}$
}
\solutionspace{80pt}

\tasknumber{13}%
\task{%
    Если $c$ — скорость света в вакууме, то с какой скоростью должна двигаться нестабильная частица относительно наблюдателя,
    чтобы её время жизни было в четыре раза больше, чем у такой же, но покоящейся относительно наблюдателя частицы?
}
\answer{%
    $\tau = \frac{\tau_0}{\sqrt{1 - \frac{v^2}{c^2}}}\implies \sqrt{1 - \frac{v^2}{c^2}} = \frac{\tau_0}{\tau}\implies v = c\sqrt{1 - \sqr{\frac{\tau_0}{\tau}} } \approx 290 \cdot 10^{6}\,\frac{\text{м}}{\text{с}}.$
}
\solutionspace{80pt}

\tasknumber{14}%
\task{%
    Время жизни нестабильной частицы, входящего в состав космических лучей, измеренное земным наблюдателем,
    относительно которого частица двигалась со скоростью, составляющей 75\% скорости света в вакууме, оказалось равным $3{,}7\,\text{мкс}$.
    Каково время жизни частицы, покоящейся относительно наблюдателя?
}
\answer{%
    $t = \frac{t_0}{\sqrt{1 - \frac{v^2}{c^2}}}\implies t_0 = t\sqrt{1 - \frac{v^2}{c^2}} \approx 2{,}4 \cdot 10^{-6}\,\text{с}.$
}
\solutionspace{80pt}

\tasknumber{15}%
\task{%
    Частица увеличила в ускорителе свою скорость с $0{,}01c$ до $0{,}50c$.
    Во сколько раз выросла её кинетическая энергия?
}
\answer{%
    \begin{align*}
    E_{\text{кин.}} &= E - E_0 = \frac{mc^2}{\sqrt{1 - \frac{v^2}{c^2}}} - mc^2 = mc^2\cbr{ \frac1{\sqrt{1 - \frac{v^2}{c^2} }} - 1}.
    \\
    \frac{E_{\text{кин.
    2}}}{E_{\text{кин.
    1}}} &= \frac{\frac1{\sqrt{1 - \frac{v_2^2}{c^2} }} - 1}{\frac1{\sqrt{1 - \frac{v_1^2}{c^2} }} - 1}\approx 3093{,}78
    \end{align*}
}
\solutionspace{120pt}

\tasknumber{16}%
\task{%
    Для частицы, движущейся с релятивистской скоростью,
    выразите $p$ и $E_\text{кин}$ через $c$, $E_0$ и $v$,
    где $E_\text{кин}$~--- кинетическая энергия частицы,
    а $E_0$, $p$ и $v$~--- её энергия покоя, импульс и скорость.
}
\answer{%
    \begin{align*}
    E_\text{кин}, E_0:\quad&E = E_\text{кин} + E_0 = \frac{E_0}{\sqrt{1 - \frac{v^2}{c^2}}} \implies \sqrt{1 - \frac{v^2}{c^2}} = \frac{E_0}{{E_0} + {E_\text{кин}}} \implies v = c\sqrt{1 - \sqr{\frac{E_0}{{E_0} + {E_\text{кин}}}}} \\
    &p = \frac{mv}{\sqrt{1 - \frac{v^2}{c^2}}} = \frac{E_0}{c^2} \cdot \sqrt{1 - \sqr{\frac{E_0}{{E_0} + {E_\text{кин}}}}} \cdot \frac{{E_\text{кин}} + {E_0}}{E_0} = \frac{E_0}{c^2} \cdot \sqrt{\sqr{\frac{{E_\text{кин}} + {E_0}}{E_0}} - 1}.
    \\
    E_\text{кин}, p:\quad&E_\text{кин} = E - E_0 = mc^2\cbr{\frac 1{\sqrt{1 - \frac{v^2}{c^2}}} - 1}, p = \frac{mv}{\sqrt{1 - \frac{v^2}{c^2}}} \implies \frac{E_\text{кин}}{p} = \frac{\frac 1{\sqrt{1 - \frac{v^2}{c^2}}} - 1}{\sqrt{1 - \frac{v^2}{c^2}}} \implies v = \ldots \\
    &E_0 = E - E_\text{кин} = \frac{E_0}{\sqrt{1 - \frac{v^2}{c^2}}} - E_\text{кин} \implies E_0 = \frac{E_\text{кин}}{\frac 1{\sqrt{1 - \frac{v^2}{c^2}}} - 1} = \ldots \\
    E_\text{кин}, v:\quad&E_\text{кин} = E - E_0 = mc^2\cbr{\frac 1{\sqrt{1 - \frac{v^2}{c^2}}} - 1} \implies m = \frac{E_\text{кин}}{c^2\cbr{\frac 1{\sqrt{1 - \frac{v^2}{c^2}}} - 1}} \\
    &E_0 = mc^2 = \frac{E_\text{кин}}{\frac 1{\sqrt{1 - \frac{v^2}{c^2}}} - 1} \\
    &p = \frac{mv}{\sqrt{1 - \frac{v^2}{c^2}}} = \frac{E_\text{кин}}{c^2\cbr{\frac 1{\sqrt{1 - \frac{v^2}{c^2}}} - 1}} \cdot \frac{v}{\sqrt{1 - \frac{v^2}{c^2}}} = \frac{{E_\text{кин}} v}{c^2\cbr{1 - {\sqrt{1 - \frac{v^2}{c^2}}}}} \\
    E_0, p:\quad&E_0 = mc^2, \quad p = \frac{mv}{\sqrt{1 - \frac{v^2}{c^2}}} \implies \frac{E_0}{p} = \frac{c^2}v{\sqrt{1 - \frac{v^2}{c^2}}} = c\sqrt{\frac{c^2}{v^2} - 1} \\
    &\sqr{\frac{E_0}{pc}} = \frac{c^2}{v^2} - 1 \implies \frac{v^2}{c^2} = \frac 1{1 + \frac{E_0^2}{p^2c^2}} \implies v = \frac c{\sqrt{1 + \frac{E_0^2}{p^2c^2}}} \\
    &{E_\text{кин}} = E - E_0 = \sqrt{E_0^2 + p^2c^2} - E_0 \\
    E_0, v:\quad&E_0 = mc^2 \implies m = \frac{E_0}{c^2} \qquad p = \frac{mv}{\sqrt{1 - \frac{v^2}{c^2}}} = \frac{E_0}{c^2} \cdot \frac{v}{\sqrt{1 - \frac{v^2}{c^2}}} \\
    &E_\text{кин}= mc^2\cbr{\frac 1{\sqrt{1 - \frac{v^2}{c^2}}} - 1} = \frac{E_0}{c^2}\cbr{\frac 1{\sqrt{1 - \frac{v^2}{c^2}}} - 1} \\
    p, v:\quad&p = \frac{mv}{\sqrt{1 - \frac{v^2}{c^2}}} \implies m = \frac p v {\sqrt{1 - \frac{v^2}{c^2}}} \implies E_0 = mc^2 =\frac {pc^2} v {\sqrt{1 - \frac{v^2}{c^2}}} \\
    &E_\text{кин} = mc^2\cbr{\frac 1{\sqrt{1 - \frac{v^2}{c^2}}} - 1} = \frac p v {\sqrt{1 - \frac{v^2}{c^2}}}\cbr{\frac 1{\sqrt{1 - \frac{v^2}{c^2}}} - 1} = \frac p v \cbr{1 - {\sqrt{1 - \frac{v^2}{c^2}}}}
    \end{align*}
}

\variantsplitter

\addpersonalvariant{Ярослав Лавровский}

\tasknumber{1}%
\task{%
    Запишите
    \begin{itemize}
        \item постулаты специальной теории относительности,
        \item пример релятивистского эффекта, обнаружимый при скоростях гораздо меньше скорости света.
    \end{itemize}
}
\solutionspace{120pt}

\tasknumber{2}%
\task{%
    Запишите формулу для ...
    \begin{itemize}
        \item релятивистского замедления времени,
        \item классической полной механической энергии тела,
        \item релятивистской энергии тела,
        \item релятивистской кинетической энергии,
        \item связь между релятивистским импульсом и релятивистской энергией.
    \end{itemize}
    Обязательно подпишите все физические величины.
}
\solutionspace{150pt}

\tasknumber{3}%
\task{%
    Электрон движется со скоростью $0{,}7\,c$, где $c$~--- скорость света в вакууме.
    Каково при этом отношение полной энергии частицы $E$ к его энергии покоя $E_0$?
}
\answer{%
    \begin{align*}
    E &= \frac{E_0}{\sqrt{1 - \frac{v^2}{c^2}}}
            \implies \frac E{E_0}
                = \frac 1{\sqrt{1 - \frac{v^2}{c^2}}}
                = \frac 1{\sqrt{1 - \sqr{0{,}7}}}
                \approx 1{,}400,
         \\
        {E_{\text{кин}}} &= E - E_0
            \implies \frac{E_{\text{кин}}}{E_0}
                = \frac E{E_0} - 1
                = \frac 1{\sqrt{1 - \frac{v^2}{c^2}}} - 1
                = \frac 1{\sqrt{1 - \sqr{0{,}7}}} - 1
                \approx 0{,}400.
    \end{align*}
}
\solutionspace{80pt}

\tasknumber{4}%
\task{%
    Полная энергия релятивистской частицы в четыре раза больше её энергии покоя.
    Найти скорость этой частицы: в долях $c$ и численное значение.
    Скорость света в вакууме $c = 3 \cdot 10^{8}\,\frac{\text{м}}{\text{с}}$.
}
\answer{%
    \begin{align*}
    E &= \frac{E_0}{\sqrt{1 - \frac{v^2}{c^2}}}\implies \sqrt{1 - \frac{v^2}{c^2}} = \frac{E_0}{E}\implies \frac{v^2}{c^2} = 1 - \sqr{\frac{E_0}{E}}\implies v = c \sqrt{1 - \sqr{\frac{E_0}{E}}} \approx 0{,}968c \approx 290 \cdot 10^{6}\,\frac{\text{м}}{\text{с}}.
    \end{align*}
}
\solutionspace{80pt}

\tasknumber{5}%
\task{%
    Кинетическая энергия релятивистской частицы в четыре раза больше её энергии покоя.
    Найти скорость этой частицы.
    Скорость света в вакууме $c = 3 \cdot 10^{8}\,\frac{\text{м}}{\text{с}}$.
}
\answer{%
    \begin{align*}
    E &= E_0 + E_{\text{кин}} \\
    E &= \frac{E_0}{\sqrt{1 - \frac{v^2}{c^2}}}\implies \sqrt{1 - \frac{v^2}{c^2}} = \frac{E_0}{E}\implies \frac{v^2}{c^2} = 1 - \sqr{\frac{E_0}{E}} \implies \\
    \implies &v = c \sqrt{1 - \sqr{\frac{E_0}{E}}} = c \sqrt{1 - \sqr{\frac{E_0}{E_0 + E_{\text{кин}} }}} = c \sqrt{1 - \frac 1 {\sqr{ 1 + \frac{E_{\text{кин}}}{E_0} }} }\approx 0{,}980c \approx 294 \cdot 10^{6}\,\frac{\text{м}}{\text{с}}.
    \end{align*}
}


\variantsplitter


\addpersonalvariant{Ярослав Лавровский}

\tasknumber{6}%
\task{%
    Электрон движется со скоростью $0{,}75\,c$, где $c$~--- скорость света в вакууме.
    Определите его кинетическую энергию (в ответе приведите формулу и укажите численное значение).
}
\answer{%
    \begin{align*}
    E &= \frac{mc^2}{\sqrt{1 - \frac{v^2}{c^2}}}
            \approx \frac{9{,}1 \cdot 10^{-31}\,\text{кг} \cdot \sqr{3 \cdot 10^{8}\,\frac{\text{м}}{\text{с}}}}{\sqrt{1 - 0{,}75^2}}
            \approx 0{,}1238 \cdot 10^{-12}\,\text{Дж},
         \\
        E_{\text{кин}} &= \frac{mc^2}{\sqrt{1 - \frac{v^2}{c^2}}} - mc^2
            = mc^2 \cbr{\frac 1{\sqrt{1 - \frac{v^2}{c^2}}} - 1} \approx \\
            &\approx \cbr{9{,}1 \cdot 10^{-31}\,\text{кг} \cdot \sqr{3 \cdot 10^{8}\,\frac{\text{м}}{\text{с}}}}
            \cdot \cbr{\frac 1{\sqrt{1 - 0{,}75^2}} - 1}
            \approx 41{,}9 \cdot 10^{-15}\,\text{Дж},
         \\
        p &= \frac{mv}{\sqrt{1 - \frac{v^2}{c^2}}}
            \approx \frac{9{,}1 \cdot 10^{-31}\,\text{кг} \cdot 0{,}75 \cdot 3 \cdot 10^{8}\,\frac{\text{м}}{\text{с}}}{\sqrt{1 - 0{,}75^2}}
            \approx 0{,}310 \cdot 10^{-21}\,\frac{\text{кг}\cdot\text{м}}{\text{с}}.
    \end{align*}
}
\solutionspace{100pt}

\tasknumber{7}%
\task{%
    Кинетическая энергия частицы космических лучей в четыре раза превышает её энергию покоя.
    Определить отношение скорости частицы к скорости света.
}
\answer{%
    \begin{align*}
    E &= E_0 + E_{\text{кин}} \\
    E &= \frac{E_0}{\sqrt{1 - \frac{v^2}{c^2}}}\implies \sqrt{1 - \frac{v^2}{c^2}} = \frac{E_0}{E}\implies \frac{v^2}{c^2} = 1 - \sqr{\frac{E_0}{E}} \implies \\
    \implies \frac vc &= \sqrt{1 - \sqr{\frac{E_0}{E}}} = \sqrt{1 - \sqr{\frac{E_0}{E_0 + E_{\text{кин}} }}} \approx 0{,}980.
    \end{align*}
}
\solutionspace{80pt}

\tasknumber{8}%
\task{%
    Некоторая частица, пройдя ускоряющую разность потенциалов, приобрела импульс $3{,}8 \cdot 10^{-19}\,\frac{\text{кг}\cdot\text{м}}{\text{с}}$.
    Скорость частицы стала равной $2 \cdot 10^{8}\,\frac{\text{м}}{\text{с}}$.
    Найти массу частицы.
}
\answer{%
    $p = \frac{ mv }{\sqrt{1 - \frac{v^2}{c^2} }}\implies m = \frac pv \sqrt{1 - \frac{v^2}{c^2}}= \frac {3{,}8 \cdot 10^{-19}\,\frac{\text{кг}\cdot\text{м}}{\text{с}}}{2 \cdot 10^{8}\,\frac{\text{м}}{\text{с}}} \sqrt{1 - \sqr{\frac{2 \cdot 10^{8}\,\frac{\text{м}}{\text{с}}}{3 \cdot 10^{8}\,\frac{\text{м}}{\text{с}}}} } \approx 1{,}416 \cdot 10^{-27}\,\text{кг}.$
}
\solutionspace{80pt}

\tasknumber{9}%
\task{%
    При какой скорости движения (в км/ч) релятивистское сокращение длины движущегося тела
    составит 50\%?
}
\answer{%
    \begin{align*}
    l_0 &= \frac l{\sqrt{1 - \frac{v^2}{c^2}}}
        \implies 1 - \frac{v^2}{c^2} = \sqr{\frac l{l_0}}
        \implies \frac v c = \sqrt{1 - \sqr{\frac l{l_0}}} \implies
         \\
        \implies v &= c\sqrt{1 - \sqr{\frac l{l_0}}}
        = 3 \cdot 10^{8}\,\frac{\text{м}}{\text{с}} \cdot \sqrt{1 - \sqr{\frac {l_0 - 0{,}50l_0}{l_0}}}
        = 3 \cdot 10^{8}\,\frac{\text{м}}{\text{с}} \cdot \sqrt{1 - \sqr{1 - 0{,}50}} \approx  \\
        &\approx 0{,}866c
        \approx 260 \cdot 10^{6}\,\frac{\text{м}}{\text{с}}
        \approx 935 \cdot 10^{6}\,\frac{\text{км}}{\text{ч}}.
    \end{align*}
}
\solutionspace{80pt}

\tasknumber{10}%
\task{%
    Стержень движется в продольном направлении с постоянной скоростью относительно инерциальной системы отсчёта.
    При каком значении скорости (в долях скорости света) длина стержня в этой системе отсчёта
    будет в  4  раза меньше его собственной длины?
}
\answer{%
    $l_0 = \frac l{\sqrt{1 - \frac{v^2}{c^2}}}\implies \sqrt{1 - \frac{v^2}{c^2}} = \frac{ l }{ l_0 }\implies \frac v c = \sqrt{1 - \sqr{\frac{ l }{ l_0 }}} \approx 0{,}968.$
}
\solutionspace{80pt}

\tasknumber{11}%
\task{%
    Какую скорость должно иметь движущееся тело, чтобы его продольные размеры уменьшились в четыре раза?
    Скорость света $c = 3 \cdot 10^{8}\,\frac{\text{м}}{\text{с}}$.
}
\answer{%
    $l_0 = \frac l{\sqrt{1 - \frac{v^2}{c^2}}}\implies \sqrt{1 - \frac{v^2}{c^2}} = \frac{ l }{ l_0 }\implies v = c\sqrt{1 - \sqr{\frac{ l }{ l_0 }}} \approx 290 \cdot 10^{6}\,\frac{\text{м}}{\text{с}}.$
}


\variantsplitter


\addpersonalvariant{Ярослав Лавровский}

\tasknumber{12}%
\task{%
    Время жизни мюона, измеренное наблюдателем, относительно которого мюон покоился, равно $\tau_0$
    Какое расстояние пролетит мюон в системе отсчёта, относительно которой он движется со скоростью $v$,
    сравнимой со скоростью света в вакууме $c$?
}
\answer{%
    $\ell = v\tau = v \frac{\tau_0}{\sqrt{1 - \frac{v^2}{c^2}}}$
}
\solutionspace{80pt}

\tasknumber{13}%
\task{%
    Если $c$ — скорость света в вакууме, то с какой скоростью должна двигаться нестабильная частица относительно наблюдателя,
    чтобы её время жизни было в четыре раза больше, чем у такой же, но покоящейся относительно наблюдателя частицы?
}
\answer{%
    $\tau = \frac{\tau_0}{\sqrt{1 - \frac{v^2}{c^2}}}\implies \sqrt{1 - \frac{v^2}{c^2}} = \frac{\tau_0}{\tau}\implies v = c\sqrt{1 - \sqr{\frac{\tau_0}{\tau}} } \approx 290 \cdot 10^{6}\,\frac{\text{м}}{\text{с}}.$
}
\solutionspace{80pt}

\tasknumber{14}%
\task{%
    Время жизни нестабильной частицы, входящего в состав космических лучей, измеренное земным наблюдателем,
    относительно которого частица двигалась со скоростью, составляющей 85\% скорости света в вакууме, оказалось равным $6{,}4\,\text{мкс}$.
    Каково время жизни частицы, покоящейся относительно наблюдателя?
}
\answer{%
    $t = \frac{t_0}{\sqrt{1 - \frac{v^2}{c^2}}}\implies t_0 = t\sqrt{1 - \frac{v^2}{c^2}} \approx 3{,}4 \cdot 10^{-6}\,\text{с}.$
}
\solutionspace{80pt}

\tasknumber{15}%
\task{%
    Частица увеличила в ускорителе свою скорость с $0{,}01c$ до $0{,}50c$.
    Во сколько раз выросла её кинетическая энергия?
}
\answer{%
    \begin{align*}
    E_{\text{кин.}} &= E - E_0 = \frac{mc^2}{\sqrt{1 - \frac{v^2}{c^2}}} - mc^2 = mc^2\cbr{ \frac1{\sqrt{1 - \frac{v^2}{c^2} }} - 1}.
    \\
    \frac{E_{\text{кин.
    2}}}{E_{\text{кин.
    1}}} &= \frac{\frac1{\sqrt{1 - \frac{v_2^2}{c^2} }} - 1}{\frac1{\sqrt{1 - \frac{v_1^2}{c^2} }} - 1}\approx 3093{,}78
    \end{align*}
}
\solutionspace{120pt}

\tasknumber{16}%
\task{%
    Для частицы, движущейся с релятивистской скоростью,
    выразите $v$ и $E_\text{кин}$ через $c$, $E_0$ и $p$,
    где $E_\text{кин}$~--- кинетическая энергия частицы,
    а $E_0$, $p$ и $v$~--- её энергия покоя, импульс и скорость.
}
\answer{%
    \begin{align*}
    E_\text{кин}, E_0:\quad&E = E_\text{кин} + E_0 = \frac{E_0}{\sqrt{1 - \frac{v^2}{c^2}}} \implies \sqrt{1 - \frac{v^2}{c^2}} = \frac{E_0}{{E_0} + {E_\text{кин}}} \implies v = c\sqrt{1 - \sqr{\frac{E_0}{{E_0} + {E_\text{кин}}}}} \\
    &p = \frac{mv}{\sqrt{1 - \frac{v^2}{c^2}}} = \frac{E_0}{c^2} \cdot \sqrt{1 - \sqr{\frac{E_0}{{E_0} + {E_\text{кин}}}}} \cdot \frac{{E_\text{кин}} + {E_0}}{E_0} = \frac{E_0}{c^2} \cdot \sqrt{\sqr{\frac{{E_\text{кин}} + {E_0}}{E_0}} - 1}.
    \\
    E_\text{кин}, p:\quad&E_\text{кин} = E - E_0 = mc^2\cbr{\frac 1{\sqrt{1 - \frac{v^2}{c^2}}} - 1}, p = \frac{mv}{\sqrt{1 - \frac{v^2}{c^2}}} \implies \frac{E_\text{кин}}{p} = \frac{\frac 1{\sqrt{1 - \frac{v^2}{c^2}}} - 1}{\sqrt{1 - \frac{v^2}{c^2}}} \implies v = \ldots \\
    &E_0 = E - E_\text{кин} = \frac{E_0}{\sqrt{1 - \frac{v^2}{c^2}}} - E_\text{кин} \implies E_0 = \frac{E_\text{кин}}{\frac 1{\sqrt{1 - \frac{v^2}{c^2}}} - 1} = \ldots \\
    E_\text{кин}, v:\quad&E_\text{кин} = E - E_0 = mc^2\cbr{\frac 1{\sqrt{1 - \frac{v^2}{c^2}}} - 1} \implies m = \frac{E_\text{кин}}{c^2\cbr{\frac 1{\sqrt{1 - \frac{v^2}{c^2}}} - 1}} \\
    &E_0 = mc^2 = \frac{E_\text{кин}}{\frac 1{\sqrt{1 - \frac{v^2}{c^2}}} - 1} \\
    &p = \frac{mv}{\sqrt{1 - \frac{v^2}{c^2}}} = \frac{E_\text{кин}}{c^2\cbr{\frac 1{\sqrt{1 - \frac{v^2}{c^2}}} - 1}} \cdot \frac{v}{\sqrt{1 - \frac{v^2}{c^2}}} = \frac{{E_\text{кин}} v}{c^2\cbr{1 - {\sqrt{1 - \frac{v^2}{c^2}}}}} \\
    E_0, p:\quad&E_0 = mc^2, \quad p = \frac{mv}{\sqrt{1 - \frac{v^2}{c^2}}} \implies \frac{E_0}{p} = \frac{c^2}v{\sqrt{1 - \frac{v^2}{c^2}}} = c\sqrt{\frac{c^2}{v^2} - 1} \\
    &\sqr{\frac{E_0}{pc}} = \frac{c^2}{v^2} - 1 \implies \frac{v^2}{c^2} = \frac 1{1 + \frac{E_0^2}{p^2c^2}} \implies v = \frac c{\sqrt{1 + \frac{E_0^2}{p^2c^2}}} \\
    &{E_\text{кин}} = E - E_0 = \sqrt{E_0^2 + p^2c^2} - E_0 \\
    E_0, v:\quad&E_0 = mc^2 \implies m = \frac{E_0}{c^2} \qquad p = \frac{mv}{\sqrt{1 - \frac{v^2}{c^2}}} = \frac{E_0}{c^2} \cdot \frac{v}{\sqrt{1 - \frac{v^2}{c^2}}} \\
    &E_\text{кин}= mc^2\cbr{\frac 1{\sqrt{1 - \frac{v^2}{c^2}}} - 1} = \frac{E_0}{c^2}\cbr{\frac 1{\sqrt{1 - \frac{v^2}{c^2}}} - 1} \\
    p, v:\quad&p = \frac{mv}{\sqrt{1 - \frac{v^2}{c^2}}} \implies m = \frac p v {\sqrt{1 - \frac{v^2}{c^2}}} \implies E_0 = mc^2 =\frac {pc^2} v {\sqrt{1 - \frac{v^2}{c^2}}} \\
    &E_\text{кин} = mc^2\cbr{\frac 1{\sqrt{1 - \frac{v^2}{c^2}}} - 1} = \frac p v {\sqrt{1 - \frac{v^2}{c^2}}}\cbr{\frac 1{\sqrt{1 - \frac{v^2}{c^2}}} - 1} = \frac p v \cbr{1 - {\sqrt{1 - \frac{v^2}{c^2}}}}
    \end{align*}
}

\variantsplitter

\addpersonalvariant{Анастасия Ламанова}

\tasknumber{1}%
\task{%
    Запишите
    \begin{itemize}
        \item постулаты специальной теории относительности,
        \item пример релятивистского эффекта, обнаружимый при скоростях гораздо меньше скорости света.
    \end{itemize}
}
\solutionspace{120pt}

\tasknumber{2}%
\task{%
    Запишите формулу для ...
    \begin{itemize}
        \item релятивистского сжатия,
        \item классической полной механической энергии тела,
        \item релятивистского импульса тела,
        \item энергии покоя тела,
        \item связь между релятивистским импульсом и релятивистской энергией.
    \end{itemize}
    Обязательно подпишите все физические величины.
}
\solutionspace{150pt}

\tasknumber{3}%
\task{%
    Электрон движется со скоростью $0{,}6\,c$, где $c$~--- скорость света в вакууме.
    Каково при этом отношение полной энергии частицы $E$ к его энергии покоя $E_0$?
}
\answer{%
    \begin{align*}
    E &= \frac{E_0}{\sqrt{1 - \frac{v^2}{c^2}}}
            \implies \frac E{E_0}
                = \frac 1{\sqrt{1 - \frac{v^2}{c^2}}}
                = \frac 1{\sqrt{1 - \sqr{0{,}6}}}
                \approx 1{,}250,
         \\
        {E_{\text{кин}}} &= E - E_0
            \implies \frac{E_{\text{кин}}}{E_0}
                = \frac E{E_0} - 1
                = \frac 1{\sqrt{1 - \frac{v^2}{c^2}}} - 1
                = \frac 1{\sqrt{1 - \sqr{0{,}6}}} - 1
                \approx 0{,}250.
    \end{align*}
}
\solutionspace{80pt}

\tasknumber{4}%
\task{%
    Полная энергия релятивистской частицы в пять раз больше её энергии покоя.
    Найти скорость этой частицы: в долях $c$ и численное значение.
    Скорость света в вакууме $c = 3 \cdot 10^{8}\,\frac{\text{м}}{\text{с}}$.
}
\answer{%
    \begin{align*}
    E &= \frac{E_0}{\sqrt{1 - \frac{v^2}{c^2}}}\implies \sqrt{1 - \frac{v^2}{c^2}} = \frac{E_0}{E}\implies \frac{v^2}{c^2} = 1 - \sqr{\frac{E_0}{E}}\implies v = c \sqrt{1 - \sqr{\frac{E_0}{E}}} \approx 0{,}980c \approx 294 \cdot 10^{6}\,\frac{\text{м}}{\text{с}}.
    \end{align*}
}
\solutionspace{80pt}

\tasknumber{5}%
\task{%
    Кинетическая энергия релятивистской частицы в пять раз больше её энергии покоя.
    Найти скорость этой частицы.
    Скорость света в вакууме $c = 3 \cdot 10^{8}\,\frac{\text{м}}{\text{с}}$.
}
\answer{%
    \begin{align*}
    E &= E_0 + E_{\text{кин}} \\
    E &= \frac{E_0}{\sqrt{1 - \frac{v^2}{c^2}}}\implies \sqrt{1 - \frac{v^2}{c^2}} = \frac{E_0}{E}\implies \frac{v^2}{c^2} = 1 - \sqr{\frac{E_0}{E}} \implies \\
    \implies &v = c \sqrt{1 - \sqr{\frac{E_0}{E}}} = c \sqrt{1 - \sqr{\frac{E_0}{E_0 + E_{\text{кин}} }}} = c \sqrt{1 - \frac 1 {\sqr{ 1 + \frac{E_{\text{кин}}}{E_0} }} }\approx 0{,}986c \approx 296 \cdot 10^{6}\,\frac{\text{м}}{\text{с}}.
    \end{align*}
}


\variantsplitter


\addpersonalvariant{Анастасия Ламанова}

\tasknumber{6}%
\task{%
    Протон движется со скоростью $0{,}65\,c$, где $c$~--- скорость света в вакууме.
    Определите его полную энергию (в ответе приведите формулу и укажите численное значение).
}
\answer{%
    \begin{align*}
    E &= \frac{mc^2}{\sqrt{1 - \frac{v^2}{c^2}}}
            \approx \frac{1{,}673 \cdot 10^{-27}\,\text{кг} \cdot \sqr{3 \cdot 10^{8}\,\frac{\text{м}}{\text{с}}}}{\sqrt{1 - 0{,}65^2}}
            \approx 0{,}19809 \cdot 10^{-9}\,\text{Дж},
         \\
        E_{\text{кин}} &= \frac{mc^2}{\sqrt{1 - \frac{v^2}{c^2}}} - mc^2
            = mc^2 \cbr{\frac 1{\sqrt{1 - \frac{v^2}{c^2}}} - 1} \approx \\
            &\approx \cbr{1{,}673 \cdot 10^{-27}\,\text{кг} \cdot \sqr{3 \cdot 10^{8}\,\frac{\text{м}}{\text{с}}}}
            \cdot \cbr{\frac 1{\sqrt{1 - 0{,}65^2}} - 1}
            \approx 47{,}55 \cdot 10^{-12}\,\text{Дж},
         \\
        p &= \frac{mv}{\sqrt{1 - \frac{v^2}{c^2}}}
            \approx \frac{1{,}673 \cdot 10^{-27}\,\text{кг} \cdot 0{,}65 \cdot 3 \cdot 10^{8}\,\frac{\text{м}}{\text{с}}}{\sqrt{1 - 0{,}65^2}}
            \approx 0{,}4292 \cdot 10^{-18}\,\frac{\text{кг}\cdot\text{м}}{\text{с}}.
    \end{align*}
}
\solutionspace{100pt}

\tasknumber{7}%
\task{%
    Кинетическая энергия частицы космических лучей в пять раз превышает её энергию покоя.
    Определить отношение скорости частицы к скорости света.
}
\answer{%
    \begin{align*}
    E &= E_0 + E_{\text{кин}} \\
    E &= \frac{E_0}{\sqrt{1 - \frac{v^2}{c^2}}}\implies \sqrt{1 - \frac{v^2}{c^2}} = \frac{E_0}{E}\implies \frac{v^2}{c^2} = 1 - \sqr{\frac{E_0}{E}} \implies \\
    \implies \frac vc &= \sqrt{1 - \sqr{\frac{E_0}{E}}} = \sqrt{1 - \sqr{\frac{E_0}{E_0 + E_{\text{кин}} }}} \approx 0{,}986.
    \end{align*}
}
\solutionspace{80pt}

\tasknumber{8}%
\task{%
    Некоторая частица, пройдя ускоряющую разность потенциалов, приобрела импульс $4{,}2 \cdot 10^{-19}\,\frac{\text{кг}\cdot\text{м}}{\text{с}}$.
    Скорость частицы стала равной $2{,}4 \cdot 10^{8}\,\frac{\text{м}}{\text{с}}$.
    Найти массу частицы.
}
\answer{%
    $p = \frac{ mv }{\sqrt{1 - \frac{v^2}{c^2} }}\implies m = \frac pv \sqrt{1 - \frac{v^2}{c^2}}= \frac {4{,}2 \cdot 10^{-19}\,\frac{\text{кг}\cdot\text{м}}{\text{с}}}{2{,}4 \cdot 10^{8}\,\frac{\text{м}}{\text{с}}} \sqrt{1 - \sqr{\frac{2{,}4 \cdot 10^{8}\,\frac{\text{м}}{\text{с}}}{3 \cdot 10^{8}\,\frac{\text{м}}{\text{с}}}} } \approx 1{,}050 \cdot 10^{-27}\,\text{кг}.$
}
\solutionspace{80pt}

\tasknumber{9}%
\task{%
    При какой скорости движения (в м/с) релятивистское сокращение длины движущегося тела
    составит 10\%?
}
\answer{%
    \begin{align*}
    l_0 &= \frac l{\sqrt{1 - \frac{v^2}{c^2}}}
        \implies 1 - \frac{v^2}{c^2} = \sqr{\frac l{l_0}}
        \implies \frac v c = \sqrt{1 - \sqr{\frac l{l_0}}} \implies
         \\
        \implies v &= c\sqrt{1 - \sqr{\frac l{l_0}}}
        = 3 \cdot 10^{8}\,\frac{\text{м}}{\text{с}} \cdot \sqrt{1 - \sqr{\frac {l_0 - 0{,}10l_0}{l_0}}}
        = 3 \cdot 10^{8}\,\frac{\text{м}}{\text{с}} \cdot \sqrt{1 - \sqr{1 - 0{,}10}} \approx  \\
        &\approx 0{,}436c
        \approx 130{,}8 \cdot 10^{6}\,\frac{\text{м}}{\text{с}}
        \approx 471 \cdot 10^{6}\,\frac{\text{км}}{\text{ч}}.
    \end{align*}
}
\solutionspace{80pt}

\tasknumber{10}%
\task{%
    Стержень движется в продольном направлении с постоянной скоростью относительно инерциальной системы отсчёта.
    При каком значении скорости (в долях скорости света) длина стержня в этой системе отсчёта
    будет в  1{,}25  раза меньше его собственной длины?
}
\answer{%
    $l_0 = \frac l{\sqrt{1 - \frac{v^2}{c^2}}}\implies \sqrt{1 - \frac{v^2}{c^2}} = \frac{ l }{ l_0 }\implies \frac v c = \sqrt{1 - \sqr{\frac{ l }{ l_0 }}} \approx 0{,}600.$
}
\solutionspace{80pt}

\tasknumber{11}%
\task{%
    Какую скорость должно иметь движущееся тело, чтобы его продольные размеры уменьшились в шесть раз?
    Скорость света $c = 3 \cdot 10^{8}\,\frac{\text{м}}{\text{с}}$.
}
\answer{%
    $l_0 = \frac l{\sqrt{1 - \frac{v^2}{c^2}}}\implies \sqrt{1 - \frac{v^2}{c^2}} = \frac{ l }{ l_0 }\implies v = c\sqrt{1 - \sqr{\frac{ l }{ l_0 }}} \approx 296 \cdot 10^{6}\,\frac{\text{м}}{\text{с}}.$
}


\variantsplitter


\addpersonalvariant{Анастасия Ламанова}

\tasknumber{12}%
\task{%
    Время жизни мюона, измеренное наблюдателем, относительно которого мюон покоился, равно $\tau_0$
    Какое расстояние пролетит мюон в системе отсчёта, относительно которой он движется со скоростью $v$,
    сравнимой со скоростью света в вакууме $c$?
}
\answer{%
    $\ell = v\tau = v \frac{\tau_0}{\sqrt{1 - \frac{v^2}{c^2}}}$
}
\solutionspace{80pt}

\tasknumber{13}%
\task{%
    Если $c$ — скорость света в вакууме, то с какой скоростью должна двигаться нестабильная частица относительно наблюдателя,
    чтобы её время жизни было в девять раз больше, чем у такой же, но покоящейся относительно наблюдателя частицы?
}
\answer{%
    $\tau = \frac{\tau_0}{\sqrt{1 - \frac{v^2}{c^2}}}\implies \sqrt{1 - \frac{v^2}{c^2}} = \frac{\tau_0}{\tau}\implies v = c\sqrt{1 - \sqr{\frac{\tau_0}{\tau}} } \approx 298 \cdot 10^{6}\,\frac{\text{м}}{\text{с}}.$
}
\solutionspace{80pt}

\tasknumber{14}%
\task{%
    Время жизни нестабильной частицы, входящего в состав космических лучей, измеренное земным наблюдателем,
    относительно которого частица двигалась со скоростью, составляющей 65\% скорости света в вакууме, оказалось равным $6{,}4\,\text{мкс}$.
    Каково время жизни частицы, покоящейся относительно наблюдателя?
}
\answer{%
    $t = \frac{t_0}{\sqrt{1 - \frac{v^2}{c^2}}}\implies t_0 = t\sqrt{1 - \frac{v^2}{c^2}} \approx 4{,}9 \cdot 10^{-6}\,\text{с}.$
}
\solutionspace{80pt}

\tasknumber{15}%
\task{%
    Частица увеличила в ускорителе свою скорость с $0{,}03c$ до $0{,}70c$.
    Во сколько раз выросла её кинетическая энергия?
}
\answer{%
    \begin{align*}
    E_{\text{кин.}} &= E - E_0 = \frac{mc^2}{\sqrt{1 - \frac{v^2}{c^2}}} - mc^2 = mc^2\cbr{ \frac1{\sqrt{1 - \frac{v^2}{c^2} }} - 1}.
    \\
    \frac{E_{\text{кин.
    2}}}{E_{\text{кин.
    1}}} &= \frac{\frac1{\sqrt{1 - \frac{v_2^2}{c^2} }} - 1}{\frac1{\sqrt{1 - \frac{v_1^2}{c^2} }} - 1}\approx 888{,}91
    \end{align*}
}
\solutionspace{120pt}

\tasknumber{16}%
\task{%
    Для частицы, движущейся с релятивистской скоростью,
    выразите $p$ и $E_0$ через $c$, $E_\text{кин}$ и $v$,
    где $E_\text{кин}$~--- кинетическая энергия частицы,
    а $E_0$, $p$ и $v$~--- её энергия покоя, импульс и скорость.
}
\answer{%
    \begin{align*}
    E_\text{кин}, E_0:\quad&E = E_\text{кин} + E_0 = \frac{E_0}{\sqrt{1 - \frac{v^2}{c^2}}} \implies \sqrt{1 - \frac{v^2}{c^2}} = \frac{E_0}{{E_0} + {E_\text{кин}}} \implies v = c\sqrt{1 - \sqr{\frac{E_0}{{E_0} + {E_\text{кин}}}}} \\
    &p = \frac{mv}{\sqrt{1 - \frac{v^2}{c^2}}} = \frac{E_0}{c^2} \cdot \sqrt{1 - \sqr{\frac{E_0}{{E_0} + {E_\text{кин}}}}} \cdot \frac{{E_\text{кин}} + {E_0}}{E_0} = \frac{E_0}{c^2} \cdot \sqrt{\sqr{\frac{{E_\text{кин}} + {E_0}}{E_0}} - 1}.
    \\
    E_\text{кин}, p:\quad&E_\text{кин} = E - E_0 = mc^2\cbr{\frac 1{\sqrt{1 - \frac{v^2}{c^2}}} - 1}, p = \frac{mv}{\sqrt{1 - \frac{v^2}{c^2}}} \implies \frac{E_\text{кин}}{p} = \frac{\frac 1{\sqrt{1 - \frac{v^2}{c^2}}} - 1}{\sqrt{1 - \frac{v^2}{c^2}}} \implies v = \ldots \\
    &E_0 = E - E_\text{кин} = \frac{E_0}{\sqrt{1 - \frac{v^2}{c^2}}} - E_\text{кин} \implies E_0 = \frac{E_\text{кин}}{\frac 1{\sqrt{1 - \frac{v^2}{c^2}}} - 1} = \ldots \\
    E_\text{кин}, v:\quad&E_\text{кин} = E - E_0 = mc^2\cbr{\frac 1{\sqrt{1 - \frac{v^2}{c^2}}} - 1} \implies m = \frac{E_\text{кин}}{c^2\cbr{\frac 1{\sqrt{1 - \frac{v^2}{c^2}}} - 1}} \\
    &E_0 = mc^2 = \frac{E_\text{кин}}{\frac 1{\sqrt{1 - \frac{v^2}{c^2}}} - 1} \\
    &p = \frac{mv}{\sqrt{1 - \frac{v^2}{c^2}}} = \frac{E_\text{кин}}{c^2\cbr{\frac 1{\sqrt{1 - \frac{v^2}{c^2}}} - 1}} \cdot \frac{v}{\sqrt{1 - \frac{v^2}{c^2}}} = \frac{{E_\text{кин}} v}{c^2\cbr{1 - {\sqrt{1 - \frac{v^2}{c^2}}}}} \\
    E_0, p:\quad&E_0 = mc^2, \quad p = \frac{mv}{\sqrt{1 - \frac{v^2}{c^2}}} \implies \frac{E_0}{p} = \frac{c^2}v{\sqrt{1 - \frac{v^2}{c^2}}} = c\sqrt{\frac{c^2}{v^2} - 1} \\
    &\sqr{\frac{E_0}{pc}} = \frac{c^2}{v^2} - 1 \implies \frac{v^2}{c^2} = \frac 1{1 + \frac{E_0^2}{p^2c^2}} \implies v = \frac c{\sqrt{1 + \frac{E_0^2}{p^2c^2}}} \\
    &{E_\text{кин}} = E - E_0 = \sqrt{E_0^2 + p^2c^2} - E_0 \\
    E_0, v:\quad&E_0 = mc^2 \implies m = \frac{E_0}{c^2} \qquad p = \frac{mv}{\sqrt{1 - \frac{v^2}{c^2}}} = \frac{E_0}{c^2} \cdot \frac{v}{\sqrt{1 - \frac{v^2}{c^2}}} \\
    &E_\text{кин}= mc^2\cbr{\frac 1{\sqrt{1 - \frac{v^2}{c^2}}} - 1} = \frac{E_0}{c^2}\cbr{\frac 1{\sqrt{1 - \frac{v^2}{c^2}}} - 1} \\
    p, v:\quad&p = \frac{mv}{\sqrt{1 - \frac{v^2}{c^2}}} \implies m = \frac p v {\sqrt{1 - \frac{v^2}{c^2}}} \implies E_0 = mc^2 =\frac {pc^2} v {\sqrt{1 - \frac{v^2}{c^2}}} \\
    &E_\text{кин} = mc^2\cbr{\frac 1{\sqrt{1 - \frac{v^2}{c^2}}} - 1} = \frac p v {\sqrt{1 - \frac{v^2}{c^2}}}\cbr{\frac 1{\sqrt{1 - \frac{v^2}{c^2}}} - 1} = \frac p v \cbr{1 - {\sqrt{1 - \frac{v^2}{c^2}}}}
    \end{align*}
}

\variantsplitter

\addpersonalvariant{Виктория Легонькова}

\tasknumber{1}%
\task{%
    Запишите
    \begin{itemize}
        \item постулаты специальной теории относительности,
        \item пример релятивистского эффекта, обнаружимый при скоростях гораздо меньше скорости света.
    \end{itemize}
}
\solutionspace{120pt}

\tasknumber{2}%
\task{%
    Запишите формулу для ...
    \begin{itemize}
        \item релятивистского сжатия,
        \item классического импульса,
        \item релятивистского импульса тела,
        \item энергии покоя тела,
        \item связь между релятивистским импульсом и релятивистской энергией.
    \end{itemize}
    Обязательно подпишите все физические величины.
}
\solutionspace{150pt}

\tasknumber{3}%
\task{%
    Электрон движется со скоростью $0{,}9\,c$, где $c$~--- скорость света в вакууме.
    Каково при этом отношение кинетической энергии частицы $E_\text{кин.}$ к его энергии покоя $E_0$?
}
\answer{%
    \begin{align*}
    E &= \frac{E_0}{\sqrt{1 - \frac{v^2}{c^2}}}
            \implies \frac E{E_0}
                = \frac 1{\sqrt{1 - \frac{v^2}{c^2}}}
                = \frac 1{\sqrt{1 - \sqr{0{,}9}}}
                \approx 2{,}294,
         \\
        {E_{\text{кин}}} &= E - E_0
            \implies \frac{E_{\text{кин}}}{E_0}
                = \frac E{E_0} - 1
                = \frac 1{\sqrt{1 - \frac{v^2}{c^2}}} - 1
                = \frac 1{\sqrt{1 - \sqr{0{,}9}}} - 1
                \approx 1{,}294.
    \end{align*}
}
\solutionspace{80pt}

\tasknumber{4}%
\task{%
    Полная энергия релятивистской частицы в три раза больше её энергии покоя.
    Найти скорость этой частицы: в долях $c$ и численное значение.
    Скорость света в вакууме $c = 3 \cdot 10^{8}\,\frac{\text{м}}{\text{с}}$.
}
\answer{%
    \begin{align*}
    E &= \frac{E_0}{\sqrt{1 - \frac{v^2}{c^2}}}\implies \sqrt{1 - \frac{v^2}{c^2}} = \frac{E_0}{E}\implies \frac{v^2}{c^2} = 1 - \sqr{\frac{E_0}{E}}\implies v = c \sqrt{1 - \sqr{\frac{E_0}{E}}} \approx 0{,}943c \approx 283 \cdot 10^{6}\,\frac{\text{м}}{\text{с}}.
    \end{align*}
}
\solutionspace{80pt}

\tasknumber{5}%
\task{%
    Кинетическая энергия релятивистской частицы в три раза больше её энергии покоя.
    Найти скорость этой частицы.
    Скорость света в вакууме $c = 3 \cdot 10^{8}\,\frac{\text{м}}{\text{с}}$.
}
\answer{%
    \begin{align*}
    E &= E_0 + E_{\text{кин}} \\
    E &= \frac{E_0}{\sqrt{1 - \frac{v^2}{c^2}}}\implies \sqrt{1 - \frac{v^2}{c^2}} = \frac{E_0}{E}\implies \frac{v^2}{c^2} = 1 - \sqr{\frac{E_0}{E}} \implies \\
    \implies &v = c \sqrt{1 - \sqr{\frac{E_0}{E}}} = c \sqrt{1 - \sqr{\frac{E_0}{E_0 + E_{\text{кин}} }}} = c \sqrt{1 - \frac 1 {\sqr{ 1 + \frac{E_{\text{кин}}}{E_0} }} }\approx 0{,}968c \approx 290 \cdot 10^{6}\,\frac{\text{м}}{\text{с}}.
    \end{align*}
}


\variantsplitter


\addpersonalvariant{Виктория Легонькова}

\tasknumber{6}%
\task{%
    Протон движется со скоростью $0{,}65\,c$, где $c$~--- скорость света в вакууме.
    Определите его импульс (в ответе приведите формулу и укажите численное значение).
}
\answer{%
    \begin{align*}
    E &= \frac{mc^2}{\sqrt{1 - \frac{v^2}{c^2}}}
            \approx \frac{1{,}673 \cdot 10^{-27}\,\text{кг} \cdot \sqr{3 \cdot 10^{8}\,\frac{\text{м}}{\text{с}}}}{\sqrt{1 - 0{,}65^2}}
            \approx 0{,}19809 \cdot 10^{-9}\,\text{Дж},
         \\
        E_{\text{кин}} &= \frac{mc^2}{\sqrt{1 - \frac{v^2}{c^2}}} - mc^2
            = mc^2 \cbr{\frac 1{\sqrt{1 - \frac{v^2}{c^2}}} - 1} \approx \\
            &\approx \cbr{1{,}673 \cdot 10^{-27}\,\text{кг} \cdot \sqr{3 \cdot 10^{8}\,\frac{\text{м}}{\text{с}}}}
            \cdot \cbr{\frac 1{\sqrt{1 - 0{,}65^2}} - 1}
            \approx 47{,}55 \cdot 10^{-12}\,\text{Дж},
         \\
        p &= \frac{mv}{\sqrt{1 - \frac{v^2}{c^2}}}
            \approx \frac{1{,}673 \cdot 10^{-27}\,\text{кг} \cdot 0{,}65 \cdot 3 \cdot 10^{8}\,\frac{\text{м}}{\text{с}}}{\sqrt{1 - 0{,}65^2}}
            \approx 0{,}4292 \cdot 10^{-18}\,\frac{\text{кг}\cdot\text{м}}{\text{с}}.
    \end{align*}
}
\solutionspace{100pt}

\tasknumber{7}%
\task{%
    Кинетическая энергия частицы космических лучей в три раза превышает её энергию покоя.
    Определить отношение скорости частицы к скорости света.
}
\answer{%
    \begin{align*}
    E &= E_0 + E_{\text{кин}} \\
    E &= \frac{E_0}{\sqrt{1 - \frac{v^2}{c^2}}}\implies \sqrt{1 - \frac{v^2}{c^2}} = \frac{E_0}{E}\implies \frac{v^2}{c^2} = 1 - \sqr{\frac{E_0}{E}} \implies \\
    \implies \frac vc &= \sqrt{1 - \sqr{\frac{E_0}{E}}} = \sqrt{1 - \sqr{\frac{E_0}{E_0 + E_{\text{кин}} }}} \approx 0{,}968.
    \end{align*}
}
\solutionspace{80pt}

\tasknumber{8}%
\task{%
    Некоторая частица, пройдя ускоряющую разность потенциалов, приобрела импульс $3 \cdot 10^{-19}\,\frac{\text{кг}\cdot\text{м}}{\text{с}}$.
    Скорость частицы стала равной $2 \cdot 10^{8}\,\frac{\text{м}}{\text{с}}$.
    Найти массу частицы.
}
\answer{%
    $p = \frac{ mv }{\sqrt{1 - \frac{v^2}{c^2} }}\implies m = \frac pv \sqrt{1 - \frac{v^2}{c^2}}= \frac {3 \cdot 10^{-19}\,\frac{\text{кг}\cdot\text{м}}{\text{с}}}{2 \cdot 10^{8}\,\frac{\text{м}}{\text{с}}} \sqrt{1 - \sqr{\frac{2 \cdot 10^{8}\,\frac{\text{м}}{\text{с}}}{3 \cdot 10^{8}\,\frac{\text{м}}{\text{с}}}} } \approx 1{,}12 \cdot 10^{-27}\,\text{кг}.$
}
\solutionspace{80pt}

\tasknumber{9}%
\task{%
    При какой скорости движения (в м/с) релятивистское сокращение длины движущегося тела
    составит 30\%?
}
\answer{%
    \begin{align*}
    l_0 &= \frac l{\sqrt{1 - \frac{v^2}{c^2}}}
        \implies 1 - \frac{v^2}{c^2} = \sqr{\frac l{l_0}}
        \implies \frac v c = \sqrt{1 - \sqr{\frac l{l_0}}} \implies
         \\
        \implies v &= c\sqrt{1 - \sqr{\frac l{l_0}}}
        = 3 \cdot 10^{8}\,\frac{\text{м}}{\text{с}} \cdot \sqrt{1 - \sqr{\frac {l_0 - 0{,}30l_0}{l_0}}}
        = 3 \cdot 10^{8}\,\frac{\text{м}}{\text{с}} \cdot \sqrt{1 - \sqr{1 - 0{,}30}} \approx  \\
        &\approx 0{,}714c
        \approx 214 \cdot 10^{6}\,\frac{\text{м}}{\text{с}}
        \approx 771 \cdot 10^{6}\,\frac{\text{км}}{\text{ч}}.
    \end{align*}
}
\solutionspace{80pt}

\tasknumber{10}%
\task{%
    Стержень движется в продольном направлении с постоянной скоростью относительно инерциальной системы отсчёта.
    При каком значении скорости (в долях скорости света) длина стержня в этой системе отсчёта
    будет в  1{,}67  раза меньше его собственной длины?
}
\answer{%
    $l_0 = \frac l{\sqrt{1 - \frac{v^2}{c^2}}}\implies \sqrt{1 - \frac{v^2}{c^2}} = \frac{ l }{ l_0 }\implies \frac v c = \sqrt{1 - \sqr{\frac{ l }{ l_0 }}} \approx 0{,}801.$
}
\solutionspace{80pt}

\tasknumber{11}%
\task{%
    Какую скорость должно иметь движущееся тело, чтобы его продольные размеры уменьшились в два раза?
    Скорость света $c = 3 \cdot 10^{8}\,\frac{\text{м}}{\text{с}}$.
}
\answer{%
    $l_0 = \frac l{\sqrt{1 - \frac{v^2}{c^2}}}\implies \sqrt{1 - \frac{v^2}{c^2}} = \frac{ l }{ l_0 }\implies v = c\sqrt{1 - \sqr{\frac{ l }{ l_0 }}} \approx 260 \cdot 10^{6}\,\frac{\text{м}}{\text{с}}.$
}


\variantsplitter


\addpersonalvariant{Виктория Легонькова}

\tasknumber{12}%
\task{%
    Время жизни мюона, измеренное наблюдателем, относительно которого мюон покоился, равно $\tau_0$
    Какое расстояние пролетит мюон в системе отсчёта, относительно которой он движется со скоростью $v$,
    сравнимой со скоростью света в вакууме $c$?
}
\answer{%
    $\ell = v\tau = v \frac{\tau_0}{\sqrt{1 - \frac{v^2}{c^2}}}$
}
\solutionspace{80pt}

\tasknumber{13}%
\task{%
    Если $c$ — скорость света в вакууме, то с какой скоростью должна двигаться нестабильная частица относительно наблюдателя,
    чтобы её время жизни было в три раза больше, чем у такой же, но покоящейся относительно наблюдателя частицы?
}
\answer{%
    $\tau = \frac{\tau_0}{\sqrt{1 - \frac{v^2}{c^2}}}\implies \sqrt{1 - \frac{v^2}{c^2}} = \frac{\tau_0}{\tau}\implies v = c\sqrt{1 - \sqr{\frac{\tau_0}{\tau}} } \approx 283 \cdot 10^{6}\,\frac{\text{м}}{\text{с}}.$
}
\solutionspace{80pt}

\tasknumber{14}%
\task{%
    Время жизни нестабильной частицы, входящего в состав космических лучей, измеренное земным наблюдателем,
    относительно которого частица двигалась со скоростью, составляющей 75\% скорости света в вакууме, оказалось равным $4{,}8\,\text{мкс}$.
    Каково время жизни частицы, покоящейся относительно наблюдателя?
}
\answer{%
    $t = \frac{t_0}{\sqrt{1 - \frac{v^2}{c^2}}}\implies t_0 = t\sqrt{1 - \frac{v^2}{c^2}} \approx 3{,}2 \cdot 10^{-6}\,\text{с}.$
}
\solutionspace{80pt}

\tasknumber{15}%
\task{%
    Частица увеличила в ускорителе свою скорость с $0{,}02c$ до $0{,}50c$.
    Во сколько раз выросла её кинетическая энергия?
}
\answer{%
    \begin{align*}
    E_{\text{кин.}} &= E - E_0 = \frac{mc^2}{\sqrt{1 - \frac{v^2}{c^2}}} - mc^2 = mc^2\cbr{ \frac1{\sqrt{1 - \frac{v^2}{c^2} }} - 1}.
    \\
    \frac{E_{\text{кин.
    2}}}{E_{\text{кин.
    1}}} &= \frac{\frac1{\sqrt{1 - \frac{v_2^2}{c^2} }} - 1}{\frac1{\sqrt{1 - \frac{v_1^2}{c^2} }} - 1}\approx 773{,}27
    \end{align*}
}
\solutionspace{120pt}

\tasknumber{16}%
\task{%
    Для частицы, движущейся с релятивистской скоростью,
    выразите $p$ и $v$ через $c$, $E_\text{кин}$ и $E_0$,
    где $E_\text{кин}$~--- кинетическая энергия частицы,
    а $E_0$, $p$ и $v$~--- её энергия покоя, импульс и скорость.
}
\answer{%
    \begin{align*}
    E_\text{кин}, E_0:\quad&E = E_\text{кин} + E_0 = \frac{E_0}{\sqrt{1 - \frac{v^2}{c^2}}} \implies \sqrt{1 - \frac{v^2}{c^2}} = \frac{E_0}{{E_0} + {E_\text{кин}}} \implies v = c\sqrt{1 - \sqr{\frac{E_0}{{E_0} + {E_\text{кин}}}}} \\
    &p = \frac{mv}{\sqrt{1 - \frac{v^2}{c^2}}} = \frac{E_0}{c^2} \cdot \sqrt{1 - \sqr{\frac{E_0}{{E_0} + {E_\text{кин}}}}} \cdot \frac{{E_\text{кин}} + {E_0}}{E_0} = \frac{E_0}{c^2} \cdot \sqrt{\sqr{\frac{{E_\text{кин}} + {E_0}}{E_0}} - 1}.
    \\
    E_\text{кин}, p:\quad&E_\text{кин} = E - E_0 = mc^2\cbr{\frac 1{\sqrt{1 - \frac{v^2}{c^2}}} - 1}, p = \frac{mv}{\sqrt{1 - \frac{v^2}{c^2}}} \implies \frac{E_\text{кин}}{p} = \frac{\frac 1{\sqrt{1 - \frac{v^2}{c^2}}} - 1}{\sqrt{1 - \frac{v^2}{c^2}}} \implies v = \ldots \\
    &E_0 = E - E_\text{кин} = \frac{E_0}{\sqrt{1 - \frac{v^2}{c^2}}} - E_\text{кин} \implies E_0 = \frac{E_\text{кин}}{\frac 1{\sqrt{1 - \frac{v^2}{c^2}}} - 1} = \ldots \\
    E_\text{кин}, v:\quad&E_\text{кин} = E - E_0 = mc^2\cbr{\frac 1{\sqrt{1 - \frac{v^2}{c^2}}} - 1} \implies m = \frac{E_\text{кин}}{c^2\cbr{\frac 1{\sqrt{1 - \frac{v^2}{c^2}}} - 1}} \\
    &E_0 = mc^2 = \frac{E_\text{кин}}{\frac 1{\sqrt{1 - \frac{v^2}{c^2}}} - 1} \\
    &p = \frac{mv}{\sqrt{1 - \frac{v^2}{c^2}}} = \frac{E_\text{кин}}{c^2\cbr{\frac 1{\sqrt{1 - \frac{v^2}{c^2}}} - 1}} \cdot \frac{v}{\sqrt{1 - \frac{v^2}{c^2}}} = \frac{{E_\text{кин}} v}{c^2\cbr{1 - {\sqrt{1 - \frac{v^2}{c^2}}}}} \\
    E_0, p:\quad&E_0 = mc^2, \quad p = \frac{mv}{\sqrt{1 - \frac{v^2}{c^2}}} \implies \frac{E_0}{p} = \frac{c^2}v{\sqrt{1 - \frac{v^2}{c^2}}} = c\sqrt{\frac{c^2}{v^2} - 1} \\
    &\sqr{\frac{E_0}{pc}} = \frac{c^2}{v^2} - 1 \implies \frac{v^2}{c^2} = \frac 1{1 + \frac{E_0^2}{p^2c^2}} \implies v = \frac c{\sqrt{1 + \frac{E_0^2}{p^2c^2}}} \\
    &{E_\text{кин}} = E - E_0 = \sqrt{E_0^2 + p^2c^2} - E_0 \\
    E_0, v:\quad&E_0 = mc^2 \implies m = \frac{E_0}{c^2} \qquad p = \frac{mv}{\sqrt{1 - \frac{v^2}{c^2}}} = \frac{E_0}{c^2} \cdot \frac{v}{\sqrt{1 - \frac{v^2}{c^2}}} \\
    &E_\text{кин}= mc^2\cbr{\frac 1{\sqrt{1 - \frac{v^2}{c^2}}} - 1} = \frac{E_0}{c^2}\cbr{\frac 1{\sqrt{1 - \frac{v^2}{c^2}}} - 1} \\
    p, v:\quad&p = \frac{mv}{\sqrt{1 - \frac{v^2}{c^2}}} \implies m = \frac p v {\sqrt{1 - \frac{v^2}{c^2}}} \implies E_0 = mc^2 =\frac {pc^2} v {\sqrt{1 - \frac{v^2}{c^2}}} \\
    &E_\text{кин} = mc^2\cbr{\frac 1{\sqrt{1 - \frac{v^2}{c^2}}} - 1} = \frac p v {\sqrt{1 - \frac{v^2}{c^2}}}\cbr{\frac 1{\sqrt{1 - \frac{v^2}{c^2}}} - 1} = \frac p v \cbr{1 - {\sqrt{1 - \frac{v^2}{c^2}}}}
    \end{align*}
}

\variantsplitter

\addpersonalvariant{Семён Мартынов}

\tasknumber{1}%
\task{%
    Запишите
    \begin{itemize}
        \item постулаты специальной теории относительности,
        \item пример релятивистского эффекта, обнаружимый при скоростях гораздо меньше скорости света.
    \end{itemize}
}
\solutionspace{120pt}

\tasknumber{2}%
\task{%
    Запишите формулу для ...
    \begin{itemize}
        \item релятивистского замедления времени,
        \item классической полной механической энергии тела,
        \item релятивистской энергии тела,
        \item энергии покоя тела,
        \item связь между релятивистским импульсом и релятивистской энергией.
    \end{itemize}
    Обязательно подпишите все физические величины.
}
\solutionspace{150pt}

\tasknumber{3}%
\task{%
    Позитрон движется со скоростью $0{,}7\,c$, где $c$~--- скорость света в вакууме.
    Каково при этом отношение кинетической энергии частицы $E_\text{кин.}$ к его энергии покоя $E_0$?
}
\answer{%
    \begin{align*}
    E &= \frac{E_0}{\sqrt{1 - \frac{v^2}{c^2}}}
            \implies \frac E{E_0}
                = \frac 1{\sqrt{1 - \frac{v^2}{c^2}}}
                = \frac 1{\sqrt{1 - \sqr{0{,}7}}}
                \approx 1{,}400,
         \\
        {E_{\text{кин}}} &= E - E_0
            \implies \frac{E_{\text{кин}}}{E_0}
                = \frac E{E_0} - 1
                = \frac 1{\sqrt{1 - \frac{v^2}{c^2}}} - 1
                = \frac 1{\sqrt{1 - \sqr{0{,}7}}} - 1
                \approx 0{,}400.
    \end{align*}
}
\solutionspace{80pt}

\tasknumber{4}%
\task{%
    Полная энергия релятивистской частицы в шесть раз больше её энергии покоя.
    Найти скорость этой частицы: в долях $c$ и численное значение.
    Скорость света в вакууме $c = 3 \cdot 10^{8}\,\frac{\text{м}}{\text{с}}$.
}
\answer{%
    \begin{align*}
    E &= \frac{E_0}{\sqrt{1 - \frac{v^2}{c^2}}}\implies \sqrt{1 - \frac{v^2}{c^2}} = \frac{E_0}{E}\implies \frac{v^2}{c^2} = 1 - \sqr{\frac{E_0}{E}}\implies v = c \sqrt{1 - \sqr{\frac{E_0}{E}}} \approx 0{,}986c \approx 296 \cdot 10^{6}\,\frac{\text{м}}{\text{с}}.
    \end{align*}
}
\solutionspace{80pt}

\tasknumber{5}%
\task{%
    Кинетическая энергия релятивистской частицы в шесть раз больше её энергии покоя.
    Найти скорость этой частицы.
    Скорость света в вакууме $c = 3 \cdot 10^{8}\,\frac{\text{м}}{\text{с}}$.
}
\answer{%
    \begin{align*}
    E &= E_0 + E_{\text{кин}} \\
    E &= \frac{E_0}{\sqrt{1 - \frac{v^2}{c^2}}}\implies \sqrt{1 - \frac{v^2}{c^2}} = \frac{E_0}{E}\implies \frac{v^2}{c^2} = 1 - \sqr{\frac{E_0}{E}} \implies \\
    \implies &v = c \sqrt{1 - \sqr{\frac{E_0}{E}}} = c \sqrt{1 - \sqr{\frac{E_0}{E_0 + E_{\text{кин}} }}} = c \sqrt{1 - \frac 1 {\sqr{ 1 + \frac{E_{\text{кин}}}{E_0} }} }\approx 0{,}990c \approx 297 \cdot 10^{6}\,\frac{\text{м}}{\text{с}}.
    \end{align*}
}


\variantsplitter


\addpersonalvariant{Семён Мартынов}

\tasknumber{6}%
\task{%
    Электрон движется со скоростью $0{,}85\,c$, где $c$~--- скорость света в вакууме.
    Определите его кинетическую энергию (в ответе приведите формулу и укажите численное значение).
}
\answer{%
    \begin{align*}
    E &= \frac{mc^2}{\sqrt{1 - \frac{v^2}{c^2}}}
            \approx \frac{9{,}1 \cdot 10^{-31}\,\text{кг} \cdot \sqr{3 \cdot 10^{8}\,\frac{\text{м}}{\text{с}}}}{\sqrt{1 - 0{,}85^2}}
            \approx 0{,}1555 \cdot 10^{-12}\,\text{Дж},
         \\
        E_{\text{кин}} &= \frac{mc^2}{\sqrt{1 - \frac{v^2}{c^2}}} - mc^2
            = mc^2 \cbr{\frac 1{\sqrt{1 - \frac{v^2}{c^2}}} - 1} \approx \\
            &\approx \cbr{9{,}1 \cdot 10^{-31}\,\text{кг} \cdot \sqr{3 \cdot 10^{8}\,\frac{\text{м}}{\text{с}}}}
            \cdot \cbr{\frac 1{\sqrt{1 - 0{,}85^2}} - 1}
            \approx 73{,}6 \cdot 10^{-15}\,\text{Дж},
         \\
        p &= \frac{mv}{\sqrt{1 - \frac{v^2}{c^2}}}
            \approx \frac{9{,}1 \cdot 10^{-31}\,\text{кг} \cdot 0{,}85 \cdot 3 \cdot 10^{8}\,\frac{\text{м}}{\text{с}}}{\sqrt{1 - 0{,}85^2}}
            \approx 0{,}441 \cdot 10^{-21}\,\frac{\text{кг}\cdot\text{м}}{\text{с}}.
    \end{align*}
}
\solutionspace{100pt}

\tasknumber{7}%
\task{%
    Кинетическая энергия частицы космических лучей в шесть раз превышает её энергию покоя.
    Определить отношение скорости частицы к скорости света.
}
\answer{%
    \begin{align*}
    E &= E_0 + E_{\text{кин}} \\
    E &= \frac{E_0}{\sqrt{1 - \frac{v^2}{c^2}}}\implies \sqrt{1 - \frac{v^2}{c^2}} = \frac{E_0}{E}\implies \frac{v^2}{c^2} = 1 - \sqr{\frac{E_0}{E}} \implies \\
    \implies \frac vc &= \sqrt{1 - \sqr{\frac{E_0}{E}}} = \sqrt{1 - \sqr{\frac{E_0}{E_0 + E_{\text{кин}} }}} \approx 0{,}990.
    \end{align*}
}
\solutionspace{80pt}

\tasknumber{8}%
\task{%
    Некоторая частица, пройдя ускоряющую разность потенциалов, приобрела импульс $3 \cdot 10^{-19}\,\frac{\text{кг}\cdot\text{м}}{\text{с}}$.
    Скорость частицы стала равной $1{,}8 \cdot 10^{8}\,\frac{\text{м}}{\text{с}}$.
    Найти массу частицы.
}
\answer{%
    $p = \frac{ mv }{\sqrt{1 - \frac{v^2}{c^2} }}\implies m = \frac pv \sqrt{1 - \frac{v^2}{c^2}}= \frac {3 \cdot 10^{-19}\,\frac{\text{кг}\cdot\text{м}}{\text{с}}}{1{,}8 \cdot 10^{8}\,\frac{\text{м}}{\text{с}}} \sqrt{1 - \sqr{\frac{1{,}8 \cdot 10^{8}\,\frac{\text{м}}{\text{с}}}{3 \cdot 10^{8}\,\frac{\text{м}}{\text{с}}}} } \approx 1{,}33 \cdot 10^{-27}\,\text{кг}.$
}
\solutionspace{80pt}

\tasknumber{9}%
\task{%
    При какой скорости движения (в м/с) релятивистское сокращение длины движущегося тела
    составит 30\%?
}
\answer{%
    \begin{align*}
    l_0 &= \frac l{\sqrt{1 - \frac{v^2}{c^2}}}
        \implies 1 - \frac{v^2}{c^2} = \sqr{\frac l{l_0}}
        \implies \frac v c = \sqrt{1 - \sqr{\frac l{l_0}}} \implies
         \\
        \implies v &= c\sqrt{1 - \sqr{\frac l{l_0}}}
        = 3 \cdot 10^{8}\,\frac{\text{м}}{\text{с}} \cdot \sqrt{1 - \sqr{\frac {l_0 - 0{,}30l_0}{l_0}}}
        = 3 \cdot 10^{8}\,\frac{\text{м}}{\text{с}} \cdot \sqrt{1 - \sqr{1 - 0{,}30}} \approx  \\
        &\approx 0{,}714c
        \approx 214 \cdot 10^{6}\,\frac{\text{м}}{\text{с}}
        \approx 771 \cdot 10^{6}\,\frac{\text{км}}{\text{ч}}.
    \end{align*}
}
\solutionspace{80pt}

\tasknumber{10}%
\task{%
    Стержень движется в продольном направлении с постоянной скоростью относительно инерциальной системы отсчёта.
    При каком значении скорости (в долях скорости света) длина стержня в этой системе отсчёта
    будет в  1{,}25  раза меньше его собственной длины?
}
\answer{%
    $l_0 = \frac l{\sqrt{1 - \frac{v^2}{c^2}}}\implies \sqrt{1 - \frac{v^2}{c^2}} = \frac{ l }{ l_0 }\implies \frac v c = \sqrt{1 - \sqr{\frac{ l }{ l_0 }}} \approx 0{,}600.$
}
\solutionspace{80pt}

\tasknumber{11}%
\task{%
    Какую скорость должно иметь движущееся тело, чтобы его продольные размеры уменьшились в пять раз?
    Скорость света $c = 3 \cdot 10^{8}\,\frac{\text{м}}{\text{с}}$.
}
\answer{%
    $l_0 = \frac l{\sqrt{1 - \frac{v^2}{c^2}}}\implies \sqrt{1 - \frac{v^2}{c^2}} = \frac{ l }{ l_0 }\implies v = c\sqrt{1 - \sqr{\frac{ l }{ l_0 }}} \approx 294 \cdot 10^{6}\,\frac{\text{м}}{\text{с}}.$
}


\variantsplitter


\addpersonalvariant{Семён Мартынов}

\tasknumber{12}%
\task{%
    Время жизни мюона, измеренное наблюдателем, относительно которого мюон покоился, равно $\tau_0$
    Какое расстояние пролетит мюон в системе отсчёта, относительно которой он движется со скоростью $v$,
    сравнимой со скоростью света в вакууме $c$?
}
\answer{%
    $\ell = v\tau = v \frac{\tau_0}{\sqrt{1 - \frac{v^2}{c^2}}}$
}
\solutionspace{80pt}

\tasknumber{13}%
\task{%
    Если $c$ — скорость света в вакууме, то с какой скоростью должна двигаться нестабильная частица относительно наблюдателя,
    чтобы её время жизни было в десять раз больше, чем у такой же, но покоящейся относительно наблюдателя частицы?
}
\answer{%
    $\tau = \frac{\tau_0}{\sqrt{1 - \frac{v^2}{c^2}}}\implies \sqrt{1 - \frac{v^2}{c^2}} = \frac{\tau_0}{\tau}\implies v = c\sqrt{1 - \sqr{\frac{\tau_0}{\tau}} } \approx 298 \cdot 10^{6}\,\frac{\text{м}}{\text{с}}.$
}
\solutionspace{80pt}

\tasknumber{14}%
\task{%
    Время жизни нестабильной частицы, входящего в состав космических лучей, измеренное земным наблюдателем,
    относительно которого частица двигалась со скоростью, составляющей 65\% скорости света в вакууме, оказалось равным $4{,}8\,\text{мкс}$.
    Каково время жизни частицы, покоящейся относительно наблюдателя?
}
\answer{%
    $t = \frac{t_0}{\sqrt{1 - \frac{v^2}{c^2}}}\implies t_0 = t\sqrt{1 - \frac{v^2}{c^2}} \approx 3{,}6 \cdot 10^{-6}\,\text{с}.$
}
\solutionspace{80pt}

\tasknumber{15}%
\task{%
    Частица увеличила в ускорителе свою скорость с $0{,}05c$ до $0{,}50c$.
    Во сколько раз выросла её кинетическая энергия?
}
\answer{%
    \begin{align*}
    E_{\text{кин.}} &= E - E_0 = \frac{mc^2}{\sqrt{1 - \frac{v^2}{c^2}}} - mc^2 = mc^2\cbr{ \frac1{\sqrt{1 - \frac{v^2}{c^2} }} - 1}.
    \\
    \frac{E_{\text{кин.
    2}}}{E_{\text{кин.
    1}}} &= \frac{\frac1{\sqrt{1 - \frac{v_2^2}{c^2} }} - 1}{\frac1{\sqrt{1 - \frac{v_1^2}{c^2} }} - 1}\approx 123{,}53
    \end{align*}
}
\solutionspace{120pt}

\tasknumber{16}%
\task{%
    Для частицы, движущейся с релятивистской скоростью,
    выразите $E_0$ и $p$ через $c$, $E_\text{кин}$ и $v$,
    где $E_\text{кин}$~--- кинетическая энергия частицы,
    а $E_0$, $p$ и $v$~--- её энергия покоя, импульс и скорость.
}
\answer{%
    \begin{align*}
    E_\text{кин}, E_0:\quad&E = E_\text{кин} + E_0 = \frac{E_0}{\sqrt{1 - \frac{v^2}{c^2}}} \implies \sqrt{1 - \frac{v^2}{c^2}} = \frac{E_0}{{E_0} + {E_\text{кин}}} \implies v = c\sqrt{1 - \sqr{\frac{E_0}{{E_0} + {E_\text{кин}}}}} \\
    &p = \frac{mv}{\sqrt{1 - \frac{v^2}{c^2}}} = \frac{E_0}{c^2} \cdot \sqrt{1 - \sqr{\frac{E_0}{{E_0} + {E_\text{кин}}}}} \cdot \frac{{E_\text{кин}} + {E_0}}{E_0} = \frac{E_0}{c^2} \cdot \sqrt{\sqr{\frac{{E_\text{кин}} + {E_0}}{E_0}} - 1}.
    \\
    E_\text{кин}, p:\quad&E_\text{кин} = E - E_0 = mc^2\cbr{\frac 1{\sqrt{1 - \frac{v^2}{c^2}}} - 1}, p = \frac{mv}{\sqrt{1 - \frac{v^2}{c^2}}} \implies \frac{E_\text{кин}}{p} = \frac{\frac 1{\sqrt{1 - \frac{v^2}{c^2}}} - 1}{\sqrt{1 - \frac{v^2}{c^2}}} \implies v = \ldots \\
    &E_0 = E - E_\text{кин} = \frac{E_0}{\sqrt{1 - \frac{v^2}{c^2}}} - E_\text{кин} \implies E_0 = \frac{E_\text{кин}}{\frac 1{\sqrt{1 - \frac{v^2}{c^2}}} - 1} = \ldots \\
    E_\text{кин}, v:\quad&E_\text{кин} = E - E_0 = mc^2\cbr{\frac 1{\sqrt{1 - \frac{v^2}{c^2}}} - 1} \implies m = \frac{E_\text{кин}}{c^2\cbr{\frac 1{\sqrt{1 - \frac{v^2}{c^2}}} - 1}} \\
    &E_0 = mc^2 = \frac{E_\text{кин}}{\frac 1{\sqrt{1 - \frac{v^2}{c^2}}} - 1} \\
    &p = \frac{mv}{\sqrt{1 - \frac{v^2}{c^2}}} = \frac{E_\text{кин}}{c^2\cbr{\frac 1{\sqrt{1 - \frac{v^2}{c^2}}} - 1}} \cdot \frac{v}{\sqrt{1 - \frac{v^2}{c^2}}} = \frac{{E_\text{кин}} v}{c^2\cbr{1 - {\sqrt{1 - \frac{v^2}{c^2}}}}} \\
    E_0, p:\quad&E_0 = mc^2, \quad p = \frac{mv}{\sqrt{1 - \frac{v^2}{c^2}}} \implies \frac{E_0}{p} = \frac{c^2}v{\sqrt{1 - \frac{v^2}{c^2}}} = c\sqrt{\frac{c^2}{v^2} - 1} \\
    &\sqr{\frac{E_0}{pc}} = \frac{c^2}{v^2} - 1 \implies \frac{v^2}{c^2} = \frac 1{1 + \frac{E_0^2}{p^2c^2}} \implies v = \frac c{\sqrt{1 + \frac{E_0^2}{p^2c^2}}} \\
    &{E_\text{кин}} = E - E_0 = \sqrt{E_0^2 + p^2c^2} - E_0 \\
    E_0, v:\quad&E_0 = mc^2 \implies m = \frac{E_0}{c^2} \qquad p = \frac{mv}{\sqrt{1 - \frac{v^2}{c^2}}} = \frac{E_0}{c^2} \cdot \frac{v}{\sqrt{1 - \frac{v^2}{c^2}}} \\
    &E_\text{кин}= mc^2\cbr{\frac 1{\sqrt{1 - \frac{v^2}{c^2}}} - 1} = \frac{E_0}{c^2}\cbr{\frac 1{\sqrt{1 - \frac{v^2}{c^2}}} - 1} \\
    p, v:\quad&p = \frac{mv}{\sqrt{1 - \frac{v^2}{c^2}}} \implies m = \frac p v {\sqrt{1 - \frac{v^2}{c^2}}} \implies E_0 = mc^2 =\frac {pc^2} v {\sqrt{1 - \frac{v^2}{c^2}}} \\
    &E_\text{кин} = mc^2\cbr{\frac 1{\sqrt{1 - \frac{v^2}{c^2}}} - 1} = \frac p v {\sqrt{1 - \frac{v^2}{c^2}}}\cbr{\frac 1{\sqrt{1 - \frac{v^2}{c^2}}} - 1} = \frac p v \cbr{1 - {\sqrt{1 - \frac{v^2}{c^2}}}}
    \end{align*}
}

\variantsplitter

\addpersonalvariant{Варвара Минаева}

\tasknumber{1}%
\task{%
    Запишите
    \begin{itemize}
        \item постулаты специальной теории относительности,
        \item пример релятивистского эффекта, обнаружимый при скоростях гораздо меньше скорости света.
    \end{itemize}
}
\solutionspace{120pt}

\tasknumber{2}%
\task{%
    Запишите формулу для ...
    \begin{itemize}
        \item релятивистского замедления времени,
        \item классического импульса,
        \item релятивистского импульса тела,
        \item релятивистской кинетической энергии,
        \item связь между релятивистским импульсом и релятивистской энергией.
    \end{itemize}
    Обязательно подпишите все физические величины.
}
\solutionspace{150pt}

\tasknumber{3}%
\task{%
    Протон движется со скоростью $0{,}7\,c$, где $c$~--- скорость света в вакууме.
    Каково при этом отношение полной энергии частицы $E$ к его энергии покоя $E_0$?
}
\answer{%
    \begin{align*}
    E &= \frac{E_0}{\sqrt{1 - \frac{v^2}{c^2}}}
            \implies \frac E{E_0}
                = \frac 1{\sqrt{1 - \frac{v^2}{c^2}}}
                = \frac 1{\sqrt{1 - \sqr{0{,}7}}}
                \approx 1{,}400,
         \\
        {E_{\text{кин}}} &= E - E_0
            \implies \frac{E_{\text{кин}}}{E_0}
                = \frac E{E_0} - 1
                = \frac 1{\sqrt{1 - \frac{v^2}{c^2}}} - 1
                = \frac 1{\sqrt{1 - \sqr{0{,}7}}} - 1
                \approx 0{,}400.
    \end{align*}
}
\solutionspace{80pt}

\tasknumber{4}%
\task{%
    Полная энергия релятивистской частицы в четыре раза больше её энергии покоя.
    Найти скорость этой частицы: в долях $c$ и численное значение.
    Скорость света в вакууме $c = 3 \cdot 10^{8}\,\frac{\text{м}}{\text{с}}$.
}
\answer{%
    \begin{align*}
    E &= \frac{E_0}{\sqrt{1 - \frac{v^2}{c^2}}}\implies \sqrt{1 - \frac{v^2}{c^2}} = \frac{E_0}{E}\implies \frac{v^2}{c^2} = 1 - \sqr{\frac{E_0}{E}}\implies v = c \sqrt{1 - \sqr{\frac{E_0}{E}}} \approx 0{,}968c \approx 290 \cdot 10^{6}\,\frac{\text{м}}{\text{с}}.
    \end{align*}
}
\solutionspace{80pt}

\tasknumber{5}%
\task{%
    Кинетическая энергия релятивистской частицы в четыре раза больше её энергии покоя.
    Найти скорость этой частицы.
    Скорость света в вакууме $c = 3 \cdot 10^{8}\,\frac{\text{м}}{\text{с}}$.
}
\answer{%
    \begin{align*}
    E &= E_0 + E_{\text{кин}} \\
    E &= \frac{E_0}{\sqrt{1 - \frac{v^2}{c^2}}}\implies \sqrt{1 - \frac{v^2}{c^2}} = \frac{E_0}{E}\implies \frac{v^2}{c^2} = 1 - \sqr{\frac{E_0}{E}} \implies \\
    \implies &v = c \sqrt{1 - \sqr{\frac{E_0}{E}}} = c \sqrt{1 - \sqr{\frac{E_0}{E_0 + E_{\text{кин}} }}} = c \sqrt{1 - \frac 1 {\sqr{ 1 + \frac{E_{\text{кин}}}{E_0} }} }\approx 0{,}980c \approx 294 \cdot 10^{6}\,\frac{\text{м}}{\text{с}}.
    \end{align*}
}


\variantsplitter


\addpersonalvariant{Варвара Минаева}

\tasknumber{6}%
\task{%
    Электрон движется со скоростью $0{,}75\,c$, где $c$~--- скорость света в вакууме.
    Определите его импульс (в ответе приведите формулу и укажите численное значение).
}
\answer{%
    \begin{align*}
    E &= \frac{mc^2}{\sqrt{1 - \frac{v^2}{c^2}}}
            \approx \frac{9{,}1 \cdot 10^{-31}\,\text{кг} \cdot \sqr{3 \cdot 10^{8}\,\frac{\text{м}}{\text{с}}}}{\sqrt{1 - 0{,}75^2}}
            \approx 0{,}1238 \cdot 10^{-12}\,\text{Дж},
         \\
        E_{\text{кин}} &= \frac{mc^2}{\sqrt{1 - \frac{v^2}{c^2}}} - mc^2
            = mc^2 \cbr{\frac 1{\sqrt{1 - \frac{v^2}{c^2}}} - 1} \approx \\
            &\approx \cbr{9{,}1 \cdot 10^{-31}\,\text{кг} \cdot \sqr{3 \cdot 10^{8}\,\frac{\text{м}}{\text{с}}}}
            \cdot \cbr{\frac 1{\sqrt{1 - 0{,}75^2}} - 1}
            \approx 41{,}9 \cdot 10^{-15}\,\text{Дж},
         \\
        p &= \frac{mv}{\sqrt{1 - \frac{v^2}{c^2}}}
            \approx \frac{9{,}1 \cdot 10^{-31}\,\text{кг} \cdot 0{,}75 \cdot 3 \cdot 10^{8}\,\frac{\text{м}}{\text{с}}}{\sqrt{1 - 0{,}75^2}}
            \approx 0{,}310 \cdot 10^{-21}\,\frac{\text{кг}\cdot\text{м}}{\text{с}}.
    \end{align*}
}
\solutionspace{100pt}

\tasknumber{7}%
\task{%
    Кинетическая энергия частицы космических лучей в четыре раза превышает её энергию покоя.
    Определить отношение скорости частицы к скорости света.
}
\answer{%
    \begin{align*}
    E &= E_0 + E_{\text{кин}} \\
    E &= \frac{E_0}{\sqrt{1 - \frac{v^2}{c^2}}}\implies \sqrt{1 - \frac{v^2}{c^2}} = \frac{E_0}{E}\implies \frac{v^2}{c^2} = 1 - \sqr{\frac{E_0}{E}} \implies \\
    \implies \frac vc &= \sqrt{1 - \sqr{\frac{E_0}{E}}} = \sqrt{1 - \sqr{\frac{E_0}{E_0 + E_{\text{кин}} }}} \approx 0{,}980.
    \end{align*}
}
\solutionspace{80pt}

\tasknumber{8}%
\task{%
    Некоторая частица, пройдя ускоряющую разность потенциалов, приобрела импульс $3{,}5 \cdot 10^{-19}\,\frac{\text{кг}\cdot\text{м}}{\text{с}}$.
    Скорость частицы стала равной $1{,}5 \cdot 10^{8}\,\frac{\text{м}}{\text{с}}$.
    Найти массу частицы.
}
\answer{%
    $p = \frac{ mv }{\sqrt{1 - \frac{v^2}{c^2} }}\implies m = \frac pv \sqrt{1 - \frac{v^2}{c^2}}= \frac {3{,}5 \cdot 10^{-19}\,\frac{\text{кг}\cdot\text{м}}{\text{с}}}{1{,}5 \cdot 10^{8}\,\frac{\text{м}}{\text{с}}} \sqrt{1 - \sqr{\frac{1{,}5 \cdot 10^{8}\,\frac{\text{м}}{\text{с}}}{3 \cdot 10^{8}\,\frac{\text{м}}{\text{с}}}} } \approx 2{,}0 \cdot 10^{-27}\,\text{кг}.$
}
\solutionspace{80pt}

\tasknumber{9}%
\task{%
    При какой скорости движения (в км/ч) релятивистское сокращение длины движущегося тела
    составит 50\%?
}
\answer{%
    \begin{align*}
    l_0 &= \frac l{\sqrt{1 - \frac{v^2}{c^2}}}
        \implies 1 - \frac{v^2}{c^2} = \sqr{\frac l{l_0}}
        \implies \frac v c = \sqrt{1 - \sqr{\frac l{l_0}}} \implies
         \\
        \implies v &= c\sqrt{1 - \sqr{\frac l{l_0}}}
        = 3 \cdot 10^{8}\,\frac{\text{м}}{\text{с}} \cdot \sqrt{1 - \sqr{\frac {l_0 - 0{,}50l_0}{l_0}}}
        = 3 \cdot 10^{8}\,\frac{\text{м}}{\text{с}} \cdot \sqrt{1 - \sqr{1 - 0{,}50}} \approx  \\
        &\approx 0{,}866c
        \approx 260 \cdot 10^{6}\,\frac{\text{м}}{\text{с}}
        \approx 935 \cdot 10^{6}\,\frac{\text{км}}{\text{ч}}.
    \end{align*}
}
\solutionspace{80pt}

\tasknumber{10}%
\task{%
    Стержень движется в продольном направлении с постоянной скоростью относительно инерциальной системы отсчёта.
    При каком значении скорости (в долях скорости света) длина стержня в этой системе отсчёта
    будет в  1{,}5  раза меньше его собственной длины?
}
\answer{%
    $l_0 = \frac l{\sqrt{1 - \frac{v^2}{c^2}}}\implies \sqrt{1 - \frac{v^2}{c^2}} = \frac{ l }{ l_0 }\implies \frac v c = \sqrt{1 - \sqr{\frac{ l }{ l_0 }}} \approx 0{,}745.$
}
\solutionspace{80pt}

\tasknumber{11}%
\task{%
    Какую скорость должно иметь движущееся тело, чтобы его продольные размеры уменьшились в шесть раз?
    Скорость света $c = 3 \cdot 10^{8}\,\frac{\text{м}}{\text{с}}$.
}
\answer{%
    $l_0 = \frac l{\sqrt{1 - \frac{v^2}{c^2}}}\implies \sqrt{1 - \frac{v^2}{c^2}} = \frac{ l }{ l_0 }\implies v = c\sqrt{1 - \sqr{\frac{ l }{ l_0 }}} \approx 296 \cdot 10^{6}\,\frac{\text{м}}{\text{с}}.$
}


\variantsplitter


\addpersonalvariant{Варвара Минаева}

\tasknumber{12}%
\task{%
    Время жизни мюона, измеренное наблюдателем, относительно которого мюон покоился, равно $\tau_0$
    Какое расстояние пролетит мюон в системе отсчёта, относительно которой он движется со скоростью $v$,
    сравнимой со скоростью света в вакууме $c$?
}
\answer{%
    $\ell = v\tau = v \frac{\tau_0}{\sqrt{1 - \frac{v^2}{c^2}}}$
}
\solutionspace{80pt}

\tasknumber{13}%
\task{%
    Если $c$ — скорость света в вакууме, то с какой скоростью должна двигаться нестабильная частица относительно наблюдателя,
    чтобы её время жизни было в четыре раза больше, чем у такой же, но покоящейся относительно наблюдателя частицы?
}
\answer{%
    $\tau = \frac{\tau_0}{\sqrt{1 - \frac{v^2}{c^2}}}\implies \sqrt{1 - \frac{v^2}{c^2}} = \frac{\tau_0}{\tau}\implies v = c\sqrt{1 - \sqr{\frac{\tau_0}{\tau}} } \approx 290 \cdot 10^{6}\,\frac{\text{м}}{\text{с}}.$
}
\solutionspace{80pt}

\tasknumber{14}%
\task{%
    Время жизни нестабильной частицы, входящего в состав космических лучей, измеренное земным наблюдателем,
    относительно которого частица двигалась со скоростью, составляющей 85\% скорости света в вакууме, оказалось равным $4{,}8\,\text{мкс}$.
    Каково время жизни частицы, покоящейся относительно наблюдателя?
}
\answer{%
    $t = \frac{t_0}{\sqrt{1 - \frac{v^2}{c^2}}}\implies t_0 = t\sqrt{1 - \frac{v^2}{c^2}} \approx 2{,}5 \cdot 10^{-6}\,\text{с}.$
}
\solutionspace{80pt}

\tasknumber{15}%
\task{%
    Частица увеличила в ускорителе свою скорость с $0{,}04c$ до $0{,}60c$.
    Во сколько раз выросла её кинетическая энергия?
}
\answer{%
    \begin{align*}
    E_{\text{кин.}} &= E - E_0 = \frac{mc^2}{\sqrt{1 - \frac{v^2}{c^2}}} - mc^2 = mc^2\cbr{ \frac1{\sqrt{1 - \frac{v^2}{c^2} }} - 1}.
    \\
    \frac{E_{\text{кин.
    2}}}{E_{\text{кин.
    1}}} &= \frac{\frac1{\sqrt{1 - \frac{v_2^2}{c^2} }} - 1}{\frac1{\sqrt{1 - \frac{v_1^2}{c^2} }} - 1}\approx 312{,}12
    \end{align*}
}
\solutionspace{120pt}

\tasknumber{16}%
\task{%
    Для частицы, движущейся с релятивистской скоростью,
    выразите $E_0$ и $E_\text{кин}$ через $c$, $p$ и $v$,
    где $E_\text{кин}$~--- кинетическая энергия частицы,
    а $E_0$, $p$ и $v$~--- её энергия покоя, импульс и скорость.
}
\answer{%
    \begin{align*}
    E_\text{кин}, E_0:\quad&E = E_\text{кин} + E_0 = \frac{E_0}{\sqrt{1 - \frac{v^2}{c^2}}} \implies \sqrt{1 - \frac{v^2}{c^2}} = \frac{E_0}{{E_0} + {E_\text{кин}}} \implies v = c\sqrt{1 - \sqr{\frac{E_0}{{E_0} + {E_\text{кин}}}}} \\
    &p = \frac{mv}{\sqrt{1 - \frac{v^2}{c^2}}} = \frac{E_0}{c^2} \cdot \sqrt{1 - \sqr{\frac{E_0}{{E_0} + {E_\text{кин}}}}} \cdot \frac{{E_\text{кин}} + {E_0}}{E_0} = \frac{E_0}{c^2} \cdot \sqrt{\sqr{\frac{{E_\text{кин}} + {E_0}}{E_0}} - 1}.
    \\
    E_\text{кин}, p:\quad&E_\text{кин} = E - E_0 = mc^2\cbr{\frac 1{\sqrt{1 - \frac{v^2}{c^2}}} - 1}, p = \frac{mv}{\sqrt{1 - \frac{v^2}{c^2}}} \implies \frac{E_\text{кин}}{p} = \frac{\frac 1{\sqrt{1 - \frac{v^2}{c^2}}} - 1}{\sqrt{1 - \frac{v^2}{c^2}}} \implies v = \ldots \\
    &E_0 = E - E_\text{кин} = \frac{E_0}{\sqrt{1 - \frac{v^2}{c^2}}} - E_\text{кин} \implies E_0 = \frac{E_\text{кин}}{\frac 1{\sqrt{1 - \frac{v^2}{c^2}}} - 1} = \ldots \\
    E_\text{кин}, v:\quad&E_\text{кин} = E - E_0 = mc^2\cbr{\frac 1{\sqrt{1 - \frac{v^2}{c^2}}} - 1} \implies m = \frac{E_\text{кин}}{c^2\cbr{\frac 1{\sqrt{1 - \frac{v^2}{c^2}}} - 1}} \\
    &E_0 = mc^2 = \frac{E_\text{кин}}{\frac 1{\sqrt{1 - \frac{v^2}{c^2}}} - 1} \\
    &p = \frac{mv}{\sqrt{1 - \frac{v^2}{c^2}}} = \frac{E_\text{кин}}{c^2\cbr{\frac 1{\sqrt{1 - \frac{v^2}{c^2}}} - 1}} \cdot \frac{v}{\sqrt{1 - \frac{v^2}{c^2}}} = \frac{{E_\text{кин}} v}{c^2\cbr{1 - {\sqrt{1 - \frac{v^2}{c^2}}}}} \\
    E_0, p:\quad&E_0 = mc^2, \quad p = \frac{mv}{\sqrt{1 - \frac{v^2}{c^2}}} \implies \frac{E_0}{p} = \frac{c^2}v{\sqrt{1 - \frac{v^2}{c^2}}} = c\sqrt{\frac{c^2}{v^2} - 1} \\
    &\sqr{\frac{E_0}{pc}} = \frac{c^2}{v^2} - 1 \implies \frac{v^2}{c^2} = \frac 1{1 + \frac{E_0^2}{p^2c^2}} \implies v = \frac c{\sqrt{1 + \frac{E_0^2}{p^2c^2}}} \\
    &{E_\text{кин}} = E - E_0 = \sqrt{E_0^2 + p^2c^2} - E_0 \\
    E_0, v:\quad&E_0 = mc^2 \implies m = \frac{E_0}{c^2} \qquad p = \frac{mv}{\sqrt{1 - \frac{v^2}{c^2}}} = \frac{E_0}{c^2} \cdot \frac{v}{\sqrt{1 - \frac{v^2}{c^2}}} \\
    &E_\text{кин}= mc^2\cbr{\frac 1{\sqrt{1 - \frac{v^2}{c^2}}} - 1} = \frac{E_0}{c^2}\cbr{\frac 1{\sqrt{1 - \frac{v^2}{c^2}}} - 1} \\
    p, v:\quad&p = \frac{mv}{\sqrt{1 - \frac{v^2}{c^2}}} \implies m = \frac p v {\sqrt{1 - \frac{v^2}{c^2}}} \implies E_0 = mc^2 =\frac {pc^2} v {\sqrt{1 - \frac{v^2}{c^2}}} \\
    &E_\text{кин} = mc^2\cbr{\frac 1{\sqrt{1 - \frac{v^2}{c^2}}} - 1} = \frac p v {\sqrt{1 - \frac{v^2}{c^2}}}\cbr{\frac 1{\sqrt{1 - \frac{v^2}{c^2}}} - 1} = \frac p v \cbr{1 - {\sqrt{1 - \frac{v^2}{c^2}}}}
    \end{align*}
}

\variantsplitter

\addpersonalvariant{Леонид Никитин}

\tasknumber{1}%
\task{%
    Запишите
    \begin{itemize}
        \item постулаты специальной теории относительности,
        \item пример релятивистского эффекта, обнаружимый при скоростях гораздо меньше скорости света.
    \end{itemize}
}
\solutionspace{120pt}

\tasknumber{2}%
\task{%
    Запишите формулу для ...
    \begin{itemize}
        \item релятивистского замедления времени,
        \item классической полной механической энергии тела,
        \item релятивистского импульса тела,
        \item энергии покоя тела,
        \item связь между релятивистским импульсом и релятивистской энергией.
    \end{itemize}
    Обязательно подпишите все физические величины.
}
\solutionspace{150pt}

\tasknumber{3}%
\task{%
    Электрон движется со скоростью $0{,}6\,c$, где $c$~--- скорость света в вакууме.
    Каково при этом отношение кинетической энергии частицы $E_\text{кин.}$ к его энергии покоя $E_0$?
}
\answer{%
    \begin{align*}
    E &= \frac{E_0}{\sqrt{1 - \frac{v^2}{c^2}}}
            \implies \frac E{E_0}
                = \frac 1{\sqrt{1 - \frac{v^2}{c^2}}}
                = \frac 1{\sqrt{1 - \sqr{0{,}6}}}
                \approx 1{,}250,
         \\
        {E_{\text{кин}}} &= E - E_0
            \implies \frac{E_{\text{кин}}}{E_0}
                = \frac E{E_0} - 1
                = \frac 1{\sqrt{1 - \frac{v^2}{c^2}}} - 1
                = \frac 1{\sqrt{1 - \sqr{0{,}6}}} - 1
                \approx 0{,}250.
    \end{align*}
}
\solutionspace{80pt}

\tasknumber{4}%
\task{%
    Полная энергия релятивистской частицы в четыре раза больше её энергии покоя.
    Найти скорость этой частицы: в долях $c$ и численное значение.
    Скорость света в вакууме $c = 3 \cdot 10^{8}\,\frac{\text{м}}{\text{с}}$.
}
\answer{%
    \begin{align*}
    E &= \frac{E_0}{\sqrt{1 - \frac{v^2}{c^2}}}\implies \sqrt{1 - \frac{v^2}{c^2}} = \frac{E_0}{E}\implies \frac{v^2}{c^2} = 1 - \sqr{\frac{E_0}{E}}\implies v = c \sqrt{1 - \sqr{\frac{E_0}{E}}} \approx 0{,}968c \approx 290 \cdot 10^{6}\,\frac{\text{м}}{\text{с}}.
    \end{align*}
}
\solutionspace{80pt}

\tasknumber{5}%
\task{%
    Кинетическая энергия релятивистской частицы в четыре раза больше её энергии покоя.
    Найти скорость этой частицы.
    Скорость света в вакууме $c = 3 \cdot 10^{8}\,\frac{\text{м}}{\text{с}}$.
}
\answer{%
    \begin{align*}
    E &= E_0 + E_{\text{кин}} \\
    E &= \frac{E_0}{\sqrt{1 - \frac{v^2}{c^2}}}\implies \sqrt{1 - \frac{v^2}{c^2}} = \frac{E_0}{E}\implies \frac{v^2}{c^2} = 1 - \sqr{\frac{E_0}{E}} \implies \\
    \implies &v = c \sqrt{1 - \sqr{\frac{E_0}{E}}} = c \sqrt{1 - \sqr{\frac{E_0}{E_0 + E_{\text{кин}} }}} = c \sqrt{1 - \frac 1 {\sqr{ 1 + \frac{E_{\text{кин}}}{E_0} }} }\approx 0{,}980c \approx 294 \cdot 10^{6}\,\frac{\text{м}}{\text{с}}.
    \end{align*}
}


\variantsplitter


\addpersonalvariant{Леонид Никитин}

\tasknumber{6}%
\task{%
    Протон движется со скоростью $0{,}75\,c$, где $c$~--- скорость света в вакууме.
    Определите его импульс (в ответе приведите формулу и укажите численное значение).
}
\answer{%
    \begin{align*}
    E &= \frac{mc^2}{\sqrt{1 - \frac{v^2}{c^2}}}
            \approx \frac{1{,}673 \cdot 10^{-27}\,\text{кг} \cdot \sqr{3 \cdot 10^{8}\,\frac{\text{м}}{\text{с}}}}{\sqrt{1 - 0{,}75^2}}
            \approx 0{,}2276 \cdot 10^{-9}\,\text{Дж},
         \\
        E_{\text{кин}} &= \frac{mc^2}{\sqrt{1 - \frac{v^2}{c^2}}} - mc^2
            = mc^2 \cbr{\frac 1{\sqrt{1 - \frac{v^2}{c^2}}} - 1} \approx \\
            &\approx \cbr{1{,}673 \cdot 10^{-27}\,\text{кг} \cdot \sqr{3 \cdot 10^{8}\,\frac{\text{м}}{\text{с}}}}
            \cdot \cbr{\frac 1{\sqrt{1 - 0{,}75^2}} - 1}
            \approx 77{,}05 \cdot 10^{-12}\,\text{Дж},
         \\
        p &= \frac{mv}{\sqrt{1 - \frac{v^2}{c^2}}}
            \approx \frac{1{,}673 \cdot 10^{-27}\,\text{кг} \cdot 0{,}75 \cdot 3 \cdot 10^{8}\,\frac{\text{м}}{\text{с}}}{\sqrt{1 - 0{,}75^2}}
            \approx 0{,}5690 \cdot 10^{-18}\,\frac{\text{кг}\cdot\text{м}}{\text{с}}.
    \end{align*}
}
\solutionspace{100pt}

\tasknumber{7}%
\task{%
    Кинетическая энергия частицы космических лучей в четыре раза превышает её энергию покоя.
    Определить отношение скорости частицы к скорости света.
}
\answer{%
    \begin{align*}
    E &= E_0 + E_{\text{кин}} \\
    E &= \frac{E_0}{\sqrt{1 - \frac{v^2}{c^2}}}\implies \sqrt{1 - \frac{v^2}{c^2}} = \frac{E_0}{E}\implies \frac{v^2}{c^2} = 1 - \sqr{\frac{E_0}{E}} \implies \\
    \implies \frac vc &= \sqrt{1 - \sqr{\frac{E_0}{E}}} = \sqrt{1 - \sqr{\frac{E_0}{E_0 + E_{\text{кин}} }}} \approx 0{,}980.
    \end{align*}
}
\solutionspace{80pt}

\tasknumber{8}%
\task{%
    Некоторая частица, пройдя ускоряющую разность потенциалов, приобрела импульс $4{,}2 \cdot 10^{-19}\,\frac{\text{кг}\cdot\text{м}}{\text{с}}$.
    Скорость частицы стала равной $2{,}4 \cdot 10^{8}\,\frac{\text{м}}{\text{с}}$.
    Найти массу частицы.
}
\answer{%
    $p = \frac{ mv }{\sqrt{1 - \frac{v^2}{c^2} }}\implies m = \frac pv \sqrt{1 - \frac{v^2}{c^2}}= \frac {4{,}2 \cdot 10^{-19}\,\frac{\text{кг}\cdot\text{м}}{\text{с}}}{2{,}4 \cdot 10^{8}\,\frac{\text{м}}{\text{с}}} \sqrt{1 - \sqr{\frac{2{,}4 \cdot 10^{8}\,\frac{\text{м}}{\text{с}}}{3 \cdot 10^{8}\,\frac{\text{м}}{\text{с}}}} } \approx 1{,}050 \cdot 10^{-27}\,\text{кг}.$
}
\solutionspace{80pt}

\tasknumber{9}%
\task{%
    При какой скорости движения (в км/ч) релятивистское сокращение длины движущегося тела
    составит 50\%?
}
\answer{%
    \begin{align*}
    l_0 &= \frac l{\sqrt{1 - \frac{v^2}{c^2}}}
        \implies 1 - \frac{v^2}{c^2} = \sqr{\frac l{l_0}}
        \implies \frac v c = \sqrt{1 - \sqr{\frac l{l_0}}} \implies
         \\
        \implies v &= c\sqrt{1 - \sqr{\frac l{l_0}}}
        = 3 \cdot 10^{8}\,\frac{\text{м}}{\text{с}} \cdot \sqrt{1 - \sqr{\frac {l_0 - 0{,}50l_0}{l_0}}}
        = 3 \cdot 10^{8}\,\frac{\text{м}}{\text{с}} \cdot \sqrt{1 - \sqr{1 - 0{,}50}} \approx  \\
        &\approx 0{,}866c
        \approx 260 \cdot 10^{6}\,\frac{\text{м}}{\text{с}}
        \approx 935 \cdot 10^{6}\,\frac{\text{км}}{\text{ч}}.
    \end{align*}
}
\solutionspace{80pt}

\tasknumber{10}%
\task{%
    Стержень движется в продольном направлении с постоянной скоростью относительно инерциальной системы отсчёта.
    При каком значении скорости (в долях скорости света) длина стержня в этой системе отсчёта
    будет в  1{,}25  раза меньше его собственной длины?
}
\answer{%
    $l_0 = \frac l{\sqrt{1 - \frac{v^2}{c^2}}}\implies \sqrt{1 - \frac{v^2}{c^2}} = \frac{ l }{ l_0 }\implies \frac v c = \sqrt{1 - \sqr{\frac{ l }{ l_0 }}} \approx 0{,}600.$
}
\solutionspace{80pt}

\tasknumber{11}%
\task{%
    Какую скорость должно иметь движущееся тело, чтобы его продольные размеры уменьшились в четыре раза?
    Скорость света $c = 3 \cdot 10^{8}\,\frac{\text{м}}{\text{с}}$.
}
\answer{%
    $l_0 = \frac l{\sqrt{1 - \frac{v^2}{c^2}}}\implies \sqrt{1 - \frac{v^2}{c^2}} = \frac{ l }{ l_0 }\implies v = c\sqrt{1 - \sqr{\frac{ l }{ l_0 }}} \approx 290 \cdot 10^{6}\,\frac{\text{м}}{\text{с}}.$
}


\variantsplitter


\addpersonalvariant{Леонид Никитин}

\tasknumber{12}%
\task{%
    Время жизни мюона, измеренное наблюдателем, относительно которого мюон покоился, равно $\tau_0$
    Какое расстояние пролетит мюон в системе отсчёта, относительно которой он движется со скоростью $v$,
    сравнимой со скоростью света в вакууме $c$?
}
\answer{%
    $\ell = v\tau = v \frac{\tau_0}{\sqrt{1 - \frac{v^2}{c^2}}}$
}
\solutionspace{80pt}

\tasknumber{13}%
\task{%
    Если $c$ — скорость света в вакууме, то с какой скоростью должна двигаться нестабильная частица относительно наблюдателя,
    чтобы её время жизни было в восемь раз больше, чем у такой же, но покоящейся относительно наблюдателя частицы?
}
\answer{%
    $\tau = \frac{\tau_0}{\sqrt{1 - \frac{v^2}{c^2}}}\implies \sqrt{1 - \frac{v^2}{c^2}} = \frac{\tau_0}{\tau}\implies v = c\sqrt{1 - \sqr{\frac{\tau_0}{\tau}} } \approx 298 \cdot 10^{6}\,\frac{\text{м}}{\text{с}}.$
}
\solutionspace{80pt}

\tasknumber{14}%
\task{%
    Время жизни нестабильной частицы, входящего в состав космических лучей, измеренное земным наблюдателем,
    относительно которого частица двигалась со скоростью, составляющей 85\% скорости света в вакууме, оказалось равным $7{,}1\,\text{мкс}$.
    Каково время жизни частицы, покоящейся относительно наблюдателя?
}
\answer{%
    $t = \frac{t_0}{\sqrt{1 - \frac{v^2}{c^2}}}\implies t_0 = t\sqrt{1 - \frac{v^2}{c^2}} \approx 3{,}7 \cdot 10^{-6}\,\text{с}.$
}
\solutionspace{80pt}

\tasknumber{15}%
\task{%
    Частица увеличила в ускорителе свою скорость с $0{,}05c$ до $0{,}60c$.
    Во сколько раз выросла её кинетическая энергия?
}
\answer{%
    \begin{align*}
    E_{\text{кин.}} &= E - E_0 = \frac{mc^2}{\sqrt{1 - \frac{v^2}{c^2}}} - mc^2 = mc^2\cbr{ \frac1{\sqrt{1 - \frac{v^2}{c^2} }} - 1}.
    \\
    \frac{E_{\text{кин.
    2}}}{E_{\text{кин.
    1}}} &= \frac{\frac1{\sqrt{1 - \frac{v_2^2}{c^2} }} - 1}{\frac1{\sqrt{1 - \frac{v_1^2}{c^2} }} - 1}\approx 199{,}62
    \end{align*}
}
\solutionspace{120pt}

\tasknumber{16}%
\task{%
    Для частицы, движущейся с релятивистской скоростью,
    выразите $E_0$ и $v$ через $c$, $E_\text{кин}$ и $p$,
    где $E_\text{кин}$~--- кинетическая энергия частицы,
    а $E_0$, $p$ и $v$~--- её энергия покоя, импульс и скорость.
}
\answer{%
    \begin{align*}
    E_\text{кин}, E_0:\quad&E = E_\text{кин} + E_0 = \frac{E_0}{\sqrt{1 - \frac{v^2}{c^2}}} \implies \sqrt{1 - \frac{v^2}{c^2}} = \frac{E_0}{{E_0} + {E_\text{кин}}} \implies v = c\sqrt{1 - \sqr{\frac{E_0}{{E_0} + {E_\text{кин}}}}} \\
    &p = \frac{mv}{\sqrt{1 - \frac{v^2}{c^2}}} = \frac{E_0}{c^2} \cdot \sqrt{1 - \sqr{\frac{E_0}{{E_0} + {E_\text{кин}}}}} \cdot \frac{{E_\text{кин}} + {E_0}}{E_0} = \frac{E_0}{c^2} \cdot \sqrt{\sqr{\frac{{E_\text{кин}} + {E_0}}{E_0}} - 1}.
    \\
    E_\text{кин}, p:\quad&E_\text{кин} = E - E_0 = mc^2\cbr{\frac 1{\sqrt{1 - \frac{v^2}{c^2}}} - 1}, p = \frac{mv}{\sqrt{1 - \frac{v^2}{c^2}}} \implies \frac{E_\text{кин}}{p} = \frac{\frac 1{\sqrt{1 - \frac{v^2}{c^2}}} - 1}{\sqrt{1 - \frac{v^2}{c^2}}} \implies v = \ldots \\
    &E_0 = E - E_\text{кин} = \frac{E_0}{\sqrt{1 - \frac{v^2}{c^2}}} - E_\text{кин} \implies E_0 = \frac{E_\text{кин}}{\frac 1{\sqrt{1 - \frac{v^2}{c^2}}} - 1} = \ldots \\
    E_\text{кин}, v:\quad&E_\text{кин} = E - E_0 = mc^2\cbr{\frac 1{\sqrt{1 - \frac{v^2}{c^2}}} - 1} \implies m = \frac{E_\text{кин}}{c^2\cbr{\frac 1{\sqrt{1 - \frac{v^2}{c^2}}} - 1}} \\
    &E_0 = mc^2 = \frac{E_\text{кин}}{\frac 1{\sqrt{1 - \frac{v^2}{c^2}}} - 1} \\
    &p = \frac{mv}{\sqrt{1 - \frac{v^2}{c^2}}} = \frac{E_\text{кин}}{c^2\cbr{\frac 1{\sqrt{1 - \frac{v^2}{c^2}}} - 1}} \cdot \frac{v}{\sqrt{1 - \frac{v^2}{c^2}}} = \frac{{E_\text{кин}} v}{c^2\cbr{1 - {\sqrt{1 - \frac{v^2}{c^2}}}}} \\
    E_0, p:\quad&E_0 = mc^2, \quad p = \frac{mv}{\sqrt{1 - \frac{v^2}{c^2}}} \implies \frac{E_0}{p} = \frac{c^2}v{\sqrt{1 - \frac{v^2}{c^2}}} = c\sqrt{\frac{c^2}{v^2} - 1} \\
    &\sqr{\frac{E_0}{pc}} = \frac{c^2}{v^2} - 1 \implies \frac{v^2}{c^2} = \frac 1{1 + \frac{E_0^2}{p^2c^2}} \implies v = \frac c{\sqrt{1 + \frac{E_0^2}{p^2c^2}}} \\
    &{E_\text{кин}} = E - E_0 = \sqrt{E_0^2 + p^2c^2} - E_0 \\
    E_0, v:\quad&E_0 = mc^2 \implies m = \frac{E_0}{c^2} \qquad p = \frac{mv}{\sqrt{1 - \frac{v^2}{c^2}}} = \frac{E_0}{c^2} \cdot \frac{v}{\sqrt{1 - \frac{v^2}{c^2}}} \\
    &E_\text{кин}= mc^2\cbr{\frac 1{\sqrt{1 - \frac{v^2}{c^2}}} - 1} = \frac{E_0}{c^2}\cbr{\frac 1{\sqrt{1 - \frac{v^2}{c^2}}} - 1} \\
    p, v:\quad&p = \frac{mv}{\sqrt{1 - \frac{v^2}{c^2}}} \implies m = \frac p v {\sqrt{1 - \frac{v^2}{c^2}}} \implies E_0 = mc^2 =\frac {pc^2} v {\sqrt{1 - \frac{v^2}{c^2}}} \\
    &E_\text{кин} = mc^2\cbr{\frac 1{\sqrt{1 - \frac{v^2}{c^2}}} - 1} = \frac p v {\sqrt{1 - \frac{v^2}{c^2}}}\cbr{\frac 1{\sqrt{1 - \frac{v^2}{c^2}}} - 1} = \frac p v \cbr{1 - {\sqrt{1 - \frac{v^2}{c^2}}}}
    \end{align*}
}

\variantsplitter

\addpersonalvariant{Тимофей Полетаев}

\tasknumber{1}%
\task{%
    Запишите
    \begin{itemize}
        \item постулаты специальной теории относительности,
        \item пример релятивистского эффекта, обнаружимый при скоростях гораздо меньше скорости света.
    \end{itemize}
}
\solutionspace{120pt}

\tasknumber{2}%
\task{%
    Запишите формулу для ...
    \begin{itemize}
        \item релятивистского замедления времени,
        \item классического импульса,
        \item релятивистского импульса тела,
        \item релятивистской кинетической энергии,
        \item связь между релятивистским импульсом и релятивистской энергией.
    \end{itemize}
    Обязательно подпишите все физические величины.
}
\solutionspace{150pt}

\tasknumber{3}%
\task{%
    Позитрон движется со скоростью $0{,}9\,c$, где $c$~--- скорость света в вакууме.
    Каково при этом отношение полной энергии частицы $E$ к его энергии покоя $E_0$?
}
\answer{%
    \begin{align*}
    E &= \frac{E_0}{\sqrt{1 - \frac{v^2}{c^2}}}
            \implies \frac E{E_0}
                = \frac 1{\sqrt{1 - \frac{v^2}{c^2}}}
                = \frac 1{\sqrt{1 - \sqr{0{,}9}}}
                \approx 2{,}294,
         \\
        {E_{\text{кин}}} &= E - E_0
            \implies \frac{E_{\text{кин}}}{E_0}
                = \frac E{E_0} - 1
                = \frac 1{\sqrt{1 - \frac{v^2}{c^2}}} - 1
                = \frac 1{\sqrt{1 - \sqr{0{,}9}}} - 1
                \approx 1{,}294.
    \end{align*}
}
\solutionspace{80pt}

\tasknumber{4}%
\task{%
    Полная энергия релятивистской частицы в четыре раза больше её энергии покоя.
    Найти скорость этой частицы: в долях $c$ и численное значение.
    Скорость света в вакууме $c = 3 \cdot 10^{8}\,\frac{\text{м}}{\text{с}}$.
}
\answer{%
    \begin{align*}
    E &= \frac{E_0}{\sqrt{1 - \frac{v^2}{c^2}}}\implies \sqrt{1 - \frac{v^2}{c^2}} = \frac{E_0}{E}\implies \frac{v^2}{c^2} = 1 - \sqr{\frac{E_0}{E}}\implies v = c \sqrt{1 - \sqr{\frac{E_0}{E}}} \approx 0{,}968c \approx 290 \cdot 10^{6}\,\frac{\text{м}}{\text{с}}.
    \end{align*}
}
\solutionspace{80pt}

\tasknumber{5}%
\task{%
    Кинетическая энергия релятивистской частицы в четыре раза больше её энергии покоя.
    Найти скорость этой частицы.
    Скорость света в вакууме $c = 3 \cdot 10^{8}\,\frac{\text{м}}{\text{с}}$.
}
\answer{%
    \begin{align*}
    E &= E_0 + E_{\text{кин}} \\
    E &= \frac{E_0}{\sqrt{1 - \frac{v^2}{c^2}}}\implies \sqrt{1 - \frac{v^2}{c^2}} = \frac{E_0}{E}\implies \frac{v^2}{c^2} = 1 - \sqr{\frac{E_0}{E}} \implies \\
    \implies &v = c \sqrt{1 - \sqr{\frac{E_0}{E}}} = c \sqrt{1 - \sqr{\frac{E_0}{E_0 + E_{\text{кин}} }}} = c \sqrt{1 - \frac 1 {\sqr{ 1 + \frac{E_{\text{кин}}}{E_0} }} }\approx 0{,}980c \approx 294 \cdot 10^{6}\,\frac{\text{м}}{\text{с}}.
    \end{align*}
}


\variantsplitter


\addpersonalvariant{Тимофей Полетаев}

\tasknumber{6}%
\task{%
    Протон движется со скоростью $0{,}75\,c$, где $c$~--- скорость света в вакууме.
    Определите его импульс (в ответе приведите формулу и укажите численное значение).
}
\answer{%
    \begin{align*}
    E &= \frac{mc^2}{\sqrt{1 - \frac{v^2}{c^2}}}
            \approx \frac{1{,}673 \cdot 10^{-27}\,\text{кг} \cdot \sqr{3 \cdot 10^{8}\,\frac{\text{м}}{\text{с}}}}{\sqrt{1 - 0{,}75^2}}
            \approx 0{,}2276 \cdot 10^{-9}\,\text{Дж},
         \\
        E_{\text{кин}} &= \frac{mc^2}{\sqrt{1 - \frac{v^2}{c^2}}} - mc^2
            = mc^2 \cbr{\frac 1{\sqrt{1 - \frac{v^2}{c^2}}} - 1} \approx \\
            &\approx \cbr{1{,}673 \cdot 10^{-27}\,\text{кг} \cdot \sqr{3 \cdot 10^{8}\,\frac{\text{м}}{\text{с}}}}
            \cdot \cbr{\frac 1{\sqrt{1 - 0{,}75^2}} - 1}
            \approx 77{,}05 \cdot 10^{-12}\,\text{Дж},
         \\
        p &= \frac{mv}{\sqrt{1 - \frac{v^2}{c^2}}}
            \approx \frac{1{,}673 \cdot 10^{-27}\,\text{кг} \cdot 0{,}75 \cdot 3 \cdot 10^{8}\,\frac{\text{м}}{\text{с}}}{\sqrt{1 - 0{,}75^2}}
            \approx 0{,}5690 \cdot 10^{-18}\,\frac{\text{кг}\cdot\text{м}}{\text{с}}.
    \end{align*}
}
\solutionspace{100pt}

\tasknumber{7}%
\task{%
    Кинетическая энергия частицы космических лучей в четыре раза превышает её энергию покоя.
    Определить отношение скорости частицы к скорости света.
}
\answer{%
    \begin{align*}
    E &= E_0 + E_{\text{кин}} \\
    E &= \frac{E_0}{\sqrt{1 - \frac{v^2}{c^2}}}\implies \sqrt{1 - \frac{v^2}{c^2}} = \frac{E_0}{E}\implies \frac{v^2}{c^2} = 1 - \sqr{\frac{E_0}{E}} \implies \\
    \implies \frac vc &= \sqrt{1 - \sqr{\frac{E_0}{E}}} = \sqrt{1 - \sqr{\frac{E_0}{E_0 + E_{\text{кин}} }}} \approx 0{,}980.
    \end{align*}
}
\solutionspace{80pt}

\tasknumber{8}%
\task{%
    Некоторая частица, пройдя ускоряющую разность потенциалов, приобрела импульс $3 \cdot 10^{-19}\,\frac{\text{кг}\cdot\text{м}}{\text{с}}$.
    Скорость частицы стала равной $2 \cdot 10^{8}\,\frac{\text{м}}{\text{с}}$.
    Найти массу частицы.
}
\answer{%
    $p = \frac{ mv }{\sqrt{1 - \frac{v^2}{c^2} }}\implies m = \frac pv \sqrt{1 - \frac{v^2}{c^2}}= \frac {3 \cdot 10^{-19}\,\frac{\text{кг}\cdot\text{м}}{\text{с}}}{2 \cdot 10^{8}\,\frac{\text{м}}{\text{с}}} \sqrt{1 - \sqr{\frac{2 \cdot 10^{8}\,\frac{\text{м}}{\text{с}}}{3 \cdot 10^{8}\,\frac{\text{м}}{\text{с}}}} } \approx 1{,}12 \cdot 10^{-27}\,\text{кг}.$
}
\solutionspace{80pt}

\tasknumber{9}%
\task{%
    При какой скорости движения (в км/ч) релятивистское сокращение длины движущегося тела
    составит 50\%?
}
\answer{%
    \begin{align*}
    l_0 &= \frac l{\sqrt{1 - \frac{v^2}{c^2}}}
        \implies 1 - \frac{v^2}{c^2} = \sqr{\frac l{l_0}}
        \implies \frac v c = \sqrt{1 - \sqr{\frac l{l_0}}} \implies
         \\
        \implies v &= c\sqrt{1 - \sqr{\frac l{l_0}}}
        = 3 \cdot 10^{8}\,\frac{\text{м}}{\text{с}} \cdot \sqrt{1 - \sqr{\frac {l_0 - 0{,}50l_0}{l_0}}}
        = 3 \cdot 10^{8}\,\frac{\text{м}}{\text{с}} \cdot \sqrt{1 - \sqr{1 - 0{,}50}} \approx  \\
        &\approx 0{,}866c
        \approx 260 \cdot 10^{6}\,\frac{\text{м}}{\text{с}}
        \approx 935 \cdot 10^{6}\,\frac{\text{км}}{\text{ч}}.
    \end{align*}
}
\solutionspace{80pt}

\tasknumber{10}%
\task{%
    Стержень движется в продольном направлении с постоянной скоростью относительно инерциальной системы отсчёта.
    При каком значении скорости (в долях скорости света) длина стержня в этой системе отсчёта
    будет в  4  раза меньше его собственной длины?
}
\answer{%
    $l_0 = \frac l{\sqrt{1 - \frac{v^2}{c^2}}}\implies \sqrt{1 - \frac{v^2}{c^2}} = \frac{ l }{ l_0 }\implies \frac v c = \sqrt{1 - \sqr{\frac{ l }{ l_0 }}} \approx 0{,}968.$
}
\solutionspace{80pt}

\tasknumber{11}%
\task{%
    Какую скорость должно иметь движущееся тело, чтобы его продольные размеры уменьшились в шесть раз?
    Скорость света $c = 3 \cdot 10^{8}\,\frac{\text{м}}{\text{с}}$.
}
\answer{%
    $l_0 = \frac l{\sqrt{1 - \frac{v^2}{c^2}}}\implies \sqrt{1 - \frac{v^2}{c^2}} = \frac{ l }{ l_0 }\implies v = c\sqrt{1 - \sqr{\frac{ l }{ l_0 }}} \approx 296 \cdot 10^{6}\,\frac{\text{м}}{\text{с}}.$
}


\variantsplitter


\addpersonalvariant{Тимофей Полетаев}

\tasknumber{12}%
\task{%
    Время жизни мюона, измеренное наблюдателем, относительно которого мюон покоился, равно $\tau_0$
    Какое расстояние пролетит мюон в системе отсчёта, относительно которой он движется со скоростью $v$,
    сравнимой со скоростью света в вакууме $c$?
}
\answer{%
    $\ell = v\tau = v \frac{\tau_0}{\sqrt{1 - \frac{v^2}{c^2}}}$
}
\solutionspace{80pt}

\tasknumber{13}%
\task{%
    Если $c$ — скорость света в вакууме, то с какой скоростью должна двигаться нестабильная частица относительно наблюдателя,
    чтобы её время жизни было в восемь раз больше, чем у такой же, но покоящейся относительно наблюдателя частицы?
}
\answer{%
    $\tau = \frac{\tau_0}{\sqrt{1 - \frac{v^2}{c^2}}}\implies \sqrt{1 - \frac{v^2}{c^2}} = \frac{\tau_0}{\tau}\implies v = c\sqrt{1 - \sqr{\frac{\tau_0}{\tau}} } \approx 298 \cdot 10^{6}\,\frac{\text{м}}{\text{с}}.$
}
\solutionspace{80pt}

\tasknumber{14}%
\task{%
    Время жизни нестабильной частицы, входящего в состав космических лучей, измеренное земным наблюдателем,
    относительно которого частица двигалась со скоростью, составляющей 65\% скорости света в вакууме, оказалось равным $6{,}4\,\text{мкс}$.
    Каково время жизни частицы, покоящейся относительно наблюдателя?
}
\answer{%
    $t = \frac{t_0}{\sqrt{1 - \frac{v^2}{c^2}}}\implies t_0 = t\sqrt{1 - \frac{v^2}{c^2}} \approx 4{,}9 \cdot 10^{-6}\,\text{с}.$
}
\solutionspace{80pt}

\tasknumber{15}%
\task{%
    Частица увеличила в ускорителе свою скорость с $0{,}05c$ до $0{,}90c$.
    Во сколько раз выросла её кинетическая энергия?
}
\answer{%
    \begin{align*}
    E_{\text{кин.}} &= E - E_0 = \frac{mc^2}{\sqrt{1 - \frac{v^2}{c^2}}} - mc^2 = mc^2\cbr{ \frac1{\sqrt{1 - \frac{v^2}{c^2} }} - 1}.
    \\
    \frac{E_{\text{кин.
    2}}}{E_{\text{кин.
    1}}} &= \frac{\frac1{\sqrt{1 - \frac{v_2^2}{c^2} }} - 1}{\frac1{\sqrt{1 - \frac{v_1^2}{c^2} }} - 1}\approx 1033{,}38
    \end{align*}
}
\solutionspace{120pt}

\tasknumber{16}%
\task{%
    Для частицы, движущейся с релятивистской скоростью,
    выразите $v$ и $E_\text{кин}$ через $c$, $E_0$ и $p$,
    где $E_\text{кин}$~--- кинетическая энергия частицы,
    а $E_0$, $p$ и $v$~--- её энергия покоя, импульс и скорость.
}
\answer{%
    \begin{align*}
    E_\text{кин}, E_0:\quad&E = E_\text{кин} + E_0 = \frac{E_0}{\sqrt{1 - \frac{v^2}{c^2}}} \implies \sqrt{1 - \frac{v^2}{c^2}} = \frac{E_0}{{E_0} + {E_\text{кин}}} \implies v = c\sqrt{1 - \sqr{\frac{E_0}{{E_0} + {E_\text{кин}}}}} \\
    &p = \frac{mv}{\sqrt{1 - \frac{v^2}{c^2}}} = \frac{E_0}{c^2} \cdot \sqrt{1 - \sqr{\frac{E_0}{{E_0} + {E_\text{кин}}}}} \cdot \frac{{E_\text{кин}} + {E_0}}{E_0} = \frac{E_0}{c^2} \cdot \sqrt{\sqr{\frac{{E_\text{кин}} + {E_0}}{E_0}} - 1}.
    \\
    E_\text{кин}, p:\quad&E_\text{кин} = E - E_0 = mc^2\cbr{\frac 1{\sqrt{1 - \frac{v^2}{c^2}}} - 1}, p = \frac{mv}{\sqrt{1 - \frac{v^2}{c^2}}} \implies \frac{E_\text{кин}}{p} = \frac{\frac 1{\sqrt{1 - \frac{v^2}{c^2}}} - 1}{\sqrt{1 - \frac{v^2}{c^2}}} \implies v = \ldots \\
    &E_0 = E - E_\text{кин} = \frac{E_0}{\sqrt{1 - \frac{v^2}{c^2}}} - E_\text{кин} \implies E_0 = \frac{E_\text{кин}}{\frac 1{\sqrt{1 - \frac{v^2}{c^2}}} - 1} = \ldots \\
    E_\text{кин}, v:\quad&E_\text{кин} = E - E_0 = mc^2\cbr{\frac 1{\sqrt{1 - \frac{v^2}{c^2}}} - 1} \implies m = \frac{E_\text{кин}}{c^2\cbr{\frac 1{\sqrt{1 - \frac{v^2}{c^2}}} - 1}} \\
    &E_0 = mc^2 = \frac{E_\text{кин}}{\frac 1{\sqrt{1 - \frac{v^2}{c^2}}} - 1} \\
    &p = \frac{mv}{\sqrt{1 - \frac{v^2}{c^2}}} = \frac{E_\text{кин}}{c^2\cbr{\frac 1{\sqrt{1 - \frac{v^2}{c^2}}} - 1}} \cdot \frac{v}{\sqrt{1 - \frac{v^2}{c^2}}} = \frac{{E_\text{кин}} v}{c^2\cbr{1 - {\sqrt{1 - \frac{v^2}{c^2}}}}} \\
    E_0, p:\quad&E_0 = mc^2, \quad p = \frac{mv}{\sqrt{1 - \frac{v^2}{c^2}}} \implies \frac{E_0}{p} = \frac{c^2}v{\sqrt{1 - \frac{v^2}{c^2}}} = c\sqrt{\frac{c^2}{v^2} - 1} \\
    &\sqr{\frac{E_0}{pc}} = \frac{c^2}{v^2} - 1 \implies \frac{v^2}{c^2} = \frac 1{1 + \frac{E_0^2}{p^2c^2}} \implies v = \frac c{\sqrt{1 + \frac{E_0^2}{p^2c^2}}} \\
    &{E_\text{кин}} = E - E_0 = \sqrt{E_0^2 + p^2c^2} - E_0 \\
    E_0, v:\quad&E_0 = mc^2 \implies m = \frac{E_0}{c^2} \qquad p = \frac{mv}{\sqrt{1 - \frac{v^2}{c^2}}} = \frac{E_0}{c^2} \cdot \frac{v}{\sqrt{1 - \frac{v^2}{c^2}}} \\
    &E_\text{кин}= mc^2\cbr{\frac 1{\sqrt{1 - \frac{v^2}{c^2}}} - 1} = \frac{E_0}{c^2}\cbr{\frac 1{\sqrt{1 - \frac{v^2}{c^2}}} - 1} \\
    p, v:\quad&p = \frac{mv}{\sqrt{1 - \frac{v^2}{c^2}}} \implies m = \frac p v {\sqrt{1 - \frac{v^2}{c^2}}} \implies E_0 = mc^2 =\frac {pc^2} v {\sqrt{1 - \frac{v^2}{c^2}}} \\
    &E_\text{кин} = mc^2\cbr{\frac 1{\sqrt{1 - \frac{v^2}{c^2}}} - 1} = \frac p v {\sqrt{1 - \frac{v^2}{c^2}}}\cbr{\frac 1{\sqrt{1 - \frac{v^2}{c^2}}} - 1} = \frac p v \cbr{1 - {\sqrt{1 - \frac{v^2}{c^2}}}}
    \end{align*}
}

\variantsplitter

\addpersonalvariant{Андрей Рожков}

\tasknumber{1}%
\task{%
    Запишите
    \begin{itemize}
        \item постулаты специальной теории относительности,
        \item пример релятивистского эффекта, обнаружимый при скоростях гораздо меньше скорости света.
    \end{itemize}
}
\solutionspace{120pt}

\tasknumber{2}%
\task{%
    Запишите формулу для ...
    \begin{itemize}
        \item релятивистского сжатия,
        \item классической полной механической энергии тела,
        \item релятивистского импульса тела,
        \item энергии покоя тела,
        \item связь между релятивистским импульсом и релятивистской энергией.
    \end{itemize}
    Обязательно подпишите все физические величины.
}
\solutionspace{150pt}

\tasknumber{3}%
\task{%
    Электрон движется со скоростью $0{,}7\,c$, где $c$~--- скорость света в вакууме.
    Каково при этом отношение кинетической энергии частицы $E_\text{кин.}$ к его энергии покоя $E_0$?
}
\answer{%
    \begin{align*}
    E &= \frac{E_0}{\sqrt{1 - \frac{v^2}{c^2}}}
            \implies \frac E{E_0}
                = \frac 1{\sqrt{1 - \frac{v^2}{c^2}}}
                = \frac 1{\sqrt{1 - \sqr{0{,}7}}}
                \approx 1{,}400,
         \\
        {E_{\text{кин}}} &= E - E_0
            \implies \frac{E_{\text{кин}}}{E_0}
                = \frac E{E_0} - 1
                = \frac 1{\sqrt{1 - \frac{v^2}{c^2}}} - 1
                = \frac 1{\sqrt{1 - \sqr{0{,}7}}} - 1
                \approx 0{,}400.
    \end{align*}
}
\solutionspace{80pt}

\tasknumber{4}%
\task{%
    Полная энергия релятивистской частицы в три раза больше её энергии покоя.
    Найти скорость этой частицы: в долях $c$ и численное значение.
    Скорость света в вакууме $c = 3 \cdot 10^{8}\,\frac{\text{м}}{\text{с}}$.
}
\answer{%
    \begin{align*}
    E &= \frac{E_0}{\sqrt{1 - \frac{v^2}{c^2}}}\implies \sqrt{1 - \frac{v^2}{c^2}} = \frac{E_0}{E}\implies \frac{v^2}{c^2} = 1 - \sqr{\frac{E_0}{E}}\implies v = c \sqrt{1 - \sqr{\frac{E_0}{E}}} \approx 0{,}943c \approx 283 \cdot 10^{6}\,\frac{\text{м}}{\text{с}}.
    \end{align*}
}
\solutionspace{80pt}

\tasknumber{5}%
\task{%
    Кинетическая энергия релятивистской частицы в три раза больше её энергии покоя.
    Найти скорость этой частицы.
    Скорость света в вакууме $c = 3 \cdot 10^{8}\,\frac{\text{м}}{\text{с}}$.
}
\answer{%
    \begin{align*}
    E &= E_0 + E_{\text{кин}} \\
    E &= \frac{E_0}{\sqrt{1 - \frac{v^2}{c^2}}}\implies \sqrt{1 - \frac{v^2}{c^2}} = \frac{E_0}{E}\implies \frac{v^2}{c^2} = 1 - \sqr{\frac{E_0}{E}} \implies \\
    \implies &v = c \sqrt{1 - \sqr{\frac{E_0}{E}}} = c \sqrt{1 - \sqr{\frac{E_0}{E_0 + E_{\text{кин}} }}} = c \sqrt{1 - \frac 1 {\sqr{ 1 + \frac{E_{\text{кин}}}{E_0} }} }\approx 0{,}968c \approx 290 \cdot 10^{6}\,\frac{\text{м}}{\text{с}}.
    \end{align*}
}


\variantsplitter


\addpersonalvariant{Андрей Рожков}

\tasknumber{6}%
\task{%
    Протон движется со скоростью $0{,}65\,c$, где $c$~--- скорость света в вакууме.
    Определите его импульс (в ответе приведите формулу и укажите численное значение).
}
\answer{%
    \begin{align*}
    E &= \frac{mc^2}{\sqrt{1 - \frac{v^2}{c^2}}}
            \approx \frac{1{,}673 \cdot 10^{-27}\,\text{кг} \cdot \sqr{3 \cdot 10^{8}\,\frac{\text{м}}{\text{с}}}}{\sqrt{1 - 0{,}65^2}}
            \approx 0{,}19809 \cdot 10^{-9}\,\text{Дж},
         \\
        E_{\text{кин}} &= \frac{mc^2}{\sqrt{1 - \frac{v^2}{c^2}}} - mc^2
            = mc^2 \cbr{\frac 1{\sqrt{1 - \frac{v^2}{c^2}}} - 1} \approx \\
            &\approx \cbr{1{,}673 \cdot 10^{-27}\,\text{кг} \cdot \sqr{3 \cdot 10^{8}\,\frac{\text{м}}{\text{с}}}}
            \cdot \cbr{\frac 1{\sqrt{1 - 0{,}65^2}} - 1}
            \approx 47{,}55 \cdot 10^{-12}\,\text{Дж},
         \\
        p &= \frac{mv}{\sqrt{1 - \frac{v^2}{c^2}}}
            \approx \frac{1{,}673 \cdot 10^{-27}\,\text{кг} \cdot 0{,}65 \cdot 3 \cdot 10^{8}\,\frac{\text{м}}{\text{с}}}{\sqrt{1 - 0{,}65^2}}
            \approx 0{,}4292 \cdot 10^{-18}\,\frac{\text{кг}\cdot\text{м}}{\text{с}}.
    \end{align*}
}
\solutionspace{100pt}

\tasknumber{7}%
\task{%
    Кинетическая энергия частицы космических лучей в три раза превышает её энергию покоя.
    Определить отношение скорости частицы к скорости света.
}
\answer{%
    \begin{align*}
    E &= E_0 + E_{\text{кин}} \\
    E &= \frac{E_0}{\sqrt{1 - \frac{v^2}{c^2}}}\implies \sqrt{1 - \frac{v^2}{c^2}} = \frac{E_0}{E}\implies \frac{v^2}{c^2} = 1 - \sqr{\frac{E_0}{E}} \implies \\
    \implies \frac vc &= \sqrt{1 - \sqr{\frac{E_0}{E}}} = \sqrt{1 - \sqr{\frac{E_0}{E_0 + E_{\text{кин}} }}} \approx 0{,}968.
    \end{align*}
}
\solutionspace{80pt}

\tasknumber{8}%
\task{%
    Некоторая частица, пройдя ускоряющую разность потенциалов, приобрела импульс $3{,}5 \cdot 10^{-19}\,\frac{\text{кг}\cdot\text{м}}{\text{с}}$.
    Скорость частицы стала равной $1{,}8 \cdot 10^{8}\,\frac{\text{м}}{\text{с}}$.
    Найти массу частицы.
}
\answer{%
    $p = \frac{ mv }{\sqrt{1 - \frac{v^2}{c^2} }}\implies m = \frac pv \sqrt{1 - \frac{v^2}{c^2}}= \frac {3{,}5 \cdot 10^{-19}\,\frac{\text{кг}\cdot\text{м}}{\text{с}}}{1{,}8 \cdot 10^{8}\,\frac{\text{м}}{\text{с}}} \sqrt{1 - \sqr{\frac{1{,}8 \cdot 10^{8}\,\frac{\text{м}}{\text{с}}}{3 \cdot 10^{8}\,\frac{\text{м}}{\text{с}}}} } \approx 1{,}56 \cdot 10^{-27}\,\text{кг}.$
}
\solutionspace{80pt}

\tasknumber{9}%
\task{%
    При какой скорости движения (в м/с) релятивистское сокращение длины движущегося тела
    составит 50\%?
}
\answer{%
    \begin{align*}
    l_0 &= \frac l{\sqrt{1 - \frac{v^2}{c^2}}}
        \implies 1 - \frac{v^2}{c^2} = \sqr{\frac l{l_0}}
        \implies \frac v c = \sqrt{1 - \sqr{\frac l{l_0}}} \implies
         \\
        \implies v &= c\sqrt{1 - \sqr{\frac l{l_0}}}
        = 3 \cdot 10^{8}\,\frac{\text{м}}{\text{с}} \cdot \sqrt{1 - \sqr{\frac {l_0 - 0{,}50l_0}{l_0}}}
        = 3 \cdot 10^{8}\,\frac{\text{м}}{\text{с}} \cdot \sqrt{1 - \sqr{1 - 0{,}50}} \approx  \\
        &\approx 0{,}866c
        \approx 260 \cdot 10^{6}\,\frac{\text{м}}{\text{с}}
        \approx 935 \cdot 10^{6}\,\frac{\text{км}}{\text{ч}}.
    \end{align*}
}
\solutionspace{80pt}

\tasknumber{10}%
\task{%
    Стержень движется в продольном направлении с постоянной скоростью относительно инерциальной системы отсчёта.
    При каком значении скорости (в долях скорости света) длина стержня в этой системе отсчёта
    будет в  1{,}25  раза меньше его собственной длины?
}
\answer{%
    $l_0 = \frac l{\sqrt{1 - \frac{v^2}{c^2}}}\implies \sqrt{1 - \frac{v^2}{c^2}} = \frac{ l }{ l_0 }\implies \frac v c = \sqrt{1 - \sqr{\frac{ l }{ l_0 }}} \approx 0{,}600.$
}
\solutionspace{80pt}

\tasknumber{11}%
\task{%
    Какую скорость должно иметь движущееся тело, чтобы его продольные размеры уменьшились в пять раз?
    Скорость света $c = 3 \cdot 10^{8}\,\frac{\text{м}}{\text{с}}$.
}
\answer{%
    $l_0 = \frac l{\sqrt{1 - \frac{v^2}{c^2}}}\implies \sqrt{1 - \frac{v^2}{c^2}} = \frac{ l }{ l_0 }\implies v = c\sqrt{1 - \sqr{\frac{ l }{ l_0 }}} \approx 294 \cdot 10^{6}\,\frac{\text{м}}{\text{с}}.$
}


\variantsplitter


\addpersonalvariant{Андрей Рожков}

\tasknumber{12}%
\task{%
    Время жизни мюона, измеренное наблюдателем, относительно которого мюон покоился, равно $\tau_0$
    Какое расстояние пролетит мюон в системе отсчёта, относительно которой он движется со скоростью $v$,
    сравнимой со скоростью света в вакууме $c$?
}
\answer{%
    $\ell = v\tau = v \frac{\tau_0}{\sqrt{1 - \frac{v^2}{c^2}}}$
}
\solutionspace{80pt}

\tasknumber{13}%
\task{%
    Если $c$ — скорость света в вакууме, то с какой скоростью должна двигаться нестабильная частица относительно наблюдателя,
    чтобы её время жизни было в семь раз больше, чем у такой же, но покоящейся относительно наблюдателя частицы?
}
\answer{%
    $\tau = \frac{\tau_0}{\sqrt{1 - \frac{v^2}{c^2}}}\implies \sqrt{1 - \frac{v^2}{c^2}} = \frac{\tau_0}{\tau}\implies v = c\sqrt{1 - \sqr{\frac{\tau_0}{\tau}} } \approx 297 \cdot 10^{6}\,\frac{\text{м}}{\text{с}}.$
}
\solutionspace{80pt}

\tasknumber{14}%
\task{%
    Время жизни нестабильной частицы, входящего в состав космических лучей, измеренное земным наблюдателем,
    относительно которого частица двигалась со скоростью, составляющей 85\% скорости света в вакууме, оказалось равным $5{,}3\,\text{мкс}$.
    Каково время жизни частицы, покоящейся относительно наблюдателя?
}
\answer{%
    $t = \frac{t_0}{\sqrt{1 - \frac{v^2}{c^2}}}\implies t_0 = t\sqrt{1 - \frac{v^2}{c^2}} \approx 2{,}8 \cdot 10^{-6}\,\text{с}.$
}
\solutionspace{80pt}

\tasknumber{15}%
\task{%
    Частица увеличила в ускорителе свою скорость с $0{,}05c$ до $0{,}50c$.
    Во сколько раз выросла её кинетическая энергия?
}
\answer{%
    \begin{align*}
    E_{\text{кин.}} &= E - E_0 = \frac{mc^2}{\sqrt{1 - \frac{v^2}{c^2}}} - mc^2 = mc^2\cbr{ \frac1{\sqrt{1 - \frac{v^2}{c^2} }} - 1}.
    \\
    \frac{E_{\text{кин.
    2}}}{E_{\text{кин.
    1}}} &= \frac{\frac1{\sqrt{1 - \frac{v_2^2}{c^2} }} - 1}{\frac1{\sqrt{1 - \frac{v_1^2}{c^2} }} - 1}\approx 123{,}53
    \end{align*}
}
\solutionspace{120pt}

\tasknumber{16}%
\task{%
    Для частицы, движущейся с релятивистской скоростью,
    выразите $p$ и $E_0$ через $c$, $E_\text{кин}$ и $v$,
    где $E_\text{кин}$~--- кинетическая энергия частицы,
    а $E_0$, $p$ и $v$~--- её энергия покоя, импульс и скорость.
}
\answer{%
    \begin{align*}
    E_\text{кин}, E_0:\quad&E = E_\text{кин} + E_0 = \frac{E_0}{\sqrt{1 - \frac{v^2}{c^2}}} \implies \sqrt{1 - \frac{v^2}{c^2}} = \frac{E_0}{{E_0} + {E_\text{кин}}} \implies v = c\sqrt{1 - \sqr{\frac{E_0}{{E_0} + {E_\text{кин}}}}} \\
    &p = \frac{mv}{\sqrt{1 - \frac{v^2}{c^2}}} = \frac{E_0}{c^2} \cdot \sqrt{1 - \sqr{\frac{E_0}{{E_0} + {E_\text{кин}}}}} \cdot \frac{{E_\text{кин}} + {E_0}}{E_0} = \frac{E_0}{c^2} \cdot \sqrt{\sqr{\frac{{E_\text{кин}} + {E_0}}{E_0}} - 1}.
    \\
    E_\text{кин}, p:\quad&E_\text{кин} = E - E_0 = mc^2\cbr{\frac 1{\sqrt{1 - \frac{v^2}{c^2}}} - 1}, p = \frac{mv}{\sqrt{1 - \frac{v^2}{c^2}}} \implies \frac{E_\text{кин}}{p} = \frac{\frac 1{\sqrt{1 - \frac{v^2}{c^2}}} - 1}{\sqrt{1 - \frac{v^2}{c^2}}} \implies v = \ldots \\
    &E_0 = E - E_\text{кин} = \frac{E_0}{\sqrt{1 - \frac{v^2}{c^2}}} - E_\text{кин} \implies E_0 = \frac{E_\text{кин}}{\frac 1{\sqrt{1 - \frac{v^2}{c^2}}} - 1} = \ldots \\
    E_\text{кин}, v:\quad&E_\text{кин} = E - E_0 = mc^2\cbr{\frac 1{\sqrt{1 - \frac{v^2}{c^2}}} - 1} \implies m = \frac{E_\text{кин}}{c^2\cbr{\frac 1{\sqrt{1 - \frac{v^2}{c^2}}} - 1}} \\
    &E_0 = mc^2 = \frac{E_\text{кин}}{\frac 1{\sqrt{1 - \frac{v^2}{c^2}}} - 1} \\
    &p = \frac{mv}{\sqrt{1 - \frac{v^2}{c^2}}} = \frac{E_\text{кин}}{c^2\cbr{\frac 1{\sqrt{1 - \frac{v^2}{c^2}}} - 1}} \cdot \frac{v}{\sqrt{1 - \frac{v^2}{c^2}}} = \frac{{E_\text{кин}} v}{c^2\cbr{1 - {\sqrt{1 - \frac{v^2}{c^2}}}}} \\
    E_0, p:\quad&E_0 = mc^2, \quad p = \frac{mv}{\sqrt{1 - \frac{v^2}{c^2}}} \implies \frac{E_0}{p} = \frac{c^2}v{\sqrt{1 - \frac{v^2}{c^2}}} = c\sqrt{\frac{c^2}{v^2} - 1} \\
    &\sqr{\frac{E_0}{pc}} = \frac{c^2}{v^2} - 1 \implies \frac{v^2}{c^2} = \frac 1{1 + \frac{E_0^2}{p^2c^2}} \implies v = \frac c{\sqrt{1 + \frac{E_0^2}{p^2c^2}}} \\
    &{E_\text{кин}} = E - E_0 = \sqrt{E_0^2 + p^2c^2} - E_0 \\
    E_0, v:\quad&E_0 = mc^2 \implies m = \frac{E_0}{c^2} \qquad p = \frac{mv}{\sqrt{1 - \frac{v^2}{c^2}}} = \frac{E_0}{c^2} \cdot \frac{v}{\sqrt{1 - \frac{v^2}{c^2}}} \\
    &E_\text{кин}= mc^2\cbr{\frac 1{\sqrt{1 - \frac{v^2}{c^2}}} - 1} = \frac{E_0}{c^2}\cbr{\frac 1{\sqrt{1 - \frac{v^2}{c^2}}} - 1} \\
    p, v:\quad&p = \frac{mv}{\sqrt{1 - \frac{v^2}{c^2}}} \implies m = \frac p v {\sqrt{1 - \frac{v^2}{c^2}}} \implies E_0 = mc^2 =\frac {pc^2} v {\sqrt{1 - \frac{v^2}{c^2}}} \\
    &E_\text{кин} = mc^2\cbr{\frac 1{\sqrt{1 - \frac{v^2}{c^2}}} - 1} = \frac p v {\sqrt{1 - \frac{v^2}{c^2}}}\cbr{\frac 1{\sqrt{1 - \frac{v^2}{c^2}}} - 1} = \frac p v \cbr{1 - {\sqrt{1 - \frac{v^2}{c^2}}}}
    \end{align*}
}

\variantsplitter

\addpersonalvariant{Рената Таржиманова}

\tasknumber{1}%
\task{%
    Запишите
    \begin{itemize}
        \item постулаты специальной теории относительности,
        \item пример релятивистского эффекта, обнаружимый при скоростях гораздо меньше скорости света.
    \end{itemize}
}
\solutionspace{120pt}

\tasknumber{2}%
\task{%
    Запишите формулу для ...
    \begin{itemize}
        \item релятивистского сжатия,
        \item классического импульса,
        \item релятивистского импульса тела,
        \item энергии покоя тела,
        \item связь между релятивистским импульсом и релятивистской энергией.
    \end{itemize}
    Обязательно подпишите все физические величины.
}
\solutionspace{150pt}

\tasknumber{3}%
\task{%
    Протон движется со скоростью $0{,}7\,c$, где $c$~--- скорость света в вакууме.
    Каково при этом отношение кинетической энергии частицы $E_\text{кин.}$ к его энергии покоя $E_0$?
}
\answer{%
    \begin{align*}
    E &= \frac{E_0}{\sqrt{1 - \frac{v^2}{c^2}}}
            \implies \frac E{E_0}
                = \frac 1{\sqrt{1 - \frac{v^2}{c^2}}}
                = \frac 1{\sqrt{1 - \sqr{0{,}7}}}
                \approx 1{,}400,
         \\
        {E_{\text{кин}}} &= E - E_0
            \implies \frac{E_{\text{кин}}}{E_0}
                = \frac E{E_0} - 1
                = \frac 1{\sqrt{1 - \frac{v^2}{c^2}}} - 1
                = \frac 1{\sqrt{1 - \sqr{0{,}7}}} - 1
                \approx 0{,}400.
    \end{align*}
}
\solutionspace{80pt}

\tasknumber{4}%
\task{%
    Полная энергия релятивистской частицы в четыре раза больше её энергии покоя.
    Найти скорость этой частицы: в долях $c$ и численное значение.
    Скорость света в вакууме $c = 3 \cdot 10^{8}\,\frac{\text{м}}{\text{с}}$.
}
\answer{%
    \begin{align*}
    E &= \frac{E_0}{\sqrt{1 - \frac{v^2}{c^2}}}\implies \sqrt{1 - \frac{v^2}{c^2}} = \frac{E_0}{E}\implies \frac{v^2}{c^2} = 1 - \sqr{\frac{E_0}{E}}\implies v = c \sqrt{1 - \sqr{\frac{E_0}{E}}} \approx 0{,}968c \approx 290 \cdot 10^{6}\,\frac{\text{м}}{\text{с}}.
    \end{align*}
}
\solutionspace{80pt}

\tasknumber{5}%
\task{%
    Кинетическая энергия релятивистской частицы в четыре раза больше её энергии покоя.
    Найти скорость этой частицы.
    Скорость света в вакууме $c = 3 \cdot 10^{8}\,\frac{\text{м}}{\text{с}}$.
}
\answer{%
    \begin{align*}
    E &= E_0 + E_{\text{кин}} \\
    E &= \frac{E_0}{\sqrt{1 - \frac{v^2}{c^2}}}\implies \sqrt{1 - \frac{v^2}{c^2}} = \frac{E_0}{E}\implies \frac{v^2}{c^2} = 1 - \sqr{\frac{E_0}{E}} \implies \\
    \implies &v = c \sqrt{1 - \sqr{\frac{E_0}{E}}} = c \sqrt{1 - \sqr{\frac{E_0}{E_0 + E_{\text{кин}} }}} = c \sqrt{1 - \frac 1 {\sqr{ 1 + \frac{E_{\text{кин}}}{E_0} }} }\approx 0{,}980c \approx 294 \cdot 10^{6}\,\frac{\text{м}}{\text{с}}.
    \end{align*}
}


\variantsplitter


\addpersonalvariant{Рената Таржиманова}

\tasknumber{6}%
\task{%
    Протон движется со скоростью $0{,}65\,c$, где $c$~--- скорость света в вакууме.
    Определите его полную энергию (в ответе приведите формулу и укажите численное значение).
}
\answer{%
    \begin{align*}
    E &= \frac{mc^2}{\sqrt{1 - \frac{v^2}{c^2}}}
            \approx \frac{1{,}673 \cdot 10^{-27}\,\text{кг} \cdot \sqr{3 \cdot 10^{8}\,\frac{\text{м}}{\text{с}}}}{\sqrt{1 - 0{,}65^2}}
            \approx 0{,}19809 \cdot 10^{-9}\,\text{Дж},
         \\
        E_{\text{кин}} &= \frac{mc^2}{\sqrt{1 - \frac{v^2}{c^2}}} - mc^2
            = mc^2 \cbr{\frac 1{\sqrt{1 - \frac{v^2}{c^2}}} - 1} \approx \\
            &\approx \cbr{1{,}673 \cdot 10^{-27}\,\text{кг} \cdot \sqr{3 \cdot 10^{8}\,\frac{\text{м}}{\text{с}}}}
            \cdot \cbr{\frac 1{\sqrt{1 - 0{,}65^2}} - 1}
            \approx 47{,}55 \cdot 10^{-12}\,\text{Дж},
         \\
        p &= \frac{mv}{\sqrt{1 - \frac{v^2}{c^2}}}
            \approx \frac{1{,}673 \cdot 10^{-27}\,\text{кг} \cdot 0{,}65 \cdot 3 \cdot 10^{8}\,\frac{\text{м}}{\text{с}}}{\sqrt{1 - 0{,}65^2}}
            \approx 0{,}4292 \cdot 10^{-18}\,\frac{\text{кг}\cdot\text{м}}{\text{с}}.
    \end{align*}
}
\solutionspace{100pt}

\tasknumber{7}%
\task{%
    Кинетическая энергия частицы космических лучей в четыре раза превышает её энергию покоя.
    Определить отношение скорости частицы к скорости света.
}
\answer{%
    \begin{align*}
    E &= E_0 + E_{\text{кин}} \\
    E &= \frac{E_0}{\sqrt{1 - \frac{v^2}{c^2}}}\implies \sqrt{1 - \frac{v^2}{c^2}} = \frac{E_0}{E}\implies \frac{v^2}{c^2} = 1 - \sqr{\frac{E_0}{E}} \implies \\
    \implies \frac vc &= \sqrt{1 - \sqr{\frac{E_0}{E}}} = \sqrt{1 - \sqr{\frac{E_0}{E_0 + E_{\text{кин}} }}} \approx 0{,}980.
    \end{align*}
}
\solutionspace{80pt}

\tasknumber{8}%
\task{%
    Некоторая частица, пройдя ускоряющую разность потенциалов, приобрела импульс $3 \cdot 10^{-19}\,\frac{\text{кг}\cdot\text{м}}{\text{с}}$.
    Скорость частицы стала равной $1{,}8 \cdot 10^{8}\,\frac{\text{м}}{\text{с}}$.
    Найти массу частицы.
}
\answer{%
    $p = \frac{ mv }{\sqrt{1 - \frac{v^2}{c^2} }}\implies m = \frac pv \sqrt{1 - \frac{v^2}{c^2}}= \frac {3 \cdot 10^{-19}\,\frac{\text{кг}\cdot\text{м}}{\text{с}}}{1{,}8 \cdot 10^{8}\,\frac{\text{м}}{\text{с}}} \sqrt{1 - \sqr{\frac{1{,}8 \cdot 10^{8}\,\frac{\text{м}}{\text{с}}}{3 \cdot 10^{8}\,\frac{\text{м}}{\text{с}}}} } \approx 1{,}33 \cdot 10^{-27}\,\text{кг}.$
}
\solutionspace{80pt}

\tasknumber{9}%
\task{%
    При какой скорости движения (в км/ч) релятивистское сокращение длины движущегося тела
    составит 50\%?
}
\answer{%
    \begin{align*}
    l_0 &= \frac l{\sqrt{1 - \frac{v^2}{c^2}}}
        \implies 1 - \frac{v^2}{c^2} = \sqr{\frac l{l_0}}
        \implies \frac v c = \sqrt{1 - \sqr{\frac l{l_0}}} \implies
         \\
        \implies v &= c\sqrt{1 - \sqr{\frac l{l_0}}}
        = 3 \cdot 10^{8}\,\frac{\text{м}}{\text{с}} \cdot \sqrt{1 - \sqr{\frac {l_0 - 0{,}50l_0}{l_0}}}
        = 3 \cdot 10^{8}\,\frac{\text{м}}{\text{с}} \cdot \sqrt{1 - \sqr{1 - 0{,}50}} \approx  \\
        &\approx 0{,}866c
        \approx 260 \cdot 10^{6}\,\frac{\text{м}}{\text{с}}
        \approx 935 \cdot 10^{6}\,\frac{\text{км}}{\text{ч}}.
    \end{align*}
}
\solutionspace{80pt}

\tasknumber{10}%
\task{%
    Стержень движется в продольном направлении с постоянной скоростью относительно инерциальной системы отсчёта.
    При каком значении скорости (в долях скорости света) длина стержня в этой системе отсчёта
    будет в  2{,}5  раза меньше его собственной длины?
}
\answer{%
    $l_0 = \frac l{\sqrt{1 - \frac{v^2}{c^2}}}\implies \sqrt{1 - \frac{v^2}{c^2}} = \frac{ l }{ l_0 }\implies \frac v c = \sqrt{1 - \sqr{\frac{ l }{ l_0 }}} \approx 0{,}917.$
}
\solutionspace{80pt}

\tasknumber{11}%
\task{%
    Какую скорость должно иметь движущееся тело, чтобы его продольные размеры уменьшились в пять раз?
    Скорость света $c = 3 \cdot 10^{8}\,\frac{\text{м}}{\text{с}}$.
}
\answer{%
    $l_0 = \frac l{\sqrt{1 - \frac{v^2}{c^2}}}\implies \sqrt{1 - \frac{v^2}{c^2}} = \frac{ l }{ l_0 }\implies v = c\sqrt{1 - \sqr{\frac{ l }{ l_0 }}} \approx 294 \cdot 10^{6}\,\frac{\text{м}}{\text{с}}.$
}


\variantsplitter


\addpersonalvariant{Рената Таржиманова}

\tasknumber{12}%
\task{%
    Время жизни мюона, измеренное наблюдателем, относительно которого мюон покоился, равно $\tau_0$
    Какое расстояние пролетит мюон в системе отсчёта, относительно которой он движется со скоростью $v$,
    сравнимой со скоростью света в вакууме $c$?
}
\answer{%
    $\ell = v\tau = v \frac{\tau_0}{\sqrt{1 - \frac{v^2}{c^2}}}$
}
\solutionspace{80pt}

\tasknumber{13}%
\task{%
    Если $c$ — скорость света в вакууме, то с какой скоростью должна двигаться нестабильная частица относительно наблюдателя,
    чтобы её время жизни было в четыре раза больше, чем у такой же, но покоящейся относительно наблюдателя частицы?
}
\answer{%
    $\tau = \frac{\tau_0}{\sqrt{1 - \frac{v^2}{c^2}}}\implies \sqrt{1 - \frac{v^2}{c^2}} = \frac{\tau_0}{\tau}\implies v = c\sqrt{1 - \sqr{\frac{\tau_0}{\tau}} } \approx 290 \cdot 10^{6}\,\frac{\text{м}}{\text{с}}.$
}
\solutionspace{80pt}

\tasknumber{14}%
\task{%
    Время жизни нестабильной частицы, входящего в состав космических лучей, измеренное земным наблюдателем,
    относительно которого частица двигалась со скоростью, составляющей 85\% скорости света в вакууме, оказалось равным $6{,}4\,\text{мкс}$.
    Каково время жизни частицы, покоящейся относительно наблюдателя?
}
\answer{%
    $t = \frac{t_0}{\sqrt{1 - \frac{v^2}{c^2}}}\implies t_0 = t\sqrt{1 - \frac{v^2}{c^2}} \approx 3{,}4 \cdot 10^{-6}\,\text{с}.$
}
\solutionspace{80pt}

\tasknumber{15}%
\task{%
    Частица увеличила в ускорителе свою скорость с $0{,}05c$ до $0{,}90c$.
    Во сколько раз выросла её кинетическая энергия?
}
\answer{%
    \begin{align*}
    E_{\text{кин.}} &= E - E_0 = \frac{mc^2}{\sqrt{1 - \frac{v^2}{c^2}}} - mc^2 = mc^2\cbr{ \frac1{\sqrt{1 - \frac{v^2}{c^2} }} - 1}.
    \\
    \frac{E_{\text{кин.
    2}}}{E_{\text{кин.
    1}}} &= \frac{\frac1{\sqrt{1 - \frac{v_2^2}{c^2} }} - 1}{\frac1{\sqrt{1 - \frac{v_1^2}{c^2} }} - 1}\approx 1033{,}38
    \end{align*}
}
\solutionspace{120pt}

\tasknumber{16}%
\task{%
    Для частицы, движущейся с релятивистской скоростью,
    выразите $E_\text{кин}$ и $v$ через $c$, $p$ и $E_0$,
    где $E_\text{кин}$~--- кинетическая энергия частицы,
    а $E_0$, $p$ и $v$~--- её энергия покоя, импульс и скорость.
}
\answer{%
    \begin{align*}
    E_\text{кин}, E_0:\quad&E = E_\text{кин} + E_0 = \frac{E_0}{\sqrt{1 - \frac{v^2}{c^2}}} \implies \sqrt{1 - \frac{v^2}{c^2}} = \frac{E_0}{{E_0} + {E_\text{кин}}} \implies v = c\sqrt{1 - \sqr{\frac{E_0}{{E_0} + {E_\text{кин}}}}} \\
    &p = \frac{mv}{\sqrt{1 - \frac{v^2}{c^2}}} = \frac{E_0}{c^2} \cdot \sqrt{1 - \sqr{\frac{E_0}{{E_0} + {E_\text{кин}}}}} \cdot \frac{{E_\text{кин}} + {E_0}}{E_0} = \frac{E_0}{c^2} \cdot \sqrt{\sqr{\frac{{E_\text{кин}} + {E_0}}{E_0}} - 1}.
    \\
    E_\text{кин}, p:\quad&E_\text{кин} = E - E_0 = mc^2\cbr{\frac 1{\sqrt{1 - \frac{v^2}{c^2}}} - 1}, p = \frac{mv}{\sqrt{1 - \frac{v^2}{c^2}}} \implies \frac{E_\text{кин}}{p} = \frac{\frac 1{\sqrt{1 - \frac{v^2}{c^2}}} - 1}{\sqrt{1 - \frac{v^2}{c^2}}} \implies v = \ldots \\
    &E_0 = E - E_\text{кин} = \frac{E_0}{\sqrt{1 - \frac{v^2}{c^2}}} - E_\text{кин} \implies E_0 = \frac{E_\text{кин}}{\frac 1{\sqrt{1 - \frac{v^2}{c^2}}} - 1} = \ldots \\
    E_\text{кин}, v:\quad&E_\text{кин} = E - E_0 = mc^2\cbr{\frac 1{\sqrt{1 - \frac{v^2}{c^2}}} - 1} \implies m = \frac{E_\text{кин}}{c^2\cbr{\frac 1{\sqrt{1 - \frac{v^2}{c^2}}} - 1}} \\
    &E_0 = mc^2 = \frac{E_\text{кин}}{\frac 1{\sqrt{1 - \frac{v^2}{c^2}}} - 1} \\
    &p = \frac{mv}{\sqrt{1 - \frac{v^2}{c^2}}} = \frac{E_\text{кин}}{c^2\cbr{\frac 1{\sqrt{1 - \frac{v^2}{c^2}}} - 1}} \cdot \frac{v}{\sqrt{1 - \frac{v^2}{c^2}}} = \frac{{E_\text{кин}} v}{c^2\cbr{1 - {\sqrt{1 - \frac{v^2}{c^2}}}}} \\
    E_0, p:\quad&E_0 = mc^2, \quad p = \frac{mv}{\sqrt{1 - \frac{v^2}{c^2}}} \implies \frac{E_0}{p} = \frac{c^2}v{\sqrt{1 - \frac{v^2}{c^2}}} = c\sqrt{\frac{c^2}{v^2} - 1} \\
    &\sqr{\frac{E_0}{pc}} = \frac{c^2}{v^2} - 1 \implies \frac{v^2}{c^2} = \frac 1{1 + \frac{E_0^2}{p^2c^2}} \implies v = \frac c{\sqrt{1 + \frac{E_0^2}{p^2c^2}}} \\
    &{E_\text{кин}} = E - E_0 = \sqrt{E_0^2 + p^2c^2} - E_0 \\
    E_0, v:\quad&E_0 = mc^2 \implies m = \frac{E_0}{c^2} \qquad p = \frac{mv}{\sqrt{1 - \frac{v^2}{c^2}}} = \frac{E_0}{c^2} \cdot \frac{v}{\sqrt{1 - \frac{v^2}{c^2}}} \\
    &E_\text{кин}= mc^2\cbr{\frac 1{\sqrt{1 - \frac{v^2}{c^2}}} - 1} = \frac{E_0}{c^2}\cbr{\frac 1{\sqrt{1 - \frac{v^2}{c^2}}} - 1} \\
    p, v:\quad&p = \frac{mv}{\sqrt{1 - \frac{v^2}{c^2}}} \implies m = \frac p v {\sqrt{1 - \frac{v^2}{c^2}}} \implies E_0 = mc^2 =\frac {pc^2} v {\sqrt{1 - \frac{v^2}{c^2}}} \\
    &E_\text{кин} = mc^2\cbr{\frac 1{\sqrt{1 - \frac{v^2}{c^2}}} - 1} = \frac p v {\sqrt{1 - \frac{v^2}{c^2}}}\cbr{\frac 1{\sqrt{1 - \frac{v^2}{c^2}}} - 1} = \frac p v \cbr{1 - {\sqrt{1 - \frac{v^2}{c^2}}}}
    \end{align*}
}

\variantsplitter

\addpersonalvariant{Андрей Щербаков}

\tasknumber{1}%
\task{%
    Запишите
    \begin{itemize}
        \item постулаты специальной теории относительности,
        \item пример релятивистского эффекта, обнаружимый при скоростях гораздо меньше скорости света.
    \end{itemize}
}
\solutionspace{120pt}

\tasknumber{2}%
\task{%
    Запишите формулу для ...
    \begin{itemize}
        \item релятивистского сжатия,
        \item классического импульса,
        \item релятивистской энергии тела,
        \item релятивистской кинетической энергии,
        \item связь между релятивистским импульсом и релятивистской энергией.
    \end{itemize}
    Обязательно подпишите все физические величины.
}
\solutionspace{150pt}

\tasknumber{3}%
\task{%
    Позитрон движется со скоростью $0{,}7\,c$, где $c$~--- скорость света в вакууме.
    Каково при этом отношение кинетической энергии частицы $E_\text{кин.}$ к его энергии покоя $E_0$?
}
\answer{%
    \begin{align*}
    E &= \frac{E_0}{\sqrt{1 - \frac{v^2}{c^2}}}
            \implies \frac E{E_0}
                = \frac 1{\sqrt{1 - \frac{v^2}{c^2}}}
                = \frac 1{\sqrt{1 - \sqr{0{,}7}}}
                \approx 1{,}400,
         \\
        {E_{\text{кин}}} &= E - E_0
            \implies \frac{E_{\text{кин}}}{E_0}
                = \frac E{E_0} - 1
                = \frac 1{\sqrt{1 - \frac{v^2}{c^2}}} - 1
                = \frac 1{\sqrt{1 - \sqr{0{,}7}}} - 1
                \approx 0{,}400.
    \end{align*}
}
\solutionspace{80pt}

\tasknumber{4}%
\task{%
    Полная энергия релятивистской частицы в четыре раза больше её энергии покоя.
    Найти скорость этой частицы: в долях $c$ и численное значение.
    Скорость света в вакууме $c = 3 \cdot 10^{8}\,\frac{\text{м}}{\text{с}}$.
}
\answer{%
    \begin{align*}
    E &= \frac{E_0}{\sqrt{1 - \frac{v^2}{c^2}}}\implies \sqrt{1 - \frac{v^2}{c^2}} = \frac{E_0}{E}\implies \frac{v^2}{c^2} = 1 - \sqr{\frac{E_0}{E}}\implies v = c \sqrt{1 - \sqr{\frac{E_0}{E}}} \approx 0{,}968c \approx 290 \cdot 10^{6}\,\frac{\text{м}}{\text{с}}.
    \end{align*}
}
\solutionspace{80pt}

\tasknumber{5}%
\task{%
    Кинетическая энергия релятивистской частицы в четыре раза больше её энергии покоя.
    Найти скорость этой частицы.
    Скорость света в вакууме $c = 3 \cdot 10^{8}\,\frac{\text{м}}{\text{с}}$.
}
\answer{%
    \begin{align*}
    E &= E_0 + E_{\text{кин}} \\
    E &= \frac{E_0}{\sqrt{1 - \frac{v^2}{c^2}}}\implies \sqrt{1 - \frac{v^2}{c^2}} = \frac{E_0}{E}\implies \frac{v^2}{c^2} = 1 - \sqr{\frac{E_0}{E}} \implies \\
    \implies &v = c \sqrt{1 - \sqr{\frac{E_0}{E}}} = c \sqrt{1 - \sqr{\frac{E_0}{E_0 + E_{\text{кин}} }}} = c \sqrt{1 - \frac 1 {\sqr{ 1 + \frac{E_{\text{кин}}}{E_0} }} }\approx 0{,}980c \approx 294 \cdot 10^{6}\,\frac{\text{м}}{\text{с}}.
    \end{align*}
}


\variantsplitter


\addpersonalvariant{Андрей Щербаков}

\tasknumber{6}%
\task{%
    Протон движется со скоростью $0{,}75\,c$, где $c$~--- скорость света в вакууме.
    Определите его полную энергию (в ответе приведите формулу и укажите численное значение).
}
\answer{%
    \begin{align*}
    E &= \frac{mc^2}{\sqrt{1 - \frac{v^2}{c^2}}}
            \approx \frac{1{,}673 \cdot 10^{-27}\,\text{кг} \cdot \sqr{3 \cdot 10^{8}\,\frac{\text{м}}{\text{с}}}}{\sqrt{1 - 0{,}75^2}}
            \approx 0{,}2276 \cdot 10^{-9}\,\text{Дж},
         \\
        E_{\text{кин}} &= \frac{mc^2}{\sqrt{1 - \frac{v^2}{c^2}}} - mc^2
            = mc^2 \cbr{\frac 1{\sqrt{1 - \frac{v^2}{c^2}}} - 1} \approx \\
            &\approx \cbr{1{,}673 \cdot 10^{-27}\,\text{кг} \cdot \sqr{3 \cdot 10^{8}\,\frac{\text{м}}{\text{с}}}}
            \cdot \cbr{\frac 1{\sqrt{1 - 0{,}75^2}} - 1}
            \approx 77{,}05 \cdot 10^{-12}\,\text{Дж},
         \\
        p &= \frac{mv}{\sqrt{1 - \frac{v^2}{c^2}}}
            \approx \frac{1{,}673 \cdot 10^{-27}\,\text{кг} \cdot 0{,}75 \cdot 3 \cdot 10^{8}\,\frac{\text{м}}{\text{с}}}{\sqrt{1 - 0{,}75^2}}
            \approx 0{,}5690 \cdot 10^{-18}\,\frac{\text{кг}\cdot\text{м}}{\text{с}}.
    \end{align*}
}
\solutionspace{100pt}

\tasknumber{7}%
\task{%
    Кинетическая энергия частицы космических лучей в четыре раза превышает её энергию покоя.
    Определить отношение скорости частицы к скорости света.
}
\answer{%
    \begin{align*}
    E &= E_0 + E_{\text{кин}} \\
    E &= \frac{E_0}{\sqrt{1 - \frac{v^2}{c^2}}}\implies \sqrt{1 - \frac{v^2}{c^2}} = \frac{E_0}{E}\implies \frac{v^2}{c^2} = 1 - \sqr{\frac{E_0}{E}} \implies \\
    \implies \frac vc &= \sqrt{1 - \sqr{\frac{E_0}{E}}} = \sqrt{1 - \sqr{\frac{E_0}{E_0 + E_{\text{кин}} }}} \approx 0{,}980.
    \end{align*}
}
\solutionspace{80pt}

\tasknumber{8}%
\task{%
    Некоторая частица, пройдя ускоряющую разность потенциалов, приобрела импульс $4{,}2 \cdot 10^{-19}\,\frac{\text{кг}\cdot\text{м}}{\text{с}}$.
    Скорость частицы стала равной $2 \cdot 10^{8}\,\frac{\text{м}}{\text{с}}$.
    Найти массу частицы.
}
\answer{%
    $p = \frac{ mv }{\sqrt{1 - \frac{v^2}{c^2} }}\implies m = \frac pv \sqrt{1 - \frac{v^2}{c^2}}= \frac {4{,}2 \cdot 10^{-19}\,\frac{\text{кг}\cdot\text{м}}{\text{с}}}{2 \cdot 10^{8}\,\frac{\text{м}}{\text{с}}} \sqrt{1 - \sqr{\frac{2 \cdot 10^{8}\,\frac{\text{м}}{\text{с}}}{3 \cdot 10^{8}\,\frac{\text{м}}{\text{с}}}} } \approx 1{,}565 \cdot 10^{-27}\,\text{кг}.$
}
\solutionspace{80pt}

\tasknumber{9}%
\task{%
    При какой скорости движения (в м/с) релятивистское сокращение длины движущегося тела
    составит 50\%?
}
\answer{%
    \begin{align*}
    l_0 &= \frac l{\sqrt{1 - \frac{v^2}{c^2}}}
        \implies 1 - \frac{v^2}{c^2} = \sqr{\frac l{l_0}}
        \implies \frac v c = \sqrt{1 - \sqr{\frac l{l_0}}} \implies
         \\
        \implies v &= c\sqrt{1 - \sqr{\frac l{l_0}}}
        = 3 \cdot 10^{8}\,\frac{\text{м}}{\text{с}} \cdot \sqrt{1 - \sqr{\frac {l_0 - 0{,}50l_0}{l_0}}}
        = 3 \cdot 10^{8}\,\frac{\text{м}}{\text{с}} \cdot \sqrt{1 - \sqr{1 - 0{,}50}} \approx  \\
        &\approx 0{,}866c
        \approx 260 \cdot 10^{6}\,\frac{\text{м}}{\text{с}}
        \approx 935 \cdot 10^{6}\,\frac{\text{км}}{\text{ч}}.
    \end{align*}
}
\solutionspace{80pt}

\tasknumber{10}%
\task{%
    Стержень движется в продольном направлении с постоянной скоростью относительно инерциальной системы отсчёта.
    При каком значении скорости (в долях скорости света) длина стержня в этой системе отсчёта
    будет в  4  раза меньше его собственной длины?
}
\answer{%
    $l_0 = \frac l{\sqrt{1 - \frac{v^2}{c^2}}}\implies \sqrt{1 - \frac{v^2}{c^2}} = \frac{ l }{ l_0 }\implies \frac v c = \sqrt{1 - \sqr{\frac{ l }{ l_0 }}} \approx 0{,}968.$
}
\solutionspace{80pt}

\tasknumber{11}%
\task{%
    Какую скорость должно иметь движущееся тело, чтобы его продольные размеры уменьшились в шесть раз?
    Скорость света $c = 3 \cdot 10^{8}\,\frac{\text{м}}{\text{с}}$.
}
\answer{%
    $l_0 = \frac l{\sqrt{1 - \frac{v^2}{c^2}}}\implies \sqrt{1 - \frac{v^2}{c^2}} = \frac{ l }{ l_0 }\implies v = c\sqrt{1 - \sqr{\frac{ l }{ l_0 }}} \approx 296 \cdot 10^{6}\,\frac{\text{м}}{\text{с}}.$
}


\variantsplitter


\addpersonalvariant{Андрей Щербаков}

\tasknumber{12}%
\task{%
    Время жизни мюона, измеренное наблюдателем, относительно которого мюон покоился, равно $\tau_0$
    Какое расстояние пролетит мюон в системе отсчёта, относительно которой он движется со скоростью $v$,
    сравнимой со скоростью света в вакууме $c$?
}
\answer{%
    $\ell = v\tau = v \frac{\tau_0}{\sqrt{1 - \frac{v^2}{c^2}}}$
}
\solutionspace{80pt}

\tasknumber{13}%
\task{%
    Если $c$ — скорость света в вакууме, то с какой скоростью должна двигаться нестабильная частица относительно наблюдателя,
    чтобы её время жизни было в восемь раз больше, чем у такой же, но покоящейся относительно наблюдателя частицы?
}
\answer{%
    $\tau = \frac{\tau_0}{\sqrt{1 - \frac{v^2}{c^2}}}\implies \sqrt{1 - \frac{v^2}{c^2}} = \frac{\tau_0}{\tau}\implies v = c\sqrt{1 - \sqr{\frac{\tau_0}{\tau}} } \approx 298 \cdot 10^{6}\,\frac{\text{м}}{\text{с}}.$
}
\solutionspace{80pt}

\tasknumber{14}%
\task{%
    Время жизни нестабильной частицы, входящего в состав космических лучей, измеренное земным наблюдателем,
    относительно которого частица двигалась со скоростью, составляющей 75\% скорости света в вакууме, оказалось равным $5{,}3\,\text{мкс}$.
    Каково время жизни частицы, покоящейся относительно наблюдателя?
}
\answer{%
    $t = \frac{t_0}{\sqrt{1 - \frac{v^2}{c^2}}}\implies t_0 = t\sqrt{1 - \frac{v^2}{c^2}} \approx 3{,}5 \cdot 10^{-6}\,\text{с}.$
}
\solutionspace{80pt}

\tasknumber{15}%
\task{%
    Частица увеличила в ускорителе свою скорость с $0{,}05c$ до $0{,}90c$.
    Во сколько раз выросла её кинетическая энергия?
}
\answer{%
    \begin{align*}
    E_{\text{кин.}} &= E - E_0 = \frac{mc^2}{\sqrt{1 - \frac{v^2}{c^2}}} - mc^2 = mc^2\cbr{ \frac1{\sqrt{1 - \frac{v^2}{c^2} }} - 1}.
    \\
    \frac{E_{\text{кин.
    2}}}{E_{\text{кин.
    1}}} &= \frac{\frac1{\sqrt{1 - \frac{v_2^2}{c^2} }} - 1}{\frac1{\sqrt{1 - \frac{v_1^2}{c^2} }} - 1}\approx 1033{,}38
    \end{align*}
}
\solutionspace{120pt}

\tasknumber{16}%
\task{%
    Для частицы, движущейся с релятивистской скоростью,
    выразите $v$ и $E_0$ через $c$, $p$ и $E_\text{кин}$,
    где $E_\text{кин}$~--- кинетическая энергия частицы,
    а $E_0$, $p$ и $v$~--- её энергия покоя, импульс и скорость.
}
\answer{%
    \begin{align*}
    E_\text{кин}, E_0:\quad&E = E_\text{кин} + E_0 = \frac{E_0}{\sqrt{1 - \frac{v^2}{c^2}}} \implies \sqrt{1 - \frac{v^2}{c^2}} = \frac{E_0}{{E_0} + {E_\text{кин}}} \implies v = c\sqrt{1 - \sqr{\frac{E_0}{{E_0} + {E_\text{кин}}}}} \\
    &p = \frac{mv}{\sqrt{1 - \frac{v^2}{c^2}}} = \frac{E_0}{c^2} \cdot \sqrt{1 - \sqr{\frac{E_0}{{E_0} + {E_\text{кин}}}}} \cdot \frac{{E_\text{кин}} + {E_0}}{E_0} = \frac{E_0}{c^2} \cdot \sqrt{\sqr{\frac{{E_\text{кин}} + {E_0}}{E_0}} - 1}.
    \\
    E_\text{кин}, p:\quad&E_\text{кин} = E - E_0 = mc^2\cbr{\frac 1{\sqrt{1 - \frac{v^2}{c^2}}} - 1}, p = \frac{mv}{\sqrt{1 - \frac{v^2}{c^2}}} \implies \frac{E_\text{кин}}{p} = \frac{\frac 1{\sqrt{1 - \frac{v^2}{c^2}}} - 1}{\sqrt{1 - \frac{v^2}{c^2}}} \implies v = \ldots \\
    &E_0 = E - E_\text{кин} = \frac{E_0}{\sqrt{1 - \frac{v^2}{c^2}}} - E_\text{кин} \implies E_0 = \frac{E_\text{кин}}{\frac 1{\sqrt{1 - \frac{v^2}{c^2}}} - 1} = \ldots \\
    E_\text{кин}, v:\quad&E_\text{кин} = E - E_0 = mc^2\cbr{\frac 1{\sqrt{1 - \frac{v^2}{c^2}}} - 1} \implies m = \frac{E_\text{кин}}{c^2\cbr{\frac 1{\sqrt{1 - \frac{v^2}{c^2}}} - 1}} \\
    &E_0 = mc^2 = \frac{E_\text{кин}}{\frac 1{\sqrt{1 - \frac{v^2}{c^2}}} - 1} \\
    &p = \frac{mv}{\sqrt{1 - \frac{v^2}{c^2}}} = \frac{E_\text{кин}}{c^2\cbr{\frac 1{\sqrt{1 - \frac{v^2}{c^2}}} - 1}} \cdot \frac{v}{\sqrt{1 - \frac{v^2}{c^2}}} = \frac{{E_\text{кин}} v}{c^2\cbr{1 - {\sqrt{1 - \frac{v^2}{c^2}}}}} \\
    E_0, p:\quad&E_0 = mc^2, \quad p = \frac{mv}{\sqrt{1 - \frac{v^2}{c^2}}} \implies \frac{E_0}{p} = \frac{c^2}v{\sqrt{1 - \frac{v^2}{c^2}}} = c\sqrt{\frac{c^2}{v^2} - 1} \\
    &\sqr{\frac{E_0}{pc}} = \frac{c^2}{v^2} - 1 \implies \frac{v^2}{c^2} = \frac 1{1 + \frac{E_0^2}{p^2c^2}} \implies v = \frac c{\sqrt{1 + \frac{E_0^2}{p^2c^2}}} \\
    &{E_\text{кин}} = E - E_0 = \sqrt{E_0^2 + p^2c^2} - E_0 \\
    E_0, v:\quad&E_0 = mc^2 \implies m = \frac{E_0}{c^2} \qquad p = \frac{mv}{\sqrt{1 - \frac{v^2}{c^2}}} = \frac{E_0}{c^2} \cdot \frac{v}{\sqrt{1 - \frac{v^2}{c^2}}} \\
    &E_\text{кин}= mc^2\cbr{\frac 1{\sqrt{1 - \frac{v^2}{c^2}}} - 1} = \frac{E_0}{c^2}\cbr{\frac 1{\sqrt{1 - \frac{v^2}{c^2}}} - 1} \\
    p, v:\quad&p = \frac{mv}{\sqrt{1 - \frac{v^2}{c^2}}} \implies m = \frac p v {\sqrt{1 - \frac{v^2}{c^2}}} \implies E_0 = mc^2 =\frac {pc^2} v {\sqrt{1 - \frac{v^2}{c^2}}} \\
    &E_\text{кин} = mc^2\cbr{\frac 1{\sqrt{1 - \frac{v^2}{c^2}}} - 1} = \frac p v {\sqrt{1 - \frac{v^2}{c^2}}}\cbr{\frac 1{\sqrt{1 - \frac{v^2}{c^2}}} - 1} = \frac p v \cbr{1 - {\sqrt{1 - \frac{v^2}{c^2}}}}
    \end{align*}
}

\variantsplitter

\addpersonalvariant{Михаил Ярошевский}

\tasknumber{1}%
\task{%
    Запишите
    \begin{itemize}
        \item постулаты специальной теории относительности,
        \item пример релятивистского эффекта, обнаружимый при скоростях гораздо меньше скорости света.
    \end{itemize}
}
\solutionspace{120pt}

\tasknumber{2}%
\task{%
    Запишите формулу для ...
    \begin{itemize}
        \item релятивистского замедления времени,
        \item классической полной механической энергии тела,
        \item релятивистского импульса тела,
        \item энергии покоя тела,
        \item связь между релятивистским импульсом и релятивистской энергией.
    \end{itemize}
    Обязательно подпишите все физические величины.
}
\solutionspace{150pt}

\tasknumber{3}%
\task{%
    Позитрон движется со скоростью $0{,}7\,c$, где $c$~--- скорость света в вакууме.
    Каково при этом отношение кинетической энергии частицы $E_\text{кин.}$ к его энергии покоя $E_0$?
}
\answer{%
    \begin{align*}
    E &= \frac{E_0}{\sqrt{1 - \frac{v^2}{c^2}}}
            \implies \frac E{E_0}
                = \frac 1{\sqrt{1 - \frac{v^2}{c^2}}}
                = \frac 1{\sqrt{1 - \sqr{0{,}7}}}
                \approx 1{,}400,
         \\
        {E_{\text{кин}}} &= E - E_0
            \implies \frac{E_{\text{кин}}}{E_0}
                = \frac E{E_0} - 1
                = \frac 1{\sqrt{1 - \frac{v^2}{c^2}}} - 1
                = \frac 1{\sqrt{1 - \sqr{0{,}7}}} - 1
                \approx 0{,}400.
    \end{align*}
}
\solutionspace{80pt}

\tasknumber{4}%
\task{%
    Полная энергия релятивистской частицы в три раза больше её энергии покоя.
    Найти скорость этой частицы: в долях $c$ и численное значение.
    Скорость света в вакууме $c = 3 \cdot 10^{8}\,\frac{\text{м}}{\text{с}}$.
}
\answer{%
    \begin{align*}
    E &= \frac{E_0}{\sqrt{1 - \frac{v^2}{c^2}}}\implies \sqrt{1 - \frac{v^2}{c^2}} = \frac{E_0}{E}\implies \frac{v^2}{c^2} = 1 - \sqr{\frac{E_0}{E}}\implies v = c \sqrt{1 - \sqr{\frac{E_0}{E}}} \approx 0{,}943c \approx 283 \cdot 10^{6}\,\frac{\text{м}}{\text{с}}.
    \end{align*}
}
\solutionspace{80pt}

\tasknumber{5}%
\task{%
    Кинетическая энергия релятивистской частицы в три раза больше её энергии покоя.
    Найти скорость этой частицы.
    Скорость света в вакууме $c = 3 \cdot 10^{8}\,\frac{\text{м}}{\text{с}}$.
}
\answer{%
    \begin{align*}
    E &= E_0 + E_{\text{кин}} \\
    E &= \frac{E_0}{\sqrt{1 - \frac{v^2}{c^2}}}\implies \sqrt{1 - \frac{v^2}{c^2}} = \frac{E_0}{E}\implies \frac{v^2}{c^2} = 1 - \sqr{\frac{E_0}{E}} \implies \\
    \implies &v = c \sqrt{1 - \sqr{\frac{E_0}{E}}} = c \sqrt{1 - \sqr{\frac{E_0}{E_0 + E_{\text{кин}} }}} = c \sqrt{1 - \frac 1 {\sqr{ 1 + \frac{E_{\text{кин}}}{E_0} }} }\approx 0{,}968c \approx 290 \cdot 10^{6}\,\frac{\text{м}}{\text{с}}.
    \end{align*}
}


\variantsplitter


\addpersonalvariant{Михаил Ярошевский}

\tasknumber{6}%
\task{%
    Протон движется со скоростью $0{,}85\,c$, где $c$~--- скорость света в вакууме.
    Определите его кинетическую энергию (в ответе приведите формулу и укажите численное значение).
}
\answer{%
    \begin{align*}
    E &= \frac{mc^2}{\sqrt{1 - \frac{v^2}{c^2}}}
            \approx \frac{1{,}673 \cdot 10^{-27}\,\text{кг} \cdot \sqr{3 \cdot 10^{8}\,\frac{\text{м}}{\text{с}}}}{\sqrt{1 - 0{,}85^2}}
            \approx 0{,}2858 \cdot 10^{-9}\,\text{Дж},
         \\
        E_{\text{кин}} &= \frac{mc^2}{\sqrt{1 - \frac{v^2}{c^2}}} - mc^2
            = mc^2 \cbr{\frac 1{\sqrt{1 - \frac{v^2}{c^2}}} - 1} \approx \\
            &\approx \cbr{1{,}673 \cdot 10^{-27}\,\text{кг} \cdot \sqr{3 \cdot 10^{8}\,\frac{\text{м}}{\text{с}}}}
            \cdot \cbr{\frac 1{\sqrt{1 - 0{,}85^2}} - 1}
            \approx 0{,}13523 \cdot 10^{-9}\,\text{Дж},
         \\
        p &= \frac{mv}{\sqrt{1 - \frac{v^2}{c^2}}}
            \approx \frac{1{,}673 \cdot 10^{-27}\,\text{кг} \cdot 0{,}85 \cdot 3 \cdot 10^{8}\,\frac{\text{м}}{\text{с}}}{\sqrt{1 - 0{,}85^2}}
            \approx 0{,}8097 \cdot 10^{-18}\,\frac{\text{кг}\cdot\text{м}}{\text{с}}.
    \end{align*}
}
\solutionspace{100pt}

\tasknumber{7}%
\task{%
    Кинетическая энергия частицы космических лучей в три раза превышает её энергию покоя.
    Определить отношение скорости частицы к скорости света.
}
\answer{%
    \begin{align*}
    E &= E_0 + E_{\text{кин}} \\
    E &= \frac{E_0}{\sqrt{1 - \frac{v^2}{c^2}}}\implies \sqrt{1 - \frac{v^2}{c^2}} = \frac{E_0}{E}\implies \frac{v^2}{c^2} = 1 - \sqr{\frac{E_0}{E}} \implies \\
    \implies \frac vc &= \sqrt{1 - \sqr{\frac{E_0}{E}}} = \sqrt{1 - \sqr{\frac{E_0}{E_0 + E_{\text{кин}} }}} \approx 0{,}968.
    \end{align*}
}
\solutionspace{80pt}

\tasknumber{8}%
\task{%
    Некоторая частица, пройдя ускоряющую разность потенциалов, приобрела импульс $3{,}5 \cdot 10^{-19}\,\frac{\text{кг}\cdot\text{м}}{\text{с}}$.
    Скорость частицы стала равной $1{,}8 \cdot 10^{8}\,\frac{\text{м}}{\text{с}}$.
    Найти массу частицы.
}
\answer{%
    $p = \frac{ mv }{\sqrt{1 - \frac{v^2}{c^2} }}\implies m = \frac pv \sqrt{1 - \frac{v^2}{c^2}}= \frac {3{,}5 \cdot 10^{-19}\,\frac{\text{кг}\cdot\text{м}}{\text{с}}}{1{,}8 \cdot 10^{8}\,\frac{\text{м}}{\text{с}}} \sqrt{1 - \sqr{\frac{1{,}8 \cdot 10^{8}\,\frac{\text{м}}{\text{с}}}{3 \cdot 10^{8}\,\frac{\text{м}}{\text{с}}}} } \approx 1{,}56 \cdot 10^{-27}\,\text{кг}.$
}
\solutionspace{80pt}

\tasknumber{9}%
\task{%
    При какой скорости движения (в м/с) релятивистское сокращение длины движущегося тела
    составит 50\%?
}
\answer{%
    \begin{align*}
    l_0 &= \frac l{\sqrt{1 - \frac{v^2}{c^2}}}
        \implies 1 - \frac{v^2}{c^2} = \sqr{\frac l{l_0}}
        \implies \frac v c = \sqrt{1 - \sqr{\frac l{l_0}}} \implies
         \\
        \implies v &= c\sqrt{1 - \sqr{\frac l{l_0}}}
        = 3 \cdot 10^{8}\,\frac{\text{м}}{\text{с}} \cdot \sqrt{1 - \sqr{\frac {l_0 - 0{,}50l_0}{l_0}}}
        = 3 \cdot 10^{8}\,\frac{\text{м}}{\text{с}} \cdot \sqrt{1 - \sqr{1 - 0{,}50}} \approx  \\
        &\approx 0{,}866c
        \approx 260 \cdot 10^{6}\,\frac{\text{м}}{\text{с}}
        \approx 935 \cdot 10^{6}\,\frac{\text{км}}{\text{ч}}.
    \end{align*}
}
\solutionspace{80pt}

\tasknumber{10}%
\task{%
    Стержень движется в продольном направлении с постоянной скоростью относительно инерциальной системы отсчёта.
    При каком значении скорости (в долях скорости света) длина стержня в этой системе отсчёта
    будет в  3  раза меньше его собственной длины?
}
\answer{%
    $l_0 = \frac l{\sqrt{1 - \frac{v^2}{c^2}}}\implies \sqrt{1 - \frac{v^2}{c^2}} = \frac{ l }{ l_0 }\implies \frac v c = \sqrt{1 - \sqr{\frac{ l }{ l_0 }}} \approx 0{,}943.$
}
\solutionspace{80pt}

\tasknumber{11}%
\task{%
    Какую скорость должно иметь движущееся тело, чтобы его продольные размеры уменьшились в два раза?
    Скорость света $c = 3 \cdot 10^{8}\,\frac{\text{м}}{\text{с}}$.
}
\answer{%
    $l_0 = \frac l{\sqrt{1 - \frac{v^2}{c^2}}}\implies \sqrt{1 - \frac{v^2}{c^2}} = \frac{ l }{ l_0 }\implies v = c\sqrt{1 - \sqr{\frac{ l }{ l_0 }}} \approx 260 \cdot 10^{6}\,\frac{\text{м}}{\text{с}}.$
}


\variantsplitter


\addpersonalvariant{Михаил Ярошевский}

\tasknumber{12}%
\task{%
    Время жизни мюона, измеренное наблюдателем, относительно которого мюон покоился, равно $\tau_0$
    Какое расстояние пролетит мюон в системе отсчёта, относительно которой он движется со скоростью $v$,
    сравнимой со скоростью света в вакууме $c$?
}
\answer{%
    $\ell = v\tau = v \frac{\tau_0}{\sqrt{1 - \frac{v^2}{c^2}}}$
}
\solutionspace{80pt}

\tasknumber{13}%
\task{%
    Если $c$ — скорость света в вакууме, то с какой скоростью должна двигаться нестабильная частица относительно наблюдателя,
    чтобы её время жизни было в три раза больше, чем у такой же, но покоящейся относительно наблюдателя частицы?
}
\answer{%
    $\tau = \frac{\tau_0}{\sqrt{1 - \frac{v^2}{c^2}}}\implies \sqrt{1 - \frac{v^2}{c^2}} = \frac{\tau_0}{\tau}\implies v = c\sqrt{1 - \sqr{\frac{\tau_0}{\tau}} } \approx 283 \cdot 10^{6}\,\frac{\text{м}}{\text{с}}.$
}
\solutionspace{80pt}

\tasknumber{14}%
\task{%
    Время жизни нестабильной частицы, входящего в состав космических лучей, измеренное земным наблюдателем,
    относительно которого частица двигалась со скоростью, составляющей 85\% скорости света в вакууме, оказалось равным $4{,}8\,\text{мкс}$.
    Каково время жизни частицы, покоящейся относительно наблюдателя?
}
\answer{%
    $t = \frac{t_0}{\sqrt{1 - \frac{v^2}{c^2}}}\implies t_0 = t\sqrt{1 - \frac{v^2}{c^2}} \approx 2{,}5 \cdot 10^{-6}\,\text{с}.$
}
\solutionspace{80pt}

\tasknumber{15}%
\task{%
    Частица увеличила в ускорителе свою скорость с $0{,}02c$ до $0{,}60c$.
    Во сколько раз выросла её кинетическая энергия?
}
\answer{%
    \begin{align*}
    E_{\text{кин.}} &= E - E_0 = \frac{mc^2}{\sqrt{1 - \frac{v^2}{c^2}}} - mc^2 = mc^2\cbr{ \frac1{\sqrt{1 - \frac{v^2}{c^2} }} - 1}.
    \\
    \frac{E_{\text{кин.
    2}}}{E_{\text{кин.
    1}}} &= \frac{\frac1{\sqrt{1 - \frac{v_2^2}{c^2} }} - 1}{\frac1{\sqrt{1 - \frac{v_1^2}{c^2} }} - 1}\approx 1249{,}62
    \end{align*}
}
\solutionspace{120pt}

\tasknumber{16}%
\task{%
    Для частицы, движущейся с релятивистской скоростью,
    выразите $E_0$ и $v$ через $c$, $p$ и $E_\text{кин}$,
    где $E_\text{кин}$~--- кинетическая энергия частицы,
    а $E_0$, $p$ и $v$~--- её энергия покоя, импульс и скорость.
}
\answer{%
    \begin{align*}
    E_\text{кин}, E_0:\quad&E = E_\text{кин} + E_0 = \frac{E_0}{\sqrt{1 - \frac{v^2}{c^2}}} \implies \sqrt{1 - \frac{v^2}{c^2}} = \frac{E_0}{{E_0} + {E_\text{кин}}} \implies v = c\sqrt{1 - \sqr{\frac{E_0}{{E_0} + {E_\text{кин}}}}} \\
    &p = \frac{mv}{\sqrt{1 - \frac{v^2}{c^2}}} = \frac{E_0}{c^2} \cdot \sqrt{1 - \sqr{\frac{E_0}{{E_0} + {E_\text{кин}}}}} \cdot \frac{{E_\text{кин}} + {E_0}}{E_0} = \frac{E_0}{c^2} \cdot \sqrt{\sqr{\frac{{E_\text{кин}} + {E_0}}{E_0}} - 1}.
    \\
    E_\text{кин}, p:\quad&E_\text{кин} = E - E_0 = mc^2\cbr{\frac 1{\sqrt{1 - \frac{v^2}{c^2}}} - 1}, p = \frac{mv}{\sqrt{1 - \frac{v^2}{c^2}}} \implies \frac{E_\text{кин}}{p} = \frac{\frac 1{\sqrt{1 - \frac{v^2}{c^2}}} - 1}{\sqrt{1 - \frac{v^2}{c^2}}} \implies v = \ldots \\
    &E_0 = E - E_\text{кин} = \frac{E_0}{\sqrt{1 - \frac{v^2}{c^2}}} - E_\text{кин} \implies E_0 = \frac{E_\text{кин}}{\frac 1{\sqrt{1 - \frac{v^2}{c^2}}} - 1} = \ldots \\
    E_\text{кин}, v:\quad&E_\text{кин} = E - E_0 = mc^2\cbr{\frac 1{\sqrt{1 - \frac{v^2}{c^2}}} - 1} \implies m = \frac{E_\text{кин}}{c^2\cbr{\frac 1{\sqrt{1 - \frac{v^2}{c^2}}} - 1}} \\
    &E_0 = mc^2 = \frac{E_\text{кин}}{\frac 1{\sqrt{1 - \frac{v^2}{c^2}}} - 1} \\
    &p = \frac{mv}{\sqrt{1 - \frac{v^2}{c^2}}} = \frac{E_\text{кин}}{c^2\cbr{\frac 1{\sqrt{1 - \frac{v^2}{c^2}}} - 1}} \cdot \frac{v}{\sqrt{1 - \frac{v^2}{c^2}}} = \frac{{E_\text{кин}} v}{c^2\cbr{1 - {\sqrt{1 - \frac{v^2}{c^2}}}}} \\
    E_0, p:\quad&E_0 = mc^2, \quad p = \frac{mv}{\sqrt{1 - \frac{v^2}{c^2}}} \implies \frac{E_0}{p} = \frac{c^2}v{\sqrt{1 - \frac{v^2}{c^2}}} = c\sqrt{\frac{c^2}{v^2} - 1} \\
    &\sqr{\frac{E_0}{pc}} = \frac{c^2}{v^2} - 1 \implies \frac{v^2}{c^2} = \frac 1{1 + \frac{E_0^2}{p^2c^2}} \implies v = \frac c{\sqrt{1 + \frac{E_0^2}{p^2c^2}}} \\
    &{E_\text{кин}} = E - E_0 = \sqrt{E_0^2 + p^2c^2} - E_0 \\
    E_0, v:\quad&E_0 = mc^2 \implies m = \frac{E_0}{c^2} \qquad p = \frac{mv}{\sqrt{1 - \frac{v^2}{c^2}}} = \frac{E_0}{c^2} \cdot \frac{v}{\sqrt{1 - \frac{v^2}{c^2}}} \\
    &E_\text{кин}= mc^2\cbr{\frac 1{\sqrt{1 - \frac{v^2}{c^2}}} - 1} = \frac{E_0}{c^2}\cbr{\frac 1{\sqrt{1 - \frac{v^2}{c^2}}} - 1} \\
    p, v:\quad&p = \frac{mv}{\sqrt{1 - \frac{v^2}{c^2}}} \implies m = \frac p v {\sqrt{1 - \frac{v^2}{c^2}}} \implies E_0 = mc^2 =\frac {pc^2} v {\sqrt{1 - \frac{v^2}{c^2}}} \\
    &E_\text{кин} = mc^2\cbr{\frac 1{\sqrt{1 - \frac{v^2}{c^2}}} - 1} = \frac p v {\sqrt{1 - \frac{v^2}{c^2}}}\cbr{\frac 1{\sqrt{1 - \frac{v^2}{c^2}}} - 1} = \frac p v \cbr{1 - {\sqrt{1 - \frac{v^2}{c^2}}}}
    \end{align*}
}

\variantsplitter

\addpersonalvariant{Алексей Алимпиев}

\tasknumber{1}%
\task{%
    Запишите
    \begin{itemize}
        \item постулаты специальной теории относительности,
        \item пример релятивистского эффекта, обнаружимый при скоростях гораздо меньше скорости света.
    \end{itemize}
}
\solutionspace{120pt}

\tasknumber{2}%
\task{%
    Запишите формулу для ...
    \begin{itemize}
        \item релятивистского сжатия,
        \item классического импульса,
        \item релятивистской энергии тела,
        \item энергии покоя тела,
        \item связь между релятивистским импульсом и релятивистской энергией.
    \end{itemize}
    Обязательно подпишите все физические величины.
}
\solutionspace{150pt}

\tasknumber{3}%
\task{%
    Позитрон движется со скоростью $0{,}9\,c$, где $c$~--- скорость света в вакууме.
    Каково при этом отношение полной энергии частицы $E$ к его энергии покоя $E_0$?
}
\answer{%
    \begin{align*}
    E &= \frac{E_0}{\sqrt{1 - \frac{v^2}{c^2}}}
            \implies \frac E{E_0}
                = \frac 1{\sqrt{1 - \frac{v^2}{c^2}}}
                = \frac 1{\sqrt{1 - \sqr{0{,}9}}}
                \approx 2{,}294,
         \\
        {E_{\text{кин}}} &= E - E_0
            \implies \frac{E_{\text{кин}}}{E_0}
                = \frac E{E_0} - 1
                = \frac 1{\sqrt{1 - \frac{v^2}{c^2}}} - 1
                = \frac 1{\sqrt{1 - \sqr{0{,}9}}} - 1
                \approx 1{,}294.
    \end{align*}
}
\solutionspace{80pt}

\tasknumber{4}%
\task{%
    Полная энергия релятивистской частицы в пять раз больше её энергии покоя.
    Найти скорость этой частицы: в долях $c$ и численное значение.
    Скорость света в вакууме $c = 3 \cdot 10^{8}\,\frac{\text{м}}{\text{с}}$.
}
\answer{%
    \begin{align*}
    E &= \frac{E_0}{\sqrt{1 - \frac{v^2}{c^2}}}\implies \sqrt{1 - \frac{v^2}{c^2}} = \frac{E_0}{E}\implies \frac{v^2}{c^2} = 1 - \sqr{\frac{E_0}{E}}\implies v = c \sqrt{1 - \sqr{\frac{E_0}{E}}} \approx 0{,}980c \approx 294 \cdot 10^{6}\,\frac{\text{м}}{\text{с}}.
    \end{align*}
}
\solutionspace{80pt}

\tasknumber{5}%
\task{%
    Кинетическая энергия релятивистской частицы в пять раз больше её энергии покоя.
    Найти скорость этой частицы.
    Скорость света в вакууме $c = 3 \cdot 10^{8}\,\frac{\text{м}}{\text{с}}$.
}
\answer{%
    \begin{align*}
    E &= E_0 + E_{\text{кин}} \\
    E &= \frac{E_0}{\sqrt{1 - \frac{v^2}{c^2}}}\implies \sqrt{1 - \frac{v^2}{c^2}} = \frac{E_0}{E}\implies \frac{v^2}{c^2} = 1 - \sqr{\frac{E_0}{E}} \implies \\
    \implies &v = c \sqrt{1 - \sqr{\frac{E_0}{E}}} = c \sqrt{1 - \sqr{\frac{E_0}{E_0 + E_{\text{кин}} }}} = c \sqrt{1 - \frac 1 {\sqr{ 1 + \frac{E_{\text{кин}}}{E_0} }} }\approx 0{,}986c \approx 296 \cdot 10^{6}\,\frac{\text{м}}{\text{с}}.
    \end{align*}
}


\variantsplitter


\addpersonalvariant{Алексей Алимпиев}

\tasknumber{6}%
\task{%
    Протон движется со скоростью $0{,}85\,c$, где $c$~--- скорость света в вакууме.
    Определите его импульс (в ответе приведите формулу и укажите численное значение).
}
\answer{%
    \begin{align*}
    E &= \frac{mc^2}{\sqrt{1 - \frac{v^2}{c^2}}}
            \approx \frac{1{,}673 \cdot 10^{-27}\,\text{кг} \cdot \sqr{3 \cdot 10^{8}\,\frac{\text{м}}{\text{с}}}}{\sqrt{1 - 0{,}85^2}}
            \approx 0{,}2858 \cdot 10^{-9}\,\text{Дж},
         \\
        E_{\text{кин}} &= \frac{mc^2}{\sqrt{1 - \frac{v^2}{c^2}}} - mc^2
            = mc^2 \cbr{\frac 1{\sqrt{1 - \frac{v^2}{c^2}}} - 1} \approx \\
            &\approx \cbr{1{,}673 \cdot 10^{-27}\,\text{кг} \cdot \sqr{3 \cdot 10^{8}\,\frac{\text{м}}{\text{с}}}}
            \cdot \cbr{\frac 1{\sqrt{1 - 0{,}85^2}} - 1}
            \approx 0{,}13523 \cdot 10^{-9}\,\text{Дж},
         \\
        p &= \frac{mv}{\sqrt{1 - \frac{v^2}{c^2}}}
            \approx \frac{1{,}673 \cdot 10^{-27}\,\text{кг} \cdot 0{,}85 \cdot 3 \cdot 10^{8}\,\frac{\text{м}}{\text{с}}}{\sqrt{1 - 0{,}85^2}}
            \approx 0{,}8097 \cdot 10^{-18}\,\frac{\text{кг}\cdot\text{м}}{\text{с}}.
    \end{align*}
}
\solutionspace{100pt}

\tasknumber{7}%
\task{%
    Кинетическая энергия частицы космических лучей в пять раз превышает её энергию покоя.
    Определить отношение скорости частицы к скорости света.
}
\answer{%
    \begin{align*}
    E &= E_0 + E_{\text{кин}} \\
    E &= \frac{E_0}{\sqrt{1 - \frac{v^2}{c^2}}}\implies \sqrt{1 - \frac{v^2}{c^2}} = \frac{E_0}{E}\implies \frac{v^2}{c^2} = 1 - \sqr{\frac{E_0}{E}} \implies \\
    \implies \frac vc &= \sqrt{1 - \sqr{\frac{E_0}{E}}} = \sqrt{1 - \sqr{\frac{E_0}{E_0 + E_{\text{кин}} }}} \approx 0{,}986.
    \end{align*}
}
\solutionspace{80pt}

\tasknumber{8}%
\task{%
    Некоторая частица, пройдя ускоряющую разность потенциалов, приобрела импульс $3{,}5 \cdot 10^{-19}\,\frac{\text{кг}\cdot\text{м}}{\text{с}}$.
    Скорость частицы стала равной $1{,}5 \cdot 10^{8}\,\frac{\text{м}}{\text{с}}$.
    Найти массу частицы.
}
\answer{%
    $p = \frac{ mv }{\sqrt{1 - \frac{v^2}{c^2} }}\implies m = \frac pv \sqrt{1 - \frac{v^2}{c^2}}= \frac {3{,}5 \cdot 10^{-19}\,\frac{\text{кг}\cdot\text{м}}{\text{с}}}{1{,}5 \cdot 10^{8}\,\frac{\text{м}}{\text{с}}} \sqrt{1 - \sqr{\frac{1{,}5 \cdot 10^{8}\,\frac{\text{м}}{\text{с}}}{3 \cdot 10^{8}\,\frac{\text{м}}{\text{с}}}} } \approx 2{,}0 \cdot 10^{-27}\,\text{кг}.$
}
\solutionspace{80pt}

\tasknumber{9}%
\task{%
    При какой скорости движения (в м/с) релятивистское сокращение длины движущегося тела
    составит 10\%?
}
\answer{%
    \begin{align*}
    l_0 &= \frac l{\sqrt{1 - \frac{v^2}{c^2}}}
        \implies 1 - \frac{v^2}{c^2} = \sqr{\frac l{l_0}}
        \implies \frac v c = \sqrt{1 - \sqr{\frac l{l_0}}} \implies
         \\
        \implies v &= c\sqrt{1 - \sqr{\frac l{l_0}}}
        = 3 \cdot 10^{8}\,\frac{\text{м}}{\text{с}} \cdot \sqrt{1 - \sqr{\frac {l_0 - 0{,}10l_0}{l_0}}}
        = 3 \cdot 10^{8}\,\frac{\text{м}}{\text{с}} \cdot \sqrt{1 - \sqr{1 - 0{,}10}} \approx  \\
        &\approx 0{,}436c
        \approx 130{,}8 \cdot 10^{6}\,\frac{\text{м}}{\text{с}}
        \approx 471 \cdot 10^{6}\,\frac{\text{км}}{\text{ч}}.
    \end{align*}
}
\solutionspace{80pt}

\tasknumber{10}%
\task{%
    Стержень движется в продольном направлении с постоянной скоростью относительно инерциальной системы отсчёта.
    При каком значении скорости (в долях скорости света) длина стержня в этой системе отсчёта
    будет в  2{,}5  раза меньше его собственной длины?
}
\answer{%
    $l_0 = \frac l{\sqrt{1 - \frac{v^2}{c^2}}}\implies \sqrt{1 - \frac{v^2}{c^2}} = \frac{ l }{ l_0 }\implies \frac v c = \sqrt{1 - \sqr{\frac{ l }{ l_0 }}} \approx 0{,}917.$
}
\solutionspace{80pt}

\tasknumber{11}%
\task{%
    Какую скорость должно иметь движущееся тело, чтобы его продольные размеры уменьшились в три раза?
    Скорость света $c = 3 \cdot 10^{8}\,\frac{\text{м}}{\text{с}}$.
}
\answer{%
    $l_0 = \frac l{\sqrt{1 - \frac{v^2}{c^2}}}\implies \sqrt{1 - \frac{v^2}{c^2}} = \frac{ l }{ l_0 }\implies v = c\sqrt{1 - \sqr{\frac{ l }{ l_0 }}} \approx 283 \cdot 10^{6}\,\frac{\text{м}}{\text{с}}.$
}


\variantsplitter


\addpersonalvariant{Алексей Алимпиев}

\tasknumber{12}%
\task{%
    Время жизни мюона, измеренное наблюдателем, относительно которого мюон покоился, равно $\tau_0$
    Какое расстояние пролетит мюон в системе отсчёта, относительно которой он движется со скоростью $v$,
    сравнимой со скоростью света в вакууме $c$?
}
\answer{%
    $\ell = v\tau = v \frac{\tau_0}{\sqrt{1 - \frac{v^2}{c^2}}}$
}
\solutionspace{80pt}

\tasknumber{13}%
\task{%
    Если $c$ — скорость света в вакууме, то с какой скоростью должна двигаться нестабильная частица относительно наблюдателя,
    чтобы её время жизни было в девять раз больше, чем у такой же, но покоящейся относительно наблюдателя частицы?
}
\answer{%
    $\tau = \frac{\tau_0}{\sqrt{1 - \frac{v^2}{c^2}}}\implies \sqrt{1 - \frac{v^2}{c^2}} = \frac{\tau_0}{\tau}\implies v = c\sqrt{1 - \sqr{\frac{\tau_0}{\tau}} } \approx 298 \cdot 10^{6}\,\frac{\text{м}}{\text{с}}.$
}
\solutionspace{80pt}

\tasknumber{14}%
\task{%
    Время жизни нестабильной частицы, входящего в состав космических лучей, измеренное земным наблюдателем,
    относительно которого частица двигалась со скоростью, составляющей 65\% скорости света в вакууме, оказалось равным $3{,}7\,\text{мкс}$.
    Каково время жизни частицы, покоящейся относительно наблюдателя?
}
\answer{%
    $t = \frac{t_0}{\sqrt{1 - \frac{v^2}{c^2}}}\implies t_0 = t\sqrt{1 - \frac{v^2}{c^2}} \approx 2{,}8 \cdot 10^{-6}\,\text{с}.$
}
\solutionspace{80pt}

\tasknumber{15}%
\task{%
    Частица увеличила в ускорителе свою скорость с $0{,}02c$ до $0{,}50c$.
    Во сколько раз выросла её кинетическая энергия?
}
\answer{%
    \begin{align*}
    E_{\text{кин.}} &= E - E_0 = \frac{mc^2}{\sqrt{1 - \frac{v^2}{c^2}}} - mc^2 = mc^2\cbr{ \frac1{\sqrt{1 - \frac{v^2}{c^2} }} - 1}.
    \\
    \frac{E_{\text{кин.
    2}}}{E_{\text{кин.
    1}}} &= \frac{\frac1{\sqrt{1 - \frac{v_2^2}{c^2} }} - 1}{\frac1{\sqrt{1 - \frac{v_1^2}{c^2} }} - 1}\approx 773{,}27
    \end{align*}
}
\solutionspace{120pt}

\tasknumber{16}%
\task{%
    Для частицы, движущейся с релятивистской скоростью,
    выразите $E_\text{кин}$ и $v$ через $c$, $E_0$ и $p$,
    где $E_\text{кин}$~--- кинетическая энергия частицы,
    а $E_0$, $p$ и $v$~--- её энергия покоя, импульс и скорость.
}
\answer{%
    \begin{align*}
    E_\text{кин}, E_0:\quad&E = E_\text{кин} + E_0 = \frac{E_0}{\sqrt{1 - \frac{v^2}{c^2}}} \implies \sqrt{1 - \frac{v^2}{c^2}} = \frac{E_0}{{E_0} + {E_\text{кин}}} \implies v = c\sqrt{1 - \sqr{\frac{E_0}{{E_0} + {E_\text{кин}}}}} \\
    &p = \frac{mv}{\sqrt{1 - \frac{v^2}{c^2}}} = \frac{E_0}{c^2} \cdot \sqrt{1 - \sqr{\frac{E_0}{{E_0} + {E_\text{кин}}}}} \cdot \frac{{E_\text{кин}} + {E_0}}{E_0} = \frac{E_0}{c^2} \cdot \sqrt{\sqr{\frac{{E_\text{кин}} + {E_0}}{E_0}} - 1}.
    \\
    E_\text{кин}, p:\quad&E_\text{кин} = E - E_0 = mc^2\cbr{\frac 1{\sqrt{1 - \frac{v^2}{c^2}}} - 1}, p = \frac{mv}{\sqrt{1 - \frac{v^2}{c^2}}} \implies \frac{E_\text{кин}}{p} = \frac{\frac 1{\sqrt{1 - \frac{v^2}{c^2}}} - 1}{\sqrt{1 - \frac{v^2}{c^2}}} \implies v = \ldots \\
    &E_0 = E - E_\text{кин} = \frac{E_0}{\sqrt{1 - \frac{v^2}{c^2}}} - E_\text{кин} \implies E_0 = \frac{E_\text{кин}}{\frac 1{\sqrt{1 - \frac{v^2}{c^2}}} - 1} = \ldots \\
    E_\text{кин}, v:\quad&E_\text{кин} = E - E_0 = mc^2\cbr{\frac 1{\sqrt{1 - \frac{v^2}{c^2}}} - 1} \implies m = \frac{E_\text{кин}}{c^2\cbr{\frac 1{\sqrt{1 - \frac{v^2}{c^2}}} - 1}} \\
    &E_0 = mc^2 = \frac{E_\text{кин}}{\frac 1{\sqrt{1 - \frac{v^2}{c^2}}} - 1} \\
    &p = \frac{mv}{\sqrt{1 - \frac{v^2}{c^2}}} = \frac{E_\text{кин}}{c^2\cbr{\frac 1{\sqrt{1 - \frac{v^2}{c^2}}} - 1}} \cdot \frac{v}{\sqrt{1 - \frac{v^2}{c^2}}} = \frac{{E_\text{кин}} v}{c^2\cbr{1 - {\sqrt{1 - \frac{v^2}{c^2}}}}} \\
    E_0, p:\quad&E_0 = mc^2, \quad p = \frac{mv}{\sqrt{1 - \frac{v^2}{c^2}}} \implies \frac{E_0}{p} = \frac{c^2}v{\sqrt{1 - \frac{v^2}{c^2}}} = c\sqrt{\frac{c^2}{v^2} - 1} \\
    &\sqr{\frac{E_0}{pc}} = \frac{c^2}{v^2} - 1 \implies \frac{v^2}{c^2} = \frac 1{1 + \frac{E_0^2}{p^2c^2}} \implies v = \frac c{\sqrt{1 + \frac{E_0^2}{p^2c^2}}} \\
    &{E_\text{кин}} = E - E_0 = \sqrt{E_0^2 + p^2c^2} - E_0 \\
    E_0, v:\quad&E_0 = mc^2 \implies m = \frac{E_0}{c^2} \qquad p = \frac{mv}{\sqrt{1 - \frac{v^2}{c^2}}} = \frac{E_0}{c^2} \cdot \frac{v}{\sqrt{1 - \frac{v^2}{c^2}}} \\
    &E_\text{кин}= mc^2\cbr{\frac 1{\sqrt{1 - \frac{v^2}{c^2}}} - 1} = \frac{E_0}{c^2}\cbr{\frac 1{\sqrt{1 - \frac{v^2}{c^2}}} - 1} \\
    p, v:\quad&p = \frac{mv}{\sqrt{1 - \frac{v^2}{c^2}}} \implies m = \frac p v {\sqrt{1 - \frac{v^2}{c^2}}} \implies E_0 = mc^2 =\frac {pc^2} v {\sqrt{1 - \frac{v^2}{c^2}}} \\
    &E_\text{кин} = mc^2\cbr{\frac 1{\sqrt{1 - \frac{v^2}{c^2}}} - 1} = \frac p v {\sqrt{1 - \frac{v^2}{c^2}}}\cbr{\frac 1{\sqrt{1 - \frac{v^2}{c^2}}} - 1} = \frac p v \cbr{1 - {\sqrt{1 - \frac{v^2}{c^2}}}}
    \end{align*}
}

\variantsplitter

\addpersonalvariant{Евгений Васин}

\tasknumber{1}%
\task{%
    Запишите
    \begin{itemize}
        \item постулаты специальной теории относительности,
        \item пример релятивистского эффекта, обнаружимый при скоростях гораздо меньше скорости света.
    \end{itemize}
}
\solutionspace{120pt}

\tasknumber{2}%
\task{%
    Запишите формулу для ...
    \begin{itemize}
        \item релятивистского замедления времени,
        \item классической полной механической энергии тела,
        \item релятивистского импульса тела,
        \item энергии покоя тела,
        \item связь между релятивистским импульсом и релятивистской энергией.
    \end{itemize}
    Обязательно подпишите все физические величины.
}
\solutionspace{150pt}

\tasknumber{3}%
\task{%
    Электрон движется со скоростью $0{,}7\,c$, где $c$~--- скорость света в вакууме.
    Каково при этом отношение кинетической энергии частицы $E_\text{кин.}$ к его энергии покоя $E_0$?
}
\answer{%
    \begin{align*}
    E &= \frac{E_0}{\sqrt{1 - \frac{v^2}{c^2}}}
            \implies \frac E{E_0}
                = \frac 1{\sqrt{1 - \frac{v^2}{c^2}}}
                = \frac 1{\sqrt{1 - \sqr{0{,}7}}}
                \approx 1{,}400,
         \\
        {E_{\text{кин}}} &= E - E_0
            \implies \frac{E_{\text{кин}}}{E_0}
                = \frac E{E_0} - 1
                = \frac 1{\sqrt{1 - \frac{v^2}{c^2}}} - 1
                = \frac 1{\sqrt{1 - \sqr{0{,}7}}} - 1
                \approx 0{,}400.
    \end{align*}
}
\solutionspace{80pt}

\tasknumber{4}%
\task{%
    Полная энергия релятивистской частицы в три раза больше её энергии покоя.
    Найти скорость этой частицы: в долях $c$ и численное значение.
    Скорость света в вакууме $c = 3 \cdot 10^{8}\,\frac{\text{м}}{\text{с}}$.
}
\answer{%
    \begin{align*}
    E &= \frac{E_0}{\sqrt{1 - \frac{v^2}{c^2}}}\implies \sqrt{1 - \frac{v^2}{c^2}} = \frac{E_0}{E}\implies \frac{v^2}{c^2} = 1 - \sqr{\frac{E_0}{E}}\implies v = c \sqrt{1 - \sqr{\frac{E_0}{E}}} \approx 0{,}943c \approx 283 \cdot 10^{6}\,\frac{\text{м}}{\text{с}}.
    \end{align*}
}
\solutionspace{80pt}

\tasknumber{5}%
\task{%
    Кинетическая энергия релятивистской частицы в три раза больше её энергии покоя.
    Найти скорость этой частицы.
    Скорость света в вакууме $c = 3 \cdot 10^{8}\,\frac{\text{м}}{\text{с}}$.
}
\answer{%
    \begin{align*}
    E &= E_0 + E_{\text{кин}} \\
    E &= \frac{E_0}{\sqrt{1 - \frac{v^2}{c^2}}}\implies \sqrt{1 - \frac{v^2}{c^2}} = \frac{E_0}{E}\implies \frac{v^2}{c^2} = 1 - \sqr{\frac{E_0}{E}} \implies \\
    \implies &v = c \sqrt{1 - \sqr{\frac{E_0}{E}}} = c \sqrt{1 - \sqr{\frac{E_0}{E_0 + E_{\text{кин}} }}} = c \sqrt{1 - \frac 1 {\sqr{ 1 + \frac{E_{\text{кин}}}{E_0} }} }\approx 0{,}968c \approx 290 \cdot 10^{6}\,\frac{\text{м}}{\text{с}}.
    \end{align*}
}


\variantsplitter


\addpersonalvariant{Евгений Васин}

\tasknumber{6}%
\task{%
    Электрон движется со скоростью $0{,}85\,c$, где $c$~--- скорость света в вакууме.
    Определите его кинетическую энергию (в ответе приведите формулу и укажите численное значение).
}
\answer{%
    \begin{align*}
    E &= \frac{mc^2}{\sqrt{1 - \frac{v^2}{c^2}}}
            \approx \frac{9{,}1 \cdot 10^{-31}\,\text{кг} \cdot \sqr{3 \cdot 10^{8}\,\frac{\text{м}}{\text{с}}}}{\sqrt{1 - 0{,}85^2}}
            \approx 0{,}1555 \cdot 10^{-12}\,\text{Дж},
         \\
        E_{\text{кин}} &= \frac{mc^2}{\sqrt{1 - \frac{v^2}{c^2}}} - mc^2
            = mc^2 \cbr{\frac 1{\sqrt{1 - \frac{v^2}{c^2}}} - 1} \approx \\
            &\approx \cbr{9{,}1 \cdot 10^{-31}\,\text{кг} \cdot \sqr{3 \cdot 10^{8}\,\frac{\text{м}}{\text{с}}}}
            \cdot \cbr{\frac 1{\sqrt{1 - 0{,}85^2}} - 1}
            \approx 73{,}6 \cdot 10^{-15}\,\text{Дж},
         \\
        p &= \frac{mv}{\sqrt{1 - \frac{v^2}{c^2}}}
            \approx \frac{9{,}1 \cdot 10^{-31}\,\text{кг} \cdot 0{,}85 \cdot 3 \cdot 10^{8}\,\frac{\text{м}}{\text{с}}}{\sqrt{1 - 0{,}85^2}}
            \approx 0{,}441 \cdot 10^{-21}\,\frac{\text{кг}\cdot\text{м}}{\text{с}}.
    \end{align*}
}
\solutionspace{100pt}

\tasknumber{7}%
\task{%
    Кинетическая энергия частицы космических лучей в три раза превышает её энергию покоя.
    Определить отношение скорости частицы к скорости света.
}
\answer{%
    \begin{align*}
    E &= E_0 + E_{\text{кин}} \\
    E &= \frac{E_0}{\sqrt{1 - \frac{v^2}{c^2}}}\implies \sqrt{1 - \frac{v^2}{c^2}} = \frac{E_0}{E}\implies \frac{v^2}{c^2} = 1 - \sqr{\frac{E_0}{E}} \implies \\
    \implies \frac vc &= \sqrt{1 - \sqr{\frac{E_0}{E}}} = \sqrt{1 - \sqr{\frac{E_0}{E_0 + E_{\text{кин}} }}} \approx 0{,}968.
    \end{align*}
}
\solutionspace{80pt}

\tasknumber{8}%
\task{%
    Некоторая частица, пройдя ускоряющую разность потенциалов, приобрела импульс $3 \cdot 10^{-19}\,\frac{\text{кг}\cdot\text{м}}{\text{с}}$.
    Скорость частицы стала равной $1{,}8 \cdot 10^{8}\,\frac{\text{м}}{\text{с}}$.
    Найти массу частицы.
}
\answer{%
    $p = \frac{ mv }{\sqrt{1 - \frac{v^2}{c^2} }}\implies m = \frac pv \sqrt{1 - \frac{v^2}{c^2}}= \frac {3 \cdot 10^{-19}\,\frac{\text{кг}\cdot\text{м}}{\text{с}}}{1{,}8 \cdot 10^{8}\,\frac{\text{м}}{\text{с}}} \sqrt{1 - \sqr{\frac{1{,}8 \cdot 10^{8}\,\frac{\text{м}}{\text{с}}}{3 \cdot 10^{8}\,\frac{\text{м}}{\text{с}}}} } \approx 1{,}33 \cdot 10^{-27}\,\text{кг}.$
}
\solutionspace{80pt}

\tasknumber{9}%
\task{%
    При какой скорости движения (в км/ч) релятивистское сокращение длины движущегося тела
    составит 10\%?
}
\answer{%
    \begin{align*}
    l_0 &= \frac l{\sqrt{1 - \frac{v^2}{c^2}}}
        \implies 1 - \frac{v^2}{c^2} = \sqr{\frac l{l_0}}
        \implies \frac v c = \sqrt{1 - \sqr{\frac l{l_0}}} \implies
         \\
        \implies v &= c\sqrt{1 - \sqr{\frac l{l_0}}}
        = 3 \cdot 10^{8}\,\frac{\text{м}}{\text{с}} \cdot \sqrt{1 - \sqr{\frac {l_0 - 0{,}10l_0}{l_0}}}
        = 3 \cdot 10^{8}\,\frac{\text{м}}{\text{с}} \cdot \sqrt{1 - \sqr{1 - 0{,}10}} \approx  \\
        &\approx 0{,}436c
        \approx 130{,}8 \cdot 10^{6}\,\frac{\text{м}}{\text{с}}
        \approx 471 \cdot 10^{6}\,\frac{\text{км}}{\text{ч}}.
    \end{align*}
}
\solutionspace{80pt}

\tasknumber{10}%
\task{%
    Стержень движется в продольном направлении с постоянной скоростью относительно инерциальной системы отсчёта.
    При каком значении скорости (в долях скорости света) длина стержня в этой системе отсчёта
    будет в  2{,}5  раза меньше его собственной длины?
}
\answer{%
    $l_0 = \frac l{\sqrt{1 - \frac{v^2}{c^2}}}\implies \sqrt{1 - \frac{v^2}{c^2}} = \frac{ l }{ l_0 }\implies \frac v c = \sqrt{1 - \sqr{\frac{ l }{ l_0 }}} \approx 0{,}917.$
}
\solutionspace{80pt}

\tasknumber{11}%
\task{%
    Какую скорость должно иметь движущееся тело, чтобы его продольные размеры уменьшились в пять раз?
    Скорость света $c = 3 \cdot 10^{8}\,\frac{\text{м}}{\text{с}}$.
}
\answer{%
    $l_0 = \frac l{\sqrt{1 - \frac{v^2}{c^2}}}\implies \sqrt{1 - \frac{v^2}{c^2}} = \frac{ l }{ l_0 }\implies v = c\sqrt{1 - \sqr{\frac{ l }{ l_0 }}} \approx 294 \cdot 10^{6}\,\frac{\text{м}}{\text{с}}.$
}


\variantsplitter


\addpersonalvariant{Евгений Васин}

\tasknumber{12}%
\task{%
    Время жизни мюона, измеренное наблюдателем, относительно которого мюон покоился, равно $\tau_0$
    Какое расстояние пролетит мюон в системе отсчёта, относительно которой он движется со скоростью $v$,
    сравнимой со скоростью света в вакууме $c$?
}
\answer{%
    $\ell = v\tau = v \frac{\tau_0}{\sqrt{1 - \frac{v^2}{c^2}}}$
}
\solutionspace{80pt}

\tasknumber{13}%
\task{%
    Если $c$ — скорость света в вакууме, то с какой скоростью должна двигаться нестабильная частица относительно наблюдателя,
    чтобы её время жизни было в три раза больше, чем у такой же, но покоящейся относительно наблюдателя частицы?
}
\answer{%
    $\tau = \frac{\tau_0}{\sqrt{1 - \frac{v^2}{c^2}}}\implies \sqrt{1 - \frac{v^2}{c^2}} = \frac{\tau_0}{\tau}\implies v = c\sqrt{1 - \sqr{\frac{\tau_0}{\tau}} } \approx 283 \cdot 10^{6}\,\frac{\text{м}}{\text{с}}.$
}
\solutionspace{80pt}

\tasknumber{14}%
\task{%
    Время жизни нестабильной частицы, входящего в состав космических лучей, измеренное земным наблюдателем,
    относительно которого частица двигалась со скоростью, составляющей 75\% скорости света в вакууме, оказалось равным $6{,}4\,\text{мкс}$.
    Каково время жизни частицы, покоящейся относительно наблюдателя?
}
\answer{%
    $t = \frac{t_0}{\sqrt{1 - \frac{v^2}{c^2}}}\implies t_0 = t\sqrt{1 - \frac{v^2}{c^2}} \approx 4{,}2 \cdot 10^{-6}\,\text{с}.$
}
\solutionspace{80pt}

\tasknumber{15}%
\task{%
    Частица увеличила в ускорителе свою скорость с $0{,}02c$ до $0{,}80c$.
    Во сколько раз выросла её кинетическая энергия?
}
\answer{%
    \begin{align*}
    E_{\text{кин.}} &= E - E_0 = \frac{mc^2}{\sqrt{1 - \frac{v^2}{c^2}}} - mc^2 = mc^2\cbr{ \frac1{\sqrt{1 - \frac{v^2}{c^2} }} - 1}.
    \\
    \frac{E_{\text{кин.
    2}}}{E_{\text{кин.
    1}}} &= \frac{\frac1{\sqrt{1 - \frac{v_2^2}{c^2} }} - 1}{\frac1{\sqrt{1 - \frac{v_1^2}{c^2} }} - 1}\approx 3332{,}33
    \end{align*}
}
\solutionspace{120pt}

\tasknumber{16}%
\task{%
    Для частицы, движущейся с релятивистской скоростью,
    выразите $p$ и $v$ через $c$, $E_0$ и $E_\text{кин}$,
    где $E_\text{кин}$~--- кинетическая энергия частицы,
    а $E_0$, $p$ и $v$~--- её энергия покоя, импульс и скорость.
}
\answer{%
    \begin{align*}
    E_\text{кин}, E_0:\quad&E = E_\text{кин} + E_0 = \frac{E_0}{\sqrt{1 - \frac{v^2}{c^2}}} \implies \sqrt{1 - \frac{v^2}{c^2}} = \frac{E_0}{{E_0} + {E_\text{кин}}} \implies v = c\sqrt{1 - \sqr{\frac{E_0}{{E_0} + {E_\text{кин}}}}} \\
    &p = \frac{mv}{\sqrt{1 - \frac{v^2}{c^2}}} = \frac{E_0}{c^2} \cdot \sqrt{1 - \sqr{\frac{E_0}{{E_0} + {E_\text{кин}}}}} \cdot \frac{{E_\text{кин}} + {E_0}}{E_0} = \frac{E_0}{c^2} \cdot \sqrt{\sqr{\frac{{E_\text{кин}} + {E_0}}{E_0}} - 1}.
    \\
    E_\text{кин}, p:\quad&E_\text{кин} = E - E_0 = mc^2\cbr{\frac 1{\sqrt{1 - \frac{v^2}{c^2}}} - 1}, p = \frac{mv}{\sqrt{1 - \frac{v^2}{c^2}}} \implies \frac{E_\text{кин}}{p} = \frac{\frac 1{\sqrt{1 - \frac{v^2}{c^2}}} - 1}{\sqrt{1 - \frac{v^2}{c^2}}} \implies v = \ldots \\
    &E_0 = E - E_\text{кин} = \frac{E_0}{\sqrt{1 - \frac{v^2}{c^2}}} - E_\text{кин} \implies E_0 = \frac{E_\text{кин}}{\frac 1{\sqrt{1 - \frac{v^2}{c^2}}} - 1} = \ldots \\
    E_\text{кин}, v:\quad&E_\text{кин} = E - E_0 = mc^2\cbr{\frac 1{\sqrt{1 - \frac{v^2}{c^2}}} - 1} \implies m = \frac{E_\text{кин}}{c^2\cbr{\frac 1{\sqrt{1 - \frac{v^2}{c^2}}} - 1}} \\
    &E_0 = mc^2 = \frac{E_\text{кин}}{\frac 1{\sqrt{1 - \frac{v^2}{c^2}}} - 1} \\
    &p = \frac{mv}{\sqrt{1 - \frac{v^2}{c^2}}} = \frac{E_\text{кин}}{c^2\cbr{\frac 1{\sqrt{1 - \frac{v^2}{c^2}}} - 1}} \cdot \frac{v}{\sqrt{1 - \frac{v^2}{c^2}}} = \frac{{E_\text{кин}} v}{c^2\cbr{1 - {\sqrt{1 - \frac{v^2}{c^2}}}}} \\
    E_0, p:\quad&E_0 = mc^2, \quad p = \frac{mv}{\sqrt{1 - \frac{v^2}{c^2}}} \implies \frac{E_0}{p} = \frac{c^2}v{\sqrt{1 - \frac{v^2}{c^2}}} = c\sqrt{\frac{c^2}{v^2} - 1} \\
    &\sqr{\frac{E_0}{pc}} = \frac{c^2}{v^2} - 1 \implies \frac{v^2}{c^2} = \frac 1{1 + \frac{E_0^2}{p^2c^2}} \implies v = \frac c{\sqrt{1 + \frac{E_0^2}{p^2c^2}}} \\
    &{E_\text{кин}} = E - E_0 = \sqrt{E_0^2 + p^2c^2} - E_0 \\
    E_0, v:\quad&E_0 = mc^2 \implies m = \frac{E_0}{c^2} \qquad p = \frac{mv}{\sqrt{1 - \frac{v^2}{c^2}}} = \frac{E_0}{c^2} \cdot \frac{v}{\sqrt{1 - \frac{v^2}{c^2}}} \\
    &E_\text{кин}= mc^2\cbr{\frac 1{\sqrt{1 - \frac{v^2}{c^2}}} - 1} = \frac{E_0}{c^2}\cbr{\frac 1{\sqrt{1 - \frac{v^2}{c^2}}} - 1} \\
    p, v:\quad&p = \frac{mv}{\sqrt{1 - \frac{v^2}{c^2}}} \implies m = \frac p v {\sqrt{1 - \frac{v^2}{c^2}}} \implies E_0 = mc^2 =\frac {pc^2} v {\sqrt{1 - \frac{v^2}{c^2}}} \\
    &E_\text{кин} = mc^2\cbr{\frac 1{\sqrt{1 - \frac{v^2}{c^2}}} - 1} = \frac p v {\sqrt{1 - \frac{v^2}{c^2}}}\cbr{\frac 1{\sqrt{1 - \frac{v^2}{c^2}}} - 1} = \frac p v \cbr{1 - {\sqrt{1 - \frac{v^2}{c^2}}}}
    \end{align*}
}

\variantsplitter

\addpersonalvariant{Вячеслав Волохов}

\tasknumber{1}%
\task{%
    Запишите
    \begin{itemize}
        \item постулаты специальной теории относительности,
        \item пример релятивистского эффекта, обнаружимый при скоростях гораздо меньше скорости света.
    \end{itemize}
}
\solutionspace{120pt}

\tasknumber{2}%
\task{%
    Запишите формулу для ...
    \begin{itemize}
        \item релятивистского сжатия,
        \item классической полной механической энергии тела,
        \item релятивистского импульса тела,
        \item релятивистской кинетической энергии,
        \item связь между релятивистским импульсом и релятивистской энергией.
    \end{itemize}
    Обязательно подпишите все физические величины.
}
\solutionspace{150pt}

\tasknumber{3}%
\task{%
    Протон движется со скоростью $0{,}8\,c$, где $c$~--- скорость света в вакууме.
    Каково при этом отношение кинетической энергии частицы $E_\text{кин.}$ к его энергии покоя $E_0$?
}
\answer{%
    \begin{align*}
    E &= \frac{E_0}{\sqrt{1 - \frac{v^2}{c^2}}}
            \implies \frac E{E_0}
                = \frac 1{\sqrt{1 - \frac{v^2}{c^2}}}
                = \frac 1{\sqrt{1 - \sqr{0{,}8}}}
                \approx 1{,}667,
         \\
        {E_{\text{кин}}} &= E - E_0
            \implies \frac{E_{\text{кин}}}{E_0}
                = \frac E{E_0} - 1
                = \frac 1{\sqrt{1 - \frac{v^2}{c^2}}} - 1
                = \frac 1{\sqrt{1 - \sqr{0{,}8}}} - 1
                \approx 0{,}667.
    \end{align*}
}
\solutionspace{80pt}

\tasknumber{4}%
\task{%
    Полная энергия релятивистской частицы в три раза больше её энергии покоя.
    Найти скорость этой частицы: в долях $c$ и численное значение.
    Скорость света в вакууме $c = 3 \cdot 10^{8}\,\frac{\text{м}}{\text{с}}$.
}
\answer{%
    \begin{align*}
    E &= \frac{E_0}{\sqrt{1 - \frac{v^2}{c^2}}}\implies \sqrt{1 - \frac{v^2}{c^2}} = \frac{E_0}{E}\implies \frac{v^2}{c^2} = 1 - \sqr{\frac{E_0}{E}}\implies v = c \sqrt{1 - \sqr{\frac{E_0}{E}}} \approx 0{,}943c \approx 283 \cdot 10^{6}\,\frac{\text{м}}{\text{с}}.
    \end{align*}
}
\solutionspace{80pt}

\tasknumber{5}%
\task{%
    Кинетическая энергия релятивистской частицы в три раза больше её энергии покоя.
    Найти скорость этой частицы.
    Скорость света в вакууме $c = 3 \cdot 10^{8}\,\frac{\text{м}}{\text{с}}$.
}
\answer{%
    \begin{align*}
    E &= E_0 + E_{\text{кин}} \\
    E &= \frac{E_0}{\sqrt{1 - \frac{v^2}{c^2}}}\implies \sqrt{1 - \frac{v^2}{c^2}} = \frac{E_0}{E}\implies \frac{v^2}{c^2} = 1 - \sqr{\frac{E_0}{E}} \implies \\
    \implies &v = c \sqrt{1 - \sqr{\frac{E_0}{E}}} = c \sqrt{1 - \sqr{\frac{E_0}{E_0 + E_{\text{кин}} }}} = c \sqrt{1 - \frac 1 {\sqr{ 1 + \frac{E_{\text{кин}}}{E_0} }} }\approx 0{,}968c \approx 290 \cdot 10^{6}\,\frac{\text{м}}{\text{с}}.
    \end{align*}
}


\variantsplitter


\addpersonalvariant{Вячеслав Волохов}

\tasknumber{6}%
\task{%
    Протон движется со скоростью $0{,}75\,c$, где $c$~--- скорость света в вакууме.
    Определите его полную энергию (в ответе приведите формулу и укажите численное значение).
}
\answer{%
    \begin{align*}
    E &= \frac{mc^2}{\sqrt{1 - \frac{v^2}{c^2}}}
            \approx \frac{1{,}673 \cdot 10^{-27}\,\text{кг} \cdot \sqr{3 \cdot 10^{8}\,\frac{\text{м}}{\text{с}}}}{\sqrt{1 - 0{,}75^2}}
            \approx 0{,}2276 \cdot 10^{-9}\,\text{Дж},
         \\
        E_{\text{кин}} &= \frac{mc^2}{\sqrt{1 - \frac{v^2}{c^2}}} - mc^2
            = mc^2 \cbr{\frac 1{\sqrt{1 - \frac{v^2}{c^2}}} - 1} \approx \\
            &\approx \cbr{1{,}673 \cdot 10^{-27}\,\text{кг} \cdot \sqr{3 \cdot 10^{8}\,\frac{\text{м}}{\text{с}}}}
            \cdot \cbr{\frac 1{\sqrt{1 - 0{,}75^2}} - 1}
            \approx 77{,}05 \cdot 10^{-12}\,\text{Дж},
         \\
        p &= \frac{mv}{\sqrt{1 - \frac{v^2}{c^2}}}
            \approx \frac{1{,}673 \cdot 10^{-27}\,\text{кг} \cdot 0{,}75 \cdot 3 \cdot 10^{8}\,\frac{\text{м}}{\text{с}}}{\sqrt{1 - 0{,}75^2}}
            \approx 0{,}5690 \cdot 10^{-18}\,\frac{\text{кг}\cdot\text{м}}{\text{с}}.
    \end{align*}
}
\solutionspace{100pt}

\tasknumber{7}%
\task{%
    Кинетическая энергия частицы космических лучей в три раза превышает её энергию покоя.
    Определить отношение скорости частицы к скорости света.
}
\answer{%
    \begin{align*}
    E &= E_0 + E_{\text{кин}} \\
    E &= \frac{E_0}{\sqrt{1 - \frac{v^2}{c^2}}}\implies \sqrt{1 - \frac{v^2}{c^2}} = \frac{E_0}{E}\implies \frac{v^2}{c^2} = 1 - \sqr{\frac{E_0}{E}} \implies \\
    \implies \frac vc &= \sqrt{1 - \sqr{\frac{E_0}{E}}} = \sqrt{1 - \sqr{\frac{E_0}{E_0 + E_{\text{кин}} }}} \approx 0{,}968.
    \end{align*}
}
\solutionspace{80pt}

\tasknumber{8}%
\task{%
    Некоторая частица, пройдя ускоряющую разность потенциалов, приобрела импульс $3 \cdot 10^{-19}\,\frac{\text{кг}\cdot\text{м}}{\text{с}}$.
    Скорость частицы стала равной $1{,}8 \cdot 10^{8}\,\frac{\text{м}}{\text{с}}$.
    Найти массу частицы.
}
\answer{%
    $p = \frac{ mv }{\sqrt{1 - \frac{v^2}{c^2} }}\implies m = \frac pv \sqrt{1 - \frac{v^2}{c^2}}= \frac {3 \cdot 10^{-19}\,\frac{\text{кг}\cdot\text{м}}{\text{с}}}{1{,}8 \cdot 10^{8}\,\frac{\text{м}}{\text{с}}} \sqrt{1 - \sqr{\frac{1{,}8 \cdot 10^{8}\,\frac{\text{м}}{\text{с}}}{3 \cdot 10^{8}\,\frac{\text{м}}{\text{с}}}} } \approx 1{,}33 \cdot 10^{-27}\,\text{кг}.$
}
\solutionspace{80pt}

\tasknumber{9}%
\task{%
    При какой скорости движения (в км/ч) релятивистское сокращение длины движущегося тела
    составит 50\%?
}
\answer{%
    \begin{align*}
    l_0 &= \frac l{\sqrt{1 - \frac{v^2}{c^2}}}
        \implies 1 - \frac{v^2}{c^2} = \sqr{\frac l{l_0}}
        \implies \frac v c = \sqrt{1 - \sqr{\frac l{l_0}}} \implies
         \\
        \implies v &= c\sqrt{1 - \sqr{\frac l{l_0}}}
        = 3 \cdot 10^{8}\,\frac{\text{м}}{\text{с}} \cdot \sqrt{1 - \sqr{\frac {l_0 - 0{,}50l_0}{l_0}}}
        = 3 \cdot 10^{8}\,\frac{\text{м}}{\text{с}} \cdot \sqrt{1 - \sqr{1 - 0{,}50}} \approx  \\
        &\approx 0{,}866c
        \approx 260 \cdot 10^{6}\,\frac{\text{м}}{\text{с}}
        \approx 935 \cdot 10^{6}\,\frac{\text{км}}{\text{ч}}.
    \end{align*}
}
\solutionspace{80pt}

\tasknumber{10}%
\task{%
    Стержень движется в продольном направлении с постоянной скоростью относительно инерциальной системы отсчёта.
    При каком значении скорости (в долях скорости света) длина стержня в этой системе отсчёта
    будет в  3  раза меньше его собственной длины?
}
\answer{%
    $l_0 = \frac l{\sqrt{1 - \frac{v^2}{c^2}}}\implies \sqrt{1 - \frac{v^2}{c^2}} = \frac{ l }{ l_0 }\implies \frac v c = \sqrt{1 - \sqr{\frac{ l }{ l_0 }}} \approx 0{,}943.$
}
\solutionspace{80pt}

\tasknumber{11}%
\task{%
    Какую скорость должно иметь движущееся тело, чтобы его продольные размеры уменьшились в два раза?
    Скорость света $c = 3 \cdot 10^{8}\,\frac{\text{м}}{\text{с}}$.
}
\answer{%
    $l_0 = \frac l{\sqrt{1 - \frac{v^2}{c^2}}}\implies \sqrt{1 - \frac{v^2}{c^2}} = \frac{ l }{ l_0 }\implies v = c\sqrt{1 - \sqr{\frac{ l }{ l_0 }}} \approx 260 \cdot 10^{6}\,\frac{\text{м}}{\text{с}}.$
}


\variantsplitter


\addpersonalvariant{Вячеслав Волохов}

\tasknumber{12}%
\task{%
    Время жизни мюона, измеренное наблюдателем, относительно которого мюон покоился, равно $\tau_0$
    Какое расстояние пролетит мюон в системе отсчёта, относительно которой он движется со скоростью $v$,
    сравнимой со скоростью света в вакууме $c$?
}
\answer{%
    $\ell = v\tau = v \frac{\tau_0}{\sqrt{1 - \frac{v^2}{c^2}}}$
}
\solutionspace{80pt}

\tasknumber{13}%
\task{%
    Если $c$ — скорость света в вакууме, то с какой скоростью должна двигаться нестабильная частица относительно наблюдателя,
    чтобы её время жизни было в семь раз больше, чем у такой же, но покоящейся относительно наблюдателя частицы?
}
\answer{%
    $\tau = \frac{\tau_0}{\sqrt{1 - \frac{v^2}{c^2}}}\implies \sqrt{1 - \frac{v^2}{c^2}} = \frac{\tau_0}{\tau}\implies v = c\sqrt{1 - \sqr{\frac{\tau_0}{\tau}} } \approx 297 \cdot 10^{6}\,\frac{\text{м}}{\text{с}}.$
}
\solutionspace{80pt}

\tasknumber{14}%
\task{%
    Время жизни нестабильной частицы, входящего в состав космических лучей, измеренное земным наблюдателем,
    относительно которого частица двигалась со скоростью, составляющей 75\% скорости света в вакууме, оказалось равным $7{,}1\,\text{мкс}$.
    Каково время жизни частицы, покоящейся относительно наблюдателя?
}
\answer{%
    $t = \frac{t_0}{\sqrt{1 - \frac{v^2}{c^2}}}\implies t_0 = t\sqrt{1 - \frac{v^2}{c^2}} \approx 4{,}7 \cdot 10^{-6}\,\text{с}.$
}
\solutionspace{80pt}

\tasknumber{15}%
\task{%
    Частица увеличила в ускорителе свою скорость с $0{,}04c$ до $0{,}50c$.
    Во сколько раз выросла её кинетическая энергия?
}
\answer{%
    \begin{align*}
    E_{\text{кин.}} &= E - E_0 = \frac{mc^2}{\sqrt{1 - \frac{v^2}{c^2}}} - mc^2 = mc^2\cbr{ \frac1{\sqrt{1 - \frac{v^2}{c^2} }} - 1}.
    \\
    \frac{E_{\text{кин.
    2}}}{E_{\text{кин.
    1}}} &= \frac{\frac1{\sqrt{1 - \frac{v_2^2}{c^2} }} - 1}{\frac1{\sqrt{1 - \frac{v_1^2}{c^2} }} - 1}\approx 193{,}14
    \end{align*}
}
\solutionspace{120pt}

\tasknumber{16}%
\task{%
    Для частицы, движущейся с релятивистской скоростью,
    выразите $E_\text{кин}$ и $p$ через $c$, $E_0$ и $v$,
    где $E_\text{кин}$~--- кинетическая энергия частицы,
    а $E_0$, $p$ и $v$~--- её энергия покоя, импульс и скорость.
}
\answer{%
    \begin{align*}
    E_\text{кин}, E_0:\quad&E = E_\text{кин} + E_0 = \frac{E_0}{\sqrt{1 - \frac{v^2}{c^2}}} \implies \sqrt{1 - \frac{v^2}{c^2}} = \frac{E_0}{{E_0} + {E_\text{кин}}} \implies v = c\sqrt{1 - \sqr{\frac{E_0}{{E_0} + {E_\text{кин}}}}} \\
    &p = \frac{mv}{\sqrt{1 - \frac{v^2}{c^2}}} = \frac{E_0}{c^2} \cdot \sqrt{1 - \sqr{\frac{E_0}{{E_0} + {E_\text{кин}}}}} \cdot \frac{{E_\text{кин}} + {E_0}}{E_0} = \frac{E_0}{c^2} \cdot \sqrt{\sqr{\frac{{E_\text{кин}} + {E_0}}{E_0}} - 1}.
    \\
    E_\text{кин}, p:\quad&E_\text{кин} = E - E_0 = mc^2\cbr{\frac 1{\sqrt{1 - \frac{v^2}{c^2}}} - 1}, p = \frac{mv}{\sqrt{1 - \frac{v^2}{c^2}}} \implies \frac{E_\text{кин}}{p} = \frac{\frac 1{\sqrt{1 - \frac{v^2}{c^2}}} - 1}{\sqrt{1 - \frac{v^2}{c^2}}} \implies v = \ldots \\
    &E_0 = E - E_\text{кин} = \frac{E_0}{\sqrt{1 - \frac{v^2}{c^2}}} - E_\text{кин} \implies E_0 = \frac{E_\text{кин}}{\frac 1{\sqrt{1 - \frac{v^2}{c^2}}} - 1} = \ldots \\
    E_\text{кин}, v:\quad&E_\text{кин} = E - E_0 = mc^2\cbr{\frac 1{\sqrt{1 - \frac{v^2}{c^2}}} - 1} \implies m = \frac{E_\text{кин}}{c^2\cbr{\frac 1{\sqrt{1 - \frac{v^2}{c^2}}} - 1}} \\
    &E_0 = mc^2 = \frac{E_\text{кин}}{\frac 1{\sqrt{1 - \frac{v^2}{c^2}}} - 1} \\
    &p = \frac{mv}{\sqrt{1 - \frac{v^2}{c^2}}} = \frac{E_\text{кин}}{c^2\cbr{\frac 1{\sqrt{1 - \frac{v^2}{c^2}}} - 1}} \cdot \frac{v}{\sqrt{1 - \frac{v^2}{c^2}}} = \frac{{E_\text{кин}} v}{c^2\cbr{1 - {\sqrt{1 - \frac{v^2}{c^2}}}}} \\
    E_0, p:\quad&E_0 = mc^2, \quad p = \frac{mv}{\sqrt{1 - \frac{v^2}{c^2}}} \implies \frac{E_0}{p} = \frac{c^2}v{\sqrt{1 - \frac{v^2}{c^2}}} = c\sqrt{\frac{c^2}{v^2} - 1} \\
    &\sqr{\frac{E_0}{pc}} = \frac{c^2}{v^2} - 1 \implies \frac{v^2}{c^2} = \frac 1{1 + \frac{E_0^2}{p^2c^2}} \implies v = \frac c{\sqrt{1 + \frac{E_0^2}{p^2c^2}}} \\
    &{E_\text{кин}} = E - E_0 = \sqrt{E_0^2 + p^2c^2} - E_0 \\
    E_0, v:\quad&E_0 = mc^2 \implies m = \frac{E_0}{c^2} \qquad p = \frac{mv}{\sqrt{1 - \frac{v^2}{c^2}}} = \frac{E_0}{c^2} \cdot \frac{v}{\sqrt{1 - \frac{v^2}{c^2}}} \\
    &E_\text{кин}= mc^2\cbr{\frac 1{\sqrt{1 - \frac{v^2}{c^2}}} - 1} = \frac{E_0}{c^2}\cbr{\frac 1{\sqrt{1 - \frac{v^2}{c^2}}} - 1} \\
    p, v:\quad&p = \frac{mv}{\sqrt{1 - \frac{v^2}{c^2}}} \implies m = \frac p v {\sqrt{1 - \frac{v^2}{c^2}}} \implies E_0 = mc^2 =\frac {pc^2} v {\sqrt{1 - \frac{v^2}{c^2}}} \\
    &E_\text{кин} = mc^2\cbr{\frac 1{\sqrt{1 - \frac{v^2}{c^2}}} - 1} = \frac p v {\sqrt{1 - \frac{v^2}{c^2}}}\cbr{\frac 1{\sqrt{1 - \frac{v^2}{c^2}}} - 1} = \frac p v \cbr{1 - {\sqrt{1 - \frac{v^2}{c^2}}}}
    \end{align*}
}

\variantsplitter

\addpersonalvariant{Герман Говоров}

\tasknumber{1}%
\task{%
    Запишите
    \begin{itemize}
        \item постулаты специальной теории относительности,
        \item пример релятивистского эффекта, обнаружимый при скоростях гораздо меньше скорости света.
    \end{itemize}
}
\solutionspace{120pt}

\tasknumber{2}%
\task{%
    Запишите формулу для ...
    \begin{itemize}
        \item релятивистского замедления времени,
        \item классического импульса,
        \item релятивистского импульса тела,
        \item энергии покоя тела,
        \item связь между релятивистским импульсом и релятивистской энергией.
    \end{itemize}
    Обязательно подпишите все физические величины.
}
\solutionspace{150pt}

\tasknumber{3}%
\task{%
    Протон движется со скоростью $0{,}9\,c$, где $c$~--- скорость света в вакууме.
    Каково при этом отношение полной энергии частицы $E$ к его энергии покоя $E_0$?
}
\answer{%
    \begin{align*}
    E &= \frac{E_0}{\sqrt{1 - \frac{v^2}{c^2}}}
            \implies \frac E{E_0}
                = \frac 1{\sqrt{1 - \frac{v^2}{c^2}}}
                = \frac 1{\sqrt{1 - \sqr{0{,}9}}}
                \approx 2{,}294,
         \\
        {E_{\text{кин}}} &= E - E_0
            \implies \frac{E_{\text{кин}}}{E_0}
                = \frac E{E_0} - 1
                = \frac 1{\sqrt{1 - \frac{v^2}{c^2}}} - 1
                = \frac 1{\sqrt{1 - \sqr{0{,}9}}} - 1
                \approx 1{,}294.
    \end{align*}
}
\solutionspace{80pt}

\tasknumber{4}%
\task{%
    Полная энергия релятивистской частицы в пять раз больше её энергии покоя.
    Найти скорость этой частицы: в долях $c$ и численное значение.
    Скорость света в вакууме $c = 3 \cdot 10^{8}\,\frac{\text{м}}{\text{с}}$.
}
\answer{%
    \begin{align*}
    E &= \frac{E_0}{\sqrt{1 - \frac{v^2}{c^2}}}\implies \sqrt{1 - \frac{v^2}{c^2}} = \frac{E_0}{E}\implies \frac{v^2}{c^2} = 1 - \sqr{\frac{E_0}{E}}\implies v = c \sqrt{1 - \sqr{\frac{E_0}{E}}} \approx 0{,}980c \approx 294 \cdot 10^{6}\,\frac{\text{м}}{\text{с}}.
    \end{align*}
}
\solutionspace{80pt}

\tasknumber{5}%
\task{%
    Кинетическая энергия релятивистской частицы в пять раз больше её энергии покоя.
    Найти скорость этой частицы.
    Скорость света в вакууме $c = 3 \cdot 10^{8}\,\frac{\text{м}}{\text{с}}$.
}
\answer{%
    \begin{align*}
    E &= E_0 + E_{\text{кин}} \\
    E &= \frac{E_0}{\sqrt{1 - \frac{v^2}{c^2}}}\implies \sqrt{1 - \frac{v^2}{c^2}} = \frac{E_0}{E}\implies \frac{v^2}{c^2} = 1 - \sqr{\frac{E_0}{E}} \implies \\
    \implies &v = c \sqrt{1 - \sqr{\frac{E_0}{E}}} = c \sqrt{1 - \sqr{\frac{E_0}{E_0 + E_{\text{кин}} }}} = c \sqrt{1 - \frac 1 {\sqr{ 1 + \frac{E_{\text{кин}}}{E_0} }} }\approx 0{,}986c \approx 296 \cdot 10^{6}\,\frac{\text{м}}{\text{с}}.
    \end{align*}
}


\variantsplitter


\addpersonalvariant{Герман Говоров}

\tasknumber{6}%
\task{%
    Протон движется со скоростью $0{,}75\,c$, где $c$~--- скорость света в вакууме.
    Определите его полную энергию (в ответе приведите формулу и укажите численное значение).
}
\answer{%
    \begin{align*}
    E &= \frac{mc^2}{\sqrt{1 - \frac{v^2}{c^2}}}
            \approx \frac{1{,}673 \cdot 10^{-27}\,\text{кг} \cdot \sqr{3 \cdot 10^{8}\,\frac{\text{м}}{\text{с}}}}{\sqrt{1 - 0{,}75^2}}
            \approx 0{,}2276 \cdot 10^{-9}\,\text{Дж},
         \\
        E_{\text{кин}} &= \frac{mc^2}{\sqrt{1 - \frac{v^2}{c^2}}} - mc^2
            = mc^2 \cbr{\frac 1{\sqrt{1 - \frac{v^2}{c^2}}} - 1} \approx \\
            &\approx \cbr{1{,}673 \cdot 10^{-27}\,\text{кг} \cdot \sqr{3 \cdot 10^{8}\,\frac{\text{м}}{\text{с}}}}
            \cdot \cbr{\frac 1{\sqrt{1 - 0{,}75^2}} - 1}
            \approx 77{,}05 \cdot 10^{-12}\,\text{Дж},
         \\
        p &= \frac{mv}{\sqrt{1 - \frac{v^2}{c^2}}}
            \approx \frac{1{,}673 \cdot 10^{-27}\,\text{кг} \cdot 0{,}75 \cdot 3 \cdot 10^{8}\,\frac{\text{м}}{\text{с}}}{\sqrt{1 - 0{,}75^2}}
            \approx 0{,}5690 \cdot 10^{-18}\,\frac{\text{кг}\cdot\text{м}}{\text{с}}.
    \end{align*}
}
\solutionspace{100pt}

\tasknumber{7}%
\task{%
    Кинетическая энергия частицы космических лучей в пять раз превышает её энергию покоя.
    Определить отношение скорости частицы к скорости света.
}
\answer{%
    \begin{align*}
    E &= E_0 + E_{\text{кин}} \\
    E &= \frac{E_0}{\sqrt{1 - \frac{v^2}{c^2}}}\implies \sqrt{1 - \frac{v^2}{c^2}} = \frac{E_0}{E}\implies \frac{v^2}{c^2} = 1 - \sqr{\frac{E_0}{E}} \implies \\
    \implies \frac vc &= \sqrt{1 - \sqr{\frac{E_0}{E}}} = \sqrt{1 - \sqr{\frac{E_0}{E_0 + E_{\text{кин}} }}} \approx 0{,}986.
    \end{align*}
}
\solutionspace{80pt}

\tasknumber{8}%
\task{%
    Некоторая частица, пройдя ускоряющую разность потенциалов, приобрела импульс $4{,}2 \cdot 10^{-19}\,\frac{\text{кг}\cdot\text{м}}{\text{с}}$.
    Скорость частицы стала равной $2{,}4 \cdot 10^{8}\,\frac{\text{м}}{\text{с}}$.
    Найти массу частицы.
}
\answer{%
    $p = \frac{ mv }{\sqrt{1 - \frac{v^2}{c^2} }}\implies m = \frac pv \sqrt{1 - \frac{v^2}{c^2}}= \frac {4{,}2 \cdot 10^{-19}\,\frac{\text{кг}\cdot\text{м}}{\text{с}}}{2{,}4 \cdot 10^{8}\,\frac{\text{м}}{\text{с}}} \sqrt{1 - \sqr{\frac{2{,}4 \cdot 10^{8}\,\frac{\text{м}}{\text{с}}}{3 \cdot 10^{8}\,\frac{\text{м}}{\text{с}}}} } \approx 1{,}050 \cdot 10^{-27}\,\text{кг}.$
}
\solutionspace{80pt}

\tasknumber{9}%
\task{%
    При какой скорости движения (в м/с) релятивистское сокращение длины движущегося тела
    составит 10\%?
}
\answer{%
    \begin{align*}
    l_0 &= \frac l{\sqrt{1 - \frac{v^2}{c^2}}}
        \implies 1 - \frac{v^2}{c^2} = \sqr{\frac l{l_0}}
        \implies \frac v c = \sqrt{1 - \sqr{\frac l{l_0}}} \implies
         \\
        \implies v &= c\sqrt{1 - \sqr{\frac l{l_0}}}
        = 3 \cdot 10^{8}\,\frac{\text{м}}{\text{с}} \cdot \sqrt{1 - \sqr{\frac {l_0 - 0{,}10l_0}{l_0}}}
        = 3 \cdot 10^{8}\,\frac{\text{м}}{\text{с}} \cdot \sqrt{1 - \sqr{1 - 0{,}10}} \approx  \\
        &\approx 0{,}436c
        \approx 130{,}8 \cdot 10^{6}\,\frac{\text{м}}{\text{с}}
        \approx 471 \cdot 10^{6}\,\frac{\text{км}}{\text{ч}}.
    \end{align*}
}
\solutionspace{80pt}

\tasknumber{10}%
\task{%
    Стержень движется в продольном направлении с постоянной скоростью относительно инерциальной системы отсчёта.
    При каком значении скорости (в долях скорости света) длина стержня в этой системе отсчёта
    будет в  3  раза меньше его собственной длины?
}
\answer{%
    $l_0 = \frac l{\sqrt{1 - \frac{v^2}{c^2}}}\implies \sqrt{1 - \frac{v^2}{c^2}} = \frac{ l }{ l_0 }\implies \frac v c = \sqrt{1 - \sqr{\frac{ l }{ l_0 }}} \approx 0{,}943.$
}
\solutionspace{80pt}

\tasknumber{11}%
\task{%
    Какую скорость должно иметь движущееся тело, чтобы его продольные размеры уменьшились в пять раз?
    Скорость света $c = 3 \cdot 10^{8}\,\frac{\text{м}}{\text{с}}$.
}
\answer{%
    $l_0 = \frac l{\sqrt{1 - \frac{v^2}{c^2}}}\implies \sqrt{1 - \frac{v^2}{c^2}} = \frac{ l }{ l_0 }\implies v = c\sqrt{1 - \sqr{\frac{ l }{ l_0 }}} \approx 294 \cdot 10^{6}\,\frac{\text{м}}{\text{с}}.$
}


\variantsplitter


\addpersonalvariant{Герман Говоров}

\tasknumber{12}%
\task{%
    Время жизни мюона, измеренное наблюдателем, относительно которого мюон покоился, равно $\tau_0$
    Какое расстояние пролетит мюон в системе отсчёта, относительно которой он движется со скоростью $v$,
    сравнимой со скоростью света в вакууме $c$?
}
\answer{%
    $\ell = v\tau = v \frac{\tau_0}{\sqrt{1 - \frac{v^2}{c^2}}}$
}
\solutionspace{80pt}

\tasknumber{13}%
\task{%
    Если $c$ — скорость света в вакууме, то с какой скоростью должна двигаться нестабильная частица относительно наблюдателя,
    чтобы её время жизни было в девять раз больше, чем у такой же, но покоящейся относительно наблюдателя частицы?
}
\answer{%
    $\tau = \frac{\tau_0}{\sqrt{1 - \frac{v^2}{c^2}}}\implies \sqrt{1 - \frac{v^2}{c^2}} = \frac{\tau_0}{\tau}\implies v = c\sqrt{1 - \sqr{\frac{\tau_0}{\tau}} } \approx 298 \cdot 10^{6}\,\frac{\text{м}}{\text{с}}.$
}
\solutionspace{80pt}

\tasknumber{14}%
\task{%
    Время жизни нестабильной частицы, входящего в состав космических лучей, измеренное земным наблюдателем,
    относительно которого частица двигалась со скоростью, составляющей 85\% скорости света в вакууме, оказалось равным $6{,}4\,\text{мкс}$.
    Каково время жизни частицы, покоящейся относительно наблюдателя?
}
\answer{%
    $t = \frac{t_0}{\sqrt{1 - \frac{v^2}{c^2}}}\implies t_0 = t\sqrt{1 - \frac{v^2}{c^2}} \approx 3{,}4 \cdot 10^{-6}\,\text{с}.$
}
\solutionspace{80pt}

\tasknumber{15}%
\task{%
    Частица увеличила в ускорителе свою скорость с $0{,}02c$ до $0{,}60c$.
    Во сколько раз выросла её кинетическая энергия?
}
\answer{%
    \begin{align*}
    E_{\text{кин.}} &= E - E_0 = \frac{mc^2}{\sqrt{1 - \frac{v^2}{c^2}}} - mc^2 = mc^2\cbr{ \frac1{\sqrt{1 - \frac{v^2}{c^2} }} - 1}.
    \\
    \frac{E_{\text{кин.
    2}}}{E_{\text{кин.
    1}}} &= \frac{\frac1{\sqrt{1 - \frac{v_2^2}{c^2} }} - 1}{\frac1{\sqrt{1 - \frac{v_1^2}{c^2} }} - 1}\approx 1249{,}62
    \end{align*}
}
\solutionspace{120pt}

\tasknumber{16}%
\task{%
    Для частицы, движущейся с релятивистской скоростью,
    выразите $v$ и $E_\text{кин}$ через $c$, $E_0$ и $p$,
    где $E_\text{кин}$~--- кинетическая энергия частицы,
    а $E_0$, $p$ и $v$~--- её энергия покоя, импульс и скорость.
}
\answer{%
    \begin{align*}
    E_\text{кин}, E_0:\quad&E = E_\text{кин} + E_0 = \frac{E_0}{\sqrt{1 - \frac{v^2}{c^2}}} \implies \sqrt{1 - \frac{v^2}{c^2}} = \frac{E_0}{{E_0} + {E_\text{кин}}} \implies v = c\sqrt{1 - \sqr{\frac{E_0}{{E_0} + {E_\text{кин}}}}} \\
    &p = \frac{mv}{\sqrt{1 - \frac{v^2}{c^2}}} = \frac{E_0}{c^2} \cdot \sqrt{1 - \sqr{\frac{E_0}{{E_0} + {E_\text{кин}}}}} \cdot \frac{{E_\text{кин}} + {E_0}}{E_0} = \frac{E_0}{c^2} \cdot \sqrt{\sqr{\frac{{E_\text{кин}} + {E_0}}{E_0}} - 1}.
    \\
    E_\text{кин}, p:\quad&E_\text{кин} = E - E_0 = mc^2\cbr{\frac 1{\sqrt{1 - \frac{v^2}{c^2}}} - 1}, p = \frac{mv}{\sqrt{1 - \frac{v^2}{c^2}}} \implies \frac{E_\text{кин}}{p} = \frac{\frac 1{\sqrt{1 - \frac{v^2}{c^2}}} - 1}{\sqrt{1 - \frac{v^2}{c^2}}} \implies v = \ldots \\
    &E_0 = E - E_\text{кин} = \frac{E_0}{\sqrt{1 - \frac{v^2}{c^2}}} - E_\text{кин} \implies E_0 = \frac{E_\text{кин}}{\frac 1{\sqrt{1 - \frac{v^2}{c^2}}} - 1} = \ldots \\
    E_\text{кин}, v:\quad&E_\text{кин} = E - E_0 = mc^2\cbr{\frac 1{\sqrt{1 - \frac{v^2}{c^2}}} - 1} \implies m = \frac{E_\text{кин}}{c^2\cbr{\frac 1{\sqrt{1 - \frac{v^2}{c^2}}} - 1}} \\
    &E_0 = mc^2 = \frac{E_\text{кин}}{\frac 1{\sqrt{1 - \frac{v^2}{c^2}}} - 1} \\
    &p = \frac{mv}{\sqrt{1 - \frac{v^2}{c^2}}} = \frac{E_\text{кин}}{c^2\cbr{\frac 1{\sqrt{1 - \frac{v^2}{c^2}}} - 1}} \cdot \frac{v}{\sqrt{1 - \frac{v^2}{c^2}}} = \frac{{E_\text{кин}} v}{c^2\cbr{1 - {\sqrt{1 - \frac{v^2}{c^2}}}}} \\
    E_0, p:\quad&E_0 = mc^2, \quad p = \frac{mv}{\sqrt{1 - \frac{v^2}{c^2}}} \implies \frac{E_0}{p} = \frac{c^2}v{\sqrt{1 - \frac{v^2}{c^2}}} = c\sqrt{\frac{c^2}{v^2} - 1} \\
    &\sqr{\frac{E_0}{pc}} = \frac{c^2}{v^2} - 1 \implies \frac{v^2}{c^2} = \frac 1{1 + \frac{E_0^2}{p^2c^2}} \implies v = \frac c{\sqrt{1 + \frac{E_0^2}{p^2c^2}}} \\
    &{E_\text{кин}} = E - E_0 = \sqrt{E_0^2 + p^2c^2} - E_0 \\
    E_0, v:\quad&E_0 = mc^2 \implies m = \frac{E_0}{c^2} \qquad p = \frac{mv}{\sqrt{1 - \frac{v^2}{c^2}}} = \frac{E_0}{c^2} \cdot \frac{v}{\sqrt{1 - \frac{v^2}{c^2}}} \\
    &E_\text{кин}= mc^2\cbr{\frac 1{\sqrt{1 - \frac{v^2}{c^2}}} - 1} = \frac{E_0}{c^2}\cbr{\frac 1{\sqrt{1 - \frac{v^2}{c^2}}} - 1} \\
    p, v:\quad&p = \frac{mv}{\sqrt{1 - \frac{v^2}{c^2}}} \implies m = \frac p v {\sqrt{1 - \frac{v^2}{c^2}}} \implies E_0 = mc^2 =\frac {pc^2} v {\sqrt{1 - \frac{v^2}{c^2}}} \\
    &E_\text{кин} = mc^2\cbr{\frac 1{\sqrt{1 - \frac{v^2}{c^2}}} - 1} = \frac p v {\sqrt{1 - \frac{v^2}{c^2}}}\cbr{\frac 1{\sqrt{1 - \frac{v^2}{c^2}}} - 1} = \frac p v \cbr{1 - {\sqrt{1 - \frac{v^2}{c^2}}}}
    \end{align*}
}

\variantsplitter

\addpersonalvariant{София Журавлёва}

\tasknumber{1}%
\task{%
    Запишите
    \begin{itemize}
        \item постулаты специальной теории относительности,
        \item пример релятивистского эффекта, обнаружимый при скоростях гораздо меньше скорости света.
    \end{itemize}
}
\solutionspace{120pt}

\tasknumber{2}%
\task{%
    Запишите формулу для ...
    \begin{itemize}
        \item релятивистского замедления времени,
        \item классической полной механической энергии тела,
        \item релятивистской энергии тела,
        \item энергии покоя тела,
        \item связь между релятивистским импульсом и релятивистской энергией.
    \end{itemize}
    Обязательно подпишите все физические величины.
}
\solutionspace{150pt}

\tasknumber{3}%
\task{%
    Электрон движется со скоростью $0{,}7\,c$, где $c$~--- скорость света в вакууме.
    Каково при этом отношение полной энергии частицы $E$ к его энергии покоя $E_0$?
}
\answer{%
    \begin{align*}
    E &= \frac{E_0}{\sqrt{1 - \frac{v^2}{c^2}}}
            \implies \frac E{E_0}
                = \frac 1{\sqrt{1 - \frac{v^2}{c^2}}}
                = \frac 1{\sqrt{1 - \sqr{0{,}7}}}
                \approx 1{,}400,
         \\
        {E_{\text{кин}}} &= E - E_0
            \implies \frac{E_{\text{кин}}}{E_0}
                = \frac E{E_0} - 1
                = \frac 1{\sqrt{1 - \frac{v^2}{c^2}}} - 1
                = \frac 1{\sqrt{1 - \sqr{0{,}7}}} - 1
                \approx 0{,}400.
    \end{align*}
}
\solutionspace{80pt}

\tasknumber{4}%
\task{%
    Полная энергия релятивистской частицы в шесть раз больше её энергии покоя.
    Найти скорость этой частицы: в долях $c$ и численное значение.
    Скорость света в вакууме $c = 3 \cdot 10^{8}\,\frac{\text{м}}{\text{с}}$.
}
\answer{%
    \begin{align*}
    E &= \frac{E_0}{\sqrt{1 - \frac{v^2}{c^2}}}\implies \sqrt{1 - \frac{v^2}{c^2}} = \frac{E_0}{E}\implies \frac{v^2}{c^2} = 1 - \sqr{\frac{E_0}{E}}\implies v = c \sqrt{1 - \sqr{\frac{E_0}{E}}} \approx 0{,}986c \approx 296 \cdot 10^{6}\,\frac{\text{м}}{\text{с}}.
    \end{align*}
}
\solutionspace{80pt}

\tasknumber{5}%
\task{%
    Кинетическая энергия релятивистской частицы в шесть раз больше её энергии покоя.
    Найти скорость этой частицы.
    Скорость света в вакууме $c = 3 \cdot 10^{8}\,\frac{\text{м}}{\text{с}}$.
}
\answer{%
    \begin{align*}
    E &= E_0 + E_{\text{кин}} \\
    E &= \frac{E_0}{\sqrt{1 - \frac{v^2}{c^2}}}\implies \sqrt{1 - \frac{v^2}{c^2}} = \frac{E_0}{E}\implies \frac{v^2}{c^2} = 1 - \sqr{\frac{E_0}{E}} \implies \\
    \implies &v = c \sqrt{1 - \sqr{\frac{E_0}{E}}} = c \sqrt{1 - \sqr{\frac{E_0}{E_0 + E_{\text{кин}} }}} = c \sqrt{1 - \frac 1 {\sqr{ 1 + \frac{E_{\text{кин}}}{E_0} }} }\approx 0{,}990c \approx 297 \cdot 10^{6}\,\frac{\text{м}}{\text{с}}.
    \end{align*}
}


\variantsplitter


\addpersonalvariant{София Журавлёва}

\tasknumber{6}%
\task{%
    Протон движется со скоростью $0{,}75\,c$, где $c$~--- скорость света в вакууме.
    Определите его импульс (в ответе приведите формулу и укажите численное значение).
}
\answer{%
    \begin{align*}
    E &= \frac{mc^2}{\sqrt{1 - \frac{v^2}{c^2}}}
            \approx \frac{1{,}673 \cdot 10^{-27}\,\text{кг} \cdot \sqr{3 \cdot 10^{8}\,\frac{\text{м}}{\text{с}}}}{\sqrt{1 - 0{,}75^2}}
            \approx 0{,}2276 \cdot 10^{-9}\,\text{Дж},
         \\
        E_{\text{кин}} &= \frac{mc^2}{\sqrt{1 - \frac{v^2}{c^2}}} - mc^2
            = mc^2 \cbr{\frac 1{\sqrt{1 - \frac{v^2}{c^2}}} - 1} \approx \\
            &\approx \cbr{1{,}673 \cdot 10^{-27}\,\text{кг} \cdot \sqr{3 \cdot 10^{8}\,\frac{\text{м}}{\text{с}}}}
            \cdot \cbr{\frac 1{\sqrt{1 - 0{,}75^2}} - 1}
            \approx 77{,}05 \cdot 10^{-12}\,\text{Дж},
         \\
        p &= \frac{mv}{\sqrt{1 - \frac{v^2}{c^2}}}
            \approx \frac{1{,}673 \cdot 10^{-27}\,\text{кг} \cdot 0{,}75 \cdot 3 \cdot 10^{8}\,\frac{\text{м}}{\text{с}}}{\sqrt{1 - 0{,}75^2}}
            \approx 0{,}5690 \cdot 10^{-18}\,\frac{\text{кг}\cdot\text{м}}{\text{с}}.
    \end{align*}
}
\solutionspace{100pt}

\tasknumber{7}%
\task{%
    Кинетическая энергия частицы космических лучей в шесть раз превышает её энергию покоя.
    Определить отношение скорости частицы к скорости света.
}
\answer{%
    \begin{align*}
    E &= E_0 + E_{\text{кин}} \\
    E &= \frac{E_0}{\sqrt{1 - \frac{v^2}{c^2}}}\implies \sqrt{1 - \frac{v^2}{c^2}} = \frac{E_0}{E}\implies \frac{v^2}{c^2} = 1 - \sqr{\frac{E_0}{E}} \implies \\
    \implies \frac vc &= \sqrt{1 - \sqr{\frac{E_0}{E}}} = \sqrt{1 - \sqr{\frac{E_0}{E_0 + E_{\text{кин}} }}} \approx 0{,}990.
    \end{align*}
}
\solutionspace{80pt}

\tasknumber{8}%
\task{%
    Некоторая частица, пройдя ускоряющую разность потенциалов, приобрела импульс $3{,}5 \cdot 10^{-19}\,\frac{\text{кг}\cdot\text{м}}{\text{с}}$.
    Скорость частицы стала равной $1{,}5 \cdot 10^{8}\,\frac{\text{м}}{\text{с}}$.
    Найти массу частицы.
}
\answer{%
    $p = \frac{ mv }{\sqrt{1 - \frac{v^2}{c^2} }}\implies m = \frac pv \sqrt{1 - \frac{v^2}{c^2}}= \frac {3{,}5 \cdot 10^{-19}\,\frac{\text{кг}\cdot\text{м}}{\text{с}}}{1{,}5 \cdot 10^{8}\,\frac{\text{м}}{\text{с}}} \sqrt{1 - \sqr{\frac{1{,}5 \cdot 10^{8}\,\frac{\text{м}}{\text{с}}}{3 \cdot 10^{8}\,\frac{\text{м}}{\text{с}}}} } \approx 2{,}0 \cdot 10^{-27}\,\text{кг}.$
}
\solutionspace{80pt}

\tasknumber{9}%
\task{%
    При какой скорости движения (в м/с) релятивистское сокращение длины движущегося тела
    составит 30\%?
}
\answer{%
    \begin{align*}
    l_0 &= \frac l{\sqrt{1 - \frac{v^2}{c^2}}}
        \implies 1 - \frac{v^2}{c^2} = \sqr{\frac l{l_0}}
        \implies \frac v c = \sqrt{1 - \sqr{\frac l{l_0}}} \implies
         \\
        \implies v &= c\sqrt{1 - \sqr{\frac l{l_0}}}
        = 3 \cdot 10^{8}\,\frac{\text{м}}{\text{с}} \cdot \sqrt{1 - \sqr{\frac {l_0 - 0{,}30l_0}{l_0}}}
        = 3 \cdot 10^{8}\,\frac{\text{м}}{\text{с}} \cdot \sqrt{1 - \sqr{1 - 0{,}30}} \approx  \\
        &\approx 0{,}714c
        \approx 214 \cdot 10^{6}\,\frac{\text{м}}{\text{с}}
        \approx 771 \cdot 10^{6}\,\frac{\text{км}}{\text{ч}}.
    \end{align*}
}
\solutionspace{80pt}

\tasknumber{10}%
\task{%
    Стержень движется в продольном направлении с постоянной скоростью относительно инерциальной системы отсчёта.
    При каком значении скорости (в долях скорости света) длина стержня в этой системе отсчёта
    будет в  4  раза меньше его собственной длины?
}
\answer{%
    $l_0 = \frac l{\sqrt{1 - \frac{v^2}{c^2}}}\implies \sqrt{1 - \frac{v^2}{c^2}} = \frac{ l }{ l_0 }\implies \frac v c = \sqrt{1 - \sqr{\frac{ l }{ l_0 }}} \approx 0{,}968.$
}
\solutionspace{80pt}

\tasknumber{11}%
\task{%
    Какую скорость должно иметь движущееся тело, чтобы его продольные размеры уменьшились в два раза?
    Скорость света $c = 3 \cdot 10^{8}\,\frac{\text{м}}{\text{с}}$.
}
\answer{%
    $l_0 = \frac l{\sqrt{1 - \frac{v^2}{c^2}}}\implies \sqrt{1 - \frac{v^2}{c^2}} = \frac{ l }{ l_0 }\implies v = c\sqrt{1 - \sqr{\frac{ l }{ l_0 }}} \approx 260 \cdot 10^{6}\,\frac{\text{м}}{\text{с}}.$
}


\variantsplitter


\addpersonalvariant{София Журавлёва}

\tasknumber{12}%
\task{%
    Время жизни мюона, измеренное наблюдателем, относительно которого мюон покоился, равно $\tau_0$
    Какое расстояние пролетит мюон в системе отсчёта, относительно которой он движется со скоростью $v$,
    сравнимой со скоростью света в вакууме $c$?
}
\answer{%
    $\ell = v\tau = v \frac{\tau_0}{\sqrt{1 - \frac{v^2}{c^2}}}$
}
\solutionspace{80pt}

\tasknumber{13}%
\task{%
    Если $c$ — скорость света в вакууме, то с какой скоростью должна двигаться нестабильная частица относительно наблюдателя,
    чтобы её время жизни было в шесть раз больше, чем у такой же, но покоящейся относительно наблюдателя частицы?
}
\answer{%
    $\tau = \frac{\tau_0}{\sqrt{1 - \frac{v^2}{c^2}}}\implies \sqrt{1 - \frac{v^2}{c^2}} = \frac{\tau_0}{\tau}\implies v = c\sqrt{1 - \sqr{\frac{\tau_0}{\tau}} } \approx 296 \cdot 10^{6}\,\frac{\text{м}}{\text{с}}.$
}
\solutionspace{80pt}

\tasknumber{14}%
\task{%
    Время жизни нестабильной частицы, входящего в состав космических лучей, измеренное земным наблюдателем,
    относительно которого частица двигалась со скоростью, составляющей 75\% скорости света в вакууме, оказалось равным $4{,}8\,\text{мкс}$.
    Каково время жизни частицы, покоящейся относительно наблюдателя?
}
\answer{%
    $t = \frac{t_0}{\sqrt{1 - \frac{v^2}{c^2}}}\implies t_0 = t\sqrt{1 - \frac{v^2}{c^2}} \approx 3{,}2 \cdot 10^{-6}\,\text{с}.$
}
\solutionspace{80pt}

\tasknumber{15}%
\task{%
    Частица увеличила в ускорителе свою скорость с $0{,}01c$ до $0{,}80c$.
    Во сколько раз выросла её кинетическая энергия?
}
\answer{%
    \begin{align*}
    E_{\text{кин.}} &= E - E_0 = \frac{mc^2}{\sqrt{1 - \frac{v^2}{c^2}}} - mc^2 = mc^2\cbr{ \frac1{\sqrt{1 - \frac{v^2}{c^2} }} - 1}.
    \\
    \frac{E_{\text{кин.
    2}}}{E_{\text{кин.
    1}}} &= \frac{\frac1{\sqrt{1 - \frac{v_2^2}{c^2} }} - 1}{\frac1{\sqrt{1 - \frac{v_1^2}{c^2} }} - 1}\approx 13332{,}33
    \end{align*}
}
\solutionspace{120pt}

\tasknumber{16}%
\task{%
    Для частицы, движущейся с релятивистской скоростью,
    выразите $E_0$ и $p$ через $c$, $v$ и $E_\text{кин}$,
    где $E_\text{кин}$~--- кинетическая энергия частицы,
    а $E_0$, $p$ и $v$~--- её энергия покоя, импульс и скорость.
}
\answer{%
    \begin{align*}
    E_\text{кин}, E_0:\quad&E = E_\text{кин} + E_0 = \frac{E_0}{\sqrt{1 - \frac{v^2}{c^2}}} \implies \sqrt{1 - \frac{v^2}{c^2}} = \frac{E_0}{{E_0} + {E_\text{кин}}} \implies v = c\sqrt{1 - \sqr{\frac{E_0}{{E_0} + {E_\text{кин}}}}} \\
    &p = \frac{mv}{\sqrt{1 - \frac{v^2}{c^2}}} = \frac{E_0}{c^2} \cdot \sqrt{1 - \sqr{\frac{E_0}{{E_0} + {E_\text{кин}}}}} \cdot \frac{{E_\text{кин}} + {E_0}}{E_0} = \frac{E_0}{c^2} \cdot \sqrt{\sqr{\frac{{E_\text{кин}} + {E_0}}{E_0}} - 1}.
    \\
    E_\text{кин}, p:\quad&E_\text{кин} = E - E_0 = mc^2\cbr{\frac 1{\sqrt{1 - \frac{v^2}{c^2}}} - 1}, p = \frac{mv}{\sqrt{1 - \frac{v^2}{c^2}}} \implies \frac{E_\text{кин}}{p} = \frac{\frac 1{\sqrt{1 - \frac{v^2}{c^2}}} - 1}{\sqrt{1 - \frac{v^2}{c^2}}} \implies v = \ldots \\
    &E_0 = E - E_\text{кин} = \frac{E_0}{\sqrt{1 - \frac{v^2}{c^2}}} - E_\text{кин} \implies E_0 = \frac{E_\text{кин}}{\frac 1{\sqrt{1 - \frac{v^2}{c^2}}} - 1} = \ldots \\
    E_\text{кин}, v:\quad&E_\text{кин} = E - E_0 = mc^2\cbr{\frac 1{\sqrt{1 - \frac{v^2}{c^2}}} - 1} \implies m = \frac{E_\text{кин}}{c^2\cbr{\frac 1{\sqrt{1 - \frac{v^2}{c^2}}} - 1}} \\
    &E_0 = mc^2 = \frac{E_\text{кин}}{\frac 1{\sqrt{1 - \frac{v^2}{c^2}}} - 1} \\
    &p = \frac{mv}{\sqrt{1 - \frac{v^2}{c^2}}} = \frac{E_\text{кин}}{c^2\cbr{\frac 1{\sqrt{1 - \frac{v^2}{c^2}}} - 1}} \cdot \frac{v}{\sqrt{1 - \frac{v^2}{c^2}}} = \frac{{E_\text{кин}} v}{c^2\cbr{1 - {\sqrt{1 - \frac{v^2}{c^2}}}}} \\
    E_0, p:\quad&E_0 = mc^2, \quad p = \frac{mv}{\sqrt{1 - \frac{v^2}{c^2}}} \implies \frac{E_0}{p} = \frac{c^2}v{\sqrt{1 - \frac{v^2}{c^2}}} = c\sqrt{\frac{c^2}{v^2} - 1} \\
    &\sqr{\frac{E_0}{pc}} = \frac{c^2}{v^2} - 1 \implies \frac{v^2}{c^2} = \frac 1{1 + \frac{E_0^2}{p^2c^2}} \implies v = \frac c{\sqrt{1 + \frac{E_0^2}{p^2c^2}}} \\
    &{E_\text{кин}} = E - E_0 = \sqrt{E_0^2 + p^2c^2} - E_0 \\
    E_0, v:\quad&E_0 = mc^2 \implies m = \frac{E_0}{c^2} \qquad p = \frac{mv}{\sqrt{1 - \frac{v^2}{c^2}}} = \frac{E_0}{c^2} \cdot \frac{v}{\sqrt{1 - \frac{v^2}{c^2}}} \\
    &E_\text{кин}= mc^2\cbr{\frac 1{\sqrt{1 - \frac{v^2}{c^2}}} - 1} = \frac{E_0}{c^2}\cbr{\frac 1{\sqrt{1 - \frac{v^2}{c^2}}} - 1} \\
    p, v:\quad&p = \frac{mv}{\sqrt{1 - \frac{v^2}{c^2}}} \implies m = \frac p v {\sqrt{1 - \frac{v^2}{c^2}}} \implies E_0 = mc^2 =\frac {pc^2} v {\sqrt{1 - \frac{v^2}{c^2}}} \\
    &E_\text{кин} = mc^2\cbr{\frac 1{\sqrt{1 - \frac{v^2}{c^2}}} - 1} = \frac p v {\sqrt{1 - \frac{v^2}{c^2}}}\cbr{\frac 1{\sqrt{1 - \frac{v^2}{c^2}}} - 1} = \frac p v \cbr{1 - {\sqrt{1 - \frac{v^2}{c^2}}}}
    \end{align*}
}

\variantsplitter

\addpersonalvariant{Константин Козлов}

\tasknumber{1}%
\task{%
    Запишите
    \begin{itemize}
        \item постулаты специальной теории относительности,
        \item пример релятивистского эффекта, обнаружимый при скоростях гораздо меньше скорости света.
    \end{itemize}
}
\solutionspace{120pt}

\tasknumber{2}%
\task{%
    Запишите формулу для ...
    \begin{itemize}
        \item релятивистского сжатия,
        \item классической полной механической энергии тела,
        \item релятивистской энергии тела,
        \item релятивистской кинетической энергии,
        \item связь между релятивистским импульсом и релятивистской энергией.
    \end{itemize}
    Обязательно подпишите все физические величины.
}
\solutionspace{150pt}

\tasknumber{3}%
\task{%
    Позитрон движется со скоростью $0{,}8\,c$, где $c$~--- скорость света в вакууме.
    Каково при этом отношение полной энергии частицы $E$ к его энергии покоя $E_0$?
}
\answer{%
    \begin{align*}
    E &= \frac{E_0}{\sqrt{1 - \frac{v^2}{c^2}}}
            \implies \frac E{E_0}
                = \frac 1{\sqrt{1 - \frac{v^2}{c^2}}}
                = \frac 1{\sqrt{1 - \sqr{0{,}8}}}
                \approx 1{,}667,
         \\
        {E_{\text{кин}}} &= E - E_0
            \implies \frac{E_{\text{кин}}}{E_0}
                = \frac E{E_0} - 1
                = \frac 1{\sqrt{1 - \frac{v^2}{c^2}}} - 1
                = \frac 1{\sqrt{1 - \sqr{0{,}8}}} - 1
                \approx 0{,}667.
    \end{align*}
}
\solutionspace{80pt}

\tasknumber{4}%
\task{%
    Полная энергия релятивистской частицы в пять раз больше её энергии покоя.
    Найти скорость этой частицы: в долях $c$ и численное значение.
    Скорость света в вакууме $c = 3 \cdot 10^{8}\,\frac{\text{м}}{\text{с}}$.
}
\answer{%
    \begin{align*}
    E &= \frac{E_0}{\sqrt{1 - \frac{v^2}{c^2}}}\implies \sqrt{1 - \frac{v^2}{c^2}} = \frac{E_0}{E}\implies \frac{v^2}{c^2} = 1 - \sqr{\frac{E_0}{E}}\implies v = c \sqrt{1 - \sqr{\frac{E_0}{E}}} \approx 0{,}980c \approx 294 \cdot 10^{6}\,\frac{\text{м}}{\text{с}}.
    \end{align*}
}
\solutionspace{80pt}

\tasknumber{5}%
\task{%
    Кинетическая энергия релятивистской частицы в пять раз больше её энергии покоя.
    Найти скорость этой частицы.
    Скорость света в вакууме $c = 3 \cdot 10^{8}\,\frac{\text{м}}{\text{с}}$.
}
\answer{%
    \begin{align*}
    E &= E_0 + E_{\text{кин}} \\
    E &= \frac{E_0}{\sqrt{1 - \frac{v^2}{c^2}}}\implies \sqrt{1 - \frac{v^2}{c^2}} = \frac{E_0}{E}\implies \frac{v^2}{c^2} = 1 - \sqr{\frac{E_0}{E}} \implies \\
    \implies &v = c \sqrt{1 - \sqr{\frac{E_0}{E}}} = c \sqrt{1 - \sqr{\frac{E_0}{E_0 + E_{\text{кин}} }}} = c \sqrt{1 - \frac 1 {\sqr{ 1 + \frac{E_{\text{кин}}}{E_0} }} }\approx 0{,}986c \approx 296 \cdot 10^{6}\,\frac{\text{м}}{\text{с}}.
    \end{align*}
}


\variantsplitter


\addpersonalvariant{Константин Козлов}

\tasknumber{6}%
\task{%
    Протон движется со скоростью $0{,}65\,c$, где $c$~--- скорость света в вакууме.
    Определите его полную энергию (в ответе приведите формулу и укажите численное значение).
}
\answer{%
    \begin{align*}
    E &= \frac{mc^2}{\sqrt{1 - \frac{v^2}{c^2}}}
            \approx \frac{1{,}673 \cdot 10^{-27}\,\text{кг} \cdot \sqr{3 \cdot 10^{8}\,\frac{\text{м}}{\text{с}}}}{\sqrt{1 - 0{,}65^2}}
            \approx 0{,}19809 \cdot 10^{-9}\,\text{Дж},
         \\
        E_{\text{кин}} &= \frac{mc^2}{\sqrt{1 - \frac{v^2}{c^2}}} - mc^2
            = mc^2 \cbr{\frac 1{\sqrt{1 - \frac{v^2}{c^2}}} - 1} \approx \\
            &\approx \cbr{1{,}673 \cdot 10^{-27}\,\text{кг} \cdot \sqr{3 \cdot 10^{8}\,\frac{\text{м}}{\text{с}}}}
            \cdot \cbr{\frac 1{\sqrt{1 - 0{,}65^2}} - 1}
            \approx 47{,}55 \cdot 10^{-12}\,\text{Дж},
         \\
        p &= \frac{mv}{\sqrt{1 - \frac{v^2}{c^2}}}
            \approx \frac{1{,}673 \cdot 10^{-27}\,\text{кг} \cdot 0{,}65 \cdot 3 \cdot 10^{8}\,\frac{\text{м}}{\text{с}}}{\sqrt{1 - 0{,}65^2}}
            \approx 0{,}4292 \cdot 10^{-18}\,\frac{\text{кг}\cdot\text{м}}{\text{с}}.
    \end{align*}
}
\solutionspace{100pt}

\tasknumber{7}%
\task{%
    Кинетическая энергия частицы космических лучей в пять раз превышает её энергию покоя.
    Определить отношение скорости частицы к скорости света.
}
\answer{%
    \begin{align*}
    E &= E_0 + E_{\text{кин}} \\
    E &= \frac{E_0}{\sqrt{1 - \frac{v^2}{c^2}}}\implies \sqrt{1 - \frac{v^2}{c^2}} = \frac{E_0}{E}\implies \frac{v^2}{c^2} = 1 - \sqr{\frac{E_0}{E}} \implies \\
    \implies \frac vc &= \sqrt{1 - \sqr{\frac{E_0}{E}}} = \sqrt{1 - \sqr{\frac{E_0}{E_0 + E_{\text{кин}} }}} \approx 0{,}986.
    \end{align*}
}
\solutionspace{80pt}

\tasknumber{8}%
\task{%
    Некоторая частица, пройдя ускоряющую разность потенциалов, приобрела импульс $4{,}2 \cdot 10^{-19}\,\frac{\text{кг}\cdot\text{м}}{\text{с}}$.
    Скорость частицы стала равной $2{,}4 \cdot 10^{8}\,\frac{\text{м}}{\text{с}}$.
    Найти массу частицы.
}
\answer{%
    $p = \frac{ mv }{\sqrt{1 - \frac{v^2}{c^2} }}\implies m = \frac pv \sqrt{1 - \frac{v^2}{c^2}}= \frac {4{,}2 \cdot 10^{-19}\,\frac{\text{кг}\cdot\text{м}}{\text{с}}}{2{,}4 \cdot 10^{8}\,\frac{\text{м}}{\text{с}}} \sqrt{1 - \sqr{\frac{2{,}4 \cdot 10^{8}\,\frac{\text{м}}{\text{с}}}{3 \cdot 10^{8}\,\frac{\text{м}}{\text{с}}}} } \approx 1{,}050 \cdot 10^{-27}\,\text{кг}.$
}
\solutionspace{80pt}

\tasknumber{9}%
\task{%
    При какой скорости движения (в долях скорости света) релятивистское сокращение длины движущегося тела
    составит 30\%?
}
\answer{%
    \begin{align*}
    l_0 &= \frac l{\sqrt{1 - \frac{v^2}{c^2}}}
        \implies 1 - \frac{v^2}{c^2} = \sqr{\frac l{l_0}}
        \implies \frac v c = \sqrt{1 - \sqr{\frac l{l_0}}} \implies
         \\
        \implies v &= c\sqrt{1 - \sqr{\frac l{l_0}}}
        = 3 \cdot 10^{8}\,\frac{\text{м}}{\text{с}} \cdot \sqrt{1 - \sqr{\frac {l_0 - 0{,}30l_0}{l_0}}}
        = 3 \cdot 10^{8}\,\frac{\text{м}}{\text{с}} \cdot \sqrt{1 - \sqr{1 - 0{,}30}} \approx  \\
        &\approx 0{,}714c
        \approx 214 \cdot 10^{6}\,\frac{\text{м}}{\text{с}}
        \approx 771 \cdot 10^{6}\,\frac{\text{км}}{\text{ч}}.
    \end{align*}
}
\solutionspace{80pt}

\tasknumber{10}%
\task{%
    Стержень движется в продольном направлении с постоянной скоростью относительно инерциальной системы отсчёта.
    При каком значении скорости (в долях скорости света) длина стержня в этой системе отсчёта
    будет в  4  раза меньше его собственной длины?
}
\answer{%
    $l_0 = \frac l{\sqrt{1 - \frac{v^2}{c^2}}}\implies \sqrt{1 - \frac{v^2}{c^2}} = \frac{ l }{ l_0 }\implies \frac v c = \sqrt{1 - \sqr{\frac{ l }{ l_0 }}} \approx 0{,}968.$
}
\solutionspace{80pt}

\tasknumber{11}%
\task{%
    Какую скорость должно иметь движущееся тело, чтобы его продольные размеры уменьшились в два раза?
    Скорость света $c = 3 \cdot 10^{8}\,\frac{\text{м}}{\text{с}}$.
}
\answer{%
    $l_0 = \frac l{\sqrt{1 - \frac{v^2}{c^2}}}\implies \sqrt{1 - \frac{v^2}{c^2}} = \frac{ l }{ l_0 }\implies v = c\sqrt{1 - \sqr{\frac{ l }{ l_0 }}} \approx 260 \cdot 10^{6}\,\frac{\text{м}}{\text{с}}.$
}


\variantsplitter


\addpersonalvariant{Константин Козлов}

\tasknumber{12}%
\task{%
    Время жизни мюона, измеренное наблюдателем, относительно которого мюон покоился, равно $\tau_0$
    Какое расстояние пролетит мюон в системе отсчёта, относительно которой он движется со скоростью $v$,
    сравнимой со скоростью света в вакууме $c$?
}
\answer{%
    $\ell = v\tau = v \frac{\tau_0}{\sqrt{1 - \frac{v^2}{c^2}}}$
}
\solutionspace{80pt}

\tasknumber{13}%
\task{%
    Если $c$ — скорость света в вакууме, то с какой скоростью должна двигаться нестабильная частица относительно наблюдателя,
    чтобы её время жизни было в пять раз больше, чем у такой же, но покоящейся относительно наблюдателя частицы?
}
\answer{%
    $\tau = \frac{\tau_0}{\sqrt{1 - \frac{v^2}{c^2}}}\implies \sqrt{1 - \frac{v^2}{c^2}} = \frac{\tau_0}{\tau}\implies v = c\sqrt{1 - \sqr{\frac{\tau_0}{\tau}} } \approx 294 \cdot 10^{6}\,\frac{\text{м}}{\text{с}}.$
}
\solutionspace{80pt}

\tasknumber{14}%
\task{%
    Время жизни нестабильной частицы, входящего в состав космических лучей, измеренное земным наблюдателем,
    относительно которого частица двигалась со скоростью, составляющей 65\% скорости света в вакууме, оказалось равным $5{,}3\,\text{мкс}$.
    Каково время жизни частицы, покоящейся относительно наблюдателя?
}
\answer{%
    $t = \frac{t_0}{\sqrt{1 - \frac{v^2}{c^2}}}\implies t_0 = t\sqrt{1 - \frac{v^2}{c^2}} \approx 4{,}0 \cdot 10^{-6}\,\text{с}.$
}
\solutionspace{80pt}

\tasknumber{15}%
\task{%
    Частица увеличила в ускорителе свою скорость с $0{,}01c$ до $0{,}70c$.
    Во сколько раз выросла её кинетическая энергия?
}
\answer{%
    \begin{align*}
    E_{\text{кин.}} &= E - E_0 = \frac{mc^2}{\sqrt{1 - \frac{v^2}{c^2}}} - mc^2 = mc^2\cbr{ \frac1{\sqrt{1 - \frac{v^2}{c^2} }} - 1}.
    \\
    \frac{E_{\text{кин.
    2}}}{E_{\text{кин.
    1}}} &= \frac{\frac1{\sqrt{1 - \frac{v_2^2}{c^2} }} - 1}{\frac1{\sqrt{1 - \frac{v_1^2}{c^2} }} - 1}\approx 8005{,}00
    \end{align*}
}
\solutionspace{120pt}

\tasknumber{16}%
\task{%
    Для частицы, движущейся с релятивистской скоростью,
    выразите $E_0$ и $v$ через $c$, $p$ и $E_\text{кин}$,
    где $E_\text{кин}$~--- кинетическая энергия частицы,
    а $E_0$, $p$ и $v$~--- её энергия покоя, импульс и скорость.
}
\answer{%
    \begin{align*}
    E_\text{кин}, E_0:\quad&E = E_\text{кин} + E_0 = \frac{E_0}{\sqrt{1 - \frac{v^2}{c^2}}} \implies \sqrt{1 - \frac{v^2}{c^2}} = \frac{E_0}{{E_0} + {E_\text{кин}}} \implies v = c\sqrt{1 - \sqr{\frac{E_0}{{E_0} + {E_\text{кин}}}}} \\
    &p = \frac{mv}{\sqrt{1 - \frac{v^2}{c^2}}} = \frac{E_0}{c^2} \cdot \sqrt{1 - \sqr{\frac{E_0}{{E_0} + {E_\text{кин}}}}} \cdot \frac{{E_\text{кин}} + {E_0}}{E_0} = \frac{E_0}{c^2} \cdot \sqrt{\sqr{\frac{{E_\text{кин}} + {E_0}}{E_0}} - 1}.
    \\
    E_\text{кин}, p:\quad&E_\text{кин} = E - E_0 = mc^2\cbr{\frac 1{\sqrt{1 - \frac{v^2}{c^2}}} - 1}, p = \frac{mv}{\sqrt{1 - \frac{v^2}{c^2}}} \implies \frac{E_\text{кин}}{p} = \frac{\frac 1{\sqrt{1 - \frac{v^2}{c^2}}} - 1}{\sqrt{1 - \frac{v^2}{c^2}}} \implies v = \ldots \\
    &E_0 = E - E_\text{кин} = \frac{E_0}{\sqrt{1 - \frac{v^2}{c^2}}} - E_\text{кин} \implies E_0 = \frac{E_\text{кин}}{\frac 1{\sqrt{1 - \frac{v^2}{c^2}}} - 1} = \ldots \\
    E_\text{кин}, v:\quad&E_\text{кин} = E - E_0 = mc^2\cbr{\frac 1{\sqrt{1 - \frac{v^2}{c^2}}} - 1} \implies m = \frac{E_\text{кин}}{c^2\cbr{\frac 1{\sqrt{1 - \frac{v^2}{c^2}}} - 1}} \\
    &E_0 = mc^2 = \frac{E_\text{кин}}{\frac 1{\sqrt{1 - \frac{v^2}{c^2}}} - 1} \\
    &p = \frac{mv}{\sqrt{1 - \frac{v^2}{c^2}}} = \frac{E_\text{кин}}{c^2\cbr{\frac 1{\sqrt{1 - \frac{v^2}{c^2}}} - 1}} \cdot \frac{v}{\sqrt{1 - \frac{v^2}{c^2}}} = \frac{{E_\text{кин}} v}{c^2\cbr{1 - {\sqrt{1 - \frac{v^2}{c^2}}}}} \\
    E_0, p:\quad&E_0 = mc^2, \quad p = \frac{mv}{\sqrt{1 - \frac{v^2}{c^2}}} \implies \frac{E_0}{p} = \frac{c^2}v{\sqrt{1 - \frac{v^2}{c^2}}} = c\sqrt{\frac{c^2}{v^2} - 1} \\
    &\sqr{\frac{E_0}{pc}} = \frac{c^2}{v^2} - 1 \implies \frac{v^2}{c^2} = \frac 1{1 + \frac{E_0^2}{p^2c^2}} \implies v = \frac c{\sqrt{1 + \frac{E_0^2}{p^2c^2}}} \\
    &{E_\text{кин}} = E - E_0 = \sqrt{E_0^2 + p^2c^2} - E_0 \\
    E_0, v:\quad&E_0 = mc^2 \implies m = \frac{E_0}{c^2} \qquad p = \frac{mv}{\sqrt{1 - \frac{v^2}{c^2}}} = \frac{E_0}{c^2} \cdot \frac{v}{\sqrt{1 - \frac{v^2}{c^2}}} \\
    &E_\text{кин}= mc^2\cbr{\frac 1{\sqrt{1 - \frac{v^2}{c^2}}} - 1} = \frac{E_0}{c^2}\cbr{\frac 1{\sqrt{1 - \frac{v^2}{c^2}}} - 1} \\
    p, v:\quad&p = \frac{mv}{\sqrt{1 - \frac{v^2}{c^2}}} \implies m = \frac p v {\sqrt{1 - \frac{v^2}{c^2}}} \implies E_0 = mc^2 =\frac {pc^2} v {\sqrt{1 - \frac{v^2}{c^2}}} \\
    &E_\text{кин} = mc^2\cbr{\frac 1{\sqrt{1 - \frac{v^2}{c^2}}} - 1} = \frac p v {\sqrt{1 - \frac{v^2}{c^2}}}\cbr{\frac 1{\sqrt{1 - \frac{v^2}{c^2}}} - 1} = \frac p v \cbr{1 - {\sqrt{1 - \frac{v^2}{c^2}}}}
    \end{align*}
}

\variantsplitter

\addpersonalvariant{Наталья Кравченко}

\tasknumber{1}%
\task{%
    Запишите
    \begin{itemize}
        \item постулаты специальной теории относительности,
        \item пример релятивистского эффекта, обнаружимый при скоростях гораздо меньше скорости света.
    \end{itemize}
}
\solutionspace{120pt}

\tasknumber{2}%
\task{%
    Запишите формулу для ...
    \begin{itemize}
        \item релятивистского замедления времени,
        \item классического импульса,
        \item релятивистской энергии тела,
        \item энергии покоя тела,
        \item связь между релятивистским импульсом и релятивистской энергией.
    \end{itemize}
    Обязательно подпишите все физические величины.
}
\solutionspace{150pt}

\tasknumber{3}%
\task{%
    Электрон движется со скоростью $0{,}8\,c$, где $c$~--- скорость света в вакууме.
    Каково при этом отношение полной энергии частицы $E$ к его энергии покоя $E_0$?
}
\answer{%
    \begin{align*}
    E &= \frac{E_0}{\sqrt{1 - \frac{v^2}{c^2}}}
            \implies \frac E{E_0}
                = \frac 1{\sqrt{1 - \frac{v^2}{c^2}}}
                = \frac 1{\sqrt{1 - \sqr{0{,}8}}}
                \approx 1{,}667,
         \\
        {E_{\text{кин}}} &= E - E_0
            \implies \frac{E_{\text{кин}}}{E_0}
                = \frac E{E_0} - 1
                = \frac 1{\sqrt{1 - \frac{v^2}{c^2}}} - 1
                = \frac 1{\sqrt{1 - \sqr{0{,}8}}} - 1
                \approx 0{,}667.
    \end{align*}
}
\solutionspace{80pt}

\tasknumber{4}%
\task{%
    Полная энергия релятивистской частицы в три раза больше её энергии покоя.
    Найти скорость этой частицы: в долях $c$ и численное значение.
    Скорость света в вакууме $c = 3 \cdot 10^{8}\,\frac{\text{м}}{\text{с}}$.
}
\answer{%
    \begin{align*}
    E &= \frac{E_0}{\sqrt{1 - \frac{v^2}{c^2}}}\implies \sqrt{1 - \frac{v^2}{c^2}} = \frac{E_0}{E}\implies \frac{v^2}{c^2} = 1 - \sqr{\frac{E_0}{E}}\implies v = c \sqrt{1 - \sqr{\frac{E_0}{E}}} \approx 0{,}943c \approx 283 \cdot 10^{6}\,\frac{\text{м}}{\text{с}}.
    \end{align*}
}
\solutionspace{80pt}

\tasknumber{5}%
\task{%
    Кинетическая энергия релятивистской частицы в три раза больше её энергии покоя.
    Найти скорость этой частицы.
    Скорость света в вакууме $c = 3 \cdot 10^{8}\,\frac{\text{м}}{\text{с}}$.
}
\answer{%
    \begin{align*}
    E &= E_0 + E_{\text{кин}} \\
    E &= \frac{E_0}{\sqrt{1 - \frac{v^2}{c^2}}}\implies \sqrt{1 - \frac{v^2}{c^2}} = \frac{E_0}{E}\implies \frac{v^2}{c^2} = 1 - \sqr{\frac{E_0}{E}} \implies \\
    \implies &v = c \sqrt{1 - \sqr{\frac{E_0}{E}}} = c \sqrt{1 - \sqr{\frac{E_0}{E_0 + E_{\text{кин}} }}} = c \sqrt{1 - \frac 1 {\sqr{ 1 + \frac{E_{\text{кин}}}{E_0} }} }\approx 0{,}968c \approx 290 \cdot 10^{6}\,\frac{\text{м}}{\text{с}}.
    \end{align*}
}


\variantsplitter


\addpersonalvariant{Наталья Кравченко}

\tasknumber{6}%
\task{%
    Электрон движется со скоростью $0{,}85\,c$, где $c$~--- скорость света в вакууме.
    Определите его полную энергию (в ответе приведите формулу и укажите численное значение).
}
\answer{%
    \begin{align*}
    E &= \frac{mc^2}{\sqrt{1 - \frac{v^2}{c^2}}}
            \approx \frac{9{,}1 \cdot 10^{-31}\,\text{кг} \cdot \sqr{3 \cdot 10^{8}\,\frac{\text{м}}{\text{с}}}}{\sqrt{1 - 0{,}85^2}}
            \approx 0{,}1555 \cdot 10^{-12}\,\text{Дж},
         \\
        E_{\text{кин}} &= \frac{mc^2}{\sqrt{1 - \frac{v^2}{c^2}}} - mc^2
            = mc^2 \cbr{\frac 1{\sqrt{1 - \frac{v^2}{c^2}}} - 1} \approx \\
            &\approx \cbr{9{,}1 \cdot 10^{-31}\,\text{кг} \cdot \sqr{3 \cdot 10^{8}\,\frac{\text{м}}{\text{с}}}}
            \cdot \cbr{\frac 1{\sqrt{1 - 0{,}85^2}} - 1}
            \approx 73{,}6 \cdot 10^{-15}\,\text{Дж},
         \\
        p &= \frac{mv}{\sqrt{1 - \frac{v^2}{c^2}}}
            \approx \frac{9{,}1 \cdot 10^{-31}\,\text{кг} \cdot 0{,}85 \cdot 3 \cdot 10^{8}\,\frac{\text{м}}{\text{с}}}{\sqrt{1 - 0{,}85^2}}
            \approx 0{,}441 \cdot 10^{-21}\,\frac{\text{кг}\cdot\text{м}}{\text{с}}.
    \end{align*}
}
\solutionspace{100pt}

\tasknumber{7}%
\task{%
    Кинетическая энергия частицы космических лучей в три раза превышает её энергию покоя.
    Определить отношение скорости частицы к скорости света.
}
\answer{%
    \begin{align*}
    E &= E_0 + E_{\text{кин}} \\
    E &= \frac{E_0}{\sqrt{1 - \frac{v^2}{c^2}}}\implies \sqrt{1 - \frac{v^2}{c^2}} = \frac{E_0}{E}\implies \frac{v^2}{c^2} = 1 - \sqr{\frac{E_0}{E}} \implies \\
    \implies \frac vc &= \sqrt{1 - \sqr{\frac{E_0}{E}}} = \sqrt{1 - \sqr{\frac{E_0}{E_0 + E_{\text{кин}} }}} \approx 0{,}968.
    \end{align*}
}
\solutionspace{80pt}

\tasknumber{8}%
\task{%
    Некоторая частица, пройдя ускоряющую разность потенциалов, приобрела импульс $4{,}2 \cdot 10^{-19}\,\frac{\text{кг}\cdot\text{м}}{\text{с}}$.
    Скорость частицы стала равной $1{,}8 \cdot 10^{8}\,\frac{\text{м}}{\text{с}}$.
    Найти массу частицы.
}
\answer{%
    $p = \frac{ mv }{\sqrt{1 - \frac{v^2}{c^2} }}\implies m = \frac pv \sqrt{1 - \frac{v^2}{c^2}}= \frac {4{,}2 \cdot 10^{-19}\,\frac{\text{кг}\cdot\text{м}}{\text{с}}}{1{,}8 \cdot 10^{8}\,\frac{\text{м}}{\text{с}}} \sqrt{1 - \sqr{\frac{1{,}8 \cdot 10^{8}\,\frac{\text{м}}{\text{с}}}{3 \cdot 10^{8}\,\frac{\text{м}}{\text{с}}}} } \approx 1{,}87 \cdot 10^{-27}\,\text{кг}.$
}
\solutionspace{80pt}

\tasknumber{9}%
\task{%
    При какой скорости движения (в км/ч) релятивистское сокращение длины движущегося тела
    составит 50\%?
}
\answer{%
    \begin{align*}
    l_0 &= \frac l{\sqrt{1 - \frac{v^2}{c^2}}}
        \implies 1 - \frac{v^2}{c^2} = \sqr{\frac l{l_0}}
        \implies \frac v c = \sqrt{1 - \sqr{\frac l{l_0}}} \implies
         \\
        \implies v &= c\sqrt{1 - \sqr{\frac l{l_0}}}
        = 3 \cdot 10^{8}\,\frac{\text{м}}{\text{с}} \cdot \sqrt{1 - \sqr{\frac {l_0 - 0{,}50l_0}{l_0}}}
        = 3 \cdot 10^{8}\,\frac{\text{м}}{\text{с}} \cdot \sqrt{1 - \sqr{1 - 0{,}50}} \approx  \\
        &\approx 0{,}866c
        \approx 260 \cdot 10^{6}\,\frac{\text{м}}{\text{с}}
        \approx 935 \cdot 10^{6}\,\frac{\text{км}}{\text{ч}}.
    \end{align*}
}
\solutionspace{80pt}

\tasknumber{10}%
\task{%
    Стержень движется в продольном направлении с постоянной скоростью относительно инерциальной системы отсчёта.
    При каком значении скорости (в долях скорости света) длина стержня в этой системе отсчёта
    будет в  3  раза меньше его собственной длины?
}
\answer{%
    $l_0 = \frac l{\sqrt{1 - \frac{v^2}{c^2}}}\implies \sqrt{1 - \frac{v^2}{c^2}} = \frac{ l }{ l_0 }\implies \frac v c = \sqrt{1 - \sqr{\frac{ l }{ l_0 }}} \approx 0{,}943.$
}
\solutionspace{80pt}

\tasknumber{11}%
\task{%
    Какую скорость должно иметь движущееся тело, чтобы его продольные размеры уменьшились в пять раз?
    Скорость света $c = 3 \cdot 10^{8}\,\frac{\text{м}}{\text{с}}$.
}
\answer{%
    $l_0 = \frac l{\sqrt{1 - \frac{v^2}{c^2}}}\implies \sqrt{1 - \frac{v^2}{c^2}} = \frac{ l }{ l_0 }\implies v = c\sqrt{1 - \sqr{\frac{ l }{ l_0 }}} \approx 294 \cdot 10^{6}\,\frac{\text{м}}{\text{с}}.$
}


\variantsplitter


\addpersonalvariant{Наталья Кравченко}

\tasknumber{12}%
\task{%
    Время жизни мюона, измеренное наблюдателем, относительно которого мюон покоился, равно $\tau_0$
    Какое расстояние пролетит мюон в системе отсчёта, относительно которой он движется со скоростью $v$,
    сравнимой со скоростью света в вакууме $c$?
}
\answer{%
    $\ell = v\tau = v \frac{\tau_0}{\sqrt{1 - \frac{v^2}{c^2}}}$
}
\solutionspace{80pt}

\tasknumber{13}%
\task{%
    Если $c$ — скорость света в вакууме, то с какой скоростью должна двигаться нестабильная частица относительно наблюдателя,
    чтобы её время жизни было в семь раз больше, чем у такой же, но покоящейся относительно наблюдателя частицы?
}
\answer{%
    $\tau = \frac{\tau_0}{\sqrt{1 - \frac{v^2}{c^2}}}\implies \sqrt{1 - \frac{v^2}{c^2}} = \frac{\tau_0}{\tau}\implies v = c\sqrt{1 - \sqr{\frac{\tau_0}{\tau}} } \approx 297 \cdot 10^{6}\,\frac{\text{м}}{\text{с}}.$
}
\solutionspace{80pt}

\tasknumber{14}%
\task{%
    Время жизни нестабильной частицы, входящего в состав космических лучей, измеренное земным наблюдателем,
    относительно которого частица двигалась со скоростью, составляющей 65\% скорости света в вакууме, оказалось равным $3{,}7\,\text{мкс}$.
    Каково время жизни частицы, покоящейся относительно наблюдателя?
}
\answer{%
    $t = \frac{t_0}{\sqrt{1 - \frac{v^2}{c^2}}}\implies t_0 = t\sqrt{1 - \frac{v^2}{c^2}} \approx 2{,}8 \cdot 10^{-6}\,\text{с}.$
}
\solutionspace{80pt}

\tasknumber{15}%
\task{%
    Частица увеличила в ускорителе свою скорость с $0{,}03c$ до $0{,}80c$.
    Во сколько раз выросла её кинетическая энергия?
}
\answer{%
    \begin{align*}
    E_{\text{кин.}} &= E - E_0 = \frac{mc^2}{\sqrt{1 - \frac{v^2}{c^2}}} - mc^2 = mc^2\cbr{ \frac1{\sqrt{1 - \frac{v^2}{c^2} }} - 1}.
    \\
    \frac{E_{\text{кин.
    2}}}{E_{\text{кин.
    1}}} &= \frac{\frac1{\sqrt{1 - \frac{v_2^2}{c^2} }} - 1}{\frac1{\sqrt{1 - \frac{v_1^2}{c^2} }} - 1}\approx 1480{,}48
    \end{align*}
}
\solutionspace{120pt}

\tasknumber{16}%
\task{%
    Для частицы, движущейся с релятивистской скоростью,
    выразите $v$ и $E_0$ через $c$, $p$ и $E_\text{кин}$,
    где $E_\text{кин}$~--- кинетическая энергия частицы,
    а $E_0$, $p$ и $v$~--- её энергия покоя, импульс и скорость.
}
\answer{%
    \begin{align*}
    E_\text{кин}, E_0:\quad&E = E_\text{кин} + E_0 = \frac{E_0}{\sqrt{1 - \frac{v^2}{c^2}}} \implies \sqrt{1 - \frac{v^2}{c^2}} = \frac{E_0}{{E_0} + {E_\text{кин}}} \implies v = c\sqrt{1 - \sqr{\frac{E_0}{{E_0} + {E_\text{кин}}}}} \\
    &p = \frac{mv}{\sqrt{1 - \frac{v^2}{c^2}}} = \frac{E_0}{c^2} \cdot \sqrt{1 - \sqr{\frac{E_0}{{E_0} + {E_\text{кин}}}}} \cdot \frac{{E_\text{кин}} + {E_0}}{E_0} = \frac{E_0}{c^2} \cdot \sqrt{\sqr{\frac{{E_\text{кин}} + {E_0}}{E_0}} - 1}.
    \\
    E_\text{кин}, p:\quad&E_\text{кин} = E - E_0 = mc^2\cbr{\frac 1{\sqrt{1 - \frac{v^2}{c^2}}} - 1}, p = \frac{mv}{\sqrt{1 - \frac{v^2}{c^2}}} \implies \frac{E_\text{кин}}{p} = \frac{\frac 1{\sqrt{1 - \frac{v^2}{c^2}}} - 1}{\sqrt{1 - \frac{v^2}{c^2}}} \implies v = \ldots \\
    &E_0 = E - E_\text{кин} = \frac{E_0}{\sqrt{1 - \frac{v^2}{c^2}}} - E_\text{кин} \implies E_0 = \frac{E_\text{кин}}{\frac 1{\sqrt{1 - \frac{v^2}{c^2}}} - 1} = \ldots \\
    E_\text{кин}, v:\quad&E_\text{кин} = E - E_0 = mc^2\cbr{\frac 1{\sqrt{1 - \frac{v^2}{c^2}}} - 1} \implies m = \frac{E_\text{кин}}{c^2\cbr{\frac 1{\sqrt{1 - \frac{v^2}{c^2}}} - 1}} \\
    &E_0 = mc^2 = \frac{E_\text{кин}}{\frac 1{\sqrt{1 - \frac{v^2}{c^2}}} - 1} \\
    &p = \frac{mv}{\sqrt{1 - \frac{v^2}{c^2}}} = \frac{E_\text{кин}}{c^2\cbr{\frac 1{\sqrt{1 - \frac{v^2}{c^2}}} - 1}} \cdot \frac{v}{\sqrt{1 - \frac{v^2}{c^2}}} = \frac{{E_\text{кин}} v}{c^2\cbr{1 - {\sqrt{1 - \frac{v^2}{c^2}}}}} \\
    E_0, p:\quad&E_0 = mc^2, \quad p = \frac{mv}{\sqrt{1 - \frac{v^2}{c^2}}} \implies \frac{E_0}{p} = \frac{c^2}v{\sqrt{1 - \frac{v^2}{c^2}}} = c\sqrt{\frac{c^2}{v^2} - 1} \\
    &\sqr{\frac{E_0}{pc}} = \frac{c^2}{v^2} - 1 \implies \frac{v^2}{c^2} = \frac 1{1 + \frac{E_0^2}{p^2c^2}} \implies v = \frac c{\sqrt{1 + \frac{E_0^2}{p^2c^2}}} \\
    &{E_\text{кин}} = E - E_0 = \sqrt{E_0^2 + p^2c^2} - E_0 \\
    E_0, v:\quad&E_0 = mc^2 \implies m = \frac{E_0}{c^2} \qquad p = \frac{mv}{\sqrt{1 - \frac{v^2}{c^2}}} = \frac{E_0}{c^2} \cdot \frac{v}{\sqrt{1 - \frac{v^2}{c^2}}} \\
    &E_\text{кин}= mc^2\cbr{\frac 1{\sqrt{1 - \frac{v^2}{c^2}}} - 1} = \frac{E_0}{c^2}\cbr{\frac 1{\sqrt{1 - \frac{v^2}{c^2}}} - 1} \\
    p, v:\quad&p = \frac{mv}{\sqrt{1 - \frac{v^2}{c^2}}} \implies m = \frac p v {\sqrt{1 - \frac{v^2}{c^2}}} \implies E_0 = mc^2 =\frac {pc^2} v {\sqrt{1 - \frac{v^2}{c^2}}} \\
    &E_\text{кин} = mc^2\cbr{\frac 1{\sqrt{1 - \frac{v^2}{c^2}}} - 1} = \frac p v {\sqrt{1 - \frac{v^2}{c^2}}}\cbr{\frac 1{\sqrt{1 - \frac{v^2}{c^2}}} - 1} = \frac p v \cbr{1 - {\sqrt{1 - \frac{v^2}{c^2}}}}
    \end{align*}
}

\variantsplitter

\addpersonalvariant{Матвей Кузьмин}

\tasknumber{1}%
\task{%
    Запишите
    \begin{itemize}
        \item постулаты специальной теории относительности,
        \item пример релятивистского эффекта, обнаружимый при скоростях гораздо меньше скорости света.
    \end{itemize}
}
\solutionspace{120pt}

\tasknumber{2}%
\task{%
    Запишите формулу для ...
    \begin{itemize}
        \item релятивистского замедления времени,
        \item классического импульса,
        \item релятивистского импульса тела,
        \item релятивистской кинетической энергии,
        \item связь между релятивистским импульсом и релятивистской энергией.
    \end{itemize}
    Обязательно подпишите все физические величины.
}
\solutionspace{150pt}

\tasknumber{3}%
\task{%
    Протон движется со скоростью $0{,}6\,c$, где $c$~--- скорость света в вакууме.
    Каково при этом отношение полной энергии частицы $E$ к его энергии покоя $E_0$?
}
\answer{%
    \begin{align*}
    E &= \frac{E_0}{\sqrt{1 - \frac{v^2}{c^2}}}
            \implies \frac E{E_0}
                = \frac 1{\sqrt{1 - \frac{v^2}{c^2}}}
                = \frac 1{\sqrt{1 - \sqr{0{,}6}}}
                \approx 1{,}250,
         \\
        {E_{\text{кин}}} &= E - E_0
            \implies \frac{E_{\text{кин}}}{E_0}
                = \frac E{E_0} - 1
                = \frac 1{\sqrt{1 - \frac{v^2}{c^2}}} - 1
                = \frac 1{\sqrt{1 - \sqr{0{,}6}}} - 1
                \approx 0{,}250.
    \end{align*}
}
\solutionspace{80pt}

\tasknumber{4}%
\task{%
    Полная энергия релятивистской частицы в три раза больше её энергии покоя.
    Найти скорость этой частицы: в долях $c$ и численное значение.
    Скорость света в вакууме $c = 3 \cdot 10^{8}\,\frac{\text{м}}{\text{с}}$.
}
\answer{%
    \begin{align*}
    E &= \frac{E_0}{\sqrt{1 - \frac{v^2}{c^2}}}\implies \sqrt{1 - \frac{v^2}{c^2}} = \frac{E_0}{E}\implies \frac{v^2}{c^2} = 1 - \sqr{\frac{E_0}{E}}\implies v = c \sqrt{1 - \sqr{\frac{E_0}{E}}} \approx 0{,}943c \approx 283 \cdot 10^{6}\,\frac{\text{м}}{\text{с}}.
    \end{align*}
}
\solutionspace{80pt}

\tasknumber{5}%
\task{%
    Кинетическая энергия релятивистской частицы в три раза больше её энергии покоя.
    Найти скорость этой частицы.
    Скорость света в вакууме $c = 3 \cdot 10^{8}\,\frac{\text{м}}{\text{с}}$.
}
\answer{%
    \begin{align*}
    E &= E_0 + E_{\text{кин}} \\
    E &= \frac{E_0}{\sqrt{1 - \frac{v^2}{c^2}}}\implies \sqrt{1 - \frac{v^2}{c^2}} = \frac{E_0}{E}\implies \frac{v^2}{c^2} = 1 - \sqr{\frac{E_0}{E}} \implies \\
    \implies &v = c \sqrt{1 - \sqr{\frac{E_0}{E}}} = c \sqrt{1 - \sqr{\frac{E_0}{E_0 + E_{\text{кин}} }}} = c \sqrt{1 - \frac 1 {\sqr{ 1 + \frac{E_{\text{кин}}}{E_0} }} }\approx 0{,}968c \approx 290 \cdot 10^{6}\,\frac{\text{м}}{\text{с}}.
    \end{align*}
}


\variantsplitter


\addpersonalvariant{Матвей Кузьмин}

\tasknumber{6}%
\task{%
    Электрон движется со скоростью $0{,}85\,c$, где $c$~--- скорость света в вакууме.
    Определите его импульс (в ответе приведите формулу и укажите численное значение).
}
\answer{%
    \begin{align*}
    E &= \frac{mc^2}{\sqrt{1 - \frac{v^2}{c^2}}}
            \approx \frac{9{,}1 \cdot 10^{-31}\,\text{кг} \cdot \sqr{3 \cdot 10^{8}\,\frac{\text{м}}{\text{с}}}}{\sqrt{1 - 0{,}85^2}}
            \approx 0{,}1555 \cdot 10^{-12}\,\text{Дж},
         \\
        E_{\text{кин}} &= \frac{mc^2}{\sqrt{1 - \frac{v^2}{c^2}}} - mc^2
            = mc^2 \cbr{\frac 1{\sqrt{1 - \frac{v^2}{c^2}}} - 1} \approx \\
            &\approx \cbr{9{,}1 \cdot 10^{-31}\,\text{кг} \cdot \sqr{3 \cdot 10^{8}\,\frac{\text{м}}{\text{с}}}}
            \cdot \cbr{\frac 1{\sqrt{1 - 0{,}85^2}} - 1}
            \approx 73{,}6 \cdot 10^{-15}\,\text{Дж},
         \\
        p &= \frac{mv}{\sqrt{1 - \frac{v^2}{c^2}}}
            \approx \frac{9{,}1 \cdot 10^{-31}\,\text{кг} \cdot 0{,}85 \cdot 3 \cdot 10^{8}\,\frac{\text{м}}{\text{с}}}{\sqrt{1 - 0{,}85^2}}
            \approx 0{,}441 \cdot 10^{-21}\,\frac{\text{кг}\cdot\text{м}}{\text{с}}.
    \end{align*}
}
\solutionspace{100pt}

\tasknumber{7}%
\task{%
    Кинетическая энергия частицы космических лучей в три раза превышает её энергию покоя.
    Определить отношение скорости частицы к скорости света.
}
\answer{%
    \begin{align*}
    E &= E_0 + E_{\text{кин}} \\
    E &= \frac{E_0}{\sqrt{1 - \frac{v^2}{c^2}}}\implies \sqrt{1 - \frac{v^2}{c^2}} = \frac{E_0}{E}\implies \frac{v^2}{c^2} = 1 - \sqr{\frac{E_0}{E}} \implies \\
    \implies \frac vc &= \sqrt{1 - \sqr{\frac{E_0}{E}}} = \sqrt{1 - \sqr{\frac{E_0}{E_0 + E_{\text{кин}} }}} \approx 0{,}968.
    \end{align*}
}
\solutionspace{80pt}

\tasknumber{8}%
\task{%
    Некоторая частица, пройдя ускоряющую разность потенциалов, приобрела импульс $4{,}2 \cdot 10^{-19}\,\frac{\text{кг}\cdot\text{м}}{\text{с}}$.
    Скорость частицы стала равной $2{,}4 \cdot 10^{8}\,\frac{\text{м}}{\text{с}}$.
    Найти массу частицы.
}
\answer{%
    $p = \frac{ mv }{\sqrt{1 - \frac{v^2}{c^2} }}\implies m = \frac pv \sqrt{1 - \frac{v^2}{c^2}}= \frac {4{,}2 \cdot 10^{-19}\,\frac{\text{кг}\cdot\text{м}}{\text{с}}}{2{,}4 \cdot 10^{8}\,\frac{\text{м}}{\text{с}}} \sqrt{1 - \sqr{\frac{2{,}4 \cdot 10^{8}\,\frac{\text{м}}{\text{с}}}{3 \cdot 10^{8}\,\frac{\text{м}}{\text{с}}}} } \approx 1{,}050 \cdot 10^{-27}\,\text{кг}.$
}
\solutionspace{80pt}

\tasknumber{9}%
\task{%
    При какой скорости движения (в м/с) релятивистское сокращение длины движущегося тела
    составит 10\%?
}
\answer{%
    \begin{align*}
    l_0 &= \frac l{\sqrt{1 - \frac{v^2}{c^2}}}
        \implies 1 - \frac{v^2}{c^2} = \sqr{\frac l{l_0}}
        \implies \frac v c = \sqrt{1 - \sqr{\frac l{l_0}}} \implies
         \\
        \implies v &= c\sqrt{1 - \sqr{\frac l{l_0}}}
        = 3 \cdot 10^{8}\,\frac{\text{м}}{\text{с}} \cdot \sqrt{1 - \sqr{\frac {l_0 - 0{,}10l_0}{l_0}}}
        = 3 \cdot 10^{8}\,\frac{\text{м}}{\text{с}} \cdot \sqrt{1 - \sqr{1 - 0{,}10}} \approx  \\
        &\approx 0{,}436c
        \approx 130{,}8 \cdot 10^{6}\,\frac{\text{м}}{\text{с}}
        \approx 471 \cdot 10^{6}\,\frac{\text{км}}{\text{ч}}.
    \end{align*}
}
\solutionspace{80pt}

\tasknumber{10}%
\task{%
    Стержень движется в продольном направлении с постоянной скоростью относительно инерциальной системы отсчёта.
    При каком значении скорости (в долях скорости света) длина стержня в этой системе отсчёта
    будет в  3  раза меньше его собственной длины?
}
\answer{%
    $l_0 = \frac l{\sqrt{1 - \frac{v^2}{c^2}}}\implies \sqrt{1 - \frac{v^2}{c^2}} = \frac{ l }{ l_0 }\implies \frac v c = \sqrt{1 - \sqr{\frac{ l }{ l_0 }}} \approx 0{,}943.$
}
\solutionspace{80pt}

\tasknumber{11}%
\task{%
    Какую скорость должно иметь движущееся тело, чтобы его продольные размеры уменьшились в три раза?
    Скорость света $c = 3 \cdot 10^{8}\,\frac{\text{м}}{\text{с}}$.
}
\answer{%
    $l_0 = \frac l{\sqrt{1 - \frac{v^2}{c^2}}}\implies \sqrt{1 - \frac{v^2}{c^2}} = \frac{ l }{ l_0 }\implies v = c\sqrt{1 - \sqr{\frac{ l }{ l_0 }}} \approx 283 \cdot 10^{6}\,\frac{\text{м}}{\text{с}}.$
}


\variantsplitter


\addpersonalvariant{Матвей Кузьмин}

\tasknumber{12}%
\task{%
    Время жизни мюона, измеренное наблюдателем, относительно которого мюон покоился, равно $\tau_0$
    Какое расстояние пролетит мюон в системе отсчёта, относительно которой он движется со скоростью $v$,
    сравнимой со скоростью света в вакууме $c$?
}
\answer{%
    $\ell = v\tau = v \frac{\tau_0}{\sqrt{1 - \frac{v^2}{c^2}}}$
}
\solutionspace{80pt}

\tasknumber{13}%
\task{%
    Если $c$ — скорость света в вакууме, то с какой скоростью должна двигаться нестабильная частица относительно наблюдателя,
    чтобы её время жизни было в три раза больше, чем у такой же, но покоящейся относительно наблюдателя частицы?
}
\answer{%
    $\tau = \frac{\tau_0}{\sqrt{1 - \frac{v^2}{c^2}}}\implies \sqrt{1 - \frac{v^2}{c^2}} = \frac{\tau_0}{\tau}\implies v = c\sqrt{1 - \sqr{\frac{\tau_0}{\tau}} } \approx 283 \cdot 10^{6}\,\frac{\text{м}}{\text{с}}.$
}
\solutionspace{80pt}

\tasknumber{14}%
\task{%
    Время жизни нестабильной частицы, входящего в состав космических лучей, измеренное земным наблюдателем,
    относительно которого частица двигалась со скоростью, составляющей 65\% скорости света в вакууме, оказалось равным $5{,}3\,\text{мкс}$.
    Каково время жизни частицы, покоящейся относительно наблюдателя?
}
\answer{%
    $t = \frac{t_0}{\sqrt{1 - \frac{v^2}{c^2}}}\implies t_0 = t\sqrt{1 - \frac{v^2}{c^2}} \approx 4{,}0 \cdot 10^{-6}\,\text{с}.$
}
\solutionspace{80pt}

\tasknumber{15}%
\task{%
    Частица увеличила в ускорителе свою скорость с $0{,}02c$ до $0{,}50c$.
    Во сколько раз выросла её кинетическая энергия?
}
\answer{%
    \begin{align*}
    E_{\text{кин.}} &= E - E_0 = \frac{mc^2}{\sqrt{1 - \frac{v^2}{c^2}}} - mc^2 = mc^2\cbr{ \frac1{\sqrt{1 - \frac{v^2}{c^2} }} - 1}.
    \\
    \frac{E_{\text{кин.
    2}}}{E_{\text{кин.
    1}}} &= \frac{\frac1{\sqrt{1 - \frac{v_2^2}{c^2} }} - 1}{\frac1{\sqrt{1 - \frac{v_1^2}{c^2} }} - 1}\approx 773{,}27
    \end{align*}
}
\solutionspace{120pt}

\tasknumber{16}%
\task{%
    Для частицы, движущейся с релятивистской скоростью,
    выразите $v$ и $E_\text{кин}$ через $c$, $p$ и $E_0$,
    где $E_\text{кин}$~--- кинетическая энергия частицы,
    а $E_0$, $p$ и $v$~--- её энергия покоя, импульс и скорость.
}
\answer{%
    \begin{align*}
    E_\text{кин}, E_0:\quad&E = E_\text{кин} + E_0 = \frac{E_0}{\sqrt{1 - \frac{v^2}{c^2}}} \implies \sqrt{1 - \frac{v^2}{c^2}} = \frac{E_0}{{E_0} + {E_\text{кин}}} \implies v = c\sqrt{1 - \sqr{\frac{E_0}{{E_0} + {E_\text{кин}}}}} \\
    &p = \frac{mv}{\sqrt{1 - \frac{v^2}{c^2}}} = \frac{E_0}{c^2} \cdot \sqrt{1 - \sqr{\frac{E_0}{{E_0} + {E_\text{кин}}}}} \cdot \frac{{E_\text{кин}} + {E_0}}{E_0} = \frac{E_0}{c^2} \cdot \sqrt{\sqr{\frac{{E_\text{кин}} + {E_0}}{E_0}} - 1}.
    \\
    E_\text{кин}, p:\quad&E_\text{кин} = E - E_0 = mc^2\cbr{\frac 1{\sqrt{1 - \frac{v^2}{c^2}}} - 1}, p = \frac{mv}{\sqrt{1 - \frac{v^2}{c^2}}} \implies \frac{E_\text{кин}}{p} = \frac{\frac 1{\sqrt{1 - \frac{v^2}{c^2}}} - 1}{\sqrt{1 - \frac{v^2}{c^2}}} \implies v = \ldots \\
    &E_0 = E - E_\text{кин} = \frac{E_0}{\sqrt{1 - \frac{v^2}{c^2}}} - E_\text{кин} \implies E_0 = \frac{E_\text{кин}}{\frac 1{\sqrt{1 - \frac{v^2}{c^2}}} - 1} = \ldots \\
    E_\text{кин}, v:\quad&E_\text{кин} = E - E_0 = mc^2\cbr{\frac 1{\sqrt{1 - \frac{v^2}{c^2}}} - 1} \implies m = \frac{E_\text{кин}}{c^2\cbr{\frac 1{\sqrt{1 - \frac{v^2}{c^2}}} - 1}} \\
    &E_0 = mc^2 = \frac{E_\text{кин}}{\frac 1{\sqrt{1 - \frac{v^2}{c^2}}} - 1} \\
    &p = \frac{mv}{\sqrt{1 - \frac{v^2}{c^2}}} = \frac{E_\text{кин}}{c^2\cbr{\frac 1{\sqrt{1 - \frac{v^2}{c^2}}} - 1}} \cdot \frac{v}{\sqrt{1 - \frac{v^2}{c^2}}} = \frac{{E_\text{кин}} v}{c^2\cbr{1 - {\sqrt{1 - \frac{v^2}{c^2}}}}} \\
    E_0, p:\quad&E_0 = mc^2, \quad p = \frac{mv}{\sqrt{1 - \frac{v^2}{c^2}}} \implies \frac{E_0}{p} = \frac{c^2}v{\sqrt{1 - \frac{v^2}{c^2}}} = c\sqrt{\frac{c^2}{v^2} - 1} \\
    &\sqr{\frac{E_0}{pc}} = \frac{c^2}{v^2} - 1 \implies \frac{v^2}{c^2} = \frac 1{1 + \frac{E_0^2}{p^2c^2}} \implies v = \frac c{\sqrt{1 + \frac{E_0^2}{p^2c^2}}} \\
    &{E_\text{кин}} = E - E_0 = \sqrt{E_0^2 + p^2c^2} - E_0 \\
    E_0, v:\quad&E_0 = mc^2 \implies m = \frac{E_0}{c^2} \qquad p = \frac{mv}{\sqrt{1 - \frac{v^2}{c^2}}} = \frac{E_0}{c^2} \cdot \frac{v}{\sqrt{1 - \frac{v^2}{c^2}}} \\
    &E_\text{кин}= mc^2\cbr{\frac 1{\sqrt{1 - \frac{v^2}{c^2}}} - 1} = \frac{E_0}{c^2}\cbr{\frac 1{\sqrt{1 - \frac{v^2}{c^2}}} - 1} \\
    p, v:\quad&p = \frac{mv}{\sqrt{1 - \frac{v^2}{c^2}}} \implies m = \frac p v {\sqrt{1 - \frac{v^2}{c^2}}} \implies E_0 = mc^2 =\frac {pc^2} v {\sqrt{1 - \frac{v^2}{c^2}}} \\
    &E_\text{кин} = mc^2\cbr{\frac 1{\sqrt{1 - \frac{v^2}{c^2}}} - 1} = \frac p v {\sqrt{1 - \frac{v^2}{c^2}}}\cbr{\frac 1{\sqrt{1 - \frac{v^2}{c^2}}} - 1} = \frac p v \cbr{1 - {\sqrt{1 - \frac{v^2}{c^2}}}}
    \end{align*}
}

\variantsplitter

\addpersonalvariant{Сергей Малышев}

\tasknumber{1}%
\task{%
    Запишите
    \begin{itemize}
        \item постулаты специальной теории относительности,
        \item пример релятивистского эффекта, обнаружимый при скоростях гораздо меньше скорости света.
    \end{itemize}
}
\solutionspace{120pt}

\tasknumber{2}%
\task{%
    Запишите формулу для ...
    \begin{itemize}
        \item релятивистского замедления времени,
        \item классического импульса,
        \item релятивистского импульса тела,
        \item релятивистской кинетической энергии,
        \item связь между релятивистским импульсом и релятивистской энергией.
    \end{itemize}
    Обязательно подпишите все физические величины.
}
\solutionspace{150pt}

\tasknumber{3}%
\task{%
    Электрон движется со скоростью $0{,}8\,c$, где $c$~--- скорость света в вакууме.
    Каково при этом отношение кинетической энергии частицы $E_\text{кин.}$ к его энергии покоя $E_0$?
}
\answer{%
    \begin{align*}
    E &= \frac{E_0}{\sqrt{1 - \frac{v^2}{c^2}}}
            \implies \frac E{E_0}
                = \frac 1{\sqrt{1 - \frac{v^2}{c^2}}}
                = \frac 1{\sqrt{1 - \sqr{0{,}8}}}
                \approx 1{,}667,
         \\
        {E_{\text{кин}}} &= E - E_0
            \implies \frac{E_{\text{кин}}}{E_0}
                = \frac E{E_0} - 1
                = \frac 1{\sqrt{1 - \frac{v^2}{c^2}}} - 1
                = \frac 1{\sqrt{1 - \sqr{0{,}8}}} - 1
                \approx 0{,}667.
    \end{align*}
}
\solutionspace{80pt}

\tasknumber{4}%
\task{%
    Полная энергия релятивистской частицы в четыре раза больше её энергии покоя.
    Найти скорость этой частицы: в долях $c$ и численное значение.
    Скорость света в вакууме $c = 3 \cdot 10^{8}\,\frac{\text{м}}{\text{с}}$.
}
\answer{%
    \begin{align*}
    E &= \frac{E_0}{\sqrt{1 - \frac{v^2}{c^2}}}\implies \sqrt{1 - \frac{v^2}{c^2}} = \frac{E_0}{E}\implies \frac{v^2}{c^2} = 1 - \sqr{\frac{E_0}{E}}\implies v = c \sqrt{1 - \sqr{\frac{E_0}{E}}} \approx 0{,}968c \approx 290 \cdot 10^{6}\,\frac{\text{м}}{\text{с}}.
    \end{align*}
}
\solutionspace{80pt}

\tasknumber{5}%
\task{%
    Кинетическая энергия релятивистской частицы в четыре раза больше её энергии покоя.
    Найти скорость этой частицы.
    Скорость света в вакууме $c = 3 \cdot 10^{8}\,\frac{\text{м}}{\text{с}}$.
}
\answer{%
    \begin{align*}
    E &= E_0 + E_{\text{кин}} \\
    E &= \frac{E_0}{\sqrt{1 - \frac{v^2}{c^2}}}\implies \sqrt{1 - \frac{v^2}{c^2}} = \frac{E_0}{E}\implies \frac{v^2}{c^2} = 1 - \sqr{\frac{E_0}{E}} \implies \\
    \implies &v = c \sqrt{1 - \sqr{\frac{E_0}{E}}} = c \sqrt{1 - \sqr{\frac{E_0}{E_0 + E_{\text{кин}} }}} = c \sqrt{1 - \frac 1 {\sqr{ 1 + \frac{E_{\text{кин}}}{E_0} }} }\approx 0{,}980c \approx 294 \cdot 10^{6}\,\frac{\text{м}}{\text{с}}.
    \end{align*}
}


\variantsplitter


\addpersonalvariant{Сергей Малышев}

\tasknumber{6}%
\task{%
    Электрон движется со скоростью $0{,}85\,c$, где $c$~--- скорость света в вакууме.
    Определите его полную энергию (в ответе приведите формулу и укажите численное значение).
}
\answer{%
    \begin{align*}
    E &= \frac{mc^2}{\sqrt{1 - \frac{v^2}{c^2}}}
            \approx \frac{9{,}1 \cdot 10^{-31}\,\text{кг} \cdot \sqr{3 \cdot 10^{8}\,\frac{\text{м}}{\text{с}}}}{\sqrt{1 - 0{,}85^2}}
            \approx 0{,}1555 \cdot 10^{-12}\,\text{Дж},
         \\
        E_{\text{кин}} &= \frac{mc^2}{\sqrt{1 - \frac{v^2}{c^2}}} - mc^2
            = mc^2 \cbr{\frac 1{\sqrt{1 - \frac{v^2}{c^2}}} - 1} \approx \\
            &\approx \cbr{9{,}1 \cdot 10^{-31}\,\text{кг} \cdot \sqr{3 \cdot 10^{8}\,\frac{\text{м}}{\text{с}}}}
            \cdot \cbr{\frac 1{\sqrt{1 - 0{,}85^2}} - 1}
            \approx 73{,}6 \cdot 10^{-15}\,\text{Дж},
         \\
        p &= \frac{mv}{\sqrt{1 - \frac{v^2}{c^2}}}
            \approx \frac{9{,}1 \cdot 10^{-31}\,\text{кг} \cdot 0{,}85 \cdot 3 \cdot 10^{8}\,\frac{\text{м}}{\text{с}}}{\sqrt{1 - 0{,}85^2}}
            \approx 0{,}441 \cdot 10^{-21}\,\frac{\text{кг}\cdot\text{м}}{\text{с}}.
    \end{align*}
}
\solutionspace{100pt}

\tasknumber{7}%
\task{%
    Кинетическая энергия частицы космических лучей в четыре раза превышает её энергию покоя.
    Определить отношение скорости частицы к скорости света.
}
\answer{%
    \begin{align*}
    E &= E_0 + E_{\text{кин}} \\
    E &= \frac{E_0}{\sqrt{1 - \frac{v^2}{c^2}}}\implies \sqrt{1 - \frac{v^2}{c^2}} = \frac{E_0}{E}\implies \frac{v^2}{c^2} = 1 - \sqr{\frac{E_0}{E}} \implies \\
    \implies \frac vc &= \sqrt{1 - \sqr{\frac{E_0}{E}}} = \sqrt{1 - \sqr{\frac{E_0}{E_0 + E_{\text{кин}} }}} \approx 0{,}980.
    \end{align*}
}
\solutionspace{80pt}

\tasknumber{8}%
\task{%
    Некоторая частица, пройдя ускоряющую разность потенциалов, приобрела импульс $3 \cdot 10^{-19}\,\frac{\text{кг}\cdot\text{м}}{\text{с}}$.
    Скорость частицы стала равной $2{,}4 \cdot 10^{8}\,\frac{\text{м}}{\text{с}}$.
    Найти массу частицы.
}
\answer{%
    $p = \frac{ mv }{\sqrt{1 - \frac{v^2}{c^2} }}\implies m = \frac pv \sqrt{1 - \frac{v^2}{c^2}}= \frac {3 \cdot 10^{-19}\,\frac{\text{кг}\cdot\text{м}}{\text{с}}}{2{,}4 \cdot 10^{8}\,\frac{\text{м}}{\text{с}}} \sqrt{1 - \sqr{\frac{2{,}4 \cdot 10^{8}\,\frac{\text{м}}{\text{с}}}{3 \cdot 10^{8}\,\frac{\text{м}}{\text{с}}}} } \approx 0{,}75 \cdot 10^{-27}\,\text{кг}.$
}
\solutionspace{80pt}

\tasknumber{9}%
\task{%
    При какой скорости движения (в м/с) релятивистское сокращение длины движущегося тела
    составит 10\%?
}
\answer{%
    \begin{align*}
    l_0 &= \frac l{\sqrt{1 - \frac{v^2}{c^2}}}
        \implies 1 - \frac{v^2}{c^2} = \sqr{\frac l{l_0}}
        \implies \frac v c = \sqrt{1 - \sqr{\frac l{l_0}}} \implies
         \\
        \implies v &= c\sqrt{1 - \sqr{\frac l{l_0}}}
        = 3 \cdot 10^{8}\,\frac{\text{м}}{\text{с}} \cdot \sqrt{1 - \sqr{\frac {l_0 - 0{,}10l_0}{l_0}}}
        = 3 \cdot 10^{8}\,\frac{\text{м}}{\text{с}} \cdot \sqrt{1 - \sqr{1 - 0{,}10}} \approx  \\
        &\approx 0{,}436c
        \approx 130{,}8 \cdot 10^{6}\,\frac{\text{м}}{\text{с}}
        \approx 471 \cdot 10^{6}\,\frac{\text{км}}{\text{ч}}.
    \end{align*}
}
\solutionspace{80pt}

\tasknumber{10}%
\task{%
    Стержень движется в продольном направлении с постоянной скоростью относительно инерциальной системы отсчёта.
    При каком значении скорости (в долях скорости света) длина стержня в этой системе отсчёта
    будет в  2{,}5  раза меньше его собственной длины?
}
\answer{%
    $l_0 = \frac l{\sqrt{1 - \frac{v^2}{c^2}}}\implies \sqrt{1 - \frac{v^2}{c^2}} = \frac{ l }{ l_0 }\implies \frac v c = \sqrt{1 - \sqr{\frac{ l }{ l_0 }}} \approx 0{,}917.$
}
\solutionspace{80pt}

\tasknumber{11}%
\task{%
    Какую скорость должно иметь движущееся тело, чтобы его продольные размеры уменьшились в три раза?
    Скорость света $c = 3 \cdot 10^{8}\,\frac{\text{м}}{\text{с}}$.
}
\answer{%
    $l_0 = \frac l{\sqrt{1 - \frac{v^2}{c^2}}}\implies \sqrt{1 - \frac{v^2}{c^2}} = \frac{ l }{ l_0 }\implies v = c\sqrt{1 - \sqr{\frac{ l }{ l_0 }}} \approx 283 \cdot 10^{6}\,\frac{\text{м}}{\text{с}}.$
}


\variantsplitter


\addpersonalvariant{Сергей Малышев}

\tasknumber{12}%
\task{%
    Время жизни мюона, измеренное наблюдателем, относительно которого мюон покоился, равно $\tau_0$
    Какое расстояние пролетит мюон в системе отсчёта, относительно которой он движется со скоростью $v$,
    сравнимой со скоростью света в вакууме $c$?
}
\answer{%
    $\ell = v\tau = v \frac{\tau_0}{\sqrt{1 - \frac{v^2}{c^2}}}$
}
\solutionspace{80pt}

\tasknumber{13}%
\task{%
    Если $c$ — скорость света в вакууме, то с какой скоростью должна двигаться нестабильная частица относительно наблюдателя,
    чтобы её время жизни было в четыре раза больше, чем у такой же, но покоящейся относительно наблюдателя частицы?
}
\answer{%
    $\tau = \frac{\tau_0}{\sqrt{1 - \frac{v^2}{c^2}}}\implies \sqrt{1 - \frac{v^2}{c^2}} = \frac{\tau_0}{\tau}\implies v = c\sqrt{1 - \sqr{\frac{\tau_0}{\tau}} } \approx 290 \cdot 10^{6}\,\frac{\text{м}}{\text{с}}.$
}
\solutionspace{80pt}

\tasknumber{14}%
\task{%
    Время жизни нестабильной частицы, входящего в состав космических лучей, измеренное земным наблюдателем,
    относительно которого частица двигалась со скоростью, составляющей 65\% скорости света в вакууме, оказалось равным $5{,}3\,\text{мкс}$.
    Каково время жизни частицы, покоящейся относительно наблюдателя?
}
\answer{%
    $t = \frac{t_0}{\sqrt{1 - \frac{v^2}{c^2}}}\implies t_0 = t\sqrt{1 - \frac{v^2}{c^2}} \approx 4{,}0 \cdot 10^{-6}\,\text{с}.$
}
\solutionspace{80pt}

\tasknumber{15}%
\task{%
    Частица увеличила в ускорителе свою скорость с $0{,}02c$ до $0{,}90c$.
    Во сколько раз выросла её кинетическая энергия?
}
\answer{%
    \begin{align*}
    E_{\text{кин.}} &= E - E_0 = \frac{mc^2}{\sqrt{1 - \frac{v^2}{c^2}}} - mc^2 = mc^2\cbr{ \frac1{\sqrt{1 - \frac{v^2}{c^2} }} - 1}.
    \\
    \frac{E_{\text{кин.
    2}}}{E_{\text{кин.
    1}}} &= \frac{\frac1{\sqrt{1 - \frac{v_2^2}{c^2} }} - 1}{\frac1{\sqrt{1 - \frac{v_1^2}{c^2} }} - 1}\approx 6468{,}85
    \end{align*}
}
\solutionspace{120pt}

\tasknumber{16}%
\task{%
    Для частицы, движущейся с релятивистской скоростью,
    выразите $E_0$ и $v$ через $c$, $E_\text{кин}$ и $p$,
    где $E_\text{кин}$~--- кинетическая энергия частицы,
    а $E_0$, $p$ и $v$~--- её энергия покоя, импульс и скорость.
}
\answer{%
    \begin{align*}
    E_\text{кин}, E_0:\quad&E = E_\text{кин} + E_0 = \frac{E_0}{\sqrt{1 - \frac{v^2}{c^2}}} \implies \sqrt{1 - \frac{v^2}{c^2}} = \frac{E_0}{{E_0} + {E_\text{кин}}} \implies v = c\sqrt{1 - \sqr{\frac{E_0}{{E_0} + {E_\text{кин}}}}} \\
    &p = \frac{mv}{\sqrt{1 - \frac{v^2}{c^2}}} = \frac{E_0}{c^2} \cdot \sqrt{1 - \sqr{\frac{E_0}{{E_0} + {E_\text{кин}}}}} \cdot \frac{{E_\text{кин}} + {E_0}}{E_0} = \frac{E_0}{c^2} \cdot \sqrt{\sqr{\frac{{E_\text{кин}} + {E_0}}{E_0}} - 1}.
    \\
    E_\text{кин}, p:\quad&E_\text{кин} = E - E_0 = mc^2\cbr{\frac 1{\sqrt{1 - \frac{v^2}{c^2}}} - 1}, p = \frac{mv}{\sqrt{1 - \frac{v^2}{c^2}}} \implies \frac{E_\text{кин}}{p} = \frac{\frac 1{\sqrt{1 - \frac{v^2}{c^2}}} - 1}{\sqrt{1 - \frac{v^2}{c^2}}} \implies v = \ldots \\
    &E_0 = E - E_\text{кин} = \frac{E_0}{\sqrt{1 - \frac{v^2}{c^2}}} - E_\text{кин} \implies E_0 = \frac{E_\text{кин}}{\frac 1{\sqrt{1 - \frac{v^2}{c^2}}} - 1} = \ldots \\
    E_\text{кин}, v:\quad&E_\text{кин} = E - E_0 = mc^2\cbr{\frac 1{\sqrt{1 - \frac{v^2}{c^2}}} - 1} \implies m = \frac{E_\text{кин}}{c^2\cbr{\frac 1{\sqrt{1 - \frac{v^2}{c^2}}} - 1}} \\
    &E_0 = mc^2 = \frac{E_\text{кин}}{\frac 1{\sqrt{1 - \frac{v^2}{c^2}}} - 1} \\
    &p = \frac{mv}{\sqrt{1 - \frac{v^2}{c^2}}} = \frac{E_\text{кин}}{c^2\cbr{\frac 1{\sqrt{1 - \frac{v^2}{c^2}}} - 1}} \cdot \frac{v}{\sqrt{1 - \frac{v^2}{c^2}}} = \frac{{E_\text{кин}} v}{c^2\cbr{1 - {\sqrt{1 - \frac{v^2}{c^2}}}}} \\
    E_0, p:\quad&E_0 = mc^2, \quad p = \frac{mv}{\sqrt{1 - \frac{v^2}{c^2}}} \implies \frac{E_0}{p} = \frac{c^2}v{\sqrt{1 - \frac{v^2}{c^2}}} = c\sqrt{\frac{c^2}{v^2} - 1} \\
    &\sqr{\frac{E_0}{pc}} = \frac{c^2}{v^2} - 1 \implies \frac{v^2}{c^2} = \frac 1{1 + \frac{E_0^2}{p^2c^2}} \implies v = \frac c{\sqrt{1 + \frac{E_0^2}{p^2c^2}}} \\
    &{E_\text{кин}} = E - E_0 = \sqrt{E_0^2 + p^2c^2} - E_0 \\
    E_0, v:\quad&E_0 = mc^2 \implies m = \frac{E_0}{c^2} \qquad p = \frac{mv}{\sqrt{1 - \frac{v^2}{c^2}}} = \frac{E_0}{c^2} \cdot \frac{v}{\sqrt{1 - \frac{v^2}{c^2}}} \\
    &E_\text{кин}= mc^2\cbr{\frac 1{\sqrt{1 - \frac{v^2}{c^2}}} - 1} = \frac{E_0}{c^2}\cbr{\frac 1{\sqrt{1 - \frac{v^2}{c^2}}} - 1} \\
    p, v:\quad&p = \frac{mv}{\sqrt{1 - \frac{v^2}{c^2}}} \implies m = \frac p v {\sqrt{1 - \frac{v^2}{c^2}}} \implies E_0 = mc^2 =\frac {pc^2} v {\sqrt{1 - \frac{v^2}{c^2}}} \\
    &E_\text{кин} = mc^2\cbr{\frac 1{\sqrt{1 - \frac{v^2}{c^2}}} - 1} = \frac p v {\sqrt{1 - \frac{v^2}{c^2}}}\cbr{\frac 1{\sqrt{1 - \frac{v^2}{c^2}}} - 1} = \frac p v \cbr{1 - {\sqrt{1 - \frac{v^2}{c^2}}}}
    \end{align*}
}

\variantsplitter

\addpersonalvariant{Алина Полканова}

\tasknumber{1}%
\task{%
    Запишите
    \begin{itemize}
        \item постулаты специальной теории относительности,
        \item пример релятивистского эффекта, обнаружимый при скоростях гораздо меньше скорости света.
    \end{itemize}
}
\solutionspace{120pt}

\tasknumber{2}%
\task{%
    Запишите формулу для ...
    \begin{itemize}
        \item релятивистского замедления времени,
        \item классического импульса,
        \item релятивистской энергии тела,
        \item энергии покоя тела,
        \item связь между релятивистским импульсом и релятивистской энергией.
    \end{itemize}
    Обязательно подпишите все физические величины.
}
\solutionspace{150pt}

\tasknumber{3}%
\task{%
    Протон движется со скоростью $0{,}9\,c$, где $c$~--- скорость света в вакууме.
    Каково при этом отношение полной энергии частицы $E$ к его энергии покоя $E_0$?
}
\answer{%
    \begin{align*}
    E &= \frac{E_0}{\sqrt{1 - \frac{v^2}{c^2}}}
            \implies \frac E{E_0}
                = \frac 1{\sqrt{1 - \frac{v^2}{c^2}}}
                = \frac 1{\sqrt{1 - \sqr{0{,}9}}}
                \approx 2{,}294,
         \\
        {E_{\text{кин}}} &= E - E_0
            \implies \frac{E_{\text{кин}}}{E_0}
                = \frac E{E_0} - 1
                = \frac 1{\sqrt{1 - \frac{v^2}{c^2}}} - 1
                = \frac 1{\sqrt{1 - \sqr{0{,}9}}} - 1
                \approx 1{,}294.
    \end{align*}
}
\solutionspace{80pt}

\tasknumber{4}%
\task{%
    Полная энергия релятивистской частицы в три раза больше её энергии покоя.
    Найти скорость этой частицы: в долях $c$ и численное значение.
    Скорость света в вакууме $c = 3 \cdot 10^{8}\,\frac{\text{м}}{\text{с}}$.
}
\answer{%
    \begin{align*}
    E &= \frac{E_0}{\sqrt{1 - \frac{v^2}{c^2}}}\implies \sqrt{1 - \frac{v^2}{c^2}} = \frac{E_0}{E}\implies \frac{v^2}{c^2} = 1 - \sqr{\frac{E_0}{E}}\implies v = c \sqrt{1 - \sqr{\frac{E_0}{E}}} \approx 0{,}943c \approx 283 \cdot 10^{6}\,\frac{\text{м}}{\text{с}}.
    \end{align*}
}
\solutionspace{80pt}

\tasknumber{5}%
\task{%
    Кинетическая энергия релятивистской частицы в три раза больше её энергии покоя.
    Найти скорость этой частицы.
    Скорость света в вакууме $c = 3 \cdot 10^{8}\,\frac{\text{м}}{\text{с}}$.
}
\answer{%
    \begin{align*}
    E &= E_0 + E_{\text{кин}} \\
    E &= \frac{E_0}{\sqrt{1 - \frac{v^2}{c^2}}}\implies \sqrt{1 - \frac{v^2}{c^2}} = \frac{E_0}{E}\implies \frac{v^2}{c^2} = 1 - \sqr{\frac{E_0}{E}} \implies \\
    \implies &v = c \sqrt{1 - \sqr{\frac{E_0}{E}}} = c \sqrt{1 - \sqr{\frac{E_0}{E_0 + E_{\text{кин}} }}} = c \sqrt{1 - \frac 1 {\sqr{ 1 + \frac{E_{\text{кин}}}{E_0} }} }\approx 0{,}968c \approx 290 \cdot 10^{6}\,\frac{\text{м}}{\text{с}}.
    \end{align*}
}


\variantsplitter


\addpersonalvariant{Алина Полканова}

\tasknumber{6}%
\task{%
    Электрон движется со скоростью $0{,}75\,c$, где $c$~--- скорость света в вакууме.
    Определите его импульс (в ответе приведите формулу и укажите численное значение).
}
\answer{%
    \begin{align*}
    E &= \frac{mc^2}{\sqrt{1 - \frac{v^2}{c^2}}}
            \approx \frac{9{,}1 \cdot 10^{-31}\,\text{кг} \cdot \sqr{3 \cdot 10^{8}\,\frac{\text{м}}{\text{с}}}}{\sqrt{1 - 0{,}75^2}}
            \approx 0{,}1238 \cdot 10^{-12}\,\text{Дж},
         \\
        E_{\text{кин}} &= \frac{mc^2}{\sqrt{1 - \frac{v^2}{c^2}}} - mc^2
            = mc^2 \cbr{\frac 1{\sqrt{1 - \frac{v^2}{c^2}}} - 1} \approx \\
            &\approx \cbr{9{,}1 \cdot 10^{-31}\,\text{кг} \cdot \sqr{3 \cdot 10^{8}\,\frac{\text{м}}{\text{с}}}}
            \cdot \cbr{\frac 1{\sqrt{1 - 0{,}75^2}} - 1}
            \approx 41{,}9 \cdot 10^{-15}\,\text{Дж},
         \\
        p &= \frac{mv}{\sqrt{1 - \frac{v^2}{c^2}}}
            \approx \frac{9{,}1 \cdot 10^{-31}\,\text{кг} \cdot 0{,}75 \cdot 3 \cdot 10^{8}\,\frac{\text{м}}{\text{с}}}{\sqrt{1 - 0{,}75^2}}
            \approx 0{,}310 \cdot 10^{-21}\,\frac{\text{кг}\cdot\text{м}}{\text{с}}.
    \end{align*}
}
\solutionspace{100pt}

\tasknumber{7}%
\task{%
    Кинетическая энергия частицы космических лучей в три раза превышает её энергию покоя.
    Определить отношение скорости частицы к скорости света.
}
\answer{%
    \begin{align*}
    E &= E_0 + E_{\text{кин}} \\
    E &= \frac{E_0}{\sqrt{1 - \frac{v^2}{c^2}}}\implies \sqrt{1 - \frac{v^2}{c^2}} = \frac{E_0}{E}\implies \frac{v^2}{c^2} = 1 - \sqr{\frac{E_0}{E}} \implies \\
    \implies \frac vc &= \sqrt{1 - \sqr{\frac{E_0}{E}}} = \sqrt{1 - \sqr{\frac{E_0}{E_0 + E_{\text{кин}} }}} \approx 0{,}968.
    \end{align*}
}
\solutionspace{80pt}

\tasknumber{8}%
\task{%
    Некоторая частица, пройдя ускоряющую разность потенциалов, приобрела импульс $3 \cdot 10^{-19}\,\frac{\text{кг}\cdot\text{м}}{\text{с}}$.
    Скорость частицы стала равной $1{,}5 \cdot 10^{8}\,\frac{\text{м}}{\text{с}}$.
    Найти массу частицы.
}
\answer{%
    $p = \frac{ mv }{\sqrt{1 - \frac{v^2}{c^2} }}\implies m = \frac pv \sqrt{1 - \frac{v^2}{c^2}}= \frac {3 \cdot 10^{-19}\,\frac{\text{кг}\cdot\text{м}}{\text{с}}}{1{,}5 \cdot 10^{8}\,\frac{\text{м}}{\text{с}}} \sqrt{1 - \sqr{\frac{1{,}5 \cdot 10^{8}\,\frac{\text{м}}{\text{с}}}{3 \cdot 10^{8}\,\frac{\text{м}}{\text{с}}}} } \approx 1{,}73 \cdot 10^{-27}\,\text{кг}.$
}
\solutionspace{80pt}

\tasknumber{9}%
\task{%
    При какой скорости движения (в долях скорости света) релятивистское сокращение длины движущегося тела
    составит 10\%?
}
\answer{%
    \begin{align*}
    l_0 &= \frac l{\sqrt{1 - \frac{v^2}{c^2}}}
        \implies 1 - \frac{v^2}{c^2} = \sqr{\frac l{l_0}}
        \implies \frac v c = \sqrt{1 - \sqr{\frac l{l_0}}} \implies
         \\
        \implies v &= c\sqrt{1 - \sqr{\frac l{l_0}}}
        = 3 \cdot 10^{8}\,\frac{\text{м}}{\text{с}} \cdot \sqrt{1 - \sqr{\frac {l_0 - 0{,}10l_0}{l_0}}}
        = 3 \cdot 10^{8}\,\frac{\text{м}}{\text{с}} \cdot \sqrt{1 - \sqr{1 - 0{,}10}} \approx  \\
        &\approx 0{,}436c
        \approx 130{,}8 \cdot 10^{6}\,\frac{\text{м}}{\text{с}}
        \approx 471 \cdot 10^{6}\,\frac{\text{км}}{\text{ч}}.
    \end{align*}
}
\solutionspace{80pt}

\tasknumber{10}%
\task{%
    Стержень движется в продольном направлении с постоянной скоростью относительно инерциальной системы отсчёта.
    При каком значении скорости (в долях скорости света) длина стержня в этой системе отсчёта
    будет в  2{,}5  раза меньше его собственной длины?
}
\answer{%
    $l_0 = \frac l{\sqrt{1 - \frac{v^2}{c^2}}}\implies \sqrt{1 - \frac{v^2}{c^2}} = \frac{ l }{ l_0 }\implies \frac v c = \sqrt{1 - \sqr{\frac{ l }{ l_0 }}} \approx 0{,}917.$
}
\solutionspace{80pt}

\tasknumber{11}%
\task{%
    Какую скорость должно иметь движущееся тело, чтобы его продольные размеры уменьшились в пять раз?
    Скорость света $c = 3 \cdot 10^{8}\,\frac{\text{м}}{\text{с}}$.
}
\answer{%
    $l_0 = \frac l{\sqrt{1 - \frac{v^2}{c^2}}}\implies \sqrt{1 - \frac{v^2}{c^2}} = \frac{ l }{ l_0 }\implies v = c\sqrt{1 - \sqr{\frac{ l }{ l_0 }}} \approx 294 \cdot 10^{6}\,\frac{\text{м}}{\text{с}}.$
}


\variantsplitter


\addpersonalvariant{Алина Полканова}

\tasknumber{12}%
\task{%
    Время жизни мюона, измеренное наблюдателем, относительно которого мюон покоился, равно $\tau_0$
    Какое расстояние пролетит мюон в системе отсчёта, относительно которой он движется со скоростью $v$,
    сравнимой со скоростью света в вакууме $c$?
}
\answer{%
    $\ell = v\tau = v \frac{\tau_0}{\sqrt{1 - \frac{v^2}{c^2}}}$
}
\solutionspace{80pt}

\tasknumber{13}%
\task{%
    Если $c$ — скорость света в вакууме, то с какой скоростью должна двигаться нестабильная частица относительно наблюдателя,
    чтобы её время жизни было в три раза больше, чем у такой же, но покоящейся относительно наблюдателя частицы?
}
\answer{%
    $\tau = \frac{\tau_0}{\sqrt{1 - \frac{v^2}{c^2}}}\implies \sqrt{1 - \frac{v^2}{c^2}} = \frac{\tau_0}{\tau}\implies v = c\sqrt{1 - \sqr{\frac{\tau_0}{\tau}} } \approx 283 \cdot 10^{6}\,\frac{\text{м}}{\text{с}}.$
}
\solutionspace{80pt}

\tasknumber{14}%
\task{%
    Время жизни нестабильной частицы, входящего в состав космических лучей, измеренное земным наблюдателем,
    относительно которого частица двигалась со скоростью, составляющей 65\% скорости света в вакууме, оказалось равным $6{,}4\,\text{мкс}$.
    Каково время жизни частицы, покоящейся относительно наблюдателя?
}
\answer{%
    $t = \frac{t_0}{\sqrt{1 - \frac{v^2}{c^2}}}\implies t_0 = t\sqrt{1 - \frac{v^2}{c^2}} \approx 4{,}9 \cdot 10^{-6}\,\text{с}.$
}
\solutionspace{80pt}

\tasknumber{15}%
\task{%
    Частица увеличила в ускорителе свою скорость с $0{,}03c$ до $0{,}50c$.
    Во сколько раз выросла её кинетическая энергия?
}
\answer{%
    \begin{align*}
    E_{\text{кин.}} &= E - E_0 = \frac{mc^2}{\sqrt{1 - \frac{v^2}{c^2}}} - mc^2 = mc^2\cbr{ \frac1{\sqrt{1 - \frac{v^2}{c^2} }} - 1}.
    \\
    \frac{E_{\text{кин.
    2}}}{E_{\text{кин.
    1}}} &= \frac{\frac1{\sqrt{1 - \frac{v_2^2}{c^2} }} - 1}{\frac1{\sqrt{1 - \frac{v_1^2}{c^2} }} - 1}\approx 343{,}55
    \end{align*}
}
\solutionspace{120pt}

\tasknumber{16}%
\task{%
    Для частицы, движущейся с релятивистской скоростью,
    выразите $E_0$ и $E_\text{кин}$ через $c$, $p$ и $v$,
    где $E_\text{кин}$~--- кинетическая энергия частицы,
    а $E_0$, $p$ и $v$~--- её энергия покоя, импульс и скорость.
}
\answer{%
    \begin{align*}
    E_\text{кин}, E_0:\quad&E = E_\text{кин} + E_0 = \frac{E_0}{\sqrt{1 - \frac{v^2}{c^2}}} \implies \sqrt{1 - \frac{v^2}{c^2}} = \frac{E_0}{{E_0} + {E_\text{кин}}} \implies v = c\sqrt{1 - \sqr{\frac{E_0}{{E_0} + {E_\text{кин}}}}} \\
    &p = \frac{mv}{\sqrt{1 - \frac{v^2}{c^2}}} = \frac{E_0}{c^2} \cdot \sqrt{1 - \sqr{\frac{E_0}{{E_0} + {E_\text{кин}}}}} \cdot \frac{{E_\text{кин}} + {E_0}}{E_0} = \frac{E_0}{c^2} \cdot \sqrt{\sqr{\frac{{E_\text{кин}} + {E_0}}{E_0}} - 1}.
    \\
    E_\text{кин}, p:\quad&E_\text{кин} = E - E_0 = mc^2\cbr{\frac 1{\sqrt{1 - \frac{v^2}{c^2}}} - 1}, p = \frac{mv}{\sqrt{1 - \frac{v^2}{c^2}}} \implies \frac{E_\text{кин}}{p} = \frac{\frac 1{\sqrt{1 - \frac{v^2}{c^2}}} - 1}{\sqrt{1 - \frac{v^2}{c^2}}} \implies v = \ldots \\
    &E_0 = E - E_\text{кин} = \frac{E_0}{\sqrt{1 - \frac{v^2}{c^2}}} - E_\text{кин} \implies E_0 = \frac{E_\text{кин}}{\frac 1{\sqrt{1 - \frac{v^2}{c^2}}} - 1} = \ldots \\
    E_\text{кин}, v:\quad&E_\text{кин} = E - E_0 = mc^2\cbr{\frac 1{\sqrt{1 - \frac{v^2}{c^2}}} - 1} \implies m = \frac{E_\text{кин}}{c^2\cbr{\frac 1{\sqrt{1 - \frac{v^2}{c^2}}} - 1}} \\
    &E_0 = mc^2 = \frac{E_\text{кин}}{\frac 1{\sqrt{1 - \frac{v^2}{c^2}}} - 1} \\
    &p = \frac{mv}{\sqrt{1 - \frac{v^2}{c^2}}} = \frac{E_\text{кин}}{c^2\cbr{\frac 1{\sqrt{1 - \frac{v^2}{c^2}}} - 1}} \cdot \frac{v}{\sqrt{1 - \frac{v^2}{c^2}}} = \frac{{E_\text{кин}} v}{c^2\cbr{1 - {\sqrt{1 - \frac{v^2}{c^2}}}}} \\
    E_0, p:\quad&E_0 = mc^2, \quad p = \frac{mv}{\sqrt{1 - \frac{v^2}{c^2}}} \implies \frac{E_0}{p} = \frac{c^2}v{\sqrt{1 - \frac{v^2}{c^2}}} = c\sqrt{\frac{c^2}{v^2} - 1} \\
    &\sqr{\frac{E_0}{pc}} = \frac{c^2}{v^2} - 1 \implies \frac{v^2}{c^2} = \frac 1{1 + \frac{E_0^2}{p^2c^2}} \implies v = \frac c{\sqrt{1 + \frac{E_0^2}{p^2c^2}}} \\
    &{E_\text{кин}} = E - E_0 = \sqrt{E_0^2 + p^2c^2} - E_0 \\
    E_0, v:\quad&E_0 = mc^2 \implies m = \frac{E_0}{c^2} \qquad p = \frac{mv}{\sqrt{1 - \frac{v^2}{c^2}}} = \frac{E_0}{c^2} \cdot \frac{v}{\sqrt{1 - \frac{v^2}{c^2}}} \\
    &E_\text{кин}= mc^2\cbr{\frac 1{\sqrt{1 - \frac{v^2}{c^2}}} - 1} = \frac{E_0}{c^2}\cbr{\frac 1{\sqrt{1 - \frac{v^2}{c^2}}} - 1} \\
    p, v:\quad&p = \frac{mv}{\sqrt{1 - \frac{v^2}{c^2}}} \implies m = \frac p v {\sqrt{1 - \frac{v^2}{c^2}}} \implies E_0 = mc^2 =\frac {pc^2} v {\sqrt{1 - \frac{v^2}{c^2}}} \\
    &E_\text{кин} = mc^2\cbr{\frac 1{\sqrt{1 - \frac{v^2}{c^2}}} - 1} = \frac p v {\sqrt{1 - \frac{v^2}{c^2}}}\cbr{\frac 1{\sqrt{1 - \frac{v^2}{c^2}}} - 1} = \frac p v \cbr{1 - {\sqrt{1 - \frac{v^2}{c^2}}}}
    \end{align*}
}

\variantsplitter

\addpersonalvariant{Сергей Пономарёв}

\tasknumber{1}%
\task{%
    Запишите
    \begin{itemize}
        \item постулаты специальной теории относительности,
        \item пример релятивистского эффекта, обнаружимый при скоростях гораздо меньше скорости света.
    \end{itemize}
}
\solutionspace{120pt}

\tasknumber{2}%
\task{%
    Запишите формулу для ...
    \begin{itemize}
        \item релятивистского сжатия,
        \item классической полной механической энергии тела,
        \item релятивистской энергии тела,
        \item энергии покоя тела,
        \item связь между релятивистским импульсом и релятивистской энергией.
    \end{itemize}
    Обязательно подпишите все физические величины.
}
\solutionspace{150pt}

\tasknumber{3}%
\task{%
    Электрон движется со скоростью $0{,}9\,c$, где $c$~--- скорость света в вакууме.
    Каково при этом отношение полной энергии частицы $E$ к его энергии покоя $E_0$?
}
\answer{%
    \begin{align*}
    E &= \frac{E_0}{\sqrt{1 - \frac{v^2}{c^2}}}
            \implies \frac E{E_0}
                = \frac 1{\sqrt{1 - \frac{v^2}{c^2}}}
                = \frac 1{\sqrt{1 - \sqr{0{,}9}}}
                \approx 2{,}294,
         \\
        {E_{\text{кин}}} &= E - E_0
            \implies \frac{E_{\text{кин}}}{E_0}
                = \frac E{E_0} - 1
                = \frac 1{\sqrt{1 - \frac{v^2}{c^2}}} - 1
                = \frac 1{\sqrt{1 - \sqr{0{,}9}}} - 1
                \approx 1{,}294.
    \end{align*}
}
\solutionspace{80pt}

\tasknumber{4}%
\task{%
    Полная энергия релятивистской частицы в пять раз больше её энергии покоя.
    Найти скорость этой частицы: в долях $c$ и численное значение.
    Скорость света в вакууме $c = 3 \cdot 10^{8}\,\frac{\text{м}}{\text{с}}$.
}
\answer{%
    \begin{align*}
    E &= \frac{E_0}{\sqrt{1 - \frac{v^2}{c^2}}}\implies \sqrt{1 - \frac{v^2}{c^2}} = \frac{E_0}{E}\implies \frac{v^2}{c^2} = 1 - \sqr{\frac{E_0}{E}}\implies v = c \sqrt{1 - \sqr{\frac{E_0}{E}}} \approx 0{,}980c \approx 294 \cdot 10^{6}\,\frac{\text{м}}{\text{с}}.
    \end{align*}
}
\solutionspace{80pt}

\tasknumber{5}%
\task{%
    Кинетическая энергия релятивистской частицы в пять раз больше её энергии покоя.
    Найти скорость этой частицы.
    Скорость света в вакууме $c = 3 \cdot 10^{8}\,\frac{\text{м}}{\text{с}}$.
}
\answer{%
    \begin{align*}
    E &= E_0 + E_{\text{кин}} \\
    E &= \frac{E_0}{\sqrt{1 - \frac{v^2}{c^2}}}\implies \sqrt{1 - \frac{v^2}{c^2}} = \frac{E_0}{E}\implies \frac{v^2}{c^2} = 1 - \sqr{\frac{E_0}{E}} \implies \\
    \implies &v = c \sqrt{1 - \sqr{\frac{E_0}{E}}} = c \sqrt{1 - \sqr{\frac{E_0}{E_0 + E_{\text{кин}} }}} = c \sqrt{1 - \frac 1 {\sqr{ 1 + \frac{E_{\text{кин}}}{E_0} }} }\approx 0{,}986c \approx 296 \cdot 10^{6}\,\frac{\text{м}}{\text{с}}.
    \end{align*}
}


\variantsplitter


\addpersonalvariant{Сергей Пономарёв}

\tasknumber{6}%
\task{%
    Протон движется со скоростью $0{,}65\,c$, где $c$~--- скорость света в вакууме.
    Определите его импульс (в ответе приведите формулу и укажите численное значение).
}
\answer{%
    \begin{align*}
    E &= \frac{mc^2}{\sqrt{1 - \frac{v^2}{c^2}}}
            \approx \frac{1{,}673 \cdot 10^{-27}\,\text{кг} \cdot \sqr{3 \cdot 10^{8}\,\frac{\text{м}}{\text{с}}}}{\sqrt{1 - 0{,}65^2}}
            \approx 0{,}19809 \cdot 10^{-9}\,\text{Дж},
         \\
        E_{\text{кин}} &= \frac{mc^2}{\sqrt{1 - \frac{v^2}{c^2}}} - mc^2
            = mc^2 \cbr{\frac 1{\sqrt{1 - \frac{v^2}{c^2}}} - 1} \approx \\
            &\approx \cbr{1{,}673 \cdot 10^{-27}\,\text{кг} \cdot \sqr{3 \cdot 10^{8}\,\frac{\text{м}}{\text{с}}}}
            \cdot \cbr{\frac 1{\sqrt{1 - 0{,}65^2}} - 1}
            \approx 47{,}55 \cdot 10^{-12}\,\text{Дж},
         \\
        p &= \frac{mv}{\sqrt{1 - \frac{v^2}{c^2}}}
            \approx \frac{1{,}673 \cdot 10^{-27}\,\text{кг} \cdot 0{,}65 \cdot 3 \cdot 10^{8}\,\frac{\text{м}}{\text{с}}}{\sqrt{1 - 0{,}65^2}}
            \approx 0{,}4292 \cdot 10^{-18}\,\frac{\text{кг}\cdot\text{м}}{\text{с}}.
    \end{align*}
}
\solutionspace{100pt}

\tasknumber{7}%
\task{%
    Кинетическая энергия частицы космических лучей в пять раз превышает её энергию покоя.
    Определить отношение скорости частицы к скорости света.
}
\answer{%
    \begin{align*}
    E &= E_0 + E_{\text{кин}} \\
    E &= \frac{E_0}{\sqrt{1 - \frac{v^2}{c^2}}}\implies \sqrt{1 - \frac{v^2}{c^2}} = \frac{E_0}{E}\implies \frac{v^2}{c^2} = 1 - \sqr{\frac{E_0}{E}} \implies \\
    \implies \frac vc &= \sqrt{1 - \sqr{\frac{E_0}{E}}} = \sqrt{1 - \sqr{\frac{E_0}{E_0 + E_{\text{кин}} }}} \approx 0{,}986.
    \end{align*}
}
\solutionspace{80pt}

\tasknumber{8}%
\task{%
    Некоторая частица, пройдя ускоряющую разность потенциалов, приобрела импульс $3{,}5 \cdot 10^{-19}\,\frac{\text{кг}\cdot\text{м}}{\text{с}}$.
    Скорость частицы стала равной $1{,}5 \cdot 10^{8}\,\frac{\text{м}}{\text{с}}$.
    Найти массу частицы.
}
\answer{%
    $p = \frac{ mv }{\sqrt{1 - \frac{v^2}{c^2} }}\implies m = \frac pv \sqrt{1 - \frac{v^2}{c^2}}= \frac {3{,}5 \cdot 10^{-19}\,\frac{\text{кг}\cdot\text{м}}{\text{с}}}{1{,}5 \cdot 10^{8}\,\frac{\text{м}}{\text{с}}} \sqrt{1 - \sqr{\frac{1{,}5 \cdot 10^{8}\,\frac{\text{м}}{\text{с}}}{3 \cdot 10^{8}\,\frac{\text{м}}{\text{с}}}} } \approx 2{,}0 \cdot 10^{-27}\,\text{кг}.$
}
\solutionspace{80pt}

\tasknumber{9}%
\task{%
    При какой скорости движения (в км/ч) релятивистское сокращение длины движущегося тела
    составит 50\%?
}
\answer{%
    \begin{align*}
    l_0 &= \frac l{\sqrt{1 - \frac{v^2}{c^2}}}
        \implies 1 - \frac{v^2}{c^2} = \sqr{\frac l{l_0}}
        \implies \frac v c = \sqrt{1 - \sqr{\frac l{l_0}}} \implies
         \\
        \implies v &= c\sqrt{1 - \sqr{\frac l{l_0}}}
        = 3 \cdot 10^{8}\,\frac{\text{м}}{\text{с}} \cdot \sqrt{1 - \sqr{\frac {l_0 - 0{,}50l_0}{l_0}}}
        = 3 \cdot 10^{8}\,\frac{\text{м}}{\text{с}} \cdot \sqrt{1 - \sqr{1 - 0{,}50}} \approx  \\
        &\approx 0{,}866c
        \approx 260 \cdot 10^{6}\,\frac{\text{м}}{\text{с}}
        \approx 935 \cdot 10^{6}\,\frac{\text{км}}{\text{ч}}.
    \end{align*}
}
\solutionspace{80pt}

\tasknumber{10}%
\task{%
    Стержень движется в продольном направлении с постоянной скоростью относительно инерциальной системы отсчёта.
    При каком значении скорости (в долях скорости света) длина стержня в этой системе отсчёта
    будет в  4  раза меньше его собственной длины?
}
\answer{%
    $l_0 = \frac l{\sqrt{1 - \frac{v^2}{c^2}}}\implies \sqrt{1 - \frac{v^2}{c^2}} = \frac{ l }{ l_0 }\implies \frac v c = \sqrt{1 - \sqr{\frac{ l }{ l_0 }}} \approx 0{,}968.$
}
\solutionspace{80pt}

\tasknumber{11}%
\task{%
    Какую скорость должно иметь движущееся тело, чтобы его продольные размеры уменьшились в два раза?
    Скорость света $c = 3 \cdot 10^{8}\,\frac{\text{м}}{\text{с}}$.
}
\answer{%
    $l_0 = \frac l{\sqrt{1 - \frac{v^2}{c^2}}}\implies \sqrt{1 - \frac{v^2}{c^2}} = \frac{ l }{ l_0 }\implies v = c\sqrt{1 - \sqr{\frac{ l }{ l_0 }}} \approx 260 \cdot 10^{6}\,\frac{\text{м}}{\text{с}}.$
}


\variantsplitter


\addpersonalvariant{Сергей Пономарёв}

\tasknumber{12}%
\task{%
    Время жизни мюона, измеренное наблюдателем, относительно которого мюон покоился, равно $\tau_0$
    Какое расстояние пролетит мюон в системе отсчёта, относительно которой он движется со скоростью $v$,
    сравнимой со скоростью света в вакууме $c$?
}
\answer{%
    $\ell = v\tau = v \frac{\tau_0}{\sqrt{1 - \frac{v^2}{c^2}}}$
}
\solutionspace{80pt}

\tasknumber{13}%
\task{%
    Если $c$ — скорость света в вакууме, то с какой скоростью должна двигаться нестабильная частица относительно наблюдателя,
    чтобы её время жизни было в пять раз больше, чем у такой же, но покоящейся относительно наблюдателя частицы?
}
\answer{%
    $\tau = \frac{\tau_0}{\sqrt{1 - \frac{v^2}{c^2}}}\implies \sqrt{1 - \frac{v^2}{c^2}} = \frac{\tau_0}{\tau}\implies v = c\sqrt{1 - \sqr{\frac{\tau_0}{\tau}} } \approx 294 \cdot 10^{6}\,\frac{\text{м}}{\text{с}}.$
}
\solutionspace{80pt}

\tasknumber{14}%
\task{%
    Время жизни нестабильной частицы, входящего в состав космических лучей, измеренное земным наблюдателем,
    относительно которого частица двигалась со скоростью, составляющей 85\% скорости света в вакууме, оказалось равным $7{,}1\,\text{мкс}$.
    Каково время жизни частицы, покоящейся относительно наблюдателя?
}
\answer{%
    $t = \frac{t_0}{\sqrt{1 - \frac{v^2}{c^2}}}\implies t_0 = t\sqrt{1 - \frac{v^2}{c^2}} \approx 3{,}7 \cdot 10^{-6}\,\text{с}.$
}
\solutionspace{80pt}

\tasknumber{15}%
\task{%
    Частица увеличила в ускорителе свою скорость с $0{,}01c$ до $0{,}50c$.
    Во сколько раз выросла её кинетическая энергия?
}
\answer{%
    \begin{align*}
    E_{\text{кин.}} &= E - E_0 = \frac{mc^2}{\sqrt{1 - \frac{v^2}{c^2}}} - mc^2 = mc^2\cbr{ \frac1{\sqrt{1 - \frac{v^2}{c^2} }} - 1}.
    \\
    \frac{E_{\text{кин.
    2}}}{E_{\text{кин.
    1}}} &= \frac{\frac1{\sqrt{1 - \frac{v_2^2}{c^2} }} - 1}{\frac1{\sqrt{1 - \frac{v_1^2}{c^2} }} - 1}\approx 3093{,}78
    \end{align*}
}
\solutionspace{120pt}

\tasknumber{16}%
\task{%
    Для частицы, движущейся с релятивистской скоростью,
    выразите $p$ и $E_\text{кин}$ через $c$, $v$ и $E_0$,
    где $E_\text{кин}$~--- кинетическая энергия частицы,
    а $E_0$, $p$ и $v$~--- её энергия покоя, импульс и скорость.
}
\answer{%
    \begin{align*}
    E_\text{кин}, E_0:\quad&E = E_\text{кин} + E_0 = \frac{E_0}{\sqrt{1 - \frac{v^2}{c^2}}} \implies \sqrt{1 - \frac{v^2}{c^2}} = \frac{E_0}{{E_0} + {E_\text{кин}}} \implies v = c\sqrt{1 - \sqr{\frac{E_0}{{E_0} + {E_\text{кин}}}}} \\
    &p = \frac{mv}{\sqrt{1 - \frac{v^2}{c^2}}} = \frac{E_0}{c^2} \cdot \sqrt{1 - \sqr{\frac{E_0}{{E_0} + {E_\text{кин}}}}} \cdot \frac{{E_\text{кин}} + {E_0}}{E_0} = \frac{E_0}{c^2} \cdot \sqrt{\sqr{\frac{{E_\text{кин}} + {E_0}}{E_0}} - 1}.
    \\
    E_\text{кин}, p:\quad&E_\text{кин} = E - E_0 = mc^2\cbr{\frac 1{\sqrt{1 - \frac{v^2}{c^2}}} - 1}, p = \frac{mv}{\sqrt{1 - \frac{v^2}{c^2}}} \implies \frac{E_\text{кин}}{p} = \frac{\frac 1{\sqrt{1 - \frac{v^2}{c^2}}} - 1}{\sqrt{1 - \frac{v^2}{c^2}}} \implies v = \ldots \\
    &E_0 = E - E_\text{кин} = \frac{E_0}{\sqrt{1 - \frac{v^2}{c^2}}} - E_\text{кин} \implies E_0 = \frac{E_\text{кин}}{\frac 1{\sqrt{1 - \frac{v^2}{c^2}}} - 1} = \ldots \\
    E_\text{кин}, v:\quad&E_\text{кин} = E - E_0 = mc^2\cbr{\frac 1{\sqrt{1 - \frac{v^2}{c^2}}} - 1} \implies m = \frac{E_\text{кин}}{c^2\cbr{\frac 1{\sqrt{1 - \frac{v^2}{c^2}}} - 1}} \\
    &E_0 = mc^2 = \frac{E_\text{кин}}{\frac 1{\sqrt{1 - \frac{v^2}{c^2}}} - 1} \\
    &p = \frac{mv}{\sqrt{1 - \frac{v^2}{c^2}}} = \frac{E_\text{кин}}{c^2\cbr{\frac 1{\sqrt{1 - \frac{v^2}{c^2}}} - 1}} \cdot \frac{v}{\sqrt{1 - \frac{v^2}{c^2}}} = \frac{{E_\text{кин}} v}{c^2\cbr{1 - {\sqrt{1 - \frac{v^2}{c^2}}}}} \\
    E_0, p:\quad&E_0 = mc^2, \quad p = \frac{mv}{\sqrt{1 - \frac{v^2}{c^2}}} \implies \frac{E_0}{p} = \frac{c^2}v{\sqrt{1 - \frac{v^2}{c^2}}} = c\sqrt{\frac{c^2}{v^2} - 1} \\
    &\sqr{\frac{E_0}{pc}} = \frac{c^2}{v^2} - 1 \implies \frac{v^2}{c^2} = \frac 1{1 + \frac{E_0^2}{p^2c^2}} \implies v = \frac c{\sqrt{1 + \frac{E_0^2}{p^2c^2}}} \\
    &{E_\text{кин}} = E - E_0 = \sqrt{E_0^2 + p^2c^2} - E_0 \\
    E_0, v:\quad&E_0 = mc^2 \implies m = \frac{E_0}{c^2} \qquad p = \frac{mv}{\sqrt{1 - \frac{v^2}{c^2}}} = \frac{E_0}{c^2} \cdot \frac{v}{\sqrt{1 - \frac{v^2}{c^2}}} \\
    &E_\text{кин}= mc^2\cbr{\frac 1{\sqrt{1 - \frac{v^2}{c^2}}} - 1} = \frac{E_0}{c^2}\cbr{\frac 1{\sqrt{1 - \frac{v^2}{c^2}}} - 1} \\
    p, v:\quad&p = \frac{mv}{\sqrt{1 - \frac{v^2}{c^2}}} \implies m = \frac p v {\sqrt{1 - \frac{v^2}{c^2}}} \implies E_0 = mc^2 =\frac {pc^2} v {\sqrt{1 - \frac{v^2}{c^2}}} \\
    &E_\text{кин} = mc^2\cbr{\frac 1{\sqrt{1 - \frac{v^2}{c^2}}} - 1} = \frac p v {\sqrt{1 - \frac{v^2}{c^2}}}\cbr{\frac 1{\sqrt{1 - \frac{v^2}{c^2}}} - 1} = \frac p v \cbr{1 - {\sqrt{1 - \frac{v^2}{c^2}}}}
    \end{align*}
}

\variantsplitter

\addpersonalvariant{Егор Свистушкин}

\tasknumber{1}%
\task{%
    Запишите
    \begin{itemize}
        \item постулаты специальной теории относительности,
        \item пример релятивистского эффекта, обнаружимый при скоростях гораздо меньше скорости света.
    \end{itemize}
}
\solutionspace{120pt}

\tasknumber{2}%
\task{%
    Запишите формулу для ...
    \begin{itemize}
        \item релятивистского сжатия,
        \item классического импульса,
        \item релятивистского импульса тела,
        \item релятивистской кинетической энергии,
        \item связь между релятивистским импульсом и релятивистской энергией.
    \end{itemize}
    Обязательно подпишите все физические величины.
}
\solutionspace{150pt}

\tasknumber{3}%
\task{%
    Позитрон движется со скоростью $0{,}9\,c$, где $c$~--- скорость света в вакууме.
    Каково при этом отношение полной энергии частицы $E$ к его энергии покоя $E_0$?
}
\answer{%
    \begin{align*}
    E &= \frac{E_0}{\sqrt{1 - \frac{v^2}{c^2}}}
            \implies \frac E{E_0}
                = \frac 1{\sqrt{1 - \frac{v^2}{c^2}}}
                = \frac 1{\sqrt{1 - \sqr{0{,}9}}}
                \approx 2{,}294,
         \\
        {E_{\text{кин}}} &= E - E_0
            \implies \frac{E_{\text{кин}}}{E_0}
                = \frac E{E_0} - 1
                = \frac 1{\sqrt{1 - \frac{v^2}{c^2}}} - 1
                = \frac 1{\sqrt{1 - \sqr{0{,}9}}} - 1
                \approx 1{,}294.
    \end{align*}
}
\solutionspace{80pt}

\tasknumber{4}%
\task{%
    Полная энергия релятивистской частицы в три раза больше её энергии покоя.
    Найти скорость этой частицы: в долях $c$ и численное значение.
    Скорость света в вакууме $c = 3 \cdot 10^{8}\,\frac{\text{м}}{\text{с}}$.
}
\answer{%
    \begin{align*}
    E &= \frac{E_0}{\sqrt{1 - \frac{v^2}{c^2}}}\implies \sqrt{1 - \frac{v^2}{c^2}} = \frac{E_0}{E}\implies \frac{v^2}{c^2} = 1 - \sqr{\frac{E_0}{E}}\implies v = c \sqrt{1 - \sqr{\frac{E_0}{E}}} \approx 0{,}943c \approx 283 \cdot 10^{6}\,\frac{\text{м}}{\text{с}}.
    \end{align*}
}
\solutionspace{80pt}

\tasknumber{5}%
\task{%
    Кинетическая энергия релятивистской частицы в три раза больше её энергии покоя.
    Найти скорость этой частицы.
    Скорость света в вакууме $c = 3 \cdot 10^{8}\,\frac{\text{м}}{\text{с}}$.
}
\answer{%
    \begin{align*}
    E &= E_0 + E_{\text{кин}} \\
    E &= \frac{E_0}{\sqrt{1 - \frac{v^2}{c^2}}}\implies \sqrt{1 - \frac{v^2}{c^2}} = \frac{E_0}{E}\implies \frac{v^2}{c^2} = 1 - \sqr{\frac{E_0}{E}} \implies \\
    \implies &v = c \sqrt{1 - \sqr{\frac{E_0}{E}}} = c \sqrt{1 - \sqr{\frac{E_0}{E_0 + E_{\text{кин}} }}} = c \sqrt{1 - \frac 1 {\sqr{ 1 + \frac{E_{\text{кин}}}{E_0} }} }\approx 0{,}968c \approx 290 \cdot 10^{6}\,\frac{\text{м}}{\text{с}}.
    \end{align*}
}


\variantsplitter


\addpersonalvariant{Егор Свистушкин}

\tasknumber{6}%
\task{%
    Электрон движется со скоростью $0{,}85\,c$, где $c$~--- скорость света в вакууме.
    Определите его импульс (в ответе приведите формулу и укажите численное значение).
}
\answer{%
    \begin{align*}
    E &= \frac{mc^2}{\sqrt{1 - \frac{v^2}{c^2}}}
            \approx \frac{9{,}1 \cdot 10^{-31}\,\text{кг} \cdot \sqr{3 \cdot 10^{8}\,\frac{\text{м}}{\text{с}}}}{\sqrt{1 - 0{,}85^2}}
            \approx 0{,}1555 \cdot 10^{-12}\,\text{Дж},
         \\
        E_{\text{кин}} &= \frac{mc^2}{\sqrt{1 - \frac{v^2}{c^2}}} - mc^2
            = mc^2 \cbr{\frac 1{\sqrt{1 - \frac{v^2}{c^2}}} - 1} \approx \\
            &\approx \cbr{9{,}1 \cdot 10^{-31}\,\text{кг} \cdot \sqr{3 \cdot 10^{8}\,\frac{\text{м}}{\text{с}}}}
            \cdot \cbr{\frac 1{\sqrt{1 - 0{,}85^2}} - 1}
            \approx 73{,}6 \cdot 10^{-15}\,\text{Дж},
         \\
        p &= \frac{mv}{\sqrt{1 - \frac{v^2}{c^2}}}
            \approx \frac{9{,}1 \cdot 10^{-31}\,\text{кг} \cdot 0{,}85 \cdot 3 \cdot 10^{8}\,\frac{\text{м}}{\text{с}}}{\sqrt{1 - 0{,}85^2}}
            \approx 0{,}441 \cdot 10^{-21}\,\frac{\text{кг}\cdot\text{м}}{\text{с}}.
    \end{align*}
}
\solutionspace{100pt}

\tasknumber{7}%
\task{%
    Кинетическая энергия частицы космических лучей в три раза превышает её энергию покоя.
    Определить отношение скорости частицы к скорости света.
}
\answer{%
    \begin{align*}
    E &= E_0 + E_{\text{кин}} \\
    E &= \frac{E_0}{\sqrt{1 - \frac{v^2}{c^2}}}\implies \sqrt{1 - \frac{v^2}{c^2}} = \frac{E_0}{E}\implies \frac{v^2}{c^2} = 1 - \sqr{\frac{E_0}{E}} \implies \\
    \implies \frac vc &= \sqrt{1 - \sqr{\frac{E_0}{E}}} = \sqrt{1 - \sqr{\frac{E_0}{E_0 + E_{\text{кин}} }}} \approx 0{,}968.
    \end{align*}
}
\solutionspace{80pt}

\tasknumber{8}%
\task{%
    Некоторая частица, пройдя ускоряющую разность потенциалов, приобрела импульс $3{,}5 \cdot 10^{-19}\,\frac{\text{кг}\cdot\text{м}}{\text{с}}$.
    Скорость частицы стала равной $2{,}4 \cdot 10^{8}\,\frac{\text{м}}{\text{с}}$.
    Найти массу частицы.
}
\answer{%
    $p = \frac{ mv }{\sqrt{1 - \frac{v^2}{c^2} }}\implies m = \frac pv \sqrt{1 - \frac{v^2}{c^2}}= \frac {3{,}5 \cdot 10^{-19}\,\frac{\text{кг}\cdot\text{м}}{\text{с}}}{2{,}4 \cdot 10^{8}\,\frac{\text{м}}{\text{с}}} \sqrt{1 - \sqr{\frac{2{,}4 \cdot 10^{8}\,\frac{\text{м}}{\text{с}}}{3 \cdot 10^{8}\,\frac{\text{м}}{\text{с}}}} } \approx 0{,}875 \cdot 10^{-27}\,\text{кг}.$
}
\solutionspace{80pt}

\tasknumber{9}%
\task{%
    При какой скорости движения (в долях скорости света) релятивистское сокращение длины движущегося тела
    составит 30\%?
}
\answer{%
    \begin{align*}
    l_0 &= \frac l{\sqrt{1 - \frac{v^2}{c^2}}}
        \implies 1 - \frac{v^2}{c^2} = \sqr{\frac l{l_0}}
        \implies \frac v c = \sqrt{1 - \sqr{\frac l{l_0}}} \implies
         \\
        \implies v &= c\sqrt{1 - \sqr{\frac l{l_0}}}
        = 3 \cdot 10^{8}\,\frac{\text{м}}{\text{с}} \cdot \sqrt{1 - \sqr{\frac {l_0 - 0{,}30l_0}{l_0}}}
        = 3 \cdot 10^{8}\,\frac{\text{м}}{\text{с}} \cdot \sqrt{1 - \sqr{1 - 0{,}30}} \approx  \\
        &\approx 0{,}714c
        \approx 214 \cdot 10^{6}\,\frac{\text{м}}{\text{с}}
        \approx 771 \cdot 10^{6}\,\frac{\text{км}}{\text{ч}}.
    \end{align*}
}
\solutionspace{80pt}

\tasknumber{10}%
\task{%
    Стержень движется в продольном направлении с постоянной скоростью относительно инерциальной системы отсчёта.
    При каком значении скорости (в долях скорости света) длина стержня в этой системе отсчёта
    будет в  1{,}25  раза меньше его собственной длины?
}
\answer{%
    $l_0 = \frac l{\sqrt{1 - \frac{v^2}{c^2}}}\implies \sqrt{1 - \frac{v^2}{c^2}} = \frac{ l }{ l_0 }\implies \frac v c = \sqrt{1 - \sqr{\frac{ l }{ l_0 }}} \approx 0{,}600.$
}
\solutionspace{80pt}

\tasknumber{11}%
\task{%
    Какую скорость должно иметь движущееся тело, чтобы его продольные размеры уменьшились в пять раз?
    Скорость света $c = 3 \cdot 10^{8}\,\frac{\text{м}}{\text{с}}$.
}
\answer{%
    $l_0 = \frac l{\sqrt{1 - \frac{v^2}{c^2}}}\implies \sqrt{1 - \frac{v^2}{c^2}} = \frac{ l }{ l_0 }\implies v = c\sqrt{1 - \sqr{\frac{ l }{ l_0 }}} \approx 294 \cdot 10^{6}\,\frac{\text{м}}{\text{с}}.$
}


\variantsplitter


\addpersonalvariant{Егор Свистушкин}

\tasknumber{12}%
\task{%
    Время жизни мюона, измеренное наблюдателем, относительно которого мюон покоился, равно $\tau_0$
    Какое расстояние пролетит мюон в системе отсчёта, относительно которой он движется со скоростью $v$,
    сравнимой со скоростью света в вакууме $c$?
}
\answer{%
    $\ell = v\tau = v \frac{\tau_0}{\sqrt{1 - \frac{v^2}{c^2}}}$
}
\solutionspace{80pt}

\tasknumber{13}%
\task{%
    Если $c$ — скорость света в вакууме, то с какой скоростью должна двигаться нестабильная частица относительно наблюдателя,
    чтобы её время жизни было в семь раз больше, чем у такой же, но покоящейся относительно наблюдателя частицы?
}
\answer{%
    $\tau = \frac{\tau_0}{\sqrt{1 - \frac{v^2}{c^2}}}\implies \sqrt{1 - \frac{v^2}{c^2}} = \frac{\tau_0}{\tau}\implies v = c\sqrt{1 - \sqr{\frac{\tau_0}{\tau}} } \approx 297 \cdot 10^{6}\,\frac{\text{м}}{\text{с}}.$
}
\solutionspace{80pt}

\tasknumber{14}%
\task{%
    Время жизни нестабильной частицы, входящего в состав космических лучей, измеренное земным наблюдателем,
    относительно которого частица двигалась со скоростью, составляющей 65\% скорости света в вакууме, оказалось равным $7{,}1\,\text{мкс}$.
    Каково время жизни частицы, покоящейся относительно наблюдателя?
}
\answer{%
    $t = \frac{t_0}{\sqrt{1 - \frac{v^2}{c^2}}}\implies t_0 = t\sqrt{1 - \frac{v^2}{c^2}} \approx 5{,}4 \cdot 10^{-6}\,\text{с}.$
}
\solutionspace{80pt}

\tasknumber{15}%
\task{%
    Частица увеличила в ускорителе свою скорость с $0{,}05c$ до $0{,}80c$.
    Во сколько раз выросла её кинетическая энергия?
}
\answer{%
    \begin{align*}
    E_{\text{кин.}} &= E - E_0 = \frac{mc^2}{\sqrt{1 - \frac{v^2}{c^2}}} - mc^2 = mc^2\cbr{ \frac1{\sqrt{1 - \frac{v^2}{c^2} }} - 1}.
    \\
    \frac{E_{\text{кин.
    2}}}{E_{\text{кин.
    1}}} &= \frac{\frac1{\sqrt{1 - \frac{v_2^2}{c^2} }} - 1}{\frac1{\sqrt{1 - \frac{v_1^2}{c^2} }} - 1}\approx 532{,}33
    \end{align*}
}
\solutionspace{120pt}

\tasknumber{16}%
\task{%
    Для частицы, движущейся с релятивистской скоростью,
    выразите $v$ и $p$ через $c$, $E_0$ и $E_\text{кин}$,
    где $E_\text{кин}$~--- кинетическая энергия частицы,
    а $E_0$, $p$ и $v$~--- её энергия покоя, импульс и скорость.
}
\answer{%
    \begin{align*}
    E_\text{кин}, E_0:\quad&E = E_\text{кин} + E_0 = \frac{E_0}{\sqrt{1 - \frac{v^2}{c^2}}} \implies \sqrt{1 - \frac{v^2}{c^2}} = \frac{E_0}{{E_0} + {E_\text{кин}}} \implies v = c\sqrt{1 - \sqr{\frac{E_0}{{E_0} + {E_\text{кин}}}}} \\
    &p = \frac{mv}{\sqrt{1 - \frac{v^2}{c^2}}} = \frac{E_0}{c^2} \cdot \sqrt{1 - \sqr{\frac{E_0}{{E_0} + {E_\text{кин}}}}} \cdot \frac{{E_\text{кин}} + {E_0}}{E_0} = \frac{E_0}{c^2} \cdot \sqrt{\sqr{\frac{{E_\text{кин}} + {E_0}}{E_0}} - 1}.
    \\
    E_\text{кин}, p:\quad&E_\text{кин} = E - E_0 = mc^2\cbr{\frac 1{\sqrt{1 - \frac{v^2}{c^2}}} - 1}, p = \frac{mv}{\sqrt{1 - \frac{v^2}{c^2}}} \implies \frac{E_\text{кин}}{p} = \frac{\frac 1{\sqrt{1 - \frac{v^2}{c^2}}} - 1}{\sqrt{1 - \frac{v^2}{c^2}}} \implies v = \ldots \\
    &E_0 = E - E_\text{кин} = \frac{E_0}{\sqrt{1 - \frac{v^2}{c^2}}} - E_\text{кин} \implies E_0 = \frac{E_\text{кин}}{\frac 1{\sqrt{1 - \frac{v^2}{c^2}}} - 1} = \ldots \\
    E_\text{кин}, v:\quad&E_\text{кин} = E - E_0 = mc^2\cbr{\frac 1{\sqrt{1 - \frac{v^2}{c^2}}} - 1} \implies m = \frac{E_\text{кин}}{c^2\cbr{\frac 1{\sqrt{1 - \frac{v^2}{c^2}}} - 1}} \\
    &E_0 = mc^2 = \frac{E_\text{кин}}{\frac 1{\sqrt{1 - \frac{v^2}{c^2}}} - 1} \\
    &p = \frac{mv}{\sqrt{1 - \frac{v^2}{c^2}}} = \frac{E_\text{кин}}{c^2\cbr{\frac 1{\sqrt{1 - \frac{v^2}{c^2}}} - 1}} \cdot \frac{v}{\sqrt{1 - \frac{v^2}{c^2}}} = \frac{{E_\text{кин}} v}{c^2\cbr{1 - {\sqrt{1 - \frac{v^2}{c^2}}}}} \\
    E_0, p:\quad&E_0 = mc^2, \quad p = \frac{mv}{\sqrt{1 - \frac{v^2}{c^2}}} \implies \frac{E_0}{p} = \frac{c^2}v{\sqrt{1 - \frac{v^2}{c^2}}} = c\sqrt{\frac{c^2}{v^2} - 1} \\
    &\sqr{\frac{E_0}{pc}} = \frac{c^2}{v^2} - 1 \implies \frac{v^2}{c^2} = \frac 1{1 + \frac{E_0^2}{p^2c^2}} \implies v = \frac c{\sqrt{1 + \frac{E_0^2}{p^2c^2}}} \\
    &{E_\text{кин}} = E - E_0 = \sqrt{E_0^2 + p^2c^2} - E_0 \\
    E_0, v:\quad&E_0 = mc^2 \implies m = \frac{E_0}{c^2} \qquad p = \frac{mv}{\sqrt{1 - \frac{v^2}{c^2}}} = \frac{E_0}{c^2} \cdot \frac{v}{\sqrt{1 - \frac{v^2}{c^2}}} \\
    &E_\text{кин}= mc^2\cbr{\frac 1{\sqrt{1 - \frac{v^2}{c^2}}} - 1} = \frac{E_0}{c^2}\cbr{\frac 1{\sqrt{1 - \frac{v^2}{c^2}}} - 1} \\
    p, v:\quad&p = \frac{mv}{\sqrt{1 - \frac{v^2}{c^2}}} \implies m = \frac p v {\sqrt{1 - \frac{v^2}{c^2}}} \implies E_0 = mc^2 =\frac {pc^2} v {\sqrt{1 - \frac{v^2}{c^2}}} \\
    &E_\text{кин} = mc^2\cbr{\frac 1{\sqrt{1 - \frac{v^2}{c^2}}} - 1} = \frac p v {\sqrt{1 - \frac{v^2}{c^2}}}\cbr{\frac 1{\sqrt{1 - \frac{v^2}{c^2}}} - 1} = \frac p v \cbr{1 - {\sqrt{1 - \frac{v^2}{c^2}}}}
    \end{align*}
}

\variantsplitter

\addpersonalvariant{Дмитрий Соколов}

\tasknumber{1}%
\task{%
    Запишите
    \begin{itemize}
        \item постулаты специальной теории относительности,
        \item пример релятивистского эффекта, обнаружимый при скоростях гораздо меньше скорости света.
    \end{itemize}
}
\solutionspace{120pt}

\tasknumber{2}%
\task{%
    Запишите формулу для ...
    \begin{itemize}
        \item релятивистского сжатия,
        \item классического импульса,
        \item релятивистской энергии тела,
        \item релятивистской кинетической энергии,
        \item связь между релятивистским импульсом и релятивистской энергией.
    \end{itemize}
    Обязательно подпишите все физические величины.
}
\solutionspace{150pt}

\tasknumber{3}%
\task{%
    Электрон движется со скоростью $0{,}8\,c$, где $c$~--- скорость света в вакууме.
    Каково при этом отношение полной энергии частицы $E$ к его энергии покоя $E_0$?
}
\answer{%
    \begin{align*}
    E &= \frac{E_0}{\sqrt{1 - \frac{v^2}{c^2}}}
            \implies \frac E{E_0}
                = \frac 1{\sqrt{1 - \frac{v^2}{c^2}}}
                = \frac 1{\sqrt{1 - \sqr{0{,}8}}}
                \approx 1{,}667,
         \\
        {E_{\text{кин}}} &= E - E_0
            \implies \frac{E_{\text{кин}}}{E_0}
                = \frac E{E_0} - 1
                = \frac 1{\sqrt{1 - \frac{v^2}{c^2}}} - 1
                = \frac 1{\sqrt{1 - \sqr{0{,}8}}} - 1
                \approx 0{,}667.
    \end{align*}
}
\solutionspace{80pt}

\tasknumber{4}%
\task{%
    Полная энергия релятивистской частицы в пять раз больше её энергии покоя.
    Найти скорость этой частицы: в долях $c$ и численное значение.
    Скорость света в вакууме $c = 3 \cdot 10^{8}\,\frac{\text{м}}{\text{с}}$.
}
\answer{%
    \begin{align*}
    E &= \frac{E_0}{\sqrt{1 - \frac{v^2}{c^2}}}\implies \sqrt{1 - \frac{v^2}{c^2}} = \frac{E_0}{E}\implies \frac{v^2}{c^2} = 1 - \sqr{\frac{E_0}{E}}\implies v = c \sqrt{1 - \sqr{\frac{E_0}{E}}} \approx 0{,}980c \approx 294 \cdot 10^{6}\,\frac{\text{м}}{\text{с}}.
    \end{align*}
}
\solutionspace{80pt}

\tasknumber{5}%
\task{%
    Кинетическая энергия релятивистской частицы в пять раз больше её энергии покоя.
    Найти скорость этой частицы.
    Скорость света в вакууме $c = 3 \cdot 10^{8}\,\frac{\text{м}}{\text{с}}$.
}
\answer{%
    \begin{align*}
    E &= E_0 + E_{\text{кин}} \\
    E &= \frac{E_0}{\sqrt{1 - \frac{v^2}{c^2}}}\implies \sqrt{1 - \frac{v^2}{c^2}} = \frac{E_0}{E}\implies \frac{v^2}{c^2} = 1 - \sqr{\frac{E_0}{E}} \implies \\
    \implies &v = c \sqrt{1 - \sqr{\frac{E_0}{E}}} = c \sqrt{1 - \sqr{\frac{E_0}{E_0 + E_{\text{кин}} }}} = c \sqrt{1 - \frac 1 {\sqr{ 1 + \frac{E_{\text{кин}}}{E_0} }} }\approx 0{,}986c \approx 296 \cdot 10^{6}\,\frac{\text{м}}{\text{с}}.
    \end{align*}
}


\variantsplitter


\addpersonalvariant{Дмитрий Соколов}

\tasknumber{6}%
\task{%
    Протон движется со скоростью $0{,}75\,c$, где $c$~--- скорость света в вакууме.
    Определите его кинетическую энергию (в ответе приведите формулу и укажите численное значение).
}
\answer{%
    \begin{align*}
    E &= \frac{mc^2}{\sqrt{1 - \frac{v^2}{c^2}}}
            \approx \frac{1{,}673 \cdot 10^{-27}\,\text{кг} \cdot \sqr{3 \cdot 10^{8}\,\frac{\text{м}}{\text{с}}}}{\sqrt{1 - 0{,}75^2}}
            \approx 0{,}2276 \cdot 10^{-9}\,\text{Дж},
         \\
        E_{\text{кин}} &= \frac{mc^2}{\sqrt{1 - \frac{v^2}{c^2}}} - mc^2
            = mc^2 \cbr{\frac 1{\sqrt{1 - \frac{v^2}{c^2}}} - 1} \approx \\
            &\approx \cbr{1{,}673 \cdot 10^{-27}\,\text{кг} \cdot \sqr{3 \cdot 10^{8}\,\frac{\text{м}}{\text{с}}}}
            \cdot \cbr{\frac 1{\sqrt{1 - 0{,}75^2}} - 1}
            \approx 77{,}05 \cdot 10^{-12}\,\text{Дж},
         \\
        p &= \frac{mv}{\sqrt{1 - \frac{v^2}{c^2}}}
            \approx \frac{1{,}673 \cdot 10^{-27}\,\text{кг} \cdot 0{,}75 \cdot 3 \cdot 10^{8}\,\frac{\text{м}}{\text{с}}}{\sqrt{1 - 0{,}75^2}}
            \approx 0{,}5690 \cdot 10^{-18}\,\frac{\text{кг}\cdot\text{м}}{\text{с}}.
    \end{align*}
}
\solutionspace{100pt}

\tasknumber{7}%
\task{%
    Кинетическая энергия частицы космических лучей в пять раз превышает её энергию покоя.
    Определить отношение скорости частицы к скорости света.
}
\answer{%
    \begin{align*}
    E &= E_0 + E_{\text{кин}} \\
    E &= \frac{E_0}{\sqrt{1 - \frac{v^2}{c^2}}}\implies \sqrt{1 - \frac{v^2}{c^2}} = \frac{E_0}{E}\implies \frac{v^2}{c^2} = 1 - \sqr{\frac{E_0}{E}} \implies \\
    \implies \frac vc &= \sqrt{1 - \sqr{\frac{E_0}{E}}} = \sqrt{1 - \sqr{\frac{E_0}{E_0 + E_{\text{кин}} }}} \approx 0{,}986.
    \end{align*}
}
\solutionspace{80pt}

\tasknumber{8}%
\task{%
    Некоторая частица, пройдя ускоряющую разность потенциалов, приобрела импульс $4{,}2 \cdot 10^{-19}\,\frac{\text{кг}\cdot\text{м}}{\text{с}}$.
    Скорость частицы стала равной $1{,}8 \cdot 10^{8}\,\frac{\text{м}}{\text{с}}$.
    Найти массу частицы.
}
\answer{%
    $p = \frac{ mv }{\sqrt{1 - \frac{v^2}{c^2} }}\implies m = \frac pv \sqrt{1 - \frac{v^2}{c^2}}= \frac {4{,}2 \cdot 10^{-19}\,\frac{\text{кг}\cdot\text{м}}{\text{с}}}{1{,}8 \cdot 10^{8}\,\frac{\text{м}}{\text{с}}} \sqrt{1 - \sqr{\frac{1{,}8 \cdot 10^{8}\,\frac{\text{м}}{\text{с}}}{3 \cdot 10^{8}\,\frac{\text{м}}{\text{с}}}} } \approx 1{,}87 \cdot 10^{-27}\,\text{кг}.$
}
\solutionspace{80pt}

\tasknumber{9}%
\task{%
    При какой скорости движения (в м/с) релятивистское сокращение длины движущегося тела
    составит 10\%?
}
\answer{%
    \begin{align*}
    l_0 &= \frac l{\sqrt{1 - \frac{v^2}{c^2}}}
        \implies 1 - \frac{v^2}{c^2} = \sqr{\frac l{l_0}}
        \implies \frac v c = \sqrt{1 - \sqr{\frac l{l_0}}} \implies
         \\
        \implies v &= c\sqrt{1 - \sqr{\frac l{l_0}}}
        = 3 \cdot 10^{8}\,\frac{\text{м}}{\text{с}} \cdot \sqrt{1 - \sqr{\frac {l_0 - 0{,}10l_0}{l_0}}}
        = 3 \cdot 10^{8}\,\frac{\text{м}}{\text{с}} \cdot \sqrt{1 - \sqr{1 - 0{,}10}} \approx  \\
        &\approx 0{,}436c
        \approx 130{,}8 \cdot 10^{6}\,\frac{\text{м}}{\text{с}}
        \approx 471 \cdot 10^{6}\,\frac{\text{км}}{\text{ч}}.
    \end{align*}
}
\solutionspace{80pt}

\tasknumber{10}%
\task{%
    Стержень движется в продольном направлении с постоянной скоростью относительно инерциальной системы отсчёта.
    При каком значении скорости (в долях скорости света) длина стержня в этой системе отсчёта
    будет в  2{,}5  раза меньше его собственной длины?
}
\answer{%
    $l_0 = \frac l{\sqrt{1 - \frac{v^2}{c^2}}}\implies \sqrt{1 - \frac{v^2}{c^2}} = \frac{ l }{ l_0 }\implies \frac v c = \sqrt{1 - \sqr{\frac{ l }{ l_0 }}} \approx 0{,}917.$
}
\solutionspace{80pt}

\tasknumber{11}%
\task{%
    Какую скорость должно иметь движущееся тело, чтобы его продольные размеры уменьшились в три раза?
    Скорость света $c = 3 \cdot 10^{8}\,\frac{\text{м}}{\text{с}}$.
}
\answer{%
    $l_0 = \frac l{\sqrt{1 - \frac{v^2}{c^2}}}\implies \sqrt{1 - \frac{v^2}{c^2}} = \frac{ l }{ l_0 }\implies v = c\sqrt{1 - \sqr{\frac{ l }{ l_0 }}} \approx 283 \cdot 10^{6}\,\frac{\text{м}}{\text{с}}.$
}


\variantsplitter


\addpersonalvariant{Дмитрий Соколов}

\tasknumber{12}%
\task{%
    Время жизни мюона, измеренное наблюдателем, относительно которого мюон покоился, равно $\tau_0$
    Какое расстояние пролетит мюон в системе отсчёта, относительно которой он движется со скоростью $v$,
    сравнимой со скоростью света в вакууме $c$?
}
\answer{%
    $\ell = v\tau = v \frac{\tau_0}{\sqrt{1 - \frac{v^2}{c^2}}}$
}
\solutionspace{80pt}

\tasknumber{13}%
\task{%
    Если $c$ — скорость света в вакууме, то с какой скоростью должна двигаться нестабильная частица относительно наблюдателя,
    чтобы её время жизни было в пять раз больше, чем у такой же, но покоящейся относительно наблюдателя частицы?
}
\answer{%
    $\tau = \frac{\tau_0}{\sqrt{1 - \frac{v^2}{c^2}}}\implies \sqrt{1 - \frac{v^2}{c^2}} = \frac{\tau_0}{\tau}\implies v = c\sqrt{1 - \sqr{\frac{\tau_0}{\tau}} } \approx 294 \cdot 10^{6}\,\frac{\text{м}}{\text{с}}.$
}
\solutionspace{80pt}

\tasknumber{14}%
\task{%
    Время жизни нестабильной частицы, входящего в состав космических лучей, измеренное земным наблюдателем,
    относительно которого частица двигалась со скоростью, составляющей 85\% скорости света в вакууме, оказалось равным $3{,}7\,\text{мкс}$.
    Каково время жизни частицы, покоящейся относительно наблюдателя?
}
\answer{%
    $t = \frac{t_0}{\sqrt{1 - \frac{v^2}{c^2}}}\implies t_0 = t\sqrt{1 - \frac{v^2}{c^2}} \approx 1{,}95 \cdot 10^{-6}\,\text{с}.$
}
\solutionspace{80pt}

\tasknumber{15}%
\task{%
    Частица увеличила в ускорителе свою скорость с $0{,}05c$ до $0{,}70c$.
    Во сколько раз выросла её кинетическая энергия?
}
\answer{%
    \begin{align*}
    E_{\text{кин.}} &= E - E_0 = \frac{mc^2}{\sqrt{1 - \frac{v^2}{c^2}}} - mc^2 = mc^2\cbr{ \frac1{\sqrt{1 - \frac{v^2}{c^2} }} - 1}.
    \\
    \frac{E_{\text{кин.
    2}}}{E_{\text{кин.
    1}}} &= \frac{\frac1{\sqrt{1 - \frac{v_2^2}{c^2} }} - 1}{\frac1{\sqrt{1 - \frac{v_1^2}{c^2} }} - 1}\approx 319{,}62
    \end{align*}
}
\solutionspace{120pt}

\tasknumber{16}%
\task{%
    Для частицы, движущейся с релятивистской скоростью,
    выразите $E_0$ и $E_\text{кин}$ через $c$, $v$ и $p$,
    где $E_\text{кин}$~--- кинетическая энергия частицы,
    а $E_0$, $p$ и $v$~--- её энергия покоя, импульс и скорость.
}
\answer{%
    \begin{align*}
    E_\text{кин}, E_0:\quad&E = E_\text{кин} + E_0 = \frac{E_0}{\sqrt{1 - \frac{v^2}{c^2}}} \implies \sqrt{1 - \frac{v^2}{c^2}} = \frac{E_0}{{E_0} + {E_\text{кин}}} \implies v = c\sqrt{1 - \sqr{\frac{E_0}{{E_0} + {E_\text{кин}}}}} \\
    &p = \frac{mv}{\sqrt{1 - \frac{v^2}{c^2}}} = \frac{E_0}{c^2} \cdot \sqrt{1 - \sqr{\frac{E_0}{{E_0} + {E_\text{кин}}}}} \cdot \frac{{E_\text{кин}} + {E_0}}{E_0} = \frac{E_0}{c^2} \cdot \sqrt{\sqr{\frac{{E_\text{кин}} + {E_0}}{E_0}} - 1}.
    \\
    E_\text{кин}, p:\quad&E_\text{кин} = E - E_0 = mc^2\cbr{\frac 1{\sqrt{1 - \frac{v^2}{c^2}}} - 1}, p = \frac{mv}{\sqrt{1 - \frac{v^2}{c^2}}} \implies \frac{E_\text{кин}}{p} = \frac{\frac 1{\sqrt{1 - \frac{v^2}{c^2}}} - 1}{\sqrt{1 - \frac{v^2}{c^2}}} \implies v = \ldots \\
    &E_0 = E - E_\text{кин} = \frac{E_0}{\sqrt{1 - \frac{v^2}{c^2}}} - E_\text{кин} \implies E_0 = \frac{E_\text{кин}}{\frac 1{\sqrt{1 - \frac{v^2}{c^2}}} - 1} = \ldots \\
    E_\text{кин}, v:\quad&E_\text{кин} = E - E_0 = mc^2\cbr{\frac 1{\sqrt{1 - \frac{v^2}{c^2}}} - 1} \implies m = \frac{E_\text{кин}}{c^2\cbr{\frac 1{\sqrt{1 - \frac{v^2}{c^2}}} - 1}} \\
    &E_0 = mc^2 = \frac{E_\text{кин}}{\frac 1{\sqrt{1 - \frac{v^2}{c^2}}} - 1} \\
    &p = \frac{mv}{\sqrt{1 - \frac{v^2}{c^2}}} = \frac{E_\text{кин}}{c^2\cbr{\frac 1{\sqrt{1 - \frac{v^2}{c^2}}} - 1}} \cdot \frac{v}{\sqrt{1 - \frac{v^2}{c^2}}} = \frac{{E_\text{кин}} v}{c^2\cbr{1 - {\sqrt{1 - \frac{v^2}{c^2}}}}} \\
    E_0, p:\quad&E_0 = mc^2, \quad p = \frac{mv}{\sqrt{1 - \frac{v^2}{c^2}}} \implies \frac{E_0}{p} = \frac{c^2}v{\sqrt{1 - \frac{v^2}{c^2}}} = c\sqrt{\frac{c^2}{v^2} - 1} \\
    &\sqr{\frac{E_0}{pc}} = \frac{c^2}{v^2} - 1 \implies \frac{v^2}{c^2} = \frac 1{1 + \frac{E_0^2}{p^2c^2}} \implies v = \frac c{\sqrt{1 + \frac{E_0^2}{p^2c^2}}} \\
    &{E_\text{кин}} = E - E_0 = \sqrt{E_0^2 + p^2c^2} - E_0 \\
    E_0, v:\quad&E_0 = mc^2 \implies m = \frac{E_0}{c^2} \qquad p = \frac{mv}{\sqrt{1 - \frac{v^2}{c^2}}} = \frac{E_0}{c^2} \cdot \frac{v}{\sqrt{1 - \frac{v^2}{c^2}}} \\
    &E_\text{кин}= mc^2\cbr{\frac 1{\sqrt{1 - \frac{v^2}{c^2}}} - 1} = \frac{E_0}{c^2}\cbr{\frac 1{\sqrt{1 - \frac{v^2}{c^2}}} - 1} \\
    p, v:\quad&p = \frac{mv}{\sqrt{1 - \frac{v^2}{c^2}}} \implies m = \frac p v {\sqrt{1 - \frac{v^2}{c^2}}} \implies E_0 = mc^2 =\frac {pc^2} v {\sqrt{1 - \frac{v^2}{c^2}}} \\
    &E_\text{кин} = mc^2\cbr{\frac 1{\sqrt{1 - \frac{v^2}{c^2}}} - 1} = \frac p v {\sqrt{1 - \frac{v^2}{c^2}}}\cbr{\frac 1{\sqrt{1 - \frac{v^2}{c^2}}} - 1} = \frac p v \cbr{1 - {\sqrt{1 - \frac{v^2}{c^2}}}}
    \end{align*}
}

\variantsplitter

\addpersonalvariant{Арсений Трофимов}

\tasknumber{1}%
\task{%
    Запишите
    \begin{itemize}
        \item постулаты специальной теории относительности,
        \item пример релятивистского эффекта, обнаружимый при скоростях гораздо меньше скорости света.
    \end{itemize}
}
\solutionspace{120pt}

\tasknumber{2}%
\task{%
    Запишите формулу для ...
    \begin{itemize}
        \item релятивистского замедления времени,
        \item классической полной механической энергии тела,
        \item релятивистского импульса тела,
        \item релятивистской кинетической энергии,
        \item связь между релятивистским импульсом и релятивистской энергией.
    \end{itemize}
    Обязательно подпишите все физические величины.
}
\solutionspace{150pt}

\tasknumber{3}%
\task{%
    Позитрон движется со скоростью $0{,}8\,c$, где $c$~--- скорость света в вакууме.
    Каково при этом отношение полной энергии частицы $E$ к его энергии покоя $E_0$?
}
\answer{%
    \begin{align*}
    E &= \frac{E_0}{\sqrt{1 - \frac{v^2}{c^2}}}
            \implies \frac E{E_0}
                = \frac 1{\sqrt{1 - \frac{v^2}{c^2}}}
                = \frac 1{\sqrt{1 - \sqr{0{,}8}}}
                \approx 1{,}667,
         \\
        {E_{\text{кин}}} &= E - E_0
            \implies \frac{E_{\text{кин}}}{E_0}
                = \frac E{E_0} - 1
                = \frac 1{\sqrt{1 - \frac{v^2}{c^2}}} - 1
                = \frac 1{\sqrt{1 - \sqr{0{,}8}}} - 1
                \approx 0{,}667.
    \end{align*}
}
\solutionspace{80pt}

\tasknumber{4}%
\task{%
    Полная энергия релятивистской частицы в четыре раза больше её энергии покоя.
    Найти скорость этой частицы: в долях $c$ и численное значение.
    Скорость света в вакууме $c = 3 \cdot 10^{8}\,\frac{\text{м}}{\text{с}}$.
}
\answer{%
    \begin{align*}
    E &= \frac{E_0}{\sqrt{1 - \frac{v^2}{c^2}}}\implies \sqrt{1 - \frac{v^2}{c^2}} = \frac{E_0}{E}\implies \frac{v^2}{c^2} = 1 - \sqr{\frac{E_0}{E}}\implies v = c \sqrt{1 - \sqr{\frac{E_0}{E}}} \approx 0{,}968c \approx 290 \cdot 10^{6}\,\frac{\text{м}}{\text{с}}.
    \end{align*}
}
\solutionspace{80pt}

\tasknumber{5}%
\task{%
    Кинетическая энергия релятивистской частицы в четыре раза больше её энергии покоя.
    Найти скорость этой частицы.
    Скорость света в вакууме $c = 3 \cdot 10^{8}\,\frac{\text{м}}{\text{с}}$.
}
\answer{%
    \begin{align*}
    E &= E_0 + E_{\text{кин}} \\
    E &= \frac{E_0}{\sqrt{1 - \frac{v^2}{c^2}}}\implies \sqrt{1 - \frac{v^2}{c^2}} = \frac{E_0}{E}\implies \frac{v^2}{c^2} = 1 - \sqr{\frac{E_0}{E}} \implies \\
    \implies &v = c \sqrt{1 - \sqr{\frac{E_0}{E}}} = c \sqrt{1 - \sqr{\frac{E_0}{E_0 + E_{\text{кин}} }}} = c \sqrt{1 - \frac 1 {\sqr{ 1 + \frac{E_{\text{кин}}}{E_0} }} }\approx 0{,}980c \approx 294 \cdot 10^{6}\,\frac{\text{м}}{\text{с}}.
    \end{align*}
}


\variantsplitter


\addpersonalvariant{Арсений Трофимов}

\tasknumber{6}%
\task{%
    Протон движется со скоростью $0{,}65\,c$, где $c$~--- скорость света в вакууме.
    Определите его импульс (в ответе приведите формулу и укажите численное значение).
}
\answer{%
    \begin{align*}
    E &= \frac{mc^2}{\sqrt{1 - \frac{v^2}{c^2}}}
            \approx \frac{1{,}673 \cdot 10^{-27}\,\text{кг} \cdot \sqr{3 \cdot 10^{8}\,\frac{\text{м}}{\text{с}}}}{\sqrt{1 - 0{,}65^2}}
            \approx 0{,}19809 \cdot 10^{-9}\,\text{Дж},
         \\
        E_{\text{кин}} &= \frac{mc^2}{\sqrt{1 - \frac{v^2}{c^2}}} - mc^2
            = mc^2 \cbr{\frac 1{\sqrt{1 - \frac{v^2}{c^2}}} - 1} \approx \\
            &\approx \cbr{1{,}673 \cdot 10^{-27}\,\text{кг} \cdot \sqr{3 \cdot 10^{8}\,\frac{\text{м}}{\text{с}}}}
            \cdot \cbr{\frac 1{\sqrt{1 - 0{,}65^2}} - 1}
            \approx 47{,}55 \cdot 10^{-12}\,\text{Дж},
         \\
        p &= \frac{mv}{\sqrt{1 - \frac{v^2}{c^2}}}
            \approx \frac{1{,}673 \cdot 10^{-27}\,\text{кг} \cdot 0{,}65 \cdot 3 \cdot 10^{8}\,\frac{\text{м}}{\text{с}}}{\sqrt{1 - 0{,}65^2}}
            \approx 0{,}4292 \cdot 10^{-18}\,\frac{\text{кг}\cdot\text{м}}{\text{с}}.
    \end{align*}
}
\solutionspace{100pt}

\tasknumber{7}%
\task{%
    Кинетическая энергия частицы космических лучей в четыре раза превышает её энергию покоя.
    Определить отношение скорости частицы к скорости света.
}
\answer{%
    \begin{align*}
    E &= E_0 + E_{\text{кин}} \\
    E &= \frac{E_0}{\sqrt{1 - \frac{v^2}{c^2}}}\implies \sqrt{1 - \frac{v^2}{c^2}} = \frac{E_0}{E}\implies \frac{v^2}{c^2} = 1 - \sqr{\frac{E_0}{E}} \implies \\
    \implies \frac vc &= \sqrt{1 - \sqr{\frac{E_0}{E}}} = \sqrt{1 - \sqr{\frac{E_0}{E_0 + E_{\text{кин}} }}} \approx 0{,}980.
    \end{align*}
}
\solutionspace{80pt}

\tasknumber{8}%
\task{%
    Некоторая частица, пройдя ускоряющую разность потенциалов, приобрела импульс $3 \cdot 10^{-19}\,\frac{\text{кг}\cdot\text{м}}{\text{с}}$.
    Скорость частицы стала равной $2{,}4 \cdot 10^{8}\,\frac{\text{м}}{\text{с}}$.
    Найти массу частицы.
}
\answer{%
    $p = \frac{ mv }{\sqrt{1 - \frac{v^2}{c^2} }}\implies m = \frac pv \sqrt{1 - \frac{v^2}{c^2}}= \frac {3 \cdot 10^{-19}\,\frac{\text{кг}\cdot\text{м}}{\text{с}}}{2{,}4 \cdot 10^{8}\,\frac{\text{м}}{\text{с}}} \sqrt{1 - \sqr{\frac{2{,}4 \cdot 10^{8}\,\frac{\text{м}}{\text{с}}}{3 \cdot 10^{8}\,\frac{\text{м}}{\text{с}}}} } \approx 0{,}75 \cdot 10^{-27}\,\text{кг}.$
}
\solutionspace{80pt}

\tasknumber{9}%
\task{%
    При какой скорости движения (в м/с) релятивистское сокращение длины движущегося тела
    составит 30\%?
}
\answer{%
    \begin{align*}
    l_0 &= \frac l{\sqrt{1 - \frac{v^2}{c^2}}}
        \implies 1 - \frac{v^2}{c^2} = \sqr{\frac l{l_0}}
        \implies \frac v c = \sqrt{1 - \sqr{\frac l{l_0}}} \implies
         \\
        \implies v &= c\sqrt{1 - \sqr{\frac l{l_0}}}
        = 3 \cdot 10^{8}\,\frac{\text{м}}{\text{с}} \cdot \sqrt{1 - \sqr{\frac {l_0 - 0{,}30l_0}{l_0}}}
        = 3 \cdot 10^{8}\,\frac{\text{м}}{\text{с}} \cdot \sqrt{1 - \sqr{1 - 0{,}30}} \approx  \\
        &\approx 0{,}714c
        \approx 214 \cdot 10^{6}\,\frac{\text{м}}{\text{с}}
        \approx 771 \cdot 10^{6}\,\frac{\text{км}}{\text{ч}}.
    \end{align*}
}
\solutionspace{80pt}

\tasknumber{10}%
\task{%
    Стержень движется в продольном направлении с постоянной скоростью относительно инерциальной системы отсчёта.
    При каком значении скорости (в долях скорости света) длина стержня в этой системе отсчёта
    будет в  1{,}25  раза меньше его собственной длины?
}
\answer{%
    $l_0 = \frac l{\sqrt{1 - \frac{v^2}{c^2}}}\implies \sqrt{1 - \frac{v^2}{c^2}} = \frac{ l }{ l_0 }\implies \frac v c = \sqrt{1 - \sqr{\frac{ l }{ l_0 }}} \approx 0{,}600.$
}
\solutionspace{80pt}

\tasknumber{11}%
\task{%
    Какую скорость должно иметь движущееся тело, чтобы его продольные размеры уменьшились в два раза?
    Скорость света $c = 3 \cdot 10^{8}\,\frac{\text{м}}{\text{с}}$.
}
\answer{%
    $l_0 = \frac l{\sqrt{1 - \frac{v^2}{c^2}}}\implies \sqrt{1 - \frac{v^2}{c^2}} = \frac{ l }{ l_0 }\implies v = c\sqrt{1 - \sqr{\frac{ l }{ l_0 }}} \approx 260 \cdot 10^{6}\,\frac{\text{м}}{\text{с}}.$
}


\variantsplitter


\addpersonalvariant{Арсений Трофимов}

\tasknumber{12}%
\task{%
    Время жизни мюона, измеренное наблюдателем, относительно которого мюон покоился, равно $\tau_0$
    Какое расстояние пролетит мюон в системе отсчёта, относительно которой он движется со скоростью $v$,
    сравнимой со скоростью света в вакууме $c$?
}
\answer{%
    $\ell = v\tau = v \frac{\tau_0}{\sqrt{1 - \frac{v^2}{c^2}}}$
}
\solutionspace{80pt}

\tasknumber{13}%
\task{%
    Если $c$ — скорость света в вакууме, то с какой скоростью должна двигаться нестабильная частица относительно наблюдателя,
    чтобы её время жизни было в четыре раза больше, чем у такой же, но покоящейся относительно наблюдателя частицы?
}
\answer{%
    $\tau = \frac{\tau_0}{\sqrt{1 - \frac{v^2}{c^2}}}\implies \sqrt{1 - \frac{v^2}{c^2}} = \frac{\tau_0}{\tau}\implies v = c\sqrt{1 - \sqr{\frac{\tau_0}{\tau}} } \approx 290 \cdot 10^{6}\,\frac{\text{м}}{\text{с}}.$
}
\solutionspace{80pt}

\tasknumber{14}%
\task{%
    Время жизни нестабильной частицы, входящего в состав космических лучей, измеренное земным наблюдателем,
    относительно которого частица двигалась со скоростью, составляющей 75\% скорости света в вакууме, оказалось равным $5{,}3\,\text{мкс}$.
    Каково время жизни частицы, покоящейся относительно наблюдателя?
}
\answer{%
    $t = \frac{t_0}{\sqrt{1 - \frac{v^2}{c^2}}}\implies t_0 = t\sqrt{1 - \frac{v^2}{c^2}} \approx 3{,}5 \cdot 10^{-6}\,\text{с}.$
}
\solutionspace{80pt}

\tasknumber{15}%
\task{%
    Частица увеличила в ускорителе свою скорость с $0{,}03c$ до $0{,}60c$.
    Во сколько раз выросла её кинетическая энергия?
}
\answer{%
    \begin{align*}
    E_{\text{кин.}} &= E - E_0 = \frac{mc^2}{\sqrt{1 - \frac{v^2}{c^2}}} - mc^2 = mc^2\cbr{ \frac1{\sqrt{1 - \frac{v^2}{c^2} }} - 1}.
    \\
    \frac{E_{\text{кин.
    2}}}{E_{\text{кин.
    1}}} &= \frac{\frac1{\sqrt{1 - \frac{v_2^2}{c^2} }} - 1}{\frac1{\sqrt{1 - \frac{v_1^2}{c^2} }} - 1}\approx 555{,}18
    \end{align*}
}
\solutionspace{120pt}

\tasknumber{16}%
\task{%
    Для частицы, движущейся с релятивистской скоростью,
    выразите $E_0$ и $v$ через $c$, $E_\text{кин}$ и $p$,
    где $E_\text{кин}$~--- кинетическая энергия частицы,
    а $E_0$, $p$ и $v$~--- её энергия покоя, импульс и скорость.
}
\answer{%
    \begin{align*}
    E_\text{кин}, E_0:\quad&E = E_\text{кин} + E_0 = \frac{E_0}{\sqrt{1 - \frac{v^2}{c^2}}} \implies \sqrt{1 - \frac{v^2}{c^2}} = \frac{E_0}{{E_0} + {E_\text{кин}}} \implies v = c\sqrt{1 - \sqr{\frac{E_0}{{E_0} + {E_\text{кин}}}}} \\
    &p = \frac{mv}{\sqrt{1 - \frac{v^2}{c^2}}} = \frac{E_0}{c^2} \cdot \sqrt{1 - \sqr{\frac{E_0}{{E_0} + {E_\text{кин}}}}} \cdot \frac{{E_\text{кин}} + {E_0}}{E_0} = \frac{E_0}{c^2} \cdot \sqrt{\sqr{\frac{{E_\text{кин}} + {E_0}}{E_0}} - 1}.
    \\
    E_\text{кин}, p:\quad&E_\text{кин} = E - E_0 = mc^2\cbr{\frac 1{\sqrt{1 - \frac{v^2}{c^2}}} - 1}, p = \frac{mv}{\sqrt{1 - \frac{v^2}{c^2}}} \implies \frac{E_\text{кин}}{p} = \frac{\frac 1{\sqrt{1 - \frac{v^2}{c^2}}} - 1}{\sqrt{1 - \frac{v^2}{c^2}}} \implies v = \ldots \\
    &E_0 = E - E_\text{кин} = \frac{E_0}{\sqrt{1 - \frac{v^2}{c^2}}} - E_\text{кин} \implies E_0 = \frac{E_\text{кин}}{\frac 1{\sqrt{1 - \frac{v^2}{c^2}}} - 1} = \ldots \\
    E_\text{кин}, v:\quad&E_\text{кин} = E - E_0 = mc^2\cbr{\frac 1{\sqrt{1 - \frac{v^2}{c^2}}} - 1} \implies m = \frac{E_\text{кин}}{c^2\cbr{\frac 1{\sqrt{1 - \frac{v^2}{c^2}}} - 1}} \\
    &E_0 = mc^2 = \frac{E_\text{кин}}{\frac 1{\sqrt{1 - \frac{v^2}{c^2}}} - 1} \\
    &p = \frac{mv}{\sqrt{1 - \frac{v^2}{c^2}}} = \frac{E_\text{кин}}{c^2\cbr{\frac 1{\sqrt{1 - \frac{v^2}{c^2}}} - 1}} \cdot \frac{v}{\sqrt{1 - \frac{v^2}{c^2}}} = \frac{{E_\text{кин}} v}{c^2\cbr{1 - {\sqrt{1 - \frac{v^2}{c^2}}}}} \\
    E_0, p:\quad&E_0 = mc^2, \quad p = \frac{mv}{\sqrt{1 - \frac{v^2}{c^2}}} \implies \frac{E_0}{p} = \frac{c^2}v{\sqrt{1 - \frac{v^2}{c^2}}} = c\sqrt{\frac{c^2}{v^2} - 1} \\
    &\sqr{\frac{E_0}{pc}} = \frac{c^2}{v^2} - 1 \implies \frac{v^2}{c^2} = \frac 1{1 + \frac{E_0^2}{p^2c^2}} \implies v = \frac c{\sqrt{1 + \frac{E_0^2}{p^2c^2}}} \\
    &{E_\text{кин}} = E - E_0 = \sqrt{E_0^2 + p^2c^2} - E_0 \\
    E_0, v:\quad&E_0 = mc^2 \implies m = \frac{E_0}{c^2} \qquad p = \frac{mv}{\sqrt{1 - \frac{v^2}{c^2}}} = \frac{E_0}{c^2} \cdot \frac{v}{\sqrt{1 - \frac{v^2}{c^2}}} \\
    &E_\text{кин}= mc^2\cbr{\frac 1{\sqrt{1 - \frac{v^2}{c^2}}} - 1} = \frac{E_0}{c^2}\cbr{\frac 1{\sqrt{1 - \frac{v^2}{c^2}}} - 1} \\
    p, v:\quad&p = \frac{mv}{\sqrt{1 - \frac{v^2}{c^2}}} \implies m = \frac p v {\sqrt{1 - \frac{v^2}{c^2}}} \implies E_0 = mc^2 =\frac {pc^2} v {\sqrt{1 - \frac{v^2}{c^2}}} \\
    &E_\text{кин} = mc^2\cbr{\frac 1{\sqrt{1 - \frac{v^2}{c^2}}} - 1} = \frac p v {\sqrt{1 - \frac{v^2}{c^2}}}\cbr{\frac 1{\sqrt{1 - \frac{v^2}{c^2}}} - 1} = \frac p v \cbr{1 - {\sqrt{1 - \frac{v^2}{c^2}}}}
    \end{align*}
}
% autogenerated
