\setdate{17~февраля~2022}
\setclass{11«Б»}

\addpersonalvariant{Михаил Бурмистров}

\tasknumber{1}%
\task{%
    Запишите формулу для ...
    \begin{itemize}
        \item релятивистского сжатия,
        \item классического импульса,
        \item релятивистской энергии тела,
        \item релятивистской кинетической энергии,
        \item связь между релятивистским импульсом и релятивистской энергией.
    \end{itemize}
    Обязательно подпишите все физические величины.
}
\solutionspace{150pt}

\tasknumber{2}%
\task{%
    Протон движется со скоростью $0{,}8\,c$, где $c$~--- скорость света в вакууме.
    Каково при этом отношение полной энергии частицы $E$ к его энергии покоя $E_0$?
}
\answer{%
    \begin{align*}
    E &= \frac{E_0}{\sqrt{1 - \frac{v^2}{c^2}}}
            \implies \frac E{E_0}
                = \frac 1{\sqrt{1 - \frac{v^2}{c^2}}}
                = \frac 1{\sqrt{1 - \sqr{0{,}8}}}
                \approx 1{,}667,
         \\
        {E_{\text{кин}}} &= E - E_0
            \implies \frac{E_{\text{кин}}}{E_0}
                = \frac E{E_0} - 1
                = \frac 1{\sqrt{1 - \frac{v^2}{c^2}}} - 1
                = \frac 1{\sqrt{1 - \sqr{0{,}8}}} - 1
                \approx 0{,}667.
    \end{align*}
}
\solutionspace{80pt}

\tasknumber{3}%
\task{%
    Полная энергия релятивистской частицы в три раза больше её энергии покоя.
    Найти скорость этой частицы: в долях $c$ и численное значение.
    Скорость света в вакууме $c = 3 \cdot 10^{8}\,\frac{\text{м}}{\text{с}}$.
}
\answer{%
    \begin{align*}
    E &= \frac{E_0}{\sqrt{1 - \frac{v^2}{c^2}}}\implies \sqrt{1 - \frac{v^2}{c^2}} = \frac{E_0}{E}\implies \frac{v^2}{c^2} = 1 - \sqr{\frac{E_0}{E}}\implies v = c \sqrt{1 - \sqr{\frac{E_0}{E}}} \approx 0{,}943c \approx 283 \cdot 10^{6}\,\frac{\text{м}}{\text{с}}.
    \end{align*}
}
\solutionspace{80pt}

\tasknumber{4}%
\task{%
    Кинетическая энергия релятивистской частицы в три раза больше её энергии покоя.
    Найти скорость этой частицы.
    Скорость света в вакууме $c = 3 \cdot 10^{8}\,\frac{\text{м}}{\text{с}}$.
}
\answer{%
    \begin{align*}
    E &= E_0 + E_{\text{кин}} \\
    E &= \frac{E_0}{\sqrt{1 - \frac{v^2}{c^2}}}\implies \sqrt{1 - \frac{v^2}{c^2}} = \frac{E_0}{E}\implies \frac{v^2}{c^2} = 1 - \sqr{\frac{E_0}{E}} \implies \\
    \implies &v = c \sqrt{1 - \sqr{\frac{E_0}{E}}} = c \sqrt{1 - \sqr{\frac{E_0}{E_0 + E_{\text{кин}} }}} = c \sqrt{1 - \frac 1 {\sqr{ 1 + \frac{E_{\text{кин}}}{E_0} }} }\approx 0{,}968c \approx 290 \cdot 10^{6}\,\frac{\text{м}}{\text{с}}.
    \end{align*}
}
\solutionspace{80pt}

\tasknumber{5}%
\task{%
    Кинетическая энергия частицы космических лучей в три раза превышает её энергию покоя.
    Определить отношение скорости частицы к скорости света.
}
\answer{%
    \begin{align*}
    E &= E_0 + E_{\text{кин}} \\
    E &= \frac{E_0}{\sqrt{1 - \frac{v^2}{c^2}}}\implies \sqrt{1 - \frac{v^2}{c^2}} = \frac{E_0}{E}\implies \frac{v^2}{c^2} = 1 - \sqr{\frac{E_0}{E}} \implies \\
    \implies \frac vc &= \sqrt{1 - \sqr{\frac{E_0}{E}}} = \sqrt{1 - \sqr{\frac{E_0}{E_0 + E_{\text{кин}} }}} \approx 0{,}968.
    \end{align*}
}
\solutionspace{80pt}

\tasknumber{6}%
\task{%
    Как (и на сколько) изменяется масса $1\,\text{т}$ воды при нагревании от $0\celsius$ до $100\celsius$?
}
\answer{%
    $\Delta m = \frac Q{c^2} \approx 5 \cdot 10^{-9}\,\text{кг}$
}

\variantsplitter

\addpersonalvariant{Снежана Авдошина}

\tasknumber{1}%
\task{%
    Запишите формулу для ...
    \begin{itemize}
        \item релятивистского замедления времени,
        \item классического импульса,
        \item релятивистского импульса тела,
        \item энергии покоя тела,
        \item связь между релятивистским импульсом и релятивистской энергией.
    \end{itemize}
    Обязательно подпишите все физические величины.
}
\solutionspace{150pt}

\tasknumber{2}%
\task{%
    Позитрон движется со скоростью $0{,}7\,c$, где $c$~--- скорость света в вакууме.
    Каково при этом отношение полной энергии частицы $E$ к его энергии покоя $E_0$?
}
\answer{%
    \begin{align*}
    E &= \frac{E_0}{\sqrt{1 - \frac{v^2}{c^2}}}
            \implies \frac E{E_0}
                = \frac 1{\sqrt{1 - \frac{v^2}{c^2}}}
                = \frac 1{\sqrt{1 - \sqr{0{,}7}}}
                \approx 1{,}400,
         \\
        {E_{\text{кин}}} &= E - E_0
            \implies \frac{E_{\text{кин}}}{E_0}
                = \frac E{E_0} - 1
                = \frac 1{\sqrt{1 - \frac{v^2}{c^2}}} - 1
                = \frac 1{\sqrt{1 - \sqr{0{,}7}}} - 1
                \approx 0{,}400.
    \end{align*}
}
\solutionspace{80pt}

\tasknumber{3}%
\task{%
    Полная энергия релятивистской частицы в четыре раза больше её энергии покоя.
    Найти скорость этой частицы: в долях $c$ и численное значение.
    Скорость света в вакууме $c = 3 \cdot 10^{8}\,\frac{\text{м}}{\text{с}}$.
}
\answer{%
    \begin{align*}
    E &= \frac{E_0}{\sqrt{1 - \frac{v^2}{c^2}}}\implies \sqrt{1 - \frac{v^2}{c^2}} = \frac{E_0}{E}\implies \frac{v^2}{c^2} = 1 - \sqr{\frac{E_0}{E}}\implies v = c \sqrt{1 - \sqr{\frac{E_0}{E}}} \approx 0{,}968c \approx 290 \cdot 10^{6}\,\frac{\text{м}}{\text{с}}.
    \end{align*}
}
\solutionspace{80pt}

\tasknumber{4}%
\task{%
    Кинетическая энергия релятивистской частицы в четыре раза больше её энергии покоя.
    Найти скорость этой частицы.
    Скорость света в вакууме $c = 3 \cdot 10^{8}\,\frac{\text{м}}{\text{с}}$.
}
\answer{%
    \begin{align*}
    E &= E_0 + E_{\text{кин}} \\
    E &= \frac{E_0}{\sqrt{1 - \frac{v^2}{c^2}}}\implies \sqrt{1 - \frac{v^2}{c^2}} = \frac{E_0}{E}\implies \frac{v^2}{c^2} = 1 - \sqr{\frac{E_0}{E}} \implies \\
    \implies &v = c \sqrt{1 - \sqr{\frac{E_0}{E}}} = c \sqrt{1 - \sqr{\frac{E_0}{E_0 + E_{\text{кин}} }}} = c \sqrt{1 - \frac 1 {\sqr{ 1 + \frac{E_{\text{кин}}}{E_0} }} }\approx 0{,}980c \approx 294 \cdot 10^{6}\,\frac{\text{м}}{\text{с}}.
    \end{align*}
}
\solutionspace{80pt}

\tasknumber{5}%
\task{%
    Кинетическая энергия частицы космических лучей в четыре раза превышает её энергию покоя.
    Определить отношение скорости частицы к скорости света.
}
\answer{%
    \begin{align*}
    E &= E_0 + E_{\text{кин}} \\
    E &= \frac{E_0}{\sqrt{1 - \frac{v^2}{c^2}}}\implies \sqrt{1 - \frac{v^2}{c^2}} = \frac{E_0}{E}\implies \frac{v^2}{c^2} = 1 - \sqr{\frac{E_0}{E}} \implies \\
    \implies \frac vc &= \sqrt{1 - \sqr{\frac{E_0}{E}}} = \sqrt{1 - \sqr{\frac{E_0}{E_0 + E_{\text{кин}} }}} \approx 0{,}980.
    \end{align*}
}
\solutionspace{80pt}

\tasknumber{6}%
\task{%
    Как (и на сколько) изменяется масса $2500\,\text{т}$ воды при испарении при $100\celsius$?
}
\answer{%
    $\Delta m = \frac Q{c^2} \approx 64 \cdot 10^{-6}\,\text{кг}$
}

\variantsplitter

\addpersonalvariant{Марьяна Аристова}

\tasknumber{1}%
\task{%
    Запишите формулу для ...
    \begin{itemize}
        \item релятивистского сжатия,
        \item классического импульса,
        \item релятивистского импульса тела,
        \item релятивистской кинетической энергии,
        \item связь между релятивистским импульсом и релятивистской энергией.
    \end{itemize}
    Обязательно подпишите все физические величины.
}
\solutionspace{150pt}

\tasknumber{2}%
\task{%
    Электрон движется со скоростью $0{,}7\,c$, где $c$~--- скорость света в вакууме.
    Каково при этом отношение кинетической энергии частицы $E_\text{кин.}$ к его энергии покоя $E_0$?
}
\answer{%
    \begin{align*}
    E &= \frac{E_0}{\sqrt{1 - \frac{v^2}{c^2}}}
            \implies \frac E{E_0}
                = \frac 1{\sqrt{1 - \frac{v^2}{c^2}}}
                = \frac 1{\sqrt{1 - \sqr{0{,}7}}}
                \approx 1{,}400,
         \\
        {E_{\text{кин}}} &= E - E_0
            \implies \frac{E_{\text{кин}}}{E_0}
                = \frac E{E_0} - 1
                = \frac 1{\sqrt{1 - \frac{v^2}{c^2}}} - 1
                = \frac 1{\sqrt{1 - \sqr{0{,}7}}} - 1
                \approx 0{,}400.
    \end{align*}
}
\solutionspace{80pt}

\tasknumber{3}%
\task{%
    Полная энергия релятивистской частицы в три раза больше её энергии покоя.
    Найти скорость этой частицы: в долях $c$ и численное значение.
    Скорость света в вакууме $c = 3 \cdot 10^{8}\,\frac{\text{м}}{\text{с}}$.
}
\answer{%
    \begin{align*}
    E &= \frac{E_0}{\sqrt{1 - \frac{v^2}{c^2}}}\implies \sqrt{1 - \frac{v^2}{c^2}} = \frac{E_0}{E}\implies \frac{v^2}{c^2} = 1 - \sqr{\frac{E_0}{E}}\implies v = c \sqrt{1 - \sqr{\frac{E_0}{E}}} \approx 0{,}943c \approx 283 \cdot 10^{6}\,\frac{\text{м}}{\text{с}}.
    \end{align*}
}
\solutionspace{80pt}

\tasknumber{4}%
\task{%
    Кинетическая энергия релятивистской частицы в три раза больше её энергии покоя.
    Найти скорость этой частицы.
    Скорость света в вакууме $c = 3 \cdot 10^{8}\,\frac{\text{м}}{\text{с}}$.
}
\answer{%
    \begin{align*}
    E &= E_0 + E_{\text{кин}} \\
    E &= \frac{E_0}{\sqrt{1 - \frac{v^2}{c^2}}}\implies \sqrt{1 - \frac{v^2}{c^2}} = \frac{E_0}{E}\implies \frac{v^2}{c^2} = 1 - \sqr{\frac{E_0}{E}} \implies \\
    \implies &v = c \sqrt{1 - \sqr{\frac{E_0}{E}}} = c \sqrt{1 - \sqr{\frac{E_0}{E_0 + E_{\text{кин}} }}} = c \sqrt{1 - \frac 1 {\sqr{ 1 + \frac{E_{\text{кин}}}{E_0} }} }\approx 0{,}968c \approx 290 \cdot 10^{6}\,\frac{\text{м}}{\text{с}}.
    \end{align*}
}
\solutionspace{80pt}

\tasknumber{5}%
\task{%
    Кинетическая энергия частицы космических лучей в три раза превышает её энергию покоя.
    Определить отношение скорости частицы к скорости света.
}
\answer{%
    \begin{align*}
    E &= E_0 + E_{\text{кин}} \\
    E &= \frac{E_0}{\sqrt{1 - \frac{v^2}{c^2}}}\implies \sqrt{1 - \frac{v^2}{c^2}} = \frac{E_0}{E}\implies \frac{v^2}{c^2} = 1 - \sqr{\frac{E_0}{E}} \implies \\
    \implies \frac vc &= \sqrt{1 - \sqr{\frac{E_0}{E}}} = \sqrt{1 - \sqr{\frac{E_0}{E_0 + E_{\text{кин}} }}} \approx 0{,}968.
    \end{align*}
}
\solutionspace{80pt}

\tasknumber{6}%
\task{%
    Как (и на сколько) изменяется масса $1\,\text{т}$ воды при испарении при $100\celsius$?
}
\answer{%
    $\Delta m = \frac Q{c^2} \approx 30 \cdot 10^{-9}\,\text{кг}$
}

\variantsplitter

\addpersonalvariant{Никита Иванов}

\tasknumber{1}%
\task{%
    Запишите формулу для ...
    \begin{itemize}
        \item релятивистского замедления времени,
        \item классического импульса,
        \item релятивистского импульса тела,
        \item релятивистской кинетической энергии,
        \item связь между релятивистским импульсом и релятивистской энергией.
    \end{itemize}
    Обязательно подпишите все физические величины.
}
\solutionspace{150pt}

\tasknumber{2}%
\task{%
    Электрон движется со скоростью $0{,}6\,c$, где $c$~--- скорость света в вакууме.
    Каково при этом отношение полной энергии частицы $E$ к его энергии покоя $E_0$?
}
\answer{%
    \begin{align*}
    E &= \frac{E_0}{\sqrt{1 - \frac{v^2}{c^2}}}
            \implies \frac E{E_0}
                = \frac 1{\sqrt{1 - \frac{v^2}{c^2}}}
                = \frac 1{\sqrt{1 - \sqr{0{,}6}}}
                \approx 1{,}250,
         \\
        {E_{\text{кин}}} &= E - E_0
            \implies \frac{E_{\text{кин}}}{E_0}
                = \frac E{E_0} - 1
                = \frac 1{\sqrt{1 - \frac{v^2}{c^2}}} - 1
                = \frac 1{\sqrt{1 - \sqr{0{,}6}}} - 1
                \approx 0{,}250.
    \end{align*}
}
\solutionspace{80pt}

\tasknumber{3}%
\task{%
    Полная энергия релятивистской частицы в пять раз больше её энергии покоя.
    Найти скорость этой частицы: в долях $c$ и численное значение.
    Скорость света в вакууме $c = 3 \cdot 10^{8}\,\frac{\text{м}}{\text{с}}$.
}
\answer{%
    \begin{align*}
    E &= \frac{E_0}{\sqrt{1 - \frac{v^2}{c^2}}}\implies \sqrt{1 - \frac{v^2}{c^2}} = \frac{E_0}{E}\implies \frac{v^2}{c^2} = 1 - \sqr{\frac{E_0}{E}}\implies v = c \sqrt{1 - \sqr{\frac{E_0}{E}}} \approx 0{,}980c \approx 294 \cdot 10^{6}\,\frac{\text{м}}{\text{с}}.
    \end{align*}
}
\solutionspace{80pt}

\tasknumber{4}%
\task{%
    Кинетическая энергия релятивистской частицы в пять раз больше её энергии покоя.
    Найти скорость этой частицы.
    Скорость света в вакууме $c = 3 \cdot 10^{8}\,\frac{\text{м}}{\text{с}}$.
}
\answer{%
    \begin{align*}
    E &= E_0 + E_{\text{кин}} \\
    E &= \frac{E_0}{\sqrt{1 - \frac{v^2}{c^2}}}\implies \sqrt{1 - \frac{v^2}{c^2}} = \frac{E_0}{E}\implies \frac{v^2}{c^2} = 1 - \sqr{\frac{E_0}{E}} \implies \\
    \implies &v = c \sqrt{1 - \sqr{\frac{E_0}{E}}} = c \sqrt{1 - \sqr{\frac{E_0}{E_0 + E_{\text{кин}} }}} = c \sqrt{1 - \frac 1 {\sqr{ 1 + \frac{E_{\text{кин}}}{E_0} }} }\approx 0{,}986c \approx 296 \cdot 10^{6}\,\frac{\text{м}}{\text{с}}.
    \end{align*}
}
\solutionspace{80pt}

\tasknumber{5}%
\task{%
    Кинетическая энергия частицы космических лучей в пять раз превышает её энергию покоя.
    Определить отношение скорости частицы к скорости света.
}
\answer{%
    \begin{align*}
    E &= E_0 + E_{\text{кин}} \\
    E &= \frac{E_0}{\sqrt{1 - \frac{v^2}{c^2}}}\implies \sqrt{1 - \frac{v^2}{c^2}} = \frac{E_0}{E}\implies \frac{v^2}{c^2} = 1 - \sqr{\frac{E_0}{E}} \implies \\
    \implies \frac vc &= \sqrt{1 - \sqr{\frac{E_0}{E}}} = \sqrt{1 - \sqr{\frac{E_0}{E_0 + E_{\text{кин}} }}} \approx 0{,}986.
    \end{align*}
}
\solutionspace{80pt}

\tasknumber{6}%
\task{%
    Как (и на сколько) изменяется масса $1\,\text{т}$ воды при нагревании от $0\celsius$ до $100\celsius$?
}
\answer{%
    $\Delta m = \frac Q{c^2} \approx 5 \cdot 10^{-9}\,\text{кг}$
}

\variantsplitter

\addpersonalvariant{Анастасия Князева}

\tasknumber{1}%
\task{%
    Запишите формулу для ...
    \begin{itemize}
        \item релятивистского сжатия,
        \item классического импульса,
        \item релятивистской энергии тела,
        \item релятивистской кинетической энергии,
        \item связь между релятивистским импульсом и релятивистской энергией.
    \end{itemize}
    Обязательно подпишите все физические величины.
}
\solutionspace{150pt}

\tasknumber{2}%
\task{%
    Позитрон движется со скоростью $0{,}7\,c$, где $c$~--- скорость света в вакууме.
    Каково при этом отношение кинетической энергии частицы $E_\text{кин.}$ к его энергии покоя $E_0$?
}
\answer{%
    \begin{align*}
    E &= \frac{E_0}{\sqrt{1 - \frac{v^2}{c^2}}}
            \implies \frac E{E_0}
                = \frac 1{\sqrt{1 - \frac{v^2}{c^2}}}
                = \frac 1{\sqrt{1 - \sqr{0{,}7}}}
                \approx 1{,}400,
         \\
        {E_{\text{кин}}} &= E - E_0
            \implies \frac{E_{\text{кин}}}{E_0}
                = \frac E{E_0} - 1
                = \frac 1{\sqrt{1 - \frac{v^2}{c^2}}} - 1
                = \frac 1{\sqrt{1 - \sqr{0{,}7}}} - 1
                \approx 0{,}400.
    \end{align*}
}
\solutionspace{80pt}

\tasknumber{3}%
\task{%
    Полная энергия релятивистской частицы в три раза больше её энергии покоя.
    Найти скорость этой частицы: в долях $c$ и численное значение.
    Скорость света в вакууме $c = 3 \cdot 10^{8}\,\frac{\text{м}}{\text{с}}$.
}
\answer{%
    \begin{align*}
    E &= \frac{E_0}{\sqrt{1 - \frac{v^2}{c^2}}}\implies \sqrt{1 - \frac{v^2}{c^2}} = \frac{E_0}{E}\implies \frac{v^2}{c^2} = 1 - \sqr{\frac{E_0}{E}}\implies v = c \sqrt{1 - \sqr{\frac{E_0}{E}}} \approx 0{,}943c \approx 283 \cdot 10^{6}\,\frac{\text{м}}{\text{с}}.
    \end{align*}
}
\solutionspace{80pt}

\tasknumber{4}%
\task{%
    Кинетическая энергия релятивистской частицы в три раза больше её энергии покоя.
    Найти скорость этой частицы.
    Скорость света в вакууме $c = 3 \cdot 10^{8}\,\frac{\text{м}}{\text{с}}$.
}
\answer{%
    \begin{align*}
    E &= E_0 + E_{\text{кин}} \\
    E &= \frac{E_0}{\sqrt{1 - \frac{v^2}{c^2}}}\implies \sqrt{1 - \frac{v^2}{c^2}} = \frac{E_0}{E}\implies \frac{v^2}{c^2} = 1 - \sqr{\frac{E_0}{E}} \implies \\
    \implies &v = c \sqrt{1 - \sqr{\frac{E_0}{E}}} = c \sqrt{1 - \sqr{\frac{E_0}{E_0 + E_{\text{кин}} }}} = c \sqrt{1 - \frac 1 {\sqr{ 1 + \frac{E_{\text{кин}}}{E_0} }} }\approx 0{,}968c \approx 290 \cdot 10^{6}\,\frac{\text{м}}{\text{с}}.
    \end{align*}
}
\solutionspace{80pt}

\tasknumber{5}%
\task{%
    Кинетическая энергия частицы космических лучей в три раза превышает её энергию покоя.
    Определить отношение скорости частицы к скорости света.
}
\answer{%
    \begin{align*}
    E &= E_0 + E_{\text{кин}} \\
    E &= \frac{E_0}{\sqrt{1 - \frac{v^2}{c^2}}}\implies \sqrt{1 - \frac{v^2}{c^2}} = \frac{E_0}{E}\implies \frac{v^2}{c^2} = 1 - \sqr{\frac{E_0}{E}} \implies \\
    \implies \frac vc &= \sqrt{1 - \sqr{\frac{E_0}{E}}} = \sqrt{1 - \sqr{\frac{E_0}{E_0 + E_{\text{кин}} }}} \approx 0{,}968.
    \end{align*}
}
\solutionspace{80pt}

\tasknumber{6}%
\task{%
    Как (и на сколько) изменяется масса $2500\,\text{т}$ воды при замерзании при $0\celsius$?
}
\answer{%
    $\Delta m = \frac Q{c^2} \approx -9{,}44 \cdot 10^{-6}\,\text{кг}$
}

\variantsplitter

\addpersonalvariant{Елизавета Кутумова}

\tasknumber{1}%
\task{%
    Запишите формулу для ...
    \begin{itemize}
        \item релятивистского замедления времени,
        \item классического импульса,
        \item релятивистской энергии тела,
        \item энергии покоя тела,
        \item связь между релятивистским импульсом и релятивистской энергией.
    \end{itemize}
    Обязательно подпишите все физические величины.
}
\solutionspace{150pt}

\tasknumber{2}%
\task{%
    Протон движется со скоростью $0{,}7\,c$, где $c$~--- скорость света в вакууме.
    Каково при этом отношение полной энергии частицы $E$ к его энергии покоя $E_0$?
}
\answer{%
    \begin{align*}
    E &= \frac{E_0}{\sqrt{1 - \frac{v^2}{c^2}}}
            \implies \frac E{E_0}
                = \frac 1{\sqrt{1 - \frac{v^2}{c^2}}}
                = \frac 1{\sqrt{1 - \sqr{0{,}7}}}
                \approx 1{,}400,
         \\
        {E_{\text{кин}}} &= E - E_0
            \implies \frac{E_{\text{кин}}}{E_0}
                = \frac E{E_0} - 1
                = \frac 1{\sqrt{1 - \frac{v^2}{c^2}}} - 1
                = \frac 1{\sqrt{1 - \sqr{0{,}7}}} - 1
                \approx 0{,}400.
    \end{align*}
}
\solutionspace{80pt}

\tasknumber{3}%
\task{%
    Полная энергия релятивистской частицы в шесть раз больше её энергии покоя.
    Найти скорость этой частицы: в долях $c$ и численное значение.
    Скорость света в вакууме $c = 3 \cdot 10^{8}\,\frac{\text{м}}{\text{с}}$.
}
\answer{%
    \begin{align*}
    E &= \frac{E_0}{\sqrt{1 - \frac{v^2}{c^2}}}\implies \sqrt{1 - \frac{v^2}{c^2}} = \frac{E_0}{E}\implies \frac{v^2}{c^2} = 1 - \sqr{\frac{E_0}{E}}\implies v = c \sqrt{1 - \sqr{\frac{E_0}{E}}} \approx 0{,}986c \approx 296 \cdot 10^{6}\,\frac{\text{м}}{\text{с}}.
    \end{align*}
}
\solutionspace{80pt}

\tasknumber{4}%
\task{%
    Кинетическая энергия релятивистской частицы в шесть раз больше её энергии покоя.
    Найти скорость этой частицы.
    Скорость света в вакууме $c = 3 \cdot 10^{8}\,\frac{\text{м}}{\text{с}}$.
}
\answer{%
    \begin{align*}
    E &= E_0 + E_{\text{кин}} \\
    E &= \frac{E_0}{\sqrt{1 - \frac{v^2}{c^2}}}\implies \sqrt{1 - \frac{v^2}{c^2}} = \frac{E_0}{E}\implies \frac{v^2}{c^2} = 1 - \sqr{\frac{E_0}{E}} \implies \\
    \implies &v = c \sqrt{1 - \sqr{\frac{E_0}{E}}} = c \sqrt{1 - \sqr{\frac{E_0}{E_0 + E_{\text{кин}} }}} = c \sqrt{1 - \frac 1 {\sqr{ 1 + \frac{E_{\text{кин}}}{E_0} }} }\approx 0{,}990c \approx 297 \cdot 10^{6}\,\frac{\text{м}}{\text{с}}.
    \end{align*}
}
\solutionspace{80pt}

\tasknumber{5}%
\task{%
    Кинетическая энергия частицы космических лучей в шесть раз превышает её энергию покоя.
    Определить отношение скорости частицы к скорости света.
}
\answer{%
    \begin{align*}
    E &= E_0 + E_{\text{кин}} \\
    E &= \frac{E_0}{\sqrt{1 - \frac{v^2}{c^2}}}\implies \sqrt{1 - \frac{v^2}{c^2}} = \frac{E_0}{E}\implies \frac{v^2}{c^2} = 1 - \sqr{\frac{E_0}{E}} \implies \\
    \implies \frac vc &= \sqrt{1 - \sqr{\frac{E_0}{E}}} = \sqrt{1 - \sqr{\frac{E_0}{E_0 + E_{\text{кин}} }}} \approx 0{,}990.
    \end{align*}
}
\solutionspace{80pt}

\tasknumber{6}%
\task{%
    Как (и на сколько) изменяется масса $1000\,\text{т}$ воды при нагревании от $0\celsius$ до $100\celsius$?
}
\answer{%
    $\Delta m = \frac Q{c^2} \approx 4{,}7 \cdot 10^{-6}\,\text{кг}$
}

\variantsplitter

\addpersonalvariant{Роксана Мехтиева}

\tasknumber{1}%
\task{%
    Запишите формулу для ...
    \begin{itemize}
        \item релятивистского замедления времени,
        \item классического импульса,
        \item релятивистского импульса тела,
        \item энергии покоя тела,
        \item связь между релятивистским импульсом и релятивистской энергией.
    \end{itemize}
    Обязательно подпишите все физические величины.
}
\solutionspace{150pt}

\tasknumber{2}%
\task{%
    Позитрон движется со скоростью $0{,}7\,c$, где $c$~--- скорость света в вакууме.
    Каково при этом отношение кинетической энергии частицы $E_\text{кин.}$ к его энергии покоя $E_0$?
}
\answer{%
    \begin{align*}
    E &= \frac{E_0}{\sqrt{1 - \frac{v^2}{c^2}}}
            \implies \frac E{E_0}
                = \frac 1{\sqrt{1 - \frac{v^2}{c^2}}}
                = \frac 1{\sqrt{1 - \sqr{0{,}7}}}
                \approx 1{,}400,
         \\
        {E_{\text{кин}}} &= E - E_0
            \implies \frac{E_{\text{кин}}}{E_0}
                = \frac E{E_0} - 1
                = \frac 1{\sqrt{1 - \frac{v^2}{c^2}}} - 1
                = \frac 1{\sqrt{1 - \sqr{0{,}7}}} - 1
                \approx 0{,}400.
    \end{align*}
}
\solutionspace{80pt}

\tasknumber{3}%
\task{%
    Полная энергия релятивистской частицы в шесть раз больше её энергии покоя.
    Найти скорость этой частицы: в долях $c$ и численное значение.
    Скорость света в вакууме $c = 3 \cdot 10^{8}\,\frac{\text{м}}{\text{с}}$.
}
\answer{%
    \begin{align*}
    E &= \frac{E_0}{\sqrt{1 - \frac{v^2}{c^2}}}\implies \sqrt{1 - \frac{v^2}{c^2}} = \frac{E_0}{E}\implies \frac{v^2}{c^2} = 1 - \sqr{\frac{E_0}{E}}\implies v = c \sqrt{1 - \sqr{\frac{E_0}{E}}} \approx 0{,}986c \approx 296 \cdot 10^{6}\,\frac{\text{м}}{\text{с}}.
    \end{align*}
}
\solutionspace{80pt}

\tasknumber{4}%
\task{%
    Кинетическая энергия релятивистской частицы в шесть раз больше её энергии покоя.
    Найти скорость этой частицы.
    Скорость света в вакууме $c = 3 \cdot 10^{8}\,\frac{\text{м}}{\text{с}}$.
}
\answer{%
    \begin{align*}
    E &= E_0 + E_{\text{кин}} \\
    E &= \frac{E_0}{\sqrt{1 - \frac{v^2}{c^2}}}\implies \sqrt{1 - \frac{v^2}{c^2}} = \frac{E_0}{E}\implies \frac{v^2}{c^2} = 1 - \sqr{\frac{E_0}{E}} \implies \\
    \implies &v = c \sqrt{1 - \sqr{\frac{E_0}{E}}} = c \sqrt{1 - \sqr{\frac{E_0}{E_0 + E_{\text{кин}} }}} = c \sqrt{1 - \frac 1 {\sqr{ 1 + \frac{E_{\text{кин}}}{E_0} }} }\approx 0{,}990c \approx 297 \cdot 10^{6}\,\frac{\text{м}}{\text{с}}.
    \end{align*}
}
\solutionspace{80pt}

\tasknumber{5}%
\task{%
    Кинетическая энергия частицы космических лучей в шесть раз превышает её энергию покоя.
    Определить отношение скорости частицы к скорости света.
}
\answer{%
    \begin{align*}
    E &= E_0 + E_{\text{кин}} \\
    E &= \frac{E_0}{\sqrt{1 - \frac{v^2}{c^2}}}\implies \sqrt{1 - \frac{v^2}{c^2}} = \frac{E_0}{E}\implies \frac{v^2}{c^2} = 1 - \sqr{\frac{E_0}{E}} \implies \\
    \implies \frac vc &= \sqrt{1 - \sqr{\frac{E_0}{E}}} = \sqrt{1 - \sqr{\frac{E_0}{E_0 + E_{\text{кин}} }}} \approx 0{,}990.
    \end{align*}
}
\solutionspace{80pt}

\tasknumber{6}%
\task{%
    Как (и на сколько) изменяется масса $200\,\text{т}$ воды при замерзании при $0\celsius$?
}
\answer{%
    $\Delta m = \frac Q{c^2} \approx -0{,}756 \cdot 10^{-6}\,\text{кг}$
}

\variantsplitter

\addpersonalvariant{Дилноза Нодиршоева}

\tasknumber{1}%
\task{%
    Запишите формулу для ...
    \begin{itemize}
        \item релятивистского сжатия,
        \item классического импульса,
        \item релятивистского импульса тела,
        \item энергии покоя тела,
        \item связь между релятивистским импульсом и релятивистской энергией.
    \end{itemize}
    Обязательно подпишите все физические величины.
}
\solutionspace{150pt}

\tasknumber{2}%
\task{%
    Позитрон движется со скоростью $0{,}7\,c$, где $c$~--- скорость света в вакууме.
    Каково при этом отношение полной энергии частицы $E$ к его энергии покоя $E_0$?
}
\answer{%
    \begin{align*}
    E &= \frac{E_0}{\sqrt{1 - \frac{v^2}{c^2}}}
            \implies \frac E{E_0}
                = \frac 1{\sqrt{1 - \frac{v^2}{c^2}}}
                = \frac 1{\sqrt{1 - \sqr{0{,}7}}}
                \approx 1{,}400,
         \\
        {E_{\text{кин}}} &= E - E_0
            \implies \frac{E_{\text{кин}}}{E_0}
                = \frac E{E_0} - 1
                = \frac 1{\sqrt{1 - \frac{v^2}{c^2}}} - 1
                = \frac 1{\sqrt{1 - \sqr{0{,}7}}} - 1
                \approx 0{,}400.
    \end{align*}
}
\solutionspace{80pt}

\tasknumber{3}%
\task{%
    Полная энергия релятивистской частицы в шесть раз больше её энергии покоя.
    Найти скорость этой частицы: в долях $c$ и численное значение.
    Скорость света в вакууме $c = 3 \cdot 10^{8}\,\frac{\text{м}}{\text{с}}$.
}
\answer{%
    \begin{align*}
    E &= \frac{E_0}{\sqrt{1 - \frac{v^2}{c^2}}}\implies \sqrt{1 - \frac{v^2}{c^2}} = \frac{E_0}{E}\implies \frac{v^2}{c^2} = 1 - \sqr{\frac{E_0}{E}}\implies v = c \sqrt{1 - \sqr{\frac{E_0}{E}}} \approx 0{,}986c \approx 296 \cdot 10^{6}\,\frac{\text{м}}{\text{с}}.
    \end{align*}
}
\solutionspace{80pt}

\tasknumber{4}%
\task{%
    Кинетическая энергия релятивистской частицы в шесть раз больше её энергии покоя.
    Найти скорость этой частицы.
    Скорость света в вакууме $c = 3 \cdot 10^{8}\,\frac{\text{м}}{\text{с}}$.
}
\answer{%
    \begin{align*}
    E &= E_0 + E_{\text{кин}} \\
    E &= \frac{E_0}{\sqrt{1 - \frac{v^2}{c^2}}}\implies \sqrt{1 - \frac{v^2}{c^2}} = \frac{E_0}{E}\implies \frac{v^2}{c^2} = 1 - \sqr{\frac{E_0}{E}} \implies \\
    \implies &v = c \sqrt{1 - \sqr{\frac{E_0}{E}}} = c \sqrt{1 - \sqr{\frac{E_0}{E_0 + E_{\text{кин}} }}} = c \sqrt{1 - \frac 1 {\sqr{ 1 + \frac{E_{\text{кин}}}{E_0} }} }\approx 0{,}990c \approx 297 \cdot 10^{6}\,\frac{\text{м}}{\text{с}}.
    \end{align*}
}
\solutionspace{80pt}

\tasknumber{5}%
\task{%
    Кинетическая энергия частицы космических лучей в шесть раз превышает её энергию покоя.
    Определить отношение скорости частицы к скорости света.
}
\answer{%
    \begin{align*}
    E &= E_0 + E_{\text{кин}} \\
    E &= \frac{E_0}{\sqrt{1 - \frac{v^2}{c^2}}}\implies \sqrt{1 - \frac{v^2}{c^2}} = \frac{E_0}{E}\implies \frac{v^2}{c^2} = 1 - \sqr{\frac{E_0}{E}} \implies \\
    \implies \frac vc &= \sqrt{1 - \sqr{\frac{E_0}{E}}} = \sqrt{1 - \sqr{\frac{E_0}{E_0 + E_{\text{кин}} }}} \approx 0{,}990.
    \end{align*}
}
\solutionspace{80pt}

\tasknumber{6}%
\task{%
    Как (и на сколько) изменяется масса $200\,\text{т}$ воды при замерзании при $0\celsius$?
}
\answer{%
    $\Delta m = \frac Q{c^2} \approx -0{,}756 \cdot 10^{-6}\,\text{кг}$
}

\variantsplitter

\addpersonalvariant{Жаклин Пантелеева}

\tasknumber{1}%
\task{%
    Запишите формулу для ...
    \begin{itemize}
        \item релятивистского замедления времени,
        \item классического импульса,
        \item релятивистской энергии тела,
        \item релятивистской кинетической энергии,
        \item связь между релятивистским импульсом и релятивистской энергией.
    \end{itemize}
    Обязательно подпишите все физические величины.
}
\solutionspace{150pt}

\tasknumber{2}%
\task{%
    Протон движется со скоростью $0{,}9\,c$, где $c$~--- скорость света в вакууме.
    Каково при этом отношение полной энергии частицы $E$ к его энергии покоя $E_0$?
}
\answer{%
    \begin{align*}
    E &= \frac{E_0}{\sqrt{1 - \frac{v^2}{c^2}}}
            \implies \frac E{E_0}
                = \frac 1{\sqrt{1 - \frac{v^2}{c^2}}}
                = \frac 1{\sqrt{1 - \sqr{0{,}9}}}
                \approx 2{,}294,
         \\
        {E_{\text{кин}}} &= E - E_0
            \implies \frac{E_{\text{кин}}}{E_0}
                = \frac E{E_0} - 1
                = \frac 1{\sqrt{1 - \frac{v^2}{c^2}}} - 1
                = \frac 1{\sqrt{1 - \sqr{0{,}9}}} - 1
                \approx 1{,}294.
    \end{align*}
}
\solutionspace{80pt}

\tasknumber{3}%
\task{%
    Полная энергия релятивистской частицы в три раза больше её энергии покоя.
    Найти скорость этой частицы: в долях $c$ и численное значение.
    Скорость света в вакууме $c = 3 \cdot 10^{8}\,\frac{\text{м}}{\text{с}}$.
}
\answer{%
    \begin{align*}
    E &= \frac{E_0}{\sqrt{1 - \frac{v^2}{c^2}}}\implies \sqrt{1 - \frac{v^2}{c^2}} = \frac{E_0}{E}\implies \frac{v^2}{c^2} = 1 - \sqr{\frac{E_0}{E}}\implies v = c \sqrt{1 - \sqr{\frac{E_0}{E}}} \approx 0{,}943c \approx 283 \cdot 10^{6}\,\frac{\text{м}}{\text{с}}.
    \end{align*}
}
\solutionspace{80pt}

\tasknumber{4}%
\task{%
    Кинетическая энергия релятивистской частицы в три раза больше её энергии покоя.
    Найти скорость этой частицы.
    Скорость света в вакууме $c = 3 \cdot 10^{8}\,\frac{\text{м}}{\text{с}}$.
}
\answer{%
    \begin{align*}
    E &= E_0 + E_{\text{кин}} \\
    E &= \frac{E_0}{\sqrt{1 - \frac{v^2}{c^2}}}\implies \sqrt{1 - \frac{v^2}{c^2}} = \frac{E_0}{E}\implies \frac{v^2}{c^2} = 1 - \sqr{\frac{E_0}{E}} \implies \\
    \implies &v = c \sqrt{1 - \sqr{\frac{E_0}{E}}} = c \sqrt{1 - \sqr{\frac{E_0}{E_0 + E_{\text{кин}} }}} = c \sqrt{1 - \frac 1 {\sqr{ 1 + \frac{E_{\text{кин}}}{E_0} }} }\approx 0{,}968c \approx 290 \cdot 10^{6}\,\frac{\text{м}}{\text{с}}.
    \end{align*}
}
\solutionspace{80pt}

\tasknumber{5}%
\task{%
    Кинетическая энергия частицы космических лучей в три раза превышает её энергию покоя.
    Определить отношение скорости частицы к скорости света.
}
\answer{%
    \begin{align*}
    E &= E_0 + E_{\text{кин}} \\
    E &= \frac{E_0}{\sqrt{1 - \frac{v^2}{c^2}}}\implies \sqrt{1 - \frac{v^2}{c^2}} = \frac{E_0}{E}\implies \frac{v^2}{c^2} = 1 - \sqr{\frac{E_0}{E}} \implies \\
    \implies \frac vc &= \sqrt{1 - \sqr{\frac{E_0}{E}}} = \sqrt{1 - \sqr{\frac{E_0}{E_0 + E_{\text{кин}} }}} \approx 0{,}968.
    \end{align*}
}
\solutionspace{80pt}

\tasknumber{6}%
\task{%
    Как (и на сколько) изменяется масса $2500\,\text{т}$ воды при нагревании от $0\celsius$ до $100\celsius$?
}
\answer{%
    $\Delta m = \frac Q{c^2} \approx 11{,}7 \cdot 10^{-6}\,\text{кг}$
}

\variantsplitter

\addpersonalvariant{Артём Переверзев}

\tasknumber{1}%
\task{%
    Запишите формулу для ...
    \begin{itemize}
        \item релятивистского замедления времени,
        \item классического импульса,
        \item релятивистской энергии тела,
        \item релятивистской кинетической энергии,
        \item связь между релятивистским импульсом и релятивистской энергией.
    \end{itemize}
    Обязательно подпишите все физические величины.
}
\solutionspace{150pt}

\tasknumber{2}%
\task{%
    Электрон движется со скоростью $0{,}9\,c$, где $c$~--- скорость света в вакууме.
    Каково при этом отношение полной энергии частицы $E$ к его энергии покоя $E_0$?
}
\answer{%
    \begin{align*}
    E &= \frac{E_0}{\sqrt{1 - \frac{v^2}{c^2}}}
            \implies \frac E{E_0}
                = \frac 1{\sqrt{1 - \frac{v^2}{c^2}}}
                = \frac 1{\sqrt{1 - \sqr{0{,}9}}}
                \approx 2{,}294,
         \\
        {E_{\text{кин}}} &= E - E_0
            \implies \frac{E_{\text{кин}}}{E_0}
                = \frac E{E_0} - 1
                = \frac 1{\sqrt{1 - \frac{v^2}{c^2}}} - 1
                = \frac 1{\sqrt{1 - \sqr{0{,}9}}} - 1
                \approx 1{,}294.
    \end{align*}
}
\solutionspace{80pt}

\tasknumber{3}%
\task{%
    Полная энергия релятивистской частицы в три раза больше её энергии покоя.
    Найти скорость этой частицы: в долях $c$ и численное значение.
    Скорость света в вакууме $c = 3 \cdot 10^{8}\,\frac{\text{м}}{\text{с}}$.
}
\answer{%
    \begin{align*}
    E &= \frac{E_0}{\sqrt{1 - \frac{v^2}{c^2}}}\implies \sqrt{1 - \frac{v^2}{c^2}} = \frac{E_0}{E}\implies \frac{v^2}{c^2} = 1 - \sqr{\frac{E_0}{E}}\implies v = c \sqrt{1 - \sqr{\frac{E_0}{E}}} \approx 0{,}943c \approx 283 \cdot 10^{6}\,\frac{\text{м}}{\text{с}}.
    \end{align*}
}
\solutionspace{80pt}

\tasknumber{4}%
\task{%
    Кинетическая энергия релятивистской частицы в три раза больше её энергии покоя.
    Найти скорость этой частицы.
    Скорость света в вакууме $c = 3 \cdot 10^{8}\,\frac{\text{м}}{\text{с}}$.
}
\answer{%
    \begin{align*}
    E &= E_0 + E_{\text{кин}} \\
    E &= \frac{E_0}{\sqrt{1 - \frac{v^2}{c^2}}}\implies \sqrt{1 - \frac{v^2}{c^2}} = \frac{E_0}{E}\implies \frac{v^2}{c^2} = 1 - \sqr{\frac{E_0}{E}} \implies \\
    \implies &v = c \sqrt{1 - \sqr{\frac{E_0}{E}}} = c \sqrt{1 - \sqr{\frac{E_0}{E_0 + E_{\text{кин}} }}} = c \sqrt{1 - \frac 1 {\sqr{ 1 + \frac{E_{\text{кин}}}{E_0} }} }\approx 0{,}968c \approx 290 \cdot 10^{6}\,\frac{\text{м}}{\text{с}}.
    \end{align*}
}
\solutionspace{80pt}

\tasknumber{5}%
\task{%
    Кинетическая энергия частицы космических лучей в три раза превышает её энергию покоя.
    Определить отношение скорости частицы к скорости света.
}
\answer{%
    \begin{align*}
    E &= E_0 + E_{\text{кин}} \\
    E &= \frac{E_0}{\sqrt{1 - \frac{v^2}{c^2}}}\implies \sqrt{1 - \frac{v^2}{c^2}} = \frac{E_0}{E}\implies \frac{v^2}{c^2} = 1 - \sqr{\frac{E_0}{E}} \implies \\
    \implies \frac vc &= \sqrt{1 - \sqr{\frac{E_0}{E}}} = \sqrt{1 - \sqr{\frac{E_0}{E_0 + E_{\text{кин}} }}} \approx 0{,}968.
    \end{align*}
}
\solutionspace{80pt}

\tasknumber{6}%
\task{%
    Как (и на сколько) изменяется масса $2500\,\text{т}$ воды при нагревании от $0\celsius$ до $100\celsius$?
}
\answer{%
    $\Delta m = \frac Q{c^2} \approx 11{,}7 \cdot 10^{-6}\,\text{кг}$
}

\variantsplitter

\addpersonalvariant{Варвара Пранова}

\tasknumber{1}%
\task{%
    Запишите формулу для ...
    \begin{itemize}
        \item релятивистского сжатия,
        \item классической полной механической энергии тела,
        \item релятивистской энергии тела,
        \item энергии покоя тела,
        \item связь между релятивистским импульсом и релятивистской энергией.
    \end{itemize}
    Обязательно подпишите все физические величины.
}
\solutionspace{150pt}

\tasknumber{2}%
\task{%
    Электрон движется со скоростью $0{,}9\,c$, где $c$~--- скорость света в вакууме.
    Каково при этом отношение кинетической энергии частицы $E_\text{кин.}$ к его энергии покоя $E_0$?
}
\answer{%
    \begin{align*}
    E &= \frac{E_0}{\sqrt{1 - \frac{v^2}{c^2}}}
            \implies \frac E{E_0}
                = \frac 1{\sqrt{1 - \frac{v^2}{c^2}}}
                = \frac 1{\sqrt{1 - \sqr{0{,}9}}}
                \approx 2{,}294,
         \\
        {E_{\text{кин}}} &= E - E_0
            \implies \frac{E_{\text{кин}}}{E_0}
                = \frac E{E_0} - 1
                = \frac 1{\sqrt{1 - \frac{v^2}{c^2}}} - 1
                = \frac 1{\sqrt{1 - \sqr{0{,}9}}} - 1
                \approx 1{,}294.
    \end{align*}
}
\solutionspace{80pt}

\tasknumber{3}%
\task{%
    Полная энергия релятивистской частицы в шесть раз больше её энергии покоя.
    Найти скорость этой частицы: в долях $c$ и численное значение.
    Скорость света в вакууме $c = 3 \cdot 10^{8}\,\frac{\text{м}}{\text{с}}$.
}
\answer{%
    \begin{align*}
    E &= \frac{E_0}{\sqrt{1 - \frac{v^2}{c^2}}}\implies \sqrt{1 - \frac{v^2}{c^2}} = \frac{E_0}{E}\implies \frac{v^2}{c^2} = 1 - \sqr{\frac{E_0}{E}}\implies v = c \sqrt{1 - \sqr{\frac{E_0}{E}}} \approx 0{,}986c \approx 296 \cdot 10^{6}\,\frac{\text{м}}{\text{с}}.
    \end{align*}
}
\solutionspace{80pt}

\tasknumber{4}%
\task{%
    Кинетическая энергия релятивистской частицы в шесть раз больше её энергии покоя.
    Найти скорость этой частицы.
    Скорость света в вакууме $c = 3 \cdot 10^{8}\,\frac{\text{м}}{\text{с}}$.
}
\answer{%
    \begin{align*}
    E &= E_0 + E_{\text{кин}} \\
    E &= \frac{E_0}{\sqrt{1 - \frac{v^2}{c^2}}}\implies \sqrt{1 - \frac{v^2}{c^2}} = \frac{E_0}{E}\implies \frac{v^2}{c^2} = 1 - \sqr{\frac{E_0}{E}} \implies \\
    \implies &v = c \sqrt{1 - \sqr{\frac{E_0}{E}}} = c \sqrt{1 - \sqr{\frac{E_0}{E_0 + E_{\text{кин}} }}} = c \sqrt{1 - \frac 1 {\sqr{ 1 + \frac{E_{\text{кин}}}{E_0} }} }\approx 0{,}990c \approx 297 \cdot 10^{6}\,\frac{\text{м}}{\text{с}}.
    \end{align*}
}
\solutionspace{80pt}

\tasknumber{5}%
\task{%
    Кинетическая энергия частицы космических лучей в шесть раз превышает её энергию покоя.
    Определить отношение скорости частицы к скорости света.
}
\answer{%
    \begin{align*}
    E &= E_0 + E_{\text{кин}} \\
    E &= \frac{E_0}{\sqrt{1 - \frac{v^2}{c^2}}}\implies \sqrt{1 - \frac{v^2}{c^2}} = \frac{E_0}{E}\implies \frac{v^2}{c^2} = 1 - \sqr{\frac{E_0}{E}} \implies \\
    \implies \frac vc &= \sqrt{1 - \sqr{\frac{E_0}{E}}} = \sqrt{1 - \sqr{\frac{E_0}{E_0 + E_{\text{кин}} }}} \approx 0{,}990.
    \end{align*}
}
\solutionspace{80pt}

\tasknumber{6}%
\task{%
    Как (и на сколько) изменяется масса $2500\,\text{т}$ воды при испарении при $100\celsius$?
}
\answer{%
    $\Delta m = \frac Q{c^2} \approx 64 \cdot 10^{-6}\,\text{кг}$
}

\variantsplitter

\addpersonalvariant{Марьям Салимова}

\tasknumber{1}%
\task{%
    Запишите формулу для ...
    \begin{itemize}
        \item релятивистского замедления времени,
        \item классической полной механической энергии тела,
        \item релятивистского импульса тела,
        \item энергии покоя тела,
        \item связь между релятивистским импульсом и релятивистской энергией.
    \end{itemize}
    Обязательно подпишите все физические величины.
}
\solutionspace{150pt}

\tasknumber{2}%
\task{%
    Протон движется со скоростью $0{,}9\,c$, где $c$~--- скорость света в вакууме.
    Каково при этом отношение кинетической энергии частицы $E_\text{кин.}$ к его энергии покоя $E_0$?
}
\answer{%
    \begin{align*}
    E &= \frac{E_0}{\sqrt{1 - \frac{v^2}{c^2}}}
            \implies \frac E{E_0}
                = \frac 1{\sqrt{1 - \frac{v^2}{c^2}}}
                = \frac 1{\sqrt{1 - \sqr{0{,}9}}}
                \approx 2{,}294,
         \\
        {E_{\text{кин}}} &= E - E_0
            \implies \frac{E_{\text{кин}}}{E_0}
                = \frac E{E_0} - 1
                = \frac 1{\sqrt{1 - \frac{v^2}{c^2}}} - 1
                = \frac 1{\sqrt{1 - \sqr{0{,}9}}} - 1
                \approx 1{,}294.
    \end{align*}
}
\solutionspace{80pt}

\tasknumber{3}%
\task{%
    Полная энергия релятивистской частицы в пять раз больше её энергии покоя.
    Найти скорость этой частицы: в долях $c$ и численное значение.
    Скорость света в вакууме $c = 3 \cdot 10^{8}\,\frac{\text{м}}{\text{с}}$.
}
\answer{%
    \begin{align*}
    E &= \frac{E_0}{\sqrt{1 - \frac{v^2}{c^2}}}\implies \sqrt{1 - \frac{v^2}{c^2}} = \frac{E_0}{E}\implies \frac{v^2}{c^2} = 1 - \sqr{\frac{E_0}{E}}\implies v = c \sqrt{1 - \sqr{\frac{E_0}{E}}} \approx 0{,}980c \approx 294 \cdot 10^{6}\,\frac{\text{м}}{\text{с}}.
    \end{align*}
}
\solutionspace{80pt}

\tasknumber{4}%
\task{%
    Кинетическая энергия релятивистской частицы в пять раз больше её энергии покоя.
    Найти скорость этой частицы.
    Скорость света в вакууме $c = 3 \cdot 10^{8}\,\frac{\text{м}}{\text{с}}$.
}
\answer{%
    \begin{align*}
    E &= E_0 + E_{\text{кин}} \\
    E &= \frac{E_0}{\sqrt{1 - \frac{v^2}{c^2}}}\implies \sqrt{1 - \frac{v^2}{c^2}} = \frac{E_0}{E}\implies \frac{v^2}{c^2} = 1 - \sqr{\frac{E_0}{E}} \implies \\
    \implies &v = c \sqrt{1 - \sqr{\frac{E_0}{E}}} = c \sqrt{1 - \sqr{\frac{E_0}{E_0 + E_{\text{кин}} }}} = c \sqrt{1 - \frac 1 {\sqr{ 1 + \frac{E_{\text{кин}}}{E_0} }} }\approx 0{,}986c \approx 296 \cdot 10^{6}\,\frac{\text{м}}{\text{с}}.
    \end{align*}
}
\solutionspace{80pt}

\tasknumber{5}%
\task{%
    Кинетическая энергия частицы космических лучей в пять раз превышает её энергию покоя.
    Определить отношение скорости частицы к скорости света.
}
\answer{%
    \begin{align*}
    E &= E_0 + E_{\text{кин}} \\
    E &= \frac{E_0}{\sqrt{1 - \frac{v^2}{c^2}}}\implies \sqrt{1 - \frac{v^2}{c^2}} = \frac{E_0}{E}\implies \frac{v^2}{c^2} = 1 - \sqr{\frac{E_0}{E}} \implies \\
    \implies \frac vc &= \sqrt{1 - \sqr{\frac{E_0}{E}}} = \sqrt{1 - \sqr{\frac{E_0}{E_0 + E_{\text{кин}} }}} \approx 0{,}986.
    \end{align*}
}
\solutionspace{80pt}

\tasknumber{6}%
\task{%
    Как (и на сколько) изменяется масса $50\,\text{т}$ воды при испарении при $100\celsius$?
}
\answer{%
    $\Delta m = \frac Q{c^2} \approx 1{,}28 \cdot 10^{-6}\,\text{кг}$
}

\variantsplitter

\addpersonalvariant{Юлия Шевченко}

\tasknumber{1}%
\task{%
    Запишите формулу для ...
    \begin{itemize}
        \item релятивистского замедления времени,
        \item классического импульса,
        \item релятивистской энергии тела,
        \item релятивистской кинетической энергии,
        \item связь между релятивистским импульсом и релятивистской энергией.
    \end{itemize}
    Обязательно подпишите все физические величины.
}
\solutionspace{150pt}

\tasknumber{2}%
\task{%
    Электрон движется со скоростью $0{,}7\,c$, где $c$~--- скорость света в вакууме.
    Каково при этом отношение полной энергии частицы $E$ к его энергии покоя $E_0$?
}
\answer{%
    \begin{align*}
    E &= \frac{E_0}{\sqrt{1 - \frac{v^2}{c^2}}}
            \implies \frac E{E_0}
                = \frac 1{\sqrt{1 - \frac{v^2}{c^2}}}
                = \frac 1{\sqrt{1 - \sqr{0{,}7}}}
                \approx 1{,}400,
         \\
        {E_{\text{кин}}} &= E - E_0
            \implies \frac{E_{\text{кин}}}{E_0}
                = \frac E{E_0} - 1
                = \frac 1{\sqrt{1 - \frac{v^2}{c^2}}} - 1
                = \frac 1{\sqrt{1 - \sqr{0{,}7}}} - 1
                \approx 0{,}400.
    \end{align*}
}
\solutionspace{80pt}

\tasknumber{3}%
\task{%
    Полная энергия релятивистской частицы в четыре раза больше её энергии покоя.
    Найти скорость этой частицы: в долях $c$ и численное значение.
    Скорость света в вакууме $c = 3 \cdot 10^{8}\,\frac{\text{м}}{\text{с}}$.
}
\answer{%
    \begin{align*}
    E &= \frac{E_0}{\sqrt{1 - \frac{v^2}{c^2}}}\implies \sqrt{1 - \frac{v^2}{c^2}} = \frac{E_0}{E}\implies \frac{v^2}{c^2} = 1 - \sqr{\frac{E_0}{E}}\implies v = c \sqrt{1 - \sqr{\frac{E_0}{E}}} \approx 0{,}968c \approx 290 \cdot 10^{6}\,\frac{\text{м}}{\text{с}}.
    \end{align*}
}
\solutionspace{80pt}

\tasknumber{4}%
\task{%
    Кинетическая энергия релятивистской частицы в четыре раза больше её энергии покоя.
    Найти скорость этой частицы.
    Скорость света в вакууме $c = 3 \cdot 10^{8}\,\frac{\text{м}}{\text{с}}$.
}
\answer{%
    \begin{align*}
    E &= E_0 + E_{\text{кин}} \\
    E &= \frac{E_0}{\sqrt{1 - \frac{v^2}{c^2}}}\implies \sqrt{1 - \frac{v^2}{c^2}} = \frac{E_0}{E}\implies \frac{v^2}{c^2} = 1 - \sqr{\frac{E_0}{E}} \implies \\
    \implies &v = c \sqrt{1 - \sqr{\frac{E_0}{E}}} = c \sqrt{1 - \sqr{\frac{E_0}{E_0 + E_{\text{кин}} }}} = c \sqrt{1 - \frac 1 {\sqr{ 1 + \frac{E_{\text{кин}}}{E_0} }} }\approx 0{,}980c \approx 294 \cdot 10^{6}\,\frac{\text{м}}{\text{с}}.
    \end{align*}
}
\solutionspace{80pt}

\tasknumber{5}%
\task{%
    Кинетическая энергия частицы космических лучей в четыре раза превышает её энергию покоя.
    Определить отношение скорости частицы к скорости света.
}
\answer{%
    \begin{align*}
    E &= E_0 + E_{\text{кин}} \\
    E &= \frac{E_0}{\sqrt{1 - \frac{v^2}{c^2}}}\implies \sqrt{1 - \frac{v^2}{c^2}} = \frac{E_0}{E}\implies \frac{v^2}{c^2} = 1 - \sqr{\frac{E_0}{E}} \implies \\
    \implies \frac vc &= \sqrt{1 - \sqr{\frac{E_0}{E}}} = \sqrt{1 - \sqr{\frac{E_0}{E_0 + E_{\text{кин}} }}} \approx 0{,}980.
    \end{align*}
}
\solutionspace{80pt}

\tasknumber{6}%
\task{%
    Как (и на сколько) изменяется масса $1\,\text{т}$ воды при испарении при $100\celsius$?
}
\answer{%
    $\Delta m = \frac Q{c^2} \approx 30 \cdot 10^{-9}\,\text{кг}$
}
% autogenerated
