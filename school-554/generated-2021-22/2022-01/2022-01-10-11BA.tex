\setdate{10~января~2022}
\setclass{11«БА»}

\addpersonalvariant{Михаил Бурмистров}

\tasknumber{1}%
\task{%
    В каком месте на главной оптической оси двояковыпуклой линзы
    нужно поместить точечный источник света,
    чтобы его изображение оказалось в главном фокусе линзы?
}
\answer{%
    $\text{для мнимого - на половине фокусного, для действительного - на бесконечности}$
}
\solutionspace{120pt}

\tasknumber{2}%
\task{%
    На экране, расположенном иа расстоянии $80\,\text{см}$ от собирающей линзы,
    получено изображение точечного источника, расположенного на главной оптической оси линзы.
    На какое расстояние переместится изображение на экране,
    если при неподвижном источнике переместить линзу на $3\,\text{см}$ в плоскости, перпендикулярной главной оптической оси?
    Фокусное расстояние линзы равно $20\,\text{см}$.
}
\answer{%
    \begin{align*}
    &\frac 1F = \frac 1a + \frac 1b \implies a = \frac{bF}{b-F} \implies \Gamma = \frac ba = \frac{b-F}F \\
    &y = x \cdot \Gamma = x \cdot \frac{b-F}F \implies d = x + y = 12\,\text{см}.
    \end{align*}
}
\solutionspace{120pt}

\tasknumber{3}%
\task{%
    Оптическая сила двояковыпуклой линзы в воздухе $5\,\text{дптр}$, а в воде $1{,}4\,\text{дптр}$.
    Определить показатель преломления $n$ материала, из которого изготовлена линза.
}
\answer{%
    \begin{align*}
    D_1 &=\cbr{\frac n{n_1} - 1}\cbr{\frac 1{R_1} + \frac 1{R_2}}, \\
    D_2 &=\cbr{\frac n{n_2} - 1}\cbr{\frac 1{R_1} + \frac 1{R_2}}, \\
    \frac {D_2}{D_1} &=\frac{\frac n{n_2} - 1}{\frac n{n_1} - 1} \implies {D_2}\cbr{\frac n{n_1} - 1} = {D_1}\cbr{\frac n{n_2} - 1}  \implies n\cbr{\frac{D_2}{n_1} - \frac{D_1}{n_2}} = D_2 - D_1, \\
    n &= \frac{D_2 - D_1}{\frac{D_2}{n_1} - \frac{D_1}{n_2}} = \frac{n_1 n_2 (D_2 - D_1)}{D_2n_2 - D_1n_1} \approx 1{,}526.
    \end{align*}
}
\solutionspace{120pt}

\tasknumber{4}%
\task{%
    На каком расстоянии от собирающей линзы с фокусным расстоянием $40\,\text{дптр}$
    следует надо поместить предмет, чтобы расстояние
    от предмета до его действительного изображения было наименьшим?
}
\answer{%
    \begin{align*}
    \frac 1a &+ \frac 1b = D \implies b = \frac 1{D - \frac 1a} \implies \ell = a + b = a + \frac a{Da - 1} = \frac{ Da^2 }{Da - 1} \implies \\
    \implies \ell'_a &= \frac{ 2Da \cdot (Da - 1) - Da^2 \cdot D }{\sqr{Da - 1}}= \frac{ D^2a^2 - 2Da}{\sqr{Da - 1}} = \frac{ Da(Da - 2)}{\sqr{Da - 1}}\implies a_{\min} = \frac 2D \approx 50\,\text{мм}.
    \end{align*}
}
\solutionspace{120pt}

\tasknumber{5}%
\task{%
    Предмет в виде отрезка длиной $\ell$ расположен вдоль оптической оси
    собирающей линзы с фокусным расстоянием $F$.
    Середина отрезка расположена
    иа расстоянии $a$ от линзы, которая даёт действительное изображение
    всех точек предмета.
    Определить продольное увеличение предмета.
}
\answer{%
    \begin{align*}
    \frac 1{a + \frac \ell 2} &+ \frac 1b = \frac 1F \implies b = \frac{F\cbr{a + \frac \ell 2}}{a + \frac \ell 2 - F} \\
    \frac 1{a - \frac \ell 2} &+ \frac 1c = \frac 1F \implies c = \frac{F\cbr{a - \frac \ell 2}}{a - \frac \ell 2 - F} \\
    \abs{b - c} &= \abs{\frac{F\cbr{a + \frac \ell 2}}{a + \frac \ell 2 - F} - \frac{F\cbr{a - \frac \ell 2}}{a - \frac \ell 2 - F}}= F\abs{\frac{\cbr{a + \frac \ell 2}\cbr{a - \frac \ell 2 - F} - \cbr{a - \frac \ell 2}\cbr{a + \frac \ell 2 - F}}{ \cbr{a + \frac \ell 2 - F} \cbr{a - \frac \ell 2 - F} }} =  \\
    &= F\abs{\frac{a^2 - \frac {a\ell} 2 - Fa + \frac {a\ell} 2 - \frac {\ell^2} 4 - \frac {F\ell}2 - a^2 - \frac {a\ell}2 + aF + \frac {a\ell}2 + \frac {\ell^2} 4 - \frac {F\ell} 2}{\cbr{a + \frac \ell 2 - F} \cbr{a - \frac \ell 2 - F} }} = \\
    &= F\frac{F\ell}{\sqr{a-F} - \frac {\ell^2}4} = \frac{F^2\ell}{\sqr{a-F} - \frac {\ell^2}4}\implies \Gamma = \frac{\abs{b - c}}\ell = \frac{F^2}{\sqr{a-F} - \frac {\ell^2}4}.
    \end{align*}
}
\solutionspace{120pt}

\tasknumber{6}%
\task{%
    Даны точечный источник света $S$, его изображение $S_1$, полученное с помошью собирающей линзы,
    и ближайший к источнику фокус линзы $F$ (см.
    рис.
    на доске).
    Расстояния $SF = \ell$ и $SS_1 = L$.
    Определить положение линзы и её фокусное расстояние.
}
\answer{%
    \begin{align*}
    \frac 1a + \frac 1b &= \frac 1F, \ell = a - F, L = a + b \implies a = \ell + F, b = L - a = L - \ell - F \\
    \frac 1{\ell + F} + \frac 1{L - \ell - F} &= \frac 1F \\
    F\ell + F^2 + LF - F\ell - F^2 &= L\ell - \ell^2 - F\ell + LF - F\ell - F^2 \\
    0 &= L\ell - \ell^2 - 2F\ell - F^2 \\
    0 &=  F^2 + 2F\ell - L\ell + \ell^2 \\
    F &= -\ell \pm \sqrt{\ell^2 +  L\ell - \ell^2} = -\ell \pm \sqrt{L\ell} \implies F = \sqrt{L\ell} - \ell \\
    a &= \ell + F = \ell + \sqrt{L\ell} - \ell = \sqrt{L\ell}.
    \end{align*}
}
\solutionspace{120pt}

\tasknumber{7}%
\task{%
    Расстояние от освещённого предмета до экрана $80\,\text{см}$.
    Линза, помещенная между ними, даёт чёткое изображение предмета на
    экране при двух положениях, расстояние между которыми $20\,\text{см}$.
    Найти фокусное расстояние линзы.
}
\answer{%
    \begin{align*}
    \frac 1a + \frac 1b &= \frac 1F, \frac 1{a-\ell} + \frac 1{b+\ell} = \frac 1F, a + b = L \\
    \frac 1a + \frac 1b &= \frac 1{a-\ell} + \frac 1{b+\ell}\implies \frac{a + b}{ab} = \frac{(a-\ell) + (b+\ell)}{(a-\ell)(b+\ell)} \\
    ab  &= (a - \ell)(b+\ell) \implies 0  = -b\ell + a\ell - \ell^2 \implies 0 = -b + a - \ell \implies b = a - \ell \\
    a + (a - \ell) &= L \implies a = \frac{L + \ell}2 \implies b = \frac{L - \ell}2 \\
    F &= \frac{ab}{a + b} = \frac{L^2 -\ell^2}{4L} \approx 18{,}8\,\text{см}.
    \end{align*}
}
\solutionspace{120pt}

\tasknumber{8}%
\task{%
    Предмет находится на расстоянии $70\,\text{см}$ от экрана.
    Между предметом и экраном помещают линзу, причём при одном
    положении линзы на экране получается увеличенное изображение предмета,
    а при другом — уменьшенное.
    Каково фокусное расстояние линзы, если
    линейные размеры первого изображения в три раза больше второго?
}
\answer{%
    \begin{align*}
    \frac 1a + \frac 1{L-a} &= \frac 1F, h_1 = h \cdot \frac{L-a}a, \\
    \frac 1b + \frac 1{L-b} &= \frac 1F, h_2 = h \cdot \frac{L-b}b, \\
    \frac{h_1}{h_2} &= 3 \implies \frac{(L-a)b}{(L-b)a} = 3, \\
    \frac 1F &= \frac{ L }{a(L-a)} = \frac{ L }{b(L-b)} \implies \frac{L-a}{L-b} = \frac b a \implies \frac {b^2}{a^2} = 3.
    \\
    \frac 1a + \frac 1{L-a} &= \frac 1b + \frac 1{L-b} \implies \frac L{a(L-a)} = \frac L{b(L-b)} \implies \\
    \implies aL - a^2 &= bL - b^2 \implies (a-b)L = (a-b)(a+b) \implies b = L - a, \\
    \frac{\sqr{L-a}}{a^2} &= 3 \implies \frac La - 1 = \sqrt{3} \implies a = \frac{ L }{\sqrt{3} + 1} \\
    F &= \frac{a(L-a)}L = \frac 1L \cdot \frac L{\sqrt{3} + 1} \cdot \frac {L\sqrt{3}}{\sqrt{3} + 1}= \frac { L\sqrt{3} }{ \sqr{\sqrt{3} + 1} } \approx 16{,}2\,\text{см}.
    \end{align*}
}

\variantsplitter

\addpersonalvariant{Ирина Ан}

\tasknumber{1}%
\task{%
    В каком месте на главной оптической оси двояковыгнутой линзы
    нужно поместить точечный источник света,
    чтобы его изображение оказалось в главном фокусе линзы?
}
\answer{%
    $\text{на половине фокусного расстояния}$
}
\solutionspace{120pt}

\tasknumber{2}%
\task{%
    На экране, расположенном иа расстоянии $80\,\text{см}$ от собирающей линзы,
    получено изображение точечного источника, расположенного на главной оптической оси линзы.
    На какое расстояние переместится изображение на экране,
    если при неподвижной линзе переместить источник на $1\,\text{см}$ в плоскости, перпендикулярной главной оптической оси?
    Фокусное расстояние линзы равно $30\,\text{см}$.
}
\answer{%
    \begin{align*}
    &\frac 1F = \frac 1a + \frac 1b \implies a = \frac{bF}{b-F} \implies \Gamma = \frac ba = \frac{b-F}F \\
    &y = x \cdot \Gamma = x \cdot \frac{b-F}F \implies d = y = 1{,}67\,\text{см}.
    \end{align*}
}
\solutionspace{120pt}

\tasknumber{3}%
\task{%
    Оптическая сила двояковыпуклой линзы в воздухе $5\,\text{дптр}$, а в воде $1{,}4\,\text{дптр}$.
    Определить показатель преломления $n$ материала, из которого изготовлена линза.
}
\answer{%
    \begin{align*}
    D_1 &=\cbr{\frac n{n_1} - 1}\cbr{\frac 1{R_1} + \frac 1{R_2}}, \\
    D_2 &=\cbr{\frac n{n_2} - 1}\cbr{\frac 1{R_1} + \frac 1{R_2}}, \\
    \frac {D_2}{D_1} &=\frac{\frac n{n_2} - 1}{\frac n{n_1} - 1} \implies {D_2}\cbr{\frac n{n_1} - 1} = {D_1}\cbr{\frac n{n_2} - 1}  \implies n\cbr{\frac{D_2}{n_1} - \frac{D_1}{n_2}} = D_2 - D_1, \\
    n &= \frac{D_2 - D_1}{\frac{D_2}{n_1} - \frac{D_1}{n_2}} = \frac{n_1 n_2 (D_2 - D_1)}{D_2n_2 - D_1n_1} \approx 1{,}526.
    \end{align*}
}
\solutionspace{120pt}

\tasknumber{4}%
\task{%
    На каком расстоянии от собирающей линзы с фокусным расстоянием $30\,\text{дптр}$
    следует надо поместить предмет, чтобы расстояние
    от предмета до его действительного изображения было наименьшим?
}
\answer{%
    \begin{align*}
    \frac 1a &+ \frac 1b = D \implies b = \frac 1{D - \frac 1a} \implies \ell = a + b = a + \frac a{Da - 1} = \frac{ Da^2 }{Da - 1} \implies \\
    \implies \ell'_a &= \frac{ 2Da \cdot (Da - 1) - Da^2 \cdot D }{\sqr{Da - 1}}= \frac{ D^2a^2 - 2Da}{\sqr{Da - 1}} = \frac{ Da(Da - 2)}{\sqr{Da - 1}}\implies a_{\min} = \frac 2D \approx 66{,}7\,\text{мм}.
    \end{align*}
}
\solutionspace{120pt}

\tasknumber{5}%
\task{%
    Предмет в виде отрезка длиной $\ell$ расположен вдоль оптической оси
    собирающей линзы с фокусным расстоянием $F$.
    Середина отрезка расположена
    иа расстоянии $a$ от линзы, которая даёт действительное изображение
    всех точек предмета.
    Определить продольное увеличение предмета.
}
\answer{%
    \begin{align*}
    \frac 1{a + \frac \ell 2} &+ \frac 1b = \frac 1F \implies b = \frac{F\cbr{a + \frac \ell 2}}{a + \frac \ell 2 - F} \\
    \frac 1{a - \frac \ell 2} &+ \frac 1c = \frac 1F \implies c = \frac{F\cbr{a - \frac \ell 2}}{a - \frac \ell 2 - F} \\
    \abs{b - c} &= \abs{\frac{F\cbr{a + \frac \ell 2}}{a + \frac \ell 2 - F} - \frac{F\cbr{a - \frac \ell 2}}{a - \frac \ell 2 - F}}= F\abs{\frac{\cbr{a + \frac \ell 2}\cbr{a - \frac \ell 2 - F} - \cbr{a - \frac \ell 2}\cbr{a + \frac \ell 2 - F}}{ \cbr{a + \frac \ell 2 - F} \cbr{a - \frac \ell 2 - F} }} =  \\
    &= F\abs{\frac{a^2 - \frac {a\ell} 2 - Fa + \frac {a\ell} 2 - \frac {\ell^2} 4 - \frac {F\ell}2 - a^2 - \frac {a\ell}2 + aF + \frac {a\ell}2 + \frac {\ell^2} 4 - \frac {F\ell} 2}{\cbr{a + \frac \ell 2 - F} \cbr{a - \frac \ell 2 - F} }} = \\
    &= F\frac{F\ell}{\sqr{a-F} - \frac {\ell^2}4} = \frac{F^2\ell}{\sqr{a-F} - \frac {\ell^2}4}\implies \Gamma = \frac{\abs{b - c}}\ell = \frac{F^2}{\sqr{a-F} - \frac {\ell^2}4}.
    \end{align*}
}
\solutionspace{120pt}

\tasknumber{6}%
\task{%
    Даны точечный источник света $S$, его изображение $S_1$, полученное с помошью собирающей линзы,
    и ближайший к источнику фокус линзы $F$ (см.
    рис.
    на доске).
    Расстояния $SF = \ell$ и $SS_1 = L$.
    Определить положение линзы и её фокусное расстояние.
}
\answer{%
    \begin{align*}
    \frac 1a + \frac 1b &= \frac 1F, \ell = a - F, L = a + b \implies a = \ell + F, b = L - a = L - \ell - F \\
    \frac 1{\ell + F} + \frac 1{L - \ell - F} &= \frac 1F \\
    F\ell + F^2 + LF - F\ell - F^2 &= L\ell - \ell^2 - F\ell + LF - F\ell - F^2 \\
    0 &= L\ell - \ell^2 - 2F\ell - F^2 \\
    0 &=  F^2 + 2F\ell - L\ell + \ell^2 \\
    F &= -\ell \pm \sqrt{\ell^2 +  L\ell - \ell^2} = -\ell \pm \sqrt{L\ell} \implies F = \sqrt{L\ell} - \ell \\
    a &= \ell + F = \ell + \sqrt{L\ell} - \ell = \sqrt{L\ell}.
    \end{align*}
}
\solutionspace{120pt}

\tasknumber{7}%
\task{%
    Расстояние от освещённого предмета до экрана $80\,\text{см}$.
    Линза, помещенная между ними, даёт чёткое изображение предмета на
    экране при двух положениях, расстояние между которыми $30\,\text{см}$.
    Найти фокусное расстояние линзы.
}
\answer{%
    \begin{align*}
    \frac 1a + \frac 1b &= \frac 1F, \frac 1{a-\ell} + \frac 1{b+\ell} = \frac 1F, a + b = L \\
    \frac 1a + \frac 1b &= \frac 1{a-\ell} + \frac 1{b+\ell}\implies \frac{a + b}{ab} = \frac{(a-\ell) + (b+\ell)}{(a-\ell)(b+\ell)} \\
    ab  &= (a - \ell)(b+\ell) \implies 0  = -b\ell + a\ell - \ell^2 \implies 0 = -b + a - \ell \implies b = a - \ell \\
    a + (a - \ell) &= L \implies a = \frac{L + \ell}2 \implies b = \frac{L - \ell}2 \\
    F &= \frac{ab}{a + b} = \frac{L^2 -\ell^2}{4L} \approx 17{,}2\,\text{см}.
    \end{align*}
}
\solutionspace{120pt}

\tasknumber{8}%
\task{%
    Предмет находится на расстоянии $80\,\text{см}$ от экрана.
    Между предметом и экраном помещают линзу, причём при одном
    положении линзы на экране получается увеличенное изображение предмета,
    а при другом — уменьшенное.
    Каково фокусное расстояние линзы, если
    линейные размеры первого изображения в пять раз больше второго?
}
\answer{%
    \begin{align*}
    \frac 1a + \frac 1{L-a} &= \frac 1F, h_1 = h \cdot \frac{L-a}a, \\
    \frac 1b + \frac 1{L-b} &= \frac 1F, h_2 = h \cdot \frac{L-b}b, \\
    \frac{h_1}{h_2} &= 5 \implies \frac{(L-a)b}{(L-b)a} = 5, \\
    \frac 1F &= \frac{ L }{a(L-a)} = \frac{ L }{b(L-b)} \implies \frac{L-a}{L-b} = \frac b a \implies \frac {b^2}{a^2} = 5.
    \\
    \frac 1a + \frac 1{L-a} &= \frac 1b + \frac 1{L-b} \implies \frac L{a(L-a)} = \frac L{b(L-b)} \implies \\
    \implies aL - a^2 &= bL - b^2 \implies (a-b)L = (a-b)(a+b) \implies b = L - a, \\
    \frac{\sqr{L-a}}{a^2} &= 5 \implies \frac La - 1 = \sqrt{5} \implies a = \frac{ L }{\sqrt{5} + 1} \\
    F &= \frac{a(L-a)}L = \frac 1L \cdot \frac L{\sqrt{5} + 1} \cdot \frac {L\sqrt{5}}{\sqrt{5} + 1}= \frac { L\sqrt{5} }{ \sqr{\sqrt{5} + 1} } \approx 17{,}1\,\text{см}.
    \end{align*}
}

\variantsplitter

\addpersonalvariant{Софья Андрианова}

\tasknumber{1}%
\task{%
    В каком месте на главной оптической оси двояковыгнутой линзы
    нужно поместить точечный источник света,
    чтобы его изображение оказалось в главном фокусе линзы?
}
\answer{%
    $\text{на половине фокусного расстояния}$
}
\solutionspace{120pt}

\tasknumber{2}%
\task{%
    На экране, расположенном иа расстоянии $80\,\text{см}$ от собирающей линзы,
    получено изображение точечного источника, расположенного на главной оптической оси линзы.
    На какое расстояние переместится изображение на экране,
    если при неподвижном источнике переместить линзу на $1\,\text{см}$ в плоскости, перпендикулярной главной оптической оси?
    Фокусное расстояние линзы равно $30\,\text{см}$.
}
\answer{%
    \begin{align*}
    &\frac 1F = \frac 1a + \frac 1b \implies a = \frac{bF}{b-F} \implies \Gamma = \frac ba = \frac{b-F}F \\
    &y = x \cdot \Gamma = x \cdot \frac{b-F}F \implies d = x + y = 2{,}7\,\text{см}.
    \end{align*}
}
\solutionspace{120pt}

\tasknumber{3}%
\task{%
    Оптическая сила двояковыпуклой линзы в воздухе $5{,}5\,\text{дптр}$, а в воде $1{,}5\,\text{дптр}$.
    Определить показатель преломления $n$ материала, из которого изготовлена линза.
}
\answer{%
    \begin{align*}
    D_1 &=\cbr{\frac n{n_1} - 1}\cbr{\frac 1{R_1} + \frac 1{R_2}}, \\
    D_2 &=\cbr{\frac n{n_2} - 1}\cbr{\frac 1{R_1} + \frac 1{R_2}}, \\
    \frac {D_2}{D_1} &=\frac{\frac n{n_2} - 1}{\frac n{n_1} - 1} \implies {D_2}\cbr{\frac n{n_1} - 1} = {D_1}\cbr{\frac n{n_2} - 1}  \implies n\cbr{\frac{D_2}{n_1} - \frac{D_1}{n_2}} = D_2 - D_1, \\
    n &= \frac{D_2 - D_1}{\frac{D_2}{n_1} - \frac{D_1}{n_2}} = \frac{n_1 n_2 (D_2 - D_1)}{D_2n_2 - D_1n_1} \approx 1{,}518.
    \end{align*}
}
\solutionspace{120pt}

\tasknumber{4}%
\task{%
    На каком расстоянии от собирающей линзы с фокусным расстоянием $30\,\text{дптр}$
    следует надо поместить предмет, чтобы расстояние
    от предмета до его действительного изображения было наименьшим?
}
\answer{%
    \begin{align*}
    \frac 1a &+ \frac 1b = D \implies b = \frac 1{D - \frac 1a} \implies \ell = a + b = a + \frac a{Da - 1} = \frac{ Da^2 }{Da - 1} \implies \\
    \implies \ell'_a &= \frac{ 2Da \cdot (Da - 1) - Da^2 \cdot D }{\sqr{Da - 1}}= \frac{ D^2a^2 - 2Da}{\sqr{Da - 1}} = \frac{ Da(Da - 2)}{\sqr{Da - 1}}\implies a_{\min} = \frac 2D \approx 66{,}7\,\text{мм}.
    \end{align*}
}
\solutionspace{120pt}

\tasknumber{5}%
\task{%
    Предмет в виде отрезка длиной $\ell$ расположен вдоль оптической оси
    собирающей линзы с фокусным расстоянием $F$.
    Середина отрезка расположена
    иа расстоянии $a$ от линзы, которая даёт действительное изображение
    всех точек предмета.
    Определить продольное увеличение предмета.
}
\answer{%
    \begin{align*}
    \frac 1{a + \frac \ell 2} &+ \frac 1b = \frac 1F \implies b = \frac{F\cbr{a + \frac \ell 2}}{a + \frac \ell 2 - F} \\
    \frac 1{a - \frac \ell 2} &+ \frac 1c = \frac 1F \implies c = \frac{F\cbr{a - \frac \ell 2}}{a - \frac \ell 2 - F} \\
    \abs{b - c} &= \abs{\frac{F\cbr{a + \frac \ell 2}}{a + \frac \ell 2 - F} - \frac{F\cbr{a - \frac \ell 2}}{a - \frac \ell 2 - F}}= F\abs{\frac{\cbr{a + \frac \ell 2}\cbr{a - \frac \ell 2 - F} - \cbr{a - \frac \ell 2}\cbr{a + \frac \ell 2 - F}}{ \cbr{a + \frac \ell 2 - F} \cbr{a - \frac \ell 2 - F} }} =  \\
    &= F\abs{\frac{a^2 - \frac {a\ell} 2 - Fa + \frac {a\ell} 2 - \frac {\ell^2} 4 - \frac {F\ell}2 - a^2 - \frac {a\ell}2 + aF + \frac {a\ell}2 + \frac {\ell^2} 4 - \frac {F\ell} 2}{\cbr{a + \frac \ell 2 - F} \cbr{a - \frac \ell 2 - F} }} = \\
    &= F\frac{F\ell}{\sqr{a-F} - \frac {\ell^2}4} = \frac{F^2\ell}{\sqr{a-F} - \frac {\ell^2}4}\implies \Gamma = \frac{\abs{b - c}}\ell = \frac{F^2}{\sqr{a-F} - \frac {\ell^2}4}.
    \end{align*}
}
\solutionspace{120pt}

\tasknumber{6}%
\task{%
    Даны точечный источник света $S$, его изображение $S_1$, полученное с помошью собирающей линзы,
    и ближайший к источнику фокус линзы $F$ (см.
    рис.
    на доске).
    Расстояния $SF = \ell$ и $SS_1 = L$.
    Определить положение линзы и её фокусное расстояние.
}
\answer{%
    \begin{align*}
    \frac 1a + \frac 1b &= \frac 1F, \ell = a - F, L = a + b \implies a = \ell + F, b = L - a = L - \ell - F \\
    \frac 1{\ell + F} + \frac 1{L - \ell - F} &= \frac 1F \\
    F\ell + F^2 + LF - F\ell - F^2 &= L\ell - \ell^2 - F\ell + LF - F\ell - F^2 \\
    0 &= L\ell - \ell^2 - 2F\ell - F^2 \\
    0 &=  F^2 + 2F\ell - L\ell + \ell^2 \\
    F &= -\ell \pm \sqrt{\ell^2 +  L\ell - \ell^2} = -\ell \pm \sqrt{L\ell} \implies F = \sqrt{L\ell} - \ell \\
    a &= \ell + F = \ell + \sqrt{L\ell} - \ell = \sqrt{L\ell}.
    \end{align*}
}
\solutionspace{120pt}

\tasknumber{7}%
\task{%
    Расстояние от освещённого предмета до экрана $80\,\text{см}$.
    Линза, помещенная между ними, даёт чёткое изображение предмета на
    экране при двух положениях, расстояние между которыми $20\,\text{см}$.
    Найти фокусное расстояние линзы.
}
\answer{%
    \begin{align*}
    \frac 1a + \frac 1b &= \frac 1F, \frac 1{a-\ell} + \frac 1{b+\ell} = \frac 1F, a + b = L \\
    \frac 1a + \frac 1b &= \frac 1{a-\ell} + \frac 1{b+\ell}\implies \frac{a + b}{ab} = \frac{(a-\ell) + (b+\ell)}{(a-\ell)(b+\ell)} \\
    ab  &= (a - \ell)(b+\ell) \implies 0  = -b\ell + a\ell - \ell^2 \implies 0 = -b + a - \ell \implies b = a - \ell \\
    a + (a - \ell) &= L \implies a = \frac{L + \ell}2 \implies b = \frac{L - \ell}2 \\
    F &= \frac{ab}{a + b} = \frac{L^2 -\ell^2}{4L} \approx 18{,}8\,\text{см}.
    \end{align*}
}
\solutionspace{120pt}

\tasknumber{8}%
\task{%
    Предмет находится на расстоянии $90\,\text{см}$ от экрана.
    Между предметом и экраном помещают линзу, причём при одном
    положении линзы на экране получается увеличенное изображение предмета,
    а при другом — уменьшенное.
    Каково фокусное расстояние линзы, если
    линейные размеры первого изображения в три раза больше второго?
}
\answer{%
    \begin{align*}
    \frac 1a + \frac 1{L-a} &= \frac 1F, h_1 = h \cdot \frac{L-a}a, \\
    \frac 1b + \frac 1{L-b} &= \frac 1F, h_2 = h \cdot \frac{L-b}b, \\
    \frac{h_1}{h_2} &= 3 \implies \frac{(L-a)b}{(L-b)a} = 3, \\
    \frac 1F &= \frac{ L }{a(L-a)} = \frac{ L }{b(L-b)} \implies \frac{L-a}{L-b} = \frac b a \implies \frac {b^2}{a^2} = 3.
    \\
    \frac 1a + \frac 1{L-a} &= \frac 1b + \frac 1{L-b} \implies \frac L{a(L-a)} = \frac L{b(L-b)} \implies \\
    \implies aL - a^2 &= bL - b^2 \implies (a-b)L = (a-b)(a+b) \implies b = L - a, \\
    \frac{\sqr{L-a}}{a^2} &= 3 \implies \frac La - 1 = \sqrt{3} \implies a = \frac{ L }{\sqrt{3} + 1} \\
    F &= \frac{a(L-a)}L = \frac 1L \cdot \frac L{\sqrt{3} + 1} \cdot \frac {L\sqrt{3}}{\sqrt{3} + 1}= \frac { L\sqrt{3} }{ \sqr{\sqrt{3} + 1} } \approx 21\,\text{см}.
    \end{align*}
}

\variantsplitter

\addpersonalvariant{Владимир Артемчук}

\tasknumber{1}%
\task{%
    В каком месте на главной оптической оси двояковыгнутой линзы
    нужно поместить точечный источник света,
    чтобы его изображение оказалось в главном фокусе линзы?
}
\answer{%
    $\text{на половине фокусного расстояния}$
}
\solutionspace{120pt}

\tasknumber{2}%
\task{%
    На экране, расположенном иа расстоянии $120\,\text{см}$ от собирающей линзы,
    получено изображение точечного источника, расположенного на главной оптической оси линзы.
    На какое расстояние переместится изображение на экране,
    если при неподвижном источнике переместить линзу на $2\,\text{см}$ в плоскости, перпендикулярной главной оптической оси?
    Фокусное расстояние линзы равно $20\,\text{см}$.
}
\answer{%
    \begin{align*}
    &\frac 1F = \frac 1a + \frac 1b \implies a = \frac{bF}{b-F} \implies \Gamma = \frac ba = \frac{b-F}F \\
    &y = x \cdot \Gamma = x \cdot \frac{b-F}F \implies d = x + y = 12\,\text{см}.
    \end{align*}
}
\solutionspace{120pt}

\tasknumber{3}%
\task{%
    Оптическая сила двояковыпуклой линзы в воздухе $5{,}5\,\text{дптр}$, а в воде $1{,}6\,\text{дптр}$.
    Определить показатель преломления $n$ материала, из которого изготовлена линза.
}
\answer{%
    \begin{align*}
    D_1 &=\cbr{\frac n{n_1} - 1}\cbr{\frac 1{R_1} + \frac 1{R_2}}, \\
    D_2 &=\cbr{\frac n{n_2} - 1}\cbr{\frac 1{R_1} + \frac 1{R_2}}, \\
    \frac {D_2}{D_1} &=\frac{\frac n{n_2} - 1}{\frac n{n_1} - 1} \implies {D_2}\cbr{\frac n{n_1} - 1} = {D_1}\cbr{\frac n{n_2} - 1}  \implies n\cbr{\frac{D_2}{n_1} - \frac{D_1}{n_2}} = D_2 - D_1, \\
    n &= \frac{D_2 - D_1}{\frac{D_2}{n_1} - \frac{D_1}{n_2}} = \frac{n_1 n_2 (D_2 - D_1)}{D_2n_2 - D_1n_1} \approx 1{,}538.
    \end{align*}
}
\solutionspace{120pt}

\tasknumber{4}%
\task{%
    На каком расстоянии от собирающей линзы с фокусным расстоянием $50\,\text{дптр}$
    следует надо поместить предмет, чтобы расстояние
    от предмета до его действительного изображения было наименьшим?
}
\answer{%
    \begin{align*}
    \frac 1a &+ \frac 1b = D \implies b = \frac 1{D - \frac 1a} \implies \ell = a + b = a + \frac a{Da - 1} = \frac{ Da^2 }{Da - 1} \implies \\
    \implies \ell'_a &= \frac{ 2Da \cdot (Da - 1) - Da^2 \cdot D }{\sqr{Da - 1}}= \frac{ D^2a^2 - 2Da}{\sqr{Da - 1}} = \frac{ Da(Da - 2)}{\sqr{Da - 1}}\implies a_{\min} = \frac 2D \approx 40\,\text{мм}.
    \end{align*}
}
\solutionspace{120pt}

\tasknumber{5}%
\task{%
    Предмет в виде отрезка длиной $\ell$ расположен вдоль оптической оси
    собирающей линзы с фокусным расстоянием $F$.
    Середина отрезка расположена
    иа расстоянии $a$ от линзы, которая даёт действительное изображение
    всех точек предмета.
    Определить продольное увеличение предмета.
}
\answer{%
    \begin{align*}
    \frac 1{a + \frac \ell 2} &+ \frac 1b = \frac 1F \implies b = \frac{F\cbr{a + \frac \ell 2}}{a + \frac \ell 2 - F} \\
    \frac 1{a - \frac \ell 2} &+ \frac 1c = \frac 1F \implies c = \frac{F\cbr{a - \frac \ell 2}}{a - \frac \ell 2 - F} \\
    \abs{b - c} &= \abs{\frac{F\cbr{a + \frac \ell 2}}{a + \frac \ell 2 - F} - \frac{F\cbr{a - \frac \ell 2}}{a - \frac \ell 2 - F}}= F\abs{\frac{\cbr{a + \frac \ell 2}\cbr{a - \frac \ell 2 - F} - \cbr{a - \frac \ell 2}\cbr{a + \frac \ell 2 - F}}{ \cbr{a + \frac \ell 2 - F} \cbr{a - \frac \ell 2 - F} }} =  \\
    &= F\abs{\frac{a^2 - \frac {a\ell} 2 - Fa + \frac {a\ell} 2 - \frac {\ell^2} 4 - \frac {F\ell}2 - a^2 - \frac {a\ell}2 + aF + \frac {a\ell}2 + \frac {\ell^2} 4 - \frac {F\ell} 2}{\cbr{a + \frac \ell 2 - F} \cbr{a - \frac \ell 2 - F} }} = \\
    &= F\frac{F\ell}{\sqr{a-F} - \frac {\ell^2}4} = \frac{F^2\ell}{\sqr{a-F} - \frac {\ell^2}4}\implies \Gamma = \frac{\abs{b - c}}\ell = \frac{F^2}{\sqr{a-F} - \frac {\ell^2}4}.
    \end{align*}
}
\solutionspace{120pt}

\tasknumber{6}%
\task{%
    Даны точечный источник света $S$, его изображение $S_1$, полученное с помошью собирающей линзы,
    и ближайший к источнику фокус линзы $F$ (см.
    рис.
    на доске).
    Расстояния $SF = \ell$ и $SS_1 = L$.
    Определить положение линзы и её фокусное расстояние.
}
\answer{%
    \begin{align*}
    \frac 1a + \frac 1b &= \frac 1F, \ell = a - F, L = a + b \implies a = \ell + F, b = L - a = L - \ell - F \\
    \frac 1{\ell + F} + \frac 1{L - \ell - F} &= \frac 1F \\
    F\ell + F^2 + LF - F\ell - F^2 &= L\ell - \ell^2 - F\ell + LF - F\ell - F^2 \\
    0 &= L\ell - \ell^2 - 2F\ell - F^2 \\
    0 &=  F^2 + 2F\ell - L\ell + \ell^2 \\
    F &= -\ell \pm \sqrt{\ell^2 +  L\ell - \ell^2} = -\ell \pm \sqrt{L\ell} \implies F = \sqrt{L\ell} - \ell \\
    a &= \ell + F = \ell + \sqrt{L\ell} - \ell = \sqrt{L\ell}.
    \end{align*}
}
\solutionspace{120pt}

\tasknumber{7}%
\task{%
    Расстояние от освещённого предмета до экрана $100\,\text{см}$.
    Линза, помещенная между ними, даёт чёткое изображение предмета на
    экране при двух положениях, расстояние между которыми $40\,\text{см}$.
    Найти фокусное расстояние линзы.
}
\answer{%
    \begin{align*}
    \frac 1a + \frac 1b &= \frac 1F, \frac 1{a-\ell} + \frac 1{b+\ell} = \frac 1F, a + b = L \\
    \frac 1a + \frac 1b &= \frac 1{a-\ell} + \frac 1{b+\ell}\implies \frac{a + b}{ab} = \frac{(a-\ell) + (b+\ell)}{(a-\ell)(b+\ell)} \\
    ab  &= (a - \ell)(b+\ell) \implies 0  = -b\ell + a\ell - \ell^2 \implies 0 = -b + a - \ell \implies b = a - \ell \\
    a + (a - \ell) &= L \implies a = \frac{L + \ell}2 \implies b = \frac{L - \ell}2 \\
    F &= \frac{ab}{a + b} = \frac{L^2 -\ell^2}{4L} \approx 21\,\text{см}.
    \end{align*}
}
\solutionspace{120pt}

\tasknumber{8}%
\task{%
    Предмет находится на расстоянии $80\,\text{см}$ от экрана.
    Между предметом и экраном помещают линзу, причём при одном
    положении линзы на экране получается увеличенное изображение предмета,
    а при другом — уменьшенное.
    Каково фокусное расстояние линзы, если
    линейные размеры первого изображения в два раза больше второго?
}
\answer{%
    \begin{align*}
    \frac 1a + \frac 1{L-a} &= \frac 1F, h_1 = h \cdot \frac{L-a}a, \\
    \frac 1b + \frac 1{L-b} &= \frac 1F, h_2 = h \cdot \frac{L-b}b, \\
    \frac{h_1}{h_2} &= 2 \implies \frac{(L-a)b}{(L-b)a} = 2, \\
    \frac 1F &= \frac{ L }{a(L-a)} = \frac{ L }{b(L-b)} \implies \frac{L-a}{L-b} = \frac b a \implies \frac {b^2}{a^2} = 2.
    \\
    \frac 1a + \frac 1{L-a} &= \frac 1b + \frac 1{L-b} \implies \frac L{a(L-a)} = \frac L{b(L-b)} \implies \\
    \implies aL - a^2 &= bL - b^2 \implies (a-b)L = (a-b)(a+b) \implies b = L - a, \\
    \frac{\sqr{L-a}}{a^2} &= 2 \implies \frac La - 1 = \sqrt{2} \implies a = \frac{ L }{\sqrt{2} + 1} \\
    F &= \frac{a(L-a)}L = \frac 1L \cdot \frac L{\sqrt{2} + 1} \cdot \frac {L\sqrt{2}}{\sqrt{2} + 1}= \frac { L\sqrt{2} }{ \sqr{\sqrt{2} + 1} } \approx 19{,}4\,\text{см}.
    \end{align*}
}

\variantsplitter

\addpersonalvariant{Софья Белянкина}

\tasknumber{1}%
\task{%
    В каком месте на главной оптической оси двояковыпуклой линзы
    нужно поместить точечный источник света,
    чтобы его изображение оказалось в главном фокусе линзы?
}
\answer{%
    $\text{для мнимого - на половине фокусного, для действительного - на бесконечности}$
}
\solutionspace{120pt}

\tasknumber{2}%
\task{%
    На экране, расположенном иа расстоянии $80\,\text{см}$ от собирающей линзы,
    получено изображение точечного источника, расположенного на главной оптической оси линзы.
    На какое расстояние переместится изображение на экране,
    если при неподвижном источнике переместить линзу на $2\,\text{см}$ в плоскости, перпендикулярной главной оптической оси?
    Фокусное расстояние линзы равно $20\,\text{см}$.
}
\answer{%
    \begin{align*}
    &\frac 1F = \frac 1a + \frac 1b \implies a = \frac{bF}{b-F} \implies \Gamma = \frac ba = \frac{b-F}F \\
    &y = x \cdot \Gamma = x \cdot \frac{b-F}F \implies d = x + y = 8\,\text{см}.
    \end{align*}
}
\solutionspace{120pt}

\tasknumber{3}%
\task{%
    Оптическая сила двояковыпуклой линзы в воздухе $5\,\text{дптр}$, а в воде $1{,}6\,\text{дптр}$.
    Определить показатель преломления $n$ материала, из которого изготовлена линза.
}
\answer{%
    \begin{align*}
    D_1 &=\cbr{\frac n{n_1} - 1}\cbr{\frac 1{R_1} + \frac 1{R_2}}, \\
    D_2 &=\cbr{\frac n{n_2} - 1}\cbr{\frac 1{R_1} + \frac 1{R_2}}, \\
    \frac {D_2}{D_1} &=\frac{\frac n{n_2} - 1}{\frac n{n_1} - 1} \implies {D_2}\cbr{\frac n{n_1} - 1} = {D_1}\cbr{\frac n{n_2} - 1}  \implies n\cbr{\frac{D_2}{n_1} - \frac{D_1}{n_2}} = D_2 - D_1, \\
    n &= \frac{D_2 - D_1}{\frac{D_2}{n_1} - \frac{D_1}{n_2}} = \frac{n_1 n_2 (D_2 - D_1)}{D_2n_2 - D_1n_1} \approx 1{,}575.
    \end{align*}
}
\solutionspace{120pt}

\tasknumber{4}%
\task{%
    На каком расстоянии от собирающей линзы с фокусным расстоянием $50\,\text{дптр}$
    следует надо поместить предмет, чтобы расстояние
    от предмета до его действительного изображения было наименьшим?
}
\answer{%
    \begin{align*}
    \frac 1a &+ \frac 1b = D \implies b = \frac 1{D - \frac 1a} \implies \ell = a + b = a + \frac a{Da - 1} = \frac{ Da^2 }{Da - 1} \implies \\
    \implies \ell'_a &= \frac{ 2Da \cdot (Da - 1) - Da^2 \cdot D }{\sqr{Da - 1}}= \frac{ D^2a^2 - 2Da}{\sqr{Da - 1}} = \frac{ Da(Da - 2)}{\sqr{Da - 1}}\implies a_{\min} = \frac 2D \approx 40\,\text{мм}.
    \end{align*}
}
\solutionspace{120pt}

\tasknumber{5}%
\task{%
    Предмет в виде отрезка длиной $\ell$ расположен вдоль оптической оси
    собирающей линзы с фокусным расстоянием $F$.
    Середина отрезка расположена
    иа расстоянии $a$ от линзы, которая даёт действительное изображение
    всех точек предмета.
    Определить продольное увеличение предмета.
}
\answer{%
    \begin{align*}
    \frac 1{a + \frac \ell 2} &+ \frac 1b = \frac 1F \implies b = \frac{F\cbr{a + \frac \ell 2}}{a + \frac \ell 2 - F} \\
    \frac 1{a - \frac \ell 2} &+ \frac 1c = \frac 1F \implies c = \frac{F\cbr{a - \frac \ell 2}}{a - \frac \ell 2 - F} \\
    \abs{b - c} &= \abs{\frac{F\cbr{a + \frac \ell 2}}{a + \frac \ell 2 - F} - \frac{F\cbr{a - \frac \ell 2}}{a - \frac \ell 2 - F}}= F\abs{\frac{\cbr{a + \frac \ell 2}\cbr{a - \frac \ell 2 - F} - \cbr{a - \frac \ell 2}\cbr{a + \frac \ell 2 - F}}{ \cbr{a + \frac \ell 2 - F} \cbr{a - \frac \ell 2 - F} }} =  \\
    &= F\abs{\frac{a^2 - \frac {a\ell} 2 - Fa + \frac {a\ell} 2 - \frac {\ell^2} 4 - \frac {F\ell}2 - a^2 - \frac {a\ell}2 + aF + \frac {a\ell}2 + \frac {\ell^2} 4 - \frac {F\ell} 2}{\cbr{a + \frac \ell 2 - F} \cbr{a - \frac \ell 2 - F} }} = \\
    &= F\frac{F\ell}{\sqr{a-F} - \frac {\ell^2}4} = \frac{F^2\ell}{\sqr{a-F} - \frac {\ell^2}4}\implies \Gamma = \frac{\abs{b - c}}\ell = \frac{F^2}{\sqr{a-F} - \frac {\ell^2}4}.
    \end{align*}
}
\solutionspace{120pt}

\tasknumber{6}%
\task{%
    Даны точечный источник света $S$, его изображение $S_1$, полученное с помошью собирающей линзы,
    и ближайший к источнику фокус линзы $F$ (см.
    рис.
    на доске).
    Расстояния $SF = \ell$ и $SS_1 = L$.
    Определить положение линзы и её фокусное расстояние.
}
\answer{%
    \begin{align*}
    \frac 1a + \frac 1b &= \frac 1F, \ell = a - F, L = a + b \implies a = \ell + F, b = L - a = L - \ell - F \\
    \frac 1{\ell + F} + \frac 1{L - \ell - F} &= \frac 1F \\
    F\ell + F^2 + LF - F\ell - F^2 &= L\ell - \ell^2 - F\ell + LF - F\ell - F^2 \\
    0 &= L\ell - \ell^2 - 2F\ell - F^2 \\
    0 &=  F^2 + 2F\ell - L\ell + \ell^2 \\
    F &= -\ell \pm \sqrt{\ell^2 +  L\ell - \ell^2} = -\ell \pm \sqrt{L\ell} \implies F = \sqrt{L\ell} - \ell \\
    a &= \ell + F = \ell + \sqrt{L\ell} - \ell = \sqrt{L\ell}.
    \end{align*}
}
\solutionspace{120pt}

\tasknumber{7}%
\task{%
    Расстояние от освещённого предмета до экрана $100\,\text{см}$.
    Линза, помещенная между ними, даёт чёткое изображение предмета на
    экране при двух положениях, расстояние между которыми $20\,\text{см}$.
    Найти фокусное расстояние линзы.
}
\answer{%
    \begin{align*}
    \frac 1a + \frac 1b &= \frac 1F, \frac 1{a-\ell} + \frac 1{b+\ell} = \frac 1F, a + b = L \\
    \frac 1a + \frac 1b &= \frac 1{a-\ell} + \frac 1{b+\ell}\implies \frac{a + b}{ab} = \frac{(a-\ell) + (b+\ell)}{(a-\ell)(b+\ell)} \\
    ab  &= (a - \ell)(b+\ell) \implies 0  = -b\ell + a\ell - \ell^2 \implies 0 = -b + a - \ell \implies b = a - \ell \\
    a + (a - \ell) &= L \implies a = \frac{L + \ell}2 \implies b = \frac{L - \ell}2 \\
    F &= \frac{ab}{a + b} = \frac{L^2 -\ell^2}{4L} \approx 24\,\text{см}.
    \end{align*}
}
\solutionspace{120pt}

\tasknumber{8}%
\task{%
    Предмет находится на расстоянии $80\,\text{см}$ от экрана.
    Между предметом и экраном помещают линзу, причём при одном
    положении линзы на экране получается увеличенное изображение предмета,
    а при другом — уменьшенное.
    Каково фокусное расстояние линзы, если
    линейные размеры первого изображения в два раза больше второго?
}
\answer{%
    \begin{align*}
    \frac 1a + \frac 1{L-a} &= \frac 1F, h_1 = h \cdot \frac{L-a}a, \\
    \frac 1b + \frac 1{L-b} &= \frac 1F, h_2 = h \cdot \frac{L-b}b, \\
    \frac{h_1}{h_2} &= 2 \implies \frac{(L-a)b}{(L-b)a} = 2, \\
    \frac 1F &= \frac{ L }{a(L-a)} = \frac{ L }{b(L-b)} \implies \frac{L-a}{L-b} = \frac b a \implies \frac {b^2}{a^2} = 2.
    \\
    \frac 1a + \frac 1{L-a} &= \frac 1b + \frac 1{L-b} \implies \frac L{a(L-a)} = \frac L{b(L-b)} \implies \\
    \implies aL - a^2 &= bL - b^2 \implies (a-b)L = (a-b)(a+b) \implies b = L - a, \\
    \frac{\sqr{L-a}}{a^2} &= 2 \implies \frac La - 1 = \sqrt{2} \implies a = \frac{ L }{\sqrt{2} + 1} \\
    F &= \frac{a(L-a)}L = \frac 1L \cdot \frac L{\sqrt{2} + 1} \cdot \frac {L\sqrt{2}}{\sqrt{2} + 1}= \frac { L\sqrt{2} }{ \sqr{\sqrt{2} + 1} } \approx 19{,}4\,\text{см}.
    \end{align*}
}

\variantsplitter

\addpersonalvariant{Варвара Егиазарян}

\tasknumber{1}%
\task{%
    В каком месте на главной оптической оси двояковыгнутой линзы
    нужно поместить точечный источник света,
    чтобы его изображение оказалось в главном фокусе линзы?
}
\answer{%
    $\text{на половине фокусного расстояния}$
}
\solutionspace{120pt}

\tasknumber{2}%
\task{%
    На экране, расположенном иа расстоянии $60\,\text{см}$ от собирающей линзы,
    получено изображение точечного источника, расположенного на главной оптической оси линзы.
    На какое расстояние переместится изображение на экране,
    если при неподвижном источнике переместить линзу на $1\,\text{см}$ в плоскости, перпендикулярной главной оптической оси?
    Фокусное расстояние линзы равно $30\,\text{см}$.
}
\answer{%
    \begin{align*}
    &\frac 1F = \frac 1a + \frac 1b \implies a = \frac{bF}{b-F} \implies \Gamma = \frac ba = \frac{b-F}F \\
    &y = x \cdot \Gamma = x \cdot \frac{b-F}F \implies d = x + y = 2\,\text{см}.
    \end{align*}
}
\solutionspace{120pt}

\tasknumber{3}%
\task{%
    Оптическая сила двояковыпуклой линзы в воздухе $4{,}5\,\text{дптр}$, а в воде $1{,}6\,\text{дптр}$.
    Определить показатель преломления $n$ материала, из которого изготовлена линза.
}
\answer{%
    \begin{align*}
    D_1 &=\cbr{\frac n{n_1} - 1}\cbr{\frac 1{R_1} + \frac 1{R_2}}, \\
    D_2 &=\cbr{\frac n{n_2} - 1}\cbr{\frac 1{R_1} + \frac 1{R_2}}, \\
    \frac {D_2}{D_1} &=\frac{\frac n{n_2} - 1}{\frac n{n_1} - 1} \implies {D_2}\cbr{\frac n{n_1} - 1} = {D_1}\cbr{\frac n{n_2} - 1}  \implies n\cbr{\frac{D_2}{n_1} - \frac{D_1}{n_2}} = D_2 - D_1, \\
    n &= \frac{D_2 - D_1}{\frac{D_2}{n_1} - \frac{D_1}{n_2}} = \frac{n_1 n_2 (D_2 - D_1)}{D_2n_2 - D_1n_1} \approx 1{,}626.
    \end{align*}
}
\solutionspace{120pt}

\tasknumber{4}%
\task{%
    На каком расстоянии от собирающей линзы с фокусным расстоянием $50\,\text{дптр}$
    следует надо поместить предмет, чтобы расстояние
    от предмета до его действительного изображения было наименьшим?
}
\answer{%
    \begin{align*}
    \frac 1a &+ \frac 1b = D \implies b = \frac 1{D - \frac 1a} \implies \ell = a + b = a + \frac a{Da - 1} = \frac{ Da^2 }{Da - 1} \implies \\
    \implies \ell'_a &= \frac{ 2Da \cdot (Da - 1) - Da^2 \cdot D }{\sqr{Da - 1}}= \frac{ D^2a^2 - 2Da}{\sqr{Da - 1}} = \frac{ Da(Da - 2)}{\sqr{Da - 1}}\implies a_{\min} = \frac 2D \approx 40\,\text{мм}.
    \end{align*}
}
\solutionspace{120pt}

\tasknumber{5}%
\task{%
    Предмет в виде отрезка длиной $\ell$ расположен вдоль оптической оси
    собирающей линзы с фокусным расстоянием $F$.
    Середина отрезка расположена
    иа расстоянии $a$ от линзы, которая даёт действительное изображение
    всех точек предмета.
    Определить продольное увеличение предмета.
}
\answer{%
    \begin{align*}
    \frac 1{a + \frac \ell 2} &+ \frac 1b = \frac 1F \implies b = \frac{F\cbr{a + \frac \ell 2}}{a + \frac \ell 2 - F} \\
    \frac 1{a - \frac \ell 2} &+ \frac 1c = \frac 1F \implies c = \frac{F\cbr{a - \frac \ell 2}}{a - \frac \ell 2 - F} \\
    \abs{b - c} &= \abs{\frac{F\cbr{a + \frac \ell 2}}{a + \frac \ell 2 - F} - \frac{F\cbr{a - \frac \ell 2}}{a - \frac \ell 2 - F}}= F\abs{\frac{\cbr{a + \frac \ell 2}\cbr{a - \frac \ell 2 - F} - \cbr{a - \frac \ell 2}\cbr{a + \frac \ell 2 - F}}{ \cbr{a + \frac \ell 2 - F} \cbr{a - \frac \ell 2 - F} }} =  \\
    &= F\abs{\frac{a^2 - \frac {a\ell} 2 - Fa + \frac {a\ell} 2 - \frac {\ell^2} 4 - \frac {F\ell}2 - a^2 - \frac {a\ell}2 + aF + \frac {a\ell}2 + \frac {\ell^2} 4 - \frac {F\ell} 2}{\cbr{a + \frac \ell 2 - F} \cbr{a - \frac \ell 2 - F} }} = \\
    &= F\frac{F\ell}{\sqr{a-F} - \frac {\ell^2}4} = \frac{F^2\ell}{\sqr{a-F} - \frac {\ell^2}4}\implies \Gamma = \frac{\abs{b - c}}\ell = \frac{F^2}{\sqr{a-F} - \frac {\ell^2}4}.
    \end{align*}
}
\solutionspace{120pt}

\tasknumber{6}%
\task{%
    Даны точечный источник света $S$, его изображение $S_1$, полученное с помошью собирающей линзы,
    и ближайший к источнику фокус линзы $F$ (см.
    рис.
    на доске).
    Расстояния $SF = \ell$ и $SS_1 = L$.
    Определить положение линзы и её фокусное расстояние.
}
\answer{%
    \begin{align*}
    \frac 1a + \frac 1b &= \frac 1F, \ell = a - F, L = a + b \implies a = \ell + F, b = L - a = L - \ell - F \\
    \frac 1{\ell + F} + \frac 1{L - \ell - F} &= \frac 1F \\
    F\ell + F^2 + LF - F\ell - F^2 &= L\ell - \ell^2 - F\ell + LF - F\ell - F^2 \\
    0 &= L\ell - \ell^2 - 2F\ell - F^2 \\
    0 &=  F^2 + 2F\ell - L\ell + \ell^2 \\
    F &= -\ell \pm \sqrt{\ell^2 +  L\ell - \ell^2} = -\ell \pm \sqrt{L\ell} \implies F = \sqrt{L\ell} - \ell \\
    a &= \ell + F = \ell + \sqrt{L\ell} - \ell = \sqrt{L\ell}.
    \end{align*}
}
\solutionspace{120pt}

\tasknumber{7}%
\task{%
    Расстояние от освещённого предмета до экрана $100\,\text{см}$.
    Линза, помещенная между ними, даёт чёткое изображение предмета на
    экране при двух положениях, расстояние между которыми $20\,\text{см}$.
    Найти фокусное расстояние линзы.
}
\answer{%
    \begin{align*}
    \frac 1a + \frac 1b &= \frac 1F, \frac 1{a-\ell} + \frac 1{b+\ell} = \frac 1F, a + b = L \\
    \frac 1a + \frac 1b &= \frac 1{a-\ell} + \frac 1{b+\ell}\implies \frac{a + b}{ab} = \frac{(a-\ell) + (b+\ell)}{(a-\ell)(b+\ell)} \\
    ab  &= (a - \ell)(b+\ell) \implies 0  = -b\ell + a\ell - \ell^2 \implies 0 = -b + a - \ell \implies b = a - \ell \\
    a + (a - \ell) &= L \implies a = \frac{L + \ell}2 \implies b = \frac{L - \ell}2 \\
    F &= \frac{ab}{a + b} = \frac{L^2 -\ell^2}{4L} \approx 24\,\text{см}.
    \end{align*}
}
\solutionspace{120pt}

\tasknumber{8}%
\task{%
    Предмет находится на расстоянии $80\,\text{см}$ от экрана.
    Между предметом и экраном помещают линзу, причём при одном
    положении линзы на экране получается увеличенное изображение предмета,
    а при другом — уменьшенное.
    Каково фокусное расстояние линзы, если
    линейные размеры первого изображения в пять раз больше второго?
}
\answer{%
    \begin{align*}
    \frac 1a + \frac 1{L-a} &= \frac 1F, h_1 = h \cdot \frac{L-a}a, \\
    \frac 1b + \frac 1{L-b} &= \frac 1F, h_2 = h \cdot \frac{L-b}b, \\
    \frac{h_1}{h_2} &= 5 \implies \frac{(L-a)b}{(L-b)a} = 5, \\
    \frac 1F &= \frac{ L }{a(L-a)} = \frac{ L }{b(L-b)} \implies \frac{L-a}{L-b} = \frac b a \implies \frac {b^2}{a^2} = 5.
    \\
    \frac 1a + \frac 1{L-a} &= \frac 1b + \frac 1{L-b} \implies \frac L{a(L-a)} = \frac L{b(L-b)} \implies \\
    \implies aL - a^2 &= bL - b^2 \implies (a-b)L = (a-b)(a+b) \implies b = L - a, \\
    \frac{\sqr{L-a}}{a^2} &= 5 \implies \frac La - 1 = \sqrt{5} \implies a = \frac{ L }{\sqrt{5} + 1} \\
    F &= \frac{a(L-a)}L = \frac 1L \cdot \frac L{\sqrt{5} + 1} \cdot \frac {L\sqrt{5}}{\sqrt{5} + 1}= \frac { L\sqrt{5} }{ \sqr{\sqrt{5} + 1} } \approx 17{,}1\,\text{см}.
    \end{align*}
}

\variantsplitter

\addpersonalvariant{Владислав Емелин}

\tasknumber{1}%
\task{%
    В каком месте на главной оптической оси двояковыпуклой линзы
    нужно поместить точечный источник света,
    чтобы его изображение оказалось в главном фокусе линзы?
}
\answer{%
    $\text{для мнимого - на половине фокусного, для действительного - на бесконечности}$
}
\solutionspace{120pt}

\tasknumber{2}%
\task{%
    На экране, расположенном иа расстоянии $120\,\text{см}$ от собирающей линзы,
    получено изображение точечного источника, расположенного на главной оптической оси линзы.
    На какое расстояние переместится изображение на экране,
    если при неподвижном источнике переместить линзу на $2\,\text{см}$ в плоскости, перпендикулярной главной оптической оси?
    Фокусное расстояние линзы равно $30\,\text{см}$.
}
\answer{%
    \begin{align*}
    &\frac 1F = \frac 1a + \frac 1b \implies a = \frac{bF}{b-F} \implies \Gamma = \frac ba = \frac{b-F}F \\
    &y = x \cdot \Gamma = x \cdot \frac{b-F}F \implies d = x + y = 8\,\text{см}.
    \end{align*}
}
\solutionspace{120pt}

\tasknumber{3}%
\task{%
    Оптическая сила двояковыпуклой линзы в воздухе $5{,}5\,\text{дптр}$, а в воде $1{,}5\,\text{дптр}$.
    Определить показатель преломления $n$ материала, из которого изготовлена линза.
}
\answer{%
    \begin{align*}
    D_1 &=\cbr{\frac n{n_1} - 1}\cbr{\frac 1{R_1} + \frac 1{R_2}}, \\
    D_2 &=\cbr{\frac n{n_2} - 1}\cbr{\frac 1{R_1} + \frac 1{R_2}}, \\
    \frac {D_2}{D_1} &=\frac{\frac n{n_2} - 1}{\frac n{n_1} - 1} \implies {D_2}\cbr{\frac n{n_1} - 1} = {D_1}\cbr{\frac n{n_2} - 1}  \implies n\cbr{\frac{D_2}{n_1} - \frac{D_1}{n_2}} = D_2 - D_1, \\
    n &= \frac{D_2 - D_1}{\frac{D_2}{n_1} - \frac{D_1}{n_2}} = \frac{n_1 n_2 (D_2 - D_1)}{D_2n_2 - D_1n_1} \approx 1{,}518.
    \end{align*}
}
\solutionspace{120pt}

\tasknumber{4}%
\task{%
    На каком расстоянии от собирающей линзы с фокусным расстоянием $30\,\text{дптр}$
    следует надо поместить предмет, чтобы расстояние
    от предмета до его действительного изображения было наименьшим?
}
\answer{%
    \begin{align*}
    \frac 1a &+ \frac 1b = D \implies b = \frac 1{D - \frac 1a} \implies \ell = a + b = a + \frac a{Da - 1} = \frac{ Da^2 }{Da - 1} \implies \\
    \implies \ell'_a &= \frac{ 2Da \cdot (Da - 1) - Da^2 \cdot D }{\sqr{Da - 1}}= \frac{ D^2a^2 - 2Da}{\sqr{Da - 1}} = \frac{ Da(Da - 2)}{\sqr{Da - 1}}\implies a_{\min} = \frac 2D \approx 66{,}7\,\text{мм}.
    \end{align*}
}
\solutionspace{120pt}

\tasknumber{5}%
\task{%
    Предмет в виде отрезка длиной $\ell$ расположен вдоль оптической оси
    собирающей линзы с фокусным расстоянием $F$.
    Середина отрезка расположена
    иа расстоянии $a$ от линзы, которая даёт действительное изображение
    всех точек предмета.
    Определить продольное увеличение предмета.
}
\answer{%
    \begin{align*}
    \frac 1{a + \frac \ell 2} &+ \frac 1b = \frac 1F \implies b = \frac{F\cbr{a + \frac \ell 2}}{a + \frac \ell 2 - F} \\
    \frac 1{a - \frac \ell 2} &+ \frac 1c = \frac 1F \implies c = \frac{F\cbr{a - \frac \ell 2}}{a - \frac \ell 2 - F} \\
    \abs{b - c} &= \abs{\frac{F\cbr{a + \frac \ell 2}}{a + \frac \ell 2 - F} - \frac{F\cbr{a - \frac \ell 2}}{a - \frac \ell 2 - F}}= F\abs{\frac{\cbr{a + \frac \ell 2}\cbr{a - \frac \ell 2 - F} - \cbr{a - \frac \ell 2}\cbr{a + \frac \ell 2 - F}}{ \cbr{a + \frac \ell 2 - F} \cbr{a - \frac \ell 2 - F} }} =  \\
    &= F\abs{\frac{a^2 - \frac {a\ell} 2 - Fa + \frac {a\ell} 2 - \frac {\ell^2} 4 - \frac {F\ell}2 - a^2 - \frac {a\ell}2 + aF + \frac {a\ell}2 + \frac {\ell^2} 4 - \frac {F\ell} 2}{\cbr{a + \frac \ell 2 - F} \cbr{a - \frac \ell 2 - F} }} = \\
    &= F\frac{F\ell}{\sqr{a-F} - \frac {\ell^2}4} = \frac{F^2\ell}{\sqr{a-F} - \frac {\ell^2}4}\implies \Gamma = \frac{\abs{b - c}}\ell = \frac{F^2}{\sqr{a-F} - \frac {\ell^2}4}.
    \end{align*}
}
\solutionspace{120pt}

\tasknumber{6}%
\task{%
    Даны точечный источник света $S$, его изображение $S_1$, полученное с помошью собирающей линзы,
    и ближайший к источнику фокус линзы $F$ (см.
    рис.
    на доске).
    Расстояния $SF = \ell$ и $SS_1 = L$.
    Определить положение линзы и её фокусное расстояние.
}
\answer{%
    \begin{align*}
    \frac 1a + \frac 1b &= \frac 1F, \ell = a - F, L = a + b \implies a = \ell + F, b = L - a = L - \ell - F \\
    \frac 1{\ell + F} + \frac 1{L - \ell - F} &= \frac 1F \\
    F\ell + F^2 + LF - F\ell - F^2 &= L\ell - \ell^2 - F\ell + LF - F\ell - F^2 \\
    0 &= L\ell - \ell^2 - 2F\ell - F^2 \\
    0 &=  F^2 + 2F\ell - L\ell + \ell^2 \\
    F &= -\ell \pm \sqrt{\ell^2 +  L\ell - \ell^2} = -\ell \pm \sqrt{L\ell} \implies F = \sqrt{L\ell} - \ell \\
    a &= \ell + F = \ell + \sqrt{L\ell} - \ell = \sqrt{L\ell}.
    \end{align*}
}
\solutionspace{120pt}

\tasknumber{7}%
\task{%
    Расстояние от освещённого предмета до экрана $80\,\text{см}$.
    Линза, помещенная между ними, даёт чёткое изображение предмета на
    экране при двух положениях, расстояние между которыми $40\,\text{см}$.
    Найти фокусное расстояние линзы.
}
\answer{%
    \begin{align*}
    \frac 1a + \frac 1b &= \frac 1F, \frac 1{a-\ell} + \frac 1{b+\ell} = \frac 1F, a + b = L \\
    \frac 1a + \frac 1b &= \frac 1{a-\ell} + \frac 1{b+\ell}\implies \frac{a + b}{ab} = \frac{(a-\ell) + (b+\ell)}{(a-\ell)(b+\ell)} \\
    ab  &= (a - \ell)(b+\ell) \implies 0  = -b\ell + a\ell - \ell^2 \implies 0 = -b + a - \ell \implies b = a - \ell \\
    a + (a - \ell) &= L \implies a = \frac{L + \ell}2 \implies b = \frac{L - \ell}2 \\
    F &= \frac{ab}{a + b} = \frac{L^2 -\ell^2}{4L} \approx 15\,\text{см}.
    \end{align*}
}
\solutionspace{120pt}

\tasknumber{8}%
\task{%
    Предмет находится на расстоянии $70\,\text{см}$ от экрана.
    Между предметом и экраном помещают линзу, причём при одном
    положении линзы на экране получается увеличенное изображение предмета,
    а при другом — уменьшенное.
    Каково фокусное расстояние линзы, если
    линейные размеры первого изображения в пять раз больше второго?
}
\answer{%
    \begin{align*}
    \frac 1a + \frac 1{L-a} &= \frac 1F, h_1 = h \cdot \frac{L-a}a, \\
    \frac 1b + \frac 1{L-b} &= \frac 1F, h_2 = h \cdot \frac{L-b}b, \\
    \frac{h_1}{h_2} &= 5 \implies \frac{(L-a)b}{(L-b)a} = 5, \\
    \frac 1F &= \frac{ L }{a(L-a)} = \frac{ L }{b(L-b)} \implies \frac{L-a}{L-b} = \frac b a \implies \frac {b^2}{a^2} = 5.
    \\
    \frac 1a + \frac 1{L-a} &= \frac 1b + \frac 1{L-b} \implies \frac L{a(L-a)} = \frac L{b(L-b)} \implies \\
    \implies aL - a^2 &= bL - b^2 \implies (a-b)L = (a-b)(a+b) \implies b = L - a, \\
    \frac{\sqr{L-a}}{a^2} &= 5 \implies \frac La - 1 = \sqrt{5} \implies a = \frac{ L }{\sqrt{5} + 1} \\
    F &= \frac{a(L-a)}L = \frac 1L \cdot \frac L{\sqrt{5} + 1} \cdot \frac {L\sqrt{5}}{\sqrt{5} + 1}= \frac { L\sqrt{5} }{ \sqr{\sqrt{5} + 1} } \approx 14{,}9\,\text{см}.
    \end{align*}
}

\variantsplitter

\addpersonalvariant{Артём Жичин}

\tasknumber{1}%
\task{%
    В каком месте на главной оптической оси двояковыпуклой линзы
    нужно поместить точечный источник света,
    чтобы его изображение оказалось в главном фокусе линзы?
}
\answer{%
    $\text{для мнимого - на половине фокусного, для действительного - на бесконечности}$
}
\solutionspace{120pt}

\tasknumber{2}%
\task{%
    На экране, расположенном иа расстоянии $120\,\text{см}$ от собирающей линзы,
    получено изображение точечного источника, расположенного на главной оптической оси линзы.
    На какое расстояние переместится изображение на экране,
    если при неподвижном источнике переместить линзу на $1\,\text{см}$ в плоскости, перпендикулярной главной оптической оси?
    Фокусное расстояние линзы равно $20\,\text{см}$.
}
\answer{%
    \begin{align*}
    &\frac 1F = \frac 1a + \frac 1b \implies a = \frac{bF}{b-F} \implies \Gamma = \frac ba = \frac{b-F}F \\
    &y = x \cdot \Gamma = x \cdot \frac{b-F}F \implies d = x + y = 6\,\text{см}.
    \end{align*}
}
\solutionspace{120pt}

\tasknumber{3}%
\task{%
    Оптическая сила двояковыпуклой линзы в воздухе $5{,}5\,\text{дптр}$, а в воде $1{,}6\,\text{дптр}$.
    Определить показатель преломления $n$ материала, из которого изготовлена линза.
}
\answer{%
    \begin{align*}
    D_1 &=\cbr{\frac n{n_1} - 1}\cbr{\frac 1{R_1} + \frac 1{R_2}}, \\
    D_2 &=\cbr{\frac n{n_2} - 1}\cbr{\frac 1{R_1} + \frac 1{R_2}}, \\
    \frac {D_2}{D_1} &=\frac{\frac n{n_2} - 1}{\frac n{n_1} - 1} \implies {D_2}\cbr{\frac n{n_1} - 1} = {D_1}\cbr{\frac n{n_2} - 1}  \implies n\cbr{\frac{D_2}{n_1} - \frac{D_1}{n_2}} = D_2 - D_1, \\
    n &= \frac{D_2 - D_1}{\frac{D_2}{n_1} - \frac{D_1}{n_2}} = \frac{n_1 n_2 (D_2 - D_1)}{D_2n_2 - D_1n_1} \approx 1{,}538.
    \end{align*}
}
\solutionspace{120pt}

\tasknumber{4}%
\task{%
    На каком расстоянии от собирающей линзы с фокусным расстоянием $40\,\text{дптр}$
    следует надо поместить предмет, чтобы расстояние
    от предмета до его действительного изображения было наименьшим?
}
\answer{%
    \begin{align*}
    \frac 1a &+ \frac 1b = D \implies b = \frac 1{D - \frac 1a} \implies \ell = a + b = a + \frac a{Da - 1} = \frac{ Da^2 }{Da - 1} \implies \\
    \implies \ell'_a &= \frac{ 2Da \cdot (Da - 1) - Da^2 \cdot D }{\sqr{Da - 1}}= \frac{ D^2a^2 - 2Da}{\sqr{Da - 1}} = \frac{ Da(Da - 2)}{\sqr{Da - 1}}\implies a_{\min} = \frac 2D \approx 50\,\text{мм}.
    \end{align*}
}
\solutionspace{120pt}

\tasknumber{5}%
\task{%
    Предмет в виде отрезка длиной $\ell$ расположен вдоль оптической оси
    собирающей линзы с фокусным расстоянием $F$.
    Середина отрезка расположена
    иа расстоянии $a$ от линзы, которая даёт действительное изображение
    всех точек предмета.
    Определить продольное увеличение предмета.
}
\answer{%
    \begin{align*}
    \frac 1{a + \frac \ell 2} &+ \frac 1b = \frac 1F \implies b = \frac{F\cbr{a + \frac \ell 2}}{a + \frac \ell 2 - F} \\
    \frac 1{a - \frac \ell 2} &+ \frac 1c = \frac 1F \implies c = \frac{F\cbr{a - \frac \ell 2}}{a - \frac \ell 2 - F} \\
    \abs{b - c} &= \abs{\frac{F\cbr{a + \frac \ell 2}}{a + \frac \ell 2 - F} - \frac{F\cbr{a - \frac \ell 2}}{a - \frac \ell 2 - F}}= F\abs{\frac{\cbr{a + \frac \ell 2}\cbr{a - \frac \ell 2 - F} - \cbr{a - \frac \ell 2}\cbr{a + \frac \ell 2 - F}}{ \cbr{a + \frac \ell 2 - F} \cbr{a - \frac \ell 2 - F} }} =  \\
    &= F\abs{\frac{a^2 - \frac {a\ell} 2 - Fa + \frac {a\ell} 2 - \frac {\ell^2} 4 - \frac {F\ell}2 - a^2 - \frac {a\ell}2 + aF + \frac {a\ell}2 + \frac {\ell^2} 4 - \frac {F\ell} 2}{\cbr{a + \frac \ell 2 - F} \cbr{a - \frac \ell 2 - F} }} = \\
    &= F\frac{F\ell}{\sqr{a-F} - \frac {\ell^2}4} = \frac{F^2\ell}{\sqr{a-F} - \frac {\ell^2}4}\implies \Gamma = \frac{\abs{b - c}}\ell = \frac{F^2}{\sqr{a-F} - \frac {\ell^2}4}.
    \end{align*}
}
\solutionspace{120pt}

\tasknumber{6}%
\task{%
    Даны точечный источник света $S$, его изображение $S_1$, полученное с помошью собирающей линзы,
    и ближайший к источнику фокус линзы $F$ (см.
    рис.
    на доске).
    Расстояния $SF = \ell$ и $SS_1 = L$.
    Определить положение линзы и её фокусное расстояние.
}
\answer{%
    \begin{align*}
    \frac 1a + \frac 1b &= \frac 1F, \ell = a - F, L = a + b \implies a = \ell + F, b = L - a = L - \ell - F \\
    \frac 1{\ell + F} + \frac 1{L - \ell - F} &= \frac 1F \\
    F\ell + F^2 + LF - F\ell - F^2 &= L\ell - \ell^2 - F\ell + LF - F\ell - F^2 \\
    0 &= L\ell - \ell^2 - 2F\ell - F^2 \\
    0 &=  F^2 + 2F\ell - L\ell + \ell^2 \\
    F &= -\ell \pm \sqrt{\ell^2 +  L\ell - \ell^2} = -\ell \pm \sqrt{L\ell} \implies F = \sqrt{L\ell} - \ell \\
    a &= \ell + F = \ell + \sqrt{L\ell} - \ell = \sqrt{L\ell}.
    \end{align*}
}
\solutionspace{120pt}

\tasknumber{7}%
\task{%
    Расстояние от освещённого предмета до экрана $80\,\text{см}$.
    Линза, помещенная между ними, даёт чёткое изображение предмета на
    экране при двух положениях, расстояние между которыми $30\,\text{см}$.
    Найти фокусное расстояние линзы.
}
\answer{%
    \begin{align*}
    \frac 1a + \frac 1b &= \frac 1F, \frac 1{a-\ell} + \frac 1{b+\ell} = \frac 1F, a + b = L \\
    \frac 1a + \frac 1b &= \frac 1{a-\ell} + \frac 1{b+\ell}\implies \frac{a + b}{ab} = \frac{(a-\ell) + (b+\ell)}{(a-\ell)(b+\ell)} \\
    ab  &= (a - \ell)(b+\ell) \implies 0  = -b\ell + a\ell - \ell^2 \implies 0 = -b + a - \ell \implies b = a - \ell \\
    a + (a - \ell) &= L \implies a = \frac{L + \ell}2 \implies b = \frac{L - \ell}2 \\
    F &= \frac{ab}{a + b} = \frac{L^2 -\ell^2}{4L} \approx 17{,}2\,\text{см}.
    \end{align*}
}
\solutionspace{120pt}

\tasknumber{8}%
\task{%
    Предмет находится на расстоянии $80\,\text{см}$ от экрана.
    Между предметом и экраном помещают линзу, причём при одном
    положении линзы на экране получается увеличенное изображение предмета,
    а при другом — уменьшенное.
    Каково фокусное расстояние линзы, если
    линейные размеры первого изображения в три раза больше второго?
}
\answer{%
    \begin{align*}
    \frac 1a + \frac 1{L-a} &= \frac 1F, h_1 = h \cdot \frac{L-a}a, \\
    \frac 1b + \frac 1{L-b} &= \frac 1F, h_2 = h \cdot \frac{L-b}b, \\
    \frac{h_1}{h_2} &= 3 \implies \frac{(L-a)b}{(L-b)a} = 3, \\
    \frac 1F &= \frac{ L }{a(L-a)} = \frac{ L }{b(L-b)} \implies \frac{L-a}{L-b} = \frac b a \implies \frac {b^2}{a^2} = 3.
    \\
    \frac 1a + \frac 1{L-a} &= \frac 1b + \frac 1{L-b} \implies \frac L{a(L-a)} = \frac L{b(L-b)} \implies \\
    \implies aL - a^2 &= bL - b^2 \implies (a-b)L = (a-b)(a+b) \implies b = L - a, \\
    \frac{\sqr{L-a}}{a^2} &= 3 \implies \frac La - 1 = \sqrt{3} \implies a = \frac{ L }{\sqrt{3} + 1} \\
    F &= \frac{a(L-a)}L = \frac 1L \cdot \frac L{\sqrt{3} + 1} \cdot \frac {L\sqrt{3}}{\sqrt{3} + 1}= \frac { L\sqrt{3} }{ \sqr{\sqrt{3} + 1} } \approx 18{,}6\,\text{см}.
    \end{align*}
}

\variantsplitter

\addpersonalvariant{Дарья Кошман}

\tasknumber{1}%
\task{%
    В каком месте на главной оптической оси двояковыгнутой линзы
    нужно поместить точечный источник света,
    чтобы его изображение оказалось в главном фокусе линзы?
}
\answer{%
    $\text{на половине фокусного расстояния}$
}
\solutionspace{120pt}

\tasknumber{2}%
\task{%
    На экране, расположенном иа расстоянии $80\,\text{см}$ от собирающей линзы,
    получено изображение точечного источника, расположенного на главной оптической оси линзы.
    На какое расстояние переместится изображение на экране,
    если при неподвижной линзе переместить источник на $1\,\text{см}$ в плоскости, перпендикулярной главной оптической оси?
    Фокусное расстояние линзы равно $20\,\text{см}$.
}
\answer{%
    \begin{align*}
    &\frac 1F = \frac 1a + \frac 1b \implies a = \frac{bF}{b-F} \implies \Gamma = \frac ba = \frac{b-F}F \\
    &y = x \cdot \Gamma = x \cdot \frac{b-F}F \implies d = y = 3\,\text{см}.
    \end{align*}
}
\solutionspace{120pt}

\tasknumber{3}%
\task{%
    Оптическая сила двояковыпуклой линзы в воздухе $4{,}5\,\text{дптр}$, а в воде $1{,}6\,\text{дптр}$.
    Определить показатель преломления $n$ материала, из которого изготовлена линза.
}
\answer{%
    \begin{align*}
    D_1 &=\cbr{\frac n{n_1} - 1}\cbr{\frac 1{R_1} + \frac 1{R_2}}, \\
    D_2 &=\cbr{\frac n{n_2} - 1}\cbr{\frac 1{R_1} + \frac 1{R_2}}, \\
    \frac {D_2}{D_1} &=\frac{\frac n{n_2} - 1}{\frac n{n_1} - 1} \implies {D_2}\cbr{\frac n{n_1} - 1} = {D_1}\cbr{\frac n{n_2} - 1}  \implies n\cbr{\frac{D_2}{n_1} - \frac{D_1}{n_2}} = D_2 - D_1, \\
    n &= \frac{D_2 - D_1}{\frac{D_2}{n_1} - \frac{D_1}{n_2}} = \frac{n_1 n_2 (D_2 - D_1)}{D_2n_2 - D_1n_1} \approx 1{,}626.
    \end{align*}
}
\solutionspace{120pt}

\tasknumber{4}%
\task{%
    На каком расстоянии от собирающей линзы с фокусным расстоянием $40\,\text{дптр}$
    следует надо поместить предмет, чтобы расстояние
    от предмета до его действительного изображения было наименьшим?
}
\answer{%
    \begin{align*}
    \frac 1a &+ \frac 1b = D \implies b = \frac 1{D - \frac 1a} \implies \ell = a + b = a + \frac a{Da - 1} = \frac{ Da^2 }{Da - 1} \implies \\
    \implies \ell'_a &= \frac{ 2Da \cdot (Da - 1) - Da^2 \cdot D }{\sqr{Da - 1}}= \frac{ D^2a^2 - 2Da}{\sqr{Da - 1}} = \frac{ Da(Da - 2)}{\sqr{Da - 1}}\implies a_{\min} = \frac 2D \approx 50\,\text{мм}.
    \end{align*}
}
\solutionspace{120pt}

\tasknumber{5}%
\task{%
    Предмет в виде отрезка длиной $\ell$ расположен вдоль оптической оси
    собирающей линзы с фокусным расстоянием $F$.
    Середина отрезка расположена
    иа расстоянии $a$ от линзы, которая даёт действительное изображение
    всех точек предмета.
    Определить продольное увеличение предмета.
}
\answer{%
    \begin{align*}
    \frac 1{a + \frac \ell 2} &+ \frac 1b = \frac 1F \implies b = \frac{F\cbr{a + \frac \ell 2}}{a + \frac \ell 2 - F} \\
    \frac 1{a - \frac \ell 2} &+ \frac 1c = \frac 1F \implies c = \frac{F\cbr{a - \frac \ell 2}}{a - \frac \ell 2 - F} \\
    \abs{b - c} &= \abs{\frac{F\cbr{a + \frac \ell 2}}{a + \frac \ell 2 - F} - \frac{F\cbr{a - \frac \ell 2}}{a - \frac \ell 2 - F}}= F\abs{\frac{\cbr{a + \frac \ell 2}\cbr{a - \frac \ell 2 - F} - \cbr{a - \frac \ell 2}\cbr{a + \frac \ell 2 - F}}{ \cbr{a + \frac \ell 2 - F} \cbr{a - \frac \ell 2 - F} }} =  \\
    &= F\abs{\frac{a^2 - \frac {a\ell} 2 - Fa + \frac {a\ell} 2 - \frac {\ell^2} 4 - \frac {F\ell}2 - a^2 - \frac {a\ell}2 + aF + \frac {a\ell}2 + \frac {\ell^2} 4 - \frac {F\ell} 2}{\cbr{a + \frac \ell 2 - F} \cbr{a - \frac \ell 2 - F} }} = \\
    &= F\frac{F\ell}{\sqr{a-F} - \frac {\ell^2}4} = \frac{F^2\ell}{\sqr{a-F} - \frac {\ell^2}4}\implies \Gamma = \frac{\abs{b - c}}\ell = \frac{F^2}{\sqr{a-F} - \frac {\ell^2}4}.
    \end{align*}
}
\solutionspace{120pt}

\tasknumber{6}%
\task{%
    Даны точечный источник света $S$, его изображение $S_1$, полученное с помошью собирающей линзы,
    и ближайший к источнику фокус линзы $F$ (см.
    рис.
    на доске).
    Расстояния $SF = \ell$ и $SS_1 = L$.
    Определить положение линзы и её фокусное расстояние.
}
\answer{%
    \begin{align*}
    \frac 1a + \frac 1b &= \frac 1F, \ell = a - F, L = a + b \implies a = \ell + F, b = L - a = L - \ell - F \\
    \frac 1{\ell + F} + \frac 1{L - \ell - F} &= \frac 1F \\
    F\ell + F^2 + LF - F\ell - F^2 &= L\ell - \ell^2 - F\ell + LF - F\ell - F^2 \\
    0 &= L\ell - \ell^2 - 2F\ell - F^2 \\
    0 &=  F^2 + 2F\ell - L\ell + \ell^2 \\
    F &= -\ell \pm \sqrt{\ell^2 +  L\ell - \ell^2} = -\ell \pm \sqrt{L\ell} \implies F = \sqrt{L\ell} - \ell \\
    a &= \ell + F = \ell + \sqrt{L\ell} - \ell = \sqrt{L\ell}.
    \end{align*}
}
\solutionspace{120pt}

\tasknumber{7}%
\task{%
    Расстояние от освещённого предмета до экрана $100\,\text{см}$.
    Линза, помещенная между ними, даёт чёткое изображение предмета на
    экране при двух положениях, расстояние между которыми $40\,\text{см}$.
    Найти фокусное расстояние линзы.
}
\answer{%
    \begin{align*}
    \frac 1a + \frac 1b &= \frac 1F, \frac 1{a-\ell} + \frac 1{b+\ell} = \frac 1F, a + b = L \\
    \frac 1a + \frac 1b &= \frac 1{a-\ell} + \frac 1{b+\ell}\implies \frac{a + b}{ab} = \frac{(a-\ell) + (b+\ell)}{(a-\ell)(b+\ell)} \\
    ab  &= (a - \ell)(b+\ell) \implies 0  = -b\ell + a\ell - \ell^2 \implies 0 = -b + a - \ell \implies b = a - \ell \\
    a + (a - \ell) &= L \implies a = \frac{L + \ell}2 \implies b = \frac{L - \ell}2 \\
    F &= \frac{ab}{a + b} = \frac{L^2 -\ell^2}{4L} \approx 21\,\text{см}.
    \end{align*}
}
\solutionspace{120pt}

\tasknumber{8}%
\task{%
    Предмет находится на расстоянии $70\,\text{см}$ от экрана.
    Между предметом и экраном помещают линзу, причём при одном
    положении линзы на экране получается увеличенное изображение предмета,
    а при другом — уменьшенное.
    Каково фокусное расстояние линзы, если
    линейные размеры первого изображения в два раза больше второго?
}
\answer{%
    \begin{align*}
    \frac 1a + \frac 1{L-a} &= \frac 1F, h_1 = h \cdot \frac{L-a}a, \\
    \frac 1b + \frac 1{L-b} &= \frac 1F, h_2 = h \cdot \frac{L-b}b, \\
    \frac{h_1}{h_2} &= 2 \implies \frac{(L-a)b}{(L-b)a} = 2, \\
    \frac 1F &= \frac{ L }{a(L-a)} = \frac{ L }{b(L-b)} \implies \frac{L-a}{L-b} = \frac b a \implies \frac {b^2}{a^2} = 2.
    \\
    \frac 1a + \frac 1{L-a} &= \frac 1b + \frac 1{L-b} \implies \frac L{a(L-a)} = \frac L{b(L-b)} \implies \\
    \implies aL - a^2 &= bL - b^2 \implies (a-b)L = (a-b)(a+b) \implies b = L - a, \\
    \frac{\sqr{L-a}}{a^2} &= 2 \implies \frac La - 1 = \sqrt{2} \implies a = \frac{ L }{\sqrt{2} + 1} \\
    F &= \frac{a(L-a)}L = \frac 1L \cdot \frac L{\sqrt{2} + 1} \cdot \frac {L\sqrt{2}}{\sqrt{2} + 1}= \frac { L\sqrt{2} }{ \sqr{\sqrt{2} + 1} } \approx 17{,}0\,\text{см}.
    \end{align*}
}

\variantsplitter

\addpersonalvariant{Анна Кузьмичёва}

\tasknumber{1}%
\task{%
    В каком месте на главной оптической оси двояковыгнутой линзы
    нужно поместить точечный источник света,
    чтобы его изображение оказалось в главном фокусе линзы?
}
\answer{%
    $\text{на половине фокусного расстояния}$
}
\solutionspace{120pt}

\tasknumber{2}%
\task{%
    На экране, расположенном иа расстоянии $80\,\text{см}$ от собирающей линзы,
    получено изображение точечного источника, расположенного на главной оптической оси линзы.
    На какое расстояние переместится изображение на экране,
    если при неподвижной линзе переместить источник на $2\,\text{см}$ в плоскости, перпендикулярной главной оптической оси?
    Фокусное расстояние линзы равно $20\,\text{см}$.
}
\answer{%
    \begin{align*}
    &\frac 1F = \frac 1a + \frac 1b \implies a = \frac{bF}{b-F} \implies \Gamma = \frac ba = \frac{b-F}F \\
    &y = x \cdot \Gamma = x \cdot \frac{b-F}F \implies d = y = 6\,\text{см}.
    \end{align*}
}
\solutionspace{120pt}

\tasknumber{3}%
\task{%
    Оптическая сила двояковыпуклой линзы в воздухе $4{,}5\,\text{дптр}$, а в воде $1{,}6\,\text{дптр}$.
    Определить показатель преломления $n$ материала, из которого изготовлена линза.
}
\answer{%
    \begin{align*}
    D_1 &=\cbr{\frac n{n_1} - 1}\cbr{\frac 1{R_1} + \frac 1{R_2}}, \\
    D_2 &=\cbr{\frac n{n_2} - 1}\cbr{\frac 1{R_1} + \frac 1{R_2}}, \\
    \frac {D_2}{D_1} &=\frac{\frac n{n_2} - 1}{\frac n{n_1} - 1} \implies {D_2}\cbr{\frac n{n_1} - 1} = {D_1}\cbr{\frac n{n_2} - 1}  \implies n\cbr{\frac{D_2}{n_1} - \frac{D_1}{n_2}} = D_2 - D_1, \\
    n &= \frac{D_2 - D_1}{\frac{D_2}{n_1} - \frac{D_1}{n_2}} = \frac{n_1 n_2 (D_2 - D_1)}{D_2n_2 - D_1n_1} \approx 1{,}626.
    \end{align*}
}
\solutionspace{120pt}

\tasknumber{4}%
\task{%
    На каком расстоянии от собирающей линзы с фокусным расстоянием $50\,\text{дптр}$
    следует надо поместить предмет, чтобы расстояние
    от предмета до его действительного изображения было наименьшим?
}
\answer{%
    \begin{align*}
    \frac 1a &+ \frac 1b = D \implies b = \frac 1{D - \frac 1a} \implies \ell = a + b = a + \frac a{Da - 1} = \frac{ Da^2 }{Da - 1} \implies \\
    \implies \ell'_a &= \frac{ 2Da \cdot (Da - 1) - Da^2 \cdot D }{\sqr{Da - 1}}= \frac{ D^2a^2 - 2Da}{\sqr{Da - 1}} = \frac{ Da(Da - 2)}{\sqr{Da - 1}}\implies a_{\min} = \frac 2D \approx 40\,\text{мм}.
    \end{align*}
}
\solutionspace{120pt}

\tasknumber{5}%
\task{%
    Предмет в виде отрезка длиной $\ell$ расположен вдоль оптической оси
    собирающей линзы с фокусным расстоянием $F$.
    Середина отрезка расположена
    иа расстоянии $a$ от линзы, которая даёт действительное изображение
    всех точек предмета.
    Определить продольное увеличение предмета.
}
\answer{%
    \begin{align*}
    \frac 1{a + \frac \ell 2} &+ \frac 1b = \frac 1F \implies b = \frac{F\cbr{a + \frac \ell 2}}{a + \frac \ell 2 - F} \\
    \frac 1{a - \frac \ell 2} &+ \frac 1c = \frac 1F \implies c = \frac{F\cbr{a - \frac \ell 2}}{a - \frac \ell 2 - F} \\
    \abs{b - c} &= \abs{\frac{F\cbr{a + \frac \ell 2}}{a + \frac \ell 2 - F} - \frac{F\cbr{a - \frac \ell 2}}{a - \frac \ell 2 - F}}= F\abs{\frac{\cbr{a + \frac \ell 2}\cbr{a - \frac \ell 2 - F} - \cbr{a - \frac \ell 2}\cbr{a + \frac \ell 2 - F}}{ \cbr{a + \frac \ell 2 - F} \cbr{a - \frac \ell 2 - F} }} =  \\
    &= F\abs{\frac{a^2 - \frac {a\ell} 2 - Fa + \frac {a\ell} 2 - \frac {\ell^2} 4 - \frac {F\ell}2 - a^2 - \frac {a\ell}2 + aF + \frac {a\ell}2 + \frac {\ell^2} 4 - \frac {F\ell} 2}{\cbr{a + \frac \ell 2 - F} \cbr{a - \frac \ell 2 - F} }} = \\
    &= F\frac{F\ell}{\sqr{a-F} - \frac {\ell^2}4} = \frac{F^2\ell}{\sqr{a-F} - \frac {\ell^2}4}\implies \Gamma = \frac{\abs{b - c}}\ell = \frac{F^2}{\sqr{a-F} - \frac {\ell^2}4}.
    \end{align*}
}
\solutionspace{120pt}

\tasknumber{6}%
\task{%
    Даны точечный источник света $S$, его изображение $S_1$, полученное с помошью собирающей линзы,
    и ближайший к источнику фокус линзы $F$ (см.
    рис.
    на доске).
    Расстояния $SF = \ell$ и $SS_1 = L$.
    Определить положение линзы и её фокусное расстояние.
}
\answer{%
    \begin{align*}
    \frac 1a + \frac 1b &= \frac 1F, \ell = a - F, L = a + b \implies a = \ell + F, b = L - a = L - \ell - F \\
    \frac 1{\ell + F} + \frac 1{L - \ell - F} &= \frac 1F \\
    F\ell + F^2 + LF - F\ell - F^2 &= L\ell - \ell^2 - F\ell + LF - F\ell - F^2 \\
    0 &= L\ell - \ell^2 - 2F\ell - F^2 \\
    0 &=  F^2 + 2F\ell - L\ell + \ell^2 \\
    F &= -\ell \pm \sqrt{\ell^2 +  L\ell - \ell^2} = -\ell \pm \sqrt{L\ell} \implies F = \sqrt{L\ell} - \ell \\
    a &= \ell + F = \ell + \sqrt{L\ell} - \ell = \sqrt{L\ell}.
    \end{align*}
}
\solutionspace{120pt}

\tasknumber{7}%
\task{%
    Расстояние от освещённого предмета до экрана $80\,\text{см}$.
    Линза, помещенная между ними, даёт чёткое изображение предмета на
    экране при двух положениях, расстояние между которыми $20\,\text{см}$.
    Найти фокусное расстояние линзы.
}
\answer{%
    \begin{align*}
    \frac 1a + \frac 1b &= \frac 1F, \frac 1{a-\ell} + \frac 1{b+\ell} = \frac 1F, a + b = L \\
    \frac 1a + \frac 1b &= \frac 1{a-\ell} + \frac 1{b+\ell}\implies \frac{a + b}{ab} = \frac{(a-\ell) + (b+\ell)}{(a-\ell)(b+\ell)} \\
    ab  &= (a - \ell)(b+\ell) \implies 0  = -b\ell + a\ell - \ell^2 \implies 0 = -b + a - \ell \implies b = a - \ell \\
    a + (a - \ell) &= L \implies a = \frac{L + \ell}2 \implies b = \frac{L - \ell}2 \\
    F &= \frac{ab}{a + b} = \frac{L^2 -\ell^2}{4L} \approx 18{,}8\,\text{см}.
    \end{align*}
}
\solutionspace{120pt}

\tasknumber{8}%
\task{%
    Предмет находится на расстоянии $70\,\text{см}$ от экрана.
    Между предметом и экраном помещают линзу, причём при одном
    положении линзы на экране получается увеличенное изображение предмета,
    а при другом — уменьшенное.
    Каково фокусное расстояние линзы, если
    линейные размеры первого изображения в два раза больше второго?
}
\answer{%
    \begin{align*}
    \frac 1a + \frac 1{L-a} &= \frac 1F, h_1 = h \cdot \frac{L-a}a, \\
    \frac 1b + \frac 1{L-b} &= \frac 1F, h_2 = h \cdot \frac{L-b}b, \\
    \frac{h_1}{h_2} &= 2 \implies \frac{(L-a)b}{(L-b)a} = 2, \\
    \frac 1F &= \frac{ L }{a(L-a)} = \frac{ L }{b(L-b)} \implies \frac{L-a}{L-b} = \frac b a \implies \frac {b^2}{a^2} = 2.
    \\
    \frac 1a + \frac 1{L-a} &= \frac 1b + \frac 1{L-b} \implies \frac L{a(L-a)} = \frac L{b(L-b)} \implies \\
    \implies aL - a^2 &= bL - b^2 \implies (a-b)L = (a-b)(a+b) \implies b = L - a, \\
    \frac{\sqr{L-a}}{a^2} &= 2 \implies \frac La - 1 = \sqrt{2} \implies a = \frac{ L }{\sqrt{2} + 1} \\
    F &= \frac{a(L-a)}L = \frac 1L \cdot \frac L{\sqrt{2} + 1} \cdot \frac {L\sqrt{2}}{\sqrt{2} + 1}= \frac { L\sqrt{2} }{ \sqr{\sqrt{2} + 1} } \approx 17{,}0\,\text{см}.
    \end{align*}
}

\variantsplitter

\addpersonalvariant{Алёна Куприянова}

\tasknumber{1}%
\task{%
    В каком месте на главной оптической оси двояковыпуклой линзы
    нужно поместить точечный источник света,
    чтобы его изображение оказалось в главном фокусе линзы?
}
\answer{%
    $\text{для мнимого - на половине фокусного, для действительного - на бесконечности}$
}
\solutionspace{120pt}

\tasknumber{2}%
\task{%
    На экране, расположенном иа расстоянии $120\,\text{см}$ от собирающей линзы,
    получено изображение точечного источника, расположенного на главной оптической оси линзы.
    На какое расстояние переместится изображение на экране,
    если при неподвижной линзе переместить источник на $1\,\text{см}$ в плоскости, перпендикулярной главной оптической оси?
    Фокусное расстояние линзы равно $30\,\text{см}$.
}
\answer{%
    \begin{align*}
    &\frac 1F = \frac 1a + \frac 1b \implies a = \frac{bF}{b-F} \implies \Gamma = \frac ba = \frac{b-F}F \\
    &y = x \cdot \Gamma = x \cdot \frac{b-F}F \implies d = y = 3\,\text{см}.
    \end{align*}
}
\solutionspace{120pt}

\tasknumber{3}%
\task{%
    Оптическая сила двояковыпуклой линзы в воздухе $5\,\text{дптр}$, а в воде $1{,}5\,\text{дптр}$.
    Определить показатель преломления $n$ материала, из которого изготовлена линза.
}
\answer{%
    \begin{align*}
    D_1 &=\cbr{\frac n{n_1} - 1}\cbr{\frac 1{R_1} + \frac 1{R_2}}, \\
    D_2 &=\cbr{\frac n{n_2} - 1}\cbr{\frac 1{R_1} + \frac 1{R_2}}, \\
    \frac {D_2}{D_1} &=\frac{\frac n{n_2} - 1}{\frac n{n_1} - 1} \implies {D_2}\cbr{\frac n{n_1} - 1} = {D_1}\cbr{\frac n{n_2} - 1}  \implies n\cbr{\frac{D_2}{n_1} - \frac{D_1}{n_2}} = D_2 - D_1, \\
    n &= \frac{D_2 - D_1}{\frac{D_2}{n_1} - \frac{D_1}{n_2}} = \frac{n_1 n_2 (D_2 - D_1)}{D_2n_2 - D_1n_1} \approx 1{,}549.
    \end{align*}
}
\solutionspace{120pt}

\tasknumber{4}%
\task{%
    На каком расстоянии от собирающей линзы с фокусным расстоянием $40\,\text{дптр}$
    следует надо поместить предмет, чтобы расстояние
    от предмета до его действительного изображения было наименьшим?
}
\answer{%
    \begin{align*}
    \frac 1a &+ \frac 1b = D \implies b = \frac 1{D - \frac 1a} \implies \ell = a + b = a + \frac a{Da - 1} = \frac{ Da^2 }{Da - 1} \implies \\
    \implies \ell'_a &= \frac{ 2Da \cdot (Da - 1) - Da^2 \cdot D }{\sqr{Da - 1}}= \frac{ D^2a^2 - 2Da}{\sqr{Da - 1}} = \frac{ Da(Da - 2)}{\sqr{Da - 1}}\implies a_{\min} = \frac 2D \approx 50\,\text{мм}.
    \end{align*}
}
\solutionspace{120pt}

\tasknumber{5}%
\task{%
    Предмет в виде отрезка длиной $\ell$ расположен вдоль оптической оси
    собирающей линзы с фокусным расстоянием $F$.
    Середина отрезка расположена
    иа расстоянии $a$ от линзы, которая даёт действительное изображение
    всех точек предмета.
    Определить продольное увеличение предмета.
}
\answer{%
    \begin{align*}
    \frac 1{a + \frac \ell 2} &+ \frac 1b = \frac 1F \implies b = \frac{F\cbr{a + \frac \ell 2}}{a + \frac \ell 2 - F} \\
    \frac 1{a - \frac \ell 2} &+ \frac 1c = \frac 1F \implies c = \frac{F\cbr{a - \frac \ell 2}}{a - \frac \ell 2 - F} \\
    \abs{b - c} &= \abs{\frac{F\cbr{a + \frac \ell 2}}{a + \frac \ell 2 - F} - \frac{F\cbr{a - \frac \ell 2}}{a - \frac \ell 2 - F}}= F\abs{\frac{\cbr{a + \frac \ell 2}\cbr{a - \frac \ell 2 - F} - \cbr{a - \frac \ell 2}\cbr{a + \frac \ell 2 - F}}{ \cbr{a + \frac \ell 2 - F} \cbr{a - \frac \ell 2 - F} }} =  \\
    &= F\abs{\frac{a^2 - \frac {a\ell} 2 - Fa + \frac {a\ell} 2 - \frac {\ell^2} 4 - \frac {F\ell}2 - a^2 - \frac {a\ell}2 + aF + \frac {a\ell}2 + \frac {\ell^2} 4 - \frac {F\ell} 2}{\cbr{a + \frac \ell 2 - F} \cbr{a - \frac \ell 2 - F} }} = \\
    &= F\frac{F\ell}{\sqr{a-F} - \frac {\ell^2}4} = \frac{F^2\ell}{\sqr{a-F} - \frac {\ell^2}4}\implies \Gamma = \frac{\abs{b - c}}\ell = \frac{F^2}{\sqr{a-F} - \frac {\ell^2}4}.
    \end{align*}
}
\solutionspace{120pt}

\tasknumber{6}%
\task{%
    Даны точечный источник света $S$, его изображение $S_1$, полученное с помошью собирающей линзы,
    и ближайший к источнику фокус линзы $F$ (см.
    рис.
    на доске).
    Расстояния $SF = \ell$ и $SS_1 = L$.
    Определить положение линзы и её фокусное расстояние.
}
\answer{%
    \begin{align*}
    \frac 1a + \frac 1b &= \frac 1F, \ell = a - F, L = a + b \implies a = \ell + F, b = L - a = L - \ell - F \\
    \frac 1{\ell + F} + \frac 1{L - \ell - F} &= \frac 1F \\
    F\ell + F^2 + LF - F\ell - F^2 &= L\ell - \ell^2 - F\ell + LF - F\ell - F^2 \\
    0 &= L\ell - \ell^2 - 2F\ell - F^2 \\
    0 &=  F^2 + 2F\ell - L\ell + \ell^2 \\
    F &= -\ell \pm \sqrt{\ell^2 +  L\ell - \ell^2} = -\ell \pm \sqrt{L\ell} \implies F = \sqrt{L\ell} - \ell \\
    a &= \ell + F = \ell + \sqrt{L\ell} - \ell = \sqrt{L\ell}.
    \end{align*}
}
\solutionspace{120pt}

\tasknumber{7}%
\task{%
    Расстояние от освещённого предмета до экрана $80\,\text{см}$.
    Линза, помещенная между ними, даёт чёткое изображение предмета на
    экране при двух положениях, расстояние между которыми $40\,\text{см}$.
    Найти фокусное расстояние линзы.
}
\answer{%
    \begin{align*}
    \frac 1a + \frac 1b &= \frac 1F, \frac 1{a-\ell} + \frac 1{b+\ell} = \frac 1F, a + b = L \\
    \frac 1a + \frac 1b &= \frac 1{a-\ell} + \frac 1{b+\ell}\implies \frac{a + b}{ab} = \frac{(a-\ell) + (b+\ell)}{(a-\ell)(b+\ell)} \\
    ab  &= (a - \ell)(b+\ell) \implies 0  = -b\ell + a\ell - \ell^2 \implies 0 = -b + a - \ell \implies b = a - \ell \\
    a + (a - \ell) &= L \implies a = \frac{L + \ell}2 \implies b = \frac{L - \ell}2 \\
    F &= \frac{ab}{a + b} = \frac{L^2 -\ell^2}{4L} \approx 15\,\text{см}.
    \end{align*}
}
\solutionspace{120pt}

\tasknumber{8}%
\task{%
    Предмет находится на расстоянии $60\,\text{см}$ от экрана.
    Между предметом и экраном помещают линзу, причём при одном
    положении линзы на экране получается увеличенное изображение предмета,
    а при другом — уменьшенное.
    Каково фокусное расстояние линзы, если
    линейные размеры первого изображения в три раза больше второго?
}
\answer{%
    \begin{align*}
    \frac 1a + \frac 1{L-a} &= \frac 1F, h_1 = h \cdot \frac{L-a}a, \\
    \frac 1b + \frac 1{L-b} &= \frac 1F, h_2 = h \cdot \frac{L-b}b, \\
    \frac{h_1}{h_2} &= 3 \implies \frac{(L-a)b}{(L-b)a} = 3, \\
    \frac 1F &= \frac{ L }{a(L-a)} = \frac{ L }{b(L-b)} \implies \frac{L-a}{L-b} = \frac b a \implies \frac {b^2}{a^2} = 3.
    \\
    \frac 1a + \frac 1{L-a} &= \frac 1b + \frac 1{L-b} \implies \frac L{a(L-a)} = \frac L{b(L-b)} \implies \\
    \implies aL - a^2 &= bL - b^2 \implies (a-b)L = (a-b)(a+b) \implies b = L - a, \\
    \frac{\sqr{L-a}}{a^2} &= 3 \implies \frac La - 1 = \sqrt{3} \implies a = \frac{ L }{\sqrt{3} + 1} \\
    F &= \frac{a(L-a)}L = \frac 1L \cdot \frac L{\sqrt{3} + 1} \cdot \frac {L\sqrt{3}}{\sqrt{3} + 1}= \frac { L\sqrt{3} }{ \sqr{\sqrt{3} + 1} } \approx 13{,}9\,\text{см}.
    \end{align*}
}

\variantsplitter

\addpersonalvariant{Ярослав Лавровский}

\tasknumber{1}%
\task{%
    В каком месте на главной оптической оси двояковыгнутой линзы
    нужно поместить точечный источник света,
    чтобы его изображение оказалось в главном фокусе линзы?
}
\answer{%
    $\text{на половине фокусного расстояния}$
}
\solutionspace{120pt}

\tasknumber{2}%
\task{%
    На экране, расположенном иа расстоянии $120\,\text{см}$ от собирающей линзы,
    получено изображение точечного источника, расположенного на главной оптической оси линзы.
    На какое расстояние переместится изображение на экране,
    если при неподвижной линзе переместить источник на $1\,\text{см}$ в плоскости, перпендикулярной главной оптической оси?
    Фокусное расстояние линзы равно $20\,\text{см}$.
}
\answer{%
    \begin{align*}
    &\frac 1F = \frac 1a + \frac 1b \implies a = \frac{bF}{b-F} \implies \Gamma = \frac ba = \frac{b-F}F \\
    &y = x \cdot \Gamma = x \cdot \frac{b-F}F \implies d = y = 5\,\text{см}.
    \end{align*}
}
\solutionspace{120pt}

\tasknumber{3}%
\task{%
    Оптическая сила двояковыпуклой линзы в воздухе $4{,}5\,\text{дптр}$, а в воде $1{,}5\,\text{дптр}$.
    Определить показатель преломления $n$ материала, из которого изготовлена линза.
}
\answer{%
    \begin{align*}
    D_1 &=\cbr{\frac n{n_1} - 1}\cbr{\frac 1{R_1} + \frac 1{R_2}}, \\
    D_2 &=\cbr{\frac n{n_2} - 1}\cbr{\frac 1{R_1} + \frac 1{R_2}}, \\
    \frac {D_2}{D_1} &=\frac{\frac n{n_2} - 1}{\frac n{n_1} - 1} \implies {D_2}\cbr{\frac n{n_1} - 1} = {D_1}\cbr{\frac n{n_2} - 1}  \implies n\cbr{\frac{D_2}{n_1} - \frac{D_1}{n_2}} = D_2 - D_1, \\
    n &= \frac{D_2 - D_1}{\frac{D_2}{n_1} - \frac{D_1}{n_2}} = \frac{n_1 n_2 (D_2 - D_1)}{D_2n_2 - D_1n_1} \approx 1{,}593.
    \end{align*}
}
\solutionspace{120pt}

\tasknumber{4}%
\task{%
    На каком расстоянии от собирающей линзы с фокусным расстоянием $50\,\text{дптр}$
    следует надо поместить предмет, чтобы расстояние
    от предмета до его действительного изображения было наименьшим?
}
\answer{%
    \begin{align*}
    \frac 1a &+ \frac 1b = D \implies b = \frac 1{D - \frac 1a} \implies \ell = a + b = a + \frac a{Da - 1} = \frac{ Da^2 }{Da - 1} \implies \\
    \implies \ell'_a &= \frac{ 2Da \cdot (Da - 1) - Da^2 \cdot D }{\sqr{Da - 1}}= \frac{ D^2a^2 - 2Da}{\sqr{Da - 1}} = \frac{ Da(Da - 2)}{\sqr{Da - 1}}\implies a_{\min} = \frac 2D \approx 40\,\text{мм}.
    \end{align*}
}
\solutionspace{120pt}

\tasknumber{5}%
\task{%
    Предмет в виде отрезка длиной $\ell$ расположен вдоль оптической оси
    собирающей линзы с фокусным расстоянием $F$.
    Середина отрезка расположена
    иа расстоянии $a$ от линзы, которая даёт действительное изображение
    всех точек предмета.
    Определить продольное увеличение предмета.
}
\answer{%
    \begin{align*}
    \frac 1{a + \frac \ell 2} &+ \frac 1b = \frac 1F \implies b = \frac{F\cbr{a + \frac \ell 2}}{a + \frac \ell 2 - F} \\
    \frac 1{a - \frac \ell 2} &+ \frac 1c = \frac 1F \implies c = \frac{F\cbr{a - \frac \ell 2}}{a - \frac \ell 2 - F} \\
    \abs{b - c} &= \abs{\frac{F\cbr{a + \frac \ell 2}}{a + \frac \ell 2 - F} - \frac{F\cbr{a - \frac \ell 2}}{a - \frac \ell 2 - F}}= F\abs{\frac{\cbr{a + \frac \ell 2}\cbr{a - \frac \ell 2 - F} - \cbr{a - \frac \ell 2}\cbr{a + \frac \ell 2 - F}}{ \cbr{a + \frac \ell 2 - F} \cbr{a - \frac \ell 2 - F} }} =  \\
    &= F\abs{\frac{a^2 - \frac {a\ell} 2 - Fa + \frac {a\ell} 2 - \frac {\ell^2} 4 - \frac {F\ell}2 - a^2 - \frac {a\ell}2 + aF + \frac {a\ell}2 + \frac {\ell^2} 4 - \frac {F\ell} 2}{\cbr{a + \frac \ell 2 - F} \cbr{a - \frac \ell 2 - F} }} = \\
    &= F\frac{F\ell}{\sqr{a-F} - \frac {\ell^2}4} = \frac{F^2\ell}{\sqr{a-F} - \frac {\ell^2}4}\implies \Gamma = \frac{\abs{b - c}}\ell = \frac{F^2}{\sqr{a-F} - \frac {\ell^2}4}.
    \end{align*}
}
\solutionspace{120pt}

\tasknumber{6}%
\task{%
    Даны точечный источник света $S$, его изображение $S_1$, полученное с помошью собирающей линзы,
    и ближайший к источнику фокус линзы $F$ (см.
    рис.
    на доске).
    Расстояния $SF = \ell$ и $SS_1 = L$.
    Определить положение линзы и её фокусное расстояние.
}
\answer{%
    \begin{align*}
    \frac 1a + \frac 1b &= \frac 1F, \ell = a - F, L = a + b \implies a = \ell + F, b = L - a = L - \ell - F \\
    \frac 1{\ell + F} + \frac 1{L - \ell - F} &= \frac 1F \\
    F\ell + F^2 + LF - F\ell - F^2 &= L\ell - \ell^2 - F\ell + LF - F\ell - F^2 \\
    0 &= L\ell - \ell^2 - 2F\ell - F^2 \\
    0 &=  F^2 + 2F\ell - L\ell + \ell^2 \\
    F &= -\ell \pm \sqrt{\ell^2 +  L\ell - \ell^2} = -\ell \pm \sqrt{L\ell} \implies F = \sqrt{L\ell} - \ell \\
    a &= \ell + F = \ell + \sqrt{L\ell} - \ell = \sqrt{L\ell}.
    \end{align*}
}
\solutionspace{120pt}

\tasknumber{7}%
\task{%
    Расстояние от освещённого предмета до экрана $100\,\text{см}$.
    Линза, помещенная между ними, даёт чёткое изображение предмета на
    экране при двух положениях, расстояние между которыми $20\,\text{см}$.
    Найти фокусное расстояние линзы.
}
\answer{%
    \begin{align*}
    \frac 1a + \frac 1b &= \frac 1F, \frac 1{a-\ell} + \frac 1{b+\ell} = \frac 1F, a + b = L \\
    \frac 1a + \frac 1b &= \frac 1{a-\ell} + \frac 1{b+\ell}\implies \frac{a + b}{ab} = \frac{(a-\ell) + (b+\ell)}{(a-\ell)(b+\ell)} \\
    ab  &= (a - \ell)(b+\ell) \implies 0  = -b\ell + a\ell - \ell^2 \implies 0 = -b + a - \ell \implies b = a - \ell \\
    a + (a - \ell) &= L \implies a = \frac{L + \ell}2 \implies b = \frac{L - \ell}2 \\
    F &= \frac{ab}{a + b} = \frac{L^2 -\ell^2}{4L} \approx 24\,\text{см}.
    \end{align*}
}
\solutionspace{120pt}

\tasknumber{8}%
\task{%
    Предмет находится на расстоянии $80\,\text{см}$ от экрана.
    Между предметом и экраном помещают линзу, причём при одном
    положении линзы на экране получается увеличенное изображение предмета,
    а при другом — уменьшенное.
    Каково фокусное расстояние линзы, если
    линейные размеры первого изображения в три раза больше второго?
}
\answer{%
    \begin{align*}
    \frac 1a + \frac 1{L-a} &= \frac 1F, h_1 = h \cdot \frac{L-a}a, \\
    \frac 1b + \frac 1{L-b} &= \frac 1F, h_2 = h \cdot \frac{L-b}b, \\
    \frac{h_1}{h_2} &= 3 \implies \frac{(L-a)b}{(L-b)a} = 3, \\
    \frac 1F &= \frac{ L }{a(L-a)} = \frac{ L }{b(L-b)} \implies \frac{L-a}{L-b} = \frac b a \implies \frac {b^2}{a^2} = 3.
    \\
    \frac 1a + \frac 1{L-a} &= \frac 1b + \frac 1{L-b} \implies \frac L{a(L-a)} = \frac L{b(L-b)} \implies \\
    \implies aL - a^2 &= bL - b^2 \implies (a-b)L = (a-b)(a+b) \implies b = L - a, \\
    \frac{\sqr{L-a}}{a^2} &= 3 \implies \frac La - 1 = \sqrt{3} \implies a = \frac{ L }{\sqrt{3} + 1} \\
    F &= \frac{a(L-a)}L = \frac 1L \cdot \frac L{\sqrt{3} + 1} \cdot \frac {L\sqrt{3}}{\sqrt{3} + 1}= \frac { L\sqrt{3} }{ \sqr{\sqrt{3} + 1} } \approx 18{,}6\,\text{см}.
    \end{align*}
}

\variantsplitter

\addpersonalvariant{Анастасия Ламанова}

\tasknumber{1}%
\task{%
    В каком месте на главной оптической оси двояковыгнутой линзы
    нужно поместить точечный источник света,
    чтобы его изображение оказалось в главном фокусе линзы?
}
\answer{%
    $\text{на половине фокусного расстояния}$
}
\solutionspace{120pt}

\tasknumber{2}%
\task{%
    На экране, расположенном иа расстоянии $80\,\text{см}$ от собирающей линзы,
    получено изображение точечного источника, расположенного на главной оптической оси линзы.
    На какое расстояние переместится изображение на экране,
    если при неподвижном источнике переместить линзу на $1\,\text{см}$ в плоскости, перпендикулярной главной оптической оси?
    Фокусное расстояние линзы равно $40\,\text{см}$.
}
\answer{%
    \begin{align*}
    &\frac 1F = \frac 1a + \frac 1b \implies a = \frac{bF}{b-F} \implies \Gamma = \frac ba = \frac{b-F}F \\
    &y = x \cdot \Gamma = x \cdot \frac{b-F}F \implies d = x + y = 2\,\text{см}.
    \end{align*}
}
\solutionspace{120pt}

\tasknumber{3}%
\task{%
    Оптическая сила двояковыпуклой линзы в воздухе $5\,\text{дптр}$, а в воде $1{,}6\,\text{дптр}$.
    Определить показатель преломления $n$ материала, из которого изготовлена линза.
}
\answer{%
    \begin{align*}
    D_1 &=\cbr{\frac n{n_1} - 1}\cbr{\frac 1{R_1} + \frac 1{R_2}}, \\
    D_2 &=\cbr{\frac n{n_2} - 1}\cbr{\frac 1{R_1} + \frac 1{R_2}}, \\
    \frac {D_2}{D_1} &=\frac{\frac n{n_2} - 1}{\frac n{n_1} - 1} \implies {D_2}\cbr{\frac n{n_1} - 1} = {D_1}\cbr{\frac n{n_2} - 1}  \implies n\cbr{\frac{D_2}{n_1} - \frac{D_1}{n_2}} = D_2 - D_1, \\
    n &= \frac{D_2 - D_1}{\frac{D_2}{n_1} - \frac{D_1}{n_2}} = \frac{n_1 n_2 (D_2 - D_1)}{D_2n_2 - D_1n_1} \approx 1{,}575.
    \end{align*}
}
\solutionspace{120pt}

\tasknumber{4}%
\task{%
    На каком расстоянии от собирающей линзы с фокусным расстоянием $30\,\text{дптр}$
    следует надо поместить предмет, чтобы расстояние
    от предмета до его действительного изображения было наименьшим?
}
\answer{%
    \begin{align*}
    \frac 1a &+ \frac 1b = D \implies b = \frac 1{D - \frac 1a} \implies \ell = a + b = a + \frac a{Da - 1} = \frac{ Da^2 }{Da - 1} \implies \\
    \implies \ell'_a &= \frac{ 2Da \cdot (Da - 1) - Da^2 \cdot D }{\sqr{Da - 1}}= \frac{ D^2a^2 - 2Da}{\sqr{Da - 1}} = \frac{ Da(Da - 2)}{\sqr{Da - 1}}\implies a_{\min} = \frac 2D \approx 66{,}7\,\text{мм}.
    \end{align*}
}
\solutionspace{120pt}

\tasknumber{5}%
\task{%
    Предмет в виде отрезка длиной $\ell$ расположен вдоль оптической оси
    собирающей линзы с фокусным расстоянием $F$.
    Середина отрезка расположена
    иа расстоянии $a$ от линзы, которая даёт действительное изображение
    всех точек предмета.
    Определить продольное увеличение предмета.
}
\answer{%
    \begin{align*}
    \frac 1{a + \frac \ell 2} &+ \frac 1b = \frac 1F \implies b = \frac{F\cbr{a + \frac \ell 2}}{a + \frac \ell 2 - F} \\
    \frac 1{a - \frac \ell 2} &+ \frac 1c = \frac 1F \implies c = \frac{F\cbr{a - \frac \ell 2}}{a - \frac \ell 2 - F} \\
    \abs{b - c} &= \abs{\frac{F\cbr{a + \frac \ell 2}}{a + \frac \ell 2 - F} - \frac{F\cbr{a - \frac \ell 2}}{a - \frac \ell 2 - F}}= F\abs{\frac{\cbr{a + \frac \ell 2}\cbr{a - \frac \ell 2 - F} - \cbr{a - \frac \ell 2}\cbr{a + \frac \ell 2 - F}}{ \cbr{a + \frac \ell 2 - F} \cbr{a - \frac \ell 2 - F} }} =  \\
    &= F\abs{\frac{a^2 - \frac {a\ell} 2 - Fa + \frac {a\ell} 2 - \frac {\ell^2} 4 - \frac {F\ell}2 - a^2 - \frac {a\ell}2 + aF + \frac {a\ell}2 + \frac {\ell^2} 4 - \frac {F\ell} 2}{\cbr{a + \frac \ell 2 - F} \cbr{a - \frac \ell 2 - F} }} = \\
    &= F\frac{F\ell}{\sqr{a-F} - \frac {\ell^2}4} = \frac{F^2\ell}{\sqr{a-F} - \frac {\ell^2}4}\implies \Gamma = \frac{\abs{b - c}}\ell = \frac{F^2}{\sqr{a-F} - \frac {\ell^2}4}.
    \end{align*}
}
\solutionspace{120pt}

\tasknumber{6}%
\task{%
    Даны точечный источник света $S$, его изображение $S_1$, полученное с помошью собирающей линзы,
    и ближайший к источнику фокус линзы $F$ (см.
    рис.
    на доске).
    Расстояния $SF = \ell$ и $SS_1 = L$.
    Определить положение линзы и её фокусное расстояние.
}
\answer{%
    \begin{align*}
    \frac 1a + \frac 1b &= \frac 1F, \ell = a - F, L = a + b \implies a = \ell + F, b = L - a = L - \ell - F \\
    \frac 1{\ell + F} + \frac 1{L - \ell - F} &= \frac 1F \\
    F\ell + F^2 + LF - F\ell - F^2 &= L\ell - \ell^2 - F\ell + LF - F\ell - F^2 \\
    0 &= L\ell - \ell^2 - 2F\ell - F^2 \\
    0 &=  F^2 + 2F\ell - L\ell + \ell^2 \\
    F &= -\ell \pm \sqrt{\ell^2 +  L\ell - \ell^2} = -\ell \pm \sqrt{L\ell} \implies F = \sqrt{L\ell} - \ell \\
    a &= \ell + F = \ell + \sqrt{L\ell} - \ell = \sqrt{L\ell}.
    \end{align*}
}
\solutionspace{120pt}

\tasknumber{7}%
\task{%
    Расстояние от освещённого предмета до экрана $80\,\text{см}$.
    Линза, помещенная между ними, даёт чёткое изображение предмета на
    экране при двух положениях, расстояние между которыми $20\,\text{см}$.
    Найти фокусное расстояние линзы.
}
\answer{%
    \begin{align*}
    \frac 1a + \frac 1b &= \frac 1F, \frac 1{a-\ell} + \frac 1{b+\ell} = \frac 1F, a + b = L \\
    \frac 1a + \frac 1b &= \frac 1{a-\ell} + \frac 1{b+\ell}\implies \frac{a + b}{ab} = \frac{(a-\ell) + (b+\ell)}{(a-\ell)(b+\ell)} \\
    ab  &= (a - \ell)(b+\ell) \implies 0  = -b\ell + a\ell - \ell^2 \implies 0 = -b + a - \ell \implies b = a - \ell \\
    a + (a - \ell) &= L \implies a = \frac{L + \ell}2 \implies b = \frac{L - \ell}2 \\
    F &= \frac{ab}{a + b} = \frac{L^2 -\ell^2}{4L} \approx 18{,}8\,\text{см}.
    \end{align*}
}
\solutionspace{120pt}

\tasknumber{8}%
\task{%
    Предмет находится на расстоянии $90\,\text{см}$ от экрана.
    Между предметом и экраном помещают линзу, причём при одном
    положении линзы на экране получается увеличенное изображение предмета,
    а при другом — уменьшенное.
    Каково фокусное расстояние линзы, если
    линейные размеры первого изображения в три раза больше второго?
}
\answer{%
    \begin{align*}
    \frac 1a + \frac 1{L-a} &= \frac 1F, h_1 = h \cdot \frac{L-a}a, \\
    \frac 1b + \frac 1{L-b} &= \frac 1F, h_2 = h \cdot \frac{L-b}b, \\
    \frac{h_1}{h_2} &= 3 \implies \frac{(L-a)b}{(L-b)a} = 3, \\
    \frac 1F &= \frac{ L }{a(L-a)} = \frac{ L }{b(L-b)} \implies \frac{L-a}{L-b} = \frac b a \implies \frac {b^2}{a^2} = 3.
    \\
    \frac 1a + \frac 1{L-a} &= \frac 1b + \frac 1{L-b} \implies \frac L{a(L-a)} = \frac L{b(L-b)} \implies \\
    \implies aL - a^2 &= bL - b^2 \implies (a-b)L = (a-b)(a+b) \implies b = L - a, \\
    \frac{\sqr{L-a}}{a^2} &= 3 \implies \frac La - 1 = \sqrt{3} \implies a = \frac{ L }{\sqrt{3} + 1} \\
    F &= \frac{a(L-a)}L = \frac 1L \cdot \frac L{\sqrt{3} + 1} \cdot \frac {L\sqrt{3}}{\sqrt{3} + 1}= \frac { L\sqrt{3} }{ \sqr{\sqrt{3} + 1} } \approx 21\,\text{см}.
    \end{align*}
}

\variantsplitter

\addpersonalvariant{Виктория Легонькова}

\tasknumber{1}%
\task{%
    В каком месте на главной оптической оси двояковыпуклой линзы
    нужно поместить точечный источник света,
    чтобы его изображение оказалось в главном фокусе линзы?
}
\answer{%
    $\text{для мнимого - на половине фокусного, для действительного - на бесконечности}$
}
\solutionspace{120pt}

\tasknumber{2}%
\task{%
    На экране, расположенном иа расстоянии $120\,\text{см}$ от собирающей линзы,
    получено изображение точечного источника, расположенного на главной оптической оси линзы.
    На какое расстояние переместится изображение на экране,
    если при неподвижной линзе переместить источник на $1\,\text{см}$ в плоскости, перпендикулярной главной оптической оси?
    Фокусное расстояние линзы равно $40\,\text{см}$.
}
\answer{%
    \begin{align*}
    &\frac 1F = \frac 1a + \frac 1b \implies a = \frac{bF}{b-F} \implies \Gamma = \frac ba = \frac{b-F}F \\
    &y = x \cdot \Gamma = x \cdot \frac{b-F}F \implies d = y = 2\,\text{см}.
    \end{align*}
}
\solutionspace{120pt}

\tasknumber{3}%
\task{%
    Оптическая сила двояковыпуклой линзы в воздухе $4{,}5\,\text{дптр}$, а в воде $1{,}4\,\text{дптр}$.
    Определить показатель преломления $n$ материала, из которого изготовлена линза.
}
\answer{%
    \begin{align*}
    D_1 &=\cbr{\frac n{n_1} - 1}\cbr{\frac 1{R_1} + \frac 1{R_2}}, \\
    D_2 &=\cbr{\frac n{n_2} - 1}\cbr{\frac 1{R_1} + \frac 1{R_2}}, \\
    \frac {D_2}{D_1} &=\frac{\frac n{n_2} - 1}{\frac n{n_1} - 1} \implies {D_2}\cbr{\frac n{n_1} - 1} = {D_1}\cbr{\frac n{n_2} - 1}  \implies n\cbr{\frac{D_2}{n_1} - \frac{D_1}{n_2}} = D_2 - D_1, \\
    n &= \frac{D_2 - D_1}{\frac{D_2}{n_1} - \frac{D_1}{n_2}} = \frac{n_1 n_2 (D_2 - D_1)}{D_2n_2 - D_1n_1} \approx 1{,}563.
    \end{align*}
}
\solutionspace{120pt}

\tasknumber{4}%
\task{%
    На каком расстоянии от собирающей линзы с фокусным расстоянием $50\,\text{дптр}$
    следует надо поместить предмет, чтобы расстояние
    от предмета до его действительного изображения было наименьшим?
}
\answer{%
    \begin{align*}
    \frac 1a &+ \frac 1b = D \implies b = \frac 1{D - \frac 1a} \implies \ell = a + b = a + \frac a{Da - 1} = \frac{ Da^2 }{Da - 1} \implies \\
    \implies \ell'_a &= \frac{ 2Da \cdot (Da - 1) - Da^2 \cdot D }{\sqr{Da - 1}}= \frac{ D^2a^2 - 2Da}{\sqr{Da - 1}} = \frac{ Da(Da - 2)}{\sqr{Da - 1}}\implies a_{\min} = \frac 2D \approx 40\,\text{мм}.
    \end{align*}
}
\solutionspace{120pt}

\tasknumber{5}%
\task{%
    Предмет в виде отрезка длиной $\ell$ расположен вдоль оптической оси
    собирающей линзы с фокусным расстоянием $F$.
    Середина отрезка расположена
    иа расстоянии $a$ от линзы, которая даёт действительное изображение
    всех точек предмета.
    Определить продольное увеличение предмета.
}
\answer{%
    \begin{align*}
    \frac 1{a + \frac \ell 2} &+ \frac 1b = \frac 1F \implies b = \frac{F\cbr{a + \frac \ell 2}}{a + \frac \ell 2 - F} \\
    \frac 1{a - \frac \ell 2} &+ \frac 1c = \frac 1F \implies c = \frac{F\cbr{a - \frac \ell 2}}{a - \frac \ell 2 - F} \\
    \abs{b - c} &= \abs{\frac{F\cbr{a + \frac \ell 2}}{a + \frac \ell 2 - F} - \frac{F\cbr{a - \frac \ell 2}}{a - \frac \ell 2 - F}}= F\abs{\frac{\cbr{a + \frac \ell 2}\cbr{a - \frac \ell 2 - F} - \cbr{a - \frac \ell 2}\cbr{a + \frac \ell 2 - F}}{ \cbr{a + \frac \ell 2 - F} \cbr{a - \frac \ell 2 - F} }} =  \\
    &= F\abs{\frac{a^2 - \frac {a\ell} 2 - Fa + \frac {a\ell} 2 - \frac {\ell^2} 4 - \frac {F\ell}2 - a^2 - \frac {a\ell}2 + aF + \frac {a\ell}2 + \frac {\ell^2} 4 - \frac {F\ell} 2}{\cbr{a + \frac \ell 2 - F} \cbr{a - \frac \ell 2 - F} }} = \\
    &= F\frac{F\ell}{\sqr{a-F} - \frac {\ell^2}4} = \frac{F^2\ell}{\sqr{a-F} - \frac {\ell^2}4}\implies \Gamma = \frac{\abs{b - c}}\ell = \frac{F^2}{\sqr{a-F} - \frac {\ell^2}4}.
    \end{align*}
}
\solutionspace{120pt}

\tasknumber{6}%
\task{%
    Даны точечный источник света $S$, его изображение $S_1$, полученное с помошью собирающей линзы,
    и ближайший к источнику фокус линзы $F$ (см.
    рис.
    на доске).
    Расстояния $SF = \ell$ и $SS_1 = L$.
    Определить положение линзы и её фокусное расстояние.
}
\answer{%
    \begin{align*}
    \frac 1a + \frac 1b &= \frac 1F, \ell = a - F, L = a + b \implies a = \ell + F, b = L - a = L - \ell - F \\
    \frac 1{\ell + F} + \frac 1{L - \ell - F} &= \frac 1F \\
    F\ell + F^2 + LF - F\ell - F^2 &= L\ell - \ell^2 - F\ell + LF - F\ell - F^2 \\
    0 &= L\ell - \ell^2 - 2F\ell - F^2 \\
    0 &=  F^2 + 2F\ell - L\ell + \ell^2 \\
    F &= -\ell \pm \sqrt{\ell^2 +  L\ell - \ell^2} = -\ell \pm \sqrt{L\ell} \implies F = \sqrt{L\ell} - \ell \\
    a &= \ell + F = \ell + \sqrt{L\ell} - \ell = \sqrt{L\ell}.
    \end{align*}
}
\solutionspace{120pt}

\tasknumber{7}%
\task{%
    Расстояние от освещённого предмета до экрана $80\,\text{см}$.
    Линза, помещенная между ними, даёт чёткое изображение предмета на
    экране при двух положениях, расстояние между которыми $30\,\text{см}$.
    Найти фокусное расстояние линзы.
}
\answer{%
    \begin{align*}
    \frac 1a + \frac 1b &= \frac 1F, \frac 1{a-\ell} + \frac 1{b+\ell} = \frac 1F, a + b = L \\
    \frac 1a + \frac 1b &= \frac 1{a-\ell} + \frac 1{b+\ell}\implies \frac{a + b}{ab} = \frac{(a-\ell) + (b+\ell)}{(a-\ell)(b+\ell)} \\
    ab  &= (a - \ell)(b+\ell) \implies 0  = -b\ell + a\ell - \ell^2 \implies 0 = -b + a - \ell \implies b = a - \ell \\
    a + (a - \ell) &= L \implies a = \frac{L + \ell}2 \implies b = \frac{L - \ell}2 \\
    F &= \frac{ab}{a + b} = \frac{L^2 -\ell^2}{4L} \approx 17{,}2\,\text{см}.
    \end{align*}
}
\solutionspace{120pt}

\tasknumber{8}%
\task{%
    Предмет находится на расстоянии $80\,\text{см}$ от экрана.
    Между предметом и экраном помещают линзу, причём при одном
    положении линзы на экране получается увеличенное изображение предмета,
    а при другом — уменьшенное.
    Каково фокусное расстояние линзы, если
    линейные размеры первого изображения в три раза больше второго?
}
\answer{%
    \begin{align*}
    \frac 1a + \frac 1{L-a} &= \frac 1F, h_1 = h \cdot \frac{L-a}a, \\
    \frac 1b + \frac 1{L-b} &= \frac 1F, h_2 = h \cdot \frac{L-b}b, \\
    \frac{h_1}{h_2} &= 3 \implies \frac{(L-a)b}{(L-b)a} = 3, \\
    \frac 1F &= \frac{ L }{a(L-a)} = \frac{ L }{b(L-b)} \implies \frac{L-a}{L-b} = \frac b a \implies \frac {b^2}{a^2} = 3.
    \\
    \frac 1a + \frac 1{L-a} &= \frac 1b + \frac 1{L-b} \implies \frac L{a(L-a)} = \frac L{b(L-b)} \implies \\
    \implies aL - a^2 &= bL - b^2 \implies (a-b)L = (a-b)(a+b) \implies b = L - a, \\
    \frac{\sqr{L-a}}{a^2} &= 3 \implies \frac La - 1 = \sqrt{3} \implies a = \frac{ L }{\sqrt{3} + 1} \\
    F &= \frac{a(L-a)}L = \frac 1L \cdot \frac L{\sqrt{3} + 1} \cdot \frac {L\sqrt{3}}{\sqrt{3} + 1}= \frac { L\sqrt{3} }{ \sqr{\sqrt{3} + 1} } \approx 18{,}6\,\text{см}.
    \end{align*}
}

\variantsplitter

\addpersonalvariant{Семён Мартынов}

\tasknumber{1}%
\task{%
    В каком месте на главной оптической оси двояковыгнутой линзы
    нужно поместить точечный источник света,
    чтобы его изображение оказалось в главном фокусе линзы?
}
\answer{%
    $\text{на половине фокусного расстояния}$
}
\solutionspace{120pt}

\tasknumber{2}%
\task{%
    На экране, расположенном иа расстоянии $60\,\text{см}$ от собирающей линзы,
    получено изображение точечного источника, расположенного на главной оптической оси линзы.
    На какое расстояние переместится изображение на экране,
    если при неподвижной линзе переместить источник на $1\,\text{см}$ в плоскости, перпендикулярной главной оптической оси?
    Фокусное расстояние линзы равно $30\,\text{см}$.
}
\answer{%
    \begin{align*}
    &\frac 1F = \frac 1a + \frac 1b \implies a = \frac{bF}{b-F} \implies \Gamma = \frac ba = \frac{b-F}F \\
    &y = x \cdot \Gamma = x \cdot \frac{b-F}F \implies d = y = 1\,\text{см}.
    \end{align*}
}
\solutionspace{120pt}

\tasknumber{3}%
\task{%
    Оптическая сила двояковыпуклой линзы в воздухе $5{,}5\,\text{дптр}$, а в воде $1{,}5\,\text{дптр}$.
    Определить показатель преломления $n$ материала, из которого изготовлена линза.
}
\answer{%
    \begin{align*}
    D_1 &=\cbr{\frac n{n_1} - 1}\cbr{\frac 1{R_1} + \frac 1{R_2}}, \\
    D_2 &=\cbr{\frac n{n_2} - 1}\cbr{\frac 1{R_1} + \frac 1{R_2}}, \\
    \frac {D_2}{D_1} &=\frac{\frac n{n_2} - 1}{\frac n{n_1} - 1} \implies {D_2}\cbr{\frac n{n_1} - 1} = {D_1}\cbr{\frac n{n_2} - 1}  \implies n\cbr{\frac{D_2}{n_1} - \frac{D_1}{n_2}} = D_2 - D_1, \\
    n &= \frac{D_2 - D_1}{\frac{D_2}{n_1} - \frac{D_1}{n_2}} = \frac{n_1 n_2 (D_2 - D_1)}{D_2n_2 - D_1n_1} \approx 1{,}518.
    \end{align*}
}
\solutionspace{120pt}

\tasknumber{4}%
\task{%
    На каком расстоянии от собирающей линзы с фокусным расстоянием $40\,\text{дптр}$
    следует надо поместить предмет, чтобы расстояние
    от предмета до его действительного изображения было наименьшим?
}
\answer{%
    \begin{align*}
    \frac 1a &+ \frac 1b = D \implies b = \frac 1{D - \frac 1a} \implies \ell = a + b = a + \frac a{Da - 1} = \frac{ Da^2 }{Da - 1} \implies \\
    \implies \ell'_a &= \frac{ 2Da \cdot (Da - 1) - Da^2 \cdot D }{\sqr{Da - 1}}= \frac{ D^2a^2 - 2Da}{\sqr{Da - 1}} = \frac{ Da(Da - 2)}{\sqr{Da - 1}}\implies a_{\min} = \frac 2D \approx 50\,\text{мм}.
    \end{align*}
}
\solutionspace{120pt}

\tasknumber{5}%
\task{%
    Предмет в виде отрезка длиной $\ell$ расположен вдоль оптической оси
    собирающей линзы с фокусным расстоянием $F$.
    Середина отрезка расположена
    иа расстоянии $a$ от линзы, которая даёт действительное изображение
    всех точек предмета.
    Определить продольное увеличение предмета.
}
\answer{%
    \begin{align*}
    \frac 1{a + \frac \ell 2} &+ \frac 1b = \frac 1F \implies b = \frac{F\cbr{a + \frac \ell 2}}{a + \frac \ell 2 - F} \\
    \frac 1{a - \frac \ell 2} &+ \frac 1c = \frac 1F \implies c = \frac{F\cbr{a - \frac \ell 2}}{a - \frac \ell 2 - F} \\
    \abs{b - c} &= \abs{\frac{F\cbr{a + \frac \ell 2}}{a + \frac \ell 2 - F} - \frac{F\cbr{a - \frac \ell 2}}{a - \frac \ell 2 - F}}= F\abs{\frac{\cbr{a + \frac \ell 2}\cbr{a - \frac \ell 2 - F} - \cbr{a - \frac \ell 2}\cbr{a + \frac \ell 2 - F}}{ \cbr{a + \frac \ell 2 - F} \cbr{a - \frac \ell 2 - F} }} =  \\
    &= F\abs{\frac{a^2 - \frac {a\ell} 2 - Fa + \frac {a\ell} 2 - \frac {\ell^2} 4 - \frac {F\ell}2 - a^2 - \frac {a\ell}2 + aF + \frac {a\ell}2 + \frac {\ell^2} 4 - \frac {F\ell} 2}{\cbr{a + \frac \ell 2 - F} \cbr{a - \frac \ell 2 - F} }} = \\
    &= F\frac{F\ell}{\sqr{a-F} - \frac {\ell^2}4} = \frac{F^2\ell}{\sqr{a-F} - \frac {\ell^2}4}\implies \Gamma = \frac{\abs{b - c}}\ell = \frac{F^2}{\sqr{a-F} - \frac {\ell^2}4}.
    \end{align*}
}
\solutionspace{120pt}

\tasknumber{6}%
\task{%
    Даны точечный источник света $S$, его изображение $S_1$, полученное с помошью собирающей линзы,
    и ближайший к источнику фокус линзы $F$ (см.
    рис.
    на доске).
    Расстояния $SF = \ell$ и $SS_1 = L$.
    Определить положение линзы и её фокусное расстояние.
}
\answer{%
    \begin{align*}
    \frac 1a + \frac 1b &= \frac 1F, \ell = a - F, L = a + b \implies a = \ell + F, b = L - a = L - \ell - F \\
    \frac 1{\ell + F} + \frac 1{L - \ell - F} &= \frac 1F \\
    F\ell + F^2 + LF - F\ell - F^2 &= L\ell - \ell^2 - F\ell + LF - F\ell - F^2 \\
    0 &= L\ell - \ell^2 - 2F\ell - F^2 \\
    0 &=  F^2 + 2F\ell - L\ell + \ell^2 \\
    F &= -\ell \pm \sqrt{\ell^2 +  L\ell - \ell^2} = -\ell \pm \sqrt{L\ell} \implies F = \sqrt{L\ell} - \ell \\
    a &= \ell + F = \ell + \sqrt{L\ell} - \ell = \sqrt{L\ell}.
    \end{align*}
}
\solutionspace{120pt}

\tasknumber{7}%
\task{%
    Расстояние от освещённого предмета до экрана $100\,\text{см}$.
    Линза, помещенная между ними, даёт чёткое изображение предмета на
    экране при двух положениях, расстояние между которыми $30\,\text{см}$.
    Найти фокусное расстояние линзы.
}
\answer{%
    \begin{align*}
    \frac 1a + \frac 1b &= \frac 1F, \frac 1{a-\ell} + \frac 1{b+\ell} = \frac 1F, a + b = L \\
    \frac 1a + \frac 1b &= \frac 1{a-\ell} + \frac 1{b+\ell}\implies \frac{a + b}{ab} = \frac{(a-\ell) + (b+\ell)}{(a-\ell)(b+\ell)} \\
    ab  &= (a - \ell)(b+\ell) \implies 0  = -b\ell + a\ell - \ell^2 \implies 0 = -b + a - \ell \implies b = a - \ell \\
    a + (a - \ell) &= L \implies a = \frac{L + \ell}2 \implies b = \frac{L - \ell}2 \\
    F &= \frac{ab}{a + b} = \frac{L^2 -\ell^2}{4L} \approx 22{,}8\,\text{см}.
    \end{align*}
}
\solutionspace{120pt}

\tasknumber{8}%
\task{%
    Предмет находится на расстоянии $60\,\text{см}$ от экрана.
    Между предметом и экраном помещают линзу, причём при одном
    положении линзы на экране получается увеличенное изображение предмета,
    а при другом — уменьшенное.
    Каково фокусное расстояние линзы, если
    линейные размеры первого изображения в пять раз больше второго?
}
\answer{%
    \begin{align*}
    \frac 1a + \frac 1{L-a} &= \frac 1F, h_1 = h \cdot \frac{L-a}a, \\
    \frac 1b + \frac 1{L-b} &= \frac 1F, h_2 = h \cdot \frac{L-b}b, \\
    \frac{h_1}{h_2} &= 5 \implies \frac{(L-a)b}{(L-b)a} = 5, \\
    \frac 1F &= \frac{ L }{a(L-a)} = \frac{ L }{b(L-b)} \implies \frac{L-a}{L-b} = \frac b a \implies \frac {b^2}{a^2} = 5.
    \\
    \frac 1a + \frac 1{L-a} &= \frac 1b + \frac 1{L-b} \implies \frac L{a(L-a)} = \frac L{b(L-b)} \implies \\
    \implies aL - a^2 &= bL - b^2 \implies (a-b)L = (a-b)(a+b) \implies b = L - a, \\
    \frac{\sqr{L-a}}{a^2} &= 5 \implies \frac La - 1 = \sqrt{5} \implies a = \frac{ L }{\sqrt{5} + 1} \\
    F &= \frac{a(L-a)}L = \frac 1L \cdot \frac L{\sqrt{5} + 1} \cdot \frac {L\sqrt{5}}{\sqrt{5} + 1}= \frac { L\sqrt{5} }{ \sqr{\sqrt{5} + 1} } \approx 12{,}8\,\text{см}.
    \end{align*}
}

\variantsplitter

\addpersonalvariant{Варвара Минаева}

\tasknumber{1}%
\task{%
    В каком месте на главной оптической оси двояковыгнутой линзы
    нужно поместить точечный источник света,
    чтобы его изображение оказалось в главном фокусе линзы?
}
\answer{%
    $\text{на половине фокусного расстояния}$
}
\solutionspace{120pt}

\tasknumber{2}%
\task{%
    На экране, расположенном иа расстоянии $80\,\text{см}$ от собирающей линзы,
    получено изображение точечного источника, расположенного на главной оптической оси линзы.
    На какое расстояние переместится изображение на экране,
    если при неподвижном источнике переместить линзу на $2\,\text{см}$ в плоскости, перпендикулярной главной оптической оси?
    Фокусное расстояние линзы равно $40\,\text{см}$.
}
\answer{%
    \begin{align*}
    &\frac 1F = \frac 1a + \frac 1b \implies a = \frac{bF}{b-F} \implies \Gamma = \frac ba = \frac{b-F}F \\
    &y = x \cdot \Gamma = x \cdot \frac{b-F}F \implies d = x + y = 4\,\text{см}.
    \end{align*}
}
\solutionspace{120pt}

\tasknumber{3}%
\task{%
    Оптическая сила двояковыпуклой линзы в воздухе $5\,\text{дптр}$, а в воде $1{,}5\,\text{дптр}$.
    Определить показатель преломления $n$ материала, из которого изготовлена линза.
}
\answer{%
    \begin{align*}
    D_1 &=\cbr{\frac n{n_1} - 1}\cbr{\frac 1{R_1} + \frac 1{R_2}}, \\
    D_2 &=\cbr{\frac n{n_2} - 1}\cbr{\frac 1{R_1} + \frac 1{R_2}}, \\
    \frac {D_2}{D_1} &=\frac{\frac n{n_2} - 1}{\frac n{n_1} - 1} \implies {D_2}\cbr{\frac n{n_1} - 1} = {D_1}\cbr{\frac n{n_2} - 1}  \implies n\cbr{\frac{D_2}{n_1} - \frac{D_1}{n_2}} = D_2 - D_1, \\
    n &= \frac{D_2 - D_1}{\frac{D_2}{n_1} - \frac{D_1}{n_2}} = \frac{n_1 n_2 (D_2 - D_1)}{D_2n_2 - D_1n_1} \approx 1{,}549.
    \end{align*}
}
\solutionspace{120pt}

\tasknumber{4}%
\task{%
    На каком расстоянии от собирающей линзы с фокусным расстоянием $40\,\text{дптр}$
    следует надо поместить предмет, чтобы расстояние
    от предмета до его действительного изображения было наименьшим?
}
\answer{%
    \begin{align*}
    \frac 1a &+ \frac 1b = D \implies b = \frac 1{D - \frac 1a} \implies \ell = a + b = a + \frac a{Da - 1} = \frac{ Da^2 }{Da - 1} \implies \\
    \implies \ell'_a &= \frac{ 2Da \cdot (Da - 1) - Da^2 \cdot D }{\sqr{Da - 1}}= \frac{ D^2a^2 - 2Da}{\sqr{Da - 1}} = \frac{ Da(Da - 2)}{\sqr{Da - 1}}\implies a_{\min} = \frac 2D \approx 50\,\text{мм}.
    \end{align*}
}
\solutionspace{120pt}

\tasknumber{5}%
\task{%
    Предмет в виде отрезка длиной $\ell$ расположен вдоль оптической оси
    собирающей линзы с фокусным расстоянием $F$.
    Середина отрезка расположена
    иа расстоянии $a$ от линзы, которая даёт действительное изображение
    всех точек предмета.
    Определить продольное увеличение предмета.
}
\answer{%
    \begin{align*}
    \frac 1{a + \frac \ell 2} &+ \frac 1b = \frac 1F \implies b = \frac{F\cbr{a + \frac \ell 2}}{a + \frac \ell 2 - F} \\
    \frac 1{a - \frac \ell 2} &+ \frac 1c = \frac 1F \implies c = \frac{F\cbr{a - \frac \ell 2}}{a - \frac \ell 2 - F} \\
    \abs{b - c} &= \abs{\frac{F\cbr{a + \frac \ell 2}}{a + \frac \ell 2 - F} - \frac{F\cbr{a - \frac \ell 2}}{a - \frac \ell 2 - F}}= F\abs{\frac{\cbr{a + \frac \ell 2}\cbr{a - \frac \ell 2 - F} - \cbr{a - \frac \ell 2}\cbr{a + \frac \ell 2 - F}}{ \cbr{a + \frac \ell 2 - F} \cbr{a - \frac \ell 2 - F} }} =  \\
    &= F\abs{\frac{a^2 - \frac {a\ell} 2 - Fa + \frac {a\ell} 2 - \frac {\ell^2} 4 - \frac {F\ell}2 - a^2 - \frac {a\ell}2 + aF + \frac {a\ell}2 + \frac {\ell^2} 4 - \frac {F\ell} 2}{\cbr{a + \frac \ell 2 - F} \cbr{a - \frac \ell 2 - F} }} = \\
    &= F\frac{F\ell}{\sqr{a-F} - \frac {\ell^2}4} = \frac{F^2\ell}{\sqr{a-F} - \frac {\ell^2}4}\implies \Gamma = \frac{\abs{b - c}}\ell = \frac{F^2}{\sqr{a-F} - \frac {\ell^2}4}.
    \end{align*}
}
\solutionspace{120pt}

\tasknumber{6}%
\task{%
    Даны точечный источник света $S$, его изображение $S_1$, полученное с помошью собирающей линзы,
    и ближайший к источнику фокус линзы $F$ (см.
    рис.
    на доске).
    Расстояния $SF = \ell$ и $SS_1 = L$.
    Определить положение линзы и её фокусное расстояние.
}
\answer{%
    \begin{align*}
    \frac 1a + \frac 1b &= \frac 1F, \ell = a - F, L = a + b \implies a = \ell + F, b = L - a = L - \ell - F \\
    \frac 1{\ell + F} + \frac 1{L - \ell - F} &= \frac 1F \\
    F\ell + F^2 + LF - F\ell - F^2 &= L\ell - \ell^2 - F\ell + LF - F\ell - F^2 \\
    0 &= L\ell - \ell^2 - 2F\ell - F^2 \\
    0 &=  F^2 + 2F\ell - L\ell + \ell^2 \\
    F &= -\ell \pm \sqrt{\ell^2 +  L\ell - \ell^2} = -\ell \pm \sqrt{L\ell} \implies F = \sqrt{L\ell} - \ell \\
    a &= \ell + F = \ell + \sqrt{L\ell} - \ell = \sqrt{L\ell}.
    \end{align*}
}
\solutionspace{120pt}

\tasknumber{7}%
\task{%
    Расстояние от освещённого предмета до экрана $100\,\text{см}$.
    Линза, помещенная между ними, даёт чёткое изображение предмета на
    экране при двух положениях, расстояние между которыми $20\,\text{см}$.
    Найти фокусное расстояние линзы.
}
\answer{%
    \begin{align*}
    \frac 1a + \frac 1b &= \frac 1F, \frac 1{a-\ell} + \frac 1{b+\ell} = \frac 1F, a + b = L \\
    \frac 1a + \frac 1b &= \frac 1{a-\ell} + \frac 1{b+\ell}\implies \frac{a + b}{ab} = \frac{(a-\ell) + (b+\ell)}{(a-\ell)(b+\ell)} \\
    ab  &= (a - \ell)(b+\ell) \implies 0  = -b\ell + a\ell - \ell^2 \implies 0 = -b + a - \ell \implies b = a - \ell \\
    a + (a - \ell) &= L \implies a = \frac{L + \ell}2 \implies b = \frac{L - \ell}2 \\
    F &= \frac{ab}{a + b} = \frac{L^2 -\ell^2}{4L} \approx 24\,\text{см}.
    \end{align*}
}
\solutionspace{120pt}

\tasknumber{8}%
\task{%
    Предмет находится на расстоянии $80\,\text{см}$ от экрана.
    Между предметом и экраном помещают линзу, причём при одном
    положении линзы на экране получается увеличенное изображение предмета,
    а при другом — уменьшенное.
    Каково фокусное расстояние линзы, если
    линейные размеры первого изображения в пять раз больше второго?
}
\answer{%
    \begin{align*}
    \frac 1a + \frac 1{L-a} &= \frac 1F, h_1 = h \cdot \frac{L-a}a, \\
    \frac 1b + \frac 1{L-b} &= \frac 1F, h_2 = h \cdot \frac{L-b}b, \\
    \frac{h_1}{h_2} &= 5 \implies \frac{(L-a)b}{(L-b)a} = 5, \\
    \frac 1F &= \frac{ L }{a(L-a)} = \frac{ L }{b(L-b)} \implies \frac{L-a}{L-b} = \frac b a \implies \frac {b^2}{a^2} = 5.
    \\
    \frac 1a + \frac 1{L-a} &= \frac 1b + \frac 1{L-b} \implies \frac L{a(L-a)} = \frac L{b(L-b)} \implies \\
    \implies aL - a^2 &= bL - b^2 \implies (a-b)L = (a-b)(a+b) \implies b = L - a, \\
    \frac{\sqr{L-a}}{a^2} &= 5 \implies \frac La - 1 = \sqrt{5} \implies a = \frac{ L }{\sqrt{5} + 1} \\
    F &= \frac{a(L-a)}L = \frac 1L \cdot \frac L{\sqrt{5} + 1} \cdot \frac {L\sqrt{5}}{\sqrt{5} + 1}= \frac { L\sqrt{5} }{ \sqr{\sqrt{5} + 1} } \approx 17{,}1\,\text{см}.
    \end{align*}
}

\variantsplitter

\addpersonalvariant{Леонид Никитин}

\tasknumber{1}%
\task{%
    В каком месте на главной оптической оси двояковыпуклой линзы
    нужно поместить точечный источник света,
    чтобы его изображение оказалось в главном фокусе линзы?
}
\answer{%
    $\text{для мнимого - на половине фокусного, для действительного - на бесконечности}$
}
\solutionspace{120pt}

\tasknumber{2}%
\task{%
    На экране, расположенном иа расстоянии $60\,\text{см}$ от собирающей линзы,
    получено изображение точечного источника, расположенного на главной оптической оси линзы.
    На какое расстояние переместится изображение на экране,
    если при неподвижном источнике переместить линзу на $1\,\text{см}$ в плоскости, перпендикулярной главной оптической оси?
    Фокусное расстояние линзы равно $20\,\text{см}$.
}
\answer{%
    \begin{align*}
    &\frac 1F = \frac 1a + \frac 1b \implies a = \frac{bF}{b-F} \implies \Gamma = \frac ba = \frac{b-F}F \\
    &y = x \cdot \Gamma = x \cdot \frac{b-F}F \implies d = x + y = 3\,\text{см}.
    \end{align*}
}
\solutionspace{120pt}

\tasknumber{3}%
\task{%
    Оптическая сила двояковыпуклой линзы в воздухе $5\,\text{дптр}$, а в воде $1{,}6\,\text{дптр}$.
    Определить показатель преломления $n$ материала, из которого изготовлена линза.
}
\answer{%
    \begin{align*}
    D_1 &=\cbr{\frac n{n_1} - 1}\cbr{\frac 1{R_1} + \frac 1{R_2}}, \\
    D_2 &=\cbr{\frac n{n_2} - 1}\cbr{\frac 1{R_1} + \frac 1{R_2}}, \\
    \frac {D_2}{D_1} &=\frac{\frac n{n_2} - 1}{\frac n{n_1} - 1} \implies {D_2}\cbr{\frac n{n_1} - 1} = {D_1}\cbr{\frac n{n_2} - 1}  \implies n\cbr{\frac{D_2}{n_1} - \frac{D_1}{n_2}} = D_2 - D_1, \\
    n &= \frac{D_2 - D_1}{\frac{D_2}{n_1} - \frac{D_1}{n_2}} = \frac{n_1 n_2 (D_2 - D_1)}{D_2n_2 - D_1n_1} \approx 1{,}575.
    \end{align*}
}
\solutionspace{120pt}

\tasknumber{4}%
\task{%
    На каком расстоянии от собирающей линзы с фокусным расстоянием $50\,\text{дптр}$
    следует надо поместить предмет, чтобы расстояние
    от предмета до его действительного изображения было наименьшим?
}
\answer{%
    \begin{align*}
    \frac 1a &+ \frac 1b = D \implies b = \frac 1{D - \frac 1a} \implies \ell = a + b = a + \frac a{Da - 1} = \frac{ Da^2 }{Da - 1} \implies \\
    \implies \ell'_a &= \frac{ 2Da \cdot (Da - 1) - Da^2 \cdot D }{\sqr{Da - 1}}= \frac{ D^2a^2 - 2Da}{\sqr{Da - 1}} = \frac{ Da(Da - 2)}{\sqr{Da - 1}}\implies a_{\min} = \frac 2D \approx 40\,\text{мм}.
    \end{align*}
}
\solutionspace{120pt}

\tasknumber{5}%
\task{%
    Предмет в виде отрезка длиной $\ell$ расположен вдоль оптической оси
    собирающей линзы с фокусным расстоянием $F$.
    Середина отрезка расположена
    иа расстоянии $a$ от линзы, которая даёт действительное изображение
    всех точек предмета.
    Определить продольное увеличение предмета.
}
\answer{%
    \begin{align*}
    \frac 1{a + \frac \ell 2} &+ \frac 1b = \frac 1F \implies b = \frac{F\cbr{a + \frac \ell 2}}{a + \frac \ell 2 - F} \\
    \frac 1{a - \frac \ell 2} &+ \frac 1c = \frac 1F \implies c = \frac{F\cbr{a - \frac \ell 2}}{a - \frac \ell 2 - F} \\
    \abs{b - c} &= \abs{\frac{F\cbr{a + \frac \ell 2}}{a + \frac \ell 2 - F} - \frac{F\cbr{a - \frac \ell 2}}{a - \frac \ell 2 - F}}= F\abs{\frac{\cbr{a + \frac \ell 2}\cbr{a - \frac \ell 2 - F} - \cbr{a - \frac \ell 2}\cbr{a + \frac \ell 2 - F}}{ \cbr{a + \frac \ell 2 - F} \cbr{a - \frac \ell 2 - F} }} =  \\
    &= F\abs{\frac{a^2 - \frac {a\ell} 2 - Fa + \frac {a\ell} 2 - \frac {\ell^2} 4 - \frac {F\ell}2 - a^2 - \frac {a\ell}2 + aF + \frac {a\ell}2 + \frac {\ell^2} 4 - \frac {F\ell} 2}{\cbr{a + \frac \ell 2 - F} \cbr{a - \frac \ell 2 - F} }} = \\
    &= F\frac{F\ell}{\sqr{a-F} - \frac {\ell^2}4} = \frac{F^2\ell}{\sqr{a-F} - \frac {\ell^2}4}\implies \Gamma = \frac{\abs{b - c}}\ell = \frac{F^2}{\sqr{a-F} - \frac {\ell^2}4}.
    \end{align*}
}
\solutionspace{120pt}

\tasknumber{6}%
\task{%
    Даны точечный источник света $S$, его изображение $S_1$, полученное с помошью собирающей линзы,
    и ближайший к источнику фокус линзы $F$ (см.
    рис.
    на доске).
    Расстояния $SF = \ell$ и $SS_1 = L$.
    Определить положение линзы и её фокусное расстояние.
}
\answer{%
    \begin{align*}
    \frac 1a + \frac 1b &= \frac 1F, \ell = a - F, L = a + b \implies a = \ell + F, b = L - a = L - \ell - F \\
    \frac 1{\ell + F} + \frac 1{L - \ell - F} &= \frac 1F \\
    F\ell + F^2 + LF - F\ell - F^2 &= L\ell - \ell^2 - F\ell + LF - F\ell - F^2 \\
    0 &= L\ell - \ell^2 - 2F\ell - F^2 \\
    0 &=  F^2 + 2F\ell - L\ell + \ell^2 \\
    F &= -\ell \pm \sqrt{\ell^2 +  L\ell - \ell^2} = -\ell \pm \sqrt{L\ell} \implies F = \sqrt{L\ell} - \ell \\
    a &= \ell + F = \ell + \sqrt{L\ell} - \ell = \sqrt{L\ell}.
    \end{align*}
}
\solutionspace{120pt}

\tasknumber{7}%
\task{%
    Расстояние от освещённого предмета до экрана $80\,\text{см}$.
    Линза, помещенная между ними, даёт чёткое изображение предмета на
    экране при двух положениях, расстояние между которыми $20\,\text{см}$.
    Найти фокусное расстояние линзы.
}
\answer{%
    \begin{align*}
    \frac 1a + \frac 1b &= \frac 1F, \frac 1{a-\ell} + \frac 1{b+\ell} = \frac 1F, a + b = L \\
    \frac 1a + \frac 1b &= \frac 1{a-\ell} + \frac 1{b+\ell}\implies \frac{a + b}{ab} = \frac{(a-\ell) + (b+\ell)}{(a-\ell)(b+\ell)} \\
    ab  &= (a - \ell)(b+\ell) \implies 0  = -b\ell + a\ell - \ell^2 \implies 0 = -b + a - \ell \implies b = a - \ell \\
    a + (a - \ell) &= L \implies a = \frac{L + \ell}2 \implies b = \frac{L - \ell}2 \\
    F &= \frac{ab}{a + b} = \frac{L^2 -\ell^2}{4L} \approx 18{,}8\,\text{см}.
    \end{align*}
}
\solutionspace{120pt}

\tasknumber{8}%
\task{%
    Предмет находится на расстоянии $70\,\text{см}$ от экрана.
    Между предметом и экраном помещают линзу, причём при одном
    положении линзы на экране получается увеличенное изображение предмета,
    а при другом — уменьшенное.
    Каково фокусное расстояние линзы, если
    линейные размеры первого изображения в пять раз больше второго?
}
\answer{%
    \begin{align*}
    \frac 1a + \frac 1{L-a} &= \frac 1F, h_1 = h \cdot \frac{L-a}a, \\
    \frac 1b + \frac 1{L-b} &= \frac 1F, h_2 = h \cdot \frac{L-b}b, \\
    \frac{h_1}{h_2} &= 5 \implies \frac{(L-a)b}{(L-b)a} = 5, \\
    \frac 1F &= \frac{ L }{a(L-a)} = \frac{ L }{b(L-b)} \implies \frac{L-a}{L-b} = \frac b a \implies \frac {b^2}{a^2} = 5.
    \\
    \frac 1a + \frac 1{L-a} &= \frac 1b + \frac 1{L-b} \implies \frac L{a(L-a)} = \frac L{b(L-b)} \implies \\
    \implies aL - a^2 &= bL - b^2 \implies (a-b)L = (a-b)(a+b) \implies b = L - a, \\
    \frac{\sqr{L-a}}{a^2} &= 5 \implies \frac La - 1 = \sqrt{5} \implies a = \frac{ L }{\sqrt{5} + 1} \\
    F &= \frac{a(L-a)}L = \frac 1L \cdot \frac L{\sqrt{5} + 1} \cdot \frac {L\sqrt{5}}{\sqrt{5} + 1}= \frac { L\sqrt{5} }{ \sqr{\sqrt{5} + 1} } \approx 14{,}9\,\text{см}.
    \end{align*}
}

\variantsplitter

\addpersonalvariant{Тимофей Полетаев}

\tasknumber{1}%
\task{%
    В каком месте на главной оптической оси двояковыпуклой линзы
    нужно поместить точечный источник света,
    чтобы его изображение оказалось в главном фокусе линзы?
}
\answer{%
    $\text{для мнимого - на половине фокусного, для действительного - на бесконечности}$
}
\solutionspace{120pt}

\tasknumber{2}%
\task{%
    На экране, расположенном иа расстоянии $80\,\text{см}$ от собирающей линзы,
    получено изображение точечного источника, расположенного на главной оптической оси линзы.
    На какое расстояние переместится изображение на экране,
    если при неподвижной линзе переместить источник на $3\,\text{см}$ в плоскости, перпендикулярной главной оптической оси?
    Фокусное расстояние линзы равно $30\,\text{см}$.
}
\answer{%
    \begin{align*}
    &\frac 1F = \frac 1a + \frac 1b \implies a = \frac{bF}{b-F} \implies \Gamma = \frac ba = \frac{b-F}F \\
    &y = x \cdot \Gamma = x \cdot \frac{b-F}F \implies d = y = 5{,}0\,\text{см}.
    \end{align*}
}
\solutionspace{120pt}

\tasknumber{3}%
\task{%
    Оптическая сила двояковыпуклой линзы в воздухе $5\,\text{дптр}$, а в воде $1{,}4\,\text{дптр}$.
    Определить показатель преломления $n$ материала, из которого изготовлена линза.
}
\answer{%
    \begin{align*}
    D_1 &=\cbr{\frac n{n_1} - 1}\cbr{\frac 1{R_1} + \frac 1{R_2}}, \\
    D_2 &=\cbr{\frac n{n_2} - 1}\cbr{\frac 1{R_1} + \frac 1{R_2}}, \\
    \frac {D_2}{D_1} &=\frac{\frac n{n_2} - 1}{\frac n{n_1} - 1} \implies {D_2}\cbr{\frac n{n_1} - 1} = {D_1}\cbr{\frac n{n_2} - 1}  \implies n\cbr{\frac{D_2}{n_1} - \frac{D_1}{n_2}} = D_2 - D_1, \\
    n &= \frac{D_2 - D_1}{\frac{D_2}{n_1} - \frac{D_1}{n_2}} = \frac{n_1 n_2 (D_2 - D_1)}{D_2n_2 - D_1n_1} \approx 1{,}526.
    \end{align*}
}
\solutionspace{120pt}

\tasknumber{4}%
\task{%
    На каком расстоянии от собирающей линзы с фокусным расстоянием $30\,\text{дптр}$
    следует надо поместить предмет, чтобы расстояние
    от предмета до его действительного изображения было наименьшим?
}
\answer{%
    \begin{align*}
    \frac 1a &+ \frac 1b = D \implies b = \frac 1{D - \frac 1a} \implies \ell = a + b = a + \frac a{Da - 1} = \frac{ Da^2 }{Da - 1} \implies \\
    \implies \ell'_a &= \frac{ 2Da \cdot (Da - 1) - Da^2 \cdot D }{\sqr{Da - 1}}= \frac{ D^2a^2 - 2Da}{\sqr{Da - 1}} = \frac{ Da(Da - 2)}{\sqr{Da - 1}}\implies a_{\min} = \frac 2D \approx 66{,}7\,\text{мм}.
    \end{align*}
}
\solutionspace{120pt}

\tasknumber{5}%
\task{%
    Предмет в виде отрезка длиной $\ell$ расположен вдоль оптической оси
    собирающей линзы с фокусным расстоянием $F$.
    Середина отрезка расположена
    иа расстоянии $a$ от линзы, которая даёт действительное изображение
    всех точек предмета.
    Определить продольное увеличение предмета.
}
\answer{%
    \begin{align*}
    \frac 1{a + \frac \ell 2} &+ \frac 1b = \frac 1F \implies b = \frac{F\cbr{a + \frac \ell 2}}{a + \frac \ell 2 - F} \\
    \frac 1{a - \frac \ell 2} &+ \frac 1c = \frac 1F \implies c = \frac{F\cbr{a - \frac \ell 2}}{a - \frac \ell 2 - F} \\
    \abs{b - c} &= \abs{\frac{F\cbr{a + \frac \ell 2}}{a + \frac \ell 2 - F} - \frac{F\cbr{a - \frac \ell 2}}{a - \frac \ell 2 - F}}= F\abs{\frac{\cbr{a + \frac \ell 2}\cbr{a - \frac \ell 2 - F} - \cbr{a - \frac \ell 2}\cbr{a + \frac \ell 2 - F}}{ \cbr{a + \frac \ell 2 - F} \cbr{a - \frac \ell 2 - F} }} =  \\
    &= F\abs{\frac{a^2 - \frac {a\ell} 2 - Fa + \frac {a\ell} 2 - \frac {\ell^2} 4 - \frac {F\ell}2 - a^2 - \frac {a\ell}2 + aF + \frac {a\ell}2 + \frac {\ell^2} 4 - \frac {F\ell} 2}{\cbr{a + \frac \ell 2 - F} \cbr{a - \frac \ell 2 - F} }} = \\
    &= F\frac{F\ell}{\sqr{a-F} - \frac {\ell^2}4} = \frac{F^2\ell}{\sqr{a-F} - \frac {\ell^2}4}\implies \Gamma = \frac{\abs{b - c}}\ell = \frac{F^2}{\sqr{a-F} - \frac {\ell^2}4}.
    \end{align*}
}
\solutionspace{120pt}

\tasknumber{6}%
\task{%
    Даны точечный источник света $S$, его изображение $S_1$, полученное с помошью собирающей линзы,
    и ближайший к источнику фокус линзы $F$ (см.
    рис.
    на доске).
    Расстояния $SF = \ell$ и $SS_1 = L$.
    Определить положение линзы и её фокусное расстояние.
}
\answer{%
    \begin{align*}
    \frac 1a + \frac 1b &= \frac 1F, \ell = a - F, L = a + b \implies a = \ell + F, b = L - a = L - \ell - F \\
    \frac 1{\ell + F} + \frac 1{L - \ell - F} &= \frac 1F \\
    F\ell + F^2 + LF - F\ell - F^2 &= L\ell - \ell^2 - F\ell + LF - F\ell - F^2 \\
    0 &= L\ell - \ell^2 - 2F\ell - F^2 \\
    0 &=  F^2 + 2F\ell - L\ell + \ell^2 \\
    F &= -\ell \pm \sqrt{\ell^2 +  L\ell - \ell^2} = -\ell \pm \sqrt{L\ell} \implies F = \sqrt{L\ell} - \ell \\
    a &= \ell + F = \ell + \sqrt{L\ell} - \ell = \sqrt{L\ell}.
    \end{align*}
}
\solutionspace{120pt}

\tasknumber{7}%
\task{%
    Расстояние от освещённого предмета до экрана $80\,\text{см}$.
    Линза, помещенная между ними, даёт чёткое изображение предмета на
    экране при двух положениях, расстояние между которыми $40\,\text{см}$.
    Найти фокусное расстояние линзы.
}
\answer{%
    \begin{align*}
    \frac 1a + \frac 1b &= \frac 1F, \frac 1{a-\ell} + \frac 1{b+\ell} = \frac 1F, a + b = L \\
    \frac 1a + \frac 1b &= \frac 1{a-\ell} + \frac 1{b+\ell}\implies \frac{a + b}{ab} = \frac{(a-\ell) + (b+\ell)}{(a-\ell)(b+\ell)} \\
    ab  &= (a - \ell)(b+\ell) \implies 0  = -b\ell + a\ell - \ell^2 \implies 0 = -b + a - \ell \implies b = a - \ell \\
    a + (a - \ell) &= L \implies a = \frac{L + \ell}2 \implies b = \frac{L - \ell}2 \\
    F &= \frac{ab}{a + b} = \frac{L^2 -\ell^2}{4L} \approx 15\,\text{см}.
    \end{align*}
}
\solutionspace{120pt}

\tasknumber{8}%
\task{%
    Предмет находится на расстоянии $70\,\text{см}$ от экрана.
    Между предметом и экраном помещают линзу, причём при одном
    положении линзы на экране получается увеличенное изображение предмета,
    а при другом — уменьшенное.
    Каково фокусное расстояние линзы, если
    линейные размеры первого изображения в пять раз больше второго?
}
\answer{%
    \begin{align*}
    \frac 1a + \frac 1{L-a} &= \frac 1F, h_1 = h \cdot \frac{L-a}a, \\
    \frac 1b + \frac 1{L-b} &= \frac 1F, h_2 = h \cdot \frac{L-b}b, \\
    \frac{h_1}{h_2} &= 5 \implies \frac{(L-a)b}{(L-b)a} = 5, \\
    \frac 1F &= \frac{ L }{a(L-a)} = \frac{ L }{b(L-b)} \implies \frac{L-a}{L-b} = \frac b a \implies \frac {b^2}{a^2} = 5.
    \\
    \frac 1a + \frac 1{L-a} &= \frac 1b + \frac 1{L-b} \implies \frac L{a(L-a)} = \frac L{b(L-b)} \implies \\
    \implies aL - a^2 &= bL - b^2 \implies (a-b)L = (a-b)(a+b) \implies b = L - a, \\
    \frac{\sqr{L-a}}{a^2} &= 5 \implies \frac La - 1 = \sqrt{5} \implies a = \frac{ L }{\sqrt{5} + 1} \\
    F &= \frac{a(L-a)}L = \frac 1L \cdot \frac L{\sqrt{5} + 1} \cdot \frac {L\sqrt{5}}{\sqrt{5} + 1}= \frac { L\sqrt{5} }{ \sqr{\sqrt{5} + 1} } \approx 14{,}9\,\text{см}.
    \end{align*}
}

\variantsplitter

\addpersonalvariant{Андрей Рожков}

\tasknumber{1}%
\task{%
    В каком месте на главной оптической оси двояковыгнутой линзы
    нужно поместить точечный источник света,
    чтобы его изображение оказалось в главном фокусе линзы?
}
\answer{%
    $\text{на половине фокусного расстояния}$
}
\solutionspace{120pt}

\tasknumber{2}%
\task{%
    На экране, расположенном иа расстоянии $60\,\text{см}$ от собирающей линзы,
    получено изображение точечного источника, расположенного на главной оптической оси линзы.
    На какое расстояние переместится изображение на экране,
    если при неподвижной линзе переместить источник на $1\,\text{см}$ в плоскости, перпендикулярной главной оптической оси?
    Фокусное расстояние линзы равно $40\,\text{см}$.
}
\answer{%
    \begin{align*}
    &\frac 1F = \frac 1a + \frac 1b \implies a = \frac{bF}{b-F} \implies \Gamma = \frac ba = \frac{b-F}F \\
    &y = x \cdot \Gamma = x \cdot \frac{b-F}F \implies d = y = 0{,}50\,\text{см}.
    \end{align*}
}
\solutionspace{120pt}

\tasknumber{3}%
\task{%
    Оптическая сила двояковыпуклой линзы в воздухе $5\,\text{дптр}$, а в воде $1{,}5\,\text{дптр}$.
    Определить показатель преломления $n$ материала, из которого изготовлена линза.
}
\answer{%
    \begin{align*}
    D_1 &=\cbr{\frac n{n_1} - 1}\cbr{\frac 1{R_1} + \frac 1{R_2}}, \\
    D_2 &=\cbr{\frac n{n_2} - 1}\cbr{\frac 1{R_1} + \frac 1{R_2}}, \\
    \frac {D_2}{D_1} &=\frac{\frac n{n_2} - 1}{\frac n{n_1} - 1} \implies {D_2}\cbr{\frac n{n_1} - 1} = {D_1}\cbr{\frac n{n_2} - 1}  \implies n\cbr{\frac{D_2}{n_1} - \frac{D_1}{n_2}} = D_2 - D_1, \\
    n &= \frac{D_2 - D_1}{\frac{D_2}{n_1} - \frac{D_1}{n_2}} = \frac{n_1 n_2 (D_2 - D_1)}{D_2n_2 - D_1n_1} \approx 1{,}549.
    \end{align*}
}
\solutionspace{120pt}

\tasknumber{4}%
\task{%
    На каком расстоянии от собирающей линзы с фокусным расстоянием $40\,\text{дптр}$
    следует надо поместить предмет, чтобы расстояние
    от предмета до его действительного изображения было наименьшим?
}
\answer{%
    \begin{align*}
    \frac 1a &+ \frac 1b = D \implies b = \frac 1{D - \frac 1a} \implies \ell = a + b = a + \frac a{Da - 1} = \frac{ Da^2 }{Da - 1} \implies \\
    \implies \ell'_a &= \frac{ 2Da \cdot (Da - 1) - Da^2 \cdot D }{\sqr{Da - 1}}= \frac{ D^2a^2 - 2Da}{\sqr{Da - 1}} = \frac{ Da(Da - 2)}{\sqr{Da - 1}}\implies a_{\min} = \frac 2D \approx 50\,\text{мм}.
    \end{align*}
}
\solutionspace{120pt}

\tasknumber{5}%
\task{%
    Предмет в виде отрезка длиной $\ell$ расположен вдоль оптической оси
    собирающей линзы с фокусным расстоянием $F$.
    Середина отрезка расположена
    иа расстоянии $a$ от линзы, которая даёт действительное изображение
    всех точек предмета.
    Определить продольное увеличение предмета.
}
\answer{%
    \begin{align*}
    \frac 1{a + \frac \ell 2} &+ \frac 1b = \frac 1F \implies b = \frac{F\cbr{a + \frac \ell 2}}{a + \frac \ell 2 - F} \\
    \frac 1{a - \frac \ell 2} &+ \frac 1c = \frac 1F \implies c = \frac{F\cbr{a - \frac \ell 2}}{a - \frac \ell 2 - F} \\
    \abs{b - c} &= \abs{\frac{F\cbr{a + \frac \ell 2}}{a + \frac \ell 2 - F} - \frac{F\cbr{a - \frac \ell 2}}{a - \frac \ell 2 - F}}= F\abs{\frac{\cbr{a + \frac \ell 2}\cbr{a - \frac \ell 2 - F} - \cbr{a - \frac \ell 2}\cbr{a + \frac \ell 2 - F}}{ \cbr{a + \frac \ell 2 - F} \cbr{a - \frac \ell 2 - F} }} =  \\
    &= F\abs{\frac{a^2 - \frac {a\ell} 2 - Fa + \frac {a\ell} 2 - \frac {\ell^2} 4 - \frac {F\ell}2 - a^2 - \frac {a\ell}2 + aF + \frac {a\ell}2 + \frac {\ell^2} 4 - \frac {F\ell} 2}{\cbr{a + \frac \ell 2 - F} \cbr{a - \frac \ell 2 - F} }} = \\
    &= F\frac{F\ell}{\sqr{a-F} - \frac {\ell^2}4} = \frac{F^2\ell}{\sqr{a-F} - \frac {\ell^2}4}\implies \Gamma = \frac{\abs{b - c}}\ell = \frac{F^2}{\sqr{a-F} - \frac {\ell^2}4}.
    \end{align*}
}
\solutionspace{120pt}

\tasknumber{6}%
\task{%
    Даны точечный источник света $S$, его изображение $S_1$, полученное с помошью собирающей линзы,
    и ближайший к источнику фокус линзы $F$ (см.
    рис.
    на доске).
    Расстояния $SF = \ell$ и $SS_1 = L$.
    Определить положение линзы и её фокусное расстояние.
}
\answer{%
    \begin{align*}
    \frac 1a + \frac 1b &= \frac 1F, \ell = a - F, L = a + b \implies a = \ell + F, b = L - a = L - \ell - F \\
    \frac 1{\ell + F} + \frac 1{L - \ell - F} &= \frac 1F \\
    F\ell + F^2 + LF - F\ell - F^2 &= L\ell - \ell^2 - F\ell + LF - F\ell - F^2 \\
    0 &= L\ell - \ell^2 - 2F\ell - F^2 \\
    0 &=  F^2 + 2F\ell - L\ell + \ell^2 \\
    F &= -\ell \pm \sqrt{\ell^2 +  L\ell - \ell^2} = -\ell \pm \sqrt{L\ell} \implies F = \sqrt{L\ell} - \ell \\
    a &= \ell + F = \ell + \sqrt{L\ell} - \ell = \sqrt{L\ell}.
    \end{align*}
}
\solutionspace{120pt}

\tasknumber{7}%
\task{%
    Расстояние от освещённого предмета до экрана $100\,\text{см}$.
    Линза, помещенная между ними, даёт чёткое изображение предмета на
    экране при двух положениях, расстояние между которыми $40\,\text{см}$.
    Найти фокусное расстояние линзы.
}
\answer{%
    \begin{align*}
    \frac 1a + \frac 1b &= \frac 1F, \frac 1{a-\ell} + \frac 1{b+\ell} = \frac 1F, a + b = L \\
    \frac 1a + \frac 1b &= \frac 1{a-\ell} + \frac 1{b+\ell}\implies \frac{a + b}{ab} = \frac{(a-\ell) + (b+\ell)}{(a-\ell)(b+\ell)} \\
    ab  &= (a - \ell)(b+\ell) \implies 0  = -b\ell + a\ell - \ell^2 \implies 0 = -b + a - \ell \implies b = a - \ell \\
    a + (a - \ell) &= L \implies a = \frac{L + \ell}2 \implies b = \frac{L - \ell}2 \\
    F &= \frac{ab}{a + b} = \frac{L^2 -\ell^2}{4L} \approx 21\,\text{см}.
    \end{align*}
}
\solutionspace{120pt}

\tasknumber{8}%
\task{%
    Предмет находится на расстоянии $90\,\text{см}$ от экрана.
    Между предметом и экраном помещают линзу, причём при одном
    положении линзы на экране получается увеличенное изображение предмета,
    а при другом — уменьшенное.
    Каково фокусное расстояние линзы, если
    линейные размеры первого изображения в два раза больше второго?
}
\answer{%
    \begin{align*}
    \frac 1a + \frac 1{L-a} &= \frac 1F, h_1 = h \cdot \frac{L-a}a, \\
    \frac 1b + \frac 1{L-b} &= \frac 1F, h_2 = h \cdot \frac{L-b}b, \\
    \frac{h_1}{h_2} &= 2 \implies \frac{(L-a)b}{(L-b)a} = 2, \\
    \frac 1F &= \frac{ L }{a(L-a)} = \frac{ L }{b(L-b)} \implies \frac{L-a}{L-b} = \frac b a \implies \frac {b^2}{a^2} = 2.
    \\
    \frac 1a + \frac 1{L-a} &= \frac 1b + \frac 1{L-b} \implies \frac L{a(L-a)} = \frac L{b(L-b)} \implies \\
    \implies aL - a^2 &= bL - b^2 \implies (a-b)L = (a-b)(a+b) \implies b = L - a, \\
    \frac{\sqr{L-a}}{a^2} &= 2 \implies \frac La - 1 = \sqrt{2} \implies a = \frac{ L }{\sqrt{2} + 1} \\
    F &= \frac{a(L-a)}L = \frac 1L \cdot \frac L{\sqrt{2} + 1} \cdot \frac {L\sqrt{2}}{\sqrt{2} + 1}= \frac { L\sqrt{2} }{ \sqr{\sqrt{2} + 1} } \approx 22\,\text{см}.
    \end{align*}
}

\variantsplitter

\addpersonalvariant{Рената Таржиманова}

\tasknumber{1}%
\task{%
    В каком месте на главной оптической оси двояковыгнутой линзы
    нужно поместить точечный источник света,
    чтобы его изображение оказалось в главном фокусе линзы?
}
\answer{%
    $\text{на половине фокусного расстояния}$
}
\solutionspace{120pt}

\tasknumber{2}%
\task{%
    На экране, расположенном иа расстоянии $80\,\text{см}$ от собирающей линзы,
    получено изображение точечного источника, расположенного на главной оптической оси линзы.
    На какое расстояние переместится изображение на экране,
    если при неподвижном источнике переместить линзу на $1\,\text{см}$ в плоскости, перпендикулярной главной оптической оси?
    Фокусное расстояние линзы равно $30\,\text{см}$.
}
\answer{%
    \begin{align*}
    &\frac 1F = \frac 1a + \frac 1b \implies a = \frac{bF}{b-F} \implies \Gamma = \frac ba = \frac{b-F}F \\
    &y = x \cdot \Gamma = x \cdot \frac{b-F}F \implies d = x + y = 2{,}7\,\text{см}.
    \end{align*}
}
\solutionspace{120pt}

\tasknumber{3}%
\task{%
    Оптическая сила двояковыпуклой линзы в воздухе $5{,}5\,\text{дптр}$, а в воде $1{,}6\,\text{дптр}$.
    Определить показатель преломления $n$ материала, из которого изготовлена линза.
}
\answer{%
    \begin{align*}
    D_1 &=\cbr{\frac n{n_1} - 1}\cbr{\frac 1{R_1} + \frac 1{R_2}}, \\
    D_2 &=\cbr{\frac n{n_2} - 1}\cbr{\frac 1{R_1} + \frac 1{R_2}}, \\
    \frac {D_2}{D_1} &=\frac{\frac n{n_2} - 1}{\frac n{n_1} - 1} \implies {D_2}\cbr{\frac n{n_1} - 1} = {D_1}\cbr{\frac n{n_2} - 1}  \implies n\cbr{\frac{D_2}{n_1} - \frac{D_1}{n_2}} = D_2 - D_1, \\
    n &= \frac{D_2 - D_1}{\frac{D_2}{n_1} - \frac{D_1}{n_2}} = \frac{n_1 n_2 (D_2 - D_1)}{D_2n_2 - D_1n_1} \approx 1{,}538.
    \end{align*}
}
\solutionspace{120pt}

\tasknumber{4}%
\task{%
    На каком расстоянии от собирающей линзы с фокусным расстоянием $40\,\text{дптр}$
    следует надо поместить предмет, чтобы расстояние
    от предмета до его действительного изображения было наименьшим?
}
\answer{%
    \begin{align*}
    \frac 1a &+ \frac 1b = D \implies b = \frac 1{D - \frac 1a} \implies \ell = a + b = a + \frac a{Da - 1} = \frac{ Da^2 }{Da - 1} \implies \\
    \implies \ell'_a &= \frac{ 2Da \cdot (Da - 1) - Da^2 \cdot D }{\sqr{Da - 1}}= \frac{ D^2a^2 - 2Da}{\sqr{Da - 1}} = \frac{ Da(Da - 2)}{\sqr{Da - 1}}\implies a_{\min} = \frac 2D \approx 50\,\text{мм}.
    \end{align*}
}
\solutionspace{120pt}

\tasknumber{5}%
\task{%
    Предмет в виде отрезка длиной $\ell$ расположен вдоль оптической оси
    собирающей линзы с фокусным расстоянием $F$.
    Середина отрезка расположена
    иа расстоянии $a$ от линзы, которая даёт действительное изображение
    всех точек предмета.
    Определить продольное увеличение предмета.
}
\answer{%
    \begin{align*}
    \frac 1{a + \frac \ell 2} &+ \frac 1b = \frac 1F \implies b = \frac{F\cbr{a + \frac \ell 2}}{a + \frac \ell 2 - F} \\
    \frac 1{a - \frac \ell 2} &+ \frac 1c = \frac 1F \implies c = \frac{F\cbr{a - \frac \ell 2}}{a - \frac \ell 2 - F} \\
    \abs{b - c} &= \abs{\frac{F\cbr{a + \frac \ell 2}}{a + \frac \ell 2 - F} - \frac{F\cbr{a - \frac \ell 2}}{a - \frac \ell 2 - F}}= F\abs{\frac{\cbr{a + \frac \ell 2}\cbr{a - \frac \ell 2 - F} - \cbr{a - \frac \ell 2}\cbr{a + \frac \ell 2 - F}}{ \cbr{a + \frac \ell 2 - F} \cbr{a - \frac \ell 2 - F} }} =  \\
    &= F\abs{\frac{a^2 - \frac {a\ell} 2 - Fa + \frac {a\ell} 2 - \frac {\ell^2} 4 - \frac {F\ell}2 - a^2 - \frac {a\ell}2 + aF + \frac {a\ell}2 + \frac {\ell^2} 4 - \frac {F\ell} 2}{\cbr{a + \frac \ell 2 - F} \cbr{a - \frac \ell 2 - F} }} = \\
    &= F\frac{F\ell}{\sqr{a-F} - \frac {\ell^2}4} = \frac{F^2\ell}{\sqr{a-F} - \frac {\ell^2}4}\implies \Gamma = \frac{\abs{b - c}}\ell = \frac{F^2}{\sqr{a-F} - \frac {\ell^2}4}.
    \end{align*}
}
\solutionspace{120pt}

\tasknumber{6}%
\task{%
    Даны точечный источник света $S$, его изображение $S_1$, полученное с помошью собирающей линзы,
    и ближайший к источнику фокус линзы $F$ (см.
    рис.
    на доске).
    Расстояния $SF = \ell$ и $SS_1 = L$.
    Определить положение линзы и её фокусное расстояние.
}
\answer{%
    \begin{align*}
    \frac 1a + \frac 1b &= \frac 1F, \ell = a - F, L = a + b \implies a = \ell + F, b = L - a = L - \ell - F \\
    \frac 1{\ell + F} + \frac 1{L - \ell - F} &= \frac 1F \\
    F\ell + F^2 + LF - F\ell - F^2 &= L\ell - \ell^2 - F\ell + LF - F\ell - F^2 \\
    0 &= L\ell - \ell^2 - 2F\ell - F^2 \\
    0 &=  F^2 + 2F\ell - L\ell + \ell^2 \\
    F &= -\ell \pm \sqrt{\ell^2 +  L\ell - \ell^2} = -\ell \pm \sqrt{L\ell} \implies F = \sqrt{L\ell} - \ell \\
    a &= \ell + F = \ell + \sqrt{L\ell} - \ell = \sqrt{L\ell}.
    \end{align*}
}
\solutionspace{120pt}

\tasknumber{7}%
\task{%
    Расстояние от освещённого предмета до экрана $80\,\text{см}$.
    Линза, помещенная между ними, даёт чёткое изображение предмета на
    экране при двух положениях, расстояние между которыми $40\,\text{см}$.
    Найти фокусное расстояние линзы.
}
\answer{%
    \begin{align*}
    \frac 1a + \frac 1b &= \frac 1F, \frac 1{a-\ell} + \frac 1{b+\ell} = \frac 1F, a + b = L \\
    \frac 1a + \frac 1b &= \frac 1{a-\ell} + \frac 1{b+\ell}\implies \frac{a + b}{ab} = \frac{(a-\ell) + (b+\ell)}{(a-\ell)(b+\ell)} \\
    ab  &= (a - \ell)(b+\ell) \implies 0  = -b\ell + a\ell - \ell^2 \implies 0 = -b + a - \ell \implies b = a - \ell \\
    a + (a - \ell) &= L \implies a = \frac{L + \ell}2 \implies b = \frac{L - \ell}2 \\
    F &= \frac{ab}{a + b} = \frac{L^2 -\ell^2}{4L} \approx 15\,\text{см}.
    \end{align*}
}
\solutionspace{120pt}

\tasknumber{8}%
\task{%
    Предмет находится на расстоянии $80\,\text{см}$ от экрана.
    Между предметом и экраном помещают линзу, причём при одном
    положении линзы на экране получается увеличенное изображение предмета,
    а при другом — уменьшенное.
    Каково фокусное расстояние линзы, если
    линейные размеры первого изображения в пять раз больше второго?
}
\answer{%
    \begin{align*}
    \frac 1a + \frac 1{L-a} &= \frac 1F, h_1 = h \cdot \frac{L-a}a, \\
    \frac 1b + \frac 1{L-b} &= \frac 1F, h_2 = h \cdot \frac{L-b}b, \\
    \frac{h_1}{h_2} &= 5 \implies \frac{(L-a)b}{(L-b)a} = 5, \\
    \frac 1F &= \frac{ L }{a(L-a)} = \frac{ L }{b(L-b)} \implies \frac{L-a}{L-b} = \frac b a \implies \frac {b^2}{a^2} = 5.
    \\
    \frac 1a + \frac 1{L-a} &= \frac 1b + \frac 1{L-b} \implies \frac L{a(L-a)} = \frac L{b(L-b)} \implies \\
    \implies aL - a^2 &= bL - b^2 \implies (a-b)L = (a-b)(a+b) \implies b = L - a, \\
    \frac{\sqr{L-a}}{a^2} &= 5 \implies \frac La - 1 = \sqrt{5} \implies a = \frac{ L }{\sqrt{5} + 1} \\
    F &= \frac{a(L-a)}L = \frac 1L \cdot \frac L{\sqrt{5} + 1} \cdot \frac {L\sqrt{5}}{\sqrt{5} + 1}= \frac { L\sqrt{5} }{ \sqr{\sqrt{5} + 1} } \approx 17{,}1\,\text{см}.
    \end{align*}
}

\variantsplitter

\addpersonalvariant{Андрей Щербаков}

\tasknumber{1}%
\task{%
    В каком месте на главной оптической оси двояковыгнутой линзы
    нужно поместить точечный источник света,
    чтобы его изображение оказалось в главном фокусе линзы?
}
\answer{%
    $\text{на половине фокусного расстояния}$
}
\solutionspace{120pt}

\tasknumber{2}%
\task{%
    На экране, расположенном иа расстоянии $60\,\text{см}$ от собирающей линзы,
    получено изображение точечного источника, расположенного на главной оптической оси линзы.
    На какое расстояние переместится изображение на экране,
    если при неподвижной линзе переместить источник на $1\,\text{см}$ в плоскости, перпендикулярной главной оптической оси?
    Фокусное расстояние линзы равно $20\,\text{см}$.
}
\answer{%
    \begin{align*}
    &\frac 1F = \frac 1a + \frac 1b \implies a = \frac{bF}{b-F} \implies \Gamma = \frac ba = \frac{b-F}F \\
    &y = x \cdot \Gamma = x \cdot \frac{b-F}F \implies d = y = 2\,\text{см}.
    \end{align*}
}
\solutionspace{120pt}

\tasknumber{3}%
\task{%
    Оптическая сила двояковыпуклой линзы в воздухе $5{,}5\,\text{дптр}$, а в воде $1{,}4\,\text{дптр}$.
    Определить показатель преломления $n$ материала, из которого изготовлена линза.
}
\answer{%
    \begin{align*}
    D_1 &=\cbr{\frac n{n_1} - 1}\cbr{\frac 1{R_1} + \frac 1{R_2}}, \\
    D_2 &=\cbr{\frac n{n_2} - 1}\cbr{\frac 1{R_1} + \frac 1{R_2}}, \\
    \frac {D_2}{D_1} &=\frac{\frac n{n_2} - 1}{\frac n{n_1} - 1} \implies {D_2}\cbr{\frac n{n_1} - 1} = {D_1}\cbr{\frac n{n_2} - 1}  \implies n\cbr{\frac{D_2}{n_1} - \frac{D_1}{n_2}} = D_2 - D_1, \\
    n &= \frac{D_2 - D_1}{\frac{D_2}{n_1} - \frac{D_1}{n_2}} = \frac{n_1 n_2 (D_2 - D_1)}{D_2n_2 - D_1n_1} \approx 1{,}499.
    \end{align*}
}
\solutionspace{120pt}

\tasknumber{4}%
\task{%
    На каком расстоянии от собирающей линзы с фокусным расстоянием $30\,\text{дптр}$
    следует надо поместить предмет, чтобы расстояние
    от предмета до его действительного изображения было наименьшим?
}
\answer{%
    \begin{align*}
    \frac 1a &+ \frac 1b = D \implies b = \frac 1{D - \frac 1a} \implies \ell = a + b = a + \frac a{Da - 1} = \frac{ Da^2 }{Da - 1} \implies \\
    \implies \ell'_a &= \frac{ 2Da \cdot (Da - 1) - Da^2 \cdot D }{\sqr{Da - 1}}= \frac{ D^2a^2 - 2Da}{\sqr{Da - 1}} = \frac{ Da(Da - 2)}{\sqr{Da - 1}}\implies a_{\min} = \frac 2D \approx 66{,}7\,\text{мм}.
    \end{align*}
}
\solutionspace{120pt}

\tasknumber{5}%
\task{%
    Предмет в виде отрезка длиной $\ell$ расположен вдоль оптической оси
    собирающей линзы с фокусным расстоянием $F$.
    Середина отрезка расположена
    иа расстоянии $a$ от линзы, которая даёт действительное изображение
    всех точек предмета.
    Определить продольное увеличение предмета.
}
\answer{%
    \begin{align*}
    \frac 1{a + \frac \ell 2} &+ \frac 1b = \frac 1F \implies b = \frac{F\cbr{a + \frac \ell 2}}{a + \frac \ell 2 - F} \\
    \frac 1{a - \frac \ell 2} &+ \frac 1c = \frac 1F \implies c = \frac{F\cbr{a - \frac \ell 2}}{a - \frac \ell 2 - F} \\
    \abs{b - c} &= \abs{\frac{F\cbr{a + \frac \ell 2}}{a + \frac \ell 2 - F} - \frac{F\cbr{a - \frac \ell 2}}{a - \frac \ell 2 - F}}= F\abs{\frac{\cbr{a + \frac \ell 2}\cbr{a - \frac \ell 2 - F} - \cbr{a - \frac \ell 2}\cbr{a + \frac \ell 2 - F}}{ \cbr{a + \frac \ell 2 - F} \cbr{a - \frac \ell 2 - F} }} =  \\
    &= F\abs{\frac{a^2 - \frac {a\ell} 2 - Fa + \frac {a\ell} 2 - \frac {\ell^2} 4 - \frac {F\ell}2 - a^2 - \frac {a\ell}2 + aF + \frac {a\ell}2 + \frac {\ell^2} 4 - \frac {F\ell} 2}{\cbr{a + \frac \ell 2 - F} \cbr{a - \frac \ell 2 - F} }} = \\
    &= F\frac{F\ell}{\sqr{a-F} - \frac {\ell^2}4} = \frac{F^2\ell}{\sqr{a-F} - \frac {\ell^2}4}\implies \Gamma = \frac{\abs{b - c}}\ell = \frac{F^2}{\sqr{a-F} - \frac {\ell^2}4}.
    \end{align*}
}
\solutionspace{120pt}

\tasknumber{6}%
\task{%
    Даны точечный источник света $S$, его изображение $S_1$, полученное с помошью собирающей линзы,
    и ближайший к источнику фокус линзы $F$ (см.
    рис.
    на доске).
    Расстояния $SF = \ell$ и $SS_1 = L$.
    Определить положение линзы и её фокусное расстояние.
}
\answer{%
    \begin{align*}
    \frac 1a + \frac 1b &= \frac 1F, \ell = a - F, L = a + b \implies a = \ell + F, b = L - a = L - \ell - F \\
    \frac 1{\ell + F} + \frac 1{L - \ell - F} &= \frac 1F \\
    F\ell + F^2 + LF - F\ell - F^2 &= L\ell - \ell^2 - F\ell + LF - F\ell - F^2 \\
    0 &= L\ell - \ell^2 - 2F\ell - F^2 \\
    0 &=  F^2 + 2F\ell - L\ell + \ell^2 \\
    F &= -\ell \pm \sqrt{\ell^2 +  L\ell - \ell^2} = -\ell \pm \sqrt{L\ell} \implies F = \sqrt{L\ell} - \ell \\
    a &= \ell + F = \ell + \sqrt{L\ell} - \ell = \sqrt{L\ell}.
    \end{align*}
}
\solutionspace{120pt}

\tasknumber{7}%
\task{%
    Расстояние от освещённого предмета до экрана $80\,\text{см}$.
    Линза, помещенная между ними, даёт чёткое изображение предмета на
    экране при двух положениях, расстояние между которыми $30\,\text{см}$.
    Найти фокусное расстояние линзы.
}
\answer{%
    \begin{align*}
    \frac 1a + \frac 1b &= \frac 1F, \frac 1{a-\ell} + \frac 1{b+\ell} = \frac 1F, a + b = L \\
    \frac 1a + \frac 1b &= \frac 1{a-\ell} + \frac 1{b+\ell}\implies \frac{a + b}{ab} = \frac{(a-\ell) + (b+\ell)}{(a-\ell)(b+\ell)} \\
    ab  &= (a - \ell)(b+\ell) \implies 0  = -b\ell + a\ell - \ell^2 \implies 0 = -b + a - \ell \implies b = a - \ell \\
    a + (a - \ell) &= L \implies a = \frac{L + \ell}2 \implies b = \frac{L - \ell}2 \\
    F &= \frac{ab}{a + b} = \frac{L^2 -\ell^2}{4L} \approx 17{,}2\,\text{см}.
    \end{align*}
}
\solutionspace{120pt}

\tasknumber{8}%
\task{%
    Предмет находится на расстоянии $60\,\text{см}$ от экрана.
    Между предметом и экраном помещают линзу, причём при одном
    положении линзы на экране получается увеличенное изображение предмета,
    а при другом — уменьшенное.
    Каково фокусное расстояние линзы, если
    линейные размеры первого изображения в два раза больше второго?
}
\answer{%
    \begin{align*}
    \frac 1a + \frac 1{L-a} &= \frac 1F, h_1 = h \cdot \frac{L-a}a, \\
    \frac 1b + \frac 1{L-b} &= \frac 1F, h_2 = h \cdot \frac{L-b}b, \\
    \frac{h_1}{h_2} &= 2 \implies \frac{(L-a)b}{(L-b)a} = 2, \\
    \frac 1F &= \frac{ L }{a(L-a)} = \frac{ L }{b(L-b)} \implies \frac{L-a}{L-b} = \frac b a \implies \frac {b^2}{a^2} = 2.
    \\
    \frac 1a + \frac 1{L-a} &= \frac 1b + \frac 1{L-b} \implies \frac L{a(L-a)} = \frac L{b(L-b)} \implies \\
    \implies aL - a^2 &= bL - b^2 \implies (a-b)L = (a-b)(a+b) \implies b = L - a, \\
    \frac{\sqr{L-a}}{a^2} &= 2 \implies \frac La - 1 = \sqrt{2} \implies a = \frac{ L }{\sqrt{2} + 1} \\
    F &= \frac{a(L-a)}L = \frac 1L \cdot \frac L{\sqrt{2} + 1} \cdot \frac {L\sqrt{2}}{\sqrt{2} + 1}= \frac { L\sqrt{2} }{ \sqr{\sqrt{2} + 1} } \approx 14{,}6\,\text{см}.
    \end{align*}
}

\variantsplitter

\addpersonalvariant{Михаил Ярошевский}

\tasknumber{1}%
\task{%
    В каком месте на главной оптической оси двояковыпуклой линзы
    нужно поместить точечный источник света,
    чтобы его изображение оказалось в главном фокусе линзы?
}
\answer{%
    $\text{для мнимого - на половине фокусного, для действительного - на бесконечности}$
}
\solutionspace{120pt}

\tasknumber{2}%
\task{%
    На экране, расположенном иа расстоянии $80\,\text{см}$ от собирающей линзы,
    получено изображение точечного источника, расположенного на главной оптической оси линзы.
    На какое расстояние переместится изображение на экране,
    если при неподвижном источнике переместить линзу на $2\,\text{см}$ в плоскости, перпендикулярной главной оптической оси?
    Фокусное расстояние линзы равно $40\,\text{см}$.
}
\answer{%
    \begin{align*}
    &\frac 1F = \frac 1a + \frac 1b \implies a = \frac{bF}{b-F} \implies \Gamma = \frac ba = \frac{b-F}F \\
    &y = x \cdot \Gamma = x \cdot \frac{b-F}F \implies d = x + y = 4\,\text{см}.
    \end{align*}
}
\solutionspace{120pt}

\tasknumber{3}%
\task{%
    Оптическая сила двояковыпуклой линзы в воздухе $5\,\text{дптр}$, а в воде $1{,}5\,\text{дптр}$.
    Определить показатель преломления $n$ материала, из которого изготовлена линза.
}
\answer{%
    \begin{align*}
    D_1 &=\cbr{\frac n{n_1} - 1}\cbr{\frac 1{R_1} + \frac 1{R_2}}, \\
    D_2 &=\cbr{\frac n{n_2} - 1}\cbr{\frac 1{R_1} + \frac 1{R_2}}, \\
    \frac {D_2}{D_1} &=\frac{\frac n{n_2} - 1}{\frac n{n_1} - 1} \implies {D_2}\cbr{\frac n{n_1} - 1} = {D_1}\cbr{\frac n{n_2} - 1}  \implies n\cbr{\frac{D_2}{n_1} - \frac{D_1}{n_2}} = D_2 - D_1, \\
    n &= \frac{D_2 - D_1}{\frac{D_2}{n_1} - \frac{D_1}{n_2}} = \frac{n_1 n_2 (D_2 - D_1)}{D_2n_2 - D_1n_1} \approx 1{,}549.
    \end{align*}
}
\solutionspace{120pt}

\tasknumber{4}%
\task{%
    На каком расстоянии от собирающей линзы с фокусным расстоянием $40\,\text{дптр}$
    следует надо поместить предмет, чтобы расстояние
    от предмета до его действительного изображения было наименьшим?
}
\answer{%
    \begin{align*}
    \frac 1a &+ \frac 1b = D \implies b = \frac 1{D - \frac 1a} \implies \ell = a + b = a + \frac a{Da - 1} = \frac{ Da^2 }{Da - 1} \implies \\
    \implies \ell'_a &= \frac{ 2Da \cdot (Da - 1) - Da^2 \cdot D }{\sqr{Da - 1}}= \frac{ D^2a^2 - 2Da}{\sqr{Da - 1}} = \frac{ Da(Da - 2)}{\sqr{Da - 1}}\implies a_{\min} = \frac 2D \approx 50\,\text{мм}.
    \end{align*}
}
\solutionspace{120pt}

\tasknumber{5}%
\task{%
    Предмет в виде отрезка длиной $\ell$ расположен вдоль оптической оси
    собирающей линзы с фокусным расстоянием $F$.
    Середина отрезка расположена
    иа расстоянии $a$ от линзы, которая даёт действительное изображение
    всех точек предмета.
    Определить продольное увеличение предмета.
}
\answer{%
    \begin{align*}
    \frac 1{a + \frac \ell 2} &+ \frac 1b = \frac 1F \implies b = \frac{F\cbr{a + \frac \ell 2}}{a + \frac \ell 2 - F} \\
    \frac 1{a - \frac \ell 2} &+ \frac 1c = \frac 1F \implies c = \frac{F\cbr{a - \frac \ell 2}}{a - \frac \ell 2 - F} \\
    \abs{b - c} &= \abs{\frac{F\cbr{a + \frac \ell 2}}{a + \frac \ell 2 - F} - \frac{F\cbr{a - \frac \ell 2}}{a - \frac \ell 2 - F}}= F\abs{\frac{\cbr{a + \frac \ell 2}\cbr{a - \frac \ell 2 - F} - \cbr{a - \frac \ell 2}\cbr{a + \frac \ell 2 - F}}{ \cbr{a + \frac \ell 2 - F} \cbr{a - \frac \ell 2 - F} }} =  \\
    &= F\abs{\frac{a^2 - \frac {a\ell} 2 - Fa + \frac {a\ell} 2 - \frac {\ell^2} 4 - \frac {F\ell}2 - a^2 - \frac {a\ell}2 + aF + \frac {a\ell}2 + \frac {\ell^2} 4 - \frac {F\ell} 2}{\cbr{a + \frac \ell 2 - F} \cbr{a - \frac \ell 2 - F} }} = \\
    &= F\frac{F\ell}{\sqr{a-F} - \frac {\ell^2}4} = \frac{F^2\ell}{\sqr{a-F} - \frac {\ell^2}4}\implies \Gamma = \frac{\abs{b - c}}\ell = \frac{F^2}{\sqr{a-F} - \frac {\ell^2}4}.
    \end{align*}
}
\solutionspace{120pt}

\tasknumber{6}%
\task{%
    Даны точечный источник света $S$, его изображение $S_1$, полученное с помошью собирающей линзы,
    и ближайший к источнику фокус линзы $F$ (см.
    рис.
    на доске).
    Расстояния $SF = \ell$ и $SS_1 = L$.
    Определить положение линзы и её фокусное расстояние.
}
\answer{%
    \begin{align*}
    \frac 1a + \frac 1b &= \frac 1F, \ell = a - F, L = a + b \implies a = \ell + F, b = L - a = L - \ell - F \\
    \frac 1{\ell + F} + \frac 1{L - \ell - F} &= \frac 1F \\
    F\ell + F^2 + LF - F\ell - F^2 &= L\ell - \ell^2 - F\ell + LF - F\ell - F^2 \\
    0 &= L\ell - \ell^2 - 2F\ell - F^2 \\
    0 &=  F^2 + 2F\ell - L\ell + \ell^2 \\
    F &= -\ell \pm \sqrt{\ell^2 +  L\ell - \ell^2} = -\ell \pm \sqrt{L\ell} \implies F = \sqrt{L\ell} - \ell \\
    a &= \ell + F = \ell + \sqrt{L\ell} - \ell = \sqrt{L\ell}.
    \end{align*}
}
\solutionspace{120pt}

\tasknumber{7}%
\task{%
    Расстояние от освещённого предмета до экрана $100\,\text{см}$.
    Линза, помещенная между ними, даёт чёткое изображение предмета на
    экране при двух положениях, расстояние между которыми $40\,\text{см}$.
    Найти фокусное расстояние линзы.
}
\answer{%
    \begin{align*}
    \frac 1a + \frac 1b &= \frac 1F, \frac 1{a-\ell} + \frac 1{b+\ell} = \frac 1F, a + b = L \\
    \frac 1a + \frac 1b &= \frac 1{a-\ell} + \frac 1{b+\ell}\implies \frac{a + b}{ab} = \frac{(a-\ell) + (b+\ell)}{(a-\ell)(b+\ell)} \\
    ab  &= (a - \ell)(b+\ell) \implies 0  = -b\ell + a\ell - \ell^2 \implies 0 = -b + a - \ell \implies b = a - \ell \\
    a + (a - \ell) &= L \implies a = \frac{L + \ell}2 \implies b = \frac{L - \ell}2 \\
    F &= \frac{ab}{a + b} = \frac{L^2 -\ell^2}{4L} \approx 21\,\text{см}.
    \end{align*}
}
\solutionspace{120pt}

\tasknumber{8}%
\task{%
    Предмет находится на расстоянии $90\,\text{см}$ от экрана.
    Между предметом и экраном помещают линзу, причём при одном
    положении линзы на экране получается увеличенное изображение предмета,
    а при другом — уменьшенное.
    Каково фокусное расстояние линзы, если
    линейные размеры первого изображения в два раза больше второго?
}
\answer{%
    \begin{align*}
    \frac 1a + \frac 1{L-a} &= \frac 1F, h_1 = h \cdot \frac{L-a}a, \\
    \frac 1b + \frac 1{L-b} &= \frac 1F, h_2 = h \cdot \frac{L-b}b, \\
    \frac{h_1}{h_2} &= 2 \implies \frac{(L-a)b}{(L-b)a} = 2, \\
    \frac 1F &= \frac{ L }{a(L-a)} = \frac{ L }{b(L-b)} \implies \frac{L-a}{L-b} = \frac b a \implies \frac {b^2}{a^2} = 2.
    \\
    \frac 1a + \frac 1{L-a} &= \frac 1b + \frac 1{L-b} \implies \frac L{a(L-a)} = \frac L{b(L-b)} \implies \\
    \implies aL - a^2 &= bL - b^2 \implies (a-b)L = (a-b)(a+b) \implies b = L - a, \\
    \frac{\sqr{L-a}}{a^2} &= 2 \implies \frac La - 1 = \sqrt{2} \implies a = \frac{ L }{\sqrt{2} + 1} \\
    F &= \frac{a(L-a)}L = \frac 1L \cdot \frac L{\sqrt{2} + 1} \cdot \frac {L\sqrt{2}}{\sqrt{2} + 1}= \frac { L\sqrt{2} }{ \sqr{\sqrt{2} + 1} } \approx 22\,\text{см}.
    \end{align*}
}

\variantsplitter

\addpersonalvariant{Алексей Алимпиев}

\tasknumber{1}%
\task{%
    В каком месте на главной оптической оси двояковыпуклой линзы
    нужно поместить точечный источник света,
    чтобы его изображение оказалось в главном фокусе линзы?
}
\answer{%
    $\text{для мнимого - на половине фокусного, для действительного - на бесконечности}$
}
\solutionspace{120pt}

\tasknumber{2}%
\task{%
    На экране, расположенном иа расстоянии $80\,\text{см}$ от собирающей линзы,
    получено изображение точечного источника, расположенного на главной оптической оси линзы.
    На какое расстояние переместится изображение на экране,
    если при неподвижной линзе переместить источник на $1\,\text{см}$ в плоскости, перпендикулярной главной оптической оси?
    Фокусное расстояние линзы равно $20\,\text{см}$.
}
\answer{%
    \begin{align*}
    &\frac 1F = \frac 1a + \frac 1b \implies a = \frac{bF}{b-F} \implies \Gamma = \frac ba = \frac{b-F}F \\
    &y = x \cdot \Gamma = x \cdot \frac{b-F}F \implies d = y = 3\,\text{см}.
    \end{align*}
}
\solutionspace{120pt}

\tasknumber{3}%
\task{%
    Оптическая сила двояковыпуклой линзы в воздухе $5\,\text{дптр}$, а в воде $1{,}5\,\text{дптр}$.
    Определить показатель преломления $n$ материала, из которого изготовлена линза.
}
\answer{%
    \begin{align*}
    D_1 &=\cbr{\frac n{n_1} - 1}\cbr{\frac 1{R_1} + \frac 1{R_2}}, \\
    D_2 &=\cbr{\frac n{n_2} - 1}\cbr{\frac 1{R_1} + \frac 1{R_2}}, \\
    \frac {D_2}{D_1} &=\frac{\frac n{n_2} - 1}{\frac n{n_1} - 1} \implies {D_2}\cbr{\frac n{n_1} - 1} = {D_1}\cbr{\frac n{n_2} - 1}  \implies n\cbr{\frac{D_2}{n_1} - \frac{D_1}{n_2}} = D_2 - D_1, \\
    n &= \frac{D_2 - D_1}{\frac{D_2}{n_1} - \frac{D_1}{n_2}} = \frac{n_1 n_2 (D_2 - D_1)}{D_2n_2 - D_1n_1} \approx 1{,}549.
    \end{align*}
}
\solutionspace{120pt}

\tasknumber{4}%
\task{%
    На каком расстоянии от собирающей линзы с фокусным расстоянием $50\,\text{дптр}$
    следует надо поместить предмет, чтобы расстояние
    от предмета до его действительного изображения было наименьшим?
}
\answer{%
    \begin{align*}
    \frac 1a &+ \frac 1b = D \implies b = \frac 1{D - \frac 1a} \implies \ell = a + b = a + \frac a{Da - 1} = \frac{ Da^2 }{Da - 1} \implies \\
    \implies \ell'_a &= \frac{ 2Da \cdot (Da - 1) - Da^2 \cdot D }{\sqr{Da - 1}}= \frac{ D^2a^2 - 2Da}{\sqr{Da - 1}} = \frac{ Da(Da - 2)}{\sqr{Da - 1}}\implies a_{\min} = \frac 2D \approx 40\,\text{мм}.
    \end{align*}
}
\solutionspace{120pt}

\tasknumber{5}%
\task{%
    Предмет в виде отрезка длиной $\ell$ расположен вдоль оптической оси
    собирающей линзы с фокусным расстоянием $F$.
    Середина отрезка расположена
    иа расстоянии $a$ от линзы, которая даёт действительное изображение
    всех точек предмета.
    Определить продольное увеличение предмета.
}
\answer{%
    \begin{align*}
    \frac 1{a + \frac \ell 2} &+ \frac 1b = \frac 1F \implies b = \frac{F\cbr{a + \frac \ell 2}}{a + \frac \ell 2 - F} \\
    \frac 1{a - \frac \ell 2} &+ \frac 1c = \frac 1F \implies c = \frac{F\cbr{a - \frac \ell 2}}{a - \frac \ell 2 - F} \\
    \abs{b - c} &= \abs{\frac{F\cbr{a + \frac \ell 2}}{a + \frac \ell 2 - F} - \frac{F\cbr{a - \frac \ell 2}}{a - \frac \ell 2 - F}}= F\abs{\frac{\cbr{a + \frac \ell 2}\cbr{a - \frac \ell 2 - F} - \cbr{a - \frac \ell 2}\cbr{a + \frac \ell 2 - F}}{ \cbr{a + \frac \ell 2 - F} \cbr{a - \frac \ell 2 - F} }} =  \\
    &= F\abs{\frac{a^2 - \frac {a\ell} 2 - Fa + \frac {a\ell} 2 - \frac {\ell^2} 4 - \frac {F\ell}2 - a^2 - \frac {a\ell}2 + aF + \frac {a\ell}2 + \frac {\ell^2} 4 - \frac {F\ell} 2}{\cbr{a + \frac \ell 2 - F} \cbr{a - \frac \ell 2 - F} }} = \\
    &= F\frac{F\ell}{\sqr{a-F} - \frac {\ell^2}4} = \frac{F^2\ell}{\sqr{a-F} - \frac {\ell^2}4}\implies \Gamma = \frac{\abs{b - c}}\ell = \frac{F^2}{\sqr{a-F} - \frac {\ell^2}4}.
    \end{align*}
}
\solutionspace{120pt}

\tasknumber{6}%
\task{%
    Даны точечный источник света $S$, его изображение $S_1$, полученное с помошью собирающей линзы,
    и ближайший к источнику фокус линзы $F$ (см.
    рис.
    на доске).
    Расстояния $SF = \ell$ и $SS_1 = L$.
    Определить положение линзы и её фокусное расстояние.
}
\answer{%
    \begin{align*}
    \frac 1a + \frac 1b &= \frac 1F, \ell = a - F, L = a + b \implies a = \ell + F, b = L - a = L - \ell - F \\
    \frac 1{\ell + F} + \frac 1{L - \ell - F} &= \frac 1F \\
    F\ell + F^2 + LF - F\ell - F^2 &= L\ell - \ell^2 - F\ell + LF - F\ell - F^2 \\
    0 &= L\ell - \ell^2 - 2F\ell - F^2 \\
    0 &=  F^2 + 2F\ell - L\ell + \ell^2 \\
    F &= -\ell \pm \sqrt{\ell^2 +  L\ell - \ell^2} = -\ell \pm \sqrt{L\ell} \implies F = \sqrt{L\ell} - \ell \\
    a &= \ell + F = \ell + \sqrt{L\ell} - \ell = \sqrt{L\ell}.
    \end{align*}
}
\solutionspace{120pt}

\tasknumber{7}%
\task{%
    Расстояние от освещённого предмета до экрана $80\,\text{см}$.
    Линза, помещенная между ними, даёт чёткое изображение предмета на
    экране при двух положениях, расстояние между которыми $40\,\text{см}$.
    Найти фокусное расстояние линзы.
}
\answer{%
    \begin{align*}
    \frac 1a + \frac 1b &= \frac 1F, \frac 1{a-\ell} + \frac 1{b+\ell} = \frac 1F, a + b = L \\
    \frac 1a + \frac 1b &= \frac 1{a-\ell} + \frac 1{b+\ell}\implies \frac{a + b}{ab} = \frac{(a-\ell) + (b+\ell)}{(a-\ell)(b+\ell)} \\
    ab  &= (a - \ell)(b+\ell) \implies 0  = -b\ell + a\ell - \ell^2 \implies 0 = -b + a - \ell \implies b = a - \ell \\
    a + (a - \ell) &= L \implies a = \frac{L + \ell}2 \implies b = \frac{L - \ell}2 \\
    F &= \frac{ab}{a + b} = \frac{L^2 -\ell^2}{4L} \approx 15\,\text{см}.
    \end{align*}
}
\solutionspace{120pt}

\tasknumber{8}%
\task{%
    Предмет находится на расстоянии $80\,\text{см}$ от экрана.
    Между предметом и экраном помещают линзу, причём при одном
    положении линзы на экране получается увеличенное изображение предмета,
    а при другом — уменьшенное.
    Каково фокусное расстояние линзы, если
    линейные размеры первого изображения в два раза больше второго?
}
\answer{%
    \begin{align*}
    \frac 1a + \frac 1{L-a} &= \frac 1F, h_1 = h \cdot \frac{L-a}a, \\
    \frac 1b + \frac 1{L-b} &= \frac 1F, h_2 = h \cdot \frac{L-b}b, \\
    \frac{h_1}{h_2} &= 2 \implies \frac{(L-a)b}{(L-b)a} = 2, \\
    \frac 1F &= \frac{ L }{a(L-a)} = \frac{ L }{b(L-b)} \implies \frac{L-a}{L-b} = \frac b a \implies \frac {b^2}{a^2} = 2.
    \\
    \frac 1a + \frac 1{L-a} &= \frac 1b + \frac 1{L-b} \implies \frac L{a(L-a)} = \frac L{b(L-b)} \implies \\
    \implies aL - a^2 &= bL - b^2 \implies (a-b)L = (a-b)(a+b) \implies b = L - a, \\
    \frac{\sqr{L-a}}{a^2} &= 2 \implies \frac La - 1 = \sqrt{2} \implies a = \frac{ L }{\sqrt{2} + 1} \\
    F &= \frac{a(L-a)}L = \frac 1L \cdot \frac L{\sqrt{2} + 1} \cdot \frac {L\sqrt{2}}{\sqrt{2} + 1}= \frac { L\sqrt{2} }{ \sqr{\sqrt{2} + 1} } \approx 19{,}4\,\text{см}.
    \end{align*}
}

\variantsplitter

\addpersonalvariant{Евгений Васин}

\tasknumber{1}%
\task{%
    В каком месте на главной оптической оси двояковыгнутой линзы
    нужно поместить точечный источник света,
    чтобы его изображение оказалось в главном фокусе линзы?
}
\answer{%
    $\text{на половине фокусного расстояния}$
}
\solutionspace{120pt}

\tasknumber{2}%
\task{%
    На экране, расположенном иа расстоянии $80\,\text{см}$ от собирающей линзы,
    получено изображение точечного источника, расположенного на главной оптической оси линзы.
    На какое расстояние переместится изображение на экране,
    если при неподвижном источнике переместить линзу на $3\,\text{см}$ в плоскости, перпендикулярной главной оптической оси?
    Фокусное расстояние линзы равно $20\,\text{см}$.
}
\answer{%
    \begin{align*}
    &\frac 1F = \frac 1a + \frac 1b \implies a = \frac{bF}{b-F} \implies \Gamma = \frac ba = \frac{b-F}F \\
    &y = x \cdot \Gamma = x \cdot \frac{b-F}F \implies d = x + y = 12\,\text{см}.
    \end{align*}
}
\solutionspace{120pt}

\tasknumber{3}%
\task{%
    Оптическая сила двояковыпуклой линзы в воздухе $5{,}5\,\text{дптр}$, а в воде $1{,}6\,\text{дптр}$.
    Определить показатель преломления $n$ материала, из которого изготовлена линза.
}
\answer{%
    \begin{align*}
    D_1 &=\cbr{\frac n{n_1} - 1}\cbr{\frac 1{R_1} + \frac 1{R_2}}, \\
    D_2 &=\cbr{\frac n{n_2} - 1}\cbr{\frac 1{R_1} + \frac 1{R_2}}, \\
    \frac {D_2}{D_1} &=\frac{\frac n{n_2} - 1}{\frac n{n_1} - 1} \implies {D_2}\cbr{\frac n{n_1} - 1} = {D_1}\cbr{\frac n{n_2} - 1}  \implies n\cbr{\frac{D_2}{n_1} - \frac{D_1}{n_2}} = D_2 - D_1, \\
    n &= \frac{D_2 - D_1}{\frac{D_2}{n_1} - \frac{D_1}{n_2}} = \frac{n_1 n_2 (D_2 - D_1)}{D_2n_2 - D_1n_1} \approx 1{,}538.
    \end{align*}
}
\solutionspace{120pt}

\tasknumber{4}%
\task{%
    На каком расстоянии от собирающей линзы с фокусным расстоянием $40\,\text{дптр}$
    следует надо поместить предмет, чтобы расстояние
    от предмета до его действительного изображения было наименьшим?
}
\answer{%
    \begin{align*}
    \frac 1a &+ \frac 1b = D \implies b = \frac 1{D - \frac 1a} \implies \ell = a + b = a + \frac a{Da - 1} = \frac{ Da^2 }{Da - 1} \implies \\
    \implies \ell'_a &= \frac{ 2Da \cdot (Da - 1) - Da^2 \cdot D }{\sqr{Da - 1}}= \frac{ D^2a^2 - 2Da}{\sqr{Da - 1}} = \frac{ Da(Da - 2)}{\sqr{Da - 1}}\implies a_{\min} = \frac 2D \approx 50\,\text{мм}.
    \end{align*}
}
\solutionspace{120pt}

\tasknumber{5}%
\task{%
    Предмет в виде отрезка длиной $\ell$ расположен вдоль оптической оси
    собирающей линзы с фокусным расстоянием $F$.
    Середина отрезка расположена
    иа расстоянии $a$ от линзы, которая даёт действительное изображение
    всех точек предмета.
    Определить продольное увеличение предмета.
}
\answer{%
    \begin{align*}
    \frac 1{a + \frac \ell 2} &+ \frac 1b = \frac 1F \implies b = \frac{F\cbr{a + \frac \ell 2}}{a + \frac \ell 2 - F} \\
    \frac 1{a - \frac \ell 2} &+ \frac 1c = \frac 1F \implies c = \frac{F\cbr{a - \frac \ell 2}}{a - \frac \ell 2 - F} \\
    \abs{b - c} &= \abs{\frac{F\cbr{a + \frac \ell 2}}{a + \frac \ell 2 - F} - \frac{F\cbr{a - \frac \ell 2}}{a - \frac \ell 2 - F}}= F\abs{\frac{\cbr{a + \frac \ell 2}\cbr{a - \frac \ell 2 - F} - \cbr{a - \frac \ell 2}\cbr{a + \frac \ell 2 - F}}{ \cbr{a + \frac \ell 2 - F} \cbr{a - \frac \ell 2 - F} }} =  \\
    &= F\abs{\frac{a^2 - \frac {a\ell} 2 - Fa + \frac {a\ell} 2 - \frac {\ell^2} 4 - \frac {F\ell}2 - a^2 - \frac {a\ell}2 + aF + \frac {a\ell}2 + \frac {\ell^2} 4 - \frac {F\ell} 2}{\cbr{a + \frac \ell 2 - F} \cbr{a - \frac \ell 2 - F} }} = \\
    &= F\frac{F\ell}{\sqr{a-F} - \frac {\ell^2}4} = \frac{F^2\ell}{\sqr{a-F} - \frac {\ell^2}4}\implies \Gamma = \frac{\abs{b - c}}\ell = \frac{F^2}{\sqr{a-F} - \frac {\ell^2}4}.
    \end{align*}
}
\solutionspace{120pt}

\tasknumber{6}%
\task{%
    Даны точечный источник света $S$, его изображение $S_1$, полученное с помошью собирающей линзы,
    и ближайший к источнику фокус линзы $F$ (см.
    рис.
    на доске).
    Расстояния $SF = \ell$ и $SS_1 = L$.
    Определить положение линзы и её фокусное расстояние.
}
\answer{%
    \begin{align*}
    \frac 1a + \frac 1b &= \frac 1F, \ell = a - F, L = a + b \implies a = \ell + F, b = L - a = L - \ell - F \\
    \frac 1{\ell + F} + \frac 1{L - \ell - F} &= \frac 1F \\
    F\ell + F^2 + LF - F\ell - F^2 &= L\ell - \ell^2 - F\ell + LF - F\ell - F^2 \\
    0 &= L\ell - \ell^2 - 2F\ell - F^2 \\
    0 &=  F^2 + 2F\ell - L\ell + \ell^2 \\
    F &= -\ell \pm \sqrt{\ell^2 +  L\ell - \ell^2} = -\ell \pm \sqrt{L\ell} \implies F = \sqrt{L\ell} - \ell \\
    a &= \ell + F = \ell + \sqrt{L\ell} - \ell = \sqrt{L\ell}.
    \end{align*}
}
\solutionspace{120pt}

\tasknumber{7}%
\task{%
    Расстояние от освещённого предмета до экрана $100\,\text{см}$.
    Линза, помещенная между ними, даёт чёткое изображение предмета на
    экране при двух положениях, расстояние между которыми $20\,\text{см}$.
    Найти фокусное расстояние линзы.
}
\answer{%
    \begin{align*}
    \frac 1a + \frac 1b &= \frac 1F, \frac 1{a-\ell} + \frac 1{b+\ell} = \frac 1F, a + b = L \\
    \frac 1a + \frac 1b &= \frac 1{a-\ell} + \frac 1{b+\ell}\implies \frac{a + b}{ab} = \frac{(a-\ell) + (b+\ell)}{(a-\ell)(b+\ell)} \\
    ab  &= (a - \ell)(b+\ell) \implies 0  = -b\ell + a\ell - \ell^2 \implies 0 = -b + a - \ell \implies b = a - \ell \\
    a + (a - \ell) &= L \implies a = \frac{L + \ell}2 \implies b = \frac{L - \ell}2 \\
    F &= \frac{ab}{a + b} = \frac{L^2 -\ell^2}{4L} \approx 24\,\text{см}.
    \end{align*}
}
\solutionspace{120pt}

\tasknumber{8}%
\task{%
    Предмет находится на расстоянии $90\,\text{см}$ от экрана.
    Между предметом и экраном помещают линзу, причём при одном
    положении линзы на экране получается увеличенное изображение предмета,
    а при другом — уменьшенное.
    Каково фокусное расстояние линзы, если
    линейные размеры первого изображения в три раза больше второго?
}
\answer{%
    \begin{align*}
    \frac 1a + \frac 1{L-a} &= \frac 1F, h_1 = h \cdot \frac{L-a}a, \\
    \frac 1b + \frac 1{L-b} &= \frac 1F, h_2 = h \cdot \frac{L-b}b, \\
    \frac{h_1}{h_2} &= 3 \implies \frac{(L-a)b}{(L-b)a} = 3, \\
    \frac 1F &= \frac{ L }{a(L-a)} = \frac{ L }{b(L-b)} \implies \frac{L-a}{L-b} = \frac b a \implies \frac {b^2}{a^2} = 3.
    \\
    \frac 1a + \frac 1{L-a} &= \frac 1b + \frac 1{L-b} \implies \frac L{a(L-a)} = \frac L{b(L-b)} \implies \\
    \implies aL - a^2 &= bL - b^2 \implies (a-b)L = (a-b)(a+b) \implies b = L - a, \\
    \frac{\sqr{L-a}}{a^2} &= 3 \implies \frac La - 1 = \sqrt{3} \implies a = \frac{ L }{\sqrt{3} + 1} \\
    F &= \frac{a(L-a)}L = \frac 1L \cdot \frac L{\sqrt{3} + 1} \cdot \frac {L\sqrt{3}}{\sqrt{3} + 1}= \frac { L\sqrt{3} }{ \sqr{\sqrt{3} + 1} } \approx 21\,\text{см}.
    \end{align*}
}

\variantsplitter

\addpersonalvariant{Вячеслав Волохов}

\tasknumber{1}%
\task{%
    В каком месте на главной оптической оси двояковыпуклой линзы
    нужно поместить точечный источник света,
    чтобы его изображение оказалось в главном фокусе линзы?
}
\answer{%
    $\text{для мнимого - на половине фокусного, для действительного - на бесконечности}$
}
\solutionspace{120pt}

\tasknumber{2}%
\task{%
    На экране, расположенном иа расстоянии $60\,\text{см}$ от собирающей линзы,
    получено изображение точечного источника, расположенного на главной оптической оси линзы.
    На какое расстояние переместится изображение на экране,
    если при неподвижной линзе переместить источник на $2\,\text{см}$ в плоскости, перпендикулярной главной оптической оси?
    Фокусное расстояние линзы равно $20\,\text{см}$.
}
\answer{%
    \begin{align*}
    &\frac 1F = \frac 1a + \frac 1b \implies a = \frac{bF}{b-F} \implies \Gamma = \frac ba = \frac{b-F}F \\
    &y = x \cdot \Gamma = x \cdot \frac{b-F}F \implies d = y = 4\,\text{см}.
    \end{align*}
}
\solutionspace{120pt}

\tasknumber{3}%
\task{%
    Оптическая сила двояковыпуклой линзы в воздухе $5\,\text{дптр}$, а в воде $1{,}4\,\text{дптр}$.
    Определить показатель преломления $n$ материала, из которого изготовлена линза.
}
\answer{%
    \begin{align*}
    D_1 &=\cbr{\frac n{n_1} - 1}\cbr{\frac 1{R_1} + \frac 1{R_2}}, \\
    D_2 &=\cbr{\frac n{n_2} - 1}\cbr{\frac 1{R_1} + \frac 1{R_2}}, \\
    \frac {D_2}{D_1} &=\frac{\frac n{n_2} - 1}{\frac n{n_1} - 1} \implies {D_2}\cbr{\frac n{n_1} - 1} = {D_1}\cbr{\frac n{n_2} - 1}  \implies n\cbr{\frac{D_2}{n_1} - \frac{D_1}{n_2}} = D_2 - D_1, \\
    n &= \frac{D_2 - D_1}{\frac{D_2}{n_1} - \frac{D_1}{n_2}} = \frac{n_1 n_2 (D_2 - D_1)}{D_2n_2 - D_1n_1} \approx 1{,}526.
    \end{align*}
}
\solutionspace{120pt}

\tasknumber{4}%
\task{%
    На каком расстоянии от собирающей линзы с фокусным расстоянием $40\,\text{дптр}$
    следует надо поместить предмет, чтобы расстояние
    от предмета до его действительного изображения было наименьшим?
}
\answer{%
    \begin{align*}
    \frac 1a &+ \frac 1b = D \implies b = \frac 1{D - \frac 1a} \implies \ell = a + b = a + \frac a{Da - 1} = \frac{ Da^2 }{Da - 1} \implies \\
    \implies \ell'_a &= \frac{ 2Da \cdot (Da - 1) - Da^2 \cdot D }{\sqr{Da - 1}}= \frac{ D^2a^2 - 2Da}{\sqr{Da - 1}} = \frac{ Da(Da - 2)}{\sqr{Da - 1}}\implies a_{\min} = \frac 2D \approx 50\,\text{мм}.
    \end{align*}
}
\solutionspace{120pt}

\tasknumber{5}%
\task{%
    Предмет в виде отрезка длиной $\ell$ расположен вдоль оптической оси
    собирающей линзы с фокусным расстоянием $F$.
    Середина отрезка расположена
    иа расстоянии $a$ от линзы, которая даёт действительное изображение
    всех точек предмета.
    Определить продольное увеличение предмета.
}
\answer{%
    \begin{align*}
    \frac 1{a + \frac \ell 2} &+ \frac 1b = \frac 1F \implies b = \frac{F\cbr{a + \frac \ell 2}}{a + \frac \ell 2 - F} \\
    \frac 1{a - \frac \ell 2} &+ \frac 1c = \frac 1F \implies c = \frac{F\cbr{a - \frac \ell 2}}{a - \frac \ell 2 - F} \\
    \abs{b - c} &= \abs{\frac{F\cbr{a + \frac \ell 2}}{a + \frac \ell 2 - F} - \frac{F\cbr{a - \frac \ell 2}}{a - \frac \ell 2 - F}}= F\abs{\frac{\cbr{a + \frac \ell 2}\cbr{a - \frac \ell 2 - F} - \cbr{a - \frac \ell 2}\cbr{a + \frac \ell 2 - F}}{ \cbr{a + \frac \ell 2 - F} \cbr{a - \frac \ell 2 - F} }} =  \\
    &= F\abs{\frac{a^2 - \frac {a\ell} 2 - Fa + \frac {a\ell} 2 - \frac {\ell^2} 4 - \frac {F\ell}2 - a^2 - \frac {a\ell}2 + aF + \frac {a\ell}2 + \frac {\ell^2} 4 - \frac {F\ell} 2}{\cbr{a + \frac \ell 2 - F} \cbr{a - \frac \ell 2 - F} }} = \\
    &= F\frac{F\ell}{\sqr{a-F} - \frac {\ell^2}4} = \frac{F^2\ell}{\sqr{a-F} - \frac {\ell^2}4}\implies \Gamma = \frac{\abs{b - c}}\ell = \frac{F^2}{\sqr{a-F} - \frac {\ell^2}4}.
    \end{align*}
}
\solutionspace{120pt}

\tasknumber{6}%
\task{%
    Даны точечный источник света $S$, его изображение $S_1$, полученное с помошью собирающей линзы,
    и ближайший к источнику фокус линзы $F$ (см.
    рис.
    на доске).
    Расстояния $SF = \ell$ и $SS_1 = L$.
    Определить положение линзы и её фокусное расстояние.
}
\answer{%
    \begin{align*}
    \frac 1a + \frac 1b &= \frac 1F, \ell = a - F, L = a + b \implies a = \ell + F, b = L - a = L - \ell - F \\
    \frac 1{\ell + F} + \frac 1{L - \ell - F} &= \frac 1F \\
    F\ell + F^2 + LF - F\ell - F^2 &= L\ell - \ell^2 - F\ell + LF - F\ell - F^2 \\
    0 &= L\ell - \ell^2 - 2F\ell - F^2 \\
    0 &=  F^2 + 2F\ell - L\ell + \ell^2 \\
    F &= -\ell \pm \sqrt{\ell^2 +  L\ell - \ell^2} = -\ell \pm \sqrt{L\ell} \implies F = \sqrt{L\ell} - \ell \\
    a &= \ell + F = \ell + \sqrt{L\ell} - \ell = \sqrt{L\ell}.
    \end{align*}
}
\solutionspace{120pt}

\tasknumber{7}%
\task{%
    Расстояние от освещённого предмета до экрана $80\,\text{см}$.
    Линза, помещенная между ними, даёт чёткое изображение предмета на
    экране при двух положениях, расстояние между которыми $30\,\text{см}$.
    Найти фокусное расстояние линзы.
}
\answer{%
    \begin{align*}
    \frac 1a + \frac 1b &= \frac 1F, \frac 1{a-\ell} + \frac 1{b+\ell} = \frac 1F, a + b = L \\
    \frac 1a + \frac 1b &= \frac 1{a-\ell} + \frac 1{b+\ell}\implies \frac{a + b}{ab} = \frac{(a-\ell) + (b+\ell)}{(a-\ell)(b+\ell)} \\
    ab  &= (a - \ell)(b+\ell) \implies 0  = -b\ell + a\ell - \ell^2 \implies 0 = -b + a - \ell \implies b = a - \ell \\
    a + (a - \ell) &= L \implies a = \frac{L + \ell}2 \implies b = \frac{L - \ell}2 \\
    F &= \frac{ab}{a + b} = \frac{L^2 -\ell^2}{4L} \approx 17{,}2\,\text{см}.
    \end{align*}
}
\solutionspace{120pt}

\tasknumber{8}%
\task{%
    Предмет находится на расстоянии $60\,\text{см}$ от экрана.
    Между предметом и экраном помещают линзу, причём при одном
    положении линзы на экране получается увеличенное изображение предмета,
    а при другом — уменьшенное.
    Каково фокусное расстояние линзы, если
    линейные размеры первого изображения в пять раз больше второго?
}
\answer{%
    \begin{align*}
    \frac 1a + \frac 1{L-a} &= \frac 1F, h_1 = h \cdot \frac{L-a}a, \\
    \frac 1b + \frac 1{L-b} &= \frac 1F, h_2 = h \cdot \frac{L-b}b, \\
    \frac{h_1}{h_2} &= 5 \implies \frac{(L-a)b}{(L-b)a} = 5, \\
    \frac 1F &= \frac{ L }{a(L-a)} = \frac{ L }{b(L-b)} \implies \frac{L-a}{L-b} = \frac b a \implies \frac {b^2}{a^2} = 5.
    \\
    \frac 1a + \frac 1{L-a} &= \frac 1b + \frac 1{L-b} \implies \frac L{a(L-a)} = \frac L{b(L-b)} \implies \\
    \implies aL - a^2 &= bL - b^2 \implies (a-b)L = (a-b)(a+b) \implies b = L - a, \\
    \frac{\sqr{L-a}}{a^2} &= 5 \implies \frac La - 1 = \sqrt{5} \implies a = \frac{ L }{\sqrt{5} + 1} \\
    F &= \frac{a(L-a)}L = \frac 1L \cdot \frac L{\sqrt{5} + 1} \cdot \frac {L\sqrt{5}}{\sqrt{5} + 1}= \frac { L\sqrt{5} }{ \sqr{\sqrt{5} + 1} } \approx 12{,}8\,\text{см}.
    \end{align*}
}

\variantsplitter

\addpersonalvariant{Герман Говоров}

\tasknumber{1}%
\task{%
    В каком месте на главной оптической оси двояковыгнутой линзы
    нужно поместить точечный источник света,
    чтобы его изображение оказалось в главном фокусе линзы?
}
\answer{%
    $\text{на половине фокусного расстояния}$
}
\solutionspace{120pt}

\tasknumber{2}%
\task{%
    На экране, расположенном иа расстоянии $120\,\text{см}$ от собирающей линзы,
    получено изображение точечного источника, расположенного на главной оптической оси линзы.
    На какое расстояние переместится изображение на экране,
    если при неподвижной линзе переместить источник на $3\,\text{см}$ в плоскости, перпендикулярной главной оптической оси?
    Фокусное расстояние линзы равно $40\,\text{см}$.
}
\answer{%
    \begin{align*}
    &\frac 1F = \frac 1a + \frac 1b \implies a = \frac{bF}{b-F} \implies \Gamma = \frac ba = \frac{b-F}F \\
    &y = x \cdot \Gamma = x \cdot \frac{b-F}F \implies d = y = 6\,\text{см}.
    \end{align*}
}
\solutionspace{120pt}

\tasknumber{3}%
\task{%
    Оптическая сила двояковыпуклой линзы в воздухе $5{,}5\,\text{дптр}$, а в воде $1{,}5\,\text{дптр}$.
    Определить показатель преломления $n$ материала, из которого изготовлена линза.
}
\answer{%
    \begin{align*}
    D_1 &=\cbr{\frac n{n_1} - 1}\cbr{\frac 1{R_1} + \frac 1{R_2}}, \\
    D_2 &=\cbr{\frac n{n_2} - 1}\cbr{\frac 1{R_1} + \frac 1{R_2}}, \\
    \frac {D_2}{D_1} &=\frac{\frac n{n_2} - 1}{\frac n{n_1} - 1} \implies {D_2}\cbr{\frac n{n_1} - 1} = {D_1}\cbr{\frac n{n_2} - 1}  \implies n\cbr{\frac{D_2}{n_1} - \frac{D_1}{n_2}} = D_2 - D_1, \\
    n &= \frac{D_2 - D_1}{\frac{D_2}{n_1} - \frac{D_1}{n_2}} = \frac{n_1 n_2 (D_2 - D_1)}{D_2n_2 - D_1n_1} \approx 1{,}518.
    \end{align*}
}
\solutionspace{120pt}

\tasknumber{4}%
\task{%
    На каком расстоянии от собирающей линзы с фокусным расстоянием $40\,\text{дптр}$
    следует надо поместить предмет, чтобы расстояние
    от предмета до его действительного изображения было наименьшим?
}
\answer{%
    \begin{align*}
    \frac 1a &+ \frac 1b = D \implies b = \frac 1{D - \frac 1a} \implies \ell = a + b = a + \frac a{Da - 1} = \frac{ Da^2 }{Da - 1} \implies \\
    \implies \ell'_a &= \frac{ 2Da \cdot (Da - 1) - Da^2 \cdot D }{\sqr{Da - 1}}= \frac{ D^2a^2 - 2Da}{\sqr{Da - 1}} = \frac{ Da(Da - 2)}{\sqr{Da - 1}}\implies a_{\min} = \frac 2D \approx 50\,\text{мм}.
    \end{align*}
}
\solutionspace{120pt}

\tasknumber{5}%
\task{%
    Предмет в виде отрезка длиной $\ell$ расположен вдоль оптической оси
    собирающей линзы с фокусным расстоянием $F$.
    Середина отрезка расположена
    иа расстоянии $a$ от линзы, которая даёт действительное изображение
    всех точек предмета.
    Определить продольное увеличение предмета.
}
\answer{%
    \begin{align*}
    \frac 1{a + \frac \ell 2} &+ \frac 1b = \frac 1F \implies b = \frac{F\cbr{a + \frac \ell 2}}{a + \frac \ell 2 - F} \\
    \frac 1{a - \frac \ell 2} &+ \frac 1c = \frac 1F \implies c = \frac{F\cbr{a - \frac \ell 2}}{a - \frac \ell 2 - F} \\
    \abs{b - c} &= \abs{\frac{F\cbr{a + \frac \ell 2}}{a + \frac \ell 2 - F} - \frac{F\cbr{a - \frac \ell 2}}{a - \frac \ell 2 - F}}= F\abs{\frac{\cbr{a + \frac \ell 2}\cbr{a - \frac \ell 2 - F} - \cbr{a - \frac \ell 2}\cbr{a + \frac \ell 2 - F}}{ \cbr{a + \frac \ell 2 - F} \cbr{a - \frac \ell 2 - F} }} =  \\
    &= F\abs{\frac{a^2 - \frac {a\ell} 2 - Fa + \frac {a\ell} 2 - \frac {\ell^2} 4 - \frac {F\ell}2 - a^2 - \frac {a\ell}2 + aF + \frac {a\ell}2 + \frac {\ell^2} 4 - \frac {F\ell} 2}{\cbr{a + \frac \ell 2 - F} \cbr{a - \frac \ell 2 - F} }} = \\
    &= F\frac{F\ell}{\sqr{a-F} - \frac {\ell^2}4} = \frac{F^2\ell}{\sqr{a-F} - \frac {\ell^2}4}\implies \Gamma = \frac{\abs{b - c}}\ell = \frac{F^2}{\sqr{a-F} - \frac {\ell^2}4}.
    \end{align*}
}
\solutionspace{120pt}

\tasknumber{6}%
\task{%
    Даны точечный источник света $S$, его изображение $S_1$, полученное с помошью собирающей линзы,
    и ближайший к источнику фокус линзы $F$ (см.
    рис.
    на доске).
    Расстояния $SF = \ell$ и $SS_1 = L$.
    Определить положение линзы и её фокусное расстояние.
}
\answer{%
    \begin{align*}
    \frac 1a + \frac 1b &= \frac 1F, \ell = a - F, L = a + b \implies a = \ell + F, b = L - a = L - \ell - F \\
    \frac 1{\ell + F} + \frac 1{L - \ell - F} &= \frac 1F \\
    F\ell + F^2 + LF - F\ell - F^2 &= L\ell - \ell^2 - F\ell + LF - F\ell - F^2 \\
    0 &= L\ell - \ell^2 - 2F\ell - F^2 \\
    0 &=  F^2 + 2F\ell - L\ell + \ell^2 \\
    F &= -\ell \pm \sqrt{\ell^2 +  L\ell - \ell^2} = -\ell \pm \sqrt{L\ell} \implies F = \sqrt{L\ell} - \ell \\
    a &= \ell + F = \ell + \sqrt{L\ell} - \ell = \sqrt{L\ell}.
    \end{align*}
}
\solutionspace{120pt}

\tasknumber{7}%
\task{%
    Расстояние от освещённого предмета до экрана $80\,\text{см}$.
    Линза, помещенная между ними, даёт чёткое изображение предмета на
    экране при двух положениях, расстояние между которыми $40\,\text{см}$.
    Найти фокусное расстояние линзы.
}
\answer{%
    \begin{align*}
    \frac 1a + \frac 1b &= \frac 1F, \frac 1{a-\ell} + \frac 1{b+\ell} = \frac 1F, a + b = L \\
    \frac 1a + \frac 1b &= \frac 1{a-\ell} + \frac 1{b+\ell}\implies \frac{a + b}{ab} = \frac{(a-\ell) + (b+\ell)}{(a-\ell)(b+\ell)} \\
    ab  &= (a - \ell)(b+\ell) \implies 0  = -b\ell + a\ell - \ell^2 \implies 0 = -b + a - \ell \implies b = a - \ell \\
    a + (a - \ell) &= L \implies a = \frac{L + \ell}2 \implies b = \frac{L - \ell}2 \\
    F &= \frac{ab}{a + b} = \frac{L^2 -\ell^2}{4L} \approx 15\,\text{см}.
    \end{align*}
}
\solutionspace{120pt}

\tasknumber{8}%
\task{%
    Предмет находится на расстоянии $80\,\text{см}$ от экрана.
    Между предметом и экраном помещают линзу, причём при одном
    положении линзы на экране получается увеличенное изображение предмета,
    а при другом — уменьшенное.
    Каково фокусное расстояние линзы, если
    линейные размеры первого изображения в два раза больше второго?
}
\answer{%
    \begin{align*}
    \frac 1a + \frac 1{L-a} &= \frac 1F, h_1 = h \cdot \frac{L-a}a, \\
    \frac 1b + \frac 1{L-b} &= \frac 1F, h_2 = h \cdot \frac{L-b}b, \\
    \frac{h_1}{h_2} &= 2 \implies \frac{(L-a)b}{(L-b)a} = 2, \\
    \frac 1F &= \frac{ L }{a(L-a)} = \frac{ L }{b(L-b)} \implies \frac{L-a}{L-b} = \frac b a \implies \frac {b^2}{a^2} = 2.
    \\
    \frac 1a + \frac 1{L-a} &= \frac 1b + \frac 1{L-b} \implies \frac L{a(L-a)} = \frac L{b(L-b)} \implies \\
    \implies aL - a^2 &= bL - b^2 \implies (a-b)L = (a-b)(a+b) \implies b = L - a, \\
    \frac{\sqr{L-a}}{a^2} &= 2 \implies \frac La - 1 = \sqrt{2} \implies a = \frac{ L }{\sqrt{2} + 1} \\
    F &= \frac{a(L-a)}L = \frac 1L \cdot \frac L{\sqrt{2} + 1} \cdot \frac {L\sqrt{2}}{\sqrt{2} + 1}= \frac { L\sqrt{2} }{ \sqr{\sqrt{2} + 1} } \approx 19{,}4\,\text{см}.
    \end{align*}
}

\variantsplitter

\addpersonalvariant{София Журавлёва}

\tasknumber{1}%
\task{%
    В каком месте на главной оптической оси двояковыпуклой линзы
    нужно поместить точечный источник света,
    чтобы его изображение оказалось в главном фокусе линзы?
}
\answer{%
    $\text{для мнимого - на половине фокусного, для действительного - на бесконечности}$
}
\solutionspace{120pt}

\tasknumber{2}%
\task{%
    На экране, расположенном иа расстоянии $80\,\text{см}$ от собирающей линзы,
    получено изображение точечного источника, расположенного на главной оптической оси линзы.
    На какое расстояние переместится изображение на экране,
    если при неподвижном источнике переместить линзу на $2\,\text{см}$ в плоскости, перпендикулярной главной оптической оси?
    Фокусное расстояние линзы равно $20\,\text{см}$.
}
\answer{%
    \begin{align*}
    &\frac 1F = \frac 1a + \frac 1b \implies a = \frac{bF}{b-F} \implies \Gamma = \frac ba = \frac{b-F}F \\
    &y = x \cdot \Gamma = x \cdot \frac{b-F}F \implies d = x + y = 8\,\text{см}.
    \end{align*}
}
\solutionspace{120pt}

\tasknumber{3}%
\task{%
    Оптическая сила двояковыпуклой линзы в воздухе $5{,}5\,\text{дптр}$, а в воде $1{,}6\,\text{дптр}$.
    Определить показатель преломления $n$ материала, из которого изготовлена линза.
}
\answer{%
    \begin{align*}
    D_1 &=\cbr{\frac n{n_1} - 1}\cbr{\frac 1{R_1} + \frac 1{R_2}}, \\
    D_2 &=\cbr{\frac n{n_2} - 1}\cbr{\frac 1{R_1} + \frac 1{R_2}}, \\
    \frac {D_2}{D_1} &=\frac{\frac n{n_2} - 1}{\frac n{n_1} - 1} \implies {D_2}\cbr{\frac n{n_1} - 1} = {D_1}\cbr{\frac n{n_2} - 1}  \implies n\cbr{\frac{D_2}{n_1} - \frac{D_1}{n_2}} = D_2 - D_1, \\
    n &= \frac{D_2 - D_1}{\frac{D_2}{n_1} - \frac{D_1}{n_2}} = \frac{n_1 n_2 (D_2 - D_1)}{D_2n_2 - D_1n_1} \approx 1{,}538.
    \end{align*}
}
\solutionspace{120pt}

\tasknumber{4}%
\task{%
    На каком расстоянии от собирающей линзы с фокусным расстоянием $40\,\text{дптр}$
    следует надо поместить предмет, чтобы расстояние
    от предмета до его действительного изображения было наименьшим?
}
\answer{%
    \begin{align*}
    \frac 1a &+ \frac 1b = D \implies b = \frac 1{D - \frac 1a} \implies \ell = a + b = a + \frac a{Da - 1} = \frac{ Da^2 }{Da - 1} \implies \\
    \implies \ell'_a &= \frac{ 2Da \cdot (Da - 1) - Da^2 \cdot D }{\sqr{Da - 1}}= \frac{ D^2a^2 - 2Da}{\sqr{Da - 1}} = \frac{ Da(Da - 2)}{\sqr{Da - 1}}\implies a_{\min} = \frac 2D \approx 50\,\text{мм}.
    \end{align*}
}
\solutionspace{120pt}

\tasknumber{5}%
\task{%
    Предмет в виде отрезка длиной $\ell$ расположен вдоль оптической оси
    собирающей линзы с фокусным расстоянием $F$.
    Середина отрезка расположена
    иа расстоянии $a$ от линзы, которая даёт действительное изображение
    всех точек предмета.
    Определить продольное увеличение предмета.
}
\answer{%
    \begin{align*}
    \frac 1{a + \frac \ell 2} &+ \frac 1b = \frac 1F \implies b = \frac{F\cbr{a + \frac \ell 2}}{a + \frac \ell 2 - F} \\
    \frac 1{a - \frac \ell 2} &+ \frac 1c = \frac 1F \implies c = \frac{F\cbr{a - \frac \ell 2}}{a - \frac \ell 2 - F} \\
    \abs{b - c} &= \abs{\frac{F\cbr{a + \frac \ell 2}}{a + \frac \ell 2 - F} - \frac{F\cbr{a - \frac \ell 2}}{a - \frac \ell 2 - F}}= F\abs{\frac{\cbr{a + \frac \ell 2}\cbr{a - \frac \ell 2 - F} - \cbr{a - \frac \ell 2}\cbr{a + \frac \ell 2 - F}}{ \cbr{a + \frac \ell 2 - F} \cbr{a - \frac \ell 2 - F} }} =  \\
    &= F\abs{\frac{a^2 - \frac {a\ell} 2 - Fa + \frac {a\ell} 2 - \frac {\ell^2} 4 - \frac {F\ell}2 - a^2 - \frac {a\ell}2 + aF + \frac {a\ell}2 + \frac {\ell^2} 4 - \frac {F\ell} 2}{\cbr{a + \frac \ell 2 - F} \cbr{a - \frac \ell 2 - F} }} = \\
    &= F\frac{F\ell}{\sqr{a-F} - \frac {\ell^2}4} = \frac{F^2\ell}{\sqr{a-F} - \frac {\ell^2}4}\implies \Gamma = \frac{\abs{b - c}}\ell = \frac{F^2}{\sqr{a-F} - \frac {\ell^2}4}.
    \end{align*}
}
\solutionspace{120pt}

\tasknumber{6}%
\task{%
    Даны точечный источник света $S$, его изображение $S_1$, полученное с помошью собирающей линзы,
    и ближайший к источнику фокус линзы $F$ (см.
    рис.
    на доске).
    Расстояния $SF = \ell$ и $SS_1 = L$.
    Определить положение линзы и её фокусное расстояние.
}
\answer{%
    \begin{align*}
    \frac 1a + \frac 1b &= \frac 1F, \ell = a - F, L = a + b \implies a = \ell + F, b = L - a = L - \ell - F \\
    \frac 1{\ell + F} + \frac 1{L - \ell - F} &= \frac 1F \\
    F\ell + F^2 + LF - F\ell - F^2 &= L\ell - \ell^2 - F\ell + LF - F\ell - F^2 \\
    0 &= L\ell - \ell^2 - 2F\ell - F^2 \\
    0 &=  F^2 + 2F\ell - L\ell + \ell^2 \\
    F &= -\ell \pm \sqrt{\ell^2 +  L\ell - \ell^2} = -\ell \pm \sqrt{L\ell} \implies F = \sqrt{L\ell} - \ell \\
    a &= \ell + F = \ell + \sqrt{L\ell} - \ell = \sqrt{L\ell}.
    \end{align*}
}
\solutionspace{120pt}

\tasknumber{7}%
\task{%
    Расстояние от освещённого предмета до экрана $100\,\text{см}$.
    Линза, помещенная между ними, даёт чёткое изображение предмета на
    экране при двух положениях, расстояние между которыми $40\,\text{см}$.
    Найти фокусное расстояние линзы.
}
\answer{%
    \begin{align*}
    \frac 1a + \frac 1b &= \frac 1F, \frac 1{a-\ell} + \frac 1{b+\ell} = \frac 1F, a + b = L \\
    \frac 1a + \frac 1b &= \frac 1{a-\ell} + \frac 1{b+\ell}\implies \frac{a + b}{ab} = \frac{(a-\ell) + (b+\ell)}{(a-\ell)(b+\ell)} \\
    ab  &= (a - \ell)(b+\ell) \implies 0  = -b\ell + a\ell - \ell^2 \implies 0 = -b + a - \ell \implies b = a - \ell \\
    a + (a - \ell) &= L \implies a = \frac{L + \ell}2 \implies b = \frac{L - \ell}2 \\
    F &= \frac{ab}{a + b} = \frac{L^2 -\ell^2}{4L} \approx 21\,\text{см}.
    \end{align*}
}
\solutionspace{120pt}

\tasknumber{8}%
\task{%
    Предмет находится на расстоянии $60\,\text{см}$ от экрана.
    Между предметом и экраном помещают линзу, причём при одном
    положении линзы на экране получается увеличенное изображение предмета,
    а при другом — уменьшенное.
    Каково фокусное расстояние линзы, если
    линейные размеры первого изображения в два раза больше второго?
}
\answer{%
    \begin{align*}
    \frac 1a + \frac 1{L-a} &= \frac 1F, h_1 = h \cdot \frac{L-a}a, \\
    \frac 1b + \frac 1{L-b} &= \frac 1F, h_2 = h \cdot \frac{L-b}b, \\
    \frac{h_1}{h_2} &= 2 \implies \frac{(L-a)b}{(L-b)a} = 2, \\
    \frac 1F &= \frac{ L }{a(L-a)} = \frac{ L }{b(L-b)} \implies \frac{L-a}{L-b} = \frac b a \implies \frac {b^2}{a^2} = 2.
    \\
    \frac 1a + \frac 1{L-a} &= \frac 1b + \frac 1{L-b} \implies \frac L{a(L-a)} = \frac L{b(L-b)} \implies \\
    \implies aL - a^2 &= bL - b^2 \implies (a-b)L = (a-b)(a+b) \implies b = L - a, \\
    \frac{\sqr{L-a}}{a^2} &= 2 \implies \frac La - 1 = \sqrt{2} \implies a = \frac{ L }{\sqrt{2} + 1} \\
    F &= \frac{a(L-a)}L = \frac 1L \cdot \frac L{\sqrt{2} + 1} \cdot \frac {L\sqrt{2}}{\sqrt{2} + 1}= \frac { L\sqrt{2} }{ \sqr{\sqrt{2} + 1} } \approx 14{,}6\,\text{см}.
    \end{align*}
}

\variantsplitter

\addpersonalvariant{Константин Козлов}

\tasknumber{1}%
\task{%
    В каком месте на главной оптической оси двояковыпуклой линзы
    нужно поместить точечный источник света,
    чтобы его изображение оказалось в главном фокусе линзы?
}
\answer{%
    $\text{для мнимого - на половине фокусного, для действительного - на бесконечности}$
}
\solutionspace{120pt}

\tasknumber{2}%
\task{%
    На экране, расположенном иа расстоянии $80\,\text{см}$ от собирающей линзы,
    получено изображение точечного источника, расположенного на главной оптической оси линзы.
    На какое расстояние переместится изображение на экране,
    если при неподвижном источнике переместить линзу на $2\,\text{см}$ в плоскости, перпендикулярной главной оптической оси?
    Фокусное расстояние линзы равно $40\,\text{см}$.
}
\answer{%
    \begin{align*}
    &\frac 1F = \frac 1a + \frac 1b \implies a = \frac{bF}{b-F} \implies \Gamma = \frac ba = \frac{b-F}F \\
    &y = x \cdot \Gamma = x \cdot \frac{b-F}F \implies d = x + y = 4\,\text{см}.
    \end{align*}
}
\solutionspace{120pt}

\tasknumber{3}%
\task{%
    Оптическая сила двояковыпуклой линзы в воздухе $5\,\text{дптр}$, а в воде $1{,}4\,\text{дптр}$.
    Определить показатель преломления $n$ материала, из которого изготовлена линза.
}
\answer{%
    \begin{align*}
    D_1 &=\cbr{\frac n{n_1} - 1}\cbr{\frac 1{R_1} + \frac 1{R_2}}, \\
    D_2 &=\cbr{\frac n{n_2} - 1}\cbr{\frac 1{R_1} + \frac 1{R_2}}, \\
    \frac {D_2}{D_1} &=\frac{\frac n{n_2} - 1}{\frac n{n_1} - 1} \implies {D_2}\cbr{\frac n{n_1} - 1} = {D_1}\cbr{\frac n{n_2} - 1}  \implies n\cbr{\frac{D_2}{n_1} - \frac{D_1}{n_2}} = D_2 - D_1, \\
    n &= \frac{D_2 - D_1}{\frac{D_2}{n_1} - \frac{D_1}{n_2}} = \frac{n_1 n_2 (D_2 - D_1)}{D_2n_2 - D_1n_1} \approx 1{,}526.
    \end{align*}
}
\solutionspace{120pt}

\tasknumber{4}%
\task{%
    На каком расстоянии от собирающей линзы с фокусным расстоянием $50\,\text{дптр}$
    следует надо поместить предмет, чтобы расстояние
    от предмета до его действительного изображения было наименьшим?
}
\answer{%
    \begin{align*}
    \frac 1a &+ \frac 1b = D \implies b = \frac 1{D - \frac 1a} \implies \ell = a + b = a + \frac a{Da - 1} = \frac{ Da^2 }{Da - 1} \implies \\
    \implies \ell'_a &= \frac{ 2Da \cdot (Da - 1) - Da^2 \cdot D }{\sqr{Da - 1}}= \frac{ D^2a^2 - 2Da}{\sqr{Da - 1}} = \frac{ Da(Da - 2)}{\sqr{Da - 1}}\implies a_{\min} = \frac 2D \approx 40\,\text{мм}.
    \end{align*}
}
\solutionspace{120pt}

\tasknumber{5}%
\task{%
    Предмет в виде отрезка длиной $\ell$ расположен вдоль оптической оси
    собирающей линзы с фокусным расстоянием $F$.
    Середина отрезка расположена
    иа расстоянии $a$ от линзы, которая даёт действительное изображение
    всех точек предмета.
    Определить продольное увеличение предмета.
}
\answer{%
    \begin{align*}
    \frac 1{a + \frac \ell 2} &+ \frac 1b = \frac 1F \implies b = \frac{F\cbr{a + \frac \ell 2}}{a + \frac \ell 2 - F} \\
    \frac 1{a - \frac \ell 2} &+ \frac 1c = \frac 1F \implies c = \frac{F\cbr{a - \frac \ell 2}}{a - \frac \ell 2 - F} \\
    \abs{b - c} &= \abs{\frac{F\cbr{a + \frac \ell 2}}{a + \frac \ell 2 - F} - \frac{F\cbr{a - \frac \ell 2}}{a - \frac \ell 2 - F}}= F\abs{\frac{\cbr{a + \frac \ell 2}\cbr{a - \frac \ell 2 - F} - \cbr{a - \frac \ell 2}\cbr{a + \frac \ell 2 - F}}{ \cbr{a + \frac \ell 2 - F} \cbr{a - \frac \ell 2 - F} }} =  \\
    &= F\abs{\frac{a^2 - \frac {a\ell} 2 - Fa + \frac {a\ell} 2 - \frac {\ell^2} 4 - \frac {F\ell}2 - a^2 - \frac {a\ell}2 + aF + \frac {a\ell}2 + \frac {\ell^2} 4 - \frac {F\ell} 2}{\cbr{a + \frac \ell 2 - F} \cbr{a - \frac \ell 2 - F} }} = \\
    &= F\frac{F\ell}{\sqr{a-F} - \frac {\ell^2}4} = \frac{F^2\ell}{\sqr{a-F} - \frac {\ell^2}4}\implies \Gamma = \frac{\abs{b - c}}\ell = \frac{F^2}{\sqr{a-F} - \frac {\ell^2}4}.
    \end{align*}
}
\solutionspace{120pt}

\tasknumber{6}%
\task{%
    Даны точечный источник света $S$, его изображение $S_1$, полученное с помошью собирающей линзы,
    и ближайший к источнику фокус линзы $F$ (см.
    рис.
    на доске).
    Расстояния $SF = \ell$ и $SS_1 = L$.
    Определить положение линзы и её фокусное расстояние.
}
\answer{%
    \begin{align*}
    \frac 1a + \frac 1b &= \frac 1F, \ell = a - F, L = a + b \implies a = \ell + F, b = L - a = L - \ell - F \\
    \frac 1{\ell + F} + \frac 1{L - \ell - F} &= \frac 1F \\
    F\ell + F^2 + LF - F\ell - F^2 &= L\ell - \ell^2 - F\ell + LF - F\ell - F^2 \\
    0 &= L\ell - \ell^2 - 2F\ell - F^2 \\
    0 &=  F^2 + 2F\ell - L\ell + \ell^2 \\
    F &= -\ell \pm \sqrt{\ell^2 +  L\ell - \ell^2} = -\ell \pm \sqrt{L\ell} \implies F = \sqrt{L\ell} - \ell \\
    a &= \ell + F = \ell + \sqrt{L\ell} - \ell = \sqrt{L\ell}.
    \end{align*}
}
\solutionspace{120pt}

\tasknumber{7}%
\task{%
    Расстояние от освещённого предмета до экрана $80\,\text{см}$.
    Линза, помещенная между ними, даёт чёткое изображение предмета на
    экране при двух положениях, расстояние между которыми $40\,\text{см}$.
    Найти фокусное расстояние линзы.
}
\answer{%
    \begin{align*}
    \frac 1a + \frac 1b &= \frac 1F, \frac 1{a-\ell} + \frac 1{b+\ell} = \frac 1F, a + b = L \\
    \frac 1a + \frac 1b &= \frac 1{a-\ell} + \frac 1{b+\ell}\implies \frac{a + b}{ab} = \frac{(a-\ell) + (b+\ell)}{(a-\ell)(b+\ell)} \\
    ab  &= (a - \ell)(b+\ell) \implies 0  = -b\ell + a\ell - \ell^2 \implies 0 = -b + a - \ell \implies b = a - \ell \\
    a + (a - \ell) &= L \implies a = \frac{L + \ell}2 \implies b = \frac{L - \ell}2 \\
    F &= \frac{ab}{a + b} = \frac{L^2 -\ell^2}{4L} \approx 15\,\text{см}.
    \end{align*}
}
\solutionspace{120pt}

\tasknumber{8}%
\task{%
    Предмет находится на расстоянии $90\,\text{см}$ от экрана.
    Между предметом и экраном помещают линзу, причём при одном
    положении линзы на экране получается увеличенное изображение предмета,
    а при другом — уменьшенное.
    Каково фокусное расстояние линзы, если
    линейные размеры первого изображения в три раза больше второго?
}
\answer{%
    \begin{align*}
    \frac 1a + \frac 1{L-a} &= \frac 1F, h_1 = h \cdot \frac{L-a}a, \\
    \frac 1b + \frac 1{L-b} &= \frac 1F, h_2 = h \cdot \frac{L-b}b, \\
    \frac{h_1}{h_2} &= 3 \implies \frac{(L-a)b}{(L-b)a} = 3, \\
    \frac 1F &= \frac{ L }{a(L-a)} = \frac{ L }{b(L-b)} \implies \frac{L-a}{L-b} = \frac b a \implies \frac {b^2}{a^2} = 3.
    \\
    \frac 1a + \frac 1{L-a} &= \frac 1b + \frac 1{L-b} \implies \frac L{a(L-a)} = \frac L{b(L-b)} \implies \\
    \implies aL - a^2 &= bL - b^2 \implies (a-b)L = (a-b)(a+b) \implies b = L - a, \\
    \frac{\sqr{L-a}}{a^2} &= 3 \implies \frac La - 1 = \sqrt{3} \implies a = \frac{ L }{\sqrt{3} + 1} \\
    F &= \frac{a(L-a)}L = \frac 1L \cdot \frac L{\sqrt{3} + 1} \cdot \frac {L\sqrt{3}}{\sqrt{3} + 1}= \frac { L\sqrt{3} }{ \sqr{\sqrt{3} + 1} } \approx 21\,\text{см}.
    \end{align*}
}

\variantsplitter

\addpersonalvariant{Наталья Кравченко}

\tasknumber{1}%
\task{%
    В каком месте на главной оптической оси двояковыпуклой линзы
    нужно поместить точечный источник света,
    чтобы его изображение оказалось в главном фокусе линзы?
}
\answer{%
    $\text{для мнимого - на половине фокусного, для действительного - на бесконечности}$
}
\solutionspace{120pt}

\tasknumber{2}%
\task{%
    На экране, расположенном иа расстоянии $120\,\text{см}$ от собирающей линзы,
    получено изображение точечного источника, расположенного на главной оптической оси линзы.
    На какое расстояние переместится изображение на экране,
    если при неподвижном источнике переместить линзу на $2\,\text{см}$ в плоскости, перпендикулярной главной оптической оси?
    Фокусное расстояние линзы равно $40\,\text{см}$.
}
\answer{%
    \begin{align*}
    &\frac 1F = \frac 1a + \frac 1b \implies a = \frac{bF}{b-F} \implies \Gamma = \frac ba = \frac{b-F}F \\
    &y = x \cdot \Gamma = x \cdot \frac{b-F}F \implies d = x + y = 6\,\text{см}.
    \end{align*}
}
\solutionspace{120pt}

\tasknumber{3}%
\task{%
    Оптическая сила двояковыпуклой линзы в воздухе $5\,\text{дптр}$, а в воде $1{,}4\,\text{дптр}$.
    Определить показатель преломления $n$ материала, из которого изготовлена линза.
}
\answer{%
    \begin{align*}
    D_1 &=\cbr{\frac n{n_1} - 1}\cbr{\frac 1{R_1} + \frac 1{R_2}}, \\
    D_2 &=\cbr{\frac n{n_2} - 1}\cbr{\frac 1{R_1} + \frac 1{R_2}}, \\
    \frac {D_2}{D_1} &=\frac{\frac n{n_2} - 1}{\frac n{n_1} - 1} \implies {D_2}\cbr{\frac n{n_1} - 1} = {D_1}\cbr{\frac n{n_2} - 1}  \implies n\cbr{\frac{D_2}{n_1} - \frac{D_1}{n_2}} = D_2 - D_1, \\
    n &= \frac{D_2 - D_1}{\frac{D_2}{n_1} - \frac{D_1}{n_2}} = \frac{n_1 n_2 (D_2 - D_1)}{D_2n_2 - D_1n_1} \approx 1{,}526.
    \end{align*}
}
\solutionspace{120pt}

\tasknumber{4}%
\task{%
    На каком расстоянии от собирающей линзы с фокусным расстоянием $40\,\text{дптр}$
    следует надо поместить предмет, чтобы расстояние
    от предмета до его действительного изображения было наименьшим?
}
\answer{%
    \begin{align*}
    \frac 1a &+ \frac 1b = D \implies b = \frac 1{D - \frac 1a} \implies \ell = a + b = a + \frac a{Da - 1} = \frac{ Da^2 }{Da - 1} \implies \\
    \implies \ell'_a &= \frac{ 2Da \cdot (Da - 1) - Da^2 \cdot D }{\sqr{Da - 1}}= \frac{ D^2a^2 - 2Da}{\sqr{Da - 1}} = \frac{ Da(Da - 2)}{\sqr{Da - 1}}\implies a_{\min} = \frac 2D \approx 50\,\text{мм}.
    \end{align*}
}
\solutionspace{120pt}

\tasknumber{5}%
\task{%
    Предмет в виде отрезка длиной $\ell$ расположен вдоль оптической оси
    собирающей линзы с фокусным расстоянием $F$.
    Середина отрезка расположена
    иа расстоянии $a$ от линзы, которая даёт действительное изображение
    всех точек предмета.
    Определить продольное увеличение предмета.
}
\answer{%
    \begin{align*}
    \frac 1{a + \frac \ell 2} &+ \frac 1b = \frac 1F \implies b = \frac{F\cbr{a + \frac \ell 2}}{a + \frac \ell 2 - F} \\
    \frac 1{a - \frac \ell 2} &+ \frac 1c = \frac 1F \implies c = \frac{F\cbr{a - \frac \ell 2}}{a - \frac \ell 2 - F} \\
    \abs{b - c} &= \abs{\frac{F\cbr{a + \frac \ell 2}}{a + \frac \ell 2 - F} - \frac{F\cbr{a - \frac \ell 2}}{a - \frac \ell 2 - F}}= F\abs{\frac{\cbr{a + \frac \ell 2}\cbr{a - \frac \ell 2 - F} - \cbr{a - \frac \ell 2}\cbr{a + \frac \ell 2 - F}}{ \cbr{a + \frac \ell 2 - F} \cbr{a - \frac \ell 2 - F} }} =  \\
    &= F\abs{\frac{a^2 - \frac {a\ell} 2 - Fa + \frac {a\ell} 2 - \frac {\ell^2} 4 - \frac {F\ell}2 - a^2 - \frac {a\ell}2 + aF + \frac {a\ell}2 + \frac {\ell^2} 4 - \frac {F\ell} 2}{\cbr{a + \frac \ell 2 - F} \cbr{a - \frac \ell 2 - F} }} = \\
    &= F\frac{F\ell}{\sqr{a-F} - \frac {\ell^2}4} = \frac{F^2\ell}{\sqr{a-F} - \frac {\ell^2}4}\implies \Gamma = \frac{\abs{b - c}}\ell = \frac{F^2}{\sqr{a-F} - \frac {\ell^2}4}.
    \end{align*}
}
\solutionspace{120pt}

\tasknumber{6}%
\task{%
    Даны точечный источник света $S$, его изображение $S_1$, полученное с помошью собирающей линзы,
    и ближайший к источнику фокус линзы $F$ (см.
    рис.
    на доске).
    Расстояния $SF = \ell$ и $SS_1 = L$.
    Определить положение линзы и её фокусное расстояние.
}
\answer{%
    \begin{align*}
    \frac 1a + \frac 1b &= \frac 1F, \ell = a - F, L = a + b \implies a = \ell + F, b = L - a = L - \ell - F \\
    \frac 1{\ell + F} + \frac 1{L - \ell - F} &= \frac 1F \\
    F\ell + F^2 + LF - F\ell - F^2 &= L\ell - \ell^2 - F\ell + LF - F\ell - F^2 \\
    0 &= L\ell - \ell^2 - 2F\ell - F^2 \\
    0 &=  F^2 + 2F\ell - L\ell + \ell^2 \\
    F &= -\ell \pm \sqrt{\ell^2 +  L\ell - \ell^2} = -\ell \pm \sqrt{L\ell} \implies F = \sqrt{L\ell} - \ell \\
    a &= \ell + F = \ell + \sqrt{L\ell} - \ell = \sqrt{L\ell}.
    \end{align*}
}
\solutionspace{120pt}

\tasknumber{7}%
\task{%
    Расстояние от освещённого предмета до экрана $80\,\text{см}$.
    Линза, помещенная между ними, даёт чёткое изображение предмета на
    экране при двух положениях, расстояние между которыми $20\,\text{см}$.
    Найти фокусное расстояние линзы.
}
\answer{%
    \begin{align*}
    \frac 1a + \frac 1b &= \frac 1F, \frac 1{a-\ell} + \frac 1{b+\ell} = \frac 1F, a + b = L \\
    \frac 1a + \frac 1b &= \frac 1{a-\ell} + \frac 1{b+\ell}\implies \frac{a + b}{ab} = \frac{(a-\ell) + (b+\ell)}{(a-\ell)(b+\ell)} \\
    ab  &= (a - \ell)(b+\ell) \implies 0  = -b\ell + a\ell - \ell^2 \implies 0 = -b + a - \ell \implies b = a - \ell \\
    a + (a - \ell) &= L \implies a = \frac{L + \ell}2 \implies b = \frac{L - \ell}2 \\
    F &= \frac{ab}{a + b} = \frac{L^2 -\ell^2}{4L} \approx 18{,}8\,\text{см}.
    \end{align*}
}
\solutionspace{120pt}

\tasknumber{8}%
\task{%
    Предмет находится на расстоянии $70\,\text{см}$ от экрана.
    Между предметом и экраном помещают линзу, причём при одном
    положении линзы на экране получается увеличенное изображение предмета,
    а при другом — уменьшенное.
    Каково фокусное расстояние линзы, если
    линейные размеры первого изображения в два раза больше второго?
}
\answer{%
    \begin{align*}
    \frac 1a + \frac 1{L-a} &= \frac 1F, h_1 = h \cdot \frac{L-a}a, \\
    \frac 1b + \frac 1{L-b} &= \frac 1F, h_2 = h \cdot \frac{L-b}b, \\
    \frac{h_1}{h_2} &= 2 \implies \frac{(L-a)b}{(L-b)a} = 2, \\
    \frac 1F &= \frac{ L }{a(L-a)} = \frac{ L }{b(L-b)} \implies \frac{L-a}{L-b} = \frac b a \implies \frac {b^2}{a^2} = 2.
    \\
    \frac 1a + \frac 1{L-a} &= \frac 1b + \frac 1{L-b} \implies \frac L{a(L-a)} = \frac L{b(L-b)} \implies \\
    \implies aL - a^2 &= bL - b^2 \implies (a-b)L = (a-b)(a+b) \implies b = L - a, \\
    \frac{\sqr{L-a}}{a^2} &= 2 \implies \frac La - 1 = \sqrt{2} \implies a = \frac{ L }{\sqrt{2} + 1} \\
    F &= \frac{a(L-a)}L = \frac 1L \cdot \frac L{\sqrt{2} + 1} \cdot \frac {L\sqrt{2}}{\sqrt{2} + 1}= \frac { L\sqrt{2} }{ \sqr{\sqrt{2} + 1} } \approx 17{,}0\,\text{см}.
    \end{align*}
}

\variantsplitter

\addpersonalvariant{Матвей Кузьмин}

\tasknumber{1}%
\task{%
    В каком месте на главной оптической оси двояковыгнутой линзы
    нужно поместить точечный источник света,
    чтобы его изображение оказалось в главном фокусе линзы?
}
\answer{%
    $\text{на половине фокусного расстояния}$
}
\solutionspace{120pt}

\tasknumber{2}%
\task{%
    На экране, расположенном иа расстоянии $60\,\text{см}$ от собирающей линзы,
    получено изображение точечного источника, расположенного на главной оптической оси линзы.
    На какое расстояние переместится изображение на экране,
    если при неподвижном источнике переместить линзу на $3\,\text{см}$ в плоскости, перпендикулярной главной оптической оси?
    Фокусное расстояние линзы равно $40\,\text{см}$.
}
\answer{%
    \begin{align*}
    &\frac 1F = \frac 1a + \frac 1b \implies a = \frac{bF}{b-F} \implies \Gamma = \frac ba = \frac{b-F}F \\
    &y = x \cdot \Gamma = x \cdot \frac{b-F}F \implies d = x + y = 4{,}5\,\text{см}.
    \end{align*}
}
\solutionspace{120pt}

\tasknumber{3}%
\task{%
    Оптическая сила двояковыпуклой линзы в воздухе $5\,\text{дптр}$, а в воде $1{,}4\,\text{дптр}$.
    Определить показатель преломления $n$ материала, из которого изготовлена линза.
}
\answer{%
    \begin{align*}
    D_1 &=\cbr{\frac n{n_1} - 1}\cbr{\frac 1{R_1} + \frac 1{R_2}}, \\
    D_2 &=\cbr{\frac n{n_2} - 1}\cbr{\frac 1{R_1} + \frac 1{R_2}}, \\
    \frac {D_2}{D_1} &=\frac{\frac n{n_2} - 1}{\frac n{n_1} - 1} \implies {D_2}\cbr{\frac n{n_1} - 1} = {D_1}\cbr{\frac n{n_2} - 1}  \implies n\cbr{\frac{D_2}{n_1} - \frac{D_1}{n_2}} = D_2 - D_1, \\
    n &= \frac{D_2 - D_1}{\frac{D_2}{n_1} - \frac{D_1}{n_2}} = \frac{n_1 n_2 (D_2 - D_1)}{D_2n_2 - D_1n_1} \approx 1{,}526.
    \end{align*}
}
\solutionspace{120pt}

\tasknumber{4}%
\task{%
    На каком расстоянии от собирающей линзы с фокусным расстоянием $50\,\text{дптр}$
    следует надо поместить предмет, чтобы расстояние
    от предмета до его действительного изображения было наименьшим?
}
\answer{%
    \begin{align*}
    \frac 1a &+ \frac 1b = D \implies b = \frac 1{D - \frac 1a} \implies \ell = a + b = a + \frac a{Da - 1} = \frac{ Da^2 }{Da - 1} \implies \\
    \implies \ell'_a &= \frac{ 2Da \cdot (Da - 1) - Da^2 \cdot D }{\sqr{Da - 1}}= \frac{ D^2a^2 - 2Da}{\sqr{Da - 1}} = \frac{ Da(Da - 2)}{\sqr{Da - 1}}\implies a_{\min} = \frac 2D \approx 40\,\text{мм}.
    \end{align*}
}
\solutionspace{120pt}

\tasknumber{5}%
\task{%
    Предмет в виде отрезка длиной $\ell$ расположен вдоль оптической оси
    собирающей линзы с фокусным расстоянием $F$.
    Середина отрезка расположена
    иа расстоянии $a$ от линзы, которая даёт действительное изображение
    всех точек предмета.
    Определить продольное увеличение предмета.
}
\answer{%
    \begin{align*}
    \frac 1{a + \frac \ell 2} &+ \frac 1b = \frac 1F \implies b = \frac{F\cbr{a + \frac \ell 2}}{a + \frac \ell 2 - F} \\
    \frac 1{a - \frac \ell 2} &+ \frac 1c = \frac 1F \implies c = \frac{F\cbr{a - \frac \ell 2}}{a - \frac \ell 2 - F} \\
    \abs{b - c} &= \abs{\frac{F\cbr{a + \frac \ell 2}}{a + \frac \ell 2 - F} - \frac{F\cbr{a - \frac \ell 2}}{a - \frac \ell 2 - F}}= F\abs{\frac{\cbr{a + \frac \ell 2}\cbr{a - \frac \ell 2 - F} - \cbr{a - \frac \ell 2}\cbr{a + \frac \ell 2 - F}}{ \cbr{a + \frac \ell 2 - F} \cbr{a - \frac \ell 2 - F} }} =  \\
    &= F\abs{\frac{a^2 - \frac {a\ell} 2 - Fa + \frac {a\ell} 2 - \frac {\ell^2} 4 - \frac {F\ell}2 - a^2 - \frac {a\ell}2 + aF + \frac {a\ell}2 + \frac {\ell^2} 4 - \frac {F\ell} 2}{\cbr{a + \frac \ell 2 - F} \cbr{a - \frac \ell 2 - F} }} = \\
    &= F\frac{F\ell}{\sqr{a-F} - \frac {\ell^2}4} = \frac{F^2\ell}{\sqr{a-F} - \frac {\ell^2}4}\implies \Gamma = \frac{\abs{b - c}}\ell = \frac{F^2}{\sqr{a-F} - \frac {\ell^2}4}.
    \end{align*}
}
\solutionspace{120pt}

\tasknumber{6}%
\task{%
    Даны точечный источник света $S$, его изображение $S_1$, полученное с помошью собирающей линзы,
    и ближайший к источнику фокус линзы $F$ (см.
    рис.
    на доске).
    Расстояния $SF = \ell$ и $SS_1 = L$.
    Определить положение линзы и её фокусное расстояние.
}
\answer{%
    \begin{align*}
    \frac 1a + \frac 1b &= \frac 1F, \ell = a - F, L = a + b \implies a = \ell + F, b = L - a = L - \ell - F \\
    \frac 1{\ell + F} + \frac 1{L - \ell - F} &= \frac 1F \\
    F\ell + F^2 + LF - F\ell - F^2 &= L\ell - \ell^2 - F\ell + LF - F\ell - F^2 \\
    0 &= L\ell - \ell^2 - 2F\ell - F^2 \\
    0 &=  F^2 + 2F\ell - L\ell + \ell^2 \\
    F &= -\ell \pm \sqrt{\ell^2 +  L\ell - \ell^2} = -\ell \pm \sqrt{L\ell} \implies F = \sqrt{L\ell} - \ell \\
    a &= \ell + F = \ell + \sqrt{L\ell} - \ell = \sqrt{L\ell}.
    \end{align*}
}
\solutionspace{120pt}

\tasknumber{7}%
\task{%
    Расстояние от освещённого предмета до экрана $100\,\text{см}$.
    Линза, помещенная между ними, даёт чёткое изображение предмета на
    экране при двух положениях, расстояние между которыми $40\,\text{см}$.
    Найти фокусное расстояние линзы.
}
\answer{%
    \begin{align*}
    \frac 1a + \frac 1b &= \frac 1F, \frac 1{a-\ell} + \frac 1{b+\ell} = \frac 1F, a + b = L \\
    \frac 1a + \frac 1b &= \frac 1{a-\ell} + \frac 1{b+\ell}\implies \frac{a + b}{ab} = \frac{(a-\ell) + (b+\ell)}{(a-\ell)(b+\ell)} \\
    ab  &= (a - \ell)(b+\ell) \implies 0  = -b\ell + a\ell - \ell^2 \implies 0 = -b + a - \ell \implies b = a - \ell \\
    a + (a - \ell) &= L \implies a = \frac{L + \ell}2 \implies b = \frac{L - \ell}2 \\
    F &= \frac{ab}{a + b} = \frac{L^2 -\ell^2}{4L} \approx 21\,\text{см}.
    \end{align*}
}
\solutionspace{120pt}

\tasknumber{8}%
\task{%
    Предмет находится на расстоянии $80\,\text{см}$ от экрана.
    Между предметом и экраном помещают линзу, причём при одном
    положении линзы на экране получается увеличенное изображение предмета,
    а при другом — уменьшенное.
    Каково фокусное расстояние линзы, если
    линейные размеры первого изображения в два раза больше второго?
}
\answer{%
    \begin{align*}
    \frac 1a + \frac 1{L-a} &= \frac 1F, h_1 = h \cdot \frac{L-a}a, \\
    \frac 1b + \frac 1{L-b} &= \frac 1F, h_2 = h \cdot \frac{L-b}b, \\
    \frac{h_1}{h_2} &= 2 \implies \frac{(L-a)b}{(L-b)a} = 2, \\
    \frac 1F &= \frac{ L }{a(L-a)} = \frac{ L }{b(L-b)} \implies \frac{L-a}{L-b} = \frac b a \implies \frac {b^2}{a^2} = 2.
    \\
    \frac 1a + \frac 1{L-a} &= \frac 1b + \frac 1{L-b} \implies \frac L{a(L-a)} = \frac L{b(L-b)} \implies \\
    \implies aL - a^2 &= bL - b^2 \implies (a-b)L = (a-b)(a+b) \implies b = L - a, \\
    \frac{\sqr{L-a}}{a^2} &= 2 \implies \frac La - 1 = \sqrt{2} \implies a = \frac{ L }{\sqrt{2} + 1} \\
    F &= \frac{a(L-a)}L = \frac 1L \cdot \frac L{\sqrt{2} + 1} \cdot \frac {L\sqrt{2}}{\sqrt{2} + 1}= \frac { L\sqrt{2} }{ \sqr{\sqrt{2} + 1} } \approx 19{,}4\,\text{см}.
    \end{align*}
}

\variantsplitter

\addpersonalvariant{Сергей Малышев}

\tasknumber{1}%
\task{%
    В каком месте на главной оптической оси двояковыпуклой линзы
    нужно поместить точечный источник света,
    чтобы его изображение оказалось в главном фокусе линзы?
}
\answer{%
    $\text{для мнимого - на половине фокусного, для действительного - на бесконечности}$
}
\solutionspace{120pt}

\tasknumber{2}%
\task{%
    На экране, расположенном иа расстоянии $80\,\text{см}$ от собирающей линзы,
    получено изображение точечного источника, расположенного на главной оптической оси линзы.
    На какое расстояние переместится изображение на экране,
    если при неподвижной линзе переместить источник на $2\,\text{см}$ в плоскости, перпендикулярной главной оптической оси?
    Фокусное расстояние линзы равно $30\,\text{см}$.
}
\answer{%
    \begin{align*}
    &\frac 1F = \frac 1a + \frac 1b \implies a = \frac{bF}{b-F} \implies \Gamma = \frac ba = \frac{b-F}F \\
    &y = x \cdot \Gamma = x \cdot \frac{b-F}F \implies d = y = 3{,}3\,\text{см}.
    \end{align*}
}
\solutionspace{120pt}

\tasknumber{3}%
\task{%
    Оптическая сила двояковыпуклой линзы в воздухе $5\,\text{дптр}$, а в воде $1{,}4\,\text{дптр}$.
    Определить показатель преломления $n$ материала, из которого изготовлена линза.
}
\answer{%
    \begin{align*}
    D_1 &=\cbr{\frac n{n_1} - 1}\cbr{\frac 1{R_1} + \frac 1{R_2}}, \\
    D_2 &=\cbr{\frac n{n_2} - 1}\cbr{\frac 1{R_1} + \frac 1{R_2}}, \\
    \frac {D_2}{D_1} &=\frac{\frac n{n_2} - 1}{\frac n{n_1} - 1} \implies {D_2}\cbr{\frac n{n_1} - 1} = {D_1}\cbr{\frac n{n_2} - 1}  \implies n\cbr{\frac{D_2}{n_1} - \frac{D_1}{n_2}} = D_2 - D_1, \\
    n &= \frac{D_2 - D_1}{\frac{D_2}{n_1} - \frac{D_1}{n_2}} = \frac{n_1 n_2 (D_2 - D_1)}{D_2n_2 - D_1n_1} \approx 1{,}526.
    \end{align*}
}
\solutionspace{120pt}

\tasknumber{4}%
\task{%
    На каком расстоянии от собирающей линзы с фокусным расстоянием $30\,\text{дптр}$
    следует надо поместить предмет, чтобы расстояние
    от предмета до его действительного изображения было наименьшим?
}
\answer{%
    \begin{align*}
    \frac 1a &+ \frac 1b = D \implies b = \frac 1{D - \frac 1a} \implies \ell = a + b = a + \frac a{Da - 1} = \frac{ Da^2 }{Da - 1} \implies \\
    \implies \ell'_a &= \frac{ 2Da \cdot (Da - 1) - Da^2 \cdot D }{\sqr{Da - 1}}= \frac{ D^2a^2 - 2Da}{\sqr{Da - 1}} = \frac{ Da(Da - 2)}{\sqr{Da - 1}}\implies a_{\min} = \frac 2D \approx 66{,}7\,\text{мм}.
    \end{align*}
}
\solutionspace{120pt}

\tasknumber{5}%
\task{%
    Предмет в виде отрезка длиной $\ell$ расположен вдоль оптической оси
    собирающей линзы с фокусным расстоянием $F$.
    Середина отрезка расположена
    иа расстоянии $a$ от линзы, которая даёт действительное изображение
    всех точек предмета.
    Определить продольное увеличение предмета.
}
\answer{%
    \begin{align*}
    \frac 1{a + \frac \ell 2} &+ \frac 1b = \frac 1F \implies b = \frac{F\cbr{a + \frac \ell 2}}{a + \frac \ell 2 - F} \\
    \frac 1{a - \frac \ell 2} &+ \frac 1c = \frac 1F \implies c = \frac{F\cbr{a - \frac \ell 2}}{a - \frac \ell 2 - F} \\
    \abs{b - c} &= \abs{\frac{F\cbr{a + \frac \ell 2}}{a + \frac \ell 2 - F} - \frac{F\cbr{a - \frac \ell 2}}{a - \frac \ell 2 - F}}= F\abs{\frac{\cbr{a + \frac \ell 2}\cbr{a - \frac \ell 2 - F} - \cbr{a - \frac \ell 2}\cbr{a + \frac \ell 2 - F}}{ \cbr{a + \frac \ell 2 - F} \cbr{a - \frac \ell 2 - F} }} =  \\
    &= F\abs{\frac{a^2 - \frac {a\ell} 2 - Fa + \frac {a\ell} 2 - \frac {\ell^2} 4 - \frac {F\ell}2 - a^2 - \frac {a\ell}2 + aF + \frac {a\ell}2 + \frac {\ell^2} 4 - \frac {F\ell} 2}{\cbr{a + \frac \ell 2 - F} \cbr{a - \frac \ell 2 - F} }} = \\
    &= F\frac{F\ell}{\sqr{a-F} - \frac {\ell^2}4} = \frac{F^2\ell}{\sqr{a-F} - \frac {\ell^2}4}\implies \Gamma = \frac{\abs{b - c}}\ell = \frac{F^2}{\sqr{a-F} - \frac {\ell^2}4}.
    \end{align*}
}
\solutionspace{120pt}

\tasknumber{6}%
\task{%
    Даны точечный источник света $S$, его изображение $S_1$, полученное с помошью собирающей линзы,
    и ближайший к источнику фокус линзы $F$ (см.
    рис.
    на доске).
    Расстояния $SF = \ell$ и $SS_1 = L$.
    Определить положение линзы и её фокусное расстояние.
}
\answer{%
    \begin{align*}
    \frac 1a + \frac 1b &= \frac 1F, \ell = a - F, L = a + b \implies a = \ell + F, b = L - a = L - \ell - F \\
    \frac 1{\ell + F} + \frac 1{L - \ell - F} &= \frac 1F \\
    F\ell + F^2 + LF - F\ell - F^2 &= L\ell - \ell^2 - F\ell + LF - F\ell - F^2 \\
    0 &= L\ell - \ell^2 - 2F\ell - F^2 \\
    0 &=  F^2 + 2F\ell - L\ell + \ell^2 \\
    F &= -\ell \pm \sqrt{\ell^2 +  L\ell - \ell^2} = -\ell \pm \sqrt{L\ell} \implies F = \sqrt{L\ell} - \ell \\
    a &= \ell + F = \ell + \sqrt{L\ell} - \ell = \sqrt{L\ell}.
    \end{align*}
}
\solutionspace{120pt}

\tasknumber{7}%
\task{%
    Расстояние от освещённого предмета до экрана $100\,\text{см}$.
    Линза, помещенная между ними, даёт чёткое изображение предмета на
    экране при двух положениях, расстояние между которыми $30\,\text{см}$.
    Найти фокусное расстояние линзы.
}
\answer{%
    \begin{align*}
    \frac 1a + \frac 1b &= \frac 1F, \frac 1{a-\ell} + \frac 1{b+\ell} = \frac 1F, a + b = L \\
    \frac 1a + \frac 1b &= \frac 1{a-\ell} + \frac 1{b+\ell}\implies \frac{a + b}{ab} = \frac{(a-\ell) + (b+\ell)}{(a-\ell)(b+\ell)} \\
    ab  &= (a - \ell)(b+\ell) \implies 0  = -b\ell + a\ell - \ell^2 \implies 0 = -b + a - \ell \implies b = a - \ell \\
    a + (a - \ell) &= L \implies a = \frac{L + \ell}2 \implies b = \frac{L - \ell}2 \\
    F &= \frac{ab}{a + b} = \frac{L^2 -\ell^2}{4L} \approx 22{,}8\,\text{см}.
    \end{align*}
}
\solutionspace{120pt}

\tasknumber{8}%
\task{%
    Предмет находится на расстоянии $70\,\text{см}$ от экрана.
    Между предметом и экраном помещают линзу, причём при одном
    положении линзы на экране получается увеличенное изображение предмета,
    а при другом — уменьшенное.
    Каково фокусное расстояние линзы, если
    линейные размеры первого изображения в два раза больше второго?
}
\answer{%
    \begin{align*}
    \frac 1a + \frac 1{L-a} &= \frac 1F, h_1 = h \cdot \frac{L-a}a, \\
    \frac 1b + \frac 1{L-b} &= \frac 1F, h_2 = h \cdot \frac{L-b}b, \\
    \frac{h_1}{h_2} &= 2 \implies \frac{(L-a)b}{(L-b)a} = 2, \\
    \frac 1F &= \frac{ L }{a(L-a)} = \frac{ L }{b(L-b)} \implies \frac{L-a}{L-b} = \frac b a \implies \frac {b^2}{a^2} = 2.
    \\
    \frac 1a + \frac 1{L-a} &= \frac 1b + \frac 1{L-b} \implies \frac L{a(L-a)} = \frac L{b(L-b)} \implies \\
    \implies aL - a^2 &= bL - b^2 \implies (a-b)L = (a-b)(a+b) \implies b = L - a, \\
    \frac{\sqr{L-a}}{a^2} &= 2 \implies \frac La - 1 = \sqrt{2} \implies a = \frac{ L }{\sqrt{2} + 1} \\
    F &= \frac{a(L-a)}L = \frac 1L \cdot \frac L{\sqrt{2} + 1} \cdot \frac {L\sqrt{2}}{\sqrt{2} + 1}= \frac { L\sqrt{2} }{ \sqr{\sqrt{2} + 1} } \approx 17{,}0\,\text{см}.
    \end{align*}
}

\variantsplitter

\addpersonalvariant{Алина Полканова}

\tasknumber{1}%
\task{%
    В каком месте на главной оптической оси двояковыпуклой линзы
    нужно поместить точечный источник света,
    чтобы его изображение оказалось в главном фокусе линзы?
}
\answer{%
    $\text{для мнимого - на половине фокусного, для действительного - на бесконечности}$
}
\solutionspace{120pt}

\tasknumber{2}%
\task{%
    На экране, расположенном иа расстоянии $120\,\text{см}$ от собирающей линзы,
    получено изображение точечного источника, расположенного на главной оптической оси линзы.
    На какое расстояние переместится изображение на экране,
    если при неподвижном источнике переместить линзу на $1\,\text{см}$ в плоскости, перпендикулярной главной оптической оси?
    Фокусное расстояние линзы равно $40\,\text{см}$.
}
\answer{%
    \begin{align*}
    &\frac 1F = \frac 1a + \frac 1b \implies a = \frac{bF}{b-F} \implies \Gamma = \frac ba = \frac{b-F}F \\
    &y = x \cdot \Gamma = x \cdot \frac{b-F}F \implies d = x + y = 3\,\text{см}.
    \end{align*}
}
\solutionspace{120pt}

\tasknumber{3}%
\task{%
    Оптическая сила двояковыпуклой линзы в воздухе $5\,\text{дптр}$, а в воде $1{,}4\,\text{дптр}$.
    Определить показатель преломления $n$ материала, из которого изготовлена линза.
}
\answer{%
    \begin{align*}
    D_1 &=\cbr{\frac n{n_1} - 1}\cbr{\frac 1{R_1} + \frac 1{R_2}}, \\
    D_2 &=\cbr{\frac n{n_2} - 1}\cbr{\frac 1{R_1} + \frac 1{R_2}}, \\
    \frac {D_2}{D_1} &=\frac{\frac n{n_2} - 1}{\frac n{n_1} - 1} \implies {D_2}\cbr{\frac n{n_1} - 1} = {D_1}\cbr{\frac n{n_2} - 1}  \implies n\cbr{\frac{D_2}{n_1} - \frac{D_1}{n_2}} = D_2 - D_1, \\
    n &= \frac{D_2 - D_1}{\frac{D_2}{n_1} - \frac{D_1}{n_2}} = \frac{n_1 n_2 (D_2 - D_1)}{D_2n_2 - D_1n_1} \approx 1{,}526.
    \end{align*}
}
\solutionspace{120pt}

\tasknumber{4}%
\task{%
    На каком расстоянии от собирающей линзы с фокусным расстоянием $40\,\text{дптр}$
    следует надо поместить предмет, чтобы расстояние
    от предмета до его действительного изображения было наименьшим?
}
\answer{%
    \begin{align*}
    \frac 1a &+ \frac 1b = D \implies b = \frac 1{D - \frac 1a} \implies \ell = a + b = a + \frac a{Da - 1} = \frac{ Da^2 }{Da - 1} \implies \\
    \implies \ell'_a &= \frac{ 2Da \cdot (Da - 1) - Da^2 \cdot D }{\sqr{Da - 1}}= \frac{ D^2a^2 - 2Da}{\sqr{Da - 1}} = \frac{ Da(Da - 2)}{\sqr{Da - 1}}\implies a_{\min} = \frac 2D \approx 50\,\text{мм}.
    \end{align*}
}
\solutionspace{120pt}

\tasknumber{5}%
\task{%
    Предмет в виде отрезка длиной $\ell$ расположен вдоль оптической оси
    собирающей линзы с фокусным расстоянием $F$.
    Середина отрезка расположена
    иа расстоянии $a$ от линзы, которая даёт действительное изображение
    всех точек предмета.
    Определить продольное увеличение предмета.
}
\answer{%
    \begin{align*}
    \frac 1{a + \frac \ell 2} &+ \frac 1b = \frac 1F \implies b = \frac{F\cbr{a + \frac \ell 2}}{a + \frac \ell 2 - F} \\
    \frac 1{a - \frac \ell 2} &+ \frac 1c = \frac 1F \implies c = \frac{F\cbr{a - \frac \ell 2}}{a - \frac \ell 2 - F} \\
    \abs{b - c} &= \abs{\frac{F\cbr{a + \frac \ell 2}}{a + \frac \ell 2 - F} - \frac{F\cbr{a - \frac \ell 2}}{a - \frac \ell 2 - F}}= F\abs{\frac{\cbr{a + \frac \ell 2}\cbr{a - \frac \ell 2 - F} - \cbr{a - \frac \ell 2}\cbr{a + \frac \ell 2 - F}}{ \cbr{a + \frac \ell 2 - F} \cbr{a - \frac \ell 2 - F} }} =  \\
    &= F\abs{\frac{a^2 - \frac {a\ell} 2 - Fa + \frac {a\ell} 2 - \frac {\ell^2} 4 - \frac {F\ell}2 - a^2 - \frac {a\ell}2 + aF + \frac {a\ell}2 + \frac {\ell^2} 4 - \frac {F\ell} 2}{\cbr{a + \frac \ell 2 - F} \cbr{a - \frac \ell 2 - F} }} = \\
    &= F\frac{F\ell}{\sqr{a-F} - \frac {\ell^2}4} = \frac{F^2\ell}{\sqr{a-F} - \frac {\ell^2}4}\implies \Gamma = \frac{\abs{b - c}}\ell = \frac{F^2}{\sqr{a-F} - \frac {\ell^2}4}.
    \end{align*}
}
\solutionspace{120pt}

\tasknumber{6}%
\task{%
    Даны точечный источник света $S$, его изображение $S_1$, полученное с помошью собирающей линзы,
    и ближайший к источнику фокус линзы $F$ (см.
    рис.
    на доске).
    Расстояния $SF = \ell$ и $SS_1 = L$.
    Определить положение линзы и её фокусное расстояние.
}
\answer{%
    \begin{align*}
    \frac 1a + \frac 1b &= \frac 1F, \ell = a - F, L = a + b \implies a = \ell + F, b = L - a = L - \ell - F \\
    \frac 1{\ell + F} + \frac 1{L - \ell - F} &= \frac 1F \\
    F\ell + F^2 + LF - F\ell - F^2 &= L\ell - \ell^2 - F\ell + LF - F\ell - F^2 \\
    0 &= L\ell - \ell^2 - 2F\ell - F^2 \\
    0 &=  F^2 + 2F\ell - L\ell + \ell^2 \\
    F &= -\ell \pm \sqrt{\ell^2 +  L\ell - \ell^2} = -\ell \pm \sqrt{L\ell} \implies F = \sqrt{L\ell} - \ell \\
    a &= \ell + F = \ell + \sqrt{L\ell} - \ell = \sqrt{L\ell}.
    \end{align*}
}
\solutionspace{120pt}

\tasknumber{7}%
\task{%
    Расстояние от освещённого предмета до экрана $100\,\text{см}$.
    Линза, помещенная между ними, даёт чёткое изображение предмета на
    экране при двух положениях, расстояние между которыми $30\,\text{см}$.
    Найти фокусное расстояние линзы.
}
\answer{%
    \begin{align*}
    \frac 1a + \frac 1b &= \frac 1F, \frac 1{a-\ell} + \frac 1{b+\ell} = \frac 1F, a + b = L \\
    \frac 1a + \frac 1b &= \frac 1{a-\ell} + \frac 1{b+\ell}\implies \frac{a + b}{ab} = \frac{(a-\ell) + (b+\ell)}{(a-\ell)(b+\ell)} \\
    ab  &= (a - \ell)(b+\ell) \implies 0  = -b\ell + a\ell - \ell^2 \implies 0 = -b + a - \ell \implies b = a - \ell \\
    a + (a - \ell) &= L \implies a = \frac{L + \ell}2 \implies b = \frac{L - \ell}2 \\
    F &= \frac{ab}{a + b} = \frac{L^2 -\ell^2}{4L} \approx 22{,}8\,\text{см}.
    \end{align*}
}
\solutionspace{120pt}

\tasknumber{8}%
\task{%
    Предмет находится на расстоянии $80\,\text{см}$ от экрана.
    Между предметом и экраном помещают линзу, причём при одном
    положении линзы на экране получается увеличенное изображение предмета,
    а при другом — уменьшенное.
    Каково фокусное расстояние линзы, если
    линейные размеры первого изображения в три раза больше второго?
}
\answer{%
    \begin{align*}
    \frac 1a + \frac 1{L-a} &= \frac 1F, h_1 = h \cdot \frac{L-a}a, \\
    \frac 1b + \frac 1{L-b} &= \frac 1F, h_2 = h \cdot \frac{L-b}b, \\
    \frac{h_1}{h_2} &= 3 \implies \frac{(L-a)b}{(L-b)a} = 3, \\
    \frac 1F &= \frac{ L }{a(L-a)} = \frac{ L }{b(L-b)} \implies \frac{L-a}{L-b} = \frac b a \implies \frac {b^2}{a^2} = 3.
    \\
    \frac 1a + \frac 1{L-a} &= \frac 1b + \frac 1{L-b} \implies \frac L{a(L-a)} = \frac L{b(L-b)} \implies \\
    \implies aL - a^2 &= bL - b^2 \implies (a-b)L = (a-b)(a+b) \implies b = L - a, \\
    \frac{\sqr{L-a}}{a^2} &= 3 \implies \frac La - 1 = \sqrt{3} \implies a = \frac{ L }{\sqrt{3} + 1} \\
    F &= \frac{a(L-a)}L = \frac 1L \cdot \frac L{\sqrt{3} + 1} \cdot \frac {L\sqrt{3}}{\sqrt{3} + 1}= \frac { L\sqrt{3} }{ \sqr{\sqrt{3} + 1} } \approx 18{,}6\,\text{см}.
    \end{align*}
}

\variantsplitter

\addpersonalvariant{Сергей Пономарёв}

\tasknumber{1}%
\task{%
    В каком месте на главной оптической оси двояковыпуклой линзы
    нужно поместить точечный источник света,
    чтобы его изображение оказалось в главном фокусе линзы?
}
\answer{%
    $\text{для мнимого - на половине фокусного, для действительного - на бесконечности}$
}
\solutionspace{120pt}

\tasknumber{2}%
\task{%
    На экране, расположенном иа расстоянии $60\,\text{см}$ от собирающей линзы,
    получено изображение точечного источника, расположенного на главной оптической оси линзы.
    На какое расстояние переместится изображение на экране,
    если при неподвижном источнике переместить линзу на $1\,\text{см}$ в плоскости, перпендикулярной главной оптической оси?
    Фокусное расстояние линзы равно $30\,\text{см}$.
}
\answer{%
    \begin{align*}
    &\frac 1F = \frac 1a + \frac 1b \implies a = \frac{bF}{b-F} \implies \Gamma = \frac ba = \frac{b-F}F \\
    &y = x \cdot \Gamma = x \cdot \frac{b-F}F \implies d = x + y = 2\,\text{см}.
    \end{align*}
}
\solutionspace{120pt}

\tasknumber{3}%
\task{%
    Оптическая сила двояковыпуклой линзы в воздухе $5{,}5\,\text{дптр}$, а в воде $1{,}4\,\text{дптр}$.
    Определить показатель преломления $n$ материала, из которого изготовлена линза.
}
\answer{%
    \begin{align*}
    D_1 &=\cbr{\frac n{n_1} - 1}\cbr{\frac 1{R_1} + \frac 1{R_2}}, \\
    D_2 &=\cbr{\frac n{n_2} - 1}\cbr{\frac 1{R_1} + \frac 1{R_2}}, \\
    \frac {D_2}{D_1} &=\frac{\frac n{n_2} - 1}{\frac n{n_1} - 1} \implies {D_2}\cbr{\frac n{n_1} - 1} = {D_1}\cbr{\frac n{n_2} - 1}  \implies n\cbr{\frac{D_2}{n_1} - \frac{D_1}{n_2}} = D_2 - D_1, \\
    n &= \frac{D_2 - D_1}{\frac{D_2}{n_1} - \frac{D_1}{n_2}} = \frac{n_1 n_2 (D_2 - D_1)}{D_2n_2 - D_1n_1} \approx 1{,}499.
    \end{align*}
}
\solutionspace{120pt}

\tasknumber{4}%
\task{%
    На каком расстоянии от собирающей линзы с фокусным расстоянием $30\,\text{дптр}$
    следует надо поместить предмет, чтобы расстояние
    от предмета до его действительного изображения было наименьшим?
}
\answer{%
    \begin{align*}
    \frac 1a &+ \frac 1b = D \implies b = \frac 1{D - \frac 1a} \implies \ell = a + b = a + \frac a{Da - 1} = \frac{ Da^2 }{Da - 1} \implies \\
    \implies \ell'_a &= \frac{ 2Da \cdot (Da - 1) - Da^2 \cdot D }{\sqr{Da - 1}}= \frac{ D^2a^2 - 2Da}{\sqr{Da - 1}} = \frac{ Da(Da - 2)}{\sqr{Da - 1}}\implies a_{\min} = \frac 2D \approx 66{,}7\,\text{мм}.
    \end{align*}
}
\solutionspace{120pt}

\tasknumber{5}%
\task{%
    Предмет в виде отрезка длиной $\ell$ расположен вдоль оптической оси
    собирающей линзы с фокусным расстоянием $F$.
    Середина отрезка расположена
    иа расстоянии $a$ от линзы, которая даёт действительное изображение
    всех точек предмета.
    Определить продольное увеличение предмета.
}
\answer{%
    \begin{align*}
    \frac 1{a + \frac \ell 2} &+ \frac 1b = \frac 1F \implies b = \frac{F\cbr{a + \frac \ell 2}}{a + \frac \ell 2 - F} \\
    \frac 1{a - \frac \ell 2} &+ \frac 1c = \frac 1F \implies c = \frac{F\cbr{a - \frac \ell 2}}{a - \frac \ell 2 - F} \\
    \abs{b - c} &= \abs{\frac{F\cbr{a + \frac \ell 2}}{a + \frac \ell 2 - F} - \frac{F\cbr{a - \frac \ell 2}}{a - \frac \ell 2 - F}}= F\abs{\frac{\cbr{a + \frac \ell 2}\cbr{a - \frac \ell 2 - F} - \cbr{a - \frac \ell 2}\cbr{a + \frac \ell 2 - F}}{ \cbr{a + \frac \ell 2 - F} \cbr{a - \frac \ell 2 - F} }} =  \\
    &= F\abs{\frac{a^2 - \frac {a\ell} 2 - Fa + \frac {a\ell} 2 - \frac {\ell^2} 4 - \frac {F\ell}2 - a^2 - \frac {a\ell}2 + aF + \frac {a\ell}2 + \frac {\ell^2} 4 - \frac {F\ell} 2}{\cbr{a + \frac \ell 2 - F} \cbr{a - \frac \ell 2 - F} }} = \\
    &= F\frac{F\ell}{\sqr{a-F} - \frac {\ell^2}4} = \frac{F^2\ell}{\sqr{a-F} - \frac {\ell^2}4}\implies \Gamma = \frac{\abs{b - c}}\ell = \frac{F^2}{\sqr{a-F} - \frac {\ell^2}4}.
    \end{align*}
}
\solutionspace{120pt}

\tasknumber{6}%
\task{%
    Даны точечный источник света $S$, его изображение $S_1$, полученное с помошью собирающей линзы,
    и ближайший к источнику фокус линзы $F$ (см.
    рис.
    на доске).
    Расстояния $SF = \ell$ и $SS_1 = L$.
    Определить положение линзы и её фокусное расстояние.
}
\answer{%
    \begin{align*}
    \frac 1a + \frac 1b &= \frac 1F, \ell = a - F, L = a + b \implies a = \ell + F, b = L - a = L - \ell - F \\
    \frac 1{\ell + F} + \frac 1{L - \ell - F} &= \frac 1F \\
    F\ell + F^2 + LF - F\ell - F^2 &= L\ell - \ell^2 - F\ell + LF - F\ell - F^2 \\
    0 &= L\ell - \ell^2 - 2F\ell - F^2 \\
    0 &=  F^2 + 2F\ell - L\ell + \ell^2 \\
    F &= -\ell \pm \sqrt{\ell^2 +  L\ell - \ell^2} = -\ell \pm \sqrt{L\ell} \implies F = \sqrt{L\ell} - \ell \\
    a &= \ell + F = \ell + \sqrt{L\ell} - \ell = \sqrt{L\ell}.
    \end{align*}
}
\solutionspace{120pt}

\tasknumber{7}%
\task{%
    Расстояние от освещённого предмета до экрана $80\,\text{см}$.
    Линза, помещенная между ними, даёт чёткое изображение предмета на
    экране при двух положениях, расстояние между которыми $40\,\text{см}$.
    Найти фокусное расстояние линзы.
}
\answer{%
    \begin{align*}
    \frac 1a + \frac 1b &= \frac 1F, \frac 1{a-\ell} + \frac 1{b+\ell} = \frac 1F, a + b = L \\
    \frac 1a + \frac 1b &= \frac 1{a-\ell} + \frac 1{b+\ell}\implies \frac{a + b}{ab} = \frac{(a-\ell) + (b+\ell)}{(a-\ell)(b+\ell)} \\
    ab  &= (a - \ell)(b+\ell) \implies 0  = -b\ell + a\ell - \ell^2 \implies 0 = -b + a - \ell \implies b = a - \ell \\
    a + (a - \ell) &= L \implies a = \frac{L + \ell}2 \implies b = \frac{L - \ell}2 \\
    F &= \frac{ab}{a + b} = \frac{L^2 -\ell^2}{4L} \approx 15\,\text{см}.
    \end{align*}
}
\solutionspace{120pt}

\tasknumber{8}%
\task{%
    Предмет находится на расстоянии $70\,\text{см}$ от экрана.
    Между предметом и экраном помещают линзу, причём при одном
    положении линзы на экране получается увеличенное изображение предмета,
    а при другом — уменьшенное.
    Каково фокусное расстояние линзы, если
    линейные размеры первого изображения в пять раз больше второго?
}
\answer{%
    \begin{align*}
    \frac 1a + \frac 1{L-a} &= \frac 1F, h_1 = h \cdot \frac{L-a}a, \\
    \frac 1b + \frac 1{L-b} &= \frac 1F, h_2 = h \cdot \frac{L-b}b, \\
    \frac{h_1}{h_2} &= 5 \implies \frac{(L-a)b}{(L-b)a} = 5, \\
    \frac 1F &= \frac{ L }{a(L-a)} = \frac{ L }{b(L-b)} \implies \frac{L-a}{L-b} = \frac b a \implies \frac {b^2}{a^2} = 5.
    \\
    \frac 1a + \frac 1{L-a} &= \frac 1b + \frac 1{L-b} \implies \frac L{a(L-a)} = \frac L{b(L-b)} \implies \\
    \implies aL - a^2 &= bL - b^2 \implies (a-b)L = (a-b)(a+b) \implies b = L - a, \\
    \frac{\sqr{L-a}}{a^2} &= 5 \implies \frac La - 1 = \sqrt{5} \implies a = \frac{ L }{\sqrt{5} + 1} \\
    F &= \frac{a(L-a)}L = \frac 1L \cdot \frac L{\sqrt{5} + 1} \cdot \frac {L\sqrt{5}}{\sqrt{5} + 1}= \frac { L\sqrt{5} }{ \sqr{\sqrt{5} + 1} } \approx 14{,}9\,\text{см}.
    \end{align*}
}

\variantsplitter

\addpersonalvariant{Егор Свистушкин}

\tasknumber{1}%
\task{%
    В каком месте на главной оптической оси двояковыпуклой линзы
    нужно поместить точечный источник света,
    чтобы его изображение оказалось в главном фокусе линзы?
}
\answer{%
    $\text{для мнимого - на половине фокусного, для действительного - на бесконечности}$
}
\solutionspace{120pt}

\tasknumber{2}%
\task{%
    На экране, расположенном иа расстоянии $80\,\text{см}$ от собирающей линзы,
    получено изображение точечного источника, расположенного на главной оптической оси линзы.
    На какое расстояние переместится изображение на экране,
    если при неподвижном источнике переместить линзу на $3\,\text{см}$ в плоскости, перпендикулярной главной оптической оси?
    Фокусное расстояние линзы равно $40\,\text{см}$.
}
\answer{%
    \begin{align*}
    &\frac 1F = \frac 1a + \frac 1b \implies a = \frac{bF}{b-F} \implies \Gamma = \frac ba = \frac{b-F}F \\
    &y = x \cdot \Gamma = x \cdot \frac{b-F}F \implies d = x + y = 6\,\text{см}.
    \end{align*}
}
\solutionspace{120pt}

\tasknumber{3}%
\task{%
    Оптическая сила двояковыпуклой линзы в воздухе $5{,}5\,\text{дптр}$, а в воде $1{,}6\,\text{дптр}$.
    Определить показатель преломления $n$ материала, из которого изготовлена линза.
}
\answer{%
    \begin{align*}
    D_1 &=\cbr{\frac n{n_1} - 1}\cbr{\frac 1{R_1} + \frac 1{R_2}}, \\
    D_2 &=\cbr{\frac n{n_2} - 1}\cbr{\frac 1{R_1} + \frac 1{R_2}}, \\
    \frac {D_2}{D_1} &=\frac{\frac n{n_2} - 1}{\frac n{n_1} - 1} \implies {D_2}\cbr{\frac n{n_1} - 1} = {D_1}\cbr{\frac n{n_2} - 1}  \implies n\cbr{\frac{D_2}{n_1} - \frac{D_1}{n_2}} = D_2 - D_1, \\
    n &= \frac{D_2 - D_1}{\frac{D_2}{n_1} - \frac{D_1}{n_2}} = \frac{n_1 n_2 (D_2 - D_1)}{D_2n_2 - D_1n_1} \approx 1{,}538.
    \end{align*}
}
\solutionspace{120pt}

\tasknumber{4}%
\task{%
    На каком расстоянии от собирающей линзы с фокусным расстоянием $40\,\text{дптр}$
    следует надо поместить предмет, чтобы расстояние
    от предмета до его действительного изображения было наименьшим?
}
\answer{%
    \begin{align*}
    \frac 1a &+ \frac 1b = D \implies b = \frac 1{D - \frac 1a} \implies \ell = a + b = a + \frac a{Da - 1} = \frac{ Da^2 }{Da - 1} \implies \\
    \implies \ell'_a &= \frac{ 2Da \cdot (Da - 1) - Da^2 \cdot D }{\sqr{Da - 1}}= \frac{ D^2a^2 - 2Da}{\sqr{Da - 1}} = \frac{ Da(Da - 2)}{\sqr{Da - 1}}\implies a_{\min} = \frac 2D \approx 50\,\text{мм}.
    \end{align*}
}
\solutionspace{120pt}

\tasknumber{5}%
\task{%
    Предмет в виде отрезка длиной $\ell$ расположен вдоль оптической оси
    собирающей линзы с фокусным расстоянием $F$.
    Середина отрезка расположена
    иа расстоянии $a$ от линзы, которая даёт действительное изображение
    всех точек предмета.
    Определить продольное увеличение предмета.
}
\answer{%
    \begin{align*}
    \frac 1{a + \frac \ell 2} &+ \frac 1b = \frac 1F \implies b = \frac{F\cbr{a + \frac \ell 2}}{a + \frac \ell 2 - F} \\
    \frac 1{a - \frac \ell 2} &+ \frac 1c = \frac 1F \implies c = \frac{F\cbr{a - \frac \ell 2}}{a - \frac \ell 2 - F} \\
    \abs{b - c} &= \abs{\frac{F\cbr{a + \frac \ell 2}}{a + \frac \ell 2 - F} - \frac{F\cbr{a - \frac \ell 2}}{a - \frac \ell 2 - F}}= F\abs{\frac{\cbr{a + \frac \ell 2}\cbr{a - \frac \ell 2 - F} - \cbr{a - \frac \ell 2}\cbr{a + \frac \ell 2 - F}}{ \cbr{a + \frac \ell 2 - F} \cbr{a - \frac \ell 2 - F} }} =  \\
    &= F\abs{\frac{a^2 - \frac {a\ell} 2 - Fa + \frac {a\ell} 2 - \frac {\ell^2} 4 - \frac {F\ell}2 - a^2 - \frac {a\ell}2 + aF + \frac {a\ell}2 + \frac {\ell^2} 4 - \frac {F\ell} 2}{\cbr{a + \frac \ell 2 - F} \cbr{a - \frac \ell 2 - F} }} = \\
    &= F\frac{F\ell}{\sqr{a-F} - \frac {\ell^2}4} = \frac{F^2\ell}{\sqr{a-F} - \frac {\ell^2}4}\implies \Gamma = \frac{\abs{b - c}}\ell = \frac{F^2}{\sqr{a-F} - \frac {\ell^2}4}.
    \end{align*}
}
\solutionspace{120pt}

\tasknumber{6}%
\task{%
    Даны точечный источник света $S$, его изображение $S_1$, полученное с помошью собирающей линзы,
    и ближайший к источнику фокус линзы $F$ (см.
    рис.
    на доске).
    Расстояния $SF = \ell$ и $SS_1 = L$.
    Определить положение линзы и её фокусное расстояние.
}
\answer{%
    \begin{align*}
    \frac 1a + \frac 1b &= \frac 1F, \ell = a - F, L = a + b \implies a = \ell + F, b = L - a = L - \ell - F \\
    \frac 1{\ell + F} + \frac 1{L - \ell - F} &= \frac 1F \\
    F\ell + F^2 + LF - F\ell - F^2 &= L\ell - \ell^2 - F\ell + LF - F\ell - F^2 \\
    0 &= L\ell - \ell^2 - 2F\ell - F^2 \\
    0 &=  F^2 + 2F\ell - L\ell + \ell^2 \\
    F &= -\ell \pm \sqrt{\ell^2 +  L\ell - \ell^2} = -\ell \pm \sqrt{L\ell} \implies F = \sqrt{L\ell} - \ell \\
    a &= \ell + F = \ell + \sqrt{L\ell} - \ell = \sqrt{L\ell}.
    \end{align*}
}
\solutionspace{120pt}

\tasknumber{7}%
\task{%
    Расстояние от освещённого предмета до экрана $80\,\text{см}$.
    Линза, помещенная между ними, даёт чёткое изображение предмета на
    экране при двух положениях, расстояние между которыми $20\,\text{см}$.
    Найти фокусное расстояние линзы.
}
\answer{%
    \begin{align*}
    \frac 1a + \frac 1b &= \frac 1F, \frac 1{a-\ell} + \frac 1{b+\ell} = \frac 1F, a + b = L \\
    \frac 1a + \frac 1b &= \frac 1{a-\ell} + \frac 1{b+\ell}\implies \frac{a + b}{ab} = \frac{(a-\ell) + (b+\ell)}{(a-\ell)(b+\ell)} \\
    ab  &= (a - \ell)(b+\ell) \implies 0  = -b\ell + a\ell - \ell^2 \implies 0 = -b + a - \ell \implies b = a - \ell \\
    a + (a - \ell) &= L \implies a = \frac{L + \ell}2 \implies b = \frac{L - \ell}2 \\
    F &= \frac{ab}{a + b} = \frac{L^2 -\ell^2}{4L} \approx 18{,}8\,\text{см}.
    \end{align*}
}
\solutionspace{120pt}

\tasknumber{8}%
\task{%
    Предмет находится на расстоянии $70\,\text{см}$ от экрана.
    Между предметом и экраном помещают линзу, причём при одном
    положении линзы на экране получается увеличенное изображение предмета,
    а при другом — уменьшенное.
    Каково фокусное расстояние линзы, если
    линейные размеры первого изображения в пять раз больше второго?
}
\answer{%
    \begin{align*}
    \frac 1a + \frac 1{L-a} &= \frac 1F, h_1 = h \cdot \frac{L-a}a, \\
    \frac 1b + \frac 1{L-b} &= \frac 1F, h_2 = h \cdot \frac{L-b}b, \\
    \frac{h_1}{h_2} &= 5 \implies \frac{(L-a)b}{(L-b)a} = 5, \\
    \frac 1F &= \frac{ L }{a(L-a)} = \frac{ L }{b(L-b)} \implies \frac{L-a}{L-b} = \frac b a \implies \frac {b^2}{a^2} = 5.
    \\
    \frac 1a + \frac 1{L-a} &= \frac 1b + \frac 1{L-b} \implies \frac L{a(L-a)} = \frac L{b(L-b)} \implies \\
    \implies aL - a^2 &= bL - b^2 \implies (a-b)L = (a-b)(a+b) \implies b = L - a, \\
    \frac{\sqr{L-a}}{a^2} &= 5 \implies \frac La - 1 = \sqrt{5} \implies a = \frac{ L }{\sqrt{5} + 1} \\
    F &= \frac{a(L-a)}L = \frac 1L \cdot \frac L{\sqrt{5} + 1} \cdot \frac {L\sqrt{5}}{\sqrt{5} + 1}= \frac { L\sqrt{5} }{ \sqr{\sqrt{5} + 1} } \approx 14{,}9\,\text{см}.
    \end{align*}
}

\variantsplitter

\addpersonalvariant{Дмитрий Соколов}

\tasknumber{1}%
\task{%
    В каком месте на главной оптической оси двояковыпуклой линзы
    нужно поместить точечный источник света,
    чтобы его изображение оказалось в главном фокусе линзы?
}
\answer{%
    $\text{для мнимого - на половине фокусного, для действительного - на бесконечности}$
}
\solutionspace{120pt}

\tasknumber{2}%
\task{%
    На экране, расположенном иа расстоянии $120\,\text{см}$ от собирающей линзы,
    получено изображение точечного источника, расположенного на главной оптической оси линзы.
    На какое расстояние переместится изображение на экране,
    если при неподвижной линзе переместить источник на $1\,\text{см}$ в плоскости, перпендикулярной главной оптической оси?
    Фокусное расстояние линзы равно $20\,\text{см}$.
}
\answer{%
    \begin{align*}
    &\frac 1F = \frac 1a + \frac 1b \implies a = \frac{bF}{b-F} \implies \Gamma = \frac ba = \frac{b-F}F \\
    &y = x \cdot \Gamma = x \cdot \frac{b-F}F \implies d = y = 5\,\text{см}.
    \end{align*}
}
\solutionspace{120pt}

\tasknumber{3}%
\task{%
    Оптическая сила двояковыпуклой линзы в воздухе $5{,}5\,\text{дптр}$, а в воде $1{,}6\,\text{дптр}$.
    Определить показатель преломления $n$ материала, из которого изготовлена линза.
}
\answer{%
    \begin{align*}
    D_1 &=\cbr{\frac n{n_1} - 1}\cbr{\frac 1{R_1} + \frac 1{R_2}}, \\
    D_2 &=\cbr{\frac n{n_2} - 1}\cbr{\frac 1{R_1} + \frac 1{R_2}}, \\
    \frac {D_2}{D_1} &=\frac{\frac n{n_2} - 1}{\frac n{n_1} - 1} \implies {D_2}\cbr{\frac n{n_1} - 1} = {D_1}\cbr{\frac n{n_2} - 1}  \implies n\cbr{\frac{D_2}{n_1} - \frac{D_1}{n_2}} = D_2 - D_1, \\
    n &= \frac{D_2 - D_1}{\frac{D_2}{n_1} - \frac{D_1}{n_2}} = \frac{n_1 n_2 (D_2 - D_1)}{D_2n_2 - D_1n_1} \approx 1{,}538.
    \end{align*}
}
\solutionspace{120pt}

\tasknumber{4}%
\task{%
    На каком расстоянии от собирающей линзы с фокусным расстоянием $50\,\text{дптр}$
    следует надо поместить предмет, чтобы расстояние
    от предмета до его действительного изображения было наименьшим?
}
\answer{%
    \begin{align*}
    \frac 1a &+ \frac 1b = D \implies b = \frac 1{D - \frac 1a} \implies \ell = a + b = a + \frac a{Da - 1} = \frac{ Da^2 }{Da - 1} \implies \\
    \implies \ell'_a &= \frac{ 2Da \cdot (Da - 1) - Da^2 \cdot D }{\sqr{Da - 1}}= \frac{ D^2a^2 - 2Da}{\sqr{Da - 1}} = \frac{ Da(Da - 2)}{\sqr{Da - 1}}\implies a_{\min} = \frac 2D \approx 40\,\text{мм}.
    \end{align*}
}
\solutionspace{120pt}

\tasknumber{5}%
\task{%
    Предмет в виде отрезка длиной $\ell$ расположен вдоль оптической оси
    собирающей линзы с фокусным расстоянием $F$.
    Середина отрезка расположена
    иа расстоянии $a$ от линзы, которая даёт действительное изображение
    всех точек предмета.
    Определить продольное увеличение предмета.
}
\answer{%
    \begin{align*}
    \frac 1{a + \frac \ell 2} &+ \frac 1b = \frac 1F \implies b = \frac{F\cbr{a + \frac \ell 2}}{a + \frac \ell 2 - F} \\
    \frac 1{a - \frac \ell 2} &+ \frac 1c = \frac 1F \implies c = \frac{F\cbr{a - \frac \ell 2}}{a - \frac \ell 2 - F} \\
    \abs{b - c} &= \abs{\frac{F\cbr{a + \frac \ell 2}}{a + \frac \ell 2 - F} - \frac{F\cbr{a - \frac \ell 2}}{a - \frac \ell 2 - F}}= F\abs{\frac{\cbr{a + \frac \ell 2}\cbr{a - \frac \ell 2 - F} - \cbr{a - \frac \ell 2}\cbr{a + \frac \ell 2 - F}}{ \cbr{a + \frac \ell 2 - F} \cbr{a - \frac \ell 2 - F} }} =  \\
    &= F\abs{\frac{a^2 - \frac {a\ell} 2 - Fa + \frac {a\ell} 2 - \frac {\ell^2} 4 - \frac {F\ell}2 - a^2 - \frac {a\ell}2 + aF + \frac {a\ell}2 + \frac {\ell^2} 4 - \frac {F\ell} 2}{\cbr{a + \frac \ell 2 - F} \cbr{a - \frac \ell 2 - F} }} = \\
    &= F\frac{F\ell}{\sqr{a-F} - \frac {\ell^2}4} = \frac{F^2\ell}{\sqr{a-F} - \frac {\ell^2}4}\implies \Gamma = \frac{\abs{b - c}}\ell = \frac{F^2}{\sqr{a-F} - \frac {\ell^2}4}.
    \end{align*}
}
\solutionspace{120pt}

\tasknumber{6}%
\task{%
    Даны точечный источник света $S$, его изображение $S_1$, полученное с помошью собирающей линзы,
    и ближайший к источнику фокус линзы $F$ (см.
    рис.
    на доске).
    Расстояния $SF = \ell$ и $SS_1 = L$.
    Определить положение линзы и её фокусное расстояние.
}
\answer{%
    \begin{align*}
    \frac 1a + \frac 1b &= \frac 1F, \ell = a - F, L = a + b \implies a = \ell + F, b = L - a = L - \ell - F \\
    \frac 1{\ell + F} + \frac 1{L - \ell - F} &= \frac 1F \\
    F\ell + F^2 + LF - F\ell - F^2 &= L\ell - \ell^2 - F\ell + LF - F\ell - F^2 \\
    0 &= L\ell - \ell^2 - 2F\ell - F^2 \\
    0 &=  F^2 + 2F\ell - L\ell + \ell^2 \\
    F &= -\ell \pm \sqrt{\ell^2 +  L\ell - \ell^2} = -\ell \pm \sqrt{L\ell} \implies F = \sqrt{L\ell} - \ell \\
    a &= \ell + F = \ell + \sqrt{L\ell} - \ell = \sqrt{L\ell}.
    \end{align*}
}
\solutionspace{120pt}

\tasknumber{7}%
\task{%
    Расстояние от освещённого предмета до экрана $80\,\text{см}$.
    Линза, помещенная между ними, даёт чёткое изображение предмета на
    экране при двух положениях, расстояние между которыми $20\,\text{см}$.
    Найти фокусное расстояние линзы.
}
\answer{%
    \begin{align*}
    \frac 1a + \frac 1b &= \frac 1F, \frac 1{a-\ell} + \frac 1{b+\ell} = \frac 1F, a + b = L \\
    \frac 1a + \frac 1b &= \frac 1{a-\ell} + \frac 1{b+\ell}\implies \frac{a + b}{ab} = \frac{(a-\ell) + (b+\ell)}{(a-\ell)(b+\ell)} \\
    ab  &= (a - \ell)(b+\ell) \implies 0  = -b\ell + a\ell - \ell^2 \implies 0 = -b + a - \ell \implies b = a - \ell \\
    a + (a - \ell) &= L \implies a = \frac{L + \ell}2 \implies b = \frac{L - \ell}2 \\
    F &= \frac{ab}{a + b} = \frac{L^2 -\ell^2}{4L} \approx 18{,}8\,\text{см}.
    \end{align*}
}
\solutionspace{120pt}

\tasknumber{8}%
\task{%
    Предмет находится на расстоянии $60\,\text{см}$ от экрана.
    Между предметом и экраном помещают линзу, причём при одном
    положении линзы на экране получается увеличенное изображение предмета,
    а при другом — уменьшенное.
    Каково фокусное расстояние линзы, если
    линейные размеры первого изображения в два раза больше второго?
}
\answer{%
    \begin{align*}
    \frac 1a + \frac 1{L-a} &= \frac 1F, h_1 = h \cdot \frac{L-a}a, \\
    \frac 1b + \frac 1{L-b} &= \frac 1F, h_2 = h \cdot \frac{L-b}b, \\
    \frac{h_1}{h_2} &= 2 \implies \frac{(L-a)b}{(L-b)a} = 2, \\
    \frac 1F &= \frac{ L }{a(L-a)} = \frac{ L }{b(L-b)} \implies \frac{L-a}{L-b} = \frac b a \implies \frac {b^2}{a^2} = 2.
    \\
    \frac 1a + \frac 1{L-a} &= \frac 1b + \frac 1{L-b} \implies \frac L{a(L-a)} = \frac L{b(L-b)} \implies \\
    \implies aL - a^2 &= bL - b^2 \implies (a-b)L = (a-b)(a+b) \implies b = L - a, \\
    \frac{\sqr{L-a}}{a^2} &= 2 \implies \frac La - 1 = \sqrt{2} \implies a = \frac{ L }{\sqrt{2} + 1} \\
    F &= \frac{a(L-a)}L = \frac 1L \cdot \frac L{\sqrt{2} + 1} \cdot \frac {L\sqrt{2}}{\sqrt{2} + 1}= \frac { L\sqrt{2} }{ \sqr{\sqrt{2} + 1} } \approx 14{,}6\,\text{см}.
    \end{align*}
}

\variantsplitter

\addpersonalvariant{Арсений Трофимов}

\tasknumber{1}%
\task{%
    В каком месте на главной оптической оси двояковыгнутой линзы
    нужно поместить точечный источник света,
    чтобы его изображение оказалось в главном фокусе линзы?
}
\answer{%
    $\text{на половине фокусного расстояния}$
}
\solutionspace{120pt}

\tasknumber{2}%
\task{%
    На экране, расположенном иа расстоянии $120\,\text{см}$ от собирающей линзы,
    получено изображение точечного источника, расположенного на главной оптической оси линзы.
    На какое расстояние переместится изображение на экране,
    если при неподвижной линзе переместить источник на $3\,\text{см}$ в плоскости, перпендикулярной главной оптической оси?
    Фокусное расстояние линзы равно $30\,\text{см}$.
}
\answer{%
    \begin{align*}
    &\frac 1F = \frac 1a + \frac 1b \implies a = \frac{bF}{b-F} \implies \Gamma = \frac ba = \frac{b-F}F \\
    &y = x \cdot \Gamma = x \cdot \frac{b-F}F \implies d = y = 9\,\text{см}.
    \end{align*}
}
\solutionspace{120pt}

\tasknumber{3}%
\task{%
    Оптическая сила двояковыпуклой линзы в воздухе $4{,}5\,\text{дптр}$, а в воде $1{,}6\,\text{дптр}$.
    Определить показатель преломления $n$ материала, из которого изготовлена линза.
}
\answer{%
    \begin{align*}
    D_1 &=\cbr{\frac n{n_1} - 1}\cbr{\frac 1{R_1} + \frac 1{R_2}}, \\
    D_2 &=\cbr{\frac n{n_2} - 1}\cbr{\frac 1{R_1} + \frac 1{R_2}}, \\
    \frac {D_2}{D_1} &=\frac{\frac n{n_2} - 1}{\frac n{n_1} - 1} \implies {D_2}\cbr{\frac n{n_1} - 1} = {D_1}\cbr{\frac n{n_2} - 1}  \implies n\cbr{\frac{D_2}{n_1} - \frac{D_1}{n_2}} = D_2 - D_1, \\
    n &= \frac{D_2 - D_1}{\frac{D_2}{n_1} - \frac{D_1}{n_2}} = \frac{n_1 n_2 (D_2 - D_1)}{D_2n_2 - D_1n_1} \approx 1{,}626.
    \end{align*}
}
\solutionspace{120pt}

\tasknumber{4}%
\task{%
    На каком расстоянии от собирающей линзы с фокусным расстоянием $50\,\text{дптр}$
    следует надо поместить предмет, чтобы расстояние
    от предмета до его действительного изображения было наименьшим?
}
\answer{%
    \begin{align*}
    \frac 1a &+ \frac 1b = D \implies b = \frac 1{D - \frac 1a} \implies \ell = a + b = a + \frac a{Da - 1} = \frac{ Da^2 }{Da - 1} \implies \\
    \implies \ell'_a &= \frac{ 2Da \cdot (Da - 1) - Da^2 \cdot D }{\sqr{Da - 1}}= \frac{ D^2a^2 - 2Da}{\sqr{Da - 1}} = \frac{ Da(Da - 2)}{\sqr{Da - 1}}\implies a_{\min} = \frac 2D \approx 40\,\text{мм}.
    \end{align*}
}
\solutionspace{120pt}

\tasknumber{5}%
\task{%
    Предмет в виде отрезка длиной $\ell$ расположен вдоль оптической оси
    собирающей линзы с фокусным расстоянием $F$.
    Середина отрезка расположена
    иа расстоянии $a$ от линзы, которая даёт действительное изображение
    всех точек предмета.
    Определить продольное увеличение предмета.
}
\answer{%
    \begin{align*}
    \frac 1{a + \frac \ell 2} &+ \frac 1b = \frac 1F \implies b = \frac{F\cbr{a + \frac \ell 2}}{a + \frac \ell 2 - F} \\
    \frac 1{a - \frac \ell 2} &+ \frac 1c = \frac 1F \implies c = \frac{F\cbr{a - \frac \ell 2}}{a - \frac \ell 2 - F} \\
    \abs{b - c} &= \abs{\frac{F\cbr{a + \frac \ell 2}}{a + \frac \ell 2 - F} - \frac{F\cbr{a - \frac \ell 2}}{a - \frac \ell 2 - F}}= F\abs{\frac{\cbr{a + \frac \ell 2}\cbr{a - \frac \ell 2 - F} - \cbr{a - \frac \ell 2}\cbr{a + \frac \ell 2 - F}}{ \cbr{a + \frac \ell 2 - F} \cbr{a - \frac \ell 2 - F} }} =  \\
    &= F\abs{\frac{a^2 - \frac {a\ell} 2 - Fa + \frac {a\ell} 2 - \frac {\ell^2} 4 - \frac {F\ell}2 - a^2 - \frac {a\ell}2 + aF + \frac {a\ell}2 + \frac {\ell^2} 4 - \frac {F\ell} 2}{\cbr{a + \frac \ell 2 - F} \cbr{a - \frac \ell 2 - F} }} = \\
    &= F\frac{F\ell}{\sqr{a-F} - \frac {\ell^2}4} = \frac{F^2\ell}{\sqr{a-F} - \frac {\ell^2}4}\implies \Gamma = \frac{\abs{b - c}}\ell = \frac{F^2}{\sqr{a-F} - \frac {\ell^2}4}.
    \end{align*}
}
\solutionspace{120pt}

\tasknumber{6}%
\task{%
    Даны точечный источник света $S$, его изображение $S_1$, полученное с помошью собирающей линзы,
    и ближайший к источнику фокус линзы $F$ (см.
    рис.
    на доске).
    Расстояния $SF = \ell$ и $SS_1 = L$.
    Определить положение линзы и её фокусное расстояние.
}
\answer{%
    \begin{align*}
    \frac 1a + \frac 1b &= \frac 1F, \ell = a - F, L = a + b \implies a = \ell + F, b = L - a = L - \ell - F \\
    \frac 1{\ell + F} + \frac 1{L - \ell - F} &= \frac 1F \\
    F\ell + F^2 + LF - F\ell - F^2 &= L\ell - \ell^2 - F\ell + LF - F\ell - F^2 \\
    0 &= L\ell - \ell^2 - 2F\ell - F^2 \\
    0 &=  F^2 + 2F\ell - L\ell + \ell^2 \\
    F &= -\ell \pm \sqrt{\ell^2 +  L\ell - \ell^2} = -\ell \pm \sqrt{L\ell} \implies F = \sqrt{L\ell} - \ell \\
    a &= \ell + F = \ell + \sqrt{L\ell} - \ell = \sqrt{L\ell}.
    \end{align*}
}
\solutionspace{120pt}

\tasknumber{7}%
\task{%
    Расстояние от освещённого предмета до экрана $80\,\text{см}$.
    Линза, помещенная между ними, даёт чёткое изображение предмета на
    экране при двух положениях, расстояние между которыми $40\,\text{см}$.
    Найти фокусное расстояние линзы.
}
\answer{%
    \begin{align*}
    \frac 1a + \frac 1b &= \frac 1F, \frac 1{a-\ell} + \frac 1{b+\ell} = \frac 1F, a + b = L \\
    \frac 1a + \frac 1b &= \frac 1{a-\ell} + \frac 1{b+\ell}\implies \frac{a + b}{ab} = \frac{(a-\ell) + (b+\ell)}{(a-\ell)(b+\ell)} \\
    ab  &= (a - \ell)(b+\ell) \implies 0  = -b\ell + a\ell - \ell^2 \implies 0 = -b + a - \ell \implies b = a - \ell \\
    a + (a - \ell) &= L \implies a = \frac{L + \ell}2 \implies b = \frac{L - \ell}2 \\
    F &= \frac{ab}{a + b} = \frac{L^2 -\ell^2}{4L} \approx 15\,\text{см}.
    \end{align*}
}
\solutionspace{120pt}

\tasknumber{8}%
\task{%
    Предмет находится на расстоянии $90\,\text{см}$ от экрана.
    Между предметом и экраном помещают линзу, причём при одном
    положении линзы на экране получается увеличенное изображение предмета,
    а при другом — уменьшенное.
    Каково фокусное расстояние линзы, если
    линейные размеры первого изображения в три раза больше второго?
}
\answer{%
    \begin{align*}
    \frac 1a + \frac 1{L-a} &= \frac 1F, h_1 = h \cdot \frac{L-a}a, \\
    \frac 1b + \frac 1{L-b} &= \frac 1F, h_2 = h \cdot \frac{L-b}b, \\
    \frac{h_1}{h_2} &= 3 \implies \frac{(L-a)b}{(L-b)a} = 3, \\
    \frac 1F &= \frac{ L }{a(L-a)} = \frac{ L }{b(L-b)} \implies \frac{L-a}{L-b} = \frac b a \implies \frac {b^2}{a^2} = 3.
    \\
    \frac 1a + \frac 1{L-a} &= \frac 1b + \frac 1{L-b} \implies \frac L{a(L-a)} = \frac L{b(L-b)} \implies \\
    \implies aL - a^2 &= bL - b^2 \implies (a-b)L = (a-b)(a+b) \implies b = L - a, \\
    \frac{\sqr{L-a}}{a^2} &= 3 \implies \frac La - 1 = \sqrt{3} \implies a = \frac{ L }{\sqrt{3} + 1} \\
    F &= \frac{a(L-a)}L = \frac 1L \cdot \frac L{\sqrt{3} + 1} \cdot \frac {L\sqrt{3}}{\sqrt{3} + 1}= \frac { L\sqrt{3} }{ \sqr{\sqrt{3} + 1} } \approx 21\,\text{см}.
    \end{align*}
}
% autogenerated
