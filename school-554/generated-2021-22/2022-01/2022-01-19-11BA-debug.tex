\newcommand\rootpath{../../..}
\documentclass[12pt,a4paper]{amsart}%DVI-mode.
\usepackage{graphics,graphicx,epsfig}%DVI-mode.
%\documentclass[pdftex,12pt]{amsart} %PDF-mode.
%\usepackage[pdftex]{graphicx}       %PDF-mode.

%\usepackage{a4wide}                 % Fit the text to A4 page tightly.
\usepackage[utf8]{inputenc}
\usepackage[T2A]{fontenc}
\usepackage[english,russian]{babel} % Download Russian fonts.
\usepackage{amsmath,amsfonts,amssymb,amsthm,amscd,mathrsfs} % Use AMS symbols.
\usepackage{tikz}
\usetikzlibrary{circuits.ee.IEC}
\usetikzlibrary{shapes.geometric}
\usetikzlibrary{decorations.markings}
%\usetikzlibrary{dashs}
%\usetikzlibrary{info}


\textheight=29cm % высота текста
\textwidth=18cm % ширина текста
\topmargin=-2.5cm % отступ от верхнего края
\parskip=6pt % интервал между абзацами
\oddsidemargin=-1.5cm
\evensidemargin=-1.5cm 

% wide docs
% \oddsidemargin=0cm
% \evensidemargin=0cm 
% \textheight=29cm % высота текста
% \textwidth=15cm % ширина текста
% \topmargin=-1.5cm % отступ от верхнего края
% \parskip=18pt % интервал между абзацами


\parindent=0pt % абзацный отступ
\tolerance=500 % терпимость к "жидким" строкам
\binoppenalty=10000 % штраф за перенос формул - 10000 - абсолютный запрет
\relpenalty=10000
\flushbottom % выравнивание высоты страниц
\def\baselinestretch{1.00}
\pagenumbering{gobble}

\begin{document}
\newcommand\bivec[2]{\begin{pmatrix} #1 \\ #2 \end{pmatrix}}

\newcommand\ol[1]{\overline{#1}}

\newcommand\p[1]{\ensuremath{\Prob\!\left(#1\right)}}
\def\cond{\,|\,}
\newcommand\e[1]{\mathsf{E}\!\left(#1\right)}
\newcommand\disp[1]{\mathsf{D}\!\left(#1\right)}
%\newcommand\norm[2]{\mathcal{N}\!\cbr{#1,#2}}
\newcommand\sign{\text{ sign }}

\newcommand\al[1]{\begin{align*} #1 \end{align*}}
\newcommand\begcas[1]{\begin{cases}#1\end{cases}}
\newcommand\tab[2]{	\vspace{-#1pt}
						\begin{tabbing} 
						#2
						\end{tabbing}
					\vspace{-#1pt}
					}


\newcommand\maintext[1]{{\bfseries\sffamily{#1}}}
\newcommand\simpletitle[1]{\begin{center} \maintext{#1} \end{center}}

\def\le{\leqslant}
\def\ge{\geqslant}
\def\Ell{\mathcal{L}}
\def\eps{\varepsilon}
\def\x{\ensuremath{\textbf{x}}}
\def\y{\ensuremath{\textbf{y}}}
\def\Rn{\ensuremath{\mathbb{R}^n}}
\def\RSS{\mathsf{RSS}}

\newcommand\mb[1]{\ensuremath{\boldsymbol{\mathbf{#1}}}}
\newcommand\argmax[1]{\arg\underset{#1}\max\,} % \operatornamewithlimits
%\newcommand{\prodl}{\mathop{\textstyle\prod}\limits}
\newcommand{\prodl}{\prod\limits}
\newcommand{\suml}{\sum\limits}
\newcommand\foral[1]{\forall\,#1\:}
\newcommand\exist[1]{\exists\,#1\:\colon}

\newcommand\cbr[1]{\left(#1\right)} %circled brackets
\newcommand\fbr[1]{\left\{#1\right\}} %figure brackets
\newcommand\sbr[1]{\left[#1\right]} %square brackets
\newcommand\modul[1]{\left|#1\right|}
\newcommand\cdf[2]{\cdot\frac{#1}{#2}}
\newcommand\integr[3]{\int\limits_{#1}^{#2}{#3}}
\newcommand\obol[1]{O\!\cbr{#1}}
\newcommand\norm[1]{\ensuremath{\left\|{#1}\right\|}}

\newcommand\dd[2]{\frac{\partial#1}{\partial#2}}

\newcommand\addeps[2]{
	\begin{figure} [!ht] %lrp
		\centering
		\includegraphics[height=240px]{#1.eps}
		\vspace{-10pt}
		\caption{#2}
		\label{eps:#1}
	\end{figure}
}

\newcommand\addtikz[4]{
	\begin{figure} [!ht] %lrp
		\centering
		\begin{tikzpicture}[x=#2cm,y=#2cm,#3]
			\input{#1.tikz}
		\end{tikzpicture}
		\vspace{-10pt}
		\caption{#4}
		\label{tikz:#1}	
	\end{figure}
}



\newcommand\addepssize[3]{
	\begin{figure} [!ht] %lrp hp
		\centering
		\includegraphics[height=#3px]{#1.eps}
		\vspace{-10pt}
		\caption{#2}
		\label{eps:#1}
	\end{figure}
}

\def\algorithmicrequire{\textbf{Вход:}}
\def\algorithmicensure{\textbf{Выход:}}
\def\algorithmicif{\textbf{если}}
\def\algorithmicthen{\textbf{то}}
\def\algorithmicelse{\textbf{иначе}}
\def\algorithmicelsif{\textbf{иначе если}}
\def\algorithmicfor{\textbf{для}}
\def\algorithmicforall{\textbf{для всех}}
\def\algorithmicdo{}
\def\algorithmicwhile{\textbf{пока}}
\def\algorithmicrepeat{\textbf{повторять}}
\def\algorithmicuntil{\textbf{пока}}
\def\algorithmicloop{\textbf{цикл}}
% переопределение стиля комментариев
\def\algorithmiccomment#1{\quad// {\sl #1}}
%\raggedright
\classdate{7}{20 апреля 2018}

\task 1
Площадь большого поршня гидравлического домкрата $S_1 = 20\units{см}^2$, а малого $S_2 = 0{,}5\units{см}^2.$ Груз какой максимальной массы можно поднять этим домкратом, если на малый поршень давить с силой не более $F=200\units{Н}?$ Силой трения от стенки цилиндров пренебречь.

\task 2
В сосуд налита вода. Расстояние от поверхности воды до дна $H = 0{,}5\units{м},$ площадь дна $S = 0{,}1\units{м}^2.$ Найти гидростатическое давление $P_1$ и полное давление $P_2$ вблизи дна. Найти силу давления воды на дно. Плотность воды \rhowater

\task 3
На лёгкий поршень площадью $S=900\units{см}^2,$ касающийся поверхности воды, поставили гирю массы $m=3\units{кг}$. Высота слоя воды в сосуде с вертикальными стенками $H = 20\units{см}$. Определить давление жидкости вблизи дна, если плотность воды \rhowater

\task 4
Давление газов в конце сгорания в цилиндре дизельного двигателя трактора $P = 9\units{МПа}.$ Диаметр цилиндра $d = 130\units{мм}.$ С какой силой газы давят на поршень в цилиндре? Площадь круга диаметром $D$ равна $S = \cfrac{\pi D^2}4.$

\task 5
Площадь малого поршня гидравлического подъёмника $S_1 = 0{,}8\units{см}^2$, а большого $S_2 = 40\units{см}^2.$ Какую силу $F$ надо приложить к малому поршню, чтобы поднять груз весом $P = 8\units{кН}?$

\task 6
Герметичный сосуд полностью заполнен водой и стоит на столе. На небольшой поршень площадью $S$ давят рукой с силой $F$. Поршень находится ниже крышки сосуда на $H_1$, выше дна на $H_2$ и может свободно перемещаться. Плотность воды $\rho$, атмосферное давление $P_A$. Найти давления $P_1$ и $P_2$ в воде вблизи крышки и дна сосуда.
\\ \\
\classdate{7}{20 апреля 2018}

\task 1
Площадь большого поршня гидравлического домкрата $S_1 = 20\units{см}^2$, а малого $S_2 = 0{,}5\units{см}^2.$ Груз какой максимальной массы можно поднять этим домкратом, если на малый поршень давить с силой не более $F=200\units{Н}?$ Силой трения от стенки цилиндров пренебречь.

\task 2
В сосуд налита вода. Расстояние от поверхности воды до дна $H = 0{,}5\units{м},$ площадь дна $S = 0{,}1\units{м}^2.$ Найти гидростатическое давление $P_1$ и полное давление $P_2$ вблизи дна. Найти силу давления воды на дно. Плотность воды \rhowater

\task 3
На лёгкий поршень площадью $S=900\units{см}^2,$ касающийся поверхности воды, поставили гирю массы $m=3\units{кг}$. Высота слоя воды в сосуде с вертикальными стенками $H = 20\units{см}$. Определить давление жидкости вблизи дна, если плотность воды \rhowater

\task 4
Давление газов в конце сгорания в цилиндре дизельного двигателя трактора $P = 9\units{МПа}.$ Диаметр цилиндра $d = 130\units{мм}.$ С какой силой газы давят на поршень в цилиндре? Площадь круга диаметром $D$ равна $S = \cfrac{\pi D^2}4.$

\task 5
Площадь малого поршня гидравлического подъёмника $S_1 = 0{,}8\units{см}^2$, а большого $S_2 = 40\units{см}^2.$ Какую силу $F$ надо приложить к малому поршню, чтобы поднять груз весом $P = 8\units{кН}?$

\task 6
Герметичный сосуд полностью заполнен водой и стоит на столе. На небольшой поршень площадью $S$ давят рукой с силой $F$. Поршень находится ниже крышки сосуда на $H_1$, выше дна на $H_2$ и может свободно перемещаться. Плотность воды $\rho$, атмосферное давление $P_A$. Найти давления $P_1$ и $P_2$ в воде вблизи крышки и дна сосуда.

\newpage

\adddate{8 класс. 20 апреля 2018}

\task 1
Между точками $A$ и $B$ электрической цепи подключены последовательно резисторы $R_1 = 10\units{Ом}$ и $R_2 = 20\units{Ом}$ и параллельно им $R_3 = 30\units{Ом}.$ Найдите эквивалентное сопротивление $R_{AB}$ этого участка цепи.

\task 2
Электрическая цепь состоит из последовательности $N$ одинаковых звеньев, в которых каждый резистор имеет сопротивление $r$. Последнее звено замкнуто резистором сопротивлением $R$. При каком соотношении $\cfrac{R}{r}$ сопротивление цепи не зависит от числа звеньев?

\task 3
Для измерения сопротивления $R$ проводника собрана электрическая цепь. Вольтметр $V$ показывает напряжение $U_V = 5\units{В},$ показание амперметра $A$ равно $I_A = 25\units{мА}.$ Найдите величину $R$ сопротивления проводника. Внутреннее сопротивление вольтметра $R_V = 1{,}0\units{кОм},$ внутреннее сопротивление амперметра $R_A = 2{,}0\units{Ом}.$

\task 4
Шкала гальванометра имеет $N=100$ делений, цена деления $\delta = 1\units{мкА}$. Внутреннее сопротивление гальванометра $R_G = 1{,}0\units{кОм}.$ Как из этого прибора сделать вольтметр для измерения напряжений до $U = 100\units{В}$ или амперметр для измерения токов силой до $I = 1\units{А}?$

\\ \\ \\ \\ \\ \\ \\ \\
\adddate{8 класс. 20 апреля 2018}

\task 1
Между точками $A$ и $B$ электрической цепи подключены последовательно резисторы $R_1 = 10\units{Ом}$ и $R_2 = 20\units{Ом}$ и параллельно им $R_3 = 30\units{Ом}.$ Найдите эквивалентное сопротивление $R_{AB}$ этого участка цепи.

\task 2
Электрическая цепь состоит из последовательности $N$ одинаковых звеньев, в которых каждый резистор имеет сопротивление $r$. Последнее звено замкнуто резистором сопротивлением $R$. При каком соотношении $\cfrac{R}{r}$ сопротивление цепи не зависит от числа звеньев?

\task 3
Для измерения сопротивления $R$ проводника собрана электрическая цепь. Вольтметр $V$ показывает напряжение $U_V = 5\units{В},$ показание амперметра $A$ равно $I_A = 25\units{мА}.$ Найдите величину $R$ сопротивления проводника. Внутреннее сопротивление вольтметра $R_V = 1{,}0\units{кОм},$ внутреннее сопротивление амперметра $R_A = 2{,}0\units{Ом}.$

\task 4
Шкала гальванометра имеет $N=100$ делений, цена деления $\delta = 1\units{мкА}$. Внутреннее сопротивление гальванометра $R_G = 1{,}0\units{кОм}.$ Как из этого прибора сделать вольтметр для измерения напряжений до $U = 100\units{В}$ или амперметр для измерения токов силой до $I = 1\units{А}?$


% \begin{flushright}
\textsc{ГБОУ школа №554, 20 ноября 2018\,г.}
\end{flushright}

\begin{center}
\LARGE \textsc{Математический бой, 8 класс}
\end{center}

\problem{1} Есть тридцать карточек, на каждой написано по одному числу: на десяти карточках~–~$a$,  на десяти других~–~$b$ и на десяти оставшихся~–~$c$ (числа  различны). Известно, что к любым пяти карточкам можно подобрать ещё пять так, что сумма чисел на этих десяти карточках будет равна нулю. Докажите, что~одно из~чисел~$a, b, c$ равно нулю.

\problem{2} Вокруг стола стола пустили пакет с орешками. Первый взял один орешек, второй — 2, третий — 3 и так далее: каждый следующий брал на 1 орешек больше. Известно, что на втором круге было взято в сумме на 100 орешков больше, чем на первом. Сколько человек сидело за столом?

% \problem{2} Натуральное число разрешено увеличить на любое целое число процентов от 1 до 100, если при этом получаем натуральное число. Найдите наименьшее натуральное число, которое нельзя при помощи таких операций получить из~числа 1.

% \problem{3} Найти сумму $1^2 - 2^2 + 3^2 - 4^2 + 5^2 + \ldots - 2018^2$.

\problem{3} В кружке рукоделия, где занимается Валя, более 93\% участников~—~девочки. Какое наименьшее число участников может быть в таком кружке?

\problem{4} Произведение 2018 целых чисел равно 1. Может ли их сумма оказаться равной~0?

% \problem{4} Можно ли все натуральные числа от~1 до~9 записать в~клетки таблицы~$3\times3$ так, чтобы сумма в~любых двух соседних (по~вертикали или горизонтали) клетках равнялось простому числу?

\problem{5} На доске написано 2018 нулей и 2019 единиц. Женя стирает 2 числа и, если они были одинаковы, дописывает к оставшимся один ноль, а~если разные — единицу. Потом Женя повторяет эту операцию снова, потом ещё и~так далее. В~результате на~доске останется только одно число. Что это за~число?

\problem{6} Докажите, что в~любой компании людей найдутся 2~человека, имеющие равное число знакомых в этой компании (если $A$~знаком с~$B$, то~и $B$~знаком с~$A$).

\problem{7} Три колокола начинают бить одновременно. Интервалы между ударами колоколов соответственно составляют $\cfrac43$~секунды, $\cfrac53$~секунды и $2$~секунды. Совпавшие по времени удары воспринимаются за~один. Сколько ударов будет услышано за 1~минуту, включая первый и последний удары?

\problem{8} Восемь одинаковых момент расположены по кругу. Известно, что три из~них~— фальшивые, и они расположены рядом друг с~другом. Вес фальшивой монеты отличается от~веса настоящей. Все фальшивые монеты весят одинаково, но неизвестно, тяжелее или легче фальшивая монета настоящей. Покажите, что за~3~взвешивания на~чашечных весах без~гирь можно определить все фальшивые монеты.

\end{document}

\begin{document}

\setdate{19~января~2022}
\setclass{11«БА»}

\addpersonalvariant{Михаил Бурмистров}

\tasknumber{1}%
\task{%
    Запишите известные вам виды классификации изображений.
}
\solutionspace{60pt}

\tasknumber{2}%
\task{%
    В каких линзах можно получить прямое изображение объекта?
}
\answer{%
    $\text{ собирающие и рассеивающие }$
}
\solutionspace{40pt}

\tasknumber{3}%
\task{%
    Какое изображение называют мнимым?
}
\solutionspace{40pt}

\tasknumber{4}%
\task{%
    Есть две линзы, обозначим их 1 и 2.
    Известно что фокусное расстояние линзы 1 больше, чем у линзы 2.
    Какая линза сильнее преломляет лучи?
}
\answer{%
    $2$
}
\solutionspace{40pt}

\tasknumber{5}%
\task{%
    Предмет находится на расстоянии $20\,\text{см}$ от собирающей линзы с фокусным расстоянием $40\,\text{см}$.
    Определите тип изображения, расстояние между предметом и его изображением, увеличение предмета.
    Сделайте схематичный рисунок (не обязательно в масштабе, но с сохранением свойств линзы и изображения).
}
\answer{%
    $b = \frac{aF}{a - F} \approx -40\,\text{см}, l = \abs{a + b} = 20\,\text{см}, \Gamma = 2{,}00, \text{мнимое, прямое, увеличенное}$
}
\solutionspace{100pt}

\tasknumber{6}%
\task{%
    Объект находится на расстоянии $45\,\text{см}$ от линзы, а его мнимое изображение — в $40\,\text{см}$ от неё.
    Определите увеличение предмета, фокусное расстояние линзы, оптическую силу линзы и её тип.
}
\answer{%
    $\frac 1F = D = \frac 1a + \frac 1b, D \approx-0{,}3\,\text{дптр}, F\approx-360\,\text{см}, \Gamma\approx0{,}89,\text{рассеивающая}$
}
\solutionspace{80pt}

\tasknumber{7}%
\task{%
    Известно, что из формулы тонкой линзы $\cbr{\frac 1F = \frac 1a + \frac 1b}$
    и определения увеличения $\cbr{\Gamma_y = \frac ba}$ можно получить выражение
    для увеличения: $\Gamma_y = \frac {aF}{a - F} \cdot \frac 1a = \frac {F}{a - F}.$
    Назовём такое увеличение «поперечным»: поперёк главной оптической оси (поэтому и ${}_y$).
    Получите формулу для «продольного» увеличения $\Gamma_x$ небольшого предмета, находящегося на главной оптической оси.
    Можно ли применить эту формулу для предмета, не лежащего на главной оптической оси, почему?
}
\answer{%
    \begin{align*}
    \frac 1F &= \frac 1a + \frac 1b \implies b = \frac {aF}{a - F} \\
    \frac 1F &= \frac 1{a + x} + \frac 1c \implies c = \frac {(a+x)F}{a + x - F} \\
    x' &= \abs{b - c} = \frac {aF}{a - F} - \frac {(a+x)F}{a + x - F} = F\cbr{\frac {a}{a - F} - \frac {a+x}{a + x - F}} =  \\
    &= F \cdot \frac {a^2 + ax - aF - a^2 - ax + aF + xF}{(a - F)(a + x - F)} = F \cdot \frac {xF}{(a - F)(a + x - F)} \\
    \Gamma_x &= \frac{x'}x = \frac{F^2}{(a - F)(a + x - F)} \to \frac{F^2}{\sqr{a - F}}.
    \\
    &\text{Нельзя: изображение по-разному растянет по осям $x$ и $y$ и понадобится теорема Пифагора}
    \end{align*}
}


\variantsplitter


\addpersonalvariant{Михаил Бурмистров}

\tasknumber{8}%
\task{%
    Доказать формулу тонкой линзы для рассеивающей линзы.
}
\solutionspace{120pt}

\tasknumber{9}%
\task{%
    Постройте ход луча $CL$ в тонкой линзе.
    Известно положение линзы и оба её фокуса (см.
    рис.
    на доске).
    Рассмотрите оба типа линзы, сделав 2 рисунка: собирающую и рассеивающую.
}
\solutionspace{120pt}

\tasknumber{10}%
\task{%
    На экране, расположенном на расстоянии $60\,\text{см}$ от собирающей линзы,
    получено изображение точечного источника, расположенного на главной оптической оси линзы.
    На какое расстояние переместится изображение на экране,
    если при неподвижном источнике переместить линзу на $3\,\text{см}$ в плоскости, перпендикулярной главной оптической оси?
    Фокусное расстояние линзы равно $20\,\text{см}$.
}
\answer{%
    \begin{align*}
    &\frac 1F = \frac 1a + \frac 1b \implies a = \frac{bF}{b-F} \implies \Gamma = \frac ba = \frac{b-F}F \\
    &y = x \cdot \Gamma = x \cdot \frac{b-F}F \implies d = x + y = 9\,\text{см}.
    \end{align*}
}
\solutionspace{120pt}

\tasknumber{11}%
\task{%
    Оптическая сила двояковыпуклой линзы в воздухе $5{,}5\,\text{дптр}$, а в воде $1{,}5\,\text{дптр}$.
    Определить показатель преломления $n$ материала, из которого изготовлена линза.
    Показатель преломления воды равен $1{,}33$.
}
\answer{%
    \begin{align*}
    D_1 &=\cbr{\frac n{n_1} - 1}\cbr{\frac 1{R_1} + \frac 1{R_2}}, \\
    D_2 &=\cbr{\frac n{n_2} - 1}\cbr{\frac 1{R_1} + \frac 1{R_2}}, \\
    \frac {D_2}{D_1} &=\frac{\frac n{n_2} - 1}{\frac n{n_1} - 1} \implies {D_2}\cbr{\frac n{n_1} - 1} = {D_1}\cbr{\frac n{n_2} - 1}  \implies n\cbr{\frac{D_2}{n_1} - \frac{D_1}{n_2}} = D_2 - D_1, \\
    n &= \frac{D_2 - D_1}{\frac{D_2}{n_1} - \frac{D_1}{n_2}} = \frac{n_1 n_2 (D_2 - D_1)}{D_2n_2 - D_1n_1} \approx 1{,}518.
    \end{align*}
}
\solutionspace{120pt}

\tasknumber{12}%
\task{%
    На каком расстоянии от собирающей линзы с фокусным расстоянием $40\,\text{дптр}$
    следует надо поместить предмет, чтобы расстояние
    от предмета до его действительного изображения было наименьшим?
}
\answer{%
    \begin{align*}
    \frac 1a &+ \frac 1b = D \implies b = \frac 1{D - \frac 1a} \implies \ell = a + b = a + \frac a{Da - 1} = \frac{ Da^2 }{Da - 1} \implies \\
    \implies \ell'_a &= \frac{ 2Da \cdot (Da - 1) - Da^2 \cdot D }{\sqr{Da - 1}}= \frac{ D^2a^2 - 2Da}{\sqr{Da - 1}} = \frac{ Da(Da - 2)}{\sqr{Da - 1}}\implies a_{\min} = \frac 2D \approx 50\,\text{мм}.
    \end{align*}
}


\variantsplitter


\addpersonalvariant{Михаил Бурмистров}

\tasknumber{13}%
\task{%
    Даны точечный источник света $S$, его изображение $S_1$, полученное с помощью собирающей линзы,
    и ближайший к источнику фокус линзы $F$ (см.
    рис.
    на доске).
    Расстояния $SF = \ell$ и $SS_1 = L$.
    Определить положение линзы и её фокусное расстояние.
}
\answer{%
    \begin{align*}
    \frac 1a + \frac 1b &= \frac 1F, \ell = a - F, L = a + b \implies a = \ell + F, b = L - a = L - \ell - F \\
    \frac 1{\ell + F} + \frac 1{L - \ell - F} &= \frac 1F \\
    F\ell + F^2 + LF - F\ell - F^2 &= L\ell - \ell^2 - F\ell + LF - F\ell - F^2 \\
    0 &= L\ell - \ell^2 - 2F\ell - F^2 \\
    0 &=  F^2 + 2F\ell - L\ell + \ell^2 \\
    F &= -\ell \pm \sqrt{\ell^2 +  L\ell - \ell^2} = -\ell \pm \sqrt{L\ell} \implies F = \sqrt{L\ell} - \ell \\
    a &= \ell + F = \ell + \sqrt{L\ell} - \ell = \sqrt{L\ell}.
    \end{align*}
}
\solutionspace{120pt}

\tasknumber{14}%
\task{%
    Расстояние от освещённого предмета до экрана $80\,\text{см}$.
    Линза, помещенная между ними, даёт чёткое изображение предмета на
    экране при двух положениях, расстояние между которыми $40\,\text{см}$.
    Найти фокусное расстояние линзы.
}
\answer{%
    \begin{align*}
    \frac 1a + \frac 1b &= \frac 1F, \frac 1{a-\ell} + \frac 1{b+\ell} = \frac 1F, a + b = L \\
    \frac 1a + \frac 1b &= \frac 1{a-\ell} + \frac 1{b+\ell}\implies \frac{a + b}{ab} = \frac{(a-\ell) + (b+\ell)}{(a-\ell)(b+\ell)} \\
    ab  &= (a - \ell)(b+\ell) \implies 0  = -b\ell + a\ell - \ell^2 \implies 0 = -b + a - \ell \implies b = a - \ell \\
    a + (a - \ell) &= L \implies a = \frac{L + \ell}2 \implies b = \frac{L - \ell}2 \\
    F &= \frac{ab}{a + b} = \frac{L^2 -\ell^2}{4L} \approx 15\,\text{см}.
    \end{align*}
}
\solutionspace{120pt}

\tasknumber{15}%
\task{%
    Предмет находится на расстоянии $80\,\text{см}$ от экрана.
    Между предметом и экраном помещают линзу, причём при одном
    положении линзы на экране получается увеличенное изображение предмета,
    а при другом — уменьшенное.
    Каково фокусное расстояние линзы, если
    линейные размеры первого изображения в два раза больше второго?
}
\answer{%
    \begin{align*}
    \frac 1a + \frac 1{L-a} &= \frac 1F, h_1 = h \cdot \frac{L-a}a, \\
    \frac 1b + \frac 1{L-b} &= \frac 1F, h_2 = h \cdot \frac{L-b}b, \\
    \frac{h_1}{h_2} &= 2 \implies \frac{(L-a)b}{(L-b)a} = 2, \\
    \frac 1F &= \frac{ L }{a(L-a)} = \frac{ L }{b(L-b)} \implies \frac{L-a}{L-b} = \frac b a \implies \frac {b^2}{a^2} = 2.
    \\
    \frac 1a + \frac 1{L-a} &= \frac 1b + \frac 1{L-b} \implies \frac L{a(L-a)} = \frac L{b(L-b)} \implies \\
    \implies aL - a^2 &= bL - b^2 \implies (a-b)L = (a-b)(a+b) \implies b = L - a, \\
    \frac{\sqr{L-a}}{a^2} &= 2 \implies \frac La - 1 = \sqrt{2} \implies a = \frac{ L }{\sqrt{2} + 1} \\
    F &= \frac{a(L-a)}L = \frac 1L \cdot \frac L{\sqrt{2} + 1} \cdot \frac {L\sqrt{2}}{\sqrt{2} + 1}= \frac { L\sqrt{2} }{ \sqr{\sqrt{2} + 1} } \approx 19{,}4\,\text{см}.
    \end{align*}
}
\solutionspace{120pt}

\tasknumber{16}%
\task{%
    (Задача-«гроб»: решать на обратной стороне) Квадрат со стороной $d = 2\,\text{см}$ расположен так,
    что 2 его стороны параллельны главной оптической оси собирающей линзы,
    его центр удален на $h = 4\,\text{см}$ от этой оси и на $a = 15\,\text{см}$ от плоскости линзы.
    Определите площадь изображения квадрата, если фокусное расстояние линзы составляет $F = 20\,\text{см}$.
    % (и сравните с площадью объекта, умноженной на квадрат увеличения центра квадрата).
}
\answer{%
    \begin{align*}
    &\text{Все явные вычисления — в см и $\text{см}^2$,} \\
    \frac 1 F &= \frac 1{a + \frac d2} + \frac 1b \implies b = \frac 1{\frac 1 F - \frac 1{a + \frac d2}} = \frac{F(a + \frac d2)}{a + \frac d2 - F} = -80, \\
    \frac 1 F &= \frac 1{a - \frac d2} + \frac 1c \implies c = \frac 1{\frac 1 F - \frac 1{a - \frac d2}} = \frac{F(a - \frac d2)}{a - \frac d2 - F} = -\frac{140}3, \\
    c - b &= \frac{F(a - \frac d2)}{a - \frac d2 - F} - \frac{F(a + \frac d2)}{a + \frac d2 - F} = F\cbr{ \frac{a - \frac d2}{a - \frac d2 - F} - \frac{a + \frac d2}{a + \frac d2 - F} } =  \\
    &= F \cdot \frac{a^2 + \frac {ad}2 - aF - \frac{ad}2 - \frac{d^2}4 + \frac{dF}2 - a^2 + \frac {ad}2 + aF - \frac{ad}2 + \frac{d^2}4 + \frac{dF}2}{\cbr{a + \frac d2 - F}\cbr{a - \frac d2 - F}}= F \cdot \frac {dF}{\cbr{a + \frac d2 - F}\cbr{a - \frac d2 - F}} = \frac{100}3.
    \\
    \Gamma_b &= \frac b{a + \frac d2} = \frac{ F }{a + \frac d2 - F} = -5, \\
    \Gamma_c &= \frac c{a - \frac d2} = \frac{ F }{a - \frac d2 - F} = -\frac{10}3, \\
    &\text{ тут интересно отметить, что } \Gamma_x = \frac{ c - b}{ d } = \frac{ F^2 }{\cbr{a + \frac d2 - F}\cbr{a - \frac d2 - F}} \ne \Gamma_b \text{ или } \Gamma_c \text{ даже при малых $d$}.
    \\
    S' &= \frac{d \cdot \Gamma_b + d \cdot \Gamma_c}2 \cdot (c - b) = \frac d2 \cbr{\frac{ F }{a + \frac d2 - F} + \frac{ F }{a - \frac d2 - F}} \cdot \cbr{c - b} =  \\
    &=\frac {dF}2 \cbr{\frac 1{a + \frac d2 - F} + \frac 1{a - \frac d2 - F}} \cdot \frac {dF^2}{\cbr{a + \frac d2 - F}\cbr{a - \frac d2 - F}} =  \\
    &=\frac {dF}2 \cdot \frac{a - \frac d2 - F + a + \frac d2 - F}{\cbr{a + \frac d2 - F}\cbr{a - \frac d2 - F}} \cdot \frac {dF^2}{\cbr{a + \frac d2 - F}\cbr{a - \frac d2 - F}} =  \\
    &= \frac {d^2F^3}{2\sqr{a + \frac d2 - F}\sqr{a - \frac d2 - F}} \cdot (2a - 2F) = \frac {d^2F^3(a - F)}{ \sqr{\sqr{a - F} - \frac{d^2}4} } = -\frac{2500}9.
    \end{align*}
}

\end{document}
% autogenerated
